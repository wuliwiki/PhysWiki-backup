% 高斯映射
% keys Gauss map|形状算子
\pentry{可定向曲面\upref{OriSur}}

\subsection{高斯映射和形状算子}
\addTODO{高斯映射和形状算子的关系可能需要解释(形状算子是高斯映射的微分的负数);自伴性的证明尚缺;Meusnier定理的证明或解释尚缺.}
\begin{definition}{高斯映射}
给定可定向曲面$S\subseteq\mathbb{R}^3$和其一个定向$N$.由于定向的值都是单位向量,因此$N$是一个$S\to S^2$的映射,称为$S$的一个\textbf{高斯映射(Gauss map)}.
\end{definition}

高斯映射是为了研究曲面的内蕴性质而诞生的,在现代微分几何中通常又改用\textbf{形状算子}来描述.我在此提一下形状算子的概念,作为补充.

\begin{definition}{形状算子}
给定流形(曲面)$S$上一点$p$,则$p$处的形状算子$L_p$是一个$T_pS\to T_pS$的一个映射.对于任意切向量$\bvec{v}\in T_pS$,取$\bvec{v}$对应的一条曲线$\alpha(t)$,都有$L_p(\bvec{v})=-\frac{\dd}{\dd t}N(\alpha(t))$,其中$N$是$S$上的一个定向.
\end{definition}


高斯映射的一个关键性质是\textbf{自伴(self-adjoint)},表述如下:

\begin{theorem}{高斯映射的自伴性}
给定高斯映射$N:S\to S^2$,且$\bvec{x}(u, v)$是$S$的一个局部坐标系,则有$N_u\cdot \bvec{x}_v=N_v\cdot \bvec{x}_u$.
\end{theorem}

这一点和形状算子的自伴性是等价的.

\begin{theorem}{形状算子的性质}
给定流形(曲面)$S$上一点$p$处的形状算子$L_p$,则对于$S$上任意切向量场$X, Y$,都有$L_p(X)\cdot Y=D_XY\cdot N$.
\end{theorem}


\begin{definition}{共轭}
如果$T_pS$上的两个向量$\bvec{v}_1$和$\bvec{v}_2$满足$L_p(\bvec{v}_1)\cdot \bvec{v}2=L_p(\bvec{v}_2)\cdot \bvec{v}1=0$,那么称$\bvec{v}_1$和$\bvec{v}_2$是\textbf{共轭(conjugate)}的向量,它们所在的方向也是彼此共轭的.
\end{definition}

高斯映射与高斯曲率的联系,由以下定理揭示:

\begin{theorem}{}
对于曲面$S$上的一点$p$,如果其高斯曲率$K(p)\not=0$,那么$K(p)=\lim\limits_{A\to 0}\frac{A'}{A}$,其中$A$是一块包含了$p$的邻域的面积,$A'$是$N(A)$的面积.
\end{theorem}

