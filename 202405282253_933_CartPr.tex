% 笛卡尔积
% keys 直积|有序数对|笛卡尔积
% license Usr
% type Tutor

\begin{issues}
\issueDraft
\end{issues}

在刚刚接触到乘法计算时,作为一个运算结果,“积”是作为与“乘法”相关的概念被引入的。后来,随着对向量的学习的深入,内积和外积逐渐也成为了熟悉的概念,二者分别与点乘($\cdot$)和叉乘($\times$)相对应。或许,“卷积”和“张量积”等概念也偶尔会出现在你的视野中。他们往往是与一个逐渐抽象的“乘法”相对应,说他逐渐抽象,是因为他与我们熟知的数的乘法的样子和计算方法相去甚远。而还称呼它是乘法,是因为某种程度上,它保留了乘法的一些特性。

在物理上,常常会通过“乘法”来定义一个新的物理量,比如:功是力与位移的乘积,力矩是力与力臂的乘积,电路中功率是电流与电压的乘积(先忽略这个乘积具体的形式)等。这个新的物理量与原有的两个物理量之间都存在关系。而“加法”往往是在一个概念内部量的多少的计算,基本是不涉及其他概念的。

数学上使用\textbf{直积}(direct product)来组合两个同类的已知对象,从而定义新对象,例如集合、群、模、拓扑空间等都可以进行直积运算。而作用在两个集合上的直积便称为\textbf{笛卡尔积}(Cartesian product)。下面会先直接给出笛卡尔积的定义,然后就定义中的一些概念进行讲解。

\begin{definition}{笛卡尔积}\label{def_CartPr_1}
从集合$X,Y$中各取出一个元素$x,y$形成的有序对的集合,称作\textbf{笛卡尔积}。
\begin{equation}
X\times Y:=\{(x,y)|x\in X,y\in Y\}.~
\end{equation}

\end{definition}

\subsection{有序对}

在刚开始接触集合这个概念的时候,一定会知道,集合有这样一个性质——“集合内的元素的具有无序性”。而现实生活中,“序”这个概念又是必不可少的。有序对(ordered pair)的出现就是为了能够有概念来表示两个元素顺序。因此,定义时就希望这个概念能够满足两个性质:

\begin{enumerate}
\item 唯一性:每一个有序对是唯一定义的,即$(a, b) = (c, d)\implies (a = c) \land (b = d)$。
\item 顺序性:有序对中的元素顺序是固定的,即$a\neq b\iff(a, b)\neq (b, a)$。
\end{enumerate}

这样,“序”的概念就能在表达序的想法(顺序性)的同时,保持稳定(唯一性)。

\begin{definition}{有序对}
\textbf{有序对}有以下几种常见的定义方法:
\begin{itemize}
\item 维纳对(Wiener pair,1914年):$(a, b):= \{\{\emptyset,\{ a\}\}, \{\{b\}\}\} $
\item 豪斯多夫对(Hausdorff Pair,1914年):$ (a, b):= \{\{a, 1\}, \{b, 2\}\} $
\item 库拉托夫斯基对(Kuratoswki pair,1921年)$(a, b) := \{\{a\}, \{a, b\}\}$
\end{itemize}
\end{definition}
其中:
\begin{itemize}
\item 维纳对通常配合类型论使用。
\item 豪斯多夫对由于使用了数字作为序的描述,如果要考虑尚未定义数的场景,或需要研究数的序时,可能会造成循环论证。
\item 库拉托夫斯基对定义较为简洁,目前使用也最为广泛。\textbf{下面的描述中,会采取此定义}。
\end{itemize}

需要注意的是,每个定义在某些具体领域(如研究集合的理论)使用时,都存在各自的局限性。也存在一些新的定义,由于与内容关系不大没有在此给出。

定义能够比较好地描述之前提到的特性,下面以唯一性举例。
\begin{example}{唯一性证明}
根据定义有$(a, b) = \{\{a\}, \{a, b\}\} , (c, d) = \{\{c\}, \{c, d\}\} $,若$(a, b)=(c,d)$,即$\{\{a\}, \{a, b\}\}=\{\{c\}, \{c, d\}\}$,则根据集合相等的定义,且$card(\{a\})=1$,必有$\{a\}=\{c\},\{a, b\}=\{c, d\}$。继续根据集合相等,有$a=c$,进而有$b=d$。

证毕。
\end{example}

有序对的概念在关系、函数、拓扑等领域都有应用。

\subsection{笛卡尔积}

现在,根据\autoref{def_CartPr_1},可以看出,当前笛卡尔积的结果是对应两个集合的。如果出现3个或者到n个集合,只要依次进行即可。

参考文献
Halmos, Paul R. *Naive Set Theory*. Springer, 1974.
Wikipedia: [Ordered Pair](https://en.wikipedia.org/wiki/Ordered_pair)