% 黑体辐射定律
% keys 黑体辐射|普朗克|电磁波

\begin{issues}
\issueDraft
\issueNeedCite
\end{issues}

单位体积单位频率的能量
\begin{equation}
S_\nu(\nu) = \frac{8\pi h}{c^3}\frac{\nu^3}{\E^{h\nu/(k_B T)} - 1}
\end{equation}
如果要计算波长的分布, 根据随机变量的变换\upref{RandCV}, 由 $\abs{S_\lambda(\lambda) \dd{\lambda}} = \abs{S_\nu(\nu)\dd{\nu}}$ 得
\begin{equation}\label{BBdLaw_eq1}
S_\lambda(\lambda) = \frac{c}{\lambda^2}S_\nu\qty(\frac{c}{\lambda}) =
\frac{8\pi ch}{\lambda^5} \frac{1}{\E^{hc/(k_B T\lambda)} - 1}
\end{equation}

单位面积单位频率单位立体角\upref{SolAng}的功率
\begin{equation}\label{BBdLaw_eq2}
B(\nu) = \frac{c}{4\pi}S_\nu(\nu) = \frac{2h}{c^2} \frac{\nu^3}{\E^{h\nu/(k_B T)} - 1}
\end{equation}
在黑体内部, 辐射是各向同性的, 但在黑体表面, 对于给定的一个平面微元, $B(\nu)$ 是垂直于平面的值, 与法向量夹角为 $\theta$ 的方向的值为 $B(\nu)\cos\theta$.

每个能级($n\omega\hbar$)的平均能量
\begin{equation}
E(\nu) = \frac{h\nu}{\E^{h\nu/(k_B T)} - 1}
\end{equation}

态密度
\begin{equation}
\rho(\nu) = \frac{8\pi}{c^3}\nu^2
\end{equation}

\subsection{推导}
\pentry{盒中的电磁波盒中的电磁波\upref{EBBox}}
