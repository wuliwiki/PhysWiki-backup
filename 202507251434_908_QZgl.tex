% 乔治·格林(综述)
% license CCBYSA3
% type Wiki

本文根据 CC-BY-SA 协议转载翻译自维基百科\href{https://en.wikipedia.org/wiki/George_Green_(mathematician)}{相关文章}。

乔治·格林(George Green,1793年7月14日-1841年5月31日)是一位英国数学物理学家,他于1828年撰写了《数学分析在电学与磁学理论中的应用论文》【2】【3】。这篇论文引入了多个重要概念,包括一个类似于现代“格林公式”的定理、如今在物理学中广泛使用的势函数思想,以及现在被称为“格林函数”的概念。格林是第一个构建电学与磁学数学理论的人,他的理论为詹姆斯·克拉克·麦克斯韦、威廉·汤姆逊(开尔文勋爵)等科学家的研究奠定了基础。他在势理论方面的工作与卡尔·弗里德里希·高斯的研究是并行发展的。

格林的人生经历极为非凡,因为他几乎完全是自学成才。他在童年时期仅接受过大约一年的正规教育,年龄在8岁到9岁之间。
\subsection{早年生活}
\begin{figure}[ht]
\centering
\includegraphics[width=6cm]{./figures/e426677496c6ac27.png}
\caption{} \label{fig_QZgl_1}
\end{figure}
格林出生并大部分时间生活在英格兰诺丁汉郡的斯尼顿镇,如今该地已划入诺丁汉市。他的父亲也叫乔治,是一位面包师,还建造并拥有一座砖结构的风车,用于碾磨谷物【1】。

在年轻时,格林被形容为体质虚弱,不喜欢在父亲的面包房里干活。然而,他并没有选择的余地;就像那个时代许多孩子一样,他很可能从五岁起便开始每天劳作谋生。
