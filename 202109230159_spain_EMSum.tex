% Euler-Maclaurin求和公式
\pentry{渐近展开\upref{Asympt}}

\subsubsection{Euler求和公式}
\begin{theorem}{}
设函数 $f:X\to Y$ 在区间$[0,+\infty)$上连续可微,则有如下Euler求和公式成立	
        \begin{equation}\label{EMSum_eq1} 
            \sum_{k=1}^{n}f(k)=\int_{0}^{n}f(x)\,\dd x
            +\frac{f(n)-f(0)}{2}+\int_{0}^{n}\psi(x)f'(x)\dd x
        \end{equation}
    其中 $\psi(x)=x-\lfloor x \rfloor-1/2=\{x\}-1/2$
\end{theorem}
证明如下:将区间$[0,n]$划分为长度为$1$的小区间$[k-1,k](k=1,2,\cdots,n)$,则
​\[
    \int_{k-1}^{k}\psi(x)f'(x)\,\dd x
    =\frac{f(k)+f(k-1)}{2}-\int_{k-1}^{k}f(x)\,\dd x
\]
​对$k$从1到$n$求和,于是
​\[
    \int_{0}^{n}\psi(x)f'(x)\,\dd x
    =\sum_{k=1}^{n}f(k)-\frac{f(n)-f(0)}{2}-\int_{0}^{n}f(x)\,\dd x
\]
这样就证明了Euler求和公式.
\\注意到\autoref{EMSum_eq1} 右端积分比较复杂,一般情况下,仅仅估计其上界便可得到较为不错的结果.设函数 $f\in{C^1[0,+\infty)}$ ,再设函数
$\varphi(x)=\displaystyle{\int_{0}^{x}\psi(t)\,\dd t}$,
易知该函数是周期为$1$的函数,且
 $-1/8\leqslant\varphi(x)\leqslant 0$,因此
\[
    \left|\int_{0}^{n}\psi(x)f'(x)\,\dd x\right|
    =\left|\int_{0}^{n}\varphi(x)D^2(f)\mathrm{d}x\right|
    \leqslant\frac{|f'(n)-f'(0)|}{8}
\]
\begin{figure}[ht]
\centering
\includegraphics[width=10cm]{./figures/EMSum_1.pdf}
\caption{函数$\varphi(x)$} \label{EMSum_fig1}
\end{figure}

在实际情况中,更为常见的是函数 $f$ 在$[1,+\infty)$ 上可微,则补充定义$ f\equiv 0,x\in[0,1)$.在 \autoref{EMSum_eq1} 式中令$n=1$,从而
\[
    \int_{0}^{1-\delta}\psi(x)f'(x)\mathrm{d}x
    =\frac{f(1-\delta)+f(0)}{2}-\int_{0}^{1-\delta}f(x)\,\dd x=0
\]
由于令$\delta\to0+0$时,极限存在且为$0$,则对于在区间$[1,+\infty)$连续可微函数$f$就有
\begin{equation}\label{EMSum_eq2}
    \sum_{k=1}^{n}f(k)
    =\int_{1}^{n}f(x)\,\dd x+\frac{f(n)+f(1)}{2}+\int_{1}^{n}\psi(x)f'(x)\,\dd x
\end{equation}
在 \autoref{EMSum_eq2}  式中,若积分$\displaystyle{\int_{1}^{\infty}\psi(x)f'(x)\,\dd x}$收敛,则
\begin{equation}\label{EMSum_eq3}
    \sum_{k=1}^{n}f(k)=\int_{1}^{n}f(x)\,\dd x
    +\frac{f(n)}{2}-\int_{n}^{\infty}\psi(x)f'(x)\,\dd x
    +\int_{1}^{\infty}\psi(x)f'(x)\,\dd x+\frac{f(1)}{2}
\end{equation}
\begin{exercise}{}
Euler-Mascheroni常数\autoref{Masche_eq1}~\upref{Masche}由调和级数与自然对数的差值的极限所给出,利用\autoref{EMSum_eq3} 和该常数可以给出类调和级数前$n$项和较好的计算公式,即
\begin{equation}\label{EMSum_eq4}
    H_{n}=\sum_{k=1}^{n}\frac{1}{k}=\log n+\frac{1}{2n}+C_{2}
    +\int_{n}^{\infty}\frac{\psi(x)}{x^2}\,\dd x
\end{equation}
其中
\begin{equation}\label{EMSum_eq5}
    C_{2}=\frac{1}{2}-\int_{1}^{\infty}\frac{\psi(x)}{x^2}\,\dd x
\end{equation}
由Dirichlet定理可知,\autoref{EMSum_eq5} 积分存在,因此将\autoref{EMSum_eq4} 结合
Euler-Mascheroni常数代入\autoref{EMSum_eq3} 就有$C_{2}=\gamma$,从而
\begin{equation}
    H_{n}=\log n+\frac{1}{2n}+\gamma+\int_{n}^{\infty}\frac{\psi(x)}{x^2}\,\dd x
\end{equation}
\end{exercise}


\subsection{Euler-Maclaurin求和公式}

\begin{definition}{}
设Bernoulli数为 $B:\mathbb{N}\to\mathbb{R} $,它有多种定义方式,其生成函数定义为
    \begin{equation}{}
        \frac{z}{e^{z}-1}=\sum_{n=0}^{\infty}B_{n}\frac{z^{n}}{n!}
        \quad(|z|<2\pi,z\in{\mathbb C})
    \end{equation}
\end{definition}
	
依据Taylor公式和Cauchy乘积,对照系数就有
\begin{equation}
    \sum_{k=0}^{n}\dbinom{n+1}{k}B_{k}=0\quad(n\in{\mathbb N^{+}})
\end{equation}
依据式,计算前10项Bernoulli数(由生成函数定义就有$B_{0}=1$).列举如下:
\begin{table}[ht]
\centering
\caption{Bernoulli数}
\begin{tabular}{|c|c|c|c|c|c|c|c|c|c|c|}
\hline
$ n $ & 0 & 1 & 2 & 3 & 4 & 5 & 6 & 7 & 8 & 9\\ 
\hline
$ B_{n} $ & $1$ & $-\dfrac{1}{2}$ & $ \dfrac{1}{6}$ & $0$ & $-\dfrac{1}{30}$ & $0$ & $\dfrac{1}{42}$ & $0$ & $-\dfrac{1}{30}$ & $0$\\
\hline
\end{tabular}
\end{table}

\begin{definition}{}
设Bernoulli多项式为 $B_{n}:X\to Y$ ,利用生成函数可定义为
\begin{equation}{}
    \frac{ze^{xz}}{e^{z}-1}=\sum_{n=0}^{\infty}B_{n}(x)\frac{z^{n}}{n!}
    \quad(|z|<2\pi,z\in{\mathbb C})
\end{equation}	
\end{definition}
利用Cauchy乘积,并比较系数可得
\begin{equation}
    B_{n}(x)=\sum_{k=0}^{n}\dbinom{n}{k}B_{k}x^{n-k}
\end{equation}
对式求导,故有
\begin{equation}
    \frac{\dd}{\dd x}(B_{n+1}(x))=\sum_{k=0}^{n}\dbinom{n+1}{k}(n-k+1)B_{k}x^{n-k}=(n+1)B_{n}(x)
\end{equation}
两边同时积分,因此
\begin{equation}
    B_{n+1}(x)-B_{n+1}=(n+1)\int_{0}^{x}B_{n}(t)\,\dd t
\end{equation}
不妨记$\bar{B}_{n}(x)=B_{n}(\{x\})$,注意到$\bar{B}_{n}(x)$是一个周期为1的函数,则
\begin{equation}\label{EMSum_eq13}
    \int_{0}^{x}\bar{B}_{n}(t)\mathrm{d}t
    =\lfloor x\rfloor\int_{0}^{1}B_{n}(t)\mathrm{d}t+\int_{0}^{\{x\}}B_{n}(t)\mathrm{d}t
    =\frac{\bar{B}_{n}(x)-B_{n+1}}{n+1}
\end{equation}
应用式,计算前5项Bernoulli多项式,列举如下:
\begin{table}[ht]
\centering
\caption{请输入表格标题}
\begin{tabular}{|c|c|c|c|c|c|}
\hline
$n$ & 0 & 1 & 2 & 3 & 4\\
\hline
$B_{n}(x)$ & $1$ & $x-\dfrac{1}{2}$ & $x^{2}-x+\dfrac{1}{6}$ & $x^{3}-\dfrac{3}{2}x^{2}+\dfrac{1}{2}x$ & $x^{4}-2x^{3}+x^{2}-\dfrac{1}{30}$\\
\hline
\end{tabular}
\end{table}

考虑在区间$[0,1]$上将$B_{2n}(x),n\in{\mathbb N^{+}}$展开为Fourier级数:
\[
    B_{2n}(x)=\frac{a_{0}}{2}+\sum_{m=1}^{\infty}[a_{m}\cos(2m\pi x)+b_{m}\sin(2m\pi x)]
\]
可求得其系数为
\begin{equation}
    \begin{cases}
        a_{0}=0\\
        a_{m}=(-1)^{n+1}\dfrac{2(2n)!}{(2m\pi)^{2n}}\\
        b_{m}=0
    \end{cases}
\end{equation}
于是就有
\begin{equation}\label{EMSum_eq15}
    B_{2n}(x)=2(-1)^{n+1}(2n)!\sum_{k=1}^{\infty}\frac{\cos(2m\pi x)}{(2m\pi)^{2n}},0\leqslant x\leqslant1
\end{equation}
令$x=1$,可推出
\begin{equation}
    \zeta(2n)=\frac{(-1)^{n+1}B_{2n}2^{2n-1}\pi^{2n}}{(2n)!}
\end{equation}

Euler-Maclaurin求和公式是有关定积分的一种数值计算公式,它建立了函数的积分与其导数的联系.在数值积分理论与级数求和法中,Euler-Maclaurin求和公式是一个极有用的工具.

\begin{theorem}{}
    (Euler-Maclaurin求和公式)设$f:X\to Y$在区间$[0,+\infty)$上至少$2K(K\in{\mathbb N})$阶连续可微,则
    \begin{equation}\label{EMSum_eq17}
        \sum_{k=1}^{n}f(k)=\int_{0}^{n}f(x)\mathrm{d}x
        +\frac{f(n)-f(0)}{2}+\sum_{k=1}^{K}\frac{B_{2k}}{(2k)!}[f^{(2k-1)}(n)-f^{(2k-1)}(0)]
        +R_{2K+1}(n)
    \end{equation}
    其中
    \begin{equation}
        R_{2K+1}(n)=\frac{1}{(2K+1)!}\int_{0}^{n}\bar{B}_{2K+1}(x)f^{(2k+1)}\,\dd x
    \end{equation}
\end{theorem}
由\autoref{EMSum_eq13} 可知
    \begin{equation}\label{EMSum_eq19}
        \int_{0}^{n}\bar{B}_{k}(x)f^{(\alpha)}\,\dd x
        =\frac{B_{k+1}}{k+1}[f^{(\alpha)}(n)-f^{(\alpha)}(0)]
        -\frac{1}{k+1}\int_{0}^{n}\bar{B}_{k+1}(x)f^{(\alpha+1)}\,\dd x
    \end{equation}
\autoref{EMSum_eq19} 代入\autoref{EMSum_eq17} 即可得证.

\begin{theorem}{}
    设$f:X\to Y$在区间$[0,+\infty)$无穷阶可微且各阶导函数在其上一致有界,则
    \begin{equation}\label{EMSum_eq20}
        \sum_{k=1}^{n}f(k)=\int_{0}^{n}f(x)\mathrm{d}x+\frac{f(n)-f(0)}{2}
        +\sum_{k=1}^{\infty}\frac{B_{2k}}{(2k)!}[f^{(2k-1)}(n)-f^{(2k-1)}(0)]
    \end{equation}
\end{theorem}
同样地,\autoref{EMSum_eq17} 和\autoref{EMSum_eq20} 可如同\autoref{EMSum_eq2} 进行类似地改写.
只须证明$\lim\limits_{K\to\infty}R_{2K+1}(n)=0$对于任意给定的$n\in{\mathbb N}$都成立.
可设$|f^{(\alpha)}|\leqslant M$,结合\autoref{EMSum_eq15} ,就有
\begin{equation}
        |R_{2K+1}(n)|<\frac{2M(n+2)}{(2\pi)^{2K+2}}
\end{equation}
从而
\[
    \lim\limits_{K\to\infty}R_{2K+1}(n)=0	
\]	


