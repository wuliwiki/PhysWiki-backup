% Mathematica 脚本模式
% keys Mathematica|命令行脚本|wolframscript
% license Usr
% type Tutor

\begin{issues}
\issueDraft
\issueTODO
\end{issues}

\subsection{命令行二进制程序}

Mathematica 有个命令行解释程序 \verb`wolframscript`,可以用来运行用 \verb`Wolfram` 语言编写的脚本。
可以使用\verb`bash`命令查看:
\begin{lstlisting}[language=bash]
file -L $(which wolframscript)
wolframscript --help #查看使用帮助
\end{lstlisting}

\subsection{命令行选项} 

\verb`wolframscript --help` 的输出如下:

\subsubsection{辅助功能} 
\begin{itemize}
\item \verb`-v,-verbose`: 在执行过程中打印附加信息。
\item \verb`-h,-help` : 打印帮助文本。
\item \verb`-version`: 打印 WolframScript 的版本。
\end{itemize}

\subsubsection{运行模式} 
\begin{itemize}
\item \verb`-c,-code WL`: 给出要执行的 Wolfram Language 代码。
\item \verb`-f,-file PATH`: 给出要执行的包含 Wolfram 语言代码的文件。
\item \verb`-a,-api URL|UUID`: 使用指定 \verb`URL` 的 \verb`API`,或来自指定 \verb`UUID` 的云或本地对象。 
在\verb`-args`之后,以 \verb`key=value` 的形式提供参数。
\item \verb`-fun,-function WL`: 使用函数,参数为使用 \verb`-args` 给出的 \verb`字符串`。
并将参数解析为 \verb`-signature` 给出的类型。
\item \verb`--,-args ARGS。。。`: 与 \verb`-api` 或 \verb`-function` 一起使用,以提供参数。
\item \verb`-s,-signature TYPE。。。`: 与 \verb`-function` 和 \verb`-args` 一起使用,为提供的参数指定解释器类型。
\end{itemize}

\subsubsection{运行选项}
\begin{itemize}
\item \verb`-charset ENCODING`: 使用\verb`ENCODING`进行输出。 编码可以是 \verb`None`,用于输出原始字节,
或是 \verb`$CharacterEncodings` 中的任何条目,除了 \verb`Unicode`。 默认情况继承终端的 \verb`语言设置` 中的值。
\item \verb`-format TYPE`: 指定输出的格式。 可以使用 \verb`Export` 所理解的任何格式。
\item \verb`-print [all]`: 当运行 \verb`脚本` 时,打印脚本 \verb`最后一行` 的结果。 如果给了 \verb`all` 参数,则打印 \verb`每一行`。
\item \verb`-linewise`: 执行读取到的 \verb`标准输入` 中的每行代码。
\item \verb`-script ARGS。。。`: 与 \verb`wolfram -script` 相对应,作用是设置 \verb`$ScriptCommandLine`。
\end{itemize}

\subsection{Wolfram内部预定义变量}

\subsubsection{常用调用方式}

建立后缀名为 \verb`.wl` or \verb`.wls` 的文件,然后按平常写 \verb`Mathematica` 笔记本的语法编写脚本。
但最好用 \verb`字符串函数` 代替\verb`Box` 相关的函数,如 \verb`TableForm`。
运行时,用 \verb`wolframscript` 唤起脚本(如 windows 为 \verb`wolframscript.exe`)。

假如当前目录下有脚本 \verb`test.wl`,用类似下面的方式运行:
\begin{lstlisting}[language=bash]
wolframscript -script test.wl \
--Lbd-num '0.90' --Lbd-fit '0.90'  --ord '\$ord0'
\end{lstlisting}

其中 \verb`--Lbd-num '0.90' --Lbd-fit '0.90'  --ord '\$ord0'` 将作为参数提供给脚本 \verb`test.wl`,可在脚本内部使用。
注意 \verb`$ScriptCommandLine` 中的参数,其类型均为 \verb`String`。
也就是命令行接收到的 \verb`参数`,会被强制转码为 \verb`字符串`。
所以上面收到的参数列表为:
\begin{lstlisting}[language=bash]
{"./test.wl","--Lbd-num","0.90","--Lbd-fit","0.90","--ord","$ord0"}
\end{lstlisting}
如果需要使用 \verb`数字`或其他类型的 \verb`值`,需要在脚本内部自行转换(\verb`ToExpression`)。

此外,如果命令行参数中包含 \verb`$`,需要用 \verb`单引号 + 反斜杠`转义(escape)。
例如\verb`$ord` -> \verb`'\$ord0'`。

在 \verb`WolframScript` 开始执行时,会预定义一些变量。
因此在脚本内部,可以从这些 \verb`预设变量` 读取 \verb`shell` 中提供的参数。

\begin{itemize}
\item \verb`$CommandLine`: \verb`字符串列表`,给出运行 \verb`WolframKernel` 使用的 \verb`完整命令行`。
\item \verb`$ScriptCommandLine`: 为 \verb`正运行的脚本` 准备的 \verb`命令行参数列表`。
这些参数出现在 \verb`-option` 给出的选项之后。
\item \verb`$ScriptInputString`: 代表在 \verb`标准输入通道` 上对 \verb`原始操作系统命令` 的输入,
正运行的 Wolfram Language 实例即来自 \verb`此命令的调用`(invoke)
脚本迭代一次,\verb`-linewise` 选项就把 \verb`标准输入` 的一行载入该变量。
可在交互式脚本中使用。
\end{itemize}


在上面的例子中,三个变量被填充如下,不同平台可能稍有不同

\begin{lstlisting}[language=bash]
$CommandLine->{/usr/local/Wolfram/Mathematica/12.2/
SystemFiles/Kernel/Binaries/Linux-x86-64/WolframKernel,
-wlbanner,-script,test.wl,--,test.wl,
--Lbd-num,0.90,--Lbd-fit,0.90,--ord,$ord0}

$ScriptCommandLine->{test.wl,--Lbd-num,0.90,--Lbd-fit,0.90,--ord,ord0}

$ScriptInputString->None
\end{lstlisting}

也可用 \verb`wolfram` 命令运行脚本,但是 \verb`wolfram` 不会设置 \verb`$ScriptCommandLine`,只会填充 \verb`$CommandLine`:

\begin{lstlisting}[language=bash]
wolfram -script test.wl  --Lbd-num '0.90' --Lbd-fit '0.90'  --ord '\$ord0'
\end{lstlisting}

\subsubsection{Unix平台} 

\verb`Unix` 还可以加上 \verb`Shebang` 行。 即在脚本文件 **首行** 添加
\begin{lstlisting}[language=bash]
#!/usr/bin/env wolframscript [其他选项]
\end{lstlisting}

运行的时候,不需要先输入 \verb`wolframscript`,传递参数的方法和上面相同。

\begin{lstlisting}[language=bash]
./test.wl para1 para2
wolframscript -script ./init.wl &>\
~/test/log.txt & # 在后台运行,把输出重定向到日志文件 log.txt
\end{lstlisting}

经常用到的 mma 系统变量,参考

\begin{itemize}
\item \verb`guide/SystemInformation` : mma 系统信息
\item \verb`guide/WolframSystemSetup`: 更一般的系统设置
\item \verb`$InputFileName`: 脚本的绝对路径。
\item \verb`$Notebooks`: 如果是用前端运行的,则为\verb`True`。
\item \verb`$BatchInput`: 输入是否来自批处理
\item \verb`$BatchOutput`: 如果在命令行中输出,则为\verb`True`。
\item \verb`$CommandLine`: 唤醒环境变量所使用的命令行,
\item \verb`$ProcessID`: 进程ID
\item \verb`$ParentProcessID`:
\item \verb`$Username`: 用户的登陆名
\item \verb`Environment["var"]`: 操作系统的环境变量,如\verb`Environment["HOME"]`
\end{itemize}
