% 詹姆斯·查德威克(综述)
% license CCBYNCSA3
% type Wiki

本文根据 CC-BY-SA 协议转载翻译自维基百科\href{https://en.wikipedia.org/wiki/James_Chadwick}{相关文章}。

詹姆斯·查德威克爵士(1891年10月20日-1974年7月24日)是英国核物理学家,因发现中子于1935年获得诺贝尔物理学奖。1941年,他撰写了《MAUD报告》的最终稿,促使美国政府开始认真进行原子弹研究工作。二战期间,他担任英国参与曼哈顿计划团队的负责人。由于在核物理领域的成就,他于1945年在英国被授予爵士荣誉。

查德威克于1911年毕业于曼彻斯特维多利亚大学,在那里师从被誉为“核物理之父”的欧内斯特·卢瑟福。在曼彻斯特,他继续在卢瑟福的指导下学习,并于1913年获得硕士学位。同年,查德威克获得了1851年皇家博览会委员会颁发的1851年研究奖学金。他选择前往柏林在汉斯·盖革手下研究β射线。利用盖革新近研发的盖革计数器,查德威克能够证明β射线产生的是连续谱,而非此前认为的离散谱。第一次世界大战在欧洲爆发时,他仍在德国,之后在鲁勒本拘留营度过了四年。

战争结束后,查德威克追随卢瑟福来到剑桥大学卡文迪许实验室,在卢瑟福的指导下于1921年6月在剑桥冈维尔与凯斯学院获得博士学位。他在卡文迪许实验室担任卢瑟福的助理研究主任超过十年,当时该实验室是世界上最重要的物理研究中心之一,吸引了约翰·考克饶夫特、诺曼·费瑟、马克·奥利芬特等学生前来求学。在发现中子后,查德威克测量了中子的质量。他预见到中子将在抗击癌症中成为一项重要武器。1935年,查德威克离开卡文迪许实验室,成为利物浦大学物理学教授,他改造了陈旧的实验室,并通过安装回旋加速器,将其建设成为核物理研究的重要中心。
\subsection{教育与早年生活}
詹姆斯·查德威克于1891年10月20日出生在柴郡博林顿,\(^\text{[4][5]}\)是纺棉工约翰·约瑟夫·查德威克与家庭女佣安妮·玛丽·诺尔斯的长子。他以祖父詹姆斯的名字命名。1895年,他的父母搬到曼彻斯特,将他留在外祖父母家照顾。他曾就读于博林顿十字小学,并曾获得曼彻斯特文法学校的奖学金,但由于家庭仍需支付少量费用而负担不起,不得不放弃这一机会。随后,他进入曼彻斯特男孩中央文法学校就读,并在那里与父母团聚。当时他已有两个弟弟,哈里和休伯特;他还有一个妹妹在婴儿时期便夭折。16岁时,他参加了两场大学奖学金考试,并双双获得。\(^\text{[6][7]}\)

1908年,查德威克选择进入曼彻斯特维多利亚大学就读。他原本打算学习数学,但误选了物理专业。和大多数学生一样,他住在家里,每天步行往返于家和大学之间,单程4英里(6.4公里)。在第一学年结束时,他获得了赫金伯顿奖学金,用于继续学习物理。当时物理系由欧内斯特·卢瑟福领导,他会给毕业班学生分配研究项目,查德威克被指派设计一种方法,用以比较两种不同放射源的放射能量。卢瑟福的想法是以1克(0.035盎司)镭的放射活度作为测量单位,这一单位后来被称为“居里”。卢瑟福提出的方法不可行——查德威克知道这一点,但不敢告诉卢瑟福——于是他继续尝试,最终设计出了所需的方法。这项结果成为查德威克的第一篇论文,由他与卢瑟福合作,于1912年发表。\(^\text{[8]}\)1911年,他以一等荣誉毕业。\(^\text{[9]}\)