% 大数定律(综述)
% license CCBYSA3
% type Wiki

本文根据 CC-BY-SA 协议转载翻译自维基百科\href{https://en.wikipedia.org/wiki/Law_of_large_numbers}{相关文章}。

在概率论中,大数定律是一条数学规律,它指出:在大量相互独立的随机样本中获得的结果的平均值将收敛于真实值(如果这个真实值存在)【1】。更正式地说,大数定律表明:对于一组独立同分布的样本,其样本均值将趋于真实的数学期望。

大数定律的重要性在于,它为某些随机事件的平均值提供了长期稳定的保证【1】【2】。例如,一个赌场在某次轮盘赌中可能会亏损,但在大量旋转后,其收益将趋近于一个可预测的百分比。任何玩家的连胜最终也会被游戏设定所“拉回”。需要注意的是,大数定律(顾名思义)只在观测次数足够大时才适用。并没有任何原则表明少数几次的观测结果就会接近期望值,或者说某种结果的连发会立即被其他结果“平衡”(参见赌徒谬误)。

大数定律仅适用于重复试验结果的平均值,并声称这个平均值会收敛到期望值;它并不意味着随着试验次数 $n$ 的增加,结果的总和一定会接近 $n$ 倍的期望值。

在其发展历程中,许多数学家对大数定律进行了不断完善。如今,大数定律被广泛应用于统计学、概率论、经济学以及保险学等多个领域【3】。
\subsection{示例}
例如,掷一次六面骰子会得到 1、2、3、4、5 或 6 中的一个结果,每个数字出现的概率相等。因此,这次掷骰子的**期望值**是:
$$
\frac{1 + 2 + 3 + 4 + 5 + 6}{6} = 3.5~
$$
根据**大数定律**,如果掷出大量的六面骰子,这些点数的**平均值**(有时称为**样本均值**)将趋近于 3.5,且随着掷骰数量的增加,结果的**精确度**也会提高。

大数定律还意味着,在一系列**伯努利试验**中,**成功的经验概率**会收敛于**理论概率**。对于伯努利随机变量,其期望值就是成功的理论概率,而 n 个这样的变量的平均值(假设它们是**独立同分布(i.i.d.)**的)恰好就是**相对频率**。
