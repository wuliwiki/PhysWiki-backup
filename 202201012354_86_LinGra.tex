% 线性引力
% keys 线性近似|线性爱因斯坦方程|弱引力场

\begin{issues}
\issueMissDepend
\issueDraft
\end{issues}



\subsection{线性引力理论}
爱因斯坦场方程\footnote{在广义相对论中,常常采用几何单位制\upref{NatUni},也即是$c=G=k_B=1$}如下

\begin{equation}
G_{\mu \nu} = R_{\mu \nu} - \frac{1}{2}g_{\mu\nu}R = 8 G\pi T_{\mu\nu}
\end{equation}

由于方程采用几何语言描述,十分简洁,但它包含着一系列复杂的非线性微分方程.一方面,寻求严格满足爱因斯坦场方程的解是一个漫长而艰难的过程,许多数学天才也投入其中.另一方面,我们试图简化爱因斯坦场方程,因为在大多数情况中引力场都很微弱,我们可以采用近似处理的方式使爱因斯坦场方程线性化,简而言之,就是要对时空进行一阶线性微扰.


\subsection{广义相对论中的时空微扰}

对已知的某一背景时空进行微扰的核心观点非常简单,假设我们可以用下列展开来近似描述真实的物理时空.

\begin{equation}
g_{\mu\nu}=g^{(0)}_{\mu\nu} + \epsilon g^{(1)}_{\mu\nu} + \frac{1}{2!}\epsilon^2 g^{(2)}_{\mu\nu}+\cdots
\end{equation}

其中,$\epsilon $仅作为计算过程中判断各小量阶数的指标,必要的时候可以令 $\epsilon = 1 $,而$g^{(0)}_{\mu\nu} $ 是已知的背景时空,一阶项$g^{(2)}_{\mu\nu}$是线性微扰项.

\subsection{闵氏时空}

我们先考虑背景时空为闵氏时空的简单情况,之后也可将背景时空推广为一般时空.

\begin{equation}
g_{\mu\nu} = \eta_{\mu\nu} + h_{\mu\nu},\quad \abs{h_{\mu\nu}}<<1
\end{equation}

例如,对于太阳系来说,$\abs{h_{\mu\nu}} \sim 10^{-6}$.


\subsection{史瓦西时空}

我们可以将史瓦西时空看作对于平直闵氏时空的微扰.


\subsection{规范不变性}


\subsection{推广到一般时空}

背景时空的选择其实是任意的,我们同样可以对其进行线性微扰.

