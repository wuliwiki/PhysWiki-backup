% 统计物理·微观与宏观之间的桥梁
% 统计物理|科普|微观

物理学史上,人们对大自然的探索主要有两条路线.其中之一,是对物质的结构不断“细分”,通过对微观世界的研究来探索最根本的自然法则.这个思想源自于古希腊时期德谟克里特的“原子论”,随后被发扬光大,对物理学、化学等学科都产生了深远的影响.而在十九世纪相对论与量子力学的革命之后,伴随着加速器技术的发展,人们发现了重子、介子、轻子等形形色色的粒子,在许许多多物理学家的共同努力下,“粒子物理标准模型”诞生,人们对微观世界的认识达到了一个崭新的高度.这一条探索路线上发生过无数次物理学的“革命”,带来了各种各样的惊喜.

另一条路线则与之不同,人们不是仅仅着眼于那些自然法则,而是去挑战复杂系统所特有的物理学性质,从我们所处的宇宙,到我们身边的空气、河流……这纷繁的世界之间似乎有一种遥远的相似性,似乎存在着许许多多重要的关于复杂系统的规律,独立于那些最根本的自然法则.

20世纪,以玻尔兹曼、吉布斯为代表的物理学家们开始对大量粒子系统的微观与宏观之间的联系产生兴趣.