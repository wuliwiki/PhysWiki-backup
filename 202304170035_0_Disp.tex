% 位置矢量、位移
% keys 位移|几何矢量|位置矢量

\pentry{全微分\upref{TDiff},矢量的微分%未完成链接
, 矢量内积\upref{Dot}}

\begin{figure}[ht]
\centering
\includegraphics[width=8cm]{./figures/ad985e632e26959b.pdf}
\caption{位矢与位移} \label{fig_Disp_1}
\end{figure}

\subsection{位矢}

\textbf{位置矢量(位矢)}是从坐标原点 $O$ 指向某一点 $P$ 的矢量\upref{GVec},可记为 $\bvec r$ 或 $\overrightarrow{OP}$。位矢常用于表示坐标系中一点的位置。

有时候可将位矢 $\bvec r$ 作为自变量以表示一个关于位置的函数。例如一个物体内密度关于位置的分布可以表示为 $\rho(\bvec r)$。 在直角坐标系中,就相当于 $\rho(x,y,z)$,在球坐标系\upref{Sph}中就相当于 $\rho(r,\theta,\phi)$ 。这么做的好处是书写简洁,而且不需要指定坐标系的种类。

\subsection{位移}
在物体运动过程中,可以把物体的坐标(以位矢表示)看做时间的矢量函数 $\bvec r=\bvec r(t)$,则\textbf{位移} $\Delta \bvec r$ 是一段时间 $[t_1,t_2]$ 内物体初末位矢的矢量差
\begin{equation}
\Delta \bvec r = \bvec r(t_2) - \bvec r(t_1)
\end{equation}
注意位移只与一段时间内物体的初末位置有关,与路径无关。 由位移的概念可以进一步定义速度和加速度\upref{VnA}。

\begin{example}{证明 $\dd{r} = \uvec r \vdot \dd{\bvec r}$}\label{ex_Disp_1}
这个证明的几何意义是, 位矢模长的微小变化等于位矢的微小变化在位矢方向的投影。

这里以平面直角坐标系中的位矢为例证明。 令位矢 $\bvec r$ 的坐标为 $(x, y)$, 模长为 $r = \sqrt{x^2 + y^2}$,
模长的全微分为
\begin{equation}
\dd{r} = \pdv{r}{x} \dd{x} + \pdv{r}{y} \dd{y} = \frac{x}{\sqrt{x^2 + y^2}} \dd{x} + \frac{y}{\sqrt{x^2 + y^2}} \dd{y}
\end{equation}
考虑到 $x/\sqrt{x^2 + y^2}$ 和 $y/\sqrt{x^2 + y^2}$ 分别为 $\uvec r = \bvec r/r$ 的两个分量, $\dd{x}$ 和 $\dd{y}$ 分别为 $\dd{\bvec r}$ 的两个分量, 根据内积的定义\upref{Dot}上式变为
\begin{equation}
\dd{r} = \uvec r \vdot \dd{\bvec r}
\end{equation}
\end{example}
