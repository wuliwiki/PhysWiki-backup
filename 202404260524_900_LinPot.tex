% 线性势能的定态薛定谔方程
% license Xiao
% type Tutor

\begin{issues}
\issueDraft
\end{issues}

\pentry{定态薛定谔方程\nref{nod_SchEq}, 艾里函数\nref{nod_AiryF}}{nod_4ebf}

\begin{equation}
-\frac{1}{2m}\dv[2]{x}\psi(x) + ax \psi(x) = E\psi(x)~.
\end{equation}
变形为
\begin{equation}
\psi''(x) - 2m(ax - E)\psi(x) = 0~.
\end{equation}
根据 \autoref{eq_AiryF_1}~\upref{AiryF}, 方程的通解为
\begin{equation}
\psi(x) = C_1\opn{Ai}\qty(\frac{\alpha x+\beta}{\abs{\alpha}^{2/3}}) + C_2 \opn{Bi}\qty(\frac{\alpha x+\beta}{\abs{\alpha}^{2/3}})~.
\end{equation}
其中 $\alpha = 2ma$, $\beta = -2mE$。

容易证明, 若把势能改为 $V(x) = a(x-x_0)$, 即向右平移 $x_0$, 那么把波函数也平移 $x_0$ 变为 $\psi(x-x_0)$ 即可。

包络线为(未完成:引用 Ai 相关性质)(符合 WKB 近似)
\begin{equation}
\psi(x) \to \frac{1-1/\E}{[2m(E-ax)]^{1/4}} \E^{\I f(x)}~.
\end{equation}
