% 乔赛亚·威拉德·吉布斯(综述)
% license CCBYSA3
% type Wiki

本文根据 CC-BY-SA 协议转载翻译自维基百科\href{https://en.wikipedia.org/wiki/Josiah_Willard_Gibbs}{相关文章}。

约西亚·威拉德·吉布斯(Josiah Willard Gibbs,/ɡɪbz/,1839年2月11日-1903年4月28日)是一位美国机械工程师和科学家,在物理学、化学和数学领域做出了基础性的理论贡献。他关于热力学应用的研究在将物理化学转变为一门严谨的演绎科学方面起到了关键作用。吉布斯与詹姆斯·克拉克·麦克斯韦和路德维希·玻尔兹曼一起创立了统计力学(该术语由他提出),将热力学定律解释为由大量粒子组成的物理系统可能状态集合的统计特性所导致的结果。吉布斯还研究了麦克斯韦方程在物理光学问题中的应用。作为数学家,他独立于英国科学家奥利弗·赫维赛德(后者在同一时期进行了类似的工作)创建了现代向量分析,并在傅里叶分析理论中描述了“吉布斯现象”。

1863年,耶鲁大学授予吉布斯美国首个工程学博士学位。在欧洲度过三年后,吉布斯将余生的职业生涯都奉献给了耶鲁大学,自1871年起担任数学物理学教授,直到1903年去世。他在相对孤立的环境中工作,成为美国最早获得国际声誉的理论科学家之一,曾被阿尔伯特·爱因斯坦称为“美国历史上最伟大的头脑”。

1901年,吉布斯因其在数学物理方面的贡献,获得了当时国际科学界最高荣誉——由伦敦皇家学会颁发的科普利奖章。

评论家和传记作家们曾指出,吉布斯宁静而孤独的新英格兰生活方式与其思想在国际上的巨大影响之间形成了鲜明对比。尽管他的研究几乎完全是理论性的,但随着20世纪上半叶工业化学的发展,吉布斯成果的实际价值逐渐显现。正如罗伯特·A·密立根所言,在纯科学领域,吉布斯“对于统计力学和热力学的贡献,就如拉普拉斯之于天体力学,麦克斯韦之于电动力学——他几乎将这个领域构建成一个完整的理论体系”。
\subsection{传记}
\subsubsection{家庭背景}
\begin{figure}[ht]
\centering
\includegraphics[width=6cm]{./figures/0453fffcb9e69ba5.png}
\caption{青年时期的威拉德·吉布斯} \label{fig_QSY_1}
\end{figure}
吉布斯出生于康涅狄格州纽黑文。他出身于一个古老的“洋基”家族,自17世纪以来,该家族不断涌现出杰出的美国牧师和学者。他是家中五个孩子中排行第四的孩子,也是父亲约西亚·威拉德·吉布斯与母亲玛丽·安娜(娘家姓范·克里夫,Mary Anna,née Van Cleve)唯一的儿子。在父系方面,他是塞缪尔·威拉德的后代,后者于1701年至1707年间曾担任哈佛学院代理校长。在母系方面,他的一位祖先是乔纳森·迪金森牧师,新泽西学院(后来的普林斯顿大学)首任校长。
“约西亚·威拉德”这个名字在吉布斯家族中代代相传,他与父亲及其他一些家族成员都使用这个名字。它源自他的一位祖先——18世纪曾任马萨诸塞湾省国务秘书的约西亚·威拉德。他的祖母默西·普雷斯科特·吉布斯(Mercy (Prescott) Gibbs)是丽贝卡·米诺特·普雷斯科特·舍曼的妹妹,而后者是美国开国元勋罗杰·舍曼的妻子。因此,吉布斯是舍曼家族的第二代近亲,也与后来涉及“阿米斯塔德号案件”的罗杰·舍曼·鲍德温是堂表亲。

吉布斯的父亲在家庭和学术界通常被称为“约西亚”,而他本人则被称为“威拉德”(Willard)。约西亚·吉布斯是一位语言学家和神学家,自1824年起担任耶鲁大学神学院的圣经文学教授,直到1861年去世。他如今最广为人知的事迹,是作为废奴主义者,在“阿米斯塔德号”事件中为非洲船员找到口译员,使他们能在审判中作证,讲述自己反抗被贩卖为奴的经历。
\subsubsection{教育经历}
威拉德·吉布斯在霍普金斯学校接受教育,并于1854年15岁时进入耶鲁学院。在耶鲁,吉布斯因数学和拉丁语方面的优异成绩而获得奖项,并于1858年以班级前列的成绩毕业。他随后留在耶鲁,成为谢菲尔德科学学院的研究生。19岁时,也就是他刚从本科毕业不久,吉布斯被选入康涅狄格艺术与科学学院,这是一个由耶鲁大学教师为主组成的学术机构。这一时期留下的文献资料相对较少,因此很难精确还原吉布斯早期职业生涯的细节。据传记作者推测,吉布斯在耶鲁大学及康涅狄格学院的主要导师与支持者,很可能是天文学家兼数学家休伯特·安森·牛顿,他是当时研究流星领域的权威,也一直是吉布斯的终生朋友和知己。1861年吉布斯父亲去世后,他继承了一笔足以维持经济独立的遗产。

年轻时期的吉布斯长期受到反复发作的肺部疾病困扰,医生担心他可能容易感染肺结核——他的母亲便因这种病去世。他还患有散光,而当时眼科医生对这种病的治疗尚不熟悉,因此吉布斯不得不自行诊断,并亲自打磨适合自己的眼镜镜片。虽然在后来,他只在阅读或从事近距离工作时才佩戴眼镜,但他体质虚弱以及视力不佳,很可能是他在1861至1865年的南北战争期间没有主动参军的原因。他也未被征召入伍,而是一直留在耶鲁大学度过了整个战时期间。

1863年,吉布斯获得了美国授予的首个工程学博士(PhD)学位,其论文题为《论直齿轮中齿的形状》,他在其中运用几何方法研究齿轮的最优设计。耶鲁大学在1861年成为美国首所提供博士学位的大学,而吉布斯的博士学位是美国在所有学科中授予的第五个博士学位。
\begin{figure}[ht]
\centering
\includegraphics[width=6cm]{./figures/20f35aef807fa103.png}
\caption{吉布斯在耶鲁担任讲师期间的照片\(^\text{[17]}\)} \label{fig_QSY_2}
\end{figure}
\subsubsection{职业生涯,1863–1873年}
获得博士学位后,吉布斯被任命为耶鲁学院的讲师,任期三年。前两年他教授拉丁语,第三年则讲授“自然哲学”(即物理学)。

1866年,吉布斯获得了一项铁路制动器的专利,并在康涅狄格艺术与科学学院发表了一篇题为《长度单位的适当量级》的论文,提出了一套使力学中所使用的度量单位系统更加合理化的方案。

讲师任期结束后,吉布斯与两位姐妹一同前往欧洲旅行。1866至1867年的冬季,他们在巴黎度过,吉布斯在那里听取了索邦大学和法兰西学院的讲座,授课人包括著名数学科学家约瑟夫·刘维尔和米歇尔·沙尔。

由于学习强度过大,吉布斯染上重感冒,医生担心他患上肺结核,建议他前往里维埃拉休养。他与姐妹在那里待了几个月,最终完全康复。

随后,吉布斯前往柏林,听取了数学家卡尔·魏尔施特拉斯和利奥波德·克罗内克以及化学家海因里希·古斯塔夫·马格努斯的课程。1867年8月,吉布斯的妹妹朱莉娅在柏林与亚迪森·范·内姆结婚,后者是吉布斯在耶鲁的同班同学。新婚夫妇返回纽黑文,而吉布斯与妹妹安娜继续留在德国。

在海德堡,吉布斯接触到了物理学家古斯塔夫·基尔霍夫、赫尔曼·冯·亥姆霍兹(Hermann von Helmholtz)以及化学家罗伯特·本生(Robert Bunsen)的研究成果。当时,德国学术界在自然科学领域,尤其是在化学与热力学方面处于世界领先地位。

1869年6月,吉布斯返回耶鲁,并短期教授工程学生法语。据推测,他也大约在此时设计了一种新的蒸汽机调速器,这是他在机械工程领域最后一次重要研究。

1871年,吉布斯被任命为耶鲁大学的数学物理学教授,这是美国首个此类教授职位。当时的吉布斯有经济独立来源,尚未发表任何著作,因此学校安排他只教授研究生课程,并以无薪身份聘用他。
\subsubsection{职业生涯,1863–1873}
毕业后,吉布斯被任命为耶鲁学院的讲师,任期三年。前三年中,他前两年教授拉丁语,第三年教授“自然哲学”(即物理学)。1866年,吉布斯为一种铁路制动器的设计申请了专利,并在康涅狄格艺术与科学学院宣读了一篇题为《长度单位的适当量级》的论文,提出了一个使力学中所用度量单位体系更为合理化的方案。

讲师任期结束后,吉布斯与两位姐妹前往欧洲旅行。1866至1867年的冬季,他们在巴黎度过,吉布斯在那里听取了索邦大学和法兰西学院的讲座,授课者包括著名数学科学家约瑟夫·刘维尔和米歇尔·沙尔。由于学习强度极大,吉布斯染上重感冒,医生担心他患上肺结核,建议他前往里维埃拉(地中海沿岸)休养。他与姐妹在那里度过了几个月,最终完全康复。

随后,吉布斯前往柏林,听取了数学家卡尔·魏尔施特拉斯、利奥波德·克罗内克以及化学家海因里希·古斯塔夫·马格努斯的课程。1867年8月,吉布斯的妹妹朱莉娅在柏林与亚迪森·范·内姆结婚,后者是吉布斯在耶鲁的同班同学。新婚夫妇返回纽黑文,吉布斯与妹妹安娜则继续留在德国。在海德堡,吉布斯接触到物理学家古斯塔夫·基尔霍夫和赫尔曼·冯·亥姆霍兹以及化学家罗伯特·本生的研究。当时,德国学界在自然科学,尤其是化学与热力学方面处于世界领先地位。

1869年6月,吉布斯返回耶鲁,曾短期为工程专业学生教授法语。据推测,他也大约在这段时间设计了一种新的蒸汽机调速器,这是他在机械工程领域的最后一项重要研究。1871年,吉布斯被任命为耶鲁大学数学物理学教授,这是美国历史上首个此类教授职位。当时吉布斯有经济来源,尚未发表任何学术论文,因此学校安排他专门教授研究生课程,并以无薪身份聘任他。
\subsubsection{职业生涯,1873–1880 年}
吉布斯于1873年发表了他的第一部作品。他关于热力学量几何表示法的论文刊登在《康涅狄格艺术与科学学院学报》上。这些论文引入了不同类型的相图,这些图形工具是他在研究过程中最喜欢的想象辅助工具——相比之下,他并不依赖麦克斯韦在构建其电磁理论时所使用的机械模型,因为后者可能无法完全代表所对应的物理现象。尽管该期刊的读者很少有人能够理解吉布斯的研究内容,他仍将论文的抽印本寄给欧洲的同行,其中一位是剑桥大学的詹姆斯·克拉克·麦克斯韦,后者对此热情回应。麦克斯韦甚至亲手用黏土制作了一个模型,以展示吉布斯的构造。他随后还制作了两个石膏模型,其中一个寄送给了吉布斯。该石膏模型如今陈列在耶鲁大学物理系。

麦克斯韦在1875年出版的新版《热学理论》中专门增设了一章介绍吉布斯的研究成果。他在伦敦化学学会的一次讲座中讲解了吉布斯图示方法的实用性,并在他为《大英百科全书》撰写的“图示”条目中也提到了这项工作。然而,他与吉布斯可能展开合作的希望随着麦克斯韦于1879年英年早逝(享年48岁)而终结。纽黑文后来流传起这样一句玩笑话:“世上只有一个人能看懂吉布斯的论文,那就是麦克斯韦——可他已经去世了。”

随后,吉布斯将他的热力学分析扩展到多相化学系统(即包含多种物质形态的系统),并探讨了各种具体应用。他将这些研究成果整理为一篇题为《非均相物质的平衡》的专著,由康涅狄格艺术与科学学院出版,分两部分分别于1875年和1878年发表。这部作品长约三百页,共包含整整七百个编号的数学方程。开篇引用了鲁道夫·克劳修斯的一句名言,这句话后来被视为热力学第一定律和第二定律的简明表述:“世界的能量是恒定的;世界的熵趋于最大。”

吉布斯的这部专著以严谨而巧妙的方式将他的热力学技术应用于物理化学现象的解释,将原本零散的事实与观察加以统一并相互关联。这部作品被誉为“热力学的《原理》”,也是一部“几乎没有限制的巨著”。它坚实地奠定了物理化学的基础。

将该专著翻译成德文的威廉·奥斯特瓦尔德称吉布斯为“化学能学的奠基人”。据现代评论者指出:

“这一著作的发表被公认为化学史上具有首要意义的事件……尽管如此,这部作品的价值在相当多年后才广泛为人所知。这种延迟主要源于其高度数学化的表达形式与严密的演绎过程,使得阅读极为困难,尤其对那些恰恰最相关的实验化学学生而言更是如此。”
——J. J. O'Connor 与 E. F. Robertson,1997年

吉布斯一直无薪工作,直到1880年马里兰州巴尔的摩的新成立的约翰斯·霍普金斯大学向他提供了一份年薪3,000美元的职位。作为回应,耶鲁大学为他开出每年2,000美元的薪资,吉布斯对此表示满意并接受了。

1879年,吉布斯推导出了吉布斯–阿佩尔运动方程,该方程在1900年被保罗·埃米尔·阿佩尔重新发现。

