% 麦克斯韦-玻尔兹曼统计(综述)
% license CCBYNCSA3
% type Wiki

本文根据 CC-BY-SA 协议转载翻译自维基百科\href{https://en.wikipedia.org/wiki/Maxwell\%E2\%80\%93Boltzmann_statistics}{相关文章}。

在统计力学中,麦克斯韦–玻尔兹曼统计描述了经典物质粒子在热平衡中不同能级上的分布。当温度足够高或粒子密度足够低,以至于量子效应可以忽略不计时,这一统计方法适用。
\begin{figure}[ht]
\centering
\includegraphics[width=10cm]{./figures/edce00b1e7050d6f.png}
\caption{麦克斯韦–玻尔兹曼统计可用于推导理想气体中粒子速度的麦克斯韦–玻尔兹曼分布。展示了:在 -100°C、20°C 和 600°C 下,106 个氧分子的速度分布。} \label{fig_MBTJ_1}
\end{figure}
对于麦克斯韦–玻尔兹曼统计,能量为\(\varepsilon_i\)的粒子的期望数为:
\[
\langle N_i \rangle = \frac{g_i}{e^{(\varepsilon_i - \mu) / kT}} = \frac{N}{Z} g_i e^{-\varepsilon_i / kT}~
\]
其中:
\begin{itemize}
\item \(\varepsilon_i\)是第\( i \)个能级的能量,
\item \( \langle N_i \rangle \)是能量为\( \varepsilon_i \)的状态集合中粒子的平均数,
\item \( g_i \)是第\( i \)个能级的简并度,即具有能量 \( \varepsilon_i \)的状态的数量,这些状态可以通过其他方式加以区分[注1],
\item \( \mu \)是化学势,
\item \( k \)是玻尔兹曼常数,
\item \( T \)是绝对温度,
\item \( N \)是总粒子数:
  \[
  N = \sum_i N_i~
  \]
\item \( Z \) 是配分函数:
  \[
  Z = \sum_i g_i e^{-\varepsilon_i / kT}~
  \]
\item \( e \) 是欧拉数。
\end{itemize}
等效地,粒子数有时表示为:
\[
\langle N_i \rangle = \frac{1}{e^{(\varepsilon_i - \mu) / kT}} = \frac{N}{Z} e^{-\varepsilon_i / kT}~
\]
其中,索引\( i \)现在指定了一个特定的状态,而不是所有能量为\( \varepsilon_i \)的状态集合,并且配分函数\( Z \)由以下式子给出:
\[
Z = \sum_i e^{-\varepsilon_i / kT}~
\]
\subsection{历史}  
麦克斯韦–玻尔兹曼统计源于麦克斯韦–玻尔兹曼分布,很可能是对基础技术的提炼。该分布最早由麦克斯韦于1860年根据启发式方法推导出来。后来,玻尔兹曼在1870年代对该分布的物理起源进行了重要研究。该分布可以通过最大化系统熵的原理推导出来。
\subsection{与麦克斯韦–玻尔兹曼分布的关系}  
麦克斯韦–玻尔兹曼分布和麦克斯韦–玻尔兹曼统计密切相关。麦克斯韦–玻尔兹曼统计是统计力学中的一个更为一般的原理,用于描述经典粒子处于特定能量状态的概率:
\[
P_i = \frac{e^{-E_i / k_B T}}{Z}~
\]
其中:
\begin{itemize}
\item \( Z \) 是配分函数:
  \[
  Z = \sum_i e^{-E_i / k_B T}~
  \]
\item \( E_i \) 是状态 \( i \) 的能量,
\item \( k_B \) 是玻尔兹曼常数,
\item \( T \) 是绝对温度。
\end{itemize}
麦克斯韦–玻尔兹曼分布是麦克斯韦–玻尔兹曼统计在气体粒子动能上的具体应用。理想气体中粒子速度(或速率)的分布可以从统计假设推导得出,该假设认为气体分子的能级由其动能给出:
\[
f(v) = \left( \frac{m}{2 \pi k_B T} \right)^{3/2} 4 \pi v^2 e^{-\frac{m v^2}{2 k_B T}}~
\]
其中:
\begin{itemize}
\item \( f(v) \)是粒子速度的概率密度函数,
\item \( m \)是粒子的质量,
\item \( k_B \)是玻尔兹曼常数,
\item \( T \)是绝对温度,
\item \( v \)是粒子的速度。
\end{itemize}
\subsubsection{推导}
我们可以从麦克斯韦–玻尔兹曼统计推导出麦克斯韦–玻尔兹曼分布,首先从麦克斯韦–玻尔兹曼的能级概率出发,并将动能\( E = \frac{1}{2} m v^2 \)代入,来将概率表示为速度的函数:
\[
P(E) = \frac{1}{Z} \exp \left( \frac{-E}{k_B T} \right) \rightarrow P(v) = \frac{1}{Z} \exp \left( \frac{-m v^2}{2 k_B T} \right)~
\]
在三维空间中,这与球面面积成正比,\( 4 \pi v^2 \)。因此,速度\( v \)的概率密度函数(PDF)变为:
\[
f(v) = C \cdot 4 \pi v^2 \exp \left( -\frac{m v^2}{2 k_B T} \right)~
\]
为了找到归一化常数\( C \),我们要求概率密度函数对所有可能速度的积分为 1:
\[
\int_0^\infty f(v) dv = 1 \quad \rightarrow \quad C \int_0^\infty 4 \pi v^2 \exp \left( -\frac{m v^2}{2 k_B T} \right) dv = 1~
\]
通过使用已知的积分结果:
\[
\int_0^\infty v^2 e^{-a v^2} dv = \frac{\sqrt{\pi}}{4 a^{3/2}}~
\]
其中 \( a = \frac{m}{2 k_B T} \),我们得到:
\[
C \cdot 4 \pi \cdot \frac{\sqrt{\pi}}{4 \left( \frac{m}{2 k_B T} \right)^{3/2}} = 1 \quad \rightarrow \quad C = \left( \frac{m}{2 \pi k_B T} \right)^{3/2}~
\]
因此,\textbf{麦克斯韦–玻尔兹曼速度}分布为:
\[
f(v) = \left( \frac{m}{2 \pi k_B T} \right)^{3/2} 4 \pi v^2 \exp \left( -\frac{m v^2}{2 k_B T} \right)~
\]
\subsection{适用性}
\begin{figure}[ht]
\centering
\includegraphics[width=10cm]{./figures/b1e4965e444d5eff.png}
\caption{} \label{fig_MBTJ_2}
\end{figure}
麦克斯韦–玻尔兹曼统计用于推导理想气体的麦克斯韦–玻尔兹曼分布。然而,它也可以用于将该分布扩展到具有不同能量–动量关系的粒子,如相对论性粒子(从而得到麦克斯韦–尤特纳分布),以及扩展到非三维空间。

麦克斯韦–玻尔兹曼统计通常被描述为“可区分”的经典粒子的统计方法。换句话说,粒子\(A\)处于状态 1 且粒子\(B\)处于状态 2 的配置与粒子\(B\)处于状态 1 且粒子\(A\)处于状态 2 的配置是不同的。这一假设导致了粒子在能量状态中的正确(玻尔兹曼)统计,但对于熵会产生不符合物理的结果,正如吉布斯悖论中所体现的那样。

同时,实际上没有任何粒子具备麦克斯韦–玻尔兹曼统计所要求的特征。实际上,如果我们将某一类型的所有粒子(例如电子、质子、光子等)视为本质上不可区分的,那么吉布斯悖论就得以解决。一旦做出这一假设,粒子统计就发生了变化。在混合熵的例子中,熵的变化可以视为一个非广延熵的例子,这种变化源于两种类型粒子混合时的可区分性。

量子粒子要么是玻色子(遵循玻色–爱因斯坦统计),要么是费米子(遵循费米–狄拉克统计,受保利排斥原理的约束)。这两种量子统计在高温和低粒子密度的极限下都会趋近于麦克斯韦–玻尔兹曼统计。
\subsection{推导}  
麦克斯韦–玻尔兹曼统计可以在各种统计力学热力学集合中推导得出:\(^\text{[1]}\)
\begin{itemize}
\item 从大正则集合精确推导。
\item 从正则集合精确推导。
\item 从微正则集合推导,但仅在热力学极限下有效。
\end{itemize}
在每种情况下,都必须假设粒子是非相互作用的,并且多个粒子可以占据相同的状态,并且是独立的。
\subsubsection{从微正则集合的推导}
假设我们有一个容器,其中包含大量具有相同物理特征(如质量、电荷等)非常小的粒子。我们称这个容器为系统。假设尽管这些粒子具有相同的属性,但它们是可区分的。例如,我们可以通过持续观察它们的轨迹,或通过在每个粒子上做标记(例如,在每个粒子上画上不同的数字,就像在彩票球上做的那样)来识别每个粒子。

这些粒子在容器内朝各个方向高速运动。由于粒子快速运动,它们拥有一定的能量。麦克斯韦–玻尔兹曼分布是一个数学函数,用来描述容器中有多少粒子具有某一特定能量。更准确地说,\textbf{麦克斯韦–玻尔兹曼分布}给出了与特定能量对应的状态被占据的非归一化概率(这意味着这些概率的总和不等于1)。

通常,可能有许多粒子具有相同的能量\( \varepsilon \)。设具有相同能量\( \varepsilon_1\)的粒子数为\( N_1 \),具有另一个能量\(\varepsilon_2 \)的粒子数为\(N_2\),依此类推,对于所有可能的能量\(\{\varepsilon_i \mid i = 1, 2, 3, \dots\}\)。为了描述这种情况,我们说\( N_i \)是能级 \(i\)的占据数。

如果我们知道所有的占据数\(\{N_i \mid i = 1, 2,3,\dots\}\),那么我们就知道系统的总能量。然而,因为我们可以区分哪些粒子占据了每个能级,所以占据数集合\(\{N_i \mid i = 1, 2, 3, \dots\}\)并不能完全描述系统的状态。为了完全描述系统的状态,或者说微观状态,我们必须准确地指定哪些粒子在每个能级上。因此,当我们计算系统可能的状态数时,我们必须计算每一个微观状态,而不仅仅是可能的占据数集合。

首先,假设每个能级 \( i \) 只有一个状态(没有简并)。接下来是一些组合学思考,这与准确描述粒子库的状态关系不大。例如,假设总共有 \( k \) 个盒子,标记为 \( a, b, \dots, k \)。使用组合的概念,我们可以计算将 \( N \) 个粒子排列到这些盒子中的方式数,其中每个盒子内球的顺序不被追踪。首先,我们从 \( N \) 个粒子中选择 \( N_a \) 个放入盒子 \( a \),然后继续为每个盒子从剩余的粒子中选择,确保每个粒子都放入一个盒子中。球的排列方式的总数是:
\[
W = \frac{N!}{N_a!(N - N_a)!} \times \frac{(N - N_a)!}{N_b!(N - N_a - N_b)!} \times \frac{(N - N_a - N_b)!}{N_c!(N - N_a - N_b - N_c)!} \times \cdots \times \frac{(N - \cdots - N_\ell)!}{N_k!(N - \cdots - N_\ell - N_k)!}~
\]
简化后得到:
\[
W = \frac{N!}{N_a! N_b! N_c! \cdots N_k!(N - N_a - \cdots - N_\ell - N_k)!}~
\]
由于每个球都已放入一个盒子中,\((N - N_a - N_b - \cdots - N_k)! = 0! = 1\),我们可以简化表达式为:
\[
W = N! \prod_{\ell = a,b,\ldots}^{k} \frac{1}{N_{\ell}!}~
\]
这正是多项式系数,表示将\( N \)个物品排列到\( k \)个盒子中的方式数,第\( l \)个盒子包含\( N_l \)个物品,忽略每个盒子内物品的排列。

现在,考虑有多种方法将 \( N_i \) 个粒子放入盒子 \( i \) 的情况(即考虑简并性问题)。如果第 \( i \) 个盒子有一个“简并度” \( g_i \),即它有 \( g_i \) 个“子盒子”(即有 \( g_i \) 个能量为 \( \varepsilon_i \) 的盒子,这些具有相同能量的状态/盒子被称为简并状态),那么每种填充第 \( i \) 个盒子的方法,其中子盒子中的数量发生变化,都是一种不同的填充方式。因此,填充第 \( i \) 个盒子的方式数必须增加 \( N_i \) 个物体在 \( g_i \) 个“子盒子”中的分布方式数。

将\(N_i\)个可区分的物体放入\(g_i\)个“子盒子”中的方式数是\(g_i^{N_i} \)(第一个物体可以放入任意一个\(g_i\)个盒子中,第二个物体也可以放入任意一个\(g_i\)个盒子中,依此类推)。因此,将总共\( N \)个粒子根据能量分类到能级中,并且每个能级\(i\) 有\( g_i\)个不同状态,使得第\(i\)个能级容纳\( N_i \)个粒子的方式数\(W\)为:
\[
W = N! \prod_{i} \frac{g_i^{N_i}}{N_i!}~
\]
这是玻尔兹曼首次推导出的\(W\)的形式。博尔兹曼的基本方程:\(S = k \ln W\)将热力学熵\( S \)与微观状态数\( W \)相关联,其中\( k \)是玻尔兹曼常数。然而,吉布斯指出,上述\( W \)的表达式并不产生一个广延熵,因此是有缺陷的。这个问题被称为吉布斯悖论。问题在于,上述方程考虑的粒子是可区分的。换句话说,对于位于两个能量子级的两个粒子(\(A\)和\(B\)),由\([A, B]\)表示的人群被认为与由\([B, A]\)表示的人群不同,而对于不可区分的粒子,它们是不一样的。如果我们对不可区分的粒子进行推导,我们会得到 Bose–Einstein 形式的\(W\):
\[
W = \prod_{i} \frac{(N_i + g_i - 1)!}{N_i!(g_i - 1)!}~
\]
当温度远高于绝对零度时,麦克斯韦-玻尔兹曼分布可以从玻色-爱因斯坦分布中推导出来,这意味着\( g_i \gg 1 \)。麦克斯韦-玻尔兹曼分布还要求低密度,这意味着\( g_i \gg N_i \)。在这些条件下,我们可以使用斯特林近似来计算阶乘:
\[
N! \approx N^N e^{-N}~
\]
因此可以写成:
\[
W \approx \prod_{i} \frac{(N_i + g_i)^{N_i + g_i}}{N_i^{N_i} g_i^{g_i}} \approx \prod_{i} \frac{g_i^{N_i} (1 + N_i / g_i)^{g_i}}{N_i^{N_i}}~
\]
利用\( (1 + N_i / g_i)^{g_i} \approx e^{N_i} \)(对于 \( g_i \gg N_i \))的事实,我们可以再次使用 Stirling 近似,得到:
\[
W \approx \prod_{i} \frac{g_i^{N_i}}{N_i!}~
\]
这本质上是对玻尔兹曼原始表达式中的\( N! \)进行除法操作,这种修正被称为正确的玻尔兹曼计数。

我们希望找到使函数\(W\)最大化的\(N_i\),同时考虑到固定的粒子数量\((N = \sum N_i)\)和固定的能量 \((E = \sum N_i \varepsilon_i)\) 的约束条件。\(W\)和\(\ln(W)\)的最大值是由相同的\(N_i\)值实现的,并且由于数学上更容易操作,我们将最大化后者的函数。我们通过拉格朗日乘子法约束我们的解,构造如下的函数:
\[
f(N_1, N_2, \dots, N_n) = \ln(W) + \alpha (N - \sum N_i) + \beta (E - \sum N_i \varepsilon_i)~
\]
\[
\ln W = \ln \left[\prod_{i=1}^{n} \frac{g_i^{N_i}}{N_i!}\right] \approx \sum_{i=1}^{n} \left(N_i \ln g_i - N_i \ln N_i + N_i \right)~
\]
最后,
\[
f(N_1, N_2, \dots, N_n) = \alpha N + \beta E + \sum_{i=1}^{n} \left( N_i \ln g_i - N_i \ln N_i + N_i - (\alpha + \beta \varepsilon_i) N_i \right)~
\]
为了最大化上述表达式,我们应用费马定理(驻点),根据该定理,如果存在局部极值,则必须位于临界点(偏导数为零):
\[
\frac{\partial f}{\partial N_i} = \ln g_i - \ln N_i - (\alpha + \beta \varepsilon_i) = 0~
\]
通过解上述方程(\(i = 1 \ldots n\)),我们得到关于 \(N_i\) 的表达式:
\[
N_i = \frac{g_i}{e^{\alpha + \beta \varepsilon_i}}~
\]
将该表达式代入\(\ln W\)的方程,并假设\(N \gg 1\),我们得到:
\[
\ln W = (\alpha + 1)N + \beta E~
\]
或者重新排列为:
\[
E = \frac{\ln W}{\beta} - \frac{N}{\beta} - \frac{\alpha N}{\beta}~
\]
认识到这只是热力学基本方程的欧拉积分形式的一个表达式。将 \(E\) 识别为内能,欧拉积分基本方程表明:
\[
E = TS - PV + \mu N~
\]
其中,\(T\) 是温度,\(P\) 是压力,\(V\) 是体积,\(\mu\) 是化学势。玻尔兹曼方程 \(S = k \ln W\) 实际上是熵与 \(\ln W\) 成正比,比例常数为玻尔兹曼常数的认识。利用理想气体状态方程(\(PV = NkT\)),可以立即得出:\(
\beta = \frac{1}{kT} \quad \text{和} \quad \alpha = -\frac{\mu}{kT}\)因此,粒子数目可以写为:
\[
N_i = \frac{g_i}{e^{(\varepsilon_i - \mu) / (kT)}}~
\]
注意,上面的公式有时也可以写成:
\[
N_i = \frac{g_i}{e^{\varepsilon_i / kT} / z}~
\]
其中,\(z = \exp(\mu / kT)\) 是绝对活度。

另外,我们可以利用以下事实:
\[
\sum_i N_i = N~
\]
来得到粒子数目为:
\[
N_i = N \frac{g_i e^{-\varepsilon_i / kT}}{Z}~
\]
其中,\(Z\)是配分函数,定义为:
\[Z = \sum_i g_i e^{-\varepsilon_i / kT}~\]
在一个近似中,其中 \(\varepsilon_i\) 被认为是连续变量,托马斯–费米近似给出了一个与 \(\varepsilon\) 成正比的连续简并度 \(g\),从而得到:
\[
\frac{{\sqrt {\varepsilon }} \, e^{-\varepsilon / kT}}{\int_0^\infty {\sqrt {\varepsilon }} \, e^{-\varepsilon / kT}} = \frac{{\sqrt {\varepsilon }} \, e^{-\varepsilon / kT}}{\frac{{\sqrt {\pi}}}{2} (kT)^{3/2}} = \frac{{2 \sqrt {\varepsilon }} \, e^{-\varepsilon / kT}}{\sqrt {\pi (kT)^3}}~
\]
这就是能量的麦克斯韦–玻尔兹曼分布。
\subsubsection{从正则系综推导}
在上述讨论中,玻尔兹曼分布函数是通过直接分析系统的多重性得到的。另一种方法是利用正则系综。在正则系综中,系统与一个热库处于热接触状态。虽然系统与热库之间可以自由交换能量,但假设热库具有无限大的热容量,从而保持整个系统的温度 \(T\)恒定。

在当前的情境下,假设我们的系统具有能级\( \varepsilon_i \)和简并度 \( g_i \)。与之前一样,我们希望计算系统具有能量\( \varepsilon_i \)的概率。

如果我们的系统处于状态\(s_1\),那么热库对应的可用微态数量为\(\Omega_R(s_1)\)。根据假设,整个系统(包括我们感兴趣的系统和热库)是孤立的,因此所有的微态都是等可能的。因此,例如,如果\(\Omega_R(s_1) = 2 \Omega_R(s_2)\),我们可以得出结论,系统处于状态\(s_1 \)的可能性是状态\(s_2\)的两倍。一般来说,如果\(P(s_i)\)是系统处于状态\(s_i\)的概率, 则有:
\[
\frac{P(s_1)}{P(s_2)} = \frac{\Omega_R(s_1)}{\Omega_R(s_2)}.~
\]
由于热库的熵\( S_R = k \ln \Omega_R \),上述关系变为:
\[
\frac{P(s_1)}{P(s_2)} = \frac{e^{S_R(s_1)/k}}{e^{S_R(s_2)/k}} = e^{(S_R(s_1) - S_R(s_2))/k}.~
\]
接下来,我们回顾热力学恒等式(来自\textbf{热力学第一定律}):
\[
dS_R = \frac{1}{T}(dU_R + P\,dV_R - \mu\,dN_R).~
\]
在经典系综中,没有粒子的交换,因此\( dN_R \)项为零。同样,\( dV_R = 0 \)。这给出了:
\[
S_R(s_1) - S_R(s_2) = \frac{1}{T} \left( U_R(s_1) - U_R(s_2) \right) = - \frac{1}{T} \left( E(s_1) - E(s_2) \right),~
\]
其中\( U_R(s_i) \)和\( E(s_i) \)分别表示热库和系统在状态\( s_i \)时的能量。对于第二个等式,我们使用了能量守恒。代入第一个关于\( P(s_1), P(s_2) \)的方程:
\[
\frac{P(s_1)}{P(s_2)} = \frac{e^{-E(s_1)/kT}}{e^{-E(s_2)/kT}},~
\]
这意味着对于系统的任何状态\( s \),有:
\[
P(s) = \frac{1}{Z} e^{-E(s)/kT},~
\]
其中\( Z \)是一个适当选择的“常数”,用于使总概率为1。(如果温度\( T \)不变,\( Z \)是常数。)
\[
Z = \sum_s e^{-E(s)/kT}.~
\]