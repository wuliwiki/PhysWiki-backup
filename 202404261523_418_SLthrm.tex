% 施图姆—刘维尔理论
% license Xiao
% type Tutor

\begin{issues}
\issueDraft
\end{issues}

\pentry{二阶常系数齐次微分方程\nref{nod_Ode2}}{nod_c960}

\textbf{施图姆—刘维尔定理(Sturm–Liouville theorem)} 简称施—刘定理或 S-L 定理。提供了一种找到一类正交函数集,使得能将一个函数展开为这一类正交函数集所构成的级数的方法。

\begin{theorem}{施图姆—刘维尔定理}
微分方程
\begin{equation}
\dv{x}\qty[p(x)\dv{y}{x}] + q(x) y = -\lambda w(x) y~.
\end{equation}
其中 $w(x)$ 又被称为权函数,又写作 $\rho(x)$。
\addTODO{不同本征值的解正交, 零点的数量, 等}
\end{theorem}

研究 S-L 定理前,模仿向量、我们先讨论函数有关的一些定义。
\begin{definition}{函数的内积}

\end{definition}
 
应用: 定态薛定谔方程(束缚态) 为什么不能应用到散射态?
