% 詹姆斯·克拉克·麦克斯韦(综述)
% license CCBYSA3
% type Wiki

本文根据 CC-BY-SA 协议转载翻译自维基百科\href{https://en.wikipedia.org/wiki/James_Clerk_Maxwell}{相关文章}。
\begin{figure}[ht]
\centering
\includegraphics[width=6cm]{./figures/86970eb17e0c5fea.png}
\caption{Maxwell, 大约 1870 年代} \label{fig_Clerk_1}
\end{figure}
詹姆斯·克拉克·麦克斯韦(James Clerk Maxwell FRS FRSE,1831年6月13日-1879年11月5日)是一位苏格兰物理学家和数学家[1],他提出了电磁辐射的经典理论,这是第一个将电、磁和光视为同一现象的不同表现形式的理论。麦克斯韦的电磁学方程实现了物理学中的第二次伟大统一[2],第一次统一则由艾萨克·牛顿实现。麦克斯韦在统计力学的创立中也起到了关键作用[3][4]。

1865年,麦克斯韦发表了《电磁场的动力学理论》,在该文中他证明了电场和磁场作为波动通过空间传播,传播速度与光速相同。他提出光是同一介质中的波动,这种介质也是电磁现象的来源[5]。光与电现象的统一促使他预言了无线电波的存在,并且这篇论文包含了他自1856年以来一直在研究的方程的最终版本[6]。由于他的方程以及他在解决网络问题和线性导体问题上提出的有效方法,他被视为现代电气工程学科的奠基人之一[7]。1871年,麦克斯韦成为第一任卡文迪许物理学教授,并一直担任此职直到1879年去世。

麦克斯韦是第一个推导出麦克斯韦–玻尔兹曼分布的人,这是一种描述气体动理论各个方面的统计方法,他在职业生涯中断断续续地研究了这一课题[8]。他还因于1861年展示了第一张持久的彩色照片而闻名,并且在分析杆接框架(如许多桥梁中的桁架)的刚性方面做出了基础性贡献。麦克斯韦帮助建立了CGS测量系统[9],并且他对现代尺寸分析的贡献也不可忽视[10][11]。麦克斯韦还因奠定了混沌理论的基础而受到认可[12][13]。他正确预测土星的环是由许多未附着的小碎片组成的[14]。他于1863年发表的《关于调速器》一文为控制理论和控制论提供了重要的基础,也为控制系统的早期数学分析[15][16]。在1867年,他提出了著名的思想实验——麦克斯韦妖[17]。

他的发现帮助开启了现代物理学的时代,为相对论等领域奠定了基础,他也是第一个将这一术语引入物理学的人[10],并对量子力学的发展做出了贡献[18][19]。许多物理学家认为,麦克斯韦是19世纪对20世纪物理学影响最大的科学家。他对科学的贡献被许多人认为与艾萨克·牛顿和阿尔伯特·爱因斯坦的贡献同样重要[20]。在麦克斯韦诞辰一百周年时,爱因斯坦曾形容他的工作是“自牛顿时代以来物理学所经历的最深刻且最富有成果的工作”[21]。当爱因斯坦于1922年访问剑桥大学时,他的主人告诉他,他之所以取得伟大的成就,是因为站在牛顿的肩膀上;爱因斯坦回答说:“不,我不是。我站在麦克斯韦的肩膀上。”[22] 汤姆·西格弗里德(Tom Siegfried)形容麦克斯韦是“那种百年一遇的天才,他比周围的人更敏锐地感知到物理世界。”[23]
\subsection{生活}  
\subsubsection{早期生活,1831–1839}
詹姆斯·克拉克·麦克斯韦于1831年6月13日出生在爱丁堡印度街14号[24],父亲是中贝比的约翰·克拉克·麦克斯韦(John Clerk Maxwell),一名律师,母亲是弗朗西斯·凯(Frances Cay),她是罗伯特·霍奇森·凯(Robert Hodshon Cay)的女儿,也是约翰·凯(John Cay)的妹妹[25][26]。他出生的地方现在是由詹姆斯·克拉克·麦克斯韦基金会运营的博物馆。麦克斯韦的父亲来自拥有相当财富的克拉克家族,该家族位于佩尼库克,并持有佩尼库克克拉克的男爵头衔[27]。他父亲的兄弟是第六代男爵[28]。他的父亲出生时名为“约翰·克拉克”,在1793年继承了中贝比庄园(一个位于邓弗里郡的麦克斯韦家族财产)后,便将“麦克斯韦”作为自己的姓氏[25]。詹姆斯是艺术家杰米玛·布莱克本(Jemima Blackburn)的表姐[29](她是他父亲妹妹的女儿)和土木工程师威廉·戴斯·凯(William Dyce Cay)的表兄[30](他是他母亲兄弟的儿子)。凯和麦克斯韦是亲密的朋友,凯在麦克斯韦结婚时担任了他的伴郎。

麦克斯韦的父母在三十多岁时相识并结婚[31];他的母亲在他出生时已经接近40岁。他们之前曾有一个孩子,一个名叫伊丽莎白的女儿,但她在婴儿时期去世[32]。

麦克斯韦年幼时,家人搬到了基尔库布赖特郡的格伦莱尔(Glenlair),这是他的父母在一个包含1500英亩(610公顷)土地的庄园上建造的房子[33]。所有迹象表明,麦克斯韦从小就保持着一种无法熄灭的好奇心[34]。到三岁时,他对任何能移动、发光或发出声音的东西都会产生疑问:“那是什么?”[35]。在他父亲1834年写给小姑简·凯(Jane Cay)的信中添加的一段话中,他的母亲描述了这种与生俱来的好奇心:

“他是一个非常快乐的孩子,自从天气变得温和以来,他进步了很多;他对门、锁、钥匙等非常感兴趣,‘告诉我怎么做’几乎是他嘴边常挂的话。他还会研究小溪和铃线的隐蔽路线,水是如何从池塘通过墙壁流过来的……”[36]
\subsubsection{教育,1839–1847}  
意识到这个男孩的潜力,麦克斯韦的母亲弗朗西斯承担了他的早期教育责任,在维多利亚时代,这通常是家庭主妇的工作[37]。八岁时,他就能背诵约翰·弥尔顿的长篇段落和完整的《诗篇》第119篇(176节)。实际上,他对圣经的了解已经相当详尽;几乎能准确说出《诗篇》中的任何引文的章节和节数。他的母亲因腹部癌症生病,在一次未成功的手术后,于1839年12月去世,当时他年仅八岁。此后,他的教育由父亲和父亲的嫂子简共同负责,两人在他的一生中扮演了关键角色[37]。他的正式学校教育开始时并不顺利,由一位16岁的聘请导师指导。关于这位年轻导师的信息很少,只有知道他对麦克斯韦态度严厉,经常责备他反应迟钝、行为古怪[37]。这位导师于1841年11月被解雇。1842年2月12日,詹姆斯的父亲带他去参加罗伯特·戴维森(Robert Davidson)关于电推进和磁力的示范,这次经历对麦克斯韦产生了深远的影响[38]。

麦克斯韦被送往了著名的爱丁堡学院[39]。在学期期间,他住在姑妈伊莎贝拉的家中。在这段时间里,他对绘画的热情得到了表姐杰米玛的鼓励[40]。10岁的麦克斯韦,由于在父亲的乡村庄园里孤立地长大,并不太适应学校生活[41]。第一年他因人数已满,只得和比自己年长一岁的同班同学一起加入了第二年级[41]。他的举止和高洛韦口音使得其他男孩觉得他有些乡土气息。第一次上学时,他穿着一双自制鞋和一件长袍,这使得他得到了不友好的绰号“傻小子”(Daftie)[42]。他似乎从未对这个绰号感到不满,许多年里毫无怨言地承受着它[43]。在学院的社交孤立局面直到他遇到了刘易斯·坎贝尔(Lewis Campbell)和彼得·格思里·泰特(Peter Guthrie Tait)才得以改变,这两位年纪相仿的男孩后来都成了著名的学者,他们也成为了终生的朋友[25]。

麦克斯韦从小就对几何学充满兴趣,在接受任何正式教育之前,他就重新发现了正多面体[40]。尽管他在第二年获得了学校的圣经传记奖,但他的学术成绩一直未受到关注[40],直到13岁时,他赢得了学校的数学奖牌,并获得了英语和诗歌的第一奖[44]。

麦克斯韦的兴趣远超学校课程,他并未特别关注考试成绩[44]。14岁时,他写了他的第一篇科学论文。在这篇论文中,他描述了一种利用一根细绳绘制数学曲线的机械方法,并探讨了椭圆、笛卡尔椭圆和具有多个焦点的相关曲线的性质。这篇1846年的论文《关于椭圆曲线及具有多个焦点的曲线的描述》[46]由爱丁堡大学自然哲学教授詹姆斯·福布斯(James Forbes)提交给爱丁堡皇家学会,因为麦克斯韦被认为年纪太小,无法亲自提交这篇论文[25][45]。这项工作并不完全原创,因为17世纪的勒内·笛卡尔也研究过这种多焦点椭圆的性质,但麦克斯韦简化了它们的构造[47]。
\subsubsection{爱丁堡大学,1847–1850}
麦克斯韦于1847年16岁时离开了爱丁堡学院,开始在爱丁堡大学上课[48]。虽然他有机会就读剑桥大学,但在第一学期后,他决定完成自己在爱丁堡的本科课程。大学的学术教职员工中有许多备受尊敬的名字;他第一年的导师包括威廉·汉密尔顿爵士(Sir William Hamilton),他教授逻辑学和形而上学,菲利普·凯兰德(Philip Kelland)教授数学,詹姆斯·福布斯(James Forbes)教授自然哲学[25]。他觉得这些课程并不难[49],因此他能够在大学的空闲时间,特别是在回到家里格伦莱尔时,专心进行个人研究[50]。在那里,他会用简易的化学、电学和磁学设备进行实验;然而,他主要关注的是偏振光的性质[51]。他制造了形状各异的明胶块,施加不同的压力,并用威廉·尼科尔(William Nicol)给他的一对偏振棱镜观察胶体中形成的彩色边带[52]。通过这种实验,他发现了光弹性,这是确定物理结构中应力分布的一种方法[53]。

18岁时,麦克斯韦为《爱丁堡皇家学会会刊》贡献了两篇论文。其中一篇《弹性固体的平衡》奠定了他一生中一项重要发现的基础,即剪切应力在粘性液体中产生的暂时双折射现象[54]。另一篇论文是《滚动曲线》,与他在爱丁堡学院所写的《椭圆曲线》论文一样,他再次被认为太年轻,无法亲自站在讲台上发表这篇论文。于是,论文由他的导师凯兰德代为提交给皇家学会[55]。
\subsubsection{剑桥大学,1850–1856}
1850年10月,麦克斯韦已经是一位出色的数学家,他离开苏格兰前往剑桥大学。他最初就读于彼得学院,但在第一学期结束前转学到了三一学院,他认为在那里更容易获得奖学金[56]。在三一学院,他被选入了剑桥大学的精英秘密社团——剑桥使徒社[57]。在剑桥的岁月里,麦克斯韦对基督教信仰和科学的理解迅速加深。他加入了“使徒社”——一个由知识精英组成的辩论社团,在那里通过他的论文,他努力探索并深化对这些思想的理解。