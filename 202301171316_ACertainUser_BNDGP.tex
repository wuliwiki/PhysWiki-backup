% 能带模型(科普)

\begin{issues}
\issueDraft
\end{issues}


\footnote{本文参考了Calister的Materials Science and Engineering An Introduction与Carter et al 的Ceramic Materials}
\subsection{分子轨道与能带模型}
在我们的常规认识中,化学键似乎只是两个原子之间的联系(例如我们常说,共价键是“两个原子公用一个电子”)。然而实际情况复杂的多,电子实则可以理解为在整个分子中运动,而非拘泥于某两个原子。
\begin{figure}[ht]
\centering
\includegraphics[width=10cm]{./figures/BNDGP_1.pdf}
\caption{电子在整个分子中运动} \label{BNDGP_fig1}
\end{figure}
用更逼格的话术说,这是一种量子现象。我们已经了解过原子轨道的相关概念:在一个原子中,电子只能在特定的电子层、电子亚层轨道上运动。而在分子中,这些原子轨道相互重叠,并组合成新的轨道,电子正是在这些新的分子轨道上运动(也只能在这些轨道上运动)。

\begin{example}{苯与石墨}
苯似乎是解释电子非局域性的最简单例子。在苯中,碳的$p$轨道互相重叠从而形成横跨分子的大$\pi$键。各个碳$p$轨道上的孤电子由此可以在整个分子中运动,而非拘泥于某两个碳之间。这给予了苯良好的化学稳定性。在有机化学中我们会继续遇到大量例子“共轭效应”\footnote{题外话:蛋白质中的肽键$C=O-N-H$得利于此。电子的非局域性让这一结构趋近于平面,而不能自由扭转。}。

同理,石墨中碳的$p$轨道也互相重叠,从而电子可以在整个固体中运动。这也是为什么石墨具有良好的导电性。
\end{example}

对于少数原子组成的小分子,分子轨道的数量不多。
\begin{figure}[ht]
\centering
\includegraphics[width=10cm]{./figures/BNDGP_2.pdf}
\caption{氢的分子轨道} \label{BNDGP_fig2}
\end{figure}
而对于由大量原子组成的固体,分子轨道的数量也是巨大的。一些轨道间的能量差异十分小,因此可以忽略不计,将这一系列分子轨道看出连续的“能带”。而在某些固体中,某些能带之间却存在显著的能量差异、存在一个不允许电子存在的能量区间,称为“禁带”。
\begin{figure}[ht]
\centering
\includegraphics[width=8cm]{./figures/BNDGP_3.pdf}
\caption{将一系列分子轨道看成连续的“能带”} \label{BNDGP_fig3}
\end{figure}
\begin{figure}[ht]
\centering
\includegraphics[width=8cm]{./figures/BNDGP_4.pdf}
\caption{在某些固体中,某些能带之间存在显著的能量差异} \label{BNDGP_fig4}
\end{figure}

\subsection{费米能}
在固体中,只有足够高能量的电子才能在外场的驱动下定向运动,称为自由电子\footnote{“自由电子”这一术语在此次的含义似乎不同于介质极化话题中的含义,请注意区分。},而其余电子只是被束缚在原子核的周围。这个能量的下限被称为“费米能”。固体中,能量高于费米能的能带称为“导带”,而低于费米能的能带称为“价带”。
\begin{figure}[ht]
\centering
\includegraphics[width=5cm]{./figures/BNDGP_5.pdf}
\caption{请添加图片描述} \label{BNDGP_fig5}
\end{figure}
费米能有什么更深刻的物理含义吗?我们先从锂原子(Li, 第三号元素)入手,不严谨地理解费米能。

我们知道锂的核外具有三个电子,根据不相容原理,这些电子得占据原子的两个电子层。而我们还知道,第二电子层的能量高于第一电子层。也就是说,电子处于第二电子层会导致额外的能量。
我们可以通过降温的方式(例如,让原子的温度降低到绝对零度)让处于第二电子层的电子回到第一电子层并释放能量吗?答案似乎是不行的,不相容原理禁止三个电子共存于第一电子层\footnote{在恒星末期的大坍塌中,极端的引力可能将这些电子压入原子核,并让原来的恒星核心成为中子星。}。也就是说,这部分由于电子处于第二电子层而产生的能量几乎是原子固有的性质。这就是“锂原子的费米能”。

在固体中我们也会遇到相同的问题。假设最开始分子轨道上没有一个电子,而我们要逐个将电子加入轨道。同样根据不相容原理,即使固体处于绝对零度,我们也不能将所有电子都放入能量最低的分子轨道,而总要逐渐使用更高能量的轨道来容纳电子。最终,当填充完所有电子后,能量最高的电子所具有的能量即为费米能。

\subsection{热激活}
我们大部分时候都是在至少常温(而不是绝对零度)的环境下使用固体。此时,部分电子由于额外的热能而具有比费米能更高的能量,从而跃升至导带并成为自由电子。显然,自由电子的数量与温度与价带-导带间的能差(禁带的宽度)有关:温度越高、能差越低,自由电子的数量越多。

\subsection{导体、半导体与绝缘体}
现在,我们\textsl{已经储备了足够的知识来运用能带模型解释世间万物}了。至少,我们可以说明几个例子。

如果价带与导带间没有禁带,那么电子非常容易通过热激活到达导带中。由于导带中总有大量自由电子,因此导电性十分良好。这就是导体的能带结构,常见于金属等固体。对于石墨等非金属,他的能带结构近似如此,因此即使是非金属物质也能体现导电性。

如果禁带的宽度较窄,那么只有少量电子能到达导带,因此这种材料的导电性不是很好,称为半导体。

\begin{example}{不只是瓶瓶罐罐}
说到陶瓷,我们的第一印象似乎总是\textsl{家里的瓶瓶罐罐与博物馆里的瓶瓶罐罐},似乎与导电性毫不沾边。但是,由于某些陶瓷具有较窄的禁带,因此也能传递电流。例如,热敏电阻的核心就是陶瓷材料。(\textsl{话说回来,LED和芯片不也都是陶瓷基半导体吗...})
\end{example}

如果禁带的宽度很宽,那么几乎没有电子能到达导带,因此这种材料几乎不能导电,称为绝缘体。

\subsection{能带理论实践}
除了解释导电性,能带理论还有什么运用呢?

\begin{example}{热情洋溢的电子与高冷的半导体}
温度升高,电阻会如何变化呢?金属与半导体的主导因素不尽相同,得分类讨论。

对于金属,温度升高,电子的运动速率上升,与原子核或晶体缺陷的碰撞频率也增加。总体而言,温度升高使电阻上升。

对于半导体,温度升高,更多的电子得以跃升至导带并成为自由电子。总体而言,温度升高使电阻下降。
\end{example}

\begin{example}{玻璃之心}
为什么金属是不透明的,而玻璃是透明的?

除了热之外,光携带的能量(光子)也能使电子跃迁。

由于金属没有禁带,几乎任何频率的光都能使电子跃迁至导带,因此金属能吸收所有频率的光,从而不透明。

而玻璃的禁带很宽,可见光的能量不足以使电子跃迁至导带,因此玻璃中的电子不会吸收这些光,这也就让玻璃呈现透明。

或许你已经猜到,有些半导体的能带不宽不窄,因此能吸收特定频段的光。例如,允许红光(低能量光)透过但吸收蓝光(高能量光)。
\end{example}
