% 惠更斯原理

水面上有一个波传播时,如果没有遇到障碍物,波将保持原来的波面形状前进,若在前进中遇到一个有小孔的障碍物$AB$,如\autoref{Huygen_fig1}所示.只要小孔的孔径$a $比波长$\lambda$小,我们就可看到,穿过小孔的波总是半圆形的波,与原来波的形状无关,这说明小孔可以看作是一个新波源,所发射的波称为子波.

\begin{figure}[ht]
\centering
\includegraphics[width=8cm]{./figures/Huygen_1.pdf}
\caption{障碍物的小孔成为新的波源} \label{Huygen_fig1}
\end{figure}

惠更斯(C. Huygens)于1678年提出了关于波传播的几何法则: 在波的传播过程中,波阵面(波前)上的每一点都可行做是发射子波的波源,在其后的任一时刻,这些子波的包迹就成为新的波阵面.这就是\textbf{惠更斯原理(Huygens principle)}.
\begin{figure}[ht]
\centering
\includegraphics[width=8cm]{./figures/Huygen_2.pdf}
\caption{惠更斯原理} \label{Huygen_fig2}
\end{figure}