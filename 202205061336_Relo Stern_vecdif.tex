% 向量函数的微分
% 多元函数|向量函数|微分
\begin{definition}{(全微分)}
设有向量函数 $F:\mathbb{R}^{n}\rightarrow\mathbb{R}^{m}$. 若存在线性变换 $A:\mathbb{R}^{n}\rightarrow\mathbb{R}^{m}$
使成立
\[
F(x)=F(p)+A(x-p)+o(\left\Vert x-p\right\Vert ),\quad\quad(x\rightarrow p\in\mathbb{R}^{n}),
\]
即
\[
{\displaystyle \lim_{x\rightarrow p}{\displaystyle \frac{F(x)-F(p)-A(x-p)}{\left\Vert x-p\right\Vert }=0,}}
\]
则称 $F(x)$ 在 $x=p$ 处\textbf{可微}, 并称线性变换 $A$ 为 $F(x)$ 在 $x=p$ 处的\textbf{全微分},
记作 $\mathrm{d}F(p)$, 即 $\mathrm{d}F(p)=A.$ 
\end{definition}

\begin{definition}{(导矩阵)}
设有向量函数 $F:\mathbb{R}^{n}\rightarrow\mathbb{R}^{m}$, $F$ 可表为分量函数形式
$$
F(x)=\left(\begin{array}{c}
F_{1}(x)\\
F_{2}(x)\\
\vdots\\
F_{m}(x)
\end{array}\right),\quad x:=(x_{1},x_{2},\ldots,x_{n})\in\mathbb{R}^{n},
$$
其中 $F_{j}:\mathbb{R}^{n}\rightarrow\mathbb{R}$ 是数量函数 $(j=1,2,\ldots,m)$.
若每个 $F_{j}$ 在 $x=p$ 处可微, 则 ${\displaystyle \frac{\partial F_{j}}{\partial x_{i}}(p)}$
都有意义 $(i=1,2,\ldots,n),$ 因而可定义
$$
{\displaystyle \frac{\partial F}{\partial x_{i}}(p):=\left(\begin{array}{c}
\frac{\partial F_{1}}{\partial x_{i}}(p)\\
\frac{\partial F_{2}}{\partial x_{i}}(p)\\
\vdots\\
\frac{\partial F_{m}}{\partial x_{i}}(p)
\end{array}\right),\quad i=1,2,\ldots,n,}
$$
称为 $F(x)$ 在 $x=p$ 处关于 $x_{i}$ 变元的\textbf{偏导数}, 以及也可定义
\[
\mathrm{D}F(p)=(\frac{\partial F}{\partial x_{1}}(p),\frac{\partial F}{\partial x_{2}}(p),\ldots,\frac{\partial F}{\partial x_{n}}(p))
\]
称为 $F(x)$ 在 $x=p$ 处的\textbf{导映射}或\textbf{导矩阵}, 也叫\textbf{雅可比 (Jocobi)
矩阵}, 即
$$
\mathrm{D}F(p)=\left(\begin{array}{cccc}
\frac{\partial F_{1}}{\partial x_{1}}(p) & \frac{\partial F_{1}}{\partial x_{2}}(p) & \cdots & \frac{\partial F_{1}}{\partial x_{n}}(p)\\
\frac{\partial F_{2}}{\partial x_{1}}(p) & \frac{\partial F_{2}}{\partial x_{2}}(p) & \cdots & \frac{\partial F_{2}}{\partial x_{n}}(p)\\
\vdots & \vdots &  & \vdots\\
\frac{\partial F_{m}}{\partial x_{1}}(p) & \frac{\partial F_{m}}{\partial x_{2}}(p) & \ldots & \frac{\partial F_{m}}{\partial x_{n}}(p)
\end{array}\right)_{m\times n}
$$
关于 Jocobi 矩阵的记号, 有些书也将其记为 $\mathrm{D}F(p):={\displaystyle \left.\frac{\mathrm{D}(F_{1},F_{2},\ldots,F_{m})}{\mathrm{D}(x_{1},x_{2},\ldots,x_{n})}\right|_{x=p}.}$
\end{definition}



下面的定理表明, 上面两个定义是完全等价的. 

\begin{theorem}{}
向量函数 $F:\mathbb{R}^{n}\rightarrow\mathbb{R}^{m}$ 在 $p\in\mathbb{R}^{n}$
处可微的充要条件是它的每个分量函数 $F_{j}:\mathbb{R}^{n}\rightarrow\mathbb{R}$ ($j=1,2,\ldots,$ $m$)
都在 $p$ 处可微. 因此, $\mathrm{d}F(p)=\mathrm{D}F(p).$
\end{theorem}
$$$$

下面给一些向量函数的微分的例子.
\begin{example}{}
若矩阵 $A\in\mathbb{R}^{m\times n}$ ($m$ 行 $n$ 列), 向量函数 $F:\mathbb{R}^{n}\rightarrow\mathbb{R}^{m},$
$F(x)=Ax,$ 则 $F(x)$ 在任意点 $x=p$ 处的微分都等于 $A,$ 即
\[
\mathrm{d}F(p)\equiv A.
\]
\end{example}

\begin{example}{}
设 $F(x,y)=(x^{2}+xy,y^{2}+xy)$, 求 $\mathrm{D}F(x,y).$
$$
\mathrm{D}F(x,y)=\left(\begin{array}{cc}
2x+y\quad & x\\
y & 2y+x
\end{array}\right).
$$
\end{example}