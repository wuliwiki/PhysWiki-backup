% 反常霍尔效应
% 晶体|电子|霍尔效应
\pentry{电子运动的准经典模型\upref{cryele}}
\begin{issues}
\issueDraft
\end{issues}
一般霍尔效应的产生需要磁场,并且满带不出现霍尔效应。但是反常霍尔效应不需要这些条件。
\subsection{介绍}
我们知道一个布洛赫态可以写成:
\begin{equation}
\psi_{n,\boldsymbol{k}}=e^{i\boldsymbol{k}\cdot\boldsymbol{r}}u_{n,\boldsymbol{k}}
\end{equation}
其哈密顿量为:$\widehat{H_0}=-\frac{h^2}{2m}\nabla^2+V(\boldsymbol{r})$,对应的能量是$E_{n,\boldsymbol{k}}$。其中$V(\boldsymbol{r})$是一个周期函数,有$V(\boldsymbol{r}+\boldsymbol{R})=V(\boldsymbol{r})$,$\boldsymbol{R}$是任意一个格矢。

外力作用下哈密顿量变成$\widehat{H}=\widehat{H_0}-\boldsymbol{F}\cdot\boldsymbol{r}$,则dt时间后,布洛赫态变成:
\begin{equation}
\psi(\boldsymbol{r},dt)=e^{-\frac{i\,\widehat{H}\,dt}{\hbar}}\psi_{n,\boldsymbol{k}} \approx \psi_{n,\boldsymbol{k}}-\frac{idt}{\hbar}(\widehat{H_0}-\boldsymbol{F}\cdot\boldsymbol{r})\psi_{n,\boldsymbol{k}}
=(1-i\frac{E_{n,\boldsymbol{k}}dt}{\hbar}+i\frac{\boldsymbol{F}\cdot\boldsymbol{r}dt}{\hbar})e^{i\boldsymbol{k}\cdot\boldsymbol{r}}u_{n,\boldsymbol{k}}
\approx e^{-\frac{i}{\hbar}E_{n,\boldsymbol{k}}dt}e^{i(\boldsymbol{k}+\frac{\boldsymbol{F}dt}{\hbar})\cdot \boldsymbol{r}}u_{n,\boldsymbol{k}}
\end{equation}
在推导电子运动的准经典模型时,我们忽略了$u_{n,\boldsymbol{k}}$到$u_{n,\boldsymbol{k}+\frac{\boldsymbol{F}dt}{\hbar}}$之间的变化(包络近似),从而得出了$d\boldsymbol{k}=\frac{\boldsymbol{F}dt}{\hbar}$的结论。现在我们不忽略它的变化,从而推导出反常霍尔效应来。

有:
\begin{equation}
\psi(\boldsymbol{r},dt)\approx e^{-\frac{i}{\hbar}E_{n,\boldsymbol{k}}dt}e^{i(\boldsymbol{k}+\frac{\boldsymbol{F}dt}{\hbar})\cdot \boldsymbol{r}}u_{n,\boldsymbol{k}}=e^{-\frac{i}{\hbar}E_{n,\boldsymbol{k}}dt}e^{i(\boldsymbol{k}+\frac{\boldsymbol{F}dt}{\hbar})\cdot \boldsymbol{r}}(u_{n,\boldsymbol{k}}-u_{n,\boldsymbol{k}+\frac{\boldsymbol{F}}{\hbar}dt}+u_{n,\boldsymbol{k}+\frac{\boldsymbol{F}}{\hbar}dt})
\end{equation}
\begin{equation}
=e^{-\frac{i}{\hbar}E_{n,\boldsymbol{k}}dt}e^{i(\boldsymbol{k}+\frac{\boldsymbol{F}dt}{\hbar})\cdot \boldsymbol{r}}u_{n,\boldsymbol{k}+\frac{\boldsymbol{F}}{\hbar}dt}-e^{-\frac{i}{\hbar}E_{n,\boldsymbol{k}}dt}e^{i(\boldsymbol{k}+\frac{\boldsymbol{F}dt}{\hbar})\cdot \boldsymbol{r}}(u_{n,\boldsymbol{k}+\frac{\boldsymbol{F}}{\hbar}dt}-u_{n,\boldsymbol{k}})=e^{-\frac{i}{\hbar}E_{n,\boldsymbol{k}}dt}\psi_{n,\boldsymbol{k}+\frac{\boldsymbol{F}dt}{\hbar}}-e^{-\frac{i}{\hbar}E_{n,\boldsymbol{k}}dt}e^{i(\boldsymbol{k}+\frac{\boldsymbol{F}dt}{\hbar})\cdot \boldsymbol{r}}(\nabla_{\boldsymbol{k}}u)
\end{equation}

