% 全同粒子

% 未完成: 这应该是二级词条

\subsection{粒子交换算符}
\pentry{正交子空间\upref{OrthSp}}

定义
\begin{equation}
P_{1,2}ψ(\bvec r_1, \bvec r_2) = ψ(\bvec r_2, \bvec r_1)
\end{equation}
可以证明这是一个厄米算符, 即
\begin{equation}
\braket{φ}{P_{1,2}ψ} = \braket{P_{1,2}φ}{ψ}
\end{equation}
该算符有 $1$ 和 $-1$ 两个本征值, 对应两个正交子空间\upref{OrthSp}, 分别是对称波函数和反对称波函数(即满足下式)构成的空间.
\begin{equation}
ψ_±(\bvec r_2, \bvec r_1) = ±ψ_±(\bvec r_1, \bvec r_2)
\end{equation}

这两个子空间外的波函数既非对称也非反对称.

对称波函数描述全同玻色子, 反对称波函数描述全同费米子.

\subsection{与哈密顿算符对易}

\pentry{守恒量(量子力学)\upref{QMcons}}
对于全同粒子, 交换算符与哈密顿算符对易\footnote{对非全同粒子则不成立, 例如两个质量不同的粒子的哈密顿算符与交换算符不对易}. 这保证了 $P$ 是一个守恒量. 也就是全同粒子的波函数在演化过程中将一直保持对称性或反对称性.

\subsection{本征态与测量}
两个全同粒子的本征态(或者其他任何态)也必须是必须是对称或反对称的, 例如位置本征态 $(δ_{\bvec x_1} δ_{\bvec x_2}± δ_{\bvec x_2} δ_{\bvec x_1})/\sqrt 2$ 只能告诉我们一个粒子在 $\bvec r_1$ 处另一个粒子在 $\bvec r_2$ 处, 仍然不能区分它们.

\begin{equation}
\begin{aligned}
P(r_1, r_2) &= \abs{\frac{1}{\sqrt{2}}∫ [δ_{\bvec x_1}(\bvec r_1) δ_{\bvec x_2}(\bvec r_2) ± δ_{\bvec x_2}(\bvec r_1) δ_{\bvec x_1}(\bvec r_2)] ψ_±(\bvec r_1,\bvec r_2) \dd[3]{r_1}\dd[3]{r_2}}^2\\
&= \abs{\frac{1}{\sqrt{2}} ψ_±(\bvec x_1, \bvec x_2) + }
\end{aligned}
\end{equation}
