% 用狄拉克 delta 函数表示点电荷的散度
% 点电荷|散度|无穷大|狄拉克 delta 函数

\begin{issues}
\issueDraft
\end{issues}

\pentry{电场的高斯定律证明\upref{EGausP}, 多元狄拉克 delta 函数\upref{deltaN}}

\footnote{参考 \cite{GriffE}.}在电场的高斯定律证明\upref{EGausP}, 我们看到通常意义下, 如果取一个包含点电荷的闭合高斯面, 那么散度定理(\autoref{Divgnc_eq1}~\upref{Divgnc})并不能直接使用, 因为在点电荷处电场的散度没有定义, 而在曲面内的其他位置电场的散度处处为零.

但是我们可以先把每个点电荷替换成一个半径为 $R$ 的带电小球, 然后令 $R\to 0$ 即可. 考虑一个在原点的均匀带电小球, 半径为 $R$, 电荷为 $Q$, 电荷密度为 $\rho = Q/V$, $V = 4\pi R^3/3$. 我们已知小球之外电场散度处处为零, 而小球内部根据高斯定律的微分形式(\autoref{EGauss_eq1}~\upref{EGauss}) 有
\begin{equation}\label{CEfDiv_eq1}
\div \bvec E = \frac{\rho}{\epsilon_0} = \frac{Q}{\epsilon_0 V} = \frac{3Q}{4\pi \epsilon_0 R^3}
\end{equation}
这说明电场散度在小球内为常量, 所以散度在高斯面内的体积分就是 $\div \bvec E$ 乘以球体体积 $V$ 得 $Q/\epsilon_0$, 这就验证了高斯定理的积分形式(\autoref{EGauss_eq2}~\upref{EGauss}).

接下来当我们令 $R\to 0$ 的时候, 小球的半径越来越小, 而电荷密度却越来越大. 所以取极限时, 电荷密度并不是一个通常意义的函数, 而狄拉克 delta 函数\upref{deltaN}就是专门用来描述这种 “极限” 的数学工具. 由于三维的狄拉克函数的体积分等于 1($\int \delta(\bvec r)\dd{V} = 1$), 而\autoref{CEfDiv_eq1} 的体积分等于 $Q/\epsilon_0$, 所以我们可以令 $\div \bvec E = Q\delta(\bvec r)/\epsilon_0$
\begin{equation}
\rho = Q\delta(\bvec r)
\end{equation}



未完成: 

========= 回收内容 ==============

相同. 同样, 这个结论也包含.

可以推出原点处有无穷大的散度, 使得积分后恰好等于 $q/\epsilon_0$. 这让我们马上想到(三维)狄拉克 $\delta$ 函数\upref{Delta}, 在原点处为无穷大, 但体积分为有限值
\begin{equation}
\int \delta(\bvec r)\dd{V} = 1
\end{equation}

所以如果我们假设电场在原点处的散度为
\begin{equation}
\div \bvec E = \frac{q}{\epsilon_0}\delta(\bvec r)
\end{equation}
那么恰好可使电场满足散度定理
\begin{equation}
\int \div \bvec E \dd{V} = \frac{q}{\epsilon_0}
\end{equation}
