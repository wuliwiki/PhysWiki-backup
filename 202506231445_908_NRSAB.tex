% 尼尔斯·阿贝尔(综述)
% license CCBYSA3
% type Wiki

本文根据 CC-BY-SA 协议转载翻译自维基百科 \href{https://en.wikipedia.org/wiki/Niels_Henrik_Abel}{相关文章}。

尼尔斯·亨利克·阿贝尔(Niels Henrik Abel,/ˈɑːbəl/ AH-bəl,挪威语发音:[ˌnɪls ˈhɛ̀nːɾɪk ˈɑ̀ːbl̩],1802年8月5日-1829年4月6日)是挪威著名数学家,在多个数学领域做出了开创性的贡献。\(^\text{[1]}\)他最著名的成果是第一个完整地证明了一般五次方程无法用根式求解的定理。这个问题曾是当时最重要的未解难题之一,悬而未决长达250多年。\(^\text{[2]}\)此外,他还是椭圆函数领域的创新者,并发现了阿贝尔函数。他是在贫困中取得这些发现的,26岁时因肺结核早逝。

他的大部分研究成果都集中在六七年间完成。\(^\text{[3]}\)法国数学家查尔斯·埃尔米特曾评价说:“阿贝尔留给数学家的东西,足够他们研究五百年。”\(^\text{[3][4]}\) 另一位法国数学家阿德里安-玛丽·勒让德则说:“这个年轻的挪威人真是个天才!”\(^\text{[5]}\)
\subsection{生平}
\subsubsection{早年生活}
尼尔斯·亨利克·阿贝尔出生于挪威内斯特兰,是牧师索伦·乔治·阿贝尔和安娜·玛丽·西蒙森的第二个孩子,出生时为早产。阿贝尔出生时,家人住在芬内伊的牧师住宅。许多迹象表明,他可能出生在邻近的教区,因为他的父母在他出生当年的七八月曾作为客人住在内斯特兰的县官家中。\(^\text{[6][注1]}\)

阿贝尔的父亲索伦·乔治·阿贝尔拥有神学和哲学学位,当时担任芬内伊的牧师。索伦的父亲、阿贝尔的祖父汉斯·马蒂亚斯·阿贝尔也是一位牧师,服务于里瑟尔附近的耶尔斯塔教堂。索伦在耶尔斯塔度过了童年,也曾在那里担任副牧师。1804年他父亲去世后,索伦接任耶尔斯塔教区的牧师一职,举家迁往当地。阿贝尔家族起源于石勒苏益格,17世纪迁居挪威。

19世纪90年代的耶尔斯塔教堂与牧师住宅明信片
明信片中牧师住宅的主楼与阿贝尔居住时为同一建筑

安娜·玛丽·西蒙森来自里瑟尔,她的父亲尼尔斯·亨利克·萨克西尔·西蒙森是当地的商人兼船主,被认为是里瑟尔最富有的人。安娜·玛丽在相对奢华的环境中长大,经历了两个继母的抚养。她在耶尔斯塔的牧师住宅中喜欢组织舞会和社交聚会。有诸多证据显示,她很早便染上了酒瘾,对孩子的教育几乎毫不上心。尼尔斯·亨利克和他的兄弟们由父亲亲自教授,用的是手写的教科书。在一本数学书中,有一张加法表写着:“1+0=0”。\(^\text{[6]}\)
\subsubsection{大教堂学校与皇家腓特烈大学}
随着挪威在1814年获得独立并举行第一次全国选举,索伦·阿贝尔当选为议会(Storting)代表。当时议会的会议在克里斯蒂安尼亚(即今日奥斯陆)的大教堂学校主楼内举行,直到1866年。几乎可以确定,正是在这样的背景下,他与这所学校建立了联系,并决定让长子汉斯·马蒂亚斯于次年入学。然而,临近启程之际,汉斯因要离家而十分沮丧和伤心,父亲因此不敢让他出发,最终决定改派次子尼尔斯前往。\(^\text{[6]}\)

1815年,13岁的尼尔斯·阿贝尔进入大教堂学校。次年,他的哥哥汉斯也加入了他们,并与他共住一室,一起上课。在学业成绩上,汉斯优于尼尔斯。然而,1818年学校新聘了一位数学教师——本特·米歇尔·霍姆博(Bernt Michael Holmboe)。霍姆博给学生布置了课后数学题,并很快发现了尼尔斯·亨利克在数学方面的天赋。他鼓励尼尔斯深入学习数学,甚至在课后为他提供私人辅导。
\begin{figure}[ht]
\centering
\includegraphics[width=8cm]{./figures/358b055808a76d79.png}
\caption{} \label{fig_NRSAB_1}
\end{figure}
1818年,索伦·阿贝尔因1806年出版的《教理问答》与神学家斯泰纳·约翰内斯·斯泰纳申爆发了公开的神学争论,此事在媒体上得到了广泛报道。索伦因此获得了一个外号“阿贝尔·夸夸其谈”(挪威语:“Abel Spandabel”)。据说尼尔斯对这场争论的反应是“过度的兴奋”。与此同时,索伦还因侮辱了挪威制宪会议东道主卡斯滕·安克而差点面临弹劾。同年9月,他在政治生涯彻底崩溃的情况下返回了耶尔斯塔德。他开始酗酒,并于两年后的1820年去世,年仅48岁。

本特·米歇尔·霍姆博设法为尼尔斯·亨利克·阿贝尔争取到一份奖学金,使他能够继续留在学校学习,并向自己的朋友募资,以资助阿贝尔进入皇家腓特烈大学(即今天的奥斯陆大学)深造。

1821年,阿贝尔进入皇家腓特烈大学时,已是当时挪威最博学的数学家。霍姆博已经无课可教,而阿贝尔则研读了大学图书馆中所有最新的数学文献。在此期间,他开始研究用根式解五次方程的问题。这个问题困扰数学界已有250多年。1821年,阿贝尔认为自己找到了答案。克里斯蒂安尼亚(今奥斯陆)的两位数学教授索伦·拉斯穆森和克里斯托弗·汉斯滕没有在阿贝尔的公式中发现错误,于是将他的成果转交给当时北欧地区最权威的数学家——哥本哈根的卡尔·费迪南德·德根。德根也未发现错误,但仍对这位来自遥远克里斯蒂安尼亚、籍籍无名的学生真的能解开这一历经无数杰出数学家未解之谜表示怀疑。不过,他指出阿贝尔的头脑极其敏锐,认为这样一位有天赋的年轻人不该把才华浪费在“如此贫瘠的课题”——即五次方程式——上,而应投身于椭圆函数和超越数领域。德根写道,阿贝尔将因此“发现通往广袤分析海洋中大片区域的麦哲伦式航道”。\(^\text{[6]}\)德根要求阿贝尔举一个数字例子来验证他的方法。在尝试提供例子时,阿贝尔发现了自己论文中的一个错误。\(^\text{[9]}\)这一发现最终在1823年引导他证明:对于五次或更高次代数方程,一般情况下是不存在根式解的。\(^\text{[10]}\)

1822年,阿贝尔顺利毕业。他在数学方面表现尤为卓越,其它课程则为中等水平。
\subsection{生涯}
\begin{figure}[ht]
\centering
\includegraphics[width=8cm]{./figures/a4e2fa6ab392689c.png}
\caption{} \label{fig_NRSAB_2}
\end{figure}