% 平均值(量子力学)
% 量子力学|平均值|本征值|本征态

\pentry{测量理论\upref{QMPos}}

我们先来回顾测量理论. 假设某个物理量 $Q$ 有离散的本征态 $\ket{\phi_i}$ ($i = 1,2\dots$), 对应的本征值为 $q_i$, 满足
\begin{equation}\label{QMavg_eq3}
Q\ket{\phi_i} = q_i \ket{\phi_i}
\end{equation}
且满足正交归一条件
\begin{equation}\label{QMavg_eq4}
\braket{\phi_i}{\phi_j} = \delta_{i,j}
\end{equation}

记粒子处于 $\ket{\psi}$ 状态, 可表示为本征态的线性组合
\begin{equation}\label{QMavg_eq2}
\ket{\psi} = \sum_i c_i \ket{\phi_i}
\end{equation}
对其测量 $Q$, 得到第 $q_i$ 的概率为
\begin{equation}
P_i = \abs{c_i}^2 = \abs{\braket{\phi_i}{\psi}}^2
\end{equation}

现在我们可以定义 $Q$ 的平均值为
\begin{equation}\label{QMavg_eq1}
\ev{Q} = \sum_i P_i q_i = \abs{c_i}^2 q_i
\end{equation}
这意味着, 如果我们取大量处于 $\ket{\psi}$ 状态的系统, 分别测量 $Q$ 再取平均, 结果就是该式.

\begin{example}{一维线性谐振子}
频率为 $\omega$ 的一维线性谐振子,状态为 $\psi(x)=c_1 \psi_0(x)+c_3\psi_3(x)$ ,其中 $\psi_n(x)$ 为第 $n$ 个能量本征态.给出能量的平均值.

首先应当保证波函数归一化,不难写出归一化后的波函数为
\begin{equation}
\psi(x)=\frac{c_1}{\sqrt{c_1^2+c_3^2}}\psi_0(x)+\frac{c_3}{\sqrt{c_1^2+c_3^2}}\psi_3(x)
\end{equation}

由\autoref{QMavg_eq1} ,
\end{example}

平均值还有一个更常见的公式, 与\autoref{QMavg_eq1} 等效.
\begin{equation}\label{QMavg_eq5}
\ev{Q} = \mel{\psi}{Q}{\psi}
\end{equation}
要验证, 可以将\autoref{QMavg_eq2} 代入该式, 得
\begin{equation}
\begin{aligned}
\ev{Q} &= \qty(\sum_i c_i^* \bra{\phi_i}) Q \qty(\sum_j c_j \ket{\phi_j})\\
&= \sum_{i,j} c_i^* c_j \bra{\phi_i} Q \ket{\phi_j}
\end{aligned} 
\end{equation}
再代入\autoref{QMavg_eq3} 和\autoref{QMavg_eq4} 得
\begin{equation}
\ev{Q} = \sum_{i,j} c_i^* c_j q_i \braket{\phi_i}{\phi_j}
= \sum_{i,j} c_i^* c_j q_i \delta_{i,j} = \sum_i \abs{c_i}^2 q_i
\end{equation}
证毕.

% 未完成: 无限深势阱中的三角波包, 用两种方法求能量的平均值






