% 浸渐不变量
% 浸渐不变量|绝热不变量

当系统由一些缓变参数 $\lambda_i$ 确定时,系统在运动过程中保持不变的量称为系统的\textbf{浸渐不变量}。这样的系统可以理解为处于一个外场当中,而参数 $\lambda_i$ 描述了系统所处外场的性质。例如处于三维静电场中的二维平面上的电荷系统,系统受到的场的作用与平面所处的位置有关,那么系统所处电场的性质可以用平面所处的 $z$ 坐标来描述(以平面作为 $xOy$ 平面)。为简单起见,我们假设只有一个参数 $\lambda$。

所谓的“缓变”,是指在一个运动周期 $T$ 内 $\lambda$ 的变化很小,即 
\begin{equation}
\frac{\lambda(t+T)-\lambda(t)}{\lambda(t)}\rightarrow0
\end{equation}
由于 $\lambda(t+T)-\lambda(t)\approx T\dv{\lambda}{t}$,上式可写为(这里简单假设了 $\lambda>0$)
\begin{equation}
T\dv{\lambda}{t}\ll\lambda
\end{equation}
若 $\lambda$ 为常数,则系统是封闭的且能量守恒;若 $\lambda$ 非常数,则系统不封闭,能量不守恒。
\subsection{浸渐不变量的具体形式}
设 $H(p,q,\lambda)$ 是依赖于参数 $\lambda$ 的系统的哈密顿量。由(链接)(注意哈密顿量就是能量):
\begin{equation}
\dv{E}{t}=\pdv{H}{t}=\pdv{H}{\lambda}\dv{\lambda}{t}
\end{equation}
上式右端不仅依赖于缓变量 $\lambda$,还依赖于快变量 $p,q$。为了消除快变量的影响,可以按周期取平均
\begin{figure}[ht]
\centering
\includegraphics[width=5cm]{./figures/ConAdi_1.pdf}
\caption{请添加图片描述} \label{ConAdi_fig1}
\end{figure}