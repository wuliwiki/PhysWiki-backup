% n 维球的度规
% keys n维球|高维球|度规
% license Usr
% type Tutor

\pentry{度规张量与指标升降\nref{nod_TofEuc}}{nod_2f31}
$n$ 维球是 $n+1$ 维空间中的超球面,其中的“超”字在数学上定义为 $n$ 维空间中的 $n-1$ 为曲面。因此,$n$ 维球作为三维空间球面的推广,代表着 $n$ 维空间中到某一(称为球心的)点距离恒定的所有点的全体。通常球心取为坐标原点。本节将推导球坐标下 $n$ 维球的\enref{度规}{TofEuc}。

\subsection{$n+1$ 维空间的球坐标}
\begin{definition}{笛卡尔坐标}
在 $n+1$ 空间中,若在坐标 $\{x^i|i=1,\cdots,n+1\}$ 下,线元 $\dd s^2$ 可写为
\begin{equation}
\dd s^2=\dd x^i+\cdots+\dd x^{n+1},~
\end{equation}
 则称坐标 $\{x^i|i=1,\cdots,n+1\}$ 为\textbf{笛卡尔坐标}。
\end{definition}

高维空间中的球坐标可以通过三维空间的球坐标推广得到。在三维空间中,笛卡尔坐标 $(x,y,z)$ 和球坐标 $(r,\theta,\varphi)$











