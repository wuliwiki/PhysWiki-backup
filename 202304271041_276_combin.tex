% 组合
% 排列|组合|阶乘

\begin{issues}
\issueDraft
\end{issues}

\pentry{排列\upref{permut}}

\footnote{参考 Wikipedia \href{https://en.wikipedia.org/wiki/Combination}{相关页面}。}数学上, \textbf{组合(combination)}用于计算从 $n$ ($n = 0, 1, 2, \dots$)个不同的物体中选取 $0 \leqslant r \leqslant n$ 个, 有几种方法。 国内高中课本常用 $C_n^r$ 来表示, 欧美教材常记为 $\pmat{n\\r}$。 组合可以用阶乘表示(推导见下文)。
\begin{equation}
C_n^r = \frac{n!}{(n-r)!r!} \qquad (n, r \in \mathbb N,\, r \leqslant n)~,
\end{equation}
注意 $0! = 1$。

我们可以从排列来推导组合,在组合中认为 ${a,b}$, ${b,a}$ 是等效的, 在排列中认为两者是不等的, 现在我们有 $n$ 个元素,我们要从中选取 $m$ 个元素, 即 $C_n^m$。
我们先根据组合的知识可以知道 $A_n^m$ 表示 $n$ 个元素,从中选取 $m$ 个再对选中的 $m$ 个元素进行全排,我们就可以得到如下公式
\begin{equation}
A_n^m = C_n^m A_m^m
\end{equation}
对等式进行变换,可得
\begin{equation}\label{eq_combin_1}
C_n^m = \frac{A_n^m}{A_m^m}
\end{equation}
对于这个公式,我们可以理解为,从排列中排除组合中认为等效的组合。
我们将\autoref{eq_combin_1} 展开,可得
\begin{equation}\label{eq_combin_2}
C_n^m = \frac{n(n - 1) \cdots (n - m + 1)}{m(m-1)\cdots 1}
\end{equation}
进一步变换,可得
\begin{equation}\label{eq_combin_3}
C_n^m = \frac{n!}{m!(n-m)!}
\end{equation}

我们将\autoref{eq_combin_3} 中的 $m$ 用 $n-(n-m)$ 代换,可得组合的\textbf{性质1}
\begin{equation}\label{eq_combin_4}
C_n^m = \frac{n!}{(n-m)![n-(n-m)]!} = C_n^{n-m}
\end{equation}
对于性质1我们可以用一种直观的方式理解,到我们取m个元素时,剩余的元素本身就是取 $n-m$ 个元素时的组合

当我们的总元素个数从 $n$ 变为 $n+1$ 时,我们可以分为两类,一类为不包含新元素的组合,一类为包含新元素的组合,由分类加法计数原理,可得组合的\textbf{性质2}
\begin{equation}
C_{n + 1}^m = C_n^m + C_n^{m - 1}
\end{equation}
当然我们也可以对性质2进行数学证明,
\begin{equation}
\begin{aligned}
C_n^m + C_n^{m - 1} &= \frac{n!}{m!(n-m)!} + \frac{n!}{(m - 1)!(n - m + 1)!}\\
&=\frac{n!(n - m + 1)! + n!\cdot m}{m!(n - m + 1)!}\\
&=\frac{n![(n - m + 1) + m]}{m!(n - m + 1)!}\\
&=\frac{(n + 1)!}{m!(n - m + 1)!}\\
&=C_{n+1}^m
\end{aligned}
\end{equation}