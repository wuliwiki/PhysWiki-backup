% 因式分解唯一性定理
% 因式分解|唯一性

\pentry{多项式的可约性质\upref{RedPol}}
把一个多项式分解成几个多项式乘积的形式叫做这个多项式的\textbf{因式分解}.在数论中,我们知道任何大于1的整数都可以分解成素数的乘积,并且这种分解除素数的顺序之外是唯一确定的.对于多项式,也有类似的性质.
\begin{theorem}{因式分解唯一性定理}
数域 $\mathbb{F}$ 上任意一个次数大于1的多项式 $f(x)$ 都可以分解成数域 $\mathbb{F}$ 上有限个不可约多项式的乘积,并且这个分解式除因式顺序之外是唯一确定的.即若有两个分解式
 \begin{equation}
 \begin{aligned}
 &f(x)=p_1(x)p_2(x)\cdots p_s(x)\\
 &f(x)=q_1(x)q_2(x)\cdots q_t(x)
 \end{aligned}
 \end{equation}
 则必有 $s=t$,并且适当调整因式的次序后有
 \begin{equation}
 p_i(x)=c_iq_i(x),i=1,2,\cdots,s
 \end{equation}
 其中,$c_i(i=1,2,\cdots,s)$ 都是非零常数.
\end{theorem}
\textbf{证明}:此处用数学归纳法来证明

1.存在性证明:因为1次多项式都是不可约的,所以当多项式 $f(x)$ 的次数 $\mathrm{deg}\;f(x)=1$ 时分解式必存在.

假设 $\mathrm{deg}\;f(x)\leq n$ 时分解式存在,则在 $\mathrm{deg}\;f(x)=n+1$ 时,只考虑 $f(x)$ 为可约的情形,否则存在性就成立了.即
\begin{equation}
f(x)=f_1(x)f_2(x)
\end{equation}
其中 $f_1(x),f_2(x)$ 的次数都小于 $n$.由归纳假定 $f_1(x),f_2(x)$ 都可分解为数域 $\mathbb{F}$ 上有限个不可约多项式的乘积.把 $f_1(x),f_2(x)$ 的分解式合起来就得到 $f(x)$ 的一个分解式.由数学归纳法原理,次数大于1的多项式分解式必存在.

 2.唯一性证明: