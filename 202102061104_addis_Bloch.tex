% 布洛赫理论
% 晶格|波函数|薛定谔方程|周期函数|bloch|布洛赫

\begin{issues}
\issueDraft
\end{issues}

\pentry{定态薛定谔方程\upref{SchEq}}

\subsection{一维薛定谔方程}
\footnote{参考\cite{GriffQ}}\textbf{布洛赫(Bloch)}理论也叫 Floquet 理论. 一维薛定谔方程中, 如果 $V(x)$ 是一个以 $a$ 为周期的函数, 那么解将满足
\begin{equation}\label{Bloch_eq1}
\psi(x+a) = \E^{\I K a}\psi(x)
\end{equation}
其中 $K$ 是一个常数. 令 $D(a)$ 为平移算符, 向右平移 $a$, 那么 $[D,H] = 0$. 所以存在能量和 $D$ 的共同本征态, 马上就得到\autoref{Bloch_eq1} .

波函数也可以记为
\begin{equation}
\psi(x) = \E^{\I K x} u(x)
\end{equation}
其中 $u(x)$ 是一个周期为 $a$ 的函数. 也就是波函数是一个振幅受周期性调制的平面波.

如果我们施加循环边界条件($N$ 是晶体一个方向的原子数, 阿伏伽德罗常数数量级)
\begin{equation}
\psi(x+Na) = \psi(x)
\end{equation}
得
\begin{equation}
K = \frac{2\pi}{a} \frac{n}{N} \qquad (n \in \mathbb Z)
\end{equation}

一个例子见一维 delta 势能晶格\upref{DelCry}.

\subsection{三维薛定谔方程}

(这参考的是 Brandsen?)布洛赫(Bloch)波函数定义为
\begin{equation}
\phi(\bvec r) = \E^{\I \bvec k \vdot \bvec r} u(\bvec r)
\end{equation}
其中 $u(\bvec r)$ 是一个周期函数. $\bvec k$ 是 crystal momentum vector

the energy eigenstates for an electron in a crystal can be written as Bloch waves.
