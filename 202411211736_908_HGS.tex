% 克里斯蒂安·惠更斯(综述)
% license CCBYSA3
% type Wiki

本文根据 CC-BY-SA 协议转载翻译自维基百科\href{https://en.wikipedia.org/wiki/Christiaan_Huygens}{相关文章}。

\begin{figure}[ht]
\centering
\includegraphics[width=6cm]{./figures/759a661ac1a7d67e.png}
\caption{惠更斯肖像,由卡斯帕·内彻绘于1671年,现藏于莱顿博尔哈夫博物馆[1]} \label{fig_HGS_1}
\end{figure}

克里斯蒂安·惠更斯,泽尔亨领主,英国皇家学会院士(/ˈhaɪɡənz/,音译‘海根斯’,[2] 美国亦发音为 /ˈhɔɪɡənz/,音译‘霍伊根斯’;[3] 荷兰语:[ˈkrɪstijaːn ˈɦœyɣə(n)s] ⓘ;也拼作 Huyghens;拉丁语:Hugenius;1629年4月14日-1695年7月8日),是一位荷兰数学家、物理学家、工程师、天文学家和发明家,被视为科学革命中的关键人物之一。[4][5] 在物理学领域,惠更斯在光学和力学方面做出了开创性的贡献;作为天文学家,他研究了土星的光环并发现了土星最大的卫星——泰坦。作为工程师和发明家,他改进了望远镜的设计,并发明了摆钟,这种时钟在近300年内是最精确的计时工具。他是一位才华横溢的数学家和物理学家,其著作首次通过一组数学参数对物理问题进行了理想化描述,[6] 并首次对一种无法直接观测的物理现象进行了数学和机械论的解释。[7]

惠更斯在其著作《De Motu Corporum ex Percussione》中首次正确地确定了弹性碰撞的定律,该书完成于1656年,但于1703年才在他去世后出版。[8] 1659年,惠更斯在其著作《De vi Centrifuga》中以几何方法推导出了经典力学中描述离心力的公式,这比牛顿早了十年。[9] 在光学领域,他因提出光的波动理论而闻名,这一理论发表于其1690年的《光论》(Traité de la Lumière)。惠更斯的光波理论最初被牛顿的光微粒理论所取代,直到1821年,奥古斯丁-让·菲涅耳(Augustin-Jean Fresnel)改进了惠更斯的原理,完整解释了光的直线传播和衍射现象。今天,这一原理被称为“惠更斯-菲涅耳原理”。

1657年,惠更斯发明了摆钟,并于同年获得专利。他对钟表的研究最终在《摆动时钟》(Horologium Oscillatorium,1673年)中发表,该书被认为是17世纪关于力学的重要著作之一。[6] 虽然书中包含了钟表设计的描述,但大部分内容是对摆动运动的分析和曲线理论。1655年,惠更斯与其兄弟康斯坦丁(Constantijn)开始研磨透镜,制作折射望远镜。他发现了土星最大的卫星——泰坦,并首次解释了土星的奇特外观是由于“一个薄而平坦的环,其不与土星接触,并倾斜于黄道面”。[10] 1662年,惠更斯开发了如今称为“惠更斯目镜”的装置,这是一种采用两个透镜的望远镜,能够减少色散现象。[11]

作为数学家,惠更斯发展了渐伸线的理论,并在《赌博中的计算》(Van Rekeningh in Spelen van Gluck)中研究了几何概率和点数问题。该书由弗朗斯·范·斯库腾(Frans van Schooten)翻译并以《赌博中的推理》(De Ratiociniis in Ludo Aleae,1657年)出版。[12] 惠更斯及其他人对期望值的使用后来启发了雅各布·伯努利(Jacob Bernoulli)对概率论的研究。[13][14]
\subsection{传记}
\begin{figure}[ht]
\centering
\includegraphics[width=6cm]{./figures/805ceabebaf64426.png}
\caption{康斯坦丁与他的五个孩子合影(克里斯蒂安位于右上方)。海牙莫里茨皇家美术馆。} \label{fig_HGS_2}
\end{figure}
克里斯蒂安·惠更斯于1629年4月14日出生在海牙的一个富有且有影响力的荷兰家庭,[15][16] 是康斯坦丁·惠更斯的次子。他以祖父的名字命名。[17][18] 他的母亲苏珊娜·范·巴尔勒(Suzanna van Baerle)在生下惠更斯的妹妹后不久去世。[19] 夫妇俩育有五个孩子:康斯坦丁(1628年)、克里斯蒂安(1629年)、洛德维克(1631年)、菲利普斯(1632年)和苏珊娜(1637年)。[20]

康斯坦丁·惠更斯是奥兰治家族的外交官和顾问,同时也是一位诗人和音乐家。他与欧洲各地的知识分子有广泛的书信往来,他的朋友包括伽利略·伽利莱、马林·梅森和勒内·笛卡尔。[21] 克里斯蒂安在16岁之前接受家庭教育,从小喜欢玩弄磨坊和其他机器的模型。他从父亲那里接受了全面的教育,学习语言、音乐、历史、地理、数学、逻辑和修辞,同时还学习舞蹈、击剑和骑马。[17][20]

1644年,惠更斯的数学导师是扬·扬斯·斯塔姆皮恩(Jan Jansz Stampioen),他为15岁的惠更斯布置了一份有关当代科学的高难度阅读清单。[22] 后来,笛卡尔对他在几何学方面的能力印象深刻,梅森则称他为“新阿基米德”。[23][16][24]
\subsubsection{学生时代}
16岁时,康斯坦丁送克里斯蒂安·惠更斯到莱顿大学学习法律和数学,他从1645年5月学到1647年3月。[17] 从1646年开始,弗朗斯·范·斯库滕(Frans van Schooten)成为莱顿大学的一名学者,并在笛卡尔的建议下接替斯塔姆皮恩(Stampioen),成为惠更斯及其兄长康斯坦丁·小惠更斯的私人导师。[25][26] 范·斯库滕为惠更斯提供了最新的数学教育,向他介绍了韦达(Viète)、笛卡尔和费马(Fermat)的研究成果。[27]

1647年3月起,惠更斯继续在新成立的布雷达奥兰治学院(Orange College)学习两年,该学院的管理者之一是他的父亲康斯坦丁。康斯坦丁深度参与了这所学院的事务,但学院仅持续到1669年,校长是安德烈·里韦(André Rivet)。[28] 惠更斯在学习期间寄住在法学家约翰·亨里克·道伯(Johann Henryk Dauber)家中,并由英国讲师约翰·佩尔(John Pell)教授数学。他在布雷达的学习结束于其兄弟洛德维克因与另一名学生决斗事件而中断学业的时期。[5][29] 惠更斯于1649年8月完成学业后离开布雷达,并短暂以外交官身份随纳骚公爵亨利(Henry, Duke of Nassau)出使。[17] 这次任务带他去了本特海姆(Bentheim)和弗伦斯堡(Flensburg)。随后,他前往丹麦,访问了哥本哈根和赫尔辛格(Helsingør),并希望穿越厄勒海峡(Øresund)前往斯德哥尔摩拜访笛卡尔,但因笛卡尔此时已去世而未能成行。[5][30]

尽管他的父亲康斯坦丁希望克里斯蒂安成为一名外交官,但种种原因阻止了他的从政之路。1650年开始的第一次无护国主时期(First Stadtholderless Period)导致奥兰治家族失去权力,从而削弱了康斯坦丁的影响力。此外,他也意识到儿子对这条职业道路毫无兴趣。[31]
\subsubsection{早期通信}
\begin{figure}[ht]
\centering
\includegraphics[width=6cm]{./figures/4dfc68904d4ad9e9.png}
\caption{惠更斯手稿中悬链线(catenary)图的插图。} \label{fig_HGS_3}
\end{figure}
惠更斯通常用法语或拉丁语写作。[32] 1646年,他还是莱顿大学的一名学生时,就开始与父亲的朋友马林·梅森(Marin Mersenne)通信,但梅森在1648年不久后去世。[17] 1647年1月3日,梅森在给康斯坦丁的信中称赞其子在数学上的才能,甚至恭维地将他比作阿基米德。[33]

这些信件显示了惠更斯早期对数学的兴趣。1646年10月,他讨论了悬索桥,并证明悬链线并非伽利略所认为的抛物线。[34] 后来,惠更斯在1690年与戈特弗里德·莱布尼茨通信时,将这种曲线命名为“悬链线”(catenaria,catenary)。[35]

在接下来的两年间(1647–1648),惠更斯写给梅森的信件涵盖了多种主题,包括自由落体定律的数学证明、格雷瓜尔·德·圣文森特(Grégoire de Saint-Vincent)提出的圆的求积问题(惠更斯证明其错误)、椭圆的求长、抛射体的运动和振动弦问题。[36] 梅森当时关注的一些问题,例如摆线(他将托里拆利关于这一曲线的论文寄给惠更斯)、振动中心和引力常数,成为惠更斯在17世纪晚期才认真研究的课题。[6] 梅森还研究了音乐理论。惠更斯偏好中全音律(meantone temperament),他在31平均律上进行了创新(虽然这一想法并非全新,早在弗朗西斯科·德·萨利纳斯就已知晓),并使用对数对其进一步研究,显示其与中全音律的紧密关系。[37]

1654年,惠更斯回到父亲在海牙的家中,完全投入研究工作。[17] 他们家还有另一座夏天常用的宅邸,位于附近的霍夫维克(Hofwijck)。尽管惠更斯在学术上非常活跃,他的学术生活并未使他摆脱时而袭来的抑郁情绪。[38]

之后,惠更斯发展了一批广泛的通信伙伴,但自1648年法国的五年内乱(“投石党之乱”)以来,这种联系遇到了一些困难。1655年访问巴黎时,惠更斯拜访了伊斯梅尔·布利奥(Ismael Boulliau)自我介绍,对方带他见了克劳德·米隆(Claude Mylon)。[39] 梅森周围的巴黎学者团体在1650年代仍保持联系,而担任秘书角色的米隆费尽心思让惠更斯与他们保持沟通。[40] 通过皮埃尔·德·卡卡维(Pierre de Carcavi),惠更斯于1656年与他非常敬仰的皮埃尔·德·费马(Pierre de Fermat)建立通信。尽管这一经历令人又喜又忧也颇费解,因为显然费马已经淡出主流研究,某些优先权的主张可能也难以成立。此外,此时惠更斯正寻求将数学应用于物理,而费马则更关注纯数学问题。[41]
\subsubsection{科学首秀}
\begin{figure}[ht]
\centering
\includegraphics[width=6cm]{./figures/ab6894350d5911e1.png}
\caption{克里斯蒂安·惠更斯,尚-雅克·克莱里翁创作的浮雕(约1670年)。} \label{fig_HGS_4}
\end{figure}
像他的一些同时代人一样,惠更斯常常不急于将自己的研究成果和发现发表成文,而是更喜欢通过书信传播自己的工作。[42] 在他早期的日子里,他的导师弗朗斯·范·斯科滕(Frans van Schooten)为他的工作提供了技术反馈,同时出于声誉考虑表现得非常谨慎。[43]

在1651年至1657年间,惠更斯发表了一系列作品,展示了他在数学方面的才华以及他对经典几何和解析几何的精通,这也使他在数学家中获得了更广泛的影响和声誉。[33] 大约在同一时期,惠更斯开始质疑笛卡尔关于碰撞定律的理论。笛卡尔的碰撞定律大部分是错误的,而惠更斯通过代数方法推导出了正确的定律,后来又用几何方法验证。他证明了对于任何物体系统,该系统的重心的速度和方向保持不变,这就是惠更斯所谓的“运动量守恒”。虽然当时其他人也在研究碰撞现象,但惠更斯的碰撞理论更为普遍。[5] 这些研究成果成为进一步讨论的主要参考点,并通过通信和《学者杂志》(Journal des Sçavans)上的一篇短文传播开来,但直到1703年出版的《物体碰撞运动》(De Motu Corporum ex Percussione)中才被更广泛的公众所知晓。[45][44]

除了数学和力学上的成就,惠更斯还做出了重要的科学发现:1655年,他首次确认土卫六是土星的一颗卫星;1657年,他发明了摆钟;1659年,他解释了土星奇怪的外观是由于其环形结构。这些发现使他在整个欧洲声名远扬。[17] 1661年5月3日,惠更斯与天文学家托马斯·斯特里特(Thomas Streete)和理查德·里夫(Richard Reeve)在伦敦使用里夫的望远镜观测了水星经过太阳的凌日现象。[46] 之后,斯特里特与赫维留斯(Hevelius)的记录发生了争论,这一争论由亨利·奥登堡(Henry Oldenburg)调解。[47] 惠更斯将耶利米·霍罗克斯(Jeremiah Horrocks)关于1639年金星凌日的手稿传递给赫维留斯,该手稿首次于1662年印刷出版。[48]

同年,罗伯特·莫雷爵士(Sir Robert Moray)向惠更斯寄送了约翰·格朗特(John Graunt)的生命表。不久后,惠更斯与他的兄弟洛德维克(Lodewijk)一起研究了寿命预期问题。[42][49] 惠更斯最终在假设死亡率均匀的前提下绘制出了首个连续分布函数图,并利用该图解决了联合年金的问题。[50] 与此同时,惠更斯也对西蒙·斯蒂文(Simon Stevin)的音乐理论产生了兴趣(惠更斯会演奏羽管键琴),但他对发表自己关于和声的理论兴趣不大,其中一些理论甚至在几个世纪内被遗忘。[51][52] 由于他对科学的贡献,伦敦皇家学会于1663年选举惠更斯为会员,使他成为该学会历史上首位外国成员,当时他仅34岁。[53][54]
\subsubsection{法国}
\begin{figure}[ht]
\centering
\includegraphics[width=6cm]{./figures/a624697b8c8e9004.png}
\caption{惠更斯位于画面中心偏右,《科学院的成立与天文台的创建,1666年》,作者亨利·泰斯特兰(Henri Testelin),约1675年创作。} \label{fig_HGS_5}
\end{figure}
蒙莫尔学会(Montmor Academy)成立于17世纪50年代中期,是梅森圈(Mersenne circle)在梅森去世后发展的形式。[55] 惠更斯参与了学会的辩论,并支持那些主张通过实验验证来避免业余态度的人。[56] 1663年,惠更斯第三次访问巴黎;当蒙莫尔学会于次年解散时,惠更斯提倡在科学上实施更符合培根主义的计划。两年后,即1666年,他受邀迁居巴黎,担任路易十四新成立的法国科学院(Académie des sciences)的领导职务。[57]

在巴黎的科学院期间,惠更斯得到了路易十四的首席大臣让-巴蒂斯特·柯尔贝尔(Jean-Baptiste Colbert)的重要支持和通信。[58] 然而,他与法国科学院的关系并不总是顺利的。1670年,因病重而担心自己可能去世,惠更斯选择弗朗西斯·弗农(Francis Vernon)来将他的研究资料捐赠给伦敦皇家学会。[59] 然而,法荷战争(1672–1678)及英国在其中的角色可能对他与皇家学会后来的关系产生了负面影响。[60] 皇家学会代表罗伯特·胡克(Robert Hooke)在1673年处理相关事务时缺乏应有的技巧。[61]

1671年,物理学家兼发明家丹尼斯·帕平(Denis Papin)成为惠更斯的助手。[62] 他们的一个合作项目是火药发动机,但未能直接取得成果。[63][64] 在科学院期间,惠更斯利用1672年刚建成的天文台进行进一步的天文观测。他在1678年将尼古拉斯·哈茨苏克尔(Nicolaas Hartsoeker)引荐给法国科学家,如尼古拉·马勒伯朗士(Nicolas Malebranche)和乔瓦尼·卡西尼(Giovanni Cassini)。[5][65]

1672年,年轻的外交官莱布尼茨在访问巴黎时与惠更斯会面,当时他正在徒劳地尝试与法国外交部长阿尔诺·德·蓬波纳(Arnauld de Pomponne)见面。莱布尼茨当时正在研究计算机,并在1673年初短暂访问伦敦后,从1673年到1676年间向惠更斯学习数学。[66] 在接下来的多年中,两人展开了广泛的通信。起初,惠更斯对接受莱布尼茨的微积分优势表现出一定的犹豫。[67]
\subsubsection{晚年}
\begin{figure}[ht]
\centering
\includegraphics[width=8cm]{./figures/ab39de254979c92f.png}
\caption{} \label{fig_HGS_6}
\end{figure}
1681年,惠更斯在经历了一次严重的抑郁症发作后搬回了海牙。1684年,他发表了关于他新发明的无筒空中望远镜的著作《简便天文镜》(Astroscopia Compendiaria)。1685年,他试图重返法国,但因《南特敕令》的废除而未能成行。1687年,他的父亲去世,他继承了霍夫维克庄园(Hofwijck),并于次年将其作为自己的住所。[31]

在他第三次访问英格兰期间,1689年6月12日,惠更斯与艾萨克·牛顿(Isaac Newton)亲自会面。他们讨论了冰洲石,并随后就阻力运动展开通信。[68]

在生命的最后几年,惠更斯重新关注数学领域,并于1693年观察到现在被称为“镶边”(flanging)的声学现象。[69] 两年后,即1695年7月8日,惠更斯在海牙去世,并与他的父亲一样被安葬在海牙大教堂(Grote Kerk)的一座无标记墓穴中。[70]

惠更斯终身未婚。[71]