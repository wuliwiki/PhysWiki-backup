% 偶极子矩阵

\pentry{长度规范\upref{LenGau}}

\footnote{本文参考 \cite{Brandsen}}本文使用原子单位. 使用长度规范\upref{LenGau}, 对于某种势能 $V$ 的束缚态 $\ket{\psi_a}$ 和 $\ket{\psi_b}$, 可以证明以下三种形式的偶极子矩阵是一样的.
\begin{equation}
q\mel{\psi_b}{\bvec r}{\psi_a} = -\frac{q}{m(E_b - E_a)}\mel{\psi_b}{\grad}{\psi_a} = \frac{q}{m(E_b-E_a)^2} \mel{\psi_b}{\grad V}{\psi_a}
\end{equation}
这三种形式分别称为偶极子矩阵的长度、速度和加速度规范.

\subsection{证明}
\begin{equation}
H_0 = -\frac{\laplacian}{2m} + V(\bvec r)
\end{equation}
\begin{equation}\label{DipEle_eq1}
\mel{\psi_b}{\bvec r}{\psi_a} = \frac{\mel{H_0\psi_b}{\bvec r}{\psi_a} - \mel{\psi_b}{\bvec r}{H_0\psi_a}}{E_b - E_a} = \frac{\mel{\psi_b}{H_0\bvec r - \bvec rH_0}{\psi_a}}{E_b - E_a}
\end{equation}
其中
\begin{equation}
H_0\bvec r - \bvec rH_0 = -\frac{1}{2m}(\laplacian \bvec r - \bvec r \laplacian)
\end{equation}
注意这里的算符 $\laplacian \bvec r$ 作用在波函数上是指 $\laplacian (\bvec r \psi)$ 而不是 $(\laplacian \bvec r) \psi$. 不难证明
\begin{equation}
\laplacian (\bvec r \psi) = 2 \grad \psi + \bvec r \laplacian \psi
\end{equation}
所以
\begin{equation}
H_0\bvec r - \bvec rH_0 = -\frac{1}{m}\grad \psi
\end{equation}
代入\autoref{DipEle_eq1} 得
\begin{equation}
\mel{\psi_b}{\bvec r}{\psi_a} = \frac{\mel{\psi_b}{\grad}{\psi_a}}{E_b - E_a}
\end{equation}


\autoref{VopEq_eq5}~\upref{VopEq}
