% 华东师范大学 2013 年 考研 量子力学
% license Usr
% type Note

\textbf{声明}:“该内容来源于网络公开资料,不保证真实性,如有侵权请联系管理员”

\subsection{简答题(每小题7分,共56分)}
\begin{enumerate}
\item 为什么波函数$\psi$,必定是复函数?
\item 试叙述量子力学基本假设一测量共设的要点。
\item $\psi(x,t)$ 和 $e^{i\theta}\psi(\vec x,t)$ 是否代表同一个量子态?并说明为什么,其$\theta(x)$ 是实函数。
\item 力学量之间的对易关系是否具有传递性?即:如果$A$与$B$对易,$B$与$C$对易成立,是否必有$A$与$C$对易成立?试用举例来证明你的结论。
\item 两个有限深方势阱深度相同,但宽度不同,与窄的势阱哪一个束缚态的个数多?为什么?
现有三种系统,其能级与其量子数$n$,的关系分别是正比于$n^2,n^{-2}$以及与$n$满足线性关系,请举例指出它们对应的分别可能是什么系统?
\item 什么是反常Zeeman效应?产生该效应的根源是什么?
\end{enumerate}
