% 半双线性形式

\begin{issues}
\issueDraft
\end{issues}

\footnote{参考 Wikipedia \href{https://en.wikipedia.org/wiki/Sesquilinear_form}{相关页面}.}\textbf{半双线性形式(sesquilinear form)}是双线性映射的一个变形. 它关于第一个变量是反线性(也叫\textbf{共轭线性 conjugate linear})的, 关于第二个变量是线性的.
\begin{definition}{}\label{sequil_def1}
复数域上的线性空间 $V$ 中, 若映射 $f:V\times V\to \mathbb C$ 对任意 $u, v, w\in V$, $a,b\in \mathbb C$ 满足($a^*$ 表示 $a$ 的复共轭\upref{CplxNo})
\begin{equation}\label{sequil_eq2}
f(au+bv, w) = a^*f(u, w) + b^*f(v, w)
\end{equation}
\begin{equation}\label{sequil_eq1}
f(u, av+bw) = af(u, v) + bf(u, w)
\end{equation}
那么就说该映射是\textbf{半双线性的(sesquilinear)}.
\end{definition}
如果满足 $f(u, v) = f(v, u)^*$, 就说它是\textbf{对称的}.

可以发现, 若把\autoref{sequil_eq2} 中的共轭符号去掉, 那么该定义就是双线性的定义(\autoref{Tensor_eq2}~\upref{Tensor}).

任意半双线性形式 $f(u, v)$ 可以唯一地由 $g(v) = f(v, v)$ 确定:
\begin{equation}
f(u, v) =\frac{1}{2}[g(u+v)-g(u)-g(v)]
-\frac{\I}{2}[g(u+\I v)-g(u)-g(\I v)]
\end{equation}
这叫做\textbf{极化恒等式(polarization identity)}. 这可以由定义直接证明.

和双线性函数(\autoref{Tensor_def1}~\upref{Tensor})一样, 半双线性形式也可以用矩阵表示, 若两个 $V$ 空间的基底分别为 $\{e_i\}$ 和 $\{e'_i\}$(当然也可以使用同一组基底), 那么表示为矩阵 $\mat F$ 后, 矩阵元就是 $F_{i,j} = f(e_i, e'_j)$, 且有
\begin{equation}
f(u, v) = \bvec u\Her \mat F \bvec v
\end{equation}

和二次型同理, 若半双线性形式 $f(u, v)$ 在两组基底下的矩阵分别为 $\mat F$ 和 $\mat F'$, 即 $F_{i,j} = f(e_i, e_j)$, $F'_{i,j} = f(e'_i, e'_j)$, 且任意 $u, v\in V$ 关于 $\{e_i\}$ 的坐标列矢量为 $\bvec u, \bvec v$, 关于 $\{e'_i\}$ 的坐标列矢量为 $\bvec u', \bvec v'$. 那么
\begin{equation}
f(u, v) = \bvec u\Her \mat F \bvec v = \bvec u'\Her \mat F' \bvec v'
\end{equation}
令两组基底的变换矩阵为 $\mat A$, 即 $\bvec u = \mat A\bvec u'$, $\bvec v = \mat A \bvec v$, 那么代入上式得
\begin{equation}
f(u, v) = \bvec u'\Her \mat A\Her \mat F \mat A\bvec v
\end{equation}



\begin{example}{半双线性型对应矩阵在不同基底下的转换规则}\label{sequil_ex1}
设 $F,F'$ 分别是半双线性型 $f$ 在基底 $\bvec e_i$ 和 $\bvec e'_i$ 下对应的矩阵,试证明 $F'=A^\dagger FA$,其中, $A$ 是基底 $\bvec e_i$ 到 $\bvec e'_i$ 的转换矩阵(或称过渡矩阵),$A^\dagger\equiv\qty(A^{T})^*$.

\textbf{证:}对任意双线性型 $g$ ,其满足关系
\begin{equation}
\begin{aligned}
&g(\bvec x,\bvec y)=\sum_{i,j}x_i y_j g_{ij}=X^TGY,\\
&X=(x_1,\cdots,x_n),\quad Y=(y_1,\cdots,y_n)
\end{aligned}
\end{equation}
而对半双线性型 $f$,则是
\begin{equation}
\begin{aligned}
&f(\bvec x,\bvec y)=\sum_{i,j}x_i^* y_j f_{ij}=\qty(X^T)^*FY=X^\dagger FY,\\
&X=(x_1,\cdots,x_n),\quad Y=(y_1,\cdots,y_n)
\end{aligned}
\end{equation}
于是对半双线性型成立
\begin{equation}
X^\dagger FY=(AX')^\dagger F(AY')={X'}^\dagger(A^\dagger FA)Y'
\end{equation}
又 $X^\dagger F Y=X'^\dagger F'Y'$,对比即得
\begin{equation}\label{sequil_eq3}
F'=A^\dagger F A
\end{equation}
事实上,由 ${X'}^\dagger(A^\dagger FA-F')Y'=0$ 对任意 $X'^\dagger$ 成立,只能是 $(A^\dagger FA-F')Y'=\bvec 0$,而 $Y'$ 是任意的,说明 $(A^\dagger FA-F')$ 是零矩阵,于是便得\autoref{sequil_eq3} .
\end{example}