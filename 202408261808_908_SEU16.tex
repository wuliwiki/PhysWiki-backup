% 东南大学 2016 年 考研 量子力学
% license Usr
% type Note

\textbf{声明}:“该内容来源于网络公开资料,不保证真实性,如有侵权请联系管理员”

\textbf{1.(15 分)}以下叙述是否正确:(1)宁称贫符成是危密练特,为是么汇梦待:(2)或海公浪高数一定是可归一化的:(3)时间反演对称性导致能尿分恒:(4)自欣角从经是最子为学中特有的为学藏、哈会具力学中没有对应的力学量:(5)角动最算符知广作在共同木念。

\textbf{2.(15 分)}设质量为 $m$ 的粒子在势场 $V(r)$ 中运动,波函数为 $\psi(r,t)$。

\begin{enumerate}
    \item  试证明粒子的能量平均值为
    \[
    E(t) = \int d^3r \, \omega(r,t), \quad \omega = \frac{\hbar^2}{2m} \nabla \psi^* \cdot \nabla \psi + \psi^* V \psi~
    \]
    
    \item  试证明能量守恒公式
    \[
    \frac{\partial \omega}{\partial t} + \nabla \cdot \mathbf{S} = 0, \quad S(r,t) = -\frac{\hbar^2}{2m} \left(\frac{\partial \psi^*}{\partial t} \nabla \psi + \frac{\partial \psi}{\partial t} \nabla \psi^*\right)~
    \]
\end{enumerate}
\textbf{3.(15 分)}粒子的轨道角动最算符定义为:
\[
\hat{l}_x = \hat{y} \hat{p}_z - \hat{z} \hat{p}_y, \quad \hat{l}_y = \hat{z} \hat{p}_x - \hat{x} \hat{p}_z, \quad \hat{l}_z = \hat{x} \hat{p}_y - \hat{y} \hat{p}_x,~
\]
基本对易关系为
$[\hat{x}_\alpha, \hat{p}_\beta] = i\hbar \delta_{\alpha\beta}$, 求对易式: $[\hat{l}_x, \hat{y}] ,  [\hat{l}_x, \hat{p}_y] [\hat{l}_x, \hat{l}_y] , .$

\textbf{4.(15 分)}

\textbf{5.(15 分)}

\textbf{6.(15 分)}

\textbf{7.(15 分)}