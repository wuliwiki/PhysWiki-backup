% 埃托雷·马约拉纳(综述)
% license CCBYSA3
% type Wiki

本文根据 CC-BY-SA 协议转载翻译自维基百科\href{https://en.wikipedia.org/wiki/Arthur_Compton}{相关文章}。

\begin{figure}[ht]
\centering
\includegraphics[width=6cm]{./figures/7dc439ef645ec5c1.png}
\caption{马约拉纳在1930年代} \label{fig_ATL_1}
\end{figure}
埃托雷·马约拉纳(Ettore Majorana,/maɪəˈrɑːnə/,[2] 意大利语:[ˈɛttore majoˈraːna];1906年8月5日出生——可能在1959年或之后去世)是意大利理论物理学家,曾研究中微子质量。1938年3月25日,他在购买了从那不勒斯到巴勒莫的船票后神秘失踪。

马约拉纳方程和马约拉纳费米子以他的名字命名。2006年,为了纪念他,设立了马约拉纳奖。
\subsection{生活与工作}
1938年,恩里科·费米曾这样评价马约拉纳:“世界上有几类科学家;第二或第三流的科学家尽最大努力,但永远不会走得太远。然后是第一流的科学家,他们做出了对科学进步至关重要的发现。但还有一些天才,比如伽利略和牛顿。马约拉纳就是其中之一。”
\subsection{数学天赋} 
马约拉纳出生在西西里岛的卡塔尼亚。他的叔叔奎里诺·马约拉纳也是一位物理学家。马约拉纳自幼展现出卓越的数学天赋,1923年开始在大学学习工程学,但在1928年,在埃米利奥·塞格雷的鼓励下转学物理。[4]: 69–72  他很年轻时就加入了恩里科·费米在罗马的研究小组,成为了著名的“潘尼斯佩尔纳街的男孩”之一,这个名字来源于他们实验室所在的街道地址。
\subsection{首次发表的学术论文}
\begin{figure}[ht]
\centering
\includegraphics[width=6cm]{./figures/99782c83004cd405.png}
\caption{} \label{fig_ATL_2}
\end{figure}

马约拉纳的第一篇论文涉及原子光谱学中的问题。他的第一篇论文发表于1928年,当时他还是一名本科生,这篇论文是与罗马物理研究所的助理教授乔瓦尼·詹蒂尔(Giovanni Gentile Jr.)共同撰写的。这项工作是费米原子结构统计模型(现称为托马斯-费米模型)在原子光谱学中的早期定量应用。

在这篇论文中,马约拉纳和詹蒂尔在该模型的框架内进行了从头算的计算,成功地解释了实验观察到的镧和铀的核心电子能量,以及在光谱中观察到的铯线的精细结构分裂。1931年,马约拉纳发表了关于原子光谱中自电离现象的第一篇论文,他称其为“自发电离”;同年,普林斯顿大学的艾伦·申斯通(Allen Shenstone)也发表了一篇独立的论文,将这一现象称为“自电离”,这个术语最早由皮埃尔·奥热(Pierre Auger)提出。自那时以来,这一术语便成为了该现象的标准称呼。

马约拉纳于1929年在罗马大学(La Sapienza)获得物理学博士学位。1932年,他发表了一篇关于原子光谱学中对准原子在时变磁场中的行为的论文。这个问题也被I.I. 拉比等人研究过,后来发展成为原子物理学的一个重要分支——射频光谱学。同年,马约拉纳发表了关于具有任意内禀动量的粒子相对论理论的论文,他在其中发展并应用了洛伦兹群的无限维表示,并为基本粒子的质量谱提供了理论依据。和马约拉纳的大多数论文一样,这篇论文也是用意大利语写的,因此在几十年里一直鲜为人知。[5]

1932年,伊雷内·约里奥-居里(Irène Joliot-Curie)和弗雷德里克·约里奥(Frédéric Joliot)的实验表明存在一种未知粒子,他们认为这可能是伽马射线。马约拉纳是第一个正确解读这一实验的学者,认为它需要一种中性电荷且质量大约和质子相同的新粒子;这种粒子就是中子。费米建议他写一篇关于这一话题的文章,但马约拉纳没有这么做。詹姆斯·查德威克(James Chadwick)在同年通过实验证明了中子的存在,并因此获得了诺贝尔奖。[6]

马约拉纳以不追求个人功劳而著称,他认为自己的工作微不足道。他一生只写了九篇论文。
\subsubsection{与海森堡和玻尔的合作}
“在费米的敦促下,马约拉纳于1933年初凭借国家研究委员会的资助离开意大利。他在德国莱比锡与维尔纳·海森堡会面。在后来写给海森堡的信中,马约拉纳透露,他在海森堡身上不仅发现了一位科学同事,还找到了一位温暖的个人朋友。”[4]: 71 

马约拉纳到达德国时,纳粹刚刚上台。他研究了一种核理论(1933年以德文发表),其对交换力的处理代表了对海森堡核理论的进一步发展。同年,他还前往哥本哈根,与诺贝尔奖得主尼尔斯·玻尔合作,玻尔既是海森堡的朋友,也是他的导师。
\subsubsection{疾病与孤立}
“1933年秋天,马约拉纳回到罗马,健康状况不佳。他在德国患上了急性胃炎,并显然遭受了神经衰弱的折磨。遵医嘱实行严格饮食后,他变得隐居起来,对家人的态度变得冷漠无情。他从德国写信给之前关系融洽的母亲,表示不会陪她去往常的海边度假。他在研究所的出现次数逐渐减少,很快几乎不再离家。这位前途无量的年轻物理学家变成了一位隐士。接下来的近四年时间里,他切断了与朋友的联系,也停止了发表文章。”[4]: 71 
\subsubsection{最后的研究}
在这些几年中,尽管很少发表文章,马约拉纳却撰写了许多关于地球物理学、电气工程、数学和相对论的小型研究。这些未发表的论文现存于比萨的伽利略研究中心,由埃拉斯莫·雷卡米和萨尔瓦托雷·埃斯波西托整理编辑。

马约拉纳最后一篇发表的论文于1937年完成,其中详细阐述了一种电子与正电子的对称理论。他预言,在费米子这一类粒子中,应该存在一些既是自身粒子又是自身反粒子的粒子。对马约拉纳方程的解产生了这些粒子,现称为马约拉纳费米子。有推测认为,宇宙中无法探测到但通过引力影响推测其存在的“暗物质”中,至少一部分可能由马约拉纳粒子组成。

1938年,年仅32岁的马约拉纳在没有参加任何考试的情况下,直接被任命为那不勒斯大学理论物理学的正教授。这一任命是基于他“在理论物理领域中享有的独特卓越声誉”。[7][8]
\subsubsection{关于中微子质量的研究}
马约拉纳在中微子质量理论方面进行了前瞻性的工作,截至2017年,这一领域仍是活跃的研究课题。[9]
\subsection{失踪}
据报道,马约拉纳在从那不勒斯前往巴勒莫的旅程之前,从自己的银行账户中提取了所有资金。[7] 他可能是希望前往巴勒莫拜访他的朋友埃米利奥·塞格雷(当时是当地大学的教授),但塞格雷当时正在加利福尼亚。

在巴勒莫,马约拉纳于1938年3月25日购买了一张返回那不勒斯的船票。他随后在未知情况下失踪。尽管进行了多次调查,他的遗体始终没有被找到,他的命运至今仍然不明。

在失踪当天,马约拉纳从巴勒莫写了一封信给那不勒斯物理研究所的主任安东尼奥·卡雷利:

亲爱的卡雷利:

我做出了一个不可避免的决定。这一点自私也没有,但我意识到,我的突然失踪会给您和学生们带来麻烦。为此,我请求您的原谅,尤其是因为我辜负了过去几个月里您对我的信任、真诚的友谊和同情。

我恳请您代我向研究所里我学会欣赏的每一位同事问好,特别是Sciuti。我将永远怀念他们,至少直到今晚11点,或许还会更久。

——E.马约拉纳

此信之后,马约拉纳迅速发送了一份电报,取消了之前的旅行计划。据称,他随后购买了从巴勒莫返回那不勒斯的船票,但之后再也没有人见过他。[7]
\subsubsection{调查与猜测}
意大利作家莱昂纳多·夏夏总结了关于马约拉纳失踪事件的一些调查结果和假说;[10] 然而,夏夏的一些结论遭到马约拉纳的前同事(包括E.阿马尔迪和E.塞格雷)的反驳。雷卡米对关于马约拉纳失踪的各种假说进行了批判性分析,包括夏夏提出的理论,并提出了马约拉纳可能前往阿根廷的证据和线索。[11][12][13]

意大利哲学家乔尔乔·阿甘本在2016年出版了一本书,研究了马约拉纳失踪的案件。[14]
\subsubsection{提出的解释}

关于马约拉纳失踪的几个假说包括:

\textbf{自杀假说} 

   由他的同事阿马尔迪(Amaldi)、塞格雷(Segrè)等人提出。[15]  

\textbf{移居阿根廷假说} 

   由埃拉斯莫·雷卡米(Erasmo Recami)和卡洛·阿尔特米(Carlo Artemi)提出,后者详细推测了马约拉纳可能移居阿根廷并在当地生活的假想重构。[需要引用]  

\textbf{移居委内瑞拉假说}

   Rai 3的访谈节目《谁见过他?》(Chi l'ha Visto?)发表声明称,马约拉纳在1955年至1959年间曾居住在委内瑞拉瓦伦西亚,并使用“比尼”(Bini)的姓氏。[16]  

\textbf{隐退修道院假说}

   由莱昂纳多·夏夏(Sciascia)提出,可能隐居于塞拉圣布鲁诺(Serra San Bruno)的修道院。[需要引用]  

\textbf{绑架或谋杀假说} 

   由贝拉(Bella)、巴尔托奇(Bartocci)等人提出,推测是为了避免马约拉纳参与原子武器的研发。[需要引用]  

\textbf{选择成为乞丐假说}  

   由巴斯科内(Bascone)和文图里尼(Venturini)提出,被称为“狗人”(omu cani)假说。[17]  
\subsubsection{2011年案件重启与结案 }
2011年3月,意大利媒体报道称,罗马检察院宣布对一名目击者的声明展开调查,该目击者称在二战后曾在布宜诺斯艾利斯与马约拉纳见过面。[18][19] 2011年6月7日,意大利媒体报道意大利宪兵的犯罪科学调查部门(RIS)分析了1955年在阿根廷拍摄的一张男子照片,发现其面部有10个特征与马约拉纳的面部相符。[20]  

2015年2月4日,罗马检察院发布声明称,马约拉纳在1955年至1959年间确实居住在委内瑞拉瓦伦西亚。[1] 根据新证据,检察院宣布此失踪案件正式结案,没有发现与他失踪相关的犯罪证据,认为失踪可能是出于个人选择,并推定他移居了委内瑞拉。[1][21]
\subsection{纪念百年诞辰}

2006年是马约拉纳诞辰一百周年。

为纪念马约拉纳诞辰百年,2006年10月5日至6日在他的出生地卡塔尼亚举行了题为“埃托雷·马约拉纳的遗产与21世纪的物理学”的国际会议。[22] 此次会议的论文集由国际著名科学家撰写,包括A. Bianconi、D. Brink、N. Cabibbo、R. Casalbuoni、G. Dragoni、S. Esposito、E. Fiorini、M. Inguscio、R. W. Jackiw、L. Maiani、R. Mantegna、E. Migneco、R. Petronzio、B. Preziosi、R. Pucci、E. Recami和Antonino Zichichi。论文集由SISSA的**POS Proceedings of Science**出版,编辑团队包括Andrea Rapisarda(主席)、Paolo Castorina、Francesco Catara、Salvatore Lo Nigro、Emilio Migneco、Francesco Porto和Emanuele Rimini。

一本收录了马约拉纳九篇论文的纪念书籍,包括评论和英文翻译,也由意大利物理学会于2006年出版。[23]

同样为了纪念百年诞辰《理论物理电子期刊》(EJTP)出版了包含20篇文章的特刊,专门探讨马约拉纳遗产的现代发展。EJTP还以他的名义设立了一项奖项,以纪念他的百年诞辰。马约拉纳奖章(Majorana Medal)或马约拉纳奖(Majorana Prize)是一项年度奖,旨在表彰在理论物理学(广义上的理论物理)中展现出非凡创造力、批判精神和数学严谨性的研究人员。2006年马约拉纳奖的获奖者是埃拉斯莫·雷卡米(Erasmo Recami,贝加莫大学与INFN)和乔治·苏达山(George Sudarshan,德克萨斯大学)。2007年马约拉纳奖的获奖者是李·斯莫林(Lee Smolin,加拿大周边理论物理研究所)、埃利亚诺·佩萨(Eliano Pessa,意大利帕维亚大学认知科学跨学科中心及心理学系)和马切洛·奇尼(Marcello Cini,罗马拉萨皮恩扎大学物理系)。
\subsection{另见}
\begin{itemize}
\item 克利福德模(Clifford module)  
\item 交换力(Exchange force)  
\item 法诺共振(Fano resonance)  
\item 《潘尼斯佩尔纳街的少年》(电影)*I ragazzi di via Panisperna*  
\item 兰道-泽纳公式(Landau-Zener formula)  
\item 神秘失踪于海上的人物名单(List of people who disappeared mysteriously at sea)  
\item 马约拉纳奖(Majorana Prize)  
\item 马约拉纳费米子(Majorana fermion)  
\item 马约子(Majoron)
\end{itemize}
\subsection{参考文献}
\begin{enumerate}
\item Palma, Ester (2015年2月4日)。“La Procura: Ettore Majorana vivo in Venezuela fra il 1955 e il 1959”。roma.corriere.it. 《晚邮报》。检索日期:2015年2月4日。
\item “Quantum Computation possible with Majorana Fermions” [在YouTube上],上传日期:2013年4月19日,检索日期:2019年12月14日。
\item “Ettore Majorana: genius and mystery”。《CERN通讯》。2006年7月24日。检索日期:2021年7月19日。
\item 《过去的伟大谜团》。纽约普莱森维尔:读者文摘协会。1991年。ISBN 0-89577-377-5。
\item D.M. (1966)。关于Majorana有关基本粒子论文的详细讨论。“Comments on a Paper by Majorana Concerning Elementary Particles”。《美国物理学杂志》34(4): 314–318。Bibcode:1966AmJPh..34..314F。CiteSeerX 10.1.1.522.8279。doi:10.1119/1.1972947。
\item “Ettore Majorana: genius and mystery”。《CERN通讯》。CERN。2006年7月24日。
\item Holstein, B.(2008年5月16日)。《埃托雷·马约拉纳神秘失踪》(PDF)。USC中微子研讨会。检索日期:2009年4月5日。
\item Holstein, Barry R(2009年6月1日)。《埃托雷·马约拉纳神秘失踪》。《物理学杂志:会议系列》。173 (1): 012019。Bibcode:2009JPhCS.173a2019H。doi:10.1088/1742-6596/173/1/012019。ISSN 1742-6596。
\item Barranco, J.; Delepine, D.; Napsuciale, M.; Yebra, A.(2017年)。《利用天体物理通量区分狄拉克和马约拉纳中微子》。arXiv:1704.01549 [hep-ph]。
\item Sciascia, Leonardo(1987年)[1975年]。《马约拉纳的失踪》(《摩罗事件与马约拉纳之谜》)。Einaudi出版社,1975年;Carcanet出版社,1987年。ISBN 978-0-85635-700-8。
\item Recami, Erasmo(2000年)。《马约拉纳事件:信件、证词与文件》。意大利罗马:Di Renzo出版社。
\item Recami, Erasmo(1975年)。《关于物理学家埃托雷·马约拉纳失踪的新证据》(意大利语)。《科学》杂志,110: 577–588。
\item Recami, Erasmo(1975年)。《关于物理学家埃托雷·马约拉纳失踪的新证据》。《科学》杂志,110: 589及后续。
\item Agamben, Giorgio(2016年)。《什么是真实的?马约拉纳的失踪》。Neri Pozza出版社。
\item Roncoron, Stefano(2012年3月15日)。《“突尼斯”备忘录:马约拉纳案的新拼图》(PDF,意大利语)。《新科学家》杂志,第58–68页。原文(PDF)存档于2014年1月15日。检索日期:2020年10月28日。
\item Mosconi, Barbara(2015年)。《“马约拉纳之谜通过《谁见过他?》得以解决”》。检索日期:2020年12月20日。
\item Bascone; Venturini, Franco(2010年)。《“埃托雷·马约拉纳的真实故事”在弗拉维奥剧院》。原文存档于2016年3月4日。检索日期:2015年10月17日。
\item 《马约拉纳之谜在73年后重新揭开 - 时事》。ANSA.it。检索日期:2013年4月29日。
\item 《Adnkronos 时事》。Adnkronos.com。2011年4月1日。检索日期:2013年4月29日。
\item 《这是马约拉纳的脸,10处相同点》。晚邮报。检索日期:2013年4月29日。
\item 《埃托雷·马约拉纳之谜可能已被解开》。Quantum Diaries。2014年8月27日。79823。
\item 《埃托雷·马约拉纳的遗产与21世纪物理学》。科学会议论文集。检索日期:2017年7月11日。
\item Majorana, Ettore(2006年)。Bassani, G.F.(编辑)。《科学论文集:为纪念出生一百周年》。博洛尼亚:SIF出版社。ISBN 978-88-7438-031-2。
\end{enumerate}
\subsection{进一步阅读}
\begin{itemize}
\item Amaldi, E.(1968年)。《埃托雷·马约拉纳的科学工作》(意大利语)。《Physis》,第X卷:173–187。——马约拉纳科学成果的总结(意大利语)。
\item Majorana, Ettore(2006年)。Bassani, G.F.(编辑)。《科学论文集:纪念诞辰百年》。博洛尼亚:SIF出版社。ISBN 978-88-7438-031-2。存档日期:2012年2月7日。——马约拉纳论文集,包含英文翻译和注释。
\item Recami, E.; Esposito, S.(编辑)(2006年)。《理论物理学未发表笔记》。Zanichelli出版社。
\item Artemi, Carlo(2007年)。《马约拉纳计划:一次完美的逃脱》。意大利罗马:De Rocco出版社。
\item Amaldi, E.(1968年)。《埃托雷·马约拉纳的回忆》。Giornale di Fisica,9期。
\item Recami, E.(2001年)。《马约拉纳事件》。意大利罗马:Di Renzo出版社。
\item Bascone, I.(1999年)。《关于埃托雷·马约拉纳失踪的猜测:在法西斯主义时期的西西里物理学家》。Ananke出版社。
\item Licata, I.(编辑)(2006年)。《当代物理学中的马约拉纳遗产》。意大利罗马:Di Renzo出版社。
\item Castellani, L.(2006年)[1974年]。《马约拉纳档案》,Fabbri兄弟出版社(第2版)。
\item Sciascia, L.(1975年)。《马约拉纳的失踪》。Adelphi出版社。
\item Bella, S.(1975年)。《关于一位科学家失踪的揭示:埃托雷·马约拉纳》。意大利文学。
\item Esposito, S.(2008年)。《埃托雷·马约拉纳及其七十年后的遗产》。Annalen der Physik,17(5): 302–318。arXiv:0803.3602。Bibcode:2008AnP...520..302E。doi:10.1002/andp.200810296。S2CID 14599270。
\item Majorana, Ettore(2009年)。Esposito, S.; Recami, E.; van der Merwe, A.(编辑)。《埃托雷·马约拉纳:理论物理学的未发表研究笔记》。基础物理学理论系列。第159卷。Springer出版社。ISBN 978-1-4020-9113-1,电子ISBN 978-1-4020-9114-8。
\item 《过去的伟大谜团》。纽约普莱森维尔:读者文摘协会。1991年。第69–72页。ISBN 0-89577-377-5。
\item Bartocci, U.(1999年)。《马约拉纳的失踪:一个国家事务?》。Andromeda出版社。
\item Sarasua, L.。《迈特纳教授的戒指:一位女性科学家的斗争》——通过Amazon.com提供。
\item Magueijo, J.(2009年)。《闪耀的黑暗》。纽约市,纽约:Basic Books出版社。ISBN 978-0-465-00903-9。
\item Pizzi, M.(2014年)[2012年]。《马约拉纳之海:三幕剧》。亚马逊KDP出版。
\item Esposito, Salvatore(2017年)。《埃托雷·马约拉纳——被揭示的天才与无尽的谜团》。Springer传记系列。Springer国际出版。ISBN 978-3-319-54318-5。
\end{itemize}
\subsection{外部链接}
\begin{itemize}
\item [《埃托雷·马约拉纳的遗产与21世纪物理学》](https://pos.sissa.it)。科学会议论文集 (POS)。意大利的的里雅斯特:SISSA。

- Zichichi, Antonino。《埃托雷·马约拉纳:天才与谜团》(PDF)。原文(PDF)存档于2019年4月4日。检索日期:2009年8月22日。

- [《埃托雷·马约拉纳:天才与谜团》](https://cerncourier.com)。书评。《CERN通讯》。

- [埃托雷·马约拉纳.eu主页](http://www.ettoremajorana.eu)。存档日期:2014年6月4日。

- [埃托雷·马约拉纳.it主页](http://www.ettoremajorana.it)。

- [埃托雷·马约拉纳基金会和科学文化中心](http://www.majorana-foundation.org)。

- [《当代物理学中的马约拉纳遗产》](http://www.majorana-legacy.org)。

- [马约拉纳奖](http://www.majoranaprize.org)。
\end{itemize}