% 帕斯夸尔·约尔丹(综述)
% license CCBYSA3
% type Wiki

本文根据 CC-BY-SA 协议转载翻译自维基百科\href{https://en.wikipedia.org/wiki/Pascual_Jordan}{相关文章}。

\begin{figure}[ht]
\centering
\includegraphics[width=6cm]{./figures/1d93dfb61493db05.png}
\caption{帕斯库尔·约当在1920年代} \label{fig_YRD_1}
\end{figure}
恩斯特·帕斯库尔·约当(Ernst Pascual Jordan,德语发音:[ˈɛʁnst pasˈku̯al ˈjɔʁdaːn];1902年10月18日-1980年7月31日)是一位德国理论与数学物理学家,对量子力学和量子场论做出了重要贡献。他为矩阵力学的数学形式奠定了基础,并发展了费米子的规范反对易关系。他还引入了**约当代数**,试图形式化量子场论;此后,这些代数在数学中得到了广泛应用。[1]

约当于1933年加入了纳粹党,但并未追随当时拒绝由阿尔伯特·爱因斯坦及其他犹太物理学家发展出的量子物理的德意志物理运动。二战后,他加入了保守派政党基督教民主联盟(CDU),并于1957年至1961年担任议会议员。
\subsection{家庭背景与教育}
约当出生于恩斯特·帕斯夸尔·约当(1858–1924)和艾娃·费舍尔(Eva Fischer)的家庭。恩斯特·约当是一位以肖像画和风景画闻名的画家,并在汉诺威工业大学担任艺术副教授。家族姓氏原为“Jorda”,起源于西班牙,家族中长子均取名为“Pasqual”或其变体“Pascual”。1815年滑铁卢战役后,家族定居于汉诺威,并在某个时期将姓氏改为“Jordan”(德语发音:[ˈjɔʁdaːn])。恩斯特·约当于1892年与艾娃·费舍尔结婚。

约当的祖先帕斯夸尔·约当(Pascual Jordan)是一位西班牙贵族及骑兵军官,在拿破仑战争期间与战后为英国效力,最终定居汉诺威。那时,汉诺威王室统治着英国。家族传统规定,每一代的长子必须以“Pascual”命名。[3] 约当自幼接受传统宗教教育,12岁时尝试将《圣经》的字面解读与达尔文进化论调和。他的宗教老师说服他科学与宗教并不矛盾(约当在其一生中撰写了许多关于两者关系的文章)。[3]

1921年,约当进入汉诺威工业大学,学习动物学、数学和物理学。按照当时德国大学生的常规,他在获得学位前转学至另一所大学。1923年,他来到当时在数学和物理科学领域处于巅峰的哥廷根大学,由数学家大卫·希尔伯特(David Hilbert)和物理学家阿诺德·索末菲(Arnold Sommerfeld)指导。在哥廷根期间,他曾短暂担任数学家理查德·柯朗(Richard Courant)的助手,随后在马克斯·玻恩(Max Born)的指导下研究物理学,并在遗传学家兼种族科学家阿尔弗雷德·库恩(Alfred Kühn)的指导下攻读博士学位。[4]

约当一生中受口吃困扰,在即兴讲话时常常严重结巴。[5] 1926年,尼尔斯·玻尔曾主动提出支付治疗费用。在威廉·伦茨(Wilhelm Lenz)的建议下,约当前往维也纳的阿尔弗雷德·阿德勒诊所寻求治疗。[6][7]
\subsection{科学研究}
帕斯库尔·约当与马克斯·玻恩和维尔纳·海森堡共同发表了一系列关于量子力学的重要论文。[8] 他在早期量子场论的研究中起到了开创性作用,但在二战前逐渐将重心转向宇宙学。[8]

约当提出了一种非结合代数,现称为**约当代数**,试图为量子力学和量子场论创建一个可观测量的代数体系。尽管这些代数未能用于其初衷,但它们后来在数学中得到了广泛应用。[9] 约当代数被应用于投影几何、数论、复分析、优化以及其他纯数学和应用数学领域。

1966年,约当出版了182页的著作《地球的膨胀:狄拉克引力假说的推论》(*Die Expansion der Erde. Folgerungen aus der Diracschen Gravitationshypothese*)[10],在其中根据保罗·狄拉克关于宇宙历史中引力逐渐减弱的假说,提出地球可能从一个直径仅约7,000公里(约4,300英里)的球体膨胀至目前的大小。他认为,这一理论可以解释地壳中柔韧的下层西玛层厚度相对均匀,而脆性的上层硅铝层则断裂成主要的大陆板块。随着地球膨胀成更平坦的表面,地表的山脉可能在这一过程中形成为受挤压的褶皱。[11] 尽管约当对地球膨胀理论投入了大量精力,但他的地质学工作从未被物理学家或地质学家认真对待。[12]
\subsection{政治活动}
第一次世界大战德国战败及《凡尔赛条约》对约当的政治信念产生了深远影响。他与许多同事一样认为条约不公正,但他更进一步,变得愈加民族主义和右翼化。他在1920年代后期撰写了许多提倡侵略性和好战立场的文章。他是反共产主义者,尤其对俄国革命和布尔什维克的崛起深感忧虑。[3] 约当曾以笔名“恩斯特·多梅尔”(Ernst Domeier)在多个极右翼期刊上发表文章,这一身份于1990年代被揭露。[9]

1933年,约当与菲利普·莱纳德和约翰内斯·斯塔克一样,加入了纳粹党,并加入了一支冲锋队(SA)单位。他支持纳粹的民族主义和反共产主义立场,但同时仍然“维护爱因斯坦”和其他犹太科学家。约当似乎希望能够影响新政权,其中一个项目是试图说服纳粹,现代物理学(尤其是由爱因斯坦发展、以及哥本哈根学派代表的量子理论)可以成为对抗“布尔什维克唯物主义”的解毒剂。然而,尽管纳粹欣赏他对政权的支持,他对犹太科学家及其理论的持续支持使他被视为政治上不可靠。[13][14]

1939年,约当加入德国空军,在佩内明德火箭中心担任一段时间的气象分析员。在战争期间,他试图让纳粹党对各种先进武器的计划产生兴趣,但这些建议因其政治不可靠而被忽视。这很可能是因为他过去与犹太科学家(特别是玻恩、理查德·柯朗和沃尔夫冈·泡利)的联系以及与“犹太物理学”的关联。回应路德维希·比伯巴赫时,约当写道:“德国与法国数学的差异不比德国与法国机枪的差异更本质。”[3]

如果约当没有加入纳粹党,他很可能因与玻恩的合作而获得诺贝尔物理学奖。而玻恩最终于1954年与瓦尔特·博特分享了物理学奖。[15][16]

战后,沃尔夫冈·泡利向西德当局表示约当已被“恢复名誉”,使他在经历两年的职业中断后得以重新获得学术职位。1953年,他在汉堡大学恢复了终身教授的地位,并一直任职到1971年退休。

尽管泡利劝阻,约当在冷战压力下重返政治。1957年,他代表保守派的基督教民主联盟当选德国联邦议院议员。同年,他支持阿登纳政府为联邦国防军装备战术核武器,而**哥廷根十八人**(包括玻恩和海森堡在内的一组德国物理学家)发布了《哥廷根宣言》以示抗议。这些问题进一步加剧了他与前同事和朋友之间的紧张关系。[3]
\subsection{精选作品}
\begin{itemize}
\item Born, M.; Jordan, P.(1925年)。《关于量子力学》(Zur Quantenmechanik)。《物理学杂志》(Zeitschrift für Physik),34 (1): 858。Bibcode:1925ZPhy...34..858B。doi:10.1007/BF01328531。S2CID 186114542。
\item Born, M.; Heisenberg, W.; Jordan, P.(1926年)。《关于量子力学. II》(Zur Quantenmechanik. II)。《物理学杂志》(Zeitschrift für Physik),35 (8–9): 557。Bibcode:1926ZPhy...35..557B。doi:10.1007/BF01379806。S2CID 186237037。
\item Jordan, P.(1927年)。《关于量子跳跃的量子力学表示》(Über quantenmechanische Darstellung von Quantensprüngen)。《物理学杂志》(Zeitschrift für Physik),40 (9): 661–666。Bibcode:1927ZPhy...40..661J。doi:10.1007/BF01451860。S2CID 122253028。
\item Jordan, P.(1927年)。《关于量子力学的新基础》(Über eine neue Begründung der Quantenmechanik)。《物理学杂志》(Zeitschrift für Physik),40 (11–12): 809–838。Bibcode:1927ZPhy...40..809J。doi:10.1007/BF01390903。S2CID 121258722。
\item Jordan, P.(1927年)。《现代物理学中的因果性与统计学》(Kausalität und Statistik in der modernen Physik)。《自然科学》(Die Naturwissenschaften),15 (5): 105–110。Bibcode:1927NW.....15..105J。doi:10.1007/BF01504228。S2CID 26167543。
\item Jordan, P.(1927年)。《量子力学统计解释的注解》(Anmerkung zur statistischen Deutung der Quantenmechanik)。《物理学杂志》(Zeitschrift für Physik),41 (4–5): 797–800。Bibcode:1927ZPhy...41..797J。doi:10.1007/BF01395485。S2CID 121174605。
\item Jordan, P.(1927年)。《关于量子力学的新基础 II》(Über eine neue Begründung der Quantenmechanik II)。《物理学杂志》(Zeitschrift für Physik),44 (1–2): 1–25。Bibcode:1927ZPhy...44....1J。doi:10.1007/BF01391714。S2CID 186228140。
\item Jordan, P.; von Neumann, J.; Wigner, E.(1934年)。《量子力学形式的代数推广》(On an Algebraic Generalization of the Quantum Mechanical Formalism)。《数学年鉴》(Annals of Mathematics),35 (1): 29–64。doi:10.2307/1968117。JSTOR 1968117。
\end{itemize}
\subsection{参考文献}

1. McCrimmon, Kevin(2004)。《约当代数的品味》(*A taste of Jordan algebras*)(PDF)。纽约:Springer出版社。ISBN 0-387-95447-3。

2. Jones, Sheilla(2008)。《量子十人:一个关于激情、悲剧、野心和科学的故事》(*The quantum ten: a story of passion, tragedy, ambition and science*)。牛津:牛津大学出版社。ISBN 9780195369090。

3. Schroer, Bert(2003)。《帕斯库尔·约当:他对量子力学的贡献及其在当代局域量子物理中的遗产》(*Pascual Jordan, his contributions to quantum mechanics and his legacy in contemporary local quantum physics*)。arXiv:hep-th/0303241。

4. Rechenberg, Helmut(2010)。《维尔纳·海森堡——原子的语言:生活与工作》(*Werner Heisenberg – Die Sprache der Atome. Leben und Wirken*)。Springer出版社。第367页。ISBN 978-3-540-69221-8。

5. Mehra, Jagdish; Rechenberg, Helmut(1982)。《量子理论的历史发展》(*The Historical Development of Quantum Theory*)。Springer出版社。第xvi页。

6. Rechenberg, Helmut。《维尔纳·海森堡——原子的语言》。Springer出版社,第367、549页。ISBN 978-3-540-69221-8。

7. Ehlers, Jürgen; Schücking, Engelbert(2002)。《但是约当是第一个》(*Aber Jordan war der Erste*)。**《物理学期刊》(Physik Journal)**,1 (11)。检索日期:2023年8月25日。

8. Silvan S. Schweber,《QED及其创造者:戴森、费曼、施温格与朝永振一郎》(*QED and the Men Who Made It: Dyson, Feynman, Schwinger, and Tomonaga*)。普林斯顿:普林斯顿大学出版社,1994年。ISBN 0-691-03327-7。

9. Dahn, Ryan(2023年1月1日)。《纳粹、移民与抽象数学》(*Nazis, émigrés, and abstract mathematics*)。**《今日物理》(Physics Today)**,76 (1): 44–50。Bibcode:2023PhT....76a..44D。doi:10.1063/PT.3.5158。S2CID 255619971。

10. *Die Wissenschaft*,第124卷。弗里德里希·维沃格与儿子出版社,布伦瑞克,1966年。

11. Heinz Haber:《地球的膨胀》(*Die Expansion der Erde*)。《我们的蓝色星球》(*Unser blauer Planet*)。Rororo非虚构书系列。莱因贝克:Rowohlt出版社,1967 [1965]。第48、52、54–55页。Bibcode:1967ubp..book.....H。

12. Kragh, Helge(2015)。《帕斯库尔·约当、引力变化与地球的膨胀》(*Pascual Jordan, Varying Gravity, and the Expanding Earth*)。**《物理学视角》(Physics in Perspective)**,17 (2): 107–134。Bibcode:2015PhP....17..107K。doi:10.1007/s00016-015-0157-9。S2CID 120065274。

13. Schücking, E. L.(1999)。《约当、泡利、政治、布莱希特和可变引力常数》(*Jordan, Pauli, Politics, Brecht, and a Variable Gravitational Constant*)。**《今日物理》(Physics Today)**,52 (10): 26–31。Bibcode:1999PhT....52j..26S。doi:10.1063/1.882858。

14. Schroer, Bert(2003年3月27日)。《帕斯库尔·约当:他对量子力学的贡献及其在当代局域量子物理中的遗产》(*Pascual Jordan, his contributions to quantum mechanics and his legacy in contemporary local quantum physics*)。arXiv:hep-th/0303241。

15. Bernstein, Jeremy(2005)。《马克斯·玻恩与量子理论》(*Max Born and the quantum theory*)。**《美国物理学杂志》(Am. J. Phys.)**,73 (11): 999–1008。Bibcode:2005AmJPh..73..999B。doi:10.1119/1.2060717。

16. Schroer, Bert(2003)。《帕斯库尔·约当:他对量子力学的贡献及其在当代局域量子物理中的遗产》(*Pascual Jordan, his contributions to quantum mechanics and his legacy in contemporary local quantum physics*)。arXiv:hep-th/0303241。