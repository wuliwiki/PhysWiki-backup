% 泰勒公式(综述)
% license CCBYSA3
% type Wiki

本文根据 CC-BY-SA 协议转载翻译自维基百科\href{https://en.wikipedia.org/wiki/Taylor\%27s_theorem}{相关文章}。

\begin{figure}[ht]
\centering
\includegraphics[width=10cm]{./figures/a2db9ae0e729b15e.png}
\caption{指数函数 \( y = e^x \)(红色)及其在原点附近的四阶泰勒多项式(绿色虚线)。} \label{fig_TLGS_1}
\end{figure}
在微积分中,泰勒定理给出了一个\( k \)次可导函数在某个给定点的近似,通过一个\( k \)次多项式,称为\( k \)阶泰勒多项式。对于一个光滑函数,泰勒多项式是该函数泰勒级数在 \( k \)阶的截断。一级泰勒多项式是该函数的线性近似,二级泰勒多项式通常称为二次近似\(^\text{[1]}\)。泰勒定理有多个版本,其中一些版本给出了函数通过其泰勒多项式近似的误差的明确估计。

泰勒定理以数学家布鲁克·泰勒命名,他在1715年提出了该定理的一个版本,\(^\text{[2]}\)尽管早在1671年,詹姆斯·格雷戈里就已提到过该结果的早期版本\(^\text{[3]}\)。

泰勒定理在初级微积分课程中教授,是数学分析中的一个核心基本工具。它提供了简单的算术公式,用于准确计算许多超越函数的值,如指数函数和三角函数。它是解析函数研究的起点,并在数学的各个领域、数值分析以及数学物理中具有基础性意义。泰勒定理也可以推广到多变量和向量值函数。它为一些开创性的早期计算机提供了数学基础:查尔斯·巴贝奇的差分机通过数值积分其泰勒级数的前七项来计算正弦、余弦、对数和其他超越函数。
\subsection{动机}
\begin{figure}[ht]
\centering
\includegraphics[width=10cm]{./figures/4b527c76614bf0b7.png}
\caption{函数 \( f(x) = e^x \)(蓝色)及其在 \( a = 0 \) 处的线性近似 \( P_1(x) = 1 + x \)(红色)。} \label{fig_TLGS_2}
\end{figure}
如果实值函数\( f(x) \)在点\( x = a \)处可导,那么它在该点附近有一个线性近似。也就是说,存在一个函数\( h_1(x) \),使得
\[
f(x) = f(a) + f'(a)(x - a) + h_1(x)(x - a), \quad \lim_{x \to a} h_1(x) = 0.~
\]
这里,
\[
P_1(x) = f(a) + f'(a)(x - a)~
\]
是\( f(x) \)在\( x \) 接近点\( a \)时的线性近似,其图像\( y = P_1(x) \)是\( y = f(x) \)在\( x = a \)处的切线。近似的误差是:\(R_1(x) = f(x) - P_1(x) = h_1(x)(x - a)\).

当\( x \)趋近于\( a \)时,这个误差比\( (x - a) \)收敛得更快,这使得\( f(x) \approx P_1(x) \)成为一个有用的近似。
\begin{figure}[ht]
\centering
\includegraphics[width=10cm]{./figures/5a6062276bf4fb22.png}
\caption{函数 \( f(x) = e^x \)(蓝色)及其在 \( a = 0 \) 处的二次近似 \( P_2(x) = 1 + x + \frac{x^2}{2} \)(红色)。注意到近似的改进。} \label{fig_TLGS_3}
\end{figure}
为了更好地近似\( f(x) \),我们可以拟合一个二次多项式,而不是线性函数:
\[
P_2(x) = f(a) + f'(a)(x - a) + \frac{f''(a)}{2}(x - a)^2.~
\]
这个多项式不仅匹配了\( f(x) \)在\( x = a \)处的一阶导数,而且匹配了二阶导数,正如通过微分可以看出的一样。

泰勒定理确保了在\( x = a \)的足够小邻域内,二次近似比线性近似更准确。具体来说,
\[
f(x) = P_2(x) + h_2(x)(x - a)^2, \quad \lim_{x \to a} h_2(x) = 0.~
\]
这里,近似的误差是
\[
R_2(x) = f(x) - P_2(x) = h_2(x)(x - a)^2,~
\]
考虑到\( h_2 \)的极限行为,随着\( x \)趋近于\( a \),这个误差比\( (x - a)^2 \)收敛得更快。

类似地,如果我们使用更高阶的多项式来近似 \( f \),我们可能得到更好的近似,因为这样我们可以在选定的基点匹配更多的导数。

一般来说,通过一个\( k \)次多项式近似函数时,误差会比\( (x - a)^k \)收敛得更快,随着\( x \)趋近于\( a \)。然而,也存在一些函数,即使是无限可导的函数,对于这些函数,增加近似多项式的阶数并不会提高近似的准确度:我们说这样的函数在\( x = a \)处不具备解析性:它不能仅通过该点的导数来(局部)确定。
\begin{figure}[ht]
\centering
\includegraphics[width=10cm]{./figures/49ba2aed0eecf75b.png}
\caption{函数 \( f(x) = \frac{1}{1 + x^2} \)(蓝色)通过其泰勒多项式 \( P_k \)(红色和绿色)的近似,其中 \( k = 1, \ldots, 16 \),以 \( x = 0 \)(红色)和 \( x = 1 \)(绿色)为中心。近似在区间 \( (-1, 1) \) 和 \( (1 - \sqrt{2}, 1 + \sqrt{2}) \) 之外没有任何改进。} \label{fig_TLGS_4}
\end{figure}
泰勒定理具有渐近性质:它仅告诉我们,使用\( k \)阶泰勒多项式\( P_k \)近似时,误差 \( R_k \)相对于任何非零的 \( k \) 次多项式,随着\( x \to a \)时趋向零的速度更快。它并没有告诉我们在扩展中心的任何具体邻域内误差有多大,但为此目的,存在针对余项的明确公式(如下所示),这些公式在对\( f \)进行某些额外规则性假设时有效。这些增强版的泰勒定理通常会在扩展中心的一个小邻域内提供对近似误差的统一估计,但这些估计不一定适用于过大的邻域,即使函数\( f \)是解析的。在这种情况下,可能需要选择多个具有不同扩展中心的泰勒多项式,才能对原始函数进行可靠的泰勒近似(参见图4的动画)。

我们可以通过余项来使用几种方法:
\begin{enumerate}
\item 估计误差:对于一个度数为\( k \)的多项式\( P_k(x) \),在给定的区间\( (a - r, a + r) \)上近似\( f(x) \)时,估计误差。(给定区间和度数,我们可以找到误差。)
\item 找到最小的度数\( k \):找到最小的度数\( k \),使得多项式\( P_k(x) \)在给定区间 \( (a - r, a + r) \) 上能够将\( f(x) \)近似到给定的误差容限内。(给定区间和误差容限,我们可以找到度数。)
\item 找到最大的区间\( (a - r, a + r) \):找到最大的区间\( (a - r, a + r) \),使得\( P_k(x) \)在该区间内将\( f(x) \)近似到给定的误差容限内。(给定度数和误差容限,我们可以找到区间。)
\end{enumerate}