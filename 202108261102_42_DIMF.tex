% 量纲式
% keys 量纲式|物理规律

\pentry{单位制和量纲\upref{USD}}
在单位制和量纲\upref{USD}这一节中我们提到了量纲式,并且给出了\autoref{USD_def1}~\upref{USD},本节将证明量纲式的通用表达式,并给出另外一种较为更“数学”的定义.本节的定理将给出一个极其重要的结论,它使得我们对物理规律有一个更深刻的认识.
\subsection{定义方程}
在单位制和量纲\upref{USD}这一节中,\autoref{USD_ex1}~\upref{USD} 和\autoref{USD_ex2}~\upref{USD} 提到了定义方程和终极定义方程,我们先给出它们的定义.
\begin{definition}{}
在一单位制中,定义导出量类 $\tilde{\boldsymbol{C}}$ 单位 $\hat{\boldsymbol{C}}$ 的物理规律对应的方程称为该单位制中$\hat{\boldsymbol{C}}$ 的\textbf{定义方程}.
\end{definition}
\begin{definition}{}
在某一单位制,若将导出量类 $\tilde{\boldsymbol{C}}$ 的单位 $\hat{\boldsymbol{C}}$ 的\textbf{定义方程} 中涉及的量类(该导出量类 $\tilde{\boldsymbol{C}}$ 除外)都用基本量类来表示,得到的定义方程称\t
\end{definition}

\begin{theorem}{}
任一单位制的任一导出单位的终定方程都是幂单项式,即
\end{theorem}