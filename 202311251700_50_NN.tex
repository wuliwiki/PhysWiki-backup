% 神经网络
% keys 神经网络 人工神经网络
% license Xiao
% type Tutor

\textbf{神经网络}(Neural network, NN),准确地说,是\textbf{人工神经网络}(Artificial neural network, ANN),在机器学习领域中是指“由具有适应性的简单单元组成的广泛并行互联的网络,其组织能够模拟生物神经系统对真实世界物体所做出的交互反应”[1]。

神经网络是机器学习中广泛使用的一种基本方法。该方法具有较好的曲线拟合能力,能够从数据中学习离散型、连续型或者向量型函数。

\subsection{动机}

神经网络最初是受到生物神经系统结构的启发,而提出的机器学习模型。生物的神经系统,比如人脑,从结构上讲,是由大量的基本单位——神经元通过各种复杂的互相连接而构成。从功能的角度,神经系统中的每个神经元都可以接收别的神经元传来的信号,然后做出处理,将处理后的结果,通过信号发送给其它与之连接的神经元。大量神经元能够同时协调工作,从而使得整个神经系统具有对各种环境刺激做出反应的能力,即\textbf{智能}(Intelligence)。

由此,人们受到启发,如果能够模拟生物神经系统的结构,并赋予其类似的信息传送和处理机制,则可以定义一个能够具有一定智能的数学模型。值得注意的是,虽然人工神经网络最初的想法是受到生物学的启发,但是在其后续实际研究过程中,站在计算机科学家的角度上来说,并不追求在每一个细节上都模拟生物神经系统。例如,人工神经元输出单一不变的值,然而生物神经元输出的是复杂的时序脉冲[2]。

\subsection{基本结构}

\subsubsection{1.神经元}

生物神经系统的基本单位是神经元,能够接收、处理和发送信号。人们将生物神经元抽象出一个简单模型,即\textbf{人工神经元}(Artificial neuraon),在机器学习领域内,通常就称\textbf{神经元}(Neuron)。在该模型中,神经元接收到其它多个神经元传来的输入信号,这些输入信号通过带有权重的连接进行传递,神经元接收到的总输入值将与神经元的阈值进行比较,然后通过\textbf{激活函数}\upref{ActFun}处理以产生神经元的输出[3]。
\begin{figure}[ht]
\centering
\includegraphics[width=10cm]{./figures/4181cdc1351396c0.png}
\caption{神经元示意图} \label{fig_NN_1}
\end{figure}
神经元的基本结构如\autoref{fig_NN_1} 所示。其中,$x_1, x_2, ..., x_i, ..., x_n$表示神经元的输入,$w_1, w_2, ..., w_i, ..., w_n$表示每个输入所对应的权重,$g$为激活函数,$w_0$为偏移量,$y$为神经元的输出值。输出和输入的关系是
\begin{equation}
y=g(w_1x_1+w_2x_2+...+w_ix_i+...+w_nx_n+w_0).~
\end{equation}
也可以写成向量形式:
\begin{equation}
y=g(\bvec w \bvec x).~
\end{equation}
其中,$\bvec w=(w_0, w_1, w_2, ..., w_i, ..., w_n)$,$\bvec x=(x_0, x_1, x_2, ..., x_i, ..., x_n)$,$x_0=1$.


\subsubsection{2.感知机与前馈神经网络}
\textbf{感知机}(Perceptron)是有两层神经元构成的最简单的神经网络。其结构中主要包含输入层和输出层。输入层能够接收外界传送来的输入信号,并传递给输出层。输出层就是一个神经元,功能是接收来自输入层传递来的信号,然后做出处理,并输出结果。
\begin{figure}[ht]
\centering
\includegraphics[width=5cm]{./figures/5aea3a7e0b71ef24.png}
\caption{简单的感知机示意图} \label{fig_NN_2}
\end{figure}
\autoref{fig_NN_2} 表示的是一个简单的感知机。其中有两层,分别为输入层和输出层。输入层有两个输入神经元,分别接收输入信号$x_1$和$x_2$。输出层有一个输出神经元,产生输出结果$y$。此感知机的输入层只是接收输入数据而不做处理,只有输出层的那个神经元有计算功能,因此,该感知机只有一层功能神经元(Functional neuron)。

感知机能够表示所有的原子逻辑函数:与(AND)、或(OR)、与非(NAND)、或非(NOR)[2]。然而,对于某些逻辑函数,单一的感知器无法表示。比如,异或运算。

\begin{figure}[ht]
\centering
\includegraphics[width=8cm]{./figures/ea59f6d033a62a51.png}
\caption{逻辑与问题$x_1x_2$} \label{fig_NN_4}
\end{figure}

\begin{figure}[ht]
\centering
\includegraphics[width=8cm]{./figures/5760f1ee034fa814.png}
\caption{逻辑或问题} \label{fig_NN_5}
\end{figure}

\autoref{fig_NN_4} 和\autoref{fig_NN_5} 分别表示的是两个输入变量$x_1$,$x_2$之间的逻辑与问题($x_1 \wedge x_2$)和逻辑或问题($x_1 \vee x_2$)。这两个基本的布尔运算很容易由感知机解决。

\begin{figure}[ht]
\centering
\includegraphics[width=10cm]{./figures/85760f884011b87c.png}
\caption{异或问题} \label{fig_NN_6}
\end{figure}
\autoref{fig_NN_6} 表示的是异或问题($x_1 \oplus x_2$)。这是一个非线性可分问题,靠单一的感知器无法解决。

感知机可以堆叠,层数也能加深,每层也可以有多个神经元。由多层神经元堆叠而成的具有层级结构的神经网络,称为\textbf{多层感知机}(Multi-layer perceptron, MLP),或者\textbf{前馈神经网络}(Feedforward neural network).之所以称之为“前馈”神经网络,是因为数据在网络中的流动方向始终是单向的,数据从浅层网络向深层网络传播,而不会返回。在多层网络结构中,只有相邻层的神经元之间才有连接,同一层的神经元之间是没有连接的,跨层的神经元也不会互相连接。
% 前馈神经网络是因为数据单向流动?此种表述可能有问题。无论是什么神经网络,即使是后面的循环神经网络,数据在两个互相连接的神经元之间的流动也是单向的。从数学图论的角度来说,全都是有向图。如果数据能够在两个相邻的神经元之间双向流动,即正反两个方向都能流动,那网络应该是无向图。因此,前馈网络与循环网络的区别是网络中是否存在环结构。

\begin{figure}[ht]
\centering
\includegraphics[width=8cm]{./figures/d2e6a4c6b99c91b4.png}
\caption{两层感知机} \label{fig_NN_3}
\end{figure}
\autoref{fig_NN_3} 表示的是一个简单的多层网络。其中,中间一层为\textbf{隐含层}(Hidden layer)。原始的输入信号,经过隐含层的处理,送到输出层,再由输出层做处理,最终产生结果作为整个网络的输出值。网络的输入层不对数据做处理,只有中间的隐含层和最后的输出层对数据做加工处理。因此,该网络拥有两层功能神经元。通常将具有两层功能神经元的感知机,称为“两层感知机”,或者“两层网络”,也有文献称其为“单隐层网络”。

% 是隐含层中的激活函数作用于仿射变换,从而产生非线性性?
隐含层的存在使得神经网络具有非线性可分的能力。事实上,两层感知机就能表示所有的逻辑函数[2]。隐含层的神经元的激活函数具有非线性性。将激活函数作用于线性变换的输出能够产生非线性变换[4]。

\subsection{感知机训练算法}





\subsubsection{参考文献}
\begin{enumerate}
\item T. Kohonen, “An introduction to neural computing,” Neural Networks, vol. 1, no. 1, pp. 3–16, 1988.
\item T. M. Mitchell, Machine learning. 1997.
\item 周志华. 机器学习[M]. 北京:清华大学出版社. 2016: 97
\item I. Goodfellow, Y. Bengio, and A. Courville, Deep learning. MIT press, 2016: 174.
\end{enumerate}