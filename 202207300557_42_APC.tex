% 布置、排列、组合
% keys 布置|排列|组合

\begin{issues}
\issueDraft
\end{issues}

\pentry{映射\upref{map}}
组合学的一个重要方面在与计数问题.实际上,在很长一段时间里,大多数数学工作者把组合学与“计数”当作一回事.“计数”往往是指找出进行某个确定的运算方法的个数,与之紧密联系的内容便是排列组合.

“排列”和“组合”这两个术语在高中部分已经熟悉.本节将给予这些概念于一个更完整、更直观的描述,这意味着要建立一些术语,而这些术语给出了这种直观的意义.排列组合在很多方面都有应用,本节的术语将使得这种应用更加直观.
\subsection{布置Arrangement}
排列组合的很多应用在于将一定数量的“物体”放置在一定数量的“房间”(盒子)中,下面的定义使得这种应用更加明显.
\begin{definition}{物体、房间、布置}
设 $X,Y$ 是有限集,把 $X$ 的元素叫做\textbf{物体},把 $Y$ 的元素叫\textbf{房间}.映射 $f:X\rightarrow Y$ 称为\textbf{物体集合 $X$ 到房间集合 $Y$ 的一种\textbf{布置}(arrangement)}.
\end{definition}
物体集合 $X$ 到房间集合 $Y$ 的映射 $f:X\rightarrow Y$ 使得 $y_i$ 与 $X$ 的一些元素构成的集合
\begin{equation}
\{x|x\in X,f(x)=y_i\}
\end{equation}
相对应.称 $f$ 为布置暗示着 $f$ 将 $X$ 中的物体放置或分配到 $Y$ 中的房间内.

和很多命名一样,关于集合的元素是为了方便人们的记忆和应用.比如研究矢量的集合称为矢量空间,集合中的元素称为矢量;研究点集拓扑学的集合称为拓扑空间,其上的元素称为点.对应的,与组合数学对应的集合分别称为物体(定义域)和房间(值域),而物体到房间的映射称为布置.

当 $\abs{X}=n,\abs{Y}=m$ 时,每个函数 $f$ 和一个字符串“$f(x_1)\cdots f(x_n)$” 相对应.往往字符串代表着一个字,比如字 word 是由字母 “w”,“o”,“r”,“d” 构成的, 下面的定义给出了这种直观上的意义.
\begin{definition}{字母,字}\label{APC_def1}
设 $X,Y$ 分别是基数为 $\abs{X}=n,\abs{Y}=m$ 的有限集, $f:X\rightarrow Y$ 是 $X$ 到 $Y$ 上的映射,则称 $Y$ 的元素为\textbf{字母}, $f$ 是由 $Y$ 中的字母形成的长度为 $n$ 的一个\textbf{字} $f(x_1)\cdots f(x_n)$.
\end{definition}
上面定义中,$X$ 可看成给出了字的一个顺序.

通过这些概念,得到以下一些定理.

\begin{theorem}{}
设 $X,Y$ 分别是基数为 $\abs{X}=n,\abs{Y}=m$ 的有限集,则映射 $f:X\rightarrow Y$ 的个数等于 $m^n$.
\end{theorem}
\textbf{证明:}由