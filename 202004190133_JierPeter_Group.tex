% 群论
\pentry{朴素集合论\upref{NSet}}
\pentry{逻辑量词}

\subsection{基本概念}

\begin{definition}{二元运算}
给定一个非空集合$A$,取A中任意两个元素$a, b$(a和b可能是同一个元素).如果有一条规则使得两个元素可以组合,指向另一个元素$c$,则称这条规则为一个\textbf{二元运算}(operation).运算的符号可以任意决定,如果使用
\end{definition}

运算的符号可以任意决定,如果使用$\cdot$作为运算符,那么定义中的情况就可以简单写为$a\cdot b=c$.注意,这里的$\cdot$只是表示某一个运算,不一定是我们通常的乘法或点乘运算.

和二元运算类似,我们也可以称更多元素间相组合的规则为多元运算.

\begin{definition}{群}
一个群$(G, \cdot)$是在集合$G$上赋予了一个运算$\cdot$的结构,该运算满足以下要求:\\


1.封闭性:$\forall x, y\in G, x\cdot y\in G$,即:任意G中元素x,y满足$x\cdot y$仍是G中元素;\\

2.结合性:$\forall x, y, z\in in G,\cdot x·\cdot y· z)=\cdot x· \cdot)· z$;\\

3.单位元存在性:$\exists e\in in G\forall ∀\in x∈ G,\cdot e· x\cdot x· e=x$;\\

4.逆元存在性:$\forall x\in G, \exists y\in G, x\cdot y=y\cdot x=e$.通常我们会把这样的$y$称作$x$的逆元,并记为$x^{-1}$. 

\end{definition}

实际上,我们可以用更为弱化的公理系统来定义群,比如只要求存在左逆元,即第4条中只要求$\forall x\in G, \exists y\in G, y\cdot x=e$. 在这种情况下我们仍然可以证明左逆元都是右逆元.但是为了方便理解,我们用了以上对称的公理系统.

\subsection{群的例子}


\begin{example}{二元群}\label{Group_ex1}
定义一个只含有两个元素的集合,记为$\{0, 1\}$.在这个集合上定义运算$\cdot$,由于只有四种运算方式,所以可以通过列举出每一个运算的结果来定义这个运算:
$$0\cdot0=0$$
$$0\cdot1=1$$
$$1\cdot0=1$$
$$1\cdot1=0$$

容易验证,这个运算满足四个群运算的定义,因此这个二元集合配上运算$\cdot$构成一个群.

\end{example}

在以上例子中,尽管$0\cdot1=1\cdot0$,我依然把它们分别单独写了出来,这是因为群的运算没有要求可交换,也就是说,群运算允许$x\cdot y\neq y\cdot x$的存在.