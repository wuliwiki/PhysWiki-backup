% Hubbard模型
% Hubbard model|hubbard model|Hubbard模型|hubbard模型


\subsection{hubbard模型的哈密顿量}

在本节中,通过对电子多体系统之间的相互作用进行近似计算,仅考虑其中的库伦相互作用来得到Hubbard模型的哈密顿量。

选择原子轨道基$|j\rangle =c_j^\dagger |0\rangle$,写出系统的哈密顿量:

\begin{equation}
H=\sum\limits_{i,j,\sigma}\langle i |H_0| j \rangle c_{i,\sigma}^\dagger c_{j,\sigma}^~+\sum\limits_{l,m,k,n,\sigma,\sigma'}\langle l,m|V|k,n\rangle c_{l,\sigma}^\dagger  c_{m,\sigma'}^\dagger c_{n,\sigma'}^~c_{k,\sigma}^~~.
\end{equation}

上式中的$H_0$代表动能部分的哈密顿量,其对应的$\sum\limits_{i,j,\sigma}\langle i |H_0| j \rangle c_{i,\sigma}^\dagger c_{j,\sigma}^~$部分为跃迁项。

上式中后一部分代表相互作用项,由于库仑相互作用并不影响自旋,所以电子被相互作用散射后自旋不变,这一点在哈密顿量中的体现则为发生散射后原本自旋为$\sigma$的粒子自旋还为$\sigma$,自旋为$
\sigma'$的粒子散射后自旋还为$\sigma'$。也就是说上式中后一项中$l$和$k$对应的是同一粒子的两个态,而$m$和$n$对应的是同一个粒子的两个态。

下面分别对哈密顿量的两项进行近似计算。

\subsubsection{跃迁项}
作为近似,我们仅考虑最近邻跃迁:

\begin{equation}
\langle i|H_0|j \rangle=\left\{
\begin{array}{lc}
~~\varepsilon~,~~~~~ i=j, \\
~-t~,~i,j\text{最近邻,} \\
~~0~,~~~~\text{其他情况。}
\end{array}\right.~
\end{equation}

这样便有:

\begin{equation}
\langle i|H_0|j \rangle=\sum\limits_{i,\sigma}\varepsilon c_{i,\sigma}^\dagger c_{i,\sigma}^~-\sum\limits_{i,\Delta,\sigma}t\left(c_{i+\Delta,\sigma}^\dagger c_{i,\sigma}^~+h.c.\right)~.
\end{equation}

\subsub


