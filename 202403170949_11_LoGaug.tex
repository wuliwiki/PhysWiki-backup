% 洛伦兹规范
% keys 规范变换|标势|矢势|麦克斯韦方程组
% license Xiao
% type Tutor

\begin{issues}
\issueDraft
\end{issues}

\pentry{规范变换\nref{nod_Gauge}}{nod_7917}

如果令
\begin{equation}
\div \bvec A = -\mu_0 \epsilon_0 \pdv{\varphi}{t}~,
\end{equation}
那么标势和矢势就符合\textbf{洛伦兹规范}。 

麦克斯韦方程组(\autoref{eq_EMPot_5}~\upref{EMPot})将变为十分对称的形式
\begin{equation}\label{eq_LoGaug_1}
\laplacian \varphi - \mu_0\epsilon_0 \pdv[2]{\varphi}{t} = -\frac{\rho}{\epsilon_0}~,
\end{equation}
\begin{equation}\label{eq_LoGaug_2}
\laplacian \bvec A - \mu_0\epsilon_0 \pdv[2]{\bvec A}{t} = -\mu_0 \bvec j~.
\end{equation}

\addTODO{应该先讲电动力学再讲相对论, 参考格里菲斯的顺序。 另开词条。}
这个形式的优点是按照相对论章节中的习惯,我们令 $\mu_0=\epsilon_0=1$,那么势的麦克斯韦方程组就可以写成

\begin{equation}
\laplacian \varphi -  \pdv[2]{\varphi}{t} = -\rho~,
\end{equation}
\begin{equation}
\laplacian \bvec A - \ \pdv[2]{\bvec A}{t} = - \bvec j~.
\end{equation}

选取符号差为$(-+++)$的闵可夫斯基度规,我们可以简化上述方程为:

\begin{equation}\label{eq_LoGaug_4}
\square^2 \varphi = -\frac{\rho}{\epsilon_0}~,
\end{equation}
\begin{equation}\label{eq_LoGaug_5}
\square^2 \bvec{A} = -\mu_0 \bvec j~.
\end{equation}

其中 $\square^2$ 被称为\textbf{达朗贝尔算子(d' Alembertian operator)},在该度规下定义为:
\begin{equation}\label{eq_LoGaug_3}
\square^2=\square^\mu\square_\mu=\square^\mu\square^\nu g_{\mu\nu}=-(\frac{\partial}{\partial x^0})^2+(\frac{\partial}{\partial x^1})^2+(\frac{\partial}{\partial x^2})^2+(\frac{\partial}{\partial x^3})^2~,
\end{equation}
其中又有
\begin{equation}
\square^\mu=\frac{\partial}{\partial x^\mu}~.
\end{equation}
可见 $\square^\mu$ 直接就是 $\nabla^i$ 加了时间项之后的推广。

注意\autoref{eq_LoGaug_3} 中 $(\partial/\partial x^0)^2$ 前面的负号,这是从闵可夫斯基度规 $g_{\mu\nu}$ 中时间项的负号得来的。这一点和我们在相对论中的规范不同,在相对论中闵可夫斯基度规的时间项为正、空间项为负。







