% 连续统假设(综述)
% license CCBYSA3
% type Wiki

本文根据 CC-BY-SA 协议转载翻译自维基百科\href{https://en.wikipedia.org/wiki/Continuum_hypothesis}{相关文章}。

在数学中,特别是集合论中,连续统假设(缩写为CH)是一个关于无限集合可能大小的假设。它陈述了:

没有一个集合,其基数严格介于整数和实数之间。

或者等价地:

实数的任何子集要么是有限的,要么是可数无限的,要么具有实数的基数。

在包含选择公理的泽梅洛–弗兰克尔集合论(ZFC)中,这与以下的阿列夫数方程等价:\()
2^{\aleph_0} = \aleph_1\)或者用贝斯数表示更简洁地写作:\(\beth_1 = \aleph_1\)

连续统假设由格奥尔格·康托尔于1878年提出,\(^\text{[1]}\)并且确定其真伪是1900年希尔伯特提出的23个问题中的第一个。这个问题的答案与ZFC独立,因此可以将连续统假设或其否定作为公理加入到ZFC集合论中,且如果ZFC是一致的,则所得到的理论是一致的。1963年,保罗·科恩证明了这一独立性,补充了1940年库尔特·哥德尔的早期工作。\(^\text{[2]}\)

该假设的名称来源于实数的“连续统”一词。
\subsection{历史} 
康托尔认为连续统假设为真,并且多年来他徒劳地尝试证明它。\(^\text{[3]}\)它成为大卫·希尔伯特在1900年巴黎国际数学家大会上提出的“重要未解问题”列表中的第一个。那个时候,公理化集合论尚未形成。1940年,库尔特·哥德尔证明了连续统假设的否定,即存在一个具有中间基数的集合,不能在标准集合论中证明。\(^\text{[2]}\)连续统假设的独立性的第二部分——即无法证明不存在中间大小的集合——在1963年由保罗·科恩证明。\(^\text{[4]}\)
\subsection{无限集合的基数} 
如果两个集合之间存在一个双射(即一一对应),则称它们具有相同的基数或基数。直观上,两个集合\( S \)和\( T \)具有相同的基数意味着可以将\( S \)中的元素与\( T \)中的元素配对,使得\( S \)中的每个元素都与\( T \)中的恰好一个元素配对,反之亦然。因此,集合\( \{\text{banana}, \text{apple}, \text{pear}\} \)与集合\( \{\text{yellow}, \text{red}, \text{green}\} \)具有相同的基数,尽管这两个集合包含不同的元素。

对于像整数集合或有理数集合这样的无限集合,证明两个集合之间存在双射变得更加困难。有理数\( \mathbb{Q} \)看似对连续统假设形成了一个反例:整数形成有理数的一个真子集,而有理数又是实数的真子集,因此直观上有理数比整数多,实数比有理数多。然而,这种直观分析是错误的,因为它没有考虑到所有三个集合都是无限的。或许更重要的是,它实际上将集合\( \mathbb{Q} \)的“大小”概念与其上的顺序或拓扑结构混淆了。实际上,事实证明有理数集合可以与整数集合一一对应,因此有理数集合的大小(基数)与整数集合相同:它们都是可数集合。\(^\text{[5]}\)

康托尔给出了两个证明,证明整数集合的基数严格小于实数集合的基数(参见康托尔的第一次不可数性证明和康托尔的对角线论证)。然而,他的证明并未给出整数的基数比实数的基数小的程度。康托尔提出了连续统假设,作为解决这个问题的可能方案。

简单来说,连续统假设(CH)声明实数集合的基数是比整数集合的基数大的最小可能基数。即,实数的每个子集\( S \subseteq \mathbb{R} \)要么可以一一映射到整数,要么实数可以一一映射到\( S \)。由于实数的基数与整数的幂集的基数相等,即\( |\mathbb{R}| = 2^{\aleph_0} \),所以连续统假设可以重述为:

\textbf{连续统假设}— \(\nexists S : \aleph_0 < |S| < 2^{\aleph_0}\)

假设选择公理成立,存在一个唯一的最小基数\( \aleph_1 \),它大于\( \aleph_0 \),而连续统假设则等价于等式:\(2^{\aleph_0} = \aleph_1\)\(^\text{[6][7]}\)
\subsection{从ZFC的独立性} 
连续统假设(CH)与泽梅洛–弗兰克尔集合论(ZF)的独立性来源于库尔特·哥德尔和保罗·科恩的联合研究。

哥德尔\(^\text{[8][2]}\)证明了,即使采用选择公理(AC),也无法从ZF中反驳连续统假设(CH),即在ZFC中也无法反驳。哥德尔的证明表明,假设只有ZF的公理,连续统假设(CH)和选择公理(AC)在构造宇宙\( L \)中成立,\( L \)是ZF集合论的一个内模型。ZF的内模型中存在附加公理的存在表明这些附加公理与ZF是(相对)一致的,前提是ZF本身是一致的。由于哥德尔的不完全性定理,ZF本身无法证明这一条件,但这一条件被广泛认为是正确的,并且可以在更强的集合论中证明。

科恩\(^\text{[4][9]}\)证明了连续统假设(CH)无法从ZFC公理中证明,从而完成了整体的独立性证明。为了证明这一结果,科恩发展了强迫法,这一方法已成为集合论中的标准工具。本质上,这种方法从一个包含连续统假设成立的ZF模型开始,并构造另一个模型,使得新模型包含的集合比原始模型更多,并且在新模型中连续统假设不成立。科恩因其证明在1966年获得了菲尔兹奖。

科恩的独立性证明表明,连续统假设(CH)与ZFC是独立的。进一步的研究表明,在ZFC的框架下,CH与所有已知的大基数公理也是独立的。\(^\text{[10]}\)此外,已证明连续统的基数\( \mathfrak{c} = 2^{\aleph_0} \)可以是与Kőnig定理一致的任何基数。Solovay的一个结果,证明在科恩关于连续统假设独立性的结果之后不久,显示在任何ZFC模型中,如果 \( \kappa \)是一个不可数的末尾基数,则存在一个强迫扩展,使得\( 2^{\aleph_0} = \kappa \)。然而,根据Kőnig定理,假设\( 2^{\aleph_0} \)是\( \aleph_\omega \)或\( \aleph_{\omega_1 + \omega} \)或任何具有末尾基数\( \omega \)的基数是不可一致的。

连续统假设与分析学、点集拓扑学和测度论中的许多陈述密切相关。由于其独立性,许多在这些领域中的重大猜想随后也被证明是独立的。

从ZFC的独立性意味着在ZFC内证明或反驳CH是不可能的。然而,哥德尔和科恩的负面结果并未被普遍接受为消除对连续统假设的所有兴趣。连续统假设仍然是一个活跃的研究话题:有关当前研究状态的概述,请参见Woodin\(^\text{[11][12]}\)和Koellner\(^\text{[13]}\)

连续统假设和选择公理是最早被证明与ZF集合论独立的真正数学陈述之一。尽管早在二十多年前就已知存在一些与ZFC独立的陈述:例如,假设ZFC的良好健全性和一致性,哥德尔于1931年发布的不完全性定理证明了存在一个正式陈述Con(ZFC)(对于每个适当的哥德尔编号方案),它表达了ZFC的一致性,并且该陈述也与ZFC独立。后者的独立性结果确实适用于许多理论。
\subsection{支持和反对连续统假设的论点}  
哥德尔认为连续统假设(CH)是错误的,他认为他证明CH与ZFC一致仅仅表明泽梅洛–弗兰克尔公理并没有充分地刻画集合的宇宙。哥德尔是一个柏拉图主义者,因此他在断言命题的真伪时并不关心它们是否可证明。尽管科恩是一个形式主义者,\(^\text{[14]}\)他也倾向于否定CH。

历史上,支持“丰富”和“大”集合宇宙的数学家反对CH,而支持“整洁”和“可控”集合宇宙的数学家则支持CH。类似的论点也被提出,用于支持和反对构造性公理,该公理隐含着CH。最近,马修·福尔曼指出,本体论的极大主义实际上可以用来支持CH,因为在具有相同实数的模型中,具有“更多”实数集合的模型更有可能满足CH。\(^\text{[15]}\)

另一种观点是,集合的概念不够具体,无法确定CH是否为真或为假。这个观点早在1923年就被斯科勒姆提出,甚至在哥德尔的第一不完全性定理之前。斯科勒姆基于现在被称为斯科勒姆悖论的观点进行了论证,后来通过CH与ZFC公理的独立性得到了支持,因为这些公理足以确立集合和基数的基本性质。为了反对这一观点,足够做出新的公理,这些公理得到直觉的支持,并解决CH的真伪问题。尽管构造性公理确实解决了CH,但它通常不被认为是直观上为真的,就像CH通常不被认为是错误的一样。\(^\text{[16]}\)

至少还有两个其他的公理被提出,这些公理对连续统假设(CH)有影响,尽管这些公理目前在数学界尚未得到广泛接受。1986年,克里斯·弗赖林\(^\text{[17]}\)提出了一个反对CH的论点,证明了CH的否定等同于弗赖林的对称公理,这是一个基于特定的概率直觉推导出的命题。弗赖林认为这个公理是“直观上清晰的”\(^\text{[17]}\),但其他人对此持不同意见。\(^\text{[18][19]}\)

由W·休·伍丁提出的一个反对CH的困难论点,自2000年以来引起了相当大的关注。\(^\text{[11][12]}\)福尔曼并没有完全否定伍丁的论点,但他强调了谨慎的重要性。\(^\text{[20]}\)伍丁提出了一个新的假设,他将其标记为“(*)-公理”或“星公理”。星公理将意味着\( 2^{\aleph_0} = \aleph_2 \),从而证伪CH。星公理得到了2021年5月独立证明的支持,证明显示星公理可以从马丁最大化的变体中推导出来。然而,伍丁在2010年代表示,他现在认为CH为真,这基于他对自己新的“终极L”猜想的信念。\(^\text{[21][22]}\)

所罗门·费费尔曼(Solomon Feferman)认为,连续统假设(CH)不是一个明确的数学问题。\(^\text{[23]}\)他提出了一个“明确性”理论,使用ZF的半直觉主义子系统,该子系统对有界量词接受经典逻辑,但对无界量词使用直觉主义逻辑,并且建议,如果半直觉主义理论能够证明\( (\phi \lor \neg \phi) \),则命题\( \phi \)是数学上“明确的”。他猜测根据这一概念,CH并不是明确的,并提出因此应该认为CH没有真值。彼得·科尔纳(Peter Koellner)对费费尔曼的文章写了批评性的评论。\(^\text{[24]}\)

乔尔·大卫·哈姆金斯(Joel David Hamkins)提出了一种集合论的多重宇宙方法,并认为“在多重宇宙视角下,连续统假设已经得到解决,因为我们对它在多重宇宙中的行为有了广泛的了解,因此它不再能够以以前期望的方式得到解决”。\(^\text{[25]}\)在一个相关的观点中,萨哈隆·谢拉(Saharon Shelah)写道,他“不赞同纯粹柏拉图主义的观点,即集合论中的有趣问题可以被决定,我们只需要发现额外的公理。我的思维图景是我们有许多可能的集合论,它们都符合ZFC”。\(^\text{[26]}\)
\subsection{广义连续统假设}  
广义连续统假设(GCH)声明,如果一个无限集合的基数介于一个无限集合\( S \)的基数和\( S \)的幂集\( \mathcal{P}(S) \)的基数之间,则它的基数要么与\( S \)相同,要么与\( \mathcal{P}(S) \) 相同。也就是说,对于任何无限基数\( \lambda \),不存在一个基数\( \kappa \)使得 \( \lambda < \kappa < 2^\lambda \)。GCH 等价于:
\[
\aleph_{\alpha + 1} = 2^{\aleph_{\alpha}}~
\]
对于每个序数\( \alpha \)\(^\text{[6]}\)  

(有时称为康托尔的阿列夫假设)。

贝斯数提供了该条件的另一种表示法:\(\aleph_{\alpha} = \beth_{\alpha}\)对于每个序数\( \alpha \)。连续统假设是\( \alpha = 1 \)的特例。GCH最早由菲利普·乔丹提出。\(^\text{[27]}\)有关GCH的早期历史,请参见Moore。\(^\text{[28]}\) 

像选择公理(CH)一样,广义连续统假设(GCH)也与ZFC独立,但谢尔平斯基证明了ZF + GCH蕴含选择公理(AC)(因此也蕴含决定性公理的否定,AD),因此在ZF中,选择公理和GCH不是独立的;没有ZF模型同时满足GCH成立且选择公理失败的情况。为了证明这一点,谢尔平斯基显示GCH意味着每个基数\( n \)都小于某个阿列夫数,从而可以排序。这通过证明\( n \)小于\( 2^{\aleph_0 + n} \)来完成,而\( 2^{\aleph_0 + n} \)小于它自己的哈托格数——这使用了等式\( 2^{\aleph_0 + n} = 2 \cdot 2^{\aleph_0 + n} \);完整的证明请参见吉尔曼。\(^\text{[29]}\)

库尔特·哥德尔证明了GCH是ZF + V=L(即每个集合相对于序数是可构造的公理)的结果,因此与ZFC一致。由于GCH蕴含CH,科恩的模型中CH失败的情况也是GCH失败的模型,因此GCH不能从ZFC中证明。W.B. 伊斯顿利用科恩发展出的强迫法证明了伊斯顿定理,表明对于任意大的基数\( \aleph_\alpha \),它不满足\( 2^{\aleph_\alpha} = \aleph_{\alpha + 1} \) 是与ZFC一致的。更晚些时候,福尔曼和伍丁证明了(假设非常大的基数一致性)对于每个无限基数\( \kappa \),\( 2^\kappa > \kappa^+ \) 与ZFC一致。后来,伍丁通过证明对于每个\( \kappa \),\( 2^\kappa = \kappa^{++} \) 的一致性,扩展了这一结果。卡尔米·梅里莫维奇\(^\text{[30]}\)表示,对于每个\( n \geq 1 \),如果假设某些大基数公理的一致性,那么对于每个无限基数\( \kappa \),\( 2^\kappa \)是\( \kappa \)的第 \( n \) 个后继是与ZFC一致的。另一方面,拉兹洛·帕泰\(^\text{[31]}\)证明了如果\( \gamma \)是一个序数,并且对于每个无限基数\( \kappa \), \( 2^\kappa \)是\( \kappa \)的第\( \gamma \)个后继,那么\( \gamma \)是有限的。

对于任何无限集合\( A \)和\( B \),如果从\( A \)到\( B \)存在一个单射,那么从\( A \)的子集到\( B \)的子集也存在一个单射。因此,对于任何无限基数\( A \)和\( B \),有\( A < B \to 2^A \leq 2^B \)。如果\( A \)和\( B \)是有限的,则更强的不等式\( A < B  \to 2^A < 2^B \)成立。GCH意味着对于无限基数和有限基数,这个严格的、更强的不等式也成立。
\subsubsection{广义连续统假设对基数指数运算的影响} 
虽然广义连续统假设直接仅涉及基数指数运算,其中基数为2,但可以从中推导出所有情况下的基数指数运算值\( \aleph_{\alpha}^{\aleph_{\beta}} \)。GCH意味着,对于序数\( \alpha \)和\( \beta \):\(^\text{[32]}\)
\begin{itemize}
\item 当 \( \alpha \leq \beta + 1 \) 时,\( \aleph_{\alpha}^{\aleph_{\beta}} = \aleph_{\beta + 1} \);
\item 当 \( \beta + 1 < \alpha \) 且 \( \aleph_{\beta} < \operatorname{cf}(\aleph_{\alpha}) \) 时,\( \aleph_{\alpha}^{\aleph_{\beta}} = \aleph_{\alpha} \),其中 \( \operatorname{cf} \) 是末尾基数操作;
\item 当 \( \beta + 1 < \alpha \) 且 \( \aleph_{\beta} \geq \operatorname{cf}(\aleph_{\alpha}) \) 时,\( \aleph_{\alpha}^{\aleph_{\beta}} = \aleph_{\alpha + 1} \)。
\end{itemize}
第一个等式(当\( \alpha \leq \beta + 1 \)时)来自于:
\[
\aleph_{\alpha}^{\aleph_{\beta}} \leq \aleph_{\beta + 1}^{\aleph_{\beta}} = (2^{\aleph_{\beta}})^{\aleph_{\beta}} = 2^{\aleph_{\beta} \cdot \aleph_{\beta}} = 2^{\aleph_{\beta}} = \aleph_{\beta + 1}~
\]
而:
\[
\aleph_{\beta + 1} = 2^{\aleph_{\beta}} \leq \aleph_{\alpha}^{\aleph_{\beta}}.~
\]
第三个等式(当\( \beta + 1 < \alpha \)且\( \aleph_{\beta} \geq \operatorname{cf}(\aleph_{\alpha}) \)时)来自于:
\[
\aleph_{\alpha}^{\aleph_{\beta}} \geq \aleph_{\alpha}^{\operatorname{cf}(\aleph_{\alpha})} > \aleph_{\alpha}~
\]
由Kőnig定理得出,而:
\[
\aleph_{\alpha}^{\aleph_{\beta}} \leq \aleph_{\alpha}^{\aleph_{\alpha}} \leq (2^{\aleph_{\alpha}})^{\aleph_{\alpha}} = 2^{\aleph_{\alpha} \cdot \aleph_{\alpha}} = 2^{\aleph_{\alpha}} = \aleph_{\alpha + 1}~
\]
\subsection{另见}  
\begin{itemize}
\item 绝对无限  
\item 贝斯数  
\item 基数  
\item \( \Omega \)-逻辑  
\item 第二连续统假设  
\item 韦策尔问题
\end{itemize}
\subsection{参考文献}
\begin{enumerate}
\item 康托尔,格奥尔格(1878)。《集合论研究的一个贡献》(德文)。《纯粹与应用数学杂志》(Journal für die Reine und Angewandte Mathematik)。1878年卷84期,第242–258页。doi:10.1515/crll.1878.84.242(自2024年11月1日失效)。
\item 哥德尔,库尔特(1940)。《连续统假设的一致性》(The Consistency of the Continuum-Hypothesis)。普林斯顿大学出版社。
\item 道本,约瑟夫·沃伦(1990)。《格奥尔格·康托尔:他的数学与无限哲学》(Georg Cantor: His mathematics and philosophy of the infinite)。普林斯顿大学出版社,第134–137页。ISBN 9780691024479。
\item 科恩,保罗·J·(1963年12月15日)。《连续统假设的独立性,[第一部分]》。《美国国家科学院院刊》(Proceedings of the National Academy of Sciences of the United States of America)。50卷6期,第1143–1148页。Bibcode:1963PNAS...50.1143C。doi:10.1073/pnas.50.6.1143。JSTOR 71858。PMC 221287。PMID 16578557。
\item 关于证明概要,参见《有理数的可数性》(Countability of the Rationals)。
\item 戈德雷,德里克(1996)。《经典集合论》(Classic Set Theory)。查普曼与霍尔出版社(Chapman & Hall)。
\item 卡拉吉拉,阿萨夫(数学栈交换用户,个人主页:https://math.stackexchange.com/users/622/asaf-karagila)。《如何在不使用选择公理的情况下表述连续统假设?》(*How to formulate continuum hypothesis without the axiom of choice?*),链接(版本:2017-04-13):https://math.stackexchange.com/q/404813
\item 哥德尔,库尔特(1938)。《选择公理与广义连续统假设的一致性》(The Consistency of the Axiom of Choice and of the Generalized Continuum-Hypothesis)。《美国国家科学院院刊》(Proceedings of the National Academy of Sciences)24卷12期,第556–557页。Bibcode:1938PNAS...24..556G。doi:10.1073/pnas.24.12.556。PMC 1077160。PMID 16577857。
\item 科恩,保罗·J·(1964年1月15日)。《连续统假设的独立性,[第二部分]》。《美国国家科学院院刊》(Proceedings of the National Academy of Sciences of the United States of America)51卷1期,第105–110页。Bibcode:1964PNAS...51..105C。doi:10.1073/pnas.51.1.105。JSTOR 72252。PMC 300611。PMID 16591132。
\item 费弗曼,所罗门(1999年2月)。《数学需要新公理吗?》(Does mathematics need new axioms?)。《美国数学月刊》(American Mathematical Monthly)106卷2期,第99–111页。CiteSeerX 10.1.1.37.295。doi:10.2307/2589047。JSTOR 2589047。
\item 伍丁,W·休(2001)。《连续统假设,第一部分》(PDF)。《美国数学会通讯》(Notices of the AMS)。48卷6期,第567–576页。原始PDF存档于2022年10月10日。
\item 伍丁,W·休(2001)。《连续统假设,第二部分》(PDF)。《美国数学会通讯》(Notices of the AMS)。48卷7期,第681–690页。原始PDF存档于2022年10月10日。
\item 克尔纳,彼得(2011)。《连续统假设》(PDF)。载于《探索独立性的前沿》(Exploring the Frontiers of Independence)。哈佛大学讲座系列。原始PDF存档于2012年1月24日。
\item 古德曼,尼古拉斯·D·(1979)。《数学作为一门客观科学》。《美国数学月刊》(The American Mathematical Monthly)。86卷7期,第540–551页。doi:10.2307/2320581。JSTOR 2320581。MR 0542765。(注:此观点常被称为“形式主义”,类似立场可见于哈斯凯尔·柯里 [5]、亚伯拉罕·鲁宾逊 [17] 及保罗·科恩 [4] 的著作。)
\item 马迪 1988,第500页。
\item 库能,肯尼斯(1980)。《集合论:独立性证明导论》(Set Theory: An Introduction to Independence Proofs)。荷兰阿姆斯特丹:北荷兰出版社。第171页。ISBN 978-0-444-85401-8。
\item 弗雷林,克里斯(1986)。《对称性公理:向实数轴投掷飞镖》(Axioms of Symmetry: Throwing darts at the real number line)。《符号逻辑杂志》(Journal of Symbolic Logic)51卷1期(符号逻辑协会),第190–200页。doi:10.2307/2273955。JSTOR 2273955。S2CID 38174418。
\item 巴格米尔,F·(1989–1990)。《对弗雷林反对连续统假设的论证投掷一镖》(Throwing a dart at Freiling's argument against the continuum hypothesis)。《实分析交流》(Real Analysis Exchange)15卷1期,第342–345页。doi:10.2307/44152014。JSTOR 44152014。MR 1042552。
\item 哈姆金斯,乔尔·戴维(2015年1月)。《连续统假设的“理想解”能否实现?》(Is the Dream Solution of the Continuum Hypothesis Attainable?)。《圣母大学形式逻辑杂志》(Notre Dame Journal of Formal Logic)56卷1期。arXiv:1203.4026。doi:10.1215/00294527-2835047。
\item 福尔曼,马特(2003)。《连续统假设解决了吗?》(Has the Continuum Hypothesis been settled?)(PDF)。原始PDF存档于2022年10月10日。检索于2006年2月25日。
\item 沃尔乔弗,纳塔莉(2021年7月15日)。《存在多少数字?关于无穷的证明推动数学接近答案》(How Many Numbers Exist? Infinity Proof Moves Math Closer to an Answer)。《量子杂志》(Quanta Magazine)。检索于2021年12月30日。
\end{enumerate}