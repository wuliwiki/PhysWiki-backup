% 数学归纳法(高中)
% keys 数学归纳法|递推|归纳法
% license Usr
% type Tutor

\begin{issues}
\issueDraft
\end{issues}

\pentry{数列\nref{nod_HsSeFu}}{nod_aa25}


在数列部分,研究的是从自然数到实数的映射。尤其是数列的通项公式,似乎像一个准则,只要给出一个n就能得到一个结果。如果将数列的通项公式看作一个命题,(解释一下怎么看)。那么只有在n与对应的$a_n$时,才能得到结果是对的。这个判断过程可以抽象出来成为一种证明命题的方式。数学归纳法是一种证明数学命题的重要方法,主要用于证明某些关于正整数的命题。

\subsection{从多米诺骨牌开始}

在介绍数学归纳法之前,先来看看一种常见的常见连锁效应——多米诺骨牌。(形容一下多米诺骨牌搭建和推倒的过程。)。设想一下,该如何用最简单的方法来描述这个过程?

一个直接的思路是,搭建的人员在搭建时,要保证每一张骨牌倒下时,都会带动下一个骨牌倒下。然后,在推倒时,保证自己能推倒第一块。这样,第一块倒下就会带动第二块,第二块倒下就会带动第三块,一直这样倒下去,那么所有的骨牌都会倒下。

这种“传递性”的现象,就是数学归纳法背后的原理。

\begin{definition}{数学归纳法}
对某个与自然数相关的命题$P(n)$,利用\textbf{数学归纳法}证明$P(n)$成立的过程分为三步:
\begin{enumerate}
\item 验证:检查命题对某个特定起始点(通常是最小值)是否成立。
\item 假设:假设命题在某个自然数 $k$ 成立,即假设 $n = k$ 时命题成立。
\item 推导:证明命题在归纳假设的前提下,可以推导出$n = k+1$ 时也成立。
\end{enumerate}
如果以上三步都完成,就可以得出结论:该命题对所有自然数$n$ 都成立。
\end{definition}

可以看到,验证对应的就是第一块能推倒,假设就是指某张骨牌倒下,而推导得出的结论就是下一张骨牌也会倒下。

尽管通常称为数学归纳法,但本身只是叫做归纳法,“数学”是与其他领域分隔开。虽然数学归纳法名字中有“归纳”,但是数学归纳法并非逻辑上不严谨的\aref{归纳推理法}{sub_HsLogi_1},它属于完全严谨的演绎推理法。数学归纳法是一种公理模式存在,如果满足则判断正确。也就是说它本身是不可以证明的。其实它是定义自然数的公理之一,也就是说,只要有自然数存在的场合,就天然存在数学归纳法。

效果:
\begin{itemize}
\item 保证自然数的完备性:确保性质对所有自然数都成立,不必逐一检查每一个自然数。
\item 建立递归定义的基础:提供严格的证明这些递归定义正确性的方法。
\end{itemize}


示例
\begin{example}{证明:对于任意正整数 $n$,数列 $a_n=n$ 的数列和为 $S_n = \frac{n(n+1)}{2}$。}
证明:

基础步骤:当 $n = 1$ 时,$S_1 = 1$。公式 $\frac{1(1+1)}{2} = 1$ 成立。

归纳步骤:
假设对于 $n = k$,命题成立,即:
\begin{equation}
S_k = 1 + 2 + 3 + \cdots + k = \frac{k(k+1)}{2}~.
\end{equation}
需要证明 $n = k+1$ 时命题也成立:
\begin{equation}
S_{k+1} = S_k + (k+1)~.
\end{equation}

根据归纳假设,将 $S_k$ 代入:
\begin{equation}
S_{k+1} = \frac{k(k+1)}{2} + (k+1)~.
\end{equation}

提取公因式 $(k+1)$,化简得:
\begin{equation}
S_{k+1} = \frac{(k+1)(k+2)}{2}~.
\end{equation}

这与命题的形式一致,因此归纳步骤成立。

综上,利用数学归纳法,可以证明命题对所有正整数 $n$ 都成立。
\end{example}

数学归纳法是一种逻辑性强、步骤清晰的方法,不仅在数列中有广泛应用,还可以用于证明多项式、几何问题等。

数学归纳法把复杂的证明过程转化为了对最终结果的猜想。由此,也使得在高中阶段绕过复杂困难的证明,直接使用其它方法得到结果并使用数学归纳法给出证明成为可能。