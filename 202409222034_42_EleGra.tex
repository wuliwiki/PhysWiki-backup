% 电磁力和引力
% keys 电磁力|引力|作用量
% license Usr
% type Tutor

\pentry{光与物质粒子的统一(相对论点粒子的作用量)\nref{nod_RAct},相对论补全\nref{nod_Comple}}{nod_e590}
本节将以一种“统一”的角度给出电磁力和引力的作用量。
\subsection{从自由粒子到势阱中的粒子}
在光与物质粒子的统一(相对论点粒子的作用量)\upref{RAct}一节,我们得到了自由粒子的作用量,其具有下面的形式
\begin{equation}\label{eq_EleGra_1}
S=-m\int\sqrt{-\eta_{\mu\nu}\dd x^\mu\dd x^\nu}=-m\int\sqrt{\dd t^2-\dd{\vec x}^2}~.
\end{equation}
现在考虑粒子处于势 $V(x)$ 中。尽管非相对论情形(Newton力学)时处于势为 $V(x)$ 的粒子作用量为
\begin{equation}
S_{NR}=\int\dd t\qty(\frac{1}{2}m\qty(\dv{\vec x}{t})^2-V(x)),~
\end{equation}
但是我们并不能理所当然的将 $V(x)$ 加入\autoref{eq_EleGra_1} 中得到相对论情形处于势阱 $V(x)$ 中的粒子。换言之,我们不知道如何将 $V(x)$ 放入\autoref{eq_EleGra_1} 中。

尽管如此,然而可以肯定的是,将 $V(x)$ 放入\autoref{eq_EleGra_1} 中,只有两种可能:a.根号外面,b.根号里面。因此,我们有如下的两种选择:
\begin{enumerate}
\item E:
\begin{equation}
S=-\int{m\sqrt{-\eta_{\mu\nu}\dd x^\mu\dd x^\nu}+V(x)\dd t}.~
\end{equation}
\item G:
\begin{equation}
S=-m\int\sqrt{\qty(1+\frac{2V}{m})\dd t^2-\dd{\vec x}^2}~.
\end{equation}
\end{enumerate}
选项G来源于非相对论极限:首先,$\abs{\dd{\vec x}}\ll \dd t$,所以
\begin{equation}\label{eq_EleGra_2}
S\approx-m\int \qty{\sqrt{\qty(1+\frac{2V}{m})}\dd t-\frac{\dd{\vec{x}}}{2\sqrt{1+\frac{2V}{m}}\dd t}}~.
\end{equation}
其次,令 $V\ll m$(即 $V\ll mc^2$),则
\begin{equation}
\sqrt{1+\frac{2V}{m}}\approx1+\frac{V}{m}.~
\end{equation}
由于在\autoref{eq_EleGra_2} 中,第二项已经远小于第一项了,因此第二项不需要再保留到 $\frac{V}{m}$ 的修正项,因此
\begin{equation}
\begin{aligned}
S\approx&-m\int \qty{(1+\frac{V}{m})\dd t-\frac{\dd{\vec x}^2}{2\dd t}}\\
=&\int \dd t\qty{\frac{1}{2}m\qty(\dv{\vec x}{t})^2-V-m}~.
\end{aligned}
\end{equation}
上式表明选项 $G$ 的作用量在适宜的极限下取Newton作用量的形式,除了多出一个常数 $-m$,而这一项我们已经知道代表着什么(见\autoref{sub_RAct_1})。

然而,无论是选项 E 还是 G,添加在作用量中的项都不是Lorentz不变的。因此,为了保持理论的Lorentz不变性(见相对论补全\upref{Comple}开头的说明),我们必须要进行修正。

\subsection{相对论补全}
无论是 E 还是 G ,若我们认为 $V$ 是通过外界固定和施加的,那么作用量无论如何都不能是Lorentz不变的。因此,为了保持Lorentz不变性,必须改变 $V(x)$ 的形式。改变的关键便是相对论补全\upref{Comple}。

首先看 E,注意 $V(x)$ 是和 $\dd t$ 结合的,因此可将 $V(x)$ 视为Lorentz矢量场 $A_\mu(x)$ 的时间分量 $A_0(x)$,而 $V(x)\dd t$ 仅是 $A_\mu(x)\dd x^\mu=A_0(x)\dd t+A_i(x)\dd x^i$ 的第一项。因此,我们只需引入一个矢量场 $A_\mu(x)$,从而得到如下作用量
\begin{equation}
S=\int\qty{-m\sqrt{-\eta_{\mu\nu}\dd x^\mu\dd x^\nu}+A_\mu(x)\dd x^\mu}~.
\end{equation}







