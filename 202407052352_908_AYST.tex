% 阿尔伯特·爱因斯坦
% license CCBYSA3
% type Wiki

(本文根据 CC-BY-SA 协议转载自原搜狗科学百科对英文维基百科的翻译)

阿尔伯特·爱因斯坦(/ˈaɪnstaɪn/\textbf{EYEN}-styne;[1] 德语:[ˈalbɛɐ̯t ˈʔaɪnʃtaɪn]( 发音);1879年3月14日 ——1955年4月18日)是出生于德国的理论物理学家,[2]他发展了现代物理学的两大支柱之一(与量子力学一起)——相对论。[3][4]他的工作也因其对科学哲学的影响而闻名。[5][6]他最为人所知的质能等价公式 E = mc2被称为“世界上最著名的方程式”。[7]他获得了1921年诺贝尔物理学奖,“基于他对理论物理学的贡献,尤其是因为他发现了光电效应定律”,[8]这是量子理论发展的关键一步。

爱因斯坦在职业生涯初期认为牛顿力学已经不足以调和经典力学和电磁场的定律,这让他在伯尔尼瑞士专利局工作期间(1902-1909年)发展了狭义相对论。同时,他意识到相对论原理也可以扩展到引力场,于是在1916年发表了一篇关于广义相对论的论文,其中包含了他的引力理论。他继续处理统计力学和量子理论的问题,这使粒子理论和分子运动能够被解释。他还研究了光的热性质,这为光的光子理论奠定了基础。1917年,他应用广义相对论来模拟宇宙的结构。[9][10]

除了在布拉格的一年之外,爱因斯坦在1895年至1914年间都生活在瑞士。在此期间,于1896年放弃了德国国籍,于1900年在苏黎世获得了瑞士联邦理工学院(后来的瑞士联邦技术学院)的学术文凭。在无国籍五年多之后,他、于1901年获得了瑞士公民身份,并将其保留终身。1905年,被苏黎世大学授予博士学位。同年,在著名的奇迹年发表了四篇开创性的论文并引起了学术界的注意,此时他才26岁。爱因斯坦于1912年至1914年间在苏黎士教理论物理,然后前往柏林,在那里他被选为普鲁士科学院院士。

1933年,当爱因斯坦访问美国时,阿道夫·希特勒上台执政。由于爱因斯坦的犹太背景,他没有返回德国,[11]在美国定居,并于1940年成为美国公民。[12]在第二次世界大战前夕,他签署了一封致富兰克林·罗斯福总统的信,提醒他注意“新型超强炸弹”的潜在发展,并建议美国开始类似的研究,这最终导致了曼哈顿计划的诞生。爱因斯坦支持同盟国,但他谴责将核裂变作为武器的想法。他与英国哲学家伯特兰·罗素签署了《罗素—爱因斯坦宣言条约》,强调了核武器的危险性。他隶属于新泽西州的普林斯顿高等研究院,直到1955年去世。

爱因斯坦发表了300多篇科学论文和150多篇非科学著作。[9][13]他的智力成就和独创性使“爱因斯坦”这个词成为“天才”的同义词。[14]尤金·维格纳将爱因斯坦与他同时代的人相比写道,“爱因斯坦的理解甚至比扬西·冯·诺伊曼更深刻。他的思想比冯·诺伊曼的更具穿透力和独创性。这是一个非常了不起的声明。”[15]

\subsection{生活和事业}
\subsubsection{1.1 早期生活和教育}
\begin{figure}[ht]
\centering
\includegraphics[width=6cm]{./figures/28110856e7f46d2c.png}
\caption{1882年,爱因斯坦三岁时} \label{fig_AYST_1}
\end{figure}
阿尔伯特·爱因斯坦于1879年3月14日出生在德意志帝国符腾堡王国的乌尔姆。[2]他的父亲是赫尔曼·爱因斯坦,母亲是鲍林·科赫,分别是一名工程师和推销员。1880年,爱因斯坦一家人搬到了慕尼黑,他的父亲和叔叔雅各布在那里创立了这家基于直流电制造电气设备的公司。[2]

爱因斯坦一家是不遵守犹太教义的犹太人,他从5岁开始在慕尼黑的一所天主教小学就读了三年。8岁时,他被转到了路易斯波特体育馆(现称阿尔伯特·爱因斯坦体育馆),在那里接受了高级小学和中学教育,直到7年后离开德意志帝国。[16]
\begin{figure}[ht]
\centering
\includegraphics[width=6cm]{./figures/92a19247301e3ac0.png}
\caption{1893年,爱因斯坦14岁时} \label{fig_AYST_2}
\end{figure}
1894年,由于赫尔曼和雅各布缺乏资金将他们的设备从直流(DC)标准转换为更高效的交流(AC)标准,他们的公司失去了向慕尼黑提供电照明的投标。[17]损失迫使他们将慕尼黑工厂出售。为了寻找生意,爱因斯坦一家搬到了意大利,先是去了米兰,几个月后去了帕维亚。当全家搬到帕维亚时,15岁的爱因斯坦留在慕尼黑,在路易斯波特体育馆完成学业。他的父亲打算让他从事电气工程,但爱因斯坦与当局发生了冲突,并对学校的制度和教学方法表示不满。他后来写道,学习和创造性思维的精神在严格的死记硬背学习中丧失了。1894年12月底,用医生的证明说服学校让他离开后,爱因斯坦去意大利和他在帕维亚的家人团聚。[18]在意大利期间,他写了一篇题为“关于磁场中以太状态的研究”的短论文。[19][20]

爱因斯坦从小就擅长数学和物理,比同龄人早几年达到数学水平。十二岁的爱因斯坦在一个夏天自学了代数和欧几里得几何。同时爱因斯坦在12岁时也独立地发现了毕达哥拉斯定理的原始证明。[21]在给了12岁的爱因斯坦一本几何教科书后,家庭教师马克斯·塔木德说:“爱因斯坦在短时间内就完成了整本书。随后他致力于高等数学...很快,他的数学天赋就飞得如此之高,我都跟不上了。”[22]他对几何和代数的热情使这个12岁的孩子相信自然可以被理解为“数学结构”。[22]爱因斯坦12岁开始自学微积分,14岁时他说自己“掌握了积分和微分”。[23]

13岁时,爱因斯坦被介绍给康德学习纯粹理性批判,康德成为了他最喜欢的哲学家,他的导师说:“当时他还是个孩子,只有十三岁,但是康德的作品,常人无法理解,对他来说似乎很清楚。”[22]
\begin{figure}[ht]
\centering
\includegraphics[width=6cm]{./figures/b0289fde805edc8f.png}
\caption{爱因斯坦17岁时获得的入学证书,显示了他在阿戈维亚州学校的毕业成绩(Aargauische Kantonsschule,分数为1-6分,其中6分是最高分数)。他的分数:德语5分;法语3分;意大利语5分;历史6分;地理4分;代数6分;几何6分;画法几何6分;物理6分;化学5分;自然历史5分;艺术与技术制图4分} \label{fig_AYST_3}
\end{figure}
1895年,16岁的爱因斯坦参加了苏黎世瑞士联邦理工学院(后拉来的瑞士联邦科技学院)的入学考试。他在考试的一般课程上没有达到要求的标准,[24]但是在物理和数学上取得了优异的成绩。[25]根据理工学院校长的建议,他于1895年和1896年在瑞士阿劳的阿尔戈维亚州学校(体育馆)完成了中学教育。在寄宿于约斯特·温特尔教授的家庭时,他爱上了温特尔的女儿玛丽。爱因斯坦的姐姐玛嘉后来嫁给了温特尔的儿子保罗。[26]1896年1月,在他父亲的批准下,爱因斯坦放弃了他在德国符腾堡王国的国籍,以逃避服兵役。[27]1896年9月,他以优异的成绩通过了瑞士马图拉考试,包括物理和数学科目的6级最高分,分数范围为1-6。[28]17岁时,他报名参加了苏黎世理工学院为期四年的数学和物理教学文凭课程。玛丽·温特尔比他大一岁,她搬到了瑞士的奥尔森堡担任教师。

爱因斯坦未来的妻子,一位20岁的塞尔维亚女性米列娃·马利奇,也在那一年就读于理工学院。她是数学和物理教学文凭课程部分六名学生中唯一的女性。在接下来的几年里,爱因斯坦和马利奇的友谊发展成了爱情,他们一起阅读课外物理书籍,爱因斯坦对此越来越感兴趣。1900年,爱因斯坦通过了数学和物理考试,并被授予联邦理工学院教学文凭。[29]有人声称马利奇和爱因斯坦在他1905年的论文上合作过,[30][31]这篇论文被称为奇迹年 论文,但是研究过这个问题的物理历史学家没有发现任何证据表明马利奇做出了任何实质性的贡献。[32][33][34][35]

\subsubsection{1.2 婚姻和儿童}
\begin{figure}[ht]
\centering
\includegraphics[width=6cm]{./figures/c19fe20fe3b3fddf.png}
\caption{请添加图片标题} \label{fig_AYST_4}
\end{figure}
