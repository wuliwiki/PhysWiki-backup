% 约束及其分类
% 约束|完整约束|非完整约束|约束方程|定常约束

\pentry{矢量力学}

牛顿力学(或称矢量力学),原则上已经可以处理一切经典力学的问题,只需列出所有质点的受力之后进行求解.但是这种方法也有不足之处,比如在面对复杂约束时方程将变得难以求解,矢量的特性更给求解带来复杂度.为此,拉格朗日在前人的基础上提出了\textbf{分析力学(Analytical Mechanics)}.

在分析力学中,我们使用标量(拉格朗日量)来描述一个系统,系统的演化由拉格朗日方程来决定.为了进一步介绍有关内容,我们需要先叙述系统受到的约束及其分类.

\textbf{约束(Restrict)},顾名思义就是对系统中的每个质点的坐标及其速度所设的约束条件.一般我们认为对加速度不会有约束.描述约束条件的方程称为约束方程.

例一:铰接在地面上的轻杆

在上例中,我们发现约束方程可写成f(x,y,z)=0的形式.一般地,假设一个系统中共有N个质点,其笛卡尔坐标依次记为$x
_1$,$x_2$,…$x_3N$,如果约束方程可写成f($x_1$,$x_2$,…$x_3N$,$\dot x_1$,$\dot x_2$,…$\dot x_3N$,t)=0的形式,则我们称这个约束是\textbf{完整}的.