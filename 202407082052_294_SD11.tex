% 苏州大学 2011 年硕士物理考试试题
% keys 苏州大学|考研|物理|2011年
% license Copy
% type Tutor
\textbf{科目代码:832}
\begin{enumerate}
\item 两个长方形物体$A$和$B$紧靠放在光滑的水平桌面上,已知$m_A=2kg,m_B=3kg$,有一质量$ m=100g$的子弹以速率 $v_0=800m/s$水平射入长方体 $A$,经$t=0.01s$,又射入长方体$B$,最后停留在长方体$B$内未射出。设子弹射入$A$时所受的摩擦力$F_r=3x10^3N$,求:\\
(1)子弹在射入$A$的过程中,$B $受到$A$的作用力的大小;\\
(2)当子弹留在$ B$ 中时,$A$ 和$B$的速度大小。
\item 一长为$L$,质量为$m$的均质细棒,一端可绕固定的水平光滑轴在0竖直平面内转动,在0点还系有一长为b(6<L)的细绳,绳的另一端悬挂一质量也为m的小球。当小球悬线偏离竖直方向某一角度时,由静止释放。已知小球与细棒发生完全弹性碰撞,要使碰撞后小球刚好停止,问绳的长度b应为多少?
\end{enumerate}
