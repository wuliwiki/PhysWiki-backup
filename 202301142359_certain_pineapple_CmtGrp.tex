% 换位子群
% 换位子|commutator|导子群|交换子|交换子群|群论|正规子群

\pentry{正规子群和商群\upref{NormSG}}


\begin{definition}{换位子}\label{CmtGrp_def1}
给定群 $G$ 中的任意元素 $a$ 和 $b$,则称元素 $b^{-1}a^{-1}ba$ 为一个\textbf{换位子(commutator)},或者\textbf{导子}。
\end{definition}

给\autoref{CmtGrp_def1} 中的概念取换位子这一名称,是因为它作用在两个元素的乘积上可以交换其乘积顺序:$(ab)(b^{-1}a^{-1}ba)=ba$。它可以用来更细致地刻画群的交换性,而不仅仅是粗糙的“交换 / 不交换”的描述。具体来说,我们需要用\textbf{换位子群}来描述群的交换性:

\begin{definition}{换位子群}
给定群 $G$。$G$ 的全体换位子所\textbf{生成}的子群,称为 $G$ 的\textbf{换位子群(commutator group)},记为 $G'$ 或 $[G, G]$。
\end{definition}

注意,全体换位子构成的集合,通常不是一个群,因此换位子群并不仅仅是换位子构成的,而是换位子以及换位子的乘积构成的。比如说,形如 $b^{-1}a^{-1}bad^{-1}c^{-1}dc$ 的元素可能不是一个换位子,但它是换位子群中的元素。

\begin{theorem}{}
给定群 $G$,则 $G'\triangleleft G$。
\end{theorem}

\textbf{证明}:

换位子群中的元素都可以写成换位子相乘的形式:$x_1x_2x_3\cdots x_n$,其中各 $x_i$ 为换位子。因此,要证明 $g^{-1}G'g=G'$,只需要证明 $g^{-1}x_1g$ 是一个换位子即可,这样 $g^{-1}x_1x_2x_3\cdots x_ng=g^{-1}x_1gg^{-1}x_2gg^{-1}x_3gg^{-1}\cdots gg^{-1}x_ng$ 也是换位子的乘积,故在换位子群中。

对于任意 $g\in G$,都有
\begin{equation}
\begin{aligned}
g^{-1}b^{-1}a^{-1}bag&=g^{-1}b^{-1}gg^{-1}a^{-1}gg^{-1}bgg^{-1}ag\\
&=(g^{-1}bg)^{-1}(g^{-1}ag)^{-1}(g^{-1}bg)(g^{-1}ag)
\end{aligned}
\end{equation}
也是一个换位子。

\textbf{证毕}。

换位子群刻画交换性的方式由以下\autoref{CmtGrp_the1} 描述:

\begin{theorem}{}\label{CmtGrp_the1}
设 $N$ 是群 $G$ 的正规子群,则商群 $G/N$ 交换当且仅当 $G'\subseteq N$。
\end{theorem}

\textbf{证明}:

任取 $x, y\in G$,则 $G'\subseteq N$ 当且仅当 $(xy)^{-1}yx\in N$。

而 $(xy)^{-1}yx\in N$ 等价于 $(xy)^{-1}yxN=N$,这又等价于 $yxN=xyN$,也就是商群运算的交换性。

% 先设 $G/N$ 交换。于是对于任意 $x, y\in G$,都有 $xyN=yxN$,即 $(xy)^{-1}yx\in N$,故任意换位子都在 $N$ 中,故换位子群必在 $N$ 中。

% 再设 $G'\subseteq N$,那么

\textbf{证毕}。

\autoref{CmtGrp_the1} 表明,一个群 $G$ 如果不交换,我们也可能找到一个正规子群使其商群为交换群,好比是抹去了一些细节之后陪集的运算体现出交换性。当然,\textbf{必须}抹去的细节越多,我们就可以说 $G$ 的交换性越差;没有抹去的必要,那 $G$ 就是完全交换的。而这个必须抹去的细节,就是换位子群。换位子群越大,则 $G$ 的交换性越低;换位子群只含单位元,那么 $G$ 本身就是交换的。












