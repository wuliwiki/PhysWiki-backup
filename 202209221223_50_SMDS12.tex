% 上海海事大学 2012 年数据结构
% keys 上海海事大学 2012 年数据结构

\subsection{一.判断题(本题20分, 每小题2分)}

1.线性的数据结构可以顺序存储,也可以链接存储;非线性的数据结构只能链接存储.

2. B+树中所有叶子结点都处在同一层次上,且每个叶子结点中关键字个数都是相等的.

3.单链表从任何一个结点出发,都能访问到所有结点.

4.排序的目的就是要将一组无序的记录序列按从小到大的顺序调整.

5.存储在顺序存储器上的顺序文件不能进行折半查找.

6.网络的最小代价生成树是唯一的.

7.磁带是顺序存取的外存储设备.

8.将一棵树转换成二叉树后,二叉树的根结点一定没有右子树.

9.所有叶子结点都处于同一层的二叉树一定是完全二叉树.

10.在AOE网中,任何一个关键活动提前完成,都将使整个工程提前完成.

\subsection{二、填空题(本题30分,每空2分) .}

1.分析下列程序段,其时间复杂度分别为:_ (1) ._ (2)_
\begin{lstlisting}[language=cpp]
i=1;
while(i<=n*n)
    i=i*2;

int i=0, s=0;
while (s<n*n){
    i++;
    s=s+i;
}
\end{lstlisting}

2.顺序表.栈和队列都是( (3) )结构, 顺序表可以在其( (4) )位置插入和删除元素;对于栈只能在( (5) )插入和删除元素; 对于队列只能在( (6) )插入元素和另一端删除元素.

3.广义表A=(a,b,(c ,d),(e,(f, g)))的长度是( (7) ).深度是( (8) ),取表头和表尾函数分别为head()和ail(),则head (head (tail (tail (A))))=( (9) ),而从表中取出原子项d的运算为( (10) ).

4.有一个二维数组A[1..6][0..7].每个数组元素占用6个存储单元,并且a[3][4]的存储地址为1280,若按行序为主序方式存储,数组元素A[2][3]的存储地址是( (11) );若按列序为主序方式存储,数组元素A[2][3]的存储地址是( (12) ).

5.在堆排序、快速排序和归并排序三种算法中,若仅从存储空间考虑,则应首先选取( (13) )方法:若只从平均情况下排序最快考虑.则应选取( (14) )方法;若只从排序结果的稳定性考虑,则应选取( (15) )方法.

