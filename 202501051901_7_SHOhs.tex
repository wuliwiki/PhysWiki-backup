% 简谐振子(高中)
% keys 弹簧振子|simple harmonic oscillation|单摆
% license Usr
% type Tutor

\begin{issues}
\issueDraft
\end{issues}

进一步了解简谐振子,请参考\enref{简谐振子(经典力学)}{SHO}。


\subsection{概念}


若一个物体受力所产生的加速度总是指向一个位置,且该加速度的大小正比于物体到这个位置的距离,我们就说这个物体结合它所受的力,构成了一个\textbf{简谐振子},加速度指向的位置被称为该简谐振子的\textbf{平衡位置}。

叫这个名字,首先是因为简谐振子会反复在平衡位置附近做周期性运动,而不会无限远离平衡位置;其次是因为这种周期性运动是最简单的形式,从而被冠以“简单而和谐”之名。我们首先看一些简谐振子的例子,再学着分析它们的运动。




\begin{example}{弹簧振子}

将轻质弹簧的一端固定,另一端连接在一个质量为$m$的物体上,物体可以沿着一个方向运动,弹簧也在这个方向上。此时弹簧和物体构成了一个简谐振子,我们可以称之为“弹簧振子”。

当弹簧处于原长时,物体不受弹簧的作用力,此时物体所在位置就是该简谐振子的平衡位置。当物体到平衡位置的距离为$x$时,物体受到大小为$kx$的拉力,方向指向平衡位置,其中$k$是弹簧的劲度系数。于是,物体的加速度总是指向平衡位置,且其大小正比于到平衡位置的距离:$a=\frac{k}{m}x$。

\end{example}


\begin{example}{单摆}

将轻质杆的一端固定,另一端在

\end{example}



















