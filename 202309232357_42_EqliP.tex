% 平衡位置和圈
% keys 平衡位置|圈|闭轨线|常微分方程的解
% license Xiao
% type Tutor

\pentry{自治系统解的特点\upref{AuSy}}
本节介绍自治系统(\autoref{def_ODEPr_2}~\upref{ODEPr})解的分类。概括来说,自治系统的解(或轨线\upref{AuSy})有三种情形:
\begin{enumerate}
\item 定点解,即轨线成为个不依赖自变量 $x$ 的点,称为\textbf{平衡位置};
\item 周期解,即轨线自身相交,称为\textbf{闭轨线}或\textbf{圈(环)};
\item 自身不相交的轨线。
\end{enumerate}
\begin{theorem}{}
设 $y^i=\varphi_i(x)$ 是自治系统的解,其最大存在区间为 $(m_1,m_2)$,并且存在 $m_1<x_1\neq x_2<m_2$,满足
\begin{equation}
\varphi_i(x_1)=\varphi_i(x_2)~.
\end{equation}
则\textbf{要么}:对一切值 $x$ 成立 $\varphi_i(x)=a^i$,其中 $a^i$ 是与 $x$ 无关的常数,这时轨线成为解的定义空间中的点。此时解本身称为\textbf{平衡位置}。\\
\textbf{要么}:存在正数 $T$,使得对任意 $t$ 成立
\begin{equation}
\varphi_i(t+T)=\varphi_i(t)~.
\end{equation}
且当 $\abs{\tau_1-\tau_2}<T$ 时,至少对一个 $i$,成立 $\varphi_i(\tau_1)\neq\varphi_i(\tau_2)$。此时解称为\textbf{周期的},$T$ 称为\textbf{周期}。此时轨线称为\textbf{闭轨线}或\textbf{圈(环)}。

\end{theorem}














