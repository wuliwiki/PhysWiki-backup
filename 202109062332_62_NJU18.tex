% 南大 2018 年考研普通物理
% keys 普通物理|普物|南大

\subsection{力学}
1. 在竖直平面内有一光滑的轨道,轨道左边是光滑的弧线,右边是足够长的水平直线.现有 $\mathrm{A}$、$\mathrm{B}$ 两个质点,质量分别 $m_{A}$ 和 $m_B$.质点 $B$ 静止于水平轨道上.将质点 $A$ 置于弧形轨道上,无初速释放.假定质点 $\mathrm{A}$、$\mathrm{B}$ 间的碰撞是完全弹性的.求 $\mathrm{A}$、$\mathrm{B}$ 至少能发生两次碰撞的条件.

2. 在光滑的水平桌面上开有一个小孔,一条不可伸长的轻绳穿过小孔.绳的两头各系一个小球.置于桌面上的小球以速率 $v_{0}$ 绕小孔作匀速圆周运动.桌面下的小球悬在空中,保持静止.假定桌面下面的小球的质量是桌面上小球的2倍\\
(1)求桌面上绳子的长度 $l_{0}$;\\
(2)若给桌面上小球一个径向的小动量,则桌面下的小球将作上下小振动,求振动周期.