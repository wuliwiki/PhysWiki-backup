% 艾里函数

\begin{issues}
\issueDraft
\end{issues}

\footnote{参考 Wikipedia \href{https://en.wikipedia.org/wiki/Airy_function}{相关页面}.}\textbf{艾里函数(Airy function)}是微分方程
\begin{equation}
y'' - xy = 0
\end{equation}
的两个线性无关解, 分别记为 $\opn{Ai}(x)$ 和 $\opn{Bi}(x)$. 其积分表达式为(待补充:推导方法)
\begin{equation}
\opn{Ai}(x)=\frac 1\pi \int_0^\infty \cos(\frac{t^3}{3}+xt)\dd t
\end{equation}
\begin{equation}
\opn{Bi}(x)=\frac 1\pi \int_0^\infty \qty[\exp(-\frac{t^3}{3}+xt)+\sin(\frac{t^3}{3}+xt)]\dd t
\end{equation}

\begin{figure}[ht]
\centering
\includegraphics[width=8cm]{./figures/AiryF_1.pdf}
\caption{艾里函数} \label{AiryF_fig1}
\end{figure}

\subsection{渐进形式}
\footnote{参考 \cite{GriffQ} 中的 WKB 近似章节.}渐进形式为 $x \to \infty$
\begin{align}
% 已验证
\opn{Ai}(x) \to \frac{1}{2\sqrt{\pi} x^{1/4}} \exp(-\frac{2}{3}x^{3/2})\\
\opn{Bi}(x) \to \frac{1}{\sqrt{\pi} x^{1/4}} \exp(\frac{2}{3}x^{3/2})
\end{align}
$x \to -\infty$
\begin{align}
% 已验证
\opn{Ai}(x) \to \frac{1}{\sqrt{\pi} \abs{x}^{1/4}} \sin(\frac{2}{3}\abs{x}^{3/2}+\frac{\pi}{4})\\
\opn{Bi}(x) \to \frac{1}{\sqrt{\pi} \abs{x}^{1/4}} \cos(\frac{2}{3}\abs{x}^{3/2}+\frac{\pi}{4})
\end{align}

\begin{lstlisting}[language=Mathematica]
Plot[{AiryAi[x], 
  1/(Sqrt[\[Pi]] (-x)^(1/4)) Sin[2/3 (-x)^(3/2) + \[Pi]/4]}, {x, -10, 
  1}]
Plot[{AiryBi[x], 
  1/(Sqrt[\[Pi]] (-x)^(1/4)) Cos[2/3 (-x)^(3/2) + \[Pi]/4]}, {x, -10, 
  1}]
Plot[{AiryAi[x], 
  1/(2 Sqrt[\[Pi]] x^(1/4)) Exp[-(2/3) x^(3/2)]}, {x, -1/2, 5}]
Plot[{AiryBi[x], 1/(Sqrt[\[Pi]] x^(1/4)) Exp[2/3 x^(3/2)]}, {x, -1/2, 
  5}]
\end{lstlisting}
\addTODO{把以上代码替换成 Mathematica 的图, 用田字格}
\subsection{性质}
\subsection{微分方程变形}
令 $a, b\in \mathbb R$, 那么
\begin{equation}
y'' - (ax + b) y = 0
\end{equation}
的通解是
\begin{equation}\label{AiryF_eq1}
% 已数值验证
y(x) = C_1\opn{Ai}\qty(\frac{ax+b}{\abs{a}^{2/3}}) + C_2 \opn{Bi}\qty(\frac{ax+b}{\abs{a}^{2/3}})
\end{equation}
