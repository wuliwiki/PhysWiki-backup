% 二元关系
\pentry{朴素集合论\upref{NSet}}
\subsection{关系}

在朴素集合论\upref{NSet}中我们只关心了集合的基数,即集合中元素的数目.在这种语境下,任何两个元素数量相同的集合都可以看作是同一个集合.但是仅仅讨论集合的基数未免太过单调,缺少了很多有意思的理论,于是我们希望在集合的元素之间建立一些结构,来进行更细致的划分和研究.

\textbf{关系(relation)}是集合上最基础的一种结构.给定一个关系,我们就可以讨论一些元素之间\textbf{是否具有}这个关系.比如说,如果取一家三口构成一个集合,$\sim$代表的是“年龄大于”,那么我们可以说“爸爸对于孩子具有这个关系”,但是反过来“孩子对于爸爸不具有这个关系”.从这个例子可以看出,关系的表达方式很灵活,而且可以是有方向性的.讨论关系时,我们唯一关心的是给定元素之间是否具有这样的关系.

关系可以用在两个元素之间,也可以用在三个元素之间,甚至可以用在不特定的元素之间.

\subsection{二元关系}

在绝大多数数学和物理领域,我们只关心集合上的\textbf{二元关系(binary relation)}.如果$\sim$是在集合$A$上定义的一个二元关系,那么任意给定两个元素,我们都可以讨论它们之间是否具有这种关系,但如果给定三个元素,讨论就没有意义了.比如,如果$\sim$的定义是“年龄大于”,那么把三个人的年龄都拿过来比较就没有意义;不过,如果$\sim$的定义是“比后面两个人的年龄都大”,那么$\sim$就可以用在三个人身上.

对于集合$A$上的二元关系$\sim$,如果$x, y\in A$满足这个关系,我们可以把这句话表述为$x\sim y$.如果不满足



