% Linux 基础笔记

\subsection{基础}
\begin{itemize}
\item \href{http://faculty.salina.k-state.edu/tim/unix_sg/index.html}{一个教程}
\item Linux 命令行区分大小写
\item \verb`man <命令>`  会给出命令的帮助 (manual)
\item \verb`clear` 清空命令行(其实是向下滚动一页)
\item 几乎任何时候, Ctrl+C (有时候 Ctrl+D) 可以强制终止当前的工作 (注意只用于可以安全终止的任务!)
\item \verb`^` 代表 Ctrl 键
\item Ubuntu 中, Ctrl++ 就是放大, 记住, 是 + 不是 =.
\item Linux 并没有文件拓展名...
\item 可执行文件: 对 linux 来说缀名没有意义. 可执行文件如果在 system path, 就可以直接打文件名执行命令(一般没有拓展名). 如果在某个目录, 就打 \verb`目录/文件名` 执行. 如果在当前目录就用 \verb`./文件名` 执行
\item 除了二进制可执行文件, 文本文件也可以被看作可执行文件(与拓展名无关). 默认用 bash 执行, 但如果有 shebang, 就用 shebang 指定的程序执行
\item shebang 在文件第一行以 \verb`#!` 开头, 如 \verb`#!/bin/bash` (默认) 或者 \verb`#!/usr/bin/python`
\item 命令的短选项可以合并, 例如 \verb`ls -l -h` 可以写成 \verb`ls -lh`. 有些段选项有对应的长选项, 如 \verb`ls -l` 相当于 \verb`ls --human-readable`.
\item \verb`ls -1` 可以将文件名列成一列, \verb`ls -v` 可以按照数字的大小排序(否则是逐位排序, 如 1,10,100,2,10,200), \verb`ls *.md` 可以只列出 md 文件
\item 要区分命令的 argument 和命令选项的 argument, 后者紧跟在选项后面.
\item 使用 Tab 键可以自动完成命令
\item \verb`Ctrl-D` 可以关闭当前 terminal, 安装 bash-completion 软件可以让该功能更完善
\item \verb`$<var>` 表示变量的值, 自定义变量可以用 \verb`<var>=<val>`, 注意等号两边不能有空格.
\item \verb`env` 命令可以查看环境变量
\item 在 \verb`etc/profile` 文件的底部定义变量可以让所有人登录后都获得该变量.如果要只对一个用户创建该变量, 在其 home 目录创建 .profile 文件即可.
\item 每次打开新的 shell,\verb`/etc/bashrc` 和 \verb`~/.bashrc` 文件都会被执行
\item history 命令可以输出最近 1000 条输入的命令
\item 用 \verb`!<number>` 可以重新输入第 \verb`<number>` 条历史命令
\item \verb`/etc/passwd` 文件中记录了系统的所有 user
\item Ctrl+L 相当于 \verb`clear` 命令
\item \verb`df` 用于查看磁盘空间, \verb`-h` 选项可以显示 \verb`Mb`, \verb`Gb` 等.
\item \verb`du /some/path` 用于查看指定目录中所有子文件夹的大小, \verb`-a` 选项可以显示文件夹和文件. \verb`--max-depth=N` is used to control the level of subfolders to display, \verb`-h` is for human readable file size (\verb`Mb`, \verb`Gb`, etc)
\item \verb|du -hd 0| 用于查看当前目录的大小. \verb|-h| 是 human readable, \verb|-d| 是 depth
\item \verb`ctrl+u` 可以删除光标前面的所有内容
\item \verb`ctrl+c` 可以放弃当前输入直接开始新的一行
\end{itemize}

\subsection{目录与文件}
\begin{itemize}
\item \verb`pwd` (present working directory)
\item \verb`pwd -P` 可以显示当前的绝对目录, 即不含 \verb`~` 等符号以及 symbolic link
\item \verb`cd` (change directory) \verb`/` (硬盘根目录) \verb`~` (缩写 /home/parallels) (用户的 home 文件夹)
\item 只用 \verb`cd` 相当于 \verb`cd ~`
\item \verb`./` 表示当前目录,常用于执行可执行文件或者 \verb`.sh`
\item 常用目录有 \verb`~/Documents` ,  \verb`~/Desktop`, \verb`/usr/bin` (gfortran 安装在这里)
\item \verb`cd <文件夹名>` 只能用于当前目录中的文件夹
\item \verb`cd ..` 返回上层目录
\item \verb`cd -` 返回刚才所在的目录
\item 要 cd 到当前目录的绝对目录, 可以用 \verb|cd `pwd -P`|, 两个 \verb|`| (backtick)内的东西会先展开
\item \verb`rm <filename>` 或者 \verb`rm /<dir>/<filename>` 删除文件  (注意是永久删除!)
\item \verb`mv <filename> /<dir>` 可以移动文件, \verb`mv <filename> <newname>` 可以重命名. 注意 \verb|mv| 会覆盖文件, 但是不会覆盖文件夹(只要目标的同名文件夹非空就不行, 即使里面没有同名文件). 此时可以用 \verb|rsync|.
\item \verb`cp <filename> /<dir>` 复制文件 \verb`cp <filename> <newname>` 复制到当前目录且重命名. 复制多个文件用 \verb`cp <file1> <file2> <dir>`, 也可以在 \verb|<dir>| 前面加 \verb|-t|. 注意 \verb|cp| 会覆盖目标文件以及文件夹.
\item \verb|cp| 会默认拷贝 symlink 指向的文件而不是 symlink 本身. \verb|-d| 选项可以只拷贝 link.
\item \verb|cp| 会默认覆盖目标的同名文件, \verb|cp -n| 可以不覆盖且跳过(但不会提示). 如果要提示且手动逐个选择是否覆盖, 用 \verb|-i| 选项.
\item 打开文件的一般方式 \verb`<program_name> <filename.extension>`
\item 要使用外接储存器(U盘, 移动硬盘等), 需要新建一个文件夹, 然后连接外接储存器, 叫做 Mount The Drive. 习惯上, 新建文件夹都放在 \verb`/mnt` 目录下
\item 要连接储存器, 用 \verb`sudo fdisk -l` 显示所有的外接硬盘的路径名
\item \verb`sudo mount <硬盘地址> <新建文件夹路径>`
\item \verb`sudo umount <硬盘地址>` 断开连接
\item \verb`sudo mount -a` 把 \verb|/etc/fstab| 中的设置全部 mount 一次
\item 创建 symbolic link (相当于快捷方式), \verb`ln -s /path/to/file /path/to/symlink`
\item \verb`mkdir` 新建文件夹, \verb`rmdir` 删除空文件夹, \verb`rmdir -r` (recursive) 删除所包含的文件夹 (其实不用 \verb`rmdir` 用 \verb`rm` 也可以!), \verb`-P` 选项可以一次创建多层目录
\item \verb`dd` 命令可以直接在当前位置生成一个大小为 100 MB 的文件, 内容为随机, 还会显示写入速度
\verb`dd if=/dev/urandom of=./dump.txt bs=1048576 count=1024` 写 1024Mb 随机文件
\item \verb`locate ???` 命令可以从数据库中寻找计算机上所有文件名包含 \verb`???` 的文件, 但刚创建的文件可能找不到
\item \verb`sudo updatedb` 可以将刚创建的文件加到数据库中, 让 \verb`locate` 可以找到(\verb|-v| 选项显示哪些文件夹被更新)
\item 对于整个系统, \verb`locate` 比 \verb`find` 要快得多, 因为数据库已经是 index 过的 \verb`-c` 选项可以显示匹配结果的个数, \verb`-i` 选项可以不区分大小写
\item \verb|plocate| 是一个比较快的实现, \verb|mlocate| 是一个比较安全的实现, 命令都是 \verb|locate|, \verb|/etc/updatedb.conf| 中可以设置 \verb|updatedb| 的一些选项, 例如忽略挂载到某个目录的硬盘, 忽略 \verb|.git| 文件夹等, 忽略某些文件系统等.
\item \verb|rename 's/老名字/新名字/' *|. \verb|-n| 选项显示重命名的预览, 不真的重命名
\item \verb|sudo swapoff -a| 不使用 swap file, \verb|sudo swapon -a| 恢复使用. 可能还需要在 \verb|/etc/fstab| 里面注释掉相应的行.
\end{itemize}

\subsection{包管理}
\begin{itemize}
\item \verb`sudo apt-get install <appname>` 安装程序(99\%的程序都可以这样安装)(需要联网)
\item \verb`sudo apt-get remove <appname>` 卸载程序
\item \verb`sudo apt-get upgrade <appname>` 升级程序
\item \verb`apt-get` (和 \verb`apt-cash`, 如果听说过)中常用的功能都在 \verb`apt` 中有同样的, 所以大部分情况下只需要 \verb`apt` 就可以了, 例如 \verb`apt install`, \verb`apt purge`. \verb`apt` 对新手更友好, 例如会默认显示进度条, 更新的 package 数等.
\item linux 里面的程序设置什么的基本都储存在对应的 configuration file 里面. 通常这种文件以 .conf 结尾
\item \verb`apt show <packagename>` 可以显示某个包的信息
\item 如果安装了例如 \verb`libblas`, \verb`libgsl` 等 library, 用 \verb`dpkg -L <packagename>` 可以查看头文件和库文件(\verb`.a`, \verb`.so`) 的位置.
\item 如果包的名字以 \verb`-dev` 结尾, 通常意味着这个包提供头文件和库文件, 可以让你在写程序时调用
\item 如果包的名字以 \verb`-dbg` 解为, 通常意味着这个包的二进制中有调试信息, 可以在 debugger 里面分析
\item \verb|apt list --installed| 列出所有安装了的包
\item 不同的 ubuntu 版本可以用 apt 安装的包的版本也是有一些区别的.
\item \verb|dpkg| 用于手动(通过安装包文件)安装和卸载指定的包, 不会自动安装或者卸载 dependency, 也不会从网上下载包. \verb|apt| 会调用 \verb|dpkg|.
\item \verb|aptitude| 是比 \verb|apt| 更高级的包管理器, 还有 GUI 界面.
\end{itemize}

\subsection{文本相关}
\begin{itemize}
\item \verb`touch <filename>` 命令可以生成一个空的文本文件
\item \verb`nano` 是一个命令行应用, 可以编辑文本文件. 打开文件命令: \verb`nano <filename.txt>`
\item 但是比较著名的文本编辑还是 VIM 和 VI 
\item \verb`head` 和 \verb`tail` 命令可以预览一个文件的前几行和后几行
\item \verb`echo <text>` 可以在命令行显示该文字, 也就是把命令的 params 输出到 stdout
\item \verb`echo <text> > <filename>` 可以把文件中的文本替换 \verb`>>` 可以附加在文本最后. 如果用双引号, 甚至还可以换行.
\item \verb`>` 是 output redirect 操作, 类似, \verb`<` 是 input redirection 操作, 可以把文档中的内容输入到命令行.
\item linux 命令行有两个主要的 output, stdout 和 stderr, \verb`>` 只用于将 stdout 输出到文件, 如果也要输出 stderr 有三种办法: 1.\verb`<command> > file1.out 2 > file2.out`, 2. \verb`<command> > file.out 2 > &1`, 3. \verb`<command> &> file.out` 其中后两种将 stdout 和 stderr 输出到同一文件.
\item 运行一个程序,指定输入输出文件, 并让 standartd err 显示在 standard out \verb`./program.x < <inputfile> > <outputfile> 2>&1` 
\item \verb`>` 的输出文件如果不存在, 就会新建一个.
\item  \verb`cat 文件` 把文件的内容输出到 stdout
\item  \verb`cat` 把 stdin 的内容输出到 stdout
\item \verb`<命令> | less` 可以用键盘滚动输出结果(用于长输出以及没有滚动条的情况), 按 q 键退出
\item \verb`|` 的作用相当于把前面一个命令的输出放到后面一个命令的输入中.
\item \verb`<命令> > <文件名.txt>` 可以把命令行的输出写入文件中 !! 例如: \verb`date > time.txt , cal > calander.txt`
\item \verb`grep <string> < <filename>` 可以在文档中高亮标出指定字符串
\item \verb`cat <filename> | grep <string>` 可以达到同样的效果
\item 用 vim 的时候, 最好在前面加上 \verb`sudo`, 否则有可能编辑了保存不了(例如一些 configuration file). 甚至要首先用 \verb`chown` 把文件所有权变成自己的.
\item \verb`#` 在文件中起标注的作用 (如果文件中有 linux 代码), 在命令行, \verb`#` 后面的代码同样会被忽略.
\item 要转换文档 encoding,用 \verb|iconv -f [encoding] -t [encoding] -o [newfilename] [filename]|. 微软的 ANSI 格式其实叫做 iso-8859-1, 例如 `iconv -f iso-8859-1 -t utf-8 -o out.txt in.txt`.
\item \verb`sed` 命令可以用来处理文件或 standard input 中的文字, 并输出到文件或者 standard out. 在 \verb`SCID_TDSE` 的 makefile 中遇见, 命令是
\verb|sed -e 's/^!\*nq/    /' inputfile.f90 > outputfile.f90|
其中 \verb`-e` 是将后面的命令作用到 \verb`inputfile.f90` 上. \verb`s/str1/str2` 是将文本中的 \verb`str1` 替换成 \verb`str2`, \verb`\*` 转义为 \verb`*` (直接用 \verb`*` 就是通配符), \verb`^` 大概是要求 \verb`str1` 出现在行首.
\end{itemize}

\subsection{账户与权限}
\begin{itemize}
\item \verb`sudo` (super user do) 加在命令前可以获得管理员权限, \verb`su` (super user) 可以把以下所有操作改为管理员权限 (不建议常用).
\item \verb`su <usr_name>` 可以临时改变登录的身份, \verb|sudo -i| 或者 \verb|su| 可以持续管理员身份.
\item 使用 \verb|sudo| 以后, 命令的执行者就会临时变为 \verb|root|, 一些 config 文件也会使用 \verb|/root/| 文件夹中的而不是 \verb|/home/用户名| 中的, 例如 \verb|/root/.ssh/config|.
\item \verb`sudo` 的密码是用户密码, 而 \verb`su` 的密码是 root 密码, 后者的权限更高
\item \verb`useradd -m` (给新用户创建 home 目录) \verb`-g users` (设置用户组为 users) \verb`<username>`
\item \verb`passwd <username>` 可以修改密码
\item \verb`userdel <username>` 删除账户
\item \verb`users` 命令显示目前所有登录的用户, \verb`who` 命令可以显示更详细的信息
\item \verb`ls -l` 最左边第一列是 owner 的权限, 第二列是 group 的权限, 第三列是其他人的权限, \verb`r (read) w (write) x (execute)`
\item \verb`ls` 选项: \verb`-a` (all)(显示隐藏) \verb`-m` (挤到一起) \verb`-1` 显示为一列 \verb`-v` 按照数字大小排序
\item \verb`chown <user> <file>` 修改文件的所有者
\item \verb`chmod <权限> <file>` 修改文件的权限. <权限> 是三位8进制数, 每一位是 x(1),w(2),r(4)之和.
\item \verb`groups` 或者 \verb`groups <user>` 可以查看用户所在的组
\item \verb`id` 可以查看用户的信息 \verb`uid (user id) gid (group id)`
\item \verb`groupadd <groupname>` 新建用户组
\item \verb`groupdel <groupname>` 移除用户组
\item \verb`gpasswd -a <username> <groupname>` 可以在组内新增用户
\item \verb`vim /etc/group` 可以查看记录 group 信息的文件
\item \verb`gpasswd -d <username> <groupname>` 可以在组中移除用户
\item \verb`hostname` 可以显示当前主机的名字. \verb`$(hostname)` 是一个环境变量, 可以试试 \verb`echo $(hostname)`
\item \verb|sudo hostname 新名字| 可以改变 hostname
\item 其实真正和 windows 的任务管理器相对应的是 \verb`top`, 会显示实时进程. 
\item \verb`K <PID>` 会结束对应编号的进程. 系统的关键进程是杀不掉的.
\item \verb`last -数字` 可以查看最近若干次登录的记录
\end{itemize}

\subsection{进程}
\begin{itemize}
\item \verb`ps aux` 会显示当前所有进程 (processes), PID 是 process ID, 用于终止进程
\item \verb`ps aux | grep <name>` 会搜索当前进程中包含 \verb`<name>` 的内容
\item 一个很占 cpu 的命令可以用于测试
\verb`dd if=/dev/zero of=/dev/null &`
其中 \verb`&` 表示在后台运行,按回车后会输出 PID (进程号)
连续输入这个进程 4 次, 然后用 \verb`top` 命令打开进程管理
\item top 中的 \verb`NI` 就是 nice value
\item 在 top 界面中用 \verb`r` (renice) 可以调整进程的 nice value (优先级,越小越优先)按 ESC 可以退出
\item 在 top 界面中用 \verb`k` 可以杀进程,输入进程的 PID,两次回车即可
\item \verb`pstree` 可以看到哪些进程生成了哪些子进程, 例如 shell 里面运行 .sh 文件会生成一个子进程用于运行一个新的 shell
\item \verb`kill -STOP 进程号` 可以暂停程序
\item \verb`kill -CONT` 可以继续暂停的程序
\end{itemize}

\subsection{文件压缩和加密}
\begin{itemize}
\item \verb`tar -czvf <path/name.tar.gz> <some/path/or/file>` 是常见的压缩命令. 其中 \verb`-c` 是 create, \verb`-v` 是 verbose, \verb`-f` 指定目标文件(一定要紧接文件名), \verb`-z` 使用 gzip 压缩. 如果不需要压缩就用 \verb`-cvf <path/name.tar>`.
\item 要解压, windows 中可以用 好压或 Matlab, linux 中把以上命令中的 \verb`-c` 换成 \verb`-x` (extract)即可, 如 \verb`tar -xzvf <path/name.tar.gz> -C <some/path/>` 如果省略 \verb`-C` 选项就解压到当前文件夹.
\item openssl 可以用来给文件或字符串加密或解密
\item 文件加密: \verb`openssl aes-256-cbc -nosalt -in 文件名 -out 加密文件名 -pass pass:密码`
\item 文件解密: \verb`openssl aes-256-cbc -nosalt -d -in 加密文件名 -out 文件名 -pass pass:密码`
注意: 文件加密得到的结果的 hash 是不稳定的!
\item 字符串加密: \verb`echo '需要加密的字符串' | openssl enc -base64 -e -aes-256-cbc -nosalt -pass pass:密码`
\item 字符串解密: \verb`echo "需要解密的字符串" | openssl enc -base64 -d -aes-256-cbc -nosalt -pass pass:密码`
\item \verb`-salt` 和 \verb`-nosalt` 是什么?
\item \verb`-base64` 是什么?
\item 这样也可以给文件加密得到稳定 hash: \verb`cat 文件名 | openssl enc -base64 -e -aes-256-cbc -pass pass:密码 > 加密文件名`
\end{itemize}

\subsection{sftp}
\begin{itemize}
\item SFTP 其实是 ssh 的一个子功能, 通过 ssh 协议传输文件, 只要能 ssh 到某个机器, 就可以 sftp
\item 连接方法为 \verb`sftp [usr@][url]` 就和 \verb`ssh [usr@][url]` 一样. 成功以后会出现提示符 \verb`sftp>`, 可以进行下一步操作
\item \verb`sftp>` 环境下同样可以用 \verb`cd`, \verb`ls`, \verb`pwd` 等命令
\item \verb`get /path/to/file` 可以传输某个文件到本地(进入 sftp 时的目录)
\item \verb`get /remote/file /local/file` 可以传输某个文件到本地指定的目录和文件名, 但该目录必须存在
\item \verb`get -r 文件夹` 可以传送整个文件夹
\item 如果要指定端口用 \verb`sftp -oPort=xxx xxx@xxx.xxx.xxx.xxx`
\end{itemize}

\subsection{rsync}
\begin{itemize}
```bash
for ((i=275;i<=294;++i)); do rsync -zvh addis1@headnode.beocat.ksu.edu:/homes/addis1/github/FEDVR-TDSE/data/${i}/analysis/* ./${i}/analysis/; done
```
\item 格式 \verb`rsync -avzh file1 file2 folder3 file4 dest_path` 其中 file1 到 file4 等文件或文件夹会自动同步到 dest_path 文件夹中(例如生成 \verb`dest_path/file1`, \verb`dest_path/folder3`). 如果 dest_path 不存在, 会自动用 mkdir 生成,但只能生成一层. file, folder 和 dest_path 都可以是远程或者本地
\item \verb`-a` 是 archive (包括 permission 和 symlink), \verb`-v` 是 verbose (显示详情), \verb`-z` 是 zip (传输过程压缩) \verb`-h` 人可读
\item \verb`rsync -avzh source/ dest/ --delete` 会把 dest 文件夹中多余的文件删除
\item 如果远程目录有需要 \verb`\` 的字符, 例如 \verb`\(...\)`, 需要用三次, 例如 \verb`\\\(...\\\)`. 本地的 shell 会先把 \verb`\\` 变为 \verb`\`, \verb`\(` 变为 \verb`(`
\item 若要在每个文件传输完成之后从原来文件夹删除, 用 \verb`--remove-source-files`
\item 若要删除目标文件夹中, 源文件夹不存在的内容, 用 \verb`--delete`
\end{itemize}

\subsection{curl}
\begin{itemize}
\item \verb|curl| 是 client URL 的意思, 用于通过 url 从服务器接受或发送信息.
\item \verb|curl "网址" -so 文件| 可以把某个 url 返回的字符串(如 html 和 json)保存到文件中, 其中 \verb|-s| 是 silent, \verb|-o| 是 output file. 这大概相当于在浏览器中输入 \verb|网址| 然后复制 html 源码等.
\item \verb|curl --header "<HeaderName>: 123" www.example.com|. 可以发送一个 header 的 url, 也可以用 \verb|-H|
\item \verb|curl -u 用户名:密码 www.example.com| 使用 Basic Auth 进行用户登录
\end{itemize}

\subsection{服务器}
\begin{itemize}
\item \verb`sudo tasksel` 快速安装某一类型的服务器需要所有的软件 (email server , file server, ...). 进入界面以后, 用空格选择想要的项目然按空格确认安装.
\item \verb`sudo /etc/init.d/<aptname> start` 或者 \verb`stop` 或者 \verb`restart` 可以进行对某个软件的 service 进行操作, 而不需要重启 linux 系统.
\end{itemize}

\subsection{环境变量}
\begin{itemize}
\item \verb|echo $VAR| 显示环境变量或任何变量
\item \verb|export VAR=XXX| 修改环境变量
\item \verb|export VAR=~/opt/bin:$VAR| 可以 append 某个变量(如 PATH)
\item \verb|PWD| 是当前路径, 修改该变量相当于 \verb|cd|
\item \verb|PATH| 是命令搜索路径, 只有这里面的目录中的执行文件才不指定路径
\item \verb|HOME| 是用户的 \verb|~| 文件夹
\item \verb|LD_LIBRARY_PATH| 是动态链接库的默认搜索地址, 见这里\upref{gppLib}.
\item \verb|LD_RUN_PATH| 编译的链接阶段, 如果没有指定 \verb|rpath|, 就用该变量中的 path 替代\upref{gppLib}.
\end{itemize}

\subsection{其他}
\begin{itemize}
\item \verb`alias` 可以给一个长命令设置别名, 例如 \verb|alias abcde='echo hello'|, 仅限于当前 terminal, 可以添加到 \verb|~/.bashrc| 中每次 login 都定义.
\end{itemize}


\begin{figure}[ht]
\centering
\includegraphics[width=14cm]{./figures/LinNt_1.png}
\caption{参考 \href{https://www.howtogeek.com/}{How to geek}} \label{LinNt_fig1}
\end{figure}
