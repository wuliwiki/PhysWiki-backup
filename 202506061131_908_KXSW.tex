% 柯西-施瓦茨不等式(综述)
% license CCBYNCSA3
% type Wiki

本文根据 CC-BY-SA 协议转载翻译自维基百科 \href{https://en.wikipedia.org/wiki/Cauchy\%E2\%80\%93Schwarz_inequality}{相关文章}。

柯西–施瓦茨不等式(也称为柯西–布尼亚科夫斯基–施瓦茨不等式)是对内积空间中两个向量的内积绝对值的一个上界,其上界由这两个向量范数的乘积给出。它被认为是数学中最重要且应用最广泛的不等式之一。

向量的内积可以用于描述有限和(通过有限维向量空间)、无穷级数(通过序列空间中的向量)以及积分(通过希尔伯特空间中的向量)。柯西于1821年首次发表了关于求和形式的不等式。相应的积分形式的不等式由布尼亚科夫斯基于1859年发表,赫尔曼·施瓦茨于1888年发表了积分形式的现代证明。
