% BRST量子化
弦理论的量子化方案有三种:协变量子化,光锥量子化和BRST量子化.三种方案的优劣比较如下:
\begin{itemize}
\item 协变量子化:洛伦兹不变能明显表现出来.有鬼.时空维数是26维难以证明.
\item 光锥量子化:洛伦兹不变不再明显.没有鬼.时空维数是26维容易证明.
\item BRST量子化:洛伦兹不变能明显表现出来.有鬼.时空维数是26维容易证明.
\end{itemize}
\subsubsection{BRST算符}
首先我们来考虑李代数.考虑算符$K_i$,这些算符满足如下的李代数
\begin{equation}
[K_i,K_j] = f_{ij}{}^k K_k~.
\end{equation}
其中$f_{ij}{}^k$被称作理论的结构常数.这些结构常数满足
\begin{equation}
f_{ij}{}^m f_{mk}{}^i + f_{jk}{}^m f_{mi}{}^l+f_{ki}{}^m f_{mj}{}^l = 0 ~. 
\end{equation}
现在我们引入两个鬼场,记作$b_i$和$c_j$,它们满足反对易关系
\begin{equation}
\{ c^i, b_j \} = \delta^i_j~.
\end{equation}

