% 东南大学 2016 年 考研 量子力学
% license Usr
% type Note

\textbf{声明}:“该内容来源于网络公开资料,不保证真实性,如有侵权请联系管理员”

\textbf{1.(15 分)}以下叙述是否正确:(1)宁称贫符成是危密练特,为是么汇梦待:(2)或海公浪高数一定是可归一化的:(3)时间反演对称性导致能尿分恒:(4)自欣角从经是最子为学中特有的为学藏、哈会具力学中没有对应的力学量:(5)角动最算符知广作在共同木念。

\textbf{2.(15 分)}设质量为 $m$ 的粒子在势场 $V(r)$ 中运动,波函数为 $\psi(r,t)$。

\begin{enumerate}
    \item  试证明粒子的能量平均值为
    \[
    E(t) = \int d^3r \, \omega(r,t), \quad \omega = \frac{\hbar^2}{2m} \nabla \psi^* \cdot \nabla \psi + \psi^* V \psi~
    \]
    
    \item  试证明能量守恒公式
    \[
    \frac{\partial \omega}{\partial t} + \nabla \cdot \mathbf{S} = 0, \quad S(r,t) = -\frac{\hbar^2}{2m} \left(\frac{\partial \psi^*}{\partial t} \nabla \psi + \frac{\partial \psi}{\partial t} \nabla \psi^*\right)~
    \]
\end{enumerate}
\textbf{3.(15 分)}粒子的轨道角动最算符定义为:
\[
\hat{l}_x = \hat{y} \hat{p}_z - \hat{z} \hat{p}_y, \quad \hat{l}_y = \hat{z} \hat{p}_x - \hat{x} \hat{p}_z, \quad \hat{l}_z = \hat{x} \hat{p}_y - \hat{y} \hat{p}_x,~
\]
基本对易关系为
$[\hat{x}_\alpha, \hat{p}_\beta] = i\hbar \delta_{\alpha\beta}$, 求对易式: $[\hat{l}_x, \hat{y}] ,  [\hat{l}_x, \hat{p}_y] [\hat{l}_x, \hat{l}_y] , .$

\textbf{4.(15 分)}质量为 $\mu$ 的粒子处于一维无限深方势阱中,设$(x,y)=0$ (当 $0 < x < a, 0 < y < b$);$V(x, y) =\infty$(我它区城),
试求:
\begin{enumerate}
    \item  粒子的能量本征值和本征函数;
    \item  当 $a = b$ 时,最低4个能级的简并度,。
\end{enumerate}

\textbf{5.(15 分)}质量为 $m$ 的粒子以一定的能量 $E (E > 0)$ 沿 $x$ 轴方向入射。碰到如下的势量:
\[
V(x) = 0, (x < 0), \quad V(x) = =V_0 x/a, (0 \leq x \leq a), \quad V(x) = V_0, (x > a).~
\]

\begin{enumerate}
    \item  分别写出入射区 ($x < 0$) 和透射区 ($x > a$) 的定态薛定谔方程,并给出定态解 $\psi(x)$ 的形式。
    
    \item  利用几率流密度公式
    \[
    j = -\frac{i\hbar}{2m} \left(\psi^* \frac{d\psi}{dx} - \psi \frac{d\psi^*}{dx} \right)~
    \]
    求出入射浓几率流密度 $j_i$、反射波几率流密度 $j_r$、及透射波几率流密度 $j_t$的表达式,并写出求反射系数 $r$ 和透射系数 $t$ 的表达式。
    \item  利用几率守恒定理证明:$r+t=1$
\end{enumerate}

\textbf{6.(15 分)}设体系的哈密顿算符 $\hat{H}(\lambda)$ 含有一个实参数 $\lambda$,其束缚态能级为 $E_n(\lambda)$ 归一化能量本征态为 $\ket{\psi_n(\lambda)}$ ,试证明
\[
\partial E_n/\partial \lambda = \left\langle \psi_n \left| (\partial \hat{H}/\partial \lambda) \right| \psi_n \right\rangle.~
\]

\textbf{7.(15 分)}质量为 $\mu$ 带电 $q$ 的粒子限制在半径为 $R$ 的圆环上运动,环心处通有一条细长的磁通管中,磁场限制在细管内,圆环上及其近区域的磁场为 $0$。设通过圆环的磁通量为 $\Phi$,则圆环邻近区域的矢量势可取为(在适当的标示中)
\[
\Lambda(\rho, \phi, z) = \left( \Phi/2\pi\rho \right) e_\phi.~
\]

\begin{enumerate}
    \item 写出粒子的哈密顿算符;
    \item 求能量本征值,并讨论能级简并情况。
\end{enumerate}3

\textbf{8.(15 分)}设粒子的波函数为 $\psi(\theta, \varphi) = aY_{1,1}(\theta, \varphi) + bY_{2,1}(\theta, \varphi)$,其中 $Y_{l,m}(\theta, \varphi)$ 为球谐函数。并且
\[
|a|^2 + |b|^2 = 1~
\]
试求:
\begin{enumerate}
    \item [(1)] 角动量 $\hat{L}_z$ 的可能测量值及其平均值;
    \item [(2)] $\hat{L}^2$ 的可能测量值及相应的几率。
\end{enumerate}


\textbf{9.(15 分)}

\textbf{10.(15 分)}