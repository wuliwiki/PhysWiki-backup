% 量子隐形传态
% license CCBYSA3
% type Wiki

(本文根据 CC-BY-SA 协议转载自原搜狗科学百科对英文维基百科的翻译)

\begin{figure}[ht]
\centering
\includegraphics[width=6cm]{./figures/4e2cfec8fd44a89d.png}
\caption{量子隐形传态} \label{fig_LZYXCT_1}
\end{figure}

量子传输是一个传输量子信息(例如,原子或光子的确切状态)的过程,在这个过程中,借助于经典通信和先前发送和接收位置之间共享的量子纠缠,量子信息可以从一个位置传输到另一个位置。因为它依赖于经典通信,其速度不能超过光速,所以不能用于超光速传输或经典位通信。虽然已经证明可以在两个(纠缠的)量子之间传送一个或多个量子位的信息, 但在比分子大的任何东西还没有实现这一点。[1]

虽然这个名字的灵感来自小说中常用的传送,但量子传输仅限于信息的传送,而不是物质本身。量子传输不是一种传输方式,而是一种通信方式:它提供了一种将量子位从一个位置传输到另一个位置的方式,而不需要移动物理粒子。

这个术语是物理学家查尔斯·贝内特创造的。1993年,C. H. Bennett、G. Brassard、C. Crépeau、R. Jozsa、A. Peres和W. K. Wootters发表了第一篇阐述了量子传输概念的开创性论文[2] 。量子传输最初是在单个光子中实现的,[3] 后来在原子、离子、电子和超导电路等各种材料系统中得到证明。最新报道的量子传输距离是 1,400 km (870 mi) ,由潘建伟小组利用Micius卫星进行空间量子传输。[4][5][6]

\subsection{非技术总结}

在与量子或经典信息理论相关的问题中,使用最简单的可能的信息单位——双态系统是很方便的。在经典信息中,这是一个比特,通常用1或0(或真或假)来表示。比特的量子模拟是一个量子比特,或量子比特。量子比特编码一种称为量子信息的信息,它与“经典”信息截然不同。例如,量子信息既不能被复制(不克隆定理),也不能被销毁(不删除定理)。

量子传输提供了一种将一个量子比特从一个位置移动到另一个位置的机制,而不需要物理传输该量子比特通常所附着的底层粒子。就像电报的发明允许经典比特在大陆间高速传输一样,量子传输有朝一日也能以同样的方式移动量子比特。[7][8] 截至2015年,单光子、光子模式、单原子、原子系综、固体缺陷中心、单电子和超导电路的量子态已经被用作信息载体。[9]

量子比特的移动不像互联网上的通信那样需要“事物”的移动:它不需要传输任何量子对象,但是每传送一个从发送者到接收者的量子比特,就需要传输两个经典比特。实际的传送协议要求创建纠缠量子态或贝尔态,并且它的两个部分在两个位置(源和目的地,或爱丽丝与鲍伯)之间共享。本质上,在移动一个量子比特之前,必须首先在两个位置之间建立某种量子通道。传送也需要建立一个经典的信息通道,因为每个量子比特必须传输两个经典比特。其原因是测量结果必须进行通信,并且必须通过普通的经典通信信道进行。起初,对这种经典渠道的需求似乎令人失望;然而,这与普通通信没有什么不同,普通通信需要电线、无线电或激光。更重要的是,贝尔态最容易被激光光子共享。因此,原则上,远程传输可以通过开放空间来完成,即不需要通过电缆或光纤发送光。

单个原子的量子态已经被传送了。[10][10][10] 一个原子由几个部分组成:电子态的量子比特或原子核周围的电子层、原子核本身的量子比特、最后是组成原子的电子、质子和中子。物理学家已经传输了以原子电子状态编码的量子比特;他们没有传送核状态,也没有传送核本身。因此说“一个原子被传送了”是不准确的,应该说原子的量子态被传送了。因此,进行这种传输需要在接收位置有一批原子,可以让量子比特印在上面。传送核状态的重要性还不清楚:核状态确实会影响原子,例如在超精细分裂中,但是在一些未来的“实际”应用中,这种状态是否需要传输是有争议的。



量子信息理论的一个重要方面是纠缠,它在不同的物理系统之间施加统计相关性。这些相关性即使在选择和独立执行测量时也是成立的,这是由于彼此之间的因果联系,正如贝尔试验实验所证实的那样。因此,在时空的一个点上进行的测量选择所产生的观测似乎会瞬间影响到另一个区域的结果,即使光还没有时间传播这个距离;这一结论似乎与狭义相对论不符(EPR悖论)。然而,这种相关性永远不能被用来于以比光速更快的速度传输任何信息,这是一个包含在非通信定理中的陈述。因此,作为一个整体,量子传输永远不会超光速,因为量子比特在伴随的经典信息到来之前是无法重建的。

理解量子传输需要在有限维线性代数、希尔伯特空间和投影矩阵中有良好的基础。利用二维复数值向量空间(希尔伯特空间)来描述量子比特,这是下面给出的形式化操作的主要基础。理解量子传输的数学并不绝对需要量子力学的工作知识,尽管没有这样的了解,方程的深层含义可能仍然相当神秘。

\subsection{协议}

\begin{figure}[ht]
\centering
\includegraphics[width=6cm]{./figures/a499bb5ea758b0d3.png}
\caption{光子量子传输} \label{fig_LZYXCT_3}
\end{figure}
