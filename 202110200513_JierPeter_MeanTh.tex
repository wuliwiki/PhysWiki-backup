% 微分中值定理
% 拉格朗日中值定理|柯西中值定理|罗尔中值定理|mean value theorem

\pentry{导数\upref{Der}}

\subsection{一点准备}

\begin{definition}{函数极值}
考虑实函数$f(x)$.如果\textbf{存在}一个实数轴上的开集$O$,且有$x_0\in O$,使得对于任意的$x\in O$,都有$f(x_0)\geq f(x)$,则称$f(x_0)$是$f$在$O$上的一个\textbf{极大值(maximum)};如果$x_0$满足的条件改为对于任意的$x\in O$,都有$f(x_0)\leq f(x)$,则称$f(x_0)$是$f$在$O$上的一个\textbf{极小值(minimum)}.

极大值和极小值统称为\textbf{极值(extremum)}.

如果$f(x_0)$是一个极大值,那么称$x_0$是一个\textbf{极大值点(maximum point)};相应地,极小值对应的自变量$x_0$是一个\textbf{极小值点(minimum point)}.极大值点和极小值点统称\textbf{极值点(extremum point)}.
\end{definition}


简单来说,极大值的意思就是,取包含极大值点的足够小的范围,那么范围内的所有函数值都小于等于极大值.极小值则反过来,范围内的函数值都大于等于它.

我们要求“\textbf{存在}一个开集$O$”,实际上就是在说存在一个范围.

\begin{example}{}\label{MeanTh_ex1}
考虑实函数$f(x)=x^3-x$,如\autoref{MeanTh_fig1} 所示.

\begin{figure}[ht]
\centering
\includegraphics[width=8cm]{./figures/MeanTh_1.pdf}
\caption{$f(x)=x^3-x$的函数图像.} \label{MeanTh_fig1}
\end{figure}

在$x=\frac{\sqrt{3}}{3}$处,$f$取极小值,但显然不是最小值.

在$x=-\frac{\sqrt{3}}{3}$处,$f$取极大值,也显然不是最大值.


\end{example}


\begin{theorem}{Fermat定理}\label{MeanTh_the1}
考虑实函数$f(x)$.如果$x_0$是$f$的一个极值点,且$f(x)$在$x_0$处可导,那么$f'(x_0)=0$.
\end{theorem}

\textbf{证明}:

假设$f(x)$在$x_0$处可导且取极大值,并\textbf{反设}$f'(x_0)>0$\footnote{取极小值和/或反设$f'(x_0)<0$的情况可以类比,在此不赘述.反设就是指“反过来假设定理不成立”.}.

那么由于可导,$f(x)$在$x_0$处的右极限存在且等于导数,即右极限大于零.这样一来,取任意正向接近$x_0$的数列$\{a_n\}$,则对于任意正整数$N$,必然总有编号大于$N$的$a_n$使得$f(a_n)>f(x_0)$\footnote{否则,如果存在一个正整数$N$使得所有编号大于$N$的$a_n$都小于等于$f(x_0)$,则这些$a_n$计算出的割线斜率$\frac{f(a_n)-f(x_0)}{a_n-x_0}$就小于等于零了,取极限以后,可得$f'(x_0)\leq 0$,而这和我们反设的“$f'(x_0)<0$”相矛盾.}.又由于$\lim\limits_{n\to\infty}a_n=0$,故得,对于任意包含$x_0$的开集(范围)$O$(对应编号$N$),总有$f(a_n)>f(x_0)$,于是$f(x_0)$就不是极大值了.

结论和假设矛盾,故反设部分不成立.将以上讨论推广到$f(x_0)$取极小值和/或反设$f'(x_0)<0$的情况后,可得最终结论:$f'(x_0)$必为零.

\textbf{证毕}.

\autoref{MeanTh_the1} 就可以用来快速计算出\autoref{MeanTh_ex1} 里的两个极值点的位置,即导数为零的地方.


\subsection{三个中值定理}

%缺例图

\begin{definition}{Rolle中值定理}
设$f(x)$在区间$[a, b]$上连续,在$(a, b)$内可导,且$f(a)=f(b)$,那么存在一个$x_0\in(a, b)$,使得$f'(x_0)=0$.
\end{definition}

%需要区间套定理,先把预备知识写掉好了.

\textbf{证明}:

如果直接引用“闭区间上的连续函数必有极值”这一定理,那么根据

\textbf{证毕}.














