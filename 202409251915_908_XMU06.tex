% 厦门大学 2006 年 考研 量子力学
% license Usr
% type Note

\textbf{声明}:“该内容来源于网络公开资料,不保证真实性,如有侵权请联系管理员”

\subsection{(20 分)简述(每小题5分)}
(1)什么是玻色(Bose)子和费米(Fermi)子?简要介绍玻色子和费米子的主要特性;

(2)正常塞受(Zeeman)效应及其解释:

(3)解释能级简并的概念并指出其起因:

(4)什么是跃迁选择定则?简单解释其起因.
\subsection{(10 分)}
对低速运动的一维自由粒子,指出下列推导过程中的错误所在:

由 $E = h \nu, \\ \nu = \frac{v}{\lambda}, \\ p = \frac{h}{\lambda}$ 和 $ p = mv$,

得$E = h \frac{v}{\lambda} = \frac{h}{\lambda} v = p v = mv^2 = 2 \cdot \frac{1}{2} mv^2 = 2E$.
\subsection{(20分)}
假设质量为 $m$ 的粒子在一维无限深势阱

\subsection{(20分)}