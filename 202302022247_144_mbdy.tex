% 介质的边界条件
% keys 边界条件 介质

\begin{issues}
\issueTODO 需要添加A的边界条件;补充相应的证明
\end{issues}
\pentry{麦克斯韦方程组(介质)\upref{MWEq1}}

\footnote{本文参考自\cite{GriffE}与周磊教授的\href{http://fdjpkc.fudan.edu.cn/d200927/2009/0314/c8569a14801/page.htm}{讲义}。}
在解决场与势在电、磁介质边界的问题时,微分形式的麦克斯韦方程组不再适用,但积分形式的麦克斯韦方程组仍然适用。(边界处的场可以是不连续的)

本文中,$\sigma_f$指自由面电荷密度,$\sigma$指总面电荷密度,即包括所有自由电荷与因介质极化而产生的感应电荷。在计算时,使用势的边界条件计算,往往比使用场更为简便。

\subsection{电场}

\subsubsection{E场}
\begin{equation}\label{mbdy_eq3}
E^\perp_{above} - E^\perp_{below} = \frac{\sigma}{\epsilon_0}
\end{equation}
\begin{equation}\label{mbdy_eq1}
\epsilon_{above}E^\perp_{above} - \epsilon_{below}E^\perp_{below} = \sigma_f
\end{equation}
\begin{equation}
E^\parallel_{above} - E^\parallel_{below} = 0
\end{equation}

\subsubsection{D场}
\begin{equation}\label{mbdy_eq6}
D^\perp_{above} - D^\perp_{below} = \sigma_f
\end{equation}
将 $D_{above}=\epsilon_{above} E, D_{below}=\epsilon_{below} E$ 代入该式,即可推导出\autoref{mbdy_eq1} 。

\subsubsection{电势 $\varphi$}
\begin{equation}
\varphi_{above}-\varphi_{below}=0
\end{equation}
\begin{equation}\label{mbdy_eq4}
\pdv{\varphi_{above}}{n} - \pdv{\varphi_{below}}{n}  = -\frac{\sigma}{\epsilon_0}
\end{equation}
\begin{equation}\label{mbdy_eq5}
\epsilon_{above}\pdv{\varphi_{above}}{n} - \epsilon_{below}\pdv{\varphi_{below}}{n}  = -\sigma_f
\end{equation}
将$\bvec E= -\grad \varphi$分别代入\autoref{mbdy_eq3} \autoref{mbdy_eq1} ,即可得\autoref{mbdy_eq4} \autoref{mbdy_eq5} 。


\begin{example}{}
初步说明 \autoref{mbdy_eq6} 。
\begin{figure}[ht]
\centering
\includegraphics[width=5cm]{./figures/mbdy_1.pdf}
\caption{边界处的高斯面.仿自\cite{GriffE}} \label{mbdy_fig1}
\end{figure}
在两介质的边界处绘制一个高斯面,对其运用麦克斯韦方程组(介质)\upref{MWEq1}的积分形式 $\oint \bvec D \vdot \dd{\bvec A} = q_f$。

当高度 $d\rightarrow0$时,D场的垂直分量只存在于上下表面,即$D^\perp_{above} A- D^\perp_{below} A= q_f$ (二者符号不同是因为积分时总是取曲面向外为面法向量的正方向)。两边同除以A,得 $D^\perp_{above} - D^\perp_{below} = \sigma_f$
\end{example}


\subsection{磁场}

\subsubsection{B场}
\begin{equation}
B^\perp_{above} - B^\perp_{below} = 0
\end{equation}
\begin{equation}
\bvec B^\parallel_{above} - \bvec B^\parallel_{below} = \mu_0\bvec K \times \hat n 
\end{equation}
\begin{equation}\label{mbdy_eq2}
\frac{1}{\mu_{above}}\bvec B^\parallel_{above} - \frac{1}{\mu_{below}} \bvec B^\parallel_{below} = \bvec K_f \times \hat n 
\end{equation}
K: 面电流密度

\subsubsection{H场}
\begin{equation}
\bvec H^\parallel_{above} - \bvec H^\parallel_{below} = \bvec K_f \times \hat n 
\end{equation}

\autoref{mbdy_eq2}  是他的推论。
% \subsubsection{磁矢势A}
% \begin{equation}
% \bvec A_{above}-\bvec A_{below}=0
% \end{equation}

% \begin{equation}
% \pdv{\bvec A_{above}}{n}-\pdv{\bvec A_{below}}{n}=-\mu_0 \bvec K
% \end{equation}

\subsubsection{磁标势$\varphi$ (如果有定义)}
\begin{equation}
\varphi_{above}-\varphi_{below}=0
\end{equation}
\begin{equation}
\mu_{above}\pdv{\varphi_{above}}{n} - \mu_{below}\pdv{\varphi_{below}}{n}  = 0
\end{equation}
