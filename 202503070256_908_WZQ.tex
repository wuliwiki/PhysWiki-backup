% 完整群(综述)
% license CCBYSA3
% type Wiki

本文根据 CC-BY-SA 协议转载翻译自维基百科\href{https://en.wikipedia.org/wiki/Holonomy}{相关文章}。

\begin{figure}[ht]
\centering
\includegraphics[width=6cm]{./figures/8ee31b3dfe98af26.png}
\caption{球面上沿分段光滑路径的平行移动。初始向量标记为 \(V\),它沿着曲线被平行移动,最终得到的向量标记为 \( \mathcal{P}_\gamma (V) \)。如果路径发生变化,平行移动的结果也会不同。} \label{fig_WZQ_1}
\end{figure}
在微分几何中,光滑流形上一个联络的\textbf{平行迁移群}(holonomy)描述的是:沿着闭合回路进行平行移动时,几何数据未被保持的程度。平行迁移群是联络曲率所导致的一种普遍几何效应。对于平坦联络,相关的平行迁移群是一种单值延拓(monodromy),并且本质上是一个全局概念。而对于曲率非零的联络,平行迁移群同时具有非平凡的局部和全局特征。

任何流形上的联络都会通过其平行移动映射引出某种\textbf{平行迁移群}(holonomy)的概念。最常见的平行迁移群形式是具有某种对称性的联络。重要的例子包括:黎曼几何中Levi-Civita联络的平行迁移群(称为黎曼平行迁移群),向量丛上联络的平行迁移群,Cartan联络的平行迁移群,以及主丛上联络的平行迁移群。在这些情形下,联络的平行迁移群都可以与某个李群(即\textbf{平行迁移群})对应起来。根据Ambrose-Singer定理,联络的平行迁移群与该联络的曲率密切相关。

对\textbf{黎曼平行迁移群}(Riemannian holonomy)的研究推动了许多重要的发展。\textbf{平行迁移群}最早由\textbf{埃利·嘉当}(Élie Cartan)于1926年引入,用于研究和分类对称空间。然而,直到很久之后,平行迁移群才被用来在更一般的背景下研究黎曼几何。

1952年,\textbf{乔治·德拉姆}(Georges de Rham)证明了\textbf{德拉姆分解定理},该定理通过将切丛分解为在局部平行迁移群作用下的不可约子空间,从而将黎曼流形分解为多个黎曼流形的笛卡尔积。随后在1953年,\textbf{马塞尔·贝尔热}(Marcel Berger)对可能出现的不可约平行迁移群进行了分类。

黎曼平行迁移群的分解和分类在物理学和弦理论中都有应用。
\subsection{定义} 
\subsubsection{向量丛上联络的平行迁移群}  
设\(E\)是光滑流形\(M\)上的一个秩为\(k\)的向量丛,\(\nabla\)是\(E\)上的一个联络。对于一个以\(M\)中点 \(x\) 为基点的分段光滑闭合路径\(\gamma : [0,1] \to M\),联络\(\nabla\)定义了一个\textbf{平行移动映射}\(P_\gamma : E_x \to E_x\)它作用在 \(E\)在点\(x\)处的纤维上。这个映射是线性的且可逆的,因此它定义了一个属于一般线性群\(GL(E_x)\)的元素。联络\(\nabla\)在基点\(x\)处的\textbf{平行迁移群}(holonomy group)定义为:
\[
\operatorname{Hol}_x(\nabla) = \{P_\gamma \in \mathrm{GL}(E_x) \mid \gamma \text{ 是以 } x \text{ 为基点的闭合路径}\}.~
\]
其中\(P_\gamma\) 是路径 \(\gamma\)对应的平行移动映射。基点\(x\)处的\textbf{限制平行迁移群}(restricted holonomy group)是子群:\(\operatorname{Hol}_x^0(\nabla)\)它由\textbf{可缩至点的闭合路径} \(\gamma\)所对应的平行移动映射组成。

如果 \(M\) 是连通的,那么平行迁移群(holonomy group)对于基点 \(x\) 的依赖仅体现在一般线性群 \(GL(k, \operatorname{R})\) 中的共轭关系上。具体来说,如果 \(\gamma\) 是从 \(x\) 到 \(y\) 的一条路径,那么:
\[
\operatorname{Hol}_y(\nabla) = P_\gamma \operatorname{Hol}_x(\nabla) P_\gamma^{-1}.~
\]
此外,选择不同的方式将纤维 \(E_x\) 与 \(\operatorname{R}^k\) 进行同构,也会得到共轭的子群。  

因此,有时,特别是在一般性的或非正式的讨论中(比如下面的内容),人们会省略对基点的引用,此时默认该定义仅确定到共轭的意义下。

关于平行迁移群(holonomy group),以下是一些重要性质:
\begin{itemize}
\item \(\operatorname{Hol}^0(\nabla)\)是\(GL(k,\operatorname{R})\) 的一个连通的李子群。
\item \(\operatorname{Hol}^0(\nabla)\)是\(\operatorname{Hol}(\nabla)\) 的单位连通分量(identity component)。
\item 存在一个从基本群\(\pi_1(M)\)到商群\(\operatorname{Hol}(\nabla)/\operatorname{Hol}^0(\nabla)\) 的自然满群同态:\(\pi_1(M) \to \operatorname{Hol}(\nabla)/\operatorname{Hol}^0(\nabla)\)
其中\(\pi_1(M)\)是\(M\)的基本群,该同态将同伦类\([\gamma]\)映射为:\(P_\gamma \cdot \operatorname{Hol}^0(\nabla)\).
\item 如果\(M\)是单连通的,则:\(\operatorname{Hol}(\nabla) = \operatorname{Hol}^0(\nabla)\).
\item 当且仅当\(\nabla\)是平坦的(即曲率为零),\(\operatorname{Hol}^0(\nabla)\)是平凡群(只包含单位元)。
\end{itemize}
\subsubsection{主丛上联络的平行迁移群}  
主丛上联络的平行迁移群的定义与向量丛上的定义类似。  

设 \(G\) 是一个李群,\(P\) 是一个在光滑流形 \(M\) 上的\textbf{主 \(G\)-丛},并且 \(M\) 是可度量紧化的(paracompact)。设 \(\omega\) 是 \(P\) 上的一个联络。  

给定一条以 \(M\) 中点 \(x\) 为基点的分段光滑闭合路径 \(\gamma : [0,1] \to M\),以及纤维上点 \(p \in P_x\)(\(p\) 属于 \(x\) 处的纤维),该联络定义了一条唯一的\textbf{水平提升}路径:\(\tilde{\gamma} : [0,1] \to P\)
满足:\(\tilde{\gamma}(0) = p\).这条提升路径的终点 \(\tilde{\gamma}(1)\) 一般不会等于 \(p\),而是纤维中某个形如 \(p \cdot g\) 的点,其中 \(g \in G\)。

在 \(P\) 上定义一个等价关系“\(\sim\)”,规定 \(p \sim q\) 当且仅当 \(p\) 和 \(q\) 可以通过一条分段光滑的\textbf{水平路径}连接。

\(\omega\) 在基点 \(p\) 处的\textbf{平行迁移群}(holonomy group)定义为:
\[
\operatorname{Hol}_p(\omega) = \{g \in G \mid p \sim p \cdot g\}.~
\]
基点 \(p\) 处的\textbf{限制平行迁移群}(restricted holonomy group)是子群 \(\operatorname{Hol}_p^0(\omega)\),它由可缩闭合路径 \(\gamma\) 的水平提升所产生。

如果 \(M\) 和 \(P\) 都是连通的,那么平行迁移群对于基点 \(p\) 的依赖仅表现为在 \(G\) 中的共轭。具体来说,如果 \(q\) 是平行迁移群选取的另一个基点,则存在唯一的 \(g \in G\),使得:\(q \sim p \cdot g\)在这个 \(g\) 下,有:
\[
\operatorname{Hol}_q(\omega) = g^{-1}\operatorname{Hol}_p(\omega)g~
\]
特别地:
\[
\operatorname{Hol}_{p \cdot g}(\omega) = g^{-1}\operatorname{Hol}_p(\omega)g~
\]
此外,如果\(p \sim q\),则有:\(\operatorname{Hol}_p(\omega)=\operatorname{Hol}_q(\omega)\)与前面类似,有时在讨论平行迁移群时会省略基点的标注,此时默认该定义只在共轭的意义下成立。

关于平行迁移群和限制平行迁移群的一些重要性质包括:
\begin{itemize}
\item \(\operatorname{Hol}_p^0(\omega)\)是李群\(G\) 的一个\textbf{连通李子群}。
\item \(\operatorname{Hol}_p^0(\omega)\)是\(\operatorname{Hol}_p(\omega)\)的\textbf{单位连通分量}。
\item 存在一个从基本群 \(\pi_1\) 到商群 \(\operatorname{Hol}_p(\omega)/\operatorname{Hol}_p^0(\omega)\) 的\textbf{自然的满群同态}:\(\pi_1 \to \operatorname{Hol}_p(\omega)/\operatorname{Hol}_p^0(\omega)\)
\item 如果 \(M\) 是\textbf{单连通}的,那么:\(\operatorname{Hol}_p(\omega) = \operatorname{Hol}_p^0(\omega)\)
\item \(\omega\) 是\textbf{平坦的}(即曲率为零)当且仅当:\(\operatorname{Hol}_p^0(\omega)\)是\textbf{平凡子群}(只包含单位元)。
\end{itemize}
\subsubsection{平行迁移丛}  
设\(M\)是连通的、可度量紧化的光滑流形,\(P\)是一个主\(G\)-丛,并带有如前所述的联络\(\omega\)。取\(p \in P\)为主丛中的任意一点,定义\(H(p)\)为可以通过\textbf{水平曲线}连接到\(p\)的所有点的集合。可以证明,\(H(p)\)配上自然的投影映射,是一个以\(\operatorname{Hol}_p(\omega)\)为结构群的主丛。这个主丛称为联络\(\omega\)的\textbf{平行迁移丛}(holonomy bundle)(通过点\(p\))。联络\(\omega\)可以限制为\(H(p)\)上的一个联络,因为平行移动映射保持\(H(p)\)内部的点。因此,\(H(p)\)是该联络的一个\textbf{约化子丛}(reduced bundle)。此外,由于不存在比\(H(p)\)更小的子丛同时能被平行移动保持,因此\(H(p)\)是这样的最小约化子丛。

与平行迁移群一样,平行迁移丛在环境主丛\(P\)内也会\textbf{协变变换}。具体来说,如果\(q \in P\)是为平行迁移选择的另一个基点,则存在唯一的\(g \in G\),使得 \(q \sim p \cdot g\)(因为根据假设,\(M\) 是路径连通的)。因此,\(H(q) = H(p) \cdot g\)。由此可得,对于不同基点选择所对应的平行迁移丛,它们上的诱导联络是相容的:它们的平行迁移映射将精确地相差同一个元素\(g\)。
\subsubsection{单调群}  
平行迁移丛 \(H(p)\) 是一个主丛,结构群为 \(\operatorname{Hol}_p(\omega)\),因此也承载了限制平行迁移群 \(\operatorname{Hol}_p^0(\omega)\) 的作用(这是完整平行迁移群的正规子群)。离散群 \(\operatorname{Hol}_p(\omega)/\operatorname{Hol}_p^0(\omega)\) 被称为该联络的\textbf{单调群}(monodromy group);它作用于商丛 \(H(p)/\operatorname{Hol}_p^0(\omega)\)。  

存在一个满射同态映射 \(\varphi : \pi_1 \to \operatorname{Hol}_p(\omega)/\operatorname{Hol}_p^0(\omega)\),使得 \(\varphi(\pi_1(M))\) 作用于商丛 \(H(p)/\operatorname{Hol}_p^0(\omega)\)。  

这个基本群的作用是基本群的单调表示。
\subsubsection{局部和平行迁移的微分形式} 
如果 \(\pi: P \to M\) 是一个主丛,并且 \(\omega\) 是在 \(P\) 上的联络,那么 \(\omega\) 的平行迁移可以限制在 \(M\) 上某个开子集的纤维上。事实上,如果 \(U\) 是 \(M\) 上的一个连通开子集,那么 \(\omega\) 限制到 \(U\) 上将给出在丛 \(\pi^{-1}(U)\) 上的联络。这个丛的平行迁移(分别是限制平行迁移)将被记作:\(\operatorname{Hol}_p(\omega, U)\quad (\text{平行迁移}),\quad \operatorname{Hol}_p^0(\omega, U) \quad (\text{限制平行迁移}),\)对于每个 \(p\),使得 \(\pi(p) \in U\)。

如果 \(U \subset V\) 是两个包含 \(\pi(p)\) 的开集,则有一个显然的包含关系:
\[
\operatorname{Hol}_p^0(\omega, U) \subset \operatorname{Hol}_p^0(\omega, V).~
\]
在点 \(p\) 处的局部平行迁移群定义为:
\[
\operatorname{Hol}^*(\omega) = \bigcap_{k=1}^{\infty} \operatorname{Hol}_p^0(\omega, U_k)~
\]
对于任何一族嵌套的连通开集 \(U_k\),满足:\(\bigcap_k U_k = \pi(p)\).

局部平行迁移群具有以下性质:
\begin{enumerate}
\item 它是受限平行迁移群 \(\operatorname{Hol}_p^0(\omega)\) 的一个连通李子群。
   
\item 每个点 \(p\) 都有一个邻域 \(V\),使得\(\operatorname{Hol}_p^*(\omega) = \operatorname{Hol}_p^0(\omega, V)\).特别地,局部平行迁移群仅依赖于点 \(p\),而不依赖于定义它所使用的嵌套开集序列 \(U_k\)。
\item 局部平行迁移是关于结构群 \(G\) 的元素的平移等变的;即,\(\operatorname{Hol}_{pg}^*(\omega) = \operatorname{Ad}(g^{-1}) \operatorname{Hol}_p^*(\omega)\)对于所有 \(g \in G\)。(注意,依据性质1,局部平行迁移群是 \(G\) 的一个连通李子群,因此伴随作用是良定义的。)
\end{enumerate}
局部平行迁移群作为一个全局对象的行为并不好。特别地,它的维数可能不是恒定的。然而,以下定理是成立的:

如果局部平行迁移群的维数是恒定的,那么局部平行迁移群和受限平行迁移群相等:\(\operatorname{Hol}_p^*(\omega) = \operatorname{Hol}_p^0(\omega)\).
\subsection{安布罗斯–辛格定理}  
安布罗斯–辛格定理(由沃伦·安布罗斯和伊萨多尔·M·辛格(1953年)提出)将主束中的连接的平行迁移与连接的曲率形式联系起来。为了使这个定理显得合理,可以考虑一个熟悉的情况,即仿射连接(或切丛中的连接——例如,勒维-奇维塔连接)。当沿着一个无穷小的平行四边形行进时,曲率就会出现。

详细来说,如果 σ: [0, 1] × [0, 1] → M 是一个在 M 中由一对变量 x 和 y 参数化的曲面,那么一个向量 V 可以沿着 σ 的边界进行平行迁移:首先沿着 (x, 0) 方向,然后沿着 (1, y),接着沿着 (x, 1) 方向以负方向行进,最后沿着 (0, y) 返回到原点。这是一个平行迁移回路的特例:向量 V 受对应于 σ 边界提升的平行迁移群元素的作用。当平行四边形收缩到零时,曲率显式地出现,通过沿 [0, x] × [0, y] 区域遍历更小的平行四边形的边界。这对应于在 x = y = 0 时对平行迁移映射求导:
\[
\frac{D}{dx} \frac{D}{dy} V - \frac{D}{dy} \frac{D}{dx} V = R\left(\frac{\partial \sigma}{\partial x}, \frac{\partial \sigma}{\partial y}\right) V~
\]
其中,\(R\)是曲率张量。[3] 所以,粗略地说,曲率给出了封闭回路(即无穷小平行四边形)上的无穷小平行迁移。更正式地,曲率是平行迁移群在单位元处作用的微分。换句话说,\(R(X, Y)\) 是平行迁移群\(\operatorname{Hol}_p(\omega)\) 的李代数的一个元素。

一般地,考虑一个主丛 \( P \to M \) 上的联络的平行迁移,其中主丛的结构群为 \( G \)。令 \( g \) 表示 \( G \) 的李代数,联络的曲率形式是一个值域为 \( g \) 的 2-形式 \( \Omega \) 定义在 \( P \) 上。Ambrose–Singer 定理指出:[4]

平行迁移群 \( \operatorname{Hol}_p(\omega) \) 的李代数由所有形如 \( \Omega_q(X, Y) \) 的 \( g \) 元素所张成,其中 \( q \) 取遍所有能够通过水平曲线与 \( p \) 连通的点(即 \( q \sim p \)),而 \( X \) 和 \( Y \) 是 \( q \) 处的水平切向量。

或者,该定理可以用平行迁移丛的语言重新表述:[5]

平行迁移群 \( \operatorname{Hol}_p(\omega) \) 的李代数是李代数 \( g \) 的子空间,由形如 \( \Omega_q(X, Y) \) 的元素所张成,其中 \( q \in H(p) \),而 \( X \) 和 \( Y \) 是 \( q \) 处的水平向量。
\subsection{黎曼平行迁移}
黎曼流形的平行迁移是指黎曼流形 \((M, g)\) 上切丛的 Levi-Civita 连接的平行迁移群。一个“典型的”\(n\)-维黎曼流形具有 \(O(n)\) 的平行迁移群,若流形是可定向的,则平行迁移群为 \(SO(n)\)。如果流形的平行迁移群是 \(O(n)\) 或 \(SO(n)\) 的真子群,则该流形具有特殊的性质。

关于黎曼平行迁移的早期基本结果之一是 Borel 和 Lichnerowicz(1952)的定理,该定理断言限制的平行迁移群是 \(O(n)\) 的闭李子群。特别地,它是紧致的。

设 \( x \in M \) 为任意一点。则平行迁移群 \( \text{Hol}(M) \) 在切空间 \( T_xM \) 上作用。这个作用可以是不可约的(作为群表示),也可以是可约的,即存在一个将 \( T_xM \) 分解为正交子空间 \( T_xM = T'_xM \oplus T''_xM \) 的分裂,其中每个子空间在 \( \text{Hol}(M) \) 的作用下是不变的。在后一种情况下,称 \( M \) 为可约的。