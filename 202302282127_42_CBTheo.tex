% Cantor-Bernstein 定理
% keys 对等性|康托尔-伯恩斯坦定理
\pentry{映射\upref{map}}
康托尔-伯恩斯坦(Cantor-Bernstein)定理是集论中的一个基本定理,它给出了一种两个集对等的判据。
\begin{definition}{集的对等性}
如果集 $M$ 与 $N$ 的元素之间可建立一一对应,则称 $M$ 与 $N$ 是\textbf{对等的},记作 $M\sim N$。
\end{definition}
\begin{theorem}{Cantor-Bernstein定理}
设 $A,B$ 是两个任意的集。如果存在集 $A$ 到集 $B$ 的子集 $B_1$ 上的一一映射 $f$,及集 $B$ 到集 $A$ 的子集 $A_1$ 上的一一映射 $g$,那么 $A$ 与 $B$ 对等。
\end{theorem}
\textbf{证明:}不失一般性,可以认为 $A\cap B=\emptyset$,因为不然的话,相交的部分可以建立自身的一一对应关系,这时又回到只需讨论不相交部分的情形。

设 $x$ 是 $A$ 中任一元素。令 $x=x_0$,我们这样定义序列 $\{x_n\}$:设 $x_n$ 已经确定,于是当 $n$ 为偶数时,取满足 $g(x_{n+1})=x_n$ 的 $B$ 中元素作为 $x_{n+1}$(如果它存在的话);而当 $n$ 为奇数时,取满足 $f(x_{n+1})=x_n$ 的 $A$ 中元素作为 $x_{n+1}$(若存在的话)。于是,可能出现以下两种情况:
\begin{enumerate}
\item 对某一 $n$ ,满足上述条件的元素 $x_{n+1}$ 不存在。这样的数 $n$ 叫作元素 $x$ 的阶;
\item 序列 $\{x_n\}$ 是无限的。这时 $x$ 称为无限阶的元素。
\end{enumerate}
显然,在序列 $\{x_n\}$ 中,偶数项的$x_n\in A$,奇数项的 $x_n\in B$。现在集 $A$ 被分成三类:由偶数阶元素组成的集 $A_E$,奇数阶元素组成的集 $A_O$,及一切无穷阶元素组成的集 $A_I$。对于集 $B$ 也由类似的方法分成三类集。

显然,若 $x=x_0\in A$ 是偶数(奇数)阶的,那么 $x_1\in B$ 就是奇数(偶数)阶的,因为由 $x=x_0\in A$ 定义的序列 $\{x_n\}$ 与 $x_1=y_0$ 定义的序列 $\{y_n\}$ 仅相差一个 $x_0$,于是这两序列的元素个数必定一偶一奇。而 $g(x_1)=x_0$,所以 $g$ 将 $B_E$ 映射到 $A_O$ 及将 $B_I$ 的元素映到 $A_I$ 上;同理, $f^{-1}$ 将 $B_O$ 映到 $A_E$ 上。于是与 $g$ 在 $B_E\cup B_I$ 上重合及与 $f^{-1}$ 在 $B_O$ 上重合的一一映射 $\psi$ 是全 $B$ 到全 $A$ 的一一映射。 

\textbf{证毕!}