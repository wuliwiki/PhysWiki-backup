% 用 Filebrowser 搭建个人网盘

\begin{issues}
\issueDraft
\end{issues}

\subsection{特性}
\begin{itemize}
\item 在线视频播放
\item 上传多个文件/文件夹
\item 创建用户并管理用户权限
\item 使用 bash
\item 分享文件/文件夹链接(文件夹内容实时更新), 无需登录即可下载
\item 适配手机浏览器
\end{itemize}

\subsection{使用说明}
docker 的详细教程参考 Docker 笔记\upref{Docker}. 首先克隆 docker 镜像: \verb|docker pull filebrowser/filebrowser|
启动的格式如
\begin{lstlisting}[language=bash]
docker run --name filebrowser -d --restart always \
   -v 本地网盘文件目录:/srv \
   -v /root/filebrowserconfig.json:/etc/config.json \
   -v /root/filebrowser/database.db:/etc/database.db \
   -p 端口号:80 filebrowser/filebrowser
\end{lstlisting}
具体例子如
\begin{lstlisting}[language=bash]
docker run --name filebrowser -d --restart always \
   -v /mnt/drive/filebrowser/root:/srv \
   -v /mnt/drive/filebrowser/filebrowserconfig.json:/etc/config.json \
   -v /mnt/drive/filebrowser/database.db:/etc/database.db \
   -p 5000:80 filebrowser/filebrowser
\end{lstlisting}

\begin{itemize}
\item 设置好后, 浏览器访问 \verb`网站域名:端口` 就可以看到初始页面了, 用户名和密码都是 \verb`admin`
\item 也可以直接用 sftp 把文件同步到 \verb`本地网盘文件目录` 可能会更稳定和方便.
\item \verb`本地网盘文件目录` 不支持 symlink 和 mount
\end{itemize}
