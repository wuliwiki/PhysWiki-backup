% 平行移动
% keys 平行移动|仿射联络

\pentry{曲线坐标系下的张量坐标变换(仿射空间)\upref{TinCur}}
在仿射空间 $(\mathbb A,V)$ 中,任一点都可看成 $V$ 与 $\mathbb A$ 之间的双射(\autoref{AfSp_eq4}~\upref{AfSp}).这就是相当于在仿射空间中的任一点可作出任一给定的矢量.如此便有这样的问题出现:在区域 $\Omega\in\mathbb A$ 取曲线坐标 $x^i$ 时,如何在任一点作出一已知矢量.设已知矢量 $v_0$ 在点 $M_0$ 的坐标为 $v_0^i$,现要求从另一点 $M_1$ 作出这个矢量.这就是说,我们必须确定:如何变更 $v_0^i$ ,以便在 $M_1$ 的局部标架中,使它们能定出以前的矢量 $v_0$.

如果将 $v_0$ 不经过任何中间点而从 $M_0$ 直接移到 $M_1$ 上,则问题的解答就失去了意义.值得注意的是矢量 $v_0$ 沿任一曲线 $\overset{\Huge\frown}{M_0M_1}$ 的连续移动,且坐标 $v_0^i$ 在路径上每一无限小段上连续变化的过程,即\textbf{平行移动}.
\subsection{平行移动公式}
设路径 $\overset{\Huge\frown}{M_0M_1}$ 的参数方程为
\begin{equation}
x^i=x^i(t),\quad t_0\leq t\leq t_1
\end{equation}
式中 $x^i(t)$ 是连续可微的函数.于是曲线上一点的向径 $x$ 是 $t$ 的函数
\begin{equation}
x=x(t)
\end{equation}
$v_0$ 在路径上平行移动就是要在路径上每一点 $M(t)$ 作一常矢量 $v_0$.由于局部标架(\autoref{CFinAf_def1}~\upref{CFinAf})随点而变化,因而矢量 $v_0$ 的坐标 $v^i$ 也随点的位置而变,所以
\begin{equation}
v^i=v^i(t)
\end{equation}
因为 $x^i(t)$ 是连续可微的,因此局部标架的矢量 $\partial_i x(x^1,\cdots,x^n)$ 以及 $v^i$ 沿此路径是 $t$ 的连续可微函数.

在点 $M(t)$ 处,按该处局部标架展开矢量 $v_0$:
\begin{equation}
v_0=v^i(t)\partial_i x(x^1,\cdots,x^n)
\end{equation}
注意,这里 $x^i$ 是依赖于路径参数 $t$ 的.关于 $t$ 逐项微分,因为 $v_0$ 是给定的,即是常矢量,得
\begin{equation}\label{PTinAS_eq1}
0=\dd v^i\partial_i x+v^i\dd \partial_i x
\end{equation}
由全微分公式得
\begin{equation}\label{PTinAS_eq2}
\dd \partial_i x(x^1,\cdots,x^n)=\partial_{ij}x\dd x^j
\end{equation}
其中
\begin{equation}
\partial_{ij}x=\pdv{x(x^1,\cdots,x^n)}{x^i\partial x^j}
\end{equation}
是在 $M(t)$ 处确定的矢量,按该点的局部标架展开,其系数记为 $\Gamma^k_{ij}$,于是
\begin{equation}\label{PTinAS_eq3}
\partial_{ij}x=\Gamma^k_{ij}\partial_k x
\end{equation}
由 $\partial_{ij} x=\partial_{ji} x$,且按标架矢量展开的唯一性,得
\begin{equation}
\Gamma^k_{ij}=\Gamma^k_{ji}
\end{equation}
 $\Gamma^k_{ij}$ 显然依赖于点 $M(t)$ ,即
 \begin{equation}
 \Gamma^k_{ij}(M)=\Gamma^k_{ij}(x^1,\cdots,x^n)
 \end{equation}
 
联立\autoref{PTinAS_eq1} ,\autoref{PTinAS_eq2} 和\autoref{PTinAS_eq3} ,就有
\begin{equation}
0=\dd v^i\partial_i x+v^i\Gamma^k_{ij}\partial_k x\dd x^j
\end{equation}
由于标架矢量 $\partial_i x$ 的线性无关性,故
\begin{equation}
\dd v^i+v^j\Gamma^i_{jk}\dd x^k=0
\end{equation}
上式或记作
\begin{equation}
\dd v^i=-v^j\Gamma^i_{jk}\dd x^k
\end{equation}
