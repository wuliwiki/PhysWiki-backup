% 二次多项式与二次型

\pentry{二次型\upref{QuaFor}, 正定矩阵\upref{DefMat}, 线性方程组解的结构\upref{LinEq}}

所有的 $N$ 元二次多项式都可以表示为
\begin{equation}
P(x_1,\dots,x_N) = \sum_{i=1}^N\sum_{j=1}^N a_{i,j}x_i x_j
\end{equation}
其中 $a_{i,j}$ 是对称矩阵 $\mat A$ 的矩阵元. 事实上这就是一个二次型\upref{QuaFor}的矩阵表示($\bvec x\Tr \mat A \bvec x$). 注意 $x_i x_j = x_j x_i$, 所以规定 $\mat A$ 为对称矩阵并不影响一般性, 反而可以简化运算.

\subsubsection{极值}
先求驻点:
\begin{equation}
\pdv{x_i}P(x_1,\dots,x_N) = 2\sum_j a_{ij} x_j = 0 \qquad (i = 1,\dots,N)
\end{equation}
所以这相当于解齐次线性方程组
\begin{equation}
\mat A \bvec x = \bvec 0
\end{equation}
即可. 可见它的解集要么是 $\bvec x = \bvec 0$ 一个点, 要么是一个子空间(零空间)\upref{LinEq}.

可以判断, 若矩阵 $\mat A$ 是正定的, 那么马上得到只有 $\bvec x = \bvec 0$ 一个点是方程组的解, 且该点是全局最小值, 也是唯一一个极值. 若 $\mat A$ 是负定的也同理.
