% 狄拉克场的量子化
% 狄拉克场|费米子|量子化

现在我们来构造自由狄拉克场的量子理论.我们从如下的拉式量出发
\begin{equation}
\mathcal L = \bar \psi (i \partial\!\!\!/ - m)\psi = \bar \psi (i \gamma^\mu \partial_\mu - m)\psi~,
\end{equation}
$\psi$的共轭动量是$i\psi^\dagger$,因此哈密顿量为
\begin{equation}
H = \int d^3 x \bar \psi (-i \boldsymbol \gamma \cdot \boldsymbol \nabla + m)\psi = \int d^3 x \psi^\dagger [-i\gamma^0\boldsymbol\gamma \cdot \boldsymbol\nabla + \gamma^0]\psi ~.
\end{equation}
定义$\alpha = \gamma^0 \boldsymbol\gamma$, $\beta = \gamma^0$, 你们可以认出括号内的量就是单粒子量子力学的狄拉克哈密顿量
\begin{equation}
h_D = - i \boldsymbol\alpha \cdot \boldsymbol \nabla + m \beta ~.
\end{equation} 

\subsection{错误的量子化狄拉克场的方法}
我们首先尝试下面的量子化狄拉克场的办法
\begin{equation}
[\psi_a(\mathbf x),\psi_b^\dagger(\mathbf y)] = \delta^{(3)}(\mathbf x - \mathbf y)\delta_{ab}~. {\rm (equal times)}
\end{equation}

\begin{equation}
[i\gamma^0\partial_0+i\boldsymbol \gamma \cdot \nabla - m ] u^s (p) e^{-ip\cdot x} = 0~.
\end{equation}








