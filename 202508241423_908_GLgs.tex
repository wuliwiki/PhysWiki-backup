% 格林公式(综述)
% license CCBYSA3
% type Wiki

本文根据 CC-BY-SA 协议转载翻译自维基百科\href{https://en.wikipedia.org/wiki/Green\%27s_theorem}{相关文章}。

在向量分析中,格林公式把围绕一条简单闭合曲线 $C$ 的曲线积分与该曲线所围平面区域 $D$(即 $\mathbb{R}^2$ 中的曲面)的二重积分联系起来。它是斯托克斯定理在二维空间($\mathbb{R}^2$)中的特例。在一维情形下,它等价于微积分基本定理;在三维情形下,它则等价于散度定理。
\subsection{定理}
设 $C$ 是平面上一条按正向(逆时针)取向、分段光滑的简单闭合曲线,$D$ 是 $C$ 所围成的区域。如果 $L$ 和 $M$ 是定义在包含 $D$ 的某个开区域上的函数,且它们在该区域内具有连续偏导数,则有
$$
\oint_{C} (L\,dx + M\,dy) 
= 
\iint_{D} 
\left( 
\frac{\partial M}{\partial x} 
- 
\frac{\partial L}{\partial y} 
\right) dA~
$$
其中,曲线 $C$ 上的积分路径方向为逆时针。
\subsection{应用}
在物理学中,格林公式有许多应用。例如,在处理二维流体积分问题时,可以用它说明:一个区域内流体的总外流量等于该区域边界曲线上的总外流量。在平面几何中,尤其是在面积测量中,格林公式还能用于仅通过对边界积分来求解平面图形的面积和形心位置。
\subsection{当 $D$ 是单连通区域时的证明}
以下是对简化区域 $D$ 的一半定理的证明。这里 $D$ 是I型区域,其边界曲线 $C_1$ 和 $C_3$ 由垂直线段(长度可能为零)连接。对于II型区域(边界曲线 $C_2$ 和 $C_4$ 由水平线段连接)的另一半定理,也存在类似的证明。将这两部分结合起来,就可以证明适用于III 型区域(既是 I 型又是 II 型区域)的格林公式。通过将一般区域 $D$ 分解为一组 III 型区域,还可以将结论推广到更一般的情形。

如果能够证明以下两式:
$$
\oint_{C} L\,dx = \iint_{D} \left( -\frac{\partial L}{\partial y} \right) dA
\tag{1}~
$$
以及
$$
\oint_{C} M\,dy = \iint_{D} \left( \frac{\partial M}{\partial x} \right) dA
\tag{2}~
$$
那么对于区域 $D$ 格林公式就立即成立。对于I 型区域,式 (1) 容易证明;对于 II 型区域,式 (2) 也可以类似地证明。由此,格林公式便适用于III 型区域。

假设区域 $D$ 是I 型区域,因此它可以表示为(如右图所示):
$$
D = \{(x, y) \mid a \leq x \leq b, \; g_1(x) \leq y \leq g_2(x)\},~
$$

其中 $g_1$ 和 $g_2$ 是区间 $[a, b]$ 上的连续函数。计算式 (1) 中的二重积分:
$$
\begin{aligned}
\iint_D \frac{\partial L}{\partial y} \, dA
&= \int_a^b \int_{g_1(x)}^{g_2(x)} \frac{\partial L}{\partial y}(x, y) \, dy \, dx \\
&= \int_a^b \big[ L(x, g_2(x)) - L(x, g_1(x)) \big] \, dx.
\end{aligned}
\tag{3}~
$$
计算式 (1) 中的曲线积分:曲线 $C$ 可以分解为四段:$C_1, C_2, C_3, C_4$。
沿 $C_1$,参数方程为:$x = x,\; y = g_1(x),\; a \leq x \leq b$.因此:
  $$
  \int_{C_1} L(x, y) \, dx
  = \int_a^b L(x, g_1(x)) \, dx~
  $$
沿$C_3$,参数方程为:$x = x,\; y = g_2(x),\; a \leq x \leq b$.因为方向相反,积分为:
  $$
  \int_{C_3} L(x, y) \, dx
  = \int_b^a L(x, y) \, dx
  = -\int_a^b L(x, g_2(x)) \, dx~
  $$
在 $C_3$ 上积分需要取负号,是因为曲线从 $b$ 到 $a$ 的方向与正向(逆时针方向)相反。
在 $C_2$ 和 $C_4$ 上,$x$ 保持不变,因此:

$$
\int_{C_4} L(x, y)\, dx = \int_{C_2} L(x, y)\, dx = 0~
$$

因此:

$$
\begin{aligned}
\oint_{C} L\, dx
&= \int_{C_1} L(x, y)\, dx
 + \int_{C_2} L(x, y)\, dx
 + \int_{C_3} L(x, y)\, dx
 + \int_{C_4} L(x, y)\, dx \\
&= \int_a^b L(x, g_1(x))\, dx
 - \int_a^b L(x, g_2(x))\, dx。
\end{aligned}
\tag{4}~
$

将式 (3) 与式 (4) 结合,就得到了适用于 **I 型区域**的式 (1)。用相同的思路处理相应端点,可以得到适用于 **II 型区域**的式 (2)。将这两部分结合,就得到了适用于 **III 型区域**的最终结果。
