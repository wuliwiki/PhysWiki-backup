% 深度卷积生成对抗网络
% 卷积 生成对抗

\pentry{卷积神经网络\upref{ConvNe},激活函数\upref{ActFun}}

\textbf{深度卷积生成对抗网络}(Deep Convolutional Generative Adversarial Networks, DCGAN)是生成对抗网络的一种基本实现,最初由文献\cite{DCGAN}提出。顾名思义,该网络模型的生成器采用的是深层卷积网络,即由多层卷积层堆叠而成。

建立稳定的深度卷积生成对抗网络的要点是\cite{DCGAN}:
\begin{enumerate}
\item 在判别器中,用跳跃卷积代替池化层;在生成器中,用分数步长跳跃卷积代替池化层。
\item 生成器和判别器中均采用批归一化层。
\item 移除全连接隐含层,从而构建更深层的网络结构。
\item 在生成器中,除了输出层以外,均使用线性阈值单元($ReLU$)作为激活函数。输出层使用双曲正切函数($Tanh$)作为激活函数。
\item 在判别器的每一层中使用倾斜线性阈值单元($Leaky ReLU$)作为激活函数。
\end{enumerate}

图$1$是深度卷积生成对抗网络生成器结构的一个典型例子。在该模型中,输入的是维度为$100$的随机噪音,服从均匀分布。数据经过一系列分数步长跳跃卷积(转置卷积,又称为反卷积),最终转换为$64 \times64$像素的图像。该网络结构中没有全连接层和池化层。
\begin{figure}[ht]
\centering
\includegraphics[width=14.25cm]{./figures/8163452ce7e1ea4b.png}
\caption{深度卷积生成对抗网络生成器结构示意图 \cite{DCGAN}} \label{fig_DCGAN_1}
\end{figure}

图$2$表示的是用深度卷积生成对抗网络生成的一些卧室的图片。这些图片是模型经过$5$轮训练之后生成的结果 \cite{DCGAN}。
\begin{figure}[ht]
\centering
\includegraphics[width=14.25cm]{./figures/d86559a51730cc6d.png}
\caption{卧室图片生成结果展示} \label{fig_DCGAN_2}
\end{figure}

一个很有趣的事情是,当生成模型训练好时,内隐空间中的表示矢量经过一定运算之后,再送入模型,能够产生有意义的输出结果,而这些新的结果还在原来的数据空间中。该事实表明模型学习到的内隐表示空间是具有线性结构的 \cite{DCGAN}。图$3$展示的是一个典型的例子。每一列的三个图片所对应的表示矢量求平均值,生成最下面一行的三个图片,然后再对这三个图片的表示矢量做运算:vector(”smiling woman”) - vector(”neutral woman”) + vector(”neutral man”) = vector("smiling man").

\begin{figure}[ht]
\centering
\includegraphics[width=14.25cm]{./figures/ba047b5936016a21.png}
\caption{表示矢量的运算} \label{fig_DCGAN_3}
\end{figure}




参考文献:
\begin{enumerate}
\item A. Radford, L. Metz, and S. Chintala, “Unsupervised representation learning with deep convolutional generative adversarial networks,” arXiv preprint arXiv:1511.06434, 2015.
\end{enumerate}