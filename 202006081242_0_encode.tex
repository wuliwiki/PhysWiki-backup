% 字符编码

这里要讨论的一个基本的问题是, 计算机如何保存字符? 例如我们打开一个 txt 文件, 在里面写一篇文章, 保存的时候这些数据以什么形式储存在硬盘上? 答案是:某个整数. 例如在著名的 ascii 编码中, 每个字符被储存为一个 0 到 127 的整数, 这些包括大写和小写字母, 数字, 常见标点, 以及一些格式上的符号如空格, 换行符, 制表符等. ascii 编码支持通常的英语写作, 但不支持其他语言

当我们在 windows 中编写
