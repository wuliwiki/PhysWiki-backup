% 状态压缩动态规划
% 动态规划|算法

状态压缩是将一个长度为 $n$ 的布尔数组用一个长度为 $n$ 的二进制数表示的方法。状态压缩动态规划即为将状态用一个二进制数保存起来,从而可以减少空间开销。存储状态一般是用 \verb|int| 类型存储,用位运算进行状态计算。简单来讲就是十进制存储,二进制计算。讲状压 dp 之前首先要学习位运算。

位运算的基本操作:异或 \verb|xor|、与 \verb|and|、或 \verb|or|、左移 \verb|<<|、右移 \verb|>>|。

^ 异或预算,若 $x$ 和 $y$ 双方都为 $1$(TRUE),与操作之后的结果为 $0$,若其中一方为 $0$,则答案为 $1$,所以异或运算也称不进位加法。

\verb|1011 ^ 1100 = 111|。

& 与运算,若 $x$ 和 $y$ 双方都为 $1$,与操作之后的结果才为 $1$,否则为 $0$。

\verb|1011 & 1100 = 1000|。

|(回车下面那个键) 或运算,只要 $x$ 和 $y$ 一方为 $1$ 答案为 $1$,双反都为 $0$ 答案才是 $0$。

\verb|1000 or 1011 = 1011|。

<< 左移运算,把二进制数向左移,高位越界后舍弃,低位补 $0$。\verb|1 << n = 2^n, n << 1 = 2n|。

\verb|1011 << 2 = 101100|。

>> 右移运算,把二进制数向右移,高位以符号位填充,低位越界后舍弃。\verb|n >> 1 = n / 2.0|。

\verb|1011 >> 2 = 10|。

\begin{figure}[ht]
\centering
\includegraphics[width=14cm]{./figures/dp4_1.png}
\caption{状态压缩位运算常用操作} \label{dp4_fig1}
\end{figure}
