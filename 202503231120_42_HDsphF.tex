% 高维球谐函数
% keys 球谐函数|高维
% license Usr
% type Tutor

\pentry{球谐函数\nref{nod_SphHar},高维弯曲空间中的拉普拉斯算符\nref{nod_GLapOp},高维空间球坐标及其度规\nref{nod_nDSM}}{nod_94ea}

在三维的情形,球谐函数是在球坐标下,求解拉普拉斯方程时,通过分离变量得到的。其是拉普拉斯方程角度部分的解。即下面的微分方程
\begin{equation}
\Nabla_\Omega^2 Y+\lambda Y=0.~
\end{equation}
其中 $\Omega$ 代表只含角 $(\theta,\phi)$ 的部分,$\lambda$ 是常数,且
\begin{equation}
\Nabla^2_\Omega=\pdv[2]{}{\theta}+\cot\theta\pdv{}{\theta}+\frac{1}{\sin^2\theta}\pdv[2]{}{\varphi}.~
\end{equation}
本词条将仿照三维情形推导球谐函数的方法,推导高维弯曲空间中的球谐函数。

\subsection{高维拉普拉斯方程的角坐标部分}
对一般高维空间的拉普拉斯方程,可由对应空间的\enref{拉普拉斯算符}{GLapOp}获得,即从 $\Delta u=0$ 一般高维空间的拉普拉斯方程如下:
\begin{equation}
\begin{aligned}

\end{aligned}
\frac{1}{\sqrt{ \left\lvert g \right\rvert }} \frac{\partial }{\partial x^i} \left(\sqrt{ \left\lvert g \right\rvert }g^{ij} \frac{\partial u}{\partial x^j} \right)=0 .~
\end{equation}



\subsection{将球谐函数分离变量}









