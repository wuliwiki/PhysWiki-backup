% 椭圆(高中)
% keys 极坐标系|直角坐标系|圆锥曲线|椭圆
% license Xiao
% type Tutor

\begin{issues}
\issueDraft
\end{issues}

\pentry{解析几何\nref{nod_JXJH},圆\nref{nod_HsCirc}}{nod_32e0}

人们常说地球绕着太阳转。听到“绕着转”这个说法时,往往会自然地联想到一个完美的圆形轨道。但实际上,真正沿着正圆轨道运动的行星极为罕见。几乎所有行星围绕恒星的运动轨道虽然依旧是一圈,但这其实是一个有些“压扁”的圆。这种被“压扁”的圆在日常生活中非常常见。比如,一个圆形水杯的杯口,在不是正面观察时,也会呈现出这种变形的轮廓。这种形状被称为“椭圆”。在行星绕太阳运动的轨道里,也有他的身影。

如果你在教室里轻声说话,声音会在墙壁间反弹,有时会被某个角落的人清晰听见。有些老式的博物馆里甚至设计了“回声墙”:一个人站在博物馆里的某个点上轻声说话,哪怕另一个和他中间隔着一大块弧形的墙,但只要站在对面的某个点,他也能清楚听到对面的声音。人们研究这类现象时发现,有一种神奇的曲线,它的一个特点就是:从某个点发出的声音(或光线),经过反射后,总会精确地传到另一个特定的点。巧的是这条曲线,也是椭圆。

椭圆是圆的“兄弟”,但比圆更自由,也更丰富。接下来,本篇的内容将从椭圆的几何定义、椭圆的标准方程和椭圆的一些几何参数进行介绍,一步步认识这个优雅又实用的图形。

\subsection{椭圆的几何定义}

椭圆的存在如此普遍,以至于早在在公元前3世纪的古希腊时期,数学家阿波罗尼奥斯(Apollonius of Perga)就对它进行了研究。

从圆的定义开始,如果想要引申圆的定义,可以这样看,假设圆心是两个重合的点,然后圆上每个点到这两个点的距离相等,都是定值,现在如果把这两个点移开,那么显然如果在平面上所有到这两个点距离相等的点就构成了他们连线的垂直平分线。那么,如果加上一些不同的设定,或许会得到不同的效果。比如,圆上点到的这两个点到距离是定值,这启发我们或许可以是点到这两个点的距离之和是定值。

下面给出古希腊时代从几何视角给出的椭圆定义,这也常被称作是椭圆的第一定义。

\begin{definition}{椭圆的几何定义}
平面上到两定点的距离之和为有限定值的几何图形,称为\textbf{椭圆}。两个定点称作椭圆的两个\textbf{焦点}。
\end{definition}

可以这样看,当一个椭圆的两个焦点重合时,椭圆就变成了圆。所以可以这样说,圆是椭圆的一个特例。

椭圆是到两个定点(焦点)距离之和为定值的点的集合。这个定值等于椭圆的长轴长(记作 $2a$),即对任意点 $P$,有
$$ PF_1 + PF_2 = 2a ~.$$
\subsection{椭圆的标准方程}

我们已经知道用焦点和准线如何定义\enref{椭圆}{Cone},下面介绍另外三种。 其中 “圆锥截面定义” 揭示了 “圆锥曲线” 一词的由来。

\begin{definition}{椭圆的标准方程}
\begin{equation}\label{eq_Elips3_3}
\frac{x^2}{a^2} + \frac{y^2}{b^2} = 1~.
\end{equation}
\end{definition}

从椭圆的极坐标公式难以看出椭圆的对称性, 另一种定义椭圆的方法是直接在直角坐标系中给出椭圆的方程

\subsubsection{参数介绍}

**长轴(major axis)**长度为 $2a$。
	•	**短轴(minor axis)**长度为 $2b$。
	•	焦点之间的距离为 $2c$,其中
$$ c^2 = a^2 - b^2~. $$
所以 $c < a$,表示焦点在中心两侧。

	•	若 $a > b$,焦点在 $x$ 轴上;
	•	若 $b > a$,焦点在 $y$ 轴上;
	•	若 $a = b$,就是圆。


长轴

短轴
焦距

\subsection{椭圆的参数方程}
表示为参数方程
\begin{equation}\label{eq_Elips3_1}
\leftgroup{
&x(t) = a\cos t\\
&y(t) = b\sin t
},\quad t \in [0, 2\pi) ~.
\end{equation}
\subsection{椭圆的性质}

椭圆上点与焦点连线的斜率之积为定值。

关于 $x$ 轴、$y$ 轴和原点对称。是中心对称图形,中心即椭圆中心。


一条从一个焦点出发射到椭圆上的射线,在椭圆上反射后,会朝向另一个焦点。这个性质被用于椭圆形镜面(如椭圆房顶)。椭圆任一点的切线,使得该点到两个焦点的连线之间的夹角相等(即入射角=反射角)。

椭圆面积为:
\begin{equation}
S = \pi a b~.
\end{equation}

可看作是“拉扁”的圆。



\addTODO{下面移走}


这相当于把一个单位圆(方程 $x^2 + y^2 = 1$)在 $x$ 轴和 $y$ 轴分别拉长了 $a$ 倍和 $b$ 倍。 我们这里用焦点和准线的定义来推导出上式, 以证明它们等价。 我们不妨先以一个焦点为原点定义直角坐标系, 且令 $x$ 轴指向另一个焦点, 则有
\begin{equation}
r = \sqrt{x^2 + y^2}~, \qquad \cos\theta = \frac{x}{\sqrt{x^2 + y^2}}~.
\end{equation}
代入椭圆的极坐标方程\autoref{eq_Cone_5}  得
\begin{equation}
\sqrt{x^2 + y^2} = p + ex~.
\end{equation}
两边平方并整理得
\begin{equation}\label{eq_Elips3_2}
(1 - e^2) \qty( x - \frac{ep}{1 - e^2} )^2 + y^2 = \frac{p^2}{1 - e^2}~.
\end{equation}
由此可见,如果我们把椭圆左移 $ep/(1 - e^2)$,椭圆将具有\autoref{eq_Elips3_3} 的形式。 其中 $a$ 为\textbf{半长轴}, $b$ 为\textbf{半短轴}。这就是椭圆的第二种定义, 即把单位圆沿两个垂直方向分别均匀拉长 $a$ 和 $b$。 所以也可以

离心率(eccentricity)
	•	表示“扁平程度”:
$$ e = \frac{c}{a} = \sqrt{1 - \frac{b^2}{a^2}} \in [0, 1) ~.$$
	•	$e = 0$ 是圆,越接近 1 越扁。

下面来看系数的关系。首先定义椭圆的焦距为焦点到椭圆中心的距离(即以上左移的距离)为
\begin{equation}\label{eq_Elips3_5}
c = \frac{ep}{1 - e^2}~.
\end{equation}
\autoref{eq_Elips3_2} 和\autoref{eq_Elips3_3} 对比系数得
\begin{equation}\label{eq_Elips3_6}
a = \frac{p}{1 - e^2}~, \qquad b = \frac{p}{\sqrt {1 - e^2} }~.
\end{equation}
以上两式可以将椭圆的极坐标方程转为直角坐标方程。 另外易证
\begin{equation}\label{eq_Elips3_7}
a^2 = b^2 + c^2~.
\end{equation}
若要从直角坐标方程变回极坐标方程, 将\autoref{eq_Elips3_5} \autoref{eq_Elips3_6} 逆转得
\begin{equation}\label{eq_Elips3_8}
e = \frac{c}{a}~,\qquad
p = \frac{b^2}{a}~.
\end{equation}


用圆锥截面定义椭圆。椭圆之所以叫做圆锥曲线, 是因为它们可以由平面截取圆锥面得到, 详见“\enref{圆锥曲线和圆锥}{ConSec}”。



由直角坐标方程可知对称性,可在椭圆的两边做两条准线,令椭圆上任意一点到两焦点的距离分别为 $r_1$ 和 $r_2$,到两准线的距离分别为 $d_1$ 和 $d_2$,则有
\begin{equation}
e = \frac{r_1}{d_1} = \frac{r_2}{d_2} = \frac{r_1 + r_2}{d_1 + d_2}~,
\end{equation}
所以
\begin{equation}\label{eq_Elips3_9}
r_1 + r_2 = e(d_1+d_2) = 2e(c + h) = 2\frac{c}{a} \qty( c + \frac{b^2}{c} ) = 2a~,
\end{equation}
证毕。