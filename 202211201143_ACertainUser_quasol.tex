% 二次方程求根公式

\subsection{求根公式}
解决二次方程最常规的方法是运用求根公式.
对于二次方程$$ax^2+bx+c=0$$,定义判别式$$\Delta = b^2-4ac$$,那么方程的两个根分别为$$
\begin{aligned}
x_1&=\frac{-b+\sqrt{\Delta}}{2a}\\
x_2&=\frac{-b-\sqrt{\Delta}}{2a}
\end{aligned}
$$

可见,当
$$
\begin{aligned}
\Delta &> 0 \Rightarrow \text{方程有两个不同的根}\\
\Delta &= 0 \Rightarrow \text{方程有两个相同的根}\\
\Delta &< 0 \Rightarrow \text{方程无实数根(但有两个共轭的复数根)}\\
\end{aligned}
$$

\subsection{配方法}
对于一些特定的问题,可以将方程配方并求解,有时这比直接使用求根公式更为简便.

例如,可以将方程配方为如下形式:
$$(x-a)(x-b)=0\Rightarrow x_1=a, x_2=b$$

或者
$$(x-a)^2=b\Rightarrow x_1=\sqrt{b}+a, x_2=-\sqrt{b}+a$$
