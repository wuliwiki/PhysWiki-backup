% 南京理工大学 2008 年 研究生入学考试试题 普通物理(B)
% license Usr
% type Note

\textbf{声明}:“该内容来源于网络公开资料,不保证真实性,如有侵权请联系管理员”

\subsection{填空题(每空2分,共30分)}

1. 一质点作直线运动,运动方程为$x=3+2t^2+t^3(t>0)\text{(SI制)}$ ,则该质点在$t = 3s$ 时,$v = \underline{\hspace{3cm}}$,在 $t = \underline{\hspace{3cm}}$ 时,质点开始作减速直线运动。

2. 力度系数为 $K$ 的轻弹簧,一端固定,另一端连接一小质量为 $m$ 的物体,$m$ 与地面间的滑动摩擦系数为 $\mu_k$。在弹簧为原长时,对静止物体 $m$ 施一沿 $X$ 轴方向的恒力为 $F (F > f, f = \mu_k mg)$ 时,则该弹簧的最大伸长量为 $\underline{\hspace{3cm}}$,该过程恒力 $\vec{F}$ 作功为 $\underline{\hspace{3cm}}$。

3. 一质量为 $m$,长为 $4r$ 的均匀直尺,一端系于 0 点,另一端连接一质量为$2m$,半径为 $r$ 的匀质小圆盘边缘。该系统可绕通过0 点垂直于纸面的轴转动,则该系统对0 轴的转动惯量 $I = \underline{\hspace{3cm}}$,直尺在水平位置静止并开始转动时的角加速度 $\beta = \underline{\hspace{3cm}}$,直尺转到竖直位置时的角速度 $\omega = \underline{\hspace{3cm}}$。

4. 对于刚性双原子分子理想气体,其定容摩尔热容 $C_v =\underline{\hspace{3cm}}$,等温摩尔热容 $C_T = \underline{\hspace{3cm}}$。

5. $t = 27^\circ C$ 下,$1 \\ mol$ 氧气分子的平均动能为 $\underline{\hspace{3cm}}$;$t = 27^\circ C$ 下,$1 \ mol$ 氧气分子的总动能为 $\underline{\hspace{3cm}}$。

6. 一质点作简谐振动的位移时间曲线如图所示,振幅为 $A$,周期为 $T$,则质点振动的初相角是 $\underline{\hspace{3cm}}$,振动方程为 $\underline{\hspace{3cm}}$。
\begin{figure}[ht]
\centering
\includegraphics[width=6cm]{./figures/4d636b95d8dff3a1.png}
\caption{} \label{fig_NJU08_1}
\end{figure}
7.一半径为$R$的导体球的球心为0点,在距0点$2R$处放置一点电荷,电量为$q$,则0点的电场强度 $E=\underline{\hspace{3cm}}$,电势$U=\underline{\hspace{3cm}}$。
\subsection{填空题(每空2分,共30分)}
1. 若一电容器上标明 $200 \\ \text{pF}, \\ 500 \\ \text{V}$,它能储存的最大电能为 $\underline{\hspace{3cm}}$;若该电容器与标明 $300 \\ \text{pF}, \\ 900 \\ \text{V}$ 的另一电容器串联后在两端加 $1000 \\ \text{V}$ 的电压,则此时两电容器的工作状态分别是 $\underline{\hspace{3cm}}$ 和 $\underline{\hspace{3cm}}$。

2. 一无限长载流直线 $l$ 与一载流矩形 ABCDA 共面,其尺寸如图所示,则载流线段 $AB$ 受力为 $\underline{\hspace{3cm}}$;载流线段 $BC$ 受力为 $\underline{\hspace{3cm}}$,方向为 $\underline{\hspace{3cm}}$。

3. 油轮泄漏油 $(n = 1.2)$ 入海,在海面上形成一大片油膜,如有人在膜厚为 $4600 \\ \mathring{A}$ 的油膜上空的飞机上垂直往下看,能看到 $\lambda= \underline{\hspace{3cm}}\\ \mathring{A}$ 的光;如有人潜入油膜下方的海里往上看,又能看到 $4 = \underline{\hspace{3cm}}\\ \mathring{A}$ 的光。
\begin{figure}[ht]
\centering
\includegraphics[width=6cm]{./figures/4b5b14c73053cee0.png}
\caption{} \label{fig_NJU08_2}
\end{figure}

4. 一部分偏振光经过一个可旋转的线偏振片后,得到 $I_{\max} / I_{\min} = 3 / 1$,则该部分偏振光中线偏振光的比例为 $\underline{\hspace{3cm}}$;如用自然光通过,则 $I_{\max} / I_{\min} = \underline{\hspace{3cm}}$。

5. 钠的光电效应应允许波长长达 $6250 \\ \mathring{A}$,则钠中电子的逸出功是 $\underline{\hspace{3cm}}$。

6. 电子的静止质量是 $m_0$,当电子以 $v = 0.5c$ 的速度运动时,它的总能量为 $\underline{\hspace{3cm}}$,动能为 $\underline{\hspace{3cm}}$。

7. 在氢原子光谱的巴尔末系中,最短波长为 $\underline{\hspace{3cm}}$,最长波长为 $\underline{\hspace{3cm}}$。
\subsection{(12 分)}
已知光栅狭缝的宽度 $a = 1.5 \times 10^{-4} \\, \text{cm}$, 当用波长 $\lambda = 600 \\, \text{nm}$ 的单色光垂直照射在光栅上时,发现第 4 级缺级(第一个缺级),透镜的焦距 $f = 1 \\, \text{m}$,求

(1) 光栅常数;

(2) 屏幕上所呈现的全部明级的级数和条数。
\subsection{(12 分)}
一半径为 $R$ 的均匀带电的介质球(介电常数 $\varepsilon$),带电量为 $q$,求

(1) 电场强度分布;

(2) 球内电场能量和球外电场能量之比。
\subsection{(10 分)}
试证明:理想气体的定压摩尔热容量为 $C_p = \frac{i + 2}{2} R$,$R$ 为普适气体常量。
\subsection{(10 分)}
无限长螺线管半径为 $R$,管内磁场的变化率 $\frac{dB}{dt} = k (k < 0)$,导体棒 $ab = bc = R$,试求导体棒中感应电动势的大小和方向。
\begin{figure}[ht]
\centering
\includegraphics[width=6cm]{./figures/d9d3eb49ddf8e2d8.png}
\caption{} \label{fig_NJU08_3}
\end{figure}
\subsection{(12 分)}
如图所示,一均匀直棒长度为 \( l \),质量为 \( M \),上端挂在水平轴 0 上,自由下垂。今有一质量为 \( m \) 的子弹水平射入杆下端而不复出,以后棒摆至水平位置又开始下落,设子弹射入到停止在棒内均极短,空气阻力不计。求子弹进入棒前的速度。
\subsection{(12 分)}

\subsection{(10 分)}

\subsection{(12 分)}