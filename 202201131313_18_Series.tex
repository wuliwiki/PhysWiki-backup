% 数项级数

\pentry{序列的极限\upref{SeqLim} 极限存在的判据、柯西序列\upref{CauSeq}}

\subsection{基本定义}
设$\{a_n\}_{n\in\mathbb{N}}$是一个实数序列. 形式表达式
\[
\sum_{n=1}^\infty a_n
\]
称为以$a_n$为一般项的级数 (series). 有限和
\[
S_N:=\sum_{n=1}^N a_n
\]
是良好定义的, 它称为级数$\sum_{n=1}^\infty a_n$的部分和 (partial sum). 如果部分和序列$\{S_N\}_{N\in\mathbb{N}}$有极限, 也就是说存在实数$S$使得
\[
\lim_{N\to\infty}\sum_{n=1}^N a_n=S,
\] 
则称级数$\sum_{n=1}^\infty a_n$收敛到$S$(converges to $S$), 这$S$称为它的和 (sum). 如果部分和序列不存在极限, 则称级数发散(divergent). 

根据极限运算的简单性质, 容易看出部分和序列$\{S_N\}$若有极限, 则必定有$\lim_{n\to\infty}a_n=0$, 由此即得到级数收敛的一个简单的必要条件. 但它并不是充分条件. 一般级数收敛性的唯一充分必要条件是
\begin{theorem}{级数收敛的柯西判据}
级数
\[
\sum_{n=1}^\infty a_n
\]
收敛, 当且仅当任给$\varepsilon>0$, 都存在正整数$N_\varepsilon$, 使得当$N'>N>N_\varepsilon$时有
$$
\left|\sum_{n=N}^{N'} a_n\right|<\varepsilon.
$$
\end{theorem}
这是序列收敛的柯西判据的直接推论.

\begin{example}{发散级数}
根据定义, 一般项不趋于零的级数当然发散. 但一般项趋于零也并不能保证收敛性. 一个著名的例子是调和级数
$$
\sum_{n=1}^\infty\frac{1}{n}.
$$
有许多种办法能证明它发散. 例如, 对于正整数$N$, 和式
$$
\sum_{n=N}^{2N}\frac{1}{n}
$$
总共有$N+1$项, 除了最后一项之外, 每一项都大于$1/2N$, 所以
$$
\sum_{n=N}^{2N}\frac{1}{n}>\frac{N+1}{2N}>\frac{1}{2}.
$$
按照柯西判据, 这个级数是发散的.

通过积分的办法, 可以得到一些有用的渐近公式:
$$
\sum_{n=1}^{N}\frac{1}{n}=\ln N+O(1),\quad N\to\infty;
$$
如果$0<\alpha<1$, 那么
$$
\sum_{n=1}^{N}\frac{1}{n^\alpha}=\frac{N^{1-\alpha}}{1-\alpha}+O(1).
$$
\end{example}



\subsection{定义辨析}
对于收敛的级数, 有时也可以用形式表达式来代表它的和. 但是需要注意, 形式等式$\sum_{n=1}^\infty a_n=S$的真正含义是$\lim_{N\to\infty}S_N=S$, 必须要在这个意义下来理解它. 由此, 级数的和与有限个实数的和有所不同.