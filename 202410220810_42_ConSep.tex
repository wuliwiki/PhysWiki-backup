% 凸集的可分离性
% keys 凸集|分离性
% license Usr
% type Tutor

\pentry{凸集和凸体\nref{nod_ConSet},泛函与线性泛函\nref{nod_Funal}}{nod_959a}

线性空间中集 $M,N$ 的分离性是指存在一个\enref{线性泛函}{Funal} $f$,使得泛函在这两个集上的值由一个常数区分开。凸集的分离性是指线性空间中两个凸集,若其中之一的\enref{核}{ConSet}非空且不与另一集相交,则必存在非零线性泛函将这两个凸集分离。

\begin{theorem}{分离性}
设 $M,N$ 是实线性空间的两个子集,$f:L\rightarrow\mathbb R$ 是线性泛函,则
\end{theorem}
