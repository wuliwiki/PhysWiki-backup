% yolov5 算法代码解析
% keys CNN
% license Usr
% type Wiki

\pentry{深度学习 CNN 入门\nref{nod_CNN1},深度学习 CNN 入门 2\nref{nod_CNN2},深度学习 CNN 入门 3\nref{nod_CNN3}}{nod_9fd7}

通过对深度学习CNN结构的学习,现在对CNN实践代码Yolo进行解析。
这部分和CNN相关的在后面,前面主要是一些代码的解析。

首先了解一下,项目的目录结构。\begin{figure}[ht]
\centering
\includegraphics[width=13cm]{./figures/bfaa89d38baa2c9a.png}
\caption{目录} \label{fig_yolov5_1}
\end{figure}
对于github文件夹是放配置文件的,这个不重要,可以忽略。

对于data文件是放一些超参数的配置文件(如.yaml),是用来配置训练集、测试集、验证集的路径的。还有一些官方提供的测试的图片。特别注意,如果训练自己的数据集的话需要修改这里的yaml文件。但是自己的数据集最好在外面新建一个dataset的文件夹。

models文件夹里放的是yaml文件,里面的模型文件有s,m,l,x,训练速度从快到慢,精确的从低到高,这个参数是表示模型参数量的大小的。

runs文件夹里放的是我们运行时的一些输出的文件,每次运行都会生成一个exp的文件夹。其中的detect文件是测试模型,输出图片并在图片中标注出物体和概率。,train文件夹里是训练模型,有模型的权重文件、混淆矩阵、F1曲线、超参数文件、P曲线、R曲线、PR曲线、结果文件等expn。

utils文件,存放的是工具类函数,里面有loss函数,metrics函数,plot函数等等。

\begin{enumerate}
\item detect代码详细分析
\end{enumerate}
首先导入python的包。
\begin{lstlisting}[language=python]
import argparse # 解析命令行参数的库
import os # 与操作系统进行交互的文件库 包含文件路径操作与解析
import sys # sys模块包含了与python解释器和它的环境有关的函数。
from pathlib import Path # Path能够更加方便得对字符串路径进行处理
 
import cv2 # sys模块包含了与python解释器和它的环境有关的函数。
import torch #pytorch 深度学习库
import torch.backends.cudnn as cudnn #让内置的cudnn的 auto-tuner 自动寻找最
#适合当前配置的高效算法,来达到优化运行效率的问题
\end{lstlisting}
因为有第一个库,我们可以直接在命令行里输入相关参数,而不用进py文件里一个个去修改,具体后面来介绍。接着获取当前文件的绝对路径。
\begin{lstlisting}[language=python]
FILE = Path(__file__).resolve()  # __file__指的是当前文件(即detect.py),FILE
#最终保存着当前文件的绝对路径,比如D://yolov5/detect.py
ROOT = FILE.parents[0]  # YOLOv5 root directory  ROOT保存着当前项目的父目录,#比如 D://yolov5
if str(ROOT) not in sys.path:  # sys.path即当前python环境可以运行的路径,假如当#前项目不在该路径中,就无法运行其中的模块,所以就需要加载路径
    sys.path.append(str(ROOT))  # add ROOT to PATH  把ROOT添加到运行路径上
ROOT = Path(os.path.relpath(ROOT, Path.cwd()))  # relative ROOT设置为相对路径
\end{lstlisting}