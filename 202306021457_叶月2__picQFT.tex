% 绘景变换与时间演化
\subsection{绘景选择}
为了简便地解决问题,我们可以根据力学量算符$F$和态矢$\ket{s}$的含时关系,选择不同的绘景:
\begin{equation}
\begin{aligned}
\text{薛定谔绘景}\frac{\partial \ket{s}}{\partial t}\neq 0 \quad \frac{\partial F}{\partial t}&= 0\\
\text{海森堡绘景}\frac{\partial \ket{s}}{\partial t}= 0 \quad \frac{\partial F}{\partial t}&\neq 0\\
\text{相互作用绘景}\frac{\partial \ket{s}}{\partial t}\neq 0 \quad \frac{\partial F}{\partial t}&\neq 0
\end{aligned}
\end{equation}
我们需要注意到两点基本事实,第一:绘景只是系统演化的不同图景,因而态矢及力学量的初始值必定是相同的。第二:如同表象变换,观测值也不随绘景的选择而改变。第三:由于时间演化算符并非力学量算符,所以在不同绘景里的变换关系需要额外推导。
\subsection{绘景变换}
设薛定谔绘景里的时间演化算符为$U_s$,那么某时刻的力学量期待值为$\bra{s}U^\dagger_sF U_s\ket{s}$。由于在海森堡绘景里,态矢不变,则$\ket{s,t}_H=\ket{s}=U^\dagger_s\ket{s,t}_s$,因此,任意时刻的海森堡力学量算符为$F_H=U^\dagger_sF U_s$。

相互作用绘景,顾名思义,哈密顿量$H^I$包含自由哈密顿量$H^I_0$和相互作用哈密顿量$H^I_\mathrm{i}$,含时微扰可以在该绘景下处理。(不含时的也可以,哈密顿量的含时性并非一个硬性条件,从海森堡绘景和薛定谔绘景的关系也可以看出。)

相互作用绘景与薛定谔绘景的关系,其实就是自由理论下海森堡绘景与薛定谔绘景的关系。即
\begin{equation}
\ket{s,t}_I=\mathrm{e} ^{\mathrm{i}H_0}\ket{s,t}_s\quad F_I=\mathrm{e}^{\mathrm{i}H_0}F_s \mathrm{e}^{\mathrm{i}H_0}
\end{equation}
从该绘景变换我们也可以发现,两个绘景的自由哈密顿量一致,但总哈密顿量却是不同的。
\subsection{时间演化}
时间演化虽称之为算符,却不是我们一般处理的厄米算符,因此其含时关系与厄米算符不同。根据态矢的演化,我们也能得到时间演化算符的动力学关系:
\begin{equation}
\mathrm{i}\partial_tU_s=H^s U_s\quad \mathrm{i}\partial_tU_I=H^I_\mathrm{i} U_I
\end{equation}
在薛定谔绘景里,总哈密顿量驱动了时间演化算符,而在相互作用绘景里,则是相互作用哈密顿量驱动了该算符。

由于在薛定谔绘景里,哈密顿量不含时(我们往往如此设定),那么我们可以轻松得到时间演化算符的形式解$\mathrm{e}^{-\mathrm{i}H}$。但在相互作用绘景里,哈密顿量显含时间,在更一般的情况下,不同时间的哈密顿量是不对易的,因此指数上的因子不能简单地写成单一的哈密顿量积分形式。对戴森级数的简化可以给出一个简洁的形式解。
在场论里,相互作用绘景与海森堡绘景会联系到一起。



