% 弱引力透镜
% license Usr
% type Tutor


今天,关于星系团尺度上暗物质存在的最引人注目的证据之一来自对一对碰撞星系团的观测,这些星系团被称为子弹星系团,位于3.7G年远的地方,目录名为1E0657-558(或1E0657-56),首次于2006年进行了详细观测,以及类似的系统。子弹星系团中大部分重子质量以热气体的形式存在,其分布可以通过X射线发射来追踪。总质量的分布,包括可见和暗物质,是通过弱透镜独立测量的。子弹星系团系统的特殊之处在于,可见物质和暗物质在空间上是分离的,见\autoref{fig_Weakls_1} 。\begin{figure}[ht]
\centering
\includegraphics[width=14cm]{./figures/95028d589604d526.png}
\caption{子弹星系团} \label{fig_Weakls_1}
\end{figure}
在过去,这两个星系团是普通的系统,可见物质和暗物质混合在一起。这两个物体在1.5亿年前发生了碰撞。可见物质与自身有显著的相互作用,以至于两个星系团的热气体经历了碰撞性冲击波。另一方面,暗物质与自身和正常物质的碰撞可以忽略不计,以至于两个系统中的暗物质云仅仅是相互穿过。这导致了目前可见和暗物质组分的分离,如图1.3所示。在观察子弹星系团之后,许多类似的系统已经被研究。Harvey等人(2015)报告了对72个类似系统的观测结果,并得出结论,暗物质的存在可以以超过7σ的显著性建立。这种观测对那些用修改引力代替暗物质的替代解释施加了严重的压力。这样的修改不能从正常物质中空间分离出来(除非它们也引入了实际上表现为暗物质的东西),因此异常的透镜信号将

8详细研究重建了两个星系团碰撞前大约3000 km/s的初始相对速度。有人声称这个速度异常高:根据ΛCDM宇宙学中速度分布的尾部,观察到这种事件的概率声称太低(假设合理的物质非均匀性,大约是10^-5)。因此,子弹星系团在相对速度的这个特定方面被用作反对暗物质的证据。后来的研究对此提出了质疑,并发现与ΛCDM宇宙学一致的概率[15]。