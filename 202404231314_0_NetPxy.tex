% 网络代理(计算机网络课程)
% license Xiao
% type Tutor

\subsubsection{方法}
\begin{itemize}
\item 手动设置 Proxy(通过控制面板): \verb`Contrl Panel > Network and Internet > Internet Options > Connections > LAN settings`
\item 手动设置 Proxy(通过 Win10 或以上的设置菜单): \verb`Network & Internet > Proxy`
\item \verb`Proxy Auto-Configuration (PAC)` 设置文件, \verb`Web Proxy Auto-Discovery Protocol (WPAD)`
\item 通过设置\textbf{组策略}
\item 命令行的 \verb`netsh`
\end{itemize}

\subsubsection{种类}
\begin{itemize}
\item \textbf{HTTP/HTTPS 代理}
\item \textbf{FTP 代理}
\item \textbf{SOCKS Proxy} (SOCKS4、 SOCKS5) 可以代理任意 TCP 和 UDP 连接。Windows 并不原生支持,但是浏览器一般支持,也就是浏览器直接把数据和第三方 proxy 软件交互而不是走系统的 http api。 一般来说其他软件如果想要支持也要专门给 SOCKS 做适配。 但 Proxifier 和一些客户端据说可以把一般应用的网络也通过 SOCKS 代理。
\item \textbf{VPN}: 不严格属于 proxy, 比 proxy 更底层,从操作系统内核的\textbf{网络栈}上实现,通常使用 full tunnel,也就是任何离开电脑的数据都会通过 VPN 服务器中转。 一般的 app 无法自己选择避开 VPN。 当然一些高级 VPN 客户端也会有 split tunnel 的功能也就是让某些 app、协议或端口不使用 VPN。
\end{itemize}

\subsubsection{应用可以选择不遵守系统 proxy}
\begin{itemize}
\item 如果 HTTP 请求使用 Windows API 例如 \verb`WinHTTP` or \verb`WinINET`, 那将受系统 proxy 控制。
\end{itemize}
