% 电磁场中的薛定谔方程及规范变换
% keys 电磁场|薛定谔方程|哈密顿算符|标势|矢势|广义动量
% license Xiao
% type Tutor

\pentry{点电荷的拉格朗日和哈密顿量\nref{nod_EMLagP}, 量子化\nref{nod_QMPos}, 原子单位制\nref{nod_AU},电磁场的规范变换\nref{nod_Gauge}}{nod_3411}

\footnote{本文参考 \cite{Bransden}。}本文如无特殊说明使用\enref{原子单位制}{AU}。 电动力学中,电磁场中电荷量为 $q$ 的粒子的哈密顿量为(\autoref{eq_EMLagP_3}~\upref{EMLagP})
\begin{equation}\label{eq_QMEM_1}
H = \frac{1}{2m} (\bvec p - q\bvec A)^2 + q\varphi + V(\bvec r)~.
\end{equation}
其中 $\varphi$ 和 $\bvec A$ 分别是\enref{电磁场的标势和矢势}{EMPot},都是位置 $\bvec r$ 和时间的函数。 $\bvec p$ 是 $\bvec r$ 的\enref{广义动量}{HamCan},
\begin{equation}\label{eq_QMEM_6}
\bvec p = m \bvec v + q\bvec A~.
\end{equation}
其中 $V(\bvec r)$ 是系统的所有其他势能。 在原子分子物理中,\autoref{eq_QMEM_1} 可以计算氢原子在外部电磁场中的变化, 此时原子核对电子的作用通常被包含在 $V(\bvec r)$ 中, 而 \textbf{$\bvec A, \varphi$ 仅表示外部电磁场的作用}。

现在要把经典的 $H$ 做量子化,也就是将 ${\bvec p} = -\I\grad$ 代入得量子哈密顿算符为
\begin{equation}\label{eq_QMEM_2}
\ali{
H &= \frac{\bvec p^2}{2m} - \frac{q}{2m} (\bvec A \vdot \bvec p + \bvec p \vdot \bvec A)
+ \frac{q^2}{2m} \bvec A^2 + q \varphi + V(\bvec r)\\
&= -\frac{1}{2m} \laplacian + \I \frac{q}{2m} (\bvec A \vdot \Nabla + \Nabla \vdot \bvec A) + \frac{q^2}{2m} \bvec A^2 + q\varphi + V(\bvec r)~,
}\end{equation}
注意算符 $\Nabla \vdot \bvec A$ 是指先把波函数乘以矢势再取散度而不是直接对 $\bvec A$ 取散度(想想量子力学中算符相乘的定义)。

另外要注意 $\bvec p = -\I\Nabla$ 代表的是\autoref{eq_QMEM_6} 的\textbf{广义动量}而不是 $m\bvec v$。 所以一般规范下的平面波 $\exp(\I \bvec k \vdot \bvec r)$ 的能量是
\begin{equation}
E = \frac{(\bvec k - q\bvec A)^2}{2m}~.
\end{equation}
在\enref{长度规范}{LenGau}下, $\bvec A \equiv \bvec 0$, 这时才有常见的 $E = \bvec k^2/(2m)$。

如果对电磁场进行规范变换(\autoref{eq_Gauge_3}~\upref{Gauge})
\begin{equation}\label{eq_QMEM_5}
\bvec A = \bvec A' + \grad \chi~,
\qquad
\varphi = \varphi' - \pdv{\chi}{t}~.
\end{equation}
其中 $\chi$ 是\autoref{eq_Gauge_3}~\upref{Gauge} 中的任意标量函数 $\lambda$。 规范变换后的哈密顿算符哈密顿量为
\begin{equation}\label{eq_QMEM_8}
H' = \frac{1}{2m} (\bvec p - q\bvec A')^2 + q\varphi' + V(\bvec r)~.
\end{equation}
考虑变换前后的含时薛定谔方程,
\begin{equation}\label{eq_QMEM_9}
H\Psi = \I \pdv{t}\Psi~,
\end{equation}
\begin{equation}\label{eq_QMEM_10}
H'\Psi' = \I \pdv{t}\Psi'~.
\end{equation}

那么 $\Psi$ 和 $\Psi'$ 之间要如何做规范变换才能使两个方程都成立呢?可以证明该变换为 
\begin{equation}\label{eq_QMEM_3}
\Psi(\bvec r, t) = \exp(\I \frac{q}{\hbar}\chi) \Psi'(\bvec r, t)~,
\end{equation}
证明见下文。 所以对于任意规范, \autoref{eq_QMEM_2} 和\autoref{eq_QMEM_8} 都保持相同的形式(gauge invariant)。

在量子力学中,常见的规范如\enref{库仑规范}{CouGau},以及\enref{偶极子近似}{DipApr}下的\enref{长度规范}{LenGau}和\enref{速度规范}{LVgaug}。

\subsection{高斯单位制}
\pentry{高斯单位制\nref{nod_GaussU}}{nod_71e8}
注意高斯单位制中 $\hbar$ 不是 1, 不可省略。 电磁场中单个粒子的哈密顿量变为
\begin{equation}
H = \frac{\hbar^2(\bvec p - q\bvec A/c)^2}{2m} + q\varphi + V(\bvec r)~.
\end{equation}
$\bvec p$ 是广义动量 $\bvec p=m\bvec v+q\bvec A/c$。
如果对电磁场进行规范变换
\begin{equation}
\bvec A=\bvec A'+\grad \chi,\qquad \varphi =\varphi' - \pdv{\chi}{t}~.
\end{equation}
波函数也要乘一个相位因子:
\begin{equation}
\Psi(\bvec r,t)=\exp(\I\frac{q}{c\hbar} \chi)\Psi'(\bvec r,t)~.
\end{equation}

\subsection{多粒子薛定谔方程}
电磁场中多个带电粒子的含时薛定谔方程
\begin{equation}
H = \sum_i \frac{(-\I\Nabla_i - q\bvec A)^2}{2m_i} + q_i\varphi + \sum_i V(\bvec r_i) + \sum_{i,j}\frac{q_iq_j}{\abs{\bvec r_i-\bvec r_j}}~.
\end{equation}
不难证明
\begin{equation}
\Psi(\bvec r_1, \dots,t)= \prod_i \exp[\I q\chi(\bvec r_i, t)]\Psi'(\bvec r_1, \dots,t)~.
\end{equation}

\subsection{证明}\label{sub_QMEM_1}
现在证明若\autoref{eq_QMEM_9} 成立, 且 $H', \Psi'$ 由\autoref{eq_QMEM_8} \autoref{eq_QMEM_3} 定义, 那么\autoref{eq_QMEM_10} 也成立。

这个证明并没有想象中那么复杂。 首先证明
\begin{equation}
(-\I\Nabla - q\bvec A)\Psi = \exp(\I q\chi)(-\I\Nabla - q\bvec A')\Psi'~.
\end{equation}
同理
\begin{equation}\label{eq_QMEM_4}
\frac{(-\I\Nabla - q\bvec A)^2}{2m}\Psi = \exp(\I q\chi)\frac{(-\I\Nabla - q\bvec A')^2}{2m}\Psi'~.
\end{equation}
然后证明
\begin{equation}\label{eq_QMEM_7}
\qty(q\varphi - \I \pdv{t})\Psi= \exp(\I q\chi)\qty(q\varphi' - \I \pdv{t})\Psi'~
\end{equation}
\autoref{eq_QMEM_4} 和\autoref{eq_QMEM_7} 相加可证\autoref{eq_QMEM_10}。

该推导容易拓展到多粒子的情况。 另外,无论使用\enref{库仑}{CouGau}、\enref{长度}{LenGau}、\enref{速度}{LVgaug}中哪种常见的规范,把原子核与电子间的库仑作用包含在 $\varphi$ 中还是分离到 $V$ 中都不影响上述推导。 我们一般选择后者。
