% 卡尔·魏尔施特拉斯(综述)
% license CCBYSA3
% type Wiki

本文根据 CC-BY-SA 协议转载翻译自维基百科 \href{https://en.wikipedia.org/wiki/Karl_Weierstrass}{相关文章}。

卡尔·西奥多·威廉·魏尔斯特拉斯(Karl Theodor Wilhelm Weierstrass,/ˈvaɪərˌstrɑːs, -ˌʃtrɑːs/,[德语发音:Weierstraß [ˈvaɪɐʃtʁaːs];1815年10月31日-1897年2月19日)是德国数学家,常被誉为“现代分析学之父”。尽管他大学未取得学位便退学,但他自学数学并接受了师范培训,最终在学校教授数学、物理、植物学和体操。后来他获得了荣誉博士学位,并成为柏林大学的数学教授。

在众多贡献中,魏尔斯特拉斯形式化了函数连续性的定义和复分析理论,证明了中值定理和博尔查诺–魏尔斯特拉斯定理,并利用后者研究了闭有界区间上连续函数的性质。
\subsection{生平}
魏尔斯特拉斯出生于普鲁士威斯特法伦省恩尼格尔洛附近的奥斯滕费尔德村的一个罗马天主教家庭。[4]

卡尔·魏尔斯特拉斯是威廉·魏尔斯特拉斯与特奥多拉·冯德福斯特的儿子,父亲是一名政府官员,父母皆为信奉天主教的莱茵兰人。他在帕德博恩的特奥多里安文理中学(Theodorianum)求学期间便对数学产生了浓厚兴趣。中学毕业后,他被送往波恩大学,目的是为将来从政做准备,因此被安排学习法律、经济和财政等科目——这与他一心想学习数学的志向发生了直接冲突。他通过对既定学业置之不理、私下自学数学的方式来解决这种矛盾,这也最终导致他未能获得学位便中途退学。

魏尔斯特拉斯随后在明斯特学院继续数学学习(该学院当时已以数学著称),他的父亲还为他争取到了明斯特师范学院的一个名额,他在那里努力学习,最终获得了教师资格。在此期间,他听了克里斯托夫·古德曼的课程,并由此对椭圆函数产生了浓厚兴趣。
