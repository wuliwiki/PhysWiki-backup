% 约翰·查尔斯·菲尔兹(综述)
% license CCBYSA3
% type Wiki

本文根据 CC-BY-SA 协议转载翻译自维基百科\href{https://en.wikipedia.org/wiki/John_Charles_Fields}{相关文章}。

约翰·查尔斯·菲尔兹(John Charles Fields),英国皇家学会院士(FRS),加拿大皇家学会院士(FRSC)(1863年5月14日—1932年8月9日)是一位加拿大数学家,也是菲尔兹奖的创始人,该奖项用于表彰在数学领域的杰出成就。
\subsection{职业生涯}  
菲尔兹出生于加拿大西部的汉密尔顿,父亲是皮革店老板。菲尔兹于1880年从汉密尔顿中学毕业,并于1884年毕业于多伦多大学,随后前往美国巴尔的摩的约翰霍普金斯大学深造。1887年,菲尔兹获得了博士学位。他的博士论文题为《符号有限解与通过确定积分求解方程 \(d^ny/dx^n = x^my\)》,该论文于1886年发表在《美国数学杂志》上。

菲尔兹在约翰霍普金斯大学教授了两年后,加入了宾夕法尼亚州米德维尔的阿勒格尼学院教职。在对北美当时的数学研究状态感到失望后,他于1891年前往欧洲,主要在柏林、哥廷根和巴黎定居,在那里他与当时一些最伟大的数学家交往,包括卡尔·魏尔斯特拉斯、费利克斯·克莱因、费尔迪南德·乔治·弗罗贝纽斯和马克斯·普朗克。菲尔兹还与哥斯塔·米塔格-莱夫勒建立了终生的友谊。他开始发表有关代数函数的新研究,这一领域将成为他职业生涯中最有成果的研究方向。

1902年,菲尔兹回到加拿大,在多伦多大学讲学。回到自己出生的国家后,他不遗余力地提升数学在学术界和公众中的地位。他成功地游说安大略省立法机关,为大学争取到每年75,000美元的研究经费,并帮助建立了加拿大国家研究委员会和安大略省研究基金会。菲尔兹于1919年至1925年期间担任加拿大皇家学会主席,在此期间,他希望将学会发展成为一个领先的科学研究中心,尽管取得的成功不尽相同。然而,他的努力对多伦多成为1924年国际数学家大会(ICM)举办地起到了关键作用。菲尔兹曾在1912年剑桥、1924年多伦多和1928年博洛尼亚的国际数学家大会上担任特邀讲者。

菲尔兹因其创立菲尔兹奖而闻名,该奖被一些人视为数学界的诺贝尔奖,尽管两者之间存在一些差异。菲尔兹奖首次颁发是在1936年,1950年重新引入,并且自那时以来每四年颁发一次。该奖项颁发给两位、三位或四位年龄不超过40岁的数学家,以表彰他们在数学领域的杰出贡献。

菲尔兹在1920年代末开始筹划这一奖项,但由于健康状况恶化,他未能在生前看到奖项的实施。他于1932年8月9日因三个月的病痛去世;在遗嘱中,他留下了47,000美元作为菲尔兹奖基金。菲尔兹奖的设立计划由J. L. Synge完成。
\subsection{荣誉}  
菲尔兹于1907年当选为加拿大皇家学会会士,1913年当选为伦敦皇家学会会士。

多伦多大学的菲尔兹研究所以他的名字命名,以表彰他的贡献。
\subsection{参考文献}  
\begin{enumerate}
\item Synge, J. L. (1933). "John Charles Fields. 1863–1932". *皇家学会会士讣告*,1 (2): 131–138. doi:10.1098/rsbm.1933.0010.  
\item Fields, J. C. (1886). "Symbolic Finite Solutions and Solutions by Definite Integrals of the Equation dny/dxn = xmy". *美国数学杂志*,8 (4): 367–388. doi:10.2307/2369393. ISSN 0002-9327. JSTOR 2369393.  
\item Van Brummelen, Glen; Kinyon, Michael, eds. (2005). *数学与历史学家的手艺:肯尼斯·O·梅讲座*。Springer. p. 173. ISBN 9780387252841.  
\item Fields, J. C. "直接从有理函数的基本性质推导补充定理"。在 Hobson, E.W.; Love, A. E. H. (编). *第五届国际数学大会会议录*(剑桥,1912年8月22-28日)(PDF),第1卷,312-326页。  
\item Fields, J. C. "理想理论的基础"。在 Fields, J.C. (编). *国际数学大会会议录*(多伦多,1924年8月11-16日)(PDF),第1卷,245-298页。  
\item Zanichelli, Nicola (编). *国际数学大会会议录*(博洛尼亚,1928年9月3-10日)(PDF),第2卷。  
\item "关于我们:菲尔兹奖"。多伦多大学菲尔兹研究所。检索于2010年8月21日。
\end{enumerate}
\subsection{进一步阅读}  
\begin{itemize}
\item Riehm, Elaine; Hoffman, Frances (2011). *数学中的动荡时代:J. C. 菲尔兹的生平与菲尔兹奖的历史*。普罗维登斯,罗德岛:美国数学学会。ISBN 978-0-8218-6914-7。
\end{itemize}
\subsection{外部链接}
\begin{itemize}
\item 菲尔兹研究所传记  
\item O'Connor, John J.; Robertson, Edmund F.,“John Charles Fields”,《麦克图尔数学历史档案》,圣安德鲁斯大学  
\item 约翰·查尔斯·菲尔兹在数学家家谱项目中的资料  
\item 约翰·查尔斯·菲尔兹的档案文件由多伦多大学档案与记录管理服务保存
\end{itemize}