% Linux 分区和文件系统操作笔记(Gparted, fdisk, resize2fs, grub, Clonezilla)
% license Xiao
% type Note

\begin{issues}
\issueDraft
\end{issues}

\subsection{文件操作}
\begin{itemize}
\item \verb|sudo dd| 是一个很危险的命令, 它直接操作硬盘上的任意位置的数据, 无论是否被挂载。 \verb|sudo dd if=/dev/random of=/dev/sda| 会把整个硬盘包括分区方式(MBR,GPT)等所有信息抹掉
\item \textbf{MBR (Master Boot Record)} 和 \textbf{GPT (GUID Partition Table)} 几乎是唯一两种硬盘的\textbf{分区方式(partition schemes)}
\item 要备份 MBR,用 \verb|dd if=/dev/sda of=/data/disk.img bs=446 count=1| , 还原同理, 但还原最好只用 440 字节
\item 用 \verb|dd| 把一个硬盘的分区克隆的另一个硬盘的分区后(两个分区大小必须完全相同), 再 clone MBR 用 \verb|dd if=/dev/sda of=/dev/sdc bs=440 count=1|
\item 要擦除 MBR 中的 grub, 首先备份! 然后用 \verb|sudo dd if=/dev/zero of=/dev/sdx bs=440 count=1|
\item 不同的分区当然可以有不同的文件系统
\item \verb|Gparted| 是一个 GUI 硬盘分区工具, 在 Ubuntu live CD 中自带
\item \verb|GParted| 使用 \verb|e2image -ra -p /dev/sda1 /dev/sdb1| 来复制分区, 没有数据的部分不会复制,Arch Wiki 中的 \href{https://wiki.archlinux.org/title/disk_cloning}{disk cloning} 词条列出了很多工具。 用 \verb|e2image| 来备份和恢复参考\href{https://stackoverflow.com/questions/51755887/backup-and-restore-e2image-how-do-i-properly-pipe-the-output-from-lzop}{这里}。
\item \verb|sda, sdb| 等编号是不稳定的, 可能重启等操作之后会变
\item \verb|fdisk -l| 查看所有挂载的硬盘
\item \verb|lsblk| 也可以查看
\item 用 \verb|sudo e2fsck -f test.img| 可以检查并自动修复文件系统。
\item \verb|resize2fs -p /dev/sd? ???K| 可以改变 ext4 文件系统的大小, Gparted 用的就是这个命令。 这个命令需要很长时间。
\item \verb|mklabel| 修改分区的 label, \verb|ntfslabel /dev/nvme0n1p3 win| 改变 ntfs 分区的 label, 还有 \verb|e2label| 用法一样。 注意用 \verb|gparted| 里面的 label 功能似乎并不是真正的 label, \verb|/etc/fstab| 识别不出后者。 用 \verb|e2label /dev/sda1| 可以查看某个分区的 label。
\item \verb|mkfs.ext4 /dev/sdx1| 或者 \verb|mke4fs -t ext4 /dev/sdb1| 把某个分区格式化为 \verb|ext4|
\item Clonezilla 的分区备份对 GPT 和 MBR 都是通用的, 如果发生任何问题, 用 live cd 重装系统,确保能正常启动, 然后还原对应的分区就行,无需任何额外设置。
\item 之前 SMART 提示 threshold exceeded,但并没有看到那一项有问题,现在重新写 GPT,安装,clonezilla 还原, 又可以正常用了。
\item  正常的 ext4 系统分区用 Gparted 缩小以后, 突然就不能启动了!貌似并不是 grub 的问题,内核已经加载然后弄出一堆错误。
\item 有一个工具叫做 \href{https://help.ubuntu.com/community/Boot-Repair}{boot-repair} 可能可以用来修复 bootloader 的问题
\item \verb|extFAT| 在 Ubuntu 中并不怎么支持 resize 和移动, 还是用 windows 的傲梅比较好
\end{itemize}

\subsubsection{fstab (File System Table)}
\begin{itemize}
\item \verb|/etc/fstab| 用于设置如何把不同的分区或网络硬盘挂载到不同文件夹。
\item 格式为 \verb|<file system> <mount point> <type> <options> <dump> <pass>|
\item \verb|uid=1000,gid=1000| 是当前登录用户的 user/group id,也可以直接把数字换成用户名和组名。
\item \verb|NTFS| 的设置是 \verb|LABEL=mydrive2 /mnt/drive2 ntfs-3g nofail,x-gvfs-show,uid=1000,gid=1000 0 0|
\item \verb|LABEL=win /mnt/win ntfs defaults,auto,rw,nofail,umask=000,uid=addis,gid=addis 0 0|
\item \verb|exFAT| 的设置是 \verb|LABEL=mydrive1 /mnt/drive1 exfat nofail,x-gvfs-show,uid=1000,gid=1000 0 0|
\item 第一列可以是 \verb|LABEL=| 或者 \verb|UUID=|, 第二列是 mount point, 第三列是文件系统, 用 \verb|auto| 可以自动检测。 第四列是一些选项, \verb|nofail| 是即使硬盘没有插入也不会报错, \verb|x-gvfs-show| 是在 gnome 文件浏览器和 taskbar 中显示。 \verb|uid| 和 \verb|gid| 是挂载后文件的 owner, \verb|1000| 是当前用户。 \verb|auto| 是开机的时候自动挂载。
\item \verb|defaults| 选项相当于 \verb|rw, suid, dev, exec, auto, nouser, async|.
\item \verb|rw| 是支持读写, \verb|suid| 支持 SetUID 权限, \verb|exec| 支持可执行文件。 \verb|dev| 选项可以支持挂载文件夹中的设备文件(block device 和 character device), \verb|nodev| 则禁止。 \verb|nouser| 只允许管理员挂载或弹出硬盘。 \verb|user| 允许任何用户挂载或弹出硬盘。 \verb|users| 允许任何用户挂载,但只有挂载的用户可以弹出某个硬盘。
\item \verb|user| 和 \verb|users| 自动开启 \verb|noexec|, 除非明确指定 \verb|exec| 选项。 此时即使文件显示有 \verb|x| 权限也无法执行(但 script 仍然可以用 \verb|source| 执行)。
\item \verb|sudo mount -a| 会挂载 \verb|fstab| 中的所有硬盘(如果没有已经挂载)。 \verb|sudo umount -a| 同理(如果被占用则不会弹出)。
\item \verb|mount| 未必需要 sudo, 如果被挂载的目录不需要管理员权限。
\item \verb|sudo mount 挂载目录| 可以根据 \verb|fstab| 中该目录的设置挂载。 \verb|umount 挂载目录| 同理。 可以是相对或绝对目录。
\item 注意如果某个目录中有打开的 bash (包括当前正在用的), 那么 \verb|umount| 就会失败(\verb|target is busy|)。
\end{itemize}

\subsubsection{挂载文件作为硬盘}
\begin{itemize}
\item 参考\href{https://www.tecmint.com/create-virtual-harddisk-volume-in-linux/}{这个教程}。
\item 先随便创建一个文件, 文件大小就是硬盘的初始大小(下面可以 resize)。 如 \verb|sudo dd if=/dev/zero of=/路径/虚拟硬盘名.img bs=1M count=100|。 其中镜像文件 \verb|.img| 是不必须的。
\item 格式化为 ext4: \verb|sudo mkfs -t ext4 /路径/虚拟硬盘名.img|。 注意运行完以后, 镜像文件会自动缩小的刚好够用, 所以第一步中不需要弄太大的文件。
\item 临时挂载: \verb|sudo mount -t auto -o loop /路径/虚拟硬盘名.img 挂载目录| 其中 \verb|-t| 是文件系统类型。 镜像文件甚至可以存在于 \verb|挂载目录| 中。 但是挂载以后并不能看到该文件。
\item 要永久挂载, 可以用 \verb|/路径/虚拟硬盘名.img /挂载目录 ext4 nofail 0 0|
\item 使用 \verb|df -h| 查看文件系统会看到例如 \verb|/dev/loop13 488M  24K  452M  1% /mnt/test_dir|
\item \verb|resize2fs 虚拟硬盘名.img 1G| 可以改变分区大小。
\item 如果文件在 \verb|ext4| 系统中 \verb|du| 镜像文件, 那么 resize 后所占硬盘空间仍然会是刚好够用。 这只是 ext4 特有的优化, 实际上文件的大小的确是 \verb|resize2fs| 的目标大小(可以用 \verb|stat 文件| 确认)。 如果把这个文件移动到比如说 exFat 文件系统中, 那么占用空间就会变成后者的大小。
\end{itemize}

\subsubsection{逻辑卷管理(LVM)}
可以把多个物理硬盘融合为一个逻辑硬盘, 见 Logical Volume Management(LVM)笔记\upref{LVMnt}。

\subsection{Bootloader}
\begin{itemize}
\item 开机的时候,BIOS/UEFI 会先启动 bootloader 然后再由 bootloader 加载系统(也就是按 F2 或 Del 以后进入的那个界面选择的其实是 bootloader)
\item 有一个设置 Grub 的 GUI 软件叫做 \verb|grub-customizer|, 可以设置 grub 菜单, 记住上次的选择, 检测新系统, 等。
\item \verb|os-prober| 也可以发现新系统并加入 grub 菜单(会被 grub-customizer 使用)
\item grub 可以装在和系统不同的盘上
\item MBR 的 grub 设置文件在 \verb|/boot/grub/grub.cfg| , GPT 的在 \verb|/boot/efi/EFI/ubuntu/grub.cfg|
\item \verb|lilo| 是一个可以安装 generic windows boot loader on MBR 的命令
\item \verb|sda, sdb| 等在 grub2 命令行里面的 ls 名字叫做 \verb|hd1, hd2...|
\item windows的bootloader自动加载windows,而grub会搜索每个硬盘和分区,并列出所有选项
\item 在GPT中,bootloader总是会在EFI分区,而在MBR中,bootloader在masterrecord中
\item \verb|sudo grub-install /dev/sd#| 可以在某个硬盘中安装 grub (注意如果硬盘只有一个分区的话, 分区名和硬盘名相同, 安装可能就会出错)
\item 注意 \verb|grub-install| 可能使用 \verb|grub-pc| 包也可能使用 \verb|grub-efi-amd64| 包。 前者给 MBR 硬盘安装, 后者给 GPT 硬盘安装。 它们可以同时安装。
\item 如果遇到 \verb|failed to get canonical path of...| 就用 \verb|sudo grub-install /dev/sdX --root-directory=/mnt/[mount point directory]|
\item 如果安装遇到 \verb|will not proceed with blocklist|, 就在安装命令后面加 \verb|--force|。
\item \verb|df| 只会显示已挂载的硬盘, 而 \verb|fdisk| 或者 \verb|disk, gparted| 软件才会显示所有连接的硬盘
\item 要安装 ubuntu live 镜像到 disk, 用 \verb|sudo dd bs=4M if=/path/to/ISOfile of=/dev/sda status=progress oflag=sync| 注意 \verb|sda| 后面不能有数字! 这样 \verb|dd| 会把整个硬盘克隆成 iso 的内容而无视之前的任何 partition table 和分区。 此时 u 盘和光盘完全等效,都是只读的(亲测成功).
\end{itemize}

\begin{itemize}
\item grub 在菜单中按 c 进入命令行模式, 然后 ls 可以看见有哪些硬盘和分区, 例如 (hd0,msdos1), 然后继续 ls (hd0,msdos1)/ 按 tab,就会提示该目录下有什么
\item grub 在菜单中按 e 就可以编辑临时的 grub config 文件,填入 ls 获得的信息即可
\item grub 的菜单是由 \verb|/boot/grub/grub.cfg| 或者 \verb|grub.config| 定义的
\item 如何克隆 MBR 硬盘上安装的 Ubuntu?
\item 如果克隆 MBR 硬盘的分区, 那么 grub 并不会一起克隆过去, 所以还需要 \verb|sudo grub-install ...| 在新硬盘上面安装 grub (如果说什么不推荐用啥不稳定的,就在后面加 --force). 然后再把原来的 grub.cfg 拷贝到新硬盘就好了(不确定新安装 grub 会不会把 cfg 重置).
\item 注意克隆分区以后, 目标分区的 uuid 会和原来的 uuid 一模一样,如果拔掉原来的硬盘没什么问题, 但如果要一起插的话还是需要改一下
\item \verb|blkid| 可以查看所有分区的 uuid
\item uuid 需要 umount 才可以修改, 随机生成 uuid 用 \verb|tune2fs -U random /dev/sdb?| , disk 中会看到变化, 如果 blkid 看见 uuid 没有变化, 就 mount 再 umount 即可刷新
\item \verb|fdisk| 也可以改变 uuid, \verb|sudo fdisk /dev/sd?| 然后按 x 再按 i 即可
\item 理论上 GPT 是可以在 BIOS 主板上 boot 的(实际上对大部分 linux 可以), 见\href{https://superuser.com/questions/1337344/is-it-possible-to-boot-linux-from-a-gpt-disk-on-a-bios-system}{这里}。
\end{itemize}

\subsection{Clone Ubuntu 硬盘}
以下步骤亲测成功。
\begin{itemize}
\item 启动 Ubuntu 的 live cd 的 usb
\item 如果目标硬盘/分区比原硬盘/分区小, 那么用 GParted 缩小原分区。
\item 用 GParted 把每个分区逐一复制粘贴, 分区尺寸要一模一样。
\item 如果 part table 是 MBR, 用上面的 \verb|dd| 命令备份并拷贝 MBR
\item 用 GParted 把目标分区的 UUID 随机生成新的, 避免两个分区具有一样的 UUID 发生冲突
\item 用 \verb|grub-install /dev/sd?| 重新安装 grub, 确认目标硬盘中 \verb|/boot/grub/grub.cfg| 配置文件中所有 UUID 都是新分区的 UUID 而不是老的。 如果是老的就全部替换。
\end{itemize}

\subsection{常见文件系统}
\subsubsection{NTFS}
\begin{itemize}
\item 微软专属
\item 支持在线扩容(瞬间完成)
\item linux 中读写巨慢
\item 不支持 symlink, WSL 支持但在 file explorer 中用不了(据说开发者模式可以支持)
\end{itemize}

\subsubsection{exFAT}
\begin{itemize}
\item exFAT 是 Windows 发明的, 适用于几乎所有操作系统的文件系统
\item 不支持 symlink
\item linux 中性能比 NTFS 要好一些
\item 几乎不可能改变分区大小, 只能先把文件移走, 重新分区再移动回来
\item 一般用于移动硬盘, 常用操作系统都兼容
\item Linux 内核在 5.几之后自动支持 exFat 文件系统, 但之前不支持, 简单的手动安装即可
\begin{lstlisting}[language=bash]
sudo add-apt-repository universe
sudo apt update
sudo apt install exfat-fuse exfat-utils
\end{lstlisting}
\end{itemize}

\subsubsection{ext4}
\begin{itemize}
\item linux 常用的文件系统, linux 下速度快(尤其是小文件读写)
\item 支持在线扩容(时间较长)
\end{itemize}

\subsubsection{btrfs}
\begin{itemize}
\item Fedora 桌面版的默认文件系统
\item 比较现代和 fancy 的文件系统, 支持创建和恢复到某个时间点的状态。
\item 似乎可以检测到 bitrot, 但不能恢复
\end{itemize}

\subsubsection{zfs}
\begin{itemize}
\item 和 btrfs 功能差不多, 还多了 volume 管理, 可以在多块硬盘上创建文件系统, 可以通过增加硬盘的方式扩容, 多用于企业。
\item 另外参考储存服务器管理软件 \href{https://www.truenas.com/truenas-core/}{TrueNAS}。
\item zfs 可以直接创建 RAID-Z 阵列,不需要其他依赖。
\item Ubuntu 19 开始支持 zfs 作为系统盘。
\end{itemize}


\subsection{Windows 上面挂载 ext4 分区}
\begin{itemize}
\item 最流行的免费软件: \href{https://sourceforge.net/projects/ext2fsd/}{Ext2Fsd}
\item 一个半免费软件(pro 要付费): \href{https://www.diskinternals.com/linux-reader/}{Linux Reader}
\item 高性能付费软件: \href{https://www.paragon-software.com/business/extfs-for-windows/}{Paragon ExtFS}
\end{itemize}

\subsection{S.M.A.R.T}
