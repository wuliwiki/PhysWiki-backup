% 数学结构
% 结构


\subsection{引言}
一堆砖块,如果只是散乱的堆在一起,意义并不是很大。
然而,如果根据设计图,将砖块堆砌起来,让它们之间拥有各种关系(比如这块砖隶属于地基区域,为其它砖奠基),这些砖块就可能成为一个有组织的有机整体,一个具有功能与价值建筑物。
在数学上,也是一样的道理。集合内的零散元素往往缺乏价值,需要我们赋予某种“结构”。

\begin{equation}
a^2 + b^2 = c^2 % is this correct?
\end{equation}

\subsection{基本释义}
对全体实数的集合,我们往往默认实数间的加法与乘法(减法与除法作为它们的逆运算自然诱导出),但这些运算并不是理所当然的,我们可以考虑一个没有任何运算的纯数集。
此时,集合内除了散乱的数,没有别的东西。显然,这样的集合也没有太大的意义。我们无法问两个数加起来等于什么,无法问两个数之间的距离,因为这些东西都尚无定义。 

若对该集合赋予一些满足公理(如加法与乘法的交换律、结合律、分配律等)的运算,且这些运算描述了数集中元素的关系,我们就可以称赋予了这些运算的集合为一个代数结构。代数结构的例子有群、环、域等。

又比如,对于一个集合A={纽约,莫斯科,巴黎},我们可以为该集合附加一个满足相应条件(比如必须大于0)的“度量”,或者说距离函数,将任意两个城市组成的二元组,例如(纽约,莫斯科)作为自变量,以一个数作为因变量。我们可以将这个数当作是这两个城市间的距离。换言之,度量(结构)使得我们可以询问任意两个元素间的距离。此时,集合A也就成为了一个(数学)结构,即度量空间。
一般而言,集合+(数学)结构=空间。常见的例子有\href{https://wuli.wiki/online/LSpace.html}{线性空间}、线性赋范空间、内积空间、n维欧几里得空间、希尔伯特空间、拓扑空间等。

\subsection{常见分类}
我们往往可以看到一些代数结构(Algebraic Structure) ,比如交换代数、结合代数、外代数、李代数等。

也可能会看见一些“几何结构”(Geometric Structure),比如n维欧几里得空间等。

代数结构常指被赋予关系与运算的集合,由这些关系与运算所能得到的结果,往往也就构成了一个“代数”,此时“代数”一词是作为结构而存在的(如上文的李代数)。

几何结构,最广泛地说,可以被视为定义了一些集合子集的分类,以及它们之间运算的规则。可选地,我们可以将一个集合S上的几何结构G看作满足(确定性质的)S的幂集的一个或多个子集。 

举例来说,欧几里得的平面几何,平面是一个点集,点、线、平面图形等都是平面的子集,它们自身也是点集。两点之间只有一连线之类的规则,定义了它们之间的关系与运算。

需要注意的是,这两种结构并不是泾渭分明的,例如,对于一个平面,我们可以将其视为一个欧几里得的几何结构,也可以将其视为一个解析几何的代数结构。

\subsection{举例说明}
对一个实数三元组的集合{(x,y,z)|x,y,z∈R},作为(其中是笛卡尔积)即,可以给其附加一个线性结构,一个欧式内积(结构),由内积诱导出度量(结构)与范数(结构),使得其成为一个实数域R上的3维欧几里得空间。 

又例如,对于一个(实)拓扑流形,赋予其C类微分结构,使其成为一个类(实)微分流形。
