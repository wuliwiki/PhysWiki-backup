% 多元函数泛函的极值
% 多元函数|泛函极值|Euler-Ostrogradsky方程

一元函数的泛函极值问题已经研究过,现在研究 $n$ 元函数的泛函极值问题,并推导 $n$ 元函数泛函极值问题中的Euler-Ostrogradsky方程.“一元”在于泛函的宗量(泛函的“自变量”)为一元函数(即可取曲线\upref{DesCur}),“ $n$ 元” 在于泛函的宗量为 $n$ 元函数(这时也可称为可取曲线,只不过变元为 $n$ 个).

在 $n$ 元函数的情形,我们仍然限定泛函 $J[\varphi]$ 的宗量 $\varphi(x_1,\cdots,x_n)$ 属于 $C_1$ 类函数,即在它的定义域上连续,并有关于所有变元 $x_i$ 的连续偏导数 $\partial_i\varphi=\pdv{\varphi}{x_i}=\varphi'_{x_i}$.

为明确说明我们的问题,先推广在一元情形中距离和邻区(\upref{DesCur})的定义.


具体来说,我们的问题可归结为:在所有定义在 $n$ 维空间中有界区域 $Q$ 上的 $C_1$ 类函数中,以 $\overline{C_1}$ 表示那些在 $Q$ 的边界 $\Omega$ 上取给定值的属于 $C_1$ 类函数的全体,即若 $\varphi\in\overline{C_1}$,则 $\varphi(\Omega)=f(\Omega)$ 而 $f(\Omega)$ 是给定的.在 $\overline{C_1}$ 中找出一个函数 $\varphi$,使它给出泛函 $J[\varphi]$ 的极值.



