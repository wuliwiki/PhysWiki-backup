% 微积分学(综述)
% license CCBY4
% type Wiki

本文根据 CC-BY-SA 协议转载翻译自维基百科\href{https://en.wikipedia.org/wiki/Calculus}{相关文章}。

微积分是研究连续变化的数学分支,正如几何学研究形状,代数学研究算术运算的推广一样。

微积分最初被称为“无穷小微积分”或“无穷小量的演算”,它有两个主要分支:微分学和积分学。微分学关注的是瞬时变化率和曲线的斜率,而积分学则研究量的累积以及曲线下方或两条曲线之间的面积。这两大分支通过微积分基本定理相互联系。它们都依赖于无穷数列和无穷级数收敛于确定极限的基本概念。\(^\text{[1]}\)微积分是处理变量随时间或其他参照变量变化问题的“数学支柱”。\(^\text{[2]}\)

无穷小微积分在17世纪末分别由艾萨克·牛顿和戈特弗里德·威廉·莱布尼茨独立创立。\(^\text{[3][4]}\)后来的工作,包括对极限概念的形式化,使这些发展建立在更坚实的概念基础之上。如今,微积分被广泛应用于科学、工程、生物学,甚至在社会科学和其他数学分支中也有重要应用。\(^\text{[5][6]}\)