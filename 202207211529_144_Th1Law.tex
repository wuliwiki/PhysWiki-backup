% 热力学第一定律
% 热力学第一定律|能量守恒|做功|传热|内能

\begin{issues}
\issueDraft
\end{issues}

\pentry{压力体积图\upref{PVgraf}, 理想气体内能\upref{IdgEng}}
\footnote{本文参考自朱文涛《简明物理化学》}

\begin{figure}[ht]
\centering
\includegraphics[width=14cm]{./figures/Th1Law_1.png}
\caption{热力学第一定律} \label{Th1Law_fig1}
\end{figure}

\begin{theorem}{热力学第一定律}
热力学第一定律表明, 系统的\textbf{内能} $U$ 增加等于外部对系统传递的热量 $Q$(流入系统为正) 减去系统对外做功 $W$(系统对外做功为正):
\begin{equation}\label{Th1Law_eq1}
\Delta U = Q - W
\end{equation}
%我记得这个公式有一个适用条件?%
热力学第一定律写成微分形式是
\begin{equation} \label{Th1Law_eq2}
\dd U = \dd Q - \dd W =\dd Q - P \dd V
\end{equation}
\end{theorem}

\textbf{热力学第一定律}是\textbf{能量守恒}在热力学中的形式.热力学第一定律的另一种表述是:\textbf{第一类永动机}是不可能的.
\addTODO{什么是第一类永动机?链接到永动机词条}

\subsection{内能、功、热}
\textbf{内能U} 指系统包括的总能量(例如系统内分子的动能、势能等),是一个状态量.系统的内能有时也用 $E$ 表示. 

\textbf{热Q} 是指系统与环境由于温度差而引起的能量转移,是通过微观粒子的无规则相互作用传递的,是一个过程量.

\textbf{功W\upref{Fwork}} 是指除热之外的其他能量传递形式,一般是由宏观的作用力和宏观位移产生的,例如机械功等于压力乘以体积变化、表面功等于表面张力乘以表面积变化,也是一个过程量.

\begin{example}{膨胀的气体}
\begin{figure}[ht]
\centering
\includegraphics[width=5cm]{./figures/Th1Law_2.png}
\caption{仅仅是个表情包:)} \label{Th1Law_fig2}
\end{figure}
\footnote{图片来自网络}

对于活塞容器中克服外界压力膨胀的气体, 把外界压强记为 $P$, 系统体积记为 $V$, 那么系统克服外压、对外做的功可以写成:
\begin{equation}
W = \int_{V_1}^{V_2} P \dd{V}
\end{equation}
\end{example}

\subsection{状态量与过程量}
内能 $E$ 只和系统的状态有关, 被称为\textbf{状态量}\upref{StaPro}. 所以 $\Delta E$ 也只与系统的初始和最终的状态有关, 与中间的过程无关.

但 $Q$ 和 $W$ 和系统变化的过程本身有关: 也就是说即使系统的初末状态确定, 中间的过程不一样也会导致它们的不同. 这样的量被称为\textbf{过程量}\upref{StaPro}.

\addTODO{写一些理想气体的例题, 例如 PV 图种, 计算两点间延着不同轨迹的热量}


对理想气体\upref{Igas}, 令分子自由度为 $i$, 有
\begin{equation}
E = \frac{i}{2}n RT
\end{equation}
