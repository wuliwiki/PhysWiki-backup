% 刚体的静力平衡
% 静力平衡|合外力|刚体|合力矩

\pentry{动量定理\upref{PLaw}, 角动量定理\upref{AMLaw}}

\footnote{参考 Wikipedia \href{https://en.wikipedia.org/wiki/Mechanical_equilibrium}{相关页面}和 \cite{新力}.}在惯性系中, 如果刚体所受的所有合外力与合外力矩\upref{Torque}都为零, 则我们说它处于\textbf{静力平衡(static equilibrium)}. 其中合外力(矩)是指所有施加在刚体上的力(矩)的矢量和. 通常来说, 静力平衡意味着刚体保持静止不动, 但严格来说刚体质心可以做匀速运动, 刚体也可以绕质心做匀速转动. 在非惯性系中, 若加入惯性力的修正, 该结论仍然成立.

注意当合外力为零时, 合外力矩与参考点(参考系)的选取无关(\autoref{Torque_eq5}~\upref{Torque}).

\subsection{证明}
把刚体看做由许多质点组成,合外力为零时刚体动量守恒\upref{PLaw}, 而动量等于 “质心的动量” \upref{SysMom}
$\bvec p_c = M_c \bvec v_c$,所以质心做匀速运动或不动.

刚体合外力矩为零时,质点系角动量守恒\upref{AMLaw},而角动量等于质心的角动量 $\bvec L_c =\bvec r_c\cross \bvec p_c$ 加上质心系中的角动量(\autoref{AngMom_eq5}~\upref{AngMom}). 当质心匀速直线运动或不动时 $\bvec L_c$ 不变,所以质心系中刚体的角动量也不变,所以刚体绕质心做匀速转动或不转动.

\subsection{例题}

\addTODO{插入一个简单的例题, ep1 轻杆的三力平衡, BTW,如果是重杆怎么办, 为什么重力可以看成作用在质心上面?}
\begin{example}{轻杆三力平衡}
如图, 一个长度为 $L$ 质量不计的细杆, 中间和两端受力分别为 $\bvec F_1, \bvec F_2, \bvec F_3$.
\end{example}

\begin{example}{}\label{RBSt_ex1}
如\autoref{RBSt_fig1}, 一个质量为 $m$ 的线轴被斜挂在墙上, 线轴与墙面的摩擦系数为 $\mu$,线轴的大圆半径为 $R$, 小圆半径为 $r$, 求当 $\alpha$ 满足什么条件时, 线轴才能不滑落.
\begin{figure}[ht]
\centering
\includegraphics[width=5cm]{./figures/RBSt_1.pdf}
\caption{线轴的平衡} \label{RBSt_fig1}
\end{figure}

我们先来看线轴受哪几个力:重力 $mg$, 绳的拉力 $T$, 墙的支持力 $N$ 和摩擦力 $f$. 由摩擦系数的定义和刚体平衡条件可得
\begin{equation}
\begin{cases}
f \leqslant \mu N & \text{(摩擦系数)}\\
N - T\sin\alpha = 0 & \text{(水平方向受力平衡)}\\
T\cos\alpha + f - mg = 0 & \text{(竖直方向受力平衡)}\\
Tr - fR = 0 & \text{(力矩平衡)}
\end{cases}
\end{equation}
其中最后一条力矩平衡是以圆心为原点计算力矩, 虽然原则上我们可以取任意点计算力矩, 但取在圆心计算最为简单. 除了 $\alpha$ 我们有三个未知数 $T, f, N$, 用以上三条等式恰好可以把这三个未知数消去, 可得关于 $\alpha$ 的不等式
\begin{equation}
\sin\alpha \geqslant \frac{r}{\mu R}
\end{equation}

一个有趣的地方在于, 不等式中没有出现质量 $m$. 事实上, 我们不使用那条含有 $mg$ 的等式也可以顺利得到答案.
\end{example}
