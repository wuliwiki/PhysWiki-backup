% 轨道方程、比耐公式
% 轨道方程|比耐公式

\pentry{中心力场问题\upref{CenFrc}, 二阶常系数非齐次微分方程\upref{Ode2N}}
\subsection{比耐公式(Binet 公式)}
我们来看 “中心力场问题\upref{CenFrc}” 中得到的两条运动方程(\autoref{CenFrc_eq5} 和\autoref{CenFrc_eq4})
\begin{equation}
\ddot{r} - r \dot\theta^2 = F(r)/m \label{Binet_eq1}
\end{equation}
\begin{equation}
mr^2\dot \theta = L \label{Binet_eq2}
\end{equation}
为了得到极坐标中 $r(\theta)$ 的微分方程(\textbf{轨道方程}), 我们以下用\autoref{Binet_eq2} 消去\autoref{Binet_eq1} 中的 $t$. 首先可以把 $r$ 看做复合函数 $r[\theta(t)]$, 再用链式法则\upref{ChainR}处理\autoref{Binet_eq1} 的第一项
\begin{equation}\label{Binet_eq10}\ali{
\ddot{r} & = \dv{t} \qty( \dv{r}{t} ) = \dv{t} \qty( \dv{r}{\theta} \dv{\theta}{t} ) = \dv{\theta}\qty( \dv{r}{\theta} ) \qty( \dv{\theta}{t} )^2 + \dv{r}{\theta}\dv[2]{\theta}{t}\\
& = \dv[2]{r}{\theta} \qty( \dv{\theta}{t} )^2 + \dv{r}{\theta}\dv{\theta} \qty( \dv{\theta}{t} )\dv{\theta}{t}
}\end{equation}
然后把\autoref{Binet_eq2} 代入\autoref{Binet_eq1} 消去所有 $\dot\theta = \dv*{\theta}{t}$, 得到 $r$ 关于 $\theta$ 的微分方程
\begin{equation}
\dv[2]{r}{\theta} \qty( \frac{L}{r^2} )^2 + \dv{r}{\theta}\dv{\theta} \qty( \frac{L}{r^2} )\frac{L}{r^2} - r \qty( \frac{L}{r^2} )^2 =  m F(r)
\end{equation}
即
\begin{equation}\label{Binet_eq5}
\dv[2]{r}{\theta} + r^2\dv{r}{\theta}\dv{\theta} \qty( \frac{1}{r^2} ) - r =  \frac{m r^4}{L^2} F(r)
\end{equation}
这就是轨道方程. 这个方程比较复杂, 但可以通过换元法% 未完成:介绍微分方程的换元
化为十分简洁的形式.令
\begin{equation}\label{Binet_eq13}
u \equiv \frac{1}{r}
\end{equation}
代入\autoref{Binet_eq5},  得到 $u$ 关于 $\theta $ 的微分方程
\begin{equation}\label{Binet_eq15}
\dv[2]{u}{\theta} + u = -\frac{m}{L^2 u^2} F\qty(\frac 1u)
\end{equation}
这是一个阶常系数非齐次微分方程\upref{Ode2N}, 被称为\textbf{比耐公式}.
将\autoref{Binet_eq15} 两边同乘 $\frac{\dd u}{\dd \theta}$,再对 $\theta$ 作积分,可以得到一阶的微分方程:
\begin{equation}\label{Binet_eq3}
\qty(\frac{\dd u}{\dd \theta})^2+u^2=-\frac{2m}{L^2}V\qty(\frac{1}{u})+\frac{2 m E}{L^2}
\end{equation}
其中最后一项 $2mE/L^2$ 是积分过程中产生的常量,并且可以验证 $E$ 就是系统的总能量.有了一阶微分方程之后,就可以分离变量法进行积分,求解 $u$ 关于 $\theta$ 的函数,即求解轨道形状.
\subsection{开普勒问题的轨道形状}
在开普勒问题中,相互作用势为 $V(\rho)=-GMm/\rho=-k/\rho$.那么 \autoref{Binet_eq3}  变为
\begin{equation}
\begin{aligned}
\left|\frac{\dd u}{\dd \theta}\right|&=\sqrt{-u^2+\frac{2m k}{L^2}u+\frac{2m E}{L^2}}\\
&=\sqrt{-\qty(u-\frac{m k}{L^2})^2+\frac{2m E}{L^2}+\frac{m^2 k^2}{L^4}}
\end{aligned}
\end{equation}
该一阶偏微分方程的解的形式为
\begin{equation}
u-\frac{m k}{L^2}=\alpha\cos(\phi-\beta)
\end{equation}
可以解得
\begin{equation}
\begin{aligned}
\alpha=\sqrt{\frac{2m E}{L^2}+\frac{m^2 k^2}{L^4}}\\
\end{aligned}
\end{equation}
这样就求得了 $u$ 关于 $\phi$ 的表达式.最后将 $u$ 用 $\rho=1/u$ 表示,得到
\begin{equation}
\rho=\frac{p}{1+e\cos(\phi-\beta)}
\end{equation}
其中 $e$ 为轨道的偏心率(或者称离心率).$p,e$ 由下式给出:
\begin{equation}
\begin{aligned}
&p=\frac{L^2}{m k}\\
&e=\frac{L^2}{m k}\alpha=\sqrt{1+\frac{2L^2E}{m k^2}}
\end{aligned}
\end{equation}

根据圆锥曲线的极坐标方程\upref{Cone},可以知道开普勒问题的轨道呈椭圆、抛物线或双曲线形状.