% Softmax 函数(综述)
% license CCBYSA3
% type Wiki

本文根据 CC-BY-SA 协议转载翻译自维基百科\href{https://en.wikipedia.org/wiki/Softmax_function}{相关文章}。

Softmax 函数,也称为 softargmax1: 184  或归一化指数函数2: 198 ,能够将一个长度为 K 的实数向量转换为 K 个可能结果的概率分布。它是逻辑函数在多维空间的推广形式,常用于多项式逻辑回归中。Softmax 函数通常作为神经网络中最后一层的激活函数,用于将网络输出归一化为对各个预测类别的概率分布。
\subsection{定义}
Softmax 函数以一个长度为 $K$ 的实数向量 $\mathbf{z}$ 作为输入,并将其归一化为一个概率分布:该分布由 $K$ 个概率值组成,每个概率值与输入中对应元素的指数成正比。也就是说,在应用 Softmax 之前,向量中的某些分量可能为负,或大于 1,且它们的和不一定为 1;但在应用 Softmax 之后,每个分量都将位于区间 $(0, 1)$ 之内,并且所有分量之和为 1,因此可以将它们解释为概率。此外,输入值越大的分量,对应的概率也越大。

标准(单位)Softmax 函数$\sigma: \mathbb{R}^K \to (0,1)^K$,其中 $K > 1$,它接收一个向量$\mathbf{z} = (z_1, \dotsc, z_K) \in \mathbb{R}^K$,并计算输出向量$\sigma(\mathbf{z}) \in (0,1)^K$的每个分量,定义为:
$$
\sigma(\mathbf{z})_i = \frac{e^{z_i}}{\sum_{j=1}^{K} e^{z_j}}.~
$$
换句话说,Softmax 对输入向量 $\mathbf{z}$ 中的每个元素 $z_i$ 应用标准指数函数(即 $e^{z_i}$),然后将所有指数值归一化——即每个指数值除以所有指数值的总和。这个归一化操作保证了输出向量 $\sigma(\mathbf{z})$ 所有分量的和为 1,从而可以被解释为概率分布。

“Softmax”一词来源于指数函数对输入向量中最大值的放大作用。例如,对向量 $(1, 2, 8)$ 进行标准 Softmax 运算,其结果大约为$(0.001, 0.002, 0.997)$,也就是说,几乎所有的权重都被分配给了最大值 8 所在的位置。

一般情况下,Softmax 函数中不一定非要使用自然底数 $e$,可以使用任意大于 0 的底数 $b$。和之前一样:如果 $b > 1$,那么输入向量中较大的分量会对应较大的输出概率;并且当 $b$ 增大时,所得的概率分布将更加集中在最大值所在的位置;相反,如果 $0 < b < 1$,那么输入中较小的分量反而会对应较大的输出概率;随着 $b$ 的减小,概率分布将更多地集中在最小值所在的位置。我们可以写成如下形式:$b = e^{\beta}$,或$b = e^{-\beta}$,其中 $\beta$ 为实数。这将导致 Softmax 函数有如下表达式:
$$
\sigma(\mathbf{z})_i = \frac{e^{\beta z_i}}{\sum_{j=1}^{K} e^{\beta z_j}} \quad \text{或} \quad \sigma(\mathbf{z})_i = \frac{e^{-\beta z_i}}{\sum_{j=1}^{K} e^{-\beta z_j}}, \quad \text{其中 } i = 1, \dotsc, K~
$$
其中与 $\beta$ 的倒数成正比的值,有时被称为温度:$\beta = 1/kT$这里的 $k$ 通常取 1 或玻尔兹曼常数,$T$ 是“温度”。较高的温度(较小的 $\beta$)会使输出分布更均匀(即熵更高,更“随机”);较低的温度(较大的 $\beta$)则会使分布更尖锐,即一个值占主导地位。

在某些领域中,底数 $b$ 是固定的,对应于某种固定的尺度;而在另一些领域中,会改变参数 $\beta$ 或 $T$ 来调整分布的形状。
\subsection{解释}
\subsubsection{平滑的 arg max}
Softmax 函数是 arg max 函数(即返回向量中最大元素索引的函数)的一个平滑近似。尽管如此,“softmax”这个名称可能具有误导性:它并不是最大值函数的平滑近似,而只是arg max的平滑版本。“softmax”一词有时也被用来指代与之密切相关的 LogSumExp 函数,而后者确实是最大值函数的平滑近似。因此,为了更准确地表达其本质,一些人更倾向于使用“softargmax”这一术语,尽管在机器学习领域,“softmax”已经是习惯用法。为避免混淆,本节中使用“softargmax”这一更清晰的表述。

与其将 arg max 看作一个输出为类别索引(如 $1, 2, \dots, n$)的函数,我们可以将其视为一个输出为独热编码的函数(假设最大值是唯一的):
$$
\operatorname{arg,max}(z_1, \dots, z_n) = (y_1, \dots, y_n) = (0, \dots,0,1,0,\dots, 0)~
$$
当且仅当索引 $i$ 是向量$(z_1, \dots, z_n)$的最大值所在位置时,输出坐标 $y_i = 1$,也就是说,$z_i$是该向量的**唯一最大值**。例如,在这种编码下:$\operatorname{arg\,max}(1, 5, 10) = (0, 0, 1)$因为第三个元素是最大值。

这一表示可以推广到存在多个最大值(即多个 $z_i$ 相等且为最大值)的情况。此时,可以将值 1 平均分配给所有最大值所在的位置;形式上,对应位置取值为 $1/k$,其中 $k$ 是最大值的个数。例如:$\operatorname{arg\,max}(1, 5, 5) = (0, 1/2, 1/2)$,因为第二和第三个元素都是最大值。如果所有元素都相等,例如:$\operatorname{arg\,max}(z, \dots, z) = \left( 1/n, \dots, 1/n \right)$表示每个位置都等可能是最大值。具有多个最大值的点 $\mathbf{z}$ 被称为奇异点(singular points或 singularities),它们构成所谓的奇异集——这些是 arg max 函数不连续的点(存在跳跃不连续);而只有一个最大值的点则称为非奇异点或常规点(non-singular或 regular points)。

根据引言中的最后一个表达式,softargmax 是 arg max函数的一个平滑近似:当$\beta \to \infty$时,softargmax 逐点收敛于 arg max。也就是说,对于任意固定的输入向量 $\mathbf{z}$,当 $\beta \to \infty$ 时,有:$\sigma_{\beta}(\mathbf{z}) \to \operatorname{arg\,max}(\mathbf{z})$。然而,softargmax 不以一致方式收敛于 arg max。这意味着不同的输入点收敛速度不同,甚至可能非常缓慢。实际上,softargmax 是连续的,而 arg max 在 奇异集(即两个或多个坐标相等的位置)上是不连续的。由于连续函数的一致极限也是连续函数,而 arg max 不连续,因此 softargmax 不可能以一致方式收敛于它。其不一致收敛的原因在于:当输入中两个坐标接近相等(其中一个略大于另一个)时,arg max 的输出会发生剧烈跳跃(从一个位置变到另一个)。例如:
$\sigma_\beta(1, 1.0001) \to (0, 1)$,$\sigma_\beta(1, 0.9999) \to (1, 0)$,而对于完全相等的输入 $(1, 1)$,无论 $\beta$ 为多少,都有:$\sigma_\beta(1, 1) = (1/2, 1/2)$。这说明:越接近奇异点 $(x, x)$,收敛速度越慢。尽管如此,在非奇异点集(即最大值唯一)上,softargmax 会在紧集上收敛。这是一种更强的逐点收敛形式,适用于没有不连续跳变的区域。

相反地,当$\beta \to -\infty$时,softargmax 会以相同的方式收敛到 arg min,这时的奇异集是具有两个或多个最小值的点。在热带分析的语言中,softmax 被视为对 arg max 和 arg min 的一种变形或“量化”。具体而言,这种变形是将 max-plus 半环(对应 arg max)或 min-plus 半环(对应 arg min)替换为 对数半环。通过取极限来恢复 arg max 或 arg min 的过程,被称为“热带化”或“去量化”。

同样地,对于任意固定的 $\beta$,如果某个输入 $z_i$ 相对于温度$T = 1/\beta$来说远大于其他输入分量,那么 Softmax 的输出就会近似于 arg max。例如,当温度为 1 时,输入间的差值为 10 被认为是“很大”的:
$$
\sigma(0, 10) := \sigma_1(0, 10) = \left( 1/(1 + e^{10}), e^{10}/(1 + e^{10})\right) \approx (0.00005, 0.99995)~
$$
在这种情况下,输出几乎完全集中在最大值对应的位置上。

但如果输入差值相对于温度来说较小,则输出就不会接近 arg max。例如,当温度为 100 时,差值 10 相对较小:
$$
\sigma_{1/100}(0, 10) = \left( 1/(1 + e^{1/10}), e^{1/10}/(1 + e^{1/10})\right) \approx (0.475, 0.525)~
$$
此时两个分量的概率接近平均,远不如高温差下那么“明确”。随着$\beta \to \infty$,意味着温度$T = 1/\beta \to 0$,此时即便是很小的输入差异,相对于趋近于零的温度也变得“巨大”,这为 Softmax 在极限情形下趋于 arg max 提供了另一种解释。
\subsubsection{统计力学}
在统计力学中,softargmax 函数被称为玻尔兹曼分布,也称为吉布斯分布:\(^\text{[5]: 7}\)索引集合 $\{1, \dots, k\}$ 表示系统的微观状态;输入值 $z_i$ 表示对应状态的能量;分母称为配分函数,通常记为 $Z$;因子 $\beta$ 被称为冷度,也称为热力学贝塔或逆温度。
\subsection{应用}
Softmax 函数被广泛应用于多类分类任务中,例如多项逻辑回归(又称 softmax 回归)\(^\text{[2]: 206–209 [6]}\)、多类线性判别分析、朴素贝叶斯分类器以及人工神经网络中的输出层\(^\text{[7]}\)。具体来说,在多项逻辑回归和线性判别分析中,Softmax 函数的输入是由 $K$ 个不同线性函数的输出组成,其对某个样本向量 $\mathbf{x}$ 和权重向量 $\mathbf{w}$ 给出的第 $j$ 类的预测概率为:
$$
P(y = j \mid \mathbf{x}) = \frac{e^{\mathbf{x}^\mathsf{T} \mathbf{w}_j}}{\sum_{k=1}^{K} e^{\mathbf{x}^\mathsf{T} \mathbf{w}_k}}~
$$
这个表达式可以看作是以下两个步骤的组合:应用 $K$ 个线性函数:$\mathbf{x} \mapsto \mathbf{x}^\mathsf{T} \mathbf{w}_1, \dotsc, \mathbf{x} \mapsto \mathbf{x}^\mathsf{T} \mathbf{w}_K$(其中 $\mathbf{x}^\mathsf{T} \mathbf{w}$ 表示 $\mathbf{x}$ 和 $\mathbf{w}$ 的内积);将上述结果输入到 Softmax 函数中。这个过程等价于将一个线性算子(由权重向量 $\mathbf{w}$ 定义)应用于输入向量 $\mathbf{x}$,从而将原始的(可能是高维的)输入转换为一个位于 $\mathbb{R}^K$ 空间中的向量。这个向量可以理解为输入属于每个类别的“激活度”,再通过 Softmax 转换为概率分布。
\subsubsection{神经网络}
标准的 softmax 函数常被用作基于神经网络的分类器的最后一层激活函数。在这种设置下,网络通常在对数损失函数或交叉熵框架下进行训练,从而形成一种非线性的多项逻辑回归模型。

由于 softmax 函数的输出依赖于输入向量和特定索引 $i$,因此在计算导数时需要考虑该索引的影响:
$$
\frac{\partial}{\partial q_k} \sigma(\mathbf{q}, i) = \sigma(\mathbf{q}, i) \left( \delta_{ik} - \sigma(\mathbf{q}, k) \right)~
$$
这个表达式在索引 $i$ 和 $k$ 上是对称的,因此也可以等价地写为:
$$
\frac{\partial}{\partial q_k} \sigma(\mathbf{q}, i) = \sigma(\mathbf{q}, k) \left( \delta_{ik} - \sigma(\mathbf{q}, i) \right)~
$$
其中,$\delta_{ik}$ 是克罗内克$\delta$符号,用于简化表达式(类似于 sigmoid 函数导数通过函数自身表示的方式)。

这些导数形式在反向传播中非常重要,用于计算 softmax 层中各个神经元的梯度。

为确保数值计算的稳定性,通常会从输入向量中减去其最大值。这种做法在理论上不会改变 Softmax 的输出或导数,但在实际计算中可以显著提升稳定性,因为它有效控制了所计算指数项的最大值,避免了指数溢出的问题。

如果 Softmax 函数中引入了缩放参数 $\beta$,则上述导数表达式需要乘以该参数 $\beta$。

关于使用 Softmax 激活函数的概率模型,可参考多项逻辑模型。
\subsubsection{强化学习}
在强化学习领域,Softmax 函数可用于将动作值转换为各动作的概率。常用的形式如下:\(^\text{[8]}\)
$$
P_t(a) = \frac{\exp(q_t(a) / \tau)}{\sum_{i=1}^{n} \exp(q_t(i) / \tau)}~
$$
其中:$q_t(a)$ 表示在时间 $t$ 选择动作 $a$ 的期望奖励;$\tau$ 被称为温度参数,借用了统计力学中的术语。当温度较高(即 $\tau \to \infty$)时,所有动作的概率几乎相同;而温度越低,动作的期望奖励对概率的影响越大。当温度趋近于零(即 $\tau \to 0^+$)时,具有最高期望奖励的动作的概率趋近于 1,其他动作的概率趋近于 0。此机制常用于在探索(高温)与利用(低温)之间做平衡。
\subsection{计算复杂度与解决方案}
在神经网络应用中,可能结果的数量 $K$ 通常非常大,例如在神经语言模型中,需要从一个可能包含数百万词汇的词表中预测最可能的输出单词。\(^\text{[9]}\)这会使 softmax 层的计算(即先进行矩阵乘法以得到各个 $z_i$,再应用 softmax 函数本身)变得非常昂贵。\(^\text{[9][10]}\)更进一步地,在使用梯度下降的反向传播方法训练此类神经网络时,每一个训练样本都需要计算一次 softmax,而训练样本的数量往往也很庞大。因此,Softmax 的计算开销成为了构建更大规模神经语言模型的主要限制因素,这推动了各种用于缩短训练时间的解决方案的发展。\(^\text{[9][10]}\)

重构 Softmax 层以提高计算效率的方法包括:层次化 softmax和 分化 softmax。\(^\text{[9]}\)其中,层次化 softmax(由 Morin 和 Bengio 于 2005 年提出)采用二叉树结构,将所有可能输出(如词汇表中的单词)作为叶节点,而中间节点则是适当选择的一些“类别”,作为潜在变量。\(^\text{[10][11]}\)对于某个叶节点(即输出结果)的 softmax 概率,可以通过从根节点到该叶节点路径上所有节点概率的乘积来计算。\(^\text{[10]}\)在理想情况下,如果该树是平衡树,其计算复杂度可由$O(K)$降低至$O(\log_2 K)$。\(^\text{[11]}\)实际效果则依赖于如何将输出结果合理地聚类成类别。\(^\text{[10][11]}\)例如,Google 于 2013 年推出的 word2vec 模型中就使用了Huffman 树来实现这一结构,以提高可扩展性。\(^\text{[9]}\)

第二类解决方案是在训练过程中,通过改写损失函数来近似计算 softmax,从而避免完整归一化因子的计算。\(^\text{[9]}\)此类方法包括仅在一个输出样本子集上进行归一化的策略,例如:  重要性采样,目标采样。\(^\text{[9][10]}\)这些方法有效地减少了计算开销,尤其适用于处理具有超大类别数的输出空间。
\subsection{数值算法}
标准的 softmax 在数值上可能不稳定,因为涉及到较大的指数运算。为了解决这一问题,安全的 softmax 方法采用以下计算方式:
$$
\sigma(\mathbf{z})_i = \frac{e^{\beta (z_i - m)}}{\sum_{j=1}^{K} e^{\beta (z_j - m)}}~
$$
其中:$m = \max_i z_i$即 $m$ 是输入向量中最大的分量。通过从每个 $z_i$ 中减去最大值 $m$,可以保证指数运算的结果不会超过 1,从而避免因指数过大而导致的数值溢出问题。这是一种常见且有效的数值稳定化技巧。

Transformer 中的注意力机制接收三个参数:一个“查询向量” $q$、一组“键向量” $k_1, \dots, k_N$,以及一组“值向量” $v_1, \dots, v_N$。其输出是一个对值向量的 softmax 加权求和:
$$
o = \sum_{i=1}^{N} \frac{e^{q^{T}k_i - m}}{\sum_{j=1}^{N} e^{q^{T}k_j - m}} v_i~
$$
标准的 softmax 实现通常需要对输入进行多次循环操作,这些操作受限于内存带宽,成为计算瓶颈。而FlashAttention方法是一种避免通信的算法,它将这些操作融合为一个单一循环,从而提高了算术强度。FlashAttention 是一种在线算法,它计算以下量:[12][13]
$$
\begin{aligned}
z_i &= q^T k_i \\
m_i &= \max(z_1, \dots, z_i) = \max(m_{i-1}, z_i) \\
l_i &= e^{z_1 - m_i} + \dots + e^{z_i - m_i} = e^{m_{i-1} - m_i} l_{i-1} + e^{z_i - m_i} \\
o_i &= e^{z_1 - m_i} v_1 + \dots + e^{z_i - m_i} v_i = e^{m_{i-1} - m_i} o_{i-1} + e^{z_i - m_i} v_i
\end{aligned}~
$$
最终返回的结果为:$o_N/l_N$.在实际应用中,FlashAttention 会在每次循环迭代中处理多个 query 和 key 向量对,其方式类似于分块矩阵乘法。当需要反向传播时,FlashAttention 会缓存中间变量数组:最大值序列 $[m_1, \dots, m_N]$权重归一化因子序列 $[l_1, \dots, l_N]$在反向传播阶段,注意力矩阵会根据这些中间结果重新构建,从而实现类似于梯度检查点的机制,节省了显存使用。
\subsection{数学性质}
从几何角度来看,Softmax 函数将向量空间$\mathbb{R}^K$映射到标准 $(K - 1)$-单纯形的边界上,即维度从 $K$ 降为 $K - 1$。这是因为 Softmax 输出满足一个线性约束:所有分量之和为 1,因此其输出位于一个超平面上。换句话说,Softmax 的值域是嵌入在 $K$-维空间中的一个 $(K - 1)$-维单纯形。

沿着主对角线$(x, x, \dots, x)$方向,Softmax 的输出是均匀分布:$(1/n, \dots, 1/n)$——即当所有输入得分相等时,输出的概率也相等。

更一般地说,Softmax 对于**每个坐标都加上相同常数的平移是不变的:若对输入向量 $\mathbf{z}$ 加上 $\mathbf{c} = (c, \dots, c)$,则有:$\sigma(\mathbf{z} + \mathbf{c}) = \sigma(\mathbf{z})$这是因为对所有指数项都乘以了同一个因子 $e^c$,即:$e^{z_i + c} = e^{z_i} \cdot e^c$由于在 softmax 中计算的是比值关系,所以这个公共因子会被约掉,不影响最终结果:
$$
\sigma(\mathbf{z} + \mathbf{c})_j = \frac{e^{z_j + c}}{\sum_{k=1}^{K} e^{z_k + c}} = \frac{e^{z_j} \cdot e^c}{\sum_{k=1}^{K} e^{z_k} \cdot e^c} = \sigma(\mathbf{z})_j~
$$
从几何角度来看,Softmax 在主对角线方向上是恒定的:这是被消除的维度,对应于 Softmax 输出不受输入得分整体平移的影响(即“设置某个得分为 0”这一选择不会改变结果)。我们可以通过假设输入得分之和为 0 来规范化输入(即减去平均值):$\mathbf{c} \quad \text{其中} \quad c = \frac{1}{n} \sum z_i$这样,Softmax 就将输入得分和为 0 的超平面(即$\sum z_i = 0$)映射为一个所有分量为正且总和为 1 的开放单纯形(即  
$
\sum \sigma(\mathbf{z})_i = 1
$)。  
这就类似于指数函数将 0 映射为 1:$e^0 = 1$而且输出始终为正。

---

与此相对,Softmax **对缩放不具不变性**。
例如:

$$
\sigma((0, 1)) = \left( \frac{1}{1 + e}, \frac{e}{1 + e} \right)
$$

但:

$$
\sigma((0, 2)) = \left( \frac{1}{1 + e^2}, \frac{e^2}{1 + e^2} \right)
$$

尽管两者的差值分别为 1 和 2,但 Softmax 输出不同。因此,Softmax 对**加法平移不变**,但对**乘法缩放敏感**。
