% 开普勒问题
% 平方反比|开普勒问题|圆锥曲线轨道

\pentry{万有引力\upref{Gravty}, 角动量\upref{AngMo}, 椭圆\upref{Elips3}, 双曲线\upref{Hypb3}, 抛物线\upref{Para3}}

在中心力场问题\upref{CenFrc} 中, 若 $F(r)$ 是平方反比的力(斥力为正引力为负), 即
\begin{equation}
F(r) = \frac{k}{r^2}  \qquad V(r) = \frac{k}{r}
\end{equation}
则该问题被称为\textbf{开普勒问题}. 其中 $k$ 为非零实数. 例如对于万有引力 $k = -GMm$, 对于异种电荷间的库仑力\upref{ClbFrc}, 有\footnote{高中所学的库仑定律的系数 $k$ 在大学物理中通常记为 $1/(4\pi\epsilon_0)$, 其中 $\epsilon_0$ 为真空中的电介质常数.} $k = Qq/(4\pi\epsilon_0)$.

在开普勒问题中, 可以证明\upref{Keple1}质点的运动轨道是圆锥曲线的一种, 力心处于焦点. 质点的机械能(动能加势能) $E$ 和角动量\upref{AngMo} $L$ 可以唯一地确定轨道的形状和大小. 轨道的形状一般由离心率 $e$ 描述, 大小由半通径 $p$ 描述(\autoref{Cone_eq5}~\upref{Cone}). $E < 0$ 对应椭圆轨道, $E = 0$ 对应抛物线轨道\footnote{显然只有引力($k < 0$)可以产生非正机械能, 即椭圆轨或抛物线轨道.}, $E > 0$ 对应双曲线轨道. 注意双曲线轨道有两支, 当 $k < 0$ (引力)时取离中心天体较近的一支, $k > 0$ (斥力)时取较远的一支.
\begin{equation}\label{CelBd_eq2}
e = \sqrt{1 + \frac{2EL^2}{mk^2}}
\end{equation}
\begin{equation}\label{CelBd_eq3}
p = \frac{L^2}{m\abs{k}}
\end{equation}
椭圆或双曲线的大小和形状也可以由参数 $a,b$ 描述. $a,b$ 与 $e,p$ 的对应关系见“椭圆\upref{Elips3}”和“双曲线\upref{Hypb3}”.
\begin{equation}\label{CelBd_eq7}
a = \frac{\abs{k}}{2\abs{E}}
\end{equation}
\begin{equation}\label{CelBd_eq8}
b = \frac{L}{\sqrt{2m\abs{E}}}
\end{equation}
要求位置和时间的关系, 见 “开普勒问题的运动方程\upref{EqMoKp}”.

\subsection{证明}
如果我们已知质点轨道为圆锥曲线, 只需要简单的代数方法就可以得到上述关系. 而证明轨道是圆锥曲线则要复杂得多, 见 “普勒第一定律的证明\upref{Keple1}”.

\subsubsection{椭圆轨道}
令椭圆轨道($k<0$)距离焦点的最近和最远距离分别为 $r_1$ 和 $r_2$,列出总能量(动能加势能)守恒
\begin{equation}\label{CelBd_eq4}
\frac12 m v_1^2 + \frac{k}{r_1} = \frac12 mv_2^2 + \frac{k}{r_2}
\end{equation}
以及角动量守恒
\begin{equation}\label{CelBd_eq5}
mv_1 r_1 = mv_2 r_2
\end{equation}
把\autoref{CelBd_eq5} 中的 $v_2$ 代入\autoref{CelBd_eq4},可得
\begin{equation}\label{CelBd_eq6}
v_1^2 = \frac{-2k/m}{r_1 + r_2} \frac{r_2}{r_1}
\end{equation}
代入\autoref{CelBd_eq4} 的左边,并使用 $r_1+r_2=2a$ (\autoref{Elips3_eq9}~\upref{Elips3})得到总能量
\begin{equation}\label{CelBd_eq9}
E = \frac{k}{2a}
\end{equation}
把\autoref{CelBd_eq6} 代入\autoref{CelBd_eq5} 的左边,并使用 $r_1 r_2 = (a+c)(a-c) =b^2$ %未完成: 此处该引用公式
得角动量
\begin{equation}\label{CelBd_eq10}
L = b\sqrt{\frac{-mk}{a}}
\end{equation}
将\autoref{CelBd_eq9} 和\autoref{CelBd_eq10} 逆转即可得到\autoref{CelBd_eq7} 和\autoref{CelBd_eq8}. 要得到\autoref{CelBd_eq2} \autoref{CelBd_eq3}, 只需使用\autoref{Elips3_eq7}~\upref{Elips3} 和\autoref{Elips3_eq8}~\upref{Elips3} 即可.

\subsubsection{抛物线轨道}
已知抛物线轨道($k<0$)的总能量为零, 抛物线轨道离焦点的最近距离为焦距 $p/2$, 该点处, 动量和能量为
\begin{equation}
L = mv_0 \frac p2
\end{equation}
\begin{equation}
0 = E = \frac 12 mv_0^2 + \frac{k}{p/2}
\end{equation}
两式消去 $v_0$ 得角动量为 $L = \sqrt{mkp}$. 证毕.

\subsubsection{双曲线轨道}
无论 $k$ 的正负如何, 令双曲线轨道离焦点最近的距离为 $r_1$, 可列出总能量守恒
\begin{equation}\label{CelBd_eq11}
\frac12 mv_0^2 = \frac12 mv_1^2 + \frac{k}{r_1}
\end{equation}
该式左边表示质点在无穷远处的总能量, 此时势能为 $0$, 总能量等于动能.再来看角动量守恒
\begin{equation}\label{CelBd_eq12}
m v_0 b = m v_1 r_1
\end{equation}
该式左边为无穷远处的角动量. 由\autoref{Hypb3_eq11}~\upref{Hypb3} 可知, 在无穷远处, 双曲线的渐近线与焦点的距离为 $b$.

用以上两式消去 $v_1$, 再利用 $r_1 = a - c$, 得
\begin{equation}\label{CelBd_eq13}
E = \frac 12 m v_0^2 = \frac{\abs{k}}{2a}
\end{equation}
再将该式的 $v_0$ 代入\autoref{CelBd_eq12} 左边得
\begin{equation}
L = b\sqrt{\frac{m\abs{k}}{a}}
\end{equation}
