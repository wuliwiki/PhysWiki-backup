% Python 函数
% python|函数|def|定义|调用|实参|形参

\pentry{Python 入门\upref{Python}}

函数是组织好的,可重复使用的,用来实现相关功能的代码段.它能提高代码的重复利用率.Python提供了许多\textbf{内建函数},比如 \verb|print()|.我们也可以自己创建函数,这被叫做\textbf{自定义函数}. 需要在程序中多次执行同一项任务时, 你无需反复编写完成该任务的代码, 而只需调用执行该任务的函数, 让Python运行其中的代码. 你将发现, 通过使用函数,程序的编写、阅读、测试和修复都将更容易.

\subsection{函数的定义与调用}
我们通过一个简单的例子开始介绍:
\begin{lstlisting}[language=python]
def  func1():
    # 定义一个函数,函数名为func1
     print('hello python')
func1()
\end{lstlisting}
这个示例演示了最简单的函数结构. 第一行的代码行使用关键字\verb|def| 来告诉Python我们要定义一个函数.然后向Python指出了函数名, 还可能在括号内指出函数为完成其任务需要什么样的信息. 在这里, 函数名为\verb|func1|, 它不需要任何信息就能完成其工作, 因此括号是空的(括号必不可少); 最后,定义以\textbf{冒号}结尾, 后面的所有\textbf{缩进行}构成了函数体. 第二行处的文本是\textbf{函数注释},以\verb|#|开始,通常用来描述了函数是做什么的.
第三行\verb|print('hello  python')| 是函数体内的唯一一行代码.
\textbf{函数调用}让Python执行函数的代码. 要调用函数,可依次指定函数名以及用括号括起的必要信息, 由于这个函数不需要任何信息, 因
此调用它时只需输入\verb|func1()| 即可.

\subsection{向函数传递信息}
只需对上面函数稍作修改, 就可以让函数\verb|func1()| 不仅向用户显示 \verb|hello python|, 还将可以打印其他信息. 在函数定义\verb|def func1()| 的括号内添加信息. 通过在这里添加信息, 就可让函数接受你给定的任何值.
\begin{lstlisting}[language=python]
def  func2(str):
    '''
     改进第一个函数,使得它可以输出更多信息
     输入:字符串
    '''
     print('hello'+str)
func2('C++')
func2('Tom')
\end{lstlisting}
输出
\begin{lstlisting}[language=python]
hello C++
hello Tom
\end{lstlisting}
上述代码在第一行函数定义处多了一个信息,函数功能变得更加丰富.同时我们注意到第二个用三个\textbf{单引号}开头,不再是\verb|#|. 三个单引号是用来进行多行注释的.另外注意到Python中字符串可以通过\verb|+|来连接.

\subsection{实参和形参}
前面定义函数\verb|func2|时, 要求函数定义括号里面指定一个信息.调用这个函数并提供这种信息.实际上这个所谓的信息是一个\textbf{形参}, 而在函数调用过程中,如\verb|func2('C++')|, 值 \verb|C++| 是一个\textbf{实参}. 实参是调用函数时传递给函数的信息. 我们调用函数时, 将要让函数使用的信息放在括号内. 通俗的说,将函数定义中的变量称为形参,将函数调用中的变量称为实参.

\subsection{传递参数}
函数定义中可能包含多个形参, 因此函数调用中也可能包含多个实参. 向函数传递实参的方式很多, 可使用\textbf{位置实参}, 这要求实参的顺序与形参的顺序相同; 也可使用\textbf{关键字实参}, 其中每个实参都由变量名和值组成.
\subsubsection{位置实参}
在调用函数时, Python必须将函数调用中的每个实参都关联到函数定义中的一个形参. 为此, 最简单的关联方式是\textbf{基于实参的顺序}, 这种关联方式被称为位置实参. 如
一个用来展示个人基本信息的函数定义如下:
\begin{lstlisting}[language=python]
def ShowInfo(name,age,ID):
    '''
    定义一个函数,来展示个人信息
    name: 姓名
    age: 年龄
    ID: 学号
    '''
    print("my name is "+name+",my age is "+str(age)," and my ID  is"+ID)
\end{lstlisting}
\begin{lstlisting}[language=python]
ShowInfo('XiaoMin', 16,'123456')
\end{lstlisting}
输出\verb|my name is XiaoMin,my age is 16  and my ID  is123456|.这个函数有三个参数,\verb|name|, \verb|age|, \verb|ID|, 调用函数时候也需要指定三个参数,否则会报错,这三个参数的顺序与定义函数顺序一致,这就是位置参数.
\subsubsection{关键字实参}
关键字实参是传递给函数的\textbf{名称-值}对. 在实参中将名称和值关联起来, 因此向函数传递实参时不会混淆. 关键字实参无需考虑函数调用中的实参顺序.

函数 \verb|ShowInfo()|还是原来那样, 但调用这个函数时, 我们向Python明确地指出了各个实参对应的形参. 如:
\begin{lstlisting}[language=python]
ShowInfo(name='XiaoMin',ID='123456',age=16)
\end{lstlisting}
输出结果与之前一样,这时候不需要考虑参数的顺序. 也可以按顺序填写位置参数, 然后指定剩下参数的名字, 如 \verb|ShowInfo('XiaoMin', ID='123456', age=16)|.

\subsection{参数默认值}
编写函数时, 可给每个形参指定默认值. 在调用函数中给形参提供了实参时, Python将使用指定的实参值; 否则, 将使用形参的默认值. 因此, 给形参指定默认值后, 可在函数
调用中省略相应的实参. 使用默认值可简化函数调用, 还可清楚地指出函数的典型用法.我们再次以展示个人信息为例:
\begin{lstlisting}[language=python]
def ShowInfo2(name,age=16,ID):
    '''
    定义一个函数,来展示个人信息
    name: 姓名
    age: 年龄
    ID: 学号
    '''
    print("my name is "+name+",my age is "+str(age)," and my ID  is"+ID)
\end{lstlisting}
调用函数
\begin{lstlisting}[language=python]
ShowInfo(name='XiaoMin',ID='123456')
ShowInfo(name='XiaoLi',ID='10086')
\end{lstlisting}
这里我们只给定了两个实参, 而\verb|age|使用了默认值, 结果就会使用默认值.

\subsection{函数返回值}
函数并非总是直接显示输出, 相反, 它可以处理一些数据, 并返回一个或一组值. 函数返回的值被称为\textbf{返回值}. 在Python函数中, 可使用\verb|return|语句将值返回到调用函数的代码行. 返回值让你能够将程序的大部分繁重工作移到函数中去完成, 从而简化主程序.

下面我们定义一个函数,通过给定分数来返回对应的成绩登记.
\begin{lstlisting}[language=python]
def score(num):
    '''
    打分
    :param num:
    :return: 等级
    '''
    num = float(num)
    if num>100 or num<0:
        print('输入有误!')
    if num>90:
        return 'A'
    elif num>80:
        return 'B'
    elif num>60:
        return 'C'
    else:
        return 'D'
state =score(89)
print("state",state)
\end{lstlisting}
通函数\verb|score|, 返回成绩等级.

如果没有 \verb|return|, 那么返回 \verb|None|. 它的类型是 \verb|NoneType|.

若要返回多个值, 用 \verb|return (val1, val2, ...)|, 即返回一个 \verb|tuple| 类型括号可省略. 调用时用 \verb|(a, b) = fun(1, 2)|, 第一个括号可省略. 如果不需要某一个返回值, 可以用 \verb|_| 代替, 例如 \verb|_, b = fun(1, 2)|. 事实上这是 \verb|tuple| 的语法而不是函数特有的, 例如 \verb|(a, b) = (123, 234)| 可以分别对 \verb|a, b| 赋值, 两个括号都分别可以省略.

\subsection{传递列表}
向函数传递列表很有用, 这种列表包含的可能是名字、 数字或更复杂的对象. 将列表传递给函数后, 函数就能直接访问其内容. 下面使用函数来提高处理列表的效率.假设有一个用户列表, 我们要问候其中的每位用户.下面的示例将一个名字列表传递给一个名为\verb|greet_users()| 的函数, 这个函数问候列表中的每个人.
\begin{lstlisting}[language=python]
def greet_users(names):
"""向列表中的每位用户都发出简单的问候"""
for name in names:
msg = "Hello, " + name.title() + "!"
print(msg)
usernames = ['xiaomin', 'toy', 'mar']
greet_users(usernames)
\end{lstlisting}
将列表传递给函数后, 函数就可对其进行修改. 在函数中对这个列表所做的任何修改都是\textbf{永久性}的, 即他可以改变原始列表.

\subsection{传递任意数量的实参}
有时候, 我们预先不知道函数需要接受多少个实参, 好在Python允许函数从调用语句中收集任意数量的实参. 为了达到这个目的,在使用形参前加\verb|*para| 但不管调用语句提供了多少实参, 这个形参都将它们包括进去. 例如求和
\begin{lstlisting}[language=python]
def sum(*num):
    s=0
    for i in num:
        s = s+i
    return s
print(sum(1,2,3))
print(sum(1,2,3,4))
\end{lstlisting}
两次调用函数分别传递了三个, 四个参数.