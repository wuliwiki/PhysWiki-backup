% 多元系热力学导引
% keys 多元系|平衡态
\pentry{热力学与统计力学导航\upref{StatMe},吉布斯自由能\upref{GibbsG},热动平衡判据\upref{equcri}}

我们一般将多种物质构成的热力学系统称作\textbf{“多元系统”},而在宏观上均匀的物质被称为\textbf{相}(具体的介绍见相简介(热力学)\upref{PHS})。在热力学中,我们会关心处于平衡态的多元系统的各种有趣性质,例如一些合金在特定的温度和压强下能处于多相共存状态,且这些合金的相图会拥有许多单元复相系所不具有的复杂特征,这涉及到\textbf{多元系的复相平衡}了。此外,对多元系热力学的建立能帮助我们建立\textbf{化学反应}(见化学反应平衡\upref{chemBl})的一般热力学理论,从而可以通过热动平衡判据\upref{equcri}判断化学反应的方向以及计算反应过程的焓变。
\subsection{一些例子}
让我们从几个例子出发来审视多元系问题的复杂之处,我们主要考察处于平衡态的系统\footnote{了解一个热力学系统的最好的方式是先从平衡态入手。}。

设一个开口玻璃管底端有半透膜将管中糖的水溶液与容器内的水隔开,半透膜只让水透过,不让糖透过。实验发现,糖水溶液的液面比容器内水的液面上升一个高度 $h$\footnote{参考相变平衡条件\upref{PhEquv} 的例二}。这个实验事实表明半透膜上方的水压要比下方的水压大 $p-p_0=\rho g h$。这一压强差也被称为渗透压。在压强差存在的情况下,两侧水的化学势“似乎”不再相同了\footnote{事实上我们应该重新审视多元系化学势、压强、温度等强度量的定义和关系式,以及重新思考复相系的平衡条件。}。之前我们得到的单元体系平衡条件很难用来分析这种情况,很难用来解释为什么半透膜的上下表面会产生一个压强差,更别提定量计算了。

一个有趣的例子是\textbf{饱和蒸气压}\upref{VaporP}。实验表明,水的饱和蒸气压随水温度变化而变化(处于气液相变平衡线上,它的方程可以由克拉伯龙方程\upref{Clapey}计算),在这个压强和温度下,蒸气和液体可以平衡共存。但让我们来考虑这样一个问题,如果蒸气中混入了空气,会变成什么情况呢?考虑水蒸气中混入了大量空气,水蒸气的量变的微不足道,而同时维持体系的压强不变。这时系统显然不处于相平衡。那么问题来了,混入了空气后水蒸气的化学势是否改变了\footnote{这里已经在启发我们:我们的确需要审视化学势以及吉布斯函数的概念,这些热力学量在多元体系中将变得很不一样。}?

另一个有趣的例子是\textbf{油水混合物}\footnote{参考相简介(热力学)\upref{PHS} 中的例子。}。如果将油与水倒在同一个杯子中并混合摇匀,会发现静置后杯中的油与水将逐渐分为两层,上层是混入水的油、下层是混入油的水,它们构成了油水混合物的两种液相。这两种不同浓度的油水混合物能够稳定共存而组成平衡的多元复相热力学系统,是一件值得思考的事情,借助多元系的热力学方程,我们能够从理论上去分析多元复相系平衡所要求的条件。

\subsection{多元系热力学函数}

\textbf{保持系统的温度和压强不变},设系统有 $k$ 种组元(例如冰、水、糖是互不相同的组元,混合的氮气和氧气是不同的组员),选 $T,P,n_1,\cdots,n_k$ 为状态参量,$n_i$ 为第 $i$ 个组员的物质的量。三个基本热力学函数为 $V,U,S$——体积、内能、熵:

\begin{equation}\label{eq_mulTh_1}
\begin{aligned}
V=V(T,P,n_1,\cdots,n_k)\\
U=U(T,P,n_1,\cdots,n_k)\\
S=S(T,P,n_1,\cdots,n_k)
\end{aligned}
\end{equation}

根据前面提到的糖水、水蒸气与空气等混合物的例子,我们不能再同以前一样简单地定义化学势(即摩尔吉布斯自由能\upref{GibbsG}),体积、内能、熵也是如此。但我们可以对\autoref{eq_mulTh_1}  这里我们定义\textbf{偏摩尔体积、偏摩尔内能和偏摩尔熵}。

\begin{equation}
\begin{aligned}
v_i=\qty(\frac{\partial V}{\partial n_i})_{T,P,n_j(j\neq i)}\\
u_i=\qty(\frac{\partial U}{\partial n_i})_{T,P,n_j(j\neq i)}\\
s_i=\qty(\frac{\partial S}{\partial n_i})_{T,P,s_j}
\end{aligned}
\end{equation}

下角标的 $T,P,n_j$ 表示在 $T,P,n_j(j\neq i)$ 不变的情况下热力学函数对 $n_i$ 的偏微商。

$V,U,S$ 是关于 $n_1,\cdots,n_k$ 的齐次函数\footnote{当所有的 $n_i$ 扩大 $\lambda$ 倍,系统各组分的比例不变,容易理解其广延量也扩大 $\lambda$ 倍}。所以根据齐次函数欧拉定理,我们有
\begin{equation}
V=\sum_i n_iv_i,U=\sum_i n_iu_i,S=\sum_i n_is_i
\end{equation}

对吉布斯函数\upref{GibbsG} ,可以定义偏摩尔吉布斯函数(称它为 $i$ 组元的化学势),则有:
\begin{equation}\label{eq_mulTh_4}
\mu_i=\qty(\frac{\partial G}{\partial n_i})_{T,P,n_j}
\end{equation}

\begin{equation}\label{eq_mulTh_2}
G=\sum n_i \mu_i
\end{equation}

\subsection{多元系热力学方程}
热力学第一定律告诉我们,在一个可逆过程中 $\dd U=T\dd S-P\dd V$。对于吉布斯函数有类似的方程 $\dd G=-S\dd T+V\dd P$。更好的表达是将化学势包含进去,即 $\dd G=-S\dd T+V\dd P+\mu\dd n$(此时 $\dd U$ 的表达式也要做相应的变换)。对于多元体系,也有类似的关系。

在所有组元的量不发生变化的条件下,我们有
\begin{equation}
\qty(\frac{\partial G}{\partial T})_{P,n_i}=-S,\qty(\frac{\partial G}{\partial P})_{T,n_i}=V
\end{equation}

所以多元系统吉布斯函数的全微分可以写为(注意 $\mu_i$ 的定义已经是偏摩尔吉布斯函数了)

\begin{equation}\label{eq_mulTh_3}
\dd G=-S\dd T+V\dd P+\sum_i\mu_i\dd n_i
\end{equation}
对 \autoref{eq_mulTh_2} 求全微分又可以得到
\begin{equation}
\dd G=\sum_i\mu_i\dd n_i+n_i\dd \mu_i
\end{equation}
两式相减,就得到了著名的\textbf{吉布斯关系}:
\begin{equation}
S\dd T-V\dd p+\sum_i n_i\dd \mu_i=0
\end{equation}
这个公式具有重要意义,它指出,在 $k+2$ 个强度量变数 $T,P,\mu_i(i=1,\cdots,k)$ 中,只有 $k+1$ 个是独立的。基于这个结论,我们能够分析任意多元体系的自由度数量(吉布斯相率\upref{GBPL})。

对于多元复相系,每一相都有它的热力学函数和热力学基本方程,例如 $\alpha$ 相的基本方程为
\begin{equation}
\dd U^\alpha=T^\alpha\dd S^\alpha-p^\alpha\dd V^\alpha+\sum_i\mu_i^\alpha\dd n_i^\alpha
\end{equation}

根据体积、内能、熵和物质的量的广延性质,整个复相系的体积、内能、熵为
\begin{equation}
V=\sum_\alpha V^\alpha,U=\sum_\alpha U^\alpha,S=\sum_\alpha S^\alpha
\end{equation}
