% 函数回顾(高中)
% keys 初中|函数|正比例|反比例|二次
% license Xiao
% type Tutor

\begin{issues}
\issueDraft
\end{issues}

在初中阶段,函数的概念初步展示了变量之间的关系。初中接触的函数主要包括正比例函数、反比例函数和一次函数、二次函数。接下来会在介绍实数和坐标系的概念后,逐一回顾每种函数的特性和相关概念。

\subsection{实数与坐标系}

\textbf{实数(real number)}包含了常见的整数、分数、小数和无理数(如π和√2)。它们既可以是正数,也可以是负数,还包括零。实数的意义在于它们可以代表连续的量,帮助我们精确地描述事物的大小、位置等属性。无论是温度、距离,还是速度,实数都为这些变量的表达提供了基础。

坐标系的引入:在数学中,理解空间位置很重要。平面直角坐标系就是我们在二维空间中定位的工具。这个系统包含两条互相垂直的数轴,横轴叫做x轴,竖轴叫做y轴。通过 x 和 y 的组合,我们可以确定平面上任意一点的位置。例如,点 (3, 2) 表示从原点向右移动 3 个单位,再向上移动 2 个单位的位置。这个坐标系统不仅仅是位置的标记,更是我们表示函数图像的基础。它让我们能够“看见”数字之间的关系,而不仅仅是计算结果。



\subsection{正比例函数}

\textbf{正比例函数(proportional function)}

对应的是一条过原点的直线,

斜率


\subsection{反比例函数}

、\textbf{反比例函数(inversely proportional function)}
对应的是两条双曲线。你还学过如何根据反比例函数的表达式,通过已知的点来求解函数的值。

中心对称性

矩形面积相同


\subsection{一次函数}

截距、\textbf{一次函数(linear function)}

\subsection{二次函数}

二次函数(quadratic function)的图像可能与  x  轴有两个交点,并且具有一个对称轴和一个最低点。

轴对称性:图像上对称点到对称轴的距离相等,且连线与对称轴垂直。

关于二次函数和一元二次方程的关系参见\enref{因式分解与一元二次方程}{quasol}。