% Python 解释器
% Python|解释器

\begin{issues}
\issueTODO
\end{issues}

\pentry{计算机语言 — 脚本语言(解释型语言)\upref{CpLgg}}

Python解释器由编译器和虚拟机构成,编译器将源代码转换成字节码,然后再通Python虚拟机来逐行执行这些字节码.

python程序执行过程:

\begin{enumerate}
\item 执行 .py 文件,就会启动python解释器

\item 编译器将源文件解释成字节码

\item 虚拟机将字节码转化成机器语言,与操作系统交互

\item 程序运行结束后,将字节码存到pyc文件,便于后续直接执行
\end{enumerate}

python解释器有很多种:

\begin{itemize}
\item CPython:C语言开发,使用最广,默认的解释器

\item IPython:基于CPython之上的交互式解释器

\item PyPy:采用JIT技术,对python代码进行动态编译,追求执行速度

\item Jython:运行在Java平台上的解释器,可以直接编译成Java字节码执行
IronPython:同理Jython,运行在 .Net 平台上
\end{itemize}
