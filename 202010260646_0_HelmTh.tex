% 亥姆霍兹定理
% keys 散度|旋度|调和场|亥姆霍兹

\begin{issues}
\issueDraft
\end{issues}

参考 \cite{GriffE}

任何矢量场都可以分解为一个无旋场 $\bvec F_{div}$ 和一个无散场 $\bvec F_{curl}$, 也可以选择性地添加一个调和场 $\bvec h$.% 链接未完成
\begin{equation}
\bvec F(\bvec r) = \bvec F_{d}(\bvec r) + \bvec F_{c}(\bvec r) + \bvec h(\bvec r)
\end{equation}
一种分解的计算方法是
\begin{equation}\label{HelmTh_eq1}
\bvec F_{d} = \frac{1}{4\pi}\int \frac{(\div \bvec F)\bvec R}{R^3} \dd{V'}
\end{equation}
\begin{equation}
\bvec F_{c} = \frac{1}{4\pi}\int \frac{(\curl \bvec F)\cross \bvec R}{R^3} \dd{V'}
\end{equation}
以上两式可以分别类比静电学中的库伦定律(\autoref{Efield_eq9}~\upref{Efield})和静磁学中的比奥萨法尔定律(\autoref{BioSav_eq3}~\upref{BioSav}): 把 $\bvec F_c$ 看作电场, 满足高斯定理(\autoref{EGauss_eq1}~\upref{EGauss}); 把 $\bvec F_c$ 看作磁场, 满足安培环路定理(\autoref{AmpLaw_eq2}~\upref{AmpLaw}).

他们显然满足无散无旋的条件. 但是还需要证明 $\bvec F - \bvec F_{div} - \bvec F_{curl}$ 是一个调和场.
(未完成)

无旋场总能表示为某个标量函数 $V(\bvec r)$ 的梯度(证明见势能\upref{V}), 而无散场总能表示为另一个矢量场 $\bvec G$ 的旋度\upref{HlmPr2}, 所以
\begin{equation}
\bvec F_{d} = \grad V\qquad \bvec F_{c} = \curl \bvec G
\end{equation}
调和场 $\bvec h(\bvec r)$ 既没有散度也没有旋度, 所以可以合并到前两项中任意一个中. 所以亥姆霍兹分解可以记为
\begin{equation}
\bvec F = \grad V + \curl \bvec A
\end{equation}


