% 拉格朗日方程的证明、达朗贝尔定理
% 拉格朗日方程证明|广义力|非约束力|达朗贝尔定理

\pentry{拉格朗日方程\upref{Lagrng}}

以下使用牛顿第二定律证明拉格朗日方程, 其中会使用到一个关于约束的定理叫做达朗贝尔定理

% 首先可以用牛顿定律来推导,然后再用最小作用量推导
\subsection{由牛顿第二定律证明拉格朗日方程}
注意以下所有的函数偏导都是把 $q_1, q_2\dots q_N,\ \dot q_1, \dot q_2\dots \dot q_N,\ t$ 作为变量,即对一个变量求导数而把其他变量看做常数.另外,势能 $V$ 和位矢 $\bvec r_j$ 与 $\dot q_i$ 无关,偏导为 0.

把系统看成质点组, 每个质点质量为 $m_j$, 位置矢量是广义坐标的函数 $\bvec r_j(q_1,\dots,q_N,t)$, 那么系统动能为
\begin{equation}
T = \frac12 \sum_j m_j \bvec v_j^2 = \frac12 \sum_j m_j \dot{\bvec r}_j \vdot \dot{\bvec r}_j
\end{equation}
由矢量内积的求导法则
\begin{equation}\label{dAlbt_eq5}
\pdv{T}{\dot q_i} = \sum_j m_j \dot{\bvec r}_j \vdot \pdv{\dot{\bvec r}_j}{\dot q_i}
\end{equation}
其中质点 $j$ 的速度可以用全导数\upref{TotDer} 公式
\begin{equation}
\dot{\bvec r}_j = \sum_k \pdv{\bvec r_j}{q_k} \dot q_k  + \pdv{\bvec r_j}{t}
\end{equation}
对 $\dot q_i$ 求偏导,注意位矢与$\dot q_i$ 无关,所以求偏导时 $\pdv*{r_j}{q_k}$ 与 $\pdv*{r_j}{t}$ 可看做常数.
\begin{equation}\label{dAlbt_eq27}
\pdv{\dot{\bvec r}_j}{\dot q_i} = \pdv{\bvec r_j}{q_i}
\end{equation}
代入\autoref{dAlbt_eq5} 并对时间求导得到拉格朗日方程的左边(我们暂时只讨论 $\pdv*{V}{\dot q_i} = 0$ 的情况)
\begin{equation}\label{dAlbt_eq8}
\dv{t} \pdv{T}{\dot q_i} = \sum_j m_j \ddot{\bvec r}_j \vdot \pdv{\bvec r_j}{q_i}  + \sum_j m_j \dot{\bvec r}_j \vdot \dv{t} \pdv{\bvec r_j}{q_i}
\end{equation}
拉格朗日方程的右边为
\begin{equation}
\pdv{L}{q_i} = \pdv{T}{q_i} - \pdv{V}{q_i}
\end{equation}
其中右边第一项为
\begin{equation}\label{dAlbt_eq28}
\pdv{T}{q_i} = \sum_j m_j \dot{\bvec r}_j \vdot \pdv{\dot{\bvec r}_j}{q_i} = \sum_j m_j \dot{\bvec r}_j \vdot \pdv{q_i} \dv{\bvec r_j}{t}
\end{equation}
第二项被定义为\textbf{广义力}
\begin{equation}\label{dAlbt_eq29}
Q_i = - \pdv{V}{q_i} = \sum_j \qty(-\pdv{V}{x_j} \pdv{x_j}{q_i} - \pdv{V}{y_j}\pdv{y_j}{q_i} - \pdv{V}{z_j} \pdv{z_j}{q_i}) = \sum_j \bvec F_j^{(a)} \vdot \pdv{\bvec r_j}{q_i}
\end{equation}
其中
\begin{equation}
\bvec F_j^{(a)} = - \grad_j V = -\pdv{V}{x_j}\uvec x - \pdv{V}{y_j}\uvec y - \pdv{V}{z_j} \uvec z
\end{equation}
被称为\textbf{非约束力}.%未完成,最好以上的例子中能说明非约束力
所以要证明拉格朗日方程,即证明\autoref{dAlbt_eq8} 等于\autoref{dAlbt_eq28} 加\autoref{dAlbt_eq29}, 首先需要证明
\begin{equation}
\dv{t} \pdv{\bvec r_j}{q_i} = \pdv{q_i} \dv{\bvec r_j}{t}
\end{equation}
也就是证明全导数和偏导数运算可对易.使用全导数\upref{TotDer} 的定义,以及混合偏导\upref{ParDer} 的性质,有
\begin{equation}
\dv{t} \pdv{\bvec r_j}{q_i} = \sum_k \pdv{q_k} \pdv{\bvec r_i}{q_i} \dot q_k  + \pdv{t} \pdv{\bvec r_j}{q_i} = \sum_k \pdv{q_i} \pdv{\bvec r_i}{q_k} \dot q_k + \pdv{q_i} \pdv{\bvec r_j}{t} = \pdv{q_i} \dv{\bvec r_j}{t}
\end{equation}
然后我们需要证明
\begin{equation}
\sum_j \qty(\bvec F_j^{(a)} - m\ddot{\bvec r}_j) \vdot \pdv{\bvec r_j}{q_i}  = 0
\qquad (i = 1\dots N)
\end{equation}
即可证明拉格朗日方程.该式被称为\textbf{达朗贝尔定理}.注意由于这里的 $\bvec F_j^{(a)}$ 为质点 $j$ 所受的非约束力而不是合力,所以求和项的小括号一般不为 0.

\subsection{达朗贝尔定理证明}
令第 $j$ 个质点所受和力为 $\bvec F_j = \bvec F_j^{(a)} + \bvec F_j^{(c)}$, 两项分别为非约束力和约束力.由牛顿第二定律 $\bvec F_j - m\ddot{\bvec r}_j = 0$, 所以
\begin{equation}
\sum_j \qty(\bvec F_j^{(a)}+\bvec F_j^{(c)} - m\ddot{\bvec r}_j) \vdot \pdv{\bvec r_j}{q_i} = 0
\qquad (i = 1\dots N)
\end{equation}
现在我们只需证明
\begin{equation}
\sum_j  \bvec F_j^{(c)} \vdot \pdv{\bvec r_j}{q_i}  = 0
\qquad (i = 1\dots N)
\end{equation}
由于以上偏微分中时间保持不变,约束力不做功,该求和为零,证毕.%未完成: 这样的解释连我都不明白... 真的要结合有约束的例题来解释了
