% 浙江大学 1999 年 考研 量子力学
% license Usr
% type Note

\textbf{声明}:“该内容来源于网络公开资料,不保证真实性,如有侵权请联系管理员”

\subsection{第一题:(10分)}
\begin{enumerate}
  \item 试求出 100 ev 的自由电子及能量为 0.1 ev,质量为 1 克的质点的德布罗意波长

  \[  \text{lev} = 1.6 \times 10^{-19} \\ \text{J}, \ h = 6.6 \times 10^{-34} \\ \text{J.s}~\]

  \item 证明一个自由运动的微观粒子对应的德布罗意群速度 $v_g$,即为其运动速度 $v$。
\end{enumerate}
\subsection{第二题:(10分)}
(1)证明定态中几率流密度与 时间无关。

(2)求一维无限深势阱中运动的粒子在第$n$个能级时的几率流密度。
\subsection{第三题:(15分)}
(粒子处于一维势阱$V(x) = \begin{cases}    \infty, & x < 0 \\\\    -U_0, & 0 \leq x \leq a \\\\    0, & x > a  \end{cases}$(取的恒定常量)中运动,)

\begin{enumerate}
  \item 面出势能 $V(x)$ 的示意图,设粒子质量为 $\mu$。
  \item 求解粒子的能级 $E$。($-U_0 < E < 0$) (写出 $E$ 所满足的方程。)
\end{enumerate}
\subsection{第四题:(10分)}
一维谐振子,其势能为:$V(x) = \frac{1}{2} k x^2$
($k$ 为常量)。若该谐振子又受一恒力 $F$ 作用,试求其本征能量及能量本征函数。该振子的质量为 $\mu$。
\subsection{第五题:(20分)}
\begin{enumerate}
  \item 写出线性厄米算符的定义。
  
  \item 判断下列算符中,哪一个是线性厄米算符?
  
  $ a. \hat{F}_1 = -i \hbar \frac{\partial}{\partial x}, \quad b. \hat{F}_2 = a p_x + b x,$ ($a, b$ 为恒定实常数) 
  $$c. \hat{F}_3 = e^{i\hat A}$,&\hat A \\text{为线性厄米算符}, i \\text{为虚宗量})  \\]
  
  \item 证明厄米算符对应的本征值得实值。
  
  \item 若算符 $\hat{B}$、$\hat{C}$ 为厄米算符,$[\hat{B}, \hat{C}]_1 = \hat{B} \hat{C} - \hat{C} \hat{B} = 0$。若在 $b, c$ 分别为 $\hat{B}, \hat{C}$ 的本征值,
  
  证明:$ b c = 0$若 $c^2 = 1$,则 $c$ 必取 $c = \pm 1$。
\end{enumerate}