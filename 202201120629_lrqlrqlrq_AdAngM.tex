% 角动量的叠加原理

假设氢原子处于基态(这样就不用考虑轨道角动量),其电子和质子都是自旋为$1/2$的粒子.这两个粒子都分别可以自旋向上或者自旋向下,也就是说由四种自旋的可能性:
\begin{equation}
\uparrow\uparrow;\ \uparrow\downarrow;\ \downarrow\uparrow;\ \downarrow\downarrow
\end{equation}
其中$\chi^{(1)}$也就是第一个箭头代表电子自旋,第二个\chi^{(2)}代表质子自旋.那么问题来了,这个原子的总角动量是什么?