% 约翰·蓝道尔(综述)
% license CCBYSA3
% type Wiki

本文根据 CC-BY-SA 协议转载翻译自维基百科 \href{https://en.wikipedia.org/wiki/John_Randall_(physicist)}{相关文章}。

\begin{figure}[ht]
\centering
\includegraphics[width=6cm]{./figures/fc0ea7384a703f1d.png}
\caption{} \label{fig_YHldr_1}
\end{figure}
约翰·特顿·兰德尔爵士(Sir John Turton Randall,FRS FRSE,\(^\text{[2]}\)1905年3月23日-1984年6月16日)是一位英国物理学家和生物物理学家,以对腔体磁控管的重大改进而闻名。腔体磁控管是厘米波雷达的重要组成部分,也是盟军在第二次世界大战中取得胜利的关键技术之一,同时也是微波炉的核心部件。\(^\text{[3][4]}\)

兰德尔与哈里·布特合作,制造出一种能够以10厘米波长发射微波脉冲的电子管。\(^\text{[3]}\)对于他们发明的重要性,不列颠哥伦比亚大学维多利亚分校军事史教授大卫·齐默曼指出:“磁控管依然是所有类型短波无线电信号的关键电子管。它不仅通过使我们能够开发机载雷达系统改变了战争进程,它仍然是今天微波炉核心的关键技术。腔体磁控管的发明改变了世界。”\(^\text{[3]}\)

兰德尔还曾领导伦敦国王学院的团队研究DNA的结构。他的副手莫里斯·威尔金斯与剑桥大学卡文迪许实验室的詹姆斯·沃森和弗朗西斯·克里克共同因DNA结构的解析获得了1962年诺贝尔生理学或医学奖。他的其他团队成员还包括罗莎琳德·富兰克林、雷蒙德·戈斯林、亚历克斯·斯托克斯和赫伯特·威尔逊,他们都参与了DNA结构研究。
\subsection{教育与早年生活}
约翰·兰德尔于1905年3月23日出生在兰开夏郡纽顿利威洛斯,是悉尼·兰德尔(Sidney Randall,苗圃及种子商人)与其妻汉娜·考利(Hannah Cawley,约翰·特顿[John Turton]之女,后者是当地煤矿经理)唯一的儿子,也是三名孩子中的长子。\(^\text{[2]}\)他在阿什顿因梅克菲尔德(文法学校和曼彻斯特维多利亚大学接受教育,1925年获得物理学一级荣誉学位和毕业奖学金,1926年获得理学硕士学位。\(^\text{[2]}\)

1928年,他与多丽丝·达克沃斯结婚。
\subsection{职业与研究}
1926年至1937年间,兰德尔在通用电气公司位于温布利的实验室从事研究工作,他在为放电灯研发发光粉方面发挥了重要作用。 他还积极研究这些发光机制。\(^\text{[2]}\)

到1937年,他已被公认为英国该领域的领先研究者,并获得了伯明翰大学英国皇家学会奖学金,在马克·奥利芬特领导的物理系与莫里斯·威尔金斯合作,研究磷光的电子陷阱理论。\(^\text{[5][6][7][8]}\)
\subsubsection{磁控管}
\begin{figure}[ht]
\centering
\includegraphics[width=6cm]{./figures/6040822fe73927cf.png}
\caption{伯明翰大学彭廷物理楼} \label{fig_YHldr_2}
\end{figure}
当战争在1939年爆发时,英国海军部找到奥利芬特,探讨是否有可能建造一种能在微波频率下工作的无线电发射源。这类系统能够使雷达探测到小型物体,例如潜艇的潜望镜。当时在萨福克海岸博兹西庄园的空军部雷达研究人员也表达了对10厘米波段系统的兴趣,因为这将大幅减小发射天线的尺寸,使其能够更轻松地安装在飞机机头,而不像现有系统那样需要安装在机翼和机身上。\(^\text{[9]}\)
\begin{figure}[ht]
\centering
\includegraphics[width=6cm]{./figures/b978444bc8af013b.png}
\caption{原始的六腔磁控管} \label{fig_YHldr_3}
\end{figure}
奥利芬特开始使用速调管开展研究,这是一种由拉塞尔和西格德·瓦里安在1937年至1939年间发明的设备,也是当时唯一已知能够高效产生微波的系统。然而,当时的速调管功率非常低,奥利芬特的工作重点是大幅提高其输出功率。如果这项工作成功,就会产生一个次要问题:速调管仅是放大器,因此需要一个低功率的信号源供其放大。奥利芬特指派兰德尔和哈里·布特解决微波振荡器的研制问题,让他们探索使用微型巴克豪森–库尔茨管作为信号源,这种设计已用于超高频(UHF)系统。他们的研究很快表明,这种管在微波频率范围内无法带来改进。\(^\text{[10]}\)速调管研究很快遇到瓶颈,研制出的设备只能产生约400瓦的微波功率,这仅够用于测试,但远远达不到用于实际雷达系统所需的数千瓦功率水平。

1939年11月,由于没有其他项目需要负责,兰德尔和布特开始考虑这一问题的解决方案。当时已知的另一种微波设备是分阳极磁控管,这种设备能够产生少量功率,但效率低下,且输出通常比速调管更低。然而,他们注意到这种设备相较于速调管有一个巨大优势:速调管的信号依赖于电子枪产生的电子束流,而电子枪的电流能力决定了设备最终能处理的功率上限。相比之下,磁控管使用的是普通的热阴极灯丝阴极,这种系统已被广泛用于产生数百千瓦功率的无线电系统中。这似乎为实现更高功率提供了更可行的途径。\(^\text{[10]}\)

现有磁控管的问题不在于功率,而在于效率。在速调管中,电子束会穿过一个金属盘状谐振腔,谐振腔的机械结构会影响电子,使其加速和减速,从而释放出微波。这种方法效率较高,功率受限于电子枪。而在磁控管中,谐振腔被两块带相反电荷的金属板取代,通过磁场使电子在板间运动以实现交替加速。这种方式在理论上对可加速电子数量没有限制,但微波释放过程的效率极低。
