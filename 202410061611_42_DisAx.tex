% 分离性公理成立的充要条件
% keys 分离性公理|充要条件
% license Usr
% type Tutor

\pentry{分离性\nref{nod_Topo5}}{nod_cf78}
拓扑空间是较度量空间更一般的对象,很多度量空间的概念可以拓广到拓扑空间。然而正因为这一一般性,拓扑空间会出现本质上不同于度量空间的情况。例如,有限点集可能不是闭的,收敛序列的极限点可能不唯一等等。为了获得和度量空间更相近的空间来,需要添加一些补充条件。分离性公理\upref{Topo5}就是人们提出的一类重要条件,它衡量了拓扑空间中点的分离程度。本节将给出其中几个分离性公理成立的充要条件。

\subsection{分离性公理成立的充要条件}
第一分离性公理 $T_1$ 是指对拓扑空间中的任意两个不同点,每一点都存在不包含另一点的邻域。这样的拓扑空间被称为\textbf{ $T_1$ 空间}。
\begin{theorem}{$T_1$ 分离性公理成立的充要条件}
第一分离性公理 $T_1$ 成立的充要条件是:空间中任意单点集都是闭的。
\end{theorem}
\textbf{证明:}
\textbf{必要性:}假设第一分离性公理成立,设 $x\neq y$,则存在 $y$ 的邻域 $O_y$,使得
\begin{equation}
x\notin O_y~.
\end{equation}
由闭包的定义,$y\notin [x]$\footnote{为了方便,本文约定单点集 $\{x\}$ 直接写为 $x$。},即任意不同于 $x$ 的点都不在 $[x]$ 中,因此 $[x]=x$。因此任意单点集都是闭的。

\textbf{充分性:}假设任意单点集都是闭的。设 $x\neq y$,则那么 $x$ 的补集 $C(x)$ 是开的,并且 $y\in C(X),x\notin C(X)$,于是 $C(x)$ 是 $y$ 的满足 $x\notin C(x)$ 的邻域。同理 $C(y)$ 是 $x$ 的满足 $y\notin C(y)$ 的邻域。因此第一分离性公理成立。

\textbf{证毕!}



第三分离性公理 $T_3$ 是指任一点和不包含它的开集都各自有一邻域存在,使得两邻域不相交。包含一个集合的邻域是指这个邻域包含一个包含该集合的开集。


\begin{theorem}{$T_3$ 分离性公理成立的充要条件}
第三分离性公理 $T_3$ 成立的充要条件是:任意单点集 $x$ 和其邻域 $O_x$,都存在 $x$ 的邻域 $ O'_x$,使得 $[O'_x]\subset O_x$。
\end{theorem}

\textbf{证明:}
\textbf{必要性:}假设 $T_3$ 分离性公理成立。则对任意 $x$ 和其邻域 $O_x$,存在开集 $U\subset O_x$ 包含 $x$。于是 $C(U)$ 是闭的,因此由 $T_3$ 公理,存在开集 $U_x, O$,使得
\begin{equation}
x\in U_x,\quad C(U)\subset O,\quad U_x\cap O=\emptyset~.
\end{equation}
因此
\begin{equation}
U_x\subset C(O)=[C(O)]\subset U\subset O_x.~
\end{equation}
$U_x$ 刚好是所需要的 $O_x'$。

\textbf{充分性:} 假设任意单点集 $x$ 和其邻域 $O_x$,都存在 $x$ 的邻域 $ O'_x$,使得 $[O'_x]\subset O_x$。设 $x$ 是任意点,$A$ 是不包含 $x$ 的任一闭集。那么 $C(A)$ 是包含 $x$ 的邻域。于是由假设,存在 $x$ 的邻域 $O$,使得 
\begin{equation}
[O]\subset C(A)~.
\end{equation}
于是 $C([O])$ 是包含 $A$ 的邻域。且 $C([O])\cap O=\emptyset $。显然 $O,C([O])$ 正是所需要的。
\textbf{证毕!}









\textbf{证明:}


\textbf{证毕!}




