% 原子符号
% keys 原子符号|角动量|泡利不相容|量子数

\pentry{电子轨道与元素周期表\upref{Ptable}}

参考 Griffiths 量子力学简介, 2.2 The Periodic Table

Term symbol 描述了原子的角动量状态
\begin{equation}
^{2S + 1} L_J
\end{equation}
其中 $L$ 表示总轨道角动量(量子数), 例如可以是 $S, P, D, F \dots$, $S$ 是总自旋角动量, $2S + 1$ 叫做 multiplicity. $J$ 是总角动量, 包括轨道和自旋.

\subsection{Hund's Rule}

\begin{enumerate}
\item 
\item 
\item 
\end{enumerate}

另外有一种符号例如碳原子($Z = 6$)是 $(1s)^2 (2s)^2 (2p)^2$ 不知道叫什么.

氦原子的基态是 $(1s)^2$, 总轨道角动量 $L = 0$. 由泡利不相容原理, 两个电子自旋方向相反(singlet, 反对称, $S = 0$), 所以波函数是对称的, 符号为 $^1 S_0$.
