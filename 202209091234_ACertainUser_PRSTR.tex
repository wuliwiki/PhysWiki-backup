% 主应力

%书本不在手边,这篇文章主要是按个人理解和笔记写的,可能有所疏漏,需要dalao指正.
\begin{issues}
\issueDraft
\end{issues}
\pentry{应力\upref{STRESS}}

假设我们有一个受力状况相对复杂的二维微元体,我们能不能改变划分微元体的方式,从而简化他的受力?
\begin{figure}[ht]
\centering
\includegraphics[width=5cm]{./figures/PRSTR_1.png}
\caption{二维微元体.仿自P. Beer的Mechanics of Materials} \label{PRSTR_fig1}
\end{figure}

答案难得的是...可以的.在一点处,我们总能够找到一种选取微元体的方式,使其只受正应力而不受切应力.
\begin{figure}[ht]
\centering
\includegraphics[width=6cm]{./figures/PRSTR_2.png}
\caption{通过改变选取微元体的方式,使其只受正应力而不受切应力} \label{PRSTR_fig2}
\end{figure}
在这种情况下,这些正应力也被称为主应力(Principal Stress). 可见,应力的大小与微元体的选取方式有关,而主应力的大小则不然.因此,某种意义上,主应力比单纯的应力更具有

现在,眼前最显然的一个问题是,如果已知了微元体的受力情况,如何计算出他的主应力呢?方法是多种多样的,这里主要介绍最常见的两种,Mohr圆与特征值法.
