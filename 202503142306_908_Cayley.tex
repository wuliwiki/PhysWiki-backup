% 阿瑟·凯利(综述)
% license CCBYSA3
% type Wiki

本文根据 CC-BY-SA 协议转载翻译自维基百科\href{https://en.wikipedia.org/wiki/Arthur_Cayley}{相关文章}。

\begin{figure}[ht]
\centering
\includegraphics[width=6cm]{./figures/54015c4f6823f2b0.png}
\caption{} \label{fig_Cayley_1}
\end{figure}
阿瑟·凯利(Arthur Cayley,FRS,1821年8月16日-1895年1月26日)是英国数学家,主要从事代数方面的研究。他帮助创立了现代英国纯数学学派,并在剑桥大学三一学院担任教授长达35年。

他提出了如今被称为凯利-哈密顿定理的观点——即每个方阵都是其自身特征多项式的根,并验证了2阶和3阶矩阵的情况。[1]他是第一个定义抽象群概念的人,抽象群是满足某些运算规律的集合,[2]区别于艾瓦里斯特·伽罗瓦(Évariste Galois)对置换群的定义。在群论中,凯利表、凯利图以及凯利定理都以他命名,而在组合数学中,也有凯利公式以纪念他。
\subsection{早年生活}  
阿瑟·凯利于1821年8月16日出生在英国伦敦的里士满。他的父亲亨利·凯利是航空工程师乔治·凯利的远亲,来自约克郡的一个古老家族,并作为商人定居在俄罗斯圣彼得堡。他的母亲是玛丽亚·安托尼娅·道蒂(Maria Antonia Doughty),威廉·道蒂的女儿。根据一些作家的说法,她是俄罗斯人,但她父亲的名字表明她有英格兰血统。他的兄弟是语言学家查尔斯·巴戈特·凯利。阿瑟在圣彼得堡度过了他生命的最初八年。1829年,他的父母定居在伦敦的布莱克希斯,阿瑟在那里上了一所私立学校。

14岁时,他被送到国王学院学校。年轻的凯利喜欢复杂的数学问题,学校的校长注意到他在数学方面的天赋,并建议父亲不要让儿子从事商业工作,而是送他去剑桥大学学习。
\subsection{教育}  
17岁时,凯利开始在剑桥大学三一学院学习,在那里他在希腊语、法语、德语、意大利语以及数学方面表现出色。此时,分析学会的事业已取得胜利,剑桥数学杂志由格雷戈里(Gregory)和罗伯特·莱斯利·埃利斯(Robert Leslie Ellis)创办。20岁时,凯利向该杂志投稿了三篇论文,这些论文的主题受到了他阅读约瑟夫·路易·拉格朗日的《解析力学》和拉普拉斯的一些著作的启发。

凯利在剑桥的导师是乔治·皮科克(George Peacock),他的私人教练是威廉·霍普金斯(William Hopkins)。他通过赢得高级学者(Senior Wrangler)和第一个史密斯奖学金(Smith's Prize)顺利完成了本科课程。接下来,他计划获得硕士学位,并通过竞争考试赢得奖学金。[3]他继续在剑桥大学居住了四年,在此期间,他接收了一些学生,但他的主要工作是为数学杂志准备28篇论文。
\subsection{法律事业}  
由于奖学金的任期有限,他必须选择一个职业;像德·摩根一样,凯利选择了法律,并于1846年4月20日以24岁的年龄被录取为林肯律师协会(Lincoln's Inn)的成员。[4]他专攻不动产转让。在参加律师资格考试期间,他曾前往都柏林聆听威廉·罗文·汉密尔顿关于四元数的讲座。

他的朋友J.J. 西尔维斯特(J. J. Sylvester),比他大五岁,曾在剑桥大学与他为同学,当时是一名常驻伦敦的精算师;他们常一起在林肯律师协会的院子里散步,讨论不变量和协变理论。在这十四年间,凯利创作了大约二三百篇论文。[5]
\subsection{教授职位}  
大约在1860年,剑桥大学的卢卡谢恩数学教授席位(牛顿的讲座)被新的萨德莱尔教授职位所补充,该职位是由萨德莱夫人遗赠的基金资助的,42岁的凯利成为第一位担任此职的人。他的职责是“阐明和教授纯数学的原理,并致力于该科学的进步。”他放弃了收入丰厚的法律事业,选择了一个 modest 的工资,但他从未后悔这一决定,因为它使他能够将精力投入到自己最喜爱的事业中。他立即结婚并定居在剑桥,并且(与哈密尔顿不同)享受着非常幸福的家庭生活。曾经,大学时期的朋友西尔维斯特表达了对凯利宁静家庭生活的羡慕,而未婚的西尔维斯特则终其一生不得不与世界作斗争。

最初,萨德莱尔教授的薪水仅足够在学年中的一个学期讲课,但1886年的大学财政改革为其讲座提供了资金,使得讲座可以扩展至两个学期。在许多年里,他的课程仅有几位完成考试准备的学生参加,但改革后,出席人数约为十五人。他通常会讲解他当前的研究课题。

至于他在数学科学进步方面的职责,他出版了一系列漫长而富有成效的论文,涵盖了纯数学的各个领域。他还成为了许多国内外学会的常驻评审,负责评估数学论文的价值。

1872年,他被授予三一学院荣誉院士称号,三年后成为正式院士,担任有薪职位。大约在这个时候,他的朋友们为他捐款,送上了一幅肖像画。麦克斯韦写了一篇致辞,赞扬凯利的主要著作,包括《n维解析几何章节》;《行列式理论》;《矩阵理论论文》;《偏斜曲面的论文,也叫做螺旋曲面》;以及《三次螺旋曲线的描绘》。[6] 

除了在代数方面的工作外,凯利还对代数几何作出了基础性的贡献。凯利和萨尔蒙发现了立方曲面上的27条直线。凯利构造了所有投影三维空间曲线的周空间。[7]他创立了有规律曲面的代数几何理论。他在组合数学方面的贡献包括通过生成函数的开创性使用,计算了\(n\)个标记顶点上的\(n^{n-2}\)棵树。

1876年,他出版了《椭圆函数论》一书。他对女性大学教育运动表现出浓厚的兴趣。在剑桥,第一所女子学院是吉尔顿学院和纽纳姆学院。在吉尔顿学院的早期,他直接参与了教学工作,并且在许多年里,他担任了纽纳姆学院理事会主席,对学院的进展一直保持着浓厚的兴趣,直至最后。

1881年,他收到了当时麦克斯韦担任剑桥大学数学教授的约翰斯·霍普金斯大学(位于巴尔的摩)的邀请,邀请他进行一系列讲座。他接受了邀请,并于1882年上半年在巴尔的摩讲授关于阿贝尔函数和θ函数的主题。

1893年,凯利成为荷兰皇家艺术与科学院的外籍会员。[8]
\subsection{英国学会会长}  
1883年,凯利成为了英国科学促进会的会长。会议在英格兰北部的南港举行。由于会长的致辞是会议中的重要公众活动之一,并吸引了具有一般文化背景的听众,因此通常会尽量避免使用过于技术性的内容。凯利(1996年)以《纯数学的进展》为主题进行致辞。
\subsection{《论文集》}  
1889年,剑桥大学出版社开始出版他的论文集,这让他十分欣慰。他亲自编辑了其中七卷四开本,尽管当时正饱受严重的内疾折磨。他于1895年1月26日去世,享年73岁。他的葬礼在三一学院教堂举行,英国的顶尖科学家们出席了仪式,甚至来自俄罗斯和美国的官方代表也前来悼念。  

其余的论文由他的继任者、萨德利里安数学教授安德鲁·福赛斯编辑,总共十三卷四开本,收录了967篇论文。他的研究成果至今仍被广泛使用,仅在21世纪就已被200多篇数学论文引用。  

凯莱至生命最后一刻仍钟爱小说阅读和旅行。他对绘画和建筑尤为喜爱,并练习水彩画,这在绘制数学图示时偶尔派上用场。
\subsection{遗产}
凯莱被安葬在剑桥的米尔路公墓。  

由洛厄斯·凯托·狄金森(Lowes Cato Dickinson)于1874年所绘的凯莱肖像以及由威廉·朗梅德(William Longmaid)于1884年所绘的肖像,现收藏于剑桥大学三一学院。