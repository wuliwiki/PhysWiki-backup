% 基本知识(常微分方程)
% keys 常微分方程|小结

\pentry{常微分方程简介\upref{ODEint}}
在常微分方程简介\upref{ODEint}一节中,可以知道,如果一个方程未知的是函数,且方程含有未知函数的导数,则称这样的方程为\textbf{微分方程}.若微分方程中的未知函数是一元函数(即未知函数仅有一个自变量),则称着方程为\textbf{常微分方程};相反的,若未知函数是多元函数,则称\textbf{偏微分方程}.所以,在自变量为 $x$,且仅有一个未知函数 $y$ 时,一般的常微分方程形式为
\begin{equation}
F\qty(x,y,y',\cdots,y^{(n)})=0
\end{equation}
其中, $y^{(n)}=\dv[n]{y}{x}$.

常微分方程理论通常也研究更一般的方程组.通常,常微分方程组中方程的个数与其中出现的未知函数个数相同,而所有未知函数都是同一自变量的函数.所以,一般的常微分方程组形式为
\begin{equation}
\left\{\begin{aligned}
&F_1\qty(x,y_1,y_1',\cdots,y_1^{(n_1)},\cdots,y_n,y_n',\cdots,y_n^{(n_m)})=0\\
\vdots\\
&F_n\qty(x,y_1,y_1',\cdots,y_1^{(n_1)},\cdots,y_n,y_n',\cdots,y_n^{(n_m)})=0
\end{aligned}\right.
\end{equation}
这里,$y_i$ 都是 $x$ 的未知函数,函数 $F_i$ 是 $\sum\limits_{j=1}^{m}n_j+n+1$ 个变量的函数. 