% 理想气体的状态密度(相空间)
% 理想气体|状态密度|球体体积|相空间

\begin{issues}
\issueDraft
\end{issues}

\pentry{$N$ 维球体的体积\upref{NSphV}}

相空间中能量小于 $E$ 的状态数
\begin{equation}\label{IdSDp_eq2}
\Omega_0 = \frac{V^N}{N! h^{3N}} \frac{(2\pi mE)^{3N/2}}{\Gamma(3N/2+1)}
\end{equation}
关于能量的状态密度函数为
\begin{equation}\label{IdSDp_eq3}
g(E) = \frac{V^N}{N! h^{3N}} \frac{(2\pi m)^{3N/2}}{\Gamma(3N/2)} E^{3N/2 - 1}
\end{equation}
关于动量绝对值的状态密度函数为
\begin{equation}
g(p) = \frac{V^N}{N! h^{3N}} \frac{2\pi^{3N/2}}{\Gamma(3N/2)} p^{3N - 1}
\end{equation}


\subsection{推导}
在 $N$ 个独立的,可区分粒子的相空间中, 能量小于 $E$ 的状态数(体积除以 $h^{3N}$, 无量纲)为
\begin{equation}\label{IdSDp_eq1}
\Omega_0 = \frac{1}{N! h^{3N}} \int\limits_{\sum p^2 \leqslant 2mE} \dd[3N]{q} \dd[3N]{p} = \frac{V^N}{N! h^{3N}} \int\limits_{\sum p^2 \leqslant 2mE} \dd[3N]{p}
\end{equation}
其中积分 $\int_{\dots} \dd[3N]{p} $ 可以看做 $n=3N$ 维球体的体积, 半径为 $R = \sqrt{2mE}$. 

$n$ 维空间中球体的体积\upref{NSphV}为
\begin{equation}
V_n = \frac{\pi^{n/2}}{\Gamma(n/2+1)}R^n
\end{equation}
代入 $n=3N$ 和 $R = \sqrt{2mE} $, 得
\begin{equation}
\int\limits_{\sum p^2 \leqslant 2mE} \dd[3N]{p} = \frac{(2\pi mE)^{3N/2}}{\Gamma(3N/2+1)}
\end{equation}
代入\autoref{IdSDp_eq1} 得\autoref{IdSDp_eq2}. 再对 $E$ 求导得\autoref{IdSDp_eq3}.
