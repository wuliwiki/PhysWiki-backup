% 洛伦兹力
% keys 磁场|点电荷|洛伦兹力|磁场力

\pentry{磁场\upref{MagneF}}
\begin{figure}[ht]
\centering
\includegraphics[width=8cm]{./figures/Lorenz_1.pdf}
\caption{洛伦兹力示意图.图中$\bvec B$指向纸面内} \label{Lorenz_fig1}
\end{figure}
磁场 $\bvec B$ 中,电荷为 $q$,以速度 $\bvec v$ 运动的点电荷受到的洛伦兹力通过叉乘\upref{Cross}定义
\begin{equation}\label{Lorenz_eq1}
\bvec F = q\bvec v \cross \bvec B
\end{equation}
即洛伦兹力与速度和磁场的方向垂直,大小等于 $qvB$ 乘以速度与磁场夹角的正弦值.可见当速度与磁场垂直时洛伦兹力最大,平行时没有洛伦兹力.

\subsection{磁场对电荷不做功}
由于任意时刻,磁场力的方向垂直于运动方向,所以静磁场不对电荷做功(类比向心力不对圆周运动做功),证明如下.洛伦兹力的瞬时功率为
\begin{equation}
P = \bvec F \vdot \bvec v = q\,\bvec v \cross \bvec B \vdot \bvec v
\end{equation}
由矢量混合积\upref{TriVM}的运算
\begin{equation}
\bvec v \cross \bvec B \vdot \bvec v = \bvec v \cross \bvec v \vdot \bvec B = 0
\end{equation}
因为矢量叉乘本身等于0.


\subsection{广义洛伦兹力}
\pentry{极限\upref{Lim}}
\textbf{广义上的洛伦兹力} 是指电磁场给电荷施加的所有作用力,即电场力加洛伦兹力.麦克斯韦方程组\upref{MWEq}描述了由电荷的分布及运动情况如何计算电磁场,而广义洛伦兹力则解释了已知电磁场分布如何计算电荷的受力.

对于点电荷
\begin{equation}\label{Lorenz_eq2}
\bvec F = q (\bvec E + \bvec v \cross \bvec B)
\end{equation}
对于连续的电荷分布,上式可写为积分形式
\begin{equation}
\int \bvec f \dd V= \int \rho(\bvec E + \bvec v \cross \bvec B) \dd V
\end{equation}
由于该公式对任意体积都成立,因此
\begin{equation}
\bvec f = \rho(\bvec E + \bvec v \cross \bvec B)
\end{equation}
其中$\rho$是单位体积电荷密度,$\bvec f$ 是单位体积电荷的受力密度,可用极限的方法定义为无穷小体积的受力除以该体积
\begin{equation}
\bvec f = \lim_{V \to 0} \frac{\bvec F}{V}
\end{equation}

