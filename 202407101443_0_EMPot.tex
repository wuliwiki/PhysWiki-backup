% 电磁场标势和矢势
% keys 法拉第电磁感应|麦克斯韦方程组
% license Xiao
% type Tutor

\begin{issues}
\issueDraft
\end{issues}

\pentry{法拉第电磁感应\nref{nod_FaraEB}, 磁矢势\nref{nod_BvecA}}{nod_5de1}

经典电动力学中, 可以用标量势 $\varphi(\bvec r, t)$ 和矢量势 $\bvec A(\bvec r, t)$ 表示电磁场,使一些计算更为方便:
\begin{equation}\label{eq_EMPot_1}
\bvec E = -\grad \varphi - \pdv{\bvec A}{t}~.
\end{equation}
\begin{equation}\label{eq_EMPot_2}
\bvec B = \curl \bvec A~.
\end{equation}

\subsection{推导}
首先定义 $\bvec A$, 则由法拉第电磁感应定律(\autoref{eq_MWEq_2})
\begin{equation}
\curl \qty(\bvec E + \pdv{\bvec A}{t}) = \curl \bvec E + \pdv{\bvec B}{t} = \bvec 0~.
\end{equation}
这说明括号中的矢量可以表示为一个标量函数的梯度,即标势 $\varphi$, 负号是为了在静电场的情况下使得标势等于电势。
\subsection{应用}
\subsubsection{麦克斯韦方程组}
将\autoref{eq_EMPot_1} 和\autoref{eq_EMPot_2} 代入麦克斯韦方程组可以得到两条与麦克斯韦方程组等效的方程
\begin{equation}\label{eq_EMPot_4}
\laplacian \varphi + \pdv{t} (\div \bvec A) = -\frac{\rho}{\epsilon_0}~,
\end{equation}
\begin{equation}\label{eq_EMPot_5}
\qty(\laplacian \bvec A - \mu_0\epsilon_0 \pdv[2]{\bvec A}{t}) - \grad \qty(\div \bvec A + \mu_0\epsilon_0 \pdv{\varphi}{t}) = -\mu_0\bvec J~.
\end{equation}
这两条方程可以根据规范条件的选取进行简化。
\subsubsection{带电粒子在电磁场中运动}
速度为$\bvec v$,电荷量为$q$的粒子在电磁场中会受到广义洛伦兹力:
\begin{equation}\label{eq_EMPot_3}
\bvec F=q\bvec E+q\bvec v \times \bvec B
~,\end{equation}
由$\grad \cdot \bvec A\neq 0$可知,洛伦兹力不是保守力。那么该粒子的拉格朗日方程为:
\begin{equation}
\frac{\mathrm{d}}{\mathrm{d} t} \frac{\partial T}{\partial \dot{q}_a}-\frac{\partial T}{\partial q_a}=F_a~.
\end{equation}
但若能把广义力表示为势能函数:
\begin{equation}
\bvec F_{\alpha}=-\frac{\partial U}{\partial q_{\alpha}}+\frac{\mathrm{d}}{\mathrm{d}t}\frac{\partial U}{\partial \dot q_{\alpha}}~,
\end{equation}
我们就能得到形式上的保守系中拉格朗日方程,同时得到电磁场中带电粒子的广义势能和拉格朗日量。

采用直角坐标系,由\autoref{eq_EMPot_3} 得:$\bvec F_x=q\bvec E_x+q\bvec v_y\bvec B_z-q\bvec v_z\bvec B_y$。代入标势和矢势得:
\begin{equation}\label{eq_EMPot_6}
\bvec F_x=-q\frac{\partial \varphi}{\partial x}-q\frac{\partial \bvec A_x}{\partial t}+q\bvec v_y\qty(\frac{\partial \bvec A_y}{\partial x}-\frac{\partial \bvec A_x}{\partial y})-q\bvec v_z\qty(\frac{\partial \bvec A_z}{\partial x}-\frac{\partial \bvec A_x}{\partial z})~,
\end{equation}
又因$\frac{\mathrm{d}\bvec A(\bvec x,t)}{\mathrm{d}t}=\frac{\partial \bvec A}{\partial t}+\frac{\partial \bvec A}{\partial \bvec x^i}\cdot \bvec v^i$,把偏微分项代入\autoref{eq_EMPot_6} 后,还需要将结果化简为对速度或者坐标的偏导数。利用
\begin{equation}
\frac{\mathrm{d} \bvec A_x}{\mathrm{~d} t}=\frac{\mathrm{d}}{\mathrm{d} t}\left[\frac{\partial}{\partial \bvec v_x}(\bvec {A} \cdot \bvec v)\right]=\frac{\mathrm{d}}{\mathrm{d} t} \frac{\partial}{\partial \bvec v_x}(-\varphi+\bvec {A} \cdot \bvec v)~,
\end{equation}
\begin{equation}
\left(\bvec v_x \frac{\partial \bvec A_x}{\partial x}+\bvec v_y \frac{\partial \bvec A_y}{\partial x}+\bvec v_z \frac{\partial \bvec A_z}{\partial x}\right)=\bvec v \cdot \frac{\partial \bvec {A}}{\partial x}=\frac{\partial}{\partial x}(\bvec v \cdot \bvec {A}) ~.
\end{equation}
把最终结果化简为
\begin{equation}
\bvec F_x=q\left[-\frac{\partial}{\partial x}(\varphi-\bvec {A} \cdot v)+\frac{\mathrm{d}}{\mathrm{d} t} \frac{\partial}{\partial \bvec v_x}(\varphi-\bvec {A} \cdot \bvec v)\right]~,
\end{equation}
因此,广义势能为:
\begin{equation}
U=q\varphi -\bvec A\cdot \bvec v~.
\end{equation}