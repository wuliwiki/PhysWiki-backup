% 质点系(摘要)

\begin{table}[ht]
\centering
\caption{质点系相关物理定律}\label{MPTA_tab1}
\begin{tabular}{|c|c|c|c|c|}
\hline
名称 & 公式 & 涉及的物理量1 & 涉及的物理量2 & 相应的守恒律\\
\hline
动量定理 & $\bvec F = \bvec F^{out}= \dv{\bvec P}{t}$  & $\bvec P = \sum \bvec P_i = \sum (m_i \bvec v_i)$ 系统的总动量是系统中各质点的动量之和 & $\bvec F=\sum \bvec F_i$ 是系统所受的合力。由于内力和总为$0$,只有合外力会改变系统的动量 & 若$\bvec F = \bvec 0$,则$\dv{\bvec P}{t} = 0$,即外力和为零时,系统动量守恒。\\
\hline
角动量定理 & $\bvec \tau =\bvec \tau^{out}= \dv{\bvec L}{t}$ & $\bvec L = \sum \bvec L_i= \sum (\bvec r_i \times \bvec P_i)$ 系统的总角动量是系统中各质点的角动量之和 & $\bvec \tau=\sum \bvec \tau_i = \sum \bvec (\bvec r_i \times \bvec F_i)$ 是系统所受的力矩和。由于内力矩和总为$0$,只有外力矩和会改变系统的角动量 & 外力矩和为零时,系统角动量守恒\\
\hline
动能定理 & $W = \Delta E_k$  & $E_k = \frac{1}{2}\sum (m_i v_i^2)$ 系统总动能是各质点的总动能之和 & $W = \sum (\bvec F_i \cdot \Delta \bvec r_i)$ 是各力对系统做的功之和。注意$\Delta \bvec r_i$的含义不同于力矩中的符号,且内、外力做的功都会影响系统的总动能。 & 功之和为零时,系统动能守恒\\
\hline
保守力的功-能关系&$W_{CON} = -\Delta E_p$&$E_P$是系统的势能&$W_{CON}$是保守内力的功之和&$W_{C} = 0 \Rightarrow \Delta E_{P} = 0$\\
\hline
机械能定理& $W_{NC} = \Delta E_{mech}$ & $E_{mech} = E_k+E_P$ 系统机械能是系统动能、势能之和 & $W_{NC}$是外力与非保守内力的功之和& $W_{NC} = 0 \Rightarrow \Delta E_{mech} = 0$
\\
\hline
\end{tabular}
\end{table}

\subsection{质心与质心系}
有时,我们会抽象出系统的质心,并构建质心系。

\begin{table}[ht]
\centering
\caption{质心的物理量}\label{MPTA_tab2}
\begin{tabular}{|c|c|}
\hline
名称 & 定义\\
\hline
质量 & $M=\sum m_i$\\
\hline
位置 & $\bvec r_c =\frac{\sum (m_i \bvec r_i)}{M} $\\
\hline
速度 & $\bvec v_c = \dv{\bvec r_c}{t}=\frac{\sum (m_i \dv{\bvec r_i}{t})}{M}=\frac{\sum (m_i \bvec v_i)}{M} $\\
\hline
加速度 & $\bvec a_c = \dv{\bvec v_c}{t}=\frac{\sum (m_i \bvec a_i)}{M} $\\
\hline
动量 & $\bvec p_c = M \bvec v_c = \sum (m_i \bvec v_i) = \sum \bvec p_i$\\
\hline
角动量 &$\bvec L_c = \bvec r_c \times \bvec P_c$\\
\hline
动能 &$E_{k,c} = \frac{1}{2} M v_c^2$\\
\hline
\end{tabular}
\end{table}

\begin{table}[ht]
\centering
\caption{一些定理}\label{MPTA_tab3}
\begin{tabular}{|c|c|c|}
\hline
相关物理量 & 公式 & 描述 \\
\hline
动量 & $\bvec P= \bvec P_c$ & 某参考系中质点系的总动量=该参考系中质心的动量。质心系中系统的总动量始终为零。 \\
\hline
角动量 & $\bvec L= \bvec L_c + \bvec L_{CMC}$ & 某参考系中质点系的总动量=该参考系中质心的角动量+质心系中系统的总角动量 \\
\hline
动能 & $E_k = E_{k,c} + E_{k,CMC}$ & 某参考系中质点系的总动能=该参考系中质心的动能+质心系中系统的总动能。此即为柯尼希定理。 \\
\hline
\end{tabular}
\end{table}