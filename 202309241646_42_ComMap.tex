% 压缩映射
% keys 压缩映射
% license Xiao
% type Tutor
\pentry{映射\upref{map},度量空间\upref{Metric}}
在正式定义之前,我们先讨论根据“压缩映射”这一名词应该怎么定义它。首先,压缩映射是集合到自身的一个映射(因为“压缩”就应该是种包含关系)。其次,“压缩”是指对集合中任意两点,在压缩映射下两点的像点的“距离”比两原像更小。这就是说,压缩映射是定义在带有“距离”的集合上的,这样的集合便是度量空间\upref{Metric}。容易验证,满足上面定义的压缩映射 $A$ 的定义集合 $X$ 是个无限集。事实上,若 $X$ 有限,且 $d:X\times X\rightarrow \mathbb R$ 是距离函数\footnote{表示"距离"的集合的元素应都是可比较的,就是说它是个有序集\upref{OrdRel},不妨将这个集合取作实数集。}。我们构建两个序列
\begin{equation}
\{x,Ax,A^2x,\cdots\},\{y,Ay,A^2y,\cdots\}~,
\end{equation}
由于 $X$ 有限,所以必有$m_1<n_1,m_2<n_2$ 存在,使得 $A^{m_1}x=A^{n_1}x,A^{m_2}y=A^{n_2}y$ 。那么
\begin{equation}
\begin{aligned}
A^{m_1+(n_1-m_1)(n_2-m_2)}x&=\underbrace{A^{(n_1-m_1)}\cdots A^{(n_1-m_1)}}_{(n_2-m_2)\text{个}}A^{m_1}x\\
&=\underbrace{A^{(n_1-m_1)}\cdots A^{(n_1-m_1)}}_{(n_2-m_2-1)\text{个}}A^{n_1}x\\
&=\underbrace{A^{(n_1-m_1)}\cdots A^{(n_1-m_1)}}_{(n_2-m_2-1)\text{个}}A^{m_1}x\\
&=A^{m_1}x\\~.
\end{aligned}
\end{equation}
同理
\begin{equation}
\begin{aligned}
A^{m_2+(n_1-m_1)(n_2-m_2)}y=A^{m_2}y\\~.
\end{aligned}
\end{equation}
于是
\begin{equation}\label{eq_ComMap_1}
\begin{aligned}
&d(A^{(n_1-m_1)(n_2-m_2)}(A^{m_1}x),A^{(n_1-m_1)(n_2-m_2)}(A^{m_2}y))\\
&=d(A^{m_1}x,A^{m_2}y)~.
\end{aligned}
\end{equation}
但是 $A$ 是压缩映射要求任意 $n>0$,都有 $d(A^n x,A^n y)<d(x,y)$\footnote{$A^n x$表示把$x$压缩 $n$ 次}。\autoref{eq_ComMap_1} 与这是矛盾的,这表明压缩映射必须定义在无限集上。上面的论证中用到了这样的事实,即映射\upref{map}可作用在其定义集合的所有元上。

于是下面的定义就是容易理解的了。
\begin{definition}{压缩映射}
设 $A:M\rightarrow M$ 是度量空间 $M$ (度量为 $d$)到它自身的一个映射,若存在一常数 $\lambda\in(0,1)$,使得
\begin{equation}
d(Ax,Ay)\leq \lambda d(x,y),\quad \forall x,y\in M~.
\end{equation}
 则称 $A$ 为 $M$ 上的\textbf{压缩映射}。
\end{definition}
\begin{theorem}{压缩映射必定连续}
压缩映射是个连续映射。
\end{theorem}
\textbf{证明:}度量空间中连续映射可以描述为:

\textbf{证毕!}