% 复旦大学 1998 量子真题
% license Usr
% type Note

\textbf{声明}:“该内容来源于网络公开资料,不保证真实性,如有侵权请联系管理员”

1. 原子核线度约 $10^{-13} \text{cm}$, 试用不确定性原理估算核内质子的动能。(以电子伏特为单位)\hfill (20分)

2. 一维无限深势阱中, 质量为 $m$ 的粒子在 $t=0$ 时的状态为 
$\psi(x,0) = A \cos \frac{\pi x}{a} \left( \sin \frac{3 \pi x}{a} - 3 \sin \frac{\pi x}{a} \right)$
其中 $a$ 为阱宽, $A$ 是归一化系数, 试求 \\hfill (共20分) \\\\
\quad (a) $t$ 时刻粒子所在的状态; \\\\
\quad (b) $t>0$ 及 $t=0$ 时粒子的平均能量; \\\\
\quad (c) 若粒子在 $0 < x < \frac{a}{2}$ 中发现粒子的几率。

3. 设氢原子处于$\psi(r, \\theta, \\varphi) = \\frac{1}{\\sqrt{\\pi a^3}} e^{-r/a}$的 $a$ 是玻尔半径, 求 
\begin{itemize}
\item (a) $r$ 的平均值; \\\\
\item  (b) 动量 $p$ 的几率分布函数。

\end{itemize}