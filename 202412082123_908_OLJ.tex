% 欧拉角(综述)
% license CCBYSA3
% type Wiki

本文根据 CC-BY-SA 协议转载翻译自维基百科\href{https://en.wikipedia.org/wiki/Euler_angles}{相关文章}。
\begin{figure}[ht]
\centering
\includegraphics[width=8cm]{./figures/f7361f2b5f088c22.png}
\caption{} \label{fig_OLJ_1}
\end{figure}
欧拉角是由莱昂哈德·欧拉提出的三个角度,用于描述刚体相对于固定坐标系的方向。[1]

它们也可以表示物理学中运动参考系的方向,或三维线性代数中一般基的方向。

经典欧拉角通常采用倾斜角度的方式,其中零度表示垂直方向。后来,由彼得·古思里·泰特(Peter Guthrie Tait)和乔治·H·布赖恩(George H. Bryan)提出了替代形式,主要用于航空学和工程学中,其中零度表示水平位置。
\subsection{链式旋转等价性}
欧拉角可以通过元素几何或旋转组合(即链式旋转)来定义。几何定义表明,三个元素旋转(绕坐标系的轴旋转)总是足够将物体定向到任何目标参考系。

这三个元素旋转可以是外在旋转(绕原始坐标系xyz轴旋转,假设坐标系保持静止),也可以是内在旋转(绕旋转坐标系XYZ轴旋转,该坐标系与运动体固连,在每次元素旋转后,物体相对于外部参考系的方向会发生变化)。

在下面的各节中,带有撇号标记的轴(例如,z″)表示元素旋转后的新轴。

欧拉角通常用 α、β、γ 或 ψ、θ、φ 来表示。不同的作者可能会使用不同的旋转轴集来定义欧拉角,或者使用不同的名称来表示相同的角度。因此,任何涉及欧拉角的讨论都应该首先明确它们的定义。

在不考虑使用两种不同约定来定义旋转轴(内在或外在)的情况下,旋转轴有十二种可能的旋转顺序,可以分为两组:

\begin{itemize}
\item \textbf{正确的欧拉角}(\(z-x-z,x-y-x,y-z-y,z-y-z,x-z-x,y-x-y\))
\item \textbf{泰特-布赖恩角}(\(x-y-z,y-z-x,z-x-y,x-z-y,z-y-x,y-x-z\))。
\end{itemize}

泰特-布赖恩角也被称为卡尔丹角、航海角、航向、仰角和倾斜角,或偏航、俯仰和滚转角。有时,这两类旋转顺序都被称为“欧拉角”。在这种情况下,第一组旋转顺序被称为正确的或经典的欧拉角。
\subsection{经典欧拉角} 
欧拉角是瑞士数学家莱昂哈德·欧拉(1707–1783)引入的三个角度,用于描述刚体相对于固定坐标系统的方向。
\subsubsection{几何定义}  
\begin{figure}[ht]
\centering
\includegraphics[width=10cm]{./figures/1c8b217760d79068.png}
\caption{左:一个万向架组合,展示了 z-x-z 旋转序列。外部坐标系显示在底座中,内部坐标轴以红色表示。右:一个简单的图示,展示了类似的欧拉角。} \label{fig_OLJ_2}
\end{figure}
原始坐标系的轴表示为 \(x\)、\(y\)、\(z\),旋转后坐标系的轴表示为 \(X\)、\(Y\)、\(Z\)。几何定义(有时称为静态定义)首先定义节点线(N)为平面 \(xy\) 和 \(XY\) 的交线(也可以定义为轴 \(z\) 和 \(Z\) 的公垂线,然后表示为向量积 \(N = z \times Z\))。基于这个定义,三个欧拉角可以如下定义:

\(\alpha\)(或 \(\varphi\))是 \(x\) 轴与 \(N\) 轴之间的带符号角度(\(x\)-惯例——也可以定义为 \(y\) 轴与 \(N\) 轴之间的角度,称为 \(y\)-惯例)。  
\(\beta\)(或 \(\theta\))是 \(z\) 轴与 \(Z\) 轴之间的角度。  
\(\gamma\)(或 \(\psi\))是 \(N\) 轴与 \(X\) 轴之间的带符号角度(\(x\)-惯例)。  

只有当两个参考系具有相同的手性时,才能定义这两个参考系之间的欧拉角。

\subsubsection{内在旋转的约定}
内在旋转是发生在附着于运动物体的坐标系统 XYZ 的轴上的元素旋转。因此,它们在每次元素旋转后会改变其方向。XYZ 系统会旋转,而 xyz 系统保持固定。从 XYZ 与 xyz 初始重合开始,三个内在旋转的组合可以用来达到 XYZ 的任何目标方向。

欧拉角可以通过内在旋转来定义。旋转后的坐标系 XYZ 可以想象为最初与 xyz 对齐,然后经历由欧拉角表示的三个元素旋转。其连续的方向可以表示如下:

\begin{itemize}
\item x-y-z 或 x0-y0-z0(初始)  
\item x′-y′-z′ 或 x1-y1-z1(第一次旋转后)  
\item x″-y″-z″ 或 x2-y2-z2(第二次旋转后)  
\item X-Y-Z 或 x3-y3-z3(最终)
\end{itemize})

对于上述列出的旋转序列,节点线 N 可以简单地定义为第一次元素旋转后 X 的方向。因此,N 可以简单地表示为 x′。此外,由于第三次元素旋转是围绕 Z 轴进行的,它不会改变 Z 的方向。因此,Z 与 z″ 重合。这使得我们可以简化欧拉角的定义如下:

α(或 φ)表示围绕 z 轴的旋转,  
β(或 θ)表示围绕 x′ 轴的旋转,  
γ(或 ψ)表示围绕 z″ 轴的旋转。