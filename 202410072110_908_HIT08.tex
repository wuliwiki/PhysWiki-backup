% 哈尔滨工业大学 2008 年 考研 量子力学
% license Usr
% type Note

\textbf{声明}:“该内容来源于网络公开资料,不保证真实性,如有侵权请联系管理员”

\subsection{二、(20 分)}
 \( t = 0 \),时,其归一化的波函数为:

\[
\psi(x,0) = \sqrt{\frac{8}{5a}} (1 + \cos\frac{\pi x}{a}) \sin\frac{\pi x}{a}~
\]
求:(1)$t>0$时,粒子的状态波甬数$\psi(x,t)$:\\\\
(2)$t=0$及$t=t_0$,时系统的平均能量:\\\\
(3)$t=0$及$t=t_0$时,在$0\leq x\leq \frac{a}{0}$的区域内发现粒子的概率
\subsection{三、(20 分)}
已知$l=1$时,$L_z$的本征函数在坐标表象中用球函数客示为

\[Y_{1,-1} = \sqrt{\frac{3}{8\pi}} \sin \theta e^{-i\phi}~
\]
\[
Y_{1,0} = \sqrt{\frac{3}{4\pi}} \cos \theta~
\]
\[
Y_{1,1} = -\sqrt{\frac{3}{8\pi}} \sin \theta e^{i\phi}~
\]
写出$l=1$时,$\hat{L_x}$, 和 $\hat{L}_y $
\subsection{四、(20 分)}
某力学量的算符 $\hat{A}$ 有两个归一化的本征函数 $\varphi_1$ 和 $\varphi_2$,相应的本征值分别为 $\alpha_1$ 和 $\alpha_2$;另一个力学量的算符 $\hat{B}$ 也有两个归一化的本征函数 $\chi_1$ 和 $\chi_2$,相应的本征值分别为 $\beta_1$ 和 $\beta_2$。已知:
$$\varphi_1 = \frac{1}{\sqrt{5}} (4 \chi_1 + 3 \chi_2), \quad \varphi_2 = \frac{1}{\sqrt{5}} (3 \chi_1 - 4 \chi_2)~$$

在某状态下,测量力学量 $A$ 后得到的结果为 $\alpha_1$;若在此之后再测量力学量 $\hat{}$,接着再一次测量力学量 $\hat{A}$,则第二次测量 $\hat{A}$ 得到结果为 $\alpha_1$ 的概率为多少?
\subsection{五、(20 分)}
某体系的哈密顿算符为 $\hat{H} = \begin{pmatrix} 1 & 0 & 0 \\\\ 0 & 3 & 0 \\\\ 0 & 0 & 2 \end{pmatrix}$,对应的能量分别为 $1, 3, 2$。

(1) 若 $a \ll 1$,用微扰论论求 $\hat{H}$ 本征值的近似结果(二级微扰理论)。

(2) 求 $\hat{H}$ 的精确本征值。

(3) 在什么条件下,(1) 与 (2) 的结果一致。
\subsection{六、(20 分)}
有一电子处于确定的自旋态 $|\psi \rangle$,已知在此自旋态中 $S_z$取$\frac{\hbar}{2}$的概率是 $\frac{1}{3}$,且 $\hat{S}_x$ 的平均值小于零,求 $\hat{S}_x$ 的测量结果为 $-\frac{\hbar}{2}$ 的概率是多少?



