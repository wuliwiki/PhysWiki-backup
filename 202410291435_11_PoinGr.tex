% 庞家莱群
% license Usr
% type Tutor


\begin{issues}
\issueTODO 引用半直积
\end{issues}


\subsection{庞家莱群}
时空平移群,顾名思义,能对任意时空矢量$x\in \mathbb R^{1,3} $进行平移,所以群元就是李群$\mathbb R^{1,3}$本身,群乘法为加法。时空平移群与洛伦兹群的半直积构成\textbf{庞加莱群(Poincaré group)},用$G$简便表示为:
\begin{equation}
G= \mathrm{O}(1,3) \ltimes\mathbb{R}^{1,3}~,
\end{equation}
因此群元素包含了所有可能的坐标变换。对于任意$(\Lambda_1,a),(\Lambda_2,b),(\Lambda_1,a)\in G$,从物理意义出发,群乘法必然为:

\begin{equation}
(\Lambda_1,a)(\Lambda_2,b)=(\Lambda_1\Lambda_2,\Lambda_1b+a)~.
\end{equation}
显然与半直积的乘法运算自洽。此外,半直积的符号表明$(I,\mathbb R^{1,3})\vartriangleleft G$。接下来我们证明庞家莱群满足群定义,且时空平移群是正规子群。

显然庞家莱群的单位元为$(I,0)$,设任意$(\Lambda,a)\in G$,设其逆元素为$(\Lambda^{-1},b)$,对任意坐标矢量$x$作用,有
\begin{equation}
(\Lambda^{-1},b)(\Lambda,a)x=\Lambda^{-1}(\Lambda x+a)+b=x~,
\end{equation}
解得$b=-\Lambda^{-1} a$\footnote{或者直接利用群乘法求解。$(\Lambda^{-1},b)(\Lambda,a)=(I,b+\Lambda^{-1}a)=(I,0)$},显然在庞家莱群内。再由定义可知,封闭性和结合性自然满足,所以这确实是群。

再设任意$(I,b)\in G$,因为
\begin{equation}
\begin{aligned}
(\Lambda,a)^{-1}(I,b)(\Lambda,a)x&=(\Lambda^{-1},-\Lambda^{-1}a)(I,b)(\Lambda,a)x\\
&=\left\{\Lambda^{-1}\left [\left(\Lambda x+a \right)+b\right]-\Lambda^{-1}a\right\}x\\
&=(I,\Lambda^{-1}b)x~.
\end{aligned}
\end{equation}
因此,时空平移群同构于庞家莱群的正规子群。



\subsubsection{庞家莱群的李代数}
庞家莱群的李代数,即单位元处的切空间,由洛伦兹群的切空间$\mathfrak {so}(1,3)$与时空平移群的切空间直和而成。为了得到李代数的形式,我们需要找到这两个群的共同表示空间。

假设这两个群的群元都有线性算符形式,分别表示为$L(\Lambda),L(a)$,且共同作用在标量函数上。在坐标系进行洛伦兹变换后,
\begin{equation}
\begin{aligned}
\phi'(x')&=\phi(x)\\
\phi'(x)&=\phi(\Lambda^{-1}x)\equiv L(\Lambda)\phi(x)~.
\end{aligned}
\end{equation}

这个定义自然满足线性与群同态,对于任意$\Lambda_1,\Lambda_2\in SO^+(1,3);k\in\mathbb R$,我们有$L(\Lambda_1\Lambda_2)=L(\Lambda_1)L(\Lambda_2),L(k\Lambda)=kL(\lambda)$


