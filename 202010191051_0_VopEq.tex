% 矢量算符常用公式
% keys 矢量算符|微积分|叉乘|点乘|内积

\footnote{参考教材: David Griffiths, Introduction to Electrodynamics}(未完成: 把 griffiths 书中的全部列出)

\subsubsection{梯度相关}

\subsubsection{散度相关}
\begin{equation}
\grad (fg) = f\grad g + g \grad f
\end{equation}



\begin{equation}
\div (f \bvec A) = f (\div \bvec A) + \bvec A \vdot (\grad f)
\end{equation}

\subsubsection{散度和旋度}
\begin{equation}
\curl(U\bvec A) = (\grad U) \cross\bvec A + U \curl\bvec A
\end{equation}

\begin{equation}
\curl(\curl \bvec E) = \grad(\div\bvec E) - \laplacian \bvec E
\end{equation}

\begin{equation}
\grad (\bvec A \vdot \bvec B) = \bvec A \cross (\curl \bvec B) + \bvec B \cross (\curl \bvec A) + (\bvec A \vdot \grad) \bvec B + (\bvec B \vdot \grad) \bvec A
\end{equation}

\subsection{证明}
理论上, 我们可以直接根据定义, 将各个矢量记为分量的形式证明, 但直接写出来非常繁琐. 一种简单的记号是使用 $\delta_{i,j}$ 和 $\epsilon_{i,j,k}$ 符号, 再结合爱因斯坦求和约定(未完成).

我们另外推荐一种不需要写出分量的推导方法, 见 “一种矢量算符的运算方法\upref{MyNab}”.
