% 阿基米德(综述)
% license CCBYSA3
% type Wiki

本文根据 CC-BY-SA 协议转载翻译自维基百科\href{https://en.wikipedia.org/wiki/Archimedes}{相关文章}。

阿基米德(约公元前287年 – 约公元前212年)是古希腊的数学家、物理学家、工程师、天文学家和发明家,来自西西里岛古城锡拉库萨。[3] 虽然他的生平细节不多,但他被认为是古典时代的领先科学家之一。阿基米德被誉为古代历史上最伟大的数学家,也是所有时代最伟大的数学家之一。[4] 阿基米德通过应用“无穷小”概念和“穷尽法”,为现代微积分和分析学奠定了基础,推导并严格证明了一系列几何定理。[5][6][7] 这些定理包括圆的面积、球的表面积和体积、椭圆的面积、抛物线下的面积、旋转抛物面段的体积、旋转双曲面段的体积,以及螺旋的面积。[8][9]

阿基米德的其他数学成就包括推导出π(圆周率)的近似值、定义并研究了阿基米德螺旋,并设计了一种使用指数法表示非常大数字的系统。他还是最早将数学应用于物理现象的人之一,致力于静力学和流体静力学的研究。阿基米德在这一领域的成就包括杠杆定律的证明,[10] 重心概念的广泛应用,[11] 以及著名的阿基米德原理,即浮力定律的表述。[12] 他还被认为设计了许多创新的机械装置,如螺旋泵、复合滑轮和用于保护家乡锡拉库萨免受入侵的防御性战争机器。

与他的发明不同,阿基米德的数学著作在古代鲜为人知。亚历山大的数学家曾阅读并引用过他的作品,但直到大约公元530年,米利都的伊西多尔在拜占庭的君士坦丁堡才首次做出了全面的汇编,而尤托修斯在同世纪对阿基米德著作的注释,第一次使这些著作获得了更广泛的读者群体。那些在中世纪幸存下来的阿基米德著作的少数副本,成为了文艺复兴时期和17世纪科学家思想的重要来源,[13][14] 而1906年在阿基米德古抄本中发现的失落作品,提供了新的洞见,揭示了他是如何获得数学成果的。[15][16][17][18]
\subsection{传记}  
\subsubsection{早年生活}
\begin{figure}[ht]
\centering
\includegraphics[width=8cm]{./figures/b835f7aaef6d5bdd.png}
\caption{本杰明·韦斯特(Benjamin West)创作的《西塞罗发现阿基米德的墓》(1805)} \label{fig_Archim_1}
\end{figure}
阿基米德大约于公元前287年出生在锡拉库萨的港口城市,该城市当时是意大利南部大希腊地区的一个自治殖民地。出生日期基于拜占庭希腊学者约翰·特泽茨(John Tzetzes)的一段话,他提到阿基米德在公元前212年去世时活了75年。[9] 普鲁塔克在《希腊罗马英杰传》中写道,阿基米德与锡拉库萨国王赫罗二世有亲戚关系,尽管西塞罗则认为他出身卑微。[19][20] 在《沙数计》中,阿基米德提到父亲的名字是菲迪亚斯(Phidias),他是一位天文学家,其他关于他的事迹没有记载。[20][21] 阿基米德的朋友赫拉克利德斯(Heracleides)曾为他写过传记,但这部作品已遗失,使得他的一生仍然笼罩在谜团中。例如,至今无法确认他是否结过婚、有过子女,或者年轻时是否曾访问过埃及的亚历山大。[22] 从他现存的著作中可以看出,他与当时在亚历山大的学者保持着良好的学术关系,包括他的朋友萨摩斯的科农(Conon of Samos)和塞内的馆长埃拉托斯特尼(Eratosthenes of Cyrene)。[b]
\subsubsection{职业生涯}  
阿基米德的生活标准版本是由古希腊和古罗马历史学家在他去世后很久才写成的。关于阿基米德的最早记载出现在波利比乌斯(约公元前200–118年)的《历史》中,写于他去世约70年后。这篇记载对阿基米德作为一个人的描写甚少,主要聚焦于他为防御罗马人而制造的战争机器。波利比乌斯提到,在第二次布匿战争期间,锡拉库萨从罗马转向迦太基,导致由马库斯·克劳狄乌斯·马塞勒斯和阿皮乌斯·克劳狄乌斯·普尔库尔指挥的军事行动,于公元前213年至212年围攻该城。他指出,罗马人低估了锡拉库萨的防御,并提到阿基米德设计的几种机器,包括改进型的投石机、可以弯曲旋转的起重机以及其他投石器。尽管罗马最终攻占了锡拉库萨,但他们因阿基米德的创新而遭受了重大损失。

西塞罗(公元前106–43年)在他的几部作品中提到阿基米德。在担任西西里的财务官时,西塞罗发现了被认为是阿基米德墓的遗址,位于锡拉库萨的阿格里金特门附近,墓地被忽视,四周长满了灌木丛。西塞罗将墓地清理一番,并看到了雕刻并能读出一些作为铭文的诗句。墓碑上有一座雕塑,描绘了阿基米德最喜欢的数学证明:球体的体积和表面积是包含其底面的圆柱体的三分之二。他还提到,马塞勒斯曾将阿基米德建造的两个天文仪器带到罗马。罗马历史学家李维(公元前59年–公元17年)在其著作中重述了波利比乌斯关于锡拉库萨被攻陷以及阿基米德在其中作用的故事。
\subsubsection{死亡}
\begin{figure}[ht]
\centering
\includegraphics[width=6cm]{./figures/d9449557ab0db727.png}
\caption{《阿基米德之死》(1815年),托马斯·德乔治(Thomas Degeorge)画作[29]} \label{fig_Archim_2}
\end{figure}
普鲁塔克(公元45–119年)提供了关于阿基米德在锡拉库萨沦陷后如何死去的至少两种说法。根据最流行的说法,阿基米德正在思考一个数学图形时,城市被攻陷。一名罗马士兵命令他前去见马塞勒斯,但阿基米德拒绝了,称他必须先完成问题的研究。这激怒了士兵,士兵用剑杀死了阿基米德。另一个故事则说,阿基米德携带着数学工具,后来被杀,因为一名士兵认为这些物品是珍贵的。马塞勒斯听闻阿基米德死讯后 reportedly 感到愤怒,因为他认为阿基米德是一个宝贵的科学资产(他曾称阿基米德为“几何学的百臂巨人”),并命令不允许伤害他。

据说阿基米德的遗言是“不要打扰我的圆圈”(拉丁语:Noli turbare circulos meos;希腊语:μὴ μου τοὺς κύκλους τάραττε),这指的是他在被罗马士兵打扰时所研究的数学图形。没有可靠的证据表明阿基米德曾说过这句话,这句话也没有出现在普鲁塔克的记载中。一个类似的引述出现在瓦莱里乌斯·马克西穆斯(约公元30年)所著的《名言和事迹》中,他写道:“...但用手保护着灰尘时说‘我求你,不要打扰这东西’”。
\subsection{发现与发明} 
\subsubsection{阿基米德原理}
\begin{figure}[ht]
\centering
\includegraphics[width=8cm]{./figures/8ef374a6acf4a6cc.png}
\caption{通过置换法测量体积:(a) 物体未浸没前,(b) 物体浸没后;液体在量筒中上升的体积(∆V)等于物体的体积。} \label{fig_Archim_3}
\end{figure}
关于阿基米德最广为人知的轶事是他发明了一种测定不规则形状物体体积的方法。据维特鲁威记载,叙拉古国王希耶罗二世为一座神庙订制了一顶王冠,他提供了用于制作的纯金。王冠可能被制成了一种类似祭祀花环的形状。阿基米德受命检测金匠是否掺入了银,但不能破坏王冠,也无法将其熔化成规则形状以计算其密度。

据说,阿基米德在洗澡时注意到,当他进入浴缸时,水位上升了。他意识到这种现象可以用来测定黄金王冠的体积。阿基米德因这一发现过于兴奋,以至于忘记穿衣服就跑到街上,大喊“尤里卡!”(希腊语:“εὕρηκα, heúrēka!”,意为“我发现了!”)。在实际操作中,水可以视为不可压缩的,因此浸没的王冠会排开与其自身体积相等的水量。通过将王冠的质量除以排开的水的体积,可以得出王冠的密度。如果掺入了较便宜且密度较低的金属,密度将低于纯金的密度。阿基米德发现确实如此,从而证明了王冠中掺杂了银。

关于这顶黄金王冠的故事并未出现在阿基米德已知的任何著作中。由于测量水排量所需的精确度极高,这种方法的实际可行性受到质疑。阿基米德可能转而使用了一种基于流体静力学原理的解决方案,即阿基米德原理,这一原理记录在他的著作《论浮体》中:浸没在流体中的物体会受到一个等于所排开流体重量的浮力。利用这一原理,可以通过将王冠与相同重量的纯金样品平衡在天平上,然后将装置浸入水中,比较两者的密度。由于密度的差异,天平会相应倾斜。

伽利略·伽利莱在1586年发明了一种受阿基米德启发的静力平衡仪。他认为“很可能阿基米德使用的正是这种方法,因为这种方法不仅非常精确,而且基于阿基米德自己发现的原理。”
\subsubsection{杠杆定律}
虽然阿基米德并未发明杠杆,但他在《平面图形的平衡》中对杠杆原理给出了数学证明。更早关于杠杆原理的描述可以在欧几里得的著作以及亚里士多德追随者逍遥学派的《机械问题》中找到,有些人将后者归功于阿尔喀塔斯的作者身份。

关于阿基米德使用杠杆举起重物的壮举,有多个描述,且往往相互矛盾。普鲁塔克提到,阿基米德设计了滑轮组系统,使得水手们可以利用杠杆原理举起原本无法移动的重物。据亚历山大的帕普斯记载,阿基米德在研究杠杆及其机械优势时曾说过:“给我一个立足之地,我就能撬动地球”(希腊语:δῶς μοι πᾶ στῶ καὶ τὰν γᾶν κινάσω)。后来,奥林匹奥多鲁斯将同样的名言归因于阿基米德发明的一种叫“baroulkos”的机械装置,这是一种类似绞盘的工具,而非杠杆。
\subsubsection{阿基米德螺旋泵}
瑙克拉提斯的阿西纳俄斯引用了一位名叫莫斯希翁的人的描述,其中提到叙拉古国王希耶罗二世委托设计了一艘巨大的船——\textbf{叙拉古号},这艘船既可以用于豪华旅行、运输物资,也可以作为海上力量的象征。据莫斯希翁的记载,\textbf{叙拉古号}由阿基米德主持建造,被认为是古典时代建造的最大船只。根据描述,这艘船能够容纳约600人,还配备了花园装饰、健身房以及一座献给女神阿芙洛狄忒的神庙等设施。  

记载中还提到,为了清除可能通过船体渗入的水,阿基米德设计了一种带有旋转螺旋状叶片的装置,该装置被安装在一个圆柱体内。  

\textbf{阿基米德螺旋泵}由手动驱动,可以将低洼水域的水输送到灌溉渠中。这种装置至今仍用于抽取液体和颗粒状固体(如煤炭和谷物)。据维特鲁威描述,阿基米德的装置可能是对一种螺旋泵的改进,而这种螺旋泵曾用于灌溉巴比伦空中花园。  

世界上第一艘配备螺旋推进器的远洋蒸汽船——\textbf{阿基米德号},于1839年下水,以纪念阿基米德及其在螺旋装置上的研究成果。
\subsubsection{阿基米德的抓钩}
据说阿基米德设计了一种抓钩作为防御叙拉古城的武器。这种抓钩也被称为“摇船器”,由一个类似起重机的臂杆组成,悬挂着一个巨大的金属抓钩。当抓钩落到进攻的船只上时,臂杆会向上摆动,将船只从水中提起,有可能将其摧毁或沉没。  

现代实验测试了这种抓钩的可行性。在2005年的一部名为《古代世界的超级武器》的电视纪录片中,重建了抓钩的一个版本,并得出结论认为这是一种可行的装置。  

此外,阿基米德还被认为改进了投石机的威力和精度,并在第一次布匿战争期间发明了里程计。里程计被描述为一种带有齿轮机构的推车,每行驶一英里,齿轮就会将一个球掉入容器中。
\subsubsection{热射线}
\begin{figure}[ht]
\centering
\includegraphics[width=6cm]{./figures/ad0a3da056a0a5aa.png}
\caption{利用镜子构成抛物面反射器攻击来袭的船只} \label{fig_Archim_4}
\end{figure}
据传说,阿基米德排列镜子形成抛物面反射器,利用聚焦的阳光烧毁进攻叙拉古的船只。虽然没有任何当时的直接证据支持这一壮举,而且现代学者认为这一事件可能并未发生,但阿基米德可能撰写过一部名为《反射光学》(*Catoptrica*)的著作。公元2世纪,卢奇安和盖伦曾提到,在叙拉古围城战期间,阿基米德曾用火烧毁敌方船只。近400年后,尽管存有疑问,安泰米乌斯尝试重建阿基米德假想的反射器几何结构。  

这一被称为“阿基米德热射线”的装置,自文艺复兴时期以来便引发了关于其真实性的持续争论。勒内·笛卡尔否定了这一说法,认为它不可能实现,而现代研究人员试图仅用阿基米德时代可能具备的工具重现这一效果,结果大多未能成功。  

有观点认为,可以使用一大片高度抛光的青铜或铜盾作为镜子,将阳光聚焦到一艘船上。然而,这种装置的实际效果可能更多是使船员目眩或分心,而非引发火灾。利用现代材料和更大规模的装置,聚光型太阳炉确实能够达到非常高的温度,并且有时被用于发电。
\subsubsection{天文学仪器}
在《数沙者》中,阿基米德讨论了地球、太阳和月亮的天文测量,并提到了阿里斯塔克斯的日心宇宙模型。在没有使用三角学或弦表的情况下,阿基米德通过描述用于观测的程序和仪器(如带有插销或凹槽的直杆),确定了太阳的视直径。他应用了修正因子来调整这些测量值,最终以上下界的形式给出结果,以考虑观测误差。托勒密在《天文学大成》中引用了喜帕恰斯的记载,也提到阿基米德对夏至的观测。这使得阿基米德成为已知的第一位连续多年记录多次夏至日期和时间的希腊人。  

西塞罗的《论共和国》中描绘了一场发生在公元前129年的虚构对话。根据对话内容,在第二次布匿战争中叙拉古被攻陷后,马尔凯卢斯从叙拉古带回了两个由阿基米德制造的装置,这些装置展示了太阳、月亮和五颗行星的运动。西塞罗还提到由米利都的泰勒斯和克尼多斯的欧多克索斯设计的类似装置。对话中提到,马尔凯卢斯将其中一个装置作为他从叙拉古带回的唯一个人战利品,另一个则捐献给了罗马的美德神庙。据西塞罗记载,盖乌斯·苏尔皮丘斯·伽鲁斯曾向卢修斯·富里乌斯·菲卢斯展示了马尔凯卢斯的装置,并描述道:  

“当伽鲁斯转动这个球体时,月亮在那铜制装置上以与天空中相同的转数跟随太阳,由此,在天空中太阳球体也会出现同样的日食,月亮到达地球影子的位置时,太阳正好与之对齐。”

这是对一种小型天体仪的描述。亚历山大的帕普斯提到,阿基米德曾撰写了一部关于构建这些装置的失传著作,名为《制球术》。现代关于这一领域的研究主要集中在安提凯希拉机械装置上,这是一种建于公元前约100年的设备,可能具有类似的目的。  

构建这类装置需要对差动齿轮具有深刻的理解。过去人们认为这种技术超出了古代技术的范围,但1902年发现的安提凯希拉机械装置证实了古希腊人已经掌握了这类设备的技术。
\subsection{数学} 
虽然阿基米德常被视为机械装置的设计者,但他也对数学领域做出了重要贡献。普鲁塔克曾写道,阿基米德“将他所有的热情和志向投入到了那些更纯粹的思考中,这些思考与生活的庸常需求无关”,尽管一些学者认为这可能是对他的误解。  
\subsubsection{穷尽法} 
\begin{figure}[ht]
\centering
\includegraphics[width=8cm]{./figures/19aec6f8df177aee.png}
\caption{阿基米德通过已知六边形的边长来计算十二边形的边长,并对每次边数加倍的正多边形进行类似计算。} \label{fig_Archim_5}
\end{figure}
阿基米德能够使用不可分割量(即无穷小量的前身),其方式类似于现代的积分学。[6] 通过反证法(归谬法),他能够为问题提供任意精度的答案,同时明确给出答案所在的范围。这一技巧被称为\textbf{穷尽法},他运用这一方法来逼近图形的面积和圆周率的值。

在《圆的测量》中,他通过在圆外绘制一个较大的正六边形,然后在圆内绘制一个较小的正六边形,逐步将每个正多边形的边数加倍,在每一步中计算每个多边形边长。随着边数的增加,正多边形越来越接近圆形。经过四次这样的步骤,当多边形的边数达到96时,他能够确定圆周率的值介于3⁠1/7(约为3.1429)和3⁠10/71(约为3.1408)之间,这与其实际值3.1416非常接近。[71] 他还证明了圆的面积等于圆周率乘以半径的平方(πr²)。
\subsubsection{阿基米德性质}
在《球与圆柱》中,阿基米德假设任何量,当它与自身相加足够多次时,将超过任何给定的量。今天,这被称为实数的\textbf{阿基米德性质}。[72]

在《圆的测量》中,阿基米德给出了√3的值,介于265/153(约为1.7320261)和1351/780(约为1.7320512)之间。实际值约为1.7320508,这使得该估算非常准确。他提出这一结果时并未解释他是如何得出的。阿基米德这一做法使得约翰·沃利斯评论道,他“似乎有意隐藏了他的研究痕迹,好像他不愿让后人知道他的研究方法,而只希望从他们那里得到对他结果的认同。”[73] 有人认为他可能使用了一种迭代法来计算这些值。[74][75]
\subsubsection{无限级数}
\begin{figure}[ht]
\centering
\includegraphics[width=6cm]{./figures/36960adb7a735b97.png}
\caption{《抛物线的求积》中,证明上图中的抛物线段的面积等于下图中内切三角形面积的4/3。} \label{fig_Archim_6}
\end{figure}
在《抛物线的求积》中,阿基米德证明了由抛物线和一条直线围成的区域的面积是相应内切三角形面积的4/3,如右图所示。他将这个问题的解表示为一个无限几何级数,其公比为1/4:
\[
\sum_{n=0}^{\infty} 4^{-n} = 1 + 4^{-1} + 4^{-2} + 4^{-3} + \cdots = \frac{4}{3}.~
\]
如果这个级数的首项是三角形的面积,那么第二项就是两个三角形面积的和,这两个三角形的底边分别是两条较小的割线,而它们的第三个顶点是抛物线与一条平行于其轴且通过底边中点的直线交点,以此类推。这个证明使用了级数 1/4 + 1/16 + 1/64 + 1/256 + · · ·,它的和为1/3。
\subsubsection{万万的万万}
在《沙子计》中,阿基米德提出要计算一个比填满整个宇宙所需沙粒还要大的数字。在此过程中,他挑战了“沙粒数量太大,无法计算”的观念。他写道:

“有些人,吉洛王,认为沙粒的数量是无穷无尽的;而我所说的沙粒,不仅是指那些存在于锡拉库萨和西西里岛的沙粒,还包括在任何地区,无论是有人居住的还是无人居住的地方存在的沙粒。”

为了解决这个问题,阿基米德设计了一种基于“万万”计数的系统。这个词本身来源于希腊语**μυριάς**(murias),表示数字10,000。他提出了一种使用万万的万万(即1亿,即10,000 × 10,000)的幂的数字系统,并得出结论,填满宇宙所需的沙粒数量将是8维京特翁(vigintillion),即8×10⁶³。[76]
\subsection{著作}
\begin{figure}[ht]
\centering
\includegraphics[width=6cm]{./figures/60d22206b6aaa81d.png}
\caption{《阿基米德全集》封面,希腊语和拉丁语版,由大卫·里沃(David Rivault)编辑(1615年)。} \label{fig_Archim_7}
\end{figure}
阿基米德的著作是用多利克方言的希腊语写成的,这是古代锡拉库萨的方言。[77] 许多阿基米德的著作没有保存下来,或者仅以大量编辑过的片段存在;至少有七篇他的著作由于其他作者的引用而为人所知。[9] 亚历山大的帕普斯提到《球体制造论》和另一篇关于多面体的著作,而亚历山大的西奥纳引用了来自现已失传的《反射论》(Catoptrica)的一段话。[c]

阿基米德通过与亚历山大数学家的书信往来使他的著作广为人知。阿基米德的著作首先由拜占庭希腊建筑师米利都的伊西多尔(约530年)汇编,而尤托修斯在同一世纪所作的注释帮助阿基米德的著作更广泛地传播。阿基米德的著作由萨比特·伊本·库拉(Thābit ibn Qurra,836–901年)翻译成阿拉伯语,再由克雷莫纳的杰拉尔(Gerard of Cremona,约1114–1187年)通过阿拉伯语翻译成拉丁语。后来,威廉·莫尔贝克(William of Moerbeke,约1215–1286年)和雅各布·克雷莫纳(Iacobus Cremonensis,约1400–1453年)完成了从希腊语到拉丁语的直接翻译。[78][79]

在文艺复兴时期,阿基米德的作品《初版》(Editio princeps)由约翰·赫尔瓦根(Johann Herwagen)于1544年在巴塞尔出版,包含希腊语和拉丁语版的阿基米德著作。[80]
\subsubsection{现存著作} 
以下作品按诺尔(Knorr,1978年)和佐藤(Sato,1986年)设定的新术语和历史标准按时间顺序排列。[81][82]

\textbf{圆的测量}

本文是一部简短的作品,由三个命题组成。它以与来自佩卢修姆的多西修斯的信件形式写成,多西修斯是萨摩斯的科农的学生。在第二命题中,阿基米德给出了π(圆周率)的近似值,显示出它大于 \( \frac{223}{71} \)(3.1408...)且小于 \( \frac{22}{7} \)(3.1428...)。

\textbf{沙粒计算者}

在这部作品中,阿基米德找到了一个大于填充宇宙所需的沙粒数量的数值。这本书提到了萨摩斯的阿里斯塔克斯提出的日心说理论,以及当时关于地球大小和各天体之间距离的观点。通过使用基于万万的幂次的数字系统,阿基米德得出结论,填充宇宙所需的沙粒数量是 8×10^63(现代记法)。引言信中指出,阿基米德的父亲是天文学家菲狄亚斯。**沙粒计算者**是唯一一部阿基米德讨论天文学观点的存世作品。

\textbf{平面平衡论}

本书有两卷:第一卷包含七个公设和十五个命题,第二卷包含十个命题。在第一卷中,阿基米德证明了杠杆法则,指出:

在杠杆两端,物体的大小与它们的重量成反比。

阿基米德利用这一原理计算了各种几何图形(如三角形、平行四边形和抛物线)的面积和重心。

\textbf{抛物线的求积}
本书共有24个命题,写给多西修斯。在这部作品中,阿基米德通过两种方法证明了由抛物线与直线所围成的面积是一个底边和高度相等的三角形面积的 \( \frac{4}{3} \) 。在其中一个证明中,他通过计算一个几何级数的值来实现,该级数的公比为 \( \frac{1}{4} \),且求和至无穷大。

\textbf{《球体与圆柱》}

在这部写给多西修斯的两卷本著作中,阿基米德得出了他最自豪的结果,即一个球体与一个外接圆柱之间的关系,二者具有相同的高度和直径。球体的体积为 \( \frac{4}{3} \pi r^3 \),圆柱的体积为 \( 2 \pi r^3 \)。球体的表面积为 \( 4 \pi r^2 \),圆柱的表面积为 \( 6 \pi r^2 \)(包括两个底面),其中 \( r \) 是球体和圆柱的半径。
\begin{figure}[ht]
\centering
\includegraphics[width=6cm]{./figures/fb470a5f1866eb0e.png}
\caption{一个球体的体积和表面积是其外接圆柱体(包括其底面)体积和表面积的2/3。} \label{fig_Archim_8}
\end{figure}
\textbf{《螺线》}

本著作包含28个命题,同样是写给多西修的。该论文定义了现在被称为“阿基米德螺线”的曲线。它是一个点的轨迹,这个点以恒定速度沿一条直线远离固定点,同时该直线以恒定角速度旋转。等价地,在现代极坐标系(\(r,\theta\))中,可以用方程 \( r = a + b\theta \) 来描述,其中 \(a\) 和 \(b\) 为实数。

这是希腊数学家研究的早期机械曲线的一个例子(即由移动点描绘的曲线)。

\textbf{《圆锥体与球体》}

这是一部由32个命题组成的著作,写给Dositheus。在这篇论文中,阿基米德计算了圆锥体、球体和抛物面体的截面面积和体积。

\textbf{《浮体》}

《浮体》有两卷。在第一卷中,阿基米德阐述了流体的平衡定律,并证明水将围绕重心呈球形。这个观点可能是他试图解释当时希腊天文学家如埃拉托斯特尼等人关于地球是圆形的理论。阿基米德描述的流体并非自引力作用的流体,因为他假设存在一个所有物体都向其下落的点来推导球形状。阿基米德的浮力原理在这部作品中给出,陈述如下:

任何完全或部分浸入流体中的物体,都将经历一个向上的浮力,大小等于但方向相反于所排开流体的重量。

在第二部分中,他计算了抛物面体的截面的平衡位置。这可能是对船体形状的理想化。一些截面在水面下浮动,顶部浮出水面,类似冰山浮动的方式。

\textbf{《奥斯托马基翁》}
\begin{figure}[ht]
\centering
\includegraphics[width=6cm]{./figures/14e214577724c0f3.png}
\caption{《奥斯托马基翁》是一个拼图,出现在《阿基米德手稿》中。} \label{fig_Archim_9}
\end{figure}
也被称为阿基米德的“盒子”(Loculus of Archimedes)或“阿基米德的箱子”(Archimedes' Box),这是一种与七巧板相似的拼图,描述它的论文在《阿基米德手稿》中以更完整的形式被发现。阿基米德计算了可以拼成正方形的14块拼图的面积。斯坦福大学的Reviel Netz在2003年提出,阿基米德可能是试图确定这些拼图可以组合成正方形的方式有多少种。Netz计算出这些拼图可以组成正方形的方式有17,152种。[88] 排除通过旋转和反射得到的等效解后,排列数为536种。[89] 这个谜题代表了组合数学中的一个早期问题。

这个谜题名称的来源尚不明确,有人认为它取自古希腊语中的“喉咙”或“食道”一词——stomachos(στόμαχος)。[90] 奥苏尼乌斯称这个谜题为“奥斯托马基翁”(Ostomachion),这是一个由古希腊词根“骨头”(osteon,ὀστέον)和“战斗”(machē,μάχη)构成的复合词。[87]

\textbf{《牛群问题》}  

Gotthold Ephraim Lessing 于 1773 年在德国沃尔芬比特尔的赫尔佐格·奥古斯图斯图书馆发现了这部作品,它是由一首包含 44 行诗句的希腊手稿组成。该作品是写给埃拉托斯特尼和亚历山大里的数学家的。在这部作品中,阿基米德挑战他们通过解一系列同余方程来计算太阳神牛群的牲畜数量。问题有一个更困难的版本,其中要求部分答案为平方数。A. Amthor 于 1880 年首次解出了这个版本的问题,答案是一个非常大的数字,约为 7.760271×10²⁰⁶⁴。

\textbf{《机械定理的方法》}  

这部著作在 1906 年阿基米德手稿的发现之前被认为已经遗失。在这部作品中,阿基米德使用了“不可分割”(indivisibles),并展示了如何通过将一个图形分割成无数个无限小的部分来确定其面积或体积。他可能认为这种方法缺乏形式上的严谨性,因此他还使用了极限法(exhaustion method)来推导结果。与《牛群问题》一样,《机械定理的方法》也是以写信给埃拉托斯特尼的形式呈现的。

\subsubsection{伪书}  
《阿基米德的命题集》或《假设书》是一本包含15个命题的论著,讨论了圆的性质。已知的最早版本是阿拉伯语的。T. L. Heath 和 Marshall Clagett 认为,这本书不可能是阿基米德以现有形式所写,因为它引用了阿基米德的言论,暗示其内容由其他作者修改。该命题集可能基于阿基米德的某部早期已失传的作品。[93]

也有人声称,阿基米德知道如何通过三角形边长来计算三角形的面积,尽管这一公式首次出现在公元1世纪亚历山大的赫伦的作品中。[d] 其他可疑的归属于阿基米德的作品包括拉丁诗《重量与度量之歌》(4世纪或5世纪),其中描述了使用静力天平解决皇冠问题的技术,以及12世纪的《金属图谱》一书,其中包含了通过计算金属的比重来进行金属检验的操作方法。[95][96]
\subsubsection{《阿基米德帕林普塞斯特》}
\begin{figure}[ht]
\centering
\includegraphics[width=8cm]{./figures/43cb0872a594c847.png}
\caption{在1906年,阿基米德帕林普塞斯特手稿揭示了被认为已经失传的阿基米德作品。} \label{fig_Archim_10}
\end{figure}
包含阿基米德作品的最重要文献是《阿基米德帕林普塞斯特手稿》。1906年,丹麦教授约翰·路德维希·海贝格访问了君士坦丁堡,检查了一卷174页的山羊皮羊皮纸祈祷文,这些祈祷文写于13世纪,此前他曾阅读过七年前由帕帕多普洛斯-凯拉梅乌斯出版的简短抄本。[97][98]他确认这确实是一本帕林普塞斯特手稿,即在删除旧作品后,新的文字覆盖其上。帕林普塞斯特是通过刮除现有作品的墨水并重新使用纸张制作的,这在中世纪是一种常见的做法,因为羊皮纸昂贵。学者们通过分析,确认手稿中的旧作品是10世纪的阿基米德失传论文的抄本。[97][99]这卷羊皮纸在君士坦丁堡的一座修道院图书馆中存放了数百年,直到1920年代才卖给了私人收藏家。1998年10月29日,它以220万美元的价格在拍卖会上被匿名买家购买。[100][101]

这份帕林普塞斯特手稿包含了七篇论文,其中包括《浮体原理》的唯一幸存的原始希腊文版本。它是《机械定理法》的唯一已知来源,曾被《苏达词典》提及,并被认为永远失传。此外,手稿中还发现了《胃题》拼图的完整版本,提供了比以前文献中更为详细的分析。这份手稿曾存放在马里兰州巴尔的摩的沃尔特斯艺术博物馆,并进行了现代化的一系列测试,包括使用紫外线和X射线光读取覆盖的文字。[102]之后它被返回给了匿名所有者。[103][104]

《阿基米德帕林普塞斯特手稿》中的论文包括:
\begin{itemize}
\item 《平面平衡论》
\item 《螺旋论》
\item 《圆的测量》
\item 《球体与圆柱体》
\item 《浮体原理》
\item 《机械定理法》
\item 《胃题》
\item 公元前4世纪政治家海佩里得斯的演讲
\item 亚里士多德《范畴》注释
\item 其他作品
\end{itemize}
\subsection{遗产}  
阿基米德有时被称为数学和数学物理学的奠基人,他对数学和科学产生了广泛的影响。[105]
\subsubsection{数学与物理}
\begin{figure}[ht]
\centering
\includegraphics[width=6cm]{./figures/fdc3fb0882c3d726.png}
\caption{柏林的阿基米德青铜雕像} \label{fig_Archim_11}
\end{figure}
科学和数学史学家几乎一致认为,阿基米德是古代最杰出的数学家。例如,埃里克·坦普尔·贝尔(Eric Temple Bell)写道:

任何列出三位‘最伟大’数学家的名单都会包括阿基米德的名字。通常与他并列的还有牛顿和高斯。一些人认为,考虑到这些巨人生活的时代在数学和物理科学上的相对财富——或贫乏——以及将他们的成就与时代背景进行比较,他们可能会将阿基米德排在首位。

同样,阿尔弗雷德·诺思·怀特海德和乔治·F·西蒙斯也曾评价阿基米德:

…公元1500年,欧洲知道的知识远不及公元前212年去世的阿基米德……[107]

如果我们考虑从古至今,在每一个大洲、每一种文明中,所有其他人在数学与物理学方面的成就,从时间的起点一直到17世纪的西欧,阿基米德的成就超过了一切。他是一个伟大的文明。[108]

斯坦福大学希腊数学与天文学教授、阿基米德专家瑞维尔·内茨指出:

因此,由于阿基米德在形成微积分方面的贡献无人能出其右,并且他是将数学应用于物理世界的先驱,事实证明,西方科学不过是阿基米德的脚注。因此,阿基米德是有史以来最重要的科学家。[109]

列奥纳多·达·芬奇多次表达对阿基米德的钦佩,并将他的发明“阿基托纳雷”归功于阿基米德。[110][111][112] 伽利略称他为“超人”和“我的导师”[113][114],而惠更斯则说:“我认为阿基米德无与伦比”,在早期的工作中有意识地模仿他。[115] 莱布尼茨曾说:“理解了阿基米德和阿波罗尼乌斯的人,会更少去钦佩后世伟人的成就。”[116] 高斯的英雄是阿基米德和牛顿,[117] 莫里茨·坎托尔(曾在哥廷根大学师从高斯)报告说,高斯曾在谈话中提到:“历史上只有三位划时代的数学家:阿基米德、牛顿和爱因斯坦。”[118]

发明家尼古拉·特斯拉称赞他:

阿基米德是我的理想。我钦佩艺术家的作品,但在我看来,那些作品仅仅是影像和表象。我认为,发明家才是为世界创造出可触摸、能活生生工作的新事物的人。[119]
\subsubsection{荣誉与纪念}
月球上有一个以阿基米德命名的陨石坑(位于29.7°N 4.0°W),以及一座山脉,名为阿基米德山脉(位于25.3°N 4.6°W)[120]。

菲尔兹奖(数学领域的杰出成就奖)上刻有阿基米德的肖像,并雕刻有他关于球体与圆柱体的证明图案。阿基米德头像周围的铭文引用了公元1世纪诗人马尼利乌斯的名句,拉丁文原文为:“Transire suum pectus mundoque potiri”(“超越自我,掌控世界”)[121][122][123]。

阿基米德曾出现在东德(1973年)、希腊(1983年)、意大利(1983年)、尼加拉瓜(1971年)、圣马力诺(1982年)和西班牙(1963年)发行的邮票上[124]。

阿基米德的名言“Eureka!”(我找到了!)是加利福尼亚州的州座右铭。在此背景下,这个词指的是1848年在萨特磨坊附近发现黄金的事件,激发了加利福尼亚淘金热[125]。