% 数学分析笔记 2

本文参考 \cite{Rudin}.

\subsection{Chap 4. 连续性}

\begin{itemize}
\item 4.1 \textbf{函数的极限}:令 $f:E\subset X\to Y$, $X,Y$ 为度量空间, 且 $p$ 是 $E$ 的极限点. 凡是我们写当 $x\to p$ 时 $f(x)\to q$, 或 $\lim_{x\to p}f(x)=q$, 就是存在 $q\in Y$ 具有以下性质: 对每个 $\epsilon>0$, 存在 $\delta>0$, 使 $d_{Y}(f(x),q)<\epsilon$ 对满足 $0<d_X(x,p)<\delta$ 的一切点 $x\in E$ 成立.

\item 4.2 $\lim_{x\to p}f(x)=q$ 当且仅当 $\lim_{n\to\infty} f(p_n)=q$ 对 $E$ 中满足 $p_n\ne p$, $\lim_{n\to\infty} p_n=p$ 的每个序列 $\{p_n\}$ 成立.

\item 如果 $f$ 在 $p$ 有极限, 那么它是唯一的

\item 4.4 $\lim_{x\to p}f(x)=A$, $\lim_{x\to p}g(x)=B$, 那么 (a) $\lim_{x\to p}(f+g)(x)=A+B$, (b) $\lim_{x\to p}(fg)(x)=AB$, (c) $\lim_{x\to p}(f/g)(x)=A/B$(若 $B\ne 0$).

\item 4.6 连续性: $\lim_{x\to p}f(x)=f(p)$.

\item 4.7 $h(x) = g(f(x))$. 如果 $f$ 在点 $p$ 连续, 且 $g$ 在点 $f(p)$ 连续, 那么 $h$ 在点 $p$ 连续.

\item 4.8 定理: 度量空间 $X,Y$ 的函数 $f:X\to Y$ 连续, 当且仅当对 $Y$ 的每个开集 $V$, $f^{-1}(V)$ 是 $X$ 中的开集. 这是连续性的一个极有用的特征.

\item 推论: 度量空间 $X,Y$ 的函数 $f:X\to Y$ 连续, 当且仅当对 $Y$ 的每个闭集 $C$, $f^{-1}(C)$ 是 $X$ 中的闭集.

\item 4.9 设 $f,g$ 是度量空间 $X$ 上的复连续函数, 那么 $f+g$, $fg$ 与 $f/g$ 在 $X$ 上连续. 在最后一个中, 假定对一切 $x\in X$, $g(x)\ne 0$.

\item (a) 度量空间上的$\bvec f:X\to R^k$ 连续当且仅当每个分量函数都连续. (b) 如果 $\bvec f, \bvec g:X\to R^k$ 连续, 那么 $\bvec f+\bvec g$ 与 $\bvec f\vdot \bvec g$ 都在 $X$ 上连续.

\item 4.13 有界: $\abs{\bvec f(x)}\leqslant M$.

\item 4.14 若度量空间 $X,Y$ 的函数 $f:X\to Y$ 是连续的, 那么 $f(X)$ 是紧的.

\item 4.15 如果 $\bvec f$ 是把紧度量空间 $X$ 映入 $R^k$ 内的连续映射, 那么 $\bvec f(X)$ 是闭的和有界的. 因此 $\bvec f$ 是有界的.

\item 4.16 如果 $\bvec f$ 是紧度量空间 $X$ 上的连续实函数, 且 $M = \sup_{p\in X} f(p)$, $m=\inf_{p\in X} f(p)$, 那么一定存在 $r,s\in X$ 使 $f(r)=M$ 以及 $f(x)=m$.

\item 4.17 设 $f$ 是把紧度量空间 $X$ 映满度量空间 $Y$ 的连续 1-1 映射, 那么逆映射 $f^{-1}$ 是 $Y$ 映满 $X$ 的连续映射.

\item 4.18 对度量空间 $X,Y$ 的函数 $f:X\to Y$, 称 $f$ 在 $X$ 上一致连续, 若对每个 $\epsilon>0$ 总存在 $\delta >0$ 对一切满足 $d_X(p,q)<\delta$ 的 $p,q\in X$ 都能使 $d_\gamma(f(p),f(q))<\epsilon$.

\item 4.19 设 $f$ 是把紧度量空间 $X$ 映入度量空间 $Y$ 的连续映射. 那么 $f$ 在 $X$ 上一致连续.

\item 4.20 设 $E$ 是 $R^1$ 中的非紧集, 那么 (a) 有在 $E$ 上连续却无界的函数. (b) 有在 $E$ 上连续且有界, 却没有最大值的函数. (c) 如果 $E$ 是有界的, 有在 $E$ 上连续却不一致连续的函数.

\item 4.22 设 $f$ 是把连通的度量空间 $X$ 映入度量空间 $Y$ 内的连续映射, $E$ 是 $X$ 的连通子集, 那么 $f(E)$ 是连通的.

\item 4.25 设 $f$ 定义在 $(a,b)$ 上, 定义一点的\textbf{左极限}和\textbf{右极限}…… 显然极限存在当且仅当左极限和右极限存在且相等. 如果函数在一点不连续, 就说在这点\textbf{间断(discontinuous)}.

\item 4.26 设 $f$ 定义在 $(a,b)$ 上, 如果 $f$ 在一点 $x$ 间断, 并且如果 $f(x+)$ 和 $f(x-)$ 都存在, 就说 $f$ 在 $x$ 发生了\textbf{第一类间断}. 其他间断称为\textbf{第二类间断}.

\item 4.28 设 $f:(a,b)\to R$, 若 $a<x<y<b$ 时有 $f(x)\leqslant f(y)$, 就说 $f$ 在 $(a,b)$ 上\textbf{单调递增}; 若有 $f(x)\geqslant f(y)$ 就是\textbf{单调递减}. 二者统称为\textbf{单调函数}.

\item 4.30 设 $f$ 在 $(a,b)$ 上单调, 那么 $(a,b)$ 中使 $f$ 间断的点最多是可数的.

\item 4.31 间断点不一定是孤立点. 

\item 4.32 对任意 $c\in R$, 集合 ${x|x>c}$ 叫做 $+\infty$ 的一个邻域, 记为 $(c,+\infty)$. 类似地, $(-\infty, c)$ 是 $-\infty$ 的一个邻域.

\item 4.33 把函数的极限用领域的语言拓展到了广义实数系.
\end{itemize}

\subsection{Chap 5. 微分法}
\begin{itemize}
\item 5.1 \textbf{导数(导函数)}: 定义在 $[a,b]$ 上的实值函数, …… $f'(x) = \lim_{t\to x} [f(t)-f(x)]/(t-x)$.

\item 如果 $f'$ 在点 $x$ 有定义, 就说 $f$ 在 $x$ 点\textbf{可微}或\textbf{可导}. 如果在 $E\subset [a,b]$ 的每一点有定义, 就说 $f$ 在 $E$ 上可微.

\item 5.2 …… 若 $f$ 在 $x\in [a,b]$ 可微(可导), 那么它在 $x$ 点连续.

\item 5.5 …… $h'(x) = g'(f(x))f'(x)$.

\item 设 $f$ 是定义在度量空间 $X$ 上的实函数, 说 $f$ 在点 $p\in X$ 取得\textbf{局部极大值}, 如果存在 $\delta>0$, 对任意 $q\in X$ 且 $d(p,q)<\delta$ 有 $f(q)\leqslant f(p)$. 局部极小值的定义类似.

\item 5.8 设 $f$ 定义在 $[a,b]$ 上; $x\in [a,b]$, 如果 $f$ 在点 $x$ 取得局部极大值而且 $f'(x)$ 存在, 那么 $f'(x) = 0$.

\item 5.9 设 $f,g$ 是 $[a,b]$ 上的连续实函数, 他们在 $(a,b)$ 内可微, 那么就有一点 $x\in (a,b)$, 使 $[f(b)-f(a)]g'(x) = [g(b)-g(a)]f'(x)$

\item 5.10 \textbf{一般中值定理}: 设 $f$ 是定义在 $[a,b]$ 的连续实函数, 在 $(a,b)$ 内可微, 那么一定有一点 $x\in (a,b)$, 使得 $f(b)-f(a) = (b-a)f'(x)$.

\item 5.12 设 $f$ 是 $[a,b]$ 上实值可微函数, 设 $f'(a)<\lambda<f'(b)$, 那么必有一点 $x\in (a,b)$ 使 $f'(x) = \lambda$.

\item 推论: 如果 $f$ 在 $[a,b]$ 上可微, 那么 $f'$ 在 $[a,b]$ 上不能有简单间断. 但 $f'$ 可能有第二类间断.

\item 5.13 洛必达(L'Hospital)法则: 假设实函数 $f,g$ 在 $(a,b)$ 内可微, 而且对所有 $x\in (a,b)$, $g'(x)\ne 0$. 这里 $-\infty\leqslant a<b\leqslant+\infty$. 已知 $\lim_{x\to a} f'(x)/g'(x) = A$, 如果 $\lim_{x\to a} f(x) = \lim_{x\to a} g(x) = 0$, 或 $\lim_{x\to a} g(x) = +\infty$, 那么 $\lim_{x\to a} f(x)/g(x)= A$.

\item 5.14 如果 $f$ 在一个区间由导数 $f'$, 而 $f'$ 自身又是可微的, 把 $f'$ 的导数记为 $f''$, 叫做\textbf{二阶导数}. 这样继续下去得到 $f, f', f'', f^{(3)},\dots, f^{(n)}$, 其中 $f^{(n)}$ 叫做 $n$ 阶导数.

\item 5.15 Taylor 定理: 设 $f$ 是 $[a, b]$ 上的实函数, $n$ 是正整数, $f^{(n-1)}$ 在 $[a, b]$ 上连续, $f^{(n)}(t)$ 对每个 $t \in(a, b)$ 存在. 设 $\alpha, \beta$ 是 $[a, b]$ 中的不同的两点, 再规定 $P(t)=\sum_{k=0}^{n-1} \frac{f^{(k)}(\alpha)}{k !}(t-\alpha)^{k}$. 那么, 在 $\alpha$ 与 $\beta$ 之间一定存在着一点, $x$, 使得 $f(\beta)=P(\beta)+\frac{f^{(n)}(x)}{n !}(\beta-\alpha)^{n}$.

\item 5.19 设 $\mathbf{f}$ 是把 $[a, b]$ 映人 $R^{k}$ 内的连续映射, 并且 $\mathbf{f}$ 在 $(a, b)$ 内可微, 那么, 必有 $x \in(a, b)$, 使得 $|\mathbf{f}(b)-\mathbf{f}(a)| \leqslant(b-a)\left|\mathbf{f}^{\prime}(x)\right|$.
\end{itemize}


\subsection{Chap 6. Riemann-Stieltjes 积分}
\begin{itemize}
\item 6.2 函数 $f$ 关于单调递增函数 $\alpha$ 在 Riemann 意义上可积, 记为 $f\in \mathscr{R}(\alpha)$: $\int_a^b f(x) \dd\alpha(x)$ 或 $\int_a^b f\dd{\alpha}$.

\item 6.6 在 $[a,b]$ 上 $f\in\mathscr{R}(\alpha)$ 当且仅当对任意的 $\epsilon>0$, 存在一个分法 $P$ 使 $U(P,f,\alpha)-L(P,f,\alpha)<\epsilon$.

\item 6.8 如果 $f$ 在 $[a,b]$ 上连续, 那么在 $[a,b]$ 上 $f\in \mathscr{R}(\alpha)$.

\item 6.9 如果 $f$ 在 $[a,b]$ 上单调, $\alpha$ 在 $[a,b]$ 上连续, 那么 $f\in \mathscr{R}(\alpha)$.

\item 6.10 假设 $f$ 在 $[a,b]$ 上有界, 只有有限个间断点. $\alpha$ 在 $f$ 的每个间断点上连续, 那么 $f\in \mathscr{R}(\alpha)$.

\item 6.11 假设在 $[a,b]$ 上 $f\in \mathscr{R}(\alpha)$, $m\leqslant f\leqslant M$. $\phi$ 在 $[m, M]$ 上连续, 并且在 $[a,b]$ 上 $h(x) = \phi(f(x))$. 那么在 $[a,b]$ 上 $h\in \mathscr{R}(\alpha)$.

\item 6.12 (a) 如果在 $f_1,f_2 \in \mathscr{R}(\alpha)$, 那么 $f_1+f_2 \in \mathscr{R}(\alpha)$. 对任意常数 $c$, $cf\in \mathscr{R}(\alpha)$, 并且 $\int_a^b (f_1+f_2)\dd{\alpha} = \int_a^bf_1\dd{\alpha} + \int_a^bf_2\dd{\alpha}$. $\int_a^b cf\dd{\alpha} = c\int_a^b f\dd{\alpha}$.

\item  6.12 (b) 如果在 $[a,b]$ 上 $f_1\leqslant f_2$, 那么 $\int_a^bf_1\dd{\alpha} \leqslant \int_a^bf_2\dd{\alpha}$.

\item  6.12 (c) 如果在 $[a,b]$ 上 $f\in \mathscr{R}(\alpha)$, 并且 $a<c< b$, 那么在 $[a,c]$ 及 $[c,b]$ 上 $f\in \mathscr{R}(\alpha)$, 并且 $\int_a^cf\dd{\alpha}+\int_c^bf\dd{\alpha} = \int_a^bf\dd{\alpha}$

\item 6.12 (d) 如果在 $[a,b]$ 上 $f\in \mathscr{R}(\alpha)$ 并且 $[a,b]$ 上 $\abs{f(x)}\leqslant M$, 那么 $\abs{\int_a^bf\dd{\alpha}} \leqslant M[\alpha(b)-\alpha(a)]$.

\item 6.12 (e) 如果 $f\in \mathscr{R}(\alpha_1)$ 并且 $f\in \mathscr{R}(\alpha_2)$, 那么 $f\in \mathscr{R}(\alpha_1+\alpha_2)$ 并且 $\int_a^bf\dd{(\alpha_1+\alpha_2)} = \int_a^bf\dd{\alpha_1}+ \int_a^bf\dd{\alpha_2}$. 如果 $f\in \mathscr{R}(\alpha)$ 且 $c$ 是正常数, 那么 $f\in \mathscr{R}(c\alpha)$ 而且 $\int_a^bf\dd{(c\alpha)} = c\int_a^bf\dd{\alpha}$

\item 如果在 $[a,b]$ 上 $f,g\in \mathscr{R}(\alpha)$ 那么 (a) $fg\in \mathscr{R}(\alpha)$; (b) $\abs{f}\in \mathscr{R}(\alpha)$ 而且 $\abs{\int_a^bf\dd{\alpha}} \leqslant \int_a^b\abs{f}\dd{\alpha}$

\item 6.17 $\int_a^b f\dd{\alpha} = \int_a^b f(x)\alpha'(x)\dd{x}$.

\item 6.20 设在 $[a,b]$ 上 $f\in \mathscr{R}$, 对于 $a\leqslant x\leqslant b$, 令 $F(x) = \int_a^x f(t)\dd{t}$. 那么 $F$ 在 $[a,b]$ 上连续; 如果 $f$ 又在 $[a,b]$ 的 $x_0$ 点连续, 那么 $F$ 在 $x_0$ 可微, 并且 $F'(x_0) = f(x_0)$.

\item 6.21 微积分基本定理: 如果在 $[a,b]$ 上 $f\in \mathscr{R}$. 在 $[a,b]$ 上又有可微函数 $F$ 满足 $F' = f$, 那么 $\int_a^b f(x)\dd{x} = F(b)-F(a)$.

\item 6.22 分部积分:假定 $F$ 和 $G$ 都是 $[a, b]$ 上的可微函数. $F^{\prime}=f \in \mathscr{R}$, $G^{\prime}=g \in \mathscr{R}$. 那么 $\int_{a}^{b} F(x) g(x) \mathrm{d} x=F(b) G(b)-F(a) G(a)-\int_{a}^{b} f(x) G(x) \mathrm{d} x$.

\item 6.23 设 $f_{1}, \cdots, f_{k}$ 是 $[a, b]$ 上的实函数, 并设 $\boldsymbol{f}=\left(f_{1}, \cdots, f_{k}\right)$ 是将 $[a, b]$ 映人 $R^{k}$ 内的映射. 如果 $\alpha$ 在 $[a, b]$ 上单调递增, 那么说 $\boldsymbol f \in \mathscr{R}(\alpha)$, 指的就是对于 $j=1,2, \cdots, k$, $f_{j} \in \mathscr{R}(\alpha)$. 果真如此的话, 就定义 $\int_{a}^{b} \boldsymbol{f} \mathrm{d} \alpha=\left(\int_{a}^{b} f_{1} \mathrm{~d} \alpha, \cdots, \int_{a}^{b} f_{k} \mathrm{~d} \alpha\right)$. 换句话说, $\int f \mathrm{~d} \alpha$ 是 $R^{k}$ 中的点, 而 $\int f_{j} \mathrm{~d} \alpha$ 是它的第 $j$ 个坐标.

\item 6.24 设 $\boldsymbol{f}$ 及 $\boldsymbol{F}$ 是把 $[a, b]$ 映入 $R^{k}$ 的映射, $\boldsymbol{f}$ 在 $[a, b]$ 上 $\in \mathscr{R}$ 并且 $F^{\prime}=f$, 那么 $\int_{a}^{b} \boldsymbol{f}(t) \mathrm{d} t=\boldsymbol{F}(b)-\boldsymbol{F}(a)$

\item 6.25 如果 $\boldsymbol f$ 是把 $[a, b]$ 映入 $R^{k}$ 内的映射, 并且对于 $[a, b]$ 上的某个单调递增函数 $\alpha$, $\boldsymbol f \in \mathscr{R}(\alpha)$, 那么 $|\boldsymbol f| \in \mathscr{R}(\alpha)$, 而且 $\left|\int_{a}^{b} \boldsymbol f \mathrm{~d} \alpha\right| \leqslant \int_{a}^{b}|\boldsymbol f| \mathrm{d} \alpha$.

\item 6.26 将闭区间 $[a, b]$ 映人 $R^{k}$ 的映射 $\gamma$ 叫做 $R^{k}$ 里的曲线. 也可以说 $\gamma$ 是 $[a , b]$ 上的曲线. 假如 $\gamma$ 是一对一的, $\gamma$ 就称作\textbf{弧}. 假如 $\gamma(a)=\gamma(b)$; 就说 $\gamma$ 是\textbf{闭曲线}. 结合着 $R^{k}$ 里的每个曲线 $\gamma$, 总有 $R^{k}$ 的一个子集,即是 $\gamma$ 的\textbf{值域},但是不同的曲线可以有相同的值域.
我们给 $[a, b]$ 的每个分法 $P=\left\{x_{0}, \cdots, x_{n}\right\}$ 和 $[a, b]$ 上的每个曲线 $\gamma$, 配置一个数 $\Lambda(P, \gamma)=\sum_{i=1}^{n}\left|\gamma\left(x_{i}\right)-\gamma\left(x_{i-1}\right)\right|$. 我们把 $\Lambda(\gamma)=\sup \Lambda(P, \gamma)$
定义作 $\gamma$ 之长; 这里的 sup 是对 $[a, b]$ 的一切分法来取的. 假若 $\Lambda(\gamma)<\infty$, 就说 $\gamma$ 是可求长的.

\item 6.27 假如 $\gamma^{\prime}$ 在 $[a, b]$ 上连续, $\gamma$ 便是可求长的, 而且 $\Lambda(\gamma)=\int_{a}^{b}\left|\gamma^{\prime}(t)\right| \mathrm{d} t$
\end{itemize}

\subsection{Chap 7. 函数序列与函数项级数}

\begin{itemize}
\item 7.1 假设 $n=1,2,...$, $\{f_n\}$ 是一个定义在集 $E$ 上的函数序列, 再假设数列 $\{f_n(x)\}$ 对每个 $x\in E$ 收敛. 我们便可以由 $f(x) = \lim_{n\to\infty} f_n(x)$ ($x\in E$) 确定一个函数 $f$. 这时我们说 $\{f_n\}$ 在 $E$ 上收敛. $f$ 是 $\{f_n\}$ 的极限或\textbf{极限函数}.

\item 类似地, 如果对每个 $x\in E$, $\sum f_n(x)$ 收敛, 如果定义 $f(x) = \sum_{n=1}^\infty f_n(x)$ ($x\in E$), 就说函数 $f$ 是级数 $\sum f_n$ 的和.

\item …… 所以, 积分的极限和极限的积分, 即使两者都是有限的, 也未必相等.

\item 7.7 如果对每一个 $\epsilon >0$, 有一个整数 $N$, 使得 $n\geqslant N$ 时, 对一切 $x\in E$, 有 $\abs{f_n(x)-f(x)} \leqslant \epsilon$, 我们就说函数序列在 $E$ 上\textbf{一致收敛}于函数 $f$. 一致收敛必定\textbf{逐点收敛}.

\item 7.8 $E$ 上的函数序列 $\{f_n\}$ 在 $E$ 上一致收敛, 当且仅当对每个 $\varepsilon>0$, 存在一个整数 $N$ 使得 $m,n\geqslant N$ 和 $x\in E$ 时, $\abs{f_n(x)-f_m(x)}\leqslant \varepsilon$.

\item 7.9 假设 $\lim_{n\to\infty} f_n(x) = f(x)$ ($x\in E$). 令 $M_n = \sup_{x\in E} \abs{f_n(x)-f(x)}$. 那么在 $E$ 上 $f_n\to f$ 是一致的, 当且仅当 $n\to \infty$ 时, $M_n\to 0$.

\item 7.10 假设 $\left\{f_{n}\right\}$ 是定义在 $E$ 上的函数序列. 并且, 假设 $\left|f_{n}(x)\right| \leqslant M_{n} \quad(x \in E, n=1,2,3, \cdots)$ 如果 $\Sigma M_{n}$ 收敛, 那么, $\Sigma f_{n}$ 便在 $E$ 上一致收敛.

\item 7.11 假设在度量空间内的集 $E$ 上 $f_{n}$ 一致收敛于 $f$. 设 $x$ 是 $E$ 的极限点, 那么 $\lim_{t\to x}\lim_{n\to\infty} f_n(t) = \lim_{n\to\infty}\lim_{t\to x} f_n(t)$

\item 7.12 如果 $\{f_n\}$ 是 $E$ 上的连续函数序列. 并且在 $E$ 上 $f_n$ 一致收敛于 $f$. 那么, $f$ 在 $E$ 上连续.

\item 7.13 假定 $K$ 是紧集. 并且 (a) $\left\{f_{n}\right\}$ 是 $K$ 上的连续函数序列, (b) $\left\{f_{n}\right\}$ 在 $K$ 上逐点收敛于连续函数 $f$, (c) 对于一切 $x \in K$ 和 $n=1,2,3, \cdots, f_{n}(x) \geqslant f_{n+1}(x)$. 那么在 $K$ 上 $f_{n} \rightarrow f$ 是一致的.

\item 7.14 如果 $X$ 是度量空间, $\mathscr C(X)$ 就表示以 $X$ 为定义域的复值连续有界函数的集. 给每个 $f\in\mathscr C(X)$ 配置\textbf{上确范数} $\norm{f} = \sup_{x\in X} \abs{f(x)}$. 因为 $f$ 有界, 所以 $\norm{f}<\infty$. 只有当 $f(x)=0$ 时才有 $\norm{f}=0$. 如果 $h=f+g$, 那么对一切 $x\in X$ 有 $\abs{h(x)}\leqslant \abs{f(x)} + \abs{g(x)}\leqslant \norm{f}+\norm{g}$. 所以 $\norm{f+g}\leqslant \norm{f}+\norm{g}$. 定义 $f,g\in\mathscr C(X)$ 之间的距离为 $\norm{f-g}$. 于是 $\mathscr C(X)$ 变为度量空间.

\item 复述 7.9: 对于 $\mathscr C(X)$ 度量来说, 序列 $\{f_n\}$ 收敛于 $f$ 当且仅当 $f_n$ 在 $X$ 上一致收敛于 $f$.

\item 7.15 以上度量使 $\mathscr C(X)$ 成为完备度量空间.

\item 7.16 设函数 $\alpha$ 在 $[a,b]$ 上单调递增. 假定在 $[a,b]$ 上 $f_n\in\mathscr R(\alpha)$, $n=1,2,\dots$. 再假设 $[a,b]$ 上 $f_n\to f$ 是一致的, 那么 $[a,b]$ 上 $f\in\mathscr R$, 而且 $\int_a^b f\dd{\alpha} = \lim_{n\to\infty}\int_a^b f_n\dd{\alpha}$.

\item 7.17 设 $\{f_n\}$ 是 $[a,b]$ 上的可微函数序列, 且 $[a,b]$ 上有某点 $x_0$ 使 $\{f_n(x_0)\}$ 收敛. 如果 $\{f'_n\}$ 在 $[a,b]$ 上一致收敛, 那么 $\{f_n\}$ 就在 $[a,b]$ 上一致收敛于某函数 $f$; 且 $f'(x) = \lim_{n\to\infty} f'_n(x)$($a\leqslant x \leqslant b$).

\item 7.18 实轴上有处处不可微的实连续函数.

\item 7.19 令 $\{f_n\}$ 为集合 $E$ 上的函数序列. 说 $\{f_n\}$ 在 $E$ 上\textbf{逐点有界}, 如果对每个 $x\in E$, 序列 $\{f_n\}$ 是有界的. 也就是说: 如果存在一个定义在 $E$ 上的有限值函数 $\phi$, 使 $\abs{f_n(x)}<\phi(x)$ ($x\in E$, $n=1,2,\dots$). 我们说 $\{f_n\}$ 在 $E$ 上\textbf{一致有界}, 如果存在一个数 $M$, 使 $\abs{f_n(x)}<M$ ($x\in E$, $n=1,2,\dots$).

\item 7.22 $f$ 是定义在度量空间 $X$ 的子集 $E$ 上的函数, $\mathscr F$ 是 $f$ 的族. 说 $\mathscr F$ 在 $E$ 上\textbf{等度连续(equicontinuous)}, 就是对每个 $\varepsilon>0$ 存在一个 $\delta >0$, 只要 $d(x,y)<\delta$, $x,y\in E$ 以及 $f\in \mathscr F$ 就能使 $\abs{f(x)-f(y)}<\varepsilon$.

\item 7.23 假若 $\left\{f_{n}\right\}$ 是在可数集 $E$ 上逐点有界的复值函数序列, 那么 $\left\{f_{n}\right\}$ 便有子序列 $\left\{f_{n_{k}}\right\}$, 使得 $\left\{f_{n_{k}}(x)\right\}$ 对于每个 $x \in E$ 收敛.

\item 7.24 若 $K$ 是紧度量空间, $f_n\in \mathscr C(K)$, $n=1,2,\dots$, 而且 $\{f_n\}$ 在 $K$ 上一致收敛, 那么 $\{f_n\}$ 在 $K$ 上等度连续.

\item 7.25 若 $K$ 是紧集, $f_n\in \mathscr C(K)$, $n=1,2,\dots$, 而且 $\{f_n\}$ 在 $K$ 上逐点有界又等度连续, 那么 (a) $\{f_n\}$ 在 $K$ 上一致有界, (b) $\{f_n\}$ 含有一致收敛的子序列.

\item 7.26 \textbf{Stone-Weierstrass 定理}: 如果 $f$ 是 $[a,b]$ 上的一个连续复函数, 那么就有多项式 $P_n$ 的序列, 使得 $\lim_{n\to\infty} P_n(x) = f(x)$ 在 $[a,b]$ 上一致成立. 如果 $f$ 是实函数, $P_n$ 可以是实多项式.

\item 7.27 在每个闭区间 $[-a,a]$ 上, 必有实多项式 $P_n$ 的序列, 满足 $P_n(0) = 0$ 且 $\lim_{n\to\infty} P_n(x) = \abs{x}$ 一致地成立.

\item 7.28 定义在 $E$ 上的复函数族 $\mathscr A$ 为\textbf{代数}, 若对于一切 $f,g\in \mathscr A$ 来说, (i) $f+g\in \mathscr A$, (ii) $fg\in \mathscr A$, (iii) 对一切复常数 $c$, $cf\in \mathscr A$. 也就是说, 加法、乘法、数乘是封闭的.

\item 如果 $\mathscr A$ 满足, 只要 $f_n\in \mathscr A$($n=1,2,\dots$), 且在 $E$ 上 $f_n\to f$ 一致成立, 就有 $f\in \mathscr A$. 就说 $\mathscr A$ 是\textbf{一致闭}的.

\item 设 $\mathscr B$ 是由 $\mathscr A$ 中所有一致收敛函数序列的及先函数组成的集, 就说 $\mathscr B$ 是 $\mathscr A$ 的\textbf{一致闭包}.

\item Weierstrass 定理可以叙述为: $[a,b]$ 上连续函数的集合是 $[a,b]$ 上的多项式集的一致闭包.

\item 7.29 设 $\mathscr B$ 是有界函数的代数 $\mathscr A$ 的一致闭包. 那么 $\mathscr B$ 是一致闭的代数.

\item 7.30 设 $\mathscr A$ 是集合 $E$ 上的函数族. 说 $\mathscr A$ \textbf{能分离 $E$ 的点}, 就是说对不同的 $x_1,x_2\in E$, 总有函数 $f\in \mathscr A$ , 使 $f(x_1)\ne f(x_2)$.
\end{itemize}


\subsection{Chap 8. 一些特殊函数}

\begin{itemize}
\item 幂级数 $f(x) = \sum_{n=0}^\infty c_n x^n$ 或更一般地 $f(x) = \sum_{n=0}^\infty c_n (x-a)^n$, 都称为\textbf{解析函数}. 我们限制 $x$ 取实数.

\item 8.1 假设对 $\abs{x}<R$, 级数 $\sum_{n=0}^\infty c_n x^n$ 收敛, 并规定 $f(x) = \sum_{n=0}^\infty c_n x^n$ ($\abs{x}<R$). 那么无论选取怎样的 $\varepsilon >0$, 级数在 $[-R+\epsilon, R-\epsilon]$ 上一致收敛, 函数 $f$ 在 $(-R,R)$ 内连续、可微, 且 $f'(x)=\sum_{n=1}^\infty nc_n x^{n-1}$($\abs{x}<R$)

\item 在 8.1 中, $f$ 在 $(-R,R)$ 内有任意阶导数. 他们是 ……

\item 8.3 设有双重序列 $\{a_{ij}\}$, $i,j=1,2,3\dots$. 假设 $\sum_{j=1}^\infty = b_i$ ($i=1,2,\dots$)并且 $\sum b_i$ 收敛, 那么 $\sum_{i=1}^\infty\sum_{j=1}^\infty a_{ij} = \sum_{j=1}^\infty\sum_{i=1}^\infty a_{ij}$

\item 8.4 设 $f(x)=\sum_{n=0}^\infty c_nx^n$, 这级数在 $\abs{x}<R$ 内收敛. 若 $-R<a<R$, $f$ 就可以在 $x=a$ 附近展开为幂级数, 这幂级数在 $\abs{x-a}<R-\abs{a}$ 中收敛, 且有泰勒公式……

\item 定义 $E(z) = \sum_{n=0}^\infty z^n/n!$ 为指数函数, 有加法公式 $E(z+w)=E(z)E(w)$.……  对一切实数 $x$, $E(x)=\E^x$.

\item 定义 $C(x) = [E(ix)+E(-ix)]/2$, $S(x)=[E(ix)-E(-ix)]/(2i)$.

\item 8.9 三角多项式是形如 $f(x) = a_0 + \sum_{n=1}^N (a_n\cos nx + b_n \sin nx)$ ($x\in R$) 的有限和. 其中系数都是复数. 也可以记为 $f(x) = \sum_{-N}^N c_n \E^{inx}$ ($x\in R$). 每个三角多项式以 $2\pi$ 为周期. $c_m = (1/2\pi) \int_{-\pi}^{\pi} f(x)\E^{-\I mx}\dd{x}$. $f$ 是实函数当且仅当 $c_{-n} = \bar c_n$, $n=0,\dots,N$. 定义三角级数为 $\sum_{-\infty}^\infty c_n \E^{inx}$.

\item 8.10 $\left\{\phi_{n}\right\}(n=1,2,3, \cdots)$ 是 $[a, b]$ 上合于 $\int_{a}^{b} \phi_{n}(x) \overline{\phi_{m}(x)} \mathrm{d} x=0 \quad(n \neq m)$ 的复值函数序列. 那么, $\left\{\phi_{n}\right\}$ 叫做 $[a, b]$ 上的函数的正交系. 此外, 若是 对于一切 $n$, $\int_{a}^{b}\left|\phi_{n}(x)\right|^{2} \mathrm{~d} x=1$, $\left\{\phi_{n}\right\}$ 便叫作\textbf{正规正交系 (orthonormal)}.

\item 假若 $\left\{\phi_{n}\right\}$ 是 $[a, b]$ 上的正规正交系, 而且 $c_{n}=\int_{a}^{b} f(t) \overline{\phi_{n}(t)} \mathrm{d} t \quad(n=1,2,3, \cdots)$, 我们便说 $c_{n}$ 是 $f$ 关于 $\left\{\phi_{n}\right\}$ 的第 $n$ 个 Fourier 系数. 我们写作 $f(x) \sim \sum_{1}^{\infty} c_{n} \phi_{n}(x)$. 这并不意味着任何关于级数收敛性的事实.

\item 8.11 设 $\left\{\phi_{n}\right\}$ 是 $[a, b]$ 上的正规正交系. 令 $s_{n}(x)=\sum_{m=1}^{n} c_{m} \phi_{m}(x)$ 是 $f$ 的 Fouricr 级数的第 $n$ 个部分和. 又假定 $t_{n}(x)=\sum_{m=1}^{n} \gamma_{m} \phi_{m}(x) .$ 那么 $\int_{a}^{b}\left|f-s_{n}\right|^{2} \mathrm{~d} x \leqslant \int_{a}^{b}\left|f-t_{n}\right|^{2} \mathrm{~d} x$ 并且, 当且仅当 $\gamma_{m}=c_{m} \quad(m=1, \cdots, n)$ 时才能使等式成立.
\end{itemize}

