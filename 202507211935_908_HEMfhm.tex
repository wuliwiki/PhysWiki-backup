% 赫尔曼·冯·亥姆霍兹(综述)
% license CCBYSA3
% type Wiki

本文根据 CC-BY-SA 协议转载翻译自维基百科\href{https://en.wikipedia.org/wiki/Hermann_von_Helmholtz}{相关文章}。

赫尔曼·路德维希·费迪南德·冯·亥姆霍兹(Hermann Ludwig Ferdinand von Helmholtz,/ˈhɛlmhoʊlts/;德语:[ˈhɛʁman fɔn ˈhɛlmˌhɔlts];1821年8月31日-1894年9月8日,自1883年起冠以“冯(von)”的贵族头衔)是一位德国物理学家和医生,在多个科学领域作出了重要贡献,尤其以流体动力学稳定性理论而闻名【2】。以他命名的亥姆霍兹协会是德国最大的科研机构联合体【3】。

在生理学和心理学领域,亥姆霍兹以其关于眼睛的数学研究、视觉理论、空间视觉感知的观点、色觉研究、音调感觉与听觉感知理论,以及对感知生理学中经验主义的探讨而著称。在物理学中,他以能量守恒定律、电双层理论、电动力学、化学热力学,以及热力学的力学基础研究而闻名。尽管能量守恒原则的发展也归功于尤利乌斯·冯·迈尔、詹姆斯·焦耳和丹尼尔·伯努利等人,但亥姆霍兹被认为是第一个以最一般形式提出能量守恒原理的人【4】。

作为哲学家,亥姆霍兹以其科学哲学、关于知觉规律与自然规律之间关系的见解、美学科学思想,以及关于科学的文明力量等理念而受到关注。到19世纪末,亥姆霍兹发展出一种广义的康德方法论,包括对知觉空间中可能取向的先验确定,这不仅激发了对康德的新解读【4】,也对现代后期的新康德主义哲学运动产生了重要影响【5】。
