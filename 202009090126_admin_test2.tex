% 1

\documentclass[addpoints,letter,11pt]{exam}

% \noaddpoints % if you don't want to count the points
% Specifies the way question are displayed:
\qformat{\textbf{Question~\thequestion}\quad(\thepoints)\hfill}
\usepackage{color} % defines a new color
\definecolor{SolutionColor}{rgb}{0.8,0.9,1} % light blue
\shadedsolutions % defines the style of the solution environment
% \framedsolutions % defines the style of the solution environment
% Defines the title of the solution environment:
\renewcommand{\solutiontitle}{\noindent\textbf{Solution:}\par\noindent}

\usepackage{algpseudocode}
\usepackage{algorithm}
\definecolor{unlred}{RGB}{245,0,47}
\usepackage[colorlinks=true,linkcolor=unlred]{hyperref}

\title{Homework $01$ (Due: Tuesday, September $8, 2020, 11:59:00$PM (Central Time))}
\author{CSCE $310$}
\date{}
\usepackage[margin=1in]{geometry}
\usepackage{amsmath}
\begin{document}
\maketitle

\section*{Instructions}
This assignment consists of $5$ analytical problems and $2$ programming problems. Your solutions to the analytical problems must be submitted, as one PDF \textbf{without spaces, tabs, parentheses, pound signs, and percent signs in the filename}, via webhandin. While handwritten (then scanned) solutions to the analytical problems are acceptable, you are strongly encouraged to typeset your solutions in \LaTeX{} or a word processor with an equation editor. The legibility of your solutions is of great importance.

\subsection*{Programming Assignment}
Your methods will be tested on the \verb+cse.unl.edu+ server, using\\
\verb!gcc version 7.5.0 (SUSE Linux)! and be compiled for C++ 2014. To ensure proper execution, you should test your submission in the \href{http://cse.unl.edu/~cse310/grade}{\color{unlred}webgrader}

You will submit \verb+csce310hmwrk01prt01.h+, \verb+csce310hmwrk01prt02.h+, \verb+csce310hmwrk01prt01.cpp+,  and \verb+csce310hmwrk01prt02.cpp+, along with your PDF, via webhandin.

\subsubsection*{\texttt{isAnagram}}
\verb+isAnagram+ is a function that should take two strings as input a return a Boolean value. \verb+isAnagram+ should return \verb+true+ if the first word is an anagram of the second (consisting of the exact same letters as the second), and \verb+false+ otherwise.

\subsubsection*{\texttt{printPermutations}}
\verb+printPermutations+ is a function that should take one string and one integer $n$ as input and print out the first $n$ permutations of that string in reverse lexicographic (alphabetical) order. Each permutation should appear on a separate line. You may assume the string is already in reverse lexicographic order. Section $4.3$ in \emph{The Design and Analysis of Algorithms} may be of some use.

\subsubsection*{General Guidelines}
Sample header, source, and testing files have been provided. You may modify the \verb+.h+ and \verb+.cpp+ files as needed, but you will only be turning in the four files mentioned above. The webgrader will be compiling the code with the command\\ \verb|g++ -std=c++14 -o /path/to/executable.out /path/to/source/files/*.cpp| for each part. 

\subsection*{Written Assignment}
\begin{questions}
  \question[10]
  %Question $2.1.7$ in \emph{The Design and Analysis of Algorithms}
  \emph{Adapted from Problem $2.1.7$ in The Design and Analysis of Algorithms}\\
  Gaussian elimination, the classic algorithm for solving systems of $n$ linear equations in $n$ unknowns, requires about $\frac{1}{3}n^3$ multiplications, which is the algorithm's basic operation.

  \begin{enumerate}
    \item[a.] How much longer should you expect Gaussian elimination to work on a system of $35$ equations versus a system of $5$ equations?
    \item[b.] You are considering buying a computer that is $25$ times faster than the one you currently have. By what factor will the faster computer increase the sizes of systems solvable in the same amount of time as on the old computer?
  \end{enumerate}

  \question[10] 
  Question $2.1.9$ in \emph{The Design and Analysis of Algorithms}
  
  \question[10] 
  %Question $2.4.1$ in \emph{The Design and Analysis of Algorithms}
  \emph{Adapted from Problem $2.4.1$ in The Design and Analysis of Algorithms}\\
  
  Solve the following recurrence relations.
  \begin{enumerate}
    \item[a.] $x\left(n\right)=x\left(n-1\right)+3$ for $n>1$ and $x\left(1\right)=3$.
    \item[b.] $x\left(n\right)=4x\left(n-1\right)+7$ for $n>0$ and $x\left(0\right)=8$.
    \item[c.] $x\left(n\right)=x\left(n-1\right)+n^2$ for $n>0$ and $x\left(0\right)=9$.
    \item[d.] $x\left(n\right)=x\left(\frac{n}{3}\right)+n$ for $n>1$ and $x\left(1\right)=8$. Solve for $n=3^{k}$.
    \item[e.] $x\left(n\right)=x\left(\frac{n}{7}\right)+5$ for $n>1$ and $x\left(1\right)=9$. Solve for $n=7^{k}$.
  \end{enumerate}
  
  \question[10]
  Question $1.3.1$ in \emph{The Design and Analysis of Algorithms}

  \question[10]
  Question $2.2.5$ in \emph{The Design and Analysis of Algorithms}

  \question[10]
  \textbf{Extra Credit or Honors Contract}
  Question $2.1.4$ in \emph{The Design and Analysis of Algorithms}

\end{questions}

\section*{webgrader Notes}
The webgrader should take roughly $60$ seconds to complete running.

\section*{Point Allocation}
\begin{table}[H]
  \center
  \begin{tabular}{|l|r|}
    \hline
    Question & Points\\
    \hline
    \hline
    Question $1$ & $10$\\
    Question $2$ & $10$\\
    Question $3$ & $10$\\
    Question $4$ & $10$\\
    Question $5$ & $10$\\
    \hline
    \verb+isAnagram+ & \\
    Test Cases & $1\times 15$\\
    Compilation & $10$\\
    \verb+isAnagram+ Total & $25$\\
    \verb+printPermutations+ & \\
    Test Cases & $1\times 15$\\
    Compilation & $10$\\
    \verb+printPermutations+ Total & $25$\\
    \hline
    \hline
    Total & $100$\\
    \hline
  \end{tabular}
\end{table}

\end{document}
