% 晶格热容的德拜理论
% keys 晶格振动|德拜|热容

\pentry{玻尔兹曼分布(统计力学)\upref{MBsta},晶格热容的爱因斯坦理论\upref{EScap}}

根据量子理论,晶格的各个简谐振动模式的能量本征值都是量子化的,为
\begin{equation}
\qty(n_j+\frac{1}{2})\hbar \omega_j
\end{equation}
固体的一部分内能来自于晶格的振动,因此我们这里考虑的晶格热容就与这些简谐振动模式有关.我们需要知道晶格内能关于温度 $T$ 的函数,那么我们希望知道在特定温度下不同简谐振动模式的平均热能.根据玻尔兹曼分布,能量本征值 $\epsilon$ 出现的概率与 $e^{-\epsilon}$ 成正比,因此我们有
\begin{equation}
\overline E_j(T)=\frac{1}{2}\hbar \omega_j + \frac{\sum_{n_j} n_j\hbar \omega_j e^{-n_j \hbar \omega_j / kT}}{\sum_{n_j} e^{-n_j \hbar \omega_j / kT}}
\end{equation}
$\overline E_j$ 代表简谐振动模式 $j$ 的平均能量.令 $\beta=1/kT$,上式可以写成
\begin{equation}
\overline E_j(T)=\frac{1}{2}\hbar \omega_j - \frac{\partial}{\partial \beta} \ln \sum_{n_j} e^{-n_j\beta\hbar\omega_j}=\frac{1}{2}\hbar \omega_j - \frac{\partial}{\partial \beta} \ln Z
\end{equation}
$Z$ 就是单个振动模式下的配分函数.