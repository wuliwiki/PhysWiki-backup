% 地球表面的力

\begin{issues}
\issueDraft
\issueOther{需要verify一下...我看网上好像说支持力不是指向圆心的?}
\end{issues}

\begin{figure}[ht]
\centering
\includegraphics[width=8cm]{./figures/FOE_1.pdf}
\caption{地球表面的力(地球参考系)} \label{FOE_fig1}
\end{figure}

在地球参考系中,地球表面的物体受到多个来源的力:
\begin{itemize}
\item 引力\upref{Gravty}:$G=\frac{GMm}{R^2}$
\item 地面的支持力 $N$
\item 离心力\upref{Centri}\upref{AccTra}:由于地球参考系是一个非惯性参考系,因此地球表面的人感受到额外的离心力。 $f_c = m \omega^2 R$
\item 科氏力\upref{ErthCf}:由于地球参考系是一个非惯性参考系,在地球表面的人观察到运动的物体受额外的科氏力 $f_{col} = 2 m v \omega$
\item 摩擦力;切线方向的支持力 $f$(例如,河床侧岸对水流的支持)
\end{itemize}

本文做了一些简化近似(更准确地说,地球是椭球,因此支持力的方向不完全指向质心),并忽略了一些其他来源的力(例如,月球引起的潮汐力\upref{Tidal})。

\begin{example}{}
估算一个行走的成年人所受到的离心力、科氏力。

取地球半径为
$$R=6400\Si{km}=6.4\e6 \Si{km}$$
地球自转的角速度为
$$\omega = \frac{2\pi}{24*3600}=7.27\e{-5}$$
假定体重为$60 \Si{kg}$, 步行速度为$1 \Si{m/s}$,那么
$$
f_c = m \omega^2 R = 2.03 \Si{N}
$$
$$
f_{col} = 2 m v\omega = 8.27\e{-3} \Si{N}
$$
二者合力为 $2.03 \Si{N}$。

相比于重力的$G=600 \Si{N}$,这二者的合力只是重力的$0.3\%$。但是,这不意味这科氏力不重要,在更大尺度、更高精度的事物上(例如,河流、气旋、制导导弹),科氏力起重要作用。
\end{example}

\begin{example}{地球自转的速度}
本问题受\href{https://www.zhihu.com/question/579214803}{该讨论}(站外链接)启发。

地球的自转线速度约为 $$v=\omega R = 465.3 \Si{m/s}$$,超过了空气中的声速。为什么我们几乎感觉不到地球自转的速度?我们再计算一下自转的加速度
$$a=\omega^2 R= 0.03 \Si{m/s}$$
可见,尽管自转的线速度很大,但是加速度却很小,在一般情况下地球参考系可以近似为惯性参考系。根据相对性原理,任何惯性参考系中物理规律相同,与“参考系的速度”无关\footnote{严谨地说,你甚至不能定义“参考系的速度”,你只能定义一个参考系相对于另一个参考系的相对速度}。因此,我们感受不到地球的线速度是正常的。
\end{example}
