% 康托尔悖论(综述)
% license CCBYSA3
% type Wiki

本文根据 CC-BY-SA 协议转载翻译自维基百科\href{https://en.wikipedia.org/wiki/Cantor\%27s_paradox}{相关文章}。

在集合论中,康托尔悖论指出不存在所有基数的集合。这是由没有最大基数的定理推导出来的。非正式地说,这个悖论的意思是,所有可能的“无限大小”的集合不仅是无限的,而且其无限大小如此之大,以至于它自己的无限大小不能是集合中的任何一个无限大小。这个困难在公理化集合论中通过声明这个集合不是一个集合,而是一个适当类来处理;在冯·诺依曼–伯尔奈–哥德尔集合论中,这个适当类与所有集合的类之间存在双射关系,并且根据大小限制公理可以得出这一点。因此,不仅存在无穷多个无限大,而且这种无限大比它枚举的任何无限大都要大。

这个悖论以乔治·康托尔的名字命名,他通常被认为是在1899年(或1895年至1897年之间)首次识别出这个悖论。像许多“悖论”一样,它实际上并不矛盾,而仅仅是表明了一种错误的直觉,尤其是在关于无限性和集合的概念上。换句话说,它在天真的集合论的框架内是悖论的,因此证明了这个理论的不谨慎公理化是不一致的。
\subsection{命题与证明}
为了陈述悖论,首先需要理解基数是完全有序的,因此可以讨论一个基数是否大于或小于另一个基数。然后,康托尔悖论是:

定理:不存在最大的基数。

这个事实是康托尔关于集合的幂集基数定理的直接结果。

\textbf{证明}:假设相反,设 $C$ 是最大的基数。那么(在冯·诺依曼的基数定义中)$C$ 是一个集合,因此有一个幂集 $2^C$,而根据康托尔定理,幂集的基数严格大于 $C$。通过证明存在一个基数(即 $2^C$ 的基数)大于 $C$,而 $C$ 被假定为最大的基数,这就否定了 $C$ 的定义。这个矛盾证明了这样的基数不能存在。

康托尔定理的另一个结果是,基数构成了一个适当类。也就是说,它们不能作为单一集合的元素全部收集在一起。这里是一个稍微更一般的结果。

\textbf{定理:}如果 $S$ 是任何集合,那么 $S$ 不能包含所有基数的元素。事实上,$S$ 的元素的基数有一个严格的上界。

\textbf{证明:}设 $S$ 是一个集合,$T$ 是 $S$ 的元素的并集。那么,$S$ 的每个元素都是 $T$ 的子集,因此它们的基数小于或等于 $T$ 的基数。根据康托尔定理,每个 $S$ 的元素的基数严格小于 $2^T$ 的基数。
\subsection{讨论与后果}
由于基数通过与序数的索引方式是良序的(见基数的正式定义),这也证明了不存在最大的序数;反过来,后者的陈述也暗示了康托尔悖论。通过将此索引应用于布拉利-福尔提悖论,我们获得了另一个证明,表明基数是一个适当类,而不是一个集合,并且(至少在 ZFC 或冯·诺依曼–伯尔奈–哥德尔集合论中)从这个结论可以得出基数类和所有集合类之间存在双射。由于每个集合是后者类的一个子集,而每个基数是一个集合的基数(根据定义!),这直观地意味着基数集合的“基数”大于任何集合的基数:它比任何真正的无穷大还要更无穷。这就是康托尔“悖论”的悖论性质。
\subsection{历史注释}
虽然通常认为康托尔首次识别出基数集合的这一性质,但一些数学家将这一荣誉归于伯特兰·罗素,他在1899年或1901年定义了一个类似的定理。
\subsection{参考文献}
\begin{itemize}
\item Anellis, I.H. (1991). Drucker, Thomas (编). "第一种罗素悖论",《数学逻辑历史的视角》. 剑桥,马萨诸塞州:Birkäuser Boston. 第33-46页。
\item Moore, G.H.; Garciadiego, A. (1981). "布拉利-福尔提悖论:对其起源的再评估",《数学史》, 8(3): 319-350. doi:10.1016/0315-0860(81)90070-7.
\end{itemize}
\subsection{外部链接}
\begin{itemize}
\item 《由抽象公理引起的集合论悖论的历史记载》:由Justin T. Miller编写,亚利桑那大学数学系报告。
\item PlanetMath.org:文章。
\end{itemize}