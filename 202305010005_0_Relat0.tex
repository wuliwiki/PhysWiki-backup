% 狭义相对论(科普)

\begin{issues}
\issueDraft
\end{issues}

\pentry{经典力学和相对论(科普)\upref{CM0}}

本文的参考系都是指惯性系。

狭义相对论有两个基本假设
\begin{enumerate}
\item 惯性系平权
\item 光速不变
\end{enumerate}

注意这仅仅是两个假设, 就像牛顿三定律一样。 不幸的是由于技术限制我们并没有能力创造一个速度足够快的(例如光速的 5\%)飞船在上面直接测量光速以及其他相对论实验, 但我们有大量间接的方法可以实验验证(链接未完成)。 光速不变常被民间科学家用于攻击相对论, 关于光速不变假设的历史背景和一些误解见 “光速不变假设的一些误解和历史\upref{SpeRel}”。

我们知道经典力学中速度是可以相加的, 如果一个人在火车上向前射击, 那么地面上看来子弹的速度等于火车的速度加上子弹相对枪口或者火车的速度。 所以如果有人告诉你说子弹的速度无论相对于火车还是相对于地面都一样快, 那你肯定会觉得他在胡说。 而爱因斯坦提出的光速不变假设恰恰就是说无论使用哪个惯性参考系, 任何光相对于该参考系的速度都是一样的, 无论这个光是从哪里发出如何发出的。 可见狭义相对论推翻了经典力学的根基——\textbf{时空观}。

\subsection{绝对时空观}\label{sub_Relat0_1}
为了方便讨论时间的概念, 我们不妨认为每个参考系中, 空间中的每一点都挂满了和这个参考系相对静止的时钟, 且这些时钟制作精良没有误差, 不受外界因素干扰, 也不取决于特定的原理(水滴、摆锤、电路等等)。 经典力学的绝对时空观认为, 我们可以一劳永逸地一次让这些钟全部彼此校准, 那么它们就会永远保持绝对的同步,使得无论从什么角度来看, 只要两个事件发生时, 它们所在位置的时钟读数相同, 那么这两个事件就是同时发生的, 毫无歧义。

绝对的空间也可以类似理解。 每个参考系中的长度都是一样的, 火车上的一截棍子无论朝向如何, 如果你在某个时刻把它贴近地面, \textbf{同时}把它的两端的位置在地面做一个标记, 那么地面上这两个标记之间的长度测量出来也等于在火车上测量的棍子的长度。 注意这个概念需要建立在上面对\textbf{同时}的定义的基础上。 做这两个标记就可以看成两个事件, 如果我们无法在所有参考系都对两个事件是否同时达成一致, 那么我们也无法明确火车上的长度和地面上的长度是否一致。 可见时间和空间的概念往往是纠缠在一起的, 这也是为什么经常把它们统称为\textbf{时空}。

由于绝对时空观在相对论提出以前根深蒂固, 所以在当时的人看来光速不变是荒谬的。 因为速度的定义是绝对的距离除以绝对的时间间隔, 那么在一段绝对的时间 $\Delta t = t_2 - t_1$ 内(上文的钟表读数相减), 光在火车上走过的距离 $s_1$ 地上的人不会有歧义, 而火车在这段时间内相对地面还走过了一段距离 $s_0$, 那么由于地面上的距离可以相加(即使在狭义相对论中, 同一个参考系中的距离也是可以相加的), 所以光相对于地面走过的距离为 $s_2 = s_0 + s_1$, 所以把 $s_1, s_2$ 分别除以时长 $\Delta t$, 得到光相对于两个参考系的速度必定是不同的, 任何物体的运动也都一样, 光没有理由例外。

\subsection{相对时空观}
所以如果一定要假设光速不变, 那么就必须要改变时空观。 我们尽量保守地改变上面关于时空的假设, 不多不少, 直到新的时空观能容纳光速不变这个现象为止。 首先是关于同时性的问题, 在一个参考系中, 我们如何确定两个不同位置的时钟同步呢? 既然我们假设光速不变, 那最直接的方法就是从这两个时钟的中点同时向它们各发射一道光, 如果当光到达两个时钟时, 它们的读数相同, 那我们就说它们\textbf{在当前参考系中同时}。 注意我们在使用文字时必须非常谨慎, 因为我们在推翻根植于常识中的认知。 我们还没开始讨论别的参考系中观察到的东西, 所以要强调只是在当前参考系中同时。 另外这个过程中我们还假设了同一个参考系中长度是绝对的, 可以用一把尺子测量任何地方的长度, 否则我们无法确定两个钟的中点在什么位置。

所以在同一个参考系中, 相对论时空观和绝对时空观的区别并不大, 处于不同位置的观察者仍然对两个事件的同时性和空间的长度么有任何争议。 所以要建立相对时空观, 我们需要从具有相对运动的不同参考系的观察者之间如何看待对方入手, 所以还是回到火车的问题。 注意即使在绝对时空观中, 也不存在这两个参考系哪个更优越的问题——地面上的人可以认为自己静止火车在动, 火车上的人也完全可以认为自己静止而火车外的所有物体都在运动(从太阳的参考系看地球也的确如此)。 而在相对论时空观中, 我们已经知道两个参考系分别已经把自己的时钟都分别校准了, 我们还没有对不同参考系之间的时钟做出任何比较。

在进一步讨论之前,我们还要把\textbf{事件}这个概念也做一个抽象。 在给定的参考系中, 仍然假设空间中每一点的时钟都无歧义地进行了同步, 每一个位置也可以用若干坐标描述, 那么一个事件就可以抽象为一个空间位置和该位置的时钟的一个读数, 例如使用三维直角坐标就有 $(x, y, z, t)$。 当然本文只讨论一个方向上的运动, 所以我们只需要一个空间坐标, 所以给定参考系中一个事件可以简化为两个坐标 $(x, t)$。 例如要在火车参考系中把棍子的两端同时在地面上做两个记号(这里的同时仅仅是对火车来说的), 就可以理解为两个事件, 用火车参考系的时空坐标分别描述为 $(x_1', t_1')$ 以及 $(x_2', t_2')$ 并要求 $t_1' = t_2'$。 那么火车参考系中棍子的长度就定义为 $L' = x_2' - x_1'$(这与 $t_1', t_2'$ 都无关, 所以测量自己参考系中静止物体的长度并不需要同时读两端的坐标)。 为了区分两个参考系, 我们把火车参考系的时空坐标后面都加一撇, 而地面参考系的坐标则不加。

在绝对时空观中, 同一个事件在不同的参考系中一般也会具有不同的空间坐标, 但时间坐标是与参考系无关的, 两个空间坐标之间的距离也与参考系无关。 为了避免回到绝对时空观, 我们需要假设同一事件在不同参考系中的每个坐标都可能是不同的, 两个事件的各个坐标之间的差值也可能不同。

现在回到测量速度的问题中。 在两个参考系中测量任何物体(视为质点)匀速运动的速度都可以简化成两个事件, 事件 1 是某时刻物体从某一点出发, 用火车参考系的坐标记为 $(x_1', t_1')$, 事件 2 是该物体做匀速运动一段时间后到达了空间中某个位置, 记为 $(x_2', t_2')$。 那么在火车中测量该物体的速度为
\begin{equation}\label{eq_Relat0_2}
v' = \frac{x_2' - x_1'}{t_2' - t_1'}~.
\end{equation}
同理, 在地面上观测这两个事件, 可以分别用 $(x_1, t_1)$ 和 $(x_2, t_2)$ 来描述两个事件, 那么测到的速度为
\begin{equation}\label{eq_Relat0_1}
v = \frac{x_2 - x_1}{t_2 - t_1}~.
\end{equation}
现在对如何测量速度已经有了严谨的定义, 那么光速不变的要求就是说, 当两个事件使得 $v' = c$ 时($c$ 表示真空中光速), 必然也有 $v = c$。 其实任何两个事件都可以, 和测不测光速无关。

要从光速不变的假设建立相对论时空观, 就是在带撇和不带撇的坐标之间建立一种转换关系, 使得无论两个事件的坐标怎么取都能满足上面的条件, 也就是无论光从什么位置发射, 向什么方向发射, 传播时间多长。 可以用数学证明对于任意两个不同的惯性系, 满足这个要求的变换是存在且唯一的, 它有一个大名鼎鼎的名字叫做\textbf{洛伦兹变换}。 虽然我们可以给出详细的推导\upref{SRLrtz}, 但这有点超出了这篇科普对数学的要求。 那么我们不妨在余下的篇幅对洛伦兹变换做一些半定量的分析。

洛伦兹变换其实就是中学数学的映射\upref{map},且可以安全地假设这个映射\textbf{是一对一的}, 不可能说一个参考系中两个不同的事件的坐标变换到另一个参考系中后具有相同的时空坐标(反之也不可能)。 这是因为我们通常对两个点的接触事件不会发生歧义, 例如当一个固定在火车参考系的物体和固定在地面的物体上的一个点发生了接触时(比如一个坐在火车上的乘客用把笔尖伸出窗外不动, 在火车匀速经过站台时, 笔尖和固定在地面的一根柱子发生了一瞬间的接触), 任何参考系的人都会同意笔尖被接触和柱子被接触发生在同一位置以及同一时刻, 可以认为是同一事件。

关于该映射的另一个假设是, \textbf{两个参考系之间的相对速度没有歧义: 它们等大, 方向相反且恒定不变}。 也就是火车上任意标记一点, 在地面参考系用\autoref{eq_Relat0_1} 测量出该点的速度都是一样的。 同理在地面任意取一点, 在火车参考系中用\autoref{eq_Relat0_2} 测速也会得到相同的速度大小, 但要取相反数(方向相反)。 这其实体现了第一个基本假设, 两个惯性系是等价的。

\addTODO{这么写又好像公式太多了}
假设 $x$ 轴和 $x'$ 轴的方向一样, 火车相对于地面延正方向运动, 那么洛伦兹变换可以记为
\begin{equation}
\leftgroup{
&x = \gamma (x' + vt') \\
&t = \gamma (t' + vx'/c^2)
}~.
\end{equation}
其中 $\gamma$ 是大于 1 的常数, 火车速度越快, $\gamma$ 越大。 解方程可得逆变换为
\begin{equation}
\leftgroup{
&x' = \gamma (x - vt) \\
&t' = \gamma (t - vx/c^2)
}~.
\end{equation}
现在来看火车上一个固定的点 $x' = x_0'$ 随着时间 $t'$ 变大选取一连串事件 $(x'_0, t'_1)$, $(x'_0, t'_2)$……, 如果令它们在地面系得坐标为 $(x_1, t_1)$, $(x_2, t_2)$……, 那么
\begin{equation}
\leftgroup{
&x_i = v t_i + \gamma x_0'\\
&t_i = \gamma t'_i + \gamma vx_0'/c^2
}~.
\end{equation}

可以看到这一串事件在地面参考系中以速度 $v$ 向右运动。 这是符合上面对相对速度的讨论的。 反之, 如果


它的逆变换为(可以通过解二元一次方程解出)



\addTODO{下面好像设计得有点失败, 似乎应该直接给出洛伦兹变换, 然后解释为什么可以让光速不变公式成立, 再根据公式解释一下尺缩短和钟变慢, 然后搞个互动演示。}

我们再来像\autoref{sub_Relat0_1} 一样, 把光相对于地面行走的距离划分为两部分, 第一部分代表火车的移动。 所以我们假设我们事先在火车上标记了光的出发点, 然后在地面参考系的 $t_2$ 时刻除了光的终点位置 $x_2$ 还记录下该点在地面系的坐标 $x_3$。 那么我们在地面参考系就可以定义第三个事件 $(x_3, t_3)$(其中 $t_3 = t_2$)。 于是光走过的路程就可以拆分成火车在这段时间内走的距离 $x_3 - x_1$, 以及从地面参考系看来光相对于火车多走的距离 $x_2 - x_3$。 所以要让光速不变, 我们想要有
\begin{equation}\label{eq_Relat0_3}
\frac{x_2' - x_1'}{t_2' - t_1'} = c = \frac{(x_3 - x_1) + (x_2 - x_3)}{t_2 - t_1}~.
\end{equation}
所以火车参考系中看来有光经过的车身长度是 $x_2' - x_1'$, 而在地面看来, 同样的一段长度是 $x_2 - x_3$。

现在我们\textbf{似乎}有不同的选择, 一个选择是仍然保持绝对的时间观, 规定两个分母不变, 而让地面的人测量到的被光经过的火车车身长度 $x_2 - x_3$ 相比于火车参考系的 $x_2'-x_1'$ 压缩一些, 使两边的分子也相同。 还记得在一个参考系中测量另一个参考系中的长度是怎么定义的吗? 要在本参考系中的同一时间! 这也是为什么我们定义事件 2, 3 在地面参考系具有相同的时间。 于是我们会得到一个很奇怪的现象, 火车上的同一截车厢, 地面测出来的长度比火车上的人自己测出来的长度要短, 这\textbf{似乎}就得到了著名的尺缩短效应。

我们\textbf{似乎}也可以试图仍然顽固地坚持绝对的空间观, 认为那段车身长度不变, 即 $x_2 - x_3 = x_2' - x_1'$ 。 而是试图让 $t_2 - t_1$ 比 $t_2' - t_1'$ 大一些。 此时如果只看事件 1 和 3, 它们在火车系的坐标分别为 $(x_1', t_1')$, $(x_1', t_3')$ (坐标不变, 时间增大), 在地面系的坐标分别为 $(x_1, t_1)$, $(x_3, t_2)$。 我们可以把\autoref{eq_Relat0_3} 两个分母的关系记为
\begin{equation}
t_2 - t_1 > t_2' - t_1' = (t_2' - t_3') + (t_3' - t_1')
\end{equation}
我们知道其中 $t_2 - t_1$ 和 $t_3' - t_1'$ 都大于零, 所以必须有 $t_2' \ne t_3'$。 这意味着什么? 我们再来单独看事件 3 和 2, 火车中的时空坐标分别为 $(x_1', t_3')$, $(x_2', t_2')$, 地面坐标分别为 $(x_3, t_2)$, $(x_2, t_2)$。 所以 $t_2' \ne t_3'$ 说明在地面系同一时刻不同位置发生的两件事, 在火车系中是不同时的。 那么 2 和 3 到在火车系中谁先呢? 由于 
