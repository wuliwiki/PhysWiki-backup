% 拓扑空间的运算
% keys 拓扑空间|商|积|不交并

\begin{issues}
% \issueOther{用 issues 环境列出文章存在的所有问题.}
\issueDraft
\issueTODO
\issueMissDepend
\issueAbstract
\issueNeedCite
\end{issues}

\subsection{概述}

暂略

\subsection{拓扑空间之间的运算}

\subsubsection{商}

详情见词条商拓扑\upref{Topo7}

\addTODO{将商拓扑词条迁移到此处}

\subsubsection{积}

详情见词条积拓扑\upref{Topo6}

\addTODO{将积拓扑词条迁移到此处}

\subsubsection{不交并}

详情见词条\autoref{Topo9_def3}~\upref{Topo9}

\addTODO{将积不交并词条迁移到此处}

\subsubsection{<待定>}

https://en.m.wikipedia.org/wiki/Join_(topology)

\addTODO{拓扑空间的join运算}

\subsubsection{锥化<待定>}

\begin{definition}{}
拓扑空间 $S$的锥化空间(记作 $C S$)定义为 $S$和一个单点空间的join.
\end{definition}

\addTODO{需要严格化一下“空间之间的运算”的定义,比如是否要定义锥化映射.}

\subsubsection{双重锥化<待定>}

\begin{definition}{}
拓扑空间 $S$的锥化空间(记作 $C S$)定义为 $S$和一个单点空间的join.
\end{definition}

\addTODO{需要严格化一下“空间之间的运算”的定义,比如是否要定义锥化映射.}

\subsection{带基点拓扑空间之间的运算}

什么是带基点拓扑空间见\autoref{Topo9_def2}~\upref{Topo9}

\subsubsection{一点并}

详情见词条积\autoref{Topo9_def1}~\upref{Topo9}

\addTODO{将一点并词条迁移到此处}

\subsubsection{压缩积}

详情见词条积\autoref{Topo9_def1}~\upref{Topo9}

\addTODO{将压缩积词条迁移到此处}



