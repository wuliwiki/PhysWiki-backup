% 惯性力:为什么月球不被太阳吸走
% license Usr
% type Tutor

为什么我们在分析月亮和卫星的运动时,可以不考虑太阳的引力? 事实上,太阳对月亮的引力比地球对月亮的引力要大!

你有兴趣的话,可以自行查找太阳、地球、月球的质量,以及他们的平均距离,然后用牛顿的万有引力公式计算:
\begin{equation}
F = \frac{Gm_1m_2}{r^2}~.
\end{equation}

如果你还记得高中的知识,也可以把地球中心作为坐标系的原点,假设太阳不存在,计算验证月球绕地球旋转的周期和轨道。 事实上你会发现这和真实观察到的情况几乎一样。

那么问题来了,为什么在计算月球绕地球旋转时不需要考虑太阳的引力?若硬要加上引力反而会得到 “月亮将被太阳吸走” 这样奇怪的结论。

在回答这个问题以前,我们先提出另一个问题。 这个问题是: 若把地球中心作为坐标系的原点,为什么地球没有被太阳吸走?难道换个参考系地球就不受太阳引力了吗?

这似乎有点循环论证的意思,我们知道运动是相对于观察者所在参考系的。 因为我们定义了地球中心是坐标原点,所以地球在这个坐标系中自然是不动的。 但为什么受了力的物体会没有加速度呢? 难道牛顿第二定律 $F = ma$ 在这里失效了? 如果太阳对地球的引力在这个参考系中对地球不起作用,你为什么会认为它对月球要起作用呢?

事实上许多人在牛顿定律时忽略了一个前提:\textbf{使用的参考系必须是惯性系}!

事实上牛顿第一定律的作用就是定义惯性系: 在某个参考系中,如果任何不受力(或者所有外力抵消)的物体都做匀速直线运动或静止不动,那么它就是一个惯性系。

那么
