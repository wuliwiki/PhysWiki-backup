% 玻尔兹曼分布(统计力学)
% 玻尔兹曼分布|等概率原理|经典统计

\pentry{麦克斯韦—玻尔兹曼分布\upref{MxwBzm},相空间\upref{PhSpace},理想气体单粒子能级密度\upref{IdED1},拉格朗日乘数法\upref{LagMul}}

根据等概率原理,对于平衡状态的孤立系统,每个可能的微观状态出现的概率是相等的。

根据统计力学中的量子力学假设,系统中单粒子态能级为分立的:$\epsilon_1,\epsilon_2,\cdots$。其中设第 $l$ 个能级的简并数为 $\omega_l$(意味着这个能级一共有 $\omega_l$ 种线性无关的态)。设第 $l$ 个能级上共有 $a_l$ 个粒子,序列 $\{a_l\}$ 构成粒子系统的一种分布。微观状态数最多的分布出现的概率最大,称为\textbf{最概然分布}。

注意等概率原理中只涉及“可能出现”的微观状态,也就是说要满足孤立系统的粒子数守恒、能量守恒条件:
\begin{equation}\label{eq_MBsta_1}
\begin{aligned}
\sum_l a_l=N~,\\
\sum_l \epsilon_l a_l=E~.
\end{aligned}
\end{equation}

我们下面要谈的是\textbf{理想气体}系统,为了方便计算尽可能快地得到有价值的结果,先忽略粒子的振动自有能和转动自由能,系统的能量完全来自粒子的平动动能。我们将通过等概率原理推出\textbf{玻尔兹曼分布},并且我们暂时\textbf{不涉及全同粒子假设}\footnote{如果我们加上全同粒子假设,则一切粒子可根据自旋分为两类粒子:玻色子和费米子,这又会带来两类不同的统计力学结果。},即粒子之间是可区分的。我们将得到同经典统计中一样的结果\upref{MxwBzm}。

\subsection{初步推导}
设能级 $\epsilon_l$ 的简并度为 $\omega_l$(可以想象该能级上有 $\omega_l$ 个房间,代表不同的单粒子微观状态)。如果给能级 $\omega_l$ 分配 $a_l$ 个粒子,那么由排列组合可得到 $N!/\Pi_l a_l!$ 种分配方法。随后要给每个能级上的粒子分配到确定的房间。对于玻尔兹曼分布,\textbf{不涉及全同粒子假设},我们认为粒子之间是可以分辨的。每种粒子都可以分配到 $\omega_l$ 种房间,于是共有 $\omega_l^{a_l}$ 种分配方式。总的微观状态数为
\begin{equation}
\Omega=\frac{N!}{\Pi_l a_l!}\prod_l\omega_l^{a_l}~.
\end{equation}

\begin{example}{}
\begin{figure}[ht]
\centering
\includegraphics[width=10cm]{./figures/f65015db35dba716.pdf}
\caption{给外地来的游客分配酒店房间} \label{fig_MBsta_1}
\end{figure}
我们先来思考一个更直观的例子:如何给外地来的游客分配酒店房间?假设一共有$N$个游客前来度假(心照不宣的假设:每个人都是不一样的、且可区分的);同时,酒店共有$l$层,第$l$层有$\omega_l$个房间,并且每个房间都可以容纳任意多的人。

我们先制定一个方案,订出各层的入住人数:第一层入住$a_1$人,第二层入住$a_2$人,...或写为 $\{a_1,a_2,...\}=\{a_l\}$。这个方案只订出了每层的人数,但没有订出具体是谁在哪间房间。

我们先从$N$位游客中选取$a_1$位游客住在第一层,有$C_N^{a_1}$种选法。于此同时,对于这$a_1$位游客,每位游客都可以随意入住本层的一间房间,由于本层有$\omega_1$间房可选,因此有${\omega_1}^{a_1}$种选法。现在,我们分配完了第一层,并产生了$$\Omega_1 = C_N^{a_1} {\omega_1}^{a_1}~$$种更具体的入住方式。

别急,我们得继续分配第二层:由于已有$a_1$位游客入住了第一层,因此我们只能从剩余的$(N-a_1)$位游客中选取$a_2$位游客住在第二层,有$C_{N-a_1}^{a_2}$种选法。同时,对于这$a_2$位游客,每一位游客都可以随意选取本层的一间房间入住,因此有${\omega_2}^{a_2}$种入住方式...
$$\Omega_2 = C_{N-a_1}^{a_2} {\omega_2}^{a_2}~.$$

以此类推,总的入住方式是
$$
\begin{aligned}
\Omega &= C_N^{a_1} {\omega_1}^{a_1} C_{N-a_1}^{a_2} {\omega_2}^{a_2} C_{N-a_1-a_2}^{a_3} {\omega_3}^{a_3}...\\
&=C_N^{a_1} C_{N-a_1}^{a_2} C_{N-a_1-a_2}^{a_3} ... {\omega_1}^{a_1}{\omega_2}^{a_2}  {\omega_3}^{a_3}...\\
& = C_N^{a_1} C_{N-a_1}^{a_2} C_{N-a_1-a_2}^{a_3} ... \Pi_l{\omega_l}^{a_l}~.\\
\end{aligned}
$$
运用组合数的定义展开前半部分。
$$
\begin{aligned}
& C_N^{a_1} C_{N-a_1}^{a_2} C_{N-a_1-a_2}^{a_3} ...\\
&= \frac{N!}{a_1!(N-a_1)!} 
\frac{(N-a_1)!}{a_2!(N-a_1-a_2)!}
\frac{(N-a_1-a_2)!}{a_3!(N-a_1-a_2-a_3)!}
...\\
&=\frac{N!}{a_1!a_2!a_3!...} \\
&=\frac{N!}{\Pi_l a_l!} ~.\\
\end{aligned}
$$
代回原式,得
$$
\Omega = \frac{N!}{\Pi_l a_l!}  \Pi_l{\omega_l}^{a_l}~.
$$
这就是给定每层人数$\{a_l\}$的情况下,具体的入住方式的个数。我们惊奇地发现,这个结论形式上与玻尔兹曼分布中的微观态个数完全相同!

在这个比喻中,
\begin{itemize}
\item 游客总数$N$代表粒子总数,
\item 酒店的楼层代表能级,
\item 每层的房间数$\omega_l$代表每个能级的简并态个数,
\item 每层的入住人数$a_l$代表该能级上的粒子个数,
\item 制定每层人数的方案$\{a_l\}$代表一种分布,
\item 每一种具体的入住方式代表这种分布下的一种微观态,
\item 入住方式的总数$\Omega$就是这种分布下的微观态总个数。
\end{itemize}
\end{example}

取对数可以得到
\begin{equation}
\ln \Omega=\ln N!-\sum_{l}a_l!+\sum_l a_l\ln \omega_l~.
\end{equation}

假设所有的 $a_l$ 都很大\footnote{我们这里先做了一个不正确的假设,因为对于真实气体模型,$a_l$ 通常并不大,不可以这样做近似。但不妨让我们先计算下去,令人惊喜的是最终结果是正确的,是与实验符合的。},根据近似等式 $\ln n! = n(\ln n-1)$,可以化简得到

\begin{equation}\label{eq_MBsta_6}
\ln \Omega=N\ln N-\sum_l a_l\ln a_l+\sum_l a_l\ln \omega_l~.
\end{equation}

在约束条件\autoref{eq_MBsta_1} 下,要求 $\ln \Omega$ 极大的分布 $\{a_l\}$,即 $\delta \ln \Omega =0$。由于在约束条件下 $\delta a_l$ 并非独立,需要利用拉格朗日乘数法,引入参量 $\alpha,\beta$,
\begin{equation}\label{eq_MBsta_7}
\delta \ln \Omega -\alpha \delta N-\beta \delta E=-\sum_l [\ln a_l-\ln \omega_l+\alpha +\beta\epsilon_l]\delta a_l=0~.
\end{equation}

所以
\begin{equation}
\ln \frac{a_l}{\omega_l}+\alpha+\beta\epsilon_l = 0~.
\end{equation}
由此求得最概然分布:
\begin{equation}\label{eq_MBsta_5}
a_l=\omega_l e^{-\alpha -\beta \epsilon_l}~,
\end{equation}
这种分布出现的概率最大。下面我们来确定 $\alpha,\beta$ 的值。现在,我们令这种分布就是\textbf{玻尔兹曼分布},并大胆地假设其他分布出现的概率为 $0$\footnote{这实际上是不正确的假设,根据等概率原理,其他可能的分布的概率总 $>0$,粒子状态分布总会有一定涨落。但当我们考虑 $N$ 很大的系统,其他分布出现的概率将远小于最概然分布出现的概率。所以这个假设是合理的。}。$\omega_l$ 是简并度,那么对于单个量子态,粒子数为 $e^{-\alpha-\beta \epsilon_l}$。现在我们列举一切能级中的一切量子态,它们的能级依次是 $\{\epsilon_s\}$。我们有
\begin{equation}\label{eq_MBsta_2}
\begin{aligned}
N=\sum_s e^{-\alpha-\beta \epsilon_s}~,\\
E=\sum_s \epsilon_s e^{-\alpha-\beta \epsilon_s}~.
\end{aligned}
\end{equation}
\subsection{玻尔兹曼分布}
我们定义\textbf{配分函数} $Z(\beta)=\sum_s e^{-\beta \epsilon_s}$,求和的时候 $s$ 覆盖了每一个能级的一切量子态。由 \autoref{eq_MBsta_2},可以得到 $E,N$ 和配分函数 $Z$ 的关系:
\begin{equation}\label{eq_MBsta_3}
\begin{aligned}
N=e^{-\alpha}Z~,\\
E=-e^{-\alpha}\frac{\partial Z}{\partial \beta}~.
\end{aligned}
\end{equation}

化简得

\begin{equation}\label{eq_MBsta_4}
\begin{aligned}
E=-N\frac{\partial \ln Z}{\partial \beta}~.
\end{aligned}
\end{equation}

根据量子力学,体积为 $V$ 的容器中,单粒子的能级为

\begin{equation}
\varepsilon = \frac{\hbar ^2}{2m} \qty[\qty(\frac{\pi n_x}{L_x})^2 + \qty(\frac{\pi n_y}{L_y})^2 + \qty(\frac{\pi n_z}{L_z})^2] = \frac{\hbar ^2}{2m} (k_x^2 + k_y^2 + k_z^2)~.
\end{equation}

$n_x,n_y,n_z$ 可以取一切正整数。计算配分函数:
\begin{equation}
\begin{aligned}
Z&\approx \int_{0}^\infty\int_{0}^\infty\int_{0}^\infty\dd n_x\dd n_y\dd n_z  e^{-\beta \frac{\hbar^2}{2m}[(\pi n_x/L_x)^2+(\pi n_y/L_y)^2+(\pi n_z/L_z)^2]}\\
&=\frac{1}{8}\int_{-\infty}^{\infty}\dd n_x e^{-\beta \frac{\hbar^2}{2m}\frac{\pi^2}{L_x^2}n_x^2}\int_{-\infty}^{\infty}\dd n_y e^{-\beta \frac{\hbar^2}{2m}\frac{\pi^2}{L_y^2}n_y^2}\int_{-\infty}^{\infty}\dd n_z e^{-\beta \frac{\hbar^2}{2m}\frac{\pi^2}{L_z^2}n_z^2}
\\
&=\frac{1}{8}\sqrt{\frac{2m}{\pi\beta}}\frac{L_x}{\hbar}\cdot \sqrt{\frac{2m}{\pi\beta}}\frac{L_y}{\hbar} \cdot \sqrt{\frac{2m}{\pi\beta}}\frac{L_z}{\hbar}
\\
&=\left(\frac{m}{2\pi\beta\hbar^2}\right)^{3/2}V~.
\end{aligned}
\end{equation}
代入\autoref{eq_MBsta_4},可得
\begin{equation}
E=\frac{3}{2}N\cdot \frac{1}{\beta}~.
\end{equation}
对于理想气体,$E=\frac{3}{2}N k T$\upref{IdgEng},因此我们求得
\begin{equation}
\beta=\frac{1}{kT}~.
\end{equation}

代入\autoref{eq_MBsta_3} 可得到 $\alpha$ 的表达式:
\begin{equation}\label{eq_MBsta_8}
\alpha=\ln\left[\frac{V(m k T/2\pi \hbar^2)^{3/2}}{N}\right]~,
\end{equation}

因此我们得到了每一个量子态上的粒子分布:

\begin{equation}
a_s=\frac{N}{V}\left(\frac{2\pi \hbar^2}{m k T}\right)^{3/2} e^{-\frac{1}{k T}\epsilon_s}~.
\end{equation}

对于单粒子相空间\upref{PhSpace}中的一个体积元 $\dd \Omega_1=\dd x\dd y\dd z\dd p_x\dd p_y\dd p_z$ 中,量子态的个数为 $\dd V/h^3=\dd V/(2\pi \hbar)^3$,所以有速度分布律:
\begin{equation}
\Delta N=\frac{N}{V}\left(\frac{1}{2\pi m k T}\right)^{3/2}e^{-\frac{1}{2mk T}(p_x^2+p_y^2+p_z^2)}\Delta x\Delta y\Delta z\Delta p_x\Delta p_y\Delta p_z~.
\end{equation}

\subsection{推广到玻色分布和费米分布}

现在我们加上全同粒子假设,要对总状态数 $\Omega$ 除以 $N!$,表示任意交换两粒子的位置,得到的状态都是同一个。微观粒子可以根据自旋分为\textbf{玻色子}和\textbf{费米子},后者满足\textbf{泡利不相容原理}——一个量子态只能容纳一个粒子。与之不同的是,一个量子态能容纳多个玻色子。

假设能级 $\epsilon_l$ 上分配 $a_l$ 个粒子,那么该能级上的分配方式不再是 $\omega_l^{a_l}$(注意我们有全同粒子假设),而是 $(\omega_l+a_l-1)!/(a_l!(\omega_l-1)!)$。总的微观状态数为
\begin{equation}
\Omega_{B.E.}=\prod_l \frac{(\omega_l+a_l-1)!}{a_l!(\omega_l-1)!}~.
\end{equation}

如果在玻色系统中,任意能级 $\epsilon_l$ 上的粒子数均远小于该能级的量子态数 $\omega_l$(即简并度),那么上式就可以近似为 $\Pi_l \omega_l^{a_l}/a_l!=\Omega_{M.B.}/N!$($\Omega_{M.B.}$ 就是前面计算的玻尔兹曼分布的微观状态数)。

我们同样可以用拉格朗日乘子法计算最概然分布下的 $a_l$。推导如下:
\begin{equation}
\begin{aligned}
\ln \Omega&=\sum_l[\ln (\omega+a_l-1)!-\ln a_l!-\ln (\omega_l-1)!]\\
&=\sum_l[(\omega_l+a_l)\ln(\omega_l+a_l)-a_l\ln a_l-\omega_l\ln \omega_l~,
\end{aligned}
\end{equation}
在极值情况下一阶微分为 $0$。由于有粒子数和总能量的约束条件,需引入拉格朗日乘子 $\alpha,\beta$。
\begin{equation}
\delta \ln \Omega-\alpha\delta N-\beta\delta E=\sum_l[\ln (\omega_l+a_l)-\ln a_l-\alpha-\beta\epsilon_l]\delta a_l~.
\end{equation}
每一个 $\delta a_l$ 的系数都为 $0$,因此
\begin{equation}
\ln(\omega_l+a_l)-\ln a_l-\alpha-\beta \epsilon_l=0~.
\end{equation}
解得\textbf{玻色-爱因斯坦分布}:
\begin{equation}\label{eq_MBsta_9}
a_l=\frac{\omega_l}{e^{\alpha+\beta\epsilon_l}-1}~.
\end{equation}

同理,对于费米分布,由于泡利不相容原理,总状态数为
\begin{equation}
\Omega=\prod \frac{\omega_l!}{(\omega-a_l)!a_l!}~.
\end{equation}
经过拉格朗日乘子法得到\textbf{费米-迪拉克分布}:
\begin{equation}\label{eq_MBsta_10}
a_l=\frac{\omega_l}{e^{\alpha+\beta\epsilon_l}+1}~.
\end{equation}

以上的推导用了诸如 $a_l\gg 1,\omega_l\gg 1$ 等条件,而实际情况下这些条件常常不能满足,所以推导过程存在严重问题。用\textbf{巨正则系综理论}可以完美地推导出粒子在其个体能级上的分布。
