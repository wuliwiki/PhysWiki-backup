% 单一算子生成的子代数
% 子代数|极小多项式|幂零算子

\pentry{线性算子代数\upref{LiOper}}
已经知道,定义在域 $\mathbb{F}$ 上的向量空间 $K$,如果它同时是一个环,它就是一个代数\autoref{LiOper_sub3}~\upref{LiOper}.如果这个向量空间 $K$ 的子空间 $L$ 对于 $K$ 作为环的乘法封闭,那么 $L$ 就称为代数 $K$ 的\textbf{子代数}.

现在要研究包含算子 $\mathcal{A}$ 的最小子代数(且含环的单位元 $\mathcal{E}$),这个子代数记作 $\mathbb{F}[\mathcal A]$ (之所以这样记,是因为这个子代数是域 $\mathbb{F}$ 上的多项式环\autoref{Ring_ex1}~\upref{Ring}).

要找包含算子 $\mathcal{A}$ 的最小子代数 $\mathbb{F}[\mathcal A]$ ,可以这样思考:首先考虑这个子代数是个向量空间并且含有 $\mathcal{E}$,那么由算子 $\mathcal{A,E,O}$ ($\mathcal{O}$ 为零算子\autoref{LiOper_ex1}~\upref{LiOper})进行向量空间的加法和数乘得到的元素形为 $a_0\mathcal{E},a_1\mathcal{A}\;\forall a_0,a_1\in\mathbb{F}$ .现在,这个子代数至少包含 $a_0\mathcal{E},a_1\mathcal{A}\;\forall a_0,a_1\in\mathbb{F}$ .考虑子代数也是个环,那么由算子 $a_0\mathcal{E},a_1\mathcal{A}\;\forall a_0,a_1\in\mathbb{F}$ 进行环的乘法得到的元素形为 $a_0\mathcal{E},a_1\mathcal{A},a_2\mathcal{A}^2\;\forall a_0,a_1,a_2\in\mathbb{F}$,所以现在这个子代数至少包含算子 $a_0\mathcal{E},a_1\mathcal{A},a_2\mathcal{A}\;\forall a_0,a_1,a_2\in\mathbb{F}$;再考虑它是个向量空间及环,如此重复可得到这个子代数至少包含所有下面形式的元素(注意 $\mathcal{A}^0=\mathcal{E}$)
\begin{equation}\label{SiOAlg_eq1}
f(\mathcal{A})=\sum_{i=0}^ma_i\mathcal{A^i},\;m\in\mathbb{N}
\end{equation}
易验证,所有形如\autoref{SiOAlg_eq1} 的元素构成一个子代数,这个子代数便是要找的 $\mathbb{F}[\mathcal A]$ ,称为由\textbf{算子 $\mathcal{A}$ 生成的子代数}.
\subsection{代数 $\mathbb{F}[\mathcal{A}]$ 的交换性}
代数 $\mathbb{F}[\mathcal A]$ 是交换的,因为 $\mathcal A^{k}\cdot \mathcal{A}^l=\mathcal{A}^{k+l}=\mathcal A^{l}\cdot \mathcal{A}^k$(利用 $\mathcal{A}$ 本身的交换性和结合性易证该性质),易验证 $f(\mathcal{A})g(\mathcal{A})=g(\mathcal{A})f(\mathcal{A})$.

与线性算子作用在向量上的方式\upref{LiOper}一样, $f(\mathcal{A})$ 通过以下方式作用在 $\bvec x\in V$ 上:
\begin{equation}
f(\mathcal{A})\bvec x=a_0\bvec x+\sum_{i=1}^m a_i\mathcal{A}^i\bvec x
\end{equation}
\subsection{极小多项式}
\begin{definition}{极小多项式}
称多项式 $f(t)$ \textbf{零化} 算子 $\mathcal{A}$,如果 $f(\mathcal{A})=\mathcal O$.次数(\autoref{OnePol_def1}~\upref{OnePol})最低且首项系数为1的多项式称为算子 $\mathcal{A}$ 的\textbf{极小多项式}.
\end{definition}
\begin{theorem}{}
所有线性算子 $\mathcal{A}$ 都有极小多项式 $\mu_\mathcal{A}(t)$,其次数与代数 $\mathbb{F}[\mathcal{A}]$ 的维数(\autoref{LiOper_sub3}~\upref{LiOper})一致.算子 $\mathcal{A}$ 可逆,当且仅当 $\mu_\mathcal{A}(t)$ 的常数项不为0.
\end{theorem}
\textbf{证明:}
(1)\textbf{定理前一部分的证明}

设
\begin{equation}
\mu_\mathcal{A}(t)=\sum_{i=0}^{m-1}\mu_i\mathcal t^i+t^m
\end{equation}
那么 $\mathcal{A}^0,\mathcal{A},\cdots \mathcal{A}^{m-1}$ 必然线性无关.因为,若 $\sum_\limits{i=0}^{m-1}\lambda_i\mathcal{A}^i=\mathcal{O}$ ,就意味着 $\sum_\limits{i=0}^{m-1}\lambda_i\mathcal{A}^i=\mathcal{O}$ 零化 $\mathcal A$,而它的次数小于 $m$,与 $\mu_\mathcal{A}(t)$ 是极小多项式矛盾.反之,若 $\mathcal{A}^0,\mathcal{A},\cdots \mathcal{A}^{m-1}$ 线性无关,而 $\mathcal{A}^m$ 可由它们线性表示,则极小多项式次数为 $m$,因为
\begin{equation}\label{SiOAlg_eq2}
\mathcal A^m=\sum_{i=0}^{m-1}\lambda_i\mathcal A^i
\end{equation}
就意味 $t^m-\sum_\limits{i=0}^{m-1}\lambda_i\mathcal t^i$是极小多项式.

由于 $\mathbb{F}[\mathcal{A}]\subset\mathcal{L}(V)$,而 $\mathrm{dim}(V)=n^2$ (因为 $\mathcal{L}(V)$ 上的线性算子和 $n$ 阶矩阵相对应,而 $n$ 阶矩阵构成 $n^2$ 维的向量空间,它以第 $i$ 行第 $j$ 列元素为 1,其余为0的矩阵 $E_{ij}$ 为基底向量),所以 $m\leq n^2$.

由于每一线性算子 $\mathcal A$ 都有与之对应的子代数 $\mathbb{F}[\mathcal A]$ ,其至少是一维的(这一维对应基底 $\mathcal{E=A}^0$),而上述表明 $\mathbb{F}[\mathcal A]$ 的维度 $m$ ( $1\leq m\leq n$ )一定存在,这对应 $m$ 个线性无关向量 $\mathcal{A}^0,\mathcal{A},\cdots \mathcal{A}^{m-1}$ ,那么\autoref{SiOAlg_eq2} 的 $\mathcal A^m$ 和它们一起便构成极小多项式 $t^m-\sum_\limits{i=0}^{m-1}\lambda_i\mathcal t^i$.即每一线性算子都有极小多项式,且次数就是代数 $\mathbb{F}[\mathcal{A}]$ 的维度.

(2)\textbf{定理后一部分证明}

若 $\mu_0= 0$,那么
\begin{equation}
\mathcal O=\mu_\mathcal{A}(\mathcal A)=\mathcal A(\sum_{i=1}^{m-1}\mu_i\mathcal A^i+\mathcal{A})
\end{equation}
即 $\mathcal{A}$ 有零因子\autoref{Domain_def1}~\upref{Domain} $\sum_\limits{i=1}^{m-1}\mu_i\mathcal A^i+\mathcal{A}\neq \mathcal O$($\mu_\mathcal{A}(t)$ 的极小性),而环中有零因子的元素无逆元(因为若 $a$ 可逆且 $ab=0,b\neq 0\Rightarrow 0\neq b=a^{-1}0=0$).反过来,若 $\mu_0\neq0$,那么
\begin{equation}
-\mathcal{A}\qty(\sum_{i=1}^{m-1}\frac{\mu_i}{\mu_0}\mathcal A^{i-1}+\frac{1}{\mu_0}\mathcal A^{m-1})=\mathcal{E}
\end{equation}
显然 $\mathcal{A}$ 可逆.

\textbf{证毕!}
\begin{theorem}{}
若 $f(t)$ 零化算子 $\mathcal A$ ,则 $\mu_{A}(t)|f(t)$\autoref{ExDiv_def1}~\upref{ExDiv}.
\end{theorem}.
\textbf{证明:}由\autoref{DivAlg_the1}~\upref{DivAlg},可设 $f(t)=q(t)\mu_\mathcal{A}(t)+r(t),\mathrm{deg}\;r(t)<\mathrm{deg}\;\mu_{\mathcal{A}}(t)}$ ,那么
\begin{equation}
\mathcal O=f(\mathcal A)=q(\mathcal A)\mathcal O+r(\mathcal A)=r(\mathcal A)
\end{equation}
这与 $\mathrm{deg}\;r(t)<\mathrm{deg}\;\mu_{\mathcal A}(t)}$ 矛盾.即 $r(t)=0$.

\textbf{证毕!}