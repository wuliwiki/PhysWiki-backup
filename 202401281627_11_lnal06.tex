% 乘积空间与商空间
% license Xiao
% type Tutor


\begin{issues}
\issueTODO 需要合并商空间,增加补空间词条
\end{issues}


和群有直积类似,线性空间也能在集合意义上求并,笛卡尔积就是积空间的向量。为了求积后依然是线性空间,我们需要规定向量的数乘和加法。

\begin{definition}{积空间}
给定域$\mathbb F $上的线性空间$U$与$V$,定义$U\times V=\{(\bvec u,\bvec v)|\bvec u\in U,\bvec v\in V\}$上的数乘和加法运算为:
\begin{equation}
\left\{\begin{aligned}
a(\bvec u,\bvec v)&=(a\bvec u,a\bvec v),\quad \forall a\in \mathbb F\\
(\bvec u_1,\bvec v_1)+(\bvec u_2,\bvec v_2)&=(\bvec u_1+\bvec u_2,\bvec v_1+\bvec v_2)
\end{aligned}\right.~
\end{equation}
\end{definition}
根据该定义,我们容易验证积空间在数乘和加法下封闭。
若令$\{\bvec x_i\}^r_{i=1}$和$\{\bvec y_i\}^s{i=1}$分别为$U$与$V$的基,我们也容易验证积空间的基为$\{\bvec x_i,\bvec 0\}^r_{i=1}\cup \{\bvec 0,\bvec y_i\}^r_{i=1}$,这实际上是两个空间的“直和”。

从群的角度上看,线性空间是一个加法群,则其任意一个子空间都是子群。由于加法群的任意子群都是正规子群,因此线性空间可以对任意一个子空间求商,商群上的运算为\textbf{向量加法}。
\begin{definition}{商空间}
给定域$F$上的线性空间$V$及其子空间$V_0$,则对任意$\bvec v\in V$,定义左陪集$\bvec v+V_0=\{\bvec v+\bvec v_0|\bvec v_0\in V_0\}$。

数乘定义为:$\forall a\in \mathbb F,a(\bvec v+V_0)=a\bvec v+V_0$
\end{definition}
可以证明,区别于群意义上的商群,配备了数乘定义的商空间是线性空间。

利用商空间的概念也能证明线性空间的任意子空间都有补空间。
\begin{theorem}{}
给定域 $\mathbb F$ 上的线性空间 $V$,设 $V_1$ 是其子空间,则 $V_1$必有补空间。
\end{theorem}
从$V_1$的左陪集${S_{\alpha}}$里各选一个元素$\bvec v_a$,$\opn{Span} \{\bvec v_{\alpha}\}$构成商空间$V/V_1$的一组基\footnote{是线性相关的向量也无碍,因为是“张成”。在商空间里,$V_1$的代表元素为$0$},张成的也是$V_1$的补空间。由等价类划分可知,$\opn{Span} \{\bvec v_{\alpha}\}$与$V_1$没有交集,因此我们只需要证明所有元素都可以表示为$b_{\alpha}\bvec v_{\alpha}+V_1$即可。

设$\bvec v$为任意元素,则有$\bvec v+V_1=b_{\alpha} S_{\alpha}$,由于$\bvec v_{\alpha}\in S_{\alpha}$,所以$b_{\alpha} \bvec v_{\alpha}\in\bvec v+V_1$。也就是说,总可以找到一个元素$\bvec v_1\in V_1$,使得$\bvec v=b\bvec v_{\alpha}+\bvec v_1$,证毕。

证明过程同时明示着商空间维数和子空间的关系。由于商空间的基可以张成被商去子空间的补空间,因而有
\begin{theorem}{}
给定域$\mathbb F$。设$V_1$为$V$的子空间,我们有
\begin{equation}
\opn {dim}(V/V_1)=\opn V-\opn V_1~.
\end{equation}
\end{theorem}
如果$V,V_1$都是无穷维,商去的结果依然是有限维。这就好比“无限集至少有一个可数集”。