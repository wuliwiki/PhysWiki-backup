% 三维直角坐标系中的亥姆霍兹方程
% 直角坐标系|亥姆霍兹方程
\pentry{一维齐次亥姆霍兹方程\upref{HmhzEq}}
三维直角坐标系中的亥姆霍兹方程为
\begin{equation}\label{RHM_eq2}
\laplacian u+k^2u=0
\end{equation}
这里,$k$为常实数.
\subsection{通解}
使用分离变量法进行求解,令
\begin{equation}\label{RHM_eq1}
u(x,y,z)=X(x)Y(y)Z(z)\quad
k^2=k_x^2+k_y^2+k_z^2
\end{equation}
于是方程分解为3个独立的一维齐次亥姆霍兹方程\upref{HmhzEq}
\begin{equation}
\begin{aligned}
\dv[2]{X}{x}+k_x^2X=0\\
\dv[2]{Y}{y}+k_y^2Y=0\\
\dv[2]{X}{x}+k_x^2Z=0
\end{aligned}
\end{equation}
由\autoref{HmhzEq_eq1}~\upref{HmhzEq},其实数解分别为
\begin{equation}
\begin{aligned}
&X(x)=C_1\cos k_xx+C_1'\sin k_xx\\
&Y(y)=C_2\cos k_yy+C_2'\sin k_yy\\
&Z(z)=C_3\cos k_zz+C_3'\sin k_zz
\end{aligned}
\end{equation}
这里,$C_i,C_i'$为常实数.由\autoref{RHM_eq1} ,得\autoref{RHM_eq2} 通解
\begin{equation}
u(x,y,z)=(C_1\cos k_xx+C_1'\sin k_xx)(C_2\cos k_yy+C_2'\sin k_yy)(C_3\cos k_zz+C_3'\sin k_zz)
\end{equation}
\begin{example}{矩形波导中的电磁波}
在高频电力系统中,为解决电磁波向外辐射的损耗以及与环境的干扰问题,常常采用波导进行电磁波的传输,波导是一根空心金属管,截面通常为矩形或圆形.如图是一矩形波导,其长宽为$a,b$,以长边为 $x$方向,短边为 $y$ 方向;$z$轴沿传播方向.则波导内电磁波满足亥姆霍兹方程\autoref{RHM_eq2} ,此时,应以 $\bvec E,\bvec H$ 代替 $u$,我们以计算 $\bvec E$ 为例.

由刚刚以证明的亥姆霍兹方程的通解
\begin{equation}
\begin{aligned}
&E_x=C_1\cos k_xx+C_1'\sin k_xx\\
&E_y=C_2\cos k_yy+C_2'\sin k_yy\\
&E_z=C_3\cos k_zz+C_3'\sin k_zz
\end{aligned}
\end{equation}
