% 字符串哈希
% 字符串哈希|哈希|数据结构|C++

字符串哈希和普通的哈希算法类似,字符串哈希是把一个很长的字符串变成一个整数,这样的好处是:如果想比较两个很长的字符串是否相等时,普通算法是遍历一遍整个字符串,如果其中一个字符串的字符和另一个字符串的字符不等,则两个字符串不一样.时间复杂度为 $O(N)$.而用字符串哈希的话可以直接比较两个字符串的哈希值是否相同,时间复杂度为 $O(1)$.下面介绍一种哈希方式可以把任意一个字符串变成一个非负整数,并且哈希冲突的概率几乎为 $0$.

哈希函数通常设计为:取一固定值 $P$,把字符串看作 $P$ 进制数,并分配一个大于 $0$ 的数值,代表每种字符.再取一固定值 $M$,求出 $P$ 进制数转换十进制的数对 $M$ 的余数,作为这个字符串的哈希值.一般来说,$p$ 取 $131$ 或 $13331$ 时出现冲突的概率几乎为 $0$,通常 $M$ 取 $2^{64}$,可以直接用 \verb|unsigned long long| 这个变量类型存储哈希值,溢出时就相当于对 $2^{64}$ 取模.

举个例子:
先把一个字符串 $\text{abcfea}$ 变成 $p$ 进制数,将字符 $a \sim z$ 映射成 $1 \sim 26$,所以原来的字符串就变成了:$(123651)_p$,转化为十进制就为:\begin{equation}
1 \times p^5 + 2 \times p^4 + 3 \times p^3 + 6 \times p^2 + 5 \times p^1 + 1 \times p^0
\end{equation}

相加得出答案:$39175337388$ 作为这个字符串的哈希值.

一般在做哈希的时候,存储哈希的数组会存字符串的前缀哈希值,比如 \verb|h| 数组存储字符串的哈希值,\verb|h[1] = a| 的哈希值,\verb|h[2] = ab| 的哈希值,这样以此类推.那怎么算前缀哈希值呢?比如我们已经知道了 \verb|hash(S)| 的哈希值,那么在字符串 $S$ 后面添加一个字符串 $T$ 构成的新字符串的哈希值为: \verb|hash(S+C) = hash(S) * P + value[C]|.乘 $P$ 就相当于左移一位,加 \verb|value[C]| 就是字符串 $C$ 所分配的数值.

举个例子:比如我们知道 \verb|hash(abc)|,想算 \verb|hash(abcd)| 的话就是:\verb|hash(abc) * P + value[d]|.

\verb|hash(abc) = 1677554|,乘 $P = 131$,就是在 $P$ 进制下左移一位,在加上 \verb|value[d]|.