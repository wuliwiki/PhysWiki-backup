% 线性相关、线性表出

\pentry{线性相关性\upref{linDpe}}

\begin{figure}[ht]
\centering
\includegraphics[width=6cm]{./figures/LnDpd2_1.png}
\caption{v1,v2,v3线性相关,而v1,v2,v4线性无关} \label{LnDpd2_fig1}
\end{figure}

\begin{theorem}{线性相关}
对于一组向量$\alpha_1, \alpha_2,\alpha_3,...$,若存在不全为0的$k_1,k_2,...$,使$k_1 \bvec \alpha_1+k_2 \bvec \alpha_2+k_3 \bvec \alpha_3+...=0$,即称$\alpha_1, \alpha_2,\alpha_3,...$线性相关.

若只当$k_1=k_2=...=0$时,$k_1 \bvec \alpha_1+k_2 \bvec \alpha_2+k_3 \bvec \alpha_3+...=0$,则$\alpha_1, \alpha_2,\alpha_3,...$线性无关.
\end{theorem}

\begin{theorem}{线性表出、线性组合}
对于一组向量$\alpha_1, \alpha_2,\alpha_3,...$与一个向量$\beta$,若存在$k_1,k_2,...$,使$k_1 \bvec \alpha_1+k_2 \bvec \alpha_2+k_3 \bvec \alpha_3+...=\beta$,则称$\beta$是$\alpha_1, \alpha_2,\alpha_3,...$的线性组合.
\end{theorem}

\begin{corollary}{}
若$\beta$是$\alpha_1, \alpha_2,\alpha_3,...$的线性组合,则$\beta, \alpha_1, \alpha_2,\alpha_3,...$必线性相关.
\end{corollary}

\subsection{线性相关、线性表出与线性方程组}
\pentry{分块矩阵\upref{BlkMat},线性方程组\upref{LinEqu}}
\subsubsection{线性相关}
根据分块矩阵\upref{BlkMat},$k_1 \bvec \alpha_1+k_2 \bvec \alpha_2+k_3 \bvec \alpha_3+...=0$可记为
$$
\begin{bmatrix}
\bvec \alpha_1& \bvec \alpha_2& \bvec \alpha_3&...
\end{bmatrix}
\begin{bmatrix}
k_{1}\\
k_{2}\\
k_{3}\\
...\\
\end{bmatrix}
=
0
\Leftrightarrow 
\mat A \bvec k = 0
$$
这将线性相关问题化为线性方程组问题:若$\mat A \bvec k = 0$有非零解,则线性相关,否则线性无关.
\subsubsection{线性表出}
同理,
$$
\begin{bmatrix}
\bvec \alpha_1& \bvec \alpha_2& \bvec \alpha_3&...
\end{bmatrix}
\begin{bmatrix}
k_{1}\\
k_{2}\\
k_{3}\\
...\\
\end{bmatrix}
=
\bvec b
\Leftrightarrow 
\mat A \bvec k = \bvec b
$$
这将线性组合问题化为线性方程组问题:若$\mat A \bvec k = \bvec b$有解,则$\beta$是$\alpha_1, \alpha_2,\alpha_3,...$的线性组合.
