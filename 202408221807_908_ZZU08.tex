% 郑州大学 2008 年 考研 量子力学
% license Usr
% type Note

\textbf{声明}:“该内容来源于网络公开资料,不保证真实性,如有侵权请联系管理员”

\subsubsection{1}
假定自旋角动量算符 $\hat{J}_1^2$ 和 $\hat{J}_2^2$ 的本征值分别为 $6\hbar^2$ 和 $\frac{15}{4}\hbar^2$,如果 $\hat{J} = \hat{J}_1 + \hat{J}_2$,则 $\hat{J}^2$ 可能的本征值为()

$\text{A.} \quad \frac{1}{4}\hbar^2, \quad \frac{5}{4}\hbar^2, \quad \frac{15}{4}\hbar^2, \quad \frac{30}{4}\hbar^2\qquad \text{B.}\frac{3}{4}\hbar^2, \quad \frac{15}{4}\hbar^2, \quad \frac{35}{4}\hbar^2, \quad \frac{63}{4}\hbar^2$

$\text{C.} \quad \frac{7}{4}\hbar^2, \quad \frac{14}{4}\hbar^2, \quad \frac{21}{4}\hbar^2, \quad \frac{28}{4}\hbar^2\qquad\text{D.} \quad \frac{3}{2}\hbar^2, \quad \frac{15}{2}\hbar^2, \quad \frac{35}{2}\hbar^2, \quad \frac{63}{2}\hbar^2$

$\text{E.} \quad \frac{7}{2}\hbar^2, \quad \frac{14}{2}\hbar^2, \quad \frac{21}{2}\hbar^2, \quad \frac{28}{2}\hbar^2$
\subsubsection{2}
如果 $\hat{A}$ 和 $\hat{C}$ 是厄米算符,并且 $[\hat{A}, \hat{C}] \neq 0$,则下列是厄米算符的是()\\
$A. \hat{A}\hat{C}\qquad B. \hat{C}\hat{A}\qquad C. \hat{A}\hat{C} - \hat{C}\hat{A}\qquad D. i[\hat{A}, \hat{C}]\qquad E. 2[\hat{A}, \hat{C}]$
\subsubsection{3}
某复原子问题中,设原子核带正电荷 $Ze$,$a_0$ 为原子的玻尔半径,处于基态的电子,其出现几率最大的径向坐标位置为()\\
$A. a_0\qquad B. \frac{a_0}{Z}\qquad C. \frac{a_0}{2Z}\qquad D. \frac{a_0}{3Z}\qquad E. \frac{a_0}{4Z}$
