% 麦克斯韦方程组(综述)
% license CCBYSA3
% type Wiki

本文根据 CC-BY-SA 协议转载翻译自维基百科\href{https://en.wikipedia.org/wiki/Maxwell\%27s_equations}{相关文章}。

\textbf{麦克斯韦方程组},或称\textbf{麦克斯韦–赫维赛德方程组},是一组耦合偏微分方程,与洛伦兹力定律一起构成了经典电磁学、经典光学、电路和磁路的基础。这些方程为电学、光学和无线电技术(如发电、电动机、无线通信、透镜、雷达等)提供了数学模型。它们描述了电场和磁场如何由电荷、电流及场的变化产生。[注1] 这些方程以物理学家和数学家詹姆斯·克拉克·麦克斯韦的名字命名,他在1861年和1862年首次发表了包含洛伦兹力定律的早期方程形式。麦克斯韦最早使用这些方程提出光是一种电磁现象。方程的现代形式及其最常见的表述归功于奥利弗·赫维赛德。[1]

麦克斯韦方程组可以组合起来,展示电磁场的波动(波)如何以恒定速度 \( c \)(真空中为 299792458 m/s[2])传播。这种波动称为**电磁辐射**,其以不同的波长出现,从而产生从无线电波到伽马射线的辐射谱。

在偏微分方程形式和一致的单位制中,麦克斯韦的微观方程可以写成:
\[
\begin{aligned}
\nabla \cdot \mathbf{E} &= \frac{\rho}{\varepsilon_0} \\
\nabla \cdot \mathbf{B} &= 0 \\
\nabla \times \mathbf{E} &= -\frac{\partial \mathbf{B}}{\partial t} \\
\nabla \times \mathbf{B} &= \mu_0 \left(\mathbf{J} + \varepsilon_0 \frac{\partial \mathbf{E}}{\partial t}\right)
\end{aligned}~
\]
其中,\(\mathbf{E}\) 表示电场,\(\mathbf{B}\) 表示磁场,\(\rho\) 为电荷密度,\(\mathbf{J}\) 为电流密度,\(\varepsilon_0\) 为真空电容率,\(\mu_0\) 为真空磁导率。
\begin{figure}[ht]
\centering
\includegraphics[width=6cm]{./figures/fc8212bde3089bd9.png}
\caption{刻在麦克斯韦爱丁堡雕像上的麦克斯韦方程。} \label{fig_MAXS_1}
\end{figure}
这些方程有两个主要变体:
\begin{enumerate}
\item \textbf{微观方程}具有普遍适用性,但在常规计算中不便使用。它们将电场和磁场与总电荷和总电流联系起来,包括原子尺度上复杂的材料内部电荷和电流。
\item \textbf{宏观方程}定义了两个新的辅助场,用于描述物质的宏观行为,而不必考虑原子尺度的电荷和量子现象(如自旋)。但其使用需要实验确定的参数,以表征材料对电磁的响应。
\end{enumerate}
“麦克斯韦方程”一词通常也用于指等效的替代形式。基于电标势和磁标势的麦克斯韦方程版本常用于显式求解边值问题、分析力学或量子力学。协变形式(在时空中,而非分离的空间和时间)使得麦克斯韦方程与狭义相对论的兼容性显而易见。在高能物理和引力物理中常用的曲时空中的麦克斯韦方程与广义相对论相容。[注2] 实际上,阿尔伯特·爱因斯坦发展了狭义和广义相对论,以容纳麦克斯韦方程中光速不变的结果,遵循只有相对运动才具有物理意义的原则。

这些方程的发表标志着对之前分别描述的现象——磁、静电、光及相关辐射——的理论统一。从20世纪中期以来,人们理解到麦克斯韦方程并未精确描述电磁现象,而是更精确的量子电动力学理论的经典极限。
\subsection{方程的历史}  
主条目:麦克斯韦方程的历史
\subsection{概念描述} 
\subsubsection{高斯定律} 
\begin{figure}[ht]
\centering
\includegraphics[width=6cm]{./figures/c46a6d0d0e00cff7.png}
\caption{电场从正电荷指向负电荷} \label{fig_MAXS_2}
\end{figure}
高斯定律描述了电场与电荷之间的关系:电场从正电荷向外指向负电荷,电场通过闭合表面的净流出量与所包围的电荷成正比,包括由于材料极化产生的束缚电荷。该比例系数为真空介电常数。
\subsubsection{磁场的高斯定律}
\begin{figure}[ht]
\centering
\includegraphics[width=6cm]{./figures/528b4e2e05dd6692.png}
\caption{磁场的高斯定律:磁场线没有起点或终点,而是形成闭合回路或延伸至无穷大,正如图中电流环所产生的磁场所示。} \label{fig_MAXS_3}
\end{figure}
磁场的高斯定律指出,电荷没有磁性对应物,称为磁单极子;不存在孤立的北极或南极。[3] 相反,材料的磁场归因于偶极子,磁场通过闭合表面的净流出量为零。磁偶极子可以表示为电流回路或不可分离的相等且相反的“磁荷”对。具体来说,通过高斯面的总磁通量为零,磁场是一个螺线型矢量场。
\subsubsection{法拉第定律}
\begin{figure}[ht]
\centering
\includegraphics[width=10cm]{./figures/053be297f99c6caf.png}
\caption{在地磁风暴中,太阳风等离子体冲击地球磁场,引起磁场的时间变化,从而在地球的大气层和导电的岩石圈中感应出电场,这可能会导致电网不稳定。(非按比例绘制。)} \label{fig_MAXS_4}
\end{figure}
法拉第感应定律的麦克斯韦-法拉第形式描述了一个随时间变化的磁场如何与电场的旋度相关联。[3] 其积分形式表明,将电荷绕闭合回路移动所需的单位电荷功等于穿过所围表面的磁通量变化率。

电磁感应是许多电力发电机的工作原理:例如,旋转的条形磁铁会产生变化的磁场,并在附近的导线上生成电场。
\subsubsection{安培–麦克斯韦定律}
\begin{figure}[ht]
\centering
\includegraphics[width=8cm]{./figures/8d31bb58a14415fb.png}
\caption{磁芯存储器(1954年)是安培环路定律的一个应用。每个磁芯存储一位数据。} \label{fig_MAXS_5}
\end{figure}
安培定律的原始形式表明磁场与电流有关。麦克斯韦的补充指出,磁场也与变化的电场有关,麦克斯韦称之为位移电流。积分形式表明,电流和位移电流与沿着任意封闭曲线的磁场成正比。

麦克斯韦对安培环路定律的修改很重要,因为否则安培和高斯的定律必须针对静电场进行调整。[4] [需要澄清] 这一修改的一个结果是预测出随着电场变化而出现的旋转磁场。[3][5] 进一步的结果是预测出自维持的电磁波在空旷空间中传播。

从电荷和电流实验中预测出的电磁波速度与光速相匹配;事实上,光是一种电磁辐射(如X射线、无线电波等)。麦克斯韦在1861年认识到电磁波与光的联系,从而统一了电磁学和光学理论。
\subsection{以电场和磁场(微观或真空版)形式的表述}
在电场和磁场的表述中,有四个方程用于确定给定电荷和电流分布的场。另有一条独立的自然法则,即洛伦兹力定律,描述了电场和磁场如何作用于带电粒子和电流。按照惯例,麦克斯韦的原始方程中包含的这一定律版本现已不再包括在内。下列矢量微积分形式是奥利弗·赫维赛德的工作成果,[6][7] 并已成为标准。该形式具有旋转不变性,因此比麦克斯韦最初以 x、y 和 z 分量表达的 20 个方程在数学上更加透明。相对论性表述更加对称且满足洛伦兹不变性。同样的方程也可以用张量微积分或微分形式表达(见“§ 替代表述”部分)。

微分和积分表述在数学上是等价的;二者均有其用途。积分形式将空间区域内的场与边界上的场联系起来,通常可以用于简化计算,并直接从对称的电荷和电流分布中计算场。另一方面,微分方程是纯粹的局部方程,是在计算更复杂(非对称)情况中的场时更自然的起点,例如使用有限元分析。[8]
\subsubsection{符号说明}

除非另有说明,加粗符号代表向量量,斜体符号代表标量量。方程引入了电场 \( E \)(一个向量场)和磁场 \( B \)(一个伪向量场),它们通常都随时间和位置变化。方程中的源项是:
\begin{itemize}
\item 总电荷密度(每单位体积的总电荷)\( \rho \),以及
\item 总电流密度(每单位面积的总电流)\( J \)。
\end{itemize}
方程中出现的通用常数(前两个常数仅在国际单位制(SI)表述中明确出现)包括:
\begin{itemize}
\item 真空介电常数 \( \varepsilon_0 \),
\item 真空磁导率 \( \mu_0 \),以及
\item 光速 \( c = (\varepsilon_0 \mu_0)^{-1/2} \)。
\end{itemize}
\textbf{微分方程}

在微分方程中,
\begin{itemize}
\item 纳布拉符号 \( \nabla \) 表示三维梯度算符(即“del”算符),
\item \( \nabla \cdot \) 符号(读作“del dot”)表示散度算符,
\item \( \nabla \times \) 符号(读作“del cross”)表示旋度算符。
\end{itemize}

\textbf{积分方程}

在积分方程中,
\begin{itemize}
\item \( \Omega \) 是具有封闭边界表面 \( \partial \Omega \) 的任意体积,
\item \( \Sigma \) 是具有封闭边界曲线 \( \partial \Sigma \) 的任意曲面。
\end{itemize}
这些方程在时间不变的表面和体积下更容易理解。时间不变的表面和体积是‘固定的’,在给定的时间间隔内不会变化。例如,由于表面是时间不变的,我们可以在法拉第定律中将微分符号带到积分符号下:
\[
\frac{\mathrm{d}}{\mathrm{d} t} \iint_{\Sigma} \mathbf{B} \cdot \mathrm{d} \mathbf{S} = \iint_{\Sigma} \frac{\partial \mathbf{B}}{\partial t} \cdot \mathrm{d} \mathbf{S} \,,~
\]
麦克斯韦方程组可以通过使用适当的高斯和斯托克斯公式,并使用可能依赖时间的表面和体积,以微分形式来表示。
\begin{itemize}
\item  \(\iint_{\partial \Omega}\) 表示在边界表面 ∂Ω 上的曲面积分,其中环路表示该表面是闭合的。
\item  \(\iiint_{\Omega}\) 表示在体积 Ω 上的体积积分。
\item  \(\oint_{\partial \Sigma}\) 表示在边界曲线 ∂Σ 上的线积分,其中环路表示该曲线是闭合的。
\item  \(\iint_{\Sigma}\) 表示在表面 Σ 上的曲面积分。
\end{itemize}

封闭在体积 Ω 内的总电荷 \( Q \) 是在 Ω 上的电荷密度 ρ 的体积积分(参见下面的‘宏观表述’部分):
\[
Q = \iiint_{\Omega} \rho \, \mathrm{d} V,~
\]
其中 \( \mathrm{d} V \) 是体积微分元。

总磁通量 \( \Phi_B \) 是通过固定表面 Σ 的磁场 B 的曲面积分:
\[
\Phi_B = \iint_{\Sigma} \mathbf{B} \cdot \mathrm{d} \mathbf{S},~
\]
总电通量 \( \Phi_E \) 是通过 Σ 的电场 E 的曲面积分:
\[
\Phi_E = \iint_{\Sigma} \mathbf{E} \cdot \mathrm{d} \mathbf{S},~
\]
总电流 \( I \) 是通过 Σ 的电流密度 J 的曲面积分:
\[
I = \iint_{\Sigma} \mathbf{J} \cdot \mathrm{d} \mathbf{S},~
\]
其中 \( \mathrm{d} \mathbf{S} \) 表示与表面 Σ 垂直的微分表面积矢量(矢量面积有时用 A 而不是 S 表示,但这与磁矢量势的符号冲突)。
\subsection{以国际单位制量进行表述}
\begin{table}[ht]
\centering
\caption\label{MAXS}
\begin{tabular}{|c|c|c|c|}
\hline
\textbf{名称} & \textbf{积分形式} & \textbf{微分形式}  \\
\hline 高斯定律 & \(\oint_{\partial \Omega} \mathbf{E} \cdot \mathrm{d} \mathbf{S} = \frac{1}{\varepsilon_0} \iiint_{\Omega} \rho \, \mathrm{d}V\) & \(\nabla \cdot \mathbf{E} = \frac{\rho}{\varepsilon_0}\)\\
\hline 磁场的高斯定律 & \(\oint_{\partial \Omega} \mathbf{B} \cdot \mathrm{d} \mathbf{S} = 0\) & \(\nabla \cdot \mathbf{B} = 0\)  \\
\hline 麦克斯韦-法拉第方程(法拉第电磁感应定律) & \(\oint_{\partial \Sigma} \mathbf{E} \cdot \mathrm{d} \ell = - \frac{\mathrm{d}}{\mathrm{d} t} \iint_{\Sigma} \mathbf{B} \cdot \mathrm{d} \mathbf{S}\) & \(\nabla \times \mathbf{E} = - \frac{\partial \mathbf{B}}{\partial t}\) \\
\hline 安培-麦克斯韦方程 & \(\oint_{\partial \Sigma} \mathbf{B} \cdot \mathrm{d} \ell = \mu_0 \left( \iint_{\Sigma} \mathbf{J} \cdot \mathrm{d} \mathbf{S} + \varepsilon_0 \frac{\mathrm{d}}{\mathrm{d} t} \iint_{\Sigma} \mathbf{E} \cdot \mathrm{d} \mathbf{S} \right)\) & \(\nabla \times \mathbf{B} = \mu_0 \left( \mathbf{J} + \varepsilon_0 \frac{\partial \mathbf{E}}{\partial t} \right)\) \\
\hline 
\end{tabular}
\end{table}
\subsection{使用高斯单位制的表述}  
可以通过将带有维度的因子 \( \varepsilon_0 \) 和 \( \mu_0 \) 吸收到单位中(从而重新定义它们),来简化电荷、电场和磁场的定义,以便于理论计算。配合洛伦兹力定律中的量值变化,这将产生相同的物理效果,例如带电粒子的轨迹或电动机所做的功。这些定义在理论物理和高能物理中更为常用,因为此时将电场和磁场设置为相同单位可以简化电磁张量的表示:统一电场和磁场的洛伦兹协变对象将包含具有统一单位和维度的分量。[9]:vii 这样的修正定义通常在高斯(CGS)单位制中使用。使用这些定义,俗称“在高斯单位制中”,麦克斯韦方程变为:[11]

\begin{table}[ht]
\centering
\caption\label{MAXS1}
\begin{tabular}{|c|c|c|c|}
\hline
\textbf{名称} & \textbf{积分方程} & \textbf{微分方程}  \\
\hline 高斯定律 &  \( \iint_{\partial \Omega} \mathbf{E} \cdot d\mathbf{S} = 4\pi \iiint_{\Omega} \rho \, dV \) &  \( \nabla \cdot \mathbf{E} = 4\pi \rho \) \\
\hline 磁场的高斯定律 & \( \iint_{\partial \Omega} \mathbf{B} \cdot d\mathbf{S} = 0 \) &  \( \nabla \cdot \mathbf{B} = 0 \)  \\
\hline 麦克斯韦-法拉第方程(法拉第感应定律) & \( \oint_{\partial \Sigma} \mathbf{E} \cdot d\boldsymbol{\ell} = -\frac{1}{c} \frac{d}{dt} \iint_{\Sigma} \mathbf{B} \cdot d\mathbf{S} \) & \( \nabla \times \mathbf{E} = -\frac{1}{c} \frac{\partial \mathbf{B}}{\partial t} \) \\
\hline 安培-麦克斯韦方程 & \( \oint_{\partial \Sigma} \mathbf{B} \cdot d\boldsymbol{\ell} = \frac{1}{c} \left( 4\pi \iint_{\Sigma} \mathbf{J} \cdot d\mathbf{S} + \frac{d}{dt} \iint_{\Sigma} \mathbf{E} \cdot d\mathbf{S} \right) \) & \( \nabla \times \mathbf{B} = \frac{1}{c} \left( 4\pi \mathbf{J} + \frac{\partial \mathbf{E}}{\partial t} \right) \)\\
\hline 
\end{tabular}
\end{table}
当选择一个以光速 \( c \) 为单位的量纲系统进行无量纲化时,方程会稍微简化,这样秒和光秒可以互换使用,并且 \( c = 1 \)。

进一步的简化可以通过吸收 \( 4\pi \) 的因子来实现。这个过程称为“有理化”,会影响库仑定律或高斯定律是否包含此类因子(参见主要用于粒子物理的海维赛德-洛伦兹单位制)。
\subsection{微分形式和积分形式之间的关系}  
微分形式和积分形式的等价性是高斯散度定理和开尔文–斯托克斯定理的结果。
\subsection{通量与散度}

根据纯粹的数学高斯散度定理,边界面 ∂Ω 上的电通量可以重写为:

\[\Oiint_{\scriptstyle \partial \Omega} \mathbf{E} \cdot \mathrm{d} \mathbf{S} = \iiint_{\Omega} \nabla \cdot \mathbf{E} \, \mathrm{d} V~\]

因此,高斯方程的积分形式可以重写为:

\[\iiint_{\Omega} \left(\nabla \cdot \mathbf{E} - \frac{\rho}{\varepsilon_0}\right) \, \mathrm{d} V = 0~\]

由于 Ω 是任意的(例如,任意小的球且中心任意),只有当被积函数在各处为零时,这个方程才成立。这便是高斯方程的微分形式,仅需进行微小的重排。

类似地,将高斯磁定律的积分形式中的磁通量重写为:

\[\oiint_{\scriptstyle \partial \Omega} \mathbf{B} \cdot \mathrm{d} \mathbf{S} = \iiint_{\Omega} \nabla \cdot \mathbf{B} \, \mathrm{d} V = 0.~\]

对于所有的 Ω,当且仅当 \(\nabla \cdot \mathbf{B} = 0\) 在各处成立时,这个等式才成立。