% 质子
% license CCBYSA3
% type Wiki

(本文根据 CC-BY-SA 协议转载自原搜狗科学百科对英文维基百科的翻译)

\textbf{中子}是亚原子粒子,符号为n或者n⁰,净电荷为零且质量略大于质子。质子和中子构成了原子的核子。因为质子和中子在原子核内的行为相似,每个质子和中子的质量大约为1原子质量单位,它们统称为核子。[1]它们的性质和相互作用由核物理描述。

原子核的化学和核性质由质子数以及中子数决定,分别称为原子序数和中子数。原子质量数是核子的总数。比如,碳原子序数为6,并且常见的碳-12有6个中子,而罕见的碳-13有7个中子。有些元素在自然界中只有一种稳定同位素,例如氟。有些元素则有许多稳定的同位素,比如锡有十种稳定同位素。

在原子核内,质子和中子通过核力结合在一起。除了单质子氢原子之外,原子核的稳定需要中子。中子在核裂变和聚变中大量产生。它们通过裂变、聚变和中子俘获过程参与在恒星内化学元素的核合成。

中子对核能生产至关重要。在詹姆斯·查德威克1932年发现中子后的十年里,中子被用来诱导许多不同类型的核嬗变。随着1938年核子裂变的发现,[2]人们很快认识到,裂变事件产生的每一个中子都可能引起进一步的裂变事件,这种级联反应被称为核链式反应。这些发现导致了第一个自持的核反应堆 ( 芝加哥一号堆,1942年)和第一个核武器 ( 三一,1945年)的产生。

自由中子虽然不是电离原子,但会导致致电离辐射。因此,当剂量较大时,它们具备生物性危害。地球上存在有小的自然的自由中子,主要来源于宇宙线射线浴以及地壳内自发裂变元素产生的辐射。用于辐射和中子散射实验的自由中子来自于专用的中子源,如中子发生器、研究堆和散裂源。

\subsection{描述}
原子核由许多质子形成,Z(原子序数)和中子数,N(中子数),由核力结合在一起。原子序数定义原子的化学性质,中子数决定同位素或核素。[3]同位素和核素这两个术语经常被当作同义词使用,但是它们分别指化学性质和核性质。严格来说,同位素是两种或多种质子数相同的核素;具有相同中子数的核素被称为同中子异荷素。原子质量数,符号$A$,等于$Z+N$。具有相同原子质量数的核素称为同量异位素。氢原子最常见的同位素(带有化学符号$^1h$),其核子是一个单独的质子。重氢同位素氘的原子核($D$或$^2h$)和氚 ($T$或$^3h$)分别包含一个和两个中子。其他所有类型的原子核都由两个或多个质子和不同数量的中子组成。化学元素铅中最常见的核素,$^{208}Pb$有82个质子和126个中子。核素表包括所有已知的核素。尽管中子不是化学元素,但它包含在这个表中。[3]

自由中子的质量为939,565,$413.3 eV /c^2$,或1.674927471×10−27 kg,或1.00866491588 u。中子的均方半径约为0.8×10−15 m,或0.8 fm ,[4]这是自旋为1的费米子。[5]质子带正电荷,直接受电场影响,而中子不带电荷,故不受电场影响。然而,中子具有磁矩,因此受磁场影响。中子的磁矩是负的,其取向与其自旋方向相反。[6]



















