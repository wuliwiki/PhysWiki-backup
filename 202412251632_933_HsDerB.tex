% 导数的计算(高中)
% keys 导数|求导
% license Usr
% type Tutor

\begin{issues}
\issueDraft
\end{issues}

\pentry{导数\nref{nod_HsDerv}}{nod_ed15}

前文提到过,为了更高效地计算导数,人们研究并总结了多种求导的方法,这些方法不仅简化了求导过程,还使得求导成为一种独立的数学运算。在高中阶段,能够快速、熟练地应用求导方法来求某个函数的导数是要求。但高中教材中对于这部分内容是空降的,相当于直接给了一个结果让学生使用。尽管熟练使用与这些公式的来源无关,但并不符合数学的严谨性。为了给

\subsection{求导法则}

为记录方便,下面记$u=f(x),v=g(x),u'=f'(x),v'=g'(x)$。

\begin{itemize}
\item 加减法:$(u\pm v)'=u'\pm v'$
\item 乘法:$(uv)'=u'v+uv'$
\item 除法:$\displaystyle\left(\frac{u}{v}\right)'=\frac{u'v-uv'}{v^2}$,特例是某个函数的倒数的导数:$\displaystyle\left(\frac{1}{v}\right)'=-\frac{v'}{v^2}$
\item 复合函数:$(f(v))'=f'(v)v'$
\end{itemize}

\subsection{基本初等函数的导数推导}

\subsection{对照表}

这里将常见的函数与导数对照表列出如下,方便查询。具体介绍需查看每个函数自己的页面。

\begin{table}[ht]
\centering
\caption{高中常见函数及其导数}\label{tab_HsDerv1}
\begin{tabular}{|c|c|c|}
\hline
\textbf{函数名称}     & \textbf{函数 $f(x)$}     & \textbf{导函数 $f'(x)$}     \\ \hline
幂函数&$x^n$                    & $n x^{n-1}$                \\ \hline
反比例函数&$\displaystyle\frac{1}{x}$             & $\displaystyle-\frac{1}{x^2}$           \\ \hline
指数函数(e为底)&$e^x$                     & $e^x$                      \\ \hline
对数函数(e为底)&$\ln(x)$                  & $\displaystyle\frac{1}{x}$              \\ \hline
指数函数&$a^x$                     & $a^x\ln a $                      \\ \hline
对数函数&$\log_a(x)$                  & $\displaystyle \frac{1}{x\ln a}$              \\ \hline
正弦函数&$\sin(x)$                 & $\cos(x)$                  \\ \hline
余弦函数&$\cos(x)$                 & $-\sin(x)$                 \\ \hline
正切函数&$\tan(x)$                 & $\displaystyle \frac{1}{\cos^2(x)}$                \\ \hline
\end{tabular}
\end{table}

