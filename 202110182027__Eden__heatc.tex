% 热传导定律
% 热传导

\subsection{傅里叶传导定律}
定义热流密度 $\bvec h$:单位时间单位面积的热流量.傅里叶热传导定律说的是

\begin{equation}
\bvec h = -\kappa \bvec \nabla T
\end{equation}

其中 $\kappa$ 为系统的热导率.这意味着热量会沿温度梯度的方向传导.

如果简化系统模型,设各层温度不均匀,在同一个高度上各处温度相等.那么傅里叶热传导定律可以简化为
\begin{equation}
H=\frac{\Delta Q}{\Delta t}=-\kappa \frac{\dd T}{\dd z}S
\end{equation}

再考虑一般的情况:\textbf{三维热传导方程}.由于单位体积内需要吸收 $c\rho \Delta T$ 的热量才能升高 $\Delta T$ 的温度,所以可以得到积分关系式
\begin{equation}
\int_V c\rho \frac{\partial T}{\partial t} \dd V = \int_V \frac{\partial U}{\partial t} \dd V= \oint_S \kappa \bvec \nabla T \cdot \bvec \dd S
\end{equation}

由此可以得到微分表达式 $c\rho \frac{\partial T}{\partial t} = \kappa \nabla^2 T$,即
\begin{equation}
\frac{\partial T}{\partial t}=k\nabla^2 T
\end{equation}
其中 $k=\kappa/c\rho$ 称作热扩散率.

注意上面讨论的是\textbf{没有热源、热扩散系数处处相等}的情况.如果有热源,则方程右端要加上 $Q$.如果热扩散系数并不处处相等,例如两个不同的介质的交界处,往往要对此设定边界条件——例如一维杆两端与空气接触的热对流现象、开水与空气接触时逐渐散热的现象等;在这类边界上如果温度差不大,有\textbf{牛顿冷却定律}成立:热流密度 $\bvec h$ 与温度的梯度成正比,但这个比例系数将与各种复杂的因素有关.


\subsection{几个例子}
\begin{example}{平衡温度分布}
一维杆 $0\le x\le L$,左端右端与恒温热源接触,左端温度恒为 $T_1$,右端温度为 $T_2$,中间部分绝热.求其平衡温度分布.
\end{example}
平衡态热流处处为 $0$,所以 $\nabla^2 T=0$,即 $\partial^2 T/\partial x^2=0$,所以 $T$ 关于 $x$ 的函数是一次函数.$T=T_1+(T_2-T_1)x/L$.

\subsection{输运过程的微观解释}
当系统处于非平衡态时,会自发地向平衡态过度,从而产生\textbf{动量、能量、质量}等宏观的流动,这些过程统称为\textbf{耗散过程}.输运过程在微观上就是耗散过程.例如当热学平衡条件不满足时,有温度梯度,从而有热传导方程(能量的传递);力学平衡条件不满足时,有黏滞现象(动量的传递),从而有牛顿黏滞定律;化学平衡条件不满足时,有扩散现象(质量的传递),从而有菲克(Fick)扩散定律.我们先给出这三个定律的表达式:
\begin{align}
h=-\kappa \frac{\dd T}{\dd z}\\
J_p=-\eta \frac{\dd u}{\dd z}\\
J_M=-D\frac{\dd \rho}{\dd z}
\end{align}

其中牛顿黏滞方程的 $J_p$ 为动量流密度,菲克定律的 $J_m$ 为质量流密度.一些参考文献上将黏滞定律写作 $\tau = -\mu \frac{\dd u}{\dd z}$,其中 $\tau$ 为剪应力,该方程则表达了剪应力与流体速度场的梯度成正比.这些方程从直观上是容易想象的,虽然宏观上代表不同的现象,但其方程形式却是相同的.经过下面的分析,我们将发现这三种输运过程在\textbf{微观上的机制}本质是相同的.

先简化模型,我们设系统的各个宏观量(例如质量密度、热运动平均能量、动量密度)在同一高度上处处相同(可写成关于 $z$ 和 $t$ 的函数).为了精简细节,简化计算,我们对系统设定几条重要的热力学假设:

\textbf{假设 1}. 微观上有大量的分子在作无规则运动,且可以运用\textbf{局部平衡假设}:每个小单元内的温度、粒子数密度、平均动量可以确定;小单元的状态量随时间与空间的变化分别为 $T(z,t),n(z,t),\bar v(z,t)$.粒子服从\textbf{麦克斯韦——玻尔兹曼分布}\upref{MxwBzm},热运动将使得 $\Delta T$ 内穿过 $\Delta S$ 的平均分子数为 $\frac{1}{4}n\bar v \Delta S\Delta t$.\textbf{由于物理量在 $z$ 方向上分布不同,热运动将使得粒子可以把界面一侧的物理量带到界面另一侧}.

设 $t$ 时刻物理量 $Q$ 的分布为 $Q(z)$,设 $q=Q/N$ 为平均意义下每个粒子携带的物理量,例如可以取 $\bar \epsilon,m,\bar v$,对应热运动能量、质量、动量.考察 $z=z_0$ 处平行于 $xy$ 面的面元 $\Delta S$,由热运动(上方的粒子到下方,下方的粒子到上方)引起的通过这一面元的物理量为
\begin{equation}
\Delta Q_{A\rightarrow B}=(\frac{1}{4}n\bar v \Delta S\Delta t\cdot q)_A-(\frac{1}{4}n\bar v \Delta S\Delta t\cdot q)_B
\end{equation}
物理量流度为 $J_{A\rightarrow B}=\frac{\Delta Q}{\Delta t}$


\textbf{假设 2}.气体足够稀薄,三分子碰撞概率可以忽略不计,从而理想气体状态方程近似成立,且分子\textbf{平均自由程公式有效}.但不能太稀薄,对每一局部平衡的小单元,平均自由程公式有效.
