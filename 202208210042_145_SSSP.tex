% 单源最短路径
% keys 单源最短路径|最短路|算法|C++

单元最短路径问题,为给定一张有向图 $G = (V, E)$,$V$ 是点集,$E$ 是边集,$|V|= n$,$|E| = m$,求给定的源点(起点)$s \in V$ 到每个结点 $v \in V$ 的最短路径.$(x, y, w)$ 表示有一条从结点 $x$ 指向结点 $y$ 的有向边,边权为 $w$.

\subsection{Dijkstra 算法}

\verb|dist[i]| 表示从起点 $s$ 到结点 $i$ 的\textbf{实际}最短路径的长度(这条路径的权值之和).

$\delta(u)$ 表示从起点 $s$ 到结点 $u$ 的估计最短路径长度.任意时刻都存在 \verb|dist[u]| $\geq \delta(u)$.

Dijkstra 算法是一种求解没有负权边的图中的单源最短路问题.将所有结点划分为两个集合,$S$ 集合存储当前已经确定了最短路的结点,$T$ 集合存储当前还未确定最短路的结点.

具体做法是:
\begin{enumerate}
\item 初始化所有点的 \verb|dist| 距离为正无穷,起点的距离为 $0$(\verb|dist[S] = 0|).
\item 每次从 $T$ 集合中选出一条 \verb|dist| 值最小的结点 $t$,并把 $t$ 结点加入 $S$ 集合中.
\item 用 $t$ 更新其他结点.
\item 重复 $2 \sim 3$ 步骤,直到所有点都被加入 $S$ 集合.
\end{enumerate}

朴素 Dijkstra 算法的时间复杂度为 $\mathcal{O}(n^2)$,使用二叉堆可以使操作 $2$ 的时间复杂度从 $\mathcal{O}(n^2)$ 的时间复杂度优化到 $\mathcal{O}(\log_2 n)$.每更新一条边 $(x, y)$,就把 $y$ 这个结点和 \verb|dist[y]| 值插入到二叉堆中.每次找最小值直接取堆顶即可.每次取堆顶时判断堆顶是不是已经被访问过了,如果被访问过了,直接忽略这次操作,否则会重复更新,导致影响时间复杂度.所以堆优化版 Dijkstra 的时间复杂度为 $\mathcal{O}(m \log_2 n)$.

\textbf{Dijkstra 算法准确性证明:}
参考算法导论中的反证法

要证明在算法结束时,每个点的实际最短距离等于估计最短距离,即证明的是对于每个结点 $u \in V$,当结点 $u$ 第一次加入到 $S$ 集合时,$dist[u] =\delta(u)$,也就是 \verb|dist[u]| 必然满足已经是起点到 $u$ 的最短距离.

初始化: $S = \varnothing$,方程显然成立,得证.

接下来使用反证法证明此结论,假设结点 $u$ 是第一次加入 $S$ 集合时使得 $dist[u] \neq \delta(u)$,因为 $s$ 结点是第一次加入 $S$ 集合中的结点,所以有 $dist[s] = \delta(s) = 0$,因为 $s$ 结点是第一个加入 $S$ 结点中的结点,所以将 $u$ 结点加入 $S$ 集合之前,必定有 $S \neq \varnothing$.此时一定有一条从 $s$ 结点到 $u$ 结点的路径,否则 $dist[u] = \delta(u) = +\infty$,而这与假设矛盾,所以一定存在一条路径从结点 $s$ 到结点 $u$.

所以这条路径上一定存在一条最短从结点 $s$ 到结点 $u$ 的最短路径 $p$.
将 $p$ 分解为:$s \thicksim x \thicksim y \thicksim u$,其中 $y$ 为第一个属于 $T$ 集合中的点,$x$ 为 $y$ 的前驱结点.有可能存在 $s = x$ 或 $y = u$ 的情况.

因为结点 $u$ 是第一次加入 $S$ 集合时不满足 $dist[u] \neq \delta(u)$ 的结点,所以在之前所以的结点都满足实际最短路径等于估计最短路径,所以在将 $x$ 结点加入到 $S$ 集合时,满足 $dist[x] = \delta(x)$.此时 $x$ 结点会更新其他结点,所以在将 $u$ 加入到 $S$ 集合时,$dist[y] = \delta(y)$.

因为结点 $y$ 是结点 $u$ 的前面的一个结点,所以存在 $\delta(y) \leq \delta(u)$.所以 $dist[y] = \delta(y) \leq \delta(u) \leq dist[u]$.又因为结点 $u$ 是算法在 $T$ 集合中选择的第一个点,所以有 \verb|dist[u] <= dist[y]|.所以上面的不等式其实都为等式,所以 $\delta(u) = dist[u]$ 成立,这与假设矛盾,所以证明得证.

\textbf{C++ 代码:}

朴素版 Dijkstra
\begin{lstlisting}[language=cpp]
const int N = 1e5 + 10, M = 510, INF = 0x3f3f3f3f;
int n, m, dist[N], st[N], g[M][M]; // st 为 true 表示在 S 集合,反之不在

int dijkstra()
{
    memset(dist, 0x3f, sizeof dist);  // 初始化距离
    dist[1] = 0;
    
    for (int i = 0; i < n - 1; i ++ ) // 迭代 n - 1 次
    {
        int t = -1;
        for (int j = 1; j <= n; j ++ )
            if (!st[j] && (t == -1 || dist[j] < dist[t]))
                t = j; // t 为 不在 S 集合中距离最短的点
        
        st[t] = true; // 加入 S 集合
        
        for (int j = 1; j <= n; j ++ ) // 更新
            dist[j] = min(dist[j], dist[t] + g[t][j]);
    }
    
    // 求 1 号点到 n 号点的最短距离
    return dist[n];
}

int main()
{
    cin >> n >> m;
    
    // 稠密图用邻接矩阵存图
    memset(g, 0x3f, sizeof g);
    for (int i = 0; i < m; i ++ ) 
    {
        int a, b, c;
        cin >> a >> b >> c;
        // 因为图中可能有重边,所以只保留权值小的边
        g[a][b] = min(g[a][b], c);
    }
    
    // 输出邻接矩阵,没有边的地方初始化为正无穷
    for (int i = 1; i <= 4; i ++ ) {
        for (int j = 1; j <= 4; j ++ )
            printf("%10d ", g[i][j]);
        cout << endl;
    }
    
    int t = dijkstra();
    if (t == INF) cout << -1 << endl; 
    else cout << t << endl;
    
    return 0;
}
\end{lstlisting}

堆优化版 Dijkstra

\begin{lstlisting}[language=cpp]
const int N = 1e6 + 10, INF = 0x3f3f3f3f;
typedef pair<int, int> PII; // first 存储距离,second 存储结点编号
priority_queue<PII, vector<PII>, greater<PII>> heap; // 小根堆
int n, m, dist[N], st[N], h[N], e[N], w[N], ne[N], idx; // 稀疏图用邻接表存图

int dijkstra()
{
    memset(dist, 0x3f, sizeof dist);
    dist[1] = 0;
    heap.push({0, 1});  // 把起点加入到优先队列中
    
    while (heap.size())
    {
        auto t = heap.top(); // 取出堆顶
        heap.pop();
        
        int ver = t.second;
        if (st[ver]) continue;  // 如果访问过就不重复访问
        st[ver] = true;
        
        for (int i = h[ver]; ~i; i = ne[i])
        {
            int j = e[i];
            // 更新
            if (dist[j] > dist[ver] + w[i])
            {
                dist[j] = dist[ver] + w[i];
                heap.push({dist[j], j});
            }
        }
    }
    
    // 求 1 号点到 n 号点的最短距离
    return dist[n];
}
\end{lstlisting}

\subsection{Bellman-Ford 算法}

Bellman-Ford 算法可以求解带有负权边的单源最短路径.

具体步骤是:
