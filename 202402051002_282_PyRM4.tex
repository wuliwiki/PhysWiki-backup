% Python RoboMaster EP 教程—通讯方式
% keys Robomaster|机器人|Python
% license Xiao
% type Tutor

\pentry{Python 基础\nref{nod_PyFi}}{nod_60b7}

与RoboMaster EP的3种连接方式在通讯协议上支持 TCP 和 UDP 通讯(实时的控制运动建议用 UDP, 事件型控制建议用 TCP)。

(TCP通讯内容可以参考:\href{https://en.wikipedia.org/wiki/Transmission_Control_Protocol}{TCP Wikipedia} ,UDP通讯可以参考:\href{https://en.wikipedia.org/wiki/User_Datagram_Protocol}{UDP Wikipedia})

\begin{itemize}
\item \autoref{sub_PyRM4_1} TCP 通讯
\item \autoref{sub_PyRM4_2} UDP 通讯
\end{itemize}

\subsection{TCP 通讯}\label{sub_PyRM4_1}

输入以下代码,可以测验:

\begin{lstlisting}[language=python]
from robomaster import robot

import robomaster

if __name__ == '__main__':
    # 在程序内实例化Robot对象
    ep_robot = robot.Robot()

    # 指定连接方式为STA 组网模式, 网络通讯方式 tcp
    ep_robot.initialize(conn_type='sta', proto_type='tcp')

    version = ep_robot.get_version()
    # 输出机器人固件版本号信息
    print("Robot version: {0}".format(version))
    # 关闭机器人
    ep_robot.close()

\end{lstlisting}

运行结果:
\begin{lstlisting}[language=pythonC]
Robot Version: xx.xx.xx.xx
\end{lstlisting}

\subsection{UDP 通讯}\label{sub_PyRM4_2}

输入以下代码,可以测验:

\begin{lstlisting}[language=python]
import robomaster
from robomaster import robot


if __name__ == '__main__':
    # 在程序内实例化Robot对象
    ep_robot = robot.Robot()
    
    # 指定连接方式为AP 直连模式, 网络通讯方式 udp
    ep_robot.initialize(conn_type='ap', proto_type='udp')

    version = ep_robot.get_version()
    # 输出机器人固件版本号信息
    print("Robot version: {0}".format(version))
    # 关闭机器人
    ep_robot.close()
\end{lstlisting}

运行结果:
\begin{lstlisting}[language=pythonC]
Robot Version: xx.xx.xx.xx
\end{lstlisting}
