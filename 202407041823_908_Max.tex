% 马克斯·普朗克
% license CCBYSA3
% type Wiki

(本文根据 CC-BY-SA 协议转载自原搜狗科学百科对英文维基百科的翻译)

马克斯·卡尔·恩斯特·路德维希·普朗克(德语:[ˈplaŋk];[1] English: /ˈplæŋk/;[2] 1858年4月23日 – 1947年10月4日),德国理论物理学家,1918年因发现能量量子获得诺贝尔物理学奖。[3]

普朗克对理论物理做出了许多贡献,但他作为物理学家的名声主要取决于他作为量子理论创始人的角色,[4]这彻底改变了人类对原子和亚原子过程的理解。1948年,德国科学机构凯泽·威廉学会(普朗克曾两次担任主席)更名为马克斯·普朗克学会(MPS)。议员现在包括代表广泛科学方向的83个机构。

\subsection{生活和事业}
\begin{figure}[ht]
\centering
\includegraphics[width=10cm]{./figures/25347aee0b1e0c17.png}
\caption{马克思普朗克十岁时的签名} \label{fig_Max_1}
\end{figure}
普朗克来自一个传统的知识分子家庭。他的曾祖父和祖父都是哥廷根的神学教授;他的父亲是基尔大学和慕尼黑大学的法学教授[5]。他的一个叔叔也是法官。[6]

普朗克生于荷斯坦基尔,家承约翰·朱利叶斯·威廉·普朗克和他的第二任妻子艾玛·帕齐希。他受洗时的名字是卡尔·凯尔希纳·马克思·普朗克;他的名字,马克思 (现已过时的变体马库斯或者仅仅是马克思的错误写法,实际上是马克西米兰的简写)被表示为“称谓名称”。[7]然而,到了十岁,他就用马克思这个名字签了名并且用这个名字度过余生。[8]

普朗克是这个家庭的第六个孩子,尽管他的两个兄弟姐妹来自他父亲的第一次婚姻。在普朗克早年时期,战争很普遍 他最早的记忆是普鲁士和奥地利军队在1864年进军基尔。[6]1867年,全家搬到了慕尼黑,普朗克就读于马克西米利安体育学校,在那里他接受了数学家赫尔曼·穆勒的指导,赫尔曼对青年很感兴趣,教他天文学、力学和数学。普朗克最早是从穆勒那里学到能量守恒原理的。普朗克早在17岁就毕业了。[9]这就是普朗克第一次接触物理领域的方式。

普朗克在音乐方面很有天赋。他上唱歌课,弹钢琴,管风琴和大提琴,还创作歌曲和歌剧。然而,他选择学习物理而不是音乐。
\begin{figure}[ht]
\centering
\includegraphics[width=6cm]{./figures/9dd227bb43a347f2.png}
\caption{878年,年轻时期的普朗克} \label{fig_Max_2}
\end{figure}
慕尼黑物理学教授菲利普·冯·乔利(Philipp von Jolly)建议普朗克不要进入物理学领域,他说,“在这个领域,几乎所有的东西都已经被发现,剩下的只是填补一些空白。”[10]普朗克回答说,他不想发现新的东西,只是想了解这个领域的已知基础,于是他于1874年在慕尼黑大学开始了研究。在乔利的监督下,普朗克完成了他科学生涯中唯一的实验,研究氢气在加热的铂中的扩散,但他随后转移到了理论物理方向。

1877年,他与物理学家赫尔曼·赫尔姆霍茨、古斯塔夫·基尔霍夫和数学家卡尔·魏尔斯特拉斯一起在柏林弗里德里希·威廉大学学习了一年。他写道,赫尔姆霍茨从来没有做好充分准备,说话缓慢,不断计算错误,令听众厌烦,而基尔霍夫却在枯燥乏味的讲座中精心准备并发言。他很快就和赫尔姆霍茨成了好朋友。在那里,他进行了一个主要是自学克劳修斯著作的项目,这使他选择热力学作为他的研究领域。

1878年10月,普朗克通过了资格考试,在1879年2月完成了关于论机械热理论的第二原理 论文答辩。他曾在慕尼黑的学校短暂教过数学和物理。

到1880年,普朗克获得了欧洲最高的两个学位。第一个是博士学位,他完成了详细阐述他的研究和热力学理论的论文。[6] 然后他提交了他的论文,题目是垂直温度下的各向同性卡尔珀 (不同温度下各向同性体的平衡态),这使他获得了适应训练论文资格。

\subsubsection{1.1 学业生涯}
随着他的适应训练论文的完成,普朗克成为了慕尼黑的一名无薪私人教师(德国学术等级相当于讲师/助理教授),直到他被提供一个学术职位。虽然他最初被学术界忽视,但他推进了他在热理论领域的工作,并发现了一个又一个与吉布斯相同的热力学形式,却他没有意识到这一点。克劳修斯关于熵的思想在他的工作中占据了核心地位。

1885年4月,基尔大学任命普朗克为理论物理副教授。接下来是熵及其处理的进一步研究,特别是在物理化学中的应用。在1897年,他发表了他的热力学论文 。[11]他为斯凡特·奥古斯特·阿伦尼乌斯的电解离理论提供了热力学基础。

1889年,他被任命为柏林弗里德里希-威廉-大学基尔霍夫职位的继任者,[12]大概是由于赫尔姆霍茨的调解,并在1892年成为一名正教授。1907年,普朗克获得了玻尔兹曼在维也纳的职位,但被拒绝留在柏林。1909年,作为一名柏林大学教授,他被邀请成为纽约哥伦比亚大学理论物理的欧内斯特·肯普顿·亚当斯讲师。哥伦比亚大学教授A. P .威尔斯翻译并共同出版了他的一系列讲座。[13]他于1926年1月10日从柏林退休,[14]由埃尔温·薛定谔继任。[15]

\subsubsection{1.2 家庭}
1887年3月,普朗克娶了一个校友的妹妹玛丽默克(1861-1909),并和她一起搬进基尔的一间转租公寓。他们有四个孩子:卡尔 (1888-1916),双胞胎艾玛 (1889-1919)和格雷特 (1889-1917),和欧文 (1893-1945)。

在柏林公寓后,普朗克一家住在柏林-格吕内瓦尔德(柏林)旺根海姆大街21号的一栋别墅里。其他几位来自柏林大学的教授住在附近,其中包括神学家阿道夫·冯·哈那克,他成为普朗克的密友。不久,普朗克之家成为了一个社会和文化中心。许多著名的科学家,如阿尔伯特·爱因斯坦、奥托·哈恩和莉泽·迈特纳都是常客。在赫尔姆霍茨的家里已经建立了联合表演音乐的传统。

几年快乐后,玛丽·普朗克于1909年7月去世,可能是死于肺结核。1911年3月,普朗克娶了他的第二任妻子玛格·冯·赫斯林(1882-1948);12月,他的第五个孩子赫尔曼出生。

第一次世界大战期间,普朗克的第二个儿子欧文于1914年被法国人俘虏,而他的大儿子卡尔在凡尔登阵亡。格雷特在1917年生第一个孩子时去世。两年后,她的姐姐嫁给了格雷特的鳏夫,也以同样的方式去世了。两个孙女都幸存了下来,并以她们母亲的名字命名。普朗克坚忍地忍受了这些损失。

1945年1月,与他关系特别密切的欧文被纳粹人民法院判处死刑,因为他参与了1944年7月暗杀希特勒的未遂行动。欧文于1945年1月23日被处决。[16]

\subsubsection{1.3 柏林大学教授}
\begin{figure}[ht]
\centering
\includegraphics[width=6cm]{./figures/c3b865fe6023c01e.png}
\caption{柏林洪堡大学的牌匾:“马克斯普朗克,作为基本行动量子h 的发现者,1889年至1928年在这座建筑中授课。”} \label{fig_Max_3}
\end{figure}
作为柏林弗里德里希-威廉-大学的教授,普朗克加入了当地的物理学会。他后来写道:“在那些日子里,我基本上是那里唯一的理论物理学家,那里的事情对我来说并不那么容易,因为我开始提到熵,但这并不十分流行,因为它被视为数学幽灵”。[17]由于他的倡议,1898年德国各地方物理学会合并形成了德国物理学会(Deutsche Physikalische Gesellschaft,DPG);从1905年到1909年,普朗克担任主席。
普朗克开始了一个为期六个学期的理论物理讲座课程,根据莉泽·迈特纳的说法,“枯燥,有点客观”,不使用笔记,从不犯错,从不动摇;“我听过的最好的讲师”根据一位英国参与者詹姆斯·r·帕廷顿的说法,他继续说:“总是有许多人站在房间周围。由于教室很热,而且相当近,有些听众会不时地倒在地板上,但这并没有干扰讲座”。普朗克没有建立真正的“学校”;他的研究生只有20人左右,其中包括:

1897年马克斯·亚伯拉罕(1875-1922)

1903年马克斯·冯·劳厄(1879-1960)

1904年莫里茨·施利克(1882-1936)

1906年瓦尔特·迈斯纳(1882-1974)

1907 弗里茨·赖克特(1883-1960)

1912年沃尔特·肖特基(1886-1976)

1914年瓦尔特·博特(1891-1957)  [18]

\subsubsection{1.4 黑体辐射}
1894年,普朗克把注意力转向了黑体辐射问题。他受电力公司委托,用最少的能量从灯泡中产生最大的光。基尔霍夫在1859年提出了这个问题:“黑体(一种完美的吸收体,也称为空腔辐射体)发出的电磁辐射的强度如何取决于辐射的频率(即光的颜色)和身体的温度?”。这个问题已经通过实验进行了探索,但没有理论处理与实验值一致。威廉·维恩提出了维恩定律,它正确地预测了高频下的行为,但在低频下却失败了。解决这个问题的另一种方法是,在低频时,与实验结果一致,但在高频时,产生了后来被称为“紫外线灾难”的现象。然而,与许多教科书相反,这并不是普朗克的动机。[19]

普朗克于1899年提出了第一个解决这个问题的方案,这是从普朗克所谓的“基本无序原理”开始的,这使得他可以从许多关于理想振荡器熵的假设中推导出维恩定律,从而产生了所谓的维恩-普朗克定律。很快人们发现,令普朗克沮丧的是,实验证据根本没有证实新定律。普朗克修改了他的方法,推导出著名的普朗克黑体辐射定律的第一个版本,该定律很好地描述了实验观察到的黑体光谱。它最初是在1900年10月19日的一次会议上被提出的,并于1901年发表。第一个推导不包括能量量子化,也不使用统计力学,他对此持厌恶态度。1900年11月,普朗克修改了第一种方法,依靠玻尔兹曼对热力学第二定律的统计解释,获得了对辐射定律背后原理的更基本理解。由于普朗克对玻尔兹曼方法的这种解释的哲学和物理含义深感怀疑,他求助于这些解释,正如他后来所说的,“是一种绝望的行为...我准备牺牲我以前对物理的任何信念。”[19]

他的新推导于1900年12月14日提交给了DPG,其核心假设是电磁能量只能以量子化的形式发射,换句话说,能量只能是基本单位的倍数:

$$E=hν~$$

$h$是普朗克常数,也称为普朗克作用量 (于1899年引入),ν 是辐射的频率。注意这里讨论的基本能源单位由hν 共同作用而不仅仅是通过ν来体现。物理学家现在称这些量子光子和频率光子ν 会有自己独特的能量。该频率的总能量等于hν 乘以该频率下光子的数量。