% 等差与等比数列总结(初等数学)
% license Pub
% type Wiki

\pentry{等差数列(高中)\nref{nod_HsAmPg},等比数列(高中)\nref{nod_HsGmPg}}{nod_8e10}

\begin{table}[ht]
\centering
\caption{等差与等比数列}\label{tab_AGS1}
\begin{tabular}{|c|c|c|}
\hline
 & 等差数列 & 等比数列 \\
\hline
例子 & $$1,2,3,4,5,...~$$ & $$1,2,4,8,16,...~$$ \\
\hline
通项公式 ($n=1,2,3,...$) & $$a_n = a_1 + (n-1)d~$$ & $$a_n = a_1 q^{n-1}~$$ $$q\ne0~$$ \\
\hline
递推公式 & $$a_{n+1} = a_n + d~$$ & $$a_{n+1} = a_n \cdot q~$$ \\
\hline
项数& $$n = \frac{a_n-a_1}{d} + 1~$$ & $$n = \frac{\ln{\frac{a_n}{a_1}}}{\ln{q}} + 1 = \log_q \frac{a_n}{a_1}+1~$$ \\
\hline
前$n$项和$S_n=a_1+a_2+...+a_n$& $$S_n=\frac{(a_1+a_n)n}{2}~$$ & $$S_n = \frac{a_1 (1-q^n)}{1-q}~$$ $$(q\ne 1)~$$ \\
\hline
一个小结论 & $$p+q=m+r \Rightarrow a_p+a_q = a_m + a_r~$$ & $$p+q=m+r \Rightarrow a_p \cdot a_q = a_m \cdot a_r~$$ \\
\hline
\end{tabular}
\end{table}

本文中,$n$代表序号$n=1,2,3,...$,$a_n$代表数列中的第$n$项,$d$代表等差数列的公差,$q$代表等比数列的公比。

\begin{example}{}
求$\frac{1}{2}+\frac{1}{4}+\frac{1}{8}+...$

我们发现这是一个公比为$1/2$的等比数列。尽管有无穷个数字相加,但累和结果似乎是一个有限的数:
$$\lim_{n \to +\infty} S_n =  \lim_{n \to +\infty} \frac{a_1 (1-q^n)}{1-q}
= \lim_{n \to +\infty} \frac{\frac{1}{2} (1-(\frac{1}{2})^n)}{1-\frac{1}{2}}
=1 ~$$
\textsl{所谓“一尺之棰,日取其半,万世不竭”,反过来说,把万世中取得的这些棰加起来,也不过一尺。}

这个结论可以很方便地推广至所有$0<q<1$的等比数列,在高数中这个结论非常有用:
$$\lim_{n \to +\infty} S_n =  \frac{a_1}{1-q} \qquad 0<q<1~$$
然而另一个看起来十分相似的式子(调和级数)就没有这么幸运,\textsl{主流理论}认为他是发散的:
$$S = 1+\frac{1}{2}+\frac{1}{3}+\frac{1}{4}+...\to+\infty~$$
\end{example}
