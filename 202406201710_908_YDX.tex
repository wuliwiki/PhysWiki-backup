% 运动学
% license CCBYSA3
% type Wiki

(本文根据 CC-BY-SA 协议转载自原搜狗科学百科对英文维基百科的翻译)

\textbf{运动学}是经典力学的一个分支,描述了点、体(对象)和体系统(对象组)的运动,而不考虑引起运动的力。[1][2][3]运动学作为一个研究领域,通常被称为“运动的几何”,偶尔也被视为数学的一个分支。[4][5][6]运动学问题首先描述系统的几何形状,并声明系统内任何已知点的位置、速度和/或加速度值的初始条件。然后,使用几何参数,可以确定系统任何未知部分的位置、速度和加速度。对力如何作用于物体的研究属于动力学范畴,而不是运动学。有关更多详细信息,请参见分析动力学。

运动学在天体物理学中用于描述天体的运动和这些天体的集合。在机械工程、机器人学和生物力学中[7]运动学用于描述由连接部件(多连杆系统)组成的系统的运动,例如发动机、机械臂或人体骨骼。

几何变换,也称为刚性变换,用于描述机械系统中部件的运动,简化了运动方程的推导。它们也是动态分析的核心。

运动学分析是测量用于描述运动的运动学量的过程。例如,在工程中,运动学分析可用于找到给定机构的运动范围,并反向工作,使用运动学综合来设计所需运动范围的机构。[8] 此外,运动学将代数几何学应用于研究机械系统或机构的机械效益。

\subsection{词源}



\subsection{非旋转参照系中粒子轨迹的运动学}



\subsubsection{2.1 速度和速度}



\subsubsection{2.2 加速度}



\subsubsection{2.3 相对位置向量}



\subsubsection{2.4 相对速度}



\subsubsection{2.5 相对加速度}



\subsection{恒定加速度下的粒子轨迹}



\subsection{圆柱-极坐标中的粒子轨迹}



\subsubsection{4.1 恒定半径}



\subsubsection{4.2 平面圆形轨迹}



\subsection{物体在平面内运动的点轨迹}

通过在每个零件上附加一个参考坐标系并确定各个参考坐标系是如何相对移动的,来分析机械系统部件的运动。如果零件的结构刚度足够大,则可以忽略其变形,并使用刚性变换来定义这种相对运动。这就把对复杂机械系统各部分运动的描述,简化为描述各部分的几何结构和各部分相对于其它部分的几何关联的问题。

几何学是研究在空间以各种方式变换时保持不变的图形的性质——更严格地说,是研究在一组变换下的不变量[18]。这些变换可以导致三角形在平面上的位移,同时留下顶点的角度和顶点之间的距离不变。运动学通常被描述为应用几何学,其中机械系统的运动是用欧几里德几何学的刚性变换来描述的。

平面中点的坐标是$r&2(二维空间)中的二维向量。刚性变换是那些保持任意两点之间距离的变换。n-维空间中的刚性变换集称为特殊的欧几里德群rn,并表示为SE(n) 。

\subsubsection{5.1 位移和运动}



\subsubsection{5.2 矩阵表示}



\subsection{纯平移}



\subsection{ 物体围绕固定轴的旋转}



\subsection{物体三维运动的点轨迹}

运动学中的重要公式定义了运动物体在三维空间中跟踪轨迹时的速度和加速度。这对于一个物体的质心特别重要,它用牛顿第二定律或拉格朗日方程导出运动方程。

\subsubsection{8.1 位置}



\subsubsection{8.2 速度}



\subsubsection{8.3 加速度}



\subsection{运动约束}

运动约束是对机械系统部件运动的约束。运动约束可以被认为有两种基本形式:(i)由定义系统结构的铰链、滑块和凸轮关节产生的约束,称为完整约束;(ii)对系统速度施加的约束,如平面上溜冰鞋的刀口约束,或滚动而不滑动的圆盘或球体与平面接触,这被称为非完整约束。下面是一些常见的例子。

\subsubsection{9.1 运动耦合}

运动耦合精确地约束所有6个自由度。

\subsubsection{9.2 滚动而不打滑}



\subsubsection{9.3 不可拉伸的绳索}

这种情况下,物体由一根理想化的绳索连接,该绳索保持张紧状态,并且不能改变长度。约束是绳索所有段的长度之和是总长度,因此该和的时间导数为零。[21][22][23]这种类型的动力学问题是摆。另一个例子是,一个鼓在重力的牵引下,通过不可拉伸的绳索将落在轮辋上的重物旋转。[24]这种类型的平衡问题(即非运动学问题)是悬链线。[25]

\subsubsection{9.4 运动副}

勒洛称形成机器运动对的组件之间的理想连接为运动副。他区分了据说在两个连杆之间具有线接触的较高对和在连杆之间具有面积接触的较低对。菲利普斯表明,有许多方法来构造不适合这种简单分类的对。[26]

\textbf{较低副}

较低对是理想的关节或完整约束,它保持移动实体(三维)中的点、线或平面与固定实体中相应的点画线或平面之间的接触。有以下几种情况:

\begin{itemize}
\item 转动副或铰接接头需要移动体中的线或轴保持与固定体中的线共线,并且移动体中垂直于该线的平面保持与固定体中类似垂直平面的接触。这给连杆的相对运动施加了五个约束,因此连杆具有一个自由度,即围绕铰链轴的纯旋转。
\item 棱柱形接头或滑块要求移动体中的线或轴保持与固定体中的线共线,并且移动体中平行于该线的平面保持与固定体中类似平行平面的接触。这对连杆的相对运动施加了五个限制,因此连杆具有一个自由度。这个自由度是滑块沿线的距离。
\item 圆柱形接头要求移动体中的线或轴保持与固定体中的线共线。它是转动关节和滑动关节的组合。这个关节有两个自由度。移动物体的位置由围绕轴的旋转和沿着轴的滑动来定义。
\item 球形接头要求移动体中的一个点与固定体中的一个点保持接触。这个关节有三个自由度。
\item 平面接头要求移动体中的平面与固定体中的平面保持接触。这个关节有三个自由度。
\end{itemize}


\textbf{较高副}

一般来说,较高的副是一个约束,要求移动体中的曲线或表面保持与固定体中的曲线或表面的接触。例如,凸轮与其从动件之间的接触是一个更高的副,称为凸轮关节。同样,构成两个齿轮啮合齿的渐开线曲线之间的接触也是凸轮关节。

\subsubsection{9.5 运动链}

\begin{figure}[ht]
\centering
\includegraphics[width=6cm]{./figures/bfa25192f796f49a.png}
\caption{《机械运动学》,1876年,四连杆机构示意图} \label{fig_YDX_1}
\end{figure}

由运动副(“关节”)连接的刚体(“连杆”)称为运动链。机构和机器人是运动链的例子。通过使用移动公式计算关节的数目和关节的数目和类型来确定运动链的自由度。这个公式也可以用来枚举具有给定自由度的运动链的拓扑结构,这被称为机械设计中的类型综合。

\textbf{实例}

由N个连杆和J铰接或滑动关节组装的平面一自由度连杆机构为:

\begin{itemize}
\item N=2,j=1 :作为杠杆的两杆联动装置;
\item N=4,j=4 :四连杆机构;
\item N=6,j=7 :六连杆机构。这必须有两个链接(“三元链接”)来支持三个关节。有两种不同的拓扑结构,这取决于两个三元连杆是如何连接的。在瓦特拓扑中,这两个三元链路有一个公共连接;在斯蒂芬森拓扑中,这两个三元链路没有公共连接,并且通过二元链路连接。[27]
\item N=8,j=10 :具有16种不同拓扑结构的八连杆机构;
\item N=10,j=13 :具有230种不同拓扑结构的十杆连杆机构;
\item N=12,j=16 :具有6,856种拓扑结构的十二连杆机构。
\end{itemize}

对于较大的链及其连杆拓扑,请参见R. P. Sunkari和L. C. Schmidt,“采用麦凯型算法进行平面运动链的结构综合”,机制和机器理论#41,1021–1030页(2006)。

\subsection{参考文献}

[1]
^Edmund Taylor Whittaker (1904). A Treatise on the Analytical Dynamics of Particles and Rigid Bodies. Cambridge University Press. Chapter 1. ISBN 0-521-35883-3..

[2]
^Joseph Stiles Beggs (1983). Kinematics. Taylor & Francis. p. 1. ISBN 0-89116-355-7..

[3]
^Thomas Wallace Wright (1896). Elements of Mechanics Including Kinematics, Kinetics and Statics. E and FN Spon. Chapter 1..

[4]
^Russell C. Hibbeler (2009). "Kinematics and kinetics of a particle". Engineering Mechanics: Dynamics (12th ed.). Prentice Hall. p. 298. ISBN 0-13-607791-9..

[5]
^Ahmed A. Shabana (2003). "Reference kinematics". Dynamics of Multibody Systems (2nd ed.). Cambridge University Press. ISBN 978-0-521-54411-5..

[6]
^P. P. Teodorescu (2007). "Kinematics". Mechanical Systems, Classical Models: Particle Mechanics. Springer. p. 287. ISBN 1-4020-5441-6.。.

[7]
^A. Biewener (2003). Animal Locomotion. Oxford University Press. ISBN 019850022X..

[8]
^J.M. McCarthy和G. S. Soh,2010年,连杆的几何设计,纽约斯普林格。.

[9]
^Ampère, André-Marie. Essai sur la Philosophie des Sciences. Chez Bachelier..

[10]
^Merz, John (1903). A History of European Thought in the Nineteenth Century. Blackwood, London. p. 5..

[11]
^O. Bottema & B. Roth (1990). Theoretical Kinematics. Dover Publications. preface, p. 5. ISBN 0-486-66346-9..

[12]
^Harper, Douglas. "cinema". Online Etymology Dictionary..

[13]
^https://web.archive.org/web/20221025164404/https://www.youtube.com/watch?v=jLJLXka2wEM 速成物理.

[14]
^https://web.archive.org/web/20221025164404/https://www.youtube.com/watch?v=jLJLXka2wEM 速成物理积分.

[15]
^https://web.archive.org/web/20221025164404/https://duckduckgo.com/?q =面积+的+a+矩形& ampatb=v92-4_g&amp。ia DuckDuckGo.

[16]
^https://web.archive.org/web/20221025164404/http://www . mathsisfun . com/代数/trig-area-triangle-with-not-right-angle . html没有直角的三角形面积.

[17]
^https://web.archive.org/web/20221025164404/https://www 4 . uwsp . edu/phys tar/kmen ning/phys 203/eqs/运动学. gif.

[18]
^几何学:对给定元素在特定变换下保持不变的性质的研究。"Definition of geometry". Merriam-Webster on-line dictionary..

[19]
^Paul, Richard (1981). Robot manipulators: mathematics, programming, and control : the computer control of robot manipulators. MIT Press, Cambridge, MA. ISBN 978-0-262-16082-7..

[20]
^R. Douglas Gregory (2006). Chapter 16. Cambridge, England: Cambridge University. ISBN 0-521-82678-0..

[21]
^William Thomson Kelvin & Peter Guthrie Tait (1894). Elements of Natural Philosophy. Cambridge University Press. p. 4. ISBN 1-57392-984-0..

[22]
^William Thomson Kelvin & Peter Guthrie Tait (1894). Elements of Natural Philosophy. p. 296..

[23]
^M. Fogiel (1980). "Problem 17-11". The Mechanics Problem Solver. Research & Education Association. p. 613. ISBN 0-87891-519-2..

[24]
^Irving Porter Church (1908). Mechanics of Engineering. Wiley. p. 111. ISBN 1-110-36527-6..

[25]
^Morris Kline (1990). Mathematical Thought from Ancient to Modern Times. Oxford University Press. p. 472. ISBN 0-19-506136-5..

[26]
^Phillips, Jack (2007). Freedom in Machinery, Volumes 1–2 (reprint ed.). Cambridge University Press. ISBN 978-0-521-67331-0..

[27]
^Tsai, Lung-Wen (2001). Mechanism design:enumeration of kinematic structures according to function (llustrated ed.). CRC Press. p. 121. ISBN 978-0-8493-0901-4..