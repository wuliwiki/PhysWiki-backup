% 角动量
% keys 角动量

\begin{issues}
\issueTODO
\end{issues}

在讨论物体的运动时,我们用动量来描述机械运动的状态,并讨论了在机械运动的转移过程中所遵循的动量守恒定律.同样,在讨论物体围绕某一点的运动时,我们也可以用角动量来描述物体的运动状态.
\subsection{角动量}
我们在研究物体的运动中经常会遇到物体围绕一定中心的转动的情况.例如,地球围绕太阳的公转、卫星绕地球的运转、原子中的电子围绕着原子核运转等等.为了方便起见,我们以质量为$m$作圆周运动的质点为例,来引入角动量的概念.

设圆的半径是$r$,则质点对圆心的位矢$\bvec r$的量值便是$r$,质点的速度是$v$,方向沿着圆的切线方向.从图1可以看出,质点的动量$\bvec p = m\bvec v$处处和它的位矢$\bvec r$相垂直.我们把质点动量$\bvec p$的量值$p$和位矢$\bvec r$的量值$r$的乘积定义为作圆周运动的质点对圆心$O$的角动量的量值,用$\bvec L$表示.
\begin{equation}
\bvec L = \bvec p r
\end{equation}
\begin{figure}[ht]
\centering
\includegraphics[width=6cm]{./figures/AngMo_1.pdf}
\caption{质点对圆心的角动量} \label{AngMo_fig1}
\end{figure}
\subsection{角动量守恒}
我们已经知道 一个物体不受外力或所受外力之和为零,这个物体的总动量保持不变,这个结论叫做动量守恒定律.同样的,在做圆周运动的物体也有自己的守恒定律,即角动量守恒定律.

角动量守恒定律是物理学的另一基本规律,在研究天体运动和微观粒子运动时,角动量守恒定律都起着重要作用.我们通过下面的例子来讨论角动量守恒满足的条件.

如图2所示,把一个质量为$m$的小球系在轻绳的一端,细绳穿过一竖直的管子;先使小球以速度$v_1$在水平面沿半径为$r_1$的圆做圆周运动,然后向下拉绳,使小球的半径减小到$r_2$.实验发现,这时小球的速度$v_2$就会增大.
\begin{figure}[ht]
\centering
\includegraphics[width=8cm]{./figures/AngMo_2.pdf}
\caption{角动量守恒演示} \label{AngMo_fig2}
\end{figure}

实验发现$v_1$和$v_2$之间存在下列关系:
\begin{equation}
v_1r_1=v_2r_2
\end{equation}

用小球的质量乘上式两边,得:
\begin{equation}
mv_1r_1=mv_2r_2
\end{equation}

即在半径改变的过程中小球对圆心$O$的角动量保持不变.在这个例子中,小球的动量是时时刻刻在改变的,但小球的角动量却能保持不变.因此在研究物体的转动时,角动量将代替动量起重要的作用.

在转动运动中,我们定义力的作用点相对于给定点的位矢$r$与力$F$的矢量积为力对给定点的力矩,以$M$表示.在转动的研究中,力矩是个重要的概念.虽然力矩和功都是长度和力的乘积,力矩是二者的矢积,本身是个矢量;而功却是二者的标积,本身是个标量.力矩和功的物理意义并不相同.力矩的单位采用$N\cdot m$(牛顿米).如果作用在质点上的外力对某给定点$O$的力矩为零,则质点对$O$的角动量在运动过程中保持不变,这就叫做质点的角动量守恒定律.

\begin{example}{}
滑冰运动员在冰上旋转时,为什么收起胳膊转的快,张开胳膊转的慢?

根据角动量守恒原理$L=mvr$,张开手脚的时候旋转半径变大,即$r$变大,在$L,m$不变的情况下旋转速度$v$会变小,运动员也就转的慢了;同理,收起胳膊的时候旋转半径$r$变小,所以旋转速度$v$变大运动员也就转的快了.
\end{example}
\begin{example}{}
为什么角动量是矢量?

在物理学中,有方向有大小的物理量叫做矢量,如:力、速度、加速度等.有大小没有方向的物理量叫做标量,如:质量、能量、密度、功等.
角动量和动量一样都是与物体运动有关的物理量,都与物体的运动方向有关,运动的方向的正负因而角动量除了大小之外也有方向,所以角动量也是矢量.
\end{example}

\addTODO{角动量性质、转动惯量、添加例子}
