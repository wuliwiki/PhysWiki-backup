% 竖直面内的圆周运动(高中)
% keys 高中物理|圆周运动
% license Usr
% type Tutor

\pentry{圆周运动\nref{nod_HSPM05}}{nod_HSPM04}

\subsection{引言}
本文主要介绍竖直平面内变速圆周运动的临界问题。这类问题常有如下设问:“如图所示........若物块不脱离轨道,求轨道半径R的取值范围”。这个问题也是高中物理的一个重要模型,常作为综合性大题的一个组成部分,与动能定理、动量定理、带电粒子运动、电磁感应等知识共同用作命题。

\subsection{能量观点的引入}
常见的竖直面内圆周运动模型有绳模型、杆模型、轨道模型等,无论什么模型,竖直面内受重力作用的物体总是做变速圆周运动,我们无法写出一个一劳永逸的描述物体运动特征的函数解析式。想要解决与之有关的临界问题,至少在高中范围内(截止2024年的普通高中物理教科书),不用能量观点,我们无从下手。

\subsection{可能运动1-往复运动}
现在考察一个具体物理情境与实例。
\begin{figure}[ht]
\centering
\includegraphics[width=6cm]{./figures/1cdd7a99f5c6bf3a.png}
\caption{轻绳模型} \label{fig_CirVer_1}
\end{figure}
如图1所示,一质量为$m$的小球从半径为$R$的固定光滑圆轨道的最低点$A$点,以速度$\bvec v_0$向右运动,忽略空气阻力,若它运动过程中不脱离轨道,试分析(大致想象)小球的运动情景。

首先可以设想,当时间开始流动,小球迅速开始向右运动。若无绳子束缚,它将向右水平飞出,也即沿圆形轨迹点A处切线飞出。由于绳的拉力,小球会有向心运动趋势,对其进行受力分析,则重力与绳拉力的合力指向圆心。在一个微小的时间微元后,小球正好行走了一小段圆弧。当时间t足够大时,小球将会达到圆形轨迹右下方四分之一圆弧的某处。

这时我们立即注意到,在此过程中机械能守恒且重力做功,小球做变速圆周运动,合加速度指向屏幕左侧,速率逐渐减小。

于是我们自然想到,存在一个速度临界值$\bvec v_1$,使得小于这个临界值的所有初速度$v_0$都会让小球无法通过与圆心等高处,只能以类似于单摆的方式往复运动。

此时小球无论如何也无法脱离轨道,因为此过程中重力的向心分量使小球有紧压轨道的趋势。
现在我们尝试求出这个临界值$\bvec v_1$。
