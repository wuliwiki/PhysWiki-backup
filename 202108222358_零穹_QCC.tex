% 量类的延拓
% 量类|延拓|矢量空间
\pentry{量类和单位\upref{QCU}}
在物理中,不少量类(这里指用单位测量量类中的量所得的数值)在通常情况下或不能取负值,或不连续,或只在某一范围取值.比如质量、面积、体积、密度、电容等不会取负值;电荷取值具有量子化性质;引力常量 $\boldsymbol{G}$ 只有一个等等.为建立一套完整的量纲理论,我们需要对量类进行延拓.在此之前,我们先给出一些理由,以便理解.

某些量类对于一些数值是“没有物理意义”的,比如质量非负,然而,“没有物理意义”的说法其实是非常含混的.比如对速率量类,在相对论里其数值不能超过光速,但若限于牛顿力学,任何速率都是允许的.再如温度,我们知道绝对零度不可达到,但总应将绝对零度视作温度量类 $\tidle{\boldsymbol{T}}$ 的元素,否则“绝对零度不可达到”将意义不明.绝对零度下的温度虽然“没有物理意义”,但不妨纳入温度量类 $\tidle{\boldsymbol{T}}$ 这个集合中,只需将其理解成在具体的问题中不出现而已.现在,“温度量类 $\tidle{\boldsymbol{T}}$” 指由开尔文(作为单位)的任意实数倍组成的集合.同样,对于量子化的量类,只需将其取值的性质改为可以连续,那些“没有物理意义”的值只需理解为在具体问题中不会出现或观察不到.更如,对于引力常量量类 $\tilde{\boldsymbol{G}}$ ,人们过去一直认为其是常量,但现在,越来越多的人相信,在宇宙演化的历史长河中,引力常量 $\boldsymbol{G}$ 是在非常缓慢地改变着的(由狄拉克于1973年率先提出).这一“常量不常”的现象zai