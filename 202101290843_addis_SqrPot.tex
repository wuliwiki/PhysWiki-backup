% 方势垒的定态波函数
% 势垒|散射|波函数|薛定谔方程

\begin{issues}
\issueDraft
\end{issues}

% 未完成: 完全从 docx 转换
% 实数解已经画图验证, 复数解还没有.

\pentry{无限深势阱\upref{ISW}, 原子单位制\upref{AU}}

为方便书写, 本文采用原子单位制\upref{AU}. 我们要解一维定态薛定谔方程(参考\autoref{AU_eq4}~\upref{AU})为
\begin{equation}
-\frac{1}{2m} \dv[2]{\psi}{x} + V\psi= E \psi
\end{equation}
令方势垒长度为 $2l$, 关于原点对称. 势能函数为
\begin{equation}
V(x) =
\begin{cases}
0 & (x < -l)\\
V_0 & (-l \leqslant x < l)\\
0 & (l \leqslant x)
\end{cases}
\end{equation}
一些教材把势垒放在区间 $[0, a]$, 这时只需令 $a = 2l$, 把解出的波函数做一个平移即可. 对称势能的好处是存在对称和反对称的实数解, 且它们自动正交. 如果开始时就用指数形式的波函数, 完成后还要进行施密特正交化, 归一化后两个解的形式也未必对称.

\subsection{$E \ge V_0$ 的情况}
令
\begin{equation}
k = \sqrt{2mE}
\qquad
b = \sqrt{2m(E-V_0)}
\end{equation}
令对称解和反对称的波函数分别为
\begin{equation}
\psi_{\bvec k}^e(x) =
\begin{cases}
A_1 \cos(bx) & (0 \leqslant x \leqslant l)\\
C_1 \cos(kx) + D_1 \sin(kx) & (l < x)\\
\psi_{\bvec k}^e(-x) & (x < 0)
\end{cases}
\end{equation}

\begin{equation}
\psi_{\bvec k}^o(x) =
\begin{cases}
B_2 \sin(bx) & (0 \leqslant x \leqslant l)\\
C_2 \cos(kx) + D_2 \sin(kx) & (l < x)\\
-\psi_{\bvec k}^o(-x) & (x < 0)
\end{cases}
\end{equation}
其中 $l < x$ 的部分也可以分别记为
\begin{equation}
G_i \sin(kx + \phi_i) \qquad (i = 1,2)
\end{equation}
\begin{equation}
G_i = \sqrt{C_i^2 + D_i^2}
\qquad
\phi_i = \Arctan(C_i, D_i)
\qquad
(i = 1,2)
\end{equation}


在 $x = l$ 处匹配波函数和一阶导数, 解得
\begin{equation}\label{SqrPot_eq1}
\leftgroup{
\frac{C_1}{A_1} &= \cos(kl)\cos(bl) + \frac{b}{k} \sin(kl)\sin(bl) \\
\frac{D_1}{A_1} &= -\frac{b}{k} \cos(kl) \sin(bl) + \sin(kl)\cos(bl)
}
\end{equation}

\begin{equation}\label{SqrPot_eq2}
\leftgroup{
\frac{C_2}{B_2} &= \cos(kl)\sin(bl) - \frac{b}{k} \sin(kl)\cos(bl)\\
\frac{D_2}{B_2} &= \frac{b}{k} \cos(kl) \cos(bl) + \sin(kl)\sin(bl)
}
\end{equation}
对波函数归一化, 使 $\int \psi_k'(x)\psi_k(x) = \delta(k'-k)$, 得
\begin{equation}
\begin{aligned}
A_1 = \frac{1}{\sqrt{\pi} \sqrt{\cos[2](bl) + b^2\sin[2](bl)/k^2}}\\
B_2 = \frac{1}{\sqrt{\pi} \sqrt{\sin[2](bl) + b^2\cos[2](bl)/k^2}}
\end{aligned}
\end{equation}

\subsection{$0 < E < V_0$ 的情况}
令 $\sqrt{2m(V_0-E)}$
\begin{equation}
\psi_{\bvec k}^e(x) =
\begin{cases}
A_1 \cosh(\kappa x) & (0 \leqslant x \leqslant l)\\
C_1 \cos(kx) + D_1 \sin(kx) & (l < x)
\end{cases}
\end{equation}
\begin{equation}
\psi_{\bvec k}^o(x) =
\begin{cases}
A_2 \sinh(\kappa x) & (0 \leqslant x \leqslant l)\\
C_2 \cos(kx) + D_2 \sin(kx) & (l < x)
\end{cases}
\end{equation}
在 $x = l$ 处匹配波函数和一阶导数, 分别得
\begin{equation}
\leftgroup{
&A_1 \cosh(\kappa l) = C_1 \cos(kl) + D_1 \sin(kl)\\
&\kappa A_1 \sinh(\kappa l) = - kC_1  \sin(kl) +  kD_1 \cos(kl)
}\end{equation}
\begin{equation}
\leftgroup{
&A_2 \sinh(\kappa l) = C_2 \cos(kl) + D_2 \sin(kl)\\
&\kappa A_2 \cosh(\kappa l) = -kC_2 \sin(kl) + k D_2 \cos(kl)
}\end{equation}
解得
\begin{equation}
\leftgroup{
&\frac{C_1}{A_1} = \cosh(\kappa l)\cos(kl) - \frac{\kappa}{k}\sinh(\kappa l) \sin(kl)\\
&\frac{D_1}{A_1} = \cosh(\kappa l)\sin(kl) + \frac{\kappa}{k}\sinh(\kappa l) \cos(kl)
}\end{equation}
\begin{equation}
\leftgroup{
&\frac{C_2}{A_2} = \sinh(\kappa l)\cos(kl) - \frac{\kappa}{k}\cosh(\kappa l)\sin(kl)\\
&\frac{D_2}{A_2} = \sinh(\kappa l)\sin(kl) + \frac{\kappa}{k} \cosh(\kappa l)\cos(kl)
}\end{equation}
对波函数归一化, 使 $\int \psi_k'(x)\psi_k(x) = \delta(k'-k)$, 得
\begin{equation}
A_1 = \frac{1}{\sqrt{\pi}\sqrt{\cosh^2(\kappa l) + (\kappa/k)^2 \sinh^2(kl)}}
\end{equation}
\begin{equation}
A_2 = \frac{1}{\sqrt{\pi}\sqrt{\sinh^2(\kappa l) + (\kappa/k)^2 \cosh^2(kl)}}
\end{equation}

\subsection{行波边界条件}
令势阱范围为 $[0, a]$, 左、中、右三个区间 $\exp(\pm kx)$ 的系数分别为 $A,B$、$C,D$、$E,F$. 左边右入射和出射, 右边只有出射没有反射 $F = 0$.

当 $E > V_0$ 时, 系数解为
\begin{equation}
\leftgroup{
B = \frac{(k^2 - l^2)\sin(la)}{(k^2 + l^2)\sin(la) + 2\I kl\cos(la)} A\\
C = \frac{\I k (k + l) \exp(-\I la)}{(k^2 + l^2)\sin(la) + 2\I kl \cos(la)} A\\
D = \frac{\I k(l-k) \exp(\I la)}{(k^2 + l^2)\sin(la) + 2\I kl \cos(la)} A\\
F = \frac{2\I kl \exp(-\I ka)}{(k^2 + l^2)\sin(la) + 2\I kl\cos(la)} A
}\end{equation}

透射率, 反射率分别为
\begin{equation}
T = \abs{\frac{E}{A}}^2 = \frac{4k^2l^2}{(k^2 + l^2) \sin[2](la) + 4k^2l^2\cos[2](la)}
\end{equation}

\begin{equation}
R = \abs{\frac{B}{A}}^2 = \frac{(k^2 - l^2) \sin[2](la)}{(k^2 + l^2) \sin[2](la) + 4k^2l^2\cos[2](la)}
\end{equation}

当 $0 < E < V_0$ 时, 系数解为
\begin{equation}
\leftgroup{
B = \frac{(l^2 + k^2)\sinh(la)}{(k^2 - l^2) \sinh(la) + 2\I kl \cosh(la)} A\\
C = \frac{k(\I l - k) \exp(-la)}{(k^2 - l^2) \sinh(la) + 2\I kl \cosh(la)} A\\
D = \frac{k(\I l + k) \exp(la)}{(k^2 - l^2) \sinh(la) + 2\I kl \cosh(la)} A\\
F = \frac{2\I kl \exp(-\I ka)}{(k^2 - l^2) \sinh(la) + 2\I kl \cosh(la)} A
}\end{equation}

\begin{equation}
T = \abs{\frac{E}{A}}^2 = \frac{4k^2l^2}{(l^2 - k^2) \sinh[2](la) + 4k^2l^2\cosh[2](la)}
\end{equation}

\begin{equation}
R = \abs{\frac{B}{A}}^2 = \frac{(l^2 + k^2) \sinh[2](la)}{(l^2 - k^2) \sinh[2](la) + 4k^2l^2\cosh[2](la)}
\end{equation}


