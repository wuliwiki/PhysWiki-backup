% 电路(高中)
% keys 电路|电源|电动势|内阻

\begin{issues}
\issueDraft
\issueTODO
\end{issues}

\pentry{恒定电流\upref{HSPE03}}

\subsection{常见元器件}

\textbf{电阻器}:有定值和变值两种类型.顾名思义,定值电阻器的电阻是固定的,而变值电阻器接入电路的电阻是可被改变的,如滑动变阻器、电阻箱等.

\textbf{电容器}:具有“通交流隔直流”的作用,也可以作为储能元件使用.

\textbf{电感器}:具有“通直流阻交流”的作用,也可以作为储能元件使用.

\textbf{二极管}:具有单向导电性,属于非线性元件.

\textbf{导线}:连接电源和其他元器件,起输送电能的作用,一般计算时忽略导线的电阻.

\textbf{开关}:控制电路的通断,连接时应断开,闭合后相当于导线.

利用导线,按实际需求将电源和其他元器件连接起来组成的闭合回路,叫做\textbf{电路}.

\subsection{电源}

通过非静电力做功将其他形式能转为电能的装置叫做\textbf{电源},常见的电源有干电池、蓄电池、锂电池、发电机等.

前面提及到,形成电流的条件包括:导体两端存在电势差和导体中存在自由电荷.若有带正电的$A$球和带负电的$B$球,用一根导线将它们相连,那么两球间存在电势差$U_{AB}$,$B$球中的电子会流向$A$球(产生电流),电中和后,$A$、$B$两球间没有电势差,不再形成电流.

为了在电路中形成持续的电流,就需要电源提供非静电力“搬运”自由电荷,保持导体两端的电势差.

\subsubsection{电动势}

电源用非静电力所做的功$W$与被移动的电荷量$q$之比,叫做电动势,用$E$表示:

\begin{equation}
E=\frac{W}{q}
\end{equation}

电动势的单位与电势、电势差的单位相同,都是伏特($\mathrm{V}$).

电动势是描述电源用非静电力做功本领大小的物理量,电源的电动势越大,将其他形式能转化为电能的本领越强.

电路断开时,电源的电动势和电源两极的电压值在数值上相等.

\subsubsection{内阻}