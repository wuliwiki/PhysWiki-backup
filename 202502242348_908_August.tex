% 奥古斯丁-路易·柯西(综述)
% license CCBYSA3
% type Wiki

本文根据 CC-BY-SA 协议转载翻译自维基百科\href{https://en.wikipedia.org/wiki/Pierre-Simon_Laplace}{相关文章}。

\begin{figure}[ht]
\centering
\includegraphics[width=6cm]{./figures/e9138cc29ee186d2.png}
\caption{大约在1840年的柯西。由Zéphirin Belliard根据Jean Roller的画作所绘制的石版画。} \label{fig_August_1}
\end{figure}
奥古斯丁-路易·柯西男爵(英国发音:/ˈkoʊʃi/ KOH-shee,/ˈkaʊʃi / KOW-shee;美国发音:/koʊˈʃiː / koh-SHEE;法语:[oɡystɛ̃ lwi koʃi];1789年8月21日 – 1857年5月23日)是法国数学家、工程师和物理学家。他是最早严格阐述并证明微积分核心定理的人之一(由此创建了实分析),开创了复分析领域,并研究了抽象代数中的置换群。柯西还在数学物理的多个领域做出了贡献,特别是在连续介质力学方面。

作为一位深刻的数学家,柯西对他的同时代人和后继者产生了深远的影响;汉斯·弗罗伊登塔尔曾说:

“比任何其他数学家都更多的概念和定理是以柯西命名的(仅在弹性学领域,就有十六个概念和定理是以柯西命名的)。”

柯西是一个多产的学者;他撰写了大约八百篇研究文章和五本关于数学及数学物理领域的完整教材。
\subsection{传记}  
\subsubsection{青年时期与教育}  
柯西是路易·弗朗索瓦·柯西(1760–1848)和玛丽-玛德琳·德塞斯特的儿子。柯西有两个兄弟:亚历山大·劳伦特·柯西(1792–1857),他于1847年成为上诉法院某部门的主席,并于1849年成为法国最高法院的法官;尤金·弗朗索瓦·柯西(1802–1877),一位公共知识分子,也撰写了几部数学著作。从小,柯西就擅长数学。

柯西于1818年与阿洛伊丝·德·比尔结婚。她是出版商的亲戚,该出版商出版了柯西的大部分著作。他们有两个女儿:玛丽·弗朗索瓦·阿丽西亚(1819年)和玛丽·马蒂尔德(1823年)。

柯西的父亲是旧 régime 巴黎警察部门的高级官员,但由于法国大革命(1789年7月14日)爆发,他失去了这一职位,而这场革命爆发时,奥古斯丁-路易·柯西正好出生在一个月前。[a] 柯西一家在1793至1794年的恐怖统治期间幸存下来,他们逃到了阿尔居伊尔,在那里柯西从父亲那里接受了初步教育。[6] 1794年罗伯斯比尔被处决后,柯西一家得以安全返回巴黎。在那里,路易-弗朗索瓦·柯西在1800年找到了一份官僚工作,[7] 并迅速晋升。拿破仑在1799年上台后,路易-弗朗索瓦·柯西得到了进一步的晋升,成为参议院的秘书长,直接在拉普拉斯(现因其在数学物理方面的贡献而更为人知)手下工作。数学家拉格朗日也是柯西一家的朋友。[4]

在拉格朗日的建议下,奥古斯丁-路易·柯西于1802年秋季入学了当时巴黎最好的中学——巴黎中央理工学校(École Centrale du Panthéon)。[6] 该校大部分课程是经典语言;雄心勃勃的柯西是一名出色的学生,在拉丁语和人文学科上赢得了许多奖项。尽管取得了这些成功,柯西还是选择了工程师的职业,并为进入巴黎高等工艺学院(École Polytechnique)的入学考试做准备。

1805年,他在293名考生中名列第二,顺利被录取。[6] 这所学校的主要目的是为未来的 civil 和军事工程师提供高水平的科学和数学教育。学校实行军事纪律,这使得柯西在适应上遇到了一些问题。尽管如此,他还是在18岁时于1807年完成了课程,并进入了桥梁与道路学校(École des Ponts et Chaussées)。他以最高荣誉获得土木工程学位。
\subsubsection{工程生涯}  
1810年毕业后,柯西接受了在瑟堡担任初级工程师的工作,那里拿破仑计划建设一个海军基地。柯西在这里待了三年,负责了乌尔克运河工程和圣克劳德桥工程,并在瑟堡港工作。[6] 尽管他的工作非常繁忙,他还是找到了时间准备三篇数学稿件,并将其提交给法国科学院的第一学部。[b] 柯西的前两篇稿件(关于多面体)被接受;第三篇稿件(关于圆锥曲线的导线)被拒绝。

1812年9月,23岁的柯西因过度劳累而生病,返回巴黎。[6] 他回到首都的另一个原因是他对工程工作的兴趣逐渐减退,越来越被数学的抽象美吸引;在巴黎,他有更好的机会找到与数学相关的职位。1813年,当他的健康有所好转时,柯西决定不再返回瑟堡。[6] 尽管他正式保留了工程师的职位,但他被调离了海军部的工资单,转到了内政部。在接下来的三年里,柯西主要处于无薪病假状态;他充分利用这段时间,专心研究数学(涉及对称函数、对称群和高阶代数方程的理论等相关课题)。他曾三次尝试进入法国科学院第一学部,但在1813年至1815年期间三次未能成功。1815年,拿破仑在滑铁卢战败,路易十八国王恢复了王位。法国科学院于1816年3月重新成立;拉扎尔·卡诺和贾斯帕·孟热因政治原因被移除,国王任命柯西取代其中一位的席位。柯西同行的反应非常激烈;他们认为柯西被接纳为科学院成员是一种侮辱,柯西因此在科学界结下了许多敌人。
\subsubsection{巴黎高等工艺学院教授}  
1815年11月,路易·波安索(Louis Poinsot),当时是巴黎高等工艺学院的一名副教授,因健康原因请求免除教学工作。此时,柯西已是冉冉升起的数学新星。他当时的一大成就是证明了费马的多边形数定理。他辞去了工程师的工作,获得了为巴黎高等工艺学院二年级学生教授数学的一年合同。1816年,这所非宗教的拿破仑派学校经过重组,几位自由派教授被解雇;柯西因此被晋升为正教授。

当柯西28岁时,他仍与父母同住。他的父亲认为是时候让儿子结婚了,便为他找到了一个合适的妻子——比他小五岁的阿洛伊丝·德·比尔。德·比尔家族是印刷商和书商,出版了柯西的大部分著作。[8] 阿洛伊丝与奥古斯丁于1818年4月4日在圣叙尔皮斯教堂举行了盛大的天主教婚礼。1819年,这对夫妇的第一个女儿玛丽·弗朗索瓦·阿丽西亚出生,1823年,第二个也是最后一个女儿玛丽·马蒂尔德出生。[9]

直到1830年,保守的政治气候非常适合柯西。在1824年,路易十八去世,由他更加保守的兄弟查理十世继位。在这些年里,柯西高产,接连发表了一部又一部重要的数学著作。他还在法国高等师范学院和巴黎科学院担任交叉任职。[fr]
\subsubsection{流亡}  
1830年7月,七月革命在法国爆发。查理十世逃离了国家,路易-菲利普继位。发生了暴乱,巴黎高等工艺学院的学生们积极参与其中,暴乱的发生地点离柯西的家很近。

这些事件标志着柯西生活的转折点,也使他的数学创作出现了停顿。由于政府的垮台以及对掌权的自由派的深深憎恨,柯西离开了法国,前往国外,留下了他的家人。[10] 他在瑞士弗里堡待了一段时间,在那里他必须决定是否向新政权宣誓效忠。他拒绝了这一要求,因此失去了在巴黎的所有职位,除了他的科学院成员身份,因为这个职位不需要宣誓。1831年,柯西前往意大利的都灵,在那里待了一段时间后,他接受了撒丁王国国王的邀请(撒丁王国统治都灵及周边的皮埃蒙特地区),为他专门设立了一个理论物理学的职位。他在都灵教书,任教于1832年至1833年。1831年,他被选为瑞典皇家科学院外籍成员,第二年被选为美国艺术与科学学院的外籍荣誉会员。[11]

1833年8月,柯西离开都灵前往布拉格,成为十三岁的波尔多公爵亨利·达尔图瓦(1820–1883)的科学导师。亨利·达尔图瓦是被流放的查理十世的孙子和王储。[12] 作为巴黎高等工艺学院的教授,柯西曾是一个臭名昭著的糟糕讲师,他假设学生的理解能力只有他最优秀的几个学生才能达到,并且在有限的时间内塞满了太多内容。亨利·达尔图瓦既没有兴趣也没有才能从事数学或科学。尽管柯西非常认真地对待这项任务,但他做得非常笨拙,并且对亨利·达尔图瓦缺乏令人惊讶的权威。在他担任土木工程师时,柯西曾短暂地负责过修理一些巴黎下水道,他犯了一个错误,提到这段经历给他的学生听;亨利·达尔图瓦怀着极大的恶意,开始说柯西是在巴黎的下水道里开始他的职业生涯。柯西作为导师的角色持续到亨利·达尔图瓦18岁,即1838年9月。[10] 在这五年里,柯西几乎没有做任何研究,而亨利·达尔图瓦则对数学产生了终生的厌恶。柯西被封为男爵,这个头衔对他来说非常重要。

1834年,他的妻子和两个女儿搬到了布拉格,柯西在流亡四年后终于与家人团聚。
\subsubsection{最后的岁月}  
柯西于1838年底返回巴黎,并恢复了他在法国科学院的职位。[10] 由于他仍然拒绝宣誓效忠,他未能重新获得教学职位。
\begin{figure}[ht]
\centering
\includegraphics[width=6cm]{./figures/7c1001f8a8901986.png}
\caption{柯西晚年} \label{fig_August_2}
\end{figure}
1839年8月,长久处于空缺状态的经度局(Bureau des Longitudes)出现了空缺。这个局与法国科学院有一些相似之处;例如,它有权共选成员。此外,人们认为经度局的成员可以“忽略”宣誓效忠,尽管形式上,与科学院成员不同,他们仍然被要求宣誓。经度局是一个成立于1795年的组织,旨在解决海上定位问题——主要是经度坐标的问题,因为纬度可以通过太阳的位置轻易确定。由于认为海上定位最好通过天文观测来确定,经度局已经发展成一个类似于天文科学学院的组织。

1839年8月,长久处于空缺状态的经度局(Bureau des Longitudes)出现了空缺。这个局与法国科学院有一些相似之处;例如,它有权共选成员。此外,人们认为经度局的成员可以“忽略”宣誓效忠,尽管形式上,与科学院成员不同,他们仍然被要求宣誓。经度局是一个成立于1795年的组织,旨在解决海上定位问题——主要是经度坐标的问题,因为纬度可以通过太阳的位置轻易确定。由于认为海上定位最好通过天文观测来确定,经度局已经发展成一个类似于天文科学学院的组织。

在整个十九世纪,法国教育系统一直在努力解决教会与国家的分离问题。在失去对公立教育系统的控制后,天主教会试图建立自己的教育分支,并在柯西身上找到了一个坚定而显赫的盟友。他将自己的声望和知识贡献给了巴黎的圣职教育学院(École Normale Écclésiastique),这是一所由耶稣会士管理的学校,旨在培养他们学院的教师。他还参与了天主教学院(Institut Catholique)的创立。该学院的目的是应对法国缺乏天主教大学教育的影响。这些活动并未让柯西在他的同事中获得欢迎,因为他们大多支持法国革命时期的启蒙思想。1843年,法国高等师范学院数学教席空缺,柯西申请了这个职位,但只得到了45票中的3票。

1848年,路易-菲利普国王逃往英国。宣誓效忠被废除,柯西的学术任命之路变得清晰。1849年3月1日,他重新被任命为法国科学院的数学天文学教授。在1848年整个政治动荡之后,法国选择成为一个共和国,由法国的拿破仑三世担任总统。1852年初,总统自封为法国皇帝,并取名拿破仑三世。

在官僚圈子里出现了一个想法,认为再次要求所有国家公务员,包括大学教授,宣誓效忠是有用的。这一次,一位内阁部长成功说服皇帝免除柯西的宣誓要求。1853年,柯西被选为美国哲学会的国际会员。[13] 柯西一直担任大学教授,直到67岁去世。他在5月23日凌晨4点接受了临终圣礼,并因支气管病症去世。[10]

他的名字是刻在埃菲尔铁塔上的72个名字之一。
\subsection{工作}  
\subsubsection{早期工作}  
柯西的天才在他对阿波罗尼乌斯问题的简单解决方案中得到了体现——即描述一个与三给定圆相切的圆,这一解法他在1805年发现;他在1811年对欧拉关于多面体公式的推广;以及他在其他几个优雅问题中的工作。更重要的是,他关于波传播的论文,这篇论文获得了法国科学院的大奖奖(Grand Prix)于1816年。柯西的著作涵盖了多个重要主题。在级数理论中,他发展了收敛性的概念,并发现了许多关于q级数的基本公式。在数论和复数理论中,他是第一个将复数定义为一对实数的人。他还写作过群论与置换理论、函数理论、微分方程和行列式的相关内容。[4]
\subsubsection{波动理论、力学、弹性}  
在光学理论中,他研究了弗涅尔的波动理论以及光的色散和偏振。他还在力学方面作出了贡献,替代了物质连续性原理,提出了几何位移连续性的概念。[14] 他撰写了关于杆件和弹性薄膜平衡的论文,并研究了弹性介质中的波动。他引入了一个3×3的对称矩阵,现在被称为柯西应力张量。[15] 在弹性理论中,他创立了应力理论,他的成果几乎与西梅翁·泊松的成果同等重要。[4]
\subsubsection{数论}  
其他重要贡献包括他是第一个证明费马多边形数定理的人。
\subsubsection{复变函数}  
柯西最著名的贡献是他独立发展了复变函数理论。柯西证明的第一个关键定理,现在称为柯西积分定理,内容如下:
\[
\oint_{C} f(z) \, dz = 0,~
\]
其中,\( f(z) \) 是一个在复平面内且在不自交的闭合曲线 \( C \) 上解析的复值函数。该曲线积分沿着曲线 \( C \) 计算。这个定理的基本内容已经可以在24岁的柯西于1814年8月11日向法国科学院(当时仍称为“第一学科”)提交的一篇论文中找到。该定理的完整形式则在1825年给出。[16]

1826年,柯西给出了函数的残数的正式定义。[17] 这一概念涉及具有极点的函数——孤立奇点,即函数值趋向正无穷或负无穷的点。如果复值函数 \( f(z) \) 可以在奇点 \( a \) 附近展开为
\[
f(z) = \varphi(z) + \frac{B_1}{z-a} + \frac{B_2}{(z-a)^2} + \cdots + \frac{B_n}{(z-a)^n}, \quad B_i, z, a \in \mathbb{C},~
\]
其中 \( \varphi(z) \) 是解析的(即没有奇点的良好行为),那么称函数 \( f \) 在点 \( a \) 处具有阶数为 \( n \) 的极点。如果 \( n = 1 \),则该极点称为简单极点。柯西称系数 \( B_1 \) 为函数 \( f \) 在 \( a \) 处的残数。如果 \( f \) 在 \( a \) 处是无奇点的,则 \( f \) 在 \( a \) 处的残数为零。显然,对于简单极点,残数为
\[
\underset{z=a}{\mathrm{Res}} f(z) = \lim_{z \to a} (z-a) f(z),~
\]
这里我们将 \( B_1 \) 替换为残数的现代符号。

1831年,在都灵期间,柯西向都灵科学院提交了两篇论文。在第一篇论文中,[18] 他提出了现在被称为柯西积分公式的公式:
\[
f(a) = \frac{1}{2\pi i} \oint_C \frac{f(z)}{z-a} \, dz,~
\]
其中 \( f(z) \) 在 \( C \) 上及其所包围的区域内是解析的,复数 \( a \) 位于该区域内。该积分沿逆时针方向进行。显然,积分被积函数在 \( z = a \) 处有一个简单极点。在第二篇论文中,[19] 他提出了残数定理:
\[
\frac{1}{2\pi i} \oint_C f(z) \, dz = \sum_{k=1}^{n} \underset{z = a_k}{\mathrm{Res}} f(z),~
\]
其中求和是在 \( f(z) \) 的所有 \( n \) 个极点(包括位于 \( C \) 上及其内的极点)上进行的。柯西的这些结果至今仍然构成复杂函数理论的核心内容,物理学家和电气工程师至今在教学中使用这些理论。长时间以来,柯西的 contemporaries 忽视了他的理论,认为其过于复杂。直到1840年代,这些理论才开始获得回应,其中皮埃尔·阿尔方斯·洛朗(Pierre Alphonse Laurent)是除了柯西之外,第一个作出重要贡献的数学家(他在1843年发表了关于现在称为洛朗级数的研究)。
\subsubsection{分析课程}
\begin{figure}[ht]
\centering
\includegraphics[width=6cm]{./figures/8dbae5a44f9833e6.png}
\caption{柯西一本教科书的标题页。} \label{fig_August_3}
\end{figure}
在他的著作《分析课程》中,柯西强调了分析中严格性的 importance。严格性在这里意味着拒绝代数的普遍性原则(早期作者如欧拉和拉格朗日的观点),并通过几何学和无穷小量来替代这一原则。[20] Judith Grabiner 写道,柯西是“教会整个欧洲严格分析的人”。[21] 这本书经常被认为是首次引入不等式以及 \(\delta - \varepsilon\) 论证到微积分中的地方。在书中,柯西将连续性定义为:如果在给定的区间内,自变量的每一个无限小增量都对应着因变量的一个无限小增量,那么函数\(f(x)\)就是连续的。

M. Barany 声称,尽管柯西本人的判断反对,École(巴黎高等师范学校)还是强制要求包括无穷小方法。[22] Gilain 指出,当分析代数部分的课程在1825年被缩减时,柯西坚持将连续函数(因此也包括无穷小)作为微分学的开篇内容。[23] Laugwitz(1989年)和 Benis-Sinaceur(1973年)指出,柯西甚至在1853年仍然在自己的研究中继续使用无穷小方法。

柯西给出了无穷小的明确定义,具体通过一个趋于零的序列来表达。关于柯西的“无穷小量”概念,已经有大量文献进行讨论,认为这一概念从传统的“ε-δ”定义到非标准分析的概念都有影响。共识是,柯西省略或隐含了许多重要的思想,未能清楚地阐明他所使用的无穷小量的确切含义。[24]
\subsubsection{泰勒定理}
他是第一个严密证明泰勒定理的人,并确立了著名的余项形式。[4] 他为自己在高等师范学校的学生编写了一本教科书[25](见插图),在书中他尽可能严谨地发展了数学分析的基本定理。在这本书中,他给出了极限存在的必要和充分条件,这一形式至今仍在教学中使用。此外,柯西著名的绝对收敛性测试也源自这本书:柯西压缩测试。1829年,他在另一本教科书中首次定义了复变量的复函数。[26] 尽管如此,柯西自己的研究论文常常使用直观而非严谨的方法;[27] 因此,他的一个定理曾被阿贝尔用“反例”揭示,后来通过引入一致连续性的概念得以修正。
\subsubsection{论点原理,稳定性}
在1855年发表的一篇论文中,柯西讨论了一些定理,其中之一类似于现代复分析教材中的“论点原理”。在现代控制理论教材中,柯西的论点原理常被用来推导奈奎斯特稳定性准则,该准则可以用来预测负反馈放大器和负反馈控制系统的稳定性。因此,柯西的工作对纯数学和实际工程都产生了深远的影响。
\subsection{已出版的著作}
Cauchy 的学术产出非常丰富,论文数量仅次于莱昂哈德·欧拉。将他所有的著作收集成27卷花费了近一个世纪的时间:
\begin{itemize}
\item 《奥古斯丁·库尔奇的全集》,由法国科学院科学指导,且在法国公共教育部长的支持下出版(27卷),存档于 Wayback Machine(2007年7月24日存档)(巴黎:高梯维拉出版社,1882–1974)
\item 《奥古斯丁·库尔奇的全集》。法国科学院(1882–1938)——通过法国国家教育部
\end{itemize}
他对数学科学的最大贡献体现在他所引入的严格方法中;这些主要体现在他的三部伟大著作中:
\begin{itemize}
\item 《代数分析》。皇家工科学院分析课程。巴黎:皇家印刷局,Debure兄弟,国王和国王图书馆的图书商。1821年。可在互联网档案馆在线查看。
\item 《无穷小计算》(1823年)
\item 《无穷小计算应用讲义;几何学》(1826–1828年)
\end{itemize}
他的其他著作包括:
\begin{itemize}
\item 《关于定义积分的论文,取自虚数界限之间》 (法语)。提交给科学院,日期为2月28日:巴黎,De Bure兄弟。1825年。
\item 《数学练习》。巴黎,1826年。
\item 《数学练习。第二年卷》。巴黎,1827年。
\item 《微分计算讲义》。巴黎:De Bure兄弟。1829年。
\item 《天体力学及其应用于大量不同问题的新计算方法》 (法语)。提交给都灵科学院,日期为1831年10月11日。
\end{itemize}