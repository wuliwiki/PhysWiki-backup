% 斯蒂芬·霍金(综述)
% license CCBYSA3
% type Wiki

本文根据 CC-BY-SA 协议转载翻译自维基百科\href{https://en.wikipedia.org/wiki/Stephen_Hawking}{相关文章}。

\begin{figure}[ht]
\centering
\includegraphics[width=6cm]{./figures/d175407efe80fdaf.png}
\caption{霍金,大约1980年} \label{fig_HJ_1}
\end{figure}
斯蒂芬·威廉·霍金(Stephen William Hawking,1942年1月8日—2018年3月14日)是英国的理论物理学家、宇宙学家和作家,曾担任剑桥大学理论宇宙学研究中心的研究主任。[6][17][18] 从1979年到2009年,他是剑桥大学的卢卡斯数学教授,这一职位被广泛认为是世界上最具声望的学术职务之一。[19]

霍金出生于牛津,来自一个医学世家。1959年10月,17岁的他开始在牛津大学大学学院学习,并获得了物理学一等荣誉学位。1962年10月,他开始在剑桥大学三一学院攻读研究生,并于1966年3月获得应用数学和理论物理学博士学位,专业方向为广义相对论和宇宙学。1963年,霍金在21岁时被诊断为一种早期发病、进展缓慢的运动神经元病,这种病症在几十年中逐渐使他瘫痪。[20][21] 失去语言能力后,他通过语音生成设备进行交流,最初使用手持开关,后来通过单个面部肌肉来控制设备。[22]

霍金的科学成就包括与罗杰·彭罗斯(Roger Penrose)合作研究广义相对论框架下的引力奇点定理,以及理论预测黑洞会发射辐射,这一现象通常被称为霍金辐射。最初,霍金辐射的预测颇具争议。但到了1970年代末,随着进一步研究的发表,这一发现被广泛接受,成为理论物理学中的重大突破。霍金是第一个提出将广义相对论与量子力学结合来解释宇宙学的理论的人。他是多世界解释的积极支持者。[23][24] 他还提出了微型黑洞的概念。[25]

霍金通过几部畅销的科普作品取得了商业成功,他在书中讨论了自己的理论和宇宙学问题。他的著作《时间简史》曾连续237周登上《星期日泰晤士报》畅销书榜,创下纪录。霍金是英国皇家学会会员、教宗科学院终身会员,并获得了美国总统自由勋章,这是美国的最高平民荣誉奖。2002年,霍金在BBC的“100位最伟大的英国人”评选中排名第25位。霍金于2018年去世,享年76岁,诊断为运动神经元病后,他活过了50多年。