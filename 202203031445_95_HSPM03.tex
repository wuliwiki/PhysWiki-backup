% 牛顿运动定律
% 牛顿|牛顿运动定律|惯性|加速度

\begin{issues}
\issueDraft
\issueTODO
\end{issues}

\pentry{机械运动基础\upref{HSPM01}, 相互作用\upref{HSPM02}}

\subsection{牛顿第一定律}
\subsubsection{内容}
一切物体总保持匀速直线运动状态或静止状态,除非作用在它上面的力迫使它改变这种状态.

\subsubsection{惯性}
物体保持原来匀速直线运动状态或静止状态的性质叫做惯性,是物体本身的固有属性.质量是物体惯性大小的唯一量度,物体的质量越大,其运动状态越难改变,惯性越大;质量越小,其运动状态越容易改变,惯性越小.

\subsubsection{对牛顿第一定律的理解}
牛顿第一定律揭示了一切物体都具有保持原来匀速直线运动状态或静止状态的性质,即一切物体都具有惯性,所以牛顿第一定律又叫\textbf{惯性定律}.

牛顿第一定律定性地揭示了运动和力的关系,说明力不是维持物体运动状态的原因,而是改变物体运动状态的原因.

牛顿第一定律是牛顿在总结前人观念的基础上得出的,是在理想实验的基础上加以科学推理和抽象得到的.

牛顿第一定律无法由实验直接验证,它所描述的是一种不受外力的理想状态.

当物体所受合力为0的时候,其效果跟不受外力时一致,但不能把“合力为0”说成“不受外力”.

\subsection{牛顿第二定律}
\subsubsection{内容}
物体加速度的大小跟它受到的作用力成正比,跟它的质量成反比,加速度的方向跟作用力的方向相同.

\subsubsection{公式}
\begin{equation}
\bvec F=m\bvec a
\end{equation}

\subsubsection{力的单位}
在国际单位制中,力的单位是牛顿($\mathrm N$),它是根据牛顿第二定律定义的,使$1\mathrm{kg}$的物体产生$1\mathrm{m/s^2}$加速度的力为$1\mathrm N$,即$1\mathrm N=1\mathrm{kg \cdot m/s^2}$.

\subsubsection{特性}
\textbf{因果性}:力是产生加速度的原因.若不存在力,则没有加速度.

\textbf{矢量性}:加速度和力都是矢量,它们的方向始终相同,加速度的方向由力的方向唯一决定.

\textbf{瞬时性}:加速度和力同时产生、变化和消失,为瞬时对应关系.

\textbf{同体性}:应用牛顿第二定律时,公式中的三个物理量都是针对同一个研究对象.

\textbf{独立性}:物体同时受到几个力作用时,每一个力都会产生一个加速度,互不干扰,这些加速度的矢量和为物体的实际加速度,即合外力产生的加速度.

\subsection{牛顿第三定律}