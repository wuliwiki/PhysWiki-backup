% 圆锥曲线的统一定义(高中)
% keys 准线|第二定义|焦点|圆锥曲线|焦点-准线定义
% license Usr
% type Tutor

\pentry{圆锥曲线与圆锥\nref{nod_ConSec}}{nod_55cd}

古希腊时代利用圆锥来研究圆锥曲线的方法虽然直观,但缺少了统一的视角,更多地是通过圆锥截得曲线后,分别对应去研究各个圆锥曲线自身的性质。随着坐标系的发明,人们开始发现这些圆锥曲线可以采用一个统一的定义来进行研究这样的研究抛却了过去基于几何视角的定义,在引入一条直线之后,人们发现这个圆锥曲线竟然有一个统一的定义,这个定义简洁而优雅,并且给圆锥曲线的研究带来了一些新的性质和视角,几何中,在摄几何中,这个统一的定义竟然带来了一些意想不到的收获,本文就来介绍他们。

古希腊时期,人们通过截取圆锥面来研究圆、椭圆、抛物线和双曲线等曲线。尽管这种方式直观,而且给予这些曲线同样的来源,但各类曲线被分别对待,缺乏统一的视角。随着坐标系的引入,数学家们逐渐发现,这些看似不同的曲线,其实可以通过一个简洁而优雅的定义统一描述——引入一个定点和一条定直线,并考察平面上点到这两者距离的比。

这一焦点-准线的定义不仅在代数和解析几何中揭示了圆锥曲线的本质,也在射影几何等更高层次的研究中带来了意想不到的收获。本文将围绕这一统一视角,介绍圆锥曲线的几何构造与深层结构。


\subsection{圆锥曲线的焦点-准线定义}

利用准线与焦点得到的。提供了一个统一的视角来看待

\textbf{圆锥曲线的焦点-准线定义(Focus-Directrix Definition of Conic Sections)}。

\begin{definition}{圆锥曲线的焦点-准线定义}
平面上到一个定点与到一条定直线的距离之比为定值的点构成的图像称为\textbf{圆锥曲线(conic section)}。

其中,定点称为圆锥曲线的\textbf{焦点(focus)},定直线称为圆锥曲线的\textbf{准线(directrix)},二者互相对应。比值称作圆锥曲线的\textbf{离心率(eccentricity)},通常记为$e$ 。特别地:
\begin{itemize}
\item 当 $e = 0$ 时,轨迹称为\textbf{圆(circle)}。
\item 当 $0 < e < 1$ 时,轨迹称为\textbf{椭圆(ellipse)}。
\item 当 $e = 1$ 时,轨迹称为\textbf{抛物线(parabola)}。
\item 当 $e > 1$ 时,轨迹称为\textbf{双曲线(hyperbola)}。
\end{itemize}
\end{definition}


显然,定点到定直线的垂线为圆锥曲线的对称轴。

\subsection{定义等价性}