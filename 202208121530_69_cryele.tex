% 晶体中电子在电场和磁场中的运动
% 晶体|电子|能带|有效质量

\pentry{近自由电子模型\upref{egasmd}}
利用近自由电子模型或者紧束缚模型\upref{tbappx},可以有效地分析晶体中电子的行为.而其中的最重要的结果就是\textbf{能带}.电子在周期性势场中运动的本征态,对应的能量本征值 $E$,和其 Bloch 波函数的波数 $\bvec k$($\hbar \bvec k$也被称为准动量)有一个对应关系,而这被称为色散关系,画在 $E$-${\bvec k}$ 图上就呈现出一个曲面,被称为能带.可以说,能带反映了晶体中电子的几乎一切行为,可以用来解释金属、半导体、绝缘体的导电行为.因此,近自由电子模型和紧束缚模型虽然对晶体模型作了许多简化,但在解释晶体的性质方面取得了很大成功.利用能带,我们不仅可以知道晶体中电子的行为,还可以知道对晶体加外场(例如电场或磁场)时电子的变化,从而计算得到晶体对外场的响应.

如何分析加外场后电子的响应呢?一种方法是求解外加场的情况下薛定谔方程的能量本征解,即
\begin{equation}
\qty[\dv{\hbar^2}{2m}\nabla^2+V(\bvec r)+U]\psi = E\psi
\end{equation}
另一种方法是将电子的运动近似当作经典粒子来处理.下面我们介绍的就是这种准经典运动的分析计算.
\subsection{准经典运动}