% 平行性(向量丛)
% keys 联络|平行|平行移动|曲率|Lipschitz 连续

\pentry{曲率(向量丛)\upref{VecCur}, 费罗贝尼乌斯定理\upref{FrobTh}}

本文使用爱因斯坦求和约定. 设 $M$ 是 $n$ 维微分流形, $E$ 是其上秩为 $k$ 的光滑向量丛. 设给定了 $E$ 上的联络 $D$.

\subsection{平行截面}

向量丛 $E$ 的截面 $\xi\in\Gamma(E)$ 称为在联络 $D$ 之下\textbf{平行的(parallel)}, 如果
$$D\xi=0.$$
在局部标架 $\{s_\alpha\}_{\alpha=1}^k$ 之下, 如果 $\xi=\xi^\alpha s_\alpha$, $\omega$ 是此标架下的联络1-形式矩阵, 则 $D\xi=0$ 等价于
$$
d\xi^\alpha+\xi^\beta\omega^\alpha_\beta=0,\,1\leq \alpha\leq k.
$$
进一步给定切丛和余切丛的局部标架 $\{e_i\},\{\theta_j\}$ (二者为对偶) 之后, 就有了克氏符 $\Gamma_{\beta i}^\alpha$, 因此上式进一步等价于方程组
$$
e_i(\xi^\alpha)+\xi^\beta\Gamma^\alpha_{\beta i}=0,\,1\leq \alpha\leq k,\,1\leq i\leq n.
$$
这是普法夫系. 在局部上, 根据费罗贝尼乌斯定理\upref{FrobTh}, 这方程组可解(局部上相当于有 $k$ 个函数 $\{\xi^\alpha\}_{\alpha=1}^k$ 满足上面的偏微分方程组)当且仅当
$$
0=d(\omega^\alpha_\beta\xi^\beta)=d\xi^\beta\wedge\omega^\alpha_\beta+\xi^\beta d\omega^\alpha_\beta=\xi^\beta\wedge\Omega_\beta^\alpha.
$$
这里 $\Omega$ 是 $D$ 的曲率方阵. 这也就表示这个方程组可解当且仅当
$$
\xi^\beta\wedge\Omega_\beta^\alpha=0.
$$
因此如果 $D$ 的曲率在某个开集上等于零, 则上述普法夫系在此开集上是局部可积的, 于是此开集上存在 $E$ 的局部平行截面.

\textbf{这样看来, 曲率是平行截面存在的障碍.}

$\mathbb{R}^n$ 上平凡丛 $\mathbb{R\times}^n\mathbb{R}^k$ 上的平凡联络就是通常的微分运算, 因此当然是可交换的, 曲率算子为零. 设 $\partial_\alpha$ 是 $\mathbb{R}^k$ 上的一个仿射坐标向量, 则它在此联络下就是平行的. 这很符合"平行"的直观意义. 

\subsection{平行移动}
向量丛在区域上的平行截面很可能不存在, 但却可以定义沿着某条道路平行的截面. 

设 $\gamma:[0,a]\to M$ 是 Lipschitz 连续的道路, $\xi_0\in E_{\gamma(0)}$ 是沿着道路的截面. 它的严格定义是一个映射: $\xi:[0,a]\to E$, 使得 $\xi(t)\in E_{\gamma(t)}$. 称此截面是\textbf{沿着道路 $\gamma$ 平行的(parallel along $\gamma$)}, 如果 $D_{\gamma'}\xi(t)=0$. 此时称 $\xi(a)$ 是 $\xi(0)$ 沿着 $\gamma$ 的\textbf{平行移动(parallel transport)}.

在 $M$ 的局部坐标系 $\{x^i\}_{i=1}^n$ 和 $E$ 的局部标架 $\{s_{\alpha}\}_{\alpha=1}^k$ 之下, 可写 $\gamma(t)=(\gamma^i(t))$, $\xi=\xi^\alpha s_\alpha$, 则 $D_{\gamma'}\xi=0$ 等价于
$$
\frac{d\xi^\alpha}{dt}+\Gamma_{\beta i}^\alpha(\gamma(t))\frac{d\gamma^i}{dt}(t)\xi^\beta=0.
$$
这是 $k$ 个函数 $\xi^\alpha$ 的齐次线性常微分方程组, 系数是有界的, 所以给定初值之后它有唯一的 Lipschitz 连续的解. 

这就给出了一个线性变换 $P_{\gamma}:E_{\gamma(0)}\to E_{\gamma(a)}$, 将任何一个 $\xi(0)\in E_{\gamma(0)}$ 通过平行移动变为 $\xi(a)\in E_{\gamma(a)}$. 从微分方程本身可得出终值与 $\gamma$ 重参数化的方法无关. 简单计算给出 $P_\gamma^{-1}$ 实为沿着反向道路 $\gamma^{-1}$ 的平行移动. 

若命 $P_\gamma^{\tau,t}$ 表示从 $E_{\gamma(\tau)}$ 到 $E_{\gamma(t)}$ 的沿着 $\gamma$ 的平行移动, 则简单地计算给出: 对于任何截面 $\xi\in\Gamma(E)$ 和任何使得 $\gamma'(t)$ 存在的 $t$, 皆有
$$
D_{\gamma'(t)}\xi(\gamma(t))=\lim_{\tau\to t}\frac{P_\gamma^{\tau,t}\xi(\gamma(\tau))-\xi(\gamma(t))}{\tau-t}.
$$
这样看来, 沿着道路的平行移动实则决定了整个联络本身的取值.

对于 $\mathbb{R}^n$ 上平凡丛 $\mathbb{R}^n\times\mathbb{R}^k$ 上的平凡联络, 沿着两条有同样起终点的道路的平行移动给出同样的结果: 它就是通常意义下实数空间中的平行移动, 移动的结果显然跟移动的路径无关. 然而对于一般的联络来说这是不对的. 这种现象是因为一般的联络曲率不为零. 这是和乐定理\upref{VecHol}要讨论的内容.
