% 一维自由高斯波包(量子)
% keys 薛定谔方程|波函数|高斯波包|傅里叶变换
% license Xiao
% type Tutor

\begin{issues}
\issueAbstract
\issueTODO
\end{issues}

% 未完成
% 插入 app 链接 https://wuli.wiki/apps/gausWP.html
% 介绍群速度和相速度
% 给出宽度和高度随时间的变化方式
% 可以从无限深势阱中的高斯波包来引入求解的过程, 中心思想都是先分解成本征态, 然后每个本征态分别演化, 再组合起来
% 未完成: 需要编程,做出动画

\pentry{无限深势阱中的高斯波包\upref{wvISW}}

我们这里要讨论的是量子力学中单个粒子的一个特定的一维波函数 $\psi(x,t)$。 它描述的粒子的位置分布是一个高斯分布(正态分布)\upref{GausPD}。

\begin{figure}[ht]
\centering
\includegraphics[width=12cm]{./figures/8264731dbbcedff2.png}
\caption{根据\autoref{eq_GausWP_4} 画出的高斯波包, 蓝色为实部, 红色为虚部。 动画见\href{https://wuli.wiki/apps/free_gauss.html}{这里}, Matlab 代码见 “自由高斯波包(量子)的动画绘制(Matlab)\upref{FreeGs}”, 互动见\href{https://wuli.wiki/apps/gausWP.html}{这里}), 注意该图 $t > 0$, 注意从左到右频率逐渐变高, 这种现象叫做啁啾(chirp)。} \label{fig_GausWP_1}
\end{figure}
\addTODO{详细讨论啁啾, 另外注意随着波函数演化, 位置和动量分部不再满足最小不确定性原理。}


设 $t = 0$ 时的波函数(已归一化)
\begin{equation}\label{eq_GausWP_1}
\psi (x,0) = \frac{1}{(2\pi\sigma_x ^2)^{1/4}} \E^{-(x - x_0)^2/(2\sigma_x)^2} \E^{\I \frac{p_0}{\hbar}x}~.
\end{equation}
那么动量表象波函数具有对称的形式\footnote{也可以把\autoref{eq_GausWP_1} 和\autoref{eq_GausWP_2} 同时除以常数 $\E^{\I p_0 x_0}$ 使\autoref{eq_GausWP_1} 最后的 $x$ 变为 $x-x_0$, \autoref{eq_GausWP_2} 最后的 $p-p_0$ 变为 $p$。 }% 链接未完成
\begin{equation}\label{eq_GausWP_2} % 已验证
\psi (p,0) = \frac{1}{(2\pi\sigma_p^2)^{1/4}} \E^{-(p - p_0)^2/(2\sigma_p)^2} \E^{-\I\frac{x_0}{\hbar }(p - p_0)}~,
\end{equation}

其中 $\sigma_x$ 为位置的标准差, $\sigma_p$ 为动量的标准差,满足不确定原理\upref{Uncert}
\begin{equation} % 已验证
\sigma_x\sigma_p = \frac{\hbar}{2}~.
\end{equation}
含时波函数为
\begin{equation}\label{eq_GausWP_4}\begin{aligned} % 已验证
\psi (x,t) = &\frac{1}{(2\pi\sigma_x^2)^{1/4} \sqrt{1 + {\I\hbar t}/(2m \sigma_x^2)}}
\exp\qty[\frac{-(x - x_0 - p_0 t/m)^2}{(2\sigma_x )^2 \qty(1 + \frac{\I\hbar t}{2m \sigma_x^2})}] \exp\qty[\frac{\I p_0}{\hbar } \qty(x - \frac{p_0 t}{2m})]~.
\end{aligned}\end{equation}

一般来说, 波包扩散是因为初始时存在一定的速度分布。 例如中心动量为零的波包经过较长时间 $t$ 后, 位置分布就会趋近于初始动量分布乘以 $t$。

记扩散后的高斯波包的标准差为 $\bar\sigma_x(t)$, 则稍加推导有
\begin{equation}
\bar\sigma^4_x(t) = \sigma_x^4 + \qty(\frac{\hbar}{2m}t)^2~.
\end{equation}
当 $t$ 很大时, 波包长度正比于 $\sqrt{t/m}$,这就是为什么质量小的物体更容易显示出波动性。 对于电子有 $\bar\sigma_x(t) \sim 0.0076\sqrt{t}$(国际单位),在厘米量级。

\subsection{推导}
\subsubsection{初状态}
\pentry{高斯积分\upref{GsInt}}

以下为书写方便使用原子单位制\upref{AU}。 如果我们想要一维波函数的概率分布为高斯分布\upref{GausPD},即
\begin{equation}
\abs{\psi (x)}^2 = \frac{1}{\sigma_x \sqrt{2\pi}} \exp[-\frac{x^2}{2\sigma_x^2}]~.
\end{equation}
先假设波函数为实数,有
\begin{equation}
\psi (x) = \frac{1}{(2\pi\sigma_x^2)^{1/4}} \exp[-\frac{x^2}{(2\sigma_x)^2}]~.
\end{equation}
变换到动量表象, 即做反傅里叶变换, 借助高斯积分\autoref{eq_GsInt_4}~\upref{GsInt} 得
\begin{equation}
\psi(p) = \frac{1}{(2\pi\sigma_p^2)^{1/4}} \exp[-\frac{p^2}{(2\sigma_p)^2}]~.
\end{equation}
其中 $\sigma_p = 1/(2\sigma_x)$, 可见高斯波包一个独特的性质就是在位置和动量表象下都是高斯分布。

由于波函数为实数,动量平均值为零(\autoref{the_QMavg_1}~\upref{QMavg})。 为了让波函数有一个动量 $p_0$,而维持 $\abs{\psi(x)}^2$ 和 $\abs{\psi(p)}^2$ %未完成: 应该区分函数名
的形状不变,我们可以直接将动量表象中的波函数平移 $p_0$, 得
\begin{equation}
\psi (p) = \frac{1}{(2\pi\sigma_p^2)^{1/4}} \exp[-\frac{(p - p_0)^2}{(2\sigma_p)^2}]~.
\end{equation}
由傅里叶变换的性质(\autoref{eq_FTExp_4}~\upref{FTExp}), 对应的位置表象波函数需要乘以因子 $\exp(\I p_0 x)$ 变为
\begin{equation}
\psi(x) = \frac{1}{(2\pi\sigma_x^2)^{1/4}} \exp[-\frac{x^2}{(2\sigma_x)^2}] {\E^{\I {p_0}x}}~.
\end{equation}
类似地,也可以再将 $\psi(x)$ 平移 $x_0$ 
\begin{equation}
\psi (x) = \frac{1}{(2\pi\sigma_x^2)^{1/4}} \exp[-\frac{(x-x_0)^2}{(2\sigma_x)^2}] \E^{\I p_0 (x-x_0)}~.
\end{equation}
而 $\psi(p)$ 则需要乘以因子 $\exp (-\I x_0 p)$
\begin{equation}\label{eq_GausWP_3}
\psi(p) = \frac{1}{(2\pi\sigma_p^2)^{1/4}} \exp[-\frac{(p - p_0)^2}{(2\sigma_p)^2}] \E^{-\I x_0 p}~.
\end{equation}
将以上两式同乘一个常数% 未完成:傅里叶变换中应该说明一个函数成一个常数,变换后的函数也乘以该常数。并在此引用
 ${\E^{\I{p_0}{x_0}}}$ 就得到\autoref{eq_GausWP_1} 和\autoref{eq_GausWP_2}。

\subsubsection{时间演化的推导}
\pentry{一维自由粒子(量子)\upref{FreeP1}}
按照\autoref{eq_FreeP1_3}~\upref{FreeP1}, 把\autoref{eq_GausWP_2} 乘以时间因子 $\exp(-\I \frac{p^2}{2m}t)$, 再做反傅里叶变换, 并把积分写为
\begin{equation}
\psi(x, t) = C\int_{-\infty}^{+\infty} \exp[-a^2(p-p_0)^2 + b(p-p_0) + c] \dd{p}~.
\end{equation}
的形式, 得
\begin{equation}
a = \sigma_x^2 + \frac{\I t}{2m}~, \qquad
b = \I(x-x_0) - \frac{\I pt}{m}~, \qquad
c = \I p_0 \qty(x - \frac{p_0}{2m} t)~.
\end{equation}
由高斯积分\autoref{eq_GsInt_4}~\upref{GsInt} 得积分结果为
\begin{equation}
\psi(x, t) = C\sqrt{\frac{\pi}{a}} \exp(\frac{b^2}{4a})\E^c~.
\end{equation}
整理后得\autoref{eq_GausWP_4}。
