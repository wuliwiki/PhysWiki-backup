% 南京航空航天大学 2007 量子真题
% license Usr
% type Note

\textbf{声明}:“该内容来源于网络公开资料,不保证真实性,如有侵权请联系管理员”

一、1.证明厄密算符的本征值为实数。10 分

2.利用不确定度关系估算-维线性谐振子的基态能量,20 分

二、一个质量为 $m$ 的粒子在一维无限深势阱 $ (0 \leq x \leq a) $ 中运动,$t = 0$ 时刻的初态波函数为
$$\psi(x, 0) = \sqrt{\frac{8}{5a}} (1 + \cos \frac{\pi x}{a}) \sin \frac{\pi x}{a} \quad (0 \leq x \leq a)~$$ 
(1)求后来的某--时刻$t_0$ 的波函数;\\
(2)求在$t=t_0$ 时刻的平均能量;\\
(3)求在$t=t_0$时在势阱左半部($0\leq  x \leq \frac{a}{2}$) 发现粒子的率。30分

三、粒子在一维无限深势阱 $(0,a)$中运动,受到微扰作用后,
$$V(x) = \begin{cases} \infty & 0 < x, x > a \\\\ \lambda \delta \left( x - \frac{a}{2} \right) & 0 \leq x \leq a \end{cases}~$$
其中$\lambda$为一个很小的正数,试求基态能量准确到$\lambda^2$的修正值,以及$\lambda$应当满足的条件。30 分

四、证明正常寒曼效应(偶极跃迁)的选择定则为
$$\Delta l = \pm 1, \Delta m = 0, \pm 1,\text{和} \Delta s = 0~$$
20 分

五、(1)考虑自旋为$\frac{1}{2}$的系统,求出算符$AS_{y}+BS_{z}$的本征值及归一化的本征函数。其中$S_y,S_z$是角动量算符,$A,B$ 是实常数、

(2)