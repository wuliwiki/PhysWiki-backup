% 列向量
% license Xiao
% type Wiki

\begin{issues}
\issueDraft
\issueMissDepend
\end{issues}
% TODO:改名为 列向量与行向量

\subsection{列向量}

几何向量的坐标让我们可以以一个全新的视角看待向量这个概念,我们可以把数组 $(a_1, \cdots, a_n)$ 称为一个向量$\bvec{a}$;由于我们常常会把它竖着记为
\begin{equation}
\pmat{a_1 \\ \vdots \\ a_n}~,
\end{equation}
这种向量被称为\textbf{列向量},$n$ 被称为 $a$ 的\textbf{维度},$a_i$ 被称为 $a$ 的第 $i$ 坐标\footnote{数学中没有规定一定要从 $1$ 开始计数,也可以从 $0$ 开始。}。对于列向量来说,存在一组特别的基底 $\{\bvec{e}_i\}_{i = 1}^n$,称为\textbf{标准基底},其中 $\bvec{e}_i$ 是第 $i$ 坐标为 $1$,其他坐标为 $0$ 的列向量,因此任何一个列向量都可以写成
\begin{equation}
\bvec{a} = a_1 \bvec{e}_1 + \cdots + a_n \bvec{e}_n~
\end{equation}
的形式。

第 $i$ 坐标 $a_i$ 的取值可以和几何向量一样取 $\mathbb{R}$,也可以取一些其他的数集,比如复数集 $\mathbb{C}$。

\addTODO{数集的定义和链接}

实数取值的 $2$ 维(或者 $3$ 维)列向量,等价于选取了坐标系的几何向量——由标准基底的存在,列向量并不是几何向量的推广。几何向量和列向量都是更一般的向量的特殊情况。

\subsection{行向量}

如果把向量“横过来”,我们就得到了\text{行向量},
\begin{equation}
\pmat{a_1 & \cdots & a_n}~,
\end{equation}
(注意,一般 $(a_1, \cdots, a_n)$ 表示的是列向量)。