% 数列的概念与函数特性(高中)
% keys 高中|数列的概念|数列的函数特性

% 改名为 数列(高中) 可以吗

\subsection{定义与相关概念}
一般地,按一定次序排列的一列数叫做\textbf{数列(sequence)},数列中的每一个数叫作这个数列的\textbf{项}。数列的一般形式可以写成
\begin{equation}
a_1,a_2,a_3,\cdots,a_n,\cdots~
\end{equation}
简记为数列 $\begin{Bmatrix} a_n \end{Bmatrix}$,其中数列的第1项 $a_1$,也称\textbf{首项};$a_n$ 是数列的第 $n$ 项,也叫数列的\textbf{通项}。

项数有限的数列,称为\textbf{有穷数列};项数无限的数列,称为\textbf{无穷数列}。

如果数列 $\begin{Bmatrix} a_n \end{Bmatrix}$ 的第 $n$ 项 $a_n$ 与 $n$ 之间的函数关系可以用一个式子表示成 $a_n = f(n)$,那么这个式子就叫作这个数列的\textbf{通项公式},数列的通项公式就是相应函数的解析式。

\textsl{注意:不是所有数列都能写出通项公式。}

\subsection{函数特性}
一般地,一个数列 $\begin{Bmatrix} a_n \end{Bmatrix}$,如果从第2项起,每一项都大于前一项,即 $a_{n+1}>a_n$,那么这个数列叫作\textbf{递增数列}。

如果从第2项起,每一项都小于前一项,即 $a_{n+1}<a_n$,那么这个数列叫作\textbf{递减数列}。

如果数列 $\begin{Bmatrix} a_n \end{Bmatrix}$ 的各项都相等,那么这个数列叫作\textbf{常数列}。

如果数列 $\begin{Bmatrix} a_n \end{Bmatrix}$ 从第2项起,有些项大于它的前一项,有些项小于它的前一项,这样的数列叫\textbf{摆动数列}。