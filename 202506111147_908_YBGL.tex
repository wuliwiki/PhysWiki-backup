% 约翰·彼得·古斯塔夫·勒热纳·狄利克雷(综述)
% license CCBYSA3
% type Wiki

本文根据 CC-BY-SA 协议转载翻译自维基百科 \href{https://en.wikipedia.org/wiki/Peter_Gustav_Lejeune_Dirichlet}{相关文章}。

约翰·彼得·古斯塔夫·勒让·狄利克雷(Johann Peter Gustav Lejeune Dirichlet,/ˌdɪərɪˈkleɪ/\(^\text{[1]}\),[德语发音:[ləˈʒœn diʁiˈkleː]\(^\text{[2]}\);1805年2月13日-1859年5月5日)是德国数学家。在数论中,他证明了费马大定理的一些特殊情形,并创立了解析数论。在分析学中,他推进了傅里叶级数理论的发展,并且是最早给出函数现代形式定义的数学家之一。在数学物理中,他研究了势理论、边值问题、热扩散和流体力学。

尽管他的姓氏是“勒让·狄利克雷”,但在引用以他命名的成果时,人们通常只使用“狄利克雷”这一名字。
\subsection{生平}
\subsubsection{早年生活(1805–1822)}
古斯塔夫·勒让·狄利克雷于1805年2月13日出生在迪伦(Düren),这是位于莱茵河左岸的一座小镇,当时属于法兰西第一帝国,1815年维也纳会议后归属普鲁士。他的父亲约翰·阿诺德·勒让·狄利克雷(Johann Arnold Lejeune Dirichlet)是一名邮政局长、商人及市议会议员。他的祖父则从比利时列日东北约5公里处的小村庄里舍莱特(Richelette,或更可能是 Richelle [fr])迁居至迪伦,因此家族姓氏“勒让·狄利克雷”(Lejeune Dirichlet,法语意为“来自里舍莱特的年轻人”)由此而来。\(^\text{[3]}\)

尽管家境并不富裕,而且狄利克雷在七个孩子中排行最小,父母仍然支持他的教育。他们先让他就读于一所小学,之后又转入私立学校,希望他日后能成为一名商人。然而年幼的狄利克雷在12岁之前就表现出了对数学的强烈兴趣,最终说服父母让他继续深造。1817年,他被送往波恩文理中学(Gymnasium Bonn [de])就读,受到家庭熟识的彼得·约瑟夫·埃尔韦尼希(Peter Joseph Elvenich,一位学生)的照顾。1820年,狄利克雷转入科隆耶稣会文理中学,在那里他在乔治·欧姆的指导下,进一步拓宽了数学知识。次年他离开了文理中学,仅获得了一张结业证书,因为他无法流利使用拉丁语,未能取得正式的中学毕业文凭。\(^\text{[3]}\)
\subsubsection{巴黎求学时期(1822–1826)}
狄利克雷再次说服了父母,为他的数学学习提供进一步的经济支持,尽管父母希望他从事法律事业。当时德国几乎没有学习高等数学的机会,唯一的例外是哥廷根大学的高斯,但高斯名义上是天文学教授,而且并不喜欢教学。因此,狄利克雷在1822年5月前往巴黎求学。在那里,他在法兰西公学院和巴黎大学听课,师从包括阿谢特在内的多位数学家,同时私下学习高斯的《算术研究》——这本书成为他终生随身携带的重要著作。
1823年,经人推荐,狄利克雷受雇于马克西米连·福伊将军,担任其子女的私人德语教师。借此工作所得的报酬,狄利克雷终于实现了经济独立,不再需要依赖父母的资助。\(^\text{[4]}\)

他的第一项原创研究是对费马大定理在 n= 5 情况下的部分证明,这项成果使他立刻声名鹊起——这是自费马本人完成 n= 4 情况的证明、以及欧拉完成 n= 3 情况的证明之后,该定理上的首次新进展。当时的审稿人之一勒让德很快完成了对 n= 5 情况的完整证明;不久后,狄利克雷也独立完成了自己的证明,并在几年后完成了对 n= 14 情况的完整证明。\(^\text{[5]}\)1825年6月,他获准在法国科学院就 n= 5 情况的部分证明进行讲演——这对于一位年仅20岁、尚无学位的学生来说,是极其罕见的成就。\(^\text{[3]}\)这次在科学院的讲座也使狄利克雷与傅里叶和泊松建立了联系,激发了他对理论物理学的兴趣,尤其是傅里叶的热传导解析理论。
\subsubsection{返回普鲁士,布雷斯劳时期(1825–1828)}

1825年11月,福伊将军去世,狄利克雷在法国也未能找到有报酬的职位,不得不返回普鲁士。傅里叶和泊松将他推荐给了亚历山大·冯·洪堡(Alexander von Humboldt)。当时洪堡受召加入普鲁士国王腓特烈·威廉三世的宫廷,正计划将柏林打造成科学研究中心。他立即向狄利克雷伸出援手,分别向普鲁士政府和普鲁士科学院写信推荐。同时,洪堡还为他争取到了高斯的推荐信。高斯在阅读了狄利克雷关于费马定理的论文后,给予了极高评价,称“狄利克雷展现出极其出色的天赋”。\[6]

在洪堡和高斯的支持下,狄利克雷获得了布雷斯劳大学的教职。不过,由于他尚未完成博士论文,便将自己关于费马定理的论文提交给波恩大学作为博士论文。然而,他缺乏拉丁语流利表达能力,无法进行论文要求的拉丁语公开答辩。经过多方讨论,波恩大学最终决定破例,于1827年2月授予他名誉博士学位。同时,教育部长也特批他豁免了进行博士资格答辩(Habilitation)时所要求的拉丁语答辩。狄利克雷顺利获得博士资格认证,并于1827–1828学年在布雷斯劳以私人讲师(Privatdozent)身份授课。\[3]

在布雷斯劳任教期间,狄利克雷继续从事数论研究,尤其是在双二次互反律方面做出了重要贡献——当时这是高斯研究的核心课题之一。洪堡借助狄利克雷这一系列的新成果(此时他也得到了天文学家贝塞尔的热烈赞誉),成功为他争取到转往柏林任教的机会。由于狄利克雷当时年纪尚轻(仅23岁),洪堡只为他在普鲁士军事学院争取到一个试用职位,同时名义上仍挂靠布雷斯劳大学。这一试用期被延长了三年,直到1831年该职位最终转为正式编制。
