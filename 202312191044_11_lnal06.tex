% 乘积空间与商空间
% license Usr
% type Tutor


\begin{issues}
\issueDraft 
需要合并商空间
\end{issues}
和群有直积类似,线性空间也能在集合意义上求并,笛卡尔积就是积空间的向量。为了求积后依然是线性空间,我们需要规定向量的数乘和加法。

\begin{definition}{积空间}
给定域$\mathbb F $上的线性空间$U$与$V$,定义$U\times V={(\bvec u,\bvec v)|\bvec u\in U,\bvec v\in V}$上的数乘和加法运算为:
\begin{equation}
\left\{\begin{aligned}
a(\bvec u,\bvec v)&=(a\bvec u,a\bvec v),\quad \forall a\in \mathbb F\\
(\bvec u_1,\bvec v_1)+(\bvec u_2,\bvec v_2)&=(\bvec u_1+\bvec u_2,\bvec v_1+\bvec v_2)
\end{aligned}\right.~
\end{equation}
\end{definition}
根据该定义,我们容易验证积空间在数乘和加法下封闭。
若令$\{\bvec x_i\}$和$\{\bvec y_i\}$分别为$U$与$V$的基,我们也容易验证积空间的基为$\{\bvec x_i,\bvec 0\}\cup \{\bvec 0,\bvec y_i\}$。

从群的角度上看,线性空间是一个加法群,则其任意一个子空间都是子群。由于加法群的任意子群都是正规子群,因此线性空间可以对任意一个子空间求商,商群上的运算为\textbf{向量加法}。
\begin{definition}{商空间}
给定域$F$上的线性空间$V$及其子空间$V_0$,则对任意$\bvec v\in V$,定义左陪集$\bvec v+V_0=\{\bvec v+\bvec v_0|\bvec v_0\in V_0\}$。

数乘定义为:$\forall a\in \mathbb F,a(\bvec v+V_0)=a\bvec v+V_0$
\end{definition}
可以证明,区别于群意义上的商群,配备了数乘定义的商空间是线性空间。在每个左陪集上取一个代表元素,加在一起可以构成商空间的基。