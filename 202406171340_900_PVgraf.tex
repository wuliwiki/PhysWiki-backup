% 压强体积图
% keys 理想气体|做功|压强|体积|状态方程
% license Xiao
% type Tutor

\begin{issues}
\issueDraft
\end{issues}

\pentry{状态量和过程量\nref{nod_StaPro}}{nod_4a92}

根据\enref{理想气体状态方程}{PVnRT}, 我们可以得知给定量的理想气体的宏观状态只有两个\enref{自由度}{DoF}。 在\autoref{ex_StaPro_2}  中我们说到可以在压强 $P$, 体积 $V$, 温度 $T$ 等宏观量中任取两个来作为该系统的状态量, 完全确定系统的宏观状态。 而最常见的作法是取压强和体积 $P,V$。 于是自然地, 我们可以用一个二维直角坐标系 $P$-$V$ 中的一点来描绘理想气体某时刻的状态及其变化, 称为\textbf{压强体积图}。 该平面就是理想气体的(宏观)状态空间。 注意在讨论 $P$-$V$ 上的状态变化时, 我们往往假设气体始终处于准平衡的状态。

对于其他的气体模型(如\enref{范德瓦尔斯气体}{Vand}), 虽然状态方程与理想气体不同, 但以上讨论仍然成立, 所以同样可以用 $P$-$V$ 图描述其状态的变化。

\addTODO{还可以举几个等体线、等压线、绝热线的图片作为例子}

\subsection{等温线}
为了直观地表现它们的关系,我们可以借助压强体积图,在上面画出一条曲线——曲线上每个点表示在当前温度下不同压强所对应的不同体积。
\begin{figure}[ht]
\centering
\includegraphics[width=8cm]{./figures/f3ad495eab7e529f.png}
\caption{理想气体的等温线} \label{fig_PVgraf_4}
\end{figure}
\begin{figure}[ht]
\centering
\includegraphics[width=8cm]{./figures/2689b60a1b35173f.png}
\caption{范德瓦尔斯气体的等温线} \label{fig_PVgraf_3}
\end{figure}

等温线帮助我们分析压强、体积与温度的关系。例如\autoref{fig_PVgraf_4} 中等温线是反比例函数,意味着理想气体在同一温度下压强与体积成反比。

\subsection{压强体积图与做功}

\subsubsection{体积、压强与做功}
现有一个\textbf{体积 $V$ 可变}的容器,在容器壁上可以安装探测器来测量内部气体的\textbf{压强 $P$}。只要气体处于\enref{平衡态}{TherEq},气体的压强就处处相等。现在我们考虑这样一个问题——改变容器的体积,需要对容器做多少\enref{功}{Fwork}(或者容器对外界做多少功)?

假设容器是长方体,横截面是 $S$,那么这块面上受气体压强的力为 $P\cdot S$。如果改变长方体的体积,使它长度增加 $\Delta x$(假设在这个过程中压强变化不大),那么气体对外做功为 $F\cdot \Delta x=P\cdot S\Delta x=P\Delta V$。于是我们得出结论 $\Delta W=P\Delta V$,这不仅对长方体容器成立,对任意形状的容器都成立。
\subsubsection{压强体积图上任意曲线的做功}
由于容器在改变体积的过程中,气体的压强可以发生变化,在一个复杂过程中做功不能用简单的 $P\Delta V$ 概括,但我们却可以划分为无穷多个小过程,每一个小过程中压强变化不大,做功近似地可以看成是 $P\Delta V$。这样一来,总的做功就是 $P\dd V$ 的积分了。

图像可以帮助我们方便地分析这个问题。对于 $P-V$ 图中的一条曲线,设容器从点 $1$ 沿着曲线变化到点 $2$——曲线上每一个点表明了气体在当时的体积和压强。
\begin{figure}[ht]
\centering
\includegraphics[width=5cm]{./figures/e3c945b54c40bb69.pdf}
\caption{$P$-$V$ 图中气体做的功} \label{fig_PVgraf_1}
\end{figure}

如\autoref{fig_PVgraf_1} 所示,做功
\begin{equation}\label{eq_PVgraf_1}
W = \int_{V_1}^{V_2}P(V) \dd{V}~.
\end{equation}

由积分关系式可得,做功就是 $P$-$V$ 曲线下方的面积,也就是图中标注的阴影部分的面积。 

\begin{figure}[ht]  
\centering
\includegraphics[width=5cm]{./figures/7b3eccf242802a8d.pdf}
\caption{$P-V$ 图中闭合路径气体做的功} \label{fig_PVgraf_2}
\end{figure}

如\autoref{fig_PVgraf_2} 所示,如果是一个闭合路径,在气体状态沿着它走的过程中,对外做功就是闭合路径围出的面积(顺时针为正)。

\begin{example}{理想气体在等温压缩需要做的功}
一个装有理想气体的体积可变的导热容器,放在温度为 $T$ 的恒温箱中,保证其温度恒为 $T$。现在缓慢地将它从体积 $V_0$ 压缩成 $V_1$,求这个过程中需要对容器壁做的功为多大。

解:气体对外界做功 $P\Delta V$,则外界对气体做功为 $-P\Delta V$。由理想气体状态方程\autoref{eq_PVnRT_1} 可得 $P=\frac{nRT}{V}$。所以
\begin{equation}
\begin{aligned}
\int_{V_0}^{V_1} -P\dd V&=\int_{V_0}^{V_1} -\frac{nRT}{V} \dd V \\
&=nRT \ln\qty(\frac{V_0}{V_1})~.
\end{aligned}
\end{equation}
\end{example}
