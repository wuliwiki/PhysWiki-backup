% 球极平面投影
% license CCBYSA3
% type Wiki

(本文根据 CC-BY-SA 协议转载自原搜狗科学百科对英文维基百科的翻译)

在几何中,球极平面投影是一种将圆球面投影到平面上的特殊映射(函数)。该映射的定义域为除投影点外的整个球面。在定义域内,该映射具有光滑性、双射性和共形性。共形性意味着其为保持曲线相交角度不变的映射。但是,该映射既不保距离也不保面积:也就是说,其既不能维持映射后图形的距离不变,也不能维持映射后图形的面积不变。

直观地说,球极平面投影是一种用平面来描绘球面的方法,但该方法在图案品质方面,也包含一些不可避免的妥协。因为球面和平面出现于数学及其应用的诸多领域,所以球极平面投影也同样很常见。在各个领域,例如复分析、制图学、地质学、摄影等,球极平面投影都有广泛的应用。在实践中,球极平面投影通常是通过计算机或手工在一种称为立体球面投影网的特殊的方格纸上绘制完成,该立体球面投影网简称立体网,或乌尔夫网。

\subsection{历史}
球极平面投影为希帕克斯(Hipparchus)、托勒密(Ptolemy),甚至可能更早期的古埃及人所知。其最初被称为“平面天体投影”。[1] 托勒密的著作《平球论》(Planisphaerium)是现存最古老的描述球极平面投影的文献,其最重要的用途之一就是描绘天体图。[1] “平面天体投影”这个术语仍然用于指代这样的图表。

在16世纪和17世纪,球极平面投影的赤道面通常被用于东半球和西半球地图上。通常认为,1507年瓜尔特里厄斯·路德(Gualterius Lud)绘制的地图[2] 使用的就是球极平面投影,包括后来的让·罗兹(Jean Roze)(1542年)、鲁莫尔德·墨卡托(Rumold Mercator)(1595年),以及其他许多人绘制的地图。[3] 而在星图上,这个赤道面也已经被像托勒密这样的古代天文学家利用了。[4]

弗朗索瓦·德阿吉隆(François d'Aguilon)在他1613年的著作《光学六册:适用于哲学家和数学家》(Opticorum libri sex philosophis juxta ac mathematicis utiles)中给球极平面投影取了现在的名字。[5]

1695年,在兴趣的驱使下,埃德蒙多·哈雷(Edmond Halley)发表了第一篇关于星图共形性的数学证明。[6] 在证明中,他使用了由他的朋友艾萨克·牛顿(Isaac Newton)在当时刚刚建立起来的微积分工具。

\subsection{定义}
\subsubsection{2.1 第一类计算公式}
