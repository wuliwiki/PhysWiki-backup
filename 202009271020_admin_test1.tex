% test1

\usepackage{mathrsfs}
\usepackage{amsmath}
\usepackage{amsthm}
\usepackage{amssymb}
\usepackage{graphicx}
\usepackage{color}
%\include{macros}
%\usepackage{floatflt}
%\usepackage{graphics}
%\usepackage{epsfig}

\newcommand{\reals}{{\mathbb{R}}}
\newcommand{\dom}{{\bf{dom}}}
\newcommand{\symm}{{\bf{S}}}

\theoremstyle{definition}
\newtheorem{theorem}{Theorem}[section]
\newtheorem{lemma}[theorem]{Lemma}
\newtheorem{proposition}[theorem]{Proposition}
\newtheorem{corollary}[theorem]{Corollary}

\theoremstyle{definition}
\newtheorem*{defition}{Definition}
\newtheorem*{example}{Example}

\theoremstyle{remark}
\newtheorem*{remark}{Remark}
\newtheorem*{note}{Note}
\newtheorem*{exercise}{Exercise}

\setlength{\oddsidemargin}{-0.25 in}
\setlength{\evensidemargin}{-0.25 in} \setlength{\topmargin}{-0.25
in} \setlength{\textwidth}{7 in} \setlength{\textheight}{8.5 in}
\setlength{\headsep}{0.25 in} \setlength{\parindent}{0 in}
\setlength{\parskip}{0.1 in}

\newcommand{\homework}[4]{
\pagestyle{myheadings} \thispagestyle{plain}
\newpage
\setcounter{page}{1} \setcounter{section}{#4} \noindent
\begin{center}
\framebox{ \vbox{\vspace{2mm} \hbox to 6.28in { {\bf
AU7021~Optimization~in~Learning and Control (Fall 2020) \hfill Homework: #1} }
\vspace{6mm} \hbox to 6.28in { {\Large \hfill #1 \hfill} }
\vspace{6mm} \hbox to 6.28in { {\it Lecturer: #2 \hfill} }
\vspace{2mm} \hbox to 6.28in { {\it Student: #3 \hfill} }
\vspace{2mm} } }
\end{center}
\markboth{#1}{#1} \vspace*{4mm} }


\begin{document}

\homework{1. Convex Set}{Xiaolin Huang \hspace{5mm} {\tt
xiaolinhuang@sjtu.edu.cn}}{XXX
\hspace{5mm} {\tt xxx@sjtu.edu.cn } }{9}

%%%%%%%%%%%%%%%%%%%%%%%%%%%%%%%%%%%%%%%%%%%%%%%%%%%%%%%%%%%%%%%%%%%%
% Section 2.  Problem
%%%%%%%%%%%%%%%%%%%%%%%%%%%%%%%%%%%%%%%%%%%%%%%%%%%%%%%%%%%%%%%%%%%%

\section*{Problem 1} \label{ex-midpoint-cvx}
\emph{Midpoint convexity.}
A set $C$ is \emph{midpoint convex} if whenever two points $a,b$ are in $C$, the average or midpoint $(a+b)/2$ is in $C$.  Obviously a convex set is midpoint convex. It can be proved that under mild conditions midpoint convexity implies convexity. As a simple case, prove that if $C$ is closed and midpoint convex, then $C$ is convex.

{\bf{Answer.}}


\section*{Problem 2}\label{e-lin-frac-image}%
\emph{Linear-fractional functions and convex sets.}
Let $f:\reals^m \rightarrow \reals^n$ be the linear-fractional function
\[
f(x) = (Ax+b)/(c^T x+ d), \qquad \dom f = \{x \;|\; c^Tx+d >0\}.
\]
In this problem we study the inverse image of a convex set $C$ under $f$, i.e.,
\[
f^{-1} (C) = \{ x \in \dom f \;|\; f(x) \in C \}.
\]
For each of the following sets $C\subseteq \reals^n$, give a simple
description of $f^{-1}(C)$.

\begin{enumerate}
\item The halfspace $C = \{ y \; | \; g^Ty \leqslant h\}$ (with $g\neq 0$).

\item The polyhedron $C = \{y \; | \; Gy \preceq h\}$.

\item The ellipsoid $\{ y \;|\; y^T P^{-1} y \leqslant 1 \}$ (where $P \in \symm^n_{++}$).

\item The solution set of a linear matrix inequality, $C = \{y\; |\; y_1 A_1+ \cdots + y_n A_n \preceq B\}$, where $A_1$, \ldots, $A_n$, $B \in \symm^p$.
\end{enumerate}

{\bf{Answer.}}

\section*{Problem 3}\label{exe-sep-hyp-strict-counterexample}
Give an example of two closed convex sets that are disjoint but cannot be strictly separated.

{\bf{Answer.}}

%%%%%%%%%%%%%%%%%%%%%%%%%%%%%%%%%%%%%%%%%%%%%%%%%%%%%%%%%%%%%%%%%%%%
% Reference
%%%%%%%%%%%%%%%%%%%%%%%%%%%%%%%%%%%%%%%%%%%%%%%%%%%%%%%%%%%%%%%%%%%%
\end{document}
