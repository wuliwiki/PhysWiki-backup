% 浙江大学 1998 年 考研 量子力学
% license Usr
% type Note

\textbf{声明}:“该内容来源于网络公开资料,不保证真实性,如有侵权请联系管理员”

\subsection{第一题:(10分)}
(1) 写出玻尔-索末菲量子化条件的形式。

(2) 求出均匀磁场中作圆周运动的电子轨道的可能半径。(利用玻尔-索末菲量子化条件,设外磁场强度为 $B$。)
\subsection{第二题:(20分)}
(1) 若一质量为 $\mu$ 的粒子在一维势场$V(x) =\begin{cases}    0, & 0 \leq x \leq a, \\\\    \infty, & x > a , x < 0\end{cases}$中运动,求粒子的可能能级。

(2) 若某一时刻加上了形如 $e \sin  \frac{\omega x}{a} $ ($e \ll 1$)的势场,求其基态能级至三级修正($\omega$ 为一已知常数)。

(3) 若势能 $V(x)$ 变成

$$V(x) =\begin{cases}    \frac{1}{2} \mu \omega^2 x^2, & x > 0, \\\\    \infty, & x < 0\end{cases}~$$

求粒子(质量为 $\mu$)的可能的能级。
\subsection{第三题:(20分)}
氢原子处于基态,其波函数形如 $\psi = c e^{-\frac{r}{a}}$,$a$ 为玻尔半径,$c$ 为归一化系数。

\begin{enumerate}
    \item 利用归一化条件,求出 $c$ 的形式。
    \item 设几率密度为 $P(r)$,试求出 $P(r)$ 的形式,并求出最可能半径 $r$。
    \item 求出势能及动能在基态时的平均值。
    \item 用何种定理可把 $\langle \hat{V} \rangle$ 及 $\langle \hat{T} \rangle$ 联系起来?
\end{enumerate}
\subsection{第四题:(15分)}
一转子,其哈密顿算符量 $\hat{H} = \frac{\hat{L}_x^2}{2I_x} + \frac{\hat{L}_y^2}{2I_y} + \frac{\hat{L}_z^2}{2I_z}$,转子的轨道角动量量子数是1。

\begin{enumerate}
    \item 试在角动量表示象中求出角动量分量 $\hat{L}_x$,$\hat{L}_y$,$\hat{L}_z$ 的形式;
    \item 求出 $\hat{H}$ 的本征值。
\end{enumerate}
\subsection{第五题:(20分)}
若基态氢原子处于平行板电场中,电场是按下列形式变化
$$\vec{E} = \begin{cases} 0, & t \leq 0 \\\\\varepsilon_0 e^{-\frac{t}{\tau}}, & t > 0\end{cases}~$$
$\tau$ 为大于零的常数,求经过长时间后,氢原子处于 $2P$ 态的几率。(设 $\hat{H}'$ 为微扰哈密顿算符,
$$\langle \hat{H}' \rangle_{100,210} = \frac{2^8}{\sqrt{2}} \frac{a\varepsilon_0 e}{3^5} e^{-\frac{t}{\tau}}~$$;当  $t > 0 \langle \hat{H}' \rangle_{100,210} = 0 $
