% 球坐标系中的定态薛定谔方程
% 球坐标|柱坐标|定态薛定谔方程|拉普拉斯算子|径向动量

\pentry{定态薛定谔方程\upref{SchEq}, 球坐标系中的亥姆霍兹方程\upref{SphHHz}, 球坐标系中的角动量算符\upref{SphAM}}

本文使用原子单位制\upref{AU}. 我们希望在球坐标中求解定态薛定谔方程
\begin{equation}\label{RadSE_eq2}
-\frac{1}{2m}\laplacian \psi(\bvec r) + V(\bvec r)\psi(\bvec r) = E\psi(\bvec r)
\end{equation}
当势能 $V(\bvec r)$ 处处为零时, 这就变成了球坐标系中的亥姆霍兹方程\upref{SphHHz}. 以下考虑球对称势能 $V(\bvec r) = V(r)$, 即势能只与到原点的距离有关. 该方程在球坐标中的求解过程也和求解亥姆霍兹方程类似\upref{SphHHz}.

我们知道一维定态薛定谔方程的解可分为两种情况, 一种是 $E < V(\pm\infty)$ 时可能存在离散的束缚态, 另一种是 $E > V(\pm\infty)$ 时存在散射态(引用未完成). 这在三维空间中同样成立, 当 $E < V(\infty)$ 时, 若存在束缚态, 那么波函数可以用球谐函数\upref{SphHar}展开为
\begin{equation}\label{RadSE_eq4}
\Psi_{n, l,m}(\bvec r) = R_{n, l}(r) Y_{l,m}(\uvec r) = \frac{1}{r}\psi_{n, l}(r) Y_{l,m}(\uvec r)
\end{equation}
其中 $\psi_{n, l}(r)$ 称为\textbf{径向波函数(radial wave function)}, $\psi_{n, l}(r)$ 称为\textbf{约化径向波函数(scaled radial wave function)}, 满足\textbf{径向薛定谔方程}
\begin{equation}\label{RadSE_eq1}
-\frac{1}{2m} \dv[2]{\psi_{n, l}}{r} + \qty[V(r) + \frac{l(l + 1)}{2mr^2}]\psi_{n, l} = E_{n, l} \psi_{n, l}
\end{equation}
该方程的推导见下文, 其中 $l$ 叫做\textbf{角量子数(angular quantum number)}, $l(l + 1)/(2mr^2)$ 项叫做\textbf{离心势能(centrifugal potential)}. 对每个固定的 $l$, 该式可能存在若干个不同的能级, 每个能级可能对应多个束缚态. 这些能级的序号一般记为 $n$, 称为\textbf{主量子数(principal quantum number)}, 主量子数一般从 0 开始.

若势能 $V(r)$ 在整个空间都是有限值, 对于每个较小的 $l$, 可能不存在束缚态, 也可能存在有限个束缚态, 即有限个 $n$. 束缚态能量 $E_{n,l}$ 由这两个量子数共同决定. 随着 $l$ 不断增加, 方括号中的等效势能越来越浅, 最终会导致径向方程不存在束缚态解. 一个简单的例子是有限深球势阱\upref{FiSph}.

一个重要的例子是氢原子: 由于 $1/r$ 势能的特殊性, 它不但有无穷个束缚态, 而且束缚态能量 $E_n$ 和 $l$ 无关, 详见 “类氢原子的定态波函数\upref{HWF}”. 另一些例子 “无限深球势阱\upref{ISphW}” 以及 “三维量子简谐振子(球坐标系)\upref{SHOSph}”, 由于简谐振的 $V(r\to \infty)$ 为无穷大, 所以同样存在无穷个束缚态.

\subsubsection{径向波函数归一化条件}
束缚态的总波函数的归一化条件为
\begin{equation}
\int_0^\infty \abs{\Psi_{n,l,m}}^2 r^2\dd{r} \dd{\Omega} = \int \abs{R_{n,l}}^2 \abs{Y_{l,m}}^2 r^2\dd{r} \dd{\Omega}  = 1
\end{equation}
球谐函数已经满足 $\int \abs{Y_{l,m}}^2 \dd{\Omega} = 1$,  所以只要求
\begin{equation}\label{RadSE_eq5}
\int_0^\infty \abs{R_{n,l}}^2 r^2 \dd{r}  = 1
\end{equation}
或者
\begin{equation}
\int_0^\infty \abs{\psi_{n,l}(r)}^2 \dd{r}  = 1
\end{equation}
再来看正交性, 根据施特恩刘维尔定理(未完成), 同一个\autoref{RadSE_eq1} (即 $l$ 不变)解出的不同 $\psi_{n,l}$ 满足正交条件
\begin{equation}
\int_0^\infty \psi_{n', l}\Cj(r) \psi_{n,l}(r) \dd{r} = 0 \qquad (n' \ne n)
\end{equation}
或者
\begin{equation}\label{RadSE_eq7}
\int_0^\infty R_{n', l}\Cj(r) R_{n,l}(r) r^2 \dd{r} = 0 \qquad (n' \ne n)
\end{equation}


\subsection{散射态}
当 $E > V(\infty)$ 时, $E$ 取任意实值都可以解出散射态, 这时我们就不能用离散的 $n$ 来区分不同的解, 而是直接使用 $E$:
\begin{equation}
\Psi_{E, l,m}(\bvec r) = \frac{1}{r}\psi_{E, l}(r) Y_{l,m}(\uvec r)
\end{equation}
径向方程则变为
\begin{equation}
-\frac{1}{2m} \dv[2]{\psi_{E, l}}{r} + \qty[V(r) + \frac{l(l + 1)}{2mr^2}]\psi_{E, l} = E \psi_{n, l}
\end{equation}
一个例子见库仑函数\upref{CulmF}, 库仑函数是类氢原子散射态的径向波函数, $V(r) = Z/r$.

\subsection{径向方程的推导}
使用球坐标的拉普拉斯算子(\autoref{SphNab_eq4}~\upref{SphNab}) 可以将哈密顿算符表示为
\begin{equation}
H = -\frac{1}{2m} \laplacian + V(\bvec r) =  K_r + \frac{L^2}{2mr^2} + V(\bvec r)
\end{equation}
其中径向动量算符和角动量平方算符为(\autoref{SphAM_eq1}~\upref{SphAM})分别为
\begin{equation}
K_r =-\frac{1}{2m} \laplacian_r =  - \frac{1}{2m} \qty(\pdv[2]{r} + \frac2r \pdv{r}) = -\frac{1}{2mr^2} \dv{r} \qty(r^2\pdv{r})
\end{equation}
\begin{equation}\label{RadSE_eq3}
L^2 = -\laplacian_\Omega = -\qty[ \frac{1}{\sin\theta} \pdv{\theta} \qty(\sin \theta \pdv{u}{\theta}) + \frac{1}{\sin^2 \theta} \pdv[2]{u}{\phi}]
\end{equation}
注意角动量算符不含 $r$.

定态薛定谔方程\autoref{RadSE_eq2} 变为
\begin{equation}
\qty(K_r + \frac{L^2}{2mr^2} + V - E)\Psi(\bvec r) = 0
\end{equation}
我们假设势能函数只与粒子到原点的距离有关, 即 $V = V(r)$. 两边乘以 $r^2$ 可以将 $r$ 与角向变量 $\theta, \phi$ (简写为 $\uvec r$)分离, 令 $\Psi = R(r)Y(\uvec r)$.

解得 $Y(\uvec r)$ 为球谐函数 $Y_{l,m}(\uvec r)$ 满足
\begin{equation}
L^2 Y_{l,m}(\uvec r) = l(l+1) Y_{l,m}(\uvec r)
\end{equation}
分离变量后 $R(r)$ 满足的方程一般被称为\textbf{径向薛定谔方程}
\begin{equation}\label{RadSE_eq6}
K_r R_l(r) + \qty[V(r) + \frac{l(l+1)}{2mr^2} ]R_l(r) = ER(r)
\end{equation}
我们可以通过变量替换将其化为更简洁的形式. 实际上是不同 $l$ 的一系列方程.

定义
\begin{equation}
\psi_l(r) = r R_l(r)
\end{equation}
代入\autoref{RadSE_eq6}, 第一项变为
\begin{equation}
K_r R_l =  - \frac{1}{2m} \qty(\dv[2]{R_l}{r} + \frac2r \dv{R_l}{r}) = -\frac{1}{2mr^2} \dv{r} \qty(r^2\dv{R_l}{r}) =  -\frac{1}{2mr} \dv[2]{\psi_l}{r}
\end{equation}
所以\autoref{RadSE_eq6} 两边乘 $r$ 后化简就得到\autoref{RadSE_eq1}. 这是径向薛定谔方程更常见的形式. 我们可以把该方程想象成是求解一维势能中粒子的能量本征态(束缚态或者散射态). 方括号中的势能可以称为\textbf{一维等效势能}, 取决于 $l$ 量子数.
