% 函数的连续性
% 极限|微积分|连续函数|左连续|右连续

\pentry{极限\upref{Lim}}

简单来说, \textbf{连续函数}定义为: 在某个区间内, 函数曲线是连续的. 例如常见的 $\sin x$, $\exp x$, $x^2$ 都在整个实数域上连续, 又例如 $\ln x$ 和 $1/x$ 在区间 $(0, \infty)$ 上连续, $\tan x$ 在所有 $x_n = (1/2 + n)\pi$($n$ 为自然数)处不连续, $1/x$ 在 $x = 0$ 处不连续. 但这只是一些简单的情况. 在一些情况下这种判断方法则显得不严谨, 例如 $\sin(1/x)$ 在原点处的连续性(不连续)根据这个定义不好判断. 所以我们需要一个更严谨的定义.

首先我们要讨论函数在一个点附近是否是连续的.这个概念的思想核心是,在函数曲线的某一点附近$(x_0, f(x_0))$,无论我们要求$f(x)$有多接近$f(x_0)$,只要$x$足够靠近$x_0$,就一定能满足条件.比如说,如果定义函数$f$为当$x<0$的时候,$f(x)=0$,其它时候$f(x)=1$,那么在$x=0$这一点处$f$就是跳跃的.如果我们要求的接近程度小于$1$,那么无论$x$多接近$0$,只要$x<0$,$f(x)$和$f(0)=1$的距离就永远满足不了需要.

准确地描述以上“连续”的概念,如下所示:

\begin{definition}{函数在一点的连续性}
函数 $f(x)$ 在某点 $x = x_0$ 处\textbf{连续}的定义是
函数$f(x)$在某点$x=x_0$处连续,当且仅当对于任何精度要求$\epsilon>0$,我们都可以找到一个对应的范围$\delta>0$,使得只要$|x-x_0|<\delta$,就有$|f(x)-f(x_0)|<\epsilon$.如果一个函数在某区间的所有点都连续, 我们就说它\textbf{在这个区间连续}.

另一个等价说法是常见的:
\begin{equation}
\lim_{x \to x_0} f(x) = f(x_0)
\end{equation}
\end{definition}

注意这里要求 $x$ 从左边和右边趋近于 $x_0$ 时的极限(即左极限和右极限)都成立. % 未完成: 极限词条中介绍左极限和右极限

% 举例: 分段函数在断点不满足该要求
