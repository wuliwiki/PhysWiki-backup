% 数据
% keys 数据|数据集|机器学习
% license Xiao
% type Tutor

\pentry{矢量空间\upref{LSpace}}

\textbf{数据}(Data)是对客观事物各种信息、各种特征的记录,能够反应事物的各种性质和状态。机器学习是基于经验的学习。对于机器学习来说,数据就是经验,是机器学习的学习素材。机器学习算法所做的事情,就是从大量已知数据中寻找出背后的规律——\textbf{数据分布}(Data distribution),从而能够预测未来可能发生的事情。

\textbf{数据集}(Data set)是多条数据记录所构成的集合。在实际应用中,通常可以以表格的形式来表示数据集。

\begin{table}[ht]
\centering
\caption{睡眠数据集}\label{tab_datast2}
\begin{tabular}{|c|c|c|c|c|c|c|c|c|c|c|}
\hline
编号 & 性别 & 年龄 & 职业 & 睡眠时间(小时) & BMI指数 & 心率 & 舒张压 & 收缩压 & 每日走路步数 & 睡眠障碍 \\\hline
1 & 男 & 27 & 软件工程师 & 6.1 & 超重 & 77 & 83 & 126 & 4200 & 无 \\
\hline
2 & 男 & 28 & 医生 & 6.2 & 正常 & 75 & 80 & 125 & 10000 & 无 \\
\hline
3 & 女 & 30 & 护士 & 6.4 & 正常 & 78 & 86 & 130 & 4100 & 睡眠暂停 \\
\hline
4 & 男 & 29 & 教师 & 6.3 & 肥胖 & 82 & 90 & 140 & 3500 & 失眠 \\
\hline
\end{tabular}
\end{table}

上述表格就是关于人的睡眠的一个简单数据集。表格中的每一行(除表头外)都是对于记录对象的描述,称为一个\textbf{实例}(Instance)或者\textbf{样本}(Sample)。表格第一行(即表头)所表示的是能够反应样本在某个方面的性质,例如,睡眠时间、心率、每日步数,称为\textbf{属性}(attribute)或者\textbf{特征}(feature)。每个样本在每个特征上都有取值,该取值称为该特征的\textbf{特征值}(feature value)。
比如,表1中第2个样本的睡眠时间这个特征的值为6.2。

站在稍微抽象一点的数学角度上,数据集中的每一个特征(例如表1中的每一列)都可以视为一个坐标分量,对应于一个坐标轴,那么全部特征就能够构成一个向量空间。我们称此向量空间为\textbf{特征空间}(feature space),\textbf{属性空间}(attribute space)、或者\textbf{输入空间}(input space)。一个样本(或实例)就可以被称为\textbf{特征向量}(feature vector)[1]。表1所表示的睡眠数据集有$10$个特征(不包括编号,因为编号实际上跟样本本身的特征没有任何关系),那么这$10$个特征就可以构造一个$10$维向量空间。

我们可以用$D$来表示整个数据集,$\bvec x$表示一个样本,加上上标后,$\bvec x^{(i)}$表示第$i$个样本。那么,整个数据集可以写成:$D=\{\bvec x^{(1)}, \bvec x^{(2)}, ..., \bvec x^{(n)}\}$。用$x$表示样本的特征值,带上下标之后,$x_i$表示第$i$个特征的值,再连同上标,$x^{(i)}_j$表示第$i$个样本的第$j$个特征的值。那么,第$i$个样本可以表示为:$\bvec x^{(i)}=(x^{(i)}_1, x^{(i)}_2, ..., x^{(i)}_j, ..., x^{(i)}_d)$。

机器学习的基本任务是发现数据中的各种关系,从而能够起到预测的作用。假设在上面那个睡眠数据集(表1)的例子中,我们想要用其它特征,比如年龄、心率、每日走路步数、睡眠时间等特征来预测“睡眠障碍”,那么,此时“睡眠障碍”这个特征就被称为\textbf{标签}(label),或者\textbf{标记}。学习算法的目的其实就是要建立起一个从其它特征到标签的映射,也就是所谓\textbf{模型}(model)。我们用$y_i$表示第$i$个样本的标签,第$i$个样本可以记为:$\bvec x=x^{(i)}, y^{(i)}$。所有标签构成的集合记为$Y$,称为\textbf{标签空间}(label space),或者\textbf{输出空间}(output space)[1]。


\textbf{参考文献:}
\begin{enumerate}
\item 周志华. 机器学习[M]. 北京:清华大学出版社. 2016, pp. 2
\end{enumerate}
