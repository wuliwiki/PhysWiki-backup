% 计算机网络笔记
% license Xiao
% type Note

\begin{issues}
\issueDraft
\end{issues}

参考: 图解计算机网络(小林)

\begin{itemize}
\item 以浏览器访问小时百科 \verb`https://wuli.wiki/index.html` 为例。 浏览器发送了一个 \verb`http` 协议的包, 发到了服务器的 Nginx 程序, 然后 Nginx 返回一个 http 包,里面包含了 \verb`index.html` 文件。 但这看似简单的过程中,要确保这个包要送到想要的应用,是非常复杂的。
\item 一个电脑可能有一张或多张网卡,一张网卡提供一个以太网插口,具有一个写死到硬件中的 MAC 地址,还有一个每那么固定的 IP 地址。 MAC 地址用于两个不同电脑的两张网卡通信,也用于交换机决定把包转发到哪个口。
\item 一个电脑可能同时运行多个程序,所以一个程序直接占用一个网卡太浪费了,还需要区分(虚拟的)端口。 端口可以数以万计,一个端口一次只能被一个程序占用(监听)。这些端口就是 TCP 协议实现的。某些特定数字的端口一般被作为一些常用软件的默认端口,但也不一定。
\item 先假定 IP 都是公网 IP,互联网每张网卡都有一个。 所以浏览器想发包给 Nginx,就直接指定 IP 和端口即可精确送达。 但要先把域名 \verb`wuli.wiki` 转为 IP, 这需要对互联网的 DNS 服务器进行询问,但我们先跳过这一步,假设已经问到了 IP。 \verb`https` 协议默认的端口是 443,所以现在浏览器把这个包填好 IP 和端口发给操作系统。
\item 操作系统有路由表,若有多块网卡,可以根据 IP 地址的范围决定下一步发到哪块网卡(获取其 IP,注意不是最终 IP 而是下一个 IP), 随机选一个端口专门用于本次连接(待会要用同一个端口接收返回的包)。 在包中除了浏览器添加的收件人(IP+端口)还添加上寄件人(本地IP+端口)。
\item 但网卡要找到下一个(别的机器的)网卡还需要 MAC 地址。 如何获取呢?
\item 电脑网卡通常用以太网线或者 wifi 直接连接到交换机。 交换机比 IP 协议更底层, 它只根据 MAC 地址把包发到下一张网卡。 网线插入交换机时,它会在一张表记录那个端口对应哪个 MAC 地址。 所以交换机没有 IP。
\item 事实上如果局域网只有两台电脑可以把它们直连,效果和连到一台交换机是一样的。
\item 若操作系统通过路由表知道了下一块网卡的 IP 但不知道 MAC, 就用 ARP 协议对所有同一个局域网的 IP 进行广播询问, 然后具有该 IP 的网卡就会用 ARP 协议回答 MAC 地址。
\item 局域网本身不是互联网,如果不连接互联网,就只能局域网内部通信,上不了 \verb`wuli.wiki`。 其实互联网本质上就是很多局域网通过路由器连起来的。
\item 路由器在 IP 层面工作,不同于交换机。 路由器的每一个端口就是一个网卡,具有自己的 IP,每个端口连接不同的局域网。 路由器相当于一个电脑, 里面有路由表。 收到一个包后, 就根据包中的最终 IP 地址决定把它转到哪个局域网的哪台机器(通常是下一个路由器),最后到达小时百科服务器的网卡。
\item 但 IP 地址不够用了,所以就有了 NAT。
\end{itemize}


======= 旧内容 ========

参考书: An Introduction to Computer Networks (Peter L Dordal)

\begin{figure}[ht]
\centering
\includegraphics[width=14.25cm]{./figures/9b54e436a0ab2af4.png}
\caption{4 层/7 层模型} \label{fig_NetNt_1}
\end{figure}

\begin{itemize}
\item \textbf{Local Area Network (LAN)}: 物理上的网络, 链接, 家庭, 学校或者公司中的设备。 例如 wifi
\item \textbf{Internet Protocol (IP)}: 将 LAN 连接成 Internet
\item \textbf{layers}: LAN 是 link layer, IP 是 internet layer 或 network layer, TCP 是 transport layer
\item 以上三个 layer 加上 application layer 叫做 \textbf{4 layer model}
\item 每个 layer 就是一个 programming interface 和 library, 只和上面和下面的 layer 直接交流
\item 每层 layer 调用下一层的 layer。 软件调用 TCP 库发送数据, TCP 调用 IP 库, IP 库使用 LAN。 软件库不会直接使用 IP 或者 LAN
\item LAN 也可以细分为两层, 底层是纯物理的, 模拟电路数字电路, 光纤, 导线。 上面一层是数据层面的对包的操作。
\item \textbf{data rate}: 数据(比特)传输的速率
\item \textbf{throughput}: 净传输量, 考虑 overhead 等
\item \textbf{bandwidth}: 可以是以上两个中的一个, 但该书中主要指 data rate。 这个词是从无线电来的。
\item \textbf{goodput}: aplication-layer throughput
\item Kbps 和 Mbps 中的 b 是 bits, K 是 $10^3$, M 是 $10^6$。
\item KB 和 MB 是 1024 bytes 和 1024*1024 bytes, 新符号也记为 KiB 和 MiB
\item packet 是传输过程中的一个整体, 每个 packet 的 header 中有 delivery information (address), 类比快递的包裹
\item 最大的 packet 大小由 LAN 决定, Ethernet 是 1500 byte
\item 大包的 LAN 如何传输数据到小包 LAN 呢? 这个由 IP 来解决
\item 一般来说, 每一个 layer 会在包前面加一个 header。 Ethernet header 一般 14 byte, IP header 一般 20 byte, TCP 20 bytes
\end{itemize}

\subsection{杂}
参考\href{https://www.cnet.com/how-to/home-networking-explained-part-1-heres-the-url-for-you/}{这篇文章}。
\begin{itemize}
\item \textbf{packet-switched network} 使用 package 可以让不同设备间在同一条线路上通讯。 这种叫做方式 connectionless
\item 如果一条线路只给两个设备传输信息, 就是 dedicated
\item Internet 基本上是一个 connectionless network.
\item \textbf{circuit-switched network} 例如电话网络, 当两个人通话时, communication circuit (path) 被设立起来在通话过程中专门给两个人使用。
\item router (路由器) 在两个或者多个 packet-switched network 之间传递信息。 router 查看 packet 的 IP 地址, 决定传递 packet 的最佳路线, 并向该路线发送 packet。
\item router 是一种常见的 gateway
\item \textbf{wide area network (WAN)} 是地理上长距离的 private 网络, 将若干个 LAN 连接起来。 例如公司用 WAN 将不同的分公司连接起来。 通常使用 router 将 LAN 连接成 WAN。
\item \textbf{virtual private network (VPN)} 是一个程序, 在一个没那么安全的网络上(例如 internet)创建一个安全加密的连接。 VPN 使用 tunneling protocols 在一端加密然后在另一端解密。 地址同样被加密。 VPN 是 point-to-point connection
\item \textbf{Ethernet} 以太网, 是一种连接 LAN 的传统技术 (LAN 基本上就是用 Ethernet), 对应的 layer 是 link layer。  IEEE 802.3 系列标准
\item Power line networking: 基本上就是用墙上的电源插座传递上网信号, 基本可以达到 Gigabit 网线的一半
\item Media Access Control (MAC) 地址是硬件的地址, 每个设备有一个永远不变的 MAC 地址 (烧在硬件里面的)。 例如 00:70:40:42:8A:60。 前三个数字通常是制造商的编号
\end{itemize}

\subsection{Wi-Fi}
\begin{itemize}
\item \textbf{wireless LAN (WLAN)} , IEEE 802.11 系列标准规定 WLAN 使用 Ethernet protocol, 简称 wifi
\item \textbf{dynamic host configuration Protocol (DHCP)} 例如公司的像衣柜一样的服务器, 用于动态给各个设备分发 IP 地址
\item \textbf{Access point (AP)} 就是发出 wifi 的设备。 可以单独购买, 连接路由器, 发射无线网络。 但是一般的做法是直接买无线路由器
\item \textbf{dual band} 路由器有两个 AP, 不同频率, 一般是 2.4GHz 和 5GHz
\item WLAN client 或 wifi client 就是连接到 wifi 的设备。 可以假设这些设备有隐形的 port 和 cable
\item wifi 加密方式: \textbf{WEP} (淘汰), \textbf{WPA, WPA2} (最安全)。 后两者的两种加密方式为 Temporal Key Integrity Protocol (TKIP) 和 Advanced Encryption Standard (AES)
\end{itemize}

\subsection{路由器}
\begin{itemize}
\item 【错】严格来说我们可以将两台电脑直接用网线连起来, 但是需要使用特殊的 crossover cable, 以及手动设置 IP 地址, 比较麻烦, 所以常见的做法还是把所有设备通过路由器相连。
\item 基本所有市场上的路由器都有 web interface, 即用浏览器设置
\item 路由器上只有一个 WAN 接口, 用来连接墙上的插头(如果不需要 internet 则不需要连接)
\item 路由器有多个 LAN 接口, 许多路由器有 4 个 LAN 口
\item 如果需要更多 LAN 接口, 可以使用 switch 或者 hub 来完成, 一般家用路由器可以最多连接 250 个设备
\item Ethernet 一般有两种, Ethernet (理论 13 MB/s, 实际约 8 MB/s) 和 Gigabit Ethernet (150 MB/s, 实际约 45-100MB/s)
\item 一般使用的网线是 category 5 (CAT5), 其实大部分是 CAT5e, 支持 Gigabit。 CAT6 和 CAT5 的性能基本差不多
\item 要使用 Gigabit Ethernet, 路由器, 电脑和网线都需要支持才行。 如果连接路由器的一台设备是 Gigabit, 另一台是普通 Ethernet, 速度就会等于后者的速度
\item hub 将所有的 LAN 端口 share 一个 channel, 而 switch 给每个端口一个 channel
\item hub 连接的设备越多, 速度越慢, 而 switch 不会变。 现在基本用 switch。
\item reset 按钮用户恢复默认设置
\item 连接好以后, 可以在浏览器中输入 IP 地址 92.168.x.1 访问浏览器设置页面, x 一般是 0, 1, 2, 3, 10 或者 11
\end{itemize}

\subsubsection{路由器设置}
\begin{itemize}
\item 登录用户名一般是 admin, 密码一般是 admin, password, default 或者 1234. 也可以查看说明书或者机身
\item 在 windows 上要查看 router 的 ip, 在 cmd 用 ipconfig 命令, 会在显示 Default Gateway
\item login 以后注意要改密码
\item 设置板块一般有 Wireless, WAN, LAN, password, System
\item Wireless 板块: 设置 wifi 名称和密码
\item WAN 设置: 大部分时候使用默认, 除非 ISP 网络提供商有特殊要求
\item LAN 设置: 可以设置路由器的 IP 地址。 可以设置设备的本地 IP 地址范围。 也可以设置 DHCP reservation list,该列表中 IP 地址是静态的。 
\end{itemize}
