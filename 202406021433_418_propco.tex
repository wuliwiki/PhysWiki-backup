% 命题的连接词
% license Usr
% type Tutor
\pentry{命题\nref{nod_prop}}{nod_2565}

\begin{definition}{否定}
对于一个命题 $p$, $\neg p$(或 $\sim p$)表示其否定。例如 $p$ 代表“我喜欢踢足球”,则 $\neg p$ 代表“我\textbf{不}喜欢踢足球”。
\end{definition}

下面给出一个命题的否定的真值表(即真值的对应关系的表格)。
\begin{table}[ht]
\centering
\caption{命题的否定的真值表}\label{tab_propco1}
\begin{tabular}{|c|c|}
\hline
$p$ & $\neg p$ \\
\hline
$1$ & $0$ \\
\hline
$0$ & $1$ \\
\hline
\end{tabular}
\end{table}


\begin{definition}{合取}
对于命题 $p$ 与命题 $q$,$p \land q$ 表示这两个命题的\textbf{合取},表示要求同时成立。$(p \land q)$ 为真当且仅当 $p$ 与 $q$ 两者都为真。
\end{definition}

\begin{definition}{析取}
对于命题 $p$ 与命题 $q$,$p \lor q$ 表示这两个命题的\textbf{析取},表示要求至少一者成立。$(p \lor q)$ 为真需 $p$ 与 $q$ 中至少一者为真。
\end{definition}

\begin{definition}{蕴含}
对于两命题 $p$ 与 $q$,复合命题“若 $p$,则 $q$”称为 $p$ 对 $q$ 的\textbf{蕴涵式},记做 $p \to q$。$p \to q$ 为假当且仅当 $p$ 为真时 $q$ 为假。其中 $p$ 称为条件而 $q$ 称为结论。
\end{definition}

\begin{definition}{等价}
对于两命题 $p$ 与 $q$,$p \leftrightarrow q$ 表示这两个命题的\textbf{等价},表示两个命题是等价的。
\end{definition}

下面给出以上二元(即两个命题间的)连接词的真值表。
\begin{table}[ht]
\centering
\caption{二元连接词的真值表}\label{tab_propco2}
\begin{tabular}{|c|c|c|c|c|c|}
\hline
$p$ & $q$ & $p \land q$ & $p \lor q$ & $p \to q$ & $p \leftrightarrow q$ \\
\hline
$0$ & $0$ & $0$ & $0$ & $1$ & $1$ \\
\hline
$0$ & $1$ & $0$ & $1$ & $1$ & $0$ \\
\hline
$1$ & $0$ & $0$ & $1$ & $0$ & $0$ \\
\hline
$1$ & $1$ & $1$ & $1$ & $1$ & $1$ \\
\hline
\end{tabular}
\end{table}

\begin{theorem}{逆否命题}
对于两个命题 $p \rightarrow q$ 的蕴含式 $p \rightarrow q$,与蕴含式的逆否命题 $(\neg q) \rightarrow (\neg p)$ 等价。
\end{theorem}