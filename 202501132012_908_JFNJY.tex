% 约翰·福布斯·纳什(综述)
% license CCBYSA3
% type Wiki

本文根据 CC-BY-SA 协议转载翻译自维基百科\href{https://en.wikipedia.org/wiki/John_Forbes_Nash_Jr.}{相关文章}。

\begin{figure}[ht]
\centering
\includegraphics[width=6cm]{./figures/1ee7155e152489e2.png}
\caption{纳什在2000年代} \label{fig_JFNJY_1}
\end{figure}
约翰·福布斯·纳什 Jr.(1928年6月13日—2015年5月23日),以约翰·纳什为人所知,是一位美国数学家,对博弈论、实代数几何、微分几何和偏微分方程作出了重要贡献。[1][2] 纳什与博弈论学者约翰·哈萨尼和赖因哈德·塞尔滕共同获得了1994年诺贝尔经济学奖。2015年,他与路易斯·尼伦贝格因在偏微分方程领域的贡献而获得阿贝尔奖。

作为普林斯顿大学数学系的研究生,纳什引入了许多概念(包括纳什均衡和纳什议价解),这些概念如今被认为是博弈论及其在各学科应用中的核心内容。在1950年代,纳什通过求解一组出现在黎曼几何中的非线性偏微分方程,发现并证明了纳什嵌入定理。这项工作也引入了纳什-莫泽定理的初步形式,后来被美国数学学会授予勒罗伊·P·斯蒂尔奖,以表彰其对研究的开创性贡献。恩里奥·德·乔尔吉与纳什通过不同的方法发现了一些成果,为系统地理解椭圆和抛物型偏微分方程铺平了道路。他们的德·乔尔吉-纳什定理解决了希尔伯特第十九问题,即变分学中的正则性问题,这个问题已经成为一个著名的悬而未解的难题近六十年。

1959年,纳什开始显现出精神疾病的明显迹象,并在精神病医院接受了多年的精神分裂症治疗。1970年后,他的病情逐渐好转,使他能够在1980年代中期重新投入学术工作。[3]

纳什的生平成为了西尔维娅·纳萨尔于1998年出版的传记《美丽心灵》的主题,书中讲述了他与病魔的斗争以及他如何恢复健康,这一过程也成为了同名电影的基础,电影由朗·霍华德执导,拉塞尔·克劳饰演纳什。[4][5][6]
\subsection{早年生活与教育}
约翰·福布斯·纳什 Jr.于1928年6月13日出生在西弗吉尼亚州的蓝田(Bluefield)。他的父亲和同名的祖父约翰·福布斯·纳什Sr.是阿巴拉契亚电力公司的电气工程师。母亲玛格丽特·维吉尼亚(原姓马丁)·纳什在结婚前曾是一名学校教师。他在圣公会教堂接受了洗礼。[7] 他有一个妹妹,玛莎(生于1930年11月16日)。[8]

纳什上过幼儿园和公立学校,他的父母和祖父母为他提供了大量书籍供他阅读和学习。[8] 纳什的父母尽力为他提供更多教育机会,在他高中最后一年时,为他安排了在附近的蓝田学院(现为蓝田大学)参加高级数学课程的机会。他通过乔治·韦斯汀豪斯奖学金(George Westinghouse Scholarship)全额资助,进入卡内基技术学院(后来成为卡内基梅隆大学),最初主修化学工程。后来他转向化学专业,并最终在他的导师约翰·莱顿·辛格(John Lighton Synge)的建议下,改学数学。1948年,他以数学学士和硕士学位毕业后,接受了普林斯顿大学的奖学金,继续攻读数学和科学的研究生学位。[8]

纳什的导师、前卡内基教授理查德·达芬(Richard Duffin)为纳什申请普林斯顿大学时写了一封推荐信,信中称他为“数学天才”。[9][10] 纳什也收到了哈佛大学的录取通知。然而,普林斯顿数学系的主任所罗门·莱夫谢茨(Solomon Lefschetz)为他提供了约翰·S·肯尼迪奖学金,成功说服纳什普林斯顿更看重他。[11] 此外,他也更倾向于选择普林斯顿,因为该校距离他在蓝田的家较近。[8] 在普林斯顿,他开始研究均衡理论,后来这个理论被称为纳什均衡。[12]
\subsection{研究贡献}
\begin{figure}[ht]
\centering
\includegraphics[width=6cm]{./figures/282eb12009d8eaac.png}
\caption{纳什在2006年11月于德国科隆举行的博弈论会议上} \label{fig_JFNJY_2}
\end{figure}
纳什的出版物并不多,尽管他的许多论文被认为是各自领域的重要里程碑。[13] 作为普林斯顿大学的研究生,他对博弈论和实代数几何做出了基础性的贡献。作为麻省理工学院的博士后,纳什转向了微分几何。尽管纳什在微分几何方面的成果使用了几何语言表达,但这些工作几乎完全与偏微分方程的数学分析有关。[14] 在证明了他的两个等距嵌入定理后,纳什转向了直接涉及偏微分方程的研究,并发现并证明了德·乔尔吉–纳什定理,从而解决了希尔伯特第十九问题的一种形式。

2011年,美国国家安全局解密了纳什在1950年代写的信件,信中他提出了一种新的加密–解密机器。[15] 这些信件显示,纳什早已预见到许多现代密码学的概念,这些概念基于计算复杂性。[16]
\subsubsection{博弈论}  
纳什于1950年获得博士学位,他的论文题为《非合作博弈》,长达28页。[17][18] 这篇论文是在博士导师阿尔伯特·W·塔克的指导下写成的,论文中定义并探讨了纳什均衡,这是非合作博弈中的一个关键概念。论文的一个版本一年后发表于《数学年刊》[19]。在1950年代初期,纳什在博弈论的多个相关概念上进行了研究,包括合作博弈理论。[20] 因其贡献,纳什成为1994年诺贝尔经济学奖的得主之一。
\subsubsection{实数代数几何}  
1949年,纳什还在研究生期间时,发现了实数代数几何领域中的一项新成果。[21] 他在1950年国际数学家大会上发表了这一定理,尽管他当时还没有完成其证明的细节。[22] 纳什的定理于1951年10月最终定稿,并提交给《数学年刊》[23]。自1930年代以来,数学界已知每个闭合光滑流形都可以与欧几里得空间中某些光滑函数的零集合进行微分同胚。在他的工作中,纳什证明了这些光滑函数可以取为多项式。[24] 这一结果被广泛视为惊人的发现,[21] 因为光滑函数和光滑流形的类通常比多项式类更加灵活。纳什的证明引入了现在被称为“纳什函数”和“纳什流形”的概念,这些概念自那时以来一直是实数代数几何学中的重要研究对象。[24][25] 纳什的定理后来被迈克尔·阿尔廷和巴里·马祖应用于动力系统的研究,他们将纳什的多项式逼近与贝祖定理结合起来。[26][27]
\subsubsection{微分几何} 
在麻省理工学院的博士后期间,纳什渴望找到具有广泛影响力的数学问题进行研究。[28] 他从微分几何学家沃伦·安布罗斯那里了解到了一种猜想,即任何黎曼流形都可以是欧几里得空间中某个子流形的等距嵌入。纳什证明这一猜想的结果现在被称为纳什嵌入定理,其中第二个定理被米哈伊尔·格罗莫夫称为“20世纪数学的主要成就之一”[29]。

纳什的第一个嵌入定理是在1953年发现的。[28] 他证明了任何黎曼流形都可以通过一个连续可微的映射嵌入到欧几里得空间中。[30] 纳什的构造允许嵌入的共维度非常小,这使得在许多情况下,逻辑上无法存在一个高可微的等距嵌入。(基于纳什的方法,尼古拉斯·库伊珀很快找到了更小的共维度,这一改进结果通常被称为纳什–库伊珀定理。)因此,纳什的嵌入限制在低可微的设定中。出于这个原因,纳什的结果在微分几何领域中略微偏离主流,因为在通常的分析中,高可微性是很重要的。[31][32]

然而,纳什的工作逻辑已被证明在数学分析中的许多其他领域中非常有用。从卡米洛·德·莱利斯和拉斯洛·塞凯利迪的工作开始,纳什证明中的思想被应用于流体力学中欧拉方程的湍流解的构造。[33][34] 在1970年代,米哈伊尔·格罗莫夫将纳什的思想发展成了凸集积分的通用框架,[32] 这一框架被斯特凡·穆勒和弗拉基米尔·什韦拉克应用于构造对变分法中的希尔伯特第十九问题广义形式的反例。[35]

纳什发现构造光滑可微的等距嵌入比预期的要困难得多。[28] 然而,在约一年半的密集工作后,他成功地证明了第二个纳什嵌入定理。[36] 证明这一第二定理的思想在很大程度上与证明第一个定理时使用的思想是分开的。证明的核心是等距嵌入的隐函数定理。由于涉及到常规性丧失现象,传统的隐函数定理无法应用。纳什通过将等距嵌入沿着普通微分方程变形,并不断注入额外的常规性,解决了这一问题,这被视为数学分析中的一种根本性新技术。[37] 纳什的论文在1999年获得了勒罗伊·P·斯蒂尔奖,该奖项表彰他在解决常规性丧失问题上的“最原始的思想”,被誉为“本世纪数学分析中的伟大成就之一”[14]。格罗莫夫评论道:[29]

你必须是分析领域的新手,或者像纳什那样的天才,才能相信这样的事情可能是真的,或者能找到一个非平凡的应用。

由于于尔根·莫泽扩展了纳什的思想,应用于其他问题(特别是在天体力学中),由此产生的隐函数定理被称为纳什–莫泽定理。它已被多位其他学者扩展和概括,包括格罗莫夫、理查德·汉密尔顿、拉尔斯·霍尔曼德、雅各布·施瓦茨和爱德华·泽德纳。[32][37] 纳什本人在解析函数的背景下分析了这个问题。[38] 施瓦茨后来评论道,纳什的思想“不仅新颖,而且非常神秘”,而且“很难搞清楚其中的真相。”[28] 格罗莫夫指出:[29]

纳什在解决经典数学问题时,面对的是一些没有人能够做的困难问题,甚至没有人能想象如何去做。……纳什在构造等距嵌入的过程中发现的东西远不是“经典的”——它改变了我们对分析和微分几何基本逻辑的理解。从经典的角度看,纳什在论文中取得的成就就像他生活的故事一样,几乎是不可能的……[H]他关于等距浸入的工作……打开了一个新的数学世界,这个世界在我们眼前展开,朝着未知的方向延伸,仍然等待着我们去探索。
\subsubsection{偏微分方程}
在纽约市的库朗数学研究所度过一段时间时,路易·尼伦伯格(Louis Nirenberg)告诉纳什,关于椭圆型偏微分方程领域的一个著名猜想。[39] 1938年,查尔斯·莫雷(Charles Morrey)为二维独立变量的函数证明了一个基本的椭圆型正则性结果,但对于多个变量的函数,类似的结果一直难以获得。在与尼伦伯格和拉尔斯·霍尔曼德的广泛讨论之后,纳什成功地将莫雷的结果扩展到不仅仅是多于两个变量的函数,还包括抛物型偏微分方程的情形。[40] 在他的工作中,像莫雷一样,通过不假设方程系数的任何可微性,实现在解的连续性上的一致控制。纳什不等式是他工作过程中发现的一个特别结果(纳什将其证明归功于以利亚·斯坦),该不等式在其他领域也有广泛应用。[41][42][43][44]

不久后,纳什从刚从意大利回来的保罗·加拉比迪安(Paul Garabedian)那里得知,当时尚不为人知的恩尼奥·德·乔治(Ennio De Giorgi)已经为椭圆型偏微分方程找到了几乎相同的结果。[39] 德·乔治和纳什的方法几乎没有关系,尽管纳什的方法在应用于椭圆型和抛物型方程时要更为强大。几年后,受到德·乔治方法的启发,于尔根·莫泽找到了不同的途径来得出相同的结果,这一系列成果现被称为德·乔治–纳什定理或德·乔治–纳什–莫泽理论(与纳什–莫泽定理不同)。德·乔治和莫泽的方法在接下来的几年里尤为具有影响力,并通过奥尔加·拉季任斯卡娅(Olga Ladyzhenskaya)、詹姆斯·塞林(James Serrin)和尼尔·特鲁丁格(Neil Trudinger)等人的研究得到了进一步发展。[45][46] 他们的工作,主要基于在偏微分方程弱形式中巧妙选择测试函数,与纳什的工作形成鲜明对比,后者基于热核的分析。纳什对德·乔治–纳什理论的处理方法,后来被尤金·法贝斯(Eugene Fabes)和丹尼尔·斯特鲁克(Daniel Stroock)重新审视,并启动了原本由德·乔治和莫泽方法得到的结果的重新推导和扩展。[41][47]

由于变分法中的许多泛函的极小化解满足椭圆型偏微分方程,希尔伯特第十九问题(关于这些极小化解的光滑性),在近六十年前的猜想,直接适用于德·乔治–纳什理论。纳什因其工作获得了立刻的认可,彼得·拉克斯(Peter Lax)称之为“一次天才之举”。[39] 纳什后来曾推测,如果不是德·乔治同时发现了这一结果,他本该在1958年获得声望极高的菲尔兹奖。[8] 尽管奖委员会的评选理由并不完全为人所知,而且并非完全基于数学上的优异性,[48] 但档案研究显示,纳什在委员会的投票中排名第三,仅次于同年获得该奖的两位数学家(克劳斯·罗斯(Klaus Roth)和雷内·汤姆(René Thom))。[49]
\subsection{精神疾病}
尽管纳什的精神疾病最初表现为偏执症,但他的妻子后来描述他的行为为不稳定。纳什认为,所有戴红领带的男人都是针对他的共产主义阴谋的一部分。他曾向华盛顿特区的使馆寄信,宣称他们正在建立一个政府。[3][50] 纳什的心理问题影响到了他的职业生涯,1959年初,他在哥伦比亚大学举办的一次美国数学学会讲座中,原本计划展示黎曼假设的证明,但讲座内容却难以理解。听众中的同事立刻意识到他出现了问题。[51]

1959年4月,纳什被送入麦克林医院住院一个月。基于他的偏执妄想、幻觉和日益加剧的社交退缩,他被诊断为精神分裂症。[52][53] 1961年,纳什被送入新泽西州特伦顿的州立医院。[54] 在接下来的九年里,他曾多次住进精神病院,并接受了抗精神病药物治疗和胰岛素休克疗法。[53][55]

尽管有时他会服用医生开具的药物,纳什后来写道,他这么做只是出于压力。据纳什所说,电影《美丽心灵》错误地暗示他正在服用非典型的抗精神病药物。他将这种描述归咎于编剧,编剧担心电影会鼓励精神疾病患者停止服药。[56]

1970年以后,纳什再也没有服药,也再未住进医院。[57] 纳什逐渐康复。[58] 在当时的前妻拉尔德(Lardé)的鼓励下,纳什住在家里,并在普林斯顿大学数学系度过时光,即使他的精神状态较差,那里的人们仍然接受了他的怪癖。拉尔德将他的康复归功于维持“安静的生活”并获得社会支持。[3]

纳什将他所称的“精神困扰”的开始追溯到1959年初,那时他的妻子怀孕。他描述了从“科学理性思维”转变为“典型的精神病学诊断为‘精神分裂症’或‘偏执型精神分裂症’的妄想性思维”的过程。[8] 对纳什来说,这包括看自己为传讯者或有某种特殊职能,拥有支持者和反对者,以及隐藏的阴谋者,同时有被迫害的感觉,并寻找代表神圣启示的迹象。[59] 在他的精神病阶段,纳什也曾以第三人称称自己为“约翰·冯·纳索”(Johann von Nassau)。[60] 纳什认为他的妄想性思维与他的不快乐、对被认可的渴望以及他特有的思维方式有关,他说:“如果我思考得更正常一点,我就不会有好的科学想法。”他还说:“如果我没有任何压力,我想我就不会走上这条路。”[61]

纳什报告说,他在1964年开始听到声音,后来开始有意识地拒绝这些声音。[62] 经过长时间的非自愿住院治疗后,他才放弃了“梦似的妄想假设”,并开始“强迫理性”。在这样做之后,他暂时能够恢复数学家的生产性工作。到了1960年代末,他复发了。[63] 最终,他“智力上拒绝了”那些“受妄想影响”和“政治取向”的思维方式,认为这些思维方式是无用的努力。[8] 1995年,他表示,由于近30年的精神疾病,他并未发挥出自己的全部潜力。[64]

1994年,纳什写道:

我曾在新泽西的医院里待过五到八个月,每次都是非自愿住院,且总是试图通过法律争取释放。的确,在我住院够长时间后,最终会放弃我的妄想假设,重新把自己看作是一个更符合常规情况的人,并恢复数学研究。在这些被迫理性化的空隙中,我确实成功地做了一些有价值的数学研究。因此,我研究了《流体的柯西问题》;我所称的“纳什爆炸变换”这一概念,得到平生池教授的认可;以及“奇点的弧结构”和“隐函数问题的解析解的分析性”这两个问题。

但在60年代后期,我重新陷入了梦幻般的妄想假设,成了一个受妄想影响的思维者,但行为相对温和,因此避免了住院治疗和精神科医生的直接关注。

这样又过了些时间。然后,我逐渐开始智力上拒绝我曾经的部分受妄想影响的思维方式。最明显的是,我拒绝了政治取向的思维方式,认为它本质上是一种徒劳的知识努力。所以,现如今我似乎再次以一种理性科学家的思维方式思考问题。[8]
\subsection{认可与后期生涯}
\begin{figure}[ht]
\centering
\includegraphics[width=8cm]{./figures/f749852973beb6f8.png}
\caption{2011年,纳什照片} \label{fig_JFNJY_3}
\end{figure}
1978年,纳什因发现非合作均衡(现称为纳什均衡)而获得约翰·冯·诺伊曼理论奖。他于1999年获得了勒罗伊·P·斯蒂尔奖。

1994年,他与约翰·哈萨尼和赖因哈德·塞尔腾共同获得了诺贝尔经济学奖,以表彰他作为普林斯顿大学研究生在博弈论方面的工作。[65] 在1980年代末,纳什开始使用电子邮件,逐步与那些意识到他是约翰·纳什并认识到他的新研究有价值的数学家建立联系。他们成为了一个小组的核心成员,联系了瑞典银行的诺贝尔奖评审委员会,能够为纳什的精神健康和获得奖项的能力提供证明。[66]

纳什的后期工作涉及到高级博弈论的研究,包括部分代理理论,表明像他早期的事业一样,他偏爱选择自己的道路和问题。在1945年至1996年间,他发表了23篇科学论文。

纳什曾提出关于精神疾病的假设。他将不符合社会常规思维方式或“疯狂”并且无法适应通常社会职能的状态,类比为从经济学的角度来看“罢工”。他还在进化心理学中提出了关于表面上看似不标准的行为或角色可能具有潜在益处的观点。[67]

纳什批评了凯恩斯主义的货币经济学思想,这种思想允许中央银行实施货币政策。[68] 他提出了一种标准的“理想货币”,将其与“工业消费价格指数”挂钩,这比“劣质货币”更为稳定。他指出,自己关于货币和货币当局职能的思考与经济学家弗里德里希·哈耶克的看法相似。[69][68]

纳什于1999年获得了卡内基梅隆大学的荣誉学位——科学与技术博士学位,2003年获得那不勒斯大学费德里科二世的经济学荣誉学位,[70] 2007年获得安特卫普大学的经济学荣誉博士学位,2011年获得香港城市大学的科学荣誉博士学位,[71] 并在一场博弈论会议上担任主旨演讲人。[72] 纳什还获得了西弗吉尼亚州两所大学的荣誉博士学位:2003年获得查尔斯顿大学的荣誉博士学位,2006年获得西弗吉尼亚大学技术学院的荣誉博士学位。他还是许多活动的多产嘉宾演讲人,比如2005年在华威大学举办的华威经济学峰会。

纳什于2006年当选为美国哲学学会会员,[73] 2012年成为美国数学学会的会员。[74]

2015年5月19日,在他去世前几天,纳什与路易斯·尼伦伯格一起获得了2015年阿贝尔奖,由挪威国王哈拉尔五世在奥斯陆的颁奖典礼上颁发。[75]
\subsection{个人生活}  
1951年,麻省理工学院(MIT)聘用纳什担任数学系的C.L.E. Moore讲师。约一年后,纳什与他在住院期间认识的护士埃莉诺·斯蒂尔开始了一段关系。两人有了一个儿子,约翰·大卫·斯蒂尔(John David Stier),但当斯蒂尔告知纳什她怀孕时,纳什离开了她。根据纳什生平改编的电影《美丽心灵》在2002年奥斯卡奖前被批评未涉及他生活中的这一方面。据称他因认为斯蒂尔的社会地位低于他而抛弃了她。

1954年,纳什在加利福尼亚州圣莫尼卡市,在二十多岁时因在一次针对同性恋男性的诱捕行动中被逮捕,罪名是猥亵暴露。虽然这些指控后来被撤销,但他被取消了最高机密的安全许可,并被从兰德公司解雇,兰德公司曾聘请他担任顾问。

在与斯蒂尔分手不久后,纳什认识了来自萨尔瓦多的美国归化公民阿丽西亚·拉尔德·洛佩兹-哈里森。拉尔德毕业于麻省理工学院,主修物理学。他们于1957年2月结婚。尽管纳什是一个无神论者,但婚礼在一座圣公会教堂举行。1958年,纳什被任命为麻省理工学院的终身教授,而他早期的精神疾病症状很快显现出来。1959年春天,他辞去了麻省理工学院的职务。几个月后,他的儿子约翰·查尔斯·马丁·纳什出生。由于阿丽西亚认为纳什应该参与为孩子命名的决定,这个名字直到一年后才确定。由于精神疾病的压力,纳什与拉尔德于1963年离婚。1970年,纳什最终从医院出院后,作为房客住在拉尔德的家里。这段稳定的生活似乎对他有帮助,他学会了有意识地摒弃他的偏执妄想。普林斯顿大学允许他旁听课程。他继续从事数学工作,最终被允许重新教授课程。到了1990年代,拉尔德与纳什恢复了关系,并于2001年再婚。约翰·查尔斯·马丁·纳什获得了罗格斯大学的数学博士学位,并在成年后被诊断为精神分裂症。
\subsection{去世}  
2015年5月23日,纳什和他的妻子在新泽西州蒙罗镇的新泽西州收费公路发生车祸去世。当时,他们正在从挪威参加阿贝尔奖颁奖典礼返回的路上。出租车司机失控,撞上了护栏。由于两人都没有系安全带,乘客被抛出车外并当场死亡。纳什去世时是新泽西州的长期居民,留下了两个儿子:当时和父母一起生活的约翰·查尔斯·马丁·纳什,以及长子约翰·斯蒂尔。

纳什去世后,世界各地的科学和大众媒体纷纷刊登了他的讣告。除了对纳什的讣告外,《纽约时报》还发布了一篇文章,其中包含了纳什的语录,这些语录来自媒体和其他已发布的资料。语录中有纳什对自己生活和成就的反思。
\subsection{遗产}  
在1970年代的普林斯顿大学,纳什被称为“芬恩大厅的幽灵”(Princeton's mathematics center),他是一个神秘人物,常在深夜在黑板上潦草地写下深奥的方程式。

他在1983年雷贝卡·戈德斯坦(Rebecca Goldstein)的小说《心身问题》中被提及。

西尔维娅·纳萨尔(Sylvia Nasar)所著关于纳什的传记《美丽心灵》于1998年出版。改编电影《美丽心灵》于2001年上映,由朗·霍华德执导,拉塞尔·克劳饰演纳什;该片赢得了四项奥斯卡奖,包括最佳影片奖。克劳因饰演纳什获得了第59届金球奖最佳男主角(电影剧情类)奖和第55届英国电影学院奖最佳男演员奖。克劳还获得了第74届奥斯卡奖最佳男主角提名,丹泽尔·华盛顿凭借《训练日》获奖。
\subsection{奖项}  
\begin{itemize}
\item 1978年 – INFORMS约翰·冯·诺依曼理论奖(与卡尔顿·莱姆基共同获得),“表彰他们在博弈论中的杰出贡献”  
\item 1994年 – 瑞典皇家银行经济学奖(与约翰·哈萨尼和赖因哈德·塞尔滕共同获得),“表彰他们在非合作博弈理论中对均衡分析的开创性贡献”  
\item 1999年 – 莱罗伊·P·斯蒂尔奖,表彰其在1956年论文《黎曼流形的嵌入问题》中的开创性研究  
\item 2002年 – 运筹学与管理科学研究院研究员荣誉  
\item 2010年 – 双螺旋奖章  
\item 2015年 – 阿贝尔奖(与路易斯·尼伦伯格共同获得),“表彰他们在非线性偏微分方程理论及其在几何分析中的应用方面的卓越和开创性贡献”
\end{itemize}
\subsection{纪录片与访谈}  
\begin{itemize}
\item 华莱士,迈克(主持人)(2002年3月17日)。“约翰·纳什的美丽心灵”。《60分钟》, 第34季,第26集,CBS。  
\item 萨梅尔斯,马克(导演)(2002年4月28日)。“一场辉煌的疯狂”。《美国经历》,公共广播服务。文字记录。取自2022年10月11日。  
\item 纳什,约翰(2004年9月1日至4日)。“约翰·F·纳什 Jr” (访谈)。访谈者:玛丽卡·格里埃塞尔。诺贝尔奖外联。  
\item 纳什,约翰(2009年12月5日)。“一对一” (访谈)。访谈者:瑞兹·汗。半岛电视台英文频道。(第一部分在YouTube,第二部分在YouTube)  
\item “与阿贝尔奖得主约翰·F·纳什 Jr的访谈”。《欧洲数学学会通讯》,第97卷。访谈者:马丁·劳森和克里斯蒂安·斯卡乌。2015年9月,26-31页。ISSN 1027-488X。MR 3409221。
\end{itemize}
\subsection{出版列表}  
\begin{itemize}
\item 纳什,约翰·F;纳什,约翰·F·Jr.(1945年)。“使用悬链线公式进行电缆和电线跨度的下垂与张力计算”。《美国电气工程师学会学报》, 64(10): 685–692. doi:10.1109/T-AIEE.1945.5059021. S2CID 51640174.  
\item 纳什,约翰·F·Jr.(1950a年)。“博弈问题”。《计量经济学》, 18(2): 155–162. doi:10.2307/1907266. JSTOR 1907266. MR 0035977. S2CID 153422092. Zbl 1202.91122.  
\item 纳什,约翰·F·Jr.(1950b年)。“n人博弈中的均衡点”。《美国国家科学院院刊》, 36(1): 48–49. Bibcode:1950PNAS...36...48N. doi:10.1073/pnas.36.1.48. MR 0031701. PMC 1063129. PMID 16588946. Zbl 0036.01104.  
\item 纳什,J·F;沙普利,L·S(1950年)。“简单的三人扑克博弈”。收录于库恩,H·W;塔克,A·W(编辑)《博弈论研究贡献,第一卷》。数学研究丛书。第24卷。普林斯顿,NJ:普林斯顿大学出版社。第105–116页。doi:10.1515/9781400881727-011. MR 0039223. Zbl 0041.25602.  
\item 纳什,约翰(1951年)。“非合作博弈”。《数学年刊》,第二辑,54(2): 286–295. doi:10.2307/1969529. JSTOR 1969529. MR 0043432. Zbl 0045.08202.  
\item 纳什,约翰(1952a年)。“流形的代数近似”。收录于格雷夫斯,劳伦斯·M;希尔,埃纳尔;史密斯,保罗·A;扎里斯基,奥斯卡(编辑)《国际数学家大会论文集:1950年,马萨诸塞州剑桥,美国》。第一卷。普罗维登斯,RI:美国数学学会,第516–517页。  
\item 纳什,约翰(1952b年)。“实代数流形”。《数学年刊》,第二辑,56(3): 405–421. doi:10.2307/1969649. JSTOR 1969649. MR 0050928. Zbl 0048.38501.  
\item 纳什,约翰(1953年)。“两人合作博弈”。《计量经济学》, 21(1): 128–140. doi:10.2307/1906951. JSTOR 1906951. MR 0053471. Zbl 0050.14102.
\item Mayberry, J. P.; Nash, J. F.; Shubik, M. (1953年)。“两家垄断企业博弈情境的处理方法比较”。《计量经济学》, 21(1): 141–154. doi:10.2307/1906952. JSTOR 1906952. MR 3363438. S2CID 154750660. Zbl 0050.15104.  
\item 纳什,约翰(1954年)。“C1等距嵌入”。《数学年刊》,第二辑,60(3): 383–396. doi:10.2307/1969840. JSTOR 1969840. MR 0065993. Zbl 0058.37703.  
\item Kalisch, G. K.; Milnor, J. W.; Nash, J. F.; Nering, E. D. (1954年)。“一些实验性的n人博弈”。收录于Thrall, R. M.; Coombs, C. H.; Davis, R. L.(编辑)《决策过程》。纽约:约翰·威利与子公司,第301–327页。MR 3363439. Zbl 0058.13904.  
\item 纳什,约翰(1955年)。“路径空间与Stiefel–Whitney类”。《美国国家科学院院刊》, 41(5): 320–321. Bibcode:1955PNAS...41..320N. doi:10.1073/pnas.41.5.320. MR 0071081. PMC 528087. PMID 16589673. Zbl 0064.17503.
\item 纳什,约翰(1956年)。“黎曼流形的嵌入问题”。《数学年刊》,第二辑,63(1): 20–63. doi:10.2307/1969989. JSTOR 1969989. MR 0075639. Zbl 0070.38603.  
\item 纳什,约翰(1957年)。“抛物方程”。《美国国家科学院院刊》, 43(8): 754–758. Bibcode:1957PNAS...43..754N. doi:10.1073/pnas.43.8.754. MR 0089986. PMC 528534. PMID 16590082. Zbl 0078.08704.  
\item 纳什,约翰(1958年)。“抛物方程和椭圆方程解的连续性”。《美国数学杂志》, 80(4): 931–954. Bibcode:1958AmJM...80..931N. doi:10.2307/2372841. JSTOR 2372841. MR 0100158. Zbl 0096.06902.  
\item 纳什,约翰(1962年)。“一般流体的Cauchy问题”。《法国数学学会公报》, 90: 487–497. doi:10.24033/bsmf.1586. MR 0149094. Zbl 0113.19405.  
\item 纳什,约翰(1966年)。“具有解析数据的隐函数问题解的解析性”。《数学年刊》,第二辑,84(3): 345–355. doi:10.2307/1970448. JSTOR 1970448. MR 0205266. Zbl 0173.09202.  
\item 纳什,约翰·F.(1995年)。“奇点的弧结构”。《杜克数学杂志》, 81(1): 31–38. doi:10.1215/S0012-7094-95-08103-4. MR 1381967. Zbl 0880.14010.  
\item 纳什,约翰(2002年)。“理想货币”。《南方经济学杂志》, 69(1): 4–11. doi:10.2307/1061553. JSTOR 1061553.  
\item 纳什,约翰·F.(2008年)。“博弈中联盟与合作建模的机构方法”。《国际博弈理论评论》, 10(4): 539–564. doi:10.1142/S0219198908002084. MR 2510706. Zbl 1178.91019.
\item 纳什,约翰·F.(2009年a)。“理想货币与渐进理想货币”。收录于彼得罗先,莱昂·A.;曾凯维奇,尼古拉·A.(编),《博弈论与管理贡献,第二卷》。圣彼得堡:圣彼得堡大学管理研究生院,第281–293页。ISBN 978-5-9924-0020-5. MR 2605109. Zbl 1184.91147.  
\item 纳什,约翰·F.(2009年b)。“使用机构方法研究合作博弈”。《数学、博弈论与代数国际杂志》,18(4–5): 413–426. MR 2642155. Zbl 1293.91015.  
\item 纳什,约翰·F. Jr.;纳戈尔,罗斯玛丽;奥肯费尔斯,阿克塞尔;塞尔腾,雷因哈德(2012年)。“实验博弈中联盟形成的机构方法”。《美国国家科学院院刊》, 109(50): 20358–20363. Bibcode:2012PNAS..10920358N. doi:10.1073/pnas.1216361109. PMC 3528550. PMID 23175792.  
\item 纳什,约翰·福布斯·Jr.;拉西亚斯,迈克尔·T.(主编)(2016年)。“数学中的开放问题”。纽约:斯普林格. doi:10.1007/978-3-319-32162-2. ISBN 978-3-319-32160-8. MR 3470099. Zbl 1351.00027.
\end{itemize}
纳什的四篇博弈论论文(Nash 1950a, 1950b, 1951, 1953)和三篇纯数学论文(Nash 1952b, 1956, 1958)被收录在以下书籍中:
\begin{itemize}
\item 库恩,哈罗德·W.;纳萨尔,西尔维亚,编(2002年)。《约翰·纳什精要》。新泽西州普林斯顿:普林斯顿大学出版社。doi:10.1515/9781400884087. ISBN 0-691-09527-2. MR 1888522. Zbl 1033.01024.
\end{itemize}
\subsection{参考文献}  
\begin{enumerate}
\item Goode, Erica (2015年5月24日)。 "约翰·F·纳什 Jr.,由‘美丽心灵’定义的数学天才,86岁去世"。《纽约时报》。
\item "约翰·F·纳什 Jr.和路易斯·尼伦伯格共享阿贝尔奖"。阿贝尔奖。2015年3月25日。原文存档于2019年6月16日。2015年5月27日检索。
\item Nasar, Sylvia (1994年11月13日)。 "诺贝尔奖得主的失落岁月"。《纽约时报》。新泽西州普林斯顿。2014年5月6日检索。
\item "奥斯卡竞赛审查基于真实故事的电影"。《今日美国》。2002年3月6日。2008年1月22日检索。
\item "奥斯卡奖获奖名单"。《今日美国》。2002年3月25日。2008年8月30日检索。
\item Yuhas, Daisy (2013年3月)。"历史上,定义精神分裂症一直是一个挑战(时间线)"。《科学美国人心智》杂志。2013年3月2日检索。
\item Nasar 1998,第1章。
\item Nash, John F. Jr. (1995)。 "约翰·F·纳什 Jr. – 传记"。在Frängsmyr, Tore(编)。《诺贝尔奖1994:讲座、传记与颁奖词》。斯德哥尔摩:诺贝尔基金会,第275–279页。ISBN 978-91-85848-24-9。
\item "纳什推荐信"(PDF)。第23页。原文存档于2017年6月7日。2015年6月5日检索。
\item Kuhn, Harold W.; Nasar, Sylvia(编)。"约翰·纳什精要"(PDF)。普林斯顿大学出版社。第xi页,导言。原文存档于2007年1月1日。2008年4月17日检索。
\item Nasar 1998,第2章。
\item Nasar (2002),第xvi–xix页。
\item Milnor, John (1998). "约翰·纳什与《美丽心灵》"(PDF)。《美国数学会通报》, 25 (10): 1329–1332.
\item "1999年斯蒂尔奖"(PDF)。《美国数学会通报》, 46 (4): 457–462. 1999年4月。原文存档于2000年8月29日。
\item "2012年新闻稿 - 国家密码学博物馆开设约翰·纳什博士新展览"。美国国家安全局。2022年7月30日检索。
\item "约翰·纳什的信件致NSA;图灵的无形之手"。2012年2月17日。2012年2月25日检索。
\item Nash, John F. (1950年5月)。"非合作博弈"(PDF)。博士论文。普林斯顿大学。原文存档于2015年4月20日。2015年5月24日检索。
\item Osborne, Martin J. (2004). 《博弈论导论》。英国牛津:牛津大学出版社,第23页。ISBN 0-19-512895-8。
\item Nash 1951。
\item Nash 1950a; Nash 1950b; Nash 1953。
\item Nasar 1998,第15章。
\item Nash 1952a。
\item Nash 1952b。
\item Bochnak, Jacek; Coste, Michel; Roy, Marie-Françoise (1998). 《实代数几何》。数学及其边界的成果,3. Folge,第36卷(从1987年法文版翻译和修订)。柏林:施普林格出版社。doi:10.1007/978-3-662-03718-8。ISBN 3-540-64663-9。MR 1659509。S2CID 118839789。Zbl 0912.14023。
\item Shiota, Masahiro (1987). 《纳什流形》。数学讲义丛书,第1269卷。柏林:施普林格出版社。doi:10.1007/BFb0078571。ISBN 3-540-18102-4。MR 0904479。Zbl 0629.58002。
\item Artin, M.; Mazur, B. (1965). "关于周期点"。《数学年刊》,第二系列,第81卷(1): 82–99. doi:10.2307/1970384. JSTOR 1970384. MR 0176482. Zbl 0127.13401.
\item Gromov, Mikhaïl (2003). "关于全纯映射的熵"(PDF)。《数学教学》,国际评论,第二系列,第49卷(3–4): 217–235. MR 2026895. Zbl 1080.37051.
\item Nasar 1998,第20章。
\item Gromov, Misha (2016). "约翰·纳什介绍:定理与思想"。在Nash, John Forbes Jr.; Rassias, Michael Th.(编)。《数学中的开放问题》。施普林格,Cham。arXiv:1506.05408. doi:10.1007/978-3-319-32162-2. ISBN 978-3-319-32160-8. MR 3470099.
\item Nash 1954。
\item Eliashberg, Y.; Mishachev, N. (2002). 《h原理导论》。研究生数学丛书,第48卷。普罗维登斯,RI:美国数学学会。doi:10.1090/gsm/048. ISBN 0-8218-3227-1. MR 1909245.
\item Gromov, Mikhael (1986). 《偏微分关系》。数学及其边界的成果(第3系列),第9卷。柏林:施普林格出版社。doi:10.1007/978-3-662-02267-2. ISBN 3-540-12177-3. MR 0864505.
\item De Lellis, Camillo; Székelyhidi, László Jr. (2013). "耗散连续欧拉流"。《数学发明》, 193 (2): 377–407. arXiv:1202.1751. Bibcode:2013InMat.193..377D. doi:10.1007/s00222-012-0429-9. MR 3090182. S2CID 2693636.
\item Isett, Philip (2018). "翁萨格猜想的证明"。《数学年刊》,第二系列,第188卷(3): 871–963. arXiv:1608.08301. doi:10.4007/annals.2018.188.3.4. MR 3866888. S2CID 119267892. 原文存档于2022年10月11日。2022年10月11日检索。
\item Müller, S.; Šverák, V. (2003). "Lipschitz映射的凸积分及正则性反例"。《数学年刊》,第二系列,第157卷(3): 715–742. arXiv:math/0402287. doi:10.4007/annals.2003.157.715. MR 1983780. S2CID 55855605.
\item Nash 1956。
\item Hamilton, Richard S. (1982). "纳什和莫泽的逆函数定理"。《美国数学会通报》,新系列,第7卷(1): 65–222. doi:10.1090/s0273-0979-1982-15004-2. MR 0656198. Zbl 0499.58003.
\item Nash 1966。
\item Nasar 1998,第30章。
\end{enumerate}