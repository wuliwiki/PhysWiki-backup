% 不确定性原理
% 量子力学|动量|不确定|海森堡

\pentry{平均值\upref{QMavg}, 柯西不等式\upref{CSNeq}, 高斯波包\upref{GausPk}}


\subsection{相容可观测量}

按照相容算符的\autoref{QMPrcp_def17}~\upref{QMPrcp},我们可以定义可观测量的相容:

\begin{definition}{相容可观测量}
设$X, Y$分别是两个可观测量对应的算符,若$[X, Y]=0$,则称它们是\textbf{相容}的.
\end{definition}

相容的可观测量具有一个关键性质:

\begin{theorem}{}
设$X, Y$是相容的可观测量,且$X$无\textbf{简并}(\autoref{QMPrcp_def16}~\upref{QMPrcp})的本征值.则$X$的本征矢量都是$Y$的本征矢量.
\end{theorem}

\textbf{证明}:

任取$X$的本征矢$\ket{s}$,设$X\ket{s}=ks$,其中$k\in\mathbb{C}$.

由于$XY-YX=0$,故有:
\begin{equation}
\begin{aligned}
XY\ket{s} &= YX\ket{s}\\
X(Y\ket{s}) &= kY\ket{s}
\end{aligned}
\end{equation}

即$Y\ket{s}$也是$X$的本征矢量,本征值也是$k$.又因为$X$没有简并本征值,故$Y\ket{s}$应该是$\ket{s}$的倍数,即$\ket{s}$是$Y$的本征矢.

\textbf{证毕}.









\subsection{位置—动量不确定原理}
单个粒子一维运动的波函数 $\psi(x)$ 的位置和动量的标准差为 $\sigma_x$ 和 $\sigma_p$
\begin{equation}\label{Uncert_eq2}
\sigma_x \sigma_p \geqslant \frac{\hbar}{2}
\end{equation}

\begin{example}{无限深势阱的束缚态}\label{Uncert_ex2}
(未完成)证明束缚态满足\autoref{Uncert_eq2}, 但不能取等号.
\end{example}

\begin{example}{高斯波包}\label{Uncert_ex1}
%(未完成)证明高斯波包可以使\autoref{Uncert_eq2} 取等号.
已知高斯波包\autoref{GausPk_eq1}~\upref{GausPk} 形为
\begin{equation}
\psi(x)=A_0\E^{-a(x-x_0)^2}\E^{\I k_0x}
\end{equation}
波函数的归一化要求 $A_0=(\frac{2a}{\pi})^{1/4}$
我们来求它的动量和位置的不确定度的乘积 $\sigma_x\sigma_p$.
\begin{equation}
\ev{x}=\int_{-\infty}^{+\infty}x\abs{\psi}^2\dd x=\abs{A_0}^2\int_{-\infty}^{+\infty}x\E^{-2a(x-x_0)^2}\dd x=0
\end{equation}
上式中,最后积分式为0是因为被积函数为奇函数.

\begin{equation}
\begin{aligned}
\ev{p}&=-\I\hbar\int_{-\infty}^{+\infty}\psi^{*}\dv{}{x}\psi\dd x\\
&=-\I\hbar\abs{A_0}^2\int_{-\infty}^{+\infty}\E^{-2a(x-x_0)^2}[-2a(x-x_0)+\I k_0]\dd x\\
&=\hbar k_0\abs{A_0}^2\sqrt{\frac{\pi}{2a}}=\hbar k_0
\end{aligned}
\end{equation}
\begin{equation}
\begin{aligned}
\ev{x^2}&=\int_{-\infty}^{+\infty}x^2\abs{\psi}^2\dd x=\abs{A_0}^2\int_{-\infty}^{+\infty}x^2\E^{-2a(x-x_0)^2}\dd x\\
&=\abs{A_0}^2\int_{-\infty}^{+\infty}\qty[(x-x_0)^2+2x_0(x-x_0)+x_0^2]\E^{-2a(x-x_0)^2}\dd x\\
&\overset{t=x-x_0}{=}\abs{A_0}^2\int_{-\infty}^{+\infty}\qty[t^2+2x_0t+x_0^2]\E^{-2at^2}\dd t\\
&=\abs{A_0}^2\int_{-\infty}^{+\infty}t^2\E^{-2at^2}\dd t
=\frac{\abs{A_0}^2}{\sqrt{(2a)^3}}\Gamma\qty(\frac{3}{2})\\
&=\frac{\abs{A_0}^2}{2}\sqrt{\frac{\pi}{(2a)^3}}=\frac{1}{4a}
\end{aligned}
\end{equation}
\begin{equation}
\begin{aligned}
\ev{p^2}&=-\hbar^2 \int_{-\infty}^{+\infty}\psi^{*}\dv[2]{}{x}\psi\dd x\\
&=-\hbar^2\abs{A_0}^2\int_{-\infty}^{+\infty}\qty[(-2a(x-x_0)+\I k_0)^2-2a]\E^{-2a(x-x_0)^2}\dd x\\
&=-\hbar^2\abs{A_0}^2\int_{-\infty}^{+\infty}\qty[4a^2(x-x_0)^2-2a-k_0^2]\E^{-2a(x-x_0)^2}\dd x\\
&=-(2a\hbar\abs{A_0})^2\frac{1}{\sqrt{(2a)^3}}\Gamma\qty(\frac{3}{2})-\frac{1}{\sqrt{2a}}\hbar^2 \abs{A_0}^2(-2a-k_0^2)\Gamma(\frac{1}{2})\\
&=-\sqrt{\frac{a\pi}{2}}(\hbar\abs{A_0})^2-\sqrt{\frac{\pi}{2a}}\hbar^2 \abs{A_0}^2(-2a-k_0^2)\\
&=\hbar^2a+\hbar^2k_0^2
\end{aligned}
\end{equation}
\begin{equation}\label{Uncert_eq3}
\begin{aligned}
\sigma_x^2&=\ev{x^2}-\ev{x}^2=\frac{1}{4a} \\
\sigma_p^2&=\ev{p^2}-\ev{p}^2=\hbar^2a
\end{aligned}
\end{equation}
由\autoref{Uncert_eq3} 
\begin{equation}
\sigma_x\sigma_p=\frac{\hbar}{2}
\end{equation}
故其满足最小不确定度,即\autoref{Uncert_eq2} 取等号.因此高斯波包常称为\textbf{最小不确定度波包}.
\end{example}

\subsection{不确定原理的拓展}
任意两个物理量 $A$ 和 $B$ 都满足
\begin{equation}\label{Uncert_eq5}
\sigma_a \sigma_b \geqslant \frac{1}{2\I}\mel{v}{[A, B]}{v}
\end{equation}
\autoref{Uncert_eq2} 可以看成是该式的特例: 令 $A = x, B = p$, 根据正则对易关系
\begin{equation}
[A, B] = [x, p] = \I\hbar
\end{equation}
代入\autoref{Uncert_eq5} 得\autoref{Uncert_eq2}. 

\subsection{证明}
令 $f = (A-a)v$, $g = (B-b)v$, 使用柯西不等式\upref{CSNeq} 有
\begin{equation}\label{Uncert_eq1}
\sigma_a^2 \sigma_b^2 = \braket{f}{f} \braket{g}{g} \geqslant \abs{\braket{f}{g}}^2 = (\Re\braket{f}{g})^2 + (\Im\braket{f}{g})^2
\end{equation}
其中
\begin{equation}
\begin{aligned}
\Im\braket{f}{g} &= \frac{1}{2\I}(\braket{f}{g} - \braket{g}{f})
= \frac{1}{2\I}\mel{v}{[(A-a), (B-b)]}{v}\\
&= \frac{1}{2\I}\mel{v}{[A, B]}{v}
\end{aligned}
\end{equation}
忽略\autoref{Uncert_eq1} 中的 $\Re$ 项(大于等于 0), 得到\autoref{Uncert_eq5}. 容易证明, $\mel{v}{[A, B]}{v}$ 必然是一个纯虚数, 所以\autoref{Uncert_eq5} 右边必为实数.
\subsection{最小不确定波包}
\autoref{Uncert_ex1} 中我们展示了高斯波包为最小不确定波包,现在我们来回答这样的问题:什么是最一般的最小不确定波包?即使\autoref{Uncert_eq5} 取等号.我们将证明对于坐标-动量的最一般的最小不确定波包是一个高斯波包.

回望对不确定原理\autoref{Uncert_eq5} 的证明,我们注意到在两个地方出现不等式:\autoref{Uncert_eq1} 中 $\braket{f}{f} \braket{g}{g} \geqslant \abs{\braket{f}{g}}^2$ 和最后舍去实部 $\Re\braket{f}{g}$.假若波函数使得这两个地方都成为等式,则这样的波函数便是最小不确定波包.

$\braket{f}{f} \braket{g}{g} \geqslant \abs{\braket{f}{g}}^2$ 是柯西不等式,要使得等号成立,必须满足 $g=cf$ ($c$ 为复常数).而对于舍去实部造成得不等式,我们只需令其为0,则此处也变为等式,即 $\Re\braket{f}{g}=\Re( c\braket{f}{f})=0$.由于内积 $\braket{f}{f}$ 一定为实数,这就意味着 $c$ 一定得是纯虚数.因此,最小不确定成立的充要条件是
\begin{equation}
g=\I af,\quad(a\in \mathbb{R})
\end{equation}

对于坐标-动量,这个判据为
\begin{equation}
\qty(-\I\hbar\dv{}{x}-\ev{p})\psi=\I a(x-\ev{x})\psi
\end{equation}
容易求得其一般解是:
\begin{equation}
\psi(x)=A_0\E^{-a(x-\ev{x_0})^2/2\hbar}\E^{\I \ev{p}x/\hbar}
\end{equation}
显然,这是一个高斯波包.故坐标-动量的最一般的最小不确定波包是一个高斯波包.

\addTODO{为什么一定要忽略实数项?为什么一定存在取等号的情况?}
