% 叉乘的矩阵形式
% 叉乘|矢量积|矩阵|反对称矩阵

\pentry{矢量叉乘\upref{Cross}, 矩阵\upref{Mat}}
对任意矢量 $\bvec a$ 和 $\bvec b$, 令
\begin{equation}\label{CrosMt_eq1}
\bvec c = \bvec a \cross \bvec b
\end{equation}
该运算可以看作列矢量 $\bvec b$ 到列矢量 $\bvec c$ 的线性变换. 我们知道线性变换可以用矩阵表示\upref{Mat}, 所以必存在矩阵 $\mat A$, 满足
\begin{equation}\label{CrosMt_eq3}
\bvec c = \mat A \bvec b
\end{equation}
令 $\bvec a$ 的坐标为 $(a_x, a_y, a_z)$, 根据叉乘的分量表达式(\autoref{Cross_eq8}~\upref{Cross}), 易得变换矩阵为
\begin{equation}\label{CrosMt_eq2}
\mat A = \pmat{
0 & -a_z & a_y\\
a_z & 0 & -a_x\\
-a_y & a_x & 0
}\end{equation}
这是一个\textbf{反对称矩阵}, 即
\begin{equation}
\mat A\Tr = -\mat A
\end{equation}

类似地, 也有
\begin{equation}
\bvec c\Tr = \bvec a\Tr \mat B
\end{equation}
其中 $\mat B$ 的定义和 $\mat A$ 一致.

同理, \autoref{CrosMt_eq1} 也可以看作是 $\bvec a$ 到 $\bvec c$ 的线性变换
\begin{equation}
\bvec c = \mat B \bvec a
\end{equation}
其中
\begin{equation}
\mat B = \pmat{
0 & b_z & -b_y\\
-b_z & 0 & b_x\\
b_y & -b_x & 0
}
\end{equation}
这恰好与\autoref{CrosMt_eq2} 符号相反.

\subsection{叉乘矩阵的旋转变换}\label{CrosMt_sub1}
\pentry{三维旋转矩阵\upref{Rot3D}}
令 $S'$ 坐标系与 $S$ 坐标系的原点重合, 且 $S'$ 坐标系中的矢量 $\bvec r$ 到 $S$ 系中的矢量 $\bvec r'$ 的旋转变换矩阵为 $\mat R$, 满足 $\bvec r' = \mat R\bvec r$. 令两个几何矢量在 $S$ 系中的坐标为 $\bvec u, \bvec v$, 在 $S'$ 中坐标为 $\bvec u', \bvec v'$. 那么有
\begin{equation}
\bvec u' = \mat R\bvec u, \qquad
\bvec v' = \mat R\bvec v
\end{equation}
由于叉乘的几何定义不依赖于坐标系, 那么必定有
\begin{equation}
\mat R(\bvec u\cross \bvec v) = \bvec u'\cross \bvec v'
\end{equation}

如果用 $\mat U$ 来代替 $\bvec u\cross$, 用 $\mat U'$ 来代替 $\bvec u'\cross$, 那么上式变为
\begin{equation}
\mat R \mat U \bvec v = \mat U' \bvec v' = \mat U'\mat R\bvec v
\end{equation}
这对所有 $\bvec v$ 都成立, 所以 $\mat R \mat U = \mat U' \bvec v' = \mat U'\mat R$, 或者
\begin{equation}
\mat U' = \mat R \mat U \mat R\Tr
\end{equation}

