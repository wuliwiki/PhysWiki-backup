% 在 Linux 上编译 C/C++ 程序
% keys linux|C++|gcc|g++
% license Xiao
% type Tutor

本文介绍如何使用 \enref{Linux 命令行}{Linux}用 \enref{g++ 编译器}{gpp}编译一个 \enref{C++}{Cpp0} 的 Hello World 程序。 阅读本文不需要任何基础(包括 C++ 和 Linux)。

如何安装 Linux 命令行? 你可以直接在电脑上安装一个 Linux 系统\footnote{推荐 Ubuntu, 因为小时百科大部分 Linux 相关的教程都是以 Ubuntu 为环境的。}, 也可以在 Windows 10 或 11 中使用 \enref{WSL}{WSLnt}。 如果使用 Win7 系统, 可以安装一个 \enref{MinGW}{Mingw}。 本文以 Ubuntu 系统为例(Linux 有不同的版本, Ubuntu 是广泛使用的一个)。

\subsection{在命令行安装 g++}
打开命令行后, 要检查是否安装了 g++, 输入以下命令并按回车。
\begin{lstlisting}[language=bash]
g++ --version
\end{lstlisting}
这个命令由两部分组成, 第一部分是 \verb`g++`, 即在命令行中运行名为 \verb`g++` 这个程序。 第二部分是 g++ 程序的 \verb`--version` 选项, 即查看已安装 g++ 的版本。 如果系统上没有安装 g++, 命令行会返回类似这样的提示(以 Ubuntu 为例)
\begin{lstlisting}[language=bash]
Command 'g++' not found, but can be installed with:
sudo apt install g++
\end{lstlisting}
要安装 g++, 只需要在 terminal 输入两个命令
\begin{lstlisting}[language=bash]
sudo apt update
sudo apt install g++
\end{lstlisting}
\verb`sudo` 是 super user do, 即使用操作系统的 root 权限运行指定的程序。 root 权限是 Linux 系统的最高权限, 几乎可以做任何事情(包括危险操作)使用时需要谨慎。 命令行中的一些命令(如安装卸载软件)一般需要 root 权限, 所以需要在 \verb`apt` 命令前面加 \verb`sudo`。 如果你已经以 root 的身份登录 Linux 则可以不用。

\verb`apt` (也可以用 \verb`apt-get`) 是 Ubuntu 系统中用来管理程序的命令(安装,卸载,更新等)。 注意系统一般要链接网络才能使用 apt 安装程序。 在安装任何程序之前, 建议新更新一下 apt 的数据库(第一行)。

\verb`update` 是 \verb`apt` 的一个\textbf{参数(argument)}, 就跟 g++ 的 \verb`--version` 参数一样。 Linux 中程序的许多参数以 \verb`--` 或者 \verb`-` 开头。

\verb`install` 是 \verb`apt` 程序的另一个参数。 \verb`g++` 是 \verb`install` 的参数, 即要安装的程序。

运行后可以按提示安装 g++, 过程中可能需要输入 Linux 的用户密码, 以及确认是否安装。 输入 \verb`Y` 然后按回车确认, 也可以直接按回车, 因为大写的 \verb`Y` 表示这是默认选项。
\begin{lstlisting}[language=bash]
After this operation, 155 MB of additional disk space will be used.
Do you want to continue? [Y/n] Y
\end{lstlisting}

安装完以后, 再次查看 \verb`g++ --version`, 得到
\begin{lstlisting}[language=bash]
g++ (Ubuntu 7.4.0-1ubuntu1~18.04.1) 7.4.0
Copyright (C) 2017 Free Software Foundation, Inc.
This is free software; see the source for copying conditions.  There is NO
warranty; not even for MERCHANTABILITY or FITNESS FOR A PARTICULAR PURPOSE.
\end{lstlisting}
可以看到我们安装的版本是 7.4.0, 不同 Ubuntu 版本的默认 g++ 版本不一样。


\subsection{创建第一个 C++ 程序}
打开一个文本编辑器, 创建一个新文本文档, 命名为 hello.cpp, 输入如下 C++ 代码并保存。
\begin{lstlisting}[language=bash, caption=hello.cpp]
// print hello world
#include <iostream>

int main()
{
    std::cout << "hello world!" << std::endl;
}
\end{lstlisting}
在编译之前, 我们首先要介绍\textbf{工作目录(working directory)}, 也叫\textbf{当前目录(current directory)}, 路径有时候也叫\textbf{路径}。 命令行一般会在输入的每一行命令之前显示当前的路径如\autoref{fig_linCpp_1}。
\begin{figure}[ht]
\centering
\includegraphics[width=14cm]{./figures/02ec163f9f02ffa5.png}
\caption{蓝色的路径就是当前路径} \label{fig_linCpp_1}
\end{figure}

这个目录一般是打开命令行的目录(例如 Ubuntu 中在某个文件夹中右键打开命令行, 或者 win10 中按 \verb`Shift + 右键` 打开 WSL 的命令行)。 另外也可以在命令行中运行 \verb`pwd` 命令显示当前路径。我们假设刚才创建的 cpp 文件就保存在这个目录下。

现在我们来用 g++ 编译这个程序, 命令行输入
\begin{lstlisting}[language=bash]
g++ hello.cpp
\end{lstlisting}
如果编译成功, 将会在同一文件夹中生成名为 \verb`a.out` 可执行文件, 相当于 windows 中的 exe 文件。 要运行该文件, 使用(\verb`./` 表示当前的 working directory)
\begin{lstlisting}[language=bash]
./a.out
\end{lstlisting}
运行后可以看到程序在命令行输出了 \verb`hello world!`。

这样, 我们就成功在 Linux 中运行了第一个 C++ 程序。
