% von Neumann 熵
% keys von Neumann entropy|von Neumann熵|冯诺依曼熵|纠缠熵
% license Xiao
% type Tutor

\begin{issues}
\issueTODO
\issueMissDepend(香农熵)
\issueDraft
\end{issues}

\pentry{密度矩阵\upref{denMat}}
%此处还应该运用香农熵的预备知识,但在我编辑这条消息的时候还没有相对应的词条,如果以后有了记得加上。

\footnote{参考了\cite{量子信息}和Wikipedia\href{https://en.wikipedia.org/wiki/Von_Neumann_entropy}{相关界面}}
von Neumann 熵的形式来自于 Shannon 熵。

\begin{definition}{von Neumann熵}
对于一个给定的密度矩阵,von Neumann熵 $S\left(\rho\right)$的定义为:

\begin{equation}
S\left( \rho \right) = \opn{tr}\left( - \rho \ln \rho \right)~.
\end{equation}

如果$\left\{ \lambda_1,~\lambda_2,~\cdots \lambda_N \right\}$是$\rho$的本征值,那么:

\begin{equation}
S\left(\rho\right) = \sum_i^N \lambda_i \ln \lambda_i~.
\end{equation}

上式中应注意我们定义$0\ln0 = 0$来规避发散。

\end{definition}

von Neumann 度量了一个混态的密度矩阵的“混乱程度”,正如\upref{partra}中提到,如果一个大系统的纯态对其中的某一个子系统取偏迹,同时如果得到了一个混态而非纯态,那么代表该子系统与剩余部分存在纠缠,这时求完偏迹的密度矩阵的von Neumann 熵就给出了一个度量纠缠的方法,这既是其纠缠熵名字的由来。

\subsection{von Neumann熵的性质}

\subsubsection{密度矩阵$\rho$的von Neumann熵当且仅当$\rho$表示纯态时为0}

\begin{itemize}
\item 密度矩阵$\rho$的von Neumann熵当且仅当$\rho$表示纯态时为0
\end{itemize}



密度矩阵的本征值的本征值代表取到对应态的概率,$\forall \lambda_i ,~ 0\leqslant \lambda_i \leqslant1$,$\sum_i^N \lambda_i = 1$。所以$\lambda_i \ln \lambda_i$非负,且仅在$\lambda_i = 0$或$\lambda_i = 1$时为$0$,因此仅仅在所有本征值中只有一个为$1$时,即纯态时,$S\left(\rho\right) = 0$。




