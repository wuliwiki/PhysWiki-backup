% 余元公式
% Gamma 函数|欧拉定义|高斯定义|魏尔斯特拉斯定义

\begin{issues}
\issueOther{本文需要和 Gamma 函数 2\upref{Gamma2} 整合}
\end{issues}

\pentry{Gamma 函数 2\upref{Gamma2}}

余元公式为
\begin{equation}
\Gamma(x)\Gamma(1-x)=\frac{\pi}{\sin  \pi x}~.
\end{equation}

\begin{lemma}{伽马函数的三种等价定义:}
欧拉定义
\begin{equation}
\Gamma(z):=\int^{\infty}_{0}e^{-t}t^{z-1} \dd{t}~.
\end{equation}
高斯定义
\begin{equation}
\Gamma(z):=\lim_{n \rightarrow \infty}{\frac{n!n^z}{z(z+1)\dots (z+n)}}=\prod_{n}^{\infty}\frac{\left( 1+\frac1n \right)^z}{\left( 1+\frac zn \right)}~.
\end{equation}
魏尔斯特拉斯定义
\begin{equation}
\frac1{\Gamma(x)}:=ze^{\gamma z}\prod_{n}^{\infty}\left[ \left( 1+\frac{z}{n}\right)\E^{-\frac{z}{n}}  \right]~.
\end{equation}
\end{lemma}
首先是 \footnote{这个证法来自于《微积分学教程》第二卷 第408目}
\begin{equation}
(\cos z+i\sin z)^n=\cos\,nz+i\sin nz~,
\end{equation}
所以有
\begin{equation}
\sin nz=n(\cos^{n-1}z)\sin z-\frac{n(n-1)(n-2)}{1\cdot 2\cdot 3}(\cos^{n-3}z \cdot \sin z)+\dots
\end{equation}
仅仅考虑 $n=2k+1$ 的情况,然后考虑到 $\cos^2 z=1-\sin^2z$ , 则必有
\begin{equation}
\sin(2k+1)z=\sin\,z\cdot P(\sin^2z)~,
\end{equation}
其中 $P(\sin^2z)$ 代表以 $ \sin^2z$ 为变量的 $\mathbf{k}$ 次多项式。

考虑到在 $z_m=\frac{m\pi}{2k+1}\,,m=1,2,\dots,k $ 时, 对于每一个 $z_m$ ,都有 $ \sin(2k+1)z=0,\sin\,z\ne 0$
则可知每一个 $z_m$ 都是 $P(\sin^2z)$ 的根。 即得
\begin{equation}
P(\sin^2z)=A(1-\frac{\sin^2z}{\sin^2z_1})(1-\frac{\sin^2z}{\sin^2z_2})\dots(1-\frac{\sin^2z}{\sin^2z_k})~.
\end{equation}
考虑到
\begin{equation}
A=P(0)=\lim_{z\rightarrow0}\frac{\sin(2k+1)z}{\sin z}=2k+1~,
\end{equation}
则有
\begin{equation}
P(\sin^2z)=(2k+1)(1-\frac{\sin^2z}{\sin^2z_1})(1-\frac{\sin^2z}{\sin^2z_2})\dots(1-\frac{\sin^2z}{\sin^2z_k})~,
\end{equation}
也即
\begin{equation}
\sin(2k+1)z=\sin z\cdot\,(2k+1)(1-\frac{\sin^2z}{\sin^2z_1})\dots(1-\frac{\sin^2z}{\sin^2z_k})~.
\end{equation}
令 $z=\frac{x}{2k+1}$,则为
\begin{equation}
\sin\,x=\sin(\frac x{2k+1})\cdot\,(2k+1)(1-\frac{\sin^2\frac x{2k+1}}{\sin^2z_1})\dots(1-\frac{\sin^2\frac x{2k+1}}{\sin^2z_k})~.
\end{equation}
现在把 $\sin x$ 乘积分成两部分,即截取
\begin{equation}
V^a_k=(1-\frac{\sin^2\frac x{2k+1}}{\sin^2z_{a+1}})\dots(1-\frac{\sin^2\frac x{2k+1}}{\sin^2z_k})~.
\end{equation}
而将前面的部分记为
\begin{equation}
U^k_a=(2k+1)\sin(\frac x{2k+1})(1-\frac{\sin^2\frac x{2k+1}}{\sin^2z_1})\dots(1-\frac{\sin^2\frac x{2k+1}}{\sin^2z_{a}})~,
\end{equation}
其中 $0<a<k $。

由  $\lim_{k\rightarrow\infty}(2k+1)\sin\frac{x}{2k+1}=x~,$
\begin{equation}
\lim_{k\rightarrow\infty}\left( \frac{\sin\frac{x}{2k+1}}{\sin\frac{m\pi}{2k+1}} \right)^2=\left( \frac{x}{m\pi} \right)^2\,,m=1,2,\dots,k~.
\end{equation}
可得
\begin{equation}
U_a=\lim_{k\rightarrow\infty}U^k_a=x(1-\frac{x^2}{\pi^2})\dots(1-\frac{x^2}{a^2\pi^2})~,
\end{equation}
现在考虑 $V^k_a$。 考虑到在 $0<\varphi<\frac\pi2 $ 时, $\frac2\pi\varphi<sin\varphi$ ,
则有
\begin{equation}
\sin^2\frac {h\pi}{2k+1}>\frac4\pi\frac{(h\pi)^2}{(2k+1)^2},\ h=a+1,\dots k 
\end{equation}
而 $\sin^2\frac x{2k+1}<\frac{x^2}{(2k+1)^2}$ 是易得的。 于是
\begin{equation}
1>V^k_a>(1-\frac{x^2}{4(a+1)^2})\dots(1-\frac{x^2}{4k^2})~.
\end{equation}
因为最开始并没有限定 $a$ 的具体数值,所以总可以找到 $a=a_0$ ,使得 $4(a_0+1)>x^2~.$
\begin{lemma}{}
对于充分大的n而言,$\prod_{n=1}^{\infty}(1+a_n) $ 收敛的充要条件是  $\sum_{n=1}^{\infty}{a_n}$ 收敛。
\end{lemma}
当然,不需要一定从 n=1 开始,去掉有限项都不改变收敛或发散的结果。
\begin{lemma}{}
若无穷乘积 $\prod_{n=1}^{\infty}p_n$ 收敛,那么总有充分大的 m 使得 $\lim_{m\rightarrow\infty}\prod_{n=m+1}^{\infty}p_n=1$。
\end{lemma}
当然,这要求 $ p_n\ne0 $,所有这时应该把 $ x=0,\pm\pi,\pm 2\pi\dots $ 分出来单独考虑。

由于
\begin{equation}
\sum_{h=a_0+1}^{\infty}{\frac{x^2}{4h^2}} \text{收敛,所以} \lim_{a_0\rightarrow\infty}\lim_{k\rightarrow\infty}(1-\frac{x^2}{4(a_0+1)^2})\dots(1-\frac{x^2}{4k^2})=1~.
\end{equation}
即为
\begin{equation}
\sin\,x=\lim_{a\rightarrow\infty}U_a=x\cdot \prod_{n=1}^{\infty}(1-\frac{x^2}{n^2\pi^2})~.
\end{equation}
这里可以直接看出,这对于 $x=0,\pm\pi,\pm 2\pi\dots $ 也是成立的。

另外,有
\begin{equation}
\sin\pi x=\pi x\cdot \prod_{n=1}^{\infty}(1-\frac{x^2}{n^2})~,
\end{equation}
考虑到(威尔斯特斯拉定义)
\begin{equation}
\Gamma(x)=\frac1x\prod_{n=1}^{\infty}\frac{(1+\frac1n)^{x}}{1+\frac xn}~.
\end{equation}
以及 $ \Gamma(1+x)=x\Gamma(x)$, 则有
\begin{equation}
\Gamma(1-x)=-x\Gamma(-x)=\prod_{n=1}^{\infty}\frac{(1+\frac1n)^{-x}}{1-\frac xn}~,
\end{equation}
于是
\begin{equation}
\Gamma(x)\Gamma(1-x)=\frac{1}{x}\cdot \prod_{n=1}^{\infty}\frac1{(1-\frac{x^2}{n^2})}=\frac\pi{\sin\pi x}~.
\end{equation}
