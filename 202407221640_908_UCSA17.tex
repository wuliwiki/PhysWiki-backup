% 中国科学院大学 2017 年考研 量子力学
% license Usr
% type Note

\textbf{声明}:“该内容来源于网络公开资料,不保证真实性,如有侵权请联系管理员”

一、已知一粒子处于一维无限深势阱时,势函数为
$$V = \begin{cases} 0 & 0 < x < a \\\\+\infty & x < 0, x > a \end{cases}~$$
能量本征函数数为 $\psi_n = \sqrt{\frac{2}{a}} \sin \frac{n\pi x}{a}$,$t = 0$ 时 $\psi(x, 0) = A x (a - x)$。

(1) 求归一化系数 $A$

(2) 求出 $t > 0$ 时的波函数 $\psi(x, t)$

(3) 该粒子处于基态和第一激发态的概率

(4) $t > 0$ 时,粒子坐标的平均值$x$.

二、如图,已知入射波函数为 $e^{ik_1x}$。
\begin{figure}[ht]
\centering
\includegraphics[width=10cm]{./figures/35ea0c2d3e63989d.png}
\caption{} \label{fig_UCSA17_2}
\end{figure}
(1) 求入射粒子的流密度

(2) 当$E = V_b$时,透射系数的取值为( )\\
A 0 \\
B $0 < T < 1$\\
C 1 \\

(3) 为了怎发透射系数,可以采取的办法有( )\\
A 保持$E = V_b$,降低Va,但保证$V_a > V_b$\\
B 在满足$V_a > V_b$的条件下,使$E > V_b$\\
C 保持$E = V_b$,降低Va,保证$V_a < V_b$\\

(4) 若已知入射波函数数为$ e^{ik_1x}$ ,反射波函数数为$R e^{ik_1x}$,透射波函数数为$S e^{ik_2x}$,其中 $k_1^2 = \frac{2mE}{h^2}, k_2^2 = \frac{2m(E - V_b)}{h^2} $,透射系数为( )\\
A. $|S|^2$ \\
B. $1 - |R|^2$ \\
C. 以上都不是

   三、
(1) 证明量子体系满足Ehrenfest定理:
$$\begin{cases}\frac{d \overline{x} }{dt} = \frac{\overline p }{m} \\\\
\frac{d\overline p }{dt} = -\left\langle \frac{\overline{\partial V}}{\partial x} \right\rangle\end{cases}~$$

(2) 已知二维谐振子处于均匀电场$(\epsilon_x, \epsilon_y)$中,体系的Hamilton为 
$H = \frac{p_x^2}{2m} + \frac{p_y^2}{2m} + \frac{1}{2}m\omega^2 x^2 + \frac{1}{2}m\omega^2 y^2 - q\epsilon_x x - q\epsilon_y y$
当电场为0时求体系的能级

(3) 以$q$为参量利用$HF$ 定理求在电场下的能级

   四、
已知二体系统哈密顿为 
$H = H_0 + \lambda H'$
其中
$H_0 = \begin{pmatrix}E_1 & 0 \\\\0 & E_2\end{pmatrix}, \quad H' = \begin{pmatrix}0 & ia \\\\-ia & 0\end{pmatrix},$
且 $\left|\frac{\lambda a}{E_2 - E_1}\right| \ll 1$.\\
(1) 求能级的二级近似和波函数的一级近似\\
(2) 求出准确的的能级表达式并和上问进行比较\\

   五、
已知晶格中两局域电子构成的体系:
$H = S_x^{(1)} s_x^{(2)} + S_y^{(1)} s_y^{(2)}$

(1) 若 $S = s^{(1)} + s^{(2)}$,$S_z = s_z^{(1)} + s_z^{(2)}$,求 $S^2$ 和 $S_z$的本征值

(2) 用 $S$ 和 $S_z$ 表示 $H$

(3) 求该体系哈密顿的能量本征值

(4) 若外加一$z$方向的磁场$B$,求体系的能级
