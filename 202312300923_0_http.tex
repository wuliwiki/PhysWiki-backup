% http 协议笔记
% license Xiao
% type Note

\begin{issues}
\issueDraft
\end{issues}

\subsubsection{常识}
\begin{itemize}
\item HTTP 是一个在计算机世界里专门在「两点」之间「传输」文字、图片、音频、视频等「超文本」数据 的「约定和规范」。
\item HTTP 协议是基于  TCP/IP,并且使用了「请求 - 应答」模式
\item HTTP 的每个 message (不是 package)都是文本的
\item http 包的结构就是 head + body, 其中 head 除了第一行都是 \verb`key : value` 的形式
\item HTTPS 也就是在 HTTP 与 TCP 层之间增加了 SSL/TLS 安全传输层,HTTP/3 甚至把 TCP 层换成了基 于 UDP 的 QUIC。
\item http 是无状态的,所以每个请求(例如购物)都需要附带登录信息和验证信息(或其他状态信息)。比常见的是 Cookie 技术,服务器在某次响应中发送 Cookie,用户之后每次请求都附上。
\item 2023 年 50\% 左右的 http 都使用 \verb`HTTP/2` 版本。 \verb`HTTP/3` 大概占 10\%。
\end{itemize}

\subsubsection{版本改进}
\begin{itemize}
\item 早期 HTTP/1.0 性能上的一个很大的问题,那就是每发起一个请求,都要新建一次 TCP 连接(三次握 手),而且是串行请求,做了无谓的 TCP 连接建立和断开,增加了通信开销。
\item HTTP/1.1 提出了长连接的通信方式,也叫持久连接。这种方式的好处在 于减少了 TCP 连接的重复建立和断开所造成的额外开销,减轻了服务器端的负载。持久连接的特点是,只要任意一端没有明确提出断开连接,则保持 TCP 连接状态。
\item HTTP/1.1 采用了长连接的方式,这使得\textbf{管道(pipeline)}网络传输成为了可能。客户端可以发起多个请求,只要第一个请求发出去了,不必等其回来, 就可以发第二个请求出去,可以减少整体的响应时间。
\item 队头堵塞:先回应 A 请求,完成后再回应 B 请求。要是前面的回应特别慢,后面就会 有许多请求排队等着
\item HTTP/2 协议是基于 HTTPS 的
\item HTTP/2 会压缩头(Header)(HPACK 算法)生成一个索引号,以后就不发送同样字段了,只发送索引号
\item 全面采用了二进制格式,头信息和数据体都是 二进制,并且统称为帧(frame):头信息帧和数据帧。
\item HTTP/2 的数据包不是按顺序发送的,同一个连接里面连续的数据包,可能属于不同的回应。因此,必 须要对数据包做标记,指出它属于哪个回应。
\item 每个请求或回应的所有数据包,称为一个数据流(    Stream  )。每个数据流都标记着一个独一无二的编 号,其中规定客户端发出的数据流编号为奇数,  服务器发出的数据流编号为偶数
\item 客户端还可以指定数据流的优先级。优先级高的请求,服务器就先响应该请求。
\item HTTP/2 是可以在一个连接中并发多个请求或回应,而不用按照顺序一一对应。
\item 移除了 HTTP/1.1 中的串行请求,不需要排队等待,也就不会再出现「队头阻塞」问题,降低了延迟, 大幅度提高了连接的利用率。
\item 举例来说,在一个 TCP 连接里,服务器收到了客户端 A 和 B 的两个请求,如果发现 A 处理过程非常耗 时,于是就回应 A 请求已经处理好的部分,接着回应 B 请求,完成后,再回应 A 请求剩下的部分。
\item HTTP/2 服务不再是被动地响应,也可以主动 向客户端发送消息。
\item 举例来说,在浏览器刚请求 HTML 的时候,就提前把可能会用到的 JS、CSS 文件等静态资源主动发给 客户端,减少延时的等待,也就是服务器推送(Server Push,也叫 Cache Push)。
\end{itemize}

\begin{lstlisting}[language=none]
<method> <URI> HTTP/<version>
<Header1>: <value1>
<Header2>: <value2>
...
<HeaderN>: <valueN>
(空行)
<body>
\end{lstlisting}

\begin{itemize}
\item URI 就是 URL, 例如 \verb`/` 代表网站根目录
\end{itemize}

例子:
\begin{lstlisting}[language=none]
GET / HTTP/1.1
Host: www.example.com
User-Agent: Mozilla/5.0 Chrome/58.0.3029.110 Safari/537.36
Accept: text/html,application/xhtml+xml,application/xml;
Accept-Language: en-US,en;q=0.5
Accept-Encoding: gzip, deflate
Connection: keep-alive
Upgrade-Insecure-Requests: 1
\end{lstlisting}
这个 request 没有 body, 因为是简单的 GET。

\subsection{方法}
\begin{itemize}
\item \verb|method| 有 \verb|GET, POST, PUT, DELETE, HEAD, OPTIONS, PATCH, CONNECT, TRACE|
\end{itemize}


\subsection{服务器回应}
服务器回应的消息有不同的结构
\begin{lstlisting}[language=none]
HTTP/<version> <Status> <Reason Phrase>
<Header1>: <value1>
<Header2>: <value2>
...
<HeaderN>: <valueN>
(空行)
<body>
\end{lstlisting}

例子:
\begin{lstlisting}[language=none]
HTTP/1.1 200 OK
Date: Fri, 29 Dec 2023 12:00:00 GMT
Server: Apache/2.4.18 (Ubuntu)
Last-Modified: Fri, 29 Dec 2023 10:00:00 GMT
Content-Type: text/html; charset=UTF-8
Content-Length: 1234
Connection: close
Set-Cookie: UserID=JohnDoe; Max-Age=3600; Version=1

<!DOCTYPE html>
<html>
<head>
    <title>Welcome to Example.com!</title>
</head>
<body>
    <h1>Hello, World!</h1>
    <p>Welcome to our website.</p>
</body>
</html>
\end{lstlisting}

\subsection{响应状态码}
\begin{itemize}
\item 1xx 中间状态,还需要后续操作。实际用到的比较少
\item 2xx 成功,已被正确处理, 如 \verb`200 OK`, \verb`204 No Content`(响应中没有 body), \verb`206 Partial Content`(分块下载或断点续传)
\item 3xx 重定向,资源位置改变,需要重新请求: \verb`301 Moved Permanently`, \verb`302 Found`(临时重定向)。 \verb`Location` 字段用于指明要跳转的 URL。 \verb`304 Not Modified` \verb`重定向已存在的缓冲文件,也称缓存重定 向,用于缓存控制`
\item 4xx 请求有误,无法处理。 \verb`400 Bad Request` 笼统错误。 \verb`403 Forbidden` 禁止访问资源。 \verb`404 Not Found` 资源未找到。
\item 5xx 服务器错误。 \verb`500 Internal Server Error` 笼统错误。 \verb`501 Not Implemented` 即将开业,敬请期待。 \verb`502 Bad Gateway` (网关或代理时返回的错误码,服务器自身工作正常,但最终资源的服务器错误)
\item \verb`503 Service Unavailable` 网络服务正忙,请稍 后重试
\end{itemize}

\subsection{常见字段}
\begin{itemize}
\item \verb`Host` 如 \verb`Host: www.A.com`。可以用于区分同一个服务器中的不同网站(只有 IP 的话就区分不了)
\item \verb`Content-Length` 回应的数据长度, 如 \verb`Content-Length: 1000`
\item \verb`Connection` 常用于客户端要求服务器使用 TCP 持久连接。 \verb`Connection: keep-alive`。 HTTP/1.1 版本的默认连接都是持久连接,但为了兼容老版本的 HTTP。这不是标准字段。
\item \verb`Content-Type`,相应的数据是什么格式, 如 \verb`Content-Type: text/html; charset=utf-8`
\item \verb`Accept` 用于请求某种格式的内容, 如 \verb`Accept: */*` 表示都行。
\item \verb`Accept-Encoding` 用于请求某种压缩格式: \verb`Accept-Encoding: gzip, deflate`。
\end{itemize}
