% Python 判断与循环
% keys Python|判断|if|循环|elif|for
% license Xiao
% type Tutor

\begin{issues}
\issueOther{介绍 break 和 continue}
\end{issues}

\pentry{Python 简介\nref{nod_Python}}{nod_d4a0}

\subsection{判断语句}
有时我们在程序中做什么将取决于前面步骤的结果。在这种情况下,您可能会发现 \verb|if| 语句很有帮助。假设我想知道一个二次方程 $ax^2+bx+c=0$ 有多少实根。下面的代码根据 $a$、$b$ 和 $c$ 的值告诉我根的数量。
\begin{lstlisting}[language=python]
a = 5
b = 7
c = -1 
D = b**2 - 4*a*c
if D > 0:
    print("two roots")
if D < 0:
    print("no roots")
if D == 0:
    print("one root")
\end{lstlisting}

通常 \verb|if| 可以嵌套使用;循环语句可以与条件判断语句结合使用。例如,一个简单的打分系统:
\begin{equation}
f=\left\{\begin{array}{lc}\begin{array}{c}A\\B \\C \end{array}&\begin{array}{c}
(x>90) \\
(x>80) \\
(x>60)~.\end{array}\end{array}\right.\\\\
\end{equation}
\begin{lstlisting}[language=python]
x = 99
if x > 90:
    print('A')
elif x > 80:
    print('B')
else:
    print('C')
\end{lstlisting}


\begin{itemize}
\item 真和假是 \verb|True|, \verb|False|
\item \verb|and|(\verb|&|), \verb|or|(\verb`|`), \verb|not|(不是 \verb|!|)
\end{itemize}


\subsection{循环语句}
我们经常需要对某个操作重复执行多次,可以用 \verb|for| 循环,例如输出 $1$ 到 $3$ 的平方:
\begin{lstlisting}[language=python]
for i in range(1,4):
    print(i,i**2)
\end{lstlisting}
输出
\begin{lstlisting}[language=python]
1 1
2 4
3 9
\end{lstlisting}
注意,python中用缩进来表示代码的范围,通常为一个 tab 制表符(即键盘上的 TAB 键), 或者四个空格缩进; \verb|for| 循环后面\textbf{冒号}必不可少。 \verb|for| 循环还可以遍历一个列表, 对每个元素执行相同的操作。例如
\begin{lstlisting}[language=python]
A = np.array([3, 2, 4, 5, 1, 76])
for aa in A:
    print(3*aa)
\end{lstlisting}
这里对列表每一个元素扩大3倍。 如果用不得循环指标的时候可以用下划线代替:
\begin{lstlisting}[language=python]
for _ in range(3):
    print('Hello Python')
\end{lstlisting}
此时连续输出   \verb|'Hello Python'|  三次。

有时候我们并不知道循环多少次,此时可以用 \verb|while| 循环。 例如找到最小的 $n$,使得
\begin{equation}
\sum_{i=1}^n i^2<50~
\end{equation}
成立。它可以通过 \verb|while| 循环实现。 代码如下
\begin{lstlisting}[language=python]
a = 0
sum0 = 0.0
while sum0 < 50:
    a = a + 1
    sum0 = sum0+a**2
    print(a, sum0)
\end{lstlisting}
输出
\begin{lstlisting}[language=python]
1 1.0
2 5.0
3 14.0
4 30.0
5 55.0
\end{lstlisting}

\subsection{迭代器(iterator)}
\begin{itemize}
\item \verb|it = iter()| 函数可以获取 set, list 等的迭代器。 set 和 list 的迭代器分别为 \verb|set_iterator|, \verb|list_iterator|。 \verb|string| 也可以。
\item \verb|next(it)| 返回下一个值(包括第一个值), 没有 \verb|last| 函数
\item 循环如 \verb|for val in it:|
\item 自定义的类型中, 用 \verb|__iter__(self)| 和 \verb|__next__(self)| 函数定义以上两个函数的行为。
\end{itemize}
