% Julia数学函数速览
% 数学函数速览

本文授权转载自郝林的 《Julia 编程基础》. 原文链接:\href{https://github.com/hyper0x/JuliaBasics/blob/master/book/ch05.md}{第 5 章 数值与运算}.


\subsection{5.7 数学函数速览}

Julia 预定义了非常丰富的数学函数.一些常用的函数如下:

- **数值类型转换:** 主要有`T(x)`和`convert(T, x)`.其中,`T`代表目的类型,`x`代表源值.
- **数值特殊性判断:** 有`isequal`、`isfinite`、`isinf`和`isnan`.
- **舍入:** 有四舍五入的`round(T, x)`、向正无穷舍入的`ceil(T, x)`、向负无穷舍入的`floor(T, x)`,以及总是向`0`舍入的`trunc(T, x)`.
- **除法:** 有`cld(x, y)`、`fld(x, y)`和`div(x, y)`,它们分别会将商向正无穷、负无穷和`0`做舍入.其中的`x`代表被除数,`y`代表除数.另外,与之相关的还有取余函数`rem(x, y)`和取模函数`mod(x, y)`,等等.
- **公约数与公倍数:** 函数`gcd(x, y...)`用于求取最大正公约数,而函数`lcm(x, y...)`则用于求取最小正公倍数.圆括号中的`...`的意思是,除了`x`和`y`,函数还允许传入更多的数值.但要注意,这里的数值都应该是整数.
- **符号获取:** 函数`sign(x)`和`signbit(x)`都用于获取一个数值的符号.但不同的是,前者对于正整数、`0`和负整数会分别返回`1`、`0`和`-1`,而后者会分别返回`false`、`false`和`true`.
- **绝对值获取:** 用于获取绝对值的函数是`abs(x)`.一个相关的函数是,用于求平方的`abs2(x)`.
- **求根:** 函数`sqrt(x)`用于求取`x`的平方根,而函数`cbrt(x)`则用于求取`x`的立方根.
- **求指数:** 函数`exp(x)`会求取`x`的自然指数.另外还有`expm1(x)`,为接近`0`的`x`计算`exp(x)-1`.
- **求对数:** `log(x)`会求取`x`的自然对数,`log(b, x)`会求以`b`为底的`x`的对数,而`log2(x)`和`log10(x)`则会分别以`2`和`10`为底求对数.另外还有`log1p(x)`,为接近`0`的`x`计算`log(1+x)`.

除了以上函数之外,Julia 的`Base`包中还定义了很多三角函数和双曲函数,比如`sin`、`cos`、`atanh`、`acoth`等等.另外,在[`SpecialFunctions.jl`包](https://github.com/JuliaMath/SpecialFunctions.jl)里还有许多特殊的数学函数.不过这个包就需要我们手动下载了.