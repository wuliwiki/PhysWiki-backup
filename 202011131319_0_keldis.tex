% Keldysh 参数
% keys 隧道电离|加速度|电场

\begin{issues}
\issueOther{需要补充推导}
\issueDraft
\issueMissDepend
\end{issues}

\subsection{Ponderomotive Energy}
\footnote{参考 \cite{Brandsen} Chap.15.3}\textbf{Ponderomotive Energy} 定义为在平面电磁波(偶极子近似,磁场为零)中的带电粒子一个周期内的平均能量.
\begin{equation}
U_p = \frac{q^2 \mathcal E_0^2}{4m\omega^2}
\end{equation}
$\mathcal E_0$ 是电场, $\omega$ 是激光频率.

令电场为
\begin{equation}
\mathcal E(t) = \mathcal E_0 \sin(\omega t + \phi)
\end{equation}
点电荷 $q$ 在电场中的加速度为
\begin{equation}
a = \frac{q\mathcal E_0}{m} \sin(\omega t + \phi)
\end{equation}
速度为
\begin{equation}
v = -\frac{q\mathcal E_0}{m\omega} \cos(\omega t + \phi) + v_0
\end{equation}
动能为
\begin{equation}
\overline{E_k} = \frac{1}{2}m \overline{v^2} = \frac{1}{2}m \qty[-\frac{q\mathcal E_0}{m\omega} \cos(\omega t + \phi) + v_0]^2 = \frac{q^2\mathcal E_0^2}{4m\omega^2} + \frac{1}{2}mv_0^2
\end{equation}
所以当电子做简谐振动, 即 $v_0 = 0$ 时的平均动能就是 $U_p$. 如果是简谐振动和平移的叠加, 就要多加上平移的动能.

\subsection{ADK Rate}
M. Ammosov, N. Delone, V. Krainov. 三个人给出了一般原子的瞬时的 tunnelling ionization rate, 叫做 \textbf{ADK rate}.
\begin{equation}
\dv{P}{t} = W_{ADK} = \omega_p \abs{C_{n^*l^*}}^2 G_{lm}\qty(\frac{4\omega_p}{\omega_T})^{2n^*-m-1}\exp(-\frac{4\omega_p}{3\omega_T})
\end{equation}
其中
\begin{equation}
\omega_p = \frac{I_p}{\hbar} \qquad \omega_T = \frac{e\mathcal E_0}{\sqrt{2mI_p}} \qquad
n^* = \sqrt{\frac{I_p^H}{I_p}}
\end{equation}
\begin{equation}
\abs{C_{n^*l^*}}^2 = \frac{2^{2^{n^*}}}{n^* \Gamma(n^* + l^* + 1) \Gamma(n^* - l^*)}
\end{equation}
\begin{equation}
G_{lm} = \frac{(2l+1)(l+\abs{m})!}{2^{\abs{m}}\abs{m}!(l-\abs{m})!}
\end{equation}
其中 $I_p^H$ 是氢原子的电离能 $I_p$, $l,m$ 是角动量量子数, effective quantum number $l^* = 0$ 当 $l\ll n$, 否则 $l^* = n^*-1$.

\subsection{Keldysh parameter}
Keldysh parameter 是一个无量纲常数, 用于判断什么时候 tunnelling ionization 占主导
\begin{equation}
\gamma = \sqrt{\frac{I_p}{2U_p}} = \frac{\omega}{e\mathcal E_0} \sqrt{2m_e I_p}
\end{equation}
$I_p$ 为电离能量, $U_p$ 为 ponderomotive 能量. $\gamma$ 越小, Tunnelling ionization 越容易发生. $\gamma < 1$ 时 tunnelling ionization 会占主要部分.
