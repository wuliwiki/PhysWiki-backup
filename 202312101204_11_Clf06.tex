% 外同态
% license Xiao
% type Tutor


注:本文参考 Jier Peter 的《代数学基础》。

几何代数有两种运算结构:依赖于二次型的标量积和与二次型无关的外代数形式。外同态既是线性空间的线性映射,亦是保外代数形式不变的同态映射(名副其实的\textbf{外同态}嘛)。
\begin{definition}{}
给定域$\mathbb F$上的几何代数$\mathcal G(V,q)$,$\mathcal G(W,q)$,及线性映射$f:V\rightarrow W$。称$f_{\wedge}:\mathcal G(V,q)\rightarrow \mathcal G(W,p)$为外同态(outermorphism)或者$\wedge$-同态,具有如下性质:
\begin{equation}
\begin{aligned}
f_\wedge(1)&=1,\\
f_\wedge|_V&=1,\\
f_\wedge(x\wedge y)&=f(x)\wedge f(y),\quad\forall x,y\in\mathcal G(V,q)~.
\end{aligned}
\end{equation}
\end{definition}
一般几何代数之间的线性映射指的就是外同态。
由第三条可知,外同态$f_{\wedge}$取决于线性映射$f$。由于$f_{\wedge}|_V=f$,我们可得映射复合的外同态:$(g\circ f)_{\wedge}=g_{\wedge} \circ f_\wedge$。

线性空间的同态有矩阵表示,几何代数的外同态自然也有。
要导出矩阵表示,需要利用定义的三条性质。为了方便,我们在$\mathbb R^3$下讨论。
由于外积与空间是否退化无关,不妨选取标准正交基$\{e_i\}$。现在设$\mathbb R^3$上有线性映射$f$,对应矩阵表示如下:
\begin{equation}
F_b^a=\left(\begin{array}{lll}
a & b & c \\
d & e & f \\
h & i & j
\end{array}\right)~.
\end{equation}
与之对应,外积的$basis$如下:
$$e_{\emptyset}e_1,e_2,e_3,e_1$$








从矩阵表示里,我们可以清楚看到外同态具有明显的分次结构。