% 整数
\pentry{逻辑量词, 二元关系\upref{Relat}}

整数的概念是大家所熟悉的.严格的数学中对于整数的定义过于抽象,主要是为了逻辑基础的严谨性;严格定义依然是建立在我们熟知的整数概念上的,所以在物理学习中没必要深入到整数的定义上,读者只需要按照通常的理解方式来认识整数就可以. 我们把整数构成的集合称为\textbf{整数集(Set of Integers)}, 记为$\mathbb{Z}$.

整数通常涉及的运算有加法和乘法.加法的逆运算被称为减法,且减法在整数集上是封闭的,即两个整数相减仍为整数;乘法的逆运算被称为除法,除法在整数集上就不封闭,例如$2/3\not\in\mathbb{Z}$,尽管$2$和$3$都属于$\mathbb{Z}$.

在中小学中我们学过带余除法,用现代数学语言可以记成如下形式:当被除数为$a$,除数为$b$,\textbf{余数}为$c$的时候,存在一个整数$k$,使得$a=kb+c$. 作为练习,用逻辑量词来表达时,这句话可以写为:当被除数为$a$,除数为$b$,余数为$c$的时候,$\exists k\in \mathbb{Z}$, s.t.$ a=kb+c$. 

将除数固定为$b$,那么如果两个整数$a_1$和$a_2$除以$b$所得到的余数相同,我们就称 $a_1$ 和 $a_2$ \textbf{模} $b$ \textbf{同余}($a_1\equiv a_2$ mod $b$).这里\textbf{模}的意思类似于“除以”,\textbf{同余}的意思是“余数相同”.在这种情况下,我们也把$a_1$和$a_2$称为彼此的\textbf{模 $b$ 的同余数}. 为了简化表达,当默认了$b$的取值以后,我们也可以把mod$b$省略,简单地称$a_1$和$a_2$同余,并记为$a_1\equiv a_2$. 

由词条\textbf{二元关系}\upref{Relat}中对\textbf{等价关系}的讨论可以推知,给定整数$b$以后,模$b$同余的关系是一个整数集上的等价关系.也就是说,我们可以利用$b$把整数集划分为若干个等价类,称作\textbf{整数集模 $b$ 的同余类}.由等式$a=kb+c$可知,$a$就在模$b$余$c$的同余类中.在词条\textbf{二元关系}\upref{Relat}中所讨论的“两数的差是3的倍数”这一关系,其实就是一个模3同余的关系.

\subsection{钟表和模算数}

我们所熟知的加法、减法和乘法运算是在一个无穷集合上讨论的,即整数集.如果把“等于”的概念替换为“模同余”的概念,我们就可以在有限集合上进行加法、减法和乘法运算.最常见的例子,是钟表上整点的计算,这是一个模12同余的运算.

模运算的概念是,给定用来作除数的$b$以后,将模$b$同余的整数都看成同一个.在钟表的例子里,$b=12$,那么我们把0,12,24,72,156等都看成同一个元素,3,15,27,147,159等也都看成同一个元素.判断整数$a_1$和$a_2$是否模12同余的方法很简单,就是看$a_1-a_2$是否是12的倍数.这样,我们可以\textbf{把整数轴卷起来},让所有彼此同余的整数都重合起来,此时12点就是0点,14点就是2点,28点就是4点.为了方便讨论,我们只取$0,1,2,\cdots,11$分别作为12个同余类的代表,然后进行运算.

我们要讨论的是,从两个同余类中随便取一个数字,将它们进行加法,减法或者乘法运算后,结果落入哪一个同余类.在这样的运算规则下,用上一段所规定的代表元来进行讨论的话,我们可以得到以下运算的例子:
\begin{enumerate}


\item $1+1\equiv 2$
\item $4+8\equiv 0$
\item $11+8\equiv 7$
\item $3-2\equiv 1$
\item $5-7\equiv 10$
\item $2\times 3\equiv 6$
\item $3\times 4\equiv 0$
\item $4\times 7\equiv 4$


\end{enumerate}

模运算有一个美妙的性质.如果$a_1$和$a_2$同余,那么我们可以将它们的关系表达为$a_1=a_2+12k$,其中$k$是唯一存在的整数.利用这个关系容易证明,对于任意的整数$a_3$,我们都有$a_1+a_3\equiv a_2+a_3$,$a_1-a_3\equiv a_2-a_3$以及$a_1\times a_3\equiv a_2\times a_3$.这三个式子实际上说明了,在进行同余类的加法、减法和乘法的时候,无论取谁作为代表元素,结果都还是在同一个同余类里.因此我们才可以放心地把同余类的运算看作是任意代表元之间的运算,比如钟表整点的计算.

在通常的钟表中,我们把整数轴卷成周长为12的钟表.我们也可以把整数轴按其它周长来卷曲.当把整数轴卷曲成周长为$b$的钟表时,在钟表上进行的加减和乘法运算就是数学家们所讨论的模$b$的加减和乘法运算.

钟表的模型可以帮助你直观地理解模运算,以及初步感受“集合间的运算”是什么样的.在\textbf{群论(group theory)}、\textbf{环论(ring theory)}和\textbf{域论(field theory)}中,我们会遇到大量基于模运算概念的运算,所以请务必认真理解本词条的内容.

\subsection{素数}

根据小学所学定义,\textbf{素数(prime numbers)}是指那些除了1和自身以外没有别的因子的整数.

在代数学中,整数集是\textbf{交换环(commutative ring)}的一个特例,而素数是更广义的\textbf{交换环的素元素(prime element of a commutative ring)}的一个特例.根据交换环中素元素的定义,整数中的素数$p$是满足如下需求的整数:
如果$p$能整除$ab$,其中$a$和$b$是任意满足条件的整数组合,那么$p$必能整除$a$和$b$中的一个.用逻辑量词来简洁地表达,就是:$\forall a, b\in \mathbb{Z}$,如果 $p|ab$,那么要么$p|a$,要么$p|b$. 