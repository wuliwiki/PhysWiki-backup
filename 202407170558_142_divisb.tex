% 整除
% keys 数论|整除
% license Usr
% type Tutor

\begin{definition}{整除}
假设对于整数 $a$ 与非 $0$ 整数 $b$,存在第 $3$ 个整数 $c$ 使得
\begin{equation}
a = bc ~,
\end{equation}
就称 $a$ 能被 $b$ \textbf{整除(divisible)}。记作 $b | a$,称 $b$ 是 $a$ 的一个\textbf{因子(divisor)}。
\end{definition}

有时也用记号 $\nmid$、$\not{\mid}$ 表示不整除,也就是做除法会有余数,例如 $2 \not{\mid}~ 5$。

\begin{corollary}{整除的性质}
对于整数 $a$ 与非零整数 $b$,显然有:
\begin{itemize}
\item $1 | a$,$a | a$;
\item $b | 0$。
\end{itemize}
\end{corollary}

另外整除还有传递性等性质。
\begin{theorem}{整除的传递性}
若 $b|a$,$c|b$,则 $c|a$。
\end{theorem}
\begin{theorem}{}
若 $b|a$,则对于非零整数 $c \neq 0$ 有 $(bc) | (ac)$。
\end{theorem}
\begin{theorem}{}
若 $c|a$ 并且 $c|b$,则对于任意整数 $m$ 与 $n$ 都有 $c | (ma + nb)$。
\end{theorem}

\begin{definition}{}
符号 $a \opn{mod} b$ 表示 $a$ 除以 $b$ 的余数。这是指若 $r = a \opn{mod} b$,则 $a = kb + r$,且 $k$ 为整数并且 $0 \le r < b$。
\end{definition}