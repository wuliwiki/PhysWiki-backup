% 北京大学 2008 年 考研 普通物理
% license Usr
% type Note

\textbf{声明}:“该内容来源于网络公开资料,不保证真实性,如有侵权请联系管理员”

力学(50分)
\subsection{(16分)}
质点作平面运动,已知其速度为$\vec{v} = (A \omega \sin \omega t) \hat{i} + (B \omega \cos \omega t) \hat{j}$,且$t=0$时,$x=0,y=0$,其中$A,B,\omega$为大于零的常量。(1)求质点的轨迹方程,画出轨迹曲线,并用箭头画出质点的运动方向;(2)求求任意时刻$t$时质点的法向加速度和切向
加速度。
\subsection{(16 分)}
水平面内的大圆环(半径为$R$,圆心为$O$’)绕过环上一点A的竖直轴以匀角速度$\omega_0$,转动。质量为$m$的小圆环套在大圆环上可以无摩擦地渭动,相对速度可表示为$\vec{v} =R\dot{\theta}\hat e_\theta$以大圆环为参考系(1)用$m,\omega_0,R,\theta,\dot{\theta},\hat e_r,$和$\hat e_\theta$等表示出质点所受的惯性力和大环对它的约束力:(2)试求出$m$在大环上的相对平衡位置和在平衡位置附近作小振动时的周期。
\begin{figure}[ht]
\centering
\includegraphics[width=8cm]{./figures/566767e44e67e468.png}
\caption{} \label{fig_PKU008_1}
\end{figure}
\subsection{(18 分)}
我国发射的娥一号卫星在正式奔月前,曾在绕地球的3条大椭圆轨道上经过7天“热身”。已知其中的一个椭圆轨道的近地点距地面高约$200kmm$、远地点距地面高约$51000km$,地球半径约为$R_e=6400km$,$\sqrt{\frac{2GM_e}{R_e}} \dot= 11.2 \, \text{(km/s)}$(1)求卫星在该轨道上的最大和最小速率;(2)求其轨道周期。

电磁学(50分)

\subsection{(20 分)}
用文字语言(不用数学表达式)表述下面的物理定律或定理。
\begin{enumerate}
\item 库仑定律
\item 楞次定律
\item 安培环路定理
\item 静电场的高斯定理
\end{enumerate}
\subsection{(15 分)}
一根无穷长沿轴线均匀磁化的圆柱形永磁棒,其直径为$D$,圆柱内磁化强度为$M$。如果用垂直于磁棒轴线的平面把磁棒分成两段,求两段磁棒之间的相互作用力。
\subsection{(15 分)}
一个半径为$R$的导体球置于均匀外电场中,该电场的电场强度为$E$。求导体球表面的电荷分布。

光学(50分)

\subsection{(10分)}
简述惠更斯一菲涅耳原理,写出菲涅耳衍射积分公式(Kirchho纩公式),并扼要说明其各个量的物理意义。
\subsection{(10分)}
什么是时间相干性,什么是空间相干性?其物理本质是什么?分别举出测量光源的时间相干性和空间相干性的装置例子,并描述测量方法。
