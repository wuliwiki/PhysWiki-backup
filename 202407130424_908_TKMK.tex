% 托卡马克
% license CCBYSA3
% type Wiki

(本文根据 CC-BY-SA 协议转载自原搜狗科学百科对英文维基百科的翻译)

\textbf{托卡马克}(俄语:Токамáк)是一种使用强磁场将热的等离子体限制为圆环形状的装置,是为了产生受控核聚变而开发的几种磁约束装置之一。[1]

托卡马克最初是在20世纪50年代由苏联物理学家伊戈尔·塔姆(Igor Tamm)和安德烈·萨哈罗夫(Andrei Sakharov)根据奥列格·拉夫伦蒂耶夫(Oleg Lavrentiev)的一封信提出的概念。同时,托卡马克的第一个工作被认为是Natan Yavlinskii在T-1上的工作。已经证明,稳定的等离子体平衡需要绕在环面上呈螺旋状的磁力线。像z-pinch和stellarator这样的设备已经尝试过这样做,但显示出严重的不稳定性。现在被称为安全系数(数学标记为q)的概念的发展引导了托卡马克的发展;通过将反应堆的关键因素q设置为大于1,托卡马克装置有效地抑制了早期设计的不稳定性。

第一台托卡马克T-1于1958年开始运行。到20世纪60年代中期,托卡马克的设计开始显示出性能的极大提高。最初的结果于1965年公布,但被忽略了;莱曼·斯皮策在注意到他们测量温度的系统存在潜在问题后,立即将其驳回。第二组结果发表于1968年,这一次声称性能远远领先于任何其他机器,同样被认为是不可靠的。这导致了联合王国的一个代表团被邀请来进行他们自己的测量。这些结果证实了苏联的结论,并在1969年发表,导致各国疯狂建设托卡马克装置。

到20世纪70年代中期,世界各地使用了数十台托卡马克。到20世纪70年代末,这些机器已经达到了实际聚变所需的所有条件,尽管不是同时进行,也不是在一个反应堆中进行。随着盈亏平衡的目标在望,一系列新的机器被设计出来,它们将使用氘和氚的聚变燃料运行。这些机器,特别是欧洲联合环流器(喷气)、托卡马克聚变试验堆 (TFTR)和JT-60 ,有明确的目标达到盈亏平衡。

相反,这些机器显现出了限制其性能的新问题。解决这些问题需要更大,更昂贵的机器,超出了任何一个国家的能力。在1985年11月罗纳德·里根和米哈伊尔·戈尔巴乔夫达成初步协议后,国际热核聚变实验堆计划 (ITER)项目应运而生,并且仍然是发展实际聚变发电的主要国际合作。许多较小的设计和分支,如球形托卡马克,继续用于研究性能参数和其他问题。

\subsection{语源}
托卡马克这个词是俄语单词токамак的音译,是以下两者的首字母缩写:
\begin{itemize}
\item "тороидальнаясасесассагнитнымикатушками"(带磁性线圈的环形室;
\end{itemize}
或者
\begin{itemize}
\item "тороидальнаясамесасаксиальным магнитным полем"(具有轴向磁场的环形腔体。[2]
\end{itemize}
这个术语是由伊戈尔·戈洛温在1957年提出的,他是科学院测量仪器实验室的副主任,现在的库尔恰托夫研究所。类似的术语“tokamag”也曾被提出过一段时间。

\subsection{历史}
\begin{figure}[ht]
\centering
\includegraphics[width=6cm]{./figures/a022bdad1bb07f4d.png}
\caption{苏联邮票,1987:托卡马克热核系统} \label{fig_TKMK_1}
\end{figure}
\subsubsection{2.1 第一步}
1934年,马克·奥列芬特、保罗·哈特克和欧内斯特·卢瑟福是第一个在地球上实现聚变的人,他们使用粒子加速器将氘原子核射入含有氘或其他原子的金属箔中。[3]这使他们能够测量各种聚变反应的核截面,并确定氘-氘反应发生的能量低于其他反应,峰值约为10万电子伏特(100 keV)。[4]

基于加速器的聚变是不现实的,因为反应截面很小;加速器里的大部分粒子会从燃料中散开,而不是与燃料融合。这些散射使粒子失去能量,达到不再发生聚变的程度。因此,注入这些粒子的能量就消失了,很容易证明这比由此产生的聚变反应释放的能量要多得多。[5]

为了维持聚变和产生净能量输出,燃料的大部分必须提高到高温,这样它的原子就会不断高速碰撞;这就产生了“热核”这个名字,因为产生它需要很高的温度。1944年,恩里科·费米(Enrico Fermi)计算出,在大约5000万K的情况下,这个反应是可以自我维持的;在那个温度下,反应释放能量的速率足够高,使周围的燃料迅速升温,以保持温度不受环境损失,继续进行反应。[5]

在曼哈顿计划期间,第一个达到这些温度的实际方法是用原子弹创造的。1944年,费米在当时假设的氢弹背景下发表了一场关于核聚变物理学的演讲。然而,人们已经想到了一种可控核聚变装置,吉姆·塔克和斯坦尼斯拉夫·乌兰姆曾尝试使用聚能炸药驱动注入氘的金属箔,尽管没有成功。[6]

第一次尝试制造实用的核聚变机器是在英国,乔治·佩吉特·汤姆森在1945年选择了箍缩效应作为一种有前途的技术。在几次试图获得资助失败后,他放弃了,并要求两名研究生斯坦·卡曾斯(Stan Cousins)和艾伦·韦尔(Alan Ware)用剩余的雷达设备制造一种设备。该系统于1948年成功运行,但没有明确的证据表明核聚变,未能获得原子能研究机构的兴趣。[7]
\subsubsection{2.2 奥列格·拉夫罗夫(Oleg Lavrentiev)的信}
1950年,时驻扎在库页岛的一名无事可做的红军中士奥列格·拉夫罗夫给苏联共产党中央委员会写了一封信。信中概述了用原子弹点燃聚变燃料的想法,然后描述了一个利用静电场来控制稳定状态的热等离子体以产生能量的系统。[8][9]

这封信被送往安德烈·德米特里耶维奇·萨哈罗夫那里征求意见。萨哈罗夫指出“作者提出了一个非常重要但不一定是无望的问题”,并发现他在构想中主要关心的是等离子会撞击电极线,“宽网格和薄载流部分将几乎所有入射的原子核反射回到反应堆中。这一要求很可能与设备的机械强度不相容。”[8]

从处理的速度可以看出拉夫罗夫的信的重要性;这封信在7月29日被中央委员会收到,萨哈罗夫在8月18日发出了他的评论,到10月,萨哈罗夫和伊戈尔·塔姆已经完成了第一个关于核聚变反应堆的详细研究,他们在1951年1月申请了建造资金。[10]
\subsubsection{2.3 磁约束}
当加热到熔化温度时,原子中的电子电离,形成一种原子核和电子的流体,称为等离子体。与电中性原子不同,等离子体是导电的,因此可以被电场或磁场控制。[11]

萨哈罗夫对电极的担心导致他考虑使用磁约束代替静电。在磁场的情况下,粒子将围绕磁力线旋转。[11]当粒子高速运动时,它们产生的路径看起来像螺旋。如果排列的磁场使得磁力线平行且靠近,围绕相邻磁力线的粒子可能会碰撞并融合。[12]

这种磁场可以在螺线管中产生,螺线管是外部缠绕有磁体的圆柱体。磁体的组合磁场产生了一组沿着圆柱体长度延伸的平行磁力线。这种布置防止粒子侧向移动到圆柱体的壁上,但并不防止它们从末端跑出。这种排列方式可以防止颗粒向圆筒壁面的侧面移动,但却不能防止颗粒跑出圆筒的末端。解决这个问题的一个明显的办法是把圆柱体弯曲成一个甜甜圈的形状,这样这些线条就形成了一系列连续的圆环。在这种排列中,粒子无休止地循环。[12]

萨哈罗夫与伊戈尔·塔姆(Igor Tamm)讨论了这个概念,到1950年10月底,两人写了一份提案,并把它寄给苏联原子弹项目主管伊戈尔·库尔恰托夫(Igor Kurchatov)及其副手伊戈尔·戈洛文(Igor Golovin)。[12]但是,这一初步建议忽略了一个基本问题;当沿直线螺线管排列时,外部磁铁的间隔是均匀的,但当它们弯曲成环形时,它们在环内比在环外靠得更近。这导致不均匀的力量,导致粒子漂移远离他们的磁力线。[13][13]

萨哈罗夫在访问苏联核研究中心苏联科学院测量仪器实验室期间,提出了解决这一问题的两个可能办法。一种方法是在环面中心悬挂载流环。环中的电流会产生一个磁场,这个磁场会与外部磁铁产生的磁场混合。由此产生的磁场将被扭曲成螺旋状,因此任何给定的粒子都会发现自己反复出现在环面内外。由不均匀场引起的漂移在内外方向相反,所以在环绕环面长轴的多次轨道运行过程中,相反的漂移会相互抵消。另外,他建议使用外部磁铁在等离子体中产生电流,而不是使用单独的金属环,这样也会产生同样的效果。[13]

1951年1月,库尔恰托夫在利潘安排了一次会议,讨论萨哈罗夫的概念。他们获得了广泛的兴趣和支持,2月,一份关于这一主题的报告被转交给了监督苏联原子能工作的拉夫连季·帕夫洛维奇·贝利亚。有一段时间,没有任何回音。[13]
\subsubsection{2.4 里克特和融合研究的诞生}
1951年3月25日,阿根廷总统胡安·裴隆宣布,前德国科学家罗纳德·里克特已经成功地在实验室规模上生产聚变,这是现在称为休穆尔项目的一部分。世界各地的科学家都对这一消息感到兴奋,但很快得出结论:这不是真的;简单的计算表明,他的实验装置不能产生足够的能量将聚变燃料加热到所需的温度。[14]

尽管遭到核研究人员的驳斥,但广泛的新闻报道意味着政治家们突然意识到并接受了聚变研究。在英国,屡次遭到拒绝的汤姆森突然获得了大量的研究资金。接下来的几个月里,有两个基于pinch系统的项目开始运行。[15]在美国,莱曼·斯必泽读了休穆尔的故事,意识到它是错误的,并着手设计一台可以工作的机器。[16]5月,他获得了5万美元,开始研究他的仿星器概念。[17]吉姆•塔克(Jim Tuck)曾短暂返回英国,参观了汤姆森的缩放机。当他回到洛斯阿拉莫斯的时候,他也和斯皮策同时申请了资金,但是被拒绝了。但直接从洛斯阿拉莫斯的预算中得到了5万美元。[18]

苏联也发生了类似的事件。4月中旬,电物理仪器科学研究所的德米特里·埃夫雷莫夫带着一本杂志闯入库尔恰托夫的研究,杂志上有一篇关于里克特工作的报道,要求知道他们为什么被阿根廷人打败。库尔恰托夫立即联系了贝利亚,提议建立一个独立的聚变研究实验室,由列夫·阿特莫维奇(Lev Artsimovich)担任主任。仅仅几天之后,即5月5日,约瑟夫·斯大林签署了这项建议。[13]
\subsubsection{2.5 新想法}
到10月份,萨哈罗夫和塔姆已经完成了对他们最初提议的更详细的考虑,要求一个整个圆环的外半径为12米,内半径为2米的设备。该系统每天可以生产100克(3.5盎司)氚,或每天生产10公斤(22磅)U233[13]

随着这一想法的进一步发展,人们认识到等离子体中的电流可以产生一个足够强的磁场来限制等离子体,从而消除了对外部磁体的需求。在这一点上,苏联研究人员重新发明了英国正在开发的箍缩系统,[6]尽管他们是从一个截然不同的起点开始设计的。

一旦提出使用箍缩效应,一个简单得多的解决方案就显而易见了。人们可以简单地将电流导入一个线性管中,而不是一个大的环形管,这将导致管内的等离子体坍缩成一条细丝。这有一个巨大的优势;等离子体中的电流会通过正常的电阻加热使其升温,但不会将等离子体加热到熔化温度。然而,当等离子体崩溃时,绝热过程将导致温度急剧上升,远远超过聚变所需的温度。随着这一发展,只有戈洛温和纳坦·亚夫林斯基继续考虑更静态的环形布置。
\subsubsection{2.6 不稳定性}
1952年7月4日,尼古拉·菲利波夫(Nikolai Filippov)的团队测量了一台线性箍缩机释放的中子。列夫·阿齐莫维奇要求他们在得出核聚变发生的结论之前检查所有的东西,在这些检查中,他们发现中子根本不是聚变产生的。英国和美国的研究人员也经历了同样的线性排列,他们的机器表现出了同样的行为。但是这项研究的高度保密意味着没有一个小组意识到其他小组正在研究,更不用说有同样的问题了。[19]

经过大量研究,发现中子是由等离子体的不稳定性引起的。有两种常见的不稳定性“香肠”主要出现在线性机器中,“扭结”在环面机器中最常见。[19]这三个国家的团队都开始研究这些不稳定性的形成以及解决这些不稳定性的潜在方法。[20]美国的马丁·大卫·克鲁斯卡尔、马丁·史瓦西和苏联的沙夫拉诺夫对该领域做出了重要贡献。[21]

来自这些研究的一个想法被称为“稳定夹点”。这一概念在腔室的外部增加了额外的磁体,将出现在等离子体之前的缩放放电。在大多数概念中,外场相对较弱,因为等离子体是抗磁性的,所以它只穿透等离子体的外部区域。[19]当收缩放电发生时,等离子体迅速收缩,这个场被“冻结”在产生的细丝上,在其外层产生一个强场。在美国,这被称为“给等离子体一个主干”[22]

萨哈罗夫重新审视了他最初的环形概念,并对如何稳定等离子体得出了略有不同的结论。布局将与稳定收缩概念相同,但两个场的作用将相反。在新的布局中,外部磁体将更强大,以提供大部分限制,而不是提供稳定的弱外部场和负责限制的强收缩电流,而电流将更小,并负责稳定效果。
\subsubsection{2.7 步骤解密}
1955年,随着线性方法仍然受到不稳定性的影响,苏联制造了第一个环形装置。TMP是一种经典的箍缩机,与英美同时期的机型相似。真空室由陶瓷制成,放电光谱显示是二氧化硅,这意味着等离子体没有完全被磁场限制,并撞击了室壁。随后出现了两台使用铜质外壳的小型机器。[23]这种导电外壳原本是用来稳定等离子体的,但在任何一台尝试过的机器上都没有完全成功。[23]

随着进展明显停滞,库尔恰托夫于1955年召开了苏联研究人员全联盟会议,其最终目标是在苏联内部开展核聚变研究。[24]1956年4月,作为尼古拉·赫鲁晓夫和尼古拉·布尔加宁广泛宣传访问的一部分,库尔恰托夫前往英国。他主动提出要在前英国皇家空军(RAF)哈维尔(Harwell)的原子能研究机构(Atomic Energy Research institution)做一次演讲。在那里,他详细介绍了苏联核聚变努力的历史概况,震惊了主办方。[30]他花了一些时间,特别注意到早期机器中出现的中子,并警告说,中子并不意味着聚变。[25]

库尔恰托夫不知道,英国的 ZETA 稳定箍缩机正在原跑道的远端建造。当时,ZETA是最大、最强的聚变机器。ZETA得到了早期设计实验的支持,这些实验已经被修改为包括稳定性,旨在产生低水平的聚变反应。这显然是一个巨大的成功,1958年1月,他们宣布基于中子释放和等离子体温度的测量,在ZETA中实现了聚变。[26]

维塔利·沙夫拉诺夫和斯坦尼斯拉夫·布拉金斯基研究了新闻报道,并试图弄清楚其中的原理。他们考虑的一种可能性是使用弱“冻结”磁场,但最终拒绝了这种可能性,因为他们认为磁场不会持续足够长的时间。然后,他们得出结论,ZETA与他们研究的设备基本相同,具有强大的外磁场。[27]
\subsubsection{2.8 第一台托卡马克装置}
那时,苏联研究人员已经决定按照萨哈罗夫建议的路线建造一台更大的环形机器。特别是,他们的设计考虑了“克鲁斯卡尔算法-沙夫拉诺夫极限”中的一个重要观点;如果粒子的螺旋路径使它们围绕等离子体圆周运动的速度比环绕环面长轴运动的速度快,则扭结不稳定性将被强烈地抑制。[20]

今天,这个基本概念被称为安全系数。表示粒子绕长轴运行的次数与短轴运行的次数之比$q$,和“克鲁斯卡尔算法-沙夫拉诺夫极限”声明一样,只要$q>1$,扭结将被抑制。这条路径由外部磁铁相对于内部电流产生的磁场的相对强度控制。为了获得$q>1$,外部磁铁必须更强大,否则,必须降低内部电流。[20]

遵循这一标准,一个新的反应堆T-1开始设计,今天被称为第一个真正的托卡马克。[23]与ZETA相比,T-1使用了更强的外部磁体和更小的电流。T-1的成功使它被认为是第一个工作的托卡马克。[28][29][30][31]由于在“气体中强大的脉冲放电,以获得热核过程所需的异常高温”方面的工作,亚夫林斯基在1958年获得了的列宁奖和斯大林奖。Yavlinskii已经在准备设计一个更大的模型,后来被命名为T-3。随着ZETA的发布,Yavlinskii的概念得到了很好的评价。[27][31]

ZETA的详细信息在1999年在《自然》杂志1月下旬的一系列文章中公之于众。。令沙夫拉诺夫惊讶的是,该系统确实使用了“冻结”领域的概念。[27]但他仍然持怀疑态度,而位于圣彼得斯贝格 Ioffe Institute 的一个团队开始计划建造一种类似的机器,名为阿尔法。仅仅几个月后,在5月份,ZETA团队发布了一份声明,称他们还没有实现聚变,并且他们被等离子体温度的错误测量误导了。[32]

T-1于1958年底开始运行。[33]它通过辐射显示了非常高的能量损失。这是由于真空系统引起的等离子体中的杂质从容器材料中逸出。这是一种内部用波纹金属制成的内衬,在550°C(1022°F)的温度下烘烤,用来燃烧被困住的气体。[33]
\subsubsection{2.9 “原子促进和平和萧条”}
作为1958年9月在日内瓦举行的第二次原子促进和平会议的一部分,苏联代表团发表了许多论文,涵盖了他们的聚变研究。其中有一组关于他们的环形机器的初步结果,当时没有显示任何值得注意的东西。[34]

该会议的“明星”是斯皮策的仿星器的一个大模型,它立即引起了苏联人的注意。与他们的设计相反,仿星器在没有电流通过的情况下在等离子体中产生了所需的扭曲路径,使用了一系列可以在稳定状态下工作的磁铁,而不是感应系统的脉冲。库尔恰托夫开始要求亚夫林斯基将他们的T-3设计改为仿星器,但他们说服他,电流在加热方面提供了一个有用的第二个效果,这是仿星器所缺乏的。[34]

在进展中,仿星器遭遇了一长串的小问题,而这些小问题刚刚被解决。结果表明,等离子体的扩散速度远快于理论预测。由于这样或那样的原因,类似的问题出现在所有的当代设计中。星状器、各种各样的夹点概念以及美苏两国的磁镜机器都证明了限制其限制时间的问题。[33]

从最初对受控聚变的研究来看,就有一个隐藏在背后的问题。在曼哈顿计划期间,戴维·玻姆是研究铀同位素分离的团队的一员。战后,他继续研究磁场中的等离子体。根据基本理论,人们可以预期等离子体将以与场强平方成反比的速度沿力线扩散,这意味着力的微小增加将大大改善约束。但根据他们的实验,玻姆发展了一个经验公式,现在被称为玻姆扩散,表明速度是线性的磁力,而不是它的平方。[35]

如果玻姆的公式是正确的,人们就没有希望建造基于磁约束的聚变反应堆。为了将等离子体限制在聚变所需的温度范围内,磁场必须比任何已知的磁体大几个数量级。斯皮策将玻色子与经典扩散速率的差异归因于等离子体中的湍流,[36]并相信仿星器的稳定长不会受到这个问题的影响。当时的各种实验表明玻姆速率不适用,经典公式是正确的。[35]

但是到了20世纪60年代初,各种各样的设计以惊人的速度泄漏等离子体,斯皮策自己得出结论:玻姆尺度是等离子体固有的性质,磁约束不起作用。[33]整个领域陷入了后来被称为“萧条”的境地,[37]一段极度悲观的时期。
\subsubsection{2.10 托卡马克在20世纪60年代的进展}
与其他设计相比,实验托卡马克似乎进展得很顺利,以至于一个小小的理论问题现在成了一个真正的问题。在重力的存在下,等离子体中有一个小的压力梯度,以前小到可以忽略,但现在变成了必须解决的问题。这导致1962年又增加了一组磁体,产生的垂直磁场抵消了这些影响。这些工作都是成功的,到20世纪60年代中期,这些机器开始显示出超越玻姆极限的迹象。[38]

在1965年的第二届国际原子能机构核聚变会议上,阿特西莫维奇报告说,他们的核聚变系统超过了玻姆极限的10倍。斯皮策回顾了这些报告,认为玻姆极限仍然适用;这些结果在星型仪的实验误差范围内,而基于磁场的温度测量结果根本不可靠。[38]

下一次重大国际聚变会议于1968年8月举行新西伯利亚。到那时,另外两个托卡马克设计也已经完成,TM-2在1965年,T-4在1968年。T-3反应堆的测试结果持续改善,新反应堆的早期测试也得出了类似的结果。在会议上,苏联代表团宣布,T-3产生的电子温度为1000 eV(相当于1000万摄氏度),限制时间至少是玻姆极限的50倍。[39]

这些结果至少是任何其他机器的10倍。如果正确的话,它们代表了聚变领域的巨大飞跃。斯皮策仍然持怀疑态度,他指出温度测量仍然基于等离子体磁性的间接计算。许多人认为这是由于一种被称为“逃逸电子”的效应,苏联人只测量了那些能量极高的电子,而没有测量体积温度。苏联人反驳说,他们测量的温度是麦克斯韦式的,争论很激烈。[40]

\textbf{“文化五人组”}

ZETA之后,英国团队开始开发新的等离子体诊断工具,以提供更精确的测量。其中包括激光器:使用汤姆孙散射直接测量体电子的温度。这项技术在聚变领域中众所周知且令人钦佩;[41]阿齐莫维奇曾公开称之为“辉煌”。阿特西莫维奇邀请了库勒姆的负责人巴斯·皮斯在苏联的反应堆上使用他们的设备。在冷战最激烈的时期,英国物理学家被允许参观库尔恰托夫研究所(Kurchatov Institute),这是苏联核弹计划的核心。[42]

这支绰号为“文化五人组”的英国小队[43]于1968年底抵达伦敦。经过漫长的安装和校准过程,该团队在多次实验运行期间测量了温度。1969年8月获得初步结果;苏联人是正确的,他们的结果是准确的。该小组将结果打电话给卡勒姆,卡勒姆随后通过秘密电话将结果传递给华盛顿。[44]最终结果发表在1969年11月的《自然》杂志上。[45]这项宣布的结果被描述为世界各地托卡马克建设的“名副其实的成功”。[46]

还有一个严重的问题。因为等离子体中的电流比箍缩机低得多,产生的压缩也小得多,这意味着等离子体的温度受限于电流的电阻加热速率。斯皮策电阻率在1950年首次提出,它指出等离子体的电阻随着温度的升高而减小,这意味着随着设备的改进和温度的升高,等离子体的升温速度会减慢。[47]这意味着随着设备的改进和温度的升高,等离子体的升温速度会减慢。计算表明,在q > 1范围内的最高温度将被限制在数百万度以下。阿特西莫维奇很快在新西伯利亚指出了这一点,称未来的进展将需要开发新的加热方法。[48]

\textbf{美国动荡}

参加1968年新西伯利亚会议的人之一是阿玛萨·斯通,美国聚变计划的领导人之一,该计划是迄今为止世界上最大的计划。在当时,有明确证据证明可以超越玻姆极限的其他设备不多,多极概念就是其中之一。劳伦斯·利弗莫尔和斯皮策仿星器的发源地普林斯顿等离子体物理实验室 (PPPL)都在多极设计上进行了改进。T-3的表现大大优于两者;毕晓普担心多极是多余的,认为美国应该考虑自己的托卡马克。[49]

当他在1968年12月的一次会议上提出这个问题时,实验室的主管们拒绝考虑。梅尔文·戈特利布普林斯顿大学的教授很恼火,问“你认为这个委员会能比科学家们想得更远吗?”[50]由于主要实验室要求他们控制自己的研究,一个实验室发现自己被遗漏了。橡树岭最初进入聚变领域是为了研究反应堆燃料系统,但后来扩展到自己的镜像计划。到了20世纪60年代中期,他们在DCX的设计已经没有创意了,提供的任何东西都比不上更有声望和政治影响力的利弗莫尔的类似项目。这使他们成为美国唯一一个高度接受新概念的主要实验室。[51]

经过相当长的内部辩论,赫尔曼·波斯特在1969年初成立了一个小组来考虑托卡马克。[51]他们想出了一个新的设计,后来命名为 Ormak ,它有几个新颖的特征。其中最主要的是外部磁场在单个大型铜块中产生的方式,它由圆环下方的大型变压器供电。这与在外部使用磁体绕组的传统设计相反。他们觉得单个块会产生更均匀的场。它还有一个优点,就是允许环面有一个更小的主半径,不需要把电缆从甜甜圈孔中穿过,从而降低了纵横比,而苏联人已经建议过这样会产生更好的结果。[52]

\textbf{美国托卡马克蓬勃发展事件}

1969年初,阿齐莫维奇访问了麻省理工学院,在那里他被那些对核聚变感兴趣的人追逐。他最终同意在4月份举办几场讲座,[48]然后允许进行冗长的问答环节。随着时间的推移,麻省理工学院本身也对托卡马克产生了兴趣,之前由于种种原因,它一直没有涉足核聚变领域。布鲁诺·科皮当时在麻省理工学院,他和波斯特马的团队遵循同样的概念,提出了自己的低宽比概念——阿尔卡特。阿尔卡特使用了传统的环形磁铁,而不是Ormak的环形变压器,但要求它们比现有的设计小得多。麻省理工学院的Francis Bitter磁铁实验室是世界上磁铁设计的领导者,他们很有信心能够建造出这样的磁铁。[48]

1969年,又有两个团体进入该领域。在通用原子公司,Tihiro Ohkawa一直在开发多极反应堆,并提出了一个基于这些想法的概念。这是一个托卡马克它有一个非圆形的等离子体截面;同样的计算表明,较低的宽高比可以提高性能,同样的计算也表明,C形或d形等离子也可以提高性能。他称新设计为双重线。[53]与此同时,德克萨斯大学奥斯汀分校的一个研究小组提出了一个相对简单的托卡马克模型,用来探索通过故意诱导的湍流——德克萨斯湍流托卡马克来加热等离子体。[54]

1969年6月,原子能委员会核聚变指导委员会的成员再次开会时,他们“听到了托卡马克提议”。[54]普林斯顿大学(Princeton)是唯一一个没有提出托卡马克(tokamak)设计方案的主要实验室,尽管他们的C星状推进器(C stellarator)模型几乎可以完美地完成这种转换,但普林斯顿拒绝考虑托卡马克。他们继续提出了一长串理由,说明为什么不应转换C型。当这些被质疑时,一场关于苏联结果是否可靠的激烈辩论爆发了。[54]

看着辩论的进行,戈特利布改变了主意。如果苏联的电子温度测量不准确,托卡马克的研究就没有意义了,所以他制定了一个计划来证明或者反驳他们的结论。午休时间在游泳池游泳时,他告诉哈罗德·弗斯他的计划,弗斯回答说:“嗯,也许你是对的。”[44]午餐后,不同的团队展示了他们的设计,这时Gottlieb提出了基于C模型的“星月星-托卡马克”的想法。[44]

常设委员会指出,这一系统可以在六个月内完成,而Ormak将需要一年时间。[44]“文化五人组”的机密结果不久后才被公布。当他们在10月再次开会时,常务委员会公布了所有这些提案的资金来源。模型C的新结构,很快被命名为对称托卡马克,目的是简单地验证苏联的结果,而其他的将探索远远超过T-3的方法[55]
\subsubsection{2.11 加热方法:美国领先}
\begin{figure}[ht]
\centering
\includegraphics[width=10cm]{./figures/6cca1c58577b8468.png}
\caption{普林斯顿大圆环面的俯视图,摄于1975年。PLT是一个非常成功的托卡马克聚变装置,它创造了无数的记录,证明了聚变所需的温度是可能的。} \label{fig_TKMK_2}
\end{figure}
对称托卡马克实验始于1970年5月,到第二年年初,他们证实了苏联的实验结果。现在,PPPL将其相当多的专业知识用于解决等离子体加热的问题。有两个概念似乎很有希望。PPPL提出了使用磁压缩的方法,这是一种类似于捏的技术,通过压缩温暖的等离子体来提高它的温度,但是通过磁铁而不是电流来提供这种压缩。[56]橡树岭实验室提出了中性束注入,一种小型粒子加速器,它可以通过周围的磁场发射燃料原子,使它们与等离子体发生碰撞并使其升温。[57]

PPPL公司的绝热环形压缩机(ATC)在1972年5月开始运行,不久之后,配备了中性点光束的Ormak也开始运行。两者都存在明显的问题,但PPPL通过在ATC上安装喷射器,超越了橡树岭,并在1973年提供了成功加热的明确证据。这一成功“抢先”了橡树岭,而橡树岭在华盛顿指导委员会中已失宠。。[58]

这时,一个基于中性束加热的更大的设计正在建造中普林斯顿大圆环,或PLT。PLT是专门设计来“明确指出托卡马克概念加上辅助加热是否可以构成未来聚变反应堆的基础”。[59]PLT是一个巨大的成功,不断提高它的内部温度,直到它达到6000万摄氏度(8000 eV,是T-3纪录的8倍)。这是托卡马克发展的一个关键点;PLT证明,在5000万至1亿摄氏度的温度下,核聚变反应可以自我维持。[59]

这些实验,尤其是PLT,使美国在托卡马克研究中遥遥领先。这主要是由于预算问题;托卡马克的成本约为50万美元,当时美国聚变预算约为2500万美元。[39]他们有能力探索所有有前途的加热方法,最终发现中性束是最有效的方法之一。[60]

在此期间,罗伯特·希尔施接管了美国原子能委员会聚变发展局。赫希认为,如果没有明显的结果,该计划就无法维持目前的资助水平。他开始重新制定整个计划。一度主要是科学探索的实验室主导的努力,现在变成了华盛顿主导的建造工作中的发电反应堆的努力。[60]这得到了第一次石油危机的推动,导致对替代能源系统的研究大大增加。[61]
\subsubsection{2.12 80年代:希望越大,失望越大}
\begin{figure}[ht]
\centering
\includegraphics[width=10cm]{./figures/b7834c47030c963e.png}
\caption{欧洲联合环流器 (JET),目前运行的最大托卡马克,自1983年开始运行} \label{fig_TKMK_3}
\end{figure}
到20世纪70年代末,托卡马克已经达到了一个实用的聚变反应堆所需的所有条件;1978年PLT演示了点火温度,次年苏联的T-7首次成功地使用了超导磁体,双重态被证明是成功的,并导致几乎所有未来的设计采用这种“形状等离子体”的方法[62]似乎建造一个发电反应堆所需要做的就是把所有这些设计理念整合到一台机器中,一台能够在燃料混合物中运行放射性氚的机器。[63]

比赛开始了。在20世纪70年代,全球资助了四项主要的第二代提案。苏联继续发展T-15,[62]而泛欧和日本发展欧洲联合环流器,开始了JT-60工作(最初称为“盈亏平衡等离子体测试设施”)。在美国,赫希开始为类似的设计制定计划,跳过了另一个直接燃烧氚的踏脚石设计的提议。这就是托卡马克聚变试验反应堆(TFTR),它直接由华盛顿运行,不与任何特定的实验室连接。[63]赫希最初支持橡树岭作为东道主,但在其他人说服他他们会尽最大努力做这件事,因为他们损失最大之后,他把它搬到了PPPL。[64]

这种兴奋是如此普遍,以至于大约在这个时候开始了几家生产商用托卡马克的商业企业。其中最著名的是1978年,《阁楼》杂志的出版人Bob Guccione遇到了Robert Bussard,成为世界上最大和最坚定的核聚变技术私人投资者,最终将自己的2000万美元投资到Bussard的托卡马克上。里格斯银行的资助导致了这项被称为里格特龙。[65]

TFTR赢得了建造比赛,并于1982年开始运营,随后是1983年的JET和1985年的JT-60。JET很快在关键的实验中取得了领先地位,从测试气体到氘以及越来越强大的“射击”。但很快就发现,这些新系统都没有达到预期的效果。出现了许多新的不稳定因素,同时还出现了一些更实际的问题,这些问题继续干扰它们的性能。除此之外,在TFTR和JET中,等离子体撞击反应堆壁的危险“短途旅行”是显而易见的。即使在工作完美的情况下,等离子体限制在聚变温度下,即所谓的“聚变三重产物”,仍然远远低于实际反应堆设计所需的水平。

到了20世纪80年代中期,许多这些问题的原因变得很清楚,并提供了各种解决方案。然而,这将显著增加机器的尺寸和复杂性。如果在后续设计中加入这些变化,将是巨大的,而且比JET或TFTR都要昂贵得多。一个新的悲观时期降临在核聚变领域。
\subsubsection{2.13 ITER}
\begin{figure}[ht]
\centering
\includegraphics[width=10cm]{./figures/6226e6ad5b688877.png}
\caption{请添加图片标题} \label{fig_TKMK_4}
\end{figure}
