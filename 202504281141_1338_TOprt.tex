% 时间演化算符(量子力学)
% keys 薛定谔方程
% license Xiao
% type Tutor

\pentry{量子力学的基本原理(量子力学)\nref{nod_QMPrcp}}{nod_1d70}




本节介绍时间演化算符,以此为切入点,引入量子态的演化方程,即薛定谔方程。


“我们应当记住的首要之点是:时间在量子力学中只是一个参量而不是一个算符。特别地,时间不是前一章所说的可观测量。像谈论位置算符一样谈论时间算符是无意义的。”——樱井纯,J. 拿波里塔诺,《现代量子力学》,2.1节。

量子力学中,时间不是一个算符,意味着量子力学认为时间是独立存在的,即采用经典时空观。


\subsection{时间演化算符的基本性质}

\begin{definition}{时间演化算符}\label{def_TOprt_1}
设一个物理系统在时间$t$时的态矢量为$\ket{s, t}$,而$t_0<t$是一个初始时间,那么定义
\begin{equation}\label{eq_TOprt_1}
\mathcal{U}(t, t_0)\ket{s, t_0}=\ket{s, t}~,
\end{equation}
其中$\mathcal{U}(t, t_0)$称为从$t_0$到$t$的\textbf{时间演化算符(time evolution operator)}。
\end{definition}

从\autoref{def_TOprt_1} 可以看出来,我们只需要研究清楚时间演化算符的性质,就能从一个初始态算出之后任意时间的态。

那么时间演化算符应该具有什么样的性质呢?


首先,时间演化算符只依赖于时间长短,即
\begin{equation}\label{eq_TOprt_2}
\mathcal{U}(t_2, t_1)\mathcal{U}(t_1, t_0) = \mathcal{U}(t_2, t_0)~,
\end{equation}
这样才能确定唯一的演化结果$\ket{s, t}$。

于是,我们可以省略掉初始时间,而把$\mathcal{U}(t, t_0)$简记为$\mathcal{U}(t-t_0)$,即把自变量由“初始时间和结束时间”替换为“演化所用时间”。同样,也可以把量子态$\ket{s, t_0}$简记为$\ket{s}$。此时$\ket{s, t_0+t}=\mathcal{U}(t)\ket{s}$。

为了\textbf{方便讨论},以下\textbf{默认}$t_0=0$,故$\ket{s, t}=\mathcal{U}(t)\ket{s}$。

接着,我们希望量子态随时间\textbf{连续地}变化,因此有
\begin{equation}\label{eq_TOprt_3}
\lim_{t\to 0}\mathcal{U}(t) = \mathcal{U}(0) = 1~,
\end{equation}
其中$1$是恒等变换。

最后,我们希望一个量子态归一化以后,在演化过程中\textbf{始终保持归一化}。也就是说,$\bra{s}\mathcal{U}^\dagger \mathcal{U}\ket{s}=\braket{s}{s}=1$对任意态$\ket{s}$成立,即
\begin{equation}\label{eq_TOprt_4}
\mathcal{U}^\dagger \mathcal{U}=I~,
\end{equation}
$I$为恒等变换。满足上式 的算符被称为\textbf{幺正(unitary)}的。



有了三条规则,\autoref{eq_TOprt_2},\autoref{eq_TOprt_3} 和\autoref{eq_TOprt_4},就可以推导时间演化算符的具体形式了。


\subsection{无穷小时间演化算符}

我们首先考虑演化用时趋于$0$时,时间演化算符的极限。这是因为微分的思想即线性近似的思想,而线性的情形是最好处理的。为了方便,我们将不使用极限语言,而是用“无穷小”的术语,这并不失严谨性。

记$\dd t$是一段无穷小时间,则由连续性\autoref{eq_TOprt_3},可知应设
\begin{equation}\label{eq_TOprt_5}
\mathcal{U}(\dd t) = 1+\Omega \dd t~,
\end{equation}
其中$\Omega$是某个确定的算符。

\autoref{eq_TOprt_5} 的形式天然满足\autoref{eq_TOprt_2} :
\begin{equation}
\begin{aligned}
\mathcal{U}(t_2)\mathcal{U}(t_1)&=(1+\Omega \dd t_2)(1+\Omega \dd t_1)\\
&=1+\Omega(\dd t_2+\dd t_1)\\
&=\mathcal{U}(t_2+t_1)~.
\end{aligned}
\end{equation}

接下来要确定$\Omega$的形式。根据幺正性\autoref{eq_TOprt_4},我们有
\begin{equation}
(1+\Omega^\dagger \dd t)(1+\Omega \dd t)=1~,
\end{equation}
展开后,剔除高阶无穷小项$\dd t^2$,可得到
\begin{equation}
\Omega^\dagger + \Omega = 0~.
\end{equation}
因此,$\Omega$是一个\textbf{反厄米算符}(\autoref{def_QMPrcp_19})。


现在,借用经典力学中“哈密顿量是时间演化生成元”的概念,令$\Omega$为哈密顿算符$H$的某个倍数。考虑到哈密顿算符是可观测量(能量),应为\textbf{厄米算符},再考虑到量纲,故可以设
\begin{equation}\label{eq_TOprt_6}
\Omega=-\frac{\I}{\hbar}H~,
\end{equation}
从而得到\textbf{无穷小时间演化算符}
\begin{equation}\label{eq_TOprt_11}
\mathcal{U}(\dd t)= 1-\frac{\I}{\hbar}H=1+\frac{1}{\I\hbar}H~,
\end{equation}

这里我们直接给出了调整量纲的常量$\hbar$。为什么是$\hbar$,而不是$h$或别的什么同量纲量呢?这个问题在\autoref{sub_TOprt_1} 中解答。



\subsection{一般的时间演化算符}

将\autoref{eq_TOprt_11} 代入\autoref{eq_TOprt_2},可得
\begin{equation}\label{eq_TOprt_7}
\mathcal{U}(t+\dd t) = \mathcal{U}(t)\mathcal{U}(\dd t) = \mathcal{U}(t)\left(1-\frac{\I}{\hbar}H \dd t\right)~.
\end{equation}

因此有\footnote{$\partial\mathcal{U}(t)/\partial t=(\mathcal{U}(t+\dd t)-\mathcal{U}(t))/\dd t$,代入\autoref{eq_TOprt_7} 即可。}
\begin{equation}\label{eq_TOprt_8}
\I\hbar\frac{\partial}{\partial t}\mathcal{U}(t) = H\mathcal{U}(t)~,
\end{equation}


\autoref{eq_TOprt_8} 被称为\textbf{时间演化算符的薛定谔方程}。由时间演化算符的定义,我们可以由此得到量子态的演化方程\footnote{$\partial\ket{s, t}/\partial t=\partial(\mathcal{U}(t)\ket{s})/\partial t=\partial \mathcal{U}(t)/\partial t \ket{s}$。}:
\begin{equation}\label{eq_TOprt_9}
\I\hbar\frac{\partial}{\partial t}\ket{s, t} = H\ket{s, t}~.
\end{equation}
当选择$\displaystyle H=\frac{\hat{\bvec{p}}^2}{2m}+V(\bvec{x})$时,\autoref{eq_TOprt_9} 正是我们熟知的薛定谔时间相关波动方程\footnote{准确来说,为了从态右矢得到波函数,还需左乘位置算子的本征矢$\bra{x}$,该本征矢在薛定谔绘景下不随时间变化。}。


我们只需要关注$\mathcal{U}(t)$的演化即可,无须求解\autoref{eq_TOprt_9}。

\autoref{eq_TOprt_8} 的解需要分三个情况讨论:


\subsubsection{哈密顿算符不依赖于时间}

如果$H$是一个不随时间改变的常量,那么根据指数映射的性质,由\autoref{eq_TOprt_8} 易得
\begin{equation}
\mathcal{U}(t) = \exp\qty(\frac{H}{\I \hbar}t)~.
\end{equation}

其中$\displaystyle \exp(X)=\sum_{i=0}^\infty \frac{1}{i!}X^i$对所有算符$X$成立。



\subsubsection{不同时间的哈密顿算符彼此对易}

如果哈密顿算符作为时间的函数$H(t)$,满足$H(t_1)H(t_2)=H(t_2)H(t_1)$,则可类比\autoref{exe_ODEa2_1},猜出形式解
\begin{equation}\label{eq_TOprt_10}
\mathcal{U}(t)=\exp\qty(\frac{1}{\I\hbar}\int^t_{t_0}\dd t' H(t'))~,
\end{equation}
易验证\autoref{eq_TOprt_10} 确实满足\autoref{eq_TOprt_8}。



\subsubsection{不同时间的哈密顿算符彼此不对易}

如果$H(t_1)H(t_2)\not=H(t_2)H(t_1)$,那么\autoref{eq_TOprt_10} 就不再满足\autoref{eq_TOprt_8} 了。此时形式解应为所谓的\textbf{戴森(Dyson)级数}:
\begin{align}
\mathcal{U}(t) &= 1+\sum_{n=1}^\infty \qty(\frac{1}{\I \hbar})\int^{t_1}_{t_0}\dd t_1\int^{t_2}_{t_0}\dd t_2\cdot \int^{t_{n-1}}_{t_0}\dd t_{n-1}H(t_1)H(t_2)\cdots H(t_n) \\
&=\hat T \exp(-\frac{i}{\hbar}\int^t_{t_0}\hat H(t_0)dt_0)~,
\end{align}

式中$\hat T$为时序算符,在 \enref{戴森级数}{Dyser}中有较为详细的介绍。

\subsection{$\hbar$的由来}\label{sub_TOprt_1}

这一小节回答之前遗留的问题,即为什么$\mathcal{U}(\dd t) = 1-\frac{\I}{\hbar} H \dd t$中选择$\hbar$。

如果尚未确定$\hbar$,只是用一个常数$C$来取代它,即设$\mathcal{U}(\dd t) = 1-\frac{\I}{C} H \dd t$,那么\autoref{eq_TOprt_9} 变为
\begin{equation}
\I C\frac{\partial}{\partial t}\ket{s, t}=H\ket{s, t}~.
\end{equation}

根据德布罗意关系$E=h\nu$和$p=h/\lambda$,可以取能量、动量的共同本征矢(位置表象)\footnote{注意到$\exp\qty[2\pi \I \qty(A\abs{\bvec{x}}-Bt)]$中,$A=1/\lambda$,$B=\nu$。}
\begin{equation}
\psi(t, \bvec{x})=\exp[2\pi\I\qty(\frac{\bvec{p}\cdot\bvec{x}}{h}-\frac{Et}{h})]~,
\end{equation}
从而有
\begin{equation}
E\ket{s, t}=H\ket{s, t}=\I C\frac{\partial}{\partial t}\psi(t, \bvec{x}) = \frac{2\pi C E}{h} \psi(t, \bvec{x})~,
\end{equation}
故
\begin{equation}
C=h/2\pi=\hbar~.
\end{equation}








