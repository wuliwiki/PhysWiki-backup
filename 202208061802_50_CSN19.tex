% 2019 年计算机学科专业基础综合全国联考卷
% 2019 年计算机学科专业基础综合全国联考卷

\subsection{一、单项选择题}
1~40小题, 每小腿2分, 共80分.下列每题输出的四个选项中,只有一个选项符合试题要求.

1.设$n$是描述问题规模的非负整数,下列程序段的时间复杂度是
\begin{lstlisting}[language=cpp]
x=0;
while(n>=(x+1)*(x+1))
    x=x+1;
\end{lstlisting}
A. $O(logn)$  $\quad$  B.$O(n)$  $\quad$   C.$O(n)$  $\quad$  D.$O(n^2)$

2. 若将一棵树$T$转化为对应的二叉树$BT$,则下列对$BT$的遍历中,其遍历序列与T的后根遍历序列相同的是

  A.先序遍历    B.中序遍历    C.后序遍历   D.按层遍历3.   n个互

  对    不相同的符号进行阶犬虽编码.若生成的哈夫曼树共有115个    n    是

    结点,则  的值

  A.56    B.57    C. 58    D.60

    一    A

4.在任意  棵iF?平衡二叉树( VL树)T中,删除某结点u之后形成

    i

  平衡二叉树T,再将u插入T形成平衡二叉树T. 下列关于Ti与

    z    z    ]

  飞的叙述中,正确的是

   I .若u是Ti的叶结点,则Ti与T]可能不相同

   II .若u不是T的叶结点,则T与T-定不相同

    i    i    3

  田.若v不是T的叶结点,则T与T一定相同

    i    i    ]

  A.仅I    B.仅H    C仅I , II    D.仅I 、田

    图所示的AOE网表示一项包含8个活动的工程  活动d的最早5.下

  开始时间和最迟开始时间分别是

    门.设外存立有120个初始归并段,进行12路Hl并时,为实现最佳归

    井,简要补充的虚段个数是

    A.I    B.2    C.3    D.4

    12.下列关于冯·诺依曼结构计算机基本思想的叙述中,错误的是

    . 5和15    A. 程序的功能都通过中央处理器执行指令实现

    . 2和12    C. 2和14    D

  A.3和7    B

    B.指令和数据部用二迸制表示,形式上无差别

    /x),需要的顶点个数至

6.    向无环图描述表达式(x+y) * ((x+y)

   用有    C.指令按地址访问,数据都在指令中直接给出

  少是    D.    和

    程序执行前,指令   数据'预先存放在在储器中

   A.5    B.      C.      D.9

    13.考虑以下C语言代码:

    ,下列队l亲巾,压芮要考

7.选   个排序算法时,除算法的时空效率外    unsigned short usi = 65535;

    择一

  虑的是    short si = usi;

    II .数据的在储方式

   I 数据的规樵    s

    执行上述程序段后,i的值是

   皿    I.数据的初始状态    A. I     B. -32767    C. -32768    D. 65535

    .n法的稳定性

   A. 仅皿    B.仅I、H    14.

    下列关于缺页处用的叙述巾,错误的是

  C    TV    D.1 、H 、皿、W

    .仅E、皿、    A.缺页是在地.bl:转换时CPU检测到的-种异常

8.    T    =

   现有|度为ll且初始为空的散列在H ,散列的数是 H(key)  key    B.缺贞处理由操作系统提供的缺页处理程序米完成

   %7,采用线性探街(线忡,探测再散列)法解决冲突  将关键字序列

    C.缺页处理程序根据页故障地址从外仔读入所缺失的页

    T    HT奇找失败的平均冕

   87  40 , 0  6  II , 22,饵,20依次插入到H 后,    D.缺贞处理完成后l剑发生缺贞的指令的下一条指令执行

    15.某计算机采用大端方式,战?节编址. 某指令中操作数的机器数

   战长度是

   A.      8. .25     C.      D. 6.29    1234FFOOH    用基址寻址方式,形式地址(用补码占

    为    ,眩操作数采

    “    ”    “    ”

9.设主申T= abaabaalcabaa be ,模式申S= abaabc ,采用KMP"法    尽)为FFl2H,基址寄存器内容为FOOOOOOOH,则该操作数的LSB进行模式匹配,到匹配成功111为止,在匹配过程中进行的单个于符    (扯低有效宁节)所在的地址是

    A. FOOO FFl2H     B. OOOFF15日

   间的比较次数是

   A.9    B. 0     C. 12    D. 5     C. EFFF FF12H     D. FFF FFl5H

10.    一    16.

    排厅过秤,巾,对尚未确定最终位置的所有元东进行  遍处理称为    下列有关处理器时钟脉冲的号的叙述中,错误的是

    ”    A.

    一“跑 . 下列序列巾,不11J能是快速排印第二趟结果的是    n,J钟脉冲信号111机器脉冲源发出的脉冲信号经整形和分频后

    A.5,2,16,12,28,60,32,72    形成

    B.时钟脉冲信号的宽度称为时钟周期,M钟周期的倒数为机器

    B. ,16,5,28,12,60,32,72

    C. ,12,16,5,28,32,72,60    主频

    C.Hf

    D. ,2,12,28,16,32,72,60    钟周期以相邻状态单元间组合逻辑rl路的故大延迟为基准    确定    则 CPU用于该设备输入/输出的时间占整个CPU时间的百分比最

   D    一

    .处理器总是在每来一个时钟脉冲信号时就开始执行  条新的

    多是

    指令    .1.25%    B.2.5%    C.5%    D.12.5%

    A

17.某    R[2]←R[ ]+M[R[rO]],其两个源操作数分别采

    指令功能为   r    rl    22.下列关于DMA方式的叙述中,正确的是

   用寄存器、寄存器间接寻址方式. 对于下列给定部件,该指令在取    I DMA传送前由设备驱动程序设置传送参数

   数及执行过程中需要用到的是    1数据传送前由DMA控制器请求总线使用权

   I 通用寄再器组(GPRs)    .

    I算术逻辑单元(ALU)    m.数据传送由DMA控制器直接控制总线完成

   皿.存储器(Memry)o    N.指令译码器(JD)

    N. MA传送结束后的处理由中断服务程序完成

    B.仅l, T、田    A.仅I、H    B.仅

   A.仅I、E    I、E、W

   C.仅E、皿、W    D仅I、田、W    C.仅E、皿、W

    D.  、E、E、W

18.    “    ”

   在采用 取指、译码/取数、执行、访在、写回 5段流水线的处理器

    23.下列关于线程的描述中,错误的是

   中,执行如下指令序列,其中sO、l、s  s2、s3和t2-曼示寄再器编号.

    A.内核级线程的调度由操作系统完成

   IL:   adds2,sl,s0     //R[s2]←R  I ] + R  Os     s

    B.操作系统为每个用户级线程建立一个线程控制块

   12    load s3  0  t2)     I  R  3 ←M[ [ 2] + 0]s

    C.用户级线程间的切换比内核级线程间的切换效率高

    /  R[ 2]←R  2] + R  3s

   I3    add 2  s2  s3    .   户级线程可以在不支持内核级线程的操作系统上实现

    D 用

   14:   store s2, 0( L2)     I  M [   t2  + 0 ]←R[ 2]    24.

    下列选项中,可能将进程唤醒的事件是

   下列指令对中,不存在数据冒险的是    I. 110结束    n.某进程退出临界区

   A.II 和I3    B.12和I3    C.12和14    D.I3和14    .

    皿 当前进程的时间片用完

19.假定一台计算机采用3通道存储器总线,配套的内存条型号为

    A.仅I    B.仅E    C.仅I、E    D. 、E、E

   DDR3-l333,即内存条所接插的存储器总线的工作频率为 1333    25.

    下列关于系统调用的叙述中,正确的是

   MHz、总线宽度为64位,则存储器总线的总带宽大约是

    I .在执行系统调用服务程序的过程中,CPU处于内核态

   A. 10.66 GB/s    B.32 GB/s    C.64 GB/s    D.96 GB/s

    n.操作系统通过提供系统调用避免用户程序直接访问外设20.下列关于磁盘再储器的叙述中,错误的是

    皿.不同的操作系统为应用程序提供了统一的系统调用接口

   A.磁盘的格式化容量比非格式化容茧小    N.系统调用是操作系统内核为应用程序提供服务的接口

   B.扇区中包含数据、地址和校验等信息    A.仅I

    、W    B.仅E、E

   C磁盘存储器的最小读写单位为一个字节    仅I、E、W    .仅

    C    D   I、皿、W

   D.    和

    磁盘存储器由磁盘控制器、磁盘驱动器  盘片组成    26.下列选项中,可用于文件系统管理空闲磁盘块的数据结构是21.某设备以中断方式与CPU进行数据交换,CPU主频为I GHz,设备

    I.位图    n.

    索引节点

   接口中的数据缓冲寄存器为32位,设备的数据传输率为50kB/s.    .

    皿 空闲磁盘块链    N.文件分配表(FAT)

   若每次中断开销(包括中断响应和中断处理)为l000个时钟周期,    A 仅I、E    B

    .    .仅I、E、W

    应拟地址20501225H对州的贞    别

   C.仅I 、皿    D仅E、田、W    口录号、反号分  是

    A.081 H、101H    B.081 H、401H

27.系统采用二级反馈队列调度算法进行进程调度. 就绪队列QI 采

    C.201  、l01     D.201  、40111

    10ms    列Q2采用短进程

   用时间片轮转调度算法,时间片为    ;就绪队

    32.在下列动态分区分配算法巾,直容劫产生内在碎片的是

   优先调度算法;系统优先调度QI队列中的进程,当Ql为空时系统

    A.首次适应11法    B.最坏适j也算法

    会    2  的进程;新创建的进程首先进入Ql; Ql中的进程

   才  调度Q 中

    一    C.MH适应算法    D.循

    ,则转   Q2. 若当前.l、Q2为空,系    环芮次适!但1;法

   执行 个时间片后,若未结束    入

    2    Pl、四百要的CPU时    33.OSI参考模咽的第5层(白下而t)完成的主要功能是

   统依次创建进程Pl、P 后即开始进程调度

    J. »

   间  别   30ms和20 ms,则进程Pl、P2在系统中的平均等待时    铺控制    B.路Lli滥炸

    分  为

    C.会iZ·管    D.数

   间为    理    据-if转换

    B.20 ms    C.15 ms    D.10 ms    34. IOOBaseT快速以太网使用的导

   A.25 ms    向传输介质是

    所有被共字的段. 丰进    八.双绞线    B.单悦光纤    C.多快J纤   D.f,d都[f电缆28.在分段再储管理系统中,用共学段在描述

    l和P2共字段S,下列叙述1p,错误的是    35.对于滑动窗口协议,如果分mrr号采用3 比特编号,发送窗口大小

   程P

   A.在物理内存中仅保在一份段S的内容    为5,则接收窗口届大是

   B.    和P2rj1应该具有相同的段号    A.2    B.3    C.4    D.5

    段S在Pl

    36.    一

   C.Pl JP2共享段S在共字段友小的段丘项    假设  个采用CSMA/CD协议的100Mhps局域网,段小帧i足128

    B    一

   D.Pl和P2都不再使用段SJf才i口l收段S所山的内行空间    ,则在  个冲突域内两个站点之间的机,1,1传播延时监多是29.    用LRU页页换n法和局部lm换策略,拧系统为迸程P1�!    A.2.56 µs    B.5.12 µs    C.10.24 µs   D.20.48 µs

   某系统采

    4    0,1,2,7,0,5,,3 5,0,    7.行

   分配了  个页框,进程P访问页号的rr列为    3    将 l01.200.16.0/20划分为5个子网,则可能的最  子网的可分

    小

   2,7,6   进程访问上述页的过程中,产生J{.'换的总次数是    Ic IP地址数是

    ,则

    B.4    C.    D.6

    A.3    5    A.126    B.254    C.510    D.102230.    一

   下夕lj关于死锁的叙述巾,正确的是    38.;有尸通过  个TCP连接向服务器发送数据的部    38

    分过程如题

    I . i可以边过剥夺进程资源解除死锁    一

    图所示  客户在lo时刻第  次收到确认rr列号ack_eq= I s    00的段,

    I.夕E锁的预防方法能确保系统不友生死锁    =

    并发送序列号seq  100的段,但发哇丢失. .:TCP支持快速重传,

   皿.    以判断系统是否处于死锁状态    =

    银行家算法可    则客户屯新发送seq 100段的时刻是

    l束态    R

    N.    统出现死锁时,必然有两个旦两个以i二的迸程处于fl[    A.t    2

    、可系    C. t    D.t

    B.仅I 、H、W    1     一    3     4

    A.仅H、皿    3

    9.店主机可1主动发起  个与主机乙的TCP连接,可1、乙选择的初始[

   C.    皿    D.仅I 、皿、W

    仅I、H 、    2018   2046

    列号分别为    和    ,则第二次握子TCP段的确认序列号是31.    节编址,采用二级分贞在储管理,地址结构如下

   某计算机主仔按宁    A.2018    B.201    C.2046    D.2047

    9

   所示    40.

    下列关于网络应用模型的叙述中,错误的是

    I求\)·( IO f,'     到号(IOfi)    J£内偏移(12位)    A.在P2P

    Ji    模型中,结点之间具有对等关系

    客户    服务器    要求:

    ( I )给出算法的基本设计思想

    �    ( 2)根据设计思想,采用C或C++语言描述算法,关键之处给出

    注释.

    to     (3)说明你所设计的算法的时间复杂度.

    42.( 10分)ij设计一个队列,要求满足:①初始时队列为空;②入队时,

    允许增加队列占用空间;③出队后,出队元素所占用的宅|时可重复

    使用,即整个队列所占用的空间只增不减;④入队操作和出队操作

    的时间复杂度始终保持为0(I).请回答下列问题:

    (1  )该队列应该选拇链式存储结构,近是顺序在储结构?

    !    (2)画出队列的初始状态,并给出判断队空和l队满的条件

    3

    !    (3)刚出第一个元亲人队后的队列状态.

    4

    时间    (4)给出入队操作和l出队操作的基本过程.

    3. ( 分)有n(n注3)位哲学家用坐在一张圆桌边,创位哲学家交替地

    题38囱    4

    就餐和l思考.在国桌中心有m(m;I)个碗,每两位哲学家之间有

    .    l根筷子.每位哲学家必须取到一个碗和两侧的筷F之后,才能就

   B 在客户/服务器(C/S)模型巾,客户与客户之间可以直接通信    餐,进餐完毕,将碗和l筷子放回原位,并继续思考.为使尽可能多C.在C/S模型中,主动发起通信的是客户,被动通信的是服务器

    的哲学家同时就餐,且防止出现死锁现象,请使用信号茸的PV操、

   0.在向多用户分发一个文件时,P2P模型通常比C/S模型所市时    at

    作(wi()、signal()操作)描述上述过程中的互斥与向步,并说明

   间短    所用信号盐及初值的含义.

    44. ( 7分)某计算机系统中的磁盘有300个柱而,每个柱面有10个磁

    二、  综合应用题:“~47小题, 其70分.

    道,每个磁道有200个扇区,扇区大小为512B.文件系统的每个41  13分)设线性表L= ( 1 ,a  a2,句,…,a.-2,an-I ,a)采用带头结点的

  . (     n    簇包含2个扇区.h'回答下列问题:

   单链表保存,链表中结点定义如下:

    ( 1 )磁盘的容业是多少?

   typedef struct node    ( 2)假设磁头在85号桂面上,此时有4个磁盘访问请求,簇号分别

    I   int data;    1   6  6    、10166   l10  60  若采用最短寻道时间

    为:0020、 0005    0和    5  .

    struct node * next;    优先(SSTF)调度算法,则系统访问簇的先后次序是什么?

    I NODE;

    (3)第100530簇在磁盘上的物理地址是什么?将簇号转换成磁

   谙设计一个空间复杂度为O(I)且时间上尽可能高效的算法,E新    盘物理地址的过程是由I/0系统的什么程序完成的?

    ’    -

    a    a    x (  -  X (   2) x

   排列L中的各结点,得到线性表L=( a1,气,鸟, .-1,鸟, .-2,…).    45. ( 16分)己知f( ) = n    n !  = n   n  I    n     …x2xl,计算J(川的C语言函数门的源程序(阴影部分)及其在32位计算机Mt的部    16行cal    ,

    知第    l指令采用相对寻址方式 该指令中的偏移量应分机器级代码如下:    是多少(给出计算过程)?已知第16行call指令的后4字节

    int  fl ( intn ) I    M

    为偏格茧,  采用大端还是小端方式?

    1  00401000    55     push  ebp    ( )f 13) = 6 227 020 800,但fl(l3    回值  l932 053 504,为

    (     )的返    为

    f1(13

    什么两者不相等?要使    )能返回正确的结果,应如何修

    if(n>l)

    改门源程序?

    11 00401018    83 7D 08 0 I     mp  dword plr [ ebp+8] , I    5    19   1『

    c    ( )第  行

    12 0040101C    7E 17     je  f1+35h (00401035)     [ ex

    l    R c ],吁乘法器输出的l、低32位乘积之问满足什么条件

    retur n * fl  -1 ;

    HJ,溢出标志OF = 1?要使CPUJ友生溢出时转}常处理,

    13 0040101E    8B 45 08     mov  e缸,dwordptr [ ebp+8]    imu

    编译器应在   l揣令后)u一条什么指令?

    14 00401021    83 E8 01     sub  eax, l    46.  7分    45    32位,采用分页存储

    (   )对于题  ,若计算机M的主存地址为

    15 00401024    50     push  eax    4  B    us    30

    管理方式,页大小为  K ,则第l行p h指令和第   行rel指令

    1  0040 I 025    E8 D  FF FF FF     fl  0040 I 000)    一

    6     6     call     (     是否在同  页中(说明理由)?行指令Cache有64行,采用4路组

    相联映射方式,主仔块大小为64B,则32位主存地址巾,哪几位表

    19 00401030    OF AF Cl     imul  eax, ecx    e    l

    示块内地址?哪儿位在示Cach 组号?哪几位表示标记(ag)信

    0 00401033    EB 05     jm   fl+3A   0040103)    一

    2    p     h (    a    息?读取第16行call指令时,只可能在指令Cache的哪  组中命

    else return I ;    中(说明理由)?

    21 00401035    B8 01 00 00 00 mov  eax,l     47    4

    .(9分)某网络拓扑如题47图所示,只中R为路由器,主机H1~H

    的IP地址配置以及R的各接f IP地址配置如图中所示. 现有若

    F台以太网交换机(无VLAN功能)和路由器两类网络互连设备可

    26 00401040     38 EC     cmp   ebp esp

    供选择.

    请回答下列问题:

    30 0040104A    C3     rel     l     l    2    3

    ( )设备 、设备  和设备  分别应选择什么类型网络设备?其中,机器级代码行包插行号、虚拟地址、 机器指令和汇编指令,计    ( )设备l、设备2和设备3巾,哪几个设备的接口需要配置IP地算机M披字节编址,int型数据占32位. 请回答下列问题:

    址?并为对应的接口配置正确的IP地址.

    10)荷要调用函数fl多少次?执行哪条指令会递lI调    3    l   4

(1)计算!(    ()为确保主机H~H 能够的问Internet,R需要提供什么服务?

    fl?    4    一

    用    ( )若主机H3发送  个口的地址为192.168.1.127   IP数据报,

    的

 2    一

( )上述代码中,哪条指令是条件转移指令?哪几条指令  定会    网络中哪几个主机会接收该数据报?

    使程序跳转执行?

(3)根据第16行call指令,第门行指令的虚拟地址应是多少?巳

    ‘…    101.1.2.10

    ·    2019年全国硕士研究生招生考试

    计算机科学与技术学科联考

    计算机学科专业基础综合试题参考答案


    !Fl    备

    单项选择题

    14    I. B    2. B    3.C    4. A    5.C

    年   �    10.D

    6. A    7.D    8.C    9. B

    :l92.168.1.66/26    IP地.f::192.168.1.67/26

IP地址·192.168.1.2/26    IP地址:192.168.1.3/26   IP地址

    11.B    12.C    13.A    14. D    15.D

    默认例关:192.168.1.65    默认闷关192.168.1.65

默认网关:192.168.1.1     默认网关:192.168.1.1

    16.D    17.B    18.C    19.B    20. C

    题47图    25. C

    21.A    22.D    23.B    24.C

    26.B    27.C    28.B    29.C    30. B

    31.A    32.C    33.C    34.A    35.B

    36.B    37.B    38.C    39.D    40.B

    -、  综合应用题

    41. [    ]

    答案要点

    ( 1)算法的基本设计思想:

    算法分3步完成  第l步,采用两个指针交替前行,找到守链

    表的中间结点;第步,将单链表的后半段结点原地逆置;第2    3

    步,从单链表前后两段中依次各取一个结点,按要求重排.

    (2)算法实现:

    void change_list( NODE * h )

    NODE * p,  * q,  * r,  * s;

    p = q = h;

    while ( q->next !  = NULL )     //寻找中间结点    (    front

    p = p  next;     II p走一步    I )采用链式仔储结构(两段式单向循环链在),队头指针为    ,

    ->

    队尾指针为rear

    q= q->nexl;

    (2)初始    一

    时,创建只有  个空闲结点的两段式单向循环链表,头指

    if ( q->next !  = NULL ) q = q->next; I I q走阔步

    front与    ar

    针    尾指针陀  均捐/oJ空闲结点  如1下图所示.

    q    、

    q二p->next;I  p所指结点为巾问结点, 为后 卡段链点

    的计结点    二

    p->next = NULL;    二里

    !    NULL )II将在盯卡段逆E、

    while( q   =     链    r

    I  r   q->next;    队空的判定条件:f ont=   rear.

    队满的判定条件:front= = rear->next.

    q->nrxt = p->next;    ( 3)插入第二个元素后的队列状态:

    >next= q;

    p-

    q = r;

    II s    一个数如结点,

    吕= h->next;     指向前中段的第

    (4)操作的基本过程:

    [JJ插入点

    = p->next;     II q指向后平段的第一个数据结点

    q    人队操作:

    p->next   NULL;    {, ( front = = rear->nexl)

    //将    、段的结点插入    //队满

    while ( q !  = NULL )     链在后    ;

    则在rear!i面恼人一个新的空闲结点

    到指定位置    人队Ji保存到rear所指纣点小;rear= rear->next;返问

    r    、

    r   q->next;     I   指向后 |段的下一个结点    Ill队操作:

    s->nexl;   //将q所指结点插入到s所指

    q->next =    ( front = = rear)

    导    //队空

    结点之后    则11队失败,返'"';

    s->next   q;    r

    J frnt所指纺点巾的元才e;front = font->neλ1;返回e.

    II s    一个插入点

    s   q->next;     指向前半段的下

    q = r;    43.

    [答案要点]

    //信号i

    semaphore bowl ;    //   于协调哲学家对碗的使用

   ( 3)算法的时间复杂度:    用

    semapl  echopsticks[ n ];

   参考答案的时间复杂度为0(n)    //用于协调听学家对筷子的使用

    for( inti= O; i <n; i++)

42. [答案要点]

    0040 102AH,故第17行指令的虚拟地址是0040I 2A   call

    chosicks [ i  ] . value   I ;   //设置两个哲学家之间筷子的数盐

    指令采用相对寻址方式,即门标地址=(PC)+偏移茧,call指

   bol value   min  n-1 ,  m ) ;    I  bol value主n-1,确保不死锁

    0040IOOOH    )=

    令的U标地址为    ,所以偏移吐=f标地址-(PC

    0040 IOOOH -0040 102AH   FFFF FFD6H   根据第16行callCoBeign

    指令的偏移茧字段为D6FF FF FF,可确定M采用小端方式.

    //    i 的程序

  while ( True )    哲学家

    I

    (4)囚为J(13) = 6 227 020 800,大于3位2  int 数据可表示的最

    思考;

    大值,因而fl(l3)的返回值是一个发生了溢出的结果.

    bol    //取碗

    P(     ;

    为使fl(l3)能返1°1正确结果,可将函数门的返回值类电改为

    ks[ i ] ) ;     //取左边筷子

    P chositc

    double(或loglog或logdouble .. floa)t 0

    chosicks[  i  + I ) MOD  ]) ;    //取右边筷子

    P(     (

    (5    0或

    就餐;    )若乘积的高33位为非全    七|全l,则OF= I

    i   l    “    ”

    s[ i    编译器应该在ru指令后加一条 溢出自陷指令 ,使得CPU

    V chositck   ] ) ;

    “    ”

    白动查询溢出标忐OF,巧OF= I时调出 溢出异常处理 程序 .

    V( chositcks[ i  + I )MOD n] ) ;

    46. [答案要点]

    V  bol ;

    0    一

    第l行指令和第3行指令的代码在同   页.

    因为页大小为4KB,所以虚拟地址的高20位为虚拟页号. 第l行

   Cond

    指令和第3行指令的虚拟地址高20位都是00401H,因此两条指0

44. [答案要点]

    5    令在同一页中.

    12/1024)KB= 3×10KB.

    1     容量=(300xl0x200x5

   ( )磁盘

    Cache组数为64/4= 16,因此,主存地址刷分巾,低6位为块内地

    0IOI660 l10 56、0 60005.

   (2    访问的簇是100 26、    、

    )依次

    址、中间位为组号4    (细索引)、自22位为标记.

   (3)第100530 簇在磁盘上的物理地址由其所在的柱面号、磁头

    读取第16行call指令时,只吁能在指令Cahe第0组中命中.

    号、扇区号构成

    因为页大小为4KB,所以虚拟地址和物理地址的最低12位完全相

    其所在的柱面号为L100530/(10x200/2)」=100

    .

    同,因而call指令虚拟地址00401025H中的025H= 0000 0010

    100530 %( 0x200/2)= 530,磁头号为L5301c 20012)」5o

    0 IO I B = 00 0000 100 O      12    ahe组

    剧区号为(530×2) %200 = 60    为物理地址的低  位,故对应C

    .

    号为0

    将簇号转换成磁监物理地址的过程由磁盘驱动程序完成.    .

    47. [答案要点]

45.[答案要点]

    ( I  设备l:路由器,设备2:以太网交换机,设备3:以太网交换机

    第16行call指令

    1     (10)芮要调用函数fl共10次  执行

   ( )计算J

    ( 2)设备l的接口市要配置IP地址;民备l的IFI、JF2和IF3  口

    用fl.    接

    会递lI调    9

    的IP地址分别是:1 2.168.1.254、192.168.1.1和192.168.1.650

   (2)第12行jle指令是条件转移指令. 第16行call指令、第2行0

    一    (3) R需要提供NAT服务

    mp    0

    j  指令、第3行削指令  定会使程序跳转执行.

    (4)主机H4会接收该数据报.

   (3)第16行call指令的下一条指令的    00401025H + 5 =

    地址为

