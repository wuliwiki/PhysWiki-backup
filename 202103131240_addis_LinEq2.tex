% 线性方程组解的结构 2

\pentry{线性映射的结构\upref{MatLS2}}

下面我们从线性映射和向量空间的角度理解线性方程组 $\bvec A \bvec x = \bvec b$.


\begin{definition}{线性方程}
对给定的线性映射 $A:X\to Y$ 和 $b \in Y$, \textbf{线性方程}为
\begin{equation}\label{LinEq2_eq2}
Ax = b
\end{equation}
所有满足该式的 $x \in X$ 的集合 $X_s$ 叫做方程的\textbf{解集}.
\end{definition}

首先注意 $A$ 未必把 $Y$ 中的每个元素都射中, 即值空间\upref{LinMap} $Y_1 = A(X) \subseteq Y$ 只是 $Y$ 的一个子空间. 所以只有 $b \in Y_1$ 时\autoref{LinEq2_eq2} 有解, 否则无解(解集为空集). 用映射的语言, 解集 $X_s$ 就是集合 $\qty{b}$ 的逆像\upref{map} $A^{-1}(\qty{b})$.

当\autoref{LinEq2_eq2} 中 $b = 0$ 时方程叫做算符 $A$ 的\textbf{齐次方程}. 根据定义, 齐次方程的解就是映射的零空间(\autoref{LinMap_the1}~\upref{LinMap}).

\begin{theorem}{}
线性方程\autoref{LinEq2_eq2} 的解集可以表示为
\begin{equation}\label{LinEq2_eq1}
X_s = X_0 + x_1
\end{equation}
其中 $x_1$ 为 $X_s$ 中的任意元素,  $X_0$ 为映射的零空间.
\end{theorem}
说明: $X_0 + x_1$ 表示把 $X_0$ 中的每一个向量与 $x_1$ 相加得到的集合. 易证当 $x_1 \ne 0$ 时解集 $X_s$ 不是一个向量空间(不存在零向量).

首先证明 $X_0 + x_1$ 中的元素满足 $Ax = b$. 令任意 $x_0 \in X_0$
\begin{equation}
A(x_0 + x_1) = Ax_0 + Ax_1 = 0 + b = b
\end{equation}
证毕. 再来证明解集中不存在 $X_0 + x_1$ 之外的矢量. 令 $x_2 \in X$ 且 $x_2 \ne x_1$. $Ax_2 = b$, 那么 $Ax_2 - Ax_1 = 0$

要证明\autoref{MatLS2_the1}~\upref{MatLS2}
