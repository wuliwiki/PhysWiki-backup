% 二项式定理(非整数幂)
% keys 二项式定理|泰勒展开|数值验证
% license Xiao
% type Tutor

\addTODO{应该结合泰勒展开来讲}

\pentry{二项式定理\upref{BiNor}}
当 $a$,  $b$ 为实数, $u$ 为非零实数时,有
\begin{equation}\label{eq_BiNorR_2}
\begin{aligned}
(a+b)^u &= a^{(u-2)}b^2 = \sum_{i=0}^{+\infty} \frac{u(u-1)\dots (u-i+1)}{i!} a^i b^{u-i}\\
&= a^u + u a^{(u-1)}b + \frac{u(u-1)}{2!}a^{(u-2)}b^2 + \dots~.
\end{aligned}
\end{equation}
容易看出,当 $u$ 为整数时,$i>u$ 的所有项为 $0$,得到整数指数的二项式定理\upref{BiNor}。

\subsection{数值验证}

在学习微积分之前,这里只给出一个数值验证的方法(而不是证明)。 在微积分中, 这个定理可以用泰勒展开\upref{Taylor}推导出来。

首先化简上式,不妨令 $|a|<|b|$, 把 $b^u$ 提出括号,再令 $x \equiv a/b$, 有$|x|<1$。 
\begin{equation}
(a+b)^u = b^u (1+x)^u = b^u \sum_{i=0}^{+\infty} \frac{u(u-1)\dots (u-i+1)}{i!} x^i~,
\end{equation}
所以只要用数值验证
\begin{equation}\label{eq_BiNorR_1}
(1+x)^u = \sum_{i=0}^{+\infty} \frac{u(u-1)\dots (u-i+1)}{i!} x^i~
\end{equation}
即可。接下来,可以用计算器或程序对\autoref{eq_BiNorR_1} 的前 $N$ 项进行求和。如果增加 $N$ 使结果趋近精确值,则验证成功(见下表)。 这里给出计算下表的 Matlab 程序\upref{BiNorM}。

\begin{table}[ht]
\centering
\caption{数值验证二项式定理(非整数幂)}\label{tab_BiNorR_1}
\begin{tabular}{|c|c|c|c|c|}
\hline
$(1+x)^u$ & $N = 5$ & $N = 20$ & $N = 100$ & 精确值前 8 位 \\
\hline
$x = 0.5, u = -0.3$ & 0.88445640 & 0.88546751 &  0.88546749 & 0.88546749 \\
\hline
$x = 0.6, u = 3.1$ & 4.2930453 & 4.2931093  & 4.2931093 & 4.2931093\\
\hline
\end{tabular}
\end{table}




