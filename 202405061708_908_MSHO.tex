% 多自由度简谐振子(经典力学)
% keys 简谐振子|mode|模
% license Xiao
% type Tutor

\pentry{简谐振子(经典力学)\nref{nod_SHO}}{nod_9c89}

\subsection{二自由度简谐振子}

考虑轻质弹簧与两个光滑滑块构成的体系,如\autoref{fig_MSHO_1} 所示。

\begin{figure}[ht]
\centering
\includegraphics[width=8.5cm]{./figures/9a36450a6311df17.pdf}
\caption{两个滑块和两根弹簧构成的简谐振子。} \label{fig_MSHO_1}
\end{figure}

和\enref{简谐振子(经典力学)}{SHO}的情况一样,弹簧的原长不重要,因此图中没给出。我们用$x_1$和$x_2$分别表示滑块$m_1$和$m_2$的位置,其中$x_1$表示弹簧$k_1$的长度减去其原长,$x_2$表示弹簧$k_2$的长度减去其原长。显然,当$x_1=x_2=0$时两根弹簧都处于原长。

则这个体系的运动方程为
\begin{equation}
\leftgroup{
    m_1\ddot{x}_1 &= -k_1x_1+k_2x_2\\
    m_2\ddot{x}_1 &= -k_2x_2~.
}
\end{equation}

这个体系也是也是一个简谐振子。由于它需要两个参数$x_1, x_2$来刻画,因此也称作\textbf{二自由度}的简谐振子。

当然,我们也可以在\autoref{fig_MSHO_1} 的右边加一堵墙,再加一根弹簧将$m_2$和右边的墙连接起来,得到的同样是二自由度简谐振子。


\begin{figure}[ht]
\centering
\includegraphics[width=8cm]{./figures/d8f923972bfb4725.pdf}
\caption{另一种二自由度简谐振子的模型。图中方块是固定的墙面,只有粉色的圆球有质量,其它构件都是轻质的。红色和蓝色分别表示只能沿着水平或垂直方向移动的滑杆,粉色圆球被固定在这两个滑杆中。两个滑杆分别连接到图示的两根弹簧上。} \label{fig_MSHO_2}
\end{figure}

\autoref{fig_MSHO_2} 所示的模型也是一个二自由度简谐振子,两个参数分别是圆球的横位移和纵位移。

一般地,二自由度简谐振子的运动方程写为
\begin{equation}\label{eq_MSHO_1}
\leftgroup{
    \ddot{x}_1 &= k_{11}x_1+k_{12}x_2\\
    \ddot{x}_2 &= k_{21}x_1+k_{22}x_2~.
}
\end{equation}


\subsubsection{二自由度简谐振子的方程解法}

类比\enref{简谐振子(经典力学)}{SHO}中的解法,我们可以先猜想\autoref{eq_MSHO_1} 的一些特解。

最简单的情况是,两个参数都做\textbf{相位}相同的单自由度简谐振动,只是\textbf{振幅}不一样,即$Ax_1(t) = x_2(t)$,其中$A$是一个常数。于是问题就化为一个单自由度简谐运动。

此时${\ddot{x}_1}/{\ddot{x}_2}=A$,代入\autoref{eq_MSHO_1} 得
\begin{equation}
\frac{k_{11}+k_{12}A}{k_{21}+k_{22}A}=A~,
\end{equation}
即
\begin{equation}
k_{22}A^2+(k_{21}-k_{12})A-k_{11} = 0~.
\end{equation}
因此$A$是确定的:
\begin{equation}
A = \frac{(k_{12}-k_{21})\pm\sqrt{(k_{12}-k_{21})^2 + 4k_{22}k_{11}}}{2k_{22}}~.
\end{equation}

这样能得到两个倍率$A$,记为$A_1$和$A_2$。在继续解方程之前,我举一个例子来方便你体会两个倍率的意义:
\begin{figure}[ht]
\centering
\includegraphics[width=10cm]{./figures/11c0cd42977bad92.pdf}
\caption{一个简单的二自由度简谐振子模型。} \label{fig_MSHO_3}
\end{figure}

\autoref{fig_MSHO_3} 所示的简谐振子体系高度对称,因此很容易发现两种特殊情况:一是两个滑块之间的距离恒为中间弹簧的原长,作\textbf{完全相同}的单自由度简谐运动,此时中间弹簧等同于不存在,这个情况对应$A=1$;二是$A=-1$,两个滑块作\textbf{完全对称}的运动。第二种情况比第一种情况的频率高一些,因为每个滑块都额外受中间弹簧的力了。

得到$A_i$以后,\autoref{eq_MSHO_1} 化为
\begin{equation}
\ddot{x}_1 = (k_{11}+A_ik_{12})x_1~,
\end{equation}
从而得\textbf{特解}
\begin{equation}
\leftgroup{
    x_1(t) &= C_i\cos(\sqrt{-k_{11}-A_ik_{12}}t+\phi_i)\\
    x_2(t) &= A_iC_i\cos(\sqrt{-k_{11}-A_ik_{12}}t+\phi_i)~.
}
\end{equation}

将各$A_i$对应的特解组合起来,就能得到通解:
\begin{equation}
\leftgroup{
    x_1(t) &= C_1\cos(\sqrt{-k_{11}-A_1k_{12}}t+\phi_1)+C_2\cos(\sqrt{-k_{11}-A_2k_{12}}t+\phi_2)\\
    x_2(t) &= A_1C_1\cos(\sqrt{-k_{11}-A_1k_{12}}t+\phi_1)+A_2C_2\cos(\sqrt{-k_{11}-A_2k_{12}}t+\phi_2)~.
}
\end{equation}
其中$C_1, C_2, \phi_1, \phi_2$都是待定常数。通过给定两组参数的初值,如初始位置和初始速度,即可得到满足初值的特解。

\subsubsection{模式}

上面的解答中,我们通过考虑两个参数成正比的特殊情况得到两个特解,用这两个特解组合出了一般的通解。这两个特解就叫做\textbf{模式(mode)}。如果能从题目中直接找到模式,那么就可以直接给出通解。




\subsection{多自由度简谐运动}

我们可以把有$n$个自由度的简谐运动方程写为
\begin{equation}\label{eq_MSHO_2}
\ddot{\bvec{x}} = \bvec{K}\bvec{x}~,
\end{equation}
其中$\bvec{x}=\pmat{x_1, x_2, \cdots, x_n}\Tr$为列矩阵,$\bvec{K}$为一个$n$阶方阵。

类似找模式的思路,我们可以设\autoref{eq_MSHO_2} 有特解
\begin{equation}\label{eq_MSHO_3}
\bvec{x}(t) = \bvec{v}\cos(\omega t + \phi)~,
\end{equation}
即$\bvec{x}$的各个分量之间的比值不变,其中$\bvec{v}$的各分量都是常数。将\autoref{eq_MSHO_3} 代入\autoref{eq_MSHO_2} 有
\begin{equation}\label{eq_MSHO_5}
\bvec{v} = -\omega^2\bvec{K}\bvec{v}~.
\end{equation}
也就是说,$\bvec{v}$是$\bvec{K}$的一个\enref{本征向量}{EigVM}。

每个本征向量能解出\autoref{eq_MSHO_2} 的一个特解,即\textbf{模式}(\autoref{eq_MSHO_2}) ,而通解则是这些模式的线性组合。具体解法请参考\autoref{ex_MSHO_1} :






\begin{example}{三自由度简谐振子}\label{ex_MSHO_1}

考虑简谐振子方程
\begin{equation}
\leftgroup{
    \ddot{x}_1 &= -2x_1+x_2-x_3\\
    \ddot{x}_2 &= -x_2-x_3\\
    \ddot{x}_3 &= -3x_3~.
}
\end{equation}
记
\begin{equation}
\bvec{x}=\pmat{x_1\\x_2\\x_3}~, \qquad \bvec{K}=\pmat{-2&1&-1\\
0&-1&-1\\
0&0&-3}~.
\end{equation}
则方程化为\autoref{eq_MSHO_2} 的形式。

求出$\bvec{K}$的三个本征向量:
\begin{equation}\label{eq_MSHO_4}
\leftgroup{
    \bvec{v}_1 &= \pmat{1\\0\\0}\qquad(\text{本征值为}-2)\\
    \bvec{v}_2 &= \pmat{1\\1\\0}\qquad(\text{本征值为}-1)\\
    \bvec{v}_3 &= \pmat{1\\1\\2}\qquad(\text{本征值为}-3)~.
}
\end{equation}
把\autoref{eq_MSHO_4} 中的三个$\bvec{v}_i$分别代入\autoref{eq_MSHO_5} 的$\bvec{v}$,得到三个模式

\begin{equation}
\leftgroup{
    \bvec{x}_1(t) &= \bvec{v}_1\cos(\sqrt{2} t + \phi_1)\\
    \bvec{x}_2(t) &= \bvec{v}_2\cos( t + \phi_2)\\
    \bvec{x}_3(t) &= \bvec{v}_3\cos(\sqrt{3} t + \phi_3)~.\\
}
\end{equation}
其中各$\phi_i$为待定常数。

通解就是
\begin{equation}
\bvec{x}(t) = C_1\bvec{x}_1(t)+C_2\bvec{x}_2(t)+C_3\bvec{x}_3(t)~,
\end{equation}
其中各$C_i$为待定常数。

\end{example}










