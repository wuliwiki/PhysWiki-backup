% 雅可比矩阵、雅可比行列式
% keys 多元微积分|坐标系变化|全微分|混合积|矩阵|雅可比矩阵|行列式|雅可比行列式
% license Xiao
% type Tutor

\begin{issues}
\issueDraft
\issueOther{补充:与方向导数的关系}
\end{issues}


\pentry{行列式与体积\nref{nod_DetVol}, 全微分\nref{nod_TDiff}}{nod_5607}

若有坐标系变换
\begin{equation}
\begin{cases}
x = x(u,v,w)\\ y = y(u,v,w)\\ z = z(u,v,w)
\end{cases}~.
\end{equation}
根据全微分关系
%(链接未完成)
\begin{equation}\label{eq_JcbDet_1}
\pmat{\dd{x}\\ \dd{y}\\ \dd{z}} =
\pmat{
\pdv*{x}{u} &  \pdv*{x}{v} & \pdv*{x}{w} \\ 
\pdv*{y}{u} & \pdv*{y}{v} & \pdv*{y}{w} \\ 
\pdv*{z}{u} & \pdv*{z}{v} & \pdv*{z}{w} }
\pmat{\dd{u}\\ \dd{v}\\ \dd{w}}~.
\end{equation}
其中 $\mat J$ 叫做雅可比矩阵,又可写为$\frac{\partial(x,y,z)}{\partial(u,v,w)}$。称其行列式 $\abs{\mat J}$为\textbf{雅可比行列式(Jacobian determinant)},或表示为$|\frac{\partial(x,y,z)}{\partial(u,v,w)}|$。在微分几何中,雅可比矩阵是切映射的表示。

\subsubsection{雅可比行列式与体积元}

由外代数的知识可知,设$V$为$n$维线性空间且$\bvec v_1,\bvec v_2,...\bvec v_n\in V$,若$\bvec v_1\wedge \bvec v_2\wedge...\wedge\bvec v_n\neq 0$(即该向量组线性无关),则其模为$\{\bvec v_i\}^n_{i=1}$张成的$n$维立方体的体积。若$\{\bvec v_i\}^n_{i=1}$是正交向量组,则其模为各向量模长的乘积。

雅可比行列式是体积元进行线性变换后的比例系数,以二维欧几里得线性空间为例。令$f:(x,y)\rightarrow (r, \phi)$,
\begin{equation}\label{eq_JcbDet_2}
\begin{aligned}
x=&r \opn {cos}\phi\\
y=&r\opn{sin}\phi~.
\end{aligned}
\end{equation}
则我们有:
\begin{equation}
\begin{aligned}
dx\wedge dy&= (dr\opn{cos}\phi-r\opn{sin}\phi d\phi )\wedge(dr\opn{sin}\phi+r\opn{cos}\phi d\phi )\\
&=(r\opn{cos}^2\phi +r\opn{sin}^2\phi)dr\wedge d\phi=rdr\wedge d\phi=|\bvec J|dr\wedge d\phi~.
\end{aligned}
\end{equation}


可以验证,若$\{x_i\}^n_{i=1}$和$\{y_i\}^n_{i=1}$都是$n$维线性空间中线性无关的坐标变量,则我们总可以进行类似\autoref{eq_JcbDet_2} 的变量代换。微积分中的多重积分实际上是对变量的外积进行积分,因此常利用该结论进行坐标系的变换。

\begin{theorem}{}

\end{theorem}
