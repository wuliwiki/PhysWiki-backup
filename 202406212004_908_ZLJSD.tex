% 重力加速度
% license CCBYSA3
% type Wiki

(本文根据 CC-BY-SA 协议转载自原搜狗科学百科对英文维基百科的翻译)


重力加速度(又名自由落体加速度)是一个物体受重力作用的情况下所具有的加速度。

通常指地面附近物体受地球引力作用在真空中下落的加速度,记为g。为了便于计算,其近似标准值通常取为980cm/s²或9.8m/s²。在月球、其他行星或星体表面附近物体的下落加速度,则分别称月球重力加速度、某行星或星体重力加速度。

\subsection{对于点质量}

牛顿万有引力定律指出,任何两个质点之间都有一个引力,该引力对每个质点的大小相等,并且两个质点处于同一直线,使它们相互靠近。万有引力公式为:

$F = G \frac{m_1 m_2}{r^2}$

其中 $m_1$ 和 $m_2$ 是两个质点的质量,$G$ 是万有引力常数,并且 $r$ 是两个质点之间的距离。这个公式是由行星运动推导出来的,在行星运动中,行星和太阳之间的距离使得把天体看作质点是合理的。(对于在轨卫星来说,“距离”指的是距质心的距离,而不是行星表面的高度。)

如果其中一个质点的质量比另一个大得多,就可以方便地在较大质量的质点周围定义一个引力场,如下所示:

$\mathbf{g} = -\frac{GM}{r^2} \hat{\mathbf{r}}$

其中 $M$ 是较大物体的质量,且 $\hat{\mathbf{r}}$ 是从质量较大质点指向质量较小质点的单位向量。负号表示力是一种吸引力。

这样,作用在较小质量质点上的力可以计算为:

$\mathbf{F} = m\mathbf{g}$

其中 $\mathbf{F}$ 是力矢量,$m$ 是较小质量,并且 $\mathbf{g}$ 是指向较大质量物体的向量。请注意    这个模型代表了与大质量物体相关的“远场”重力加速度。当物体的尺寸与测量距离相比并不微不足道时,叠加原理可以用于微分质量,假设整个物体的密度分布,从而获得一个更详细的“近场”重力加速度模型。对于在轨卫星,远场模型足以粗略计算高度与周期的关系,但不适用于多轨道后未来位置的精确估计。

更详细的模型包括(除其他外)地球赤道膨胀和月球不规则质量浓度(由于流星撞击)。重力恢复和气候实验(GRACE)任务于2002年发射,由两个昵称为“汤姆”和“杰瑞”的探测器组成,在环绕地球的极地轨道上测量两个探测器之间的距离差异,以便更精确地确定环绕地球的重力场,并跟踪随时间发生的变化。同样,2011-2012年的重力恢复和内部实验室(GRAIL)任务包括围绕月球的极地轨道上的两个探测器(“落潮”和“流动”),以更精确地确定重力场,用于未来的导航目的,并推断有关月球物理组成的信息。

具有加速度单位,是相对于较大物体位置的矢量函数,与较小质量的大小(甚至存在与否)无关。

\subsection{地球重力模型}

用于地球的重力模型的类型取决于给定问题所需的保真度。对于许多问题,例如飞机模拟,将重力视为常数就足够了,定义为:[2]

$\mathbf{g} =9.80665(32.1740\text{英尺}) m/s^2$

基于1984年世界大地测量系统(WGS-84)的数据,其中$\mathbf{g}$被理解为在当前参照系中指向“下”。

如果希望将地球上物体的重量作为纬度的函数进行建模,可以使用以下方法([2] 第41页):

$g = g_{45} - \frac{1}{2} \left( g_{\text{poles}} - g_{\text{equator}} \right) \cos \left( 2 \varphi \cdot \frac{\pi}{180} \right)$

其中

\begin{itemize}
\item $g_\text{poles}=9.832 (32.26\text{英尺})m/s^2$
\end{itemize}



\subsection{广义相对论}

在爱因斯坦的广义相对论中,引力是弯曲时空的一个属性,而不是由于物体间传播的力。在爱因斯坦的理论中,质量扭曲了附近的时空,其他粒子沿着时空几何形状决定的轨迹运动。重力是一种虚构的力。 没有重力加速度,因为物体在自由落体中的适当加速度和四加速度为零。自由落体中的物体不会经历加速度,而是沿着弯曲时空上的直线(测地线)运动。

\subsection{参考文献}

[1]
^Fredrick J. Bueche (1975). Introduction to Physics for Scientists and Engineers, 2nd Ed. USA: Von Hoffmann Press. ISBN 978-0-07-008836-8..

[2]
^Brian L. Stevens; Frank L. Lewis (2003). Aircraft Control And Simulation, 2nd Ed. Hoboken, New Jersey: John Wiley & Sons, Inc. ISBN 978-0-471-37145-8..

[3]
^Richard B. Noll; Michael B. McElroy (1974), Models of Mars' Atmosphere [1974], Greenbelt, Maryland: NASA Goddard Space Flight Center, SP-8010..