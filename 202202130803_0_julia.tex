% Julia 入门笔记
% keys Julia 语言|科学计算|数值计算|REPL

\begin{issues}
\issueDraft
\end{issues}

Julia 从 Matlab 和 C/C++ 借鉴了不少语法, 如果你两种语言都已经会了, 那么 Julia 是十分简单的. 本章的目的就是为已经学习过类似语言的读者提供一个 Julia 语言的快速入门笔记.

安装方法
\begin{lstlisting}[language=bash]
sudo snap install Julia --classic
\end{lstlisting}

另外 Jupyter Notebook 也支持 Julia.

\subsection{命令行}
\begin{itemize}
\item 最简单的 Hello World 程序如 \verb|println("hello world")|

\item 为了区分, 系统的命令行叫做 terminal, 而 Julia 的命令行叫做 \textbf{REPL (read-eval-print-loop)}
\item 注释用 \verb|#|, 或者 \verb|#=一些注释=#|
\item \verb|Ctrl + D| 退出或者 \verb|exit()| 退出
\item \verb|ans| 变量和 Matlab 一样, 用于显示上一次运算的输出结果(结果赋值给变量的除外)
\item 在一行中运行多个命令, 用分号隔开
\item 要清空 workspace 中的变量只能重启
\item \verb|println("$var1 some words $var2")| 中 \verb|var1, var2| 可以是数字、字符串等
\end{itemize}

\subsection{计算器}
\begin{itemize}
\item \verb|b = 2a| 和 \verb|c = 2(b+3)| 都可以省略乘号. 注意 \verb|1/2a| 相当于 \verb|1/(2a)|
\item \verb|2/3| 返回双精度类型 \verb|Float64| 而不是整数
\item \verb|2//3| 返回 \verb|Rational{Int64}| 类型, 相当于用两个 \verb|Int64| 整数来表示分数, 分数之间的运算没有误差
\item 表示复数如 \verb|1 + 2im|, 相当于 \verb|ComplexF64(0,2)|. 又例如 \verb|1//2 + (3//4)im| 的类型是 \verb|Complex{Rational{Int64}}|
\end{itemize}

\subsection{变量}
\begin{itemize}
\item 用 \verb|typeof()| 查看某个变量的类型
\item 类型完全是动态的, 但可以在表达式或者函数定义后面加上 \verb|::类型| 限制变量类型
\item 字符串用双引号, 单个字符用单引号
\item 查看类型的最大和最小值如 \verb|typemax(Int64)|, \verb|typemin(Int64)|
\item 标准库提供任意精度类型 \verb|BigInt| 和 \verb|BigFloat| (底层是 GMP)
\item 变量名可以用 UTF-8 字符, 在一些编辑器中可以用反斜杠 latex 命令打出对应的字符
\item 注意和 C++ 不同, Julia 中的变量名只是 object 的标签
\item 要释放一个变量所占的内存, 把它赋值为 \verb|nothing|, 然后运行垃圾回收 \verb|GC.gc()| 即可.
\end{itemize}

\subsection{矩阵}
\begin{itemize}
\item 索引从 1 开始, 和 Matlab 一样
\item 矩阵(数组)使用列主元排序, 和 Matlab 一样
\item 一维数组 \verb|Array{T, 1}| 不区分行和列, 但是二维数组 \verb|Array{T, 2}| 可以区分行向量和列向量
\item 矩阵切割如 \verb|Psi[:, j, :]|
\item 矩阵尺寸 \verb|size(Psi, 维度)|
\item 零向量矩阵 \verb|zeros(整数)| 或 \verb|zeros(整数, 整数)| 分别返回 \verb|Array{Float64,1}| 和 \verb|Array{Float64,2}|
\item 64 位机器上, 整数 literal 的类型默认是 \verb|Int64|
\item \verb|a = [1 2; 3 4]| 得到 \verb|Array{Float64,2}|
\item \verb|a[:, 1]| 仍然得到 \verb|Array{Float64,2}|
\item \verb|a[:, 1] = [3; 4]| 可以修改 \verb|a| 的值
\item \verb|b = a[:, 1]| 却是复制数据.
\item \verb|a| 作为参数传给函数是 pass by reference, 相当于 C 语言传递指针
\item \verb|a[:, 1]| 作为参数传给函数是 pass by value, 如果需要 by reference, 使用 \verb|view| 函数获取 \verb|SubArray| 类
\item \verb|SubArray| 不存数据, 可以用来给函数参数 pass by reference
\item \verb|view(a, :, 1)| 得到的类型是 \verb|SubArray|, 具体类型是 \verb|SubArray{Int64,1,Array{Int64,2}, Tuple{Base.Slice{Base.OneTo{Int64}},Int64},true}|
\item \verb|SubArray <: AbstractArray{T,Ndim} <: Any|
\item \verb|Array{T,Ndim} <: DenseArray{T,Ndim} <: AbstractArray{T,Ndim} <: Any|
\item 所以只要函数参数接受 \verb|AbstractArray|, 就可以同时接受 \verb|Array, SubArray|.
\end{itemize}

\verb|SubArray| 的结构
\begin{lstlisting}[language=julia]
struct SubArray{T,N,P,I,L} <: AbstractArray{T,N}
    parent::P
    indices::I
    offset1::Int  # for linear indexing and pointer, only valid when L==true
    stride1::Int  # used only for linear indexing
    ...
end
\end{lstlisting}

\begin{itemize}
\item 随机矩阵如 \verb|rand(ComplexF64, Nr1, Nr2, Npw)|.
\item 求矩阵的哈希值如 \verb|hash(矩阵)|. 把运算结果的 hash 输出可以保证计算过程不被优化掉.
\end{itemize}

\subsection{脚本}
\begin{itemize}
\item 在系统命令行用 \verb|Julia <file>| 运行脚本, 用 \verb|Julia <file> <arg1> <arg2>| 给出 arguments
\item 在 REPL 中运行脚本如 \verb|include("/Users/addis/Desktop/main.jl")|(路径不区分大小写), 也支持反斜杠, 但要转义成 \verb|\\|
\item 在文件中, 如果要确定当前文件是不是主文件, 用 \verb|abspath(PROGRAM_FILE) == @__FILE__|
\end{itemize}

\subsection{画图}
第三方画图包 \verb|Plots|, 无需安装
\begin{lstlisting}[language=Julia]
using Plots
x = 1:10; y = rand(10); # These are the plotting data
plot(x,y)
\end{lstlisting}

\subsection{控制流}
\begin{lstlisting}[language=Julia]
b = 3;
for a = 1:5
    if a < b
        println(a, " < ", b)
    elseif a > b
        println(a, " > ", b)
    else
        println(a, " = ", b)
    end
end
\end{lstlisting}
其中 \verb|1:5| 的类型是 \verb|UnitRange{Int64}|. 若令 \verb|a = 1:5|, 可以使用 \verb|a[i]| 获取 “元素”. 类似地, \verb|1:2:5| 的类型是 \verb|StepRange{Int64, Int64}|.

for 循环也可以使用 \verb|for i in [1,4,0]|

\subsection{函数}
函数定义
\begin{lstlisting}[language=Julia]
function sphere_vol(r)
    return 4/3*pi*r^3
end
\end{lstlisting}

\subsection{常用函数}
\begin{itemize}
\item 当前路径 \verb|pwd()|
\item 更改路径 \verb|cd()|
\item 当前时间 \verb|time()|
\end{itemize}

\subsection{储存变量}
要把任意变量存到文件, 用 \verb|using Serialization|, 除此之外常用的还有 \verb|using JLD2|.
\begin{lstlisting}[language=julia]
using Serialization
a = randn(100,100)
f = serialize("file1.dat", a)
a′ = deserialize("file1.dat")
println(a == a′)
\end{lstlisting}

\subsection{文件读写}
\begin{itemize}
\item \verb|f = open("文件名", "r")| 打开文件读取, \verb|close(f)| 关闭文件.
\item \verb|eof(f)| 用于判断是否到达文件末尾
\item \verb|s = readline(f)| 读取下一行
\item \verb|s = read(f, String)| 整个文件读入字符串
\item , \verb|read()|, \verb|close()|
\end{itemize}
