% Stern-Gerlach实验
% 斯特恩-盖拉赫实验|狄拉克符号|自旋|坍缩|量子态

\pentry{量子力学的基本原理(量子力学)\upref{QMPrcp}}

Stern-Gerlach实验由O. Stern于1921年构想,由Stern和W. Gerlach于1922年在法兰克福完成\footnote{可参见Bretislav Fridrich和Dudley Herschbach发表的\textsl{Stern and Gerlach: How a Bad Cigar Helped Reorient Atomic Physics}, \textsl{Physics Today}, Dec. 2003.}.

引用樱井纯《现代量子力学》中的描述:“在某种意义上,Stern-Gerlach类型的双态系统是最少经典力学而最多量子力学的系统.对涉及双态系统问题的坚实理解将对任何认真学习量子力学的学生都是有益的.”\textbf{量子力学的基本原理(量子力学)}\upref{QMPrcp}词条中也建议配合本词条内容来理解抽象的概念.


\subsection{实验描述}

用一个炉子加热银原子,使之获得动能,从炉子上的一个小洞跑出来.出射的银原子会经过一个准直器,之后朝已经建立好的非均匀磁场飞去.这个磁场由如图所示的两磁极构成,其中一磁极有尖锐的边缘.之后,银原子会打到一块接收屏上,形成可观测的光斑.实验如\autoref{SGExp_fig1} 所示.


\begin{figure}[ht]
\centering
\includegraphics[width=14cm]{./figures/SGExp_1.pdf}
\caption{Stern-Gerlach实验示意图.} \label{SGExp_fig1}
\end{figure}

银原子由一个原子核、47个电子组成,其中46个电子构成了总角动量为零的球对称电子云,而剩下那个电子的自旋则为整个原子提供了非零的角动量.原子核自旋与本实验无关,故不讨论.

于是,原子整体上有一个磁矩$\bvec{\mu}$,正比于剩下那个电子的自旋$\bvec{S}$:
\begin{equation}
\bvec{\mu}\propto\bvec{S}
\end{equation}

按照\autoref{SGExp_fig1} 所示的装置,银原子通过非均匀磁场时会因为存在磁矩而受到竖直方向的力,这个力正比于原子磁矩的竖直分量$\mu_z$.因此,银原子最后击中接收屏的位置,反映了银原子偏转的大小,从而可以看作是对银原子的$\mu_z$进行的测量,也可以看作是对电子自旋的测量.




\begin{figure}[ht]
\centering
\includegraphics[width=10cm]{./figures/SGExp_2.pdf}
\caption{Stern-Gerlach实验的结果,左边是经典理论的预言,右边是实际观测的结果.} \label{SGExp_fig2}
\end{figure}


经典理论认为,银原子的磁矩取向是完全随机的,因此$\mu_z$可能是$\pm\abs{\bvec{\mu}}$之间的任何值.因此,如果我们放出大量的银原子轰击接收屏幕,并记录银原子所击中的位置,综合起来,应该得到\autoref{SGExp_fig2} 左边所示的结果.

但实验观测到的却是\autoref{SGExp_fig2} 右边所示的结果.


实验结果意味着,如果我们测量电子自旋,那只能得到一正一负两个数值,而不是连续的结果;换言之,自旋算符只有两个特征值,数值上是$\pm\hbar/2$.


如果我们把\autoref{SGExp_fig1} 中的磁场转动90度,测量银原子磁矩的水平分量$\mu_y$或者说电子自旋的水平分量,也会得到同样的结果.



注意,实验结果并不意味着“银原子的磁矩”或电子的自旋,在$z$方向上只有两种可能性;而是意味着,如果我们去测量,就只能得到两种结果.此时银原子的量子态已经坍缩成对应结果的本征态了,通常已不再是测量前的状态.



\subsection{序列Stern-Gerlach实验}


现在,把\autoref{SGExp_fig1} 所示的实验装置视为一个整体,一个“测量银原子某方向上磁矩”的装置.把测量$z$方向磁矩的装置记为$\opn{SG}\uvec{z}$,测量$y$方向磁矩的装置记为$\opn{SG}\uvec{y}$,测量$x$方向的记为$\opn{SG}\uvec{x}$.

首先我们要确定,一个$\opn{SG}\uvec{z}$装置如量子力学的基本原理所述,并不会改变该测量的本征态.先让银原子通过一台$\opn{SG}\uvec{z}$仪器,按照理论预言,此时它已经变成本征态了.我们把本征值为负的本征态剔除——实验上很容易做到,在装置出射口用挡板挡住其中一个偏转方向的银原子射流,只让另一个方向的射流顺利离开.此时,筛选出的射流应该处于正的本征值.让它通过下一台$\opn{SG}\uvec{z}$仪器,结果屏幕上确实只剩下了一个光斑,对应本征值为正的状态.

现在,用同样的手法,筛选出$\opn{SG}\uvec{z}$的本征值为正的射流,但接下来让它通过一台$\opn{SG}\uvec{x}$仪器.实验结果是两个光斑.

接下来做最后一个实验.筛选出$\opn{SG}\uvec{z}$的本征值为正的射流,让它通过一台$\opn{SG}\uvec{x}$仪器,筛选出其中一个方向的射流,再让它通过一台$\opn{SG}\uvec{z}$仪器.结果是我们又得到了两个光斑.
\addTODO{需要添加上类似樱井的《现代量子力学》中的示意图.}



实验结果表明,我们没法同时测量出银原子在$z$和$x$两个方向上的磁矩,对$x$方向的测量破坏了$z$方向的状态.按照量子力学的\textbf{坍缩假设}\footnote{即测量后,量子态会返回测量算符的一个本征值,并坍缩成该本征值的一个本征态.},这意味着没有任何一个量子态,同时是$x$、$z$方向上磁矩测量算符的本征态.



\begin{exercise}{与光的偏振态进行类比}
这个实验现象本质上和光的偏振实验\autoref{QMPrcp_ex3}~\upref{QMPrcp}是同样的原理.对照这两个实验,比较它们的异同,观察它们是如何满足量子力学基本原理的.
\end{exercise}





\subsection{对自旋$1/2$系统的讨论}

















