% 南京理工大学 2010 年 研究生入学考试试题 普通物理(B)
% license Usr
% type Note

\textbf{声明}:“该内容来源于网络公开资料,不保证真实性,如有侵权请联系管理员”

\subsection{一。填空题(32分,每空2分)}
\begin{enumerate}
    \item 已知一电子的运动方程可表示为 $r = b \cos \omega t + b \sin \omega t + ct\hat{e}_z$,式中 $a,b$ 为常数,以秒计,随在$t$时刻,电子的速度为 \underline{\hspace{1cm}} ,加速度为 \underline{\hspace{1cm}} 。
    \item 一质量为$m$的小球系在长为$L$的细绳的一端,绳的另一端固定于$O$点。先使小球以$v_0$速度做圆周水平匀速运动,然后细绳逐渐缩短,绳始终与运动方向夹角为$\theta$的小球的速度表达式为 \underline{\hspace{1cm}} ,细绳的张力为多大为 \underline{\hspace{1cm}} 。
    \item 设一平面简谐波沿$z$轴正方向传播,已知$t = 0$ 处质点的振动方程为$y = A \cos (\omega t + \frac{\pi}{3})$,已知质点的最大位移$A=0.05m$,频率$f=30Hz$,波速$v=4m/s$,求波函数表达式为 \underline{\hspace{1cm}} 。
    \item 如图所示,$x$ 方向上传播简谐波的振动周期为$T = 0.2s$,波长$\lambda = 20m$,当$x = 0$处质点的振动方程为 \underline{\hspace{1cm}} ,求波速为 \underline{\hspace{1cm}} 。
    \item $2 mol$ 氧气在$27°C$时的内能等于 \underline{\hspace{1cm}} ,其分子的平均动能是 \underline{\hspace{1cm}} ,平均平动动能是 \underline{\hspace{1cm}} 。
    \item 设一个气体分子的密度分布函数为$f(v)$,则单位体积中,$v_1$与$v_2$区间内的分子数为 \underline{\hspace{1cm}} 。
    \item 假设气体是平衡态的理想气体,外界压力、外界体积、分子数和温度稳定且和高度无关。假设自由运动的气体分子速度为$B$,则从高度$h$处逃逸的分子速率为 \underline{\hspace{1cm}} ,逃逸时间为 \underline{\hspace{1cm}} ,逃逸气体的动能为 \underline{\hspace{1cm}} 。
\end{enumerate}
