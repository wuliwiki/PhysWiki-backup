% 尤金·维格纳(综述)
% license CCBYSA3
% type Wiki

本文根据 CC-BY-SA 协议转载翻译自维基百科\href{https://en.wikipedia.org/wiki/Eugene_Wigner}{相关文章}。

\begin{figure}[ht]
\centering
\includegraphics[width=6cm]{./figures/d45abc1dfbc9adc2.png}
\caption{} \label{fig_YJwgn_1}
\end{figure}
尤金·保罗·维格纳(匈牙利语:Wigner Jenő Pál,发音:[ˈviɡnɛr ˈjɛnøː ˈpaːl];1902年11月17日-1995年1月1日)是匈牙利裔美国理论物理学家,同时也对数学物理作出了贡献。他因“在原子核和基本粒子理论上的贡献,特别是发现并应用基本对称性原理”获得1963年诺贝尔物理学奖。[1]

维格纳毕业于柏林工业高等学校(今柏林工业大学),曾在柏林的威廉皇帝研究所担任卡尔·魏森伯格和理查德·贝克尔的助理,并在哥廷根大学跟随大卫·希尔伯特工作。维格纳与赫尔曼·外尔共同将群论引入物理学,尤其是引入物理对称性理论。在此过程中,他在纯数学领域也做出了开创性工作,发表了多篇重要数学定理,特别是“维格纳定理”,该定理是量子力学数学表述的基石。他还因对原子核结构的研究而闻名。1930年,普林斯顿大学招募了维格纳和约翰·冯·诺伊曼,他随即移居美国,并于1937年获得美国公民身份。

维格纳曾与利奥·西拉德和阿尔伯特·爱因斯坦共同参加会议,促成了著名的“爱因斯坦-西拉德信”,促使时任总统富兰克林·罗斯福批准成立铀顾问委员会,研究核武器的可行性。维格纳担心德国核武器项目会率先研制出原子弹。在曼哈顿计划期间,他领导团队设计核反应堆,将铀转化为武器级钚。当时,核反应堆还仅存在于纸面上,还没有任何反应堆实现临界。维格纳对杜邦公司不仅承担建造任务,还获得了反应堆详细设计权感到失望。1946年初,他成为克林顿实验室(今橡树岭国家实验室)研发负责人,但由于对美国原子能委员会官僚干预感到沮丧,重返普林斯顿。

战后,维格纳曾在多个政府机构任职,包括1947年至1951年在国家标准局任职,1951年至1954年在国家研究委员会数学小组任职,曾参与国家科学基金会物理小组工作,并于1952年至1957年及1959年至1964年两次担任美国原子能委员会有影响力的总咨询委员会成员。晚年时期,他更加关注哲学领域,并发表了《数学在自然科学中非凡有效性的非理性》一文,这篇文章成为他在技术数学和物理学以外最著名的作品。
\subsection{早年生活与教育}
\begin{figure}[ht]
\centering
\includegraphics[width=6cm]{./figures/d952b73589cd1ed6.png}
\caption{沃纳·海森堡与尤金·维格纳(1928年)} \label{fig_YJwgn_3}
\end{figure}
维格纳·耶诺·帕尔于1902年11月17日出生在奥匈帝国布达佩斯,家庭为中产阶级犹太家庭,父母分别是伊丽莎白·埃尔莎·爱因霍恩和安塔尔·安东·维格纳,父亲是一名皮革鞣制商。他有一个姐姐贝尔塔(Berta,称为Biri),以及一个妹妹玛吉特(Margit,称为Manci),[2] 后来玛吉特嫁给了英国理论物理学家保罗·狄拉克。[3] 他在9岁之前由专业教师在家中授课,在三年级时进入学校学习。在此期间,维格纳对数学问题产生了兴趣。[4] 11岁时,维格纳被医生误诊为患有肺结核,父母将他送到奥地利山区的疗养院住了六周,后来医生确认此前的诊断有误。[5]

维格纳的家庭是犹太人,但并不虔诚,他的成人礼也是以世俗形式举行的。1915年至1919年期间,他在法索里路福音中学学习,该校也是他父亲曾经就读的学校。宗教教育是必修课程,他曾跟随一位拉比学习犹太教课程。[6] 他的同学中有小他一岁的约翰·冯·诺依曼。两人都受益于著名数学教师拉斯洛·拉茨的指导。[7] 1919年,为了逃避贝拉·库恩领导的匈牙利共产政权,维格纳一家短暂逃往奥地利,在库恩政权垮台后返回匈牙利。[8] 部分由于库恩政权中犹太人担任重要职位引发的社会反弹,维格纳一家改宗路德宗。[9] 维格纳晚年曾解释说,他们家当时改宗路德宗“本质上并非出于宗教原因,而是出于反共原因”。[9]

1920年中学毕业后,维格纳进入布达佩斯科技大学(Műegyetem,布达佩斯理工大学)学习,但对学校开设的课程并不满意,[10] 于是在1921年转入柏林工业高等学校(Technische Hochschule Berlin,今柏林工业大学)学习化学工程。[11] 他还参加了德国物理学会每周三下午举行的学术讨论会,这些讨论会聚集了包括马克斯·普朗克、马克斯·冯·劳厄、鲁道夫·拉登堡、维尔纳·海森堡、瓦尔特·能斯特、沃尔夫冈·泡利和阿尔伯特·爱因斯坦在内的著名研究人员。[12] 维格纳还结识了物理学家利奥·西拉德,两人迅速成为最亲密的朋友。[13] 柏林的第三段经历对他也具有重要影响,他曾在威廉皇帝物理化学与电化学研究所(今弗里茨·哈伯研究所)工作,在那里遇到了迈克尔·波兰尼,波兰尼成为继拉斯洛·拉茨之后对维格纳影响最大的老师。波兰尼指导维格纳完成了他的DSc论文《分子的形成与衰变》。[14]
\subsection{中年时期}
维格纳回到布达佩斯,在父亲的制革厂工作,但在1926年,他接受了卡尔·魏森贝格在柏林威廉皇帝研究所的邀请。魏森贝格需要有人协助他进行X射线晶体学的研究,而波兰尼推荐了维格纳。维格纳担任魏森贝格助手六个月后,又为理查德·贝克尔工作了两个学期。在此期间,维格纳开始探索量子力学,研究埃尔温·薛定谔的相关工作,同时深入学习了费迪南德·弗罗贝尼乌斯和爱德华·冯·韦伯的群论。[15]

随后,维格纳收到阿诺德·索末菲尔德的邀请,前往哥廷根大学担任伟大数学家戴维·希尔伯特的助理。但这段经历让他失望,因为年事已高的希尔伯特能力逐渐衰退,且兴趣已转向逻辑领域。然而,维格纳依然坚持独立研究。[16] 他为量子力学中对称性理论奠定了基础,并在1927年引入了如今被称为维格纳D矩阵的概念。[17] 维格纳和赫尔曼·外尔共同推动了群论在量子力学中的引入。外尔撰写了经典著作《群论与量子力学》(Group Theory and Quantum Mechanics,1928年),但这本书较难理解,尤其对年轻物理学家来说更是如此。维格纳的著作《群论及其在原子光谱量子力学中的应用》(Group Theory and Its Application to the Quantum Mechanics of Atomic Spectra,1931年)使群论更加易于理解,扩大了其受众群体。[18]
\begin{figure}[ht]
\centering
\includegraphics[width=6cm]{./figures/02f216555541cc58.png}
\caption{维格纳 6-j 符号的 Jucys 图示。图中节点上的加号表示其周围连线按逆时针方向读取。由于其对称性,该图可以有多种不同的绘制方式。通过取其镜像可以得到等价的配置,此时节点上的加号会变为减号。} \label{fig_YJwgn_2}
\end{figure}
在这些工作中,维格纳为量子力学中对称性理论奠定了基础。[19] 他于1931年证明的“维格纳定理”是量子力学数学表述的基石。该定理规定了旋转、平移和CPT对称性等物理对称性在态的希尔伯特空间上的表示方式。根据该定理,任何对称变换都可在希尔伯特空间中表示为线性且酉的变换,或反线性且反酉的变换。对称群在希尔伯特空间上的表示要么是通常表示,要么是投影表示。[20][21]

在1930年代后期,维格纳将研究扩展到原子核领域。到1929年时,他的论文已在物理学界引起关注。1930年,普林斯顿大学以他在欧洲所得工资的7倍薪水聘请维格纳担任为期一年的讲师,同时还聘请了冯·诺伊曼。维格纳和冯·诺伊曼在1928年共同发表过三篇论文,在1929年又合作发表了两篇论文。他们分别将自己的名字英文化为“Eugene”和“John”。[22] 在一年合同结束后,普林斯顿又提供了为期五年的兼职访问教授合同,而德国柏林工业大学则安排了另一半时间的教学任务。这一安排十分及时,因为纳粹不久便在德国上台。[23] 1934年在普林斯顿,维格纳将妹妹玛吉特(Margit “Manci” Wigner)介绍给物理学家保罗·狄拉克,两人随后结婚。[24]

1936年合同到期后,普林斯顿没有续聘维格纳。[25] 在格雷戈里·布莱特的帮助下,维格纳在威斯康星大学找到了新工作。在那里,他遇到了第一任妻子阿米莉亚·弗兰克,她当时是物理系的学生。然而,她于1937年意外去世,维格纳深受打击。因此,他在1938年接受了普林斯顿的邀请重返任教。[26] 1937年1月8日,维格纳成为美国公民,并将父母接到美国生活。[27]
\subsection{曼哈顿计划}
\begin{figure}[ht]
\centering
\includegraphics[width=6cm]{./figures/b22267d79a2a59d6.png}
\caption{} \label{fig_YJwgn_4}
\end{figure}
虽然他自称是政治外行,但在1939年8月2日,他参与了与利奥·西拉德和阿尔伯特·爱因斯坦的一次会面,促成了“爱因斯坦–西拉德信”的诞生,这封信促使富兰克林·D·罗斯福总统授权成立“铀顾问委员会”,以调查制造原子弹的可行性。[28] 威格纳担心德国核武器项目会率先制造出原子弹,甚至拒绝按指纹,因为这可能在德国获胜后被用来追踪他的位置。[29] 他后来回忆道:“被谋杀的念头,会让你异常专注。”[29]

1941年6月4日,威格纳与第二任妻子玛丽·安妮特·惠勒结婚,她是瓦萨学院的物理学教授,1932年在耶鲁大学获得博士学位。战后,她在新泽西州罗格斯大学道格拉斯学院任物理教师,直至1964年退休。他们一直婚姻稳定,直到她于1977年11月去世。[30][31] 他们育有一子一女:大卫·威格纳和玛莎·威格纳·厄普顿。[32]

在曼哈顿计划期间,威格纳领导了一个团队,成员包括J·欧内斯特·威尔金斯、阿尔文·M·温伯格、凯瑟琳·韦、盖尔·杨和爱德华·克鲁兹。该团队的任务是设计用于将铀转化为武器级钚的生产核反应堆。当时,反应堆还仅存在于理论图纸上,没有任何一个反应堆实现临界状态。1942年7月,威格纳选择了一个保守的100兆瓦设计,使用石墨作为中子减速剂,并以水作为冷却剂。[33] 1942年12月2日,威格纳在芝加哥大学废弃的斯塔格体育场看台下的一个改造过的球场内见证了世界上第一座原子反应堆——芝加哥一号堆(CP-1)实现受控核链式反应。[34]
\begin{figure}[ht]
\centering
\includegraphics[width=6cm]{./figures/5a56e734e490c5cf.png}
\caption{} \label{fig_YJwgn_5}
\end{figure}