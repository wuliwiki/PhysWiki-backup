% 振动的指数形式
% keys 振动|欧拉方程|复数|复振幅|幅角|相位|简谐

\pentry{二阶常系数齐次微分方程\upref{Ode2}}

简谐振子的微分方程
\begin{equation}\label{VbExp_eq1}
m\ddot x =  - kx
\end{equation}
是一个二阶常系数齐次微分方程. 其复数域的通解可以表示为
\begin{equation}
x(t) = C_1 \E^{\I \omega t} + C_2 \E^{-\I \omega t}
\end{equation}
其中 $C_1, C_2$ 是任意复常数. 由于指数函数的运算往往比三角函数方便, 物理或工程中常常用指数函数表示振动, 即把\autoref{VbExp_eq1} 的通解记为\footnote{\autoref{VbExp_eq3} 中 $\E^{-\I \omega t}$ 里的负号是一种习惯, 有些教材中也会使用正号. 无论使用哪一种, 必须在计算中保持一致.}
\begin{equation}\label{VbExp_eq3}
\tilde x(t) = \tilde A \E^{-\I \omega t}
\end{equation}
其中 $\tilde A$ 是一个复数\footnote{在变量上方加波浪线通常为了强调该变量是一个复数, 但为了书写方便有时候也会省略, 需要从语境中判断.}, 称为\textbf{复振幅}, $\tilde A$ 的模长 $A = |\tilde A|$ 就是振幅, $\tilde A$ 幅角的相反数 $\varphi_0 = -\arg(\tilde A)$ 就是初相位\footnote{如果\autoref{VbExp_eq3} 中没有负号, 则初相位定义为 $\varphi_0 = \arg(\tilde A)$}. 当我们用\autoref{VbExp_eq3} 表示振动时, 其实部表示质点的坐标, 虚部没有物理意义.

为了验证\autoref{VbExp_eq3} 的确包含了实数域的通解, 我们可以先把复振幅表示为 $\tilde A = A\E^{-\I \varphi_0}$%未完成:检查之前是否有提过
, 代入\autoref{VbExp_eq3}, 再取实部得
\begin{equation}
x(t) = \Re[\tilde x(t)] = A\Re \qty[\E^{-\I (\omega t + \varphi_0)}] = A \cos(\omega t + \varphi_0)
\end{equation}

\subsection{振动的叠加}
这里举一个例子说明使用指数函数比三角函数方便. 假设有若干个频率相同但振幅和初相位各不相同的振动 $x_i(t) = A_i \cos(\omega t + \varphi_{0i})$, 现在我们来计算它们叠加的结果, 即 $\sum_i x_i(t)$. 若用两角和公式(\autoref{TriEqv_eq2}\upref{TriEqv})直接计算, 得
\begin{equation}\ali{
\sum_i x_i(t) &= \sum_i [ A_i \cos\varphi_{0i}\cos(\omega t) - A_i\sin\varphi_{0i}\sin(\omega t)]\\
&= \sum_i (A_i \cos\varphi_{0i})\cos(\omega t) - \sum_i(A_i\sin\varphi_{0i})\sin(\omega t)
}\end{equation}
分别令 $C = \sum_i A_i \cos\varphi_{0i}$, $D = \sum_i A_i \sin\varphi_{0i}$, 且令 $A = \sqrt{C^2 + D^2}$, 以及令 $\varphi_0$ 满足 $\cos\varphi_0 = C/A$, $\sin\varphi_0 = D/A$, 则上式变为
\begin{equation}\label{VbExp_eq6}
\sum_i x_i(t) = A [\cos\varphi_0\cos(\omega t) - \sin\varphi_0\sin(\omega t)]
= A\cos(\omega t + \varphi_0)
\end{equation}
可见任意多个相同频率的简谐波叠加仍然是该频率的一个简谐波.

若我们用指数形式的振动来进行同样的计算, 第 $i$ 个振动可表示为 $\tilde x_i(t) = \tilde A_i \E^{-\I\omega t}$, 其中 $\tilde A_i = A_i\E^{-\I\omega\varphi_{0i}}$. 求和得
\begin{equation}
\sum_i \tilde x_i(t) = \qty(\sum_i \tilde A_i ) \E^{-\I\omega t}
\end{equation}
令 $\tilde A = \sum_i \tilde A_i$, $A = |\tilde A|$, $\varphi_0 = -\arg(\tilde A)$, 则最后结果为
\begin{equation}
\sum_i \tilde x_i(t) = \tilde A \E^{-\I\omega t}
\end{equation}
\begin{equation}
\sum_i x_i(t) = \Re \qty[\tilde A \E^{-\I\omega t}] = A \cos(\omega t + \varphi_0)
\end{equation}
不难证明上式的 $A$ 和 $\varphi_0$ 与\autoref{VbExp_eq6} 得到的 $A$ 和 $\varphi_0$ 相同, 但是这里的推导的过程却更为简洁.
