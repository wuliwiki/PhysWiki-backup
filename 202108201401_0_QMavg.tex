% 平均值(量子力学)
% keys 量子力学|平均值|本征值|本征态

\pentry{测量理论\upref{QMPos}}

\subsection{离散本征态}

本文使用原子单位制\upref{AU}. 我们先来回顾测量理论. 假设某个测量量 $q$ 对应的算符为 $Q$, 有离散的本征态(可以是无穷多个) $\ket{\phi_i}$ ($i = 1,2\dots$), 对应的本征值为 $q_i$, 满足
\begin{equation}\label{QMavg_eq3}
Q\ket{\phi_i} = q_i \ket{\phi_i}
\end{equation}
且满足正交归一条件
\begin{equation}\label{QMavg_eq4}
\braket{\phi_i}{\phi_j} = \delta_{i,j}
\end{equation}
记粒子处于 $\ket{\psi}$ 状态, 可表示为离散本征态的线性组合
\begin{equation}\label{QMavg_eq2}
\ket{\psi} = \sum_i c_i \ket{\phi_i}
\end{equation}
对其测量 $Q$, 得到第 $q_i$ 的概率为
\begin{equation}
P_i = \abs{c_i}^2 = \abs{\braket{\phi_i}{\psi}}^2
\end{equation}
现在我们可以习惯定义 $Q$ 的平均值为
\begin{equation}\label{QMavg_eq1}
\ev{q} = \sum_i q_i P_i = \sum_i q_i \abs{c_i}^2
\end{equation}
这意味着, 如果我们取大量处于 $\ket{\psi}$ 状态的系统, 分别测量 $Q$ 再取平均, 结果就是该式.

\begin{example}{一维简谐振子}
谐振子内容详见——量子简谐振子(升降算符法)\upref{QSHOop}. 频率为 $\omega$ 的一维线性谐振子, 若状态未
\begin{equation}
\psi(x)=c_0 \psi_0(x)+c_3\psi_3(x)
\end{equation}
其中 $\psi_n(x)$ 为第 $n$ 个归一化的能量本征态,系数 $c_1$, $c_3$ 满足归一化条件 $\abs{c_0}^2+\abs{c_3}^2 = 1$. 球能量的平均值.

由\autoref{QMavg_eq1}
\begin{equation}\label{QMavg_eq8}
\ev{E}=\abs{c_0}^2 E_0+\abs{c_3}^2 E_3
\end{equation}

其中 $E_n$ 为第 $n$ 个能量本征值:$E_n=\omega(n+\frac 12)$,其中 $\hbar$ 已取为1.代入\autoref{QMavg_eq8} 即得该态的能量平均值.
\end{example}

\subsubsection{另一个种形式}
平均值还有一个更常见的公式, 与\autoref{QMavg_eq1} 等效.
\begin{equation}\label{QMavg_eq5}
\ev{q} = \mel{\psi}{Q}{\psi}
\end{equation}
要验证, 可以将\autoref{QMavg_eq2} 代入该式, 得
\begin{equation}
\begin{aligned}
\ev{q} &= \qty(\sum_i c_i^* \bra{\phi_i}) Q \qty(\sum_j c_j \ket{\phi_j})\\
&= \sum_{i,j} c_i^* c_j \bra{\phi_i} Q \ket{\phi_j}
\end{aligned} 
\end{equation}
再代入\autoref{QMavg_eq3} 和\autoref{QMavg_eq4} 得
\begin{equation}
\ev{q} = \sum_{i,j} c_i^* c_j q_i \braket{\phi_i}{\phi_j}
= \sum_{i,j} c_i^* c_j q_i \delta_{i,j} = \sum_i \abs{c_i}^2 q_i
\end{equation}
证毕.

\subsection{连续本征态}
\pentry{连续正交归一基底与傅里叶变换\upref{COrNoB}}

若 $Q$ 的本征态是连续的, 记为 $\ket{q}$ ($q\in [a,b]$), 且满足正交归一条件(链接未完成,引用量子力学散射态相关词条)
\begin{equation}
\braket{q'}{q} = \delta(q'-q)
\end{equation}
那么波函数的展开变为
\begin{equation}\label{QMavg_eq7}
\ket{\psi} = \int_a^b c(q) \ket{q} \dd{q}
\end{equation}
其中系数 $c(q)$ 是关于 $q$ 的复值函数
\begin{equation}
c(q) = \braket{q}{\psi}
\end{equation}
而 $\abs{c(q)}^2$ 则是测得 $q$ 的概率密度函数. 这时可以定义平均值为
\begin{equation}
\ev{q} = \int_a^b q \abs{c(q)}^2 \dd{q}
\end{equation}
同样容易证明等效形式为
\begin{equation}
\ev{q} = \mel{\psi}{Q}{\psi}
\end{equation}
证明: 把\autoref{QMavg_eq7} 代入, 得
\begin{equation}
\begin{aligned}
\ev{q} &= \int_a^b\int_a^b \abs{c(q)}^2\mel{q'}{Q}{q} \dd{q'}\dd{q} = \int_a^b\int_a^b q\abs{c(q)}^2\braket{q'}{q} \dd{q'}\dd{q}\\
&= \int_a^b\int_a^b q\abs{c(q)}^2\delta(q'-q) \dd{q'}\dd{q}\\
&= \int_a^b q\abs{c(q)}^2 \dd{q}
\end{aligned}
\end{equation}
证毕.

\subsection{动量的平均值(一维)}
动量的算符为 $P = -\I\pdv{x}$, 本征值记为 $k = p$, 因为在原子单位中波数等于动量(链接未完成). 归一化的本征函数为
\begin{equation}
\ket{p} = \frac{\E^{\I k x}}{\sqrt{2\pi}}
\end{equation}
满足归一化条件(链接未完成,delta 函数相关公式)
\begin{equation}
\braket{p'}{p} = \int_{-\infty}^{\infty} \frac{\E^{-\I k x}}{\sqrt{2\pi}} \frac{\E^{\I k x}}{\sqrt{2\pi}} \dd{x} = \delta(k' - k)
\end{equation}
若要计算函数 $\ket{\psi} = \psi(x)$ 的动量平均值, 那么可以直接计算
\begin{equation}
\ev{p} = \mel{\psi}{P}{\psi} = -\I\int_{-\infty}^{\infty} \psi\Cj(x) \pdv{x} \psi(x) \dd{x}
\end{equation}
也可以先计算展开系数, 即傅里叶变换\upref{FTExp}
\begin{equation}
c(p) = \braket{p}{\psi} = \frac{1}{\sqrt{2\pi}}\int_{-\infty}^{\infty} \psi(x)\E^{-\I kx} \dd{x}
\end{equation}
然后计算平均值
\begin{equation}
\ev{p} = \int_{-\infty}^{\infty} p \abs{c(p)}^2\dd{p}
\end{equation}

特殊地, 若 $\psi(x)$ 是一个实函数(或者乘以一个任意相位因子 $\E^{\I \theta}$), 那么其频谱 $\abs{c(p)}^2$ 必定是一个偶函数(链接未完成), 可得动量平均值为零.


\begin{example}{波函数的坐标、动量平均值}
假设一粒子波函数为
\begin{equation}
\psi(x)=
\begin{cases}
C(a^2-x^2) &(-a\leqslant x\leqslant a)\\
0 &(x<-a \ \text{或}\ x>a) 
\end{cases}
\end{equation}
\begin{figure}[ht]
\centering
\includegraphics[width=7cm]{./figures/QMavg_2.pdf}
\caption{粒子波函数示意图} \label{QMavg_fig2}
\end{figure}
其中 $C$ 已归一化.求粒子的坐标、动量平均值.

由\autoref{QMavg_eq5} 
\begin{equation}\label{QMavg_eq6}
\ev{x}= \int_{-a}^a C\Cj(a^2-x^2)xC(a^2-x^2)\dd{x} = 0
\end{equation}

同理,将\autoref{QMavg_eq6} 中的算符 $x$ 用动量算符 $p$ 替换,可以得到动量平均值
\begin{equation}
\begin{aligned}
\ev{p} &= \int_{-a}^a C\Cj(a^2-x^2)(-\I\dv{x})C(a^2-x^2)\dd{x}\\
&= -\I\abs{C}^2\int_{-a}^a (a^2-x^2)(-2x)\dd{x}\\
&= 0
\end{aligned}
\end{equation}
\end{example}
% 未完成: 无限深势阱中的三角波包, 用两种方法求能量的平均值

\subsection{离散本征态和连续本征态}
与上面两节同理, 若算符 $Q$ 同时存在离散本征态 $\ket{q_i}$ 和连续本征态 $\ket{q}$, 且各自满足上文的正交归一化条件, 以及
\begin{equation}
\braket{q_i}{q} = 0
\end{equation}
使 $\ket{q_i}$ 和 $\ket{q}$ 共同组成一组完备正交归一基底.

那么波函数可以展开为
\begin{equation}
\ket{\psi} = \sum_i c_i \ket{q_i} + \int_a^b c(q) \ket{q}
\end{equation}
那么平均值被定义为
\begin{equation}
\ev{q} = \sum_i q_i \abs{c_i}^2 + \int_a^b q \abs{c(q)}^2 \dd{q}
\end{equation}
同样有等效形式
\begin{equation}
\ev{q} = \mel{\psi}{Q}{\psi}
\end{equation}

同时具有离散和连续本征值的典型的例子就是一个有限深势阱的哈密顿算符(能量算符)$H$.
