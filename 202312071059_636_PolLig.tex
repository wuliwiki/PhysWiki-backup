% 偏振光
% license Xiao
% type Tutor

\begin{issues}
\issueDraft
\end{issues}

\pentry{平面简谐波\upref{PWave}}

\subsection{偏振光}

光是一种电磁波,电磁波是一种横波。所谓横波,即波的传播方向与振动方向垂直的波,例如人在奔跑时,挂在脖子上的围巾的运动就可以类比为一种横波。偏振,顾名思义,是偏好某个方向振动。振动方向对于传播的方向不对称就叫做偏振。在这里,偏振光可以理解为沿着确定方向做振动的波。

根据光的偏振方向的不同,可以分为线偏振、圆偏振和椭圆偏振。线偏振,顾名思义为偏振方向呈直线;圆偏振,依据偏振的旋转方向可分为左旋圆偏振与右旋圆偏振,或简称为左(右)旋圆偏;椭圆偏振可分为左(右)旋椭偏。

下面是几种偏振光的形态。


\subsection{偏振片}
偏振片的工作原理是在光的偏振方向上对光的选择性吸收。理想情况下,线性偏振片通过与透光轴平行的光场分量,吸收与透光轴垂直的光场分量。

