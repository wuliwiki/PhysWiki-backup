% 牛顿第二定律的矢量形式

我们高中最熟悉的是一维直线运动的牛顿第二定律, 也就是常见的标量形式 $F = ma$. 高中物理已经明确过力和加速度都是矢量, 本书中矢量用黑体和正体表示, 即 $\bvec F$ 和 $\bvec a$. 我们知道, 矢量既有长度也有方向, 但是在一维情况下, 矢量只有两个方向, 我们把其中一个定义为正方向, 那么另一个就是反方向. 

物理中有一个简单的约定, 就是一维的矢量可以用标量(也就是一个实数)表示, 对于正方向的矢量, 就使用正数, 反方向的矢量就使用负数. 所以在一维情况下, 矢量于标量可以一一对应.

事实上牛顿第二定律适用于任何曲线运动, 它的矢量形式是 $\bvec F = m\bvec a$, 比标量形式所包含的意义要丰富得多. 一个典型的例子就是在圆周运动中, 质点存在向心加速度.


\begin{equation}
p = m\bvec v = m\int a \dd{t} = 
\end{equation}
