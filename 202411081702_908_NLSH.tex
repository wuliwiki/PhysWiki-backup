% 能量守恒定律(综述)
% license CCBYSA3
% type Wiki

本文根据 CC-BY-SA 协议转载翻译自维基百科\href{https://en.wikipedia.org/wiki/Conservation_of_energy}{相关文章}。

能量守恒定律表明,孤立系统的总能量保持不变;即它随着时间的推移是守恒的。[1] 对于封闭系统,该原理表明系统内的总能量只能通过能量的进出而改变。能量既不能被创造,也不能被销毁;它只能从一种形式转化或转移为另一种形式。例如,当一根炸药爆炸时,化学能转化为动能。如果把爆炸中释放的所有形式的能量相加,如碎片的动能和势能以及热和声,就可以得到炸药燃烧过程中化学能的精确减少量。

在经典物理中,能量守恒与质量守恒是不同的。然而,狭义相对论表明质量和能量是相互关联的,反之亦然,其关系由方程 \( E = mc^2 \) 表示,即质能等价,现在科学认为整体的质能是守恒的。从理论上讲,这意味着任何具有质量的物体本身可以转化为纯能量,反之亦然。然而,人们认为这只有在最极端的物理条件下才可能发生,例如宇宙大爆炸后的极短时间内,或黑洞发射霍金辐射时。

根据驻行动原理,通过诺特定理可以严格证明能量守恒是连续时间平移对称性的结果;即物理定律不随时间改变这一事实。

能量守恒定律的一个结果是第一类永动机不可能存在;也就是说,没有外部能量供应的系统不能向其周围环境提供无限的能量。[2] 根据能量的定义,能量守恒在宇宙尺度上可以被认为在广义相对论中被违背。[3]
\subsection{历史}
