% 倒格空间
% keys 倒格矢,倒格空间,劳厄方程

为了观测晶体结构,我们需要用特殊的方式对晶体进行探测。先了解一些基本知识:原子间距大约$\sim 1$ nm,而人眼一般能看见的可见光波段在 $380 \sim 780 $ nm,这就使得利用可见光的干涉消长现象来推断晶体结构是不可能的事情。

为什么?假设你有一把尺子,每一格刻度最小为 $1$ mm。现在有一根头发丝,我们用尺子去测量它时,有办法直接精确测量出头发丝直径吗?显然很困难,我们的估算将会有很大误差。用波的干涉消长测量原子的间距,观察原子排布方式也是类似的原理。假设一束波长为$100$ nm的波入射,原子将紧密分布在波的每一个地方,相邻原子感受到的光场几乎相同,由此其散射光也几乎没有差异,我们无法得到干涉现象。

因此,需要将观测的入射波长缩小至$\sim 1$ nm 乃至更小才能实现观测。怎么办?由物质波的
\subsection{布拉格定律}
1912年劳厄(Laue)和1913年布拉格父子(Bragg)基于X射线衍射的晶体结构分析及其方程的建立,标志着近代固体物理学的开端。


\subsection{劳厄方程}