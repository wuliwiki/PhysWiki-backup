% 零点能量(综述)
% license CCBYSA3
% type Wiki

本文根据 CC-BY-SA 协议转载翻译自维基百科\href{https://en.wikipedia.org/wiki/Zero-point_energy}{相关文章}。

\begin{figure}[ht]
\centering
\includegraphics[width=8cm]{./figures/9b23f9fc5e8f2f7f.png}
\caption{液态氦由于零点能的存在,在标准大气压下无论温度如何都保持动能,不会冻结。当其温度降到 Lambda 点以下时,它表现出超流体性特性。} \label{fig_LD_1}
\end{figure}
\textbf{零点能(ZPE)}是量子力学系统可能具有的最低能量。与经典力学不同,量子系统即使在最低能量状态下也会不断波动,这可以通过海森堡不确定性原理来描述[1]。因此,即使在绝对零度下,原子和分子也会保持某些振动运动。除了原子和分子外,真空的空旷空间也具有这些性质。根据量子场论,宇宙可以被看作不仅是孤立的粒子,而是连续波动的场:物质场,其量子是费米子(即轻子和夸克),以及力场,其量子是玻色子(例如光子和胶子)。所有这些场都有零点能[2]。这些波动的零点场导致了一种在物理学中重新引入以太的现象[1][3],因为某些系统可以探测到这种能量的存在[需要引用]。然而,如果这个以太要保持洛伦兹不变性,以保证与爱因斯坦的相对论没有矛盾,那么它就不能被视为一种物理介质[1]。

零点能的概念对宇宙学也非常重要,然而,物理学目前缺乏一个完整的理论模型来理解宇宙学中的零点能;特别是理论上与观测到的宇宙真空能量之间的差异,成为了一个重大争议问题[4]。然而,根据爱因斯坦的广义相对论,任何这种能量都会引起引力,而来自宇宙膨胀、暗能量和Casimir效应的实验证据表明,任何这种能量都极其微弱。一个试图解决这一问题的提案是认为费米子场具有负的零点能,而玻色子场具有正的零点能,因此这些能量会以某种方式相互抵消[5][6]。如果超对称是自然界的精确对称性,这个想法是成立的;然而,欧洲核子研究中心的大型强子对撞机至今未找到支持这一理论的证据。此外,已知如果超对称是有效的,它最多也只是一个破缺的对称性,仅在极高的能量下才成立,目前没有人能够展示一个低能宇宙中发生零点能抵消的理论[6]。这一差异被称为宇宙学常数问题,是物理学中最大的未解之谜之一。许多物理学家认为,“真空是理解自然的关键”[7]。
\subsection{词源和术语}  
零点能(ZPE)一词是从德语“Nullpunktsenergie”翻译过来的。[8] 有时,它与零点辐射和基态能量互换使用。零点场(ZPF)一词可以用来指代特定的真空场,例如量子电动力学(QED)真空,它专门处理量子电动力学(如光子、电子和真空之间的电磁相互作用),或者量子色动力学(QCD)真空,它涉及量子色动力学(如夸克、胶子和真空之间的色荷相互作用)。真空可以被视为不是空的空间,而是所有零点场的组合。在量子场论中,这种场的组合被称为真空态,与之相关的零点能量被称为真空能量,平均能量值称为真空期望值(VEV),也称为其凝聚态。
\subsection{概述}
\begin{figure}[ht]
\centering
\includegraphics[width=8cm]{./figures/429e3fe66ba0fbb1.png}
\caption{动能与温度} \label{fig_LD_2}
\end{figure}
在经典力学中,所有粒子都可以被认为具有某种能量,这种能量由它们的势能和动能组成。例如,温度来自于由动能引起的随机粒子运动的强度(称为布朗运动)。当温度降低到绝对零度时,可以认为所有运动都停止,粒子完全静止。然而,实际上,即使在最低的温度下,粒子仍然保持动能。与这种零点能量对应的随机运动永远不会消失;它是量子力学不确定性原理的结果。
\begin{figure}[ht]
\centering
\includegraphics[width=8cm]{./figures/a6ab0822f805ed5d.png}
\caption{零点辐射不断地对电子施加随机冲动,使得电子永远无法完全停止。零点辐射赋予振荡器的平均能量等于振荡频率乘以普朗克常数的一半。} \label{fig_LD_3}
\end{figure}
不确定性原理表明,任何物体无法同时拥有精确的位置和速度值。量子力学物体的总能量(包括势能和动能)由其哈密顿量描述,哈密顿量也描述了该系统作为一个简谐振子或波函数,在不同的能量状态之间波动(参见波粒二象性)。所有量子力学系统即使在其基态下也会经历波动,这是它们波动性本质的结果。不确定性原理要求每个量子力学系统必须具有大于经典势阱最小值的波动零点能量。这导致即使在绝对零度下也会有运动。例如,液氦在大气压力下无论温度如何都不会冻结,这正是由于其零点能量。

根据阿尔伯特·爱因斯坦的质量与能量等价关系 \( E = mc^2 \),任何包含能量的空间点都可以被看作具有质量,从而产生粒子。现代物理学已经发展出了量子场论(QFT),用以理解物质与力之间的基本相互作用;它将空间中的每个点视为一个量子简谐振子。根据量子场论,宇宙由物质场组成,物质场的量子是费米子(如轻子和夸克),以及力场,力场的量子是玻色子(如光子和胶子)。所有这些场都具有零点能量。最近的实验支持这样一个观点:粒子本身可以看作是基础量子真空的激发态,物质的所有属性只是由零点场相互作用引起的真空波动。

‘空’空间可以具有内在能量,并且没有‘真正的真空’这一概念,这一观点似乎是反直觉的。通常认为,整个宇宙完全浸泡在零点辐射中,因此它只能在计算中增加一个常数值。因此,物理测量将仅揭示该值的偏差。[10] 对于许多实际计算,零点能量通常被在数学模型中以强制性的方式忽略,作为一个没有物理效应的项。然而,这种处理方式会引发问题,因为在爱因斯坦的广义相对论中,空间的绝对能量值不是一个任意常数,而是产生了宇宙学常数。几十年来,大多数物理学家认为,存在某种尚未发现的基本原理,能够消除无限的零点能量并使其完全消失。如果真空没有内在的、绝对的能量值,它就不会发生引力效应。人们曾认为,随着宇宙从大爆炸的余波中膨胀,任何单位的空旷空间中所包含的能量将减少,因为总能量扩展以填满宇宙的体积;宇宙中的星系和所有物质应该开始减速。然而,1998年通过发现宇宙的膨胀并没有减缓,而是加速了,这一可能性被排除。这意味着空空间确实具有某种内在的能量。暗能量的发现最好通过零点能量来解释,尽管目前仍然是一个谜,为什么其值与通过理论得到的巨大值相比如此之小——这就是宇宙学常数问题。[5]

许多归因于零点能量的物理效应已经通过实验验证,如自发辐射、卡西米尔力、兰姆位移、电子的磁矩和德尔布吕克散射。[11][12] 这些效应通常被称为‘辐射修正’。[13] 在更复杂的非线性理论中(例如量子色动力学,QCD),零点能量可以引发各种复杂的现象,如多稳定态、对称性破缺、混沌和涌现。当前的研究领域包括虚拟粒子的效应,[14] 量子纠缠,[15] 惯性质量和引力质量之间的差异(如果有的话),[16] 光速变化,[17] 观察到的宇宙学常数的原因,[18] 以及暗能量的性质。[19][20]
\subsection{历史 } 
\subsubsection{早期的以太理论}
\begin{figure}[ht]
\centering
\includegraphics[width=6cm]{./figures/36a1f632beb96010.png}
\caption{詹姆斯·克拉克·麦克斯韦} \label{fig_LD_4}
\end{figure}
零点能量源自关于真空的历史思想。对于亚里士多德来说,真空是 τὸ κενόν,‘空的’;即与物体无关的空间。他认为这个概念违反了基本的物理原则,并主张火、空气、地球和水的元素并非由原子构成,而是连续的。对于原子论者来说,‘空’的概念具有绝对性质:它是存在与不存在的区别。关于真空特性的辩论大多局限于哲学领域,直到文艺复兴时期才开始有所突破,当时奥托·冯·格里克发明了第一个真空泵,科学上可验证的理论才开始出现。人们认为,简单地去除所有气体,就能创造出完全空的空间,这是第一个被普遍接受的真空概念。

然而,到了19世纪末,显然即使在抽空的区域中,仍然存在热辐射。以太存在作为真实空无的替代品是当时最流行的理论。根据基于麦克斯韦电动力学的成功电磁以太理论,这种无所不包的以太被赋予了能量,因此与虚无是非常不同的。电磁现象和引力现象能够在空空间中传播,被认为是它们相关的以太是空间本身的一部分。然而,麦克斯韦指出,这些以太在大多数情况下是临时设定的:

“对于那些坚持认为以太作为哲学原则存在的人来说,自然对真空的厌恶是想象一种包围一切的以太的充分理由……以太被发明出来,让行星在其中游动,构成电气大气层和磁气流,传递从身体一部分到另一部分的感觉等等,直到空间被以太填充了三四次。”

此外,1887年的迈克尔逊–莫雷实验结果是第一次强烈证据表明当时流行的以太理论存在严重缺陷,并启动了最终导致狭义相对论的研究路线,后者完全排除了静止以太的概念。对于当时的科学家来说,似乎可以通过冷却并消除所有辐射或能量来在空间中创造一个真正的真空。由此演变出了第二个实现真正真空的概念:将一个区域的空间冷却至绝对零度温度,然后进行抽空。绝对零度在19世纪技术上是无法实现的,因此这一辩论没有得到解决。
\subsubsection{第二量子理论}
\begin{figure}[ht]
\centering
\includegraphics[width=6cm]{./figures/1755ae4a531d63e4.png}
\caption{普朗克在1918年,因其在量子理论方面的工作获得诺贝尔物理学奖。} \label{fig_LD_6}
\end{figure}
在1900年,马克斯·普朗克推导了单个能量辐射体(例如振动的原子单位)平均能量 \(\varepsilon\) 与绝对温度的关系公式:
\[
\varepsilon = \frac{h\nu}{e^{h\nu / (kT)} - 1}~
\]
其中,\(h\) 是普朗克常数,\(\nu\) 是频率,\(k\) 是玻尔兹曼常数,\(T\) 是绝对温度。零点能量并未对普朗克的原始定律做出贡献,因为在1900年时,普朗克尚不知晓零点能量的存在。

零点能量的概念是由马克斯·普朗克在1911年在德国提出的,作为对他在1900年提出的原始量子理论中零基态公式的修正项。

在1912年,普朗克发表了第一篇关于辐射不连续发射的期刊文章,基于能量的离散量子。他的“第二量子理论”中,共振器连续吸收能量,但仅当它们达到相空间的有限单元边界时,才会以离散的能量量子发射能量,其中它们的能量成为 \(h\nu\) 的整数倍。这一理论促使普朗克得出了新的辐射定律,但在这个版本中,能量共振器具有零点能量,即共振器能够取的最小平均能量。普朗克的辐射方程包含一个残余能量因子 \(\frac{h\nu}{2}\) 作为额外项,这个项依赖于频率 \(\nu\),且大于零(其中 \(h\) 是普朗克常数)。因此,普遍认为“普朗克方程标志着零点能量概念的诞生”。在1911到1913年间的一系列论文中,[29]普朗克发现了振荡器的平均能量为:
\[
\varepsilon = \frac{h\nu}{2} + \frac{h\nu}{e^{h\nu / (kT)} - 1}~
\]
\begin{figure}[ht]
\centering
\includegraphics[width=6cm]{./figures/4b6500c57682c25f.png}
\caption{爱因斯坦在获得1921年诺贝尔物理学奖后的官方肖像。} \label{fig_LD_5}
\end{figure}
不久,零点能量的概念引起了阿尔伯特·爱因斯坦及其助手奥托·斯特恩的关注。[31] 他们在1913年发表了一篇论文,试图通过计算氢气的比热并与实验数据进行比较来证明零点能量的存在。然而,在他们认为自己成功之后,他们很快撤回了对这一理论的支持,因为他们发现普朗克的第二理论可能并不适用于他们的例子。在同年的一封信中,爱因斯坦向保罗·艾伦费斯特声明,零点能量“死得像钉子一样”。[32] 零点能量也被彼得·德拜提及,[33] 他指出,即使温度接近绝对零度,晶格中原子的零点能量也会导致X射线衍射辐射强度的降低。1916年,瓦尔特·能斯特提出,空旷的空间充满了零点电磁辐射。[34] 随着广义相对论的发展,爱因斯坦发现真空的能量密度有助于产生一个宇宙常数,以便得到他场方程的静态解;空旷空间或真空可以具有某种内在能量的观点重新出现了。爱因斯坦在1920年表示:

有一个强有力的论点支持以太假说。否认以太最终意味着假设空旷空间没有任何物理性质。力学的基本事实与这种观点并不和谐……根据广义相对论,空间是赋予物理性质的;因此,在这个意义上,存在一个以太。根据广义相对论,没有以太的空间是不可思议的;因为在这样的空间中,不仅没有光的传播,而且没有时间和空间的标准存在(即测量杆和时钟),因此也就没有物理意义上的时空间隔。但是,这个以太不能被看作是具有可追溯时间的物质特性,也不能应用于可追踪的部分。运动的概念不能应用于它。[35][36]
\begin{figure}[ht]
\centering
\includegraphics[width=6cm]{./figures/d83b5e7eb6dd8e91.png}
\caption{} \label{fig_LD_7}
\end{figure}
Kurt Bennewitz 和 Francis Simon(1923年),他们在沃尔特·能斯特(Walther Nernst)位于柏林的实验室工作,研究了低温下化学物质的熔化过程。他们计算了氢气、氩气和水银的熔点,并得出结论,认为这些结果为零点能提供了证据。此外,他们还正确地提出(后来由西蒙(1934年)验证)这一量是导致氦气即使在绝对零度下也难以固化的原因。1924年,罗伯特·穆立肯(Robert Mulliken)通过比较10BO和11BO的带谱,提供了分子振动的零点能的直接证据:如果没有零点能,不同电子能级基态的振动频率的同位素差异应该会消失,这与观察到的谱线相矛盾。仅仅一年后的1925年,随着在维尔纳·海森堡的文章《量子理论对运动学和力学关系的重新解释》中矩阵力学的发展,零点能从量子力学中得到了推导。

1913年,尼尔斯·玻尔(Niels Bohr)提出了现在被称为玻尔模型的原子模型,但尽管如此,为什么电子不会坠入原子核依然是一个谜。根据经典理论,考虑到加速电荷通过辐射损失能量,意味着电子应该会螺旋式地坠入原子核,原子也不应该是稳定的。这个经典力学的问题在1915年被詹姆斯·霍普伍德·吉恩斯(James Hopwood Jeans)总结得很好:“假设力学定律 \(\frac{1}{r^2}\) 在r趋近于零时仍然成立,将会是一个非常真实的困难。因为两电荷在零距离时的相互作用力将是无限大的;我们应该看到异号电荷不断相互吸引,且一旦相遇,没有力会使它们进一步收缩或无限减小。” 这一难题的解决出现在1926年,当时厄尔温·薛定谔(Erwin Schrödinger)提出了薛定谔方程。该方程解释了这样一个新的非经典事实:当电子被限制靠近原子核时,它必然会拥有很大的动能,因此最小的总能量(动能加势能)实际上会出现在某个正的距离上,而不是零距离;换句话说,零点能对于原子稳定性至关重要。