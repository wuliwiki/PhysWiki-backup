% 绝热近似(量子力学)
% license Usr
% type Tutor

\begin{issues}
\issueTODO
\end{issues}

\footnote{参考 \cite{GriffE} 的章节: The Adiabatic Approximation、 \cite{Shankar} Chap18-P478、 Wikipedia \href{https://en.wikipedia.org/wiki/Adiabatic_theorem}{相关页面}。}若系统初始时处于某个离散非简并的本征态,那么当哈密顿量随时间缓慢改变时(改变的特征时间远大于本征态的周期), 那改变过程中波函数将仍然处于同一个本征态,除了一个相位因子的改变。

具体来说,令含时薛定谔方程为(用函数上一点表示关于时间的偏导数)
\begin{equation}
H(t)\Psi(t) = \I\hbar\dot\Psi(t)~.
\end{equation}
当系统不存在简并时, 渐进近似下含时薛定谔方程的通解可以表示为($c_n$ 为常数,由初始波函数决定)
\begin{equation}\label{eq_AdiaQM_2}
\Psi(t) \approx \sum_n c_n \psi_n(t) \E^{\I\gamma_n(t)}\E^{\I\theta_n(t)}~.
\end{equation}
其中 $\psi_n(t)$ 是 $H(t)$ 一组正交归一本征态(为了方便本文把它们限制为实值函数),满足不含时薛定谔方程(时间看作数学参数)
\begin{equation}
H(t)\psi_n(t) = E_n\psi_n(t)~.
\end{equation}
两个相位因子分别为
\begin{equation}
\gamma_m(t) = \I \int_0^t \braket*{\psi_m(t')}{\dot\psi_m(t')}\dd{t'}~,
\end{equation}
\begin{equation}
\theta_n(t) = -\frac{\I}{\hbar} \int_0^t E_n(t')\dd{t'}~.
\end{equation}
容易看出当 $H(t)$ 不随时间变化时,通解就回到了熟悉的通解(链接未完成)
\begin{equation}
\Psi(t) = \sum_n c_n \psi_n \E^{-\I E_n t}~.
\end{equation}


也就是说,若 $t=0$ 时波函数是若干非简并离散本征态的线性组合, 那么每个系数的模长不会发生改变。

\begin{example}{}
\begin{enumerate}
\item 当无限深势阱\upref{ISW}缓慢变长。
\item 量子简谐振子(升降算符法)\upref{QSHOop}的劲度系数 $k$ 缓慢变化。
\end{enumerate}
\end{example}

该理论在对分子的计算中有广泛的应用,且有一个响亮的名字,叫\textbf{波恩—奥本海默近似(Born–Oppenheimer approximation)}。 这是因为在分子运动中,原子核的运动速度通常要比电子慢得多,使渐进近似效果较好。

同为含时近似理论,绝热近似和含时微扰理论\upref{TDPTc}有什么区别呢? 前者不要求 $H(t)$ 缓慢变化,例如用激光波包对原子光电离时,电场随时间的变化并不算慢。 那可以使用绝热近似的情况是否可以使用含时微扰理论呢? 理论上可以,但可能计算比较麻烦,因为含时微扰使用初始的本征态展开任意时刻的波函数。

\subsection{简并情况}
\addTODO{大胆猜测,若初始波函数所处的简并空间维数不变,那么和非简并情况一样。 但若随着 $H$ 的变化分裂成多个子空间, 则需要先把初始波函数投影到每个子空间上,再分别演化。}

\begin{example}{}
给氢原子的任意束缚态 $\psi_{2,l,m}$ 缓慢施加外电场或磁场(参考 “类氢原子斯塔克效应(微扰)\upref{HStark}”)。
\addTODO{推导}
\end{example}

\subsection{推导}
若哈密顿量不随时间改变,
\begin{equation}
\Psi_n(t) = \psi_n \E^{-\I E_n t}~.
\end{equation}
若随时间改变, 本征态和本征值都变为时间的函数 $\psi_n(t)$ 和 $E_n(t)$。 但仍然正交归一。 此时的含时波函数仍然可以用它们展开
\begin{equation}
\Psi(t) = \sum_n c_n(t) \psi_n(t) \E^{\I \theta_n(t)}~,
\end{equation}
其中
\begin{equation}
\theta_n(t) = -\frac{1}{\hbar} \int_0^t E_n(t')\dd{t'}~.
\end{equation}
代入含时薛定谔方程
\begin{equation}\label{eq_AdiaQM_1}
H(t)\Psi(t) = \I \dot \Psi(t)~,
\end{equation}
得
\begin{equation}
\dot c_m(t) = -c_m \braket*{\psi_m}{\dot\psi_m} - \sum_{n}^{E_n\ne E_m} c_n \frac{\mel*{\psi_m}{\dot H}{\psi_n}}{E_n-E_m} \exp\qty{{-\frac{\I}{\hbar}\int_0^t[E_n(t')-E_m(t')]\dd{t'}}}~.
\end{equation}
现在还没有使用任何近似。 绝热近似就在于假设 $\dot H$ 非常小,从而忽略该式中第二项。 那么有
\begin{equation}
c_m(t) = c_m(0)\E^{\I\gamma_m(t)}~,
\end{equation}
其中
\begin{equation}
\gamma_m = \I \int_0^t \braket{\psi_m(t')}{\dot\psi_m(t')}\dd{t'}~.
\end{equation}
这就得到了\autoref{eq_AdiaQM_2}。
