% 高锟
% license CCBYSA3
% type Wiki

(本文根据 CC-BY-SA 协议转载自原搜狗科学百科对英文维基百科的翻译)

\textbf{查尔斯·高锟爵士} GBM KBE FRS [1][2][3] (1933年11月4日至2018年9月23日)是一位物理学家和电气工程师,他率先在电信领域开发和使用光纤。20世纪60年代,高锟发明了多种方法将玻璃纤维和激光结合起来传输数字数据,这为互联网的发展奠定了基础。

被誉为“宽带教父” 、“光纤之父”、[4][5][6] 和“光纤通信之父”[7] 的高锟因“在光纤通信中光传输方面的突破性成就”获得2009年诺贝尔物理学奖。[8]

高锟出生于中国上海,是香港永久居民[9] ,并在英国和美国拥有公民身份。

\subsection{早期生活和教育}
高锟于1933年出生于中国上海,[10] 他的祖籍在金山附近,[10] 当时是一个独立的行政区。[11] 他和哥哥在家庭教师的指导下学习中国古典文学。[12][10] 他还在上海法租界的一所国际学校学习英语和法语,[13] 这所学校是由包括蔡元培在内的一些先驱的中国教育家创办的。[14]

高锟的家人于1948年移居台湾,随后又移居英属香港,[10][15] 1952年,他在圣约瑟夫学院完成了中学教育(香港中学会考,HKCEE的前身)。[16][17]他在伍尔维奇理工学院(现在的格林威治大学) 完成了电气工程本科学习,获得了工程学士学位。[10]

之后,他继续研究,并于1965年在伦敦大学获得电气工程博士学位,当时他是伦敦大学学院哈罗德·巴洛教授的旁听生,在英国哈洛的标准电信实验室(STL)工作,该实验室是标准电话和电缆的研究中心。[18] 正是在那里,高锟在亚历克·里维斯的管理下,作为一名工程师和研究员,与乔治·霍克汉姆并肩作战,完成了他的第一项开创性工作。

\subsubsection{1.1 祖先和家庭}
高锟的父亲高君湘[10] 是一名律师, 1925年从密歇根大学法学院获得法学博士学位。[19] 他是中国苏州大学(当时在上海)比较法学院的教授。[20][21]

他的祖父高燮是晚清时期的学者、诗人、艺术家[12] 和南方社会的领军人物。[22] 包括高旭、姚光在内的几位作家和高增也都是高锟的近亲。

他父亲的堂兄弟是天文学家高平子[12][23] (高平子环形山就是以他的名字命名的[24])。高锟的弟弟高铻是华盛顿特区美国天主教大学的土木工程师和荣誉退休教授。他的研究领域是流体力学。[25]

毕业后,高锟在伦敦遇到了他未来的妻子黄美芸,他们都在伦敦的标准电话电缆公司当工程师。[10][26] 黄美芸是英国华裔。[10] 他们于1959年在伦敦结婚,[10][27] 有两个孩子,一个儿子和一个女儿,[27] 他们都在加利福尼亚的硅谷生活和工作。[28][29][26] 根据高锟的自传,高锟是天主教徒,而他的妻子参加英国圣公会。[10]

\subsection{学术生涯}
\subsubsection{2.1 光纤和通信}
\begin{figure}[ht]
\centering
\includegraphics[width=6cm]{./figures/a28d89f54d895c5f.png}
\caption{一束用于光通信的二氧化硅玻璃纤维,这是全球事实标准。高锟还首次公开表示,高纯二氧化硅玻璃是一种理想的长距离光通信材料。[1]} \label{fig_GK_1}
\end{figure}
20世纪60年代,在位于埃塞克斯哈洛的标准电信实验室(STL),高锟和他的同事在实现光纤作为电信媒介方面做了开创性的工作,证明现有光纤的高损耗是由玻璃中的杂质引起的,而不是技术本身的根本问题。[30]

1963年,当高锟首次加入光通信研究团队时,他记录总结了当时的背景情况[31] 和可用技术,并确定了涉及的关键人物。[31] 最初,高锟在安东尼·卡尔鲍伊克(Toni Karbowiak)的团队工作,他在亚历克·里维斯手下研究通信光波导。高锟的任务是研究纤维衰减,为此他从不同的纤维制造商那里收集样本,并仔细研究大块玻璃的性能。高锟的研究初步让他相信是材料中的杂质导致了这些纤维的高光损失。[32] 那年晚些时候,高锟被任命为标准电信实验室光电研究小组的组长。[33] 他于1964年12月接管了标准电信实验室的光通信项目,因为他的导师卡尔鲍伊克(Karbowiak)去了澳大利亚悉尼新南威尔士大学(UNSW)电气工程学院,担任通信教授。[34]

虽然高锟接替卡尔鲍伊克担任光通信研究经理,但他立即决定放弃卡尔博维克的计划(薄膜波导),与同事乔治·霍克汉姆(George Hockham)一起全面改变了研究方向。[32][34] 他们不仅考虑了光学物理,还考虑了材料特性。1966年1月,高锟在伦敦向英国电机工程师学会首次提交了研究结果,并于7月与乔治·霍克汉(1964-1965年与高锟合作)一起发表了进一步的研究成果。[35] 这项研究首先从理论上提出用玻璃纤维实现光通信,所描述的思想(特别是结构特征和材料)在很大程度上是当今光纤通信的基础。

1965年,[33][36] 霍克汉姆和高锟得出结论,玻璃光衰减的基本极限低于20 dB/km(分贝/km,是信号在一定距离内衰减的量度),这是光通信的关键阈值。[37] 然而,在测定时,光纤通常表现出高达1000分贝/公里甚至更高的光损耗。这一结论开启了寻找低损耗材料和适合达到这一标准的合适纤维的激烈竞争。

高锟和他的新团队(成员包括戴维、琼斯和赖特)通过测试各种材料来努力达成这个目标。他们精确测量了不同波长的光在玻璃和其他材料中的衰减。在此期间,高锟指出高纯度的熔融石英(二氧化硅)是光通信的理想候选。高锟还指出,玻璃材料的杂质是玻璃纤维内部光传输急剧衰减的主要原因,而不是像当时许多物理学家认为的像散射这样的基本物理效应,这种杂质可以被去除。这导致了高纯度玻璃纤维的全球研究和生产。[38] 当高先生第一次提出这种玻璃纤维可以用于远距离信息传输,并且可以代替那个时代用于电信的铜线时,他的想法被广泛地怀疑;后来人们才意识到高锟的想法彻底改变了整个通信技术和行业。[39]

他还在光通信的工程和商业实现的早期阶段发挥了主导作用。 1966年春天,高锟前往美国,但未能引起贝尔实验室(Bell Labs)的兴趣,当时贝尔实验室是STL在通信技术方面的竞争对手。[40] 他后来去了日本并获得了支持。[40] 高锟参观了许多玻璃和聚合物工厂,与包括工程师、科学家、商人在内的许多人讨论了玻璃纤维制造的技术和改进。1969年,高锟和琼斯测量了体积熔融石英的固有损耗为4 dB/km,这是超透明玻璃可能存在的第一个证据。贝尔实验室开始认真考虑光纤。[40]

高锟开发了玻璃纤维波导的重要技术和配置,并为满足民用和军用应用要求的不同纤维类型和系统设备以及光纤通信外围支持系统的开发做出了贡献。 20世纪70年代中期,他在玻璃纤维疲劳强度方面做了开创性的工作。 当被任命为第一位ITT执行科学家时,高锟推出了“兆比特技术”计划,解决信号处理的高频极限,因此高锟也被称为“兆比特技术概念之父”。[41] 高锟已经发表了100多篇论文,获得了30多项专利, 其中包括防水高强度纤维(与马克拉德公司合作)。[42]

在光纤开发的早期阶段,高锟已经强烈地倾向于使用单模进行长距离光通信,而不是使用多模系统。他的设想后来被采纳,现在几乎被完全应用。[38][43]高锟也是现代海底通信电缆的梦想家,并在很大程度上推动了这一想法。他在1983年预测,世界海洋将遍布光纤,比这种跨海洋光纤电缆首次投入使用提前了五年。[44]

阿里·贾文引进稳定的氦氖激光器和高锟发现光纤损耗特性在现在被认为是光纤通信发展的两个重要里程碑。[34]

\subsubsection{2.2 之后的工作}
高锟于1970年加入香港中文大学(CUHK)成立了电子系,后来成为电子工程系。在此期间,高锟是读者,然后是香港中文大学的电子系教授;他建立了电子专业的本科生和研究生课程,并管理他的第一批学生毕业。在他的领导下,香港中文大学成立了教育学院和其他新的研究机构。他于1974年回到美国国际电视电报公司(当时卫星电视公司的母公司),在弗吉尼亚州罗诺克工作,先是担任首席科学家,后来担任工程总监。1982年,他成为第一位国际电视电报公司执行科学家,常驻在康涅狄格州的高级技术中心。[45] 在那里,他担任耶鲁大学特鲁姆布尔学院的兼职教授和研究员。1985年,高锟在西德SEL研究中心呆了一年。1986年,高锟担任国际电话电报公司研究公司董事。

他是最早研究香港填海造地对环境影响的人之一,并于1972年在爱丁堡举行的英联邦大学协会(ACU)会议上陈述了他的第一份相关研究报告。[45]

高锟在1987年至1996年期间担任香港中文大学副校长。[46] 自一1991年起,高锟成为香港精电国际有限公司独立非执行董事及审核委员会成员。[47][48] 1993年至1994年,他担任东南亚高等教育机构协会主席。[49] 1996年,高锟向耶鲁大学捐款,并设立了高琨基金研究基金,以支持耶鲁在亚洲的研究、研究和创新项目。[50] 该基金目前由耶鲁大学东亚和东南亚研究委员会管理。[51]。1996年从香港中文大学退休后,高锟在伦敦帝国理工学院电气与电子工程系休了六个月的假;从1997年到2002年,他还在同一系担任客座教授。[52]

高先生曾担任香港能源咨询委员会主席及委员两年,并于2000年7月15日退休。[53][54] 高锟先生是香港创新科技顾问委员会成员,于2000年4月20日获委任。[55] 2000年,高锟参与创立了位于香港数码港的独立学校基金会学院。[56] 他在2000年是ISF的创始主席,并于2008年12月从ISF董事会卸任。[56] 高锟是2002年在台湾台北举行的电气电子工程师协会全球电信展的主讲人。2003年,高锟被任命为国立台湾大学电子工程与计算机科学学院电子研究所的教授。 高锟随后担任香港电信咨询公司环球科技服务有限公司的董事长兼首席执行官。他是ITX服务有限公司的创始人、董事长兼首席执行官。自2003年至2009年1月30日,高锟担任独立非执行董事及未来媒体审计委员会委员。[57][58]

\subsection{荣誉和奖项}
高锟获得了许多荣誉和奖项,其中最著名的是诺贝尔物理学奖。他的奖项包括:
\subsubsection{3.1 荣誉}
\begin{itemize}
\item 1993年:英帝国高级勋爵士。[59]
\item 2010年:英帝国最高级巴思爵士。[60][60]
\item 2010年:香港特别行政区大紫荆勋章。[61]
\end{itemize}
\subsubsection{3.2 社会和学术认可}
\begin{itemize}
\item 
\end{itemize}
