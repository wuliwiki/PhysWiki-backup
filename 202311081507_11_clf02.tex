% Clifford 代数的基本运算
% license Xiao
% type Tutor


本节利用集合语言,介绍Clifford代数上的二元线性运算\footnote{本文参考Jie Peter《代数学讲义》}。我们也会证明,更改线性空间的正交基并不会改变运算的形式。
\subsubsection{外积与正交基}
\begin{definition}{外积}
给定Clifford代数$\mathrm {Cl(X,R,s)}$,对于$A,B\in 2^x$,定义
\begin{equation}
A \wedge B=\left\{\begin{aligned}
A B,\quad& A \cap B=\varnothing \\
0,\quad& A \cap B \neq \varnothing~,
\end{aligned}\right.
\end{equation}
并称之为外积(outer product or exterior product)或楔积(wedge product)。我们可以通过线性性将该运算拓展到任意元素之间的外积。
\end{definition}

在上一节,我们说过,集合语言的阐述实际上是指定了线性空间的正交基。正交性体现为CLifford积的反对称性。因而对于正交基$\{\mathrm {e_i}\}$,我们有$\mathrm{e_i\wedge e_j=-e_j\wedge e_i}$。所以对线性空间的任意两个向量作外积,反对称性亦能满足。

不仅如此,基底之间的正交关系不随线性变换而改变。例如某线性空间下有两组正交基$\{e_i\},\{\theta_i\}$,且$\mathrm {\theta_1=a^ie_i,\theta_2=b^ie_i}$,那么由线性性我们可以展开得到后一组基的Clifford积:
\begin{equation}
\left\{\begin{array}{l}
\theta_1 \theta_2=\left(a^i e_i\right) \wedge\left(b^j e_j\right)+\sum_k\left(a^k b^k s(e_k)\right) \\
\theta_2 \theta_1=\left(b^j e_j\right) \wedge\left(a^i e_i\right)+\sum_k\left(b^k a^k s(k)\right)
\end{array}\right.~,
\end{equation}
$a^i,b^i$是过渡矩阵里的两个列向量。由于从标准正交基到标准正交基的矩阵为正交矩阵,因而这两个列向量正交。
