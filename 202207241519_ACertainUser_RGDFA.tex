% 简单刚体系统的静力学分析
% 静力学 刚体 系统 受力分析

% 简单刚体系统的静力学分析
本文简要介绍分析处于静力平衡的简单刚体系统的受力的方法.\footnote{本文参考了张娟著《理论力学》}

\subsection{轻绳与轻杆模型}
\subsubsection{轻绳}
\begin{figure}[ht]
\centering
\includegraphics[width=4cm]{./figures/RGDFA_1.png}
\caption{轻绳} \label{RGDFA_fig1}
\end{figure}
拉紧的轻绳只能受拉而不能受压

拉紧的轻绳向与其相连的刚体提供拉力,但不能提供压力(轻绳不能被压缩);如果轻绳被拉直,那么轻绳提供的拉力方向总是平行于绳

\subsubsection{轻杆}
\begin{figure}[ht]
\centering
\includegraphics[width=4cm]{./figures/RGDFA_2.png}
\caption{轻杆} \label{RGDFA_fig2}
\end{figure}
轻杆可以受压或受拉,同时轻杆提供的力的方向也不一定平行于轻杆.

\subsection{常见约束条件}
\subsubsection{接触面 Surface Constraint}
\begin{figure}[ht]
\centering
\includegraphics[width=5cm]{./figures/RGDFA_3.png}
\caption{接触面模型1} \label{RGDFA_fig3}
\end{figure}
\begin{figure}[ht]
\centering
\includegraphics[width=4cm]{./figures/RGDFA_4.png}
\caption{接触面模型2} \label{RGDFA_fig4}
\end{figure}
放置于平坦接触面的刚体受一个垂直于公切面的支持力.如果接触面粗糙,还可以提供一个平行于接触面的静摩擦力.
\subsubsection{小车模型 Roller Support}
\begin{figure}[ht]
\centering
\includegraphics[width=5cm]{./figures/RGDFA_5.png}
\caption{小车模型} \label{RGDFA_fig5}
\end{figure}
小车可以为与其相连的杆提供一个垂直接触面向上的支持力,但是不能提供一个垂直表面向下的拉力;如果接触面是粗糙的,那还能提供一个平行于接触面的静摩擦力.
\subsubsection{铰链模型 Fixed Hinge}
\begin{figure}[ht]
\centering
\includegraphics[width=5cm]{./figures/RGDFA_6.png}
\caption{铰链模型} \label{RGDFA_fig6}
\end{figure}
铰链可以为与其相连的杆提供一个任意方向的力,但是不能提供力偶.在初步的受力分析中,力的方向不能确定,因此可以先记为两个互相垂直的分力.
\subsubsection{钉子模型 Fixed Rod}
\begin{figure}[ht]
\centering
\includegraphics[width=4cm]{./figures/RGDFA_7.png}
\caption{钉子模型} \label{RGDFA_fig7}
\end{figure}
杆钉入墙中的部分可以提供一个任意方向的力与一个力偶.同铰链一样,力的方向无法立刻确定,先记为两个互相垂直的分力.

\subsection{主动力与约束力}
先简要定义约束力与主动力:
\begin{figure}[ht]
\centering
\includegraphics[width=5cm]{./figures/RGDFA_8.png}
\caption{主动力与约束力}} \label{RGDFA_fig8}
\end{figure}
\begin{itemize}
\item 约束力:由于约束条件而产生的力,例如桌面对物体产生的支持力,绳子对物体提供的拉力等
\item 主动力:由于其他外界因素而产生的力,例如重力、外加的压力等
\end{itemize}

\subsection{受力分析}
总体思路是系统法与隔离法,即先分析系统的整体受力,再逐个分析系统内各个组件的受力.分析系统受力不是一蹴而就,经常是一个反复试错、修正的过程.
\begin{itemize}
\item 先把系统视作一个整体,进行受力分析.此时,系统中各刚体之间的约束力是内力,均可被忽略
\item 随后逐个分析系统内各个组件的受力.一般而言,先分析受到主动力的组件,以及只受两个力的组件(由二力平衡定理立刻得知该两力等大、反向、作用点共线)
\end{itemize}

在分析一个组件时,标注主动力,
\begin{figure}[ht]
\centering
\includegraphics[width=5cm]{./figures/RGDFA_12.png}
\caption{“标注主动力”} \label{RGDFA_fig12}
\end{figure}
并根据约束条件标注约束力(注意到,此时系统中其他组件对该组件的约束力不能再被忽视),
\begin{figure}[ht]
\centering
\includegraphics[width=5cm]{./figures/RGDFA_13.png}
\caption{“标注约束力”} \label{RGDFA_fig13}
\end{figure}
最终根据刚体平衡条件计算,并得到各个约束力.

根据牛顿第三定律,若A对B施加约束力,那么B对A也会施加一个等大反向的约束力.标注组件的约束力时必须注意到这一点.同时,若A对B施加的约束力已被确定,那么B对A施加的约束力也随之确定.

\subsection{刚体的平衡条件}
\pentry{刚体的静力平衡\upref{RBSt}}
对于一个刚体,处于平衡的条件是合力为0,且力矩和为0.

对于合力,一般把所有力分解至水平与竖直方向,在这两个方向上的合力分别为0.力是有方向的,在这种平面情况下,以正负号代表方向.可选取水平向右、垂直向上为正方向.
\begin{equation}
\sum F_x=0, \sum F_y=0
\end{equation}
如果一个力的计算结果是负数,那么说明力的实际方向与假定的相反.

对于力矩和,一般选取未知力较多的点作为参考点,因为作用点在该点上的未知力关于该点的力矩均为0.同时,计算力矩时,也是一般把力分解至水平与竖直方向,分别计算力矩.力矩也是有方向的,在这种平面情况下,以正负号代表方向.可使用右手法则确定力矩的正负:拇指放在参考点,四指指向力的方向.拇指向纸面内为正,朝纸面外为负.
\begin{equation}
\sum M=0
\end{equation}

\begin{example}{框架结构的受力}
\begin{figure}[ht]
\centering
\includegraphics[width=5cm]{./figures/RGDFA_9.png}
\caption{框架结构} \label{RGDFA_fig9}
\end{figure}
如图,杆的重力均不计,分析框架的受力

先分析系统的整体受力.但此时未知力太多,无法得到有价值的信息.

\begin{figure}[ht]
\centering
\includegraphics[width=5cm]{./figures/RGDFA_10.png}
\caption{AB杆受力} \label{RGDFA_fig10}
\end{figure}

再分析AB杆的受力.显然,AB杆只受两个力(分别由两个铰链提供),根据二力平衡定理,两力等大、反向、作用点共线.

\begin{figure}[ht]
\centering
\includegraphics[width=5cm]{./figures/RGDFA_11.png}
\caption{BC杆受力} \label{RGDFA_fig11}
\end{figure}
再分析BC杆的受力.注意根据牛顿第三定律,B处力的方向已经可以确定.

最后,根据各组件力的平衡条件列方程组,即可解得$\bvec F_A, \bvec F_B, \bvec F_C$.例如,在BC杆中,可列出平衡方程:

x方向合力为0: $F_{Cx}-P-F_{Bx}=0$

y方向合力为0: $F_{Cy}-P-F_{By}=0$

关于C点的力矩和为0: $P \cdot l_{CP} + F_{By} \cdot l_{CB} = 0$
\end{example}

\begin{exercise}{“反重力积木”?}
试受力分析该“反重力积木”,并说明其实它并没有反重力.\footnote{图片来自网络}
\end{exercise}
