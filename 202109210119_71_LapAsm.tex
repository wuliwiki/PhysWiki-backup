% 拉普拉斯方法

\pentry{极值\upref{DerMax}, 积分换元\upref{IntCV}, 高斯积分\upref{GsInt}, 渐近展开\upref{Asympt},  Gamma函数\upref{Gamma}}

在分析数学中,拉普拉斯方法是一种计算含参数积分的渐近展开式的办法. 所考察的积分一般具有如下形式:
\begin{equation}\label{LapAsm_eq1}
I(s)=\int_a^b\psi(x)\E^{-s\Phi(x)}\,\dd x
\end{equation}
其中 $s$ 是正的实参数. 要考察当 $s\to+\infty$ 时积分的行为. 拉普拉斯方法背后的想法很简单: 如果函数 $\Phi$ 在某点处达到极大值, 那么当 $s$ 很大时, 只有极大值点附近的贡献才比较可观, 其余部分相比起来都要小得多.

\subsection{Watson 引理}
拉普拉斯方法基于 Watson 引理. 它本身也是很有用的.

\begin{lemma}{Watson 引理}
设实数 $\alpha>0$, 函数 $\phi:X\to Y$ 在闭区间 $[0,T]$ 上连续, 且当 $x\to0$ 时有渐近展开式
\[
  \phi(x)\sim c_0+c_1x+c_2x^2+\cdots
\]
则当 $s\to+\infty$ 时, 含参变量积分
\[
  I(s)=\int_0^T \phi(x)\E^{-s x^\alpha}\,\dd x
\]
有渐近展开式
\[
  I(s)\sim\sum_{n=0}^\infty \frac{c_n}{\alpha}\Gamma\left(\frac{n+1}{\alpha}\right)s^{-(n+1)/\alpha}.
\]
\end{lemma}

证明是直接的计算 固定一个正整数 $n$,按照渐近展开的定义, 存在正数 $\delta >0$ 使得当 $0\leq x<\delta$ 时, $\phi(x)-(c_0+...+c_nx^n)=O(x^{n+1})$. 邻域$B(0,\delta)$外的点 $x\in[0,b]$ 对积分的贡献是可以忽略的:
\[
  \left|\int_\delta^T\phi(x)\E^{-sx^\alpha}\,\dd x\right|
\leq\max_{x\in[0,T]}|\phi(x)|\E^{-s\delta^\alpha}.
\]
而在区间 $[0,\delta]$ 上, 通过换元可得
$$
\begin{aligned}
\int_0^\delta x^k\E^{-tx^\alpha}\,\dd x
&=\frac{s^{-(k+1)/\alpha}}{\alpha}\int_0^{s\delta^\alpha}t^{-(k+1)/\alpha-1}\E^{-t}\,\dd t\\
&=\frac{s^{-(k+1)/\alpha}}{\alpha}\Gamma\left(\frac{k+1}{\alpha}\right)+O(\E^{-s\delta^\alpha/2}).
\end{aligned}
$$
于是就有
$$
\begin{aligned}
I(s)
&=\int_0^\delta\phi(x)\E^{-sx^\alpha}\,\dd x+O(\E^{-s\delta^\alpha})\\
&=\sum_{k=0}^n\frac{c_n}{\alpha}\Gamma\left(\frac{k+1}{\alpha}\right)s^{-(k+1)/\alpha}+O(s^{-(n+2)/\alpha}).
\end{aligned}
$$
这就是所求的结果.

\subsection{拉普拉斯方法}
拉普拉斯方法是将 Watson 引理用于\autoref{LapAsm_eq1} 得到的. 作出如下假设:

\begin{enumerate}
\item 函数 $\Phi,\psi:X\to Y$ 都是光滑函数.

\item 函数 $\Phi(x)$ 在积分区间$I=[a,b]$内部仅有一个严格的极小值点 $x_0\in(a,b)$,同时$\Phi''(x_0)>0$, 且 $x_0$ 是整个区间上的最小值点.

\item \autoref{LapAsm_eq1} 对于 $s=s_0$ 绝对收敛.
\end{enumerate}

于是 $\Phi'(x_0)=0$. 从泰勒公式知道, 有一正数 $\delta>0$ ,使得在邻域 $B(x_0,\delta)$ 上, 
\[
  \Phi(x)=\Phi(x_0)+\frac{1}{2}\Phi''(x_0)(x-x_0)^2+\varphi(x)(x-x_0)^3
\]

其中函数 $\phi$ 有界. 当 $s$ 很大时, 邻域 $B(x_0,\delta)$ 以外的积分的贡献比 $s$ 的任何负幂衰减得都快:根据假定 2,当 $x\in I\setminus B(x_0,\delta)$ 时, 存在正数 $\epsilon_0>0$ 使得 $f(x)<f(x_0)-\epsilon_0$, 从而邻域 $B(x_0,\delta)$ 之外的积分估计为
\[
  \left|\int_{I\setminus B(x_0,\delta)}\psi(x)\E^{-s\Phi(x)}\,\dd x\right|
  \leq \E^{-\epsilon_0(s-s_0)}\int_a^b|\psi(x)|\E^{-s_0\Phi(x)}\,\dd x
  =O(\E^{-\epsilon_0s})
\]
而在邻域 $B(x_0,\delta)$ 内, 可作如下换元: $y^2=\Phi(x)-\Phi(x_0)$, 即
\[
  y=\sqrt{\frac{\Phi''(x_0)}{2}}(x-x_0)\left(1+\frac{2}{\Phi''(x_0)}(x-x_0)\varphi(x)\right)^{1/2}.
\]
令 $\alpha,\beta$分别等于$\pm\sqrt{\Phi(x_0\pm\delta)-\Phi(x_0)}$, 又设有泰勒展开
\begin{equation}\label{LapAsm_eq2}
\tilde{\psi}(y):=\psi(x(y))\frac{\dd}{\dd x}[x(y)]\sim c_0+c_1y+...,y\to 0
\end{equation}
(这里显然有 $c_0=\psi(x_0)\sqrt{2/\Phi''(x_0)}$), 则
$$
\begin{aligned}
\int_{B(x_0,\delta)}\psi(x)\E^{-s\Phi(x)}\,\dd x
&=\E^{-s\Phi(x_0)}\int_{\alpha}^{\beta}\psi(y)\E^{-sy^2}\,\dd y\\
&\sim\E^{-s\Phi(x_0)}\int_{0}^{c}[\tilde \psi(y)+\tilde \psi(-y)]\E^{-sy^2}\,\dd y.
\end{aligned}
$$
应用 Watson 引理, 最后终于得到\autoref{LapAsm_eq1} 的渐近展开:
\begin{equation}\label{LapAsm_eq3}
I(t)\sim \E^{-s\Phi(x_0)}\sum_{n=0}^\infty c_{2n}\Gamma\left(n+\frac{1}{2}\right)s^{-n-1/2},
\end{equation}
其中 $c_n$ 由\autoref{LapAsm_eq2} 给出. 特别地, 由于 $\Gamma(1/2)=\sqrt{\pi}$, \autoref{LapAsm_eq1} 渐近公式的首项是
\[
I(s)=\E^{-s\Phi(x_0)}\left(As^{-1/2}+O(s^{-1/2})\right).
\]
其中$A=\sqrt{\frac{2\pi}\psi(x_0){s\Phi''(x_0)}}$

\subsection{斯特林公式}
拉普拉斯方法最基本的应用就是导出 $\Gamma$ 函数的\textbf{斯特林公式(Stirling formula)}. 按照定义,
$$
\Gamma(t+1)
=\int_0^\infty \E^{-x}x^tdx
=t^{t+1}\int_0^\infty \E^{-t(x-\log x)}dx.
$$
最后一步用到换元 $x\to tx$. 最后这个积分刚好符合上一小节所要求的三条假设:这里 $\phi(x)\equiv1$,$f(x)=-(x-\log x)$. 于是 $x=1$ 是 $f(x)$ 唯一的极大值点,同时也是唯一的最大值点.代入\autoref{LapAsm_eq3}, 就得到渐近公式
$$
\Gamma(t+1)\simeq\sqrt{2\pi t}\left(\frac{t}{\E}\right)^t.
$$
\autoref{LapAsm_eq3} 还给出了更精细的渐近级数:
$$
  \Gamma(t+1)
  =\sqrt{2\pi t}\left({t\over \E}\right)^t
  \left(
   1
   +{1\over12t}
   +{1\over288t^2}
   -{139\over51840t^3}
   -{571\over2488320t^4}
   + \cdots
  \right).
$$
毫无疑问这是一个可以对大的 $n$ 来近似计算 $n!$ 的公式.
