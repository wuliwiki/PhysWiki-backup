% C 和 C++ 的整数(笔记)
% c++|int|long|整数|二进制|八进制|十六进制|转换|溢出

\pentry{C++ 基础\upref{Cpp0}}

\footnote{参考 C++ Primer\cite{CppPr}。}本文的\textbf{模(modulo)}运算都是指 $M$ 加减整数个 $N$ 后使结果范围在 $0$ 到 $N-1$ 之间(包含), $N$ 只能是正整数。 本文中 $n$ 表示整数类型的比特数。

\subsection{无符号整数}
\begin{itemize}
\item 取值范围为 $0$ 到 $2^n-1$
\item 如果溢出就把二进制的高位截去, 也就是模 $2^n$
\end{itemize}

\subsection{有符号整数}
\begin{itemize}
\item 几乎所有现代 cpu 使用 2 的补表示有符号整数, 此时取值范围为 $-2^{n-1}$ 到 $2^{n-1}-1$
\item 有符号整数运算溢出结果\textbf{无定义}, 虽然大部分机器会 wrap around, 但不要依赖该行为。
\item 不要用有符号整数表示 bit field, 一律用无符号。
\end{itemize}

\subsection{转换规则}
整数之间的转换规则:
\begin{itemize}
\item 超出范围的任何类型整数值转换为无符号整型时, 将其模 $2^n$ ($n$ 是目标类型的比特数), 例如 $n = 8$ 时 -1 转换为 255。
\item 超出范围的任何值转换为有符号整型, 结果无定义。
\item 如果你想要把一个变量转换为另一个同长度变量并保持每个 bit 不变, 可以用 \verb|memcpy()| 函数。
\end{itemize}
其他基本类型转换规则:
\begin{itemize}
\item 其他类型转换为 \verb|bool|: 0 变为 \verb|false|, 否则变为 \verb|true|
\item \verb|bool| 转换为其他类型: \verb|true| 变为 1, \verb|false| 变为 0
\item 浮点类型转换为整数类型: 向 0 取整
\item 整型转为浮点: 如果位数太多会不精确
\item \verb|signed| 和 \verb|unsigned| 整数之间比较大小或相等或时(如果开了 \verb|-Wall|, 编译器会给出警告), 前者会先转换为后者。 这就导致负数有可能大于正数。做加减乘除时也一样, 这叫做 \textbf{usual arithmetic conversions}。
\end{itemize}

\subsection{2 的补}
有符号整数类型的负数在内存中的二进制表示常采用 \textbf{2 的补(2's complement)}储存(例如 x86-64\upref{x86} 架构)。 若采用 2 的补, 以下几点成立
\begin{itemize}
\item 一个整数和它的相反数相加等于 $2^n$
\item 范围内最大的整数加上 1 等于范围内最小的整数, 即加 1 后取相反数(如 $n = 8$ 时 $127 + 1 = -128$)
\item 若保持 bit 不变, 无符号类型变为有符号类型只需减去 $2^n$, 反之则加上 $2^n$。 例如 $n = 8$ 时无符号 156 和有符号 -100 的各个 bit 相同。
\end{itemize}

也有其他表示负数的方法, 例如最左边的 bit 表示符号, 剩余表示大小。

数学上, 可以将无符号的整数类型及其加法构成一个 $2^n$ 元循环群(见\autoref{ex_Group_2}~\upref{Group})。 若采用 2 的补, 由于群元素可以用任何符号, 有符号整数类型可以看作将无符号整数类型的后一半元素减去 $2^n$, 而群运算保持不变。

\subsection{Integer Literal}
\footnote{参考 \href{https://en.cppreference.com/w/cpp/language/integer_literal}{cppreference}。}这里讨论 c++11 标准。 literal 中的字母不区分大小写, \verb|0| 开头表示 8 进制,  \verb|0x| 开头表示 16 进制, 数字后面可以加 \verb|u|, 以及 \verb|l| 或 \verb|ll| 的一个(可以是大写)。 literal 的类型会根据数值的大小而变化。

\begin{itemize}
\item 最小是 \verb|int|, 最大是 \verb|long long int| 每一种前面都可以加 \verb|unsigned|
\item 有后缀 \verb|u| 必定是 \verb|unsigned|
\item 没有 \verb|u| 的十进制必定是 signed, 其他进制可以是 signed 或 unsigned (首选 signed)
\item 后缀 \verb|l| 和 \verb|ll| 分别规定类型至少为 \verb|long| 和 \verb|long long|, 如果数值超出就不断升级
\end{itemize}

\subsection{explicit 类型转换}
\begin{itemize}
\item \verb|static_cast<新类型>(变量)| 和 \verb|(新类型)变量| 效果相同, 但有一些转换不允许。
\item \verb|reinterpret_cast<新类型&>(变量)| 和 \verb|reinterpret_cast<新类型*>(变量指针)| 并不会检查 \textbf{strict aliasing rule}, 需要用户自己保证。
\item \textbf{strict aliasing rule}: 不允许不同类型的指针或索引指向同一个。
\end{itemize}
