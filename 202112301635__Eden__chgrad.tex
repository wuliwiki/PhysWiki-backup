% 带电粒子的辐射
% 带电粒子的辐射|李纳维谢尔势

\pentry{李纳维谢尔势\upref{LWP}}

我们继续使用自然单位制,令 $\mu_0=\epsilon_0=c=1$ 来简化表达.依照习惯,上下标使用希腊字母如 $\mu, \nu$ 时,取值范围为 $\{0, 1, 2, 3\}$;使用拉丁字母如 $i, j$ 时,取值范围为 $\{1, 2, 3\}$.约定闵氏时空度规为 $(-1,1,1,1)$.

根据李纳维谢尔势公式,运动电荷将向外辐射电磁场,从而产生能量损耗.下面我们将定量地推导由运动电荷带来的空间的电磁场分布.

\begin{equation}
\begin{aligned}
\phi(\bvec r,t)=\frac{q}{|\bvec r-\bvec r'|-\bvec v'\cdot (\bvec r-\bvec r')}\\
\bvec A(\bvec r,t)=\frac{q\bvec v'}{|\bvec r-\bvec r'|-\bvec v'\cdot (\bvec r-\bvec r')}
\end{aligned}
\end{equation}

其中 $\bvec r',\bvec v'$ 是 $t'$ 时刻粒子的位置和速度,满足 $t-t'=|\bvec r-\bvec r'|$.下面将用 $\bvec R$ 来表示 $\bvec r-\bvec r'$,表示 $t'$ 处电荷位置到当前位置的位矢.那么电磁势可以写为
\begin{equation}
\begin{aligned}
\phi(\bvec r,t)=\frac{q}{R-\bvec v'\cdot \bvec R}\\
\bvec A(\bvec r,t)=\frac{q\bvec v'}{R-\bvec v'\cdot \bvec R}
\end{aligned}
\end{equation}

现在只要利用 $\bvec E=-\nabla \phi-\frac{\partial \bvec A}{\partial t},\bvec B=\nabla\times \bvec A$ 就可以计算空间的电磁场分布了.