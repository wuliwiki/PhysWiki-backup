% 位移电流、广义安培环路定理
% 位移电流|麦克斯韦修正|麦克斯韦方程组|安培环路定理|旋度|散度

\begin{issues}
\issueDraft
\end{issues}

\pentry{安培环路定理\upref{AmpLaw}, 电荷守恒、电流连续性方程\upref{ChgCsv}}

\textbf{位移电流(displacement current)}是电动力学中的一个物理量.

定义为
\begin{equation}
\bvec j_d = \epsilon_0 \pdv{\bvec E}{t}
\end{equation}

为什么定义这么一个量呢?

\subsection{广义安培环路定理}

注意广义安培环路定理是基本假设之一, 所以以下并不是推导, 而是一种把安培环路定理拓展到非静电学情况的可能思路.

在静电学问题(引用未完成)中, 安培环路定理为
\begin{equation}\label{DisCur_eq1}
\curl\bvec B = \mu_0\bvec j
\end{equation}
这要求等式右边的矢量场必须是一个无散场. 在静电学问题中 $\bvec j$ 的确是无散场($\pdv*{\rho}{t} = 0$), 然而若要拓展到一般情况, 我们需要给\autoref{DisCur_eq1} 右边加上一个修正项, 使等式右边的散度恒为零.

由电荷连续性方程
\begin{equation}
\div \bvec j + \pdv{\rho}{t} = 0
\end{equation}
使用电场高斯定律
\begin{equation}
\div \qty(\bvec j + \epsilon_0\pdv{\bvec E}{t}) = 0
\end{equation}
可见括号中恒为无散场. 所以不妨猜测
\begin{equation}
\curl\bvec B = \mu_0\bvec j + \epsilon_0\mu_0\pdv{\bvec E}{t}
\end{equation}
这就是说, 和法拉第电磁感应\upref{FaraEB}相似, 变化的电场也会产生磁场, 并且和电流产生的磁场叠加得到总磁场. 该式称为\textbf{广义安培环路定理}或者\textbf{麦克斯韦—安培公式}.
