% 洛朗级数
% keys 复变函数|留数|residue|全纯函数|级数

\pentry{全纯函数,柯西积分定理\upref{CauGou}}

如果一个单复变函数处处解析,那它处处都可以展开成泰勒级数的形式.但我们常遇到的很多函数并不是处处解析的,从而在非解析点处无法泰勒展开.但是我们依然有一种类似级数的办法来展开其中一些函数,这就是\textbf{洛朗级数}.

\begin{definition}{洛朗级数}
形如
\begin{equation}
f(z)=\sum\limits_{n=-\infty}^{\infty} a_n(z-c)^n
\end{equation}
的单复变函数,被称为一个关于点$c\in\mathbb{C}$的\textbf{洛朗级数(Laurent series)},其中$a_n$都是常数.


\end{definition}

容易看到,洛朗级数就是将泰勒级数的幂次拓展到了负的情况.在$z\neq c$的点处,$f(z)$可以看成两个级数$\sum\limits_{n=0}^{\infty} a_n(z-c)^n$和$\sum\limits_{n=1}^{\infty} a_{-n}\qty(\frac{1}{z-c})^n$的和;在$z=c$处则是发散的.

$\sum\limits_{n=0}^{\infty} a_n(z-c)^n$称为$f$的\textbf{正则部分(normal part)},而剩下的$\sum\limits_{n=1}^{\infty} a_{-n}\qty(\frac{1}{z-c})^n$则称为$f$的\textbf{主要部分(principal part)}.两个术语可以分别简称为\textbf{正则部}和\textbf{主部}.



















