% 平面波的正交归一
% keys 平面波|矢量空间|傅里叶变换|正交归一

\begin{issues}
\issueOther{重复词条, 待删除}
\end{issues}

\pentry{函数空间\upref{FunSpc}, 傅里叶变换与矢量空间\upref{FTvec}}

\subsection{一维情况}
要计算一维平面波在实数域的正交归一积分, 由傅里叶变换已知
\begin{equation}
\braket{k'}{k} = \delta(k - k')
\end{equation}
所以
\begin{equation}
\int_{-\infty}^{\infty} \E^{-\I k' x} \E^{\I k x} \dd{x} = 2\pi\braket{k'}{k} = 2\pi\delta(k - k')
\end{equation}
\begin{equation}
\begin{aligned}
&\quad \int_{-\infty}^{\infty} \sin(\I k' x) \sin(\I k x) \dd{x}\\
&= \qty(\I\sqrt{\frac{\pi}{2}}\bra{k'} - \I \sqrt{\frac{\pi}{2}}\bra{-k'})\qty(-\I\sqrt{\frac{\pi}{2}}\ket{k} + \I \sqrt{\frac{\pi}{2}}\ket{-k})\\
&= \pi\delta(k - k')
\end{aligned}
\end{equation}
\begin{equation}
\begin{aligned}
&\quad \int_{-\infty}^{\infty} \cos(\I k' x) \cos(\I k x) \dd{x}\\
&= \qty(\sqrt{\frac{\pi}{2}}\bra{k} + \sqrt{\frac{\pi}{2}}\bra{-k})\qty(\sqrt{\frac{\pi}{2}}\ket{k} + \sqrt{\frac{\pi}{2}}\ket{-k})\\
&= \pi\delta(k - k')
\end{aligned}
\end{equation}
要证明 $\sin(kx)$ 和 $\cos(kx)$ 正交,
\begin{equation}\ali{
&\quad \int_{-\infty}^{\infty} \sin(k'x)^*\cos(kx) \dd{x}\\
& = \int_{-\infty}^{\infty}\qty(\I\sqrt{\frac{\pi}{2}}\bra{k'} - \I \sqrt{\frac{\pi}{2}}\bra{-k'}) \qty(\sqrt{\frac{\pi}{2}}\ket{k} + \sqrt{\frac{\pi}{2}}\ket{-k}) \dd{x} = 0
}\end{equation}
所以实数域上完备正交归一的实波函数分别是
\begin{equation}
\frac{1}{\sqrt{\pi}} \cos(kx) \qquad
\frac{1}{\sqrt{\pi}} \sin(kx) \qquad (k \geqslant 0)
\end{equation}
如果只考虑 $[0, \infty)$, 则是
\begin{equation}
\frac{2}{\sqrt{\pi}} \cos(kx) \qquad
\frac{2}{\sqrt{\pi}} \sin(kx) \qquad (k \geqslant 0)
\end{equation}

\subsection{附加相位}
现在再来看波函数中有相位 $\phi(k)$ 的情况, 例如
\begin{equation}
\int_{-\infty}^{\infty} \frac{1}{\sqrt{\pi}}\exp[-\I k' x + \I \phi(k')] \frac{1}{\sqrt{\pi}}\exp[\I k x + \I \phi(k)] \dd{x}
\end{equation}
结果会不会仍然等于 $\delta(k - k')$ 呢? 我们利用 $\exp(a + b) = \exp a\exp b$ 和已知的归一化积分得到结果为 $\exp[\I\phi(k) - \I\phi(k')] \delta(k - k')$. 如果我们假设 $\phi(k)$ 是连续的, 那么结果就是 $\delta(k - k')$.

同理可得, 以上所有正交归一基底中给波函数添加相移 $\phi(k)$ (连续函数), 也满足正交归一.
