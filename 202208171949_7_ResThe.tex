% 留数定理
% keys 复变函数|residue|柯西积分定理|洛朗级数|围道积分|约尔当引理|Jordan 引理|Jordan's Lemma|若尔当引理|若当引理

\pentry{洛朗级数\upref{LaurSr}}

留数定理是用来计算围道积分的工具,也可以借用围道积分的性质来计算实函数的积分.

\subsection{定理的导出}

我们首先讨论最简单的洛朗级数 $f(z)=1/z$ 的围道积分.

\begin{example}{}\label{ResThe_ex1}
考虑 $f(z)=1/z$ 的围道积分.

令积分路径为 $\Gamma(t)=\rho\E^{\I t}$,其中 $\rho$ 是一个常数.$\Gamma$ 就是一个绕着一个半径为 $\rho$ 的圆的逆时针路径.$f$ 沿 $\Gamma$ 的围道积分就是:
\begin{equation}\label{ResThe_eq1}
\begin{aligned}
\int_{\Gamma}\frac{1}{z}\dd z&=\int^{2\pi}_0\frac{1}{\rho\E^{\I t}}\cdot\frac{\dd\rho\E^{\I t}}{\dd t}\dd t\\
&=\int_0^{2\pi}\frac{1}{\rho\E^{\I t}}\cdot (\I \rho \E^{\I t})\dd t\\
&=\int_0^{2\pi}\I\dd t\\
&=2\pi\I
\end{aligned}
\end{equation}

对于任意逆时针环绕原点的路径 $P(t)$,我们可以求 $\int_{\Gamma}\frac{1}{z}\dd z-\int_{P}\frac{1}{z}\dd z$,其中 $\Gamma$ 完全包裹在 $P$ 中.这相当于求 $\Gamma$ 和 $P$ 之间区域的围道积分.这个中间区域是不包含原点的,因此由于 $1/z$ 在原点之外处处解析,由\textbf{柯西积分定理}\upref{CauGou}知,中间区域的围道积分为 $0$.从而我们可以推知,$\int_{P}\frac{1}{z}\dd z=2\pi\I$.

更一般地,对于任意环绕原点的路径 $P$,有 $\int_{P}\frac{1}{z}\dd z=\pm 2\pi\I$,其中 $P$ 为逆时针时取正号,反之取负号.

\end{example}

接着是幂次更低的洛朗级数的围道积分.

\begin{example}{}\label{ResThe_ex2}
考虑 $f(z)=1/z^n$ 的围道积分,其中整数 $n>1$.

同样令积分路径为 $\Gamma(t)=\rho\E^{\I t}$,其中 $\rho$ 是一个常数.$f$ 沿 $\Gamma$ 的围道积分就是:

\begin{equation}
\begin{aligned}
\int_{\Gamma}\frac{1}{z^n}\dd z&=\int^{2\pi}_0\frac{1}{(\rho\E^{\I t})^n}\cdot\frac{\dd\rho\E^{\I t}}{\dd t}\dd t\\
&=\int_0^{2\pi}\frac{1}{(\rho\E^{\I t})^n}\cdot (\I \rho \E^{\I t})\dd t\\
&=\int_0^{2\pi}\frac{\I}{\rho^{n-1}}\E^{(1-n)\I t}\dd t\\
&=\frac{1}{(1-n)\rho^{n-1}}\E^{(1-n)\I t}\mid^{t=2\pi}_{t=0}\\
&=0
\end{aligned}
\end{equation}

和\autoref{ResThe_ex1} 一样地,可以将结论推广为,对于任意积分路径 $P$,都有
\begin{equation}
\int_P\frac{1}{z^n}\dd z=0
\end{equation}

\end{example}

而洛朗级数的正则部分是一个泰勒级数,其围道积分处处为 $0$.结合\autoref{ResThe_ex1} 和\autoref{ResThe_ex2} ,我们可以得到以下定理:

\begin{definition}{留数}

取洛朗级数 $f(z)=\sum\limits_{n=-\infty}^{\infty} a_n(z-c)^n$,定义 $a_{-1}$ 为 $f$ 在 $c$ 处的\textbf{留数(residue)},记为 $\opn{Res}[f, c]$,或者 $\opn{Res}_cf$.

\end{definition}

\begin{theorem}{留数定理(单个极点)}\label{ResThe_the1}
洛朗级数 $f(z)=\sum\limits_{n=-\infty}^{\infty} a_n(z-c)^n$ 沿闭合路径 $\Gamma$ 逆时针绕点 $c$ 一周的围道积分为
\begin{equation}
\int_\Gamma f(z)\dd z=2a_{-1}\pi\I
\end{equation}
这一结果也可以表述为
\begin{equation}
\int_\Gamma f(z)\dd z=2\pi\I\opn{Res}_cf
\end{equation}

\end{theorem}

有时候回路中会有多个极点,此时我们要做的就是把这些极点的留数求和:

\begin{theorem}{留数定理}

如果连续复变函数 $f(z)$ 在闭合路径 $\Gamma$ 中有多个极点,那么沿 $\Gamma$ 的回路积分为

\begin{equation}
\int_\Gamma f(z)\dd z=2\pi\I\sum_{c}\opn{Res}_cf
\end{equation}
其中 $c$ 遍历所有包含在 $\Gamma$ 中的极点.


\end{theorem}



\begin{example}{}

考虑函数$f(z)=\E^{\I z^2}$在$x$轴上的积分.

设$C_\rho$为起点在$(\rho, 0)$、终点在$(-\rho, 0)$的逆时针半圆弧路径,则类比\autoref{ResThe_eq1} 有
\begin{equation}
\ali{
    \int_{C_\rho} f(z)\dd z &= \int_{C_\rho} \E^{\I z^2}\dd z\\
    &= \int_0^\pi \E^{\I \rho^2 \E^{2\I t}}\cdot (\I\rho \E^{\I t})\dd t\\
    &= \qty(-\frac{\I}{2\rho})\E^{\I\rho^2\E^{2\I t}}\mid^{t=\pi}_{t=0}
}
\end{equation}

\end{example}









\subsection{留数的计算}



如果能把给定函数 $f$ 在给定点的洛朗展开写出来,我们直接就能得到对应的留数,用于计算围道积分.但那样常常非常麻烦,也没有必要,因为留数只是洛朗级数中的一个系数,通常没必要为了求这一个系数而把整个级数算出来.

\begin{lemma}{}\label{ResThe_lem1}
如果 $f(z)$ 是一个在 $c\in\mathbb{C}$ 处解析的函数,那么 $\opn{Rez}_c\frac{f(z)}{z-c}=f(c)$.
\end{lemma}

\autoref{ResThe_lem1} 的证明很简单,将 $f(z)$ 在 $c$ 处泰勒展开以后再除以 $z-c$ 即可.

利用\autoref{ResThe_lem1} ,我们可以快速计算出很多常见函数的留数.

\begin{example}{}
求 $\frac{1}{z(z-2)}$ 在 $z=2$ 处的留数.

由于 $\frac{1}{z}$ 在 $z=2$ 处解析,因此根据\autoref{ResThe_lem1} ,代入 $f(z)=\frac{1}{z}$ 和 $c=2$ 可知,
\begin{equation}
\opn{Rez}_2\frac{1}{z(z-2)}=\frac{1}{2}
\end{equation}
\end{example}

用证明\autoref{ResThe_lem1} 的方法进行推广,还可以得到更一般的定理:

\begin{theorem}{}\label{ResThe_the2}
如果 $f(z)$ 是一个在 $c\in\mathbb{C}$ 处解析的函数,$k$ 是一个正整数,那么 $\opn{Rez}_c\frac{f(z)}{(z-c)^k}=\frac{f^{(k-1)}(c)}{(k-1)!}$.
\end{theorem}

\textbf{证明}:

由于$f(z)$在$c$处解析,因此可以在$c$处展开为
\begin{equation}
f(z) = \sum_{n=0}^\infty a_n(z-c)^n
\end{equation}
于是
\begin{equation}
\frac{f(z)}{(z-c)^k} = \sum_{n=0}^\infty a_n(z-c)^{n-k} = \sum_{n=-k}^\infty a_{n+k}(z-c)^n
\end{equation}
因此
\begin{equation}
\opn{Rez}_c\frac{f(z)}{(z-c)^k} = a_{k-1}
\end{equation}

而
\begin{equation}
f^{k-1}(c)=(k-1)!a_{k-1}
\end{equation}

\textbf{证毕}.




在很多材料中会把点 $c$ 称为 $f(z)/(z-c)^k$ 的 $k$\textbf{级极点},其中 $f(z)$ 在 $c$ 处解析.

\begin{example}{}
计算 $\opn{Rez}_0 \frac{\E^{2z}}{z^2}$.

在 $0$ 处,$\frac{\E^{2z}}{z^2}$ 有一个二级极点.根据\autoref{ResThe_the2} ,将 $k=2$、$f(z)=\E^{2z}$ 和 $c=0$ 代入,可得:
\begin{equation}
\opn{Rez}_0\frac{\E^{2z}}{z^2}=\frac{(\E^{2z})'\mid_{z=0}}{1!}=2
\end{equation}


\end{example}












