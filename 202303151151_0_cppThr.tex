% C++ 多线程笔记(std::thread)

\begin{itemize}
\item 在 Linux 或其他 POSIX 系统上 \verb|std::thread| 底层使用 pthread。
\item \verb|#include <thread>|, \verb|#include <mutex>|
\item \verb|std::thread th(函数, arg1, arg2, ...)|; 创建一个线程, 调用\verb|函数|(可以是函数指针, 函数对象, lambda), \verb|arg| 是\verb|函数|的变量。
\item \verb|th.join()| 可以让主程序等待某个线程自己退出。
\item 声明一个全局变量 \verb|std::mutex m| 相当于 openmp 的 atomic 操作, 可以避免多个线程操作同一数据。 \verb|m.lock()| 给 mutex 上锁, 如果已经被别人上锁就暂停该线程并等待解锁。 \verb|m.try_lock()| 试图上锁, 如果已经被别人上锁就返回 \verb|false|。 \verb|m.unlock()| 解锁。
\item 为了避免在 \verb|m.lock()| 和 \verb|m.unlock()| 之间发生 throw, 通常不直接调用他们, 而是用 \verb|std::lock_guard<std::mutex> guard(m)| (相当于 lock), 当 \verb|guard| 被 destroy 时, 会自动 unlock。
\item \verb|std::this_thread::sleep_for(std::chrono::seconds(2));| 可以让某个线程暂停。
\end{itemize}

例程(编译时要给 linker 加上 \verb|-l pthread|):
\begin{lstlisting}[language=cpp]
#include <thread>
#include <mutex>
#include <chrono>
using namespace std;

mutex mut;

// A dummy function
void myfun(int id, int *i)
{
    if (id == 1)
        this_thread::sleep_for(chrono::milliseconds(100));
    lock_guard<mutex> guard(mut);
    printf("id = %d, i = %d\n", id, *i);
    int j = *i+1;
    if (id == 2)
        this_thread::sleep_for(chrono::milliseconds(100));
    *i = j;
}

int main()
{
    int i = 0;
    thread th1(myfun, 1, &i);
    thread th2(myfun, 2, &i);
    myfun(0, &i);
    th1.join();
    th2.join();
    return 0;
}
\end{lstlisting}

\addTODO{互斥锁、条件锁、读写锁以及自旋锁分别什么时候用?}
% 参考 https://www.zhihu.com/question/66733477
