% 爱因斯坦场方程
% 引力场|引力|gravity|场方程|Ricci张量|测地线|geodesic|广义相对论|相对论|relativity|时空|spacetime|弯曲|曲率

\pentry{引力的弱场近似\upref{WeakG},尘埃云的能动张量\upref{SRFld},曲率张量场\upref{RicciC}}

\subsection{爱因斯坦张量}

我们直接介绍一个有用的性质,在稍后猜测爱因斯坦场方程的时候我们自然会讨论到它的用处.

先考虑黎曼曲率张量的第二Bianchi恒等式\autoref{RicciC_eq7}~\upref{RicciC}:
\begin{equation}
\partial_\lambda R_{\mu\nu\rho\sigma}+\partial_\rho R_{\mu\nu\sigma\lambda}+\partial_\sigma R_{\mu\nu\lambda\rho}=0
\end{equation}

等式两端同时乘以$g^{\nu\sigma}g^{\mu\lambda}$后,考虑到联络对度量的相容性\footnote{即$\nabla_ag_{ij}=0$.},得到:
\begin{equation}
\begin{aligned}
0&=g^{\nu\sigma}g^{\mu\lambda}\partial_\lambda R_{\mu\nu\rho\sigma}+\partial_\rho R_{\mu\nu\sigma\lambda}+\partial_\sigma R_{\mu\nu\lambda\rho}\\
&=\nabla^\mu R_{\rho}
\end{aligned}
\end{equation}




\subsection{能动张量}

爱因斯坦场方程的引出过程中,我们考虑的是最简单的宏观模型,即\textbf{尘埃云的能动张量}\upref{SRFld}中所介绍的\textbf{理想流体}.对于任何物质,其四动量的各分量,都会随着参考系的不同而有不同取值,因此这些量只能是流形上某种量在具体坐标系中的坐标分量而已.这样一来,在流形上描述能量质量分布的量,就不能是简单的光滑函数,或者说标量场,而只能是更高阶的张量场.能描述理想流体四动量分布的张量,可以是四动量本身,也可以是能动张量,而我们会选择能动张量,这样才能和Ricci曲率张量的阶数吻合.

















