% 恩斯特·策梅洛(综述)
% license CCBYSA3
% type Wiki

本文根据 CC-BY-SA 协议转载翻译自维基百科\href{https://en.wikipedia.org/wiki/Ernst_Zermelo}{相关文章}。

\begin{figure}[ht]
\centering
\includegraphics[width=6cm]{./figures/e3826f3f611894f0.png}
\caption{} \label{fig_Zerme_1}
\end{figure}
恩斯特·弗里德里希·费迪南德·策梅洛(Ernst Friedrich Ferdinand Zermelo,发音:/zɜːrˈmɛloʊ/;德语:[tsɛɐ̯ˈmeːlo];1871年7月27日-1953年5月21日)是德国的逻辑学家和数学家,他的工作对数学基础具有重要影响。他因在发展策梅洛-弗伦克尔公理化集合论以及证明良序定理方面的贡献而闻名。此外,他在1929年关于国际象棋棋手排名的研究,首次描述了一种对偶比较模型,这一方法在多个应用领域中继续产生深远的影响。
\subsection{生活}
恩斯特·策梅洛于1889年毕业于柏林的路易森斯特第高等学校(现为海因里希·施利曼中学)。随后,他在柏林大学、哈雷大学和弗赖堡大学学习数学、物理和哲学。他于1894年在柏林大学完成了博士学位,论文题目为变分法(Untersuchungen zur Variationsrechnung)。策梅洛继续留在柏林大学,成为普朗克的助手,并在其指导下开始研究流体动力学。1897年,策梅洛前往哥廷根大学,这时的哥廷根大学是世界领先的数学研究中心,他于1899年完成了博士后资格论文。

1910年,策梅洛离开哥廷根,受聘为苏黎世大学数学系的教授,直到1916年辞职。他于1926年被授予弗赖堡大学的名誉教授职位,但因不满阿道夫·希特勒的政权,於1935年辞去该职位。在第二次世界大战结束后,应策梅洛的请求,他被重新恢复了弗赖堡的名誉教授职务。
\subsection{集合论研究}
1900年,在巴黎举行的国际数学家大会上,大卫·希尔伯特向数学界提出了著名的希尔伯特问题,这是一个列出23个未解的基础性问题的清单,数学家们应在接下来的一个世纪内进行攻克。第一个问题是集合论中的问题,即由康托尔在1878年提出的连续统假设,在阐述这个问题时,希尔伯特还提到了需要证明良序定理。
\begin{figure}[ht]
\centering
\includegraphics[width=6cm]{./figures/4e5a90def210d0d7.png}
\caption{恩斯特·策梅洛(1953年,弗赖堡) } \label{fig_Zerme_2}
\end{figure}
在希尔伯特的影响下,策梅洛开始研究集合论问题,并于1902年发表了关于超限基数相加的第一篇论文。到那时,他已经发现了所谓的拉塞尔悖论。1904年,他成功地迈出了希尔伯特所建议的关于连续统假设的第一步,证明了良序定理(即每个集合都可以良序)。这一结果使策梅洛声名鹊起,并于1905年被任命为哥廷根大学的教授。他基于幂集公理和选择公理的良序定理证明并未被所有数学家接受,主要因为选择公理是非构造性数学的典型范例。1908年,策梅洛成功地提出了改进版的证明,利用了德德金德对集合“链”的概念,这一版本得到了更广泛的接受;主要是因为在同一年,他还提出了集合论的公理化方案。

策梅洛于1905年开始对集合论进行公理化;尽管他未能证明其公理系统的一致性,但他于1908年仍然发表了他的结果。有关这篇论文的概要以及原始公理和编号,请参见策梅洛集合论的相关文章。

1922年,亚伯拉罕·弗伦克尔和托尔尔夫·斯科勒姆独立地改进了策梅洛的公理系统。由此产生的系统,现在被称为策梅洛-弗伦克尔公理(ZF),是目前最常用的集合论公理化系统。
\subsection{策梅洛的导航问题}
策梅洛导航问题提出于1931年,是一个经典的最优控制问题。该问题涉及一艘船在水域中航行,从起点O出发,前往目的地D。船具有一定的最大速度,我们希望推导出最佳的控制策略,以便在最短的时间内到达D。

在不考虑外部因素如水流和风的情况下,最优控制策略是船始终朝向D航行。此时,船的航迹就是从O到D的线段,这显然是最优的。但如果考虑到水流和风的影响,当施加在船上的合力不为零时,忽略水流和风的控制策略将无法得到最优路径。
