% 等式与不等式(高中)
% keys 方程|不等式|代数基本定理
% license Xiao
% type Tutor

\begin{issues}
\issueDraft
\end{issues}

\pentry{函数\nref{nod_functi},集合\nref{nod_HsSet}}{nod_85e3}

相等和不等关系是从小学阶段就开始接触的基础概念,但由此延伸出的方程、不等式、恒等式、方程组、解等概念,许多人往往会感到不知所谓。人教版初中教材中给出的“方程”定义是“含有未知数的等式称作方程(equation)”\footnote{注意并非含有\textbf{字母}的等式。},给“未知数”的定义则是“方程的求解目标”,这看上去是一种令人迷惑的循环定义,进而造成有些人困惑形如$x=3$的等式是否也是一个方程。到高中阶段,在高中教材中依然没有系统性的澄清这些概念。

因此,很多学生在阅读题目时,对解题任务的理解感到模糊,不清楚解方程、联立方程究竟意味着什么,这种认识上的模糊甚至延续到大学阶段,影响对更复杂概念的掌握和后续学习的进展,很多研究者在使用这些术语时也很混乱。本文旨在解决上面提到的问题。

\subsection{一些相关的基础概念}

下面会先介绍一些基础概念。这些概念的数量很多且前后勾连,且有不少与教材上语焉不详的定义存在出入。这里不要求完整记忆,只需要认真理解并清楚自己脑海中习惯的表达与下面概念的对应即可。

\subsubsection{运算符与关系符}

表示数学运算的符号,如加、减、乘、除及各种函数(例如$\sin,\cos$)等,被称为\textbf{运算符(operator)}\footnote{在数学领域深入研究后,这一概念被称为\textbf{算子(operator)},并且具有更严格的定义,会在\enref{泛函分析}{FnalNt}中学习。在计算机科学、物理等其他领域中也会使用“运算符”这个术语,但其定义可能有所不同。}。在数学学习的基础阶段经常会接触到的五种基本运算符——加、减、乘、除及有理数次的乘方——被归类为\textbf{代数运算符(algebraic operators)}。这些代数运算符专注于基本的算术计算,是构建许多数学表达式的核心部分。

用于表示两个数学元素之间关系的符号称为\textbf{关系符(relation)}\footnote{在一些领域中,如计算机科学,关系符也可以视作一种特殊的运算符,称作“关系运算符”,其运算结果是关系判定的真值。例如,$2=3$的运算结果为$\rm False$,而$2=1+1$的运算结果为$\rm True$。}。例如,“$>$”、“$<$”、“$\leq$”、“$\geq$”、“$\neq$”这些符号称为\textbf{不等号(inequality symbols)},而“$=$”称为\textbf{等号(equality symbol)}。此外,还有许多关系符号,比如:在集合论中,有表示包含关系“$\subset$”和表示属于关系的“$\in$”;在几何中,有表示平行关系的“$\mathrel{/\mskip-2.5mu/}$”和垂直关系的“$\perp$”;在数理逻辑中,有表示等价关系的“$\equiv$”、表示蕴含关系的“$\Rightarrow$”以及表示互为充要条件关系的“$\iff$”等。

\subsubsection{表达式}

由数字、变量和运算符组成的数学符号组合称为\textbf{数学表达式(mathematical expression)},或简称\textbf{表达式(expression)},也叫\textbf{式子}。表达式可以被看作一种“描述工具”,其主要作用在于用符号表示某种数学上的数量关系或状态,而不一定需要得到一个具体的数值。例如,$3x + 2$ 和 $\sin(x)$ 都是表达式,它们描述了一种数量关系或函数的性质,而不是一道必须求解的题目。

表达式可以通过各种数学操作来进行简化和转化,比如合并同类项、约分等,这些操作称为“恒等变形”。在特定情况下,如果知道表达式中变量的具体值,还可以将该值代入表达式,从而计算出一个数值结果。

在初中阶段学习的\textbf{代数式(algebraic expression)}是一个特定类型的数学表达式。代数式仅由代数运算符连接数或字母组成,比如$2x + 3$或$x^2 - 4x + 4$。这也意味着代数式的运算范围相对有限,通常只涉及基本的代数运算,而不会涉及如三角函数或对数等更复杂的运算。可以说,代数式是表达式中的“基础款”。代数式的核心特点在于运算简单且易于处理,是数学学习的入门工具。

\subsection{等式和不等式}

\textbf{等式(equation)}和\textbf{不等式(inequality)}在定义上几乎完全相同,两者的区别主要在于所使用的关系符号不同,以及由此导致的操作规则的变化。由于本章主要讨论定义,为了简化理解,可以先专注于等式的定义。一旦掌握了等式的概念和性质,再推广到不等式就会变得更加容易。

\subsubsection{等式}

在介绍下面的概念之前,有必要先引入一个英语单词“equation”。它在中文中通常翻译为“方程”或“等式”。这里提到的“方程”并不完全等同于日常所理解的方程。为了避免混淆,本节接下来的内容将统一使用“等式”一词\footnote{当然,很多时候,为了区分会将“equality”译作等式,而“equation”译作方程。}。\textbf{等式(equation)}是指由等号($=$)连接两个表达式构成的数学符号组合,表示二者之间的相等关系。

如果将等式中的表达式视为函数的对应关系,那么等式可以看作是描述两个函数之间的相等关系。这时,称函数的自变量称为\textbf{未知数(unknown)},而使等式成立的条件称为等式的\textbf{解(solution)},这里对应的就是未知数的取值\footnote{在大学阶段,还会研究各种各样的微分方程,微分方程也是一种方程,只不过它将某个不确定的函数关系作为未知量或求解目标。这时方程的解,也就是使等式成立的条件,就是某个或某类特定的函数。具体可以参见\enref{常微分方程简介}{ODEint}或\enref{常微分方程}{ODE}}。所有满足等式的解构成的集合称为\textbf{解集(solution set)}。如果没有值使二者相等,则解集为空集。

根据未知数允许的取值范围$M$与解集$S$的交集$C=M\cap S$的不同情况,可以将等式分为以下几类:
\begin{itemize}
\item 矛盾等式:如果$C = \varnothing$,则称原等式为\textbf{矛盾等式(contradictory equation)},表示该等式在给定的取值范围内没有任何解,通常称之为\textbf{方程无解}。这时有两种情况,一种是$S=\varnothing$,比如$0\times x=1$,一种是$S\neq\varnothing,C=\varnothing$,比如$x^2=-1$在实数范围内无解,此时$M=\mathbb{R},S={\pm \I}$。
\item 条件等式:如果$C \neq \varnothing$且$C \subsetneqq M$,即等式只在某些特定的自变量取值下成立,则称原等式为\textbf{条件等式(conditional equation)}。在高中范围内研究的“方程”或者说日常生活中说的“方程”,指的就是这种狭义上的条件等式。方程的解就是使条件等式成立的条件。这时,如果$S$中只有一个元素,则称方程有\textbf{唯一解},否则称方程有\textbf{多解}。
\item 恒等式:如果$C = M$,即在允许的取值范围$M$内所有值都能使等式成立,则称原等式为$M$上的\textbf{恒等式(identity)}。\enref{恒等式}{HsIden}常用于定义某种数学量或关系,例如,三角恒等式$\sin^2 x + \cos^2 x = 1$在所有实数$x$的范围内都成立。在求解方程时通常需要进行恒等变换。
\end{itemize}

总之,理解概念是最重要的。在中文用词上,通常“等式”这个词包含了上面提到的三种类型——矛盾式、条件等式和恒等式。而“方程”则特指条件等式,表示在特定条件下成立的等式,“方程无解”则指矛盾等式,之后不引起歧义时也会如此使用。在解题和讨论时,明确所指的对象,可以避免不必要的混淆。

现在回答“如何看待$x = 3$?”这个问题。$x = 3$既可以看作是一个\textbf{方程(条件等式)},即一个条件下成立的等式;也可以看作是这个方程的\textbf{解(恒等式)},即它的取值范围是解集$\{3\}$。从解的角度来看,可以将$x = 3$表示为$x \in \{3\}$。这种写法清晰地表明了$x$是要表示的值,而非条件等式的情况,可以避免歧义。因此,在不产生歧义的情况下,通常直接使用$x = 3$来表示解,而需要特别强调解集时采用解集的写法会更为精确,尤其是在讨论不等式时。

\subsubsection{不等式}

不等式这里就容易得多,不论是何种情况只要包含不等号,就会称为\textbf{不等式(inequality)}。但由此,也造成了使用时忽略了到底是指“不等式方程”还是“恒等不等式”(这两个都不是规范用语,只为了表达意思)。

其余的概念如:未知数、解、解集以及矛盾不等式、条件不等式、恒成立不等式,除符号与等式不同之外,均与等式中的定义相同,需要注意两点:
\begin{itemize}
\item 不等式通常没有单一的解,因此一般以解集的形式来描述结果,在高中阶段,这个结果通常会写成区间的形式,如$x^2-1>0$的解集为$(-\infty,-1)\cup(1,+\infty)$。
\item 由于不存在方程与恒等式的名称区分,或者说由于不等式通常都是非单一解的形式,不等式并不专指某一种不等式。因此,不等式有时指条件不等式,有时指恒成立的不等式,需要根据使用情况来判定,一般而言,条件不等式会在计算题中出现,或者与方程一同出现;而恒成立的不等式会在证明题出现,或者与恒等式一同出现。
\end{itemize}

\subsubsection{方程组}

初中时就已经接触过的方程组,当时给出的定义是“把两个必须同时满足的方程合在一起组成一个方程组”,这样的定义显然是朴素的,但它给出了一个很重要的条件就是,方程组涉及的等式需要同时成立,而这也是用大括号括起来几个方程时所传达的意思。

从而,一组需要同时成立的方程称为\textbf{方程组(system of equations)}。

注意到它使用的是“system”,一般一个常见的翻译是“系统”,因此其实可以将方程组看作是一个系统\footnote{很多时候,如果看到数学领域的某某“系统”,可能是指“某某方程组”。}。每个等式的解集$S_i$都相当于从限定的范围$M$中划出一部分,这个过程就像在描述这个系统的限定词,就像描述“硬的、红色的、圆的水果”中,前面三个词就分别从水果中划分出了一部分,最终的结果是他们的交集。这种限定行为一般称作\textbf{约束}。从约束来看待方程组是一个很重要的视角。同时使所有方程成立的条件称为方程组的\textbf{解},这是从方程组的定义中自然而然得到的,它也符合约束处提到的交集的定义。

通常情况下一个方程会给出方程组中一个变量的约束,从而,通常方程组是针对多个变量的。

事实上,描述约束的方式除了方程,还有不等式。通常一组只由不等式构成的称为\textbf{不等式组},不等式和等式共同构成的称作\textbf{约束系统(system of constraints)}

\subsection{解与解集}

根据前面的定义可知,方程和不等式只在解集中成立。

如果两个方程组或方程的解完全相同,则称二者等价。求一个方程的解的过通常是把它变换成一个等价的方程,有时,为了方便求解或产生某些错误时,把给定的方程变换成另一个方程后,会使得原本方程的解集成为新方程解集的真子集。所以,在解方程的过程中,如果进行了这种变换,就可能出现多余的解。因此,一般会将方程得到的解代人原方程进行验算,从而避免这类情况。这也是一个良好的解题习惯。


方程的解可以分为两大类:解析解和数值解。如果方程的解可以通过有限次的代数运算得到,这种解称为\textbf{解析解(Analytical Solution)}。这时,解的表达式可以用代数形式清晰地表示出来。有些复杂的方程很难找到解析解,甚至解析解根本不存在。在这种情况下,可以使用数值分析方法,如二分法、牛顿法等,通过迭代和近似计算来求解方程。此时得到的解称为\textbf{数值解(Numerical Solution)}。数值解通常通过计算机来计算,能够为复杂问题提供高精度的近似解。

总的来说,解析解是精确的,但不总是存在;数值解是近似的,却总是能提供实用的近似结果。在高中阶段,一般只涉及解析解,但存在大量的方程无法获得解析解,或难以获得解析解。

\subsubsection{求解规则}
等式的

不等式的

\subsubsection{有理不等式的解集}

穿针法

\subsubsection{解与零点}

\begin{definition}{代数学基本定理}
任何一个 $n$ 次多项式函数在复数域上都有 $n$ 个零点(重数计入)。
\end{definition}
这意味着在复数范围内,可以找到所有多项式方程的解。



