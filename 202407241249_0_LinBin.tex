% Linux 的二进制兼容问题
% license Xiao
% type Tutor

\begin{issues}
\issueDraft
\end{issues}

\pentry{g++ 编译器创建静态和动态链接库\nref{nod_gppLib}}{nod_bade}

\subsection{CPU 兼容性}

\begin{itemize}
\item 参考 \href{https://en.wikipedia.org/wiki/Binary-code_compatibility}{Wikipedia}。
\item 注意即使都是 x86-64 架构, 也可能有不同的拓展命令, 例如 AVX。 例如\href{https://stackoverflow.com/questions/50945287/illegal-instruction-when-run-precompiled-program-on-other-machine}{这个帖子}。
\item 如果指令集不支持, 会在运行时出现 \verb`Illegal Instruction` 错误, 也就是 \verb`SIGILL` 信号\upref{LinSig}。
\item gcc 编译器会默认使用 \verb`-march=native`, 也就是针对当前 cpu 的指令集编译, 这里面就可能包含一些其他 cpu 没有的拓展。 所以为了保证兼容性, 应该用 \verb`-march=x86-64`。 详见 gcc \href{https://gcc.gnu.org/onlinedocs/gcc/x86-Options.html}{文档}。
\item 如果想知道 \verb`-march=native` 具体用的是什么设置, 用 \verb`gcc -march=native -Q --help=target | grep march` 即可。
\item 如果是编译别人的 lib, 不好设置, 那可以试试在比较老的机器上面编译。
\end{itemize}

\subsection{库兼容性}
参考这里的\href{https://stackoverflow.com/questions/20183883/determining-binary-compatibility-under-linux}{回答}。

静态链接的程序一般\href{https://stackoverflow.com/questions/31801824/is-static-linking-in-linux-portable}{没什么问题}(除了 GLIBC)。 但对于动态链接的程序, 可以把所有非基础的 so 依赖库打包到一个文件夹里面并设置 \verb`rpath/runpath` 或者 \verb`LD_LIBRARY_PATH`。 但是这里面不能包括一些基础的系统库例如
\begin{itemize}
\item \verb`linux-vdso.so` 直接由内核提供,不存在于文件系统, 见\href{https://unix.stackexchange.com/questions/476971/ldd-shows-no-location-after-arrow-library-does-not-exist-on-system}{这里} 和 \href{https://en.wikipedia.org/wiki/VDSO}{vDSO - Wikipedia}。
\item \verb`glibc` 不兼容是链接时和运行时都很常见的问题,详见下文。
\item 动态链接到 glibc 会提高不同 glibc 版本的兼容性。
\item \verb`libm.so` 是 GNU C library 的一部分, 提供数学函数。
\item \verb`libgcc_s.so` 是 GCC runtime library 例如在 32bit 系统中提供 64bit 整数运算。
\item \verb`ld-linux-x86-64.so` 是 dynamic linker
\item \verb`libgomp.so` 跟 openmp 有关。
\item \verb`libstdc++.so` 是 C++ 库
\item \verb`libgfortran.so` 是 fortran 库
\item \verb`libquadmath.so` 是和 G++ 一起安装的四精度计算库。
\end{itemize}

最兼容的方法大概使用 \enref{chroot}{chroot} 或者 \enref{docker}{Docker}, 可以把除了 \verb`vdso` 以外的所有动态库都放到 fake root 文件夹里面。

\subsubsection{关于 GLIBC 兼容性}
\begin{itemize}
\item \verb`libc.so/.a`(\verb`GLIBC`) 是操作系统的核心库,如果不兼容比较难办。Ubuntu 只有升级版本才能升级 GLIBC 版本,一般是向下兼容的,所以只要保证足够新即可。 GLIBC 是 GNU C library, 提供 C 语言支持, 包括 system call 的 wrapper。 可以用 \verb`ldd --version` 或者 \verb`ldconfig -v | grep libc` 查看版本。
\item 为了提高 glibc 兼容性,可以选择在比较老的系统上面编译库文件,那么在更新的系统上 link 也应该会兼容。
\item 静态链接程序时,GCC 编译器经常会给出经典警告: \verb`warning: Using 'dlopen' in statically linked applications requires at runtime the shared libraries from the glibc version used for linking`。 其中 \verb`dlopen` 是 C 语言中用于动态加载动态链接库的函数。 有可能你的代码中没有直接调用 \verb`dlopen`,但可能一些库的 .a 文件会使用, 例如 \verb`sqlite3.a`。 而且也并不是说运行时一定需要完全相同版本的 glibc 才可以兼容, 由于 glibc 向后兼容一般只需要运行时 glibc 版本高于编译时即可。
\item 如果想要动态链接到 glibc,而其他所有库都静态链接, 那么这要求 \verb`其他库.a` 文件不直接静态链接 glibc,而是等到你的程序最终链接时再一次性动态链接到 glibc。 所以可能需要一些 hacking 重新编译 \verb`其他库.a`。
\end{itemize}
