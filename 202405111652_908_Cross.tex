% 几何向量的叉乘
% keys 线性代数|向量|叉乘|cross product|叉积|cross product|向量积|vector product|分配率
% license Xiao
% type Tutor

\pentry{几何向量\nref{nod_GVec}}{nod_8777}

向量叉乘在物理定律中十分常见, 例如在讨论力学中的力矩\upref{Torque}, 角动量\upref{AngMom}, 以及电磁学中的洛伦兹力\upref{Lorenz}, 安培力\upref{FAmp}时都会使用。 以下我们讨论的向量都是三维中的空间向量\upref{GVec}; 在讨论它们的坐标时, 我们默认取正交归一基底。

\subsection{叉乘的几何定义}

\begin{figure}[ht]
\centering
\includegraphics[width=8cm]{./figures/6bf04211b7159b15.pdf}
\caption{叉乘的示意图} \label{fig_Cross_2}
\end{figure}

两个几何向量 $\bvec A$,  $\bvec B$ 的\textbf{叉乘(cross product)} 也叫\textbf{叉积},\textbf{向量积(vector product)}或\textbf{向量积}。 叉乘的结果是一个向量。  叉乘用 $\bvec A \cross \bvec B$ 表示,且“$\cross$”不可省略。

\begin{definition}{向量叉乘}\label{def_Cross_1}
要确定一个几何向量,只需分别确定模长和方向:
\begin{enumerate}
\item $\bvec C$ 的模长等于 $\bvec A, \bvec B$ 的模长之积与夹角 $\theta$ ($0 \leqslant \theta \leqslant \pi$)的正弦值相乘。
\begin{equation}\label{eq_Cross_1}
\abs{\bvec C}  = \abs{\bvec A} \abs{\bvec B} \sin\theta ~.
\end{equation}
\item $\bvec C$ 的方向垂直于 $\bvec A, \bvec B$ 所在的平面,且由\enref{右手定则}{RHRul}决定。
\end{enumerate}
\end{definition}

结合\autoref{fig_Cross_2} 与\autoref{eq_Cross_1} ,可看出$\bvec C$的大小就是$\bvec A, \bvec B$所围成的平行四边形的面积, 亦即用平行四边形法则\upref{GVecOp}计算 $\bvec A + \bvec B$ 时的平行四边形。

\subsection{叉乘运算律}
\subsubsection{一个结论}
特别地,由\autoref{eq_Cross_1} 可知, 当两向量平行时, 叉乘为零向量; 当它们垂直, 叉乘的模长等于两模长直接相乘。%对于两个模长相同的向量, 

\subsubsection{叉乘的 “交换律”}
与内积和数乘不同,叉乘\textbf{不满足一般意义上的交换律}。根据几何定义, $\bvec B \cross \bvec A$ 与 $\bvec A \cross \bvec B$ 模长相同,方向却相反。表示某个向量的反方向,就是在前面加负号,所以有
\begin{equation}\label{eq_Cross_11}
\bvec B \cross \bvec A = -\bvec A \cross \bvec B~.
\end{equation}

\subsubsection{叉乘与数乘的结合律}

在 $\bvec A \cross \bvec B = \bvec C$ 中, $\bvec C$ 的方向仅由 $\bvec A$ 和 $\bvec B$ 的方向决定。当 $\bvec A$ 和 $\bvec B$ 的方向不变时, $\bvec C$ 的模长正比于 $\bvec A$ 和 $\bvec B$ 的模长之积。假设 $\lambda $ 为常数(标量),显然有
\begin{equation}
(\lambda \bvec A) \cross \bvec B = \bvec A \cross (\lambda \bvec B) = \lambda (\bvec A \cross \bvec B)~.
\end{equation}
即标量的位置可以任意变换,但向量与乘号的位置关系始终要保持不变。

\subsubsection{叉乘的分配律}

叉乘一个最重要的特性,就是它满足分配律。
\begin{equation}\label{eq_Cross_6}
\bvec A \cross (\bvec B +\bvec C) = \bvec A \cross \bvec B + \bvec A \cross \bvec C~,
\end{equation}
由\autoref{eq_Cross_11} 及上式可以推出
\begin{equation}\label{eq_Cross_7}
(\bvec A + \bvec B) \cross \bvec C =  - \bvec C \cross (\bvec A + \bvec B) =  - \bvec C \cross \bvec A - \bvec C \cross \bvec B = \bvec A \cross \bvec C + \bvec B \cross \bvec C~.
\end{equation}

从几何的角度理解,这个结论并不显然(见向量叉乘分配律的几何证明\upref{CrossP})。

\subsection{叉乘的坐标运算}
\subsubsection{坐标轴向量的叉乘}
\begin{figure}[ht]
\centering
\includegraphics[width=4cm]{./figures/5b03b49cd3e8f622.png}
\caption{坐标轴} \label{fig_Cross_3}
\end{figure}
按照上面的定义,在右手系中,三个坐标轴的单位向量 $\uvec x, \uvec y, \uvec z$ 满足
\begin{equation}\label{eq_Cross_3}
\uvec x \cross \uvec y = \uvec z
\qquad
\uvec y \cross \uvec z = \uvec x
\qquad
\uvec z \cross \uvec x = \uvec y~
\end{equation}
由\autoref{eq_Cross_11} 可得
\begin{equation}\label{eq_Cross_4}
\uvec y \cross \uvec x =  - \uvec z
\qquad
\uvec z \cross \uvec y =  - \uvec x
\qquad
\uvec x \cross \uvec z =  - \uvec y~
\end{equation}
根据定义,一个向量叉乘自身,模长为 $0$。 所以叉乘结果是零向量 $\bvec 0$。 于是又有
\begin{equation}\label{eq_Cross_5}
\uvec x \cross \uvec x = \bvec 0
\qquad
\uvec y \cross \uvec y = \bvec 0
\qquad
\uvec z \cross \uvec z = \bvec 0~
\end{equation}
\autoref{eq_Cross_3},\autoref{eq_Cross_4} 和\autoref{eq_Cross_5} 中共 9 条等式描述了 $\uvec x, \uvec y, \uvec z$ 中任意两个叉乘的结果。

\subsubsection{任意向量的叉乘}
把向量 $\bvec A$ 和 $\bvec B$ 分别在直角坐标系的三个单位向量展开,得到
\begin{equation}
\bvec A = a_x\,\uvec x + a_y\,\uvec y + a_z\,\uvec z \qquad \bvec B = b_x\,\uvec x + b_y\,\uvec y + b_z\,\uvec z~
\end{equation}
$(a_x,a_y,a_z)$ 和 $(b_x,b_y,b_z)$ 分别是 $\bvec A$ 和 $\bvec B$ 的坐标。根据叉乘的分配律(\autoref{eq_Cross_6} \autoref{eq_Cross_7}),可得到如下 9 项
\begin{equation}
\ali{
\bvec A \cross \bvec B ={} &(a_x\,\uvec x + a_y\,\uvec y + a_z\,\uvec z) \cross (b_x\,\uvec x + b_y\,\uvec y + b_z\,\uvec z)\\
={} &+ a_x b_x(\uvec x \cross \uvec x) + a_x b_y(\uvec x \cross \uvec y) + a_x b_z(\uvec x \cross \uvec z)\\
&+ a_y b_x(\uvec y \cross \uvec x) + a_y b_y(\uvec y \cross \uvec y) + a_y b_z(\uvec y \cross \uvec z)\\
&+ a_z b_x(\uvec z \cross \uvec x) + a_z b_y(\uvec z \cross \uvec y) + a_z b_z(\uvec z \cross \uvec z)~.
}\end{equation}
注意每一项中的运算在\autoref{eq_Cross_3} 到\autoref{eq_Cross_5} 中都能找到答案,于是上式化为
\begin{equation}\label{eq_Cross_2}
\bvec A \cross \bvec B = (a_y b_z - a_z b_y)\,\uvec x + (a_z b_x - a_x b_z)\,\uvec y + (a_x b_y - a_y b_x)\,\uvec z~.
\end{equation}
令 $\bvec C = \bvec A \cross \bvec B$, 则 $\bvec C$ 的分量表达式为
\begin{equation}\label{eq_Cross_8}
\begin{cases}
c_x = a_y b_z - a_z b_y\\
c_y = a_z b_x - a_x b_z\\
c_z = a_x b_y - a_y b_x
\end{cases}~
\end{equation}
\autoref{eq_Cross_2} 可以用三阶行列式\upref{Deter}表示为
\begin{equation}\label{eq_Cross_13}
\bvec A \cross \bvec B = 
\begin{vmatrix}
\uvec x & \uvec y & \uvec z\\
a_x & a_y & a_z\\
b_x & b_y & b_z
\end{vmatrix} ~\end{equation}
与普通行列式不同的是,这个行列式中第一行的元素是向量,所以得出的结果也是向量。

%叉乘在物理中的应用如: 圆周运动的线速度(\autoref{eq_CMVD_5}~\upref{CMVD}), 洛伦兹力(\autoref{eq_Lorenz_1}~\upref{Lorenz}), 安培力(\autoref{eq_FAmp_1}~\upref{FAmp})。
%已经在引言中提及

\subsection{等效的叉乘}\label{sub_Cross_1}
\begin{figure}[ht]
\centering
\includegraphics[width=3.5cm]{./figures/2630d2018b496c7c.pdf}
\caption{$\bvec A$ 与 $\bvec B_i$ 相乘结果相同} \label{fig_Cross_1}
\end{figure}

根据\autoref{def_Cross_1} , 当两向量叉乘时, 若把\textbf{一个向量延着与另一个向量垂直的方向投影, 叉乘结果不变}。 例如在\autoref{fig_Cross_1} 中有
\begin{equation}
\bvec A\cross\bvec B_1 = \bvec A \cross \bvec B_3 = \bvec A\cross \bvec B_2~.
\end{equation}
这是因为 $\abs{\bvec B_i}\sin\theta_i$ 就是投影后的 $\bvec{\bvec B_2}$ 的模长。 另一种证明方法:所有指向虚线的 $\bvec B_i$ 都可以表示为 $\bvec B_2 + \lambda \bvec A$($\lambda$ 为任意实数)。 根据分配律,有
\begin{equation}
\bvec A \cross (\bvec B_2 + \lambda \bvec A) = \bvec A \cross \bvec B_2 + \bvec A \cross (\lambda \bvec A) = \bvec A \cross \bvec B_2~.
\end{equation}

\subsection{面积向量}
在高中阶段面积只是一个标量概念, 但在微积分和大学物理中, 我们也可以把它拓展为向量。 定义\textbf{面积向量}的模长是一个平面几何图形的面积, 方向是其所在平面的两个法向量之一。 具体到这里讨论的平行四边形, 我们可以根据叉乘结果来明确定义平行四边形的面积向量。

\begin{example}{求三角形或平行四边形的面积与法向量}\label{ex_Cross_1}
空间直角坐标系中三角形的三点分别为 $O(0,0,0)$,  $A(1,1,0)$,  $B(-1,1,1)$。 求三角形的面积和一个单位法向量。 以这三点为顶点的平行四边形的面积显然是该三角形的两倍, 所以下面只讨论三角形。

令 $O$ 到 $A$ 的向量和  $O$ 到 $B$ 的向量分别为
\begin{equation}
\ali{
\bvec a  &= (1,1,0) - (0,0,0) = (1,1,0)\\
\bvec b  &= (-1,1,1) - (0,0,0) = (-1,1,1)
}~\end{equation}
三角形的面积为
 \begin{equation}
S = \frac12 ab \sin \theta ~.
\end{equation}
其中 $\theta $ 是 $\bvec a$ 与 $\bvec b$ 的夹角。根据\autoref{eq_Cross_1}, 有\footnote{可见 $\abs{\bvec a\cross\bvec b}$ 是以 $\bvec a$ 和 $\bvec b$ 为边的平行四边形的面积。}
\begin{equation}
S = \frac12 ab \sin \theta  = \frac12 \abs{\bvec a\cross\bvec b}~.
\end{equation}
令
\begin{equation}
\bvec v = \bvec a \cross \bvec b = 
\begin{vmatrix} \uvec x & \uvec y & \uvec z \\ 1&1&0\\-1&1&1 \end{vmatrix}
= \uvec x - \uvec y + 2\,\uvec z ~.
\end{equation}
坐标为 $(1,-1,2)$,模长为 $\abs{\bvec v} = \sqrt{1 + 1 + 2^2} = \sqrt 6$, 所以面积为 $S = \sqrt 6 /2$。 

根据叉乘的几何定义, $\bvec v = (1,-1,2)$ 就是三角形的法向量,进行\textbf{归一化}\upref{GVecOp}
得单位法向量为
 \begin{equation}
\uvec v = \frac{\bvec v}{\abs{\bvec v}} = \frac{(1,-1,2)}{\sqrt 6} = \qty( \frac{\sqrt 6 }{6}, - \frac{\sqrt 6 }{6}, \frac{\sqrt 6 }{3} )~.
\end{equation}
\end{example}

