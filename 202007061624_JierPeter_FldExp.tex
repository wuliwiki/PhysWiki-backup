% 域的扩张
\pentry{域\upref{field}}


\subsection{域的单扩张}
如果在一个域中添加不属于域集合的元素,我们可以得到一个更大的集合.要让这个新集合成为域,我们就得定义新元素和原来域中元素相加和相乘的结果;无论怎么定义,这个结果必须满足域的公理.如果在集合中任何元素都无法成为某个运算结果,那么我们就必须再引入新的元素来作为这个结果.以此类推,不停地添加新元素,直到最后不需要添加新元素了,那最后这个集合就是一个新的域,它包含了原来的域.这个域是原来的域的\textbf{扩张},并且包含最初那个新元素的最小的域,因此被称为\textbf{单元素扩张},简称\textbf{单扩张}.

域的单扩张具体是怎么进行的呢?我们将从例子开始说明,最后引入域的单扩张的定义.

\begin{exercise}{有理数域的$\sqrt{2}$扩张}
给定有理数域$\mathbb{Q}$,则$\sqrt{2}$并不是$\mathbb{Q}$的元素.将$\sqrt{2}$添加进去,那么由于$\sqrt{2}$是实数域$\mathbb{R}$的元素,这提示我们可以把新的运算结果按照$\mathbb{R}$中的运算来定义.这样,$\mathbb{Q}$中添加$\sqrt{2}$的单扩域的集合就是$\{a+b\sqrt{2}|a, b\in\mathbb{Q}\}$.

请验证,任意$a+b\sqrt{2}$都必须在扩域中,而这个集合也满足了加法和乘法的封闭性,所以构成了一个域;换句话说,这个集合是包含$\sqrt{2}$和全体有理数的最小的域.
\end{exercise}

\begin{exercise}{超越扩域}
依然给定有理数域$\mathbb{Q}$,这次添加的元素是$\pi$.那么扩域应当是$\{\}$
\end{exercise}


