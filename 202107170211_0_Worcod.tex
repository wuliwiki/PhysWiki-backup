% 世界坐标系
% keys 世界坐标系 全局坐标系


一个三维场景中通常都不会只有一个物体.我们真正需要的是把我们建立的物体按照我们所需要的形式摆放在场景之中.每个物体分布在场景的适当的位置上.整个场景的坐标系就称为\textbf{世界坐标系(world coordinate system)}.

从建模坐标系到世界坐标系有一个坐标变换,即建模变换(modeling transformation)或称模型矩阵(model matrix).建模变换通过平移(translate)、旋转(rotate)和缩放(scale)将物体摆放在场景中的适当位置上.

图形学软件包OpenGL通常是采用右手坐标系.在右手系中,用右手握住z轴,大拇指指向z轴正方向,四指环绕方向是从x轴正半轴到y轴正半轴.




参考文献:
\begin{enumerate}
\item Donald Hearn, Pauline Baker, Carithers著, 蔡士杰, 杨若瑜译. 计算机图形学[M]. 电子工业出版社. 2014
\item https://learnopengl.com/Getting-started/Coordinate-Systems
\item https://learnopengl-cn.readthedocs.io/zh/latest/01\%20Getting\%20started/08\%20Coordinate\%20Systems/
\end{enumerate}