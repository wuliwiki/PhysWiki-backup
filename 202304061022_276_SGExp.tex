% Stern-Gerlach 实验
% 斯特恩-盖拉赫实验|狄拉克符号|自旋|坍缩|量子态

\pentry{量子力学的基本原理(量子力学)\upref{QMPrcp}}


Stern-Gerlach实验由O. Stern于1921年构想,由Stern和W. Gerlach于1922年在法兰克福完成\footnote{可参见Bretislav Fridrich和Dudley Herschbach发表的\textsl{Stern and Gerlach: How a Bad Cigar Helped Reorient Atomic Physics}, \textsl{Physics Today}, Dec. 2003。}。

引用樱井纯《现代量子力学》中的描述:“在某种意义上,Stern-Gerlach类型的双态系统是最少经典力学而最多量子力学的系统。对涉及双态系统问题的坚实理解将对任何认真学习量子力学的学生都是有益的。”\textbf{量子力学的基本原理(量子力学)}\upref{QMPrcp}词条中也建议配合本词条内容来理解抽象的概念。

另见\textbf{自旋角动量}\upref{Spin}。


\subsection{实验描述}

用一个炉子加热银原子,使之获得动能,从炉子上的一个小洞跑出来。出射的银原子会经过一个准直器,之后朝已经建立好的非均匀磁场飞去。这个磁场由如图所示的两磁极构成,其中一磁极有尖锐的边缘。之后,银原子会打到一块接收屏上,形成可观测的光斑。实验如\autoref{fig_SGExp_1} 所示。


\begin{figure}[ht]
\centering
\includegraphics[width=14cm]{./figures/1f41333cbde7d8f9.pdf}
\caption{Stern-Gerlach实验示意图。} \label{fig_SGExp_1}
\end{figure}

银原子由一个原子核、47个电子组成,其中46个电子构成了总角动量为零的球对称电子云,而剩下那个电子的自旋则为整个原子提供了非零的角动量。原子核自旋与本实验无关,故不讨论。

于是,原子整体上有一个磁矩$\bvec{\mu}$,正比于剩下那个电子的自旋$\bvec{S}$:
\begin{equation}
\bvec{\mu}\propto\bvec{S}
\end{equation}

按照\autoref{fig_SGExp_1} 所示的装置,银原子通过非均匀磁场时会因为存在磁矩而受到竖直方向的力,这个力正比于原子磁矩的竖直分量$\mu_z$。因此,银原子最后击中接收屏的位置,反映了银原子偏转的大小,从而可以看作是对银原子的$\mu_z$进行的测量,也可以看作是对电子自旋的测量。




\begin{figure}[ht]
\centering
\includegraphics[width=10cm]{./figures/7dccc697c1418554.pdf}
\caption{Stern-Gerlach实验的结果,左边是经典理论的预言,右边是实际观测的结果。} \label{fig_SGExp_2}
\end{figure}


经典理论认为,银原子的磁矩取向是完全随机的,因此$\mu_z$可能是$\pm\abs{\bvec{\mu}}$之间的任何值。因此,如果我们放出大量的银原子轰击接收屏幕,并记录银原子所击中的位置,综合起来,应该得到\autoref{fig_SGExp_2} 左边所示的结果。

但实验观测到的却是\autoref{fig_SGExp_2} 右边所示的结果,是两个\textbf{强度相等}的光斑。


实验结果意味着,如果我们测量电子自旋,那只能得到一正一负两个数值,而不是连续的结果;换言之,自旋算符只有两个特征值,数值上是$\pm\hbar/2$。


如果我们把\autoref{fig_SGExp_1} 中的磁场转动90度,测量银原子磁矩的水平分量$\mu_y$或者说电子自旋的水平分量,也会得到同样的结果。



注意,实验结果并不意味着“银原子的磁矩”或“电子的自旋”,在$z$方向上只有两种可能性;而是意味着,如果我们去测量,就只能得到两种结果。此时银原子的量子态已经坍缩成对应结果的本征态了,通常已不再是测量前的状态。



\subsection{序列Stern-Gerlach实验}\label{sub_SGExp_1}

上一小节末尾强调了“状态是xxx”和“测量能得到xxx”是不一样的概念。本小节介绍的序列实验能更好地体现出,并非是原子处在$\pm\hbar/2$的自旋态,而是它处于一个能吐出$\pm\hbar/2$这两个测量值的状态。

现在,把\autoref{fig_SGExp_1} 所示的实验装置视为一个整体,一个“测量银原子某方向上磁矩”的装置。把测量$z$方向磁矩的装置记为$\opn{SG}\uvec{z}$,测量$y$方向磁矩的装置记为$\opn{SG}\uvec{y}$,测量$x$方向的记为$\opn{SG}\uvec{x}$。

首先我们要确定,一个$\opn{SG}\uvec{z}$装置如量子力学的基本原理所述,并不会改变该测量的本征态。先让银原子通过一台$\opn{SG}\uvec{z}$仪器,按照理论预言,此时它已经变成本征态了。我们把本征值为负的本征态剔除——实验上很容易做到,在装置出射口用挡板挡住其中一个偏转方向的银原子射流,只让另一个方向的射流顺利离开。此时,筛选出的射流应该处于正的本征值。让它通过下一台$\opn{SG}\uvec{z}$仪器,结果屏幕上确实只剩下了一个光斑,对应本征值为正的状态。

现在,用同样的手法,筛选出$\opn{SG}\uvec{z}$的本征值为正的射流,但接下来让它通过一台$\opn{SG}\uvec{x}$仪器。实验结果是\textbf{强度相同}的两个光斑。

接下来做最后一个实验。筛选出$\opn{SG}\uvec{z}$的本征值为正的射流,让它通过一台$\opn{SG}\uvec{x}$仪器,筛选出其中一个方向的射流,再让它通过一台$\opn{SG}\uvec{z}$仪器。结果是我们又得到了\textbf{强度相同}的两个光斑。
\addTODO{需要添加上类似樱井的《现代量子力学》中的示意图。}



实验结果表明,我们没法同时测量出银原子在$z$和$x$两个方向上的磁矩,对$x$方向的测量破坏了$z$方向的状态。按照量子力学的\textbf{坍缩假设}\footnote{即测量后,量子态会返回测量算符的一个本征值,并坍缩成该本征值的一个本征态。},这意味着没有任何一个量子态,同时是$x$、$z$方向上磁矩测量算符的本征态。



\begin{exercise}{与光的偏振态进行类比}
这个实验现象本质上和光的偏振实验\autoref{ex_QMPrcp_3}~\upref{QMPrcp}是同样的原理。对照这两个实验,比较它们的异同,观察它们是如何满足量子力学基本原理的。
\end{exercise}






\subsection{对自旋$1/2$系统的讨论}

电子、银原子都可以描述为具有$1/2$自旋的系统。自旋和地球的自转不同,如果你测量地球在某个方向上的角动量,可能得到的值应像\autoref{fig_SGExp_2} 左边经典理论预言的那样“连续”;另外,我们可以把自转角动量描述成一个几何矢量,把各个方向的角动量测量值理解为这个几何矢量的分量大小。显然,自旋$1/2$系统没有这么直接。

由实验可知,对银原子磁矩、或者说电子自旋的\textbf{测量},只有两个本征值,$\pm\hbar/2$。由于$\hbar$是常用的常数,因此\textbf{自然单位制}\upref{NatUni}下我们可以设$\hbar=1$,而将电子自旋描述为$\pm 1/2$。不过为了避免混淆,我们接下来还是带着$\hbar$进行讨论。根据\autoref{the_QMPrcp_7}~\upref{QMPrcp},这意味着刻画一个电子自旋的态空间可以是二维的。这是一个抽象的矢量空间,不可和几何的二维空间相混淆。

\subsubsection{用本征态表出自旋算符}

考虑对$z$方向的自旋进行测量,将该测量算符记为$S_z$,其本征矢表示为$\ket{S_z; \pm}$,分别对应本征值$\pm \hbar/2$。这两个向量能张成整个自旋态空间,即所有自旋态都能表示为$a\ket{S_z; +}+b\ket{S_z; -}$,其中$a, b\in\mathbb{C}$是复数,且$\abs{a}^2+\abs{b}^2=1$。

于是,根据\autoref{cor_QMPrcp_3}~\upref{QMPrcp},单位算符可以表示为
\begin{equation}
\mathbb{1}=\ket{S_z; +}\bra{S_z; +} + \ket{S_z; -}\bra{S_z; -}~.
\end{equation}
根据\autoref{cor_QMPrcp_2}~\upref{QMPrcp},自旋测量算符可以表示为
\begin{equation}
S_z=\frac{\hbar}{2}\qty(\ket{S_z; +}\bra{S_z; +} - \ket{S_z; -}\bra{S_z; -})
\end{equation}

以$\ket{S_z; \pm}$为基向量,则$S_z$的矩阵表示为
\begin{equation}\label{eq_SGExp_1}
\frac{\hbar}{2}\pmat{1&0\\0&-1}
\end{equation}


显然,如果取$\ket{S_x; \pm}$为基向量,则$S_x$的矩阵表示也和\autoref{eq_SGExp_1} 一致;$S_y$也一样。



\subsubsection{其它方向的本征态与自旋算符}


接下来,我们讨论在给定$\ket{S_z; \pm}$为基向量时,$S_x$和$S_z$应怎么表达。

根据序列Stern-Gerlach实验\autoref{sub_SGExp_1} 的结果,一束处于$\ket{S_x; +}$的射流通过一台$\opn{SG}\uvec{z}$仪器后,劈裂为两束\textbf{强度相等}的射流。由概率的定义\autoref{def_QMPrcp_15}~\upref{QMPrcp}和量子态的等价\autoref{def_QMPrcp_2}~\upref{QMPrcp},我们可以写出:
\begin{equation}\label{eq_SGExp_2}
\ket{S_x; +}=\frac{1}{\sqrt{2}}\ket{S_z; +}+\frac{\E^{\I \delta_1}}{\sqrt{2}}\ket{S_z; -}
\end{equation}
其中$\delta_1$是一个实数,尚无法确定其取值。这已经是自由度最大的形式了。

类似地,处于$\ket{S_x; -}$的射流在进行$S_z$测量后也会劈裂为\textbf{强度相等}的射流,再考虑到$\ket{S_x; \pm}$彼此正交,可得\footnote{注意,$\qty(\frac{\E^{\I \delta_1}}{\sqrt{2}}\ket{S_z; -})^*=\frac{\E^{-\I \delta_1}}{\sqrt{2}}\bra{S_z; -}$。}
\begin{equation}\label{eq_SGExp_3}
\ket{S_x; -}=\frac{1}{\sqrt{2}}\ket{S_z; +}-\frac{\E^{\I \delta_1}}{\sqrt{2}}\ket{S_z; -}
\end{equation}


由\autoref{cor_QMPrcp_2}~\upref{QMPrcp},代入\autoref{eq_SGExp_2} 和\autoref{eq_SGExp_3} 得
\begin{equation}\label{eq_SGExp_4}
\begin{aligned}
S_x &= \frac{\hbar}{2}\qty(\ket{S_x; +}\bra{S_x; +}-\ket{S_x; -}\bra{S_x; -})\\
&=\frac{\hbar}{2}\qty(\E^{\I\delta_1}\ket{S_z; -}\bra{S_z; +} + \E^{-\I\delta_1}\ket{S_z; +}\bra{S_z; -})
\end{aligned}
\end{equation}

类似地,还可以得出
\begin{equation}\label{eq_SGExp_7}
\ket{S_y; \pm} = \frac{1}{\sqrt{2}}\ket{S_z; +}\pm\frac{\E^{\I\delta_2}}{\sqrt{2}}\ket{S_z; -}
\end{equation}
和
\begin{equation}\label{eq_SGExp_5}
S_y=\frac{\hbar}{2}\qty(\E^{\I\delta_2}\ket{S_z; -}\bra{S_z; +} + \E^{-\I\delta_2}\ket{S_z; +}\bra{S_z; -})
\end{equation}
其中$\delta_2$是另一个待定实数。

为了算出$\delta_1$和$\delta_2$,考虑将$S_x$的本征态通过$\opn{SG}\uvec{y}$仪器,则结果和$S_z$的本征态通过$\opn{SG}\uvec{x}$是一样的:两个强度相等的光斑。于是我们有
\begin{equation}\label{eq_SGExp_6}
\abs{\braket{S_y; \pm}{S_x; +}}=\abs{\braket{S_y; \pm}{S_x; -}}=\frac{1}{\sqrt{2}}
\end{equation}

将\autoref{eq_SGExp_2} 或\autoref{eq_SGExp_3} ,和\autoref{eq_SGExp_7} 一起代入\autoref{eq_SGExp_6} ,得
\begin{equation}\label{eq_SGExp_8}
\begin{aligned}
\abs{\frac{1}{2}\braket{S_z; +}{S_z; +} \pm \frac{\E^{\I(\delta_1-\delta_2)}}{2}\braket{S_z; -}{S_z; -}} &= \frac{1}{\sqrt{2}}\\
\frac{1}{2}\abs{1 \pm \E^{\I(\delta_1-\delta_2)}} &= \frac{1}{\sqrt{2}}
\end{aligned}
\end{equation}

由于$\abs{\E^{\I(\delta_1-\delta_2)}}=1$,故从几何上可以很容易将\autoref{eq_SGExp_8} 化为
\begin{equation}\label{eq_SGExp_10}
\delta_1-\delta_2=\pm\frac{\pi}{2}
\end{equation}

如果取$\delta_1=0$,$\delta_2=\frac{\pi}{2}$,那么应有
\begin{equation}
\leftgroup{
    \ket{S_x; \pm}&=\frac{1}{\sqrt{2}}\ket{S_z; +} \pm \frac{1}{\sqrt{2}}\ket{S_z; -};\\
    \ket{S_y; \pm}&=\frac{1}{\sqrt{2}}\ket{S_z; +} \pm \frac{\I}{\sqrt{2}}\ket{S_z; -};\\
}
\end{equation}
和
\begin{equation}\label{eq_SGExp_9}
\leftgroup{
    S_x&=\frac{\hbar}{2}\qty(\ket{S_z; -}\bra{S_z; +} + \ket{S_z; +}\bra{S_z; -})\\
    S_y&=\frac{\I\hbar}{2}\qty(\ket{S_z; -}\bra{S_z; +} - \ket{S_z; +}\bra{S_z; -})
}
\end{equation}


\subsubsection{自旋算符的对易关系}

由\autoref{eq_SGExp_9} 可算出自旋算符的对易性质,而实际上任何满足\autoref{eq_SGExp_10} 的也都具有同样的性质
\addTODO{需要检查一下“任何满足\autoref{eq_SGExp_10} 的也都具有同样的性质”是否成立,检查后删除此“未完成”。}



\begin{theorem}{自旋算符的对易关系}

\begin{equation}
[\hat{S}_z, \hat{S}_z]=0
\end{equation}

\begin{equation}
[\hat{S}_x, \hat{S}_y]=\I\hbar S_z
\end{equation}

\begin{equation}\label{eq_SGExp_11}
\{\hat{S}_z, \hat{S}_z\}=\frac{\hbar^2}{2}
\end{equation}

\begin{equation}
\{\hat{S}_x, \hat{S}_y\}=0
\end{equation}

\begin{equation}\label{eq_SGExp_12}
[\bvec{S}^2, S_x]=0
\end{equation}

其中$\bvec{S}^2=S_x^2+S_y^2+S_z^2=\frac{3\hbar^2}{4}$。

\end{theorem}

实际上,若自旋大于$1/2$,则$\bvec{S}^2$不再是单位算符的倍数,但\autoref{eq_SGExp_12} 依然成立。
\addTODO{有了相关讨论后,在此引用。}










