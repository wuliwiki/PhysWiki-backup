% Fortran 入门笔记

\subsection{常识}
\begin{itemize}
\item Fortran 代码不区分大小写
\item 感叹号用于写注释
\item \verb`&` 用于续行, 如果一个 token 需要续行, 那么下一行的行首也需要 \verb`&`. 另外, 必须放在注释前面.
\item 程序首尾  \verb`program <name> ....  end program <name>`, 或者直接用 \verb`end`
\item 没有子程序的时候, \verb`program <name>` 可以省略, 最后一行 \verb`end program <name>` 也可以省略
\item \verb`print*, <item1>, <item2>....` 星号代表默认格式
\item 用 \verb`read*, item1, item2, ...` 来输入变量, 要先定义变量类型. 
\item \verb`read` 读取 character 类型的时候, 只能读取字母, 不能有任何其他符号
\item 字符串可以用单引号也可以用双引号
\item 可以用 \verb`;` 符号在一行写多个语句, 叫 statement separator. 同一行中的每个 statement 不能超过 132 个字符
\item \verb`stop` 可用于终止主程序, \verb`return` 终止子程序
\item 所有的小括号前面可以有空格
\item \verb`include "<file>"` 就是把文件的内容复制到当前位置, 应该相当于 C++ 的 \verb|#include|
\end{itemize}

\subsection{算符}
\begin{itemize}
\item 指数算符是 \verb`**`
\item 比较算符有: \verb`==   >=   <=  /= `.  注意不等是用斜杠不是感叹号. 感叹号是用于注释的
\item 逻辑算符有 \verb`.and.   .or.   .not.   .eqv.` (同或)  \verb`.neqv.` (异或)
\item \verb`//` 用于连接字符串
\end{itemize}

\subsection{结构}
\begin{itemize}
\item \verb`if (<logical>) then ....  else .... end if` 注意要有括号
\item \verb`if (<logical>) then ... else if (<logical>) then ... else ... end if`
\item \verb`if` 前面可以加上 \verb`<name>:` 如果这样做,  end if 后面也一定要加上 \verb`<name>`  , 也可以选择在 \verb`else if .. then` 或 \verb`else` 后面加上 \verb`<name>`. 注意只有开头有冒号, 后面没有冒号 .
\item select case, 见\href{http://www.tutorialspoint.com/fortran/select_case_construct.htm}{这里}
\item do 循环:   \verb`do ii = 1,3; ...; end do;` 也可以设置步长 \verb`ii = int1, int2, step`. 其中 \verb`step` 可以是负值.
\item do while 循环: \verb`do while (<logical>) ....  end do`
\item \verb`exit` 相当于 C 语言的 break, \verb`exit <name>` 用于退出指定循环.
\item \verb`cycle` 相当于 C 语言的 continue
\item do 前面可以加上 \verb`<name>:` 如果这样, \verb`end do` 后面也一定要加上 \verb`<name>`
\item \verb`cycle` 执行下一个当前循环, \verb`cycle <name>` 执行下一个指定循环
\item \verb`do` 后面什么都不加相当于无条件循环, 需要用 \verb`exit` 退出.
\item statement label 是在 statement 前面的 1-5 位数字.
\item \verb`go to <label>` 可以使程序跳转到 \verb`<label>` 的语句.
\end{itemize}

\subsection{数据类型}
\begin{itemize}
\item \verb`integer, real, complex, character, logical` 是五种 intrinsic types, 前三种是 numeric, 剩下的是 non-numeric.
\item 从 fortran90 开始, 所有变量都要定义. 用 implicit none 语句可以使未定义的变量出错. 亲自证明了不声明变量类型会导致很严重的错误且很难调试.
\item 除了 imlicit none, 另一种声明如 implicit double precision (a-h,o-z), 作用是默认所有 a-h, o-z 开头的变量都是 double precision, 而 i-n 开头的所有变量都是普通 integer*4. 这种做法不建议使用.
\item 所有的变量只能在 implicit none 后面紧接着, 不能在程序中定义.
\item 声明变量格式:  \verb`<类型> :: a=1, b=2, ....`  初值可选
\item 变量声明时 \verb`::` 可以省略
\item 5 种基本变量类型可以定义 constant 格式, 如  \verb`real, parameter :: pi = 3.1415927`
\item 声明变量后, 也可以把某个变量变成常数, 直接用 parameter (var1 = 1, var2 = 2) 即可. 括号不可省略
\item \verb`selected_int_kind(8)` 可以返回 integer 类型的 kind, 是的 integer 有 8 位有效数字
\item \verb`selected_real_kind(15, 307)` 可以返回 real 类型的 kind, 使得 real 有 15 位有效数字和最大指数 307
\item \verb`huge()` 可以测试某个变量的最大值
\item \verb`tiny()` 可以测试某个变量的最小值
\item \verb`<类型>(kind=..) :: <name>` 的括号中 kind = 可以省略, 直接写数字. 这个数字叫做 kind type parameter
\item \verb`kind()` 函数可以检查某个变量的字节数.
\item \verb`precision()` 函数可以检查某个变量的精度(小数位数)
\item 整型和整型相除得到整数, real 和 整型相除得到 real. real 赋值给 integer 向下取整.
\item \verb`1` 和 \verb`2` 都是整型! 尤其注意 \verb`1/2 = 0` 而不是 1.   至少要加个小数点 !
\item literal constant 指的是程序中写入的值, 例如 \verb`123, 'good', 1.7e-6` 等.
\item 声明矩阵的时候, 只能直接用 constant 指定矩阵大小和初值.
\end{itemize}

\subsubsection{real}
\begin{itemize}
\item 指定 literal constant 的类型: 1.3e8 只有 real(4) 的精度 (7位有效数字, 默认), 1.3d8 是 real(8) 的精度(16位有效数字, 即 double), 1.3q8 是 real(16) 的精度(33位有效数字). d 代表 double, q 代表 quad. 另外也可以用 \verb|1.3e8_16| 的格式 , 得到类型  real(16) 与 1.3q8 等效, 这么做的好处是可以用一个常数变量 (parameter) 来指定数据类型,例如 \verb|1.3e8_rk|
\item real 的默认是单精度 real(4). 双精度 real(8) 需要声明.
\item real 直接赋值溢出会出错, 算符运算结果溢出产生 Infinity
\item underflow, 例如 1e-40. 会产生 warning.
\item 强制类型转换 real (var, kind)
\end{itemize}

\subsubsection{integer}
\begin{itemize}
\item \verb`integer(kind = <n>)` 设置占用的字节, 可以取 1, 2, 4, 8 或者 16. 即最大值. 默认为 
   kind = 4. (huge 约 2E9)
\item 强制类型转换 int() 注意不是 integer()
\item \verb`nint()` 采用四舍五入进行类型转换, 而 int() 采用向下取整
\item literal constants 设定 kind 用下划线, 例如 \verb`3_4` 是 \verb`kind = 4` 的整数 3.
\end{itemize}

\subsubsection{other}
\begin{itemize}
\item literal constant 是 (<实部>, <虚部>). 赋值时如果需要包含变量, 用强制类型转换 cmplx (Re, Im, 8) 其中 kind = 8 是 double complex
\item literal constant 是 .true. 或 .false.
\item 强制类型转换 logical (var, kind)
\end{itemize}

\subsubsection{char}
\begin{itemize}
\item \verb`character (len = <n>)` 其中 n 是最大长度, 可以等于任意正整数, 默认为 1.  \verb`len =` 可以省略
\item character 不是一个数组, 是一个独立的变量, 所以不能只改变其中一个元素. 声明数组: character(5) :: char(3,3) 生成 3*3 的矩阵, 每个矩阵元是 len=5.
\item \verb`str(n:m)` 获取 character 类型变量 str 的第 n 到第 m 个字符, 注意不能用 str(n) 用而是用 str(n:n)
\item \verb`str(1,2)(n:m)` 获取 character 矩阵的 (1,2) 元的第 n 到第 m 个字符
\item \verb`str = 'some caracters'` 用于赋值(用双引号也行), 字符串左对齐, 剩下的位置补 null (ascii 的 0), 如果长度超出, 多出的部分被截去,且不会报错.
\item str = str1(n:m) 同样是左对齐,但是后面补空格
\item 要获取某个字符的 ascii 代码, 用 iachar(), 如 iachar(str(1,2)(3:3)) . 如果输入的变量有多于一个字符, 返回首字符的 ascii 代码
\item 用 \verb`//` 可以链接两个字符变量,然后把所有的 null 变为空格
\item 用 \verb`trim()` 可以把字符变量最后的空格或者 null 去掉并返回 len 较短的字符常量
\item \verb`len()` 可以用于返回字符变量/常量的长度, 如 len('1234 ') 返回 5, len(trim('1234 ')) 返回 4
\item \verb`write(var, *) var1, var2, ...` 可以像输出到命令行一样将字符串或变量写到 character 变量中
\end{itemize}

\subsection{数组和矩阵}
\subsubsection{基础}
\begin{itemize}
\item 声明矩阵例子:  \verb`integer, dimension (5,5) :: mymatrix`
\item 更简单地: 可以用 \verb`integer :: matrix(5,5)  或者 integer matrix(5,5)`
\item 如果用第二种格式, 可以在同一行声明很多不同尺寸的变量
\item 所有的小括号前面可以有空格.
\item 甚至可以指定矩阵的索引范围!   integer :: matrix(-3:1, 0:4). 这样 matrix(-3,0) 等表达都是合法的!
\item 但是要注意 \verb`a:b` 中 a <= b, 否则返回空矩阵 (某些维度的长度为 0)
\end{itemize}

\subsubsection{Dynamic Matrix}
\begin{itemize}
\item \item 可用于变化尺寸声明: \verb`real, allocatable :: Mat (:,:)`
\item 尺寸声明: \verb`allocate ( Mat1 (size1, size2), Mat2 (size1, size2))`.
\item 用完了以后: \verb`deallocate (Mat)` 养成好习惯, 务必写上.
\item 即使没有 \verb`implicit none`, 也需要 \verb`allocate`
\item allocate 以后, 所有元素都是 0.
\item 子程序中定义的 allocatable array 如果没有 save attribute, 在程序结束时会自动 deallocate.
\end{itemize}

\subsubsection{索引}
\begin{itemize}
\item \verb`rank()` 相当于 Matlab 的 ndims, 返回一个 integer
\item \verb`size()` 相当于Matlab的 numel. 但是 \verb`size(Mat, dim)` 相当于 Matlab 的 size
\item \verb`shape()` 相当于 Matlab 的 size, 返回一个 \verb`integer(2)` 数组.
\item "extent" 指的是一个维度上面的索引数量 (这并不是个函数)
\item \verb`lbound(Mat)` 返回在每个维度 index 的最小值, \verb`lbound (Mat, dim)` 返回指定维度的最小值
\item 矩阵的赋值和 Matlab 一样, \verb`matrix(1:2,3:4) = 1`
\item \verb`a((/1,3,4/))` 也是合法的
\item 如果直接 \verb`print` 矩阵, 会把所有列展开成一行.
\item \verb`n1 : n2 : step` 的格式叫做 subscript triplet
\end{itemize}

\subsubsection{赋值}
\begin{itemize}
\item 数组可以与常数相乘
\item 数组可以用一个常数赋值
\item array constructor 的格式 \verb`(/1,2,3,4/)` 相当于 Matlab 里面的中括号, 但只能表达一维矩阵. 例程 \verb`a(1:4,2) = (/1,2,3,4/)`. 里面可以是变量.
\item 逻辑矩阵可以用 1 和 0 来赋值. \verb`logical :: Barray(4) = (/1,1,0,1/)`, 
   或者 \verb`(/.true.,.true.,.false.,.true./)`
\item \verb`reshape` 可以指定高维度矩阵的赋值: 
   \verb`reshape ( (/5,9,6,10,8,12/), (/3,2/) )`, 生成 3 行 2 列的数组, 排序方式和 Matlab 相同
\item \verb`reshape (  array, shape, pad, order )` 可以按照矢量 order 指定的顺序来填充矩阵, 默认
\item reshape 的 pad 参数是用于 vector 不能把矩阵填满的情况下, 继续填满矩阵.
\item 用函数生成数列  \verb`(/ ( (ii**2+3), ii=0,N ) /)`.  可以在后面加上 *2 之类的. 这种格式中 N 不能超过 1e5. 否则只能用循环.
\end{itemize}

\subsubsection{数学运算}
\begin{itemize}
\item \verb`sum (array, dim, mask)` 所有元素求和
\item \verb`product (array, dim, mask)` 所有元素求积
\item \verb`dot_product(v1, v2)` 矢量点乘 (仅限于一维矢量)
\item \verb`matmul(mat1, mat2)` 矩阵乘法
\item \verb`maxval(array, dim, mask)` 矩阵的最大值 (不知道 mask 是干啥的)
\item \verb`minval` 矩阵的最小值
\item \verb`maxloc(Mat, mask)` 找到矩阵中最大元素的位置
\item \verb`minloc(Mat, mask)` 找到矩阵中最小元素的位置
\item \verb`transpose()` 求转置矩阵
\end{itemize}

\subsubsection{逻辑运算}
\begin{itemize}
\item \verb`all (logical, dim)` 判断逻辑矩阵是否都是 .true.  dim 可选
\item \verb`any (logical, dim)` 判断逻辑矩阵是否有任何 .true.
\item \verb`count (logical, dim) `
\item \verb`>, <, /=` 等算符用于两个矩阵返回逻辑数组. 进而使用 \verb`.and. .or.` 等
\item where statement : 
\verb`where ( a < 0 ); a = 1; elsewhere; a = 5; endwhere`
\end{itemize}

\subsubsection{矩阵处理}
\begin{itemize}
\item \verb`pack(Mat, mask)` 把 Mat 中对应 mask 中为 .true. 的矩阵元按顺序装到 rank1 向量. 如果被赋值的矢量有多余元素, 则随机赋值. 这相当于 Matlab 中的 Mat (mask)
\item \verb`v = pack(Mat, mask, vector)` 多余元素赋值为 vector 中的对应元素. vector 要和 v 的长度一致.
\item \verb`pack(M,.true.)` 相当于 Matlab 中的 M(:)
\item \verb`unpack(vector, mask, array)` 是 pack 的逆过程
\item \verb`spread(Mat, dim, ncopies)` 把 Mat 在 dim 的维度排列 ncopies 次. dim 只能从 1 到 \verb`rank(Mat) +1`
\item \verb`merge(MatTrue, MatFalse, mask)` 在两个矩阵中选取矩阵元组成新的矩阵. mask 是逻辑数组, 用于选择矩阵元.
\item \verb`cshift(Mat, shift, dim)` 是 circle shift, 把矩阵沿着 dim 的维度循环移动, shift > 0 是向 ind 变小的方向移动.
\item \verb`eoshift()` 非循环移动, 规则类似.
\end{itemize}

\subsection{全局变量}
\begin{itemize}
\item \verb`common/group/var1, var2, ...` 其中 var1, var2 的变量名在不同的声明中可以不一致, 但是 group 的名字要一样
\item 全局数组 \verb`common/grou/var1(N1), var2(N2)` 变量要在 common 之前再定义一次
\end{itemize}

\subsection{输入输出}

\subsubsection{命令行输入输出}
\begin{itemize}
\item \verb`print (<格式>), item1, item2, item3...`
\item 注意格式周围一定要有括号!!!
\item free format 格式 (list directed form)
\begin{lstlisting}
   print *, item1, item2...
   read (*,*) item1, item2... 
   write (*,*) item1, item2...
\end{lstlisting}
   星号代表默认, print 的星号可以替换成 \verb`<format>`, read 和 write 的第一个星号可替换成输入输出的位置代号 (默认是命令行), 第二个星号可替换成 \verb`<format>`
\item \verb`write (var, <format>)` 可以将内容写到变量中 (不确定是否只支持字符串变量)
\item \verb`<format>` 可以是一个数字, 如 1010, 这样, 必定有另一行指令 1010 format(...)
\item 一个 format 的例子如下
\begin{lstlisting}
write(6, 901) a, b, c
901 format(1X,I3/'X=',T7,F5.2/D8.2)
\end{lstlisting}
这两个命令相当于在命令行中输出 1 个空格, 一个整型(a), 下一行, 'X=', tab 到第 7 列, 一个 real (b), 下一行, 一个 real (c).
\item \verb`print <format>` 相当于 \verb`write(*,<format>)`
\item \verb`<format>` 是一个指定输出格式的字符串, 例如 '(10ES12.3)' , '(5I6)' 等.
\item 在命令行给 read 输入数据的时候, 用空格隔开, 按回车确认.

\verb`<format>` 字符串的定义
\item 任意格式中, 如果需要的空间超出列宽, 将显示为 \verb`****`

real type
\item 小数格式 <列数>F<列宽>.<小数位数>
\item real 类型的科学计数法  <列数>ES<列宽>.<小数位数>, 显示出来的位数 = <小数位数>+6
  所以<列宽> 需大于等于 <小数位数> +7 (正数+6, 负数带负号+7)
\item 用 / 表示空白行

integer type
\item <列数>I<列宽>.<显示位数>  对于 <显示位数>, 不够的前面用0补齐, 超出的不
   处理)
\item 如果 write/read 命令后面是一个 list, 那么可以给 list 的每一项设置格式, 用 '(<fmt1>, <fmt2>...)'  (我猜测<列数>在该情况仍然可以使用)

character type
\item <列数>A<列宽>

\subsection{文件输入输出}
\item 在 visual studio 中, 生成的文件与源文件保存在同一目录下.
\item 文件可以用 txt 编辑器打开, 不需要有拓展名或者可以有任何拓展名.
\item 创建新数据文件并写入数据如下
\begin{lstlisting}
open (1, file = 'data1.txt', status = 'new')
do ii = 1,100
	write(1,*) x(ii), y(ii)
end do
close (1)
\end{lstlisting}
\item 这里的 1 是文件在程序中的编号, 可以取 1-99, 除了 5 和 6 (分别代表从键盘读写).
\item 'new' 是创建新文件, 若打开已存在文件, 替换成 'old'
\item 如果不指定 status, 则相当于 status = 'unknown', 如果文件存在就用存在的文件, 不存在就创建新文件
\item 在读写文件时, 每个 write 语句写在新的一行, 每个 read 语句读取一行.

\subsection{Unformatted File}
\item 如果不需要人读取, 或者其他程序读取, 用这种文件效率更高
\item ifort 写入的时候不用打开, 但是关闭是好习惯
\begin{lstlisting}
open(unit=7, file='fort.7', form='Unformatted') 这行可以省略!
write(7) array1
write(7) array2
close(unit=7)
\end{lstlisting}
两个写入的矩阵尺寸可以不一样!

\item 读取方法
\begin{lstlisting}
open(unit=7, file='fort.7', form='Unformatted') 这行可以省略!
read(7) array1
read(7) array2
\end{lstlisting}
\end{itemize}

\subsection{常用函数}
\begin{itemize}
\item 已验证指数运算兼容复数, \verb`real` 和 \verb`imag` 兼容数组.
\item 数学函数 \verb`sqrt(), exp(), log10(), abs(), aimag()` 取虚部, \verb`**` 指数, 
\item 三角函数 \verb`sin(), cos(), tan(), asin(), acos(), atan(), atan2(), sinh(), cosh(), tanh(), atanh()`
\item \verb`real(array, kind),  imag(array, kind), iabs(int), sign(), isign(int)`
\item \verb`aimag(), dimag(), qimag()` 分别适用 kind = 4, 8, 16 的 complex.
\item \verb`amod(),  mod(),  idim(), dim()`
\item \verb`aint()` 向下取整, \verb`anint()` 最近取整, \verb`ceiling()`, \verb`floor()`, \verb`complx(x, y)` 生成复数, \verb`conjg()` 复共轭, \verb`dim(x, y)` 差值(正数)
\end{itemize}

\subsection{文字处理}
\begin{itemize}
\item \verb`index(string,subtring)` 返回 substring 在 string 中的位置
\item \verb`len(string)` 返回 string 的最大长度
\item \verb`achar()` 把 integer 转换成 character
\item \verb`iachar()` 把 character 转换成 integer
\item \verb`trim()` 把后面的空格清空. 应用  trim(char)//trim(char) 会消除中间多余的空格.
\item 更多文字处理见 \href{http://www.tutorialspoint.com/fortran/fortran_characters.htm}{这里}
\end{itemize}

\subsection{子程序}
\subsubsection{internal subroutine/function}
\begin{itemize}
\item 统称为 internal subprogram
\item 在主程序或者外部子程序或者模块中(统称为 host), internal subprogram 可以在 host 的 end 之前紧接着定义, 前面加上 contains 即可.
\item  stop <5位数/字符串> 可以终止程序, 且显示 <5位数/字符串>
\item internal subprogram 可以不用自变量而获取 host 中的所有变量
\item internal subprogram 不可以嵌套
\item 只有在 host 外定义的 subprogram 才需要声明 interface, 叫做 implicit interface, host 内的叫 explicit interface, 不能声明 interface.
\item 即使不需要使用 interface, 最好也用 external <subprogram-list> 声明一下, 这样如果有 intrinsic 的同名子函数, 就会覆盖.
\item 声明默认变量用 optional, 如 logical, intent(out), optional :: fail
\item 用 present() 检测 optional 变量是否被输入, 如 if (present(fail)) then
\end{itemize}

\subsubsection{subroutine}

\begin{itemize}
\item subroutine 可以任意修改其变量 (声明了 intent 的除外)
\item 声明: \verb`subroutine <name> ( arg1, arg2 ) .... end subroutine  <name>`
\item 调用:  \verb`call <name> (arg1, arg2)`
\item 同样, 如果加了 \verb`implicit none`,  需要声明变量 (包括输入变量).
\item 声明变量的用途: \verb`integer, dimension (5), intent (out) :: arg1, arg2`. 使用 \verb`intent(in)` 使变量不可修改, \verb`intent(out)` 在输入时重置 allocatable 矩阵 (常数及普通矩阵保持不变). intent(inout) 有输入且可以修改
\item 若函数自变量中有 allocatable 矩阵, 必须要在主函数中声 Interface
\item \verb`interface` 必须要在定义变量后面其他语句前面
\item 格式
\begin{lstlisting}
interface
    subroutine funName (Mat)
         integer,  intent (out):: Mat(:, :)
end subroutine funName
\end{lstlisting}
\item 同时在定义 subroutine 时也要使用 \verb`(:,:)` 声明变量维度, 冒号的个数代表维度.
\item interface 中只需声明变长度的变量, 其他变量可以省略
\item 用 integer, dimension(:, :) :: Mat 或 integer :: Mat(:, :) 等效
\item subroutine 中不能出现与 subroutine 名字同名的变量, 否则编译器会认为这是一个 function.
\item subroutine 如果需要输入矩阵, 调用的时候可以只输入矩阵的一个矩阵元.
\end{itemize}

\subsubsection{function}
\begin{itemize}
\item 定义 \verb`function(var1,var2)`
\item 如果声明了 implicit none, 那么应该声明一个与函数名同名的变量, 类型是输出变量的类型.
\item 如果返回变量有其他名字, 用 function(var1,var2) return <output>
\item 函数内部使用的输入变量名叫做 dummy arguments, 调用时输入的叫做 actual arguments. 两者必须是完全同一类型, 没有任何自动转换.
\item 函数的自变量不能做任何修改, 否则会出错.
\item 函数内外的变量互不通用. (黑盒子)
\item 函数中的一维数组不能超过 1e5 个元素, 否则会出错. 用 subroutine.
\item 函数一般进行一些小型运算, subroutine 才是大型运算.
\end{itemize}

\subsection{模块}

\begin{itemize}
\item module 可以是一个独立的文件, 编译的时候一起编译.
\item 调用模块命令:  \verb|use <module_name>|.  必须放在 implicit none 之前.
\item 模块文件格式
\begin{lstlisting}
  module <name>
  <变量声明>
  contains
      subroutine <name>
          ...
      function <name>
          ...
      interface
          ...
  end module <name>
\end{lstlisting}
\item contains 命令前面声明的变量可以在 module 内部和声明了 module 的任何程序中使用.
\item 如果 <变量声明> 中使用 private 声明, 则变量只在 module 内部通用. 例如
   integer, private :: a(5)
\item 模块中声明的变量, 可以在调用程序中修改. 所以可以通过模块来在函数间传递变量, 相当于 global.
\end{itemize}
