% 常微分方程解的存在、唯一及对参数的连续依赖定理
% keys 存在|唯一|连续依赖|常微分方程
% license Xiao
% type Tutor

\pentry{皮卡映射\upref{PicMap},基本知识(常微分方程)\upref{ODEPr}}
本节证明常微分方程的解的存在、唯一、及对参数的连续依赖定理。所谓 “解对参数的连续依赖” 是指微分方程
\begin{equation}\label{eq_ODEUC_2}
\dot x=v(x,t)~,
\end{equation}
 的解 $\varphi$ 也是某些参数 $\mu=(\mu_1,\cdots,\mu_m)$ 的函数,即 $\varphi(\mu,t)$。于是 \autoref{eq_ODEUC_2} 右边也因写成 $v(x,\mu,t)$。 即需要证明形如
\begin{equation}\label{eq_ODEUC_1}
\dot x=v(x,\mu,t)~
\end{equation}
的微分方程的存在唯一且对参数 $\mu$ 的连续依赖定理。然而,可以验证,对\autoref{eq_ODEUC_2} 证明存在唯一性和对初始点的连续依赖性,等价于证明\autoref{eq_ODEUC_1} 的存在唯一性和对参数 $\mu$ 的连续依赖性。事实上:
若微分方程\autoref{eq_ODEUC_2} 解 $\varphi$ 存在唯一且对初始点 $x$ 的连续依赖。记 
\begin{equation}
y=(x,\mu),\quad v=(v_x,v_{\mu})~,
\end{equation}
则 \autoref{eq_ODEUC_1} 等价于
\begin{equation}\label{eq_ODEUC_3}
\begin{aligned}
\dot y=v(y,t)~,
\end{aligned}
\end{equation}
其中 $v_{\mu}=0$。由假设,\autoref{eq_ODEUC_3} 的解 $y=\varphi(t)$存在唯一,且连续依赖于qi's


\subsection{存在、唯一及对参数的连续依赖定理}
\begin{theorem}{存在、唯一及对参数的连续依赖定理}
设微分方程
\begin{equation}
\dot x=v(x,\mu,t)~
\end{equation}
中向量场 $v$ 及其关于变量 $x$ 的导数 $v_*\equiv(\pdv{v^i}{x^j})$ 在其自变量所在欧氏空间中的区域 $\tilde\Gamma$ 上有定义且连续,则对任一点 $(x_0,\mu_0,t_0)\in\tilde\Gamma$,存在
\begin{equation}
\abs{}~
\end{equation}

\end{theorem}
