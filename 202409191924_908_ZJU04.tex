% 浙江大学 2004 年 考研 量子力学
% license Usr
% type Note

\textbf{声明}:“该内容来源于网络公开资料,不保证真实性,如有侵权请联系管理员”

\subsubsection{第一题(35分):}
\begin{enumerate}
    \item 由正则对易关系 $[\hat{x}, \hat{p}] = i\hbar$ 导出角动量的三个分量
    \[    L_x = y \frac{\partial}{\partial z} - z \frac{\partial}{\partial y}, \quad L_y = z \frac{\partial}{\partial x} - x \frac{\partial}{\partial z}, \quad L_z = x \frac{\partial}{\partial y} - y \frac{\partial}{\partial x}~\]
    的对易关系。
    
    \item 证明厄米算符的本征值为实数。
    
    \item 什么是量子力学中的守恒量,它们有什么性质。
    
    \item 写出测不准关系,并简要说明其物理含义。
    
    \item 写出泡利矩阵
    \[    \sigma^x =     \begin{pmatrix}    0 & 1 \\\\    1 & 0    \end{pmatrix},    \quad    \sigma^y =     \begin{pmatrix}    0 & -i \\\\    i & 0    \end{pmatrix},    \quad    \sigma^z =     \begin{pmatrix}    1 & 0 \\\\    0 & -1    \end{pmatrix} ~\]
    满足的对易关系。
\end{enumerate}
\subsection{第二题(30分):}
二维谐振子的哈密顿量为
\[H = \frac{1}{2m} \left( \hat{p}_x^2 + \hat{p}_y^2 \right) + \frac{1}{2} m \left( \omega_1 x^2 + \omega_2 y^2 \right)~\]

\begin{enumerate}
    \item 求出其能级。
    
    \item 给出基态波函数。
    
    \item 如果 $\omega_1 = \omega_2$,试求能级的简并度。
\end{enumerate}
\subsection{第三题(30 分):}
有一个质量为 \(m\) 的粒子处在如下势阱中

\[
V(x) = 
\begin{cases} 
\infty, & x < 0 \\
-V_0, & 0 < x < a \\
V_0, & a < x < a + b \\
0, & a + b < x
\end{cases}
~\]
(这里$V_0 > 0$)

1. 试求其能级与波函数。

2. 问通过调节势阱宽度 \(a\),能否让势阱中的粒子有一定的几率穿透出来。

3. 如果你认为可以,试确定参数 \(a\) 的取值范围。
\subsection{第四题(20 分):}

\subsection{第五题(20 分):}