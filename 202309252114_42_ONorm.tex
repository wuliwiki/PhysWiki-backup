% 线性算子度量空间
% keys 线性算子|范数|度量空间|完备性
% license Xiao
% type Tutor
\pentry{线性算子代数\upref{LiOper}}
线性算子是矢量空间上的线性映射(\upref{LiOper}),其不仅能在加法和数乘的定义下构成代数,而且能在其上定义范数使其变成完备度量空间。本节旨在说明后一论断。
\begin{definition}{}\label{def_ONorm_1}
设 $\mathcal A$ 是 $\mathbb R^n$ 上的线性算子,则其\textbf{范数}定义为
\begin{equation}
\norm{\mathcal A}=\sup_{x\neq0}\frac{\abs{Ax}}{\abs{x}}~.
\end{equation}
其中 $\abs{x}\equiv\sqrt{(x,x)}$ 是矢量 $x$ 的范数, $(\cdot,\cdot)$ 是 $\mathbb R^n$ 的内积。
\end{definition}
范数的\autoref{def_ONorm_1} 是几何意义恰好是算子 $\mathcal A$ 的最大\textbf{伸缩系数}(矢量模之比)。
\begin{example}{}
试证明上面定义的算子的范数满足范数的定义(\autoref{def_NormV_1}~\upref{NormV}),即:
\begin{enumerate}
\item $\norm{\mathcal A}\geq0$,且 $\norm{\mathcal A}=0$ 当且仅当 $\norm{\mathcal A}=0$;
\item $\norm{\lambda \mathcal A}=\abs{\lambda}\norm{\mathcal A}$;
\item 三角不等式:$\norm{\mathcal A+\mathcal B}\leq\norm{\mathcal A}+\norm{\mathcal B}$。
\end{enumerate}

\end{example}