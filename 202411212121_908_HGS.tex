% 克里斯蒂安·惠更斯(综述)
% license CCBYSA3
% type Wiki

本文根据 CC-BY-SA 协议转载翻译自维基百科\href{https://en.wikipedia.org/wiki/Christiaan_Huygens}{相关文章}。

\begin{figure}[ht]
\centering
\includegraphics[width=6cm]{./figures/759a661ac1a7d67e.png}
\caption{惠更斯肖像,由卡斯帕·内彻绘于1671年,现藏于莱顿博尔哈夫博物馆[1]} \label{fig_HGS_1}
\end{figure}

克里斯蒂安·惠更斯,泽尔亨领主,英国皇家学会院士(/ˈhaɪɡənz/,音译‘海根斯’,[2] 美国亦发音为 /ˈhɔɪɡənz/,音译‘霍伊根斯’;[3] 荷兰语:[ˈkrɪstijaːn ˈɦœyɣə(n)s] ⓘ;也拼作 Huyghens;拉丁语:Hugenius;1629年4月14日-1695年7月8日),是一位荷兰数学家、物理学家、工程师、天文学家和发明家,被视为科学革命中的关键人物之一。[4][5] 在物理学领域,惠更斯在光学和力学方面做出了开创性的贡献;作为天文学家,他研究了土星的光环并发现了土星最大的卫星——泰坦。作为工程师和发明家,他改进了望远镜的设计,并发明了摆钟,这种时钟在近300年内是最精确的计时工具。他是一位才华横溢的数学家和物理学家,其著作首次通过一组数学参数对物理问题进行了理想化描述,[6] 并首次对一种无法直接观测的物理现象进行了数学和机械论的解释。[7]

惠更斯在其著作《De Motu Corporum ex Percussione》中首次正确地确定了弹性碰撞的定律,该书完成于1656年,但于1703年才在他去世后出版。[8] 1659年,惠更斯在其著作《De vi Centrifuga》中以几何方法推导出了经典力学中描述离心力的公式,这比牛顿早了十年。[9] 在光学领域,他因提出光的波动理论而闻名,这一理论发表于其1690年的《光论》(Traité de la Lumière)。惠更斯的光波理论最初被牛顿的光微粒理论所取代,直到1821年,奥古斯丁-让·菲涅耳(Augustin-Jean Fresnel)改进了惠更斯的原理,完整解释了光的直线传播和衍射现象。今天,这一原理被称为“惠更斯-菲涅耳原理”。

1657年,惠更斯发明了摆钟,并于同年获得专利。他对钟表的研究最终在《摆动时钟》(Horologium Oscillatorium,1673年)中发表,该书被认为是17世纪关于力学的重要著作之一。[6] 虽然书中包含了钟表设计的描述,但大部分内容是对摆动运动的分析和曲线理论。1655年,惠更斯与其兄弟康斯坦丁(Constantijn)开始研磨透镜,制作折射望远镜。他发现了土星最大的卫星——泰坦,并首次解释了土星的奇特外观是由于“一个薄而平坦的环,其不与土星接触,并倾斜于黄道面”。[10] 1662年,惠更斯开发了如今称为“惠更斯目镜”的装置,这是一种采用两个透镜的望远镜,能够减少色散现象。[11]

作为数学家,惠更斯发展了渐伸线的理论,并在《赌博中的计算》(Van Rekeningh in Spelen van Gluck)中研究了几何概率和点数问题。该书由弗朗斯·范·斯库腾(Frans van Schooten)翻译并以《赌博中的推理》(De Ratiociniis in Ludo Aleae,1657年)出版。[12] 惠更斯及其他人对期望值的使用后来启发了雅各布·伯努利(Jacob Bernoulli)对概率论的研究。[13][14]
\subsection{传记}
\begin{figure}[ht]
\centering
\includegraphics[width=6cm]{./figures/805ceabebaf64426.png}
\caption{康斯坦丁与他的五个孩子合影(克里斯蒂安位于右上方)。海牙莫里茨皇家美术馆。} \label{fig_HGS_2}
\end{figure}
克里斯蒂安·惠更斯于1629年4月14日出生在海牙的一个富有且有影响力的荷兰家庭,[15][16] 是康斯坦丁·惠更斯的次子。他以祖父的名字命名。[17][18] 他的母亲苏珊娜·范·巴尔勒(Suzanna van Baerle)在生下惠更斯的妹妹后不久去世。[19] 夫妇俩育有五个孩子:康斯坦丁(1628年)、克里斯蒂安(1629年)、洛德维克(1631年)、菲利普斯(1632年)和苏珊娜(1637年)。[20]

康斯坦丁·惠更斯是奥兰治家族的外交官和顾问,同时也是一位诗人和音乐家。他与欧洲各地的知识分子有广泛的书信往来,他的朋友包括伽利略·伽利莱、马林·梅森和勒内·笛卡尔。[21] 克里斯蒂安在16岁之前接受家庭教育,从小喜欢玩弄磨坊和其他机器的模型。他从父亲那里接受了全面的教育,学习语言、音乐、历史、地理、数学、逻辑和修辞,同时还学习舞蹈、击剑和骑马。[17][20]

1644年,惠更斯的数学导师是扬·扬斯·斯塔姆皮恩(Jan Jansz Stampioen),他为15岁的惠更斯布置了一份有关当代科学的高难度阅读清单。[22] 后来,笛卡尔对他在几何学方面的能力印象深刻,梅森则称他为“新阿基米德”。[23][16][24]
\subsubsection{学生时代}
16岁时,康斯坦丁送克里斯蒂安·惠更斯到莱顿大学学习法律和数学,他从1645年5月学到1647年3月。[17] 从1646年开始,弗朗斯·范·斯库滕(Frans van Schooten)成为莱顿大学的一名学者,并在笛卡尔的建议下接替斯塔姆皮恩(Stampioen),成为惠更斯及其兄长康斯坦丁·小惠更斯的私人导师。[25][26] 范·斯库滕为惠更斯提供了最新的数学教育,向他介绍了韦达(Viète)、笛卡尔和费马(Fermat)的研究成果。[27]

1647年3月起,惠更斯继续在新成立的布雷达奥兰治学院(Orange College)学习两年,该学院的管理者之一是他的父亲康斯坦丁。康斯坦丁深度参与了这所学院的事务,但学院仅持续到1669年,校长是安德烈·里韦(André Rivet)。[28] 惠更斯在学习期间寄住在法学家约翰·亨里克·道伯(Johann Henryk Dauber)家中,并由英国讲师约翰·佩尔(John Pell)教授数学。他在布雷达的学习结束于其兄弟洛德维克因与另一名学生决斗事件而中断学业的时期。[5][29] 惠更斯于1649年8月完成学业后离开布雷达,并短暂以外交官身份随纳骚公爵亨利(Henry, Duke of Nassau)出使。[17] 这次任务带他去了本特海姆(Bentheim)和弗伦斯堡(Flensburg)。随后,他前往丹麦,访问了哥本哈根和赫尔辛格(Helsingør),并希望穿越厄勒海峡(Øresund)前往斯德哥尔摩拜访笛卡尔,但因笛卡尔此时已去世而未能成行。[5][30]

尽管他的父亲康斯坦丁希望克里斯蒂安成为一名外交官,但种种原因阻止了他的从政之路。1650年开始的第一次无护国主时期(First Stadtholderless Period)导致奥兰治家族失去权力,从而削弱了康斯坦丁的影响力。此外,他也意识到儿子对这条职业道路毫无兴趣。[31]
\subsubsection{早期通信}
\begin{figure}[ht]
\centering
\includegraphics[width=6cm]{./figures/4dfc68904d4ad9e9.png}
\caption{惠更斯手稿中悬链线(catenary)图的插图。} \label{fig_HGS_3}
\end{figure}
惠更斯通常用法语或拉丁语写作。[32] 1646年,他还是莱顿大学的一名学生时,就开始与父亲的朋友马林·梅森(Marin Mersenne)通信,但梅森在1648年不久后去世。[17] 1647年1月3日,梅森在给康斯坦丁的信中称赞其子在数学上的才能,甚至恭维地将他比作阿基米德。[33]

这些信件显示了惠更斯早期对数学的兴趣。1646年10月,他讨论了悬索桥,并证明悬链线并非伽利略所认为的抛物线。[34] 后来,惠更斯在1690年与戈特弗里德·莱布尼茨通信时,将这种曲线命名为“悬链线”(catenaria,catenary)。[35]

在接下来的两年间(1647–1648),惠更斯写给梅森的信件涵盖了多种主题,包括自由落体定律的数学证明、格雷瓜尔·德·圣文森特(Grégoire de Saint-Vincent)提出的圆的求积问题(惠更斯证明其错误)、椭圆的求长、抛射体的运动和振动弦问题。[36] 梅森当时关注的一些问题,例如摆线(他将托里拆利关于这一曲线的论文寄给惠更斯)、振动中心和引力常数,成为惠更斯在17世纪晚期才认真研究的课题。[6] 梅森还研究了音乐理论。惠更斯偏好中全音律(meantone temperament),他在31平均律上进行了创新(虽然这一想法并非全新,早在弗朗西斯科·德·萨利纳斯就已知晓),并使用对数对其进一步研究,显示其与中全音律的紧密关系。[37]

1654年,惠更斯回到父亲在海牙的家中,完全投入研究工作。[17] 他们家还有另一座夏天常用的宅邸,位于附近的霍夫维克(Hofwijck)。尽管惠更斯在学术上非常活跃,他的学术生活并未使他摆脱时而袭来的抑郁情绪。[38]

之后,惠更斯发展了一批广泛的通信伙伴,但自1648年法国的五年内乱(“投石党之乱”)以来,这种联系遇到了一些困难。1655年访问巴黎时,惠更斯拜访了伊斯梅尔·布利奥(Ismael Boulliau)自我介绍,对方带他见了克劳德·米隆(Claude Mylon)。[39] 梅森周围的巴黎学者团体在1650年代仍保持联系,而担任秘书角色的米隆费尽心思让惠更斯与他们保持沟通。[40] 通过皮埃尔·德·卡卡维(Pierre de Carcavi),惠更斯于1656年与他非常敬仰的皮埃尔·德·费马(Pierre de Fermat)建立通信。尽管这一经历令人又喜又忧也颇费解,因为显然费马已经淡出主流研究,某些优先权的主张可能也难以成立。此外,此时惠更斯正寻求将数学应用于物理,而费马则更关注纯数学问题。[41]
\subsubsection{科学首秀}
\begin{figure}[ht]
\centering
\includegraphics[width=6cm]{./figures/ab6894350d5911e1.png}
\caption{克里斯蒂安·惠更斯,尚-雅克·克莱里翁创作的浮雕(约1670年)。} \label{fig_HGS_4}
\end{figure}
像他的一些同时代人一样,惠更斯常常不急于将自己的研究成果和发现发表成文,而是更喜欢通过书信传播自己的工作。[42] 在他早期的日子里,他的导师弗朗斯·范·斯科滕(Frans van Schooten)为他的工作提供了技术反馈,同时出于声誉考虑表现得非常谨慎。[43]

在1651年至1657年间,惠更斯发表了一系列作品,展示了他在数学方面的才华以及他对经典几何和解析几何的精通,这也使他在数学家中获得了更广泛的影响和声誉。[33] 大约在同一时期,惠更斯开始质疑笛卡尔关于碰撞定律的理论。笛卡尔的碰撞定律大部分是错误的,而惠更斯通过代数方法推导出了正确的定律,后来又用几何方法验证。他证明了对于任何物体系统,该系统的重心的速度和方向保持不变,这就是惠更斯所谓的“运动量守恒”。虽然当时其他人也在研究碰撞现象,但惠更斯的碰撞理论更为普遍。[5] 这些研究成果成为进一步讨论的主要参考点,并通过通信和《学者杂志》(Journal des Sçavans)上的一篇短文传播开来,但直到1703年出版的《物体碰撞运动》(De Motu Corporum ex Percussione)中才被更广泛的公众所知晓。[45][44]

除了数学和力学上的成就,惠更斯还做出了重要的科学发现:1655年,他首次确认土卫六是土星的一颗卫星;1657年,他发明了摆钟;1659年,他解释了土星奇怪的外观是由于其环形结构。这些发现使他在整个欧洲声名远扬。[17] 1661年5月3日,惠更斯与天文学家托马斯·斯特里特(Thomas Streete)和理查德·里夫(Richard Reeve)在伦敦使用里夫的望远镜观测了水星经过太阳的凌日现象。[46] 之后,斯特里特与赫维留斯(Hevelius)的记录发生了争论,这一争论由亨利·奥登堡(Henry Oldenburg)调解。[47] 惠更斯将耶利米·霍罗克斯(Jeremiah Horrocks)关于1639年金星凌日的手稿传递给赫维留斯,该手稿首次于1662年印刷出版。[48]

同年,罗伯特·莫雷爵士(Sir Robert Moray)向惠更斯寄送了约翰·格朗特(John Graunt)的生命表。不久后,惠更斯与他的兄弟洛德维克(Lodewijk)一起研究了寿命预期问题。[42][49] 惠更斯最终在假设死亡率均匀的前提下绘制出了首个连续分布函数图,并利用该图解决了联合年金的问题。[50] 与此同时,惠更斯也对西蒙·斯蒂文(Simon Stevin)的音乐理论产生了兴趣(惠更斯会演奏羽管键琴),但他对发表自己关于和声的理论兴趣不大,其中一些理论甚至在几个世纪内被遗忘。[51][52] 由于他对科学的贡献,伦敦皇家学会于1663年选举惠更斯为会员,使他成为该学会历史上首位外国成员,当时他仅34岁。[53][54]
\subsubsection{法国}
\begin{figure}[ht]
\centering
\includegraphics[width=6cm]{./figures/a624697b8c8e9004.png}
\caption{惠更斯位于画面中心偏右,《科学院的成立与天文台的创建,1666年》,作者亨利·泰斯特兰(Henri Testelin),约1675年创作。} \label{fig_HGS_5}
\end{figure}
蒙莫尔学会(Montmor Academy)成立于17世纪50年代中期,是梅森圈(Mersenne circle)在梅森去世后发展的形式。[55] 惠更斯参与了学会的辩论,并支持那些主张通过实验验证来避免业余态度的人。[56] 1663年,惠更斯第三次访问巴黎;当蒙莫尔学会于次年解散时,惠更斯提倡在科学上实施更符合培根主义的计划。两年后,即1666年,他受邀迁居巴黎,担任路易十四新成立的法国科学院(Académie des sciences)的领导职务。[57]

在巴黎的科学院期间,惠更斯得到了路易十四的首席大臣让-巴蒂斯特·柯尔贝尔(Jean-Baptiste Colbert)的重要支持和通信。[58] 然而,他与法国科学院的关系并不总是顺利的。1670年,因病重而担心自己可能去世,惠更斯选择弗朗西斯·弗农(Francis Vernon)来将他的研究资料捐赠给伦敦皇家学会。[59] 然而,法荷战争(1672–1678)及英国在其中的角色可能对他与皇家学会后来的关系产生了负面影响。[60] 皇家学会代表罗伯特·胡克(Robert Hooke)在1673年处理相关事务时缺乏应有的技巧。[61]

1671年,物理学家兼发明家丹尼斯·帕平(Denis Papin)成为惠更斯的助手。[62] 他们的一个合作项目是火药发动机,但未能直接取得成果。[63][64] 在科学院期间,惠更斯利用1672年刚建成的天文台进行进一步的天文观测。他在1678年将尼古拉斯·哈茨苏克尔(Nicolaas Hartsoeker)引荐给法国科学家,如尼古拉·马勒伯朗士(Nicolas Malebranche)和乔瓦尼·卡西尼(Giovanni Cassini)。[5][65]

1672年,年轻的外交官莱布尼茨在访问巴黎时与惠更斯会面,当时他正在徒劳地尝试与法国外交部长阿尔诺·德·蓬波纳(Arnauld de Pomponne)见面。莱布尼茨当时正在研究计算机,并在1673年初短暂访问伦敦后,从1673年到1676年间向惠更斯学习数学。[66] 在接下来的多年中,两人展开了广泛的通信。起初,惠更斯对接受莱布尼茨的微积分优势表现出一定的犹豫。[67]
\subsubsection{晚年}
\begin{figure}[ht]
\centering
\includegraphics[width=8cm]{./figures/ab39de254979c92f.png}
\caption{} \label{fig_HGS_6}
\end{figure}
1681年,惠更斯在经历了一次严重的抑郁症发作后搬回了海牙。1684年,他发表了关于他新发明的无筒空中望远镜的著作《简便天文镜》(Astroscopia Compendiaria)。1685年,他试图重返法国,但因《南特敕令》的废除而未能成行。1687年,他的父亲去世,他继承了霍夫维克庄园(Hofwijck),并于次年将其作为自己的住所。[31]

在他第三次访问英格兰期间,1689年6月12日,惠更斯与艾萨克·牛顿(Isaac Newton)亲自会面。他们讨论了冰洲石,并随后就阻力运动展开通信。[68]

在生命的最后几年,惠更斯重新关注数学领域,并于1693年观察到现在被称为“镶边”(flanging)的声学现象。[69] 两年后,即1695年7月8日,惠更斯在海牙去世,并与他的父亲一样被安葬在海牙大教堂(Grote Kerk)的一座无标记墓穴中。[70]

惠更斯终身未婚。[71]
\subsection{数学}
惠更斯因其数学研究首次在国际上获得声誉,他发表了许多重要成果,引起了众多欧洲几何学家的关注。[72] 惠更斯在出版的作品中更倾向于使用阿基米德的方法,尽管他在个人笔记中更广泛地使用了笛卡尔的解析几何和费马的微小量技术。[17][27]
\subsubsection{出版作品}
\textbf{《面积求积定理》(Theoremata de Quadratura)}
\begin{figure}[ht]
\centering
\includegraphics[width=6cm]{./figures/2adc8baae1776ba2.png}
\caption{惠更斯的首部出版物属于面积求积领域。} \label{fig_HGS_7}
\end{figure}
惠更斯的第一本出版作品是《双曲线、椭圆和圆的面积求积定理》(Theoremata de Quadratura Hyperboles, Ellipsis et Circuli),由莱顿的爱尔维修出版商于1651年出版。[42] 该书的第一部分包含了计算双曲线、椭圆和圆的面积的定理,与阿基米德关于圆锥曲线的研究(特别是《抛物线求积》)相呼应。[33] 第二部分包括对格雷戈瓦·德·圣文森特(Grégoire de Saint-Vincent)关于圆面积求积的主张的反驳,这些主张是他此前与梅森讨论过的。

惠更斯证明了任意双曲线、椭圆或圆的弧段的重心与该弧段的面积直接相关。他进而展示了内接于圆锥曲线的三角形与这些曲线的重心之间的关系。通过将这些定理推广到所有圆锥曲线,惠更斯扩展了经典方法,从而得出了一些新的结果。[17]

在17世纪50年代,面积求积是一个热门话题。通过梅隆(Mylon),惠更斯参与了关于托马斯·霍布斯(Thomas Hobbes)数学理论的讨论。他坚持解释霍布斯所犯的错误,从而赢得了国际声誉。[73]

\textbf{《圆的大小之发现》}

惠更斯的下一部出版作品是《圆的大小之发现》(De Circuli Magnitudine Inventa),于1654年出版。在这本书中,惠更斯缩小了阿基米德在《圆的测量》中提出的内接与外接多边形之间的差距,表明圆周与直径的比值(即π)必须位于该区间的前三分之一内。[42]

通过一种相当于理查森外推法(Richardson extrapolation)的技术,[74] 惠更斯缩短了阿基米德方法中使用的不等式。这次,他利用抛物线弧段的重心来逼近圆弧段的重心,从而更快、更准确地近似圆的面积求积问题。[75] 通过这些定理,惠更斯得到了两组π的值:第一组介于3.1415926与3.1415927之间,第二组介于3.1415926533与3.1415926538之间。[76]

惠更斯还表明,对于双曲线,使用抛物线弧段进行相同的近似可提供一种快速简便的方法来计算对数。[77] 在书的末尾,他附加了一些经典问题的解答集,并以《若干著名问题的构造》(Illustrium Quorundam Problematum Constructiones)为标题。[42]

\textbf{《关于博弈中推理的方法》}

1655年,惠更斯访问巴黎后,对博弈论产生了兴趣,他此前已接触过费马、布莱兹·帕斯卡和吉拉尔·德萨尔格的相关研究。[78] 最终,他在《关于博弈中推理的方法》(De Ratiociniis in Ludo Aleae)中发表了当时对博弈进行数学分析最为系统的阐述。[79][80] 该书的原始荷兰语手稿由弗朗斯·范·斯科滕(Frans van Schooten)翻译为拉丁文,并于1657年收入他的《数学练习集》(Exercitationum Mathematicarum)中出版。[81][12]

此书包含早期的博弈论思想,特别讨论了“点数问题”(problem of points)。[14][12] 惠更斯从帕斯卡那里借用了“公平游戏”和公平契约的概念(即在概率相等时平分),并将其拓展为一种非标准的期望值理论。[82] 他成功地将代数应用于看似无法被数学家掌控的机遇领域,展示了将欧几里得式的综合证明与维耶特和笛卡尔作品中的符号推理相结合的强大能力。[83]

惠更斯在书的末尾附上了五道极具挑战性的难题,这些问题在接下来的六十年中成为任何想展示自己在博弈数学中技巧的标准考验。[84] 曾研究这些问题的人包括亚伯拉罕·德·莫阿夫、雅各布·伯努利、约翰内斯·胡德、巴鲁克·斯宾诺莎和莱布尼茨。
\subsubsection{未发表的作品}
\begin{figure}[ht]
\centering
\includegraphics[width=8cm]{./figures/2979e3bfad1a1579.png}
\caption{惠更斯关于浮动长方体稳定性的研究结果。} \label{fig_HGS_8}
\end{figure}
惠更斯早年曾完成了一部以阿基米德的《论浮体》为模板的手稿,题为《论浮于液体之物》(De Iis quae Liquido Supernatant)。这部手稿写于约1650年,共由三卷组成。虽然他将完整的作品寄给弗朗斯·范·斯科滕(Frans van Schooten)征求意见,但最终惠更斯选择不予发表,并曾一度建议将其烧毁。[33][85] 其中的一些研究成果直到18世纪和19世纪才被重新发现。[8]

惠更斯首先通过巧妙应用托里切利原理(即系统中的物体只有在其重心下降时才会移动)重新推导了阿基米德关于球体和抛物体稳定性的解法。[86] 随后,他证明了一个普遍定理:对于处于平衡状态的浮体,其重心与其浸没部分之间的距离达到最小值。[8] 惠更斯利用这一定理,提出了关于圆锥体、平行六面体和圆柱体稳定性的原创解法,在某些情况下包括完整的旋转周期。[87] 他的研究方法实际上等同于虚功原理。此外,惠更斯是第一个认识到,对于这些均匀固体,其比重和形状比例是决定流体静力学稳定性的关键参数的人。[88][89]
\subsection{自然哲学}
在笛卡尔和牛顿之间的时期,惠更斯是欧洲最重要的自然哲学家。[17][90] 然而,与许多同时代人不同,惠更斯对宏大的理论或哲学体系不感兴趣,并且通常避免处理形而上学问题(如果必须表态,他倾向于支持当时的笛卡尔哲学)。[7][33] 相比之下,惠更斯擅长扩展其前辈(如伽利略)的工作,从而推导出尚未解决但可以用数学分析的方法处理的物理问题。尤其是,他寻求依赖物体之间接触的解释,避免涉及远距离作用的概念。[17][91]

与罗伯特·波义耳(Robert Boyle)和雅克·罗奥尔特(Jacques Rohault)一样,惠更斯在巴黎期间提倡一种实验导向的机械自然哲学。[92] 早在1661年首次访问英格兰时,惠更斯就在格雷沙姆学院的一次会议上了解到波义耳的空气泵实验。不久之后,他重新评估了波义耳的实验设计,并开发了一系列实验来测试一个新的假设。[93] 这一过程持续了数年,揭示了一些实验和理论问题,直到他成为皇家学会会员前后才告一段落。[94] 尽管波义耳实验结果的复制过程混乱且有争议,惠更斯最终接受了波义耳关于真空的观点,这一观点与笛卡尔对真空的否认相对立。[95]

牛顿对约翰·洛克的影响是通过惠更斯间接实现的。惠更斯向洛克保证牛顿的数学是可靠的,这促使洛克接受了以微粒-机械论为基础的物理学。[96]
\subsubsection{运动、碰撞与引力定律}
\textbf{弹性碰撞}
\begin{figure}[ht]
\centering
\includegraphics[width=8cm]{./figures/c1fdfc4cde0b753d.png}
\caption{在惠更斯的《全集》(Oeuvres Complètes)中,使用划船隐喻来思考相对运动,以简化碰撞理论。} \label{fig_HGS_9}
\end{figure}
机械哲学的整体方法是提出现在称为“接触作用”的理论。 惠更斯采用了这种方法,但他也意识到其局限性,[97] 而他的学生莱布尼茨后来在巴黎放弃了这种方法。[98] 通过这种方式理解宇宙使得碰撞理论成为物理学的核心,因为只有涉及物质运动的解释才能被真正理解。虽然惠更斯受到笛卡尔方法的影响,但他并不拘泥于教条。[99] 他在17世纪50年代研究了弹性碰撞,但延迟了十多年才发表成果。[100]

惠更斯很早就得出结论,认为笛卡尔关于弹性碰撞的定律大部分是错误的。他制定了正确的定律,包括硬体的质量乘以速度平方的积守恒,以及所有物体在某一方向上的运动量守恒。[101] 一个重要的进展是他认识到伽利略不变性在这些问题中的重要性。[102] 惠更斯在1652年至1656年间完成了一份题为《物体碰撞运动》(De Motu Corporum ex Percussione)的手稿,详细推导了碰撞定律,但他的成果需要许多年才能广泛传播。1661年,他亲自将这些成果传递给了伦敦的威廉·布朗克(William Brouncker)和克里斯托弗·雷恩(Christopher Wren)。[103] 在1666年英荷第二次战争期间,斯宾诺莎写信给亨利·奥尔登堡提到这些成果时表现得非常谨慎。[104] 战争于1667年结束,惠更斯于1668年向皇家学会宣布了他的研究结果。随后,他于1669年在《学者杂志》(Journal des Sçavans)上发表了这些成果。[100]

\textbf{离心力}

1659年,惠更斯发现了重力加速度常数,并以二次形式阐述了如今被称为牛顿运动定律第二定律的内容。他通过几何方法推导出现在标准形式的离心力公式,用于描述在旋转参考系中观察到的物体所受到的离心力,例如在汽车转弯时的情形。用现代符号表示为:
\[{\displaystyle F_{c}={m\ \omega ^{2}}{r}}~\]
其中,\(m\) 是物体的质量,\(\omega\)是角速度,\(r\)是半径。[8] 惠更斯将其研究成果汇集成了一篇名为《离心力论》(De vi Centrifuga)的论文,该论文直到1703年才发表。在这篇论文中,他利用自由落体的运动学首次构建了关于力的广义概念,这一成果早于牛顿的相关研究。[106]

\textbf{引力}

然而,关于离心力的概念在1673年被发表,这是天文学中研究轨道运动的一个重要进展。这一研究促成了从开普勒行星运动第三定律向引力反平方定律的过渡。[107] 然而,惠更斯对牛顿引力研究的解释与罗杰·科茨等牛顿派的看法不同:他并不坚持笛卡尔的先验态度,但也不接受无法原则上归因于粒子之间接触的引力作用。[108]

惠更斯采用的方法也遗漏了一些数学物理学的核心概念,而这些概念被其他学者注意到了。在他的钟摆研究中,惠更斯非常接近简单谐振动理论;但这一主题首次被完整阐述是在牛顿1687年《自然哲学的数学原理》(Principia Mathematica)第二卷中。[109] 1678年,莱布尼茨从惠更斯关于碰撞的研究中提取出一个隐含的思想,即守恒定律。[110]
\subsubsection{钟表学}
\textbf{摆钟}
\begin{figure}[ht]
\centering
\includegraphics[width=8cm]{./figures/2b692a1767f03764.png}
\caption{弹簧驱动的摆钟,由惠更斯设计并由萨洛蒙·科斯特于1657年制造,[111] 连同《摆动时钟》(Horologium Oscillatorium,1673年)的副本,[112] 收藏于莱顿的布尔哈夫博物馆(Museum Boerhaave)。} \label{fig_HGS_10}
\end{figure}
1657年,受先前关于钟摆作为调节机制研究的启发,惠更斯发明了摆钟。这一发明是计时领域的突破,成为近300年来(直到1930年代)最精确的计时器。[113] 摆钟的精确度远高于当时的传统锚形擒纵器和摆轮钟表,立即受到欢迎,并迅速在欧洲传播。在此之前的钟表每天会误差约15分钟,而惠更斯的摆钟每天仅误差约15秒。[114] 尽管惠更斯为其钟表设计申请了专利,并将钟表的制造合同交由海牙的萨洛蒙·科斯特(Salomon Coster)负责,[115] 他并未从这项发明中赚取大量收益。皮埃尔·塞吉耶(Pierre Séguier)拒绝给予他任何法国方面的权利,而1658年鹿特丹的西蒙·多夫(Simon Douw)和伦敦的阿哈苏鲁斯·弗洛曼蒂尔(Ahasuerus Fromanteel)均仿制了他的设计。[116] 最古老的已知惠更斯式摆钟可以追溯到1657年,现收藏于莱顿的布尔哈夫博物馆(Museum Boerhaave)。[117][118][119][120]

发明摆钟的部分动机是为了制造一种精确的航海天文钟,用以在海上航行中通过天文导航测定经度。然而,由于船只摇摆的运动干扰了钟摆的运行,这种钟表作为航海计时器并不成功。1660年,洛德维克·惠更斯(Lodewijk Huygens)在前往西班牙的航行中进行了试验,但他报告称恶劣天气使钟表完全无法使用。1662年,亚历山大·布鲁斯(Alexander Bruce)加入了这一领域,惠更斯随即邀请罗伯特·莫雷爵士(Sir Robert Moray)和皇家学会介入,以调解和保护他的部分权利。[121][117]

试验持续到1660年代,其中最好的消息来自1664年英国皇家海军船长罗伯特·霍尔姆斯(Robert Holmes)的一次行动,他在对抗荷兰殖民地的任务中进行了试验。[122] 然而,丽莎·贾丁(Lisa Jardine)对此试验结果的准确性表示怀疑,因为当时塞缪尔·佩皮斯(Samuel Pepys)也对此表达了质疑。[123]

法国科学院在一次前往法属圭亚那的探险中进行的试验也以失败告终。让·里歇(Jean Richer)建议对地球形状进行修正。到了1686年荷兰东印度公司前往好望角的探险时,惠更斯已经能够提供这一修正的补充结果。[124]

\textbf{摆动时钟}
\begin{figure}[ht]
\centering
\includegraphics[width=10cm]{./figures/af8cf7fee6097614.png}
\caption{显示曲线渐屈线的示意图} \label{fig_HGS_11}
\end{figure}
在摆钟发明16年后,1673年,惠更斯发表了他关于钟表学的重要著作《摆动时钟》(Horologium Oscillatorium: Sive de Motu Pendulorum ad Horologia Aptato Demonstrationes Geometricae,意为《摆钟:或关于摆的运动应用于钟表的几何论证》)。这是第一部通过将物理问题理想化为一组参数并进行数学分析的现代力学著作。[6]

惠更斯的动机来自梅森(Mersenne)和其他人提出的观察,即摆并非完全等时的:其周期依赖于摆幅的大小,大摆幅的周期稍长于小摆幅。[125] 他通过解决一个称为等时性问题(tautochrone problem)的难题来应对这一现象,即研究在重力作用下,无论起点如何,质量沿曲线下滑所需时间始终相等的轨迹。他通过几何方法(这些方法预示了微积分的出现)证明该曲线为摆线(cycloid),而非摆球通常的圆弧轨迹。因此,为了实现等时性,摆需要沿摆线运动。解决这一问题所需的数学分析促使惠更斯发展了他的演化曲线理论,并在《摆动时钟》第三部分中提出了该理论。[6][126]

他还解决了梅森早些时候提出的另一个问题:如何计算由任意形状刚体组成的单摆的周期。这涉及找到振动中心及其与支点的互为倒数关系。在同一部著作中,他还分析了锥摆(conical pendulum),即通过离心力概念分析在圆周上运动的线悬重物。[6][127]

惠更斯是第一个推导出理想数学摆(具有无质量杆或绳,且长度远大于摆幅)的周期公式的人,其现代形式为:
\[{\displaystyle T=2\pi {\sqrt {\frac {l}{g}}}}~\]
其中,\(T\) 表示周期,\(l\) 表示摆的长度,\(g\) 表示重力加速度。通过对复摆振动周期的研究,惠更斯为转动惯量概念的发展做出了关键贡献。[128]

惠更斯还观察到了耦合振动现象:当两个摆钟并排安装在同一个支架上时,它们常常会同步摆动,但方向相反。他通过信件向皇家学会报告了这一结果,在学会的会议记录中,这一现象被称为“某种奇特的共鸣”。[129] 这一现象如今被称为同步现象(entrainment)。[130]

\textbf{螺旋游丝怀表}
\begin{figure}[ht]
\centering
\includegraphics[width=8cm]{./figures/dc017d256a866968.png}
\caption{惠更斯发明的螺旋游丝示意图} \label{fig_HGS_12}
\end{figure}
1675年,在研究摆线的振动特性时,惠更斯通过几何学和高级数学的结合,成功将摆线摆转换为振动弹簧。[131] 同年,惠更斯设计了一种螺旋游丝,并为一款怀表申请了专利。这些怀表的显著特点是没有用于平衡发条扭矩的链轮装置(fusee)。这表明惠更斯认为螺旋游丝可以像他钟表中的摆线形悬挂约束摆那样,使摆轮实现等时性。[132]

后来,他将螺旋游丝应用于更为传统的手表,这些手表由巴黎的图雷特(Thuret)为他制造。这种游丝在现代使用独立杠杆擒纵机构的手表中是必不可少的,因为它们可以调节以实现等时性。然而,惠更斯时代的手表使用效率极低的锚形擒纵器(verge escapement),这一机构会干扰任何形式的游丝(包括螺旋游丝)的等时性能。[133]

惠更斯的设计几乎与罗伯特·胡克(Robert Hooke)的设计同时问世,但两者是独立发展的。关于游丝优先发明权的争议持续了几个世纪。2006年2月,在英格兰汉普郡的一个柜子中发现了胡克几十年来皇家学会会议的手写笔记的遗失副本,这似乎为胡克的优先权提供了支持的证据。[134][135]
\subsubsection{光学}
\textbf{屈光学}
\begin{figure}[ht]
\centering
\includegraphics[width=6cm]{./figures/4eeada1352a0ee32.png}
\caption{惠更斯的空中望远镜,摘自《简明天文观测》(Astroscopia Compendiaria,1684年)} \label{fig_HGS_13}
\end{figure}
惠更斯长期对光的折射、透镜及屈光学的研究抱有浓厚兴趣。[136] 早在1652年,他便撰写了一部关于屈光理论的拉丁文论述的初稿,名为《光学论》(Tractatus),其中包含了关于望远镜的全面且严谨的理论。惠更斯是为数不多的提出望远镜性质及其工作原理理论问题的学者之一,几乎也是唯一将数学才能直接应用于天文学仪器实际使用中的研究者。[137]

惠更斯多次向同僚宣布即将出版这部著作,但最终将其推迟,以便撰写一部更全面的著作,取名为《屈光学》(Dioptrica)。这部作品分为三个部分:第一部分集中讨论折射的一般原理,第二部分涉及球差和色差,第三部分涵盖望远镜和显微镜的所有构造方面。与笛卡尔的《屈光学》仅讨论理想的透镜(如椭圆形和双曲形)不同,惠更斯专门研究了球面透镜,因为这是当时唯一能够实际制造并用于显微镜和望远镜的透镜类型。[138]

惠更斯还提出了一些实用方法以减少球差和色差的影响,例如采用长焦距作为望远镜物镜、使用内置光圈以减小孔径,以及设计一种新型目镜,即惠更斯目镜(Huygenian eyepiece)。[138] 《屈光学》在惠更斯生前从未出版,直到1703年才问世,但其中的大部分内容在当时科学界已广为人知。

\textbf{透镜}

1655年,惠更斯与其兄弟康斯坦丁(Constantijn)开始自己研磨透镜,以改进望远镜。[139] 1662年,他设计了如今被称为惠更斯目镜(Huygenian eyepiece)的装置,这是一组由两个平凸透镜组成的望远镜目镜。[140][141] 惠更斯制造的透镜以卓越的品质著称,其打磨始终符合他的规格要求;然而,他的望远镜成像并不十分清晰,这导致有人猜测他可能患有近视。[142]

透镜也是惠更斯在17世纪60年代与斯宾诺莎(Spinoza)社交时的共同兴趣点,后者是一名专业的透镜研磨师。他们对科学的看法存在相当大的差异,斯宾诺莎更倾向于笛卡尔主义。他们的一些讨论保留在书信中。[143] 在显微镜领域,惠更斯接触到了另一位透镜研磨师安东尼·范·列文虎克(Antoni van Leeuwenhoek)的研究,这一领域也引起了他父亲的兴趣。[6] 惠更斯还研究了透镜在投影设备中的应用。他被认为是幻灯机的发明者,并在1659年的信件中对此进行了描述。[144] 不过,也有其他人被认为发明了类似的幻灯装置,例如贾姆巴蒂斯塔·德拉·波塔(Giambattista della Porta)和科内利斯·德雷贝尔(Cornelis Drebbel);然而,惠更斯的设计采用了透镜以实现更好的投影效果(阿塔纳修斯·基歇尔(Athanasius Kircher)也被认为对此有所贡献)。[145]

\textbf{《光论》}
\begin{figure}[ht]
\centering
\includegraphics[width=8cm]{./figures/09f6273dc8cd08b2.png}
\caption{平面波的折射,使用惠更斯在《光论》(Traité de la Lumière,1690年)中提出的原理进行解释。} \label{fig_HGS_14}
\end{figure}
惠更斯在光学领域中因其光的波动理论而备受铭记。他于1678年首次向巴黎科学院(Académie des sciences)介绍了这一理论。最初作为《屈光学》(Dioptrica)的一章,惠更斯的光学理论最终于1690年以《光论》(Traité de la Lumière)的形式出版。这本书包含了对不可观测物理现象(即光的传播)的第一个完全数学化和机械化的解释。[7][147] 惠更斯在书中提到伊尼亚斯-加斯东·帕尔迪斯(Ignace-Gaston Pardies),他的光学手稿对惠更斯的波动理论提供了帮助。[148]

当时的主要挑战是解释几何光学,因为大多数物理光学现象(如衍射)尚未被观察到或视为问题。1672年,惠更斯对冰洲石(一种方解石)中的双折射现象进行了实验,这一现象由拉斯穆斯·巴托林(Rasmus Bartholin)于1669年发现。起初,他无法解释其发现,但后来通过其波前理论和渐屈线概念成功加以说明。[147] 他还提出了关于焦散线的理论。[6] 惠更斯认为光速是有限的,这一观点基于1677年奥勒·克里斯滕森·罗默(Ole Christensen Rømer)的报告,但据推测,惠更斯可能早已相信这一点。[149]

惠更斯的理论认为,光是以波前形式辐射的,光线常被视为与波前垂直传播的路径。波前的传播被解释为每个波前点都会发射球面波的结果(这被称为惠更斯-菲涅尔原理)。[150] 这一理论假设了无处不在的以太介质,通过完全弹性的粒子进行传输,这是对笛卡尔观点的修订。因此,他认为光的本质是一种纵波。[149]

惠更斯的光学理论在当时并未得到广泛接受,而牛顿在其1704年出版的《光学》(Opticks)中提出的光的粒子理论获得了更多支持。对惠更斯理论的一个主要反对意见是,纵波只有单一的极化,无法解释观察到的双折射现象。然而,1801年托马斯·杨(Thomas Young)的干涉实验和1819年弗朗索瓦·阿拉戈(François Arago)发现泊松亮斑(Poisson spot)时,牛顿的粒子理论或其他任何粒子理论都无法解释这些现象,从而使惠更斯的波动理论和模型重新受到关注。菲涅尔(Fresnel)了解到惠更斯的研究后,于1821年证明光实际上是一种横波,而非假设中的纵波,这也解释了双折射现象。[151]

后被称为惠更斯-菲涅尔原理的理论成为物理光学发展的基础,能够解释光传播的所有方面,直到麦克斯韦的电磁理论以及量子力学和光子的发现使光学进一步发展。[138][152]
\subsubsection{天文学}
\textbf{《土星体系》(Systema Saturnium)}
\begin{figure}[ht]
\centering
\includegraphics[width=8cm]{./figures/1acd339a169a9d16.png}
\caption{惠更斯对土星外观的解释,摘自《土星体系》(Systema Saturnium,1659年)。} \label{fig_HGS_15}
\end{figure}
1655年,惠更斯发现了土星的第一颗卫星——泰坦(Titan),并使用自己设计的放大倍率为43倍的折射望远镜观测并绘制了猎户座星云的图像。[11][10] 他成功地将星云分解为不同的恒星(星云较亮的内部区域现在被命名为“惠更斯区域”以纪念他),并发现了几片星际星云以及一些双星系统。[153] 他也是第一个提出土星的外观之谜是由“一个薄而平的环形物体形成的,这个环与土星不接触,并且倾斜于黄道平面”的人。[154]

三年多后,1659年,惠更斯在《土星体系》(Systema Saturnium)中发表了他的理论和研究成果。这被认为是自伽利略五十年前的《星际信使》(Sidereus Nuncius)以来,关于望远镜天文学最重要的著作。[155] 《土星体系》远不仅仅是关于土星的报告,惠更斯提供了行星距离太阳的相对测量值,首次引入了微测量仪(micrometer)的概念,并展示了一种测量行星角直径的方法,从而使望远镜不仅仅作为观测工具,而成为一种测量天体的仪器。[156] 他也是第一个在望远镜研究上质疑伽利略权威的人,这种观点在其著作发表后数年间变得更加普遍。

同年,惠更斯成功观测了火星上的大火山平原——大瑟提斯(Syrtis Major)。通过多日重复观测这一地貌的移动,他精确地估算出火星的自转周期为24.5小时。这一数据与火星真实的自转周期(24小时37分钟)仅相差几分钟。[157]

\textbf{行星仪}

在让-巴蒂斯特·柯尔贝尔(Jean-Baptiste Colbert)的推动下,惠更斯承担了建造机械行星仪的任务,该仪器能够展示当时已知的所有行星及其卫星围绕太阳的运行轨迹。惠更斯于1680年完成了设计,并于次年由钟表匠约翰内斯·范·瑟伦(Johannes van Ceulen)制作。然而,期间柯尔贝尔去世,惠更斯未能将其行星仪交付给法国科学院,因为新任大臣弗朗索瓦-米歇尔·勒·泰利耶(François-Michel le Tellier)决定不续签惠更斯的合同。[158][159]

在他的设计中,惠更斯巧妙地利用了连分数,找到最佳的有理数近似值,以选择齿轮的齿数。两个齿轮之间的比率决定了两颗行星的轨道周期。为了让行星围绕太阳运动,惠更斯使用了一种能够向前和向后计时的钟表机制。惠更斯声称他的行星仪比同时期奥勒·罗默(Ole Rømer)制造的类似装置更精确,但他的行星仪设计直到他去世后才在《遗作集》(Opuscula Posthuma,1703年)中发表。[158]

\textbf{《宇宙理论》}
\begin{figure}[ht]
\centering
\includegraphics[width=8cm]{./figures/f723a22d2979fd0d.png}
\caption{《宇宙理论》(Cosmotheoros,1698年)中太阳和行星的相对大小。} \label{fig_HGS_16}
\end{figure}
1695年临终前不久,惠更斯完成了他最具推测性的著作,名为《宇宙理论》(Cosmotheoros)。根据他的指示,这本书由他的兄弟在他死后出版,康斯坦丁·惠更斯(Constantijn Jr.)于1698年实现了这一愿望。[160] 在这部作品中,惠更斯推测了地外生命的存在,并设想这些生命与地球上的生命类似。这种推测在当时并不罕见,多基于哥白尼学说或“充足性原则”,但惠更斯进行了更为详细的探讨,尽管他没有提及牛顿的引力定律,也未意识到行星大气由不同气体组成的事实。[161][162] 《宇宙理论》被翻译成英文为《发现天体世界》(The Celestial Worlds Discover’d),本质上是一部乌托邦式的作品,部分灵感来源于彼得·海林(Peter Heylin)的著作,同时也可能被当时的读者视为一部类似弗朗西斯·戈德温(Francis Godwin)、约翰·威尔金斯(John Wilkins)和西拉诺·德·贝热拉克(Cyrano de Bergerac)作品传统的小说。[163][164][165]

惠更斯写道,液态水的存在是生命形成的必要条件,并认为水的性质需要根据不同星球的温度范围而变化。他将火星和木星表面暗点和亮点的观察结果视为这些行星上存在水和冰的证据。[166] 他认为,圣经既没有证实也没有否认地外生命的存在,并质疑如果其他行星仅供地球上的人类欣赏,那么上帝为何要创造它们。惠更斯提出,行星之间巨大的距离表明上帝并未打算让一种行星上的生物了解其他行星上的生物,也未预见到人类科学知识会如此迅速地进步。[167]

在这本书中,惠更斯还发表了他对太阳系相对大小的估算以及计算恒星距离的方法。[5] 他通过在一个面对太阳的屏幕上打出一系列小孔,直到认为孔的光强与天狼星的亮度相等为止。他随后计算出这个小孔的角度为太阳直径的1/27,664,并由此推断天狼星的距离约为太阳距离的30,000倍(基于天狼星与太阳具有相同光度的错误假设)。直到皮埃尔·布格尔(Pierre Bouguer)和约翰·海因里希·兰伯特(Johann Heinrich Lambert)的时代,光度学这一学科才从起步阶段走向成熟。[168]
\subsection{遗产}
惠更斯被称为第一位理论物理学家,也是现代数学物理学的奠基人之一。[169][170] 尽管他在世时影响深远,但在他去世后不久,这种影响开始逐渐减弱。他作为几何学家和机械设计天才的能力赢得了许多同时代人的钦佩,包括牛顿、莱布尼茨、洛必达以及伯努利家族的成员。[42] 在物理学领域,惠更斯被认为是科学革命中最伟大的科学家之一,其洞察深度和取得成果的数量仅次于牛顿,与之并驾齐驱。[4][171] 此外,惠更斯还为欧洲大陆的科学研究建立了制度框架,成为现代科学确立过程中的重要推动者。[172]
\subsubsection{数学与物理}
\begin{figure}[ht]
\centering
\includegraphics[width=8cm]{./figures/d7005a2a0ddea591.png}
\caption{克里斯蒂安·惠更斯的肖像,由伯纳德·瓦扬特绘于1686年。} \label{fig_HGS_17}
\end{figure}
在数学领域,惠更斯精通古希腊几何学的方法,尤其是阿基米德的研究成果,同时也熟练运用笛卡尔和费马的解析几何及微量技术。[85] 他的数学风格可以被描述为几何化的曲线和运动的微量分析。从力学中汲取灵感和类比,他的数学研究形式上仍保持纯粹的几何学特点。[72] 惠更斯将这种几何分析方法推向终结,因为越来越多的数学家转向微积分以处理微量、极限过程和运动问题。[38]

此外,惠更斯能够充分利用数学来回答物理问题。他通常通过引入简单模型来描述复杂情境,从基本论点出发,分析其逻辑推论,并在此过程中发展出必要的数学工具。在《离心力论》(De vi Centrifuga)的一份草稿末尾,他写道:[33]

“无论你假设与重力、运动或其他事物相关的任何假设,只要它不被证明为不可能,那么你可以通过证明线、面或体的某些性质得出真理。例如,阿基米德关于抛物线求积的论述,其中假设重物的作用力沿平行线方向。”

惠更斯偏爱以公理化的形式呈现其研究结果,这种方法需要严格的几何论证方法:尽管他允许在选择基本公理和假设时存在一定的不确定性,但从这些基础出发推导出的定理的证明则必须无可置疑。[33] 惠更斯的出版风格对牛顿在其重大著作中的呈现方式产生了影响。[173][174]

除了将数学应用于物理和物理应用于数学之外,惠更斯还将数学作为一种方法论,特别是它产生关于世界的新知识的能力。[175] 不同于伽利略主要将数学用作论证或综合的工具,惠更斯始终以数学作为发现和发展各种现象理论的手段,并坚持将物理问题的几何化简化必须符合现实与理想之间严格的标准。[125] 惠更斯对数学可解性和精确性的高要求为18世纪的科学家如约翰·伯努利(Johann Bernoulli)、让·勒龙·达朗贝尔(Jean le Rond d'Alembert)以及查尔斯-奥古斯丁·库仑(Charles-Augustin de Coulomb)树立了榜样。[33][169]

尽管从未计划公开发表,惠更斯在其关于碰撞的少数手稿中使用了代数表达式来表示物理实体。[44] 这使他成为最早使用数学公式描述物理关系的学者之一,这种方法正是当今物理学中的标准做法。[5] 此外,惠更斯在研究《屈光学》时接近了现代极限的概念,尽管他从未在几何光学之外使用过这一概念。[176]

\textbf{后期影响}

尽管惠更斯的成就在某些重要方面超过了牛顿,正如休·奥尔德西-威廉斯(Hugh Aldersey-Williams)所指出的,[177] 但到了17世纪末,惠更斯作为欧洲最伟大科学家的地位已被牛顿取代。尽管他的期刊发表形式预示了现代科学论文的风格,[93] 但他的经典主义倾向和对公开发表研究成果的迟疑在科学革命之后极大地削弱了他的影响力,因为莱布尼茨微积分和牛顿物理学的追随者成为了主流。[38][85]

惠更斯对满足特定物理性质的曲线(如摆线)的分析,促成了对其他类似曲线的研究,例如焦散线、最速降线、帆曲线和悬链线。[24][35] 他将数学应用于物理的研究,如对碰撞和双折射的研究,为随后几个世纪数学物理学和理性力学的新发展提供了灵感(尽管使用的是微积分这一新语言)。[7] 此外,惠更斯开发了振动计时机制,包括摆钟和游丝,这些机制自此以来一直用于机械表和钟表。这些装置是第一批适合科学用途的可靠计时器(例如,用于精确测量太阳日的不均匀性,这在以前是无法实现的)。[6][125] 他在这一领域的工作预示了随后的几个世纪中应用数学与机械工程的结合。[132]

\subsubsection{肖像}
在惠更斯的一生中,他和他的父亲委托创作了多幅肖像。这些包括:

\begin{itemize}
\item 1639年 – 阿德里安·汉尼曼(Adriaen Hanneman)创作的《康斯坦丁·惠更斯与他的五个子女》,带有奖章的画作,收藏于海牙莫里茨之家博物馆(Mauritshuis)[178]
\item 1671年 – 卡斯帕·内策尔(Caspar Netscher)绘制的肖像,现藏于莱顿布尔哈夫博物馆(Museum Boerhaave),由海牙历史博物馆(Haags Historisch Museum)借展[178]
\item 约1675年 – 亨利·特斯特兰(Henri Testelin)创作的《科学学院的建立及天文台的创立,1666年》中对惠更斯的描绘。画中柯尔贝尔向法国国王路易十四介绍新成立的科学学院成员,收藏于凡尔赛宫及特里亚农国家博物馆(Musée National du Château et des Trianons de Versailles),凡尔赛[179]
\item 1679年 – 法国雕塑家让-雅克·克莱里翁(Jean-Jacques Clérion)创作的浮雕奖章肖像[178]
\item 1686年 – 伯纳德·瓦扬特(Bernard Vaillant)创作的粉彩肖像,收藏于福尔堡霍夫维克博物馆(Museum Hofwijck),福尔堡[178]
\item 1684至1687年 – 卡斯帕·内策尔绘画作品的版画,由G. 埃德林克(G. Edelinck)雕刻[178]
\item 1688年 – 皮埃尔·布尔吉尼翁(Pierre Bourguignon)绘制的肖像,收藏于阿姆斯特丹皇家荷兰艺术与科学学院(Royal Netherlands Academy of Arts and Sciences)[178]
\end{itemize}
\subsubsection{纪念}
2005年降落在土星最大卫星泰坦上的欧洲航天局探测器以惠更斯的名字命名。[180]

在荷兰的重要城市,包括鹿特丹、代尔夫特和莱顿,可以找到多处纪念克里斯蒂安·惠更斯的纪念碑。

\subsection{作品}
来源:

\begin{itemize}
\item 1650 – De Iis Quae Liquido Supernatant(《关于漂浮在液体表面的物体》),未发表。[181]
\item 1651 – Theoremata de Quadratura Hyperboles, Ellipsis et Circuli(《关于双曲线、椭圆和圆的面积求解定理》),重刊于《全集》第十一卷。[42]
\item 1651 – Epistola, qua diluuntur ea quibus 'Ἐξέτασις [Exetasis] Cyclometriae Gregori à Sto. Vincentio impugnata fuit(《书信:反驳圣文森特的格雷戈里对圆测法的批评》),补编。[182]
\item 1654 – De Circuli Magnitudine Inventa(《圆的大小的发现》)。[33]
\item 1654 – Illustrium Quorundam Problematicum Constructiones(《某些著名问题的构造方法》),补编。[182]
\item 1655 – Horologium(《时钟》),关于摆钟的一本短小小册子。[6]
\item 1656 – De Saturni Luna Observatione Nova(《关于土星卫星的新观测》),描述了泰坦的发现。[183]
\item 1656 – De Motu Corporum ex Percussione(《关于碰撞中物体的运动》),死后于1703年发表。[184]
\item 1657年 – De Ratiociniis in Ludo Aleae(《关于博弈概率的推理》),由弗朗斯·范·斯霍滕(Frans van Schooten)翻译成拉丁文。[12]
\item 1659年 – Systema Saturnium(《土星体系》)。[182]
\item 1659年 – De vi Centrifuga(《关于离心力》),死后于1703年发表。[185]
\item 1673年 – Horologium Oscillatorium Sive de Motu Pendulorum ad Horologia Aptato Demonstrationes Geometricae(《摆钟:或关于钟摆运动的几何论证》),包括摆线理论和摆钟设计,献给法国国王路易十四。[126]
\item 1684年 – Astroscopia Compendiaria Tubi Optici Molimine Liberata(《无筒望远镜的简明天文观测》)。[42]
\item 1685年 – Memoriën aengaende het slijpen van glasen tot verrekkijckers(《关于研磨望远镜透镜的备忘录》),讨论透镜的研磨。[7]
\item 1686年 – Kort onderwijs aengaende het gebruijck der horologiën tot het vinden der lenghten van Oost en West(古荷兰语:《使用钟表确定东西经的简明指南》),关于如何利用钟表在海上确定经度。[186]
\item 1690年 – Traité de la Lumière(《光论》),探讨光的传播性质。[23]
\item 1690年 – Discours de la Cause de la Pesanteur(《关于重力原因的讨论》),补编。[42]
\item 1691年 – Lettre Touchant le Cycle Harmonique(《关于31平均律的书信》),一篇关于31平均律的短文。[37]
\item 1698年 – Cosmotheoros(《宇宙理论》),讨论太阳系、宇宙学和地外生命。[167]
\item 1703年 – Opuscula Posthuma(《遗作集》),包括:[42]
De Motu Corporum ex Percussione(《关于碰撞中物体的运动》),提出了关于碰撞的正确定律,最初完成于1656年。
Descriptio Automati Planetarii(《行星仪描述》),提供了一种行星仪的设计和说明。
\item 1724年 – Novus Cyclus Harmonicus(《新和谐循环》),惠更斯去世后在莱顿出版的一部音乐论著。[37]
\item 1728年 – Christiani Hugenii Zuilichemii, dum viveret Zelhemii Toparchae, Opuscula Posthuma(《克里斯蒂安·惠更斯遗作集》),副标题为《遗作》(Opera Reliqua),包括光学和物理学的研究。[185]
\item 1888–1950 – 惠更斯,克里斯蒂安:《全集》(Oeuvres complètes),共22卷,由D. Bierens de Haan(第1–5卷)、J. Bosscha(第6–10卷)、D.J. Korteweg(第11–15卷)、A.A. Nijland(第15卷)、J.A. Vollgraf(第16–22卷)编辑,海牙出版。[182]

\item 卷 I: 《书信集 1638–1656》(1888年)
\item 卷 II: 《书信集 1657–1659》(1889年)
\item 卷 III: 《书信集 1660–1661》(1890年)
\item 卷 IV: 《书信集 1662–1663》(1891年)
\item 卷 V: 《书信集 1664–1665》(1893年)
\item 卷 VI: 《书信集 1666–1669》(1895年)
\item 卷 VII: 《书信集 1670–1675》(1897年)
\item 卷 VIII: 《书信集 1676–1684》(1899年)
卷 IX: 《书信集 1685–1690》(1901年)
卷 X: 《书信集 1691–1695》(1905年)
卷 XI: 《数学著作 1645–1651》(1908年)
卷 XII: 《纯数学著作 1652–1656》(1910年)
卷 XIII, 第一册: 《屈光学 1653, 1666》(1916年)
卷 XIII, 第二册: 《屈光学 1685–1692》(1916年)
卷 XIV: 《概率计算:1655–1666年的纯数学研究》(1920年)
卷 XV: 《天文学观测:土星体系,1658–1666年的天文学研究》(1925年)
卷 XVI: 《力学研究直到1666年:碰撞、绝对运动存在性及感知性的问题、离心力》(1929年)
卷 XVII: 《1651至1666年摆钟研究,1650至1666年的物理学、力学及技术研究,1662或1663年的光冠和幻日研究》(1932年)
卷 XVIII: 《1666至1695年摆钟与平衡器研究,佚文》(1934年)
卷 XIX: 《1666至1695年的理论与物理力学研究,惠更斯在皇家科学院的工作》(1937年)
卷 XX: 《音乐与数学,1666至1695年的音乐与数学研究》(1940年)
卷 XXI: 《宇宙学》(1944年)






\end{itemize}


