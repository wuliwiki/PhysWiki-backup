% 李代数的复化
% license Usr
% type Tutor


\begin{definition}{}
若$\mathfrak g$是实数域上的李代数,我们可以将其扩展到复数域上。定义一个新的集合为
\begin{equation}
\mathfrak g+\I \mathfrak g=\{X+\I Y|X\,,Y\in\mathfrak g\}~.
\end{equation}
\end{definition}
可以验证李括号运算在其上封闭,且李括号依然有双线性和结合性。所以这也是一个李代数,称为$\mathfrak g$的\textbf{复化(complification)}。
\begin{example}{一般线性群的李代数}
类似于$\opn{Lie}GL(n,\mathbb R)\cong M(n,\mathbb R)$的证明,易证$\opn{Lie}GL(n,\mathbb C)\cong M(n,\mathbb C)$。下面主要证明这个复数版本的李代数恰为$\mathfrak u(n)$的复化,简单表示为$$
\end{example}
\begin{example}{SU(2)}

\end{example}
\begin{example}{洛伦兹群}

\end{example}