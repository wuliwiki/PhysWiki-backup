% 欧几里得空间
% 欧几里得空间|直角坐标系|线线夹角|线段|点面距离|垂线|面面距离

\begin{issues}
\issueDraft
\end{issues}

\pentry{仿射空间\upref{AfSp},欧几里得矢量空间\upref{EuVS}}
有了欧几里得矢量空间\upref{EuVS}和仿射空间\upref{AfSp}之后,终于可以更加贴近于我们所处的3维物理空间。简单来说,欧几里得空间就是与欧几里得矢量空间相伴随的仿射空间。在它上面,可以获得高中时候早就熟知的各种几何图景,比如直角坐标系,线段,线线夹角等概念。不同的是,这里是更一般的空间,其维度不止于3维空间。
\subsection{欧几里得空间}
\begin{definition}{欧几里得空间}
称仿射空间 $(\mathbb E,V)$ 是个\textbf{欧几里得(点)空间},如果 $V$ 是欧几里得矢量空间。并把 $V$ 的维数称为欧几里得空间的\textbf{维数}
\end{definition}
\begin{definition}{点点距离}
欧几里得空间 $(\mathbb E,V)$ 中,函数 $\rho:\mathbb E\times\mathbb E\rightarrow\mathbb R$:
\begin{equation}
\rho(\dot p,\dot q):=\norm{\overrightarrow{pq}}=\sqrt{(\overrightarrow{pq}|\overrightarrow{pq})}
\end{equation}
称为点点之间的\textbf{距离函数}。
\end{definition}
也就是说,欧几里得空间是个三元组 $(\mathbb E,V,\rho)$。

由矢量内积的性质\upref{EuVS},距离函数有以下性质;
\begin{enumerate}
\item \textbf{对称性:}$\rho(\dot p,\dot q)=\rho(\dot q,\dot p)$;
\item \textbf{正定性:}$\rho(\dot p,\dot q)\geq 0$ 且 $\rho(\dot p,\dot q)= 0\Leftrightarrow\dot p=\dot q$;
\item \textbf{三角不等式:}$\rho(\dot p,\dot q)+\rho(\dot q,\dot r)\geq\rho(\dot p,\dot r)$
\end{enumerate}
为方便起见,过点 $\dot p,\dot q$ 之间的直线(\autoref{SAfSp_eq4}~\upref{SAfSp})记作 $\Pi_{\dot p\dot q}$。
\begin{definition}{线线夹角}
矢量 $\overrightarrow{pq}$ 和 $\overrightarrow{rs}$ 之间的夹角 $\varphi$
\begin{equation}
\cos\varphi=\frac{(\overrightarrow{pq}|\overrightarrow{rs})}{\norm{\overrightarrow{pq}}\cdot\norm{\overrightarrow{rs}}}
\end{equation}
称为直线 $\Pi_{\dot p\dot q}$ 和 $\Pi_{\dot r\dot s}$ 之间的\textbf{夹角}。
\end{definition}
\begin{definition}{直角坐标系}
在欧几里得空间 $(\mathbb E,V)$ 中,称坐标系 $\{\dot o;\bvec{e_1},\cdots,\bvec{e_n}\}$ 为\textbf{直角坐标系},如果 $\bvec{e_1},\cdots,\bvec{e_n}$ 是矢量空间 $V$ 中的一个标准正交基底:$(\bvec e_i|\bvec e_j)=\delta_{ij}$。其中,$n$ 为 $V$ 的维数。
\end{definition}
\begin{example}{}
在直角坐标系中,若点 $\dot p,\dot q$ 的坐标分别为 $x_1,\cdots,x_n$ 和 $y_1,\cdots,y_n$ ,则 $\overrightarrow{pq}$ 坐标为 $y_1-x_1,\cdots,y_n-x_n$(\autoref{AfSp_the2}~\upref{AfSp})。于是,点 $\dot p,\dot q$ 距离为(\autoref{EuVS_eq5}~\upref{EuVS})
\begin{equation}
\rho(\dot p,\dot q)=\sqrt{\norm{\sum_{i}(y_i-x_i)\bvec e_i}^2}=\sqrt{\sum_{i}(y_i-x_i)^2\norm{\bvec e_i}^2}=\sqrt{\sum_{i}(y_i-x_i)^2}~.
\end{equation}
\end{example}
\begin{definition}{线段,中点}
称集合
\begin{equation}
\dot p\dot q=\{\dot p+\lambda\overrightarrow{pq}|0\leq\lambda\leq1\}
\end{equation}
是仿射空间中连接点 $\dot p,\dot q$ 的\textbf{线段}。满足条件 $\overrightarrow{pr}=\overrightarrow{rq}$ 的点 $\dot r\in\dot p\dot q$ 称为\textbf{线段 $\dot p\dot q$ 的中点}。而
\begin{equation}
\abs{\dot p\dot q}:=\norm{\overrightarrow{pq}}
\end{equation}
称为线段的\textbf{长度}。
\end{definition}
\begin{example}{}
试证明:$\dot p\dot q=\dot q\dot p$。

\textbf{证明:}
\begin{equation}
\begin{aligned}
\dot p\dot q&=\{\dot p+\lambda\overrightarrow{pq}|0\leq\lambda\leq1\}\\
&=\{\dot q+(1-\lambda)\overrightarrow{qp}|0\leq\lambda\leq1\}\\
&=\{\dot q+\lambda'\overrightarrow{qp}|0\leq\lambda'\leq1\}=\dot q\dot p
\end{aligned}
\end{equation}
\end{example}
\begin{definition}{点面距离,垂线}
设 $\dot q$ 是平面 $\Pi$ 上一点,称直线 $\Pi_{\dot p\dot q}$ \textbf{垂直}于平面 $\Pi$ ,记作 $\Pi_{\dot p\dot q}\perp\Pi$。若
\begin{equation}
\forall \dot r,\dot s\in\Pi\Rightarrow (\overrightarrow{pq}|\overrightarrow{rs})=0
\end{equation}
此时,称 $\rho(\dot p,\dot q)$ 是\textbf{点 $\dot p$ 到平面 $\Pi$ 的距离}。而 $\dot p,\dot q$ 之间的线段 $\dot p\dot q$ 就称为点 $\dot p$ 在 $\Pi$ 上的\textbf{垂线},记为 $\dot p\dot q\perp\Pi$。
\end{definition}
显然,当 $\dot p\in\Pi$ 是,$\Pi_{\dot p\dot q}$ 不可能垂直于 $\Pi$。因为 $(\overrightarrow{pq}|\overrightarrow{pq})>0,\dot p\neq \dot q$,而当 $\dot p=\dot q$ 时,由直线定义(\autoref{SAfSp_def1}~\upref{SAfSp}),$\{\dot p+\lambda\dot q|\lambda\in \mathbb R\}=\{\dot p\}$ 是个点,并不是直线。