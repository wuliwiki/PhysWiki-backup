% 幂函数(高中)
% 幂函数

\subsection{有理数次幂函数}
我们先来看实参数的幂函数 $f(x) = x^a$ 在 $a\in\mathbb R$ 和 $x > 0$ 时函数曲线如\autoref{fig_power_1} 所示。 注意在该区间 $x^{1/a}$ 是 $x^a$ 的反函数。

\begin{figure}[ht]
\centering
\includegraphics[width=8cm]{./figures/86604297d1436480.pdf}
\caption{实参数的幂函数(相同颜色的函数互为反函数)}\label{fig_power_1}
\end{figure}

由图可知, 对正数次幂($a > 0$), 其定义域可以包含 $0$, 且 $0^a = 0$。

严格来说, $0^0$ 是无法定义的,但在有些场合我们定义 $0^0$,因为我们希望幂级数\upref{powerS} $\sum_{n=0}^\infty c_n x^n$ 在 $x = 0$ 处的值是 $c_0$, 即 $n=0$ 时的项是常数项。 如果没有 $0^0 = 1$, 那么该幂级数将只能更繁琐地记为 $c_0 + \sum_{n=1}^\infty c_n x^n$。

\subsubsection{负数的幂函数}
显然, 负数的整数次幂是良好定义的, 因为这只涉及实数的乘法运算。 当 $a$ 为偶数时, $x^a = (-x)^a$ 是偶函数, $a$ 为奇数时, $x^a = -(-x)^a$ 是奇函数。 这样我们就可以把(TODO:插图) 中整数次幂的曲线根据对称性延申到负半轴。

而当我们试图将非整数次幂扩展到负实数时, 便需要把函数值拓展到复数域中, 并且可能有多个不同的函数值, 例如 $(-1)^{1/2} = \pm\I$. 

有理数次幂函数 $x^{n/m}$ ($x\in \mathbb R$, $n$ 为整数, $m$ 为正整数) 总是有 $m$ 个可能的值
\begin{equation}
x^{n/m} = \leftgroup{
&\abs{x^n}^{1/m}\E^{\I 2\pi k/m} & (x^n > 0)\\
&\abs{x^n}^{1/m}\E^{\I 2\pi (k+1/2)/m} & (x^n < 0)
}\qquad (k = 0,1,\dots, m-1)~.
\end{equation}

\subsection{无理数次幂函数}
