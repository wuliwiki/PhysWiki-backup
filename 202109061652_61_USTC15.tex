% 中科大 2015 年考研普通物理
% keys 中科大|考研|普通物理

\begin{issues}
\issueTODO
xia
\end{issues}


\begin{enumerate}
\item (15分)
一枚质量为M的火箭,依靠向正下方喷气在空中保持静止,如果喷出气体的速度为v,求火箭发动机的功率.
\item (20分)
如图所示,两个可看作质点的物
体,用胶粘在一起,由一段长为l=1m的
不可伸长的轻绳吊在高度为h=3m的天
花板上.使物体在水平面内以角速度
O=5rad/s做圆周运动.建立如图所示坐
标系,使物体运动的平面为XY平面.当
物体经过Y轴正半轴时,胶突然裂开,
其中一个物体脱落.求物体掉在地上的坐
标.(重力加速度g取10m/s2)
\item (20分)
一个原长为l、弹性系数为k的弹性绳固定在一个光滑平面上的固
定点O,另一端系一质量为m的质点,原先静止在平面上,突然给质点施
加一个大小为I垂直于绳的冲量,在以后运动中绳被拉伸到最大长度3l,
求冲量I.
\item (15分)
(15分)产生动生电动势的非静电力是洛伦兹力,该力推动载流子做功.
而另一方面,根据洛伦兹力公式PqB,洛伦兹力对带电粒子不做功.这
两个陈述是否矛盾,为什么?
\item (20分)
(20分)中性氢原子由原子核和电子云组成,电子云电荷密度分布为
p(r)=-Cqe2°,其中q为核电荷,a为玻尔半径,C为待定系数.
(1)确定C的数值;
(2)求氢原子核与电子云的相互作用能.
\item (20分)
在半径为a的细长螺线管中,均匀磁场的磁感应强度随时间均匀增大,可以表示为B=B+bt.均匀导线弯成等腰梯形回路 ABCDA,上底长为a,下底长为2a,总电阻为R,试求\\
(1)梯形各边上的感生电动势及整个回路的感生电动势;\\
(2)B、C两点之间的电势差.
\item (10分)
(10分)若钙原子(Z=20)除了一个电子之外的所有电子都被移去.\\
(1)计算该离子基态能量\\
(2)求该离子第一激发态的激发能量及相应跃迁的波长.
\item (10分)
氮原子中两个电子处在npn2P组态,(1)设n≠n2,有哪些可能的原子态?共有多少可能的状态数目?(2)若n=n2’又有哪些可能的原子态?为什么3D态不可能存在?
\item (20分) 
钠是 $Z=11$ 的碱金属.\\问:(1)钠基态的电子组态是什么?\\(2)该态的量子数各是多少?光谱项表达式是什么?\\(3)其第一激发态的电子组态是什么?光谱项如何写?\\(4)第一激发态向基态的跃迁,是否是允许跃迁?如果是,写出允许的电偶极跃迁.
\end{enumerate}