% 等压过程
% 等压过程|体积|压强|状态方程|做功

\pentry{理想气体状态方程\upref{PVnRT}}

等压过程的特征是系统的压强保持不变,即$p $为常量,$\mathrm dp =0$.设想气缸连续地与一系列有微小温度差的恒温热源相接触,同时活塞上所加的外力保持不变.那么接触产生什么效果呢?就是将有微小的热量传给气体,使气体温度稍微升高,气体对活塞的压强也随之较外界所施的压强增加一微量,于是稍微推动活塞对外做功.由于体积的膨胀,压强降低,从而保证气体在内、外压强的量值保持不变的情况下进行膨胀.所以这一准静态过程是一个\textbf{等压过程(isobaric process)}.

现在我们来计算气体的体积增加$\mathrm d V $时所做的功$\delta W$.根据理想气体状态方程,如果气体的体积从$V $增加到$V+\mathrm dV$,温度从$T $增加到$T+\mathrm dT$,那么气体所做的功
\begin{equation}
\delta W=p \mathrm{d} V=\frac{m}{M} R \mathrm{d} T
\end{equation}