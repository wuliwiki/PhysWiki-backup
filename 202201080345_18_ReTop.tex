% 实数集的拓扑

\pentry{上确界与下确界\upref{SupInf}}

\subsection{开集与闭集}

首先给出如下定义.

\begin{definition}{开集与闭集}
设 $x$ 是实数. 任意包含 $x$ 的开区间 $U(x)$ 都称作 $x$ 的一个开邻域 (open neighbourhood). 如果将点 $x$ 挖去, 得到的集合称为 $x$ 的去心邻域 (deleted neighbourhood), 常记为 $\mathring U(x)$.

实数集 $\mathbb{R}$ 的子集 $U$ 称为开集 (open set), 如果对于任意 $x\in U$, 都存在 $x$ 的开邻域 $V(x)$ 使得 $V(x)\subset U$. 

实数集 $\mathbb{R}$ 的子集 $C$ 称为闭集 (closed set), 如果 $\mathbb{R}\setminus C$ 是开集.

规定空集既是开集也是闭集.
\end{definition}

注意, 按照这个定义, 有许多集合不是开集也不是闭集. 直观上来说, 开集的"开"在于包含每个点的邻域, 而闭集的"闭"在于它的接触点无法跑出它的范围 (下详).

容易证明如下性质:

\begin{theorem}{开集和闭集的运算}
任意多个开集的并集仍然是开集. 有限多个开集的交集仍然是开集.

等价地, 任意多个闭集的交集仍然是闭集. 有限多个闭集的并集仍然是闭集.
\end{theorem}

\begin{exercise}{}
证明这个定理. 提示: 设 $\{U_\alpha\}_{\alpha\in A}$ 是一族开集, 那么若 $x\in \cup_{\alpha\in A}U_\alpha$, 则必定有一 $\alpha$ 使得 $x\in U_\alpha$. 如果 $U_1,...,U_n$ 是有限多个开集, $x\in\cap_{k=1}^nU_k$, 而 $V^k(x)$ 是 $x$ 的包含在 $U^k(x)$ 中的开邻域, 那么 $\cap_{k=1}^nV^k(x)$ 还是 $x$ 的开邻域.
\end{exercise}

\begin{exercise}{}
在证明"有限多开集的交集还是开集"时, "有限"这个条件究竟被用在哪里? 可以参考下面的例子.
\end{exercise}

\begin{example}{例子}
在实数轴上, 任何开区间本身都是开集. 根据下面给出的开集结构定理, 开集总是可数个互不相交的开区间的并.

实数轴上单独一点构成的集合是闭集. 以此类推, 有限多个点构成的集合是闭集. 整数集合$\mathbb{Z}$是闭集, 因为它的补集是$\cup_{k\in\mathbb{Z}}(k,k+1)$. 闭区间$[a,b]$是闭集, 因为它的补集是$(-\infty,a]\cup[b,\infty)$; 另外, 它不是开集, 因为点$a$的任何邻域都有不包含于闭区间$[a,b]$的部分.
\end{example}

\begin{example}{反例}
无限多个开集的交集不一定是开集. 例如, 设开区间 $U_k=(-1/k,1/k)$, 那么 $\cap_{k=1}^\infty=\{0\}$. 相应地, 无限多个闭集的并集也不一定是闭集, 例如, 设闭区间 $I_k=[0,1-1/k]$, 则 $\cup_{k=1}^\infty I_k=[0,1)$, 它不是开集也不是闭集.

一个更不平凡的例子是有理数集$\mathbb{Q}$. 它是可数多个单点集合的并集. 但由于有理数集在实数集中稠密, 它既不是开集也不是闭集.
\end{example}

粗略地说, 在一个集合上给定拓扑, 就是给定一个衡量元素之间的"远近关系"的尺度. 在实数集 $\mathbb{R}$ 中, 一个给定的实数 $x$ 的全体开邻域就划定了距离这个实数的"远近关系". 如上定义的开集的全体符合抽象的拓扑的定义; 详见词条拓扑空间\upref{Topol}.

\begin{definition}{子集上的拓扑}
设$E\subset\mathbb{R}$是实数集的非空子集. 称形如$E\cap U$的集合为在$E$中开 (这里$U\subset\mathbb{R}$是开集), 形如$E\cap C$的集合为在$E$中闭 (这里$C\subset\mathbb{R}$是闭集). 对于点$x\in E$, 称形如$E\cap U(x)$的集合为$x$在$E$中的邻域 (这里$U(x)$是包含$x$的开区间), 而形如$E\cap \mathring U(x)$的集合为$x$在$E$中的去心邻域.
\end{definition}

\subsection{开集的结构}
在实数集 $\mathbb{R}$ 中, 开集的结构可以被清楚地刻画出来. 首先引入一个定义: 包含于非空开集 $G\subset\mathbb{R}$ 中的开区间 $(a,b)$ 称为一个分支 (component), 如果端点 $a,b\notin G$. 容易看出, 任何非空开集中的两个分支一定不相交.

\begin{theorem}{实数集中开集的结构}
每一个非空开集 $G\subset\mathbb{R}$ 都是至多可数个分支的并集. 
\end{theorem}
\textbf{证明.} 首先注意到任何一点 $x\in\mathbb{R}$ 都一定属于某个分支 $U$: 这个分支是所有包含在 $G$ 中且包含 $x$ 的开区间的并集. 为了说明它符合分支的定义, 首先注意到 $U$ 当然是个区间. 进一步, 可以反设, 例如, $U$ 的左端点 $a\in G$; 那么有 $a$ 的开邻域 $U(a)\subset G$, 从而 $U(a)\cup U$ 也是包含 $x$ 的区间, 但它严格包含了 $U$, 同 $U$ 的定义相违背.

接下来, 按照这个推理, 注意到 $G$ 中的任何有理数 $r$ 都属于某个分支 $U_r$. 这些分支或者不相交, 或者重合. 由此, $G$ 的全体分支被 $G$ 中所包含的有理数所标记, 从而分支的个数一定是至多可数的. \textbf{证毕.}

这个定理显示出: 在实数集中, 既开又闭的非空集合只能是实数集本身.

\subsection{距离, 接触点与闭包}
实数集 $\mathbb{R}$ 是一个度量空间 (metric space). 关于一般的度量空间理论, 详见词条度量空间\upref{Metric}. 在实数集上, 最自然的度量是绝对值函数 $d(x,y)=|x-y|$, 它显然满足如下三条性质:

\begin{itemize}
\item $|x-y|=0$ 当且仅当 $x=y$.
\item $|x-y|=|y-x|$.
\item 三角不等式: 对于 $x,y,z\in\mathbb{R}$, 总有
$$
|x-z|\leq|x-y|+|y-z|.
$$
\end{itemize}

显然, 开区间 $(x_0-\delta,x_0+\delta)$ 恰好等同于集合
$$
\{x:|x-x_0|<\delta\}.
$$
在这种情况下, 我们说实数集的拓扑是由这个度量诱导得到的.

\begin{definition}{接触点}
给定点集 $E\subset\mathbb{R}$, 函数
$$
\text{dist}(x;E):=\inf_{y\in E}|x-y|
$$
称为到点集 $E$ 的距离函数 (distance function). 与 $E$ 的距离为零的点称为 $E$ 的接触点 (contact point). 
\end{definition}
$x$ 是 $E$ 的接触点, 当且仅当 $x$ 的任何邻域都与 $E$ 有非空交集. 实际上, $\inf_{y\in E}|x-y|=0$ 等价于如下命题: 任何 $\delta>0$ 都不是数集 $\{|x-y|:y\in E\}$ 的下界, 或者换句话说, 对于任何一个 $\delta>0$, 都存在 $y_\delta\in E$ 使得 $|x-y_\delta|<\delta$. 这恰好等价于"$x$ 的任何邻域都与 $E$ 有非空交集".

如果点$x$的每个去心邻域都与$E$相交, 那么称$x$是$E$的聚点 (accumulation point). 显然, 聚点是接触点, 但反过来不一定: $x$是$E$的聚点意味着$x$周围有无穷多个属于$E$的点. 与聚点相对立的是孤立点 (isolated point): 它本身是$E$的元素, 但它的某个邻域里不再有$E$的其它元素.

\begin{example}{接触点与聚点}
有限点集只有孤立点而没有聚点. 例如对于点集$\{-1,1\}$来说, 如果实数$x\neq\pm1$, 那么它与这个点集的距离$\delta$便不可能是零, 从而$x$的邻域$(x-\delta,x+\delta)$便不与$\{-1,1\}$相交.

对于开区间$(a,b)$, 端点$a,b$都是它的聚点.
\end{example}

\begin{definition}{闭包}
$E$ 的接触点的全体称为 $E$ 的闭包 (closure), 常常记为 $\bar E$.
\end{definition}

\begin{definition}{稠密}
如果集合$E\subset F$, 而$\bar E\supset F$, 则称$E$在$F$中稠密 ($E$ is dense in $F$).
\end{definition}

有如下定理:

\begin{theorem}{}
$E$ 的闭包是包含 $E$ 的所有闭集之交.
\end{theorem}
实际上, 如果 $C$ 是包含 $E$ 的闭集, 那么 $\mathbb{R}\setminus C$ 是包含于 $\mathbb{R}\setminus E$ 的开集, 因此任意一点 $x\in\mathbb{R}\setminus C$ 都有开邻域 $U(x)\subset\mathbb{R}\setminus C$, 而这开邻域显然同 $E$ 不相交. 这表示任意的 $x\in\mathbb{R}\setminus C$ 都不是 $E$ 的接触点, 或者反过来说, $E$ 的接触点集必然包含于 $C$. 

有如下显然的推论:
\begin{corollary}{}
集合 $E\subset\mathbb{R}$ 为闭集, 当且仅当它的闭包等于它自己.
\end{corollary}

最后:
\begin{theorem}{稠密性}
任何点集$E\subset\mathbb{R}$都有可数的稠密子集.
\end{theorem}
实际上, 从所有端点为有理数的开区间同$E$的交集中选出一个点 (如果交集非空), 即可组成$E$的可数的稠密子集$H$.