% 皮埃尔-西蒙·拉普拉斯(综述)
% license CCBYSA3
% type Wiki

本文根据 CC-BY-SA 协议转载翻译自维基百科\href{https://en.wikipedia.org/wiki/Pierre-Simon_Laplace}{相关文章}。

\begin{figure}[ht]
\centering
\includegraphics[width=6cm]{./figures/12e7e5c72d2ec562.png}
\caption{皮埃尔-西蒙·拉普拉斯作为法兰西第一帝国参议院的参议长} \label{fig_LPLS_1}
\end{figure}
皮埃尔-西蒙·拉普拉斯(Pierre-Simon, Marquis de Laplace,1749年3月23日-1827年3月5日)是法国学者,他的工作对工程学、数学、统计学、物理学、天文学和哲学的发展具有重要意义。他在五卷本《天体力学》(Mécanique céleste)(1799–1825)中总结并扩展了前人的工作。这部著作将经典力学的几何学研究转化为基于微积分的研究,从而开辟了更广泛的研究领域。[2]拉普拉斯还推广并进一步确认了艾萨克·牛顿爵士的工作。在统计学中,贝叶斯概率解释主要是由拉普拉斯发展而来的。[3]

拉普拉斯提出了拉普拉斯方程,并开创了拉普拉斯变换,这在数学物理的许多分支中都有应用,而数学物理领域正是他主导发展的一个重要领域。广泛应用于数学中的拉普拉斯算子也以他的名字命名。他重新阐述并发展了太阳系起源的星云假说,是最早提出类似黑洞概念的科学家之一,[4] 斯蒂芬·霍金曾表示:“拉普拉斯基本上预言了黑洞的存在”。[1]他提出了拉普拉斯恶魔,这是一个假设的全能预测智慧。他还改进了牛顿关于声速的计算,得出了更精确的测量结果。[5]

拉普拉斯被认为是历史上最伟大的科学家之一。有时被称为“法国的牛顿”或“法国的牛顿”,他被描述为拥有卓越的数学天赋,超越了几乎所有同时代的人。[6] 1785年,拉普拉斯曾担任拿破仑从巴黎军事学院毕业时的考官。[7] 拉普拉斯于1806年成为帝国伯爵,并在1817年波旁王朝复辟后被封为侯爵。
\subsection{早年}
\begin{figure}[ht]
\centering
\includegraphics[width=6cm]{./figures/1abf9bfb2f2cc079.png}
\caption{皮埃尔-西蒙·拉普拉斯的肖像,作者:约翰·恩斯特·海因修斯(1775年)} \label{fig_LPLS_2}
\end{figure}
拉普拉斯生活中的一些细节不为人知,因为1925年与他的曾曾孙科尔贝尔-拉普拉斯伯爵的家族 château(位于利厄市附近的圣朱利安·德·梅约)一起被烧毁。还有一些记录早在1871年,当他的位于巴黎附近阿尔居伊的住所被掠夺时就已被销毁。[8]

拉普拉斯于1749年3月23日出生在诺曼底的博蒙-昂-奥热(Beaumont-en-Auge),这个小村庄位于Pont l'Évêque以西约四英里处。根据W. W. Rouse Ball的说法,[9] 他的父亲皮埃尔·德·拉普拉斯(Pierre de Laplace)拥有并耕种着马尔基斯的小庄园。他的曾叔父奥利维尔·德·拉普拉斯(Maitre Oliver de Laplace)曾担任皇家外科医生。似乎他从一名学生逐渐升职为博蒙学校的监督员;但在获得了给达朗贝尔(d'Alembert)的介绍信后,他前往巴黎以谋求更好的发展。然而,卡尔·皮尔逊(Karl Pearson)对Rouse Ball叙述中的不准确之处进行了严厉批评,并指出:

的确,在拉普拉斯时代,卡昂可能是诺曼底所有城市中最具知识活力的地方。正是在这里,拉普拉斯接受了教育,并暂时担任教授。也是在这里,他写下了发表在都灵皇家学会《混合文集》中的第一篇论文,卷四,1766-1769年,至少比他22或23岁时前往巴黎的1771年早了两年。因此,在他不到20岁时,他已经与位于都灵的拉格朗日保持联系。他并不是一个仅有农民背景的未经教化的乡村小伙子,贸然前往巴黎!1765年,16岁的拉普拉斯离开了博蒙的“奥尔良公爵学校”,前往卡昂大学,他似乎在这里学习了五年,并且是斯芬克斯社团的成员。博蒙的军事学校直到1776年才取代了这所旧学校。

他的父母,皮埃尔·拉普拉斯和玛丽-安妮·索雄,来自较为富裕的家庭。拉普拉斯家族至少在1750年之前一直从事农业,但皮埃尔·拉普拉斯(父亲)还是一位苹果酒商人和博蒙镇的市政官。

皮埃尔·西蒙·拉普拉斯曾就读于该村的一所由本笃会修道院经营的学校,他的父亲原本打算让他成为天主教会的神职人员。16岁时,为了进一步实现父亲的意图,他被送往卡昂大学学习神学。[10]

在大学期间,他受到两位热衷数学的教师克里斯托夫·加德布莱和皮埃尔·勒卡努的指导,正是他们激发了他对该学科的热情。在这里,拉普拉斯作为数学家的才华迅速得到了认可,并且在仍在卡昂时,他写下了《关于无穷小和有限差分的积分计算的论文》。这篇论文也标志着拉普拉斯与拉格朗日之间的第一次通信。拉格朗日比拉普拉斯年长十三岁,最近在他故乡都灵创办了一本名为《都灵文集》(Miscellanea Taurinensia)的期刊,许多他的早期作品都刊登在其中,拉普拉斯的论文正是在该系列的第四卷中发表的。大约在这个时候,拉普拉斯意识到自己并没有成为神职人员的天赋,于是决定成为一名职业数学家。有些资料指出,他此后与教会决裂并成为无神论者。[需要引用] 拉普拉斯没有获得神学学位,而是带着勒卡努写给让·勒朗·达朗贝尔的介绍信前往巴黎,而达朗贝尔在当时是科学界的权威。[10][11]

根据他的曾曾孙的说法,[8] 达朗贝尔最初对拉普拉斯的接待并不好,为了摆脱他,他给了拉普拉斯一本厚厚的数学书,说让他读完后再回来。几天后,当拉普拉斯回来时,达朗贝尔更为冷淡,并且没有掩饰他对拉普拉斯不可能读懂并理解那本书的看法。但在与拉普拉斯交谈后,达朗贝尔意识到拉普拉斯确实理解了书中的内容,从那时起,他开始对拉普拉斯给予指导。

另一种说法是,拉普拉斯在一夜之间解决了达朗贝尔要求他提交的下周的问题,然后在接下来的夜晚又解决了一个更难的问题。达朗贝尔对此印象深刻,并推荐他进入军事学院任教。[12]

有了稳定的收入和不太苛求的教学任务,拉普拉斯开始全身心投入原创研究,在接下来的十七年里(1771年至1787年),他在天文学领域创作了大量原创性作品。[13]

从1780年到1784年,拉普拉斯与法国化学家安托万·拉瓦锡(Antoine Lavoisier)合作进行了一些实验研究,为这些任务设计了自己的设备。[14] 1783年,他们共同发表了论文《关于热的回忆录》,在其中讨论了分子运动的动理论。[15] 在他们的实验中,他们测量了不同物体的比热和金属在温度升高时的膨胀情况。他们还测量了乙醇和醚在压力下的沸点。

拉普拉斯进一步给孔多塞侯爵留下了深刻的印象,早在1771年,拉普拉斯便觉得自己有资格加入法国科学院。然而,那一年,学籍被授予了亚历山大-泰奥菲尔·范德蒙德(Alexandre-Théophile Vandermonde),1772年则授予了雅克·安托万·约瑟夫·库桑(Jacques Antoine Joseph Cousin)。拉普拉斯对此感到不满,1773年初,达朗贝尔写信给位于柏林的拉格朗日,询问是否可以为拉普拉斯找到一个职位。然而,孔多塞于2月成为了科学院的常任秘书,拉普拉斯在3月31日被选为院士候补成员,时年24岁。[16] 1773年,拉普拉斯在法国科学院前宣读了他的关于行星运动不变性的论文。同年3月,他被选为学院成员,之后他在该学术机构进行了大部分的科学研究。[17]

在1788年3月15日,拉普拉斯(Laplace)在39岁时娶了来自贝桑松一个“良好”家庭的18岁女孩玛丽-夏洛特·德·库尔蒂·德·罗曼热(Marie-Charlotte de Courty de Romanges)。婚礼在巴黎的圣叙尔皮斯教堂举行。夫妻俩有一个儿子,查尔斯-埃米尔(1789–1874),和一个女儿,索菲-苏珊娜(1792–1813)。
\subsection{分析、概率与天文稳定性}
拉普拉斯的早期发表作品始于1771年,主要涉及微分方程和有限差分,但他已经开始思考概率和统计的数学及哲学概念。[22] 然而,在1773年当选为法兰西科学院成员之前,他已经草拟了两篇论文,这些论文为他建立了声誉。第一篇论文《通过事件的概率推测因果关系》最终于1774年发表,第二篇论文则在1776年发表,进一步阐述了他的统计思想,并开始了他对天体力学和太阳系稳定性的系统研究。这两个学科始终在他的思维中紧密相连。“拉普拉斯将概率作为修正知识缺陷的工具。”[23] 拉普拉斯关于概率和统计的工作将在下文中讨论,包括他关于概率解析理论的成熟工作。
\subsubsection{太阳系的稳定性}  
艾萨克·牛顿爵士于1687年出版了《自然哲学的数学原理》(Philosophiæ Naturalis Principia Mathematica),在其中他根据自己的运动定律和万有引力定律推导出了开普勒定律,描述了行星的运动。然而,尽管牛顿在私下里已经发展了微积分方法,他的所有公开作品仍然使用繁琐的几何推理,这种方法并不适合解释行星间相互作用的更微妙的高阶效应。牛顿本人曾怀疑是否有可能找到整个问题的数学解法,甚至认为定期的神力干预是保证太阳系稳定所必需的。摒弃神力干预的假设将成为拉普拉斯科学生涯中的一项重大任务。[24] 现在普遍认为,尽管拉普拉斯的方法对理论的发展至关重要,但仅凭这些方法并不足够精确以证明太阳系的稳定性;今天我们理解太阳系在精细尺度上通常是混乱的,尽管在粗尺度上目前相对稳定。[25]: 83, 93

天文学中的一个特定问题是木星的轨道似乎在缩小,而土星的轨道在扩展。这个问题曾被莱昂哈德·欧拉于1748年和约瑟夫·路易·拉格朗日于1763年尝试解决,但都未成功。[26] 1776年,拉普拉斯发表了一篇论文,首次探讨了假设的光以太或不瞬时作用的引力定律可能的影响。最终,他回到了对牛顿引力的知识投资。[27] 欧拉和拉格朗日通过忽略运动方程中的小项,做出了一个实际的近似。拉普拉斯注意到,尽管这些项本身很小,但当它们在时间上积累时,可能会变得重要。拉普拉斯将他的分析扩展到了更高阶的项,直到包括三次项。通过这种更精确的分析,拉普拉斯得出结论,任何两颗行星和太阳必须处于相互平衡中,从而开启了他对太阳系稳定性的研究。[28] 杰拉尔德·詹姆斯·惠特罗称这一成就为“自牛顿以来物理天文学中最重要的进展”。[24]

拉普拉斯对所有科学都有广泛的知识,并主导了法兰西科学院的所有讨论。[29] 拉普拉斯似乎仅将分析视为解决物理问题的一种手段,尽管他发明所需分析的能力几乎是惊人的。只要他的结果是正确的,他对解释他得出这些结果的过程几乎不做任何努力;他从不研究过程中的优雅或对称性,对他来说,只要能以任何方式解决他所讨论的特定问题就足够了。[13]
\subsection{潮汐动力学}    
\subsubsection{潮汐的动态理论}  
虽然牛顿通过描述潮汐产生力来解释潮汐现象,伯努利则描述了地球上水体对潮汐势的静态反应,拉普拉斯于1775年发展出的潮汐动态理论[30] 描述了海洋对潮汐力的真实反应。[31] 拉普拉斯的海洋潮汐理论考虑了摩擦、共振和海洋盆地的自然周期。它预测了全球海洋盆地中大的安菲德罗米克系统,并解释了实际观测到的海洋潮汐。[32][33]

基于太阳和月球引力梯度的平衡理论,忽略了地球的自转、大陆的影响及其他重要因素,无法解释真实的海洋潮汐。[34][35][36][32][37][38][39][40][41]
\begin{figure}[ht]
\centering
\includegraphics[width=8cm]{./figures/e256f0da20f057c3.png}
\caption{} \label{fig_LPLS_3}
\end{figure}
由于测量结果已证实该理论,现在许多现象有了可能的解释,例如潮汐如何与深海脊和海山链相互作用,从而产生深层漩涡,将营养物质从深海输送到海面。[42] 平衡潮汐理论计算出的潮汐波高度不到半米,而动态理论则解释了潮汐为什么能达到15米。[43] 卫星观测证实了动态理论的准确性,全球的潮汐现已被测量到几厘米的误差范围内。[44][45] 来自CHAMP卫星的测量结果与基于TOPEX数据的模型非常吻合。[46][47][48] 全球潮汐的准确模型对研究至关重要,因为在计算重力和海平面变化时,必须从测量中去除潮汐引起的变化。[49]
\subsubsection{拉普拉斯的潮汐方程}
\begin{figure}[ht]
\centering
\includegraphics[width=6cm]{./figures/9b2d5a988b79470b.png}
\caption{A. 月球引力势:这表示从北半球上方看,月球正位于北纬30°(或南纬30°)的上空。} \label{fig_LPLS_4}
\end{figure}
\begin{figure}[ht]
\centering
\includegraphics[width=6cm]{./figures/aad7e166278f0243.png}
\caption{B. 这个视图显示的是与视图A相差180°的相同引力势。 从北半球上方看,红色朝上,蓝色朝下。} \label{fig_LPLS_5}
\end{figure}
1776年,拉普拉斯提出了一组描述潮汐流动的线性偏微分方程,潮汐流被描述为一种条形流动的二维薄层流动。引入了科氏效应以及由引力引起的侧向强迫力。拉普拉斯通过简化流体动力学方程获得了这些方程,但它们也可以通过拉格朗日方程从能量积分推导出来。

