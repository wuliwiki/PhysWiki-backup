% 狄拉克方程
% dirac function

\subsection{背景}
薛定谔方程并不能描述高速运动的微观粒子,而且由于不满足洛伦兹协变性,无法保证在别的惯性系成立.自然人们希望有满足狭义相对论及量子力学的理论出现.1927年, Klein-Gordon 方程被顺势提出.该方程满足狭义相对论中的质壳条件($E^{2}=\boldsymbol{p}^{2}+m^{2}$),描述的是无自旋的自由粒子:
\begin{equation}\label{qed4_eq1}
\left(\frac{\partial^{2}}{\partial t^{2}}-\nabla^{2}+m^{2}\right) \phi(\boldsymbol{x}, t)=0
\end{equation}
然而它产生两个无法忽视的问题:

\begin{enumerate}
\item \textbf{负能解}
\autoref{qed4_eq1} 的平面波解为$\phi_{\boldsymbol{p}}(\boldsymbol{x}, t)=\frac{1}{\sqrt{V}} e^{-i\left(E_{\boldsymbol{p}} t-\boldsymbol{p} \cdot \boldsymbol{x}\right)}$,能量本征值为
$E_{\boldsymbol{p}}=\pm \sqrt{\boldsymbol{p}^{2}+m^{2}}$,然而负能量在量子力学理论中没有物理意义.
\item \textbf{负概率密度}
K-G方程的荷密度为$\rho \equiv i\left(\phi^{*} \frac{\partial \phi}{\partial t}-\phi \frac{\partial \phi^{*}}{\partial t}\right)$,代入平面波函数解,显然荷密度可以取为负值.
\end{enumerate}
\subsection{狄拉克方程}
狄拉克提出,在三维空间中运动的自由粒子应满足:
\begin{equation}\label{qed4_eq2}
E=\boldsymbol{\alpha} \cdot \boldsymbol{p}+\beta m=\alpha_{x} p_{x}+\alpha_{y} p_{y}+\alpha_{z} p_{z}+\beta m
\end{equation}
平方后有
\begin{equation}\label{qed4_eq3}
\begin{aligned}
E^{2} &=\alpha_{x}^{2} p_{x}^{2}+\alpha_{y}^{2} p_{y}^{2}+\alpha_{z}^{2} p_{z}^{2}+\beta^{2} m^{2} \\
&+\left(\alpha_{x} \alpha_{y}+\alpha_{y} \alpha_{x}\right) p_{x} p_{y}+\left(\alpha_{y} \alpha_{z}+\alpha_{z} \alpha_{y}\right) p_{y} p_{z}+\left(\alpha_{x} \alpha_{z}+\alpha_{z} \alpha_{x}\right) p_{x} p_{z} \\
&+\left(\alpha_{x} \beta+\beta \alpha_{x}\right) p_{x} m+\left(\alpha_{y} \beta+\beta \alpha_{y}\right) p_{y} m+\left(\alpha_{z} \beta+\beta \alpha_{z}\right) p_{z} m
\end{aligned}
\end{equation}
若质壳关系依然成立,需要:
\begin{equation}\label{qed4_eq4}
\begin{aligned}
\beta^{2} &=1 & & \\
\alpha_j^{2} &=1 & &(j=1,2,3) \\
\alpha_j \alpha_k+\alpha_k \alpha_j &=0 & &(j, k=1,2,3, j \neq k) \\
\alpha_j \beta+\beta \alpha_j &=0 & &(j=1,2,3)
\end{aligned}
\end{equation}
显然,$\boldsymbol{\alpha}, \beta$不能是数,必须(至少)为四阶矩阵.$\alpha_j, \beta$满足反对易关系.
\\又因为$E$为厄米算符,即$E=\boldsymbol{\alpha} \cdot \boldsymbol{p}+\beta m=\boldsymbol{\alpha}^\dagger \cdot \boldsymbol{p}+\beta^\dagger m$,所以
\begin{equation}\label{qed4_eq5}
\alpha_j^{\dagger}=\alpha_j, \quad \beta^{\dagger}=\beta
\end{equation}
同时满足\autoref{qed4_eq4} 和\autoref{qed4_eq5} 的不同表示\textbf{(representation)}之间通过相似变换联系.
\\常用的表示有以下两种:
\\\textbf{标准表示(standard representation):}
\begin{equation}\label{qed4_eq6}
\alpha^{j}=\left(\begin{array}{cc}
0 & \sigma^{j} \\
\sigma^{j} & 0
\end{array}\right), \quad \beta=\left(\begin{array}{rr}
I & 0 \\
0 & -I
\end{array}\right)
\end{equation}
其中$\sigma^{j}$为泡利矩阵
\begin{equation}\label{qed4_eq7}
\sigma^{1}=\left(\begin{array}{ll}
0 & 1 \\
1 & 0
\end{array}\right), \quad \sigma^{2}=\left(\begin{array}{rr}
0 & -i \\
i & 0
\end{array}\right), \quad \sigma^{3}=\left(\begin{array}{rr}
1 & 0 \\
0 & -1
\end{array}\right)
\end{equation}
\\\textbf{威尔/手征表示(Wely/Chiral representation):}
\begin{equation}\label{qed4_eq8}
\alpha^{j}=\left(\begin{array}{cc}
-\sigma^{j} & 0 \\
0 & \sigma^{j}
\end{array}\right), \quad \beta=\left(\begin{array}{cc}
0 & I \\
I & 0
\end{array}\right)
\end{equation}
对\autoref{qed4_eq2} 进行一次量子化.动量被动量算符代替,由含时薛定谔方程得:\begin{equation}\label{qed4_eq9}
\boxed{i \frac{\partial}{\partial t} \psi(\boldsymbol{x}, t)=\left[-i \sum_{j} \alpha^{j} \frac{\partial}{\partial x^{j}}+\beta m\right] \psi(\boldsymbol{x}, t), \quad j=1,2,3}
\end{equation}
这就是狄拉克方程的一般形式.由于两个表示的$\alpha,\beta$都是四阶矩阵,所以狄拉克方程的解为具有四个分量的波函数,该解又称为\textbf{Dirac旋量}
\begin{equation}\label{qed4_eq10}
\psi=\left(\begin{array}{l}
\psi_{1} \\
\psi_{2} \\
\psi_{3} \\
\psi_{4}
\end{array}\right)
\end{equation}
与量子力学中的表象概念类似,不同的表示并不影响方程背后的物理内涵.具有物理意义的本征值不变,证明如下:
\\\textbf{标准表示}
采取质心系,此时粒子动量为0,\autoref{qed4_eq9} 变为
\begin{equation}\label{qed4_eq11}
i \frac{\partial}{\partial t} \psi= \beta m\psi
\end{equation}
令$\psi=\left(\begin{array}{l}\psi_{+} \\ \psi_{-}\end{array}\right)$,代入$\beta$得:
\begin{equation}\label{qed4_eq14}
i \frac{\partial}{\partial t} \psi_{+}=+m \psi_{+}
\end{equation}
\begin{equation}\label{qed4_eq15}
i \frac{\partial}{\partial t} \psi_{-}=-m \psi_{-}
\end{equation}
以上两式表明$\psi_{+}$与$\psi_{-}$分别对应能量本征值为$m$与$-m$的波函数解.\\\textbf{手征表示} 同样采取质心系,令$\psi=\left(\begin{array}{l}\psi_{L} \\ \psi_{R}\end{array}\right)$,代入该表示中的$\beta$于\autoref{qed4_eq11} ,则:
\begin{equation}\label{qed4_eq12}
i \frac{\partial}{\partial t} \psi_{L}=m \psi_{R}
\end{equation}
\begin{equation}\label{qed4_eq13}
i \frac{\partial}{\partial t} \psi_{R}=m \psi_{L}
\end{equation}
上式耦合了$\psi_{L}$与$\psi_{R}$,然而只要令
\begin{equation}
\psi_{+}=\frac{1}{\sqrt{2}}\left(\psi_{L}+\psi_{R}\right)
\end{equation}
\begin{equation}
\psi_{-}=\frac{1}{\sqrt{2}}\left(\psi_{L}-\psi_{R}\right)
\end{equation}
代入\autoref{qed4_eq12} 与\autoref{qed4_eq13} ,并进行相加减,最后可得到\autoref{qed4_eq14} 与\autoref{qed4_eq15} ,意味着不同表示并没有改变能量本征值与对应的本征态.
\subsubsection{描述对象}
\textbf{狄拉克方程描述的是自旋为$1/2$的正反费米子.}
\subsubsection{拉氏量}
\subsection{狄拉克方程对K-G方程疑难的解释}
\subsubsection{负能量}
\subsubsection{概率密度}




