% 二项分布

如果记$X $为$n $重伯努利试验中成功(记为事件$A$)的次数,则$X $的可能取值为$0,1,···,n$.记$p $为每次试验中$A $发生的概率,即$P(A)=p$,则$P(\overline{A})=1-p$.

而我们知道,$n$重伯努利试验的基本结果可以记作
\begin{equation}
\omega=\left(\omega_{1}, \omega_{2}, \cdots, \omega_{n}\right)
\end{equation}
其中$\omega_i$为$A$,或者为$\overline{A}$.这样的$\omega$共有$2^n$个,这$2^n$个样本点$\omega$组成了样本空间$\Omega$.

下面求事件$X$的分布列,即$\{X=k\}$的概率.若某个样本点
\begin{equation}
\omega=\left(\omega_{1}, \omega_{2}, \cdots, \omega_{n}\right) \in\{X=k\}
\end{equation}
意味着$\omega_1,\omega_2,\cdots,\omega_n$中有$k$个$A$,$n-k$个$\overline A$.由事件的独立性知:
\begin{equation}
P(\omega)=p^{k}(1-p)^{n-k}
\end{equation}
而事件$\{X=k\}$中这样的$\omega$共有$\binom nk$个,所以$X$的分布列为:
\begin{equation}
P(X=k)=\binom nk p^{k}(1-p)^{n-k}, k=0,1, \cdots, n
\end{equation}
这个分布成为\textbf{二项分布},记为$X\sim b(n, p)$.

那么它的和是不是为$1$呢?这很容易验证.我们有
\begin{equation}
\sum_{k=0}^{n}\binom nk p^{k}(1-p)^{n-k}=[p+(1-p)]^{n}=1
\end{equation}
并且从上式可以看出,二项概率$\binom nk p^{k}(1-p)^{n-k}$恰好是二项式$[p+(1-p)]^{n}$的展开式中的第$k+1$项,这正是其名称的由来.

显然,二项概率是一种离散分布.它很常见,举例来说:
\begin{example}{二项分布的例子}
\begin{itemize}
\item 检查$10$件产品,$ 10 $件产品中不合格品的个数$X $服从二项分布$b(10,p)$,其中$p$为不合格品率.
\item 调查$50 $个人,$ 50 $个人中患色盲的人数$Y $服从二项分布$b(50,p)$,其中$p$为色盲率.
\item 射击$5 $次,$ 5 $次中命中次数$Z $服从二项分布$b(5,p)$,其中$p $为射手的命中率.
\end{itemize}
\end{example}

下面来计算一些具体的例题.
\begin{example}{治愈的概率}
某特效药的临床有效率为$0. 95$,今有$10 $人服用,问至少有$8 $人治愈的概率是多少?

直接计算,我们有
\begin{equation}
\begin{aligned} P(X \geqslant 8) &=P(X=8)+P(X=9)+P(X=10) \\ &=\binom{10}{8} 0.95^{8} 0.05^{2}+\binom{10}{9} 0.95^{9} 0.05+\binom{10}{10} 0.95^{10} \\ &=0.0746+0.3151+0.5988=0.9885 \end{aligned}
\end{equation}
即$10 $人中有$8 $人以上被治愈的概率为$0. 988 5$.
\end{example}

\begin{example}{}
设随机变量$X\sim b(2,p),Y\sim b(3,p)$.若$P (X\geqslant1) = 5/9$, 试求$P(Y\geqslant1)$.

由$P(X\geqslant1)=5/9$,知$P(X=0)=4/9$,所以$(1-p)^2=4/9$,由此得$p=1/3$.再由$Y\sim b(3,p)$可得
\begin{equation}
P(Y \geqslant 1)=1-P(Y=0)=1-\left(1-\frac{1}{3}\right)^{3}=\frac{19}{27}
\end{equation}
\end{example}