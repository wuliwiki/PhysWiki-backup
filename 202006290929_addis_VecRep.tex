% 矢量空间的表示
\pentry{线性相关和线性无关\upref{LinDep}}

由于矢量空间中运算的线性性,可以使用矩阵来表示任何一个矢量空间中的元素和线性变换.对于一个域$\mathbb{F}$上的$n$维线性空间中的矢量,我们惯例上使用一个$n$行$1$列的矩阵来表示,称为\textbf{列向量}.线性变换被表示成一个$n\times n$的矩阵.这些矩阵中的元素都必须取自$\mathbb{F}$.

需要注意的是,这些表示都依赖于该矢量空间的\textbf{基}的选取.

\subsection{基和基向量}

\begin{definition}{线性组合}\label{VecRep_def1}
给定域$\mathbb{F}$上的$n$维矢量空间$V$.对于任意的$\bvec{v}_i\in V$,$a_i\in\mathbb{F}$,称$\{\bvec{v}_i\}_{i\in\Gamma}$是$V$中的一个\textbf{矢量组},其中$\Gamma$是某个表示指标的集合;称$a_1\bvec{v}_1+a_2\bvec{v}_2+\cdots=\sum\limits_{i\in\Gamma}a_i\bvec{v}_i$是该矢量组的一个\textbf{线性组合}\footnote{从这个角度来说,矢量空间的定义也可以简单记成“$V$的任何元素的任何线性组合还是$V$的元素”.}.
\end{definition}

\begin{definition}{矢量组张成的空间}
设定同\autoref{VecRep_def1}.矢量组$\{\bvec{v}_i\}_{i\in\Gamma}$的全体线性组合所构成的集合,仍然是一个线性空间.称这个空间是$\{\bvec{v}_i\}_{i\in\Gamma}$所\textbf{张成(span)}的子空间,记为$<\{\bvec{v}_i\}_{i\in\Gamma}>$.
\end{definition}

\begin{exercise}{}
设定同\autoref{VecRep_def1}.证明:
\begin{itemize}
\item 矢量组$\{\bvec{v}_i\}_{i\in\Gamma}$中的各矢量如果是线性相关的,则总可以一个一个地拿走若干矢量使得剩下的矢量仍然张成同一个子空间,直到剩下的各矢量线性无关.
\item 无论如何选择被拿走的矢量,剩下的线性无关的矢量的数目一定是$n$.
\end{itemize}
\end{exercise}

\begin{definition}{极大线性无关组}

设定同\autoref{VecRep_def1}.如果从$\{\bvec{v}_i\}_{i\in\Gamma}$中任意拿走一些矢量使之变成线性无关的矢量组,那么称剩下的矢量构成一个\textbf{线性无关组}.如果一个线性无关组张成的空间与原矢量组一样,那么称这个线性无关组是一个\textbf{极大线性无关组}.

\end{definition}

假设线性无关组$\{\bvec{e}_i\}_{i=1}^{n}$张成空间$V$.$V$中的每一个矢量都可以表示成$\{\bvec{v}_i\}_{i\in\Gamma}$中矢量的某个线性组合,并且由于线性无关,这种表示方法还是唯一的.这时,我们称这个线性无关组是$V$的一个\textbf{基},基中的矢量都称为一个\textbf{基向量}.

\subsection{用基向量来表示向量和线性变换}

给定域$\mathbb{F}$上的$n$维线性空间$V$和它的一个基$\{\bvec{e}_i\}_{i=1}^{n}$.由于$V$中的每一个向量都可以唯一地表示成基向量的线性组合,因此我们可以用线性组合的系数来构成一个列向量,作为这个向量在基$\{\bvec{e}_i\}_{i=1}^{n}$下的\textbf{坐标}.比如,向量$a_1\bvec{e}_1+\cdots+a_n\bvec{e}_n$在这个基下的坐标就是
\begin{equation}
\pmat{a_1\\ \vdots\\ a_n}
\end{equation}
基的选择不同,同一个向量的坐标也就不一样.

在研究线性变换的时候,我们只需要关注线性变换对基向量的变换,就可以据此计算出任意向量的线性变换.如果某一个线性变换$T$把基向量$\bvec{e}_i$变换成$a_{i1}\bvec{e}_1+\cdots+a_{in}\bvec{e}_n$,那么我们可以在这个基下把$T$表示成一个矩阵:
\begin{equation}
M=\pmat{a_{11},a_{12},\cdots,a_{1n}\\ a_{21},a_{22},\cdots,a_{2n}\\ \vdots\ddots\vdots\\ a_{n1},a_{n2},\cdots,a_{nn}}
\end{equation}

这样,如果把$\bvec{v}$的坐标是列向量$\bvec{c}$,那么$M\bvec{c}$就是$T\bvec{v}$的坐标.

同样地,线性变换的矩阵表示,也依赖于基的选取.