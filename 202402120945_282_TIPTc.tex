% 含连续态的不含时微扰(量子力学)
% license Usr
% type Tutor

\begin{issues}
\issueDraft
\end{issues}

若不含时微扰需要包含连续态(例如氢原子 stark 效应), 那么一阶微扰的\autoref{eq_TIPT_1}~\upref{TIPT} 到\autoref{eq_TIPT_2}~\upref{TIPT} 以及二阶微扰的\autoref{eq_TIPT2_1}~\upref{TIPT2} 中, $\ket{\psi_n^0}$ 仍然是束缚态,$\ket{\psi_n^1}$ 是束缚态和连续态叠加s。 $\bra{\psi_m^0}$ 可以是束缚态或者连续态。

这样的改变对束缚态的 $E_n^1$ 修正没有任何影响,但\autoref{eq_TIPT_5}~\upref{TIPT}和\autoref{eq_TIPT2_2}~\upref{TIPT2} 的求和中需要包含连续态。

对于氢原子来说,包含连续态是很重要的。 若不含, 则 $E_n^2$ 会比实际值偏小!
