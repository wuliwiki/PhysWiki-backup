% 兹威基关于暗物质的工作
% license Usr
% type Tutor


一般认为暗物质是由兹威基(Fritz Zwicky)在1933年提出的。那时候暗物质的概念和术语在天文学界已经存在一段时间了,例如在H. Poincaré、W. Thomson(即Lord Kelvin)、E. Öpik、J. Kapteyn、K. Lundmark和J. Oort以及S. Smith的工作中。兹威基的工作在历史上很重要,而且特别简单,所以值得我们好好讨论一下。兹威基通过观察Coma星系团的速度分布,发现保持星系团的凝聚力需要额外的物质。星系团是宇宙中最大的引力束缚系统。它们包含数百到数千个星系,延伸到数Mpc的大小。由于它们的规模,星系团是探测“平均”宇宙的好探针。虽然目前最精确的暗物质密度测量不是来自星系团,但它们确实导致了$\Omega_{DM} \simeq 0.2$。Zwicky的确定是基于位力定理。位力定理将平均动能与平均势能联系起来,%$\langle K \rangle  = −1/2 \langle V \rangle $。
在一个系统中,有$N \gg 1$个质量为$m$的物体在相等的距离$r$上通过引力相互作用,这允许从速度$v$和大小$R$来确定它们的总质量$mN$:
\begin{equation}\label{eq_DMZWI_1}
N mv^2 / 2 = 1/2 N2 G m^2 / R \rightarrow mN = 2R v^2 / G  ~.
\end{equation}


将这种考虑应用于Coma星系团,Zwicky认为星系团的总质量大于可见质量,因此需要额外的暗物质。暗物质的假设并没有被广泛接受,但也没有被忽视。一个常见的解释是需要更多的信息才能理解这些系统。

自20世纪80年代以来,X射线观测已成为评估星系团中普通和暗物质数量的更有效方法(见[14]的综述)。星系团包含大量的电离氢和氦。当这些气体坍缩到星系团的势阱中时,它会发生冲击和绝热压缩,加热,并达到温度$T_{gas} \sim m_pv^2 esc \sim 10 keV \sim  10^8 K$,这将是核反应的典型温度。然而,由于环境密度低,核聚变可以忽略不计。然后,气体主要通过热轫致辐射发出X射线。假设球对称性和流体静力平衡,可以写出将气体压力梯度℘gas与引力势梯度ϕ联系起来的平衡方程:
\begin{equation}\label{eq_DMZWI_2}
\rho_{gas}(r) d\varphi_{gas}/dr = - d\varphi / dr = GM(r)/ r^2~. 
\end{equation}
 

其中$\rho_{gas}(r)$是气体密度,$M(r)$是半径r内总的引力质量(即,氢/氦气体和暗物质)。假设理想气体,压力和密度通过$\varphi_{gas} = \rho_{gas}kT_{gas}/\mu$

$mp$相关联,其中$\mu\simeq 0.6$是由约75\%的氢和约25\%的氦混合而成的平均分子量。气体的密度ρgas和温度Tgas可以从X射线发射的强度和光谱中测量出来,从而允许使用\autoref{eq_DMZWI_2} 重建星系团的总质量和其分布。结果证实,气体只占星系团总质量的一部分。在某种程度上,使用X射线发射是一种应用Zwicky方法到微观尺度:气体分子的动能(即它们的温度)取代了星系的动能,然后从中推断出保持星系团引力束缚的总引力质量。X射线测量在分析独特的空间配置方面也非常有用,例如星系团的碰撞,我们将在接下来讨论。

Zwicky使用了一个与\autoref{eq_DMZWI_1} 不同的玩具模型来近似星系团:一个恒定密度$\rho$和半径$R$的球体。结果的差异是一个数量级因子,平均势能现在由$-\langle V\rangle = \int^R_0 G \rho 4\pi r^2 dr M(r)/r = 3GM^2/5 R$给出,其中M(r)是r内包含的质量,M是总质量。如今我们测量的是,在典型的星系团中,星系中的恒星质量占总质量的1-2\%,星系间的气体占5-15\%。其余的都是缺失的,并被解释为暗物质。