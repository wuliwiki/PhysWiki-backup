% 无限深圆形势阱
% 薛定谔方程|无限深势阱|圆势阱|贝塞尔函数

\begin{issues}
\issueDraft
\end{issues}

\pentry{定态薛定谔方程\upref{SchEq}, 柱坐标中的亥姆霍兹方程\upref{CylHlm}}

\begin{equation}
-\frac{1}{2m}\laplacian \psi(\bvec r) = E\psi(\bvec r)
\end{equation}
令 $k = \sqrt{2mE}$, 得到标准形式的亥姆霍兹方程
\begin{equation}
\laplacian \psi(\bvec r) = -k^2\psi(\bvec r)
\end{equation}
使用分离变量法, 角向波函数为
\begin{equation}
\Theta_{m_z}(\theta) = \frac{1}{\sqrt{2\pi}} \E^{im_z \theta}
\end{equation}
径向方程为\autoref{CylHlm_eq1}~\upref{CylHlm}($l = 0$, $x = kr$)
\begin{equation}
x \dv{x} \qty(x\dv{y}{x}) + (x^2 - m_z^2)y = 0
\end{equation}
所以径向波函数为
\begin{equation}
R_{m_z}(r) = J_{m_z}(kr)
\end{equation}
从物理上来看, 能量分为角速度产生的能量以及径向速度产生的能量, 角动量已经由 $m_z$ 固定, 所以 $r \to 0$ 时角速度产生的能量占主导, $J_{m_z}(kr)$ 的局部波长变长, 而 $r\to \infty$ 时径向速度几乎占据所有能量, 这时 $J_{m_z}(kr)$ 的局部波数趋近于 $k$.

再考虑到边界条件要求 $R_{m_z}(a)  = 0$, 可以得到每个能级的 $k_n$, $E_n = k_n^2/2$.
