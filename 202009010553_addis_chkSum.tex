% 校验和

\textbf{校验和(checksum)}是指一类算法, 可以将任意字节的数据转换为一个简短的值. 我们通常也将其称为文件的\textbf{指纹(fingerprint)}或者\textbf{哈希值(hash)}\footnote{严格来说这些不是同义词, 但人们经常混用}. 这可以用于检查两个文件的内容是否相同(即使文件名不同). % 未完成: 文件的结构: 文件名,内容, metadata, 文件系统?

通常来说, 文件在储存或传输的过程中有一定可能会被损坏, 例如网络故障, 黑客攻击, 或者储存介质的个别字节出现故障. 如果我们知道文件原来的校验和, 就可以随时重新做一次校验和, 如果结果和以前不一样, 就说明文件被改变了. 这也是为什么一些网站的提供文件下载时同时也会提供各种校验和, 以便用户下载以后对照校验和确保文件无误.

\subsection{碰撞}
然而要达到上述目的, 理想的情况就是确保任何文件和它的指纹都是一一对应\upref{map}的. 但实际上, 我们只能确保内容相同的文件得到同样的指纹, 或者说不同的指纹必定对应内容不同的文件. 不同的文件有可能对应相同的指纹, 我们把这种情况叫做\textbf{碰撞(collision/clash)}.

哈希碰撞的概率通常很小, 但确是无法避免的. 例如某文件有 100 个比特, 那么就有可能有 $2^{100}$ 种不同的内容, 但如果哈希值只有 10 个比特, 那么只可能有 $2^{10}$ 个不同的哈希值, 所以必定会出现碰撞. 为了减小碰撞的可能性, 我们可以使用更好的算法, 或者用更长的指纹.

为什么不同的算法会影响碰撞的可能性呢? 我们可以考虑一个最简单的算法: 把文件中每个字节对应的整数相加得到指纹. 这显然是一个不好的算法, 因为实际操作中很可能会出现两个文件仅仅存在字节顺序上的不同, 例如 \verb|abcd| 和 \verb|dcba| 按照这个算法得到的结果都是一样的. 所以好的算法并不在于避免随机文件发生碰撞的概率(因为这已经由字节数决定了), 而是尽量减小实际使用时发生碰撞的概率.

\subsection{安全性}
除了碰撞, 一个校验和算法的安全性也十分重要. 安全性是指能否创造或修改某个文件, 使得它具有指定的指纹. 这种操作越难实现, 算法就越安全.

例如一个黑客在某文件被传输时将其篡改, 为了防止用户发现, 将其篡改的文件略作不重要的修改使其具有和源文件相同的指纹, 那么即使文件的接收者检查指纹, 也不会有所察觉, 这就使得这个算法不安全.

\subsubsection{用于服务器保存密码}
网站的服务器中经常使用某种安全的指纹算法保存密码. 这是因为如果一个网站的服务器中直接保存用户的密码明文, 那么一旦这些明文密码被泄露, 那么得到密码的人将可以任意访问用户的数据.

更好的做法是服务器仅仅保存每个用户密码的哈希值, 用户每次输入密码时, 服务器先将密码转换为哈希值再验证是否正确. 这样即使这些哈希值泄露, 也不能把他们作为密码直接使用, 或者逆向得到密码原文.

\subsection{常用算法}
\textbf{SHA1(Secure Hash Algorithm 1)}著名文件版本控制软件 Git % 链接未完成
就是使用 SHA1 散列值来检查文件是否发生变化. 这是一个广泛使用的算法, 但 SHA1 已被破解, 所以安全性无保障.

\textbf{MD5} 是一个更安全的算法.
