% 有限深势阱
% 束缚态|能级|薛定谔方程|定态

% 束缚态的平均动量为零

% 这是区分束缚态(E<0) 和连续态(E>0)的最简单例子了.
% 当我们有势阱时, 都会有这种规律, 例如氢原子.

% 图未完成(量子力学简介里面似乎有图)

\pentry{定态薛定谔方程\upref{SchEq}}

\footnote{参考 Wikipedia \href{https://en.wikipedia.org/wiki/Finite_potential_well}{相关页面}.}本文使用原子单位. 代入定态薛定谔方程\autoref{SchEq_eq1}~\upref{SchEq}
\begin{equation}
-\frac{1}{2m}\pdv[2]{x}\psi(x) + V(x) \psi(x) = E \psi(x)
\end{equation}

中,令势能函数为
\begin{equation}
V(x) = \begin{cases}
-V_0 \quad &(-L/2 \leqslant x \leqslant L/2)\\
0 \quad &(\text{其他})
\end{cases}
\end{equation}

\subsubsection{束缚态}
由于 $V(x)$ 是对称的, 波函数必定是奇函数或者偶函数(\autoref{SchEq_eq3}~\upref{SchEq}). 令
\begin{equation}
k = \sqrt{2mE} \qquad \kappa = \sqrt{2m(E + V_0)}
\end{equation}
第 1,3 区间的通解为 $E < V$
\begin{equation}
\psi(x) = C_1 \E^{kx} + C_2 \E^{-kx}
\end{equation}
为了让无穷远处波函可归一化, 所以
\begin{equation}
\psi_1 = A \E^{kx} \qquad \psi_3 = D\E^{-kx}
\end{equation}
第 2 区间的通解为 $E > V$
\begin{equation}
\psi_2(x) = B \cos(\kappa x) + C\sin(\kappa x)
\end{equation}

奇函数要求波函数为奇函数, 所以 $\psi(0) = 0$, $B = 0$, 且 $A = -D$. 在考虑 2,3 区间交界处波函数及一阶导数连续有
\begin{equation}
\begin{aligned}
&C\sin(\kappa L/2) = D \exp(-kL/2)\\
&\kappa C \cos(\kappa L/2) = -kD \exp(-kL/2)
\end{aligned}
\end{equation}



==================================







波函数在边界处满足函数值和一阶导数都连续. 两个边界就可以得到 4 个边界条件, 得到 4 个系数的 4 条齐次方程. 最后加上归一化条件就可以唯一确定实函数(或者乘以任意相位因子). 事实上


由于 $V(x)$ 是偶函数, $\psi(x)$ 必须是奇函数或者偶函数.




什么时候有束缚态什么时候没有?
