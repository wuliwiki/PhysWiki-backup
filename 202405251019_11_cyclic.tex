% 循环群
% license Xiao
% type Tutor

\begin{issues}
\issueTODO 
\begin{enumerate}
\item 循环群与自同构群的关系。
\item 阿贝尔群是循环群的条件,相关定理应该拆分为新的一节。
\item 需要大量实例,某些定理需要韦恩图说明。
\end{enumerate}
\end{issues}

\begin{definition}{}
若群$G$的元素都是由某个元素生成,即$G=\ev  a$,则称群$G$为\textbf{循环群}(cyclic group),称$a$是循环群的生成元。
\end{definition}
显然,循环群的群元都可以表示为生成元的整数次幂,因此循环群实际上是阿贝尔群。由群的封闭性可知,给定群$G$的一个群元素$g$,那么$\ev{g}\subset G$。
\begin{definition}{}
对于群$G$,若$H$是其循环子群,且不是其他循环子群的真子群,则称$H$为$G$的\textbf{极大循环子群}。
\end{definition}

\begin{example}{}
整数群$\mathbb Z$是有无限元素的循环群,群乘法为加法。
\end{example}
\begin{example}{}
模$n$同余类$\mathbb Z_n$。
\end{example}
\begin{example}{}
群$G=\{-1,-\mathrm i,1,\mathrm i\}=\ev {\mathrm i}$。
\end{example}
循环群的形式看似没有什么规律,但我们可以借助同构来缩减研究对象。
\begin{theorem}{}
无限循环群同构于整数加群;$n$元有限循环群同构于$\{\mathbb Z_n;+\}$。
\end{theorem}
\textbf{proof.}\footnote{参考《抽象代数》,邓少强祝,朱富海著。}

设$G$是无限循环群,建立$\mathbb Z\rightarrow G$的同态映射,使得对于任意$n\in \mathbb Z$,都有$f(n)=a^n$。根据群同态基本定理\autoref{exe_Group2_1}~\upref{Group2},我们有$\mathbb Z/\opn{ker}f\cong G$。由于$\mathbb Z$的正规子群都是$n\mathbb Z,n\in \mathbb N$,因此模$n$同余类与$G$同构。当该$n=0$时,对应无限循环群;当$m\neq 0$时,$\mathbb Z_n$与n元循环群同构。

因为有限循环群可以继承整数群的乘法,因此还是一个环。可以证明,$\mathbb Z_n$环上的零因子是$n$的因子。所以,如果$n$是素数,那么这个环就是\textbf{无零因子交换幺环}了,我们一般简称其为\textbf{整环}。
\begin{theorem}{}
有限整环必是域。
\end{theorem}
只要证明任意环元都有逆在环内即可。
因为是有限整环,假设生成元为$a$,则由封闭性知对于每个非零同余类都必有$a^m=a^n,m\neq n$。设$m n$,因为$a^{m-n}a^n=1\cdot a^n$,所以$a^{m-n}=1$\footnote{注意这是环上的乘法,由于乘法运算构成半群,消去律未必成立。若对于环上元素有$ab=cb$且$b\neq 0$,则$(a-c)b=0$。由于整环没有零因子,所以$a=c$,即消去律对整环必然成立。}。因此,$a^{m-n-1}$为$a$的逆元,$a^{k(m-n-1)}$是$a^k$的逆元,证毕。
\begin{exercise}{}
无限循环群的生成元只有两个。
\end{exercise}
\begin{exercise}{}
若$n$元有限循环群$G=\ev a$,那任意元素$a^k,k\in N$的阶是多少。一个推论是:若$k$与$n$互素,则$\ev a=\ev a^k$,因此我们可以得知一个$n$元循环群的生成元个数。
\end{exercise}
\subsection{循环群的子群结构}
\begin{theorem}{}\label{the_cyclic_1}
循环群的子群必是循环群。
\end{theorem}
\textbf{proof.}
设循环群$G=\ev a$,则其子群元素必定包含$a$的某次幂。设任意子群为$G_1$包含元素的最小次幂为$k$,则该子群包含$a^{kn},n\in \mathbb Z$。若该群不是循环群,必然包含元素形如$a^{kn+r}$,其中$0< r<k$。由封闭性知也包含$a^{-kn}=(a^{kn})^{-1}$。则$a^{r}\in G^{1}$,与假设里$k$是最小正整数矛盾,所以循环群的子群必是循环群。
\begin{theorem}{}
设$G$是$n$元循环群,若$d|n$,则$G$内存在唯一一个$d$阶子群。
\end{theorem}
\textbf{proof.}

因为$a^{\frac{n}{d}d}=e$,因此$\ev a^{\frac{n}{d}}$是一个$d$阶子群。下面证明唯一性。

设$H=\{x\in G|x^d=e\}$,易证这是$G$的子群且具有唯一性。又设$x=a^k$,则$a^{kd}=e$意味着$n|kd$,则$\frac{n}{d}|k$。因为$G$是$n$阶子群,所以$k$的可能取值范围为$\{1,2...n\}$,也即$\{\frac{n}{d},2\frac{n}{d}...d\frac{n}{d}\}$。所以$H$的阶小于或等于$d$。

又因为$a^{\frac{n}{d}}\in H$,则$\ev a^{\frac{n}{d}}\subset H$,所以$H$的阶必等于$d$且$H=\ev a^{\frac{n}{d}}$。
换句话说,循环群的不同子群阶数不同。实际上,其逆命题也是成立的。即$G$是循环群$\Longleftrightarrow G$的不同子群阶数不同。
\begin{theorem}{}\label{the_cyclic_2}
若群$G$的不同子群阶数不同,则$G$是循环群。
\end{theorem}
\textbf{proof.}\footnote{引自《代数学基础》,Jier Peter著。}

设$H$为$G$的任意子群,因为共轭子群的阶与原群相等,即对于任意$g\in G$都有$|gHg^{-1}|=|H|$,题设条件使得$G$的任意子群都是正规子群——$H\lhd G$。设$G'=G/H$且$H_1$是$G'$的任意子群,那么对于$G$而言,$H_1$是运算封闭的左陪集之并,也就是说,$H_1$也是原群的子群,题设及“任意子群都是正规子群”在商群意义上得以继承。

因此,我们可以利用循环子群来构造商群列。从$G$里选任意元素$x_1$,构造商群$G_1=G/\ev x_1$,从$G_1$中选任意元素$x_2$,构造商群$G_2=G_1/\ev x_2$,以此类推——
\begin{equation}
G>G_1>G_2>...>G_{s-1}>G_S=\{e\}~.
\end{equation}
我们知道,根据拉格朗日定理,商群的基数必是原群基数的因子,因此商群列总是有限的,最后终结于平凡群。又因为该平凡群是$G_{s-1}$商去循环群得到的,所以$G_{s-1}$必是循环群。

接下来我们只需要证明,对于$G_{n}=G_{n-1}/\ev x_n$,若满足$G_{n}$为循环群,则$G_{n-1}$也必是循环群即可。这其实是从商群列逆向推导出$G$是循环群。

设$r$是$G_n$的生成元代表元素,即$r\ev x_n$生成了这个循环群。设$|G_{n}|=d$,则对于任意$r_1,r_2\in r\ev x_n$都有$r_1^d,r_2^d\in\ev x_n$。$r_1\rightarrow r_1^d$是从左陪集$r\ev x_n$到正规子群$\ev x_n$的映射,下面证明这是一个双射。

设$r_1\neq r_2$,且$r_1^d=r_2^d$。在$G_{n-1}$中,$\ev r_1$和$\ev r_2$是两个阶数相同,但元素不同的子群,与题设矛盾。所以$r_1\rightarrow r_1^d$是单射。又因为正规子群和陪集的基数相同,所以该映射既单又满。

因此在$r\ev x_n$中存在唯一的$a$使得$a^d=x$,把陪集表示为$a^i\ev {a^d},i=1,2...d$,则$G_{n-1}$确实是一个循环群,证毕。

取$d=5$,上述的$G_{n-1}$及其陪集划分如下图所示。
\begin{figure}[ht]
\centering
\includegraphics[width=12cm]{./figures/89878e22b416e68c.png}
\caption{$G_{n-1}/x_n$} \label{fig_cyclic_2}
\end{figure}

利用商群列的思路,我们可以证明一个阿贝尔群的常用性质:\textbf{阿贝尔群都是循环群的直积。
}要得到这个推论,我们还需要引入2个定理。
\begin{theorem}{}\label{the_cyclic_3}
给定阿贝尔群$G$,对其基数作素因子分解为$|G|=\prod_{i=1}^{k} p_{i}^{l_{i}},l_i\in\mathbb{Z}^+$,各$p_i$不相等。
定义$G_i = \{x \in G \mid x^{p_i^{l_i}} = e\}$,则$|G_i|=p_i^{l_i}$,且对于任意$x\in G$都有分解形式$x=x_1x_2...x_k$。
\end{theorem}
\textbf{证明\footnote{该证明总结自Jier Peter的《代数学基础》。}:}(为方便起见,以下用$O(x)$表示群元素的阶。)

\textbf{首先,我们需要证明对于$|G|$的任意素因子$p$,$G$总有$p$阶子群。}

设$x\in G$,则$\ev x$的基数是$|G|$的因子,设为$s$。$s$至少有一个质因子,设为$p$,则由于$x^{\frac{s}{p}\cdot p}=e$,所以$\ev {x^{\frac{s}{p}}}$是一个$p$阶循环子群。因此任意群都至少包含一个素数阶循环群。现在,我们可以构造商群列,仿若\autoref{the_cyclic_2} ,区别在于每一步所取的循环子群都是素数阶子群,$|G_n|=|G|/(p_1p_2...p_n)$,$p_i$都是$G$的素因子,所以总有一步取到的循环子群阶数就是我们需要的$p$。对于这个循环子群$\ev {x}_p$,其生成元的任意一个代表元素$g$的阶数都是$kp,k\in \mathbb N^{+}$,则$\ev{g^k}$就是$p$阶循环子群。\footnote{也就是说,生成元的整个左陪集的元素阶数都是$p$的整数倍。这是由于该左陪集里任意元素经过$p$步自乘后,都是正规子群的某个元素,而这个元素的阶数又是某个整数。}

现在我们可以构造商群列,每一次在商群中取$p_i$阶循环子群$\ev{x}_{n}$,直到$l_i$步终止\footnote{因为在这之后$|G_i|/|p_i^{l_i}|$里再无因子$p_i$}。从$G$的视角上看,$\ev x_{l_i}$的阶数是$p_i^{l_i}$,下面证明这个子群就是$G_i$。
为证明方便,设商群列为$G_n=G_{n-1}/\ev x_n$,其中$ n=1,2...l_i$。

对于一个素数阶群,其任意非单位元的元素阶数都是该素数。因此,从$\ev x_2$中任选一元素,其阶数要么是$p_i$,要么是$p_i^2$(因为经$p_i$步后要么是$e$要么是$\ev{x}_1-e$)。以此类推,直到$l_i$步后,对于任意$g\in\ev{x}_{l_i}$都有$O(g)\in\{p,p^2...p^{l_i}\}$,所以$\ev{x}_{l_i}\subset G_i$。又因为$\qty (G-\ev{X}_{l_i})\cap G_i=0$(不然$\qty (G-\ev{X}_{l_i})$包含$p^i(i=\mathbb N^{+})$阶子群),所以$\ev{x}_{l_i}=G_i$,则$|G_i|=p_i^{l_i}$。这样,我们就证明了定理的第一点。接着,我们来证明对于任意$x\in G$,定理所示的分解形式都是存在且唯一的。

设$t=p_2^{l_2}p_3^{l_3}...p_k^{l_k}$,则$t$与$p_1^{l_1}$互素,存在正整数$m$使得$mt\equiv 1(\mathrm {mod} p_1^{l_1})$。任取$x\in G$,则$x^t\in G_1$,设$x_1=x^{tm}$。利用$x^{p_1^{l_1}}=e$,我们有
\begin{equation}
\begin{aligned}
(x_1^{-1}x)^t&=x^{-t^2m}x^t\\
&=x^{-t(-1+mt)}\\
&=x^tx^{-t\cdot mt}\\
&=e~.
\end{aligned}
\end{equation}
阶数是$t$的因子,因此$x_1^{-1}x$的阶数不含素因子$p_1$。
仿照上述过程,设$t_2=p_1^{l_1}p_3^{l_3}...p_k^{l_k}$,存在正整数$\mathbb N^{+}$使得$m_2t_2\equiv 1(\opn{mod}p_2^{l_2})$。设$x_2=x'^{t_2m},x'=x^{-1}x$,则求得$(x_2^{-1}x')^t=(x_2^{-1}x_1^{-1}x)^t=e$。同理可知$x_2^{-1}x_1^{-1}x$的阶数不含有素因子$p_2$。由于整体$x^{-1}x$的幂次,因此阶数是$x^{-1}x$阶数的因子,因此也不含有$p_1$因子。以此类推,最后得到$x_k^{-1}x_{k-1}^{-1}...x_1^{-1}x$,其阶数不含有素因子$p_1,p_2...p_k$,因此为1,相应的$x_1x_2x_3...x_k$就是$x$的分解形式。

设分解形式不唯一,即至少存在两组分量使得$x_1x_2...x_k=h_1h_2...h_k$。由交换性得:$x_1h_1^{-1}x_2h_2^{-1}...x_kh_k^{-1}=e$。然而$e$的分解形式只能是$e$的乘积,因而$x_i=h_i$。

该定理实际上表明的是——阿贝尔群总可以分解为若干阶数为素因子幂的子群内直积,即$G\cong G_{l_1}\times G_{l_2}...G_{l_k}$。对于任意$x\in G$,可同构为$f(x)=(x_1,x_2,...,x_k)$。

在这基础上,我们只需要证明素数阶交换群总可以分解为有限个循环群内直积即可证明阿贝尔群的核心性质。
\begin{example}{循环群的分解形式}

\end{example}
\addTODO{需要更多解释}
\begin{theorem}{}\label{the_cyclic_4}
给定$p^k$阶阿贝尔群$G$($p$为素数),则它可以表示为循环子群的直积。
\end{theorem}
当$G$是循环群时,定理自然成立(待证),下面证明$G$并非循环群的一般情况。

设$G_1$是$G$的一个非平凡极大循环子群,再取另一个极大循环子群的形式为$\ev x G_1$,其中$\ev x$是$G$的一个循环群。由\autoref{the_NormSG_1}~\upref{NormSG}和\autoref{the_NormSG_2}~\upref{NormSG}可知,$\ev{x}G_1/G_1\cong \ev{x}/(\ev{x}\cap G_1)$,且若$\ev{x}\cap G_1\neq {e}$,总可以找到$y\in \ev x$使得$\ev y\cap G_1={e}$,同时满足$\ev xG_1=\ev y G_1$。

易证$\ev yG_1\cong \ev{y}\times G_1$\footnote{由于$G_1$是不变的,验证该性质时使之为$e$即可。画出韦恩图,你会发现,该同构实际上是把陪集映射到对应的点上,这也是我们要找到$\ev y$的原因,使得每个陪集只包含$\ev{y}$的一个点。}。然后我们来看$G/G_1\ev{y}$,若结果是$\{e\}=eG_1\ev{y}$,定理自然得证。若结果并不是这个极大循环子群,那么我们可以用$G$继续商去$G_1\ev{y}$,在商群里找到另一个元素$y_2$,重复上述过程,不断扩充正规子群,在有限步后$G/{\ev y_n}=\{e\}$,相应的,我们就成功把该阿贝尔群分解为循环子群的直积。

结合\autoref{the_cyclic_3} 和\autoref{the_cyclic_4} ,我们就得到了这节的重要推论。

\begin{corollary}{有限阿贝尔群基本定理}
阿贝尔群都可以分解为循环子群的直积,且循环子群的阶数是素数的幂。
\end{corollary}