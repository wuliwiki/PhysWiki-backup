% 欧几里得空间中的曲线

\pentry{光滑映射(简明微积分)\upref{SmthM}}

\subsection{一般欧几里得空间中曲线的概念}
% 原作者:JierPeter

\begin{definition}{参数曲线}
令 $I$ 是实数轴 $\mathbb{R}$ 上的一个开区间,则称\textbf{连续函数}$f:I\to \mathbb{R}^n$ 为 $\mathbb{R}^n$ 中的一条\textbf{参数曲线(curve)}(若$I$是一个闭区间则$f$被称为\textbf{道路(path)}),$f$的值域$f(I)$被称为参数曲线的\textbf{轨迹}。此处连续是指函数的 $n$ 个分量都是 $\mathbb{R}\to\mathbb{R}$ 的连续函数。
\end{definition}

我们可以任意取定一个坐标系,把向量值函数 $f$ 分为 $n$ 个标量值函数,简称为 $f$ 的分量,由此来理解连续的含义。你可能自然会想确认,$f$ 的分量的连续性,和取定坐标系的方式是否有关?答案是无关的。这是因为我们可以用另一种方式来理解此处的“连续”,那就是取集合 $I$ 与 $\mathbb{R}^3$,配上通常的拓扑——即 $I$ 取 $\mathbb{R}$ 的子拓扑,$\mathbb{R}^3$ 取 $\mathbb{R}$ 的乘积拓扑——所得到的拓扑空间,那么 $f$ 就是拓扑空间之间的映射,其连续性取决于拓扑意义上的连续性,和具体坐标系的选择就无关了。

要强调的一点是,即便两条参数曲线的轨迹一样,它们也不一定是同一条参数曲线。比如说,取 $f, g:\mathbb{R}\to\mathbb{R}^2$ 这两个函数,定义为 $f(t)=\pmat{\cos t\\\sin t}$ 和 $g(t)=\pmat{\cos 2t\\\sin 2t}$,那么尽管它们的轨迹都是平面上的单位圆,但由于两条参数曲线的“速度”不一样,我们依然把它们认为是不同的参数曲线。

\begin{definition}{连续可微参数曲线}
如果参数曲线 $f:I\to\mathbb{R}^n$ 对于任意分量都是连续可微的(即其导函数连续),那么称 $f$ 是一个\textbf{连续可微参数曲线(differentiable curve)}或\textbf{连续可微道路(differentiable path)},有时也记为 $C^1$ 的参数曲线。
\end{definition}

同样地,$f$ 本身的连续可微性,和具体坐标系的选择有关系吗?由于“微分”这一概念是实数空间特有的,我们没法像前面一样直接用拓扑的概念绕过坐标系的选择;但答案是一样的,\textbf{无关}。证明这一点的思路也可以应用到之前对一般参数曲线的讨论上去:考虑变换坐标系后各点的变换,会发现变换后的分量就是过渡矩阵乘以原先的函数列矩阵,也就是说,变换坐标后新的函数分量,是旧分量的某种线性组合。这样一来,变换前连续可微当且仅当变换后也连续可微,从而可知和变换无关。

\begin{definition}{光滑参数曲线}
如果参数曲线 $f:I\to\mathbb{R}^n$ 对于任意分量都是光滑的,那么称 $f$ 是一个\textbf{光滑参数曲线(smooth curve)}($I$为闭曲线时称为\textbf{光滑道路(smooth path)}),有时也记为 $C^\infty$ 的参数曲线。
\end{definition}

和上述坐标变换的讨论类似,可证明光滑参数曲线的光滑性不依赖于坐标系的选择。

除了参数曲线之外,我们还可以定义曲线为 $\mathbb{R}^n$ 的子集。我们可以把曲线li j

\begin{definition}{正则曲线}

\end{definition}
