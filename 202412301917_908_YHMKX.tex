% 约翰·麦卡锡(综述)
% license CCBYSA3
% type Wiki

本文根据 CC-BY-SA 协议转载翻译自维基百科\href{https://en.wikipedia.org/wiki/Maxwell\%27s_equations}{相关文章}。

\begin{figure}[ht]
\centering
\includegraphics[width=6cm]{./figures/b71f34967eb60ef0.png}
\caption{麦卡锡在2006年的一次会议上} \label{fig_YHMKX_1}
\end{figure}
约翰·麦卡锡(John McCarthy,1927年9月4日-2011年10月24日)是一位美国计算机科学家和认知科学家。他是人工智能学科的创始人之一。[1] 他共同撰写了提出“人工智能”(Artificial Intelligence, AI)这一术语的文献,开发了编程语言家族Lisp,深刻影响了ALGOL语言的设计,普及了分时共享技术,并发明了垃圾回收机制。

麦卡锡的大部分职业生涯都在斯坦福大学度过。[2] 他因对人工智能领域的贡献获得了诸多奖项和荣誉,例如1971年的图灵奖、美国国家科学奖章和京都奖。
\subsection{早年生活与教育}
约翰·麦卡锡于1927年9月4日出生在马萨诸塞州波士顿,他的父亲是一位爱尔兰移民,母亲是立陶宛犹太移民。[4] 他的父母分别是约翰·帕特里克·麦卡锡(John Patrick McCarthy)和艾达·格拉特·麦卡锡(Ida Glatt McCarthy)。大萧条期间,家庭多次搬迁,直到父亲在加利福尼亚州洛杉矶找到一份为服装工会(Amalgamated Clothing Workers)担任组织者的工作。他的父亲来自爱尔兰凯里郡的一个小渔村克罗曼(Cromane)。[5] 他的母亲于1957年去世。[6]

麦卡锡的父母在1930年代是共产党的积极成员,他们鼓励孩子学习和批判性思考。在进入高中之前,麦卡锡通过阅读一本名为《十万个为什么》(*100,000 Whys*)的俄文儿童科普书籍,对科学产生了兴趣。[7] 他精通俄语,并在多次访问苏联期间结交了俄罗斯科学家,但在访问东欧集团后与其拉开了距离,并成为一名保守派共和党人。[8]

麦卡锡提早两年从贝尔蒙特高中毕业,[9] 并于1944年被加州理工学院(Caltech)录取。

他展现出数学的早期天赋;在青少年时期,他通过自学加州理工学院使用的大学数学教材掌握了大学数学知识。因此,他进入加州理工学院后得以跳过前两年的数学课程。[10] 他因未参加体育课程而被加州理工学院停学,[11] 后来服役于美国陆军,之后重新被录取,并于1948年获得数学学士学位(BS)。[12]

麦卡锡在加州理工学院听过约翰·冯·诺依曼的讲座,这对他的未来事业产生了重要启发。

麦卡锡在加州理工完成了研究生学习后,前往普林斯顿大学,并于1951年在唐纳德·C·斯宾塞(Donald C. Spencer)的指导下完成了题为《投影算子与偏微分方程》(*Projection operators and partial differential equations*)的博士论文,获得了数学博士学位。[13]
\subsection{学术生涯}

在普林斯顿大学和斯坦福大学短期任职后,麦卡锡于1955年成为达特茅斯学院的助理教授。

一年后,他于1956年秋天转至麻省理工学院(MIT)担任研究员。在麻省理工学院的最后几年,他已经被学生们亲切地称为“约翰叔叔”。[14]

1962年,麦卡锡成为斯坦福大学的正教授,并一直任职至2000年退休。

麦卡锡倡导数学方法,如λ演算,并为人工智能中的常识推理设计了逻辑系统。