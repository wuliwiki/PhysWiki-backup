% 张量的对称化和交错化
% 对称化|交错化
\pentry{张量\upref{Tensor},置换群\upref{Perm}}
张量可按选出来的共变或逆变指标的集合来讨论对称性和斜对称性.由于张量的共边和逆变指标的对称性(即共边和逆变或矢量和对偶矢量的区别仅在于原矢量空间的选择),在讨论张量的对称性和斜对称性时可直接将张量限制在 $(p,0)$ 型或 $(0,p)$ 型张量,而不减少一般性.比如:讨论张量 $T^{i_1 i_2\cdots i_p}$ 和 ${S_{i_1}}^{i_2\cdots i_p}$ 在指标 $\{i_1,i_2\}$ 上的对称性质没有任何区别,只需将 $T$ 的上指标 $i_1$ 当作下指标即可移值到张量 $S$.故在本词条里,我们总是将张量限制为 $(p,0)$ 型张量.并且讨论的是这 $p$ 个指标的对称性和斜对称性,而不是它的一部分(因为在讨论某些指标的(斜)对称性质的时候,只需将其它指标当作固定的就变成这里讨论的情形).
\subsection{张量的对称化}
\subsubsection{对称张量}


