% 驻波

\begin{issues}
\issueDraft
\end{issues}

\pentry{平面简谐波\upref{PWave}}

\subsection{一维驻波}
想象一根紧绷的弦的两端固定, 那么弦上显然不可能出现稳定的简谐波\upref{PWave}, 因为简谐波要求两个端点也必须随做简谐运动\footnote{“两端固定” 这个条件在解波动方程\upref{WEq1D}的语境下被称为边界条件, 所以前面讨论的简谐波作为波动方程的解, 不满足所要求的边界条件}.

一般来说, 波都是可叠加的, 所以我们不妨看看两列振幅、频率、波长都相同而方向相反的简谐波
$$
\begin{aligned}
f_1(x,t)&=\sin(kx+\omega t)\\
f_2(x,t)&=\sin(kx-\omega t)\\
\end{aligned}
$$
叠加后能否满足条件,即
$$f=f_1+f_2=\sin(kx+\omega t)+\sin(kx-\omega t)$$
根据和差化积\upref{TriEqv}公式,
\begin{equation}
f(x,t)=-2\sin(\omega t)\sin(kx)
\end{equation}
形如
\begin{equation}\label{StaWav_eq1}
f(x,t)=A\sin(\omega t)\sin(kx)
\end{equation}
的波可以理解为单一频率的驻波.

\subsubsection{驻波、波节与波腹}
\begin{figure}[ht]
\centering
\includegraphics[width=12cm]{./figures/StaWav_1.pdf}
\caption{驻波、波节与波腹} \label{StaWav_fig1}
\end{figure}

如\autoref{StaWav_fig1} 所示,单一频率的驻波的最大特点是“只在原地振动”.

根据\autoref{StaWav_eq1} ,如果我们把$\sin(kx)$一项视为振幅的一部分,那么我们会发现振幅大小与位置有关,即$A(x) = A \sin(kx)$.

也就是说,当$\sin(kx)=0, kx=n\pi$时(图中黑色点),振幅始终为零,此处的质元从不运动,称为波节;当$\sin(kx)=\pm1, kx=\frac{\pi}{2} (2n-1)$时,振幅达到最大值$A$,称为波腹(图中红色点).

\subsubsection{振动频率}
\begin{figure}[ht]
\centering
\includegraphics[width=10cm]{./figures/StaWav_2.pdf}
\caption{请添加图片描述} \label{StaWav_fig2}
\end{figure}
在简谐波中,振动频率是可以任意取值的;但是在驻波中,振动频率只能取离散的值.这是由于弦边界的约束条件.具体的求解需要数学物理方法(分离变量法解偏微分方程\upref{SepVar})的相应知识,此处仅做简单介绍.

假定弦长为$L$,驻波方程\autoref{StaWav_eq1} 满足弦的边界条件 
$$
\begin{aligned}
f(0)&=0\\
f(L)&=0\\
\end{aligned}
$$
很明显,第一个条件自动满足,而由第二个条件得到
$$
f(L,t)=A\sin(\omega t)\sin(kL)
$$
为使该式始终为零,必须有$\sin(kL)$始终为零,即
$$kL = n\pi \qquad n=1,2,3...$$
即
\begin{equation}
k=\frac{n\pi}{L}
\end{equation}
由此,为了符合边界条件,$k$只能取离散的值,因此弦上可存在的驻波的波长、周期等,也必须是离散的.
\begin{equation}
\begin{aligned}
\lambda &=\frac{2L}{n}\\
T &=\frac{2L}{nv}\\
\end{aligned}
\end{equation}
\addTODO{介绍: 波节、波腹、驻波的波动方程; 如何计算一根线所支持的振动频率、 谐波是什么. 大学物理程度即可, 不要讲太深}
