% 东南大学 2010 年 考研 量子力学
% license Usr
% type Note

\textbf{声明}:“该内容来源于网络公开资料,不保证真实性,如有侵权请联系管理员”

\textbf{1.(15 分)}设粒子能级 $E_n$ 的简并度为 $f_n$,归一化的能量本征函数为 $\phi_{n\alpha}(r)$ ($\alpha = 1,2,\dots,f_n$),$t = 0$ 时刻粒子的归一化波函数为 $\psi(r,0) = \sum_{n\alpha} c_{n\alpha} \phi_{n\alpha}(r)$,试求:

\begin{enumerate}
    \item $t$ 时刻粒子的波函数 $\psi(r,t)$;
    \item $t$ 时刻粒子的能量平均值 $ \overline{H} $。
\end{enumerate}

\textbf{2.(15 分)}设 $\psi_1(x)$ 和 $\psi_2(x)$ 均为与能级 $E$ 对应的能量本征函数,试证:
\begin{enumerate}
    \item $\psi_1 \psi_2' - \psi_1' \psi_2 = c$ (常数);
    \item 若 $\psi_1$ 和 $\psi_2$ 均为束缚态波函数,则 $\psi_1 \psi_2' = \psi_1' \psi_2$。
\end{enumerate}

\textbf{3.(15 分)}质量为 $m$ 的粒子处于一维势场中 $V(x) = 0, (0 < x < a)$, $V(x) = \infty, (x < 0, x > a)$,试求能量本征值和归一化的能量本征函数。

\textbf{4.(15 分)}设粒子的能量 $E > V_0$ 从左入射,碰到势场 $V(x) = 0, (x < 0)$, $V(x) = V_0, (x > a)$,在 $0 < x < a$ 的区域 $V(x)$ 是连续有限的函数,反射系数和透射系数分别为 $r$ 和 $t$。试证明:$r + t = 1$。

提示:几率流密度公式为 $j(x) = -(i\hbar/2m)(\psi^*\psi'  - \psi\psi^{*'})$。

\textbf{5.(15 分)}试用测不准关系估计以下体系的基态能量:
\begin{enumerate}
    \item 质量为 $m$ 的粒子处于长度为 $a$ 的一维无限深方势阱中;
    \item 频率为 $\omega$ 的一维谐振子。
\end{enumerate}

\textbf{6.(15 分)}