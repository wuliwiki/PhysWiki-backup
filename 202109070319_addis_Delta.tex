% 狄拉克 delta 函数
% 微积分|定积分|狄拉克 delta 函数|\delta 函数

\pentry{定积分\upref{DefInt}, sinc 函数\upref{sinc}}
\footnote{参考 \cite{Arfken} 相关内容.}在物理中我们经常会遇到一些模型, 如质点和点电荷等, 这类模型使用了极限的思想(如令体积趋于无穷小). 如果考察质点的密度或点电荷的电荷密度, 将得到无穷大, 然而将其密度(电荷密度)在空间中积分却又能得到有限的质量与电荷. 为了描述这样的密度(电荷密度)分布, 我们引入\textbf{狄拉克 $\delta$ 函数(Dirac delta function)}.

$\delta(x)$ 并不是数学中一个严格意义上的函数, 而是在泛函分析中被称为\textbf{广义函数(generalized function)}或\textbf{分布(distribution)}, 详见泛函分析教材如\cite{Zeidler}.

\begin{figure}[ht]
\centering
\includegraphics[width=11cm]{./figures/Delta_1.pdf}
\caption{$\delta(x - x_0)$ 函数列的简单例子} \label{Delta_fig1}
\end{figure}

我们来考虑一个函数(\autoref{Delta_fig1} 左)
\begin{equation}\label{Delta_eq6}
f_h(x) =
\begin{cases}
h & \qty( \abs{x - x_0} \leqslant \frac{1}{2h} )\\
0 & \qty( \abs{x - x_0} > \frac{1}{2h} )
\end{cases}
\end{equation}
其中 $h, x_0$ 是常数. 由函数图像易得函数曲线下面的面积为 $\int_{-\infty}^{+\infty} f_h(x) \dd{x} = 1$. 现在我们令 $h \to \infty$, 长方形的高将趋于无穷大, 宽将趋于零, 而定积分结果不变.

这样, 上面的 $f_h(x)$ 就可以表示为 $\delta(x - x_0)$. 当然, 我们还可以选取其他含有参数的 $f(x)$ 来逼近 $\delta$ 函数, 如\autoref{Delta_fig1} 中的另外两种情况.

\subsection{用函数列严格定义}

本文中我们给出另一个严谨的定义: 把 $\delta$ 函数看作是满足一定条件的\textbf{函数序列}, 即无穷个函数按一定顺序排列.
\begin{definition}{狄拉克 $\delta$ 函数}\label{Delta_def1}
令 $\delta_1(x), \delta_2(x), \dots$ 为一个连续实函数的序列. 若 $\delta_n(x)$ 满足以下两个条件, 那么我们把该函数列称为狄拉克 $\delta$ 函数(列):
\begin{equation}\label{Delta_eq5}
\lim_{n\to\infty}\int_{-\infty}^{+\infty} \delta_n(x) \dd{x} = 1
\end{equation}
对任意给定的不包含 0 的区间 $(a,b)$($a,b \ne 0$, 可取 $\pm\infty$), 有
\begin{equation}\label{Delta_eq4}
\lim_{n\to\infty} \int_{a}^{b} \delta_n(x) \dd{x} = 0
\end{equation}
\end{definition}
事实上, 教材上(如\cite{Arfken})常使用以下定义, 但笔者认为这两种定义是等价的\footnote{容易证明\autoref{Delta_def1} 是\autoref{Delta_def2} 的必要条件(只需要令\autoref{Delta_def2} 中的 $f(x) = 1$ 证明\autoref{Delta_eq5} ; 再令 $f(x)$ 在区间 $(a,b)$ 为 1 , 否则为 0, 证明\autoref{Delta_eq4}.). 充分条件笔者不会证明, 欢迎指正. 如果有误以\autoref{Delta_def2} 为准.}, 且前者更容易理解

\begin{definition}{狄拉克 $\delta$ 函数 2}\label{Delta_def2}
令 $\delta_1(x), \delta_2(x), \dots$ 为一个连续实函数的序列. 若 $\delta_n(x)$ 满足以下两个条件, 那么我们把该函数列称为狄拉克 $\delta$ 函数(列):

对所有性质良好(例如在 $x = 0$ 连续)的 $f(x)$, 都有
\begin{equation}
\lim_{n\to\infty}\int_{-\infty}^{+\infty} \delta_n(x)f(x) \dd{x} = f(0)
\end{equation}
\end{definition}

一些物理教材会把 $\delta(x)$ 的性质简单记为
\begin{equation}\label{Delta_eq2}
\delta(x) =
\begin{cases}
+\infty & (x = 0)\\
0 & (x \ne 0)
\end{cases}
\end{equation}
\begin{equation}\label{Delta_eq3}
\int_{-\infty}^{+\infty} \delta(x) \dd{x} = 1
\end{equation}
这是十分不严谨的, $\delta(x)$ 不是函数, 更不可积, $x \ne 0$ 时极限 $\lim_{n\to \infty}\delta_n(x)$ 也未必存在(\autoref{Delta_ex2} ).

\begin{exercise}{高斯函数}\label{Delta_exe2}
证明高斯分布函数\upref{GausPD} 可以构成以下 $\delta$ 函数列
\begin{equation}
\delta_n(x) = \frac{n}{\sqrt{\pi}} \E^{-{n ^2}x^2} \qquad (n = 1, 2, \dots)
\end{equation}
\end{exercise}
\begin{exercise}{sinc 函数}\label{Delta_exe1}
证明 $\sinc$ 函数\upref{sinc}可以构成以下 $\delta$ 函数列(提示: \autoref{sinc_eq1}~\upref{sinc}).
\begin{equation}
\delta_n(x) = \frac{n}{\pi}\sinc^2(n x) \qquad (n = 1, 2, \dots)
\end{equation}
\end{exercise}

\begin{example}{sinc 函数}\label{Delta_ex2}
证明 $\sinc$ 函数\upref{sinc}可以构成以下 $\delta$ 函数列
\begin{equation}\label{Delta_eq10}
\delta_n(x) = \frac{n}{\pi}\sinc(n x) \qquad (n = 1, 2, \dots)
\end{equation}
该式在傅里叶分析和量子力学中有重要应用\cite{Arfken}, 但证明起来比较困难只好从略(参考\autoref{sinc_eq2}~\upref{sinc}). 注意即使对于 $x \ne 0$ 上式也不存在 $n\to\infty$ 的极限而是做简谐振动, 可见\autoref{Delta_eq2} 是十分不严谨的.
\end{example}
\begin{figure}[ht]
\centering
\includegraphics[width=14.25cm]{./figures/Delta_2.pdf}
\caption{\autoref{Delta_ex2} 中的 $n\sinc(n x)$} \label{Delta_fig2}
\end{figure}

\subsection{常见性质}
\subsubsection{性质 1}
若一个等式中出现了所谓的 $\delta$ 函数 $\delta(x)$, 那么其严格的定义是先将 $\delta(x)$ 替换为符合\autoref{Delta_def1} 的任意函数列 $\delta_n(x)$, 令\textbf{等式在 $n\to\infty$ 的极限时成立}.

例如 $\delta(x)$ 一个重要的性质是: 对任意在 $x = x_0$ 处连续函数 $f(x)$, 有
\begin{equation}\label{Delta_eq7}
\int_{-\infty}^{+\infty} f(x) \delta(x - x_0) \dd{x}= f(x_0)
\end{equation}
使用\autoref{Delta_def1} 中的 $\delta_n(x)$ 表示, 该等式的严格意义是(注意极限和积分不可交换, 且极限必须在最外面)
\begin{equation}\label{Delta_eq11}
\lim_{n\to\infty}\int_{-\infty}^{+\infty} f(x) \delta_n(x - x_0) \dd{x}= f(x_0)
\end{equation}
由于我们假设\autoref{Delta_def1} 和\autoref{Delta_def2} 等效, 这是成立的.

\subsubsection{性质 2}
\begin{equation}\label{Delta_eq1}
\delta(ax) = \frac{1}{\abs{a}} \delta(x)
\end{equation}
我们从\autoref{Delta_eq6} 的几何上来不严谨地证明这个性质: 与 $\delta(x)$ 相比较, $\delta(ax)$ 的图像在 $x$ 方向变窄了 $\abs{a}$ 倍, 所以函数曲线下的面积变为原来的 $1/\abs{a}$ 倍, 故 $\abs{a}\delta(ax)$ 下的面积是 $1$, 证毕.

\subsubsection{性质 3}
作为\autoref{Delta_eq1} 的拓展, 令 $f(x)$ 的根为 $x_1, x_2, \dots$, 在这些根处的导数为 $f'(x_i)$, 那么
\begin{equation}\label{Delta_eq12}
\delta[f(x)] = \sum_i \frac{1}{\abs{f'(x_i)}} \delta(x - x_i)
\end{equation}
证明和\autoref{Delta_eq1} 类似.

\subsubsection{性质 4}
对性质良好的函数 $g(x)$ 有
\begin{equation}
g(x)\delta(x) = g(0)\delta(x)
\end{equation}
证明: 对于性质良好的 $f(x)$, $\int f(x) g(x)\delta(x) \dd{x} = f(0)g(0) = \int f(x)g(0)\delta(x) \dd{x}$. 证毕.

\subsection{其他应用}
再次强调我们不能 “按字面意思” 理解任何含有 $\delta(x)$ 的等式.
\begin{example}{}\label{Delta_ex1}
在傅里叶分析中, 时常会看到
\begin{equation}\label{Delta_eq8}
\int_{-\infty}^{+\infty} \E^{\I kx}\dd{x} = 2\pi \delta(k)
\end{equation}
然而严格来说, 左边的积分并不收敛, 所以不能 “按字面意思” 理解该式. 要严格证明, 上式两边除以 $2\pi$, 令有限区间 $[-n,n]$ 内的积分为
\begin{equation}
\delta_n(k) = \frac{1}{2\pi}\int_{-n}^{n} \E^{\I kx}\dd{x} = \frac{\sin(n k)}{\pi k} = \frac{n}{\pi} \sinc(n k)
\end{equation}
由\autoref{Delta_exe1} 中的结论得
\begin{equation}
\lim_{n\to\infty} \delta_n(k) = \delta(k)
\end{equation}
证毕.
\end{example}

\begin{exercise}{}\label{Delta_exe3}
证明
\begin{equation}\label{Delta_eq9}
\begin{aligned}% 已验证
&\int_{-\infty}^{+\infty} \sin(k'x) \sin(kx) \dd{x} = \int_{-\infty}^{+\infty} \cos(k'x) \cos(kx) \dd{x}\\
&= \pi\delta(k' - k) - \pi\delta(k' + k)
\end{aligned}
\end{equation}
提示:使用\autoref{CTrig_eq1}~\upref{CTrig},\autoref{CTrig_eq2}~\upref{CTrig} 和\autoref{Delta_eq8} .
\end{exercise}

\begin{example}{}
请证明
\begin{equation}\label{Delta_eq13}
\int_{-\infty}^{+\infty} \delta(x-x_1) \delta(x-x_2) \dd{x} = \delta(x_1 - x_2)
\end{equation}
注意由此可得积分 $\int_{-\infty}^{+\infty}\delta(x)^2 \dd{x} = +\infty$, 即不收敛.

证明: 考虑和上文一样的含参函数 $\delta_n(x)$, 令 $I_n = \int_{-\infty}^{+\infty} \delta_n(x) \dd{x}$, 有 $\lim_{n\to\infty} I_n = 1$. 再令
\begin{equation}
f_n(x_1, x_2) = \int_{-\infty}^{+\infty} \delta_n(x-x_1) \delta_n(x-x_2) \dd{x}
\end{equation}
\addTODO{以下证明不严谨, 待更正}
我们希望证明 $\lim_{n\to\infty }f_n(x_1, x_2) = \delta(x_1 - x_2)$. 首先对于给定的 $x_1 \ne x_2$ 显然有 $\lim_{n\to\infty }f_n(x_1, x_2) = 0$. 所以只需证明
\begin{equation}
\lim_{n\to\infty }\int_{-\infty}^{+\infty} f_n(x_1, x_2) \dd{x_2} = 1
\end{equation}
交换积分顺序得
\begin{equation}
\begin{aligned}
&\lim_{n\to\infty }\int_{-\infty}^{+\infty}\int_{-\infty}^{+\infty} \delta_n(x-x_1) \delta_n(x-x_2) \dd{x}\dd{x_2}\\
&= \lim_{n\to\infty }\int_{-\infty}^{+\infty} \delta_n(x-x_1) \int_{-\infty}^{+\infty} \delta_n(x-x_2)\dd{x_2} \dd{x}\\
&= \lim_{n\to\infty } I_n \int_{-\infty}^{+\infty} \delta_n(x-x_1)\dd{x}\\
&= \lim_{n\to\infty } I_n^2 = 1
\end{aligned}
\end{equation}
证毕.
\end{example}
