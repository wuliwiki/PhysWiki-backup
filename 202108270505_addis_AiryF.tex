% 艾里函数

\begin{issues}
\issueTODO
\end{issues}

\pentry{常微分方程\upref{ODE}, 二阶线性微分方程(未完成)}

\footnote{参考 Wikipedia \href{https://en.wikipedia.org/wiki/Airy_function}{相关页面}.}\textbf{艾里函数(Airy function)}是微分方程
\begin{equation}
y'' - xy = 0
\end{equation}
的两个线性无关解, 分别记为 $\opn{Ai}(x)$ 和 $\opn{Bi}(x)$.

\begin{figure}[ht]
\centering
\includegraphics[width=11cm]{./figures/AiryF_1.pdf}
\caption{艾里函数} \label{AiryF_fig1}
\end{figure}

艾里函数可以用反常积分\upref{impro}定义(未完成:推导)
\begin{equation}
% 已数值验证
\opn{Ai}(x)=\frac 1\pi \int_0^\infty \cos(\frac{t^3}{3}+xt)\dd t
\end{equation}
\begin{equation}
% 已数值验证
\opn{Bi}(x)=\frac 1\pi \int_0^\infty \qty[\exp(-\frac{t^3}{3}+xt)+\sin(\frac{t^3}{3}+xt)]\dd t
\end{equation}
乍看之下, 三角函数在 $[0,\infty)$ 的积分似乎不收敛, 但由于相位以 $t^3$ 变化, 原函数在 $t\to+\infty$ 的震荡会原来越快, 幅度越来越小, 最终收敛.


\subsection{性质}
\subsubsection{渐近形式}
\footnote{参考 \cite{GriffQ} 中的 WKB 近似章节.}艾里函数一种较简单的渐近形式为
\begin{align}
% 已验证
\opn{Ai}(x) \overset{x \to +\infty}{\longrightarrow} \frac{1}{2\sqrt{\pi} x^{1/4}} \exp(-\frac{2}{3}x^{3/2})\\
\opn{Bi}(x) \overset{x \to +\infty}{\longrightarrow} \frac{1}{\sqrt{\pi} x^{1/4}} \exp(\frac{2}{3}x^{3/2})
\end{align}
$x \to -\infty$
\begin{align}
% 已验证
\opn{Ai}(x) \overset{x \to -\infty}{\longrightarrow} \frac{1}{\sqrt{\pi} \abs{x}^{1/4}} \sin(\frac{2}{3}\abs{x}^{3/2}+\frac{\pi}{4})\\
\opn{Bi}(x) \overset{x \to -\infty}{\longrightarrow} \frac{1}{\sqrt{\pi} \abs{x}^{1/4}} \cos(\frac{2}{3}\abs{x}^{3/2}+\frac{\pi}{4})
\end{align}
\begin{figure}[ht]
\centering
\includegraphics[width=14.25cm]{./figures/AiryF_2.pdf}
\caption{蓝线为艾里函数,橙线为渐进形式} \label{AiryF_fig2}
\end{figure}

\subsubsection{正交性}
艾里函数 $\opn{Ai}$ 的正交性定义为(未完成:证明)
\begin{equation}
\int_{-\infty}^\infty\opn{Ai}(x+a)\opn{Ai}(x+b)\dd x=\delta(a-b)
\end{equation}

\subsubsection{与贝塞尔函数关系}
\pentry{贝塞尔函数\upref{Bessel}}
对 $x>0$,有
\begin{equation}
\opn{Ai}(x)=\frac1\pi \sqrt{\frac x3} K_{\frac13} \qty(\frac23 x^{\frac32})
\end{equation}\begin{equation}
\opn{Bi}(x)=\sqrt{\frac x3} \qty[I_{\frac13}\qty(\frac23 x^{\frac32})+I_{-\frac13}\qty(\frac23 x^{\frac32})]
\end{equation}
其中 $K$ 和 $I$ 分别为第一类、第二类修正贝塞尔函数(见\autoref{Bessel_eq2}~\upref{Bessel}).
对 $x<0$,有
\begin{equation}
\opn{Ai}(x)=\sqrt{\frac{\abs x}{9}}\qty[J_{\frac13}\qty(\frac23 \abs{x}^{\frac32})+J_{-\frac13}\qty(\frac23 \abs{x}^{\frac32})]
\end{equation}
\begin{equation}
\opn{Bi}(x)=\sqrt{\frac{\abs x}{3}} \qty[J_{-\frac13}\qty(\frac23 \abs{x}^{\frac32})-J_{\frac13}\qty(\frac23 \abs{x}^{\frac32})]
\end{equation}
其中, $J$ 为第一类贝塞尔函数(\autoref{Bessel_eq3}~\upref{Bessel}).

\subsection{应用}
\subsubsection{微分方程变形}
令 $a, b\in \mathbb R$, 那么
\begin{equation}\label{AiryF_eq13}
y'' - (ax + b) y = 0
\end{equation}
的通解是
\begin{equation}\label{AiryF_eq1}
% 已数值验证
y(x) = C_1\opn{Ai}\qty(\frac{ax+b}{\abs{a}^{2/3}}) + C_2 \opn{Bi}\qty(\frac{ax+b}{\abs{a}^{2/3}})
\end{equation}

\subsubsection{线性势能的薛定谔方程}
在一维定态薛定谔方程(连接未完成)中, 若势能函数是线性的, 即 $V(x) = px + q$, 那么方程整理后可变为\autoref{AiryF_eq13} 的形式, 所以波函数就是\autoref{AiryF_eq1} 的形式. 详见 “线性势能的定态薛定谔方程\upref{LinPot}”. 该势能的一个应用是 WKB 近似\upref{WKB}, WKB 近似是量子力学中解定态薛定谔方程的一种近似方法.
