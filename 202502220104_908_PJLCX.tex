% 庞加莱猜想(综述)
% license CCBYSA3
% type Wiki

本文根据 CC-BY-SA 协议转载翻译自维基百科\href{https://en.wikipedia.org/wiki/Poincar\%C3\%A9_conjecture}{相关文章}。

在几何拓扑的数学领域,庞加莱猜想(英国发音:/ˈpwæ̃kæreɪ/,美国发音:/ˌpwæ̃kɑːˈreɪ/,法语发音:[pwɛ̃kaʁe])是一个关于3-球体的定理,3-球体是四维空间中界定单位球的超球面。

该猜想最初由亨利·庞加莱于1904年提出,定理涉及到一些局部上看起来像普通三维空间,但在规模上是有限的空间。庞加莱假设,如果这样的空间具有一个额外的特性,即该空间中的每一条环路都可以连续地收缩到一个点,那么这个空间必定是一个三维球体。解决该猜想的努力推动了20世纪几何拓扑领域的许多进展。

最终的证明基于理查德·S·汉密尔顿使用Ricci流来解决该问题的方案。通过在Ricci流理论中发展一系列新的技术和结果,格里戈里·佩雷尔曼能够修改并完成汉密尔顿的方案。在2002年和2003年发布于arXiv的论文中,佩雷尔曼展示了他证明庞加莱猜想(以及威廉·瑟尔斯顿的更强的几何化猜想)工作的过程。随后几年,几位数学家研究了他的论文,并详细阐述了他的工作。

汉密尔顿和佩雷尔曼在该猜想上的工作被广泛认为是数学研究的一个里程碑。汉密尔顿因其贡献获得了邵逸夫奖和勒罗伊·P·斯蒂尔研究奠基奖。《科学》杂志将佩雷尔曼证明庞加莱猜想的成果评为2006年年度科学突破。[5] 克雷数学研究所将庞加莱猜想列为著名的千年奖问题之一,并为该猜想的解决提供了100万美元的奖赏。[6] 佩雷尔曼拒绝了这一奖项,称汉密尔顿的贡献与他自己的贡献相等。[7][8]
\subsection{概述}
\begin{figure}[ht]
\centering
\includegraphics[width=6cm]{./figures/bb28959dfa154fe6.png}
\caption{这个环面上的两个彩色环都不能连续地收缩到一个点。环面与球面不是同胚的。} \label{fig_PJLCX_1}
\end{figure}
庞加莱猜想是几何拓扑学领域的一个数学问题。用该领域的术语来说,它表述如下:

庞加莱猜想:  
每一个封闭、连通且基本群为平凡的三维拓扑流形,与三维球面是同胚的。

