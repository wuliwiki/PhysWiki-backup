% Egoroff定理
% 叶戈洛夫定理|实变函数|一致收敛

在讨论函数极限的相关问题时,我们常需要函数列具有\textbf{一致收敛}的性质.反过来,观察一个不一致收敛的函数列,比如$\{f_n(x)=x^n\}$在区间$[0, 1]$上就不一致收敛,我们会发现如果把区间挖掉长度$\epsilon$\textbf{任意}小的一部分,那么$\{f_n\}$在$[0, 1-\epsilon]$上总是一致收敛的.这提示我们研究,任意收敛函数列是否可以去掉一个小部分以后是一致收敛的?

答案是肯定的,这就是本节要讨论的Egoroff定理.

\subsection{Egoroff定理}

\begin{theorem}{Egoroff定理}

设$E$是$\mathbb{R}^n$上测度有限的可测集合.如果$\{f_n\}$是$E$上的一列函数,都几乎处处取有限函数值,而存在$E$上的函数$f(x)$使得\footnote{即$f_n$几乎处处收敛于$f$,或者说不收敛点构成一个零测集.}\begin{equation}
\lim\limits_{n\to\infty}f_n(x)=f(x)a. e. 
\end{equation}

那么对于任意正数$\epsilon$,必存在一个可测集$E_\epsilon$,使得$\opn{m}E_\epsilon<\epsilon$,且$f_n$在$E-E_\epsilon$上一致收敛于$f$.

\end{theorem}

\textbf{证明}:



\textbf{证毕}.










