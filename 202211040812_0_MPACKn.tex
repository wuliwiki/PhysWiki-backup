% MPACK 笔记

\begin{issues}
\issueDraft
\end{issues}

\begin{itemize}
\item MPACK 是一个高精度计算库, 支持四精度(double double, \verb|dd|), 八精度(qd),乃至任意精度的 BLAS\upref{BLAS} 和 LAPACK\upref{Lapack}. 源码主要是 C++.
\item \href{https://mplapack.sourceforge.net/}{官方主页} (然而这里下载的安装包在 ubuntu 22.04 下编译出错.
\item 【更新】最新版改名为 MPLAPACK, 见 \href{https://github.com/nakatamaho/mplapack}{GitHub 仓库}.
\item 另见一个 \href{https://github.com/Auto-Mech/MPACK}{github 页面}, 这里说可以直接用 conda 下载编译好的 lib: \verb|conda install -c auto-mech mpack|. 亲测有效, 安装路径在 \verb|~/miniconda3/pkgs/mpack-0.6.8-0/|
\item \verb|x86| 下的 Ubuntu 22.04 也可以直接从笔者的 \href{https://github.com/MacroUniverse/MPACK-source}{github 仓库}下载.
\item 四精度需要使用的 lib 应该是 \verb|libmlapack_dd.so|, 头文件 \verb|mlapack_dd.h|
\item 源码也可以在笔者的 \href{https://github.com/MacroUniverse/MPACK-source}{github} 下载.
\item 编译例子 \verb|inv_mat_dd| 成功! 路径在 \verb|mpack-0.6.7/examples/mlapack/|:
\item \verb|g++ inv_mat_dd.cpp -I ~/miniconda3/pkgs/mpack-0.6.8-0/include/mpack -I ~/miniconda3/pkgs/mpack-0.6.8-0/include -L ~/miniconda3/pkgs/mpack-0.6.8-0/lib -Wl,-rpath,/home/addis/miniconda3/envs/mpack/lib -l mlapack_dd -l mblas_dd -l qd|
\end{itemize}

\subsection{四精度 LAPACK, 基于 QD 库}
\begin{itemize}
\item 虽然 MPACK 主页上说可以使用 GCC 的 \verb|__float128|, 但源码中并没有搜到任何相关内容, 似乎目前还是只能使用 \verb|QD| 库.
\item 见 \verb|example| 里面的 \verb|*_dd.cpp| 源码
\item 普通整数 \verb|mpackint| 类型为 \verb|long long| (8 字节符号整数)
\item 四精度实数 \verb|dd_real| 类的定义为(头文件 \verb|include/qd/dd_real.h|, 这其实是 QD 库\upref{libQD}的头文件, 有微量修改.)
\begin{lstlisting}[language=cpp]
struct dd_real {
    double x[2];
  public:
    dd_real(double, double);
    dd_real(void);
    dd_real(double);
    dd_real(int);
    dd_real(const char *);
    dd_real(const double *);
    ...
};
\end{lstlisting}
\item 四精度复数 \verb|dd_complex| 类的定义为(头文件 \verb|include/mpack/dd_complex.h|)
\begin{lstlisting}[language=cpp]
class dd_complex {
  private:
    dd_real re;
    dd_real im;

  public:
    dd_complex(void);
    dd_complex(const dd_complex &);
    dd_complex(const dd_real &, const dd_real &);
    dd_complex(const dd_real &);
    dd_complex(const std::complex<double> &);
    const dd_real & real(void) const;
    dd_real & real(void);
    const dd_real & imag(void) const;
    dd_real & imag(void);
    dd_complex & operator=(std::complex<double>);
    dd_complex & operator=(dd_real);
    dd_complex & operator=(double);
    dd_complex & operator+=(const dd_complex &);
    dd_complex & operator+=(const dd_real &);
\end{lstlisting}
\item \verb|mlapack_dd.h| 中的 lapack 函数以 \verb|R|(实数)或者 \verb|C|(复数) 开头, 例如 \verb|?gbsv| 函数为 \verb|Rgbsv|. 函数的用法应该都是一样的.
\begin{lstlisting}[language=cpp]
void Rgbsv(mpackint n, mpackint kl, mpackint ku, mpackint nrhs,
dd_real *AB, mpackint ldab, mpackint *ipiv, dd_real *B,
mpackint ldb, mpackint *info);

void Cgbsv(mpackint n, mpackint kl, mpackint ku, mpackint nrhs,
dd_complex *ab, mpackint ldab, mpackint *ipiv, dd_complex *b,
mpackint ldb, mpackint *info);
\end{lstlisting}
\end{itemize}
