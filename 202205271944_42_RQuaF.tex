% 实二次型
% keys 实二次型|惯性定律|惯性指数

\begin{issues}
\issueDraft
\end{issues}

\pentry{二次型的规范型\upref{GuaOQu}}
\subsection{实二次型}
在二次型的规范型\upref{GuaOQu}一节里,已经知道,在任意域 $\mathbb F$ 上的二次型都具有规范型(或对角型)\autoref{QuaFor_def1}~\upref{QuaFor}.一般来说,规范型是二次型最简单的形式.但是,在域为实数域,即 $\mathbb F=\mathbb R$ 时,可以让规范型\autoref{GuaOQu_eq1}~\upref{GuaOQu}
\begin{equation}
q(\bvec{x})=\lambda_1 x_1^2+\cdots+\lambda_r x_r^2
\end{equation}
的所有系数 $\lambda_i$ 均为 $\pm 1$.
\begin{definition}{实二次型}
若二次型 $q$ 所配备的矢量空间 $V$ 定义在实数域 $\mathbb R$ 上,则 $q$ 称为\textbf{实二次型}.
\end{definition}
适当置换基底矢量,可认为前 $s$ 个系数 $\lambda_1,\cdots,\lambda_s$ 是正的,而其余的系数是负的.进行替换
\begin{equation}
\begin{aligned}
&x'_i=\sqrt{\lambda_i}x_i,\;&1\leq i\leq s;\\
&x'_i=\sqrt{-\lambda_i}x_i,\;&s+1\leq i\leq r;\\
&x'_i=x_i,\; &r+1\leq i\leq n
\end{aligned}
\end{equation}
 即得
 \begin{equation}
 q(\bvec{x})=\sum_{i=1}^{s}{x'_i}^2-\sum_{i=s+1}^{r} {x'_{i}}^{2}.
 \end{equation}
 而若对于有理数域 $\mathbb Q$,在 $\sqrt{\lambda_i}$ 为无理数时并不能作这样的简化.
 
\begin{definition}{标准型}
称可以按公式
\begin{equation}
q(\bvec x)=\sum_{i=1}^{s}{x_i}^2-\sum_{i=s+1}^{r} {x_{i}}^{2}
\end{equation}
计算值的二次型有\textbf{标准型}.
\end{definition}
由上面的讨论立刻得
\begin{theorem}{}
实矢量空间 $V$ 上的所有二次型 $q$ 均可化为标准型.
\end{theorem}
\subection{惯性定理}
