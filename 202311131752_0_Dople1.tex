% 多普勒效应(一维匀速)
% keys 多普勒效应|声波|频率|波长|周期
% license Xiao
% type Tutor

\pentry{平面波\upref{PWave}}

\textbf{多普勒效应(Doppler effect)}讨论的是, 当机械波的波源和(或)接收者相对于波的介质运动时, 发射的频率和接收到的频率之间的变化关系。 本文只讨论波源和接收者沿同一直线匀速运动的情况。 另外本文不讨论相对论效应, 即假设波速远小于真空中的光速(另见“光的多普勒效应\upref{RelDop}”)。

\begin{figure}[ht]
\centering
\includegraphics[width=9cm]{./figures/8d78838f91bf5288.pdf}
\caption{多普勒效应} \label{fig_Dople1_1}
\end{figure}

\begin{example}{}
生活中一种常见的多普勒效应是, 一辆疾驰的车一边鸣笛一边驶过行人, 人听到的音调就会先高后低。 这是因为, 车经过人之前不断靠近人, 经过人后再不断远离人。 可见多普勒效应和运动的速度有关。
\end{example}

在分析多普勒效应时, 一种方便的做法是选取介质为参考系, 例如有均匀的风时, 参考系随风运动。 假设介质处处均匀且静止, 波在介质中传播的速度(\textbf{波速}) $v$ 处处相等, 且与方向无关(\textbf{各向同性})。

令\autoref{fig_Dople1_1} 中波源 1 以速度 $v_1$ (向右为正)做匀速运动,当前位置为 $x_1$。 接收者 2 以速度 $v_2$ (向右为正)做匀速运动,当前位置为 $x_2$。 令波源的频率为 $f_1$, 接收者收到的频率为 $f_2$。 我们假设二者的速度均小于波速($\abs{v_1}, \abs{v_2} < v$), 有
\begin{equation}\label{eq_Dople1_1}
\frac{f_2}{f_1} = \leftgroup{
&\frac{v - v_2}{v - v_1} \qquad (x_1 < x_2)\\
&\frac{v + v_2}{v + v_1} \qquad (x_1 > x_2)~.
}\end{equation}
推导见下文。 注意 $x_1=x_2$ 时刻不做讨论,因为该时刻的频率没有良好的定义。

作为一个形象的解释, 一维多普勒效应可以等效为追及问题, 可以想象 1 以一定的频率 $f_1$ 向 2 发射速度为 $v$ 的子弹, 子弹的位置对应波峰的位置, 两个相邻子弹之间的间距对应波长。 若 1,2 相对介质静止不动或者以相同的速度运动, 则 2 接收到子弹的频率和 1 发射的频率是一样的, 但若 1, 2 不断靠近, 则接收子弹的频率就会更高, 若不断远离, 则接受的频率就更低。

\begin{exercise}{}
在一条长直马路上, 风速为 $10\Si{m/s}$, 一个单车以 $5\Si{m/s}$ 向顺风而行, 迎面驶来一辆摩托车, 速度为 $20\Si{m/s}$。 若摩托车一直以 $600\Si{Hz}$ 的频率鸣喇叭, 声速为 $340\Si{m/s}$, 求摩托车经过单车前后骑单车的人听到的频率。

令风马路为 $x$ 轴, 风向为正方向, 在风的参考系中, 摩托车速度为
\begin{equation}
v_1 = -20 - 10 = -30(\Si{m/s})~.
\end{equation}
单车速度为
\begin{equation}
v_2 = 5 - 10 = -5\Si{m/s}~.
\end{equation}
另外 $v = 340\Si{m/s}$, 相遇以前 $x_2 < x_1$, 所以
\begin{equation}
f_2 = \frac{v + v_2}{v + v_1} f_1 = 648.4\Si{Hz}~.
\end{equation}
相遇以后 $x_1 < x_2$, 所以
\begin{equation}
f_2 = \frac{v - v_2}{v - v_1} f_1 = 559.5\Si{Hz}~.
\end{equation}
\end{exercise}
注意我们必须在风的参考系中计算, 否则得不到正确的结果(除非 $v_1 = v_2$)。

\subsection{反射的多普勒效应}
当我们考虑波从一个运动的点上反射。 每当这个点接收一个周期的波,就同时反射一个周期的波。 所以我们可以把这个点作为一个新的波源, 发射的频率与接收的频率相同。 这样我们就可以重复使用\autoref{eq_Dople1_1} 得到最终接收者的频率。

\begin{example}{}
假设 $x_3 < x_1 < x_2 < x_4$, 当波从 $x_1$ 发出, 经过 $x_2, x_3$ 分别反射, 被 $x_4$ 接收, 有
\begin{equation}
\frac{f_4}{f_1} = \frac{f_4}{f_3}\frac{f_3}{f_2}\frac{f_2}{f_1}
= \frac{v - v_4}{v - v_3}\frac{v + v_3}{v + v_2}\frac{v - v_2}{v - v_1}~.
\end{equation}
可见如果这四个点都向中间运动, 波每反射一次都会使频率变快。
\end{example}

\subsection{推导}
现在推导\autoref{eq_Dople1_1}。 以 $x_1 < x_2$ 为例, 假设波源在某时刻在 $x_{10}$ 位置向右发射一枚子弹, 经过周期 $T_1 = 1/f_1$ 后又发射第二枚时, 第一枚的位置为 $x_{10} + vT_1$, 而此时波源的位置(也就是第二枚子弹的位置)为 $x_{10} + v_1 T_1$。 所以两枚子弹相距(也就是波长)为
\begin{equation}
\lambda = (v - v_1)T_1~.
\end{equation}
同理, 接收者收到这两枚子弹的时间间隔为
\begin{equation}
T_2 = \frac{\lambda}{v - v_2}~.
\end{equation}
以上两式消去 $\lambda$, 有
\begin{equation}
\frac{f_2}{f_1} = \frac{T_1}{T_2} = \frac{v - v_2}{v - v_1}~,
\end{equation}
这样就得到了\autoref{eq_Dople1_1}。 $x_2 < x_1$ 的推导同理可得。
