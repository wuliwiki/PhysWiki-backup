% 约翰·冯诺依曼(综述)
% license CCBYSA3
% type Wiki

本文根据 CC-BY-SA 协议转载翻译自维基百科\href{https://en.wikipedia.org/wiki/John_von_Neumann}{相关文章}。

\begin{figure}[ht]
\centering
\includegraphics[width=6cm]{./figures/c4c9226c223e913e.png}
\caption{冯·诺依曼在1940年代} \label{fig_Neuman_1}
\end{figure}
约翰·冯·诺依曼(John von Neumann,1903年12月28日—1957年2月8日)是一位匈牙利裔美国数学家、物理学家、计算机科学家和工程师。冯·诺依曼可能是他那个时代涵盖面最广泛的数学家之一,他将纯粹科学和应用科学相结合,并对许多领域作出了重要贡献,包括数学、物理学、经济学、计算机学和统计学。他是量子物理学数学框架建设的先驱,在泛函分析和博弈论的发展中也做出了突出贡献,提出或规范了包括细胞自动机、通用构造器和数字计算机等概念。他对自我复制结构的分析,早于DNA结构的发现。

在第二次世界大战期间,冯·诺依曼参与了曼哈顿计划,他开发了用于爆炸透镜的数学模型,这些透镜在内爆型核武器中起到了重要作用。战前和战后,他为许多组织提供咨询服务,包括科学研究与发展办公室、陆军弹道研究实验室、武装部队特殊武器计划和橡树岭国家实验室等。在1950年代的巅峰时期,他主持了多个国防部委员会,包括战略导弹评估委员会和洲际弹道导弹科学顾问委员会。他还是负责全国所有原子能开发的影响力巨大的原子能委员会的成员。在与伯纳德·施里弗和特雷弗·加德纳的合作中,他在美国首个洲际弹道导弹(ICBM)项目的设计和开发中扮演了关键角色。那时,他被认为是美国核武器方面的顶尖专家,也是美国国防部的首席防御科学家。

冯·诺依曼的贡献和智力能力得到了物理学、数学及其他领域同事的高度赞扬。他所获得的荣誉包括自由勋章以及以他名字命名的月球陨石坑。
\subsection{生活与教育}  
\subsubsection{家庭背景}  
冯·诺依曼于1903年12月28日出生在匈牙利王国的布达佩斯(当时是奥匈帝国的一部分),出生于一个富裕的、不信教的犹太家庭。他的出生名为Neumann János Lajos。在匈牙利语中,姓氏排在前面,而他的名字相当于英语中的John Louis。

他是家中三个兄弟中的长子,两个弟弟分别是米哈伊(Mihály,迈克尔)和尼克拉斯(Miklós,尼古拉斯)。他的父亲Neumann Miksa(Max von Neumann)是一位银行家,拥有法学博士学位。父亲于1880年代末从佩奇(Pécs)搬到布达佩斯。Miksa的父亲和祖父均出生于匈牙利北部的翁德(Ond,现在是泽伦茨(Szerencs)的一部分),位于泽普伦(Zemplén)县。冯·诺依曼的母亲是Kann Margit(玛格丽特·坎);她的父母是Kann Jákab和Meisels Katalin,来自Meisels家族。坎家三代人居住在布达佩斯坎-赫勒事务所上方的宽敞公寓里;冯·诺依曼的家庭占据了顶层的18个房间。

1913年2月20日,弗朗茨·约瑟夫皇帝因冯·诺依曼的父亲为奥匈帝国的服务,将他提升为匈牙利贵族。因此,诺依曼家族获得了“Margittai”这一世袭称号,意为“来自马吉塔”(如今的罗马尼亚马尔吉塔)。家族与该镇并无实际联系,这个称号是为了纪念玛格丽特而选择的,他们的家族徽章也描绘了三朵雏菊。冯·诺依曼János成为了“margittai Neumann János”(约翰·诺依曼·德·马吉塔),后来他将其更改为德语的“Johann von Neumann”。
\subsubsection{天才儿童}  
冯·诺依曼是一个天才儿童,六岁时就能在脑海中计算两个八位数的除法,并能用古希腊语交流。他和他的兄弟、表兄弟们由家庭教师教授。冯·诺依曼的父亲认为,除了母语匈牙利语之外,掌握其他语言是非常重要的,因此孩子们都接受了英语、法语、德语和意大利语的辅导。到八岁时,冯·诺依曼已经熟悉微积分,十二岁时他读过了布雷尔(Borel)的《函数论》(La Théorie des Fonctions)。他还对历史感兴趣,阅读了威廉·昂肯(Wilhelm Oncken)的46卷世界历史丛书《普通历史的个别阐述》(Allgemeine Geschichte in Einzeldarstellungen)。公寓中的一间房间被改成了图书馆和阅览室。

冯·诺依曼于1914年进入了路德教会的法索里·厄尔文吉库斯高中(Lutheran Fasori Evangélikus Gimnázium)。尤金·维格纳(Eugene Wigner)比冯·诺依曼大一岁,成为了冯·诺依曼在学校的朋友。

尽管冯·诺依曼的父亲坚持让他按年龄进入相应的年级,但他同意雇佣私人教师为冯·诺依曼提供更高水平的教学。15岁时,他开始在数学分析家加博尔·塞戈(Gábor Szegő)的指导下学习高阶微积分。塞戈第一次见到冯·诺依曼时,被他的数学才能和学习速度震惊得无以复加,正如他的妻子回忆的那样,塞戈带着泪水回到家中。19岁时,冯·诺依曼已发表了两篇重要的数学论文,其中第二篇给出了现代的序数定义,取代了乔治·康托尔(Georg Cantor)原有的定义。在完成高中学业后,冯·诺依曼申请并获得了厄尔托斯奖(Eötvös Prize),这是匈牙利的一项国家数学奖项。
\subsubsection{大学学习}
根据他的朋友西奥多·冯·卡门(Theodore von Kármán)的说法,冯·诺依曼的父亲希望约翰能够继承他的事业进入工业界,并要求冯·卡门劝说冯·诺依曼不要选择数学。冯·诺依曼和父亲决定,最好的职业道路是化学工程。冯·诺依曼对化学工程并不太了解,因此安排他在柏林大学参加为期两年的非学位化学课程,之后参加了苏黎世联邦理工学院(ETH Zurich)的入学考试,并在1923年9月通过了考试。与此同时,冯·诺依曼还进入了帕兹玛尼·彼得大学(当时称为布达佩斯大学),作为数学博士候选人。他的博士论文是关于康托尔集论的公理化。

1926年,他从苏黎世联邦理工学院获得化学工程学位,并同时以优异成绩通过了布达佩斯大学数学博士的最终考试(辅修实验物理学和化学)。

随后,他凭借洛克菲勒基金会的奖学金前往哥廷根大学,师从大卫·希尔伯特(David Hilbert)学习数学。赫尔曼·外尔(Hermann Weyl)回忆道,在1926至1927年冬季,冯·诺依曼、艾米·诺特(Emmy Noether)和他常常在课后一起走在“哥廷根寒冷、湿滑、被雨水打湿的街道上”,讨论超复数系统及其表示方法。
\subsection{职业与私生活}
\begin{figure}[ht]
\centering
\includegraphics[width=8cm]{./figures/970bcb0b4439142d.png}
\caption{摘自1928年和1928/29学年《弗里德里希·威廉大学柏林校历》的内容,宣布冯·诺依曼的讲座主题包括:函数理论II、公理化集合论和数学逻辑、数学座谈会、量子力学的最新研究综述、数学物理的特殊函数以及希尔伯特的证明理论。他还讲授了相对论理论、集合论、积分方程以及无限多个变量的分析。} \label{fig_Neuman_2}
\end{figure}
冯·诺依曼于1927年12月13日完成了他的“资格讲师”(Habilitation)资格认证,并于1928年开始在柏林大学担任资格讲师(Privatdozent)。他成为该大学历史上最年轻的资格讲师。他开始几乎每月撰写一篇重要的数学论文。1929年,他短暂地成为汉堡大学的资格讲师,因为在那里成为终身教授的前景更好;然后,他在同年10月转至普林斯顿大学,担任数学物理学的访问讲师。

冯·诺依曼于1930年接受了天主教洗礼。不久后,他与玛丽埃塔·科韦西结婚,玛丽埃塔曾在布达佩斯大学学习经济学。冯·诺依曼和玛丽埃塔育有一女,玛丽娜,生于1935年;她后来成为一名教授。夫妻俩于1937年11月2日离婚。1938年11月17日,冯·诺依曼与克拉拉·丹(Klára Dán)结婚。

1933年,冯·诺依曼接受了新泽西州高等研究院的终身教授职位,当时该机构原计划任命赫尔曼·外尔的计划似乎未能实现。1939年,冯·诺依曼的母亲、兄弟和亲家一同前往美国。冯·诺依曼将自己的名字改为英文名“John”,但保留了德式贵族姓氏“von Neumann”。冯·诺依曼于1937年成为美国公民,并立即尝试成为美国陆军后备军官团的一名中尉。他通过了考试,但因年龄问题被拒绝。

克拉拉与约翰·冯·诺依曼在当地学术界十分活跃。他们位于西科特路的白色木板房屋是普林斯顿最大的私人住宅之一。冯·诺依曼总是穿着正式的西装。他喜欢犹太语和“有点不雅”的幽默。在普林斯顿,他曾因播放极其响亮的德国进行曲音乐而受到投诉。冯·诺依曼在嘈杂、混乱的环境中做出了他的一些最佳工作。根据丘吉尔·艾森哈特(Churchill Eisenhart)的说法,冯·诺依曼曾参加过派对直到凌晨,然后在8:30准时开始讲座。

冯·诺依曼以乐于为各个层次的学者提供科学和数学建议而闻名。维格纳(Wigner)曾写道,他可能比任何其他现代数学家都更“随意地”指导了更多的工作。他的女儿写道,冯·诺依曼非常关心自己在两个方面的遗产:他的生活和他对世界的知识贡献的持久性。

许多人认为冯·诺依曼是一个出色的委员会主席,他在个人或组织事务上较为宽容,但在技术问题上则表现得非常坚持。赫伯特·约克(Herbert York)描述了冯·诺依曼参与的许多“冯·诺依曼委员会”,认为它们“在风格和成果上都非常出色”。冯·诺依曼主持的委员会与必要的军事或企业实体密切合作,这种工作方式成为了所有空军远程导弹项目的蓝图。许多曾与冯·诺依曼接触的人对他与军事及权力结构的关系感到困惑。斯坦尼斯瓦夫·乌拉姆(Stanisław Ulam)怀疑冯·诺依曼对那些能够影响他人思维和决策的人或组织有着隐秘的钦佩。

冯·诺依曼还保持着他年轻时学过的语言能力。他能流利地使用匈牙利语、法语、德语和英语,并保持一定的意大利语、犹太语、拉丁语和古希腊语会话水平。他的西班牙语水平较差。他对古代历史充满热情,并具备百科全书般的知识,喜欢用原文阅读古希腊历史学家的著作。乌拉姆怀疑这些可能影响了冯·诺依曼对未来事件的看法,以及他对人性和社会运行方式的理解。

冯·诺依曼在美国最亲密的朋友是数学家斯坦尼斯瓦夫·乌拉姆。冯·诺依曼认为,他的大部分数学思维是直觉性的;他常常在睡觉时带着一个未解答的问题,醒来后便知道了答案。乌拉姆指出,冯·诺依曼的思维方式可能并非视觉性的,而更偏向听觉性。乌拉姆回忆道:“完全不论他对抽象智慧的喜爱,他对更接地气的喜剧和幽默有着强烈的欣赏(可以说几乎是渴望)。”
\subsubsection{疾病与死亡}
\begin{figure}[ht]
\centering
\includegraphics[width=6cm]{./figures/4dbffa775d803dc1.png}
\caption{冯·诺依曼的墓碑} \label{fig_Neuman_3}
\end{figure}
1955年,冯·诺依曼的锁骨附近发现了一个肿块,最终被确诊为源自骨骼、胰腺或前列腺的癌症。(尽管普遍认为肿瘤已经发生了转移,但不同的资料来源对于原发性癌症的位置存在不同看法。)这恶性肿瘤可能是由于在洛斯阿拉莫斯国家实验室接触辐射所致。随着死亡临近,他请求了一位神父,尽管这位神父后来回忆道,冯·诺依曼在接受最后的圣事时几乎没有得到安慰——他仍然对死亡感到恐惧,且无法接受它。关于冯·诺依曼的宗教观点,据说他曾说:“只要非信徒可能面临永恒的诅咒,那么在最后时刻成为信徒更为合逻辑,”这是指帕斯卡尔的赌注。他曾对母亲倾诉:“或许必须有一个上帝。如果有的话,许多事情比没有上帝时更容易解释。”

他于1957年2月8日在沃尔特·里德陆军医疗医院去世,享年53岁,并被埋葬在普林斯顿墓地。
\subsection{数学}  
\subsubsection{集合论}
\begin{figure}[ht]
\centering
\includegraphics[width=6cm]{./figures/4efc733fe9f8afc1.png}
\caption{导致NBG集合论的各种方法的历史} \label{fig_Neuman_4}
\end{figure}
20世纪初,试图将数学建立在朴素集合论基础上的努力遭遇了挫折,原因是拉塞尔悖论(即“所有不属于自己的集合的集合”)的出现。解决集合论充分公理化的问题,大约在二十年后由恩斯特·采尔梅洛(Ernst Zermelo)和亚伯拉罕·弗兰克尔(Abraham Fraenkel)间接解决。采尔梅洛-弗兰克尔集合论提供了一系列原理,允许构造日常数学实践中使用的集合,但并未明确排除存在自包含集合的可能性。在他1925年的博士论文中,冯·诺依曼展示了两种排除这种集合的技巧——基础公理和类的概念。

基础公理提出,每个集合都可以通过采尔梅洛-弗兰克尔原理,从底部向上按顺序构造。如果一个集合属于另一个集合,那么第一个集合必须先于第二个集合出现在这个顺序中。这排除了集合自包含的可能性。为了证明添加这个新公理不会产生矛盾,冯·诺依曼引入了内模型方法,这成为集合论中的一个重要证明工具。

解决自包含集合问题的第二种方法以“类”的概念为基础,并定义一个集合为属于其他类的类,而定义一个适当类为不属于其他类的类。在采尔梅洛-弗兰克尔方法中,公理阻止了构造一个包含所有不属于自己的集合的集合。而在冯·诺依曼的方法中,可以构造所有不属于自己的集合的类,但它是一个适当类,而不是集合。

总体而言,冯·诺依曼在集合论方面的主要成就之一是“集合论的公理化和(与之相关的)优雅的序数与基数理论,以及通过超限归纳法严格表述定义原理”。

\textbf{冯·诺依曼悖论}  

在费利克斯·豪斯多夫(Felix Hausdorff)于1914年提出的豪斯多夫悖论的基础上,斯特凡·巴拿赫(Stefan Banach)和阿尔弗雷德·塔尔斯基(Alfred Tarski)于1924年展示了如何将一个三维球体分割成不相交的集合,然后通过平移和旋转这些集合,形成两个完全相同的球体副本;这就是著名的巴拿赫–塔尔斯基悖论。他们还证明了一个二维圆盘没有这样的悖论性分解。然而,在1929年,冯·诺依曼将圆盘分割成有限多个部分,并通过面积保持的仿射变换,而非平移和旋转,将这些部分重新排列成两个圆盘。这个结果依赖于找到仿射变换的自由群,这是一个重要的技术,后来冯·诺依曼在他的测度论工作中进一步扩展了这一方法。
\subsubsection{证明论} 
冯·诺依曼对集合论的贡献使得该集合理论的公理系统避免了早期系统的矛盾,并成为数学基础的可用框架,尽管没有证明其一致性。下一个问题是:它是否为所有可以在其中提出的数学问题提供了确定的答案,还是可以通过添加更强的公理来改进,从而能够证明更广泛类别的定理。[95]

到1927年,冯·诺依曼开始参与哥廷根的讨论,是否可以从佩阿诺公理中推导出基本的算术。他在阿克曼(Ackermann)的工作基础上,开始尝试用希尔伯特学派的有限方法证明一阶算术的一致性。他成功地证明了自然数算术的一个片段的一致性(通过对归纳法的限制)。[97] 他继续寻找使用证明论方法证明经典数学一致性的更一般的证明。[98]

关于这一系统是否是确定性的,1930年9月,在第二届精确科学认识论大会上给出了一个强烈的否定答案。在大会上,库尔特·哥德尔宣布了他的第一不完备定理:常见的公理系统是不完备的,意味着它们无法证明在其语言中表达的每一个真理。此外,这些系统的任何一致扩展必然仍然是不完备的。[99] 在大会上,冯·诺依曼建议哥德尔应该尝试将他的结果转化为关于整数的不可判定命题。[100]

不到一个月后,冯·诺依曼向哥德尔传达了他定理的一个有趣结论:常见的公理系统无法证明它们自身的一致性。[99] 哥德尔回复说,他已经发现了这个结论,也就是他的第二不完备定理,并且他将发送包含这两个结果的文章的预印本,但该文章从未公开。[101][102][103] 冯·诺依曼在下一封信中承认了哥德尔的优先权。[104] 然而,冯·诺依曼的证明方法与哥德尔不同,他还认为第二不完备定理对希尔伯特计划的打击要比哥德尔认为的要大得多。[105][106] 这一发现大大改变了他对数学严格性的看法,冯·诺依曼停止了对数学基础和元数学的研究,转而专注于与应用相关的问题。[107]
\subsubsection{遍历理论}  
在1932年发表的一系列论文中,冯·诺依曼对遍历理论做出了基础性贡献,遍历理论是数学的一个分支,涉及具有不变测度的动态系统的状态。[108] 对于冯·诺依曼在1932年发表的遍历理论论文,保罗·哈尔莫斯(Paul Halmos)曾写道,即使“冯·诺依曼再也没有做过其他任何事情,这些论文也足以确保他获得数学上的不朽声誉”。[109] 到那时,冯·诺依曼已经写过关于算子理论的文章,这些工作的应用在他的均值遍历定理中发挥了重要作用。[110]

该定理涉及任意的单参数酉群 \(t \to V_{t}\)并声明,对于希尔伯特空间中的每个向量 \(\phi\)
,\(\lim_{T \to \infty} \frac{1}{T} \int_0^T V_{t}(\phi) \, dt\)在由希尔伯特范数定义的度量意义下是存在的,并且是一个向量 \(\psi\),使得\(V_{t}(\psi) = \psi\)对于所有\(t\)。这一结论在第一篇论文中得到了证明。在第二篇论文中,冯·诺依曼认为他的结果足以应用于与玻尔兹曼的遍历假设相关的物理应用。他还指出,遍历性尚未实现,并将其孤立出来,作为未来研究的课题。[111]

同年晚些时候,他发表了另一篇具有影响力的论文,开始系统地研究遍历性。他给出了并证明了一个分解定理,展示了实数轴上的遍历测度保持作用是所有测度保持作用的基本构建块,所有其他测度保持作用都可以由此构建。这篇论文中给出并证明了几个其他关键定理。这些结果与保罗·哈尔莫斯的另一篇论文共同,在数学的其他领域有着重要应用。[111][112]
\subsubsection{测度理论}  
在测度理论中,n维欧几里得空间 \( \mathbb{R}^n \) 的“测度问题”可以表述为:“是否存在一个正的、归一化的、不变的、可加的集合函数,定义在 \( \mathbb{R}^n \) 所有子集的类上?”[113] 费利克斯·豪斯多夫(Felix Hausdorff)和斯特凡·巴拿赫(Stefan Banach)的工作暗示,当 \( n = 1 \) 或 \( n = 2 \) 时,测度问题有正解,而在其他情况下则为负解(因为有班纳赫–塔尔斯基悖论)。冯·诺依曼的工作认为,“这个问题本质上是群论性质的”:测度的存在可以通过观察给定空间的变换群的性质来确定。对于最多二维的空间,正解存在;对于更高维的空间,负解存在,原因在于欧几里得群对于最多二维的空间是可解群,而对于高维空间则不是可解的。“因此,根据冯·诺依曼的观点,是群的变化才决定了差异,而不是空间的变化。”[114] 大约在1942年,他告诉多萝西·马哈拉姆(Dorothy Maharam)如何证明每个完全 σ-有限测度空间都有一个乘法提升;他没有发表这个证明,后来她提出了一个新的证明。[115]

在冯·诺依曼的许多论文中,他采用的论证方法被认为比结果本身更为重要。为了预示他日后对算子代数中的维数理论的研究,冯·诺依曼使用了有限分解等价的结果,并将测度问题重新表述为关于函数的问题。[116] 冯·诺依曼对测度理论的主要贡献之一是,他写了一篇论文回答哈尔(Haar)提出的问题,关于是否存在一个所有有界函数的代数,这些函数定义在实数轴上,并且它们构成“几乎处处相等的可测有界函数类的代表系统”。[117] 他证明了这个问题的正解,并在后来的论文中与斯通(Stone)讨论了这个问题的各种推广和代数方面。[118] 他还通过新的方法证明了各种一般类型测度的分解存在性。冯·诺依曼还通过使用函数的平均值,给出了哈尔测度唯一性的一个新证明,尽管这个方法仅适用于紧群。[117] 他不得不创造全新的技术,将此应用到局部紧群。[119] 他还为拉东–尼科迪姆定理提供了一个新的巧妙证明。[120] 他在高等研究所的测度理论讲义是当时美国关于该主题的重要知识来源,后来被出版。[121][122][123]
\subsubsection{拓扑群}
利用他之前在测度理论方面的工作,冯·诺依曼对拓扑群理论做出了几项贡献,首先是关于群上几乎周期函数的论文,其中冯·诺依曼将博尔(Bohr)的几乎周期函数理论扩展到任意群[124]。他与博赫纳(Bochner)合作,进一步改进了几乎周期性的理论,使其包括取线性空间元素作为值的函数,而不仅仅是数值[125]。1938年,他因与这些论文相关的分析工作获得了博赫纪念奖[126][127]。

在1933年的一篇论文中,他在解决希尔伯特第五问题时使用了新发现的哈尔测度,特别是针对紧群的情况[128]。这个基本思想几年前就被发现,当时冯·诺依曼发表了一篇关于线性变换群的解析性质的论文,发现一般线性群的闭子群是李群[129]。这一结果后来由卡尔坦(Cartan)扩展到任意李群,形成了闭子群定理[130][117]。
\subsubsection{泛函分析}  
冯·诺依曼是第一个公理化定义抽象希尔伯特空间的人。他将其定义为具有厄米标量积的复向量空间,并且对应的范数既可分离又完备。在同一篇论文中,他还证明了柯西–施瓦茨不等式的一般形式,而该不等式之前仅在特定的例子中被知道[131]。他在1929年至1932年间的三篇开创性论文中继续发展了希尔伯特空间算子的谱理论[132]。这项工作 culminated 在他的《量子力学的数学基础》一书中,与斯通(Stone)和巴拿赫(Banach)的另外两本书一起,成为同年关于希尔伯特空间理论的首部专著[133]。

之前他人的工作表明,弱拓扑的理论不能通过序列来获得。冯·诺依曼是第一个概述如何克服这些困难的程序的人,这使得他首次定义了局部凸空间和拓扑向量空间。此外,他在当时定义的若干其他拓扑性质(他是最早将霍斯多夫(Hausdorff)从欧几里得空间到希尔伯特空间的拓扑新概念应用于数学的学者之一)[134],例如有界性和完全有界性,至今仍在使用[135]。二十年来,冯·诺依曼被认为是该领域的“无可争议的大师”[117]。

这些发展主要是受量子力学的需求推动的,在这一领域,冯·诺依曼意识到需要将厄米算子的谱理论从有界情形扩展到无界情形[136]。这些论文中的其他重大成就包括:对正规算子的谱理论的完整阐述,正算子迹的首次抽象呈现[137][138],对当时希尔伯特谱定理的里斯(Riesz)呈现的推广,以及在希尔伯特空间中发现厄米算子与自伴算子的不同,这使他能够描述所有扩展给定厄米算子的厄米算子。他还撰写了一篇论文,详细说明了当时谱理论中常用的无限矩阵作为厄米算子表示的不足之处。他在算子理论上的工作最终导致了他在纯数学中的最深刻发明——冯·诺依曼代数的研究,以及一般的算子代数研究[139]。

他在算子环方面的后续工作使他重新审视了自己的谱理论工作,并通过使用希尔伯特空间的直接积分提供了一种新的方法来处理几何内容[136]。与他在测度理论方面的工作一样,他证明了几条定理,但由于没有时间发表,他没有将这些成果公开。他告诉Nachman Aronszajn和K. T. Smith,在20世纪30年代初,他在解决不变子空间问题时证明了在希尔伯特空间中完全连续算子存在适当的不变子空间[140]。

与I. J. Schoenberg一起,他写了几篇文章,研究实数轴上的平移不变希尔伯特度量,这导致了它们的完整分类。他们的动机源自于与将度量空间嵌入希尔伯特空间相关的各种问题[141][142]。

与Pascual Jordan合作,他写了一篇简短的论文,首次通过平行四边形恒等式推导出内积所给定的范数[143]。他的迹不等式是矩阵理论中的一个关键结果,用于矩阵逼近问题[144]。他还首次提出了预范数的对偶是范数这一概念,在讨论单位不变范数和对称度量函数(现在称为对称绝对范数)理论的第一篇重要论文中[145][146][147]。这篇论文自然地引出了对称算子理想的研究,并为现代对称算子空间的研究奠定了基础[148]。

后来与Robert Schatten一起,他启动了对希尔伯特空间上的核算子的研究[149][150],研究了巴拿赫空间的张量积[151],引入并研究了迹类算子[152],它们的理想及其与紧算子的对偶关系,以及与有界算子的预对偶关系[153]。将这一主题推广到对巴拿赫空间上核算子的研究,是亚历山大·格罗滕迪克的早期成就之一[154][155]。早在1937年,冯·诺依曼就已在这一领域发表了几项成果,例如给出了不同交叉范数的1参数尺度,定义了 \({\textit {l}}_{2}^{n} \otimes {\textit {l}}_{2}^{n}\)  上的交叉范数,并证明了关于如今被称为Schatten–von Neumann理想的若干其他结果[156]。
\subsubsection{算子代数}
冯·诺依曼创立了算子环的研究,通过冯·诺依曼代数(最初称为W*-代数)。虽然他最初关于算子环的构想早在1930年就已存在,但直到他几年后遇到F·J·穆雷后才开始深入研究这一领域。冯·诺依曼代数是一个在希尔伯特空间上的有界算子的*代数,该代数在弱算子拓扑下是闭合的,并包含单位算子。冯·诺依曼的双共轭定理表明,分析定义等同于纯代数定义,即它等于双共轭。  

在阐明了交换代数的研究后,冯·诺依曼于1936年开始,在穆雷的部分协作下,着手研究非交换代数的情况,即冯·诺依曼代数的因子分类的总体研究。他在1936到1940年间发展这一理论的六篇重要论文,被誉为20世纪分析学的杰作之一;这些论文汇集了许多基础性结果,并开启了算子代数理论中的几个重要研究方向,数学家们在随后的几十年中继续深入研究。例如因子分类。此外,冯·诺依曼在1938年证明了每个在可分希尔伯特空间上的冯·诺依曼代数都是因子的直接积分;他直到1949年才找时间发表这一结果。冯·诺依曼代数与非交换积分理论紧密相关,这是冯·诺依曼在他的工作中有所暗示,但没有明确写出的一部分内容。1932年,冯·诺依曼还发布了关于极分解的另一个重要结果。
\subsubsection{格理论} 
在1935到1937年间,冯·诺依曼从事了格理论的研究,格理论是部分有序集合的理论,其中每一对元素都有一个最大下界和最小上界。正如加勒特·伯科夫所写:“约翰·冯·诺依曼的卓越才智如同一颗流星,照亮了格理论的领域。”冯·诺依曼将传统的射影几何与现代代数(线性代数、环论、格论)相结合。许多以前几何学中的结果现在可以在一般环上的模的情形下得到解释。他的工作为现代射影几何的一些研究奠定了基础。

他最大的贡献是创立了\textbf{连续几何}这一领域。这一领域延续了他在算子环方面的开创性工作。在数学中,连续几何是复杂射影几何的替代,其中子空间的维度不再是离散集 \{0, 1, ..., n\} 中的元素,而可以是单位区间 \([0, 1]\) 中的一个元素。早期,门格尔和伯科夫曾经基于线性子空间的格的性质对复杂射影几何进行了公理化。冯·诺依曼在研究算子环的基础上,放宽了这些公理,描述了一类更广泛的格,即\textbf{连续几何}。

在射影几何中,子空间的维度是一个离散集(非负整数),而连续几何中元素的维度则可以在单位区间 \([0, 1]\) 内连续变化。冯·诺依曼的动力来自于他发现冯·诺依曼代数的维度函数可以取连续的维度范围,第一个除了射影空间以外的连续几何的例子就是超有限类型 II 因子的投影。

在更纯粹的格理论工作中,他解决了一个困难的问题,即用抽象的格理论语言描述类 \( \mathit{CG(F)} \)(在任意除法环 \( \mathit{F} \) 上的连续维度射影几何)的性质。[173]冯·诺依曼提供了对完成的补全模拓扑格中维度的抽象探索(这些性质出现在内积空间的子空间格中):

维度是通过以下两个性质来确定的,直到正的线性变换为止。它由透视映射(“透视性”)保持不变,并且由包含关系排序。证明的最深部分涉及透视性与“通过分解的射影性”的等价性——其推论是透视性的传递性。

对于任意整数 \( n > 3 \),每个 \( n \)-维抽象射影几何与一个 \( n \)-维向量空间 \( V_{n}(F) \) 上的子空间格同构,该向量空间在(唯一的)对应的除法环 \( F \) 上。这被称为 \textbf{维布伦–杨定理}。冯·诺依曼将这个射影几何的基本结果扩展到连续维度的情况。[174]这个坐标化定理激发了在抽象射影几何和格理论中的大量研究,许多研究继续使用冯·诺依曼的技术。[169][175] 伯科夫这样描述这个定理:

任何具有“基”的补全模格 \( L \),其中 \( n \geq 4 \) 是一组互相透视的元素,都与适当的正规环 \( R \) 的所有主右理想的格 \( \mathcal{R}(R) \) 同构。这个结论是140页精彩而深刻的代数的巅峰,这些代数涉及完全新的公理。任何想要领略冯·诺依曼思维锋利边缘的人,只需试着自己追踪这一精确的推理链——并意识到,他经常在早餐前就写下了五页内容,坐在浴袍中的客厅写字台前。[176]

这项工作需要创建正规环。[177] 一个冯·诺依曼正规环是一个环,对于每个元素 \( a \),存在一个元素 \( x \),使得 \( axa = a \)。[176] 这些环源自并与他在冯·诺依曼代数\(AW^*-\)代数和各种类型的\(C^*-\)代数方面的工作有关。[178]

在创建和证明上述定理的过程中,证明了许多较小的技术性结果,特别是关于分配律的(例如无穷分配律),冯·诺依曼根据需要进行了发展。他还发展了格中的估值理论,并参与了度量格的通用理论的研究。[179]

伯科夫在他关于冯·诺依曼的遗作文章中指出,这些大多数结果是在一个密集的两年工作期间发展出来的,并且尽管冯·诺依曼的兴趣在1937年后依然延续在格理论中,但这些兴趣变得边缘化,主要体现在他与其他数学家的信件中。1940年的最后一项贡献是他与伯科夫在高级研究所共同主持的一个讲座,讲述了关于他发展出的σ-完备格序环的理论。他从未将这些工作写成出版物。[180]
\subsubsection{数学统计学}  
冯·诺依曼对数学统计学作出了基础性贡献。1941年,他推导出了独立且同分布的正态变量的均值平方差与样本方差比值的精确分布。[181] 这一比值应用于回归模型的残差,通常被称为杜宾-沃森统计量,用于检验错误项是否为序列独立的零假设,或它们是否遵循平稳的一阶自回归模型的备择假设。[182]

此后,Denis Sargan 和 Alok Bhargava 扩展了这一结果,用于检验回归模型的错误项是否遵循高斯随机游走(即具有单位根),与它们是平稳的一阶自回归模型的备择假设进行比较。[183]
\subsubsection{其他工作}  
在早期,冯·诺依曼发表了几篇与集合论实分析和数论相关的论文。[184] 在1925年发表的一篇论文中,他证明了对于区间 \([0,1]\) 中的任何稠密点列,存在一个重新排列的方式,使得这些点是均匀分布的。[185][186][187] 1926年,他唯一的出版物是关于普吕弗理想代数数理论的论文,他找到了构造这些代数数的新方法,从而将普吕弗的理论扩展到所有代数数的领域,并阐明了它们与p-完备数的关系。[188][189][190][191][192] 1928年,他又发表了两篇延续这些主题的论文。第一篇论文涉及将一个区间划分为可数多个全等子集,解决了Hugo Steinhaus提出的一个问题,即一个区间是否是 \(\aleph_0\)-可分的。冯·诺依曼证明了所有的区间,无论是半开、开放还是闭区间,都是 \(\aleph_0\)-可分的,即这些区间可以通过平移分解为 \(\aleph_0\) 个全等的子集。[193][194][195][196] 他接下来的论文提供了没有选择公理的构造性证明,证明了存在 \(2^{\aleph_0}\) 个代数独立的实数。他证明了 \(A_r = \sum_{n=0}^{\infty} 2^{2^{[nr]}} / 2^{2^{n^2}}\) 对于 \(r > 0\) 是代数独立的。因此,存在一个大小为连续统的完美代数独立实数集。[197][198][199][200] 他早期生涯的其他小成果包括为变分法中的最小化函数梯度提供最大值原理的证明,[201][202][203][204] 以及对几何数论中赫尔曼·闵可夫斯基线性形式定理的小简化。[205][206][207]

在职业生涯后期,冯·诺依曼与Pascual Jordan 和 Eugene Wigner 一起,撰写了一篇基础性论文,分类了所有有限维的形式实数乔丹代数,并在试图寻找量子理论更好的数学形式时发现了Albert代数。[208][209] 1936年,他尝试进一步推进将之前的希尔伯特空间公理替换为乔丹代数公理的计划,在一篇研究无限维情况的论文中进行了探讨;他计划至少再写一篇相关论文,但最终未能完成。[211] 尽管如此,这些公理成为了后续由Irving Segal 开始的代数量子力学研究的基础。[212][213]
\subsection{物理学}  
\subsubsection{量子力学}  
冯·诺依曼是第一个为量子力学建立严格数学框架的人,这一框架被称为迪拉克-冯·诺依曼公理,见于他在1932年发表的具有影响力的著作《量子力学的数学基础》中。[214] 在完成集合论的公理化后,他开始着手量子力学的公理化工作。他在1926年意识到,量子系统的状态可以用一个点来表示,这个点位于(复数)希尔伯特空间中,通常即使是对于单个粒子,这个空间也是无限维的。在这个量子力学的形式化框架中,可观察量(如位置或动量)被表示为作用在与量子系统相关的希尔伯特空间上的线性算符。[215]

量子力学的物理学因此被简化为作用在希尔伯特空间上的数学问题。例如,不确定性原理指出,粒子位置的确定会妨碍动量的确定,反之亦然,这可以转化为对应的两个算符的不交换性。这一新的数学表述包含了海森堡和薛定谔的特殊情形。[215]

冯·诺依曼的抽象处理使他能够面对决定论与非决定论的基础性问题,在这本书中,他提供了一个证明,证明量子力学的统计结果不可能是一个潜在确定的“隐藏变量”集的平均值,正如经典统计力学中所做的那样。1935年,Grete Hermann 发表了一篇论文,认为这个证明包含了概念性错误,因此是无效的。[216] Hermann 的工作在1966年约翰·贝尔提出基本相同的论点之前,基本上被忽视了。[217] 2010年,Jeffrey Bub 认为贝尔误解了冯·诺依曼的证明,并指出该证明虽然对于所有隐藏变量理论无效,但确实排除了一个定义明确且重要的子集。Bub 还建议冯·诺依曼意识到这一限制,并没有声称他的证明完全排除了隐藏变量理论。[218] Bub 论点的有效性也受到争议。1957年,Gleason 定理提供了一个与冯·诺依曼相类似的反隐藏变量论证,但其基础假设被认为有更好的动机并且更具物理意义。[219][220]

冯·诺依曼的证明开启了一条研究路线,最终通过贝尔定理和阿兰·阿斯佩的1982年实验,证明了量子物理学要么需要一种与经典物理学完全不同的现实观念,要么必须包含非局部性,似乎违反了相对论的特殊相对性原理。[221]

在《量子力学的数学基础》一书的一个章节中,冯·诺依曼深入分析了所谓的测量问题。他得出结论,整个物理宇宙可以服从普遍的波函数。由于需要某种“外部的计算”来坍缩波函数,冯·诺依曼认为波函数的坍缩是由实验者的意识引起的。他认为,量子力学的数学允许将波函数的坍缩置于从测量装置到人类观察者的“主观意识”之间的因果链中的任何位置。换句话说,虽然观察者与被观察物体之间的界限可以在不同的位置画出,但只有在某个地方存在观察者时,这一理论才有意义。[222] 尽管尤金·维格纳接受了意识导致坍缩的观点,[223] 但冯·诺依曼-维格纳解释法从未获得大多数物理学家的接受。[224]

尽管量子力学的理论仍在不断发展,但大多数量子力学方法背后的数学形式主义的基本框架可以追溯到冯·诺依曼首次使用的数学形式和技术。如今,关于理论的解释以及对其的扩展,大多是在关于数学基础的共同假设的基础上进行讨论的。[214]

从冯·诺依曼在量子力学方面的工作被视为实现希尔伯特第六问题的一部分,数学物理学家阿瑟·怀特曼在1974年表示,他对量子理论的公理化可能是迄今为止最重要的物理理论公理化。通过他在1932年出版的这本书,量子力学成为了一种成熟的理论,因为它拥有了精确的数学形式,这使得对概念问题能够提供明确的答案。[225] 然而,冯·诺依曼在晚年觉得他在这一科学工作方面失败了,因为尽管他发展了所有这些数学,他并没有找到一个令人满意的量子理论整体的数学框架。[226][227]

\textbf{冯·诺依曼熵}  

冯·诺依曼熵在量子信息理论的框架中被广泛应用于不同的形式(如条件熵、相对熵等)。[228] 纠缠度量基于与冯·诺依曼熵直接相关的某些量。给定一个具有密度矩阵 \(\rho\) 的量子力学系统的统计集合,它由以下公式给出:\(
S(\rho) = -\operatorname{Tr}(\rho \ln \rho).\)许多经典信息理论中的熵度量也可以推广到量子情形,例如霍尔沃熵[229] 和条件量子熵。量子信息理论在很大程度上关注冯·诺依曼熵的解释和应用,它是该领域发展的基石;而香农熵则应用于经典信息理论。[230]

\textbf{密度矩阵}  
  
密度算子和矩阵的形式主义由冯·诺依曼在1927年提出[231],而列夫·兰道(Lev Landau)[232]和费利克斯·布洛赫(Felix Bloch)[233]分别在1927年和1946年独立提出,尽管他们的提出方式较为不系统。密度矩阵允许表示量子态的概率混合(混合态),与波函数不同,波函数只能表示纯态。[234]

\textbf{冯·诺依曼测量方案}  

冯·诺依曼测量方案,量子退相干理论的前身,通过考虑测量装置作为量子物体来将测量表示为投影式。冯·诺依曼提出的“投影测量”方案促成了量子退相干理论的发展。[235][236]

\textbf{量子逻辑}  

冯·诺依曼在他的1932年著作《量子力学的数学基础》中首次提出了量子逻辑,他指出在希尔伯特空间上的投影可以被视为关于物理可观察量的命题。随后,冯·诺依曼与加勒特·伯克霍夫在1936年发表的论文中开创了量子逻辑领域,这是第一次引入量子逻辑[237]。在这篇论文中,冯·诺依曼和伯克霍夫首次证明了量子力学需要一种与所有经典逻辑显著不同的命题演算,并严格地为量子逻辑隔离出一种新的代数结构。量子逻辑的命题演算的概念最初在冯·诺依曼1932年的作品中简要提到,但在1936年通过几项证明展示了对新命题演算的需求。例如,光子不能通过两个方向垂直的连续滤光片(如水平和垂直滤光片),因此,如果在它们的前后再添加一个对角线极化的滤光片,它也无法通过,但如果第三个滤光片放在两个滤光片之间,光子则会通过。这一实验事实可以翻译成逻辑中的结合性不交换性:\((A \land B) \neq (B \land A)\)还证明了经典逻辑中的分配律( \(P \lor (Q \land R) = (P \lor Q) \land (P \lor R)\)和\(P \land (Q \lor R) = (P \land Q) \lor (P \land R)\))在量子理论中不成立。[238]

这归因于量子析取与经典析取的不同,量子析取即使在两个析取项都为假时也可以为真,这反过来可以归因于在量子力学中,往往一对备选方案在语义上是确定的,而其每个成员则必然是不确定的。因此,经典逻辑的分配律必须被替换为一个更弱的条件。[238] 量子系统的命题不是构成分配格,而是形成与该系统相关的希尔伯特空间的子空间格同构的正交模格。[239]

尽管如此,他从未对自己在量子逻辑方面的工作感到满意。他原本希望将其作为形式逻辑与概率论的联合综合,但当他试图撰写他在1945年于华盛顿哲学学会上发表的亨利·约瑟夫讲座论文时,他发现自己无法完成,尤其是在他当时忙于战争工作时。在1954年国际数学家大会上的演讲中,他将这一问题列为未来数学家可以继续研究的未解问题之一。[240][241]
\subsubsection{流体动力学}  
冯·诺伊曼在流体动力学领域做出了基础性贡献,包括爆炸波的经典流动解[242],以及与雅科夫·博里索维奇·泽尔多维奇(Yakov Borisovich Zel'dovich)和沃纳·多林(Werner Döring)独立共同发现的ZND爆炸物引爆模型[243]。在1930年代,冯·诺伊曼成为了成形装药数学的权威[244]。

后来,冯·诺伊曼与罗伯特·D·里希迈尔(Robert D. Richtmyer)共同开发了人工粘性算法,这一算法改善了对冲击波的理解。当计算机解决流体动力学或空气动力学问题时,它们将过多的计算网格点放置在尖锐不连续性(冲击波)区域。人工粘性的数学方法平滑了冲击波过渡,而没有牺牲基本物理原理[245]。

冯·诺伊曼很快将计算机建模应用于这一领域,为他的弹道学研究开发了软件。在第二次世界大战期间,他向美国陆军弹道研究实验室主任R.H.肯特(R. H. Kent)提出了一种用于计算一维100分子模型的计算机程序,以模拟冲击波。冯·诺伊曼为包括他的朋友西奥多·冯·卡门(Theodore von Kármán)在内的听众做了关于这个程序的研讨会。冯·诺伊曼讲完后,冯·卡门说:“当然,你知道拉格朗日也用数字模型来模拟连续介质力学。”冯·诺伊曼当时并不知道拉格朗日的《解析力学》[246]。
\subsubsection{其他工作}
\begin{figure}[ht]
\centering
\includegraphics[width=6cm]{./figures/5654142e178647a1.png}
\caption{冯·诺伊曼的纪念 plaque,位于他出生地布达佩斯第五区巴托里街26号的墙上。} \label{fig_Neuman_6}
\end{figure}
尽管冯·诺伊曼在物理学方面的贡献不如在数学领域那么丰富,但他仍然做出了几项值得注意的贡献。他与苏布拉马尼扬·钱德拉塞卡(Subrahmanyan Chandrasekhar)共同撰写的关于由随机分布的恒星产生的波动引力场统计的开创性论文,被认为是一项杰出的工作[247]。在这篇论文中,他们发展了双体弛豫理论[248],并使用霍尔茨马克分布来模拟[249]恒星系统的动力学[250]。他还写了几篇未出版的手稿,涉及恒星结构的课题,其中一些内容被包含在钱德拉塞卡的其他著作中[251][252]。

在由奥斯瓦尔德·费布伦(Oswald Veblen)领导的早期工作中,冯·诺伊曼帮助发展了与旋量(spinors)相关的基本思想,这些思想最终促成了罗杰·彭罗斯的拓扑理论(twistor theory)[253][254]。许多工作是在1930年代在高级研究院(IAS)举行的研讨会中完成的[255]。在这些工作中,他与A·H·陶布(A. H. Taub)和费布伦共同撰写了一篇论文,将狄拉克方程扩展到射影相对论中,重点讨论了保持坐标、旋转和规范变换下的不变性,作为1930年代量子引力理论早期研究的一部分[256]。

在同一时期,他向同事们提出了几项处理新创建的量子场论问题以及量子化时空的提案;然而,他和他的同事们都认为这些想法并不具有实际意义,因此没有继续深入研究[257][258][259]。尽管如此,他仍然保持着一定的兴趣,并在1940年写了一篇关于狄拉克方程在德西特空间中的应用的手稿[260]。
\subsection{经济学}  
\subsubsection{博弈论}  
冯·诺伊曼创立了博弈论这一数学学科。[261] 他在1928年证明了最小最大定理。该定理表明,在完美信息的零和博弈中(即玩家在每个时刻都知道到目前为止发生的所有动作),存在一对策略,使得每个玩家都能最小化他们的最大损失。[262] 这样的策略被称为最优策略。冯·诺伊曼证明了他们的最小最大值在绝对值上相等(符号相反)。他进一步改进并扩展了最小最大定理,将其应用于不完全信息的博弈以及多于两名玩家的博弈,并将这一结果在1944年与奥斯卡·莫根斯坦合著的《博弈论与经济行为》一书中发表。此书的公众关注度很高,《纽约时报》还在头版刊登了报道。[263] 在这本书中,冯·诺伊曼声明经济理论需要使用泛函分析,尤其是凸集和拓扑不动点定理,而非传统的微分学,因为最大算子无法保持可微函数。[261]

冯·诺伊曼的泛函分析技术——使用实向量空间的对偶配对表示价格和数量、使用支持平面和分离超平面、以及凸集和不动点理论——一直是数学经济学的主要工具。[264]
\subsubsection{数学经济学}  
冯·诺伊曼通过几篇有影响力的出版物提高了经济学的数学水平。在他关于扩展经济模型的研究中,他通过推广布劳尔不动点定理,证明了均衡的存在性和唯一性。[261] 冯·诺伊曼的扩展经济模型考虑了非负矩阵 A 和 B 的矩阵铅笔形式 \(A - \lambda B\);他寻求概率向量 \(p\) 和 \(q\) 以及一个正数 \(\lambda\),使得它们能够解满足互补条件的方程:\(p^{T}(A - \lambda B)q = 0\)同时还要求满足两个表达经济效率的不等式系统。在该模型中,(转置的)概率向量 \(p\) 代表商品的价格,而概率向量 \(q\) 代表生产过程的“强度”。唯一解 \(\lambda\) 代表增长因子,其值为1加上经济增长率;而增长率等于利率。[265][266]

冯·诺伊曼的结果被视为线性规划的特例,其中他的模型仅使用非负矩阵。他的扩展经济模型的研究至今仍吸引着数学经济学家的兴趣。[267][268] 这篇论文被几位作者称为数学经济学中最伟大的论文,他们认为它引入了不动点定理、线性不等式、互补松弛条件和鞍点对偶性。[269] 在关于冯·诺伊曼增长模型的会议记录中,保罗·萨缪尔森表示,许多数学家开发了对经济学家有用的方法,但冯·诺伊曼在对经济理论本身做出显著贡献方面是独一无二的。[270] 该研究在一般均衡理论和不动点定理方法学方面的持久重要性通过1972年颁发给肯尼斯·阿罗、1983年颁发给热拉尔·德布雷和1994年颁发给约翰·纳什的诺贝尔奖得到了突出展示,他们使用不动点定理为非合作博弈和谈判问题建立均衡解,而纳什在其博士论文中正是如此。阿罗和德布雷也使用了线性规划,诺贝尔奖得主塔林·库普曼斯、列昂尼德·坎托罗维奇、瓦西里·列昂季耶夫、保罗·萨缪尔森、罗伯特·多夫曼、罗伯特·索洛和列昂尼德·赫尔维茨也使用了线性规划。[271]

冯·诺伊曼对这一课题的兴趣始于1928年和1929年他在柏林的讲座。他和经济学家尼古拉斯·卡尔多尔一起在布达佩斯度过夏天;卡尔多尔建议冯·诺伊曼阅读数学经济学家莱昂·瓦尔拉斯的书。冯·诺伊曼注意到,瓦尔拉斯的《一般均衡理论》和瓦尔拉斯定律导致了同时线性方程组的产生,这可能得出荒谬的结果——通过生产和销售负数量的产品可以最大化利润。他将方程替换为不等式,介绍了动态均衡等内容,最终形成了他的论文。[272]
\subsubsection{线性规划}  
基于他在矩阵博弈和扩展经济模型方面的研究成果,冯·诺伊曼在乔治·丹齐格(George Dantzig)简短描述他工作的几分钟后,发明了线性规划中的对偶理论,冯·诺伊曼不耐烦地要求丹齐格直接切入重点。接着,丹齐格目瞪口呆地听着冯·诺伊曼进行了一个小时的讲座,内容涉及凸集、不动点理论和对偶性,冯·诺伊曼提出了矩阵博弈与线性规划等价的猜想。[273]

后来,冯·诺伊曼提出了一种新的线性规划方法,使用保罗·戈尔丹(Paul Gordan,1873年)的齐次线性系统,这一方法后来通过卡尔马克尔(Karmarkar)的算法得到了普及。冯·诺伊曼的方法使用了在单纯形之间的旋转算法,旋转的决策由一个具有凸性约束的非负最小二乘子问题决定(将零向量投影到活跃单纯形的凸包上)。冯·诺伊曼的算法是线性规划中的第一个内点方法。[274]
\subsection{计算机科学}  
冯·诺伊曼是计算机科学的奠基人物之一,[275] 对计算硬件设计、理论计算机科学、科学计算以及计算机科学哲学作出了重要贡献。
\subsubsection{硬件}
\begin{figure}[ht]
\centering
\includegraphics[width=6cm]{./figures/5cd4d044454ba452.png}
\caption{AVIDAC计算机部分基于冯·诺伊曼开发的IAS机器的架构。} \label{fig_Neuman_7}
\end{figure}
冯·诺伊曼为陆军弹道研究实验室提供咨询,最著名的贡献是在ENIAC项目中,[276] 他是该实验室科学顾问委员会的成员。[277] 尽管单一内存、存储程序架构通常被称为冯·诺伊曼架构,但该架构实际上是基于J·普雷斯珀·埃克特(J. Presper Eckert)和约翰·莫赫利(John Mauchly)的工作,他们是ENIAC及其继任者EDVAC的发明人。在为宾夕法尼亚大学的EDVAC项目提供咨询时,冯·诺伊曼撰写了《EDVAC报告的初稿》。这篇论文的过早分发使埃克特和莫赫利的专利申请无效,文中描述了一种将数据和程序存储在同一地址空间中的计算机,这与早期的计算机不同,后者将程序分别存储在纸带或插头板上。这一架构成为了大多数现代计算机设计的基础。[278]

接下来,冯·诺伊曼在新泽西州普林斯顿的高等研究院设计了IAS机器。他安排了其资金,并且组件在附近的RCA研究实验室设计和制造。冯·诺伊曼建议IBM 701(绰号“防御计算机”)采用磁鼓。它是IAS机器的一个更快版本,并成为商业上成功的IBM 704的基础。[279][280]
\subsubsection{算法}
\begin{figure}[ht]
\centering
\includegraphics[width=6cm]{./figures/b5edaf182c3dba06.png}
\caption{冯·诺伊曼在1947年发表的《电子计算仪器问题的规划与编码》中的流程图} \label{fig_Neuman_8}
\end{figure}
1945年,冯·诺伊曼发明了归并排序算法,该算法通过递归地对数组的前半部分和后半部分进行排序,然后将它们合并。[281][282]

作为冯·诺伊曼氢弹研究的一部分,他与斯坦尼斯瓦夫·乌拉姆(Stanisław Ulam)共同开发了用于流体动力学计算的模拟方法。他还为蒙特卡罗方法的发展做出了贡献,该方法使用随机数来近似解决复杂问题。[283]

冯·诺伊曼提出的利用有偏硬币模拟公平硬币的算法被用于某些硬件随机数生成器的“软件白化”阶段。[284] 由于获得“真正”随机数在实践中不可行,冯·诺伊曼开发了一种伪随机性的方法,使用了中心平方法。他将这一粗略方法的快速性作为其优点,写道:“任何考虑使用算术方法生成随机数字的人当然都处于一种‘罪恶状态’。”[284] 他还指出,当这种方法出错时,问题显而易见,而其他方法可能会产生微妙的错误。[284]

1953年,冯·诺伊曼引入了随机计算的概念,[285] 但直到20世纪60年代计算技术的进步才得以实现。[286][287] 大约在1950年,他也是最早讨论计算时间复杂度的人之一,这一研究最终发展成为计算复杂性理论领域。[288]


\begin{figure}[ht]
\centering
\includegraphics[width=8cm]{./figures/909670ecd41ab09e.png}
\caption{冯·诺依曼自我复制通用构造器的首次实现。[289] 显示了三代机器:第二代几乎完成了第三代的构建。向右延伸的线是遗传指令带,这些指令与机器的主体一起被复制。} \label{fig_Neuman_9}
\end{figure}
冯·诺依曼对自我复制结构的数学分析早于DNA结构的发现。[290] 乌拉姆和冯·诺依曼通常被认为是细胞自动机领域的创始人,该领域始于20世纪40年代,作为生物系统的简化数学模型。[291]

在1948年和1949年的讲座中,冯·诺依曼提出了一个运动学自我复制自动机。[292][293] 到1952年,他对这个问题进行了更抽象的处理。他设计了一个复杂的二维细胞自动机,可以自动复制其初始配置的单元。[294] 基于冯·诺依曼细胞自动机的冯·诺依曼通用构造器在他去世后通过《自我复制自动机的理论》得以完善。[295] 冯·诺依曼邻域中,每个二维网格中的单元有四个正交相邻的网格单元作为邻居,这一概念至今仍被用于其他细胞自动机。[296]