% 求定积分的一些方法

在实操中,定积分的计算一般交给计算机进行(运用符号积分或者数值积分);不过,既然考试还是喜欢考定积分的计算,你就得稍微了解一些定积分的方法.

\subsection{微积分基本定理}
原则上说,因为你总能通过微积分基本定理(牛顿—莱布尼兹公式)\upref{NLeib}联系不定积分\upref{Int}(原函数)与定积分,因此你可以先找到相应的原函数再赋值计算.这使得不定积分的积分技巧\upref{intech}同样适用于定积分,此处不再复述.

不过值得注意的是,在运用分部积分法或换元法后,你必须相应的改变积分上下限.例如,$\int^b_a uv'dx = uv|^b_a-\int^b_a vu'dx$, $\int^b_a f'(g(x))g'(x)dx = \int ^{u(b)}_{u(a)} f'(u)du$

\subsection{对称性}
但另一方面,由于定积分有确定的积分上下限,比起不定积分,定积分有着更多样的(也更方便的)求解技巧.因此,做定积分时不要一心总想着求原函数,这往往过于复杂.例如,可以运用被积函数的对称性(奇偶性)
$$
\int ^a_{-a} f(x) = 
\left \{
\begin{aligned}
0,\text{f是奇函数,f(-x)=-f(x)}\\
2\int ^a_0 f(x),\text{f是偶函数,f(-x)=f(x)}\\
\end{aligned}
\right.
$$
在实操中,运用对称性时,往往需要拆分积分区间或积分函数,从而更好地运用对称性.

\subsection{周期性}
如果T是f的一个周期,那么 $\int ^{a+nT}_{a} f(x)dx= n\int^{a}_0 f(x)dx$