% 时空的几何
% keys 类时|本征时|时空|光锥|类空|类光
% license Usr
% type Tutor

学习狭义相对论的“现代”方式时强调时空的几何,这种方法将引领我们自然的到达广义相对论和Einstein的引力。本部分以一种更为严格的形式展现狭义相对论的几何。

\subsection{基本定义}
相对论认为,任一事件都由它发生的时间和地点确定,因此描述事件的所在“空间”(数学概念,可视为拓扑空间的空间)称为时空。
\begin{definition}{时空,事件,距离}
若四维矢量空间 $\mathbb R^4$ 上定义了如下的距离函数
\begin{equation}
\Delta s^2_{AB}:=\rho^2(A,B):=\eta_{\alpha\beta}\Delta x^\alpha\Delta x^\beta,\quad \forall A,B\in\mathbb R^4,~
\end{equation}
其中,$\Delta x=x_B-x_A$,$\eta=\mathrm{diag}[-1,1,1,1]$ 是对角化矩阵。则称该矢量空间为4维\textbf{时空},其上的点称为\textbf{事件}(event),$\Delta s_{AB}=\sqrt{\Delta s^2_{AB}}$ 称为 $A,B$ 的\textbf{间隔}(separation),$\eta$ 称为\textbf{度规}(matric)。规定坐标从0标记,即 $\alpha=0,1,2,3$,并称坐标 $x^0$ 称为\textbf{时间(time)坐标},可记为 $t$, $(x^1,x^2,x^3)$ 称为\textbf{空间(space)坐标},可记作 $\bvec r$。
\end{definition}
\textbf{注:}事实上,更严格的相对论时空几何的定义需要借助仿射空间定义在伪欧几里得空间上(指数有限度量空间\upref{EFSp})。即一个点空间配上一个带有不定二次型的矢量空间,

注意到度规 $\eta$ 是一个不定型(\autoref{def_DeQua_2}),因此 $s^2(A,B)$ 有三种可能,即 $\Delta s^2_{AB}>0,\Delta s^2_{AB}=0,\Delta s^2_{AB}<0$。

\begin{definition}{类时,类空,类光}
设 $A,B$ 是时空中的两个事件,则称间隔 $\Delta s_{AB})$ 为\textbf{类时间隔}(timelike),若 $\Delta s^2_{AB}<0$;为\textbf{类空间隔}(sapcelike),若 $\Delta s^2_{AB}>0$;为\textbf{类光间隔}(lightlike)或\textbf{零值}(null),若 $\Delta s^2_{AB}=0$。
\end{definition}
“类光” 一词来源于相对论中光的轨迹为时空中的直线,即 $\Delta t^2=\Delta \bvec r^2$。显然,
\begin{equation}
\begin{aligned}
\Delta t^2>\Delta \bvec r^2,\quad \text{类时},\\
\Delta t^2=\Delta \bvec r^2,\quad \text{类光},\\
\Delta t^2<\Delta \bvec r^2,\quad \text{类空}.
\end{aligned}~
\end{equation}

在相对论中,对于(有质量)粒子,其在两邻近点的间隔满足 $\dd s^2<0$。
\begin{definition}{本征时间}
称由
\begin{equation}
\Delta \tau^2:=-\Delta s^2~
\end{equation}
定义的 $\tau$ 为\textbf{本征时间}(proper time)。
\end{definition}

\subection{洛伦兹变换}
在数学上,运动是使得两点间距离保持不变的线性变换,而在伪欧几里得空间,为了在实数情况下讨论,运动被定义为使得距离平方不变的变换\upref{EFSp}。对应在相对论中,使得间隔平方不变的线性变换则称为Lorentz变换。
\begin{definition}{Lorentz变换}
设 $f:\mathbb R^4\rightarrow\mathbb R^4$ 是线性变换,若满足 $\forall A,B\in\mathbb R^4$,都有
\begin{equation}
\Delta s^2_{f(A)f(B)}=\Delta s^2_{AB},~
\end{equation}
则称 $f$ 为\textbf{Lorentz变换}(Lorentx transformation)。
\end{definition}




















