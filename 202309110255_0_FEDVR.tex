% FEDVR 网格
% keys 薛定谔方程|数值解|偏微分方程|三维
% license Xiao
% type Tutor

\pentry{高斯积分(Gauss-Lobatto)\upref{GLquad}}
\addTODO{简单介绍什么是 FEDVR,有什么用}

\subsection{FEDVR基底}

\footnote{参考 Aihua Liu 的\href{https://jrm.phys.ksu.edu/Research/Theses/Liu/AihuaLiu2015.zip}{博士论文} (KSU, 2015)。}现在以一维 FEDVR 为例,把整个区间划分成 $N_e$ 个\textbf{有限元},第 $i$ 个有限元的区间为 $[x_i,x_{i+1}]$。每个有限元内进一步加入格点,令 $x_{ij}\ (j = 1\dots N)$ 为第 $i$ 有限元的 $N$ 阶 Gauss-Lobatto 数值积分 %链接未完成
的 $N$ 个采样点。这样,$x_{j1}=x_j$,  $x_{jN}=x_{j+1}$。 

现在来定义 FEDVR 基底 $u_{ij}(x)$。 令 $a_i$ 是第 $i$ 个 FE 长度的一半, $b_i$ 是第 $i$ 个 FE 的中点坐标, 把\autoref{eq_GLquad_3}~\upref{GLquad} 中的 $f_n(x)$ ($x$ 超出 $[-1,1]$ 时令 $f_n(x) = 0$) 记为 $f_n(t)$, 并把 $t \in [-1,1]$ 线性依次映射到每个有限元的区间 $x \in [x_i, x_{i+1}]$ 上, 有(下文中 $t$ 和 $x$ 都分别使用该定义)
\begin{equation}
x_{ij} = a_i t_j + b_i~,
\end{equation}
\begin{equation}
x = a_i t + b_i \qquad (x_{i1} \leqslant x \leqslant x_{iN})~.
\end{equation}
则基底的定义为
\begin{equation}
u_{ij}(x) = \leftgroup{
&\frac{1}{\sqrt{a_i}} f_j\qty(\frac{x-b_i}{a_i}) &&( 1 < j < N) \\
& \frac{1}{\sqrt{a_i+a_{i+1}}} \qty[ f_N\qty(\frac{x-b_i}{a_i}) + f_1\qty(\frac{x-b_{i+1}}{a_{i+1}})]
&&(j = N) 
} ~.\end{equation}
每个基底都是由一个 $N-1$ 阶多项式(或两个拼接而成)。 易证正交归一关系
\begin{equation}
\int_{x_1}^{x_{N_e+1}} u_{ij}(x) u_{i'j'}(x) \dd{x} = \delta_{ii'} \delta_{jj'}~.
\end{equation}
在总区间的两端我们一般采用函数值为零的边界条件, 所以不定义 $u_{11}(x)$ 和 $u_{N_e, N}(x)$。 这样,我们最终共有 $N_e(N-1)-1$ 个节点 $x_{ij}$ 和对应的基底 $u_{ij}(x)$。 令 $w_{0i}$ 为\autoref{eq_GLquad_3}~\upref{GLquad} 中的 $w_i$, 则每个基底在每个节点处的函数值为
\begin{equation}
u_{ij}(x_{i'j'}) = \delta_{ii'}\delta_{jj'}\times \leftgroup{
&\frac{1}{\sqrt{a_i w_{0j}}} &&( 1 < j < N) \\
& \frac{1}{\sqrt{(a_i+a_{i+1}) w_{0N}}} &&(j = N) 
}~.\end{equation}

易证, 若要对整个区间积分, 所需权重为
\begin{equation}
w_{ij} = \frac{1}{u_{ij}^2(x_{ij})} =
\begin{cases}
a_i w_{0j} &(1 < j < N) \\
(a_i + a_{i+1}) w_{0N} &(j = N)
\end{cases}~.
\end{equation}
但如果只需要在某个 FE 区间内做积分, 只需用 $w_{ij} = a_i w_{0j}$ 即可。

一个函数 $f(x)$ 用基底展开为
\begin{equation}
\braket{u_{ij}(x)}{f(x)} = \int_{x_1}^{x_{N_e+1}} u_{ij}(x) f(x) \dd{x} \approx \sqrt{w_{ij}} f(x_{ij})~.
\end{equation}
把所有基底按对应的 $x_{ij}$ 从小到大排序。 $f(x)$ 的展开系数就可以记为一个列矢量。

\subsection{二阶导数矩阵(动能矩阵)}

虽然不是所有的 $u_i$ 基底都处处二阶可导, 但我们可以假设它们所表示的波函数 $\psi(x) = \sum_k c_k u_k(x)$ 处处二阶可导。 求导后,系数为
\begin{equation}
c_i^{(2)} = \int_{-\infty}^{+\infty} {u_i}\dv[2]{x}\sum_k c_k u_k \dd{x}~.
\end{equation}
所以为了避免处理有限元的边界,我们可以在积分中直接忽略所有的边界点(忽略有限个点不会影响积分值)。
\begin{equation}
c_i^{(2)} = \sum_k c_k \int u_i u''_k \dd{x}~.
\end{equation}
此时求和中的每个积分范围可以缩小到一个有限元 $[x_\text{FE,min},x_\text{FE,max}]$ 内,并看作二阶导数算符的矩阵元
\begin{equation}\ali{
D^{(2)}_{ij} &= \mel*{u_i}{\dv[2]{x}}{u_j} = \int u_i u''_k \dd{x}\\
&= \eval{u_i(x) u'_j(x)}_{x_\text{FE,min}}^{x_\text{FE,max}} - \int u'_i u'_j \dd{x}\\
&= -\braket*{u'_i(x)}{u_j'(x)}~.
}\end{equation}
可以看出 $\mat D^{(2)}$ 是实对称矩阵。 由于被积多项式的阶数只有 $2(N-2)$(小于 $2N-3$), 这个积分可以精确地用求和表示, 而且只有同一个 FE 里面的 $u_i(x), u_j(x)$ 才能使积分不为零(bridge function 属于两个 FE), 所以就得到了几乎是 block diagonalized 的矩阵, 只是每一个 block 左上角和右下角的一个矩阵元与其他 block 重叠。

那如何计算基底函数的导数呢? 可以通过计算拉格朗日插值多项式\upref{LagraP}的导数得到
\begin{equation}\label{eq_FEDVR_9}
\ali{
&f_i'(t_j) =\frac{1}{\sqrt{w_{0i}}} \times\\
&\quad\leftgroup{
&\sum_{k\ne i}\frac{t_j-t_1}{t_i-t_1} \frac{t_j-t_2}{t_i-t_2} \dots \frac{1}{t_i-t_k} \dots \frac{t_j-t_{i-1}}{t_i-t_{i-1}}\frac{t_j-t_{i+1}}{t_i-t_{i+1}} \dots \frac{t_j-t_N}{t_i-t_N} &&(i \ne j) \\
&\sum_{k\ne i} \frac{1}{t_i-t_k} &&(i = j)~.
}}\end{equation}
代入得
\begin{equation}
u'_{ij}(x_{ij'}) = \frac{1}{a_i^{3/2}} f'_j\qty(t_{j'})  \qquad ( 1 < j < N)~,
\end{equation}
\begin{equation}
u'_{iN}(x_{i j'}) = \frac{1}{a_i\sqrt{a_i+a_{i+1}}} f'_N\qty(t_{j'})~,
\end{equation}
\begin{equation}
u'_{iN}(x_{i+1, j'}) = \frac{1}{a_{i+1}\sqrt{a_i+a_{i+1}}} f'_1\qty(t_{j'})~.
\end{equation}
注意 $u'_{iN}$ 在 $x_{iN}$ 处不连续无定义,左右导数分别为 $u'_{iN}(x_{iN})$ 和 $u'_{iN}(x_{i+1, 1})$。

现在就可以求动能矩阵的矩阵元了(由于是对称矩阵, 我们只列出半边), 先看不含 bridge function 的矩阵元
\begin{equation} \ali{
K_{(im), (in)} &=  \frac12 \int_{x_{i1}}^{x_{iN}} u'_{im}(x) u'_{in}(x) \dd{x} \\
&= \frac12 \sum_k  a_i w_{0k} u'_{im}(x_{ik}) u'_{in}(x_{ik}) \\
&= \frac{1}{2a_i^2} \sum_k w_{0k} f'_m(t_k) f'_n(t_k)
\qquad (1 < m \leqslant n < N)~.
} \end{equation}
再看含 bridge function 的矩阵元
\begin{equation} \ali{
K_{(im), (iN)} &= \frac12 \int_{x_{i1}}^{x_{iN}} u'_{im}(x) u'_{iN}(x) \dd{x} \\
&= \frac{1}{2a_i^{3/2} \sqrt{a_i + a_{i+1}}} \sum_k w_{0k} f'_m(t_k) f'_N(t_k)
\qquad (1< m < N)~,
} \end{equation}

\begin{equation} \ali{
K_{(iN), (iN)} &= \frac12 \int_{x_{i1}}^{x_{iN}} u'_{iN}(x)^2 \dd{x} + \frac12 \int_{x_{i+1,1}}^{x_{i+1,N}} u'_{i+1,1}(x)^2 \dd{x}\\
&= \frac{1}{2(a_i + a_{i+1})} \sum_k w_{0k} \qty[\frac{1}{a_i} f'_N(t_k)^2 + \frac{1}{a_{i+1}} f'_1(t_k)^2]~,
} \end{equation}

\begin{equation} \ali{
K_{(iN), (i+1,n)} &= \frac12 \int_{x_{i+1,1}}^{x_{i+1,N}} u'_{i,N}(x) u'_{i+1,n}(x) \dd{x}\\
&= \frac{1}{2 a_{i+1}^{3/2} \sqrt{a_i + a_{i+1}}} \sum_k w_{0k} f'_1(t_k) f'_n(t_k)
\qquad (1< n < N)~,
} \end{equation}

\begin{equation} \ali{
K_{(iN), (i+1,N)} &= \frac12 \int_{x_{i+1,1}}^{x_{i+1,N}} u'_{i,N}(x) u'_{i+1,N}(x) \dd{x}\\
&= \frac{1}{2a_{i+1}\sqrt{(a_i + a_{i+1})(a_{i+1}+a_{i+2})}} \sum_k w_{0k} f'_1(t_k) f'_N(t_k)~.
} \end{equation}

\subsection{一阶导数矩阵}
\begin{equation}
\mel{u_i}{\dv{x}}{u_j} = u_i~.
\end{equation}


\subsection{势能矩阵}
\begin{equation}
V_{\alpha\beta} = \int_{x_1}^{x_{N_e+1}} u_n^j (x) u_{n'}^{j'} (x) V(x) \dd{x} = ~% 未完成
\end{equation}
\addTODO{未完成}
