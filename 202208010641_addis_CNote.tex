% C 语言笔记

% C 语言笔记

\begin{issues}
\issueDraft
\end{issues}

\subsection{printf 和 scanf}
\begin{lstlisting}[language=cpp]
#include <stdio.h>
void main( )
{
  int x,y,z;
  scanf("%d+\n,\n=%d",&x,&y);
  z=x*y;
  printf("x=%d,y=%d\n",x,y);
  printf("xy=%d\n",z);
}
\end{lstlisting}
引号内除了特殊字符,其它都需要输入一摸一样的,否则会出错.但是,1.变量前面可以多打任意多个空格和回车,2.任意多个空格、回车相连等效.
 
\verb|%d,%c,%f|, 都行 \verb|%s| 输入字符串
\begin{lstlisting}[language=cpp]
#include<stdio.h>

void main()
{
char s[20];
scanf("%s",&s);
printf("%s\n",s);
}
\end{lstlisting}
注意字符串不能包含空格回车

\subsection{getchar}
\begin{lstlisting}[language=cpp]
#include <stdio.h>
#include <string.h>
void main()
{
 int i=1;
 char str[5]={0};
    while(i<=5)
   {str[i]=getchar();i++;}

 i=1;
 while(i<=5)
 {printf("%d  ",str[i]);i++;}
 printf("\n");

 i=1;
 while(i<=5)
 {printf("%c",str[i]);i++;}
 printf("\n");
}
\end{lstlisting}

\subsection{自加自减}
含有 a++ 的表达式一律统一取a再依次自加. 含有 ++a 的表达式,按照扫描规律依次自加计算、自加计算...
\begin{lstlisting}[language=cpp]
#include <stdio.h>
#include <math.h>
void main()
{
  int a,b,c,d,e,f,g;
  a=2;
  b=(++a)+((++a)+((++a)+(++a))); // 6*4=24
  a=2;
  c=(++a)+(++a)+(++a)+(++a)+(++a); // 4+4+5+6+7=26
  a=2;
  g=(++a)+(++a)*(++a)+(++a)*2; // 5+5*5+6*2=42
  
  a=2;
  d=(a++)+((a++)+((a++)+(a++))); // 2*4=8
  a=2;
  e=(a++)+(a++)+(a++)+(a++); // 2*4=8
  a=2;
  f=(a++)+(a++)*(a++)+(a++); // 2+2*2+2=8
  printf("b=%d\nc=%d\ng=%d\nd=%d\ne=%d\nf=%d\n",b,c,g,d,e,f);
}
\end{lstlisting}
