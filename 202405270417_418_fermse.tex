% 费马小定理与欧拉定理
% keys 费马小定理|费马定理|欧拉定理|费马-欧拉定理
% license Usr
% type Tutor

\pentry{线性同余\nref{nod_linmod},同余与剩余类\nref{nod_modres},欧拉函数(数论)\nref{nod_EulFun}}{nod_d0eb}

\begin{theorem}{费马小定理}
若 $p$ 是素数,且 $p \not{\mid}~ a$,则
\begin{equation}
a^{p-1} \equiv 1 \pmod p ~.
\end{equation}
\end{theorem}

费马小定理只是费马-欧拉定理的一个特例,我们只要证明了费马-欧拉定理就自动证明了费马小定理。其中,费马-欧拉定理是指:
\begin{theorem}{费马-欧拉定理}
若 $(a, m) = 1$,则
\begin{equation}
a^{\varphi(m)} \equiv 1 \pmod m ~.
\end{equation}
\end{theorem}
\textbf{证明}:考虑 $h$ 取遍 $m$ 的一个缩系,则由于 $(a, m) = 1$,利用\autoref{the_EulFun_1}~\upref{EulFun},$ah$ 也将取遍 $m$ 的一个缩系。这就使得
\begin{equation}
\prod_{h^*(m)}{a h} \equiv \prod _{h^*(m)}h \pmod m ~,
\end{equation}
也就是
\begin{equation}
a^{\varphi(m)} \prod_{h^*(m)} h \equiv \prod_{h^*(m)} h \pmod m ~,
\end{equation}
而由于 $h$ 取 $m$ 的缩系,故每个 $h$ 都与 $m$ 互质,这必将导致他们的乘积也与 $m$ 互质,从而使得有
\begin{equation}
a^{\varphi(m)} \equiv 1 \pmod m ~.
\end{equation}
证毕!


