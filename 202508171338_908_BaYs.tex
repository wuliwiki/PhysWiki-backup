% 八元数(综述)
% license CCBYSA3
% type Wiki

本文根据 CC-BY-SA 协议转载翻译自维基百科\href{https://en.wikipedia.org/wiki/Octonion}{相关文章}。

在数学中,八元数是一种实数域上的赋范除代数,是一种超复数系统。八元数通常用大写字母 $\mathbf{O}$ 表示,也可以写作黑板粗体 $\mathbb{O}$。八元数具有 8 个维度,是四元数的 2 倍维数,而它们正是四元数的扩展。八元数是非交换的、非结合的,但满足一种较弱的结合性,即所谓的可交替性。此外,它们还具有幂结合性。

八元数不像四元数和复数那样广为人知,后者在研究和应用上更为广泛。八元数与数学中的一些例外结构(exceptional structures,需要进一步澄清)有关,其中包括例外李群。八元数在弦理论、狭义相对论和量子逻辑等领域都有应用。将 Cayley–Dickson 构造应用于八元数,可以得到十六元数。
\subsection{历史}
八元数是在 1843 年 12 月由约翰·T·格雷夫斯发现的,他的灵感来自好友威廉·罗恩·哈密顿发现四元数。就在格雷夫斯发现八元数前不久,他在 1843 年 10 月 26 日写给哈密顿的一封信中写道:“如果凭借你的炼金术,你能炼出三磅黄金,为什么要止步于此呢?”\(^\text{[1]}\)

格雷夫斯把他的发现称为 “octaves”(八度数),并在 1843 年 12 月 26 日写给哈密顿的信中提到这一点。\(^\text{[2]}\)他最早发表研究结果的时间,比阿瑟·凯莱的文章稍晚一些。\(^\text{[3]}\)八元数也被凯莱独立发现,有时被称为 Cayley 数或凯莱代数。\(^\text{[4]}\)哈密顿后来描述过格雷夫斯发现八元数的早期经过。\(^\text{[5]}\)
\subsection{定义}
八元数可以被看作是实数的 八元组(octets 或 8-元组)。
每一个八元数都是单位八元数的实线性组合:
$$
\{e_{0}, e_{1}, e_{2}, e_{3}, e_{4}, e_{5}, e_{6}, e_{7}\},~
$$
其中,$e_{0}$ 是标量或实数元,可以与实数 1 对应。

也就是说,每个八元数 $x$ 都可以写成以下形式:
$$
x = x_{0} e_{0} + x_{1} e_{1} + x_{2} e_{2} + x_{3} e_{3} 
  + x_{4} e_{4} + x_{5} e_{5} + x_{6} e_{6} + x_{7} e_{7},~
$$
其中系数 $x_{i} \in \mathbb{R}$。
\subsubsection{凯莱–迪克森构造}
一种更系统地定义八元数的方法是通过 凯莱–迪克森构造。将凯莱–迪克森构造应用于四元数,就能得到八元数。可以表示为$\mathbb{O} = \mathcal{CD}(\mathbb{H}, 1)$.\(^\text{[6]}\)

就像四元数可以定义为复数的有序对一样,八元数也可以定义为四元数的有序对。加法按分量逐一进行。若 $(a, b)$ 和 $(c, d)$ 是两对四元数,则它们的乘法定义为
$$
(a, b)(c, d) = (ac - d^{*}b,\; da + bc^{*}),~
$$
其中 $z^{\*}$ 表示四元数 $z$ 的共轭。

当将八个单位八元数与以下有序对对应时,这一定义与前面给出的等价:
$$
(1, 0),\; (i, 0),\; (j, 0),\; (k, 0),\; (0, 1),\; (0, i),\; (0, j),\; (0, k).~
$$
\subsection{算术与运算}
\subsubsection{加法与减法}
八元数的加法与减法是逐项进行的,即对应项的系数相加或相减,这与四元数的情况相同。
\subsubsection{乘法}
八元数的乘法要复杂得多。乘法对加法是分配的,因此两个八元数的积可以通过逐项相乘再求和得到,这一点和四元数类似。

每一对项的乘积由系数的乘法以及单位八元数的乘法表共同决定。一个这样的乘法表(由 阿瑟·凯莱(Arthur Cayley, 1845) 和 约翰·T·格雷夫斯(John T. Graves, 1843)分别给出)如下所示:\(^\text{[7]}\)
\begin{figure}[ht]
\centering
\includegraphics[width=10cm]{./figures/2816714b696facec.png}
\caption{} \label{fig_BaYs_1}
\end{figure}
大多数乘法表的**非对角元素**都是**反对称的**,因此它几乎是一个**斜对称矩阵**(skew-symmetric matrix),只是主对角线元素,以及 $e\_{0}$ 所在的行和列例外。

这个乘法表可以总结为如下形式:[8]
$$
e_{\ell} e_{m} =
\begin{cases}
e_{m}, & \text{if } \ell = 0, \\
e_{\ell}, & \text{if } m = 0, \\
-\delta_{\ell m} e_{0} + \varepsilon_{\ell mn} e_{n}, & \text{otherwise},
\end{cases}~
$$
其中:$\delta_{\ell m}$ 是 Kronecker delta(当 $\ell = m$ 时等于 1,当 $\ell \neq m$ 时等于 0);$\varepsilon_{\ell mn}$ 是一个完全反对称张量:当 $(\ell m n)$ 等于下列之一时,取值为 $+1$:$(123),\; (145),\; (176),\; (246),\; (257),\; (347),\; (365),$以及这些三元组的偶数次排列;对于这些三元组的奇数次排列,则取值为 $-1$。例如:$\varepsilon_{123} = +1, \quad 
\varepsilon_{132} = \varepsilon_{213} = -1, \quad\varepsilon_{312} = \varepsilon_{231} = +1$若三个指标中有任意两个相同,则 $\varepsilon_{\ell mn} = 0$。

然而,上述定义并不是唯一的。事实上,这只是 480 种可能的八元数乘法定义之一(其中 $e_{0} = 1$)。其他定义可以通过对非标量基元素 ${ e\_{1}, e_{2}, e_{3}, e_{4}, e_{5}, e_{6}, e_{7} }$ 进行置换和符号变换得到。所有这 $480$ 种代数都是同构的,因此通常没有必要区分具体采用哪一种乘法规则。
\begin{figure}[ht]
\centering
\includegraphics[width=14.25cm]{./figures/e879232d050a3043.png}
\caption{} \label{fig_BaYs_2}
\end{figure}
有时会采用一种变体,即将基底元素标记为射影线上 ${\infty, 0, 1, 2, \ldots, 6}$,该射影线是定义在有限域 $\mathrm{GF}(7)$ 上的。此时,乘法规则为$e_{\infty} = 1, \quad e_{0} e_{1} = e_{3}$,以及所有通过给下标加上常数(模 7)得到的方程。换句话说,利用以下七个三元组:$(0,1,3), \; (1,2,4), \; (2,3,5), \; (3,4,6), \; (4,5,0), \; (5,6,1), \; (6,0,2)$.这些正是**长度为 7 的二次剩余码在 $\mathrm{GF}(2)$ 上的非零码字。在这里存在两个对称性:一个是阶为 7的对称性,即对所有下标加上模 7 的常数;另一个是阶为 3的对称性,即将所有下标乘以模 7 的一个二次剩余(1,2,4)。这七个三元组也可以看作是集合 {1,2,4} 的七个平移,它们形成有限域 $\mathrm{GF}(7)$(含七个元素)上的一个循环 (7,3,1) 差集。

前文所示的法诺平面,配合 $e\_{n}$ 和 IJKL 乘法矩阵,也包含了一个符号为 $(-,-,-,-)$ 的几何代数基。它由以下七个四元数型三元组给出(省略了标量单位元):
$$
(I, j, k), \; (i, J, k), \; (i, j, K), \; (I, J, K), \; (\star I, i, l), \; (\star J, j, l), \; (\star K, k, l),~
$$
或者等价地写作:
$$
(\sigma_{1}, j, k), \; (i, \sigma_{2}, k), \; (i, j, \sigma_{3}), \; (\sigma_{1}, \sigma_{2}, \sigma_{3}), \; 
(\star \sigma_{1}, i, l), \; (\star \sigma_{2}, j, l), \; (\star \sigma_{3}, k, l).~
$$
其中:小写符号 ${i, j, k, l}$ 表示向量,例如${\gamma_{0}, \gamma_{1}, \gamma_{2}, \gamma_{3}}$大写符号 ${I, J, K} = {\sigma_{1}, \sigma_{2}, \sigma_{3}}$ 表示双向量,例如$\gamma_{{1,2,3}} \gamma_{0},$算子 $\star = i j k l$ 是伪标量元。

如果强制 $\star$ 等于单位元,那么乘法将不再是结合的,但此时可以将 $\star$ 从乘法表中移除,从而得到一个八元数的乘法表。若保持 $\star = i j k l$ 作为结合的运算元(因此不将四维几何代数约化为八元数代数),则整个乘法表都可以由 $\star$ 的定义导出。考虑上文给出的 $\gamma$ 矩阵,第五个 $\gamma$ 矩阵的定义式:$\gamma_{5}$,
正表明它是一个由 $\gamma$ 矩阵形成的四维几何代数的 $\star$ 运算结果。
\subsubsection{法诺平面助记法}
\begin{figure}[ht]
\centering
\includegraphics[width=8cm]{./figures/2548bf53b7958ddb.png}
\caption{单位八元数乘积的助记法\(^\text{[11]}\)} \label{fig_BaYs_3}
\end{figure}
一种记忆单位八元数乘法的便利助记法是通过图示来完成的,这个图表示了凯莱和格雷夫斯给出的乘法表。\(^\text{[7][12]}\)该图由 7 个点和 7 条线组成(其中通过点 1,2,3 的圆也被视为一条线),被称为法诺平面。这些线具有方向性。这 7 个点对应于 $\operatorname{\mathcal{I_{m}}}!\bigl[\mathbb{O}\bigr]$ 的 7 个标准基元素(见下文定义)。任意两个不同的点恰好确定一条唯一的线,每条线也恰好穿过 3 个点。

设 $(a, b, c)$ 是位于某条线上且按箭头方向排序的三元组,则乘法规则为:
$$
ab = c, \quad ba = -c,~
$$
并结合循环置换。

此外,还需满足以下规则:
\begin{itemize}
\item 1 是乘法单位元;
\item 对于图中的每个点$e_{i}$,都有$ e_{i}^{2} = -1$.
\end{itemize}
这些规则共同完全确定了八元数的乘法结构。此外,法诺平面中的每一条线都生成了一个与四元数 $\mathbf{H}$ 同构的子代数。
\subsubsection{共轭、范数与逆元}
\begin{figure}[ht]
\centering
\includegraphics[width=6cm]{./figures/486180f7c8c7f832.png}
\caption{一种三维助记可视化方法,将前述八元数示例中的实数顶点 $e_{0}$ 作为公共顶点,展示 7 个三元组对应的超平面\(^\text{[11]}\)} } \label{fig_BaYs_4}
\end{figure}
一个八元数
$$
x = x_{0} e_{0} + x_{1} e_{1} + x_{2} e_{2} + x_{3} e_{3} + x_{4} e_{4} + x_{5} e_{5} + x_{6} e_{6} + x_{7} e_{7}~
$$
的共轭定义为
$$
x^{*} = x_{0} e_{0} - x_{1} e_{1} - x_{2} e_{2} - x_{3} e_{3} - x_{4} e_{4} - x_{5} e_{5} - x_{6} e_{6} - x_{7} e_{7}.~
$$
共轭运算是 $\mathbb{O}$ 上的一个自反,并满足$(xy)^{*} = y^{*} x^{*} \quad \text{(注意次序变化)}$.

八元数 $x$ 的实部为
$$
\frac{x + x^{*}}{2} = x_{0} e_{0},~
$$
虚部(纯部)为
$$
\frac{x - x^{*}}{2} = x_{1} e_{1} + x_{2} e_{2} + x_{3} e_{3} + x_{4} e_{4} + x_{5} e_{5} + x_{6} e_{6} + x_{7} e_{7}.~
$$
所有纯虚八元数组成 $\mathbb{O}$ 的一个 7 维子空间,记作$\operatorname{\mathcal{I_{m}}}\!\bigl[\mathbb{O}\bigr]$.

八元数的共轭还满足以下关系:
$$
-6x^{*} = x + (e_{1}x)e_{1} + (e_{2}x)e_{2} + (e_{3}x)e_{3} + (e_{4}x)e_{4} + (e_{5}x)e_{5} + (e_{6}x)e_{6} + (e_{7}x)e_{7}.~
$$

八元数与其共轭的乘积总是非负实数:
$$
x^{*}x = x_{0}^{2} + x_{1}^{2} + x_{2}^{2} + x_{3}^{2} + x_{4}^{2} + x_{5}^{2} + x_{6}^{2} + x_{7}^{2}.~
$$
因此,可以定义八元数的范数为
$$
\|x\| = \sqrt{x^{*}x}.~
$$
这个范数与 $\mathbb{R}^{8}$ 上的标准 8 维欧几里得范数一致。

由于 $\mathbb{O}$ 上存在范数,所以每个非零元素都有逆元。对 $x \neq 0$,其唯一的逆元 $x^{-1}$(满足 $xx^{-1} = x^{-1}x = 1$)为
$$
x^{-1} = \frac{x^{*}}{\|x\|^{2}}.~
$$
\subsubsection{指数与极坐标形式}
任意一个八元数 $x$ 都可以分解为其实部和虚部:
$$
x = \mathfrak{R}(x) + \mathfrak{I}(x),~
$$
其中,实部 $\mathfrak{R}(x)$ 有时也称为标量部分,虚部 $\mathfrak{I}(x)$ 称为向量部分。

我们定义与 $x$ 对应的单位向量 $u$ 为:
$$
u = \frac{\mathfrak{I}(x)}{\|\mathfrak{I}(x)\|}.~
$$
它是一个范数为 1 的纯八元数。

可以证明\(^\text{[13]}\),任何非零八元数都可以写成:
$$
o = \|o\|\bigl(\cos \theta + u \sin \theta \bigr) = \|o\| e^{u\theta},~
$$
从而给出了八元数的极坐标形式。
