% 约翰·考克饶夫(综述)
% license CCBYSA3
% type Wiki

本文根据 CC-BY-SA 协议转载翻译自维基百科\href{https://en.wikipedia.org/wiki/John_Cockcroft}{相关文章}。

\begin{figure}[ht]
\centering
\includegraphics[width=6cm]{./figures/ed431b9628bc8c4d.png}
\caption{1961年的考克饶夫} \label{fig_YHkrf_1}
\end{figure}
约翰·道格拉斯·考克饶夫爵士(Sir John Douglas Cockcroft,1897年5月27日-1967年9月18日)是英国核物理学家,因与欧内斯特·沃尔顿共同实现原子核裂变而获1951年诺贝尔物理学奖,这一成就对核能的发展起到了关键作用。

在第一次世界大战期间,考克饶夫曾在西线担任皇家野战炮兵服役。战后,他在曼彻斯特市立工艺学院学习电气工程,同时在大都会维克斯特拉福德园区担任学徒,并成为该公司研究部门的一员。随后,他获得奖学金进入剑桥大学圣约翰学院,并于1924年6月参加三一试,成为Wrangler(剑桥数学优等生)。欧内斯特·卢瑟福接纳考克饶夫在卡文迪许实验室攻读研究生,考克饶夫于1928年在卢瑟福的指导下完成博士学位。在沃尔顿和马克·奥利芬特的合作下,他建造了后来被称为考克饶夫–沃尔顿发生器的装置。考克饶夫和沃尔顿利用这一装置首次实现了对原子核的人造裂变,这一壮举被大众称为“劈开原子”。

在第二次世界大战期间,考克饶夫担任英国供应部科研助理主任,负责雷达相关工作。他还是处理弗里施–佩尔斯备忘录(该备忘录计算出原子弹在技术上可行)相关问题的委员会成员,并参与了随后成立的MAUD委员会。1940年,作为提泽德代表团的一员,他将英国技术与美国同行共享。战争后期,提泽德代表团成果以SCR-584雷达和近炸引信的形式返回英国,并被用于协助击落V-1飞弹。

1944年5月,他出任蒙特利尔实验室主任,负责监督ZEEP和NRX反应堆的开发,以及乔克河实验室的创建。

战后,考克饶夫出任哈韦尔原子能研究机构(AERE)主任,1947年8月15日,低功率、石墨慢化的GLEEP反应堆在哈韦尔启动,成为西欧首座投入运行的核反应堆。随后在1948年又建成了英国实验堆0号(BEPO)。哈韦尔参与了温斯凯尔反应堆和化学分离工厂的设计。在他的领导下,哈韦尔还参与了前沿聚变研究,包括ZETA计划。他坚持要求在温斯凯尔反应堆的排气烟囱上安装过滤器,这一做法曾被讥讽为“考克饶夫的愚行”,但在1957年温斯凯尔火灾导致其中一座反应堆堆芯燃烧并释放放射性物质后,这一措施证明了其重要性。

1959年至1967年,他出任剑桥大学丘吉尔学院首任院长。1961年至1965年,他还担任堪培拉澳大利亚国立大学校监。
\subsection{早年经历}
约翰·道格拉斯·考克饶夫,也被称作“Johnny W.”,于1897年5月27日出生在英格兰约克郡西区托德莫登,是纺织厂主约翰·阿瑟·考克饶夫和妻子安妮·莫德(娘家姓菲尔登,Annie Maude née Fielden)的长子。他有四个弟弟:埃里克、菲利普、基思和莱昂内尔。1901年至1908年,他在沃尔斯登的英格兰教会学校接受早期教育,1908年至1909年就读于托德莫登小学,1909年至1914年就读于托德莫登中学,在校期间,他参加了足球和板球运动。在这所学校就读的女生中,有他未来的妻子尤尼斯·伊丽莎白·克拉布特里。1914年,他获得了约克郡西区的郡优秀奖学金,进入曼彻斯特维多利亚大学学习数学。

1914年8月,第一次世界大战爆发。考克饶夫于1915年6月完成在曼彻斯特的第一学年。他加入了校内的军官训练团,但并不希望成为军官。在暑假期间,他在威尔士金梅尔军营的基督教青年会食堂工作。1915年11月24日,他参军入伍。1916年3月29日,他加入了皇家野战炮兵第59训练旅,在此接受通信兵训练。随后,他被分配到西线战场第20(轻型)师所属的第92野战炮兵旅B炮兵连服役。