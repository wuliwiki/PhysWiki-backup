% 等式与不等式(高中)
% keys 方程|不等式|代数基本定理
% license Xiao
% type Tutor

\begin{issues}
\issueDraft
\end{issues}

\pentry{函数\nref{nod_functi},集合\nref{nod_HsSet}}{nod_85e3}

相等和不等关系是从小学阶段就开始接触的基础概念,但由此延伸出的方程、不等式、恒等式、方程组、解等概念,许多人往往只有一个模糊的感觉,而并不能清晰描述它们是什么。人教版初中教材中给出的“方程”定义是“含有未知数的等式称作方程(equation)”\footnote{注意并非是“含有\textbf{字母}的等式”,这是一个讹传。},给“未知数”的定义则是“方程的求解目标”,这看上去是一种令人迷惑的循环定义,进而造成有些人困惑形如$x=3$的等式是否也是一个方程。到高中阶段,在高中教材中依然没有系统性的澄清这些概念。

因此,很多学生在阅读题目时,对解题任务的理解感到模糊,不清楚解方程、联立方程究竟意味着什么,这种认识上的模糊甚至延续到大学阶段,影响对更复杂概念的掌握和后续学习的进展,很多研究者在使用这些术语时也很混乱。本文旨在解决上面提到的问题。

\subsection{一些相关的基础概念}

下面会先介绍一些基础概念。这些概念的数量很多且前后勾连,且有不少与教材上语焉不详的定义存在出入。这里不要求完整记忆,只需要认真理解并清楚自己脑海中习惯的表达与下面概念的对应即可。

\subsubsection{运算符与关系符}

表示数学运算的符号,如加、减、乘、除及各种函数(例如$\sin,\cos$)等,被称为\textbf{运算符(operator)}\footnote{在数学领域深入研究后,这一概念被称为\textbf{算子(operator)},并且具有更严格的定义,会在\enref{泛函分析}{FnalNt}中学习。在计算机科学、物理等其他领域中也会使用“运算符”这个术语,但其定义可能有所不同。}。在数学学习的基础阶段经常会接触到的五种基本运算符——加、减、乘、除及有理数次的乘方——被归类为\textbf{代数运算符(algebraic operators)}。这些代数运算符专注于基本的算术计算,是构建许多数学表达式的核心部分。

用于表示两个数学元素之间关系的符号称为\textbf{关系符(relation)}\footnote{在一些领域中,如计算机科学,关系符也可以视作一种特殊的运算符,称作“关系运算符”,其运算结果是关系判定的真值。例如,$2=3$的运算结果为$\rm False$,而$2=1+1$的运算结果为$\rm True$。}。例如,“$>$”、“$<$”、“$\leq$”、“$\geq$”、“$\neq$”这些符号称为\textbf{不等号(inequality symbols)},而“$=$”称为\textbf{等号(equality symbol)}。此外,还有许多关系符号,比如:在集合论中,有表示包含关系“$\subset$”和表示属于关系的“$\in$”;在几何中,有表示平行关系的“$\mathrel{/\mskip-2.5mu/}$”和垂直关系的“$\perp$”;在数理逻辑中,有表示等价关系的“$\equiv$”、表示蕴含关系的“$\Rightarrow$”以及表示互为充要条件关系的“$\iff$”等。

\subsubsection{表达式}

由数字、变量和运算符组成的数学符号组合称为\textbf{数学表达式(mathematical expression)},或简称\textbf{表达式(expression)},也叫\textbf{式子}。表达式可以被看作一种“描述工具”,其主要作用在于用符号表示某种数学上的数量关系或状态,而不一定需要得到一个具体的数值。例如,$3x + 2$ 和 $\sin(x)$ 都是表达式,它们描述了一种数量关系或函数的性质,而不是一道必须求解的题目。

表达式可以通过各种数学操作来进行简化和转化,比如合并同类项、约分等,这些操作称为“恒等变形”。在特定情况下,如果知道表达式中变量的具体值,还可以将该值代入表达式,从而计算出一个数值结果。

在初中阶段学习的\textbf{代数式(algebraic expression)}是一个特定类型的数学表达式。代数式仅由代数运算符连接数或字母组成,比如$2x + 3$或$x^2 - 4x + 4$。这也意味着代数式的运算范围相对有限,通常只涉及基本的代数运算,而不会涉及如三角函数或对数等更复杂的运算。可以说,代数式是表达式中的“基础款”。代数式的核心特点在于运算简单且易于处理,是数学学习的入门工具。

\subsection{等式和不等式}

\textbf{等式(equation)}和\textbf{不等式(inequality)}在定义上几乎完全相同,两者的区别主要在于所使用的关系符号不同,以及由此导致的操作规则的变化。由于本章主要讨论定义,为了简化理解,可以先专注于等式的定义。一旦掌握了等式的概念和性质,再推广到不等式就会变得更加容易。

\subsubsection{等式}

在介绍下面的概念之前,有必要先引入一个英语单词“equation”。它在中文中通常翻译为“方程”或“等式”。这里提到的“方程”并不完全等同于日常所理解的方程。为了避免混淆,本节接下来的内容将统一使用“等式”一词\footnote{当然,很多时候,为了区分会将“equality”译作等式,而“equation”译作方程。}。\textbf{等式(equation)}是指由等号($=$)连接两个表达式构成的数学符号组合,表示二者之间的相等关系。

如果将等式中的表达式视为函数的对应关系,那么等式可以看作是描述两个函数之间的相等关系。这时,称函数的自变量称为\textbf{未知数(unknown)},而使等式成立的条件称为等式的\textbf{解(solution)},这里对应的就是未知数的取值\footnote{在大学阶段,还会研究各种各样的微分方程,微分方程也是一种方程,只不过它将某个不确定的函数关系作为未知量或求解目标。这时方程的解,也就是使等式成立的条件,就是某个或某类特定的函数。具体可以参见\enref{常微分方程简介}{ODEint}或\enref{常微分方程}{ODE}},也称“解满足给定的方程”,所有满足等式的解构成的集合称为\textbf{解集(solution set)}。如果没有值使条件成立,则解集为空集。

根据未知数允许的取值范围$M$与解集$S$的交集$C=M\cap S$的不同情况,可以将等式分为以下几类:
\begin{itemize}
\item 矛盾等式:如果$C = \varnothing$,则称原等式为\textbf{矛盾等式(contradictory equation)},表示该等式在给定的取值范围内没有任何解,通常称之为\textbf{方程无解}。这时有两种情况,一种是$S=\varnothing$,比如$0\times x=1$,一种是$S\neq\varnothing,C=\varnothing$,比如$x^2=-1$在实数范围内无解,此时$M=\mathbb{R},S={\pm \I}$。
\item 条件等式:如果$C \neq \varnothing$且$C \subsetneqq M$,即等式只在某些特定的自变量取值下成立,则称原等式为\textbf{条件等式(conditional equation)}。在高中范围内研究的“方程”或者说日常生活中说的“方程”,指的就是这种狭义上的条件等式。方程的解就是使条件等式成立的条件。这时,如果$S$中只有一个元素,则称方程有\textbf{唯一解},否则称方程有\textbf{多解}。
\item 恒等式:如果$C = M$,即在允许的取值范围$M$内所有值都能使等式成立,则称原等式为$M$上的\textbf{恒等式(identity)}。\enref{恒等式}{HsIden}常用于定义某种数学量或关系,例如,三角恒等式$\sin^2 x + \cos^2 x = 1$在所有实数$x$的范围内都成立。在求解方程时通常需要进行恒等变换。
\end{itemize}

总之,理解概念是最重要的。在中文用词上,通常“等式”这个词包含了上面提到的三种类型——矛盾式、条件等式和恒等式。而“方程”则特指条件等式,表示在特定条件下成立的等式,“方程无解”则指矛盾等式,之后不引起歧义时也会如此使用。在解题和讨论时,明确所指的对象,可以避免不必要的混淆。

现在回答“如何看待$x = 3$?”这个问题。$x = 3$既可以看作是一个\textbf{方程(条件等式)},即一个条件下成立的等式;也可以看作是这个方程的\textbf{解(恒等式)},即它的取值范围是解集$\{3\}$。从解的角度来看,可以将$x = 3$表示为$x \in \{3\}$。这种写法清晰地表明了$x$是要表示的值,而非条件等式的情况,可以避免歧义。因此,在不产生歧义的情况下,通常直接使用$x = 3$来表示解,而需要特别强调解集时采用解集的写法会更为精确,尤其是在讨论不等式时。

\subsubsection{不等式}

在理解了等式那些错综复杂的概念之后,不等式的定义就更容易掌握了。\textbf{不等式(inequation)}是由不等号连接两个表达式组成的数学符号组合,用来表示两个表达式之间的不等关系\footnote{由于一般情况下$<,\leq$以及$>,\geq$的情况接近,而$\neq$的结论一般是平凡的。下面在讨论时为免麻烦,基本只讨论$\leq,\geq$的情况。}。无论具体形式为何,只要包含不等号的表达式组合都称为不等式,而由于“不等式”是一个宽泛的称谓,在使用时会容易忽略它到底是指一个具有条件解的“不等式方程”还是一个在所有取值范围上都成立的“恒等不等式”(这两个术语并非正式定义,仅用作便于理解的表述)。

在不等式的研究中,许多概念,例如未知数、解、解集,以及矛盾不等式、条件不等式和恒成立不等式,与等式中的定义类似,仅仅因为不等号的存在而带来了一些细微的差别。在理解不等式时,以下两个要点尤其重要:
\begin{itemize}
\item 由于不等式往往会对应多个值或范围,而非单一解,它的解通常以解集的形式给出。在高中阶段,通常会用区间来描述解集。例如,对于不等式$x^2 - 1 > 0$,其解集为$(-\infty, -1) \cup (1, +\infty)$。
\item 不等式通常与等式不同,没有明确的“方程”和“恒等式”之分,或者更准确地说,不等式一般不具备单一解的特性,因此其类型需要根据上下文来判定。因此,不等式有时指条件不等式,有时指恒成立的不等式,需要根据使用情况来判定。一般而言,条件不等式通常在解题过程或计算题中出现。例如,不等式$x + 2 > 5$只有在$x > 3$时成立,这种情况与等式中的“方程”类似,也经常会一同出现;恒成立不等式则通常出现在证明题或者与恒等式一同出现。例如不等式$x^2 \geq 0$在所有实数$x$的取值下均成立。恒成立不等式在数学中用于描述某些始终满足的约束条件,因此常见于分析和推理中。
\end{itemize}

\subsubsection{方程组}

在初中时,就已经接触过方程组,当时给出的定义是“把两个必须同时满足的方程合在一起组成一个方程组”。这个定义虽然简单,但传达了一个核心概念,即方程组涉及的每个方程都必须同时成立。用大括号将多个方程括在一起,正是为了表示它们的联合条件。

\textbf{方程组(system of equations)}指的就是一组需要同时成立的方程。注意到这里使用的英语单词是“system”,中文里它通常翻译为“系统”。因此其实可以将方程组看作是一个“系统”,或者说方程组本身就是描述了一个系统\footnote{很多时候,如果看到数学领域的“某某系统”,可能是指“某某方程组”。}。每个等式的解集$S_i$都相当于要从总体范围$M$中划分出一部分,这一过程就像描述“硬的、红色的、圆的水果”,每个描述词(“硬的”、“红色的”、“圆的”)都对“水果”进行了限定,而每个方程也都类似于为系统添加了一个限制条件,这个过程中的限制条件称为\textbf{约束(constraint)},从“约束”的角度理解方程组提供了一个重要的视角。最终,所有的形容词同时成立的水果指的是包含“苹果”水果等构成的集合,而同样地,同时使所有方程成立的条件称为方程组的\textbf{解(solution)},而方程组的解集也就是各方程解集的交集。

方程组中的方程经过变形后,在大多数情况下\footnote{如果一个方程不提供约束时,称其与其他的方程\enref{线性相关}{linDpe}。},一个方程会对一个变量提供约束,从而,通常方程组是针对多个变量的。

不仅方程可以用来描述约束,\textbf{不等式}也可以发挥类似的作用。通常,只有不等式构成的一组称为\textbf{不等式组(system of inequalities)}。如果既包含等式又包含不等式,则直接称为\textbf{约束系统(system of constraints)}\footnote{有些时候,也会将上述两种情况都称作“不等式组”,或粗糙地将所有的约束系统都称为“方程组”。下面统称约束系统}。约束系统通常出现在\enref{优化问题}{Optimi}中,尤其是在线性规划和非线性规划中。在这些情境中,\textbf{线性系统(linear system)}和\textbf{非线性系统(nonlinear system)}的定义取决于方程组中方程的类型(线性或非线性)。无论是方程组、不等式组还是约束系统,其解集都由所有约束条件共同定义,即由使所有约束同时成立的变量值构成。

对于两个表达式$f$和$g$,方程组$\begin{cases}f = 0 \\ g = 0\end{cases}$与方程$f \cdot g = 0$的区别体现了“交集与并集”,“且与或”的性质,这个性质在\enref{解析几何}{JXJH}中有着重要的应用:

\begin{itemize}
\item 方程组 $\begin{cases}f = 0 \\ g = 0\end{cases}$ 表示的是一个同时满足两个条件的系统,要求$f = 0$和$g = 0$这两个等式必须同时成立。即,解集只包含那些能使$f$和$g$都为零的变量值。这种情况下,解集通常会比单个方程更为严格(更“小”),因为变量必须满足两个独立的条件。
\item 方程 $f \cdot g = 0$ 表示$f$和$g$至少有一个为零。由于乘积为零只需要其中一个因子为零,因此这个方程的解集包含了满足$f = 0$的所有解以及满足$g = 0$的所有解。解集因此是满足这两个条件的所有解的并集,而不要求两者一定同时为零。
\end{itemize}

简单来说,方程组$\begin{cases}f = 0 \\ g = 0\end{cases}$的解集是$f = 0$和$g = 0$解集的交集,而方程$f \cdot g = 0$的解集是它们的并集。

\subsection{解}

根据前面的定义,方程和不等式仅在其解集中成立。换句话说,方程或不等式的解集包含了所有能够满足该等式或不等式的条件。方程的解可以分为两大类:

\begin{itemize}
\item 如果一个方程的解可以通过有限次的代数运算得出,则称该解为\textbf{解析解(Analytical Solution)}。这类解通常可以用代数表达式明确地表示。例如,一元二次方程的解可以用平方根和加减运算清晰表达出来。解析解的优点在于其精确性,但对于一些复杂方程,解析解可能难以找到,甚至不存在。
\item 对于复杂方程或无解析解的方程,可以使用\enref{数值分析}{NLinEq}方法(如二分法、牛顿法等)进行近似计算,以获得方程的解。此时得到的解称为\textbf{数值解(Numerical Solution)}。数值解通常由计算机进行迭代计算,可以达到很高的精度,因此适用于求解复杂问题。但数值解始终是近似解,是存在精度限定的非“精确”解。
\end{itemize}

总体而言,解析解精确,但并不总是存在;数值解虽然是近似的,却能够为几乎所有方程提供可用的结果。在高中阶段基本只关注解析解,但需要意识到在实际数学应用中,存在许多方程无法获得解析解,或解析求解极为复杂,在保证精度的前提下,数值解是一种重要的替代方案。

\subsubsection{求解规则}

如果两个方程组或两个方程的解集完全相同,则称它们\textbf{等价(equivalent)}。求解一个方程(或方程组)时,通常通过变换将其转换为一个等价的方程(组),以便更容易求解。然而,在一些情况下,为了简化方程或产生某些错误时,把给定的方程变换成另一个方程后,会使得原本方程的解集成为新方程解集的真子集。这意味着在求解过程中可能引入\textbf{增根(extraneous solution,也称伪解)}。为了避免这种情况,一般会将求得的解代入原方程进行验算,以确保所有解均为原方程的有效解。这是一种良好的解题习惯,能够有效减少因变换操作导致的错误。同时,对增根的研究,也促进了对同一问题不同的审视视角,产生了许多新理论,如\enref{复数}{CplxNo}等。而了解清楚求解规则成立的原因,对解析几何的学习非常重要。下面介绍的具体规则,都是前面等价原则的具体体现,如果遇到陌生情况拿不准,一定要回归到等价的原则上来分析。

等式的等价变换包括在等式两侧同时进行相同的运算操作,例如加(任意)、乘(非零)的等价表达式\footnote{其中,减法可看作加上表达式的相反数,除法则相当于乘以表达式的倒数。由于倒数定义要求$0$没有倒数,因此不能除以$0$。}(包括相同或恒等两种情况),以及在等式两侧套用相同的函数。加、乘相同表达式的操作一般被简化为移项(如$x+1=0\to x=-1$)、消去(如$x+2=2x\to 2=x$)和约化(如$2x=2\to x=1$)。而函数作用时可能会带来增根\footnote{此时施加的函数不是单调函数。},例如$x+1=1$两侧套用$f(x)=x^2$后,变成$(x+1)^2=1$,带来增根$x=-2$。乘以与零等价的表达式是一种需要特别注意的操作,因为它可能会引入增根,甚至导致解集错误。例如,对于方程$x+1=0$,如果两侧分别乘以$x-2=0$两侧的表达式,则得到$(x+1)(x-2)=0$,从而引入了增根$x=-2$。而若在$x+1=1$的两侧乘以$x-2=0$的表达式,则得到$(x+1)(x-2)=0$,与原方程没有共同解。因此,只有在与$0$等价的方程情境下,才能合理地乘以与零等价的表达式,以确保不会产生错误。

不等式的等价变换包括在不等式两侧同时加(任意)、乘(非零)等价的表达式以及施加相同的单调函数。与等式不同的是,在乘以负数以及施加单调递减的函数时,需要改变不等号方向,即$\{<,\leq\}$与$\{>,\geq\}$的互换(如$x>2\to-x<-2$以及$\displaystyle x-2>3\to\left({1\over2}\right)^{x-2}<\left({1\over2}\right)^3$)。由于不等式通常以解集的形式给出,验证增根较为困难,因此确保按照条件变换,避免变换后不等式的解集与原不等式不同尤为重要。乘以与零等价的表达式,会导致不等式变成等式。

因式分解则常用来将复杂的表达式简化为更易处理的形式。根据之前的讨论,通过因式分解,可以将原本复杂的表达式化为$h=f\cdot g=0$的形式,一般$f,g$的求解会较$h$更容易。根据方程解集的特性,分别求得二者再取并集即可。不等式部分稍微复杂,若$h=f\cdot g\geq0$,则解集为:$\begin{cases}f \geq 0 \\ g \geq 0\end{cases}$与$\begin{cases}f \leq 0 \\ g \leq 0\end{cases}$的并集,或者说$f,g$同号;若$h=f\cdot g\leq0$,则解集为:$\begin{cases}f \leq 0 \\ g \geq 0\end{cases}$与$\begin{cases}f \geq 0 \\ g \leq 0\end{cases}$的并集,或者说$f,g$异号。

方程组的求解过程可以看作是分别求每个方程的解集,然后取其交集。不过相较于求解方程,求解方程组有一个好处是,由于方程组中的方程需要同时成立,于是这些等式可以视为在变量取值范围为解集的前提下恒成立,从而允许在一个方程$A$进行变换时,将另一个方程$B$的等式两侧作为等价量代入。这个操作进一步简化为方程加减(如$\begin{cases}x+y= 0 \\ x-y =2\end{cases}\to 2x=2$)、代入(如$\begin{cases}x+y= 0 \\ x=2\end{cases}\to 2+y=0$)等操作。

\subsubsection{有理不等式的解集}

有一类常见的不等式,形如$F(x) \geq 0$或$F(x) \leq 0$,其中$\displaystyle F(x)={P(x)\over Q(x)}$为一个有理表达式,即$P(x),Q(x)$为多项式,称为\textbf{有理不等式(rational inequality)}。

求解有理不等式时,通常会使用\textbf{根轴法(sign chart method)},它通过分析函数分子和分母的符号变化,找到$x$的解集区间。根轴法的基本步骤如下:
\begin{enumerate}
\item 标准化:将$\displaystyle F(x)={P(x)\over Q(x)}$化作$P(x)Q(x)$,分别因式分解得到$P^{m_1}_1(x)\cdots P^{m_p}_p(x)Q^{n_1}_1(x)\cdots Q^{n_q}_q(x)$,其中$m_i$和$n_j$表示每个因式的指数。要求每个因式的最高次都是正的,如果有负号放在最外面(如$(x-1)(1-x)\to-(x-1)^2$),因式分解要彻底(如:$x^2+bx+c$如果$b^2-4c\geq0$则继续分解为两个因式,否则因式分解到此为止。)
\item 标出零点:将分解后的每个因式$P_i(x)$和$Q_j(x)$分别设为零,得到方程$P_i(x) = 0$和$Q_j(x) = 0$,解得各自的零点并将它们标在数轴上。如果某个因式无解,则不标记。
\item 穿轴画线:根据第一步的最外层符号确定画线的起始位置。如果外层符号为正,从数轴右上方开始;如果为负,从数轴右下方开始。然后从右至左画线,经过每个零点时在数轴的上方和下方切换。当笔尖到达一个零点时,判断是否“穿过”该点:如果该因式的指数为奇数($m_i$或$n_j$为奇数),则穿过零点,从上方切到下方或从下方切到上方;如果指数为偶数,则不穿过该点,笔尖接触该点后停留在同一侧继续向左。
\item 确定结果:画线完成后,曲线在数轴上方的区域表示$F(x) > 0$,下方的区域表示$F(x) < 0$。根据题目要求选择符合不等式条件的区间。如果不等式包含$\geq$或$\leq$符号,还需判断区间端点的取值情况。检查每个端点属于分子$P(x)$还是分母$Q(x)$:如果端点来自$P(x)$,则可以取该点;如果端点来自$Q(x)$,则该点不能取。
\end{enumerate}
\addTODO{根轴法画图}
\begin{example}{求解不等式$\displaystyle \frac{x^2 - 1}{x + 2} \leq 0$}
解:
\begin{enumerate}
\item 原不等式等价于$\begin{cases}(x-1)(x+1)(x+2)\leq0 \\ x+2\neq0\end{cases}$
\item 求解的零点分别为$x=-2,-1,1$,标记在数轴上。
\item 由于外面的符号为正,因此从右上角开始,依次穿过$1,-1,-2$。
\item 在数轴下方的部分是$(-\infty,-2)\cup(-1,1)$。分子的零点分别是$-1,1$,在这两点处可取等,因此最终结果为$(-\infty,-2)\cup[-1,1]$。
\end{enumerate}
\end{example}

根轴法本质上是通过数轴上的符号分析来粗略描绘不等式表达式对应的函数图像,以便确定解集。对于形如$\displaystyle F(x) = \frac{P(x)}{Q(x)} \geq 0$的不等式,由于分母$Q(x) \neq 0$,即$\left(Q(x)\right)^2 > 0$恒成立,因此可以在不等式两侧同时乘以正数$\left(Q(x)\right)^2$,从而将不等式$\displaystyle F(x) \geq 0$转化为$\displaystyle\begin{cases}P(x)Q(x) \geq 0 \\ Q(x) \neq 0\end{cases}$,这与$F(x) \geq 0$具有相同的解集。从数轴\textbf{右上方}还是\textbf{右下方}开始,取决于$F(x)$在$+\infty$处的性质。当$x$趋向于正无穷时,表达式$F(x)$的符号决定了初始的绘图方向(即开始的位置)。至于是否“穿越”每个零点及为何“穿越”,穿越或保持在同侧取决于零点附近$P(x)$或$Q(x)$的变化方向——即它们的指数(奇次还是偶次),这涉及极限分析,此处暂不介绍。

\subsubsection{解与零点}

\begin{definition}{代数学基本定理}
任何一个 $n$ 次多项式函数在复数域上都有 $n$ 个零点(重数计入)。
\end{definition}
这意味着在复数范围内,可以找到所有多项式方程的解。



