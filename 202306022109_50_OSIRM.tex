% 开放系统互联基本参考模型
% 网络 模型 五层 OSI

\textbf{开放系统互联基本参考模型}(Open System Interconnection Reference Model, OSI/RM)是由国际标准化组织提出的一种试图使得各种计算机在世界范围内互联成网的标准框架。该模型的目标是使得全球计算机能够遵循同一个协议,互相连接并交换数据。

OSI模型具有七个逻辑层次。从下到上依次为:物理层、数据链路层、网络层、传输层、会话层、表示层、应用层。逻辑层次的意思是对于网络上所传输的数据流的封装的层次,而并非是真实物理存在的实体。
\begin{figure}[ht]
\centering
\includegraphics[width=5cm]{./figures/fe3db6a185508986.png}
\caption{OSI七层模型} \label{fig_OSIRM_1}
\end{figure}

(1)物理层(physical layer)

物理层协议其实就是最底层的通讯协议。该层上传输的是数据的二进制位,即比特(bit)。物理层要做的是用适当的高、低电压来表示0和1。传输信号所用的通讯线路,例如网线、电缆、光缆、微波信道等不属于物理层,也有人将其称为第0层。

(2)数据链路层(data link layer)

数据链路层常简称为链路层。数据在网络上传输的基本单位是包含两个节点主机的链路。数据链路层的基本任务就是将

我们知道,两台主机之间的数据传输,总是在一段段的链路上传送的,这就需要使用专门的链路层的协议。在两个相邻结点之间传送数据时,数据链路层将网络层交下来的IP数据报组装成帧(framing),在两个相邻结点间的链路上传送,帧(frame)。每一-帧包括数据和必要的控制信息(如同步信息、地址信息、差错控制等)。在接收数据时,控制信息使接收端能够知道一个帧从哪个比特开始和到哪个比特结束。这样,数据链路层在收到一个帧后,就可从中提取出数据部分,.上交给网络层 。