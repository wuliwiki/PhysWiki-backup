% LSZ约化公式
% LSZ约化公式|散射理论|编时格林函数


S 矩阵被定义为入态和出态的内积,因此可以用 $C_\alpha,D_\alpha$ 表示为
\begin{equation}\label{LSZ_eq1} \begin{aligned}
\langle\psi^-|\psi^+\rangle=\langle\bvec k'_1 \cdots \bvec k'_{n'}|S|\bvec k_1\cdots\bvec k_n\rangle=\bra{\Omega} T D_1^\dagger \cdots D_{n'}^\dagger C_1\cdots C_n \ket{\Omega}
\end{aligned}\end{equation} 
上面插入了编时算符 $T$ 不改变表达式,因为 $D_\alpha$ 出现在无穷远未来而 $C_\alpha$ 出现在无穷远过去.定义两组新的算符 $\bar C_\alpha$ 和 $\bar D_\alpha$:
\begin{equation} \begin{aligned}
&\bar C_\alpha=g(t_\alpha-T_+)\int \frac{\dd[3]{\bvec p}}{(2\pi)^3} f_\alpha(\bvec p)e^{-i\omega_{\bvec p}t_\alpha+i\bvec p\cdot \bvec x_\alpha},\ T_-\rightarrow -\infty\\
&\bar D_\alpha=g(t'_\alpha-T_-)\int \frac{\dd[3]{\bvec p}}{(2\pi)^3} f'_\alpha(\bvec p)e^{-i\omega_{\bvec p}t'_\alpha+i\bvec p\cdot \bvec x'_\alpha},\ T_+\rightarrow +\infty
\end{aligned}\end{equation}
如果将 \autoref{LSZ_eq1}  中的某个 $C_\alpha$ 替换为 $\bar{C}_\alpha$,那么在编时算符的作用下它被移到最左侧,于是湮灭 $\bra{\Omega}$.如果将某个 $D_\alpha^\dagger$ 替换为 $\bar{D}_\alpha^\dagger$,则它被移到最右侧后湮灭 $\ket{0}$.因此可以将 \autoref{LSZ_eq1}  改写为
\[
\bra{\Omega} T (D_1^\dagger-\bar{D}_1^\dagger) \cdots (D_n^\dagger-\bar{D}_n^\dagger) (C_1-\bar{C}_1)\cdots (C_n-\bar{C}_n) \ket{\Omega}
\]
算符 $C_\alpha$ 与 $\bar{C}_\alpha$ 的波包调制函数,其 $f_\alpha(\bvec p)$ 分量是完全相同的.对于在壳的 $l$,两个波包调制函数的在壳傅里叶分量 $\tilde{u}_\alpha(-l)$ 是相同的.对于自由标量场论来说,满足自由 Klein-Gordon 方程的场算符只有在壳的傅里叶分量,那么 $C_\alpha$ 和 $\bar{C}_\alpha$ 将没有差别.这意味着两个算符 $C_\alpha$ 与 $\bar{C}_\alpha$ 的差别来自于理论的相互作用,可以对它作具体的计算:
\begin{equation} \begin{aligned}
&C_\alpha-\bar{C}_\alpha=\int \dd[4] x\left[g\left(t-T_{-}\right)-g\left(t-T_{+}\right)\right] \int \frac{d^3 {\bvec p}}{(2 \pi)^3} f_\alpha(\mathbf{p}) e^{-i \omega_{\mathbf{p}} t+i \bvec p \cdot \bvec x} \phi(x)\\
&=\int \frac{\dd[3]{\bvec p}}{(2 \pi)^3} \frac{\dd \nu}{2 \pi} f_\alpha(\mathbf{p}) \tilde{g}(\nu) \int \dd[4] x\left[e^{-i \nu\left(t-T_{-}\right)}-e^{-i \nu\left(t-T_{+}\right)}\right] e^{-i \omega_{\mathrm{p}} t+i \bvec p \cdot \bvec{x}} \phi(x)\\
&=\int \frac{\dd[3]{\bvec p}}{(2 \pi)^3} \frac{\dd \nu}{2 \pi} f_\alpha(\mathbf{p}) \tilde{g}(\nu) \frac{\left(e^{i \nu T_{-}}-e^{i \nu T_{+}}\right)}{m^2-\left(\omega_{\mathbf{p}}+\nu\right)^2+|\mathbf{p}|^2} \int \dd[4] x\left[\left(m^2+\partial^2\right) e^{-i\left(\omega_{\mathbf{p}}+\nu\right) t+i \mathbf{p} \cdot \mathbf{x}}\right] \phi(x)\\
&=\int \frac{\dd[3]{\bvec p}}{(2 \pi)^3} \frac{\dd \nu}{2 \pi} f_\alpha(\mathbf{p}) \tilde{g}(\nu) \frac{\left(e^{i \nu T_{-}}-e^{i\nu T_{+}}\right)}{-2\omega_{\bvec p}\nu - \nu^2} \int \dd[4] x e^{-i\left(\omega_{\mathbf{p}}+\nu\right) t+i \mathbf{p} \cdot \mathbf{x}}\left[\left(m^2+\partial^2\right) \phi(x)\right]\\
&=\int \frac{\dd[3]{\bvec p}}{(2 \pi)^3} \frac{\dd \nu}{2 \pi} f_\alpha(\mathbf{p}) \tilde{g}(\nu) \frac{2 \pi i \delta(\nu)}{2 \omega_{\mathbf{p}}} \int \dd[4] x e^{-i\left(\omega_{\mathbf{p}}+\nu\right) t+i \bvec p \cdot \bvec{x}}\left[\left(m^2+\partial^2\right) \phi(x)\right]\\
&=i \int \frac{\dd[3]{\bvec p}}{(2 \pi)^3 2 \omega_{\mathbf{p}}} f_\alpha(\mathbf{p}) \int \dd[4]x e^{-ipx}\left[\left(m^2+\partial^2\right) \phi(x)\right]
\end{aligned}\end{equation}
其中第四行到第五行的推导利用了 $\tilde{g}(\nu)$ 宽度远小于 $m$ 的性质将分母中的 $\nu^2$ 当作无穷小量略去,当 $T_+\rightarrow +\infty,T_-\rightarrow -\infty$ 时 $(e^{i\nu T_-}-e^{i\nu T_+})/(-\nu)$ 可以改写为 $2\pi i\delta(\nu)$.在最后一步中我们取定了 $\tilde g(0)=\int \dd{t} g(t)=1$,如果它不为 $1$,我们可以将系数吸收进 $f_\alpha(\bvec p)$ 中.用类似的方法可以计算 $D_\alpha^\dagger-\bar{D}_\alpha^\dagger$.
\[
D_\alpha^\dagger-\bar{D}_\alpha^\dagger=i \int \frac{\dd[3]{\bvec p}}{(2 \pi)^3 2 \omega_{\mathbf{p}}} f'^{*}_\alpha(\mathbf{p}) \int \dd[4]x e^{ipx}\left[\left(m^2+\partial^2\right) \phi(x)\right]
\]
现在离 LSZ 约化公式的导出只差一步之遥.S-矩阵依赖于归一化的选择,也就依赖于 $f_\alpha(\bvec p)$ 和 $f'\alpha(\bvec p)$,因此我们需要对 $C_\alpha\ket{\Omega}$ 和 $D_\alpha\ket{\Omega}$ 进行适当的归一化.利用 autoref47 和 autoref48 以及 $\tilde g(0)=1$,可以得到:
\begin{equation} \begin{aligned}
C_\alpha\ket{\Omega}&=\int \frac{\dd[4]{l}}{(2\pi)^4} \tilde{u}_\alpha(-l)\tilde{\phi}(l)\ket{\Omega}\\
&=\sqrt{Z}\int \frac{\dd[3]{\bvec p}}{(2\pi)^3 2\omega_{\bvec p}} \left.\tilde{u}_\alpha(-p)
\right|_{p^0=\omega_{\bvec p}} \ket{\bvec{p}}\\
&=\sqrt{Z}\int \frac{\dd[3]{\bvec p}}{(2\pi)^3 2\omega_{\bvec p}} f_\alpha(\bvec p) \ket{\bvec{p}}
\end{aligned}\end{equation}
$D_\alpha\ket{\Omega}$ 也是类似的.为了使这些构造的入态和出态具有和物理态一样的归一化条件,可以约定
\begin{equation} \begin{aligned}
&f_\alpha(\bvec p)=Z^{-1/2}(2\pi)^3 2\omega_{\bvec p}\delta(\bvec p-\bvec k_\alpha)\\
&f'_\alpha(\bvec p)=Z^{-1/2}(2\pi)^3 2\omega_{\bvec p}\delta(\bvec p-\bvec k'_\alpha)
\end{aligned}\end{equation}
那么由 $C_\alpha,D_\alpha$ 构造的入态和出态就是
\begin{equation} \begin{aligned}
\ket{\psi^+} &=C_1 \cdots C_n|\Omega\rangle=\left|\mathbf{k}_1, \cdots, \mathbf{k}_n\right\rangle_{\text {in }} \\
\ket{\psi^-} &=D_1 \cdots D_{n'}|\Omega\rangle=\left|\mathbf{k}_1^{\prime}, \cdots, \mathbf{k}_{n'}^{\prime}\right\rangle_{\text {out }}
\end{aligned}\end{equation}
最终我们得到 LSZ 约化公式:
\begin{equation} \begin{aligned}
\notag\langle\psi^-|\psi^+\rangle&=\langle\bvec k'_1 \cdots \bvec k'_{n'}|S|\bvec k_1\cdots\bvec k_n\rangle= i^{n+n'} Z^{-\left(n+n'\right) / 2}
\prod_{i=1}^{n'}\int \dd[4]{x_i'} e^{i k_i' x_i'}\left(m^2+\partial_{x_i'}^2\right)  \\
& \prod_{j=1}^n\int\dd[4] x_j e^{-i k_j x_j}\left(m^2+\partial_{x_j}^2\right)
\left\langle \Omega\left|T \phi\left(x_1'\right) \cdots \phi\left(x_n'\right) \phi\left(x_{1}\right) \cdots \phi\left(x_{n}\right)\right| 0\right\rangle
\end{aligned}\end{equation} 