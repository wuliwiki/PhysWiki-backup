% 阿德里安-马里·勒让德(综述)
% license CCBYSA3
% type Wiki

本文根据 CC-BY-SA 协议转载翻译自维基百科\href{https://en.wikipedia.org/wiki/Adrien-Marie_Legendre}{相关文章}。

\begin{figure}[ht]
\centering
\includegraphics[width=6cm]{./figures/ce46ca8eff284986.png}
\caption{根据1823年素描绘制的肖像} \label{fig_adla_1}
\end{figure}
阿德里安-马里·勒让德(Adrien-Marie Legendre,/ləˈʒɑːndər, -ˈʒɑːnd/\(^\text{[2]}\);法语发音:[adʁiɛ̃ maʁi ləʒɑ̃dʁ];1752年9月18日-1833年1月9日)是一位法国数学家,对数学做出了众多贡献。以他命名的重要概念包括勒让德多项式和勒让德变换。他还因在最小二乘法上的贡献而著称,尽管卡尔·弗里德里希·高斯在他之前已发现这一方法,勒让德却是第一个正式发表该方法的人。\(^\text{[3][4]}\)
\subsection{生平}
阿德里安-马里·勒让德于1752年9月18日出生在巴黎的一个富裕家庭。他在巴黎马扎兰学院接受教育,并于1770年在物理与数学方面完成了论文答辩。1775年至1780年间,他在巴黎的军事学院任教,1795年起又在国立高等师范学院任教。同时,他还隶属于法国经度局。1782年,柏林科学院因其关于阻力介质中抛体运动的论文授予勒让德奖项,这篇论文也使他引起了拉格朗日的注意。\(^\text{[5]}\)

法国科学院于1783年任命勒让德为副成员,1785年成为正式院士。1789年,他被选为英国皇家学会会士。\(^\text{[6]}\)

他参与了英法联合测量项目(Anglo-French Survey, 1784–1790),该项目旨在通过三角测量法精确计算巴黎天文台与格林尼治天文台之间的距离。为此,他于1787年与多米尼克·卡西尼伯爵和皮埃尔·梅尚一同前往多佛和伦敦。他们三人还拜访了天王星的发现者威廉·赫歇尔。

勒让德在1793年法国大革命期间失去了他的私人财产。同年,他与玛格丽特-克洛迪娜·库安结婚,后者帮助他理顺了财务事务。1795年,勒让德成为重组后的法国科学院(当时改名为“国家科学与艺术研究院”Institut National des Sciences et des Arts)数学部的六位成员之一。1803年,拿破仑对该研究院进行了重组,勒让德成为几何学部的成员。

1799年至1812年间,勒让德担任巴黎军事学院毕业炮兵生的数学考官;1799年至1815年,他担任巴黎综合理工学院常任数学考官。\(^\text{[7]}\)1824年,因他拒绝在国家研究院投票支持政府候选人,勒让德失去了来自军事学院的退休金。1831年,他被授予荣誉军团勋章军官级别。\(^\text{[5]}\)
\begin{figure}[ht]
\centering
\includegraphics[width=6cm]{./figures/7f9e43d88374097d.png}
\caption{阿德里安-玛丽·勒让德的纹章(1811年受封为骑士时所授)} \label{fig_adla_2}
\end{figure}
勒让德于1833年1月9日因长期严重的疾病在巴黎去世。他的遗孀悉心保存了他的遗物,以纪念他。1856年她去世后,被安葬在他们曾居住的奥特伊村,与勒让德合葬,并将他们最后的乡间住宅遗赠给该村。勒让德的名字被刻在埃菲尔铁塔上的72位名字之一中,以示纪念。
\subsection{数学工作}
\begin{figure}[ht]
\centering
\includegraphics[width=8cm]{./figures/35432ec5c82dc2ec.png}
\caption{勒让德在奥特伊公墓的墓地} \label{fig_adla_3}
\end{figure}
阿贝尔在椭圆函数方面的研究是建立在勒让德工作的基础之上的,而高斯在统计学和数论方面的一些成果则完善了勒让德的工作。他发展并最早在高斯之前向同时代人传播了最小二乘法\(^\text{[8]}\),这一方法在线性回归、信号处理、统计学和曲线拟合中有广泛应用;该方法于1806年以附录形式刊登在他关于彗星轨道的著作中。如今,“最小二乘法”一词正是从法语“méthode des moindres carrés”直译而来。

他的主要著作是《积分演算练习》,分三卷分别于1811年、1817年和1819年出版。在第一卷中,他引入了椭圆积分、贝塔函数和伽马函数的基本性质,首次使用了符号$\Gamma$,并将其归一化定义为$\Gamma(n+1) = n!$。第二卷中进一步研究了贝塔函数和伽马函数,并将其应用于力学领域,例如地球自转与椭球体的引力问题\(^\text{[9]}\)。1830年,他给出了费马大定理在指数 $n = 5$情形下的一个证明,这一命题也曾在1828年被勒让德·狄利克雷证明过\(^\text{[9]}\)。

在数论中,勒让德提出了二次互反律的猜想,该定律随后由高斯证明;与此相关的“勒让德符号”也因此以他的名字命名。他还在素数分布及将分析方法应用于数论方面做出了开创性的工作。他于1798年提出了素数定理的猜想,这一猜想最终在1896年由阿达马和德拉瓦莱-普桑严格证明。

勒让德在椭圆函数方面也做出了大量重要工作,包括对椭圆积分的分类,但要到阿贝尔研究雅可比函数反函数时,这一问题才被完全解决。

他还因提出勒让德变换而闻名,这一变换可用于将拉格朗日力学形式转换为哈密顿力学形式。在热力学中,该变换也用于从内能导出焓、以及赫姆霍兹和吉布斯自由能。此外,“勒让德多项式”亦以他命名,它是勒让德微分方程的解,在物理学和工程学中广泛出现,如静电学等领域。

勒让德最广为人知的成就是其著作《几何原本》,该书于1794年出版,约一百年内一直是几何学的主要入门教材。这本书对欧几里得《几何原本》中的大量命题进行了重新编排与简化,使之成为更高效的教学读物。
\subsection{荣誉}
\begin{itemize}
\item 1832年,当选为美国艺术与科学学院外籍荣誉院士。\(^\text{[10]}\)
\item 月球上的勒让德陨石坑以他命名。
\item 主带小行星 26950 Legendre 也以他命名。
\item 勒让德是72位被纪念于埃菲尔铁塔首层铭牌上的杰出法国科学家之一。
\item 巴黎第十七区有一条街道以他的名字命名。
\end{itemize}
\subsection{著作}
\textbf{论文}
\begin{itemize}
\item 1782年:《在阻力介质中弹道轨道的研究》(*Recherches sur la trajectoire des projectiles dans les milieux résistants*),为柏林科学院所设弹道学奖项所撰。
\end{itemize}

\textbf{书籍}

* 《几何原本》(*Éléments de géométrie*),教科书,1794年出版\[11]
* 《数论试探》(*Essai sur la Théorie des Nombres*),1797–1798年(法兰西共和国六年),第二版1808年,第三版于1830年分两卷出版
* 《测定彗星轨道的新方法》(*Nouvelles Méthodes pour la Détermination des Orbites des Comètes*),1805年
* 《积分演算练习》(*Exercices de Calcul Intégral*),三卷本,分别出版于1811年、1817年和1819年
* 《椭圆函数论》(*Traité des Fonctions Elliptiques*),三卷本,分别出版于1825年、1826年和1830年
