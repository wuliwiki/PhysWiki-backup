% ADM形式
% keys ADM形式|时空3+1分解

\begin{issues}
\issueTODO
\end{issues}

将4维时空分解成1维的时间和3维的空间称之为时空的\textbf{3+1分解},用数学的语言来描述就是:若 $W$ 是4维时空(可看成4维矢量空间\upref{LSpace}),$T$ 是1维的时间(1维矢量空间),$V$ 是3维空间 (3维矢量空间),那么$V=T\oplus V$ (\autoref{DirSum_def1}~\upref{DirSum}) 就是时空 $W$ 的 $3+1$ 分解.在一般的框架下对时空进行 $3+1$ 的分解称之为\textbf{ADM 形式}(\textbf{ADM formalism}).ADM 形式来源于Arnowitt, Deser and Misner 1962年的工作,并以三人的首字母命名.

考虑4维时空中的空间超曲面 $\bvec X$( $n$ 维空间的超曲面指其 $n-1$ 维的子空间,所以空间超曲面就是我们所处的3维空间,用黑体字母表示暗示着它的每一元素都可用一3维的空间矢量表示),其由3个坐标 $x^i,i=1,2,3$ 所定义,即 
\begin{equation}
\bvec X=\bvec X(x^i)
\end{equation}
在超曲面中的任一点处,都有一基底 $\{\bvec e_i\}$ 与之对应.其中
\begin{equation}
\bvec e_i=\partial_i \bvec X=\pdv{\bvec X}{x^i}
\end{equation}
