% 乔治·布尔(综述)
% license CCBYSA3
% type Wiki

本文根据 CC-BY-SA 协议转载翻译自维基百科\href{https://en.wikipedia.org/wiki/George_Boole#}{相关文章}。

\begin{figure}[ht]
\centering
\includegraphics[width=6cm]{./figures/228c41af0dd53eca.png}
\caption{布尔的肖像,摘自《插图伦敦新闻》,1865年1月21日} \label{fig_George_1}
\end{figure}
乔治·布尔(George Boole,发音:/buːl/ 布尔,1815年11月2日-1864年12月8日)是一位主要自学成才的英国数学家、哲学家和逻辑学家,他的大部分短暂职业生涯都在爱尔兰科克的女王学院担任数学教授。他在微分方程和代数逻辑领域有所研究,最著名的作品是《思维的法则》(The Laws of Thought,1854),该书中包含了布尔代数。布尔逻辑对计算机编程至关重要,并被认为为信息时代奠定了基础。

布尔是一个鞋匠的儿子。他接受了初等教育,并通过各种方式学习了拉丁语和现代语言。16岁时,他开始教授工作以养家糊口。19岁时,他创办了自己的学校,后来在林肯经营了一所寄宿学校。布尔积极参与当地社团活动,并与其他数学家合作。1849年,他被任命为爱尔兰科克女王学院(现为科克大学)首任数学教授,在那里他遇见了未来的妻子玛丽·埃弗雷斯特。他继续参与社会事业,并保持与林肯的联系。1864年,布尔因患肺炎引发的胸膜积液而去世。

布尔一生发表了约50篇文章和几本单独的著作。他的一些关键作品包括关于早期不变性理论的论文和《逻辑的数学分析》(The Mathematical Analysis of Logic),该书引入了符号逻辑。布尔还写了两部系统性的专著:《微分方程论》和《有限差分法则论》。他对线性微分方程理论和有理函数的留数和研究作出了贡献。1847年,布尔发展了布尔代数,这一二进制逻辑的基本概念为逻辑代数传统奠定了基础,并构成了数字电路设计和现代计算机科学的基石。布尔还试图在概率论中发现一种通用方法,重点研究如何通过逻辑连接给定概率的事件来确定其后果概率。

布尔的工作得到了许多学者的拓展,如查尔斯·桑德斯·皮尔士和威廉·斯坦利·杰文斯等。布尔的思想后来得到了实际应用,当克劳德·香农和维克托·谢斯塔科夫利用布尔代数优化机电继电器系统的设计时,推动了现代电子数字计算机的发展。他对数学的贡献为他赢得了各种荣誉,包括皇家学会的首个数学金奖、基思奖章以及都柏林大学和牛津大学的名誉学位。科克大学在2015年庆祝布尔诞辰200周年,强调了他对数字时代的重大影响。
\subsection{早年生活}
\begin{figure}[ht]
\centering
\includegraphics[width=6cm]{./figures/ae46c5bae0377e72.png}
\caption{3号波特门街的房子和学校} \label{fig_George_2}
\end{figure}
\begin{figure}[ht]
\centering
\includegraphics[width=6cm]{./figures/f2dd0a96ad09ef13.png}
\caption{来自房子的纪念 plaque(牌匾)} \label{fig_George_3}
\end{figure}
布尔于1815年出生在英格兰林肯市,父亲是鞋匠约翰·布尔(John Boole Snr,1779-1848),[7]母亲是玛丽·安·乔伊斯(Mary Ann Joyce)。[8]他接受了初等教育,并从父亲那里获得了教学,但由于家庭生意的严重衰退,他没有接受更多的正式和学术教育。[9]林肯的书商威廉·布鲁克(William Brooke)可能帮助他学习拉丁语,他也可能在托马斯·贝恩布里奇(Thomas Bainbridge)学校学习过拉丁语。[2]他在现代语言方面是自学成才的。[10]事实上,当当地报纸刊登了他翻译的拉丁诗歌时,一位学者指控他抄袭,声称他不可能取得这样的成就。16岁时,布尔成为家庭的经济支柱,负责养活父母和三个年幼的弟妹,并在唐卡斯特的海厄姆学校担任初级教师。[11]他还曾在利物浦短暂任教。[1]
\begin{figure}[ht]
\centering
\includegraphics[width=6cm]{./figures/c17cd6d82ded5147.png}
\caption{林肯的灰修道院,曾是机械学会的所在地} \label{fig_George_4}
\end{figure}
布尔参与了位于林肯灰修道院的林肯机械学会,该学会成立于1833年。[2][12]通过该机构,爱德华·布罗姆赫德(Edward Bromhead)认识了约翰·布尔,并帮助乔治·布尔获得了数学书籍。[13]此外,林肯圣斯威辛教堂的乔治·史蒂文斯·迪克森(Rev. George Stevens Dickson)向他赠送了西尔维斯特·弗朗索瓦·拉克鲁瓦(Sylvestre François Lacroix)的微积分教材。[14]由于没有教师指导,他花了很多年才掌握微积分。[1]

19岁时,布尔在林肯成功地创办了自己的学校:自由学校巷(Free School Lane)。[15]四年后,在罗伯特·霍尔去世后,他接管了位于林肯外的瓦丁顿的霍尔学院(Hall's Academy)。[1]1840年,他搬回林肯并经营了一所寄宿学校。布尔立即参与了林肯地理学会,担任委员会成员,并提交了一篇题为《论多神教的起源、发展和趋势,尤其是在古埃及、波斯和现代印度的表现》的论文。[16]

布尔成为了当地的知名人物,并且是约翰·凯(John Kaye)主教的崇拜者。[17]他参与了当地的早期关门运动。[2]1847年,他与埃德蒙·拉肯(Edmund Larken)等人共同成立了一个建筑协会。[18]他还与查尔斯主义者托马斯·库珀(Thomas Cooper)交往,库珀的妻子与他有亲戚关系。[19]

从1838年起,布尔开始与英国学术数学家建立联系,并广泛阅读。他研究了代数,尤其是符号方法的形式,并开始发表研究论文。[1]
\subsection{教授职务与在科克的生活}
\begin{figure}[ht]
\centering
\includegraphics[width=6cm]{./figures/566ee9f7903c9c2e.png}
\caption{} \label{fig_George_5}
\end{figure}
布尔作为数学家的地位得到了认可,1849年,他被任命为爱尔兰科克女王学院(现为科克大学UCC)的首任数学教授。他于1850年在科克遇见了未来的妻子玛丽·埃弗雷斯特,当时她正在探望她的叔叔约翰·里尔(John Ryall),他是希腊语教授。两人于1855年结婚。[20][21]他始终保持与林肯的联系,并与埃德蒙·R·拉肯一起参与了一项旨在减少卖淫的运动。[22]

1861年,布尔参与了爱尔兰女王法庭的一项判决案件,案件的被告是斯莱戈克雷格豪斯的约翰·休伊特·惠特利(John Hewitt Wheatley),他欠款400英镑。根据判决,惠特利的财产和在科克郡马亨(Maghan/Mahon)土地上的权益转交给了布尔。[23]

1863年3月,布尔租下了科克的利奇菲尔德小屋(Litchfield Cottage),他将在那里与妻子玛丽一起生活,直到第二年12月去世。根据契约,该房产被描述为“所有该住所,名为利奇菲尔德小屋的住宅及其附属建筑,后院的花园和围墙田地”。布尔的遗嘱将他在利奇菲尔德小屋租赁中的所有“财产、权益和租期”留给了他的妻子。1865年8月,布尔去世约8个月后,玛丽(此时住在伦敦哈雷街68号)将该房子转让给了科克巴林坦普尔的弗朗西斯·赫德(Francis Heard)先生,他是她陛下第87南科克团的上尉。
\subsection{荣誉与奖项}
\begin{figure}[ht]
\centering
\includegraphics[width=6cm]{./figures/b664232823a3010d.png}
\caption{细节描绘了他最喜欢的圣经章节(内容由他的遗孀建议),即上帝呼召先知撒母耳(撒母耳记上 3:1–10),一个由父母献给上帝的孩子。} \label{fig_George_6}
\end{figure}
\begin{figure}[ht]
\centering
\includegraphics[width=6cm]{./figures/ffca4d379b4e18c2.png}
\caption{窗下的牌匾} \label{fig_George_7}
\end{figure}
1844年,布尔的论文《分析中的一般方法》获得了皇家学会颁发的首个数学金奖。[27]1855年,他获得了爱丁堡皇家学会的基思奖章,[28]并于1857年当选为皇家学会院士(FRS)。[14]他还获得了都柏林大学和牛津大学的法学荣誉学位(LL.D.)。[29]
\subsection{著作}  
布尔的第一篇发表的论文是《分析变换理论研究,特别应用于二阶一般方程的简化》,该论文于1840年2月在《剑桥数学杂志》上发表(第2卷,第8期,第64–73页),这篇论文使他与该杂志的编辑邓肯·法夸森·格雷戈里(Duncan Farquharson Gregory)建立了友谊。[20]他的著作包括约50篇文章和几本单独的出版物。[30][22]

1841年,布尔发表了一篇在早期不变性理论领域具有影响力的论文。[14]他因1844年的论文《分析中的一般方法》获得了皇家学会颁发的奖章。[20]这篇论文是对线性微分方程理论的贡献,研究内容从已发布的常系数情形扩展到变系数的情形。[31]在操作方法上的创新是承认操作可能不满足交换律。[32]1847年,布尔出版了《逻辑的数学分析》,这是他关于符号逻辑的第一部著作。[33]
\subsubsection{微分方程}  
布尔一生中完成了两部关于数学的系统性著作。《微分方程论》于1859年出版,[34] 随后在次年出版了《有限差分法则论》,[35] 这是前者的续集。[20] 布尔去世后不久,托德汉特(Todhunter)重新出版了布尔的这部著作,并加入了布尔的一些修订内容,以及一份原本打算并入第二版的附录。
\subsubsection{分析}  
1857年,布尔发表了论文《超越函数的比较及其在定积分理论中的某些应用》,[36]在其中研究了有理函数的剩余和。除了其他结果外,他证明了现在被称为布尔恒等式的公式:
\[
\mathrm{mes} \left\{ x \in \mathbb{R} \mid \Re \left( \frac{1}{\pi} \sum \frac{a_k}{x - b_k} \right) \geq t \right\} = \frac{\sum a_k}{\pi t}~
\]
其中,\( a_k > 0 \),\( b_k \) 和 \( t > 0 \) 为任意实数。[37]该恒等式的推广在希尔伯特变换的理论中起着重要作用。[37]
\subsubsection{二值逻辑}  
1847年,布尔发表了小册子《逻辑的数学分析》。他后来将其视为自己逻辑体系的一种有缺陷的阐述,并希望《思维法则的研究——其上建立了逻辑与概率的数学理论》能被视为他观点的成熟表述。[20] 与广泛的看法相反,布尔从未打算批评或与亚里士多德逻辑的主要原则相对立。相反,他的目的是将其系统化,为其提供基础,并扩展其适用范围。[38] 布尔最初涉足逻辑学的原因是当时关于量化的争论,争论的双方是支持“谓词量化”理论的威廉·汉密尔顿爵士和布尔的支持者奥古斯都·德摩根(他提出了现在所称的德摩根对偶)。布尔的方法最终比争论双方的理论更具深远影响。[39] 它奠定了最初被称为“逻辑代数”传统的基础。[40]

布尔的众多创新之一是他的整体性引用原则,后来(可能是独立地)被戈特洛布·弗雷格以及支持标准一阶逻辑的逻辑学家所采纳。2003年的一篇文章[41]系统地比较并批判性地评估了亚里士多德逻辑与布尔逻辑;它还揭示了整体性引用在布尔逻辑哲学中的核心地位。

\textbf{1854年对话语领域的定义}  

在每一次对话中,无论是心灵与自身思想的对话,还是个体与他人的交往,都有一个假定或明确的界限,在这个界限内,其讨论的主题是被限定的。最自由的对话是我们使用的词汇被理解为尽可能广泛的应用,在这种情况下,对话的界限与整个宇宙的界限是重合的。然而,通常我们会将自己局限于一个较小的领域。有时,在谈论人类时,我们暗示(而非明确表达限制)我们所讨论的是在某些特定情况下和条件下的人类,例如文明人,或正值壮年的人,或处于某种其他条件或关系中的人。现在,无论我们讨论的对象所处的领域有多广泛,这个领域可以恰当地称为“对话的领域”。此外,这个对话的领域在最严格的意义上是对话的最终主题。[42]

\textbf{逻辑中加法的处理}  

布尔将他的“选择符号”视为一种代数结构。然而,这一概念当时对他来说并不可用:他没有像现代抽象代数中那样对操作的假定(公理化)属性和推导属性的区分标准。[43] 他的工作为集合代数奠定了基础,但这一概念对布尔来说并不是一个熟悉的模型。他的开创性工作遇到了具体的困难,而加法的处理在早期便是一个明显的难题。

布尔将乘法运算替换为“和”一词,将加法运算替换为“或”一词。但在布尔的原始系统中,+ 是一种部分运算:用集合论的语言来说,它仅对应于不相交子集的并集。后来,其他作者改变了这一解释,通常将其理解为“排他或”(exclusive or),或者用集合论术语来说是对称差(symmetric difference);这一变化意味着加法总是有定义的。[40][44]

事实上,还有另一种可能性来推广布尔的原始部分运算,那就是将+理解为非排他或(non-exclusive or)。[43] 处理这一歧义是该理论的早期问题之一,反映了布尔环和布尔代数的现代用法(它们只是同一类型结构的不同方面)。布尔和杰文斯在1863年就这一问题进行了争论,具体形式是正确评估x + x的结果。杰文斯主张结果应为x,这对于“或”运算中的加法是正确的。布尔则认为该结果应视为未定义的。他反对结果为0的看法,因为对于排他或运算来说,0是正确的,他认为方程x + x = 0暗示x = 0,这与普通代数的类比是错误的。[14]
\subsubsection{概率论}  
《思维法则》的第二部分包含了试图发现概率中的一般方法的对应尝试。其目标是算法性的:根据任何事件系统的给定概率,逻辑地确定与这些事件相关联的其他事件的后续概率。[45][20]
\subsection{死亡}
\begin{figure}[ht]
\centering
\includegraphics[width=6cm]{./figures/2f6c152e9947b549.png}
\caption{布尔的墓碑位于爱尔兰科克的布莱克罗克。} \label{fig_George_9}
\end{figure}
1864年11月末,布尔在大雨中步行从他位于巴林坦普尔的利奇菲尔德小屋(Lichfield Cottage)家中前往大学,路程约为三英里,并穿着湿衣服讲课。[46] 他很快生病,患上了肺炎。由于他的妻子认为疗法应与病因相似,她用湿毛毯包裹他——因为湿气引发了他的病情。[47][48][49] 布尔的病情恶化,最终于1864年12月8日因由发热引起的胸膜积液去世。[50]

他被安葬在科克市郊布莱克罗克(Blackrock)圣迈克尔教堂(St Michael's Church)路上的爱尔兰教会墓地内。在相邻的教堂内有一块纪念牌匾。[51]
\subsection{遗产}
\begin{figure}[ht]
\centering
\includegraphics[width=6cm]{./figures/ae88f29336fbfa3e.png}
\caption{布尔的半身像,位于科克大学(University College Cork)} \label{fig_George_8}
\end{figure}
布尔是布尔代数(Boolean algebra)这一代数分支的名称来源,并且布尔陨石坑(Boole)也以他的名字命名。布尔(Bool)是许多编程语言中布尔数据类型的关键字,尽管如Pascal和Java等语言使用的是完整名称Boolean。[52] 科克大学(University College Cork)内的图书馆、地下讲座剧院和布尔信息学研究中心(Boole Centre for Research in Informatics)都以他的名字命名。伯克郡布拉克内尔(Bracknell)的一条名为布尔高地(Boole Heights)的道路也以他命名。
\subsubsection{19世纪的发展}  
布尔的工作被许多学者进一步扩展和完善,其中包括威廉·斯坦利·杰文斯(William Stanley Jevons),他还在《大英百科全书》中撰写了关于布尔的文章。奥古斯都·德·摩根(Augustus De Morgan)曾研究关系逻辑,而查尔斯·桑德斯·皮尔斯(Charles Sanders Peirce)则在1870年代将他的工作与布尔的工作结合起来。[54] 其他重要人物包括普拉东·谢尔盖耶维奇·波列茨基(Platon Sergeevich Poretskii)和威廉·厄内斯特·约翰逊(William Ernest Johnson)。在命题演算等价命题的布尔代数结构的构想上,归功于休·麦克考尔(Hugh MacColl)(1877),这一工作在15年后由约翰逊进行了综述。[54] 这些发展的综述文章由恩斯特·施罗德(Ernst Schröder)、路易·库图拉特(Louis Couturat)和克拉伦斯·欧文·刘易斯(Clarence Irving Lewis)发布。
\subsubsection{20世纪的发展} 
\begin{figure}[ht]
\centering
\includegraphics[width=8cm]{./figures/4f109b158e1e352b.png}
\caption{在现代符号中,基本命题 p 和 q 上的自由布尔代数在哈塞图(Hasse diagram)中的表示。布尔组合形成了16个不同的命题,线条显示了哪些命题在逻辑上是相关的。} \label{fig_George_10}
\end{figure}
1921年,经济学家约翰·梅纳德·凯恩斯出版了关于概率论的著作《概率论Treatise of Probability》。凯恩斯认为布尔在其独立性定义上犯了一个根本性的错误,这使得他的很多分析失效。[55] 在他的著作《最后的挑战问题》中,David Miller 提供了一种与布尔系统一致的通用方法,并尝试解决凯恩斯及其他人早期所识别的问题。西奥多·海尔佩林则较早地证明了布尔在已解答问题中使用的是正确的数学独立性定义。[56]

布尔的工作以及后来的逻辑学家最初似乎没有实际的工程应用。克劳德·香农曾在密歇根大学上过一门哲学课,课程内容介绍了布尔的研究。香农认识到布尔的工作可以为现实世界中的机制和过程提供基础,因此非常具有相关性。1937年,香农在麻省理工学院写了一篇硕士论文,展示了布尔代数如何优化当时用于电话路由交换机的电机继电器系统设计。他还证明了继电器电路能够解决布尔代数问题。利用电气开关的特性处理逻辑是所有现代电子数字计算机的基本概念。莫斯科国立大学的维克多·谢斯塔科夫(1907-1987)早于香农在1935年就提出了一种基于布尔逻辑的电气开关理论,并得到了苏联逻辑学家和数学家如索菲亚·雅诺夫斯卡娅、加泽-拉波波特、罗兰·多布鲁欣、卢帕诺夫、梅德韦杰夫和乌斯宾斯基的证词。但谢斯塔科夫的研究成果直到1941年(以俄文发布)才首次公开。因此,布尔代数成为了实际数字电路设计的基础;通过香农和谢斯塔科夫,布尔为信息时代提供了理论基础。[57]
\subsubsection{21世纪的发展} 
2015年是布尔诞辰200周年。为了纪念这一双百年纪念年,科克大学(University College Cork)与世界各地的布尔崇拜者一起庆祝他的生平和遗产。

UCC的“乔治·布尔200”[58]项目包括了一系列活动、学生外展活动和学术会议,探讨布尔遗产在数字时代的影响,其中还包括2014年出版的德斯蒙德·麦克海尔(Desmond MacHale)1985年传记《乔治·布尔的生活与工作:数字时代的前奏》的新版。[59]
\begin{figure}[ht]
\centering
\includegraphics[width=6cm]{./figures/42bd2bc3389a974d.png}
\caption{2017年,科克大学修复后的格伦维尔广场5号。} \label{fig_George_12}
\end{figure}
搜索引擎Google在2015年11月2日为纪念布尔诞辰200周年,在其Google涂鸦中进行了代数重构。[4]

2022年9月,乔治·布尔作为教师的雕像在布尔的故乡林肯的林肯中央火车站揭幕。
\subsection{观点}  
布尔的观点通过四次公开演讲表达:分别是《艾萨克·牛顿爵士的天才》、《休闲的正确使用》、《科学的主张》和《知识文化的社会方面》[20]。其中第一篇是1835年,查尔斯·安德森-佩尔汉姆(第一代亚博罗伯爵)将牛顿的半身像赠送给林肯机械学会时发表的[60]。第二篇是1847年,庆祝林肯成功推动早闭运动的成果,该运动由布兰斯顿大厅的亚历山大·莱斯利-梅尔维尔领导[61]。《科学的主张》则是在1851年在科克的皇后学院发表的[62]。最后,《知识文化的社会方面》也是在1855年,布尔向科克的居维尔学会发表的[63]。
\begin{figure}[ht]
\centering
\includegraphics[width=6cm]{./figures/0ff61672830e467f.png}
\caption{布尔的铜像位于林肯中央火车站。该设计由雕塑家安东尼·杜福特(Antony Dufort)创作,部分资金来自赫斯兰信托基金(Heslam Trust)。} \label{fig_George_11}
\end{figure}
尽管他的传记作者德斯·麦克黑尔(Des MacHale)将布尔描述为“不可知的自然神论者”[64][65],布尔广泛阅读了基督教神学。将他对数学和神学的兴趣结合起来,他将基督教三位一体(父、子、圣灵)与空间的三维进行比较,并且被希伯来对上帝作为绝对统一的观念所吸引。布尔曾考虑过皈依犹太教,但最后据说他选择了一神教(Unitarianism)[引用?]。布尔开始反对他所认为的“自负”的怀疑主义,而倾向于相信“至高智能的原因”[66]。他还宣称:“我坚信,这是为了实现神圣心智的目的。”[67][68]此外,他还说:“从周围设计的丰富证据中推断出智能原因的存在,从我们自身本性中的构造和道德规定中升华出世界的道德治理者的观念——这些,尽管只是理解力和知识材料有限的理解者的微弱步伐,却比那些雄心勃勃、企图在自然宗教的基础上达到无法达到的确定性更有意义。正如这些是最古老的基础,它们至今仍是最坚实的基础,启示被排除在外,信仰认为这个世界的进程不是被任由偶然和不可避免的命运所支配。”[69][70]

布尔的妻子玛丽·埃弗雷斯特·布尔(Mary Everest Boole)后来声称,有两个因素对布尔产生了影响:一种受到犹太思想调和的普遍神秘主义和印度逻辑。[71] 玛丽·布尔表示,布尔的少年时代经历了一次神秘的体验,这为他一生的工作奠定了基础:

“我的丈夫告诉我,当他十七岁时,有一个突然的念头闪现,这成为了他所有未来发现的基础。这是一种对心理状态的深刻洞察,揭示了一个大脑最容易积累知识的条件……几年来,他曾认为自己深信‘圣经’的整体真实性,甚至打算成为英格兰教会的牧师。但在林肯一位学者犹太人的帮助下,他发现了那次突然启示的真正性质。这个发现是:人的大脑通过某种机制运作,这种机制‘正常运作朝向一元论’。”[72]

在《思维法则》第13章中,布尔引用了巴鲁赫·斯宾诺莎和塞缪尔·克拉克的命题作为例子。该书中有一些关于逻辑与宗教关系的评论,但这些评论简略且晦涩。[73] 布尔显然对这本书仅作为数学工具集的接受度感到不安:

“乔治后来得知,令他非常高兴的是,莱布尼茨(牛顿的同时代人)也持有与布尔相同的逻辑基础观念。德摩根当然理解这个公式的真正含义;他一直是布尔的合作者。赫伯特·斯宾塞、乔韦特和罗伯特·莱斯利·埃利斯,我确信他们也理解;还有一些其他人,但几乎所有的逻辑学家和数学家忽视了[953]这本书的一个声明:即这本书旨在揭示人类思维的本质;他们完全把这个公式当作一种新奇的方法,来将大量关于外部事实的证据归纳成逻辑秩序。”[72]

玛丽·布尔声称,通过她的叔叔乔治·埃弗雷斯特,印度思想,特别是印度逻辑,对乔治·布尔、奥古斯都·德摩根和查尔斯·巴贝奇产生了深远的影响:[74]

“想想印度化对像巴贝奇、德摩根和乔治·布尔这三位人物的影响,会对1830至1865年间的数学氛围产生什么样的作用。它在生成矢量分析和用于现代物理科学研究的数学方面发挥了多大作用?”[72]

布尔坚持认为:

“在概率论中,任何解决问题的一般方法都不能建立,如果它没有明确地认识到,不仅是科学的特殊数字基础,而且还必须认识到那些普遍的思维法则,这些法则是所有推理的基础,无论它们的本质如何,至少在形式上是数学的。”[75]
\subsection{家庭}  
1855年,布尔与玛丽·埃佛雷斯特(乔治·埃佛雷斯特的侄女)结婚,玛丽后来写了几本关于丈夫理论的教育著作。

布尔夫妇育有五个女儿:
\begin{itemize}
\item 玛丽·艾伦(1856–1908),她嫁给了数学家和作家查尔斯·霍华德·辛顿,并育有四个孩子。她的丈夫突然去世后,玛丽·艾伦于1908年5月在华盛顿特区自杀。
\item 乔治·辛顿(1882–1943),矿业工程师和植物学家
\item H. E. 辛顿(1912–1977),昆虫学家
\item 杰弗里·辛顿(1947年出生),认知心理学家和计算机科学家,2024年诺贝尔物理学奖得主,因在人工神经网络方面的工作而闻名
\item 埃里克·辛顿(1884年出生)
\item 威廉·辛顿(1886–1909)
\item 塞巴斯蒂安·辛顿(1887–1923),律师,丛林健身器的发明者
\item 让·辛顿(结婚后名罗斯纳)(1917–2002),和平活动家
\item 威廉·H·辛顿(1919–2004),他在1930年代和40年代访问了中国,并写了关于共产主义土地改革的有影响力的著作
\item 琼·辛顿(1921–2010),参与曼哈顿计划,并自1948年起一直生活在中国,直至2010年6月8日去世;她嫁给了西德·恩格斯特
\item 玛格丽特(1858–1935),嫁给了艺术家爱德华·英格拉姆·泰勒  
\item 她的大儿子杰弗里·英格拉姆·泰勒成为了数学家,并成为皇家学会会员
\item 她的小儿子朱利安·泰勒是一位外科教授
\item 阿莉西亚(1860–1940),对四维几何学做出了重要贡献  
\item 她的儿子伦纳德·斯托特,一位医学博士和结核病先锋,发明了便携式X光机、气胸设备和基于球面坐标的导航系统
\item 露西·埃佛雷斯特(1862–1904),是英国第一位女性化学教授
\item 艾瑟尔·莉莲(1864–1960),嫁给了波兰科学家和革命家威尔弗里德·迈克尔·沃伊尼奇,并著有小说《蚂蚁》。
\end{itemize}
\subsection{参见}
