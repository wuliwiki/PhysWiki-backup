% 正项级数的收敛性判别

\pentry{数项级数\upref{Series}}

给定一个正项级数, 也就是各项都为正实数的级数, 该怎么判别它是否收敛? 本条目将列举一些常用的判据.

\subsection{比较原理}

第一类收敛性判别法是直接将给定的级数与一个收敛级数来比较大小. 

\begin{theorem}{比较原理}
设$\{a_n\},\{b_n\}$是正数序列, 从某项开始,$a_n\leq b_n$恒成立. 那么由级数$\sum_{n=1}^\infty b_n$收敛推出级数$\sum_{n=1}^\infty a_n$收敛, 而由级数$\sum_{n=1}^\infty a_n$发散推出级数$\sum_{n=1}^\infty b_n$发散.
\end{theorem}
\begin{exercise}{}
利用级数收敛的柯西判据来证明这个比较原理. 进一步证明: 如果$\{a_n\}$是复数序列, 而$\sum_{n=1}^\infty|a_n|$是收敛级数, 那么$\sum_{n=1}^\infty a_n$也是收敛级数.
\end{exercise}

\begin{example}{}
对于$p>1$, 不论$x$是何实数, 级数
$$
\sum_{n=1}^\infty\frac{e^{inx}}{n^p}
$$
都是收敛的, 因为它的各项绝对值组成的级数是收敛的$p$-级数.

以后将会算出: 如果$x\neq 2k\pi$, 那么
$$
\sum_{n=1}^\infty\frac{e^{inx}}{n}=\ln|1-e^{ix}|+i\arg(1-e^{ix}).
$$
\end{example}

\subsection{收敛速度比较法}