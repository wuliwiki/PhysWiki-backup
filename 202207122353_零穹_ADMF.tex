% ADM形式
% ADM形式|时空3+1分解
\pentry{爱因斯坦求和约定\upref{EinSum}}
\begin{issues}
\issueTODO
\end{issues}

将4维时空分解成1维的时间和3维的空间称之为时空的\textbf{3+1分解},用数学的语言来描述就是:若 $W$ 是4维时空(可看成4维矢量空间\upref{LSpace}),$T$ 是1维的时间(1维矢量空间),$V$ 是3维空间 (3维矢量空间),那么$V=T\oplus V$ (\autoref{DirSum_def1}~\upref{DirSum}) 就是时空 $W$ 的 $3+1$ 分解.在一般的框架下对时空进行 $3+1$ 的分解称之为\textbf{ADM 形式}(\textbf{ADM formalism}).ADM 形式来源于Arnowitt, Deser and Misner 1962年的工作,并以三人的首字母命名.
\subsection{时空3+1分解}
考虑4维时空中的空间超曲面 $X^\mu$( $n$ 维空间的超曲面指其 $n-1$ 维的子空间,所以空间超曲面就是我们所处的3维空间,上标是抽象指标\upref{AbsInd},暗示着它的每一元素都与某一矢量相对应),其由3个坐标 $x^i,i=1,2,3$ 所定义,即 
\begin{equation}
X^\mu=X^\mu(x^i)
\end{equation}
在超曲面中的任一点处,都有一基底 $\{e^\mu_i\}$ 与之对应.其中
\begin{equation}
e^\mu_i=\partial_i  X^\mu=\pdv{ X^\mu}{x^i}
\end{equation}
$e^\mu_i$ 方向的坐标轴当然只有对应坐标 $x^i$ 变化,由偏导数的定义\upref{ParDer},上式的几何图像很直白.

而垂直于超曲面的单位法矢量 $n^\mu$ 满足(采用爱因斯坦求和约定\upref{EinSum})
\begin{equation}\label{ADMF_eq1}
g_{\mu\nu} e_i^\mu n^\nu=0,\quad g_{\mu\nu}n^\mu n^\nu=-1
\end{equation}
其中 $e_i^\mu,n^\nu$ 分别是四矢量 $ e_i,n$ 对应 $\mu,\nu$ 坐标轴的坐标分量,而 $g_{\mu\nu}$ 为时空的度规(亦即度量,时空的度量显然是闵可夫斯基度量(\autoref{EFSp_def1}~\upref{EFSp})).\autoref{ADMF_eq1} 中,第一式表明 $n^\mu$ 垂直于超曲面,第二式表明 $n^\mu$ 是单位矢.

现在,以一连续的方式定义超曲面,从而获得一族超曲面 
\begin{equation}
\{{X^\mu}= X^\mu(x^i,t)|t\in\mathbb R\}
\end{equation}
定义时间演化4矢量 $N^\mu$
\begin{equation}
N^\mu:=\dot{X}^\mu=\partial_t  X^\mu(x^i,t)
\end{equation}

由定义,这一矢量连接着相邻超曲面上具有统一坐标 $x^i$ 的点.将 $N^\mu$ 在基底 $\{n^\mu,e^\mu_i\}$下 进行展开,其对应分量定义如下
\begin{equation}\label{ADMF_eq2}
N^\mu:=Nn^\mu+N^ie^\mu_i
\end{equation}
其中 $N$ 称为\textbf{时移函数}(\textbf{lapse function}),$N^i$ 称为\textbf{位移函数}(\textbf{shift function}).其几何解释如\autoref{ADMF_fig1} :$N\dd t$ 是相邻两超曲面的间的固有时,$N^i\dd t$ 是上超曲面 $X^\mu(x^i,t+\dd t)$ 上空间坐标为 $x^i$ 的点,到上超曲面对应于下超曲面 $X^\mu(x^i,t)$ 上空间坐标为 $x^i$ 的点在法矢量方向上的对应点的距离.
\begin{figure}[ht]
\centering
\includegraphics[width=10cm]{./figures/ADMF_1.pdf}
\caption{$N^\mu$ 的几何解释}} \label{ADMF_fig1}
\end{figure}

现在,跟随着任一场的运动,可以将场投影到垂直和平行超曲面的方向,包括令人感兴趣的度规场.但是由于度规定义了所谓的“垂直”,所以它的两个投影是不重要的,这两个投影是:
\begin{equation}
g_{\perp\perp}:=g_{\mu\nu}n^{\mu}n^{\nu},\quad g_{\perp i}:=g_{\mu\nu}n^{\mu}e^{\nu}_i
\end{equation}
重要的仅仅是
\begin{equation}\label{ADMF_eq3}
g_{ij}:=\gamma_{ij}=g_{\mu\nu}e^{\mu}_ie^{\nu}_j
\end{equation}
它定义了由超曲面诱导的3维度规 $\gamma_{ij}$.
\subsection{度规的ADM分解}
度规的ADM分解是在基底 $\{N^{\mu},e^{\mu}_i\}$ 下进行的 (由于 $N^{\mu}$ 是时间轴):
\begin{equation}
\left\{\begin{aligned}
&g_{00}:=g_{\mu\nu}N^{\mu}N^{\nu}=\gamma_{ij}N^iN^j-N^2\\
&g_{0i}:=g_{\mu\nu}N^{\nu}e^{\nu}_i=N_i\\
&g_{ij}:=g_{\mu\nu}e^{\mu}_ie^{\nu}_i=\gamma_{ij}
\end{aligned}\right.
\end{equation}
上式的证明只需将\autoref{ADMF_eq2} 带入,并注意 \autoref{ADMF_eq3} 和$n^\mu$ 与 $e^{\mu}_i$ 垂直.

所以时空间隔 $\dd s^2$ 为
\begin{equation}
\begin{aligned}
\dd s^2&=g_{\mu\nu}\dd X^\mu\dd X^\nu=g_{\mu\nu}\qty(\pdv{X^\mu}{t}\dd t+\pdv{X^\mu}{x^i}\dd x^i)\qty(\pdv{X^\nu}{t}\dd t+\pdv{X^\nu}{x^j}\dd x^j)\\
&=g_{\mu\nu}\qty(N^\mu N^\nu\dd t^2+N^\mu e^\nu_j\dd t\dd x^j+e^\mu_iN^\nu\dd t\dd x^i+e^{\mu}_ie^{\nu}_j\dd {x^i}\dd x^j)\\
&=g_{00}dt^2+N_j\dd t\dd x^j+N_i\dd t\dd x^i+\gamma_{ij}\dd x^i\dd x^j\\
&=-N^2\dd t^2+\gamma_{ij}N^iN^j\dd t^2+\gamma_{ij}\dd x^i\dd x^j\\
&=-N^2\dd t^2+\gamma_{ij}(\dd x^i+N^i\dd t)(\dd x^j+N^j\dd t)
\end{aligned}
\end{equation}

