% 百科创作指导

\subsection{预备知识}
\begin{itemize}
\item 我们重视百科的自洽性, 所以几乎所有词条都需要有预备知识(用 \verb|\pentry| 命令实现), 预备知识相当于 “必备知识”, 如果里面的内容不掌握, 读者阅读词条内容就会遇到困难.
\item 注意百科词条目录并不按照建议阅读的顺序来排序而是按照话题, 所以不能默认读者以已经读过前面的词条.
\item \href{http://wuli.wiki/tree/}{词条目录树}页面将自动按照 “预备知识” 生成. 读者可以把任意节点作为目标生成该词条的预备知识树, 也可以将任意节点作为起点逆向生成.
\item 注意 “预备知识” 是递归的,意味着你可以认为读者已经掌握 “预备知识” 词条中的 “预备知识”.
\item 一些拓展或者选读的相关词条不需要作为预备知识, 例如 “详见……”.
\item 一般来说我们假设读者具有普通高中生数理水平. 任何超出该水平的内容都需要在 “预备知识” 中有所体现. 如果百科存在低于该水平的内容, 也需要加入预备知识.
\item 在添加预备知识时, 先浏览一下里面的内容, 确保它包含当前词条所需内容.
\item 如果你写的内容在百科中找不到预备知识所需的内容, 应该把词条标记为 “缺少预备知识” (在 \verb|issues| 环境中插入 \verb|\issueMissDepend|), 并用注释说明需要什么内容.
\end{itemize}
