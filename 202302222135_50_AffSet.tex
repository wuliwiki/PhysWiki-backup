% 仿射集
% keys 仿射  直线

有一个集合$C \subseteq R^n$,当过$C$中任意不同的两点的直线仍然在$C$中时,$C$为\textbf{仿射集}(Affine set),此时,对于任意$x_1,x_2 \in C$,$\theta \in R$,有$\theta x_1+(1-\theta)x_2 \in C$.换句话说,$C$中任意两点的线性组合仍然在$C$内,线性组合的系数之和为$1$.

\begin{figure}[ht]
\centering
\includegraphics[width=12cm]{./figures/AffSet_1.png}
\caption{仿射集定义示意图} \label{AffSet_fig1}
\end{figure}

图$1$表示的是一条直线穿过$x_1,x_2$两点,也即$x_1,x_2$两点的仿射组合$\theta x_1+(1-\theta)x_2$的全体形成了该直线。根据仿射集的定义,该条直线就是一个仿射集。当$0<\theta<1$,形成图中直线加粗的部分;反之,形成直线上细线表示的部分。

仿射集上述定义可以推广到多点的情形。我们把$\theta_1x_1+\theta_2x_2+...+\theta_kx_k$称为点$x_1,x_2,...,x_k$的\textbf{仿射组合}(Affine combination),其中$\theta_1+\theta_2+...+\theta_k=1$。

基于仿射组合的概念,仿射集有性质:若集合$C$是仿射集,有点$x_1,x_2,...,x_k \in C$,且$\theta_1+\theta_2+...+\theta_k=1$,则$\theta_1x_1+\theta_2x_2+...+\theta_kx_k$也在集合$C$内。