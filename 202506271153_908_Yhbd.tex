% 约翰·巴丁(综述)
% license CCBYSA3
% type Wiki

本文根据 CC-BY-SA 协议转载翻译自维基百科 \href{https://en.wikipedia.org/wiki/John_Bardeen}{相关文章}。

\begin{figure}[ht]
\centering
\includegraphics[width=6cm]{./figures/c261931e347b9e06.png}
\caption{巴丁于1956年} \label{fig_Yhbd_1}
\end{figure}
约翰·巴丁(John Bardeen,1908年5月23日-1991年1月30日)\(^\text{[1]}\)是一位美国物理学家。他是唯一一位两次获得诺贝尔物理学奖的人:第一次是在1956年,与威廉·肖克利和沃尔特·布拉顿因共同发明晶体管而获奖;第二次是在1972年,与利昂·库珀和约翰·施里弗因提出超导性的微观理论——BCS理论——而再次获奖。\(^\text{[4][5]}\)

巴丁出生并成长于威斯康星州,在威斯康星大学获得电气工程的学士和硕士学位,随后在普林斯顿大学取得物理学博士学位。在参加第二次世界大战后,他曾在贝尔实验室从事研究工作,之后担任伊利诺伊大学教授。

晶体管的发明彻底改变了电子工业,使得从电话到计算机几乎所有现代电子设备的出现成为可能,也由此开启了信息时代。而巴丁在超导性领域的研究成果——为他赢得第二次诺贝尔奖——则被广泛应用于核磁共振光谱(NMR)、医学磁共振成像(MRI)以及超导量子电路等领域。

巴丁是仅有的三位在同一领域获得两次诺贝尔奖的人之一(另外两位是弗雷德里克·桑格和卡尔·巴里·夏普利斯,均为化学奖获得者),也是全球仅有的五位两度获诺贝尔奖的人之一。1990年,《生活》杂志将他评为“20世纪最具影响力的一百位美国人”之一。\(^\text{[6]}\)
\subsection{教育与早年生活}
巴丁于1908年5月23日出生在威斯康星州麦迪逊市。\(^\text{[7]}\)他是查尔斯·巴丁的儿子,后者是威斯康星大学医学院的首任院长。

巴丁就读于麦迪逊的威斯康星大学附属中学。他于1923年毕业,时年仅15岁。\(^\text{[7]}\)他本可以更早毕业,但由于他在另一所高中修读课程以及母亲去世,毕业时间被推迟了。1923年,巴丁进入威斯康星大学。在大学期间,他加入了齐达赛兄弟会,并靠打台球赚取部分会费。他还被接纳为工程荣誉学会Tau Beta Pi的成员。由于不想像父亲那样走学术路线,巴丁选择了工程专业。他还认为工程专业具有良好的就业前景。\(^\text{[8]}\)
