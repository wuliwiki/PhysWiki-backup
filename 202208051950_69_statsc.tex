% 统计物理·微观与宏观之间的桥梁
% 统计物理|科普|微观

物理学史上,人们对大自然的探索有两条主线.其中之一,是对物质的结构不断“细分”,通过对微观世界的研究来探索最根本的自然法则.这个思想源自于古希腊时期德谟克里特的“原子论”,随后被发扬光大,对物理学、化学等学科都产生了深远的影响.而在十九世纪相对论与量子力学的革命之后,伴随着加速器技术的发展,人们发现了重子、介子、轻子等形形色色的粒子,在许许多多物理学家的共同努力下,“粒子物理标准模型”诞生,人们对微观世界的认识达到了一个崭新的高度.这一条探索路线上发生过无数次物理学的“革命”,带来了各种各样的惊喜.