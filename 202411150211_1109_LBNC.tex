% 戈特弗里德·莱布尼茨(综述)
% license CCBYSA3
% type Wiki

本文根据 CC-BY-SA 协议转载翻译自维基百科\href{https://en.wikipedia.org/wiki/Gottfried_Wilhelm_Leibniz}{相关文章}。

“戈特弗里德·威廉·莱布尼茨或莱布尼茨[a](1646年7月1日 [旧历6月21日] – 1716年11月14日)是一位德国博学家,活跃于数学家、哲学家、科学家和外交官等多个身份,与艾萨克·牛顿爵士一起被认为是微积分的发明者,此外还在二进制算术和统计学等数学分支做出了许多贡献。由于其在不同领域的知识和技能,以及随着工业革命的到来和专业化劳动的普及,像他这样的人变得越来越少,莱布尼茨被称为“最后的通才”。[15]他在哲学史和数学史上都是一个重要人物。他撰写了关于哲学、神学、伦理学、政治学、法律、历史、语文学、游戏、音乐和其他学科的作品。莱布尼茨还对物理学和技术做出了重大贡献,并预见了后来在概率论、生物学、医学、地质学、心理学、语言学和计算机科学中出现的概念。

此外,他在沃尔芬比特尔的赫尔佐格·奥古斯特图书馆工作时,设计了一种目录系统,为欧洲许多大型图书馆提供了指导。[16][17] 莱布尼茨在众多领域的贡献散见于各种学术期刊、成千上万封信件以及未发表的手稿中。他用多种语言写作,主要是拉丁语、法语和德语。[18][b]

作为一位哲学家,他是17世纪理性主义和唯心主义的主要代表之一。作为数学家,他的主要成就是独立于艾萨克·牛顿的同时代发展而提出了微积分的核心思想。[20] 数学家们一直偏爱莱布尼茨的符号法,因为它被认为是微积分的惯用且更为精确的表达方式。[21][22][23]

在20世纪,莱布尼茨关于连续性法则和超越同质性法则的概念通过非标准分析得到了一致的数学表述。他也是机械计算器领域的先驱。在为帕斯卡计算器添加自动乘法和除法功能的过程中,他于1685年首次描述了齿轮计算器,并发明了莱布尼茨轮,后来被用于首台大规模生产的机械计算器——算筹器中。

在哲学和神学方面,莱布尼茨以其乐观主义最为著名,即他得出的结论是,我们的世界在某种限定意义上是上帝所能创造的最好的世界,这一观点有时被其他思想家讽刺,例如伏尔泰在其讽刺小说《老实人》中就调侃了这一观点。莱布尼茨与勒内·笛卡尔和巴鲁赫·斯宾诺莎一起,是三位具有影响力的早期现代理性主义者。他的哲学还吸收了经院哲学传统的元素,尤其是假设通过从第一原理或先验定义推理,可以获得对现实的实质性知识。莱布尼茨的工作预示了现代逻辑的发展,并且仍然影响着当代分析哲学,例如采用“可能世界”这一术语来定义模态概念。
\subsection{传记}
\subsubsection{早年生活}
戈特弗里德·莱布尼茨于1646年7月1日[旧历:6月21日]出生在萨克森的莱比锡,父母是弗里德里希·莱布尼茨(1597–1652)和卡塔琳娜·施穆克(1621–1664)。[25] 他在两天后在莱比锡的圣尼古拉教堂受洗,他的教父是路德教神学家马丁·盖尔。[26] 在他六岁时,父亲去世,莱布尼茨由母亲抚养长大。[27]

莱布尼茨的父亲曾是莱比锡大学的道德哲学教授,也曾担任哲学系主任。莱布尼茨继承了父亲的私人图书馆。从七岁起,他就被允许自由使用这些书籍,正是在父亲去世后不久。他的学校教育主要集中在少量经典权威著作的学习上,但父亲的图书馆使他能够研究各种高级的哲学和神学著作——这些书籍本来他要到大学时才能接触到。[28] 父亲的图书馆大部分是用拉丁文写成的,这也帮助他在12岁时就掌握了拉丁语。13岁时,他在学校的一次特别活动中用拉丁文在一个早晨写了300行六步格诗。[29]

1661年4月,14岁的莱布尼茨进入了父亲曾任职的大学。[30][8][31] 在那里,他受到包括雅各布·托马修斯在内的导师的指导,托马修斯曾是弗里德里希的学生。莱布尼茨于1662年12月完成了哲学学士学位。1663年6月9日[旧历:5月30日],他为自己的论文《个体化原则的形而上学争论》(Disputatio Metaphysica de Principio Individui)进行了答辩,[32] 该论文探讨了个体化原则,提出了单子实在论的早期版本。1664年2月7日,莱布尼茨获得哲学硕士学位。1664年12月,他发表并为论文《从法律收集的哲学问题样本》(Specimen Quaestionum Philosophicarum ex Jure collectarum)进行了答辩,[32] 主张哲学与法律之间的理论和教学关系。经过一年的法律学习,他于1665年9月28日获得了法学学士学位。[33] 他的论文题为《论条件》(De conditionibus)。[32]

1666年初,年仅19岁的莱布尼茨写了他的第一本书《组合艺术》(De Arte Combinatoria),其第一部分也是他的哲学资格论文,并于1666年3月进行了答辩。[32][34] 《组合艺术》受拉蒙·卢尔的《大艺术》(Ars Magna)启发,包含了一个基于运动论证的几何形式的上帝存在性证明。

他的下一个目标是获得法律执业许可证和法学博士学位,这通常需要三年的学习时间。1666年,莱比锡大学拒绝了莱布尼茨的博士申请,并拒绝授予他法学博士学位,这很可能是由于他年纪尚轻。[35][36] 随后,莱布尼茨离开了莱比锡。[37]

莱布尼茨接着在阿尔特多夫大学注册,并很快提交了一篇论文,他很可能早在莱比锡时就已经在研究这篇论文。[38] 他的论文题为《法律中疑难案例的就职争论》(Disputatio Inauguralis de Casibus Perplexis in Jure)。[32] 莱布尼茨于1666年11月获得了法律执业许可证和法学博士学位。接下来,他拒绝了阿尔特多夫大学提供的教职,表示“我的想法完全朝向另一个方向”。[39]

成年后,莱布尼茨经常自称为“戈特弗里德·冯·莱布尼茨”。许多他去世后出版的作品在标题页上将他的名字标为“冯·莱布尼茨男爵G. W.”(Freiherr G. W. von Leibniz)。然而,没有发现任何当代政府的文件表明他被授予任何形式的贵族身份。[40]
\subsubsection{1666–1676}
\begin{figure}[ht]
\centering
\includegraphics[width=6cm]{./figures/f6003f86e6fa77fd.png}
\caption{戈特弗里德·威廉·莱布尼茨} \label{fig_LBNC_1}
\end{figure}
莱布尼茨的第一个职位是纽伦堡一家炼金术协会的薪酬秘书。[41] 当时他对该领域了解甚少,但表现得学识渊博。他很快遇到了约翰·克里斯蒂安·冯·博伊内堡(1622–1672),后者是美因茨选帝侯约翰·菲利普·冯·舍恩博恩的前任首席大臣。[42] 冯·博伊内堡雇佣莱布尼茨作为助手,不久之后他与选帝侯和解,并将莱布尼茨介绍给选帝侯。莱布尼茨随即将一篇法律论文献给选帝侯,希望借此获得职位。这一策略奏效了;选帝侯要求莱布尼茨协助重新起草选侯国的法律法规。[43] 1669年,莱布尼茨被任命为上诉法院的评审官。尽管冯·博伊内堡于1672年底去世,莱布尼茨仍然受雇于他的遗孀,直到1674年被解雇。[44]

冯·博伊内堡极大地提升了莱布尼茨的声誉,后者的备忘录和信件开始引起人们的关注。在莱布尼茨为选帝侯服务之后,很快就承担了一个外交角色。他以一个虚构的波兰贵族的笔名发表了一篇文章,主张(但未成功)支持德国候选人争夺波兰王位。在莱布尼茨成年期间,欧洲地缘政治的主要力量是法国国王路易十四的野心,由法国的军事和经济力量支持。同时,三十年战争使德语区的欧洲疲惫不堪,分裂且经济落后。莱布尼茨提出通过以下方式来保护德语区欧洲:将法国的注意力引向埃及,以此作为最终征服荷兰东印度群岛的跳板。作为交换,法国同意不干扰德国和荷兰。该计划获得了选帝侯的谨慎支持。1672年,法国政府邀请莱布尼茨前往巴黎讨论,[45] 但这一计划很快因法荷战争的爆发而失去意义。拿破仑1798年对埃及的失败入侵可以视为莱布尼茨计划的一种不自觉的迟来实施,而此时欧洲对东半球的殖民统治权已经从荷兰转移到了英国。

因此,莱布尼茨于1672年前往巴黎。到达后不久,他遇到了荷兰物理学家兼数学家克里斯蒂安·惠更斯,并意识到自己在数学和物理方面的知识还不够全面。在惠更斯的指导下,他开始了自学,并很快在这两个领域做出了重大贡献,包括发现了他版本的微积分。他结识了当时法国的主要哲学家尼古拉·马勒伯朗士和安托万·阿尔诺,并研究了笛卡尔和帕斯卡的著作,包括未发表和已发表的作品。[46] 他还与德国数学家埃伦弗里德·瓦尔特·冯·奇尔恩豪斯成为朋友,二人保持了终生的通信联系。
\begin{figure}[ht]
\centering
\includegraphics[width=6cm]{./figures/05d4c37c7cb9457d.png}
\caption{阶梯计算器} \label{fig_LBNC_2}
\end{figure}
当法国显然不会执行莱布尼茨的埃及计划时,选帝侯派遣他的侄子,由莱布尼茨陪同,于1673年初前往伦敦执行与英国政府相关的任务。[47] 在那里,莱布尼茨结识了亨利·奥尔登堡和约翰·柯林斯。他与皇家学会会面,展示了他自1670年起设计并建造的计算机。这台机器能够执行加、减、乘、除四种基本运算,学会很快将他列为外部会员。

当选帝侯去世的消息(1673年2月12日)传来时,这次任务突然中止。莱布尼茨迅速返回巴黎,而不是按计划返回美因茨。[48] 在同一个冬天,他的两位赞助人相继去世,这意味着莱布尼茨需要寻找新的事业基础。

在这方面,1669年布伦瑞克的约翰·弗里德里希公爵曾邀请莱布尼茨访问汉诺威,这一邀请后来被证明是命运的转折点。莱布尼茨最初拒绝了邀请,但从1671年起与公爵开始了通信。1673年,公爵向莱布尼茨提供了顾问的职位。直到两年后,当巴黎或哈布斯堡帝国宫廷中没有其他工作机会时,莱布尼茨才勉强接受了这一职位,尽管他非常享受巴黎的智力氛围。[49]

1675年,他试图以外国名誉会员的身份加入法国科学院,但由于科学院认为已有足够多的外国成员,因此未向他发出邀请。他于1676年10月离开巴黎。
\subsection{汉诺威家族,1676–1716}
\begin{figure}[ht]
\centering
\includegraphics[width=6cm]{./figures/bb0fe18b15719881.png}
\caption{汉诺威公共图书馆藏戈特弗里德·威廉·莱布尼茨肖像,1703年} \label{fig_LBNC_3}
\end{figure}
莱布尼茨设法推迟了前往汉诺威的时间,直到1676年底才抵达。在此之前,他又短暂前往伦敦,期间牛顿指责他提前看过自己未发表的微积分研究成果。[50] 这一指控被认为是后来的窃取微积分争议的证据,几十年后,有人指控莱布尼茨从牛顿那里窃取了微积分。在从伦敦前往汉诺威的旅途中,莱布尼茨在海牙停留,遇见了微生物发现者列文虎克。他还与斯宾诺莎进行了几天的深入讨论,后者刚刚完成了(但尚未出版)他的代表作《伦理学》。[51] 莱布尼茨拜访后不久,斯宾诺莎便去世了。

1677年,莱布尼茨应自己的请求被提升为司法枢密顾问,这一职位他保持至终生。莱布尼茨在勃伦瑞克家族的三位连续统治者手下服务,担任历史学家、政治顾问,最重要的是担任公爵图书馆的图书馆员。从此,他参与了涉及勃伦瑞克家族的各种政治、历史和神学事务,这些文件成为该时期宝贵的历史记录。

莱布尼茨开始推动一个利用风车改进哈尔茨山脉采矿作业的项目。该项目对采矿改进几乎没有成效,最终于1685年被恩斯特·奥古斯特公爵关闭。[49]

在北德接受莱布尼茨的人中为数不多,包括汉诺威选帝侯夫人索菲亚(1630–1714)、她的女儿汉诺威的索菲亚·夏洛特(1668–1705)——普鲁士的王后以及莱布尼茨公开的信徒,以及她的外孙、未来的乔治二世的妻子安斯巴赫的卡罗琳。对于这些女性来说,莱布尼茨既是通信者、顾问,也是朋友。她们对莱布尼茨的认可都超过了她们的配偶及未来的英国国王乔治一世。[52]

汉诺威的人口仅约一万人,其地方性最终让莱布尼茨感到不满。然而,成为勃伦瑞克家族的重要侍臣仍是一项殊荣,尤其是考虑到该家族在莱布尼茨的协助下声望的迅速上升。1692年,勃伦瑞克公爵成为神圣罗马帝国的世袭选帝侯。1701年的《英国王位继承法》指定选帝侯夫人索菲亚及其后代为英国王室成员,以应对国王威廉三世及其妯娌兼继任者安妮女王去世后的王位继承问题。莱布尼茨在该法案的倡议和谈判中发挥了作用,但并非总是有效。例如,他在英国匿名发表的一篇旨在促进勃伦瑞克家族利益的文章被英国议会正式谴责。

勃伦瑞克家族容忍了莱布尼茨在履行侍臣职责之外投入于知识追求的巨大精力,例如完善微积分,撰写数学、逻辑、物理和哲学的文章,并维持广泛的通信联系。他从1674年开始研究微积分;现存笔记中首次使用的证据出现在1675年。到1677年,他已拥有一套完整的体系,但直到1684年才出版。他最重要的数学论文发表在1682年至1692年间,通常刊载于他与奥托·门克于1682年创办的《学者纪事》期刊。该期刊在提升他的数学和科学声誉方面发挥了关键作用,这反过来也增强了他在外交、历史、神学和哲学领域的地位。
\begin{figure}[ht]
\centering
\includegraphics[width=7cm]{./figures/54d47600cd26d10a.png}
\caption{莱布尼茨论文的页面,收藏于波兰国家图书馆} \label{fig_LBNC_4}
\end{figure}
选帝侯恩斯特·奥古斯特委托莱布尼茨编写一部勃伦瑞克家族的历史,追溯至查理大帝时期甚至更早,希望这本书能有助于他的家族野心。从1687年到1690年,莱布尼茨在德国、奥地利和意大利广泛旅行,寻找并收集与此项目相关的档案材料。几十年过去了,历史书仍未完成;下一任选帝侯对莱布尼茨的拖延表现颇为不满。莱布尼茨从未完成这一项目,部分原因在于他在许多其他领域的大量创作,另一个原因则是他坚持编写一本基于档案资料、研究详尽且博学的书籍,而他的资助人其实只希望得到一本简短的通俗书籍,或许只是带有评注的家族世系,在三年或更短的时间内完成。他们从未知道,莱布尼茨实际上已经完成了相当一部分任务:当他为勃伦瑞克家族历史所撰写和收集的材料最终在19世纪出版时,共计三卷。

1691年,莱布尼茨被任命为下萨克森沃尔芬比特尔的赫尔佐格·奥古斯特图书馆馆长。