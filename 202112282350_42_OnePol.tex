% 一元多项式
% keys 多项式|多项式环

数论里有加减乘除,带余除法,(公)因子,(公)倍数,素数,辗转相除法等概念,这些都可以推广到一元多项式里面.一元多项式和 $\lambda$ -矩阵密切相关,$\lambda$ -矩阵又是所谓的 Jordan 标准形的基本概念,而 Jordan 标准形是方阵分解的特例,也为理解方阵分解提供了一个很好的样本.为掌握更高阶的知识,我们得一点点的夯实基础,而这个过程本身就是一个很具有意义的过程.当然,上面提到的概念不要求你提前掌握,若需掌握我会在文中给出链接.所以,不要害怕,我们慢慢来!

讨论多项式时,我们会预先给定一个数域 $\mathbb{F}$,所谓\textbf{数域},是关于加减乘除都封闭的数集(运算关于集合\textbf{封闭}是指该集合中元素的运算结果还是该集合的元素).若你对\textbf{域}有兴趣,点击\upref{field}即可到达.
\begin{definition}{一元多项式}
设 $n$ 是一个非负整数,$x$ 是一个符号,$a_0,a_1,\cdots,a_n\in \mathbb{F}$ ,则表达式
\begin{equation}
f(x)=\sum_{i=0}^{n}a_i x^i
\end{equation}
称为数域 $\mathbb{F}$ 上的\textbf{一元多项式},简称数域 $\mathbb{F}$ 上的\textbf{多项式}.其中,$a_ix^i(0\leq i\leq n)$ 称为这个多项式的\textbf{$i$次项},$a_i$ 称为 \textbf{$i$次项系数}.特别,$a_0$ 称为多项式的\textbf{常数项}.若 $a_n\neq 0$($n$ 为最大整数),则 $n$ 称为多项式 $f(x)$ 的\textbf{次数},记为 $\mathrm{def}f(x)$,并把 $a_nx^n$ 称为 $f(x)$ 的\textbf{首项},$a_n$ 称为\textbf{首项系数}.
\end{definition}
两个多项式 $f(x)$ 和 $g(x)$ 称为\textbf{相等}的,若它们的所有同次项的的系数都相等,并记为 $f(x)=g(x)$.
\subsection{多项式的运算}
设数域上的两个多项式
\begin{align}
f(x)=\sum_{i=0}^{n}a_ix^i\\
g(x)=\sum_{i=0}^{m}a_ix^i
\end{align}
不失一般性,设 $n\geq m$
下面定义它们的加法和乘法.
\begin{definition}{加法}
多项式的\textbf{和}(或\textbf{差})为
\begin{equation}
f(x)\pm g(x)=\sum_{i=0}^n (a_i\pm b_i)x^i
\label{OnePol_eq1}
\end{equation}
其中,约定了 $b_n=b_{n-1}=\cdots=b_{m+1}=0$.特别$f(x)+0=f(x)$.
\end{definition}

显然,两个多项式的和或差仍是多项式.\autoref{OnePol_eq1} 表明两个多项式的和的次数最多和原来次数最大的多项式一样,而系数之和可能为0,所以和的次数满足不等式
\begin{equation}
\mathrm{deg}(f(x)\pm g(x))\leq \max\{\mathrm{deg}f(x),\mathrm{deg}g(x)\}
\end{equation}
\begin{definition}{乘法}
多项式的\textbf{乘积}为
\begin{equation}
f(x)\pm g(x)=\sum_{s=0}^{m+n} c_s x^s
\end{equation}
其中,
\begin{equation}
c_s=\sum_{i=0}^s a_i b_{s-i}
\end{equation}
特别的,$f(x)\cdot 0=0$.
\end{definition}