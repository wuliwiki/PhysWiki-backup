% 库尔特·哥德尔(综述)
% license CCBYSA3
% type Wiki

本文根据 CC-BY-SA 协议转载翻译自维基百科\href{https://en.wikipedia.org/wiki/Kurt_G\%C3\%B6del}{相关文章}。

\begin{figure}[ht]
\centering
\includegraphics[width=6cm]{./figures/6b68f02159236857.png}
\caption{哥德尔,大约在1926年} \label{fig_KRT_1}
\end{figure}
库尔特·弗里德里希·哥德尔(Kurt Friedrich Gödel,1906年4月28日-1978年1月14日)是一位逻辑学家、数学家和哲学家。与亚里士多德和戈特洛布·弗雷格一起,被认为是历史上最重要的逻辑学家之一,哥德尔深刻影响了20世纪的科学和哲学思维(当时,伯特兰·罗素、阿尔弗雷德·诺斯·怀特海德和大卫·希尔伯特正在利用逻辑和集合论研究数学基础),并在弗雷格、理查德·德德金德和乔治·康托尔的早期工作基础上进行扩展。

哥德尔在数学基础方面的发现导致了他在1929年通过其维也纳大学博士论文证明的完备性定理,并且两年后在1931年发表了哥德尔不完备性定理。第一个不完备性定理指出,对于任何足够强大、能够描述自然数算术的 ω-一致递归公理系统(例如,佩亚诺算术),存在一些关于自然数的命题,这些命题既无法从公理中证明,也无法被反驳。为了证明这一点,哥德尔发展了一种现在被称为哥德尔编号的技术,该技术将形式表达式编码为自然数。第二个不完备性定理从第一个定理中推导出来,指出该系统不能证明其自身的一致性。

哥德尔还证明了,在接受的策梅洛–弗伦克尔集合论(Zermelo–Fraenkel set theory)中,选择公理和连续统假设无法被反驳,前提是其公理是一致的。前一个结果为数学家们在其证明中假设选择公理打开了大门。他还通过澄清经典逻辑、直觉主义逻辑和模态逻辑之间的联系,对证明论做出了重要贡献。
\subsection{早年生活与教育}
\subsubsection{童年时期}
哥德尔于1906年4月28日出生在奥匈帝国的布伦(现在的捷克共和国布尔诺),出生在一个讲德语的家庭。父亲鲁道夫·哥德尔(1874-1929)是一个主要纺织厂的总经理和部分股东,母亲玛丽安·哥德尔(原姓汉德舒,1879-1966)。在哥德尔出生时,该市有德语为主的居民,其中包括他的父母。父亲是天主教徒,母亲是新教徒,孩子们也被抚养成新教徒。哥德尔的祖先在布伦的文化生活中通常非常活跃。例如,他的祖父约瑟夫·哥德尔是当时著名的歌手,并且曾是布伦男子合唱团(Brünner Männergesangverein)的成员之一。

当奥匈帝国在第一次世界大战后战败并解体时,哥德尔在12岁时自动成为捷克斯洛伐克的公民。据他的同学克莱佩塔尔说,像许多生活在主要由德语人口构成的苏台德地区的人一样,"哥德尔始终认为自己是奥地利人,并且是捷克斯洛伐克的流亡者"。1929年2月,他被允许放弃捷克斯洛伐克国籍,并于4月获得了奥地利国籍。1938年,当德国吞并奥地利时,32岁的哥德尔自动成为德国公民。1948年,二战后,42岁的哥德尔成为美国公民。

在他的家庭中,年幼的哥德尔被昵称为“为什么先生”(Herr Warum),因为他有着无止境的好奇心。根据他的哥哥鲁道夫的说法,哥德尔在六七岁时曾患过风湿热,虽然完全康复,但他一生都坚信自己的心脏遭受了永久性损害。从四岁起,哥德尔就经常“健康状况不佳”,这种情况贯穿了他的一生。

哥德尔于1912年至1916年间就读于布伦的路德学校(Evangelische Volksschule),并于1916年至1924年在德国语国家文理中学(Deutsches Staats-Realgymnasium)就读,在所有科目中都名列前茅,特别是在数学、语言和宗教方面。虽然哥德尔最初在语言学上表现突出,但后来他对历史和数学产生了更大的兴趣。1920年,哥哥鲁道夫(生于1902年)前往维也纳,在维也纳大学医学院学习时,哥德尔的数学兴趣进一步加深。在青少年时期,哥德尔研究了加贝尔斯伯格速记法、艾萨克·牛顿的批评,以及伊曼努尔·康德的著作。
\subsubsection{维也纳的学习经历}
\begin{figure}[ht]
\centering
\includegraphics[width=8cm]{./figures/6713277c70ddf4c0.png}
\caption{在维也纳43-45号Josefstädter街的Gödel纪念牌匾,这里是他发现不完全性定理的地方。} \label{fig_KRT_2}
\end{figure}
18岁时,哥德尔与哥哥一起进入维也纳大学。他已经掌握了大学水平的数学。 尽管最初打算学习理论物理学,他也参加了数学和哲学的课程。在此期间,他接受了数学实在论的观点。他阅读了康德的《自然科学的形而上学基础》,并与莫里茨·施里克、汉斯·哈恩和鲁道夫·卡尔纳普一起参加了维也纳学派的活动。哥德尔随后研究了数论,但当他参加由莫里茨·施里克主持的研讨会,该研讨会研究了伯特兰·罗素的《数学哲学导论》时,他对数学逻辑产生了兴趣。根据哥德尔的说法,数学逻辑是“所有学科之前的科学,包含了所有科学背后的思想和原则。”

听完大卫·希尔伯特在博洛尼亚关于数学系统的完备性与一致性的讲座后,哥德尔可能确定了自己的人生方向。 1928年,希尔伯特和威廉·阿克曼出版了《数学逻辑基础》(Grundzüge der theoretischen Logik),这是一本介绍一阶逻辑的书,提出了完备性问题:“一个形式系统的公理是否足够推导出所有在该系统的所有模型中都为真的命题?”

这个问题成为了哥德尔选择作为博士论文主题的课题。 1929年,23岁的哥德尔在汉斯·哈恩的指导下完成了博士论文。论文中,他确立了自己命名的完备性定理,涉及一阶逻辑。他于1930年获得博士学位, 论文(附带额外的工作)由维也纳科学院出版。
\subsection{职业生涯}
\subsubsection{不完全性定理}
\begin{figure}[ht]
\centering
\includegraphics[width=6cm]{./figures/07bea3120b5a44e4.png}
\caption{1925年作为学生的哥德尔} \label{fig_KRT_3}
\end{figure}
库尔特·哥德尔在现代逻辑方面的成就堪称独一无二、具有纪念意义——事实上,它不仅是一个纪念碑,而是一个地标,远远超越时空,始终可见。……哥德尔的成就无疑彻底改变了逻辑学的本质和可能性。

——约翰·冯·诺依曼[19]

1930年,哥德尔参加了第二届精确科学认识论会议,该会议于9月5日至7日在柯尼斯堡举行。在会上,他展示了他的第一阶逻辑的完整性定理,并在讲座结束时提到,这个结果不能推广到更高阶逻辑,从而暗示了他的不完全性定理。[20]

哥德尔在《Über formal unentscheidbare Sätze der Principia Mathematica und verwandter Systeme》(英文名为《On Formally Undecidable Propositions of Principia Mathematica and Related Systems》)中发布了他的不完全性定理。在这篇文章中,他证明了对于任何足够强大的可计算公理化系统(例如,佩亚诺公理或带选择公理的泽梅洛–弗兰克尔集合论),都有以下两个结论:

1. 如果一个(逻辑或公理形式的)系统是ω-一致的,它就不能是语法上完备的。
2. 公理的一致性不能在它们自己的系统内被证明。[21]

这些定理终结了自戈特洛布·弗雷格的工作开始,直到《数学原理》和希尔伯特计划的提出,持续了半个世纪的尝试,旨在寻找足够的非相对一致公理化,来为数论提供基础(并且为其他数学领域提供基础)[22]。

哥德尔构造了一个公式,声称它在给定的形式系统中是不可证明的。如果它可以被证明,那么它就是假的。换句话说,总会有至少一个真实但无法证明的命题。也就是说,对于任何可计算枚举的算术公理集(即,原则上可以由理想化的计算机在无限资源下打印出来的集合),总存在一个关于算术的公式,它是真实的,但在该系统中无法证明。为了使这一点更加精确,哥德尔必须提出一种方法,将命题、证明和可证明性的概念编码为自然数;他通过被称为哥德尔编号的过程完成了这一点。[23]

在他1932年发布的两页论文《Zum intuitionistischen Aussagenkalkül》中,哥德尔驳斥了直觉主义逻辑的有限值性。在证明中,他隐含地使用了后来成为哥德尔–杜梅特中间逻辑(或哥德尔模糊逻辑)的方法。[24]
\subsubsection{1930年代中期:进一步的工作和美国访问}
Gödel于1932年在维也纳获得了教授资格,并于1933年成为了该校的Privatdozent(无薪讲师)。1933年,阿道夫·希特勒在德国上台,接下来的几年,纳粹在奥地利和维也纳的数学家中逐渐获得影响力。1936年6月,Moritz Schlick(曾引发Gödel对逻辑兴趣的讲座导师)被他的前学生Johann Nelböck暗杀,这引发了Gödel的“严重神经危机”。[25] 他出现了偏执症状,包括害怕被毒死,并因此在神经病疗养院里待了几个月。[26]

1933年,Gödel首次前往美国,在那里遇到了阿尔伯特·爱因斯坦,并与他成为了好朋友。[27] 他在美国数学学会年会上发表了演讲。那一年,Gödel还发展了可计算性和递归函数的思想,并能够就一般递归函数和真理的概念进行讲座。这些工作是在数论领域中完成的,使用了Gödel编号法。

1934年,Gödel在普林斯顿的高等研究院(IAS)举行了一系列讲座,题为《形式数学系统中的不可判定命题》。当时刚刚完成博士学位的斯蒂芬·克莱尼(Stephen Kleene)记下了这些讲座的笔记,并随后出版了。

1935年秋天,Gödel再次访问了IAS。旅行和高强度的工作让他感到疲惫,次年他暂停工作,休息以恢复抑郁症状。1937年,他重返教学岗位。在这段时间里,他致力于证明选择公理和连续统假设的一致性;他证明了这些假设无法从公认的集合论公理体系中被推翻。

他与已认识10多年的阿黛尔·宁布尔斯基(Adele Nimbursky,原姓Porkert,1899–1981)于1938年9月20日结婚。Gödel的父母反对他们的婚姻,因为她是一个离过婚的舞蹈演员,比他大六岁。

此后,他再次前往美国,在1938年秋季到高等研究院,并出版了《选择公理和广义连续统假设在集合论公理中的一致性》[28],这部作品是现代数学的经典。在该作品中,他引入了可构造的宇宙——一种集合论模型,其中只有那些可以从简单集合构造出的集合才存在。Gödel证明了选择公理(AC)和广义连续统假设(GCH)在可构造宇宙中都成立,因此必须与Zermelo–Fraenkel集合论公理(ZF)一致。这个结果对数学家们有重大影响,因为它意味着他们在证明Hahn-Banach定理时可以假设选择公理。后来,保罗·科恩(Paul Cohen)构造了一个ZF模型,其中AC和GCH为假;这两项证明意味着AC和GCH在ZF公理下是独立的。

Gödel在1939年春天到达了圣母大学(University of Notre Dame)。[29]
\subsubsection{普林斯顿,爱因斯坦,美国国籍} 
1938年3月12日,奥地利发生“合并”,成为纳粹德国的一部分。德国废除了Privatdozent职称,Gödel因此不得不根据新秩序申请新的职位。他与维也纳学派中犹太成员的往来,尤其是与汉恩(Hahn)的关系,对他产生了负面影响。维也纳大学拒绝了他的申请。

当德国军队认为他适合服兵役时,他的困境加剧。1939年9月,第二次世界大战爆发。在战争爆发前,Gödel和妻子离开维也纳,前往普林斯顿。为了避免横渡大西洋的困难,他们选择乘坐横穿西伯利亚的铁路到达太平洋,从日本乘船到达旧金山(于1940年3月4日抵达),然后乘火车穿越美国前往普林斯顿。[30] 在此行中,Gödel本应携带一封来自维也纳物理学家Hans Thirring的秘密信件,目的在于提醒阿尔伯特·爱因斯坦,告知罗斯福总统,希特勒可能正在研制原子弹。然而,尽管Gödel与爱因斯坦见面,他并未将该信件交给爱因斯坦,因为他并不相信希特勒能够完成这一壮举。[31] 无论如何,Leo Szilard已经将这一消息传递给了爱因斯坦,爱因斯坦也已警告了罗斯福。

在普林斯顿,Gödel接受了高等研究院(IAS)的职位,该院他曾在1933-34年访问过。[32]

此时,爱因斯坦也在普林斯顿生活。Gödel与爱因斯坦建立了深厚的友谊,两人经常一起散步,往返于高等研究院之间。其他研究院成员对此感到困惑,不知道他们的谈话内容。经济学家奥斯卡·莫根斯坦回忆说,爱因斯坦在临终时曾向他透露:“我自己的工作已经不再重要,我来到研究院只是为了享有和Gödel一起散步回家的特权。”[33]

Gödel和妻子阿黛尔于1942年夏天在缅因州的蓝山(Blue Hill)度过,这个地方位于海湾顶部的蓝山旅馆。Gödel不仅是在度假,还在工作上取得了丰硕的成果。根据Gödel尚未出版的《工作笔记》中的第15本[Heft 15],John W. Dawson Jr.推测,Gödel在1942年蓝山时发现了选择公理与有限类型理论(集合论的一个弱形式)独立性的证明。Gödel的密友Hao Wang支持这一推测,指出Gödel的蓝山笔记包含了他对这一问题最为详细的处理。

1947年12月5日,爱因斯坦和莫根斯坦陪同Gödel参加美国国籍考试,并作为见证人。Gödel曾向他们透露,他发现美国宪法中存在一个可能使美国变成独裁政体的漏洞,这一发现后来被称为“Gödel漏洞”。爱因斯坦和莫根斯坦担心他们朋友的不可预测行为可能会影响他的申请。审判官是Phillip Forman,他认识爱因斯坦,并曾在爱因斯坦的入籍听证会上宣誓。所有流程都很顺利,直到Forman问Gödel是否认为像纳粹政权这样的独裁政体能在美国出现。Gödel随即开始向Forman解释他的发现。Forman明白了情况,打断了Gödel,并将听证会转向了其他问题,并顺利结束了。[34][35]

1946年,Gödel成为普林斯顿高等研究院的正式成员。从那时起,他停止了出版工作,尽管他仍继续研究。1953年,他成为该院的正教授,1976年成为名誉教授。[36]

在高等研究院的期间,Gödel的兴趣转向了哲学和物理学。1949年,他展示了存在解决方案,涉及到爱因斯坦广义相对论的场方程中的闭合类时曲线。[37] 据说他把这一成果当作70岁生日礼物送给了爱因斯坦。[38] 他所提出的“旋转宇宙”理论可以使时间旅行成为可能,并使爱因斯坦对他自己的理论产生了怀疑。这些解被称为Gödel度规(Einstein场方程的精确解)。

他研究并钦佩戈特弗里德·莱布尼茨的著作,但他认为一些莱布尼茨的作品因敌对的阴谋而被压制。[39] 他在一定程度上研究了伊曼努尔·康德和爱德蒙·胡塞尔的思想。1970年代初,Gödel将他对莱布尼茨版安瑟尔姆本体论证明的 elaboration(对上帝存在的证明)传给了他的朋友们。这一证明现在被称为Gödel的本体论证明。
\subsection{奖项与荣誉}  
Gödel与朱利安·施温格(Julian Schwinger)共同获得了1951年的首届阿尔伯特·爱因斯坦奖,并于1974年获得了美国国家科学奖章。[40] Gödel于1961年当选为美国哲学学会的常住会员,并于1968年当选为英国皇家学会的外籍会员(ForMemRS)。[41][1] 他还曾在1950年担任美国数学会国际数学大会(ICM)的全体会议发言人,地点是马萨诸塞州的剑桥。[42]
\subsection{晚年与去世}
\begin{figure}[ht]
\centering
\includegraphics[width=6cm]{./figures/099e6246a9dfc262.png}
\caption{库尔特和阿黛尔·哥德尔的墓碑,位于新泽西州普林斯顿公墓。} \label{fig_KRT_4}
\end{figure}
在晚年,Gödel经历了精神不稳定和疾病的时期。在他亲密朋友莫里茨·施利克(Moritz Schlick)被刺杀后,[43] Gödel产生了强烈的被毒死的恐惧症,他只吃妻子阿黛尔(Adele)准备的食物。1977年底,阿黛尔被送进医院,Gödel在她缺席期间拒绝进食;[44] 他在1978年1月14日因“由于人格障碍引起的营养不良和虚弱”在普林斯顿医院去世,体重仅为29公斤(65磅)。[45] 他被埋葬在普林斯顿公墓。阿黛尔于1981年去世。[46]