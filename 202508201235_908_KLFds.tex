% 克利福德代数(综述)
% license CCBYSA3
% type Wiki

本文根据 CC-BY-SA 协议转载翻译自维基百科\href{https://en.wikipedia.org/wiki/Clifford_algebra}{相关文章}。

在数学中,克利福德代数[a] 是由具备二次型的向量空间生成的一类代数。它是一个带有单位元的结合代数,同时具有一个特定子空间这一附加结构。作为 $K$-代数,它们推广了实数、复数、四元数以及若干其他超复数体系。[1][2] 克利福德代数理论与二次型理论及正交变换理论紧密相关。克利福德代数在几何学、理论物理以及数字图像处理等诸多领域中具有重要应用。该名称源自英国数学家威廉·金登·克利福德(William Kingdon Clifford, 1845–1879)。

在所有克利福德代数中,最为常见的是正交克利福德代数,它们亦称为(伪)黎曼克利福德代数,以区别于辛克利福德代数。[b]

\subsection{引言与基本性质}
克利福德代数是一个包含并由向量空间 $V$ 生成的有单位元的结合代数,其中 $V$ 定义在一个域 $K$ 上,并配备有一个二次型$Q: V \to K$。克利福德代数 $\mathrm{Cl}(V, Q)$ 是满足如下条件的“最自由”[c] 有单位元结合代数:
$$
v^{2} = Q(v) \cdot 1 \quad \text{对所有 } v \in V,~
$$
其中左侧的乘积是代数内部的乘法,而右侧的 $1$ 表示代数的乘法单位元(需注意,不同于域 $K$ 的乘法单位元)。这里“最自由”或“最一般”的含义,可以通过泛性质的概念形式化地加以刻画,如下所述。

当 $V$ 是有限维实向量空间且 $Q$ 非退化时,$\mathrm{Cl}(V, Q)$ 可以记作 $\mathrm{Cl}_{p,q}(\mathbf{R})$。这表示 $V$ 存在一个正交基,其中 $p$ 个基向量满足 $e_i^2 = +1$,而 $q$ 个基向量满足 $e_i^2 = -1$;$\mathbf{R}$ 表明这是定义在实数上的克利福德代数,即代数元素的系数均为实数。该基底可通过正交对角化方法获得。

由向量空间 $V$ 生成的自由代数可写作张量代数$\bigoplus_{n \geq 0} V^{\otimes n}$,即所有 $n$ 重张量积的直和。于是,克利福德代数可表述为该张量代数关于双边理想的商,其中理想由以下形式的元素生成:$v \otimes v - Q(v) \cdot 1, \quad \forall v \in V$.在商代数中,由张量积诱导的乘法通常用并置表示(例如 $uv$)。其结合性直接源于张量积的结合性。

克利福德代数具有一个**特定子空间** $V$,即嵌入映射的像。然而,一般来说,给定一个与克利福德代数同构的 $K$-代数,并不能唯一确定这样的子空间。

若底域 $K$ 中的 2 可逆,则上述基本恒等式可以改写为
$$
uv + vu = 2 \langle u, v \rangle \cdot 1 \quad \forall u, v \in V,~
$$
其中
$$
\langle u, v \rangle = \tfrac{1}{2}\big(Q(u+v) - Q(u) - Q(v)\big)~
$$
是通过极化恒等式由二次型 $Q$ 所关联的对称双线性型。

在此方面,特征为 2 的情形构成一个特殊情况。特别地,当 $\mathrm{char}(K) = 2$ 时,二次型并不一定能够唯一地决定一个满足$Q(v) = \langle v, v \rangle$的对称双线性型。[3] 因此,本文中的许多叙述均附带条件:底域的特征不为 2;若去除此条件,结论即不再成立。
