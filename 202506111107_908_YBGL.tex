% 约翰·彼得·古斯塔夫·勒热纳·狄利克雷(综述)
% license CCBYSA3
% type Wiki

本文根据 CC-BY-SA 协议转载翻译自维基百科 \href{https://en.wikipedia.org/wiki/Peter_Gustav_Lejeune_Dirichlet}{相关文章}。

约翰·彼得·古斯塔夫·勒让·狄利克雷(Johann Peter Gustav Lejeune Dirichlet,/ˌdɪərɪˈkleɪ/\(^\text{[1]}\),[德语发音:[ləˈʒœn diʁiˈkleː]\(^\text{[2]}\);1805年2月13日-1859年5月5日)是德国数学家。在数论中,他证明了费马大定理的一些特殊情形,并创立了解析数论。在分析学中,他推进了傅里叶级数理论的发展,并且是最早给出函数现代形式定义的数学家之一。在数学物理中,他研究了势理论、边值问题、热扩散和流体力学。

尽管他的姓氏是“勒让·狄利克雷”,但在引用以他命名的成果时,人们通常只使用“狄利克雷”这一名字。
\subsection{生平}
\subsubsection{早年生活(1805–1822)}
古斯塔夫·勒让·狄利克雷于1805年2月13日出生在迪伦(Düren),这是位于莱茵河左岸的一座小镇,当时属于法兰西第一帝国,1815年维也纳会议后归属普鲁士。他的父亲约翰·阿诺德·勒让·狄利克雷(Johann Arnold Lejeune Dirichlet)是一名邮政局长、商人及市议会议员。他的祖父则从比利时列日东北约5公里处的小村庄里舍莱特(Richelette,或更可能是 Richelle [fr])迁居至迪伦,因此家族姓氏“勒让·狄利克雷”(Lejeune Dirichlet,法语意为“来自里舍莱特的年轻人”)由此而来。\(^\text{[3]}\)

尽管家境并不富裕,而且狄利克雷在七个孩子中排行最小,父母仍然支持他的教育。他们先让他就读于一所小学,之后又转入私立学校,希望他日后能成为一名商人。然而年幼的狄利克雷在12岁之前就表现出了对数学的强烈兴趣,最终说服父母让他继续深造。1817年,他被送往波恩文理中学(Gymnasium Bonn [de])就读,受到家庭熟识的彼得·约瑟夫·埃尔韦尼希(Peter Joseph Elvenich,一位学生)的照顾。1820年,狄利克雷转入科隆耶稣会文理中学,在那里他在乔治·欧姆的指导下,进一步拓宽了数学知识。次年他离开了文理中学,仅获得了一张结业证书,因为他无法流利使用拉丁语,未能取得正式的中学毕业文凭。\(^\text{[3]}\)
