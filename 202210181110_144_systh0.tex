% 系统的动量定理、角动量定理、动能定理及其守恒

类似于单个质点的动量、角动量、动能定理,系统也存在相关的定理,bi有相似的形式,

\subsection{系统的动量定理}
\begin{equation}
\bvec F = \bvec F_{ext} = \dv{\bvec P}{t}
\end{equation}
系统所受合力等于系统动量的变化率.由于合力等于合外力,因此也可以表述为“系统所受合外力等于系统动量的变化率”

\subsection{系统的角动量定理}
\begin{equation}
\bvec \tau = \bvec \tau_{ext} = \dv{\bvec L}{t}
\end{equation}
系统所受力矩和等于系统角动量的变化率.由于力矩和等于外力矩和,因此也可以表述为“系统所受外力矩和等于系统角动量的变化率”

\subsection{系统的动能定理}
\begin{equation}
\delta w =\delta w_{in} + \delta w_{ext} = \dd E_k
\end{equation}
系统中功之和等于系统动能的增量.
