% 数论函数
% 数论|函数

\pentry{映射\upref{map}}

\begin{issues}
\issueDraft
\issueMissDepend
\end{issues}

\begin{definition}{数论函数}
以自然数集或正整数集为定义域,以一数集\footnote{指复数域的一个子集.}为值域的函数称为数论函数.
\end{definition}

下面列出常见的数论函数(均定义在自然数集上,其中的$\lambda$均是实数):
\begin{example}{简单数论函数}
\begin{itemize}
\item 单位函数$I(n)$
\begin{equation}
I(n) =
\begin{cases}
1,\ n = 1,\\
0,\ n > 1.
\end{cases}
\end{equation}
\item 恒等函数$e(n)$
\begin{equation}
e(n)=n
\end{equation}
\item 幂函数$n^\lambda$
\item 对数函数$\log n$\footnote{在数论函数中,对数函数的底默认为自然常数$e$.}
\end{itemize}
\end{example}
\begin{example}{与因数有关的数论函数}
\begin{itemize}
\item 除数函数$d(n)$
\begin{equation}
d(n)=\sum_{d|n} 1=
\begin{cases}
1,\ n=1,\\
\prod\limits_{i=1}^{s}(\alpha_i+1),\ n=p_1^{\alpha_1}p_2^{\alpha_2}\cdots p_s^{\alpha_s}.
\end{cases}
\end{equation}
\item 除数和函数$\sigma(n)$
\begin{equation}
\sigma(n)=\sum_{d|n}d=
\begin{cases}
1,\ n=1\\
\prod\limits_{i=1}^{s}\dfrac{p_i^{\alpha_i}-1}{p_i-1},\ n=p_1^{\alpha_1}p_2^{\alpha_2}\cdots p_s^{\alpha_s}.
\end{cases}
\end{equation}
\item 除数幂和函数$\sigma_\lambda(n)$
\begin{equation}
\sigma_\lambda(n)=\sum_{d|n}d^\lambda
\end{equation}
\end{itemize}
\end{example}
\begin{example}{著名的数论函数}
\begin{itemize}
\item 素数计数函数$\pi(n)$
\begin{equation}
\pi(n)=\sum_{p\leq n} 1
\end{equation}
\item $\mathrm{M\ddot{o}bius}$函数$\mu(n)$
\begin{equation}
\mu(n)=
\begin{cases}
1,\ n=1,\\
0,\ n=l^2k,\ l\geq 1,\ l,k\in\mathbb{N},\\
(-1)^s,\ n=p_1p_2p_3\cdots p_s.
\end{cases}
\end{equation}
\item $\mathrm{Euler}$函数$\varphi(n)$
\begin{equation}
\varphi(n)=\sum_{1\leq d\leq n,(d,n)=1}1
\end{equation}
\item $\mathrm{van\ Mangoldt}$函数$\Lambda(n)$
\begin{equation}
\varLambda(n)=
\begin{cases}
\log p,\ n=p^s,\\
0,\ \omega(n)\neq 1.
\end{cases}
\end{equation}
\end{itemize}

\end{example}