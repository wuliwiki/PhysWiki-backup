% 协变性和不变性
% keys 协变性|不变性|物理
% license Usr
% type Tutor

协变性和不变性是相对论中会遇到的术语,本节给出它们具体的定义。
\subsection{概念的引入}
物理定律经常被表达为一个矢量等于另一个矢量,例如,Newton定律 
\begin{equation}
m\bvec a=\bvec F.~
\end{equation}
若换一参考系(重选基底),它和原参考系(或坐标系)由坐标变换相联系。设 $R$ 是这两参考系下矢量的变换矩阵,即若 $\bvec x$ 是旧参考系下表达的矢量,则新参考系下表达的该矢量 $\bvec x'$ 和旧参考系下的 $\bvec x$ 关系为 $\bvec x'=R\bvec x$。将 $R$ 作用于Newton定律,就有
\begin{equation}
mR\bvec a=R\bvec F.~
\end{equation}
由于加速度是矢量,因此新参考系下的加速度为 $\bvec a'=R\bvec a$。假设 $\bvec F$ 像一个矢量一样变换(虽然这里对力使用了矢量相同的符号,但是力的定义并不清晰,因此在不同坐标变换下不一定是一个矢量),那么新坐标下 $\bvec F'=R\bvec F$。于是在新坐标系下,就有
\begin{equation}
m\bvec a'=\bvec F'.~
\end{equation}
即两坐标系下的牛顿定理具有相同的形式。应该注意的是,质量是标量的例子,即在坐标变换下不变。如果改变的话,那么Newton定律在坐标变换下就不是不变的了,从而导致某个参考系比另一个更可取,这是不可接受的,物理学应当建立在平等的思想上(可视为某种假设)。

Newton定律是协变的,是指方程的两边在坐标变换下按照同一变换方式变换,即若左边的量 $x$ 在新坐标系下为 $x'f(x)$ (映射 $f$ 由新旧坐标系确定),那么右边的量 $y$ 在新坐标系下由 $y'=f(y)$ 确定。然而,由Newton定律表达的物理是不变的,即独立于通过坐标变换联系的参考系。若物理依赖于你如何偏头,那么就相当的麻烦了。物理不应该取决于物理学家,但物理学家具有使用不同方式表达物理的自由。

\subsection{定义}
每一物理学都具有一些最基本的物理定律,对应的物理学是这些基本定律下逻辑推演。物理定律在数学上表达为方程的形式,对应基本物理定律的方程称为\textbf{基本方程}。
\begin{definition}{协变性、不变性}
设
\begin{equation}
y_i= f(x_i),i=1,\ldots n.
\end{equation}
是在某一参考系下某一物理学的基本方程, $R$ 是这一参考系和另一新参考系之间的变换,使得基本方程左边的物理量在新参考系下为 $y'_i=R(y_i)$。若基本方程右边的物理量在新参考系下为 $x_i'=R(x_i)$,则

\end{definition}















