% 流体运动的描述方法
% 欧拉法|拉格朗日法

我们可以从流体与固体的区别中思考流体运动的描述方法,固体无论处于静止还是运动都可以通过有限的静变形承受剪切力,其形状不易变化,在运动学中,可以只从几何角度来描述物体的位置随时间的变化,最简单的运动学是质点运动学,考虑物体形状有刚体运动学,考虑物体变形有弹性力学和塑性力学等.

流体在静止时无法承受剪切力,而在运动状态下虽能产生剪切力,剪切力却不能维持流体内部各点位置的有序性,其内部各点在运动中相对位置关系变化可以很大,流体各部分之间不能保持规则的几何形状,单从几何角度描述其变化将复杂而困难,

\subsection{欧拉法}

\subsection{拉格朗日法}