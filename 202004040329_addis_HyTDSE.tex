% 氢原子薛定谔方程数值解
\pentry{薛定谔方程\upref{TDSE}, 原子单位\upref{AU}}

虽然最直观的方法是使用直角坐标, 但计算效率太低. 实际中一般使用球坐标系, 用球谐函数展开波函数. 如果 Hamiltonian 是轴对称的, 那么只需要 $m = 0$ 的球谐函数, 即勒让德多项式.

\begin{equation}
\Psi(\bvec r, t) = \frac{1}{r}\sum_{l,m} \psi_{l,m}(r) Y_{l,m}(\bvec r)
\end{equation}
定态薛定谔方程, 并左乘 $\bra{Y_{l,m}}$ 得
\begin{equation}
-\frac{1}{2m} \pdv[2]{\psi_{l,m}}{r} + \qty[-\frac{Z}{r} + \frac{l(l+1)}{2mr^2}]\psi_{l,m} = E \psi_{l,m}
\end{equation}
可见我们


\begin{equation}
-\frac{1}{2m} \pdv[2]{\psi_{l,m}}{r} + \qty[-\frac{Z}{r} + \frac{l(l+1)}{2mr^2}]\psi_{l,m} = \I \pdv{\psi_{l,m}}{t}
\end{equation}

