% Vim 笔记

\begin{issues}
\issueDraft
\end{issues}

\subsection{基础}
\begin{itemize}
\item 在 Ubuntu 中, \verb`vi` 就是 \verb`vim`, 打开的都是 vim 程序.
\item linux 命令行输入 \verb`vimtutor` 进入基础教程
\item \verb`ctrl+u` 向上翻半页, \verb|ctrl+d| 向下翻半页, \verb|ctrl+b| 上一页, \verb|ctrl+f| 下一页
\item 用 \verb|:w| 保存, \verb|:wq| 保存并退出, \verb`:w /<path>/<file>` 另存.
\item 撤销: \verb`u`, 重做: \verb`Ctrl+r`
\item \verb`dd` 剪切一行, \verb|p| 粘贴。
\item \verb`G` 可以跳到文档最后, \verb`gg` 可以跳到文件开始
\item 复制: \verb`y` (yank), 剪切: \verb`d` (delete)
\item 命令行用 \verb`Ctrl+D` 可以查看所有候选的补全
\item 由于 ctrl 键过于重要, 建议仿照 hhkb, 将 caps loc 和左 ctrl 键位置调换, 用 sharp keys 软件就可以\autoref{Ubuntu_sub1}~\upref{Ubuntu}。
\end{itemize}

\subsection{移动}
\begin{itemize}
\item 使用 \verb|hjkl| 按键可以上下左右移动, 不需要方向键(好远)
\item \verb`w` 可以跳到下一个单词, 比 \verb`l` 要快
\item \verb`b` 跳到上一个单词, 比 \verb`h` 要快
\item \verb`e` 跳到下个单词末尾
\item \verb`0` 使光标跳到行首
\item \verb`^` 跳到第一个不为空格/tab 的字母
\item \verb`$` 跳到行末
\item \verb`ctrl+e` 将窗口向下移动一行, \verb`ctrl+y` 向上移动一行
\item \verb`ctrl+]` 可以进入一个 tag, 相当于点击链接, 用 ctrl+o 返回
\item \verb`123G` 跳到第 123 行
\item 用 \verb`:<number>` 也可以直接跳到指定的行
\item 按 \verb`$` 跳到行末
\item 按 \verb`A` 跳到行末并编辑
\item \verb|``| 可以让光标回到上一个位置, \verb`''` 让光标回到上一个位置的行首。 连续按多次将在两个位置间跳转
\item 要在 insert 或者 command 模式移动光标(包括上键查找历史), 可以设置快捷键如下
\begin{lstlisting}[language=bash]
" insert mode 快捷键
inoremap <C-h> <Left>
inoremap <C-j> <Down>
inoremap <C-k> <Up>
inoremap <C-l> <Right>
" command mode 快捷键
cnoremap <C-h> <Left>
cnoremap <C-j> <Down>
cnoremap <C-k> <Up>
cnoremap <C-l> <Right>
\end{lstlisting}
\item \verb`z+回车` 用于将当前行移动到屏幕顶部, \verb`zz` 移动到中间
\item \verb`H` 让光标跳到屏幕第一行行首, \verb`L` 跳到最后一行行首, \verb`M` 跳到中间
\item \verb`(` 和 \verb`)` 把光标移动到上一句或者下一句。 下一句。 下一句。 下一句。
\item \verb`{` 和 \verb`}` 把光标移动到上一段或者下一段
\end{itemize}

\subsection{编辑}
\begin{itemize}
\item 按 \verb`i` 直接编辑
\item 用 \verb`x` 键删除光标处的字符
\item 在单词首字母按 \verb`dw` 可以删除当前字符到单词 \verb`d` 开启 delete 模式, \verb`w` 跳到下一个单词行首
\item \verb`d$` 可以删除当前字符到行末 
\item \verb`dw`, \verb`d$` 这样的操作叫做 operator + motion
\item 在移动前面加数字, 可以重复移动指定的次数, 例如 \verb`100w`, \verb`d3w`(删除三个字)
\item \verb`dd` 删除一行, \verb`3dd` 删除三行
\item 编辑模式或者命令模式下 \verb`ctrl+U` 可以删除光标左边的内容
\item \verb`f` 加一个字符可以跳到同一行内下一个出现该字符的位置(话说为什么不直接用 \verb`/`)
\end{itemize}

\subsection{vimtutor 笔记}
\begin{itemize}
\item \verb`set mouse=a` 可以使用鼠标!
\item 鼠标双击也可以进入 tag
\item \verb`:help 内容` 可以进入帮助页面
\item 按 \verb`esc` 进入 normal mode
\item \verb`u` 撤销, ctrl+R 重做,\verb`10u` 撤销 10 次。 注意每次撤销一个命令, 一次插入很多单词也算一个命令, \verb`10x` 也算一个命令
\item \verb`U` 用于撤销改行的所有修改
\item \verb`p` (put)用于把 \verb`dd` 删除的行插入到光标下方, 可以多次粘贴, 可以 \verb`5p`
\item \verb`r` (replace)可以替换光标处的单个字符, 例如 \verb`ra` 替换为 \verb`a`, \verb`3ra` 连续三个字符替换为 \verb`a`
\item \verb`c` (change)相当于 \verb`d` 再 \verb`a`
\item \verb`ctrl+G` 查看当前的位置和文件状态
\item \verb`%` 可以找到对应的另一半括号 (), [], {}
\end{itemize}

\subsection{Visual}
\begin{itemize}
\item 选择: \verb`v`, 选择行: \verb`V`, 选择块: \verb`ctrl+v`.
\item 复制时用 \verb`v` 粘贴(光标右侧): \verb`p`, 粘贴(光标左侧): \verb`P`
\item 剪切整行直接用 \verb`dd` (事实上是剪切!)
\item 复制时用 \verb`V` 或, 粘贴(光标下一行下移): \verb`p`, 粘贴(光标所在行下移): \verb`P`
\item 全选,用 \verb`ggVG` 即可
\end{itemize}

\subsection{搜索替换}
\begin{itemize}
\item 如果 search 的内容中包含 \verb`/`, 用 \verb`\/` 即可, 可能的转义的字符都用 \verb`\` 试试就行
\item 用 \verb`/ *<string>*` 会向下搜索文档中所有包含 \verb`string` 的位置, 如果再按 \verb`n`, 会搜索下一处. 
\item 向下搜索 \verb`/pattern`, 向上搜索 \verb`?pattern`,下一个 \verb`n`, 上一个 \verb`N`.
\item \verb`:noh` 可以取消搜索匹配的高亮(如果有设置)
\item \verb`q/` 可以打开新窗口编辑搜索历史, 用回车执行, 按两次 \verb`ctrl+c` 关闭
\item 不区分大小写搜索:  \verb`/` \verb`\c`
\item \verb`*` 可以搜索当前光标下的单词
\end{itemize}

\subsection{命令}
\begin{itemize}
\item 用 \verb`esc` 返回 Normal mode
\item 可以用 tab 键补全命令(包括查看文件目录)
\item 可以用上下方向键根据命令历史补全(同 Matlab)
\item 也可以用 \verb`q:` 打开一个窗口编辑历史命令, 用回车执行,按两次 \verb`ctrl+c` 关闭
命令模式下,\verb`ctrl+r` 快捷键可以插入许多内容。 \verb`ctrl+r %` 可以插入当前文件名(含相对目录)
\end{itemize}

\subsection{多窗口}
\begin{itemize}
\item 参考 \verb`:help usr_08.txt`
\item \verb`:split` 将当前窗口复制一份, 选中上方窗口
\item \verb`:10split` 指定新增窗口行数
\item \verb`:split [file]` 在新窗口打开文件
\item \verb`:only` 关闭其他所有窗口
\item \verb`:new` 创建新文件,在新窗口打开
\item \verb`ctrl+w w` 切换窗口, \verb`ctrl+w ctrl+w` 也一样(ctrl 按着连按两个 w)
\item 貌似按下 \verb`ctrl+w` 以后, \verb`ctrl` 都可以一直按着也没关系, 例如 \verb`ctrl+w` 后 \verb`ctrl` 不松手连续按 \verb`wj, wk`
\item \verb`ctrl+w +` 或 \verb`ctrl+w -` 用于调整窗口大小, 例如 \verb`10ctrl+w -`
\item \verb`10 ctrl+w _` 指定大小为 10 行
\item \verb`:q` 仅退出当前窗口, \verb`:qa` 退出所有窗口
\item \verb`:vsplit` 可以让窗口左右划分
\item \verb`:vnew` 可以就是左右划分的 \verb`new`
\item \verb`ctrl+w` 后加上 \verb`hjkl` 可以切换到指定方向的窗口
\item \verb`ctrl+w` 后加上 \verb`t` 或者 \verb`b` 可以跳到最上方和最下方的窗口
\item 打开 vim 时用 \verb`vim -o 1.txt 2.txt 3.txt` 可以每个文件开一个窗口
\end{itemize}

\subsection{多 Tab}
\begin{itemize}
\item 用 \verb`:tabnew` 新建一个 tab
\item \verb`:tabf 文件名` 新建一个 tab 用于打开某文件
\item \verb`q` 退出当前 tab
\item \verb`:tabn` 和 \verb`:tabp` 用于切换到后一个/前一个 tab
\end{itemize}

\subsection{设置}
\begin{itemize}
\item Tab 默认占七个格, 可以在 .vimrc 文件中设置 \verb`set ts=4`
\end{itemize}

\subsection{目录}
\begin{itemize}
\item 当前目录可以用 \verb`:pwd` 查看
\item \verb`:cd` 可以更改当前目录
\end{itemize}

\subsection{文件搜索}
\begin{itemize}
\item 用 \verb`find: **/文件名` 可以搜索当前目录和子目录下的文件名并打开(但如果有太多 match 就会出错)
\end{itemize}

\subsection{vimrc}
\begin{itemize}
\item 修改 vimrc 以后, 使用 \verb`:source ~/.vimrc` 就可以生效, 不需要重启 vim
\end{itemize}

\subsection{旧}
\begin{itemize}
\item 要养成经常保存的习惯,尤其是用 ssh
\item home, end, page up, page down 都可以正常使用
\item 用 \verb`:e <filename>` 打开文件.
\item \verb`gd` 跳到光标所在变量的声明处, 其实是搜索最早出现的地方.
\item vim 中可以查看正在被程序写入的文件, 但不要修改! \verb`:edit` 命令可以刷新
\item linux 的 \verb`file -bi <filename>` 命令可以检测文件的 encoding,注意 UTF8 文件首位要有 BOM (Byte Order Mark) 该命令才能检测到 UTF8。
\item vim 默认的 encoding 是 latin1,或者系统 environment variable 的设置. 如果是 UTF8, 有时候不会自动加 BOM, 需要在 vim 里面用 \verb`:set bomb` 才行, BOM 有时候会产生一些问题, 例如 gcc (GNU compiler collection) 不接受 BOM, 又例如 cat 两个文件的时候第二个文件的 BOM 也会复制到第一个文件的结尾.
\item 要检查当前打开文件被识别的 encoding,用 \verb`:set fileencoding`.
\item 要显示行号,用 \verb`:set number`. 要隐藏,用 \verb`:set nonumber`. 如果要默认显示行号,在 \verb`~/.vimrc` 文件中输入 \verb`set number` 即可
\end{itemize}

\subsection{NERDtree}
\begin{itemize}
\item \verb`?` 可以查看简单的说明
\item \verb`u` 可以将 root 移动到上一层
\item \verb`U` 相当于 \verb`u` 但是保持打开的子目录
\item \verb`C` 将选中的目录作为 root
\item \verb`q` 退出
\end{itemize}
