% EPR 佯谬与定域隐变量理论
% EPR佯谬|量子力学|物理实在|定域|隐变量

\pentry{量子力学的基本原理(量子力学)\upref{QMPrcp},自旋角动量\upref{Spin}}

1935年,Einstein,Podolsky 和 Rosen 发表一篇名为《Can Quantum Mechanical Description of Physical Reality be Considered Complete》文章,以定域实在论为出发点论证量子力学的不完备性,以佯谬的形式对量子力学的哥本哈根诠释提出了批评.在这篇论文中 Einstein 谈到:“我不相信上帝会掷骰子.”(原文是“I can't believe that God plays dice”.) 

EPR 佯谬是基于这样一个思想实验\footnote{后来 Bohm 对这个思想实验进行了改进,使之具有更直观的物理图像,也就是我们现在看到的版本.}.我们有一个粒子源,它能够同时产生两个自旋 $1/2$ 的粒子(为了方便表述,下面假定它为电子).单个电子的自旋方向是不确定的,但是粒子源保证了这两个电子的自旋总是相反的,即双电子体系的总自旋为 $0$.具体而言,双粒子波函数可以写为
\begin{equation}
\ket{\Psi^-}=\frac{1}{\sqrt{2}}(\ket{\uparrow}\ket{\downarrow}-\ket{\downarrow}\ket{\uparrow})
\end{equation}

\begin{figure}[ht]
\centering
\includegraphics[width=14cm]{./figures/EPR_1.png}
\caption{EPR 佯谬思想实验} \label{EPR_fig1}
\end{figure}
