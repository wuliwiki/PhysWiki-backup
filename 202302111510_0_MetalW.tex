% 导体中的电磁波
% keys 金属|电磁波|导体|电磁场|趋肤深度|相位差

\pentry{电场波动方程\upref{EWEq}}
\begin{figure}[ht]
\centering
\includegraphics[width=10cm]{./figures/MetalW_1.pdf}
\caption{导体中的平面电磁波示意图。注意它与真空电磁波\upref{VcPlWv}的差异(衰减、电场磁场不同相)} \label{MetalW_fig1}
\end{figure}

\footnote{参考 David Griffiths 的电动力学导论} 良导体的情况下, 设任何净电荷消散的时间都非常快, 可以认为 $\rho_f = 0$。 另外自由电流仅由自由电子在电场中运动产生, $\bvec j_f = \sigma \bvec E$。 代入介质中的麦克斯韦方程组得波动方程
\begin{equation}
\laplacian \bvec E = \epsilon \mu \pdv[2]{\bvec E}{t} + \mu\sigma \pdv{\bvec E}{t}
\end{equation}

\begin{example}{简要的推导}
% \pentry{}
因为$\curl\bvec E = -\pdv{\bvec B}{t}$, $\curl\bvec B = \mu \bvec j_f + \epsilon\mu \pdv{t} \bvec E$
\begin{equation}
\curl(\curl\bvec E) = -\pdv{t} (\curl\bvec B) =-\pdv{t}(\mu \bvec j_f + \epsilon\mu \pdv{t} \bvec E)
\end{equation}
代入$\bvec j_f = \sigma \bvec E$
\begin{equation}
\curl(\curl\bvec E) = -\pdv{t} (\curl\bvec B) =-\pdv{t}(\mu \sigma \bvec E + \epsilon\mu \pdv{t} \bvec E)
\end{equation}
参考\upref{EWEq},即可求解。
\end{example}


设电磁场的形式为
\begin{equation}
\bvec E = \uvec x \tilde E_0 \E^{\I (\tilde k z - \omega t)}
\qquad
\bvec B = \uvec y \tilde B_0 \E^{\I (\tilde k z - \omega t)}
\end{equation}
代入波动方程得
\begin{equation}
\tilde k^2 = \epsilon\mu \omega^2 + \I \sigma \mu \omega
\end{equation}
令 $\tilde k = k + \I \kappa$, 得
\begin{equation}
k = \omega\sqrt{\frac{\epsilon\mu}{2}} \sqrt{\sqrt{1 + \qty(\frac{\sigma}{\epsilon\omega})^2} + 1},
\qquad
\kappa = \omega\sqrt{\frac{\epsilon\mu}{2}} \sqrt{\sqrt{1 + \qty(\frac{\sigma}{\epsilon\omega})^2} - 1}
\end{equation}
对于良导体, $\sigma \gg \epsilon\omega$, 有
\begin{equation}
k = \kappa = \omega \sqrt{\frac{\mu\sigma}{2\omega}}
\end{equation}
电磁场变为
\begin{equation}
\bvec E = \uvec x \tilde E_0 \E^{-\kappa z} \E^{\I (kz - \omega t)}
\qquad
\bvec B = \uvec y \tilde B_0 \E^{-\kappa z} \E^{\I (kz - \omega t)}
\end{equation}
定义\textbf{趋肤深度(skin depth)}为
\begin{equation}
d = 1/\kappa
\end{equation}
使用 $\curl \bvec E = -\partial \bvec B / \partial t$ (这里 $\grad \equiv \I \tilde k \uvec z$)
\begin{equation}
\tilde B_0 = \tilde k \tilde E_0 / \omega
\end{equation}
所以电场磁场存在相位差
\begin{equation}
\phi = \arg (\tilde k)
\end{equation}
