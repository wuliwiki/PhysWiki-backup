% 应力-应变曲线

\begin{issues}
\issueDraft
\end{issues}
\footnote{本文参考了Callister的Material Science and Engineering An Introdcution}

\subsection{应力-应变曲线}

\begin{figure}[ht]
\centering
\includegraphics[width=10cm]{./figures/SSCUR_1.pdf}
\caption{典型的应力-应变曲线示意图} \label{SSCUR_fig1}
\end{figure}

\begin{table}[ht]
\centering
\caption{材料力学常用概念}\label{SSCUR_tab1}
\begin{tabular}{|c|c|c|}
\hline
名称 & 英文名 & 描述\\
\hline
刚度 & Stiffness & 材料抵抗弹性变形的能力,弹性区间内应力-应变曲线的斜率,即杨氏模量$E$\\
\hline
屈服强度 & Yield Strength & 弹性区间内材料可承受的最大应力,$\sigma_y$\\
\hline
极限强度 & Tensile Strength & 材料可承受的最大应力,$\sigma_{TS}, \sigma_b$\\
\hline
硬度 & Hardness & “材料抵抗局域塑性变形”的能力,一般正比于强度 $Hardness \propto \sigma_{TS}$\\
\hline
延展性 & Ductility & 材料断裂时的应变或材料变形程度,一般以百分比记。延展性较好的材料是“韧(Ductile)”的,反之是“脆(Brittle)”的。$\%EL = \frac{l-l_0}{l} $\\
\hline
韧性 & Resilience, Toughness & 断裂材料所需要的能量,即应力-应变曲线围成的面积。$U = \int_0^\varepsilon \sigma \dd \varepsilon$\\
\hline
\end{tabular}
\end{table}

\subsection{材料的力学共性}
材料的力学性能大致上如下表所示。当然,具体各种材料的“个性”不同,不一定完全符合下表规律。例如,金属汞Hg在室温下甚至是液体。

\begin{figure}[ht]
\centering
\includegraphics[width=8cm]{./figures/SSCUR_2.pdf}
\caption{材料共性力学性能示意图,笔者在课程作业所绘} \label{SSCUR_fig2}
\end{figure}

\begin{table}[ht]
\centering
\caption{材料的共性}\label{SSCUR_tab2}
\begin{tabular}{|c|c|c|c|}
\hline
类型 & 刚度 & 屈服强度 & 延展性 & 密度 \\
\hline
金属、合金 & 高 & 高 & 高 & 高\\
\hline
陶瓷、玻璃 & 高 & 高 & 低 & 中\\
\hline
塑料 & 低 & 低 & 很高 & 低\\
% \hline
% 热固塑料 & 高于热塑塑料 & 高于热塑塑料 & 低于热塑塑料 & 低\\
\hline
\end{tabular}
\end{table}
