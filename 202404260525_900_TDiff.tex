% 全微分(简明微积分)
% keys 多元微积分|导数|偏导数|微分|全微分|全微分近似
% license Xiao
% type Tutor

\pentry{偏导数\nref{nod_ParDer}}{nod_18ad}

以二元函数为例,在偏微分的几何意义中,若 $z = f(x,y)$ 在某点 $(x_0, y_0)$ 附近的曲面光滑\footnote{“光滑”即可以进行任意多次求导,对于多元函数则是任意多次偏微分。},那么如果考虑一个足够小的区域,可以把曲面近似为平面。设平面方程为
\begin{equation}
z = c_0 + c_x(x - x_0) + c_y(y - y_0)~.
\end{equation}
当 $x=x_0$, $y=y_0$ 时显然有 $c_0 = f(x_0, y_0)$,求两个偏导,又有
\begin{equation}
c_x = \pdv{f}{x}~, \qquad c_y = \pdv{f}{y}~.
\end{equation}
令坐标增量为 $\Delta x \equiv x - x_0$, $\Delta y \equiv y - y_0$,  $\Delta z \equiv z - c_0$,则平面方程变为
\begin{equation}
\Delta z = \pdv{f}{x}\Delta x + \pdv{f}{y}\Delta y~.
\end{equation}
令增量为无穷小,即
 \begin{equation}
\dd{z} = \pdv{f}{x} \dd{x} + \pdv{f}{y} \dd{y}~,
\end{equation}
这就是\textbf{全微分}关系。全微分的意义是,从某一点开始向任意方向移动 $(\dd{x}, \dd{y})$,函数的增量等于只向 $x$ 方向移动 $\dd{x}$ 的增量加上只向 $y$ 方向移动 $\dd{y}$ 的增量。类似地, $N$ 元函数的全微分关系为
\begin{equation}\label{eq_TDiff_1}
\dd{z} = \sum_{i = 1}^{N} \pdv{f}{x_i} \dd{x_i}~.
\end{equation}
事实上,偏微分也可以理解为是由该式定义的。

\subsection{全微分近似}
类比一元函数的微分近似\upref{Diff} $\Delta y \approx \dv*{f}{x} \cdot \Delta x$, 若 $N$ 元函数各个变量的一阶偏导在一小块区域内变化不大,那么函数值的变化可近似为
\begin{equation}\label{eq_TDiff_6}\ali{
\Delta z &= f(x_1+\Delta x_1, \dots, x_N + \Delta x_N) - f(x_1, \dots, x_N) \\
&\approx \pdv{f}{x_1}\Delta {x_1} +\ldots + \pdv{f}{x_N}\Delta {x_N}~.
}\end{equation}

\begin{example}{测量误差}
测量一个边长各不相同的长方体的体积,若三边的测量值和最大测量误差分别为 $a, \sigma_a, b, \sigma_b, c, \sigma_c$ (假设不确定度远小于边长),求体积的最大测量误差 $\sigma_V$ 及最大相对误差 $\sigma_V/V$。

类比“一元函数微分”中的\autoref{ex_Diff_1}, 长方体的体积为 $V(a,b,c) = abc$, 由全微分近似得
\begin{equation}
\sigma_V \approx \pdv{V}{a} \sigma_a + \pdv{V}{b} \sigma_b + \pdv{V}{c} \sigma_c = bc \sigma_a + ac \sigma_b + ab \sigma_c~.
\end{equation}
相对不确定度为
\begin{equation}
\frac{\sigma_V}{V} \approx \frac{\sigma_a}{a} + \frac{\sigma_b}{b} + \frac{\sigma_c}{c}~.
\end{equation}
\end{example}
