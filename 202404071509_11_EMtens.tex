% 电磁场的能动张量
% keys 能动张量|诺特定理|电磁场
% license Xiao
% type Tutor

\pentry{电磁场的作用量\nref{nod_ElecS},电磁场的动量守恒、动量流密度张量\nref{nod_EBP}}{nod_13da}

我们继续使用自然单位制,令 $\mu_0=\epsilon_0=c=1$ 来简化表达。依照习惯,上下标使用希腊字母如 $\mu, \nu$ 时,取值范围为 $\{0, 1, 2, 3\}$;使用拉丁字母如 $i, j$ 时,取值范围为 $\{1, 2, 3\}$。约定闵氏时空度规为 $(-1,1,1,1)$。

\subsection{电动力学的守恒量}
根据经典场论\upref{classi} 中能动张量 ${T^\mu}_\nu$ 的定义(注意,由于这里采用的度规是 $(-1,1,1,1)$,所以和 \upref{classi} 中的定义差一个负号。)
\begin{equation}
T^\mu{}_\nu \equiv -\frac{\partial \mathcal L}{\partial (\partial_\mu \phi)} \partial_\nu \phi + \mathcal L \delta^\mu{}_\nu~,
\end{equation}
可以写出电磁场的能动张量
\begin{equation}\label{eq_EMtens_1}
T^\mu{}_\nu =-\frac{\partial \mathcal L}{\partial (\partial_\mu A_\rho)} \partial_\nu A_\rho + \mathcal L \delta^\mu{}_\nu~.
\end{equation}
根据诺特定理,能动张量对应着四个守恒流:
\begin{equation}
\partial_\mu T^\mu{}_\nu=0~.
\end{equation}
其中 $T^0{}_0$ 对应着哈密顿量密度,或者说电磁场的能量密度;$T^0{}_i$ 对应着电磁场的动量密度。我们已经推导过电磁场的能量密度为 $\mathcal H=\frac{1}{8\pi}(\bvec E^2+\bvec B^2)$ \upref{EBS},也推导过电磁场的动量密度为 $\mathcal P=\frac{1}{4\pi} \bvec E\times \bvec B$\upref{EBP}。下面可以通过推导 \autoref{eq_EMtens_1} 验证这些结论。

如果先假设没有场源,即电流4-矢量 $J^\mu = 0$,\textbf{自由电磁场}的拉氏量密度为
\begin{equation}
\begin{aligned}
\mathcal{L}&=-\frac{1}{16\pi}F^{\mu\nu} F_{\mu\nu}\\
&=-\frac{1}{16\pi}(\partial^\mu A^\nu-\partial^\nu A^\mu)(\partial_\mu A_\nu-\partial_\nu A_\mu)~,
\end{aligned}
\end{equation}
所以
\begin{equation}
\begin{aligned}
T^\mu{}_\nu&=-\frac{\partial \mathcal{L}}{\partial (\partial_\mu A_\rho)}\partial_\nu A_\rho+\mathcal{L}\delta^\mu{}_\nu\\
&=\frac{1}{4\pi}F^{\mu\rho}\partial_\nu A_\rho-\frac{1}{16\pi}\delta^\mu{}_\nu F^{\sigma\lambda}F_{\sigma\lambda}~.
\end{aligned}
\end{equation}
或者将 $\nu$ 指标上升:
\begin{equation}
T^{\mu\nu}=\frac{1}{4\pi}F^{\mu\rho}\partial^\nu A_\rho - \frac{1}{16\pi} \eta^{\mu\nu}F^{\sigma\lambda}F_{\sigma\lambda}~,
\end{equation}
然而现在的 $T^{\mu\nu}$ 是不对称的。为了使 $T^{\mu\nu}$ 成为对称张量,需要增添一项 $\partial_\rho \psi^{\mu\nu\rho}$($\psi^{\mu\nu\rho}$ 是某个三阶张量,且满足 $\psi^{\mu\nu\rho}=-\psi^{\mu\rho\nu}$,因此 $\partial_\mu T^{\mu\nu}=0$ 仍然成立),最终可以将 $T^{\mu\nu}$ 写为
\begin{equation}
\begin{aligned}
T^{\mu\nu}&=\frac{1}{4\pi}F^{\mu\rho}(\partial^\nu A_\rho-\partial_\rho A^\nu) - \frac{1}{16\pi} \eta^{\mu\nu}F^{\sigma\lambda}F_{\sigma\lambda}\\
&=\frac{1}{4\pi}F^{\mu\rho}F^\nu{}_\rho - \frac{1}{16\pi} \eta^{\mu\nu}F^{\sigma\lambda}F_{\sigma\lambda}~,
\end{aligned}
\end{equation}

现在来计算 $T^{\mu\nu}$ 的每一项。
\begin{equation}\label{eq_EMtens_2}
\begin{aligned}
T^{00}&=\frac{1}{4\pi}\bvec E \cdot \bvec E+\frac{1}{16\pi}(2\bvec B^2-2\bvec E^2)\\
&=\frac{1}{8\pi}(\bvec E^2+\bvec B^2)=\mathcal{H}~,
\end{aligned}
\end{equation}
\begin{equation}\label{eq_EMtens_3}
\begin{aligned}
T^{0i}=T^{i0}&=\frac{1}{4\pi}E^j F^i{}_j=\frac{1}{4\pi}\epsilon_{ijk}E_jB_k\\
&=\frac{1}{4\pi}\bvec E\times \bvec B=\mathcal{P}_i~,
\end{aligned}
\end{equation}
这些恰好对应着电磁场的能量与动量密度。
\begin{equation}\label{eq_EMtens_4}
\begin{aligned}
T^{ij}&=\frac{1}{4\pi}F^{i0}F^j{}_0+\frac{1}{4\pi}F^{ik}F^j{}_k-\frac{1}{16\pi}\eta^{ij}F^{\sigma\lambda}F_{\sigma\lambda}\\
&=-\frac{1}{4\pi}E_iE_j+\frac{1}{4\pi}\epsilon_{ikl}B_l \epsilon_{jkm}B_m-\frac{1}{16\pi}\delta_{ij}(2\bvec B^2-2\bvec E^2)\\
&=\frac{1}{4\pi}\qty(\frac{1}{2}\delta_{ij}\bvec E^2-E_iE_j)+\frac{1}{4\pi}\qty(\delta_{ij}\delta_{lm}-\delta_{im}\delta_{lj})B_lB_m-\frac{1}{8\pi}\delta_{ij}\bvec B^2\\
&=\frac{1}{4\pi}\qty(\frac{1}{2}\delta_{ij}\bvec E^2-E_iE_j)+\frac{1}{4\pi}\qty(\frac{1}{2}\delta_{ij}\bvec B^2 - B_iB_j)~.
\end{aligned}
\end{equation}
这与麦克斯韦应力张量 \autoref{eq_EBP_2}~\upref{EBP} 的形式是一致的。

最后根据方程 $\partial_\mu T^{\mu\nu}=0$ 可以得到 4 个守恒流方程:
\begin{equation}
\begin{aligned}
&\frac{\partial }{\partial t}\qty(\frac{1}{8\pi}(\bvec E^2+\bvec B^2))+\nabla \cdot \qty(\frac{1}{4\pi}\bvec E\times \bvec B)=0~,\\
&\frac{\partial}{\partial t}\qty(\frac{1}{4\pi}\bvec E\times \bvec B)+\partial_i T_{ij}=0~.
\end{aligned}
\end{equation}

现在考虑场源的影响。根据 \autoref{eq_ElecS_2}~\upref{ElecS},作用量的完整形式为
\begin{equation}
S=-\sum{m\int \dd \tau}+\int \qty(-\frac{1}{16\pi}F^{\mu\nu}F_{\mu\nu}+A_\mu J^\mu ){\dd}^4 x`~,
\end{equation}
所以拉格朗日量可以写为
\begin{equation}
\mathcal{L}=-\sum_i \frac{m_i\delta(\bvec r-\bvec r_i(t))}{\gamma}-\frac{1}{16\pi}F^{\mu\nu}F_{\mu\nu}+A_\mu J^\mu~.
\end{equation}
现在考虑单个静质量为 $m$ 的粒子对能动张量 $T^{\mu\nu}$ 的影响,设它的运动轨迹为 $\bvec r_0(t)$。设 $\epsilon=m\delta(\bvec r-\bvec r_0(t))$。根据能动张量的物理意义,$T_{\rm{part}}^{00}=\gamma \epsilon$,$T_{\rm{part}}^{0i}=T_{\rm{part}}^{i0}=\gamma \epsilon \frac{\dd x^i}{\dd t}$。由 $\gamma=\frac{\dd t}{\dd \tau}$,并根据 $F^{\mu\nu}$ 的洛伦兹协变性,可以得到单个带电粒子的能动张量的表达式:
\begin{equation}
T_{\rm{part}}^{\mu\nu}=\frac{\epsilon}{\gamma}\frac{\dd x^\mu}{\dd \tau}\frac{\dd x^\nu}{\dd \tau}~.
\end{equation}

设电磁场的能动张量为 $T_{\rm{e.m.}}^{\mu\nu}$,由\autoref{eq_EMtens_2} \autoref{eq_EMtens_3} \autoref{eq_EMtens_4} 给出;而总的能动张量由电磁场和粒子的能动张量两部分组成。现在重写 $\partial_\mu T^{\mu\nu}=0$,可以得到
\begin{equation}\label{eq_EMtens_5}
\begin{aligned}
\partial_{\mu}T^{\mu\nu}&=\partial_\mu T_{\rm{e.m.}}^{\mu\nu}+\partial_\mu T_{\rm{part}}^{\mu\nu}\\
&=\partial_\mu T_{\rm{e.m.}}^{\mu\nu}+\partial_\mu\qty(\frac{\epsilon}{\gamma}\frac{\dd x^\mu}{\dd \tau}\frac{\dd x^\nu}{\dd \tau})\\
&=\partial_\mu T_{\rm{e.m.}}^{\mu\nu}+\epsilon \frac{\dd x^\mu}{\dd t}\partial_\mu\qty(\frac{\dd x^\nu}{\dd \tau})+\frac{\dd x^\nu}{\dd \tau}\partial_\mu\qty(\epsilon \frac{\dd x^\mu}{\dd t})\\
&=\partial_\mu T_{\rm{e.m.}}^{\mu\nu}+\epsilon \frac{\dd x^\mu}{\dd t}\partial_\mu U^\nu+\frac{m}{q}U^\nu\partial_\mu J^\mu \\
&=\partial_\mu T_{\rm{e.m.}}^{\mu\nu}+\epsilon \frac{\dd U^\nu}{\dd t}=0~.
\end{aligned}
\end{equation}
这意味着电磁场的能量和动量会与粒子的能量动量发生交换,但\textbf{粒子与场的总能量和总动量是守恒的}。

进一步,根据洛伦兹力公式\autoref{eq_EMFT_3}~\upref{EMFT},
\begin{equation}
\epsilon \frac{\dd U^\nu}{\dd t}=\rho F^\mu{}_\nu U^\nu \frac{\dd \tau}{\dd t}=\rho F^\mu{}_\nu\frac{\dd x^\nu}{\dd t}=F^\mu{}_\nu J^\nu ~.
\end{equation}
代入 \autoref{eq_EMtens_5} 将得到
\begin{equation}\label{eq_EMtens_6}
\partial_\mu T_{\rm{e.m.}}^{\mu\nu}=-F^{\mu\nu}J^\nu~.
\end{equation}
等式左边包含两部分:一项是场的能量动量密度随时间变化率,另一项是能量动量流的散度;等式右侧代表的是电磁场对电荷作的功。因此 \autoref{eq_EMtens_6} 相当于电动力学的能量动量守恒律。
