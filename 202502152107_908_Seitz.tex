% 弗雷德里克·塞茨(综述)
% license CCBYSA3
% type Wiki

本文根据 CC-BY-SA 协议转载翻译自维基百科\href{https://en.wikipedia.org/wiki/Frederick_Seitz}{相关文章}。

弗雷德里克·塞茨(1911年7月4日-2008年3月2日)是美国物理学家,固态物理学的先驱,也是气候变化否认者。塞茨曾担任洛克菲勒大学第4任校长(1968年-1978年),以及美国国家科学院第17任院长(1962年-1969年)。塞茨获得了国家科学奖章、NASA杰出公共服务奖等多项荣誉。

他在伊利诺伊大学厄本那-香槟分校创立了弗雷德里克·塞茨材料研究实验室,并在美国其他地方创立了多个材料研究实验室。[1][2] 塞茨还是乔治·C·马歇尔研究所的创始主席。[3]
\subsection{背景和个人生活} 
塞茨于1911年7月4日出生在旧金山。他的母亲也来自旧金山,而以他命名的父亲则出生于德国。[4] 塞茨在高中最后一年中途从利克-威尔默丁高中毕业,随后进入斯坦福大学学习物理,并在三年内获得学士学位,[1] 于1932年毕业。[5] 他于1935年5月18日与伊丽莎白·K·马歇尔结婚。[6]

塞茨于2008年3月2日去世,享年96岁,地点为纽约。[7][8] 他留下了一子、三名孙子和四名曾孙。[7]
\subsection{早期职业生涯}
\begin{figure}[ht]
\centering
\includegraphics[width=6cm]{./figures/88a94b6bc3e7a33c.png}
\caption{Wigner–Seitz原胞的构建。} \label{fig_Seitz_1}
\end{figure}
塞茨前往普林斯顿大学,在尤金·维格纳的指导下研究金属,[1] 并于1934年获得博士学位。[7][9] 他与维格纳一起开创了最早的晶体量子理论之一,并在固态物理学中提出了诸如Wigner–Seitz单元格[1]等概念,该概念被用于晶体材料的研究。
\subsection{学术生涯}  
在研究生学习后,赛茨继续从事固态物理研究,并于1940年出版了《固体的现代理论》一书,目的是“写出一部关于固态物理各个方面的连贯论述,为该领域提供应有的统一性”。《固体的现代理论》帮助统一并理解了冶金学、陶瓷学和电子学等领域之间的关系。他还曾为许多与第二次世界大战相关的项目提供咨询,涉及冶金学、固体辐射损伤和电子学等方面。他与希拉德·亨廷顿共同首次计算了铜中空位和间隙原子形成与迁移的能量,激发了许多关于金属中点缺陷的研究。他的出版作品范围广泛,涵盖了“光谱学、光致发光、塑性变形、辐照效应、金属物理、自扩散、金属和绝缘体中的点缺陷以及科学政策”等多个领域。