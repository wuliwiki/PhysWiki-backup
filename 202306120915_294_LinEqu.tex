% 线性方程组
% keys 线性方程组|线性代数|二元一次|行变换|齐次|行列式

% \addTODO{需要补充一篇高中版本的线性方程组作为预备}

求解解线性方程组是线性代数学习中的一个主要线索。 虽然我们在中学阶段学过一些简单的技巧, 但要讨论最一般的情况和背后的数学结构, 我们还需要经历一个较为漫长的学习。

\subsection{低维简单例子}
\subsubsection{二元一次方程组}
在许多实际生活中,我们往往需要解决类似的问题:
%大体完成,还差插入图片与表达的细节优化
%未完成: littlefeng 打算以线性方程组,以及高斯消元法作为线性代数的引入。后续的矩阵初等变换,向量组以及矩阵的秩都和这个知识点息息相关

小明拿着家长给的 $100$ 元去超市买饮料。超市里有 $3$ 元钱一瓶的可口可乐和 $5$ 元钱一瓶的奶茶,最后他带着 $28$ 瓶饮料回家过暑假了。求问小明买了多少瓶可口可乐,多少瓶奶茶?

这个问题实际上是一个二元一次方程组的问题,我们设小明买了 $x$ 瓶可口可乐,$y$ 瓶奶茶,可以列出一个方程组:\begin{equation}
\leftgroup{
100 &= 3x + 5y & &(a)\\
28 &= x + y & &(b)\\~.
}\end{equation}
一个简单的解决方法是计算 $(a)-3 \cdot (b)$,得到 $16 = 0x + 2y$ 即 $y = 8$,进一步就知道 $x = 20$.

\subsubsection{三元一次方程组}
类似的在三维坐标系里面,考察三个平面:$S_1:x - 3y-2z=3$,$S_2:-2x+y-4z=-9$ 与 $S_3:-x+3y-z=-7$ 的交点。
那么这个问题等价于解决如下三元一次方程组:\begin{equation}
\leftgroup{
x - 3y - 2z &= 3\\
- 2x + y - 4z &= -9\\
- x + 3y - z &= -7\\~.
}\end{equation}
解得\begin{equation}
\leftgroup{
x &= 2\\
y &= -1\\
z &= 1\\~.
}\end{equation}
那么 $(2,-1,1)$ 就是三维坐标系中平面 $S_1$,$S_2$ 与 $S_3$ 的交点。
\subsection{一般情况下的定义}
一般的,形如:
\begin{equation}
\leftgroup{
&a_{1,1}x_1 + a_{1,2}x_2 + \dots + a_{1,n}x_n&=\quad &y_1\\
&a_{2,1}x_1 + a_{2,2}x_2 + \dots + a_{2,n}x_n&=\quad &y_2\\
&\qquad \qquad \dots  \qquad \dots \qquad  \dots\\
&a_{m,1}x_1 + a_{m,2}x_2 + \dots + a_{m,n}x_n&=\quad &y_m}\\~.
\end{equation}
的等式组统称为线性方程组,也可以根据其未知数的个数称为 $n$ 元一次方程组。

其中 $x_1\dots x_n$ 为 $n$ 个未知量,$y_1\dots y_m$ 与 $a_{1,1} ,a_{1,2}\dots a_{1,n},a_{2,1} \dots a_{n,m}$ 为给定的系数。(形如 $a_{i.j}$ 的系数表示它是方程组中第 $i$ 个方程的 $x_j$ 对应的系数,也即第 $i$ 行第 $j$ 列系数)

\subsection{求解线性方程组}
高斯消元法\upref{GAUSS}是求解线性方程通解的一个重要方法, 适用于所有类型的解, 包括无解, 唯一解和无穷多个解。 另外高斯消元法还引入了线性代数中一种重要的变换即\textbf{行变换}。

对于有唯一解的情况, 另一种方法是使用克拉默法则\upref{kramer}, 但计算量要大许多, 一般不对较大的方程组使用。

当全部 $y_i$ 都为零时, 我们把方程叫\textbf{齐次的(homogeneous)}(见\upref{GAUSS})。 对于 $N$ 元 $N$ 次线性齐次方程组, 系数矩阵的行列式\upref{Deter}可以用于判断方程的解的个数: 若行列式不为零, 则方程有无穷多个解; 若行列式为零, 则方程有唯一解。 对于非齐次 $N$ 元 $N$ 次方程, 若行列式不为零, 方程有唯一解; 若行列式为零, 方程有无穷多个解或者无解。 这主要是因为行列式可以用于判断系数方阵中的行或列是否线性无关。 % 链接未完成

若想彻底了解线性方程组的解的结构, 我们必须要从矢量空间的角度来理解\upref{LinEq}: 首先我们需要先学习矢量空间\upref{LSpace}, 子空间\upref{SubSpc}, 线性变换\upref{LinMap}, 矩阵的秩\upref{MatRnk}, 然后才能进一步讨论为什么在不同情况下方程组的解会有不同的结构。
