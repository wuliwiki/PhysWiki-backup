% 复流形的(余)切空间
% keys 复流形
% license Usr
% type Tutor

\pentry{复流形\nref{nod_CMani}}{nod_fd19}

\subsection{全纯切丛与反全纯切丛}
对于复流形 $(M, J)$(这里的 $M$ 是一个实流形,配备了 $J$ 的复结构),其上的 $J_p: T_p M \to T_p M$ 即限定在 $p$ 点的复结构 $J$,是 $p$ 点的切空间 $T_p M$ 的自同态。

考虑将切空间复化:$T_p M \mapsto T_p M \otimes \mathbb C$,即每个切矢量的“系数”可以从实数变为复数,则 $J_p$ 给出了:
\begin{equation}
J_p : T_p M \otimes C \to T_p M \otimes C ~.
\end{equation}

$J^2 = -1$ 给出了 $J$ 在 $T_p M$ 的特征值是 $\pm \mathrm i$,就可以对 $T_p M \otimes \mathbb C$ 的直和分解,对应到两个特征空间:
\begin{equation}
T_p M \otimes C = T_p^{1, 0} M \oplus T_p^{0, 1} M ~.
\end{equation}
 
