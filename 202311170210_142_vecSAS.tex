% 矢量空间的对称/反对称幂
% keys 对称幂|反对称幂
% license Xiao
% type Wiki

\begin{issues}
\issueTODO
\issueOther{可以对照张量的对称化和交错化\upref{SIofTe}进行阅读}
\end{issues}

\pentry{空间的张量积\upref{TPofSp},置换的奇偶性\upref{permu},群作用\upref{Group3}}

\subsection{作为子空间的对称/反对称幂}

向量空间 $V$ 的 $n$ 次张量幂 $V^{\otimes n}$ 上存在一个 $n$ 阶对称群 $S_n$ \upref{Perm}的\textbf{置换作用}:
\begin{equation}
\begin{aligned}
\rho(\sigma): V^{\otimes n} &\to V^{\otimes n}~, \\
\sum v_1 \otimes \dots \otimes v_n &\mapsto \sum v_{\sigma(1)} \otimes \dots \otimes v_{\sigma(n)}~.
\end{aligned}
\end{equation}

\begin{example}{}
考虑 $n = 2$,$S_2 = \mathbb{Z}/2\mathbb{Z} = \{e, (1 2)\}$,$\rho(e)$是恒等映射,而
\begin{equation}
\begin{aligned}
\rho((1 2)): V \otimes V &\to V \otimes V~, \\
\sum v_1 \otimes v_2 &\mapsto \sum v_2 \otimes v_1~.
\end{aligned}
\end{equation}
\end{example}

特别的,我们把 $n$ 次张量幂 $V^{\otimes n}$ 的不动点集 $(V^{\otimes n})^{S_n}$ (\autoref{def_Group3_2}~\upref{Group3})称为 $V$ 的 $n$ 阶\textbf{对称幂},记做 $\opn{Sym}^n(V)$ 或者 $S^n(V)$。

\begin{example}{}\label{ex_vecSAS_1}
考虑 $V = \mathbb{R}^2$;$\opn{Sym}^2(V) = \langle e_1 \otimes e_1, e_2 \otimes e_2, e_1 \otimes e_2 + e_2 \otimes e_1 \rangle$ 是一个三维矢量空间。
\end{example}

我们定义(向量的)\textbf{对称积}
\begin{equation}
\begin{aligned}
\cdot: V \times V &\to \opn{Sym}^2(V)~, \\
v \cdot w &= v \otimes w + w \otimes v~;
\end{aligned}
\end{equation}
因此\autoref{ex_vecSAS_1} 中 $\opn{Sym}^2(V)$ 的基可以写成 $e_1 \cdot e_1, e_2 \cdot e_2, e_1 \cdot e_2$。


类似的,也可以定义 $S_n$ \upref{Perm}的\textbf{(?)置换作用}:
\begin{equation}
\begin{aligned}
\rho(\sigma): V^{\otimes n} &\to V^{\otimes n}~, \\
\sum v_1 \otimes \dots \otimes v_n &\mapsto \sum v_{\sigma(1)} \otimes \dots \otimes v_{\sigma(n)}~.
\end{aligned}
\end{equation}

\begin{example}{}
考虑 $n = 2$,$S_2 = \mathbb{Z}/2\mathbb{Z} = \{e, (1 2)\}$,$\rho(e)$是恒等映射,而
\begin{equation}
\begin{aligned}
\rho((1 2)): V \otimes V &\to V \otimes V~, \\
\sum v_1 \otimes v_2 &\mapsto \sum v_2 \otimes v_1~.
\end{aligned}
\end{equation}
\end{example}

\subsection{作为商空间的对称/反对称幂}