% 群(综述)
% license CCBYSA3
% type Wiki

本文根据 CC-BY-SA 协议转载翻译自维基百科\href{https://en.wikipedia.org/wiki/Group_(mathematics)}{相关文章}。

\begin{figure}[ht]
\centering
\includegraphics[width=6cm]{./figures/3d977276c842b7f1.png}
\caption{魔方的操作构成了魔方群。} \label{fig_GroupM_1}
\end{figure}
在数学中,群是一个具有二元运算的集合,并满足以下约束条件:该运算是结合的,它具有单位元,并且集合中的每个元素都有逆元。  

许多数学结构都是具有其他性质的群。例如,整数在加法运算下构成一个无限群,该群由一个称为\textbf{1}的单一元素生成(这些性质以独特的方式刻画了整数)。

群的概念被提出,以统一方式处理许多数学结构,例如数、几何形状和多项式的根。由于群的概念在数学内外的多个领域中无处不在,一些作者将其视为当代数学的核心组织原则之一。  

在几何学中,群自然地出现在对对称性和几何变换的研究中:一个物体的对称性构成一个群,称为该物体的\textbf{对称群},而某种特定类型的变换构成一个更一般的群。\textbf{李群}在几何中的对称群中出现,也出现在粒子物理学的\textbf{标准模型}中。\textbf{庞加莱群}是一个李群,包含狭义相对论中时空的对称性。\textbf{点群}则用于描述分子化学中的对称性。

群的概念起源于对多项式方程的研究,最早由埃瓦里斯特·伽罗瓦在 1830 年代提出,他使用 群(法语:groupe)这一术语来描述方程根的对称群,这一概念如今被称为\textbf{伽罗瓦群}。随着来自数论、几何等其他领域的贡献,群的概念得到了推广,并在1870 年左右被正式确立。\textbf{现代群论}是一个活跃的数学学科,它研究群本身的性质。为了探索群,数学家引入了各种概念,以便将群分解为更小、更易理解的部分,例如子群、商群和单群。除了研究群的抽象性质之外,群论学者还研究群的具体表现方式,包括表示论(即群的表示)和计算群论的方法。对于有限群,已经发展出一整套理论,并最终在2004年完成了有限单群的分类。自 20世纪80年代中期以来,\textbf{几何群论}这一分支迅速发展,它将\textbf{有限生成群}视为几何对象进行研究,成为群论中的一个活跃领域。
\subsection{定义与示例}  
\subsubsection{第一个例子:整数}
一个常见的群是整数集
\[
\mathbb{Z} = \{\ldots, -4, -3, -2, -1, 0, 1, 2, 3, 4, \ldots\}~
\]
配备\textbf{加法运算} \((+)\) 。对于任意两个整数 \(a\) 和 \(b\),它们的和 \(a + b\) 仍然是整数;这个封闭性表明加法是整数集 \(\mathbb{Z}\) 上的一个二元运算。  

整数加法的以下性质构成了群的基本公理,并在下面的定义中得到了推广:  
\begin{itemize}
\item 结合律(Associativity)对于所有整数 \(a, b, c\),有:\((a + b) + c = a + (b + c)\)这意味着,无论是先将 \(a\) 与 \(b\) 相加再加上 \(c\),还是先将 \(b\) 与 \(c\) 相加再加上 \(a\),最终的结果相同。  
\item 单位元(Identity Element)对于任意整数 \(a\),有:\(0 + a = a \quad \text{且} \quad a + 0 = a\) \textbf{0}被称为\textbf{加法的单位元},因为它与任意整数相加都不改变该整数的值。  
\item 逆元(Inverse Element) 对于任意整数 \(a\),存在一个整数 \(b\),使得: \(a + b = 0 \quad \text{且} \quad b + a = 0\)这个整数 \(b\) 称为 \(a\) 的\textbf{加法逆元},通常记作 \(-a\)。  
\end{itemize}
整数集 \(\mathbb{Z}\) 连同加法运算构成了一个数学结构,该结构属于一类具有相似性质的更广泛的代数对象。为了更系统地理解这类结构,下面给出正式的定义。
\subsubsection{定义} 
一个群是一个\textbf{非空集合 }\( G \),配备一个\textbf{二元运算}(在此记作“\( \cdot \)”),该运算将 \( G \) 中的任意两个元素 \( a \) 和 \( b \) 组合,得到仍属于 \( G \) 的元素 \( a \cdot b \)。此外,该运算必须满足以下三个被称为\textbf{群公理}的条件:[5][6][7][a]  

结合律(Associativity) 

对于所有 \( a, b, c \in G \),有:\((a \cdot b) \cdot c = a \cdot (b \cdot c)\)这意味着运算的计算顺序不会影响最终结果。  

单位元(Identity Element) 

存在一个元素 \( e \in G \),使得对于任意 \( a \in G \),有:\(e \cdot a = a \quad \text{且} \quad a \cdot e = a\)该元素 \( e \) 是唯一的(见下文),称为单位元(或中性元)。  

逆元(Inverse Element)

对于 \( G \) 中的每个元素 \( a \),存在一个元素 \( b \in G \),使得:\(a \cdot b = e \quad \text{且} \quad b \cdot a = e\)其中 \( e \) 是单位元。对于每个 \( a \),这个元素 \( b \) 是唯一的(见下文),称为 \( a \) 的逆元,通常记作 \( a^{-1} \)。
\subsubsection{符号与术语} 
从形式上看,一个群是一个\textbf{有序对},由一个集合和该集合上的二元运算组成,并满足群公理。这个集合称为\textbf{群的底层集合},而该运算称为群运算或群律。  

因此,群和它的底层集合是两个不同的数学对象。为了避免繁琐的符号表示,通常会滥用符号,用同一个符号表示二者。这种做法也反映了一种非正式的思维方式:即认为群只是该集合的一个“扩展版本”,其附加的结构由群运算提供。  

例如,考虑\textbf{实数集} \( \mathbb{R} \),它配备了加法运算 \( a + b \) 和乘法运算 \( ab \):形式上,\( \mathbb{R} \)只是一个集合,\( (\mathbb{R}, +) \)是一个加法群,\( (\mathbb{R}, +, \cdot) \)是一个域(field)。然而,在实际使用中,通常直接用 \( \mathbb{R} \) 来表示这三种对象之一,而不作区分。  

在域 \( \mathbb{R} \) 中,加法群(additive group)是以 \( \mathbb{R} \) 为底层集合,并以加法 \( + \) 作为运算的群,即 \( (\mathbb{R}, +) \)。乘法群记作 \( \mathbb{R}^{\times} \),其底层集合是去掉零的实数集\( \mathbb{R}\setminus \{0\} \),其运算是乘法\( \cdot \),即 \( (\mathbb{R}^\times, \cdot) \)。

当群的运算使用加法表示时,通常称之为加法群。在这种情况下,单位元通常记作\textbf{0},元素 \( x \) 的逆元记作\(-x\)。同样,当群的运算使用乘法表示时,通常称之为乘法群。在这种情况下,单位元通常记作 \textbf{1},元素 \( x \) 的逆元记作 \( x^{-1} \)。在乘法群中,运算符号通常被省略,即直接用并置表示运算,例如 \( ab \) 代替 \( a \cdot b \)。  

群的定义并不要求对所有元素 \( a, b \in G \) 都满足:\(a \cdot b = b \cdot a\) 如果满足这个额外条件,则称该运算是交换的,并称该群为\textbf{阿贝尔群}。通常的惯例是:阿贝尔群可以使用加法记号(\( + \))或乘法记号(\( \cdot \))。非阿贝尔群仅使用乘法记号(\( \cdot \))。

对于元素不是数的群,常见的还有其他记号。例如:当群的元素是函数时,运算通常是函数复合,记作:\(f \circ g\)在这种情况下,单位元通常记作\textbf{id}(即恒等函数)。在更具体的情况中,例如几何变换群、对称群、置换群和自同构群,运算符号 \( \circ \)通常被省略,类似于乘法群的记法。此外,群的记号可能会有许多其他变体,具体取决于应用领域和数学上下文。
\subsubsection{第二个例子:对称群 }
在平面上,如果一个图形可以通过旋转、反射和平移的组合变换成另一个图形,则它们是全等的。任何图形都与自身全等。然而,一些图形不仅与自身全等,而且有多种不同的方式与自身全等,这些额外的全等变换称为对称。  

例如,一个正方形有八种对称变换,它们包括:
\begin{figure}[ht]
\centering
\includegraphics[width=14.25cm]{./figures/f32a15365d3f3335.png}
\caption{} \label{fig_GroupM_2}
\end{figure}
\begin{itemize}
\item 恒等变换:保持所有内容不变,记作\textbf{id}。  
\item 旋转:以正方形的中心为旋转点,顺时针旋转90°,记作\( r_1 \) 180°,记作 \( r_2 \)270°,记作 \( r_3 \) 反射:  
\item 关于水平轴和垂直轴的反射,分别记作 \( f_h \) 和 \( f_v \);关于两条对角线的反射,分别记作 \( f_d \) 和 \( f_c \)。
\end{itemize}
这些对称变换本质上是函数,它们将正方形上的一个点映射到对称变换后的对应点。例如:\( r_1 \):将一个点顺时针旋转90°,围绕正方形的中心。\( f_h \):将一个点关于水平中轴线反射。组合两个这样的对称变换,仍然得到另一个对称变换。因此,这些对称变换构成了一个群,称为四阶二面体群,记作\( D_4 \)。该群的\textbf{底层集合}就是上述八个对称变换,而群运算是函数复合。两个对称变换的组合是按照函数复合的方式进行的,即:先应用第一个变换 \( a \) ,再应用第二个变换 \( b \)。记作:\(b \circ a\)这表示先执行对称变换 \( a \),然后对结果再应用对称变换 \( b \)。这里采用的记号是从右到左的顺序,这是函数复合的标准记法。

凯莱表列出了所有可能的对称变换组合的结果。例如:先顺时针旋转 270°(\( r_3 \)),再进行水平反射(\( f_h \)),其结果与沿对角线反射(\( f_d \))的效果相同。用上述符号表示,在凯莱表中(通常以蓝色高亮):  
\[
f_h \circ r_3 = f_d~
\]
给定这组对称变换及其运算方式,可以如下理解群公理:  

二元运算(Binary Operation):函数复合是一种二元运算,即对于任意两个对称变换 \( a \) 和 \( b \),它们的复合运算 \( a \circ b \) 仍然是一个对称变换。例如:  
 \[
 r_3 \circ f_h = f_c~
\]  
这表示先进行水平反射 \( f_h \),然后顺时针旋转 270°(\( r_3 \)),其结果等同于沿副对角线反射 \( f_c \)。实际上,任意两个对称变换的组合仍然是某种对称变换,这可以通过凯莱表(Cayley table)进行验证。