% 连续正交归一基底
% keys 正交归一|delta 函数|内积

\pentry{傅里叶变换(指数)\upref{FTExp}, 矢量空间\upref{LSpace}}

\subsection{离散的函数基底}
本文使用狄拉克符号\upref{braket}. 在 “傅里叶级数(三角)\upref{FSTri}” 中, 我们介绍了正交归一函数基底的概念, 即把满足一定条件的一元函数的集合看作一个矢量空间\upref{LSpace}, 两个函数(矢量)的内积\upref{InerPd}定义为
\begin{equation}
\braket{f}{g} = \int_{-\infty}^{+\infty} f(x)^* g(x) \dd{x}
\end{equation}
其中 $*$ 表示复共轭, 如果空间中的函数都是实函数则可忽略.

该空间中的一组正交归一基底用狄拉克符号\upref{braket}表示为 $\ket{x_i}$ ($i = 1, 2,\dots$, 可以是有限个或无限个), 满足正交归一条件(\autoref{OrNrB_eq3}~\upref{OrNrB})
\begin{equation}
\braket{x_i}{x_j} = \delta_{i,j}
\end{equation}
若这组正交归一基底是完备的, 那么任何函数都可以分解为它们的线性组合
\begin{equation}
\ket{f} = \sum_i \ket{x_i}\braket{x_i}{f}
\end{equation}


\subsection{连续的函数基底}
我们是否可以用
