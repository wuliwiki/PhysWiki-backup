% 曲线的长度
% keys 曲线|积分|长度|导数|勾股定理

\pentry{定积分\upref{DefInt}}

直角坐标系中, 曲线通常用函数 $y(x)$ 描述. 我们在曲线上某点 $(x, y)$ 附近取一小段, 根据勾股定理, 它的长度为
\begin{equation}
\dd{l} = \sqrt{\dd{x}^2 + \dd{y}^2} = \sqrt{1 + \dot y^2} \dd{x}
\end{equation}
两边积分得
\begin{equation}\label{CurLen_eq1}
l = \int_{x_1}^{x_2} \sqrt{1 + \dot y^2} \dd{x}
\end{equation}
这就是曲线在区间 $[x_1, x_2]$ 的长度.

(例题未完成)
\begin{example}{抛物线}
对抛物线$y=\frac{x^2}{2p}$,取顶点$O(x=0)$作为弧的起点,对于横坐标为$x$的任意点$M$,我们有
\begin{equation}
\begin{aligned}
l&=\overset{\frown}{OM}=\frac{1}{p}\int_{0}^{x}\sqrt{x^2+p^2}\dd{x}\\
&=\frac{1}{p}\bigg[\frac{1}{2}x\sqrt{x^2+p^2}+\frac{p^2}{2}\ln(x+\sqrt{x^2+p^2})\bigg]\Bigg\lvert_{0}^{x}\\
&=\frac{x}{2p}\sqrt{x^2+p^2}+\frac{p}{2}\ln\frac{x+\sqrt{x^2+p^2}}{p}
\end{aligned}
\end{equation}

\end{example}