% 搜狗科学百科目录

\begin{issues}
\issueDraft
\end{issues}

按照\href{https://wuli.wiki/assets/sogou}{原目录}的排列顺序来进行分类。 目前做到 1135

\subsection{数学}
1016.同胚(图论)
1022.随机图
1119.离散对数
15.梅森素数
160.图形属性
172.坐标系
176.条形统计图
178.演员模型理论
186.追逃
1.和逆变器图
218.联系
240.五个室拼图
250.希尔伯特问题
259.随机数生成
26.正弦曲线
277.布尔电路
291.Immerman-Szelepcsényi 定理
292.伪随机数发生器定理
299.非逻辑符号
\subsubsection{数值计算}
1124.过拟合
188.分支界限
319.蒙特卡洛极化

\subsection{物理}
\subsubsection{经典力学}
105.经典物理学
1002.万有引力
1041.硬度
110.接触力
1110.G力
1114.横波
112.传输介质
117.经典统一场论
\subsubsection{电磁学}
1003.电子
101.法拉第常数
1075.楞次定律
1081.直流电
1097.无线电频率
1112.电极
116.静电感应
\subsubsection{光学}
217.三棱镜
288.光度学
153.吸收 (光学)
249.散粒噪声
\subsubsection{热力学}
1215.工程热力学
167.可逆过程
199.吸热过程
255.蒸馏
\subsubsection{量子力学和场论}
100.双缝实验
1017.量子不和谐
1026.量子霍尔效应
1033.原子吸收光谱
142.斯托克斯位移
185.狄拉克之海
233.库珀对
183.英国国家量子技术项目
1026.量子霍尔效应
\subsubsection{原子分子}
138.分子轨道理论
1033.原子吸收光谱
\subsubsection{量子信息}
241.量子密码术
315.量子密钥分配
\subsubsection{粒子物理与核物理}
1217.粒子物理学
1000.夸克
1001.中子
1005.质子
1023.中微子
1027.费米子
201.玻色子
1146.离子束
223.放射源
121.Α粒子
205.快质子俘获过程
1009.加尔加梅勒
1134.第四代反应堆
1147.轻水反应堆
1148.核裂变
1149.电离辐射
219.结合能量
139.核合成
1150.国际核事件分级表
1145.加压重水反应堆
1151.核电站
1153.惯性约束核聚变
1155.核安全
1159.核聚变
1160.托卡马克
160.GEM
1144.中子散射
1222.夸克胶子等离子体
1175.X 射线衍射技术


\subsection{天文学和宇宙学}
1120.太阳系外行星
211.天体物理学
215.事件视界望远镜
328.CfA2长城
1032.宇宙射线
1074.傅里叶变换红外光谱
1141.宇宙常数
\subsubsection{物理学家}
1004.欧内斯特·卢瑟福
1028.琼·费曼
1047.卡尔·萨根
242.利昂·库珀

\subsubsection{其他}
210.因果系统

\subsection{化学}
1006.十六烷值
1030.闪点
1038.碘化钾
1039.合金
1040.钾肥
1046.硝化棉
1063.丙烯腈一丁二烯苯乙烯共聚物
1067.氰化钾
1091.聚碳酸酯
1099.碳酸钾
109.砷化镓
1115.化学合成物
1133.美拉德反应
1186.尿素
1218.固体氧化物燃料电池
1220.离子液体
125.虫胶
130.玻璃碳
132.价键理论
133.氮族元素
141.碳族元素
146.反键轨道
150.有机化学
154.醋酸纤维素
157.酚醛树脂
161.聚邻苯二甲酰胺
165.冲洗
17.重量分析法
194.茴香醛
204.丁子香酚
213.本生灯
224.三氧化二砷
22.5族元素
230.氧化镁
235.甲基碘
236.双乙酰
25.玻璃纸
279.高斯轨道
280.机油
283.硅藻土
284.溶剂化
30.磷钨酸
310.电偶阳极
316.氢氧化钾
321.羟胺
323.反氢原子
324.愈创木酚

\subsection{材料}
200.真空用材料

\subsection{生物}
1065.生物学
1029.染料测序
1031.链霉菌
1070.大肠杆菌
1036.生物医学工程
1043.致癌物
1051.血压
1052.1型糖尿病
1054.细菌生长
1058.卵细胞
1062.生物圈
1072.生态位
1064.细胞质
1069.谷类
1071.生物不朽
1076.粪便移植
1086.菊苣根
1090.柠檬酸循环
1092.颈椎
1094.癌症中的DNA甲基化
1095.再生障碍性贫血
1096.杂交
1098.细胞损伤
1111.显微镜
1123.生物发光
1129.多莉
1130.线粒体DNA
1132.姜黄素
1179.植物化学物质
1136.DNA双螺旋结构
1174.大分子
1176.细胞生物学
1177.同位素标记
1178.电泳
1180.生理学
1181.遗传
1182.聚合酶链反应
1183.遗传学
1184.RNA干扰
1185.微生物学
1187.代谢
1210.基因线路
1211.自噬
1212.RNase_P
1214.核酶
1216.第一类内含子
1223.生态毒理学
1227.毒物兴奋效应
1228.胶原
13.子囊果
140.边缘效应
168.仿生计算
189.NF-κB
191.肽核酸
192.抗冻蛋白
193.醛缩酶
216.深海巨人症
21.木栓质
228.蛋白质列表
239.生态群落
247.苔藓植物
252.二氢叶酸还原酶
254.鱼类迁徙
258.生态稳定性
268.棘鱼纲
275.单核苷酸多态性
27.物种
282.细菌DNA结合蛋白
286.好氧生物
290.生物物理环境
295.底栖生物
298.自然平衡
301.群体感应
302.基因库
303.聚乳酸
307.人类与生物圈计划
309.原核生物
311.木栓形成层
317.生物电磁学
318.植物修复
31.半棘肌
322.复杂系统
326.生态演替
327.连合纤维

\subsubsection{生物学家}
1219.托马斯·罗伯特·切赫(Thomas R. Cech)

\subsection{医学}
103.亚甲蓝
104.血清学
1059.电解液
1079.癌
1080.选择性COX2抑制剂
1121.抗磷脂综合征
1126.胆脂瘤
1137. 放射治疗
115.远程医疗
1172.核磁共振成像
1189.胸部损伤
1193.骨盆骨折
1199.头部损伤
1200.颈椎过伸性损伤
120.己烯雌酚
12.传播途径
135.四环素
14.奎宁
248.Folding@home
256.皮炎
262.癌症分期
273.解剖学

\subsection{心理学}
1128.自闭症的原因
18.VTA
129.网络去抑制效应
234.标签游戏
287.彭罗斯楼梯

\subsection{环境}
1113.回收标志
225.诺里斯大坝
237.能源工程

\subsection{计算机}
\subsubsection{基础概念}
1034.电脑程序
1083.二进制代码
1049.计算机工程
1055.中间件
111.ASCII
158.并行计算
271.竞争条件
131.软件工程
162.日志文件
\subsubsection{数据结构与算法}
1014.银行家算法
1018.原地算法
1021.图论算法
1135.归约
1011.并行回火
1020.二叉搜索树
1025.合并排序
177.超级递归算法
184.分而治之算法
19.贪心算法
1138.普里姆算法
1139.深度优先搜索算法
1140. 排序算法
1167.进化算法
1213.期望最大化算法
143.分布式树搜索
187.集束搜索
196.拓扑排序
243.隐式图
281.迷宫求解算法
305.HCS聚类算法
24.遗传算法
\subsubsection{信息论}
1010.可计算
285.不可判定问题
\subsection{具体应用}
1057.二维码
1066.计算机视觉
1068.计算机辅助设计
1035.数字货币
23.比特币
289.授权证明
123.虚拟机
137.图像文件格式
113.ZIP 格式
147.HTPC
1161.机器翻译
164.计算机设计自动化
312.AVT统计滤波算法
325.外部存储器算法
136.微软安全软件
149.实时文本  
\subsubsection{编程语言}
10.运行时
1050.Scala
1073.Haskell(编程语言)
1127.R(编程语言)
1116.递归语言
128.指针(计算机编程)
\subsubsection{机器学习和人工智能}
261.人工智能
1157.机器学习
1019.可能近似正确学习
1045.聊天机器人
1169.人工神经网络
1164.神经信息处理系统大会
1168.贝叶斯网络
1131.自然语言理解
1206.Natural language processing
1207.NLP
1208.自然语言处理(NLP)
1045.聊天机器人
314.标签传播算法
\subsubsection{数据科学}
1048.数据科学
1087.数据采集
1088.数据挖掘
\subsubsection{操作系统}
181.POSIX
108.任务管理器
1117.命题演算系统
1118.原始递归函数
1125.最大长度序列
159.VPS VM
169.DOS
174.svchost.exe
208.BIOS
126.线程
127.速率单调
\subsubsection{计算机网络}
263.计算机网络
190.网络服务器
203.网络拓扑结构
1143.超文本传输协定
114.ASP.NET
156.apache
\subsubsection{通信}
320.光纤电缆
2.信息和通信技术
278.调频广播
221.差分GPS技术
226.载波相位差分技术
\subsection{计算机科学家}
102.阿兰·图灵
244.约翰·冯·诺依曼
1156.李飞飞
1158.吴恩达
1154.周志华
\subsubsection{计算机硬件}
11.曼彻斯特马克1
144.曼彻斯特宝贝
180.ENIAC 197.麦金塔电脑
227.DDR2 SDRAM
\subsubsection{其他}
107.对象模型
1085.数模转换器
1007.计算历史
1024.信息理论的历史
1012.单向函数
1013.可满足模理论
1122.正则语言的泵引理
148.异步系统
272.美国计算机学会IEEE计算机科学逻辑专题研讨会
118.磁盘格式化
173.M-实验室
145.形式规范
1188.搜狗科学百科
265.信息寻求行为
198.统计静态时序分析
274.古特曼方法
170.形式等效性检查
257.百万美元主页
175.计算机科学中的逻辑

\subsection{交通}
1089.交通事故重建
1190.碰撞缓冲区
1191.防撞性
1194.新车碰撞测试
1196.行人安全设计
1198.Euro NCAP
1201.主动安全性
1202.车辆救援
1203.碰撞模拟
1204.安全带
1205.安全气囊
214.汽车工程
1197.立柱

\subsection{地球物理}
1037.气象学
1078.大气层

\subsection{工程与制造}
1008.工业CT
106.测量仪器
1082.轴承(机械)
1084.航空航天制造商
1053.损伤力学
1152.工业射线照相术
269.电磁制动器
276.磁异常探测器
1195.夹层玻璃
1209.高档数控机床
1221.人造鳃
151.光离子化检测器
206.传感器
209.滑动轴承
229.光电传感器
266.机器元素
293.空速指示器
297.双止回阀
161.精益六西格玛
260.机电学


212.皮埃尔·伽桑狄(哲学家)	   232_nfo 
251.环路图 (工业制造?)    304.多模光纤
