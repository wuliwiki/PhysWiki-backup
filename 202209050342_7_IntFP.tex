% 场与粒子的相互作用
% 场论|相互作用|interaction|电磁场

\pentry{自由粒子拉格朗日函数(狭义相对论)\upref{FPLSR}}

本文中$c=1$,闵可夫斯基空间度规为$\opn{diag}(1, -1, -1, -1)$.

本文的目的是简单讨论描述场与粒子相互作用的拉格朗日方法,思路启发自Suskkind的特别要强调,我们并不是把粒子和场分开讨论,得到两种作用量或者拉格朗日函数;我们研究的是粒子和场构成的整体,只有对这个整体的\textbf{一个}作用量.

\subsection{场对粒子的作用}

粒子的运动轨迹由拉格朗日函数决定,因此要体现场对粒子的作用,就需要在粒子的拉格朗日函数里有场的出现.

自由粒子的拉格朗日函数\upref{FPLSR}为$L(t, x^i, \dot{x}^i) = -m\sqrt{1-\dot{x}^i\dot{x}^jg_{ij}}$.如何添加一个“相互作用”的项呢?注意到和粒子有关的部分是$\sqrt{1-\dot{x}^i\dot{x}^jg_{ij}}$,我们带着这部分,简单地把标量场$\phi$加进去试试:
\begin{equation}\label{IntFP_eq1}
\mathcal{L}(t, x^i, \dot{x}^i) = -(m+g\phi)\sqrt{1-\dot{x}^i\dot{x}^jg_{ij}}
\end{equation}
其中$g$是一个常数,常称为\textbf{耦合常数(coupling constant)},用来表征场对粒子的作用强度.

把\autoref{IntFP_eq1} 代入粒子的欧拉-拉格朗日方程,得到粒子的运动方程
\begin{equation}\label{IntFP_eq2}
\frac{\dd }{\dd t} \qty[-\qty(m+g\phi)\frac{-\dot{x}^k}{\sqrt{1-\dot{x}^i\dot{x}^jg_{ij}}}] = -g\frac{\partial \phi}{\partial x^k}\sqrt{1-\dot{x}^i\dot{x}^jg_{ij}}
\end{equation}

如果取\textbf{非相对论}和\textbf{弱场}极限,即$\dot{x}^k\ll c$且$g\phi$和$\dot{x}^i\dot{x}^jg_{ij}$相当,那么\autoref{IntFP_eq1} 化为
\begin{equation}\label{IntFP_eq3}
\mathcal{L}(t, x^i, \dot{x}^i) = -m-g\phi+\frac{1}{2}mv^2
\end{equation}
其中$v^2=\dot{x}^i\dot{x}^jg_{ij}$.

于是\autoref{IntFP_eq2} 化为、或者说\autoref{IntFP_eq3} 的欧拉-拉格朗日方程为
\begin{equation}
\frac{\dd}{\dd t} m\dot{x}^k = -g\frac{\dd \phi}{\dd x^k}
\end{equation}
这不正是
























