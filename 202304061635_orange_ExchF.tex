% 全同粒子的交换力

\subsection{交换“力”}
当波函数出现一定程度的重叠时,整个系统好像受到外力的作用:对全同波色子,这个力是吸引力,把粒子拉近;对全同费米子,这个力是排斥力,使得粒子相互远离。我们把这个“力”称为交换力,它尽管实际上并不存。交换力仅仅是对称性导致的一个几何结果。它也是仅存在于量子力学的现象,在经典力学当中并没有对应。

接下来我们承接着?? 的简化和规定,通过计算两个粒子距离平方的期待值,推导出交换“力”的原理。

\subsection{可分辨的两个粒子}
\begin{equation}
\langle x_1^2\rangle = \int x_1|\psi_A(x_1)|^2dx_1\int |\psi_B(x_2)|^2dx_2=\langle x^2\rangle_A~.
\end{equation}
类似的可得:
\begin{equation}
\langle x_2^2\rangle = \int |\psi_A(x_1)|^2dx_1\int x_2 |\psi_B(x_2)|^2dx_2=\langle x^2\rangle_B
\end{equation}
还有:
\begin{equation}
\langle x_1x_2\rangle = \int  x_1|\psi_A(x_1)|^2dx_1\int x_2 |\psi_B(x_2)|^2dx_2=\langle x\rangle_A\langle x\rangle_B
\end{equation}
可得两个粒子距离平方的期待值为:
\begin{equation}\label{eq_ExchF_7}
\langle (x_1-x_2)^2\rangle_d=\langle x^2\rangle_A+\langle x^2\rangle_B-2\langle x\rangle_A\langle x\rangle_B
\end{equation}
反过来并拓展,粒子 $ 1 $ 处在态 $\psi_B(\bvec r)$,粒子 $ 2 $ 处于 $\psi_A(\bvec r)$ 的情况也是一样的。

\subsection{两个全同的粒子}

\begin{align}
\langle x_1^2\rangle &= \frac{1}{2}\big[\int x_1^2|\psi_A(x_1)|^2dx_1\int |\psi_B(x_2)|^2dx_2\\
&+\int x_1^2|\psi_B(x_1)|^2dx_1\int |\psi_A(x_2)|^2dx_2\\ 
&\pm\int x_1^2\psi_A(x_1)\psi_B(x_2)^*dx_1\int \psi_B(x_1)\psi_A(x_2)^*dx_2\\
&\pm\int x_1^2\psi_B(x_1)\psi_A(x_2)^*dx_1\int \psi_A(x_1)\psi_B(x_2)^*dx_2\big]\\
&=\frac{1}{2}(\langle x^2\rangle_A+\langle x^2\rangle_B)
\end{align}
同理可得:
\begin{equation}
\langle x_2^2\rangle=\frac{1}{2}(\langle x^2\rangle_B+\langle x^2\rangle_A)
\end{equation}
尽管 $\langle x_1^2\rangle=\langle x_2^2\rangle$,但是:
\begin{align}
\langle x_1x_2\rangle &= \frac{1}{2}\big[\int x_1|\psi_A(x_1)|^2dx_1\int x_2 |\psi_B(x_2)|^2dx_2\\
&+\int x_1|\psi_B(x_1)|^2dx_1\int x_2|\psi_A(x_2)|^2dx_2\\ 
&\pm\int x_1\psi_A(x_1)\psi_B(x_2)^*dx_1\int x_2\psi_B(x_1)\psi_A(x_2)^*dx_2\\
&\pm\int x_1\psi_B(x_1)\psi_A(x_2)^*dx_1\int x_2\psi_A(x_1)\psi_B(x_2)^*dx_2\big]\\
&=\frac{1}{2}\left(\langle x\rangle_A\langle x\rangle_B+\langle x\rangle_B\langle x\rangle_A\pm\langle x\rangle_{AB}\langle x\rangle_{BA}\pm\langle x\rangle_{BA}\langle x\rangle_{AB}\right)\\
&=\langle x\rangle_A\langle x\rangle_B\pm|\langle x\rangle_{AB}|^2
\end{align}
上面推导中的\begin{equation}
\langle x\rangle_{AB}\equiv \int x\psi_Ax)\psi_B(x)^*dx
\end{equation}
因此我们最后得到了全同粒子的距离平方期待值为:
\begin{equation}
\langle (x_1-x_2)^2\rangle_\pm=\langle x^2\rangle_A+\langle x^2\rangle_B-2\langle x\rangle_A\langle x\rangle_B\mp 2|\langle x\rangle_{AB}|^2
\end{equation}
结合\autoref{eq_ExchF_7} 可得:
\begin{equation}
\langle (\Delta x)^2\rangle_\pm=\langle (\Delta x)^2\rangle_d\mp 2|\langle x\rangle_{AB}|^2
\end{equation}
