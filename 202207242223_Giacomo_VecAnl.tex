% 矢量分析总结
% 标量场|矢量场|梯度|散度|旋度|斯托克斯公式

\begin{issues}
\issueOther{待整合}
\end{issues}

\subsection{矢量分析公式列表}

令 $\bvec{a}, \bvec{b}, \bvec{c}, \bvec{d}$ 为 $\mathbb{R}^3$ 中的可微矢量场,$\phi$ 为 $\mathbb{R}^3$ 中的可微标量场,$\nabla$ 为矢量微分算子 $\hat{\bvec{x}}\partial/\partial{x}+\hat{\bvec{y}}\partial/\partial{y}+\hat{\bvec{z}}\partial/\partial{z}$.

\begin{equation}
\bvec{a}\cdot(\bvec{b}\times\bvec{c})=\bvec{b}\cdot(\bvec{c}\times\bvec{a})=\bvec{c}\cdot(\bvec{a}\times\bvec{b})
\end{equation}
\begin{equation}
\bvec{a}\times(\bvec{b}\times\bvec{c})=(\bvec{a}\cdot\bvec{c})\bvec{b}-(\bvec{a}\cdot\bvec{b})\bvec{c}
\end{equation}
\begin{equation}
(\bvec{a}\times\bvec{b})\cdot(\bvec{c}\times\bvec{d})=(\bvec{a}\cdot\bvec{c})(\bvec{b}\cdot\bvec{d})-(\bvec{a}\cdot\bvec{d})(\bvec{b}\cdot\bvec{c})
\end{equation}
\begin{equation}
\nabla\times(\nabla\phi)=\bvec{0}
\end{equation}
\begin{equation}
\nabla\cdot(\nabla\times\bvec{a})=0
\end{equation}
\begin{equation}
\nabla\times(\nabla\times\bvec{a})=\nabla(\nabla\cdot\bvec{a})-(\nabla\cdot\nabla)\bvec{a}
\end{equation}
\begin{equation}
\nabla\cdot(\phi\bvec{a})=(\nabla\phi)\cdot\bvec{a}+\phi\nabla\cdot\bvec{a}
\end{equation}
\begin{equation}
\nabla\times(\phi\bvec{a})=(\nabla\phi)\times\bvec{a}+\phi\nabla\times\bvec{a}
\end{equation}

\begin{equation}\label{VecAnl_eq15}
\nabla(\bvec{a}\cdot\bvec{b})=(\bvec{a}\cdot\nabla)\bvec{b}+\bvec{b}\times(\nabla\times\bvec{a})+(\bvec{b}\cdot\nabla)\bvec{a}+\bvec{a}\times(\nabla\times\bvec{b})
\end{equation}

\begin{equation}
\nabla\cdot(\bvec{a}\cdot\bvec{b})=\bvec{b}\cdot(\nabla\times\bvec{a})-\bvec{a}\cdot(\nabla\times\bvec{b})
\end{equation}

\begin{equation}
\nabla\times(\bvec{a}\times\bvec{b})=(\bvec{b}\cdot\nabla)\bvec{a}-(\bvec{a}\cdot\nabla)\bvec{b}+\bvec{a}(\nabla\cdot\bvec{b})-\bvec{b}(\nabla\cdot\bvec{a})
\end{equation}






\subsection{标量场和矢量场}

\textbf{场(field)}是指实线性空间 $V$\footnote{或者更一般地,流形 $M$.实线性空间是\textbf{流形}\upref{Manif}的特例.}

\subsubsection{标量场}
\begin{equation}\label{VecAnl_eq1}
\Phi=\Phi(x,y,z)
\end{equation}
 $\Phi$ 的数值是空间位置的函数
 等值面
\begin{equation}\label{VecAnl_eq2}
\Phi=C
\end{equation}
 例如气压场、温度场.
\subsubsection{矢量场(详见??)}%\upref{Vfiel}) % 未完成
\begin{equation}\label{VecAnl_eq3}
\bvec {A}= \bvec{A}(x,y,z)
\end{equation}
$\bvec{A}$ 的大小、方向是空间位置的函数.
例如速度场 $\bvec{v}$、电场 $\bvec{E}$.

\textbf{场线}: 有方向的曲线,其上每一点切线方向都与 $\bvec{A}$ 的方向一致.

\textbf{场管}: 由一束场线围城的管状区域.

\subsection{标量场的梯度}%(详见\upref{Grad} 未完成
\subsubsection{方向微商}
\begin{equation}\label{VecAnl_eq4}
\frac{\partial \Phi}{\partial l}=\lim_{\Delta l \to 0}\frac{\Delta \Phi}{\Delta l}
\end{equation}
标量场 $\Phi$ 在P点沿 $\Delta \bvec{l} $ 方向的方向微商.

\subsubsection{标量场 $\Phi$ 的梯度}
沿方向微商最大的方向(即 $\Delta \bvec{n}$ 方向).
\begin{equation}\label{VecAnl_eq5}
\grad \Phi = \frac{\partial \Phi}{\partial n}
\end{equation}
$\Delta \Phi$ 方向总于 $\Phi$ 的等值面垂直.

标量场的梯度是矢量场

电势 $U$ 是标量场,其负梯度 $\bvec{E}$ 是矢量场.

\subsection{矢量场的通量和散度、高斯定理}%(\upref{Divgnc})
\subsubsection{定义}
通量
\begin{equation}\label{VecAnl_eq6}
\Phi_A=\iint\limits_{(S)} \bvec{A} \dd{\bvec S} = \iint\limits_{(S)} A\cos\theta dS
\end{equation}
流速场、流量、电通量、磁通量.

\subsubsection{散度}
\begin{equation}\label{VecAnl_eq7}
div \bvec{A}=\nabla \cdot \bvec{A}=\lim_{\Delta V \to 0}\frac{\Phi}{\Delta V}
\end{equation}
矢量场的散度是标量场.

\subsubsection{坐标表示}
\begin{equation}\label{VecAnl_eq8}
\nabla \cdot \bvec{A}= \frac{\partial A_x}{\partial x}+\frac{\partial A_y}{\partial y}+\frac{\partial A_z}{\partial z}
\end{equation}
\begin{equation}\label{VecAnl_eq9}
\oint \bvec{A}=\iiint_V \nabla \cdot \bvec{A} dV
\end{equation}
矢量场通过任意闭合曲面S的通量等于它向包围体积V内的散度积分.
\subsection{矢量场的环量和旋度、Stokes定理}%(\upref{Curl})

\subsubsection{定义}
环量 $\Gamma$
\begin{equation}\label{VecAnl_eq10}
\Gamma=\oint_L \bvec{A} \cdot d \bvec{l}
\end{equation}
矢量场 $\bvec{A}$ 沿闭合回路线积分.

$\delta S$ 为闭线L包围面积,$\bvec{n}$ 为右旋单位法向量.

旋度 $\curl\bvec{A}$
\begin{equation}\label{VecAnl_eq11}
rot \bvec{A}=\nabla \times \bvec{A}=\lim_{\Delta S \to 0} \frac{\oint_L \bvec{A} d\bvec{l}}{\Delta S}
\end{equation}
矢量场的旋度仍是矢量场.

\subsubsection{坐标表示}
\begin{equation}\label{VecAnl_eq12}
\nabla \times \bvec{A}=
\begin{vmatrix}
\bvec{i} & \bvec{j} & \bvec{k} \\
\frac{\partial}{\partial x} & \frac{\partial}{\partial y} & \frac{\partial}{\partial z}\\
A_x & A_y & A_z
\end{vmatrix}
\end{equation}
\begin{equation}\label{VecAnl_eq13}
\nabla = \bvec{i}\frac{\partial}{\partial x}+ \bvec{j} \frac{\partial}{\partial}y + \bvec{k} \frac{\partial}{\partial z}
\end{equation}

\subsubsection{stokes 定理}
\begin{equation}\label{VecAnl_eq14}
\oint_L \bvec{A} d\bvec{l}=\iint_S (\nabla \times \bvec{A})\cdot d\bvec{S}
\end{equation}
矢量场在任意闭合回路L上的环量等于它为边界的曲面S上旋度的积分.

\subsection{矢量场的分类}
\subsubsection{有散场和无散场}
散度为0,即无源,为无散场;散度不为0,即有源,有散场.
\begin{equation}\label{VecAnl_eq24}
\nabla\cdot\nabla\times\bvec{A}=0
\end{equation}
知,任何矢量场的旋度永远是无散场.

任何无散场 $\bvec{B}$ 可表达成某矢量场的旋度
\begin{equation}\label{VecAnl_eq25}
\bvec{B}=\nabla\times\bvec{A} ,\nabla\cdot\bvec{B}=0
\end{equation}

\subsubsection{有旋场和无旋场}
旋度为0,为无旋场;反之为有旋场.
\begin{equation}\label{VecAnl_eq26}
\nabla\times\nabla\Phi=0
\end{equation}
任何标量场的梯度永远是无旋场.

任何无旋场 $\bvec{A}$ 可表示为某个标量场 $\Phi$ 的梯度
\begin{equation}\label{VecAnl_eq27}
\bvec{A}=\nabla\Phi,\nabla\times\bvec{A}=0
\end{equation}

\subsubsection{谐和场}
谐和场为某一矢量 $\bvec{A}$ 在某空间内既无散又无旋,由于其无旋,所以可以由势场表示:
\begin{equation}\label{VecAnl_eq28}
\bvec{A}=\nabla\Phi,\nabla\times\bvec{A}=0
\end{equation}
同样由于其为无散场,所以有:
\begin{equation}\label{VecAnl_eq29}
\bvec{B}=\nabla\times\bvec{A} ,\nabla\cdot\bvec{B}=0
\end{equation}
故可以导出Laplace方程:
\begin{equation}\label{VecAnl_eq30}
\nabla\cdot\nabla=\nabla^2
\end{equation}
谐和场的势函数满足Laplace方程.
