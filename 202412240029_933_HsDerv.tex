% 导数(高中)
% keys 导数|高中|变化|求导
% license Xiao
% type Tutor

\begin{issues}
\issueDraft
\end{issues}

在学习函数的性质时,曾经提到过函数的\aref{变化率}{sub_HsFunC_4}。其中,\textbf{平均变化率(average rate of change)}的概念类似于计算某段时间内的平均速度,反映了函数在一个区间内的整体变化趋势。当平均变化率所涉及的两个点逐渐靠近,直至几乎重合时,这一变化率便转化为描述局部变化的工具。此时,连线逐渐成为该点处的切线,而平均变化率也演变为\textbf{瞬时变化率(instantaneous rate of change)},即函数在该点的变化速率。瞬时变化率的另一个名称是\textbf{导数(derivative)}。

平均变化率提供了宏观的变化趋势,而导数则通过精确的数学方法刻画了局部的瞬时变化。导数的应用范围极其广泛,几乎所有涉及变化的领域都能发现它的踪迹。例如,在经济学中,导数用于分析股票价格的涨跌;在气象学中,它可以测量温度的变化速度;在生物学中,它帮助研究细胞分裂的速率。作为研究变化的强大工具,导数为人们提供了一种新的思维方式,帮助深入理解和处理动态问题。

导数的理论基础依赖于\textbf{极限(limit)},但由于高中阶段未涉及极限的具体内容,因此高中的导数的学习主要聚焦于以下三个方面:
\begin{enumerate}
\item 理解背景和实际意义:掌握导数的几何意义和实际应用场景,理解它与函数其他性质的关系;
\item 熟练计算:学习导数的常见计算规则,能够快速、准确地对复杂函数求导;
\item 解决实际问题:利用导数分析函数性质,在面对恒成立、不等关系等问题时,知道如何构造辅助函数来使用导数解决问题。
\end{enumerate}

导数不仅是理解函数变化的核心工具,也是未来微积分学习的重要基础。高中阶段对导数的熟练掌握,将为进一步研究函数极限与积分打下坚实的基础。由于内容庞杂,本篇主要集中介绍导数的背景、定义和几何含义,导数的\enref{计算}{HsDerB}和\enref{性质}{HsDerC}则另外介绍。

\subsection{导数}

\subsubsection{从理解速度开始}

在谈论导数的定义之前,先从一个生活中的常见情景入手:假设某人开车从家到商场,整个行程花了30分钟,行驶了15公里。通过小学就学过的计算,可以得出这段旅程的\textbf{平均速度}为15公里 ÷ 0.5小时 = 30公里/小时。这种计算方式,是对整个行程的总体描述,表示汽车“平均”每小时行驶的距离。然而,在实际驾驶过程中,汽车的速度并不是恒定的。坐车时可以看到仪表盘上的速度表,它显示的单位同样是“公里/小时”,但随着行驶情况变化,汽车可能需要减速、加速,甚至短暂停车,它显示的数字也会有时是10公里/小时,有时是80公里/小时,甚至可能短暂为0。这些读数反映的是汽车在某一时刻的速度,也被称为\textbf{瞬时速度}。

瞬时速度与平均速度的差别在于,前者描述的是某一具体时刻的车速,而后者是对整个行程的概括。这两种速度各有用处:平均速度简单直观,但忽略了旅途中车速的变化;瞬时速度则能精确反映某一时刻的状态,但若没有速度表,就没有办法知道瞬时速度。平均速度怎么计算已经介绍过了,那么速度表\footnote{下面探讨的都是电子式的速度表,机械式的原理略有差异。}是怎么测得瞬时速度的呢?

速度表自己带有一个脉冲信号,他会从0开始不断计数,而且计数间隔的时间非常短。车轮每转一圈,它就会记录下当前的脉冲数为$N$,然后开始重新计数。对于一台车,每两个脉冲间隔的时间$t$和车轮的直径为$d$都是确定的。速度表显示的值$v$如下计算:
\begin{equation}
v={\pi d\over tN}=\frac{\Delta y}{\Delta x}~.
\end{equation}
这里,$\Delta y$表示车轮转一周的路程,$\Delta x$表示对应的时间间隔。看上去,这个计算方式本质上仍然是之前熟悉的平均速度计算公式,只不过计算的是车轮转一周时的平均速度。但如之前所说,人们通常将速度表显示的值视为瞬时速度。那么,为什么这里的平均速度可以作为瞬时速度呢?

其实,把汽车的运动位置当成一个函数的话,瞬时速度就是这个函数的导数。而上面的问题其实就和导数的定义有关。对于函数的\aref{平均变化率}{def_HsFunC_3}$\displaystyle\frac{\Delta y}{\Delta x}$,如果区间$[a, b]$的两个端点距离非常近,这时一般$\Delta x,\Delta y$都会非常小。如果此时的二者的比值能够稳定存在的话,那么这个比值就被称为导数。

而在实际使用时,只要控制这两个端点,使得$\Delta x,\Delta y$都比较小,或者说小到计算的精度是可接受的,就可以近似地认为它们的比值是导数。在速度表的例子中,车轮周长通常在2米左右(即0.002千米),对于“公里/小时”这种单位的速度,速度表需要感知的最低车速约为1公里/小时。也就是说,对应的最小时间间隔约为0.002小时(即7.2秒)\footnote{因此,车速较低时,速度表的数值可能会有一定延迟,因为需要完成计数后才能更新,不过这时对于驾驶者影响也不大了。}。也就是说,对于日常使用而言,0.002这个数字已经足够小了,更不用说一般汽车行驶的速度都是几十公里每小时,时间间隔就更小了。因此,基本认为速度表输出的就是瞬时速度,而前面讲述的这个过程,也是几乎所有数字领域计算导数的原理。

\subsubsection{导数的定义}

其实,刚才的描述中,就提到了平均变化率与导数的最主要区别——$\Delta x,\Delta y$是不是非常小,而这也就是导数定义的核心部分。

\begin{definition}{导数}
若函数$f(x)$在某点$x_0$处的极限
\begin{equation}
\lim_{\Delta x\to 0}{f(x_0+\Delta x)-f(x_0)\over \Delta x}~.
\end{equation}
存在,则称其为$f(x)$在$x_0$处的\textbf{导数(derivative)},记作$f'(x_0)$。
\end{definition}

定义中的$f(x_0+\Delta x)-f(x_0)$代表函数值的增量,也就是平均变化率中的$\Delta y$的具体形式,严格对应于自变量的增量$\Delta x$。

导数的核心定义在于让变量 $\Delta x$ 趋于0,这一点与速度表中基于有限时间间隔的近似结果有着本质区别。在速度表中,时间间隔 $\Delta x = tN$ 是一个具体的物理量,尽管可以通过某些改进方法\footnote{例如,通过测量半周或更短车轮行驶距离对应的时间。}将 $\Delta x$ 缩小到非常小的值,从而提高实际精度,但它始终是一个确定的值。在区间 $(0, \Delta x)$ 内,总可以找到一个比 $\Delta x$ 更小的数。这种“非常小”的概念与数学中讨论的“无限接近于0”存在根本差异。

“$\Delta x$无限接近于0”时,这个$\Delta x$被称为\textbf{无穷小(infinitesimal)}\footnote{注意,无穷小并非负无穷大。}。关于“无穷小”有一个常见的疑问:$\Delta x$ 到底是否等于0?一方面,根据 $\displaystyle\lim_{\Delta x \to 0} \Delta x = 0$\footnote{这符合极限的直觉,也就是说如果一个数趋于0,那么它的极限就是0。},似乎 $\Delta x$ 是 0;另一方面,$\Delta x$ 出现在分母上,因此它应该是一个非零值,而这时似乎上面的定义又回到了物理近似的情况。

这一问题曾在历史上引发争议。在牛顿(Isaac Newton)和莱布尼茨(Gottfried Wilhelm Leibniz)建立微积分体系后,无穷小的概念便被广泛使用,但它们并未对这一概念作出精确的定义。这一模糊性引来了批评,例如贝克莱(George Berkeley)称无穷小为“幽灵般的量”。直到19世纪,魏尔施特拉斯(Karl Weierstrass)通过极限的语言为无穷小概念赋予了严谨的数学定义。

现在可以明确地讲:无穷小不是一个固定的数值,而是一个动态量——它的绝对值可以任意接近 0,但永远不为 0。这一概念使数学能够研究“接近”的行为,而不仅仅是“已经到达”的状态。因此,导数描述的是一个动态变化的“接近”行为。如果不论如何接近,导数都会变成同样值,就说这个导数存在。数学通过取极限得到导数的过程,得到了一个“毫无偏差,无限精度”的“平均变化率”。

\subsubsection{几何含义}

函数与其图像之间具有对应关系,而导数作为函数的一种重要性质,也在几何上具有明确的含义。

在研究函数图像时,常通过选取曲线上两个不同的点,并将它们连成一条直线以分析函数值的变化。这条直线被称为\textbf{割线(secant line)}。割线描述了两个点之间的平均变化率,其斜率正好等于函数值在这两点间变化的比值,即 $\Delta y = y_2 - y_1$ 与 $\Delta x = x_2 - x_1$ 之比。当两个点逐渐靠近直至“几乎重合”时,割线会逐步趋近于另一种特殊的直线:\textbf{切线(tangent line)}。

切线是与曲线在某点局部范围内唯一的“紧密贴合”的直线,它描述了曲线在这一点的瞬时变化趋势。“切线”一词的几何起源是直线与圆的关系:直线与圆仅有一个交点且该交点位于圆的一侧。而在函数图像的语境中,“相切”更多地意味着在某一极小范围内,直线与曲线只有一个交点而且变化趋势非常相近。值得注意的是,切线的这些性质与导数的定义密切相关——导数本质上是切线的斜率,它揭示了函数在某一点处的瞬时变化速率。

需要注意的是,与圆的切线不同,切线与曲线不总是只有一个交点。例如,函数 $f(x) = x^3 - x$ 在 $x = 1$ 处的切线同时经过点 $(1, 0)$ 和 $(-2, -6)$。此外,切线也未必始终位于曲线的一侧,例如同一函数在 $x = 0$ 处的切线,两侧的函数图像分别位于切线的上下方。

\begin{figure}[ht]
\centering
\includegraphics[width=8cm]{./figures/c406989f9499269d.png}
\caption{$f(x)=x^3-x$在$x=1$处的切线} \label{fig_HsDerv_1}
\end{figure}

\begin{figure}[ht]
\centering
\includegraphics[width=8cm]{./figures/131b5a866d54db49.png}
\caption{$f(x)=x^3-x$在$x=0$处的切线} \label{fig_HsDerv_2}
\end{figure}

在实际应用中,有时无法直接获取函数的完整图像,而只能通过一组离散的点近似函数的行为。在这种情况下,可以采用一种“以直代曲”的方法,用相邻两点连线构成的折线图来替代函数的原始图像。这一方法本质上是将离散点之间的连线斜率,也即平均变化率,视为函数在对应区间内原本的导数值。现实生活中的区间测速就是一个典型的例子。在区间测速中,车辆的平均速度是通过测量两个位置之间的时间和距离来计算的。虽然它不能代表某一具体时刻的瞬时速度,但它却能近似车辆的整体运动趋势。这种利用平均变化率近似导数值的思想,与割线向切线的过渡过程是一致的。

\subsection{导函数}

导数也是一个对应关系,即每个自变量都对应一个导数,因此他也是一个函数,这个函数称为\textbf{导函数}(不引起歧义时,简称为导数)。导函数和原本的函数是一一对应的,因此可以根据定义或求导方法,来求一个函数的导函数,这个过程就是\textbf{求导}。

\subsection{记号}

由于历史发展和人们长久以来的使用习惯,导函数逐渐衍生出了许多不同的记法。这些记法不仅仅是使用者的偏好选择,还与特定领域的需求和表达习惯密切相关。既为方便计算和推导,也为强调不同的数学概念。了解这些符号的使用,有助于理解求导这个运算,另外在未来见到时,也不至陌生,不要求完全掌握,看个眼熟就好。下面的符号针对函数$y=f(x)$:
\begin{itemize}
\item 拉格朗日记法——$y'$或$f'(x)$,好处是记法比较简洁,便于书写,缺点是难以表达较为复杂的关系。高中数学主要采用这种表示法。某点$x_0$处的导数记作$f'(x_0)$。
\item 莱布尼茨记法——$\displaystyle\frac{\dd y}{\dd x}$  或  $\displaystyle\frac{\dd}{\dd x}f(x)$,好处是在进行某些复杂运算时,分子与分母可以直接按照乘除法的规则来进行运算,降低推导的复杂度。另外,也在形式上代表着变化率。在大学阶段的数学领域主要采用这种表示方法。某点$x_0$处的导数记作$\displaystyle\eval{\dfrac{\dd y}{\dd x}}_{x=x_0}$ 或$\displaystyle\frac{\dd}{\dd x}f(x_0)$。
\item 牛顿记法——$\dot{y}$。由于在物理学中,时间是一个较为特别的变量,一般用这种方法来表示某个变量相对于时间的导数。基本只在物理学领域使用。
\item 重导数记法——$f_x$,这种记法简洁紧凑,又能在复杂关系出现时,避免拉格朗日记法的问题。在偏微分方程和张量分析中常用,尤其是对多重导数关系。
\item 欧拉记法——$Df(x)$,采用$D$算子。主要在大学阶段的微分方程中使用,好处是$D^n f(x)$可以直接修改$n$来表示进行几次求导运算。另外,将求导(微分)视为一种算子,便于与其他运算符进行组合,适合处理复杂的微分运算。
\item 差分导数——$\Delta f(x)$,对于定义域是离散的函数(一般是$\mathbb{Z}$),通常会用这样的符号来表示它的导数,称为\textbf{差分}。
\end{itemize}

在高中阶段,一般要求只使用拉格朗日记法,即$y'$或$f'(x)$,并且不允许使用其他记法。其余的记法可以这样理解:
\begin{itemize}
\item 用$D$运算符代替$\displaystyle\frac{\dd}{\dd x}f(x)$中的$\displaystyle\frac{\dd}{\dd x}$就成了$Df(x)$;
\item 把$\displaystyle\frac{\dd}{\dd x}f(x)$中最重要的两部分$f,x$拿出来显示就成了$f_x$;
\item 在明确自变量的情况下,只强调$\displaystyle\frac{\dd y}{\dd x}$中的$y$就成了$y'$;
\item $\dot{y}$与$y'$异曲同工,只是更着眼于“时间”;
\item $\Delta f(x)$与$Df(x)$只是因为定义域不同,处理方法不同;
\end{itemize}

