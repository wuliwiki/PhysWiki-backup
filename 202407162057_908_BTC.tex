% 比特币
% license CCBYSA3
% type Wiki

(本文根据 CC-BY-SA 协议转载自原搜狗科学百科对英文维基百科的翻译)

\textbf{比特币 (₿)}是一种加密货币,即一种电子现金。它是一种去中心化的数字货币,没有中央银行或单一管理员,可以在无需中介的情况下,在点对点比特币网络上从一个用户发送到另一个用户。

交易由网络节点通过加密进行验证,并记录在名为区块链的公共分散式账本中。比特币是由一个或一群未知晓的人用“中本聪”这个名字 发明的,并于2009年作为开源软件发布。 比特币是作为对采矿过程的奖励而创造的。它们可以被兑换成其他货币、产品和服务。[1] 剑桥大学开展的研究估计,在2017年里有290万至580万不同用户使用加密货币钱包,其中大多数使用比特币。

比特币因其在非法交易中的使用、高耗电量、价格波动性、交易中存在的偷窃以及其成为经济泡沫的可能性而受到批评。 比特币也被用作一种投资,尽管几个监管机构已经就比特币向投资者发出了警告。[2]

\subsection{历史}
\subsubsection{1.1 创造}
域名“bitcoin.org”于2008年8月18日被注册。[3] 2008年10月31日,中本聪撰写的一篇题为比特币:点对点电子现金系统 的论文的链接被邮寄到一个加密邮件列表上。[4] 中本聪将比特币软件作为开源代码实现,并于2009年1月发布。[5][6][7] 中本聪的身份仍然未知。[7]

2009年1月3日,当中本聪开采出这条链的第一个区块,即创世区块时,比特币网络就建立了。[7][8] 嵌在这一区块的货币基础中的是以下文字:“2009年1月3日《泰晤士报》财政大臣即将对银行进行第二次救助。”[7] 这张消息被解释为一个带有时间戳的对部分准备金银行业务造成的不稳定性的评论。[9]

第一笔比特币交易的接收者是密码朋克哈尔·芬尼,他在2004年创建了第一个可重复使用的工作量证明系统(RPOW)。[10] 芬尼在比特币软件发布之日下载了该软件,并于2009年1月12日收到了中本聪的10枚比特币。[11][12] 其他早期密码朋克的支持者是比特币前辈的创造者:bb-money的创造者戴伟和比特黄金的创造者尼克·萨伯。[7]2010年,第一次使用比特币的商业交易发生在程序员拉兹洛·汉尼茨花10,000比特币买了Papa John's的比萨饼的时候。[13]

据估计,中本聪在2010年消失之前已经开采了100万枚比特币,[14] 当时他将网络警报密钥和代码库的控制权交给了加文·安德列森。安德烈森后来成为比特币基金会的首席开发者。[15][16] 安德烈森随后寻求去中心化控制。这为比特币未来的发展道路留下了争议。[16]
\subsubsection{1.2 2011-2012年}
在早期的“概念验证”交易后,比特币的第一批主要用户是黑市,如丝绸之路。从2011年2月开始,丝绸之路在其存在的30个月中,只接受比特币作为支付,交易990万比特币,价值约2.14亿美元。

2011年,比特币的起价为每枚0.30美元,今年上涨至5.27美元。6月8日,价格升至31.50美元。不到一个月,价格降至11美元。第二个月跌至7.80美元,另一个月跌至4.77美元。[17]

Litecoin是比特币的早期衍生产品,于2011年10月问世。[18] 从那以后,许多另类硬币被创造出来。[19]

2012年,比特币价格从5.27美元开始上涨至13.30美元。[17] 截至1月9日,价格已升至7.38美元,但随后在接下来的16天内暴跌49\%,至3.80美元。8月17日,价格升至16.41美元,但在接下来的三天里下跌了57\%,至7.10美元。[20]

比特币基金会成立于2012年9月,旨在促进比特币的发展和认识。[21]
\subsubsection{1.3 2013-2016年}
2013年,价格从13.30美元开始,到2014年1月1日升至770美元。[17]

2013年3月,区块链暂时分裂成两条不同规则的独立链条。两个区块链同时运行了六个小时,每个小时都有自己版本的交易历史。当大多数网络降级到比特币软件的0.7版时,正常操作得以恢复。[22] 在接下来的几个小时里,比特币在恢复到之前的大约48美元的水平之前,Mt. Gox曾短暂停止比特币存款,导致比特币价格下跌了23\%,至37美元。[22][23] [24] 美国金融犯罪执法网(FinCEN)为比特币等“去中心化虚拟货币”制定了监管准则,将出售其生成的比特币的美国比特币矿商归类为货币服务企业(MSBs),这些企业需接受注册或其他法律义务。[25][26][27]4月份,由于容量不足,比特币交易所BitInstant和Gox经历了处理延迟,[28] 导致比特币价格从266美元跌至76美元,6小时内又回到160美元。[29] 比特币价格在4月10日升至259美元,但随后三天暴跌83\%,至45美元。[20] 2013年5月15日,美国当局在发现Mt. Gox没有在美国金融犯罪中心注册为汇款人后,查封了其相关账户。[30][31] 2013年6月23日,美国缉毒署根据《美国法典》第21篇第881条,在美国司法部扣押通知中将11.02枚比特币列为扣押资产。[32]这标志着政府机构首次没收比特币。[33] 2013年10月,在罗斯·威廉·乌尔布里切特被捕期间,美国联邦调查局从暗网丝绸之路缴获了约26,000枚比特币。[34][35][36]比特币的价格在11月19日升至755美元,当天暴跌50\%,至378美元。2013年11月30日,该价格在开始长期崩盘前达到1,163美元,2015年1月下跌87\%,至152美元。[20] 2013年12月5日,中国人民银行禁止中国金融机构使用比特币。[37] 宣布后,比特币的价值下跌,[38] 百度也不再接受比特币的某些服务。[39] 至少自2009年以来,在中国用任何虚拟货币购买现实世界的商品都是非法的。[40]

2014年,价格从770美元开始跌至314美元。[17]

2014年7月30日,维基媒体基金会开始接受比特币捐赠。[41]

2015年。价格从314美元开始上升到434美元。2016年,价格在2017年1月1日升至998美元。[17]
\subsubsection{1.4 2017-2018年}
价格从2017年的998美元开始,在2017年12月17日达到19,783.06美元的历史高点后,于2018年1月1日上涨到13,412.44美元。[17] [42]

中国于2017年9月开始禁止比特币交易,并于2018年2月1日开始全面禁止比特币交易。比特币价格随后在2018年2月5日从9,052美元跌至6,914美元。[20] 比特币在中国人民币交易中的比例从2017年9月的90\%以上降至2018年6月的不到1\%。[43]

在2018年上半年的剩余时间里,比特币的价格在11,480美元至5,848美元之间波动。2018年7月1日,比特币的价格为6,343美元。[44][45] 2019年1月1日的价格为3747美元,比2018年下降72\%,比历史最高水平下降81\%。[44][46]

比特币价格受到了几次涉及到加密货币的黑客攻击或盗窃的负面影响,包括分别对于2018年1月的coincheck、6月的coinrail和bithumb的盗窃以及7月的班科尔的盗窃。据报道,2018年前6个月,价值7.61亿美元的密码从交易被盗。[47] 即使其他加密货币在coinrail和bancor被盗,由于投资者担心密码货币交易所的安全性,比特币的价格也会受到影响。[48][49][50]

\subsection{设计}
\subsubsection{2.1 单位}
比特币系统的记账单位是比特币。用来代表比特币的股票代码是BTC 和XBT。[51] 它的唯一编码是₿.[52] 作为少量比特币的替代单位有毫比特币(mBTC)和satoshi(sat)。satoshi是为向比特币的创造者致敬而命名的,是比特币中最小的一种,代表0.00000001个比特币,相当于比特币的1亿分之一。[53] 一枚毫比特币等于0.001枚比特币;一枚比特币的千分之一或十万satoshis。[54]
\subsubsection{2.2 块状链}
\begin{figure}[ht]
\centering
\includegraphics[width=14.25cm]{./figures/e684c31b9a7ea6e2.png}
\caption{账本中区块的数据结构} \label{fig_BTC_1}
\end{figure}
比特币区块链是记录比特币交易的公共账本。[55] 它是作为一个块链来实现的,每个块包含一个直到链的起源块 的前一个块的散列。运行比特币软件的通信节点网络维护着区块链。[56]付款人X向收款人Z发送Y比特币的交易广播到使用现成的软件应用程序的网络。

网络节点可以验证交易,将其添加到账本的副本中,然后将这些账本的增量广播到其他节点。为了实现所有权链的独立验证,每个网络节点存储自己的区块链副本。[56] 大约每10分钟,一组新的被接受的事务,称为块,被创建、添加到区块链,并迅速发布到所有节点,而不需要中央监督。这使得比特币软件能够确定一个特定比特币何时被消费,而这是防止重复消费所必需的。传统的账本记录实际票据或本票的转移,但区块链是可以说比特币以未用交易产出的形式存在的唯一地方。
\subsubsection{2.3 处理}
事务是使用类似Forth的脚本语言定义的。 交易由一个或多个输入和一个或多个输出组成。当用户发送比特币时,用户在输出中指定每个地址以及发送到该地址的比特币数量。为了防止双重支出,每一项投入都必须参考区块链以前未用的输出。[57] 多种输入相当于在现金交易中使用多种比特币。由于交易可以有多个输出,用户可以在一次交易中向多个接收者发送比特币。与现金交易一样,输入(用于支付的比特币)的总和可能超过预期的支付总和。在这种情况下,使用额外的输出,将更改返回给付款人。[57] 任何未计入交易输出的输入satoshis都将成为交易费用。[57]\subsubsection{2.4 交易费用}
尽管交易费用是可选的,但矿工可以选择处理哪些交易,并优先处理那些支付较高费用的交易。[57] 矿工可以根据相对于其仓储规模支付的费用,而不是作为费用支付的绝对金额来选择交易。这些费用通常以每字节satoshi(sat/b)来衡量。事务的大小取决于用于创建事务的输入数量和输出数量。
\subsubsection{2.5 所有权}
\begin{figure}[ht]
\centering
\includegraphics[width=14.25cm]{./figures/b15d866e7b7bd6c2.png}
\caption{比特币白皮书中所述的所有权的简化链。[10] 实际中,一个交易可以有多个输入和输出。[9]} \label{fig_BTC_2}
\end{figure}
在区块链,比特币被注册到比特币地址。创建一个比特币地址只需要选择一个随机的有效私钥并计算相应的比特币地址。这个计算可以在一瞬间完成。但是相反,计算给定比特币地址的私钥在数学上是不可行的。用户可以告诉他人或公开比特币地址,而不会泄露其相应的私钥。此外,有效私钥的数量如此之大,以至于不太可能有人计算出已经在使用并有资金的密钥对。大量有效的私钥使得暴力破解私钥变得不可行。为了能够花掉他们的比特币,所有者必须知道相应的私钥并对交易进行数字签名。网络使用公钥验证签名;私钥永远不会泄露。

如果私钥丢失,比特币网络将无法识别任何其他所有权证据;[56] 然后比特币就不能用了,即丢失了。例如,2013年,一名用户声称丢失了7500枚比特币,当时价值750万美元,当时他意外丢弃了一个包含私人密钥的硬盘。[58] 据信,大约20\%的比特币丢失了。按2018年7月的价格计算,它们的市场价值约为200亿美元。[59]

为了确保比特币的安全性,私钥必须保密。如果私钥被泄露给第三方,例如通过数据泄露,第三方可以使用它来窃取任何相关联的比特币。截至2017年12月,约有98万枚比特币从密码货币交易所被盗。
\subsubsection{2.6 采矿}
\begin{figure}[ht]
\centering
\includegraphics[width=10cm]{./figures/66a5919eaf60a027.png}
\caption{早期的比特币矿工使用图形处理器(GPU)进行采矿,因为他们比中央处理器(CPU)更适合工作量证明的算法。} \label{fig_BTC_3}
\end{figure}
采矿是通过使用计算机处理能力完成的记录保存服务。 矿工通过反复将新广播的事务分组到一个块中来保持区块链的一致性、完整性和不可更改性,然后将该块广播到网络并由接收节点验证。[55] 每个块包含前一个块的SHA-256加密散列,[55] 因此将它链接到前一个块并给区块链命名。[55]

要被网络的其他节点接受,新块必须包含工作证明(PoW)。[55] 使用的系统是基于亚当·贝克1997年的反垃圾邮件方案哈斯查什。[58][60] 工作证明要求矿工找到一个称为随机数的数字,这样当块内容与随机数一起被散列时,结果在数字上小于网络的难度目标。 这种证明对于网络中的任何节点来说都很容易验证,但是生成这种证明非常耗时,因为对于安全的加密散列,矿工必须尝试许多不同的随机数(通常测试值的序列是升序自然数:0,1,2,3,...)在达到难度目标之前。
\begin{figure}[ht]
\centering
\includegraphics[width=10cm]{./figures/7c869435794a37a2.png}
\caption{后来的业余爱好者用特制的现场可编程门阵列(FPGA)和专用的集成电路(ASIC)芯片来挖矿。由于开采难度越来越大,图中的芯片已经过时。} \label{fig_BTC_4}
\end{figure}
每2016个块(大约14天,每个块大约10分钟),难度目标根据网络最近的性能进行调整,目的是将新块之间的平均时间保持在10分钟。这样,系统会自动适应网络上的总采矿电量。 在2014年3月1日至2015年3月1日期间,在创建新区块之前,挖掘随机数的矿工的平均人数从1640万亿亿增加到20050万亿亿。[61]

工作证明系统和区块链一起使得对区块链的修改变得极其困难,因为攻击者必须修改所有后续区块才使得对一个区块的修改被接受。[62] 随着新区块的不断开采,随着时间的推移,修改区块的难度会增加,因为随后区块的数量(也称为给定区块的确认)也会增加。[55]
\begin{figure}[ht]
\centering
\includegraphics[width=10cm]{./figures/e0821cf2bd77fd53.png}
\caption{如今,比特币矿业公司将用专门的设备来存放和运行大量的高性能采矿硬件。} \label{fig_BTC_5}
\end{figure}

\textbf{集合采矿}

计算能力通常被捆绑在一起或“集中起来”,以减少矿工收入的差异。单个采矿设备通常不得不等待很长一段时间来确认一笔交易并接收付款。在池中,每次参与的服务器解决一个块时,所有参与的矿工都会得到报酬。这笔钱取决于个体矿工帮助找到该区块的工作量。[63]
\begin{figure}[ht]
\centering
\includegraphics[width=10cm]{./figures/4a36f556e913aca6.png}
\caption{相对开采难度的半对数图[d][14]} \label{fig_BTC_6}
\end{figure}
\subsubsection{2.7 供给}
找到新区块的成功矿工可以用新创造的比特币和交易费奖励自己。[64] 截至2016年7月9日,[65] 每添加一个块到区块链,奖励金额为12.5个新创建的比特币。为了获得奖励,一项被称为硬币库的特殊交易包含在已处理的付款中。 所有现存的比特币都是在这种货币基础交易中产生的。比特币协议规定,增加一个区块的奖励将在每210,000个区块被挖掘后减半(大约每四年一次)。最终,奖励将降至零,2100万比特币的上限 将在公元2140年达到;然后,记录保存将仅由交易费用奖励。[66]

换句话说,比特币的发明者中本聪(Nakamoto)在比特币诞生之初就基于人为稀缺制定了一项货币政策,即比特币总量将只有2100万枚。它们大约每十分钟发布一次,它们的生成速度每四年下降一半,直到全部流通。[67]
\begin{figure}[ht]
\centering
\includegraphics[width=10cm]{./figures/b49d42aaac092ead.png}
\caption{流通中的比特币总量[14]} \label{fig_BTC_7}
\end{figure}
\subsubsection{2.8 钱包}
钱包存储交易比特币所需的信息。虽然钱包通常被描述为存放或比特币的地方,[68] 但由于系统的性质,比特币与区块链交易分类账密不可分。描述钱包的一个更好的方法是“存储您持有比特币的数字凭证”,并允许用户访问(和消费)它们。 比特币使用公钥加密技术,其中生成两个密钥,一个公钥,一个私钥。[69] 最基本的是,钱包是这些密钥的集合。

钱包有几种工作模式。它们在不可信和计算需求方面有负相关的联系。
\begin{itemize}
\item 完整客户端通过下载区块链的完整副本(截至2018年1月超过150 GB)直接验证交易。[70] 它们是使用网络最安全和可靠的方式,因为不需要信任外部方。完整的客户端检查挖掘块的有效性,防止它们在违反或改变网络规则的链上进行交易。 由于其大小和复杂性,下载和验证整个区块链并不适合所有计算设备。
\item 轻量级客户咨询完整的客户来发送和接收交易,而不需要整个区块链的本地副本。这使得轻量级客户端的设置速度更快,并允许它们在智能手机等低功耗、低带宽设备上使用。然而,当使用轻量级钱包时,用户必须在一定程度上信任服务器,因为它可能向用户报告错误值。轻量级客户遵循最长的区块链协议,但不能确保其有效性,这需要对矿工的信任。[71]
\end{itemize}
称为在线钱包的第三方互联网服务提供类似的功能,但可能更容易使用。在这种情况下,访问资金的凭证存储在在线钱包提供商处,而不是用户的硬件上。[72] 因此,用户必须完全信任在线钱包提供商。恶意提供商或服务器安全漏洞可能会导致托管比特币被盗。2011年发生在Mt. Gox的这种安全漏洞就是一例。[73]

实体钱包存储离线使用比特币所必需的凭证,并且可以像纸质打印的私钥一样简单。 物理钱包也可以采用金属代币的形式,[74] 在背面凹陷处的安全全息图下可以访问私钥。[75] 当从令牌中移除时,安全全息图自毁,显示私钥已被访问。[76] 起初,这些代币是用黄铜制成的,但后来随着比特币价格的上涨,使用了贵金属。[75] 80个高达1000 BTC的教派获得了资金,并用黄金镀造。[75] 大英博物馆的硬币收藏包括第一批资助比特币代币中的四个样本;一个标本目前正在博物馆的花旗银行画廊展出。[77][75] 2013年,金融犯罪执法网命令这些代币的一家犹他州制造商在制造更多代币之前注册为货币服务企业。[74][75]

另一种叫做硬件钱包的钱包在促进交易的同时保持凭证离线。[78]
\subsubsection{2.9 实现}
第一个钱包程序,简称比特币,有时也被称为Satoshi客户端,于2009年作为开源软件由中本聪发布。[79] 0.5版中,客户端从wxWidgets用户界面工具包转移到Qt,整个包被称为比特币-Qt。[79] 0.9版发布后,该软件包更名为比特币核心,以区别于底层网络。[80][81]

\textbf{分叉}

比特币核心或许是最著名的实现或客户端。存在其他客户(比特币核心的分叉),如比特币XT、比特币无限、[82] 和平价比特币。[82]

2017年8月1日,比特币的硬分叉被创造出来,被称为比特币现金。[83] 比特币现金有一个更大的块大小限制,在分叉时有一个相同的区块链。2017年10月24日,另一个硬分叉比特币黄金诞生了。比特币黄金改变了采矿中使用的工作证明算法,因为开发人员觉得采矿变得过于专业化。[84]
\subsubsection{2.10 去中心化和中心化}
\textbf{分散}

比特币没有中心机构,比特币网络是去中心化的:
\begin{itemize}
\item 没有中央服务器;比特币网络是点对点的。[79]
\item 没有中央存储器;比特币账本是分布式的。
\item 账本是公开的;任何人都可以把它存储在他们的电脑上。
\item 没有单一的管理员; 账本由同等特权的矿工网络维护。
\item 任何人都可以成为矿工。
\item 账本的增加是通过竞争来维护的。在新块被添加到账本之前,不知道哪个矿工将创建该块。
\item 比特币的发行是去中心化的。它们是作为对创建新区块的奖励而发行的。[64]
\item 任何人都可以创建一个新的比特币地址(银行账户的比特币副本),而无需任何批准。
\item 任何人都可以向网络发送交易,无需任何批准;网络仅仅用来确认交易是合法的。[85]
\end{itemize}

\textbf{集中化趋势}

研究人员指出了“集中化趋势”。虽然比特币可以直接从一个用户发送到另一个用户,但实际上中介被广泛使用。[56] 比特币矿工加入大型采矿池,以将收入差异降至最低。[56][86][87] 因为网络上的交易是由矿工确认的,网络的去中心化要求没有一个矿工或矿池获得51\%的散列能力,这将允许他们重复花费硬币,阻止某些交易被验证,并阻止其他矿工赚取收入。[88]截至2013年,只有6个矿池控制了比特币散列能力的75\%。[88] 2014年,Ghash.io矿集区获得了51\%的散列能力,这引起了关于网络安全的重大争议。该池自愿将其散列能力限制在39.99\%,并要求其他池负责任地为整个网络的利益而行动。[89] 2017年至2019年间,70\%以上的散列能力和90\%的交易都是在中国进行的。[90]

据研究人员称,生态系统的其他部分也“由一小部分实体控制”,特别是客户端软件、在线钱包和简化支付验证(SPV)客户端的维护。[88]
\subsubsection{2.11 隐私}
比特币是假名,这意味着资金不与现实世界实体挂钩,而是与比特币地址挂钩。比特币地址的所有者没有明确的身份,但区块链的所有交易都是公开的。此外,交易可以通过“使用习惯用法”(例如,使用来自多个输入的硬币的交易表明这些输入可能具有共同的所有者)和用关于特定地址所有者的已知信息来证实公共交易数据来与个人和公司相联系。[91] 此外,法律可能要求比特币交易所收集个人信息。[92] 为了提高金融隐私,可以为每笔交易生成一个新的比特币地址。[93]
\subsubsection{2.12 可替代性}
钱包和类似软件在技术上把所有比特币等同处理,建立了基本的可替代性。研究人员指出,每枚比特币的历史都在区块链账本中注册并公开提供,一些用户可能会拒绝接受来自有争议交易的比特币,这将损害比特币的可替代性。[94]
\subsubsection{2.13 可量测性}
区块链的数据块最初被限制在32兆字节大小。中本聪在2010年引入了1兆字节的块大小限制。最终,1兆字节的块大小限制给事务处理带来了问题,例如增加事务费用和延迟事务处理。[95]

\subsection{意识形态}
中本聪在他的白皮书中指出:“传统货币的根本问题在于充分信任它的发挥作用。必须相信央行不会让货币贬值,但法定货币的历史充满了对这种信任的违背。”[96]
\subsubsection{3.1 奥地利经济学}
据欧洲中央银行称,比特币提供的货币去中心化有其奥地利学派的理论根源,特别是弗雷德里希·冯·哈耶克在他的著作《货币去国有化:理论的提炼》中,[97] 哈耶克主张在货币的生产、分配和管理中建立一个完全自由的市场,以结束中央银行的垄断。[98]
\subsubsection{3.2 无政府主义和自由意志主义理论}
据《纽约时报》报道,自由主义者和无政府主义者被这个想法所吸引。早期比特币支持者罗杰·佛(Roger Ver)说:“起初,几乎所有参与进来的人都是出于哲学原因。我们认为比特币是一个伟大的想法,是一种将货币从政府中分离出来的方式。”[96] 《经济学人》将比特币描述为“一个创造现金在线版本的技术无政府主义项目,这是一种人们在不受政府或银行恶意干涉的情况下进行交易的方式”。
\begin{table}[h]
    \centering
      \caption\begin{tabular}{|c}
        \hline
        \textbf{外部视频链接}\\
        \hline
        The Declaration Of Bitcoin's Independence, BraveTheWorld, 4:38[99]\\
        \hline
      \end{tabular}
     \end{table}
     奈杰尔·多德在《比特币的社会生活》一书中指出,比特币意识形态的本质是将金钱从社会和政府的控制中移除。[100] 多德引用了一段YouTube视频,罗杰·弗、杰夫·贝里克、查理·施雷姆、安德烈亚斯·安东诺普洛斯、加文·伍德、Trace Meyer和其他比特币支持者阅读了《比特币独立宣言》。声明中包含了秘密无政府主义的信息,上面写着:“比特币本质上是反体制、反系统和反国家的。比特币损害政府,扰乱机构,因为比特币从根本上讲是人道主义的。[100][99]

大卫·戈伦比亚说,影响比特币倡导者的思想来自右翼极端主义运动,如自由游说团和约翰·伯奇协会及其反央行言论,或者最近罗恩·保罗和茶党式的自由意志主义。[101] 史蒂夫·班农拥有比特币的“大量股份”,他认为比特币是“破坏性的民粹主义”。它从中央政府手中夺回控制权。这是革命性的。”[102]

然而,研究人员试图揭示人们对比特币感兴趣的原因,却没有在谷歌搜索数据中发现这与自由意志主义相关的证据。

\subsection{经济学}
比特币是一种数字资产,旨在作为一种货币在点对点交易中发挥作用。[103][103] 根据《经济学人》2015年1月的报道,比特币在货币中有三种有用的特性:难以赚取,供应有限,易于验证。 据一些研究人员称,截至2015年,比特币更多的是作为支付系统,而不是货币。[104]

经济学家将货币定义为价值储存、交换媒介和记账单位。[104] 根据《经济学人》2014年的报道,比特币作为一种交换媒介的功能最佳。[104] 然而,这是有争议的,经济学家2018年的一项评估指出,加密货币不符合这三个标准。[105]
\begin{figure}[ht]
\centering
\includegraphics[width=10cm]{./figures/af2146ae750cb081.png}
\caption{资产折现力(估计值, 美元/年, 对数比例).[7]} \label{fig_BTC_8}
\end{figure}
根据剑桥大学的研究,2017年有290万至580万独特用户使用密码钱包,其中大多数是比特币。自2013年以来,用户数量大幅增长,当时有30万至130万用户。[105]