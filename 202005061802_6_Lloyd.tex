% 劳埃德镜实验

劳埃德(H.Lloyd)于1834年提出了一种更简单的观察干涉的装置.如图所示,MN 为一块平玻璃板,用作反射镜, s, 是一狭缝光源,从光源发出的
光波,一部分掠射(即入射角接近$90°$)到玻璃平板上,经玻璃表面反射到达屏上;另一部分直接射到屏上.这两部分光也是相干光,它们同样是用分波
阵面得到的反射光可看成是由虚光源S2 发出的.s, 和S2 构成一对相干光源,对干涉条纹的分析与杨氏实验也相同. 图
中画有阴影的区域表示相干光在空间叠加的区域.这时在屏上可以观察到明暗
相间的干涉条纹.

应该指出,在劳埃德镜实验中,如果把屏幕移近到和镜面边缘N 相接触,即
图中E' 的位置,这时从s, 和S2 发出的光到达接触处的路程相等,应该出现明
纹,但实验结果却是暗纹,其他的条纹也有相应的变化这一实验事实说明了由
镜面反射出来的光和直接射到屏上的光在N 处的相位相反,即相位差为'TT. 由于
直射光的相位不会变化,所以只能认为光从空气射向玻璃平板发生反射时,反射
光的相位跃变了-rr.

进一步的实验表明: 光从光疏介}贡射到光密介队界面反射时,在掠射(入射
f(J i = 900) 或W 入射(i = 0) 的·h'i 况下,反射光的相位较之入射光的相位有T 的
突变,这一变化导致了反射光的波程在反射过程中附加了半个波长,故常称为
“ 平波损久“今后在讨论光波叠加时,若有半波损失,在计算波程差时必须计及,
否则会得出与实际情况不同的结果.