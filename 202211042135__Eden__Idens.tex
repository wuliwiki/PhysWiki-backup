% 电流密度
% 电流|流密度|电流密度|电磁学

\pentry{电流\upref{I}, 流密度\upref{CrnDen}}

电流某时刻在空间中的分布情况可以用\textbf{电流密度}(矢量) $\bvec j(\bvec r)$ 来描述, 其方向与 $\bvec r$ 处的电荷运动方向相同, 详见 “流密度\upref{CrnDen}”.

\begin{equation}
I = \int \bvec j \vdot \dd{\bvec S}
\end{equation}
我们可以这样理解上式:若作一个垂直于电流方向的横截面 $\dd S$,且穿过这一横截面的电流为 $I$(这意味着单位时间 $\Delta t$ 内有 $I\Delta t$ 的电荷经过这个横截面),那么该横截面的电流密度 $\bvec j$ 可以用 $\bvec I/\dd S$ 来估计.它衡量了单位时间内单位横截面通过的电流量.现在考虑,如果作一个横截面\textbf{不垂直于}电流方向,

换句话说,
\begin{equation}
\bvec j = \rho \bvec v = ne \bvec v
\end{equation}
其中$\rho$是载流子的体电荷密度,$n$是载流子的数密度,$e$是单个载流子的电荷量,$\bvec v$是载流子的运动速度\cite{GriffE}.
