% 欧几里得空间(综述)
% license CCBYSA3
% type Wiki

本文根据 CC-BY-SA 协议转载翻译自维基百科\href{https://en.wikipedia.org/wiki/Euclidean_space}{相关文章}

\begin{figure}[ht]
\centering
\includegraphics[width=6cm]{./figures/24da64db04284f76.png}
\caption{三维欧几里得空间中的一个点可以用三个坐标来定位。} \label{fig_OJLDkj_1}
\end{figure}
欧几里得空间是几何学中的基本空间,用来表示物理空间。最初在欧几里得的《几何原本》中,它指的是三维欧几里得几何空间;但在现代数学中,欧几里得空间可以是任意正整数维的空间$n$,当需要明确维数时,称为$n$维欧几里得空间。当$n=1$或$n=2$ 时,通常分别称为欧几里得直线和欧几里得平面。\(^\text{[1]}\)“欧几里得”这一限定词用来区分欧几里得空间与后来的物理学和现代数学中研究的其他类型的空间。

古希腊几何学家引入欧几里得空间来模拟物理空间。他们的工作由古希腊数学家欧几里得汇编成《几何原本》\(^\text{[2]}\)。该书的一大创新是从少数基本性质(称为公设)出发,将空间的所有性质都作为定理加以证明。这些公设有的被认为是不言自明的(例如:“通过两点可以作且仅可以作一条直线”),有的则看似无法证明(例如平行公设)。

在19世纪末非欧几何引入之后,传统的欧几里得几何公设被重新形式化,用公理化理论来定义欧几里得空间。另一种定义方式是通过向量空间与线性代数来刻画欧几里得空间,并且已经证明这种定义与公理化定义是等价的。现代数学中更常使用这种定义,本条目中所介绍的内容也主要基于这一形式\(^\text{[3]}\)。在所有定义中,欧几里得空间都由“点”组成,而这些点只通过它们在形成欧几里得空间时必须满足的性质来刻画。

每个维数实际上只有一个本质上的欧几里得空间;也就是说,同一维数的所有欧几里得空间都是同构的。因此,在实际工作中,人们通常使用一个特定的欧几里得空间,记作:$\mathbf{E}^n \quad \text{或} \quad \mathbb{E}^n$,并将其用笛卡尔坐标表示为实数的 $n$ 维空间:$\mathbb{R}^n$并配备标准的点积结构。
\subsection{定义}
\subsubsection{定义的发展历史}
欧几里得空间最初由古希腊人提出,用来抽象化我们所处的物理空间。他们的重大创新体现在欧几里得的《几何原本》中:从少数几个抽象自物理世界的最基本性质出发,构建并证明几何学的全部内容。由于缺乏更基本的工具,这些性质无法被数学地证明,只能作为出发点,这些性质被称为公设,在现代数学语言中称为公理。这种定义欧几里得空间的方式至今仍存在,并被称为合成几何。

1637 年,笛卡尔引入了笛卡尔坐标系,并展示了如何把几何问题转化为数的代数运算。这一思想把几何归约为代数,是一种重大的观念转变,因为在此之前,实数是通过长度和距离来定义的。

直到19 世纪,欧几里得几何才被扩展到三维以上的空间。路德维希·施莱夫利将欧几里得几何推广到$n$维空间,结合了合成方法和代数方法,并发现了欧几里得空间中所有维度下存在的正多胞体(高维柏拉图立体的类似物)\(^\text{[4]}\)。

尽管笛卡尔的解析几何方法得到广泛应用,但欧几里得空间的定义直到19世纪末都没有发生改变。抽象向量空间的引入使得人们能够以纯代数的方式来定义欧几里得空间。这种新的代数定义后来被证明与基于几何公理的经典定义等价,并成为现代介绍欧几里得空间时最常用的定义方式。
\subsubsection{现代定义的动机}
可以把欧几里得平面看作是满足某些距离和角度关系的一组点。例如,在平面上有两类基本操作(称为“运动”):平移:指将整个平面沿同一个方向移动同样的距离,使所有点都发生相同的位移;旋转:指围绕平面内某个固定点旋转,使平面上的所有点都围绕该点转动同样的角度。欧几里得几何的一个基本理念是:如果一个图形可以通过一系列平移、旋转和反射变换为另一个图形,那么这两个图形应被视为全等。

为了让这些概念在数学上更精确,理论必须清楚地定义什么是欧几里得空间,以及与之相关的距离、角度、平移、旋转等概念。即便在物理理论中使用,欧几里得空间仍是一种抽象,它与实际的物理位置、参考系或测量工具无关。纯数学的欧几里得空间定义同样忽略了长度单位和其他物理量的问题:在数学空间中,距离只是一个数值,而不是以英寸或米等单位表达的物理量。

数学上定义欧几里得空间的标准方法(本文余下部分采用的方式)是:将其视为一个点集,其上有一个实向量空间作用,这个向量空间表示平移空间,并且配备了内积。平移的作用使得这个空间成为一个仿射空间,从而可以定义直线、平面、子空间、维数和平行性。内积则用来定义距离和角度。

配备了点积的实数$n$-元组集合$\mathbb{R}^n$就是一个$n$ 维欧几里得空间。反过来说,若在一个$n$维欧几里得空间中选定一个点作为原点,并选定平移空间的一个标准正交基,则可以将该欧几里得空间与$\mathbb{R}^n$建立一个同构,把它视为标准的欧几里得空间。

因此,关于欧几里得空间的所有描述,同样都适用于$\mathbb{R}^n$。正因如此,许多作者,尤其是在基础层面上,直接把$\mathbb{R}^n$称作标准 $n$ 维欧几里得空间\(^\text{[5]}\),或简而言之称作$n$ 维欧几里得空间。
\begin{figure}[ht]
\centering
\includegraphics[width=6cm]{./figures/41788364328162ed.png}
\caption{无原点示例的欧几里得平面} \label{fig_OJLDkj_2}
\end{figure}
引入这种抽象定义的欧几里得空间,并使用$\mathbb{E}^n$而不是$\mathbb{R}^n$ 的一个原因是:在很多情况下,人们更倾向于以无坐标、无原点的方式进行研究(即不选择特定的基底或特定的原点)。另一个原因是:在物理世界中并不存在标准的原点或标准的基底。
\subsubsection{技术定义}
\textbf{欧几里得向量空间},欧几里得向量空间是一个实数域上的有限维内积空间\(^\text{[6]}\)

\textbf{欧几里得空间},欧几里得空间是一个实数域上的仿射空间,并且其关联的向量空间是一个欧几里得向量空间。有时也称为欧几里得仿射空间,以区别于欧几里得向量空间\(^\text{[6]}\)。

如果$E$ 是一个欧几里得空间,那么其关联的向量空间(欧几里得向量空间)通常记作:$\overrightarrow{E}$.欧几里得空间的维数就是其关联向量空间的维数。

$E$的元素称为点,通常用大写字母表示。$\overrightarrow{E}$ 的元素称为**欧几里得向量或自由向量,有时也称为平移。不过严格来说,平移是指欧几里得向量作用于欧几里得空间所产生的几何变换。

向量对点的作用,一个向量 $v$ 作用于一个点 $P$,会得到一个新的点,记作:$P + v$.这种作用满足:
$$
P + (v + w) = (P + v) + w.~
$$
\textbf{注:}左边表达式中的第二个 “+” 表示向量加法;而其他的 “+” 表示向量对点的作用。这个符号并不混淆,因为只需观察 “+” 左边的对象类型,就能判断含义。

自由且传递的作用,这种作用是自由且传递的,这意味着对于任意一对点$P, Q$,存在唯一的位移向量$v$,使得:$P + v = Q$.这个向量$v$记作:$Q - P\quad \text{或} \quad\overrightarrow{PQ}$.

如前所述,欧几里得空间的一些基本性质来源于其仿射空间结构,这些内容在“仿射结构”及其小节中详细介绍。而由内积带来的性质,则在“度量结构”及其小节中加以说明。
\subsection{典型示例}
对于任意向量空间,加法在该向量空间本身上都具有自由且传递的作用。因此,欧几里得向量空间也可以看作是一个其关联向量空间就是自身的欧几里得空间。

一个典型的欧几里得向量空间示例是:$\mathbb{R}^n$将其视为一个配备了点积作为内积的向量空间。这个特定的欧几里得空间之所以重要,是因为每一个欧几里得空间都与它同构。更准确地说,给定一个维数为 $n$ 的欧几里得空间 $E$,如果选择一个点作为原点,并在 $\overrightarrow{E}$ 中选择一个标准正交基,就可以定义一个从 $E$ 到:$\mathbb{R}^n$的欧几里得空间同构。

由于每个$n$维欧几里得空间都与$\mathbb{R}^n$同构,$\mathbb{R}^n$有时也被称为标准$n$维欧几里得空间\(^\text{[5]}\)。
\subsection{仿射结构}
欧几里得空间的一些基本性质仅仅依赖于它是一个仿射空间这一事实。这些性质被称为仿射性质,包括直线、子空间和平行性等概念,后续小节会对此进行详细说明。
\subsubsection{子空间}
设 $E$ 是一个欧几里得空间,$\overrightarrow{E}$ 是它的关联向量空间。

欧几里得子空间,一个平直子空间、欧几里得子空间或仿射子空间$F \subseteq E$ 满足:
$$
\overrightarrow{F} = \bigl\{ \overrightarrow{PQ} \;\big|\; P \in F,\, Q \in F \bigr\}~
$$
即$F$ 的关联向量空间 $\overrightarrow{F}$ 是$\overrightarrow{E}$ 的一个线性子空间。因此,一个欧几里得子空间 $F$ 自身也是一个欧几里得空间,其关联向量空间为 $\overrightarrow{F}$。这个线性子空间 $\overrightarrow{F}$ 也称为该子空间的方向。

如果 $P$ 是 $F$ 中的一个点,则:
$$
F = 
\bigl\{ P + v \;\big|\; v \in \overrightarrow{F} \bigr\}.~
$$
反过来,如果 $P$ 是 $E$ 中的一个点,且 $\overrightarrow{V}$ 是 $\overrightarrow{E}$ 的一个线性子空间,则:
$$
P + \overrightarrow{V} = 
\bigl\{ P + v \;\big|\; v \in \overrightarrow{V} \bigr\}~
$$
是一个方向为$\overrightarrow{V}$的欧几里得子空间。(该子空间的关联向量空间即为$\overrightarrow{V}$。)

如果欧几里得空间$E$恰好等于它的关联向量空间 $\overrightarrow{E}$,即它是一个欧几里得向量空间,则它有两类子空间:欧几里得子空间,线性子空间;其中:所有线性子空间都是欧几里得子空间;只有包含零向量的欧几里得子空间才是线性子空间。
\subsubsection{直线与线段}
在欧几里得空间中,直线是一维的欧几里得子空间。由于一维向量空间可以由任意一个非零向量生成,因此一条直线可以写作如下形式:
$$
\bigl\{\, P + \lambda \overrightarrow{PQ} \;\big|\; \lambda \in \mathbb{R} \,\bigr\},~
$$
其中 $P$ 和 $Q$ 是直线上两个不同的点。

由此可知:通过两个不同的点,恰有一条直线;两条不同的直线最多有一个交点。

一个更对称的表示法是:
$$
\bigl\{\, O + (1-\lambda)\overrightarrow{OP} + \lambda\overrightarrow{OQ} \;\big|\; \lambda \in \mathbb{R} \,\bigr\},~
$$
其中 $O$ 是一个任意点(不一定在直线上)。

在欧几里得向量空间中,通常选择零向量作为$O$,此时公式可简化为:
$$
\bigl\{\, (1-\lambda)P + \lambda Q \;\big|\; \lambda \in \mathbb{R} \,\bigr\}.~
$$
根据标准约定,这个公式也可以用于所有欧几里得空间(参见“仿射空间”章节中的仿射组合与重心部分)。

线段或简称线段是连接$P$和$Q$的点集,对应上述公式中参数$\lambda$满足$0 \leq \lambda \leq 1$。它通常记作$PQ$或$QP$:
$$
PQ = QP = 
\bigl\{\, P + \lambda \overrightarrow{PQ} \;\big|\; 0 \leq \lambda \leq 1 \,\bigr\}.~
$$
\subsubsection{平行性}
在欧几里得空间中,如果两个维数相同的子空间$S$和$T$具有相同的方向(即相同的关联向量空间),则称它们是平行的\(^\text{[a]}\)。等价地,如果存在一个平移向量$v$,使得:
$$
T = S + v,~
$$
那么$S$与$T$也是平行的。

对于任意一个点 $P$ 和一个子空间$S$,存在唯一一个包含该点并且与$S$平行的子空间,它可以表示为:$P + \overrightarrow{S}$.当 $S$ 是一条直线(一维子空间)时,这一性质正是普莱费尔公理的表达。

因此,在一个欧几里得平面中,两条直线要么恰好有一个交点,要么互相平行。

平行子空间的概念也可以推广到不同维数的子空间:如果一个子空间的方向包含在另一个子空间的方向中,那么这两个子空间也被称为平行的。
\subsection{度量结构}
与欧几里得空间 $E$ 关联的向量空间 $\overrightarrow{E}$ 是一个内积空间。这意味着存在一个对称双线性形式:
$$
\begin{aligned}
\overrightarrow{E} \times \overrightarrow{E} &\;\longrightarrow\; \mathbb{R} \\[6pt]
(x, y) &\;\longmapsto\; \langle x, y \rangle
\end{aligned}~
$$
并且它是正定的(即当 $x \neq 0$ 时,$\langle x, x \rangle > 0$)。

欧几里得空间中的内积通常称为点积,并记作$x \cdot y$。特别是在选定笛卡尔坐标系的情况下,两个向量的内积就是它们坐标向量的点积。由于这个原因,以及历史上的沿用,点积符号比尖括号符号更常见。本文将遵循这种约定:$\langle x, y \rangle $记作$x \cdot y$.

向量$x$的\textbf{欧几里得范数为}:
$$
\|x\| = \sqrt{x \cdot x}.~
$$
内积和范数使得欧几里得几何的度量性质和拓扑性质能够表达和证明。接下来的小节将介绍其中最基本的一些性质。在这些小节中:$E$ 表示一个任意的欧几里得空间;$\overrightarrow{E}$ 表示它的平移向量空间。
\subsubsection{距离与长度}
在欧几里得空间中,两点之间的距离(更准确地说是欧几里得距离)是把一个点平移到另一个点的平移向量的范数。即:
$$
d(P, Q) = \bigl\| \overrightarrow{PQ} \bigr\|.~
$$
线段$PQ$的长度就是其端点$P$和$Q$之间的距离$d(P, Q)$,通常记作:$|PQ|$.

这种距离是一个度量,因为它满足:正定性:$d(P, Q) \ge 0$,且 $d(P, Q) = 0$ 当且仅当 $P = Q$;对称性:$d(P, Q) = d(Q, P)$;三角不等式:
$$
d(P, Q) \le d(P, R) + d(R, Q).~
$$
当且仅当点$R$位于线段$PQ$上时,上式中的等号成立。这说明:三角形任一边的长度小于或等于其他两边长度之和。这正是“三角不等式”这一名称的来源。

有了欧几里得距离后,每个欧几里得空间都是一个完备的度量空间。
\subsubsection{正交性}
在欧几里得空间$E$的关联向量空间$\overrightarrow{E}$中,若两个非零向量$u$和$v$的内积为零,则它们是正交或垂直的:
$$
u \cdot v = 0~
$$
如果两个线性子空间的任意一个非零向量都与另一个子空间的所有非零向量正交,那么这两个线性子空间是正交的。这意味着它们的交集仅包含零向量。

两条直线,或者更一般地说,两个欧几里得子空间,如果它们的方向向量空间彼此正交,则称它们是正交的。特别地,若两条正交直线相交,则称它们是垂直的。

两条共享公共端点$$的线段$AB$和$AC$是垂直的(或称形成直角),当且仅当向量$\overrightarrow{AB} $和$\overrightarrow{AC}$正交。

如果$AB$和$AC$形成直角,则有:
$$
|BC|^2 = |AB|^2 + |AC|^2.~
$$
这就是勾股定理。在这种向量和内积的框架下,证明这一结果非常直接:利用内积的双线性性和对称性,可以写为:

$$
\begin{aligned}
|BC|^2 
&= \overrightarrow{BC} \cdot \overrightarrow{BC} \\[4pt]
&= (\overrightarrow{BA} + \overrightarrow{AC}) \cdot (\overrightarrow{BA} + \overrightarrow{AC}) \\[4pt]
&= \overrightarrow{BA} \cdot \overrightarrow{BA} + \overrightarrow{AC} \cdot \overrightarrow{AC} - 2 \overrightarrow{AB} \cdot \overrightarrow{AC} \\[4pt]
&= \overrightarrow{AB} \cdot \overrightarrow{AB} + \overrightarrow{AC} \cdot \overrightarrow{AC} \\[4pt]
&= |AB|^2 + |AC|^2.
\end{aligned}
$$

其中,使用了条件:

$$
\overrightarrow{AB} \cdot \overrightarrow{AC} = 0
$$

因为这两个向量是正交的。
