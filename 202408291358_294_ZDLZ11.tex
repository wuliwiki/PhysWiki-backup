% 浙江大学 2011 年硕士入学量子力学考试试题
% keys 浙江大学|考研|物理|量子力学|2011年
% license Copy
% type Tutor


\textbf{声明}:“该内容来源于网络公开资料,不保证真实性,如有侵权请联系管理员”

\subsection{简答题}
\subsubsection{第一题}
\begin{enumerate}
\item 什么是光电效应?什么是受激辐射?
\item 什么是斯达克效应?什么是塞曼效应?
\item 一维问题的能级的简并度最大是多少?
\item 写出测不准关系,并简要说明其物理含义。
\item 写出非简并微扰论的能级修正公式(到二阶)。
\item 放射性指的是束缚在某些原子核中的更小粒子有一定的概率逃逸出来,你
认为这与什么量子效应有关?
\item 由正则对易关系$[\uvec x,\uvec p]=ih$导出角动量的三个分量
\begin{equation}
L_x=y\pdv{}{z}-z\pdv{}{y} \qquad L_y=z\pdv{}{x}-x\pdv{}{z} \qquad L_z=x\pdv{}{y}-y\pdv{}{x}~
\end{equation}
的对易关系。
\end{enumerate}
\subsubsection{第二题}
有一个质量为m的粒子处在如下势阱中(这里 $V_0>0$)
\begin{equation}
v(x)=\leftgroup{&\infty   \quad &x<0\\&-V_0 \quad &0<X<a \\&0 \quad &z<x}~
\end{equation}
(1)求其能级,求其波函数。\\
(2)确定至少存在一个束缚态的条件。
\subsubsection{第三题}
原子序数较大的原子的最外层电子感受到的原子核和内层电子的总位势可表示为$v(r)=-\frac{e^2}{r}-a\frac{e^2}{r^2},a<<1$,试求其基态能量。
\subsubsection{第四题}
分别回答下列两问,\\
(1)如果把谐振子势$V(x)=\frac{1}{2}m\omega^2x^2$变成$V(x)=\leftgroup{\frac{1}{2}m\omega^2x^2 \quad \abs{x}<a\\ \frac{1}{2}m\omega^2a^2 \quad \abs{x}\ge a}$与原来能级比较,你认为会发生什么变化?\\
(2)如果把谐振子势$V(x)=\frac{1}{2}m\omega^2x^2$变成$V(x)=\leftgroup{\frac{1}{2}m\omega^2x^2 \quad \abs{x}<a\\ \infty \quad \abs{x}\ge a}$你认为又会发生什么变化?
\subsubsection{第五题}
假定一个原子阱中的单粒子能级有两个非简并的本征值ε,和ε,
们对应的单粒子空间波函数分别设为业(F)和y(F)。
(1)试求该阱中有两个自旋为零玻色型原子时,基态和第一激发态的波函数(2)如果有两个自旋为1的玻色型原子,求基态和第一激发态的能级简并度
第六题:(20 分):自旋1/2的海森堡模型是研究量子磁性系统的重要模型考虑三个自旋的简单情形,
A=-J(S,S,+S,S,+8.S),其中 J>0 为参数。
试求其能量本征值。
如果引入磁场则会有附加的Zeeman 能,那么总 Hamiltonian 成为:(2)A=-J(S,·$,+$,$,+$,S,)- B(S, +$,+S,),针对磁场 B不同的取值范围,试确定其基态。
