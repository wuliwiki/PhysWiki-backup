% 合力、力矩和、功之和

%这篇文章意图创建一个无关质心系的、对质点系量的基本描述。即把原先的“质点系与刚体”拆分为“质点系”、“质心与质心系”、“刚体”

\footnote{本文受安宇等人的《大学物理》启发。}
系统的大多数物理量等于组成他的质点的物理量之和。
\subsection{合力}
\begin{equation}
\bvec F_{tot}=\sum \bvec F_i=\sum_{in} \bvec F_i + \sum_{ext} \bvec F_i = \sum_{ext} \bvec F_i
\end{equation}

系统的合力等于系统中各质点所受力之和。

更细致地说,我们还可以定义“外力”与“内力”。内力的受力物体与施力物体都位于系统之内;而外力的施力物体则位于系统之外。

系统的合力有一个很重要的性质:合内力总为零,因此系统的合力事实上等于合外力。因为根据牛顿第三定律,内力总是“等大共线反向”的,因此内力之和总为零。

%个人喜欢使用“力矩和”,因为“合 力矩”容易被误解为“合力 矩”
\subsection{力矩和}
系统的力矩和等于系统中各质点所受力矩之和。注意,此处的“力矩和”指的是各个力矩的加和,而不是对“合力”求矩。
\begin{equation}
\bvec \tau_{tot}=\sum \bvec \tau_i=\sum_{in} \bvec \tau_i + \sum_{ext} \bvec \tau_i = \sum_{ext} \bvec \tau_i
\end{equation}

与力的情况一样,内力矩之和同样为零。因此,系统的力矩和事实上等于外力矩和。

很容易从牛顿第三定律证明内力矩之和为零也。假设系统中的两个质点,他们间存在一对内力,他们的力矩和是
$$\bvec \tau = \bvec r_1 \times \bvec F_1 + \bvec r_2 \times \bvec F_2$$
根据牛顿第三定律“等大反向”,$\bvec F_1=-\bvec F_2$,因此
$$\bvec \tau = \bvec r_1 \times \bvec F_1 - \bvec r_2 \times \bvec F_1 = (\bvec r_1 - \bvec r_2) \times \bvec F_1 = \bvec r_{rel}\times \bvec F_1$$
再根据牛顿第三定律“共线”,$\bvec r_{rel} \parallel \bvec F_1 $,因此
$$\bvec \tau = \bvec 0$$
总的内力矩和即为这样一组组的内力矩和,自然总的内力矩和也为零。

\subsection{功之和}
系统中力做的总功等于力对系统中各质点所做功之和。
\begin{equation}
\delta W = \sum \delta w_i = \sum_{in} \delta w_i + \sum_{ext} \delta w_i
\end{equation}

那么,内力做的功会互相抵消吗?答案是,\textbf{不会的}。不同于合力或力矩和,内力做的功\textbf{不会}相互抵消。在计算功之和的时候,必须计算内力功的和与外力功的和
