% 结构张量代数
% keys 代数|结构张量|结构常数

\begin{issues}
\issueDraft
\end{issues}

\pentry{张量的坐标变换\upref{TrTnsr},域上的代数\upref{AlgFie}}
在线性算子代数\upref{LiOper} 一节提到,代数是一个同时是个环的矢量空间,或者由代数的定义直接获得.那么,要使一个矢量空间构成一个代数,就得赋予矢量空间环的特性,即任意二矢量可进行乘法运算且该乘法对加法满足分配律.由于矢量都可由一组基表示,那么任意二矢量可作乘法及对加法满足分配律的要求,就变成只需规定基矢量之间的乘法.由运算的封闭性,作乘法得到的矢量仍能用基表示,这样的基矢量之间的乘法得到的矢量在该组基下的坐标就称为\textbf{结构常数},这样只需要求乘法满足结合律,矢量空间便是一个环了,于是就将矢量空间构造成了一个代数.可以证明,结构常数是某一个 $(2,1)$ 型张量的坐标,这个张量就称为\textbf{结构张量}. 

一句话来说就是:结构张量使得一个矢量空间具有了代数结构.

\subsection{将矢量空间构造成环}
设 $V$ 是域 $\mathbb F$ 上的矢量空间,$\{e_i\}$ 是它的一个基.那么任意的元都可表示成 (使用爱因斯坦求和约定\upref{EinSum})
\begin{equation}
x^i e_i
\end{equation}
的形式.下面将 $V$ 构造为一个环.

为使任意二矢量能进行乘法运算,且乘法是封闭的和对加法满足分配律,那么只需规定
\begin{equation}\label{STAlg_eq1}
e_i*e_j=\gamma_{ij}^k e_k
\end{equation}
且乘法“*”满足
\begin{equation}
\begin{aligned}
&\lambda(e_i*e_j)=(\lambda e_i)*e_j=e_i*(\lambda e_j),\quad \lambda\in\mathbb F\\
&(e_i+e_j)*e_k=e_i*e_k+e_j*e_k\\
&e_i*(e_j+e_k)=e_i*e_j+e_i*e_k
\end{aligned}
\end{equation}

现在来寻找乘法满足结合律的要求,设
\begin{equation}
u=u^ie_i,\quad v=v^je_j,\quad w=w^ke_k
\end{equation}
要 $(uv)w=u(vw)$,就要
\begin{equation}
\begin{aligned}
&u^iv^jw^k (e_i*e_j)*e_k=(u^ie_i *v^je_j)*w^ke_k=u^ie_i *(v^je_j*w^ke_k)\\
&=u^iv^jw^k e_i*(e_j*e_k)
\end{aligned}
\end{equation}
由 $u,v,w$ 的任意性,只要
\begin{equation}
\begin{aligned}
(e_i*e_j)*e_k&=e_i*(e_j*e_k)\\
&\Downarrow\\
\gamma_{ij}^le_l*e_k&=\gamma_{jk}^le_i*e_l\\
&\Downarrow\\
\gamma_{ij}^l\gamma_{lk}^{m}e_m&=\gamma_{jk}^l\gamma_{il}^m e_m\\
&\Downarrow\\
\gamma_{ij}^l\gamma_{lk}^{m}&=\gamma_{jk}^l\gamma_{il}^m
\end{aligned}
\end{equation}
结合律就满足了.

这样,矢量空间在乘法“*” 之下就构成了一个代数.这就给出了下面的定义.
\begin{definition}{结构常数}
由\autoref{STAlg_eq1} 规定的 $\gamma_{ij}^k\in\mathbb F$ 称为代数 $V$ 在给定基之下的\textbf{结构常数},其满足
\begin{equation}
\gamma_{ij}^l\gamma_{lk}^{m}=\gamma_{jk}^l\gamma_{il}^m
\end{equation}
\end{definition}
\begin{theorem}{结构张量}
结构常数 $\gamma_{ij}^k$ 是某个 $(2,1)$ 型张量 $\Gamma$ 的坐标.该张量 $\Gamma$ 称为\textbf{结构张量}.
\end{theorem}
\textbf{证明:}要证结构常数是某个张量的坐标,就只要证它的坐标满足张量坐标的变换规则,因为如果一个数满足张量的坐标变换规则,那么配上对应基底后就是一个张量.

设
\begin{equation}
e_i'=a_i^s e_s,\quad e_j'=b_j^t e_t
\end{equation}
则 $B=(b_j^t)=A^{-1}$, $A=(a_i^s)$ .于是
\begin{equation}\label{STAlg_eq2}
\begin{aligned}
{\gamma'}_{ij}^ke'_k&=e'_i*e'_j=\qty(a_i^s e_s)*\qty(a_j^t e_t)=a_i^s a_j^te_s*e_t\\
&=a_i^sa_j^t\gamma_{st}^r e_r=a_i^sa_j^t\gamma_{st}^r b_r^k e'_k\\
&\Downarrow\\
{\gamma'}_{ij}^k&=a_i^sa_j^t\gamma_{st}^r b_r^k
\end{aligned}
\end{equation}
由张量坐标变换规则(\autoref{TrTnsr_the1}~\upref{TrTnsr}),\autoref{STAlg_eq2} 表明结构常数和 $(2,1)$ 型张量的坐标变换规则一致.

\textbf{证毕!}

综上,可以说,在 $V$ 上给出了结构张量 $\Gamma$,就确定了一个代数.反过来,代数 $V$ 的结构张量 $\Gamma$ 是完全确定的.
\subsection{迹形式}

对于研究代数 $V$ 的结构,迹形式是一个重要的工具.定义映射
\begin{equation}
L_a:x\rightarrow a*x
\end{equation}
由乘法 “*” 的线性性,可知 $L_a$ 是个线性算子.
\begin{definition}{迹形式}
称
\begin{equation}
f_V(a,b)=\mathrm{tr}\, L_aL_b
\end{equation}
是代数 $V$ 上的\textbf{迹形式}.
\end{definition}
\begin{example}{}
试证明:设 $a=\alpha^i e$ 迹形式可记成张量的完整卷积的形式:
\begin{equation}
f_V(a,b)=
\end{equation}

\end{example}