% 随机变量的变换
% 概率|随机变量|分布函数|微积分|微分|微分方程|可分离变量的微分方程

\pentry{概率分布函数\upref{RandF}}

\subsection{求新变量的分布函数}

我们先来讨论这样一个问题:令两个随机变量 $x_1, x_2$ 间有函数关系 $x_1 = g(x_2)$, 若已知 $x_1$ 的分布函数为 $f_1(x_1)$, 求 $x_2$ 的分布函数 $f_2(x_2)$.

将两个概率分布写成微分形式%式未完成
, 有
\begin{gather}
\dd{P} = f_1(x_1) \dd{x_1}\label{RandCV_eq1}\\
\dd{P} = f_2(x_2) \dd{x_2} \label{RandCV_eq2}
\end{gather}
若将\autoref{RandCV_eq1} 中的 $x_1$ 替换成 $g(x_2)$, $\dd{x_1}$ 替换成 $g'(x_2)\dd{x_2}$, 有
\begin{equation}
\dd{P} = f_1[g(x_2)] g'(x_2) \dd{x_2}
\end{equation}
对比\autoref{RandCV_eq2}, 得
\begin{equation}
f_2(x) = f_1[g(x)] g'(x)
\end{equation}
这样, 就求出了 $x_2$ 的分布函数 $f_2(x)$.

\begin{example}{}
已知 $f_1(x_1) = 3x_1^2$, $x_1 \in [0, 1]$, $x_2 = x_1^2$, 求 $x_2$ 的分布函数.

用 $x_2$ 表示 $x_1$ 得 $x_1 = \sqrt{x_2}$, 代入 $\dd{P} = f_1(x_1)\dd{x_1}$, 得
\begin{equation}
\dd{P} = 3\sqrt{x_2}^2 \dd(\sqrt{x_2}) = \frac32 \sqrt{x_2} \dd{x_2}
\end{equation}
所以 $x_2$ 的分布函数为 $f_2(x_2) = 3\sqrt{x_2}/2$.
\end{example}

\subsection{求两变量的关系}
\pentry{可分离变量的微分方程}%链接未完成

另一个常见的问题是已知 $x_1$ 和 $x_2$ 的分布函数 $f_1(x_1), f_2(x_2)$, 求两个随机变量需要满足的函数关系.

对比\autoref{RandCV_eq1} 和\autoref{RandCV_eq2} 可得一个已分离变量的微分方程
\begin{equation}\label{RandCV_eq6}
f_1(x_1)\dd{x_1} = f_2(x_2)\dd{x_2}
\end{equation}
将方程两边积分即可得到两变量所满足的函数关系
\begin{equation}
F_1(x_1) = F_2(x_2) + C
\end{equation}
其中函数 $F_1, F_2$ 分别是函数 $f_1, f_2$ 的一个原函数, 待定常数 $C$ 通常可以由 $x_1$ 和 $x_2$ 的取值范围确定.

这个问题的一个常见应用是在程序中生成指定分布函数的随机变量. 在许多编程语言中, 随机数生成器只能生成一个从 0 到 1 均匀分布的随机变量(即 $f(x) = 1$), 若我们需要一个其他分布的随机变量, 就可以使用以上方法.

\begin{example}{}
已知随机变量 $x_1\ \ (x_1\in [0,1])$ 的分布函数为 $f_1(x_1) = 1$, 求函数关系 $x_2 = g(x_1)$ 使得 $x_2$ 的分布函数为 $f_2(x_2) = 2x_2\ \ (x_2\in [0,1])$.

将 $f_1, f_2$ 代入\autoref{RandCV_eq6} 并两边积分得
\begin{equation}
x_1 = x_2^2 + C
\end{equation}
由于 $x_1$ 和 $x_2$ 的区间关系得 $C = 0$, 所以有 $x_2 = \sqrt{x_1}$.
\end{example}

\begin{example}{}\label{RandCV_ex3}
给出两个随机变量 $\xi_1, \xi_2\ \ (\xi_1, \xi_2\in [0,1])$, 分布函数均为 $f(\xi_i) = 1$, 用 $\xi_1, \xi_2$ 表示某随机点的极坐标 $(r,\theta)$ 使得该点在单位圆内均匀随机分布.

要使随机点在单位圆内随机分布, $\theta$ 显然应该在 $[0,2\pi]$ 间均匀随机分布, 所以令 $\theta = 2\pi\xi_2$ 即可. 要决定 $r$ 的分布函数, 我们把单位圆划分为许多小圆环, 随机点出现在某圆环内的概率等于该圆环的面积比单位圆的面积, 即
% 未完成:定积分应该增加一道求圆面积的例题!
\begin{equation}
\dd{P} = \frac{2\pi r\dd{r}}{\pi} = 2r\dd{r}
\end{equation}
所以 $r$ 的分布函数为 $2r$. 令 $r$ 与 $\xi_1$ 间存在函数关系, 由\autoref{RandCV_eq6} 得
\begin{equation}
1\dd{\xi_1} = 2r\dd{r}
\end{equation}
两边积分得 $\xi_1 = r^2$ ($\xi_1 = 0$ 时 $r = 0$, 所以积分常数为零), 即 $r = \sqrt{\xi_1}$. 所以, 随机点的极坐标取 $(\sqrt{\xi_1}, 2\pi \xi_2)$ 即可.
\end{example}
