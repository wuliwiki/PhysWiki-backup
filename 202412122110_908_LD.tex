% 零点能量(综述)
% license CCBYSA3
% type Wiki

本文根据 CC-BY-SA 协议转载翻译自维基百科\href{https://en.wikipedia.org/wiki/Zero-point_energy}{相关文章}。

\begin{figure}[ht]
\centering
\includegraphics[width=8cm]{./figures/9b23f9fc5e8f2f7f.png}
\caption{液态氦由于零点能的存在,在标准大气压下无论温度如何都保持动能,不会冻结。当其温度降到 Lambda 点以下时,它表现出超流体性特性。} \label{fig_LD_1}
\end{figure}
\textbf{零点能(ZPE)}是量子力学系统可能具有的最低能量。与经典力学不同,量子系统即使在最低能量状态下也会不断波动,这可以通过海森堡不确定性原理来描述[1]。因此,即使在绝对零度下,原子和分子也会保持某些振动运动。除了原子和分子外,真空的空旷空间也具有这些性质。根据量子场论,宇宙可以被看作不仅是孤立的粒子,而是连续波动的场:物质场,其量子是费米子(即轻子和夸克),以及力场,其量子是玻色子(例如光子和胶子)。所有这些场都有零点能[2]。这些波动的零点场导致了一种在物理学中重新引入以太的现象[1][3],因为某些系统可以探测到这种能量的存在[需要引用]。然而,如果这个以太要保持洛伦兹不变性,以保证与爱因斯坦的相对论没有矛盾,那么它就不能被视为一种物理介质[1]。

零点能的概念对宇宙学也非常重要,然而,物理学目前缺乏一个完整的理论模型来理解宇宙学中的零点能;特别是理论上与观测到的宇宙真空能量之间的差异,成为了一个重大争议问题[4]。然而,根据爱因斯坦的广义相对论,任何这种能量都会引起引力,而来自宇宙膨胀、暗能量和Casimir效应的实验证据表明,任何这种能量都极其微弱。一个试图解决这一问题的提案是认为费米子场具有负的零点能,而玻色子场具有正的零点能,因此这些能量会以某种方式相互抵消[5][6]。如果超对称是自然界的精确对称性,这个想法是成立的;然而,欧洲核子研究中心的大型强子对撞机至今未找到支持这一理论的证据。此外,已知如果超对称是有效的,它最多也只是一个破缺的对称性,仅在极高的能量下才成立,目前没有人能够展示一个低能宇宙中发生零点能抵消的理论[6]。这一差异被称为宇宙学常数问题,是物理学中最大的未解之谜之一。许多物理学家认为,“真空是理解自然的关键”[7]。
\subsection{词源和术语}  
零点能(ZPE)一词是从德语“Nullpunktsenergie”翻译过来的。[8] 有时,它与零点辐射和基态能量互换使用。零点场(ZPF)一词可以用来指代特定的真空场,例如量子电动力学(QED)真空,它专门处理量子电动力学(如光子、电子和真空之间的电磁相互作用),或者量子色动力学(QCD)真空,它涉及量子色动力学(如夸克、胶子和真空之间的色荷相互作用)。真空可以被视为不是空的空间,而是所有零点场的组合。在量子场论中,这种场的组合被称为真空态,与之相关的零点能量被称为真空能量,平均能量值称为真空期望值(VEV),也称为其凝聚态。
\subsection{概述}
\begin{figure}[ht]
\centering
\includegraphics[width=8cm]{./figures/429e3fe66ba0fbb1.png}
\caption{动能与温度} \label{fig_LD_2}
\end{figure}
在经典力学中,所有粒子都可以被认为具有某种能量,这种能量由它们的势能和动能组成。例如,温度来自于由动能引起的随机粒子运动的强度(称为布朗运动)。当温度降低到绝对零度时,可以认为所有运动都停止,粒子完全静止。然而,实际上,即使在最低的温度下,粒子仍然保持动能。与这种零点能量对应的随机运动永远不会消失;它是量子力学不确定性原理的结果。


不确定性原理表明,任何物体无法同时拥有精确的位置和速度值。量子力学物体的总能量(包括势能和动能)由其哈密顿量描述,哈密顿量也描述了该系统作为一个简谐振子或波函数,在不同的能量状态之间波动(参见波粒二象性)。所有量子力学系统即使在其基态下也会经历波动,这是它们波动性本质的结果。不确定性原理要求每个量子力学系统必须具有大于经典势阱最小值的波动零点能量。这导致即使在绝对零度下也会有运动。例如,液氦在大气压力下无论温度如何都不会冻结,这正是由于其零点能量。

根据阿尔伯特·爱因斯坦的质量与能量等价关系 \( E = mc^2 \),任何包含能量的空间点都可以被看作具有质量,从而产生粒子。现代物理学已经发展出了量子场论(QFT),用以理解物质与力之间的基本相互作用;它将空间中的每个点视为一个量子简谐振子。根据量子场论,宇宙由物质场组成,物质场的量子是费米子(如轻子和夸克),以及力场,力场的量子是玻色子(如光子和胶子)。所有这些场都具有零点能量。最近的实验支持这样一个观点:粒子本身可以看作是基础量子真空的激发态,物质的所有属性只是由零点场相互作用引起的真空波动。