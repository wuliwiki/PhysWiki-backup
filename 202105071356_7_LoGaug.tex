% 洛伦兹规范
% 规范变换|洛伦兹规范|麦克斯韦方程组

\begin{issues}
\issueDraft
\end{issues}

\pentry{规范变换\upref{Gauge}}

如果令
\begin{equation}
\div \bvec A = -\mu_0 \epsilon_0 \pdv{\varphi}{t}
\end{equation}
那么标势和矢势就符合\textbf{洛伦兹规范}. 

麦克斯韦方程组(\autoref{EMPot_eq4}~\upref{EMPot} \autoref{EMPot_eq5}~\upref{EMPot})将变为十分对称的形式
\begin{equation}\label{LoGaug_eq1}
\laplacian \varphi - \mu_0\epsilon_0 \pdv[2]{\varphi}{t} = -\frac{\rho}{\epsilon_0}
\end{equation}
\begin{equation}\label{LoGaug_eq2}
\laplacian \bvec A - \mu_0\epsilon_0 \pdv[2]{\bvec A}{t} = -\mu_0 \bvec j
\end{equation}

这个形式的优点是可以直接进行相对论的推广.按照相对论章节中的习惯,我们令$\mu_0=\epsilon_0=1$,那么势的麦克斯韦方程组就可以写成

\begin{equation}
\laplacian \varphi -  \pdv[2]{\varphi}{t} = -\frac{\rho}{\epsilon_0}
\end{equation}
\begin{equation}
\laplacian \bvec A - \ \pdv[2]{\bvec A}{t} = -\mu_0 \bvec j
\end{equation}

这实际上就是$\nabla$算子在闵可夫斯基度规下的拓展:

\begin{equation}
\square^2 \varphi = -\frac{\rho}{\epsilon_0}
\end{equation}
\begin{equation}
\square^2 \bvec{A} = -\mu_0 \bvec j
\end{equation}

其中$\square^2$被称为\textbf{达朗贝尔算子(d' Alembertian operator)},定义为:
\begin{equation}\label{LoGaug_eq3}
\square^2=\square^\mu\square_\mu=\square^\mu\square^\nu g_{\mu\nu}=-(\frac{\partial}{\partial x^0})^2+(\frac{\partial}{\partial x^1})^2+(\frac{\partial}{\partial x^2})^2+(\frac{\partial}{\partial x^3})^2
\end{equation}
其中又有
\begin{equation}
\square^\mu=\frac{\partial}{\partial x^\mu}
\end{equation}
可见$\square^\mu$直接就是$\nabla^i$加了时间项之后的推广.

注意\autoref{LoGaug_eq3} 中$\partial/\partial x^0)^2$




