% 柱坐标中的亥姆霍兹方程
% 柱坐标|亥姆霍兹

\begin{issues}
\issueDraft
\end{issues}

\pentry{亥姆霍兹方程, 柱坐标系中的拉普拉斯方程\upref{CylLap}}

亥姆霍兹方程为
\begin{equation}
\laplacian u = -k^2 u
\end{equation}

将柱坐标系中的拉普拉斯方程右边加上一项得到亥姆霍兹方程。
\begin{equation}
\frac{1}{r} \pdv{r} \qty(r\pdv{u}{r}) + \frac{1}{r^2} \pdv[2]{u}{\theta} + \pdv[2]{u}{z} = -k^2 u
\end{equation}
使用分离变量法, 令 $u = R(r) \Phi(\theta) Z(z)$, 代入方程得
\begin{equation}
\frac{1}{rR}\pdv{r} \qty(r\pdv{R}{r}) + \frac{1}{r^2 \Phi} \pdv[2]{\Phi}{\theta} + \frac{1}{Z} \pdv[2]{Z}{z} = -k^2
\end{equation}
$\Theta(\theta)$ 和 $Z(z)$ 的常微分方程和解与拉普拉斯方程中的相同
\begin{equation}
\frac{1}{Z} \pdv[2]{Z}{z} = l^2 \qquad
\frac{1}{\Phi} \dv[2]{\Phi}{\theta} = -m^2
\end{equation}
径向方程变为
\begin{equation}
r\pdv{r} \qty(r\pdv{R}{r}) + [(l^2 + k^2)r^2 - m^2]R  = 0
\end{equation}
令 $x = \sqrt{l^2 + k^2}r$ 以及 $u(x) = R(r)$, 做变量替换得
\begin{equation}\label{CylHlm_eq1}
x \dv{x} \qty(x\dv{y}{x}) + (x^2 - m^2)y = 0
\end{equation}
与拉普拉斯方程的情况(\autoref{CylLap_eq8}~\upref{CylLap})一样, 我们得到了贝塞尔方程(\autoref{Bessel_eq1}~\upref{Bessel})。 但注意 $x, r$ 之间的缩放比例取也取决于 $k^2$。
