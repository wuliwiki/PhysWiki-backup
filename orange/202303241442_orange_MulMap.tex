% 多重线性映射
% 线性映射|线性型
\pentry{矢量空间\upref{LSpace}}
\subsection{多重线性映射}
\begin{definition}{多重线性映射}
设 $V_1,\cdots,V_p;U$ 为域\upref{field} $\mathbb{F}$ 上的矢量空间\upref{LSpace}。映射
\begin{equation}
f:V_1\times V_2\times\cdots\times V_p\rightarrow U
\end{equation}
称为\textbf{多重线性的}(\textbf{$p$-线性}),如果对任意指数 $i=1,\cdots,p$ 及任意固定的向量 $\bvec{a_j}\in V_j\quad(1\leq j\leq p,j\neq i)$ ,映射
\begin{equation}
f_i:\bvec{v}\mapsto f(\bvec{a_1},\cdots,\bvec{a_{i-1}};\bvec{v},\bvec{a_{i+1}},\cdots,\bvec{a_p})
\end{equation}
都是线性的。即
\begin{equation}
f_i(\alpha\bvec x+\beta\bvec y)=\alpha f_i(\bvec x)+\beta f_i(\bvec y)~,
\end{equation}
\end{definition}
所有 $p$-线性的映射构成的集合记为 $\mathcal{L}(V_1,\cdots,V_p;U)$ 。
\begin{definition}{加法}
\begin{equation}
\begin{aligned}
\forall \mu,\nu\in\mathbb{F},\;\bvec a_i\in V_i,\;f,g\in \mathcal{L}(V_1,\cdots,V_p;U);\\
(\mu f+\nu g)(\bvec a_1,\cdots,\bvec a_p)=\mu f(\bvec a_1,\cdots,\bvec a_p)+\nu g(\bvec a_1,\cdots,\bvec a_p)~.
\end{aligned}
\end{equation}
\end{definition}
\begin{theorem}{}\label{MulMap_the1}
任意两个 $p$-线性映射的线性组合 $\alpha f+\beta g$ 也是一个 $p$-线性映射。
\end{theorem}
\textbf{证明:}令 $h=\alpha f+\beta g$ ,且对任意指数 $i=1,\cdots,p$ 及任意固定的向量 $\bvec{a_j}\in V_j\quad(1\leq j\leq p,j\neq i)$ ,记
\begin{equation}
h_i:\bvec{v}\mapsto h(\bvec{a_1},\cdots,\bvec{a_{i-1}};\bvec{v},\bvec{a_{i+1}},\cdots,\bvec{a_p})~.
\end{equation}
则对 $\forall \bvec x,\bvec y\in V_i,\quad a,b\in\mathbb{F}$
\begin{equation}
\begin{aligned}
h_i(a \bvec x+b\bvec y)&=(\alpha f_i+\beta g_i)(a \bvec x+b\bvec y)\\
&=\alpha f_i(a \bvec x+b\bvec y)+\beta g_i(a \bvec x+b\bvec y)\\
&=a\alpha f_i(\bvec x)+b\alpha f_i(\bvec y)+a\beta g_i(\bvec x)+b\beta g_i(\bvec y)\\
&=a\qty(\alpha f_i(\bvec x)+\beta g_i(\bvec x))+b(\alpha f_i(\bvec y)+\beta g_i(\bvec y))\\
&=ah_i(\bvec x)+bh_i(\bvec y)~,
\end{aligned}
\end{equation}
这显然满足 $p$-线性映射的定义。证毕!

\autoref{MulMap_the1} 表明, $p$-线性映射的加法是封闭的。容易验证所有的 $p$-线性映射的集合 $\mathcal{L}(V_1,\cdots,V_p;U)$ 构成 $\mathbb{F}$ 的向量空间。
\subsubsection{多重线性型}
\begin{definition}{多重线性型}\label{MulMap_def2}
任意 $V_1\times V_2\times\cdots\times V_p$ 到 $\mathbb{F}$ 的多重线性映射称为 $V_1\times V_2\times\cdots\times V_p$ 上的\textbf{多重线性型}(\textbf{多重线性函数})(\textbf{multilinear form})。
\end{definition}
\begin{example}{}
设
\begin{equation}
l^i:\bvec{v_i}\mapsto l^i(\bvec{v_i}),\quad i=1,\cdots,p
\end{equation}
是 $V_i$ 上的线性函数(型),那么用关系式
\begin{equation}
f(\bvec{v_1},\cdots,\bvec{v_p})=l^1(\bvec{v_1})\cdots l^p(\bvec{v_p})
\end{equation}
定义的函数 $f$ 就是 $V_1\times V_2\times\cdots\times V_p$ 上的多重线性函数。称它为线性函数(型)$l^1,\cdots,l^p$ 的\textbf{张量积}且表成 $f=l^1\otimes l^2\otimes\cdots\otimes l^p$ 或简记为 $l^1l^2\cdots l^p$(有序)。
\end{example}
当 $V_1=\cdots=V_p=V$ 时, $V^p=V\times\cdots\times V$ (集合 $V$ 的 $p$ 个元素的笛卡尔积),此时,记
\begin{equation}
\mathcal{L}_p(V,\mathbb{F})=\mathcal{L}(V,\cdots,V;\mathbb{F})
\end{equation}
是很方便的。
\begin{definition}{多重线性型的对称性}\label{MulMap_def1}
$V^p$ 上的多重线性型 $f$ 称为\textbf{对称的}。如果对任意 $\bvec{v_1},\cdots\bvec{v_p}\in V$ 及任意置换\upref{Perm} $\pi\in S_p$ ,都有
\begin{equation}
f(\bvec{v}_{\pi(1)},\bvec{v}_{\pi(2)},\cdots,\bvec{v}_{\pi(p)})=f(\bvec{v}_1,\bvec{v}_2,\cdots,\bvec{v}_p)
\end{equation}
而 $f$ 称为\textbf{斜对称的}(或反对称)。如果
\begin{equation}
f(\bvec{v}_{\pi(1)},\bvec{v}_{\pi(2)},\cdots,\bvec{v}_{\pi(p)})=\epsilon_\pi f(\bvec{v}_1,\bvec{v}_2,\cdots,\bvec{v}_p)~,
\end{equation}
其中 $\epsilon_\pi$ 是置换的符号(偶置换取1,奇置换取负)\upref{permu}。
\end{definition}