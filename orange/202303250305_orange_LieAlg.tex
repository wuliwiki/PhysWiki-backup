% 李代数
% 李代数|结合代数|李代数的表示|线性群|一般线性群|特殊线性群|结构常数|李代数的结构常数|爱因斯坦求和约定|叉积|外积|叉乘|向量积|Jacobi恒等式|Jacobi结合性|交换李代数|阿贝尔李代数

\pentry{域上的代数\upref{AlgFie}}

\subsection{李代数}

李代数(Lie algebra)是对域上的代数\upref{AlgFie}进行的一种推广。域上的代数是指域上的线性空间配合了一个矢量乘法,使得这个线性空间在矢量乘法下也能构成一个环。李代数也是域上线性空间配合了一个矢量乘法,这个矢量乘法和构成环的乘法几乎一样,但是有一点显著不同:将环乘法的结合律替代为Jacobi结合性。

李代数在现代理论物理的时空研究中应用广泛,常用于描述光滑张量场之间的作用——要注意的是,此时我们把一个光滑张量场本身看成一个向量。

\begin{definition}{李代数}\label{LieAlg_def2}
给定域 $\mathbb{F}$ 上的一个线性空间 $V$。在 $V$ 上定义一个“乘法”运算 $[\cdot, \cdot] : V \times V \to V$,对于任意 $\bvec{v}_1, \bvec{v}_2\in V$,将它们的运算结果记为 $[\bvec{v}_1, \bvec{v}_2]$,称为\textbf{李括号}。$(V, [\cdot, \cdot])$ 被称为一个\textbf{李代数},如果它满足以下性质:

\begin{itemize}
\item \textbf{封闭性} 对于任意 $\bvec{v}_1, \bvec{v}_2\in V$,$[\bvec{v}_1, \bvec{v}_2]\in V$。% 这条其实已经在“运算”的定义里了。

\item \textbf{双线性性}\footnote{如果用爱因斯坦求和约定\upref{EinSum}来表达,双线性性还可以写为 $[a^i\bvec{v}_i, b^j\bvec{u}_j] = a^i b^j[\bvec{v}_i, \bvec{u}_j]$。} 对于任意 $c_1, c_2 \in \mathbb{F}$ 和任意 $\bvec{v}, \bvec{v}_1, \bvec{v}_2, \bvec{u}, \bvec{u}_1, \bvec{u}_2 \in V$,都有 
\begin{equation}
\begin{aligned}
{[\bvec{v}, c_1\bvec{u}_1 + c_2\bvec{u}_2]} &= c_1 [\bvec{v}, \bvec{u}_1] + c_2[\bvec{v}, \bvec{u}_2], \\
[c_1\bvec{v}_1+c_2\bvec{v}_2, \bvec{u}] &= c_1[\bvec{v}_1, \bvec{u}] + c_2[\bvec{v}_1, \bvec{u}_2]~.
\end{aligned}
\end{equation}

\item \textbf{反对称性} 对于任意 $\bvec{v}, \bvec{u}\in V$,有 $[\bvec{v}, \bvec{u}]=-[\bvec{u}, \bvec{v}]$。

\item \textbf{Jacobi结合性} 对于任意 $\bvec{x}, \bvec{y}, \bvec{z}\in V$,有

\begin{equation}
[\bvec{x}, [\bvec{y}, \bvec{z}]]+[\bvec{z}, [\bvec{x}, \bvec{y}]]+[\bvec{y}, [\bvec{z}, \bvec{x}]]=0
\end{equation}

\end{itemize}

由于李括号用有时也被视为乘法,因此有时也称之为\textbf{括积}。
\end{definition}

% 李代数是对李群的性质进行进一步的抽象,是对李群的一些核心性质的代数刻画。% 不大正确,李代数只是李群的“线性近似”

李代数定义中的双线性性可以通过反对称性和单线性性\footnote{即对于参与运算的双方中某一方线性就可以,比如对后面的线性:$[\bvec{v},a\bvec{x}+b\bvec{y}]=a[\bvec{v}, \bvec{x}]+b[\bvec{v}, \bvec{y}]$。}推出,所以和本书中群的定义一样,是有冗余的。

\begin{definition}{交换李代数}
设 $\mathfrak{g}$ 为域 $\mathbb{F}$ 上的 $n$ 维李代数,那么如果对于任意的 $\bvec{x},\bvec{y}\in\mathfrak{g}$,都有 $[\bvec{x}, \bvec{y}] = [\bvec{y}, \bvec{x}]$(等价的 $[\bvec{x}, \bvec{y}] = 0$),则称这是一个\textbf{交换李代数(commutative Lie algebra)}或\textbf{阿贝尔李代数(abelian Lie algebra)}。
\end{definition}

李代数和结合代数\autoref{AlgFie_def1}~\upref{AlgFie}(很多时候直接称代数)的关系极为紧密,每个结合代数都可以唯一地导出一个李代数:

\begin{theorem}{结合代数导出李代数}\label{LieAlg_the2}
设 $\mathfrak{A}$ 是域 $\mathbb{F}$ 上的一个结合代数,以 $\times$ 为环的乘法,那么如果定义
$$
[\bvec{x}, \bvec{y}] := \bvec{x}\times \bvec{y}-\bvec{y}\times \bvec{x}
$$
其中 $\bvec{x}, \bvec{y}\in\mathfrak{A}$,则 $(\mathfrak{A}, [\cdot, \cdot])$ 构成一个李代数。
\end{theorem}

\textbf{证明}:

封闭性是显然的,因为 $\times$ 是封闭的。

由这个括积的定义,反对称性也是显然的,因为 $\bvec{x}\times \bvec{y}-\bvec{y}\times \bvec{x}=-(\bvec{y}\times \bvec{x}-\bvec{x}\times \bvec{y})$ 恒成立。

由反对称性,我们只需要证明单线性性即可推出双线性性:$\forall \bvec{x}, \bvec{y}, \bvec{z}\in\frak{A}$ 和 $a, b\in\mathbb{F}$,有:
\begin{equation}
\begin{aligned}
{[a\bvec{x}+b\bvec{y}, \bvec{z}]}&=(a\bvec{x}+b\bvec{y})\times\bvec{z}-\bvec{z}\times(a\bvec{x}+b\bvec{y})\\
&=a\bvec{x}\times\bvec{z}+b\bvec{y}\times\bvec{z}-a\bvec{z}\times\bvec{x}-b\bvec{z}\times\bvec{y}\\
&=a(\bvec{x}\times\bvec{z}-\bvec{z}\times\bvec{x})+b(\bvec{y}\times\bvec{z}-\bvec{z}\times\bvec{y})\\
&=a[\bvec{x}, \bvec{z}]+b[\bvec{y}, \bvec{z}]
\end{aligned}
\end{equation}
%注意公式代码的第一行左边,两端加了{},这是因为不加的话第一行左边会消失,经实验发现是因为[]的影响。去掉[]则不需要额外加{}。

最后是Jacobi结合性:

\begin{equation}\label{LieAlg_eq1}
\begin{aligned}
{[\bvec{x}, [\bvec{y}, \bvec{z}]]}&=\bvec{x}\times (\bvec{y}\times\bvec{z}-\bvec{z}\times\bvec{y})-(\bvec{y}\times\bvec{z}-\bvec{z}\times\bvec{y})\times\bvec{x}\\
&=\bvec{x}\times(\bvec{y}\times\bvec{z})+(\bvec{z}\times\bvec{y})\times\bvec{x}-(\bvec{y}\times\bvec{z})\times\bvec{x}-\bvec{x}\times(\bvec{z}\times\bvec{y})\\
%&=(\bvec{x}\times\bvec{y})\times\bvec{z}+(\bvec{z}\times\bvec{y})\times\bvec{x}-(\bvec{y}\times\bvec{z})\times\bvec{x}-(\bvec{x}\times\bvec{z})\times\bvec{y}\\
\end{aligned}
\end{equation}

依次轮换$\bvec{x}, \bvec{y}$和$\bvec{z}$的位置,即可相互抵消,得到$[\bvec{x}, [\bvec{y}, \bvec{z}]]+[\bvec{z}, [\bvec{x}, \bvec{y}]]+[\bvec{y}, [\bvec{z}, \bvec{x}]]=0$。


由\autoref{LieAlg_eq1} ,轮换 $\bvec{x}, \bvec{y}$ 和 $\bvec{z}$ 的顺序后把得到的三个式子相加,即可以得到Jacobi结合性。


\addTODO{该证明不够清晰}

\textbf{证毕}。

知道了每个结合代数可以对应一个李代数后,我们自然会好奇,每个李代数是不是也能“对应”一个结合代数?答案是肯定的,见泛包络代数\upref{UEnAl}。
% 要注意的是,这种对应可能并不是直接对应,比如说下面\autoref{LieAlg_ex1} 中的三维叉积,就不是直接对应到一个三维空间的结合代数,而是用四元数结合代数导出四维李代数以后,商去其中心得到的商代数。% 我不明白这段话想表达什么


\begin{definition}{中心}\label{LieAlg_def1}
设 $\mathfrak{g}$ 为域 $\mathbb{F}$ 上的李代数,记 $C(\mathfrak{g})=\{\bvec{x}\in\mathfrak{g}|\forall\bvec{y}\in\mathfrak{g}, [\bvec{x}, \bvec{y}]=\bvec{0}\}$,称之为 $\mathfrak{g}$ 的\textbf{中心(center)}。
\end{definition}

李代数的中心,概念直接来自群等其它代数结构的中心,即“和所有元素的运算都交换的元素所构成的集合”。对李代数来说,交换意味着李括号的结果为 $0$。
% 李代数的特殊之处在于,两个元素若交换,则它们的乘积为 $\bvec{0}$,因此我们如上定义。



\subsection{例子}

结合代数对应李代数这一事实,使得我们很容易联想到一个常见的结合代数:矩阵代数。在本文中,将域 $\mathbb{F}$ 上的 $n$ 阶可逆方阵的集合记为 $\opn{gl}(n, \mathbb{F})$,那么这个集合自然构成 $\mathbb{F}$ 上的一个线性空间(以矩阵加法为向量加法),而矩阵乘法则使之构成一个环,因此这是一个结合代数。

\begin{example}{一般线性李代数}
域 $\mathbb{F}$ 上的 $\opn{gl}(n, \mathbb{F})$ 是一个结合代数。由这个结合代数可以导出李代数,其中对于矩阵 $\bvec{A}$ 和 $\bvec{B}$,括积的定义为 $[\bvec{A}, \bvec{B}]=\bvec{A}\bvec{B}-\bvec{B}\bvec{A}$。该李代数被称为一般线性李代数,名称和一般线性群 $\opn{GL}(n, \mathbb{F})$ 对应。% \footnote{见\autoref{LieGrp_ex1}~\upref{LieGrp}}。
\end{example}

\begin{example}{特殊线性李代数}
对于域 $\mathbb{F}$ 和正整数 $n$,记 $\opn{sl}(n, \mathbb{F})$ 为 $\opn{gl}(n, \mathbb{F})$ 中迹为 $0$ 的矩阵构成的结合代数,则它可以导出一个特殊线性李代数,名称也和特殊线性群 $\opn{SL}(n, \mathbb{F})$ 对应。
\end{example}

\begin{example}{三维向量叉积}\label{LieAlg_ex1}
域 $\mathbb{R}$ 上的三维线性空间 $\mathbb{R}^3$ 中,将括积定义为叉积:$\forall \bvec{v}, \bvec{u}\in \mathbb{R}^3$,有 $[\bvec{v}, \bvec{u}]=\bvec{v}\times\bvec{u}$。那么这个线性空间配上叉积可以得到一个李代数。
\end{example}

\subsection{结构常数}
\pentry{张量的坐标变换\upref{TrTnsr},爱因斯坦求和约定\upref{EinSum}}

李代数的括积的作用是把两个向量映射为一个向量,而且还要求具有双线性性,因此括积实际上是一个三阶张量。如果设 $\mathfrak{g}$ 为域 $\mathbb{F}$ 上的 $n$ 维李代数,而 $\{\bvec{x}_i\}$ 为它的一组基,那么对于任意的 $\bvec{x}_i$ 和 $\bvec{x}_j$,存在 $C^k_{ij}\in\mathbb{F}$,使得 $[\bvec{x}_i, \bvec{x}_j]=C^k_{ij}\bvec{x}_k$\footnote{注意使用爱因斯坦求和约定。}。这样的 $C^k_{ij}$ 就是括积张量的三维矩阵表示。

显然,括积张量的三维矩阵表示依赖于基的选取,和所有其它张量一样。当选定基以后,所得到的矩阵称为 $\mathfrak{g}$ 关于基 $\{\bvec{x}_i\}$ 的\textbf{结构常数}。

\begin{theorem}{}\label{LieAlg_the1}
设 $\mathfrak{g}$ 为域 $\mathbb{F}$ 上的 $n$ 维李代数,它在基 $\{\bvec{x}_i\}$ 下的\textbf{结构常数}为 $C^k_{ij}$,那么有:
\begin{enumerate}
\item $C^k_{ij}=-C^k_{ji}$。
\item $C^l_{ij}C^m_{lk}+C^l_{jk}C^m_{li}+C^l_{ki}C^m_{lj}=0$。
\end{enumerate}
\end{theorem}

\autoref{LieAlg_the1} 的证明思路简述如下,细节读者可自行补充:对于第一条,直接应用反交换性;对于第二条,考虑 $[\bvec{x}_i, [\bvec{x}_j, \bvec{x}_k]] = [\bvec{x}_i, C_{jk}^l \bvec{x}_l] = C_{jk}^l C_{il}^m \bvec{x}_m$,再应用反交换性就能直接得到式中左边的第二项,以此类推,应用Jacobi结合性即可得到整个式子。

结构常数唯一地对应李代数,也就是说,我们也可以任取一个线性空间,然后通过定义满足\autoref{LieAlg_the1} 的结构常数来定义括积,从而得到一个李代数。

\begin{theorem}{结构常数的变换}
设 $\mathfrak{g}$ 为域 $\mathbb{F}$ 上的 $n$ 维李代数,它在基 $\{\bvec{x}_i\}$ 下的\textbf{结构常数}为 $C^k_{ij}$,在基 $\{\bvec{y}_i\}$ 下的\textbf{结构常数}为 $D^k_{ij}$,而且有过渡矩阵\upref{TransM}$a_i^j$ 使得 $\bvec{y}_i=a_i^j\bvec{x}_j$,那么有变换法则:
\begin{equation}
D_{ij}^k a_k^j=a_i^m a_j^n C^j_{mn}
\end{equation}

\end{theorem}

\textbf{证明}:

\begin{equation}\label{LieAlg_eq2}
[\bvec{y}_i, \bvec{y}_j] = D_{ij}^k \bvec{y}_k = D_{ij}^k a_k^j \bvec{x}_j
\end{equation}

\begin{equation}\label{LieAlg_eq3}
[\bvec{y}_i, \bvec{y}_j]=[a_i^m \bvec{x}_m, a_j^n \bvec{x}_n]=a_i^m a_j^n[\bvec{x}_m, \bvec{x}_n]=a_i^m a_j^n C^j_{mn} \bvec{x}_j
\end{equation}

比较\autoref{LieAlg_eq2} 和\autoref{LieAlg_eq3} ,消去 $\bvec{x}_j$ 即可得 $D_{ij}^k a^k_j = a_i^m a_j^n C^j_{mn}$。

\textbf{证毕}。

回顾张量的坐标变换\upref{TrTnsr}词条,我们发现这就是三阶张量的坐标变换式。






