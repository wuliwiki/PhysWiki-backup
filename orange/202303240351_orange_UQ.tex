% 酉群
% 酉群|酉矩阵

\pentry{埃尔米特矢量空间(酉空间)\upref{HVorUV}}
如词条“埃尔米特矢量空间(酉空间)\upref{HVorUV}” 所说,酉群是埃尔米特矢量空间中所有标准正交基底之间的转换矩阵构成的群。现在来一步步的引入它。
\subsection{酉矩阵}
设 $\{ \hat e_i\},\{  \hat e'_i\}$ 是埃尔米特矢量空间 $V$ 的两个标准正交基底,从 $\{ \hat e_i\}$ 到 $\{\hat e'_i\}$ 的转换矩阵为 $\mat A=(a_{ij})$ ,即
\begin{equation}\label{UQ_eq1}
   \hat e'_j=\sum_{i}a_{ij} \hat e_i~.
\end{equation}
于是,基矢量间的纯量积满足(\autoref{HVorUV_def1}~\upref{HVorUV})
\begin{equation}
( \hat e'_j| \hat e'_k)=\sum_{i,s}\overline{a_{ij}}a_{sk}(\hat e_i|\hat e_s)=\sum_{i}\overline{a_{ij}}a_{ik}
\end{equation}
即
\begin{equation}\label{UQ_eq2}
\mat{A}^\dagger \mat{A}=\mat{E}=\mat{A} \mat{A}^\dagger
\end{equation}
其中,$\mat{A}^\dagger:=(\mat A^T)^*$,*表取共轭复数。
\begin{definition}{酉矩阵}
称矩阵 $\mat{A}$ 为\textbf{酉矩阵},若
\begin{equation}
\mat A^\dagger \mat A=\mat E
\end{equation}
\end{definition}
这就是说,在埃米尔特矢量空间里,两个标准正交基之间的转换矩阵为酉矩阵。反过来,若 $\mat A$ 为酉矩阵,则根据\autoref{UQ_eq1} ,由标准正交基底 $\{\hat e_i\}$ 得到的矢量组 $\{\hat e'_i\}$ 仍是标准正交基底。下面定理成立
\begin{theorem}{}
在埃米尔特矢量空间中,由一个标准正交基底到另一个标准正交基底的转换矩阵是酉矩阵,而且,所有酉矩阵都可以是这种转换矩阵。
\end{theorem}
\subsection{酉群}
\begin{example}{所有的酉矩阵构成一个群}
试证明:所有 $n$ 阶酉矩阵构成一个群。

\textbf{证明:}
\begin{enumerate}
\item \textbf{封闭性:}设 $\mat A,\mat B$ 是酉矩阵,则
\begin{equation}
(\mat{AB})^\dagger(\mat{AB})=\mat B^\dagger \mat A^\dagger\mat{AB}=\mat E
\end{equation}
\item \textbf{结合性:}由矩阵满足结合律,故酉矩阵满足结合律;
\item \textbf{逆元:}由于酉矩阵 $\mat A$ 的逆就是 $\mat A^\dagger$,而 $(\mat A^\dagger)^\dagger=\mat A$ ,所以\autoref{UQ_eq2} 表明 $\mat A^\dagger$ 也是酉矩阵,所以 $A^{-1}$ 即酉矩阵;
\item \textbf{单位元:} $\mat E^\dagger\mat E=\mat E$
\end{enumerate}
由群的定义(\autoref{Group_def1}~\upref{Group}),所有的酉矩阵构成一个群。

\textbf{证毕!}
\end{example}
\begin{definition}{酉群}
所有 $n$ 酉矩阵构成的集合称为\textbf{酉群},记作 $U(n)$。
\end{definition}
\begin{theorem}{}
所有 $n$ 阶实的正交矩阵构成的正交群 $O(n)$ 是酉群 $U(n)$ 的一个子群。
\end{theorem}
\textbf{证明:}由正交矩阵的定义,若 $A$ 是正交矩阵,则 $\mat A^T=\mat A$,又正交矩阵的矩阵元是实数,故 $\mat A^*=\mat A$。所以
\begin{equation}
\mat A^\dagger\mat A=(\mat A^T)^*\mat A=\mat A^T\mat A=\mat E
\end{equation}
这就是说,正交矩阵本身也是个酉矩阵,故 $O(n)\subset U(n)$。
\begin{theorem}{$SU(n)$}
所有行列式为1的 $n$ 阶酉矩阵构成一个特殊的酉群 $SU(n)$。
\end{theorem}
\textbf{证明:}
\begin{enumerate}
\item \textbf{封闭性:}设 $\mat A,\mat B\in SU(n)$,则
\begin{equation}
\det{(\mat{AB})}=\det{\mat A}\det{\mat B}=1
\end{equation}
\item \textbf{结合性、逆元存在性}显然,\textbf{单位元}即 $\mat E$
\end{enumerate}
故 $SU(n)$ 是一个群。

\textbf{证毕!}