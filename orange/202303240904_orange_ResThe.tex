% 留数定理
% 复变函数|residue|柯西积分定理|洛朗级数|围道积分|约尔当引理|Jordan 引理|Jordan's Lemma|若尔当引理|若当引理

\pentry{洛朗级数\upref{LaurSr}}

留数定理是用来计算围道积分的工具,也可以借用围道积分的性质来计算实函数的积分。

\subsection{定理的导出}

我们首先讨论最简单的洛朗级数 $f(z)=1/z$ 的围道积分。

\begin{example}{}\label{ResThe_ex1}
考虑 $f(z)=1/z$ 的围道积分。

令积分路径为 $\Gamma(t)=\rho\E^{\I t}$,其中 $\rho$ 是一个常数。$\Gamma$ 就是一个绕着一个半径为 $\rho$ 的圆的逆时针路径。$f$ 沿 $\Gamma$ 的围道积分就是:
\begin{equation}\label{ResThe_eq1}
\begin{aligned}
\int_{\Gamma}\frac{1}{z}\dd z&=\int^{2\pi}_0\frac{1}{\rho\E^{\I t}}\cdot\frac{\dd\rho\E^{\I t}}{\dd t}\dd t\\
&=\int_0^{2\pi}\frac{1}{\rho\E^{\I t}}\cdot (\I \rho \E^{\I t})\dd t\\
&=\int_0^{2\pi}\I\dd t\\
&=2\pi\I~.
\end{aligned}
\end{equation}

对于任意逆时针环绕原点的路径 $P(t)$,我们可以求 $\int_{\Gamma}\frac{1}{z}\dd z-\int_{P}\frac{1}{z}\dd z$,其中 $\Gamma$ 完全包裹在 $P$ 中。这相当于求 $\Gamma$ 和 $P$ 之间区域的围道积分。这个中间区域是不包含原点的,因此由于 $1/z$ 在原点之外处处解析,由\textbf{柯西积分定理}\upref{CauGou}知,中间区域的围道积分为 $0$。从而我们可以推知,$\int_{P}\frac{1}{z}\dd z=2\pi\I$。

更一般地,对于任意环绕原点的路径 $P$,有 $\int_{P}\frac{1}{z}\dd z=\pm 2\pi\I$,其中 $P$ 为逆时针时取正号,反之取负号。

\end{example}

接着是幂次更低的洛朗级数的围道积分。

\begin{example}{}\label{ResThe_ex2}
考虑 $f(z)=1/z^n$ 的围道积分,其中整数 $n>1$。

同样令积分路径为 $\Gamma(t)=\rho\E^{\I t}$,其中 $\rho$ 是一个常数。$f$ 沿 $\Gamma$ 的围道积分就是:

\begin{equation}
\begin{aligned}
\int_{\Gamma}\frac{1}{z^n}\dd z&=\int^{2\pi}_0\frac{1}{(\rho\E^{\I t})^n}\cdot\frac{\dd\rho\E^{\I t}}{\dd t}\dd t\\
&=\int_0^{2\pi}\frac{1}{(\rho\E^{\I t})^n}\cdot (\I \rho \E^{\I t})\dd t\\
&=\int_0^{2\pi}\frac{\I}{\rho^{n-1}}\E^{(1-n)\I t}\dd t\\
&=\frac{1}{(1-n)\rho^{n-1}}\E^{(1-n)\I t}\mid^{t=2\pi}_{t=0}\\
&=0~.
\end{aligned}
\end{equation}

和\autoref{ResThe_ex1} 一样地,可以将结论推广为,对于任意积分路径 $P$,都有
\begin{equation}
\int_P\frac{1}{z^n}\dd z=0~.
\end{equation}

\end{example}

而洛朗级数的正则部分是一个泰勒级数,其围道积分处处为 $0$。结合\autoref{ResThe_ex1} 和\autoref{ResThe_ex2} ,我们可以得到以下定理:

\begin{definition}{留数}

取洛朗级数 $f(z)=\sum\limits_{n=-\infty}^{\infty} a_n(z-c)^n$,定义 $a_{-1}$ 为 $f$ 在 $c$ 处的\textbf{留数(residue)},记为 $\opn{Res}[f, c]$,或者 $\opn{Res}_cf$。

\end{definition}

\begin{theorem}{留数定理(单个极点)}\label{ResThe_the1}
洛朗级数 $f(z)=\sum\limits_{n=-\infty}^{\infty} a_n(z-c)^n$ 沿闭合路径 $\Gamma$ 逆时针绕点 $c$ 一周的围道积分为
\begin{equation}
\int_\Gamma f(z)\dd z=2a_{-1}\pi\I~,
\end{equation}
这一结果也可以表述为
\begin{equation}
\int_\Gamma f(z)\dd z=2\pi\I\opn{Res}_cf~.
\end{equation}

\end{theorem}

有时候回路中会有多个极点,此时我们要做的就是把这些极点的留数求和:

\begin{theorem}{留数定理}\label{ResThe_the3}

如果连续复变函数 $f(z)$ 在闭合路径 $\Gamma$ 中有多个极点,那么沿 $\Gamma$ 的回路积分为

\begin{equation}
\int_\Gamma f(z)\dd z=2\pi\I\sum_{c}\opn{Res}_cf~.
\end{equation}
其中 $c$ 遍历所有包含在 $\Gamma$ 中的极点。


\end{theorem}



\begin{example}{}\label{ResThe_ex3}

考虑函数$f(z)=\E^{\I z^2}$在$x$轴上的积分,即$\int^\infty_{-\infty} f(x) \dd x$。

设$\Gamma_1$为实数轴上从$0$到$\rho$的路径,$\Gamma_2$为从$\rho$到$\E^{\pi\I/4}$的\textbf{八分之一}圆弧路径,$\Gamma_3$为上述圆弧半径上从$\E^{\pi\I/4}$到$0$的路径。于是有
\begin{equation}
\int^\rho_{0} f(x) \dd x = \int_{\Gamma_1}f(z)\dd z~.
\end{equation}

\begin{figure}[ht]
\centering
\includegraphics[width=10cm]{./figures/ResThe_1.pdf}
\caption{\autoref{ResThe_ex3} 的积分路径。} \label{ResThe_fig1}
\end{figure}

由于$f(z)$在整个复平面上都没有极点,因此据\autoref{ResThe_the3} ,
\begin{equation}
\sum_{i=1, 2, 3}\int_{\Gamma_i} f(z) \dd z = 0~.
\end{equation}

由于$f(x)$作为实函数是偶函数,故$\int^\infty_{-\infty} f(x)\dd x=2\int^\infty_0 f(x)\dd x$。于是我们可以根据$f(z)$在$\Gamma_2$和$\Gamma_3$上的围道积分计算出\autoref{ResThe_eq1} ,再取$\rho\to\infty$的极限得到所求。

先计算$f(z)$沿着$\Gamma_2$的积分:
\begin{equation}
\ali{
    \int_{\Gamma_2} f(z) \dd z &= \int_0^{\pi/4}\E^{\I \rho^2\E^{2\I t}}\cdot\I\rho\E^{\I t}\dd t\\
    &= \I\rho \cdot (-\frac{1}{2\rho^2}) \E^{\I\rho^2\E^{2\I t}}\mid^{t=\pi/4}_{t=0}\\
    &= -\frac{\I}{2\rho}\qty(\E^{-\rho^2}-\E^{\I\rho^2})~,
}
\end{equation}
于是显然可知,
\begin{equation}
\lim_{\rho\to+\infty}\int_{\Gamma_2} f(z) \dd z = 0~.
\end{equation}

再计算$f(z)$沿着$\Gamma_3$的积分:
\begin{equation}
\ali{
    \int_{\Gamma_3} f(z) \dd z &= \int^0_\rho \E^{\I \qty(\frac{\sqrt{2}}{2}t+\frac{\sqrt{2}}{2}\I t)^2} \cdot \frac{\mathrm{d}\qty(\frac{\sqrt{2}}{2}t+\frac{\sqrt{2}}{2}\I t)}{\dd t}\dd t\\
    &= \qty(\frac{\sqrt{2}}{2}+\frac{\sqrt{2}}{2}\I ) \cdot \int^0_\rho \E^{\I \qty(\frac{\sqrt{2}}{2}t+\frac{\sqrt{2}}{2}\I t)^2} \cdot \dd t\\
    &= \qty(\frac{\sqrt{2}}{2}+\frac{\sqrt{2}}{2}\I ) \cdot \int^0_\rho \E^{-t^2} \cdot \dd t\\
}
\end{equation}
由\textbf{高斯积分}\upref{GsInt}的\autoref{GsInt_eq1}~\upref{GsInt},可知
\begin{equation}
\lim_{\rho\to+\infty}\int_{\Gamma_3} f(z) \dd z = \qty(\frac{\sqrt{2}}{4}+\frac{\sqrt{2}}{4}\I )\sqrt{\pi}=\sqrt{\frac{\pi}{8}}\qty(1+\I)~,
\end{equation}

$\Gamma_2$和$\Gamma_3$的积分得到了,就可以进行最后的计算:

\begin{equation}\label{ResThe_eq2}
\ali{
    \int^\infty_{-\infty} f(x) &= 2\int^\infty_0 f(x)\dd x\\
    &=-2\lim_{\rho\to +\infty}\qty[ \int_{\Gamma_2} f(z)\dd z + \int_{\Gamma_3} f(z)\dd z ]\\
    &= -2\cdot \sqrt{\frac{\pi}{8}}\qty(1+\I)\\
    &=\sqrt{\frac{\pi}{2}}(1+\I)~,
}
\end{equation}

注意到$\sqrt{\I}=\sqrt{\frac{1}{2}}(1+\I)$,\autoref{ResThe_eq2} 也可写为
\begin{equation}\label{ResThe_eq3}
\ali{
    \int^\infty_{-\infty} f(x) &= \sqrt{\frac{\pi}{-\I}}~.
}
\end{equation}
这和\textbf{高斯积分}\upref{GsInt}的\autoref{GsInt_eq5}~\upref{GsInt}相符。

类似地可证
\begin{equation}
\int^\infty_{-\infty} \E^{-\I x^2}\dd x = \sqrt{\frac{\pi}{2}}(1-\I)~.
\end{equation}



\end{example}









\subsection{留数的计算}



如果能把给定函数 $f$ 在给定点的洛朗展开写出来,我们直接就能得到对应的留数,用于计算围道积分。但那样常常非常麻烦,也没有必要,因为留数只是洛朗级数中的一个系数,通常没必要为了求这一个系数而把整个级数算出来。

\begin{lemma}{}\label{ResThe_lem1}
如果 $f(z)$ 是一个在 $c\in\mathbb{C}$ 处解析的函数,那么 $\opn{Rez}_c\frac{f(z)}{z-c}=f(c)$。
\end{lemma}

\autoref{ResThe_lem1} 的证明很简单,将 $f(z)$ 在 $c$ 处泰勒展开以后再除以 $z-c$ 即可。

利用\autoref{ResThe_lem1} ,我们可以快速计算出很多常见函数的留数。

\begin{example}{}
求 $\frac{1}{z(z-2)}$ 在 $z=2$ 处的留数。

由于 $\frac{1}{z}$ 在 $z=2$ 处解析,因此根据\autoref{ResThe_lem1} ,代入 $f(z)=\frac{1}{z}$ 和 $c=2$ 可知,
\begin{equation}
\opn{Rez}_2\frac{1}{z(z-2)}=\frac{1}{2}~.
\end{equation}
\end{example}

用证明\autoref{ResThe_lem1} 的方法进行推广,还可以得到更一般的定理:

\begin{theorem}{}\label{ResThe_the2}
如果 $f(z)$ 是一个在 $c\in\mathbb{C}$ 处解析的函数,$k$ 是一个正整数,那么 $\opn{Rez}_c\frac{f(z)}{(z-c)^k}=\frac{f^{(k-1)}(c)}{(k-1)!}$。
\end{theorem}

\textbf{证明}:

由于$f(z)$在$c$处解析,因此可以在$c$处展开为
\begin{equation}
f(z) = \sum_{n=0}^\infty a_n(z-c)^n~.
\end{equation}
于是
\begin{equation}
\frac{f(z)}{(z-c)^k} = \sum_{n=0}^\infty a_n(z-c)^{n-k} = \sum_{n=-k}^\infty a_{n+k}(z-c)^n~,
\end{equation}
因此
\begin{equation}
\opn{Rez}_c\frac{f(z)}{(z-c)^k} = a_{k-1}~,
\end{equation}

而
\begin{equation}
f^{k-1}(c)=(k-1)!a_{k-1}~.
\end{equation}

\textbf{证毕}。




在很多材料中会把点 $c$ 称为 $f(z)/(z-c)^k$ 的 $k$\textbf{级极点},其中 $f(z)$ 在 $c$ 处解析。

\begin{example}{}
计算 $\opn{Rez}_0 \frac{\E^{2z}}{z^2}$。

在 $0$ 处,$\frac{\E^{2z}}{z^2}$ 有一个二级极点。根据\autoref{ResThe_the2} ,将 $k=2$、$f(z)=\E^{2z}$ 和 $c=0$ 代入,可得:
\begin{equation}
\opn{Rez}_0\frac{\E^{2z}}{z^2}=\frac{(\E^{2z})'\mid_{z=0}}{1!}=2~.
\end{equation}


\end{example}












