% 内积、内积空间
% 线性代数|内积空间|向量空间|线性空间

\pentry{几何向量的内积\upref{Dot}, 范数、赋范空间\upref{NormV}}

\footnote{参考 Wikipedia \href{https://en.wikipedia.org/wiki/Inner_product_space}{相关页面}。}在向量空间中, 我们可以另外定义任意两个向量的\textbf{内积(inner product)}运算, 运算的结果是一个实数或复数。 内积运算不是向量空间所必须的, 但物理中的向量空间几乎都定义了内积运算。 我们把定义了内积运算的空间称为\textbf{内积空间(inner product space)}, 完备的内积空间也叫\textbf{希尔伯特空间\upref{Hilber}(Hilbert space)}。 本词条中我们使用狄拉克符号\upref{braket}。

\begin{definition}{}
对域 $\mathbb F$ 上的向量空间\upref{LSpace}, 定义\textbf{内积}为满足以下条件的二元运算 $\braket{\cdot}{\cdot}: V\times V\to \mathbb F$。 令向量 $u, v\in V$ 以及标量 $a, b \in \mathbb F$
\begin{itemize}
\item \textbf{正定}
\begin{equation}\label{InerPd_eq3}
\braket{v}{v} > 0 \quad (v \ne 0)~.
\end{equation}
\item \textbf{线性}
\begin{equation}\label{InerPd_eq4}
\braket{w}{a u + b v} = a\braket{w}{u} + b\braket{w}{v}
\end{equation}
\item \textbf{交换律}
\begin{equation}\label{InerPd_eq2}
\braket{u}{v} = \braket{v}{u}^*
\end{equation}
\end{itemize}
\end{definition}
说明:
\begin{itemize}
\item 定义中我们使用了狄拉克符号\upref{braket}, 也有地方将内积记为 $\ev{u, v}$。
\item 向量空间依赖于一个域 $\mathbb F$ 通常取实数或复数域。 对于实数(复数)域的向量空间, 内积的结果必须是实数(复数)。 对实数的情况, 定义中的共轭(*)可以略去。
\item 由\autoref{InerPd_eq2}  得无论 $\mathbb F$ 是什么, 一个向量和自身的内积必须是\textbf{实数}(复共轭等于自身的数都是实数)。 又根据\autoref{InerPd_eq3}, 该实数大于零。
\item 令\autoref{InerPd_eq4} 中 $a,b= 0$ 可得\textbf{零向量和任何向量的内积为零}。 如果两个向量内积为零, 我们就说他们是\textbf{正交(orthogonal)}的, 这意味着\textbf{零向量和任何向量正交}。
\item 根据\autoref{InerPd_eq4}, 零向量和任何向量的内积都必定是 0。 结合\autoref{InerPd_eq2} 和\autoref{InerPd_eq4}, 可以得到 $\braket{a u + b v}{w} = a^*\braket{w}{u} + b^*\braket{w}{v}$。
\item 结合\autoref{InerPd_eq4} 和\autoref{InerPd_eq2} 可得 $\braket{\sum_i a_i u_i}{\sum_j b_j v_j} = \sum_{i,j} a_i^* b_j \braket{u_i}{v_j}$。
\end{itemize}

内积一个重要性质就是满足柯西不等式\upref{CSNeq}
\begin{equation}
\abs{\braket{u}{v}}^2 \leqslant \braket{u}{u} \cdot \braket{v}{v}
\end{equation}
即两个向量内积绝对值的平方小于它们分别和自身内积再相乘。 由柯西不等式可以证明内积空间必然可以定义范数\upref{NormV}(证明见下文)
\begin{equation}\label{InerPd_eq8}
\norm{v} = \sqrt{\braket{v}{v}}
\end{equation}
所以内积空间属于赋范空间\upref{NormV}。

\subsection{内积的坐标表示}
$N$ 维内积空间中, 必存在 $N$ 个正交归一基底, 任意向量 $v$ 可以在这组基底上找到对应的坐标 $(v_1, \dots, v_N)$, 那么任意两个向量 $u, v$ 的内积可以用坐标表示为
\begin{equation}\label{InerPd_eq6}
\braket{u}{v} = \sum_i u_i^* v_i
\end{equation}
但若基底不是正交的, 令基底为 $\beta_1, \dots, \beta_N$, 那么根据内积的线性容易证明两个向量的内积为
\begin{equation}\label{InerPd_eq7}
\braket{u}{v} = \bvec u\Her \mat M \bvec v = \sum_{i,j} M_{i,j} u_i^* v_j
\end{equation}
其中矩阵 $\mat M$ 的矩阵元为 $M_{i,j} = \braket{\beta_i}{\beta_j}$。 当 $\braket{\beta_i}{\beta_j} = \delta_{i,j}$ 时, $\mat M$ 变为单位矩阵, 就回到了\autoref{InerPd_eq6}。 可见\autoref{InerPd_eq7} 才是最一般坐标内积公式。

\subsection{勾股定理}
内积空间的\textbf{勾股定理(Pythagorean theorem)}: 对任意两个正交的向量, 有
\begin{equation}\label{InerPd_eq1}
\braket{u + v}{u + v} = \braket{u}{u} + \braket{v}{v}
\end{equation}
证明:
\begin{equation}
\braket{u + v}{u + v} = \braket{u}{u} + \braket{v}{v} + \braket{u}{v} + \braket{v}{u}
\end{equation}
根据正交的定义, $\braket{u}{v} = 0$。 证毕。

\subsection{证明内积必定是范数}
要证 $\braket{x}{x}$ 满足范数的要求, 最关键是证明性质
\begin{equation}
\norm{x+y}^2 \leqslant (\norm{x} + \norm{y})^2 = \norm{x}^2 + \norm{y}^2 + 2\norm{x}\norm{y}
\end{equation}
即证
\begin{equation}
\braket{x+y}{x+y} - \braket{x}{x} - \braket{y}{y} \leqslant 2\norm{x}\norm{y}
\end{equation}
其中
\begin{equation}
\braket{x+y}{x+y} = \braket{x}{x} + \braket{y}{y} + 2\Re[{\braket{x}{y}}]
\end{equation}
代入, 即证
\begin{equation}\label{InerPd_eq5}
\Re[{\braket{x}{y}}]^2 \leqslant \norm{x}^2\norm{y}^2 = \braket{x}{x}\braket{y}{y}
\end{equation}
由柯西不等式
\begin{equation}
\Re[\braket{x}{y}]^2 + \Im[\braket{x}{y}]^2 = \abs{\braket{x}{y}}^2 \leqslant \braket{x}{x}\braket{y}{y}
\end{equation}
而 $\Im[\braket{x}{y}]^2 > 0$。 证毕。
