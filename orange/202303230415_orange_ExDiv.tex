% 多项式的整除
% 因式|倍式|整除


\pentry{带余除法\upref{DivAlg}}
\footnote{吴群。矩阵分析[M].上海:同济大学出版社}有了带余除法\upref{DivAlg} ,我们自然会想到像整数一样来定义多项式的整除、因式、倍式等概念。同时,带余除法还给出了多项式整除性的一个判别法。
\begin{definition}{整除、因式、倍式}\label{ExDiv_def1}
设 $f(x),g(x)$ 为数域 $\mathbb{F}$ 上的多项式,若存在数域 $\mathbb{F}$ 上的多项式 $q(x)$ 使得 $f(x)=q(x)g(x)$,则称 $g(x)$ \textbf{整除} $f(x)$,记作 $g(x)|f(x)$,并称 $g(x)$ 为 $f(x)$ 的\textbf{因式},$f(x)$ 为 $g(x)$ 的\textbf{倍式}.
\end{definition}
由\autoref{ExDiv_def1} ,容易验证:
\begin{enumerate}
\item 任意多项式一定整除自身:$f(x)|f(x)$;
\item 零多项式只整除零多项式:$0|0$;
\item 任意一个非零多项式一定整除零多项式:$f(x)|0$;
\item 零次多项式整除任意多项式:$c|f(x)$,其中 $c\neq0\in\mathbb{F}$。
\end{enumerate}
根据这4条性质,规定:零次多项式,以及 $f(x)$ 的与其次数相同的因式,称为 $f(x)$ 的\textbf{平凡因式};零多项式,以及 $f(x)$ 的与其次数相同的倍式,称为 $f(x)$ 的\textbf{平凡倍式}.

有了整除的概念,考虑到带余除法,自然容易想到,两个多项式能够整除是其余式为0。因此,下面定理是显然的
\begin{theorem}{}
设 $f(x),g(x)$ 为数域 $\mathbb{F}$ 上的两个多项式,$g(x)\neq 0$ ,则 $g(x)|f(x)$ 的充要条件是 $g(x)$ 除 $f(x)$ 的余式为0,即存在多项式 $q(x)$ 使得 $f(x)=q(x)g(x)$。
\end{theorem}
\begin{definition}{多项式的商}
若 $g(x)|f(x)$,则 $g(x)$ 除 $f(x)$ 的\textbf{商} $q(x)$ 可表示 
\begin{equation}
q(x)=\frac{f(x)}{g(x)}~.
\end{equation}

\end{definition}

整除具有下面性质:
\begin{enumerate}
\item 若 $f(x)|g(x)$ 且 $g(x)|f(x)$ ,则 $f(x)=cg(x)$,$c$ 为非0常数;
\item 若 $f(x)|g(x)$ 且 $g(x)|h(x)$ ,则 $f(x)|h(x)$;
\item 若 $f(x)|g_i(x)\;(i=1,2,\cdots,n)$,则 $f(x)|\sum\limits_{i=1}^{n}u_i(x)g_i(x)$ , $u_i(x)$ 为数域 $\mathbb{F}$ 上的多项式。
\end{enumerate}

有了因式,当然可以定义两个多项式的公因式及最大公因式。
\begin{definition}{因式、最大公因式}\label{ExDiv_def2}
设 $f(x),g(x)$ 为数域 $\mathbb{F}$ 上的两个多项式,若数域 $\mathbb{F}$ 上的多项式 $d(x)$ 满足
\begin{equation}
d(x)|f(x)\quad\&\quad d(x)|g(x)~.
\end{equation}
则称 $d(x)$ 为 $f(x)$ 与 $g(x)$ 的一个\textbf{公因式}。
若 $f(x)$ 与 $g(x)$ 的任意公因式都是它们公因式 $d(x)$ 的因式,则称 $d(x)$ 为 $f(x)$ 与 $g(x)$ 的一个\textbf{最大公因式},并把 $f(x)$ 与 $g(x)$ 的\textbf{首项系数为1}的最大公因式记为 $(f(x),g(x))$。
\end{definition}

由于对任意多项式 $f(x)$ 和数域 $\mathbb{F}$ 中的数 $c\neq 0$,都有 $c|f(x)$,所以非零多项式的最大公因式总是一个非零多项式。而 $f(x)$ 是 $f(x)$ 与0的一个最大公因式。

整除性质1表明,若 $d(x),d_1(x)$ 是 $f(x)$ 和 $g(x)$ 的两个最大公因式,则 $d(x)=cd_1(x)$,其中 $c\neq 0$ 为常数。
\begin{example}{整除性质1的证明}\label{ExDiv_ex1}
试证明:若 $f(x)|g(x)$ 且 $g(x)|f(x)$ ,则 $f(x)=cg(x)$,$c$ 为非0常数。

\textbf{证明:}显然,若 $f(x)=0$,由 $f(x)|g(x)$,有 $g(x)=0$,则 $f(x)=cg(x)$ 显然成立。下面考虑 $f(x)\neq 0$ 情形。

条件 $f(x)|g(x)$ 和 $g(x)|f(x)$ 意味着
\begin{equation}
f(x)=q(x)g(x)\quad g(x)=q_1(x)f(x)~.
\end{equation}
所以
\begin{equation}
f(x)=q(x)q_1(x)f(x)~.
\end{equation}
由于\autoref{DivAlg_the1}~\upref{DivAlg},必有
\begin{equation}\label{ExDiv_eq1}
q(x)q_1(x)=1~,
\end{equation}
上式两边取次数,由\autoref{OnePol_eq4}~\upref{OnePol},
\begin{equation}
\mathrm{deg}\;q(x)+\mathrm{deg}\;q_1(x)=0~,
\end{equation}
显然 $q(x),q_1(x)\neq 0$,而非零多项式的次数必定大于等于0,所以要上式成立,只能
\begin{equation}
\mathrm{deg}\;q(x)=\mathrm{deg}\;q_1(x)=0~,
\end{equation}
结合\autoref{ExDiv_eq1} ,有
 \begin{equation}
 q(x)=1/q_1(x)=c\neq 0~.
 \end{equation}
 即 $f(x)=cg(x)$。

\end{example}

