% 斜对称映射
% 斜对称映射

\begin{issues}
\issueDraft
\issueOther{已移动到线性代数中,应当做适当的调整}
\end{issues}

\pentry{矢量空间\upref{LSpace},线性映射\upref{LinMap}}

\begin{definition}{斜对称映射}\label{AntMap_def1}
设 $X,Y$ 是任一集合,其中 $Y$ 上的任一元素 $y$ 都有一元 $-y=-1\cdot y$ 与之对应,且 $y$ 上的元素与 $-1$ 的作用满足 $\underbrace{(-1)\cdot((-1)\cdots((-1)}_{n\text{个}}\cdot y)\cdots)=(-1)^n y$。

 $n$ 元映射 $f:X^n\rightarrow Y$ 叫作\textbf{斜对称的},若对任意 $k=1,\cdots,n-1$,成立
\begin{equation}\label{AntMap_eq3}
f(\cdots,x_k,x_{k+1},\cdots)=-f(\cdots,x_{k+1},x_k,\cdots)~,
\end{equation}
即当交换任意两个相邻变量的位置时,映射值变号。
\end{definition}
\begin{theorem}{}\label{AntMap_the1}
交换任意两个变量的位置,斜对称映射都变号。
\end{theorem}
\textbf{证明:}设交换第 $i,j$ 个变量的位置,且 $i<j$,那么位于 $i,j$ 之间的自变量的个数 $l=j-i-1$。利用数学归纳法证明如下: $l=0$ 时正好满足斜对称的定义,故成立。设 $l\leq k$ 时定理成立,那么 $l=k$ 时有
\begin{equation}
\begin{aligned}
f(\cdots,x_i,x_{i+1},\cdots,x_{j-1},x_j,\cdots)&=-f(\cdots,x_{i+1},x_{i},\cdots,x_{j-1},x_j,\cdots)\\
&=(-1)^2 f(\cdots,x_{i+1},x_{j},\cdots,x_{j-1},x_i,\cdots)\\
&=(-1)^3 f(\cdots,x_{j},x_{i+1},\cdots,x_{j-1},x_i,\cdots)\\
&=-f(\cdots,x_{j},x_{i+1},\cdots,x_{j-1},x_i,\cdots)
\end{aligned}
\end{equation}
  
\textbf{证毕!}
\begin{example}{外积}\label{AntMap_ex1}
在斜对称映射的帮助下,可以直接讨论外积(wedge)的一些性质,而不关心具体的集合。外积运算 $\wedge$ 具有结合性和斜对称性的性质:
\begin{enumerate}
\item \textbf{结合性}:$x_1\wedge(x_2\wedge x_3)=(x_1\wedge x_2)\wedge x_3$
\item \textbf{斜对称性}:$x_1\wedge x_2=-x_2\wedge x_1$
\end{enumerate}

定义映射 $f:V^k\rightarrow \Lambda^k(V)$:
\begin{equation}
f(x_1,\cdots,x_k)=x_1\wedge\cdots\wedge x_k
\end{equation}
其中
\begin{equation}
\Lambda^k(V)=\{x_1\wedge\cdots\wedge x_k| x_i\in V,i=1,\cdots,k\}
\end{equation}
满足\autoref{AntMap_def1} 中集合 $Y$ 的一切性质。

$\wedge$ 的结合性意味着括号怎么摆放都不重要;斜对称性则意味着,对 $\forall i=1,\cdots,k-1$,都成立
\begin{equation}
\begin{aligned}
f(x_1,\cdots,x_i,x_{i+1},\cdots,x_k)&=-f(x_1,\cdots,x_{i+1},x_i,\cdots,x_k)\\
&\Downarrow\\
x_1\wedge\cdots\wedge (x_i\wedge x_{i+1})\wedge\cdots\wedge x_k&=-x_1\wedge\cdots\wedge (x_{i+1}\wedge x_{i})\wedge\cdots\wedge x_k
\end{aligned}
\end{equation}
于是由\autoref{AntMap_the1} ,\textbf{任意交换两个变量的位置,$f$ 变号},即
\begin{equation}
x_1\wedge\cdots\wedge x_i\wedge\cdots\wedge x_j\wedge\cdots\wedge x_k=-x_1\wedge\cdots\wedge x_j\wedge\cdots\wedge x_i\wedge\cdots\wedge x_k
\end{equation}

设 $\pi$ 是个将集合 $\{1,\cdots,k\}$ 重新排序的映射(这样的映射成为置换)。比如$k=3$, $\pi$ 将 $1,2,3$ 重新排序为 $2,1,3$ ,那么
\begin{equation}
\pi 1=2,\quad\pi2=1,\quad \pi3=3
\end{equation}
于是
\begin{equation}\label{AntMap_eq1}
x_{i_1}\wedge\cdots\wedge x_{i_k}
\end{equation}
可通过某个 $\pi$ 按升序排列为
\begin{equation}\label{AntMap_eq2}
x_{i_{\pi1}}\wedge\cdots\wedge x_{i_{\pi k}}
\end{equation}
使得 $i_{\pi1}<\cdots< i_{\pi k}$。由于任意置换都和通过依次进行两两交换\upref{permu}得到(即先交换某两个数再交换某两个数)。假如\autoref{AntMap_eq1} 到\autoref{AntMap_eq2} 进行这样两两交换的次数为 $n$,
那么由\autoref{AntMap_eq3} ,就有
\begin{equation}
x_{i_{\pi1}}\wedge\cdots\wedge x_{i_{\pi k}}=(-1)^p x_{i_1}\wedge\cdots\wedge x_{i_k}
\end{equation}
事实上,$p$ 是奇数还是偶数取决于 $\pi$,并称 $\epsilon_\pi=(-1)^p$ 为置换 $\pi$ 的符号,于是上式便为
\begin{equation}
x_{i_{\pi1}}\wedge\cdots\wedge x_{i_{\pi k}}=\epsilon_\pi (x_{i_1}\wedge\cdots\wedge x_{i_k})
\end{equation}
这就是说\textbf{任意一个乘积 $x_{i_1}\wedge\cdots\wedge x_{i_k}$ 可按因子(即$x_i$)的升序进行排列}。

关于外积,参加外代数\upref{ExtAlg}。
\end{example}
