% 极值的必要条件(变分学)
% 极值

\pentry{变分\upref{Varia}}

正如\autoref{AbPol_exe1}~\upref{AbPol}所示,绝对极值必定是相对极值,相对强的极值必定是相对弱的极值。所以,弱的相对极值的必要条件也是绝对极值和强的相对极值的必要条件。因为这个原因,弱的极值的必要条件就称为\textbf{极值的必要条件}。

曲线 $y=y(x)$ 实现了泛函 $J(y)$ 的极值的必要条件是:在曲线 $y=y(x)$ 的某个一级 $\epsilon-$ 邻区中,对任一曲线 $y=\overline{y}(x)$, $\Delta J=J(\overline{y})-J(y)$ 符号一定。

对于泛函 $J(y)=\int_a^bF(x,y,y')\dd x$ ,有下面关于极值的必要条件定理
\begin{theorem}{}\label{PolReq_the1}
设 $y(x)$ 是 $C_1$ 类的\autoref{Varia_sub1}~\upref{Varia},且 $y(a)=y_0,y(b)=y_1$。则 $y(x)$ 给出泛函 $J(y)=\int_a^bF(x,y,y')\dd x$ 的极值的必要条件是:变分
\begin{equation}
\delta J=\int_a^b[F'_y(x,y,y')\eta(x)+F'_{y'}(x,y,y')\eta'(x)]\dd x
\end{equation}
对于 $C_1$ 类的任意满足 $\eta(a)=\eta(b)=0$ 的函数 $\eta(x)$,成立 $\delta J=0$。
\end{theorem}
\subsection{证明}
由弱的相对极值的定义\upref{AbPol} ,若 $y(x)$ 实现 $J(y)$ 的弱的相对极值,则对 $y(x)$ 的某个一级 $\epsilon-$ 邻区中,任意 $\overline{y}(x)$ ,在弱极大时, $\Delta J=J(\overline{y})-J(y)\leq 0$;在若极小时,$\Delta J=J(\overline{y})-J(y)\geq 0$。即 $\Delta J$ 符号一定。

为证明\autoref{PolReq_the1} ,先引入一个引理。
\begin{lemma}{}\label{PolReq_lem1}
设 $d$ 是常数, $\alpha$ 可以任意方式趋于0, $\epsilon_\alpha$ 是 $\alpha$ 的高阶无穷小且随 $\alpha$ 趋于0。如果对于一切充分小的 $\alpha$ ,成立 
\begin{equation}\label{PolReq_eq2}
\alpha d+\epsilon_\alpha\geq0\quad or\quad\alpha d+\epsilon_\alpha\leq0~,
\end{equation}
则 $d=0$。
\end{lemma}
\textbf{证明:} 设 $d\neq 0$。则因$\epsilon_\alpha$ 是 $\alpha$ 的高阶无穷小,当 $\alpha\rightarrow0$ 时,表式
\begin{equation}
d+\frac{\epsilon_\alpha}{\alpha}
\end{equation}
 将与 $d$ 的符号一致。所以,表式
 \begin{equation}\label{PolReq_eq1}
 \alpha d+\epsilon_\alpha=\alpha\qty(d+\frac{\epsilon_\alpha}{\alpha})
 \end{equation}
 的符号取决于 $\alpha$。即若 $\alpha$ 与 $d+\frac{\epsilon_\alpha}{\alpha}$ 符号相反时,\autoref{PolReq_eq1} 为负,相同时为正。这与条件\autoref{PolReq_eq2} 相违背。因此,$d=0$。引理得证。
 \subsubsection{\autoref{PolReq_the1} 的证明}
考虑函数 $y(x)+t\eta(x)$,则由\autoref{Varia_eq3}~\upref{Varia},\autoref{Varia_eq4}~\upref{Varia}
\begin{equation}\label{PolReq_eq3}
J(y+t\eta)-J(y)=t\delta J+\epsilon_1\abs{t}r(y,y+\eta)
\end{equation}
式中,右边第二项是 $t$ 的高阶无穷小,并且式中已经使用了 $r(y,y+t\eta)=\abs{t}r(y,y+\eta)$\autoref{AbPol_def1}~\upref{AbPol}。

由于 $y(x)$ 给 $J(y)$ 于极值,由极值的必要条件,对任意的 $t\rightarrow0$, \autoref{PolReq_eq3} 符号都一定。由泛函变分 $\delta J$ 的定义,当 $y$ 和 $\eta$ 固定时,其是一个常数。所以\autoref{PolReq_eq3} 满足\autoref{PolReq_lem1} 的条件,故 $\delta J=0$。定理得证。