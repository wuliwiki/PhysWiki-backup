% 结构张量代数
% 代数|结构张量|结构常数

\pentry{张量的坐标变换\upref{TrTnsr},域上的代数\upref{AlgFie}}
\footnote{捷玻大列夫。代数学引论,高等教育出版社.1954.}
\footnote{柯斯特利金,代数学引论,卷2,第3版。高等教育出版社.2008.} 在线性算子代数\upref{LiOper}一节提到,代数是一个同时是个环的矢量空间,或者由代数的定义直接获得。那么,要使一个矢量空间构成一个代数,就得赋予矢量空间环的特性,即任意二矢量可进行乘法运算且该乘法对加法满足分配律。由于矢量都可由一组基表示,那么任意二矢量可作乘法及对加法满足分配律的要求,就变成只需规定基矢量之间的乘法。由运算的封闭性,作乘法得到的矢量仍能用基表示,这样的基矢量之间的乘法得到的矢量在该组基下的坐标就称为\textbf{结构常数},这样只需要求乘法满足结合律,矢量空间便是一个环了,于是就将矢量空间构造成了一个代数。可以证明,结构常数是某一个 $(2,1)$ 型张量的坐标,这个张量就称为\textbf{结构张量}。 

一句话来说就是:结构张量使得一个矢量空间具有了代数结构。

\subsection{将矢量空间构造成环}
设 $V$ 是域 $\mathbb F$ 上的矢量空间,$\{e_i\}$ 是它的一个基。那么任意的元都可表示成 (使用爱因斯坦求和约定\upref{EinSum})
\begin{equation}
x^i e_i
\end{equation}
的形式。下面将 $V$ 构造为一个环。

为使任意二矢量能进行乘法运算,且乘法是封闭的和对加法满足分配律,那么只需规定
\begin{equation}\label{STAlg_eq1}
e_i*e_j=\gamma_{ij}^k e_k~,
\end{equation}
且乘法“*”满足
\begin{equation}
\begin{aligned}
&\lambda(e_i*e_j)=(\lambda e_i)*e_j=e_i*(\lambda e_j),\quad \lambda\in\mathbb F\\
&(e_i+e_j)*e_k=e_i*e_k+e_j*e_k\\
&e_i*(e_j+e_k)=e_i*e_j+e_i*e_k
\end{aligned}
\end{equation}

现在来寻找乘法满足结合律的要求,设
\begin{equation}
u=u^ie_i,\quad v=v^je_j,\quad w=w^ke_k
\end{equation}
要 $(uv)w=u(vw)$,就要
\begin{equation}
\begin{aligned}
&u^iv^jw^k (e_i*e_j)*e_k=(u^ie_i *v^je_j)*w^ke_k=u^ie_i *(v^je_j*w^ke_k)\\
&=u^iv^jw^k e_i*(e_j*e_k)
\end{aligned}
\end{equation}
由 $u,v,w$ 的任意性,只要
\begin{equation}
\begin{aligned}
(e_i*e_j)*e_k&=e_i*(e_j*e_k)\\
&\Downarrow\\
\gamma_{ij}^le_l*e_k&=\gamma_{jk}^le_i*e_l\\
&\Downarrow\\
\gamma_{ij}^l\gamma_{lk}^{m}e_m&=\gamma_{jk}^l\gamma_{il}^m e_m\\
&\Downarrow\\
\gamma_{ij}^l\gamma_{lk}^{m}&=\gamma_{jk}^l\gamma_{il}^m
\end{aligned}
\end{equation}
结合律就满足了。

这样,矢量空间在乘法“*” 之下就构成了一个代数。这就给出了下面的定义。
\begin{definition}{结构常数}
由\autoref{STAlg_eq1} 规定的 $\gamma_{ij}^k\in\mathbb F$ 称为代数 $V$ 在给定基之下的\textbf{结构常数},其满足
\begin{equation}
\gamma_{ij}^l\gamma_{lk}^{m}=\gamma_{jk}^l\gamma_{il}^m
\end{equation}
\end{definition}
\begin{theorem}{结构张量}
结构常数 $\gamma_{ij}^k$ 是某个 $(2,1)$ 型张量 $\Gamma$ 的坐标。该张量 $\Gamma$ 称为\textbf{结构张量}。
\end{theorem}
\textbf{证明:}要证结构常数是某个张量的坐标,就只要证它的坐标满足张量坐标的变换规则,因为如果一个数满足张量的坐标变换规则,那么配上对应基底后就是一个张量。

设
\begin{equation}
e_i'=a_i^s e_s,\quad e_j'=b_j^t e_t
\end{equation}
则 $B=(b_j^t)=A^{-1}$, $A=(a_i^s)$ 。于是
\begin{equation}\label{STAlg_eq2}
\begin{aligned}
{\gamma'}_{ij}^ke'_k&=e'_i*e'_j=\qty(a_i^s e_s)*\qty(a_j^t e_t)=a_i^s a_j^te_s*e_t\\
&=a_i^sa_j^t\gamma_{st}^r e_r=a_i^sa_j^t\gamma_{st}^r b_r^k e'_k\\
&\Downarrow\\
{\gamma'}_{ij}^k&=a_i^sa_j^t\gamma_{st}^r b_r^k
\end{aligned}
\end{equation}
由张量坐标变换规则(\autoref{TrTnsr_the1}~\upref{TrTnsr}),\autoref{STAlg_eq2} 表明结构常数和 $(2,1)$ 型张量的坐标变换规则一致。

\textbf{证毕!}

综上,可以说,在 $V$ 上给出了结构张量 $\Gamma$,就确定了一个代数。反过来,代数 $V$ 的结构张量 $\Gamma$ 是完全确定的。
\subsection{迹形式}
对于研究代数 $V$ 的结构,迹形式是一个重要的工具。定义映射
\begin{equation}
L_a:x\rightarrow a*x
\end{equation}
由乘法 “*” 的线性性,可知 $L_a$ 是个线性算子。
\begin{definition}{迹形式}
称
\begin{equation}
f_V(a,b):=\mathrm{tr}\, L_aL_b
\end{equation}
是代数 $V$ 上的\textbf{迹形式}。
\end{definition}
\begin{example}{}\label{STAlg_ex1}
设 $a=\alpha^i e_i,b=\beta^j e_j$,试证明:迹形式 $f_V(a,b)$ 可记成张量的完整卷积的形式:
\begin{equation}
f_V(a,b)=\alpha^i\beta^j\gamma_{jt}^s\gamma_{is}^t
\end{equation}
\textbf{证明:}
\begin{equation}
\begin{aligned}
L_aL_b e_k=a*(b*e_k)=\alpha^i\beta^je_i*(e_j*e_k)=\alpha^i\beta^j\gamma_{jk}^s \gamma_{is}^t e_t
\end{aligned}
\end{equation}
故 $L_aL_b$ 的迹就是(利用\autoref{CofTen_eq4}~\upref{CofTen}的记号:$(e^k,e_t)=e^k(e_t)$)
\begin{equation}
f_V(a,b)=\mathrm{tr}\, L_a L_b=(e^k,L_aL_b e_k)=\alpha^i\beta^j\gamma_{jt}^s \gamma_{is}^t
\end{equation}\\
\textbf{证毕!}
\end{example}

\begin{theorem}{李代数中迹形式的结合性}\label{STAlg_the1}
设 $V$ 是一个有限维李代数(\autoref{LieAlg_def2}~\upref{LieAlg}),则其上的迹形式满足结合性:
\begin{equation}
f_V([a,b],c)=f_V(a,[b,c])
\end{equation}
\end{theorem}
\textbf{证明:}在李代数上,$[a,b]:=a*b$ ,利用李代数上的雅可比恒等式
\begin{equation}
\begin{aligned}
&[[x,y],z]+[[z,x],y]+[[y,z],x]=0\\
&\qquad\qquad\Downarrow\\
&[[x,y],z]=-[[z,x],y]-[[y,z],x]\\
&=[x,[y,z]]+[y,[z,x]]\\
&=[x,[y,z]]-[y,[x,z]]\\
\end{aligned}
\end{equation}
上式即
\begin{equation}
L_{[x,y]}=L_xL_y-L_yL_x
\end{equation}
注意迹的交换性 $\mathrm{tr}\, \mathcal A\mathcal B=\mathrm{tr}\, \mathcal B\mathcal A$(\autoref{trace_the1}~\upref{trace}),就有
\begin{equation}
\begin{aligned}
f_V([a,b],c)&=\mathrm{tr}\,L_{[a,b]}L_c=\mathrm{tr}\,(L_aL_bL_c-L_bL_aL_c)\\
&=\mathrm{tr}\,L_aL_bL_c-\mathrm{tr}L_bL_aL_c\\
&=\mathrm{tr}\,L_aL_bL_c-\mathrm{tr}L_aL_cL_b\\
&=\mathrm{tr}\,L_a(L_bL_c-L_cL_b)\\
&=\mathrm{tr}\,L_aL_{[b,c]}=f_V([a,[b,c]])
\end{aligned}
\end{equation}
\textbf{证毕!}

\begin{example}{3维李代数(矢量叉积)}
3维矢量的叉积相当于在3维空间定义了一个李代数,其结构常数定义如下
\begin{equation}
[e_1,e_2]=e_3,\quad [e_3,e_1]=e_2,\quad [e_2,e_3]=e_1
\end{equation}
这相当于只有3指标 $i,j,k$ 两两不同时才有 $\gamma_{ij}^k\neq0$,而且不等于零的结构常数为:
\begin{equation}
\begin{aligned}
\gamma_{12}^3=\gamma_{23}^1=\gamma_{31}^2=1,\\
\gamma_{21}^3=\gamma_{32}^1=\gamma_{13}^2=-1
\end{aligned}
\end{equation}
此时
\begin{equation}
\gamma_{jt}^s\gamma_{is}^t=-2\delta_{ij}
\end{equation}
那么,由\autoref{STAlg_ex1} ,迹形式为
\begin{equation}
f_V(a,b)=-2\alpha^i\beta^j\delta_{ij}=-2(a|b)
\end{equation}
由\autoref{STAlg_the1} 
\begin{equation}
([a,b]|c)=(a|[b,c])
\end{equation}
在中学里,我们熟悉的 $[a,b]$ 是用叉积 $a\times b$ 表示的,而 $(a|b)$ 用 $a\cdots b$ 表示,那么上式可写成
\begin{equation}
(a\times b)\cdot c=a\cdot(b\times c)
\end{equation}

\end{example}