% Egoroff 定理
% 叶戈洛夫定理|实变函数|一致收敛

\pentry{可测函数\upref{MsbFun}}

在讨论函数极限的相关问题时,我们常需要函数列具有\textbf{一致收敛}的性质。反过来,观察一个不一致收敛的函数列,比如 $\{f_n(x)=x^n\}$ 在区间 $[0, 1]$ 上就不一致收敛,我们会发现如果把区间挖掉长度 $\epsilon$\textbf{任意}小的一部分,那么 $\{f_n\}$ 在 $[0, 1-\epsilon]$ 上总是一致收敛的。这提示我们研究,任意收敛函数列是否可以去掉一个小部分以后是一致收敛的?

答案是肯定的,这就是本节要讨论的Egoroff定理。

\subsection{Egoroff定理}

为方便讨论,我们需要先证明一个引理,证明思路依赖于对集合极限的讨论。

%%注释说明:此处注释掉的内容是参照江泽坚《实变函数论》的证法提供的引理,但是本文使用了其它思路,所以把这个引理给换掉了。

% \begin{lemma}{}
% 任取可测集 $E$ 上的一列实函数 $\{f_n\}$,再任取一个 $E$ 上的实函数 $f$。

% 则 $\lim\limits_{n\to\infty}f_n(x_0)\neq f(x_0)$,当且仅当对于\textbf{任意}一列\textbf{单调递减}到 $0$ 的\textbf{正}数列 $\{\epsilon_k\}$,必有:

% \begin{equation}\label{EgrfTh_eq1}
% x_0\in\bigcup^\infty_{k=1}\bigcap^\infty_{N=1}\bigcup^\infty_{n=N} \{x\in E|\abs{f_n(x)-f(x)}>\epsilon_k\}
% \end{equation}

% \end{lemma}

% \textbf{证明}:

% 该引理简单来说,就是“\autoref{EgrfTh_eq1} 为全体使 $f_n$ 不收敛于 $f$ 的点构成的集合”。为了证明引理,我们首先要翻译一下什么叫“$f_n(x)$ 不收敛于 $f(x)$”。

% “$f_n(x_0)$\textbf{收敛}于 $f(x_0)$”等价于“对于任意 $\epsilon>0$,存在 $N_\epsilon\in\mathbb{Z}^+$,使得只要 $n>N_\epsilon$,就有 $\abs{f_n(x_0)-f(x_0)}<\epsilon$”。将该命题进行否定\footnote{回忆否定的过程,即把命题描述中的全称量词和存在量词互换,然后把最后描述的条件取反。},即得到:

% “$f_n(x_0)$\textbf{不收敛}于 $f(x_0)$”等价于“存在 $\epsilon>0$,使得对于任意 $N\in\mathbb{Z}^+$,都存在 $n>N$,使得 $\abs{f_n(x_0)-f(x_0)}\geq\epsilon$”\footnote{到这里,有的读者可能已经看出来怎么证明了:“存在 $\epsilon>0$”对应的就是\autoref{EgrfTh_eq1} 中的 $\bigcup^\infty_{k=1}$。}。

% 接下来证明\autoref{EgrfTh_eq1} 和“存在 $\epsilon>0$,使得对于任意 $N\in\mathbb{Z}^+$,都存在 $n>N$,使得 $\abs{f_n(x)-f(x)}\geq\epsilon$”是等价命题。

% 为了方便,记 $E_{n, k}=\{x\in E|\abs{f_n(x)-f(x)}>\epsilon_k\}$。那么 $\bigcap^\infty_{N=1}\bigcup^\infty_{n=N} E_{n, k}$ 就是集合的\textbf{上极限}(\autoref{SetLim_def1}~\upref{SetLim},原因见\textbf{集合的极限}\upref{SetLim}词条内容)。根据上极限的意义可知,$\bigcap^\infty_{N=1}\bigcup^\infty_{n=N} E_{n, k}$ 是全体“对于 $\epsilon_k$ 和任意的 $N\in\mathbb{Z}^+$,总存在 $n>N$ 使得 $\abs{f_n(x)-f(x)}\geq\epsilon$”的 $x$ 构成的集合。

% 再套上一个 $\bigcup^\infty_{k=1}$ 来取遍所有 $\epsilon_k$,得到的\autoref{EgrfTh_eq1} 就等价于“$f_n(x_0)$\textbf{不收敛}于 $f(x_0)$”。

% \textbf{证毕}。

\begin{lemma}{}\label{EgrfTh_lem1}
设 $E$ 是 $\mathbb{R}^n$ 上\textbf{有界}的可测集,$\{E_k\}$ 是一列集合,其中各 $E_k\subseteq E$。

如果 $\{E_k\}$ 的上极限(\autoref{SetLim_def1}~\upref{SetLim})为 $\varnothing$\footnote{注意下极限必是上极限的子集,因此这意味着 $\{E_k\}$ 的上极限等于其下极限,也就是其极限。},那么 $\lim\limits_{k\to\infty}\opn{m}E_k=0$。
\end{lemma}

\textbf{证明}:

令 $F_k=\bigcup_{i=k}^\infty E_i$。则 $\{F_k\}$ 是一个单调不增的可测集列,且其极限就是 $\{E_k\}$ 的上极限。

由于能容纳 $F_k$ 的开集必能容纳 $F_{k+1}$,知 $\{\opn{m}(F_k)\}$ 是一个单调不增的非负数列,且恒有 $\opn{m}(F_k)\geq\opn{m}(E_k)$。又因为 $E$ 有界,知 $\opn{m}F_1<+\infty$。

\textbf{反设}存在一个正数 $\epsilon$,使得存在一个 $F_N$ 无法被体积\textbf{小于}$\epsilon$ 的开集覆盖,那么对于所有 $k>N$,$F_k$ 都不能被体积小于 $\epsilon$ 的开集覆盖。

任取一个开集 $U$ 使得 $\opn{m}U<\epsilon/1024$\footnote{小于 $\epsilon$ 就可以,我这是在整活。}。则按照反设的情况,$U$ 必然不能覆盖任何一个 $F_k$。考虑到 $F_k$ 的单调性和\textbf{一致}有界性,这意味着至少存在一个点 $x$,它是所有 $F_k$ 的公共元素但并不在 $U$ 中。这么一来 $F_k$ 的极限至少含有 $x$,和题设“$E_k$ 的极限是空集”矛盾。

因此反设不成立,即任取 $\epsilon>0$,总存在正整数 $N_\epsilon$ 使得对于任意 $k>N_\epsilon$,$\opn{m}F_k<\epsilon$。

因此 $\lim\limits_{k\to\infty}\opn{m}F_k=0$。

因此 $\lim\limits_{k\to\infty}\opn{m}E_k\leq\lim\limits_{k\to\infty}\opn{m}F_k=0$。


\textbf{证毕}。

\begin{corollary}{}\label{EgrfTh_cor1}
设 $E$ 是 $\mathbb{R}^n$ 上\textbf{有界}的可测集,$\{E_k\}$ 是一列集合,其中各 $E_k\subseteq E$。

如果 $\{E_k\}$ 的上极限是零测集,那么同样有 $\lim\limits_{k\to\infty}\opn{m}E_k=0$。
\end{corollary}

\autoref{EgrfTh_cor1} 和\autoref{EgrfTh_lem1} 的区别在于,无法说明 $\{E_k\}$ 的上下极限相等了。但是由于零测集在测度意义上相当于不存在,故推论依然成立。

证明很简单,只需要把 $\{E_k\}$ 的上极限从 $E$ 中减掉,应用\autoref{EgrfTh_lem1} 即可。





接下来,就可以讨论Egoroff定理了。


\begin{theorem}{Egoroff定理}\label{EgrfTh_the1}

设 $E$ 是 $\mathbb{R}^n$ 上\textbf{测度有限}的\textbf{可测集合}。如果 $\{f_n\}$ 是 $E$ 上的一列\textbf{可测函数},都几乎处处取有限函数值,而存在 $E$ 上的函数 $f(x)$ 使得\footnote{即 $f_n$ 几乎处处收敛于 $f$,或者说不收敛点构成一个零测集。}\begin{equation}
\lim\limits_{n\to\infty}f_n(x)=f(x)a. e. 
\end{equation}

那么对于任意正数 $\delta$,必存在一个\textbf{可测集}$E_\delta$,使得 $\opn{m}E_\delta<\delta$,且 $f_n$ 在 $E-E_\delta$ 上一致收敛于 $f$。

\end{theorem}

\textbf{证明}:

考虑 $E$ 是有界可测集且 $\{f_n\}$ 在 $E$ 上\textbf{处处}收敛于 $0$ 的情况。这个情况最简单,并且证明了这个情况也就相当于证明了 $\{f_n\}$\textbf{几乎处处}收敛于\textbf{任意}$f$ 的情况,从而也相当于证明了 $E$ 是任意测度有限的可测集的情况。



任取一个单调递减的\textbf{正数}列 $\{\epsilon_k\}$,且 $\lim\limits_{k\to\infty}\epsilon_k=0$。

固定 $\epsilon_k$,记 $\{x\in E|\abs{f_n(x)}\geq\epsilon_k\}=D_{k, n}$\footnote{$D$ 是取“不听话(disobey)”的首字母。}。由于 $\{f_n\}$ 处处趋近于 $0$,故 $D_{k, n}$ 关于 $n$ 的上极限是 $\varnothing$。由\autoref{EgrfTh_lem1} ,$\lim\limits_{n\to\infty}\opn{m}D_{k, n}=0$。因此,我们可以取一个正整数 $N_k$,使得对于任意 $i\geq N_k$,都有
\begin{equation}\label{EgrfTh_eq2}
\opn{m}D_{k, i}<\delta/2^k
\end{equation}

把全体 $D_{k, N_k}$ 取并,得到 $D=\bigcup_{k=1}^\infty D_{k, N_k}$。由\autoref{EgrfTh_eq2} 直接可得
\begin{equation}
\opn{m}D<\delta
\end{equation}

按照我们的构造规则,在 $E-D$ 上,对于任意的 $\epsilon_k$,都存在正整数 $N_k$,使得对于 $i\geq N_k$ 和 $x\in E-D$ 恒有 $\abs{f_n(x)}<\epsilon_k$。即 $\{f_n\}$ 在 $E-D$ 上一致收敛到 $0$。

\textbf{证毕}。


\subsection{例子与反例}

在本节的开头,我们已经给出了一个适用于Egoroff定理的例子了。

下面给出一个反例,说明\autoref{EgrfTh_the1} 中“$E$ 的测度有限”是不可或缺的条件。这个例子是通过把本节开头的例子“拉伸”来得到的。


\begin{example}{}

考虑 $\mathbb{R}$ 上的函数列 $f_n(x)=\qty(\frac{2\arctan(x)}{\pi})^n$。显然它在 $\mathbb{R}$ 上并不一致收敛,而且必须至少挖去包含 $+\infty$ 的一个区间才能使剩下部分一致收敛,这个区间的测度自然是无穷大的。

\end{example}















