% 三角函数(复数)
% 复数|三角函数|三角恒等式|欧拉公式

\pentry{指数函数(复数)\upref{CExp}}
\subsection{定义}
定义复数域的正弦函数为
\begin{equation}\label{CTrig_eq1}
\sin z = \frac{\E^{\I z} - \E^{-\I z}}{2\I}~.
\end{equation}
定义复数域的余弦函数为
\begin{equation}\label{CTrig_eq2}
\cos z = \frac{\E^{\I z} + \E^{-\I z}}{2}~.
\end{equation}
为什么三角函数要这么定义?因为只有这么定义,才能既“兼容”实数范围内的三角函数,同时满足解析的要求, 即以后会学习的柯西—黎曼条件\upref{CauRie}。

\subsection{与实数函数的“兼容性”}
“兼容性”在这里指若将一个复变函数的自变量取实数, 那么结果与使用同名的实数函数相同。 例如将\autoref{CTrig_eq1} 中的复数 $z$ 取实数 $x$, 得
\begin{equation}
\sin x = \frac{\E^{\I x} - \E^{-\I x}}{2\I}~.
\end{equation} 
根据复数域指数函数的定义\upref{CExp}, 在这里具体指欧拉公式, 得
\begin{equation}
\E^{\I x} = \cos x + \I\sin x
\end{equation} 
\begin{equation}
\E^{-\I x} = \cos x - \I \sin x
\end{equation} 
代入得
\begin{equation}
\frac{\E^{\I x} - \E^{-\I x}}{2\I} = \frac{(\cos x + \I\sin x) - (\cos x - \I\sin x)}{2\I} = \sin x
\end{equation}  
同理可证\autoref{CTrig_eq2} 在实轴上成立。 证毕。

\subsection{两角和公式}
利用欧拉公式,容易证明,复数范围内的正余弦函数同样满足两角和公式
\begin{equation}\label{CTrig_eq3}
\sin(z_1 + z_2) = \sin {z_1}\cos {z_2} + \cos {z_1}\sin {z_2}
\end{equation}
\begin{equation}\label{CTrig_eq4}
\cos (z_1 + z_2) = \cos {z_1}\cos {z_2} - \sin {z_1}\sin {z_2}
\end{equation}
\subsection{实部和虚部}
利用两角和公式,令 $z_1$ 等于实数 $x$,  $z_2$ 等于虚数 $\I y$, 则有
 \begin{equation}
\sin z = \sin(x + \I y) = \sin x\cos \I y + \cos x\sin \I y
\end{equation} 
\begin{equation}
\cos z = \cos(x + \I y) = \cos x\cos \I y - \sin x\sin \I y
\end{equation} 
其中
\begin{equation}\label{CTrig_eq5}
\cos(\I y) = \frac{\E^{-y} + \E^y}{2} = \cosh y
\end{equation} 
\begin{equation}\label{CTrig_eq6}
\sin(\I y) = \frac{\E^{-y} - \E^y}{2\I} = \I\frac{\E^y - \E^{-y}}{2} = \I\sinh y
\end{equation} 
代入得
\begin{equation}
\sin z = \sin(x + \I y) = \sin x\cosh y + \I\cos x\sinh y
\end{equation} 
\begin{equation}
\cos z = \cos(x + \I y) = \cos x\cosh y - \I\sin x\sinh y
\end{equation}  
这样,就把正余弦的实部和虚部分开来了(当然也可以根据定义直接得到两式)。

%\subsection{解析性}
%由于 ${\E^z}$ 是解析函数,而解析函数的线性组合也是解析函数,所以正余弦函数都是解析函数%未完成:词条
%。但也可以根据柯西—黎曼公式%未完成:词条
%直接证明。






















