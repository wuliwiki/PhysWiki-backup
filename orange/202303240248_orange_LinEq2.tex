% 线性映射的结构 2
% 线性映射|齐次方程|解集|零空间|向量空间|补空间

\pentry{线性映射的结构\upref{MatLS2}, 线性方程组\upref{LinEqu}}

下面我们从线性映射和向量空间的角度理解线性方程组 $\bvec A \bvec x = \bvec b$。


\begin{definition}{线性方程}
对给定的线性映射 $A:X\to Y$ 和 $b \in Y$, \textbf{线性方程}为
\begin{equation}\label{LinEq2_eq2}
Ax = b~,
\end{equation}
所有满足该式的 $x \in X$ 的集合 $X_s$ 叫做方程的\textbf{解集}。
\end{definition}

首先注意 $A$ 未必把 $Y$ 中的每个元素都射中, 即值空间\upref{LinMap} $Y_1 = A(X) \subseteq Y$ 只是 $Y$ 的一个子空间。 所以只有 $b \in Y_1$ 时\autoref{LinEq2_eq2} 有解, 否则无解(解集为空集)。 用映射的语言, 解集 $X_s$ 就是集合 $\qty{b}$ 的逆像\upref{map} $A^{-1}(\qty{b})$。

当\autoref{LinEq2_eq2} 中 $b = 0$ 时方程叫做算符 $A$ 的\textbf{齐次方程}。 根据定义, 齐次方程的解就是映射的零空间(\autoref{LinMap_the1}~\upref{LinMap})。

\begin{theorem}{}\label{LinEq2_the1}
线性方程\autoref{LinEq2_eq2} 的解集可以表示为
\begin{equation}\label{LinEq2_eq1}
X_s = X_0 + x_1~.
\end{equation}
其中 $x_1$ 为 $X_s$ 中的任意元素,  $X_0$ 为映射的零空间。
\end{theorem}
说明: $X_0 + x_1$ 表示把 $X_0$ 中的每一个向量与 $x_1$ 相加得到的集合。 易证当 $x_1 \ne 0$ 时解集 $X_s$ 不是一个向量空间(例如不存在零向量)。

首先证明集合 $X_0 + x_1$ 中的元素满足 $Ax = b$。 令任意 $x_0 \in X_0$
\begin{equation}
A(x_0 + x_1) = Ax_0 + Ax_1 = 0 + b = b~,
\end{equation}
证毕。 再来证明解集中不存在 $X_0 + x_1$ 之外的向量。 令 $x_2 \in X_S$ 且 $x_2 \ne x_1$, 那么
\begin{equation}
A(x_2 - x_1) = Ax_2 - Ax_1 = b - b = 0~,
\end{equation}
即 $x_2 - x_1 \in X_0$, 即 $x_2 \in x_1 + X_0$。 证毕。

\subsection{线性映射的结构}
到此为止我们就可以非常清晰地勾画出多对一线性映射的结构了 $A:X\to Y$。 我们先找到\autoref{MatLS2_the1}~\upref{MatLS2} 中的零空间 $X_0$ 和它的补空间 $X_1$, 其中 $X_1$ 的元素和值域空间 $Y_1 = A(X)$ 一一对应。 那么对每个 $x_1 \in X_1$, 线性映射就会把集合 $x_1 + X_0$ 所有元素映射到 $Y_1$ 中的同一个元素 $y_1 = Ax_1$ 上。

\addTODO{需要一个 3 维几何向量空间的真实例子, 零空间 1 维, 补空间 2 维}
