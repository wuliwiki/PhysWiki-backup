% 二体系统
% 质心|质点|二体系统|自由度|动能

\pentry{自由度\upref{DoF}, 动量定理\upref{PLaw}}

我们现在考虑两个仅受相互作用的质点 $A$ 和 $B$, 它们的质量分别为 $m_A$ 和 $m_B$。 由于不受系统外力, 在任何惯性系中它们的质心\upref{CM}都会做匀速直线运动(\autoref{eq_PLaw_2}~\upref{PLaw})。 所以系统的质心参考系是一个惯性系, 以下我们选取质心参考系, 系统的质心将一直处于原点。

现在定义它们的\textbf{相对位矢}(也叫\textbf{相对坐标})为点 $A$ 指向点 $B$ 的矢量
\begin{equation}\label{eq_TwoBD_1}
\bvec R = \bvec r_B - \bvec r_A~,
\end{equation}
且定义\textbf{相对速度}和\textbf{相对加速度}分别为 $\bvec R$ 的导数 $\dot{\bvec R}$ 和二阶导数 $\ddot{\bvec R}$。
在质心系中观察, 由于质心始终处于原点, 两质点的位矢 $\bvec r_A$ 和 $\bvec r_B$ 满足
\begin{equation}\label{eq_TwoBD_2}
m_A \bvec r_A + m_B \bvec r_B = \bvec 0~.
\end{equation}
联立\autoref{eq_TwoBD_1} 和\autoref{eq_TwoBD_2} 可以发现在质心系中 $\bvec R, \bvec r_A, \bvec r_B$ 间始终存在一一对应的关系(共线且模长呈固定比例), 所以质心系中不受外力的二体系统只有三个自由度\upref{DoF}
\begin{equation}\label{eq_TwoBD_3}
\bvec r_A = \frac{-m_B}{m_A + m_B} \bvec R ~,\qquad
\bvec r_B = \frac{m_A}{m_A + m_B} \bvec R~.
\end{equation}

\subsection{运动方程}

现在令质点 $A$ 对 $B$ 的作用力为 $\bvec F$ (与 $\bvec R$ 同向), 则由牛顿第三定律, $B$ 对 $A$ 有反作用力 $- \bvec F$。 两质点加速度分别为(牛顿第二定律) $\bvec a_A =  -\bvec F/m_A$, $\bvec a_B =  \bvec F/m_B$。 所以相对加速度为
\begin{equation}
\ddot{\bvec R} = \ddot{\bvec r}_B - \ddot{\bvec r}_A = \frac{m_A+m_B}{m_Am_B} \bvec F~.
\end{equation}
若定义两质点的\textbf{约化质量}为
\begin{equation}
\mu = \frac{m_A m_B}{m_A + m_B}~,
\end{equation}
且将上式两边同乘约化质量, 我们得到相对位矢的牛顿第二定律
\begin{equation}\label{eq_TwoBD_6}
\bvec F = \mu\ddot{\bvec R}~.
\end{equation}
也就是说, 在质心系中使用相对位矢, 二体系统的运动规律就相当于单个质量为 $\mu$, 位矢为 $\bvec R$ 的质点的运动规律, 我们姑且将其称为\textbf{等效质点}。 而 $A$ 对 $B$ 的作用力可以看成等效质点的受力。

\begin{example}{两天体圆周运动}
令质量分别为 $m_1$ 和 $m_2$ 的天体距离为 $R$,在万有引力作用下绕质心做圆周运动,求角速度 $\omega$。

解:两天体之间的引力大小为(\autoref{eq_Gravty_1}~\upref{Gravty})
\begin{equation}
F = \frac{Gm_1m_2}{R^2}~.
\end{equation}
如果不使用等效质点的概念, 我们可以先用\autoref{eq_TwoBD_3} 得到两个天体做圆周运动的半径, 然后再令引力等于其中一个天体的向心力\upref{CentrF}
\begin{equation}
F = m_1 \omega^2 r_1~.
\end{equation}
也可以不求 $r_1, r_2$, 直接使用等效天体的圆周运动向心力
\begin{equation}
F = \mu \omega^2 R~,
\end{equation}
易得以上两式是等效的, 解得
\begin{equation}
\omega = \sqrt{\frac{G(m_1 + m_2)}{R^3}}~.
\end{equation}
\end{example}

\subsection{机械能}

再来看系统的动能。 令 $\bvec v = \dot {\bvec R}$, 使用\autoref{eq_TwoBD_3} 把系统在质心系中的总动能用相对位矢表示得
\begin{equation}
E_k = \frac12 (m_A \dot{\bvec r}_A^2 + m_B \dot{\bvec r}_B^2) = \frac12 \frac{m_A m_B}{m_A + m_B} {\bvec  v}^2 = \frac12 \mu {\bvec  v}^2~,
\end{equation}
这恰好是等效质点动能。

若两质点间的相互作用力的大小只是二者距离 $R = \abs{\bvec R}$ 的函数, 我们可以用一个标量函数 $F(R)$ 来表示力与距离的关系, 即
\begin{equation}
\bvec F(\bvec R) = F(R) \uvec R~.
\end{equation}
注意 $F(R)>0$ 时两质点存在斥力, $F(R)<0$ 时存在引力。

根据“势能\upref{V}” 中的\autoref{eq_V_20}, 我们可以定义势能函数 $V(R)$ 为 $F(R)$ 的一个负原函数。 现在写出二体系统在质心系中的机械能为
\begin{equation}
E = \frac12 \mu \dot{\bvec  R}^2 + V(R)~.
\end{equation}
由于系统不受外力, 机械能守恒。

\subsection{动量守恒}
在质心系中, 两质点的总动量恒为零, 动量守恒。 令它们的动量分别为 $\bvec p_A$ 和 $\bvec p_B$, 有
\begin{equation}\label{eq_TwoBD_10}
\bvec p_B = m_B \dot {\bvec r}_B = \mu \bvec v
\end{equation}
令等效质点的动量为 $\bvec p = \mu \bvec v$, 则
\begin{equation}\label{eq_TwoBD_11}
\bvec p = \bvec p_B = -\bvec p_A
\end{equation}
即等效质点的动量等于质点 $B$ 的动量或质点 $A$ 动量的逆矢量。

\subsection{角动量守恒}

由\autoref{eq_TwoBD_11} 可得二体系统总角动量为
\begin{equation}
\bvec L = \bvec r_A \cross \bvec p_A + \bvec r_B \cross \bvec p_B
= \bvec R \cross \bvec p~.
\end{equation}
即二体系统的总角动量等于等效质点的角动量。

如果两个质点之间的相互作用力沿它们连线的方向, 那么所有两个力都经过质心, 即都不受任何力矩, 所以系统角动量守恒。
