% 椭圆(高中)
% keys 极坐标系|直角坐标系|圆锥曲线|椭圆
% license Xiao
% type Tutor

\begin{issues}
\issueDraft
\end{issues}

\pentry{解析几何\nref{nod_JXJH},圆\nref{nod_HsCirc}}{nod_32e0}

% 933:这篇文章的介绍思路是主要考虑高中教材。从圆和解析几何的视角来引入椭圆本身。忽略圆锥曲线这件事。
人们常说地球绕着太阳转。提到“绕着转”,很多人会自然联想到一个完美的圆形轨道。但实际上,真正沿正圆轨道运行的行星几乎不存在。大多数行星的轨道虽然是封闭的环绕,但略微“压扁”,并不是严格的圆形。类似这种“被压扁的圆”在生活中也很常见,比如一个圆形水杯的杯口,在斜着观察时,所看到的轮廓就是这种形状。

有一种被称为“回声墙”的特殊建筑设计:只要两个人分别站在弧形墙上的两个特定位置,即使中间隔着整面墙,也能清晰地听到彼此的低语。研究发现,这种现象的产生依赖于墙体的形状满足一种特殊的曲线条件——从一个点发出的声音或光线,经过曲线表面的反射后,总能精准地传到另一个固定点。

令人惊讶的是,这两种现象中出现的曲线,实际上是同一种,它的名字叫\textbf{椭圆(ellipse)}。接下来,将从这两个视角分别出发,一步步探索这种曲线的几何特性。在研究各自的方程之后,进一步验证它们对应的是同一个图形,并由此引出椭圆的定义、标准方程和基本参数,逐步来认识这个优雅而又实用的几何对象。

\subsection{“压扁的圆”}

先来研究水杯杯口那种略微压扁的圆形轮廓,只从几何变化的角度入手,参考 \enref{函数的变换}{FunTra} 中介绍的变换方法,从单位圆出发,研究其经过轴向放缩后所形成的曲线。具体地,考虑将单位圆在 $x$ 轴和 $y$ 轴方向分别拉伸 $p$ 倍和 $q$ 倍(其中 $p, q$ 为任意正数)。单位圆的方程为:
\begin{equation}
x^2 + y^2 = 1~.
\end{equation}
其与 $x$ 轴的交点为 $(1, 0), (-1, 0)$,与 $y$ 轴的交点为 $(0, 1), (0, -1)$,分别确定了单位圆的两个直径,当然也是单位圆的两个对称轴。

根据图像变换的规律,若在 $x$ 或 $y$ 方向进行 $k$ 倍拉伸,对应变量将变为原来的 $\displaystyle\frac{1}{k}$ 。因此,拉伸后的曲线方程为:
\begin{equation}\label{eq_Elips3_2}
\left(\frac{x}{p}\right)^2 + \left(\frac{y}{q}\right)^2 = 1\implies\frac{x^2}{p^2} + \frac{y^2}{q^2} = 1~.
\end{equation}

根据图像变换的结果,曲线在 $x$ 轴上的交点为 $(p, 0), (-p, 0)$,在 $y$ 轴上的交点为 $(0, q), (0, -q)$,它们限制了图像在两个方向上的最大范围。这四个交点确定的两条线段,尽管不再是直径,但仍是图像的唯二的对称轴,两条线段长度分别为 $2p$ 和 $2q$。

参考圆中“直径”和“半径”的概念,通常,将 $p, q$ 中较大的一个对应的对称轴或其长度称为\textbf{长轴(major axis)},记作$2a$,其中 $a$ 称为\textbf{半长轴(semi-major axis)};较小的一个对应的对称轴称或其长度为\textbf{短轴(minor axis)},记作$2b$,其中 $b$ 称为\textbf{半短轴(semi-minor axis)}。二者满足的关系是$a>b$,即:
\begin{itemize}
\item $p>q$时,长轴在$x$轴上,短轴在$y$轴上,曲线方程为:
\begin{equation}\label{eq_Elips3_5}
\frac{x^2}{a^2} + \frac{y^2}{b^2} = 1,\qquad(a>b)~.
\end{equation}
\item $p<q$时,长轴在$y$轴上,短轴在$x$轴上,曲线方程为:
\begin{equation}\label{eq_Elips3_7}
\frac{x^2}{b^2} + \frac{y^2}{a^2} = 1,\qquad(a>b)~.
\end{equation}
\end{itemize}

\subsection{椭圆的几何定义}

从圆的定义开始,如果想要引申圆的定义,可以这样看,假设圆心是两个重合的点,然后圆上每个点到这两个点的距离相等,都是定值,现在如果把这两个点移开,那么显然如果在平面上所有到这两个点距离相等的点就构成了他们连线的垂直平分线。那么,如果加上一些不同的设定,或许会得到不同的效果。比如,圆上点到的这两个点到距离是定值,这启发我们或许可以是点到这两个点的距离之和是定值。

下面给出古希腊时代从几何视角给出的椭圆定义,这也常被称作是椭圆的第一定义。

\begin{definition}{椭圆的几何定义}
平面上到两定点的距离之和为有限定值的几何图形,称为\textbf{椭圆}。两个定点称作椭圆的两个\textbf{焦点}。
\end{definition}

可以这样看,当一个椭圆的两个焦点重合时,椭圆就变成了圆。所以可以这样说,圆是椭圆的一个特例。


\begin{example}{对两定点 $F_1(-c, 0)$ 和 $F_2(c, 0),(c>0)$,若点$P$满足$|PF_1| + |PF_2| = M,(M > 2c)$,求$P$方程。}\label{ex_Elips3_1}
解:

设椭圆上的任意点为 $P(x, y)$,根据题意有:
\begin{equation}
\sqrt{(x + c)^2 + y^2} + \sqrt{(x - c)^2 + y^2} = M~.
\end{equation}
移项后,两边平方有:
\begin{equation}
(x + c)^2 + y^2 = M^2 - 2M\sqrt{(x - c)^2 + y^2} + (x - c)^2 + y^2~.
\end{equation}
打开整理有:
\begin{equation}
2M\sqrt{(x - c)^2 + y^2}= M^2 - 4cx~.
\end{equation}
两边平方,打开有:
\begin{equation}
4M^2(x^2 - 2cx+c^2) + 4M^2y^2= M^4-4M^2\cdot2cx+16c^2x^2~.
\end{equation}
整理后得到:
\begin{equation}
4(M^2 -4c^2)x^2 + 4M^2y^2= M^2(M^2-4c^2)~.
\end{equation}
两侧同时除以$(M^2-4c^2)M^2$后得到:
\begin{equation}\label{eq_Elips3_4}
\frac{x^2}{\left(\displaystyle\frac{M}{2}\right)^2} + \frac{y^2}{\displaystyle\left(\frac{M}{2}\right)^2-c^2}=1~.
\end{equation}
\end{example}

由于$M>2c>0$,也就是$\displaystyle\left(\frac{M}{2}\right)^2-c^2>0$,可以看出,\autoref{ex_Elips3_1} 的结果的形式与\autoref{eq_Elips3_2} 相同,也就是说二者的结果对应着同一个曲线。同理易知,如果两个定点分别为 $F_1(0,-c)$ 和 $F_2(0,c),(c>0)$,则相当于更换$x,y$的结果位置,得到的表达式是:
\begin{equation}\label{eq_Elips3_6}
\frac{x^2}{\displaystyle\left(\frac{M}{2}\right)^2-c^2}+\frac{y^2}{\left(\displaystyle\frac{M}{2}\right)^2} =1~.
\end{equation}

令\autoref{eq_Elips3_4} 和\autoref{eq_Elips3_5} 、\autoref{eq_Elips3_6} 和\autoref{eq_Elips3_7} 对应相等,可以发现,总有:
\begin{equation}
\displaystyle\left(\frac{M}{2}\right)^2=a^2,~.
\end{equation}

椭圆是到两个定点(焦点)距离之和为定值的点的集合。这个定值等于椭圆的长轴长(记作 $2a$),即对任意点 $P$,有
\begin{equation}\label{eq_Elips3_9}
PF_1 + PF_2 = 2a ~.
\end{equation}

\begin{definition}{椭圆的标准方程}
\begin{equation}\label{eq_Elips3_3}
\frac{x^2}{a^2} + \frac{y^2}{b^2} = 1~.
\end{equation}
\end{definition}

从椭圆的极坐标公式难以看出椭圆的对称性, 另一种定义椭圆的方法是直接在直角坐标系中给出椭圆的方程

\subsubsection{参数介绍}


焦点之间的距离为 $2c$,其中半焦距 c 也叫做椭圆的线性离心率。
 c^2 = a^2 - b^2~. 
所以 $c < a$,表示焦点在中心两侧。

	•	若 $a > b$,焦点在 $x$ 轴上;
	•	若 $b > a$,焦点在 $y$ 轴上;
	•	若 $a = b$,就是圆。


长轴

短轴
焦距

\subsection{椭圆的参数方程}
表示为参数方程
\begin{equation}\label{eq_Elips3_1}
\leftgroup{
&x(t) = a\cos t\\
&y(t) = b\sin t
},\quad t \in [0, 2\pi) ~.
\end{equation}
\subsection{椭圆的性质}

\subsubsection{椭圆的面积}

参考圆的面积为$S = \pi r^2$,以及半径对应半长轴和半短轴。从椭圆是圆的伸缩的视角,一个大胆的猜想是,椭圆的面积满足:

\begin{equation}
S = \pi a b~.
\end{equation}

好消息是这个猜想是成立的,但是其证明需要使用\enref{定积分}{DefInt},此处不与赘述。


椭圆上点与焦点连线的斜率之积为定值。

关于 $x$ 轴、$y$ 轴和原点对称。是中心对称图形,中心即椭圆中心。
任何穿过两个焦点的直线段,其在椭圆上的两个端点之间的距离等于 $2a$,且椭圆的长轴为所有通过的弦中最长的。
焦点反射性质(反射定理)
一条从一个焦点 $F_1$ 发出的光线射到椭圆上的某点 $P$,会被反射到另一个焦点 $F_2$,即:
\begin{equation}
\angle F_1PT = \angle F_2PT~.
\end{equation}

($T$ 为切线的切点)
一条从一个焦点出发射到椭圆上的射线,在椭圆上反射后,会朝向另一个焦点。这个性质被用于椭圆形镜面(如椭圆房顶)。椭圆任一点的切线,使得该点到两个焦点的连线之间的夹角相等(即入射角=反射角)。路径 $F_1PF_2$ 是 $F_1 \to F_2$ 所有路径中最短的一条绕射路径。


焦半径夹角相等
在椭圆上任意一点 $P$,连到两个焦点 $F_1$, $F_2$ 的线段与该点的切线夹角相等。

	10.	切线交焦线定长性质
过椭圆上一点作切线,交焦点连线于两侧,其两个交点到该点的距离之差恒定。







