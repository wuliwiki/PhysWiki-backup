% 矩阵指数
% keys 矩阵|矩阵指数|特征值|特征子空间|特征分解


\subsection{定义}
实数域上的指数函数$\E^x$可以进行Maclaulin展开:\begin{equation}
\E^x=\sum\limits_{n=0}^\infty \frac{x^n}{n!}
\end{equation}

展开式使得我们只需要用$x$的幂就可以表示指数$\E^x$.我们把这一点应用到矩阵中,就可以用方阵的幂来定义出矩阵的指数:

\begin{definition}{矩阵指数}
给定方阵$\bvec{M}$,定义$\E^\bvec{M}=\sum_{n=0}^\infty \frac{\bvec{M}^n}{n!}$,并称之为矩阵$\bvec{M}$的\textbf{指数(matrix exponential)}.
\end{definition}

矩阵指数在常微分方程中非常常用,是用来解线性齐次方程组的利器.一个矩阵的指数本身还是一个矩阵.

\subsection{矩阵指数的性质}

\subsubsection{相似变换的统一}

由\textbf{过渡矩阵}\upref{TransM}可知,如果矩阵$\bvec{M}$在某基下表示一个线性变换,那么当基按过渡矩阵$\bvec{Q}$改变时,同一个线性变换的矩阵表示就变为$\bvec{Q}^{-1}\bvec{M}\bvec{Q}$.在原基下,$\E^\bvec{M}$可以表示另一个线性变换,而它在$\bvec{Q}$下的变换是
\begin{equation}
\bvec{Q}^{-1}\E^\bvec{M}\bvec{Q}=\E^{\bvec{Q}^{-1}\bvec{M}\bvec{Q}}
\end{equation}







