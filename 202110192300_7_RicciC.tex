% 曲率张量场
% keys 里奇曲率|曲率|测地线|广义相对论|引力|爱因斯坦场方程|relativity|gravity|Einstein Field Equation|geodesic|curvature|differential geometry|manifold|流形|联络|connection|黎曼曲率|Riemann curvature|黎曼张量|Riemann tensor|Ricci曲率|Ricci张量|Ricci curvature|毕安基恒等式|比安基恒等式|Bianchi identity|Bianchi

\pentry{仿射联络(切丛)\upref{affcon}}

本节中默认$(M, \nabla)$是一个配备了仿射联络$\nabla$的流形$M$.

\subsection{黎曼曲率张量}

我们在这里重新誊写一遍\textbf{仿射联络(切丛)}\upref{affcon}中定义的\textbf{曲率}映射.

\begin{definition}{曲率}
定义$R:\mathfrak{X}(M)\times\mathfrak{X}(M)\to\opn{End}(\mathfrak{X}(M))$为:对于任意$X, Y\in\mathfrak{X}(M)$,有$R(X, Y)=\nabla_X\nabla_Y-\nabla_Y\nabla_X-\nabla_{[X, Y]}$,称该映射为$(M, \nabla)$的\textbf{曲率(curvature)}.
\end{definition}

为什么要这么定义曲率呢?平坦流形和弯曲流形有一个关键的区别,那就是平行移动是否依赖路径.平坦的时空中,对一个向量进行平行移动,不管它沿着什么路径走,只要回到原点,它就和出发时是同一个向量;但是不平坦的流形上,如\textbf{仿射联络(切丛)}\upref{affcon}中地球表面的例子,平行移动后回到原点时是哪个向量,取决于走的是什么道路.

本质上,平坦流形上的路径无关性,来自偏微分算子的交换性,即$\partial_i\partial_j-\partial_j\partial_i=0$.那么当平行移动取决于路径时,人们自然想到通过$\nabla_i\nabla_j-\nabla_j\nabla_i$的值来研究路径是如何影响平行移动的.由于我们希望讨论的是流形上的张量,而不是局限于给定图上的函数排列,因此又额外加了一项$\nabla_{[X, Y]}$,把它凑成一个张量场.

整体来看,曲率就是把两个光滑向量场映射为一个光滑向量场.我们接下来就证明,这样构造的量确实是一个张量场.

\begin{theorem}{曲率的线性性}\label{RicciC_the1}
令$f, g, h$为$M$上的光滑函数,$X, Y, Z$为$M$上的光滑向量场.则我们有:$R(fX, gY)hZ=fgh\cdot R(X, Y)Z$.
\end{theorem}

\textbf{证明}:

由于$\nabla_{fX}=f\nabla_X$,且$[fX, gY]=fg[X, Y]+f(Xg)Y-g(Yf)X$

故有
\begin{equation}
\begin{aligned}
&\nabla_{fX}\nabla_{gY}-\nabla_{gY}\nabla_{fX}-\nabla_{[fX, gY]}\\=&f\nabla_X(g\nabla_Y)-g\nabla_Y(f\nabla_X)-fg\nabla_{[X, Y]}-\nabla_{f(Xg)Y-g(Yf)X}\\
\\=&fg\nabla_X\nabla_Y+f(Xg)\nabla_Y-gf\nabla_Y\nabla_X-g(Yf)\nabla_X\\&-fg\nabla_{[X, Y]}-f(Xg)\nabla_Y+g(Yf)\nabla_X\\
=&fg\nabla_X\nabla_Y-gf\nabla_Y\nabla_X-fg\nabla_{[X, Y]}\\
=&fg(\nabla_X\nabla_Y-\nabla_Y\nabla_X-\nabla_{[X, Y]})
\end{aligned}
\end{equation}
这就证明了$R(fX, gY)=fgR(X, Y)$.接下来证明$R(X, Y)hZ=hR(X, Y)Z$.

考虑到
\begin{equation}
\begin{aligned}
\nabla_X\nabla_Y(hZ)&=\nabla_X(h\nabla_YZ+(Yh)Z)\\
&=h\nabla_X\nabla_YZ+(Xh)\nabla_YZ+(Yh)\nabla_XZ+(XYh)Z
\end{aligned}
\end{equation}

故
\begin{equation}
\begin{aligned}
(\nabla_X\nabla_Y-\nabla_Y\nabla_X)hZ=& h\nabla_X\nabla_YZ-h\nabla_Y\nabla_XZ+\\&(Xh)\nabla_YZ-(Yh)\nabla_XZ+\\&(Yh)\nabla_XZ-(Xh)\nabla_YZ+\\&(XYh)z-(YXh)z\\
=& h(\nabla_X\nabla_Y-\nabla_Y\nabla_X)Z+([X, Y]h)Z
\end{aligned}
\end{equation}

于是最后有
\begin{equation}
\begin{aligned}
&(\nabla_X\nabla_Y-\nabla_Y\nabla_X-\nabla_{[X, Y]})hZ\\
&=h(\nabla_X\nabla_Y-\nabla_Y\nabla_X)Z+([X, Y]h)Z-h\nabla_{[X, Y]}Z-([X, Y]h)Z\\
&=h(\nabla_X\nabla_Y-\nabla_Y\nabla_X-\nabla_{[X, Y]})Z
\end{aligned}
\end{equation}

从而得证$R(X, Y)hZ=hR(X, Y)Z$.

\textbf{证毕}.

\autoref{RicciC_the1} 意味着,曲率是一个$M$上光滑向量场的线性映射,也就是说,是一个张量场.又因为曲率将三个向量场映射为一个向量场,我们可以在给定图中将曲率映射表示为一组光滑函数$R^r_{kij}$\footnote{注意下标的位置,$h^k$对应的下标在第一个.},使得$R(f^i\uvec{e}_i, g^j\uvec{e}_j)h^k\uvec{e}_k=(R^r_{kij}f^ig^jh^k)\uvec{e}_r$.从证明过程中也可以看到,补充的$\nabla_{[X, Y]}$的必要性.

如果用抽象指标,将$X, Y, Z$分别表示为$x^i, y^j, z^k$,那么$R(X, Y)Z=R^r_{kij}x^iy^jz^k$.


\subsubsection{曲率张量场的坐标}

在给定图中,联络$\nabla$由Christoffel符号$\Gamma^k_{ij}$完全刻画,那么我们应该也能用Christoffel符号计算出曲率张量场的坐标.记住,$\Gamma$中每个元素的类型是$M$上的光滑向量场.

\begin{equation}
\begin{aligned}
\nabla_{\partial_i}\nabla_{\partial_j}\partial_k&=\nabla_{\partial_i}(\Gamma^s_{jk}\partial_s)\\
&=(\partial_i\Gamma^s_{jk})\partial_s+\Gamma^s_{jk}\Gamma^r_{is}\partial_r\\
&=(\partial_i\Gamma^r_{jk}+\Gamma^s_{jk}\Gamma^r_{is})\partial_r
\end{aligned}
\end{equation}


考虑到偏微分的交换性,$[\partial_i, \partial_j]=0$,因此
\begin{equation}
\begin{aligned}
&(\nabla_{\partial_i}\nabla_{\partial_j}-\nabla_{\partial_j}\nabla_{\partial_i})\partial_k\\
=&(\partial_i\Gamma^r_{jk}-\partial_j\Gamma^{r}_{ik}+\Gamma^s_{jk}\Gamma^r_{is}-\Gamma^s_{ik}\Gamma^r_{js})\partial_r
\end{aligned}
\end{equation}

由定义,$R(f^i\partial_i, g^j\partial_j)h^k\partial_k=(R^r_{kij}f^ig^jh^k)\partial_r$,故有
\begin{equation}\label{RicciC_eq1}
R^r_{kij}=\partial_i\Gamma^r_{jk}-\partial_j\Gamma^{r}_{ik}+\Gamma^s_{jk}\Gamma^r_{is}-\Gamma^s_{ik}\Gamma^r_{js}
\end{equation}

\subsubsection{指标下降后的坐标}

黎曼曲率张量场经常被用在进行内积的场合中,比如高斯绝妙定理就可以表示为$K=<R(\uvec{e}_1, \uvec{e}_2)\uvec{e}_1, \uvec{e}_2>$.带上内积运算后,我们也可以把它认为是一个“将四个向量场映射为一个光滑函数”的张量,也就是说,指标下降一下:
\begin{equation}
R_{rkij}=g_{ar}R^a_{kij}
\end{equation}

由于式中出现了$g_{ar}$,再用\autoref{RicciC_eq1} 那样纯粹用Christoffel符号来描述其坐标已经不方便了,因此我们还得把Christoffel符号展开来,用$g_{ab}$表示.

以下是详细展开过程,但是\autoref{RicciC_eq6} 和\autoref{RicciC_eq2} 都只是中间步骤,请读者酌情选择跳过或跟一遍.最终结果是\autoref{RicciC_eq3} .

引用\autoref{CrstfS_eq3}~\upref{CrstfS},誊抄如下:
\begin{equation}
\Gamma^{r}_{ij}=\frac{1}{2}g^{kr}(\partial_ig_{jk}+\partial_jg_{ki}-\partial_kg_{ij})
\end{equation}

代入\autoref{RicciC_eq1} 得:

\begin{equation}\label{RicciC_eq6}
\begin{aligned}
2R^r_{kij}=&\partial_i(g^{ar}\partial_jg_{ka}+g^{ar}\partial_kg_{aj}-g^{ar}\partial_ag_{jk})-\\
&\partial_j(g^{ar}\partial_ig_{ka}+g^{ar}\partial_kg_{ai}-g^{ar}\partial_ag_{ik})+\\
&g^{as}(\partial_jg_{ka}+\partial_kg_{aj}-\partial_ag_{jk})g^{br}(\partial_ig_{sb}+\partial_sg_{bi}-\partial_bg_{is})-\\
&g^{as}(\partial_ig_{ka}+\partial_kg_{ai}-\partial_ag_{ik})g^{br}(\partial_jg_{sb}+\partial_sg_{bj}-\partial_bg_{js})\\
=&(\partial_ig^{ar})(\partial_jg_{ka})+g^{ar}\partial_i\partial_jg_{ka}+\\
&(\partial_ig^{ar})(\partial_kg_{aj})+g^{ar}\partial_i\partial_kg_{aj}-\\
&(\partial_ig^{ar})(\partial_ag_{jk})-g^{ar}\partial_i\partial_ag_{jk}-\\%First line above, expanded
&(\partial_jg^{ar})(\partial_ig_{ka})-g^{ar}\partial_j\partial_ig_{ka}-\\
&(\partial_jg^{ar})(\partial_kg_{ai})-g^{ar}\partial_j\partial_kg_{ai}+\\
&(\partial_jg^{ar})(\partial_ag_{ik})+g^{ar}\partial_j\partial_ag_{ik}+\\%Second line above
&g^{as}g^{br}(\partial_jg_{ka}+\partial_kg_{aj}-\partial_ag_{jk})(\partial_ig_{sb}+\partial_sg_{bi}-\partial_bg_{is})-\\
&g^{as}g^{br}(\partial_ig_{ka}+\partial_kg_{ai}-\partial_ag_{ik})(\partial_jg_{sb}+\partial_sg_{bj}-\partial_bg_{js})
\end{aligned}
\end{equation}

最后两行尚未展开,就已经这么长了,这也是为什么我们不用度量张量来表示黎曼张量,而是用Christoffel符号的原因.进行指标下降后,我们得到:

\begin{equation}\label{RicciC_eq2}
\begin{aligned}
2R_{tkij}=&2g_{rt}R^r_{kij}\\
=&g_{rt}(\partial_ig^{ar})(\partial_jg_{ka})+\delta^a_t\partial_i\partial_jg_{ka}+\\
&g_{rt}(\partial_ig^{ar})(\partial_kg_{aj})+\delta^a_t\partial_i\partial_kg_{aj}-\\
&g_{rt}(\partial_ig^{ar})(\partial_ag_{jk})-\delta^a_t\partial_i\partial_ag_{jk}-\\%First line above, expanded
&g_{rt}(\partial_jg^{ar})(\partial_ig_{ka})-\delta^a_t\partial_j\partial_ig_{ka}-\\
&g_{rt}(\partial_jg^{ar})(\partial_kg_{ai})-\delta^a_t\partial_j\partial_kg_{ai}+\\
&g_{rt}(\partial_jg^{ar})(\partial_ag_{ik})+\delta^a_t\partial_j\partial_ag_{ik}+\\%Second line above
&g^{as}\delta^b_t(\partial_jg_{ka}+\partial_kg_{aj}-\partial_ag_{jk})(\partial_ig_{sb}+\partial_sg_{bi}-\partial_bg_{is})-\\
&g^{as}\delta^b_t(\partial_ig_{ka}+\partial_kg_{ai}-\partial_ag_{ik})(\partial_jg_{sb}+\partial_sg_{bj}-\partial_bg_{js})\\
=&g_{rt}(\partial_ig^{ar})(\partial_jg_{ka})+\partial_i\partial_jg_{kt}+\\
&g_{rt}(\partial_ig^{ar})(\partial_kg_{aj})+\partial_i\partial_kg_{tj}-\\
&g_{rt}(\partial_ig^{ar})(\partial_ag_{jk})-\partial_i\partial_tg_{jk}-\\%First line above, expanded
&g_{rt}(\partial_jg^{ar})(\partial_ig_{ka})-\partial_j\partial_ig_{kt}-\\
&g_{rt}(\partial_jg^{ar})(\partial_kg_{ai})-\partial_j\partial_kg_{ti}+\\
&g_{rt}(\partial_jg^{ar})(\partial_ag_{ik})+\partial_j\partial_tg_{ik}+\\%Second line above
&g^{as}(\partial_jg_{ka}+\partial_kg_{aj}-\partial_ag_{jk})(\partial_ig_{st}+\partial_sg_{ti}-\partial_tg_{is})-\\
&g^{as}(\partial_ig_{ka}+\partial_kg_{ai}-\partial_ag_{ik})(\partial_jg_{st}+\partial_sg_{tj}-\partial_tg_{js})\\
=&g_{rt}(\partial_ig^{ar})(\partial_jg_{ka})+\\
&g_{rt}(\partial_ig^{ar})(\partial_kg_{aj})+\partial_i\partial_kg_{tj}-\\
&g_{rt}(\partial_ig^{ar})(\partial_ag_{jk})-\partial_i\partial_tg_{kj}-\\%First line above, expanded
&g_{rt}(\partial_jg^{ar})(\partial_ig_{ka})-\\
&g_{rt}(\partial_jg^{ar})(\partial_kg_{ai})-\partial_j\partial_kg_{ti}+\\
&g_{rt}(\partial_jg^{ar})(\partial_ag_{ik})+\partial_j\partial_tg_{ki}+\\%Second line above
&g^{as}(\partial_ig_{at}+\partial_ag_{ti}-\partial_tg_{ia})(\partial_jg_{ks}+\partial_kg_{sj}-\partial_sg_{jk})-\\
&g^{as}(\partial_ig_{ka}+\partial_kg_{ai}-\partial_ag_{ik})(\partial_jg_{st}+\partial_sg_{tj}-\partial_tg_{js})\\
\end{aligned}
\end{equation}

其中第二个等号是直接把$g_{rt}$乘到展开式中,并按升降法则将$g_{rt}g^{ar}$和$g_{rt}g^{br}$分别写为$\delta^a_t$和$\delta^b_t$;第三个等号是消去$\delta^i_j$项,进行下标替换;第四个等号删去了互相抵消的两项,按照$g_{ab}$的对称性调整了几个下标的位置,并将倒数第二项的$s$和$a$两个赝指标地位对换、两个括号顺序对换.

但是等等!还没完,我们还能进一步简化.我把这一步单独摘出来,是为了不让原式过于臃肿.这步简化是什么呢?考虑到$g^{ar}g_{rt}=\delta^a_t$,且$\partial_i\delta^a_t=0$,因此应用Leibniz律可得$g^{ar}\partial_ig_{rt}=-g_{rt}\partial_ig^{ar}$.把这一点代入\autoref{RicciC_eq2} ,即可得如下简化:

\begin{equation}\label{RicciC_eq3}
\begin{aligned}
2R_{tkij}=&2g_{rt}R^r_{kij}\\
=&-g^{ar}(\partial_ig_{rt})(\partial_jg_{ka})\\
&-g^{ar}(\partial_ig_{rt})(\partial_kg_{aj})+\partial_i\partial_kg_{tj}\\
&+g^{ar}(\partial_ig_{rt})(\partial_ag_{jk})-\partial_i\partial_tg_{kj}\\%First line above, expanded
&+g^{ar}(\partial_jg_{rt})(\partial_ig_{ka})\\
&+g^{ar}(\partial_jg_{rt})(\partial_kg_{ai})-\partial_j\partial_kg_{ti}\\
&-g^{ar}(\partial_jg_{rt})(\partial_ag_{ik})+\partial_j\partial_tg_{ki}+\\%Second line above
&g^{as}(\partial_ig_{at}+\partial_ag_{ti}-\partial_tg_{ia})(\partial_jg_{ks}+\partial_kg_{sj}-\partial_sg_{jk})-\\
&g^{as}(\partial_ig_{ka}+\partial_kg_{ai}-\partial_ag_{ik})(\partial_jg_{st}+\partial_sg_{tj}-\partial_tg_{js})\\
\end{aligned}
\end{equation}



简化到这一步,我们就已经可以清晰地看到黎曼张量的对称性了.这些对称性会在本节稍后的小节中集中讨论.











\subsection{Ricci张量场}

如果我们固定$Y, Z$,那么$R(X, Y)Z$可以看成是$X$的一个$C^{\infty}$-线性映射;更准确地说,对于任意$p\in M$,$R(X, Y)Z$都是$X_p$的一个$\mathbb{R}$-线性映射.为了方便,我们记$T_{Y, Z}$是这样的线性映射:$T_{Y, Z}(X)=R(X, Y)Z$.那么$\opn{trace}T_{Y, Z}$就是$M$上的一个光滑函数.这样一来,我们又相当于得到了一个将$Y$和$Z$映射为一个光滑函数的张量场.

用抽象指标,将$X, Y, Z$分别表示为$x^i, y^j, z^k$,那么$T_{Y, Z}(X)=R^r_{kij}y^jz^kx^i$,因此$T_{Y, Z}=R^r_{kij}y^jz^k$.于是,$\opn{trace}T_{Y, Z}=R^i_{kij}y^jz^k$\footnote{注意,就是把上指标$r$换成了$i$,因为对于矩阵(或者抽象指标表示的线性变换)$a^i_j$,其\textbf{迹}的定义就是$\opn{trace}a^i_j=a^i_i$.}

而因为我们想把$\opn{trace}T_{Y, Z}$看成是$Y, Z$的映射,因此最终得到一个张量场$R^i_{kij}$,它把两个光滑向量场映射为一个光滑函数.这个张量场,就是所谓的\textbf{Ricci曲率}.

\begin{definition}{Ricci曲率}
已知曲率张量场$R^r_{kij}$,则定义张量场$R_{kj}=R^i_{kij}$,称之为\textbf{Ricci曲率(Ricci curvature)}或者\textbf{Ricci张量场(Ricci tensor field)}.
\end{definition}

黎曼曲率张量$R^r_{kij}$涉及的向量场太多,而Ricci场通过对其进行缩并,得到了一个更简单的张量场,它同样可以很好地描绘流形的性质,同时是\textbf{爱因斯坦场方程}的核心结构.

我们上面已经计算过了黎曼曲率$R^r_{kij}$的坐标表达,只需要进行指标替换就可以得到Ricci张量场的坐标:
\begin{equation}\label{RicciC_eq9}
R_{kj}=\partial_i\Gamma^i_{jk}-\partial_j\Gamma^{i}_{ik}+\Gamma^s_{jk}\Gamma^i_{is}-\Gamma^s_{ik}\Gamma^i_{js}
\end{equation}

结合\autoref{CrstfS_eq3}~\upref{CrstfS}:
\begin{equation}
\Gamma^{r}_{ij}=\frac{1}{2}g^{kr}(\partial_ig_{jk}+\partial_jg_{ki}-\partial_kg_{ij})
\end{equation}
还可以根据度量场$g_{ij}$计算出Ricci曲率场,写起来非常长,在此从略.





%Scalor curvature









\subsection{曲率张量场的性质}

\begin{theorem}{反对称性}
对于任意$X, Y\in\mathfrak{X}(M)$,有$R(X, Y)=-R(Y, X)$.
\end{theorem}


这一点由黎曼曲率张量的定义直接可得.用抽象指标表示,就是$R^i_{cab}=-R^i_{cba}$,再两边乘以一个$g_{di}$,得到$R_{dcab}=-R_{dcba}$.





\begin{theorem}{度量反对称性}
对于任意$X, Y, U, V\in\mathfrak{X}(M)$,有$<R(X, Y)U, V>=-<R(X, Y)V, U>$.


\end{theorem}

用抽象指标表示,就是$R^i_{cab}g_{id}=-R^i_{dab}g_{ic}$,或者写为$R_{dcab}=-R_{cdab}$.我们根据\autoref{RicciC_eq3}  ,交换其中的$t$和$k$指标,就可以证明\footnote{证明难点提示:考虑$-g^{ar}(\partial_ig_{rt})(\partial_jg_{ka})+g^{ar}(\partial_jg_{rt})(\partial_ig_{ka})$这一项,由于$a, r$都是赝指标,故也可以同时调换,于是$-g^{ar}(\partial_ig_{rt})(\partial_jg_{ka})+g^{ar}(\partial_jg_{rt})(\partial_ig_{ka})=-g^{ra}(\partial_ig_{at})(\partial_jg_{kr})+g^{ra}(\partial_jg_{at})(\partial_ig_{kr})$.此时再调换$t, k$,就能得到$-g^{ar}(\partial_ig_{rk})(\partial_jg_{ta})+g^{ar}(\partial_jg_{rk})(\partial_ig_{ta})$,刚好变成相反数.其它所有项都需要这样同时调换一下赝指标,从而看出$t, k$调换以后变成相反数.}这一点.

\begin{theorem}{交换对称性}
\begin{equation}\label{RicciC_eq5}
R_{tkij}=R_{ijtk}
\end{equation}
\end{theorem}

同样,在\autoref{RicciC_eq3} 中交换$ij$和$tk$指标即可证明.

\begin{theorem}{第一Bianchi恒等式(代数)}
\begin{equation}\label{RicciC_eq4}
R_{tkij}+R_{tijk}+R_{tjki}=0
\end{equation}
\end{theorem}

这次我们根据\autoref{RicciC_eq1} ,注意到把等号右边分成两组,分别进行下标$ijk$的轮换,即可证明$R^r_{kij}+R^r_{ijk}+R^r_{jki}=0$,两边乘以一个$g_{rt}$即得证.

\begin{theorem}{第二Bianchi恒等式(微分)}
\begin{equation}\label{RicciC_eq7}
\partial_aR_{tkij}+\partial_iR_{tkja}+\partial_jR_{tkai}=0
\end{equation}
\end{theorem}

用\autoref{RicciC_eq1} 可以证明上标版本的恒等式$\partial_aR_{kij}^r+\partial_iR_{kja}^r+\partial_jR_{kai}^r=0$,把$g_{rt}$乘进去,展开\footnote{就是说,$\partial_aR_{tkij}=\partial_a(g_{rt}R^{r}_{kij})=g_{rt}\partial_aR^r_{kij}+R^r_{kij}\partial_ag_{rt}$.},再把\autoref{RicciC_eq5} 和\autoref{RicciC_eq4} 代进去,即可得证.


以上五条定理,就是黎曼张量最重要的五条对称性,我们将下标修改成顺眼的形式,集中总结如下:

\begin{theorem}{黎曼曲率张量的对称性}
设$R_{abcd}$是流形$(M, \nabla)$上的黎曼曲率张量(全下标形式)在\textbf{任意}图中的坐标表达,即$<R(x^a\partial_a, y^b\partial_b)v^c\partial_c, w^d\partial_d>=x^ay^bv^cw^dR_{abcd}$,那么我们有如下对称性:
\begin{enumerate}
\item $R_{abcd}=-R_{abdc}$;
\item $R_{abcd}=-R_{bacd}$;
\item $R_{abcd}=R_{cdab}$;
\item $R_{abcd}+R_{acdb}+R_{adbc}=0$;
\item $\partial_eR_{abcd}+\partial_cR_{abde}+\partial_dR_{abec}=0$.
\end{enumerate}
\end{theorem}

定理中强调了是在\textbf{任意}的图中都成立,因此也可以直接推广为张量的抽象指标形式.特别地,偏微分算子直接推广为图的坐标方向的联络,于是第二Bianchi恒等式变为:
\begin{equation}\label{RicciC_eq8}
\nabla_eR_{abcd}+\nabla_cR_{abde}+\nabla_dR_{abec}=0
\end{equation}

由于Ricci张量定义为黎曼张量的缩并:$R_{ij}=R^a_{iaj}=R_{aibj}g^{ab}$,因此可以直接继承黎曼张量的对称性:

\begin{theorem}{Ricci曲率张量的对称性}
$R_{ij}=R_{ji}$.
\end{theorem}

\subsubsection{符号上的拓展}

对于任何嵌套矩阵$T_{i_1i_2\cdots i_n}$,定义
\begin{equation}
T_{[i_1i_2\cdots i_n]}=\frac{1}{n}\sum\limits_{\sigma\in S_n}\opn{sgn}(\sigma)T_{i_{\sigma(1)}i_{\sigma(2)}\cdots i_{\sigma(n)}}
\end{equation}
具体到$n=3$的情况,就是$T_{[abc]}=\frac{1}{3}(T_{abc}+T_{bca}+T_{cab}-T_{acb}-T_{cba}-T_{bac})$.





这样,我们还可以把第一Bianchi恒等式表示为$R_{a[bcd]}=0$,而把第二Bianchi恒等式表示为\footnote{借用对称性$R_{abcd}=R_{cdab}$,并将$\nabla_eR_{cdab}$认为是一个整体$T_{ecbad}$,这样$\nabla_{[e}R_{cd]ab}=T_{[ecb]ad}$.}$\nabla_{[e}R_{cd]ab}=0$.








