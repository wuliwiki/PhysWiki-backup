% 霍奇星算子
% Hodge算子|Hodge 算子|Hodge star operator|星算子|Hodge 星算子|外代数|Grassmann 代数|Exterior algebra|麦克斯韦方程组|对偶

\pentry{外导数\upref{ExtDer}}

考虑$n$维线性空间$V$上的外代数$\bigwedge V$,我们注意到各阶的外积空间具有明显的对称性:$\opn{dim}\bigwedge^k V=\opn{dim}\bigwedge^{n-k} V$.这意味着这样的一对空间之间存在线性同构,我们使用星算子$\star$来描述这一同构.

星算子是一个映射,把一个$\bigwedge^k V$中的元素$\omega$映射为一个$\bigwedge^{n-k} V$中的元素$\star\omega$.为了方便描述星算子的定义,我们先引入一些新的表示方法.

选定$V$的基$\{\bvec{e}_i\}$,那么任意$\omega\in\bigwedge^k V$都可以表示为各$\bvec{e}_{i_1}\wedge\bvec{e}_{i_2}\wedge\cdots\wedge\bvec{e}_{i_k}$的线性组合,因此我们只需要描述$\star\bvec{e}_{i_1}\wedge\bvec{e}_{i_2}\wedge\cdots\wedge\bvec{e}_{i_k}$即可定义星算子.

为了方便,我们只考虑$\bvec{e}_{i_1}\wedge\bvec{e}_{i_2}\wedge\cdots\wedge\bvec{e}_{i_k}$中各$i_{r+1}>i_r$的情况,也就是下标顺序排列的情况\footnote{乱序排列的情况无非两种,奇排列和偶排列,根据外代数的定义,前者加上负号即可,后者和顺序排列是相等的.}.考虑到





