% 康普顿散射
% 康普顿散射|量子力学|相对论|能量守恒|动量守恒

\pentry{光子,相对论动量,相对论能量}
康普顿效应是射线与物质相互作用的三种效应之一.康普顿效应是指入射光子与物质原子中的核外电子产生非弹性碰撞而被散射的过程.碰撞时,入射光子把部分能量转移给电子,使它脱离原子成反冲电子,而散射光子的能量和运动方向发生变化.如\autoref{Comptn_fig1}所示,其中$h\nu$是入射$\gamma$光子的能量,$h\nu^\prime$是散射$\gamma$光子的能量,$\theta$是散射光子的散射角,$e$是反冲电子,$\phi$是反冲电子的反冲角.
\begin{figure}[ht]
\centering
\includegraphics[width=10cm]{./figures/Comptn_1.pdf}
\caption{康普顿散射} \label{Comptn_fig1}
\end{figure}
高能光子与自由电子发生弹性碰撞,要考虑相对论效应.由于光子能量很高,可假设自由电子初始不动.
\begin{equation}\label{Comptn_eq1}
\Delta \lambda  = \frac{h}{m_e c}(1 - \cos\theta )
\end{equation}           
光子的动量为 $p = h/\lambda$, 能量为 $cp$.能量守恒
\begin{equation}
m_e c^2 + c p_i - c p_f = \sqrt{m_e^2 c^4 + p_e^2 c^2}
\end{equation} h
动量守恒
\begin{equation}
\bvec p_i - \bvec p_f = \bvec p_e
\end{equation}
两式平方,消去 $p_e$ 得波长差\autoref{Comptn_eq1}. 
 