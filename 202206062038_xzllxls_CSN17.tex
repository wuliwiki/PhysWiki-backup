% 2017 年计算机学科专业基础综合全国联考卷
% 2017 年计算机 全国 联考卷


一、单项选择题:1~40小题,每小题2分,共80分.下列每题给出的四个选项中,只有一个选项符合题目要求.

    1.下列函数的时间复杂度是
    int func( int n1
    {int i=0,sum=0;
    while(sum<n)sum+=++i;
    return i:
    {
    A. O(log n)  B.O(nl/2)    C.0(n)    D. O(nlog n)
    2.下列关于栈的叙述中,错误的是
    I.采用非递归方式重写递归程序时必须使用栈
    II.函数调用时,系统要用栈保存必要的信息
    III.只要确定了入栈次序,即可确定出栈次序
    Ⅳ,栈是一种受限的线性表,允许在其两端进行操作
    A.仅I    B.仅I、II、III
    C.仅I、Ⅲ、Ⅳ    D.仅II、III、Ⅳ
    3.适用于压缩存储稀疏矩阵的两种存储结构是
    A.三元组表和十字链表  B.三元组表和邻接矩阵
    C.十字链表和二叉链表    D.邻接矩阵和十字链表
    4.要使一棵非空二叉树的先序序列与中序序列相同,其所有非叶结点须满足的条件是
    A.只有左子树    B.只有右予树
    C.结点的度均为1    D.结点的度均为2
    5.己知一棵二叉树的树形如下图所示,其后序序列为e,a,c,b.d,g,f,树中与结点a
同层的结点是
A.C    B.d    C.f    D.g
6.己知字符集{a,b,c,d,e,f,g,h},若各字符的哈夫曼编码依次是
0100, 10, 0000, 0101, 001, 011, 11, 0001,则编码序列
010001 100100101 1 1 10101的译码结果是
A. acgabfh    B. adbagbb
C. afbeagd  D. afeefgd