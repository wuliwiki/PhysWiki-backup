% SymPy 符号计算笔记

\begin{issues}
\issueDraft
\end{issues}

\pentry{Python 符号计算简介\upref{SymPy}}

\begin{itemize}
\item \verb|import sympy as sm|
\item \verb|x0, x1 = sm.symbols('x0, x1')| 声明变量, 类型为 \verb|sympy.core.symbol.Symbol|
\item \verb|x, y, z = sm.symbols('x:z')|, \verb|x4, x5, x6, x7 = sm.symbols('x4:8')|
\item 常数如 \verb|sm.pi, sm.E, sm.I, sm.oo| (无穷)
\item 整数为 \verb|sm.numer(n)|, 类型为 \verb|sympy.core.numbers.Integer|, 也可能是 \verb|sympy.core.numbers.One|
\item 有理数 \verb|sm.Rational(n,m)|. 如果 \verb|n, m| 已经是 \verb|numbers.Integer| 类型, 也可以直接 \verb|n/m|
\item 函数如 \verb|sm.sin(), sm.asin(), sm.sinh(), sm.exp(), sm.log(), sm.sqrt()|
\end{itemize}

\subsection{基本运算}
\begin{itemize}
\item \verb|sm.summation(含i的表达式, (i, 1, 5))|, 例如 \verb|sm.summation(1/2**n, (n, 1, sm.oo))| 得 \verb|1|.
\end{itemize}

