% SLISC 的矢量和矩阵

\begin{issues}
\issueDraft
\end{issues}

\pentry{SLISC 库简介\upref{SLISC}, C++ 类的定义和继承}

我们先以双精度类型 \verb|Doub| (即 \verb|double|)来举例介绍 SLISC 中的\textbf{容器(container)}. 容器指的就是像 \verb|std::vector| 这样的可以储存多个元素的对象, 下面介绍的矢量,矩阵,高维数组等都是容器.

C/C++ 的数组是储存于内存的 stack (栈)中的, 而 stack 一般只有几个 Mb, 一旦数组太大就会产生 stack overflow 错误. 第二是 stack 中的数据的大小都只能在编译时确定, 所以我们不能在运行是确定数组的长度(例如从文件中读取, 或通过运行时的计算得到). 所以在一般来说容器都会 allocate (分配)到内存的 heap (堆)中. 在 C++ 中, 在 heap 中分配内存用 \verb|new|, 释放则用 \verb|delete|.

\verb|std::vector| 可以说是使用最广的矢量容器, 理论上我们可以用 \verb|vector| 的 \verb|vector| 定义一个 heap 中的矩阵, 但是这样做数据在内存中是不连续的, 而为了性能上的考虑, 我们需要给每个容器分配一块连续的内存.
\begin{lstlisting}[language=cpp]
// 用 std::vector 的嵌套定义矩阵
vector<vector<double>> a;
// 分配内存
long N1 = 10, N2 = 10;
a.resize(N1);
for (long i = 0; i < N1; ++i)
    a[i].resize(N2);
// 全部元素赋值为 0
for (long i = 0; i < N1; ++i)
    for (long j = 0; j < N2; ++j)
        a[i][j] = 0;
\end{lstlisting}


我们构建矢量/矩阵的数据类型很简单, 首先

================= 回收 ============

虽然我们可以直接使用 \verb|std::vector|, 但是通过创建一个简单的矢量类我们可以了解写一个类的基本思路, 在这之后就可以按照同样的思路创建各种容器了.

注意 Vector 只有两个很简单的数据成员, 分别是类型 \verb|T| 的指针以及矢量的长度. 事实上矩阵元数据并不在 Vector 内部, 而是在 constructor 或 \verb|resize()| 函数中通过 \verb|new| 命令来动态分配的, 所以如果我们在程序中用 \verb|sizeof(Vector<Doub>)| 来测试占用内存, 得到的字节数将会是 \verb|m_p| 和 \verb|m_N| 的字节数相加, 而不包括矩阵元占用的空间. 如果在程序中直接用 \verb|Vector<Doub> a| 来声明变量, \verb|m_p| 和 \verb|m_N| 将会储存在 stack 中, 而矩阵元(以后如果有的话)将会在储存在 heap 中.

============ 结束回收 ====================

\subsection{Vbase 类}
所有容器都继承于 \verb|VbaseDoub|, \verb|VbaseDoub| 不直接使用, 仅用于继承.
\begin{lstlisting}[language=cpp]
class VbaseDoub
{
protected:
    Doub *m_p;
    Long m_N;
public:
    typedef Doub value_type;
    VbaseDoub();
    explicit VbaseDoub(Long_I N);
    VbaseDoub(const VbaseDoub &rhs);

    Doub* p();
    const Doub* p() const;
    Long size() const;
    void resize(Long_I N);
    Doub &operator[](Long_I i);
    const Doub &operator[](Long_I i) const;
    Doub& end();
    const Doub& end() const;
    Doub& end(Long_I i);
    const Doub& end(Long_I i) const;
    void operator<<(VbaseDoub &rhs);
    ~VbaseDoub();
};
\end{lstlisting}

简单来说, \verb|VbaseDoub| 就是一个简单的 \verb|std::vector|, 负责内存管理, 用方括号 \verb|[i]| 获取第 \verb|i| 个矩阵元, 用 \verb|end()| 获取最后一个元, \verb|end(i)| 获取第 \verb|N-i| 个元. 用 \verb|.size()| 获取元素长度, \verb|.resize(N)| 释放内存并重新分配 \verb|N| 个元素的新内存. \verb|<<| 用于把一个对象所分配的内存转移到另一个对象中. 所有继承 \verb|VbaseDoub| 的容器都具有这些功能, 除了\verb|resize()| 会根据矩阵的维度重新定义.

\subsection{Vec 容器}
为了避免歧义, 任何两个容器之间禁止使用等号.
\begin{lstlisting}[language=cpp]
class VecDoub : public VbaseDoub
{
public:
    typedef VbaseDoub Base;
    VecDoub() = default;
    explicit VecDoub(Long_I N);
    VecDoub(const VecDoub &rhs);
    VecDoub &operator=(const VecDoub &rhs) = delete;
    void operator<<(VecDoub &rhs);
};
\end{lstlisting}

\subsection{Cmat 容器}
\begin{lstlisting}[language=cpp]
class CmatDoub : public VbaseDoub
{
protected:
    typedef VbaseDoub Base;
    Long m_N1, m_N2;
public:
    CmatDoub(): m_N1(0), m_N2(0) {};
    CmatDoub(Long_I N1, Long_I N2);
    CmatDoub(const CmatDoub &rhs);
    CmatDoub &operator=(const CmatDoub &rhs) = delete;
    void operator<<(CmatDoub &rhs);
    Doub& operator()(Long_I i, Long_I j);
    const Doub& operator()(Long_I i, Long_I j) const;
    Long n1() const;
    Long n2() const;
    void resize(Long_I N1, Long_I N2);
    void reshape(Long_I N1, Long_I N2);
};
\end{lstlisting}

\subsection{Cmat3 容器}
\begin{lstlisting}[language=cpp]
class Cmat3Doub : public VbaseDoub
{
protected:
    typedef VbaseDoub Base;
    Long m_N1, m_N2, m_N3;
public:
    typedef Doub value_type;
    Cmat3Doub(): m_N1(0), m_N2(0), m_N3(0) {};
    Cmat3Doub(Long_I N1, Long_I N2, Long_I N3);
    Cmat3Doub(const Cmat3Doub &rhs);
    Cmat3Doub &operator=(const Cmat3Doub &rhs) = delete;
    void operator<<(Cmat3Doub &rhs);
    void resize(Long_I N1, Long_I N2, Long_I N3);
    void reshape(Long_I N1, Long_I N2, Long_I N3);
    Doub &operator()(Long_I i, Long_I j, Long_I k);
    const Doub &operator()(Long_I i, Long_I j, Long_I k) const;
    Long n1() const;
    Long n2() const;
    Long n3() const;
};
\end{lstlisting}

\addTODO{列出其他容器}
