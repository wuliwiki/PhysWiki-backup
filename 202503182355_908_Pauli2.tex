% 沃尔夫冈·泡利(综述)
% license CCBYSA3
% type Wiki

本文根据 CC-BY-SA 协议转载翻译自维基百科\href{https://en.wikipedia.org/wiki/Wolfgang_Pauli}{相关文章}。

\begin{figure}[ht]
\centering
\includegraphics[width=6cm]{./figures/7d0315d83d1af784.png}
\caption{泡利于1945年} \label{fig_Pauli2_1}
\end{figure}
沃尔夫冈·恩斯特·泡利(Wolfgang Ernst Pauli,/ˈpɔːli/,[5] 德语:[ˈvɔlfɡaŋ ˈpaʊli];1900年4月25日—1958年12月15日)是一位奥地利物理学家,量子力学的先驱。1945年,在阿尔伯特·爱因斯坦的提名下,[6] 泡利因其“通过发现一条新的自然法则——泡利不相容原理(Pauli Exclusion Principle)所做出的决定性贡献”而获得诺贝尔物理学奖。这一发现涉及自旋理论,该理论是物质结构理论的基础。为了保持\(\beta\)衰变中的能量守恒,他提出了一种质量极小、电中性的粒子,其后被恩里科·费米命名为中微子。该粒子最终于1956年被探测到。

泡利就读于维也纳的多布林格文理中学,并于1918年以优异成绩毕业。两个月后,他发表了第一篇论文,内容涉及阿尔伯特·爱因斯坦的广义相对论。随后,他进入慕尼黑大学学习,并在阿诺德·索末菲指导下开展研究。[1]1921年7月,他因其关于电离双原子氢(\(H_2^+\))的量子理论的论文获得博士学位。[2][9]
\subsection{职业生涯}  
索末菲邀请泡利为《数学科学百科全书》撰写相对论综述。在获得博士学位后的两个月内,泡利便完成了这篇长达 237 页的文章。爱因斯坦对此给予高度评价;该文章后来以专著形式出版,并至今仍是该领域的重要参考文献。[10]
\begin{figure}[ht]
\centering
\includegraphics[width=6cm]{./figures/5713325387bd5c08.png}
\caption{沃尔夫冈·泡利用于讲授(1929年)} \label{fig_Pauli2_2}
\end{figure}
泡利曾在哥廷根大学担任马克斯·玻恩的助理一年,随后一年在哥本哈根理论物理研究所(后来的 尼尔斯·玻尔研究所)工作。[1] 从1923年到1928年,他在汉堡大学任教。[11]在此期间,泡利对现代量子力学理论的发展发挥了关键作用,特别是提出了泡利不相容原理以及非相对论性自旋理论。

1928年,泡利被任命为瑞士苏黎世联邦理工学院的理论物理教授。[1]1930年,他获得洛伦兹奖章。[12] 他还曾在1931 年担任密歇根大学客座教授,并于1935年在普林斯顿高等研究院担任访问教授。
\subsubsection{荣格}  
1930年底,就在提出中微子假说后不久,泡利经历了一场个人危机,当时他刚刚离婚,并且他的母亲自杀。1932年1月,他向居住在苏黎世附近的心理学家和心理治疗师卡尔·荣格寻求帮助。荣格立刻开始分析泡利充满原型意象的梦境,并将其纳入自己的研究。泡利随后从科学角度批判性地探讨了荣格理论的认识论基础,这些讨论帮助澄清了荣格的某些概念,尤其是\textbf{共时性}的理论。他们的讨论记录保存在泡利—荣格通信中,后来以《原子与原型》出版。而荣格对泡利400多个梦境的详细分析,则记录在《心理学与炼金术》中。1933 年,泡利出版了他的物理学著作《物理手册》的第二部分,该书被认为是量子物理学领域的权威著作。罗伯特·奥本海默称其为“唯一一本真正成熟的量子力学入门书”。[13]

1938年德国吞并奥地利后,泡利自动成为德国公民,这在1939年第二次世界大战爆发后对他来说成了一个问题。1940年,他曾试图获得瑞士公民身份,以便能够继续留在苏黎世联邦理工学院(ETH),但最终未能成功。[14]
\subsubsection{美国与瑞士} 
1940年,泡利移居美国*,在普林斯顿高等研究院担任理论物理教授。1946 年,战后,他获得美国国籍,随后回到苏黎世,并在那里度过了余生的大部分时间。1949 年,他获得了瑞士公民身份。

1958年,泡利被授予马克斯·普朗克奖章。同年,他被诊断出患有胰腺癌。当他的最后一位助手查尔斯·恩茨在苏黎世红十字医院探望他时,泡利问道:“你看到房间号了吗?” 这是137号房间。泡利一生都在思考精细结构常数的问题,这个无量纲基本常数的值接近 1/137。[15]1958 年12月15日,泡利在那个房间里去世。[16][17]
\subsection{科学研究}
泡利作为物理学家做出了许多重要贡献,主要集中在量子力学领域。然而,他很少发表论文,而更喜欢通过书信与同行交流,如丹麦哥本哈根大学的尼尔斯·玻尔和维尔纳·海森堡,两人都是他亲密的朋友。他的许多想法和研究成果从未正式发表,而只是出现在他的信件中,这些信件常被收件人复制和传播。1921年,泡利与玻尔共同提出了构造原理,该原理描述了电子按照壳层填充的方式构建原子。该名称源自德语“Aufbau”(意为“构建”),因为玻尔同样精通德语。

泡利于1924年提出了一种新的量子自由度(或量子数),其具有两个可能的取值,以解决实验观测到的分子光谱与当时正在发展的量子力学理论之间的不一致性。他提出了泡利不相容原理,这或许是他最重要的贡献。该原理指出,任何两个电子都不能处于相同的量子态,这一态由四个量子数确定,其中包括他所提出的新的二值自由度。电子自旋的概念最初由拉尔夫·克罗尼希提出。一年后,乔治·乌伦贝克和塞缪尔·古兹密特确认了泡利提出的这一自由度即为电子自旋。然而,泡利本人长期拒绝接受这一解释。[18]  

1926 年,在海森堡发表矩阵力学版的现代量子力学后不久,泡利利用该理论推导出了氢原子光谱的观测结果。这一成果对于确立海森堡理论的可信度起到了关键作用。
\begin{figure}[ht]
\centering
\includegraphics[width=6cm]{./figures/b80ac58f66886779.png}
\caption{保罗·狄拉克、沃尔夫冈·泡利与鲁道夫·派尔斯,约 1953 年} \label{fig_Pauli2_3}
\end{figure}
泡利引入了\(2\times 2\)泡利矩阵作为自旋算符的基,从而解决了非相对论性自旋理论的问题。这项工作,包括泡利方程,有时被认为影响了保罗·狄拉克 在建立相对论电子的狄拉克方程时的思路。然而,狄拉克本人表示,他当时是独立地推导出了类似的矩阵。后来,狄拉克在其相对论性费米子自旋理论中,引入了类似但维度更大的(\(4\times4\))自旋矩阵。

1930年,泡利研究\(\beta\)衰变问题。在12月4日致莉泽·迈特纳等人的信中,他以“亲爱的放射性女士们和先生们”开头,并提出了一种尚未被观测到的中性粒子的假设,该粒子的质量极小,不超过质子质量的 1\%,以解释\(\beta\)衰变的连续能谱。1934年,恩里科·费米在其\(\beta\)衰变理论中引入了这一粒子,并将其命名为“neutrino”(意大利语,意为“小中性粒子”)。1956年,弗雷德里克·雷因斯和克莱德·考恩实验上首次确认了中微子的存在,距泡利去世还有两年半。收到消息后,泡利通过电报回应道:“感谢消息。一切都会降临于懂得等待的人。——泡利”。[19]

1940年,泡利重新推导了\textbf{自旋-统计定理},这是量子场论中的一个关键结论。该定理表明:具有半整数自旋的粒子是费米子,而具有整数自旋的粒子是玻色子。

1949年,泡利发表了一篇关于泡利-维拉斯\textbf{重整化}的论文。重整化是指一种数学技术,在计算过程中对无穷的积分进行修正,使其变为有限值,从而确定理论中本质上无穷的物理量(如质量、电荷、波函数)是否能被重新定义,使其变为有限且可计算的一组量,并通过实验值加以校正。这一过程被称为重整化,它不仅消除了量子场论中的无穷大问题,还使得微扰理论中的高阶修正计算成为可能。

泡利多次批评现代进化生物学的综合理论,[20][21]他的一些当代支持者认为表观遗传继承可以支持他当初的论点。[22]

1900年,保罗·德鲁德提出了经典电子在金属固体中运动的第一个理论模型。后来,泡利和其他物理学家对德鲁德的经典模型进行了改进。泡利意识到,金属中的自由电子必须遵守费米–狄拉克统计。基于这一概念,他于1926年提出了顺磁性理论。泡利曾戏称:“Festkörperphysik ist eine Schmutzphysik.”——“固体物理是肮脏的物理。”[23]

1953年,泡利被选为英国皇家学会外籍院士,并在1955年至1957年担任瑞士物理学会主席。[1]1958年,他成为荷兰皇家艺术与科学学院的外籍院士。[24]
\subsection{性格与友谊}
\begin{figure}[ht]
\centering
\includegraphics[width=6cm]{./figures/350bd1a45c23890a.png}
\caption{} \label{fig_Pauli2_4}
\end{figure}