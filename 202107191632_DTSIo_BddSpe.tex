% 有界算子的谱
\pentry{巴拿赫定理\upref{BanThm}} 
\textbf{线性算子的谱 (spectrum)} 推广了矩阵的本征值\upref{MatEig}这一概念. 它对于了解线性算子如何作用于线性空间有着重要意义. 在这一章中, 我们主要讨论复巴拿赫空间上有界线性算子的谱. 无界算子的谱将留待后续章节讨论.

\subsection{定义}
\begin{definition}{有界算子的谱}
设$X$是复巴拿赫空间, $T:X\to X$是有界线性算子. 复数$\lambda\in\mathbb{C}$称为算子$T$的一个\textbf{谱点 (spectral point)}, 如果$\lambda-T$不是可逆映射. $T$的全体谱点的集合记为$\sigma(T)$, 称为\textbf{谱集 (spectrum)}, 而补集$\mathbb{C}\setminus\sigma(T)$称作\textbf{预解集 (resolvent set)}, 有时记为$\rho(T)$.
\end{definition}

既然$T-\lambda$是有界算子, 根据开映像原理, 如果逆映射$(T-\lambda)^{-1}$存在, 那么它也必然是连续的. 因此$T$的谱集可以等价地定义为使得$(\lambda-T)^{-1}$不是有界线性算子的那些$\lambda$的集合.

有界算子$T$的谱集$\sigma(T)$可以分成三个互不相交的部分:

\begin{enumerate}
\item 点谱$\sigma_p(T)$, 定义为使得$\text{Ker}(\lambda-T)$为非平凡子空间的那些谱点$\lambda$的集合, 也即$T$的特征值的集合.

\item 连续谱$\sigma_c(T)$, 定义为使得$\text{Ker}(\lambda-T)=\{0\}$, 且$\text{Ran}(\lambda-T)$在空间$X$中稠密的那些谱点$\lambda$的集合.

\item 剩余谱$\sigma_r(T)$, 定义为使得$\text{Ker}(\lambda-T)=\{0\}$, 且$\text{Ran}(\lambda-T)$在空间$X$中不稠密的那些谱点$\lambda$的集合.
\end{enumerate}

接下来马上就能看到, 对于无穷维巴拿赫空间上的算子, 定义连续谱和剩余谱是十分必要的. 在讨论无界算子 (尤其是量子力学中常见的算子) 时更能看出它的用处.

\subsection{一些例子}
\begin{example}{矩阵的本征值}
设$A$是$n\times n$复矩阵. 可以把它等价地看成$\mathbb{C}^n$到自己的线性算子. 如果$\lambda I_n-A$不可逆, 那么它的行列式必定为零. 按照线性方程组的性质, 必然存在一个非零向量$v\in \mathbb{C}^n$使得$\lambda v-Av=0$, 也就是说$\lambda$是$A$的特征值. 这样看来, 对于有限维空间上的线性算子, 谱的概念和特征值是等价的.
\end{example}