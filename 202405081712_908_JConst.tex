% 雅可比常量2
% keys 限制性三体问题|拉格朗日方程|雅可比常量
% license Xiao
% type Tutor

\begin{issues}
\issueNeedCite
\end{issues}

\pentry{限制性三体问题\nref{nod_TriLim},拉格朗日方程\nref{nod_Lagrng}}{nod_a920}

\footnote{本文参考 Wikipedia \href{https://en.wikipedia.org/wiki/Jacobi_integral}{相关页面}。}在 “\enref{限制性三体问题}{TriLim}” 中,我们推导了第三天体 C 的动能、势能、角动量表达式。 这里对限制性三体问题进一步特殊化,假设原二体系统围绕质心作圆周运动,即大质量天体A和B的距离 $R$ 在运动过程中不变,则在旋转坐标系中,A、B是相对静止的。并且天体A、B将作匀速圆周运动(参考“轨道参数-时间变量\upref{OribP}”中关于圆轨道的讨论),即角速度 $\omega$ 也恒定不变。


\subsection{动力学方程}
三体系统在空间中自由度\upref{DoF}为9,但由于天体A和B的运动模式已知,故限制性三体系统的自由度为3,可用质点C在旋转坐标系中的广义坐标 $x$,$y$,$z$ 描述。将质点C的动能 $T$(\autoref{eq_TriLim_3}~\upref{TriLim})和势能 $U$(\autoref{eq_TriLim_4}~\upref{TriLim})代入拉格朗日方程\upref{Lagrng})
\begin{equation}%1
\dv{t} \pdv{(T-U)}{\dot q_i} = \pdv{(T-U)}{q_i}
\quad (q_i\; \rightarrow \; x,y,z)~.
\end{equation}
可得质点 $C$ 的动力学微分方程组
\begin{equation}\label{eq_JConst_2}
-\omega^2 x - 2\omega\dot{y} + \ddot{x} =- \pdv{U}{x}=-\frac{\mu GM}{r_1^3}[x+(1-\mu)R]-\frac{(1-\mu)GM}{r_2^3}(x-\mu R)~,
\end{equation}
\begin{equation}\label{eq_JConst_3}
-\omega^2 y +2\omega\dot{y} + \ddot{y} =- \pdv{U}{y}=-\frac{\mu GM}{r_1^3}y-\frac{(1-\mu)GM}{r_2^3}y~,
\end{equation}
\begin{equation}\label{eq_JConst_4}
\ddot{z} =- \pdv{U}{z} =-\frac{\mu GM}{r_1^3}z-\frac{(1-\mu)GM}{r_2^3}z~.
\end{equation}

\subsection{雅可比常量}
由于质点C的角动量在惯性坐标轴上的分量和机械能分别守恒,故这些物理量的线性组合也是守恒量。为统一量纲,可以令角动量分量乘以角速度。在所有这些线性组合中,有一种组合具有实际的物理意义,记这个组合量为 $E$ 
\begin{equation}%5
\begin{aligned}
E &=(T+U)-\omega L_z\\
&=\frac{1}{2}m\qty(\dot{x}^2-\dot{y}^2+\dot{z}^2)-\frac{1}{2}m\omega^2\qty(x^2+y^2)-\frac{\mu GMm}{r_1}-\frac{(1-\mu)GMm}{r_2}~.
\end{aligned}
\end{equation}
物理量 $E$ 具有能量的量纲。其中 $\frac{1}{2}m\qty(\dot{x}^2-\dot{y}^2+\dot{z}^2)$ 是质点C相对旋转坐标系的动能,$E$ 的其余部分只与位置相关,故可解释为势能。在势能部分,引力势能 $-\mu GMm/r_1-(1-\mu)GMm/r_2$ 与坐标系的选取无关,而 $-\frac{1}{2}m\omega^2\qty(x^2+y^2)$ 是由于坐标系 $xOy$ 的旋转引起的,可将其解释为质点C所受的向心力由于坐标轴自身的旋转而产生的势能。守恒量 $E$ 称为\textbf{雅可比常量},可看作是第三天体相对于旋转坐标系具有的能量。

\subsubsection{推导}
将\autoref{eq_JConst_2} 乘以 $\dot{x}$,\autoref{eq_JConst_3} 乘以 $\dot{y}$,\autoref{eq_JConst_4} 乘以 $\dot{z}$,再将三个式子相加,可得
\begin{equation}\label{eq_JConst_6} %6
\begin{aligned}
&\dot{x} \ddot{x}+\dot{y}\ddot{y}+\dot{z}\ddot{z}-\omega^2\qty(x\dot{x}+y\dot{y})\\
&=-\frac{\mu GM}{r_1^3}[x\dot{x}+(1-\mu)R\dot{x}+y\dot{y}+z\dot{z}]-\frac{(1-\mu)GM}{r_2^3}(x\dot{x}-\mu R\dot{x}+y\dot{y}+z\dot{z})~.
\end{aligned}
\end{equation}
其中,注意到
\begin{equation}
\dot{x}\ddot{x}+\dot{y}\ddot{y}+\dot{z}\ddot{z} =\frac{1}{2}\dv{t}\qty(\dot{x}^2+\dot{y}^2+\dot{z}^2)~,
\end{equation}
\begin{equation}
\omega^2\qty(x\dot{x}+y\dot{y}) =\frac{1}{2}\omega^2\dv{t}\qty(x^2+y^2)~,
\end{equation}
\begin{equation}
-\frac{\mu GM}{r_1^3}[x\dot{x}+(1-\mu)R\dot{x}+y\dot{y}+z\dot{z}] = \dv{t}\qty(\frac{\mu GM}{r_1})~,
\end{equation}
\begin{equation}
-\frac{(1-\mu)GM}{r_2^3}(x\dot{x}-\mu R\dot{x}+y\dot{y}+z\dot{z}) = \dv{t}\qty(\frac{(1-\mu) GM}{r_2})~.
\end{equation}
将以上四式代入\autoref{eq_JConst_6} 可得
\begin{equation}%11
\frac{1}{2}\dv{t}\qty(\dot{x}^2+\dot{y}^2+\dot{z}^2)-\frac{\omega^2}{2}\dv{t}\qty(x^2+y^2)-\dv{t}\frac{\mu GM}{r_1}-\dv{t}\frac{(1-\mu) GM}{r_2}=0~.
\end{equation}
于是
\begin{equation}\label{eq_JConst_12} %12
\frac{1}{2}\qty(\dot{x}^2+\dot{y}^2+\dot{z}^2)-\frac{1}{2}\omega^2\qty(x^2+y^2)-\frac{\mu GM}{r_1}-\frac{(1-\mu) GM}{r_2}=Const~,
\end{equation}
\autoref{eq_JConst_12} 就是单位质量的\textbf{雅可比常量(Jacobi constant)}。
