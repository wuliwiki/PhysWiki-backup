% 亚历山大·格罗滕迪克(综述)
% license CCBYSA3
% type Wiki

本文根据 CC-BY-SA 协议转载翻译自维基百科\href{https://en.wikipedia.org/wiki/Alexander_Grothendieck#Mathematical_work}{相关文章}。

\begin{figure}[ht]
\centering
\includegraphics[width=6cm]{./figures/a1640137f4bbaf24.png}
\caption{1970年,亚历山大·格罗滕迪克在蒙特利尔。} \label{fig_AlGr_1}
\end{figure}
亚历山大·格罗滕迪克(后来的法语名为亚历克斯·格罗滕迪克,发音:/ˈɡroʊtəndiːk/;德语发音:[ˌalɛˈksandɐ ˈɡʁoːtn̩ˌdiːk] ⓘ;法语发音:[ɡʁɔtɛndik]),1928年3月28日出生,2014年11月13日去世,是一位出生于德国的法国数学家,他在现代代数几何的创立中成为了主要人物。他的研究拓展了该领域的范围,并将交换代数、同调代数、层理论和范畴理论等元素融入了其基础中,而他所谓的“相对”视角则在纯数学的许多领域带来了革命性的进展。许多人认为他是二十世纪最伟大的数学家。

格罗滕迪克于1949年开始了他富有成效且公开的数学家生涯。1958年,他被任命为高等科学研究院(IHÉS)的研究教授,并一直在那里工作,直到1970年,由于个人和政治信念,他因与军事资金的争执而离开。1966年,他因在代数几何、同调代数和K理论方面的突破而获得了菲尔兹奖。他后来成为蒙彼利埃大学的教授,并在继续进行相关数学研究的同时,逐渐退出了数学界,投身于政治和宗教事务(最初是佛教,后来转向更为天主教的基督教观点)。1991年,他搬到了位于比利牛斯山脉的法国小村庄拉塞尔,在那里他过上了隐居生活,仍然致力于数学及其哲学和宗教思想,直至2014年去世。
\subsection{生平}  
\subsubsection{家庭与童年}  
格罗滕迪克出生于柏林,父母为无政府主义者。他的父亲亚历山大·“萨沙”·沙皮罗(也叫亚历山大·塔纳罗夫)有哈西德犹太血统,曾在俄罗斯被囚禁,后于1922年移居德国;他的母亲约翰娜·“汉卡”·格罗滕迪克来自汉堡的一个新教德国家庭,并且是一名记者。[a] 两位父母在青少年时期都脱离了他们的早期背景。[16] 在格罗滕迪克出生时,他的母亲与记者约翰内斯·拉达茨结婚,最初,他的出生名字被记录为“亚历山大·拉达茨”。该婚姻在1929年解除了,沙皮罗承认了自己的父亲身份,但并未与汉卡·格罗滕迪克结婚。[16] 格罗滕迪克有一位母亲那边的兄弟姐妹——同父异母的妹妹麦迪。

格罗滕迪克与父母一起生活在柏林,直到1933年底,父亲为了躲避纳粹主义而搬到巴黎,母亲随之而后。格罗滕迪克被交由威廉·海多恩照料,海多恩是一位路德教牧师和汉堡的教师。[17][18] 据温弗里德·沙尔劳称,在此期间,格罗滕迪克的父母作为非战斗辅助人员参与了西班牙内战。[19][20] 然而,也有其他人表示沙皮罗曾在无政府主义民兵中作战。[21]
\subsubsection{第二次世界大战}  
1939年5月,格罗滕迪克被从汉堡送上前往法国的火车。不久后,他的父亲被关押在勒维尔内(Le Vernet)集中营。[22] 他和母亲随后在1940年至1942年间作为“危险的外国人”被关押在不同的集中营。[23] 第一个营地是里厄克罗斯营地(Rieucros Camp),在那里,他的母亲感染了结核病,这种疾病最终导致她在1957年去世。在那里,格罗滕迪克设法上了当地的学校——孟德尔学校(Mendel)。有一次,他设法从营地逃脱,打算刺杀希特勒。[22] 后来,他的母亲汉卡被转移到居尔斯集中营,直到第二次世界大战结束。[22] 格罗滕迪克被允许与母亲分开生活。[24]

在勒尚邦-sur-Lignon村,他在当地的寄宿家庭或旅馆中得到了庇护和隐藏,尽管有时他必须在纳粹突袭期间躲进树林里,几天没有食物和水也能活下来。[22][24]

他的父亲在维希政府的反犹法令下被逮捕,并被送到德朗西集中营,随后由维希政府交给德国人,被送往奥斯维辛集中营,在1942年被杀害。[8][25]

在勒尚邦,格罗滕迪克就读于塞文学院(Collège Cévenol,现在被称为Le Collège-Lycée Cévenol International),这是一所由当地新教和平主义者和反战活动家于1938年创办的独特中学。许多在勒尚邦藏匿的难民儿童都在塞文学院就读,正是在这所学校里,格罗滕迪克显然第一次对数学产生了浓厚的兴趣。[26]

1990年,因冒着生命危险拯救犹太人,整个村庄被认定为“国际义人”(Righteous Among the Nations)。
\subsubsection{学业与接触研究数学}  
战争结束后,年轻的格罗滕迪克在法国学习数学,最初在蒙彼利埃大学,起初他的表现不佳,甚至在天文学等课程上不及格。[27] 他开始独立学习,重新发现了勒贝格测度。在那里经过三年的越来越独立的学习后,他于1948年继续前往巴黎深造。[17]

最初,格罗滕迪克参加了亨利·卡尔坦(Henri Cartan)在巴黎高等师范学校(École Normale Supérieure)的研讨会,但由于缺乏必要的背景,他无法跟上这个高水平研讨会的进度。在卡尔坦和安德烈·韦伊(André Weil)的建议下,他转到南锡大学(University of Nancy),那里有两位领先的专家在研究格罗滕迪克感兴趣的领域——拓扑向量空间:让·迪厄多内(Jean Dieudonné)和洛朗·施瓦茨(Laurent Schwartz)。后者刚刚获得了菲尔兹奖。迪厄多内和施瓦茨向这位新来的学生展示了他们最新的论文《空间(F)与(LF)中的对偶性》(La dualité dans les espaces (F) et (LF));论文末尾列出了14个开放问题,涉及局部凸空间。[28] 格罗滕迪克引入了新的数学方法,使他能够在几个月内解决所有这些问题。[29][30][31][32][33][34][35]

在南锡,他在这两位教授的指导下写了他的博士论文,内容是关于泛函分析,从1950年到1953年。[36] 在此期间,他成为了拓扑向量空间理论的领先专家。[37] 1953年,他移居巴西圣保罗大学,凭借难民护照(Nansen护照),因为他拒绝加入法国国籍(因为那样会要求他服兵役,而这与他的信念相悖)。他在圣保罗呆到1954年底(除了1953年10月到1954年3月期间在法国的长时间访问)。他在巴西期间发表的工作仍然集中在拓扑向量空间理论;正是在那里,他完成了关于这一主题的最后一项重要工作——关于巴拿赫空间的“度量”理论。

格罗滕迪克于1955年初搬到了堪萨斯州的劳伦斯,并在那里将他以前的研究主题放在一边,开始从事代数拓扑、同调代数,并逐渐转向代数几何。[38][39] 正是在劳伦斯,格罗滕迪克发展了他的阿贝尔范畴理论,并基于此重新构建了层的同调学,说法最终导致了具有深远影响的《东北论文》("Tôhoku paper")。[40]

1957年,他受到奥斯卡·扎里斯基(Oscar Zariski)的邀请,前往哈佛大学访问,但因为拒绝签署承诺不参与推翻美国政府的声明,这个邀请未能成行。扎里斯基警告他说,这样的拒绝可能会让他入狱。然而,入狱的前景并没有让他担心,只要他能够接触到书籍。[41]

回顾格罗滕迪克在南锡时期的情况,与当时从巴黎高等师范学校(École Normale Supérieure)训练出来的学生(如皮埃尔·萨缪尔、罗杰·戈德曼、勒内·托姆、雅克·迪克斯米尔、让·塞尔夫、伊冯娜·布吕哈、让-皮埃尔·塞尔和伯纳德·马尔格朗日)相比,莱拉·施内普斯(Leila Schneps)说:

“他对这一群人及其教授几乎完全陌生,来自如此贫困和混乱的背景,相较于他们,刚开始时他的知识如此贫乏,然而他那种闪电般的升迁至突然成名的过程是如此不可思议;在数学史上,这种现象是独一无二的。”[42]

他在1953年关于拓扑向量空间的早期工作已成功应用于物理学和计算机科学,最终在量子物理学中形成了格罗滕迪克不等式与爱因斯坦—波多尔斯基—罗森悖论之间的关系。[43]
\subsubsection{IHÉS 时代}  
1958年,格罗滕迪克被聘任到高等科学研究院(Institut des hautes études scientifiques,简称IHÉS),这是一所由私人资助的新研究机构,实际上是为了让让·迪厄多内和格罗滕迪克能够工作而创建的。[3] 格罗滕迪克通过在那里举办一系列密集且高产的研讨会吸引了大量关注(这些研讨会实际上是一些工作小组,汇聚了法国及其他年轻一代最有才华的数学家,进行基础性工作)。[17] 格罗滕迪克几乎停止了通过传统的学术期刊发表论文的方式。然而,他仍然能够在大约十年的时间里在数学领域发挥主导作用,培养出强大的数学流派。[44]

在这段时间内,他的学生包括米歇尔·德马兹尔(Michel Demazure,研究SGA3中的群方案)、莫妮克·哈基姆(Monique Hakim,研究相对方案和分类拓扑)、吕克·伊吕斯(Luc Illusie,研究余切复合体)、米歇尔·雷诺(Michel Raynaud)、米歇尔·雷诺(Michele Raynaud)、让-路易·费尔迪耶(Jean-Louis Verdier,导出范畴理论的共同创立者)和皮埃尔·德林热(Pierre Deligne)。在SGA项目中的合作伙伴还包括迈克尔·阿尔廷(Michael Artin,研究étale同调)、尼克·卡茨(Nick Katz,研究单调性理论和Lefschetz铅笔)。让·吉罗(Jean Giraud)在这里也研究了非阿贝尔同调的扭束理论扩展。许多其他数学家,如大卫·穆姆福德(David Mumford)、罗宾·哈特肖恩(Robin Hartshorne)、巴里·马祖尔(Barry Mazur)和C.P. 拉马努贾姆(C.P. Ramanujam)等,也参与了这些工作。
\subsubsection{“黄金时代”}  
在IHÉS的“黄金时代”期间,亚历山大·格罗滕迪克的工作奠定了代数几何、数论、拓扑学、范畴理论和复分析等多个领域的统一主题。[36] 他在代数几何方面的首次(IHÉS之前的)发现是格罗滕迪克–赫尔茨布鲁赫–黎曼–罗赫定理,这是对赫尔茨布鲁赫–黎曼–罗赫定理的代数推广;在这个背景下,他还引入了K理论。随后,按照他在1958年国际数学家大会上的讲话中所概述的计划,他引入了方案理论,并在其《代数几何元素》(Éléments de géométrie algébrique,简称EGA)中详细展开,提供了更为灵活和一般化的代数几何基础,自那时以来,该基础已被该领域采纳。[17] 他接着引入了方案的étale同调理论,提供了证明魏尔猜想的关键工具,并补充了结晶同调和代数德拉姆同调理论。与这些同调理论紧密相关,他提出了拓扑理论的拓扑范畴(topos theory)作为拓扑学的一种推广(在范畴逻辑中也具有相关性)。他还通过范畴化的伽罗瓦理论,提供了方案基本群的代数定义,从而诞生了如今闻名的étale基本群,随后他猜测其进一步推广的存在性,这一推广现已被称为基本群方案(fundamental group scheme)。作为他一致性对偶理论的框架,他还引入了导出范畴,该范畴后来由费尔迪耶(Verdier)进一步发展。[45]

他在这些及其他课题上的研究成果,既在《EGA》一书中发布,也以较为粗糙的形式发表在他在IHÉS主办的《代数几何研讨会讲义》(Séminaire de géométrie algébrique,简称SGA)中。[17]
\subsubsection{政治激进主义}  
格罗滕迪克的政治观点激进且和平主义。他坚决反对美国干预越南战争和苏联的军事扩张。为了抗议越南战争,他在河内周围的森林中讲授范畴理论,而当时河内正遭受轰炸。[46] 1966年,他拒绝出席在莫斯科举行的国际数学家大会(ICM),在那里他原本将获得菲尔兹奖。[7] 约在1970年,格罗滕迪克因发现IHÉS部分资金来自军事而从科学界退休。[47] 几年后,他作为蒙彼利埃大学的教授重返学术界。

尽管军事资金问题或许是格罗滕迪克离开IHÉS的最明显原因,但认识他的人表示,导致这一断裂的原因更为深刻。皮埃尔·卡尔捷(Pierre Cartier),IHÉS的长期访客,在IHÉS四十周年纪念专刊中写了一篇关于格罗滕迪克的文章。[48] 在那篇文章中,卡尔捷指出,作为一位反军事的无政府主义者的儿子,以及在被剥夺权利的环境中成长的人,格罗滕迪克始终对贫困和被压迫者抱有深切的同情。卡尔捷形容,格罗滕迪克开始觉得布尔河(Bures-sur-Yvette)像是“一个金笼子”(une cage dorée)。当格罗滕迪克在IHÉS时,反对越南战争的情绪愈发激烈,卡尔捷认为这也加剧了格罗滕迪克对成为科学界权威人物的厌恶。[3] 此外,在IHÉS待了几年后,格罗滕迪克似乎开始寻找新的智识兴趣。到1960年代末,他开始对数学以外的科学领域产生兴趣。1964年加入IHÉS的物理学家大卫·鲁埃尔(David Ruelle)表示,格罗滕迪克曾几次找他谈论物理学。[b] 生物学比物理学更吸引格罗滕迪克,他还组织了一些关于生物学话题的研讨会。[48]

1970年,格罗滕迪克与另外两位数学家克劳德·谢瓦利耶(Claude Chevalley)和皮埃尔·萨缪尔(Pierre Samuel)一起创建了一个名为“Survivre”的政治团体——这个名字后来改为“Survivre et vivre”。该团体发布了一个公报,致力于反军事和生态问题,并对科学和技术的滥用提出了强烈批评。[49] 格罗滕迪克将接下来的三年时间投入到这个团体中,并担任其公报的主编。[1]

尽管格罗滕迪克继续进行数学研究,但他的标准数学生涯在离开IHÉS后大致结束。[8] 离开IHÉS后,格罗滕迪克成为法国高等研究院(Collège de France)的临时教授,任期两年。[49] 随后,他成为蒙彼利埃大学的教授,但与数学界的关系日益疏远。他于1988年正式退休,此前他曾接受过法国国家科学研究中心(CNRS)的一项研究职位。[1]
\subsubsection{1980年代的手稿}  
尽管在1980年代格罗滕迪克没有以传统方式发布数学研究,但他编写了几本有影响力的手稿,这些手稿内容既有数学也有传记性质,并且传播范围有限。

《La Longue Marche à travers la théorie de Galois》(《通过伽罗瓦理论的长征》)是他在1980年和1981年之间写成的一本1600页的手写手稿,包含了许多导致《Esquisse d'un programme》产生的思想。[50] 其中还包括了对泰奇穆勒理论的研究。

1983年,受班戈大学的罗纳德·布朗(Ronald Brown)和蒂姆·波特(Tim Porter)的通信启发,格罗滕迪克写了一本600页的手稿,名为《Pursuing Stacks》(《追寻层叠》)。这本手稿以一封写给丹尼尔·奎伦(Daniel Quillen)的信开始。该信件和随后的部分从班戈大学分发出去(见下面的外部链接)。在这些内容中,格罗滕迪克以一种非正式、类似日记的方式,解释并发展了他关于代数同伦理论与代数几何之间关系的想法,以及对非交换层叠理论的展望。该手稿由G. Maltsiniotis编辑出版,后来成为他另一部宏伟著作《Les Dérivateurs》的基础。该书大约有2000页,写于1991年,进一步发展了《Pursuing Stacks》中的同伦思想。[7] 这项工作在很大程度上预见了1990年代中期Fabien Morel和Vladimir Voevodsky提出的动机同伦理论的发展。

1984年,格罗滕迪克为法国国家科学研究中心(CNRS)的一项职位撰写了提案《Esquisse d'un Programme》(《一个计划的草图》)。[50] 该提案描述了研究复曲线模空间的新想法。尽管格罗滕迪克没有在这一领域发表过他的研究成果,但这一提案激励了其他数学家开展该领域的工作,成为儿童图形理论和无象几何的来源。后来,这一提案被以两卷本形式出版,题为《几何伽罗瓦作用》(Geometric Galois Actions)(剑桥大学出版社,1997年)。

在此期间,格罗滕迪克还同意发布他的一些草稿,内容涉及EGA中的贝尔蒂尼类型定理(EGA V,1992-1993年在《Ulam Quarterly》上发表,后来于2004年在格罗滕迪克圆网站上发布)。

在他那本广泛的自传性著作《Récoltes et Semailles》(《收获与播种》,1986年)中,格罗滕迪克描述了他对数学的看法以及他在数学社区的经历。他回顾到,这个社区最初以开放和欢迎的态度接纳了他,但他渐渐感到,这个社区被竞争和地位所主导。他抱怨他所认为的“埋葬”他的工作,并且在他离开社区后,曾经的学生和同事们背叛了他。《Récoltes et Semailles》最终于2022年由Gallimard出版,[51] 通过法国科学历史学家阿兰·赫雷曼(Alain Herreman)的帮助,[7] 也可在互联网上获取。[52] Leila Schneps的英文翻译将由MIT Press于2025年出版。[53] 部分英文翻译可以在互联网上找到。[54] 一本完整的日文翻译版由格罗滕迪克的朋友、Survivre时期的伙伴辻裕一(Tsuji Yuichi,1938–2002)完成。前三卷(对应本书的第0至III部分)在1989年至1993年间出版,而第四卷(Part IV)已经完成,尽管尚未出版,但作为打字手稿的复印本已被流传。格罗滕迪克为翻译工作提供了帮助,并为其写了序言,称辻裕一为他的“第一位真正的合作者”。[55][56][57][58][59][60] 《Récoltes et Semailles》的部分内容已被翻译成西班牙语,[61] 也有俄文翻译版本,并已在莫斯科出版。[62]

1988年,格罗滕迪克拒绝了克拉福德奖,并通过公开信向媒体解释。他写道,他和其他资深数学家不需要额外的经济支持,并批评了他认为科学共同体伦理的衰退,尤其是他认为科学盗窃已经变得普遍并被容忍。信中还表达了他认为在本世纪结束之前将发生完全无法预见的事件,从而导致文明前所未有的崩溃。然而,格罗滕迪克补充说,他的观点“绝不是对皇家科学院在其基金管理上的目标进行批评”,并且他还表示:“我遗憾地认为我拒绝接受克拉福德奖可能给您和皇家科学院带来了不便。”[63]

《La Clef des Songes》是格罗滕迪克于1987年写成的一部315页手稿,内容讲述了他如何通过对梦境来源的思考,得出神明存在的结论。[65] 作为这份手稿的注释的一部分,格罗滕迪克描述了18位“变种人”的生活和工作,这些人是他崇拜的先知性人物,远远超越了他们的时代,预示着一个新时代的到来。[1] 他名单上的唯一一位数学家是伯恩哈德·黎曼。[66] 受天主教神秘主义者玛尔特·罗宾的影响(据称她仅依靠圣体圣血生存),格罗滕迪克在1988年几乎饿死。[1] 他对精神问题的日益关注也在1990年1月写给250位朋友的信《Lettre de la Bonne Nouvelle》中体现。在信中,他描述了与神明的相遇,并宣布一个“新时代”将于1996年10月14日开始。[7]

《格罗滕迪克纪念文集》于1990年出版,是一套三卷本的研究论文集,以纪念他1988年60岁生日。[67]

格罗滕迪克的数学及其他写作超过2万页,现存于蒙彼利埃大学并仍未出版。[68] 这些文献已被数字化保存,并通过蒙彼利埃亚历山大·格罗滕迪克研究所门户网站开放访问。[69][70]
\subsubsection{退休与隐居及去世}
1991年,格罗滕迪克搬到了一个新地址,并未与他在数学界的老朋友们分享这个地址。[1] 此后,很少有人去拜访他。[71] 当他尝试以蒲公英汤为主食时,当地村民为他提供了更多样化的饮食,以帮助维持他的生计。[72] 在某个时刻,莱拉·施内普斯和皮埃尔·洛查克 [fr] 找到了他,并与他进行了简短的通信。因此,他们成为“最后与他接触的数学界成员之一”。[73] 在他去世后,透露出他独自居住在位于阿列日(Ariège)的小村庄拉塞尔(Lasserre)的一座房子里,这里位于比利牛斯山脚下。[74]

2010年1月,格罗滕迪克写了一封名为《Déclaration d'intention de non-publication》的信给卢克·伊吕西(Luc Illusie),声称所有在他缺席期间出版的材料均未经他许可。他要求不允许任何人转载他的作品,无论是全部还是部分,并要求将这些作品从图书馆中撤回。[75] 他把一个专门致力于他工作的网页称为“亵渎”。[76] 他的要求可能在2010年发生了改变。[77]

2014年9月,几乎完全失聪和失明的他请求邻居为他购买一把左轮手枪,好让他自杀。[78] 2014年11月13日,格罗滕迪克在阿列日圣利齐医院(Saint-Lizier)或圣吉龙医院(Saint-Girons)去世,享年86岁。[26][79]
\subsubsection{国籍}
格罗滕迪克出生于魏玛德国。1938年,10岁的他作为难民迁移到法国。由于1945年纳粹德国的覆灭,他的国籍记录被销毁,并且他在战后并未申请法国国籍。因此,他至少在大部分职业生涯中是一个无国籍人士,使用的是南森护照。[4][5][6] 他不愿持有法国国籍的部分原因被认为是因为他不愿服役于法国军队,特别是由于阿尔及利亚战争(1954–62)期间的情况。[3][6][15] 他最终在1980年代初申请了法国国籍,那时他已经远超过需要服兵役的年龄。[3]
\subsubsection{家庭}
格罗滕迪克与他的母亲关系非常亲密,他将自己的论文献给了她。她于1957年因在难民营中感染结核病去世。[49]

他有五个孩子:一个是与他在南锡期间的房东女儿所生;[3] 三个是与妻子米雷伊·迪富尔所生,分别是约翰娜(1959年)、亚历山大(1961年)和马修(1965年);[1][41] 还有一个是与贾斯汀·斯卡尔巴所生,他在1970年代初期与她一起生活在一个公社中。[1]
\subsection{数学工作}
格罗滕迪克早期的数学工作集中在泛函分析领域。在1949年至1953年期间,他在南锡大学攻读博士学位,导师是让·迪厄多内和洛朗·施瓦茨。他的关键贡献包括拓扑向量空间的拓扑张量积、作为施瓦茨分布基础的核空间理论,以及Lp空间在研究拓扑向量空间之间线性映射中的应用。在短短几年内,他便成为这一领域的权威,甚至让·迪厄多内将他在这一领域的影响力与巴拿赫相提并论。[80]

然而,格罗滕迪克最重要和最具影响力的工作是在代数几何及相关领域。从大约1955年开始,他着手研究层理论和同调代数,创作了具有深远影响的《东北论文》(Sur quelques points d'algèbre homologique,1957年发表在《东北数学杂志》上),在该论文中他引入了阿贝尔范畴,并应用其理论证明了层上同调可以作为某些导出函子的定义。[17]

同调方法和层理论早已由让-皮埃尔·塞尔和其他数学家引入代数几何,层的概念由让·勒雷(Jean Leray)定义。格罗滕迪克将这些方法提升到更高的抽象层次,并将其转化为他理论的一个关键组织原则。他将关注点从研究个别代数簇转向了相对视角(由态射联系的代数簇对),这使得许多经典定理能够得到广泛的推广。[49] 首个主要应用是塞尔定理的相对版本,表明一个完全代数簇上的连通层的同调是有限维的;格罗滕迪克的定理进一步表明,在适当映射下的连通层的高阶直接像是连通的;当映射是单点空间时,这一结果就变成了塞尔定理。

1956年,他将相同的思路应用于黎曼-罗赫定理,该定理近期已被希尔茨布鲁赫推广至任意维度。格罗滕迪克-黎曼-罗赫定理于1957年在第一次数学工作会议(Mathematische Arbeitstagung)上由格罗滕迪克宣布。[49] 此结果以阿尔曼·博雷尔与塞尔的合作论文形式发表。这一成果标志着他在代数几何领域的首次重要贡献。接着,格罗滕迪克开始规划并实施一项重建代数几何基础的计划,而当时代数几何的基础正处于动荡状态,并在克劳德·谢瓦雷的研讨会上讨论中。他在1958年国际数学家大会上的报告中概述了这一计划。

他在代数几何的基础性工作达到了比以前版本更高的抽象层次。他适应了非闭泛点的使用,这最终导致了“方案”(schemes)理论的提出。格罗滕迪克还开创性地系统化使用了“幂零元”(nilpotents)。作为“函数”,这些元素只能取值为零,但它们携带着微小的信息,可以在纯代数环境中使用。他的方案理论因其表达力和技术深度,已成为该领域最为通用的基础。在这一框架下,人们可以以集成的方式使用双有理几何、数论技巧、伽罗瓦理论、交换代数以及代数拓扑方法的近似技术。[17][82][83]

格罗滕迪克以其在数学抽象方法上的精通和在表达与呈现方面的完美主义著称。[44] 1960年后,他的大部分工作并没有通过传统的学术期刊发表,而是最初以研讨会笔记的复印本形式流传;他的影响在很大程度上是个人化的。他的影响扩展到了数学的许多其他领域,例如当代的D-模块理论。尽管被誉为“数学界的爱因斯坦”,他的工作也引发了许多负面反应,许多数学家更倾向于寻找更具具体性的问题和领域。[84][85]
\subsubsection{EGA, SGA, FGA}
格罗滕迪克的主要已出版作品收录在他的宏大但不完整的《代数几何元素》(Éléments de géométrie algébrique,简称EGA)和《代数几何研讨会》(Séminaire de géométrie algébrique,简称SGA)中。 《代数几何基础》(Fondements de la Géométrie Algébrique,简称FGA)集合了在Bourbaki研讨会中发表的讲座,也包含了重要的材料。[17]

格罗滕迪克的工作包括发明了étale同调和l-adic同调理论,这些理论解释了安德烈·维尔(André Weil)提出的一个观察,即论证了一个代数簇的拓扑特征与其丢番图(数论)性质之间的联系。[49] 例如,一个方程在有限域上的解的数量反映了其在复数域上解的拓扑性质。维尔曾意识到,为了证明这种联系,需要一种新的同调理论,但直到格罗滕迪克提出这一理论时,无论是他自己还是任何其他专家都没有找到实现这一理论的方法。

这一理论最终导致了维尔猜想的证明,最后一个猜想由格罗滕迪克的学生皮埃尔·德里希(Pierre Deligne)在1970年代初解决,此时格罗滕迪克已基本退出数学界。[17]
\subsubsection{主要数学贡献}
在Grothendieck的回顾性作品《收获与播种》(*Récoltes et Semailles*)中,他列出了十二个他认为属于“伟大思想”的贡献。按时间顺序,这些贡献包括:
\begin{enumerate}
\item 拓扑张量积和核空间
\item “连续”和“离散”对偶性(导出范畴,“六个操作”)
\item Grothendieck–Riemann–Roch定理的K理论与交点理论的关系
\item 方案(Schemes)
\item 拓扑(Topoi)
\item Étale同调和l-adi同调
\item 动机和动机Galois群(Grothendieck ⊗-范畴)
\item 晶体与晶体同调,“de Rham系数”,“Hodge系数”的“瑜伽”
\item “拓扑代数”:∞-堆栈(∞-stacks),导出者(derivators);拓扑的同调形式主义作为新同伦代数的灵感来源
\item 温和拓扑(Tame topology)
\item Anabelian代数几何的“瑜伽”,Galois–Teichmüller理论
\item 规律多面体和各种规律配置的“示意图”或“算术”视角
\end{enumerate}
这里,“瑜伽”一词指的是一种“元理论”,可以作为启发式的方法来使用;Michel Raynaud将其他术语称为“Ariadne的线索”和“哲学”,作为有效的等效词。[87]

Grothendieck 写道,在这些主题中,规模最大的是拓扑(topoi),因为它们综合了代数几何、拓扑学和算术。最为深入发展的主题是方案(schemes),它们是“典型的”框架,用于支持其他八个主题(除了第1、5和12)。Grothendieck 表示,第一个和最后一个主题,即拓扑张量积和规律配置,比其他主题的规模要小。拓扑张量积更多地充当了一个工具的角色,而不是进一步发展的灵感来源;但他预计,规律配置这一主题在任何一个数学家的一生中都无法被完全研究透彻。他认为,最深刻的主题是动机、无可交换几何(anabelian geometry)和Galois–Teichmüller理论。[88]
\subsection{影响}  
Grothendieck 被许多人认为是20世纪最伟大的数学家。[11] 在一篇讣告中,David Mumford 和 John Tate 写道:

尽管20世纪的数学变得越来越抽象和一般化,但正是 Alexander Grothendieck 成为这一趋势的伟大大师。他的独特才能是消除所有不必要的假设,深入到一个领域,直到其内在模式在最抽象的层面显现出来——然后,像魔术师一样,展示如何通过揭示问题的真正本质,轻松解决那些古老的问题。[11]

到了1970年代,Grothendieck 的工作被认为不仅在代数几何及相关领域如层理论和同调代数中具有深远影响,[89] 而且对逻辑学领域,特别是范畴逻辑也有影响。[90]

根据数学家 Ravi Vakil 的说法,“整个数学领域都在使用他所建立的语言。我们生活在他建造的这个大结构里,已经习以为常——但这位建筑师已经不在了。”在同一篇文章中,Colin McLarty 说道:“今天有很多人住在 Grothendieck 的房子里,却没有意识到这是 Grothendieck 的房子。”[71]
\subsubsection{几何学}  
Grothendieck 通过澄清该领域的基础,并开发出旨在证明若干著名猜想的数学工具,来接近代数几何学。代数几何学传统上意味着通过研究几何物体的代数方程来理解几何物体,例如代数曲线和曲面。代数方程的性质反过来又通过环理论的技巧进行研究。在这种方法中,几何物体的性质与其关联环的性质相关。物体所在的空间(例如实数、复数或射影空间)是物体的外部,而环则是物体的内在。

Grothendieck 为代数几何学奠定了新的基础,重点研究内在空间(“谱”)及其相关环。为此,他开发了“方案”的理论,可以非正式地理解为一个拓扑空间,在这个空间中的每个开集上都与一个交换环相关联。方案已成为现代代数几何学研究的基本对象。作为基础的使用,使得几何学能够吸收来自其他领域的技术进展。[91]

他对经典的 Riemann–Roch 定理的推广,将复代数曲线的拓扑性质与它们的代数结构联系起来,现在被称为“Grothendieck–Hirzebruch–Riemann–Roch 定理”。他为证明该定理而开发的工具,启动了代数与拓扑 K-理论的研究,该理论通过将物体与环关联,探索物体的拓扑性质。[92] 在直接接触 Grothendieck 的思想后,拓扑 K-理论由 Michael Atiyah 和 Friedrich Hirzebruch 创立。[93]
\subsubsection{上同调理论}  
Grothendieck 构造的新型上同调理论,利用代数技巧研究拓扑物体,影响了代数数论、代数拓扑和表示论的发展。在这个项目的一部分中,他创造的拓扑理论(topos theory),作为点集拓扑的范畴理论推广,已经影响了集合论和数学逻辑领域。[89]

Weil 猜想在 1940 年代后期作为一组算术几何学中的数学问题被提出。它们描述了代数曲线或高维流形上点的数量的解析不变量,称为局部 zeta 函数的性质。Grothendieck 对 ℓ-adic étale 上同调的发现——作为 Weil 上同调理论的第一个例子——为证明 Weil 猜想开辟了道路,最终在 1970 年代由他的学生 Pierre Deligne 完成。[92] Grothendieck 的大规模方法被称为“远见卓识的计划”。ℓ-adic 上同调随后成为数论学者的基础工具,应用于 Langlands 计划。[95]

Grothendieck 的动机猜想理论旨在成为“ℓ-adic”理论,但不依赖于“ℓ”的选择,即素数。尽管它未能提供预期的路径来证明 Weil 猜想,但它成为了代数 K-理论、动机同伦理论和动机积分等现代发展的基础。[96] 该理论、Daniel Quillen 的工作和 Grothendieck 的 Chern 类理论被视为代数边界理论(algebraic cobordism)的背景,后者是拓扑思想的另一个代数类比。[97]
\subsubsection{范畴论}  
Grothendieck 强调普遍性质在各种数学结构中的作用,使范畴论成为数学中的主流组织原理。范畴论的应用之一是为描述在许多不同数学系统中看到的相似结构和技巧提供了一种共同的语言。[98] 他提出的阿贝尔范畴(abelian category)现在成为同调代数中的基本研究对象。[99] 范畴论作为一门独立的数学学科的出现,已被归因于 Grothendieck 的影响,尽管这并非有意为之。[100]
\subsection{在流行文化中}  
《Lágrimas上校》(英文版为 *Colonel Tears*),是波多黎各-哥斯达黎加作家卡洛斯·丰塞卡(Carlos Fonseca)创作的小说,讲述了 Grothendieck 的故事。[101]  

本杰明·拉巴图(Benjamín Labatut)所著《当我们无法理解世界时》一书 dedicates 了其中一章讨论 Grothendieck 的工作和生活,并通过引用日本数学家望月新一(Shinichi Mochizuki)来引入他的故事。该书是一部轻微虚构的科学探求世界的描写,并曾入围美国国家图书奖。[102]  

在科马克·麦卡锡(Cormac McCarthy)的《乘客》(*The Passenger*)及其续集《斯特拉·马里斯》(*Stella Maris*)中,一位主要角色是 Grothendieck 的学生。[103][104]  

为了纪念他,“Grothendieck 研究所”已成立。[105]
\subsection{出版物}  
\begin{itemize}
\item Grothendieck, Alexander (1955). "Produits Tensoriels Topologiques et Espaces Nucléaires" [拓扑张量积与核空间]. 美国数学学会系列文集(法文版)。16. 普罗维登斯:美国数学学会。ISBN 978-0-8218-1216-7。MR 0075539。OCLC 1315788.  
\item Grothendieck, Alexander (1973). *Topological Vector Spaces*. 由 Orlando Chaljub 翻译。纽约:Gordon and Breach Science Publishers。ISBN 978-0-677-30020-7。OCLC 886098. 
\end{itemize} 
\subsection{另见}  
\begin{itemize}
\item ∞-群体  
\item λ-环  
\item AB5 类  
\item 阿贝尔类别  
\item 可达类别  
\item 代数几何  
\item 代数堆  
\item 逼近性质 – 数学概念  
\item Barsotti–Tate 群  
\item 陈类  
\item 晶体(数学) – 某些纤维范畴的笛卡尔截面  
\item 晶体同调 – 针对基础域 k 上的方案 X 的 Weil 同调理论  
\item Δ-函子  
\item 导算子 – 提出的同调代数框架  
\item 导出类别 – 同调构造  
\item 降(数学)  
\item Dévissage  
\item DF 空间 – 特殊局部凸空间类  
\item 邓福德–佩蒂斯性质  
\item 可消除函子
\item 优良环  
\item 纤维范畴 – 范畴理论中的概念  
\item 形式平滑映射  
\item 基本群体  
\item 基本群方案  
\item 高伦斯坦环 – 交换代数中的局部环  
\item Grothendieck 的 Tôhoku 论文 – 1957 年由 Alexander Grothendieck 撰写的数学论文  
\item K-理论  
\item 希尔伯特方案  
\item 同伦假设  
\item 无穷小同调 – Grothendieck 于 1966 年引入的代数簇的同调理论  
\item 以 Alexander Grothendieck 命名的事物列表  
\item 局部同调– 代数几何中的概念  
\item 中介猜想  
\item 模空间方案 – Grothendieck 方案范畴中的模空间  
\item 动机(代数几何) – 用于统一同调理论的结构  
\item 核算符  
\item 核空间  
\item 准因子局部环  
\item 投影张量积 
\item 适当态射 – 代数几何中的概念,代数簇的适当映射的类比  
\item 追寻堆叠 – 具有开创性的数学著作  
\item 准有限态射 
\item 商方案 
\item 拉马努扬–塞缪尔定理 – 局部环的除法器为主元素的条件  
\item 方案(数学)  
\item 截面猜想  
\item 半稳定阿贝尔簇
\item 层同调 
\item 堆(数学) 
\item 代数周期的标准猜想 
\item 计划的概要 
\item 塔纳基形式主义 
\item 绝对纯度定理 
\item 正规函数定理 
\item 超本质空间  
\item Weil 猜想  
\item 代数曲线上的向量束  
\item 扎里斯基主定理
\end{itemize}
\subsection{注释} 
a.皮埃尔·卡蒂埃的证词表明,格罗滕迪克的母亲是犹太德国血统:“我所知道的关于他的一生,都是来自格罗滕迪克本人”。[15]\\ 
b.吕尔(Ruelle)发明了动力系统中“奇异吸引子”的概念,并与荷兰数学家弗洛里斯·塔肯斯(Floris Takens)一起,在1970年代提出了一个新的湍流模型。
\subsection{参考文献}  
\begin{enumerate}
\item Scharlau 2008.  
\item Cartier 等, 2007, p. 7.  
\item Cartier, 2004.  
\item Douroux, 2012.  
\item Cartier, 2004, p. 10, 注脚 12.  
\item Kleinert, 2007.  
\item Jackson, 2004b.  
\item Bruce Weber; Julie Rehmeyer (2014年11月14日). "Alexander Grothendieck, Math Enigma, Dies at 86". *The New York Times*. 2022年1月1日归档.  
\item Mumford, David; Tate, John (2015). "Alexander Grothendieck (1928–2014) Mathematician who rebuilt algebraic geometry". *Nature*. 517 (7534): 272. Bibcode:2015Natur.517..272M. doi:10.1038/517272a. ISSN 0028-0836. PMID 25592527.  
\item "Guardian obituary". Independent.co.uk. 2014年11月15日.  
\item David Mumford 和 John Tate 发表的 Alexander Grothendieck 讣告。布朗大学与哈佛大学档案馆:再版档案:14 December 2014, *Can one explain schemes to biologists*  
\item "Fields Medals 1966". mathunion.org. 2019年3月22日归档. 于2022年1月5日检索.  
\item Scharlau, Winfried. "Who is Alexander Grothendieck? Anarchy, Mathematics, Spirituality, Solitude" (PDF). 于2022年10月9日归档 (PDF).  
\item Ruelle 2007, p. 40.  
\item Cartier, 2001.  
\item "The Early Background of Genius". 于2011年6月15日归档. 于2011年6月15日检索.  
\item Jackson 2004a.  
\item Philippe Douroux (2019年5月6日). "Trésor scientifique ou vieux papiers illisibles? Les mystérieuses archives d'Alexandre Grothendieck" [科学宝藏还是难以阅读的旧纸张?亚历山大·格罗滕迪克的神秘档案]. *Le Monde* (法语).
\item Scharlau 2008, p. 931.  
\item Scharlau n.d., p. 2: "两者都参与了西班牙内战,但并非积极参战,而是提供支持。"  
\item Hersh & John-Steiner 2011, p. 109.  
\item Amir D. Aczel, *The Artist and the Mathematician*, Basic Books, 2009, pp. 8ff, pp. 8–15.  
\item Piotr Pragacz, 'Notes on the Life and Work of Alexander Grothendieck,' in Piotr Pragacz (ed.), *Topics in Cohomological Studies of Algebraic Varieties: Impanga Lecture Notes*, Springer Science & Business Media, 2006, pp. xi–xxviii, p. xii.  
\item Luca Barbieri Viale, 'Alexander Grothendieck: entusiasmo e creatività,' in C. Bartocci, R. Betti, A. Guerraggio, R. Lucchetti (eds.), *Vite matematiche: Protagonisti del '900, da Hilbert a Wiles*, Springer Science & Business Media, 2007, pp. 237–249, p. 237.  
\item Ruelle 2007, p. 35.  
\item "Alexandre Grothendieck, ou la mort d'un génie qui voulait se faire oublier". *Libération Sciences* (法语). 2014年11月13日。于2014年11月14日检索。  
\item Philippe Douroux (2012年2月8日). "Alexandre Grothendieck: Un voyage à la poursuite des choses évidentes" [亚历山大·格罗滕迪克:一场追求显而易见的事物之旅]. *Images des mathématiques* (法语). CNRS.
\item Jean Dieudonné; Laurent Schwartz (1949). "La dualité dans les espaces (F) et (LF)"。*Annales de l'Institut Fourier*,1: 61–101. ISSN 0373-0956.  
\item Peixoto, Tatiana; Bietenholza, Wolfgang (2016). "纪念亚历山大·格罗滕迪克:一位伟大而神秘的数学天才"。arXiv:1605.08112 [math.HO].  
\item Alexander Grothendieck (1950). "关于局部凸空间对偶的完备性" (PDF)。*巴黎科学院公报*,230: 605–606.  
\item Alexander Grothendieck (1950). "关于空间 F 中对偶的若干结果" (PDF)。*巴黎科学院公报*,230: 1561–1563.  
\item Alexander Grothendieck (1950). "局部凸向量空间中的紧致性一般标准。LF 空间的病理学" (PDF)。*巴黎科学院公报*,231: 940–941.  
\item Alexander Grothendieck (1951). "关于拓扑向量空间的若干结果" (PDF)。*巴黎科学院公报*,233: 839–841.  
\item Alexander Grothendieck (1951). "关于拓扑向量空间的拓扑张量积的概念,以及与此概念相关的一个特殊类的向量空间" (PDF)。*巴黎科学院公报*,233: 1556–1558.  
\item Alexander Grothendieck (1952). "一般功能空间中的紧致性标准" (PDF)。*美国数学杂志*,74 (1): 168–186. doi:10.2307/2372076. JSTOR 2372076.
\item Cartier 等人 2007,《前言》。  
\item Horvâth, John (1976年7月)。《拓扑向量空间,A. Grothendieck 著...》(PDF)。书评。*美国数学学会公报*,82 (4): 515–521. doi:10.1090/S0002-9904-1976-14076-1. 已归档 (PDF) 自原文于2022年10月9日。  
\item Schneps 未注明日期。  
\item Colmez & Serre 2004。  
\item Grothendieck, Alexander (1957),《关于一些同调代数的点》,*东北数学杂志*,第二系列(法文),9 (2): 119–221,doi:10.2748/tmj/1178244839,ISSN 0040-8735,MR 0102537。  
\item Hersh & John-Steiner 2011,第113页。  
\item 《第三章:从学生到名人:1949-1952》(PDF)。*谁是亚历山大·格罗滕迪克:无政府主义、数学、精神性*,第2卷。  
\item Guillaume Aubrun (2020年3月17日)。 "1953年:一个“摘要”的无限发展" [1953: 一个“摘要”与无限的开发]。*数学图像*(法文)。CNRS。  
\item Amir D. Aczel (2009)。*艺术家与数学家*。基础书籍。  
\item Lipman, Joseph (2009)。 "关于导出类别与格罗滕迪克对偶性的笔记"(PDF)。*格罗滕迪克对偶性的基础:方案图的讲义*。数学讲义系列,第1960卷。纽约:Springer-Verlag。第1-259页。doi:10.1007/978-3-540-85420-3。ISBN 978-3-540-85419-7。MR 2490557。已归档 (PDF) 自原文于2022年10月9日。  
\item 《亚历山大·格罗滕迪克的生平与工作》,*美国数学月刊*,第113卷,第9期,脚注6。
\item SGA1, Springer讲义系列 224,第xii, xiii页  
\item Jackson, Allyn (1999年3月)。《IHÉS四十年》(PDF)。*美国数学学会公告*,46 (3): 329–337。  
\item Pragacz 2005。  
\item Alexandre Grothendieck,《计划概述》,英文翻译。  
\item Grothendieck 2022。  
\item Grothendieck, Alexandre。《收获与播种》(PDF)(法文)。2024年9月17日检索。  
\item “出版公告”。2023年1月25日。  
\item 《收获与播种》(PDF)。2024年9月15日检索。  
\item Roy Lisker。“拜访亚历山大·格罗滕迪克”。2022年1月25日检索。  
\item Scharlau, Winfried。“第23章:收获与播种”(PDF)。已归档 (PDF) 自原文于2022年10月9日。2022年1月25日检索。  
\item Grothendieck, Alexander (2015)。*数学家的孤独冒险:进入数学与自我发现之旅*(日文)。由辻雄一翻译(第2版)。京都:现代数学社。  
\item Grothendieck, Alexander (2015)。*数学与裸王:一个梦与数学的埋葬*(日文)。由辻雄一翻译(第2版)。京都:现代数学社。
\end{enumerate}