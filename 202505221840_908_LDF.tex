% 鲁道夫·佩尔斯(综述)
% license CCBYSA3
% type Wiki

本文根据 CC-BY-SA 协议转载翻译自维基百科\href{https://en.wikipedia.org/wiki/Rudolf_Peierls}{相关文章}。

\begin{figure}[ht]
\centering
\includegraphics[width=6cm]{./figures/de58c2e988983035.png}
\caption{佩耶尔斯,摄于1966年} \label{fig_LDF_1}
\end{figure}
鲁道夫·恩斯特·佩耶尔斯爵士,CBE勋衔,英国皇家学会会士(FRS)(/ˈpaɪ.ərlz/;德语:[ˈpaɪɐls];1907年6月5日-1995年9月19日),是一位出生于德国的英国物理学家,在英国核武器计划“管合金”以及其后的“曼哈顿计划”(即盟军联合核弹计划)中发挥了重要作用。《今日物理》于1996年发表的讣告称他是“在核物理爆发进入世界事务的戏剧中扮演关键角色的人物之一”\(^\text{[1]}\)。

佩耶尔斯在柏林大学学习物理学,随后在慕尼黑大学跟随阿诺德·索末菲尔德、在莱比锡大学跟随沃尔夫冈·海森堡、以及在苏黎世联邦理工大学跟随沃尔夫冈·泡利深造。1929年,他获得莱比锡大学的博士学位后,成为泡利在苏黎世的助理。1932年,他获得了洛克菲勒奖学金,并利用这一奖学金在罗马跟随恩里科·费米学习,随后又在剑桥大学的卡文迪许实验室跟随拉尔夫·霍普金森·福勒进行研究。由于其犹太背景,他决定在1933年阿道夫·希特勒上台后不返回德国,而是留在英国,在那里他与汉斯·贝特一起工作,先是在曼彻斯特维多利亚大学,然后在剑桥大学的蒙德实验室。1937年,马克·奥利芬特,新任的伯明翰大学澳大利亚籍物理学教授,邀请他担任应用数学的新职位。

1940年3月,佩耶尔斯与奥托·罗伯特·弗里施共同撰写了《弗里施–佩耶尔斯备忘录》。这篇简短的论文首次指出,仅需少量可裂变的铀-235就可以制造出一枚原子弹。在此之前,人们普遍认为制造这种炸弹需要数吨铀,因此被认为在实际操作上不可行。这篇论文在最初引起英国、随后引起美国政府对核武器的兴趣方面起到了关键作用。佩耶尔斯还促成了他的同胞克劳斯·富克斯加入“管合金”(Tube Alloys,即英国核武器项目)的工作,但由于富克斯于1950年被揭露为苏联间谍,佩耶尔斯也因此一度受到怀疑。

战后,佩耶尔斯回到伯明翰大学工作,直至1963年。随后,他在牛津大学担任物理学怀克姆讲座教授并成为新学院(New College)的院士,直至1974年退休。\(^\text{[2]}\)在伯明翰期间,他的研究涵盖核力、散射、量子场论、原子核的集体运动、输运理论和统计力学,并担任哈威尔原子能研究机构的顾问。他获得了许多荣誉奖项,包括1968年获封爵士。他还著有多部著作,包括《固体的量子理论》(Quantum Theory of Solids)、《自然法则》(The Laws of Nature, 1955)、《理论物理的惊奇》(Surprises in Theoretical Physics, 1979)、《理论物理的更多惊奇》(More Surprises in Theoretical Physics, 1991)以及自传《过客鸟》(Bird of Passage, 1985)。由于对自己曾参与推动的核武器研发心存忧虑,他积极参与《原子科学家公报》的工作,曾担任英国原子科学家协会主席,并积极参与“普格沃什运动”。
