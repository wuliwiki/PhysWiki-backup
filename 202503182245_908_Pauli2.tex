% 沃尔夫冈·泡利(综述)
% license CCBYSA3
% type Wiki

本文根据 CC-BY-SA 协议转载翻译自维基百科\href{https://en.wikipedia.org/wiki/Wolfgang_Pauli}{相关文章}。

\begin{figure}[ht]
\centering
\includegraphics[width=6cm]{./figures/7d0315d83d1af784.png}
\caption{泡利于1945年} \label{fig_Pauli2_1}
\end{figure}
沃尔夫冈·恩斯特·泡利(Wolfgang Ernst Pauli,/ˈpɔːli/,[5] 德语:[ˈvɔlfɡaŋ ˈpaʊli];1900年4月25日—1958年12月15日)是一位奥地利物理学家,量子力学的先驱。1945年,在阿尔伯特·爱因斯坦的提名下,[6] 泡利因其“通过发现一条新的自然法则——泡利不相容原理(Pauli Exclusion Principle)所做出的决定性贡献”而获得诺贝尔物理学奖。这一发现涉及自旋理论,该理论是物质结构理论的基础。为了保持\(\beta\)衰变中的能量守恒,他提出了一种质量极小、电中性的粒子,其后被恩里科·费米命名为中微子。该粒子最终于1956年被探测到。

泡利就读于维也纳的多布林格文理中学,并于1918 年以优异成绩毕业。两个月后,他发表了第一篇论文,内容涉及阿尔伯特·爱因斯坦的广义相对论。随后,他进入慕尼黑大学学习,并在阿诺德·索末菲指导下开展研究。[1]1921年7月,他因其关于电离双原子氢(\(H_2^+\))的量子理论的论文获得博士学位。[2][9]
\begin{figure}[ht]
\centering
\includegraphics[width=6cm]{./figures/5713325387bd5c08.png}
\caption{沃尔夫冈·泡利用于讲授(1929年)} \label{fig_Pauli2_2}
\end{figure}
