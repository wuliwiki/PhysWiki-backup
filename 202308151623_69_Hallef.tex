% 霍尔效应
% keys 霍尔效应|电势差|Hall|磁场|霍尔电势差
% license Xiao
% type Tutor

\begin{issues}
\issueDraft
\end{issues}

\footnote{参考 Wikipedia \href{https://en.wikipedia.org/wiki/Hall_effect}{相关页面}。}1879 年霍尔(E. C. Hall)首先观察到,把一载流导体薄片放在磁场中时,如果磁场方向垂直于薄片平面,则在薄片的上、下两侧面会出现微弱的电势差。这一现象称为\textbf{霍尔效应(Hall effect)}。此电势差称为\textbf{霍尔电势差}。实验测定,霍尔电势差的大小与电流 $I$ 及磁感应强度 $B$ 成正比,而与薄片沿 $\mathbf B$ 方向的厚度 $d$ 成反比。它们的关系可写成:
\begin{equation}
V = R_{\mathrm{H}} \frac{I B}{d}~.
\end{equation}
其中 $R_H$ 是\textbf{霍尔系数(Hall coefficient)}, 等于自由电子电荷量的体密度 $ne$ 的倒数(\autoref{eq_Hallef_1} ),单个电子的带电量为 $-e$。

\begin{figure}[ht]
\centering
\includegraphics[width=8.5cm]{./figures/f6c43fb7857c45fc.png}
\caption{霍尔效应示意图(来自维基百科)} \label{fig_Hallef_1}
\end{figure}

\subsection{霍尔电阻的推导}
平衡时电场力与洛伦兹力相等
\begin{equation}
Ee = vBe~,
\end{equation}
令 $J$ 为电流密度 $J=nev$
\begin{equation}
E = JB/(ne)~.
\end{equation}
两端电压为
\begin{equation}\label{eq_Hallef_1}
V = IB/(d n e)~,
\end{equation}
\textbf{霍尔电阻(Hall resistance)}
\begin{equation}\label{eq_Hallef_2}
R = V/I = B/(d n e)~.
\end{equation}
于是定义 $R_H=1/(nq)$,我们最终得到了
\begin{equation}
V=R_H\frac{IB}{d}~.
\end{equation}

或者可以对\autoref{eq_Hallef_2} 换一种表达。考虑电阻率 $\rho$
\footnote{这里电阻率的定义与\autoref{eq_Resist_7}~\upref{Resist} 中的纵向电阻的定义不同。} ,由于霍尔电阻只和材料的厚度 $d$ 有关,和另两条边的长度 $w,L$ 无关,所以可以定义电阻率 $\rho$ 为 $R\equiv\rho /d$,那么代入 $R=V/I$ 可以得到
\begin{equation}
\begin{aligned}
&\rho\equiv Rd=\frac{B}{ne}~,\\
&\frac{\rho }{d} =R= \frac{V}{I}\Rightarrow \frac{\rho I}{dw}=\frac{V}{w}~,\\
&\rho \cdot J = E,\quad J= \frac{I}{dw}=\frac{I}{S},\quad E=\frac{V}{w}
\end{aligned}
\end{equation}
因此我们利用电阻率、电流密度、电场强度得到了一个更简单的表达式
\begin{equation}
\rho\cdot J=E~.
\end{equation}
假设电流 $I$ 是沿 $y$ 方向,产生的霍尔电压是沿 $x$ 方向,那么
\begin{equation}
E_x = \rho_{xy} J_y~.
\end{equation}
事实上如果加 $x$ 方向的电流,会产生 $-y$ 方向的霍尔电压,这是因为材料旋转90度以后的物理规律一般是不变的。 $E_y=-\rho_{xy} J_x\equiv \rho_{yx}J_x$。
$\rho_{xy}=-\rho_{yx}$ 被称为横向的霍尔电阻率。

\subsection{Drude 模型与经典霍尔效应}
下面我们从 Drude 模型\upref{DrudeM}的视角看待经典霍尔效应,并推导出它当对\textbf{二维材料}加 $z$ 方向的磁场时材料的电导率 $\sigma$ 与电阻率张量 $\rho$ 的理论计算结果。 $\sigma$ 与 $\rho$ 的定义为
\begin{equation}
\begin{aligned}
&\bvec J=\sigma \bvec E,\quad \sigma=\pmat{
    \sigma_{xx} & \sigma_{xy}\\
    -\sigma_{yx} & \sigma_{yy}
}
~,\\
&\bvec E=\rho \bvec J,\quad \rho = \sigma^{-1} =
\pmat{\rho_{xx}&\rho_{xy}\\-\rho_{yx} & \rho_{yy}}~,\\
&\bvec E=\pmat{E_x\\ E_y},\quad \bvec J = \pmat{J_x\\J_y}~.
\end{aligned}
\end{equation}

霍尔效应表明材料可能存在横向的电阻 $\rho_{xy}$,即如果我们通 $x$ 方向的电流,材料 $y$ 方向会产生霍尔电势差。

Drude 模型中,在电场作用的驱动下,材料中的自由电子会往一个方向加速运动,并有一定的几率撞到离子实被弹回。根据\autoref{eq_DrudeM_8}~\upref{DrudeM},我们可以将电子的运动方程改写为
\begin{equation}
m\dv{\bvec v}{t} =  -e(E+\bvec v\times \bvec B)-\frac{m\bvec v}{\tau} ~.
\end{equation}
其中方程右边的第二项为线性摩擦项, $\tau$ 被称为散射时间,是单个电子相邻两次碰撞间隔的平均时间。
可以将上式对所有自由电子求平均,这样 $\bvec v$ 理解为材料中自由电子的宏观平均速度,当电场驱动下平均速度不再随时间变化,那么体系达到了平衡态,可以求出平衡状态下材料的电流密度 $\bvec J=-ne\bvec v$ 与 $\bvec E$ 的关系。
\begin{equation}
\quad e\bvec v\times \bvec B + \frac{m\bvec v}{\tau} = -eE~.
\end{equation}
设磁场为 $z$ 方向,取电场与电流方向所在的平面为 $xy$平面。那么
\begin{equation}
\pmat{
m/\tau  & {eB}\\
-eB & m/\tau
} 
\cdot 
\pmat{v_x\\v_y}=\pmat{-eE_x \\ -eE_y}
~.
\end{equation}
将 $\bvec J=-ne\bvec v$ 带入,
\begin{equation}
\pmat{
    1 & eB\tau/m \\
    -eB\tau/m & 1
}
\cdot 
\pmat{
   J_x\\
   J_y 
}=
\frac{ne^2 \tau}{m }
\pmat{
    E_x\\
    E_y
}
~.
\end{equation}
定义电子在磁场中的回旋频率为
\begin{equation}
\omega_B=\frac{eB}{m}~.
\end{equation}
那么
\begin{equation}
\begin{aligned}
&\bvec E=\rho \bvec J,\quad \rho=\frac{m}{ne^2\tau} \pmat{1 & \omega_B\tau \\ -\omega_B\tau & 1},
\\
&\bvec J=\sigma \bvec E,\quad \sigma =\frac{ne^2\tau}{m} \frac{1}{1+\omega_B^2\tau^2}\pmat{1 & -\omega_B \tau \\ \omega_B \tau & 1} ~,\\
&\qquad \qquad\quad =\sigma_{DC} \cdot  \frac{1}{1+\omega_B^2\tau^2}\pmat{1 & -\omega_B \tau \\ \omega_B\tau &1} ~.
\end{aligned}
\end{equation}
其中
\begin{equation}
\sigma_{DC} = \frac{ne^2\tau}{m}~.
\end{equation}
为不加磁场时材料的直流电导率,它正比于自由电子密度。电导率张量为对角矩阵,意味着没有横向的电导。

从电阻率张量中可以看出,对角元即纵向电阻与磁场的大小无关
\begin{equation}
\rho_{xx}=\rho_{yy}=\frac{1}{\sigma_{DC}} ~.
\end{equation}
而非对角元随着磁场的增大而增大,材料的横向电阻率正比于电导率:
\begin{equation}
\rho_{xy}=-\rho_{yx} = \frac{1}{\sigma_{DC}}\cdot \omega_B\tau~.
\end{equation}


\addTODO{可以参考 David Tong 的 qhe.pdf 讲义}
