% 张成空间
% 矢量空间|张成|集合|span|基底|极大线性无关组

\pentry{线性相关性\upref{LinInd}}

矢量的线性组合,使得我们有可能用少量矢量来表示更多的矢量. 例如 $N$ 维几何矢量空间\upref{GVec}空间中的一组基底可以通过线性组合得到整个矢量空间中的任意矢量, 我们就说这组基底\textbf{张成(span)}了这个矢量空间. 我们来看 “张成” 更一般的定义.

\subsection{张成空间}

\begin{definition}{张成空间}
给定若干矢量构成的集合 $S=\{\bvec{v_\alpha}\}$,可以是无穷集合, 也可以线性相关, 那么从$S$中任意地选择有限个矢量进行线性组合所得到的集合 $\{\sum_i^N c_i\bvec{v_i}|N\in\mathbb{Z}, c_i\in\mathbb{R}, \bvec{v}_i\}$ 称为 $S$ 所\textbf{张成(span)的空间},记为 $\opn{span} S$ 或者 $\ev{\{\bvec v_\alpha\}}$.
\end{definition}
容易证明, 定义中的 $\opn{span} S$ 符合矢量空间\upref{LSpace}的定义.

张成空间的概念,最直观的例子就是子空间.比如说,三维的几何矢量空间$V$中,任意选定两个向量$\bvec{v}$和$\bvec{u}$,这两个向量的所有线性组合构成的几何,是$V$中的一个过原点的平面.这样,$\opn{span}\{\bvec{v}, \bvec{u}\}$构成了一个线性空间,同时还是$V$的子集,因此又被称作$V$的一个子空间.

\begin{exercise}{}\label{VecSpn_exe1}
给定一个三维线性空间,把其中的每个向量表示成一个行向量$(x,y,z)$,其中$x,y,z\in\mathbb{R}$.取三个向量$\bvec{v}_1=(1,2,0)$,$\bvec{v}_2=(3,2,4)$和$\bvec{v}_3=(1,1,1)$,那么这三个向量所张成的空间是一个点、一条线、一个平面还是整个三维空间本身?三个向量是否线性相关?它们的线性相关性和张成空间的样子有什么关系?
\end{exercise}

实际上,我们完全可以摆脱“用给定空间的向量来张成一个子空间”,而形式化地定义张成的概念.取任何一个集合$S$,我们可以用$S$在域$\mathbb{K}$上直接构造一个线性空间$V=\{\sum a_\alpha s_\alpha|a_\alpha\in\mathbb{K}, s_\alpha\in S\}$.这里的$a_\alpha s_\alpha$表示把数字$a_\alpha$和元素$s_\alpha$组合在一起,变成一个不属于$S$的新元素\footnote{只有一个情况例外,那就是当$a_\alpha=1$时,将$1s_\alpha$认为就是$s_\alpha$本身,从而这时并没有得到新元素.},而$\sum$表示这些新元素组合起来又得到了新的元素,组合符号用加号$+$表示.这时回过头来,我们可以把每个$s_\alpha$看成是$V$的一个向量,它们张成了$V$这个线性空间.

\subsection{基和基向量}

我们用向量来张成空间,就是在用这些向量来表示张成空间里面的所有向量的.从线性相关性\upref{LinInd}中我们知道,张成空间时所使用的向量,如果它们线性相关,那么就会有冗余向量,也就是说其中一些向量本身就可以被剩下的向量表示出来,因此我们就算把冗余的向量都剔除了,剩余的向量仍然可以张成同一个空间.

进一步,有冗余向量就意味着,张成空间里的每一个向量都可以有多个表示方法.

\begin{exercise}{}
\autoref{VecSpn_exe1} 中所给出的三个向量是线性相关的.请使用这个例子来讨论以下两个问题:
\begin{itemize}
\item 这三个向量中剔除几个向量后就不再有冗余向量?
\item 剔除的冗余向量有几种选择?
\item 用这三个向量的线性组合来表示向量$(1,0,0)$,组合方式是唯一的吗?
\end{itemize}
\end{exercise}


如果一组向量是线性无关的,那么就没有这讨厌的冗余向量了,我们就可以很方便地用这组向量来讨论张成空间的性质.

\begin{definition}{基和基向量}
如果线性空间$V$可以表示为一组向量$\{\bvec{v}_i\}$的张成空间,并且这组向量是线性无关的,那么我们说$\{\bvec{v}_i\}$是$V$的一组\textbf{基底(basis)}或简称\textbf{基},各$\bvec{v}_i$被称作一个\textbf{基向量(basis vector)}.
\end{definition}

有了基之后,线性空间中的\textbf{每一个}向量都可以\textbf{唯一}地表示成基向量的线性组合.这里,每一个向量都可以被表示成基向量的线性组合是因为定义中要求基底能张成整个线性空间,而线性组合的唯一性是因为定义中要求基底是线性无关的.这是极为重要的性质,它使得线性空间上的一切线性性质,比如线性函数、线性变换、张量的表示等等,都只和基底的选择有关.比如说,对于线性函数,一旦选定了基底,那么只需要计算出基向量的函数值,我们就可以得到一切向量的函数值,而不用对每一个向量都作一番计算.

\begin{theorem}{基底表示的唯一性}\label{VecSpn_the1}
对于线性空间$V$,给定一组基$\{\bvec{e}_i\}$.对于任何向量$\bvec{v}\in V$,它表示为该基底的线性组合的方式是唯一的.
\end{theorem}

\textbf{证明:}

设$\bvec{v}=\sum a_i\bvec{e}_i=\sum b_i\bvec{e}_i$,其中$a_i, b_i\in \mathbb{K}$,而$\mathbb{K}$是定义$V$所用的域.那么我们有:
\begin{equation}
\sum(a_i-b_i)\bvec{e}_i=\sum a_i\bvec{e}_i-\sum b_i\bvec{e}_i=\bvec{v}-\bvec{v}=\bvec{0}
\end{equation}

由于$\{\bvec{e}_i\}$线性无关,上式就意味着每一个$a_i-b_i$都是$0$,也就是说每一个$a_i$都等于$b_i$.因此不可能有两种不同的组合方式能得到$\bvec{v}$.

\textbf{证毕.}

\begin{definition}{向量的坐标}
对于线性空间$V$,给定一组基$\{\bvec{e}_i\}$.如果向量$\bvec{v}=\sum a_i\bvec{e}_i$,那么我们称$\{a_i\}$是$\bvec{v}$的一组\textbf{坐标(coordinates)}.
\end{definition}

\autoref{VecSpn_the1} 意味着选定基底以后,各向量的坐标是唯一的,但基底的选择本身不是唯一的.比如说,二维平面上,任何两个不平行的向量都可以构成这个空间的一个基.基的选择不一样,向量的坐标也不一样.对此的详细讨论请参见\textbf{矢量空间的表示}\upref{VecRep}.

\subsection{线性空间的维度}

对于一组有限个向量,如果它们线性相关,就总可以找出一个冗余向量,把它剔除;如果剔除一个冗余向量以后还有冗余向量,就重复这个操作,直到不再有冗余向量为止.这样剔除若干步后所得到的线性无关的向量组,被称为原先向量组的一个\textbf{极大线性无关组}.由于每一步剔除的冗余向量可以不同,最终剩下来的极大线性无关组也可以不同.

\begin{example}{}
依然利用\autoref{VecSpn_exe1} 的例子.向量组$\{\bvec{v}_1, \bvec{v}_2, \bvec{v}_3\}$的极大线性无关组一共有三个,分别是$\{\bvec{v}_1, \bvec{v}_2\}$,$\{\bvec{v}_2, \bvec{v}_3\}$和$\{\bvec{v}_1, \bvec{v}_3\}$.
\end{example}

由于每一步剔除的向量都是“冗余”的,因此每一步的剔除都不会改变向量组所张成的空间.尽管剔除的方式不同可能





