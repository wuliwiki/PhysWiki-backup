% 希尔伯特旅馆悖论(综述)
% license CCBYSA3
% type Wiki

本文根据 CC-BY-SA 协议转载翻译自维基百科\href{https://en.wikipedia.org/wiki/Hilbert\%27s_paradox_of_the_Grand_Hotel}{相关文章}。

希尔伯特的大酒店悖论(口语化:无限酒店悖论或希尔伯特的酒店)是一个思想实验,用来说明无限集合的一个违反直觉的性质。实验表明,即使一个完全占满的酒店有无限多的房间,它仍然可以容纳更多的客人,甚至是无限多的客人,而且这一过程可以无限次地重复。这个想法由大卫·希尔伯特在1925年的讲座《关于无限》中提出,并在《希尔伯特2013年论文集》(第730页)中重印,后来通过乔治·伽莫夫1947年的书籍《一二三……无限》得到了广泛传播。[1][2]
\subsection{悖论}  
希尔伯特设想了一个假想的酒店,房间按1、2、3等顺序编号,且没有上限。这被称为可数无限数量的房间。最初,每个房间都已被占满,但新客人到来,每个人都希望拥有自己的房间。一个正常的、有限的酒店在每个房间都已满员时无法容纳新客人。然而,可以证明,现有的客人和新来的客人——即使是无限多的——都可以在这个无限酒店里各自拥有一个房间。
\subsubsection{有限数量的新客人}  
如果有一个额外的客人,酒店可以容纳他们和现有的客人,只需要让所有现有客人同时换房。现在在1号房间的客人移到2号房间,在2号房间的客人移到3号房间,依此类推,把每个客人从他们当前的房间n移到房间n+1。由于无限酒店没有最终房间,所以每个客人都有新的房间可以去。这样一来,1号房间就空了,新客人可以被安排到这个房间。通过重复这个过程,就可以为任何有限数量的新客人腾出房间。一般来说,当有k个客人需要房间时,酒店可以应用相同的程序,把每个客人从房间n移到房间n+k。

\begin{figure}[ht]
\centering
\includegraphics[width=6cm]{./figures/6d84c164267a5248.png}
\caption{} \label{fig_HPGH_1}
\end{figure}