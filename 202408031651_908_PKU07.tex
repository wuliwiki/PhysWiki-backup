% 北京大学 2007 年 考研 量子力学
% license Usr
% type Note

\textbf{声明}:“该内容来源于网络公开资料,不保证真实性,如有侵权请联系管理员”

1.已知 $\phi = \left(\frac{3}{4}\right)^{1/2} Y_{11}(\theta, \phi) + \left(\frac{1}{4}\right)^{1/2} Y_{10}(\theta, \phi) + A Y_{1-1}(\theta, \phi)$, $(A > 0)$。

(a) 求 $L_z$ 和 $L^2$ 的平均值;

(b) 求 $\langle \phi | L_z | \phi \rangle$, $\langle \phi | L_- | \phi \rangle$。

2.已知 $|+\rangle$, $|-\rangle$ 表示自旋为 $z$ 方向上、向下的态。现有两个粒子,已知 $\psi(0) = \frac{1}{2} |+\rangle |+\rangle + \frac{1}{2} |+\rangle |-\rangle + (\frac{1}{2})^{1/2} |-\rangle |-\rangle$, $H = \omega_1 s_{1z} + \omega_2 s_{2z}$。求:

(a) $t$ 时刻的波函数;

(b) 求 $\langle s_{1z} \rangle$, $\langle s_{2z} \rangle$。

3.在连续势 $V = -q E_0 x$ 中,$t = 0$ 时 $\langle x \rangle = x_0$, $\langle p_x \rangle = 0$。

(a) 求 $t$ 时刻 $\langle p_x \rangle$;

(b) 求 $t$ 时刻 $\langle x \rangle$;

(c) 将上面的结果与经典结果比较。

4.一粒子自旋向上,且自由运动,当将 $V(x) = -A \delta(x)(\sigma_+ + \sigma_-)$ 散射。问:散射后的粒子自旋向下的反射的全反射几率。

5.已知 $\hat{H} = \frac{p^2}{2m} + \frac{1}{2} m \omega^2 x^2 \left(1 + \frac{1}{\cosh^2(\frac{x}{x_0})}\right) |0\rangle_{range} \langle 0| + |n\rangle$ 为第 $n$ 个激发态。求:

(a)跃迁几率$P_0\t1。$;
        \item 当 $n = 0$ 时的跃迁几率 $P_0 = ?$;
        \item 当 $t = 0$ 时的结果如何?
    \end{enumerate}
