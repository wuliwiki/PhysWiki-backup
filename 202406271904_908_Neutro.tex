% 中子
% license CCBYSA3
% type Wiki

(本文根据 CC-BY-SA 协议转载自原搜狗科学百科对英文维基百科的翻译)

\textbf{中子}是亚原子粒子,符号为$n$或者n⁰,净电荷为零且质量略大于质子。质子和中子构成了原子的核子。因为质子和中子在原子核内的行为相似,每个质子和中子的质量大约为1原子质量单位,它们统称为核子。[1]它们的性质和相互作用由核物理描述。

原子核的化学和核性质由质子数以及中子数决定,分别称为原子序数和中子数。原子质量数是核子的总数。比如,碳原子序数为6,并且常见的碳-12有6个中子,而罕见的碳-13有7个中子。有些元素在自然界中只有一种稳定同位素,例如氟。有些元素则有许多稳定的同位素,比如锡有十种稳定同位素。

在原子核内,质子和中子通过核力结合在一起。除了单质子氢原子之外,原子核的稳定需要中子。中子在核裂变和聚变中大量产生。它们通过裂变、聚变和中子俘获过程参与在恒星内化学元素的核合成。

中子对核能生产至关重要。在詹姆斯·查德威克1932年发现中子后的十年里,中子被用来诱导许多不同类型的核嬗变。随着1938年核子裂变的发现,[2]人们很快认识到,裂变事件产生的每一个中子都可能引起进一步的裂变事件,这种级联反应被称为核链式反应。这些发现导致了第一个自持的核反应堆 ( 芝加哥一号堆,1942年)和第一个核武器 ( 三一,1945年)的产生。

自由中子虽然不是电离原子,但会导致致电离辐射。因此,当剂量较大时,它们具备生物性危害。地球上存在有小的自然的自由中子,主要来源于宇宙线射线浴以及地壳内自发裂变元素产生的辐射。用于辐射和中子散射实验的自由中子来自于专用的中子源,如中子发生器、研究堆和散裂源。

\subsection{描述}
原子核由许多质子形成,$Z$(原子序数)和中子数,$N$(中子数),由核力结合在一起。原子序数定义原子的化学性质,中子数决定同位素或核素。[3]同位素和核素这两个术语经常被当作同义词使用,但是它们分别指化学性质和核性质。严格来说,同位素是两种或多种质子数相同的核素;具有相同中子数的核素被称为同中子异荷素。原子质量数,符号$A$,等于$Z+N$。具有相同原子质量数的核素称为同量异位素。氢原子最常见的同位素(带有化学符号${}^{1}h)$),其核子是一个单独的质子。重氢同位素的原子核$(D\text{或}{}^{2}H)$和氚$(T\text{或}{}^{3}H)$分别包含一个和两个中子。其他所有类型的原子核都由两个或多个质子和不同数量的中子组成。化学元素铅中最常见的核素,${}^{208}Pb)$有82个质子和126个中子。核素表包括所有已知的核素。尽管中子不是化学元素,但它包含在这个表中。[3]

自由中子的质量为939,565,$413.3 eV /c^{2}$,或$1.674927471\times10^{-27}$ kg,或1.00866491588 u。中子的均方半径约为$0.8\times10^{-15}$ m,或0.8 fm ,[4]这是自旋为1的费米子。[5]质子带正电荷,直接受电场影响,而中子不带电荷,故不受电场影响。然而,中子具有磁矩,因此受磁场影响。中子的磁矩是负的,其取向与其自旋方向相反。[6]

自由中子是不稳定的,其衰变产物为质子,电子和反中微子,平均寿命不到15分钟(881.5±1.5 s)。这种被称为β衰变的放射性衰变是可能的,因为中子的质量略大于质子且自由质子是稳定的。然而,结合在原子核中的中子或质子可以是稳定的,也可以是不稳定的,这取决于核素。在$\beta$衰变,中子衰变为质子,反之亦然,受弱力支配,它需要发射或吸收电子和中微子,或它们的反粒子。
\begin{figure}[ht]
\centering
\includegraphics[width=6cm]{./figures/a27c6c16b4c60c50.png}
\caption{铀-235吸收中子引起的核裂变。重核素分裂成更轻的成分和额外的中子。} \label{fig_Neutro_1}
\end{figure}
质子和中子在核子核力的影响下表现几乎相同。同位旋的概念视质子和中子为同一粒子的两个量子态,用于模拟核子在强力和弱子下的相互作用。由于核力在短距离内的强度,核子的结合能比原子中束缚电子的电磁能大7个数量级。因此核反应(例如核子裂变)产生的能量密度是化学反应的1000多万倍。根据质能公式,核结合能降低了原子核的质量。因此,核子内部电磁排斥所产生的能量是核能的主要来源,这也使得核反应堆和核子炸弹成为可能。在核裂变中,中子被重核素吸收(例如,铀-235)导致核素变得不稳定,并分裂成轻核素和额外的中子。带正电荷的轻核素然后排斥,释放电磁势能。

中子被归类为强子,因为它是由夸克组成的复合粒子。同时它也被分类为重子,因为它由三个价夸克组成[7]。中子的大小及其磁矩表明中子是一种复合粒子,而不是基本粒子。一个中子包含两个带电荷$-\frac{1}{3}e$的下夸克和一个带电荷$+\frac{2}{3}e$的上夸克。

像质子一样,中子的夸克通过由胶子做介质的强力结合在一起。[8]核力源于更基本的强力的次要影响。



















