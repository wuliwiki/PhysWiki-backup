% 詹姆斯·克拉克·麦克斯韦(综述)
% license CCBYSA3
% type Wiki

本文根据 CC-BY-SA 协议转载翻译自维基百科\href{https://en.wikipedia.org/wiki/James_Clerk_Maxwell}{相关文章}。
\begin{figure}[ht]
\centering
\includegraphics[width=6cm]{./figures/86970eb17e0c5fea.png}
\caption{Maxwell, 大约 1870 年代} \label{fig_Clerk_1}
\end{figure}
詹姆斯·克拉克·麦克斯韦(James Clerk Maxwell FRS FRSE,1831年6月13日-1879年11月5日)是一位苏格兰物理学家和数学家[1],他提出了电磁辐射的经典理论,这是第一个将电、磁和光视为同一现象的不同表现形式的理论。麦克斯韦的电磁学方程实现了物理学中的第二次伟大统一[2],第一次统一则由艾萨克·牛顿实现。麦克斯韦在统计力学的创立中也起到了关键作用[3][4]。

1865年,麦克斯韦发表了《电磁场的动力学理论》,在该文中他证明了电场和磁场作为波动通过空间传播,传播速度与光速相同。他提出光是同一介质中的波动,这种介质也是电磁现象的来源[5]。光与电现象的统一促使他预言了无线电波的存在,并且这篇论文包含了他自1856年以来一直在研究的方程的最终版本[6]。由于他的方程以及他在解决网络问题和线性导体问题上提出的有效方法,他被视为现代电气工程学科的奠基人之一[7]。1871年,麦克斯韦成为第一任卡文迪许物理学教授,并一直担任此职直到1879年去世。

麦克斯韦是第一个推导出麦克斯韦–玻尔兹曼分布的人,这是一种描述气体动理论各个方面的统计方法,他在职业生涯中断断续续地研究了这一课题[8]。他还因于1861年展示了第一张持久的彩色照片而闻名,并且在分析杆接框架(如许多桥梁中的桁架)的刚性方面做出了基础性贡献。麦克斯韦帮助建立了CGS测量系统[9],并且他对现代尺寸分析的贡献也不可忽视[10][11]。麦克斯韦还因奠定了混沌理论的基础而受到认可[12][13]。他正确预测土星的环是由许多未附着的小碎片组成的[14]。他于1863年发表的《关于调速器》一文为控制理论和控制论提供了重要的基础,也为控制系统的早期数学分析[15][16]。在1867年,他提出了著名的思想实验——麦克斯韦妖[17]。

他的发现帮助开启了现代物理学的时代,为相对论等领域奠定了基础,他也是第一个将这一术语引入物理学的人[10],并对量子力学的发展做出了贡献[18][19]。许多物理学家认为,麦克斯韦是19世纪对20世纪物理学影响最大的科学家。他对科学的贡献被许多人认为与艾萨克·牛顿和阿尔伯特·爱因斯坦的贡献同样重要[20]。在麦克斯韦诞辰一百周年时,爱因斯坦曾形容他的工作是“自牛顿时代以来物理学所经历的最深刻且最富有成果的工作”[21]。当爱因斯坦于1922年访问剑桥大学时,他的主人告诉他,他之所以取得伟大的成就,是因为站在牛顿的肩膀上;爱因斯坦回答说:“不,我不是。我站在麦克斯韦的肩膀上。”[22] 汤姆·西格弗里德(Tom Siegfried)形容麦克斯韦是“那种百年一遇的天才,他比周围的人更敏锐地感知到物理世界。”[23]
\subsection{生活}  
\subsubsection{早期生活,1831–1839}
\begin{figure}[ht]
\centering
\includegraphics[width=6cm]{./figures/22a3e8c25ac81a93.png}
\caption{克拉克·麦克斯韦的出生地位于爱丁堡的印度街14号,现在是詹姆斯·克拉克·麦克斯韦基金会的所在地。} \label{fig_Clerk_7}
\end{figure}
詹姆斯·克拉克·麦克斯韦于1831年6月13日出生在爱丁堡印度街14号[24],父亲是中贝比的约翰·克拉克·麦克斯韦(John Clerk Maxwell),一名律师,母亲是弗朗西斯·凯(Frances Cay),她是罗伯特·霍奇森·凯(Robert Hodshon Cay)的女儿,也是约翰·凯(John Cay)的妹妹[25][26]。他出生的地方现在是由詹姆斯·克拉克·麦克斯韦基金会运营的博物馆。麦克斯韦的父亲来自拥有相当财富的克拉克家族,该家族位于佩尼库克,并持有佩尼库克克拉克的男爵头衔[27]。他父亲的兄弟是第六代男爵[28]。他的父亲出生时名为“约翰·克拉克”,在1793年继承了中贝比庄园(一个位于邓弗里郡的麦克斯韦家族财产)后,便将“麦克斯韦”作为自己的姓氏[25]。詹姆斯是艺术家杰米玛·布莱克本(Jemima Blackburn)的表姐[29](她是他父亲妹妹的女儿)和土木工程师威廉·戴斯·凯(William Dyce Cay)的表兄[30](他是他母亲兄弟的儿子)。凯和麦克斯韦是亲密的朋友,凯在麦克斯韦结婚时担任了他的伴郎。

麦克斯韦的父母在三十多岁时相识并结婚[31];他的母亲在他出生时已经接近40岁。他们之前曾有一个孩子,一个名叫伊丽莎白的女儿,但她在婴儿时期去世[32]。

麦克斯韦年幼时,家人搬到了基尔库布赖特郡的格伦莱尔(Glenlair),这是他的父母在一个包含1500英亩(610公顷)土地的庄园上建造的房子[33]。所有迹象表明,麦克斯韦从小就保持着一种无法熄灭的好奇心[34]。到三岁时,他对任何能移动、发光或发出声音的东西都会产生疑问:“那是什么?”[35]。在他父亲1834年写给小姑简·凯(Jane Cay)的信中添加的一段话中,他的母亲描述了这种与生俱来的好奇心:

“他是一个非常快乐的孩子,自从天气变得温和以来,他进步了很多;他对门、锁、钥匙等非常感兴趣,‘告诉我怎么做’几乎是他嘴边常挂的话。他还会研究小溪和铃线的隐蔽路线,水是如何从池塘通过墙壁流过来的……”[36]
\subsubsection{教育,1839–1847}  
意识到这个男孩的潜力,麦克斯韦的母亲弗朗西斯承担了他的早期教育责任,在维多利亚时代,这通常是家庭主妇的工作[37]。八岁时,他就能背诵约翰·弥尔顿的长篇段落和完整的《诗篇》第119篇(176节)。实际上,他对圣经的了解已经相当详尽;几乎能准确说出《诗篇》中的任何引文的章节和节数。他的母亲因腹部癌症生病,在一次未成功的手术后,于1839年12月去世,当时他年仅八岁。此后,他的教育由父亲和父亲的嫂子简共同负责,两人在他的一生中扮演了关键角色[37]。他的正式学校教育开始时并不顺利,由一位16岁的聘请导师指导。关于这位年轻导师的信息很少,只有知道他对麦克斯韦态度严厉,经常责备他反应迟钝、行为古怪[37]。这位导师于1841年11月被解雇。1842年2月12日,詹姆斯的父亲带他去参加罗伯特·戴维森(Robert Davidson)关于电推进和磁力的示范,这次经历对麦克斯韦产生了深远的影响[38]。

麦克斯韦被送往了著名的爱丁堡学院[39]。在学期期间,他住在姑妈伊莎贝拉的家中。在这段时间里,他对绘画的热情得到了表姐杰米玛的鼓励[40]。10岁的麦克斯韦,由于在父亲的乡村庄园里孤立地长大,并不太适应学校生活[41]。第一年他因人数已满,只得和比自己年长一岁的同班同学一起加入了第二年级[41]。他的举止和高洛韦口音使得其他男孩觉得他有些乡土气息。第一次上学时,他穿着一双自制鞋和一件长袍,这使得他得到了不友好的绰号“傻小子”(Daftie)[42]。他似乎从未对这个绰号感到不满,许多年里毫无怨言地承受着它[43]。在学院的社交孤立局面直到他遇到了刘易斯·坎贝尔(Lewis Campbell)和彼得·格思里·泰特(Peter Guthrie Tait)才得以改变,这两位年纪相仿的男孩后来都成了著名的学者,他们也成为了终生的朋友[25]。

麦克斯韦从小就对几何学充满兴趣,在接受任何正式教育之前,他就重新发现了正多面体[40]。尽管他在第二年获得了学校的圣经传记奖,但他的学术成绩一直未受到关注[40],直到13岁时,他赢得了学校的数学奖牌,并获得了英语和诗歌的第一奖[44]。

麦克斯韦的兴趣远超学校课程,他并未特别关注考试成绩[44]。14岁时,他写了他的第一篇科学论文。在这篇论文中,他描述了一种利用一根细绳绘制数学曲线的机械方法,并探讨了椭圆、笛卡尔椭圆和具有多个焦点的相关曲线的性质。这篇1846年的论文《关于椭圆曲线及具有多个焦点的曲线的描述》[46]由爱丁堡大学自然哲学教授詹姆斯·福布斯(James Forbes)提交给爱丁堡皇家学会,因为麦克斯韦被认为年纪太小,无法亲自提交这篇论文[25][45]。这项工作并不完全原创,因为17世纪的勒内·笛卡尔也研究过这种多焦点椭圆的性质,但麦克斯韦简化了它们的构造[47]。
\subsubsection{爱丁堡大学,1847–1850}
\begin{figure}[ht]
\centering
\includegraphics[width=6cm]{./figures/954be534fffd5046.png}
\caption{爱丁堡大学旧学院} \label{fig_Clerk_6}
\end{figure}
麦克斯韦于1847年16岁时离开了爱丁堡学院,开始在爱丁堡大学上课[48]。虽然他有机会就读剑桥大学,但在第一学期后,他决定完成自己在爱丁堡的本科课程。大学的学术教职员工中有许多备受尊敬的名字;他第一年的导师包括威廉·汉密尔顿爵士(Sir William Hamilton),他教授逻辑学和形而上学,菲利普·凯兰德(Philip Kelland)教授数学,詹姆斯·福布斯(James Forbes)教授自然哲学[25]。他觉得这些课程并不难[49],因此他能够在大学的空闲时间,特别是在回到家里格伦莱尔时,专心进行个人研究[50]。在那里,他会用简易的化学、电学和磁学设备进行实验;然而,他主要关注的是偏振光的性质[51]。他制造了形状各异的明胶块,施加不同的压力,并用威廉·尼科尔(William Nicol)给他的一对偏振棱镜观察胶体中形成的彩色边带[52]。通过这种实验,他发现了光弹性,这是确定物理结构中应力分布的一种方法[53]。

18岁时,麦克斯韦为《爱丁堡皇家学会会刊》贡献了两篇论文。其中一篇《弹性固体的平衡》奠定了他一生中一项重要发现的基础,即剪切应力在粘性液体中产生的暂时双折射现象[54]。另一篇论文是《滚动曲线》,与他在爱丁堡学院所写的《椭圆曲线》论文一样,他再次被认为太年轻,无法亲自站在讲台上发表这篇论文。于是,论文由他的导师凯兰德代为提交给皇家学会[55]。
\subsubsection{剑桥大学,1850–1856}
\begin{figure}[ht]
\centering
\includegraphics[width=6cm]{./figures/80507c6607b821a2.png}
\caption{年轻的麦克斯韦在剑桥大学三一学院,手持他的色轮之一。} \label{fig_Clerk_2}
\end{figure}
1850年10月,麦克斯韦已经是一位出色的数学家,他离开苏格兰前往剑桥大学。他最初就读于彼得学院,但在第一学期结束前转学到了三一学院,他认为在那里更容易获得奖学金[56]。在三一学院,他被选入了剑桥大学的精英秘密社团——剑桥使徒社[57]。在剑桥的岁月里,麦克斯韦对基督教信仰和科学的理解迅速加深。他加入了“使徒社”——一个由知识精英组成的辩论社团,在那里通过他的论文,他努力探索并深化对这些思想的理解。

这是麦克斯韦在其信仰与科学探讨中的一段文字,翻译如下:

“现在,我的伟大计划,早已酝酿成型,...就是不允许任何事情故意不被审视。没有任何事物应当是圣地,不可触碰的,无论是积极的还是消极的。所有荒地都应被耕耘,遵循一个规律性的轮作制度。...永远不要隐藏任何东西,无论它是杂草还是其他,也不要表现出想要将其隐藏的样子。...我再次宣示,任何人设立的圣地都应该允许侵犯。...现在,我相信,只有基督徒才能真正清除这些圣地上的污秽。...我并不是说没有基督徒会设立这样的禁地。许多人拥有很多,每个人都有一些。但在嘲讽者、泛神论者、安静主义者、形式主义者、教条主义者、感官主义者以及其他一些人的领域里,存在着广阔而重要的禁忌区,这些区域是公开且庄严的被视为禁区的。...”

基督教——即圣经中的宗教——是唯一一种不承认任何这种约定形式的信仰体系。在这里,所有事物都是自由的。你可以飞到世界的尽头,找到的只有拯救之主。你可以翻阅圣经,找不到任何一句话能阻止你进行探索。...

旧约、摩西法则和犹太教通常被正统信仰认为是‘禁忌’的。怀疑者假装已经阅读过这些内容,并找到了某些机智的反对意见……这些反对意见被太多未读过的正统信徒接受,并将这一话题封闭为‘鬼魂缠绕’。但一盏灯即将到来,驱散所有的幽灵和恐惧。让我们跟随这道光明吧。[58]

在他大三的夏天,麦克斯韦在他的同学G. W. H. Tayler的叔叔C. B. Tayler牧师位于萨福克的家中度过了一段时间。Tayler一家人展现出的神的爱给麦克斯韦留下了深刻的印象,特别是在他因病体虚弱而得到牧师和妻子的悉心照料后[59]。

返回剑桥后,麦克斯韦写信给他近期的东道主,信中包括以下温馨且亲切的陈述[58]:

“…我有比任何人能为我树立的榜样更大的堕落能力,而…如果我能逃脱,这仅仅是因为神的恩典帮助我部分地从自己中解脱出来,在科学中更为明显,在社会中则更彻底——但除非完全将自己交托给神,否则是不可能完美的…”

1851年11月,麦克斯韦在威廉·霍普金斯的指导下学习,霍普金斯因在培养数学天才方面的成功而获得了“高级学士制造者”的绰号[60]。

1854年,麦克斯韦以数学学位从三一学院毕业。他在最终考试中成绩排名第二,仅次于爱德华·罗思(Edward Routh),因此获得了“第二学士”的称号。随后,他在更为严格的史密斯奖学金考试中与罗思并列[61]。获得学位后,麦克斯韦向剑桥哲学学会宣读了他的论文《关于表面通过弯曲的变换》[62]。这篇论文是他为数不多的纯数学论文之一,展示了他作为数学家的日益崭露头角[63]。麦克斯韦决定毕业后继续留在三一学院,并申请了奖学金,这个过程预计需要几年的时间[64]。凭借作为研究生的成功,他除了需要承担一些辅导和考核的职责外,可以自由地追求自己的科学兴趣,享受个人的学术时光[64]。

色彩的性质和感知是他的一项兴趣,早在他还是爱丁堡大学福布斯教授的学生时就已开始研究[65]。通过福布斯发明的彩色陀螺,麦克斯韦能够证明,红光、绿光和蓝光的混合会产生白光[65]。他的论文《关于色彩的实验》阐述了色彩组合的原理,并于1855年3月提交给了爱丁堡皇家学会[66]。这一次,麦克斯韦亲自进行了论文的宣读[66]。

麦克斯韦于1855年10月10日被任命为三一学院的研究员,成为通常流程之外的例外[66],并被要求准备有关静水学和光学的讲座以及出题[67]。次年2月,福布斯教授敦促他申请阿伯丁马里沙尔学院的自然哲学新空缺教席[68][69]。他的父亲帮助他准备所需的推荐信,但在两人都未得知麦克斯韦候选结果之前,他父亲于4月2日在格伦莱尔去世[69]。麦克斯韦接受了阿伯丁的教职,并于1856年11月离开剑桥[67]。
\subsubsection{阿伯丁马里沙尔学院,1856–1860}
25岁的麦克斯韦比马里沙尔学院的其他教授年轻了至少15年。他全身心投入到作为一个系主任的新职责中,制定课程大纲并准备讲座[70]。他每周安排15小时的授课,其中包括每周一次为当地工人学院提供的无偿讲座[70]。在学年中的六个月里,他与他的表亲威廉·戴斯·凯(苏格兰土木工程师)一起住在阿伯丁,夏天则回到他从父亲那里继承的格伦莱尔[28]。
\begin{figure}[ht]
\centering
\includegraphics[width=6cm]{./figures/c23aa2e17f55f863.png}
\caption{麦克斯韦证明了土星的环是由无数小颗粒组成的。} \label{fig_Clerk_3}
\end{figure}
后来,他的前学生如此描述麦克斯韦:

在1850年代末的某个冬日早晨,9点钟前,你很可能会看到年轻的詹姆斯·克拉克·麦克斯韦,那个二十多岁中期至晚期的年轻人,身材中等,骨架结实,步态中带着一种弹性和活力;他的穿着以舒适为主,而非优雅;面容一方面显得聪慧和愉快,另一方面又透露出深沉的思考;五官轮廓鲜明且颇具吸引力;眼睛深邃而闪亮;黑色的头发和胡须,与他苍白的肤色形成鲜明对比[71]。
\begin{figure}[ht]
\centering
\includegraphics[width=6cm]{./figures/4ccffc95d0db12b6.png}
\caption{詹姆斯·克拉克·麦克斯韦和他的妻子,杰米玛·布莱克本所绘} \label{fig_Clerk_5}
\end{figure}
他将注意力集中在一个困扰科学家们200年的问题上:土星环的性质。人们不知道它们是如何保持稳定的,既不破碎,也不漂离或撞向土星。[72] 这个问题在当时具有特别的意义,因为剑桥大学圣约翰学院选择它作为1857年亚当斯奖的主题。[73] 麦克斯韦花了两年时间研究这个问题,证明了一个规则的固体环不可能保持稳定,而流体环则会由于波动作用而分裂成小块。由于没有观察到这两种情况,他得出结论,土星的环必须是由无数小颗粒组成的,他称这些颗粒为“砖块”,每个颗粒独立地绕土星运转。[73] 1859年,麦克斯韦因其论文《论土星环运动的稳定性》获得了130英镑的亚当斯奖;[74] 他是唯一一个在此问题上取得了足够进展并提交了论文的参赛者。[75] 他的工作如此详细且令人信服,以至于当乔治·比德尔·艾里阅读它时评论道:“这是我见过的最出色的数学与物理学结合应用之一。”[76] 直到1980年代“旅行者”号飞船的直接观测证实了麦克斯韦的预测,即土星环是由颗粒组成的,这一理论才被证实。[77] 然而,现在已知这些颗粒并非完全稳定,它们被重力拉向土星。预计这些环将在未来3亿年内完全消失。[78]

1857年,麦克斯韦结识了丹尼尔·德沃牧师,当时他是马里什尔学院的院长。[79] 通过德沃,麦克斯韦认识了德沃的女儿凯瑟琳·玛丽·德沃。他们于1858年2月订婚,并于1858年6月2日在阿伯丁结婚。在结婚记录中,麦克斯韦被列为马里什尔学院的自然哲学教授。[80] 凯瑟琳比麦克斯韦大七岁。关于她的资料相对较少,尽管已知她曾在麦克斯韦的实验室帮忙,并参与了粘度实验的研究。[81] 麦克斯韦的传记作家和朋友刘易斯·坎贝尔在谈到凯瑟琳时采取了不寻常的克制态度,尽管他描述他们的婚姻生活是“无与伦比的献身”。[82]

1860年,马里什尔学院与邻近的国王学院合并,成立了阿伯丁大学。由于自然哲学学科不能有两位教授,尽管麦克斯韦享有很高的科学声誉,他还是被裁员了。他申请了爱丁堡大学福布斯教授的空缺,但未能成功,职位最终由泰特(Tait)获得。麦克斯韦则获得了伦敦国王学院的自然哲学教授职位。[83] 经过1860年一次几乎致命的天花病发作后,他与妻子一起搬到了伦敦。[84]
\subsubsection{伦敦国王学院,1860–1865}
\begin{figure}[ht]
\centering
\includegraphics[width=6cm]{./figures/1df49cf314da3258.png}
\caption{在伦敦国王学院纪念麦克斯韦方程。两块相同的IEEE里程碑纪念 plaque 分别位于麦克斯韦的出生地爱丁堡和家族故居格伦莱尔。} \label{fig_Clerk_4}
\end{figure}
麦克斯韦尔在国王学院的时间可能是他职业生涯中最富有成效的时期。他因在颜色方面的工作而在1860年获得皇家学会的伦福德奖章,并于1861年当选为皇家学会会员。[86] 这一时期他展示了世界上第一张耐光色彩照片,进一步发展了他关于气体粘度的理论,并提出了一种定义物理量的系统——现在被称为维度分析。麦克斯韦尔经常参加皇家学会的讲座,在那里他与迈克尔·法拉第有了定期接触。两人之间的关系虽然不能称得上亲密,因为法拉第比马克斯韦尔年长40岁,且已显现出老年痴呆的迹象,但他们彼此始终保持着对对方才华的深深尊重。[87]

这段时间尤其值得注意,因为麦克斯韦尔在电学和磁学领域取得了显著的进展。他在1861年发表的两部分论文《物理力线》中,研究了电场和磁场的性质。在这篇论文中,他为电磁感应提供了一个概念模型,其中磁通量由微小旋转的磁场单元组成。之后,马克斯韦尔在1862年初进一步补充了两部分内容并发表。在第一部分中,他讨论了静电学的性质和位移电流。在第二部分中,他研究了在磁场中光的偏振面旋转的现象,这一现象由法拉第发现,现在被称为法拉第效应。[88]
\subsubsection{后期,1865–1879}
\begin{figure}[ht]
\centering
\includegraphics[width=6cm]{./figures/59b268fcc91b39dc.png}
\caption{詹姆斯·克拉克·麦克斯韦、他的父母和妻子的墓碑位于帕顿教堂(加洛韦)。} \label{fig_Clerk_8}
\end{figure}
1865年,麦克斯韦辞去了伦敦国王学院的教职,和凯瑟琳一起回到了格伦莱尔。在他的论文《论调节器》(1868年)中,他数学描述了调节器——一种控制蒸汽机速度的装置——的行为,从而为控制工程奠定了理论基础。[89] 在他的论文《论对偶图形、框架和力的图示》(1870年)中,他讨论了不同类型格栅设计的刚性。[90][91] 他还撰写了《热学理论》一书(1871年)和《物质与运动》一书(1876年)。麦克斯韦也是第一个明确使用维度分析的人,时间是在1871年。[92]

1871年,他返回剑桥,成为首任卡文迪许物理学教授。[93] 麦克斯韦负责卡文迪许实验室的发展,监督了实验室建设的每个步骤以及设备的采购。[94] 麦克斯韦对科学的最后一项重大贡献之一是编辑(并附上大量原创注释)亨利·卡文迪许的研究,这些研究表明,卡文迪许研究了包括地球密度和水的成分在内的多个问题。[95] 1876年,他被选为美国哲学学会的会员。[96]
\subsubsection{去世}  
1879年4月,麦克斯韦开始出现吞咽困难,这是他致命疾病的第一个症状。[97]

麦克斯韦于1879年11月5日在剑桥因腹部癌症去世,享年48岁。[48] 他的母亲也因相同类型的癌症去世,且年龄相同。[98] 在麦克斯韦生命的最后几周,常常来探望他的牧师对他在临终时表现出的清晰思维和强大记忆力感到震惊,但他特别提到:

... 他的病痛展现了这位男子的全部心灵、思想和精神:他对道成肉身及其所有结果的坚定不移的信仰;对完全赎罪的信任;对圣灵工作的信仰。他已透彻地了解并深刻反思了所有的哲学体系,发现它们完全空洞且无法满足——“不可行”是他自己对它们的评价——于是他带着简单的信仰转向了救世主的福音。

临终时,麦克斯韦对剑桥的同事说道,[58]

我一直在思考,我是多么温和地被对待过一生。我一生从未经历过剧烈的推搡。我唯一的愿望就像大卫一样,按照上帝的旨意为我的时代服务,然后安然入睡。

麦克斯韦葬于帕顿教堂,位于加洛韦的卡斯尔道格拉斯附近,这里靠近他成长的地方。[99] 由他的旧时同学和终生朋友刘易斯·坎贝尔教授所写的《詹姆斯·克拉克·麦克斯韦传》于1882年出版。[100][101] 他的全集在1890年由剑桥大学出版社出版,共两卷。[102]

麦克斯韦遗产的执行人是他的医生乔治·爱德华·帕杰特、G·G·斯托克斯和麦克斯韦的表亲科林·麦肯齐。由于工作繁重,斯托克斯将麦克斯韦的文件交给了威廉·加内特,后者实际上一直保管这些文件直到约1884年。[103]

在西敏寺合唱坛屏附近有一块纪念碑刻有他的名字。[104]
\subsubsection{个人生活}
\begin{figure}[ht]
\centering
\includegraphics[width=6cm]{./figures/e7587ab02e93718b.png}
\caption{詹姆斯·克拉克·麦克斯韦,由杰米玛·布莱克本所画。} \label{fig_Clerk_9}
\end{figure} 
作为一位伟大的苏格兰诗歌爱好者,麦克斯韦记忆了许多诗歌,并且自己也创作诗歌。[105] 他最著名的作品是《刚体之歌》,这首诗紧密模仿了罗伯特·彭斯的《穿过麦田》,麦克斯韦显然曾在弹吉他时唱这首诗。诗的开头是:

假如一个人遇到一个人  
飞翔在空中,  
假如一个人撞到一个人,  
它会飞吗?飞到哪里?

他的诗集由他的朋友刘易斯·坎贝尔于1882年出版。[107]  

对麦克斯韦的描述指出,他那令人瞩目的智力特点与社交上的笨拙相匹配。[108]

麦克斯韦为自己作为一名科学家的行为写下了以下箴言:

“想要享受生活并自由行动的人,必须将当天的工作始终放在眼前。不是昨日的工作,以免陷入绝望;也不是明日的工作,以免成为空想家;既不是那种随着一天的结束而终止的世俗工作,也不是仅仅属于永恒的工作,因为通过它他无法塑造自己的行动。能在今天的工作中看到生活工作的一部分,并将其作为永恒工作的一种体现的人是幸福的。他的信心基础是不可动摇的,因为他已成为无限的一部分。他竭尽全力完成每日的工作,因为现在是他所拥有的财富。”[109]

麦克斯韦是一个福音派的长老会教徒,在晚年成为了苏格兰教会的长老。[110] 麦克斯韦的宗教信仰和相关活动成为了多篇论文的研究主题。[111][112][113][114] 作为孩子时,麦克斯韦同时参加过苏格兰教会(父亲的教派)和圣公会(母亲的教派)的礼拜。在1853年4月,麦克斯韦经历了一次福音派的皈依。这次皈依的一个方面可能使他倾向于反实证主义立场。[113]
\subsection{科学遗产}  
\subsubsection{认可}  
在《Physics World》进行的100位最杰出物理学家的调查中,麦克斯韦被评为历史上第三伟大的物理学家,仅次于牛顿和爱因斯坦。[115] 另一次由《PhysicsWeb》对普通物理学家进行的调查也将他排在第三位。[116]
\subsubsection{电磁学}
\begin{figure}[ht]
\centering
\includegraphics[width=6cm]{./figures/ff568165c1143260.png}
\caption{一张 Maxwell 写给 Peter Tait 的明信片} \label{fig_Clerk_10}
\end{figure}
麦克斯韦早在 1855 年就开始研究并评论电学和磁学,当时他在剑桥哲学学会宣读了他的论文《法拉第的力线》[117]。这篇论文提出了法拉第工作的一种简化模型,并阐述了电和磁之间的关系。他将所有已知的内容简化为一组相互关联的微分方程,共有 20 个方程,涉及 20 个变量。该工作后来于 1861 年 3 月作为《物理力线》发表[118]。

大约在 1862 年,麦克斯韦在金斯学院讲课时计算出电磁场的传播速度大约与光速相同。他认为这不仅仅是巧合,并评论道:“我们几乎无法避免得出结论:光是由与电磁现象相关的同一种介质的横向波动构成的。”[76]

继续研究这个问题,麦克斯韦表明,这些方程预测了电磁场振荡波的存在,这些波以电磁场的方式传播,通过真空传播的速度可以通过简单的电学实验来预测;根据当时的数据,麦克斯韦得出了 310,740,000 米每秒(\(1.0195\times10^9 \)英尺/秒)的传播速度。[119] 在他 1865 年的论文《电磁场的动力学理论》中,麦克斯韦写道:“结果的一致性似乎表明,光和磁性是同一种物质的效应,并且光是通过电磁法则传播的电磁干扰。”[5]

他著名的二十个方程,经过现代化后成为偏微分方程,首次在他 1873 年出版的《电学与磁学学说》中以完整形式出现。[120] 这些工作大多数是在麦克斯韦担任伦敦职务与担任卡文迪许讲座期间,在格伦莱尔的家中完成的。[76] 奥利弗·海维赛德将麦克斯韦理论的复杂性简化为四个偏微分方程,[121] 现在统称为麦克斯韦定律或麦克斯韦方程。尽管在十九世纪,势函数变得不再那么流行,[122] 但在解麦克斯韦方程时,标量势和矢量势的使用现在已成为标准做法。[123] 他的工作实现了物理学中的第二次伟大统一。[124]

正如巴雷特和格赖姆斯(1995)所描述的:[125]

麦克斯韦用四元数代数表达了电磁学,并将电磁势作为其理论的核心。1881年,海维赛德将电磁势场替换为力场,成为电磁理论的核心。根据海维赛德的说法,电磁势场是任意的,需要被“暗杀”(原文如此)。几年后,海维赛德与[彼得·古思里]·泰特(原文如此)之间就矢量分析与四元数的相对优点展开了辩论。结果得出结论,如果理论是纯粹局部的,那么四元数所提供的更深刻的物理见解并非必要,矢量分析因此变得广泛应用。