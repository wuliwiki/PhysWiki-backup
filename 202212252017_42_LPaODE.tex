% 一阶线性偏微分方程与常微分方程组的等价性
% 线性偏微分方程
\pentry{一般积分\upref{IntGen}}
本词条将说明,一阶线性偏微分方程与常微分方程组具有直接的联系:一阶线性偏微分方程求解问题可以化为常微分方程组的求解问题。这也是对应常微分方程组被称为一阶线性偏微分方程的\textbf{特征方程组}的原因。
\subsection{将常微分方程组写成更为对称的形式}
一般的常微分方程组都可写为下面的形式(\autoref{GO2SOD_the2}~\upref{GO2SOD})
\begin{equation}\label{LPaODE_eq4}
\dv{y_i}{x}=f_i(x,y_1,\cdots,y_n),\quad i=1,\cdots,n
\end{equation}
这可写为下面等价的形式
\begin{equation}
\dd x=\frac{\dd y_1}{f_1(x,y_1,\cdots,y_n)}=\cdots=\frac{\dd y_n}{f_n(x,y_1,\cdots,y_n)}
\end{equation}
为使第一项的分母不为1,可把该式所有的分母都乘上共同的因子。并且为对称起见,将 $x$ 记为 $x_1$,$y_i$ 记为 $x_{i+1}$,上式可写为更具对称性的等价形式:
\begin{equation}\label{LPaODE_eq1}
\frac{\dd x_1}{X_1}=\cdots=\frac{\dd x_{n+1}}{X_{n+1}}
\end{equation}
其中,$X_i$ 是变量 $x_1,\cdots,x_{n+1}$ 的函数。

当常微分方程组写成\autoref{LPaODE_eq1} 时,可看出这 $n+1$ 个变量是等价的,并没有特定哪个变量作为自变量。在新的记号下,方程组的积分是(\autoref{IntGen_def1}~\upref{IntGen})
\begin{equation}
\varphi_i(x_1,\cdots,x_{n+1})=C_i,\quad i=1,\cdots,n
\end{equation}
\subsection{从常微分方程组到线性偏微分方程}
在词条“一般积分\upref{IntGen}”的末尾,已经知道,把方程组的解代入其积分里面,积分便是常数。设 $\varphi(x_1,\cdots,x_{n+1})$ 是方程组的积分,不失一般性设 $x_1$ 是自变量,而 $x_2,\cdots,x_{n+1}$ 是 $x_1$ 的函数,它们是方程组的解。于是 $\varphi(x_1,\cdots,x_{n+1})=C$ 。这就是说,代入 $x_2,\cdots,x_{n+1}$ 后,应当消去自变量 $x_1$\footnote{这是因为代入后函数值是常数,意味着代入后的函数不显含 $x_1$}。于是 $\varphi$ 对 $x_1$ 进行全微商应等于0:
\begin{equation}
\pdv{\varphi}{x_1}+\pdv{\varphi}{x_2}\dv{x_2}{x_1}+\cdots+\pdv{\varphi}{x_{n+1}}\dv{x_{n+1}}{x_1}=0
\end{equation}
或写成
\begin{equation}\label{LPaODE_eq2}
\pdv{\varphi}{x_1}\dd x_1+\pdv{\varphi}{x_2}\dd{x_2}+\cdots+\pdv{\varphi}{x_{n+1}}\dd{x_{n+1}}=0
\end{equation}

由\autoref{LPaODE_eq1} ,无论代入方程组哪个解,$\dd x_i$ 都与 $X_i$ 的大小成正比,于是\autoref{LPaODE_eq2} 可写成:
\begin{equation}\label{LPaODE_eq3}
\pdv{\varphi}{x_1}X_1+\pdv{\varphi}{x_2}X_2+\cdots+\pdv{\varphi}{x_{n+1}}X_{n+1}=0
\end{equation}
根据常微分方程组初始条件的任意性,若我们取方程组的所有解,变量 $x_1,\cdots,x_{n+1}$ 就可能取任意的值,而对方程组任意解,$\varphi$ 都满足\autoref{LPaODE_eq3} ,所以函数 $\varphi(x_1,\cdots,x_{n+1})$ 应当恒满足\autoref{LPaODE_eq3} 。于是证得下面定理一部分
\begin{theorem}{}
 $\varphi(x_1,\cdots,x_n)=C$ 是方程组\autoref{LPaODE_eq1} 的积分,当且仅当函数 $\varphi(x_1,\cdots,x_n)$ 满足偏微分方程\autoref{LPaODE_eq3} 。
\end{theorem}
 \textbf{证明:}现在来证明定理的另一部分。设 $\varphi(x_1,\cdots,x_n)$ 是偏微分方程\autoref{LPaODE_eq3} 的解,现在代入\autoref{LPaODE_eq4} 的解于 $\varphi$,于是
 \begin{equation}
 \dd\varphi=\pdv{\varphi}{x_1}\dd x_1+\pdv{\varphi}{x_2}\dd{x_2}+\cdots+\pdv{\varphi}{x_{n+1}}\dd{x_{n+1}}
 \end{equation}
 等价于
 \begin{equation}
 \dd\varphi=\lambda\qty(\pdv{\varphi}{x_1}X_1+\pdv{\varphi}{x_2}X_2+\cdots+\pdv{\varphi}{x_{n+1}}X_{n+1})
 \end{equation}
 其中已设\autoref{LPaODE_eq1} 的比例系数为 $\lambda$。于是由\autoref{LPaODE_eq3} 
 \begin{equation}
 \dd\varphi=0
 \end{equation}
 而一阶微分具有形式不变性(链接),即将函数的变量当成自变量或自变量的函数,函数的一阶微分形式都是一样的。在现在的情形,$\varphi$ 就是一个自变量的函数,不失一般性设为 $x_1$,于是 

 \textbf{证毕!}