% 麦克斯韦—玻尔兹曼分布的数值模拟

\begin{issues}
\issueDraft
\end{issues}

\pentry{二体碰撞\upref{TwoCld}}

最简单算法: 定时间步长, 若发现某两个小球重合,或与墙重合, 则就地完全弹性碰撞.

一些变体
\begin{itemize}
\item 支持三维
\item 支持重力
\item 支持质量和大小不同的若干种小球
\item 真随机碰撞(似乎已完成): 随机找出一对小球进行碰撞, 无论他们在哪里
\end{itemize}

一些优化
\begin{itemize}
\item (先不做)把盒子形划分成小盒子, 每一步把小球先按照小盒子分类, 然后再在每个盒子中判断两小球是否碰撞. 若一个小球出现在边角处, 则它同时属于多个小盒子.
\item (先不做)每一步找出速度最快的小球, 确保时间步长小于某长度除以该速度.
\end{itemize}

\begin{lstlisting}[language=matlab]
% 两个粒子的完全弹性碰撞
% size(v1)=size(v2)=size(m1)=size(2)=[N,3];
% A 是碰撞平面的法向量
% 当不输入 m1, m2 的时候, 默认他们相等.
function [v11, v22] = collision(v1, v2, A, m1, m2)
A = A/norm(A);
% 质心的速度
vc = (m1*v1+m2*v2)/(m1+m2);
% 质心系中的速度
vr1 = v1-vc;
vr2 = v2-vc;
% 计算关于碰撞平面的反射
vr11 = reflection(vr1,A);
vr22 = reflection(vr2,A);
% 原参考系中的速度
v11 = vr11+vc;
v22 = vr22+vc;
end
\end{lstlisting}

\begin{lstlisting}[language=matlab]
% 反射
% A 是法向量. 
% size(v1) = [N,3];
% v1 是入射的方向, v2 是反射的方向
function v2 = reflection(v1,A)
vnorm = sqrt(v1' * v1);
v1unit = v1 / vnorm;
cosine = v1unit' * A;
v2 = v1 - (2*vnorm*cosine)*A;
end
\end{lstlisting}
