% 2010 年计算机学科专业基础综合全国联考卷
% keys 考研 计算机 全国联考

\subsection{一、单项选择题}
第1~40 小题,每小题2 分,共80 分.下列每题给出的四个选项中,只有一个选项最符合试题要求.

1. 若元素a、b、c、d、e、f 依次进栈,允许进栈、退栈操作交替进行,但不允许连续三次进行退栈操作,则不.可能得到的出栈序列是______. \\
A. d c e b f a $\quad$ B. c b d a e f $\quad$ C. b c a e f d $\quad$ D. a f e d c b

2. 某队列允许在其两端进行入队操作,但仅允许在一端进行出队操作.若元素a、b、c、d、e 依次入此队列后再进行出队操作,则不.可能得到的出队序列是______ \\
A. b a c d e $\quad$ B. d b a c e $\quad$ C. d b c a e $\quad$ D. e c b a d

3. 下列线索二叉树中(用虚线表示线索),符合后序线索树定义的是______.\\
\begin{figure}[ht]
\centering
\includegraphics[width=14.25cm]{./figures/CSN10_1.png}
\caption{第3题图} \label{CSN10_fig1}
\end{figure}

4. 在右图所示的平衡二叉树中,插入关键字48 后得到一棵新平衡二叉树.在新平衡二叉树中,关键字37 所在结点的左、右子结点中保存的关键字分别是______. \\
\begin{figure}[ht]
\centering
\includegraphics[width=5cm]{./figures/CSN10_2.png}
\caption{第4题图} \label{CSN10_fig2}
\end{figure}
A.13,48 $\quad$ B.24,48 $\quad$ C.24,53 $\quad$ D、24,90

5. 在一棵度为4的树T 中,若有20个度为4 的结点,10个度为3 的结点,1个度为2 的结点,10个度为1的结点,则树T 的叶结点个数是______. \\
A.41 $\quad$ B.82 $\quad$ C.113 $\quad$ D.122

6. 对n(n≥2)个权值均不相同的字符构造成哈夫曼树.下列关于该哈夫曼树的叙述中,\textbf{错误}的是______ \\
A.该树一定是一棵完全二叉树. \\
B.树中一定没有度为1 的结点. \\
C.树中两个权值最小的结点一定是兄弟结点. \\
D.树中任一非叶结点的权值一定不小于下一层任一结点的权值.

7. 若无向图G=(V, E)中含有7 个顶点,要保证图G 在任何情况下都是连通的,则需要的边数最少是_____. \\
A.6 $\quad$ B.15 $\quad$ C.16 $\quad$ D.21

8. 对下图进行拓扑排序,可以得到不同的拓扑序列的个数是_____. \\
\begin{figure}[ht]
\centering
\includegraphics[width=10cm]{./figures/CSN10_3.png}
\caption{第8题图} \label{CSN10_fig3}
\end{figure}
A.4 $\quad$ B.3 $\quad$ C.2 $\quad$ D.1

9. 已知一个长度为16的顺序表L,其元素按关键字有序排列.若采用折半查找法查找一个L中不存在的元素,则关键字的比较次数最多的是_____. \\
A.4 $\quad$ B.5 $\quad$ C.6 $\quad$ D.7

10. 采用递归方式对顺序表进行快速排序.下列关于递归次数的叙述中,正确的是______. \\
A.递归次数与初始数据的排列次序无关. \\
B.每次划分后,先处理较长的分区可以减少递归次数. \\
C.每次划分后,先处理较短的分区可以减少递归次数. \\
D.递归次数与每次划分后得到的分区的处理顺序无关.

11. 对一组数据(2,12,16,88,5,10)进行排序,若前三趟排序结果如下: \\
第一趟排序结果:2,12,16,5,10,88 \\
第二趟排序结果:2,12,5,10,16,88 \\
第三趟排序结果:2,5,10,12,16,88 \\
则采用的排序方法可能是______. \\
A.起泡排序 $\quad$ B.希尔排序 $\quad$ C.归并排序 $\quad$ D.基数排序

12. 下列选项中,能缩短程序执行时间的措施是. \\
Ⅰ. 提高CPU 时钟频率 \\
Ⅱ. 优化数据通路结构 \\
Ⅲ. 对程序进行编译优化 \\
A.仅Ⅰ 和Ⅱ $\quad$ B.仅Ⅰ 和Ⅲ $\quad$ C.仅Ⅱ 和Ⅲ $\quad$ D.Ⅰ 、Ⅱ 和Ⅲ

13. 假定有4个整数用8位补码分别表示$r1=FEH$,$r2=F2H$,$r3=90H$,$r4=F8H$,若将运算结果存放在一个8位
寄存器中,则下列运算中会发生溢出的是. \\
A.r1 x r2 $\quad$ B.r2 x r3 $\quad$ C.r1 x r4 $\quad$ D.r2 x r4

14. 假定变量$i$、$f$和$d$的数据类型分别为int,float 和double(int 用补码表示,float 和double分别用IEEE754单精度和双精度浮点数格式表示),已知$i=785$,$f=1.5678e3$,$d=1.5e100$.若在$32$位机器中执行下列关系表达式,则结果为“真”的是. \\
(I)i == (int)(float)i $\quad$ (II)f == (float)(int)f \\
(III)f == (float)(double)f $\quad$ (IV)(d+f)-d == f \\
A.仅I 和II $\quad$ B.仅I 和III $\quad$ C.仅II 和III $\quad$ D.仅III 和IV

15.假定用若干个2kx4 位的芯片组成一个8kx8位的存储器,则地址0B1FH所在芯片的最小地址是. \\
A.0000H $\quad$ B.0600H $\quad$ C.0700H $\quad$ D.0800H

16. 下列有关RAM和ROM的叙述中,正确的是. \\
I RAM是易失性存储器,ROM是非易失性存储器 \\
II RAM和ROM都采用随机存取方式进行信息访问 \\
III RAM和ROM都可用作Cache \\
IV RAM和ROM都需要进行刷新 \\
A.仅I 和II $\quad$ B.仅II 和III $\quad$ C.仅I,II 和IV $\quad$ D.仅II,III 和IV

18. 下列寄存器中,汇编语言程序员可见的是. \\
A.存储器地址寄存器(MAR) $\quad$ B.程序计数器(PC) \\
C.存储器数据寄存器(MDR) $\quad$ D.指令寄存器(IR)

19. 下列选项中,\textbf{不}会引起指令流水线阻塞的是. \\
A.数据旁路(转发) $\quad$ B.数据相关 \\
C.条件转移 $\quad$ D.资源冲突

20. 下列选项中的英文缩写均为总线标准的是______. \\
A.PCI、CRT、USB、EISA \\
B.ISA、CPI、VESA、EISA \\
C.ISA、SCSI、RAM、MIPS \\
D.ISA、EISA、PCI、PCI-Express

21. 单级中断系统中,中断服务程序内的执行顺序是______. \\
I 保护现场 $\quad$ II 开中断 $\quad$ III 关中断 $\quad$ IV 保存断点 \\
V 中断事件处理 $\quad$ VI 恢复现场 $\quad$ VII 中断返回 \\
A.I->V->VI->II->VII $\quad$ B.III->I->V->VII \\
C.III->IV->V->VI->VII $\quad$ D.IV->I->V->VI->VII

22. 假定一台计算机的显示存储器用DRAM 芯片实现,若要求显示分辨率为1600*1200,颜色深度为24位,
帧频为85Hz,显存总带宽的50\%用来刷新屏幕,则需要的显存总带宽至少约为______. \\
A.245Mbps $\quad$ B.979Mbps $\quad$ C.1958Mbps $\quad$ D.7834Mbps

23. 下列选项中,操作系统提供给应用程序的接口是_____. \\
A.系统调用 $\quad$ B.中断 $\quad$ C.库函数 $\quad$ D.原语

24. 下列选项中,导致创建新进程的操作是______. \\
Ⅰ 用户登录成功 $\quad$ Ⅱ 设备分配 $\quad$ Ⅲ 启动程序执行 \\
A.仅Ⅰ 和Ⅱ $\quad$ B.仅Ⅱ 和Ⅲ $\quad$ C.仅Ⅰ 和Ⅲ $\quad$ D.Ⅰ 、Ⅱ 和Ⅲ

25. 设与某资源关联的信号量初值为3,当前值为1.若M表示该资源的可用个数,N表示等待该资源的进程
数,则M、N 分别是______. \\
A.0、1 $\quad$ B.1、0 $\quad$ C.1、2 $\quad$ D.2、0

26. 下列选项中,降低进程优先级的合理时机是_____. \\
A. 进程的时间片用完 \\
B. 进程刚完成I/O,进入就绪列队 \\
C. 进程长期处于就绪列队中 \\
D. 进程从就绪态转为运行态 \\

27. 进程P0和P1的共享变量定义及其初值为 \\
\begin{lstlisting}[language=cpp]
boolean flag[2];
int turn = 0;
flag[0] = FALSE;
flag[1] = FALSE;
\end{lstlisting}
若进程P0和P1访问临界资源的类C伪代码实现如下:
\begin{lstlisting}[language=cpp]
void P0() // 进程P0
{
  while(TRUE)
  {
    flag[0]=TRUE; turn=1;
    while(flag[1]&&(turn==1))
    ;
    临界区;
    flag[0]=FALSE;
  }
}
\end{lstlisting}

\begin{lstlisting}[language=cpp]
void P1() // 进程P1
{
  while(TRUE)
  {
    flag[1]=TRUE; turn=0;
    while(flag[0]&&(turn==0))
    ;
    临界区;
    flag[1]=FALSE;
  }
}
\end{lstlisting}
则并发执行进程P0和P1时产生的情形是______.  \\
A. 不能保证进程互斥进入临界区,会出现“饥饿”现象 \\
B. 不能保证进程互斥进入临界区,不会出现“饥饿”现象 \\
C. 能保证进程互斥进入临界区,会出现“饥饿”现象 \\
D. 能保证进程互斥进入临界区,不会出现“饥饿”现象

28. 某基于动态分区存储管理的计算机,其主存容量为55MB(初始为空闲),采用最佳适配(Best Fit)算法,分配
和释放的顺序为:分配15MB,分配30MB,释放15MB,分配8MB,分配6MB,此时主存中最大空闲分区的大小是______. \\
A.7MB $\quad$ B.9MB $\quad$ C.10MB $\quad$ D.15MB

29. 某计算机采用二级页表的分页存储管理方式,按字节编址,页大小为210字节,页表项大小为2字节,逻辑
地址结构为:  \\
页目录号,页号,页内偏移量 \\
逻辑地址空间大小为$2^{16}$页,则表示整个逻辑地址空间的页目录表中包含表项的个数\textbf{至少}是______. \\
A. 64 $\quad$ B. 128 $\quad$ C. 256 $\quad$ D. 512

30. 设文件索引节点中有7个地址项,其中4个地址项是直接地址索引,2个地址项是一级间接地址索引,1个地址项是二级间接地址索引,每个地址项大小为4字节.若磁盘索引块和磁盘数据块大小均为256字节,则可表示的单个文件最大长度是______. \\
A.33 KB $\quad$ B.519 KB $\quad$ C.1 057 KB $\quad$ D.16 513 KB

31. 设置当前工作目录的主要目的是_______. \\
A.节省外存空间 \\
B.节省内存空间 \\
C.加快文件的检索速度 \\
D.加快文件的读/写速度

32. 本地用户通过键盘登陆系统时,首先获得键盘输入信息的程序是______.  \\
A.命令解释程序 \\
B.中断处理程序 \\
C.系统调用服务程序 \\
D.用户登录程序

33. 下列选项中,\textbf{不}属于网络体系结构所描述的内容是______.  \\
A.网络的层次  \\
B.每一层使用的协议  \\
C.协议的内部实现细节  \\
D.每一层必须完成的功能

34. 在下图所示的采用“存储-转发”方式的分组交换网络中,所有链路的数据传输速率为100Mbps,分组大小为
1000B,其中分组头大小为20B.若主机H1 向主机H2 发送一个大小为980 000B 的文件,则在不考虑分组
拆装时间和传播延迟的情况下,从 H1发送开始到 H2接收完为止,需要的时间\textbf{至少}是______.  \\
\begin{figure}[ht]
\centering
\includegraphics[width=12.5cm]{./figures/CSN10_4.png}
\caption{第34题图} \label{CSN10_fig4}
\end{figure}
A.80 ms $\quad$ B.80.08 ms
C.80.16 ms $\quad$ D.80.24 ms

35. 某自治系统内采用RIP协议,若该自治系统内的路由器R1收到其邻居路由器R2的距离矢量,距离矢量中
包含信息<net1, 16>,则能得出的结论是______.  \\
A.R2可以经过R1到达net1,跳数为17  \\
B.R2可以到达net1,跳数为16  \\
C.R1可以经过R2到达net1,跳数为17  \\
D.R1不能经过R2到达net1  \\

36. 若路由器R因为拥塞丢弃IP分组,则此时R 向发出该IP分组的源主机发送的ICMP报文类型是______ \\
A.路由重定向 $\quad$ B.目的不可达
C.源点抑制 $\quad$ D.超时

37. 某网络的IP地址空间为192.168.5.0/24,采用定长子网划分,子网掩码为255.255.255.248,则该网络中的最大子网个数、每个子网内的最大可分配地址个数分别是______.  \\
A.32,8 $\quad$ B.32,6 \\
C.8,32 $\quad$ D.8,30

38. 下列网络设备中,能够抑制广播风暴的是______. \\
Ⅰ 中继器Ⅱ 集线器Ⅲ 网桥Ⅳ 路由器
A.仅Ⅰ 和Ⅱ $\quad$ B.仅Ⅲ \\
C.仅Ⅲ 和Ⅳ $\quad$ D.仅Ⅳ

39. 主机甲和主机乙之间已建立了一个TCP连接,TCP最大段长度为1000字节.若主机甲的当前拥塞窗口为4000字节,在主机甲向主机乙连续发送两个最大段后,成功收到主机乙发送的第一个段的确认段,确认段中通告的接收窗口大小为2000字节,则此时主机甲还可以向主机乙发送的最大字节数是______. \\
A.1 000 $\quad$ B.2 000
C.3 000 $\quad$ D.4 000

40. 如果本地域名服务器无缓存,当采用递归方法解析另一网络某主机域名时,用户主机、本地域名服务器发送的域名请求消息数分别为______. \\
A.一条、一条 $\quad$ B.一条、多条
C.多条、一条 $\quad$ D.多条、多条

\subsection{二、综合应用题}
第41~47题,共70分.

41. (10 分)将关键字序列(7、8、30、11、18、9、14)散列存储到散列表中.散列表的存储空间是一个下标从0开始的一维数组,散列函数为:H(key) = (keyx3)MOD 7,处理冲突采用线性探测再散列法,要求装填(载)因子为0.7. \\
(1) 请画出所构造的散列表.  \\
(2) 分别计算等概率情况下查找成功和查找不成功的平均查找长度.

42. (13 分)设将$n(n>1)$个整数存放到一维数组$R$中.试设计一个在时间和空间两方面都尽可能高效的算法.将$R$中保存的序列循环左移$p(0<p<n)$个位置,即将R中的数据由$(X_0, X_1, ..., X_{n-1})$变换为$(Xp,Xp+1, ..., Xn-1, X0, X1, ..., Xp-1)$.要求:  \\
⑴ 给出算法的基本设计思想.  \\
⑵ 根据设计思想,采用$C$或$C++$或$JAVA$语言描述算法,关键之处给出注释.  \\
⑶ 说明你所设计算法的时间复杂度和空间复杂度.

43. (11 分)某计算机字长为16位,主存地址空间大小为128KB,按字编址.采用单字长指令格式,指令各
字段定义如下: \\
\begin{figure}[ht]
\centering
\includegraphics[width=14.25cm]{./figures/CSN10_5.png}
\caption{第43题图1} \label{CSN10_fig5}
\end{figure}
转移指令采用相对寻址方式,相对偏移量用补码表示,寻址方式定义如下: \\
\begin{figure}[ht]
\centering
\includegraphics[width=14.25cm]{./figures/CSN10_6.png}
\caption{第43题图2} \label{CSN10_fig6}
\end{figure}

请回答下列问题: \\
⑴ 该指令系统最多可有多少条指令?该计算机最多有多少个通用寄存器?存储器地址寄存器(MAR)和
存储器数据寄存器(MDR)至少各需要多少位? \\
⑵ 转移指令的目标地址范围是多少? \\
⑶ 若操作码$0010B$表示加法操作(助记符为$add$),寄存器$R4$和$R5$的编号分别为$100B$和$101B$,$R4$的内容为$1234H$,$R5$的内容为5$678H$,地址$1234H$中的内容为$5678H$,地址$5678H$中的内容为$1234H$,则汇编语言为“$add (R4), (R5)+$”(逗号前为源操作数,逗号后为目的操作数)对应的机器码是什么(用十六进制表示)?该指令执行后,哪些寄存器和存储单元中的内容会改变?改变后的内容是什么?

44. (12 分)某计算机的主存地址空间大小为256MB,按字节编址.指令Cache和数据Cache分离,均有8
个Cache行,每个Cache行大小为64B,数据Cache采用直接映射方式.现有两个功能相同的程序A 和B,
其伪代码如下所示: \\
\begin{lstlisting}[language=cpp]
程序A:
int a[256][256]
……
int sum_array1()
{
  int i, j, sum=0;
  for(i=0; i<256; i++)
    for(j=0; j<256; j++)
      sum += a[i][j];
  return sum;
}
\end{lstlisting}

\begin{lstlisting}[language=cpp]
程序B:
int a[256][256]
……
int sum_array2()
{
  int i, j, sum=0;
    for(j=0; j<256; j++)
      for(i=0; i<256; i++)
        sum += a[i][j];
  return sum;
}
\end{lstlisting}

假定int类型数据用32位补码表示,程序编译时i,j,sum均分配在寄存器中,数组a按行优先方式存放,其首地址为320(十进制数).请回答下列问题,要求说明理由或给出计算过程. \\
(1) 若不考虑用于cache 一致性维护和替换算法的控制位,则数据Cache 的总容量为多少? \\
(2) 数组元素a[0][31]和a[1][1]各自所在的主存块对应的Cache 行号分别是多少(Cache 行号从0开始)? \\
(3) 程序A和B的数据访问命中率各是多少?哪个程序的执行时间更短?

45. (7 分)假设计算机系统采用CSCAN(循环扫描)磁盘调度策略,使用2KB的内存空间记录16384个磁盘块的空闲状态. \\
(1) 请说明在上述条件下如何进行磁盘块空闲状态的管理. \\
(2) 设某单面磁盘旋转速度为每分钟6000 转,每个磁道有100 个扇区,相邻磁道间的平均移动时间为1ms.
若在某时刻,磁头位于100 号磁道处,并沿着磁道号增大的方向移动(如下图所示),磁道号请求队列为50,
90,30,120,对请求队列中的每个磁道需读取1 个随机分布的扇区,则读完这4 个扇区点共需要多少时间?要求给出计算过程. \\
(3) 如果将磁盘替换为随机访问的Flash 半导体存储器(如U 盘、SSD 等),是否有比CSCAN 更高效的磁
盘调度策略?若有,给出磁盘调度策略的名称并说明理由;若无,说明理由.

