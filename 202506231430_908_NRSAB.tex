% 尼尔斯·阿贝尔(综述)
% license CCBYSA3
% type Wiki

本文根据 CC-BY-SA 协议转载翻译自维基百科 \href{https://en.wikipedia.org/wiki/Niels_Henrik_Abel}{相关文章}。

尼尔斯·亨利克·阿贝尔(Niels Henrik Abel,/ˈɑːbəl/ AH-bəl,挪威语发音:[ˌnɪls ˈhɛ̀nːɾɪk ˈɑ̀ːbl̩],1802年8月5日-1829年4月6日)是挪威著名数学家,在多个数学领域做出了开创性的贡献。[1] 他最著名的成果是第一个完整地证明了一般五次方程无法用根式求解的定理。这个问题曾是当时最重要的未解难题之一,悬而未决长达250多年。[2] 此外,他还是椭圆函数领域的创新者,并发现了阿贝尔函数。他是在贫困中取得这些发现的,26岁时因肺结核早逝。

他的大部分研究成果都集中在六七年间完成。[3] 法国数学家查尔斯·埃尔米特曾评价说:“阿贝尔留给数学家的东西,足够他们研究五百年。”[3][4] 另一位法国数学家阿德里安-玛丽·勒让德则说:“这个年轻的挪威人真是个天才!”[5]
\subsection{生平}
\subsubsection{早年生活}
尼尔斯·亨利克·阿贝尔出生于挪威内斯特兰,是牧师索伦·乔治·阿贝尔和安娜·玛丽·西蒙森的第二个孩子,出生时为早产。阿贝尔出生时,家人住在芬内伊的牧师住宅。许多迹象表明,他可能出生在邻近的教区,因为他的父母在他出生当年的七八月曾作为客人住在内斯特兰的县官家中。[6][注1]

阿贝尔的父亲索伦·乔治·阿贝尔拥有神学和哲学学位,当时担任芬内伊的牧师。索伦的父亲、阿贝尔的祖父汉斯·马蒂亚斯·阿贝尔也是一位牧师,服务于里瑟尔附近的耶尔斯塔教堂。索伦在耶尔斯塔度过了童年,也曾在那里担任副牧师。1804年他父亲去世后,索伦接任耶尔斯塔教区的牧师一职,举家迁往当地。阿贝尔家族起源于石勒苏益格,17世纪迁居挪威。

19世纪90年代的耶尔斯塔教堂与牧师住宅明信片
明信片中牧师住宅的主楼与阿贝尔居住时为同一建筑

安娜·玛丽·西蒙森来自里瑟尔,她的父亲尼尔斯·亨利克·萨克西尔·西蒙森是当地的商人兼船主,被认为是里瑟尔最富有的人。安娜·玛丽在相对奢华的环境中长大,经历了两个继母的抚养。她在耶尔斯塔的牧师住宅中喜欢组织舞会和社交聚会。有诸多证据显示,她很早便染上了酒瘾,对孩子的教育几乎毫不上心。尼尔斯·亨利克和他的兄弟们由父亲亲自教授,用的是手写的教科书。在一本数学书中,有一张加法表写着:“1+0=0”。[6]
