% 圆锥曲线的统一定义(高中)
% keys 准线|第二定义|焦点|圆锥曲线|焦点-准线定义
% license Usr
% type Tutor

\begin{issues}
\issueDraft
\end{issues}

\pentry{圆锥曲线与圆锥\nref{nod_ConSec}}{nod_55cd}

古希腊时期,人们通过截取圆锥面来研究圆、椭圆、抛物线和双曲线等曲线。尽管这种方式直观,而且给予这些曲线同样的来源,但在研究各类曲线时,仍是分别对待,研究各自的性质。随着坐标系的引入,数学家们逐渐发现,这些看似不同的曲线,其实可以在引入一条直线后,通过一个简洁而优雅的定义统一描述。这一定义不仅在解析几何中揭示了圆锥曲线的本质,也在射影几何等更高层次的研究中带来了意想不到的收获。

可惜的是,这部分内容在高中阶段已被完全删除。为带给读者更完整的视角与更新的体验,本文将从这一统一定义出发,探索圆锥曲线的几何构造与背后的深层结构。


\subsection{圆锥曲线的焦点-准线定义}

利用准线与焦点得到的。提供了一个统一的视角来看待

\textbf{圆锥曲线的焦点-准线定义(Focus-Directrix Definition of Conic Sections)}。

\begin{definition}{圆锥曲线的焦点-准线定义}
平面上到一个定点与到一条定直线的距离之比为定值的点构成的图像称为\textbf{圆锥曲线(conic section)}。

其中,定点称为圆锥曲线的\textbf{焦点(focus)},定直线称为圆锥曲线的\textbf{准线(directrix)},二者互相对应。比值称作圆锥曲线的\textbf{离心率(eccentricity)},通常记为$e$ 。特别地:
\begin{itemize}
\item 当 $e = 0$ 时,轨迹称为\textbf{圆(circle)}。
\item 当 $0 < e < 1$ 时,轨迹称为\textbf{椭圆(ellipse)}。
\item 当 $e = 1$ 时,轨迹称为\textbf{抛物线(parabola)}。
\item 当 $e > 1$ 时,轨迹称为\textbf{双曲线(hyperbola)}。
\end{itemize}
\end{definition}

显然,定点到定直线的垂线为圆锥曲线的对称轴。

射影几何如何看待圆锥曲线。

射影几何中的视角使我们能够用一种统一且优雅的方式看待圆锥曲线。在欧几里得几何中,我们通常将抛物线、双曲线和椭圆视为彼此不同的三类曲线,因为它们的代数表达式不同,图像形状也不相同。但在射影几何中,这些差异被看作是坐标选择与观察角度所导致的表象变化,它们在更本质的层面上是一类对象的不同表现:它们都是圆锥曲线。

从射影几何的角度看,圆锥曲线定义为一个圆锥面与一个平面相交的轨迹。这个定义在欧几里得空间中也成立,但射影几何更进一步地指出:在射影平面中,所有非退化的圆锥曲线都是射影等价的。这意味着我们可以通过一个合适的射影变换(即坐标的线性变换加上归一化),将任意一个圆锥曲线变为另一个圆锥曲线——比如将一个椭圆变为一个双曲线或抛物线。

换句话说:
	•	椭圆是在射影平面中与无穷远直线没有实交点的圆锥曲线;
	•	双曲线是在射影平面中与无穷远直线有两个实交点的圆锥曲线;
	•	抛物线是恰好与无穷远直线有一个交点的极限情形。

这种分类在射影几何中失去了意义,因为无穷远直线被作为与其他直线同等地位来处理,不再是“例外的部分”。因此,抛物线、椭圆和双曲线不再是本质不同的几何对象,而只是一个对象的不同投影或表示。

此外,射影几何还强调了极点与极线的对偶性,并引入了极线极点变换的工具来研究圆锥曲线的性质,使得很多命题具有了对称且优美的形式。例如:对于一个给定的圆锥曲线,任意一点都有与之对应的一条极线,反之亦然。这种对偶关系在欧氏几何中并不自然存在。

因此,射影几何所带来的理解是:

	圆锥曲线不是三类不同的曲线,而是一个统一的几何实体的三种视角。

它让我们跳出了直观图形的束缚,从结构上理解几何对象之间的联系,也为代数几何、复几何乃至更高维的几何打下了坚实的基础。

\subsection{性质}

离心率表示“扁平程度”:
$$ e = \frac{c}{a} = \sqrt{1 - \frac{b^2}{a^2}} \in [0, 1) ~.$$
	•	$e = 0$ 是圆,越接近 1 越扁。

\subsection{定义等价性}

