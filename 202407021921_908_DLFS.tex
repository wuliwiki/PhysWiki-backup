% 电离辐射
% license CCBYSA3
% type Wiki

(本文根据 CC-BY-SA 协议转载自原搜狗科学百科对英文维基百科的翻译)

\begin{figure}[ht]
\centering
\includegraphics[width=6cm]{./figures/5477464af9fc0d2c.png}
\caption{电离辐射危害标志} \label{fig_DLFS_11}
\end{figure}

\textbf{致电离辐射(电离辐射)}是指能量足够高而能使原子或分子中的电子解离、也就是使他们电离的辐射。电离辐射通常包括高能亚原子粒子和离子、高速运动的原子(通常大于光速的1\%),以及高能电磁波。

$\gamma$射线、$X$射线,以及紫外线中的高能部分属于电离辐射,紫外线中低能部分以及所有紫外线以下的所有频谱,包括可见光(包括几乎所有类型的激光)、红外线、微波、无线电波则属于非电离辐射。因为不同的分子和原子在具有不同的电离能,紫外线中电离辐射和非电离辐射之间没有明确的边界。一般习惯将边界置于10eV 和33eV之间。

典型的致电离亚原子粒子来自放射性衰变,包括α粒子、β粒子和中子。几乎所有放射性衰变产物都是致电离的,因为放射性衰变的能量通常远远高于电离所需的能量。其他自然产生的亚原子电离粒子有μ子、介子、正电子以及宇宙射线与地球大气层相互作用后产生的产生的次级宇宙射线中的其他粒子。[1][2]宇宙射线是由恒星和某些天体事件产生的,例如超新星爆炸。宇宙射线也可能在地球上产生放射性同位素(例如碳14),其会发生衰变并产生电离辐射。宇宙射线和的衰变放射性的同位素的衰变是地球上自然电离辐射的主要来源,被称为背景辐射。电离辐射也可以通过以下方式人工产生:$X$射线管、粒子加速器以及任何产生放射性同位素人工行为。

电离辐射不能直接被人类感觉到,因此必须使用辐射检测仪器(如盖革计数器)来指示并测量它们。一些高强度的电离辐射可以与物质相互作用发出可见光,如契伦科夫辐射和辐射发光现象。电离辐射可用于各种领域,如医学、核电、研究、制造、建筑和许多其他领域,但如果不采取适当的屏蔽措施,则会对健康造成危害。暴露于电离辐射会对活体组织造成损伤,并可能导致辐射灼伤、细胞损伤、放射病、癌症和死亡。

\subsection{类型}
\begin{figure}[ht]
\centering
\includegraphics[width=6cm]{./figures/42d0cf06dd16f8f8.png}
\caption{阿尔法(α)辐射由快速移动的氦-4(4He)细胞核并被一张纸挡住。Beta(β)由电子组成的辐射被铝板阻挡。伽马(γ)由高能光子组成的辐射在穿透致密材料时最终被吸收。中子(n)辐射由被轻元素阻挡的自由中子组成,如氢,它减缓和/或俘获它们。未示出:银河宇宙射线,由高能带电核组成,例如质子、氦核和被称为 HZE 离子的高电荷核。} \label{fig_DLFS_1}
\end{figure}
\begin{figure}[ht]
\centering
\includegraphics[width=6cm]{./figures/9e60e1ff6d93f21f.png}
\caption{云室是电离辐射可视化的几种方法之一。在粒子物理学的早期,它们主要用于研究,但今天仍然是一种重要的教学工具。} \label{fig_DLFS_2}
\end{figure}
电离辐射根据产生电离效应的粒子或电磁波的性质进行分类。它们有不同的电离机制,可以分为直接电离和间接电离。

\subsubsection{1.1 直接电离辐射}
任何带电的有质量粒子只要具有足够的动能都可以通过库仑力作用直接电离原子,这些粒子包括电子、$\mu$介子、带电的$\pi$介子、质子和失去电子的高能带电原子核。当在相对论性速度下移动时,这些粒子有足够的动能产生电离,但相对论性速度不是必需的。例如,典型的α粒子是致电离的,其速度约5\% 光速,而33 eV(足以产生电离)的电子速度约为1\% 光速。

要最先发现的两种致电离粒子源被赋予了的特殊名称:从原子核中发射的氦核被称为$\alpha$粒子,而通常(但不总是)在相对论性速度发射的电子被称为$\beta$粒子。

天然宇宙射线主要由相对论性质子组成,也包括较重的原子核,如氦离子和其他高能重离子。在大气中,这种粒子通常被空气分子阻止,产生短寿期的带电$\pi$介子,它们很快衰变为$\mu$子,$\mu$子是到达地面(并在一定程度上穿透地面)的一种主要宇宙射线辐射。$\pi$介子也可以在粒子加速器中大量产生。

\textbf{$\alpha$粒子}

$\alpha$粒子由两个质子和两个中子组成,与氦核相同。$\alpha$粒子辐射通常在$\alpha$衰变过程中产生,也可能以其他方式产生。$\alpha$粒子是以希腊字母表中的第一个字母命名。$\alpha$粒子的符号是$\alpha$或$\alpha^{2+}$。因为它们与氦核相同,它们有时也被写作$He^{2+}$或者$^{4}$
$2^{He^2+}$表示氦离子带有+2电荷(缺少两个电子)。如果离子从环境中获得电子,$\alpha$粒子可以写成正常的(电中性的)氦原子$^{4}$
$2^{He^2+}$。

$\alpha$粒子是一种电离能力极强的粒子辐射。当它们由放射性$\alpha$衰变产生时,它们的穿透深度很浅,只能穿透几厘米的空气,无法穿透皮肤死层。三元裂变产生的$\alpha$粒子能量是其三倍,并且在空气中穿透得更远。氦核构成宇宙射线的10-12\%,通常也比核衰变过程产生$\alpha$粒子的能量高得多,因此当能够穿过人体和密度较高的屏蔽层。然而,这种类型的辐射能够被地球大气层大大减弱,其相当于一个大约10米深水构成的防辐射罩。[3]

\textbf{$\beta$粒子}

$\beta$粒子是由某些类型的放射性核素(如钾-40)发射的高能、高速电子或正电子。$\beta$粒子产生过程被称为$\beta$衰变。它们由希腊文字母$\beta$表示。$\beta$衰变有两种形式,$\beta^-$和$\beta^+$,分别代表电子和正电子。[4]

对于具有$\beta$放射性的东西,可以用盖革计数器或其他辐射探测器检测到。当接近$\beta$发射体时,探测器将指示放射性急剧增加。当探测器探头被屏蔽以阻挡$\beta$射线时,放射性指示将显著减少。

高能$\beta$粒子在穿过物质时会通过“韧致辐射”效应产生$X$射线或“次级电子”($\delta$射线)。这两种情况都会造成间接电离效应。

当屏蔽$\beta$发射体时,轫致辐射现象需要考虑,因为$\beta$粒子与屏蔽材料的相互作用产生轫致辐射。这种效应在高原子序数的材料中更显著,因此低原子序数的材料被用于屏蔽$\beta$源。

\textbf{正电子和其他类型的反物质}

\textbf{正电子}或者\textbf{反电子}是电子的反粒子或反物质。当低能正电子与低能电子碰撞时,发生湮灭现象产生成两个或更多的$Y$光子。

正电子可由具有$\beta^+$放射性的核衰变(通过弱相互作用)产生,也可由高能光子的电子对效应产生。正电子是医用电离辐射的常用射线源,可用于正电子发射断层扫描中(PETCT)。

由于正电子是带正电荷的粒子,它们也可以通过库仑相互作用直接电离原子。

\textbf{带电原子核}

带电原子核主要来自银河宇宙射线和太阳粒子事件,除了$\alpha$粒子(带电氦核)外在地球上没有自然来源。然而,高能质子、氦核和高能重离子可以被相对较薄的屏蔽层、衣服或皮肤阻止。然而,由此产生的相互作用将产生次级辐射,并导致级联生物效应。例如,如果只有一个组织原子被高能质子撞击并移位,碰撞将在体内引起进一步的相互作用。这被称为“线性能量传输”(LET),它利用了粒子的弹性散射过程。

线性能量传输可以直观理解为为一个台球以动量守恒的方式撞击另一个台球,将两个球都带走,第一个球的能量不相等地分配给两个球。当带电原子核撞击空间中相对运动缓慢的物体原子核时,线性能量传输发生并且通过碰撞产生中子、$\alpha$粒子、低能质子和其他原子核将,这些都会对组织的总吸收剂量做出贡献。[5]

\subsubsection{1.2 间接电离辐射}
间接电离辐射是电中性的,因此不会与物质发生强烈的相互作用。大部分电离效应是由次级电离引起的。

间接电离辐射的一个例子是中子辐射。

\textbf{光子辐射}

\begin{figure}[ht]
\centering
\includegraphics[width=10cm]{./figures/5b806c515e5da559.png}
\caption{不同类型的电磁辐射} \label{fig_DLFS_3}
\end{figure}
即使光子是电中性的,它们也可以通过光电效应和康普顿效应直接电离原子。这些相互作用中的任何一种都会导致原子中的电子以相对论性速度被弹出,将该电子转化为$\beta$粒子(次级$\beta$粒子),从而电离许多其他原子。由于大多数受影响的原子被次级$\beta$粒子直接电离,光子被称为间接电离辐射。[6]

如果光子辐射是由核内的核反应、亚原子粒子衰变或放射性衰变产生的,则称为$\gamma$射线。如果它在原子核外产生,则被称为$X$射线。光子这个通用术语被用来描述这两者。[7][8][9]
\begin{figure}[ht]
\centering
\includegraphics[width=10cm]{./figures/6f9a30288dd018a0.png}
\caption{γ射线对铅的总吸收系数(原子序数82),横坐标为γ射线能量,反映了三种效应的贡献。光电效应在低能时占主导地位。高于5兆电子伏时,粒子对效应开始占据主导地位。} \label{fig_DLFS_4}
\end{figure}
$X$射线的能量通常低于$\gamma$射线,传统上习惯以波长10-11m或能量100keV的光子作为二者分界。[10]这个阈值的确定是由于旧X射线管的能力局限性和对同质异能跃迁缺乏了解导致的。现代的技术和发现导致了$X$射线和$\gamma$射线能量区域相重叠。在许多领域,它们在功能上是相同的,不同之处仅在于辐射的来源。然而,在天文学中,辐射源通常无法可靠地确定,旧的能量区分标准被保留下来,X射线被定义为大约120 eV至120 keV之间,$\gamma$射线是任何能量高于100至120 keV的射线,无论来源如何。天文学绝大多数”伽马射线天文事件“都是已经确认不起源于核放射性过程,而更接近与产生天文学X射线的过程,除了那些是由更高能电子驱动的事件。

光电效应是有机材料与低于100 keV光子相互作用的主要机制,也是最初典型的$X$射线管的原理。当能量超过100 keV时,光子主要通过康普顿散射效应产生电离,随着能量进一步提高到超过5 MeV,则通过粒子对效应产生间接电离。随后相互作用过程中会先后发生的两次康普顿散射。在两个散射事件中,$\gamma$射线将能量传递给电子,并且以不同的方向和更低的能量继续前进。

\textbf{低能光子的能量边界确定}

电离能最低的元素为铯,其店里能为3.89 eV。然而,美国联邦通信委员会将能量大于10 eV(相当于波长124纳米的远紫外线)的光子定义为电离辐射。[11]这大致相当于氧的第一电离能氧和氢的电离能,都约为14 eV。[12]在环保局的一些参考文献中,引用了典型水分子的电离能(33 eV)[13]作为电离辐射生物阈值:该值代表所谓的w值,这是国际辐射单位与测量委员会(ICRU)对“气体中每形成的一对离子消耗的平均能量”的通俗称呼,[14]其包含了电离能以及其他过程损失的能量,如激发。[15]33 eV对应38纳米波长的电磁辐射,接近传统认定的极紫外线和X射线之间分界点:约10 nm波长,125eV。因此,$X$射线总是致电离辐射,但只有极紫外辐射满足所有电离辐射的定义。

如上所述,电离辐射对细胞的生物效应有点类似于具有更广能谱分布的分子损伤辐射,后者电离辐射重叠并包含更大范围的能量的电磁辐射,在某些系统(如叶片中的光合系统)能量下限可达到紫外线区域甚至进入可见光区域。虽然DNA总是容易受到电离辐射的损害,但辐射能量足以激发某些分子键形成嘧啶二聚体,DNA分子也会收到损害,该能量可能接近但小于电离能。一个很好的例子能量在3.1 eV (400 nm)附近的紫外线,由于胶原蛋白中的光反应(在紫外线$B$波段区域内)可以导致DNA中的损伤(例如嘧啶二聚体),从而使未受保护的皮肤被晒伤。这样虽然分子中的电子的激发不足以导致电离,但会产生类似的非热效应,使得中低能紫外线能够对生物组织造成损伤。可见光和最接近可见光能量的紫外线$A$波段已经被证明某种程度上会导致皮肤中形成活性氧,这些电子激发的分子可以造成间接损害,尽管它们不会造成晒伤(红斑)。[16]同电离辐射损伤一样,皮肤中的所有这些效应都超出了简单热效应的范畴。

\textbf{中子}

\begin{figure}[ht]
\centering
\includegraphics[width=10cm]{./figures/73f47c4c7ce4d65d.png}
\caption{辐射相互作用:伽马射线由波浪线表示,带电粒子和中子由直线表示。小圆圈显示电离发生的位置。} \label{fig_DLFS_5}
\end{figure}
中子没有电荷,因此通常不会在单一过程或者与物质相互作用中产生直接电离。然而,快中子能够通过线性能量传输与氢中的质子相互作用,使目标区域中物质的原子核发生散射,导致氢原子的直接电离。因此当中子撞击氢核时,会产生质子辐射(快质子)。这些质子本身是致电离辐射,因为它们具有高能量并且带电荷,能够与物质中的电子相互作用。

中子撞击氢以外其他原子核时,发生线性能量传输时转移的能量相对较少。但对于许多被中子撞击的原子核,也会发生非弹性散射。弹性或非弹性散射的发生取决于中子的速度,是快中子还是热中子亦或介于两者之间,同时还取决于它所撞击的原子核及其中子截面。

在非弹性散射中,中子很容易通过中子俘获过程被原子核吸收而产生中子活化。中子以这种方式与大多数类型物质的相互作用通常会产生放射性核。例如,自然界中含量丰富的氧-16 核经历中子活化后,通过质子发射快速衰变形成氮-16 ,进而迅速衰变为氧-16,并发射出高能β射线。这个过程可以写成:

$16^O (n,p)16^N$(快中子能量大于11MeV时可被俘获)

$16^N \to 16^O + \beta^-$(衰变$t_{1/2}$= 7.13秒)

高能$\beta^-$射线进一步与其他原子核相互作用,通过切割致辐射发射高能X射线

$16^O (n,p)16^N$反应是压水堆冷却水产生X射线的主要来源,并对水冷核反应堆运行时的辐射有巨大贡献。

为了最好地屏蔽中子,应使用富含氢的碳氢化合物。

在裂变材料中,次级中子可能产生核链式反应,导致裂变物产生大量电离。

在原子核内,自由中子是不稳定的,平均寿命为14分42秒。自由中子通过发射电子和反电子中微子而衰变为质子,这个过程被称为$\beta$衰变。

在右图中,中子与目标材料的原子碰撞,然后变为反冲原子而产生电离效应。在中子路径的末端,中子被原子核(n, $\gamma$)反应俘获,发射一个中子俘获光子。这样足够的能量可以被称为电离辐射。

\subsection{物理效应}
\begin{figure}[ht]
\centering
\includegraphics[width=10cm]{./figures/29cc0bd1138a432d.png}
\caption{电离空气在来自回旋加速器的粒子电离辐射束周围发出蓝色光} \label{fig_DLFS_6}
\end{figure}

\subsubsection{2.1 核效应}
分子电离可导致辐射分解(化学键断裂),并形成高活性的自由基。即使在最初的辐射停止后,这些自由基也可能与邻近的物质发生化学反应。(例如,由空气电离形成的臭氧导致聚合物的臭氧裂解)。电离辐射还可以通过提供反应所需的活化能来加速现有的化学反应,如聚合和腐蚀。光学材料在电离辐射的作用下变暗。

空气中的高强度电离辐射可以产生可见的电离空气辉光,呈蓝紫色。在临界事故,核爆炸后的蘑茹云周围、或受损核反应堆内部如切尔诺贝利灾难都可以观察到这种辉光。

单原子流体,例如熔融的钠,不存在化学键和晶格结构,因此它们对电离辐射的化学效应免疫。具有非常负的生成焓的简单双原子化合物,如氟化氢,在电离后会迅速自发重整。

\subsubsection{2.3 电效应}
物质的电离会暂时增加它们的电导率,有可能会使电流提高到具有破坏性的水平。这对于电子设备中使用的半导体微电子器件是一个需要特别注意的危险,增大的电流会引入操作误差,甚至永久损坏器件。在高辐射环境,例如核工业和大气外层(外空间)下使用的设备,应具有抗辐照能力,需要通过设计、材料选择和制造工艺来实现这一点。

太空中的质子辐射也可能导致数字电路中的单粒子翻转现象。

电离辐射的电效应在充气辐射探测器中得到利用,例如盖革计数管及电离室。

\subsection{健康影响}
一般来说,电离辐射对生物是有害甚至致命的,但是一些类型的电离辐射具有医学应用,如治疗癌症和甲状腺毒症的放射疗法。

暴露于电离辐射对健康的不利影响主要可分为两大类:
\begin{itemize}
\item 确定性效应(组织损伤反应),很大程度上是由于收到高剂量辐射灼伤后的细胞死亡或功能失常
\item 随机效应,包括癌症和可遗传效应,涉及暴露个体中由于体细胞的突变导致的癌症或其后代中由于生殖细胞突变导致的可遗传疾病。[18]
\end{itemize}
最常见的影响是随机诱发癌症,暴露后潜伏期为几年或几十年。例如,电离辐射是慢性粒细胞白血病的唯一原因。[19]发生这种情况的机制是众所周知的,但是预测风险水平的定量模型仍然有争议。最广泛接受的模型假设电离辐射导致的癌症发病率随有效辐射剂量以每西弗5.5\%的速度增加。如果这个线性模型是正确的,那么自然的背景辐射是对公众健康最危险的辐射源,其次是医学成像。电离辐射的其他随机效应有致畸、认知下降和心脏病。

\subsection{测量学}
下表以国际单位制和非国际单位制显示辐射和剂量。附图为国际放射防护委员会给出的剂量率之间的关系。
\begin{figure}[ht]
\centering
\includegraphics[width=14.25cm]{./figures/13eb0c575f89c4e6.png}
\caption\label{fig_DLFS_7}
\end{figure}

\begin{table}[ht]
\centering
\caption{放射性和检测到的电离辐射之间关系}\label{DLFS}
\begin{tabular}{|c|c|c|c|c}
\hline
\textbf{量} & \textbf{探测器} & \textbf{CGS单位} & \textbf{国际单位制} & \textbf{其他单位} \\
\hline
衰变速度 &  & 居里 & 贝克 &  \\
\hline
粒子通量 & 盖革计数器、正比计数器、闪烁体 & 个厘米$^{2}$ · 秒 & 个米$^{2}$ · 秒 & 每分钟计数,粒子数每厘米$^{2}$每秒\\
\hline
能量通量 & 热释光剂量计,胶片卡剂量计 & 兆电子伏(MeV)厘米$^{2}$ & 焦耳米$^{2}$ &  \\
\hline
辐射能量 & 正比计数器 & 电子伏特 & 焦耳 & \\
\hline
线性能量传输 & 导出量 & 兆电子伏(MeV)厘米 & 焦耳米 & keV微米 \\
\hline
比释动能 & 电离室、半导体探测器、石英纤维剂量计、卡尼沉降计 & esu(静电单位)厘米$^{3}$ & 戈瑞(gray) & 伦琴\\
\hline
吸收剂量 & 热量计 & 拉德(rad) & 戈瑞(gray) & 伦琴当量\\
\hline
等效剂量 & 导出量 & 雷姆(rem) & 西弗(sievert) & \\
\hline
有效剂量 & 导出量 & 雷姆(rem) & 西弗(sievert) & 背景辐射当量时间\\
\hline
承诺剂量 & 导出量 & 雷姆(rem) & 西弗(sievert) & 香蕉等效剂量\\
\hline
\end{tabular}
\end{table}

\subsection{应用}
电离辐射有许多工业、军事和医疗用途。随着时间推移,必须不断地平衡其有用性与危害性。例如,曾经鞋店的店员用x光检查孩子的鞋号,但当人们更好地了解电离辐射的风险时,这种做法就停止了。[20]

中子辐射对核反应堆和核武器至关重要。$X$射线、$\gamma$射线、$\beta$射线和正电子辐射的穿透力用于医学成像、无损检测和各种工业仪表。放射性示踪物用于医疗和工业应用,以及生物和辐射化学。$\alpha$射线用于静电消除器和烟雾探测器。电离辐射的杀菌效果可用于清洗医疗器械、食品辐照和昆虫无菌技术。对碳-14 的测量可以用来确定古代生物(例如有数千年历史的木材)遗骸的年代。

\subsection{ 辐射源}
电离辐射是由核反应、核衰变、甚高温或电磁场中带电粒子的加速产生的。自然来源包括太阳、闪电和超新星爆炸。人工源包括核反应堆、粒子加速器和 $X$射线管。

下表是联合国原子辐射效应科学委员会(UNSCEAR)人类受到的辐射照射类型的分类。

\begin{table}[ht]
\centering
\caption{辐射的类型}\label{DLFS_1}
\begin{tabular}{|c|c|c}
\hline
\textbf{公众照射} &  &  \\
\hline
 & 	一般事件 & 宇宙射线\\
\hline
 &  一般事件 & 地面辐射\\
\hline
自然辐射源 & 增强辐射源 & 	金属矿床开采和熔炼\\
\hline
 &  & 磷酸盐工业\\
\hline
 &  & 采煤和煤炭发电\\
\hline
 &  & 石油和天然气钻井\\
\hline
 &  & 稀土和二氧化钛工业\\
\hline
 &  & 锆和陶瓷工业\\
\hline
 &  & 镭和钍的应用\\
\hline
 &  & 其他暴露情况\\
\hline
人工辐射源 & 和平目的 &	核电生产 \\
\hline
 &  & 核材料和放射性材料的运输\\
\hline
 &  & 核能以外的应用\\
\hline
 & 军事目的 & 核试验\\
\hline
 & & 环境中的残留物。放射性沉降物\\
\hline
历史状况 & & \\
\hline
事故暴露 & & \\
\hline
\textbf{职业辐射} & & \\
\hline
自然辐射源 &  & 空勤人员和航空人员的宇宙射线照射\\
\hline
 & & 采掘和加工业\\
\hline
 & & 天然气和石油开采行业\\
\hline
 & & 矿井以外工作场所的氡暴露\\
\hline
人工辐射源 & 和平目的 & 核能工业\\
\hline
 & & 辐射的医疗用途\\
\hline
 & & 辐射的工业用途\\
\hline
 & & 其他各类用途\\
\hline
 & 军事目的 & 其他暴露的工人\\
\hline
资料来源:辐射科委2008年附件B &  &  \\
\hline
\end{tabular}
\end{table}
国际辐射防护委员会负责管理国际辐射防护体系,该体系为辐射剂量设定了建议限值。

\subsubsection{6.1 本底辐射}
本底辐射同时来自自然和人工来源。

全世界人类受电离辐射的平均照射量约为3mSv (0.3 rem)每年,其中80\%源于自然辐射,其余的20\%来自于人造辐射源,主要为医学成像。在发达国家,人工辐射剂量要高得多,主要是由于CT扫描和核医学的广泛应用。

自然背景辐射有五个主要来源:宇宙辐射、太阳辐射、地表辐射、人体内辐射和放射性氡。

自然背景辐射率随位置变化很大,在某些地点低至1.5 mSv每年,而另一些地方可以超过100 mSv每年。有记录的最高水平的纯自然辐射是90 Gy/h (0.8 Gy/a),位于独居石组成的巴西黑色海滩上。[21]有人居住的地区中背景辐射最高的是伊朗的拉姆萨尔,主要是由于天然放射性石灰石被用作建筑材料。受辐射剂量最大的约2000名居民平均辐射剂量为10mGy每年(1拉德每年),是国际辐射防护委员会建议的公众接触人工辐射源剂量限值的十倍以上。[22]最高辐射剂量位于一所房子内,其外部辐射导致的有效辐射剂量为135 mSv每年,来自氡的待积剂量为640 mSv每年。[23]这比世界平均背景辐射高200多倍。尽管拉姆萨尔居民受到高水平的背景辐射,但没有令人信服的证据表明他们面临更大的健康风险。辐射防委会的建议是保守的,可能过度估计了健康风险。一般来说,辐射安全组织从谨慎行事的角度会推荐最保守的限值,这种谨慎程度是适当的,但不应该被用来制造对背景辐射危险的恐慌。来自背景辐射的辐射危险可能是一个威胁,但与环境中的所有其他因素相比,总体上只是一个微小的风险。

\textbf{宇宙辐射}

地球和地球上的所有生物都不断受到太阳系外辐射的轰击。这种宇宙辐射由相对论性粒子组成:各种带正电荷的原子核(离子),从质子(约占85\%)至质子数 为26的铁核甚以及部分更重的元素(高原子序数的离子被称为高能重离子(HZE离子))。这种辐射的能量远远超过人类最大型粒子加速器所能达到的能量 。宇宙射线与大气发生相互作用产生各种次级辐射,包括$X$射线、$\mu$子、质子、反质子、$\alpha$粒子、介子、电子、正电子和中子等.

宇宙辐射的剂量主要来自$\mu$子、中子和电子,剂量率在世界不同地区有所不同,主要取决于地磁场、海拔和太阳周期。飞机上的宇宙辐射剂量率很高,根据联合国辐射科委2000年报告,航空公司机组人员人均辐射剂量超过其他任何职业,包括核电厂的工作人员。北极或南极附近的高海拔地区宇宙射线剂量率最大,如果机组人员经常在这些地区飞行,他们会接受到更多宇宙射线的辐射。

宇宙射线还包括高能伽马射线,其能量远远高于人类甚至太阳产生的射线能量。

\textbf{地表辐射源}

地球上的大多数材料都含有一些放射性原子,即使其数量很少。来自建筑材料、岩石和土壤中的伽马射线是地表辐射的主要成分。地表辐射的主要放射性核素是钾、铀和钍的同位素。自地球形成以来,这些放射源强度一直在减弱。

\textbf{内部辐射源}

地球上所有构成生命的物质都含有放射性成分。当人类、植物和动物消耗食物、空气和水时,生物体内会积累放射性同位素。一些放射性核素,比如钾-40能发出高能伽马射线,该射线可以由灵敏的电子辐射测量系统测量。这些内部辐射源也归类于天然本底辐射。

\textbf{氡}

自然辐射的一个重要来源是氡气,它从基岩中不断渗出,且由于密度较高,会积聚在通风不良的房屋中。

氡-222 是一种由镭 -226衰变产生的气体。两者都是天然铀衰变链的一部分。铀在世界各地的土壤中以不同的浓度存在。氡是肺癌的第二大诱因,对于不吸烟者来说,则是最大诱因。[24]

\subsection{辐射照射}
\begin{figure}[ht]
\centering
\includegraphics[width=10cm]{./figures/52972bde47603125.png}
\caption{各种剂量的辐射,从微不足道到致命。} \label{fig_DLFS_8}
\end{figure}
有三种标准方法来限制辐照:
\begin{enumerate}
\item \textbf{时间:}对于暴露在自然背景辐射之外的辐射中的人,限制或最小化暴露时间将减少来自辐射源的剂量。
\item \textbf{距离:}根据平方反比定律(在绝对真空中),辐射强度随着距离的增加而急剧下降。[25]
\item \textbf{屏蔽:}空气或皮肤足以充分衰减$\alpha$和$\beta$辐射。铅、混凝土或水的屏障通常用于防护$\gamma$射线和中子等更具穿透性的粒子。一些放射性材料被储存在水下或通过遥控器储存在由厚混凝土或铅衬构成的房间中。有特殊的塑料屏蔽层可以阻挡$\beta$粒子,空气可以阻挡大多数$\alpha$粒子。材料屏蔽辐射的有效性由它的半衰减厚度决定,该厚度是使辐射减少一半的材料厚度。该值由材料本身以及电离辐射的类型和能量确定。一些普遍使用的衰减材料厚度如下,5mm的铝毫米用于屏蔽$\beta$粒子,3英寸的铅层用于屏蔽$\gamma$辐射。
\end{enumerate}
这些都措施对于天然和人工辐射源均有效。对于人造资源,隔离室是减少剂量摄入的主要工具,能够有效地进行屏蔽和并将辐射源与开放环境隔离开。放射性材料被限制在尽可能小的空间内并远离环境,如热室(防辐射)或手套箱(防污染)。医用放射性同位素被放置在封闭的处理设施中,通常是手套箱,而核反应堆在封闭的系统中运行,具有多个屏障来保持放射性材料的容纳。工作室、热室和手套箱的略低于大气压,以防止物质通过空气传播泄漏到开放环境中。

在核冲突或民用核设施泄露事件,民防措施可以减少人群的同位素摄入和职业性辐射来帮助降低辐照剂量。其中一种措施是服用碘化钾片,它可以阻止放射性碘(核裂变的主要放射性同位素产物之一)进入人体的甲状腺。

\subsubsection{7.1 职业性照射}
受到职业性照射的个人在其工作所在国家的监管框架内受到控制,并符合当地对核许可证管理的要求。这些通常基于国际辐射防护委员会的建议。国际辐射防护委员会建议限制人工辐射。对于职业性照射,限制是一年50毫西弗,连续五年最多100毫西弗。[26]

使用剂量计和其他辐射防护仪器仔细监测这些人的受辐照情况,这些仪器将测量放射性颗粒浓度、区域γ剂量读数和放射性污染。需要对剂量进行合法记录。

关注职业性照射的事项主要包括:
\begin{itemize}
\item 航空机组人员(受辐照最多的人群)
\item 工业射线照相术
\item 医学的放射学和核医疗学[26][27]
\item 铀的开采
\item 核电厂和核燃料后处理厂工人
\item 研究实验室(政府、大学和私人)
\end{itemize}
一些人造辐射源通过直接辐射影响人体,称为有效剂量(辐射),而另一些则以放射性污染的形式从内部照射人体,后者被称为待积剂量。

\subsubsection{7.2 公众辐射}
医疗程序,如诊断x射线、核医学、放射治疗是公众受到人造辐射的最重要的来源。主要应用的放射性核素包括碘-131、锝-99m、钴-60、铱-192和铯-137。公众还会受到生活用品中的辐射,例如烟草(钋-210)、可燃燃料(天然气、煤等等。)、电视机、夜光手表和刻度盘(氚)、机场x射线系统,烟雾探测器(镅)、电子管和煤气灯罩(钍)中。

公众受到核燃料循环过程中的辐射较小,该循环包括从铀加工到乏燃料处理的整个过程。由于涉及的剂量极低,这种接触的影响尚未得到可靠的测量。一些反对者应用线性无阈值模型(LNT)计算癌症与剂量的关系并断言,核燃料循环辐射每年导致数百例癌症。

国际辐射防护委员会建议将公众的人工辐射限制在每年1mSv (0.001Sv)的有效剂量,其中不包括医疗和职业辐照。[26]

在核战争中,核武器爆炸和其沉降物的$\gamma$射线都将是辐射的来源。

\textbf{航天}

对于在地球磁场保护之外的宇航员来说,太阳质子事件(SPE)和银河宇宙射线的重粒子辐射是需要注意的。这些高能带电原子核能被地球磁场阻挡,但对于前往月球和地球轨道以外区域的宇航员来说是一个重大健康问题。尽管银河系宇宙射线的绝大部分为质子,其中的带电高能重离子仍然极具破坏性。有证据表明,过去的太阳质子事件的辐射水平对未受保护的宇航员来说是致命的。[28]

\textbf{航空旅行}

与海平面相比,空中旅行使飞机上的人收到宇宙射线和太阳耀斑等来自空间的辐射增加[29][30]诸如Epcard 、CARI、SIEVERT、PCAIRE等软件程序试图模拟空勤人员和乘客的辐照水平。[30]剂量测量结果(非模拟剂量)表明从伦敦希思罗机场到东京成田机场的高纬度极地航线的辐射剂量为每小时6μSv。[30]然而,在太阳活跃期剂量可能会增加。[30]美国联邦航空局要求航空公司向机组人员提供有关宇宙辐射的信息,国际辐射防护委员会对公众建议的辐射限制为不超过1mSv每年。[30]此外,许多航空公司按照欧洲规定不允许怀孕的机组人员登机。[30]美国联邦航空局推荐的怀孕期间辐照限值为1mSv每年,单月不超过0.5mSv。[30]上述信息基于《航天医学基础》(2008年出版)。[30]

\subsubsection{7.3 辐射危险警告标志}
具有危险水平的电离辐射由黄色背景上的三叶形标志表示。这些通常张贴在辐射控制区的边界或由于人类活动导致辐射水平明显高于背景辐射的地方。

红色电离辐射警告标志(ISO 21482)于2007年推出,旨在标识国际原子能机构(IAEA)的1、2和3类放射源,定义为可导致死亡或严重伤害的危险源,包括食品辐照器、癌症放疗仪和工业射线照相设备。该符号将放置在容纳信号源的设备上,作为不要拆除设备或靠近设备的警告。在正常使用情况下,仅当有人试图拆卸设备时,它才是不可见的。该符号将不会位于建筑物检修门、运输包装或集装箱上。[31]
\begin{figure}[ht]
\centering
\includegraphics[width=6cm]{./figures/8875d632be434d3f.png}
\caption\label{fig_DLFS_9}
\end{figure}
\begin{itemize}
\item 电离辐射危害标识
\end{itemize}
\begin{figure}[ht]
\centering
\includegraphics[width=6cm]{./figures/7883ef429539a26d.png}
\caption\label{fig_DLFS_10}
\end{figure}
\begin{itemize}
\item 2007年ISO 的放射性危害标识,针对国际原子能机构(IAEA)规定的1、 2 、3 类危险源,该源能够导致死亡或严重损伤。[31]
\end{itemize}

\subsection{参考文献}
[1]
^Woodside, Gayle (1997). Environmental, Safety, and Health Engineering. US: John Wiley & Sons. p. 476. ISBN 978-0471109327. Archived from the original on 2015-10-19..

[2]
^Stallcup, James G. (2006). OSHA: Stallcup's High-voltage Telecommunications Regulations Simplified. US: Jones & Bartlett Learning. p. 133. ISBN 978-0763743475. Archived from the original on 2015-10-17..

[3]
^每平方厘米一公斤水等于10米水。 Archived 2016-01-01 at the Wayback Machine.

[4]
^"Beta Decay". Lbl.gov. 9 August 2000. Archived from the original on 3 March 2016..

[5]
^高电荷和能量(HZE)离子在1989年9月29日太阳粒子事件中的贡献金明熙;约翰·威尔逊;弗朗西斯·库奇诺塔;西蒙森,丽莎·c;阿特威尔,威廉;弗朗西斯·巴达维;米勒、杰克、美国宇航局约翰逊航天中心;1999年5月,兰利研究中心。.

[6]
^European Centre of Technological Safety. "Interaction of Radiation with Matter" (PDF). Radiation Hazard. Archived (PDF) from the original on 12 May 2013. Retrieved 5 November 2012..

[7]
^Feynman, Richard; Robert Leighton; Matthew Sands (1963). The Feynman Lectures on Physics, Vol.1. USA: Addison-Wesley. pp. 2–5. ISBN 978-0-201-02116-5..

[8]
^L'Annunziata, Michael; Mohammad Baradei (2003). Handbook of Radioactivity Analysis. Academic Press. p. 58. ISBN 978-0-12-436603-9..

[9]
^Grupen, Claus; G. Cowan; S. D. Eidelman; T. Stroh (2005). Astroparticle Physics. Springer. p. 109. ISBN 978-3-540-25312-9..

[10]
^Charles Hodgman, Ed. (1961). CRC Handbook of Chemistry and Physics, 44th Ed. USA: Chemical Rubber Co. p. 2850..

[11]
^Robert F. Cleveland, Jr.; Jerry L. Ulcek (August 1999). "Questions and Answers about Biological Effects and Potential Hazards of Radiofrequency Electromagnetic Fields" (PDF) (4th ed.). Washington, D.C.: OET (Office of Engineering and Technology) Federal Communications Commission. Archived (PDF) from the original on 2011-10-20. Retrieved 2011-12-07..

[12]
^Jim Clark (2000). "Ionisation Energy". Archived from the original on 2011-11-26. Retrieved 2011-12-07..

[13]
^"Ionizing & Non-Ionizing Radiation". Radiation Protection. EPA. 2014-07-16. Archived from the original on 2015-02-12. Retrieved 2015-01-09..

[14]
^"Fundamental Quantities and Units for Ionizing Radiation (ICRU Report 85)". Journal of the ICRU. 11 (1). 2011. Archived from the original on 2012-04-20..

[15]
^Hao Peng. "Gas Filled Detectors" (PDF). Lecture notes for MED PHYS 4R06/6R03 - Radiation & Radioisotope Methodology. MacMaster University, Department of Medical Physics and Radiation Sciences. Archived from the original (PDF) on 2012-06-17..

[16]
^Liebel F, Kaur S, Ruvolo E, Kollias N, Southall MD (2012). "Irradiation of skin with visible light induces reactive oxygen species and matrix-degrading enzymes". J. Invest. Dermatol. 132 (7): 1901–1907. doi:10.1038/jid.2011.476. PMID 22318388..

[17]
^W.-M. Yao; et al. (2007). "Particle Data Group Summary Data Table on Baryons" (PDF). J. Phys. G. 33 (1). Archived from the original (PDF) on 2011-09-10. Retrieved 2012-08-16..

[18]
^ICRP 2007, paragraph 55..

[19]
^Huether, Sue E.; McCance, Kathryn L. (2016-01-22). Understanding pathophysiology (6th ed.). St. Louis, Missouri: Elsevier. p. 530. ISBN 9780323354097. OCLC 740632205..

[20]
^Lewis, Leon; Caplan, Paul E (January 1, 1950). "The Shoe-fitting Fluoroscope as a Radiation Hazard". California Medicine. 72 (1): 26–30 [27]. PMC 1520288. PMID 15408494..

[21]
^United Nations Scientific Committee on the Effects of Atomic Radiation (2000). "Annex B". Sources and Effects of Ionizing Radiation. vol. 1. United Nations. p. 121. Archived from the original on 4 August 2012. Retrieved 11 November 2012..

[22]
^Mortazavi, S.M.J.; P.A. Karamb (2005). "Apparent lack of radiation susceptibility among residents of the high background radiation area in Ramsar, Iran: can we relax our standards?". Radioactivity in the Environment. 7: 1141–1147. doi:10.1016/S1569-4860(04)07140-2. ISBN 9780080441375. ISSN 1569-4860..

[23]
^Sohrabi, Mehdi; Babapouran, Mozhgan (2005). "New public dose assessment from internal and external exposures in low- and elevated-level natural radiation areas of Ramsar, Iran". Proceedings of the 6th International Conference on High Levels of Natural Radiation and Radon Areas. 1276: 169–174. doi:10.1016/j.ics.2004.11.102..

[24]
^"Health Risks". Radon. EPA. Archived from the original on 2008-10-20. Retrieved 2012-03-05..

[25]
^坎普豪森·卡,劳伦斯研究中心。《放射治疗原理》 Archived 2009-05-15 at the Wayback Machine帕斯杜尔·R、瓦格纳·LD、坎普豪森·卡、霍斯金·WJ(编辑)癌症管理:多学科方法 Archived 2013-10-04 at the Wayback Machine。11版。2008..

[26]
^ICRP 2007..

[27]
^Pattison, J.E. (1999). "Finger Doses Received during Samarium-153 Injections". Health Physics. 77 (5): 530–5. doi:10.1097/00004032-199911000-00006. PMID 10524506..

[28]
^"Superflares could kill unprotected astronauts". New Scientist. 21 March 2005. Archived from the original on 27 March 2015..

[29]
^"Effective Dose Rate". NAIRAS (Nowcast of Atmospheric Ionizing Radiation System). Archived from the original on 2016-03-05..

[30]
^Jeffrey R. Davis; Robert Johnson; Jan Stepanek (2008). Fundamentals of Aerospace Medicine. pp. 221–230. ISBN 9780781774666 – via Google Books..

[31]
^"New Symbol Launched to Warn Public About Radiation Dangers". International Atomic Energy Agency (IAEA}. February 15, 2007. Archived from the original on 2007-02-17..