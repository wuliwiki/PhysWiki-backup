% 浙江大学 2000 年 考研 量子力学
% license Usr
% type Note

\textbf{声明}:“该内容来源于网络公开资料,不保证真实性,如有侵权请联系管理员”

\subsection{第一题:(20 分)}
\begin{enumerate}
  \item 下列说法哪个是正确的?不正确的说法给予修正。
  \begin{itemize}
    \item a. 量子力学适用于微观体系,而经典力学适用于宏观体系。
    \item b. 电子是粒子,又是波。
    \item c. 电子是粒子,不是波。
    \item d. 电子是波,不是粒子。
  \end{itemize}
  
  \item a. 厄米算符的定义是什么?算符 $x \frac{d}{dx}$ 是否厄米?
  \begin{itemize}
    \item b. 等式 $e^{\hat g} \cdot e^{\hat f} = e^{\hat g+\hat f}$ 何时成立?何时不成立?
  \end{itemize}
  
  \item 若太阳为一黑体,人所能感受到的太阳光能量的最大波长为 $\lambda_m = 0.48 \\, \mu m$,太阳半径 $R = 7.0 \times 10^8 \\, \text{m}$,太阳质量 $m = 2 \times 10^{30} \\ \text{kg}$,试估算太阳质量由于热辐射而损耗 1\% 所需要的时间。(斯特藩常数 $\sigma = 5.67 \times 10^{12}  \text{W} / (\text{cm}^2  \text{K}^4)$)
\end{enumerate}
\subsection{第二题:(20分)}
若有一粒子,质量为 $m$,在有限深势阱 $V(x) = \begin{cases}
0, & |x| \leq a \\\\
V_0, & |x| > a
\end{cases}$ 中运动,$V_0$ 为某一正常数。

\begin{enumerate}
  \item 试推导出其能量本征值所满足的方程。
  \item 如何求能量本征值?试作出求解本征值的草图。
  \item 若粒子不作一维运动,而是三维运动,$V(r) = \begin{cases}
0, & 0 < r < a \\\\
V_0, & r \geq a
\end{cases}$,试求出至少存在一个本征能的条件。
\end{enumerate}
\subsection{第三题:(20分)}
\begin{enumerate}
    \item 在量子力学中,若 $\hat{H}$ 不显含时间,则力学量 $\hat{A}$ 为守恒量的定义是什么?守恒量 $\hat{A}$ 的本征态有何特点?
    
    \item 本征值简并的概念是如何表达的?一维运动的粒子(势为 $V(x)$),其能级是否简并?
    
    \item 在一维势场 $V(x)$ 中运动的粒子,其动量 $\hat{P_x}$ 是否守恒?
    
    \item 试说出氢原子问题中的量子跃迁的选择定则的内容。
\end{enumerate}
\subsection{第四题: (25 分)}
若一二维谐振子哈密顿量为:
\[\hat{H} = \hat{H}_0 + \hat{H}'~\]
\[\hat{H}_0 = \frac{1}{2\mu} (\hat{p}_x^2 + \hat{p}_y^2) + \frac{1}{2}\mu \omega^2 (x^2 + y^2)~\]
\[\hat{H}' = 2\lambda xy (\lambda \\ \text{为一小量})~\]

\begin{enumerate}
    \item 用微扰论,求其基态的能量修正(至 $\lambda^2$ 项)及第一激发态的能量修正(至 $\lambda$ 项)。

    \item 如何求出非微扰论的本征能量?试求之,并同微扰论的结果比较。

    \item 相干态的定义为:
    \[    |\alpha \rangle = e^{-|\alpha|^2/2} \sum_{n=0}^{\infty} \frac{\alpha^n}{\sqrt{n!}} |n\rangle \\, \hat{H}_0  \text{为一维线性谐振子的哈密顿量} \\ \hat{H}_0|n\rangle = E_n |n\rangle , ~\]
    $E_n = \left( n + \frac{1}{2} \right) \hbar \omega$ , 试证明,相干态是测不准关系取最小值时的状态。
\end{enumerate}
\subsection{第五题:(15 分)}
质量为 $m$ 的粒子受到势能为 $V(r) = \frac{a}{r^2}$ 的场的散射($a$ 为某一正常数),在入射能量极低的条件下,计算其微分散射截面。(球贝塞尔函数 $j_l = \frac{\sin \left( x - \frac{l}{2} \pi \right)}{x} , \quad x \to \infty$)

\section{第一题: (15 分)}
(1) 试确定,在 3K 温度下,空腔辐射的最大能量密度所对应的光子的波长 $\lambda_m$ 是多少?

(2) 此时,光子的对应能量是多少?

(3) 光电效应中,如何测定某金属板的逸出功 $A$?