% 渐近线
% keys 渐近线|垂直渐近线|斜渐近线|水平渐近线|
% license Usr
% type Tutor

\begin{issues}
\issueDraft
\end{issues}

不论是在刚学习反比例函数时,还是学习。
简单来说,渐近线是函数图像的“边界线”,当函数的自变量趋近于某些极端值(如无穷大或无穷小)时,函数的图像无限接近于渐近线,但永远不会相交。渐近线是指在某个方向上,曲线无限接近于一条直线,且在无穷远处两者之间的距离无限接近于零。

\subsection{函数意义下的渐近线}
函数的\textbf{渐近线(Asymptotes)}是描述函数曲线趋向某一条直线的行为的工具。可以简单理解为:当  $x$  或  $y$  趋向无穷时,函数曲线无限接近但通常不会与这条直线相交。
\subsubsection{垂直渐近线}

\begin{definition}{垂直渐近线}
若函数$f(x)$满足$\displaystyle \lim_{x\to x_0}f(x)=\infty$,则称直线$x=x_0$是$f(x)$的\textbf{垂直渐近线(Vertical Asymptote,也称铅直渐近线)};
\end{definition}

垂直渐近线代表了函数在某些点上发生了不连续性,比如在这些点上,函数值趋于无穷大。所以垂直渐近线所在的位置一定是函数的间断点。一般此时,函数分母趋于零或是分子的高阶无穷小。

\subsubsection{水平渐近线}

\begin{definition}{水平渐近线}
若函数$f(x)$满足$\displaystyle \lim_{x\to +\infty}f(x)=y_0$或$\displaystyle \lim_{x\to -\infty}f(x)=y_0$,则称直线$y=y_0$是$f(x)$的\textbf{水平渐近线(Horizontal Asymptote)};
\end{definition}

偶尔可能会见到类似$\displaystyle \lim_{x\to \infty}f(x)=y_0$的表述,但这种写法暗含了两侧极限存在且相等的意味,也即要求渐近线只能有一条。而对于如$\displaystyle f(x)={1+|x|\over x}$等函数,事实上存在两条水平渐近线。此处选择了稍嫌麻烦但更严谨的定义。

\subsubsection{斜渐近线}

\begin{definition}{斜渐近线}
若函数$f(x)$满足$\displaystyle \lim_{x\to +\infty}f(x)-kx-b=0$或$\displaystyle \lim_{x\to -\infty}f(x)-kx-b=0$,则称直线$y=kx+b$是$f(x)$的\textbf{斜渐近线(Oblique Asymptote)};
\end{definition}

这里的情况与水平渐近线相同。其实可以看出,水平渐近线是$k=0$时的特例。

一般来讲,由于函数的性质限定,斜渐近线和水平渐近线不共存,却可以和垂直渐近线共存。比如$f(x)=x+{1\over x}$

\subsection{解析几何背景下的渐近线}

在解析几何的背景下,渐近线的概念可以被统一为一条曲线无限接近但不相交于另一曲线的直线。无论是水平渐近线、垂直渐近线还是斜渐近线,渐近线都代表了一个曲线在趋于无穷远或某特殊点时的行为。这种统一的定义可以不局限于函数,而是更广泛地适用于曲线的情况。

采用解析几何的视角,可以尝试着将渐近线定义为:

\begin{definition}{渐近线}
设以极坐标表示的平面曲线$C:r=r(\theta)$。如果存在一个方向  $\theta_0$  和一条直线  $L$ ,使得当  $\theta$  趋于  $\theta_0$  时,满足径向距离趋于无穷大且曲线到直线的距离趋于零,即;
\begin{equation}
\begin{cases}
\displaystyle\lim_{\theta \to \theta_0} r(\theta) = \infty\\
\displaystyle\lim_{\theta \to \theta_0} d(C, L) = 0~.
\end{cases}
\end{equation}
那么,称直线  $L$  是曲线  $C$  在方向  $\theta_0$  上的渐近线。
\end{definition}

前面提到的水平渐近线和垂直渐近线分别是$\theta_0=k\pi,(k\in\mathbb{Z})$以及$\displaystyle\theta_0=(2k+1){\pi\over2},(k\in\mathbb{Z})$时的特例。

$\displaystyle y={\sin x\over x}$,$y=0$
$\displaystyle y=\sin {1\over x}$,$y=0$

由上面的定义可以得知,任何闭合曲线都不可能有渐近线。