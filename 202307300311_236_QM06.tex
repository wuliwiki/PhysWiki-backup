% 相互作用绘景
% 动力学|微扰近似
\pentry{量子力学的基本原理\upref{QMPrcp}, 薛定谔绘景和海森堡绘景\upref{HsbPic},时间演化算符\upref{TOprt}}

在本章节中,使用上角标为$(s)$代表薛定谔绘景,上角标为$(I)$代表相互作用绘景,例如$H^{(s)}$,$H^{(I)}$。

\begin{definition}{}

不同于海森堡绘景和薛定谔绘景,在相互作用绘景里,态矢和算符都随时间而改变。薛定谔绘景中的哈密顿量可以分割为:
\begin{equation}
H^{(S)}=H^{(S)}_0+H^{(S)}_I~.
\end{equation}
分别定义$\mathcal U(t)$为$H^{(S)}$对应的时间演化算符,$\mathcal U_0(t)$为$H^{(S)}_0$对应的时间演化算符,具体定义方法参考\href{https://wuli.wiki/online/TOprt.html}{时间演化算符}。

定义:
\begin{align}
|\psi^{(I)}(t)\rangle&=\mathcal U_0(t)^\dagger|\psi^{(S)}(t)\rangle=\mathcal U_0(t)^\dagger\mathcal U(t)|\psi^{(S)}(0)\rangle~, \\
A^{(I)}(t)&=\mathcal U_0(t)^\dagger A^{(S)}(t)\mathcal U_0 (t)~.
\end{align}

上式中$A$为任意算符。

\end{definition}



