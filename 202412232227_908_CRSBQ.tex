% 查尔斯·巴贝奇(综述)
% license CCBYSA3
% type Wiki

本文根据 CC-BY-SA 协议转载翻译自维基百科\href{https://en.wikipedia.org/wiki/Charles_Babbage}{相关文章}。

\begin{figure}[ht]
\centering
\includegraphics[width=6cm]{./figures/2f69cba1b5206a32.png}
\caption{巴贝奇在1860年} \label{fig_CRSBQ_1}
\end{figure}
查尔斯·巴贝奇(Charles Babbage,1791年12月26日-1871年10月18日)是一位英国博学家。[1] 他是一位数学家、哲学家、发明家和机械工程师,巴贝奇提出了数字可编程计算机的概念。[2]

巴贝奇被一些人认为是“计算机之父”。[2][3][4][5] 他被认为发明了第一台机械计算机——差分机,这为更复杂的电子设计奠定了基础,尽管现代计算机的所有基本思想都可以在他的分析机中找到,该机器是通过一个明确借鉴自雅卡尔织机的原则来编程的。[2][6] 除了计算机相关工作外,巴贝奇在他1832年出版的《制造与机械经济学》一书中还涉及了广泛的兴趣领域。[7] 他是伦敦社交圈中的重要人物,并且以其举办的周六晚会而闻名,被认为将“科学晚会”从法国引入英国。[8][9] 他在其他领域的多样工作使他被描述为其世纪中“最杰出”的博学家之一。[1]

巴贝奇虽然未能完成许多设计的成功工程实现,包括他的差分机和分析机,但他在计算机理念的提出上依然是一个重要人物。他未完成的部分机械装置如今被展示在伦敦的科学博物馆。1991年,一台根据原始设计图纸构建的功能性差分机完成了建造。按照19世纪可以实现的公差制造,最终完成的差分机成功运转,证明了巴贝奇的机器本应能够正常工作。


巴贝奇的出生地存在争议,但根据《牛津国家传记词典》,他最有可能出生在英国伦敦沃尔沃思路的44号克罗斯比街。[10] 在拉科姆街与沃尔沃思路交汇处有一块蓝色纪念牌,纪念这一事件。[11]

在《泰晤士报》对巴贝奇的讣告中,他的出生日期为1792年12月26日;但随后一位侄子写信表示,巴贝奇出生在一年之前,即1791年。伦敦纽宁顿圣玛丽教区的注册簿显示巴贝奇于1792年1月6日接受洗礼,支持他出生于1791年的说法。[12][13][14]

约1850年的巴贝奇
巴贝奇是本杰明·巴贝奇和贝齐·普拉姆利·蒂普的四个孩子之一。他的父亲本杰明·巴贝奇是伦敦弗利特街普雷德银行的创始人之一,该银行与威廉·普雷德合伙,于1801年创办。[15] 1808年,巴贝奇一家搬到了东泰恩茅斯的老罗登斯宅邸。大约在八岁时,巴贝奇因患上威胁生命的高烧,被送到埃克塞特附近的阿尔菲顿乡村学校就读。短时间内,他曾在南德文托特尼斯的国王爱德华六世文法学校就读,但由于健康原因,他不得不转回私人家教。[16]

随后,巴贝奇进入了位于米德尔塞克斯恩菲尔德贝克街的霍尔姆伍德学院,学习数学,该校由史蒂芬·弗里曼牧师主持。[17] 学院里有一个图书馆,这激发了巴贝奇对数学的热爱。离开学院后,巴贝奇又请了两位私人家教。第一位家教是一位来自剑桥附近的牧师;通过这位家教,巴贝奇结识了查尔斯·西门和他的福音派追随者,但家教的内容并不适合他。[18] 他被带回家,回到托特尼斯的学校继续学习,那个时候他大约16或17岁。[19] 第二位家教是一位牛津的导师,在他的指导下,巴贝奇掌握了足够的古典学知识,并被剑桥大学录取。