% 格林公式(综述)
% license CCBYSA3
% type Wiki

本文根据 CC-BY-SA 协议转载翻译自维基百科\href{https://en.wikipedia.org/wiki/Green\%27s_theorem}{相关文章}。

在向量分析中,格林公式把围绕一条简单闭合曲线 $C$ 的曲线积分与该曲线所围平面区域 $D$(即 $\mathbb{R}^2$ 中的曲面)的二重积分联系起来。它是斯托克斯定理在二维空间($\mathbb{R}^2$)中的特例。在一维情形下,它等价于微积分基本定理;在三维情形下,它则等价于散度定理。
\subsection{定理}
设 $C$ 是平面上一条按正向(逆时针)取向、分段光滑的简单闭合曲线,$D$ 是 $C$ 所围成的区域。如果 $L$ 和 $M$ 是定义在包含 $D$ 的某个开区域上的函数,且它们在该区域内具有连续偏导数,则有
$$
\oint_{C} (L\,dx + M\,dy) 
= 
\iint_{D} 
\left( 
\frac{\partial M}{\partial x} 
- 
\frac{\partial L}{\partial y} 
\right) dA~
$$
其中,曲线 $C$ 上的积分路径方向为逆时针。
\subsection{应用}
在物理学中,格林公式有许多应用。例如,在处理二维流体积分问题时,可以用它说明:一个区域内流体的总外流量等于该区域边界曲线上的总外流量。在平面几何中,尤其是在面积测量中,格林公式还能用于仅通过对边界积分来求解平面图形的面积和形心位置。
\subsection{当 $D$ 是单连通区域时的证明}