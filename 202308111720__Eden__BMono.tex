% 磁单极子
% keys 磁单极子|麦克斯韦方程组|高斯单位
% license Xiao
% type Tutor

\pentry{磁场的高斯定律\upref{MagGau}}

\begin{issues}
\issueDraft
\end{issues}

\addTODO{狄拉克弦,狄拉克量子化条件,Wu-Yang 单极子}
\footnote{参考 Wikipedia \href{https://en.wikipedia.org/wiki/Magnetic_monopole}{相关页面}。}\textbf{磁单极子(magnetic monopole)}是电动力学中的一类假想的粒子, 至今未被真实观测到。

\subsection{麦克斯韦方程组}
\pentry{麦克斯韦方程组\upref{MWEq}, 洛伦兹力\upref{Lorenz}}

 在麦克斯韦方程组\upref{MWEq} 出现后, 人们注意到若假设磁单极子存在并能像带电粒子产生电场那样产生磁场以及像电流产生磁场那样产生电场, 那么电场 $\bvec E$ 和磁场 $\bvec B$ 的地位就完全平等了。

磁单极子是一类假想粒子的名称, 类似于把质子和电子等微观带电粒子称为 “电单极子”。 类比电荷, 我们说磁单极子中带有\textbf{磁荷(magnetic charge)}。 磁荷的国际单位是 $\Si{Am}$(安培·米), 以下把磁荷记为 $q_m$, \textbf{磁荷密度(magnetic charge density)}记为 $\rho_m$, 则麦克斯韦方程组变为
\begin{equation}
\begin{aligned}
&\div \bvec E = \frac{\rho}{\epsilon_0}~,\\
&\curl \bvec E = - \mu_0 \bvec j_m -\pdv{\bvec B}{t}~,\\
&\div \bvec B = \mu_0 \rho_m~,\\
&\curl \bvec B = \mu_0 \bvec j + \mu_0\epsilon_0 \pdv{\bvec E}{t}~.
\end{aligned}
\end{equation}
电荷和磁荷的总洛伦兹力(\autoref{eq_Lorenz_1}~\upref{Lorenz})变为
\begin{equation}
\bvec F = q \qty(\bvec E + \bvec v \cross \bvec B) +
q_m \qty(\bvec B - \bvec v \cross \bvec E)~.
\end{equation}


\subsubsection{高斯单位}
高斯单位制\upref{GaussU}下, 麦克斯韦方程组和洛伦兹力具有更对称的形式
\begin{equation}\label{eq_BMono_1}
\begin{aligned}
&\div \bvec E = 4\pi\rho~,\\
&\curl \bvec E = -\frac{1}{c}\pdv{\bvec B}{t}  - \frac{4\pi}{c}\bvec j_m~,\\
&\div \bvec B = 4\pi\rho_m~,\\
&\curl \bvec B = \frac{1}{c}\pdv{\bvec E}{t} + \frac{4\pi}{c} \bvec j~.
\end{aligned}
\end{equation}
\begin{equation}
\bvec F = q \qty(\bvec E + \frac{\bvec v}{c}\cross \bvec B) + q_m \qty(\bvec B - \frac{\bvec v}{c}\cross \bvec E)~.
\end{equation}

\subsection{磁单极子的磁场与磁矢势}
\pentry{磁通量\upref{BFlux}}
假设一个静止在原点的磁单极子,磁荷为 $e_M$,则磁场满足类似于电场高斯定律的方程,可以写出磁场的散度与磁荷成正比的方程:
\begin{equation}\label{eq_BMono_4}
\div \bvec B(\bvec x)=\mu_0 e_M\delta^3(0)~.
\end{equation}
解上述微分方程可以得到
\begin{equation}
\bvec B(\bvec x)=\frac{\mu_0 e_M}{4\pi |\bvec x|^3}\bvec x~.
\end{equation}
它对应的磁矢势为
\begin{equation}
\bvec A(\bvec x)=\frac{\mu_0 e_M}{4\pi } \frac{1-\cos\theta}{r\sin\theta} \hat e_\phi~.
\end{equation}
可以验证 $\curl \bvec A(\bvec x) = \bvec B(\bvec x)$。要注意的是上式在 $\theta=\pi$ 即 $z$ 轴负半轴处奇异。所以 $\curl \bvec A(\bvec x)$ 在 $z$ 轴负半轴上奇异。从原点出发沿着 $z$ 轴负半轴一直到无穷远处的弦被称为\textbf{狄拉克弦}。从下面的推导我们将看出狄拉克弦是没有物理意义的,可以利用 $\bvec A$ 的规范变换从一根弦变为另一根弦。
\subsubsection{stokes 定理产生的矛盾}
对于场 $\curl \bvec A(\bvec x)$,它在闭球面上面积分(不考虑狄拉克弦处的非奇异部分)为
\begin{equation}\label{eq_BMono_3}
\int \bvec B(\bvec x)\cdot \dd{\bvec s} =\int \curl \bvec A(\bvec x) \cdot \dd{\bvec s}=0~. 
\end{equation}
上式最终利用了 stokes 定理,由于闭曲面是没有边界的,所以磁矢势的旋度在闭曲面上的积分为 $0$。这与我们之前对磁单极子产生磁场的期待相违背。我们期待 $\bvec B$ 在球面上的积分应当是
\begin{equation}\label{eq_BMono_2}
\int \bvec B(\bvec x)\cdot \dd{\bvec s}=|\bvec B(|\bvec x|=R)|\cdot 4\pi R^2=\mu_0 e_M~.
\end{equation}
产生矛盾的原因正是狄拉克弦的存在。stokes 定理并不适用于积分区域内存在非奇异点的情况。或者也可以将狄拉克弦理解为一根具有无穷大磁场的磁感线,沿着狄拉克弦一直延伸到原点;它导致了一个有限的朝内的磁通量通过闭曲面,与\autoref{eq_BMono_2} 的磁单极子产生的朝闭曲面外的那部分磁通相抵消,即加起来为 $0$。
\subsection{规范变换与 WuYang 单极子}
\pentry{规范变换\upref{Gauge}}
根据\autoref{eq_Gauge_3}~\upref{Gauge} ,规范变换下磁矢势的变换行为是
\begin{equation}
\bvec A\rightarrow \bvec A+\nabla \lambda~.
\end{equation}
可以通过规范变换将一组磁矢势的解转换为另一组磁矢势的解。

另一种满足\autoref{eq_BMono_4} 的磁矢势的解为
\begin{equation}
\bvec A(\bvec x)=-\frac{\mu_0 e_M}{4\pi}\frac{1+\cos\theta}{r\sin\theta}
\hat e_\phi~.
\end{equation}
可以看到狄拉克弦的位置变为 $\theta=0$,即 $z$ 轴正半轴的位置。与之前的解的狄拉克弦的位置不同。这意味着,对于一个磁单极子它所激发出的磁场虽然是确定的,但是它的磁矢势可以有无穷多种解的情况,
