% 静电场与静磁场(摘要)

\begin{issues}
\issueDraft
\end{issues}
在本文中,我们总结一些静电场与静磁场的基本性质。此处所谓“静”,指的是系统的状态不随时间变化,例如,电荷、电流、电场等等的强度与分布都是恒定的。注意,此处“静”与“静止”的含义不完全相同,不代表没有电荷流动(电流),只是说明电荷流动的速度是均匀的,所谓“稳恒电流”。\upref{MagneF}

\subsection{电荷、电流与电荷守恒}
什么是电荷?按照Landau富有深意的话来说,电荷是“粒子与电磁场相互作用的强度”。不过,对于初学者来说,你只要知道电荷同质量一样,也是粒子/物质的一种性质。之所以引入电荷的概念,是因为人们发现物质间除了有源于质量的引力交互作用外,还有另一种交互作用,其强度与质量无关、但与物质的另一种性质有关。人们把这种性质称为“电荷”,这种交互称为“静电力”。

当电荷是离散分布时(例如,我们熟知的点电荷模型,对应“质点”),我们可以描述每一个点电荷所带电荷量$q_i$;而当电荷是连续分布时(例如,一个处处带电的物体,对应“刚体”),我们更喜欢使用电荷密度,即单位体积的电荷量$\rho = \dv{q}{V}$。

当(大量)电荷定向运动起来,就会产生\textsl{电流} \footnote{有时,我们假定电流伴随着异号静止电荷,因此电流总体上不带静电荷,也不产生电场。}。电流可由电流强度 $I$ \upref{I}与电流密度 $\bvec j$\upref{Idens} 表述:

\begin{table}[ht]
\centering
\caption{电流强度与电流密度}\label{tab_estfid3}
\begin{tabular}{|c|c|c|}
\hline
电流强度$I$ :“单位时间通过截面的电荷量” & $I = \dv{q}{t}$ \\
\hline
电流密度$\bvec j$ :“单位时间单位面积通过的电荷量” & $$I = \oint \bvec j \cdot \dd {\bvec S}~$$ $$\bvec j= \dv{I}{S_\perp} \hat n~$$ $$\bvec j = n e \bvec v~$$其中, $n$是载流子(载流子是电荷的载体)的体积数密度,$e$是每一个载流子的电荷量,$\bvec v$是载流子的速度。 \\
\hline
\end{tabular}
\end{table}


同质量一样\footnote{防杠声明:与电动力学紧密结合的狭义相对论告诉我们,这个表述是不严谨的},电荷是守恒的\upref{ChgCsv}。也就是说,如果一个区域内有电荷的流出(电流),那么这个区域内的电荷量就会相应地减小。
$$
\oint \bvec j \vdot \dd{\bvec s}  =  - \dv{t} \int \rho \dd{V} \qquad \sum_i I_i = -\dv{q}{t}~,
$$
其中,$\bvec j$是电流密度,$\rho = \dv{q}{V}$是电荷密度,$I>0$表示流出该区域的电流,否则是流入的电流。或者
$$
\div \bvec j + \pdv{\rho}{t} = 0~.
$$
在静场条件下, 空间中的电荷密度以及电流密度都不随时间变化, 所以有
$$
\div \bvec j = 0 \qquad \sum_i I_i = 0~.$$
这个结论还被称为基尔霍夫第一定律\upref{Kirch} 。

\subsection{静电场与静磁场}
\begin{figure}[ht]
\centering
\includegraphics[width=14 cm]{./figures/1b7ae852545ead32.png}
\caption{电荷与电流分别在一个截面上产生的电场与磁场示意图。CC 0} \label{fig_estfid_1}
\end{figure}
众所周知,电荷与电流会在他们周围分别产生电场与磁场。下表反映了电荷、电流与他们所产生的电场、磁场之间的联系:
\begin{table}[ht]
\centering
\caption{静电场与静磁场}\label{tab_estfid1}
\begin{tabular}{|c|c|c|}
\hline
 & 电场 $\bvec E$ \upref{Efield} & 磁场 $\bvec B$\upref{MagneF} \\
\hline
场源 & 电荷 $q$ \upref{Efield}& 电流(运动的电荷) $I$ \upref{I}\\
\hline
场源产生的场 & $$\dd {\bvec E} (\bvec r) = \frac{1}{4 \pi \epsilon_0} \frac{\dd q}{R^2} \bvec {\hat R}~ $$
不大严格地说,$\dd {\bvec E}$是由单个小电荷 $\dd q$ 产生的电场。
& $$\dd {\bvec B} (\bvec r) = \frac{\mu_0}{4\pi} \frac{I \dd{\bvec r'} \cross \uvec R}{R^2}~$$ 比奥萨伐尔定律\upref{BioSav}; $\dd {\bvec B}$是由一小段电流 $I \dd {\bvec r'}$ 产生的磁场。\footnote{不同于静电场中可以任意摆放电荷,在静磁场中我们不能“任意摆放”电流,而必须使电流成环,或者延伸到无穷远处。假如设计的“电路”不成环,那么根据电荷守恒\upref{ChgCsv},区域内的电荷密度必定变化,从而不再是静场问题。这也是为什么这个公式实际上不能准确描述“单个运动电荷的磁场”。}\\
\hline
线性叠加原理 
& $$\bvec E (\bvec r) = \int \dd {\bvec E} = \frac{1}{4 \pi \epsilon_0} \int \frac{\dd q}{R^2} \bvec {\hat R}~ $$ 由于相应的方程是线性的,因此如果空间中有多个电荷,他们导致的总电场是各个电荷产生的电场的和。 \upref{Efield}
& $$\bvec B(\bvec r) = \oint \dd {\bvec B} =  \frac{\mu_0}{4\pi} \oint \frac{I \dd{\bvec r'} \cross \uvec R}{R^2}~$$\\
\hline
散度方程 & 
$$\oint \bvec E \vdot \dd{\bvec s} = \frac{1}{\epsilon_0}\int \rho \dd{V} = \frac{Q}{\epsilon_0}~$$
$$\div \bvec E = \frac{\rho}{\epsilon_0}~$$ 电场的高斯定律\upref{EGauss}
&
$$\oint \bvec B \vdot \dd{\bvec s} = 0~$$
$$\div \bvec B = 0~$$ 磁场的高斯定律\upref{MagGau}\\
\hline
旋度方程 & 
$$ \oint \bvec E \vdot \dd{\bvec l} = \bvec 0~$$
$$ \curl \bvec E = \bvec 0 ~$$ 静电场的环路定理\upref{ELECLD},基尔霍夫第二定律\upref{Kirch} 的一种表述。
 &
$$\oint \bvec B \vdot \dd{\bvec l} = \mu_0 \int \bvec j \vdot \dd{\bvec s}=\mu_0 I ~$$ 
$$\curl \bvec B = \mu_0 \bvec j~$$ 静磁场的环路定理(专业术语:安培环路定律) \upref{AmpLaw}\\
\hline 
\end{tabular}
\end{table}
其中 $\bvec r$是场点,$\bvec r'$是场源位置,$\bvec R = \bvec r - \bvec r'$是场源指向场点的矢量,$\bvec{\hat R}$是相应的单位矢量。\upref{Efield} $\epsilon_0$,$\mu_0$ 分别是“真空介电常数”与“真空磁导率”(按照griffiths的说法,这两个常数的名称具有误导性,你只需要知道他们是两个常数就行)。

为什么有了漂亮的“场源产生的场”,还需要(微分形式的)散度与旋度方程?尽管二者可以互相“推导”,但是散度与旋度方程仍具有(潜在的、理论意义上的)优势:
\begin{itemize}
\item 在具有某些对称性的情况下,使用散度与旋度方程能避开繁杂的积分、更易于解题;
\item 散度与旋度方程是“局域性”的描述,而局域性是经典物理所偏好的。按照Feynman(的厚厚的《物理学讲义》)的话说,散度与旋度方程告诉我们(静场情况下)某处电磁场的改变只与该处的电荷、电流密度有关;
\item 在非静场情况下,散度与旋度方程在添加一些项后仍然适用,但是“场源产生的场”需要经过大幅修正;
\item 容易从散度与旋度方程中导出势的概念;而如今,势被认为是更基本的物理量。
\end{itemize}

\subsection{电(标)势与磁矢势}
基于数学与物理意义上的考量,我们可以引入势的概念。有时使用势的概念,会比场更为简洁、深刻。
\begin{table}[ht]
\centering
\caption{电(标)势与磁矢势}\label{tab_estfid2}
\begin{tabular}{|c|c|c|}
\hline
* & 电场 $\bvec E$ & 磁场 $\bvec B$ \\
\hline
势 & $$\varphi~$$  电势\upref{QEng},标量函数& $$\bvec A~$$  磁矢势\upref{BvecA},矢量函数\\
\hline
势与场 & $$\bvec E = -\grad \varphi~$$ $$ \varphi(\bvec r) = \int^{\bvec r_0}_{\bvec r} \bvec E \cdot \dd {\bvec l}~ $$ ($\bvec r_0$是零势参考点,一般选取无穷远处势为0,具体参考下文的“规范”) \upref{QEng} & $$\bvec B = \curl \bvec A~$$ \upref{BvecA} \\
\hline
势的任意性,“规范” \upref{Gauge} & $$\varphi += \lambda~$$ $\lambda$是常数 & $$\bvec A += \grad \lambda~$$ $\lambda$是标量函数。基于此,$\div \bvec A$的取值可被控制 \\
\hline
场源导致的势 
& $$\varphi(\bvec r) = \frac{1}{4\pi\epsilon_0} \int \frac{\rho(\bvec r')}{\abs{\bvec r - \bvec r'}} \dd V'~.$$ (假定无穷远处势为零)
& $$\bvec A(\bvec r) = \frac{\mu_0}{4\pi} \int \frac{\bvec j(\bvec r')}{\abs{\bvec r - \bvec r'}}\dd{V'}~.$$ (假定取规范$\div \bvec A = 0$)\\
\hline
势的方程 & $$\nabla^2 \varphi = -\frac{\rho}{\varepsilon_0}~$$ (假定无穷远处势为零)& $$\nabla^2 \bvec A = - \mu_0 \bvec j~$$ (取规范$\div \bvec A = 0$)\\
\hline
\end{tabular}
\end{table}

稍微具体一点,引入势的数学考量是,如果一个场无旋,那么他可以写成一个标量函数的梯度 $\curl \bvec F = \bvec 0 \Rightarrow \bvec F = - \nabla \varphi$;以及如果一个场无散,那么他可以写成另一个场的旋度$\div \bvec F = \bvec 0 \Rightarrow \bvec F = \curl \bvec A$,静电场和磁场天然分别满足这些条件,详见griffiths书与“矢量分析总结\upref{VecAnl} ”;物理上的考量参考“电势、电势能\upref{QEng} ”。从电场-电势的关系很容易看出,静电场其实只有一个自由度 $\varphi$,而非看起来的三个 $(E_x, E_y, E_z)$。

由于电荷直接感受到的是场而不是势(见下"洛伦兹力"),所以只要能得到相同的场,势的取值具有一定的灵活性。例如,由于 $\bvec E = -\grad \varphi$,即使势加上一个常数后,仍有 $\bvec E = -\grad (\varphi+\lambda) = -\grad \varphi - \grad \lambda = -\grad \varphi$,即相应的电场也不会变化。因此,电势总可以相差一个常数而不改变“实质性结果”。这就有点像我们做不定积分时,总会得到一个积分常数 $C$;或者对函数求导时,常数项不会改变导函数。
选取常数的方法称为“规范”,根据相应的场合,我们会选取恰当的规范以简化计算。在静场中,我们一般令电势在无穷远处为零,并让磁势满足 $\div \bvec A = 0$。

将势的定义代入Maxwell方程组,并辅以数学技巧,就可以得到相应势的方程。例如对于电势:
$$
\left \{
\begin{aligned}
\bvec E &= -\grad \varphi\\
\div \bvec E &= \frac{\rho}{\epsilon_0}\\
\end{aligned}
\right.
\Rightarrow
\laplacian \varphi = -\frac{\rho}{\epsilon_0}
~.
$$
磁势也是同理,但是需要更巧妙地运用数学技巧\upref{VecAnl} 与规范;由于磁势$\bvec A$是一个矢量,所以磁势的方程其实包括三个方程
$$
\nabla^2 \bvec A = - \mu_0 \bvec j~
\Rightarrow
\left \{
\begin{aligned}
\laplacian A_x = -\mu_0 j_x\\
\laplacian A_y = -\mu_0 j_y\\
\laplacian A_z = -\mu_0 j_z\\
\end{aligned}
\right.
~.
$$

原则上,"场源导致的势"是相应势的方程的解,\textsl{但这么做就慢了}。根据电场的性质、势的定义、以及电势与积分路径无关的性质,很容易直接得到
$\varphi = \int \bvec E \cdot \dd {\bvec l} = \frac{1}{4\pi\epsilon_0} \frac{q}{R}$
并推广为 
$\varphi = \frac{1}{4\pi\epsilon_0} \int \frac{\rho}{R} \dd V'$。根据磁势方程与电势的类似性,也就容易得到"电流导致的ci sh"

\begin{figure}[ht]
\centering
\includegraphics[width=14 cm]{./figures/cfe09e8bf7ddc9a4.pdf}
\caption{源->势->场} \label{fig_estfid_2}
\end{figure}

\subsection{洛伦兹力}
一个点电荷在电磁场中的受力由洛伦兹力\upref{Lorenz}公式给出:
$$
\bvec F = q (\bvec E + \bvec v \cross \bvec B)~.
$$
注意该式中并不包含点电荷静止或匀速运动时自身产生的电磁场。 但当点电荷做非匀速运动时会产生辐射(即电磁波), 而这个辐射又会反过来作用在点电荷上面。 从能量守恒的角度来说,电荷产生辐射损失了能量,所以动能必定会下降,所以它的辐射对自身的力整体上必然是一个阻力。 不过在一般情况下,我们假定这个辐射的功率足够小, 使得该效应可以忽略不计。

对于连续分布的电荷,我们描述单位体积(“一小块”)电荷所受的洛伦兹力。相当于上式两边“同除以” $\dd V$。
$$\bvec f = \dv{\bvec F}{V} = \dv{q}{V} (\bvec E + \bvec v \cross \bvec B) = \rho (\bvec E + \bvec v \cross \bvec B)=\rho \bvec E + \bvec j \times \bvec B ~.$$
% 较真地说,运动的电荷会导致非静场问题,并且直接引出电动力学中最引入入胜、高深莫测的话题。
