% 材料科学 Intro
% keys 
% license CCBY4
% type Wiki



不仅是单一成分…
按照经典的比喻,如果材料是一栋大楼,那么他的成分就相当于建造大楼的一块块砖头。

早在幼儿园二年级,懵懂无知的我们就开始好奇地打量生活中的各种物品、初探材料世界。也是从那时起,我们知道了那些基本材料的名称:金属、陶瓷、玻璃、塑料…你可能会开始想,如此几种材料是如何组成千变万化的世界的呢?

而到了小学二年级,事情开始变得复杂起来。我们开始发现,所谓金属、塑料,其实只是一大类材料的总称,而并非单一一种物质。譬如说,金属可以继续细分为铁、铜、铝等等等等,而塑料则可以分为热塑型塑料与热固型塑料。

到了人生知识的巅峰、中二病发的中学二年级,我们了解到,即使是铁、铝这些名词,仍然不是故事的全部。我们使用的金属材料往往不是单一成分的纯净金属,而是包括了多种元素的合金:常说的“钢铁“就至少包括了铁与碳这两种元素;塑料那边的情况则更糟糕,根据塑料中构成高分子链的小分子种类,我们开始苦恼于聚乙烯PE、聚丙烯PP、聚氯乙烯PVC、聚乳酸PLA…

到了大学二年级,如果我们有志在干饭、电竞之余继续探索材料世界,就一定会感慨“吾生也有涯,而知也无涯”。“钢铁”不只是铁与碳:一方面,钢除了铁、碳之外,还往往包括其余数(十)种元素:例如,常听说的304不锈钢就具有大量的铬Cr、镍Ni等元素,这些元素有助于提升不锈钢的性能;另一方面,为了适应不同的应用环境,各种不同成分的钢也被广泛地开发、应用,因此“钢铁”仍然还是一类材料的代称,304不锈钢也只是钢铁中的一种。别忘了,世界上不只有钢铁(铁合金),还有铝合金、铜合金在摩拳擦掌…

而塑料的情况更为复杂,塑料中除了高分子链本身,还含有大量其余成分,例如催化剂(用于在塑料生产过程中将小分子聚合为高分子)、塑化剂(相当于分子层面的润滑剂,提高塑料的可变性性、“塑性”)、色素(顾名思义)…就连高分子链本身也变幻莫测,例如高分子链还由多种小分子聚合而来,这种情况下称为共聚物,比较经典的是ABS塑料。

金属	铁、铝、…	铁合金、铝合金	各种铁合金、各种铝合金…
塑料	热塑塑料、热固塑料	PP,PE,PVC,…	高分子链+催化剂、塑化剂…

在云里雾里了一通后,我们甚至还没有开始涉及“各个成分之间的含量是什么”,光是“有什么成分”的问题就已经空前复杂。总而言之,我想说的是,材料的成分是非常复杂的问题,在生活中没有太多材料是“纯净物”。(或许钻石、石墨是个例外)
 
也不只是一种结构
接下来,我们要思考材料学的另一个重要问题,“结构”。“结构”这个词总令人有些恐惧,因为和结构有关的问题往往是重点、难点与考点(为什么这三个点总是黏在一起!)

无论在哪个领域,“结构”的含义都十分丰富;在材料科学领域,不同的细化学科也可能会着重研究不同含义下的结构。我们姑且认为结构包括原子间的排列(晶胞)与原子团之间的排列(晶粒、高分子链间的结合)等。

首先,元素组成肯定是影响结构的关键因素。我们已经知道了晶胞的概念,而在室温下,不同金属原子形成的晶胞结构是不一样的。例如,铁Fe形成体心立方BCC晶胞,而Ti形成HCP晶胞。

元素的含量比例也会影响结构。在低碳的情况下,C插入BCC晶胞的间隙;而在高碳的情况下,C与Fe形成复杂的Fe3C。这是Fe-C的两种相态,铁往往是这两相的混合物。

到这里,都还比较符合我们的生活直觉,然而,我想问的是,“如果元素组成与含量确定了,那么材料的结构就确定了吗?”你或许会觉得这句话很有道理,但是考虑到写文章半路抛出个似是而非的问题后必没有好事的套路,你又迟疑了起来。答案是,或许有点震惊的,不是。就像使用相同的积木能够搭出不同的房子一样,使用相同的元素也可以构造出不同的材料。

最耳熟能详的例子是碳C,它的固体包括两种结构:石墨与钻石。并且,据我们所知,石墨与钻石的性质相差非常大!

更有趣的例子是,成分不能完全确定原子团(晶胞)之间的排序。即使是最简单的单相合金,晶胞的尺寸也可以有大有小,形状上可以是等轴的、也可以是树枝状的,等等等等。

再比如二相混合物,不同相晶粒的形状、分布也可以千变万化。例如,晶胞可以是等轴的、层片状的或者网状的。

高分子的结构更为复杂,包括大量全新的术语与概念。为简明起见,此处我们就举一个简单的例子:支链。对于聚乙烯PE这种“简单“的高分子,我们预期他似乎应该是一根直链。然而,因为一些我也快忘了的高分子化学原因,在合成高分子链时,往往会产生一些支链,这使PE高分子链更像一条带有各种支流的河流。

如果一个PE链的支链多,那么链之间就很难凑在一起,使得链的排列很松散,宏观上看密度低,称为低密度聚乙烯LDPE;而如果一个PP链的支链少,那么链的排列更为紧密,宏观上看密度高,称为高密度聚乙烯HDPE。这就有点像你收拾衣柜时,如果你一通乱塞,那么衣柜可能放不进几件衣服;而如果你叠好衣物(或者用真空泵压缩)再放进衣柜,那么衣柜就能装不少衣服了。
综上所述,成分确实能影响结构,但是成分并不能唯一确定结构。即使成分确定,材料的结构仍有很大的变化空间。
 
不同的结构、不同的性能

除了对知识的满腔热情,还有什么动力驱动着我们去分析材料的结构呢?答案是,有用。相同成分、不同结构的材料具有截然不同的性能。换而言之,结构会影响性能。只有我们充分理解结构对性能的影响,才能设计符合产品需求的材料。

在中,我们了解了金属变形的基本原理,即金属的变形往往关乎位错的运动;而我们将要说明,金属结构如何影响位错运动。

金属结构->影响位错运动->影响金属变形->改变材料力学性能

以晶界的作用为例,晶界就像一堵墙,能阻碍位错的通过,也就能阻碍金属变形。那么为什么晶界能一夫当关万夫莫开呢?背后的原因非常复杂,容我简要列举其一二:一方面,晶界本身是一个结构相对无序的区域,原子无序堆叠导致的内力将排斥位错的靠近;另一方面,晶界两边晶粒取向不同,位错想要通过晶界就得改变自身的滑移系,而这是相当困难的。

那么我们可以如何运用这一结论?我们对比小晶粒材料与大晶粒材料。如上文所说,小晶粒材料中晶粒小、晶界多,因此位错遇到的阻碍也就多,金属也就不容易变形;此外,由于晶粒之间的相互联系,晶粒的变形将相互制约。因此一般而言,小晶粒材料中位错更难运动、变形更难发生、强度也更高(需要更大的力才能使材料发生变形)。材料学术语,“细晶强化”。

但是这也只是硬币的一面,另一面是,既然晶界堵住了位错的去路,位错自然得堆积在晶界面前;而我们知道,位错作为缺陷的一种,自身也要产生应力场;大量位错塞积产生的大应力场将可能使材料断裂。在小晶粒材料中,这个问题还不算太棘手,因为位错被分散在大量晶粒中,单处的位错塞积不明显,总体而言位错的塞积甚至还被缓解,材料的韧性也被增强(在材料断裂前,材料的变形程度更大);然而,在网状结构的材料中(见上文),位错在晶界(相界)的塞积十分明显,这使材料很容易断裂。因此,生产过程中往往要避免网状结构。

或许你还记得上文提及的PE支链问题。我们知道,支链多的 PE,链之间排列很松散,形成LDPE;而支链少的 PE,链之间排列更紧密,形成HDPE。由于HDPE相对紧密的链排序,一般而言,HDPE的强度、韧性甚至熔点都高于LDPE。

 
如何调整结构

既然我们已经知道了结构会如何影响材料的性能了,那么问题就变成了,我们如何生产特定结构的材料呢?难过的是,目前而言,精细地控制材料结构仍然是困难的(从这方面上来说,细胞似乎很厉害,可以直接在原子、分子尺度上精确地控制物质的结构),还有广阔的空间留给后人开发、探索。此处我们将介绍一点最基础的方法。

在铸造过程中,调整晶胞大小的最简单的方法,莫过于改变冷却的速率。背后的原理可以参考。简而言之,当材料从融化状态逐渐冷却到固体时,发生形核-长大过程:液体内将先形成一堆小晶核,然后这些晶核将逐渐长大、直到材料完全凝固为固体。如果冷却速度足够快,那么有利于形成大量的晶核,同时这些晶核将没有时间充分长大。因此,我们最终会得到大量小晶粒;相反,如果冷却速度比较慢,那么晶核将能充分长大,从而得到少量大晶粒。由此可见,通过调整冷却速率,我们就可以控制晶胞的大小,从而影响材料的力学性能!工业上用“退火”、“正火”与“淬火”来形容不同的冷却速度

当然实际问题也并非如此简单,单纯的加速冷却并非灵丹妙药包炼百材料,还得具体问题具体分析。

正如我们上文所述,我们一般不喜欢性能不良的网状结构,那么怎么消去网状结构呢?我们知道,相界也是一种缺陷,而网状组织具有大面积的结构,本身是热力学不稳定的,只不过由于室温固态环境下,原子的排列已经很紧密,也就很难重新排序为热力学上更稳定的结构。就像在水里跑步比在空气来跑步费力多了一样!

为给原子加油鼓气,我们选择再加热材料一段时间:在较高的温度之下,原子就能获得足够的能量来克服阻碍、重新排布为更稳定的结构。因此,网状结构将在高温下自行分解,材料的性能也得以改善。这种方法称为“回火“,比较典型的例子是回火处理高碳钢以消除网状渗碳体。

念念不忘的PE支链问题。在传统的催化合成方法中,由于催化剂本身的性质,支链的形成在所难免,很难合成支链足够少的PE;目前,“新”的表面催化方法大幅降低了支链的形成,让HDPE成为可能。

本文能涵盖的,也只(能)是加工与结构故事的冰山一角。 
无言的环境因素

在本文的最后,让我们来讨论一下环境与材料性能的关系。在之前的论述中,我们都心照不宣地假定了材料工作在室温室压的环境。然而,并非所有的材料都能如此幸运:飞机引擎叶片工作在几百甚至几千摄氏度,轮船船壳得忍受几度的盐性海水,而气罐的容器壁得承受数百倍大气压的压强…在不同的工作环境下,材料的性能肯定各不相同。

例如,对于

