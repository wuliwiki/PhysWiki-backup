% 有限对称群的性质
% license Usr
% type Tutor

\begin{definition}{}
阶数为1或2的置换,称为\textbf{对合变换(involution)。}
\end{definition}
\begin{theorem}{}
任意有限置换群都可以表示为两个对合变换的复合。
\end{theorem}
\textbf{证明:}
我们首先证明,循环置换可以分解为两个对合变换的复合。

一个n元循环置换可以看作n边形上的旋转,而我们知道,二维空间上的旋转可以分解为两个反射,只要保证两次反射轴的夹角是旋转角度的$1/2$,对合变换就是这种反射变换的置换表示。以正五边形为例,该过程如下所示:
\begin{figure}[ht]
\centering
\includegraphics[width=14cm]{./figures/f32c9320160af59c.png}
\caption{} \label{fig_AutSym_2}
\end{figure}
该循环的分解写作$(12345)=[(23)(14)][(13)(45)]$,显然,$[(13)(45)]$与$[(23)(14)]$就是图中所示的“反射变换”。

由于$n$元置换群总可以拆分成不相交的循环乘积,而每个循环都可以拆成对合之积,则这些对合可以重新组合成两组。如设某置换群可拆分成三个不相交循环之积,用$\sigma$表示对合变换,且下标首字母不同代表不同循环,则有:
\begin{equation}
\begin{aligned}
f=f_1f_2f_3&=[\sigma_{11}\sigma_{12}][\sigma_{21}\sigma_{22}][\sigma_{31}\sigma_{32}]\\
&=[\sigma_{11}\sigma_{21}\sigma_{31}][\sigma_{12}\sigma_{22}\sigma_{32}]\\
&=\sigma'_1\sigma'_2~.
\end{aligned}
\end{equation}
得证。

从将循环元素的分解过程中,我们可以看到,对合变换是不相交的对换乘积,\textbf{称这些不相交的对换为一个对合变换的组分。}
\subsubsection{有限置换群的自同构群}
在本节,我们先阐明置换群的内自同构与群本身的关系,再阐明自同构与群的关系。
\begin{theorem}{}
当$n>2$时,$\opn{Inn}S_n\cong S_n$
\end{theorem}

\textbf{证明:}

由\autoref{the_Group2_2}~\upref{Group2}得,$\opn{Inn}S_n\cong S_n/C(S_n)$,因此我们只需要证明$n>2$时,有限置换群的中心是单位元即可。

设任意$\sigma_n\in C(S_n)-e$,那么$\sigma_n$有三种可能,下面证明,每一种情况总能找到与之不交换的群元素。

\begin{itemize}
\item 若$\sigma_n$是对换,比如$(a\,b)$,则$(b\,c)$必然与之交换。
\item 若$\sigma_n$是$k$循环,比如$(a\,b\,c\,d)$,则$(a\,q)$便与之不交换。
\item 若$\sigma_n$是不相交的循环乘积,则在两个循环里各取一元素,组成的对换与之不交换。
\end{itemize}
因此,$\sigma_n$是空集,$C(S_n)=e$,证明成立。

自同构是比内自同构更广些的概念,只要求双射同态。思及前文,我们已经证明任意置换都可以拆分成两个不相关的对合之复合。令对合变换表示为$\sigma$,易证自同构将其映射为对合,因为$f(\sigma^2)=f(e)=f^2(\sigma)$,且能把共轭类映射为共轭类,$f(\tau\sigma\tau^{-1})=f(\tau)f(\sigma)f^{-1}(\tau)$。下面我们来探讨,对合变换彼此共轭需要的条件。


\begin{lemma}{}
对合变换共轭,当且仅当二者\textbf{组分一致}。
\end{lemma}
用$\tau$来做共轭变换,有几种可能结果:
\begin{itemize}
\item 
\end{itemize}