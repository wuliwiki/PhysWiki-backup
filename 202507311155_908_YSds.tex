% 域上的代数(综述)
% license CCBYSA3
% type Wiki

本文根据 CC-BY-SA 协议转载翻译自维基百科\href{https://en.wikipedia.org/wiki/Algebra_over_a_field}{相关文章}。

在数学中,一个域上的代数(通常简称为代数,algebra)是一个带有双线性乘法的向量空间。因此,代数是一种代数结构,由一个集合以及加法、乘法和域元素的数量乘法构成,并且满足“向量空间”和“双线性”所蕴含的公理。[1]

代数中的乘法运算可以是结合的,也可以不是,这对应于结合代数(假设乘法具有结合性)和非结合代数(不假设乘法具有结合性,但也不排除结合性)。例如,给定一个整数 $n$,实数域上的 $n$ 阶实方阵环,在矩阵加法和矩阵乘法下,是一个结合代数,因为矩阵乘法是结合的。另一方面,三维欧几里得空间在向量叉积运算下是一个非结合代数,因为向量叉积不满足结合律,但满足雅可比恒等式。

如果代数在乘法下具有单位元,则称其为有单位代数。例如,$n$ 阶实方阵环是一个有单位代数,因为 $n$ 阶单位矩阵是矩阵乘法的单位元。它是一个有单位结合代数的例子,即既是(有单位的)环,又是一个向量空间。

许多作者使用“代数”一词时,实际上是指结合代数,或有单位结合代数,或者在某些学科(如代数几何)中,指有单位结合交换代数。

将标量的域替换为交换环,就得到了更一般的环上的代数的概念。需要注意的是,代数不同于带双线性型的向量空间(如内积空间),因为对于这样的空间,乘积的结果不在空间本身,而是在系数域中。
\subsection{定义与动机}
\subsubsection{动机性示例}
\begin{figure}[ht]
\centering
\includegraphics[width=14.25cm]{./figures/e88003d816a27498.png}
\caption{} \label{fig_YSds_1}
\end{figure}
\subsubsection{定义}
设 $K$ 是一个域,$A$ 是一个定义在 $K$ 上的向量空间,并且带有一个从 $A \times A$ 到 $A$ 的附加二元运算,记作 $\cdot$(也就是说,若 $x$ 和 $y$ 是 $A$ 的任意两个元素,则 $x \cdot y$ 是 $A$ 的一个元素,称为 $x$ 和 $y$ 的积)。若对所有 $A$ 中的元素 $x, y, z$ 以及所有 $K$ 中的元素(通常称为标量)$a, b$,以下恒等式成立,则称 $A$ 是一个定义在 $K$ 上的代数:
\begin{itemize}
\item 右分配律:$(x + y) \cdot z = x \cdot z + y \cdot z$
\item 左分配律:$z \cdot (x + y) = z \cdot x + z \cdot y$
\item 与标量的相容性:$(a x) \cdot (b y) = (ab)(x \cdot y)$
\end{itemize}
这三个公理等价于说二元运算是双线性的。一个定义在 $K$ 上的代数有时也称为 $K$-代数,而 $K$ 被称为 $A$ 的基域。这个二元运算通常被称为 $A$ 的乘法。本文所采用的约定是:代数中元素的乘法不一定满足结合律,尽管有些作者使用“代数”一词时默认指结合代数。

当向量空间上的二元运算是交换的情况下,左分配律与右分配律是等价的,此时只需证明一个分配律即可。一般而言,对于非交换运算,左分配律与右分配律并不等价,必须分别证明。
\subsection{基本概念}
\subsubsection{代数同态}
给定两个 $K$-代数 $A$ 和 $B$,一个$K$-代数同态是一个 $K$-线性映射
$f: A \to B$使得对所有 $x, y \in A$ 都有$f(xy) = f(x) f(y)$。如果 $A$ 和 $B$ 是有单位元的代数,并且一个同态满足$f(1_A) = 1_B$,则称它为一个有单位同态。所有从 $A$ 到 $B$ 的 $K$-代数同态所构成的集合通常记作
$$
\mathbf{Hom}_{K\text{-alg}}(A, B)~
$$
一个$K$-代数同构是一个双射的 $K$-代数同态。
\subsubsection{子代数与理想}
一个定义在域 $K$ 上的代数的子代数是一个线性子空间,具有这样的性质:其任意两个元素的积仍然属于该子空间。换句话说,代数的子代数是一个在加法、乘法和数量乘法下封闭的非空元素子集。用符号表示,如果 $A$ 是一个 $K$-代数,当 $L \subseteq A$ 且对所有 $x, y \in L$ 和 $c \in K$ 有$x \cdot y \in L \quad x + y \in L \quad cx \in L$则称 $L$ 是 $A$ 的一个子代数。

在前面将复数视为实数域上的二维代数的例子中,一维的实数直线就是一个子代数。

一个 $K$-代数的左理想是一个线性子空间,具有这样的性质:其任意一个元素被代数中任意一个元素从左边相乘,结果仍然属于该子空间。用符号表示,如果 $A$ 是一个 $K$-代数,当 $L \subseteq A$ 且对所有 $x, y \in L$、$z \in A$、$c \in K$,以下三个条件成立时,称 $L$ 是一个左理想:
\begin{enumerate}
\item $x + y \in L$($L$ 在加法下封闭);
\item $cx \in L$($L$ 在数量乘法下封闭);
\item $z \cdot x \in L$($L$ 在任意元素的左乘下封闭)。
\end{enumerate}
如果将 (3) 替换为 $x \cdot z \in L$,则定义的是一个右理想。一个双边理想是既是左理想又是右理想的子集。通常单独使用“理想”一词时,指的是双边理想。当然,当代数是交换的时,这些理想的概念是等价的。条件 (1) 和 (2) 一起等价于 $L$ 是 $A$ 的一个线性子空间,由条件 (3) 可知每个左理想或右理想都是一个子代数。

这个定义与环的理想定义不同,因为这里我们需要条件 (2)。当然,如果代数是有单位的,则条件 (3) 推出条件 (2)。
\subsubsection{标量的扩张}
如果我们有一个域扩张 $F/K$,也就是说一个包含 $K$ 的更大的域 $F$,那么可以通过一种自然的方法,从任意一个定义在 $K$ 上的代数构造出一个定义在 $F$ 上的代数。这和将向量空间扩张到更大域上的方法是一样的,即使用张量积:$V_F := V \otimes_K F
$。因此,如果 $A$ 是一个定义在 $K$ 上的代数,那么$A_F$就是一个定义在 $F$ 上的代数。
\subsection{代数的种类与示例}
定义在域上的代数有很多不同的类型。这些类型由一些进一步的公理决定,例如乘法的交换性或结合性,而这些在代数的广义定义中并不是必须的。不同类型的代数对应的理论往往非常不同。
\subsubsection{有单位代数}
如果一个代数具有单位元或恒等元 $I$,并且对代数中所有的 $x$ 都满足$I x = x = x I$,则称该代数是有单位代数。
\subsubsection{零代数}
如果一个代数中对所有 $u, v$ 都有$uv = 0$,则称它为零代数,\(^\text{[2]}\)不要与“只有一个元素的代数”混淆。零代数本质上是非幺的(除非只有一个元素的情形),并且是结合的、交换的。

一个幺零代数是一个域 $K$ 与一个 $K$-向量空间 $V$ 的直和$K \oplus V$,其乘法在向量空间(或模)上恒为零,并使其成为一个有单位元的代数。

更确切地说,代数的每个元素都可以唯一地写为$k + v, \quad k \in K, v \in V,$并且乘法是唯一的双线性运算,满足对所有 $v, w \in V$:$vw = 0$.因此,若 $k_1, k_2 \in K$ 且 $v_1, v_2 \in V$,则有:
$$
(k_1 + v_1)(k_2 + v_2) = k_1k_2 + (k_1v_2 + k_2v_1)~
$$
一个经典的幺零代数的例子是对偶数代数,它是由一个一维实向量空间构造出的幺零 $\mathbb{R}$-代数。

这个定义可以直接推广到交换环上的幺零代数,将“域”和“向量空间”分别替换为“交换环”和“模”。

幺零代数允许统一给定模的子模理论和幺代数的理想理论。事实上,一个模 $V$ 的子模正好对应于包含在$K \oplus V$中的理想。

例如,Gröbner 基理论是 Bruno Buchberger 为域上的多项式环$R = K[x_1,\dots, x_n]$中的理想引入的。通过在自由 $R$-模上构造幺零代数,可以将这一理论推广为自由模的子模的 Gröbner 基理论。这一推广使得在计算子模的 Gröbner 基时,可以不加修改地使用任何计算理想 Gröbner 基的算法和软件。

类似地,幺零代数可以直接推出模(交换环上的模)的 Lasker–Noether 定理,而无需额外的证明,只需从原本针对理想的 Lasker–Noether 定理出发即可。
\subsubsection{结合代数}
结合代数的例子包括:
\begin{itemize}
\item 域(或交换环)$K$ 上所有 $n \times n$ 矩阵构成的代数,这里的乘法是普通的矩阵乘法;
\item 群代数,其中一个群作为向量空间的基,代数的乘法扩展群的乘法;
\item 交换代数 $K[x]$,即所有 $K$ 上多项式构成的代数(见多项式环);
\item 函数代数,例如区间 $[0,1]$ 上所有实值连续函数构成的 $\mathbb{R}$-代数,或者某个固定复平面开集上所有全纯函数构成的 $\mathbb{C}$-代数,这些代数也是交换的;
\item 关联代数,它们构建在某些偏序集之上;
\item 线性算子代数,例如 Hilbert 空间上的算子代数,其中代数的乘法由算子复合给出。这类代数还带有拓扑结构;许多算子代数定义在 Banach 空间上,从而成为 Banach 代数。如果还给出一个对合运算,则得到 B\*-代数 和 C\*-代数。这些代数是泛函分析中的研究对象。
\end{itemize}
\subsubsection{非结合代数}
一个非结合代数\(^\text{[3]}\)(或称分配代数,distributive algebra)是一个定义在域 $K$ 上的 $K$-向量空间 $A$,配备了一个 $K$-双线性映射$A \times A \to A$。这里“非结合”的含义是不假设结合律成立,但并不意味着结合律被禁止,即“非必然结合”。

主条目中详细介绍的例子包括:
\begin{itemize}
\item 三维欧几里得空间 $\mathbf{R}^3$,其乘法由向量叉积给出
\item 八元数
\item 李代数
\item 约旦代数
\item 交替代数
\item 柔性代数
\item 幂结合代数
\end{itemize}
\subsection{代数与环}
结合的、有单位元的 $K$-代数的定义也经常以另一种方式给出。在这种情况下,域 $K$ 上的一个代数 $A$ 被定义为一个环,并带有一个环同态:
$$
\eta : K \to Z(A)~
$$
其中 $Z(A)$ 是 $A$ 的中心。由于 $\eta$ 是一个环同态,所以必须满足:要么 $A$ 是零环,要么 $\eta$ 是单射。

该定义与前面的定义是等价的,其中标量乘法
$$
K \times A \to A~
$$
由下式给出:
$$
(k, a) \mapsto \eta(k)a~
$$
给定两个这样的结合的、有单位的 $K$-代数 $A$ 和 $B$,一个有单位的 $K$-代数同态$f : A \to B$是一个与 $\eta$ 所定义的标量乘法可交换的环同态,可以写作:
$$
f(ka) = k f(a)~
$$
对所有 $k \in K$ 和 $a \in A$ 都成立。换句话说,下列图表是可交换的:
$$
\begin{matrix}
& & K & & \\
& \eta_A \swarrow & & \eta_B \searrow & \\
A & & \xrightarrow{f} & & B
\end{matrix}~
$$
\subsection{结构系数}
对于域上的代数,从 $A \times A \to A$ 的双线性乘法完全由 $A$ 的基元素的乘法决定。反过来,一旦选定了 $A$ 的一个基,就可以任意规定基元素的乘积,然后以唯一的方式将其扩展为 $A$ 上的双线性算子,即保证所得的乘法满足代数公理。

因此,给定域 $K$,任意有限维代数可以通过给出它的维数(设为 $n$)并指定 $n^3$ 个结构系数 $c_{i,j,k}$(这些是标量)来确定到同构为止。这些结构系数通过如下规则确定 $A$ 中的乘法:
$$
\mathbf{e}_i \mathbf{e}_j = \sum_{k=1}^{n} c_{i,j,k} \mathbf{e}_k,~
$$
其中 $\mathbf{e}_1, \ldots, \mathbf{e}_n$ 构成 $A$ 的一个基。

需要注意的是,几个不同的结构系数组合可能会导出同构的代数。

在数学物理中,结构系数通常带上下标书写,以区分它们在坐标变换下的变换性质。具体来说,下标是协变指标,通过拉回变换,而上标是逆变指标,通过推前变换。因此,结构系数通常写作 $c_{i,j}{}^{k}$,其定义规则用爱因斯坦求和约定可写作:
$$
e_i e_j = c_{i,j}{}^{k} e_k.~
$$
将此应用于用指标表示的向量时,可得:
$$
(xy)^k = c_{i,j}{}^{k} x^i y^j.~
$$
如果 $K$ 只是一个交换环而不是一个域,当 $A$ 是 $K$ 上的自由模时,同样的过程仍然成立。如果不是自由模,则乘法仍然完全由它在一个生成 $A$ 的集合上的作用决定;但是在这种情况下,结构常数不能随意指定,仅知道结构常数并不能唯一确定代数的同构类。
\subsection{复数域上低维有单位结合代数的分类}
二维、三维和四维的复数域上有单位结合代数在同构意义下由 Eduard Study 完全分类。[4]

二维代数

存在两个这样的二维代数。每个代数由两个基元素 $1$(单位元)和 $a$ 的复数系数线性组合组成。根据单位元的定义:
$$
1 \cdot 1 = 1, \quad 1 \cdot a = a, \quad a \cdot 1 = a.~
$$
剩下需要指定:\\
对于第一个代数:$a a = 1$\\
对于第二个代数:$a a = 0$

三维代数

存在五个这样的三维代数。每个代数由三个基元素 $1$(单位元)、$a$ 和 $b$ 的复数系数线性组合组成。考虑单位元的定义,只需指定:\\
第一个代数:$a a = a, \quad b b = b, \quad a b = b a = 0$\\
第二个代数:$a a = a, \quad b b = 0, \quad a b = b a = 0$\\
第三个代数:$a a = b, \quad b b = 0, \quad a b = b a = 0$\\
第四个代数:$a a = 1, \quad b b = 0, \quad a b = - b a = b$\\
第五个代数:$a a = 0, \quad b b = 0, \quad a b = b a = 0$\\
其中,第四个代数是非交换的,其余都是交换的。
