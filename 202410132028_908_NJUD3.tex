% 南京理工大学 普通物理 B(845)模拟五套卷 第三套
% license Usr
% type Note

\textbf{声明}:“该内容来源于网络公开资料,不保证真实性,如有侵权请联系管理员”

\subsection{一、 填空题 I(26 分,每空 2 分)}
1. 一质点的运动方程(SI)为:$x=-10t+30t^2,y=15t-20t^2$ ,则质点的起始速度为__________,质点加速度为____________。

2. 质量为 $m$,长为 $l$ 的匀质细杆,可绕其端点的水平轴在竖直平面内自由转动。如果将细杆置于水平位置,然后让其由静止开始自由下摆,则开始转动的瞬间,细杆的角加速度为_____________,细杆转动到竖直位置时的角速度为_________________。

3. 如图所示,一长为 $l$ 的均匀直棒可绕过其一端且与棒垂直的水平光滑固定轴转动。抬起另一端使棒向上与水平面成 60°,然后无初速地将棒释放。已知棒对轴的转动惯量为$\frac{1}{3}ml^2$ ,其中 $m$ 和 $l$ 分别为棒的质量和长度,则放手时棒的角加速度为________,棒转到水平位置时的角加速度为_________________。
\begin{figure}[ht]
\centering
\includegraphics[width=6cm]{./figures/cab3d81ca7772da7.png}
\caption{} \label{fig_NJUD3_1}
\end{figure}
4. 已知一平面简谐波频率为$1000Hz$,波速为 $300m/s$,则波上相差 $\pi/4$的两点之间的距离为_______________,在某点处时间间隔为 $0.001s$的两个振动状态间的相位差为_________。

5. 一质点作简谐振动,速度最大值 $Vm=5cm/s$,振幅 $A=2cm$,若令速度具有正最大值的那一时刻为 $t=0$,则振动表达式为_____________。

6. 互感系数的物理意义是______________。

7. 在容积为 $10^{-2}m^3$ 的容易中,装有质量为 $100g$ 的气体,若气体分子的方均根速率为 $300mS^{-1}$,则气体的压强为____________。
\subsection{、 填空题 II(20 分,每空 2 分)}
1. 在棱镜$(n1=1.52)$表面镀一层增透膜$(n2=1.30)$,如使此增透膜适用于 $550.0nm$ 波长的光,则膜的厚度为____________。

2. 如图所示,半径为 $r_1$ 的小导线环,置于半径为 $r_2$ 的大导线环中心,二者在同一平面内,且 $r_1<r_2$。在大导线环中通过正弦电流$I=I_0\sin^\omega t$,其中$\omega ,I_0$为常数,$t$ 为时间,则任一时刻小导线环中感应电动势的大小为____________。设小导线环的电阻为$R$,则在 $t=0$ 到 $t=\pi/(2^\omega )$时间内,通过小导线环某截面的感应电量为_______________。

3. 自然光通过两个偏振化方向成 60°的偏振片,透射的光强为 $I_1$,今在这两个偏振片之间加入另一个偏振片,与两个夹角均为30°,透过的光强为___________。

4. 自感为 $0.25H$ 的线圈中,当电流在 $1/16s$ 内由 $2A$ 均匀减小到零时,线圈中自感电动势的大小为____________。

5. 在电荷为$-Q$ 的点电荷 $A$ 的静电场中,将另一电荷为 $q$ 的点电荷$B$从 $a$ 点移到 $b$ 点,$a,b$ 两点距离点电荷 $A$ 的距离分别为 $r_1$ 和$r_2$,如图所示,则移动过程中电场力做的功为______________。
\begin{figure}[ht]
\centering
\includegraphics[width=6cm]{./figures/a9f083700a3e1f6c.png}
\caption{} \label{fig_NJUD3_2}
\end{figure}
6. 一粒子静质量为 $m_0$,其动能是静能的 $n$ 倍,则该粒子的运动质量为_________,运动速度大小为_____________。

7. 处于 $n=4$ 激发态的氢原子,它回到基态的过程中,所发出的光波波长最短为___________$nm$,最长为____________$nm$。
\subsection{三、(13 分)}
