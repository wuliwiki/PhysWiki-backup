% 玻尔兹曼因子(综述)
% license CCBYNCSA3
% type Wiki

本文根据 CC-BY-SA 协议转载翻译自维基百科\href{https://en.wikipedia.org/wiki/Boltzmann_distribution}{相关文章}。

在统计力学和数学中,玻尔兹曼分布(也称为吉布斯分布\(^\text{[1]}\))是一种概率分布或概率测度,用于描述一个系统处于某一状态的概率,它是该状态能量和系统温度的函数。该分布的形式为:
\[
p_i \propto \exp\left(-\frac{\varepsilon_i}{kT}\right)~
\]
其中:\( p_i \)是系统处于状态\( i \)的概率;\( \exp \)是指数函数;\( \varepsilon_i \)是该状态的能量;常数\( kT \)是玻尔兹曼常数\( k \)与热力学温度\( T \)的乘积。符号\( \propto \)表示成比例关系(关于比例常数的具体内容,见“分布”一节)。

这里的“系统”一词具有广泛的含义;它可以指从“足够数量”的原子或单个原子\(^\text{[1]}\)到像天然气储罐这样的宏观系统。因此,玻尔兹曼分布可以用于解决各种各样的问题。该分布表明,能量较低的状态总是具有更高的被占据的概率。

两个状态的概率之比被称为玻尔兹曼因子,其特征是仅依赖于这两个状态的能量差:
\[
\frac{p_i}{p_j} = \exp\left(\frac{\varepsilon_j - \varepsilon_i}{kT}\right)~
\]
玻尔兹曼分布以路德维希·玻尔兹曼的名字命名,他最早于1868年在研究热平衡状态下气体的统计力学时提出了这一分布。\(^\text{[2]}\)玻尔兹曼的统计工作体现在他的论文《论机械热理论的第二基本定理与关于热平衡条件的概率计算之间的关系》中。\(^\text{[3]}\)之后,约西亚·威拉德·吉布斯于1902年在其现代通用形式上对该分布进行了深入研究。\(^\text{[4]}\)

玻尔兹曼分布不应与麦克斯韦–玻尔兹曼分布或麦克斯韦-玻尔兹曼统计混淆。玻尔兹曼分布给出了系统处于某一状态的概率,作为该状态能量的函数[5],而麦克斯韦-玻尔兹曼分布给出了理想气体中粒子速度或能量的概率。然而,一维气体中的能量分布确实遵循玻尔兹曼分布。
\subsection{分布}  
玻尔兹曼分布是一种概率分布,它给出了某一状态的概率,作为该状态能量和分布所应用的系统的温度的函数\(^\text{[6]}\)。它的表达式为:
\[
p_i = \frac{1}{Q} \exp\left(-\frac{\varepsilon_i}{kT}\right) = \frac{\exp\left(-\frac{\varepsilon_i}{kT}\right)}{\sum _{j=1}^{M}\exp\left(-\frac{\varepsilon_j}{kT}\right)}~
\]
其中:
\begin{itemize}
\item exp() 是指数函数,  
\item \(p_i\)是状态\(i\)的概率,  
\item \(\varepsilon_i\)是状态\(i\)的能量,  
\item \(K\)是玻尔兹曼常数,  
\item \(T\)是系统的绝对温度,  
\item \(M\)是系统可达的所有状态的数量,[6][5]  
\item \(Q\)某些作者用\(z\)表示)是归一化分母,即经典配分函数  
\end{itemize} 
\[{\displaystyle Q=\sum _{j=1}^{M}\exp \left(-{\tfrac {\varepsilon _{j}}{kT}}\right)}~\]
这来自于所有可达状态的概率必须加起来等于 1 的约束。

使用拉格朗日乘子法,可以证明玻尔兹曼分布是最大化熵的分布:
\[
S(p_1, p_2, \cdots, p_M) = -\sum_{i=1}^{M} p_i \log_2 p_i~
\]
在归一化约束条件下,即\(\sum p_i = 1\)以及约束条件下,即\(\sum p_i \varepsilon_i\)等于一个特定的平均能量值,除了两种特殊情况。(这些特殊情况出现在平均值是能量\(\varepsilon_i\)的最小值或最大值时。在这些情况下,最大化熵的分布是玻尔兹曼分布的极限,其中 T 分别从上方或下方趋近于零。)