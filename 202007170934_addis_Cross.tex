% 矢量叉乘
% 线性代数|矢量|叉乘|cross product|叉积|cross product|向量积|vector product|矢量积|分配率

\pentry{三阶行列式\upref{Deter}}

矢量叉乘在物理定律中十分常见, 例如在讨论力学中的力矩\upref{Torque}, 角动量\upref{AngMom}, 以及电磁学中的洛伦兹力\upref{Lorenz}, 安培力\upref{FAmp}时都会使用. 以下我们讨论的矢量都是三维空间中的几何矢量\upref{GVec}, 在讨论它们的坐标时, 我们默认取正交归一基底.

\subsection{叉乘的几何定义}
两个几何矢量 $\bvec A$,  $\bvec B$ 的\textbf{叉乘(cross product)} 也叫\textbf{叉积(cross product)},\textbf{向量积(vector product)}或\textbf{矢量积}. 叉乘的结果是一个矢量 $\bvec C$.  叉乘用“ $\cross$” 表示,且不可省略, 即 $ \bvec A \cross \bvec B = \bvec C$

\begin{definition}{矢量叉乘}
要确定一个几何矢量,只需分别确定模长和方向:
\begin{enumerate}
\item $\bvec C$ 的模长等于 $\bvec A, \bvec B$ 的模长之积与夹角 $\theta$ ($0 \leqslant \theta \leqslant \pi$)的正弦值相乘.
\begin{equation}\label{Cross_eq1}
\abs{\bvec C}  = \abs{\bvec A} \abs{\bvec B} \sin\theta 
\end{equation}
\item $\bvec C$ 的方向垂直于 $\bvec A, \bvec B$ 所在的平面,且由右手定则\upref{RHRul}决定.
\end{enumerate}
\end{definition}

与内积和数乘不同,叉乘\textbf{不满足交换律}.根据几何定义, $\bvec B \cross \bvec A$ 与 $\bvec A \cross \bvec B$ 模长相同,方向却相反.表示某个矢量的反方向,就是在前面加负号,所以有
\begin{equation}\label{Cross_eq11}
\bvec B \cross \bvec A = -\bvec A \cross \bvec B
\end{equation}

\subsection{叉乘与数乘的混合运算}

在 $\bvec A \cross \bvec B = \bvec C$ 中, $\bvec C$ 的方向仅由 $\bvec A$ 和 $\bvec B$ 的方向决定.当 $\bvec A$ 和 $\bvec B$ 的方向不变时, $\bvec C$ 的模长正比 $\bvec A$ 和 $\bvec B$ 的模长相乘.假设 $\lambda $ 为常数(标量),显然有
\begin{equation}
(\lambda \bvec A) \cross \bvec B = \bvec A \cross (\lambda \bvec B) = \lambda (\bvec A \cross \bvec B)
\end{equation}
即标量的位置可以任意变换,但矢量与乘号的位置关系始终要保持不变.

\subsection{叉乘的分配律}

叉乘一个最重要的特性,就是它满足分配律.
\begin{equation}\label{Cross_eq6}
\bvec A \cross (\bvec B +\bvec C) = \bvec A \cross \bvec B + \bvec A \cross \bvec C
\end{equation}
由\autoref{Cross_eq11} 及上式可以推出
\begin{equation}\label{Cross_eq7}
(\bvec A + \bvec B) \cross \bvec C =  - \bvec C \cross (\bvec A + \bvec B) =  - \bvec C \cross \bvec A - \bvec C \cross \bvec B = \bvec A \cross \bvec C + \bvec B \cross \bvec C
\end{equation}

从几何的角度理解,这个结论并不显然(见矢量叉乘分配律的几何证明\upref{CrossP}).

\subsection{叉乘的坐标运算}

按照上面的定义,在右手系中,三个坐标轴的单位矢量 $\uvec x, \uvec y, \uvec z$ 满足
\begin{equation}\label{Cross_eq3}
\uvec x \cross \uvec y = \uvec z
\qquad
\uvec y \cross \uvec z = \uvec x
\qquad
\uvec z \cross \uvec x = \uvec y
\end{equation}
由\autoref{Cross_eq11} 可得
\begin{equation}\label{Cross_eq4}
\uvec y \cross \uvec x =  - \uvec z
\qquad
\uvec z \cross \uvec y =  - \uvec x
\qquad
\uvec x \cross \uvec z =  - \uvec y
\end{equation}
根据定义,一个矢量叉乘自身,模长为 $0$. 所以叉乘结果是零矢量 $\bvec 0$. 于是又有
\begin{equation}\label{Cross_eq5}
\uvec x \cross \uvec x = \bvec 0
\qquad
\uvec y \cross \uvec y = \bvec 0
\qquad
\uvec z \cross \uvec z = \bvec 0
\end{equation}
\autoref{Cross_eq3},\autoref{Cross_eq4} 和\autoref{Cross_eq5} 中共 9 条等式描述了 $\uvec x, \uvec y, \uvec z$ 中任意两个叉乘的结果.



把矢量 $\bvec A$ 和 $\bvec B$ 分别在直角坐标系的三个单位矢量展开,得到
\begin{equation}
\bvec A = a_x\,\uvec x + a_y\,\uvec y + a_z\,\uvec z \qquad \bvec B = b_x\,\uvec x + b_y\,\uvec y + b_z\,\uvec z
\end{equation}
$(a_x,a_y,a_z)$ 和 $(b_x,b_y,b_z)$ 分别是 $\bvec A$ 和 $\bvec B$ 的坐标.根据叉乘的分配律(\autoref{Cross_eq6} \autoref{Cross_eq7}),可得到如下 9 项
\begin{equation}
\ali{
\bvec A \cross \bvec B ={} &(a_x\,\uvec x + a_y\,\uvec y + a_z\,\uvec z) \cross (b_x\,\uvec x + b_y\,\uvec y + b_z\,\uvec z)\\
={} &+ a_x b_x(\uvec x \cross \uvec x) + a_x b_y(\uvec x \cross \uvec y) + a_x b_z(\uvec z \cross \uvec z)\\
&+ a_y b_x(\uvec y \cross \uvec x) + a_y b_y(\uvec y \cross \uvec y) + a_y b_z(\uvec y \cross \uvec z)\\
&+ a_z b_x(\uvec z \cross \uvec x) + a_z b_y(\uvec z \cross \uvec y) + a_z b_z(\uvec z \cross \uvec z)
}\end{equation}
注意每一项中的运算在\autoref{Cross_eq3} 到\autoref{Cross_eq5} 中都能找到答案,于是上式化为
\begin{equation}\label{Cross_eq2}
\bvec A \cross \bvec B = (a_y b_z - a_z b_y)\,\uvec x + (a_z b_x - a_x b_z)\,\uvec y + (a_x b_y - a_y b_x)\,\uvec z
\end{equation}
令 $\bvec C = \bvec A \cross \bvec B$, 则 $\bvec C$ 的分量表达式为
\begin{equation}\label{Cross_eq8}
\begin{cases}
c_x = a_y b_z - a_z b_y\\
c_y = a_z b_x - a_x b_z\\
c_z = a_x b_y - a_y b_x
\end{cases}
\end{equation}
\autoref{Cross_eq2} 可以用三阶行列式\upref{Deter}表示为
\begin{equation}\label{Cross_eq13}
\bvec A \cross \bvec B = 
\begin{vmatrix}
\uvec x & \uvec y & \uvec z\\
a_x & a_y & a_z\\
b_x & b_y & b_z
\end{vmatrix} \end{equation}
与普通行列式不同的是,这个行列式中的元有部分是矢量,所以得出的结果也是矢量.

\begin{example}{求三角形面积与法向量}\label{Cross_exe1}
空间直角坐标系中三角形的三点分别为 $O(0,0,0)$,  $A(1,1,0)$,  $B(-1,1,1)$. 求三角形的面积和一个单位法向量.

令 $O$ 到 $A$ 的矢量和  $O$ 到 $B$ 的矢量分别为
\begin{equation}
\ali{
\bvec a  &= (1,1,0) - (0,0,0) = (1,1,0)\\
\bvec b  &= (-1,1,1) - (0,0,0) = (-1,1,1)
}\end{equation}
三角形的面积为
 \begin{equation}
S = \frac12 ab \sin \theta 
\end{equation}
其中 $\theta $ 是 $\bvec a$ 与 $\bvec b$ 的夹角.根据\autoref{Cross_eq1}, 有\footnote{可见 $\abs{\bvec a\cross\bvec b}$ 是以 $\bvec a$ 和 $\bvec b$ 为边的平行四边形的面积.}
\begin{equation}
S = \frac12 ab \sin \theta  = \frac12 \abs{\bvec a\cross\bvec b}
\end{equation}
令
\begin{equation}
\bvec v = \bvec a \cross \bvec b = 
\begin{vmatrix} \uvec x & \uvec y & \uvec z \\ 1&1&0\\-1&1&1 \end{vmatrix}
= \uvec x - \uvec y + 2\,\uvec z 
\end{equation}
坐标为 $(1,-1,2)$,模长为 $\abs{\bvec v} = \sqrt{1 + 1 + 2^2} = \sqrt 6$, 所以面积为 $S = \sqrt 6 /2$. 

根据叉乘的几何定义, $\bvec v = (1,-1,2)$ 就是三角形的法向量,进行\textbf{归一化}\footnote{把矢量长度变为1,方向不变}
得单位法向量为
 \begin{equation}
\uvec v = \frac{\bvec v}{\abs{\bvec v}} = \frac{(1,-1,2)}{\sqrt 6} = \qty( \frac{\sqrt 6 }{6}, - \frac{\sqrt 6 }{6}, \frac{\sqrt 6 }{3} )
\end{equation}
\end{example}
%例子: 绕轴旋转的线速度;洛伦兹力;安培力;%未完成:引用

\begin{example}{建立参考系}

\end{example}
