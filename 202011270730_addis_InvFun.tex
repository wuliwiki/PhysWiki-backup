% 反函数
% 函数|映射|反函数

\begin{issues}
\issueDraft
\end{issues}

\pentry{函数\upref{functi}}
由于函数本质上是映射\upref{map}, 那么如果一个函数的映射存在逆映射, 那么我们就把逆映射对应的函数成为\textbf{反函数(inverse function)}.

我们知道不是所有的映射都有反映射. 只有单射存在逆映射, 而多对一映射不存在逆映射.

\begin{equation}
x = f^{-1}[f(x)]
\end{equation}

我们往往可以通过缩小定义域的方式来使一个函数具有反函数. 例如 $\sin^{-1} x$ 是 $\sin(x)$ 在 $[-\pi/2, \pi/2]$ 区间上的反函数.

\subsection{图像}
如果一个实函数 $y = f(x)$($f: \mathbb R \to \mathbb R$) 可以使用图像描述, 那么反函数就是 $f(x)$ 关于直线 $y = x$ 的镜像对称.
