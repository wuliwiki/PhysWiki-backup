% 斯涅尔定律(综述)
% license CCBYSA3
% type Wiki

本文根据 CC-BY-SA 协议转载翻译自维基百科\href{https://en.wikipedia.org/wiki/Maxwell\%27s_equations}{相关文章}。

\begin{figure}[ht]
\centering
\includegraphics[width=6cm]{./figures/2be493390070c758.png}
\caption{光在两种折射率不同的介质界面处的折射示意图,其中 \( n_2 > n_1 \)。由于光在第二介质中的速度较低(\( v_2 < v_1 \)),折射角 \( \theta_2 \) 小于入射角 \( \theta_1 \);也就是说,光线在折射率较高的介质中更接近法线。} \label{fig_SNR_1}
\end{figure}

斯涅尔定律(也称为斯涅尔-笛卡尔定律、伊本·萨尔定律[^1],或折射定律)是一种描述入射角和折射角之间关系的公式,适用于光或其他波在两种不同各向同性介质(如水、玻璃或空气)的边界处传播时的情况。在光学中,该定律用于光线追踪以计算入射角或折射角,也用于实验光学中测定材料的折射率。该定律在具有负折射率的超材料中同样适用,这些材料允许光以“向后”折射的方式弯曲,形成负折射角。

定律指出,对于一对给定的介质,入射角(\( \theta_1 \))的正弦与折射角(\( \theta_2 \))的正弦之比,等于第二介质相对于第一介质的折射率(\( n_{21} \)),也等于两介质的折射率之比(\( \frac{n_2}{n_1} \)),或者等价于两介质中相位速度之比(\( \frac{v_1}{v_2} \))[^2]。
\[
\frac{\sin \theta_1}{\sin \theta_2} = n_{21} = \frac{n_2}{n_1} = \frac{v_1}{v_2}~
\]
斯涅尔定律可以从费马最短时间原理推导而来,而费马原理本身是基于光作为波传播的特性得出的。
\subsection{历史}
\begin{figure}[ht]
\centering
\includegraphics[width=6cm]{./figures/6af00fcbdd7265ed.png}
\caption{伊本·萨赫尔手稿中展示其发现折射定律的一页复刻。} \label{fig_SNR_2}
\end{figure}
托勒密在埃及亚历山大的研究中发现了关于折射角的某种关系,但对于较大的角度来说,这一关系并不准确。托勒密确信自己找到了一个精确的经验定律,这部分是因为他对数据进行了轻微的修改以符合理论(参见:确认偏误)。

这一定律最终以斯涅尔命名,尽管最早发现这一规律的是波斯科学家伊本·萨赫尔(Ibn Sahl)。984年,萨赫尔在巴格达宫廷中撰写的《关于燃烧镜与透镜》手稿中,运用该定律推导出了能够无几何像差聚焦光线的透镜形状。

海什木(Alhazen)在其1021年完成的《光学书》中几乎重新发现了折射定律,但他未能迈出最后一步。
\begin{figure}[ht]
\centering
\includegraphics[width=6cm]{./figures/a891d473cda5060f.png}
\caption{1837年对“正弦定律”历史的观点[5]} \label{fig_SNR_3}
\end{figure}
1602年,托马斯·哈里奥特(Thomas Harriot)重新发现了这一定律,但他并未发表其成果,尽管他曾与开普勒就这一主题进行过通信。1621年,荷兰天文学家威利布罗德·斯涅利乌斯(Willebrord Snellius,1580–1626,即斯涅尔)推导出一种数学上等效的表达形式,但在他生前并未出版。1637年,勒内·笛卡尔(René Descartes)通过一种基于正弦的启发性动量守恒论证,独立推导出了该定律,并在他的文章《光学》中运用该定律解决了许多光学问题。

皮埃尔·费马(Pierre de Fermat)拒绝接受笛卡尔的解法,而是基于他的最短时间原理,独立得出了同样的结论。笛卡尔假设光速是无限的,但在其推导中又假设介质越密,光速越快。费马则持相反假设,即光速是有限的,而且光速在密度较高的介质中较慢。他的推导依赖于这一假设。此外,费马的推导还使用了他发明的“等适性”(adequality),这一数学方法相当于微积分,用于求解极值和切线问题。