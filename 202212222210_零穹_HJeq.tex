% 哈密顿-雅可比方程
% 哈密顿雅可比方程

\pentry{端点可变的作用量\upref{EPAct}}
对作用量
\begin{equation}
S=\int_{t_1}^{t_2}L\dd t
\end{equation}
由\autoref{EPAct_eq1}~\upref{EPAct},就有
\begin{equation}\label{HJeq_eq1}
\pdv{S}{t^{(2)}}+H\qty(q^{(2)},p^{(2)},t^{(2)})=0
\end{equation}
同样,用\autoref{EPAct_eq1}~\upref{EPAct} 中的
\begin{equation}
p_i^{(2)}=\pdv{S}{{q^i}^{(2)}}
\end{equation}
代入\autoref{HJeq_eq1} ,就得到方程:
\begin{equation}
\pdv{S}{t^{(2)}}+H\qty({q^1}^{(2)},\cdots,{q^n}^{(2)};\pdv{S}{{q^1}^{(2)}},\cdots,\pdv{S}{{q^n}^{(2)}};t^{(2)})=0
\end{equation}

若去掉表示末时刻的上指标 $(2)$,而默认所有变量都对应末时刻的值,上式就写成
\begin{equation}\label{HJeq_eq2}
\pdv{S}{t}+H\qty({q^1},\cdots,{q^n};\pdv{S}{{q^1}},\cdots,\pdv{S}{{q^n}};t)=0
\end{equation}
这个关于 $S$ 的一阶偏微分方程,就称为\textbf{哈密顿-雅可比方程}。
\subsection{利用哈密顿-雅可比方程求解系统的运动}
\footnote{参考朗道《力学》}一阶偏微分方程的解(全积分)包含的独立常数之个数与独立变量的数目相同。由\autoref{HJeq_eq2} ,这里 $S$ 的独立变量是 $n$ 个坐标和1个时间,共 $n+1$ 个。而 $S$ 仅以其导数的形式出现在方程中,所以解中的任意常数中有一个是以相加的形式出现的,故哈密顿-雅可比方程解的全积分形式为
\begin{equation}
S=f(t,q^1,\cdots,q^n,\alpha_1,\cdots,\alpha^n)+A
\end{equation}
其中 $\alpha_1,\cdots,\alpha_n,A$ 是任意常数。
