% 有限深势阱
% 束缚态|能级|薛定谔方程|定态

% 束缚态的平均动量为零

% 这是区分束缚态(E<0) 和连续态(E>0)的最简单例子了.
% 当我们有势阱时, 都会有这种规律, 例如氢原子.

% 图未完成(量子力学简介里面似乎有图)

\pentry{定态薛定谔方程\upref{SchEq}}

\footnote{参考 Wikipedia \href{https://en.wikipedia.org/wiki/Finite_potential_well}{相关页面}.}本文使用原子单位. 代入定态薛定谔方程\autoref{SchEq_eq1}~\upref{SchEq}
\begin{equation}
-\frac{1}{2m}\pdv[2]{x}\psi(x) + V(x) \psi(x) = E \psi(x)
\end{equation}

中,令势能函数为
\begin{equation}
V(x) = \begin{cases}
-V_0 \quad &(-L/2 \leqslant x \leqslant L/2)\\
0 \quad &(\text{其他})
\end{cases}
\end{equation}

\begin{equation}
k = \sqrt{2mE} \qquad \kappa = \sqrt{2m(E + V_0)}
\end{equation}



波函数在边界处满足函数值和一阶导数都连续.

什么时候有束缚态什么时候没有?
