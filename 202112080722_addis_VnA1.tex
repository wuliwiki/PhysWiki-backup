% 速度、加速度(一维)
% 位移|符合函数|速度|加速度|二阶导数|匀加速运动|简谐振动

\pentry{位移\upref{Disp}, 复合函数求导\upref{ChainR}}

速度和加速度都是矢量, 但如果我们考虑质点的一维运动(沿直线运动), 那么我们可以指定一个正方向并沿运动方向建立坐标轴. 这样一来, 我们就可以把一维情况下的位移、速度、加速度这些矢量用一个标量来表示, 可以用标量的正负号区分矢量的方向, 正号代表指向正方向, 负号代表指向负方向, 标量的绝对值就等于矢量的模长. 所以以下我们用坐标 $x$ 来表示一维位移, 实数 $v$ 和 $a$ 来表示一维速度和加速度. 事实上, 这些标量可以看作是对应矢量的坐标\upref{GVec}.

物理学中, \textbf{速度(velocity)}和\textbf{加速度(acceleration)}通常指瞬时值. 在一维运动中, \textbf{瞬时速度(instantaneous velocity)}被定义为一段极短时间 $\Delta t$ 内质点的位移\upref{Disp} $\Delta x$ 除以这段时间, \textbf{瞬时加速度(instantaneous acceleration)}被定义为一段极短时间 $\Delta t$ 内质点的速度变化 $\Delta v$ 除以这段时间, 而这些恰好是导数\upref{Der}的定义. 用极限\upref{Lim}和导数\upref{Der}来表示, 就是

\begin{equation}
v(t) = \lim_{\Delta t\to 0} \frac{x(t+\Delta t) - x(t)}{\Delta t} = \dv{x(t)}{t}
\end{equation}

\begin{equation}
a(t) = \lim_{\Delta t\to 0} \frac{v(t+\Delta t) - v(t)}{\Delta t} = \dv{v(t)}{t}
\end{equation}
根据高阶导数\upref{HigDer}的定义, 加速度就是位矢的二阶导数, 即导数的导数
\begin{equation}
a(t) = \dv[2]{x(t)}{t}
\end{equation}

为什么经典力学中, 我们通常之关心位移的一阶和二阶导数而不关心更高阶呢? 因为牛顿第二定律中出现了加速度, 把它和质点的受力紧紧关联起来, 另一方面, 和速度成正比的动量是重要的守恒量.

\begin{example}{匀加速运动}\label{VnA1_ex1}
若已知某直线运动的位移—时间函数为
\begin{equation}\label{VnA1_eq1}
x(t) = x_0 + v_0 t + \frac{1}{2} a_0 t^2
\end{equation}
试证明这是一个匀加速运动.

对位移求导得到速度为
\begin{equation}
v(t) = v_0 + a_0 t
\end{equation}
再次求导(二阶导数)得到加速度为 $a(t) = a_0$ 是常数. 可见这是一个匀加速运动.
\end{example}
事实上, 任何匀加速直线运动都可以表示为\autoref{VnA1_eq1} 的形式, 详见 “匀加速直线运动\upref{CnstAL}”.

\begin{example}{简谐振动}
已知简谐振动(详见“简谐振子\upref{SHO}”)的位移函数为 $x(t) = A\cos(\omega t)$, 运用复合函数求导\upref{ChainR} 得速度为 $v(t) = -A\omega\sin(\omega t)$, 加速度为 $a(t) = -A\omega^2\cos(\omega t)$.
\end{example}

\subsection{由速度或加速度求位移}
\pentry{牛顿—莱布尼兹公式\upref{NLeib}}
既然一维速度 $v(t)$ 是位置 $x(t)$ 的导数, 那么 $x(t)$ 是 $v(t)$ 的原函数\upref{Int}. 由牛顿—莱布尼兹公式得速度在一段时间的定积分等于初末位置之差, 即
\begin{equation}\label{VnA1_eq4}
x(t_2) - x(t_1) = \int_{t_1}^{t_2} v(t) \dd{t}
\end{equation}
所以若已知某时刻质点的位置 $x(t_0) = x_0$, 和速度函数 $v(t)$, 就可以求得任意时刻的位置.
\begin{equation}\label{VnA1_eq5}
x(t) = x_0 + \int_{t_0}^t v(t') \dd{t'}
\end{equation}
注意为了区分积分变量和积分上限, 我们把积分变量改成 $t'$. 这是一个常见的做法.

\begin{example}{匀速直线运动}
若一维运动的质点速度始终为 $v_0$, 由\autoref{VnA1_eq5} 得
\begin{equation}
\begin{aligned}
x(t) &= x_0 + \int_{t_0}^t v_0 \dd{t}\\
&= x_0 + v_0(t-t_0)
\end{aligned}
\end{equation}
\end{example}

与\autoref{VnA1_eq4} 和\autoref{VnA1_eq5} 同理,一维速度和加速度之间也有类似关系
\begin{equation}
v(t_2) - v(t_1) = \int_{t_1}^{t_2} a(t) \dd{t}
\end{equation}
\begin{equation}\label{VnA1_eq8}
v(t) = v_0 + \int_{t_0}^t a(t') \dd{t'}
\end{equation}
把\autoref{VnA1_eq8} 带入\autoref{VnA1_eq5} 再次定积分得(注意积分变量再次重命名)
\begin{equation}\label{VnA1_eq2}
\begin{aligned}
x(t) &= x_0 + \int_{t_0}^t  \qty[v_0 + \int_{t_0}^{t'} a(t'') \dd{t''}] \dd{t'}\\
&= x_0 + v_0 (t - t_0) + \int_{t_0}^t \qty[\int_{t_0}^{t'} a(t'') \dd{t''}] \dd{t'}\\
\end{aligned}
\end{equation}
这里做了两次积分, 即二重积分\upref{IntN}.

简单的例子见 “匀加速直线运动\upref{CnstAL}”.

\begin{exercise}{}
你是否能看出, 以 $t'$ 和 $t''$ 建立直角坐标系, \autoref{VnA1_eq1} 的二重积分的区域是一个三角形?
\end{exercise}
