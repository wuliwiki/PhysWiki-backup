% 多自由度简谐振子(经典力学)
% keys 简谐振子|mode|模

\pentry{简谐振子(经典力学)\upref{SHO}}

\subsection{二自由度简谐振子}

考虑轻质弹簧与两个光滑滑块构成的体系,如\autoref{MSHO_fig1} 所示.

\begin{figure}[ht]
\centering
\includegraphics[width=10cm]{./figures/MSHO_1.pdf}
\caption{两个滑块和两根弹簧构成的简谐振子.} \label{MSHO_fig1}
\end{figure}

和\textbf{简谐振子(经典力学)}\upref{SHO}的情况一样,弹簧的原长不重要,因此图中没给出.我们用$x_1$和$x_2$分别表示滑块$m_1$和$m_2$的位置,其中$x_1$表示弹簧$k_1$的长度减去其原长,$x_2$表示弹簧$k_2$的长度减去其原长.显然,当$x_1=x_2=0$时两根弹簧都处于原长.

则这个体系的运动方程为
\begin{equation}
\leftgroup{
    m_1\ddot{x}_1 &= -k_1x_1+k_2x_2\\
    m_2\ddot{x}_1 &= -k_2x_2
}
\end{equation}

这个体系也是也是一个简谐振子.由于它需要两个参数$x_1, x_2$来刻画,因此也称作\textbf{二自由度}的简谐振子.

当然,我们也可以在\autoref{MSHO_fig1} 的右边加一堵墙,再加一根弹簧将$m_2$和右边的墙连接起来,得到的同样是二自由度简谐振子.


\begin{figure}[ht]
\centering
\includegraphics[width=8cm]{./figures/MSHO_2.pdf}
\caption{另一种二自由度简谐振子的模型.图中方块是固定的墙面,只有粉色的圆球有质量,其它构件都是轻质的.红色和蓝色分别表示只能沿着水平或垂直方向移动的滑杆,粉色圆球被固定在这两个滑杆中.两个滑杆分别连接到图示的两根弹簧上.} \label{MSHO_fig2}
\end{figure}

\autoref{MSHO_fig2} 所示的模型也是一个二自由度简谐振子,两个参数分别是圆球的横位移和纵位移.

一般地,二自由度简谐振子的运动方程写为
\begin{equation}\label{MSHO_eq1}
\leftgroup{
    \ddot{x}_1 &= k_{11}x_1+k_{12}x_2\\
    \ddot{x}_2 &= k_{21}x_1+k_{22}x_2
}
\end{equation}


\subsubsection{二自由度简谐振子的方程解法}

类比\textbf{简谐振子(经典力学)}\upref{SHO}中的解法,我们可以先猜想\autoref{MSHO_eq1} 的一些特解.

最简单的情况是,两个参数都做\textbf{相位}相同的单自由度简谐振动,只是\textbf{振幅}不一样,即$Ax_1(t) = x_2(t)$,其中$A$是一个常数.于是问题就化为一个单自由度简谐运动.

此时${\ddot{x}_1}/{\ddot{x}_2}=A$,代入\autoref{MSHO_eq1} 得
\begin{equation}
\frac{k_{11}+k_{12}A}{k_{21}+k_{22}A}=A
\end{equation}
即
\begin{equation}
k_{22}A^2+(k_{21}-k_{12})A-k_{11} = 0
\end{equation}
因此$A$是确定的:
\begin{equation}
A = \frac{(k_{12}-k_{21})\pm\sqrt{(k_{12}-k_{21})^2 + 4k_{22}k_{11}}}{2k_{22}}
\end{equation}

这样能得到两个倍率$A$,记为$A_1$和$A_2$.在继续解方程之前,我举一个例子来方便你体会两个倍率的意义:
\begin{figure}[ht]
\centering
\includegraphics[width=10cm]{./figures/MSHO_3.pdf}
\caption{一个简单的二自由度简谐振子模型.} \label{MSHO_fig3}
\end{figure}

\autoref{MSHO_fig3} 所示的简谐振子体系高度对称,因此很容易发现两种特殊情况:一是两个滑块之间的距离恒为中间弹簧的原长,作\textbf{完全相同}的单自由度简谐运动,此时中间弹簧等同于不存在,这个情况对应$A=1$;二是$A=-1$,两个滑块作\textbf{完全对称}的运动.第二种情况比第一种情况的频率高一些,因为每个滑块都额外受中间弹簧的力了.

得到$A_i$以后,\autoref{MSHO_eq1} 化为
\begin{equation}
\ddot{x}_1 = (k_{11}+A_ik_{12})x_1
\end{equation}
从而得\textbf{特解}
\begin{equation}
\leftgroup{
    x_1(t) &= C_i\cos(\sqrt{-k_{11}-A_ik_{12}}t+\phi_i)\\
    x_2(t) &= A_iC_i\cos(\sqrt{-k_{11}-A_ik_{12}}t+\phi_i)
}
\end{equation}

将各$A_i$对应的特解组合起来,就能得到通解:
\begin{equation}
\leftgroup{
    x_1(t) &= C_i\cos(\sqrt{-k_{11}-A_ik_{12}}t+\phi_i)\\
    x_2(t) &= A_iC_i\cos(\sqrt{-k_{11}-A_ik_{12}}t+\phi_i)
}
\end{equation}






