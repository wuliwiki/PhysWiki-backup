% 策梅洛-弗兰克尔集合论(综述)
% license CCBYSA3
% type Wiki

本文根据 CC-BY-SA 协议转载翻译自维基百科\href{https://en.wikipedia.org/wiki/Zermelo\%E2\%80\%93Fraenkel_set_theory}{相关文章}。

在集合论中,泽梅洛-弗兰克尔集合论,以数学家恩斯特·泽梅洛和亚伯拉罕·弗兰克尔命名,是一个公理化系统,旨在20世纪初提出,以构建一个没有类似罗素悖论之类悖论的集合理论。如今,泽梅洛-弗兰克尔集合论,包含历史上具有争议的选择公理(AC),是标准的公理化集合论形式,因此也是数学的最常见基础。包含选择公理的泽梅洛-弗兰克尔集合论简称为ZFC,其中C代表“选择”\(^\text{[1]}\),而ZF指的是没有包含选择公理的泽梅洛-弗兰克尔集合论的公理。

非正式地说\(^\text{[2]}\),泽梅洛-弗兰克尔集合论旨在形式化一个单一的基本概念,即遗传的良基集合,使得所有在讨论宇宙中的实体都是这样的集合。因此,泽梅洛-弗兰克尔集合论的公理仅涉及纯集合,并防止其模型中包含尿元素(不是集合本身的元素)。此外,适当类(由其成员共享的属性定义的数学对象集合,这些集合过大无法作为集合处理)只能间接处理。具体来说,泽梅洛-弗兰克尔集合论不允许存在一个普遍集合(包含所有集合的集合),也不允许无限制的理解,从而避免了罗素悖论。冯·诺依曼-伯奈斯-哥德尔集合论(NBG)是泽梅洛-弗兰克尔集合论的一个常用保守扩展,它确实允许对适当类进行显式处理。

泽梅洛-弗兰克尔集合论的公理有许多等价的表述。大多数公理陈述了从其他集合定义的特定集合的存在。例如,配对公理指出,给定任何两个集合\(a\)和\(b\),存在一个新的集合\(\{a,b\}\),其中恰好包含\(a\) 和 \(b\)。其他公理描述了集合成员关系的属性。这些公理的目标是,若将每个公理解释为关于冯·诺依曼宇宙中所有集合的陈述(也称为累积层次结构),那么每个公理都应该成立。

泽梅洛-弗兰克尔集合论的元数学已经得到了广泛的研究。该领域的标志性成果确立了选择公理与其他泽梅洛-弗兰克尔公理的逻辑独立性,以及连续统假设与ZFC的逻辑独立性。像ZFC这样的理论的一致性不能在该理论内部证明,正如哥德尔的第二不完备性定理所示。
\subsection{历史}
现代集合论的研究由乔治·康托尔和理查德·德德金德在1870年代发起。然而,天真的集合论中的悖论(如罗素悖论)的发现,促使人们寻求一种更加严谨的集合论形式,以避免这些悖论。

1908年,恩斯特·泽梅洛提出了第一个公理化集合论——泽梅洛集合论。然而,正如亚伯拉罕·弗兰克尔在1921年给泽梅洛的信中首次指出的,这一理论无法证明某些集合和基数的存在,而这些集合和基数在当时大多数集合论学者中被视为理所当然,特别是基数阿列夫-欧米伽(\(\aleph_\omega\))以及集合 \(\{Z_0, \mathcal{P}(Z_0), \mathcal{P}(\mathcal{P}(Z_0)), \mathcal{P}(\mathcal{P}(\mathcal{P}(Z_0))), \dots\}\),其中 \(Z_0\) 是任何无限集合,\(\mathcal{P}\) 是幂集运算。\(^\text{[3]}\)此外,泽梅洛的一个公理使用了一个概念,即“确定”属性,其操作性含义并不明确。1922年,弗兰克尔和托拉尔夫·斯科勒姆独立提出将“确定”属性操作化为能够以一阶逻辑中能够表达的良构公式来表示,其中原子公式仅限于集合成员关系和相等关系。他们还独立提出用替换公理代替规范公理公理模式。将此公理模式以及正则性公理(由约翰·冯·诺依曼首次提出)添加到泽梅洛集合论中,得到的理论被称为ZF。将选择公理(AC)或等价的陈述添加到ZF中,得到的理论就是ZFC。
\subsection{形式语言}  
形式上,ZFC 是一种一排序理论,使用一阶逻辑。等号符号可以被视为一个原始的逻辑符号,或者作为具有完全相同元素的高层缩写。前者的方法最为常见。其符号系统有一个谓词符号,通常表示为\(\in\),这是一个二元关系符号,表示集合成员关系。例如,公式\(a \in b\)表示\(a\)是集合\(b\)的元素(也可以读作\(a\)是\(b\)的成员)。

有不同的方法来表述形式语言。一些作者可能会选择不同的连接词或量词。例如,逻辑连接词 NAND 本身就可以编码其他连接词,这种特性称为功能完备性。本节旨在在简洁性和直观性之间找到平衡。
