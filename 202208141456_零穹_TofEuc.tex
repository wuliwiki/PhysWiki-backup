% 度规张量与指标升降(欧氏空间)
% 度规张量|指标升降

\pentry{欧几里得矢量空间\upref{EuVS},张量的坐标\upref{CofTen},爱因斯坦求和约定\upref{EinSum}}
欧氏矢量空间是在矢量空间中定义了一个正定的双线性型 $(*|*)$\upref{EuVS}(它也称为矢量空间上的内积),双线性型又是一个2阶共变张量,这表明可以直接用张量来描述欧氏矢量空间,这只需将欧氏矢量空间中的双线性型 $(*|*)$ 用一个张量 $G$ 来代替:
\begin{equation}
G:=g_{ij} e^i\otimes e^j
\end{equation}
其中,$g_{ij}=G(e_i,e_j)=(e_i|e_j)$.上式使用了爱因斯坦求和约定\upref{EinSum}(以下也将使用这一约定),其中张量坐标和基的指标是对立(见\upref{CofTen}).由于双线性型 $(*|*)$ 给定了欧氏矢量空间中的度量性质,所以这个欧氏矢量空间中的双线性型 $(*|*)$ 对应的张量 $G$ 叫作欧氏矢量空间的\textbf{度规张量}(\textbf{度量张量}).

在给定了 $(*|*)$ 的欧氏矢量空间 $V$ 中,$V$ 和 $V^*$ 是自然同构的(\autoref{EVOIOG_the3}~\upref{EVOIOG}),即 $v\in V $ 和 $(v|*)\in V^*$ 没有本质上的区别.这也是定义了内积的矢量空间相对于未定义内积的矢量空间多出的性质,这一性质在用张量表述时表现在指标的升降上.

\textbf{约定}:为方便描述起见,同时和大多数习惯相一致,我们将用张量的坐标表示来替代张量本身.即若 $T=T^{ij} e_{i}\otimes e_{j}$,则直接用 $T^{ij}$ 来表示张量 $T$.事实上,这样并未丢失任何信息,要还原张量 $T$ 只需在 $T^{ij}$ 后补上基矢 $e_{i}\otimes e_{j}$ 即可.使用这一约定,$g_{ij}$ 就是度规张量本身.
\subsection{度规张量}
\begin{definition}{度规张量}
设 $(V,(*|*))$ 是欧几里得矢量空间,则坐标
\begin{equation}
g_{ij}=(e_i|e_j)
\end{equation}
对应的二阶共变张量 $g_{ij}$ 称为$V$ 上的\textbf{度规张量}(或\textbf{度量张量}).
\end{definition}
有了度规张量的定义,$V$ 上两矢量的内积 $(x|y)$ 便可记成
\begin{equation}
(x|y)=g_{ij}x^i y^i
\end{equation}
于是矢量的模平方便成了
\begin{equation}
\norm{x}^2=g_{ij}x^i x^j
\end{equation}

\begin{theorem}{度规张量的性质}
\begin{enumerate}
\item \textbf{对称性:}\begin{equation}
g_{ij}=g_{ji}
\end{equation}
\item \textbf{坐标变换规律:} 设 $g'_{ij},g_{ij}$ 分别是在基底 $\{e'_i\},\{e_i\}$ 中的表示,则
\begin{equation}\label{TofEuc_eq1}
g'_{ij}=A^k_{i} A^l_{j}g_{kl}
\end{equation}
其中,$A^i_j$ 是基底 $\{e_i\}$ 到 $\{e'_i\}$ 的转换矩阵.
\end{enumerate}
\end{theorem}
\textbf{证明:}对称性由内积的对称性直接得到;\autoref{TofEuc_eq1} 证明如下:
\begin{equation}
g'_{ij}=(e'_i|e'_j)=(A^k_i e_k|A^l_j e_l)=A^k_{i} A^l_{j}(e_k|e_l)=A^k_{i} A^l_{j}g_{kl}
\end{equation}

