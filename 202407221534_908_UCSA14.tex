% 中国科学院大学 2014 年 考研 量子力学
% license Usr
% type Note

\textbf{声明}:“该内容来源于网络公开资料,不保证真实性,如有侵权请联系管理员”

\subsection{一}
若已知算符 $\hat{A} = \begin{pmatrix}
1 & 0 \\
0 & C - 1
\end{pmatrix}$ 和算符 $\hat{B} = \begin{pmatrix}
0 & -i \\
i & 0
\end{pmatrix}$,其中 $C$ 为实常数,当 $C = C_0$ 有共同的本征函数。

1. 求 $C_0$ 的值。

2. 求当 $C = C_0$ 时算符 $\hat{A}$ 和 $\hat{B}$ 的共同本征函数。

3. 求当 $C \neq C_0$,求由 $\hat{A}$ 表像到 $\hat{B}$ 表像的变换矩阵。

\subsection{二}
两个质量均为$\mu$的非全同粒子被禁锢在$0 \le x \le L$的无限深势阱中。

1. 忽略两个粒子间的相互作用,求系统的三个最低能量及相应的归一化波函数。

2. 假设粒子间的相互作用是为$v = \lambda \delta (x_1 - x_2)$的微弱相互作用,求系统的三个最低能量($\lambda$的一级近似)及相应的归一化波函数。

\subsection{三}
一个质量为$\mu$电荷为$q$自旋为$0$的粒子被限制在$x-y$平面内半径为$a$的圆周上运动。

1. 求该粒子的哈密顿量$\hat H$及自旋算符的第二分量$\hat{L}_z$的本征值及相应的归一化本征函数。

2. 若在$Z$方向上加一磁场$\vec{B}$,求系统的哈密顿量$\hat{H}$的本征值及相应的归一化本征函数,并讨论加入磁场后能级及简并度的变化。(提示取矢势$A=\frac{1}{2}(\hat{B} x\hat{r})$)