% 费曼-海尔曼定理(综述)
% license CCBYSA3
% type Wiki

本文根据 CC-BY-SA 协议转载翻译自维基百科\href{https://en.wikipedia.org/wiki/Hellmann\%E2\%80\%93Feynman_theorem}{相关文章}。

在量子力学中,赫尔曼–费曼定理将总能量相对于某个参数的导数与哈密顿量对该参数的导数的期望值关联起来。根据该定理,一旦通过求解薛定谔方程确定了电子的空间分布,系统中的所有力都可以使用经典静电学来计算。

该定理已被许多作者独立证明,包括保罗·古廷格(1932年)、沃尔夫冈·泡利(1933年)、汉斯·赫尔曼(1937年)和理查德·费曼(1939年)。

定理表述为:
\[
\frac{\mathrm{d} E_{\lambda}}{\mathrm{d} \lambda} = \langle \psi_{\lambda} | \frac{\mathrm{d} \hat{H}_{\lambda}}{\mathrm{d} \lambda} | \psi_{\lambda} \rangle~
\]
其中:
\begin{itemize}
\item \(\hat{H}_{\lambda}\) 是一个依赖于连续参数 \(\lambda\) 的厄米算符,
\item \(|\psi_{\lambda} \rangle\) 是哈密顿量的本征态(本征函数),隐含地依赖于 \(\lambda\),
\item \(E_{\lambda}\) 是状态 \(|\psi_{\lambda} \rangle\) 的能量(本征值),即 \(\hat{H}_{\lambda} |\psi_{\lambda} \rangle = E_{\lambda} |\psi_{\lambda} \rangle\)。
\end{itemize}
注意,在热力学极限下,赫尔曼-费曼定理在量子临界点附近会出现失效。[5]