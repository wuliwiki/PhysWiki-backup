% 函数视角下的三角函数(高中)
% keys 函数|三角函数|周期|性质
% license Usr
% type Tutor

\begin{issues}
\issueDraft
\end{issues}

\pentry{三角函数\nref{nod_HsTrFu},函数\nref{nod_functi},函数的性质\nref{nod_HsFunC},导数\nref{nod_HsDerv}}{nod_5a43}

在前面的内容中,已经接触过三角函数的定义,并基于这些定义推导出了诱导公式及同角三角函数之间的关系。这些推导主要依赖于任意角和三角函数的几何定义。然而,三角函数不仅仅是几何分析的工具,它们本质上也是一种函数,并具备一般函数的基本性质,如周期性、单调性和对称性。因此,本文将从函数的角度进一步分析三角函数,考察它们的性质、图像、变化趋势等。需要注意的是,这些视角本质上是等价的,它们都在描述同一数学对象。无论是几何定义还是函数分析,最终指向的都是相同的数学结构。这种多重视角的统一性,正是三角函数作为数学工具的强大之处。它不仅能够通过直观的几何形式展现对称性和变换规律,也能在函数的框架下揭示更广泛的性质,为各种数学应用提供坚实的基础。

另外,在三角函数的介绍中,有一个广为流传的动画:一个点在单位圆上运动,表示角度的变化,同时,在单位圆的右侧和上侧,将角度与对应的线段长度映射到另一坐标系,从而自然引出各个三角函数的图像。尽管这种动画能够直观展示三角函数的变化过程,更理想的方式是能够在脑海中主动演练这一过程。看到函数图像时,能够自动联想到单位圆上的点如何旋转;反之,观察圆周运动时,能够迅速在脑海中构建出相应的函数图像。这种能力不仅有助于理解三角函数的本质,也将在更深入的数学学习中提供帮助。本文内容主要关注正弦、余弦与正切函数,其余三角函数由于与它们存在倒数关系,将适当涉及,但不会展开详细推导。

\subsection{前文提及的性质总结}

\subsubsection{定义域与值域}

由于自变量是任意角,所以理论上所有三角函数的定义域都应该是实数域,然而在讨论三角函数的集合含义时曾经\aref{提及}{eq_HsTrFu_13}过,$\tan x$在$x\in$

正弦函数和余弦函数的\textbf{定义域为全体实数},正切函数的定义域为 $\begin{Bmatrix}\alpha|\alpha \neq \frac{\pi}{2}+k\pi,k\in Z\end{Bmatrix}~.$

\subsection{周期性}

正弦函数、余弦函数、正切函数的都是周期函数,根据定义易得,正弦函数和余弦函数,周期为 $2k\pi(k\in Z,k\neq0)$,正切函数的周期为 $k\pi(k\in Z,k\neq0)$.


\subsection{图像}
根据前面的推导,可以得到基本三角函数的图像如下图。
\begin{figure}[ht]
\centering
\includegraphics[width=14.25cm]{./figures/14fd66d8d1e6e0b5.png}
\caption{$\sin x$和$\cos x$} \label{fig_HsTFFv_1}
\end{figure}
可以看出正弦函数和余弦函数是定义域为 $R$ 值域为 $[-1,1]$ 最小正周期 $T = 2\pi$ 的周期函数。

\begin{figure}[ht]
\centering
\includegraphics[width=14.25cm]{./figures/6f97182187b36e36.png}
\caption{$\tan x$和$\cot x$} \label{fig_HsTFFv_3}
\end{figure}

作为扩展,下面也给出正割函数与余割函数的函数图像,他们的性质均可通过与正弦和余弦的关系分析得到,此处不予赘述。

\begin{figure}[ht]
\centering
\includegraphics[width=14.25cm]{./figures/56f93ee1a7fb0faa.png}
\caption{$\sec x$和$\csc x$} \label{fig_HsTFFv_2}
\end{figure}

\subsection{从三角函数推广得到的其他三角函数}

\subsection{导数}