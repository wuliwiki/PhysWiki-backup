% 背包问题
% keys 动态规划|背包问题|算法|dp

背包问题是 $\tt dp$ 问题中给一个很大的分支,可以归类于\textbf{组合数 $\tt dp$}.

背包问题大致为这么几类,分别为:$01$ 背包问题、多重背包问题(Ⅰ、Ⅱ)、完全背包问题、分组背包问题、混合背包问题、二维费用的背包问题、有依赖的背包问题、背包问题求方案数、背包问题求具体方案.

\subsection{$01$ 背包问题}

题意:有 $N$ 件物品和一个能被重为 $W$ 的背包.第 $i$ 件物品的重量是 $w$,价值是 $v$ .每件物品只能用一次,求解将哪些物品装入背包里物品价值总和最大.

\textbf{状态表示:}$f(i, j)$ 表示只从前 $i$ 个物品中选,并且总体积不超过 $j$ 的选法的集合.

\textbf{
状态计算:}依据为第 $i$ 的物品选还是不选划分为两个不重不漏的集合.

\begin{enumerate}
\item 不选择第 $i$ 个物品,状态转移为 $f(i - 1, j)$.
\item 选择第 $i$ 个物品,对应着:从前 $i - 1$ 个物品中选,总体积不超过 $j - v_i$,并且把第 $i$ 个物品的价值加上.状态转移为:$f(i - 1, j - v_i) + w_i$.
\end{enumerate}

\textbf{时间复杂度:}朴素做法需要两重循环,第一维枚举物品,第二维枚举体积.因此时间复杂度为:$\mathcal{O}(n \times m)$.

\textbf{朴素代码:}

\begin{lstlisting}[language=cpp]
const int M = 1010;
int N, V; // 物品数量, 背包容积
int v[M], w[M];  // 第 i 件物品的体积和价值
int f[M][M];

int main()
{
    cin >> N >> V;
    for (int i = 1; i <= N; i ++ ) cin >> v[i] >> w[i];

    for (int i = 1; i <= N; i ++ )
        for (int j = 1; j <= V; j ++ )
        {
            f[i][j] = f[i - 1][j]; // 不包含 i 的情况
            if (v[i] <= j)  // 如果要选 i, 物品的体积不能超过背包的容积
                f[i][j] = max(f[i][j], f[i - 1][j - v[i]] + w[i]);
        }

    cout << f[N][V] << endl;

    return 0;
}
\end{lstlisting}

因为状态转移每次只依赖上一个状态,即 $f(i, j)$ 只依赖于 $f(i - 1, j)$ 这个状态.因此可以优化至一维.

一维状态转移方程:$f_j = max(f_j, f_j - v_i + w_i)$.

但要注意优化至一维是枚举体积的时候要倒着循环.简单的证明一下:

如果正着循环.

\begin{lstlisting}[language=cpp]
for (int i = 1; i <= n; i ++ ) 
    for (int j = v[i]; j <= m; j ++ ) 
        f[j] = max(f[j], f[j - v[i]] + w[i]);
\end{lstlisting}

对于二维状态:$f(i, j)$ 需要由 $f(i - 1, j - v_i)$ 得来,但是化成一维时,(用二维理解)可见 $f(i, j)$ 是由 $f(i, j - v_i)$ 得来的,而不是由 $f(i - 1, j - v_i)$ 得来的.

例子:假设有 $2$ 件物品,背包的总体积为 $8$.

\begin{lstlisting}[language=cpp]
    物品    体积    价值
    1       3       6
    2       7       1
\end{lstlisting}

正序模拟过程如下:

\begin{lstlisting}[language=cpp]
f[3] = max(f[3], f[0] + w[1] = 6)  == f[3] = 6 
f[4] = max(f[4], f[1] + w[1] = 6)  == f[4] = 6 
f[5] = max(f[5], f[2] + w[1] = 6)  == f[5] = 6 

// 到 f[6] 出错了,f[3] 应该是没被计算过的,但是正序循环,导致被重复计算.
f[6] = max(f[6], f[3] + w[1] = 6)  == f[6] = 12
f[7] = max(f[7], f[4] + w[1] = 6)  == f[7] = 12 
f[8] = max(f[8], f[5] + w[1] = 6)  == f[8] = 12 
f[9] = max(f[9], f[6] + w[1] = 6)  == f[9] = 18 
f[10] = max(f[10], f[7] + w[1] = 6)  == f[10] = 18 
f[7] = max(f[7], f[0] + w[2] = 1)  == f[7] = 12 
f[8] = max(f[8], f[1] + w[2] = 1)  == f[8] = 12 
f[9] = max(f[9], f[2] + w[2] = 1)  == f[9] = 18 
f[10] = max(f[10], f[3] + w[2] = 1)  == f[10] = 18 
18
\end{lstlisting}

倒序循环模拟过程如下:

\begin{lstlisting}[language=cpp]
f[10] = max(f[10], f[7] + w[1] = 6)  == f[10] = 6 
f[9] = max(f[9], f[6] + w[1] = 6)  == f[9] = 6 
f[8] = max(f[8], f[5] + w[1] = 6)  == f[8] = 6 
f[7] = max(f[7], f[4] + w[1] = 6)  == f[7] = 6 

// f[6] 中的 f[3] 没被计算,因此可以得出正确答案.
f[6] = max(f[6], f[3] + w[1] = 6)  == f[6] = 6 
f[5] = max(f[5], f[2] + w[1] = 6)  == f[5] = 6 
f[4] = max(f[4], f[1] + w[1] = 6)  == f[4] = 6 
f[3] = max(f[3], f[0] + w[1] = 6)  == f[3] = 6 
f[10] = max(f[10], f[3] + w[2] = 1)  == f[10] = 7 
f[9] = max(f[9], f[2] + w[2] = 1)  == f[9] = 6 
f[8] = max(f[8], f[1] + w[2] = 1)  == f[8] = 6 
f[7] = max(f[7], f[0] + w[2] = 1)  == f[7] = 6 
7
\end{lstlisting}

\textbf{一维状态代码如下:}

\begin{lstlisting}[language=cpp]
int n, m, f[10005], v, w;

int main()
{
    cin >> n >> m;
    while (n -- )
    {
        cin >> v >> w;
        for (int j = m; j >= v; j -- )
            f[j] = max(f[j], f[j - v] + w);
    }
    
    cout << f[m] << endl;
}
\end{lstlisting}

\subsection{完全背包问题}

\textbf{题意:}有 $N$ 件物品和一个能被重为 $W$ 的背包.第 $i$ 件物品的重量是 $w$,价值是 $v$ .每件物品可以用\textbf{无限次},求解将哪些物品装入背包里物品价值总和最大.

\textbf{状态表示:}$f(i, j)$ 表示只从前 $i$ 个物品中选,并且总体积不超过 $j$ 的选法的集合.

\textbf{
状态计算:}依据为第 $i$ 的物品选几个划分为 $s$ 个子集.为什么不是无限个子集呢?因为总体积是有限制的.

\begin{enumerate}
\item 第 $i$ 个物品一个都不选,状态转移为:$f(i - 1, j)$.
\item 第 $i$ 个物品选一个,对应着:从 $i - 1$ 个物品中选,总体积不超过 $j - v_i$,并且把第 $i$ 个物品的价值加上.状态转移为:$f(i - 1, j - v_i) + w_i$.
\item 第 $i$ 个物品选两个,对应着:从 $i - 1$ 个物品中选,总体积不超过 $j - 2 \times v_i$,并且把第 $i$ 个物品的价值加两次.状态转移为:$f(i - 1, j - 2 \times v_i) + 2 \times w_i$.
\end{enumerate}

\textbf{时间复杂度:}朴素做法需要三重循环,第一维枚举物品,第二维枚举体积,第三维枚举决策(第 $i$ 个物品选择几个),因此时间复杂度为:$\mathcal{O}(n \times m^2)$.

\textbf{三维朴素代码:}

\begin{lstlisting}[language=cpp]
const int M = 1010;
int f[M][M], N, V, v[M], w[M];

int main()
{
    scanf("%d%d", &N, &V);

    for (int i = 1; i <= N; i ++ ) scanf("%d%d", &v[i], &w[i]);

    for (int i = 1; i <= N; i ++ ) 
        for (int j = 1; j <= V; j ++ ) 
            for (int k = 0; k * v[i] <= j; k ++ )
                f[i][j] = max(f[i][j], f[i - 1][j - k * v[i]] + k * w[i]);
    cout << f[N][V] << endl;

    return 0;
}
\end{lstlisting}

观察一下状态转移方程可以优化成二维:

\begin{equation}
f(i, j) = \max(f(i - 1, j), f(i - 1, j - v) + w, f(i - 1, j - 2v) + 2w , \cdots , f(i - 1, j - sv) + sw)
\end{equation}

有:

\begin{equation}
f(i, j - v) = max(f(i - 1, j - v), f(i - 1, j - 2v) + w, f(i - 1, j - 3v) + 2w , \cdots, f(i - 1, j - sv) + (s - 1)w)
\end{equation}

因此不难看出第二个方程的第一项到最后一项只比第一个方程的第二项到最后一项少了 $w$.

因此状态转移方程可优化为:$f(i, j) = \max(f(i - 1, j), f(i, j - v) + w$.如此一来,第三重循环就可以删去了,因此时间复杂度为 $\mathcal{O}(n \times m)$.

同样可以优化成一维:$f_j = \max(f_j, (f_j - v) + w)$,此时循环可以从小到大循环,因为 $f(i, j)$ 从 $f(i, j - v + w)$ 转移得来.

\textbf{一维代码:}

\begin{lstlisting}[language=cpp]
const int M = 1010;
int f[M], N, V, v[M], w[M];

int main()
{
    scanf("%d%d", &N, &V);
    for (int i = 1; i <= N; i ++ ) scanf("%d%d", &v[i], &w[i]);
    
    for (int i = 1; i <= N; i ++ ) 
        for (int j = v[i]; j <= V; j ++ ) 
            f[j] = max(f[j], f[j - v[i]] + w[i]);

    cout << f[V] << endl;

    return 0;
}
\end{lstlisting}

