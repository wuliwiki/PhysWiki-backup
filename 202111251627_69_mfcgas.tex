% 金属中的自由电子气体
% keys 费米气体|费米分布|费米能级|费米面|自由电子气体
\pentry{玻尔兹曼分布(统计力学)\upref{MBsta},热力学量的统计表达式(玻尔兹曼分布)\upref{TheSta},理想气体单粒子能级密度\upref{IdED1}}

经典统计理论用能量均分定理求得的金属中自由电子的热容与实际不符,这是经典统计理论的又一困难.1928年索末菲(Sommerfeld)根据费米分布成功解决了这个问题.

金属中的自由电子互相碰撞的概率可忽略不计,自由电子与原子实碰撞的概率也可忽略不计.电子为费米子,可以将金属中的自由电子看成费米气体.

经典近似条件为 $e^{\alpha}\gg 1$,在这个条件下,费米分布可近似为玻尔兹曼分布.根据\autoref{MBsta_eq8}~\upref{MBsta},该条件等价于
\begin{equation}
\frac{V}{N}\qty(\frac{2\pi mkT}{h^2})^{3/2}\gg 1
\end{equation}
电子的质量 $m$ 非常小,不满足该条件\footnote{以铜为例,铜的密度是 $8.9\times 10^3 \rm{kg\cdot m^{-3}}$,相对原子质量为 $63.5$,如果一个铜原子贡献一个自由电子,则 $N/V=n=8.5\times 10^{28} m^{-3}$.电子的质量为 $9.1\cdot 10^{-31} \rm{kg}$,于是 $e^\alpha=\frac{V}{N}\qty(\frac{2\pi m kT}{h^2})^{3/2}=2.8\cdot 10^{-8}T^{3/2}$},在 $T=300K$ 时,$e^\alpha\approx 10^{-4}$.因此有显著的量子效应.我们必须用费米分布来推导相关公式.

\subsection{自由电子气体}
根据费米分布,温度为 $T$ 时能量 $\epsilon$ 上的一个量子态上的平均电子数为
\begin{equation}
f(\epsilon)=\frac{1}{e^{\frac{\epsilon-\mu}{kT}}+1}
\end{equation}
根据\autoref{IdED1_eq2}~\upref{IdED1},在 $\epsilon~\epsilon+\dd \epsilon$ 间电子的量子态数为(考虑电子的自旋量子态,还要乘以 $2$)
\begin{equation}
D(\epsilon) \dd \epsilon = \frac{4\pi V}{h^3}(2m)^{3/2}\epsilon^{1/2}\dd \epsilon
\end{equation}
所以总粒子数为
\begin{equation}
N=\frac{4\pi V}{h^3}(2m)^{3/2}\int_0^\infty \frac{\epsilon^{1/2}\dd \epsilon}{e^{\frac{\epsilon-\mu}{kT}}+1}
\end{equation}
根据 $V,m,N$ 的值,可以由上式确定电子气体化学势 $\mu$.这个积分式被称为费米积分,很难求出解析解,但我们可以对 $T\rightarrow 0 \rm{K}$ 的情况求近似.
\begin{figure}[ht]
\centering
\includegraphics[width=12cm]{./figures/mfcgas_1.png}
\caption{当温度 $T$ 较高时,$f(\epsilon )$关于 $\epsilon$ 的图像} \label{mfcgas_fig1}
\end{figure}
\begin{figure}[ht]
\centering
\includegraphics[width=12cm]{./figures/mfcgas_2.png}
\caption{当温度 $T$ 较低时,$f(\epsilon )$关于 $\epsilon$ 的图像} \label{mfcgas_fig2}
\end{figure}

如\autoref{mfcgas_fig2} 所示,当温度 $T\rightarrow 0$ 时,$f(\epsilon)$ 几乎成为阶梯函数.我们称温度 $0\rm{K}$ 时的电子气体的化学势 $\mu(0)$ 为\textbf{费米能级}.当 $\epsilon<\mu$ 时,$f(\epsilon)=1$,当 $\epsilon>\mu$ 时,$f(\epsilon)=0$.

电子气体总能量为
\begin{equation}
E=\frac{4\pi V}{h^3}(2m)^{3/2}\int_0^\infty \frac{\epsilon^{3/2}\dd \epsilon}{e^{\frac{\epsilon-\mu}{kT}}+1}
\end{equation}
