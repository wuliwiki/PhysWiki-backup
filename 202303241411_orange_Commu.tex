% 对易算符
% 对易|交换子|分配律|交换律|算符

定义算符 $A,B$ 的\textbf{对易子(commutator)}为
\begin{equation}
[A, B] = AB - BA~,
\end{equation}
那么对易可以表示为 $[A,B] = 0$, 反之不对易表示为 $[A,B]\ne 0$。


\pentry{映射\upref{map}}

定义两个算符(即映射) $A: X\to X$ 和 $B: X\to X$ 的\textbf{交换子(commutator)}(也叫\textbf{对易算符}) 为
\begin{equation}
[A, B] = A B - B A~.
\end{equation}
两个算符相等在这里的意义是它们作用在任意 $x \in X$ 上, 结果都相等。 算符的乘法 $AB$ 在这里表示复合算符(即复合映射 $A\circ B$)。

根据定义, 当 $A B = B A$ 就有 $[A, B] = 0$, 这时我们就说它们\textbf{对易(commutes)}或者\textbf{交换}, 否则就\textbf{不对易}。

\begin{exercise}{}
一般来说, 矩阵乘法不满足交换律。 试证明任意两个平面旋转矩阵\upref{Rot2D}(二维几何矢量空间\upref{GVec}的算符)对易(实数 $\alpha \ne \beta$)
\begin{equation}
\mat A  = \pmat{\cos\alpha & -\sin\alpha\\ \sin\alpha & \cos\alpha}
\qquad
\mat B  = \pmat{\cos\beta & -\sin\beta\\ \sin\beta & \cos\beta}~.
\end{equation}
\end{exercise}

\begin{exercise}{}
令 $X$ 为所有连续可导函数的集合, 定义算符 $A$ 作用在任意 $f\in X$ 上的结果 $Af$ 为函数 $(Af)(x) = x f(x)$, 定义算符 $B$ 为 $(Bf)(x) = \dv*{f}{x}$。 试证明对易算符 $[A, B]$ 是单位算符, 即 $[A, B]f = f$。
\end{exercise}

\subsection{常见性质}
\begin{equation}
[B, A] = -[A, B]~.
\end{equation}
分配律
\begin{equation}
[A, B + C] = [A, B] + [A, C]
\qquad
[A + B, C] = [A, C] + [B, C]
\end{equation}
\begin{equation}
[A, BC] = [A, B]C + B[A, C]~.
\end{equation}
\textbf{雅可比恒等式(Jacobi identity)}
\begin{equation}
[A, [B, C]] + [B, [C, A]] + [C, [A, B]] = 0~.
\end{equation}
