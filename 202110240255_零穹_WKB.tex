% WKB 近似

\begin{issues}
\issueDraft
\end{issues}
\footnote{李蕴才.高等量子力学[M].开封:河南大学出版社,2000:337-347}在某些定态问题中,WKB近似方法可以比较容易地求解薛定谔方程.该方法基于将波函数按 $\hbar$ 作幂级数展开,就其本身而言,有两个基本问题:1.在远离转折点处的近似解;2.在转折点处的连接条件.并通过这两个问题求解薛定谔方程.

经典区域
\begin{equation}
\psi(x) \approx \frac{C}{\sqrt{p(x)}} \exp(\pm \I \int p(x) \dd{x})
\end{equation}

隧道区域
\begin{equation}
\psi(x) \approx \frac{C}{\sqrt{\abs{p(x)}}} \exp(\pm \int p(x) \dd{x})
\end{equation}

在二者的转折点, 假设势能为线性函数, 其解是艾里函数 $\opn{Ai}$, 详见 “线性势能的定态薛定谔方程\upref{LinPot}”.

从经典到非经典区域
\begin{equation}
\psi(x) = \leftgroup{
&\frac{B}{\sqrt{p(x)}} \exp(\I \int_{-x}^{x_0} p(x')\dd{x'}) + \frac{C}{\sqrt{p(x)}} \exp(-\I \int_{-x}^{x_0} p(x')\dd{x'}) \qquad &(x < x_0)\\
&\frac{D}{\sqrt{\abs{p(x)}}} \exp(-\int_{x_0}^x \abs{p(x')} \dd{x'})  \qquad &(x > x_0)
}\end{equation}
衔接以后
\begin{equation}
\psi(x) = \leftgroup{
&\frac{2D}{\sqrt{p(x)}} \sin(\int_{x}^{x_0} p(x')\dd{x'} + \frac{\pi}{4}) \quad &(x < x_0)\\
&\frac{D}{\sqrt{\abs{p(x)}}} \exp(-\int_{x_0}^x \abs{p(x')} \dd{x'}) \quad &(x > x_0)
}\end{equation}
\subsection{证明}
\subsubsection{WKB近似解}
薛定谔方程
\begin{equation}\label{WKB_eq2}
\I\hbar\pdv{\psi}{t}=-\frac{\hbar^2}{2m}\Delta\psi+V(\bvec{r})\psi
\end{equation}
的解一般总能写成如下形式
\begin{equation}\label{WKB_eq1}
\psi(\bvec r,t)=A\E^{\I W(\bvec r,t)/\hbar}
\end{equation}
\autoref{WKB_eq1} 代入\autoref{WKB_eq2} 得到 $W$ 满足的方程
\begin{equation}\label{WKB_eq3}
\pdv{W}{t}+\frac{1}{2m}(\nabla W)^2+V-\frac{\I\hbar}{2m}\Delta W=0
\end{equation}
在经典极限($\hbar\rightarrow 0$)下,\autoref{WKB_eq3} 等同 $W$ (称为主函数)的哈密顿方程
\begin{equation}
\pdv{W}{t}+\frac{1}{2m}(\nabla W)^2+V=0
\end{equation}
如果 $\psi$ 是能量本征函数 $u(\bvec E)\E^{-\I Et/\hbar}$,则 $W$ 可写成
\begin{equation}
W(\bvec r,t)=S(\bvec r)-Et
\end{equation}
在这种情况下,我们有
\begin{equation}
u(\bvec r)=A\E^{\I S(\bvec r)}
\end{equation}
且
\begin{equation}
\frac{1}{2m}(\nabla S)^2-\qty[E-V(\bvec r)]-\frac{\I\hbar}{2m}\Delta S=0
\end{equation}

在一维情况下,WKB 方法得到 $S$ 的按 $\hbar$ 幂展开式的头两项可以清楚的给出.下面以一维定态为例.

一维薛定谔方程可写为\cite{Sakurai}
\begin{equation}\label{WKB_eq4}
\begin{aligned}
&\dv[2]{u}{x}+\qty[k(x)]^2u=0\\
&k(x)\equiv\left\{\begin{aligned}
&\qty[\frac{2m}{\hbar^2}\qty(E-V(x))]^{1/2},\quad \text{when}\; E>V(x)\\
&-\I\kappa(x)\equiv-\I\qty[\frac{2m}{\hbar^2}(V(x)-E)]^{1/2},\quad \text{when}\; E<V(x)
\end{aligned}\right.
\end{aligned}
\end{equation}

我们寻求\autoref{WKB_eq4} 如下形式的解
\begin{equation}\label{WKB_eq10}
u(x)=A\E^{\I S(x)/\hbar}
\end{equation}
将上式代入\autoref{WKB_eq4} ,得
\begin{equation}\label{WKB_eq6}
\I\hbar S''-S'^2+\hbar^2\qty[k(x)]^2=0
\end{equation}
其中,撇号表示对 $x$ 求导.将 $S(x)$ 按 $\hbar$ 之幂作级数展开
\begin{equation}\label{WKB_eq5}
S(x)=\sum_{n=0}\hbar^nS_n(x)
\end{equation}
\autoref{WKB_eq5} 代入\autoref{WKB_eq6} 得
\begin{equation}
\sum_{n=0}\qty(\I S_n''-\sum_{i=0}^{n+1}S_{n+1-i}'S_i')\hbar^{n+1}+\hbar^2\qty[k(x)]^2-S_0'^2=0
\end{equation}
按幂次将上式分解为多个全微分方程
\begin{equation}
\begin{aligned}
-S_0'^2+\hbar^2\qty[k(x)]^2&=0\\
\hbar\qty(\I S_0''-2S_0'S_1')&=0\\
\cdots
\end{aligned}
\end{equation}
为求出一级近似解,通过比较系数,给出头两个方程
\begin{equation}\label{WKB_eq7}
\begin{aligned}
-S_0'^2+\hbar^2k^2=0;\\
\I S_0''-2S_0'S_1'=0
\end{aligned}
\end{equation}
容易求得\autoref{WKB_eq7} 第一式的解为
\begin{equation}\label{WKB_eq8}
S_0(x)=\pm\hbar\int^x k(y)\dd y
\end{equation}
利用上式和\autoref{WKB_eq7} 第一式可得\autoref{WKB_eq7} 第二式的解
\begin{equation}\label{WKB_eq9}
S_1(x)=\frac{\I}{2}\ln k(x)
\end{equation}
\autoref{WKB_eq8} 和\autoref{WKB_eq9} 已经将任意常数略去了.将 $S_0$ 和 $S_1$ 的结果代入\autoref{WKB_eq10} ,可得 $u(x)$ 的一级近似解
\begin{equation}\label{WKB_eq11}
u(x)=A\E^{\I(S_0+\hbar S_1)/\hbar}=\frac{A}{\sqrt{k}} \E^{\pm\I\int^x k(y)\dd y}
\end{equation}
\subsubsection{解的渐进性质}
欲使\autoref{WKB_eq11} 的解成立,显然必须要求展开式\autoref{WKB_eq5} 是合理的.为了保证一级近似解适用,应该要求
\begin{equation}\label{WKB_eq13}
\abs{\frac{\hbar(\I S_0''-2S_0'S_1')}{-S_0'^2+\hbar^2k^2}}<<1
\end{equation}
由\autoref{WKB_eq8} 和\autoref{WKB_eq9} 
\begin{equation}\label{WKB_eq12}
S_0'=\pm\hbar k,\quad S_0''=\pm\hbar k',\quad S_1'=\frac{\I k'}{2k}
\end{equation}
\autoref{WKB_eq12} 代入\autoref{WKB_eq13} ,得
\begin{equation}
\abs{\frac{k'}{k^2}}<<1
\end{equation}
引入波长 $\lambda=\frac{2\pi}{\abs{k}}$ ,则
\begin{equation}\label{WKB_eq14}
\frac{\lambda}{2\pi}\abs{\frac{k'}{k}}<<1
\end{equation}
将\autoref{WKB_eq4} 代入\autoref{WKB_eq14} 
\begin{equation}\label{WKB_eq15}
\frac{\lambda}{2\pi}\abs{\frac{\dv{V}{x}}{2(E-V)}}<<1
\end{equation}
\autoref{WKB_eq14} 表明,在可以和 $\lambda$ 相比的小距离中,“动量”的相对变化 $k'/k$ 要十分小.为保证这一点,只要 $V(x)$ 随 $x$ 变化得十分缓慢,而粒子的“动能” $(E-V)$ 又很大,使得 $\dv{V}{x}$ 与 “动能”相比要小得多,则\autoref{WKB_eq14} 和\autoref{WKB_eq15} 才成立.这表明,WKB近似方法的适用有着它的适用范围.

显然,当 $E=V(x)$ 时(满足此式的式子称为转折点),\autoref{WKB_eq13} 不成立.由此 WKB近似不适用,只有在离开最近的转折点几个波长的地方,WKB近似解才可能是有效的.

以上的讨论完全适用于中心力场的径向方程,因为在中心力场径向方程中,只需要将 $r$ 代替 $x$ ,以 $V(\bvec r)+\frac{\hbar^2l(l+1)}{2m+2}$ 代替 $V(x)$ ,以 $u(\bvec r)$ 代替 $u(x)$ ,相应方程便与一维薛定谔方程完全类同.
\subsubsection{连接公式}
既然在转折点附近\autoref{WKB_eq5} 和\autoref{WKB_eq11} 的解不适用.因此需要寻找 WKB波函数在转折点处的连接公式,这需要另辟蹊径.
\begin{figure}[ht]
\centering
\includegraphics[width=5cm]{./figures/WKB_1.pdf}
\caption{转折点 $x=a$ 附近的势能曲线} \label{WKB_fig1}
\end{figure}

为方便讨论,不失一般性,通常假定在 $x=a$ 处有一转折点,即设
\begin{equation}
\left\{\begin{aligned}
&E>V(x),\quad x>a(\text{经典允许区})\\
&E=V(x),\quad x=a\\
&E<V(x),\quad x<a(\text{经典禁区})
\end{aligned}\right.
\end{equation}
如\autoref{WKB_fig1} 所示.

建立连接公式基本思想是:首先在转折点附近求解薛定谔方程;其次给出该解在离开转折点“较远处”所具有的渐进形式;最后,将这渐进解形式与 WKB 解相比较,从而找出WKB波函数的连接公式.

\textbf{1.在转折点附近求解薛定谔方程}

由于 $V(x)$ 随 $x$ 发生缓慢变化,因此可在点a处将 $V(x)$ 泰勒展开,并取线性部分,即
\begin{equation}
V(x)=V(a)+V'(a)(x-a)=E+V'(a)(x-a)
\end{equation}
 其中,$V'(a)<0$.于是在 $a$ 点附近的薛定谔方程为
 \begin{equation}\label{WKB_eq16}
 \dv[2]{u}{x}-\frac{2mV'(a)}{\hbar^2}(x-a)u=0
 \end{equation}
令
\begin{equation}
\begin{aligned}
&\alpha=\qty(\frac{2mV'(a)}{\hbar^2})^{\frac{1}{3}}\\
&\xi=\alpha(x-a)
\end{aligned}
\end{equation}
则\autoref{WKB_eq16} 式变为
\begin{equation}\label{WKB_eq17}
\dv[2]{u}{\xi}-\xi u=0
\end{equation}
这是艾里函数\upref{AiryF}满足的微分方程\autoref{AiryF_eq2}~\upref{AiryF}.显然,$u$ 应是 $\mathrm{Ai}(\xi)$ 和 $\mathrm{Bi}(\xi)$ 的线性组合
\begin{equation}
u=c_1 \mathrm{Ai}(\xi)+c_2 \mathrm{Bi}(\xi)
\end{equation}

在经典允许区($\xi<0$)中,$\mathrm{Ai}(\xi),\mathrm{Bi}(\xi)$ 都是 \autoref{WKB_eq17} 满足有限条件的解.在经典禁区($\xi>0$),\autoref{WKB_eq17} 满足有限条件的解为$\mathrm{Ai}(\xi)$.为使得 $u$ 在 $\xi=0$ 处连续,必有 $c_2=0$,所以
\begin{equation}\label{WKB_eq18}
u=c_1\mathrm{Ai}(\xi)
\end{equation}

\textbf{2.渐进形式}

由 $\mathrm{Ai(x)}$ 函数的渐进形式\upref{AiryF} ,得\autoref{WKB_eq18} 的渐进形式
\begin{equation}
\begin{aligned}
&c_1\opn{Ai}(\xi) \overset{\xi \to +\infty}{\longrightarrow} \frac{c_1}{2\sqrt{\pi} \xi^{1/4}} \exp(-\frac{2}{3}\xi^{3/2})\\
&c_1\opn{Ai}(\xi) \overset{\xi \to -\infty}{\longrightarrow} \frac{c_1}{\sqrt{\pi} \abs{\xi}^{1/4}} \sin(\frac{2}{3}\abs{\xi}^{3/2}+\frac{\pi}{4})\\
\end{aligned}
\end{equation}

\textbf{3.连接公式}
我们的问题是要给出在“远离”转折点的WKB近似解的连接公式.为此,现在考虑\autoref{WKB_fig2} 的 $V(x)$.
\begin{figure}[ht]
\centering
\includegraphics[width=4cm]{./figures/WKB_2.pdf}
\caption{具有两个转折点的势能曲线} \label{WKB_fig2}
\end{figure}
在阱内 $(a<x<b)$ 有 $E>V(x)$,仍为经典允许区,按\autoref{WKB_eq11} ,WKB一级近似解可表示为
\begin{equation}
u(x)=\frac{c}{\sqrt{k}}\sin(\int^x k(y)\dd y+\delta)
\end{equation}
其中, $\delta$ 为待定相因子.
\addTODO{Griffiths 例题 Potential well with one vertical wall.}
