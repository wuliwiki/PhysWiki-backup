% 庞家莱群
% license Usr
% type Tutor


\begin{issues}
\issueTODO 引用半直积
\end{issues}


\subsection{庞家莱群}
时空平移群,顾名思义,能对任意时空矢量$x\in \mathbb R^{1,3} $进行平移,所以群元就是李群$\mathbb R^{1,3}$本身,群乘法为加法。时空平移群与洛伦兹群的半直积构成\textbf{庞加莱群(Poincaré group)},用$G$简便表示为:
\begin{equation}
G= \mathrm{O}(1,3) \ltimes\mathbb{R}^{1,3}~,
\end{equation}
因此群元素包含了所有可能的坐标变换。对于任意$(\Lambda_1,a),(\Lambda_2,b),(\Lambda_1,a)\in G$,从物理意义出发,群乘法必然为:

\begin{equation}
(\Lambda_1,a)(\Lambda_2,b)=(\Lambda_1\Lambda_2,\Lambda_1b+a)~.
\end{equation}
显然与半直积的乘法运算自洽。此外,半直积的符号表明$(I,\mathbb R^{1,3})\vartriangleleft G$。接下来我们证明庞家莱群满足群定义,且时空平移群是正规子群。

显然庞家莱群的单位元为$(I,0)$,设任意$(\Lambda,a)\in G$,设其逆元素为$(\Lambda^{-1},b)$,对任意坐标矢量$x$作用表示为
\begin{equation}
(\Lambda^{-1},b)(\Lambda,a)x=\Lambda^{-1}(\Lambda x+a)+b=x~,
\end{equation}
解得$b=-\Lambda^{-1} a$\footnote{或者直接利用群乘法求解。$(\Lambda^{-1},b)(\Lambda,a)=(I,b+\Lambda^{-1}a)=(I,0)$},显然在庞家莱群内。再由定义可知,封闭性和结合性自然满足,所以这确实是群。

再设任意$(I,b)\in G$,因为
\begin{equation}
\begin{aligned}
(\Lambda,a)^{-1}(I,b)(\Lambda,a)x&=(\Lambda^{-1},-\Lambda^{-1}a)(I,b)(\Lambda,a)x\\
&=\left\{\Lambda^{-1}\left [\left(\Lambda x+a \right)+b\right]-\Lambda^{-1}a\right\}x\\
&=(I,\Lambda^{-1}b)x~.
\end{aligned}
\end{equation}
因此,时空平移群同构于庞家莱群的正规子群。



\subsubsection{庞家莱群的李代数}
庞家莱群的李代数,即单位元处的切空间,由洛伦兹群的切空间$\mathfrak {so}(1,3)$与时空平移群的切空间直和而成。为了得到李代数的形式,我们需要找到这两个群的共同表示空间。

假设这两个群的群元都有线性算符形式,分别表示为$\Q L(\Lambda),\Q L(a)$,且共同作用在标量函数上。在坐标系进行洛伦兹变换后,
\begin{equation}\label{eq_PoinGr_1}
\begin{aligned}
\phi'(x')&=\phi(x)\\
\phi'(x)&=\phi(\Lambda^{-1}x)\equiv \Q L(\Lambda)\phi(x)~.
\end{aligned}
\end{equation}

这个定义自然满足对标量函数线性性以及群同态定义。因为对于任意$\Lambda_1,\Lambda_2\in SO^+(1,3)$,我们有$\Q L(\Lambda_1\Lambda_2)=\Q L(\Lambda_1)\Q L(\Lambda_2)$。由洛伦兹群一节可知,任意洛伦兹群元都可以表示为$\E^{-\I t(\bvec a^i\bvec J_i+\bvec b^i\bvec K_i)}$。为了方便,我们引入物理上另一种常见的表示方式:
\begin{equation}
\Lambda=\E^{-\frac{i}{2} \omega_{\mu \nu} J^{\mu \nu}}
~,
\end{equation}
其中$\omega_{\mu\nu}$与$J^{\mu\nu}$皆为反对称张量,且\footnote{下列定义中,定义等号取的是张量的矩阵形式。}
\begin{equation}\label{eq_PoinGr_2}
\begin{aligned}
J_i&=(J_1,J_2,J_3)\equiv(J^{23},J^{31},J^{12});a^i=(a^1,a^2,a^3)\equiv(\omega_{23},\omega_{31},\omega_{12}).\\
K_i&=(K_1,K_2,K_3)\equiv(J^{01},J^{02},J^{03});b^i=(b^1,b^2,b^3)\equiv(\omega_{01},\omega_{02},\omega_{03})~.
\end{aligned}
\end{equation}


同态对$\Lambda$而言是线性的,意味着$\Q L(\Lambda)$也有这样满的指数映射。令$\Q L(J^{\mu \nu})=L^{\mu \nu}$,则任意$\Q L(\Lambda)=\E^{-\frac{i}{2} \omega_{\mu \nu} L^{\mu \nu}}$,代入\autoref{eq_PoinGr_1} 后作泰勒展开并保留一阶项——
\begin{equation}
\begin{aligned}
\Q L(\Lambda)\phi(x)&=\phi(\Lambda^{-1}x)\\
&=\phi(x-\delta x)\\
&= \phi(x)-\delta x^{\mu} \partial_{\mu} \phi(x) \\
&= \phi(x)-\omega^{\mu}{ }_{\nu} x^{\nu} \partial_{\mu} \phi(x) \\
&=\phi(x)-\omega_{\mu \nu} x^{\nu} \partial^{\mu} \phi(x) \\
&= \phi(x)-\frac{i}{2} \omega_{\mu \nu} L^{\mu \nu} \phi(x)~.
\end{aligned}
\end{equation}
其中$L^{\mu\nu}\equiv=\I(x^{\mu}\partial^{\nu}-x^{\nu}\partial^{\mu})$。显然这就是我们需要的,$SO^+(1,3)$线性算符表示的李代数。我们还需要验证一下李括号关系。类似\autoref{eq_PoinGr_2}  ,$L_i=(L_1,L_2,L_3)\equiv(L^{23},L^{31},L^{12})$。则
\begin{equation}
\begin{aligned}
~[L_1,L_2]~&=-[x^{2}\partial^{3}-x^{3}\partial^{2},x^{3}\partial^{1}-x^{1}\partial^{3}]\\
&=-[x^2\partial^3,x^3\partial^1]-[x^3\partial^2,x^1\partial^3]\\
&=-x^2[\partial^3,x^3]\partial^1-x^1[x^3,\partial^3]\partial^2\\
&=\I L_3~.
\end{aligned}
\end{equation}
其他同理亦然。关键是利用$[x^{\mu},\partial^{\nu}]=\eta^{\mu\nu}$,$\eta^{\mu\nu}$的矩阵形式为$\opn{diag}(-1,1,1,1)$。

设$\Q L(a)$为时空平移群的线性算符表示,假设同样可以用指数映射表示对坐标系平移任意时空四矢量,即$\Q L(a)=$
