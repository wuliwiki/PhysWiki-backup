% 二次型
% 二次型|规范型|对角型

\pentry{双线性型(2-线性函数)\upref{Tensor}}
\begin{definition}{二次型}
\footnote{科斯特利金.代数学引论, 第二卷.}域 $\mathbb{F}$ 上有限维空间 $V$ 上的函数 $q:V\rightarrow\mathbb{F}$ ,若它满足如下两个性质:
\begin{enumerate}
\item $q(-\bvec{v})=q(\bvec v),\quad \forall\bvec v\in V$;
\item 由公式
\begin{equation}\label{QuaFor_eq1}
f(\bvec x,\bvec y)=\frac{1}{2}\qty[q(\bvec x+\bvec y)-q(\bvec x)-q(\bvec y)]
\end{equation}
决定的映射 $f:V\times V\rightarrow\mathbb{F}$ 是 $V$ 上的双线性型\upref{Tensor}(显然是对称的).
\end{enumerate}
则称 $q$ 是 $V$ 上的\textbf{二次型(quadratic form)},并称 $f$ 的秩为 $q$ 的秩:rank $q$=rank $f$.
\end{definition}
利用\autoref{QuaFor_eq1} ,由 $q$ 得到的对称的双线性型 $f$ 称为\textbf{极化的},或 $f$ 是与二次型 $q$ \textbf{配极} 的双线性型.
\begin{example}{}
设 $f$ 是 $V$ 上任意一个对称的双线性型,令
\begin{equation}
q_f(\bvec x)=f(\bvec x,\bvec x)
\end{equation}
就得到一个满足二次型定义的函数 $q_f:V\rightarrow\mathbb{F}$ ,因为
\begin{equation}
\begin{aligned}
q_f(-\bvec{x})&=f(-\bvec{x},-\bvec{x})=f(\bvec{x},\bvec{x})=q_f(\bvec{x}) \quad \forall\bvec x\in V\\
f(\bvec x,\bvec y)&=\frac{1}{2}\qty[f(\bvec x+\bvec y,\bvec x+\bvec y)-f(\bvec x,\bvec x)-f(\bvec y,\bvec y)]\\
&=\frac{1}{2}\qty[q(\bvec x+\bvec y)-q(\bvec x)-q(\bvec y)]
\end{aligned}
\end{equation}
\end{example}
\begin{theorem}{}\label{QuaFor_the1}
每一个二次型 $q$ 都可以按着自己的配极双线性型 $f$ 唯一地恢复原型;换言之, $q=q_f$
\end{theorem}
\textbf{证明:}在\autoref{QuaFor_eq1} 中令 $\bvec y=-\bvec x$ :
\begin{equation}
-f(\bvec x,\bvec x)=\frac{1}{2}[q(\bvec 0)-q(\bvec x)-q(-\bvec x)]
\end{equation}
从而
\begin{equation}
q(\bvec x)=f(\bvec x,\bvec x)+\frac{1}{2}q(\bvec 0)
\end{equation}
因为 $f$ 是个双线性型,所以 $f(\bvec 0,\bvec 0)=0$ .因为,当 $\bvec x=0$ 时有 $q(\bvec 0)=\frac{1}{2}q(\bvec 0)$ ,即 $q(\bvec 0)=0$,也就是说, $q(\bvec x)=f(\bvec x,\bvec x)$.

每一个二次型按\autoref{QuaFor_eq1} 定义一个与其配极对称双线性型 $f$ ,而由\autoref{QuaFor_the1} ,每一个对称的双线性型 $f$ 有唯一一个二次型 $q$ 与之对应,这就是说,二次型和对称双线性型一一对应.
\begin{definition}{二次型的矩阵}
称与 $q$ 配极的双线性型 $f$ 在空间 $V$ 的基底 $(\bvec e_1,\cdots,\bvec e_n)$ 之下的矩阵 $F$ 是二次型 $q=q_f$ 的矩阵.
\end{definition}
\begin{example}{}
设 $F=(f(\bvec e_i,\bvec e_j))=(f_{ij})$ ,则
\begin{equation}
f_{ij}=\frac{1}{2}[q(\bvec e_i+\bvec e_j)-q(\bvec e_i)-q(\bvec e_j)]\quad i,j=1,2,\cdots n.
\end{equation}
任意一个对称的矩阵 $F=(f_{ij})$ 也适应由关系式\autoref{Tensor_exe1}~\upref{Tensor}
\begin{equation}
q(\bvec x)=X^tFX=\sum_\limits{i,j} f_{ij}x_ix_j\quad i,j=1,\cdots ,n
\end{equation}
给定的二次型 $q$ .其中 $X=[x_1,\cdots,x_n]$为列向量.
\end{example}
\begin{definition}{二次型的规范型(或对角型)}
称二次型 $q$ 在 $V$ 的基底 $(\bvec e_1,\cdots,\bvec e_n)$ 之下具有\textbf{规范型}或\textbf{对角型},如果对 $\forall\bvec x=\sum x_i\bvec e_i\in V$ ,$q(\bvec x)$ 的值可用公式
\begin{equation}
q(\bvec x)=\sum_{i}f_{ii}x_i^2
\end{equation}
计算.此时基底 $(\bvec e_i)$ 称为对 $q$ 的\textbf{规范基底}.
\end{definition}