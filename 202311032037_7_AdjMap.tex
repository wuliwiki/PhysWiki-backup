% 伴随映射
% keys 线性代数|伴随映射|伴随算子|伴随变换|adjoint operator|adjoint map|adjoint transformation|dual map|对偶映射
% license Xiao
% type Tutor

\pentry{线性映射\upref{LinMap}, 对偶空间\upref{DualSp}}



\subsection{伴随映射的概念}


给定线性空间,则可由此导出其对偶空间。如果给定两个线性空间$V$和$W$之间的线性映射,则我们可以自然导出对应的$W^*$和$V^*$之间的线性映射,即所谓的伴随映射。



\begin{definition}{伴随映射}

\begin{figure}[ht]
\centering
\includegraphics[width=6cm]{./figures/22f579955f587872.pdf}
\caption{伴随映射的概念。} \label{fig_AdjMap_1}
\end{figure}

给定域$\mathbb{F}$上的线性空间$V$和$W$,对于\textbf{线性映射}$f:V\to W$,定义其伴随映射为$f^*:W^*\to V^*$,使得对于任意$\bvec{w}\in W^*$和$\bvec{v}\in V$都有
\begin{equation}
\langle f(\bvec{v}), \bvec{w} \rangle = \langle \bvec{v}, f^*(\bvec{w}) \rangle~. 
\end{equation}
其中$\langle *, * \rangle$表示对偶向量之间的相互作用。
\end{definition}

注意,定义伴随映射时,我们已知的两个元素$\bvec{v}$和$\bvec{w}$分别位于\autoref{fig_AdjMap_1} 的左上角和右下角,它们之间不能相互作用。但是,$f$能够把$\bvec{v}$挪到左下角,使得二者可以相互作用;而其对偶映射$f^*$的作用则是反过来,把$\bvec{w}$挪到右上角,同样使得二者可以相互作用。



当给定$V$的基$\{\bvec{e}_i\}_{i=1}^n$和$W$的基$\{\bvec{\varepsilon}_j\}_{j=1}^m$后,$f$可以表示为矩阵
\begin{equation}
\begin{pmatrix}
a^1_1&a^1_2&\cdots&a^1_n\\
a^1_1&a^1_2&\cdots&a^1_n\\
\vdots&\vdots
\end{pmatrix}~. 
\end{equation}
令$\{\bvec{\theta}_i\}_{i=1}^n$和$\{\bvec{\tau}_j\}_{j=1}^m$分别是这两个基的\textbf{对偶基},则$f^*$可以表示为矩阵
\begin{equation}
\begin{pmatrix}
f^*(\bvec{\theta}_1, \bvec{\tau}_1)
\end{pmatrix}~. 
\end{equation}
























