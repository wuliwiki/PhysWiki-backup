% 蒙特卡罗方法(综述)
% license CCBYSA3
% type Wiki

本文根据 CC-BY-SA 协议转载翻译自维基百科\href{https://en.wikipedia.org/wiki/Monte_Carlo_method}{相关文章}。

蒙特卡罗方法,又称蒙特卡罗实验,是一类依赖重复随机抽样以获得数值解的广义计算算法。其基本思想是利用随机性来解决一些本质上可能是确定性的问题。该名称源自摩纳哥的蒙特卡罗赌场,该方法的主要发展者、数学家斯坦尼斯瓦夫·乌拉姆受到他叔叔赌博习惯的启发而命名。

蒙特卡罗方法主要用于三类问题:优化问题,数值积分,从概率分布中抽样生成样本。此外,它还可以用于对输入存在高度不确定性的现象进行建模,例如评估核电站故障的风险。蒙特卡罗方法通常通过计算机模拟实现,能为那些无法解析求解或过于复杂的问题提供近似解。

蒙特卡罗方法广泛应用于多个科学与工程领域,包括:物理、化学、生物学、统计学、人工智能、金融、密码学,也被应用于社会科学,如:社会学、心理学与政治学。它被认为是20世纪最重要且最具影响力的思想之一,并推动了众多科学与技术突破的实现。

不过,蒙特卡罗方法也存在一些局限性和挑战,例如:精度与计算成本之间的权衡,维度灾难,随机数生成器的可靠性,结果的验证与确认。
\subsection{概述}
蒙特卡罗方法虽然形式多样,但通常遵循以下基本步骤:
\begin{enumerate}
\item 定义可能输入的范围(即问题的定义域);
\item 从该范围内的概率分布中随机生成输入样本;
\item 对每个输入进行确定性的计算,得到输出结果;
\item 汇总(聚合)所有结果,以得到最终的估计或解。
\end{enumerate}
\begin{figure}[ht]
\centering
\includegraphics[width=6cm]{./figures/db49926bb4b1422c.png}
\caption{蒙特卡罗方法用于近似计算\(\pi\)的值} \label{fig_MTKL_1}
\end{figure}
例如,考虑一个内接于单位正方形的象限圆扇形。由于它们面积的比值是 $\frac{\pi}{4}$,因此可以使用蒙特卡罗方法来近似计算 $\pi$ 的值:\(^\text{[1]}\)
\begin{enumerate}
\item 画一个正方形,并在其中内接一个象限圆(四分之一圆);
\item 在正方形内均匀随机撒布一定数量的点;
\item 统计落在象限圆内的点的数量(即其到原点的距离小于1的点);
\item 象限内点数与总点数之比是两者面积比 $\frac{\pi}{4}$ 的估计值。
将该比值乘以4,即可得到 $\pi$ 的估计值。
\end{enumerate}
在上述过程中,输入的定义域是包围该象限的正方形。可以通过在正方形内随机撒点(如撒沙粒)来生成输入,然后对每个输入执行一次计算,以判断它是否落在象限圆内。将结果汇总后,就可以得到最终的估计值——对 $\pi$ 的近似。

这里有两个重要注意事项:
\begin{enumerate}
\item 如果点不是均匀分布的,则估计结果将不准确;
\item 随机点的数量越多,估计的精度就越高。
\end{enumerate}
蒙特卡罗方法的应用需要大量的随机数,因此它的发展极大地受益于伪随机数生成器的出现,这些生成器比早期所用的随机数表要高效得多。
\subsection{应用}
蒙特卡罗方法常用于物理和数学问题,尤其在其他方法难以或无法使用时最为有用。蒙特卡罗方法主要应用于三类问题:\(^\text{[2]}\):优化问题,数值积分,从概率分布中生成样本(抽样)。

在与物理相关的问题中,蒙特卡罗方法非常适用于模拟具有多个耦合自由度的系统,例如流体、无序材料、强耦合固体以及细胞结构等(参见:细胞 Potts 模型、相互作用粒子系统、McKean–Vlasov 过程、气体动力学模型)。

其他应用示例包括:建模输入存在重大不确定性的现象,例如商业风险评估;在数学中,用于求解具有复杂边界条件的多维定积分;在系统工程问题中(如航天、石油勘探、飞机设计等),蒙特卡罗方法在预测失效概率、成本超支和进度延误方面,往往优于人类直觉或其他“软性”方法。\(^\text{[3]}\)

从原理上讲,任何具有概率解释的问题都可以使用蒙特卡罗方法求解。根据大数定律,由某随机变量期望值所定义的积分可以通过该变量的独立样本的经验平均值(即样本均值)来近似。当所研究的随机变量具有参数化的概率分布时,数学家通常会使用马尔可夫链蒙特卡罗方法(MCMC)。\(^\text{[4][5][6]}\) 其核心思想是构造一个具有指定平稳分布的合适马尔可夫链模型。也就是说,在极限下,由 MCMC 方法生成的样本将来自目标分布。\(^\text{[7][8]}\)根据遍历定理,该平稳分布可以通过 MCMC 采样器生成的随机状态的经验分布来近似。

在其他类型的问题中,目标是从满足非线性演化方程的一系列概率分布中生成样本。这些概率分布的演化流总可以解释为某个马尔可夫过程的随机状态分布,其转移概率依赖于当前状态的分布(参见:McKean–Vlasov 过程、非线性滤波方程)。\(^\text{[9][10]}\)在另外一些情形下,会出现采样复杂度不断上升的概率分布流,例如:随时间增长的路径空间模型,与降温参数相关的玻尔兹曼–吉布斯测度,以及其他类似模型。这些模型也可以看作是非线性马尔可夫链的随机状态分布的演化过程。\(^\text{[10][11]}\)一种模拟此类复杂非线性马尔可夫过程的自然方法是:对该过程生成多个独立副本,用样本的经验分布来替代演化方程中未知的状态分布。与传统蒙特卡罗或 MCMC 方法不同,这些被称为平均场粒子方法的方法依赖于一组连续相互作用的样本。术语“平均场”指的是,每一个样本(也称为粒子、个体、行者、代理、实体或表型)都与过程的经验分布发生相互作用。当系统规模趋于无穷大时,这些随机的经验测度将收敛到非线性马尔可夫链中随机状态的确定性分布,从而粒子之间的统计相互作用也将消失。
\subsection{简单蒙特卡罗方法}
假设我们希望知道某个总体的期望值 $\mu$(且已知该期望值存在),但无法通过公式直接计算。简单蒙特卡罗方法通过运行 $n$ 次模拟并对结果取平均,来对 $\mu$ 进行估计。

该方法对模拟输入的概率分布没有限制,只要求以下几点:输入是随机生成的,输入彼此相互独立,$\mu$存在。只要 $n$ 足够大,所得估计值 $m$ 就能**任意接近于**真实值 $\mu$。更正式地说,对于任意 $\epsilon > 0$,都可以有$|\mu - m| \leq \epsilon$\(^\text{[12]}\)

通常用于计算 $m$ 的算法如下:
\begin{figure}[ht]
\centering
\includegraphics[width=14.25cm]{./figures/99fbcdd21ea48ed3.png}
\caption{} \label{fig_MTKL_2}
\end{figure}
\subsubsection{一个示例}
假设我们想知道:掷三个八面骰子,总点数至少为 $T$,我们期望需要掷多少次才能满足这一条件。我们知道该期望值是存在的。骰子的每次投掷是随机的且相互独立的,因此可以应用简单蒙特卡罗方法:
\begin{figure}[ht]
\centering
\includegraphics[width=14.25cm]{./figures/679fe4eb3bb18f45.png}
\caption{} \label{fig_MTKL_3}
\end{figure}
如果 $n$ 足够大,则对于任意 $\epsilon > 0$,估计值 $m$ 将会落在真实值 $\mu$ 的 $\epsilon$ 范围内。
\subsection{确定足够大的 $n$}

\textbf{通用公式}

设$\epsilon = |\mu - m| > 0$,其中 $\mu$ 是真实期望值,$m$ 是蒙特卡罗估计值。首先,选择所需的置信水平——也就是希望在蒙特卡罗算法运行完成后,估计值$m$落在$\mu$的$\epsilon$ 范围内的概率。设对应该置信水平的标准正态$z$-分数为 $z$。

设$s^2$为估计方差(有时称为“样本方差”),它是从一个相对较小的样本数量 $k$ 次模拟中计算得到的结果的方差。选择一个$k$值;Driels 和 Shin 指出:“即使样本量比所需数量小一个数量级,这个估算结果通常也相当稳定。”\(^\text{[13]}\)

以下算法可以单遍计算 $s^2$,同时最小化累积数值误差导致的错误结果风险:\(^\text{[12]}\)
\begin{figure}[ht]
\centering
\includegraphics[width=14.25cm]{./figures/8dcfec3f3fc4e736.png}
\caption{} \label{fig_MTKL_4}
\end{figure}
请注意,当算法运行结束时,$m_k$是从 $k$ 次模拟中得到的平均值。

当满足以下条件时,$n$ 被认为是足够大的:
$$
n \geq \frac{s^2 z^2}{\epsilon^2},^\text{[12][13]}~
$$
如果$n \leq k$,那么$m_k = m$,说明已有的 $k$ 次模拟已足以保证 $m_k$ 在误差 $\epsilon$ 范围内逼近 $\mu$。如果$n > k$,那么可以选择:重新“从头开始”运行 $n$ 次模拟,或因为已完成了 $k$ 次模拟,只需再运行 $n - k$ 次模拟,并将它们的结果加入原来的样本统计中即可:
\begin{figure}[ht]
\centering
\includegraphics[width=14.25cm]{./figures/e52479c12430563f.png}
\caption{} \label{fig_MTKL_5}
\end{figure}
\textbf{当模拟结果有界时的估算公式}

在所有模拟结果都有上下界的特例中,可以使用另一种估算公式。

选择一个 $\epsilon$ 值,使其为 $\mu$ 与 $m$ 之间最大允许误差的两倍。设置信心水平 $\delta$,满足$0 < \delta < 100$,并以百分比表示。假设每次模拟结果$r_1, r_2, \ldots, r_i, \ldots, r_n$都满足$a \leq r_i \leq b$,其中 $a$ 和 $b$ 是有限实数。若希望以至少 $\delta\%$ 的置信度满足$|\mu - m| < \epsilon/2$,

则应选择满足以下条件的 $n$:
$$
n \geq 2(b - a)^2 \ln\left(2/(1 - (\delta/100))\right)/\epsilon^2~
$$
例如,若 $\delta = 99\%$,则有:$n \geq 2(b - a)^2 \ln(2 / 0.01)/\epsilon^2 \approx 10.6(b - a)^2/\epsilon^2$.\(^\text{[12]}\)
\subsection{计算成本}
尽管蒙特卡罗方法在概念和算法上都非常简单,但其计算成本可能极其高昂。一般而言,该方法需要大量样本才能获得良好的近似结果,而如果每个样本的处理时间很长,那么总运行时间可能会大得无法接受。\(^\text{[14]}\)尽管这一点在处理非常复杂的问题时是一个严重的限制,但该算法具有天然的“尴尬并行性”,因此可以通过各种并行计算策略来降低这一高昂成本,使其达到可行的水平,例如使用:本地多处理器,计算集群,云计算,GPU(图形处理器),FPGA(现场可编程门阵列)等。\(^\text{[15][16][17][18]}\)
\subsection{历史}
在蒙特卡罗方法发展之前,模拟通常是用于测试已知的确定性问题,而统计抽样则被用于估算这些模拟中的不确定性。蒙特卡罗模拟则反其道而行之,利用概率性元启发式方法(如模拟退火)来求解确定性问题。

蒙特卡罗方法的一个早期变种用于解决著名的布丰投针问题,该问题通过向平行等距条纹地板上投掷针来估算圆周率 $\pi$。20世纪30年代,恩里科·费米在研究中子扩散问题时首次尝试了蒙特卡罗方法,但他并未发表相关研究成果。\(^\text{[19]}\)

20世纪40年代末,斯坦尼斯瓦夫·乌拉姆在美国洛斯阿拉莫斯国家实验室参与核武器项目期间,发明了现代版本的马尔可夫链蒙特卡罗方法。1946年,洛斯阿拉莫斯的核武器物理学家正在研究核武器核心中中子的扩散问题。\(^\text{[19]}\)尽管他们掌握了大部分必要的数据,比如中子在物质中与原子核发生碰撞之前的平均行进距离,以及中子发生碰撞后可能释放的能量等,但他们无法用传统的确定性数学方法解决这个问题。乌拉姆提出可以使用随机实验来处理此类问题。他这样回忆当时的灵感来源:

我第一次尝试实践[蒙特卡罗方法]的想法,是源于1946年一个我在患病康复期间玩纸牌游戏(单人接龙)时想到的问题。当时我在想:用52张牌摆出的 Canfield 单人接龙游戏成功的几率有多大?我花了很长时间试图通过纯粹的组合计算来估算它,但后来我开始思考,是否有比“抽象思维”更实用的方法,比如干脆实际玩一百次,然后观察并统计成功次数。由于当时正值高速计算机新时代的开端,这个思路开始变得可行。我很快就联想到中子扩散问题及其他数学物理问题,进而更一般地思考,如何将某些微分方程所描述的过程转化为一系列可解释为随机操作的等效形式。后来(也是在1946年),我把这个想法讲给约翰·冯·诺依曼听,我们开始计划实际的计算工作。\(^\text{[20]}\)

由于相关工作属于机密,冯·诺依曼和乌拉姆的研究需要一个代号。\(^\text{[21]}\)他们的同事尼古拉斯·梅特罗波利斯提议使用“蒙特卡罗”这个名称,来源于摩纳哥的蒙特卡罗赌场,乌拉姆的叔叔曾常常向亲戚借钱去那里赌博。\(^\text{[19]}\)蒙特卡罗方法在战后核武器进一步发展的模拟计算中发挥了核心作用,包括氢弹的设计,尽管当时的计算工具极其有限。1948年春,冯·诺依曼、尼古拉斯·梅特罗波利斯及其他人使用ENIAC 计算机进行了首次完全自动化的蒙特卡罗计算,模拟对象是裂变武器核心。\(^\text{[22]}\)到了20世纪50年代,洛斯阿拉莫斯国家实验室在氢弹研发中大量应用了蒙特卡罗方法,该方法也逐渐在物理学、物理化学和运筹学等领域普及开来。兰德公司和美国空军是当时推广和资助蒙特卡罗方法研究的两大主要机构,该方法随后在许多不同领域中得到广泛应用。

更为复杂的平均场类型粒子蒙特卡罗方法的理论,最迟在20世纪60年代中期就已开始发展,起始于亨利·P·麦基恩二世关于流体力学中出现的一类非线性抛物型偏微分方程的马尔可夫解释的研究。\(^\text{[23][24]}\)而更早的一篇开创性文章由西奥多·E·哈里斯和赫尔曼·卡恩于1951年发表,文章中使用了平均场遗传类型的蒙特卡罗方法来估算粒子的传输能量。\(^\text{[25]}\)平均场遗传类型的蒙特卡罗方法也被用作进化计算中的自然启发式搜索算法(又称元启发式算法,metaheuristic)。这种平均场计算技术的起源可以追溯到:1950年和1954年,艾伦·图灵关于遗传型突变–选择学习机的工作,\(^\text{[26]}\)以及尼尔斯·奥尔·巴里切利在新泽西普林斯顿高等研究院发表的相关论文。\(^\text{[27][28]}\)

量子蒙特卡罗方法,特别是扩散蒙特卡罗方法,也可以被解释为对费曼–卡克路径积分的平均场粒子蒙特卡罗近似。\(^\text{[29][30][31][32][33][34][35]}\)量子蒙特卡罗方法的起源通常归功于恩里科·费米和罗伯特·里希特迈尔,他们于1948年发展了中子链式反应的平均场粒子解释模型。\(^\text{[36]}\)然而,第一个启发式或遗传型粒子算法(又称重采样蒙特卡罗或重构蒙特卡罗方法)用于估算量子系统基态能量(在约化矩阵模型中),是杰克·H·赫瑟林顿在1984年提出的。\(^\text{[35]}\)在分子化学领域,使用遗传启发式粒子方法(也称为剪枝与增强策略)可以追溯到1955年,由马歇尔·N·罗森布鲁斯和阿丽安娜·W·罗森布鲁斯所开展的开创性工作。\(^\text{[37]}\)

在高级信号处理与贝叶斯推断中使用序贯蒙特卡罗方法是较近才出现的进展。1993年,Gordon 等人在其开创性论文中首次将蒙特卡罗重采样算法应用于贝叶斯统计推断。\(^\text{[38]}\)作者将其算法命名为“自助滤波器”,并指出与其他滤波方法相比,该算法不需要对状态空间或系统噪声作出任何假设。该领域的另一篇奠基性文章是北川源四郎关于相关的“蒙特卡罗滤波器”的研究。\(^\text{[39]}\)同时期还有皮埃尔·德尔·莫拉尔,\(^\text{[40]}\)以及 Himilcon Carvalho、Pierre Del Moral、André Monin 与 Gérard Salut 合作发表的关于粒子滤波器的研究论文,这些成果大多发表于1990年代中期。\(^\text{[41]}\)实际上,在1989年至1992年间,P. Del Moral、J. C. Noyer、G. Rigal 和 G. Salut 在法国国家科学研究中心系统分析与架构实验室已在雷达/声纳与GPS信号处理领域开发出粒子滤波算法,这些工作曾以保密研究报告的形式提交给法国海军武器工程技术处、IT公司 DIGILOG 及 LAAS-CNRS。\(^\text{[42][43][44][45][46][47]}\)这些序贯蒙特卡罗方法可被理解为:带有相互作用重采样机制的接受–拒绝采样器。

从1950年到1996年,所有关于序贯蒙特卡罗方法的出版物——包括在计算物理和分子化学中提出的剪枝与重采样蒙特卡罗方法——都提出了自然的、类启发式的算法,被应用于各种不同情境,但并未提供算法一致性的证明,也没有讨论估计偏差或基于谱系树与祖先树的算法结构。

该领域的数学基础及对这些粒子算法的首个严格分析由皮埃尔·德尔·莫拉尔于1996年完成。\(^\text{[40][48]}\)到了20世纪90年代末,具有可变粒子数量的分支型粒子方法也得到了发展,主要由丹·克里桑、杰西卡·盖恩斯和特里·里昂斯提出。\(^\text{[49][50][51]}\)之后丹·克里桑、皮埃尔·德尔·莫拉尔与特里·里昂斯共同推动了进一步的研究。\(^\text{[52]}\)1999年至2001年,P. Del Moral、A. Guionnet 和 L. Miclo继续对该领域进行了拓展与深化。\(^\text{[30][53][54]}\)
\subsection{定义}
关于“蒙特卡罗”应如何定义,学界尚无统一共识。例如,Ripley\(^\text{[55]}\)将大多数概率建模称为随机模拟,而将“蒙特卡罗”一词专用于蒙特卡罗积分和蒙特卡罗统计检验。Sawilowsky\(^\text{[56]}\)则区分了模拟、蒙特卡罗方法与蒙特卡罗模拟三者的含义:模拟是对现实的一种虚构性再现;蒙特卡罗方法是一种用于求解数学或统计问题的技术手段;蒙特卡罗模拟是通过重复抽样来获取某种现象(或行为)的统计特性。

Kalos 和 Whitlock\(^\text{[57]}\)指出,这种区分并不总是容易维持的。例如,原子发射辐射是一个自然的随机过程,既可以直接对其进行模拟,也可以通过随机微分方程来描述其平均行为,而这些方程本身又可以通过蒙特卡罗方法求解。“实际上,同一段计算机代码既可以被视为一种‘自然模拟’,也可以被视为通过自然采样求解方程的方法。”

可以通过Gelman–Rubin 统计量来检验蒙特卡罗模拟的收敛性。
\subsubsection{蒙特卡罗方法与随机数}
该方法的核心思想是:通过反复随机抽样与统计分析来计算结果。实际上,蒙特卡罗模拟本质上是一种随机实验,尤其适用于那些实验结果尚不明确的问题情境。蒙特卡罗模拟通常具有大量未知参数,其中许多参数难以通过实验手段获得。\(^\text{[58]}\)蒙特卡罗模拟方法并不总是需要真正的随机数才能有效(尽管对于某些应用,如素性测试,不可预测性是至关重要的)。\(^\text{[59]}\)许多实用的技术采用的是确定性的伪随机序列,这使得测试和重复运行模拟变得更加容易。通常,生成良好模拟所需的唯一标准是:伪随机序列在某种意义上看起来“足够随机”。

具体标准取决于实际应用,但通常来说,应满足一系列统计检验。最基本且最常见的检验之一是:当考虑序列中足够多的元素时,这些数应当呈均匀分布或符合其他期望分布。此外,相邻样本间的弱相关性通常也是所期望或必须具备的属性。

Sawilowsky 对高质量蒙特卡罗模拟的标准进行了系统归纳,列出了以下关键特征:\(^\text{[56]}\)
\begin{itemize}
\item 伪随机数生成器具备特定性质,例如具有较长的“周期”,即在数列重复之前可生成大量不同值;
\item 伪随机数生成器能够通过随机性检验,确保其统计性质符合均匀分布或其他预期分布;
\item 样本数量充足,以确保模拟结果具有足够的统计精度;
\item 使用了合适的抽样技术,以避免偏差或非代表性样本;
\item 算法本身在数学与逻辑层面上对所建模型具有有效性;
\item 模拟过程能够准确再现目标现象的本质特征。
\end{itemize}
在模拟过程中,常通过伪随机数抽样算法将均匀分布的伪随机数转换为服从给定概率分布的随机变量。该过程确保模拟能在所需的分布特性下进行,进而提升结果的代表性。

此外,在某些应用场景中,为克服随机抽样在覆盖空间时的不均匀性,常采用低偏差序列替代纯粹的随机或伪随机序列。这类方法通过实现更均匀的样本空间覆盖,通常能提供更高的收敛速度,从而提高模拟效率。基于此类序列的模拟被称为准蒙特卡罗方法。

为研究伪随机数质量对蒙特卡罗模拟结果的影响,天体物理领域的研究者曾将Intel RDRAND 指令集生成的加密级伪随机数与常用算法(如 Mersenne Twister)生成的伪随机数进行了对比。模拟场景为褐矮星无线电耀发的蒙特卡罗模拟,在生成 $10^7$ 个随机数的实验中,未观察到使用 RDRAND 与传统伪随机数生成器之间存在统计显著差异。\(^\text{[60]}\)
\subsubsection{蒙特卡罗模拟 vs. “假设情景”(What-if)分析}
有些概率使用方式并不属于蒙特卡罗模拟,例如:采用单点估计进行的确定性建模。在这类方法中,模型中的每个不确定变量都被赋予一个“最合理猜测”的估计值。然后选择每个输入变量的情境(如最佳情况、最差情况或最可能情况),并记录相应结果。\(^\text{[61]}\)

与之相反,蒙特卡罗模拟是从每个变量的概率分布中采样,生成数百甚至数千种可能的结果。随后对这些结果进行分析,从而得到不同结果发生的概率分布。\(^\text{[62]}\)例如,将一个电子表格的建筑成本模型,分别以传统的“假设情景”方式和使用蒙特卡罗模拟(采用三角概率分布)方式运行比较会发现:蒙特卡罗分析的结果范围通常更窄,相较于“假设情景”分析。 这是因为:“假设情景”方法对所有情境赋予相等权重(详见:企业财务中的不确定性量化),而蒙特卡罗方法在概率极低的区域几乎不会采样。这种概率极低的样本被称为“稀有事件”。
\subsection{应用领域}
蒙特卡罗方法在模拟输入存在显著不确定性、或系统具有多个耦合自由度的现象时尤其有用。其应用领域包括:
\subsubsection{物理科学}
蒙特卡罗方法在计算物理、物理化学及相关的应用学科中具有极其重要的地位,并广泛应用于多个方向,例如:从复杂的量子色动力学(QCD)计算,到热防护罩与气动外形的设计,以及用于辐射剂量学计算的辐射输运建模。\(^\text{[63][64][65]}\)在统计物理中,蒙特卡罗分子建模是计算分子动力学的一种替代方案;同时,蒙特卡罗方法也被用于计算简单粒子系统和聚合物系统的统计场论。\(^\text{[37][66]}\)量子蒙特卡罗方法可用于求解多体量子系统问题。\(^\text{[9][10][29]}\)在辐射材料科学中,用于模拟离子注入的二元碰撞近似(Binary Collision Approximation, BCA)通常基于蒙特卡罗方法来选择下一个碰撞原子。\(^\text{[67]}\)在实验粒子物理中,蒙特卡罗方法用于:探测器的设计,探测器行为的建模,以及实验数据与理论模型的比对。在天体物理学中,蒙特卡罗方法被广泛用于不同任务,例如:模拟星系演化,\(^\text{[68]}\)
建模微波辐射在粗糙行星表面上的传播。\(^\text{[69]}\)此外,蒙特卡罗方法也是现代天气预报中的集合模型的核心组成部分。
\subsubsection{工程领域}
蒙特卡罗方法在工程中被广泛应用于灵敏度分析和过程设计中的定量概率分析。这是因为实际过程模拟往往存在变量间的相互作用、共线性以及非线性行为。例如:
\begin{itemize}
\item 在微电子工程中,蒙特卡罗方法被用于分析模拟与数字集成电路中相关或不相关的工艺变化;
\item 在地质统计学与地质冶金学中,蒙特卡罗方法是矿物加工流程设计的基础,并用于定量风险分析;\(^\text{[21]}\)
\item 在流体力学(特别是稀薄气体动力学)中,蒙特卡罗方法用于解有限克努森数下的玻尔兹曼方程,这通过直接模拟蒙特卡罗方法结合高效算法来实现;\(^\text{[70][71]}\)
\item 在自主机器人学中,蒙特卡罗定位法可用于确定机器人位置。它常结合用于随机滤波的算法,如卡尔曼滤波器或粒子滤波器,后者是 SLAM(同步定位与建图)算法的核心组成;
\item 在电信工程中,规划无线网络时必须确保设计能适用于大量不同情境,这些情境主要取决于用户数量、位置和所使用的服务。蒙特卡罗方法通常用于随机生成这些用户及其状态,再对网络性能进行评估。如果结果不理想,则对网络设计进行优化;
\item 在可靠性工程中,蒙特卡罗模拟用于根据组件级响应计算系统级响应;
\item 在信号处理与贝叶斯推断中,粒子滤波器和序贯蒙特卡罗技术是一类平均场粒子方法,通过交互式经验测度从部分有噪观测数据中采样,用于计算信号过程的后验分布。\(^\text{[72]}\)
\end{itemize}
\subsubsection{气候变化与辐射强迫}
联合国政府间气候变化专门委员会(IPCC)在分析辐射强迫的概率密度函数时依赖于蒙特卡罗方法。\(^\text{[73]}\)
\subsubsection{计算生物学}
蒙特卡罗方法广泛应用于计算生物学的多个领域,例如:系统发育树推断中的贝叶斯推断,或研究基因组、蛋白质和细胞膜等生物系统。\(^\text{[74][75]}\)研究可以基于粗粒度模型或从头计算模型,具体取决于所需的精度。计算机模拟使研究者能够观察某个特定分子的局部环境变化,以判断是否发生某些化学反应。当物理实验不可行时,还可以进行思想实验,例如:断裂化学键、在特定位置引入杂质、改变局部或整体结构、施加外部场等。
\subsubsection{计算机图形学}
路径追踪,有时也被称为蒙特卡罗光线追踪,是一种通过随机采样光线路径来渲染三维场景的方法。对任意像素进行多次采样后,所有样本的平均值最终将收敛到渲染方程的正确解。这使路径追踪成为目前已知最具物理精度的三维图形渲染方法之一。
\subsubsection{应用统计学}
Sawilowsky\(^\text{[76]}\)为统计学中的蒙特卡罗实验确立了标准。在应用统计学中,蒙特卡罗方法至少可用于以下四个方面:
\begin{enumerate}
\item 比较在现实数据条件下,小样本下的统计量性能:虽然在经典理论分布(如正态分布、柯西分布)和渐近条件下(即样本无限大,处理效应无限小),可以计算统计量的第一类错误率和检验功效,但真实数据往往不服从这些理论分布。\(^\text{[77]}\)
\item 实现比精确检验(如排列检验)更高效的假设检验方法:精确检验往往计算量极大甚至无法实现,而蒙特卡罗方法提供了一种比渐近分布临界值更准确、计算又更可行的替代方案。
\item 在贝叶斯推断中,从后验分布中获取随机样本:通过这些样本可以近似并总结后验分布的所有关键特征。
\item 高效地估计负对数似然函数的 Hessian 矩阵:这些随机估计值可进行平均,形成对 Fisher 信息矩阵的估计。\(^\text{[78][79]}\)
\end{enumerate}
蒙特卡罗方法也是近似随机化检验与置换检验之间的一种折中方案。近似随机化检验是基于所有置换中某个特定子集进行的,这通常需要进行大量记录和管理,以确保哪些置换已被考虑过。而蒙特卡罗方法则是基于预定数量的随机抽取置换样本。这种方式通过允许某些置换可能被重复抽取,在牺牲微小精度的同时,换来了更高的计算效率,因为无需追踪已抽取的置换。
\subsubsection{游戏中的人工智能}
蒙特卡罗方法已被发展为一种称为蒙特卡罗树搜索的技术,用于在游戏中寻找最佳走法。可能的走法被组织成一棵搜索树,并通过大量随机模拟来估计每一步的长期潜力。对手的行为则由一个黑箱模拟器来表示。\(^\text{[80]}\)

蒙特卡罗树搜索包含以下四个步骤:\(^\text{[81]}\)
\begin{enumerate}
\item 从根节点出发,依次选择最优子节点,直到到达一个叶节点;
\item 扩展该叶节点,并选择其一个子节点;
\item 从该节点开始进行一次模拟游戏;
\item 根据模拟结果更新该节点及其所有祖先节点的信息。
\end{enumerate}
随着大量模拟游戏的进行,某个表示某一步棋的节点的价值会逐渐上升或下降,理想情况下,该价值变化应反映该走法是否优秀。

蒙特卡罗树搜索已成功应用于多种游戏中,例如:围棋(Go)\(^\text{[82]}\)Tantrix \(^\text{[83]}\)战舰\(^\text{[84]}\)Havannah \(^\text{[85]}\)Arimaa \(^\text{[86]}\)
\subsubsection{设计与视觉呈现}
蒙特卡罗方法在求解辐射场与能量传输耦合的积分微分方程方面也非常高效,因此广泛应用于全局光照计算,可生成逼真的虚拟三维图像。该技术在多个领域都有应用,例如:视频游戏建筑与工业设计计算机生成影像(CG)电影特效等\(^\text{[87]}\)
\subsubsection{搜救行动}
美国海岸警卫队在其搜救建模软件 SAROPS(Search And Rescue Optimal Planning System)中采用了蒙特卡罗方法,用于计算搜救行动中船只的可能位置。每次模拟可基于所提供的变量随机生成多达一万个数据点。\(^\text{[88]}\)系统随后会基于这些数据的外推结果生成搜索路径,以最大化:包围概率,发现概率,这两者相乘即为总体成功概率。最终,这一方法作为概率分布在现实中的实用应用,能够提供最快速、最有效的搜救策略,以节省生命与资源。\(^\text{[89]}\)

