% 自然语言处理(综述)
% license CCBYSA3
% type Wiki

本文根据 CC-BY-SA 协议转载翻译自维基百科\href{https://en.wikipedia.org/wiki/Natural_language_processing}{相关文章}。

自然语言处理(NLP)是计算机科学的一个子领域,特别是人工智能领域。它主要关注赋予计算机处理自然语言编码的数据的能力,因此与信息检索、知识表示和计算语言学(语言学的一个子领域)密切相关。通常,数据通过文本语料库收集,并使用基于规则、统计方法或基于神经网络的机器学习和深度学习方法进行处理。

自然语言处理的主要任务包括语音识别、文本分类、自然语言理解和自然语言生成。
\subsection{历史} 
更多信息:自然语言处理的历史  
自然语言处理的根源可以追溯到20世纪50年代。[1] 早在1950年,阿兰·图灵就发表了一篇名为《计算机器与智能》的文章,提出了现在被称为图灵测试的智能标准,尽管当时这并没有被表述为一个与人工智能分开的问题。该测试提议包括一个任务,涉及自动化地解释和生成自然语言。
\subsubsection{符号化自然语言处理(1950年代 – 1990年代初)}  
符号化自然语言处理的前提可以通过约翰·塞尔的“中文房间”实验来简明总结:给定一组规则(例如,一本中文短语手册,包含问题及其对应的答案),计算机通过应用这些规则来模拟自然语言理解(或其他自然语言处理任务),从而应对其所遇到的数据。
\begin{itemize}
\item 1950年代:1954年的乔治城实验涉及将超过60个俄语句子完全自动翻译成英语。研究者声称,机器翻译将在三到五年内解决。[2] 然而,真正的进展要慢得多,在1966年发布的ALPAC报告之后,报告指出十年的研究未能实现预期目标,机器翻译的资金大幅减少。美国几乎没有再进行机器翻译的进一步研究(尽管其他地方如日本和欧洲仍有一些研究[3]),直到1980年代末,首个统计机器翻译系统的出现。
\item 1960年代:1960年代出现了一些成功的自然语言处理系统,其中包括SHRDLU,这是一种在受限“积木世界”中工作的自然语言系统,具有受限的词汇表;以及ELIZA,这是由约瑟夫·魏岑鲍姆于1964到1966年间编写的罗杰式心理治疗师模拟系统。ELIZA几乎不涉及人类思维或情感的信息,但有时能提供惊人的人类互动。当“病人”的问题超出了非常小的知识库时,ELIZA可能会提供一个通用的回答,例如,当被问到“我的头很痛”时,它可能会回答“你为什么说你的头痛?”罗斯·奎利安(Ross Quillian)在自然语言方面的成功工作证明了仅用20个词汇就能开发出有效的系统,因为那时计算机的内存只能容纳这么多。[4]
\item 1970年代:1970年代,许多程序员开始编写“概念本体”,将现实世界的信息结构化为计算机可理解的数据。例子包括MARGIE(Schank, 1975)、SAM(Cullingford, 1978)、PAM(Wilensky, 1978)、TaleSpin(Meehan, 1976)、QUALM(Lehnert, 1977)、Politics(Carbonell, 1979)和Plot Units(Lehnert, 1981)。在这一时期,第一批聊天机器人被编写出来(例如,PARRY)。
\item 1980年代:1980年代和1990年代初是符号方法在自然语言处理中的黄金时代。当时的研究重点包括基于规则的解析(例如,HPSG作为生成语法的计算性实现)、形态学(例如,两级形态学[5])、语义学(例如,Lesk算法)、指代(例如,在中心理论[6]中)以及自然语言理解的其他领域(例如,在修辞结构理论中)。其他研究方向也得到了延续,例如使用Racter和Jabberwacky开发聊天机器人。一个重要的进展(最终导致了1990年代统计方法的转向)是这一时期量化评估的重要性逐渐上升。[7]
\end{itemize}