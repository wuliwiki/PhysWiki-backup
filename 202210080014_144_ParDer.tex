% 偏导数(简明微积分)
% keys 多元微积分|导数|偏导数|混合偏导

\pentry{导数\upref{Der}}
\subsection{一点处的偏导数}
类似于一元函数的导数,我们先定义一点处的偏导数.对一个多元函数 $y = f(x_1, x_2 \dots x_i \dots)$,如果求导时只把 $x_i$ 看成自变量,剩下的 $x_{j \ne i}$ 都看做常数,得到的导数就叫函数(关于 $x_i$)的\textbf{偏导数}. 因此,一个多元函数具有n个偏导数,即分别对各个自变量求偏导.

\begin{figure}[ht]
\centering
\includegraphics[width=14cm]{./figures/ParDer_1.png}
\caption{偏导数} \label{ParDer_fig1}
\end{figure}

下文以二元函数 $z=f(x,y)$ 为例,求$(x_0,y_0)$处的$f$关于$x$的偏导数.在三维坐标系中,二元函数$z=f(x,y)$一般可画成一个曲面.如\autoref{ParDer_fig1} 所示,我们先固定$y$不变,即令$y=y_0$(相当于做一个截面),此时二元函数$y=f(x,y)=f(x,y_0)$,事实上化为了一个熟悉的一元函数$y=f(x)$(即曲面与截面相交的曲线),随后运用一元函数导数\upref{Der}的定义,即可计算该点处的偏导数$y=f_x(x_0,y_0)$.

更一般地,我们定义


以二元函数 $z=f(x,y)$ 为例,对 $x$ 的偏导数常记为
\begin{equation}\label{ParDer_eq1}
\pdv{z}{x} \qquad \pdv{f}{x} \qquad f_x  \qquad \qty(\pdv{f}{x})_y
\end{equation}
最后一种记号在括号右下角声明了保持不变的自变量,这在许多情况下能避免混淆.

\begin{example}{}\label{ParDer_ex1}
对于函数 $f(x,y) = x^2 + 2 y^2 + 2xy$, 两个偏导数分别为
\begin{equation}
\pdv{f}{x} = 2x + 2y  \qquad  \pdv{f}{y} = 4y + 2x
\end{equation}
\end{example}

\begin{example}{}\label{ParDer_ex2}
对于函数 $z = \sin (y\cos x) + \cos ^2 x$
\begin{equation}
\begin{aligned}
\pdv{z}{x} &=  - y\cos (y\cos x)\sin x - 2\cos x\sin x =  - y\cos (y\cos x)\sin x - \sin 2x\\
\pdv{z}{y} &= \cos (y\cos x)\cos x
\end{aligned}
\end{equation}
\end{example}

\subsection{几何意义}

类比导数的几何意义(曲线的斜率), 若在三维直角坐标系中画出曲面 $f(x,y)$,则 $\pdv*{f}{x}$ 和 $\pdv*{f}{y}$ 分别是是某点处曲面延 $x$ 方向和 $y$ 方向的斜率.所以从某点 $(x_0, y_0)$ 延 $x$ 方向移动一个微小量 $\Delta x$,假设曲面平滑,则函数值增加
\begin{equation}
\Delta f \approx \pdv{f}{x}\Delta x
\end{equation}
写成微分关系就是
\begin{equation}
\dd{f} = \pdv{f}{x} \dd{x} \quad (y\, \text{不变})
\end{equation}

\subsection{高阶偏导}
与一元函数的高阶导数类似,多元函数也可以求高阶偏导数,不同的是,由于每求一次偏导都需要指定对哪个变量.例如二元函数 $f(x,y)$ 的二阶偏导有
\begin{equation}
\pdv[2]{f}{x} \qquad
\pdv{f}{x}{y} \qquad
\pdv{f}{y}{x} \qquad
\pdv[2]{f}{y}
\end{equation}
若高阶偏导的分母中出现不止一个变量,我们就称其为\textbf{混合偏导}.混合偏导的一个重要性质就是偏导的顺序可以任意改变,例如上式中有 $\pdv*{f}{x}{y} = \pdv*{f}{y}{x}$. 这点本书不做证明,可以通过以上的例子验证.
\addTODO{并不正确,需要 $f$ 满足很强的条件}








