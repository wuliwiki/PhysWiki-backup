% 马克·奥利芬特(综述)
% license CCBYSA3
% type Wiki

本文根据 CC-BY-SA 协议转载翻译自维基百科\href{https://en.wikipedia.org/wiki/Mark_Oliphant}{相关文章}。

马库斯·劳伦斯·埃尔温·奥利凡特爵士(Sir Marcus Laurence Elwin Oliphant,AC,KBE,FRS,FAA,FTSE,1901年10月8日-2000年7月14日)是澳大利亚物理学家和人道主义者,在核聚变的首次实验演示和核武器的开发中扮演了重要角色。

奥利凡特出生并成长于南澳大利亚的阿德莱德,1922年毕业于阿德莱德大学。1927年,他凭借在汞方面的研究获得了1851年展览奖学金,前往英国,在剑桥大学的卡文迪什实验室师从欧内斯特·拉塞福德爵士。在那里,他使用粒子加速器将重氢原子核(氘核)射向不同的靶标。他发现了氦-3(氦离子)和氚(氚核)的相应原子核。他还发现,当这些粒子相互反应时,释放出来的能量远远超过它们原本的能量。这些能量来自原子核内部的释放,他意识到这正是核聚变的结果。

奥利凡特于1937年离开卡文迪什实验室,成为伯明翰大学的波因廷物理学教授。他试图在大学建造一台60英寸(150厘米)的回旋加速器,但由于1939年欧洲爆发第二次世界大战,建设工作被推迟。他开始参与雷达的开发,领导伯明翰大学的一个小组,成员包括约翰·兰德尔和哈里·布特。他们创造了一种全新的设计——腔体磁控管,使微波雷达成为可能。奥利凡特还参与了MAUD委员会的工作,该委员会于1941年7月报告称,原子弹不仅是可行的,而且可能在1943年就能生产出来。奥利凡特在美国传播这一发现的消息,进而启动了后来成为曼哈顿计划的工作。在战争后期,他与他的朋友欧内斯特·劳伦斯一起,在加利福尼亚州伯克利的辐射实验室工作,开发了电磁同位素分离技术,这为“小男孩”原子弹提供了裂变材料,该原子弹在1945年8月被用在了广岛的原子弹爆炸中。

战争结束后,奥利凡特回到澳大利亚,成为新成立的澳大利亚国立大学(ANU)物理科学与工程研究学院的首任院长,在那里他发起了世界上最大的(500兆焦耳)同极发电机的设计和建设。他于1967年退休,但在唐·邓斯坦首相的建议下,被任命为南澳大利亚州州长。他成为南澳大利亚州首位出生于该州的州长。他协助创立了澳大利亚民主党,并且是1977年墨尔本会议的主席,在会议上该党正式成立。晚年,他目睹了妻子罗莎在1987年去世前的痛苦,并成为自愿安乐死的倡导者。他于2000年在堪培拉去世。
\subsection{早年生活}
马库斯“马克”·劳伦斯·埃尔温·奥利凡特于1901年10月8日出生在阿德莱德的肯特镇,一个位于阿德莱德郊区的地方。他的父亲是哈罗德·乔治“巴伦”·奥利凡特,一名南澳大利亚工程和供水部门的公务员,兼职经济学讲师,任职于工人教育协会。他的母亲是比阿特丽斯·伊迪丝·法尼·奥利凡特(原姓塔克),一位艺术家。他的名字取自澳大利亚作家马库斯·克拉克和英国旅行家与神秘主义者劳伦斯·奥利凡特。大多数人叫他马克;1959年他被授予骑士爵位时,这个名字成为了正式名称。

他有四个年幼的弟弟:罗兰、基思、奈杰尔和唐纳德;他们出生时都登记了奥利凡特这个姓氏。他的祖父哈里·史密斯·奥利凡特(1848年11月7日-1916年1月30日)是阿德莱德邮政总局的职员,他的曾祖父詹姆斯·史密斯·奥利凡特(约1818年-1890年1月21日)和他的妻子伊丽莎(约1821年-1881年10月18日)从肯特出发,乘船Ruby号前往南澳大利亚,并于1854年3月抵达。他后来被任命为阿德莱德贫困收容所的主管,而伊丽莎·奥利凡特则于1865年被任命为该收容所的院长。马克的父母是神智学家,因此他们可能避免食用肉类。马克从小便成为了终身素食主义者,原因是他在一个农场目睹了猪的屠宰。他在一只耳朵完全失聪,并且由于严重的散光和近视,他需要戴眼镜。

奥利凡特最初在家人于1910年迁往古德伍德和迈洛尔后,在当地的小学接受教育。他在阿德莱德的昂利中学就读,并在1918年完成了最后一年,转学至阿德莱德高中。毕业后,他未能获得奖学金进入大学,于是他在阿德莱德的一家制造珠宝的公司S. Schlank & Co.找了一份工作,该公司以制作奖章而闻名。随后,他获得了南澳大利亚州立图书馆的学员职位,这使他能够在晚上参加阿德莱德大学的课程。

1919年,奥利凡特开始在阿德莱德大学学习。最初,他对医学职业感兴趣,但在同年晚些时候,物理学教授凯尔·格兰特为他提供了物理系的学员职位。该职位每周支付10先令(相当于2022年的89澳元),与奥利凡特在州立图书馆工作的薪水相同,但它允许他选修任何不与物理系工作冲突的大学课程。他于1921年获得了理学学士(BSc)学位,并在1922年进行了荣誉学位的学习,导师是格兰特。1925年,格兰特休假时,物理系的临时负责人罗伊·伯顿与奥利凡特合作,发表了两篇关于汞的性质的论文:《汞的表面张力问题与水溶液对汞表面作用的影响》和《气体在汞表面的吸附》。奥利凡特后来回忆说,伯顿教会了他“即使是物理学领域中的小发现,也能带来一种非凡的振奋感”。

奥利凡特于1925年5月23日与来自阿德莱德的罗莎·路易丝·威尔布拉姆结婚。两人从青少年时期起便认识。他在实验室里用父亲给他的库尔加迪黄金矿的金块为罗莎制作了婚戒。
