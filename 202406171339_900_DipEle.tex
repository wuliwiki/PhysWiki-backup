% 跃迁偶极子矩阵的三种形式
% license Xiao
% type Tutor

\pentry{长度规范\nref{nod_LenGau}}{nod_2aed}

\footnote{本文参考 \cite{Bransden}}本文使用\enref{原子单位制}{AU}。 在\enref{长度规范}{LenGau}下, 对于某种势能 $V$ 的束缚态 $\ket{\psi_a}$ 和 $\ket{\psi_b}$, 可以证明以下三种形式的\textbf{跃迁偶极子矩阵(transition dipole matrix)}相等。
\begin{equation}\label{eq_DipEle_4}
D_{ba}^L = q\mel{\psi_b}{\bvec r}{\psi_a}~,
\end{equation}
\begin{equation}\label{eq_DipEle_3}
D_{ba}^V = -\frac{q}{m(E_b - E_a)}\mel{\psi_b}{\grad}{\psi_a}~,
\end{equation}
\begin{equation}
D_{ba}^A = \frac{q}{m(E_b-E_a)^2} \mel{\psi_b}{\grad V}{\psi_a}~,
\end{equation}
这三种形式分别称为偶极子矩阵的\textbf{长度形式(length form)}、\textbf{速度形式(velocity form)}和\textbf{加速度形式(acceleration form)}。 注意他们都使用长度规范, 不要和\enref{速度规范}{LVgaug}和加速度规范混淆。 % \addTODO{链接}

\subsection{证明速度形式}
\begin{equation}
H_0 = -\frac{\laplacian}{2m} + V(\bvec r)~,
\end{equation}
\begin{equation}\label{eq_DipEle_1}
\mel{\psi_b}{\bvec r}{\psi_a} = \frac{\mel{H_0\psi_b}{\bvec r}{\psi_a} - \mel{\psi_b}{\bvec r}{H_0\psi_a}}{E_b - E_a} = \frac{\mel{\psi_b}{H_0\bvec r - \bvec rH_0}{\psi_a}}{E_b - E_a}~.
\end{equation}
其中
\begin{equation}
H_0\bvec r - \bvec rH_0 = -\frac{1}{2m}(\laplacian \bvec r - \bvec r \laplacian)~,
\end{equation}
注意这里的算符 $\laplacian \bvec r$ 作用在波函数上是指 $\laplacian (\bvec r \psi)$ 而不是 $(\laplacian \bvec r) \psi$。 由\autoref{eq_VopEq_5}  不难证明
\begin{equation}
\laplacian (\bvec r \psi) = 2 \grad \psi + \bvec r \laplacian \psi~.
\end{equation}
所以
\begin{equation}
H_0\bvec r - \bvec rH_0 = -\frac{1}{m}\grad \psi~,
\end{equation}
代入\autoref{eq_DipEle_1} 得
\begin{equation}
\mel{\psi_b}{\bvec r}{\psi_a} = -\frac{\mel{\psi_b}{\grad}{\psi_a}}{m(E_b - E_a)}~,
\end{equation}
代入\autoref{eq_DipEle_4} 可得速度形式。 证毕。

\subsection{证明加速度形式}
从速度形式出发, 再次进行\autoref{eq_DipEle_1} 类似的操作
\begin{equation}\label{eq_DipEle_2}
\mel{\psi_b}{\grad}{\psi_a} = \frac{\mel{\psi_b}{H_0\grad - \grad H_0}{\psi_a}}{E_b - E_a}~.
\end{equation}
其中
\begin{equation}
H_0\grad - \grad H_0 = V\grad - \grad V = -(\grad V)~,
\end{equation}
代入\autoref{eq_DipEle_2} 再带入\autoref{eq_DipEle_3} 可得加速度形式, 证毕。
