% shenry
令$\overrightarrow a  = \left( {{x_1},{y_1}} \right),\overrightarrow b  = \left( {{x_2},{y_2}} \right),\left\langle {a,b} \right\rangle  = \theta $则有
\begin{equation}
\left| {\overrightarrow a } \right| = \sqrt {x_1^2 + y_1^2}
\end{equation}
$\overrightarrow a$与$\overrightarrow b$的数量积记作$\overrightarrow a  \cdot \overrightarrow b $则有
\begin{equation}
\overrightarrow a  \cdot \overrightarrow b  = {x_1}{y_1} + {x_2}{y_2} = \left| {\overrightarrow a } \right| \cdot \left| {\overrightarrow b } \right| \cdot \cos \theta 
\end{equation}
向量的数量积满足交换律、分配律和关于数乘的结合律,即:
\begin{equation}
\overrightarrow a  \cdot \overrightarrow b  = \overrightarrow b  \cdot \overrightarrow a ,\overrightarrow a \left( {\overrightarrow b  + \overrightarrow c } \right) = \overrightarrow a  \cdot \overrightarrow b  + \overrightarrow a  \cdot \overrightarrow c ,\left( {\lambda \overrightarrow a } \right) \cdot \overrightarrow b  = \lambda \left( {\overrightarrow a  \cdot \overrightarrow b } \right)
\end{equation}
要注意向量的数量积之间不满足结合律,即一般情况下:
\begin{equation}
\left( {\overrightarrow a  \cdot \overrightarrow b } \right)\overrightarrow c  \ne \overrightarrow a \left( {\overrightarrow b  \cdot \overrightarrow c } \right)
\end{equation}
一些推论:
\begin{equation}
\overrightarrow a  \cdot \overrightarrow a  = {\left| {\overrightarrow a } \right|^2}
\end{equation}
\begin{equation}
\overrightarrow a  \cdot \overrightarrow b  = 0 \Leftrightarrow \overrightarrow a  \bot \overrightarrow b 
\end{equation}
\begin{equation}
\left| {\overrightarrow a  \cdot \overrightarrow b } \right| \le \left| {\overrightarrow a } \right| \cdot \left| {\overrightarrow b } \right|
\end{equation}