% 南京理工大学 普通物理 B(845)模拟五套卷 第二套
% license Usr
% type Note

\textbf{声明}:“该内容来源于网络公开资料,不保证真实性,如有侵权请联系管理员”

\subsection{填空题(30 分,每空 2 分)}
1. 已知一质点的运动方程为 $r=\left[(5\sin2\pi t)i+(4\cos2\pi t)j\right]$(单位为米),则该质点在 $0.25s$ 末的位置是__________,从 $0.25s$ 末到 $1s$ 末的位移是___
______。

2. 一质量为 $m$的质点在指向圆心的平方反比力的作用下,作半径为 r 的圆
周运动,此质点的速度 $v=$__________。若取距圆心无穷远处为势能零点,
它的机械能 $E$=____________。

3. 一轻绳绕于 $r=0.2m$ 的飞轮边缘,以恒力 $F=98N$ 拉绳,如图所示,已知
飞轮的转动惯量 $J=0.5kg\cdot m2$,轴承无摩擦。则飞轮的角加速度为______
_____;绳子拉下 $5m$ 时,飞轮的角速度为___________,动能为_______
\begin{figure}[ht]
\centering
\includegraphics[width=6cm]{./figures/406c268cd9b48d00.png}
\caption{} \label{fig_NJUD2_1}
\end{figure}
4. 已知一平面简谐波沿 $x$ 轴负向传播,振动周期 $T=0.5s$,波长 $l=10m$,振
幅 $A=0.1m$。当 $t=0$ 时波源振动的位移恰好为正的最大值。若波源处为原
点,则沿波传播方向距离波源为 $l/2$ 处的振动方程为 $y=$____________;
当 $t=T/2$ 时,$x=l/4$ 处质点的振动速度为____________。

5. 一 簧振子作简谐振动 ,其振动曲线图所示 。 则它的周期$T=$________ , 其余弦函数描述时初相位为 _________ 。
