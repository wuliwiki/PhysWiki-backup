% 黑体辐射定律
% 黑体辐射|普朗克|电磁波

在非绝对零度的温度下,任何物体都能辐射出电磁波(热辐射),同时也能吸收外来电磁波。假想一种黑体,它能 $100\%$ 地吸收所有辐射在其上的电磁波。并且为了能够达到热平衡,黑体也不断地辐射出能量。黑体的能量密度函数(关于频率 $\nu$ 的函数)以及辐射规律只和温度 $T$ 有关,这样才能保证热平衡定律(即热力学第零定律\upref{TherEq})的满足。

为了研究黑体辐射,我们希望能测定温度 $T$ 下黑体辐射的\textbf{能量密度}\footnote{单位体积单位频率间隔内辐射场的能量。} $S_\nu(\nu,T)$ 与辐射频率 $\nu$ 和温度 $T$ 之间的关系。这样我们就可以进一步由能量密度导出黑体单位面积上的辐射功率。

\subsection{维恩定律与瑞利金斯公式}
Wien(1894)从经典统计出发总结黑体辐射经验规律,得到了黑体辐射能量密度的公式:
\begin{equation}
S_\nu(\nu,T)\dd \nu=C_1 \frac{\nu^3}{c^3}e^{-C_2\nu/T}\dd \nu~.
\end{equation}
其中 $c$ 是真空中的光速,$C_1,C_2$ 是经验常数。该公式只在高频区适用。

瑞利(1900)和金斯(1905)则将空腔中的辐射场视为电磁驻波振子的集合,利用 Maxwell-Boltzmann 分布律与能量连续分布的观念导出
\begin{equation}
S_\nu(\nu,T)\dd \nu=\frac{8\pi\nu^2}{c^3}kT\dd \nu~.
\end{equation}


\subsection{黑体辐射定律}
在上述两个经验规律的基础上,普朗克(1900)提出了黑体辐射定律:
\begin{equation}
S_\nu(\nu,T) = \frac{8\pi h}{c^3}\frac{\nu^3}{\E^{h\nu/(k_B T)} - 1}~.
\end{equation}
如果要计算波长的分布, 根据随机变量的变换\upref{RandCV}, 由 $\abs{S_\lambda(\lambda) \dd{\lambda}} = \abs{S_\nu(\nu)\dd{\nu}}$ 得
\begin{equation}\label{eq_BBdLaw_1}
S_\lambda(\lambda,T) = \frac{c}{\lambda^2}S_\nu\qty(\frac{c}{\lambda}) =
\frac{8\pi ch}{\lambda^5} \frac{1}{\E^{hc/(k_B T\lambda)} - 1}~.
\end{equation}


由能量密度公式可以推出黑体在单位面积上的辐射功率。由于球面立体角\upref{SolAng}为 $4\pi$,单位面积单位频率单位立体角的功率为
\begin{equation}\label{eq_BBdLaw_2}
B(\nu) = \frac{c}{4\pi}S_\nu(\nu) = \frac{2h}{c^2} \frac{\nu^3}{\E^{h\nu/(k_B T)} - 1}~.
\end{equation}
在黑体内部, 辐射是各向同性的, 但在黑体表面, 对于给定的一个平面微元, $B(\nu)$ 是垂直于平面的值, 与法向量夹角为 $\theta$ 的方向的辐射功率为 $B(\nu)\cos\theta$。对 $\theta$ 从 $0$ 到 $\pi/2$ 积分得到单位面积上的辐射功率
\begin{equation}
\begin{aligned}
J(\nu)&=\int_0^{2\pi} \dd\phi \int_0^{\pi/2} \sin\theta \dd \theta \cdot \left[B(\nu)\cos\theta\right] ~,\\
&=\frac{c}{4\pi}\cdot 2\pi S_\nu(\nu) \cdot \int_0^{\pi/2} \sin\theta\cos\theta \dd\theta ~,\\
&= \frac{c}{4} S_\nu(\nu) ~.
\end{aligned}
\end{equation}
这就是著名的 Stefan-Boltzmann 定律,它指出了单位面积上辐射功率与辐射场能量密度之间的关系。

\subsection{推导}
\pentry{盒中的电磁波\upref{EBBox}}
普朗克首先提出了能量量子化假设,黑体空腔中的辐射场为电磁驻波振子的集合,并且振动能量只能取离散值\footnote{事实上对每一个振子都会有一个基态真空能 $\frac{1}{2}h\nu$ 的贡献,在这里我们忽略这一常数,即从能量密度中减去这一常数,它对我们所关心的辐射公式是没有影响的。}:
\begin{equation}
\epsilon = nh\nu, \quad n=0,1,\cdots~
\end{equation}
并且,腔中的辐射场与温度为 $T$ 的腔壁交换的能量也是一份一份的量子化的。

根据平衡态统计理论,能量为 $\epsilon=nh\nu$ 振子数目的相对值是 $e^{-nh\nu/kT}$,于是频率为 $\nu$ 的振子的平均能量为\footnote{或者我们也可以从玻色爱因斯坦分布\upref{MBsta}进行推导}(令 $\beta = 1/(k_BT)$)
\begin{equation}\label{eq_BBdLaw_4}
\overline{\epsilon_\nu}=\frac{\sum_{n=0}^\infty nh\nu e^{-nh\nu/kT}}{\sum_{n=0}^\infty e^{-nh\nu/kT}}=-\pdv{\beta}\ln \sum_{n=0}^\infty e^{-nh\nu\beta}=\frac{h\nu}{e^{h\nu\beta}-1}~.
\end{equation}


最后,我们要求出单位频率 $\dd \nu$ 内,振子的数量,即态密度。电磁波作为横场,独立自由度数为 $2$,因此单位体积的态密度为
\begin{equation}\label{eq_BBdLaw_3}
\rho(\nu) \dd \nu= \frac{1}{V}\cdot 2\cdot \frac{\dd{} ^3 k}{(2\pi/L)^3}=\frac{8\pi \omega^2 \dd \omega}{(2\pi c)^3}= \frac{8\pi}{c^3}\nu^2\dd \nu~.
\end{equation}
结合\autoref{eq_BBdLaw_4} 和 \autoref{eq_BBdLaw_3} ,我们可以得到黑体辐射公式:
\begin{equation}
S_\nu(\nu,T) = \frac{8\pi h}{c^3}\frac{\nu^3}{\E^{h\nu/(k_B T)} - 1}~.
\end{equation}
