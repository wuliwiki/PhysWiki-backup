% 二次量子化
% 二次量子化|多体系统|玻色统计|费米统计

\pentry{全同粒子\upref{IdPar},全同粒子的统计\upref{IdParS}}

薛定谔方程是关于单粒子的量子力学,而如果我们考虑一个多粒子体系,研究粒子间相互作用对多体系统的影响,则有必要建立一个关于多粒子的量子力学.一个直接的想法是,讲单个时空坐标变量的波函数拓展为 $N$ 个变量的波函数 $\psi(x_1,\cdots,x_N)$,波函数的模方具有概率的意义,因此可以乘上一个系数使它归一化.此外,如果某多体系统中两个电子可以被看作是两个独立的波函数 $\psi_1(x_1)$ 和 $\psi_2(x_2)$,那么这意味着 $\psi(x_1,x_2)$ 可以写成 $\psi_1(x_1)\psi_2(x_2)$ 也就是这两个单粒子波函数的乘积,此时这个量子态可以用 $\ket{\psi_1}\ket{\psi_2}$ 表示,它的坐标表象就是 $(\bra{x_1}\bra{x_2})(\ket{\psi_1}\ket{\psi_2})=\braket{x_1}{\psi_1} \braket{x_2}{\psi_2}=\psi_1(x_1)\psi_2(x_2)=\psi(x_1,x_2)$.

然而当我们讨论全同粒子的时候,例如 $N$ 个电子组成的体系,上述波函数的定义并没有体现粒子的全同性,波函数不具有交换对称性或交换反对称性\footnote{玻色统计满足交换对称性 $\psi(x_1,x_2,\cdots)=\psi(x_2,x_1,\cdots)$,而费米统计满足交换反对称性,$\psi(x_1,x_2,\cdots)=-\psi(x_2,x_1,\cdots)$.}.因此有必要将波函数对称化.我们令 $\xi=\pm 1$,$\xi=1$ 对应玻色统计,而 $\xi=-1$ 对应费米统计.那么可以定义新的对称化的波函数为
\begin{equation}
\psi'(x_1,\cdots,x_N)\propto \sum_{P\in S_N} \xi^P \psi(x_{P_1},x_{P_2},\cdots,x_{P_N})
\end{equation}
其中 $\xi^P$ 中的指数 $P$ 看作是排列 $P\in S_N$ 的逆序数,每交换排列 $P$ 的某两个 $P_i,P_j$,逆序数都会改变 $\pm 1$.

$N$ 体系统的波函数所在的 Hilbert 空间记为 $\mathcal{H}_N$.完整的 Hilbert 空间被称为 Fock 空间,记为 $\mathcal{F}$,那么我们可以将 $\mathcal{F}$ 表示为一切 $\mathcal{H}_N$ 的直和,即
\begin{equation}
\mathcal{F}=\mathcal{H}_1 \oplus \mathcal{H}_2 \oplus \cdots
\end{equation}
为了更清楚地了解 $\mathcal{F}$ 的结构,我们需要从几个方面进行研究:首先是研究 $\mathcal{H}_N$ 的一组正交完备基底,有了基底以后我们可以用这些基函数的线性组合来表示任意 $N$ 体系统,经过合适的基底构造还可以将单粒子态与多粒子态联系起来;然后我们需要研究产生算符和湮灭算符,它们将不同粒子数的 Hilbert 空间之间关联了起来.最后,我们需要研究多粒子系统的算符,这里面包括能量算符、动量算符、相互作用势能算符等等,这是二次量子化的一个非常重要的内容.
\subsection{Fock 空间的基底}
最简单的一组基底为坐标表象下的基底,我们将它记为 $\ket{x_1\cdots x_N}$(类似于单粒子 Hilbert 空间的 $\ket{x}$ 基底),满足正交完备性.对于多体系统的这组基底,我们约定每交换两个坐标 $x_i,x_j$,态矢 $\ket{x_1\cdots x_N}$ 将改变一个 $\xi$ 的因子,即满足交换对称性
\begin{equation}
\begin{aligned}
\ket{\cdots x_i \cdots x_j \cdots}& =
\xi \ket{\cdots x_j \cdots x_i \cdots} \\
&=\frac{1}{\sqrt{N!}}\sum_P \xi^P\ket{x_{P_1}}\ket{x_{P_2}}\cdots\ket{x_{P_N}}
\end{aligned}
\end{equation}

更一般地,我们定义这样的多粒子态,其中第 $i$ 个粒子的波函数可以由 $\psi_i(x)$ 描述.
\begin{equation}
\begin{aligned}
\ket{\psi_1\cdots\psi_N}&=\frac{1}{\sqrt{N!}}\sum_P \xi^P\ket{\psi_{P_1}}\cdots \ket{\psi_{P_N}}\\
&=\frac{1}{\sqrt{N!}}\sum_P \xi^P\psi_{P_1}(x_1)\cdots \psi_{P_N}(x_N)
\end{aligned}
\end{equation}
则经过计算,它的坐标表象为
\begin{equation}\label{SecQua_eq1}
\braket{x_1\cdots x_N}{\psi_1\cdots\psi_N}=\frac{1}{N!}\sum_P\sum_Q \xi^P \xi^Q \braket{x_{P_1}}{\psi_{Q_1}}\cdots\braket{x_{P_N}}{\psi_{Q_N}}
\end{equation}
我们可以将右侧的 $\braket{x_{P_i}}{\psi_{Q_i}}$ 重排,使得 $Q_i$ 从左往右依次是 $1,2,\cdots,N$,那么 $P_i$ 从左往右依次是 $P'_1,P'_2,\cdots,P'_N$.这意味着 $P'_{Q_i}=P_i$,或者说 $P'=P\circ Q^{-1}$.根据逆序数的性质,$\xi^P\xi^Q=\xi^{P\circ Q^{-1}}=\xi^{P'}$,\autoref{SecQua_eq1}  最终可以化简为
\begin{equation}
\begin{aligned}
\braket{x_1\cdots x_N}{\psi_1\cdots\psi_N}&=\frac{1}{N!}\sum_{P'}\sum_Q \xi^{P'} \braket{x_{P'_1}}{\psi_{1}}\cdots\braket{x_{P'_N}}{\psi_{N}}\\
&=\sum_{P}\xi^P\braket{x_{P_1}}{\psi_{1}}\cdots\braket{x_{P_N}}{\psi_{N}}\\
&=\left|\begin{matrix}
\braket{x_1}{\psi_1} &\braket{x_2}{\psi_1} &\cdots &\braket{x_N}{\psi_1}\\
\braket{x_1}{\psi_2} &\braket{x_2}{\psi_2} &\cdots & \braket{x_N}{\psi_2}\\
\vdots & \vdots & & \vdots \\
\braket{x_1}{\psi_N} &\braket{x_2}{\psi_N} &\cdots & \braket{x_N}{\psi_N}
\end{matrix}\right|_\xi
\end{aligned}
\end{equation}
上述行列式右下角的 $\xi$ 表示每一个乘积项都带有一个 $\xi^P$ 的符号.利用上述公式,我们可以构造一组正交完备基.

例如,我们取单粒子 Hilbert 空间的一组基 $\ket{1},\ket{2}\cdots$,满足正交关系 $\braket{i}{j}=\delta_{ij}$,和完备关系 $\sum_i \ket{i}\bra{i}=1$(这里的 $\sum$ 只是一个形式上的记号,如果指标 $i$ 是连续的,则应当视作积分.)那么,对于多粒子态 $\ket{\underbrace{1 1 \cdots 1}_{n_1 \text{个} 1} \ \underbrace{2 2 \cdots 2}_{n_2 \text{个} 2}\cdots }$,可以证明
\begin{equation}
\braket{\underbrace{1 1 \cdots 1}_{n_1 \text{个} 1} \ \underbrace{2 2 \cdots 2}_{n_2 \text{个} 2}\cdots }{\underbrace{1 1 \cdots 1}_{n_1 \text{个} 1} \ \underbrace{2 2 \cdots 2}_{n_2 \text{个} 2}\cdots }=\begin{cases}
n_1!n_2!\cdots\ &\xi=1\\
(n_1\le 1)   &\xi=-1
\end{cases}
\end{equation}
