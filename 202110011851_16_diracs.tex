% 狄拉克方程的自由粒子解
% 狄拉克方程|解|自由粒子

现在我们来讨论狄拉克方程的平面波解.解的形式如下
\begin{equation}
\psi(x) = u(p)e^{-ip\cdot x}~, \quad p^2 = m^2~.
\end{equation}
我们现在主要考虑正频率的解,也就是$p^0>0$的解.我们把上式代入狄拉克方程中,得
\begin{equation}\label{diracs_eq1}
(\gamma^\mu p_\mu - m) u(p) = 0~.
\end{equation}
我们可以先在静止系中分析这个方程,其中$p=p_0=(m,\boldsymbol 0 )$. 任意动量$p$的解可以通过boost $\Lambda_{\frac{1}{2}}$来得到.在静止系中,\autoref{diracs_eq1} 式变为
\begin{equation}
(m\gamma^0-m)u(p_0) = m\begin{pmatrix}
-1 & -1 \\
 1 & -1
\end{pmatrix}u(p_0) = 0 ~.
\end{equation}
解为
\begin{equation}
u(p_0) = \sqrt{m} \begin{pmatrix}
\xi \\ \xi
\end{pmatrix}~.
\end{equation}
其中$\xi$是任意的两分量的旋量.我们取归一化条件$\xi^\dagger \xi =1$. $\xi$在旋转生成元的作用下按照普通的两分量旋量进行变换.$\xi = \begin{pmatrix}
1 \\0
\end{pmatrix}$时,粒子在3-方向具有朝上的自旋.

使用狄拉克方程过后,我们可以选择$u(p)$中四个分量的其中两个分量.因为自旋为$1/2$的粒子只有两个物理态-自旋向上和向下.

现在我们来对静止系的$u(p)$来进行一个boost.我们考虑沿着3-轴方向的boost.首先我们要知道boost对4-动量的矢量有何作用.考虑无穷小的boost
\begin{equation}
\begin{pmatrix}
E \\ p^3
\end{pmatrix} = \bigg[ 1 + \eta \begin{pmatrix}
0 & 1 \\
1 & 0 
\end{pmatrix} \bigg] \begin{pmatrix}
m \\ 0
\end{pmatrix}~.
\end{equation}
其中$\eta$是无穷小的参数.对于有限的$\eta$,我们可以写
\begin{align}\nonumber
\begin{pmatrix}
E \\ p^3
\end{pmatrix} & = \exp \bigg[ \eta \begin{pmatrix}
0 & 1 \\
1& 0
\end{pmatrix} \bigg]\begin{pmatrix}
m \\ 0
\end{pmatrix} \\
& = \bigg[ \cosh \eta \begin{pmatrix}
1 & 0 \\ 0 & 1
\end{pmatrix} + \sinh \eta \begin{pmatrix}
0 & 1 \\
1 & 0
\end{pmatrix}\bigg]\begin{pmatrix}
m \\ 0
\end{pmatrix} = \begin{pmatrix}
m \cosh \eta \\ m \sinh \eta
\end{pmatrix}~.
\end{align}
$\eta$参数被称为\textbf{快度}.\textbf{在连续的boost变换下,快度是相加的}.
\begin{exercise}{对$u(p_0)$进行boost变换得到$u(p)$}
\begin{align}\nonumber
u(p) & = \exp \bigg[ - \frac{1}{2} \eta \begin{pmatrix}
\sigma^3 & 0 \\ 
0 & - \sigma^3
\end{pmatrix} \bigg] \sqrt{m} \begin{pmatrix}
\xi \\ \xi
\end{pmatrix} \\ \nonumber
& = \bigg[ \cosh\big(\frac{1}{2}\eta\big)\begin{pmatrix}
1 & 0 \\
0 & 1
\end{pmatrix} - \sinh (\frac{1}{2}\eta) \begin{pmatrix}
\sigma^3 & 0 \\
0 & - \sigma^3
\end{pmatrix} \bigg]\sqrt{m} \begin{pmatrix}
\xi \\ \xi
\end{pmatrix}
\end{align}
\end{exercise}

