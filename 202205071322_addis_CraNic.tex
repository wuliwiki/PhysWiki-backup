% Crank-Nicolson 算法解一维含时薛定谔方程(Matlab)
% 算法|薛定谔方程|数值解|偏微分方程

\pentry{薛定谔方程(单粒子一维)\upref{TDSE11}}

\footnote{参考 \cite{NR3}.}本文使用原子单位制\upref{AU}. 薛定谔方程为
\begin{equation}
-\frac12 \pdv[2]{\psi}{x} + V\psi = \I \pdv{\psi}{t}
\end{equation}
传播子作用于波函数为
\begin{equation}
\psi(x,t+\Delta t) = \exp(-\I H \Delta t)\psi(x,t)
\end{equation}
用 Crank-Nicolson 或 Caley scheme\footnote{二者是一回事, 见 \cite{NR3} 19.2 节.} 得到的结果是
\begin{equation}\label{CraNic_eq2}
\qty(1+\frac{\I}{2}\mat H_{n+1}\Delta t)\bvec\psi_{n+1} = \qty(1-\frac{\I}{2}\mat H_n\Delta t)\bvec\psi_n
\end{equation}
其中 $\bvec\psi_n$ 是时刻 $t_n$ 的波函数列矢量(已知), $\bvec\psi_{n+1}$ 为时刻 $t_{n+1}$ 的波函数列矢量(未知), $\mat H_n$ 是 $t_n$ 时刻的哈密顿矩阵.

但事实上, 还可以继续减少计算量. 若近似认为\footnote{或者更公平地, 把\autoref{CraNic_eq5} 中 $\mat H_n$ 改为 $\mat H_{n+1/2}$, 即 $(t_n+t_{n+1})/2$ 时的哈密顿矩阵.} $\mat H_{n+1}\approx \mat H_{n}$, 将\autoref{CraNic_eq2} 整理后得
\begin{equation}\label{CraNic_eq5}
\qty(\frac12 + \frac{\I}{4}\mat H_n\Delta t)\qty(\bvec\psi_{n+1}+\bvec\psi_n) = \bvec\psi_n
\end{equation}
解这个方程, 再减去 $\bvec \psi_n$ 即可.

\subsection{等间距网格}
对于等间距坐标网格 $x_1,x_2,\dots$, 可以用差分法(\autoref{DerDif_eq5}~\upref{DerDif})计算二阶导数, 表示为矩阵有
\begin{equation}
\mat D_2 = \frac{1}{\Delta x^2}\pmat{-2 & 1 & 0 & 0 & \dots\\
1 & -2 & 1 & 0 & \dots\\
0 & 1 & -2 & 1 & \dots\\
0 & 0 & 1 & -2 & \dots\\
\vdots & \vdots & \vdots & \vdots & \ddots}
\end{equation}
那么 $\mat H_n = -\mat D_2/(2m) + \mat V_n$, 其中 $\mat V_n$ 是对角矩阵, 第 $i$ 个对角元为 $V(x_i, t_n)$. 现在, 就可以代入\autoref{CraNic_eq2} 或\autoref{CraNic_eq5} 求解了.

\subsection{虚时间法}
\pentry{虚时间法求基态波函数\upref{ImgT}}
虚时间法用于求解势阱中的基态, 使用虚时间后, \autoref{CraNic_eq2} 和\autoref{CraNic_eq5} 分别变为
\begin{equation}
\qty(1+\frac12\mat H\Delta t)\bvec\psi_{n+1} = \qty(1-\frac12\mat H\Delta t)\bvec\psi_n
\end{equation}
\begin{equation}
\qty(\frac12 + \frac14\mat H\Delta t)\qty(\bvec\psi_{n+1}+\bvec\psi_n) = \bvec\psi_n
\end{equation}

\subsection{代码}
\pentry{Matlab 的稀疏矩阵\upref{MatSpa}}
Matlab 代码如下, 使用\autoref{CraNic_eq5}, 以及 Matlab 的稀疏矩阵\upref{MatSpa}. 势能函数可以在 \verb|V_fun| 中设置, 我们以方势垒为例, 所有参数和 “高斯波包的方势垒散射数值计算(Matlab)\upref{FSBsct}” 相同. 不同的是, 由于我们使用迪利克雷边界条件, 波函数到达边界后会发生全反射. 要避免反射, 可以用开放边界条件(未完成)以及吸收势能(未完成).

\begin{figure}[ht]
\centering
\includegraphics[width=14cm]{./figures/CraNic_1.png}
\caption{运行结果} \label{CraNic_fig1}
\end{figure}

\begin{lstlisting}[language=matlab, caption=TDSE\_cn1d.m]
% 数值解拉格朗日方程
% 拉格朗日量 L(qqt), qqt = [q, dq, t]
% 一阶偏微分 dL(i,qqt), 若不支持 i > N, 返回 nan
% 二阶偏微分 d2L(i,j,qqt) 可选, i = N+1:2*N, j = 1:2*N+1
% 广义力 Q(i, qqt) 可选
function [t, qq] = lagr_solve(L, qq0, tmin, tmax, Nt, RelTol, dL, d2L, Q)
h = 1e-6; % 差分长度
N = numel(qq0)/2;
if ~exist('d2L', 'var') || isempty(d2L)
    if isnan(dL(N+1,[qq0, tmin]))
        d2L = @(i,j,qqt)D2_ij(L,i,j,qqt,h);
    else
        d2L = @(i,j,qqt)d2L_dif1(dL,i,j,qqt,h);
    end
end
if ~exist('Q', 'var') || isempty(Q)
    Q = @(i,qqt) 0;
end
options = odeset('RelTol', RelTol);
[T, QQ] = ode45(@(t,qq)ode_fun(t, qq, d2L, Q), ...
        [tmin, tmax], qq0, options);
t = linspace(tmin, tmax, Nt);
qq = zeros(Nt, 2*N);
for i = 1:2*N
    qq(:,i) = interp1(T, QQ(:,i), t, 'spline');
end
end

function dqq = ode_fun(t, qq, d2L, Q)
    N = numel(qq)/2;
    dqq = zeros(N, 1);
    dqq(1:N) = qq(N+1:end);
    qqt = [qq, t];
    A = zeros(N, N);
    for i = N+1:2*N
        for j = i:2*N
            A(i,j) = d2L(i, j, qqt);
            if i ~= j
                A(j,i) = A(i,j);
            end
        end
    end
    b = zeros(N, 1);
    for i = 1:N
        b(i) = dL(i, qqt) - d2L(i+N,2*N+1) + Q(i,qqt);
        for j = 1:N
            b(i) = b(i) - d2L(N+i,j)*qqt(j+N);
        end
    end
    dqq(N:end) = A \ b;
end

% d^2 L / d(qqt_i)d(qqt_j)
% dL is dL/d(qqt_i)
function ret = d2L_dif1(dL, i, j, qqt, h)
qqt(j) = qqt(j) + h;
L2 = dL(i, qqt, h);
qqt(j) = qqt(j) - 2*h;
L1 = dL(i, qqt, h);
ret = (L2 - L1) / (2*h);
end
\end{lstlisting}
