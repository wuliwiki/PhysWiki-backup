% 相对论效应造成的近日进动
% keys 相对论|进动
% license Usr
% type Wiki

\begin{issues}
\issueDraft
\end{issues}

\pentry{开普勒问题\upref{CelBd},狭义相对论,分析力学}



对于开普勒问题,还需要考虑由于太阳的强引力对轨道产生的影响。这里仅讨论狭义相对论,不考虑广义相对论造成的修正。

\subsection{观测值与其他星体造成的影响理论值不符}
对于水星近日点的总进动值(约每世纪 $570''$),其他行星对水星的影响约仅有观测值的 $93 \%$,特别是木星、金星与地球(占约 $91\%$),人们发现有微小偏差,广义相对论的修正约是每世纪进动 $43''$,与其他星体造成的影响合并后恰好符合观测值。

特别的,广义相对论的修正包含了狭义相对论的修正,狭义相对论的修正结果只有广义相对论的约 $1/6$。

\subsection{狭义相对论修正的近日点进动}
这个问题一般被称为\textbf{相对论性开普勒问题},在经典力学中可以严格求解。第一个解决该问题的是索墨菲(Sommerfeld)。

\subsubsection{分析力学求近似解}
为简化计算,采用速度满足光速 $c=1$ 的单位制。

粒子的拉氏量在

\subsection{哈密顿力学求精确解}
