% 线性泛函的保范延拓
% keys 保范|延拓|线性泛函
% license Usr
% type Tutor

\pentry{线性泛函的延拓\nref{nod_ExLina},赋范空间\nref{nod_NormV}}{nod_b522}

由一般线性泛函延拓的\enref{Hahn-Banach定理}{ExLina},可以得到赋范空间中线性泛函延拓定理的表述,进而很容易得到,赋范空间中,定义在子空间上的线性泛函可以保范的延拓到整个空间上去。详情见下文。

\subsection{保范延拓}

\begin{theorem}{}
设 $L$ 是实赋范空间 $E$ 的子空间,$f_0$ 是 $L$ 上的有界线性泛函,则存在 $f_0$ 在 $E$ 上的保范\enref{延拓}{ExLina} $f$,即 $\norm{f}_{E}=\norm{f_0}_{L},f(x)=f_0(x),x\in L$。
\end{theorem}

\textbf{证明:}令 $k=\norm{f_0}_{L}$,则 $k\norm{x}$ 是\enref{齐次凸泛函}{ConFul}。由于 $\abs{f_0(x)}\leq k\norm{x}$,根据 \enref{Hahn-Banach延拓定理}{ExLina},$$


\textbf{证毕!}
