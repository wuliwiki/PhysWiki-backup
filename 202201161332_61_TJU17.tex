% 天津大学 2017 年考研量子力学
% 考研|天津大学|量子力学|2017

\subsection{30分}
\begin{enumerate}
\item 氢原子处于基态 $\varPsi(\gamma,\theta,\varphi)=\frac{1}{\sqrt{\pi a^{3}_{0}}}e^{-\frac{r}{a_0}}$ (10分)\\
(1)求势能 $-\frac{e^2}{\gamma}$ 的平均值.\\
(2)求最概然半径.\\
\item 为何康普顿散射验证了光的粒子性.(10分)\\
\item 三维空间转子的哈密顿量 $H=\frac{\hat{L}^2}{2I}$,其能级和相应的简并度是多少?平面转子的能级和简并度是多少?(10分)\\
\end{enumerate}
\subsection{30分}
证明:\\
\begin{enumerate}
\item $\vec{r}\times \vec{L}+ \vec{L}\times \vec{r}$
\item $\vec{p}\times\vec{L}+\vec{L}\times\vec{p}$
\end{enumerate}
\subsection{30分}
质量为M的粒子在如下势场中运动:当 $0<x<\frac{a}{2}$ 时,$V_{(x)}=0$;当 $\frac{a}{2} <x<a$ 时,$V_{(x)}=x$;当 $x \le 0$ 或 $x \ge a$ 时,$V_{(x)}=\infty$,用微扰论求波函数到一级修正,能级至二级修正.
\subsection{30分}
质量都是 $m$ 的两质点被一弹性系数为 $k$,原长为 $l_{0}$ 的轻质弹簧连在一起,整个系统可在一维空间自由运动,求这一系统的能级和波函数.
\subsection{30分}
两粒子自旋都是 $\frac{1}{2}$,自旋算符分别为 $\vec{S_{1}}$,$\vec{S_{2}}$,两粒子组成一个系统,哈密顿量 $H=aS_{1x}S_{2x}+aS_{1y}S_{2y}+bS_{1z}S_{2z}$.\\
(1)求体系的能级和波函数.\\
(2)若开始时第一个粒子自旋向上,第二个粒子自旋向下,求体系的状态.\\