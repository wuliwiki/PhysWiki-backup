% 条件概率与事件的独立性(高中)
% 高中|条件概率|相互独立事件

\begin{issues}
\issueDraft
\end{issues}

\subsection{条件概率}
对于任何两个事件 $A$ 和 $B$,在已知事件 $A$发生的条件下,事件 $B$ 发生的概率叫做\textbf{条件概率},用符号 $P(B|A)$ 来表示.

我们把事件 $A$ 和 $B$ 同时发生所构成的事件 $D$,称为事件 $A$ 与 $B$ 的 \textbf{交} (或\textbf{积}),记作 $D = A \cap B$ (或 $D = AB$).

一般地,我们有条件概率公式
\begin{equation}
P(B|A) = \frac{P(A \cap B)}{P(A)},P(A)>0
\end{equation}

\subsection{事件的独立性}
事件 $A$ 是否发生对事件 $B$ 发生的概率没有影响,即
\begin{equation}
P(B|A) = P(B)
\end{equation}
这时,我们称两个事件 $A,B$ \textbf{相互独立},并把这两个事件叫做\textbf{相互独立事件(mutually independent events)}.

在实际问题中,常常通过事件本质进行分析就可知道它们是否相互独立,而不需要进行类似上面的计算去验证.

一般地,当事件 $A$ 和 $B$ 相互独立时,$A$ 与 $\overline{B}$,$\overline{A}$ 与 $B$,$\overline{A}$ 与 $\overline{B}$ 也相互独立.

由条件概率公式和相互独立事件 $A$,$B$ 的定义,可以得到
\begin{equation}
\begin{aligned}
&P(B) = P(B|A) = \frac{P(A \cap B)}{P(A)} \\
&P(A\cap B) = P(A) \cdot P(B)
\end{aligned}
\end{equation}

我们可以进一步推得,
\begin{equation}
P(A_1\cap A_2 \cdots A_n) = P(A_1) \cdot P(A_2) \cdots P(A_n)
\end{equation}

\subsection{独立重复试验}
在相同的条件下,重复地做 $n$ 次试验,各次试验从的结果相互独立,那么一般就称它为 \textbf{$n$次