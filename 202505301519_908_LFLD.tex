% 列夫·朗道(综述)
% license CCBYSA3
% type Wiki

本文根据 CC-BY-SA 协议转载翻译自维基百科 \href{https://en.wikipedia.org/wiki/Lev_Landau}{相关文章}。

列夫·达维多维奇·朗道(俄语:Лев Дави́дович Ланда́у,1908年1月22日-1968年4月1日)是一位苏联物理学家,在理论物理的诸多领域作出了基础性的贡献。\(^\text{[1][2][3]}\)他被认为是最后一批在物理学各个分支都造诣深厚并做出开创性贡献的科学家之一。\(^\text{[4]}\)他被誉为20世纪凝聚态物理学的奠基人,\(^\text{[5]}\)同时也被广泛认为是苏联最杰出的理论物理学家。\(^\text{[6]}\)
\subsection{生平}
\subsubsection{早年时期}
\begin{figure}[ht]
\centering
\includegraphics[width=6cm]{./figures/ce294ae7da64364f.png}
\caption{朗道一家,1910年} \label{fig_LFLD_1}
\end{figure}
朗道于1908年1月22日出生在俄罗斯帝国的巴库(今属阿塞拜疆),父母是犹太人\(^\text{[11][12][13][14]}\)。他父亲达维德·列沃维奇·朗道是一位从事当地石油工业的工程师,母亲柳博芙·维尼亚米诺芙娜·加尔卡维-朗道是一名医生。两人都来自莫吉廖夫,并毕业于当地的文理中学\(^\text{[15][16]}\)。朗道12岁学习微分学,13岁学习积分学,并在1920年13岁时从中学毕业。由于父母认为他年龄太小,不适合直接升入大学,他先在巴库经济技术学校学习了一年。
1922年,年仅14岁的朗道进入巴库国立大学,同时注册了两个系:物理与数学系以及化学系。后来他中止了化学的学习,但终其一生对化学始终保有兴趣。
\subsubsection{列宁格勒与欧洲时期}
\begin{figure}[ht]
\centering
\includegraphics[width=6cm]{./figures/5c374c1bae8ad643.png}
\caption{1914年的少年朗道} \label{fig_LFLD_2}
\end{figure}
1924年,朗道前往当时苏联物理学的主要中心——列宁格勒国立大学物理系,专注于理论物理的学习,并于1927年毕业。此后,他进入列宁格勒物理技术研究所攻读研究生,并最终于1934年获得物理-数学科学博士学位。\(^\text{[17]}\)1929年至1931年,朗道首次获得出国机会,依靠苏联政府(教育人民委员部)提供的出国奖学金,同时也得到了洛克菲勒基金会的资助。在这段时间里,他已能流利地使用德语和法语,并能以英语交流。\(^\text{[18]}\)后来,他进一步提高了英语水平,并学习了丹麦语。\(^\text{[19]}\)

朗道曾短暂访问哥廷根和莱比锡,随后于1930年4月8日前往哥本哈根,在尼尔斯·玻尔理论物理研究所工作,直到同年5月3日离开。这次访问之后,朗道始终视自己为尼尔斯·玻尔的学生,而他的物理研究方法也受到玻尔深刻的影响。离开哥本哈根后,朗道于1930年中期访问剑桥,与保罗·狄拉克合作研究,\(^\text{[20]}\)同年9月至11月他再次回到哥本哈根,\(^\text{[21]}\)接着于1930年12月至1931年1月在苏黎世与沃尔夫冈·泡利共事。\(^\text{[20]}\)从苏黎世出发后,他第三次前往哥本哈根,\(^\text{[22]}\)并于1931年2月25日至3月19日再次在那里停留,然后于同年返回列宁格勒。\(^\text{[23]}\)
\subsubsection{乌克兰哈尔科夫:国家科学中心哈尔科夫物理技术研究所}
1932年至1937年间,朗道担任国家科学中心哈尔科夫物理技术研究所理论物理系主任,并在哈尔科夫大学和哈尔科夫理工学院讲授课程。除理论研究外,朗道还是乌克兰哈尔科夫理论物理学传统的主要奠基人,这一学派有时被称为“朗道学派”。在哈尔科夫,他与朋友兼前学生叶甫根尼·利夫希茨开始撰写著名的《理论物理教程》,这一套涵盖理论物理全部领域的十卷巨著,至今仍被广泛用作研究生阶段的物理教材。在大清洗期间,朗道因涉及哈尔科夫的“UPTI 案”受到调查,但他设法离开哈尔科夫,前往莫斯科接受新职。\(^\text{[3]}\)

朗道制定了一项著名的综合考试,被称为“理论最低限”,学生只有通过这项考试后才能正式进入他的学派学习。该考试涵盖理论物理的各个方面。从1934年到1961年,仅有43人通过,但这些通过者后来都成为了非常杰出的理论物理学家。\(^\text{[24][25]}\)

1932年,朗道计算出了“钱德拉塞卡极限”;\(^\text{[26]}\)然而,他当时并未将其应用于白矮星。\(^\text{[27]}\)
\subsubsection{莫斯科物理问题研究所}
\begin{figure}[ht]
\centering
\includegraphics[width=8cm]{./figures/cfc5b79c45f1b664.png}
\caption{1938–1939年狱中照片} \label{fig_LFLD_3}
\end{figure}
\begin{figure}[ht]
\centering
\includegraphics[width=8cm]{./figures/a920820c2683ad17.png}
\caption{1934年在哈尔科夫研究所} \label{fig_LFLD_4}
\end{figure}
自1937年至1962年,朗道担任莫斯科物理问题研究所理论部主任。\(^\text{[28]}\)

1938年4月27日,朗道因持有一份传单而被捕,该传单将斯大林主义与德国纳粹主义和意大利法西斯主义相提并论。\(^\text{[3][29]}\)他被关押在内务人民委员部卢比扬卡监狱,直到1939年4月29日才获释。这次获释是由于彼得·卡皮察(著名的低温实验物理学家、该研究所创始人及所长)和尼尔斯·玻尔向约瑟夫·斯大林写信为他求情。\(^\text{[30][31]}\)卡皮察亲自担保朗道的品行,并威胁若不释放朗道将辞职离所。\(^\text{[32]}\)获释后,朗道提出了解释卡皮察发现的超流现象的理论,使用声子(声波激发)与一种新的激发形式——旋子。\(^\text{[3]}\)

朗道还带领一支数学家团队,支持苏联原子弹与氢弹的研制。他曾计算出首枚苏联热核武器的动力学过程,并预测了其当量。由于这项工作,朗道在1949年与1953年两度获得“斯大林奖”,并于1954年被授予“社会主义劳动英雄”称号。\(^\text{[3]}\)

朗道的学生包括:列夫·皮塔耶夫斯基、阿列克谢·阿布里科索夫、亚历山大·阿希耶泽尔、伊戈尔·贾洛辛斯基、叶甫根尼·利夫希茨、列夫·戈尔科夫、伊萨克·哈拉托尼科夫、罗阿尔德·萨格杰耶夫和伊萨克·波梅朗丘克。
\subsubsection{科学成就}
朗道的成就包括:密度矩阵方法在量子力学中的独立共同发现(与约翰·冯·诺依曼同时提出),量子力学的抗磁性理论,超流体理论,二级相变理论,金兹堡–朗道超导理论,费米液体理论,对等离子体物理中朗道阻尼现象的解释,量子电动力学中的朗道极点,中微子二分量理论,对火焰不稳定性的解释(即达里厄–朗道不稳定性),以及用于描述 S 矩阵奇点的朗道方程。

由于他建立了一个能解释液态氦II在低于2.17K(−270.98°C)时性质的超流体数学理论,朗道于1962年荣获诺贝尔物理学奖。\(^\text{[33]}\)
\subsubsection{个人生活与观点}
1937年,朗道与来自哈尔科夫的科拉·T·德罗班泽娃结婚。\(^\text{[34]}\)他们的儿子伊戈尔后来也成为一位理论物理学家。朗道主张“自由恋爱”而非一夫一妻制,他鼓励妻子和自己的学生实践“自由恋爱”。然而,他的妻子对此并不热衷。\(^\text{[3]}\)

朗道一般被描述为无神论者,\(^\text{[35][36][37]}\)然而当苏联电影导演安德烈·塔可夫斯基问他是否相信上帝的存在时,朗道沉默了三分钟,最终回答说:“我想是的。”\(^\text{[38]}\)1957年,克格勃向苏共中央提交了一份详细报告,记录了朗道对1956年匈牙利起义、弗拉基米尔·列宁以及他所称的“红色法西斯主义”的看法。\(^\text{[39]}\)物理学家亨德里克·卡西米尔回忆说,朗道是一个充满激情的共产主义者,并受其革命意识形态所激励。朗道在建立苏联科学体系方面的热情,部分来源于他对社会主义的忠诚。1935年,朗道在苏联《消息报》上发表文章《资产阶级与当代物理学》,批判宗教迷信和资本主导地位,他认为这两者都是资产阶级的倾向。他在文中强调:“党和政府为我国物理学的发展提供了前所未有的机遇。”\(^\text{[3]}\)
\subsubsection{晚年}
1962年1月7日,朗道乘坐的汽车与一辆迎面而来的卡车相撞,他伤势严重,昏迷了两个月。尽管他在许多方面恢复了健康,但他的科学创造力遭到了彻底破坏,\(^\text{[28]}\)此后再也未能完全回归科研工作。他的伤情也使他无法亲自前往领奖,接受1962年诺贝尔物理学奖。\(^\text{[40]}\)

朗道以其犀利幽默闻名一生,以下是他在事故恢复期间与心理学家亚历山大·鲁里亚之间的一段经典对话,\(^\text{[19][41]}\)鲁里亚当时正试图评估他是否有脑部损伤:

鲁里亚:“请给我画一个圆。”
朗道画了一个十字。
鲁里亚:“嗯,那请给我画一个十字。”
朗道画了一个圆。
鲁里亚:“朗道,你为什么不按照我说的做?”
朗道:“如果我照做了,你也许会以为我变傻了。”

1965年,朗道的几位学生和同事在莫斯科郊外的切尔诺戈洛夫卡创立了朗道理论物理研究所,该所在随后三十年由伊萨克·哈拉托尼科夫领导。

同年6月,朗道与叶夫谢·利伯曼在《纽约时报》上发表了一封公开信,声明他们作为苏联犹太人反对美国介入“苏联犹太学生运动”。\(^\text{[42]}\)然而,有学者质疑这封信是否真的由朗道亲自撰写。\(^\text{[43]}\)
\subsubsection{去世}
朗道于1968年4月1日去世,享年60岁,死因是六年前车祸所致伤病的并发症。他安葬于新圣女公墓。\(^\text{[44]}\)
\subsection{研究贡献领域}
\begin{itemize}
\item DLVO理论
\item 费米液体理论
\item 准粒子理论
\item 伊万年科–朗道–凯勒方程
\item 朗道阻尼
\item 朗道分布
\item 朗道规范
\item 朗道动理学方程
\item 朗道极点
\item 朗道磁化率
\item 朗道势能
\item 朗道量子化
\item 朗道理论
\item 朗道–斯奎尔射流
\item 朗道–列维奇问题
\item 朗道–霍普湍流理论
\item 金兹堡–朗道理论
\item 达里厄–朗道不稳定性
\item 朗道–李夫希茨气动声学方程
\item 朗道–雷查杜里方程
\item 朗道–曾纳公式
\item 朗道–李夫希茨模型
\item 朗道–李夫希茨赝张量
\item 朗道–李夫希茨–吉尔伯特方程
\item 朗道–波梅朗丘克–米格达尔效应
\item 朗道–杨定理
\item 朗道原理
\item 斯图尔特–朗道方程
\item 超流性
\item 超导性
\end{itemize}
\subsubsection{教育贡献}
\begin{itemize}
\item 《理论物理教程》
\end{itemize}
\subsection{#遗产}
\begin{figure}[ht]
\centering
\includegraphics[width=6cm]{./figures/86688521fccf2708.png}
\caption{} \label{fig_LFLD_5}
\end{figure}
为纪念朗道,有两个天体以他的名字命名:
\begin{itemize}
\item 小行星 2142 Landau。\(^\text{[45]}\)
\item 月球上的朗道环形山。
\end{itemize}
俄罗斯科学院授予理论物理领域的最高奖项也以他命名:
\begin{itemize}
\item 朗道金质奖章
\end{itemize}
2019年1月22日,谷歌以涂鸦形式庆祝朗道诞辰111周年。\(^\text{[46]}\)

此外,由美国物理学会设立、旨在表彰对等离子体物理的杰出贡献以及欧美合作成就的奖项:朗道–斯皮策奖也部分以他的名字命名。\(^\text{[47]}\)
\subsection{朗道对物理学家的评级}
\begin{figure}[ht]
\centering
\includegraphics[width=6cm]{./figures/05c41d75b2e99f43.png}
\caption{2008年俄罗斯邮票上的朗道} \label{fig_LFLD_6}
\end{figure}
朗道曾列出一份物理学家名单,并按照他们在创造力、天赋和科研成果等方面的表现,以对数等级制为基础,将他们划分为从 0 到 5 的等级。\(^\text{[48][49][50]}\)等级 0 是最高级,仅授予艾萨克·牛顿;阿尔伯特·爱因斯坦被评为 0.5;等级 1 授予了量子力学的奠基人们,如尼尔斯·玻尔、维尔纳·海森堡、萨蒂扬德拉·纳特·玻色、保罗·狄拉克和埃尔温·薛定谔等人;等级 5 的人被他称为“病理学家”,意指他们在科学上的天赋和创造力极低。\(^\text{[51]}\)朗道最初将自己评为 2.5 级,后来将自己提升为 2 级。物理学家 N·戴维·默明在撰写关于朗道的文章《我与朗道的生活:一个4.5级向2级致敬》中引用了这套评级体系,并幽默地称自己为第4级半。\(^\text{[52][53]}\)

此外,朗道还设计了一个鲜为人知的评级图示体系,以图形方式衡量科学家的天赋。他将科学家分为四类图形,其中第一类是一个简单的三角形,象征那些最具原创性和天赋的科学家,例如狄拉克和爱因斯坦。图形由两条平行线组成:下方的线表示“毅力”,上方的线表示“天赋与原创性”。\(^\text{[54]}\)
\subsection{在大众文化中}
\begin{itemize}
\item 2008年上映的俄罗斯电视电影《我丈夫——天才》(俄文片名《Мой муж — гений》,英文译名 My Husband — the Genius)讲述了朗道的生平(由达尼伊尔·斯皮瓦科夫斯基饰演),但主要聚焦于他的私人生活。该片普遍受到影评人批评。一些曾亲自与朗道接触过的人,包括著名俄罗斯科学家维塔利·金兹堡在内,表示这部影片不仅质量低劣,而且在历史事实方面存在严重错误。
\item 另一部关于朗道的影片《Dau》由伊利亚·赫尔扎诺夫斯基执导,主演为非职业演员、指挥家特奥多尔·库伦齐斯,饰演朗道。“Dau”是列夫·朗道的常用昵称。\(^\text{[55]}\)
这部影片是多学科艺术项目“DAU”的一部分。\(^\text{[56][57]}\)
\end{itemize}
\subsection{著作}
朗道的第一篇论文题为《克莱因–福克方程的导出》,与德米特里·伊万年科合著,发表于1926年,当时他年仅18岁。他的最后一篇论文题为《基本问题》,发表于1960年,是为纪念沃尔夫冈·泡利而编辑的文集中的一篇文章。朗道的全部著作清单于1998年发表在俄罗斯期刊《物理学-前沿》上。\(^\text{[58]}\)
朗道对署名一篇期刊论文有两个条件:他在研究中提出了部分或全部核心思想;他亲自参与了论文中至少一部分的计算工作。因此,朗道主动将自己的名字从一些学生的论文中撤下,因为他认为自己贡献不够大。\(^\text{[55]}\)
\subsubsection{《理论物理学教程》}
\begin{itemize}
\item 这是朗道与利夫希茨(E. M. Lifshitz)共同主编的一套十卷本物理学巨著,覆盖整个理论物理领域,被广泛用作研究生教学教材:《力学》(第1卷,第3版,1976年)ISBN 978-0-7506-2896-9
\item 《场论基础》(第2卷,第4版,1975年)ISBN 978-0-7506-2768-9
\item 《量子力学:非相对论理论》(第3卷,第3版,1977年)ISBN 978-0-08-020940-1 (第2版[1965] 可在 archive.org 查阅)
\item 《量子电动力学》(第4卷,第2版,1982年,合著:贝列斯捷茨基)ISBN 978-0-7506-3371-0
\item 《统计物理·第一部分》(第5卷,第3版,1980年)ISBN 978-0-7506-3372-7
\item 《流体力学》(第6卷,第2版,1987年)ISBN 978-0-08-033933-7
\item 《弹性理论》(第7卷,第3版,1986年)ISBN 978-0-7506-2633-0
\item 《连续介质电动力学》(第8卷,第2版,1984年)
   ISBN 978-0-7506-2634-7
\item 《统计物理·第二部分》(第9卷,第1版,1980年,合著:皮塔耶夫斯基)
   ISBN 978-0-7506-2636-1
\item 《物理动力学》(第10卷,第1版,1981年,合著:皮塔耶夫斯基)
    ISBN 978-0-7506-2635-4
\end{itemize}
元素周期表的修正建议,朗道与利夫希茨在《理论物理学教程》第三卷中指出,当时标准的元素周期表存在一个错误:铥应被归入d区元素,而非f区元素。这一观点后来得到了实验证据的充分支持,\(^\text{[59–62] }\)并在1988年被国际纯粹与应用化学联合会的一份报告正式采纳。\(^\text{[63]}\)
\subsubsection{其他著作}
\begin{itemize}
\item L. D. 朗道 与 A. S. 科姆帕涅茨(1965年 [原作于1935年]):《附录 A:金属的电导率》(The Electrical Conductivity of Metals),ONTI出版社,哈尔科夫,第803–832页。doi:10.1016/B978-0-08-010586-4.50106-1,ISBN 9780080105864。
\item  L. D. 朗道 与 Ya. 斯莫罗金斯基(2011年 [原作于1958年]):《原子核理论讲义》,Dover Publications,ISBN 978-0486675138。
\item L. D. 朗道 与 G. B. 鲁默(2003年 [原作于1960年]):《什么是相对论?》,Dover Publications,ISBN 978-0-48-616348-2。
\item L. D. 朗道,A. I. 阿希耶泽尔,E. M. 利夫希茨(1967年):《普通物理学:力学与分子物理》,Pergamon Press,ISBN 978-0-08-009106-8。
\item L. D. 朗道 与 A. I. 基塔伊哥罗茨基(1978年):《给所有人的物理学》,莫斯科 Mir 出版社,共四卷:第1卷《物体》ISBN 978-0-82-851716-4;第2卷《分子》ISBN 978-0-82-851725-6;第3卷《电子》与第4卷《光子与原子核》由基塔伊哥罗茨基单独撰写。
\end{itemize}
\subsection{参见}
\begin{itemize}
\item 犹太诺贝尔奖得主名单
\item 以列夫·朗道命名的事物列表
\end{itemize}
\subsection{参考文献}
\begin{enumerate}
\item McCauley, Martin(1997年):《1900年以来的俄罗斯名人录》,Routledge出版社,第128页。朗道,列夫·达维多维奇(1908–1968),才华横溢的苏联理论物理学家,出生于巴库的一个犹太家庭,1927年毕业于列宁格勒国立大学。
\item Zubok, Vladislav(2012年):《斯大林去世后的苏联知识分子及其对冷战终结的愿景》,载于 Bozo, Frédéric;Rey, Marie-Pierre;Rother, Bernd;Ludlow, N. Piers 编,《1945–1990年欧洲冷战终结的愿景》,Berghahn Books,第78页。
\item Gorelik, Gennady(1997年8月):《列夫·朗道的绝密人生》,《科学美国人》(Scientific American),第277卷第2期,第72–77页。Bibcode:1997SciAm.277b..72G,doi:10.1038/scientificamerican0897-72。JSTOR 24995874。2018年6月18日存档并检索。
\item Smilga, Andrei V.(2017年):《可消化的量子场论》(Digestible Quantum Field Theory),施普林格出版社(Springer),第250页。ISBN 978-3-319-59922-9。
\item Gorelik, Gennady(1997年):《列夫·朗道的绝密人生》,《科学美国人》(Scientific American),第277卷第2期,第72–77页。Bibcode:1997SciAm.277b..72G,doi:10.1038/scientificamerican0897-72。ISSN 0036-8733。JSTOR 24995874。
\item Ryndina, Ella(2004年2月1日):《朗道画像中的家族脉络》,《今日物理》(Physics Today),第57卷第2期,第53–59页。Bibcode:2004PhT....57b..53R,doi:10.1063/1.1688070。ISSN 0031-9228。
\item 朗道,列夫(1927年):《波动力学中的阻尼问题》,《物理学杂志》,第45卷第5–6期,第430–441页。Bibcode:1927ZPhy...45..430L,doi:10.1007/bf01343064。S2CID 125732617。英文译文收录于 D. Ter Haar 编(1965年)《朗道文集》,牛津:佩加蒙出版社。
\item Schlüter, Michael;Lu Jeu Sham(1982年):《密度泛函理论》,《今日物理》(Physics Today),第35卷第2期,第36页。Bibcode:1982PhT....35b..36S,doi:10.1063/1.2914933。S2CID 126232754。2013年4月15日存档。
\item Fisher, Michael E.(1998年4月1日):《重整化群理论:其基础与统计物理中的形式化》,《现代物理评论》,第70卷第2期,第653–681页。Bibcode:1998RvMP...70..653F,doi:10.1103/RevModPhys.70.653。
\item Shifman, M. 编(2013年):《朗道的魔力:当理论物理塑造命运之时》,世界科学出版社,doi:10.1142/8641。ISBN 978-981-4436-56-4。
\item Kapitza, P. L.; Lifshitz, E. M.(1969):《列夫·达维多维奇·朗道(1908–1968)》,载于《英国皇家学会会员传记回忆录》,第15卷,第140–158页。doi:10.1098/rsbm.1969.0007。
\item 马丁·吉尔伯特,《二十世纪的犹太人:图文史》(The Jews in the Twentieth Century: An Illustrated History),Schocken Books 出版,2001年,ISBN 0805241906,第284页。
\item 《物理前沿:朗道纪念会议论文集》,以色列特拉维夫,1988年6月6日至10日,Pergamon 出版社,1990年,ISBN 0080369391,第13–14页。
\item 爱德华·泰勒,《回忆录:二十世纪的科学与政治之旅》,Basic Books 出版,2002年,ISBN 0738207780,第124页。
\item “伟大的巴库本地人列夫·朗道”,Vestnik Kavkaza。2019年6月10日存档。检索日期:2019年1月22日。
\item “莫吉廖夫中学毕业生”。[www.petergen.com。检索日期:2022年11月22日。](http://www.petergen.com。检索日期:2022年11月22日。)
\item 弗兰季谢克·亚努赫,《列夫·朗道:一位理论物理学家的画像,1908–1988》,物理研究所,1988年,第17页。
\item 尤里·鲁梅尔,《朗道》,berkovich-zametki.com。
\item 玛雅·别萨拉布,《朗道的生活篇章》,莫斯科工人出版社,1971年,莫斯科。
\item 贾格迪什·梅赫拉,《理论物理的黄金时代》,两卷盒装版,世界科学出版社,2001年,第952页,ISBN 9810243421。
\item 在此期间,朗道三次访问哥本哈根:1930年4月8日至5月3日,1930年9月20日至11月22日,以及1931年2月25日至3月19日(参见朗道传记——MacTutor 数学史档案馆)。
\item Sykes, J. B.(2013年):《朗道:物理学家与人:对朗道的回忆》,Elsevier 出版社,第81页,ISBN 9781483286884。
\item Haensel, P.; Potekhin, A. Y. 与 Yakovlev, D. G.(2007):《中子星1:状态方程与结构》,施普林格科学与商业媒体公司,第2页,ISBN 0387335439。
\item Stephen J. Blundell(2009):《超导:极简介绍》,牛津大学出版社,第67页,ISBN 9780191579097。
\item Ioffe, B. L.(2002):〈朗道的“理论最低限”、朗道研讨班与1950年代初期的苏联核物理研究所(ITEP)〉,arXiv\:hep-ph/0204295。
\item 《恒星理论》,收录于《朗道论文集》,D. ter Haar 编辑并作序,纽约:戈登与布里奇出版社,1965年;最初发表于《苏联物理杂志》Phys. Z. Sowjet. 第1卷(1932年),第285页。
\item Dmitrii Yakovlev 与 Pawel Haensel(2013):〈列夫·朗道与中子星概念〉,刊于《物理学进展》,第56卷第3期,第289–295页。arXiv:1210.0682。Bibcode:2013PhyU...56..289Y。doi:10.3367/UFNe.0183.201303f.0307。S2CID 119282067。
\item Alexander Dorozynsk(1965):《那个他们不让死去的人》。
\item 彼得·卡皮察纪念博物馆-书房,《院士卡皮察:传记概述》。
\item O'Connor, 2014。
\item Yakovlev, 2012。
\item 理查德·罗德斯:《黑太阳:氢弹的制造》,西蒙与舒斯特出版社,1995年,ISBN 0684824140,第33页。
\item 《列夫·达维多维奇·朗道,苏联物理学家及诺贝尔奖得主》,发表于《今日物理》,第57卷第2期,第62页,2004年。Bibcode:2004PhT....57Q..62..,doi:10.1063/1.2408530。
\item 彼得·列昂尼多维奇·卡皮察:《实验、理论与实践:论文与演讲集》,施普林格出版社,1980年,ISBN 9027710619,第329页。
\item 亨利·F·谢弗(2003):《科学与基督教:冲突还是一致?》,阿波罗信托基金会,第9页,ISBN 9780974297507。“我在此举出两位著名的无神论者,其一是列夫·朗道,二十世纪最杰出的苏联物理学家。”
\item “列夫·朗道”。Soylent Communications,2012年。访问日期:2013年5月7日。
\item 詹姆斯·D·帕特森与伯纳德·C·贝利(2019年2月20日):《固态物理:理论导论》。章节《列夫·朗道——苏联的大师》,施普林格出版社,第190页,ISBN 9783319753225。“朗道的‘理论最低限’考试非常著名,当时仅有约40人通过。这是他为进入理论物理领域设置的入门考试,内容涵盖他认为从事该领域所需的一切。与许多苏联时代的物理学家一样,他是一位无神论者。”
\item 安德烈·塔可夫斯基(1987):《雕刻时光:这位伟大的俄罗斯电影导演论述他的艺术》,亨特–布莱尔译,德克萨斯大学出版社,第229页,ISBN 0-292-77624-1。
\item 1957年12月19日(无编号),布科夫斯基档案。
\item 瑞典皇家科学院院士 I. 瓦勒教授在诺贝尔奖颁奖典礼上的演讲。来源:Nobelprize.org,检索日期:2012年1月28日。
\item 科拉·德罗班采娃的回忆录,第38章《我们的生活方式》;包括亚历山大·卢里亚测试列夫·朗道智力的片段(在俄文原文中称其为“Лурье”)。
\item 雅科夫·罗伊,《苏联犹太人移民斗争,1948–1967年》(The Struggle for Soviet Jewish Emigration, 1948–1967),剑桥大学出版社,2003年,ISBN 0521522447,第199页。
\item “Если нужен вор, его и с виселицы снимают”(俄语,意为“如果需要一个贼,即使已经吊死了也会把他从绞刑架上取下来”——比喻只要需要,就不择手段地使用人)。
\item 新圣女公墓的朗道纪念碑。来源:novodevichye.com,2008年10月26日,访问时间:2012年1月28日。
\item 施马德尔,卢茨·D((2003):《小行星名称词典》,第5版,施普林格出版社,第174页,ISBN 3-540-00238-3。
\item 希瓦利·贝斯特(2019年1月22日):“谷歌涂鸦庆祝理论物理学家列夫·朗道111岁诞辰”,Mirror网站,访问时间:2019年1月22日。
\item “朗道–斯皮策奖”,来源:APS.org,美国物理学会。
\item 埃尔克霍农·戈德堡(2018):《创造力:创新时代的人脑》,纽约:牛津大学出版社,第166页,ISBN 978-0-19-046649-7。
\item 李继超、殷逸安、桑托·福尔图纳托、王达顺(2019年4月18日):“诺贝尔奖获得者几乎和我们一样”,发表于《自然评论·物理》,第1卷第5期,第301–303页。Bibcode:2019NatRP...1..301L,doi:10.1038/s42254-019-0057-z,访问日期:2024年4月9日。
\item 亚当·阿尔特(2023):《突破的解剖:在关键时刻如何摆脱困境》,纽约:西蒙与舒斯特出版社,第214页,ISBN 978-1-9821-8296-0。
\item 安娜·利瓦诺娃(1983):《朗道》(俄文,第2次增订版),Znanie出版社,访问日期:2022年8月1日。
\item N. 大卫·默明(1990):《一路布咕姆:在平实时代传递科学》,剑桥大学出版社,第39页,ISBN 9780521388801。
\end{enumerate}
\subsection{进一步阅读}
\textbf{图书}
\begin{itemize}
\item 多罗任斯基,亚历山大(1965):《那个他们不让死去的人》,Secker and Warburg出版社,ASIN B0006DC8BA。(在朗道1962年车祸之后,他身边的物理学界聚集起来,努力挽救他的生命。他们成功地将他的生命延续到了1968年。)
\item 雅努什,弗朗季舍克(1979):《列夫·D·朗道:他的生平与工作》,欧洲核子研究中心,ASIN B0007AUCL0。
\item 哈拉特尼科夫,I. M.(主编)(1989):《朗道:物理学家与人》,朗道回忆录,由J. B. Sykes翻译,佩加蒙出版社,ISBN 0-08-036383-0。
\item 科热夫尼科夫,阿列克谢·B(2004):《斯大林的大科学:苏联物理学家的时代与历险》,《现代物理科学史》丛书,帝国理工学院出版社,ISBN 1-86094-420-5。
\item 朗道-德罗班采娃,科拉(1999):《朗道教授:我们的生活方式》(Professor Landau: How We Lived,俄文),AST出版社,ISBN 5-8159-0019-2。该书已于2005年5月4日存档。
\item 利瓦诺娃,安娜(1980):《朗道:伟大的物理学家与教师》,伯灵顿:佩加蒙出版社,ISBN 978-0-08-023076-4。
\item 希夫曼,M.(M. Shifman,主编)(2013):《朗道的魔力:当理论物理学改变命运》,世界科学出版社,doi:10.1142/8641,ISBN 978-981-4436-56-4。
\end{itemize}
\textbf{文章}
\begin{itemize}
\item 卡尔·胡夫鲍尔:“朗道年轻时对恒星理论的尝试:其起源、主张与反响”,《物理与生物科学历史研究》,第37卷(2007年),第337–354页。
\item 《作为一名学生,朗道曾在课堂上大胆纠正爱因斯坦》,发表于《全球人才新闻》。
\item 约翰·J·奥康纳、埃德蒙·F·罗伯逊:“列夫·朗道”,发表于圣安德鲁斯大学《MacTutor数学史档案》。
\item “列夫·达维多维奇·朗道”,见《诺贝尔奖得主》数据库。
\item 鲍里斯·L·约菲撰写的《朗道的理论极限、朗道的研讨班、1950年代初的苏联理论物理研究所(ITEP)》,在研讨会《QCD在第四个十年的门槛上 / Ioffefest》上发表的总结演讲。
\item 《欧洲理论物理杂志》2008年“朗道特刊”。
\item 阿马尔·萨卡吉与伊尼亚齐奥·利卡塔主编:《列夫·达维多维奇·朗道及其对当代理论物理学的影响》,纽约,Nova Science出版社,2009年,ISBN 978-1-60692-908-7。
\item 根纳季·戈雷利克:“列夫·朗道的绝密生活”,发表于《科学美国人》,1997年8月,第277卷第2期,第53–57页,JSTOR链接可查。
\item 玛雅·别萨拉布:“朗道的生命篇章(俄文)”。
\end{itemize}

