% 狄拉克 delta 函数
% 微积分|定积分|狄拉克 delta 函数|\delta 函数

\begin{issues}
\issueTODO
\end{issues}

\pentry{定积分\upref{DefInt}}
在物理中我们经常会遇到一些模型, 如质点和点电荷等, 这类模型使用了极限的思想(如令体积趋于无穷小). 如果考察质点的密度或点电荷的电荷密度, 将得到无穷大, 然而将其密度(电荷密度)在空间中积分却又能得到有限的质量与电荷. 为了描述这样的密度(电荷密度)分布, 我们引入\textbf{狄拉克 $\delta$ 函数(Dirac delta function)}.

\subsection{一维情况}

$\delta(x)$ 在泛函分析中并不是数学中一个严格意义上的函数, 而是被称为\textbf{广义函数(generalized function)}或\textbf{分布(distribution), 详见泛函分析}. 但本文中我们给出另含有 $\delta(x)$ 等式的另严谨的理解: 即把 $\delta$ 函数看作是\textbf{对一系列函数取极限的过程}.

\begin{figure}[ht]
\centering
\includegraphics[width=11cm]{./figures/Delta_1.pdf}
\caption{$\delta(x - x_0)$ 的几个例子} \label{Delta_fig1}
\end{figure}
我们来考虑一个函数(\autoref{Delta_fig1} 左)
\begin{equation}\label{Delta_eq6}
f(x) =
\begin{cases}
h & \qty( \abs{x - x_0} \leqslant \frac{1}{2h} )\\
0 & \qty( \abs{x - x_0} > \frac{1}{2h} )
\end{cases}
\end{equation}
其中 $h, x_0$ 是常数. 由函数图像易得函数曲线下面的面积为 $\int_{-\infty}^{+\infty} f(x) \dd{x} = 1$. 现在我们令 $h \to \infty$, 长方形的高将趋于无穷大, 宽将趋于零, 而定积分结果不变.

这样, 上面的 $f(x)$ 就可以表示为 $\delta(x - x_0)$. 当然, 我们还可以选取其他含有参数的 $f(x)$ 来逼近 $\delta$ 函数, 如\autoref{Delta_fig1} 中的另外两种函数.

\begin{definition}{狄拉克 $\delta$ 函数}\label{Delta_def1}
令一个含有参数 $\alpha$ 的一元函数 $f_\alpha(x)$. $\alpha$ 可以是连续的或离散的. 如果 $\alpha$ 是离散的, 我们可以将其看作一个由无穷多个函数组成的函数序列 $f_1(x), f_2(x), \dots$ 若 $f_\alpha(x)$ 在所有 $x \ne 0$ 处都满足
\begin{equation}\label{Delta_eq4}
\lim_{\alpha\to+\infty} f_\alpha(x) = 0 \qquad (x \ne 0)
\end{equation}
且(注意极限和积分不可交换)
\begin{equation}\label{Delta_eq5}
\lim_{\alpha\to+\infty} \int_{-\infty}^{\infty} f_\alpha(x) \dd{x} = 1
\end{equation}
那么我们说 $f_\alpha(x)$ 是 $\alpha\to+\infty$ 时的狄拉克 $\delta$ 函数, 记为
\begin{equation}\label{Delta_eq1}
\lim_{\alpha\to+\infty} f_\alpha(x) = \delta(x)
\end{equation}
\end{definition}
注意\autoref{Delta_eq1} 中的等号并不是通常意义的两函数相等, 而是隐含了\autoref{Delta_eq4} 和\autoref{Delta_eq5} 两个条件. 另外在一些教材会把 $\delta(x)$ 的性质简单记为
\begin{equation}\label{Delta_eq2}
\delta(x) =
\begin{cases}
+\infty & (x = 0)\\
0 & (x \ne 0)
\end{cases}
\end{equation}
\begin{equation}\label{Delta_eq3}
\int_{-\infty}^{+\infty} \delta(x) \dd{x} = 1
\end{equation}
这些表达式都需要按照\autoref{Delta_def1} 来严格理解.

\begin{example}{高斯波包}
\addTODO{……}
\end{example}
\begin{exercise}{sinc 函数}
\begin{equation}
\lim_{a\to\infty}\frac{a}{\pi}\sinc(a x) = \lim_{a\to\infty}\frac{a}{\pi}\sinc(a x) = \delta(x)
\end{equation}
提示: \autoref{sinc_eq1}~\upref{sinc}.
\end{exercise}

\subsubsection{多元 delta 函数}
\addTODO{……}

\subsection{性质}
若一个等式中出现了 $\delta$ 函数, 那么其意义是先将 $\delta(x)$ 替换为 $f_\alpha(x)$. 该\textbf{等式在 $\alpha\to+\infty$ 的极限时成立}.

例如一个重要的性质是: 对任意连续函数 $f(x)$, 有
\begin{equation}
\int_{-\infty}^{+\infty} f(x) \delta(x - x_0) \dd{x}= f(x_0)
\end{equation}
使用\autoref{Delta_def1} 中的 $f_\alpha(x)$ 表示, 该等式的严格意义是
\begin{equation}
\lim_{\alpha\to+\infty}\int_{-\infty}^{+\infty} f(x) f_\alpha(x - x_0) \dd{x}= f(x_0)
\end{equation}

我们来不严谨地证明\autoref{Delta_eq6} : 首先把积分上下限变为 $x_0 \pm \epsilon$, 这样并不会改变积分结果, 因为在区间 $[x_0+\epsilon, +\infty)$ 和 $(-\infty, x_0 - \epsilon]$ 中, $\delta(x-x_0) = 0$. 然后我们令 $\epsilon\to 0$, 这样 $f(x)$ 在 $[x_0 - \epsilon, x_0 + \epsilon]$ 区间内的值就趋近于常数 $f(x_0)$. 所以有
\begin{equation}\ali{
\int_{-\infty}^{+\infty} f(x) \delta(x - x_0) \dd{x} &= \int_{x_0-\epsilon}^{x_0+\epsilon} f(x) \delta(x - x_0)\dd{x} \\
&= f(x_0)\int_{x_0-\epsilon}^{x_0+\epsilon} \delta(x - x_0)\dd{x} = f(x_0)
}\end{equation}
证毕.

$\delta$ 函数的另一个性质是
\begin{equation}
\delta[f(x)] = \frac{1}{\abs{f'(0)}} \delta(x)
\end{equation}
我们不妨从几何上来(不严谨地)证明这个性质: 可以证明 $\delta[f(x)] = \delta[f'(0) x]$ 与 $\delta(x)$ 相比较, $\delta[f(x)]$ 的图像在 $x$ 方向变窄了 $\abs{f'(0)}$ 倍, 所以函数曲线下的面积变为原来的 $1/\abs{f'(0)}$ 倍, 故 $\abs{f'(0)}\delta[f(x)]$ 下的面积是 $1$, 且满足\autoref{Delta_eq2} 和\autoref{Delta_eq3}, 所以有 $\abs{f'(0)}\delta[f(x)] = \delta(x)$. 证毕.

\subsection{其他应用}
例如出现在微分方程中(量子力学的 $\delta$ 函数势阱, 或者格林函数发的微分方程), 意思是取微分方程序列, 然后取解的极限. 这样就不需要 “广义函数” 的概念了. 又例如傅里叶变换的归一化
\begin{example}{}
在傅里叶分析中, 时常会看到
\begin{equation}
\int_{-\infty}^\infty \E^{\I kx}\dd{x} = 2\pi \delta(k)
\end{equation}
然而严格来说, 左边的积分并不收敛, 所以我们不能 “按字面意思” 理解含有 $\delta$ 函数的等式. 要严格理解, 先令
\begin{equation}
I(l) = \int_{-l}^{l} \E^{\I kx}\dd{x} = \frac{2\sin(l k)}{l \Delta k} = 2l \sinc(l \Delta k)
\end{equation}
当 $l \to \infty$ 时, 可以证明


 但我们可以将其理解为积分 在 $l \to \infty$ 的过程中满足 $2\pi\delta(k)$ 的条件, 即 


为了证明, 我们先来把积分区间变为有限的
\begin{equation}
\int_{-l}^{l} \E^{-\I k' x}\E^{\I kx}\dd{x} = 2\pi \delta(k' - k)
\end{equation}
\end{example}
