% 弗雷德里克·塞茨(综述)
% license CCBYSA3
% type Wiki

本文根据 CC-BY-SA 协议转载翻译自维基百科\href{https://en.wikipedia.org/wiki/Frederick_Seitz}{相关文章}。

弗雷德里克·塞茨(1911年7月4日-2008年3月2日)是美国物理学家,固态物理学的先驱,也是气候变化否认者。塞茨曾担任洛克菲勒大学第4任校长(1968年-1978年),以及美国国家科学院第17任院长(1962年-1969年)。塞茨获得了国家科学奖章、NASA杰出公共服务奖等多项荣誉。

他在伊利诺伊大学厄本那-香槟分校创立了弗雷德里克·塞茨材料研究实验室,并在美国其他地方创立了多个材料研究实验室。[1][2] 塞茨还是乔治·C·马歇尔研究所的创始主席。[3]
\subsection{背景和个人生活} 
塞茨于1911年7月4日出生在旧金山。他的母亲也来自旧金山,而以他命名的父亲则出生于德国。[4] 塞茨在高中最后一年中途从利克-威尔默丁高中毕业,随后进入斯坦福大学学习物理,并在三年内获得学士学位,[1] 于1932年毕业。[5] 他于1935年5月18日与伊丽莎白·K·马歇尔结婚。[6]

塞茨于2008年3月2日去世,享年96岁,地点为纽约。[7][8] 他留下了一子、三名孙子和四名曾孙。[7]
\subsection{早期职业生涯}
\begin{figure}[ht]
\centering
\includegraphics[width=6cm]{./figures/88a94b6bc3e7a33c.png}
\caption{Wigner–Seitz原胞的构建。} \label{fig_Seitz_1}
\end{figure}
塞茨前往普林斯顿大学,在尤金·维格纳的指导下研究金属,[1] 并于1934年获得博士学位。[7][9] 他与维格纳一起开创了最早的晶体量子理论之一,并在固态物理学中提出了诸如Wigner–Seitz单元格[1]等概念,该概念被用于晶体材料的研究。
\subsection{学术生涯}  
在研究生学习后,赛茨继续从事固态物理研究,并于1940年出版了《固体的现代理论》一书,目的是“写出一部关于固态物理各个方面的连贯论述,为该领域提供应有的统一性”。《固体的现代理论》帮助统一并理解了冶金学、陶瓷学和电子学等领域之间的关系。他还曾为许多与第二次世界大战相关的项目提供咨询,涉及冶金学、固体辐射损伤和电子学等方面。他与希拉德·亨廷顿共同首次计算了铜中空位和间隙原子形成与迁移的能量,激发了许多关于金属中点缺陷的研究。他的出版作品范围广泛,涵盖了“光谱学、光致发光、塑性变形、辐照效应、金属物理、自扩散、金属和绝缘体中的点缺陷以及科学政策”等多个领域。

赛茨的学术生涯初期,他曾在罗切斯特大学(1935–1937)担任教职,随后在通用电气实验室(1937–1939)担任研究物理学家。之后,他分别在宾夕法尼亚大学(1939–1942)和卡内基技术学院(1942–1949)工作。

从1946年到1947年,赛茨担任橡树岭国家实验室的原子能培训项目主任。1949年,他被任命为伊利诺伊大学香槟分校物理学教授,1957年成为物理系主任,1964年成为研究副校长兼院长。赛茨还曾担任北约顾问。从1962年到1969年,赛茨担任美国国家科学院(NAS)主席,1965年起为全职工作。在担任NAS主席期间,他发起了大学研究协会,该协会与原子能委员会签订合同,建设当时世界上最大的粒子加速器——费米实验室(Fermilab)。

他于1968年至1978年担任洛克菲勒大学校长,在此期间,他帮助启动了分子生物学、细胞生物学和神经科学的新研究项目,并与康奈尔大学共同创建了MD-PhD联合学位项目。[7]他于1979年从洛克菲勒大学退休,并被授予名誉校长职衔。
\subsection{咨询生涯}  
在塞茨发表关于晶体变暗的论文后,杜邦公司于1939年请他帮助解决他们在铬黄稳定性方面的问题。他“深度参与”了他们的研究工作。[11] 其中,他研究了非毒性碳化硅作为白色颜料的可能性。[12]塞茨曾担任德州仪器公司(1971-1982)和阿克佐诺公司(1973-1982)的董事。[13]

在1979年从洛克菲勒大学退休前不久,塞茨开始作为R.J. Reynolds烟草公司的一名长期顾问,向他们的医学研究项目提供咨询,直到1988年。[14] 之前,雷诺公司曾为洛克菲勒大学的生物医学工作提供了“非常慷慨”的资助。[15]塞茨后来写道:“这些钱都用于基础科学、医学科学,”并指出雷诺公司资助的疯牛病和结核病研究。[7] 然而,后来关于烟草行业影响的学术研究得出结论,塞茨在帮助分配雷诺公司4500万美元的研究资金时,“在帮助烟草行业制造关于吸烟健康影响的不确定性方面发挥了关键作用。”[16][17] 根据1989年一份烟草行业的备忘录,菲利普·莫里斯国际公司的一位员工描述塞茨为“相当年老,且不够理性以提供建议。”[18]

1984年,塞茨成为乔治·C·马歇尔研究所的创始主席,并一直担任该职位直到2001年。[19][20] 该研究所成立的目的是支持里根总统的战略防御倡议(SDI),[23] 但“到了1990年代,它开始扩展,成为试图驳斥气候变化科学的领先智库之一。”[24][25] 1990年,塞茨与研究所的共同创始人罗伯特·贾斯特罗和威廉·尼尔伯格合著的一份报告“在很大程度上影响了布什政府关于人为气候变化的立场。”[26] 该研究所还广泛推广环境怀疑主义。1994年,研究所发布了塞茨的一篇论文,标题为《全球变暖与臭氧层空洞争议:对科学判断的挑战》。塞茨质疑了CFCs(氯氟烃)“是臭氧层最大威胁”这一观点。[27] 在同一篇论文中,评论了二手烟的危害时,他得出结论:“在正常情况下,没有充分的科学证据表明被动吸入烟雾真的有危险。”[28]

塞茨是全球变暖否认者中的核心人物。[7][29] 他是90年代初开始坚定反驳全球变暖是严重威胁的科学怀疑者中排名最高的科学家。[30] 塞茨认为,关于全球变暖的科学证据并不确定,并且“肯定不值得强制实施温室气体排放的限制。”[30] 2001年,塞茨和贾斯特罗质疑全球变暖是否是人为造成的。[31]

塞茨签署了1995年的《莱比锡声明》,并在一封公开信中邀请科学家签署俄勒冈州科学与医学研究所的全球变暖请愿书,呼吁美国拒绝《京都议定书》。[7] 这封信附带了一篇12页的关于气候变化的文章,其风格和格式几乎与《美国国家科学院院刊》(PNAS)的投稿相同,甚至包括了出版日期(“10月26日”)和卷号(“第13卷:149-164 1999”),但实际上并不是《美国国家科学院院刊》的出版物。在回应中,美国国家科学院采取了《纽约时报》所称的“非常举措,公开反驳其一位前总统的立场。”[7][33][34] 美国国家科学院还明确表示,“该请愿书并不代表学院专家报告的结论。”[33]

在他的咨询生涯中,塞茨与弗雷德·辛格(Fred Singer)进行了广泛合作,分别在健康和气候变化问题上为烟草和石油公司提供咨询。[35]
\subsection{出版作品}  
塞茨在他的领域写了多本科学书籍,包括《固体的现代理论》(1940年)和《金属物理学》(1943年)。后来,他与人合作编写了《晶格动力学的理论(在谐近似下)》(1971年)和《固体物理学》一书。[36] 后者始于1955年,与大卫·特恩布尔(David Turnbull)合作,到2008年已经出版了60卷,塞茨一直活跃于编辑工作,直到1984年出版的第38卷。[1] 《固体物理学》仍由Elsevier出版。[37] 在退休后,他与人合作编写了一本关于全球变暖的书籍,通过他所主持的乔治·C·马歇尔研究所出版。他于1994年出版了自己的自传。其他作品包括美国物理学家弗朗西斯·惠勒·卢米斯(Francis Wheeler Loomis,1991年)的传记和加拿大发明家雷金纳德·费森登(Reginald Fessenden,1999年)的传记、硅的历史,以及美国国家科学院的历史(2007年)。
\subsection{批评}  
在1970年代初,塞茨因支持越南战争而变得不受欢迎,这一立场与总统科学顾问委员会的大多数同事的意见不一致。在1970年代末,塞茨还在核武器准备问题上与他的科学同僚分道扬镳。塞茨致力于“通过最先进的军事技术强化军力”,而科学界普遍支持军备限制谈判和条约。塞茨也是一个坚定的反共主义者,他对进攻性武器项目的支持反映了这一点。[35]

在《怀疑的商人》一书中,科学史学者娜奥米·奥雷斯基(Naomi Oreskes)和埃里克·M·康威(Erik M. Conway)指出,塞茨和其他一群科学家在许多20世纪和21世纪最重要的问题上与科学证据作斗争,并传播混淆信息,这些问题包括烟草烟雾的危害、酸雨、氯氟烃(CFCs)、农药和全球变暖。塞茨曾表示,美国的科学已经变得“僵化”,他的同事们变得闭塞和教条主义。根据奥雷斯基和康威的说法,塞茨利用科学证据中的正常不确定性来传播有关烟草烟雾危害的怀疑。[35]

塞茨还是臭名昭著的俄勒冈请愿书的主要组织者之一,许多签署者声称没有证据表明温室气体是导致全球变暖的原因。尽管塞茨曾是美国国家科学院的前任院长,但美国国家科学院发布新闻稿称:“该请愿项目是故意试图误导科学家,并召集他们试图破坏对《京都议定书》的支持。该请愿书并未基于对全球气候变化科学的审查,也没有签署者是气候科学领域的专家。”[38] 随后,记者发现绝大多数签署者的身份无法得到验证,[39] 因为请愿书的组织者没有身份验证的程序。此外,所谓的科学文章声称反驳全球变暖(并随请愿书一起发布)实际上是一篇未经过同行评审的文章,来自《美国医学会与外科医生杂志》,该杂志由请愿书的共同组织者阿瑟·罗宾逊(Arthur Robinson)发布。[40] 该杂志提倡一些已被科学界否定的观点,例如声称HIV病毒与艾滋病之间没有关系,并且没有被PubMed索引。

奥雷斯基和康威批评了塞茨参与烟草行业的行为。他们指出,塞茨反对吸烟对人类健康的科学共识,并帮助制造了这一问题的混乱和怀疑。
\subsection{奖项与荣誉}  
塞茨于1946年当选为美国哲学学会会员。[41] 他于1952年当选为美国国家科学院院士,并在1962年至1969年期间担任该院院长。[10] 他于1962年当选为美国艺术与科学院院士。[42] 他获得了富兰克林奖章(1965年)。1973年,他因“对固体物质现代量子理论的贡献”获得了国家科学奖章。[7] 他还获得了美国国防部杰出服务奖、美国国家航空航天局杰出公共服务奖,以及康普顿奖,这是美国物理学会的最高荣誉。[7] 除了洛克菲勒大学外,美国和国外31所大学授予塞茨荣誉学位。[43] 他还是外交关系委员会的成员。[43]

塞茨曾在多个慈善机构的董事会中服务,包括(担任主席)约翰·西蒙·古根海姆纪念基金会(1976–1983年[13])和伍德罗·威尔逊国家奖学金基金会,[5] 以及(作为受托人)美国自然历史博物馆(自1975年起[13])和国际教育研究所。[5] 他还曾是战略与国际研究中心的董事会成员。[5] 其他国家和国际机构的任命包括在国防科学委员会任职,并担任美国代表团团长,参与联合国科学与技术委员会的工作。[5] 他还曾在科学服务董事会任职,现已更名为科学与公众协会,任职时间为1971年至1974年。

1981年,塞茨成为世界文化委员会的创始成员。[44]
\subsection{担任职务}  
\subsubsection{学术职务}
\begin{itemize}
\item 卡内基技术学院,物理系主任(1946年–?)[45]  
\item 伊利诺伊大学,物理学教授(1949年–1964年)[1]  
\item 美国物理学会,主席(1954年–1959年)[1]  
\item 学术出版社,编辑(1955年–1984年)[1]  
\item 北大西洋公约组织,职员(1959年–1960年)[1]  
\item 美国物理学会,主席(1961年)[1]  
\item 美国国家科学院,院长(1962年–1969年)[10]  
\item 洛克菲勒大学,荣誉校长(1968年–1978年)[1]  
\item 《固体物理学状态B》,编辑委员会成员[46]
\end{itemize}
私营部门
\begin{itemize}
\item 乔治·C·马歇尔研究所,共同创始人,主席(1984年–2001年)[20][21][22]  
\item 理查德·朗斯伯里基金会,主席(1995年–1997年),主席(自1998年起)[48][49]  
\item 科学与环境政策项目,主席(?–?)[50]  
\item 声音科学进步中心,顾问委员会成员[51]
\end{itemize}
\subsection{书籍}
\begin{itemize}
\item Frederick Seitz,《晶体学群的矩阵代数发展》,普林斯顿大学,1934年  
\item Frederick Seitz,《现代固体理论》,麦格劳-希尔,1940年  
\item Frederick Seitz,《金属物理学》,麦格劳-希尔,1943年  
\item Robert Jastrow, William Aaron Nierenberg, Frederick Seitz,《全球变暖:科学告诉我们什么?》,乔治·C·马歇尔研究所,1990年  
\item Robert Jastrow, William Aaron Nierenberg, Frederick Seitz,《温室效应问题的科学视角》,马歇尔出版社,1990年  
\item Frederick Seitz,《弗朗西斯·惠勒·卢米斯:1889年8月4日–1976年2月9日》,美国国家学会出版社,1991年  
\item Frederick Seitz (1994),《在前沿:我的科学人生》,美国物理学会出版社。ISBN 9781563961977。OCLC 231640410
\item Nikolaus Riehl 和 Frederick Seitz,《斯大林的囚徒:尼古拉斯·里尔与苏联的原子弹竞赛》(美国化学学会和化学遗产基金会,1996年),ISBN 0-8412-3310-1  
  本书是尼古拉斯·里尔的《十年金笼》(Riederer-Verlag,1988年)的翻译,Seitz撰写了长篇导言。书中包含58张照片。
\item Frederick Seitz 和 Norman G. Einspruch,《电子天才:硅的曲折历史》,伊利诺伊大学出版社,1998年
\item Frederick Seitz,《科学矩阵:旅程、艰辛与胜利》,施普林格,1998年
\item Frederick Seitz,《宇宙发明家雷金纳德·奥布里·费森登(1866–1932)》,美国哲学学会,1999年
\item Henry Ehrenreich, Frederick Seitz, David Turnbull, Frans Spaepen,《固态物理学》,学术出版社,2006年
\item Frederick Seitz,《美国国家科学院历史精选,1863–2005》,美国大学出版社,2007年
\end{itemize}
\subsection{另见}
\begin{itemize}
\item Seitz 对政府间气候变化专门委员会(IPCC)的批评:《IPCC 第二次评估报告》第8章:气候变化的探测与原因归属
\item 维格纳–塞茨半径
\item 维格纳–塞茨单元
\end{itemize}
\subsection{参考文献}
\begin{enumerate}
\item E. Goldwasser; A.V. Granato; R.O. Simmons (2008). "Frederick Seitz". *Physics Today*. 61 (7): 66–67. Bibcode:2008PhT....61g..66G. doi:10.1063/1.2963019.
\item "The 1950s in the Department of Physics at the University of Illinois at Urbana-Champaign". [已失效链接]
\item "The Marshall Institute – Founders". 归档自原文于2010-07-06. 取自2012-04-01.
\item Slichter, Charles P. (2010). "Frederick Seitz 1911–2008" (PDF). *Biographical Memoirs of the National Academy of Sciences*: 4.
\item Rockefeller University, *Biography of Frederick Seitz* 归档于2007-03-13通过Wayback Machine,1985年11月
\item *Current biography yearbook*, Volume 17, H.W. Wilson Company, 1957. p564
\item D. Hevesi (2008-03-06). "Frederick Seitz, 96, Dies; Physicist Who Led Skeptics of Global Warming". *The New York Times*. p. C12.
\item J.L. Bast. "Report #2 from the Global Warming Conference in New York City". 归档自原文于2008-03-14. 取自2008-03-04.
\item Seitz, Frederick (1934). *A matrix-algebraic development of the crystallographic groups* (Ph.D.). Princeton University. OCLC 82947413 – 通过ProQuest.
\item United States National Academy of Sciences, 2008年3月7日, *Past NAS President Frederick Seitz Dies at 96* 归档于2010-05-25通过Wayback Machine
\item Frederick Seitz, Norman G. Einspruch, *Electronic genie: the tangled history of silicon*. University of Illinois Press, 1998. pp128-9.
\item Seitz, Frederick (1981年1月26日). "Oral history interview transcript with Frederick Seitz" (采访). 由Lillian Hoddeson和Paul Henriksen采访. 美国马里兰州大学公园:美国物理学会,尼尔斯·玻尔图书馆与档案馆.
\item *The International Who's Who 2004*, Europa Publications
\item Stokes, Colin. "RJR's Support of Biomedical Research". *Tobacco Documents*. 归档自原文于2008-03-10.
\item Frederick Seitz, 1979年5月29日,向RJ Reynolds国际咨询委员会的报告 归档于2011-06-15通过Wayback Machine
\item Mark Hertsgaard (2006年5月). "While Washington Slept". *Vanity Fair*.
\item Dunlap, Riley E.; McCright, Aaron M. (2011). "Climate change denial: sources, actors, and strategies". 在Lever-Tracy, Constance (编辑). *Routledge Handbook of Climate Change and Society*. Taylor & Francis. p. 251. ISBN 9780203876213.
\item "Letter from Alexander Holtzman to Bill Murray". *Tobaccodocuments.org*. 1989年8月31日. 归档自原文于2014年3月1日. 取自2014年2月23日.
\item "The Marshall Institute – Remembering Frederick Seitz". *The Marshall Institute*. 2008年3月4日. 归档自原文于2011年1月28日. 取自2012年4月1日.
\item *The Independent Institute*, Research Fellow: Frederick Seitz. 取自2010年9月15日.
\item George C. Marshall Institute, "Untitled". 归档自原文于2001年12月14日。取自2001年12月14日。
\item Seitz, Frederick (2006年4月3日). "Interviews: Frederick Seitz" (采访). WGBH Educational Foundation. 取自2019年12月16日。
\item Naomi Oreskes 和 Erik M. Conway, 2010年8月10日, "Distorting Science While Invoking Science 归档自2010年9月19日于Wayback Machine", *Science Progress*。
\item *Daily Telegraph*, 2008年3月14日, Frederick Seitz。
\item 在《新闻周刊》关于全球变暖的封面报道中,研究所被形容为“否认机器的核心齿轮”——Begley, Sharon (2007年8月13日). "The Truth About Denial". *Newsweek*. 取自2007年10月17日。
\item George E. Marcus, *Paranoia within reason: a casebook on conspiracy as explanation*, University of Chicago Press, 1999. p.117。
\item "A Conversation with Dr. Frederick Seitz". *The George C. Marshall Institute*. 1997年9月3日。归档自原文于2010年7月6日。
\item Hirschhorn, Norbert; Aguinaga Bialous, Stella (2001). "Second hand smoke and risk assessment: what was in it for the tobacco industry?". *Tobacco Control*. 10 (4): 375–382. doi:10.1136/tc.10.4.375. PMC 1747615. PMID 11740031.
\item 根据《怀疑商人》(*Merchants of Doubt*),Seitz是气候变化否认的核心人物。
\item Hertsgaard, Mark (2006年5月). "While Washington Slept" *Vanity Fair*。
\item Seitz, Frederick; Jastrow, Robert (2001年12月1日). "Do people cause global warming?". *The Heartland Institute*. 归档自原文于2010年10月30日。取自2004年8月21日。
\item Arthur B. Robinson; Sallie L. Baliunas; Willie Soon; Zachary W. Robinson (1998年1月). "Environmental effects of increased atmospheric carbon dioxide". *OISM and the George C. Marshall Institute*. 归档自原文于2007年1月14日。取自2008年7月14日。
\item "Statement by the Council of the National Academy of Sciences regarding Global Change Petition" (新闻稿). *National Academy of Sciences*. 1998年4月20日。取自2018年12月23日。NAS理事会声明明确表示该请愿与美国国家科学院无关,且该手稿未在《美国国家科学院学报》或任何其他同行评审期刊上发表。
\item David Malakoff (1998年4月10日). "Climate change: Advocacy mailing draws fire". *Science*. 280 (5361): 195. Bibcode:1998Sci...280Q.195.. doi:10.1126/science.280.5361.195a. S2CID 152855137。
\item Oreskes, Naomi 和 Conway, Erik M. (2010). *Merchants of Doubt: How a Handful of Scientists Obscured the Truth on Issues from Tobacco Smoke to Global Warming*, Bloomsbury, 第25-29页。
\item Frederick Seitz; David Turnbull (1955). *Solid State Physics: Advances in Research and Applications*. 第1卷. 纽约: Academic Press. ISBN 978-0-08-086465-5. OCLC 646775088。
\item "Book Series: Solid State Physics". *www.elsevier.com*. 取自2021年2月12日。
\item "Statement by the Council of the National Academy of Sciences Regarding Global Change Petition". *美国国家科学院*. 1998年4月20日。取自2019年12月16日。
\item Brown, Joe. 700 Club主持人宣传气候变化怀疑论者的请愿书, reportedly signed by non-scientists, fictitious characters. *Media Matters*, 2006年2月14日。https://www.mediamatters.org/research/2006/02/14/700-club-anchor-touted-global-warming-skeptics/134878
\item Grandia, Kevin (2009年8月22日). "The 30,000 Global Warming Petition Is Easily-Debunked Propaganda". *The Huffington Post*. 更新于2017年12月6日。
\item "APS Member History". *search.amphilsoc.org*. 取自2023年3月22日。
\item "Frederick Seitz". *美国艺术与科学学院*. 2023年2月9日。取自2023年3月22日。
\item Rockefeller University, 2008年3月4日, Frederick Seitz – Lounsberry主管和前总统 – 逝世,享年96岁。归档自2011年5月20日。
\item "About Us". *World Cultural Council*. 取自2016年11月8日。
\item "Saxonburg Cyclotron 50th Reunion". 归档自2007年10月20日。
\item "Physica Status Solidi: Meet the Board Members". 归档自2013年1月5日。
\item "Richard Lounsbery Foundation".
\item Frederick Seitz, Richard Lounsbery基金会主席,参加1998年世界艺术与科学学院全球大会,温哥华 BC,加拿大。
\item "Remembering Frederick Seitz, 1911–2008". *Lounsbery Foundation*. 2008年3月4日。取自2020年5月9日。
\item *Science and Environmental Policy Project (SEPP)* 董事会成员,归档自2006年9月28日,取自2010年9月19日。
\item *New York Times*, 1998年4月26日,工业集团计划抗争气候条约。
\end{enumerate}