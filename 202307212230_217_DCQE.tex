% 延迟选择量子擦除实验
% 量子|干涉|擦除

\pentry{狄拉克符号\upref{braket},量子测量\upref{qmeas}}

如图\ref{fig:Experimental device}所示,左上角的光源发射光子,光子随后经过双缝。把红色路径标识的缝记为左缝,另一条则称为右缝。光子经过双缝后会进入$BBO$晶体。$BBO$将会吸收每一个入射的光子并发射出两个纠缠的光子。这两个光子经过棱镜后会分别射向不同的方向。我们把射向屏幕$D_0$的光子称为$A$光子,另一个称为$B$光子。$D_0$、$D_1$、$D_2$、$D_3$、$D_4$都是感光屏。$BS_a$、$BS_b$、$BS_c$是半透半反射镜,入射的光子有一半概率反射、一半概率透射。$M_a$、$M_b$是全反射镜。在经过$BS_c$后,从两条缝中射出的光子所产生的$B$光子的路径完全重合。从双缝到$D_1$、$D_2$、$D_3$、$D_4$光程远大于到$D_0$的光程,因此可以在观察到$D_0$上的现象后再对虚线框中的$B$光子观测系统进行调整。

在进行实验时,按如下的步骤操作:

(i)令光源每次只发射一个光子。

(ii)观察$D_0$屏幕上的光子落点,并记下另外四个屏幕中感光的屏幕编号。若$D_i(i=1,2,3,4)$屏幕感光,则将对应的光子落点记为$i$类点。重复进行多次。

(iii)只开放左狭缝,观察$D_0$屏幕上的光子落点,这类落点记为$L$类落点。重复进行多次。然后,在只开放右狭缝的情况下进行同样的工作,对应的落点记为$R$类落点。

实验完成后,将光子的落点画在图上,构成光强分布图。将$i(i=1,2,3,4,L,R)$类点的分布称为$i$类分布,所有光子落点的分布称为总分布。实验现象如下:

(i)总分布图中不存在干涉条纹。

(ii)$1$类分布与$2$类分布中存在干涉条纹。

(iii)$3$类分布与$4$类分布中不存在干涉条纹。

(iv)$3$类分布等于$R$类分布,$4$类分布等于$L$类分布。

(v)总分布等于$|\alpha|^2$倍的$L$类分布与$|\beta|^2$倍的$R$类分布之和。

\subsection{假设与约定}
为方便下面的讨论,这里做一些必要的简化,并对符号作出约定。

光子经过双缝后,其态矢量是确定的。显然,这个态矢可以写为如下的形式:
\begin{equation}
\ket{\Psi(t=0)} =\alpha\ket{\chi_L(t=0)}+\beta\ket{\chi_R(t=0)}~.
\end{equation}
其中$\ket{\chi_L(t=0)}$表示这样的一个态:$t=0$时刻,光子刚通过狭缝,此时对光子的位置进行测量,可以完全肯定它来自于左边的狭缝。若$t=0$时刻,从两边狭缝透射出的波函数没有任何重叠(即$\chi_L(t=0)\chi_R(t=0)$处处为零),这样的效果就可以达成。$\ket{\chi_R(t=0)}$的定义同理。所以,在光子刚经过狭缝时,对光子的位置进行测量,就可以完全肯定光子经过了哪条狭缝,且经过左边狭缝的概率为$|\alpha|^2$,经过右边狭缝的概率为$|\beta|^2$。这两个概率的和为$1$(这就保证了态矢量$\ket{\Psi}$的归一化),而它们的具体值取决于光源的位置和两条狭缝的宽度等参数,这里不讨论。通过以上的定义,还可以看出$\ket{\psi_L(t=0)}$正交于$\ket{\psi_R(t=0)}$,$\ket{\phi_L(t=0)}$正交于$\ket{\phi_R(t=0)}$,这是因为从两边狭缝透射出的波函数没有任何重叠。

接下来研究$BBO$晶体对态矢量的作用。简单起见,假设光子经过双缝后立刻受到$BBO$的作用。当一个光子射入$BBO$晶体时,将产生一对除自旋外各种属性和状态都相同的光子,且这对光子的属性和状态与入射的光子也相同。设$\mathscr{E}_A$为$A$光子的轨道态空间(即不考虑自旋的态空间),$\mathscr{E}_B$为$B$光子的轨道态空间,$\mathscr{E}_I$为入射光子的轨道态空间。经过$BBO$晶体后整个体系的轨道态空间即为$\mathscr{E}_A$和$\mathscr{E}_B$的张量积。为了保证属性相同,$\mathscr{E}_A$与、$\mathscr{E}_B$、$\mathscr{E}_I$互相自然同构。若入射光子的态矢量为$\ket{\chi}$,出射的$A$光子的态矢量为$\ket{\psi}$、出射的$B$光子的态矢量为$\ket{\phi}$,则$\ket{\chi}$、$\ket{\psi}$、$\ket{\phi}$应当在同构映射的意义下彼此相等(或相差一个无意义的相位因子)。从而,经过$BBO$晶体后,整个体系的态矢量应当为
\begin{equation}
\ket{\Psi(t=0)} =\alpha\ket{\psi_L(t=0)}\ket{\phi_L(t=0)}+\beta\ket{\psi_R(t=0)}\ket{\phi_R(t=0)}~.
\end{equation}
其中$\ket{\psi_L(t=0)}$是与$\ket{\chi_L(t=0)}$在同构映射的意义下相等的矢量,$\ket{\psi_R(t=0)}$、$\ket{\phi_L(t=0)}$、$\ket{\phi_R(t=0)}$的定义同理。
现在考虑体系的演化。在射出$BBO$晶体后,$A$光子和$B$光子将各自独立地演化,因此体系的哈密顿算符是这两者各自的哈密顿算符(的延伸算符)之和,即$H=H_A+H_B$。假定$\ket{\psi_L(t)}$是满足下列薛定谔方程的解:
\begin{equation}
\I\hbar\frac{\dd}{\dd t}\ket{\psi_L} =H_A\ket{\psi_L}~.
\end{equation}
该解的初态即为$\ket{\psi_L(t=0)}$。同理,对$\ket{\psi_R(t)}$、$\ket{\phi_L(t)}$、$\ket{\phi_R(t)}$也作同样的假定。于是,任意时刻整个体系的态可以写作
\begin{equation}
\ket{\Psi(t)} =\alpha\ket{\psi_L(t)}\ket{\phi_L(t)}+\beta\ket{\psi_R(t)}\ket{\phi_R(t)}~.
\end{equation}
这是因为这样定义的$\ket{\Psi(t)}$满足整个体系的薛定谔方程:
\begin{align}
&\I\hbar\frac{\dd}{\dd t}\ket{\Psi(t)} \notag \\
={}&\alpha((\I\hbar\frac{\dd}{\dd t}\ket{\psi_L(t)})\ket{\phi_L(t)}+\ket{\psi_L(t)}(\I\hbar\frac{\dd}{\dd t}\ket{\phi_L(t)})) \notag \\
&+\beta((\I \hbar\frac{\dd}{\dd t}\ket{\psi_R(t)})\ket{\phi_R(t)}+\ket{\psi_R(t)}(\I\hbar\frac{\dd}{\dd t}\ket{\phi_R(t)})) \notag \\
={}&\alpha(H_A\ket{\psi_L(t)}\ket{\phi_L(t)}+\ket{\psi_L(t)} H_B\ket{\phi_L(t)}) \notag \\
&+\beta(H_A\ket{\psi_R(t)}\ket{\phi_R(t)}+\ket{\psi_R(t)} H_B\ket{\phi_R(t)}) \notag \\
={}&H\ket{\Psi(t)}~.
\end{align}
哈密顿量各自独立,态矢量当然是互相独立地演化。这样体系的演化就得到了求解。为简化讨论,再对体系的演化提出以下假设:

(i)所有的态都是已归一化的。

(ii)$B$光子的波函数在$BS_c$之前的路径上是不重叠的,即$\phi_L(\boldsymbol{r_B},t)\phi_R(\boldsymbol{r_B},t)$在路径上恒为零。

(iii)$A$光子与$B$光子的波函数分布范围足够小,使得在屏幕后面的探测不到光子,也即不会出现光子从屏幕外面经过的情况。


屏幕对光子的作用相当于进行了一次位置的测量,这会导致态矢量发生跃变。假设在$t_A$时刻屏幕$D_0$感光,在$t_B$时刻$D_1$、$D_2$、$D_3$、$D_4$中的任意一个感光。若在$\boldsymbol{r_A}$处观测到了光子,则假设在$t=t_A^+$的时刻,$A$光子的态矢量跃变为$\ket{\psi'(t_A)}$,并且此后其随时间变化的规律为$\ket{\psi'(t)}(t>t_A)$。相应地,假设其对应的波函数为$\psi'(\boldsymbol{r_A},t)$。

\section{对实验结果的解释}
现在只看$D_0$上的落点分布而不关注另外四个感光屏上出现的现象,则$A$的位置分布是就是概率关于$\boldsymbol{r_A}$的边缘分布。联合概率分布为
\begin{align}
&\rho(\boldsymbol{r_A},\boldsymbol{r_B}) \notag \\
={}&|\alpha\psi_L(\boldsymbol{r_A},t_A)\phi_L(\boldsymbol{r_B},t_A)+\beta\psi_R(\boldsymbol{r_A},t_A)\phi_R(\boldsymbol{r_B},t_A)|^2 \notag \\
={}&|\alpha|^2|\psi_L(\boldsymbol{r_A},t_A)|^2|\phi_L(\boldsymbol{r_B},t_A)|^2 \notag \\
&+|\beta|^2|\psi_R(\boldsymbol{r_A},t_A)|^2|\phi_R(\boldsymbol{r_B},t_A)|^2 \notag \\
&+2Re(\alpha\beta^*\psi_L(\boldsymbol{r_A},t_A)\psi_R^*(\boldsymbol{r_A},t_A)\phi_L(\boldsymbol{r_B},t_A)\phi_R^*(\boldsymbol{r_B},t_A)).
\label{eq:联合概率分布}
\end{align}
对$\boldsymbol{r_B}$积分得到关于$\boldsymbol{r_A}$的边缘概率分布。考虑到$|\phi_L\rangle$和$|\phi_R\rangle$都是归一化的态,可得
\begin{align}
&\rho_A(\boldsymbol{r_A}) \notag \\
={}&\int \rho(\boldsymbol{r_A},\boldsymbol{r_B})d^3\boldsymbol{r_B} \notag \\
={}&|\alpha|^2|\psi_L(\boldsymbol{r_A},t_A)|^2+|\beta|^2|\psi_R(\boldsymbol{r_A},t_A)|^2 \notag \\
&+2Re(\alpha\beta^*\psi_L(\boldsymbol{r_A},t_A)\psi_R^*(\boldsymbol{r_A},t_A)\langle\phi_R(t_A)|\phi_L(t_A)\rangle).
\end{align}
注意到薛定谔方程是保内积的,即
\begin{align}
&i\hbar\frac{d}{dt}\langle\xi|\eta\rangle \notag \\
={}&i\hbar(\frac{d}{dt}\langle\xi|)|\eta\rangle+i\hbar\langle\xi|(\frac{d}{dt}|\eta\rangle) \notag \\
={}&-\langle\xi|H|\eta\rangle+\langle\xi|H|\eta\rangle \notag \\
={}&0.
\end{align}
而$|\phi_L\rangle$和$|\phi_R\rangle$两个态初始时就是正交的,于是得
\begin{equation}
\rho_A(\boldsymbol{r_A})=|\alpha|^2|\psi_L(\boldsymbol{r_A},t_A)|^2+|\beta|^2|\psi_R(\boldsymbol{r_A},t_A)|^2.
\label{eq:A边缘分布}
\end{equation}
于是实验现象$(i)$和$(v)$得到解释。

如果$D_3$或$D_4$感光,$D_0$上的光强分布函数就是$D_3$或$D_4$感光时$A$光子关于$\boldsymbol{r_A}$的条件概率分布。以$D_3$感光为例,记$D_3$屏幕的空间区域为$\Sigma(D_3)$。根据上面的假设,显然有$\phi_R(\boldsymbol{r_B}\in\Sigma(D_3),t_B)\neq 0$且$\phi_L(\boldsymbol{r_B}\in\Sigma(D_3),t_B)=0$。与式\ref{eq:A边缘分布}同理,光子$B$的边缘分布为$\rho_B(\boldsymbol{r_B})=|\alpha|^2|\phi_L(\boldsymbol{r_B},t_B)|^2+|\beta|^2|\phi_R(\boldsymbol{r_B},t_B)|^2$。于是在$D_3$感光的条件下,$A$光子的概率分布为:
\begin{align}
&\rho_{A|B}(\boldsymbol{r_A}|\boldsymbol{r_B}\in\Sigma(D_3)) \notag \\
={}&\frac{\int_{\Sigma(D_3)}\rho(\boldsymbol{r_A},\boldsymbol{r_B})d^3\boldsymbol{r_B}}{\int_{\Sigma(D_3)}\rho_B(\boldsymbol{r_B})d^3\boldsymbol{r_B}} \notag \\
={}&\frac{|\beta|^2|\psi_R(\boldsymbol{r_A},t_A)|^2\int_{\Sigma(D_3)}|\phi_R(\boldsymbol{r_B},t_A)|^2d^3\boldsymbol{r_B}}{|\beta|^2\int_{\Sigma(D_3)}|\phi_R(\boldsymbol{r_B},t_A)|^2d^3\boldsymbol{r_B}} \notag \\
={}&|\psi_R(\boldsymbol{r_A},t_A)|^2.
\end{align}
$D_4$感光的情况也是同理的。于是实验现象$(iii)$和$(iv)$得到解释。

如果$D_1$或$D_2$感光,$D_0$上的光强分布函数就是$D_1$或$D_2$感光时$A$光子关于$\boldsymbol{r_A}$的条件概率分布。以$D_1$感光为例,记$D_1$屏幕的空间区域为$\Sigma(D_1)$。根据上面的假设,显然有$\phi_L(\boldsymbol{r_B}\in\Sigma(D_1),t_B)\neq 0$和$\phi_R(\boldsymbol{r_B}\in\Sigma(D_1),t_B)\neq 0$。令
\begin{equation}
	\begin{cases}
	\mathscr{I}_1=\int_{\Sigma(D_1)}|\phi_L(\boldsymbol{r_B},t_B)|^2d^3\boldsymbol{r_B}; \\
	\mathscr{I}_2=\int_{\Sigma(D_1)}|\phi_R(\boldsymbol{r_B},t_B)|^2d^3\boldsymbol{r_B}; \\
	\mathscr{I}_3=\int_{\Sigma(D_1)}\phi_L(\boldsymbol{r_B},t_B)\phi_R^*(\boldsymbol{r_B},t_B)d^3\boldsymbol{r_B}.
	\end{cases}
\end{equation}
于是在$D_1$感光的条件下,$A$光子的概率分布为:
\begin{align}
&\rho_{A|B}(\boldsymbol{r_A}|\boldsymbol{r_B}\in\Sigma(D_1)) \notag \\
={}&\frac{\int_{\Sigma(D_1)}\rho(\boldsymbol{r_A},\boldsymbol{r_B})d^3\boldsymbol{r_B}}{\int_{\Sigma(D_1)}\rho_B(\boldsymbol{r_B})d^3\boldsymbol{r_B}} \notag \\
={}&\frac{|\alpha|^2\mathscr{I}_1|\psi_L(\boldsymbol{r_A},t_A)|^2+|\beta|^2\mathscr{I}_2|\psi_R(\boldsymbol{r_A},t_A)}{|\alpha|^2\mathscr{I}_1+|\beta|^2\mathscr{I}_2} \notag \\
&+\frac{2Re(\alpha\beta^*\mathscr{I}_3\psi_L(\boldsymbol{r_A},t_A)\psi_R^*(\boldsymbol{r_A},t_A))}{|\alpha|^2\mathscr{I}_1+|\beta|^2\mathscr{I}_2}.
\end{align}
虽然$\langle\psi_R|\psi_L\rangle=0$,也即$\psi_L\psi_R^*$在全空间的积分为零,但是一般而言$\mathscr{I}_3\neq 0$,因此干涉项存在。于是实验现象现象$(ii)$得到解释。
