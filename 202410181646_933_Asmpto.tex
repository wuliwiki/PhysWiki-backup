% 渐近线
% keys 渐近线|垂直渐近线|斜渐近线|水平渐近线|
% license Usr
% type Tutor

函数的渐近线(Asymptotes)主要分为以下几种类型:水平渐近线、垂直渐近线和斜渐近线。



\subsection{垂直渐近线}

\begin{definition}{垂直渐近线}
若函数$f(x)$满足$\displaystyle \lim_{x\to x_0}f(x)=\infty$,则称直线$x=x_0$是$f(x)$的\textbf{垂直渐近线(Vertical Asymptote,也称铅直渐近线)};
\end{definition}
一定是间断点。

\subsection{水平渐近线}

\begin{definition}{水平渐近线}
若函数$f(x)$满足$\displaystyle \lim_{x\to \infty}f(x)=y_0$,则称直线$y=y_0$是$f(x)$的\textbf{水平渐近线(Horizontal Asymptote)};
\end{definition}
注意,这里的无穷是包含正负两个方向的。
水平渐近线与函数  相关,其中  是  或  时的函数极限。对于函数 :

	•	当  或  时,计算  和 。
	•	如果极限是有限的数 ,那么  就是水平渐近线。

\subsection{斜渐近线}

\begin{definition}{斜渐近线}
若函数$f(x)$满足$\displaystyle \lim_{x\to \infty}f(x)-kx-b=0$,则称直线$y=kx+b$是$f(x)$的\textbf{斜渐近线(Oblique Asymptote)};
\end{definition}
注意,这里的无穷是包含正负两个方向的。
当函数没有水平渐近线但具有斜渐近线时,通常是因为函数的分子和分母的最高次数相差 1。

对于函数 :

	•	如果 ,则可能存在斜渐近线。
	•	使用多项式长除法将  表示为 ,其中  是渐近线方程。
	•	计算  如果结果趋近于零,则  为斜渐近线。

例子

对于函数 :

	1.	垂直渐近线:,所以  是垂直渐近线。
	2.	水平渐近线:计算 ,没有有限极限,所以没有水平渐近线。
	3.	斜渐近线:用多项式除法  可得斜渐近线方程。

\subsection{渐近线}
其实可以看出,水平渐近线是$k=0$时的特例。

任何函数$f(x)$都可以表示成$\displaystyle f(x)={P(x)\over Q(x)}$的形式,若无分母则可认为$Q(x)=1$。下面的讨论都在此基础上进行。
\begin{itemize}
\item 若函数满足$\displaystyle\lim_{x\to x_0}Q(x)=0$,且$\displaystyle\lim_{x\to x_0}P(x)\neq0$,则$x=x_0$是函数的垂直渐近线。
\item 若函数满足$\displaystyle\lim_{x\to x_0}Q(x)=0$,$\displaystyle\lim_{x\to x_0}P(x)=0$,且$\displaystyle \lim_{x\to x_0}f(x)=\infty$,则$x=x_0$是函数的垂直渐近线。
\end{itemize}