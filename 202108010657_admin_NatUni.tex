% 自然单位制 普朗克单位制
% 单位制|原子单位制|转换常数

\pentry{原子单位制\upref{AU}}

\footnote{参考 Wikipedia \href{https://en.wikipedia.org/wiki/Natural_units}{相关页面} 以及 \href{https://en.wikipedia.org/wiki/Planck_units}{普朗克单位制}.}在高能物理和场论中, 我们往往使用一套无量纲单位制\upref{NoUnit}, 使得普朗克常数 $\hbar = 1$ 真空光速 $c = 1$, 以及万有引力常数 $G = 1$. 这个单位制也称为\textbf{普朗克单位制(Plank units)}. 下面我们来进行说明, 首先给出和国际单位制\upref{SIunit}之间的转换常数:

\begin{table}[ht]
\caption{普朗克单位制换常数,括号表示最后两位的误差, 不带括号的是精确值. 参考 “物理学常数\upref{Consts}”.}\label{NatUni_tab1}
\begin{tabular}{|c|c|c|}
\hline
物理量 & 转换常数 $\beta$ & 数值(国际单位)\\
\hline
\dfracH 长度 $x$ & $\sqrt{\dfrac{\hbar G}{c^3}}$ & $1.616255(18)\e{-35}$ \\
\hline
质量 $m$ & $\sqrt{\dfrac{\hbar c}{G}}$ & $2.176434(24)\e{-8}$ \\
\hline
时间 $t$ & $\sqrt{\dfrac{\hbar G}{c^5}}$ & $5.391247(60)\e{-44}$ \\
\hline
\dfracH 速度 $v$ & $c$ & $299792458$ \\
\hline
力 $F$ & $\dfrac{c^4}{G}$ & $1.210256(27)\e{44}$ \\
\hline
\dfracH 能量 $E$ & $\sqrt{\dfrac{c^5\hbar}{G}}$ & $1.956082(22)\e9$ \\
\hline
角动量 $L$ & $\hbar$ & $1.054571817646156\ldots\e{-34}$ \\
\hline
\end{tabular}
\end{table}

\subsection{推导}
我们先来确定三个基本转换常数 $\beta_x, \beta_m, \beta_t$. 在 “原子单位制\upref{AU}” 的一开始, 为了使 $\beta_L = \hbar$ (也就是所谓的 “令 $\hbar = 1$”), 我们得到(\autoref{AU_eq6}~\upref{AU})
\begin{equation}\label{NatUni_eq1}
\beta_t = \frac{\beta_m \beta_x^2}{\hbar}
\end{equation}

现在为了让 $c = 1$, 即规定速度的转换常数为光速 $\beta_v = c$. 如果我们希望满足 $x = vt$, 那么必须有
\begin{equation}\label{NatUni_eq2}
\beta_x = \beta_v \beta _t = c\beta_t
\end{equation}
至此 $\beta_x, \beta_m, \beta_t$ 中只剩一个自由度.

为了确定力的量纲, 令牛顿定律形式不变 $F = ma$, 则
\begin{equation}
\beta_F = \frac{\beta_m \beta_x}{\beta_t^2}
\end{equation}
再令万有引力公式为(“令引力常数 $G = 1$”,见\autoref{NoUnit_ex2}~\upref{NoUnit})
\begin{equation}
F = \frac{m_1 m_2}{r^2}
\end{equation}
则有
\begin{equation}\label{NatUni_eq3}
\beta_x^3 = G \beta_m \beta_t^2
\end{equation}
联立\autoref{NatUni_eq1} ,\autoref{NatUni_eq2}  和\autoref{NatUni_eq3} 就可以求出 $\beta_x, \beta_m, \beta_t$(\autoref{NatUni_tab1} ). 另外也顺便定义了 $\beta_F$.

\subsection{另一种思路}
严格来说普朗克单位制中, 所有转换常数都使用 $c, G, \hbar$ 来定义. 例如\textbf{普朗克长度(Plank length)}为
\begin{equation}
\beta_x = \sqrt{\frac{\hbar G}{c^3}}
\end{equation}
这是 $\hbar, G, c$ 使用幂的乘积拼凑出长度量纲的唯一组合: 令 $\beta_x = \hbar^a G^b c^c$ 解出 $a, b, c$ 即可解出 $a = b = 1/2, c = -3/2$. 其他的物理量的转换常数也可以通过解 $a,b,c$ 的形式得到.

\subsection{电磁常数}

普朗克单位制并不规定电磁学常数, 但为了方便我们可以创造一套. 令以下公式成立
\begin{equation}
F = \frac{q_1q_2}{r^2}
\end{equation}
\begin{equation}
\bvec F = q(\bvec E + \bvec v\cross \bvec B )
\end{equation}
得
\addTODO{下表待验证}

\begin{table}[ht]
\caption{电磁学}\label{NatUni_tab2}
\begin{tabular}{|c|c|c|}
\hline
物理量 & 转换常数 $\beta$ & 数值(国际单位)\\
\hline
电荷 $q$ & $2\sqrt{\pi\epsilon_0 c\hbar}$ & $1.875546037779711\e{-18}$\\
\hline
\dfracH 电场强度 $E$ & $\dfrac{c^4}{2G\sqrt{\pi\epsilon_0 c\hbar}}$ & $6.4528\e{61}$ \\
\hline
\dfracH 磁感应强度 $B$ & $\dfrac{c^3}{2G\sqrt{\pi\epsilon_0 c\hbar}}$ & $2.15243\e{53}$\\
\hline
\end{tabular}
\end{table}
