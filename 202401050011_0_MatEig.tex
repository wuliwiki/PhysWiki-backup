% 矩阵的本征问题
% keys 矩阵|本征问题|本征矢|本征值|线性方程组|对角线
% license Xiao
% type Tutor

\pentry{线性方程组与矢量空间\upref{LinEq}}

若已知矩阵 $\mat A$, 我们把线性方程组
\begin{equation}\label{eq_MatEig_1}
\mat A \bvec v = \lambda \bvec v~.
\end{equation}
称为矩阵 $\mat A$ 的\textbf{本征方程}。 式中 $\mat A$ 是已知的, 而 $\lambda$ 和 $\bvec v$ 是未知的。 显然, 当 $\bvec v = \bvec 0$ 时方程恒成立, 所以我们通常只对非零解感兴趣。 也就是说, 我们希望找到一些\textbf{非零矢量} $\bvec v$, 使得矩阵 $\mat A$ 乘以该矢量以后方向不变\footnote{“方向” 只是从几何矢量\upref{GVec}中沿用过来的一个习惯说法, 注意\autoref{eq_MatEig_1} 中的所有量都可以是复数。 两个矢量方向相同意味着一个矢量乘以标量(包括复数)可以得到另一个。}。 对于每个这样的矢量, 我们用一个标量 $\lambda$ 来描述其模长的改变。 我们把这些矢量叫做\textbf{本征矢(eigen vector)}, 把对应的 $\lambda$ 叫做\textbf{本征值(eigen value)}。 一些教材也翻译成\textbf{特征矢}和\textbf{特征值}。 小时百科中, eigen 译作 “本征”, 而 characteristic 译作 “特征”。

\subsubsection{几何意义}
几何上来讲, 实数矩阵对应的线性变换相当于把坐标网格做旋转、拉伸、翻折等操作。% 链接未完成
所以一般而言, 一个非零矢量在变换后长度和方向都会改变。 但也可能存在一些特殊的非零矢量, 使得变换后只可能改变长度而不改变方向。 这些矢量就是本征方程的解。 注意这种几何理解仅适用于实数矩阵以及实数本征值和本征矢的解。

\subsection{求解本征方程}\label{sub_MatEig_1}

若令 $\mat I$ 为 $N\times N$ 的单位矩阵\footnote{即对角线上的元为 1, 其他元为 0, 见“矩阵\upref{Mat}”}, 则本征方程\autoref{eq_MatEig_1} 移项后得到一个齐次线性方程组\upref{LinEqu}
\begin{equation}\label{eq_MatEig_2}
(\mat A - \lambda\mat I)\bvec v = \bvec 0~.
\end{equation}
括号中的矩阵相当于把矩阵 $\mat A$ 的对角线上的元都减去 $\lambda$ 得到的方阵。 要确保方程有非零解, 只需令系数矩阵 $\mat A - \lambda\mat I$ 不是满秩\upref{MatRnk}的, 即行列式为零
\begin{equation}\label{eq_MatEig_4}
\abs{\mat A - \lambda\mat I} = 0~.
\end{equation}
这是一个关于 $\lambda$ 的 $N$ 阶多项式, 称为\textbf{特征多项式(characteristic polynomial)}。 特征多项式必存在 $N$ 个复数根(包括重根),% 链接未完成
记为 $\lambda_i$ ($i = 1, 2\dots N$)。 将它们依次代入\autoref{eq_MatEig_2}, 就可以分别解出对应的本征矢。 考虑到\autoref{eq_MatEig_2} 是一个齐次方程, 所以 $\mat A - \lambda_i\mat I$ 的零空间中所有矢量都是本征矢, 且零空间至少是一维的。 我们把这个空间叫做 $\lambda_i$ 的\textbf{本征矢空间}, 是 $\bvec v$ 所在的 $N$ 维矢量空间的子空间。 所以 1. 任何本征矢乘以非零常数都是本征矢; 2. 本征值相同的一组本征矢的任意线性组合都仍然是本征矢。

\addTODO{正文引用矩阵版本的零空间}

令 $\lambda_i$ 的本征矢空间的维度是 $n_i$, 若 $n_i = 1$, 我们说 $\lambda_i$ 是\textbf{非简并(non-degenerate)}的, 若 $n_i > 1$ 就说 $\lambda_i$ 是 $n_i$ 重\textbf{简并(degenerate)}的, 把 $n_i$ 叫做\textbf{简并数(degeneracy)}。

\begin{example}{二维矩阵的本征方程}
给出任意二维实数矩阵
\begin{equation}
\mat A = \pmat{a & b \\ c & d}~.
\end{equation}
要求它的本征值和本征矢, 其特正多项式(\autoref{eq_MatEig_4} )为
\begin{equation}
\vmat{a-\lambda & b \\ c & d-\lambda} = (\lambda-a)(\lambda-d) - bc = 0~.
\end{equation}
解二次方程得两个本征值为
\begin{equation}
\lambda_\pm = \frac{(a + d) \pm \sqrt{(a-d)^2 + 4bc}}{2}~.
\end{equation}
复数域中必定存在两个根, 包括重根。 若要求本征值为实数, 则需要另判别式(根号中的式子)大于零, 否则本征方程无解。

本征矢为
\begin{equation}
\bvec v_\pm = C\pmat{b\\ \lambda_\pm - a} = C\pmat{\lambda_\pm - d\\ c}~,
\end{equation}
其中 $C$ 是任意非零常数。 若两本征值相同, 则只存在一个一维的本征矢空间, 即一条直线。
\end{example}

\begin{theorem}{}
方阵 $\mat A$ 若存在两个不同的本征值 $\lambda_i, \lambda_j$, 那么它们所对应的本征矢 $\bvec v_i, \bvec v_j$ 不共线。
\end{theorem}
证明: 可以用反证法。 若共线, 则 $\bvec v_j = C \bvec v_i$, 带入 $\mat A \bvec v_j = \lambda_j \bvec v_j$ 后消去 $C$ 得 $\mat A \bvec v_i = \lambda_j \bvec v_i$, 这与 $\mat A \bvec v_i = \lambda_i \bvec v_i$ 矛盾。 证毕。

可见, 对于一般 $N$ 维方阵, 若特征多项式不存在重根, 则复数域中必有 $N$ 个线性无关的特征矢量, 每个构成一维特征子空间的基底。 若存在重根, 则线性无关的本征矢的个数小于等于 $N$ 个。

\begin{theorem}{一些简单性质}
若矩阵 $\mat A$ 的本征值和一组本征矢为 $\lambda_i,\ \bvec v_i$($i=1,\dots,N$), 则以下成立:
\begin{itemize}
\item 令 $c$ 为常数, 则 $c\mat A$ 的本征值和一组本征矢为 $c\lambda_i,\ \bvec v_i$($i=1,\dots,N$)。
\item 令 $\mat I$ 为同尺寸的单位矩阵, 那么 $\mat A + c\mat I$ 的本征值和一组本征矢为 $\lambda_i + c,\ \ \bvec v_i$。
\end{itemize}
\end{theorem}
根据\autoref{eq_MatEig_2} 易证。

\subsection{对角化与相似变换}
求解矩阵的本征方程的过程有时候也叫做矩阵的\textbf{对角化(diagonalization)}, 原因如下: 若 $N$ 维矩阵 $\mat A$ 存在 $N$ 个线性无关的本征矢(列矢量) $\bvec v_i$, 对应本征值 $\lambda_i$, 如果把本征值按顺序组成对角矩阵 $\mat \Lambda$, 把 $\bvec v_i$ 按顺序从左到右组成方阵 $\mat P$, 那么根据矩阵乘法\upref{Mat}的定义, $\mat A \mat P$ 相当于分别计算 $\mat A\bvec v_i$ 再从左到右排成方阵。 而 $\mat P\mat\Lambda$ 相当于把 $\lambda_i\bvec v_i$ 从左到右排成方阵。 二者应该相等, 所以有
\begin{equation}
\mat A \mat P = \mat P\mat\Lambda~.
\end{equation}
由于方阵 $\mat P$ 是满秩\upref{MatRnk}的(每列线性无关), 必定存在逆矩阵% 连接未完成
$\mat P^{-1}$。 两边右乘 $\mat P^{-1}$ 或者左乘 $\mat P$ 得
\begin{equation}\label{eq_MatEig_3}
\mat A = \mat P\mat\Lambda\mat P^{-1}, \qquad
\mat \Lambda = \mat P^{-1}\mat A \mat P~.
\end{equation}
这种从 $\mat A$ 和 $\mat\Lambda$ 之间的变换被称为\textbf{相似变换(similarity transform)}\upref{MatSim}。 如果能找到使 $\mat\Lambda$ 为对角矩阵的 $\mat P$ 就相当于解出了本征方程\autoref{eq_MatEig_1}, 这就是 “对角化” 名字的由来。
