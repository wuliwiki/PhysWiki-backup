% 阶乘(高中)
% keys 整数|gamma 函数|pi 符号
% license Xiao
% type Tutor

\begin{issues}
\issueDraft
\end{issues}

\pentry{\enref{求积符号(累乘)}{ProdSy}}{nod_fe64}

对自然数 $n$\footnote{学习微积分以后, 阶乘可以拓展到半整数甚至大部分实数, 见 “\enref{Gamma 函数}{Gamma}”(确切的说,Gamma 函数是阶乘的解析延拓,也就是我们通过找到一个函数使之满足原阶乘函数在各已知点处的取值而值域广于原来的阶乘函数,但不能简单的认为 Gamma 函数就是阶乘或者阶乘就是 Gamma 函数)。}, \textbf{阶乘(factorial)}定义为所有小于等于 $n$ 的正整数的乘积,即
\begin{equation}
n! := \prod_{i = 1}^n i =1 \cdot 2 \cdot 3 \dots (n - 2) (n - 1)n~.
\end{equation}
其中,$\prod_{i = 1}^n i$为\enref{求积符号}{ProdSy}。特殊地,约定
\begin{equation}
0! := 1~.
\end{equation}

\begin{example}{求5的阶乘}
小于等于5的正整数有:1,2,3,4,5,因此:
$$5! = 1\times 2\times 3\times 4\times 5 = 120.~$$
\end{example}





