% 等效原理
% keys 等效原理
% license Usr
% type Tutor

\footnote{ A.Zee,Einstein Gravity in a Nutshell.}等效原理是广义相对论的根基,它声称:在足够小的时空区域内,没有实验能够验证处于其中的物体是在引力场中还是加速参考系中。本节将通两个思维实验来理解Einstein是如何获得这一思想的。

\subsection{下落的盒子}
让我们思考这样一个愚人节恶作剧:趁我们其中一个朋友睡着的时候,把他放在一个进行精心设计的盒子里,这个盒子装饰得和这个伙计睡觉的地方一模一样。然后我们从很高的飞机上丢下这个盒子。

当我们的朋友醒来的时候,他认为他正处于他的房间里。由于他和他周围的所有东西都以盒子同样的速率向下加速,相对于他的四周来说,他感觉不到他在向下坠落。他轻轻的一跳,他发现他飘向了天花板。然而对于外面的观测者来说:我们的朋友,只不过是通过踩在地板上,降低了他的下落速度,同时增加了盒子的下落速度。他认为他是飘向天花板的,然而实际上他的坠落正在和之前一样的速率向下加速。

事实上,这一可怕且不道德的愚人节恶作剧已经被实验过了:我们的宇航员被放在一个叫做宇宙飞船的盒子里,然后从天空之外丢下它(宇航员返回地球)。人性化起见,总是给盒子一个向前的运动,以便坠入地面时和地面有个好的弯曲度感受。

为更详细的了解引力,然我们再次思考愚人节恶作剧。为了让这一恶作剧奏效,关键是要所有物体精确地以相同的速率下落。相反,若盒子比我们的朋友下落的快,那么我们的朋友将会发现自己被钉在天花板上,他可能会解释为存在一个力将他往上拉。若盒子比他下落的慢,则他会感到一个力把他拉在地面上。所有的物体以相同的速率下落而和它们的组成成分无关,这和日常经验是相反的,但是Galileo猜测我们的日常经验被空气阻力给扭曲了。

\subsubsection{引力的普遍性}
一个坠落的人不知道他是下落的,因为他周围的所有东西都以同样的速率下落,换句话说,因为引力的普遍性(所有东西都被引力往下拉,而无关他们的组成)。那么,我们可以反过来说吗?因为坠落的人不知道他是下落的,所以引力必须是普遍的。注意到在下落过程中,我们的朋友感受到自己是漂浮的,因此,某种程度上说,下落抵消了引力。那么假如我们用向上推来代替下落,我们可以产生引力吗?

\subsection{远离引力源的火箭}
为了理解Einstein的想法,让我们对我们的朋友开一个更加复杂的玩笑。这一次,趁他睡着,我们把他放进盒子里并让盒子飞向星际空间深处,远离所有的引力场。现在,加速发动机,并以恒定的速率加速装置。当他醒来后,没有发现任何异样,这一次没有漂浮感。

当他丢下苹果,苹果立刻掉在地上。然而对于漂浮在飞船外的观测者,看到的是飞船呼啸而过,下落的苹果实际上是漂浮在飞船里的(匀速运动),苹果完全没有意识到飞船正在以不断增加的速度冲向它。若我们以精确的加速度加速盒子,那么我们的朋友会看到苹果好像是在地球上以合适的速率下落。

通过加速的火箭盒子(实际上,逆转自由落体),我们可以产生引力。很明显,若我们的朋友在距离地板同样的高度落下石头和苹果,那么它们将在同一时刻“掉”在地板上。但是对他来说神秘的普遍性对于外面的观测者而言极其可笑地明显:地板向上移动,遇见了苹果和石头,所以同时和它们相遇。

所以,难道说,Pisa斜塔上下落的苹果和石头并没有掉下,而是一动不动的悬在空中,实际上是地面冲向了它们?这就解释了为什么苹果和石头是同时落地的。运动的相对性!

\subsubsection{"好像"已经够好了}
但是,这听起来完全是胡说八道。地球带着Pisa的斜塔和整个城镇冲向苹果和石头?怎么能用这种特殊的幻觉来解释引力呢?世界上所有人都在丢下东西,成熟的果实从树上掉下来,书呆子的物理学家被自己绊倒了,若真那样,地球必须要这样或那样的冲刺。

然而,地面冲向苹果和石头,对它们同时落地的现象是如此简单的解释。这其中一定有一些真实的因素。要在无意义之外制造意义,关键的见解是“好像”已经够好了。地面不需要匆忙上去,只要说,引力就表现得好像是地面在加速。我们可以通过称“好像”为“等价”来更学术地阐明这一点。


\subsection{等效原理}
现在,我们得出了Einstein的等效原理,这一原理声称:在足够小的时空区域内,没有实验能够验证处于其中的物体是在引力场中还是加速参考系中(或称,在足够小的时空区域,引力场等价于加速参考系)。

注意,“足够小的时空区域”的“足够小”是指一个区域小得可以和引力的特征尺度相比较。这是容易理解的,加入下落的苹果是在地球上而不是深空中,那么在下落的时候由于下落方向指向地心,因此苹果和石头会微微靠近,通过对此效应进行测量,实际上可以确定是处于地球的引力场还是加速的盒子里。

等效原理是对时空小区域物理的陈述。

\subsection{等效原理的预言}

\subsubsection{光线偏折}
现在xiang



