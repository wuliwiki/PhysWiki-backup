% 真因子树
% 环|因式分解|唯一析因环|因子|素理想|极大理想

真因子树的概念,是笔者优化了“因子链”的概念而得出的一套描述因式分解理论的框架.

\subsection{概念的描述}

\begin{definition}{真因子}
给定交换环$R$,对于$r, s\in R$,如果$s|r$且$r\not{|}s$,那么称$r$是$s$的真因子.
\end{definition}

\begin{definition}{单位}
给定交换环$R$,对于$u\in R$,如果$u^{-1}$是存在的,那么称$u$是$R$的一个\textbf{单位(unit)}.$R$中全体单位的集合,记为$U$.
\end{definition}

显然,如果有单位$u$使得$r=us$,那么$r$和$s$互相不是真因子.我们将这样的$r, s$视为等价的:

\begin{definition}{}
给定交换环$R$,定义集合$R$上的一个等价关系:对于$r, s\in R$,$r$等价于$s$当且仅当存在单位$u$使得$r=su$.等价的元素视为同一个元素,或者说把每个等价类看成一个元素,得到的集合是$R$模去该等价关系的商集\footnote{见\textbf{二元关系}\upref{Relat}.},记为$R_u$.
\end{definition}

有了$R_u$的概念,就可以定义本节核心的概念了:真因子树.

\begin{definition}{真因子树}
给定交换环$R$,对于$r\in R$,如果存在非单位的$a, b\in R$使得$ab=r$,那么可以从$r$画两个箭头分别指向$a$和$b$,而$(a, b)$就是$r$的一个因子分解;同样,如果$a$和$b$可以继续分解为其它非单位元素之积,那么也可以继续画出箭头指向它们对应的因子分解.如是反复,直到不能继续进行下去为止,所获得的整个结构称为$r$的一棵\textbf{真因子树}.

$r$的真因子树一般不止一棵.
\end{definition}

%要画图说明;说明等价关系是什么,以及真因子树长什么样.



