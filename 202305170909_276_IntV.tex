% 单变量矢量值函数的积分
% keys 微积分|矢量|导数|定积分|牛顿—莱布尼兹公式

% 介绍单变量的, 以及体积分(例如流体的动量)
\pentry{矢量的导数\upref{DerV},定积分\upref{DefInt}}% 未完成

\subsection{单变量不定积分}
令 $\bvec f(t)$ 为只有一个自变量的矢量函数, 则与标量函数类似, 定义其不定积分为求导的逆运算。 也就是说, 若能找到 $\bvec F(t)$, 使得 $\bvec F(t)$ 对 $t$ 求导就是 $\bvec f(t)$, 那么 $\bvec F(t) + \bvec C$ ($\bvec C$ 为任意常矢量) 就是定积分的结果, 都是 $\bvec f(t)$ 的原函数。

在直角坐标系中, 我们已经知道对矢量函数 $\bvec F(t)$ 求导就是对它的每个分量函数分别求导, 即
\begin{equation}\label{eq_IntV_1}
\bvec F'(t) = \bvec f(t)~.
\end{equation}
\begin{equation}\label{eq_IntV_2}
F'_x(t) = f_x(t) \qquad F'_y(t) = f_y(t) \qquad F'_z(t) = f_z(t)
\end{equation}
考虑到标量函数的不定积分是标量函数求导的逆运算, 所以对 $\bvec f(t)$ 不定积分, 只需对它的各个分量分别进行不定积分即可。 注意每个分量函数在不定积分后都会出现一个待定常数, 三个分量中的待定常数相加就得到一个待定常矢量 $\bvec C$。
\begin{equation}\ali{
\int \bvec f(t) \dd{t} &= \uvec x \int f_x(t) \dd{t} + \uvec y \int f_y(t) \dd{t} + \uvec z \int f_z(t) \dd{t}\\
&= [F_x(t)+C_x]\uvec x + [F_y(t)+C_y]\uvec y + [F_z(t)+C_z]\uvec z\\
&= \bvec F(t) + \bvec C
}\end{equation}
根据\autoref{eq_IntV_1} \autoref{eq_IntV_2}, 显然有 $[\bvec F(t) + \bvec C]' = \bvec f(t)$。

\subsection{单变量定积分}
类比一元标量函数定积分\upref{DefInt}的定义, 要计算一元矢量函数 $\bvec f(t)$ 从 $t_1$ 到 $t_2$ 的定积分, 就先把区间 $[t_1, t_2]$ 分为 $N$ 个小区间, 长度分别为 $\Delta t_i$, 且令 $t_i$ 为第 $i$ 个区间内的任意一点。 当我们取极限令所有区间长度 $\Delta t_i$ 都趋近于 $0$ (这时 $N\to\infty$) 时, 如果以下极限存在, 得到的矢量就是定积分的结果。
\begin{equation}
\int_{t_1}^{t_2} \bvec f(t) \dd{t} = \lim_{\Delta t_i \to 0} \sum_{i = 0}^N \bvec f(t_i) \Delta t_i
\end{equation}
唯一与标量函数的定积分不同的是, 这里的求和是矢量求和。 但在直角坐标系中, 我们可以把上式对矢量的求和表示成对各个分量分别求和, 而每个分量的极限就是一个标量定积分。 
\begin{equation}\ali{
\int_{t_1}^{t_2} \bvec f(t) \dd{t} &= \uvec x\lim_{\Delta t_i \to 0} \sum_{i = 0}^N f_x(t_i) \Delta t_i
+ \uvec y\lim_{\Delta t_i \to 0} \sum_{i = 0}^N f_y(t_i) \Delta t_i
+ \uvec y\lim_{\Delta t_i \to 0} \sum_{i = 0}^N f_z(t_i) \Delta t_i\\
&= \uvec x\int_{t_1}^{t_2} f_x(t) \dd{t} + \uvec y\int_{t_1}^{t_2} f_y(t) \dd{t} + \uvec z\int_{t_1}^{t_2} f_z(t) \dd{t}
}\end{equation}
所以 $\bvec f(t)$ 的定积分就是把直角坐标的各个分量分别进行定积分。现在对三个定积分分别运用牛顿—莱布尼兹公式\upref{NLeib}, $\bvec f(t)$ 的原函数为 $\bvec F(t)$, 各分量的原函数为 $F_x(t), F_y(t), F_z(t)$, 则上式等于
\begin{equation}\label{eq_IntV_6}\ali{
\int_{t_1}^{t_2} \bvec f(t) \dd{t} &= \uvec x [F_x(t_2) - F_x(t_1)] + \uvec y [F_y(t_2) - F_y(t_1)] + \uvec z [F_z(t_2) - F_z(t_1)]\\
&= \bvec F(t_2) - \bvec F(t_1)
}\end{equation}
这就是矢量函数的牛顿—莱布尼兹公式。

\begin{example}{加速度,速度和位移的积分关系}\label{ex_IntV_1}
由于质点的速度—时间函数 $\bvec v(t)$ 是位移—时间函数 $\bvec r(t)$ 的导函数, 后者就是前者的原函数。 所以根据牛顿—莱布尼兹公式\autoref{eq_IntV_6} 有
\begin{equation}
\bvec r(t) - \bvec r(t_0) = \int_{t_1}^{t_2} \bvec v(t) \dd{t}
\end{equation}
即
\begin{equation}
\bvec r(t) = \bvec r(t_0) + \int_{t_1}^{t_2} \bvec v(t) \dd{t}
\end{equation}
这是一维情况\autoref{eq_VnA1_5}~\upref{VnA1} 的拓展。

同理, 由于质点的加速度函数 $\bvec a(t)$ 是速度函数 $\bvec v(t)$ 的导函数, 后者可以通过前者定积分得到
\begin{equation}
\bvec v(t) = \bvec v(t_0) + \int_{t_1}^{t_2} \bvec a(t) \dd{t}
\end{equation}
\end{example}

\eentry{匀加速运动\upref{ConstA}}