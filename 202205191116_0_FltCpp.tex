% 浮点数测试(C++)

\begin{issues}
\issueDraft
\end{issues}

\pentry{双精度和变精度浮点数测试(Matlab)\upref{FltMat}}

double 和 “双精度和变精度浮点数测试(Matlab)\upref{FltMat}” 中的结果一样.

g++ 在 64bit 机器上, long double 一般是 64 bit 浮点 + 15 bit 指数, 十进制大约是 18-19 位有效数字. 然而 sizeof 却是 16 字节, 但实际上只使用了不到 10 字节.

\begin{lstlisting}
sizeof(Doub) = 8
sizeof(Ldoub) = 16

std::numeric_limits<Doub>::max() = 1.797693135e+308
std::numeric_limits<Ldoub>::max() = 1.189731495e+4932

std::numeric_limits<Doub>::min() = 2.225073859e-308
std::numeric_limits<Ldoub>::min() = 3.362103143e-4932

std::numeric_limits<Doub>::denorm_min() = 4.940656458e-324
std::numeric_limits<Ldoub>::denorm_min() = 3.645199532e-4951

std::numeric_limits<Doub>::epsilon() = 2.220446049e-16
std::numeric_limits<Ldoub>::epsilon() = 1.084202172e-19

std::numeric_limits<Doub>::digits10 = 15
std::numeric_limits<Ldoub>::digits10 = 18
\end{lstlisting}

然而 \verb|sqrt| 函数似乎并不会提高精度(仍然是 16 位):
\begin{lstlisting}
sqrt(2.)       = 1.4142135623730951454
sqrt(Ldoub(2)) = 1.4142135623730951454
precise val    = 1.4142135623730950488...
\end{lstlisting}


\subsection{四精度}
如果想完全利用 16 字节, 见 Boost\upref{Boost} 的 \verb|float128| 类型. 相当于 FORTRAN 的 \verb|QUAD real|. 有 112 bit 小数, 15 bit 指数和 1 bit 正负号(\autoref{FltCpp_fig1} ).

\begin{figure}[ht]
\centering
\includegraphics[width=10cm]{./figures/FltCpp_1.png}
\caption{见 Wikipedia }} \label{FltCpp_fig1}
\end{figure}

\begin{lstlisting}
FLT128_MAX = +1.18973149535723176509e+4932
FLT128_MIN = +3.36210314311209350626e-4932
FLT128_EPSILON = +1.92592994438723585306e-34 // 2e-112
FLT128_DENORM_MIN = +6.47517511943802511092e-4966
FLT128_MANT_DIG = +1.13000000000000000000e+02 // 包括正负位
FLT128_MAX_EXP = +1.63840000000000000000e+04 // 2e14
FLT128_DIG = +3.30000000000000000000e+01 // 33-34 位十进制有效数字
\end{lstlisting}

