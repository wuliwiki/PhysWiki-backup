% 氢原子的精细能级结构

\begin{issues}
\issueDraft
\end{issues}
讨论氢原子时,我们将哈密顿量取为:\begin{equation}
H=-\frac{h^2}{2m}\laplacian-\frac{e^2}{4\pi\epsilon_0 r}
\end{equation}

但是电子的动能加库仑势能之和并不是完整的内容.我们讨论过了对原子核运动的修正,也就是把$m$替换成约化质量.在我们研究氢原子时,还有个极为重要的现象,那就是由相对论效应修正和自旋-轨道耦合所带来的精细结构.比起数量级为$\alpha^2 mc^2$的波尔能量,精细结构是一个非常微小的扰动,其数量级为$\alpha^4 mc^2$,其中$\alpha$就是精细结构常数.
\begin{equation}
\alpha = \frac{e^2}{4\pi\epsilon_0\hbar c} \approx 0.0072973525693(11) \approx \frac{1}{137.036}
\end{equation}

\subsection{相对论修正}
哈密顿量的首项为动能:
\begin{equation}\label{HfineS_eq1}
T=\frac{1}{2}mv^2=\frac{p^2}{2m}
\end{equation}
动量$\mathbf p$的正则替换(canonical substitution)为$-\I\hbar\Nabla$,由此可得动能算符:
\begin{equation}
T=-\frac{\hbar^2}{2m}\laplacian
\end{equation}
不过,注意到\autoref{HfineS_eq1} 为经典动能的表达式;现在我们考虑相对论表达式:
\begin{equation}
T=\frac{mc^2}{\sqrt{1-(v/c)^2}}-mc^2
\end{equation}
其中的第一项为总的相对论能量,第二项为静能.那么两项的差就是动能.这里我们需要用到相对论的动量代替速度来表示动能$\mathbf T$
\begin{equation}
p=\frac{mv}{\sqrt{1-(v/c)^2}}
\end{equation}
由于
\begin{equation}
p^2c^2+m^2c^4=\frac{m^2v^2c^2+m^2c^4[1-(v/c)^2]}{1-(v/c)^2}=\frac{m^2c^4}{1-(v/c)^2}=(T+mc^2)^2
\end{equation}
因此
\begin{equation}
T=\sqrt{p^2c^2+m^2c^4}-mc^2
\end{equation}
我们将其从$p/mc$级数展开得到近似:
\begin{equation}
T = mc^2\left[\sqrt{1+\left(\frac{p}{mc}\right)^2}\right]=mc^2\left[1+\left(\frac{p}{mc}\right)^2\right]
\end{equation}



微扰项为
\begin{equation}
H'_{so} = \frac{e^2}{8\pi \epsilon_0 m^2 c^2} \frac{\bvec S \vdot \bvec L}{r^3}
\end{equation}
