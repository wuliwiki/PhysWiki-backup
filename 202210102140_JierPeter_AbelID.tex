% 阿贝尔微分方程恒等式
% 朗斯基行列式|Wronski行列式|Wronskian|线性微分方程|Abel's Identity|Abel's Formula|阿贝尔公式

本文翻译自WikiPedia的\href{https://en.wikipedia.org/wiki/Abel\%27s_identity}{Abel's Identity}.


\subsection{综述}

在数学中,\textbf{阿贝尔恒等式(Abel's identity)}(亦称\textbf{阿贝尔公式(Abel's formula)}\footnote{ Rainville, Earl David; Bedient, Phillip Edward (1969). \href{https://archive.org/details/elementarydiffer00rain}{ Elementary Differential Equations} . Collier-Macmillan International Editions.} 或者 \textbf{阿贝尔微分方程恒等式(Abel's differential equation identity)}),是一个等式,用于表示一个二阶齐次线性常微分方程的两个解的朗斯基行列式,只需要用到原方程的系数.这一关系也可以推广到$n$阶的线性微分方程.该恒等式命名自挪威数学家Niels Henrik Abel.

由于阿贝尔恒等式把微分方程不同的线性独立解联系起来了,因此它也可以用来从一个特解得到另一个特解.它给出了解之间很有用的恒等关系,同时也在参数变易法(variation of parameters)等其它技巧中功不可没.在Bessel方程等无法给出简单解析解的方程中尤其有用,因为这些情况下朗斯基行列式非常难算.

用\href{https://en.wikipedia.org/wiki/Liouville\%27s_formula}{Liouville}公式

A generalisation to first-order systems of homogeneous linear differential equations is given by Liouville's formula.







