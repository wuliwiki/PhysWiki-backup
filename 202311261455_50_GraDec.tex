% 梯度下降
% keys 梯度 下降 优化 更新
% license Xiao
% type Tutor

\textbf{梯度下降}(Gradient decent),或称\textbf{最速下降},是一种一阶优化算法,是神经网络中最基本最常用的优化方法。

梯度下降法搜索可能的权值假设空间,从而找到能够拟合训练样本的最佳权值 [1]。梯度下降法为误差反向传播算法提供了基础。而后者是多层神经网络训练的基础算法。

假设有一个没有阈值的线性神经元,其输入向量为$\bvec x$,输出为$y_o$,权值向量为$\bvec w$。正如基本的感知机训练算法一样,为了能够实现权值更新,要定义一个衡量神经元输出值与样本实际值之间误差,即训练误差(Training error)。通常可以采用最小二乘:
\begin{equation}
E=\frac{1}{2}(y-y_o)^2=\frac{1}{2}[y-y_o(\bvec x)]^2=\frac{1}{2}[y-y_o(\bvec w \bvec x)]^2~
\end{equation}



% \subsection{程序实践}


\subsubsection{参考文献}
\begin{enumerate}
\item T. M. Mitchell, Machine learning. 1997.
\item 周志华. 机器学习[M]. 北京:清华大学出版社. 2016: 97
\item I. Goodfellow, Y. Bengio, and A. Courville, Deep learning. MIT press, 2016: 174.
\end{enumerate}