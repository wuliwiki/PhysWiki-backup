% 北京航空航天大学2013年数据结构与C语言程序设计
% keys 北京航空航天大学 2013 数据结构 C语言程序设计 计算机 考研


考生注意:所有答题务必书写在考场提供的答题纸上,写在本试题单上的答题一律无效(本题单不参与阅卷).

\subsection{一、单项选择题}
(本题20分,每小题各2分)

1. 对于K度为n的线性表,建立其对应的单链表的时问复杂度为____. \\
A. $O(1)$ $\quad$ B $O(log_2n)$ $\quad$ C. $O(n)$ $\quad$ D. $O(n^2)$

2. 一般情况下.在一个双向链表中插一个新的链结点,____. \\
A.需要器改4个指针域自的指针 $\quad$ B.需要修改3个指针域自的指针 \\
C.需要修改2个指针域自的指针 $\quad$ D.只需要修改1个指针域自的指针

3. 假设用单十字母表示中糍表达式中的十运算数(或棒运算对象).井利用堆栈产生中缀表选式对应的后墒表选式.对于÷辍表达式A.B.<CfD-E)'j从左至右扫描到运算数E时-堆栈中的运算符依次是____.(注:不包古表达式的分界符、 \\
A. $+*/-$  $\quad$  B. $+*(/-$ $\quad$  C. $+*-$ $\quad$ D. $+*(-$

4. 若某二叉排序树的前序遍历序列为50,20,40,30,80,60,70.则后序遍历序列____. \\
A.30.40 20.50.70.60.80 $\quad$ B.30.40.20.W.60.80.50. \\
C.70.60.80.50.30.40.20 $\quad$ D.70.60.90.30.40.20.50.

5丹别H 6 3 8,12,5,7对应叶结点的权值构造的哈夫曼(Huffman)树的深度____. \\
A.6;    B 5:    C 4:    D 3.