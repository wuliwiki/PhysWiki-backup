% 曲线坐标系下的张量坐标变换(仿射空间)
% 坐标变换|张量|曲线坐标

\pentry{仿射空间中的曲线坐标系\upref{CFinAf},张量的坐标变换\upref{TrTnsr}}
在仿射空间中的曲线坐标系\upref{CFinAf}一节中,我们知道,在区域 $\Omega$ 中给定一个曲线坐标,就相当于在 $\Omega$ 中的每一点 $M$ 上给出了一个局部标架 $\{M;\partial_1 x,\cdots,\partial_n x\}$,其中 $x$ 是点 $M$ 的向径.在相当多的情形下,都是认为仿射空间取任意的曲线坐标 $x^i$ ,因而在每一点 $M$ 产生一个局部标架,于是点 $M$ 处的张量 $T(M)$ 的坐标,都是在这一标架下取的.这些坐标简单的叫作张量 $T(M)$ 在已知曲线坐标系 $x^i$ 中的坐标. 在 $\Omega$ 上每一点 $M$ 处给定一个张量 $T(M)$ 就叫作在 $\omega$ 上给定了一个张量场.本节将给出在两个曲线坐标系下的张量场的坐标变换规律.我们这里将继续遵守用张量坐标形式来代表张量本身,要还原张量本身只需加上基底即可.

\subsection{张量场的坐标变换规律}
由张量的坐标变换\upref{TrTnsr}知道,若基底 $\{e_i\}$ 到基底 $\{e'_i\}$ 的转换矩阵为 $A^i_j$,逆矩阵为 $B^i_j$,那么任意张量 $T$ 在这两基底下的坐标变换规律(见张量的坐标变换\upref{TrTnsr})为
\begin{equation}
T'^{j_1,\cdots,j_q}_{i_1,\cdots,i_p}=B^{j_1}_{k_1}\cdots B^{j_q}_{k_q}A^{l_1}_{i_1}\cdots A^{l_p}_{i_p}T^{k_1,\cdots,k_q}_{l_1,\cdots,l_p}
\end{equation}
对 $\Omega$ 上任一点 $M$,有(\autoref{CFinAf_sub1}~\upref{CFinAf})
\begin{equation}
A^i_j=\pdv{x^i}{x'^j}(M),\quad B^i_j=\pdv{x'^i}{x_j}(M)
\end{equation}
于是
\begin{equation}
T'^{j_1,\cdots,j_q}_{i_1,\cdots,i_p}(M)=\pdv{x'^{j_1}}{x_{k_1}}(M)\cdots \pdv{x'^{j_q}}{x^{k_q}}(M)\pdv{x^{l_1}}{x'^{i_1}}(M)\cdots \pdv{x^{l_p}}{x'^{i_p}}T^{k_1,\cdots,k_q}_{l_1,\cdots,l_p}(M)
\end{equation}
上式可记为
\begin{equation}
T'^{j_1,\cdots,j_q}_{i_1,\cdots,i_p}(M)=\qty(\pdv{x'^{j_1}}{x_{k_1}}\cdots \pdv{x'^{j_q}}{x^{k_q}}\pdv{x^{l_1}}{x'^{i_1}}\cdots \pdv{x^{l_p}}{x'^{i_p}}T^{k_1,\cdots,k_q}_{l_1,\cdots,l_p})(M)
\end{equation}



