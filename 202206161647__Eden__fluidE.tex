% 流体力学方程组
% 流体力学|守恒方程|方程组

\pentry{流体力学守恒方程\upref{fluidC},Navier-Stokes 方程\upref{NSeq}}

描述流体微元的物理量有 $\rho,p,T,\bvec u$,而完整描述它们随时间的演化需要流体力学方程组.它们分别是质量守恒方程(\autoref{fluidC_eq6}~\upref{fluidC})、动量守恒方程(\autoref{NSeq_eq2}~\upref{NSeq})、动能守恒方程(\autoref{fluidC_eq5}~\upref{fluidC})、内能守恒方程(\autoref{fluidC_eq7}~\upref{fluidC})、本构方程(\autoref{NSeq_eq1}~\upref{NSeq})和状态方程 $\rho=\rho(p,T)$.
\begin{equation}
\begin{aligned}
&\pdv{\rho}{t}+\nabla\cdot(\rho\bvec u)=0\\
&
\rho \dv{u_j}{t}=-\pdv{p}{x_j}+\rho g_j+\mu\pdv[2]{u_j}{x_i} +\qty(\mu_\nu+\frac{1}{3}\mu)\pdv{x_j}\pdv{u_m}{x_m}\\
&\rho \dv{}{t}\qty(\frac{1}{2}|\bvec u|^2)=\rho g_iu_i+u_j\pdv{T_{ij}}{x_i}=\rho g_iu_i+u_j\qty(-\pdv{p}{x_j}+\pdv{\tau_{ij}}{x_i})\\
&\rho \dv{e}{t}=-p\pdv{u_m}{x_m}+2\mu\qty(S_{ij}-\frac{1}{3}\pdv{u_m}{x_m}\delta_{ij})^2+\mu_\nu\qty(\pdv{u_m}{x_m})^2+\pdv{x_i}\qty(k\pdv{T}{x_i})
\\
&T_{ij}=-p\delta_{ij}+\tau_{ij}=-p\delta_{ij}+2\mu \qty(S_{ij}-\frac{2}{3}S_{mm}\delta_{ij})+\mu_\nu S_{mm} \delta_{ij}\\
&\rho=\rho(p,T)
\end{aligned}
\end{equation}
上面随时间演化的为前 $4$ 个方程(其中第二个方程是三分量的),因此将最后两个方程(状态方程和本构方程)代入后,总共可以得到 $6$ 个分量的独立的方程.