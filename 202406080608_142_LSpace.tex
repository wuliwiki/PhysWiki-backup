% 向量空间
% keys 线性空间|向量空间|线性代数|集合|交换律|结合律|分配率|多项式|线性相关|线性无关|基底|n维空间|内积
% license Xiao
% type Tutor

\begin{issues}
\issueTODO
\end{issues}

\pentry{函数\nref{nod_functi},几何向量的加法和数乘\nref{nod_GVecOp}}{nod_b44c}

\textbf{向量空间(vector space)} 也叫\textbf{矢量空间}或\textbf{线性空间(linear space)},是一种满足一定条件的非空\enref{集合}{Set},其元素叫做\textbf{向量}或\textbf{矢量}。 它必须满足, 在其中选择任意两个向量, 它们的线性组合(\autoref{def_LSpace_1} )仍然在这个空间中。 进行归纳后易得, 这个条件等价于 “任意有限个向量的线性组合仍然在这个空间中”(封闭性)。 

(二维、三维)欧几里得空间和列向量空间都是特别的向量空间,对应的向量为\enref{几何向量}{GVec}和\enref{列向量}{colVec}
% 这里的“向量”是\enref{几何向量}{GVec}的抽象,反过来,几何向量是向量的具象(特例)。
一个一般的向量,不一定具有长度和方向(几何向量的性质),也不一定有具体的分量(列向量的性质);任何“东西”都可以是向量,例如下面会看到函数也可以看作向量。

在小时百科中, 我们仅用粗体正体字母(如 $\bvec v$)表示几何向量或者矩阵(包括单行单列的矩阵), 而本文定义的一般向量空间的向量则使用与普通标量一样的字体(如 $v$)。 更多规则详见 “\enref{小时百科符号与规范}{Conven}”。
% Giacomo:应该是在《符号与规范》添加此条目的链接,而不是反过来。

\subsection{定义}
向量空间的定义必须依赖一个\enref{域(field)}{field} $\mathbb F$,即“域 $\mathbb{F}$ 上的向量空间”。 简单来说,域就是能进行加减乘除的对象的一个集合, 比如实数域 $\mathbb R$ 和复数域 $\mathbb C$。 这个域被称为该向量空间的\textbf{标量域(scalar field)}或\textbf{标域},它的元素被称为向量空间的\textbf{标量(scalar)},它们不是向量空间的元素,但是可以用来和向量进行数乘。我们接下来只会讨论\textbf{实数(或复数)域上的向量空间},所以即使我们不了解域的具体定义也没关系,只要知道实数和复数是两种常见的域就足够了。

\begin{definition}{向量空间}
标量域 $\mathbb F$ 上的向量空间定义了向量的集合 $X$、两个向量间的加法运算 $X\times X \to X$ 以及标量和向量间的数乘运算 $\mathbb F \times X \to X$。 令向量 $u,v,w \in X$, 标量 $a,b \in \mathbb F$, 向量加法记为 $u + v$, 数乘记为 $a u$; 两种运算必须满足以下性质:

\subsubsection{加法运算}
\begin{enumerate}
\item 满足加法\textbf{交换律} $u + v = v + u$。
\item 满足加法\textbf{结合律} $(u + v) + w = u + (v + w)$。
\item 存在\textbf{零向量} $0_X$,使得 $v + 0_X = v$。
\item 空间中任意向量 $v$ 存在\textbf{逆向量} $-v$,使得 $v + (-v) = 0_X$。
\end{enumerate}

\subsubsection{数乘运算}
\begin{enumerate}
\item 乘法\textbf{分配律}(向量) $a(u + v) = au + av$。
\item 乘法\textbf{分配律}(标量) $(a + b)v = av + bv$。
\item 乘法\textbf{结合律} $a (b v) = (ab) v$。
\end{enumerate}
\end{definition}

\textbf{说明}: 加法运算 $X \times X \to X$ 是一个二元映射(\autoref{sub_map_1}~\upref{map}), 注意运算的结果必须仍然落在 $X$ 中。 我们把这样的运算叫做\textbf{封闭(closed)}的\footnote{一些文献中也叫 “闭合”}。 数乘运算同样也是封闭的, 即一个向量数乘标量后仍然落在 $X$ 中。 我们现在还没有涉及 $X$ 以外的元素, 所以封闭性看似有些多余, 但以后会看到一个向量空间 $X$ 的子集 $X_1$ 也可以是向量空间, 称为\enref{子空间}{SubSpc}, $X_1$ 上的两种运算封闭意味着运算结果只能落在 $X_1$ 中而不能是 $X$ 的其他元素。 另外,我们要注意区分标量的零元 $0$ 和向量的零元 $0_X$。

\begin{corollary}{}
令 $a$ 为标量, $v$ 为向量, 则

1. $0\cdot v=0_X$。

$\quad$ 证明:$0\cdot v=(a-a)\cdot v=av+(-av)=0_X$.

2. $a \cdot 0_X=0_X$。

$\quad$ 证明:$a \cdot 0_X=a\cdot(v+(-v))=av+(-av)=0_X$.

3. $-1 \cdot v= - v$。

$\quad$ 证明:$v + (-1 \cdot v)= (1 - 1) \cdot v = 0_X$.

\end{corollary}

\begin{exercise}{几何向量}
证明 1,2,3 维空间中的所有\enref{几何向量}{GVec}各自构成实数域 $\mathbb R$ 上的向量空间。
\end{exercise}

作为一个非几何向量的例子, 我们来看由多项式构成向量空间。

\begin{example}{多项式}\label{ex_LSpace_1}
所有不大于 $n$ 阶的多项式 $c_n x^n + c_{n-1} x^{n-1} + \dots + c_1 x + c_0$ 可以构成一个实数或复数向量空间。定义向量加法为两多项式相加, 满足
\begin{itemize}
\item 封闭性:任意两个不大于 $n$ 阶的多项式相加仍然为不大于 $n$ 阶的多项式。
\item 交换律:多项式相加显然满足交换律。
\item 零向量:常数 0 可以看做一个 0 阶多项式, 任何多项式与之相加都不改变。
\item 逆向量:把任意多项式乘以 $-1$ 就得到它的逆向量, 任意多项式与其逆向量相加等于 0。
\end{itemize}
定义向量数乘为多项式乘以常数, 显然也满足数乘的各项要求, 不再赘述。
\end{example}

另一个重要的向量空间,是\textbf{函数空间(function space)}。

\begin{example}{函数空间}\label{ex_LSpace_2}
从任何集合 $A$ 到域 $\mathbb{F}$ 的全体函数的集合 $F := \{f \mid f: A \to \mathbb{F} \}$ 构成一个 $\mathbb{F}$ 上的线性空间,称为 $\mathbb{F}$ 在 $A$ 上的 \textbf{函数空间}。 函数空间中两个向量的加法定义为,对于任何 $x \in A$ 和函数(即向量)$f, g\in F$,有 $(f+g)(x)=f(x)+g(x)$;数乘定义为,对于任何标量 $a \in \mathbb{F}$, $x \in A$ 和函数 $f \in F$,有 $(af)(x)=af(x)$。

特别地,实数到实数、复数到复数、实数到复数等的函数都可以构成线性空间;把函数限制在连续函数、可导函数等条件下也依然构成线性空间。更特别地,复数域上的归一化可导函数,构成了复数域上的希尔伯特空间希尔伯特空间,这是一种无穷维的特殊向量空间,是量子力学的基础概念,我们将会在将来详细讨论。
\end{example}
\addTODO{链接:量子力学中的希尔伯特空间}

注意向量空间的定义并不需要包含内积(点乘) 的概念, 但我们可以在其基础上额外定义内积, 这样的空间叫做\enref{内积空间}{InerPd}, 留到以后介绍。 除了内积, 我们可以把 “\enref{几何向量的运算}{GVecOp}” 和 “\enref{线性相关性}{linDpe}” 中介绍的概念都拓展到一般的向量空间中, 这里不再重复。

\addTODO{下面这个例子应该移动到\enref{基(线性代数)}{VecSpn}}

\begin{exercise}{列向量}
我们把 $N$ 个复数 $c_1, \dots, c_N$ 按顺序排成一列(或一行, 下同), 叫做\textbf{列向量}(或\textbf{行向量}, 下同)。 列向量可以看成是 $N \times 1$ 的\enref{矩阵}{Mat}。给它们定义通常意义的加法和数乘运算, 这样所有列向量可以构成一个 $N$ 维向量空间。 注意由于我们使用了复数, 即使 $N \leqslant 3$ 时我们也无法将这些向量与几何向量对应起来。

如果我们将基底取为
$$
\pmat{1 \\ 0 \\ \vdots \\ 0}, \pmat{0 \\ 1 \\ \vdots \\ 0}, \dots, \pmat{0 \\ 0 \\ \vdots \\ 1}~
$$
那么显然任意列向量 $\pmat{c_1 \\ \vdots \\ c_N}$ 的坐标就是有序实数 $c_1, \dots, c_N$。 但我们也可以取其他基底, 这时坐标就会改变。 所以再次强调坐标和向量本身是不同的。我们将会在\enref{向量空间的表示}{VecRep}中详细区分向量本身和向量的坐标这两个概念。
\end{exercise}

\begin{exercise}{}
证明\autoref{ex_LSpace_1} 中多项式空间是 $n+1$ 维空间, $x^k$ ($k = 0, \dots, n$) 是一组基底(提示: 证明它们线性无关, 可以表示空间中的任意向量)。
\end{exercise}


\subsection{线性组合}

类比几何向量,我们也可以定义向量的线性组合:
\begin{definition}{线性组合}\label{def_LSpace_1}
对于向量空间 $V$,其中的两个向量 $v, w \in V$ 的线性组合为
$$
a v + b w ~,
$$
其中 $a, b \in \mathbb{F}$。
\end{definition}

