% 哥德巴赫猜想(综述)
% license CCBYSA3
% type Wiki

本文根据 CC-BY-SA 协议转载翻译自维基百科\href{https://en.wikipedia.org/wiki/Goldbach\%27s_conjecture}{相关文章}。

\begin{figure}[ht]
\centering
\includegraphics[width=6cm]{./figures/0b683c0c5cb7d58d.png}
\caption{哥德巴赫致欧拉的信,日期为1742年6月7日(拉丁语-德语)[1]} \label{fig_GDBHCX_1}
\end{figure}
哥德巴赫猜想是数论和整个数学中最古老且最著名的未解问题之一。它指出,所有大于2的偶数自然数都可以表示为两个质数的和。

尽管经过了大量的努力,这个猜想已经被证明对所有小于 \(4 \times 10^{18}\) 的整数成立,但仍未被完全证明。