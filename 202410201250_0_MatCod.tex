% Matlab Coder 代码生成笔记
% license Usr
% type Note

\begin{itemize}
\item 官方文档在\href{https://ch.mathworks.com/help/coder/index.html?s_tid=CRUX_lftnav}{这里}
\item 首先要安装 Matlab Coder 工具箱
\item 当前使用 Matlab 2024b 进行测试
\item 还需要独立编译器,Windows 上这里用 MinGW,在 Home 菜单中选 \verb`Add-Ons > Get Add-Ons` 即可。这里可以安装和卸载不同官方工具箱(无需安装包),也可以装 MinGW。按照提示下载安装即可。
\item 命令行运行 \verb`mex -setup`, \verb`mex -setup C++`, \verb`mex -setup FORTRAN`。会提示如 \verb`MEX configured to use 'MinGW64 Compiler (C++)' for C++ language compilation.`
\item 现在就可以在 \verb`APPS` 菜单里面搜索 \verb`Matlab Coder`,会有 GUI 引导进行代码生成。需要选择入口函数,入口函数参数类型。
\item 使用以下简单的单文件代码测试
\begin{lstlisting}[language=matlab,caption=test1.m]
% 代码生成的入口函数,arg* 限制为字符串,ret 限制为非负整数
function ret = test1(a, b)

% 输出
disp('hello, world');

% 修改入参
b = b+1;

% 简单表达式
c = a + b;

ret = "";
disp('a = '); ret = ret + 'a = ' + newline;
disp(a); ret = ret + sprintf('%0.5e',a) + newline;
disp('b = '); ret = ret + 'b = ' + newline;
disp(b); ret = ret + sprintf('%0.5e',b) + newline;
disp('c = '); ret = ret + 'c = ' + newline;
disp(c); ret = ret + sprintf('%0.5e',c) + newline;

% 判断
if c < 0
    d = b + sqrt(-c);
elseif c > 0
    d = a + sqrt(c);
else
    % 异常退出
    error('c = 0.');
end

disp('d = '); ret = ret + 'd = ' + newline;
disp(d); ret = ret + sprintf('%0.5e',d) + newline;

% 循环
for n = 1:5
    c = c + n;
end

disp('c = '); ret = ret + 'c = ' + newline;
disp(c); ret = ret + sprintf('%0.5e',c) + newline;

% 子函数调用
e = f1(a, b);

disp('e = '); ret = ret + 'e = ' + newline;
disp(e); ret = ret + sprintf('%0.5e',e) + newline;

% 正常退出
end

% 子函数
function z = f1(x, y)
z = x^2 + y^2;
end
\end{lstlisting}
\item 有几个问题要注意就是,代码生成是经过优化的。 例如如果不把 \verb`a, b` 作为参数而是直接在代码内赋值,它们很可能就会被优化掉,例如直接在 c++ 生成 \verb`c = 3`。
\item 非整数的 \verb`num2str()` 提示不支持生成代码
\item \verb`disp()` 似乎会被忽略,把每个中间变量 \verb`disp()` 似乎不能防止被优化,测试中的 \verb`sprintf()` 可以, 可以猜测 \verb`printf()` 也行。
\item 在生成路径下 \verb`mex` 路径是用于测试的编译好的 mex 文件 \verb`test1_mex.mexw64` 和编译所需的 cpp 和头文件。 \verb`lib` 路径是给终端用户生成的 cpp 和头文件。
\item \verb`mex` 模式下的 cpp 源码非常类似于我们的兼容模式,例如 \verb`disp()` 是支持的,其入参的确是 \verb`bxArray`,另外两个入参应该是 m 文件中的具体行号列号,以及调用栈信息? 
\end{itemize}
