% 位力定理
位力定理是质点组力学在统计上的一个应用。在保守系下,该定理展示了“长时间”后系统的动能平均值及势能平均值的关系。作为牛顿力学的推论,位力定理可用于热力学中玻意耳定律的证明,可用于大尺度星系质量的估算。经典力学和量子力学的关系如此密切,你也很容易猜到,位力定理必然也会“出现”于量子力学中。

(注:本文使用爱因斯坦求和约定,即$x_iy_i=\Sigma x_iy_i$。另,物理量头上一点代表对时间求导)

设n个质点组成一质点系,$G=\boldsymbol{r_i\cdot p_i}$,由链式法则我们有:

\begin{equation}
\frac{dG}{dt}=\boldsymbol{\dot{r}_i\cdot p_i}+\boldsymbol{r_i\cdot\dot{p}_i}=2T+\boldsymbol{r_i\cdot F_i}
\end{equation}
在统计上,某物理量$F$的时间平均值常被定义为$\overline{F}=\frac{1}{t}\int Fdt $。同理
\begin{equation}
\overline{\frac{dG}{dt}}=\frac{1}{t}\int \frac{dG}{dt}dt=2\overline{ T}+\overline{\boldsymbol{r_i\cdot F_i}}
\end{equation}

\begin{theorem}{位力定理}
对于 $\overline{\frac{dG}{dt}}=0,
\overline{ T}=-\frac{1}{2}\overline{\boldsymbol{r_i\cdot F_i}}$
\end{theorem}
由此可见,位力定理的应用关键在于找到一个
(此处待补充)
\begin{corollary}{势能为齐次线性函数}

\end{corollary}
\begin{exercise}{玻意耳定律的证明}

\end{exercise}
\subsubsection{星系质量估算}  