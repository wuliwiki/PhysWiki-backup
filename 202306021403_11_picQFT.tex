% 绘景变换与时间演化
\subsection{绘景选择}
为了简便地解决问题,我们可以根据力学量算符$F$和态矢$\ket{s}$的含时关系,选择不同的绘景:
\begin{equation}
\begin{aligned}
\text{薛定谔绘景}\frac{\partial \ket{s}}{\partial t}\neq 0 \quad \frac{\partial F}{\partial t}&= 0\\
\text{海森堡绘景}\frac{\partial \ket{s}}{\partial t}= 0 \quad \frac{\partial F}{\partial t}&\neq 0\\
\text{相互作用绘景}\frac{\partial \ket{s}}{\partial t}\neq 0 \quad \frac{\partial F}{\partial t}&\neq 0
\end{aligned}
\end{equation}
我们需要注意到两点基本事实,第一:绘景只是系统演化的不同图景,因而态矢及力学量的初始值必定是相同的。第二:如同表象变换,观测值也不随绘景的选择而改变。第三:由于时间演化算符并非力学量算符,所以在不同绘景里的变换关系需要额外推导。
\subsection{绘景变换}
设薛定谔绘景里的时间演化算符为$U_s$,那么某时刻的力学量期待值为$\bra{s}U^\dagger_sF U_s\ket{s}$。由于在海森堡绘景里,态矢不变,则$\ket{s,t}_H=\ket{s}=U^\dagger_s\ket{s,t}_s$,因此,任意时刻的海森堡力学量算符为$F_H=U^\dagger_sF U_s$
相互作用绘景,顾名思义,哈密顿量包含自由哈密顿量和相互作用哈密顿量,含时微扰可以在该绘景下处理

