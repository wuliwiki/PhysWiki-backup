% 晶体中电子在电场和磁场中的运动
% 晶体|电子|能带|有效质量

\pentry{近自由电子模型\upref{egasmd}}
在\textbf{单电子近似}下,利用近自由电子模型\upref{egasmd}或者紧束缚模型\upref{tbappx},可以有效地分析晶体中电子的行为.和自由电子模型相比,描述电子状态的波函数从平面波过渡到 Bloch (布洛赫)波,准连续的 $\epsilon(k)$ 也分裂为\textbf{能带}.电子在周期性势场中运动的本征态,对应的能量本征值 $\epsilon$,和其 Bloch 波函数的波数 $\bvec k$($\hbar \bvec k$也被称为准动量)有一个对应关系,被称为色散关系,画在 $\epsilon$-${\bvec k}$ 图上就呈现出一个曲面,由于周期性势场的影响这些曲线画在简约布里渊区中呈现出若干个能带.

可以说,能带反映了晶体中电子的几乎一切行为,可以用来解释金属、半导体、绝缘体的导电行为.因此,近自由电子模型或紧束缚模型虽然对晶体模型作了许多简化,但在解释晶体的性质方面取得了很大成功.利用能带,我们不仅可以知道晶体中电子的行为,还可以知道对晶体加外场(例如电场或磁场)时电子的变化,从而计算得到晶体对外场的响应.

如何分析加外场后电子的响应呢?一种方法是求解外加场的情况下薛定谔方程的能量本征解,即
\begin{equation}
\qty[-\frac{\hbar^2}{2m}\nabla^2+V(\bvec r)+U]\psi = E\psi
\end{equation}
另一种方法是将电子的运动近似当作经典粒子来处理.下面我们介绍的就是这种分析计算方法.
\subsection{晶体中电子的速度}
首先我们计算 Bloch 态 $\psi_{n\bvec k}(\bvec r)=e^{i\bvec k\cdot \bvec r}u_{n\bvec k}(\bvec r)$ 的速度期待值.其中 $n$ 对应能带的标号.我们有
\begin{equation}\label{cryele_eq1}
\begin{aligned}
\bvec v_n(\bvec k)&=\frac{1}{m}\bra{\psi_{n\bvec k}}\hat p\ket{\psi_{n\bvec k}}=\frac{1}{m}\int \dd[3]{\bvec x}\psi_{n\bvec k}^*(\bvec x)(-i\hbar\nabla)\psi_{n\bvec k}(\bvec x)\\
&=\frac{1}{m}\int \dd[3]{\bvec x}u_{n\bvec k}^*(\bvec x)(-i\hbar\nabla+\hbar \bvec k)u_{n\bvec k}(\bvec x)
\end{aligned}
\end{equation}
而 Bloch 态的能量可以表达为
\begin{equation}\label{cryele_eq2}
\begin{aligned}
\epsilon_{n}(\bvec k)&=\bra{\psi_{n\bvec k}} \qty[-\frac{\hbar^2}{2m}\nabla^2+V]\ket{\psi_{n\bvec k}}
\\&=\frac{1}{m}\int \dd[3]{\bvec x} u_{n\bvec k}^*(\bvec x)(\hbar^2k^2/2-\hbar^2\nabla^2/2-i\hbar\bvec k\cdot \nabla)u_{n\bvec k}(\bvec x)
\end{aligned}
\end{equation}
对比\autoref{cryele_eq1} 和\autoref{cryele_eq2} 可以看到
\begin{equation}\label{cryele_eq3}
\bvec v_n(\bvec k)=\frac{1}{\hbar}\nabla_{\bvec k}\epsilon_{n}(\bvec k)
\end{equation}
这表明 Bloch 态电子的速度期待值就等于能带在 $\bvec k$ 处的斜率.我们也可以通过将 $\bvec k$ 邻域内的 Bloch 态作光滑叠加得到局域的波包,并且计算波包中心的速度得到上述结果,具体推导可以参考群速度\upref{GroupV}. 从上面的分析我们可以得到以下结论:能带的带底和带顶电子速度为零,这是因为它对应能带的极值点,斜率一定为 $0$.另一方面由以上结果可以知道,因为一个 Bloch 态是定态,所以晶体中的电子在无外场的情况下平均速度永远保持不变;这意味着,尽管有电子与离子实之间相互作用的周期性势场,一个理想晶体金属将具有无穷大的电导.然而事实上,任何的晶体都不具有无穷大电导,这是因为晶体结构上的不理想性(存在杂质或缺陷),而且晶格有热振动,电子总会发生散射.
\subsection{电子运动的准经典模型}
现在我们希望讨论的是,在两次散射之间电子的行为,尤其是在外场作用下电子的行为.为了讨论这个问题,准经典模型是非常有用的.如果对晶体中电子施加一个外力(这个外力来自于外场) $\bvec F$,那么在 $\dd t$ 时间内外力对电子做功为
\begin{equation}
\bvec F\cdot \bvec v_n(\bvec k)\dd t
\end{equation}
电子的能量也要有相应的变化.设 $\bvec k$ 变化 $\dd{\bvec k}$根据功能原理可以得到
\begin{equation}
\bvec F\cdot \bvec v_n(\bvec k)\dd t
=\dd {\bvec k}\cdot (\nabla_{\bvec k}\epsilon_n(\bvec k))
\end{equation}
代入\autoref{cryele_eq3} 可以得到
\begin{equation}
\qty(\bvec F-\hbar\dv{\bvec k}{t})\cdot \bvec v_n(\bvec k)=0
\end{equation}
我们称 $\hbar \bvec k$ 为准动量,则上式表明准动量守恒在平行 $\bvec v_n(\bvec k)$ 方向上是成立的.事实上可以进一步证明垂直 $\bvec v_n(\bvec k)$ 方向上也成立,即
\begin{equation}
\dv{\hbar \bvec k}{t}=\bvec F
\end{equation}
