% 苏州大学 2005 年硕士考试试题
% keys 苏州大学|考研|物理|2005年
% license Copy
% type Tutor

\begin{enumerate}
\item 质量为 $M$,半径为$r$的均匀圆柱体放在粗糙的水平面上,柱的外围绕有轻绳,绳子绕过一个很轻的滑轮,并悬挂一质量为$m $的物体。设圆柱只滚不滑,并且圆柱体与滑轮间的绳子是水平的,求圆柱体质心加速度$ a_1$、物体的加速度$ a_2$、及绳中张力T。
\begin{figure}[ht]
\centering
\includegraphics[width=6cm]{./figures/821de792fe1174ee.png}
\caption{} \label{fig_SD05_1}
\end{figure}
\item 如图所示,长为$l$的质细杆,一端悬于O点,自由下垂。在O点同时悬一单摆,摆长也是$l$,摆的质量为 $m$,单摆从水平位置由静止开始自由下摆,与自由下垂的细杆作完全弹性碰撞,碰撞后单摆恰好静止。求:\\
(1)细棒的质量 $M;$\\
(2)细棒摆动的最大角度$\theta$。
\begin{figure}[ht]
\centering
\includegraphics[width=6cm]{./figures/f59c66a130406144.png}
\caption{} \label{fig_SD05_2}
\end{figure}
\item 一弹簧振子作简偕振动,振幅 $A=0.20m$,如果弹簧的劲度系数 $k=2.0N/m$所系物体的质量 $m=0.50kg$,试求:\\
(1)当动能和势能相等时,物体的位移是多少?\\
(2)设$t=0$时,物体在正最大位移处,达到动能和势能相等处所需的时间是多少(在一个周期内)?
\item 已知一沿$x$轴正方向传播的平面余弦波在$t=\frac{1}{3}$秒时的波形如图所示,且周期$T=2S$。求\\
(1)写出O点和P点的振动表达式。\\
(2)写出该波的波动表达式。\\
(3)求P 点离 Os点的距离。
\begin{figure}[ht]
\centering
\includegraphics[width=8cm]{./figures/c138c9384e352412.png}
\caption{} \label{fig_SD05_3}
\end{figure}
\item 图中球形区域$a<r<b$,设其体密度为$\rho=A/r$,在封闭空腔的中心($r=0$)有一个电量为Q的点电荷。证明:当A$=Q/2\pi a^2$ 时,球形区域($a<r<b$)中的电场具有恒定值。
\begin{figure}[ht]
\centering
\includegraphics[width=6cm]{./figures/153df1ebfd53fb83.png}
\caption{} \label{fig_SD05_4}
\end{figure}
\item 一平行板电容器有两层介质,$\varepsilon_{r_1}=4,\varepsilon_{r_2}=2$,厚度为$ d_1=2.0mm$,$d_2=3.0mm$,极板面积 $S=40cm^2$,两极板电压为 $200V$。计算:\\
(1)每层介质中的电场能量密度;\\
(2)每层介质中的总能量;\\
(3)用公式$\frac{1}{2}qU$计算电容器的总能量。
\begin{figure}[ht]
\centering
\includegraphics[width=6cm]{./figures/ca7510eaa2df27f8.png}
\caption{} \label{fig_SD05_5}
\end{figure}
\item 半径为$R$的平面圆形线圈中载有电流$I_2$,另一无限长直线 $AB $中载有电流$I_1$,设 $AB$ 通过圆心,并和圆形线圈在同一平面内(如图),求圆形线圈所受的磁力。
\item 在半径为$R$的圆柱形体积内充满磁感应强度为$B$的均匀磁场,有一长为$l$的金属棒放在磁场中,如图所示。设 $dB/dt $为已知,求棒两端的电势差。
\item 如图所示,沉积在玻璃衬底上的氧化钽层从$A$到$B$的厚度递减到零,从而形成一尖劈,为测定氧化钽薄层的厚度$t$,用波长为$632.8nm$ 的He-Ne 激光垂直照射到薄层上,观察到楔形部分共出现11条暗纹,且$A$处恰好为一暗纹位置,已知氧化胆的折射率$n_1$=2.21,玻璃折射率$n_2$=1.5,求氧化胆薄层的厚度$t$。
\end{enumerate}