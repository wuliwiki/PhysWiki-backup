% 贝叶斯公式
% 概率论

\pentry{条件概率 \upref{HsCpMi},从集合论角度看随机事件\upref{JNran1}}
\subsection{前言}
贝叶斯公式可常常用于概率模型中的统计推断。在开始讨论贝叶斯公式前,我们需要定义几个概念。
\begin{definition}{划分}
设S为一个样本空间,我们定义对一个样本空间的划分为一系列的事件$A_1,A_2,A_3...,A_n$,这些事件对于$\forall i,j $ 且$i\neq j$都满足$A_i \cap A_j = \emptyset$,且对于他们的并集满足$A_1\cup A_2\cup ...\cup A_n = S$。
\end{definition}
划分是对样本空间进行的两两互不相交的切割,当我们有了划分后,对于任意一个事件$B$发生的概率,就可以看做是在每个划分中$B$事件发生的概率再乘以每个划分发生的概率。
\begin{theorem}{全概率公式}
设$A_1,A_2,...A_n$是样本空间$S$的一个划分,则任意一个事件$B$发生的概率可以写为$P(B) = \sum_i P(B A_i)= \sum_i P(B| A_i) P(A_i)$
\end{theorem}
证明如下,我们可以将事件B作一个切分,$B = B \cap S = B \cap (A_1 \cup A_2 \cup ...\cup A_n)$,根据集合运算的分配率,我们有
\begin{aligned}
B = (B\cap A_1) \cup (B\cap A_2) \cup ... \cup (B \cap A_n)~
\end{aligned}
由于$\{A_i\}$是$S$的划分,有$A_i \cap A_j = \emptyset$,而 $(B\cap A_i) \cap (B\cap A_j) \subseteq A_i \cap A_j$,于是我们知道$(B\cap A_i) \cap (B \cap A_j) = \emptyset$。根据概率与条件概率的定义,我们有
\begin{align}
P(B) = \sum_i P(B\cap A_i)  = \sum_i P(B | A_i) P(A_i)~
\end{align}
\subsection{贝叶斯公式}
\begin{theorem}{贝叶斯公式}
设$S$为样本空间,$A_1,A_2,...A_n$为样本空间的一个划分,$B$为一个事件,则有
\begin{equation}
P(A_i|B) = \frac{P(B|A_i)P(A_i)}{P(B)} = \frac{P(B|A_i)P(A_i)}{\sum_j P(B|A_j)P(A_j)}~
\end{equation}
\end{theorem}
贝叶斯公式证明如下:根据条件概率公式的定义$P(A|B) = P(AB)/P(B)$,$P(B|A) = P(BA)/P(A)$,我们显然有$AB = A\cap B = BA$,于是对比得到一个等式
\begin{equation}
P(A|B)P(B) = P(B|A)P(A)~
\end{equation}
或
\begin{equation}
P(A|B) = \frac{P(B|A)P(A)}{P(B)}~
\end{equation}
将上述等式做代换$A\rightarrow A_i$即可得到贝叶斯公式
\begin{equation}
P(A_i|B) = \frac{P(B|A_i)P(A_i)}{P(B)}~
\end{equation}
再代入刚刚给出的全概率公式即可得到贝叶斯公式中第二个等式。
\subsection{贝叶斯公式的应用}
我们先来看一下贝叶斯公式说了什么,条件概率的表达式$P(A|B)$中,事件$B$作为前提,事件$A$作为结果,给出了在前提事件下作为结果的事件发生的概率,因此我们说$P(A|B)$描述了两个事件之间的因果关系强度。而贝叶斯公式的另一侧将前提事件与结果事件翻转过来了,这种翻转使得我们能够形成推断。具体讨论前我们先看例子
\begin{example}{疾病原因的推断}
有一种疾病A,它的发病率为 2 \% ,通过临床数据的统计,我们发现患者中有70\%是男性,有百分之30\%是女性,问一个男性患有疾病A的概率是多大?
\end{example}
