% 实二次型
% 实二次型|惯性定律|惯性指数

\begin{issues}
\issueDraft
\end{issues}

\pentry{二次型的规范型\upref{GuaOQu}}
\subsection{实二次型}
在二次型的规范型\upref{GuaOQu}一节里,已经知道,在任意域 $\mathbb F$ 上的二次型都具有规范型(或对角型)\autoref{QuaFor_def1}~\upref{QuaFor}.一般来说,规范型是二次型最简单的形式.但是,在域为实数域,即 $\mathbb F=\mathbb R$ 时,可以让规范型\autoref{GuaOQu_eq1}~\upref{GuaOQu}
\begin{equation}
q(\bvec{x})=\lambda_1 x_1^2+\cdots+\lambda_r x_r^2
\end{equation}
的所有系数 $\lambda_i$ 均为 $\pm 1$.
\begin{definition}{实二次型}
若二次型 $q$ 所配备的矢量空间 $V$ 定义在实数域 $\mathbb R$ 上,则 $q$ 称为\textbf{实二次型}.
\end{definition}
适当置换基底矢量,可认为前 $s$ 个系数 $\lambda_1,\cdots,\lambda_s$ 是正的,而其余的系数是负的.进行替换
\begin{equation}
\begin{aligned}
&x'_i=\sqrt{\lambda_i}x_i,\;&1\leq i\leq s;\\
&x'_i=\sqrt{-\lambda_i}x_i,\;&s+1\leq i\leq r;\\
&x'_i=x_i,\; &r+1\leq i\leq n
\end{aligned}
\end{equation}
 即得
 \begin{equation}
 q(\bvec{x})=\sum_{i=1}^{s}{x'_i}^2-\sum_{i=s+1}^{r} {x'_{i}}^{2}.
 \end{equation}
 而若对于有理数域 $\mathbb Q$,在 $\sqrt{\lambda_i}$ 为无理数时并不能作这样的简化.
 
\begin{definition}{标准型}
称可以按公式
\begin{equation}\label{RQuaF_eq1}
q(\bvec x)=\sum_{i=1}^{s}{x_i}^2-\sum_{i=s+1}^{r} {x_{i}}^{2}
\end{equation}
计算值的二次型有\textbf{标准型}.
\end{definition}
由上面的讨论立刻得
\begin{theorem}{}
实矢量空间 $V$ 上的所有二次型 $q$ 均可化为标准型.
\end{theorem}
\subsection{惯性定理}
\begin{theorem}{惯性定理}
实二次型 $q$ 的标准型\autoref{RQuaF_eq1} 中的整数 $r$ 和 $s$,$s\leq r\leq n$ 仅依赖于 $q$,即与规范基底的选择无关.
\end{theorem}
\textbf{证明:} 由于 $r$ 不变,故只需证明 $s$ 不变.

设另有一基底 $(\bvec e'_1,\cdots,\bvec e'_n)$,在其上 $q$ 具有标准型
\begin{equation}\label{RQuaF_eq2}
q(\bvec x)=\sum_{i=1}^{t}{x'_i}^2-\sum_{i=t+1}^{r} {x'_{i}}^{2}
\end{equation}
不是一般性,令 $t<s$.
对子空间
\begin{equation}
L=\langle \bvec e_1,\cdots,\bvec e_s\rangle_\mathbb{R},\quad L'=\langle \bvec e'_{t+1},\cdots,\bvec e'_n\rangle_\mathbb{R}
\end{equation}
因为 $\mathrm{dim}\;(L+L')\leq s+n-t\leq n=$,那么
\begin{equation}
\begin{aligned}
\mathrm{dim}\;(L\cap L')&=\mathrm{dim}\; L+\mathrm{dim}\; L'-\mathrm{dim}\;(L+L')\\
&\geq s+(n-t)-n=s-t> 0
\end{aligned}
\end{equation}
可见,存在一非零矢量 $\bvec a\in(L\cap L')$:
\begin{equation}
\bvec 0\neq a=\sum_{i=1}^s a_i\bvec e_i=\sum_{i=t+1}^n a'_i \bvec e'_i
\end{equation}
由\autoref{RQuaF_eq1} 
\begin{equation}\label{RQuaF_eq3}
q(\bvec a)=\sum_{i=1}^s a_i^2>0
\end{equation}
由\autoref{RQuaF_eq2} 
\begin{equation}\label{RQuaF_eq4}
q(\bvec a)=-\sum_{i=t+1}^n a_i^2\leq 0
\end{equation}
\autoref{RQuaF_eq3} \autoref{RQuaF_eq4} 联立得出矛盾.因此,只能是 $s=t$.

\textbf{证毕!}
\begin{definition}{惯性指数}
称实二次型的秩为\textbf{惯性指数},数 $s$ 为\textbf{正惯性指数}, $r-s$ 为\textbf{负惯性指数}.
\end{definition}

\textbf{注意},对于复二次型,即 $\mathbb F=\mathbb C$ 的情形,正负惯性指数失去ren'he'yi'yi