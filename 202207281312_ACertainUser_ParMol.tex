% 偏摩尔量
\pentry{态函数\upref{statef}}
\footnote{本文参考了朱文涛的《简明物理化学》}

\subsection{偏摩尔量}
假设一个多元单相系统的某一广度状态量$Z=f(p,T,n_1,n_2,...)$

对其求全微分
\begin{equation}\label{ParMol_eq1}
dZ = \left(\pdv{Z}{p}\right)_{T,n_1,n_2,...}dp + \left(\pdv{Z}{T}\right)_{p,n_1,n_2,...}dT+\left(\pdv{Z}{n_1}\right)_{p,T,n_2,...}dn_1+\left(\pdv{Z}{n_2}\right)_{p,T,n_1,n_3,...}dn_2+...
\end{equation}

定义偏摩尔量
\begin{equation}
{Z_B} = \left(\pdv{Z}{n_B}\right)_{p,T,n_i, i \neq B} 
\end{equation}
那么 \autoref{ParMol_eq1} 化为 
\begin{equation}
dZ = \left(\pdv{Z}{p}\right)_{T,n_1,n_2,...}dp + \left(\pdv{Z}{T}\right)_{p,n_1,n_2,...}dT+\sum {Z_B} dn_B
\end{equation}
特别地,如果温度、压力一定,
\begin{equation}\label{ParMol_eq3}
dZ = \sum {Z_B} dn_B
\end{equation}

${Z_B} d n_B$ 的物理含义可以理解为,某一状态下,再往系统中加入少量物质B,系统状态Z的变化.

注意,B组分偏摩尔量${Z_B}={Z_B}(p,T,n_1,n_2,...)$一般不是常数,也不等于纯B物质的摩尔量.例如,往水中融入体积为V的少量NaCl晶体后,水体积的改变量将小于V.

\subsection{偏摩尔量常用公式}
\subsubsection{集合公式}
\begin{equation}\label{ParMol_eq2}
Z=\sum {Z_B}  n_B
\end{equation}

对于有些溶液,溶质的${V_B}n_B$的值是负数(即溶解后,溶液体积反而变小),而显然物质的体积不能是负数.由此可见,集合公式\textbf{不能}简单地理解为总量等于各组分“\textsl{占有的份量}”之和.

\subsubsection{Gibbs-Duhem 公式}
\begin{equation}
\sum n_B \dd Z_B = 0
\end{equation}

推导:对集合公式(\autoref{ParMol_eq2} )两端求导,$dZ=\sum n_B \dd Z_B + \sum {Z_B}  \dd n_B$,代入 \autoref{ParMol_eq3} ,对比,即得证.

特别地,对于二元混合物,
\begin{equation}
n_1 \dd {Z_1} = - n_2 \dd {Z_2}
\end{equation}
或
\begin{equation}
\dd {Z_1} = -\frac{n_2}{n_1} \dd {Z_2}
\end{equation}

