% 导数(高中)
% keys 导数|高中|变化|求导
% license Xiao
% type Tutor

\begin{issues}
\issueDraft
\end{issues}

在学习函数的性质时,曾经提到过函数的\aref{变化率}{sub_HsFunC_4}。其中,\textbf{平均变化率(average rate of change)}的概念类似于计算某段时间内的平均速度,反映了函数在一个区间内的整体变化趋势。当平均变化率所涉及的两个点逐渐靠近,直至几乎重合时,这一变化率便转化为描述局部变化的工具。此时,连线逐渐成为该点处的切线,而平均变化率也演变为\textbf{瞬时变化率(instantaneous rate of change)},即函数在该点的变化速率。瞬时变化率的另一个名称是\textbf{导数(derivative)}。

平均变化率提供了宏观的变化趋势,而导数则通过精确的数学方法刻画了局部的瞬时变化。导数的应用范围极其广泛,几乎所有涉及变化的领域都能发现它的踪迹。例如,在经济学中,导数用于分析股票价格的涨跌;在气象学中,它可以测量温度的变化速度;在生物学中,它帮助研究细胞分裂的速率。作为研究变化的强大工具,导数为人们提供了一种新的思维方式,帮助深入理解和处理动态问题。

导数的理论基础依赖于\textbf{极限(limit)},但由于高中阶段未涉及极限的具体内容,因此高中的导数的学习主要聚焦于以下三个方面:
\begin{enumerate}
\item 理解背景和实际意义:掌握导数的几何意义和实际应用场景,理解它与函数其他性质的关系;
\item 熟练计算:学习导数的常见计算规则,能够快速、准确地对复杂函数求导;
\item 解决实际问题:利用导数分析函数性质,在面对恒成立、不等关系等问题时,知道如何构造辅助函数来使用导数解决问题。
\end{enumerate}

导数不仅是理解函数变化的核心工具,也是未来微积分学习的重要基础。高中阶段对导数的熟练掌握,将为进一步研究函数极限与积分打下坚实的基础。

\subsection{理解导数}

在谈论导数前,不妨先回忆一下生活中常见的情景:汽车仪表盘上的速度显示。假如一个人开车从家去商场,出发后司机可能会注意到仪表盘上显示的速度有时是50公里/小时,有时是80公里/小时——这是汽车在某一瞬间的速度,也叫瞬时速度。

如果我们不借助仪表盘,只知道整个行程花了30分钟,行驶了15公里,那我们可以说这段旅程的平均速度是15公里÷0.5小时,也就是30公里/小时。这是对整个行程的一个大致描述,告诉我们平均下来,汽车每小时走了多少公里。

但实际情况呢?在不同的路段,汽车可能需要减速、加速,甚至短暂停车。我们会发现,平均速度虽然简单,却忽略了这些瞬间的变化。而瞬时速度的意义在于:它告诉我们汽车在某个具体时刻的速度,比如在某条平直的路上刚刚加速后达到80公里/小时时,这一刻的速度就是它的瞬时速度。

导数的本质,就是用来研究这些瞬时的变化。它帮助我们回答一个更具体的问题:在某一时刻,事物的变化速度是多少?就像仪表盘上显示的数字,背后其实是导数在起作用。通过导数,我们可以更深入地理解事物的变化过程,而不仅仅停留在整体的平均趋势上。

\subsection{导数的定义}
一点的\textbf{导数}
\begin{equation}
f'(x_0)=\lim_{x_1\to x_0}{f(x_1)-f(x_0)\over x_1-x_0}~.
\end{equation}

导数也是一个对应关系,即每个自变量都对应一个导数,因此他也是一个函数,这个函数称为\textbf{导函数}(不引起歧义时,简称为导数)。导函数和原本的函数是一一对应的,因此可以根据定义或求导方法,来求一个函数的导函数,这个过程就是\textbf{求导}。

\subsubsection{导数的记号}
由于历史发展和人们长久以来的使用习惯,导函数逐渐衍生出了许多不同的记法。这些记法不仅仅是使用者的偏好选择,还与特定领域的需求和表达习惯密切相关。既为方便计算和推导,也为强调不同的数学概念。了解这些符号的使用,有助于理解求导这个运算,另外在未来见到时,也不至陌生,不要求完全掌握,看个眼熟就好。下面的符号针对函数$y=f(x)$:
\begin{itemize}
\item 拉格朗日记法——$y'$或$f'(x)$,好处是记法比较简洁,便于书写,缺点是难以表达较为复杂的关系。高中数学主要采用这种表示法。
\item 莱布尼茨记法——$\displaystyle\frac{\dd y}{\dd x}$  或  $\displaystyle\frac{\dd}{\dd x}f(x)$,好处是在进行某些复杂运算时,分子与分母可以直接按照乘除法的规则来进行运算,降低推导的复杂度。另外,也在形式上代表着变化率。在大学阶段的数学领域主要采用这种表示方法。
\item 牛顿记法——$\dot{y}$。由于在物理学中,时间是一个较为特别的变量,一般用这种方法来表示某个变量相对于时间的导数。基本只在物理学领域使用。
\item 重导数记法——$f_x$,这种记法简洁紧凑,又在出现复杂关系时,避免了拉格朗日记法的问题。在偏微分方程和张量分析中常用。
\item 欧拉记法——$Df(x)$,采用$D$运算符。主要在大学阶段的微分方程中使用,好处是$D^n f(x)$可以直接修改$n$来表示进行几次求导运算。另外,将导数视为一种运算符,便于与其他运算符进行组合,适合处理复杂的微分运算。
\item 差分导数——$\Delta f(x)$,对于定义域是离散的函数(一般是$\mathbb{Z}$),通常会用这样的符号来表示它的导数,称为\textbf{差分}。
\end{itemize}

在高中阶段,一般只要求使用拉格朗日记法,且不允许使用其他记法。其余的记法可以这样理解:
\begin{itemize}
\item 用$D$运算符代替$\displaystyle\frac{\dd}{\dd x}f(x)$中的$\displaystyle\frac{\dd}{\dd x}$就成了$Df(x)$;
\item 把$\displaystyle\frac{\dd}{\dd x}f(x)$中最重要的两部分$f,x$拿出来显示就成了$f_x$;
\item 在明确自变量的情况下,只强调$\displaystyle\frac{\dd y}{\dd x}$中的$y$就成了$y'$;
\item $\dot{y}$与$y'$异曲同工,只是更着眼于“时间”;
\item $\Delta f(x)$与$Df(x)$只是因为定义域不同,处理方法不同;
\end{itemize}

\subsection{求导法则}

为记录方便,下面记$u=f(x),v=g(x),u'=f'(x),v'=g'(x)$。

\begin{itemize}
\item 加减法:$(u\pm v)'=u'\pm v'$
\item 乘法:$(uv)'=u'v+uv'$
\item 除法:$\displaystyle\left(\frac{u}{v}\right)'=\frac{u'v-uv'}{v^2}$
\item 复合函数:$(f(v))'=f'(v)v'$
\end{itemize}

\subsection{基本初等函数的导数推导}

\subsection{对照表}

这里将常见的函数与导数对照表列出如下,方便查询。具体介绍需查看每个函数自己的页面。

\begin{table}[ht]
\centering
\caption{高中常见函数及其导数}\label{tab_HsDerv1}
\begin{tabular}{|c|c|c|}
\hline
\textbf{函数名称}     & \textbf{函数 $f(x)$}     & \textbf{导函数 $f'(x)$}     \\ \hline
幂函数&$x^n$                    & $n x^{n-1}$                \\ \hline
反比例函数&$\displaystyle\frac{1}{x}$             & $\displaystyle-\frac{1}{x^2}$           \\ \hline
指数函数(e为底)&$e^x$                     & $e^x$                      \\ \hline
对数函数(e为底)&$\ln(x)$                  & $\displaystyle\frac{1}{x}$              \\ \hline
指数函数&$a^x$                     & $a^x\ln a $                      \\ \hline
对数函数&$\log_a(x)$                  & $\displaystyle \frac{1}{x\ln a}$              \\ \hline
正弦函数&$\sin(x)$                 & $\cos(x)$                  \\ \hline
余弦函数&$\cos(x)$                 & $-\sin(x)$                 \\ \hline
正切函数&$\tan(x)$                 & $\displaystyle \frac{1}{\cos^2(x)}$                \\ \hline
\end{tabular}
\end{table}

