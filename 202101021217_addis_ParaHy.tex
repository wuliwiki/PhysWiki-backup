% 抛物线坐标系中的类氢原子定态波函数

\begin{issues}
\issueDraft
\end{issues}

\pentry{定态薛定谔方程\upref{SchEq}, 抛物线坐标系\upref{ParaCr}}

\footnote{参考 \cite{Brandsen} Chap 3.5.}本文使用原子单位制\upref{AU}. 令原子核不动.

\begin{equation}
-\frac{1}{2m} \laplacian \psi - \frac{Z}{r} \psi = E\psi
\end{equation}
在抛物线坐标系中变为
\begin{equation}
-\frac{1}{2m} \qty{\frac{4}{\xi + \eta} \qty[\pdv{u}{\xi}\qty(\xi\pdv{\xi}) + \pdv{u}{\eta}\qty(\eta\pdv{\eta})] + \frac{1}{\xi\eta}\pdv[2]{u}{\phi}}\psi - \frac{2Z}{\xi + \eta}\psi = E\psi
\end{equation}
分离变量, 令
\begin{equation}
\psi(\xi, \eta, \phi) = f(\xi) g(\eta) \Phi(\phi)
\end{equation}
和球坐标同理, $\Phi(\phi) = \exp(\I m \phi)$.
