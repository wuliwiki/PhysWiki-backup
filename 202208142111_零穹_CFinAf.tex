% 仿射空间中的曲线坐标系
% 曲线坐标系|仿射空间

\pentry{仿射空间\upref{AfSp}}
本节采用度规张量与指标升降(欧氏空间)\upref{TofEuc}中的约定,即:
设 $\{\dot o;e_1,\cdots,e_n\}$ 是仿射空间 $\mathbb A$ 中的坐标系.则对其上任一点 $M$ ,其坐标为矢量 $\overrightarrow{OM}=x^i e_i$ 的坐标(采取坐标和基矢指标的对立约定\upref{CofTen}及爱因斯坦求和约定\upref{EinSum}).若选择一新坐标系 $\{\dot o',e'_1,\cdots,e'_n\}$,则点的坐标变换规则为(\autoref{AfSp_the3}~\upref{AfSp})
\begin{equation}
x'^i=B^i_jx^j-B^i_j b^j
\end{equation}
其中,$B^i_j$ 是基 $\{e_1,\cdots,e_n\}$ 到基 $\{e'_1,\cdots,e'_n\}$ 的转换矩阵 $A^i_j$ 的逆矩阵.而 $b^i$ 是 $\dot o'$ 在旧坐标系 $\{\dot o;e_1,\cdots,e_n\}$  中的坐标.