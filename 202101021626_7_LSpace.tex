% 矢量空间
% keys 线性空间|向量空间|线性代数|矢量|几何矢量|矢量空间|集合|交换律|结合律|分配率|多项式|线性相关|线性无关|基底|n维空间|行矢量|列矢量|子空间|内积


\begin{issues}
\issueTODO
\issueOther{修改进行中,在取舍重复内容.}
\end{issues}

\pentry{几何矢量\upref{GVec}, 函数\upref{functi}, 抽象\upref{Abstra}}

\textbf{矢量空间(vector space)} 也叫\textbf{向量空间}或\textbf{线性空间(linear space)},是一种满足一定条件的集合\upref{Set}, 有无穷多个元素, 每个元素叫做一个\textbf{矢量(vector)}或\textbf{向量}. 它必须满足, 在其中选择任意两个矢量, 它们的线性组合仍然在这个空间中. 进行归纳后易得, 这个条件等价于 “任意有限个矢量的线性组合仍然在这个空间中”(封闭性). 这里的 “矢量” 是一个广义的概念, 是几何矢量\upref{GVec}的抽象\upref{Abstra}; 反过来,几何矢量是矢量的具象. 一个广义的矢量,不一定具有长度和方向, 例如下面会看到函数也可以看作矢量.

矢量和向量都是术语vector的翻译,其译名的来历参见几何矢量\upref{GVec}.今后我们不区分这两个译名,请注意.

\subsection{定义}
矢量空间的定义必须依赖一个\textbf{域}\footnote{见 “域\upref{field}”, 简单来说,域就是能进行加减乘除的一个集合.},比如实数域 $\mathbb R$ 和复数域 $\mathbb C$. 这个域本身被称为该矢量空间的\textbf{标量域(scalar field)}或\textbf{标域},它的元素被称为矢量空间的\textbf{标量(scalar)},它们不是矢量空间的元素,但是可以用来和矢量进行数乘. 那么, 我们就说矢量空间是\textbf{域} $\mathbb{F}$ \textbf{上的}. 通常选择的域就是 $\mathbb{R}$ 或 $\mathbb{C}$, 即讨论的是\textbf{实数(或复数)域上的矢量空间}.

任意矢量空间内必须定义两个矢量的\textbf{加法(addition)}(用 “+” 表示)和标量与矢量之间的\textbf{数乘(scalar multiplication)} 两种运算, 得到的结果也必须在同一空间中. 我们把这样的运算叫做\textbf{封闭(closed)}的\footnote{一些文献中也叫 “闭合”}. 两种运算必须满足如下性质, 其中 $\bvec u,\bvec v,\bvec w$ 为空间中任意三个矢量, $a,b$ 为任意两个标量.

\subsubsection{加法运算}
\begin{enumerate}
\item 满足加法交换律 $\bvec u + \bvec v = \bvec v + \bvec u$.
\item 满足加法结合律 $(\bvec u + \bvec v) + \bvec w = \bvec u + (\bvec v + \bvec w)$.
\item 存在零矢量,使得 $\bvec v + \bvec 0 = \bvec v$.
\item 空间中任何矢量 $\bvec v$ 存在逆矢量 $-\bvec v$,使得 $\bvec v + (-\bvec v) = \bvec 0$.
\end{enumerate}

\subsubsection{数乘运算}
\begin{enumerate}
\item 乘法分配律 $a(\bvec u + \bvec v) = a\bvec u + a\bvec v$ 
\item 乘法分配律 $(a + b)\bvec v = a\bvec v + b\bvec v$
\item 乘法结合律 $a (b \bvec v) = (ab) \bvec v$
\end{enumerate}

作为一个非几何矢量的例子, 我们来看由多项式构成矢量空间.

\begin{example}{多项式}\label{LSpace_ex1}
所有不大于 $n$ 阶的多项式 $c_n x^n + c_{n-1} x^{n-1} + \dots + c_1 x + c_0$ 可以构成一个实数矢量空间或复数矢量空间.定义矢量加法为两多项式相加, 满足
\begin{itemize}
\item 封闭性:任意两个不大于 $n$ 阶的多项式相加仍然为不大于 $n$ 阶的多项式.
\item 交换律:多项式相加显然满足交换律.
\item 零矢量:常数 0 可以看做一个 0 阶多项式, 任何多项式与之相加都不改变.
\item 逆矢量:把任意多项式乘以 $-1$ 就得到它的逆矢量, 任意多项式与其逆矢量相加等于 0.
\end{itemize}
定义矢量数乘为多项式乘以常数, 显然也满足数乘的各项要求, 不再赘述.
\end{example}

\begin{exercise}{几何矢量}
证明 1,2,3 维空间中的所有几何矢量各自构成一个实数矢量空间.
\end{exercise}

另一个重要的矢量空间,是\textbf{函数空间}.

\begin{example}{函数空间}\label{LSpace_ex2}
实数到实数的全体函数($f:\mathbb R \to \mathbb R$)的集合 $F$ 构成一个线性空间,称为 $\mathbb{R}$ \textbf{函数空间}. 函数空间中两个向量的加法定义为,对于任何实数$x$和函数(即向量)$f, g\in F$,有$(f+g)(x)=f(x)+g(x)$;数乘定义为,对于任何实数$a, x$和函数$f\in F$,有$(af)(x)=af(x)$.

类似地,复数到复数、实数到复数等的函数都可以构成线性空间;把函数限制在连续函数、可导函数等条件下也依然构成线性空间.特别地,复数域上的归一化可导函数,构成了复数域上的希尔伯特空间,这是一种无穷维的特殊矢量空间,是量子力学的基础概念,我们将会在将来详细讨论.
\end{example}

注意矢量空间的定义并不需要包含内积(点乘) 的概念, 但我们可以在其基础上额外定义内积, 这样的空间叫做\textbf{内积空间}\upref{InerPd}, 留到以后介绍. 除了内积, 我们可以把 “几何矢量的运算\upref{GVecOp}” 和 “线性相关性\upref{linDpe}” 中介绍的概念都拓展到一般的矢量空间中, 这里不再重复.

\begin{exercise}{复数列矢量}
我们把 $N$ 个复数 $c_1, \dots, c_N$ 按顺序排成一列(或一行, 下同), 叫做\textbf{列矢量}(或\textbf{行矢量}, 下同). 列矢量可以看成是 $N \times 1$ 的矩阵\upref{Mat}. 给它们定义通常意义的加法和数乘运算, 这样所有列矢量可以构成一个 $N$ 维矢量空间. 注意由于我们使用了复数, 即使 $N \leqslant 3$ 时我们也无法将这些矢量与几何矢量对应起来.

如果我们将基底取为\footnote{上标 $\mathrm T$ 表示转置, 这里是为了排版方便} $(1, 0, \dots, 0)\Tr$, $(0, 1, 0, \dots, 0)\Tr$, …, $(0, \dots, 0, 1)\Tr$, 那么显然任意列矢量 $(c_1, \dots, c_N)\Tr$ 的坐标就是有序实数 $c_1, \dots, c_N$. 但我们也可以取其他基底, 这时坐标就会改变. 所以再次强调坐标和矢量本身是不同的.我们将会在矢量空间的表示\upref{VecRep}中详细区分矢量本身和矢量的坐标这两个概念.
\end{exercise}

\begin{exercise}{}
证明\autoref{LSpace_ex1} 中多项式空间是 $n+1$ 维空间, $x^k$ ($k = 0, \dots, n$) 是一组基底(提示: 证明它们线性无关, 可以表示空间中的任意矢量).
\end{exercise}





