% 圆锥曲线的统一定义(高中)
% keys 准线|第二定义|焦点|圆锥曲线|焦点-准线定义
% license Xiao
% type Tutor

\begin{issues}
\issueDraft
\end{issues}

\pentry{圆锥曲线与圆锥\nref{nod_ConSec}}{nod_55cd}

在\enref{圆锥曲线与圆锥}{ConSec} 中曾提到,古希腊时期,人们通过截取圆锥面,得到了圆、椭圆、抛物线和双曲线这四类曲线,它们后来被统称为“圆锥曲线”。这种从几何构造出发的方法,直观地揭示了它们的共同起源。然而,尽管阿波罗尼乌将它们放入了同一个圆锥面,在后来一千多年的研究中却仍然被当作彼此独立的对象来看待。无论是图像形状还是代数表达式,它们看起来都完全不同,彼此之间似乎没有直接联系。这种割裂也延续到了现代教学中。

以椭圆为例,高中教材通常在介绍完长轴、短轴、焦点等要素之后,会突然引入一个叫“离心率”的量\footnote{本站的其他内容,为避免突兀感,并未立即引入。},并简单解释为“衡量椭圆扁平程度的参数”。它的定义是一个比值,看起来更像三角函数的形式,而不像之前那些能直接对应到具体图形的长度。更令人疑惑的是,不只是椭圆,抛物线和双曲线也各有自己的离心率。这不禁让人想问:为什么所有圆锥曲线都有离心率?这个量到底意味着什么?不同曲线之间的离心率又有什么关系?

这类困惑其实很正常。毕竟,在长轴、焦点等概念被提出的年代,离心率这个概念还没有出现,那时的研究更关注图形本身,而不是参数之间的抽象联系。也正因为如此,当它和其他几何要素一同出现在课本中时,常常会让人感到有些突兀。

随着解析几何的发展,数学家们发现,这些看似不同的圆锥曲线,在引入一个定点和一条定直线后,竟然可以通过一个简洁而优雅的定义统一起来。从这个统一的角度出发,离心率不再是某种“空降”的数值,而是整个几何结构的内在参数。更重要的是,这一定义不仅回答了之前的问题,更在射影几何等更深层的理论中展现出非凡的结构美感。遗憾的是,这部分内容在现行高中课程中已被完全删去。为了带给读者获得更全面的视角,本文将从统一定义出发,系统梳理圆锥曲线的几何构造及其背后隐藏的深层联系。

% 先以“焦点-准线”的统一定义为出发点,引出为什么引入射影几何的视角可以进一步理解这种统一性。

\subsection{关于统一定义的思考}

其实,既然圆锥曲线本就有着同一个几何来源,一个直觉的想法是:它们应当具有相同的表达形式,而现在看到的不同方程,很可能只是某些参数取值不同的结果,就像截圆锥的平面与母线或轴之间的夹角不同所带来的变化一样。既然目标是寻找这种统一形式,那么接下来的问题就是:该如何找到它?

回顾三种圆锥曲线的定义,唯一在所有定义中都共同出现的元素是焦点,因此,直觉上,统一定义中一定要有焦点的存在。而与焦点直接相关的参数,一是椭圆与双曲线的焦距 $2c$,一是抛物线的焦准距 $p$。很显然,想要想统一定义,就必然要将椭圆与双曲线纳入抛物线的定义方法中,或者将抛物线纳入到椭圆与双曲线的定义方法中。

原来人们以为行星轨道是圆的,但开普勒在使用第谷·布拉赫留下的精密观测数据分析行星运动轨道时,发现数据总对不上。直到他发现如果轨道是椭圆,而太阳在椭圆的一个焦点上,那么数据就能精确拟合。也就是开普勒第一定律:行星绕太阳运动的轨道是椭圆,太阳位于其中一个焦点上。而这时另外一个焦点却并没有一个实际的天体在那里,它单纯只是一个几何构造。

天文学者们总结认为,行星的轨道是一个点和某种约束条件下的轨迹,哪怕之前认为的圆轨道也是如此。当然,后来牛顿总结的约束条件就是万有引力定律,这是后话了。

开普勒的发现使得“焦点”具有直接的物理解释,更让人们摆脱了椭圆必须要有两个焦点的想法。于是,前面统一的问题有了一个最自然的思路——在椭圆和双曲线中也构造出一个“虚构的准线”,尝试用焦准距的方式来描述。

回顾抛物线的定义,说的是,假设点$P$到一个定点$F$到距离为$|PF|$,到一条定直线$L$的距离为$|PL|$,那么满足$|PF|=|PL|$的点$P$轨迹就是抛物线。有了之前从圆的定义出发得到其他圆锥曲线的经验,这里考虑,如何将唯一的约束条件$|PF|=|PL|$推广,一个作法是像得到阿波罗尼斯圆那样推广,得到$P$满足:
\begin{equation}
\frac{|PF|}{|PL|} = e~.
\end{equation}
此时,若 $e = 1$ 则正好是抛物线,下面分别探究$e \ne 1$ 时,会是什么?

\subsubsection{$0<e<1$}



\begin{example}{设点$C$}

\end{example}
由直角坐标方程可知对称性,可在椭圆的两边做两条准线,令椭圆上任意一点到两焦点的距离分别为 $r_1$ 和 $r_2$,到两准线的距离分别为 $d_1$ 和 $d_2$,则有
\begin{equation}
e = \frac{r_1}{d_1} = \frac{r_2}{d_2} = \frac{r_1 + r_2}{d_1 + d_2}~,
\end{equation}
所以
\begin{equation}
r_1 + r_2 = e(d_1+d_2) = 2e(c + h) = 2\frac{c}{a} \qty( c + \frac{b^2}{c} ) = 2a~,
\end{equation}
证毕。
\subsubsection{$e>1$}
双曲线的另一种定义是, 曲线上任意一点到两个焦点距离之差等于 $2a$。 这里证明前两种定义满足该性质。 由对称性, 不妨只考虑右支上的某点, 令其到右焦点和右准线的距离分别为 $r_1$ 和 $d_1$, 到左焦点和左准线的距离分别为 $r_2$ 和 $d_2$。 由离心率的定义, 有
\begin{equation}
e = \frac{r_1}{d_1} = \frac{r_2}{d_2} = \frac{r_2 - r_1}{d_2 - d_1}~,
\end{equation}
由于两准线之间的距离恒为 $2a^2/c$, 上式变为
\begin{equation}
r_2 - r_1 = e(d_2 - d_1) = 2a~,
\end{equation}
证毕。

\subsection{圆锥曲线的焦点-准线定义}



利用准线与焦点得到的。提供了一个统一的视角来看待



\textbf{圆锥曲线的焦点-准线定义(Focus-Directrix Definition of Conic Sections)}。

\begin{definition}{圆锥曲线的焦点-准线定义}\label{def_HsCsFD_1}
在平面上,所有到一个定点的距离与到一条定直线的距离的比值是一个固定常数的点的轨迹,称为\textbf{圆锥曲线(conic section)}。其中,定点称为圆锥曲线的\textbf{焦点(focus)},定直线称为圆锥曲线的\textbf{准线(directrix)},二者互相对应,对应的焦点与准线的距离称作\textbf{焦准距(focal parameter)},通常记作$p$。比值称作圆锥曲线的\textbf{离心率(eccentricity)},通常记作$e$ 。特别地:
\begin{itemize}
\item 当 $e = 0$ 时,轨迹称为\textbf{圆(circle)}\footnote{这一点会在\aref{后文}{sub_HsCsFD_1}提供说明。}。
\item 当 $0 < e < 1$ 时,轨迹称为\textbf{椭圆(ellipse)}。
\item 当 $e = 1$ 时,轨迹称为\textbf{抛物线(parabola)}。
\item 当 $e > 1$ 时,轨迹称为\textbf{双曲线(hyperbola)}。
\end{itemize}
\end{definition}

\begin{figure}[ht]
\centering
\includegraphics[width=11cm]{./figures/52670f52be70ae3b.pdf}
\caption{$p = 1$时,不同离心率 $e$ 的圆锥曲线} \label{fig_Cone_2}
\end{figure}

显然,定点到定直线的垂线为圆锥曲线的对称轴。

\addTODO{$e = 0$ 为什么是圆?}注意根据定义,圆的准线为无穷远, 所以只能使用中, 圆的半径为无穷小。
\subsection{性质}

离心率表示“扁平程度”:
$$ e = \frac{c}{a} = \sqrt{1 - \frac{b^2}{a^2}} \in [0, 1) ~.$$
椭圆越接近 1 越扁。

\subsection{*射影几何视角下的圆锥曲线}\label{sub_HsCsFD_1}

这一章在高中阶段是完全超纲的内容,大部分的老师甚至都不会提及。但在这个视角下能够观察到非常美妙的统一性质,对于理解圆锥曲线的整体为何会跟随之前给出的准线交点定义有非常重要的作用。因此,此处承接定义进行介绍供读者感受数学之美。


\subsubsection{射影几何}



射影几何中的视角使能够用一种统一且优雅的方式看待圆锥曲线。但在射影几何中,这些差异被看作是坐标选择与观察角度所导致的表象变化,它们在更本质的层面上是一类对象的不同表现:它们都是圆锥曲线。圆锥曲线不是三类不同的曲线,而是一个统一的几何实体的三种视角。它让跳出了直观图形的束缚,从结构上理解几何对象之间的联系,也为代数几何、复几何乃至更高维的几何打下了坚实的基础。

从射影几何的角度看,圆锥曲线定义为一个圆锥面与一个平面相交的轨迹。这个定义在欧几里得空间中也成立,但射影几何更进一步地指出:在射影平面中,所有非退化的圆锥曲线都是射影等价的。这意味着可以通过一个合适的射影变换(即坐标的线性变换加上归一化),将任意一个圆锥曲线变为另一个圆锥曲线——比如将一个椭圆变为一个双曲线或抛物线。

换句话说:
\begin{itemize}
\item 椭圆是在射影平面中与无穷远直线没有实交点的圆锥曲线;
\item 双曲线是在射影平面中与无穷远直线有两个实交点的圆锥曲线;
\item 抛物线是恰好与无穷远直线有一个交点的极限情形。
\end{itemize}

这种分类在射影几何中失去了意义,因为无穷远直线被作为与其他直线同等地位来处理,不再是“例外的部分”。因此,抛物线、椭圆和双曲线不再是本质不同的几何对象,而只是一个对象的不同投影或表示。

此外,射影几何还强调了极点与极线的对偶性,并引入了极线极点变换的工具来研究圆锥曲线的性质,使得很多命题具有了对称且优美的形式。例如:对于一个给定的圆锥曲线,任意一点都有与之对应的一条极线,反之亦然。这种对偶关系在欧氏几何中并不自然存在。



焦点-准线统一视角下的圆锥曲线:通往射影几何的一扇门

已经分别学习过三种圆锥曲线:椭圆、抛物线和双曲线。

它们的定义看起来各不相同,方程也各有特点。但有没有一种方法,能用一句话同时描述这三种曲线呢?答案是:有。

这一节课,将首先从“焦点-准线”的角度,找到一个统一的定义;接着,将引入一种叫做“射影几何”的新视角,让对圆锥曲线之间的联系有更深入的理解。


这个比值 $e$ 叫做离心率(eccentricity),不同的值决定了曲线的形状:
	•	$0<e<1$ 时,是椭圆;
	•	$e=1$ 时,是抛物线;
	•	$e>1$ 时,是双曲线。

这就是寻找的统一定义。只需要一个公式,就可以包含之前学的三种曲线。是不是很简洁?

但还有一个问题:

为什么 $e=1$ 是一个“分界线”?

离心率为什么只分成这三类,而不是连续变化出更多种曲线?

要理解这个问题,需要换一个“看待图形”的方式,也就是今天要介绍的新视角:射影几何。


二、普通几何的局限:为什么看不到统一?

在熟悉的欧式几何中,有一些“默认”的限制,比如:
	•	平行线不会相交;
	•	直线是无限延伸的;
	•	点只能表示有限的位置。

这些看起来都很自然,但正是这些“限制”,让无法从一个更高的角度去看清圆锥曲线之间的关系。

尤其是当在统一定义中使用了“准线”这个概念时,会发现一个问题:

准线是直线,而焦点是点,它们的地位并不对等。

比如,在抛物线中,焦点和准线之间的距离决定了曲线的开口程度;但在椭圆和双曲线中,焦点之间的关系常常比准线更显眼。这种“不对等”让难以一眼看出统一性。

要解决这个问题,需要让“点”和“直线”变得对等、互换,这正是射影几何擅长的。

三、射影几何:添加“无穷远”来重新看世界

射影几何的出发点是:不要再区分平行与相交,也不要忽略“无穷远处”的点。

在现实中早就见过类似的情形:
	•	平行的铁路轨道,在远方看起来会相交;
	•	街道两旁的建筑,在画中会汇聚到“消失点”;
	•	摄影师知道,透视图中的所有平行线,最终都会“汇聚”。

这些现象的数学表达方式,就是射影几何中的一个核心思想:

所有直线在射影几何中都相交——平行线也会在“无穷远点”相交。

于是扩展了平面,引入了一个“无穷远直线”,把所有方向的“无穷远点”放在这条线上。

更惊人的是:

在射影几何中,点和直线可以互换、对称对待;也就是说,直线也可以看成是“由点组成”的,点也可以像直线一样进行变换。

\subsubsection{对偶原理}

在十九世纪,法国数学家腾塞叶(Jean-Victor Poncelet)在俄国战俘营中写下了他的重要著作《论图形的摄影性质》。在这部作品中,他首次系统提出了“对偶原理”与“投影不变性”这两个深刻的几何思想。所谓对偶原理,是指在射影几何中,点与直线可以互换,互换后许多几何命题依然成立。例如,“两点决定一条直线”的命题,对偶后变成了“两条直线决定一个交点”,这两者在射影几何中都同样成立。这种点与线之间的对称关系,揭示了几何结构中隐藏的深层对称性,使人们重新思考“几何事实”背后的逻辑构造。

与此同时,腾塞叶还指出,一些几何性质在投影变换下是保持不变的。换句话说,即使我们改变观察角度或从不同平面进行投影,某些关系仍旧成立,这被称为“投影下的不变性”。比如,共线的点经过中心投影后仍然共线;一个圆在透视下可能变成椭圆、抛物线或双曲线,但这些曲线本质上都是圆锥曲线,因此在射影几何中是等价的。这一思想打破了古典欧几里得几何中对“形状”的执着,把几何研究的焦点从“看上去的样子”转向了“结构中的本质”。

到了二十世纪,随着公理化几何的发展,数学家们进一步发现:在许多几何定理中,把“点共线”换成“线共点”、把“点”换成“直线”后,新的表述仍然成立。这些互换后的命题不仅不是偶然巧合,而是源自整个射影几何体系中点与线的对等地位。对偶原理的提出不仅丰富了几何的思维方式,也为代数几何、拓扑学、以及更现代的数学分支奠定了基础。在这种视角下,几何的研究不再只是对现实图形的模仿,而是一种对空间逻辑结构的深刻把握。


四、重新看焦点和准线:变换下的对等性

回到的统一定义:

到焦点距离与到准线距离的比值等于 $e$

焦点是一个点,准线是一个直线,它们是不一样的。但在射影几何中,可以把直线看成是“一个方向上的点的集合”,特别是在加入了“无穷远点”之后,直线也可以被看作是特殊的“点”。

这就让焦点和准线,在某种意义上变得“对等”。

更重要的是:

在射影几何中,通过变换,可以把一个圆锥曲线变换成另一种类型的圆锥曲线,只要它们满足相同的基本结构。

举个例子:
	•	一个椭圆,通过一个适当的“射影变换”,可以变成一个抛物线;
	•	抛物线也可以变成双曲线;
	•	这些变换不会改变圆锥曲线的“本质”,只改变它在眼中的“样子”。

这就说明,射影几何的世界中,圆锥曲线是一个统一的整体,而不是三种各自孤立的图形。

五、结语:统一不是结束,而是开始

通过这节课的学习,从“到焦点和准线的距离比”这个统一定义出发,进入到了一个新的几何世界——射影几何。在这个世界中,曲线之间的关系变得清晰、自然,而且不再受到原来空间结构的限制。

学数学,不只是为了掌握解题方法,更是为了建立更高层次的理解力。射影几何给展示了:
	•	同一个对象可以从不同的角度理解;
	•	表面看起来不同的东西,背后可能有统一的结构;
	•	有时,必须打破一些“习惯的规则”,才能看到更完整的图景。

未来你会在更高层的数学中看到更多这样的“统一视角”——它们不仅改变你对数学的看法,也可能改变你看待世界的方式。
