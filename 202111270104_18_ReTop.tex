% 实数集的拓扑

\pentry{实数\upref{ReNum}}

首先给出如下定义.

\begin{definition}{开集与闭集}
设$x$是实数. 任意包含$x$的开区间都称作$x$的一个开邻域.

实数集$\mathbb{R}$的子集$U$称为开集 (open set), 如果对于任意$x\in U$, 都存在$x$的开邻域$V_x$使得$V_x\subset U$. 

实数集$\mathbb{R}$的子集$C$称为闭集 (closed set), 如果$\mathbb{R}\setminus C$是开集.

规定空集既是开集也是闭集.
\end{definition}

容易证明如下性质:

\begin{theorem}{开集和闭集的运算}
任意多个开集的并集仍然是开集. 有限多个开集的交集仍然是开集.

等价地, 任意多个闭集的交集仍然是闭集. 有限多个闭集的并集仍然是闭集.
\end{theorem}

\begin{exercise}{}
证明这个定理. 提示: 设$\{U_\alpha\}_{\alpha\in A}$是一族开集, 那么若$x\in \cup_{\alpha\in A}U_\alpha$, 则必定有一$\alpha$使得$x\in U_\alpha$. 如果$U_1,...,U_n$是有限多个开集, $x\in\cap_{k=1}^nU_k$, 而$V_x^k$是$x$的包含在$U_x^k$中的开邻域, 那么$\cap_{k=1}^nV_x^k$还是$x$的开邻域.
\end{exercise}

\begin{exercise}{}
在证明"有限多开集的交集还是开集"时, "有限"这个条件究竟被用在哪里?
\end{exercise}

\begin{example}{一些反例}
无限多个开集的交集不一定是开集. 例如, 设开区间$U_k=(-1/k,1/k)$, 那么$\cap_{k=1}^\infty=\{0\}$. 相应地, 无限多个闭集的并集也不一定是闭集, 例如, 设闭区间$I_k=[0,1-1/k]$, 则$\cup_{k=1}^\infty I_k=[0,1)$, 它不是开集也不是闭集.
\end{example}

粗略地说, 在一个集合上给定拓扑, 就是给定一个衡量元素之间的"远近关系"的尺度. 在实数集$\mathbb{R}$上有一个自然的拓扑. 