% 抛物线(高中)
% keys 极坐标系|直角坐标系|圆锥曲线|抛物线
% license Xiao
% type Tutor

\begin{issues}
\issueDraft
\end{issues}

\pentry{解析几何\nref{nod_JXJH},圆\nref{nod_HsCirc},双曲线\nref{nod_Hypb3}}{nod_7c17}

不知道读者在初次接触双曲线时,是否产生了一种似曾相识的感觉:它的一支看起来与初中阶段学习过的二次函数图像——抛物线,非常相似。二者都不封闭、有一个开口、略微弯曲、向无限延伸,甚至也拥有一条对称轴。相信一些读者可能早已不禁在心中将双曲线的一支等同于抛物线,认为双曲线不过是“两个抛物线”的组合而已。

难以否认,这两种形状在直觉上太像了。回顾初中阶段的学习,教学重点多放在二次函数的代数形式与图像特征之间的关系上。例如,通过函数表达式判断图像的开口方向、对称轴位置、顶点坐标等。这些内容帮助建立了对抛物线的基本印象,但大多仅停留在函数视角,对抛物线作为几何图形本身的性质所涉不多。

然而,进一步学习后可以发现,抛物线在几何上的独特性质,使其在现实生活中具有重要作用。例如,雷达天线的反射面常采用抛物面结构,这是因为抛物线具有精确的聚焦特性:平行电磁波经抛物面反射后,会准确汇聚于焦点;而从焦点出发的信号,也会被反射成方向一致的平行波。这一聚焦能力,使抛物面非常适合实现能量的集中与传输,使抛物线广泛应用于雷达、卫星通信设备、汽车大灯以及太阳能灶等技术场景中。


,然而经过学习后,相信读者会发现,这两类曲线虽然外形上有几分相似,却在几何定义、解析表达式,以及性质方面都存在本质区别。抛物线本身也是一种拥有独立几何意义的特殊曲线,而双曲线也并不是由两条抛物线拼接而成。


\subsection{抛物线的定义}
标准定义:平面上到定点(焦点)和定直线(准线)距离相等的点的轨迹
这就是 “\enref{圆锥曲线的极坐标方程}{Cone}” 中对抛物线的定义。
\begin{figure}[ht]
\centering
\includegraphics[width=4.2cm]{./figures/c89771dd2fef516e.pdf}
\caption{抛物线的定义} \label{fig_Para3_1}
\end{figure}

在 $x$ 轴正半轴作一条与准线平行的直线 $L$, 则抛物线上一点 $P$ 到其焦点的距离 $r$ 与 $P$ 到 $L$ 的距离之和不变。

如\autoref{fig_Para3_1}, 要证明由焦点和准线定义的抛物线满足该性质, 只需过点 $P$ 作从准线到直线 $L$ 的垂直线段 $AB$, 由于 $r$ 等于线段 $PA$ 的长度, 所以 $r$ 加上 $PB$ 的长度等于 $AB$ 的长度, 与 $P$ 的位置无关。 证毕。


\subsection{抛物线的方程}
\begin{theorem}{抛物线的标准方程}

\end{theorem}
	•	顶点在原点,轴为 $y$ 轴的标准式:$x^2=2py$
	•	讨论参数 $p$ 的意义(焦点到顶点的距离)
\begin{theorem}{抛物线的参数方程}
	•	用参数表示抛物线上的点,如 $x=pt^2,,y=2pt$ 等(视教学安排可选讲)
\end{theorem}

\subsection{抛物线的几何性质}
	•	对称性(关于轴对称)
	•	顶点、焦点、准线的定义和关系
	•	开口方向与参数正负有关
	•	通用式推导(顶点在 $(h,k)$ 时的方程)
    反射性质(光线从焦点发出反射后平行于轴)
\subsubsection{切线}
	•	给定抛物线方程和点,求切线方程
	•	切线的几何意义(过点,与焦点、准线的关系)
