% 伊本·海什木(综述)
% license CCBYSA3
% type Wiki

本文根据 CC-BY-SA 协议转载翻译自维基百科 \href{https://en.wikipedia.org/wiki/Ibn_al-Haytham}{相关文章}。

哈桑·伊本·海赛姆(Ḥasan Ibn al-Haytham,拉丁化名为 Alhazen,/ælˈhæzən/;全名:阿布·阿里·哈桑·伊本·哈桑·伊本·海赛姆,阿拉伯语:أبو علي، الحسن بن الحسن بن الهيثم;约965年—约1040年),是伊斯兰黄金时代的一位中世纪数学家、天文学家和物理学家,来自今天的伊拉克地区。[6][7][8][9]他被誉为“现代光学之父”,[10][11][12] 尤其在光学原理与视觉感知领域做出了重要贡献。他最具影响力的著作是《光学书》(阿拉伯语:كتاب المناظر,*Kitāb al-Manāẓir),写于1011年至1021年之间,现存有其拉丁文译本。[13]在科学革命时期,艾萨克·牛顿、约翰内斯·开普勒、克里斯蒂安·惠更斯和伽利略·伽利莱等人经常引用海赛姆的著作。

伊本·海赛姆是第一个正确解释视觉理论的人,[14] 他提出视觉是在大脑中形成的,并指出视觉具有主观性,会受到个体经验的影响。[15] 他还首次提出了光在折射时走最短时间路径的原理,这一原理后来被称为费马原理。[16]在镜学和透镜学领域,他通过对反射、折射以及光线成像性质的研究做出了重大贡献。[17][18]伊本·海赛姆还是最早倡导假设必须通过可验证程序或数学推理支持的实验来检验的人之一——在文艺复兴科学家出现前五个世纪,他就已是科学方法的早期奠基者,[19][20][21][22] 有时他被称为世界上“第一位真正的科学家”。[12]
此外,他还是一位博学多才的学者,著述涵盖哲学、神学和医学等多个领域。[23]

伊本·海赛姆是第一个正确解释视觉理论的人,[14] 他提出视觉是在大脑中形成的,并指出视觉具有主观性,会受到个体经验的影响。[15] 他还首次提出了光在折射时走最短时间路径的原理,这一原理后来被称为费马原理。[16] 在镜学和透镜学领域,他通过对反射、折射以及光线成像性质的研究做出了重大贡献。[17][18]伊本·海赛姆还是最早倡导假设必须通过可验证程序或数学推理支持的实验来检验的人之一——在文艺复兴科学家出现前五个世纪,他就已是科学方法的早期奠基者,[19][20][21][22] 有时他被称为世界上“第一位真正的科学家”。[12]此外,他还是一位博学多才的学者,著述涵盖哲学、神学和医学等多个领域。[23]
\subsection{生平}
伊本·海赛姆(拉丁化名 Alhazen)出生于约公元965年,其家族具有阿拉伯[9][31][32][33][34]或波斯[35][36][37][38][39]血统,出生地为伊拉克的巴士拉,当时属于布韦王朝的统治范围。他最初受宗教研究和公共服务的影响较深。当时社会中存在许多相互冲突的宗教观点,他最终决定淡出宗教领域,转而投身于数学与科学的研究。[40]他在家乡巴士拉曾担任“宰相”的职位,并因其应用数学方面的才能而闻名,其中一个例证是他曾尝试设计方案以调控尼罗河的泛滥。[41]

他回到开罗后,被任命为一个行政职务。然而,他最终也未能胜任该职,引起了哈里发哈基姆的愤怒,[42] 据说他因此被迫躲藏,直到哈里发于公元1021年去世,他才得以恢复自由,并领回被没收的财产。[43] 传说称,海赛姆假装疯癫,并在这段时间中被软禁在家中。[44] 正是在这段时期,他撰写了其最有影响力的著作《光学书》。此后,海赛姆继续留居开罗,住在著名的爱资哈尔大学附近,靠其著作收入为生,直到约公元1040年去世。[45][41]现存有一部伊本·海赛姆亲笔誊写的阿波罗尼奥斯《圆锥曲线论》手稿,藏于圣索菲亚图书馆,编号为 MS Aya Sofya 2762,第307叶,落款日期为伊斯兰历415年萨法尔月(公元1024年)。[46]:注2

他的学生中包括一位来自塞姆南的波斯人苏尔哈布(Sorkhab,或作 Sohrab),以及一位埃及王子阿布·瓦法·穆巴希尔·伊本·法泰克。[47]
\subsection{《光学书》}
伊本·海赛姆最著名的著作是其七卷本的光学论文集——《光学书》,写于公元1011年至1021年间。[48] 在该书中,他是第一个解释视觉是由于光线从物体反射后进入眼睛而产生的人,[14] 同时他还首次提出视觉是在大脑中形成的,并指出视觉具有主观性,会受到个人经验的影响。[15]

这部著作在12世纪末或13世纪初由一位不知名的学者翻译成拉丁文。[49][a]

该书在中世纪享有极高声誉。其拉丁文版本 De aspectibus 于14世纪末被译成意大利通俗语,题为 《De li aspecti》。[50]

1572年,弗里德里希·里斯纳将其印刷出版,书名为:Opticae thesaurus: Alhazeni Arabis libri septem, nunc primum editi; Eiusdem liber De Crepusculis et nubium ascensionibus(中文译名:《光学宝藏:阿拉伯人阿尔哈曾七卷本著作,首版;另附其关于黄昏与云层高度的著作》)[51]“Alhazen”这一名字变体即由里斯纳所创,在此之前他在西方被称为“Alhacen”。[52]1834年,E. A. 塞迪约在巴黎国家图书馆发现了海赛姆关于几何的若干著作。根据A. Mark Smith 的研究,目前共发现18部完整或近完整的手稿和5部残卷,分布于14个地点,包括牛津大学的博德利图书馆和布鲁日图书馆等地。[53]
\subsubsection{光学理论}
\begin{figure}[ht]
\centering
\includegraphics[width=6cm]{./figures/0669b50632580a1e.png}
\caption{} \label{fig_YBH_1}
\end{figure}
在古典时代,关于视觉的主要理论有两种:第一种是发射理论,由欧几里得和托勒密等思想家支持,他们认为视觉是由于眼睛发出光线与物体接触而产生的。第二种是摄入理论,由亚里士多德及其追随者支持,他们认为物体以某种物理形式将图像传入眼中。早期伊斯兰世界的学者(如金迪 al-Kindi)基本上沿用了欧几里得、盖伦或亚里士多德的理论体系。《光学书》最强烈的影响来源于托勒密的《光学》,而其中关于眼睛的解剖与生理结构的描述,则是基于盖伦的医学论述。[54]海赛姆的成就在于,他创造性地提出了一个理论,成功结合了:欧几里得的数学光线理论;盖伦的医学传统;以及亚里士多德的摄入理论中的部分要素。在他的摄入理论中,海赛姆继承了金迪的观点(并不同于亚里士多德),提出:“在任何被光照亮的彩色物体上,从其每一个点都会沿着所有可以从该点画出的直线,向外发出光与颜色。”[55]这就为他留下了一个重要问题:如何从如此多独立来源的辐射中形成一个连贯的图像?——特别是,当物体的每一个点都向眼睛的每一个点发送光线时,图像如何保持清晰与一致?

海赛姆所需要解决的问题是:物体上的每一个点如何只对应眼睛上的一个点。[55]为了解决这个问题,他提出了一个观点:眼睛只感知来自物体的垂直光线——也就是说,对于眼睛上的任意一点,只有那条直接进入该点、且未被眼睛其他部分折射的光线才会被感知。他使用了一个物理类比来说明垂直光线比斜射光线更“有力”:就像一个球如果垂直击中木板,可能会将其击碎;但如果是斜着打过去,就会被弹开。同样的道理,垂直入射的光线比折射偏转的光线更“强”,因此只有垂直光线才会被眼睛感知。由于从物体某一点发出的光线中,只有一条垂直光线能准确地进入眼睛的某一点,且所有这些光线在眼睛中形成一个朝向中心的光锥结构,这一观点便使他成功解决了“物体每个点发出大量光线会造成视觉混乱”的问题。换句话说,如果只有垂直光线会被感知,那么就可以实现“物体点”和“眼睛点”之间的一对一对应关系,避免了图像混乱。[56]后来,在《光学书》第七卷中,海赛姆又进一步提出:其他(非垂直)光线在进入眼睛后,会被折射,并最终“仿佛”以垂直方式被感知。[57]然而,他关于垂直光线的论证仍存在解释不足之处:[58]为什么只有垂直光线被感知?为什么较“弱”的斜射光线不会被较弱地感知?
他后来提出的“折射光线被看作垂直”的观点,[59] 从逻辑上也不具备足够的说服力。尽管如此,这一理论在当时仍是最为全面的光学体系,并具有极其深远的影响,尤其是在西欧中世纪至近代早期。海赛姆的《De Aspectibus》(即《光学书》的拉丁译本)直接或间接地激发了13至17世纪间大量光学研究活动。开普勒后来关于视网膜成像的理论,正是在海赛姆的概念框架基础上发展而来,最终彻底解决了“物体点与眼睛点一一对应”的问题。[60]
