% 费马小定理(综述)
% license CCBYNCSA3
% type Wiki

本文根据 CC-BY-SA 协议转载翻译自维基百科\href{https://en.wikipedia.org/wiki/Fermat\%27s_little_theorem}{相关文章}。

在数论中,费马小定理指出:如果 $p$ 是一个质数,那么对于任意整数 $a$,数 $a^p - a$ 都是 $p$ 的整数倍。用模运算的记号表示,就是:
$$
a^p \equiv a \pmod{p}~
$$
例如,当 $a = 2$、$p = 7$ 时,有 $2^7 = 128$,而 $128 - 2 = 126 = 7 \times 18$,正好是 7 的倍数。

如果 $a$ 不能被 $p$ 整除,也就是说 $a$ 与 $p$ 互素,那么费马小定理等价于以下陈述:$a^{p-1} - 1$ 是 $p$ 的整数倍,或者用符号表示为:
$$
a^{p-1} \equiv 1 \pmod{p}~
$$
例如,当 $a = 2$、$p = 7$ 时,有 $2^6 = 64$,而 $64 - 1 = 63 = 7 \times 9$,也是 7 的倍数。

费马小定理是费马素性检验的理论基础,也是初等数论中的一个基本定理。该定理得名于皮埃尔·德·费马,他在 1640 年提出了这一结论。之所以称为“小定理”,是为了将其与费马大定理区分开来。
\subsection{历史}
皮埃尔·德·费马(Pierre de Fermat)最早是在一封写于 **1640 年 10 月 18 日** 的信中向他的朋友兼知己弗雷尼克尔·德·贝西(Frénicle de Bessy)陈述了这一定理。他的表述等价于如下内容:

> 如果 $p$ 是质数,而 $a$ 是任意一个不能被 $p$ 整除的整数,那么 $a^{p-1} - 1$ 一定能被 $p$ 整除。

费马的原始陈述是:

> Tout nombre premier mesure infailliblement une des puissances −1 de quelque progression que ce soit, et l'exposant de la dite puissance est sous-multiple du nombre premier donné −1; et, après qu'on a trouvé la première puissance qui satisfait à la question, toutes celles dont les exposants sont multiples de l'exposant de la première satisfont tout de même à la question.

这段话翻译成中文,并加入一些注释和公式以便理解,可以表述为:

> 每一个质数 $[p]$ 必定整除某一个“某个等比级数 $[a, a^2, a^3, \dots]$”中的“某个幂减一”的结果(即存在某个 $t$,使得 $a^t - 1$ 可被 $p$ 整除),且这个幂的指数 $t$ 是 $p - 1$ 的一个约数。找到第一个满足条件的幂之后,所有指数是这个指数 $t$ 的倍数的幂,同样也满足这一性质。

也就是说,如果某个 $t$ 满足 $a^t \equiv 1 \pmod{p}$,那么所有 $t$ 的倍数也都满足 $a^{kt} \equiv 1 \pmod{p}$。

费马并没有讨论当 $a$ 是 $p$ 的倍数时的情形,也并没有给出他的证明,只是写道:

> Et cette proposition est généralement vraie en toutes progressions et en tous nombres premiers; de quoi je vous envoierois la démonstration, si je n'appréhendois d'être trop long.

翻译为中文是:

> “这个命题在所有的数列和所有的质数中都是普遍成立的;我本可以把它的证明寄给你,只是担心写得太长了。”\(\)

---

如需继续翻译后续段落,请告诉我。
