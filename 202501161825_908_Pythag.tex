% 毕达哥拉斯(综述)
% license CCBYSA3
% type Wiki

本文根据 CC-BY-SA 协议转载翻译自维基百科\href{https://en.wikipedia.org/wiki/Pythagoras}{相关文章}。

\begin{figure}[ht]
\centering
\includegraphics[width=6cm]{./figures/76de0efc99539521.png}
\caption{萨摩斯的毕达哥拉斯雕像,位于罗马的卡比托利尼博物馆[1]} \label{fig_Pythag_1}
\end{figure}
萨摩斯的毕达哥拉斯(古希腊语:Πυθαγόρας;约公元前570年-约公元前495年),通常以单名“毕达哥拉斯”而为人熟知,是一位古代伊奥尼亚希腊哲学家、博学家,也是毕达哥拉斯学派的创始人。他的政治和宗教教义在大希腊地区广为人知,并且影响了柏拉图、亚里士多德的哲学思想,进而影响了整个西方。关于他生平的知识被传说所笼罩;现代学者对毕达哥拉斯的教育背景和影响存在分歧,但他们一致认为,在公元前530年左右,他前往意大利南部的克罗顿,在那里创立了一所学校,要求学员发誓保守机密,并过着共同体的禁欲主义生活。

在古代,毕达哥拉斯被誉为许多数学和科学发现的创立者,包括毕达哥拉斯定理、毕达哥拉斯调律、五种规则立体、比例理论、地球的球形学说,以及早晨星和晚星是金星的天体认知。据说他是第一个自称为“哲学家”(即“智慧之爱者”)的人,也是第一个将地球划分为五个气候带的人。古代历史学家对毕达哥拉斯是否做出了这些发现存在争议,许多归功于他的成就可能在他之前就已被发现,或者是由他的同事或继承者完成的。一些记载提到,与毕达哥拉斯相关的哲学与数学有关,数字在其中占有重要地位,但关于他在数学或自然哲学方面的实际贡献程度仍存在争议。