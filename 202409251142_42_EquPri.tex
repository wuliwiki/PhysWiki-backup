% 等效原理
% keys 等效原理
% license Usr
% type Tutor

\footnote{ A.Zee,Einstein Gravity in a Nutshell.}等效原理是广义相对论的根基,它声称:在足够小的时空区域内,没有实验能够验证处于其中的物体是在引力场中还是加速参考系中。本节将通两个思维实验来理解Einstein是如何获得这一思想的。

\subsection{下落的盒子}
让我们思考这样一个愚人节恶作剧:趁我们其中一个朋友睡着的时候,把他放在一个进行精心设计的盒子里,这个盒子装饰得和这个伙计睡觉的地方一模一样。然后我们从很高的飞机上丢下这个盒子。

当我们的朋友醒来的时候,他认为他正处于他的房间里。由于他和他周围的所有东西都以盒子同样的速率向下加速,相对于他的四周来说,他感觉不到他在向下坠落。他轻轻的一跳,他发现他飘向了天花板。然而对于外面的观测者来说:我们的朋友,只不过是通过踩在地板上,降低了他的下落速度,同时增加了盒子的下落速度。他认为他是飘向天花板的,然而实际上他的坠落正在和之前一样的速率向下加速。

事实上,这一可怕且不道德的愚人节恶作剧已经被实验过了:我们的宇航员被放在一个叫做宇宙飞船的盒子里,然后从天空之外丢下它(宇航员返回地球)。人性化起见,总是给盒子一个向前的运动,以便坠入地面时和地面有个好的弯曲度感受。

为更详细的了解引力,然我们再次思考愚人节恶作剧。为了让这一恶作剧奏效,关键是要所有物体精确地以相同的速率下落。相反,若盒子比我们的朋友下落的快,那么我们的朋友将会发现自己被钉在天花板上,他可能会解释为存在一个力将他往上拉。若盒子比他下落的慢,则他会感到一个力把他拉在地面上。所有的物体以相同的速率下落而和它们的组成成分无关,这和日常经验是相反的,但是Galileo猜测我们的日常经验被空气阻力给扭曲了。

\subsubsection{引力的普遍性}
一个坠落的人不知道他是下落的,因为他周围的所有东西都以同样的速率下落,换句话说,因为引力的普遍性(所有东西都被引力往下拉,而无关他们的组成)。那么,我们可以反过来说吗?因为坠落的人不知道他是下落的,所以引力必须是普遍的。注意到在下落过程中,我们的朋友感受到自己是漂浮的,因此,某种程度上说,下落抵消了引力。那么假如我们用向上推来代替下落,我们可以产生引力吗?

\subsection{远离引力源的火箭}
为了理解Einstein的想法,让我们对我们的朋友开一个更加复杂的玩笑。这一次,趁他睡着,我们把他放进盒子里并让盒子飞向星际空间深处,远离所有的引力场。现在,加速发动机,并以恒定的速率加速装置。当他醒来后,没有发现任何异样,这一次没有漂浮感。

当他丢下苹果,苹果立刻掉在地上。然而对于漂浮在飞船外的观测者,看到的是飞船呼啸而过,下落的苹果实际上是漂浮在飞船里的(匀速运动),苹果完全没有意识到飞船正在以不断增加的速度冲向它。若我们以精确的加速度加速盒子,那么我们的朋友会看到苹果好像是在地球上以合适的速率下落。

通过加速的火箭盒子(实际上,逆转自由落体),我们可以产生引力。很明显,若我们的朋友在距离地板同样的高度落下石头和苹果,那么它们将在同一时刻“掉”在地板上。但是对他来说神秘的普遍性对于外面的观测者而言极其可笑地明显:地板向上移动,遇见了苹果和石头,所以同时和它们相遇。

所以,难道说,Pisa斜塔上下落的苹果和石头并没有掉下,而是一动不动的悬在空中,实际上是地面冲向了它们?这就解释了为什么苹果和石头是同时落地的。运动的相对性!

\subsubsection{"好像"已经够好了}
但是,这听起来完全是胡说八道。地球带着Pisa的斜塔和整个城镇冲向苹果和石头?怎么能用这种特殊的幻觉来解释引力呢?世界上所有人都在丢下东西,成熟的果实从树上掉下来,书呆子的物理学家被自己绊倒了,若真那样,地球必须要这样或那样的冲刺。

然而,地面冲向苹果和石头,对它们同时落地的现象是如此简单的解释。这其中一定有一些真实的因素。要在无意义之外制造意义,关键的见解是“好像”已经够好了。地面不需要匆忙上去,只要说,引力就表现得好像是地面在加速。我们可以通过称“好像”为“等价”来更学术地阐明这一点。


\subsection{等效原理}
现在,我们得出了Einstein的等效原理,这一原理声称:在足够小的时空区域内,没有实验能够验证处于其中的物体是在引力场中还是加速参考系中(或称,在足够小的时空区域,引力场等价于加速参考系)。

注意,“足够小的时空区域”的“足够小”是指一个区域小得可以和引力的特征尺度相比较。这是容易理解的,加入下落的苹果是在地球上而不是深空中,那么在下落的时候由于下落方向指向地心,因此苹果和石头会微微靠近,通过对此效应进行测量,实际上可以确定是处于地球的引力场还是加速的盒子里。

等效原理是对时空小区域物理的陈述。

\subsection{等效原理的预言}

\subsubsection{光线偏折}
想象在远离引力源的火箭盒子里,我们的朋友相当的古怪,他拿起一把激光枪射向了墙面。对于外面的观测者来说,没有引力场存在,时空度规由平坦的Minkowsk度规描述,光以匀速直线运动 $\dd t^2=\dd{\vec x}^2$(\autoref{sub_RAct_2})。然而,火箭盒子不停的向上加速,因此,在光线穿过房间的时间里,墙已经向上了,从而激光打在瞄准点的下面。然而对我们的朋友来说,他感受到他处于地球的引力场中,光线的偏折是引力场造成的。等效原理断言:光线在引力场中要发生偏折。

在地球上坠落的情形,我们的朋友感受不到引力,因此光对它来说走直线,激光点刚好打在瞄准点上。而对外面的人来说,只不过墙上的瞄准点在激光到达那里的时候下落了,这一下落距离刚好和光在引力场中的下落距离一致。

\subsubsection{引力红移}
现在我们的朋友在远离引力源的火箭盒子里将激光枪射向了天花板。那么对外面的观测者来说,因为盒子在加速,光到达探测器时的盒子速度大于发射光时的速度。注意光对于外面的观测者保持光速为 $c$ 的匀速运动,因此普通的多普勒效应(见多普勒效应(一维匀速)\upref{Dople1})告诉我们探测器将看到频率为 $\frac{c-v-gt}{c-v}f$ 的光,其低于光的频率 $f$。在起始速度 $v=0$ 的惯性系中(正如人静止站在地面上时),探测器看到的频率则为 $\frac{c-gt}{c}\omega=\frac{c-gh/c}{c}\omega$,因此
$\Delta \omega/\omega=\frac{-gh}{c^2}=-\frac{\Phi_{\text{接收器}}-\Phi_{\text{发射器}}}{c^2}$,最后的式子是因为在 $gh/c\ll c^2/c=c$ 时对应于弱引力场的势 $\Phi=gh$。

等效原理告诉我们在任意的弱引力场中这都是成立的,即
\begin{equation}
\frac{\omega_{\text{接收器}}-\omega_{\text{发射器}}}{\omega}=-\frac{\Phi_{\text{接收器}}-\Phi_{\text{发射器}}}{c^2}.~
\end{equation}

上式表明光从引力势能低的地方到引力势能高的地方频率将减小,即光的频率发生红移。这称为\textbf{引力红移}。

对于地球上下落的情形,由于盒子里的朋友感受不到引力,因此接收器接受到的光频率将和发射光的频率一模一样。而对站在地球上的人而言,接收器向下冲向光,因此接收器应当记录下更高频率的光信号,即发生多普勒蓝移,而结果频率没有移动,因此他最终怀疑一定有什么东西可以抵消多普勒蓝移:
地球的引力场必须将光的频率向红色偏移,其量与多普勒蓝移完全相同。

\subsection{注意误解}
在上面的例子中,我们看到:(在加速情形)外面的观测者看到的是Doppler红移,而火箭盒子的观测者看到的是引力红移。难道说等效原理意味着物理依赖于观测者?不是的,等效原理并没有声称这个“物理”不依赖于观测者。物理不依赖于观测者,在这里,物理指代他们都看到了同一红移,而上面的“物理” 是指观测者对同一物理(频率)具有不同的解释,一个解释为引力造成的,另一个解释为加速造成的。

有些人可能会认为Einstein的广义相对论是建立在广义协变原理之上的。然而如果注意到广义协变原理只不过是说我们可以自由的选取参考系描述物理,那么这一断言本身并没有什么内容,顶多是误导。正如在Newton力学一样,我们总可以任意选择参考系。或许人们已经忘了,我们总习惯于选择使我们的方程看上去足够简单的参考系。




