% 物理学常数

\pentry{国际单位制\upref{SIunit}}

\subsection{精确定义的常数}
2019 年的国际单位制\upref{SIunit}中精确定义了 7 个基本常数, 每个基本单位的定义都可以由这些常数的定义测量出来.
\begin{table}[ht]
\centering
\caption{精确定义的常数}\label{Consts_tab1}
\begin{tabular}{|c|c|c|}
\hline
符号 & 精确值 & 名称 \\
\hline
$\nu_{Cs}$ & $9,192,631,770\Si{Hz}$ & 铯原子 133 基态的超精细能级之间的跃迁辐射的电磁波频率 \\
\hline
$c$ & $299,792,458\Si{m/s}$ & 真空中的光速 \\
\hline
$h$ & $6.62607015\times 10^{-34}\Si{Js}$ & 普朗克常数 \\
\hline
$e$ & $1.602176634\times 10^{-19}\Si{C} $ & 元电荷 \\
\hline
$k_B$ & $1.380649\times 10^{-23} \Si{J/K}$ & 玻尔兹曼常数 \\
\hline
$N_A$ & $6.02214076\times 10^{23} $ & 阿伏伽德罗常数 \\
\hline
$K_{cd}$ & $683\Si{Im/W}$ & $540\Si{THz}$ 电磁波的照射效率 \\
\hline
\end{tabular}
\end{table}

\subsection{电动力学}

\subsubsection{真空介磁常数}
\begin{equation}
\mu_0 = 1.25663706212(19)\times 10^{−6} \Si{H/m}
\end{equation}
在 2019 年前, 它被精确定义为 $4\pi\times 10^{-7} \Si{H/m}$.

\subsubsection{真空介电常数}
\begin{equation}
\epsilon_0 = 8.8541878128(13)×10^{-12}
\end{equation}
在 2019 年前, 它被精确定义为 $1/(\mu_0 c^2)$.

\subsection{量子力学}
精细结构常数\upref{FinStr}
\begin{equation}
\alpha = 
\end{equation}


\subsection{统计力学}
理想气体常数的精确定义为
\begin{equation}
R = k_B N_A = 8.31447165136438 \,\Si{J/K}
\end{equation}
