% 有界算子的预解式

\pentry{有界算子的谱\upref{BddSpe}}

\begin{definition}{预解式}
设$X$是复巴拿赫空间, $T:X\to X$是有界线性算子. 定于$T$的\textbf{预解式 (resolvent)} 为复变量$z\in\mathbb{C}$的算子值函数
$$
R(z;T):=(z-T)^{-1},
$$
如果$z$使得上式有意义 (即$z$属于$T$的预解集).
\end{definition}

然而我们首先要确认这个式子确实对某些$z$有意义. 引入下列定义:

\begin{definition}{冯诺依曼级数}
几何级数
$$
\text{Id}+T+T^2+...
$$
称为\textbf{冯诺依曼级数 (von Neumann series)}.
\end{definition}
不难看出在$\|T\|<1$时, 它按照算子范数收敛到$(\text{Id}-T)^{-1}$, 这与通常的数值几何级数很类似.

据此便可以证明如下基本命题:

\begin{theorem}{}
\begin{enumerate}
\item 如果$T:X\to X$是有界线性算子, 那么预解集$\rho(T)$是开集, 而谱集$\sigma(T)$是紧集.
\item 如果$T:X\to X$是有界线性算子, 那么它至少有一个谱点, 也就是说$\sigma(T)$是非空紧集.
\end{enumerate}
\end{theorem}
\textbf{证明.} 
对于 1., 如果$\lambda_0\in\rho(T)$, 那么$A:=\lambda_0-T$是有界的可逆算子. 于是可作冯诺依曼级数
$$
A^{-1}(\text{Id}+zA^{-1}+z^2A^{-2}+...+z^nA^{-n}+...);
$$
如果$|z|<\|A^{-1}\|^{-1}$, 那么这个级数收敛到$A^{-1}(\text{Id}-zA^{-1})^{-1}=(\lambda_0-z-T)^{-1}$. 这说明预解集的点的某个邻域还包含在预解集内, 从而预解集是开集, 而谱集是闭集. 另一方面, 只要$|z|>\|T\|$, 冯诺依曼级数
$$
\frac{1}{z}\sum_{n=0}\frac{T^n}{z^n}
$$
就收敛到$(z-T)^{-1}$, 所以谱集包含在圆$|z|\leq\|T\|$内. 所以谱集是紧集.

对于 2., 如果$(z-T)^{-1}$对于所有复数$z$都存在, 那么根据上面的级数展开, 可以看出它是全复平面上的算子值全纯函数, 而且由于$|z|>2\|T\|$时有
$$
f(z)=\frac{1}{z}\sum_{n=0}\frac{T^n}{z^n},
$$
而右边级数的模显然小于2, 所以$f(z)$是有界的全纯函数. 按照刘维尔定理, 它只能是常值函数, 这当然不可能. \textbf{Q.E.D.}

从证明中可见$R(z;T)$是定义在$\rho(T)$上的算子值全纯函数.