% Linux 网络笔记
% license Xiao
% type Note

\begin{issues}
\issueDraft
\end{issues}

\begin{itemize}
\item \href{https://itsfoss.com/basic-linux-networking-commands/}{21 个 Linux 常用网络命令}
\item \verb`sudo ifconfig` 显示网卡信息, 以及 ip 地址 (inet addr)
\item \verb|ip -4 addr show scope global| 也可以。 \verb|ifconfig| 现在已经有点过时了(但仍然很多人和教材在用), 可以用 \verb|ip| 代替。
\item 不同的网卡有不同的 ip 地址, 通常有一张网卡是连互联网的, 其他网卡都是局域网, 甚至虚拟网卡(例如虚拟机, docker 等)。 每个局域网的 ip 范围由子网掩码决定。 对这些不同的局域网来说, 当前电脑就是一个\textbf{网关(gateway)}把它们连起来。
\item 可以想象每张网卡上都长了很多\textbf{端口(port)}, 用于给不同的应用收发各种消息。 例如 http 协议的默认端口是 80, https 是 443
\item 要找到公网网卡, 用 \verb`ip route show | grep default` 找默认网卡, 默认网卡几乎肯定是公网网卡。
\item \verb`ifconfig 网卡名 up` 启动网卡, \verb`ifconfig 网卡名 down` 关闭网卡, 可以用这两个命令重启网卡(要防止 ssh 被中断,两个命令用分号写到一行)
\item \verb`ifoncig` 用于查看网络相关信息: 其中 \verb`Link encap:Ethernet` 表示在使用 Ethernet, \verb`HWaddr` 是硬件地址, 即 MAC 地址, \verb`inet addr` 网卡的 IP 地址, \verb`Bcast` broadcast 地址, \verb`Mask` 掩码, \verb`UP` 表示 Ethernet 的 kernel module 被加载, \verb`BROADCAST` 支持 broadcasting, 从 \textbf{DHCP} 获取 IP 的必要条件, \verb`RUNNING` 准备好传输数据, \verb`mtu` (\textbf{Maximum Transmission Unit}), 就是 packet 的最大尺寸
\item \verb`netplan` 是 ubuntu 18.04 的默认管理网络设置的程序, 比如设置 DHCP, 静态 ip, 掩码, 网关等, 设置完成以后用 \verb`sudo netplan apply` 可以生效。 但有时候还需要重启网卡才能成功。
\item DHCP 的默认地址段是 100 到 149。 DHCP 开启的时候也仍然可以用 netplan 设置静态 ip, 但要避开 100 到 149。
\item 如果 \verb|ip -4 addr show scope global| 返回的 \verb|valid_lft| 和 \verb|preferred_lft| 不是 \verb|forever|, 那(理论上)这个 ip 就可能会变化。 但实际上不常改变。
\item \verb`sudo hpclient` 显示 ip 地址
\item \verb`sudo /etc/init.d/networking restart` 重启网络服务 
\item 要从某个网址下载文件, 只需安装 wget 软件即可, 使用方法如 \verb`wget http://...` 文件会下载到当前文件夹
\item \verb`ping` 可以用来检查是否有网络连接, 比如 \verb`ping google.com` 也可以用来查看某个域名的 ip 地址, 也可以直接使用 ip 地址如 \verb`ping 8.8.8.8` (8.8.8.8 是谷歌的主要 DNS 服务器)
\item \verb`ping 域名` 和 \verb|host 域名| 都可以检查域名的 ip
\item 如果连不上网, 可以参考\href{https://upcloud.com/community/tutorials/troubleshoot-network-connectivity-linux-server/}{这篇文章}的步骤调试
\item 如果想 “ping 某个端口”, 用 \verb|telnet IP 端口号|。 一般来说我们说的端口是 TCP 中的概念, \verb|telnet| 只支持 TCP 不支持 UDP。 如果有任何进程在监听该端口且网络没有问题,那么就会现实 \verb|connected|。 如果卡在 \verb|Trying xxx.xxx.xxx.xxx...| 上, 就是无法建立连接(如果 ping 可以 telnet 不行, 多半是 GFW 屏蔽了某个端口)。 这有别与 \verb|connection refused|, 后者说明这个端口没有被监听(也叫做端口关闭)。
\item 其实本来 \verb|telnet| 是一个远程命令行,也像 \verb|ssh| 一样有一个服务监听一个特定的端口。 但因为不安全,已经被 \verb|ssh| 代替。
\item 除了 \verb|telnet|, 也可以用 \verb|nc -v <ip> <port>| (netcat) 或者 \verb|nmap -p <port> <ip>| 测试 tcp 连接。
\item \verb|nmap| 也被用于测试某个接口是否开启, 以及是否在使用某个 app (如 ssh, ftp, 等)
\item 如果想获得某个 ip 的子网掩码, 用 \verb`whois 66.220.156.68 | grep CIDR` 输出如 \verb|CIDR: 66.220.144.0/20|
\item \verb|traceroute 域名或者ip| 可以查看从当前机器出发到某个机器所经过的网关的 ip (有些会隐藏)。 例如局域网里面的某台机器 traceroute 到外网的某个网址, 那么第一条显示的是当前局域网的网关(用于该局域网的网卡)的 ip, 如果该网关仍然没有处于公网, 那么第二条会显示第一条中网关所在的另一个局域网的网关的 ip…… 直到公网 ip, 然后各种互联网上的公网 ip, 最终达到网址的服务器的 ip。 注意每条中只会显示进入某个机器所用网卡的 ip, 不会显示出去时所用网卡的 ip。
\item \verb|cat /sys/class/net/enp5s0/speed| 可以查看以太网的连接速度(即是否是 Gigabit。 在 GUI 的网络设置里面也可以看到)
\item \verb|iperf3| 可以测试网络中两个主机之间的网速, 参考\href{https://www.cyberciti.biz/faq/how-to-test-the-network-speedthroughput-between-two-linux-servers/}{这个教程}。 不过貌似比较麻烦, 倒不如用 sftp, 一般不需要安装额外东西。
\item \verb|ethtool eth0| 可以查看网卡 \verb|eth0| 的详细信息。
\item \verb|bind9| 可以在 LAN 中设置 DNS 服务器。
\item 现在的网卡一般都支持自动跳线,也就是说两台电脑直接用网线连接没什么问题。
\end{itemize}

\subsection{ip}
\verb|ip a| (或者 \verb|ip addr|) 输出的的一个例子如下:
\begin{lstlisting}[language=none]
1: lo: <LOOPBACK,UP,LOWER_UP> mtu 65536 qdisc noqueue state 
UNKNOWN group default qlen 1000
    link/loopback 00:00:00:00:00:00 brd 00:00:00:00:00:00
    inet 127.0.0.1/8 scope host lo
       valid_lft forever preferred_lft forever
    inet6 ::1/128 scope host 
       valid_lft forever preferred_lft forever
2: enp3s0: <BROADCAST,MULTICAST,UP,LOWER_UP> mtu 1500 qdisc
fq_codel state UP group default qlen 1000
    link/ether d8:bb:c1:17:da:ca brd ff:ff:ff:ff:ff:ff
    inet 192.168.137.1/24 brd 192.168.137.255 scope global 
noprefixroute enp3s0
       valid_lft forever preferred_lft forever
    inet6 fe80::dabb:c1ff:fe17:daca/64 scope link 
       valid_lft forever preferred_lft forever
3: wlo1: <BROADCAST,MULTICAST,UP,LOWER_UP> mtu 1500 qdisc
noqueue state UP group default qlen 1000
    link/ether 14:18:c3:7f:91:85 brd ff:ff:ff:ff:ff:ff
    altname wlp0s20f3
    inet 10.150.61.141/17 brd 10.150.127.255 scope global 
dynamic noprefixroute wlo1
       valid_lft 66898sec preferred_lft 66898sec
    inet6 fe80::e7b9:9ec8:8534:38d2/64 scope link noprefixroute 
       valid_lft forever preferred_lft forever
\end{lstlisting}
\begin{itemize}
\item 其中第一张网卡是 \verb|enp3s0|, \verb|en| 表示 ethernet, \verb|p3|, \verb|s0| 表示某些编号。
\item 尖括号里 \verb|LOWER_UP| 表示网线已经插入, 已经连接到网络。
\item 加括号里 \verb|LOOPBACK| 表示 loop back (例如 localhost 127.0.0.1)
\item \verb|qlen|: default transmit queue length。
\item \verb|inet| ipv4 地址, \verb|inet6| ipv6 地址
\item \verb|brd| 广播(broadcast)地址
\item \verb|mtu| maximum transmission unit (ethernet 都是 1500 字节)
\item \verb|valid_lft|: valid life time
\item \verb|preferred_lft|: preferred life time
\item \verb|qdisc|: queuing discipline。 也就是排队的算法, \verb|noqueque| 是说没有队列。
\item \verb|state| 是网卡的状态, 它的值可以是 \verb|UP, DOWN, UNKNOWN|, 如果开启以后没有连接,会显示 \verb|UNKNOWN|。
\end{itemize}


\subsection{iptables 和 ufw}
\begin{itemize}
\item 参考 \href{https://www.tecmint.com/linux-iptables-firewall-rules-examples-commands/}{25 Useful IPtable Firewall Rules Every Linux Administrator Should Know}。
\item 注意与 routing table 区分
\item \verb|iptables| 是一个命令行防火墙, 它不是一个 service 而是个命令, 所以不能 turn on/off。 它直接跟 Linux 内核的 netfilter 打交道,在开机时、用户直接输入相关命令时,或者被其他服务如 \verb`ufw` 调用时才会跟跟内核沟通。
\item iptables 和 netfilter 在重启后都不会保留任何设置。
\item \textbf{UFW(uncomplicated firewall)} 是 iptable 的一个用户友好的前端。 用户可以直接使用 iptable, 但操作起来较为复杂, 所以直接用 UFW 比较方便。 UFW 是有服务的。 UFW 可以通过服务在每次开机自动恢复用户设置的规则。
\item 要关闭防火墙可以关闭 UFW 服务(如果有其他程序自动设置 iptable 或 netfilter 也需要关闭): \verb|sudo ufw status|, \verb|sudo ufw disable|(立即取消通过 ufw 设置的所有规则,且重启后不自动开启), \verb|sudo ufw enable| (马上恢复之前设置的 ufw 规则,且开机自动开启)。
\item 如果不考虑安全因素,disable 是最方便的,但会使 NAT (见下文)失效, 所以要用 NAT 的话可以允许需要的端口而不 disable。
\item 要重启 UFW, 用 \verb|sudo systemctl restart ufw|
\item \verb`sudo ufw status numbered` 可以看到当前的 ufw 规则。
\begin{lstlisting}[language=none]
Status: active
     To                         Action      From
     --                         ------      ----
[ 2] 139/tcp                    ALLOW IN    Anywhere
[ 3] 445/tcp                    ALLOW IN    Anywhere
[ 4] 22/tcp                     ALLOW IN    Anywhere
[ 6] 139/tcp (v6)               ALLOW IN    Anywhere (v6)
[ 7] 445/tcp (v6)               ALLOW IN    Anywhere (v6)
[ 8] 22/tcp (v6)                ALLOW IN    Anywhere (v6)
\end{lstlisting}
\item 要删除
\end{itemize}

\subsubsection{直接操作 iptable}
\begin{itemize}
\item 参考\href{https://www.cyberciti.biz/faq/how-to-save-iptables-firewall-rules-permanently-on-linux/}{这篇}。
\item 除了上面的 ufw, \verb`iptables-persistent` 也能让 iptable 重启后保留设置。 安装 \verb|sudo apt install iptables-persistent|。 安装过程中会把当前的所有 rule 保存到设置文件 \verb|/etc/iptables/rules.v4| 和 \verb|v6| 里面, 对应 ipv4 和 ipv6。 然后为了安全期间可以复制一个备份 \verb|rules.v4.backup|
\item 现在修改规则以后, 用 \verb|sudo iptables-save -f /etc/iptables/rules.v4| 把修改后的规则保存到设置文件中。 这样重启以后会自动按照配置文件设置。
\item \textbf{iptable} 的种类有 \verb|FILTER|, \verb|NAT|, \verb|MANGLE|, 每种又有不同的 \textbf{chain}, 例如 \verb|INPUT|, \verb|OUTPUT|, \verb|FORWARD|,  \verb|PREROUTING|, \verb|POSTROUTING| 等。
\item 查看所有防火墙规则 \verb|sudo iptables -L -n -v|
\item 查看某个 table 的规则用 \verb|-t|, 如 \verb|sudo iptables -t nat -L -n -v|
\item 屏蔽某个 ip: \verb|sudo iptables -A INPUT -s xxx.xxx.xxx.xxx -j DROP|。
\item 其中 \verb|-A| 就是添加一个规则, 如果改成 \verb|-D| 就是删除该规则。
\item 屏蔽某个 ip 的 tcp traffic: \verb|sudo iptables -A INPUT -p tcp -s xxx.xxx.xxx.xxx -j DROP|
\item 屏蔽一个 OUTPUT 端口: \verb|sudo iptables -A OUTPUT -p tcp --dport xxx -j DROP|
\item 允许一个 INPUT 端口:  \verb|sudo iptables -A INPUT  -p tcp --dport xxx -j ACCEPT|
\item 允许多个 INPUT 端口 \verb|sudo iptables -A INPUT  -p tcp -m multiport --dports 22,80,443 -j ACCEPT|。
\item 如果是一个范围, 就用例如 \verb|--dports 5000:5005|。
\item 允许多个 OUTPUT 端口 \verb|sudo iptables -A OUTPUT -p tcp -m multiport --sports 22,80,443 -j ACCEPT|.
\item 要允许所有 INPUT或OUTPUT 端口, 用 \verb|sudo iptables -A INPUT或OUTPUT -p tcp -m multiport --sports 1:65535 -j ACCEPT|。
\item 屏蔽 facebook: \verb|host facebook.com|, \verb`whois 66.220.156.68 | grep CIDR`, 例如得到 \verb|66.220.144.0/20|, 那么 \verb|sudo iptables -A OUTPUT -p tcp -d 66.220.144.0/20 -j DROP|
\item 以上操作默认对所有网卡进行, 如果要指定某个网卡, 用 \verb|-i 网卡名|。
\item Port Forwarding: \verb|sudo iptables -t nat -A PREROUTING -i eth0 -p tcp --dport 25 -j REDIRECT --to-port 2525| 就是网卡 eth0, 从端口 25 到端口 2525
\item 屏蔽某个 mac 地址 \verb|sudo iptables -A INPUT -m mac --mac-source 00:00:00:00:00:00 -j DROP|
\end{itemize}

\subsection{调试工具}
\begin{itemize}
\item \verb|iperf| 可以用于测试网速
\item \verb`sudo lsof -i -P -n | grep LISTEN` 可以查看哪些端口被占用。 \verb|lsof| 意思是 list opened files。
\end{itemize}


\subsection{用 Virtualbox 搭建局域网}\label{sub_LinWeb_1}
\begin{itemize}
\item 另见 “搭建 Linux 局域网\upref{LinLAN}”
\item 使用 Virtualbox 运行若干个 linux (ubuntu 18 lts) 系统, 并组成 LAN 网络, 以及连接 internet
\item 为了节约空间, 只需要一台 desktop 系统, 其他都是 server 系统(没有 GUI 的)即可
\item 设置好 ssh\upref{SSH} 以后, server 开机后无需登录就可以接收 ssh 连接
\item host 表示运行虚拟机的 OS, guest 表示虚拟机中运行的 OS
\item clone 虚拟机的时候选择改 mac 地址 (也可以 clone 完了以后在网络设置里面刷新 mac 地址)
\item 克隆完后进入每个 guest 在 \verb` /etc/hostname` 修改 hostname, 重启即可
\item 参考\href{https://www.thomas-krenn.com/en/wiki/Network_Configuration_in_VirtualBox}{这里}。
\item 看网页上不同 network type 的表, 其中 \verb` NAT network` 满足要求, 但不能从 host 连接虚拟机, 而 \verb` bridged networking` 可以
\item \verb` Not attached` 模式, guest 会有一张网卡, 但是没有连接 cable
\item \verb` NAT (Network Address Translation)` 模式, 新建虚拟机默认使用这个, guest 可以连接 internet, 但是外部不能访问 guest (例如 guest 上的 web server), 包括 host
\item \verb` NAT network` 模式: \textbf{需要的就是这个, 目前用得是这个} 类似于用 router 将虚拟机连起来,并连互联网, host 连不连暂时无所谓。 在选择这个之前需要在 File->Preference 菜单中的 Network 中新建一个网卡(直接按+按钮即可,无需设置), 然后在对每个虚拟机的 \verb` NAT network` 设置里面选这个网卡。
\item 完了以后好像还要在 linux 里面设置 \verb` netplan` (ubuntu 18) 或者 \verb` ifupdown` (old ubuntu), 可以参考\href{https://www.linux.com/tutorials/how-use-netplan-network-configuration-tool-linux/}{这里}
\item 一个详细的 netplan \href{https://linuxconfig.org/netplan-network-configuration-tutorial-for-beginners}{教程}。
\item netplan 配置文件目录为 \verb` cd /etc/netplan/`, 里面可以手动指定当前机器的 ip 地址
\item 首先备份 netplan 的配置文件 \verb` sudo cp 50-cloud-init.yaml 50-cloud-init.yaml.backup`
\item 然后配置文件这样设置 (host 和每个 guest 给一个不同的 ip) (注意其中 \verb` enp0s3` 是 ifconfig 的连接名)
\begin{lstlisting}[language=none]
network:
    version: 2
    ethernets:
        enp0s3:
            dhcp4: false
            addresses: [10.0.2.4/24]
\end{lstlisting}
\item 在桌面版的 ubuntu, 该文件由设置面板管理, 会有一行 \verb|renderer: NetworkManager|
\item 注意 ubuntu desktop 这么做以后好像连不上 internet, 其实也没关系, 因为 virtualbox 分配的 ip 似乎完全不会变
\item 然后用 \verb` sudo netplan try`, 然后 \verb` ifconfig` 看一下 ip 是否变为指定 ip。 如果有什么问题, 两分钟后会自动恢复原来的设置。(据说恢复时有一个 bug, 并不能恢复, 需要重启才可以)
\item 使用 \verb` sudo netplan apply` 使配置生效
\item 另外试一下 \verb` apt update` 可不可以连接外网
\item 如果上不了外网, \verb` sudo ifconfig enp0s3 down; sudo ifconfig enp0s3 up` 重启网卡即可(如果通过 ssh 链接, 一定不能拆开使用, 否则会断开链接)。
\item 要搜索局域网中所有设备, 用 \verb|sudo arp-scan -l --interface=网卡名| 其中 \verb|网卡名| 就是 \verb|ifconfig| 里面显示的。
\end{itemize}

\subsection{netplan}
\begin{itemize}
\item 一个详细的 netplan \href{https://linuxconfig.org/netplan-network-configuration-tutorial-for-beginners}{教程}。
\item 可以选择两个后端(netplan 设置文件中的 \verb|renderer|): \verb|systemd-networkd| 和 \verb|NetworkManager|。 如果不指定, 默认是前者, 此时 Ubuntu 桌面右上角的网络设置中将会显示 \verb|unmanaged|。
\item 从 Ubuntu 18.04 开始, netplan 代替了传统的配置文件 \verb|/etc/network/interfaces|, 如果要切换回后者, 见\href{https://linuxconfig.org/how-to-switch-back-networking-to-etc-network-interfaces-on-ubuntu-20-04-focal-fossa-linux}{这个教程}。
\item netplan 的目标是让网络设置更中心化以及更简单。
\item netplan 的三个路径(按优先级排序):\verb|/run/netplan|,\verb|/etc/netplan|,\verb|/lib/netplan|, 一般来说默认文件是 \verb|/etc/netplan/01-network-manager-all.yaml|, 内容是
\begin{lstlisting}[language=none]
# Let NetworkManager manage all devices on this system
network:
  version: 2
  renderer: NetworkManager
\end{lstlisting}
其中 \verb|renderer| 的另一个选项是 \verb|networkd|(如果不指定就默认这个)。 桌面版还是建议用 NetworkManager, 因为 GUI 网络设置就是用的这个,可以查看各种信息。
\end{itemize}

\begin{figure}[ht]
\centering
\includegraphics[width=6cm]{./figures/a4562de67b530bca.png}
\caption{netplan 的结构} \label{fig_LinWeb_1}
\end{figure}

\begin{itemize}
\item 要指定 gateway 的话, \verb|gateway4 xxx.xxx.xxx.xxx| 已经过时了, 应该用 \verb|routes: [{to: default, via: xxx.xxx.xxx.xxx}]|。
\end{itemize}

\subsection{配置静态 ip 以及 routing table}
\begin{itemize}
\item 一个完整的静态 ip 示例(在 LAN 中实测可以访问外网)
\begin{lstlisting}[language=none, caption=01-network-manager-all.yaml]
# Let NetworkManager manage all devices on this system
network:
  version: 2
  renderer: NetworkManager
  ethernets:
    enp5s0:
      dhcp4: false
      addresses: [192.168.137.183/24]
      routes:
        - to: default
          via: 192.168.137.214
      nameservers:
        addresses: [10.130.30.52, 10.130.30.53]
\end{lstlisting}
\item 一般来说, \verb|nameservers| 的 ip 可以设置为常用的 \verb|8.8.8.8| 等, 但笔者的校园网禁止访问公共 DNS, 所以笔者在 LAN 的网关电脑上查到了两个校园网提供的 DNS 地址并填入。
\item 参考\href{https://linuxiac.com/how-to-set-static-ip-address-and-modifying-routing-table-on-linux/}{这个教程}, 以及 “Linux DNS 笔记\upref{LinDNS}”。
\item 查看 ip 用 \verb|ip a| (即 \verb|ip address|)
\item 要给某个网卡添加一个 ip, 用 \verb|sudo ip address add 192.168.137.10/255.255.255.0 dev enp3s0| (\verb|en| 表示 ethernet, \verb|p3| 表示 bus number, \verb|s10| 表示 slot number)
\item 关闭和启动网卡 \verb|ip link set dev eth0 up|, \verb|ip link set dev eth0 down|
\item \verb|ip route show| 或者 \verb|route -n| 可以查看路由表。 前者可以看到 DHCP 服务器的位置
\item 要给路由表添加一条, 用 \verb|ip route add 192.168.1.0/255.255.255.0 via 192.168.1.1|
\item 给路由表设置默认网卡, 用 \verb|ip route add default via 192.168.1.1 dev eth0|
\item \verb|arp -a| 查看 ARP (由 ip 获取 mac 地址的协议)缓存
\item \verb|netstat -napt| 可以查看 tcp 连接状态
\end{itemize}


\subsection{NAT}
\subsubsection{Ubuntu 22.04}
\begin{itemize}
\item \verb|NAT| 的原理大概就是把不同内网的 ip 地址 + 端口 (socket) 映射到某个(例如学校的)公网 ip 地址和不同的端口, 并在链接建立以后把后者的端口回传给前者的端口
\item 所以两个不同的 NAT 后面的电脑是可以 P2P 连接的, 但是首先要通过一个公网服务器建立连接, 这几乎是常规操作了(猜测 teamviewer 应该就是这个原理)
\item 具体设置见下文
\item 比如说网关主机(Ubuntu 22.04)上有两张网卡, 一个 ethernet 连接 LAN, 一个 wifi 网卡连接外网。 要使 LAN 中的主机可以访问 internet, 需要在网关主机上设置 NAT, 参考\href{https://unix.stackexchange.com/questions/575178/sharing-wifi-internet-through-ethernet-interface}{这里}的 Setup a NAT 步骤(亲测成功):
\item \verb|/etc/ufw/sysctl.conf| 中注释 \verb|net/ipv4/ip_forward=1|.
\item \verb|/etc/default/ufw| 中设置 \verb|DEFAULT_FORWARD_POLICY="ACCEPT"|
\item \verb|/etc/ufw/before.rules| 中在 \verb`*filter` 之前添加如下几行:
\begin{lstlisting}[language=none]
*nat
:POSTROUTING ACCEPT [0:0]
-A POSTROUTING -o 链接互联网的网卡名如wlo1或enp3s0 -j MASQUERADE
COMMIT
\end{lstlisting}
\item \verb|sudo systemctl restart ufw|, \verb|sudo ufw enable|。 注意如果 disable 的话, 就访问不了外网了。
\end{itemize}

\subsubsection{Ubuntu 20.04}
\begin{itemize}
\item \verb`/etc/sysctl.conf` 中, \verb`net.ipv4.ip_forward=1` 取消注释
\item \verb`sudo sysctl -p` 使其生效
\item 假设有网的网卡为 \verb`wlo1`, 连局域网的网卡为 \verb`eth0`
\begin{lstlisting}[language=none]
sudo iptables -t nat -A POSTROUTING -o wlo1 -j MASQUERADE
sudo iptables -A FORWARD -i wlo1 -o eth0 -m state \
    --state RELATED,ESTABLISHED -j ACCEPT
sudo iptables -A FORWARD -i eth0 -o wlo1 -j ACCEPT
sudo apt-get install iptables-persistent
sudo netfilter-persistent save
\end{lstlisting}
\end{itemize}

