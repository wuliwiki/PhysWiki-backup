% 卡尔·萨根
% license CCBYSA3
% type Wiki

(本文根据 CC-BY-SA 协议转载自原搜狗科学百科对英文维基百科的翻译)

\textbf{卡尔·爱德华·萨根} (1934年11月9日 – 1996年12月20日)是美国天文学家, 宇宙学家, 天体物理学家, 天体生物学家,作家,科普工作者,以及天文学和其他自然科学的科学传播者。他最出名的是他作为科学普及者和传播者的工作。他最著名的科学贡献是在外星生命领域的研究,包括利用辐射从基本化学品中实验生产氨基酸。萨根组装了第一批发送到太空的物理信息 :先驱者号镀金铝板 还有旅行者号黄金唱片,包含有任何找到它们的外星智慧都可能理解的通用信息。萨根认为,现在普遍接受的假设是,金星表面的高温可以归因于温室效应,并且可以利用温室效应来进行计算。

萨根发表了600多篇科学论文和文章,撰写、编辑出版了20多本著作。 他撰写了很多大众科学书籍,例如伊甸园的飞龙、布罗卡的脑 和暗淡蓝点 ,并为1980年获奖的《宇宙:个人之旅》系列担任旁白和编剧。这是美国历史上最受广泛关注的大众电视系列, 宇宙 已经被60个不同国家的至少5亿人观看。 为了配合这个系列出版了同名书籍 宇宙 。他还写了科幻小说 接触,1997年在此基础上拍摄了 同名电影。他的论文中包含涉及595,000份资料,[1] 存档于 国会图书馆。[2]

萨根倡导科学怀疑调查和科学方法,开创了外太空生物学的先河,并推动了对外星智能的探索(SETI)。他的大部分职业生涯是在康奈尔大学(Cornell University)担任天文学教授,并在那里领导行星研究实验室(Laboratory for Planetary Studies)。萨根和他的作品获得了无数奖项和荣誉,包括 美国宇航局杰出公共服务奖章, 国家科学院 公益奖章,《伊甸园之龙 》获得了 普利策普通非小说奖 ,以及 《宇宙:个人航行》,获得了两次 艾美奖, 皮博迪奖以及 雨果奖。他结了三次婚,有五个孩子。萨根患有脊髓发育不良 ,1996年12月20日因 肺炎逝世,享年62岁时。

\subsection{早期生活和教育}
卡尔·萨根出生于纽约 布鲁克林。 他的父亲塞缪尔·萨根是 从 卡米内茨-波德伊尔斯基 之前属于俄罗斯帝国,[3] 如今位于乌克兰移民的 服装工人。他的母亲雷切尔·莫利·格鲁伯是来自纽约的家庭主妇。卡尔是雷切尔为了纪念 生母柴娅·克拉拉而取得名字,用萨根的话说,“她从未见过的母亲”。

他有一个妹妹卡罗尔,一家人住在布鲁克林社区本森赫斯特(Bensonhurst)的一套靠近大西洋的普通公寓里。据萨根说,他们是改革派犹太人,是北美犹太教四个主要群体中最自由的。卡尔和他的姐妹一致认为,他们的父亲并不是特别虔诚的教徒,但是他们的母亲“绝对信仰上帝,在教会里很活跃;只食用洁食肉类”。 在经济大萧条时期,他的父亲是一名戏院引座员。

根据传记作者凯伊·戴维森(Keay Davidson)的说法,萨根的“内心战争”是他与父母双方关系密切的结果,父母在许多方面是“矛盾的”。萨根将他后来的分析冲动追溯到他的母亲,她在第一次世界大战 和20世纪20年代期间,在纽约市还是一个非常贫穷的孩子。 作为一名年轻女性,她有着自己的学术抱负,但却因社会限制:她的贫穷、作为女性并且已婚以及她的犹太民族而受挫。戴维森指出,她因此“崇拜她的独生子卡尔”。他会实现她未实现的梦想。”

然而,他声称他的惊奇感来自他的父亲,他在闲暇时给穷人苹果,或者帮助缓解纽约服装业的劳资关系紧张。 尽管卡尔的智力令他震惊,但他从容地接受了儿子的好奇心,并将其视为他成长的一部分。作为作家和科学家的晚年,萨根经常利用童年记忆来阐述科学观点,就像他在 被遗忘祖先的影子书中所做的那样。萨根描述了他父母对他后来思想的影响:[4]

我的父母不是科研从业者。他们对科学一无所知。 但在同时向我介绍怀疑论和怀疑主义的过程中,他们教会了我两种不同思维模式共存,而这两种模式是科学方法的核心。

萨根回忆说,他最重要的时刻之一是当他四岁的时候,他的父母带他去了 1939年纽约世界博览会 。展品成了他一生中的转折点。他后来回忆起了 明天的美国 展品:“它展示了美丽的高速公路、立交桥和小型通用汽车公司的汽车,所有这些都把人们带到摩天大楼、有可爱尖顶的建筑、飞拱——看起来棒极了!” 在其他展览上,他记得手电筒是如何照射在光电管产生了爆裂声,以及音叉变成了一股波浪 的示波器。他还见证了未来将取代广播的媒体技术:电视。萨根写道:

显然,世界上有一种我从未猜到的奇迹。一个音调怎么能变成一幅图画,而光怎么能变成一种噪音呢

他还目睹了博览会上最广为人知的事件之一,法拉盛草地埋下了一个时空胶囊 ,其中包含了20世纪30年代的纪念品,将在数千年后,被地球上的智慧生命找回。戴维森写道,“时光胶囊让卡尔激动不已”。作为一个成年人,萨根和他的同事也创造类似的时间胶囊——一个将被发送到银河系的胶囊;这些是 先驱者号镀金铝板 还有 旅行者黄金唱片 这些都是世界博览会记忆带给萨根的灵感。

在二次世界大战期间,萨根的家人担心他们在欧洲亲戚的命运。然而,萨根对正在进行的战争的细节并不了解。他写道,“当然,我们有亲戚被卷入了大屠杀。 希特勒在我们家不受欢迎,但另一方面,我并没有受到战争恐怖的影响。”他的妹妹卡罗尔说,他们的母亲“最重要的是想保护卡尔”。她在处理第二次世界大战和大屠杀时异常艰难。" 在萨根的书 《恶魔出没的世界》 (1996年)中,描述了他对这段冲突时期的记忆,当时他的家人弱化了欧洲战争的残酷现实,防止战争破坏他的乐观精神。
\subsubsection{1.1 对自然的好奇}
小学毕业后不久,他开始表现出对自然强烈好奇心。萨根回忆起他独自一人第一次去 公共图书馆 ,那是在他五岁的时候,他妈妈给他一张借书证。他想知道什么是星星,因为他的朋友或父母都不能给他一个明确的答案:

我去图书馆借了一本关于星星的书;答案是惊人的。太阳是离我们很近的一颗恒星。星星是太阳,但在那么远的地方,它们只是小小的光点……我突然意识到宇宙是一种什么样的规模。这是一种宗教体验。那里有一种富丽堂皇的气氛,一种从未离开我的气势。从未离开过我。

大约在六七岁的时候,他和一个密友去了在东河对岸曼哈顿的美国自然历史博物馆 。在那里的时候,他们去了海登天文馆, 在博物馆的展品 例如 流星以及自然环境中恐龙和动物的展示中走来走去。萨根写道:

我被世界各地栩栩如生的动物和它们的栖息地的立体模型惊呆了。企鹅在昏暗的南极冰层上;一群大猩猩,公猩猩敲打着自己的胸膛,一只美国灰熊用后腿站立着,有十到十二英尺高,直直地盯着我的眼睛。

他的父母通过给他买化学用品和阅读材料来帮助培养他对科学日益增长的兴趣。然而,太空是他的主要关注点,尤其是在阅读了一些作家的科幻小说之后,比如 H.水井 和 埃德加·赖斯·巴勒斯这激起了他对其他星球如火星上生命的想象 。根据传记作者雷·斯潘根伯格(Ray Spangenburg)的说法,萨根试图理解行星之谜的最初几年成了“他生命中的驱动力,他智慧的持续火花,以及永远不会忘记的探索”。

1947年,他发现 惊奇科幻小说 杂志,看来到了比巴勒斯小说中的推测还要多的硬科幻小说 。 同年,“飞碟”引发了大众的 集体恐慌,年轻的卡尔怀疑”飞碟"可能是外星飞船 。
\subsubsection{1.2 高中时代}
\begin{figure}[ht]
\centering
\includegraphics[width=6cm]{./figures/5d449d8aa3dd7fa5.png}
\caption{萨根高中照片,摄于1951年} \label{fig_Sagan_1}
\end{figure}
萨根住在本森赫斯特,在那里他上了大卫·博迪初中。13岁时在本森赫斯特他度过了自己的成人礼。 第二年,1948年,他的家人搬到了附近的小镇 新泽西州拉威,萨根随后进入了他父亲的工作 拉威高中,他于1951年毕业。 拉威是一个古老的工业城镇,萨根家是当地为数不多的犹太家庭之一。

萨根是一名全优学生,但由于课程缺乏挑战和老师缺乏激情,他感到很无聊。 他的老师意识到了这一点,并试图说服他的父母送他去私立学校,校长Robert Hutchins告诉他们,“这个孩子应该去天才儿童学校,他天分了得。” 但是因为家庭拮据,没有能力将萨根送去私立学校。

萨根被任命为学校化学俱乐部的主席,在家里他建立了自己的实验室。他通过制作纸板剪纸来帮助自己想象分子是如何形成的从而自学了解 分子: “我发现这和做化学实验一样有趣”,他说。 但是萨根最感兴趣的仍然是天文学,在他大三的时候,当他得知做他一直喜欢的天文学家也可以赚钱之后,他就把天文学作为未来的职业目标:“那是美好的一天——当我开始意识到如果我努力的话,我可以做全职天文学家,而不仅仅是兼职。”

高中毕业前,他参加了一场征文比赛,在比赛中,他提出了一个问题:人类与来自另一个星球的高级生命形式的接触,对地球上的人来说,是否会像印第安人第一次与欧洲人接触时一样,带来灾难性的后果。[5] 这个话题被认为是有争议的,但他的修辞技巧赢得了评委的青睐,他们授予他一等奖。[5] 毕业时,他的同学们投票给他“最有可能成功”,并让他成为毕业告别演说者。[5]
\subsubsection{1.3 大学教育}
萨根选择了就读芝加哥大学,尽管他高中成绩优异,但这是他申请的少数几所考虑招收16岁学生的大学之一。该校校长罗伯特·哈钦斯将学校建设成一个没有年龄限制“理想的精英统治”学校。[6] 该校还聘请了包括恩利克·费米(Enrico Fermi)和爱德华·泰勒(Edward Teller)在内的多位美国顶尖科学家,并经营着著名的 耶克斯天文台。[6]

萨根在攻读荣誉学士学位期间,曾在遗传学家 H.j .穆勒(H. J. Muller)的实验室工作,并与物理化学家 Harold Urey(Harold Urey)共同撰写了一篇关于生命起源的论文。 随着 。萨根加入了瑞尔森天文学会,[7] 在笑着自称“一无所有”之后取得了一般以及特殊荣誉本科学位,[8] 在1954年获得物理学的本科学位1955年。1956年他获得物理学硕士学位,1960年 在将论文行星的物理研究 提交给天文和天体物理学系之后萨根获得博士学位。 [9][10][11][8]

他利用研究生其间的暑假与导师、行星科学家 杰拉德·柯伊伯(Gerard Kuiper)[12] 、以及物理学家乔治·盖莫夫 (George Gamow)和化学家梅尔文·卡尔文(Melvin Calvin)合作 。萨根的论文题目反映了他与柯伊伯的共同兴趣,柯伊伯在整个20世纪50年代都是 国际天文学联盟“行星和卫星物理研究”委员会的主席。[13] 在1958年,两个人从事军方的秘密工程 A119项目,空军计划在月球上引爆核弹头。[14]

萨根获得了美国空军 与美国国家航空航天局的最高机密许可。 1959年在撰写博士论文时,萨根透露了当初他申请加州大学伯克利分校奖学金时被美国政府列为机密的两篇A119项目 论文标题 。直到1999年《自然》杂志才公开披露这一泄密事件。项目负责人伦纳德·雷菲尔给《华尔街日报》的后续信件证实了萨根的安全漏洞。[15]

\subsection{ 职业和研究}
\begin{figure}[ht]
\centering
\includegraphics[width=8cm]{./figures/32f5025d39629773.png}
\caption{萨根在讨论关于其他星球上存在生命可能性的,罗伯特德鲁执导的美国宇航局获奖纪录片《谁在那?》。} \label{fig_Sagan_2}
\end{figure}
从1960年到1962年,萨根在加州大学伯克利分校担任 米勒研究员 在。[16] 与此同时,他于1961年在《科学》杂志上发表了一篇关于金星大气层的文章,同时还与美国宇航局的“水手2号”团队合作,并担兰德公司的“行星科学顾问” 。[17]

在萨根发表 《科学》上的文章之后,1961年哈佛大学的天文学家弗雷德·惠普尔和唐纳德·门泽耳 邀请萨根在哈佛举办一个座谈会,随后又给了他一个该机构的讲师职位。 但是萨根要求成为一个助理教授,最终惠普尔和门泽耳说服了哈佛大学向萨根提供他所要求的助理教授职位。[17] 从1963年到1968年,萨根在该学院授课、进行研究并且指导研究生,同时也在马萨诸塞州剑桥的史密森尼天体物理天文台兼职 。

1968年,萨根被哈佛大学拒绝终身教职。他后来表示,这个决定非常出乎意料。[18] 哈佛大学拒萨根绝终身职位有几个因素,包括他把兴趣过于广泛地分散在多个领域(而学术界的标准是成为一个狭窄专业领域的知名专家),或许是因为他广为人知的科学主张,一些科学家认为他借用他人的想法仅仅是为了自我推销。[19] 哈罗德·尤里(Harold Urey)是他大学时代的一名导师,他给终身教职委员会写了一封信,强烈反对萨根获得终身教职。[19]

科学不仅仅是一个知识体系,也是一种思维方式。我有一种不祥的预感,在我的子孙后代的时代,美国只是一个服务和信息经济的实体;当几乎所有的关键制造业都转移到其他国家时;当令人敬畏的技术力量掌握在极少数人手中,甚至没有一个代表公众利益的人能够理解这些问题时;当人民失去了捍卫自己立场的能力或对当权者提出有见地的质疑时;当我们紧紧抓住水晶,紧张地看星象时,我们的批判能力在衰退,无法分辨什么是感觉良好,什么是真实的时候,我们只能别无选择地落入迷信和黑暗之中。

\textbf{卡尔萨根,}from 恶魔降临的世界 (1995) [19]

早在这个哈佛拒绝他的终身教授过程之前,康奈尔大学 的天文学家托马斯·戈尔德(Thomas Gold)就曾向萨根示好,希望他搬到纽约伊萨卡(Ithaca)任教。哈佛大学拒绝授予萨根终身教职后,萨根接受了戈尔德的提议,并在康奈尔大学任教近30年,直到1996年去世。与哈佛不同,康奈尔大学的规模更小、更悠闲的天文系欢迎萨根这种名气日益增长的名人。[20] 做了两年副教授后,1970年在康奈尔大学萨根成为了正教授 ,并在那里指导行星研究实验室。从1972年到1981年,他担任康奈尔大学辐射物理和空间研究中心(CRSR)的副主任。1976年,他成为大卫·邓肯天文学和空间科学教授,并在这里度过了余生。[21]

萨根从一开始就与美国太空计划有关联。从20世纪50年代开始,他一直担任 美国宇航局的顾问,其中一项职责是在阿波罗 宇航员登月前向他们做简报。萨根参与了许多探索太阳系的机器人航天器任务,并在许多探索任务中安排实验。萨根组装了第一个发送到太空的物理信息:太空探测器上的一块 镀金匾。 提倡 10,于1972年推出。 先锋 11岁也携带着另一枚牌匾,于次年发射升空。他继续改进他的设计:他帮助开发和组装的最详尽的信息是1977年与太空探测器航海者一起发射的 旅行者 黄金唱片。萨根经常质疑资助航天飞机 和国际空间站 机器人任务。[22]
\subsubsection{2.1 科学成就}
萨根的学生大卫·摩利逊 将萨根描述为“一个‘有想法的人’以及“直觉物理论证”和“信封背面计算”大师”,[19] 杰拉德·柯伊伯 说“一些人在实验室里专攻一个主要项目时工作得最好;另外一些人则是将各科学之间联系做得最好。萨根博士属于后一类。”[19]

在发现金星表面高温的研究中萨根的贡献是显著的。[23][23] 在20世纪60年代早期,没有人确切知道金星表面的基本情况,萨根在后来的一份报告中列出了这些可能性,该报告在后来的 《行星报告》 一书中广为流传。他自己的观点是,金星干燥且非常热,与其他人想象的温和天堂相反。他通过研究无线电波并得出结论,金星的表面温度为 500°C(900 °F)。作为美国宇航局喷气推进实验室的访问科学家 ,他促成了水手号对金星的第一次访问,并致力于项目的设计和管理。1962年 水手2号 证实了他关于金星表面状况的结论。

萨根是第一批假设土星的卫星泰坦其表面可能拥有大量液态水以及木星的卫星木卫二可能拥有地下水的人。这将使木卫二具有潜在的宜居性。[24] 木卫二的水下海洋后来被宇宙飞船伽利略间接证实 。泰坦微红薄雾的神秘也在萨根的帮助下得以解决。红色的薄雾是由于复杂的有机分子不断向土卫六的表面降雨形成的。[25]

萨根进一步提出了关于金星和木星大气的见解 ,以及火星的季节变化的原因 。他也意识到全球变暖作为一种日益增长的人为危险,并将其比作金星通过某种失控的温室效应自然地发展成一个炎热的、不利于生命生存的行星。[26] 萨根和他康奈尔的同事埃德温·欧内斯特·萨尔彼得推测木星云层中存在的生命,因为木星的大气层成分稠密,富含有机分子。他研究了火星表面观察到的颜色变化,并得出结论,它们不是大多数人认为的季节性原因或植物变化造成的, 而是风暴造成的表面灰尘的移动。

萨根还以他对外星生命可能性的研究而闻名,包括通过辐射基本化学物质产生氨基酸的实验演示 。[27][28]

1994年萨根获美国国家科学院颁发的“公共福利奖章”,以表彰他在科学应用于公共福利方面的杰出贡献。[29] 据报道,由于他的媒体活动使他在许多其他科学家中不受欢迎,他们拒绝他成为科学院院士。[30][31][32]

截至2017年萨根是被引用最多的SETI科学家和被引用最多的行星科学家之一。[33]
\subsubsection{2.2 《宇宙》:在电视上普及科学}
\begin{figure}[ht]
\centering
\includegraphics[width=6cm]{./figures/7d826423e7bf5bfa.png}
\caption{萨根在《宇宙》 中(1980)} \label{fig_Sagan_3}
\end{figure}
1980年,萨根与人合作撰写并讲述了获奖的13集PBS电视系列剧《宇宙:个人航行》(Cosmos: A Personal Voyage),这部电视系列剧成为美国公共电视史上收视率最高的电视剧。该节目已经在60个不同国家的至少5亿观众中播出。[33][33][34] 萨根的著作《宇宙》(Cosmos)就是为了配合这个系列而出版的。[35]

由于萨根因早期的科普畅销书《伊甸之龙》(The Dragons of Eden, 1977年获得 普利策奖)而广受欢迎,他被邀请为该剧撰写剧本并担任叙事工作。它的目标受众是普通观众,萨根觉得大众已经对科学失去了兴趣,部分原因是令人窒息的教育体系并不培养大众的科学兴趣。[36]

这13集的每一集都是聚焦于一个特定的主题或人,从而展示宇宙的协同作用。[36] 他们涵盖了广泛的科学课题,包括 生命起源 以及人类在地球上的角色。

该剧获得了艾美奖 和皮博迪奖,并将萨根从一个默默无闻的天文学家变成了流行文化偶像。[37] 时代杂志在节目播出后不久就刊登了一篇关于萨根的封面故事,称他为“节目的创作者、首席编剧和主持人兼叙述者”。[38] 2000年,“宇宙”发行了一套重新灌制的DVD。
\subsubsection{2.3 “数十亿”}
\begin{figure}[ht]
\centering
\includegraphics[width=8cm]{./figures/c849a593138fef8b.png}
\caption{萨根站在海盗号火星登陆器模型旁边。萨根与 Mike Carr 和Hal Masursky检查登陆器可能的登陆点。} \label{fig_Sagan_4}
\end{figure}
萨根经常受邀出席约翰尼·卡森主演的《今夜秀》。[39] 在 《宇宙》 播出后,他与流行语“数十亿,数十亿”联系在一起,尽管他从未在《宇宙》 系列使用这个词语。[40] 他更喜欢用“数以十亿计”。[41] 然而,卡森在模仿萨根时有时会用到这个短语。[42][43]

作为对萨根的幽默致敬,以及他对“数十亿”这流行语的联想,萨根 被定义为一个非常大的测量单位,在技术上相当于至少40亿(20亿加上20亿)的量级。[44][45][46]
\subsubsection{2.4 科学和批判性思维倡导}
\begin{figure}[ht]
\centering
\includegraphics[width=6cm]{./figures/1f86e20a13afd9a5.png}
\caption{行星协会的联合创始人,萨根前排右侧} \label{fig_Sagan_5}
\end{figure}
萨根表达自己想法的能力可以让许多人更好地理解了宇宙——同时强调了人类的价值和价值,以及地球相对于宇宙的相对重要性。1977年,他在伦敦发表了一系列皇家学会圣诞讲座。[47]

萨根是搜寻外星生命的支持者。他敦促科学界用射电望远镜接收来自潜在的外星智慧生命的信号。萨根如此有说服力,以至于到1982年,他收到了一份支持SETI计划的请愿书,并发表在《科学》(Science)杂志上,有70位科学家签署了请愿书,其中包括7位诺贝尔奖 得主。这标志着这个当时备受争议的领域声望大幅提升。萨根还帮助弗兰克·德雷克(Frank Drake)撰写了阿雷西博信息,这是1976年11月通过阿雷西博射电望远镜向太空发射的无线电信息,目的是让潜在的外星人了解地球。

萨根曾担任专业行星研究杂志《伊卡洛斯》(Icarus)的首席技术官长达12年。他是行星协会的联合创始人,也是SETI研究所理事会的成员之一。萨根曾担任2000年美国天文学会行星科学部的主席 ,作为美国地球物理联盟地球科学的行星学部门的主席 ,并担任美国美国科学进步协会天文学分会主席 。

在 冷战的顶峰时期,适逢保罗·克鲁岑 的”正午的黄昏”概念的提出,大规模的核交换可能引发核黄昏,并通过冷却地表,破坏地球上生态平衡时,萨根通过推动有关核战争影响的假设,参与了核裁军。1983年,他是后来的“TTAPS”模型(这篇研究论文后来为人所知)的五位作者中“S”,该模型首次使用了他的同事 Richard P. Turco(Richard P. Turco)提出的“核冬天”一词。[48] 1984年,他与人合著了《寒冷与黑暗:核战争后的世界》 一书。1990年,《无人思考的道路:核冬天和军备竞赛的结束》一书中解释了核冬天假说并提倡核裁军。萨根对利用媒体传播一个非常不确定的假设受到了广泛的怀疑和蔑视。大约从1983年萨根开始了与核物理学家爱德华·泰勒的私人通信 ,泰勒表示支持继续研究以确定核冬天假说的可信度。然而,萨根和泰勒的通信最终导致泰勒写道:“宣传者是利用不完整的信息产生最大说服力的人。我可以称赞你确实是一个优秀的宣传者,记住一个宣传者越好,他就应该越不像一个宣传者”。[49] 萨根的传记作者也评论说,从科学角度来看,核冬天对萨根来说是一个低谷,尽管从政治角度来说,在公众中普及了他的形象。[49]

成年的萨根仍然是科幻小说的粉丝,尽管他不喜欢不符合物理定律的故事(比如忽略平方反比定律),或者说,他不打算“把心思花在构建另一种未来上面”。[50] 他出版了一些科普书籍,比如《宇宙》,这本书向大众展现了个人航行这个概念,成为有史以来最畅销的英文科学书籍;《伊甸园之龙:对人类智力进化的思考》获普利策奖;以及布罗卡的大脑:对科学的浪漫思考。萨根也写了1985年最畅销的科幻小说的《接触》,基于他的妻子安·德鲁扬在1979年一起编剧的电影《治疗》 。但他并没有看到这本书的1997电影的改编版本,主演朱迪·福斯特赢得了1998年的雨果奖最佳戏剧表演奖。

萨根写了《宇宙》的续集《浅蓝暗点:展望人类的太空家园》,这本书被《纽约时报》评选为1995年的畅销书。1995年1月,他出现在公共广播公司 查理·罗斯 节目中。[22] 萨根为 斯蒂芬·霍金的《时间简史》写了序言。萨根还以科普闻名,他努力提高公众对科学的理解,他支持科学怀疑主义反对 伪科学,比如他揭穿了贝蒂和巴尼·希尔被绑架的真相。为了纪念萨根逝世十周年,萨根的学生大卫·摩利逊(David Morrison)在《怀疑论者》(Inquirer)一书中回顾了“萨根对行星研究、公众对科学的理解和怀疑论运动的巨大贡献”。[19]
\begin{figure}[ht]
\centering
\includegraphics[width=6cm]{./figures/b773119ff5304f04.png}
\caption{淡蓝点:旅行者号从16亿公里外(冥王星之外)拍摄到的地球是一个明亮的像素。萨根建议NASA拍摄这张照片。 from 暗淡蓝点 (1994)[42] 在这个暗淡蓝点上,聚集着我们每个人的欢乐和痛苦,成千上万种我们信奉的宗教,意识形态和经济学说,每一个猎人和觅食者,每一个英雄和懦夫,每一个文明的创造者和破坏者,每一个国王和农民,每一对相爱的年轻夫妇,每一个充满希望的孩子 ,每一个母亲和父亲,每一个发明家和探索者,每一个道德楷模,每一个腐败的政客,每一个超级巨星,每一位最高领导人,在人类历史的每一个圣人和罪人,都生活在这一个暗淡微光的灰尘的上。 想想那些浴血奋战的将军和帝王得来的光荣和胜利成就了他们成为那暗淡蓝点一瞬的主人。 卡尔萨根1994年在康奈尔大学的讲座} \label{fig_Sagan_6}
\end{figure}
随着萨达姆·侯赛因威胁突袭科威特油井,为了应对伊拉克对石油资产的任何物理形式的控制权,萨根和他的“TTAPS”同事保罗·克鲁岑,在1991年1月《巴尔的摩太阳报 》和 《威尔明顿晨星 》报纸上警告说,如果大火持续燃烧几个月,1991年的600起左右的科威特石油大火 就足够“可能会高到扰乱南亚大部分地区的农业”。并且这种可能性应该会“影响战争的计划”;[51][52] 这些主张也是萨根和物理学家弗雷德·辛格 1月22日,在 美国广播公司夜生活上电视辩论的主题。[53][54]

在电视辩论中,萨根认为烟雾的影响将类似于 核冬天的影响,辛格的观点与此相反。辩论结束后,大火燃烧了好几个月才熄灭。烟雾的结果并没有产生大陆范围内的冷却。萨根后来在 《恶魔出没的世界 》一书中承认了预测结果并不正确:“正午时分,天空一片漆黑,波斯湾上空的气温下降了4.6摄氏度,但没有多少烟雾升到平流层的高度,亚洲幸免于难 ”。[55]

萨根晚年主张创建一个有组织的小行星/近地物体搜索系统,可能影响地球但是阻止或推迟开发防御近地天体所需的技术方法也有负面影响。[56] 他认为,有许多方法可以改变小行星的轨道,包括使用的核爆炸,这也造就了一个两难境地:如果能够时小行星偏转远离地球,那么人也有能力使其转向地球,从了制造一种极具破坏性的武器。[57][58]1994年在他合著的一篇论文中,他嘲笑了 洛斯阿拉莫斯国家实验室 在1993长达3天的近地物体拦截讲习班中,“甚至没有顺便”提到这种拦截和偏转技术可能有这些“潜在危险”。[57]

萨根仍然希望,近地天体撞击威胁和防止这些威胁的方法本质上的双刃剑可以成为“成熟国际关系的新的和强有力的动力”。[57][59] 后来他承认,在充分的国际监督下,将来可以采用一种“逐步改进”的办法来实施核爆炸偏转方法,并在获得足够的知识后,利用这些方法来开采小行星。[58] 他对在太空中使用核爆炸的兴趣源于他1958年在 装甲研究基金会的 A119项目中关于在月球表面引爆核装置的可能性的工作。[60]

萨根是古希腊哲学家柏拉图的批判者:“科学和数学会在商人和工匠手中消失。这种倾向在毕达哥拉斯 的追随者柏拉图那里最为明显了”[61]

柏拉图相信想法比自然世界要真实得多。他建议天文学家不要浪费时间去观察恒星和行星。他相信,只要想到他们就可以进行更好的了解。柏拉图表示反对观察和实验。他教导人们蔑视现实世界,蔑视科学知识的实际应用。柏拉图的追随者们成功地熄灭了德谟克利特和其他爱奥尼亚人点燃的科学和实验之光。
\begin{figure}[ht]
\centering
\includegraphics[width=8cm]{./figures/6fc92c954a5b1e7d.png}
\caption{萨根承认他高估了1991年科威特石油大火的危害。} \label{fig_Sagan_7}
\end{figure}
\subsubsection{2.5 普及科学}
谈到他在普及科学方面的活动,萨根说,科学家分享科学的目的及其当代状态至少有两个原因。简单的利己主义是其中之一:科学研究的大部分资金来自公众,因此公众有权知道这些资金是如何使用的。如果科学家增加公众对科学的兴趣,就很有可能获得更多的公众支持者。[62] 另一个原因是把自己对科学的兴奋与他人交流得到的满足感。[63]
\subsubsection{2.6 批评}
虽然萨根受到公众的广泛爱戴,但他在科学界的声誉却两极分化得十分严重。[64] 批评家有时形容他的作品富于幻想,不严谨,自我夸大;[65] 其他人在他晚年抱怨说,他为了提升自己的名人地位而忽视了自己作为教师的角色。[66]

萨根最严厉的批评者之一哈罗德·乌里(Harold Urey)认为,作为一名科学家,萨根太过张扬,而且对一些科学理论的处理过于随意。[67] 戴维森说,乌里和萨根秉持不同的科学哲学。乌里是一位“旧时代的经验主义者”,他避免对未知事物进行理论化,而萨根则相反,他愿意公开猜测这些事情。[68] 弗雷德·惠普尔说:“我希望哈佛让萨根留下,但是我知道因为尤里是诺贝尔奖获得者,他的观点是哈佛拒绝萨根的一个重要因素。”[67]

萨根在哈佛的朋友莱斯特·格林斯彭还说:“我非常了解哈佛,知道那里有些人肯定不喜欢直言不讳的人。” 格林斯普补充道:

无论你走到哪里,都能看到一位天文学家的名言出现在每样东西上,一位天文学家的脸出现在电视上,还有一位天文学家的书在当地书店的陈列在最受欢迎的位置。

有些人,比如乌里,后来意识到萨根流行的科学宣传品牌的建立对整个科学界是有益的。 乌里特别喜欢萨根1977年的书 《伊甸园之龙》 ,他给萨根写道:“我非常喜欢它,我很惊讶像你这样的人对这个问题的各方面有如此深入的了解...我祝贺你...你是一个在很多方面都很有天赋的人。”

萨根被指控为了自己的利益而借用他人的一些观点,他反驳了这些说法,解释说,这种盗用是他作为科学传播者和讲解员角色的不幸的一面,他希望在任何可能的情况下能得到适当的信任。[67]
\subsubsection{2.7 社会关注}
萨根认为,根据在另一种合理估计的基础上提出的德雷克方程(Drake equation),会得到存在大量的外星文明的结论,但费米悖论 (Fermi paradox)所强调的这些文明缺乏证据,表明技术文明有自我毁灭的倾向。这激发了他对确定和宣传人类可以自我毁灭的方式的兴趣,希望能避免这样的灾难,并最终成为一个太空物种。在《宇宙》(Cosmos)的最后一集《谁为地球说话》(Who Speaks for Earth)中,有一段令人难忘的电影片段,表达了萨根对核浩劫可能对人类文明造成破坏的深切关注。萨根辞去了空军科学咨询委员会UFO调查委员会的职务,并自愿交出了他的绝密许可,以抗议越南战争。[69]1981年6月,萨根与第三任妻子(小说家安·德鲁安)结婚后,他在政治上变得更加活跃,特别是反对总统罗纳德·里根领导的核军备竞赛升级。
\begin{figure}[ht]
\centering
\includegraphics[width=6cm]{./figures/220d882d2a50d224.png}
\caption{美国和苏联/俄罗斯核武库,整个冷战和冷战后时代存在的核炸弹/弹头总数} \label{fig_Sagan_8}
\end{figure}