% SLISC 容器的基本运算

\begin{issues}
\issueDraft
\end{issues}

\verb|Long size(T_I v)| 相当于 \verb|std::vector| 的 \verb|.size()|, 但返回 \verb|Long| 类型, 这是为了避免使用 \verb|unsigned| 类型.

\verb|Tr sum_abs(T_I v)| 绝对值求和

\verb|Ts max(T_I v)|, \verb|Ts min(T_I v)| 最大元素和最小元素

\verb|Ts max(Long_O ind, T_I v)|, \verb|Ts min(Long_O ind, T_I v)| 最大元素和最小元素(同时求出其位置)

\verb|Tr max_abs(T_I v)| 最大绝对值

\verb|void mod(T_O v, T1_I v1, Ts1_I s)| 求余

\verb|void mod(T1_O v, Ts1_I s)| 求余

\verb|void real(T_O v, T1_I v1)| 提取实部

\verb|void imag(T_O v, T1_I v1)| 提取虚部

\verb|void abs(T_IO v)| 求绝对值

\verb|void abs(T_O v, T1_I v1)| 求绝对值

\verb|Tret sum(T_I v)| 求和

\verb|Tret prod(T_I v)| 求积

\verb|Tr norm2(T_I v)| 绝对值平方和

\verb|Tr norm(T_I a)|

\verb|void resize_cpy(Tv_IO v, Long_I N, T_I val = 0)|

\verb|void resize_cpy(Tv_IO v, Long_I N1, Long_I N2, T_I val = 0)|

\verb|void resize_cpy(Tv_IO v, Long_I N1, Long_I N2, Long_I N3, T_I val = 0)|

\verb|void linspace(Tv_O v, Ts_I first, Ts_I last)|

\verb|void flip(Tv_IO v)|

\verb|void flip(T_O v, T1_I v1)|

\verb|void reorder(Tv_O v, To_I order)|

\verb|void trans(T_IO v)|

\verb|void trans(T_O v, T1_I v1)|

\verb|void conj(T_IO v)|

\verb|void conj(T_O v, T1_I v1)|

\verb|void her(T_IO v)|

\verb|void her(T_O v, T1_I v1)|

\verb|void operator+=(T_IO v, Ts_I s)|, \verb|void operator-=(T_IO v, Ts_I s)|, \verb|void operator*=(T_IO v, Ts_I s)|, \verb|void operator/=(T_IO v, Ts_I s)|

\verb|void operator+=(T_IO v, Ts_I s)|, \verb|void operator-=(T_IO v, Ts_I s)|, \verb|void operator*=(T_IO v, Ts_I s)|, \verb|void operator/=(T_IO v, Ts_I s)|

\verb|void operator+=(T_O &v, T1_I v1)|, \verb|void operator+=(T_O &v, T1_I v1)|, \verb|void operator-=(T_O &v, T1_I v1)|, \verb|void operator*=(T_O &v, T1_I v1)|, \verb|void operator/=(T_O &v, T1_I v1)|

\verb|void plus(T_O v, T1_I v1, Ts2_I s)|, \verb|void minus(T_O v, T1_I v1, Ts2_I s)|, \verb|void times(T_O v, T1_I v1, Ts2_I s)|, \verb|void divide(T_O v, T1_I v1, Ts2_I s)|

\verb|void pow(T_IO v, Ts_I s)|

\verb|void pow(T_O v, T1_I v1, Ts_I s)|

\verb|T dot(T1_I v1, T2_I v2)|

\verb|void cumsum(T_O v, T1_I v1)|

\verb|void mul(T_O y, T1_I a, T2_I x)|

\verb|void mul(T_IO y, T1_I a, T2_I x, Ts1_I alpha, Ts_I beta)|

\verb|void mul(T_O &y, T1_I x, T2_I a)|

\verb|void mul(T_O y, T1_I a, T2_I x)|

\verb|void mul_gen(Ty_O &y, Ta_I a, Tx_I x, Tsa_I alpha = 1, Tsa_I beta = 0)|

\verb|void mul_sym(Ty_IO &y, Ta_I a, Tx_I x, Tsa_I alpha = 1, Tsa_I beta = 0)|

\verb|void mul_gen(Ty_O y, Ta_I a, Tx_I &x)|

\verb|void uniq_elm(T_IO v)|

\verb|void uniq_rows(T_O a, T1_I a1)|

\verb|void exp(T_IO v)|

\verb|void exp(T_O v, T1_I v1)|
