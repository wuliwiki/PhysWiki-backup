% 光电离时间延迟:一维波包、氢原子与氦原子
% license Xiao
% type Tutor

\begin{issues}
\issueDraft
\end{issues}

\pentry{量子散射的延迟\nref{nod_tDelay}, 库仑函数\nref{nod_CulmF}, 选择定则\nref{nod_SelRul}, 含时微扰理论\nref{nod_TDPTc}}{nod_2298}

本文使用\enref{原子单位制}{AU}。 例如一个有限深势阱(短程势)中有一个束缚态, 被一个电场波包电离之后, 光电子波包逃出势阱。 那么光电子波包的延迟是多少呢? 我们以下使用\enref{一阶含时微扰理论}{TDPT}来分析。

\subsection{一维势阱的光电离延迟}
我们假设势阱在坐标原点, 且 $t = 0$ 时电场波包的中心刚好到达原点。

初态和末态能量分别为 $E_0, E$, 令 $\omega = E - E_0$, 不含时的末态记为 $\psi_k(x)$, 并令其为满足边界条件
\begin{equation}\label{eq_HeAna2_2}
\psi_k(x \to +\infty) = \frac{1}{\sqrt{2\pi}}\E^{\I kx}~,
\end{equation}
其中 $A(k)$ 是实函数。 那么一阶微扰系数为(\autoref{eq_TDPTc_1}~\upref{TDPTc})
\begin{equation}
c(k) = -\I \sqrt{2\pi} \mel{\psi_k}{H'}{\psi_0} \tilde f(-\omega)~.
\end{equation}
电离波包可以由末态展开:
\begin{equation}
\psi(x, t) = \int_{-\infty}^{\infty} c(k) \psi_k(x)\E^{-\I E t} \dd{k}~.
\end{equation}
那么根据\autoref{eq_tDelay_4}~\upref{tDelay} 延迟为($k$ 视为 $E$ 的函数)
\begin{equation}\label{eq_HeAna2_3}
t = \pdv{E} \arg (c(k))~.
\end{equation}
对于没有 chirp 的电场波包, $\tilde f(-\omega)$ 相位恒定(想象一个高斯波包乘以正弦函数的傅里叶变换)。 所以随 $E$ 变化的相位只有矩阵元, 绝对时间延迟为
\begin{equation}\label{eq_HeAna2_1}
t = \pdv{E} \arg \mel{\psi_k}{H'}{\psi_0}~,
\end{equation}
这个延迟是相对与 $t = 0$ 也就是电磁波包到达原点的时间。

\subsection{氢原子的光电离延迟}\label{sub_HeAna2_1}
还是考虑类氢原子, 库仑球面波(\autoref{eq_CulmWf_11}~\upref{CulmWf})记为 $\ket{C_{l,m}(k)}$, 包含球谐函数 $Y_{l,m}(\uvec r)$。 含时微扰得($E = k^2/2$)
\begin{equation}\label{eq_HeAna2_4}
\ket{\psi(t)} = \sum_{l,m} \int c_{l,m}(k) \ket{C_{l,m}(k)} \E^{-\I E t} \dd{k}~,
\end{equation}
其中系数为
\begin{equation}
c_{l,m}(k) = -\sqrt{2\pi}\I \mel{C_{l,m}(k)}{z}{i} \tilde f(-\omega)~.
\end{equation}
相位不随 $k$ 或 $E$ 变化。 注意对于氢原子束缚态作为初始态, 由于\enref{选择定则}{SelRul} $\Delta m = 0$, $\Delta l = \pm 1$ (只有这时 $\mel{C_{l,m}(k)}{z}{i}$ 不为零), 只有最多两个 $c_{l,m}$ 不为零。 如果从基态电离就只有 $c_{1,0} = 1$(线偏振光)。

库仑平面波为 $\ket{C(\bvec k)} = \sum_{l,m} a_{l,m}^{(-)}(\bvec k) \ket{C_{l,m}}$, 把 $\ket{\psi(t)}$ 投影到 $\ket{C(\bvec k)}$ 上有(\autoref{eq_CulmWf_10}~\upref{CulmWf})
\begin{equation}
\braket{C(\bvec k)}{\psi(t)} = \sum_{l,m} a_{l,m}^{(-)*}(\bvec k) c_{l,m}(k) \E^{-\I E t} 
= \sum_{l,m} \frac{\I^{-l}}{k} \E^{\I\sigma_l} Y_{l,m} (\uvec k) c_{l,m}(k) \E^{-\I E t}~.
\end{equation}
求和中每一项可以看作一个独立的波包, 具有不同的延迟。 所以和\autoref{eq_HeAna2_3} 同理, 每一项的延迟仅由库仑相移贡献($\mel{C_{l,m}(k)}{z}{i}$ 是实数, $c_{l,m}(k)$ 不贡献相位变化)
\begin{equation}\label{eq_HeAna2_6}
t_{EWS}^l = \pdv{E} \arg\braket{C(\bvec k)}{\psi(t)} = \pdv{\sigma_l}{E}~.
\end{equation}
根据\autoref{fig_HeAna2_1} 它恒大于零,随能量单调递减。
\begin{figure}[ht]
\centering
\includegraphics[width=12cm]{./figures/cd0a33d6ef1fcfb1.png}
\caption{\autoref{eq_HeAna2_6} } \label{fig_HeAna2_2}
\end{figure}

但是\autoref{eq_HeAna2_4} 中库仑平面波本身还存在关于 $r$ 不收敛的相移 $-\eta\ln (2kr)$, 这对能量求导以后对足够大的 $r$ 会得到小于零的延迟, 即提前。 从经典物理的角度来看, 提前是因为一开始电子在势阱中移动的速度比无穷远处更快,节约了一些时间。 重申,对于氢原子基态的线性光电离, 只需要 $l = 1$, 结果与方向无关。 所以氢原子 streaking 电离的角向变化完全是由 $t_{CLC}$ 决定的。

库仑渐进相位为(\autoref{eq_CulmF_7}~\upref{CulmF})
\begin{equation}
F_l(k, r) \overset{r\to \infty}{\longrightarrow} \sin\qty[kr - \frac{\pi l}{2} + \frac{Z}{k}\ln(2kr) + \sigma_l(k)]~, % Z > 0
\end{equation}
\begin{equation}
\sigma_l(k) = \arg[\Gamma(l+1-\I Z/k)] = \Im\ln\Gamma(l+1-\I Z/k) \qquad (Z > 0)~.
\end{equation}
\begin{figure}[ht]
\centering
\includegraphics[width=10cm]{./figures/bbe2fecdf936361f.png}
\caption{库仑相移 $\sigma_l(k)$} \label{fig_HeAna2_1}
\end{figure}

所以
\begin{equation}\label{eq_tEWS}
\begin{aligned}
t_{EWS}^C &= \frac{1}{k} \Im\dv{k}\ln\Gamma(l+1-\I Z/k) = \frac{1}{k} \Im\frac{\dv{k}\Gamma(l+1-\I Z/k)}{\Gamma(l+1-\I Z/k)}\\
&= \frac{Z}{k^3} \Re\frac{\Gamma'(l+1-\I Z/k)}{\Gamma(l+1-\I Z/k)}~.
\end{aligned}
\end{equation}
另外 $\Gamma'(x)/\Gamma(x)$ 被称为\enref{双 $\Gamma$ 函数}{digama}。

其中 $\sigma_l(k)$ 产生的相移与距离无关, 可以单纯看作是非直线轨道带来的延迟(不确定), 并且是 XUV 波包一瞬间带来的。 $kr\ln(2kr)$ 只和距离有关, 带来的延迟是库仑力产生的, 而且关于 $r$ 并不收敛。 那么光电离的总延迟就是
\begin{equation}\label{eq_HeAna2_5}
t = \pdv{E} \arg\qty[\sigma_l(k) + \frac{Z}{k}\ln(2kr)]~. % Z > 0
\end{equation}
第一项是瞬间的, 而第二项是波包移动过程中缓慢积累且不收敛的。

一定要强调光电离延迟 $\tau$ 和 streaking 延迟是不一样的。 在 Streaking 实验中, 前两项都会直接转换为 streaking 延迟(因为电离瞬间产生)。 第三项在上式中不收敛, 但 streaking 延迟却总是收敛的。 配合 IR 的扰动就变为了 $t_{CLC}$ 项。

\subsection{氢原子使用库仑平面波基底}\label{sub_HeAna2_2}
对于氢原子, 一阶微扰也可以直接投影到库仑平面波(\autoref{eq_CulmWf_7}~\upref{CulmWf}, 取减号) $\ket{C(\bvec k)}$ ($E = k^2/2$)
\begin{equation}
\ket{\psi(t_f)} = \int c(\bvec k)\ket{C(\bvec k)} \E^{-\I E t} \dd[3]{k}~,
\end{equation}
\begin{equation}
c(\bvec k) = -\sqrt{2\pi}\I\mel{C(\bvec k)}{z}{i} \tilde f(-\omega)~.
\end{equation}
这相对于\autoref{eq_HeAna2_4} 不过是变换了一下基底。 根据边界条件\autoref{eq_CulmWf_4}~\upref{CulmWf}, $C(\bvec k)$ 本身也自带一个不收敛的相位 $-\eta \ln 2kr$, 而电场的傅里叶变换 $\tilde f(-\omega)$ 相位恒定, 所以对某个方向 $\uvec k$, 有
\begin{equation}
\tau = \pdv{E} \arg\qty[\mel{C(\bvec k)}{z}{i} + \frac{Z}{k}\ln(2kr)]~.
\end{equation}
这和\autoref{eq_HeAna2_5} 只是使用的基底不同, 所以是等价的。
\addTODO{用 Matlab 的数值计算验证}

注意\autoref{eq_HeAna2_5} 和初态无关, 所以这里同样无关。 但实际上 TDSE 的结果是有关系的, 要考虑初态, 只有使用 Pazourek 的 dDLC 修正项。

\subsection{氦原子}
\pentry{角动量的叠加 2(量子力学)\nref{nod_AMAdd}}{nod_c791}
氦原子的延迟完全也可以分别使用\autoref{sub_HeAna2_1} 和\autoref{sub_HeAna2_2} 两种方法。 这是完全一样的。 一阶微扰的末态必须是能量本征态, 氦原子和 $H$ 对易的算符只有 $L^2, L_z$, 所以精确的球面波能量本征态记为 $\ket{E,L,M}$。 它可以分解为
\begin{equation}
\ket{E,L,M} = \frac{1}{r_1r_2}\sum_{l_1,l_2}\psi_{l_1,l_2}^{L, M}(r_1, r_2)\mathcal Y_{l_1,l_2}^{L, M}(\uvec r_1, \uvec r_2)~.
\end{equation}
理论上要解本征态 $\ket{E,L,M}$, 就用许多无限长的长方形网格, 每张网格是一个 $(l_1,l_2,L)$ 分波, 网格之间被 $1/r_{12}$ 势能耦合。 网格上 $r_1$ 在区间 $(0,a]$, $r_1 = a$ 处边界条件是波函数为 “零”, 另外两条边 $r_1 = r_2 = 0$ 也有波函数为零。 $r_2$ 的区间为 $(0,\infty)$, 这样才能保证 $E$ 可以连续取值而不是离散的。 由于边界条件和耦合项都是实函数, 解出的 $\psi_{l_1,l_2}^{L, M}(r_1, r_2)$ 也必是实函数。

但波函数有 6 维度, 我们只确定了三个量子数 $H,L,M$, 所以还可以规定在 $r_2\to\infty$ 时, 某个 $(l_1,l_2)$ 分波是束缚态和库仑波函数($Z=1$)的乘积, 其他分波无穷远处都为零
\begin{equation}
\begin{aligned}
&\psi_{l'_1,l'_2}^{L, M}(r_1, r_2) \overset{r_2\to\infty}{\longrightarrow} \delta_{l_1,l'_1}\delta_{l_2,l'_2} r_1 R_{n_1,l_1}^{(Z=2)}(r_1)
\sin\qty[k_2 r_2 - \frac{\pi l_2}{2} +\frac{1}{k_2}\ln(2k_2 r_2) + \sigma_{l_2} + \delta_{n_1,l_1,l_2}^{L,M}]~.
\end{aligned}
\end{equation}
最后一项就是库仑函数 $F_{l_2}(k_2 r_2 + \delta_{n_1,l_1,l_2}^{L,M})$ 的渐进形式。 这样就可以把这个态记为 $\ket{n_1,L,M,l_1,l_2,k_2}$。 在 Pazourek 的论文中提到一个 R-matrix 方法可以解出这样的波函数\footnote{Philip G. Burke, R-Matrix Theory of Atomic Collisions - Application to Atomic, Molecular and Optical Processes, Springer}。 这个态仍然只有 $E,L,M$ 是 well defined。 如果把这些态用 CG 系数线性组合一下, 就可以得到
\begin{equation}
\begin{aligned}
&\ket{n_1,l_1,m_1,l_2,m_2,k_2}
= 2\I \sum_{L} \bmat{l_1& l_2& L\\ m_1& m_2& M} \E^{-\I \delta_{n_1,l_1,l_2}^{L,M}} \ket{n_1,L,M,l_1,l_2,k_2}~,
\end{aligned}
\end{equation}
渐进形式为:
\begin{equation}
\ket{n_1,l_1,m_1, l_2,m_2,k_2} \overset{r_2\to\infty}{\longrightarrow} r_1 R_{n_1,l_1}(r_1) F_{l_2}(r_2 + \delta_{l_2})~.
\end{equation}
注意此时除了能量, 里面的量子数都只是在 $r_2\to+\infty$ 处是 well defined。

微扰理论中的矩阵元可以用(注意对称化) $\mel{n_1,l_1,m_1, l_2,m_2,k_2}{z_1 + z_2}{i}$, 这同样是实数, 剩下的论述就和氢原子的一样了。 要获得平面波出射的散射态, 就做线性组合
\begin{equation}
\ket{n_1, l_1, m_1, \bvec k_2} = \sum_{l_2,m_2}\frac{\I^{l_2}}{k_2} \E^{-\I(\sigma_{l_2}+\delta_{l_2})} Y_{l_2,m_2} (\uvec k_2) \ket{n_1,l_1,m_1, l_2,m_2,k_2}~,
\end{equation}
这个态中同样只包含一个库仑 $-\eta\ln 2kr$ 相位。 把一阶微扰的末态投影到上面后, 光电离的总延迟就是
\begin{equation}
t = \pdv{E} \arg\qty[\delta_{l_2} + \sigma_{l_2} + \frac{Z}{k_2}\ln(2k_2 r_2)]~, % Z > 0
\end{equation}
比起氢原子多了一项 $\delta_{l_2}$。 事实上如果给氢原子加上一个 SAE 势能同样也会多出这样一个相位。

和上文氢原子同理, 也可以直接以 $\ket{n_1, l_1, m_1, \bvec k_2}$ 作为基底计算一阶微扰, 结果相同:
\begin{equation}
\pdv{E} \arg\mel{n_1, l_1, m_1, \bvec k_2}{z_1 + z_2}{i} = \pdv{E} (\delta_{l_2} + \sigma_{l_2})~.
\end{equation}

\subsubsection{选择定则}
但氦原子的本征态是 $L,M$ 算符的本征态, 而且从基态的一阶微扰是 $L = 1, M = 0$。所以根据三角形法则(\autoref{fig_AMAdd_2}~\upref{AMAdd}), 可以支持 $\abs{l_2 - l_1} \le 1$ 除了 $(0,0)$ 的所有分波。

\subsection{回收的内容}
\addTODO{如果是长程库仑势能, 延迟就会取决于距离而不收敛。 但动量却收敛, 所以使用 streaking 仍然会获得一个延迟, 但这个延迟和上面的是两码事, 然而 Pazourek 仍然使用\autoref{eq_HeAna2_1}, 这里面还有更深的奥妙……}

如果\autoref{eq_HeAna2_2} 中本来就有额外的取决于能量的相位 $\delta(E)$ 那么同样需要加到\autoref{eq_HeAna2_1} 中
\begin{equation}
\tau = \pdv{E} \arg \mel{\psi_E}{H'}{\psi_0} + \pdv{E} \delta(E)~.
\end{equation}
并且这个相位可能还会取决于距离和其他参数 $\delta(E, x)$。 例如库仑相移中, 这个相位产生的延迟并不随距离收敛。

\addTODO{另一个问题是, 延迟不仅和 He+ 的 $n$ 有关还与 $l$ 有关, 或者说还与 Stark 效应的好量子数有关, 那么 Pazourek 使用的是哪个呢? 还是说取平均呢? 还是要仔细看 Pazo12}
