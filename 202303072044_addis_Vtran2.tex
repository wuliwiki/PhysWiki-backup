% 速度的坐标系变换

\pentry{速度的参考系变换\upref{Vtrans}, 圆周运动的速度\upref{CMVD}, 三维旋转矩阵\upref{Rot3D}}

\subsection{有相对转动}
对于任意两个坐标系, 他们之间的相对运动除了平移可能还有转动, 即 $\uvec x, \uvec y, \uvec z$ 和 $\uvec x', \uvec y', \uvec z'$ 之间的关系可能随时间变化。 这时\autoref{Vtrans_eq1}~\upref{Vtrans} 是否仍然成立呢?

要回答这个问题我们首先要修改 $\bvec v_r$ 的定义。 按照上一节的定义, 如果坐标系间存在相对转动, $\bvec v_r$ 将与固定点的位置有关。 若定义某时刻点 $P$ 在 $S'$ 中的坐标为 $(x_p', y_p', z_p')$, 则 $\bvec v_r$ 可以定义为 $S'$ 系中的固定点 $(x_p', y_p', z_p')$ 相对于 $S$ 系的瞬时速度。 这时仍有
\begin{equation}\label{Vtran2_eq2}
\bvec v = \bvec v' + \bvec v_r~.
\end{equation}
证明见下文。 再次强调, 这三个矢量也表示几何矢量。 若要记为坐标的形式需要使用同一坐标系。

\begin{example}{}\label{Vtran2_ex1}
令 $S'$ 系 $t = 0$ 时与 $S$ 系重合并绕 $z$ 轴逆时针以恒定角速度 $\omega$ 相对 $S$ 转动, 又令点 $P$ 的运动方程为 $\bvec r(t) = \alpha t \uvec x'$, 验证\autoref{Vtran2_eq2}。

首先将 $\bvec r(t)$ 用 $\uvec x, \uvec y$ 基底表示为
\begin{equation}
\bvec r(t) = \alpha t (\cos\omega t\, \uvec x + \sin\omega t\, \uvec y)~.
\end{equation}
将 $\uvec x, \uvec y, \uvec z$ 视为常矢量, $\bvec r(t)$ 关于时间求导得点 $P$ 相对于 $S$ 系的速度
\begin{equation}
\bvec v = \alpha (\cos\omega t \,\uvec x + \sin\omega t \,\uvec y)
+ \alpha\omega t (-\sin\omega t \,\uvec x + \cos\omega t \,\uvec y)~.
\end{equation}
将 $\uvec x', \uvec y', \uvec z'$ 视为常矢量, $\bvec r(t)$ 关于时间求导得点 $P$ 相对于 $S'$ 系的速度
\begin{equation}
\bvec v' = \dv{t} (\alpha t \uvec x') = \alpha \uvec x' = \alpha (\cos\omega t\, \uvec x + \sin\omega t \,\uvec y)~.
\end{equation}
最后, $t$ 时刻两坐标系在点 $P$ 处的相对速度(见\autoref{CMVD_eq5}~\upref{CMVD} )为
\begin{equation}\label{Vtran2_eq3}
\begin{aligned}
\bvec v_r &= \bvec\omega \cross \bvec r = (\omega \uvec z') \cross (\alpha t \uvec x') \\
&= \alpha\omega t \uvec z'\cross\uvec x' = \alpha\omega t \uvec y' = \alpha\omega t(-\sin \omega t \,\uvec x + \cos\omega t \,\uvec y)~.
\end{aligned}\end{equation}
将以上三式代入\autoref{Vtran2_eq2} 可验证其成立。 注意以上我们将所有的矢量用 $\uvec x, \uvec y, \uvec z$ 基底, 同理我们也可以将所有矢量用 $\uvec x', \uvec y', \uvec z'$ 基底展开。
\end{example}

一般情况下, 相对速度 $\bvec v_r$ 可以拆分成 $S'$ 的原点 $O'$ 在 $S$ 中的速度, 以及 $S'$ 相对 $S$ 旋转产生的速度(\autoref{CMVD_eq5}~\upref{CMVD})两部分, 根据\autoref{Vtrans_eq1}~\upref{Vtrans} 有
\begin{equation}\label{Vtran2_eq4}
\bvec v_r = \bvec v_{O'} + \bvec \omega \cross \bvec r'~.
\end{equation}
\textbf{证明}: 可以先在 $O'$ 处建立一个相对 $S$ 只平动不转动的坐标系 $S''$, 然后考虑 $S'$ 中的固定点相对 $S''$ 以速度 $\bvec \omega \cross \bvec r'$ 旋转即可。 证毕。

所以\autoref{Vtran2_eq2} 也可以进一步写成
\begin{equation}\label{Vtran2_eq7}
\bvec v = \bvec v_{O'} + \bvec \omega \cross \bvec r' + \bvec v'~.
\end{equation}

\subsubsection{证明}
现在用旋转变换矩阵\upref{Rot3D}在 $S$ 系中证明\autoref{Vtran2_eq2} 即\autoref{Vtran2_eq7}。

令 $S'$ 中坐标到 $S$ 坐标的旋转变换矩阵\upref{Rot3D}为 $\mat R$, 即对任意几何矢量 $\bvec A$ 有
\begin{equation}
(\bvec A)_S = \mat R (\bvec A)_{S'}~, \qquad
(\bvec A)_{S'} = \mat R\Tr (\bvec A)_S
\end{equation}
那么令 $\bvec r_{O'}$ 为 $S$ 的原点指向 $S'$ 的原点的几何矢量, 令 $\bvec r, \bvec r'$ 分别为质点在两参考系中的位置矢量, 有
\begin{equation}
\bvec r = \bvec r_{O'} + \bvec r'~.
\end{equation}
用 $S$ 中的坐标来表示, 就是
\begin{equation}
(\bvec r)_S = (\bvec r_{O'})_S + (\bvec r')_S = (\bvec r_{O'})_S + \mat R(\bvec r')_{S'}~.
\end{equation}
两边对时间求导得(未完成:引用矩阵乘法的求导法则)
\begin{equation}
\begin{aligned}
(\bvec v)_S &= ({\bvec v}_{O'})_S + \dot{\mat R}(\bvec r')_{S'} + \mat R (\bvec v')_{S'}\\
&= ({\bvec v}_{O'})_S + \dot{\mat R}(\bvec r')_{S'} + (\bvec v')_S~.
\end{aligned}
\end{equation}
其中第二项为(详见 “旋转矩阵的导数\upref{RotDer}”)
\begin{equation}
\dot{\mat R}(\bvec r')_{S'} = \dot{\mat R}\mat R\Tr (\bvec r')_S = \mat \Omega (\bvec r')_S = (\bvec \omega \cross \bvec r')_S~,
\end{equation}
代入得\autoref{Vtran2_eq7}, 证毕。
