% 同余与剩余类
% keys 同余|剩余类
% license Usr
% type Tutor

\pentry{整除\nref{nod_divisb}}{nod_e98f}

\begin{definition}{同余与剩余}
若(整数) $m$ 是(整数) $x$ 减去(整数) $a$ 的一个因子,即 $m | (x-a)$,则称 $x$ 与 $a$ 模\footnote{此处的“模”对应英文 $\operatorname{mod}$,与整数模无关。} $m$ \textbf{同余},记作
\begin{equation}
x \equiv a \pmod m ~.
\end{equation}
同时,若 $x \equiv a \pmod m$,就 说 $a$ 是作 $x$ 模 $m$ 的一个\textbf{剩余(residue)}。若还有 $0 \le a < m$,就说 $a$ 是作 $x$ 模 $m$ 的\textbf{最小剩余(least residue)}。
\end{definition}
有时候会忽略整数条件,此时可以说 $\pi \equiv -\pi \pmod{2\pi}$。即只要 $x-a$ 是 $m$ 的整数倍即可。

\begin{definition}{剩余类}
在模 $m$ 的前提下,与给定的某个剩余关于 $m$ 同余的所有数将组成一个集合,称为模 $m$ 的一个\textbf{剩余类(class of residue)}。而这剩余类中的每个元素都叫做这个类的一个\textbf{代表(represent)}。
\end{definition}

\begin{corollary}{}
显然,$m$ 共有 $m$ 个不同的剩余类,分别有代表
\begin{equation}
0, 1, \dots, (m-1) ~.
\end{equation}
\end{corollary}
\begin{definition}{完全剩余系}
对于模 $m$ 的前提下,将会有 $m$ 个不同的剩余系分别有代表 $0, 1, \dots, (m-1)$。对于任意 $m$ 个分别属于这 $m$ 个不同的剩余系的数,这 $m$ 个数组成的集合都称为\textbf{模 $m$ 的一个完全剩余系},简称模 $m$ 的一个\textbf{完系(complete system)}。
\end{definition}

