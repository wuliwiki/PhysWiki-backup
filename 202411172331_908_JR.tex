% 詹姆斯·焦耳(综述)
% license CCBYSA3
% type Wiki

本文根据 CC-BY-SA 协议转载翻译自维基百科\href{https://en.wikipedia.org/wiki/James_Prescott_Joule}{相关文章}。

詹姆斯·普雷斯科特·焦耳 (James Prescott Joule) FRS FRSE (/dʒuːl/;[1][2][a] 1818年12月24日—1889年10月11日) 是一位英国物理学家、数学家和酿酒师,出生于兰开夏郡的索尔福德。焦耳研究了热的本质,并发现了热与机械功之间的关系。这一发现促成了能量守恒定律的提出,进而导致了热力学第一定律的发展。国际单位制中能量的导出单位“焦耳”(joule)即以他的名字命名。

他与开尔文勋爵(Lord Kelvin)合作,开发了一个绝对热力学温标,后被称为开尔文温标(Kelvin scale)。焦耳还观察到了磁致伸缩现象,并发现了电阻中的电流与其散热之间的关系,即焦耳第一定律(Joule's first law)。他关于能量转化的实验成果首次发表于1843年。

\subsection{早年生活} 
詹姆斯·焦耳出生于1818年,其父本杰明·焦耳(Benjamin Joule,1784–1858)是一位富有的酿酒师,母亲是爱丽丝·普雷斯科特(Alice Prescott)。他出生在索尔福德的新贝利街(New Bailey Street)。[3] 焦耳年轻时接受了著名科学家约翰·道尔顿(John Dalton)的指导,并深受化学家威廉·亨利(William Henry)以及曼彻斯特工程师彼得·尤尔特(Peter Ewart)和伊顿·霍奇金森(Eaton Hodgkinson)的影响。他对电学充满了兴趣,他和兄弟曾通过相互施加电击以及给家里的仆人施加电击进行实验。[4]  

成年后,焦耳接管了家族的酿酒业务,科学对他来说只是一个严肃的爱好。大约在1840年,他开始研究用新发明的电动机取代酿酒厂的蒸汽机的可行性。他关于这一主题的首批科学论文发表于威廉·斯特金(William Sturgeon)的《电学年鉴》(*Annals of Electricity*)。焦耳还是伦敦电学会(London Electrical Society)的成员,该学会由斯特金等人创立。[需要引用]  

部分出于商人对经济效益量化的需求,部分出于科学上的好奇心,焦耳开始研究哪种原动机效率更高。1841年,他发现了“焦耳第一定律”:“\textbf{任何伏打电流的正确作用所产生的热量,与该电流强度的平方乘以其所经历的导电阻力成正比。}”[5] 他进一步认识到,在蒸汽机中燃烧一磅煤比在电池中消耗一磅昂贵的锌更为经济。焦耳用一个通用标准来衡量不同方法的输出——将一磅重的物体提升一英尺的能力,即“英尺-磅”标准。[需要引用]  

然而,焦耳的兴趣逐渐从单纯的财务问题转向探讨从某一给定能源中可以提取多少功,这使他开始思考能量的可转化性。1843年,他发表了实验结果,显示他在1841年量化的加热效应是由于导体内的热量生成,而非热量从设备的其他部分转移而来。这一发现直接挑战了热质理论,该理论认为热量既不能被创造也不能被销毁。自1783年安托万·拉瓦锡(Antoine Lavoisier)提出以来,热质理论一直在热学领域占据主导地位。拉瓦锡的声望以及萨迪·卡诺(Sadi Carnot)自1824年以来在热机理论中的成功实践,确保了热质理论的影响力。然而,年轻的焦耳既不属于学术界,也不属于工程专业,他的研究因此面临重重困难。热质理论的支持者通常会指出珀尔帖–塞贝克效应的对称性,以此声称热量和电流在一定程度上是可以通过可逆过程相互转化的。[需要引用]  

\subsection{热的机械当量}  
通过对电动机的进一步实验和测量,焦耳估算出热的机械当量为每卡路里4.1868焦耳,即将一克水的温度升高一开尔文所需的功。[b] 他在1843年8月于科克举行的英国科学促进会化学分会会议上宣布了这一结果,却遭遇了一片沉默。[7]  

焦耳并未气馁,他开始寻求一种纯机械的方式来证明功可以转化为热。他通过将水强制流过一个带孔的圆筒,测量液体因粘性加热而产生的轻微升温。他得到了机械当量为每英热单位770英尺磅力(4,140 J/Cal)。焦耳认为,电学方法和纯机械方法所获得的数值在至少两位有效数字上相符,是功可以转化为热的有力证据。  

“\textbf{无论机械力如何消耗,总会产生一个完全等价的热量。}”

——J.P. 焦耳,1843年8月  

焦耳尝试了第三种方法,他测量了压缩气体时产生的热量与所做功之间的关系。他得到的机械当量是每英热单位798英尺磅力(4,290 J/Cal)。在许多方面,这个实验为焦耳的批评者提供了最容易攻击的目标,但焦耳通过巧妙的实验消除了预期的反对意见。1844年6月20日,焦耳在皇家学会宣读了他的论文,[8][9] 但皇家学会拒绝发表这篇论文,最终他不得不于1845年在《哲学杂志》(*Philosophical Magazine*)上发表。[10]  

在论文中,焦耳明确拒绝了卡诺(Carnot)和埃米尔·克拉佩龙(Émile Clapeyron)的热质理论,这种拒绝部分源于神学驱动:[需要引用]  

“\textbf{我认为,这一理论……与被公认为哲学原则的观点相悖,因为它得出了一种结论,即动生力(vis viva)可能由于装置的不当配置而被销毁。例如,克拉佩龙先生得出结论,‘火炉的温度比锅炉的温度高出1000°C到2000°C,在热量从火炉传到锅炉的过程中,存在巨大的动生力损失。’我相信,只有造物主才拥有摧毁的能力,因此我断言……任何理论如果在实施时要求力的消失,必然是错误的。}”

在这里,焦耳采用了“动生力”(即能量)的语言,这可能是因为霍奇金森(Hodgkinson)在1844年4月为文学与哲学学会(*Literary and Philosophical Society*)宣读了尤尔特(Ewart)的《论动能的测量》(*On the measure of moving force*)的一篇评论。[需要引用]  