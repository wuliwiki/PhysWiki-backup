% 角动量
% 角动量

\begin{issues}
本词条有误,需大改
\end{issues}
\subsection{物体最一般的运动}
处于一个空间当中的物体,若不考虑物体发生形变,物体最一般的运动是怎样的呢?(不发生形变指物体中任意两点的位置关系不发生任何改变,这样的物体称\textbf{刚体},本文所提“物体”均指刚体)

物体的运动可用物体在空间中的位置随时间的变化来描述,而在观察物体的运动时,我们总是在某一时刻观察物体的位置,紧接着在下一时刻继续观察物体的位置,在观察时间段里,当知道了每一个观察时刻物体对应的位置,我们就认为我们知道了物体的运动,为了更精细的得到物体的运动状况,我们只是要求两相邻观察时刻的时间间隔能充分的小.然而无论如何,我们只是在将一段时间用很多时刻点来进行划分,并观察每一时刻物体的位置.这是因为在一段时间内所能做到的对物体的观察的次数只能是有限的,而任意两时刻间的时刻是无限的.幸运的是,微积分告诉我们,只需要相邻观察时刻充分的小,所获得的物体的运动就越接近于物体的真实运动,而物体的真实运动就是当相邻观察时刻的时间间隔趋于0的极限运动.所以为描述物体最一般的运动,只需要考察物体两个时刻的位置即可,而无需考虑这两时刻的中间位置.前一时刻物体的位置叫 \textbf{初位置},后一时刻的位置叫\textbf{末位置}.物体从初位置变化到末位置,就称物体完成了一个\textbf{位移}.若建立了坐标系,末初位置的坐标之差对应的向量就称为物体的\textbf{位移向量}(或\textbf{位移矢量}).现在,问题转化为“物体最一般的位移是怎样的?”

\textbf{平动位移}是指物体上所有点的位移的几何上相等的位移.\textbf{转动位移}是物体从初位置绕某个固定直线旋转得到末位置的位移,该固定直线称为\textbf{转动轴}.\textbf{螺旋位移}由平动和转动位移组成,并且平动位移沿着转动轴.

对于物体的任一位移,只需选取物体上任一点,通过平动位移使该点从初位置变化到末位置,再通过绕该点的转动便可使物体处于末位置,即任一位移可分解为随任选点的平动位移和绕该点的某个轴的转动.也就是说,物体最一般的运动是平动和转动,而螺旋位移只是这两者的组合.
\subsection{质量几何}
知道了物体最一般的运动,现在回到物体本身,研究具有质量的物体的几何性质.
\subsubsection{质心}
\textbf{质心}是描述物体质量分布平均位置的点.若把组成物体的粒子看成具有相同的质量 $m$ ,那么质量大的地方分布的粒子多.以 $n_i$ 表示在 $\bvec{r_i}$ 处的粒子数,则 $\bvec{r_i}$ 处质量为 $m_i=n_im$.则总的粒子对物体的位置贡献为 $\sum_\limits{i}n_i\bvec{r_i}$,于是平均每个粒子的位置为
\begin{equation}\label{AngMo_eq1}
\bvec{r_c}=\frac{\sum_\limits{i}n_i\bvec{r_i}}{\sum_\limits{i}n_i}=\frac{\sum_\limits{i}n_im\bvec{r_i}}{\sum_\limits{i}n_im}=\frac{\sum_\limits{i}m_i\bvec{r_i}}{M}
\end{equation}
其中 $M=\sum_\limits{i} n_im$ 为物体总质量,若组成物体的粒子质量不同

\addTODO{角动量性质、转动惯量、添加例子}
