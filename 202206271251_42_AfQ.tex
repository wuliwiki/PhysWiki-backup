% 仿射群
% 仿射群|正合列

\begin{issues}
\issueDraft
\end{issues}

\pentry{仿射空间\upref{AfSp}}
\textbf{仿射群}是在仿射空间利用仿射映射(\autoref{AfSp_def2}~\upref{AfSp})构建的一种群\upref{Group}.让我们进行如下思考:群首先要满足封闭性,即两个群元作用(或运算)得到的还是一个群元,对仿射群而言,群元是仿射映射,映射的运算很自然的用映射的复合(\autoref{map_sub2}~\upref{map})表示.那么封闭性要求对两仿射映射 $f,g$,$fg$还是仿射映射.映射的复合要求 $f$ 的定义域起码得包含 $g$ 的值域.记
 \begin{equation}
f:\mathbb A\rightarrow\mathbb A_1,\quad g:\mathbb A_2\rightarrow \mathbb A_3
 \end{equation}
 则 $\mathbb A_3\subset \mathbb A$.而任意两个群元都可以进行运算的,即还有 $gf$ 也是仿射映射,同理,这意味着 $\mathbb A_1\subset\mathbb A_2$.即
 \begin{equation}\label{AfQ_eq1}
 \mathbb A_3\subset \mathbb A,\quad \mathbb A_1\subset\mathbb A_2
 \end{equation}
 
 其次,群元必定得有逆元,对映射而言,就是逆映射得存在,这表明
\begin{equation}
 f^{-1}:\mathbb A_1\rightarrow\mathbb A,\quad g^{-1}:\mathbb A_3\rightarrow\mathbb A_2
\end{equation}
 存在,和前面一样,又有
 \begin{equation}\label{AfQ_eq2}
 \mathbb A_2\subset \mathbb A_1, \mathbb A\subset\mathbb A_3
 \end{equation}
 \autoref{AfQ_eq1} ,\autoref{AfQ_eq2} 联立,就有
 \begin{equation}
 \mathbb A=\mathbb A_1= \mathbb A_2=\mathbb A_3
 \end{equation}
 也就是说,仿射群是由仿射空间 $(\mathbb A,V)$ 上的所有自同构 $f:\mathbb A\rightarrow\mathbb A$ 实现的.由仿射映射的定义:
 \begin{equation}\label{AfQ_eq3}
 f(\dot p+v)=f(\dot p)+Df\cdot v
 \end{equation}
 显然,这里 $Df$ 是 $V\rightarrow V$ 上的可逆的线性映射(\autoref{AfSp_the1}~\upref{AfSp}).这意味着,$Df$ 是一个可逆的线性算子(\autoref{LiOper_sub4}~\upref{LiOper}),可记为 $\mathcal F=Df$.于是\autoref{AfQ_eq3} 变成
 \begin{equation}
 f(\dot p+v)=f(\dot p)+\mathcal F v
 \end{equation}
 \subsection{仿射群}
 \begin{definition}{仿射群}
 设 $n$ 仿射空间 $(\mathbb A,V)$ 定义在域 $\mathbb F$ 上,则所有仿射自同构配上映射复合构成的集合 $\mathrm{Aff}(\mathbb A)=A_n(\mathbb F)$ 称为仿射空间 $\mathbb A$ 上的 $n$ 维\textbf{仿射群}.
 \end{definition}