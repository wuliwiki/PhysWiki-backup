% 欧拉角(综述)
% license CCBYSA3
% type Wiki

本文根据 CC-BY-SA 协议转载翻译自维基百科\href{https://en.wikipedia.org/wiki/Euler_angles}{相关文章}。
\begin{figure}[ht]
\centering
\includegraphics[width=8cm]{./figures/f7361f2b5f088c22.png}
\caption{} \label{fig_OLJ_1}
\end{figure}
欧拉角是由莱昂哈德·欧拉提出的三个角度,用于描述刚体相对于固定坐标系的方向。[1]

它们也可以表示物理学中运动参考系的方向,或三维线性代数中一般基的方向。

经典欧拉角通常采用倾斜角度的方式,其中零度表示垂直方向。后来,由彼得·古思里·泰特(Peter Guthrie Tait)和乔治·H·布赖恩(George H. Bryan)提出了替代形式,主要用于航空学和工程学中,其中零度表示水平位置。
\subsection{链式旋转等价性}
欧拉角可以通过元素几何或旋转组合(即链式旋转)来定义。几何定义表明,三个元素旋转(绕坐标系的轴旋转)总是足够将物体定向到任何目标参考系。

这三个元素旋转可以是外在旋转(绕原始坐标系xyz轴旋转,假设坐标系保持静止),也可以是内在旋转(绕旋转坐标系XYZ轴旋转,该坐标系与运动体固连,在每次元素旋转后,物体相对于外部参考系的方向会发生变化)。

在下面的各节中,带有撇号标记的轴(例如,z″)表示元素旋转后的新轴。

欧拉角通常用 α、β、γ 或 ψ、θ、φ 来表示。不同的作者可能会使用不同的旋转轴集来定义欧拉角,或者使用不同的名称来表示相同的角度。因此,任何涉及欧拉角的讨论都应该首先明确它们的定义。

在不考虑使用两种不同约定来定义旋转轴(内在或外在)的情况下,旋转轴有十二种可能的旋转顺序,可以分为两组:

\begin{itemize}
\item \textbf{正确的欧拉角}(\(z-x-z,x-y-x,y-z-y,z-y-z,x-z-x,y-x-y\))
\item \textbf{泰特-布赖恩角}(\(x-y-z,y-z-x,z-x-y,x-z-y,z-y-x,y-x-z\))。
\end{itemize}

泰特-布赖恩角也被称为卡尔丹角、航海角、航向、仰角和倾斜角,或偏航、俯仰和滚转角。有时,这两类旋转顺序都被称为“欧拉角”。在这种情况下,第一组旋转顺序被称为正确的或经典的欧拉角。
\subsection{经典欧拉角} 
欧拉角是瑞士数学家莱昂哈德·欧拉(1707–1783)引入的三个角度,用于描述刚体相对于固定坐标系统的方向。
\subsubsection{几何定义}  
\begin{figure}[ht]
\centering
\includegraphics[width=10cm]{./figures/1c8b217760d79068.png}
\caption{左:一个万向架组合,展示了 z-x-z 旋转序列。外部坐标系显示在底座中,内部坐标轴以红色表示。右:一个简单的图示,展示了类似的欧拉角。} \label{fig_OLJ_2}
\end{figure}
原始坐标系的轴表示为 \(x\)、\(y\)、\(z\),旋转后坐标系的轴表示为 \(X\)、\(Y\)、\(Z\)。几何定义(有时称为静态定义)首先定义节点线(N)为平面 \(xy\) 和 \(XY\) 的交线(也可以定义为轴 \(z\) 和 \(Z\) 的公垂线,然后表示为向量积 \(N = z \times Z\))。基于这个定义,三个欧拉角可以如下定义:

\(\alpha\)(或 \(\varphi\))是 \(x\) 轴与 \(N\) 轴之间的带符号角度(\(x\)-惯例——也可以定义为 \(y\) 轴与 \(N\) 轴之间的角度,称为 \(y\)-惯例)。  
\(\beta\)(或 \(\theta\))是 \(z\) 轴与 \(Z\) 轴之间的角度。  
\(\gamma\)(或 \(\psi\))是 \(N\) 轴与 \(X\) 轴之间的带符号角度(\(x\)-惯例)。  

只有当两个参考系具有相同的手性时,才能定义这两个参考系之间的欧拉角。

\subsubsection{内在旋转的约定}
内在旋转是发生在附着于运动物体的坐标系统$XYZ$的轴上的元素旋转。因此,它们在每次元素旋转后会改变其方向。$XYZ$系统会旋转,而$xyz$系统保持固定。从$XYZ$与$xyz$初始重合开始,三个内在旋转的组合可以用来达到$XYZ$的任何目标方向。

欧拉角可以通过内在旋转来定义。旋转后的坐标系$XYZ$可以想象为最初与$xyz$对齐,然后经历由欧拉角表示的三个元素旋转。其连续的方向可以表示如下:
\begin{itemize}
\item $x-y-z$ 或 $x_0-y_0-z_0$(初始)  
\item $x^\prime-y^\prime-z^\prime$ 或 $x_1-y_1-z_1$(第一次旋转后)  
\item $x"-y"-z"$ 或 $x_2-y_2-z_2$(第二次旋转后)  
\item $X-Y-Z$ 或 $x_3-y_3-z_3$(最终)
\end{itemize}
对于上述列出的旋转序列,节点线 $N$ 可以简单地定义为第一次元素旋转后 $X$ 的方向。因此,$N$ 可以简单地表示为 $x^\prime$。此外,由于第三次元素旋转是围绕 $Z$ 轴进行的,它不会改变 $Z$ 的方向。因此,$Z$ 与 $z"$ 重合。这使得我们可以简化欧拉角的定义如下:
\begin{itemize}
\item $\alpha$(或$\varphi$)表示围绕 $z$ 轴的旋转,  
\item $\beta$(或 $\theta$)表示围绕 $x^\prime$ 轴的旋转,  
\item $\gamma$(或 $\psi$)表示围绕 $z"$ 轴的旋转。
\end{itemize}
\subsubsection{外在旋转的约定}

外在旋转是关于固定坐标系统$xyz$轴的元素旋转$XYZ$系统旋转,而$xyz$系统保持不变。从$XYZ$与$xyz$重合开始,通过三次外在旋转的组合可以实现任何目标方向。欧拉角或泰特–布莱恩角($\alpha,\beta,\gamma$)表示这些元素旋转的幅度。例如,目标方向可以按以下步骤实现(注意欧拉角应用的顺序是反向的):
\begin{enumerate}
\item $XYZ$ 系统绕$z$轴旋转角度$\gamma$。此时$X$轴与$x$轴的夹角为 γ。
\item $XYZ$ 系统再次旋转,这次绕$x$轴旋转角度 $\beta$。此时 $Z$ 轴与$z$轴的夹角为$\beta$。
\item $XYZ$ 系统第三次旋转,绕$z$轴再旋转角度$\alpha$。
\end{enumerate}
总之,三次元素旋转依次发生在$z,x$ 和$z$轴上。实际上,这个旋转序列通常表示为$z-x-z$(或 3-1-3)。与欧拉角和泰特–布莱恩角相关的旋转轴集合通常使用这种符号表示(具体细节请参见上文)。

如果每一步旋转都作用于旋转坐标系统$XYZ$,那么该旋转为内在旋转($Z-X^\prime-Z"$)。内在旋转也可以表示为 3-1-3。
\subsubsection{符号、范围和约定}
角度通常按照右手法则定义。即,当角度表示的旋转在沿轴正方向观察时呈顺时针方向时,其值为正;当旋转呈逆时针方向时,值为负。相反的约定(左手法则)较少采用。

关于范围(使用区间表示法):
\begin{itemize}
\item 对于$\alpha$和$\gamma$ ,范围定义为模$2\pi$弧度。例如,一个有效的范围可以是$[ -\pi,\pi]$。
\item 对于$\beta$,范围覆盖π弧度(但不能称其为模$\pi$的)。例如,它可以是$[0,\pi]$或$[-\pi/2,\pi/2]$。
\end{itemize}
角度 $\alpha,\beta$和$\gamma$唯一确定,除了一个特殊情况,即当$xy$平面和$XY$平面重合时,即当$z$轴和$Z$轴方向相同或相反时。实际上,如果$z$轴和 $Z$轴方向相同,则$\beta =0$,只有$(\alpha+\gamma)$被唯一确定(而不是各自的值);同样,如果$z$轴和$Z$轴方向相反,则 $\beta=\pi$,只有$(\alpha-\gamma)$被唯一确定(而不是各自的值)。这些模糊性在应用中被称为“万向节锁”。

有六种选择旋转轴的方式来确定欧拉角。在所有情况下,第一和第三个旋转轴是相同的。这六种可能的顺序是:
\begin{enumerate}
\item $z_1-x'-z_2"$(内旋转)或 $z_2-x-z_1$(外旋转)  
\item $x_1-y'-x_2"$(内旋转)或 $x_2-y-x_1$(外旋转)  
\item $y_1-z'-y_2"$(内旋转)或 $y_2-z-y_1$(外旋转)  
\item $z_1-y'-z_2"$(内旋转)或 $z_2-y-z_1$(外旋转)  
\item $x_1-z'-x_2"$(内旋转)或 $x_2-z-x_1$(外旋转)  
\item $y_1-x'-y_2"$(内旋转)或 $y_2-x-y_1$(外旋转)
\end{enumerate}

\subsubsection{进动、章动和内旋转}
进动、章动和内旋转(自旋)被定义为通过改变其中一个欧拉角,同时保持其他两个角度不变所得到的运动。这些运动不是通过外部坐标系来表示,也不是通过与物体共同旋转的坐标系来表示,而是两者的混合。它们构成了一个混合轴旋转系统,其中第一个角度使节点线绕外部轴$z$旋转,第二个角度绕节点线$N$旋转,第三个角度则是绕$Z$轴的内旋转,$Z$轴是一个固定在物体上的轴,随着物体的运动而改变。

静态定义意味着:

\begin{itemize}
\item \alpha(进动)表示绕 z 轴的旋转,
- β(章动)表示绕 N 或 x′ 轴的旋转,
- γ(内旋转)表示绕 Z 或 z″ 轴的旋转。
\end{itemize}

如果 β 为零,则没有绕 N 轴的旋转。因此,Z 与 z 重合,α 和 γ 表示绕同一轴(z)的旋转,最终的方向可以通过绕 z 轴的单一旋转来获得,旋转角度等于 α + γ。