% 连续统假设(综述)
% license CCBYSA3
% type Wiki

本文根据 CC-BY-SA 协议转载翻译自维基百科\href{https://en.wikipedia.org/wiki/Continuum_hypothesis}{相关文章}。

在数学中,特别是集合论中,连续统假设(缩写为CH)是一个关于无限集合可能大小的假设。它陈述了:

没有一个集合,其基数严格介于整数和实数之间。

或者等价地:

实数的任何子集要么是有限的,要么是可数无限的,要么具有实数的基数。

在包含选择公理的泽梅洛–弗兰克尔集合论(ZFC)中,这与以下的阿列夫数方程等价:\()
2^{\aleph_0} = \aleph_1\)或者用贝斯数表示更简洁地写作:\(\beth_1 = \aleph_1\)

连续统假设由格奥尔格·康托尔于1878年提出,\(^\text{[1]}\)并且确定其真伪是1900年希尔伯特提出的23个问题中的第一个。这个问题的答案与ZFC独立,因此可以将连续统假设或其否定作为公理加入到ZFC集合论中,且如果ZFC是一致的,则所得到的理论是一致的。1963年,保罗·科恩证明了这一独立性,补充了1940年库尔特·哥德尔的早期工作。\(^\text{[2]}\)

该假设的名称来源于实数的“连续统”一词。
\subsection{历史} 
康托尔认为连续统假设为真,并且多年来他徒劳地尝试证明它。\(^\text{[3]}\)它成为大卫·希尔伯特在1900年巴黎国际数学家大会上提出的“重要未解问题”列表中的第一个。那个时候,公理化集合论尚未形成。1940年,库尔特·哥德尔证明了连续统假设的否定,即存在一个具有中间基数的集合,不能在标准集合论中证明。\(^\text{[2]}\)连续统假设的独立性的第二部分——即无法证明不存在中间大小的集合——在1963年由保罗·科恩证明。\(^\text{[4]}\)
\subsection{无限集合的基数} 
如果两个集合之间存在一个双射(即一一对应),则称它们具有相同的基数或基数。直观上,两个集合\( S \)和\( T \)具有相同的基数意味着可以将\( S \)中的元素与\( T \)中的元素配对,使得\( S \)中的每个元素都与\( T \)中的恰好一个元素配对,反之亦然。因此,集合\( \{\text{banana}, \text{apple}, \text{pear}\} \)与集合\( \{\text{yellow}, \text{red}, \text{green}\} \)具有相同的基数,尽管这两个集合包含不同的元素。

对于像整数集合或有理数集合这样的无限集合,证明两个集合之间存在双射变得更加困难。有理数\( \mathbb{Q} \)看似对连续统假设形成了一个反例:整数形成有理数的一个真子集,而有理数又是实数的真子集,因此直观上有理数比整数多,实数比有理数多。然而,这种直观分析是错误的,因为它没有考虑到所有三个集合都是无限的。或许更重要的是,它实际上将集合\( \mathbb{Q} \)的“大小”概念与其上的顺序或拓扑结构混淆了。实际上,事实证明有理数集合可以与整数集合一一对应,因此有理数集合的大小(基数)与整数集合相同:它们都是可数集合。\(^\text{[5]}\)

康托尔给出了两个证明,证明整数集合的基数严格小于实数集合的基数(参见康托尔的第一次不可数性证明和康托尔的对角线论证)。然而,他的证明并未给出整数的基数比实数的基数小的程度。康托尔提出了连续统假设,作为解决这个问题的可能方案。

简单来说,连续统假设(CH)声明实数集合的基数是比整数集合的基数大的最小可能基数。即,实数的每个子集\( S \subseteq \mathbb{R} \)要么可以一一映射到整数,要么实数可以一一映射到\( S \)。由于实数的基数与整数的幂集的基数相等,即\( |\mathbb{R}| = 2^{\aleph_0} \),所以连续统假设可以重述为:

\textbf{连续统假设}— \(\nexists S : \aleph_0 < |S| < 2^{\aleph_0}\)

假设选择公理成立,存在一个唯一的最小基数 \( \aleph_1 \),它大于 \( \aleph_0 \),而连续统假设则等价于等式:\(2^{\aleph_0} = \aleph_1\)\(^\text{[6][7]}\)