% 粒子物理学
% license CCBYSA3
% type Wiki

(本文根据 CC-BY-SA 协议转载自原搜狗科学百科对英文维基百科的翻译)


\textbf{粒子物理(又称作“高能物理”)}是研究基本粒子之间相互作用、相互转化规律的科学,其主要目的是要找到一种既简单又普遍的物理原理来统一解释基本粒子之间五花八门的相互作用、相互转换现象[1],其研究对象就是物质的基本结构和基本相互作用。

\subsection{概论}
在过去的一百多年里,我们对基本粒子性质的认识有了长足的进步,建立发展并逐步完善了粒子物理标准模型,粒子物理标准模型的预测与实验测量达到了惊人的吻合程度。然而,时至今日,我们并没有找到一个能够统一描述所有粒子以及所有相互作用的理论。

自然界中已知的四种基本相互作用是引力相互作用、电磁相互作用、弱相互作用以及强相互作用,每一种相互作用的特点如表1.1所示。
\begin{table}[h]
\centering
\caption{四种基本相互作用性质比较}
\begin{tabular}{|c|c|c|c|c|c|}
\hline
\textbf{相互作用} & \textbf{强度} & \textbf{力程} & \textbf{媒介粒子} & \textbf{参与作用粒子} & \textbf{末端态 }\\
\hline
强相互作用 & 1 & $10^{-15}$ m & 胶子 & 夸克, 胶子 & 强子 \\
\hline
弱相互作用 & $10^{-5}$ & $<10^{-17}$ m & $W^{\pm}, Z^0$ & 夸克, 电子, 中微子等 & 无 \\
\hline
电磁相互作用 & $1/137$ & $F \propto 1/r^2$ & 光子 & 所有带电粒子 & 原子等 \\
\hline
引力 & $10^{-39}$ & $F \propto 1/r^2$ & 引力子? & 所有粒子 & 太阳系等 \\
\hline
\end{tabular}
\end{table}
