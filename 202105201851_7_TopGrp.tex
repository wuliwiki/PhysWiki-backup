% 拓扑群
% keys 拓扑群|同态|拓扑空间|topological group|topology|Lie group|李群

\pentry{连续映射\upref{Topo1}, 自由群\upref{FreGrp}}


\subsection{拓扑群}
拓扑群,顾名思义,是一种同时具有群和拓扑结构的数学对象.但是光有两个结构还不够,拓扑群还要求满足拓扑和群运算之间的一个联系.

\begin{definition}{拓扑群}
给定一个集合 $G$,若在 $G$ 上定义了一个拓扑 $\mathcal{T}$ 和一个群运算“$\cdot$”,且满足拓扑空间之间的映射 $f:G\times G\rightarrow G$ 是一个连续映射,其中 $f(g_1, g_2)=g_1\cdot g_2^{-1}$,那么称 $G$ 是一个\textbf{拓扑群(topological group)}.
\end{definition}

要求 $f(g_1, g_2)=g_1\cdot g_2^{-1}$ 是一个连续映射,也就保证了 $f_1(g_1, g_2)=g_1\cdot g_2$ 和 $f_2(g)=g^{-1}$ 都是连续映射.

更进一步,对于$h\in G$,我们还可以定义\textbf{左平移映射}$l_h: G\to G$和\textbf{右平移映射}$r_h: G\to G$,其中对于任意$g\in G$都有$l_h(g)=hg$和$r_h(g)=gh$.这两个映射也是$G\to G$的连续映射.

通常,我们也会省略运算符号,而简单把 $g_1\cdot g_2$ 记为 $g_1g_2$.

\begin{definition}{拓扑群同态}
设 $G$ 和 $H$ 是两个拓扑群,它们之间有映射 $f:G\rightarrow H$.如果 $f$ 在拓扑意义上是连续映射,在代数意义上是群同态,那么称 $f$ 是一个拓扑群之间的同态.
\end{definition}

下面我们看两个简单的例子.

\begin{example}{拓扑群的例子}\label{TopGrp_ex1}
\begin{itemize}
\item 取实数轴 $\mathbb{R}$,定义通常的度量拓扑以及加法群,则 $\mathbb{R}$ 是一个拓扑群.
\item 取复平面上的单位圆 $S^1=\{\mathrm{e}^{\theta\I}|\theta\in\mathbb{R}\}$,定义其拓扑为二维欧几里得空间中的子拓扑,群运算为复数乘法,则 $S^1$ 是一个拓扑群.
\item 定义映射 $f:\mathbb{R}\rightarrow S^1$ 为 $f(t)=\mathrm{e}^{2\pi t\I}$,则 $f$ 是一个拓扑群同态.
\end{itemize}
\end{example}

以上两个例子非常容易想象出来.下面,我们稍微深入一些,讨论一个不那么直观的拓扑群,这对将来理解微分几何富有启发意义.

\begin{example}{一般线性群}\label{TopGrp_ex2}
给定实数域 $\mathbb{R}$ 和一个正整数 $k$,则全体 $\mathbb{R}$ 上的 $k$ 阶矩阵构成一个 $k^2$ 维欧氏空间,也就是一个拓扑空间.对于这个空间中的每个点(也就是矩阵),其第 $i$ 行 $j$ 列的矩阵元就是该点在第 $k(i-1)+j$ 个坐标轴上的分量.

如上定义的矩阵空间中,所有非退化矩阵构成一个子集,记为 $\opn{GL}(k, \mathbb{R})$.在 $\opn{GL}(k,\mathbb{R})$ 上用子拓扑,则它成为一个\textbf{拓扑空间}.

$\opn{GL}(k,\mathbb{R})$ 上,用矩阵乘法作为运算,则它还构成一个\textbf{群}.

通过以上方式,在 $\opn{GL}(k, \mathbb{R})$ 上分别定义的拓扑和群结构,合在一起就使得 $\opn{GL}(k, \mathbb{R})$ 成为一个\textbf{拓扑群}.
\end{example}

$\opn{GL}(k, \mathbb{R})$ 作为拓扑空间可以理解为 $\mathbb{R}^{k^2}$ 空间里挖去了若干孤立点,每个孤立点都代表一个退化矩阵.定义 $f:\opn{GL}(k, \mathbb{R})\times\opn{GL}(k, \mathbb{R})\to\opn{GL}(k, \mathbb{R})$,其中 $f(g_1, g_2)=g_1\cdot g_2^{-1}$,那么要让它构成拓扑群就必须满足 $f$ 是一个连续映射,使得这一点成立的关键在于行列式是矩阵空间的连续函数,以及行列式的积性\footnote{即矩阵乘积的行列式等于矩阵行列式的乘积.}.




%我们知道,欧几里得空间中可以用乘积拓扑来构造基本开集,也就是说,$\mathbb{R}^k$ 的基本开集形如 $(a_1, b_1)\times(a_2, b_2)\times\cdots\times(a_k, b_k)$,其中各 $a_i, b_i$ 都是实数;$\mathbb{R}^k$ 的开集都是基本开集的并.全体 $\mathbb{R}$ 上的 $k$ 阶矩阵构成一个 $k^2$ 维欧氏空间中的基本开集,可以表示成\textbf{若干矩阵的集合}:$\{k\times k$ 实矩阵|第 $i$ 行 $j$ 列的矩阵元在区间 $(a_{ij}, b_{ij})$ 中 $\}$;这个矩阵的集合也可以表示成\textbf{开区间构成的矩阵},其第 $i$ 行 $j$ 列的矩阵元是 $(a_{ij}, b_{ij})$.

%但我们讨论的是非退化矩阵 $\opn{GL}(k, \mathbb{R})$,所以得抛弃那些不满秩的矩阵,相当于在 $\mathbb{R}^{K^2}$ 中挖去若干点.所以为了方便讨论,当我们说到开区间构成的非退化矩阵时,默认已经挖去了那些不满秩的矩阵.

%接下来,我们来证明这个既是拓扑空间又是群的 $\opn{GL}(k,\mathbb{R})$ 是一个拓扑群;换句话说,$\varphi: \opn{GL}(k,\mathbb{R})\times \opn{GL}(k,\mathbb{R})\rightarrow \opn{GL}(k,\mathbb{R})$ 是一个连续映射,其中 $\varphi(\bvec{M}, \bvec{N})=\bvec{MN}^{-1}$.思路是证明 $\varphi$ 处处连续.

%选择任意两个非退化的实矩阵 $\bvec{M}, \bvec{N}$,则 $\bvec{P}=\varphi(\bvec{M}, \bvec{N})=\bvec{MN}^{-1}$.取 $\bvec{P}$ 的一个邻域,不妨设这个邻域是一个基本开集(或者说,开区间构成的矩阵):$U_{\bvec{P}}$,其中 $U_{\bvec{P}}$ 第 $i$ 行 $j$ 列的矩阵元是 $(a_{ij}, b_{ij})$,而 $P_{ij}\in(a_{ij}, b_{ij})$.记所有 $(a_{ij}, b_{ij})$ 中,长度 $\abs{b_{ij}-a{ij}}$ 最短的是 $r$.

\subsection{子拓扑群}

\begin{definition}{子拓扑群}
给定一个拓扑群$G$,若它的一个子集$H$取了$G$的群运算和限制拓扑以后形成一个拓扑群,则称$H$是$G$的\textbf{子拓扑群(topological subgroup)}.
\end{definition}

对于连通的拓扑群,我们有如下性质:

\begin{theorem}{}
设$G$是一个\textbf{连通}的拓扑群,$H$是$G$的\textbf{开集},且$H$是$G$的\textbf{子群}\footnote{不要求一定是子拓扑群.},那么$H=G$.
\end{theorem}

\textbf{证明}:

由于$H$是子集,我们就可以将$G$拆分成\textbf{彼此不相交}的左陪集$aH$,其中$a\in G$.

考虑\textbf{左平移映射}$l_a:G\to G$,则由左平移的定义得,$l_a(H)=aH$.

由于$l_a$是连续映射,因此$aH=l_a(H)$和$H$一样,也是\textbf{开集}.

这样一来,$G$就是由彼此不相交的开集$aH$取并集而来.但是$G$是联通的,因此不可以拆分成不相交的开集之并.因此$H$只能有一个左陪集,就是$H=G$.

\textbf{证毕}.




