% 曲率 曲率半径

\pentry{切线\upref{TanL}}

我们来看一个平面上的一个光滑曲线(即处处存在切线), 我们如何描述它某点处的弯曲程度呢? 一种常用方法是在这点附近取曲线的一小段, 然后做一个尽量与它吻合的圆, 当这小段的长度趋近于 0 时, 这个圆可以唯一确定. 我们把这个圆叫做\textbf{密切圆(osculating circle)}, 把密切圆的半径叫做曲线在该点的\textbf{曲率半径(radius of curvature)}, 曲率半径的倒数叫做\textbf{曲率(curvature)}.

我们先来看一个半径为 $R$ 的圆的一小段圆弧, 令其长度为 $\Delta l$. 作这段圆弧两端的切线, 令它们的夹角为 $\Delta \theta$, 那么显然满足 $R \theta = \Delta l$. 同理, 对于任意光滑曲线上长度为 $\Delta l$ 的一段, 我们也可以做相同的处理, 但需要令 $\Delta l \to 0$
\begin{equation}\label{curvat_eq3}
R = \lim_{\Delta l \to 0} \frac{\Delta l}{\Delta \theta}
\end{equation}
曲率的具体的计算公式取决于使用什么方式定义曲线, 最常见描述方式就是在直角坐标系中通过函数 $y = f(x)$ 来定义, 点 $(x, f(x))$ 处的曲率半径为
\begin{equation}
R = \frac{[1 + f'(x)^2]^{3/2}}{f''(x)}
\end{equation}
如果通过极坐标系方程 $r = g(\theta)$ 定义, 则点 $(\theta, g(\theta))$ 处的曲率半径为
\begin{equation}

\end{equation}


\subsection{直角坐标系的推导}
\pentry{高阶导数\upref{HigDer}, 一元函数的微分\upref{Diff}}

平面上曲线的最常见描述方式就是通过定义实函数 $y(x)$. 我们可以通过导数\upref{Der}计算曲线上某点切线关于 $x$ 轴的夹角 $\theta$.
\begin{equation}\label{curvat_eq1}
\dot y = \dv{y}{x} = \tan \theta
\end{equation}
曲线长度的微分为
\begin{equation}
\dd{l} = \frac{\dd{x}}{\cos\theta}
\end{equation}
其中
\begin{equation}\label{curvat_eq2}
\cos\theta = \frac{1}{\sqrt{1 + \tan^2\theta}} = \frac{1}{\sqrt{1 + f'(x)^2}}
\end{equation}
为了得到 $\dd{\theta}$, 我们对\autoref{curvat_eq1} 两边做微分得
\begin{equation}
f''(x) \dd{x} = \frac{1}{\cos^2\theta} \dd{\theta}
\end{equation}
所以曲率半径为
\begin{equation}
R = \dv{l}{\theta} = \frac{1}{f''(x)\cos^3\theta} = \frac{[1 + f'(x)^2]^{3/2}}{f''(x)}
\end{equation}

\subsection{极坐标系的推导}
\begin{equation}
\dd{l} = \frac{r\dd{\theta}}{\cos\alpha}
\end{equation}

令
\begin{equation}
\frac{\dd{r}}{r \dd{\theta}} = \tan\alpha
\end{equation}
\begin{equation}\label{curvat_eq5}
g'(\theta) = g(\theta)\tan\alpha
\end{equation}
微分
\begin{equation}
g''(\theta)\dd{\theta} = g'(\theta)\tan\alpha\dd{\theta} + \frac{g(\theta)}{\cos^2\alpha}\dd{\alpha}
\end{equation}
即
\begin{equation}\label{curvat_eq4}
\dv{\alpha}{\theta} = (g'' - g' \tan\alpha) \frac{\cos^2\alpha}{g} = \frac{g'' - g'\tan\alpha}{g(t + \tan^2\alpha)}
\end{equation}
注意切线方向的微分是 $\dd{\theta} - \dd{\alpha}$. 所以
\begin{equation}
R = \frac{\dd{l}}{\dd{\theta} - \dd{\alpha}} = \frac{\dd{l}/\dd{\theta}}{1 - \dd{\alpha}/\dd{\theta}}
\end{equation}
把\autoref{curvat_eq3} 和\autoref{curvat_eq4} 代入, 再使用\autoref{curvat_eq5} 消去 $\alpha$ 得
\begin{equation}
R = \frac{[g^2 + g'(\theta)^2]^{3/2}}{g^2 + 2g'(\theta)^2 - gg''}
\end{equation}
