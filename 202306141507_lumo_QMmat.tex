% 量子力学与矩阵
% 量子力学|算符|矩阵|态矢|角动量|泡利矩阵

\pentry{量子力学简介\upref{QM0}, 重积分\upref{IntN}, 厄米矩阵本征值问题\upref{HerEig}, 狄拉克符号\upref{braket}}

在量子力学简介\upref{QM0} 中, 我们知道粒子的状态由波函数描述。 许多量子力学教材会从波函数和薛定谔方程讲起(如无限深势阱\upref{ISW}, 简谐振子\upref{QSHOop}), 而另一些教材会从矩阵的形式讲起。 前者由于涉及到求解(偏)微分方程, 数学要求会相对较高, 而后者只需涉及一些有限维的线性代数。 二者乍看起来不同, 但实际上几乎是等效的\footnote{最大的区别和难点是前者是无穷维而后者是有限维}。 为了让读者先对量子力学的理论结构由一个总体的了解, 避免迷失在数学计算中, 我们先介绍矩阵的形式。

以下将\autoref{ex_QM0_2}~\upref{QM0} 展开讨论, 使用线性代数的语言, 继续完善量子力学的基本假设并应用到例子中。 该例中的角动量本征态 $\ket{x+}$, $\ket{x-}$ 和 $\ket{z+}$, $\ket{z-}$ 之间的关系是线性的。 这很容易让我们联想到矢量空间\upref{LSpace}的基底。 根据矢量空间的定义, 这四个本征态符合作为矢量的要求: 可以给它们乘以常数, 可以把它们相加(见\autoref{eq_QM0_2}~\upref{QM0}), 等等(参考\autoref{ex_LSpace_1}~\upref{LSpace})。 正是由于这个原因, 我们把量子力学中的态(即波函数)称为\textbf{状态矢量(state vector)}, 简称\textbf{态矢}。

当明确了它们是矢量后, 我们就可以对这些态(波函数)定义内积\upref{InerPd}。 以(某个时刻的)一维(波)函数 $f(x)$, $g(x)$ 为例(可以用狄拉克符号记为 $\ket{f}$, $\ket{g}$), 我们将其内积定义为(注意这里的内积是有顺序的, 星号代表对函数值取复共轭)% 未完成: 这个概念是否应该已经在傅里叶级数中讲过?
\begin{equation}
\braket{f}{g} = \int_{-\infty}^{+\infty} f(x)^* g(x) \dd{x}~.
\end{equation}
如果态矢 $\ket{f}$ 和 $\ket{g}$ 是二维或三维的波函数, 我们只需要用重积分即可(积分的范围是全空间, 即三个方向的上下限都为无穷大)\footnote{再次强调, \autoref{ex_QM0_2}~\upref{QM0}讨论的态矢是抽象的, 并不能用波函数表示, 但为了方便理解, 我们姑且假设它们可以。}
\begin{equation}
\braket{f}{g} = \int f(\bvec r)^* g(\bvec r) \dd[3]{r} = \iiint f(x, y, z)^* g(x, y, z) \dd{x}\dd{y}\dd{z}~.
\end{equation}
容易证明该定义满足 $\braket{f}{g} = \braket{g}{f}^*$, 即交换内积的顺序, 相当于把结果取复共轭。

% 未完成: 介绍归一化和正交

量子力学的基本假定认为,




\subsection{厄米矩阵}






\subsection{角动量算符}
在 “量子力学简介\upref{QM0}” 中, 由于数学工具上的不足, 我们并没有提及某个物理量的本征态是怎么得到的, 这里来简单介绍。

量子力学中, 每个物理量都对应一个\textbf{算符(operator)}, 算符可以想象为对(波)函数的一种操作, 算符作用在(波)函数上可以得到一个新的函数。 例如某时刻函数为 $\sin x$, 求导算符 $\dv*{x}$ 作用在 $\sin x$ 上就得到一个新的函数 $\cos x$。 又例如坐标 $x$ 也可以作为一个算符, 我们定义将其作用在任意函数 $\Psi(x, t)$ 上, 就是将其相乘, 即 $x\Psi(x, t)$。 又例如任意函数 $f(x)$ 也可以是一个算符, 我们定义将其作用在 $\Psi(x, t)$ 上得 $f(x)\Psi(x, t)$。

不难发现这些算符都是\textbf{线性}的, 如果它们在讨论的有限维\footnote{无限维空间在数学上会产生许多麻烦。}空间\footnote{这里要提醒一下态矢所在的矢量空间的维度和物理空间的维度不是一回事, 例如将一维(物理上的空间维度, 即粒子做直线运动)波函数 $\sin(x)$ 和 $\cos(x)$ 作为基底, 可以组成一个二维态矢空间。}中是封闭的, 我们就可以用矩阵来表示它们\footnote{以后我们还会详细讲算符和矩阵之间的联系, 如果你暂时不理解, 可以先接受这个结论。}(见“矩阵与线性变换\upref{MatLS}”)。%被引用词条未完成。 应该把上面的某个算符作为例子写进去。 注意在有限维空间的情况下, 不是所有算符都闭合的。 使用不同的基底, 矩阵也是不同的

我们这里考察的空间的确是有限维(二维)的, 习惯上我们用 $\ket{z+}$ 和 $\ket{z-}$ 作为基底 (也可以用 $\ket{x+}$ 和 $\ket{x-}$ 或 $\ket{y+}$ 和 $\ket{y-}$, 进行基底变换即可)。 在这个二维空间中, 我们直接给出(以后再推导) $x, y, z$ 三个方向的角动量算符的矩阵(如果去掉 $\hbar/2$, 这三个矩阵就是著名的\textbf{泡利矩阵})为
\addTODO{引用“以后再推导”中“以后”的词条内容。}
\begin{equation}
\mat L_x = \frac{\hbar}{2}\pmat{0 & 1\\ 1 & 0}~,
\qquad
\mat L_y = \frac{\hbar}{2}\pmat{0 & -\I \\ \I & 0}~,
\qquad
\mat L_z = \frac{\hbar}{2}\pmat{1 & 0\\ 0 & -1}~.
\end{equation}
注意它们都是厄米矩阵\upref{HerMat}, 即满足 $\mat L_i\Her = \mat L_i$。% 链接未完成

那么我们为什么需要算符或者矩阵呢? 简单来说, \textbf{一个物理量的本征矢就是就是对应算符(矩阵)的本征矢, 对应的测量值就是本征值}。 这是量子力学的又一基本假设。

以 $\mat L_z$ 为例, 其本征方程为% 连接未完成
\begin{equation}
\frac{\hbar}{2}\pmat{1 & 0\\ 0 & -1} \pmat{c_1 \\ c_2} = \lambda \pmat{c_1 \\ c_2}~.
\end{equation}
由于这个矩阵已经是对角矩阵, 两个本征矢和本征值分别为
\begin{equation}\ali{
&\pmat{c_1 \\ c_2} = \pmat{1 \\ 0}~, &\qquad& \lambda = +\frac{\hbar}{2}~.\\
&\pmat{c_1 \\ c_2} = \pmat{0 \\ 1}~, &\qquad& \lambda = -\frac{\hbar}{2}~.
}\end{equation}
如我们所料, 两个态矢就是我们使用的两个基底 $\ket{z+}$ 和 $\ket{z-}$ (因为它们就是这么定义的), 本征值也同样吻合。

我们再来解 $x$ 方向角动量的本征方程
\begin{equation}
\frac{\hbar}{2}\pmat{0 & 1\\ 1 & 0} \pmat{c_1 \\ c_2} = \lambda \pmat{c_1 \\ c_2}~.
\end{equation}
解得两个态矢和本征值分别为(量子力学中我们需要把所有的本征矢归一化, 意义见下文)
\begin{equation}\label{eq_QMmat_7}
\ali{
&\pmat{c_1 \\ c_2} = \frac{1}{\sqrt2}\pmat{1 \\ 1}~, &\qquad& \lambda = +\frac{\hbar}{2}~.\\
&\pmat{c_1 \\ c_2} = \frac{1}{\sqrt2}\pmat{1 \\ -1} ~,&\qquad& \lambda = -\frac{\hbar}{2}~.
}\end{equation}
根据定义, 这两个矢量就是 $\ket{x+}$ 和 $\ket{x-}$, 考虑到我们现在用的基底是 $\ket{z+}$ 和 $\ket{z-}$, 马上得到\autoref{eq_QM0_2}~\upref{QM0}
\begin{equation}
\leftgroup{
\ket{x+} = \frac{1}{\sqrt{2}} \ket{z+} + \frac{1}{\sqrt{2}} \ket{z-}\\
\ket{x-} = \frac{1}{\sqrt{2}} \ket{z+} - \frac{1}{\sqrt{2}} \ket{z-}~.
}\end{equation}

最后, $y$ 方向的本征方程和两个解为
\begin{equation}
\frac{\hbar}{2}\pmat{0 & -\I \\ \I & 0} \pmat{c_1 \\ c_2} = \lambda \pmat{c_1 \\ c_2}~,
\end{equation}
\begin{equation}\ali{
&\pmat{c_1 \\ c_2} = \frac{1}{\sqrt2}\pmat{1 \\ \I} ~,&\qquad& \lambda = +\frac{\hbar}{2}~.\\
&\pmat{c_1 \\ c_2} = \frac{1}{\sqrt2}\pmat{1 \\ -\I} ~,&\qquad& \lambda = -\frac{\hbar}{2}
}~.\end{equation}
即
\begin{equation}
\leftgroup{
\ket{y+} = \frac{1}{\sqrt{2}} \ket{z+} + \frac{\I}{\sqrt{2}} \ket{z-}\\
\ket{y-} = \frac{1}{\sqrt{2}} \ket{z+} - \frac{\I}{\sqrt{2}} \ket{z-}~.
}\end{equation}

以上的过程中有几点需要注意。 由于测量到的物理量(本征值)必须是实数, 算符对应的矩阵必须是\textbf{厄米矩阵}。 $N$ 维厄米矩阵会有 $N$ 个\textbf{正交归一}的本征矢。 不难验证, $\braket{x+}{x-}$, $\braket{y+}{y-}$ 和 $\braket{z+}{z-}$ 都为 0。 例如
\begin{equation}
\braket{y+}{y-} = \frac{1}{\sqrt 2}\pmat{1 \\ \I}\Her \frac{1}{\sqrt 2}\pmat{1 \\ -\I} = \frac12 \pmat{1, -\I} \pmat{1 \\ -\I} = 0~.
\end{equation}

\subsection{本征矢的正交归一性}
事实上, 量子力学的所有可测量量对应的算符都必须是厄米算符, 在某种意义上, 同一个算符的不同本征波函都是正交归一的。

\addTODO{投影即可得到系数}
\addTODO{正交归一性见\autoref{the_QMPrcp_3}~\upref{QMPrcp}.}

%\subsection{回收的内容}
%
%如果我们对处于 $\ket{z+}$ 或者 $\ket{z-}$ 态的粒子测量 $z$ 方向角动量, 显然也会分别得到确定的值 $\hbar/2$ 和 $-\hbar/2$。 如果对 $\ket{z+}$ 或者 $\ket{z-}$ 测量 $x$ 方向的角动量, 就需要先将 $\ket{z+}$ 和 $\ket{z-}$ 用 $\ket{x+}$ 和 $\ket{x-}$ 表示(将\autoref{eq_QM0_2}~\upref{QM0} 中的两式相加或相减即可)。

% 回收: 如果我们把它看作 这事实上跟平面旋转变换相同(未完成)% 此处引用一个几何矢量的例子, 未完成
%相同, 我们只需要把

















