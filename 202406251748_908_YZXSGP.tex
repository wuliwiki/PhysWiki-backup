% 原子吸收光谱
% license CCBYSA3
% type Wiki

(本文根据 CC-BY-SA 协议转载自原搜狗科学百科对英文维基百科的翻译)

\begin{figure}[ht]
\centering
\includegraphics[width=6cm]{./figures/acd93a5153633fe4.png}
\caption{请火焰原子吸收光谱仪} \label{fig_YZXSGP_1}
\end{figure}

原子吸收光谱(AAS)和原子发射光谱(AES)是一种利用自由原子对气态光辐射(光)的吸收,定量测定化学元素的光谱分析方法。原子吸收光谱是以自由金属离子对光的吸收为基础的。

在分析化学中,该技术用于确定待分析样品中特定元素(被分析物)的浓度。原子吸收光谱法可用于测定溶液中70多种不同的元素,也可以通过电热蒸发直接测定固体样品中的元素,用于药理学、生物物理学,考古学和毒理学研究。

原子发射光谱学最初被用作分析技术,其基本原理是由德国海德堡大学的教授Robert Wilhelm Bunsen和Gustav Robert Kirchhoff在19世纪下半叶确立的。[1]

原子吸收光谱的现代形式主要是在20世纪50年代由一组澳大利亚化学家发展起来的。他们由澳大利亚墨尔本联邦科学与工业研究组织(CSIRO)化学物理部门的的Alan Walsh爵士领导。[2][3]

原子吸收光谱法在化学的不同领域有许多用途,如临床分析生物液体和组织中的金属,如全血、血浆、尿液、唾液、脑组织、肝脏、头发、肌肉组织、精液,在一些制药过程中,在最终药物产品中残留的微量催化剂,以及分析水中的金属含量。

\subsection{原则}
为了分析样品的原子成分,必须对样品进行雾化。现在最常用的雾化器是火焰和电热(石墨管)雾化器。用光辐射照射原子,辐射源可以是元素特异性线辐射源,也可以是连续辐射源。然后,辐射通过单色仪,单色仪将特定元素的辐射与辐射源发射的所有其他辐射分离开来,最终由检测器进行测量。