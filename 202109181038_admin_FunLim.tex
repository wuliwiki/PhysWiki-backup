% 函数的极限(简明微积分)

\pentry{数列的极限(简明微积分)\upref{Lim0}}

实函数 $f(x)$ 可以看成是一种 “连续” 的数列, 只不过把元素编号从离散的 $n$ 改为连续的 $x$. 类比数列的极限, 我们也可以定义\textbf{函数在正无穷的极限}.

\begin{definition}{函数趋于正无穷时的极限}\label{FunLim_def1}
考虑实函数 $f(x)$. 若存在实数 $A$, 使得对于\textbf{任意} $\epsilon>0$, 总存在\textbf{正实数} $X$, 使得对于所有 $x>X$, 都有 $\abs{f(x)-A}<\epsilon$, 那么我们说 $A$ 是函数 $f(x)$ 在 $x$ 趋于正无穷时的极限, 记为
\begin{equation}
\lim\limits_{x\to +\infty} f(x) = A
\end{equation}
\end{definition}

可以看到该定义和数列极限的定义(\autoref{Lim0_def2}~\upref{Lim0})非常相似, 只是简单做了替换.不过,函数并不是简单地把数列的概念拓展到连续的情况. 数列的编号只能朝着一个方向增大, 但函数的自变量 $x$ 既可以趋近正无穷也可以奔向负无穷, 另外, 由于 $x$ 是连续取值的, 也可以考察自变量 $x$ 不断趋近某一点 $x_0$ 的极限, 即 $x\to x_0$.

\begin{exercise}{}
请仿照\autoref{FunLim_def1} 给出函数趋于正无穷时极限的定义
\begin{equation}
\lim\limits_{x\to -\infty} f(x) = A
\end{equation}
\end{exercise}

如何描述 “自变量趋于一个给定的实数 $x_0$” 呢? 只需要取自变量 $x$ 使得二者间的距离 $\abs{x-x_0}$ 越来越接近 $0$ 即可.

\begin{definition}{函数在某点的极限}\label{FunLim_def3}
考虑实函数 $f(x)$, 并给定一个实数 $x_0$ 和 $A$. 如果对于\textbf{任意}给定的 $\epsilon > 0$, 存在实数 $\delta > 0$, 只要当 $\abs{x-x_0} < \delta$, 就能满足 $\abs{f(x)-A} < \epsilon$ 那么我们说 $f(x)$ 在 $x$ 趋近于 $x_0$ 时的极限为 $A$, 记为
\begin{equation}
\lim\limits_{x\to x_0}f(x)=A
\end{equation}
\end{definition}
$\delta$ 在这里的作用相当于数列极限定义中的 $N$ 或者\autoref{FunLim_def1} 中的 $X$.

\begin{example}{}
求函数在某个值处的极限时, 通常可以直接代入数值计算, 如
\begin{equation}
\lim_{x\to 1} 2x + 1 = 3 \qquad \lim_{x\to 2}\frac{x + 1}{x + 2} = \frac34
\end{equation}

当无穷大与常数相加时, 可以忽略常数, 如
\begin{equation}
\lim_{x\to +\infty} \frac{x + 1}{2x + 2} = \lim_{x\to +\infty} \frac{x}{2x} = \frac12
\end{equation}
\end{example}

但注意 $x\to x_0$ 的极限并不要求函数 $f(x)$ 在 $x_0$ 这点有定义, 因为定义中只考虑 $x$ 慢慢接近 $x_0$ 的过程, 而不考虑 $x = x_0$ 的情况. 即使我们把这点从函数定义域中挖去, 极限是否存在, 以及极限值是多少都不会被改变.

例如以后在 “小角正弦极限\upref{LimArc}” 中会看到, 虽然 $\sin x/ $


我们还可以区分函数在某点的\textbf{左极限}和\textbf{右极限}. 简而言之就是 $x$ 分别从左边和右边两个方向趋近 $x_0$ 时的极限, 具体定义留做思考. 显然, 若\autoref{FunLim_def3} 中的极限存在, 那么无论从左边还是右边趋近, 极限都是相同的.