% 陕西师范大学 2015 年 考研 量子力学
% license Usr
% type Note

\textbf{声明}:“该内容来源于网络公开资料,不保证真实性,如有侵权请联系管理员”

\subsection{一}
\begin{enumerate}
\item 态叠加原理?自由粒》波峦教一定是耳面波吗?为什么?
\item 下面表述 的是否是同一状态\\
(1)$4(x)$与$4(x)e^\lambda$,$\lambda$为应常教,(2)$4(x)$与$4(x)e^{\lambda(x)}$,$\lambda(x)$与$x$有关的函数。
\item 为什么可观测力学量都是厄米算符?
\item 为什么全同对称,反对称?
\end{enumerate}
\subsection{二}
在 $[-a, a]$ 的无限深势阱中,在 $t=0$ 时,对于最低能量的叠加态,几承都为50\%,求状态随时间的变化规律。
\subsection{三}
谐振子的能量本征态为 $\lvert n \rangle$,其 $x,p,x^2,p^2$以及$\Delta x,\Delta p$的平均值。
\subsection{四}
 $\hat{A}^\dagger = \hat{A}$, $\hat{B}^\dagger = \hat{B}$, $\hat{F} = \hat{A} + i\hat{B}$,当满是什么条件时$\hat{F}^2$为厄米算符。
\subsection{五}
在宽度习$a$的无限深势阱中,在$t=0$时,处于$\psi=a_1\psi_1+a_2\psi_2+a_2\psi_3+a_4\psi_4$求动量的概率密度。
\subsection{六}
\[
E = 
\begin{pmatrix}
0 & 1 & 0 \\
1 & 0 & 1 \\
0 & 1 & 0
\end{pmatrix}
+ \epsilon_0 
\begin{pmatrix}
0 & 0 & 0 \\
0 & 0 & 1 \\
0 & 1 & 0
\end{pmatrix}~
\]
求能量至二级微扰及一级微扰波函数,和二级态矢。
\subsection{七}
$a_1$的态下