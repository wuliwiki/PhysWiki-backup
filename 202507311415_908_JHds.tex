% 结合代数(综述)
% license CCBYSA3
% type Wiki

本文根据 CC-BY-SA 协议转载翻译自维基百科\href{https://en.wikipedia.org/wiki/Associative_algebra}{相关文章}。

在数学中,交换环(通常是一个域) $K$ 上的结合代数 $A$ 是一个环 $A$,并且带有一个从 $K$ 到 $A$ 的中心(center)的环同态。因此,它是一种代数结构,包含加法、乘法和数量乘法(即由 $K$ 中元素通过环同态的像所定义的乘法)。加法和乘法运算共同赋予 $A$ 环的结构;加法和数量乘法运算共同赋予 $A$ $K$-模或向量空间的结构。在本文中,我们也使用 $K$-代数 一词来指代 $K$ 上的结合代数。

$K$-代数的一个标准例子是定义在交换环 $K$ 上的方阵环,采用通常的矩阵乘法。

一个交换代数是乘法交换的结合代数,或者等价地,是同时也是交换环的结合代数。

在本文中,假设结合代数都有一个乘法单位元,记作 $1$;为强调这一点,有时称为有单位结合代数。在数学的一些领域中不做这一假设,这类结构称为非有单位结合代数。我们也假设所有的环都是有单位的,并且所有的环同态都是保单位元的。

每个环都是它的中心上的结合代数,也是整数环 $\mathbb{Z}$ 上的结合代数。
\subsection{定义}
设 $R$ 是一个交换环(因此 $R$ 也可以是一个域)。一个结合的 $R$-代数 $A$(或更简单地称为 $R$-代数 $A$)是一个环 $A$,并且同时是一个 $R$-模,满足环加法与模加法是同一个运算,并且数量乘法满足
$$
r \cdot (xy) = (r \cdot x)y = x(r \cdot y)~
$$
对所有 $r \in R$ 和代数中的 $x, y$ 成立。(这个定义意味着代数作为一个环是有单位的,因为假设环必须有乘法单位元。)

等价地,一个结合代数 $A$ 是一个环,并且带有一个从 $R$ 到 $A$ 的中心的环同态。如果 $f$ 是这样的同态,则数量乘法为$(r, x) \mapsto f(r) x$(此处乘法是环乘法);如果给定了数量乘法,则该环同态由$r \mapsto r \cdot 1_A$给出。(另见下文“由环同态导出”一节。)每个环都是一个结合的 $\mathbb{Z}$-代数,其中 $\mathbb{Z}$ 表示整数环。

一个交换代数是一个乘法交换的结合代数,或者等价地,是一个同时也是交换环的结合代数。
\subsubsection{作为模范畴中的幺半群对象}
这个定义等价于说:一个有单位的结合 $R$-代数是 $R$-Mod(即 $R$-模的单(幺)积范畴)中的一个幺半群对象。按照定义,一个环是阿贝尔群范畴中的幺半群对象;因此,结合代数的概念可以通过将阿贝尔群范畴替换为模范畴而得到。

进一步推广这一思想,一些作者将“广义环”定义为某个行为类似于模范畴的其他范畴中的幺半群对象。实际上,这种重新解释使得我们可以避免对代数 $A$ 的元素做显式引用。例如,结合律可以用如下方式表达:根据模张量积的泛性质,乘法(即 $R$-双线性映射)对应于一个唯一的 $R$-线性映射
$$
m: A \otimes_R A \to A~
$$
结合律则对应于以下恒等式:
$$
m \circ (\operatorname{id} \otimes m) = m \circ (m \otimes \operatorname{id})~
$$
\subsubsection{由环同态出发}
一个结合代数本质上等价于一个像落在中心中的环同态。确实,假设给定一个环 $A$ 和一个环同态$\eta : R \to A$其像落在 $A$ 的中心,那么可以通过定义
$$
r \cdot x = \eta(r) x~
$$
(对所有 $r \in R$ 和 $x \in A$)使 $A$ 成为一个 $R$-代数。反过来,如果 $A$ 是一个 $R$-代数,取 $x = 1$,同样的公式又定义了一个像落在中心的环同态$
\eta : R \to A$。

如果一个环是交换的,那么它等于它的中心,因此一个交换的 $R$-代数可以简单地定义为:一个交换环 $A$ 连同一个交换环同态$\eta : R \to A$。上述出现的环同态 $\eta$ 通常称为结构映射。在交换情形下,可以考虑这样一个范畴:其对象是固定 $R$ 的环同态 $R \to A$,即交换的 $R$-代数;其态射是“在 $R$ 下”的环同态 $A \to A'$,即图$R \to A \to A'$等于$R \to A'$也就是交换环范畴在 $R$ 下的余切范畴。素谱函子 Spec 然后给出这个范畴与 Spec $R$ 上的仿射概形范畴之间的反等价。

如何削弱交换性假设是非交换代数几何以及近来的导出代数几何的研究主题。参见:泛矩阵环。
\subsection{代数同态}
两个 $R$-代数之间的同态是一个 $R$-线性的环同态。具体来说,如果$\varphi : A_1 \to A_2$是一个结合代数同态,当且仅当满足以下条件:
$$
\begin{aligned}
\varphi(r \cdot x) &= r \cdot \varphi(x)\\
\varphi(x + y) &= \varphi(x) + \varphi(y)\\
\varphi(xy) &= \varphi(x)\varphi(y)\\
\varphi(1) &= 1
\end{aligned}~
$$
其中 $r \in R$,$x, y \in A_1$。所有 $R$-代数及其代数同态构成一个范畴,通常记作 R-Alg。

交换 $R$-代数的子范畴可以刻画为余切范畴 $R / \mathbf{CRing}$,其中$\mathbf{CRing}$ 是交换环范畴。
\subsection{例子}
最基本的例子是一个环本身;它是其中心或中心中任何子环上的代数。特别地,任何交换环都是其任意子环上的代数。其他例子在代数及数学的其他领域中也非常多见。
\subsubsection{代数}
\begin{itemize}
\item 任意一个环 $A$ 都可以看作一个 $\mathbb{Z}$-代数。从 $\mathbb{Z}$ 到 $A$ 的唯一环同态由它必须将 $1$ 映射到 $A$ 中的单位元这一事实决定。因此,环和 $\mathbb{Z}$-代数是等价的概念,就像阿贝尔群和 $\mathbb{Z}$-模是等价的一样。
\item 任意特征为 $n$ 的环也是一个 $(\mathbb{Z}/n\mathbb{Z})$-代数,方式相同。
\item 给定一个 $R$-模 $M$,$M$ 的自同态环 $\mathrm{End}_R(M)$ 是一个 $R$-代数,其数量乘法定义为$(r \cdot \varphi)(x) = r \cdot \varphi(x)$。
\item 系数在交换环 $R$ 中的矩阵环在矩阵加法和乘法下构成一个 $R$-代数。当 $M$ 是有限生成的自由 $R$-模时,这与上面的自同态环的例子一致。
\item 特别地,域 $K$ 上的 $n \times n$ 方阵构成一个定义在 $K$ 上的结合代数。
\item 复数是一个实数域上的 2 维交换代数。
\item 四元数是一个实数域上的 4 维结合代数(但不是复数域上的代数,因为复数不是四元数的中心元素)。
\item 每个多项式环 $R[x_1, \ldots, x_n]$ 是一个交换的 $R$-代数。实际上,这是集合 $\{x_1, \ldots, x_n\}$ 上的自由交换 $R$-代数。
\item 集合 $E$ 上的自由 $R$-代数是一个“多项式”代数,其系数在 $R$ 中,未知量(不交换)取自集合 $E$。
\item 一个 $R$-模的张量代数自然是一个结合的 $R$-代数。外代数和对称代数等商代数也是如此。从范畴论的角度来看,将 $R$-模映射到其张量代数的函子是左伴随于将 $R$-代数映射到其底层 $R$-模的函子(忽略乘法结构)。
\item 给定交换环 $R$ 上的一个模 $M$,模的直和 $R \oplus M$ 通过将 $M$ 看作“无穷小元素”可以赋予一个 $R$-代数结构;即乘法定义为$(a + x)(b + y) = ab + ay + bx$这个概念有时称为对偶数代数。
\item Cuntz 和 Quillen 引入的拟自由代数是一类自由代数和代数闭域上的半单代数的推广。
\end{itemize}
\subsubsection{表示论}
\begin{itemize}
\item 一个李代数的通用包络代数是一个结合代数,可以用来研究该李代数。
\item 如果 $G$ 是一个群,$R$ 是一个交换环,则所有从 $G$ 到 $R$ 的有限支撑函数构成一个 $R$-代数,乘法为卷积运算。这被称为 $G$ 的群代数。该构造是研究(离散)群应用的起点。
\item 如果 $G$ 是一个代数群(例如半单复李群),那么 $G$ 的坐标环是对应于 $G$ 的 Hopf 代数 $A$。$G$ 的许多结构可以转化为 $A$ 的结构。
\item 有向图的箭图代数(或路径代数 path algebra)是在一个域上由图中的路径生成的自由结合代数。
\end{itemize}
\subsubsection{分析}
\begin{itemize}
\item 给定任意 Banach 空间 $X$,连续线性算子$A : X \to X$构成一个结合代数(以算子复合作为乘法);这是一个 Banach 代数。
\item 给定任意拓扑空间 $X$,$X$ 上的连续实值或复值函数构成一个实或复的结合代数;此处函数按点逐一加法与乘法。
\item 在滤过概率空间$(\Omega, \mathcal{F}, (\mathcal{F}_t)_{t \geq 0}, P)$
上定义的半鞅集合在随机积分运算下构成一个环。
\item Weyl 代数
\item Azumaya 代数
\end{itemize}
\subsubsection{几何与组合学}
\begin{itemize}
\item Clifford 代数,在几何和物理中非常有用。
\item 局部有限偏序集的关联代数是组合数学中研究的结合代数。
\item 划分代数及其子代数,包括 Brauer 代数和 Temperley–Lieb 代数。
\item 微分分次代数是一个带有分次和微分的结合代数。例如,de Rham 代数$\Omega(M) = \bigoplus_{p=0}^n \Omega^p(M)$其中 $\Omega^p(M)$ 表示流形 $M$ 上的 $p$ 阶微分形式,就是一个微分分次代数。
\end{itemize}
\subsubsection{数学物理}
\begin{itemize}
\item 泊松代数是一个在某个域上的交换结合代数,并且带有李代数结构,使得李括号 $\{,\}$ 满足 Leibniz 法则,即$\{fg, h\} = f \{g, h\} + g \{f, h\}$。
\item 给定一个泊松代数 $\mathfrak{a}$,考虑形式幂级数空间 $\mathfrak{a}[\![u]\!]$。如果 $\mathfrak{a}[\![u]\!]$ 拥有一个结合代数结构,其乘法 $*$ 满足:
  对所有 $f, g \in \mathfrak{a}$,
  $$
  f * g = fg - \frac{1}{2} \{f, g\} u + \cdots~
  $$
  那么 $\mathfrak{a}[\![u]\!]$ 称为 $\mathfrak{a}$ 的一个变形量子化。
\item 量子化包络代数。这类代数的对偶是一个结合代数(参见“结合代数的对偶”一节),从哲学上讲,它是量子群的(量子化的)坐标环。
\item Gerstenhaber 代数
\end{itemize}
\subsection{构造}
\textbf{子代数}

一个 $R$-代数 $A$ 的子代数是 $A$ 的一个子集,它既是一个子环,又是一个子模。也就是说,它必须在加法、环乘法、数量乘法下封闭,并且必须包含 $A$ 的单位元。

\textbf{商代数}

设 $A$ 是一个 $R$-代数。$A$ 中任意一个环论意义下的理想 $I$ 自动是一个 $R$-模,因为$r \cdot x = (r 1_A) x$。这使得商环 $A / I$ 具有 $R$-模的结构,实际上也具有 $R$-代数的结构。因此,$A$ 的任何环同态像也是一个 $R$-代数。

\textbf{直积}

一族 $R$-代数的直积是环论意义下的直积。它通过显然的数量乘法成为一个 $R$-代数。

\textbf{自由积}

可以用类似于群自由积的方式构造 $R$-代数的自由积。自由积是 $R$-代数范畴中的余积。

\textbf{张量积}

两个 $R$-代数的张量积也自然是一个 $R$-代数。更多细节参见代数的张量积。给定一个交换环 $R$ 和任意环 $A$,张量积$R \otimes_{\mathbb{Z}} A$可以通过如下定义赋予 $R$-代数的结构:$r \cdot (s \otimes a) = (rs \otimes a)$。将 $A$ 映射到 $R \otimes_{\mathbb{Z}} A$ 的函子是左伴随于将一个 $R$-代数映射到它的底层环(忽略模结构)的函子。另见:换环

\textbf{自由代数}

自由代数是由符号生成的代数。如果施加交换性条件,即对换子取商,那么就得到一个多项式代数。
\subsection{结合代数的对偶}
设 $A$ 是一个定义在交换环 $R$ 上的结合代数。由于 $A$ 特别是一个模,我们可以取它的对偶模 $A^*$。先验地,$A^*$ 不一定具有结合代数的结构。然而,如果 $A$ 带有额外的结构(即 Hopf 代数的结构),那么它的对偶也会成为一个结合代数。

例如,取 $A$ 为一个紧群 $G$ 上的连续函数环。此时,$A$ 不仅是一个结合代数,它还带有余乘法和余单位,定义为:$\Delta(f)(g, h) = f(gh), \quad \varepsilon(f) = f(1)$。

这里的“余-”是指它们满足代数公理中通常乘法和单位元的对偶条件。因此,对偶 $A^*$ 是一个结合代数。

余乘法和余单位在构造结合代数表示的张量积时也非常重要(参见下文“表示”一节)。
