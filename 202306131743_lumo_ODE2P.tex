% 二阶齐次变系数线性微分方程的幂级数解法
% 幂级数|ODE|常微分方程|differential equation|二阶微分方程

\pentry{幂级数\upref{powerS},常系数线性齐次微分方程\upref{ODEb2}}

\addTODO{是不是还要加上差分方程作为预备知识?}

\subsection{从例子出发}

从微积分学中我们知道,许多函数是可以表示为幂级数\upref{powerS}的形式:$f(x)=a_0+a_1x+a_2x^2+\cdots$。幂级数良好的性质可以用于解二阶微分方程。

我们先看一个简单的实例。遵循微积分学的习惯,我们这里以 $x$ 为自变量了。

\begin{example}{}\label{ex_ODE2P_1}
考虑方程
\begin{equation}\label{eq_ODE2P_4}
\frac{\mathrm{d}^2 y}{\dd x^2}-2x\frac{\dd y}{\dd x}-4y=0~.
\end{equation}
在初始条件
\begin{equation}\label{eq_ODE2P_1}
\leftgroup{
    y(0)&=0\\
    y(1)&=1
}~
\end{equation}
下的\textbf{特解}。

尝试设
\begin{equation}\label{eq_ODE2P_3}
y(x)=a_0+a_1x+a_2x^2+\cdots=\sum\limits_{i=0}^\infty a_ix^i~.
\end{equation}
首先代入初值条件\autoref{eq_ODE2P_1} ,得到 $a_0=0, a_1=1$。

接着,考虑到
\begin{equation}\label{eq_ODE2P_2}
\leftgroup{
    &\frac{\mathrm{d}^2 y}{\dd x^2}=2a_2+6a_3x+12a_4x^2+\cdots=\sum\limits_{k=0}^\infty (k+1)(k+2)a_{k+2}x^k\\
    &2x\frac{\dd y}{\dd x}=2a_1x+4a_2x^2+6a_3x^3+\cdots=\cdots=\sum\limits_{k=1}^\infty 2ka_kx^k~.
}
\end{equation}

将\autoref{eq_ODE2P_2} 和\autoref{eq_ODE2P_3} 代回\autoref{eq_ODE2P_4} ,比较各 $x^k$ 的系数,得到
\begin{equation}
(k+1)(k+2)a_{k+2}=(2k+4)a_k~,
\end{equation}
整理得
\begin{equation}
a_{k+2}=\frac{2}{k+1}a_k~.
\end{equation}

这是一个二阶\textbf{差分方程}。

由于 $a_0=0$,故 $a_{2k}=0$ 对所有 $k$ 成立。我们只需要考虑奇数项即可。

令 $b_k=a_{2k-1}$\footnote{反过来就是 $a_k=b_{\frac{k+1}{2}}$。},则我们有 $b_1=a_1=1$ 和 $b_{\frac{k+3}{2}}=\frac{2}{k+1}b_{\frac{k+1}{2}}$;换个写法,就是 $b_{k+1}=\frac{1}{k}b_k$。

因此
\begin{equation}
b_k=\frac{1}{(k-1)!}~,
\end{equation}

进而
\begin{equation}\label{eq_ODE2P_6}
\begin{aligned}
y&=b_1x+b_2x^3+b_3x^5+\cdots\\
 &=x\sum\limits_{k=1}^\infty b_kx^{2k-2}\\
 &=x\sum\limits_{k=1}^\infty \frac{x^{(2k-2)}}{(k-1)!}\\
 &=x\sum\limits_{k=0}^\infty \frac{x^{2k}}{k!}\\
 &=x\E^{x^2} ~.
\end{aligned}
\end{equation}

\end{example}

\autoref{ex_ODE2P_1} 中“假设解为 $x$ 的幂级数,通过比较系数来求出解”的方法,被称为\textbf{幂级数解法}。


\subsection{幂级数解法}

\autoref{ex_ODE2P_1} 和我们之前所解的方程不一样,\autoref{eq_ODE2P_4} 中的系数 $2x$ 不再是一个常数,而是 $x$ 的函数,这使得我们应对\textbf{常系数}方程的方法无效了。对于二阶变系数方程,幂级数解法是很有用的。

哪些方程能应用幂级数解法呢?幂级数解的收敛区间又是否能覆盖所要求解的区间呢?这些问题有完善的解答,但由于较为深入,我们在此只给出重要的结论。

我们所考虑的方程是形如
\begin{equation}\label{eq_ODE2P_5}
\qty(\frac{\mathrm{d}^2}{\dd x^2}+p(x)\frac{\dd}{\dd x}+q(x))y(x)=0~
\end{equation}
在 $x=0$ 处的\textbf{特解}。

$x=x_0$ 处的特解,可以通过变量代换 $t=x-x_0$,来化为 $y(t)$ 的方程在 $t=0$ 处的特解问题。

\begin{theorem}{}\label{the_ODE2P_2}
如果\autoref{eq_ODE2P_5} 中的 $p(x)$ 和 $q(x)$ 都可以写为 $x$ 的幂级数形式,且它们在区间 $\abs{x}<X$ 上收敛,那么方程有形如
\begin{equation}
y=\sum\limits_{k=0}^\infty a_nx^n~
\end{equation}
的特解,且该特解也在 $\abs{x}<X$ 上收敛。
\end{theorem}

\autoref{ex_ODE2P_1} 中的系数为 $-2x$ 和 $-4$,它们都在整个实数轴上收敛,因此我们最终算出来的特解\autoref{eq_ODE2P_6} 也在整个实数轴上收敛。

\begin{theorem}{}\label{the_ODE2P_1}
如果 $xp(x)$ 和 $x^2q(x)$ 均能展成幂级数形式,并且都在 $\abs{x}<X$ 上收敛,那么\autoref{eq_ODE2P_5} 有形如
\begin{equation}\label{eq_ODE2P_7}
y=x^\alpha\sum\limits_{k=0}^\infty a_kx^k~
\end{equation}
的特解,其中 $a_0\neq 0$,$\alpha$ 是一个待定常数,并且\autoref{eq_ODE2P_7} 也在 $\abs{x}<X$ 上收敛。


\end{theorem}


\subsection{若干例题}

二阶变系数线性微分方程在工程和物理中应用广泛,因此我们在此举出一些例题,以帮助读者熟悉其解法。

\begin{example}{}\label{ex_ODE2P_2}
考虑方程
\begin{equation}\label{eq_ODE2P_9}
\qty(\frac{\mathrm{d}^2}{\dd x^2}+\frac{1}{x}\frac{\dd}{\dd x}-\frac{1}{x})y(x)=0~
\end{equation}
在
\begin{equation}\label{eq_ODE2P_10}
\leftgroup{
    y(0)&= 1\\
    y'(0)&= 1\\
}~
\end{equation}
下的特解。

设所求特解为
\begin{equation}\label{eq_ODE2P_8}
\sum\limits_{k=0}^\infty a_kx^k~.
\end{equation}

这次我们先求通解,再代入初值条件求特解。

首先将\autoref{eq_ODE2P_8} 代入\autoref{eq_ODE2P_9} ,得到
\begin{equation}
\sum\limits_{k=0}^\infty (k+2)(k+1)a_{k+2}x^k+(k+2)a_{k+2}x^k-a_{k+1}x^k=0~,
\end{equation}

从而得到
\begin{equation}
a_{k+2}=\frac{a_{k+1}}{(k+2)^2}~,
\end{equation}

即
\begin{equation}\label{eq_ODE2P_11}
a_k=\frac{a_{k-1}}{k^2}~,
\end{equation}

由初值条件\autoref{eq_ODE2P_10} 得
\begin{equation}
a_k=\frac{1}{(k!)^2}~,
\end{equation}

则
\begin{equation}
y=\sum\limits_{k=0}^\infty \frac{x^k}{(k!)^2}~
\end{equation}
是所求的特解。


\end{example}


% 注释掉一个出得不好的题

% \begin{example}{}\label{ex_ODE2P_4}\label{ex_ODE2P_3}
% 考虑方程
% \begin{equation}\label{eq_ODE2P_14}
% \qty(\frac{\mathrm{d}^2}{\dd x^2}+\frac{1}{x}\frac{\dd}{\dd x}-\frac{1}{x})y(x)=0
% \end{equation}
% 在
% \begin{equation}\label{eq_ODE2P_12}
% \leftgroup{
%     y(0)&= 0\\
%     y'(0)&= 1\\
% }
% \end{equation}
% 下的特解。

% 这个例子的方程和\autoref{ex_ODE2P_2} 一样, 但是初值条件不同。用\autoref{ex_ODE2P_2} 的方法是求不出给定特解的,因为初值条件不满足\autoref{eq_ODE2P_11} 。

% 应用\autoref{the_ODE2P_1} ,设所求特解为
% \begin{equation}\label{eq_ODE2P_13}
% y=x^{-\alpha}\sum\limits^\infty_{k=0}a_kx^k=\sum\limits^\infty_{k=0}a_kx^{k+\alpha}
% \end{equation}

% 将\autoref{eq_ODE2P_13} 代入\autoref{eq_ODE2P_14} ,得到
% \begin{equation}\label{eq_ODE2P_15}
% \begin{aligned}
% \sum\limits_{k=0}^\infty (k+\alpha)(k+\alpha-1)a_kx^{k+\alpha-2}+\\
% \sum\limits_{k=0}^\infty(k+\alpha)a_kx^{k+\alpha-2}-\\
% \sum\limits_{k=0}^\infty a_kx^{k+\alpha-1}\\
% =0
% \end{aligned}
% \end{equation}

% 重新排列一下,把 $x$ 的对应次幂放到一起,将\autoref{eq_ODE2P_15} 整理成
% \begin{equation}\label{eq_ODE2P_16}
% \sum\limits_{k=1}^\infty\qty[(k+\alpha)^2a_k-a_{k-1}]x^{k+\alpha-2}+\alpha^2a_0x^{\alpha-2}=0
% \end{equation}

% 令\autoref{eq_ODE2P_16} 左边各项系数为零,得到一系列代数方程:
% \begin{equation}\label{eq_ODE2P_17}
% \leftgroup{
%     \alpha^2a_0&=0\\
%     (\alpha+1)^2a_1-a_0&=0\\
%     (\alpha+2)^2a_2-a_1&=0\\
%     &\vdots\\
%     (\alpha+k)^2a_k-a_{k-1}&=0\\
%     &\vdots\\
% }
% \end{equation}

% 初值条件决定了
% \begin{equation}
% \leftgroup{
%     a_0=0\\
%     \sum\limits_{k=0}^\infty
% }
% \end{equation}



% \end{example}


\begin{example}{$n$ 阶贝塞尔方程}\label{ex_ODE2P_4}

$n$ 阶贝塞尔方程形如
\begin{equation}\label{eq_ODE2P_18}
x^2\frac{\mathrm{d}^2 y}{\dd x^2}+x\frac{\dd y}{\dd x}+(x^2-n^2)y=0~,
\end{equation}
其中 $n$ 是任意\textbf{非负}常数,\textbf{不一定是整数}。

首先把\autoref{eq_ODE2P_18} 改写为
\begin{equation}
\frac{\mathrm{d}^2 y}{\dd x^2}+\frac{1}{x}\frac{\dd y}{\dd x}+(1-\frac{n^2}{x^2})y=0~,
\end{equation}

由\autoref{the_ODE2P_1} ,它的特解形如
\begin{equation}\label{eq_ODE2P_19}
y=\sum\limits_{k=0}^\infty a_kx^{k+\alpha}~,
\end{equation}
其中 $a_k, \alpha$ 是待定常数,且 $a_0\neq 0$\footnote{总可以通过调整 $\alpha$ 使 $a_0\neq 0$。}。

将\autoref{eq_ODE2P_19} 代入\autoref{eq_ODE2P_18} ,可得
\begin{equation}\label{eq_ODE2P_20}
\begin{aligned}
&\sum\limits^\infty_{k=0}a_k(k+\alpha)(k+\alpha-1)x^{k+\alpha}\\+
&\sum\limits^\infty_{k=0}a_k(k+\alpha)x^{k+\alpha}\\+
&\sum\limits^\infty_{k=0}a_kx^{k+\alpha+2}\\-
&\sum\limits^\infty_{k=0}a_kn^2x^{k+\alpha}\\
&=0~,
\end{aligned}
\end{equation}

重新整理一下\autoref{eq_ODE2P_20} ,将 $x$ 的同次幂放在一起,得到
\begin{equation}
\begin{aligned}
\sum\limits^\infty_{k=0}[(k+\alpha)^2-n^2]a_kx^{k+\alpha}\\
+\sum\limits^\infty_{k=2}a_{k-2}x^{k+\alpha}=0~.
\end{aligned}
\end{equation}
令各项系数为 $0$,则得到一系列代数方程:
\begin{equation}\label{eq_ODE2P_21}
\leftgroup{
    (\alpha^2-n^2)a_0&=0\\
    [(1+\alpha)^2-n^2]a_1&=0\\
    [(2+\alpha)^2-n^2]a_2+a_0&=0\\
    [(3+\alpha)^2-n^2]a_3+a_1&=0\\
    &\vdots
}~
\end{equation}
由于 $a_0\neq 0$,故 $\alpha=\pm n$。

将 $\alpha=n$ 代入\autoref{eq_ODE2P_21} ,可以逐个计算出 $a_k$(除了 $a_0$)
\begin{equation}
\leftgroup{
    a_1&=0\\
    a_k&=-\frac{a_{k-2}}{k(k+2n)}, k=2, 3,\cdots
}~
\end{equation}
因此对于\textbf{奇数}$k$,$a_k=0$。

再把 $\alpha$ 和 $a_k$ 一起代入\autoref{eq_ODE2P_19} ,得到一个解
\begin{equation}\label{eq_ODE2P_22}
y_1=a_0x^n+\sum\limits^\infty_{k=2}\frac{(-1)^ka_0}{2^{2k}k!(n+1)(n+2)\cdots(n+k)}x^{n+2k}~,
\end{equation}


定义\textbf{Gamma 函数}\upref{Gamma}为:当 $s>0$ 时,$\Gamma(s)=\int_0^{+\infty}x^{s-1}\E^{-x}\dd x$;当 $s\leq 0$ 时,$\Gamma(s)=\frac{1}{s}\Gamma(s+1)$。

Gamma函数有两个性质:对正整数 $n$, $\Gamma(n+1)=n!$;$\Gamma(s+1)=s\Gamma(s)$。

\autoref{ex_ODE2P_4}  中的解\autoref{eq_ODE2P_22} 仍有一个待定常数 $a_0$。如果令 $a_0=\frac{1}{2^n\Gamma(n+1)}$,那么我们就能得到:
\begin{equation}\label{eq_ODE2P_23}
\begin{aligned}
y_1&=\sum\limits^\infty_{k=0}\frac{(-1)^k}{k!(n+1)(n+2)\cdots(n+k)\Gamma(n+1)}\qty(\frac{x}{2})^{n+2k}~.\\
 &=\sum\limits^\infty_{k=0}\frac{(-1)^k}{k!\Gamma(n+k+1)}\qty(\frac{x}{2})^{n+2k}~.
\end{aligned}
\end{equation}

\autoref{eq_ODE2P_23} 被称为\textbf{$n$ 阶贝塞尔函数},常记作 $J_n(x)$。


\end{example}



\begin{example}{}\label{ex_ODE2P_5}
依然是考虑\autoref{ex_ODE2P_4} 中的贝塞尔方程。

我们已经知道,对于 $n$ 阶贝塞尔方程,总有一个特解 $J_n(x)$。为了求出另一个线性无关的特解,我们考虑 $\alpha=-n$ 的情况。

类似地,通过\autoref{eq_ODE2P_21} 我们能得到($k$ 取遍正整数):
\begin{equation}
\leftgroup{
    &a_{2k-1}=0\\
    &a_{2k}=\frac{(-1)^ka_0}{2^{2k}k!(-n+1)(-n+2)\cdots(-n+k)}~.
}
\end{equation}

然后类似\autoref{ex_ODE2P_4} ,得到特解
\begin{equation}\label{eq_ODE2P_24}
y_2=a_0x^{-n}+\sum\limits^\infty_{k=1}\frac{(-1)^ka_0}{2^{2k}k!(-n+1)(-n+2)\cdots(-n+k)}x^{-n+2k}~.
\end{equation}

令 $a_0=\frac{1}{2^{-n}\Gamma(-n+1)}$,代入\autoref{eq_ODE2P_24} 得
\begin{equation}\label{eq_ODE2P_25}
y_2=\sum\limits^\infty_{k=0}\frac{(-1)^k}{k!\Gamma(-n+k+1)}\qty(\frac{x}{2})^{-n+2k}~.
\end{equation}
通常把\autoref{eq_ODE2P_25} 记作 $J_{-n}(x)$,称为\textbf{$-n$ 阶贝塞尔函数}。



\end{example}

$J_n$ 和 $J_{-n}$ 统称为\textbf{第一类贝塞尔函数}。设这两个函数的朗斯基行列式为$W[J_n, J_{-n}]$,那么由\textbf{阿贝尔微分方程恒等式}\upref{AbelID}得
\begin{equation}
W[J_n, J_{-n}]'(x)=-\frac{1}{x}W[J_n, J_{-n}](x)~,
\end{equation}
从而解得
\begin{equation}
W[J_n, J_{-n}](x) = \frac{C}{x}~,
\end{equation}
其中$C$是某个待定常数。

直接使用朗斯基行列式的定义,计算$xW[J_n, J_{-n}](x)$,得到一个贼长的级数和,但由于我们已经知道这个级数和是常数$C$,因此可以令$x\to 0$,结果依然应该是$C$,于是得到
\addTODO{最后一步是为什么?需要补充Gamma函数词条专门讨论。}
\begin{equation}
C = \frac{2n}{\Gamma(n+1)\Gamma(-n+1)} = \frac{2n}{n\Gamma(n)\Gamma(-n+1)} = -\frac{2\sin n\pi}{\pi}~,
\end{equation}

故
\begin{equation}\label{eq_ODE2P_26}
W[J_n, J_{-n}](x) = -\frac{2\sin n\pi}{\pi x}~,
\end{equation}

当$n$\textbf{不是整数}时,$J_{\pm n}$线性无关,因而可以用来将贝塞尔方程\autoref{eq_ODE2P_18} 的通解表示为
\begin{equation}
y(x)=C_1J_n(x)+C_2J_{-n}(x)~.
\end{equation}

但如果$n$\textbf{是整数},由\autoref{eq_ODE2P_26} 可知$J_{\pm n }$的朗斯基行列式为零,从而二式线性相关。因此,$n$为整数时常使用贝塞尔方程的另外一个特解来代替$J_{-n}$,即Neumann函数:
\begin{equation}
N_n = \frac{(\cos n \pi)J_n - J_{-n}}{\sin n \pi}~,
\end{equation}
当$n$为整数时,$N_n$定义为
\begin{equation}
N_n = \lim_{\nu\to n} N_\nu~.
\end{equation}


















