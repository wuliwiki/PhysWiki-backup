% 堆排序
% 堆排序|算法|排序|C++

前文讲到了堆\upref{heap}这个数据结构,这里来讲一下堆排序这个排序算法.

堆排序是用二叉堆这种数据结构实现的排序算法,从小到大排序的话是实现小根堆,而从大到小则相反.这里以实现从小到大排序为例.

具体做法是每次假设堆已经建好了,由于是小根堆,所以只需每次输出堆顶元素,再把堆顶删除,涉及到了三种操作:

\begin{enumerate}
\item 建堆;
\item 输出堆顶;
\item 删除堆顶.
\end{enumerate}

由于只有删除堆顶这个操作,所以只需要实现 down 操作.那如何来建堆呢?朴素方法是一个一个往堆中插入,但这种方法太慢了,有一个 $\mathcal{O}(n)$ 的建堆方式,就是从 $\dfrac{n}{2}$ down 到 $1$ 就可以了. 

为什么从 $\dfrac{n}{2}$ 开始 down 呢?假设堆中共有 $n$ 个结点,$n$ 结点的下标最大,$\dfrac{n}{2}$ 这个结点是有子结点的最大值,显然叶结点一定满足堆的性质,所以只需从 $\dfrac{n}{2}$ 开始建堆就能把堆键好.

为什么建堆的时间复杂度是 $\mathcal{O}(n)$ 呢?这里简单的地来证明一下.

\textbf{证明:}

一棵完全二叉树上有 $\left\lceil{\log_2 n}\right\rceil$ 层,叶子结点没有结点了,所以叶子结点不需要 down,所以从 $\dfrac{n}{2}$ 开始 down,所以除了叶子结点,上面的所有结点的个数为 $\dfrac{n}{2}$,除去叶子结点,上面的最后一层结点就是 $\dfrac{n}{4}$.

所以:

\begin{equation}
\dfrac{n}{2} \times 1 + \dfrac{n}{4} \times 2 + \cdots + \dfrac{n}{2^{\left\lceil{\log_2 n}\right\rceil}} \times \left\lfloor{\log_2 n}\right\rfloor \\
= \left\lfloor{\log_2 h}\right\rfloor \sum ^ {\left\lceil{\log_2 n}\right\rceil} _ {h=1} \dfrac{n}{2^h} \\
= n(\dfrac{1}{2^2} + \dfrac{1}{2^3} + \dfrac{1}{2^4} + \cdots +\dfrac{1}{2^{\left\lceil{\log_2 n}\right\rceil}}) \\
= n \sum^{\left\lceil{\log_2 n}\right\rceil} _ {h = 2} 2^h\\
\end{equation}

那么上面的公式是不是 $\mathcal{O}(n)$ 的呢?

\begin{equation}
s = \dfrac{1}{2^2} + \dfrac{1}{2^3} + \dfrac{1}{2^4} + \cdots +\dfrac{1}{2^{\left\lceil{\log_2 n}\right\rceil}} \\
, \ 2s = \dfrac{1}{2} + \dfrac{2}{2^2} + \dfrac{3}{2^3} + \cdots + \dfrac{\left\lceil{\log_2 n}\right\rceil}{2^{\left\lceil{\log_2 n}\right\rceil}} \\
, \ 2s - s = s= \dfrac{1}{2} + \dfrac{1}{2^2} + \dfrac{1}{2^3} + \cdots + \dfrac{1}{2^{\left\lceil{\log_2 n}\right\rceil}}
\end{equation}


因为 s 是小于 $1$ 的,所以 $\sum^{\left\lceil{\log_2 n}\right\rceil} _ {h = 2} 2^h < 1$,所以时间复杂度为 $\mathcal{O}(n)$.

down 操作在堆\upref{heap}这篇文章讲了,这里不再赘述.

