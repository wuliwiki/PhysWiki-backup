% cos2 波包
% license Usr
% type Tutor

\subsection{cos2 波包}
$\cos^2$ 波包也叫 $\sin^2$ 波包,比起\enref{高斯波包}{GausPk},它的优点是存在明确的范围。 它的函数形式为
\begin{equation}\label{eq_cos2wp_5}
f(x) = \leftgroup{
&A\cos^2\qty(\frac{\pi x}{L}) \E^{\I k x} && (\abs{x} < L/2)\\
&0 && (\text{otherwise})~.
}\end{equation}
其中 $L$ 是波包的总长度。 它的 FWHMI 为
\begin{equation}\label{eq_cos2wp_4} % 已数值验证
\text{FWHMI} = \frac{2}{\pi}\opn{acos}(2^{-1/4}) L \approx 0.3640567 L~.
\end{equation}
满足 $f^2(\text{FWHMI/2}) = f^2(0)/2$。

积分为(令 $a = \pi/L$)
\begin{equation}\label{eq_cos2wp_2} % Mathematica 已验证 (原 eq_GausPk_2)
\begin{aligned}
&\quad \int A\cos^2\qty(a x) \E^{\I kx} \dd{x} = -\I\frac{A}{4}\E^{\I kx}\qty(\frac{2}{k} +\frac{\E^{2\I ax}}{2a+k} -\frac{\E^{-2\I ax}}{2a-k}) + \frac{2\I Aa^2\cos[{\pi k}/{(2a)}]}{k(4 a^2 -k^2)} + C\\
&= \frac{Ax}{4} \qty(2\sinc(k x) + \sinc[(2 a+k)x] + \sinc[(2 a-k)x])\\
&\quad + \I\frac{A}{4} \qty(-\frac{2\cos(k x)}{k} -\frac{\cos[(2 a+k)x]}{2 a+k} +\frac{\cos[(2 a-k)x]}{2 a-k}) + \frac{2\I Aa^2\cos[{\pi k}/{(2a)}]}{k(4 a^2 -k^2)} + C~.
\end{aligned}
\end{equation}
$C$ 前面的部分在 $x = \pm\pi/(2a)$ 处分别为 $\frac{\pm 2 Aa^2\sin[{\pi k}/{(2a)}]}{k(4 a^2 -k^2)}$,易得无穷定积分只有实部
\begin{equation}% Mathematica 已验证
\int_{-\infty}^{+\infty} f(x) \dd{x} = \int_{-\pi/(2a)}^{\pi/(2a)} A\cos^2\qty(a x)\E^{\I k x} \dd{x} = \frac{4 Aa^2\sin[{\pi k}/{(2a)}]}{k(4 a^2 -k^2)}~.
\end{equation}
另外,实部是奇函数,虚部是偶函数。

傅里叶变换(注意是实数):
\begin{equation} % Mathematica 已验证 + 数值已验证
\tilde f(k) = \frac{1}{\sqrt{2\pi}} \int_{-\infty}^{+\infty} f(x) \E^{-\I \omega x} \dd{x}
= \frac{\sqrt{2} \pi ^{3/2} A L}{4 \pi ^2-L^2 (\omega-k)^2} \sinc[L(\omega-k)/2]~.
\end{equation}
零点的位置为
\begin{equation} % 数值已验证
k = k_0 \pm 2n\pi/L \qquad (n=2,3,\dots)~.
\end{equation}
标准差约为 $2.92544/L$, 方差 $13.15947/L^2$, FWHMI $9.05144/L$。

画图对比如下(代码见文末):
\begin{figure}[ht]
\centering
\includegraphics[width=13cm]{./figures/77264c5d4c0cdbd0.png}
\caption{高斯波包和 cos2 波包的对比} \label{fig_cos2wp_1}
\end{figure}

\subsection{附:Matlab 画图代码}
\begin{lstlisting}[language=matlab, caption=cos2\_spec.m]
% properties of cos2 wave packet spectra

A = 0.9; L = 1.12;
g = @(k) (sqrt(2)*pi^1.5*A*L)./(4*pi^2-L^2*k.^2) .* sinc(L*k/2);

k = linspace(-40, 40, 1000);
figure; plot(k, g(k));
grid on;
xlabel k;
hold on;
scatter((2:6)*2*pi/L, 0, 'k');
scatter((-6:-2)*2*pi/L, 0, 'k');
axis([-40, 40, -0.02, 0.21]);

A = 1/sqrt(integral(@(k)g(k).^2, -inf, inf));
integral(@(k)g(k).^2.*abs(k).*A^2, -inf, inf)
\end{lstlisting}

\begin{lstlisting}[language=matlab,caption=FWHMIsin2]
% FWHMI of wave packets
% FWHMI: full width half maximum intensity
% return the ratio of FWHMI of sin2 (field) wave v.s. total duration
% satisfy: |cos(pi/2 * FWHMIsin2)^2|^2 == 1/2

function ret = FWHMIsin2
    ret = 2*acos(2^(-1/4))/pi;
end
\end{lstlisting}

\begin{lstlisting}[language=matlab, caption=cos2\_gaussian\_compare.m]
% plot Gaussian vs cos2 profile

% gaussian
FWHMI = 1;
a = iFWHMIexp(FWHMI);
x = linspace(-2*FWHMI, 2*FWHMI, 1000);
field_gauss = exp(-a.*x.^2);

% cos2
field_cos2  = zeros(size(x));
dur_cos2 = FWHMI / FWHMIsin2;
mark = abs(x) < dur_cos2/2;
field_cos2(mark) = cos((pi/2)*x(mark)/(dur_cos2/2)).^2;

% plot field profile
figure;
subplot(2, 1, 1); hold on;
axis([min(x), max(x), 0, 1.1]);
plot_vert(-FWHMI/2, 'c--');
plot_vert(FWHMI/2, 'c--');
plot_hori(sqrt(1/2), 'c--');
plot(x, field_gauss, 'r');
plot(x, field_cos2, 'b--');
legend({'', '', '', 'Gaussian', 'cos2'});
% xlabel('t [FWHM]');
ylabel('field');
title('Gaussian vs cos2 profile (lines show FWHMI)');

% plot intensity profile
subplot(2, 1, 2); hold on;
axis([min(x), max(x), 0, 1.1]);
plot_vert(-FWHMI/2, 'c--');
plot_vert(FWHMI/2, 'c--');
plot_hori(1/2, 'c--');
plot(x, field_gauss.^2, 'r');
plot(x, field_cos2.^2, 'b--');
xlabel('t [FWHM]');
ylabel('intensity');
\end{lstlisting}

