% 变分的基本定理(变分学)
% 变分|基本定理


\pentry{变分的变换\upref{VarCha}}
本节给出的两个定理,将有助于引出欧拉方程。其中一个定理属于拉格朗日,另一个属于黎曼。由拉格朗日推演的欧拉方程是不精确的,这可从拉格朗日变换\upref{VarCha}的定义中看到。而由黎曼推演可精确的推出欧拉方程,并且得到在极值曲线的正规点上,$y''$ 存在,而这是事先并未假定的。本节将证明推演欧拉方程用到的这两个定理。

\begin{theorem}{(拉格朗日)}\label{VarDef_the1}
若对任意的属于 $C_1$ 类(\autoref{Varia_sub1}~\upref{Varia})的函数 $\eta(x)$,并且其满足 $\eta(a)=\eta(b)=0$,属于 $C_1$ 类的函数 $M(x)$ 都有
\begin{equation}\label{VarDef_eq1}
\int_a^b M(x)\eta(x)\dd x=0~,
\end{equation}
则对于一切的 $a\leq x\leq b,M(x)=0$.
\end{theorem}
\begin{theorem}{(黎曼)}\label{VarDef_the2}
若对任意的属于 $C_1$ 类的函数 $\eta(x)$,并且其满足 $\eta(a)=\eta(b)=0$,属于 $C_1$ 类的函数 $M(x)$ 都有
\begin{equation}\label{VarDef_eq2}
\int_a^b M(x)\eta'(x)\dd x=0
\end{equation}
则对于一切的 $a\leq x\leq b$,$M(x)$ 为常数。
\end{theorem}
\subsection{证明}
以下将用反证法证明。
\subsubsection{\autoref{VarDef_the1} 的证明}
设在区间 $[a,b]$  上某点 $c$,$M(c)\neq0$。例如 $M(c)>0$ ,由于 $M(x)$ 的连续性,取充分大的 $n$,可以得到包含在 $[a,b]$ 内的区间 $[x_0,x_0+\frac{\pi}{n}]$ ,它包含着 $c$ 点,并且在它上面 $M(x)$ 大于某一正数 $m$。
定义
\begin{equation}
\eta_0(x)=\leftgroup{
&\sin^2 [n(x-x_0)], \quad & x\in [x_0,x_0+\frac{\pi}{n}]\\
&0, \quad & other
}\end{equation}
则函数 $\eta_0(x)$ 是 $C_1$ 类的,且 $\eta_0(a)=\eta_0(b)=0$。因此,由\autoref{VarDef_the1} 条件,\autoref{VarDef_eq1} 成立。
但
\begin{equation}
\begin{aligned}
&\int_a^b M(x)\eta_0(x)\dd x\\
&=\int_{x_0}^{x_0+\frac{\pi}{n}}M(x)\sin^2[n(x-x_0)]\dd x\\
&>m\int_{x_0}^{x_0+\frac{\pi}{n}}\sin^2[n(x-x_0)]\dd x=\frac{\pi m}{2n}>0
\end{aligned}
\end{equation}
这与\autoref{VarDef_eq1} 矛盾,故\autoref{VarDef_the1} 得证。
\subsubsection{\autoref{VarDef_the2} 的证明}
设 $M(x)$ 不是常数,则在区间 $[a,b]$ 上,至少有两点 $c_1,c_2$,使得 $M(c_1)\neq M(c_2)$ ,设 $M(c_1)>M(c_2)$。由 $M(x)$ 的连续性,存在 $d_1,d_2$,使得
\begin{equation}
M(c_1)>d_1>d_2>M(c_2)
\end{equation}
并且取充分大的 $n$,可以得到包含在 $[a,b]$ 内的一对区间 $[x_0,x_0+\frac{\pi}{n}]$ 和 $[x_1,x_1+\frac{\pi}{n}]$,它们分别包含着点 $c_1$ 和 $c_2$,并且在前一区间上面 $M(x)>d_1$,而在另一区间上, $M(x)<d_2$。 定义函数
\begin{equation}
\eta'(x)=\leftgroup{
&\sin^2[n(x-x_0)],\quad &x\in[x_0,x_0+\frac{\pi}{n}]\\
&-\sin^2[n(x-x_1)],\quad &x\in[x_1,x_1+\frac{\pi}{n}]\\
&0,\quad &other
}\end{equation}
显然,$\eta'(x)$ 连续,所以函数 $\eta(x)=\int_a^x\eta'(x)\dd x$ 连续且有连续微商 $\eta'(x)$,并且 $\eta(a)=\eta(b)=0$,由\autoref{VarDef_the2} 条件,\autoref{VarDef_eq2} 成立,但
\begin{equation}
\begin{aligned}
&\int_a^b M(x)\eta'(x)\dd x=\\
&\int_{x_0}^{x_0+\frac{\pi}{n}}M(x)\sin^2[n(x-x_0)]\dd x-\int_{x_1}^{x_1+\frac{\pi}{n}}M(x)\sin^2[n(x-x_1)]\dd x\\
&>(d_1-d_2)\int_0^{\frac{\pi}{n}}\sin^2nx\dd x>0
\end{aligned}
\end{equation}
这与\autoref{VarDef_eq2} 矛盾,故\autoref{VarDef_the2} 得证。