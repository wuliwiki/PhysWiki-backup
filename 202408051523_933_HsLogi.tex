% 数理逻辑(高中)
% keys 逻辑|高中|命题|且|或|非|推理|
% license Usr
% type Tutor

\begin{issues}
\issueDraft
\end{issues}
\pentry{集合\nref{nod_HsSet}}{nod_fc7f}

\textbf{逻辑}是研究如何进行正确推理和论证的学科。逻辑能帮助你对事物进行理解和判断。在数学中,逻辑用于证明定理,确保论证的严谨性和准确性。\textbf{数理逻辑}是逻辑学的一个分支,它应用数学的方法研究逻辑。

尽管高中数学教材中在逐渐弱化逻辑的内容,但作为数学根基的内容,学习逻辑对于理解高中的知识内容(不仅是数学学科,对于理科内容甚至是文科内容)具有相当的助益。同时,这一部分内容在本科学习阶段往往也会作为学生已知的部分略讲或跳过。因此,在高中阶段接触和学习一部分逻辑内容是必要的。

\subsection{命题}

一条可供判断的陈述句被称为\textbf{命题}(proposition)。

如果一条命题被判断为真,则称为\textbf{真命题},判断为假,则称为\textbf{假命题}。一个命题要么为真,要么为假。如果一条陈述句既真又假,或者模棱两可,无法判断真假,则不能称为命题。一般这种情况出现时,或者是没有言明前提条件的限定范围,或者是陈述句中的某些概念的定义不清晰。例如:“这个班级中长得好看的人占比$30\%$”就不称为命题,因为“好看的人”是一个定义不清晰的概念。

如果一个命题里不包含变量,那么这个命题被称为\textbf{原子命题}(atomic Proposition)。这种命题直接陈述一个事实,并且它的真假值是固定的。例如,“自然数2是一个偶数”是一个简单命题,它总是真的。

如果一个命题包含变量,那么这个命题被称为\textbf{开放命题}(open proposition)。这种命题的真假值取决于变量的取值。例如,“x 是一个偶数”就是一个含变量的命题,其中 x 可以取不同的值,因此命题的真假也会随之变化。如果 x=2,为真命题;如果 x=3,为假命题。如果一个开放命题不论变量怎么取值总是真命题,则称其\textbf{恒成立}。

命题使用\textbf{真值}(truth value)来表示命题的真假,因此陈述命题有确定的真值,而开放命题的真值与变量相关。真值用\textbf{布尔值}表示。布尔值有两个,分别为“\textbf{真}”和“\textbf{假}”,分别记作:“${\rm True}$”、“$1$”、“$T$”和“${\rm False}$”、“$0$”、“$F$”。



\subsection{*与命题相似、相关的词汇}

下面给出的是一些经常使用的与命题相似或相关的词汇。他们大多是各个学科学习时,都会涉及到的概念,建议你读一读,也可以直接\aref{跳过}{sub_HsLogi_1}。

\subsubsection{概念}

\textbf{概念}(concept)是人类思维中用来表示某类事物、属性或关系的一种基本单位。它是对现实世界中事物的抽象和一般化。概念通过定义、特征或属性来描述和区分不同的对象、现象或情况。给出一个概念的定义时,通常会给出这个概念的内涵或外延,内涵就是指反映在概念中的对象的本质属性或特有属性。外延是指具有概念所反映的本质属性或特有属性的对象,即概念的适用范围。

内涵是“属性”,外延是“范围”。属性越多,则符合这个属性的范围越小。属性越少,符合这个属性的范围就越大。

概念的内涵是指概念的质的方面。
概念的外延是指概念的量的方面。

\subsubsection{定义}

定义是对某一概念的精确描述或解释。定义帮助我们清晰地理解和区分不同的概念。例如,命题的定义是“一个可以明确判断真假的陈述句”。

\subsubsection{公理}

公理(axiom)是逻辑系统中的基本假设,它们不需要证明,被视为显而易见的真理。公理是其他定理推导的基础。例如,在欧几里得几何中,“通过两点可以作一条直线”就是一个公理。

\subsubsection{猜想}

由于命题的要求是“可供”判断,至于怎么样才能判断命题的真假则不包含在内。比如,“每一个不小于6的偶数都是两个素数的和”,这句话本身是可供判断的,比如只要有人给出一个反例就说明这句话是假的,又或者如果有人证明出来就说明他是真的。但是在当下,这句话我们尚不知道它的真伪。因此,称之为猜想。刚才给出的例子就是著名的“哥德巴赫猜想”。

\subsubsection{悖论}

悖论(paradox)是逻辑中的一些在逻辑推理中出现的表面上矛盾的结论,通常是因为一些隐含假设或定义问题导致的。悖论后面蕴涵着深刻的思想,并且往往随着悖论的解决,带来哲学、逻辑、数学等领域的巨大变迁。

\subsection{复杂命题}\label{sub_HsLogi_1}

由逻辑连接词组合的命题称为\textbf{复杂命题}(compound proposition),逻辑连接词分为:且、或、非。

\subsubsection{且}

\textbf{且}(and,也称为\textbf{同}、\textbf{合取})记号为$A\land B$。

\subsubsection{或}

\textbf{或}(or,也称为\textbf{或者}、\textbf{析取})记号为$A\lor B$。


\subsubsection{非}
\textbf{非}(not,也称为\textbf{否定})记号为$\lnot A$。

\subsection{量词命题}

\subsubsection{全称量词}

在陈述中表达所述事物的全体的含义时,使用的量词称为\textbf{全称量词}(universal quantifier),记作$\forall$,可以读作“任意”、“所有”、“每一个”等。形如“集合$M$中所有的元素$x$,都满足性质$P(x)$”的命题称为\textbf{全称量词命题},记作:
\begin{equation}
\forall x\in M,P(x).~
\end{equation}


\subsubsection{存在量词}

在陈述中表达所述事物的个体、部分或特例的含义时,使用的量词称为\textbf{存在量词}(existential quantifier,也称作\textbf{特称量词},Particular Quantifier),记作$\exists$,可以读作“存在”、“有”、“至少有一个”等。形如“集合$M$中存在某个元素$x$满足性质$P(x)$”的命题称为\textbf{存在量词命题},记作:
\begin{equation}
\exists x\in M,P(x).~
\end{equation}

\subsection{条件命题}
“若  $p$  则  $q$ ” 是一种\textbf{条件命题}(conditional proposition或\textbf{蕴含命题},implication)。它表示如果命题  $p$  为真,那么命题  $q$  也为真。条件命题通常用符号$p\Rightarrow q$表示。
如果这个条件命题成立,则$p$称为$q$的充分条件,反过来看则$q$称为$p$的必要条件。

\subsubsection{充要条件}
当且仅当

逆命题、否命题、逆否命题

\subsection{推理方法}

\subsubsection{演绎推理}

从一般性原则推导出具体结论的推理方法,称作\textbf{演绎推理}(Deductive Reasoning)。演绎推理的特点便是,只要保证推理模式和前提正确,那么结论必然为真。在数学领域,最常做的便是建立公理和概念,之后进行演绎推理。最有名的演绎推理便是苏格拉底的“三段论”(Syllogism)。

\begin{example}{证明“苏格拉底会死”}
\begin{itemize}
\item 大前提:所有人都会死。
\item 小前提:苏格拉底是人。
\item 结论:苏格拉底会死。
\end{itemize}
\end{example}

演绎推理通常非常可靠,一般得出错误结论的场合,都是没有正确审视前提

\subsubsection{归纳推理}

从具体的、个别的观察出发,得出一般性的结论的推理方法,称为\textbf{归纳推理}(Inductive Reasoning)。在生活、学习、研究中,我们总是能观察到很多现象,有些现象是偶然出现的,也有些现象是循环重复的,通过观察、猜想、总结,就能得到很多的结论。

但是,就像哪怕寒假的时候,你都在家,也不是永远在家。由于,大多数能够被观察到的现象都并不是现象的全体,只是总体的一部分,也就是说,归纳推理是从基于有限的观测得出的,因此即使所有前提为真,结论也不一定为真,得到的结论并不必然,而是具有一定可能性的。就像,如果有人通过寒假观察了两周就归纳出了一个“你永远会在家”的结论,显然是不合理的。

很多人对归纳推理嗤之以鼻,觉得它不可靠,但是在探索未知领域,尤其是没有现成体系或前人经验的领域时,归纳推理非常有用。它可以从现象中从无到有,归纳出新的猜想方向。尽管归纳得出的结论受到观察样本数量的限制,可能存在以偏概全的风险,但相比于完全未知的准确结论,一个模糊但可获得的结论或者准确但未必正确的方向,对实际情况更有帮助。因此,尽管归纳推理并不总是严格,但只要在使用时保持警惕,它依然非常有用。

\subsubsection{对于处理题目的启发}

做数学题的过程,也就是进行演绎推理的过程。保证自己的基础知识牢固,对题目中的信息理解正确都是在保证演绎推理的前提不出错。对条件应用的范围准确、每一步的严丝合缝都是在保证演绎的过程准确。这样就能得到正确的结果。而在处理陌生领域、方向不甚清晰的问题时,先举出一些例子,然后通过归纳来猜想出大概的方向,往往是找到思路的关键。