% 原子单位制
% 原子|国际单位|薛定谔方程|量子力学

\pentry{无单位的物理公式\upref{NoUnit}, 玻尔原子模型\upref{BohrMd}}

\footnote{参考 Wikipedia \href{https://en.wikipedia.org/wiki/Hartree_atomic_units}{相关条目}. 以及 \cite{Brandsen} 的附录.}在量子力学的许多理论或数值计算中,选用\textbf{原子单位(Hartree atomic unit)}会更方便. 需要特别注意的是, 原子单位下的物理公式在使用时的习惯和其他单位制有所不同, 例如会出现 $E = \omega$ (能量等于角频率, 见\autoref{AU_eq2})这种看似不符合量纲分析的公式. 要理解这种记号, 这里推荐的方法是把使用原子单位的公式都理解成无单位的物理公式\upref{NoUnit}.

若声明使用原子单位, 物理量的数值后面就不需要标记单位, 例如 “氢原子基态能量为 $E \approx -0.5$”. 有时候为了强调我们使用原子单位, 我们会在数值后面加上 “a.u.”, 如 “$E \approx 0.5 \Si{a.u.}$”. 这里的 “a.u.” 可等效为 “单位 1”, 类似弧度单位 “Rad”\footnote{例如在扇形面积公式 $S = \theta R^2/2$ 中的 $\theta$ 可以看作具有单位 “Rad”, 但面积的单位却只是 “$\Si{m^2}$” 而无需记为 “$\Si{Rad\cdot m^2}$”.}.

\subsection{推导}
令角动量公式 $\bvec L = m \bvec r \cross \bvec v$ 成立, 那么
\begin{equation}
\beta_L = \frac{\beta_m \beta_x^2}{\beta_t}
\end{equation}
由于 $\$

这么做是为了

令动能公式 $E_k = mv^2/2$ 成立, 那么
\begin{equation}
\beta_E = \frac{\beta_m \beta_x^2}{\beta_t^2}
\end{equation}

\pentry{薛定谔方程\upref{TDSE}}
国际单位下的一维含时薛定谔方程为
\begin{equation}\label{AU_eq1}
-\frac{\hbar^2}{2m} \pdv[2]{\Psi}{x} + V\Psi= \I\hbar \pdv{\Psi}{t}
\end{equation}
与“无单位的物理公式\upref{NoUnit}” 中的方法类似, 我们需要给公式中出现的每个单位定义一个常数, 分别记为 $\beta_x$, $\beta_m$, $\beta_t$, $\beta_E$, $\beta_\Psi$, 且令 $x = x_a\beta_x$, $m = m_a\beta_m$, $t = t_a\beta_t$, $V = V_a\beta_E$, $\Psi = \Psi_a \beta_\Psi$ (注意 $x_a, m_a$ 等都是不带单位的).
代入\autoref{AU_eq1}, 各项同除 $\beta_E\beta_\Psi$, 得\footnote{根据偏微分的定义, 常数可以移到偏微分算符外, 如 $\pdv*[2]{(\beta_x x_a)} = (1/\beta_x^2) \pdv*[2]{x_a}$}
\begin{equation}\label{AU_eq3}
-\qty(\frac{\hbar^2}{\beta_m\beta_x^2\beta_E})\frac{1}{2m_a} \pdv[2]{\Psi_a}{x_a} + V_a\Psi_a= \I\qty(\frac{\hbar}{\beta_E\beta_t})\pdv{\Psi_a}{t_a}
\end{equation}
为了让公式尽可能简洁, 我们令两个括号都为 1, 得
\begin{equation}\label{AU_eq6}
\beta_t = \frac{\beta_m\beta_x^2}{\hbar}
\qquad
\beta_E = \frac{\hbar}{\beta_t}
\end{equation}



于是无单位的薛定谔方程为
\begin{equation}\label{AU_eq4}
-\frac{1}{2m_a} \pdv[2]{\Psi_a}{x_a} + V_a\Psi_a= \I\pdv{\Psi_a}{t_a}
\end{equation}
我们再看波函数的归一化公式
\begin{equation}
1 = \int \abs{\Psi}^2 \dd{x} = \beta_\Psi^2 \beta_x \int \abs{\Psi_a}^2 \dd{x_a}
\end{equation}
为了使归一化公式的形式不变, 必须令
\begin{equation}\label{AU_eq5}
\beta_\Psi = \beta_x^{-1/2}
\end{equation}
同理, 对 $N$ 维波函数有 $\beta_\Psi = \beta_x^{-N/2}$.
由\autoref{AU_eq5} 和\autoref{AU_eq6} 可知我们只剩下两个自由的转换常数, 例如只要确定 $\beta_x$ 和 $\beta_m$, 剩下的 $\beta$ 也就确定了.

我们来看光子的能量公式
\begin{equation}
E = \hbar \omega \Longrightarrow E_a = \frac{\hbar\beta_\omega}{\beta_E}\omega_a
\end{equation}
为满足周期的定义($T_a = 2\pi/\omega_a$), 必须令 $\beta_\omega = 1/\beta_t$. 将上文得 $\beta_E$ 和 $\beta_t$ 代入可得原子单位下光子能量等于角频率
\begin{equation}\label{AU_eq2}
E_a = \omega_a
\end{equation}
我们以后会发现, 使用原子单位后所有公式中的普朗克常数 $\hbar$ 都恰好被消去了, 所以一些文献中直接说 “原子单位中 $\hbar = 1$”. 这句话的一种理解方式是把 $\hbar$ 看成一个角动量常量, 而原子单位制中角动量的转换常数恰好为 $\beta_L = \hbar$ (见\autoref{AU_eq1}), 所以转换后该角动量常数等于 1a.u.

\subsection{原子单位}
最常见的情况下, 原子单位定义 $\beta_m$ 等于电子的质量, $\beta_x$ 等于玻尔半径, 再由\autoref{AU_eq6} 和\autoref{AU_eq5} 确定 $\beta_E, \beta_t, \beta_\Psi$, 如\autoref{AU_tab1} 所示\footnote{为了区别能量与电场,以下用 $E$ 表示能量,用 $\mathcal{E}$ 表示电场.} . 注意许多常数都与氢原子的玻尔模型\upref{BohrMd}(原子核不动)的基态(表中简称基态)有关.

\begin{table}[ht]
\caption{原子单位转换常数表}\label{AU_tab1}
\begin{tabular}{|c|c|c|c|}
\hline
物理量 & $\beta$ & 描述 & 数值(国际单位)\\
\hline
质量 $m$ & $m_e$ & 电子质量 & $9.10938215\e{-31}$ \\
\hline
\dfracH 长度 $x$ & $a_0 = \dfrac{4\pi \epsilon_0 \hbar ^2}{m_e e^2}$ & 玻尔半径 & $5.2917721067\e{-11}$ \\
\hline
\dfracH 速度 $v$ & $\dfrac{\hbar}{m_e a_0}$ & 基态电子速度 & $2.1876912633\e6$ \\
\hline
时间 $t$ & $m_e a_0^2/\hbar$ & 长度除以速度 & $2.418884326\e{-17}$\\
\hline
\dfracH 角频率 $\omega$ & $\dfrac{\hbar}{m_e a_0^2}$ & 时间的倒数 & $6.579683921 \times {10^{15}}$ \\
\hline
\dfracH 能量 $E$ & $\dfrac{\hbar^2}{m_e a_0^2} = \dfrac{e^2}{4\pi \epsilon_0 a_0}$ & 基态电子势能大小 & $4.3597446499\e{-18}$ \\
\hline
角动量 $L$ & $m_e v_0 a_0 = \hbar$ & 长度乘以动量 & $1.054571800\e{-34}$ \\
\hline
电荷 $q$ & $e$ 或 $q_e$ & 电子电荷 & $1.6021766208\e{-19}$\\
\hline
\dfracH 电场强度 $\mathcal{E}$ & $\dfrac{e}{4\pi \epsilon_0 a_0^2}$ & 基态轨道电场强度 & $5.1422067070\e{11}$ \\
\hline
\dfracH 磁感应强度 $B$ & $\dfrac{\hbar}{ea_0^2}$ &  & $2.350517567\e5$\\
\hline
\dfracH 电势 $V$ & $\dfrac{e}{4\pi\epsilon_0 a_0}$ & 基态轨道电势 & $27.211386019$ \\
\hline
\end{tabular}
\end{table}

表中还定义了一些其他的物理量的转换常数, 它们的定义可以使以下无单位公式成立(以后我们在不至于混淆的情况下省略角标 $a$)
\begin{equation}
\omega = \frac{2\pi}{T}
\end{equation}
\begin{equation}
x = v t
\end{equation}
\begin{equation}
\bvec L = m\bvec r \cross \bvec v  \qquad \text{(角动量)}
\end{equation}
\begin{equation}
\mathcal{E} = \frac{q}{r^2} \qquad \text{(点电荷电场)}
\end{equation}
\begin{equation}
\rho_{eng} = \frac{1}{8\pi} \abs{\bvec E}^2 \qquad \text{(电磁场能量密度)}
\end{equation}
\begin{equation}
\bvec F = q \bvec v \cross \bvec B \qquad \text{(洛伦兹力)}
\end{equation}
\begin{equation}
U = \frac{q}{r} = \mathcal{E} x \qquad \text{(点电荷电势)}
\end{equation}
\begin{equation}\label{AU_eq11}
V = qU = -q\mathcal{E} x \qquad \text{(匀强电场电势能)} 
\end{equation}
\begin{equation}
\bvec {\mathcal{E}} = -\grad \varphi - \pdv{t}\bvec A \qquad \text{(标势和矢势)}
\end{equation}
\begin{equation}
\bvec B = \curl \bvec A \qquad \text{(标势和矢势)}
\end{equation}

薛定谔方程为
\begin{equation}\label{AU_eq12}
-\frac{1}{2m} \pdv[2]{\Psi}{x} + V\Psi= \I\pdv{\Psi}{t}
\end{equation}
注意当考察对象为电子时, 式中 $m = 1$, 可省略.

\begin{example}{匀强电场中电子的薛定谔方程}
令\autoref{AU_eq12} 中 $m = 1$, $q = -1$, 再将\autoref{AU_eq11} 代入, 得
\begin{equation}
-\frac12 \pdv[2]{\Psi}{x} + \mathcal{E} x \Psi= \I\pdv{\Psi}{t}
\end{equation}
\end{example}

\begin{exercise}{氢原子的基态能量}
计算玻尔模型中氢原子基态的能量(答案:$-1/2$).
\end{exercise}

\subsubsection{电磁学相关}
把精细结构常数\upref{FinStr}记为 $\alpha$. 国际单位中有 $a_0 = \hbar/(\alpha c m_e)$. 原子单位中有 $c = 1/\alpha$, $\epsilon_0 = 1/(4\pi)$, $\mu_0 = 4\pi\alpha^2$, $\mu_0\epsilon_0 = \alpha^2$, 电磁场能量密度 $\rho_E = (\abs{\bvec E}^2  + \abs{c\bvec B}^2)/(8\pi)$. 坡印廷矢量为 $\bvec s = \bvec E \cross \bvec B/(4\pi\alpha^2)$. 对平面电磁波 $E = cB$.  

\subsection{另一种原子单位}

当问题涉及一主要角频率 $\omega$ 的时候(例如研究原子在单频激光中的变化),可选择 $\beta_E = \hbar\omega$ 做能量单位. 同样令 $\beta_m$ 等于电子质量, $\beta_q$ 等于元电荷, 由\autoref{AU_eq6} 得
\begin{equation}\label{AU_eq15}
\beta_x = \sqrt{\frac{\hbar}{m_e\omega}}
\qquad
\beta_t = \frac{1}{\omega}
\end{equation}
为了使\autoref{AU_eq11} 成立,得
\begin{equation}
\beta_\mathcal{E} = \frac{\hbar\omega}{e \beta_x}
\end{equation}
一种常见的情况是平面电磁波中的电场用国际单位表示为 $\mathcal{E}(t) = \mathcal{E}_0\cos(\omega t)$, 而原子单位下该式为
\begin{equation}
\mathcal{E}(t) = \mathcal{E}_0\cos t
\end{equation}
注意右边不含 $\omega$, 形式更简洁.

另一个常见的例子是简谐振子\upref{QSHOxn}, 若使用其振动的固有频率来定义 $\beta_t$, 其薛定谔方程为
\begin{equation}\label{AU_eq18}
-\frac12 \pdv[2]{\Psi}{x} + \frac12 x^2 \Psi= \I\pdv{\Psi}{t}
\end{equation}
能级为
\begin{equation}\label{AU_eq19}
E_n = \frac12 + n \qquad (n = 0, 1, 2\dots)
\end{equation}
归一化的基态波函数为
\begin{equation}
\psi_0(x) = \pi^{-1/4} \E^{-x^2/2}
\end{equation}


\begin{example}{转换为含单位的公式}
现在我们按照“无单位的物理公式\upref{NoUnit}” 中介绍的方法将\autoref{AU_eq18} 转换为含单位的公式. 即先把所有无单位的物理量替换成有单位的物理量除以对应的 $\beta$ 常数, 得(两边已同乘 $\beta_\Psi$)
\begin{equation}
-\beta_x^2\frac12 \pdv[2]{\Psi}{x} + \frac{1}{\beta_x^2}\frac12 x^2 \Psi= \beta_t\I\pdv{\Psi}{t}
\end{equation}
将\autoref{AU_eq15} 代入, 两边乘以 $\omega\hbar$ 得国际单位下的简谐振子薛定谔方程
\begin{equation}
-\frac{\hbar^2}{2m} \pdv[2]{\Psi}{x} + \frac12 m\omega^2 x^2 \Psi= \I\hbar\pdv{\Psi}{t}
\end{equation}

类似地, 也可以将\autoref{AU_eq19} 变为
\begin{equation}
E =  \qty(\frac12 + n)\beta_E = \qty(\frac12 + n)\omega\hbar \qquad (n = 0, 1, 2\dots)
\end{equation}
\end{example}
