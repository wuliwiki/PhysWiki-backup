% 合同变换
% license Xiao
% type Tutor


给定域$\mathbb F$上的线性空间$V$,$f$为对称双线性函数:$V\times V\to \mathbb F$,也就是二次型$q(v),v\in V$。再给定$V$上的一组基$\{\boldsymbol e_i\}$,则二次型可以表示为:
\begin{equation}
\eta_{ij}=f(\boldsymbol {\mathrm e}_i,\boldsymbol {\mathrm e}_j)~,
\end{equation}
对于$V$的另一组基$\{\boldsymbol \theta_i\}$,二次型可以表示为$\Theta_{ij}=f(\boldsymbol {\mathrm \theta}_{i},\boldsymbol {\mathrm \theta}_{j})$。设过渡矩阵为$T^i_j,\boldsymbol {\theta}_i=T^j_i\boldsymbol e_j$。那么我们可以看到二次型在不同基下的表示是如何通过过渡矩阵联系在一起的:
\begin{equation}
\begin{aligned}
\Theta_{ij}&=f(\boldsymbol {\mathrm \theta}_{i},\boldsymbol {\mathrm \theta}_{j})\\
&=f(T^r_i\boldsymbol e_r,T^s_j\boldsymbol e_s)\\
&=T^r_i \eta_{rs}T^s_j~,
\end{aligned}
\end{equation}
把上式两端的二次型张量写作(1,1)型矩阵,则有:
\begin{equation}
\begin{aligned}
\Theta^i_j&=T^i_r\eta^i_jT^s_j\\

\end{aligned}
\end{equation}