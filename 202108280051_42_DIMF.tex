% 量纲式
% 量纲式|物理规律

\begin{issues}
\issueTODO
\end{issues}

\pentry{单位制和量纲\upref{USD}}
在单位制和量纲\upref{USD}这一节中我们提到了量纲式,并且给出了\autoref{USD_def1}~\upref{USD},本节将证明量纲式的通用表达式,并给出另外一种较为更“数学”的定义.本节的定理将给出一个极其重要的结论,它使得我们对物理规律有一个更深刻的认识.
\subsection{定义方程}
在单位制和量纲\upref{USD}这一节中,\autoref{USD_ex1}~\upref{USD} 和\autoref{USD_ex2}~\upref{USD} 提到了定义方程和终极定义方程,我们先给出它们的定义.
\begin{definition}{}
在单位制 $\mathscr{Z}$ 中,定义导出量类 $\tilde{\boldsymbol{C}}$ 单位 $\hat{\boldsymbol{C}}_{\mathscr{Z}}$ 的物理规律对应的方程称为该单位制 $\mathscr{Z}$ 中$\hat{\boldsymbol{C}}_{\mathscr{Z}}$ 的\textbf{定义方程}.
\end{definition}
\begin{definition}{}
在某一单位制 $\mathscr{Z}$ 中,若将导出量类 $\tilde{\boldsymbol{C}}$ 的单位 $\hat{\boldsymbol{C}}_{\mathscr{Z}}$ 的\textbf{定义方程} 中涉及的量类(该导出量类 $\tilde{\boldsymbol{C}}$ 除外)都用基本量类来表示,得到的定义方程称为该单位制 $\mathscr{Z}$ 中$\hat{\boldsymbol{C}}_{\mathscr{Z}}$ 的\textbf{终极定义方程},简称 \textbf{终定方程}.
\end{definition}

\begin{theorem}{}
任一单位制 $\mathscr{Z}$ 的任一导出单位 $\hat{\boldsymbol{C}}_{ \mathscr{Z}}$ 的终定方程都是幂单项式,即
\begin{equation}
C=k_{\text{终}}J_1^{\sigma_1}\cdots J_l^{\sigma_l}
\end{equation}
\end{theorem}
\textbf{证明:}记单位制为 $\mathcal{Z}$ ,为便于陈述,设 $\mathcal{Z}$ 制有3个基本量类——$\tilde{\boldsymbol{l}}$、$\tilde{\boldsymbol{m}}$和$\tilde{\boldsymbol{t}}$.选定基本单位$\hat{\boldsymbol{l}}$、$\hat{\boldsymbol{m}}$和$\hat{\boldsymbol{t}}$ .任一导出量类 $\tilde{\boldsymbol{C}}$ 的导出单位 $\hat{\boldsymbol{C}}$ 的终定方程为
\begin{equation}
C=f(l,m,t)
\end{equation}
设 $\mathcal{Z'}$ 是 $\mathcal{Z}$ 的同族制,其基本单位是$\hat{\boldsymbol{l}}'$、$\hat{\boldsymbol{m}}'$和$\hat{\boldsymbol{t}}'$,导出量类 $\tilde{\boldsymbol{C}}$ 的单位是  $\hat{\boldsymbol{C}}'$,又有
\begin{equation}\label{DIMF_eq1}
C'=f(l',m',t')
\end{equation}
$f$ 不加撇是因为同族制有相同的定义方程.令
\begin{equation}
x\equiv \eval{\mathrm{dim}}_{\mathcal{Z,Z'}}\tilde{\boldsymbol{l}},\quad y\equiv \eval{\mathrm{dim}}_{\mathcal{Z,Z'}}\tilde{\boldsymbol{m}},\quad z\equiv \eval{\mathrm{dim}}_{\mathcal{Z,Z'}}\tilde{\boldsymbol{t}}
\end{equation}
则由\autoref{USD_eq9}~\upref{USD},得
\begin{equation}
x=\frac{\hat{\boldsymbol{l}}}{\hat{\boldsymbol{l}}'}=\frac{l'}{l},\quad y=\frac{\hat{\boldsymbol{m}}}{\hat{\boldsymbol{m}}'}=\frac{m'}{m},\quad z=\frac{\hat{\boldsymbol{t}}}{\hat{\boldsymbol{t}}'}=\frac{t'}{t},\quad 
\end{equation}
故 $l'=xl,m'=ym,t'=zt$,代入\autoref{DIMF_eq1} 得
\begin{equation}
C'=f(xl,ym,zt)
\end{equation}
由量纲的定义又知
\begin{equation}
\mathrm{\eval{dim}_{\mathcal{Z,Z'
}}}\tilde{\boldsymbol{C}}=\frac{f(xl,ym,zt)}{f(l,m,t)}
\end{equation}


