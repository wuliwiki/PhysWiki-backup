% 前推
% license Usr
% type Tutor
赋予流形以联络,便可以比较同一流形上不同点的切向量“大小”。那对于不同流形,我们能否找到一个定义去联系同一点上的不同切空间呢?

\begin{issues}
\issueDraft 缺图
\end{issues}

\subsection{前推}
\begin{definition}{前推}
设$M,N$都是光滑流形。$f,g\in C^{\infty}(N)$。对任意$p\in M,\,F:M\rightarrow N$为光滑映射。定义$F_*:T_p M\rightarrow T_{F(p)N}$为
\begin{equation}
(F_*X)(f)=X(f\circ F)~,
\end{equation}
称之为与$F$关联的\textbf{前推(push-forward)}。
\end{definition}
可记忆前推是顺着光滑映射的方向对切向量进行转移,在$M$上$p$点的切空间与$N$上$F(p)$点的切空间建立关联。这样的定义是合理的,$F_*X$确实是一个切向量,满足导子的性质:
\begin{equation}
\begin{aligned}
(F_*X)(fg)&=X((fg)\circ F)\\
&=X(f\circ F)(g\circ F)\\
&=(g\circ F)X(f\circ F(p))+(f\circ F)X(g\circ F(p))\\
&=g(F)F_*X(f)+f(F)F_*X(g)~.
\end{aligned}
\end{equation}
易证前推具有如下性质:
\begin{lemma}{}
令$F:M\rightarrow N$与$G:N\rightarrow P$都是光滑映射。且有$p\in M$。那么我们有:
\begin{enumerate}
\item $F_*:T_P M\rightarrow T_{F(p)}N$是线性的。
\item $(G\circ F)_*=G_*\circ F_*:T_p M\rightarrow T_{G\circ F(p)}P$
\item $(Id_M)_*=Id_{T_p M}:T_p M\rightarrow T_p M$
\item 若$F$是微分同胚,那么前推:$F_*: T_p M\rightarrow T_{F(p)N}$是同构映射。
\end{enumerate}
\end{lemma}

\begin{exercise}{}
设$U\subset R^n$,基矢组为$\{e_i\}$,任意坐标表示为$(x^1,x^2...x^n)$。另有$V\subset R^m$,基矢组为$\{\theta_i\}$,对应坐标表示$(y^1,y^2...y^m)$。设$F:U\rightarrow V$是光滑映射。证明:
\begin{equation}
(F_*e_i)f=(\frac{\partial F^j}{\partial x^i}(p)\theta_j)f~.
\end{equation}

\end{exercise}
\subsection{前推的应用}
\subsubsection{流形上的切空间}
令$(U,\phi)$是$M$上的光滑坐标卡,也就是说$\phi$是双向光滑的微分同胚映射。因此,$\phi_*:T_p M\rightarrow T_{\phi(p)}R^n$是同构映射。那么我们可以利用前推来定义该坐标卡上每一点切空间的basis:
\begin{equation}
\frac{\partial}{\partial x^i}\bigg|_p=\phi^{-1}_* (\frac{\partial}{\partial x^i})\bigg|_{\phi(p)}~.
\end{equation}

设$\{\widetilde{e_i}\}$为$U$上$p$点的切空间基矢,对应$\phi (U)$上切空间的基矢$\{{e_i}\}$,并设$f\in C^{\infty}(U)$,那么我们有:
\begin{equation}
\begin{aligned}
\widetilde e_i f=\frac{\partial}{\partial x^i}\bigg|_p f&= (\frac{\partial}{\partial x^i})\bigg|_{\phi(p)}(f\circ \phi^{-1})\\
&=e_i\hat f(\hat p)~.
\end{aligned}
\end{equation}

可见,固定一个图,则流形上该点切空间的基矢定义与光滑函数的定义是自洽的。称$\hat f=f\circ \phi^{-1}$为光滑函数$f$的坐标表示,$\hat p=\phi (p)$为$p$点的坐标表示,一个函数在某个图光滑等价于在“表示”里光滑。

切空间是线性空间,因此可以通过过渡矩阵来进行坐标变换。那不同图上同一点的切空间是否也存在一个过渡矩阵,建立不同基矢组的联系?答案是肯定的。
具体来说,设$(U,\phi),(V,\psi)$是$M$上相容的图。$\{\widetilde e_i\}$与$\{\widetilde \theta_i\}$分别是$U$和$V$上切空间的基矢,且图上的坐标分别为$\phi(p)=(x^1,x^2...),\psi(p)=(y^1,y^2...y^n)$。则$\phi\circ\psi^{-1}:\psi(U\cap V)\rightarrow\phi(U\cap V)$是光滑的。又设\textbf{该转移映射的坐标表示}为
\begin{equation}
\phi\circ \psi^{-1}(y)=(x'^1,x'^2...x'^n)~,
\end{equation}
对于$p\in U\cap V$,我们有:$\widetilde\theta_i|_p=T^j_i \widetilde e_j|_p$,$T^j_i$是光滑函数构成的过渡矩阵。该假设是合理的,切向量作为道路等价类,无论在什么图里,作用在光滑函数上都会得到相同的结果,即$Xf=a^i\widetilde e_if=b^i\widetilde \theta_i f$。不同点的切空间没有自然同构,因而过渡矩阵元是流形上的光滑函数。

两边作用在$\phi(p)$上,利用$e_ix^k=\delta^k_i$,计算分量得:
\begin{equation}
\begin{aligned}
 T^j_i e_j x^k&=\theta_i (\phi^k\circ \psi^{-1}(y))\\
T^k_i&=\frac{\partial x'^{k}}{\partial y^i}~.
\end{aligned}
\end{equation}
从推导中我们也可以看到,实际上$e_j,x^k$都是包含转移函数的形式。
严谨的推导如下:
\begin{equation}
\begin{aligned}
\left.\frac{\partial}{\partial y^i}\right|_p & =\left.\left(\phi^{-1}\right)_* \frac{\partial}{\partial y^i}\right|_{\psi(p)} \\
& =\left.\left(\phi^{-1}\right)_*\left(\phi \circ \psi^{-1}\right)_* \frac{\partial}{\partial y^i}\right|_{\psi(p)} \\
& =\left.\left(\phi^{-1}\right)_* \frac{\partial x'^j}{\partial y^i}(\psi(p)) \frac{\partial}{\partial x'^j}\right|_{\phi(p)} \\
& =\left.\frac{\partial x'^j}{\partial y^i}(\psi(p))\left(\phi^{-1}\right)_* \frac{\partial}{\partial x'^j}\right|_{\phi(p)} \\
& =\left.\frac{\partial x'^j}{\partial y^i}(\widehat{p}) \frac{\partial}{\partial x'^j}\right|_p,
\end{aligned}~.
\end{equation}

于是,我们得到了不同图之间基矢的“联系”,类似多元函数微分学的结果,只不过雅克比矩阵的矩阵元为转移函数对变量的偏导数。回顾切向量的导子定义,我们可以得到同一切向量的坐标变换。若$X=a^i\widetilde e_i=b^i\widetilde  \theta_i$,则有:
\begin{equation}
b^i=T^i_ja^j~,
\end{equation}
其中$T^i_j=\frac{\partial y'^{i}}{\partial x^j}(\phi(p)),y'^i=\psi\circ\phi^{-1}$。
\subsubsection{曲线上的切向量}
设光滑曲线$\gamma(t):[0,1]\rightarrow M$,定义$\gamma(t_0)$处的切向量为导数的推前:
\begin{equation}
\gamma^{\prime}\left(t_0\right)=\left.\gamma_* \frac{d}{d t}\right|_{t_0}~.
\end{equation}
设$\gamma(t)=(\gamma^1(t),\gamma^2(t)...\gamma^n(t))=(x^1,x^2...x^n),f\in C^{\infty}(M)$。由推前的定义及链式法则,可知曲线上的切向量对光滑函数作用为:
\begin{equation}
\begin{aligned}
\gamma'(t_0)f&=\frac{d(f\circ\gamma)}{dt}\\
&=\gamma^i(t_0)'\frac{\partial f}{\partial x^i}\bigg|_{\gamma(t_0)}
\end{aligned}~,
\end{equation}
也就是说,$\gamma'(t_0)=\gamma^i(t_0)'\frac{\partial }{\partial x^i}\bigg|_{\gamma(t_0)}$,这确实是一个切向量。与欧几里得空间的意义相同,流形上曲线某点处的切向量等于沿着该点切向求方向导数。

广义上来看,“前推”可以概括顺着光滑映射方向的一切复合作用。例如,若$\gamma:[0,1]\rightarrow M$,则对于光滑映射$F:M\rightarrow N$,复合映射:$F\circ \gamma:[0,1]\rightarrow M\rightarrow N$可看作把$M$上的光滑曲线迁移到$N$上,在这个过程中,每一点的切向量也在前推。
\begin{lemma}{}
$F,\gamma,M,N$的定义如上。$N$上该复合曲线的切向量为:
\begin{equation}
(F\circ\gamma)'(t_0)=F_*\gamma(t_0)'~.
\end{equation}
\end{lemma}