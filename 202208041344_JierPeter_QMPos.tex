% 量子力学的基本假设
% 基本假设|哈密顿方程|量子力学

\begin{issues}
\issueDraft
\end{issues}

% 初学的话可以把这个假设换一个简单的版本, 比如说在位置表象中, 直接给出为什么 x, p 算符有这样的形式, 这样就不需要了解哈密顿方程, 只需要知道哈密顿量是能量就可以了.

\pentry{哈密顿正则方程\upref{HamCan}}

\subsection{一维的情况}
\begin{itemize}
\item 粒子的状态由 Hilbert 空间中的波函数 $\ket{\psi(t)}$ 表示.
\item 要得到一个物理量 $\omega(x, p)$ 对应的算符, 就把 $x$ 和 $p$ 换成对应的算符 $\Q x$ 和 $\Q p$. 这两个算符的定义为
\begin{equation}
\bra{x}\Q x \ket{x'} = x\delta(x-x')
\end{equation}
\begin{equation}
\bra{x}\Q p \ket{x'} = -\I\hbar \delta'(x-x')
\end{equation}
\item 找到哈密顿量对应的哈密顿算符, 波函数的演化由薛定谔方程决定
\begin{equation}
\Q H \ket{\psi(t)} = \I \hbar \dv{t} \ket{\psi(t)}
\end{equation}
\item 如果要测量一个物理量 $\Q\Omega(x, p)$, 那么先求出所有的归一化本征函数 $\ket{\omega_i}$ 和对应的本征值 $\omega_i$, 测量到 $\omega_i$ 的概率为 $\abs{\braket{\omega}{\psi}}^2$. 测量完之后波函数由 $\ket{\psi}$ 变为 $\ket{\omega_i}$.
\end{itemize}

注意这里的 $x$ 和 $p$ 分别是哈密顿方程中的广义坐标和广义动量, 而不必是位置和动量.

希尔伯特空间既包括可以正常归一化的波函数, 也包括能用狄拉克 $\delta$ 函数归一化的波函数.

如果经典哈密顿量中出现了 $xp$ 项, 那么算符要写成 $(\Q x \Q p + \Q p \Q x)/2$ 以保证物理量的算符是 Hermitian 矩阵. 如果出现了 $x$ 和 $p$ 的更高次项, 就只能靠实验判断.

本词条参考\cite{Shankar}.
