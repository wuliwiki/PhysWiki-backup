% 自然单位制
% keys 单位制|原子单位制|转换常数

\begin{issues}
\issueDraft
\end{issues}

\pentry{原子单位制\upref{AU}}

在高能物理和场论中, 我们往往使用一套单位制, 使得 $\hbar = 1$ 以及 $c = 1$, 下面我们对这种单位制进行说明.

在 “原子单位制\upref{AU}” 的第一节中, 我们已经确定了一系列常数 $\beta$, 使得薛定谔方程变成更简洁的形式, 但还留有两个自由度 $\beta_x$ 和 $\beta_m$. 我们现在使用 $c = 1$ 这个条件, 即规定速度的转换常数为 $\beta_v = c$. 如果我们希望满足 $x = vt$, 那么必须有 $\beta_x = \beta_v \beta _t$. 将\autoref{AU_eq6}~\upref{AU} 的 $\beta_t$ 代入得
\begin{equation}
\beta_x \beta_m = \frac{\hbar}{c}
\end{equation}
所以此时转换常数只剩一个自由度. 
