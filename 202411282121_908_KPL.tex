% 开普勒定律(综述)
% license CCBYSA3
% type Wiki

本文根据 CC-BY-SA 协议转载翻译自维基百科\href{https://en.wikipedia.org/wiki/Kepler\%27s_laws_of_planetary_motion}{相关文章}。

\begin{figure}[ht]
\centering
\includegraphics[width=8cm]{./figures/e5bb3dbe32521ea0.png}
\caption{用两个行星轨道说明开普勒定律。这些轨道是椭圆形的,行星1的焦点为F1和F2,行星2的焦点为F1和F3。太阳位于F1。阴影区域A1和A2是相等的,并且是由行星1的轨道在相等的时间内扫过的。行星1的轨道周期与行星2的轨道周期的比值为 \(\left(\frac{a_1}{a_2}\right)^{3/2}\)。} \label{fig_KPL_1}
\end{figure}
在天文学中,开普勒的行星运动定律由约翰内斯·开普勒于1609年发布(除了第三定律,后者于1619年完全发布),描述了行星围绕太阳的轨道。这些定律用椭圆轨道代替了哥白尼日心说中的圆形轨道和本轮,并解释了行星速度的变化。这三条定律如下:
\begin{enumerate}
\item 行星的轨道是椭圆,太阳位于其中一个焦点上。
\item 连接行星和太阳的线段在相等的时间间隔内扫过相等的面积。
\item 行星轨道周期的平方与其轨道半长轴的立方成正比。
\end{enumerate}
行星的椭圆轨道通过火星轨道的计算得到了证明。从这些计算中,开普勒推断出太阳系中其他天体(包括距离太阳较远的天体)也具有椭圆轨道。第二定律确定了当行星靠近太阳时,其运动速度较快。第三定律表达了行星距离太阳越远,其轨道周期越长。

艾萨克·牛顿在1687年证明,像开普勒定律这样的关系,作为他自己运动定律和万有引力定律的结果,也适用于太阳系。

更精确的历史方法可以在《新天文学》和《哥白尼天文学概要》中找到。
\subsection{与哥白尼的比较 } 
约翰内斯·开普勒的定律改进了哥白尼的模型。根据哥白尼的观点:
\begin{enumerate}
\item 行星轨道是一个带有本轮的圆形。
\item 太阳大致位于轨道的中心。
\item 行星在主轨道中的速度是恒定的。
\end{enumerate}
尽管哥白尼正确地指出了行星绕太阳转,但他在定义行星轨道时是错误的。开普勒通过引入超越几何的物理解释,正确地定义了行星轨道,如下所示:
\begin{enumerate}
\item 行星轨道不是带有本轮的圆形,而是椭圆形。
\item 太阳不在轨道的中心,而是在椭圆轨道的一个焦点上。
\item 行星在轨道中的线速度和角速度都不是恒定的,但面积速度(与角动量的概念密切相关)是恒定的。
\end{enumerate}
地球轨道的离心率使得从3月春分到9月秋分的时间约为186天,不等于从9月秋分到3月春分的时间约为179天。如果用一条直径将轨道分成两部分,轨道将被平分,但通过太阳的平面与地球赤道平行的平面将轨道分成两部分,面积比约为186比179。因此,地球轨道的离心率约为:
\[
e \approx \frac{\pi}{4} \cdot \frac{186-179}{186+179} \approx 0.015~
\]
这个值接近正确值(0.016710218)。这个计算的准确性要求选择的两个日期位于椭圆轨道的短轴上,并且每一半的中点位于长轴上。由于这里选择的日期是春分和秋分,这个计算在近日点(地球最接近太阳的日期)恰好落在冬至时是正确的。目前的近日点,接近1月4日,比较接近12月21日或22日的冬至。
\subsection{命名法}  
开普勒工作的当前表述形式花了近两个世纪才定型。伏尔泰的《牛顿哲学要义》(*Eléments de la philosophie de Newton*,1738年)是第一部使用“定律”术语的出版物。[6][7] 《天文学家传记百科全书》在关于开普勒的条目中(第620页)指出,至少从约瑟夫·德·拉兰德(Joseph de Lalande)时代起,科学定律的术语就已被使用。[8] 是罗伯特·斯莫尔(Robert Small)在其1814年出版的《开普勒的天文发现概述》(*An Account of the Astronomical Discoveries of Kepler*)中,将这三条定律定型,加入了第三条。[9] 斯莫尔还声称(与历史记载不符),这些定律是基于归纳推理的经验定律。[7][10]

此外,目前使用的“开普勒第二定律”其实是一个误称。开普勒有两个版本的定律,在定性上有所关联:“距离定律”和“面积定律”。“面积定律”最终成为三条定律中的第二条;但开普勒自己并没有以这种方式优先考虑它。[11]
\subsection{历史}  
开普勒于1609年发布了他的前两条行星运动定律,[12] 这些定律是通过分析第谷·布拉赫的天文观测数据发现的。[13][14][15][5]: 53 开普勒的第三定律则在1619年发布。[16][14] 开普勒信仰哥白尼的太阳系模型,该模型要求行星沿圆形轨道运行,但他无法将布拉赫非常精确的观测数据与火星轨道的圆形拟合起来——火星的离心率恰好是除水星之外所有行星中最高的。[17] 他的第一定律反映了这一发现。

在1621年,开普勒注意到他的第三定律也适用于木星的四颗最亮的卫星。[Nb 1] 戈德弗罗伊·温德林(Godefroy Wendelin)也在1643年做出了这一观察。[Nb 2] 第二定律(以“面积定律”形式)在1664年由尼古拉斯·梅尔卡托尔(Nicolaus Mercator)在一本书中提出异议,但到了1670年,他的《哲学汇刊》已支持这一理论。[18][19] 随着世纪的推进,这一理论变得更加广泛接受。[20] 德国的接受情况在1688年(牛顿的《原理》出版并被视为基本的哥白尼主义作品)和1690年(戈特弗里德·莱布尼茨关于开普勒工作的成果出版)之间发生了显著变化。[21]

牛顿被认为理解了第二定律并非仅适用于万有引力的平方反比定律,而是仅作为该定律径向性质的结果,而其他定律则依赖于吸引力的平方反比形式。卡尔·龙格(Carl Runge)和威廉·伦茨(Wilhelm Lenz)在很晚时候在行星运动的相空间中识别出一个对称性原理(正交群 O(4) 的作用),该原理解释了牛顿引力下的第一和第三定律,类似于角动量的守恒通过旋转对称性解释第二定律。[22]
\subsection{公式} 
行星在开普勒定律作用下的运动学数学模型允许进行大量进一步的计算。  
\subsubsection{第一定律}  
开普勒的第一定律指出:

每颗行星的轨道是椭圆形的,太阳位于椭圆的两个焦点之一。  
\begin{figure}[ht]
\centering
\includegraphics[width=8cm]{./figures/16318e9d2bc7251f.png}
\caption{开普勒的第一定律将太阳置于椭圆轨道的一个焦点上。} \label{fig_KPL_2}
\end{figure}
一个椭圆可以用以下公式表示:
\[
r = \frac{p}{1 + \varepsilon \cos \theta}~
\]
其中:\( p \) 是半准线,\( \varepsilon \) 是椭圆的离心率,\( r \) 是从太阳到行星的距离,\( \theta \) 是从太阳看行星当前位置到其最近点的角度。所以 \( (r, \theta) \) 是极坐标。

对于椭圆,\( 0 < \varepsilon < 1 \);在极限情况下,当 \( \varepsilon = 0 \) 时,轨道是一个圆形,太阳位于中心(即零离心率的情况)。

当 \( \theta = 0^\circ \) 时,近日点,距离最小:
\[
r_{\min} = \frac{p}{1 + \varepsilon}~
\]
当 \( \theta = 90^\circ \) 和 \( \theta = 270^\circ \) 时,距离等于 \( p \)。
当 \( \theta = 180^\circ \) 时,远日点,距离最大(根据定义,远日点永远是近日点加180°):
\[
r_{\max} = \frac{p}{1 - \varepsilon}~
\]
半长轴 \( a \) 是 \( r_{\min} \) 和 \( r_{\max} \) 的算术平均值:
\[
a = \frac{r_{\max} + r_{\min}}{2}~
\]
\[
a = \frac{p}{1 - \varepsilon^2}~
\]
半短轴 \( b \) 是 \( r_{\min} \) 和 \( r_{\max} \) 的几何平均值:
\[
b = \sqrt{r_{\max} r_{\min}}~
\]
\[
b = \frac{p}{\sqrt{1 - \varepsilon^2}}~
\]
半准线 \( p \) 是 \( r_{\min} \) 和 \( r_{\max} \) 的调和平均值:
\[
p = \left( \frac{r_{\max}^{-1} + r_{\min}^{-1}}{2} \right)^{-1}~
\]
\[p = r_{\max} r_{\min} = b^2~\]
离心率 \( \varepsilon \) 是 \( r_{\min} \) 和 \( r_{\max} \) 之间的变异系数:
\[
\varepsilon = \frac{r_{\max} - r_{\min}}{r_{\max} + r_{\min}}~
\]
椭圆的面积是:
\[
A = \pi ab~
\]
圆的特殊情况是 \( \varepsilon = 0 \),此时 \( r = p = r_{\min} = r_{\max} = a = b \),并且 \( A = \pi r^2 \)。
\begin{figure}[ht]
\centering
\includegraphics[width=8cm]{./figures/f0817979388b8adc.png}
\caption{\textbf{椭圆的日心坐标系统} (r, θ)图中还展示了:半长轴 \(a\)、半短轴 \(b\) 和半准线 \(p\);椭圆的中心及其两个焦点以大圆点标记。对于 \( \theta = 0^\circ \),\( r = r_{\min} \),而对于 \( \theta = 180^\circ \),\( r = r_{\max} \)。} \label{fig_KPL_3}
\end{figure}
\subsubsection{第二定律}  
开普勒的第二定律指出:

连接行星和太阳的直线在相等时间间隔内扫过相等的面积。[23]
\begin{figure}[ht]
\centering
\includegraphics[width=8cm]{./figures/937e0d665a599152.png}
\caption{相同的(蓝色)区域在固定的时间内被扫过。绿色箭头表示速度。指向太阳的紫色箭头表示加速度。其他两个紫色箭头是与速度平行和垂直的加速度分量。} \label{fig_KPL_4}
\end{figure}
行星在椭圆轨道上的轨道半径和角速度会发生变化。这在动画中有所展示:当行星靠近太阳时,它的速度较快;而当行星远离太阳时,速度较慢。开普勒的第二定律指出,蓝色扇形区域的面积是恒定的。

\textbf{历史与证明}

开普勒显著地通过一些假设得出了这一定律,这些假设要么只是近似正确,要么是完全错误的,可以概括如下:
\begin{enumerate}
\item 行星是通过太阳的某种力被推动绕太阳运行的。这个错误的假设依赖于错误的亚里士多德物理学,即物体需要不断被推动才能维持运动。 
\item 太阳的推动力与太阳的距离成反比。开普勒基于这种推理,认为引力应该是三维空间扩展的,而不是平面上的浪费。因此,他提出了一个反比定律,而不是[正确的]平方反比定律。 
\item 由于开普勒认为力与速度成正比,因此从第1条和第2条假设中可以推导出,速度与距离成反比。这也是亚里士多德物理学中的错误观念。
\item 由于速度与时间成反比,因此从太阳的距离将与绕轨道小段的时间成正比。这对于椭圆轨道大致是正确的。
\item 扫过的面积与整体时间成正比。这也大致是正确的。
\item 行星的轨道是圆形的(开普勒在提出第二定律之前就发现了这一点,这与此相矛盾)。
\end{enumerate}
然而,第二定律的结果是完全正确的,因为它在逻辑上等价于角动量的守恒,这对于任何受到径向对称力作用的物体都成立。[24] 可以通过以下证明来展示这一点。由于两个向量的叉积给出了这两个向量构成的平行四边形的面积,因此在短时间内扫过的三角形面积 \( dA \) 可以通过 \( \vec{r} \) 和 \( \vec{dx} \) 向量的叉积的一半来表示,其中 \( dx \) 是轨道的小段。
\[
dA = \frac{1}{2} (\vec{r} \times \vec{dx}) = \frac{1}{2} (\vec{r} \times \vec{v} dt)~
\]
其中 \( dx \) 是轨道的小段,\( dt \) 是覆盖该段所需的时间。

因此,
\[
\frac{dA}{dt} = \frac{1}{2} (\vec{r} \times \vec{v})~
\]
\[
\frac{dA}{dt} = \frac{1}{m} \frac{1}{2} (\vec{r} \times \vec{p})~
\]
由于最终的表达式与总角动量 \( (\vec{r} \times \vec{p}) \) 成正比,开普勒的等面积定律将适用于任何守恒角动量的系统。由于任何径向力都不会对行星的运动产生力矩,角动量将得到守恒。

\textbf{关于椭圆参数}

在一个很短的时间 \( dt \) 内,行星扫过一个小三角形,底边为 \( r \),高为 \( r \, d\theta \),其面积为:
\[
dA = \frac{1}{2} \cdot r \cdot r \, d\theta~
\]
因此,恒定的面积速度为:
\[
\frac{dA}{dt} = \frac{r^{2}}{2} \frac{d\theta}{dt}~
\]
椭圆轨道所围成的面积为 \( \pi ab \),因此周期 \( T \) 满足:
\[
T \cdot \frac{r^{2}}{2} \frac{d\theta}{dt} = \pi ab~
\]
行星围绕太阳的平均运动 \( n \) 满足:
\[
n = \frac{2\pi}{T}~
\]
于是有:
\[
r^{2} d\theta = abn \, dt~
\]
因此:
\[
\frac{dA}{dt} = \frac{abn}{2} = \frac{\pi ab}{T}~
\]
\begin{figure}[ht]
\centering
\includegraphics[width=10cm]{./figures/181e72f55483a4d6.png}
\caption{\textbf{红色射线以恒定的角速度旋转,并且与行星具有相同的轨道周期,}\(T = 1\)S:太阳在主焦点,C:椭圆的中心,S':次焦点。在每种情况下,所描绘的所有扇形的面积都是相同的。} \label{fig_KPL_5}
\end{figure}
\textbf{第三定律}

开普勒的第三定律表述为:

\textbf{任何绕同一主星运行的物体,其轨道周期的平方与轨道半长轴的立方之比是相同的。}

这一规律揭示了行星与太阳的距离以及它们的轨道周期之间的关系。

开普勒在1619年表述了这一第三定律,他艰难地尝试通过精确的规律确定他所认为的“天体的音乐”,并以音乐符号的形式表达出来[25]。因此,它也被称为**谐波定律**[26]。这一定律的最初形式(提到的不是半长轴,而是“平均距离”)仅适用于偏心率接近零的行星[27]。

利用牛顿的引力定律(1687年发布),在圆形轨道的情况下,我们可以通过将向心力与引力相等来得到这一关系:
\[
mr\omega^2 = G\frac{mM}{r^2}~
\]
接着,将角速度 \(\omega\) 表达为轨道周期 \(T\) 的形式,并进行重排,最终得到开普勒的第三定律:
\[
mr\left(\frac{2\pi}{T}\right)^2 = G\frac{mM}{r^2} \implies T^2 = \left(\frac{4\pi^2}{GM}\right)r^3 \implies T^2 \propto r^3~
\]
通过更详细的推导,可以得到一般椭圆轨道的情况,而不仅仅是圆形轨道,并且考虑绕质心(而不仅仅是大质量物体)运动。这将导致将圆半径 \(r\) 替换为椭圆运动中的半长轴 \(a\),同时将大质量 \(M\) 替换为 \(M+m\)。然而,由于行星的质量远小于太阳质量,这个修正通常被忽略。完整的公式是:
\[
\frac{a^3}{T^2} = \frac{G(M+m)}{4\pi^2} \approx \frac{GM}{4\pi^2} \approx 7.496 \times 10^{-6} \frac{\text{AU}^3}{\text{days}^2} \text{是常数}~
\]
其中,\(M\) 是太阳的质量,\(m\) 是行星的质量,\(G\) 是引力常数,\(T\) 是轨道周期,\(a\) 是椭圆的半长轴,AU 是天文单位,即地球与太阳的平均距离。

\textbf{表格}  

下表显示了开普勒用来经验性地推导他的定律的数据:
\begin{table}[ht]
\centering
\caption{开普勒使用的数据(1618年)}\label{KPL}
\begin{tabular}{|c|c|c|c|}
\hline \textbf{行星} & \textbf{到太阳的平均距离(AU)} & \textbf{周期(天)} & \textbf{\({\frac {R^{3}}{T^{2}}}\) (10⁻⁶ AU³/天²)}\\
\hline 水星	 & 0.389 & 87.77 & 7.64 \\
\hline 金星	 & 0.724 & 224.70 & 7.52 \\
\hline 地球	 & 1 & 365.25 & 7.50 \\
\hline 火星	 & 1.524 & 686.95 & 7.50 \\
\hline 木星	 & 5.20 & 4332.62 & 7.49 \\
\hline 土星	 & 9.510 & 10759.2 & 7.43  \\
\hline 
\end{tabular}
\end{table}
开普勒在发现这个规律之前,意识到约翰·纳皮尔最近发明了对数和对数-对数图。[28]

在发现这个规律后,开普勒写道:[29]

我一开始以为我在做梦……但可以完全肯定并准确地说,任何两颗行星之间的周期时间比,正是它们平均距离的3/2次方的比值。

—— 摘自《世界和谐》中的开普勒(1619年)

作为比较,以下是现代的估计值:
\begin{table}[ht]
\centering
\caption{现代数据}\label{KPL1}
\begin{tabular}{|c|c|c|c|}
\hline \textbf{行星} & \textbf{半长轴(AU)} & \textbf{周期(天)} &\textbf{ \({\frac {a^{3}}{T^{2}}}\) (10⁻⁶ AU³/天²)  }\\
\hline 水星 & 0.38710 & 87.9693 & 7.496 \\  
\hline 金星 & 0.72333 & 224.7008 & 7.496 \\ 
\hline 地球 & 1 & 365.2564 & 7.496 \\ 
\hline 火星 & 1.52366 & 686.9796 & 7.495 \\ 
\hline 木星 & 5.20336 & 4332.8201 & 7.504 \\
\hline 土星 & 9.53707 & 10775.599 & 7.498 \\ 
\hline 天王星 & 19.1913 & 30687.153 & 7.506 \\ 
\hline 海王星 & 30.0690 & 60190.03 & 7.504 \\
\hline 
\end{tabular}
\end{table}
\begin{figure}[ht]
\centering
\includegraphics[width=10cm]{./figures/1625ee3cb9994c76.png}
\caption{周期T与半长轴a(近日点和远日点的平均值)的对数-对数图(交叉点表示开普勒的数值),显示出a³/T²是恒定的(绿色线)} \label{fig_KPL_6}
\end{figure}
\subsection{行星加速度}  
艾萨克·牛顿在《自然哲学的数学原理》中计算了根据开普勒的第一定律和第二定律运动的行星的加速度。
\begin{enumerate}
\item 加速度的方向指向太阳。  
\item 加速度的大小与行星到太阳的距离的平方成反比(平方反比定律)。
\end{enumerate}
这意味着太阳可能是行星加速度的物理原因。然而,牛顿在《原理》中指出,他从数学角度看待力,而不是从物理角度看,因此采取了一种工具主义的观点。[30] 此外,他没有给引力赋予一个具体的原因。[31]

牛顿定义作用在行星上的力为其质量与加速度的乘积(参见牛顿运动定律)。因此:
\begin{enumerate}
\item 每颗行星都被太阳吸引。  
\item 作用在行星上的力与行星的质量成正比,与其距离太阳的平方成反比。
\end{enumerate}
太阳在其中扮演着不对称的角色,这是没有依据的。因此,他在牛顿万有引力定律中假设:
\begin{enumerate}
\item 太阳系中的所有天体互相吸引。  
\item 两天体之间的力与它们的质量乘积成正比,与它们之间的距离的平方成反比。
\end{enumerate}  
由于行星的质量相对于太阳较小,行星的轨道大致符合开普勒定律。牛顿的模型改进了开普勒的模型,并且更准确地符合实际观测结果。(参见二体问题)

以下是根据开普勒的第一定律和第二定律运动的行星加速度的详细计算。
\subsubsection{加速度向量 }  
从日心坐标系的角度考虑,指向行星的向量为  
\(\mathbf{r} = r \hat{\mathbf{r}}\),其中  \(r\) 是到行星的距离,  \(\hat{\mathbf{r}}\) 是指向行星的单位向量。  
\[
\frac{d \hat{\mathbf{r}}}{dt} = \dot{\hat{\mathbf{r}}} = \dot{\theta} \hat{\boldsymbol{\theta}}, \quad \frac{d \hat{\boldsymbol{\theta}}}{dt} = \dot{\hat{\boldsymbol{\theta}}} = -\dot{\theta} \hat{\mathbf{r}}~
\]
其中,\(\hat{\boldsymbol{\theta}}\) 是与\(\hat{\mathbf{r}}\)的方向相反的单位向量,偏离90度,\(\theta\) 是极角,且变量上方的点表示对时间的微分。

对位置向量求两次导数,得到速度向量和加速度向量:
\[
\dot{\mathbf{r}} = \dot{r} \hat{\mathbf{r}} + r \dot{\hat{\mathbf{r}}} = \dot{r} \hat{\mathbf{r}} + r \dot{\theta} \hat{\boldsymbol{\theta}},~
\]                      
\[
\ddot{\mathbf{r}} = \left(\ddot{r} \hat{\mathbf{r}} + \dot{r} \dot{\hat{\mathbf{r}}}\right) + \left(\dot{r} \dot{\theta} \hat{\boldsymbol{\theta}} + r \ddot{\theta} \hat{\boldsymbol{\theta}} + r \dot{\theta} \dot{\hat{\boldsymbol{\theta}}}\right) = \left(\ddot{r} - r \dot{\theta}^2\right) \hat{\mathbf{r}} + \left(r \ddot{\theta} + 2 \dot{r} \dot{\theta}\right) \hat{\boldsymbol{\theta}}.~
\]
因此,  
\[
\ddot{\mathbf{r}} = a_r \hat{\boldsymbol{r}} + a_{\theta} \hat{\boldsymbol{\theta}},~
\]
其中,径向加速度为  
\[
a_r = \ddot{r} - r \dot{\theta}^2,~
\]
而横向加速度为  
\[
a_{\theta} = r \ddot{\theta} + 2 \dot{r} \dot{\theta}.~
\]























 
 
