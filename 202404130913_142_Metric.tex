% 度量空间
% keys 度量空间|欧几里得空间|距离函数
% license Xiao
% type Tutor

\pentry{集合\nref{nod_Set}}{nod_aa50}

度量空间是除拓扑空间\upref{Topol}外最广义的空间。 它在集合的基础上增加了距离或长度的概念。
\begin{figure}[ht]
\centering
\includegraphics[width=5cm]{./figures/827addbd920e62bc.pdf}
\caption{用维恩图表示几种不同空间之间的关系, 从内到外分别是内积空间\upref{InerPd}, 赋范空间\upref{NormV}, 度量空间, 拓扑空间\upref{Topol}(修改自维基百科)} \label{fig_Metric_2}
\end{figure}

\begin{definition}{度量空间}\label{def_Metric_2}
一个集合 $X$ 中任意两个元素 $u, v$ 间若定义了满足以下条件的\textbf{距离函数(distance function)} $d: X \times X \to \mathbb{R}$, 那它就是一个\textbf{度量空间(metric space)}。 集合中的每个元素就叫空间中的一个\textbf{点}。
\begin{itemize}
\item 正定性:$d(u, v) \geq 0$,$d(u, v)=0$ 当且仅当 $u=v$
\item 对称性:$d(u, v) = d(v, u)$
\item 三角不等式:$d(u, v) \leqslant d(u, w) + d(w, v)$
\end{itemize}
\end{definition}
\addTODO{笛卡尔积的用法}
\addTODO{距离有$[0, +\infty)$和$[0, \infty]$两种定义,在拓扑等价的意义下可以互相转换。}

其中 “三角不等式” 就是通常所说的 “三角形两边之和不小于第三边”, 移向后就变为 “两边之差不大于第三边”。

\begin{definition}{度量的等价性}
空间 $X$ 上的两个度量 $d_1$ 和 $d_2$ 称为\textbf{强等价的(strongly equivalent)}, 如果有正实数 $C_1, C_2 \in \mathbb{R}_+$ 使得如下不等式对于任何 $x, y \in X$ 都成立:
$$
C_1 d_1(x, y) \leq d_2(x, y) \leq C_2 d_1(x, y)~.
$$
另一方面,如果空间上的两个度量导出相同的拓扑的话,我们就称这两个度量是\textbf{拓扑等价的(topologically equivalent)}
\end{definition}

\begin{theorem}{强等价强于拓扑等价}

\end{theorem}
\addTODO{证明}


\begin{example}{平凡度量}
\addTODO{0,1 版本 和 0, +infty 版本}
\addTODO{拓扑等价:都是离散拓扑}
\end{example}


\begin{example}{欧几里得空间}\label{ex_Metric_1}
$N$ 维欧几里得空间 $\mathbb R^N$ 中通常定义距离函数为
\begin{equation}\label{eq_Metric_1}
d(x, y) = \sqrt{\sum_{i=1}^N (x_i - y_i)^2}~,
\end{equation}
那么它是一个度量空间(证明留做习题)。

特殊地, 实数域 $\mathbb R$ 通常的距离函数为 $d(x, y) = \abs{x - y}$。
\end{example}

日常生活中, 我们关于距离的直观概念都是建立在\autoref{ex_Metric_1} 的基础上的, 但度量空间是非常广义和抽象\upref{Abstra}的。 例如上例中 $d(x, y) = \abs{x^3 - y^3}$ 也可以是 $\mathbb R$ 的距离函数; 又例如我们可以把一些函数的集合看成一个度量空间:
\begin{example}{}\label{ex_Metric_2}
所有 $f:\mathbb R \to \mathbb R$ 函数\upref{functi}的集合是一个度量空间, 如果定义距离函数为
\begin{equation}
d(f, g) = \sup_x{\abs{f(x) - g(x)}}~.
\end{equation}
\addTODO{证明} % Giacomo:我不确定这个是正确的

\end{example}

\begin{corollary}{}
度量空间 $X$ 的子集 $A$ 若继承 $X$ 的距离函数, 那么 $A$ 也是一个度量空间, 称为 $X$ 的\textbf{子空间(subspace)}。
\end{corollary}
证明显然。

\subsection{对比线性空间}
虽然\autoref{ex_Metric_1} 和\autoref{ex_Metric_2} 的集合也可以用于定义线性空间, 但二者却有较大区别: 比起线性空间\upref{LSpace}, 度量空间有距离或长度的概念而线性空间却不一定(见范数\upref{NormV})。 线性空间必须定义 “加法” 和 “标量积” 两种运算而度量空间不必。 线性空间的 “零元” 在度量空间也不必存在。
