% 绳结的受力分析
% keys 绳结|力的分解|夹角|受力平衡

\pentry{力的分解与合成\upref{Fdecom}}

作为力的分解与合成的一个应用, 考虑 $N$ 根质量和粗细不计的绳子的末端连接到一个质量不计的绳结, 假设每根绳子拉绳结的力为 $\bvec F_i$, 那么绳结受力平衡的充分必要条件是合力为零
\begin{equation}
\sum_{i=1}^N \bvec F_i = \bvec 0
\end{equation}
显然, 当 $N = 2$ 时, 两个拉力必须等大反向。

\begin{example}{三点拉绳}
从给定的 $P_1,P_2,P_3$ 三点以固定大小的力 $F_1, F_2, F_3$ 拉一个绳结, 求绳结平衡的条件以及平衡位置。
\begin{figure}[ht]
\centering
\includegraphics[width=6.5cm]{./figures/Knot_1.pdf}
\caption{三点拉绳} \label{Knot_fig1}
\end{figure}

解: 首先绳结的平衡位置必定在三角形 $P_1 P_2 P_3$ 内, 否则仅分析三个力的方向就不可能平衡。 令绳结到三点的单位矢量分别为 $\uvec r_1, \uvec r_2, \uvec r_3$, 那么
\begin{equation}
F_1 \uvec r_1 + F_2 \uvec r_2 + F_3 \uvec r_3 = \bvec 0
\end{equation}
根据该关系以及矢量相加的平行四边形法则\upref{GVecOp}, 容易求出 $\uvec r_1, \uvec r_2, \uvec r_3$ 两两之间的夹角, 记为 $\theta_{12}, \theta_{23}, \theta_{13}$。 注意这三个角和 $P_1, P_2, P_3$ 的位置没有关系, 完全由三个力的大小唯一确定。 注意 $F_1, F_2, F_3$ 必须满足任意两个之和大于第三个才可能有解, 即三角不等式。

令线段 $P_1P_2$ 的长度为 $l_{12}$, 为了保证 $\theta_{12}$ 为定值, 过 $P_1, P_2$ 作一条弧线, 半径为(\autoref{SphTri_eq1}~\upref{SphTri})
\begin{equation}
R = \frac{l_{12}}{\sin\theta_{12}}
\end{equation}
那么绳结必定落在弧线上。 同理, 过 $P_2, P_3$ 再做一条弧线满足 $R = l_{23}/\sin\theta_{23}$, 两条弧线的交点若在三角形内就是平衡点, 否则不存在平衡点。
\end{example}

另一个例题见\autoref{Spring_ex1}~\upref{Spring}。
