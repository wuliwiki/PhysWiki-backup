% 泛函与线性泛函
% keys 泛函|线性泛函
% license Usr
% type Tutor

\pentry{向量空间\nref{nod_LSpace}}{nod_db0a}
\cite{Ke1}泛函是线性(向量)空间中的数值函数。泛函的英文单词为“functional”,后缀“-al”在这里表示“属于,像,相关的”\footnote{见https://www.etymonline.com/cn/word/-al}。因此,“functional” 就指代与函数相关的对象。而“泛”在这里的中文的意思是“泛指”的意思,即比“函数”更广的函数。事实上,“泛函”完美反映了泛函本身的定义。这个“更广”广在泛函的定义空间不再特指数构成的线性空间——数域,而是更一般的线性空间。

从泛函的定义,就能够避免现存的大多数误解:即把泛函理解作函数的函数。事实上这一说法不光理解错误,表述也是错误的。稍微正确一点的表述是函数(线性)空间的函数,然而这只是泛函的特殊情形。真正正确的理解是线性空间中的数值函数。

本节将给出泛函的具体定义,并将注意力集中在更重要的线性泛函的情形。

\subsection{泛函和线性泛函}
\begin{definition}{泛函}
设 $L$ 是定义在数域 $\mathbb F$ 上的线性空间,则称 $f:L\rightarrow\mathbb F$ 是 $L$ 上的\textbf{泛函}(functional)。
\end{definition}

\begin{example}{函数是泛函}
当 $L=\mathbb F$ 时,$L$ 上的泛函就是我们熟知的\textbf{函数}(function)。
\end{example}

\begin{definition}{可加,齐次,共轭齐次}
设 $f$ 是线性空间 $(L,\mathbb F)$ 上的
\end{definition}











