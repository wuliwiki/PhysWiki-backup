% 高斯散度定理(综述)
% license CCBYSA3
% type Wiki

本文根据 CC-BY-SA 协议转载翻译自维基百科\href{https://en.wikipedia.org/wiki/Divergence_theorem}{相关文章}。

在向量分析中,散度定理,又称高斯定理或奥斯特罗格拉德斯基定理\(^\text{[1]}\),是一条将向量场通过闭合曲面的通量与该曲面所包围体积内场的散度联系起来的定理。

更准确地说,散度定理表明:一个向量场在闭合曲面上的曲面积分(即该曲面的“通量”)等于该曲面所包围区域内散度的体积分。直观地理解,这意味着“一个区域内所有场源的总和(将汇点视为负源)等于该区域向外的净通量”。

散度定理在物理和工程数学中有着重要地位,尤其是在静电学和流体力学领域。在这些领域中,它通常应用于三维情形。然而,该定理可以推广到任意维度。在一维情形下,它等价于微积分基本定理;在二维情形下,它等价于格林定理。
\subsection{用液体流动来解释}
向量场常通过流体速度场(例如气体或液体)来进行说明。运动中的液体在每一点都有一个速度——即速度大小和方向,这个速度可以用一个向量表示,因此液体在某一时刻的速度分布可以看作一个向量场。假设在流体内部放置一个想象的闭合曲面 $S$,它包围了一个体积。该体积内液体的流出通量等于穿过该曲面的体积流率,即速度在该曲面上的曲面积分。

如果流体是不可压缩的,那么闭合体积内的液体总量保持不变;如果体积内没有源或汇,穿过 $S$ 的净通量就是零。即使液体在运动,它可能在曲面 $S$ 的某些位置流入体积,在另一些位置流出体积,但流入和流出的总量始终相等,因此净通量为零。

然而,如果在闭合曲面内存在一个液体源,例如一根向内部注入液体的管道,那么新增的液体会对周围液体施加压力,促使液体向各个方向流出。这会导致通过曲面 $S$ 的净外流,且该外流通量等于通过管道注入的体积流率。同样,如果在曲面 $S$ 内有一个汇或排水口,例如一根排出液体的管道,那么液体在外界压力的作用下会整体向汇点方向流动。穿过曲面 $S$ 的净内流通量等于通过该排水口排出的液体体积流率。

如果曲面 $S$ 内有多个液体的源和汇,通过该曲面的通量可以通过将所有源注入的体积流率相加,再减去所有汇排出的体积流率来计算。液体通过某个源或汇的体积流率(汇的流率取负号)等于该点速度场的散度。因此,将整个体积内的散度积分(即体积分)起来,结果就等于通过曲面 $S$ 的净通量。这就是散度定理\(^\text{[2]}\)。

散度定理应用于任何形式的守恒定律中,这类定律表明:所有源与汇的总体积(即散度的体积分)等于该体积边界上的净流量\(^\text{[3]}\)。
\subsection{数学表述}
\begin{figure}[ht]
\centering
\includegraphics[width=8cm]{./figures/62c835fec923f18e.png}
\caption{} \label{fig_GSsd_1}
\end{figure}
假设 $V$ 是 $\mathbb{R}^n$ 的一个子集(当 $n = 3$ 时,$V$ 表示三维空间中的一个体积),它是紧致的并且具有分段光滑的边界 $S$(也记作 $\partial V = S$)。如果 $\mathbf{F}$ 是定义在 $V$ 邻域上的连续可微向量场,则有\(^\text{[4][5]}\):
$$
\iiint_V (\nabla \cdot \mathbf{F})\,\mathrm{d}V 
= 
\oiint_S (\mathbf{F} \cdot \mathbf{\hat{n}})\,\mathrm{d}S.~
$$
左边是对体积 $V$ 的体积分,右边是对体积 $V$ 边界的曲面积分。闭集 $\partial V$ 由指向外的法向量来定向,$\mathbf{\hat{n}}$ 是边界 $\partial V$ 上几乎每一点的外指单位法向量。($\mathrm{d}\mathbf{S}$ 可作为 $\mathbf{n}\,\mathrm{d}S$ 的简写。)从直观描述的角度来看,方程的左边表示体积 $V$ 内所有“源”的总量,而右边表示穿过边界 $S$ 的总通量。
