% 矩阵的本征问题
% keys 矩阵|本征问题|本征矢|本征值|线性方程组|对角线
% license Xiao
% type Tutor

\pentry{线性方程组与向量空间\nref{nod_LinEq},矩阵的本征值\nref{nod_eigenM}}{nod_f021}

若已知矩阵 $\mat A$, 我们把线性方程组
\begin{equation}\label{eq_MatEig_1}
\mat A \bvec v = \lambda \bvec v~.
\end{equation}
称为矩阵 $\mat A$ 的\textbf{本征方程(eigen equation) }。 式中 $\mat A$ 是已知的, 而 $\lambda$ 和 $\bvec v$ 是未知的。 显然, 当 $\bvec v = \bvec 0$ 时方程恒成立, 所以我们通常只对非零解感兴趣。 也就是说, 我们希望找到一些\textbf{非零向量} $\bvec v$, 使得矩阵 $\mat A$ 乘以该向量以后方向不变\footnote{“方向” 只是从几何向量\upref{GVec}中沿用过来的一个习惯说法, 注意\autoref{eq_MatEig_1} 中的所有量都可以是复数。 两个向量方向相同意味着一个向量乘以标量(包括复数)可以得到另一个。}。 对于每个这样的向量, 我们用一个标量 $\lambda$ 来描述其模长的改变。 我们把这些向量叫做\textbf{本征矢(eigen vector)}, 把对应的 $\lambda$ 叫做\textbf{本征值(eigen value)}。 一些教材也翻译成\textbf{特征矢}和\textbf{特征值}。 小时百科中, eigen 译作 “本征”, 而 characteristic 译作 “特征”。

\subsubsection{几何意义}
几何上来讲, 实数矩阵对应的线性变换相当于把坐标网格做旋转、拉伸、翻折等操作。% 链接未完成
所以一般而言, 一个非零向量在变换后长度和方向都会改变。 但也可能存在一些特殊的非零向量, 使得变换后只可能改变长度而不改变方向。 这些向量就是本征方程的解。 注意这种几何理解仅适用于实数矩阵以及实数本征值和本征矢的解。

\subsection{求解本征方程}\label{sub_MatEig_1}

若令 $\mat I$ 为 $N\times N$ 的单位矩阵\footnote{即对角线上的元为 1, 其他元为 0, 见“矩阵\upref{Mat}”}, 则本征方程\autoref{eq_MatEig_1} 移项后得到一个齐次线性方程组\upref{LinEqu}
\begin{equation}\label{eq_MatEig_2}
(\mat A - \lambda\mat I)\bvec v = \bvec 0~.
\end{equation}
括号中的矩阵相当于把矩阵 $\mat A$ 的对角线上的元都减去 $\lambda$ 得到的方阵。 要确保方程有非零解, 只需令系数矩阵 $\mat A - \lambda\mat I$ 不是满秩\upref{MatRnk}的, 即行列式为零
\begin{equation}\label{eq_MatEig_4}
\abs{\mat A - \lambda\mat I} = 0~.
\end{equation}
这是一个关于 $\lambda$ 的 $N$ 阶多项式, 称为\textbf{特征多项式(characteristic polynomial)}。 并不是所有矩阵都有本征向量和相应本征值,比如非特殊角的旋转变换并不能把任意向量旋转到与原向量平行。但是根据代数学基本定理,复数域上的特征多项式(一元n次方程)必有n个根(重根的重数包括在内),即有$n$个本征值。对于$n$阶矩阵$\mat A$,如果$\lambda$为特征方程的单根,称$\lambda$为$\mat A$的\textbf{单本征值};如果$\lambda$是$k$重根,称之为$\mat A$的\textbf{$k$重本征值},并称$k$为$\lambda$的代数重数。

把本征值记为 $\lambda_i$ ($i = 1, 2\dots N$)。 将它们依次代入\autoref{eq_MatEig_2}, 就可以分别解出对应的本征矢。通常把\autoref{eq_MatEig_2} 的解空间称为矩阵$\mat A$对应于本征值$\lambda$的\textbf{本征子空间},记为$V_{\lambda}$,称该子空间的维数为\textbf{几何重数},即线性无关的本征向量。根据方程的线性可知,本征子空间的任意向量都是本征矢。

在物理上,如果本征子空间维度是$1$,我们称相应的本征值是\textbf{非简并(non-degenerate)}的,若几何重数$n_i$大于1,则称 $\lambda_i$  是$n_i$ 重\textbf{简并(degenerate)}的, 把 几何重数$n_i$ 叫做\textbf{简并数(degeneracy)}。
可以证明,任意本征值的几何重数不大于代数重数。


\begin{example}{二维矩阵的本征方程}
给出任意二维实数矩阵
\begin{equation}
\mat A = \pmat{a & b \\ c & d}~.
\end{equation}
要求它的本征值和本征矢, 其特征多项式(\autoref{eq_MatEig_4} )为
\begin{equation}
\vmat{a-\lambda & b \\ c & d-\lambda} = (\lambda-a)(\lambda-d) - bc = 0~.
\end{equation}
解二次方程得两个本征值为
\begin{equation}
\lambda_\pm = \frac{(a + d) \pm \sqrt{(a-d)^2 + 4bc}}{2}~.
\end{equation}
复数域中必定存在两个根, 包括重根。 若要求本征值为实数, 则需要另判别式(根号中的式子)大于零, 否则本征方程无解。

本征矢为
\begin{equation}
\bvec v_\pm = C\pmat{b\\ \lambda_\pm - a} = C\pmat{\lambda_\pm - d\\ c}~,
\end{equation}
其中 $C$ 是任意非零常数。 若两本征值相同, 则只存在一个一维的本征矢空间, 即一条直线。
\end{example}
下面列举与本征值相关的常用结论。
\begin{theorem}{}\label{the_MatEig_2}
设$\lambda_1,\lambda_2...\lambda_m$是方阵 $\mat A$的$m$个互不相同的本征值,分别对应本征向量$\bvec x_1,\bvec x_2...\bvec x_m$。则$\bvec x_1,\bvec x_2...\bvec x_m$线性无关。
\end{theorem}
证明: 
用数学归纳法证明。
$m=1$显然成立。设$i=m-1$成立,需要证明$i=m$成立。用反证法,假设$i=m$时,$\bvec x_1,\bvec x_2...\bvec x_m$线性相关,设有若干不全部为0的系数使得:
\begin{equation}\label{eq_MatEig_3}
\bvec x_m=a^1\bvec x_1+a^2\bvec x_2+...a^{m-1} \bvec x_{m-1}~,
\end{equation}
则有:
\begin{equation}
 A\bvec x_m=a^1\lambda_1\bvec x_1+a^2\lambda_2\bvec x_2+...a^{m-1}\lambda \bvec x_{m-1}=\lambda_m\bvec x_m~.
\end{equation}
由于\autoref{eq_MatEig_3} 中的$a^i(i=1,2...m-1)$不全为0。所以至少有部分本征值满足$\lambda_i=\lambda_m$,这与题设矛盾,因而定理成立。

上述定理可以推广为几何重数大于$1$的情况,即:
\begin{corollary}{}\label{cor_MatEig_1}
设$\lambda_1,\lambda_2...\lambda_m$是方阵 $\mat A$的$m$个互不相同的本征值,分别对应本征向量组$\{\bvec x_{1i}\},\{\bvec x_{1i}\}...\{\bvec x_{mi}\}$,$i$的最大值取遍几何重数。则$\{\bvec x_{1i}\},\{\bvec x_{1i}\}...\{\bvec x_{mi}\}$线性无关。
\end{corollary}
证明思路一致。


\begin{theorem}{}
设$\lambda$是方阵$\mat A$的一个本征值,$\bvec x$是对应的本征向量。则有:
\begin{enumerate}
\item 当$\mat A$可逆时,$\frac{1}{\lambda }$是$\mat A^{-1}$的本征值;
\item 当$\mat A$可逆时,$\frac{|A|}{\lambda }$是伴随矩阵$\mat A^{*}$的本征值;
\item $f(x)$是$x$的一元多项式,则$f(\lambda)$是$f(\mat A)$的一个本征值,并且$\bvec x$依然是矩阵$\mat A^{-1},\mat A^{*},f(\mat A)$的分别对应于本征值$\frac{1}{\lambda},\frac{|\mat A|}{\lambda},f(\lambda)$的本征向量。
\end{enumerate}
\end{theorem}
Proof.
现证第一点。由于$\mat A\mat A^{-1}\bvec x=\mat A^{-1}\mat A \bvec x=\lambda \mat A^{-1} \bvec x=\bvec x$,因此$\bvec x$依然是$\mat A^{-1}$的本征向量,且本征值为$\lambda^{-1} $。从线性变换的角度上看,$\mat A$使得$\bvec x$“伸长”为原来的$\lambda$倍,其逆操作必定是缩小为原来的$\lambda^{-1}$,才能保证本征向量不变。

第二点易证,留作习题。下证第三点,为证明方便,使用爱因斯坦求和约定。

设$f(x)=a_nx^n$,则$f(\mat A)=a_i\mat A^i,f(\lambda)=a_n\lambda ^n$。显然我们有$f(\mat A)\bvec x=a_i\mat A^i \bvec x=a_i\lambda^i \bvec x=f(\lambda)\bvec x$,得证。

\subsection{对角化与相似变换}

求解矩阵的本征方程的过程有时候也叫做矩阵的\textbf{对角化(diagonalization)},因为方阵\textbf{可对角化}的充要条件是$A$有$n$个线性无关的本征向量。
\begin{theorem}{}\label{the_MatEig_1}
$n$阶方阵$A$可对角化的充要条件是$A$有$n$个线性无关的本征向量。
\end{theorem}
Proof.
设$Q=(\bvec x_1,\bvec x_2...
\bvec x_n)$是实现该相似变换的过渡矩阵,相似变换为:
\begin{equation}
\begin{aligned}
AQ&=QB\\
A(\bvec x_1,\bvec x_2...
\bvec x_n)&=Q\opn{diag}(\lambda_1,\lambda_2...\lambda_n)\\
(A\bvec x_1,A\bvec x_2...
A\bvec x_n)&=(\lambda_1 \bvec x_1,\lambda_2\bvec x_2...\lambda_n\bvec x_n)\\
\end{aligned}~.
\end{equation}
必要性的证明同理。

因此,过渡矩阵的每一列都是$A$的本征向量。由于过渡矩阵可逆,因而这$n$个本征向量线性无关。
如果能找到使 $\mat A$ 为对角矩阵的 $\mat Q$ 就相当于解出了本征方程\autoref{eq_MatEig_1}, 这就是 “对角化” 名字的由来。

\subsection{相似不变量}
相似变换不改变矩阵的本征值和本征向量,因为这仅仅是改变线性映射和向量的“表示”,$f(\bvec x)=\lambda \bvec x$这个关系是不随基的改变而改变的。当然,你也可以利用相似变换后矩阵和向量坐标关系的变化来证明这一点。由此我们可以进一步总结基本的相似不变量:\textbf{矩阵的秩、行列式、本征值}。

实际上,\textbf{矩阵的迹也是相似不变量}。设过渡矩阵为$S$,则有$\opn {Tr}(S^{-1}AS)=\opn {Tr}(SS^{-1}A)=\opn {Tr}A$。
\begin{exercise}{}
用其他方式证明迹是相似不变量。(提示:用指标表示法。)
\end{exercise}
实际上,还有两个常见的相似不变量,可以让我们明晰矩阵元和本征值的关系。
\begin{theorem}{}\label{the_MatEig_3}
设$n$阶方阵$\mat A=(a^i_j)$的$n$个本征值为$\lambda_1,\lambda_2,...\lambda_n$,则
\begin{enumerate}
\item $\prod\limits^n_{i=1}\lambda_i=|\mat A|$
\item $\lambda_1+\lambda_2+...\lambda_n=\opn{Tr}A =a^1_1+a^2_2+...a^n_n$
\end{enumerate}
\end{theorem}
Proof.

关键是展开矩阵的特征多项式。
\begin{equation}
\begin{aligned}
|\lambda \boldsymbol{I}-\boldsymbol{A}| & =\left|\begin{array}{cccc}
\lambda-a^1_{1} & -a^1_{2} & \cdots & -a^1_{n} \\
-a^2_{1} & \lambda-a^2_{2} & \cdots & -a^2_{ n} \\
\vdots & \vdots & & \vdots \\
-a^n_{ 1} & -a^n_{ 2} & \cdots & \lambda-a^n_{ n}
\end{array}\right| \\
&=a_i\lambda^i\\
& =(\lambda-\lambda_1)(\lambda-\lambda_2)...(\lambda-\lambda_n)=0~.
\end{aligned}
\end{equation}
在第二种展开方式里,$\prod\limits^n_{i=1}\lambda_i$的系数为$(-1)^n$,因此我们只需要在第一种展开方式里找不含有$\lambda$的项即可。首先想到的是对角元连乘,里面有一项$(-1)^{n}\prod \limits^n_{i=1}a^i_i$。在其他的展开项里,总含有$\prod \limits^k_{i=1}(\lambda-a^i_i)$,其中$k\le n-2$。可以进一步展开多项式,选取不含有$\lambda$的项。由于保留了矩阵元前的系数$-1$,所以最终得到的是$(-1)^n |\mat A|$,第一点得证。

现在证明第二点。同样的,在第二种展开方式里我们可以看到这一项实际上是$(\lambda)^{n-1}(\lambda_1+\lambda_2...+\lambda_n)$。由于除了对角元连乘以外的项里连乘数目$k\le n-2$,因此这一项只在对角元连乘项里,展开这一项便可得证。
\begin{corollary}{}
设$\mat A$是$n$阶矩阵,则$|\mat A|=0$的充要条件是数$0$为矩阵$\mat A$的本征值。
\end{corollary}
关于本征值还有一个常用结论:\textbf{本征值的几何重数小于或等于对应的代数重数},这里不做证明。
\begin{exercise}{}
\begin{enumerate}
\item 证明三角矩阵(上三角或下三角)的对角元为本征值。
\item 证明\textbf{严格上三角}矩阵为幂零矩阵,即矩阵自乘若干次后为零矩阵。 提示:用Hamilton-Cayley 定理。
\end{enumerate}
\end{exercise}

