% 集合的运算
% keys 交|并|对偶原理|de Morgan定理
% license Xiao
% type Tutor

\pentry{集合\upref{Set}}
通过 “集合\upref{Set}” 的预备知识,我们应该知道,集合是由元素构成的。这是说,在一开始讨论集合的时候,必须先行给出一个集合,才能继续进行有关的讨论。这是说当你说集合 $A$ 时,我们已经知道了它的元素。具体而言,当谈论集合 $A$ 时,你应该马上将其理解成 $A=\{a_1,\cdots,a_n,\cdots\}$,其中花括号里的是 $A$ 的元素。既然提到集合相当于指出它的元素,那同时就能判断哪些使它的元素,哪些不是它的元素,$a$ 是 $A$ 的元素记作 $a\in A$,$a$ 不是 $A$ 的元素记作 $a\notin A$,这些都是在提到集合的时候就同时表明了的。有了集合后,一个重要的任务是其上运算的定义及运算规则,这便是本节要介绍的,本词条较通常的教科书创新之处在于给出了通常集合中 “且”和“或” 的精确定义。
\subsection{与 $\land$ 和 或 $\lor$}
同样在“集合\upref{Set}”里,已经知道集合间可以进行运算 $\cap$(交)、$\cup$(并)。现在来给出它们的具体定义。严格的定义需要用到两个逻辑学概念 $\land$(读作“与”) 和 $\lor$(读作 “或”)。它们仅仅代表两个函数,下面将记 $Z_2=\{0,1\}$。
\begin{definition}{与}\label{def_OpSet_2}
称函数 $f:Z_2\times Z_2\rightarrow Z_2$ 为\textbf{与},若
\begin{equation}
f(0,0)=0,\quad f(0,1)=0,\quad f(1,0)=0,\quad f(1,1)=1~.
\end{equation}
并将与函数 $f$ 记作 $\land$。
\end{definition}
\begin{definition}{或}\label{def_OpSet_1}
称函数 $f:Z_2\times Z_2\rightarrow Z_2$ 为\textbf{或},若
\begin{equation}
f(0,0)=0,\quad f(0,1)=1,\quad f(1,0)=1,\quad f(1,1)=1~.
\end{equation}
并将或函数 $f$ 记作 $\lor$ 。
\end{definition}
下面的定理起着重要作用。
\begin{theorem}{与或分配律}\label{the_OpSet_1}
对任意 $a,b,c\in Z_2$,成立恒等式:
\begin{equation}\label{eq_OpSet_1}
\begin{aligned}
\lor(\land(a,b),c)=\land(\lor(a,c),\lor(b,c)),\\
\land(\lor(a,b),c)=\lor(\land(a,c),\land(b,c))~.
\end{aligned}
\end{equation}
\end{theorem}
\textbf{证明:}
我们以第一式的证明为例,读者完全可以类似的证明出第二式。证明的关键只需要代入所有可能的 $a,b,c$ 即可,由于取值只有 $0,1$,这是方便的。

若 $c=1$,由\autoref{def_OpSet_1} ,$\lor(\land(a,b),1)=1$。于是只需证明右边无论 $a,b$ 取 $0,1$ 哪个指都为1即可。同样由 $\lor$ 的定义,右边 $\land$ 函数的两个变量都是 1:
\begin{equation}
\lor(a,1)=\lor(b,1)=1~.
\end{equation}
于是由\autoref{def_OpSet_2} ,右边为 $\land(1,1)=1$,即
\begin{equation}
\lor(\land(a,b),1)=\land(\lor(a,1),\lor(b,1))~.
\end{equation}

若 $c=0$,那么仅当 $\land(a,b)=1$ 时左边为1,而右边也是仅当 $a=b=1$ 时为1。于是
\begin{equation}
\lor(\land(a,b),0)=\land(\lor(a,0),\lor(b,0))~.
\end{equation}
\textbf{证毕!}

\autoref{eq_OpSet_1} 之所以称为分配律,是因为若记 $a\land b:=\land(a,b),a\lor b:=\lor(a,b)$ ,则\autoref{eq_OpSet_1} 成为
\begin{equation}
\begin{aligned}
(a\land b)\lor c=(a\lor c)\land(b\lor c),\\
(a\lor b)\land c=(a\land c)\lor(b\land c)~.
\end{aligned}
\end{equation}

\subsection{属于$\in$的数值化}
在精确定义集合的交和并前,还需要引入一种特定的函数,它将元素和集合的关系数值化了,不妨称为集合上的坐标函数。
\begin{definition}{}
设 $X$ 是集合,则称函数 $f_X$ 是集合 $X$ 上的\textbf{坐标},若
\begin{equation}
f_X(x)=\left\{\begin{aligned}
&1,\quad x\in X,\\
&0,\quad x\notin X~.
\end{aligned}\right.
\end{equation}
并称 $f_X(x)$ 是元素 $x$ 关于 $X$ 的\textbf{坐标}。
\end{definition}
元素的坐标其实就表达了元素是否属于对应的集合。由于坐标函数的取值是 $0,1$,所以可以将其和逻辑与或的关系联系起来。

\begin{definition}{简记约定}
记\\
 $\land(f_X(x),f_Y(y))$ 为 $x\in X\land y\in Y$ ,\\
$\lor(f_X(x),f_Y(y))$ 为 $x\in X\lor y\in Y$。
\end{definition}

利用这一约定,则可将\autoref{the_OpSet_1} 翻译成下面定理
\begin{theorem}{}
设 $X,Y,Z$ 是三个集合,则成立
\begin{equation}
\begin{aligned}
&(x\in X\land y\in Y)\lor z\in Z=(x\in X\lor z\in Z)\land(y\in Y\lor z\in Z),\\
&\qty(x\in X\lor y\in Y)\land z\in Z=(x\in X\land z\in Z)\lor(y\in Y\land z\in Z)~.
\end{aligned}
\end{equation}
\end{theorem}
这便是\autoref{eq_OpSet_1} 称为分配律的原因。

定义交和并的原材料已经制备完成了,接下来就可以造出交并的严格形式了。
\subsection{交 $\cap$ 和并 $\cup$}
\begin{definition}{交集}\label{def_OpSet_3}
设 $X,Y$ 是两集合,称集合
\begin{equation}
\{z|z\in X\land z\in Y=1 \}~
\end{equation}
为 $X,Y$ 的\textbf{交集},记作 $X\cap Y$。
\end{definition}
这事实上表达了交集的元素同时属于两集合这样的概念。
\begin{definition}{并集}\label{def_OpSet_4}
设 $X,Y$ 是两集合,称集合
\begin{equation}
\{z|z\in X\lor z\in Y=1 \}~
\end{equation}
为 $X,Y$ 的\textbf{并集},记作 $X\cup Y$。
\end{definition}
这事实上表达了并集的元素至少属于两集合之一这样的概念。
\subsection{运算性质}
现在,我们就可以严格的证明集合的交运算和并运算的分配律了。
\begin{theorem}{交并的分配律}
设 $X,Y,Z$ 是三个集合,则成立
\begin{equation}
\begin{aligned}
&(X\cap Y)\cup Z=(X\cup Z)\cap (Y\cup Z),\\
&(X\cup Y)\cap Z=(X\cap Z)\cup (Y\cap Z)~.
\end{aligned}
\end{equation}
\end{theorem}
\textbf{证明:}由\autoref{def_OpSet_3}, \autoref{def_OpSet_4} ,只需证明
\begin{equation}
\begin{aligned}
z\in (X\cap Y)\lor z\in Z=1~,
\end{aligned}
\end{equation}
 当且仅当 
 \begin{equation}
 (z\in X\cup Z)\land (z\in Y\cup Z)=1~.
 \end{equation}
 和 
 \begin{equation}
\begin{aligned}
z\in (X\cup Y)\land z\in Z=1~,
\end{aligned}
\end{equation}
 当且仅当 
 \begin{equation}
 (z\in X\cap Z)\lor (z\in Y\cap Z)=1~.
 \end{equation}



\textbf{证毕!}