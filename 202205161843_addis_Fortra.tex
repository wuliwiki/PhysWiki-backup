% Fortran 入门笔记

\subsection{常识}
\begin{itemize}
\item Fortran 代码不区分大小写
\item 感叹号用于写注释
\item `&` 用于续行, 如果一个 token 需要续行, 那么下一行的行首也需要 `&`. 另外, 必须放在注释前面.
\item 程序首尾  `program <name> ....  end program <name>`, 或者直接用 `end`
\item 没有子程序的时候, `program <name>` 可以省略, 最后一行 `end program <name>` 也可以省略
\item `print*, <item1>, <item2>....` 星号代表默认格式
\item 用 `read*, item1, item2, ...` 来输入变量, 要先定义变量类型. 
\item `read` 读取 character 类型的时候, 只能读取字母, 不能有任何其他符号
\item 字符串可以用单引号也可以用双引号
\item 可以用 `;` 符号在一行写多个语句, 叫 statement separator. 同一行中的每个 statement 不能超过 132 个字符
\item `stop` 可用于终止主程序, `return` 终止子程序
\item 所有的小括号前面可以有空格
\item `include "<file>"` 就是把文件的内容复制到当前位置, 应该相当于 C++ 的 #include
\end{itemize}

\subsection{算符}
\begin{itemize}
\item 指数算符是 `**`
\item 比较算符有: `==   >=   <=  /= `.  注意不等是用斜杠不是感叹号. 感叹号是用于注释的
\item 逻辑算符有 `.and.   .or.   .not.   .eqv.` (同或)  `.neqv.` (异或)
\item `//` 用于连接字符串
\end{itemize}

\subsection{结构}
\begin{itemize}
\item `if (<logical>) then ....  else .... end if` 注意要有括号
\item `if (<logical>) then ... else if (<logical>) then ... else ... end if`
\item `if` 前面可以加上 `<name>:` 如果这样做,  end if 后面也一定要加上 `<name>`  , 也可以选择在 `else if .. then` 或 `else` 后面加上 `<name>`. 注意只有开头有冒号, 后面没有冒号 .
\item select case, 见 http://www.tutorialspoint.com/fortran/select_case_construct.htm
\item do 循环:   `do ii = 1,3; ...; end do;` 也可以设置步长 `ii = int1, int2, step`. 其中 `step` 可以是负值.
\item do while 循环: `do while (<logical>) ....  end do`
\item `exit` 相当于 C 语言的 break, `exit <name>` 用于退出指定循环.
\item `cycle` 相当于 C 语言的 continue
\item do 前面可以加上 `<name>:` 如果这样, `end do` 后面也一定要加上 `<name>`
\item `cycle` 执行下一个当前循环, `cycle <name>` 执行下一个指定循环
\item `do` 后面什么都不加相当于无条件循环, 需要用 `exit` 退出.
\item statement label 是在 statement 前面的 1-5 位数字.
\item `go to <label>` 可以使程序跳转到 `<label>` 的语句.
\end{itemize}

\subsection{数据类型}
\begin{itemize}
\item `integer, real, complex, character, logical` 是五种 intrinsic types, 前三种是 numeric, 剩下的是 non-numeric.
\item 从 fortran90 开始, 所有变量都要定义. 用 implicit none 语句可以使未定义的变量出错. 亲自证明了不声明变量类型会导致很严重的错误且很难调试.
\item 除了 imlicit none, 另一种声明如 implicit double precision (a-h,o-z), 作用是默认所有 a-h, o-z 开头的变量都是 double precision, 而 i-n 开头的所有变量都是普通 integer*4. 这种做法不建议使用.
\item 所有的变量只能在 implicit none 后面紧接着, 不能在程序中定义.
\item 声明变量格式:  `<类型> :: a=1, b=2, ....`  初值可选
\item 变量声明时 `::` 可以省略
\item 5 种基本变量类型可以定义 constant 格式, 如  `real, parameter :: pi = 3.1415927`
\item 声明变量后, 也可以把某个变量变成常数, 直接用 parameter (var1 = 1, var2 = 2) 即可. 括号不可省略
\item `selected_int_kind(8)` 可以返回 integer 类型的 kind, 是的 integer 有 8 位有效数字
\item `selected_real_kind(15, 307)` 可以返回 real 类型的 kind, 使得 real 有 15 位有效数字和最大指数 307
\item `huge()` 可以测试某个变量的最大值
\item `tiny()` 可以测试某个变量的最小值
\item `<类型>(kind=..) :: <name>` 的括号中 kind = 可以省略, 直接写数字. 这个数字叫做 kind type parameter
\item `kind()` 函数可以检查某个变量的字节数.
\item `precision()` 函数可以检查某个变量的精度(小数位数)
\item 整型和整型相除得到整数, real 和 整型相除得到 real. real 赋值给 integer 向下取整.
\item `1` 和 `2` 都是整型! 尤其注意 `1/2 = 0` 而不是 1.   至少要加个小数点 !
\item literal constant 指的是程序中写入的值, 例如 `123, 'good', 1.7e-6` 等.
\item 声明矩阵的时候, 只能直接用 constant 指定矩阵大小和初值.
\end{itemize}

\subsubsection{real}
\begin{itemize}
\item 指定 literal constant 的类型: 1.3e8 只有 real(4) 的精度 (7位有效数字, 默认), 1.3d8 是 real(8) 的精度(16位有效数字, 即 double), 1.3q8 是 real(16) 的精度(33位有效数字). d 代表 double, q 代表 quad. 另外也可以用 1.3e8_16 的格式 , 得到类型  real(16) 与 1.3q8 等效, 这么做的好处是可以用一个常数变量 (parameter) 来指定数据类型,例如 1.3e8_rk 
\item real 的默认是单精度 real(4). 双精度 real(8) 需要声明.
\item real 直接赋值溢出会出错, 算符运算结果溢出产生 Infinity
\item underflow, 例如 1e-40. 会产生 warning.
\item 强制类型转换 real (var, kind)
\end{itemize}

\subsubsection{integer}
\begin{itemize}
\item `integer(kind = <n>)` 设置占用的字节, 可以取 1, 2, 4, 8 或者 16. 即最大值. 默认为 
   kind = 4. (huge 约 2E9)
\item 强制类型转换 int() 注意不是 integer()
\item `nint()` 采用四舍五入进行类型转换, 而 int() 采用向下取整
\item literal constants 设定 kind 用下划线, 例如 `3_4` 是 `kind = 4` 的整数 3.
\end{itemize}

\subsubsection{integer}