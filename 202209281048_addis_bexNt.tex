% 北太天元插件开发笔记

\begin{issues}
\issueDraft
\end{issues}

\begin{lstlisting}[language=cpp, caption=extern\_obj.h节选]
namespace baltam {
    /**
     * @class extern_obj_base
     * @brief 所有不属于 BALTAM 的外部对象基类.
     *
     * 当合作开发者/社区开发者使用 C++ 开发并且需要将某些对象临时托管到 BALTAM 中时,
     * 必须从这个类继承.
     *
     * 软件内核对该类对象仅定义抽象的方法,具体实现交给开发者.软件无需了解
     * @p extern_obj_base 子类的实现细节.
     */
    struct extern_obj_base {
        virtual extern_obj_base* dup() const { return nullptr; }
        virtual ~extern_obj_base() {};
        virtual std::string to_string() const { return ""; }
        virtual std::string classname() const { return "extern object"; };
        int sid; ///< 整数,用于存储内核自动分配的外部类 ID
    };
}
\end{lstlisting}


\begin{lstlisting}[language=cpp,caption=vector 插件的 main.cpp]
#define PLUGIN_NAME "vector" // 定义插件名称

// 声明要实现的函数

// 声明帮助

// 自定义类型, 必须基于 extern_obj_base
// 这里的 bxArray 相当于 Matlab api 的 mxArray, 基本上就是 Matlab 的矩阵, 什么都能装.
// https://www.mathworks.com/help/matlab/apiref/mxarray.html
struct bVector : public extern_obj_base {
    std::vector<bxArray*> data;
    BALTAM_LOCAL static int ID;
    extern_obj_base *dup() const override;
    ~bVector() override;
    // 可选:转化为字符串的实现
    std::string to_string() const override;
    // 可选:自定义类型名字
    std::string classname() const override;
};

// 并定义插件的 ID 这一静态变量(会被内核使用)
int bVector::ID = 0;

// 复制这个对象(duplicate)
// bex 就相当于 matlab 的 mex 哈哈哈
extern_obj_base *bVector::dup() const {
    bVector *ret = new bVector(*this);
    for (size_t i = 0; i < this->data.size(); ++i){
#ifdef BV_USE_DYN_LOADER
        ret->data[i] = bex::__bxDuplicateArray(data[i]);
#else
        ret->data[i] = bxDuplicateArray(data[i]);
#endif
    }
    return ret;
}

// 把 vector<> 中的每个 bxArray 毁灭
bVector::~bVector(){
    for (auto &i : data){
#ifdef BV_USE_DYN_LOADER
        bex::__bxDestroyArray(i);
#else
        bxDestroyArray(i);
#endif
    }
}

// 打印这个对象
std::string bVector::to_string() const {...}

// 自定义类型的名字
std::string bVector::classname() const {
    return "bVector";
}

// 显示 bVector 信息
// bxGetExtObj() 把 bxArray 转换为外部对象
void bv_show(int, bxArray *[], int nrhs, const bxArray *prhs[]){
    if (nrhs > 1){
        return;
    }
    bVector *bv = bxGetExtObj<bVector>(prhs[0], bex::__bxGetExtObj_impl);
    for (size_t i = 0; i < bv->data.size(); ++i){
        std::cout << '[' << i << "]    "
            << bex::__bxTypeCStr(bv->data[i]) << std::endl;
    }
}
\end{lstlisting}
