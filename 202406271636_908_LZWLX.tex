% 粒子物理学
% license CCBYSA3
% type Wiki

(本文根据 CC-BY-SA 协议转载自原搜狗科学百科对英文维基百科的翻译)


\textbf{粒子物理(又称作“高能物理”)}是研究基本粒子之间相互作用、相互转化规律的科学,其主要目的是要找到一种既简单又普遍的物理原理来统一解释基本粒子之间五花八门的相互作用、相互转换现象[1],其研究对象就是物质的基本结构和基本相互作用。

\subsection{概论}
在过去的一百多年里,我们对基本粒子性质的认识有了长足的进步,建立发展并逐步完善了粒子物理标准模型,粒子物理标准模型的预测与实验测量达到了惊人的吻合程度。然而,时至今日,我们并没有找到一个能够统一描述所有粒子以及所有相互作用的理论。

自然界中已知的四种基本相互作用是引力相互作用、电磁相互作用、弱相互作用以及强相互作用,每一种相互作用的特点如表1.1所示。
\begin{table}[h]
\centering
\caption{四种基本相互作用性质比较}
\begin{tabular}{|c|c|c|c|c|c|}
\hline
\textbf{相互作用} & \textbf{强度} & \textbf{力程} & \textbf{媒介粒子} & \textbf{参与作用粒子} & \textbf{末端态 }\\
\hline
强相互作用 & 1 & $10^{-15}$ m & 胶子 & 夸克, 胶子 & 强子 \\
\hline
弱相互作用 & $10^{-5}$ & $<10^{-17}$ m & $W^{\pm}, Z^0$ & 夸克, 电子, 中微子等 & 无 \\
\hline
电磁相互作用 & $1/137$ & $F \propto 1/r^2$ & 光子 & 所有带电粒子 & 原子等 \\
\hline
引力 & $10^{-39}$ & $F \propto 1/r^2$ & 引力子? & 所有粒子 & 太阳系等 \\
\hline
\end{tabular}
\end{table}
从上表可以看出,强相互作用与弱相互作用是短程力,其有效范围不会超出原子核的尺度($10^{-15} m$ ),因此其效应并不会在宏观中有所体现,而在日常生活中有所体现的引力和电磁相互作用是长程力,粒子物理的主要研究对象就是除引力外的其它三种相互作用。2012年LHC发现Higgs玻色子后,有人把Higgs玻色子通过“汤川耦合”使费米子获得质量的作用成为“第五种相互作用”。[2]

\subsection{ 对称性和守恒律}
对称性在现代物理学的研究中起着至关重要的作用,而对称性的破缺在粒子物理的研究中尤其重要。比如,希格斯机制就与对称性自发破缺有着密切的联系。物理学在这方面探索的一个重要进展是建立了艾米·诺特定理,这个定理首先是在经典物理学中给出的,后来推广到量子物理范围内也得到了普遍证明。

Noether定理[2]:\textbf{如果运动规律在某一不明显依赖于时间的变换下具有不变性,则必然存在一种对应的守恒定律。}

比如说,经典力学中的动量守恒定律、角动量守恒定律以及能量守恒定律,分别源于空间平移不变性、空间转动不变性以及时间平移不变性。在量子物理的情况下也有相似的结论。

\subsection{粒子的分类}
按照是否参与强相互作用,粒子可以本分为两类:轻子和强子。前者不参与强相互作用而后者参与。

\subsubsection{3.1 轻子}
轻子是不直接参与强相互作用的粒子,它们可以直接参与电磁相互作用和弱相互作用,比如电子等。目前发现的轻子一共有六种: