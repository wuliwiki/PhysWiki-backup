% 复流形的切空间和复化切空间
% keys 复流形
% license Usr
% type Tutor

\pentry{复流形\nref{nod_CMani}}{nod_fd19}

\subsection{(余)切空间}
这是直接视 $n$ 维复流形 $M$ 为 $2n$ 维实流形得到的,$T_{\mathbb R, p} M$。对余切空间同理。

\subsection{全纯切丛与反全纯切丛}
接下来是\textbf{复化切空间}的定义。对于复流形 $(M, J)$(这里的 $M$ 是一个实流形,配备了 $J$ 的复结构),其上的 $J_p: T_{\mathbb R,p} M \to T_{\mathbb R, p} M$ 即限定在 $p$ 点的复结构 $J$,是 $p$ 点的切空间 $T_{\mathbb R, p} M$ 的自同态。

考虑将切空间复化:$T_{\mathbb R, p} M \mapsto T_{\mathbb R, p} M \otimes \mathbb C$,即每个切矢量的“系数”可以从实数变为复数,则 $J_p$ 给出了:
\begin{equation}
J_p : T_{\mathbb R,p} M \otimes C \to T_{\mathbb R, p} M \otimes C ~.
\end{equation}

$J^2 = -1$ 给出了 $J$ 在 $T_{\mathbb R, p} M$ 的特征值是 $\pm \mathrm i$,就可以对 $T_{\mathbb R, p} M \otimes \mathbb C$ 的直和分解,对应到两个特征空间:
\begin{equation}
T_{\mathbb C, p} M = T_{\mathbb R, p} M \otimes C = T_p^{1, 0} M \oplus T_p^{0, 1} M ~.
\end{equation}
其中,$T_p^{1, 0} M$ 对应 $+\mathrm i$ 特征值,称为\textbf{全纯切空间},$T_p^{0, 1} M$ 对应 $-\mathrm i$ 特征值,称为\textbf{反全纯切空间}。


类似的,对整个流形而言,
\begin{equation}
T_{\mathbb C} M = T_\mathbb R M \otimes C = T^{1, 0} M \oplus T^{0, 1} M ~.
\end{equation}
其中,$T^{1, 0} M$ 对应 $+\mathrm i$ 特征值,称为\textbf{全纯切丛},$T^{0, 1} M$ 对应 $-\mathrm i$ 特征值,称为\textbf{反全纯切丛}。

可以写出:
\begin{equation}
\begin{aligned}
T^{1, 0} M &= \{v \in T_{\mathbb C} M \mid J(v) = +\mathrm i v\} \\
T^{0, 1} M &= \{v \in T_{\mathbb C} M \mid J(v) = -\mathrm i v\} 
\end{aligned} ~.
\end{equation}

他们分别诱导出基矢量:
\begin{equation}
\left\{ \pdv{z_i} := \frac{1}{2} \left( \pdv{x_i} - \mathrm i \pdv{y_i} \right) \right\} ~,
\end{equation}
\begin{equation} 
\left\{ \pdv{\bar z_i} := \frac{1}{2} \left( \pdv{x_i} + \mathrm i \pdv{y_i} \right)\right\} ~.
\end{equation}

在复几何中,可以在全纯切丛上研究一些几何。

\subsection{复化切空间中的切向量}
利用这个分解,我们可以把任意一个 $T_\mathbb C M$ 中的矢量分解为 $T^{1, 0} M$ 中的矢量和 $T^{0, 1} M$ 中的矢量的直和,一般区分拉丁字母和希腊字母:
\begin{equation}
S^a = S^\alpha + S^{\bar \alpha} ~.
\end{equation}

自然可以给出:
\begin{equation}
J^a_b = \mathrm i \delta^\alpha_\beta - \mathrm i \delta^{\bar \alpha}_{\bar \beta} = \mathrm i(\delta^\alpha_\beta - \delta^{\bar \alpha}_{\bar \beta} )~.
\end{equation}

\subsection{复化余切丛}
复化余切丛的构造方法可以类比实流形和复流形的复化切丛,对偶地,
\begin{equation}
T^*_\mathbb C M = T^*_\mathbb R M \otimes \mathbb C ~.
\end{equation}

