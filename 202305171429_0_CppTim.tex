% C++ 时间处理笔记

\begin{issues}
\issueDraft
\end{issues}

\pentry{C++ 基础\upref{Cpp0}}

\verb|std::tm| 结构(\href{https://en.cppreference.com/w/cpp/chrono/c/tm}{参考})代表一个人类可读的时间格式(数值取决于时区和是否夏令时), 是从 C 语言迁移过来的, 成员如下
\begin{itemize}
\item \verb|tm_year| 年份四位数减去 1900
\item \verb|tm_mon| 月份减去 1
\item \verb|tm_mday| 日 [1,31]
\item \verb|tm_hour| 小时 [0,23]
\item \verb|tm_min| 分钟 [0,59]
\item \verb|tm_sec| 秒 [0,60]
\end{itemize}
\begin{itemize}
\item \verb|tm_wday| 周几(0 代表周日) [0-6]
\item \verb|tm_yday| 一年中的第一天(0 代表 1/1)[0, 365]
\item \verb|tm_isdst| 是否是夏令时 (DST) 若大于零则有 DST, 零则非 DST, 负数则无信息
\end{itemize}

\verb|std::time_t| 结构是一个表示时间的一个数字类型(几乎总是整型)的 \verb|typedef|。 几乎总是用于表示从 1970 年 1 月 1 日 UTC (epoch)起流逝的秒数(不包含增加的几个闰秒\upref{TimeCa})。

\subsubsection{函数}
\begin{itemize}
\item \verb|std::time_t now = std::time(nullptr);| 获取当前时间(如果电脑时间和时区正确,返回值与时区无关)。 返回值一定是 epoch 起流逝的秒数(不包含增加的几个闰秒)。
\item \verb|std::time_t mktime(std::tm* time);| 可以从 \verb|std::tm| 转换为 \verb|time_t|(结合计算机上的时区信息)。
\item \verb|std::tm* localtime(const std::time_t *time);| 可以从 \verb|time_t| 转换为 \verb|std::tm|(结合计算机上的时区信息)。 是 \verb|mktime| 的反函数。
\item \verb|std::tm* gmtime(const std::time_t *time);| 可以把 \verb|timt_t| 转换为 GMT/UTC+0 时区的 \verb|std::tm|(结合计算机上的时区信息)。
\item 对比 \verb|localtime()| 和 \verb|gmtime()| 的结果可以获得计算机的时区(注意时区未必是整数或者是半整数)。
\item \verb|ss >> get_time(std::tm *t, "%Y 年 %m 月 %d 日 %H 时 %M 分 %S 秒")| 可以把 \verb|istringstream| 中的字符串 parse 到 \verb|std::tm| 结构。 完了以后如果 \verb|ss.fail()| 就是 parse 失败。
\item \verb|ss << put_time(const std::tm* t, "%Y-%m-%d %H:%M");| 可以把 \verb|std::tm| 变为指定格式的字符串传给 \verb|ostringstream|。
\end{itemize}
