% 核工程
% license CCBYSA3
% type Wiki

(本文根据 CC-BY-SA 协议转载自原搜狗科学百科对英文维基百科的翻译)

\textbf{核工程}是…的分支工程关心分解的应用原子核(分裂)或结合原子核(融合)或应用其他基于以下原理的亚原子过程核子物理。在核裂变的子领域,它特别包括系统和组件的设计、交互和维护,例如核反应堆s .核电站s,或核武器s.该领域还包括以下研究医学的和其他应用辐射,特别是致电离辐射,核安全,热量/热力学运输,原子核燃料、或其他相关技术(例如, 放射性废物处理)和的问题核扩散。

\subsection{专业领域}
美国目前大约18\%的电力来自核电厂。该领域的核工程师通常直接或间接在核电行业或为国家实验室工作。该行业目前的研究旨在生产具有被动安全特征的经济的和防扩散的反应器设计。一些政府(国家)实验室提供与私营行业相同领域的研究,以及其他领域的研究,如核燃料和核燃料循环、先进反应堆设计和核武器设计和维护。美国反应堆设施的主要管道/训练有素人员来源(军事和文职人员)是美国海军核电计划,包括其在南卡罗来纳州的核电学校。根据需要,到2022年,核工程领域的就业预计将增长约9\%,以取代退休的核工程师,为电厂提供安全系统的维护和更新,并推动核医学的应用。[1]
\begin{itemize}
\item \textbf{Nuclear powerplant}
\begin{figure}[ht]
\centering
\includegraphics[width=6cm]{./figures/dfb28eb7dddcb2a2.png}
\caption \label{fig_HGC_1}
\end{figure}
\item \textbf{B-61 thermonuclear weapon}
\begin{figure}[ht]
\centering
\includegraphics[width=6cm]{./figures/a7b4abdc22df4471.png}
\caption\label{fig_HGC_2}
\end{figure}
\end{itemize}

\subsubsection{1.1 核医学和医学物理学}
医学物理学是核医学的一个重要领域;其子领域包括核医学、放射疗法、健康物理学和诊断成像。[2]高度专业化且操作复杂的设备,包括 x光机、核磁共振成像和 PET 扫描仪和许多其他设备,提供了现代医学的大部分诊断能力,同时还披露了微妙的治疗选项。
\begin{itemize}
\item \textbf{X-Ray image of a male skull}
\begin{figure}[ht]
\centering
\includegraphics[width=6cm]{./figures/a6292744e8f52b76.png}
\caption\label{fig_HGC_3}
\end{figure}
\item \textbf{Magnetic Resonance Imaging (MRI) scan of a human head}
\begin{figure}[ht]
\centering
\includegraphics[width=6cm]{./figures/e79ffd4a3c134ee5.png}
\caption \label{fig_HGC_5}
\end{figure}
\oit
\end{itemize}
