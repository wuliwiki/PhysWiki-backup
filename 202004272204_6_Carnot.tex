% 卡诺热机
% keys 卡诺热机|等温过程|绝热过程|热机效率
\pentry{等温过程\upref{EqTemp}, 绝热过程\upref{Adiab},熵\upref{Entrop}}

\textbf{卡诺循环}(Carnot cycle)是一个特别的热力学循环,使用在一个假想的\textbf{卡诺热机}上,由法国人尼古拉·卡诺于1824年提出,埃米尔·克拉佩龙于1830年代至1840年代扩充,是为了找出热机的最大的工作效率而分析热机的工作过程.

\begin{figure}[ht]
\centering
\includegraphics[width=8cm]{./figures/Carnot_1.png}
\caption{热机示意图(来自维基百科)} \label{Carnot_fig1}
\end{figure}
卡诺循环由两个等温过程,两个绝热过程,下面先通过两个图来直观感受一下卡诺热机的工作过程.

压力 - 体积图,即$P-V$图,是大家十分熟悉的:
\begin{figure}[ht]
\centering
\includegraphics[width=8cm]{./figures/Carnot_2.pdf}
\caption{卡诺循环的压力-体积图} \label{Carnot_fig2}
\end{figure}

温度 - 熵图,即$T-S$图,则如下所示:
\begin{figure}[ht]
\centering
\includegraphics[width=8cm]{./figures/Carnot_3.pdf}
\caption{卡诺循环的温度-熵图} \label{Carnot_fig3}
\end{figure}

\autoref{Carnot_fig2}$1\to 2$、\autoref{Carnot_fig3}$A\to B$,可逆等温膨胀:此等温的过程中系统从高温热库吸收了热量且全部拿去做功.

\autoref{Carnot_fig2}$2\to 3$、\autoref{Carnot_fig3}$B\to C$,等熵(可逆绝热)膨胀:移开热库,系统对环境做功,其能量来自于本身的内能.

\autoref{Carnot_fig2}$3\to 4$、\autoref{Carnot_fig3}$C\to D$,可逆等温压缩:此等温的过程中系统向低温热库放出了热量.同时环境对系统做正功.

\autoref{Carnot_fig2}$4\to 1$、\autoref{Carnot_fig3}$D\to A$,等熵(可逆绝热)压缩:移开低温热库,此绝热的过程系统对环境作负功,系统在此过程后回到原来的状态.


卡诺热机效率 $T_1 < T_2$
\begin{equation}
\eta = \frac{W}{Q_H} = 1 - \frac{T_1}{T_2}
\end{equation}
