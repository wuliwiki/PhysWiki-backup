% 数理逻辑(高中)
% keys 逻辑|高中
% license Usr
% type Tutor

\begin{issues}
\issueDraft
\end{issues}
\pentry{集合\nref{nod_HsSet}}{nod_fc7f}

尽管高中数学教材中在逐渐弱化逻辑的概念,但作为数学根基的内容,学习逻辑对于理解高中的内容,不仅是数学学科,对于理科内容甚至是文科内容都具有相当的助益。同时,这一部分内容在本科学习阶段往往也会作为学生已知的部分略讲或跳过。因此,在高中阶段接触和学习一部分逻辑内容是必要的。

\textbf{逻辑}是研究如何进行正确推理和论证的学科。逻辑能帮助人们理解和判断,哪些推理是合理的,哪些推理是错误的,在数学中,逻辑用于证明定理,确保论证的严谨性和准确性。\textbf{数理逻辑}是逻辑学的一个分支,它应用数学的方法研究逻辑。


\subsection{命题}

一个可以明确判断真假的陈述句被称为\textbf{命题}(proposition)。

如果一个命题里不包含变量,那么这个命题被称为\textbf{陈述命题}(propositional statement)或\textbf{简单命题}(atomic statement)。这种命题直接陈述一个事实,并且它的真假值是固定的。例如,“自然数2是一个偶数”是一个陈述命题,它总是真的。

如果一个命题包含变量,那么这个命题被称为\textbf{开放命题}(open proposition)。这种命题的真假值取决于变量的取值。例如,“x 是一个偶数”就是一个含变量的命题,其中 x 可以取不同的值,因此命题的真假也会随之变化。如果 x=2,命题为真;如果 x=3,命题为假。

命题使用\textbf{真值}(truth value)来表示命题的真假,因此陈述命题有确定的真值,而开放命题的真值与变量相关。真值用\textbf{布尔值}表示。布尔值有两个,分别为“\textbf{真}”和“\textbf{假}”,分别记作:“${\rm True}$”、“$1$”、“$T$”和“${\rm False}$”、“$0$”、“$F$”。

\subsubsection{与命题相似、相关的概念}

概念:

定义:定义是对某一概念的精确描述或解释。定义帮助我们清晰地理解和区分不同的概念。例如,命题的定义是“一个可以明确判断真假的陈述句”。

公理:公理(axiom)是逻辑系统中的基本假设,它们不需要证明,被视为显而易见的真理。公理是其他定理推导的基础。例如,在欧几里得几何中,“通过两点可以作一条直线”就是一个公理。

悖论:悖论(paradox)是逻辑中的一些在逻辑推理中出现的表面上矛盾的结论,通常是因为一些隐含假设或定义问题导致的。悖论后面蕴涵着深刻的思想,并且往往随着悖论的解决,带来哲学、逻辑、数学等领域的巨大变迁。

\subsection{逻辑连接词}

\subsubsection{且}

\textbf{且}(and,也称为同)记号为$A\land B$。

\subsubsection{或}

\textbf{或}(or,也称为或者)记号为$A\lor B$。


\subsubsection{非}
\textbf{非}(not,也称为同)记号为$\lnot A$。

\subsection{量词}

\subsubsection{全称量词}

$\forall$

\subsubsection{存在量词}

$\exists$

\subsection{条件}
如果那么 当且仅当
\subsection{演绎与归纳}

\subsubsection{演绎}

\subsubsection{归纳}