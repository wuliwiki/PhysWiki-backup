% 状态量和过程量

\pentry{力场 保守场 势能\upref{V}}

若一个系统可以用若干参数 $x_1, x_2,  \dots, x_N$ 描述, 那么我们可以把某个状态表示成一个 $N$ 维矢量\upref{GVec} $\bvec x = (x_1, x_2, \dots, x_N)$, 叫做\textbf{状态点}, 把 $\bvec x$ 所有可能的取值范围称为\textbf{状态空间}. 系统关于时间的变化可以看作状态空间中一点划过一条轨迹, 表示为矢量函数 $\bvec x(t)$.

\subsection{状态量}
若系统的一个物理量 $Q$ 只和状态空间的位置有关, 即可以表示为多元函数 $Q(\bvec x) = Q(x_1, \dots, x_N)$, 那么就把它称为\textbf{状态量}. 最简单地, 每个 $x_i$ 本身都是一个状态量. 典型的状态量例如系统的能量, 动量, 温度, 体积, 压强等.

若给出 $t_1$ 时刻的初始状态 $\bvec x_1 = \bvec x(t_1)$ 以及 $t_2$ 时刻的末状态 $\bvec x_2 = \bvec x(t_2)$, 那么该物理量的增量为
\begin{equation}
\Delta Q = Q(\bvec x_2) - Q(\bvec x_1)
\end{equation}
注意这个增量只与初末状态 $\bvec x_1,\bvec x_2$ 有关, 而与过程无关, 也就是无论状态点以什么路径 $\bvec x(t)$ 从初状态移动到末状态, 都会得到同样的增量 $\Delta Q$.

\begin{example}{}
热力学在描述理想气体\upref{Igas}的(宏观)状态时, 只需要用压强 $P$ 和体积 $V$ 和粒子摩尔数 $n$. 当 $n$ 始终不变时, 也可以认为状态量只有 $(P,V)$. 温度 $T$ 可以根据理想气体状态方程(\autoref{PVnRT_eq1}~\upref{PVnRT}) 表示为 $P,V$ 的函数. 类似地, 内能\upref{IdgEng} $E$ 也是一个状态量.

事实上我们也可以用 $(P,T)$ 或 $(V,E)$ 等作为理想气体的状态空间参数,把其他状态量看作它们的函数.
\end{example}

\subsection{过程量}
若一个量 $Q$ 取决于状态空间中的一段运动过程 $\bvec x(t)$($t = [t_1,t_2]$), 它就是过程量.  典型的过程量如做功, 冲量, 传热等.

一种常见的过程量可以用线积分\upref{IntL}定义为
\begin{equation}\label{StaPro_eq1}
\begin{aligned}
Q_{12} &= \int_{\mathcal L} \bvec f(\bvec x_i) \vdot \dd{\bvec x} = \int_{\mathcal L} \sum_i f_i(x_1, \dots, x_N) \dd{x_i}\\
&= \int_{t_1}^{t_2} \sum_i f_i(x_1, \dots, x_N) \dv{x_i}{t} \dd{t}
\end{aligned}
\end{equation}
$\mathcal L$ 表示状态点的 “运动方程” $x_i(t)$ ($i = 1,\dots, N$) 以及起点终点 $\bvec x(t_1), \bvec x(t_2)$. 注意这样定义的过程量只可能和轨迹 $\mathcal L$ 有关而与状态点在轨迹上移动的快慢无关. 所以这里的 $t$ 可以看作轨迹的参数随时间变化而未必是时间本身.

一个具体的例子是力场对单个质点的做功, 下面会在\autoref{StaPro_ex1} 详细讨论.
\begin{equation}
W_{12} = \int_{\mathcal L} \bvec F(\bvec x) \vdot \dd{\bvec x} = \int_{\mathcal L} \bvec F(\bvec x(t)) \vdot \bvec v(t) \dd{t}
\end{equation}


从定义上来说, $Q$ 是一个过程量, 但如果在某个系统中它可以表示为某个状态量的增量, 那么\textbf{对这个系统}区分 $Q$ 是状态量和过程量将没有太大实用价值(trivial): 任何状态量在不同时间的差都能看作一个这样的过程量, 反之这个过程量只要固定了起点也能变为一个状态量. 此时\autoref{StaPro_eq1} 积分的结果不取决于路径, 只取决于初末状态. 把该状态量记为 $V(\bvec x)$, 那么总有
\begin{equation}
Q_{12} = V(\bvec x(t_2)) - V(\bvec x(t_1))
\end{equation}
例如在二维或三维状态空间, 若令矢量函数为 $\bvec f(\bvec x) = \sum_i f_i(\bvec x) \uvec x_i$, 那么当旋度 $\curl \bvec f = \bvec 0$ 时, $\bvec f(\bvec x)$ 就是一个保守场, 必存在势函数 $V(\bvec x)$. 对于高维情况, 需要使用外导数\upref{ExtDer} 来判断保守场.

但事实上远非所有情况下\autoref{StaPro_eq1} 的积分都可以表示为两个状态量之差. 此时积分的结果必须取决于路径的形状, 那么区分状态量和过程量就至关重要. 例如, 虽然我们往往写出微分关系
\begin{equation}
\dd Q = \sum_i f_i(x_1, \dots, x_N) \dd{x_i}
\end{equation}
但是却不可能把 $Q$ 表示为 $x_i$ 的函数, $f_i$ 也不能看作偏导 $\pdv*{Q}{x_i}$.

为了防止这种误解, 一些教材中把过程量的微分\footnote{严格来说应该叫做微小增量, 因为只有函数可以做全微分\upref{TDiff}}记为 $\delta Q$ 而不是 $\dd Q$.

\begin{example}{力场}\label{StaPro_ex1}
一个具体的例子是力场对单个质点的做功. 在分析力学中, 此时状态空间是 $(\bvec x, \bvec p)$ 即位置和动量, 做功的微分为
\begin{equation}
\dd{W} = \sum_i F_i(\bvec x) \dd{x_i} = \bvec F(\bvec x) \vdot \dd{\bvec x}
\end{equation}
如果力场 $\bvec F(\bvec x)$ 是保守场, 那么做功就是势能之差; 如果是非保守场, 做功只能由具体路径决定, 此时 “功”(过程量) 和 “能”(状态量) 的区分就很重要了. 例如动能总可以表示为状态 $\bvec p$ 的函数, 但做功却不行, 因为它不是状态量.
\end{example}

\begin{example}{热力学第一定律}
另一个例子是热力学第一定律\upref{Th1Law}往往记为
\begin{equation}
\dd{Q} = P\dd{V} + \dd{E}
\end{equation}
或者
\begin{equation}
Q_{12} = \int_1^2 P\dd{V} + \Delta E
\end{equation}
但状态空间中的环积分并不总是为零, 例如著名的卡诺热机\upref{Carnot}, 即积分取决于路径. 所以 $Q$ 不能看作 $V, E$ 的函数, 也不能记
\begin{equation}
\qty(\pdv{Q}{V})_E = P \qquad \qty(\pdv{Q}{E})_V = 1 \qquad \text{(错)}
\end{equation}
\end{example}
