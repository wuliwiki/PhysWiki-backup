% 电介质的简单模型

\pentry{电偶极子\upref{eleDpl}}

先来介绍一个简单的模型: 假设\textbf{电介质(dielectric)}是一种完全不导电的物质, 其中正电荷密度(假设非常大)分布为 $\rho(\bvec r)$. 若电介质中处处无电场, 那么负电荷密度为 $-\rho(\bvec r)$, 电解质呈电中性; 若某点附近有电场, 该点处负电荷不动, 正电荷向电场方向移动一个非常小的距离 $\bvec d$. 那么我们就把介质中某一点的\textbf{极化强度(polarization density 或 polarization)}定义为
\begin{equation}
\bvec P = \rho \bvec d
\end{equation}
若取一个体积为 $V$ 的体积元, 把连续电荷分布看成是由许多点电荷构成的, 令其中的正点电荷为 $q_i$, 每个正电荷对应一个负电荷 $-q_i$, 相对位移为 $\bvec d$, 则极化强度可用电偶极矩\autoref{eleDpl_eq1}~\upref{eleDpl}表示为
\begin{equation}
\bvec P = \lim_{V\to 0}\sum_i \frac{q_i}{V} \bvec d = \lim_{V\to 0}\frac{1}{V} \sum_i \bvec p_i
\end{equation}
所以极化密度也可以理解为电偶极矩的体密度.

\subsection{极化密度与电场}
若电解质中某点处电场为 $\bvec E$, 如果该介质是 (a) 各向同性 (b) 均匀 (c) 线性的, 该点处的极化强度与电场的关系就满足
\begin{equation}
\bvec P = \chi \epsilon_0 \bvec E
\end{equation}
其中常数 $\chi$ 叫做\textbf{电极化率(susceptibility)}, $\epsilon_0$ 是真空中的电容率.
