% 伊西多·拉比(综述)
% license CCBYSA3
% type Wiki

本文根据 CC-BY-SA 协议转载翻译自维基百科 \href{https://en.wikipedia.org/wiki/Isidor_Isaac_Rabi}{相关文章}。

\begin{figure}[ht]
\centering
\includegraphics[width=6cm]{./figures/6a4d8eb51e8bbdb5.png}
\caption{} \label{fig_YXDlb_1}
\end{figure}
以色列·“伊西多”·艾萨克·拉比(Israel "Isidor" Isaac Rabi,/ˈrɑːbi/;意第绪语:איזידאָר יצחק ראַבי,转写:Izidor Yitzkhok Rabi\(^\text{[2]}\);1898年7月29日-1988年1月11日)是一位美国核物理学家,因“发明用于记录原子核磁性特性的共振方法”而获得1944年诺贝尔物理学奖。他也是美国最早研究腔体磁控管的科学家之一,该装置广泛用于微波雷达和微波炉中。

拉比出生于加利西亚赖马努夫的一个传统波兰犹太家庭,婴儿时期随家人移民至美国,在纽约下东区长大。1916年,他以电气工程专业身份进入康奈尔大学学习,但不久后转向化学,后来又对物理学产生了兴趣。他在哥伦比亚大学继续深造,并因研究某些晶体的磁化率而获得博士学位。1927年,他前往欧洲,与当时许多顶尖物理学家会面并共事。

1929年,拉比返回美国,哥伦比亚大学为他提供了教职。与格雷戈里·布赖特合作时,他发展出了布赖特–拉比方程,并预测斯特恩–盖拉赫实验可以通过改进来验证原子核的某些特性。他利用核磁共振技术测定原子的磁矩和核自旋的研究,使他获得了1944年诺贝尔物理学奖。核磁共振技术随后成为核物理和化学中的重要工具,并进一步发展出磁共振成像(MRI)技术,使其在医学领域也具有重要意义。二战期间,拉比在麻省理工学院的辐射实验室从事雷达研究,同时也参与了曼哈顿计划。战后,他担任美国原子能委员会下属的一般顾问委员会(GAC)成员,并于1952年至1956年间出任委员会主席。他还参与了国防动员办公室和陆军弹道研究实验室的科学顾问委员会,并曾担任总统德怀特·艾森豪威尔的科学顾问。

拉比参与创建了布鲁克海文国家实验室,并于1946年推动其成立。作为美国教科文组织代表,他还参与了1952年欧洲核子研究中心(CERN)的创建。1964年哥伦比亚大学设立“大学教授”职位时,拉比是首位获得此殊荣的人。1985年,该校还以他的名字命名了一个特别讲席。他于1967年退休,退出教学工作,但仍积极参与系内事务,并一直保留“荣誉大学教授”与“特别讲席教授”头衔直至去世。
\subsection{早年经历}
以色列·艾萨克·拉比于1898年7月29日出生在奥匈帝国时期的加利西亚地区赖马努夫,该地现属波兰。他出生于一个波兰犹太正统家庭。不久之后,他的父亲戴维·拉比移民到了美国。几个月后,年幼的拉比和母亲谢因德尔也前往美国与父亲团聚,全家搬进了曼哈顿下东区一套两居室的公寓。在家中,他们讲意第绪语。拉比上学时,母亲告诉校方他的名字是“Izzy”,校方人员误以为是“Isidor”的昵称,于是将“Isidor”登记为他的正式名字。从此,这个名字便成为他的官方姓名。后来,为了应对反犹主义,他开始将自己的名字写作“Isidor Isaac Rabi”,并在职业场合以“I.I. Rabi”为人所知。对他的大多数亲友来说,包括1903年出生的妹妹格特鲁德,他通常被简称为“Rabi”。1907年,全家搬到布鲁克林的布朗斯维尔,并在那里经营一家杂货店。\(^\text{[3]}\)

童年时期,拉比对科学产生了浓厚兴趣。他常常从公共图书馆借阅科学书籍阅读,还自己动手制作收音机。他的第一篇科学论文——关于电容器设计的文章——在他还在小学时便发表在《现代电气》杂志上。\(^\text{[4][5]}\)在读到哥白尼的日心说后,他成为了一名无神论者。他对父母说:“这一切都很简单,谁还需要上帝?”\(^\text{[6]}\)作为对父母的妥协,他的成年礼在家中举行,他用意第绪语发表了一篇关于电灯如何工作的演讲。

他就读于布鲁克林的实用技术高中,于1916年毕业。\(^\text{[7]}\)同年晚些时候,他进入康奈尔大学,最初主修电气工程,但很快转向化学专业。1917年美国加入第一次世界大战后,他在康奈尔参加了学生陆军训练团。他的本科毕业论文研究了锰的氧化态。1919年6月,他获得了理学士学位。但当时犹太人在化学工业和学术界的就业机会非常有限,他没有收到任何工作录用通知。他曾短暂在莱德利实验室工作,之后做了一段时间的簿记员。\(^\text{[8]}\)
\subsection{教育经历}
1922年,拉比重返康奈尔大学攻读化学研究生学位,并开始学习物理。1923年,他遇见了亨特学院暑期课程的学生海伦·纽马克,并开始追求她。为了能在她回家时留在她身边,他转学至哥伦比亚大学继续深造,导师是阿尔伯特·威尔斯。1924年6月,拉比在纽约市立学院找到了一份兼职导师的工作。威尔斯专攻磁学,他建议拉比将博士论文题目定为钠蒸气的磁化率。这个题目并未引起拉比的兴趣,但在听完威廉·劳伦斯·布拉格在哥伦比亚关于某些晶体(被称为图顿盐,Tutton's salts)电化率的讲座后,拉比决定研究这些晶体的磁化率,威尔斯也同意指导他的研究。\(^\text{[9]}\)

测量晶体的磁共振首先要培育晶体,这是一个简单的过程,常由小学生完成。但随后需将晶体切割成各向异性的薄片,并精确测量其在磁场中的响应。晶体生长期间,拉比阅读了詹姆斯·克拉克·麦克斯韦1873年的著作《电与磁论》,从中获得了灵感,想出了一个更简单的方法:他将晶体系在玻璃纤维上,再接入一个扭力天平系统,把晶体浸入一个磁化率可调的溶液中并放置在两个磁极之间。当溶液的磁化率与晶体一致时,磁铁的开关不会影响晶体的位置。这个新方法不仅更为省力,还能获得更精确的结果。1926年7月16日,拉比将题为《晶体的主磁化率》的论文投稿至《物理评论》杂志。第二天,他与海伦结婚。这篇论文在学术界没有引起太大关注,尽管卡里亚玛尼卡姆·斯里尼瓦萨·克里希南读过并在研究中加以应用。拉比由此认识到,除了发表研究成果,还需要积极宣传自己的工作。\(^\text{[10][11]}\)

像许多其他年轻物理学家一样,拉比密切关注着欧洲正在发生的重大事件。他对斯特恩–盖拉赫实验感到震惊,这项实验使他确信量子力学的正确性。他与拉尔夫·克罗尼希、弗朗西斯·比特、马克·泽曼斯基等人一起,着手将薛定谔方程推广到对称刚体分子体系,并试图找出该类力学系统的能级状态。问题在于,他们都无法解出所得到的那个二阶偏微分方程。拉比在路德维希·施莱辛格(的《微分方程理论导论》中找到了答案。书中描述了一种由卡尔·古斯塔夫·雅可比最早发展出来的方法。该方程具有超几何方程的形式,而雅可比曾找到过其解。克罗尼希与拉比将他们的研究结果整理成文,并投稿至《物理评论》杂志,该论文于1927年发表。\(^\text{[12][13]}\)
\subsection{欧洲}
1927年5月,拉比获得了巴纳德奖学金的任命。这份奖学金为期从1927年9月至1928年6月,资助金额为1500美元(相当于2024年的约27000美元\(^\text{[14]}\))。他立即向纽约城市学院申请为期一年的休假前往欧洲深造。但因申请被拒,他选择辞职。拉比抵达苏黎世时,本希望能为埃尔温·薛定谔工作,却遇到了两位美国同乡——朱利叶斯·亚当斯·斯特拉顿和莱纳斯·鲍林。他们得知薛定谔即将离开,因他已被任命为柏林弗里德里希·威廉大学理论研究所的所长。于是拉比转而前往慕尼黑大学,寻求阿诺德·索末菲尔德的接纳。到达慕尼黑后,他遇到了另外两位美国人:霍华德·珀西·罗伯逊和爱德华·康登。索末菲尔德接纳拉比为博士后研究员。当时德国物理学家鲁道夫·佩耶尔斯和汉斯·贝特也在索末菲尔德门下,但三位美国人之间关系尤为亲密。\(^\text{[15]}\)

根据导师威尔斯的建议,拉比前往利兹参加了第97届英国科学促进会年会,在会上听取了维尔纳·海森堡关于量子力学的报告。会后,拉比转往哥本哈根,自愿为尼尔斯·玻尔工作。玻尔当时正在度假,拉比便着手研究分子氢的磁化率计算。玻尔于10月归来后,安排拉比与西田吉男一道,继续在汉堡大学与沃尔夫冈·泡利合作进行研究。\(^\text{[16]}\)
