% Ubuntu 笔记

\subsection{快捷键}
\begin{itemize}
\item 鼠标中键可以粘贴刚才选中的内容
\end{itemize}


\subsection{改主机名}
\begin{itemize}
\item \verb|hostname 新主机名| 可以临时改变 hostname
\item \verb|hostnamectl set-hostname 新主机名| 可以永久改变 hostname
\end{itemize}


\subsection{改键位}\label{Ubuntu_sub1}
注意 Ubuntu (20.04 亲测, 参考\href{https://manpages.ubuntu.com/manpages/focal/en/man5/keyboard.5.html}{这里})设置 xkb config 文件无效的! 需要在 \verb|/etc/default/keyboard| 里面设置(同样重新 login 以后生效)。 例如添加一行 \verb|XKBOPTIONS=ctrl:swapcaps| 对调 Ctrl 和 Capslock 按键。

\subsection{系统版本升级}
\begin{itemize}
\item 参考\href{https://www.cyberciti.biz/faq/upgrade-ubuntu-18-04-to-20-04-lts-using-command-line/#Make_a_backup}{这里}。
\item \verb|sudo apt update && sudo apt upgrade|
\item \verb|sudo reboot|
\item \verb|sudo apt install update-manager-core|
\item \verb|sudo do-release-upgrade|
\item \verb|sudo reboot|
\end{itemize}

\subsection{取消 Alt+F5 快捷键}
\begin{itemize}
\item \verb|gsettings set org.gnome.desktop.wm.keybindings unmaximize "['<Super>Down']"|
\item 参考\href{https://askubuntu.com/questions/1322105/cant-find-alt-f5-in-settings-keyboard-shortcuts}{这里}。
\end{itemize}

\subsection{Xorg / Wayland}
用 \verb|echo $XDG_SESSION_TYPE| 可以判断显示服务用的是 xorg 还是 wayland。 新版本 ubuntu 会默认用后者, 但远程桌面软件可能无法正常工作。 可以在 login 的时候在右下角的小齿轮中选择 xrog。

\subsection{自定义屏幕分辨率}
\begin{itemize}
\item 在 22.04 下, 例如要定义 1368x912 的分辨率, 先用 \verb|cvt 1368 912|, 输出第二行为 \verb|Modeline "1368x912_60.00"  103.00  1368 1448 1592 1816  912 915 925 947 -hsync +vsync|
\item 执行 \verb|xrandr --newmode "1368x912"  103.00  1368 1448 1592 1816  912 915 925 947 -hsync +vsync|, 后面的数据是复制过来的。 双引号中的名字可以随意。
\item 执行 \verb|xrandr|, 找到 connected 的显示器名称, 如 \verb|eDP-1 connected primary 1368x912+0+0 ...|
\item 执行 \verb|xrandr --addmode eDP-1 "1368x912"| 即可添加分辨率。
\item 现在打开 setting 中的 display 就可以看到这个分辨率的选项了。
\item 重启以后, 这些设置都会消失。 要永久生效参考\href{https://askubuntu.com/questions/1376391/how-to-permanently-save-an-xrandr-configuration-for-a-monitor-even-after-reboot}{这里}。
\item 创建设置文件 \verb|sudo vim /usr/share/X11/xorg.conf.d/10-monitor.conf|, 内容如下
\begin{lstlisting}[language=none]
Section "Monitor"
  Identifier "Monitor0"
  Modeline "1368x912_60.00" 复制过来
EndSection
Section "Screen"
  Identifier "Screen0"
  Device "eDP-1"
  Monitor "Monitor0"
  DefaultDepth 24
  SubSection "Display"
    Depth 24
    Modes "1368x912"
  EndSubSection
EndSection
\end{lstlisting}
\end{itemize}


\subsection{笔记本合上盖子不作为}
\begin{itemize}
\item \verb|sudo vim /etc/systemd/logind.conf|, 找到所有包含 \verb|LidSwitch| 的选项, 取消注释, 值改成 \verb|ignore|
\item 然后运行 \verb|systemctl restart systemd-logind.service| 生效。 注意系统会重新 login
\end{itemize}

\subsection{自定义桌面、下载等目录}
\begin{itemize}
\item 修改设置文件 \verb|~/.config/user-dirs.dirs| 即可
\end{itemize}

\subsection{编辑器}
\begin{itemize}
\item (貌似没用)默认文本编辑器从 gedit 替换为 vscode: 修改 \verb`/usr/bin/gnome-text-editor` 软链指向 \verb`code` 即可
\item 第二种方法: \verb|sudo update-alternatives --install /usr/bin/editor editor /usr/bin/code 100| 然后 \verb|sudo update-alternatives --config editor| 选择 \verb|code| 即可。
\item 第三种方法: 在 \verb|~/.bashrc| 中添加一行 \verb|export EDITOR="/usr/bin/code"|。
\end{itemize}

\subsection{安装搜狗输入法}
\begin{itemize}
\item \href{https://shurufa.sogou.com/linux}{官网}说支持 Ubuntu 16.04 到 20.04 亲测 22.04 也可成功。 下载安装包。
\item \href{https://shurufa.sogou.com/linux/guide}{安装说明}, 可以直接搜索 \verb|Ubuntu20.04|。
\item 先把系统切换成中文输入法。 完了可以再切换回英文。
\item \verb|sudo apt update && sudo apt install fcitx -y|
\item 语言设置中, 语言支持, 将语言选择为 fcitx
\item fcitx 开机自启动: \verb|sudo cp /usr/share/applications/fcitx.desktop /etc/xdg/autostart/|
\item \verb|sudo apt purge ibus|
\item \verb|sudo dpkg -i 安装包名|
\item \verb|sudo apt install libqt5qml5 libqt5quick5 libqt5quickwidgets5 qml-module-qtquick2|
\item \verb|sudo apt install libgsettings-qt1|
\item 重启电脑, 右上角菜单栏就可以找到搜狗。
\item 如果系统中还有一些 UI 中残留有中文, 就在语言设置里面 Manage Installed Language, 然后按 apply system-wide。
\end{itemize}

\subsection{应用图标}
\begin{itemize}
\item Launcher 中的应用图标都是 \verb|usr/shared/applications| 中的文本设置文件, 例如。
\begin{lstlisting}[language=none]
[Desktop Entry]
Version=1.0
Type=Application
Name=CLion
Icon=/home/addis/clion-2022.3.1/bin/clion.svg
Exec="/home/addis/clion.sh"
Comment=A cross-platform IDE for C and C++
Categories=Development;IDE;
Terminal=false
StartupWMClass=jetbrains-clion
StartupNotify=true
\end{lstlisting}
\end{itemize}


\subsection{recycle}
\begin{itemize}
\item 用 \verb`apt-clone` 是最方便快速的(未测试), 或者释放出 deb 文件单独安装
\item CLion 的工具栏 Tools 可以直接创建快捷方式
\item Matlab 和 Mathematica 和 Intel oneAPI 都可以直接复制目录(亲测有效), 不过前两个需要再次注册。
\item GitHub Desktop 可以在这里安装 https://github.com/shiftkey/desktop/
\item \verb`~/.config` 文件夹(很多常用软件的设置文件都在这,可以选择性拷贝)和 \verb`~/.vim` 文件夹, 以及 \verb`~/.vimrc` 都可以拷贝
\end{itemize}
