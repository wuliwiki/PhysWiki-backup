% 幂指函数之差的极限
% license Usr
% type Art

% Date: 2024/3/13 

% 想法:打算开一个大学生数学竞赛专题,总结笔记

\begin{theorem}{}
设 $f(x)$, $g(x)$, $h(x)$ 在 $x_{0}$ 的去心邻域内取正值,
且 $a>0$. 

$(1)$ 若 ${\displaystyle \lim_{x\to x_{0}}}\,f(x)^{g(x)}={\displaystyle \lim_{x\to x_{0}}}\,f(x)^{h(x)}=a$,
则当 $x\to x_{0}$ 时, 
$$
f(x)^{g(x)}-f(x)^{h(x)}\,\sim\,a\,\big(g(x)-h(x)\big)\ln f(x).~
$$
 

$(2)$ 若 ${\displaystyle {\displaystyle \lim_{x\to x_{0}}}\,}g(x)^{f(x)}={\displaystyle \lim_{x\to x_{0}}}\,h(x)^{f(x)}=a$,
且 ${\displaystyle \lim_{x\to x_{0}}}\,{\displaystyle \frac{g(x)}{h(x)}}=1,$
则当 $x\to x_{0}$ 时, 
\[
g(x)^{f(x)}-h(x)^{f(x)}\,\sim\,a\,\big(g(x)-h(x)\big)\frac{g(x)}{h(x)}.~
\]

$(3)$ 若 $f(x)$ 与 $g(x)$ 是等价无穷小量且 ${\displaystyle {\displaystyle \lim_{x\to x_{0}}}\,}f(x)^{g(x)}={\displaystyle \lim_{x\to x_{0}}}\,g(x)^{f(x)}=a$,
则当 $x\to x_{0}$ 时, 
\[
f(x)^{g(x)}-g(x)^{f(x)}\,\sim\,a\,\big(g(x)-f(x)\big)\ln g(x)\sim\,a\,\big(g(x)-f(x)\big)\ln f(x).~
\]

\verb`注记`
若 $a=0$ 但相减的两个幂指函数是等价无穷小量, 则上面各个结论中的 $a$ 换成任一幂指函数即可, 如 $(1)$ 中的结论将变成
\[
f(x)^{g(x)}-f(x)^{h(x)}\,\sim\,f(x)^{g(x)}\,\big(g(x)-h(x)\big)\ln f(x),~
\]
其余类似. 

\end{theorem}


% %\end{rem*}
% %\begin{proof}
\textsl{Proof.} 为记号简便, 将 $f(x)$, $g(x)$, $h(x)$ 分别略写为 $f$, $g,$ $h$. 

$(1)$ 显然, $(g-h)\ln f$ 是无穷小量, 因此
\[
f^{g}-f^{h}=f^{h}\big(f^{g-h}-1\big)=f^{h}\big(\,\mathrm{e}^{(g-h)\ln f}-1\big)\sim a(g-h)\ln f.~
\]
 $(2)$ 类似地, 
\[
g^{f}-h^{f}=h^{f}\big((g/h)^{f}-1\big)=h^{f}\big(\,\mathrm{e}^{f\ln\frac{g}{h}}-1\big)\sim af\ln\frac{g}{h}\sim af\left(\frac{g}{h}-1\right)\sim a(g-h)\frac{f}{h}\sim a(g-h)\frac{g}{h}.~
\]
 $(3)$ 由于 $f\sim g$ 故 $f^{f}=\big(f^{g}\big)^{\frac{f}{g}}\to a$,
所以将 $f^{g}-g^{f}$ 看成 $(f^{g}-f^{f})+(f^{f}-g^{f})$, 可利用前面两个结论: 
\[
f^{g}-f^{f}=a(g-f)\ln f,\qquad f^{f}-g^{f}=a(f-g)\frac{f}{g}\sim a(f-g),~
\]
因为 $f$ 是无穷小量, 所以 $\ln f=\infty$, 故上面的第二式被第一式控制. 综上, $f^{g}-g^{f}\sim a(g-f)\ln f.$ $\quad\square$
% %\end{proof}
% %\begin{cor*}

利用上面的定理,立即得到下面的推论:

\begin{corollary}{}
当 $x\to0$ 时成立

$(1)$ $x^{x}-x^{\sin x}\sim1\cdot(x-\sin x)\ln x\sim\frac{x^{3}}{6}\ln x.$

$(2)$ $x^{x}-(\sin x)^{x}\sim1\cdot(x-\sin x)\frac{x}{\sin x}\sim\frac{x^{3}}{6}.$

$(3)$ $x^{\sin x}-(\sin x)^{x}\sim1\cdot(\sin x-x)\ln x\sim\,\frac{-x^{3}}{6}\ln x.$
\end{corollary}
% %\end{cor*}
% %\begin{xca*}

$$~$$ 

下面来看几个例子.

\begin{example}{}
证明:当 $x\to0$ 时, $x^{(\sin x)^{x}}-(\sin x)^{x^{\sin x}}\sim\frac{x^{3}}{6}.$ 
\end{example}
% %\end{xca*}
% %\begin{proof}
\textsl{Proof.} 令 $u(x)=(\sin x)^{x}$, $v(x)=x^{\sin x},$ 则 $x^{u(x)}-(\sin x)^{v(x)}$
看成 $\big(x^{u(x)}-x^{v(x)}\big)+\big(x^{v(x)}-(\sin x)^{v(x)}\big)$,
利用命题前面两个结论: 
\[
x^{u(x)}-x^{v(x)}\sim x^{u(x)}\big(u(x)-v(x)\big)\ln x,\qquad x^{v(x)}-(\sin x)^{v(x)}\sim x^{v(x)}(x-\sin x).~
\]
另一方面, 易知 $x^{u(x)}\sim x$ 及 $x^{v(x)}\sim x$, 再利用命题的第三个结论: 
\[
u(x)-v(x)\sim(\sin x-x)\ln x\sim\frac{-x^{3}}{6}\ln x.~
\]
因而 
\[
x^{u(x)}-x^{v(x)}\sim\frac{-x^{4}}{6}\ln^{2}x,\qquad x^{v(x)}-(\sin x)^{v(x)}\sim\frac{x^{3}}{6}.~
\]
由于 ${\displaystyle \lim_{x\to0}x^{\alpha}\ln x=0}$ $(\alpha>0)$,
故上面的第一式是第二式的高阶无穷小, 因此只剩下第二式. 综上, $$x^{u(x)}-(\sin x)^{v(x)}\sim\frac{x^{3}}{6}.~$$  证毕. $\quad\square$
% %\end{proof}

\begin{exercise}{}
求极限 $\displaystyle\lim_{x\to 0^+}\frac{x^{x^x}-(\sin x)^{(\sin x)^{(\sin x)}}}{x^3}$.
\end{exercise}