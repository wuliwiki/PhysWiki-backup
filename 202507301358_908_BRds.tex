% 布尔代数(综述)
% license CCBYNCSA3
% type Wiki

本文根据 CC-BY-SA 协议转载翻译自维基百科\href{https://en.wikipedia.org/wiki/Boolean_algebra_(structure)}{相关文章}。

在抽象代数中,*布尔代数(Boolean algebra)**或**布尔格是一个带补的分配格。这种代数结构刻画了集合运算和逻辑运算的基本性质。布尔代数可以被看作是幂集代数或集合域的推广,或者其元素可以被看作是广义真值。它也是德摩根代数和克莱尼代数的一个特殊情形。

每一个布尔代数都可以产生一个**布尔环(Boolean ring)**,反之亦然,其中环的乘法对应于**合取或交运算(conjunction or meet)** ∧,环的加法对应于**异或或对称差(exclusive disjunction or symmetric difference)**(不是析取 ∨)。然而,布尔环理论在两个运算之间本质上是不对称的,而布尔代数的公理和定理则通过\*\*对偶性原理(duality principle)\*\*体现出理论的对称性。[1]
