% 光与物质粒子的统一(相对论点粒子的作用量)
% keys 相对论|非相对论|光|作用量
% license Usr
% type Tutor

本节将从作用量的视角介绍牛顿力学到相对论的自然过渡,最终给出适用于物质粒子和光的作用量。阅读本节需带着“民主”的思想:时间和空间应当被平等对待。
\subsection{Newton力学到狭义相对论}
\subsubsection{缘起}
首先看Newton力学中物质粒子的Euler-Lagrange作用量:
\begin{equation}
S=\int\dd t\qty[\frac{1}{2}m\qty(\dv{\vec x}{t})^2-V(x)]~.
\end{equation}
这一作用量是相当“笨重”的。在这里,让我们只考虑自由粒子,把势 $V(x)$ 给丢掉。那么上式可写为
\begin{equation}
S=\int\dd t\frac{1}{2}m\qty(\dv{\vec x}{t})^2=\int\dd t\frac{1}{2}m\frac{(\dd{\vec x})^2}{\dd t}~.
\end{equation}
上式对待 $\dd{\vec x}$ 和 $\dd t$ 的方式是不平等的,$\dd{\vec x}$ 平方的出现来自于旋转不变性,但是为什么 $\dd t$ 只值得1次幂?多么奇怪的组合呀,$\frac{(\dd{\vec x})^2}{\dd t}$,一个东西的平方除以另一个东西。这冒犯了我们自由主义的思想,因此,我们需要进行改变。

\subsubsection{平方根}
让我们回想一下学习平方根的概念。我们知道,25的平方根是5,36的平方根是6,等等。但是,对一个不是整数的平方的数,它的平方根是什么?比如24。使用古老的试错方法:计算 $4.9^2,4.8^2,\ldots.$ 


