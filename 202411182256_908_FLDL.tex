% 法拉第电磁感应定律(综述)
% license CCBYSA3
% type Wiki

本文根据 CC-BY-SA 协议转载翻译自维基百科\href{https://en.wikipedia.org/wiki/Faraday\%27s_law_of_induction}{相关文章}。

\begin{figure}[ht]
\centering
\includegraphics[width=8cm]{./figures/0ad787ae5c515839.png}
\caption{法拉第的实验展示了线圈之间的感应现象:右侧的液体电池提供电流,该电流流经小线圈(A),从而产生一个磁场。当线圈静止时,大线圈中没有感应电流产生。但当小线圈在大线圈中移动或移出时(B),大线圈中的磁通量发生变化,从而在大线圈中感应出电流,这一电流通过检流计(G)检测到。[1]} \label{fig_FLDL_1}
\end{figure}
法拉第电磁感应定律(简称法拉第定律)是电磁学中的一条定律,用于预测磁场如何与电路相互作用以产生电动势(emf)。这种现象被称为电磁感应,是变压器、电感器以及许多类型的电动机、发电机和螺线管的基本工作原理。[2][3]

麦克斯韦-法拉第方程(作为麦克斯韦方程组之一)描述了一个事实,即空间变化的(也可能是时间变化的,具体取决于磁场随时间的变化情况)电场总是伴随着时间变化的磁场,而法拉第定律则表明,当通过由导电回路包围的表面的磁通量随时间变化时,导电回路中会产生电动势(即单位电荷沿回路运动一圈时电磁作用所做的功)。

法拉第定律被发现后,其一个方面(变压器电动势)被表述为麦克斯韦-法拉第方程。法拉第定律的方程可以通过麦克斯韦-法拉第方程(描述变压器电动势)和洛伦兹力(描述运动电动势)推导而来。麦克斯韦-法拉第方程的积分形式仅描述变压器电动势,而法拉第定律的方程同时描述变压器电动势和运动电动势。
\subsection{历史}  
电磁感应现象分别由迈克尔·法拉第于1831年和约瑟夫·亨利于1832年独立发现。[4] 法拉第是第一个发表其实验结果的人。[5][6]
\begin{figure}[ht]
\centering
\includegraphics[width=8cm]{./figures/40394efd87bbeabe.png}
\caption{法拉第1831年的演示[7]} \label{fig_FLDL_2}
\end{figure}