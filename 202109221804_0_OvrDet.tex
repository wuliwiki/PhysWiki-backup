% 超定线性方程组
% 线性代数|矩阵|复数矩阵|线性方程组|超定方程组|线性相关|线性无关

\pentry{最小二乘法\upref{LstSqr}, 线性方程组与矢量空间\upref{LinEq}}

令 $A$ 为 $M\times N$ 的复数矩阵, $\bvec x$ 和 $\bvec y$ 为复数列矢量, 当 $M > N$ 时, 以下线性方程组称为\textbf{超定方程组}(只有 $\bvec x$ 是未知)
\begin{equation}\label{OvrDet_eq1}
\bvec y = \mat A \bvec x
\end{equation}

我们把 $\mat y$ 和 $\mat A$ 拼接成一个 $M\times(N+1)$ 的矩阵, 当这个矩阵的 $M$ 个行矢量中只有小于或等于 $N$ 个线性无关时, 我们只需取所有线性无关的行即可得到非超定的线性方程组.举一个简单的例子,如果第 2 条方程(第 2 行)是第 1 条方程(第 1 行)乘以常数,那么这两条方程中我们只需保留一条即可.

如果有大于 $N$ 个线性无关的行(由于每行只有 $N+1$ 个元,那么最多只可能有 $N+1$ 个线性无关的行), 那么超定方程无解.为什么?


但我们仍然可以寻找一个最优的 $\bvec x$, 使以下误差函数取最小值
\begin{equation}
\abs{\mat A\bvec x - \bvec y}^2 =  \sum_k  \qty(\sum_j A_{kj} x_j - y_k) \qty(\sum_j A_{kj} x_j - y_k)^*
\end{equation}

所以这是一个最小二乘法问题. 令误差函数分别对每个 $\Re[x_i]$ 和 $\Im[x_i]$ 求导等于 0, 得
\begin{equation}
\sum_j \qty(\sum_i A\Her_{ik} A_{kj}) x_j = \sum_k A\Her_{ik} y_k
\end{equation}
即
\begin{equation}\label{OvrDet_eq4}
\mat A\Her \mat A \bvec x = \mat A\Her \bvec y
\end{equation}
该方程只有一个解, 也就是最小二乘法的解. 这是因为系数行列式必大于零:
\begin{equation}
\mat A\Her \mat A
\end{equation}

% 到底会不会有多个解或者无解呢?

对比\autoref{OvrDet_eq1} 可以发现\autoref{OvrDet_eq4} 只是在左右两侧同时乘以 $\mat A$ 的厄米共轭. 所以任何能满足\autoref{OvrDet_eq1} 的解也可以通过\autoref{OvrDet_eq4} 解得.
