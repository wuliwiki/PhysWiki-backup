% ZFS 笔记

\begin{issues}
\issueDraft
\end{issues}

\begin{itemize}
\item Ubuntu/Debian 安装 zfs: \verb|sudo apt install zfsutils-linux|
\item 创建一个 500MB 文件: \verb|dd if=/dev/zero of=虚拟硬盘文件 bs=1M count=500|
\item 制作 loopback 设备 \verb|sudo losetup -fP --show 虚拟硬盘文件|。会返回设备路径如 \verb|/dev/loop0|
\item \verb|sudo zpool create pool名 /dev/loop0|
\item 检查状态: \verb|sudo zpool status|; \verb|sudo zfs list|
\item 创建文件系统: \verb|sudo zfs create pool名/fs名|
\item 创建测试文件: \verb|echo "This is a test file." > /pool名/fs名/文件|
\item 创建文件系统镜像: \verb|sudo zfs snapshot pool名/fs名@镜像1|
\item 改变测试文件: \verb|echo "This is some additional data." >> /pool名/fs名/文件|
\item 对比当前文件系统和上一次镜像 \verb|sudo zfs diff pool名/fs名@镜像1 pool名/fs名|。 输出如 \verb|M	/pool名/fs名/文件|。 其中 \verb|M| 和 git 中一样表示 modified。
\item 恢复镜像: \verb|sudo zfs rollback pool名/fs名@镜像1|。现在可以检查 \verb|文件| 恢复到了之前的版本。
\item 添加另一个 loop 设备: \verb|dd if=/dev/zero of=虚拟硬盘文件2 bs=1M count=500|
\item \verb|sudo losetup -fP --show 虚拟硬盘文件2|, 输出例如 \verb|loop1|
\item 添加设备到已有的 pool 中: \verb|sudo zpool add pool名 /dev/loop1|
\item 所以 zfs 不需要再给硬盘分区,而是把硬盘加入 pool, 再在 pool 中创建文件系统。 每个文件系统有自己的属性,有自己的镜像,共享整个 pool 的容量。 当然,作为高级功能, zfs 也允许不把整块硬盘加入 pool 而是把一个分区加入 pool。
\item 添加设备后, pool 中的所有文件系统会自动按需使用新增到 pool 中的设备,无需手动 resize。
\item 用完以后,删除 pool: \verb|sudo zpool destroy pool名|, \verb|sudo losetup -d /dev/loop0|
\end{itemize}

\subsection{RAID}
\begin{itemize}
\item 设置 RAID-Z1: \verb|sudo zpool create pool名 raidz1 /dev/sdb /dev/sdc ...|。 这会自动挂载到 \verb|/pool名| 目录。
\item 创建文件系统: \verb|sudo zfs create pool名/fs名|。
\item \verb|sudo zpool status yue| 可以查看各种状态以及 RAID 设置, 以及读写错误。
\item 改变挂载目录: \verb|sudo zfs set mountpoint=/挂载目录 pool名/fs名|。
\item \verb|sudo zfs unmount pool名| 弹出 pool, 用 \verb|zfs mount| 重新挂载。
\item \verb|sudo zfs set compression=lz4 pool名/fs名| 开启 6 级 gzip 压缩(cpu 会间断性占用 70\% 左右, 8 核 16 线程)。 可以随时打开或者关闭,只对新数据生效。
\item \verb|sudo zfs set dedup=on pool名/fs名| 开启查重(重复文件可以节约空间)。 可以随时打开或者关闭,只对新数据生效。
\item 把 \verb|set| 改成 \verb|inherit| 就相当于关闭这些选项。
\item \verb|sudo zfs get all pool名/fs名| 可以列出所有当前选项。 把 \verb|all| 换成 \verb|选项名| 可以列出制定选项。
\item \verb|compressratio| 属性可以显示当前的平均压缩率, 这只用于查看不能手动设置。
\item 的确,以上设置需要 \verb|1GB/1TB| 的内存。
\end{itemize}
