% 时空的几何
% keys 类时|本征时|时空|光锥|类空|类光
% license Usr
% type Tutor

学习狭义相对论的“现代”方式时强调时空的几何,这种方法将引领我们自然的到达广义相对论和Einstein的引力。本部分以一种更为严格的形式展现狭义相对论的几何。

\subsection{基本定义}
相对论认为,任一事件都由它发生的时间和地点确定,因此描述事件的所在“空间”(数学概念,可视为拓扑空间的空间)称为时空。
\begin{definition}{时空,事件,距离}
若四维矢量空间 $\mathbb R^4$ 上定义了如下的距离函数
\begin{equation}
s^2(A,B)=\eta_{\alpha\beta}\Delta x^\alpha\Delta x^\beta,\quad \forall A,B\in\mathbb R^4,~
\end{equation}
其中,$\Delta x=x_B-x_A$,$\eta=\mathrm{diag}[-1,1,1,1]$ 是对角化矩阵。则称该矢量空间为4维\textbf{时空},其上的点称为\textbf{事件}(event),$s(A,B)=\sqrt{s^2(A,B)}$ 称为 $A,B$ 的\textbf{间隔}(separation),$\eta$ 称为\textbf{度规}(matric)。规定坐标从0标记,即 $\alpha=0,1,2,3$,并称坐标 $x^0$ 称为\textbf{时间(time)坐标},可记为 $t$, $(x^1,x^2,x^3)$ 称为\textbf{空间坐标}(space),可记作 $\bvec r$。
\end{definition}

注意到度规 $\eta$ 是一个不定型(\autoref{def_DeQua_2}),因此 $s^2(A,B)$ 有三种可能,即 $s^2(A,B)>0,s^2(A,B)=0,s^2(A,B)<0$。

\begin{definition}{类时,类空,类光}
设 $A,B$,是时空中的两个事件,若
\end{definition}





















