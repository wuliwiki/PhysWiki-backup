% 常微分方程的几何图像
% 向量场|单参数微分同胚群|微分方程|解

\pentry{相空间和相流\upref{PSaPF},}
这里,将从几何上来理解常微分方程.这需要先引入一些概念.
\subsection{微分同胚}
\begin{definition}{可微函数}
设 $U$ 是矢量空间 $\mathbb R^n$ 上的区域,其上坐标为 $x_1,\cdots,x_n$,称函数
\begin{equation}
f:U\rightarrow\mathbb R
\end{equation}
是 $U$ 上的\textbf{可微函数},若 $f(x_1,\cdots,x_n)$ 是 $r$ 次连续可微的,此处,$1\leq r\leq\infty$.通常人们都不关心 $r$ 的具体值,因此并不指明.若有需要将指出“$r$次可微”或函数类 $C^r$.
\end{definition}
\begin{definition}{可微映射}
设 $U$ 是 $\mathbb R^n$ 中区域,$V$ 是 $\mathbb R^m$ 中区域,其中 $x_1,\cdots,x_n$ 是 $U$ 中的坐标,$y_1,\cdots,y_m$ 是 $V$ 中的坐标,称映射
\begin{equation}
f:U\rightarrow V,\quad f(x_1,\cdots,x_n)=(y_1,\cdots,y_m)
\end{equation}
为\textbf{可谓映射},若 $y_i=f_i(x_1,\cdots,x_n)$ 是可微函数.其中 $1\leq i\leq m$.
\end{definition}
\begin{definition}{微分同胚}
若映射 $f:U\rightarrow V$ 是个双射,且 $f$ 和其逆 $f^{-1}$ 都是可微映射,则称 $f$ 为\textbf{微分同胚}.
\end{definition}