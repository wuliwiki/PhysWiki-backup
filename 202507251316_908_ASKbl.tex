% 艾萨克·巴罗(综述)
% license CCBYSA3
% type Wiki

本文根据 CC-BY-SA 协议转载翻译自维基百科\href{https://en.wikipedia.org/wiki/Isaac_Barrow}{相关文章}。

\begin{figure}[ht]
\centering
\includegraphics[width=6cm]{./figures/3f0be7137842d59d.png}
\caption{} \label{fig_ASKbl_1}
\end{figure}
艾萨克·巴罗(Isaac Barrow,1630年10月-1677年5月4日)是一位英国基督教神学家和数学家,通常被认为在微积分早期的发展中扮演了重要角色,特别是在证明微积分基本定理方面作出了贡献。^\([1] 他的研究主要集中于切线的性质;巴罗是第一个计算出卡帕曲线切线的人。他还因担任著名的卢卡斯数学教授讲席第一任教授而备受瞩目,该职位后来由他的学生艾萨克·牛顿继承。
\subsection{生平}
\subsubsection{早年生活与教育}
\begin{figure}[ht]
\centering
\includegraphics[width=6cm]{./figures/37d75e02495b220e.png}
\caption{《剑桥大学公开讲座讲义》(Lectiones habitae in scholiis publicis academiae Cantabrigiensis AD 1664)} \label{fig_ASKbl_2}
\end{figure}
巴罗出生于伦敦。他的父亲托马斯·巴罗是一名亚麻布商人。1624年,托马斯娶了来自肯特郡北克雷的威廉·巴金之女安,他们的儿子艾萨克出生于1630年。看起来巴罗是这段婚姻中唯一的孩子——至少是唯一存活至幼年之后的孩子。安约在1634年去世,守寡的父亲于是将年幼的艾萨克送往他祖父、剑桥郡治安官艾萨克处抚养,祖父当时住在斯宾尼修道院【2】。然而不到两年,托马斯便再婚,新妻是来自肯特郡梅德金的亨利·奥克辛登之妹凯瑟琳·奥克辛登。这段婚姻至少育有一女伊丽莎白(生于1641年)和一子托马斯。小托马斯曾拜皮革商爱德华·米勒为师,于1647年完成学徒任务,1680年移民至巴巴多斯【3】。
\subsubsection{早期生涯}
艾萨克最初就读于查特豪斯公学,在那里他性格顽劣、好斗,甚至父亲曾被人听到祈祷说,如果上帝要带走他的一位孩子,那么他最愿意牺牲的就是艾萨克。之后他转学至费尔斯特学校,在那里他终于安定下来,并在杰出的清教徒校长马丁·霍尔比奇指导下学习。霍尔比奇在十年前曾教过约翰·沃利斯【4】。在费尔斯特学校,他学习了希腊语、希伯来语、拉丁语和逻辑,为进入大学做准备【5】。随后他进入剑桥大学三一学院继续深造;他选择三一学院的原因是沃波尔家族某位成员向他提供了资助,“这项资助可能出于沃波尔家族对巴罗忠于王党事业的同情”【6】。他的叔叔、同名的艾萨克·巴罗,后来成为圣阿萨夫主教,是彼得学院的院士。

巴罗刻苦学习,在古典学和数学方面表现出色;1648年取得学位后,他于1649年当选为三一学院的院士【7】。1652年,他在导师詹姆斯·杜波特指导下获得剑桥大学文学硕士学位(MA),随后在学院中居住了几年,并曾竞选希腊语教授职位。但在1655年,由于拒绝签署效忠英格兰共和国的《誓言》,他获得了一笔旅行资助,开始出国游历【8】。
