% 北京科技大学 2006 年考研普通物理 A 卷
% keys 北京科技大学|考研|物理
% license Copy
% type Tutor
适用专业:凝聚态物理、物理电子学
\begin{itemize}
\item 选择题
\end{itemize}
\begin{enumerate}
\item 质点作半径为R的变速圆周运动的加速度大小为(v表示任一时刻质点的速率)\\
(A) $\displaystyle \dv{v}{t}$\\
(B) $\displaystyle \frac{v^2}{R}$\\
(C) $\displaystyle \dv{v}{t}+\frac{v^2}{R}$\\
(D) $\displaystyle [(\dv{v}{t})+\frac{v^4}{R^2}]^\frac{1}{2}$
\item 一质点在如图1所示的坐标平面内作圆周运动,有一力$\vec F=F_0(x\vec i+y \vec j)$作用在质点上,在该质点从坐标原点运动到$(0,2R)$位置过程中,此力$\vec F$下对它作的功为:\\
(A) $F_0R^2S$\\
(B) $2F_0R^2$\\
(C) $3F_0R^2$\\
(D) $4F_0R^2$
\item 若一倾角为$\theta$ 的斜面上放一质量为m的物体,m与斜面间的摩擦系数为 $\mu$,斜面向左加速运动,欲使m沿斜面向上滑动,则斜面的加速度值至少应为:\\
(A) $\displaystyle \frac{(\mu \cos \theta+\sin \theta)g}{\cos 2\theta -\mu \sin \theta}$\\
(B) $\displaystyle\frac{(\mu \sin \theta+\cos \theta)g}{\cos 2\theta -\mu \sin \theta}$\\
(C) $\displaystyle \frac{(\mu \cos \theta+\sin \theta)g}{\sin \theta -\mu \cos\theta}$\\
(D)$\displaystyle \frac{(\mu \cos \theta+\sin \theta)g}{\cos \theta -\mu \sin\theta}$
\item 如图所示,两个同心球壳。内球半径为 $R_1$ ,均带有电荷 $Q$ ;外求壳半径为 $R_2$ ,壳的厚度忽略,原先不带电,但与地相连接。设地为电势零点,则在两球之间,距离球心为 $r$ 的 $P$ 点处电场强度的大小与电势分别为:\\
(A) $\displaystyle E=\frac{Q}{4\pi\varepsilon_0 r^2},U=\frac{Q}{4\pi\varepsilon_0 r}$\\
(B) $\displaystyle E=\frac{Q}{4\pi\varepsilon_0 r^2},U=\frac{Q}{4\pi\varepsilon_0} (\frac{1}{R_1}-\frac{1}{r})$\\
(C) $\displaystyle E=\frac{Q}{4\pi\varepsilon_0 r^2},U=\frac{Q}{4\pi\varepsilon_0} (\frac{1}{r}-\frac{1}{R_2}) $\\
(D)$\displaystyle E=0,\qquad U=\frac{Q}{4 \pi \varepsilon_0 R_2}$
\item 无线长直导线在 $P$ 处弯成半径为 $R$ 的圆,当通以电流 $I$ 时,则在圆心  $O$ 点的磁场感应强度大小等于\\
(A) $\displaystyle \frac{\mu_0 I}{2 \pi R}$\\
(B) $\displaystyle \frac{\mu_0 I}{4R}$\\
(C) $0$\\
(D) $\displaystyle \frac{\mu_0 I}{2R}(1-\frac{1}{\pi})$
\item 已知一定量的某种理想气体,在温度为 $T_1$ 和 $T_2$ 时的分子最概然速率分别为 $V_{p1}$ 和 $V_{p2}$ ,分子速率分布的最大值分别为 $f(V_{p1})$ 和 $f(V_{p2})$ 。若 $T_1>T_2$ ,则\\
(A) $V_{p1}>V_{p2},f(V_{p1}>f(V_{p2})$\\
(B)$V_{p1}>V_{p2},f(V_{p1}<f(V_{p2})$\\
(C)$V_{p1}<V_{p2},f(V_{p1}>f(V_{p2})$\\
(D)$V_{p1}<V_{p2},f(V_{p1}<f(V_{p2})$\\
\item 一质量为$M$的物体沿$x$轴止向运动,假设该质点在通过坐标为$x$的位置时速度的大小为$kx$($k$为正值常量),则此时作用于该质点上力$F=()$,该质点从$x=x_0$点出发运动到$x=x_1$处所经历的时间$\Delta t=()$\\
\item 把一个均匀带有电荷$+Q$的球形肥皂泡由半径$r_1$吹胀到$r_2$,则半径为$R(r_1<R<r_2)$的任一点的场强大小$E$由$(\qquad)$变为$(\qquad)$;电势$U$由$(\qquad)$。(选无穷远处为电势零点)。
\item 如图所示,一质量为$m$的滑块,两边分别与劲度系数为$kl$和$k2$的轻弹簧相连,两弹簧的另外两端分别规定在墙上。滑块$m$可在光滑的水平面上滑动,$0$点为系统的平衡位置,将滑块向右移动到$x0$,自静止释放,并从释放时开始计时。取坐标如图所示,则其振动方程为:$(\qquad)$
\item 在弦线上有一简谐波,其表达式为:$y_1=$
\end{enumerate}
