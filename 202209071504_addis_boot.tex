% bootloader 与多系统笔记

\begin{issues}
\issueDraft
\end{issues}

\begin{itemize}
\item 开机的时候,BIOS/EUFI会先启动bootloader然后再由bootloader加载系统(也就是按F2或Del以后进入的那个界面选择的其实是bootloader)
\item windows的bootloader自动加载windows,而grub会搜索每个硬盘和分区,并列出所有选项
\item 在GPT中,bootloader总是会在EFI分区,而在MBR中,bootloader在masterrecord中
\item \verb|sudo grub-install /dev/sd#| 可以在某个硬盘中安装 grub
\item \verb|df| 只会显示已挂载的硬盘, 而 \verb|fdisk| 或者 \verb|disk, gparted| 软件才会显示所有连接的硬盘
\item 要安装 ubuntu live 镜像到 disk, 用 \verb|sudo dd bs=4M if=/path/to/ISOfile of=/dev/sda status=progress oflag=sync| 注意 \verb|sda| 后面不能有数字! 这样 \verb|dd| 会把整个硬盘克隆成 iso 的内容而无视之前的任何 partition table 和分区. 此时 u 盘和光盘完全等效,都是只读的(亲测成功).
\end{itemize}

\begin{itemize}
\item grub 中按 c 进入命令行模式, 然后 ls 可以看见有哪些硬盘和分区, 例如 (hd0,msdos1)
\end{itemize}

