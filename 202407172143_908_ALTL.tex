% 阿兰·图灵
% license CCBYSA3
% type Wiki

(本文根据 CC-BY-SA 协议转载自原搜狗科学百科对英文维基百科的翻译)

艾伦·麦席森·图灵·奥博·弗莱斯(英语:Alan Mathison Turing,/ˈtjʊərɪŋ/,又译阿兰·图灵,Turing也常翻译成涂林或者杜林,1912年6月23日——1954年6月7日),英国数学家、计算机科学家、逻辑学家、密码分析学家、哲学家和理论生物学家。图灵在理论计算机科学的发展中具有很大的影响力,他为算法和计算的概念提供了一个形式化的图灵机,图灵机可以被认为是一个通用计算机的模型,为现代计算机的逻辑工作方式奠定了基础,[1]同时也被称为理论计算机科学和人工智能之父。[2]

第二次世界大战期间,图灵在破解截获的编码信息方面发挥了关键作用,这些信息使盟军能够在许多关键战役中击败纳粹,包括大西洋战役。[3][4]从长远意义上讲,据估计这项工作缩短了欧洲战争两年多,挽救了1400多万人的生命。[3]1952年因同性恋行为被起诉,拉博彻修正案规定“严重猥亵”在英国是刑事犯罪。他接受了化学阉割治疗,用DES替代监狱。1954年,图灵死于氰化物中毒,享年41岁。[5]2013年12月24日,在英国司法大臣克里斯·格雷灵的要求下,英国女王伊丽莎白二世向图灵颁发了皇家赦免。

\subsection{ 早期生活}
\subsubsection{1.1 家庭}
图灵出生在伦敦迈达谷,[6]而他的父亲朱利叶斯·马西森·图灵(1873-1947年)则离开了他在查特拉布尔(Chatrapur)的印度公务员队伍,当时是马德拉斯(Madras)总统任期,目前在印度奥迪沙邦( Odisha)。[6][7]图灵的父亲是牧师约翰·罗伯特·图灵牧师的儿子,他来自一个位于荷兰的苏格兰商人家庭,其中包括一名准男爵。图灵的母亲,朱利叶斯(Julius)的妻子,是埃塞尔·萨拉·图灵(内·斯通尼,1881-1976),[6]爱马德拉斯铁路公司(Madras Railways)总工程师爱德华·沃勒·斯通尼的女儿。斯通尼一家是蒂珀雷里县和龙福德县的新教盎格鲁爱尔兰贵族家庭,而埃塞尔本人童年大部分时间都在克莱尔县(County Clare)度过。[8]

朱利叶斯与ICS的合作将这个家庭带到了英属印度,他的祖父曾在那里的孟加拉军队中担任将军。然而,朱利叶斯和埃塞尔都想让他们的孩子在英国长大,所以他们搬到了伦敦的迈达山谷,[9]艾伦·图灵于1912年6月23日出生在那里,正如他出生的房子(后来的Colonnade酒店)外面的蓝色匾所记录的那样[10][11] [6][12]。图灵有一个哥哥约翰(约翰·德莫特·图灵爵士的父亲,图灵男爵的第12个男爵)。[13]

图灵父亲的公务员委员会仍然很活跃,在图灵童年时期,图灵的父母在英国黑斯廷斯( Hastings)和印度之间旅行,[14]留下他们的两个儿子和一对退休的陆军夫妇住在一起。在黑斯廷斯,图灵住在上梅兹山的巴斯顿小屋(Baston Lodge, Upper Maze Hill),圣莱昂纳斯(St Leonards-on-Sea),现在有一块蓝色的匾,[15]这块匾于图灵诞生一百周年的2012年6月23日揭幕。[16]

在早年,图灵展示了天才的迹象,后来他又突出地展示了这一点。[17]1927年,他的父母在吉尔福德(Guildford)买了一栋房子,图灵在学校放假期间住在那里。该位置还标有蓝色斑块。[18]
\subsubsection{1.2 学校}
图灵的父母在他六岁的时候就给他注册了圣迈克尔学校(St Michael's),这是一所位于海边的查尔斯路(Charles Road)20号的日制学校。女校长很早就发现了他的才华,他后来的许多老师也是如此。

1922年1月至1926年1月间,图灵在哈斯赫斯特预备学校(Hazelhurst Preparatory School)接受教育,这所学校是苏塞克斯郡(Sussex)(现为东苏塞克斯郡)弗兰特村的一所独立学校。[19]1926年,13岁时,他继续就读于舍伯恩(Sherborne) 学校,这是一所位于多塞特郡舍伯恩集镇的独立寄宿学校。开学的第一天正好赶上1926年英国总罢工,但他下定决心要参加,以至于他独自骑自行车从南安普敦到谢伯恩60英里(97公里),并在一家旅店过夜。[20]

图灵对数学和科学的天生爱好并没有赢得舍伯恩的一些老师的尊重,舍伯恩的老师们对教育的定义更加强调经典。他的校长写信给他的父母:“我希望他不会两头落空。如果他要留在公立学校,他必须以接受教育为目标。如果他是一名科学专家,那他在公立学校是对时间的一种浪费”。[21]尽管没有得到老师们的认可,但是图灵继续在他热爱的研究中表现出非凡的能力。1927年,他甚至在连初等微积分没接触过的基础上就解决了高级问题;1928年,16岁的图灵读到了阿尔伯特·爱因斯坦的作品;他不仅领会了这一点,而且有可能从一篇从未明确阐述过的文章中推断出爱因斯坦对牛顿运动定律的质疑。[22]

舍伯恩毕业后,图灵从1931年到1934年在剑桥国王学院学习,[6]并在那里获得了一级数学荣誉。1935年,他22岁时,凭借一篇证明中心极限定理的论文,被选为国王学院的研究员。[23]委员会不知道,这个定理已经在1922年被贾尔·瓦尔德马尔·林德伯格(Jarl Waldemar Lindeberg)证明了。[24]2012年6月23日,该学院的一块蓝色牌匾在他诞辰100周年之际揭幕,现已安装在国王阅兵式上的学院凯恩斯大楼(Keynes Building)。[25][26]
\subsubsection{1.3 朋友}
在舍伯恩,图灵和他的同学克里斯托弗·莫克姆(Christopher Morcom,1911-1930)建立了重要的友谊,克里斯托弗·莫克姆被描述为图灵的“初恋”。他们的关系为图灵未来的成就提供了灵感,但由于莫克姆于1930年2月死于牛结核病的并发症,这一关系被打断了。牛结核病是几年前喝了受感染的牛奶后感染的。[27][28][29]

这件事给图灵带来了巨大的悲痛。他通过更加努力地研究他与莫克姆共享的科学和数学主题来应对自己的悲伤。在给莫克姆的母亲的一封信中图灵说:

我相信我在任何地方都找不到另一个如此聪明、如此迷人、如此不可思议的伴侣。我认为我对我的工作和天文学(他向我介绍了天文学)的兴趣是可以和他分享的,我想他对我也有一点相同的感觉...我知道我必须把尽可能多的精力投入到我的工作中,就像他活着一样,因为那是他希望我做的。

莫克姆去世后很久,图灵和莫克姆的母亲一直保持着联系,她给图灵送礼物,而他通常在莫克姆生日那天写信。莫克姆去世三周年的前一天(1933年2月12日),他写信给莫克姆夫人:

我希望你收到这封信时会想起克里斯。我也会的,这封信只是想告诉你,明天我会想到克里斯和你。我确信他现在和曾经在这里时一样快乐。你亲爱的艾伦。

一些人推测,莫克姆的死是图灵无神论和唯物主义的原因。显然,在生命的这一点上,他仍然相信精神(独立于肉体,在死亡中幸存)的概念。图灵在后来的一封信中说:

就我个人而言,我相信精神确实永远与物质联系在一起,但肯定不是由同一种身体联系在一起的...至于精神和身体之间的实际联系,我认为身体可以抓住一个“精神”,当身体活着和清醒时,两者是紧密相连的。当身体睡着的时候,我无法猜测会发生什么,但是当身体死亡的时候,身体的“机制”就消失了,灵魂迟早会找到一个新的身体,也许是马上。

\subsection{研究成就}
图灵在科学、特别在数理逻辑和计算机科学方面,取得了举世瞩目的成就,他的一些科学成果,构成了现代计算机技术的基础。
\subsubsection{2.1 可计算性研究}
\begin{figure}[ht]
\centering
\includegraphics[width=8cm]{./figures/e29f60892c491fbe.png}
\caption{剑桥国王学院。图灵于1931年成为这里的学生,并在1935年成为研究员。微机室是以他的名字命名的。} \label{fig_ALTL_1}
\end{figure}
1936年,图灵发表了他的论文《关于可计算的数字,及其在Entscheidungsproblem中的应用》(1936)[30],在这篇论文中,图灵重新表述了库尔特·哥德尔(Kurt Gödel's)1931年关于证明和计算极限的结果,用后来被称为图灵机的形式和简单的假设装置取代了哥德尔的通用算术形式语言。决策问题最初是由德国数学家戴维·希尔伯特在1928年提出的。图灵证明了他的“通用计算机器”能够执行任何可以想象的数学计算,如果它可以表示为一种算法的话。他继续证明决策问题没有解决方案,首先证明图灵机的停机问题是不可判定的:不可能通过算法决定图灵机是否会停机。

虽然图灵的证明是在阿隆佐·邱奇(Alonzo Church's)用λ演算进行等价证明后不久发表的,[31]但图灵的方法比丘奇的方法更容易理解和直观。[32]它还包含了“通用机器”(现在被称为通用图灵机)的概念,认为这样的机器可以执行任何其他计算机器的任务(实际上教会的λ演算也可以)。根据丘奇-图灵理论,图灵机和λ演算能够计算任何可计算的东西。约翰·冯·诺依曼(John von Neumann)承认现代计算机的核心概念源于图灵的论文。[33]迄今为止,图灵机是计算理论的一个中心研究对象。

从1936年9月到1938年7月,图灵第二年大部分时间在普林斯顿大学教堂学习,[34]除了纯粹的数学工作之外,他还研究密码学,并建立了机电二进制乘法器的四个阶段中的三个[53]。1938年6月,他获得了普林斯顿数学系的博士学位;[35]他的论文《基于序数的逻辑系统》引入了序数逻辑的概念和相对计算的概念,[36][37]在这两个概念中,图灵机被所谓的预言所增强,从而可以研究图灵机无法解决的问题。约翰·冯·诺依曼想雇他做博士后助理,但他回到了英国。[38]

图灵回到剑桥后,他参加了1939年路德维希·维特斯坦根(Ludwig Wittgenstein)关于数学基础的讲座。[39]讲座逐字逐句地进行了重构,包括图灵和其他学生的感叹词,以及学生笔记。[40]图灵和维特根斯坦争论不休,图灵为形式主义辩护,维特根斯坦提出他的观点,即数学不发现任何绝对真理,而是发明了它们。[41]
\subsubsection{2.2 密码分析学}
\begin{figure}[ht]
\centering
\includegraphics[width=8cm]{./figures/66f4e52ad7bf9342.png}
\caption{布莱奇利公园的马场里有两间小屋。图灵1939年和1940年在这里工作,后来搬到8号小屋。} \label{fig_ALTL_2}
\end{figure}
在第二次世界大战期间,图灵是布莱切利公园破解德国密码的主要参与者。历史学家和战时破译员阿萨·布里格斯(Asa Briggs)说过,“你需要非凡的才能,你需要布莱切利的天才,图灵就是那个天才。”[42]

从1938年9月开始,图灵一直在英国密码破译组织——政府法典和塞弗尔学校(GC&CS)兼职工作。他和GC&CS高级密码破译员迪利·诺克斯一起专注于密码分析。[43]在1939年7月华沙会议上,波兰密码局向英国和法国提供了英格玛转子接线的细节以及他们解密英格玛密码信息的方法,此后不久,图灵和诺克斯开始研究一种不那么脆弱的方法来解决这个问题。[44]波兰的方法依赖于一个不安全的指标程序,德国人很可能会改变这个程序,他们在1940年5月就这样做了。图灵的方法更普遍,使用基于婴儿床的解密,为此他产生了炸弹的功能规范(波兰炸弹的改进)。[45]

1939年9月4日,英国对德宣战的第二天,图灵向GC&CS战时车站布莱切利公园报到。[46]指定炸弹是图灵在战争期间取得的五大密码分析进展中的第一个。其他的是:推导德国海军使用的指示程序;制定一个统计程序,以便更有效地利用被称为班布里斯穆斯的炸弹;开发了一个程序来计算Lorenz SZ 40/42 (Tunny)命名为Turingery的车轮的凸轮设置,并在战争即将结束时,在Hanslope Park开发了一个便携式安全语音扰频器,代号为Delilah。

图灵通过使用统计技术优化代码破解过程中不同可能性的试验,对这个课题做出了创新性贡献。他写了两篇讨论数学方法的论文,题目分别是《概率在密码学中的应用》和《重复统计论文》。这两篇论文对GC&CS及其继任者GCHQ非常有价值,直到2012年4月,也就是他出生一百周年前夕,才被公布到英国国家档案馆。一位GCHQ数学家,“他自称只是理查德”,当时说,内容被限制了大约70年的事实证明了它们的重要性,以及它们与战后密码分析的相关性:

他说,内容受到限制的事实“表明它在我们学科的基础上有多么重要”。...论文详细介绍了如何使用“数学分析来尝试并确定哪些设置更有可能,以便能够尽快尝试。”...理查德说,GCHQ现在已经从这两份文件中“榨出了汁液”,并且“很高兴它们被公开”。

图灵在布莱切利公园以古怪著称。他被他的同事们称为“教授”,他关于谜的论文被称为“教授的书”。[47]根据历史学家罗纳德·列文的说法,与图灵一起工作的密码分析师杰克·古德这么评价他:

每年六月的第一周,他会患上严重的花粉热,他会戴着防毒面具骑车去办公室防止花粉传播。他的自行车有一个毛病:链条会定期脱落。他会数踏板转动的次数,然后及时从自行车上下来,用手调整链条,而不是修理它。他的另一个怪癖是他把杯子锁在散热管上以防被偷。

图灵是一名优秀的长跑运动员,在布莱切利工作时,他偶尔会在需要开会时跑40英里(64公里)到伦敦,[48]他有能力达到世界级的马拉松标准。[49][50]图灵参加了1948年英国奥林匹克队的选拔,但他因受伤而受阻。他的马拉松预赛时间仅比英国银牌得主托马斯·理查兹(Thomas Richards')的奥林匹克比赛时间慢11分钟,后者为2小时35分钟。他是沃尔顿体育俱乐部最好的跑步者,这是当他独自跑步时通过团体时发现的事实。[51][52][53]

1946年,图灵因其战时服务被乔治六世国王(King George VI)任命为大英帝国勋章(OBE)的一名军官,但他的工作多年来一直保密。[54]
\subsubsection{2.3 炸弹机}
\begin{figure}[ht]
\centering
\includegraphics[width=8cm]{./figures/8888b94e60068354.png}
\caption{布莱奇利公园的国家计算机博物馆。图为一个正在工作的完整轰炸机复制品。} \label{fig_ALTL_3}
\end{figure}
在到达布莱切利公园后的几周内,[46]图灵专注于一种名为“bombe”的机电机器,这种机器可以比波兰炸弹kryptologiczna更有效地破解“谜”。这个炸弹,由数学家戈登·韦尔奇曼Gordon Welchman)提出一个改进,成为主要工具之一,也是主要的自动化工具,用来攻击英格玛加密的信息。[55]

炸弹使用合适的婴儿床(可能明文的片段)搜索用于英格玛(Enigma)消息的可能正确设置(即转子顺序、转子设置和插板设置)。[56]对于转子的每一种可能的设置(对于四转子U型船,大约有1019个状态,或者1022个状态)[80]炸弹执行了一系列基于婴儿床的逻辑推理,并实现了机电一体化。

当矛盾发生时,炸弹被探测到,并排除了这种设置,转到下一个。大多数可能的设置会导致矛盾并被丢弃,只留下少数需要详细调查。当一个加密的字母变成相同的明文字母时,就会发生矛盾,这在英格玛中是不可能的。第一枚炸弹于1940年3月18日安装。[57]

到1941年底,图灵和他的密码分析师戈登·韦尔奇曼、休·亚历山大和斯图尔特·米尔纳-巴里(Gordon Welchman, Hugh Alexander and Stuart Milner-Barry )都感到沮丧。在波兰人工作的基础上,他们建立了一个很好的解密英格玛信号的工作系统,但是他们有限的人员和炸弹意味着他们不能翻译所有的信号。夏季,他们取得了相当大的成功,航运损失降至每月不到10万吨;然而,他们迫切需要更多的资源来跟上德国的调整。他们试图通过适当的渠道吸引更多的人和资助更多的炸弹制造商,但失败了。[58]

10月28日,他们直接写信给温斯顿·丘吉尔(Winston Churchill),解释他们的困难,图灵是第一个被指名的人。他们强调,他们的需求与部队的大量人力和财力相比是多么微不足道,与他们能够向部队提供的援助水平相比又是多么微不足道。[58]正如图灵的传记作者安德鲁·霍奇斯(Andrew Hodges)后来写道,“这封信有一种电效应。”[59]丘吉尔给伊斯梅尔将军(General Ismay)写了一份备忘录,上面写道:“今天行动。确保他们把他们想要的一切放在最优先的位置,并向我报告已经完成了。”11月18日,特勤局局长报告说,正在采取一切可能的措施。[59]布莱切利公园的密码学家不知道首相的反应,但正如米尔纳-巴里(Milner-Barry)回忆的那样,“我们所注意到的是,几乎从那天起,崎岖的道路奇迹般地变得平坦起来。”[60]到战争结束时,已有两百多枚炸弹投入使用。[61]
\begin{figure}[ht]
\centering
\includegraphics[width=8cm]{./figures/b4a82ee8c68173f8.png}
\caption{斯蒂芬·凯特尔在布莱奇利公园的图灵雕像,由西德尼·弗兰克委托,由50万块威尔士石板建造。.[34]} \label{fig_ALTL_4}
\end{figure}
\subsubsection{2.4 8号小屋和海军之谜}
图灵决定解决德国海军“英格玛”这个特别困难的问题,“因为没有其他人对此做任何事,我可以独享它”。[62]1939年12月,图灵解决了海军指示系统的关键部分,这个部分比其他部门使用的指示系统更复杂。[62][63]

同一天晚上,他还想到了班布里斯穆斯(Banburismus)的想法,这是一种序列统计技术(亚伯拉罕·瓦尔德后来称之为序列分析),用来帮助破解海军之谜,“尽管我不确定它在实践中是否行得通,事实上,直到有几天真的破解了,我也不确定。”[62]为此,他发明了一种衡量证据权重的方法,称之为禁令。Banburismus可以排除英格玛转子的某些序列,从而大大减少测试炸弹设置所需的时间。[64]后来,这种利用决策层(十分之一的决策层)积累足够证据权重的连续过程被用于洛伦兹密码的密码分析。[65]

图灵于1942年11月前往美国,[66]并与美国海军密码分析师合作在华盛顿建造海军英格玛和邦贝;他还参观了他们在俄亥俄州代顿的计算机实验室。

图灵对美国炸弹设计的反应并不热烈:

美国炸弹计划将生产336枚炸弹,每个车轮订单一枚。我过去常常对这个节目暗示的Bombe hut例程的概念暗自微笑,但我认为指出我们不会真的那样使用它们,不会达到任何特定的目的。 他们的测试(换向器)很难被认为是结论性的,因为他们没有测试电子站发现装置的反弹。似乎没有人被告知关于杆或offiziers或banburismus,除非他们真的要做些什么。

在这次旅行中,他还在贝尔实验室协助开发了安全语音设备。[67]他于1943年3月回到布莱切利公园。休·亚历山大不在的时候,他正式担任了8号小屋的负责人,尽管亚历山大实际上已经当了一段时间的负责人(图灵对这一部分的日常运作不感兴趣)。图灵成为布莱切利公园密码分析的总顾问。[68]

亚历山大写道图灵的贡献:

毫无疑问,图灵的工作是8号小屋成功的最大因素。在早期,他是唯一认为这个问题值得解决的密码学家,他不仅主要负责小屋内的主要理论工作,还与韦尔奇曼和基恩分享了炸弹发明的主要功劳。很难说任何人都是“绝对不可或缺的”,但是如果说任何人对8号小屋来说都是不可或缺的,那就是图灵。当经验和例行公事使一切变得简单时,先驱的工作总是被遗忘,我们8号小屋的许多人觉得图灵的贡献从未被外界完全意识到。
\subsubsection{2.5 图灵格林}
1942年7月,图灵发明了一种被称为图灵格林(Turingery) (或戏谑的 jokingly Turingismus)[69]的技术,用于对抗德国新的盖海姆施雷伯(秘密作家)机器产生的洛伦兹密码信息。这是布莱切利公园的电传打字机转子密码附件,代号为Tunny。Turingery是一种车轮断裂的方法,即计算Tunny车轮凸轮设置的程序。[70]他还将通尼团队介绍给汤米·弗劳尔斯,汤米·弗劳尔斯在马克斯·纽曼的指导下,继续建造巨像计算机,这是世界上第一台可编程数字电子计算机,它取代了更简单的先前机器(希斯·罗宾逊,Heath Robinson),其优越的速度使统计解密技术能够有效地应用于信息。[98]有人错误地说图灵是巨像计算机设计中的关键人物。图灵格林和班布里斯穆斯的统计方法无疑为洛伦兹密码的密码分析提供了思路,[71][72]但他并没有直接参与“巨像”的发展。[73]
\subsubsection{2.6 黛利拉}
继他在美国贝尔实验室的工作之后,[74]图灵在电话系统中寻求语音电子加密的想法。在战争后期,他搬家是为了在汉斯洛普公园(Hanslope Park)为特勤局的无线电安全局工作。在公园里,他在工程师唐纳德·贝利(Donald Bayley)的帮助下进一步发展了他的电子学知识。他们一起设计和建造了一台代号为黛利拉(Delilah)的便携式安全语音通信机。[75]这台机器是为不同的应用而设计的,但它缺乏长距离无线电传输的能力。无论如何,黛利拉完成得太晚了,不能在战争中使用。虽然这个系统工作得很好,图灵通过加密和解密温斯顿·丘吉尔的一段录音向官员们展示了这个系统,但是黛利拉并没有被采用。[76]图灵还就SIGSALY的开发咨询了贝尔实验室,SIGSALY是一种在战争后期使用的安全语音系统。
\subsubsection{2.7 人工智能}
\begin{figure}[ht]
\centering
\includegraphics[width=8cm]{./figures/7860e2b99ef4d2bf.png}
\caption{牌匾,汉普顿大街78号} \label{fig_ALTL_5}
\end{figure}
1945年至1947年间,图灵住在伦敦汉普顿,[77]在国家物理实验室从事自动计算引擎的设计。1946年2月19日,他发表了一篇论文,这是第一个详细设计的存储程序计算机。[78]冯·诺依曼的不完整的教育真空科学报告初稿早于图灵的论文,但它不太详细,根据不良贷款数学部门主管约翰·沃默斯利(John R. Womersley)的说法,它“包含了许多图灵博士自己的想法”。[79]尽管ACE是一个可行的设计,但是布莱切利公园战时工作的秘密导致了项目启动的延误,他也醒悟了。1947年末,他回到剑桥休假一年,在此期间,他创作了一部关于智能机械的开创性著作,这本书在他有生之年没有出版过。[80]当他在剑桥的时候,Pilot ACE正是在他不在的时候建造。它在1950年5月10日执行了它的第一个程序,后来世界各地的许多计算机都归功于它,包括英国的电动杜斯(Electric DEUCE)和美国的本迪克斯(Bendix G-15)。图灵的完整版本ACE直到他死后才被制造出来。[81]
\subsubsection{2.8 图灵测试}
根据杜塞尔多夫根舍(Genscher, Düsseldorf)出版的马克斯·普朗克物理研究所(MPIP)德国计算机先驱海因茨·比尔(Heinz Billing)的回忆录,图灵和康拉德·楚泽之间有一次会面。[82]它于1947年在哥廷根举行。审讯采取座谈会的形式。参与者包括沃默斯利、图灵、来自英国的波特和一些德国研究人员,如苏泽、沃尔特和比尔。

1948年,图灵被任命为曼彻斯特维多利亚大学( Victoria University of Manchester)数学系的读者。一年后,他成为了计算机实验室的副主任,在那里他为最早的存储程序计算机之一——曼彻斯特马克1号——开发软件。图灵为这台机器编写了第一版《程序员手册》,并被费朗蒂(Ferranti)聘请为他们商业化机器费朗蒂马克1的开发顾问。费朗蒂一直向他支付咨询费,直到他去世。[83]在此期间,他继续在数学方面做更抽象的工作,在《计算机械与智能》(Mind,1950年10月)中,图灵提出了人工智能的问题,并提出了一个后来被称为图灵测试的实验,试图定义一个被称为“智能”的机器标准。这个想法是,如果一个人类询问者不能通过对话将计算机与人类区分开来,那么计算机可以被称为“思考”。[84]图灵在论文中建议,与其建立一个模拟成人大脑的程序,不如制作一个更简单的程序来模拟儿童的大脑,然后让其接受教育。图灵测试的反向形式在互联网上被广泛使用;验证码测试旨在确定用户是人还是计算机。

1948年,图灵和他以前的大学同事钱博诺尼(D.G. Champernowne)一起,开始为一台尚不存在的计算机编写象棋程序。到1950年,该计划完成,并被称为涡轮增压器。[85]1952年,他试图在费朗蒂马克1上实现它,但由于缺乏足够的电力,计算机无法执行该程序。相反,图灵通过翻阅算法页面并在棋盘上执行指令来“运行”程序,每次移动大约需要半个小时。比赛被记录下来了。[86]根据加里·卡斯帕罗夫(Garry Kasparov)的说法,图灵的程序“玩了一个可识别的象棋游戏”[87] 这个程序输给了图灵的同事阿利克·格伦尼,尽管据说它赢了一场与尚珀诺尼的妻子伊莎贝尔(Isabel)的比赛。[88]

他的图灵测试是对人工智能争论的一个重要的、典型的挑衅性的、持久的贡献,这场争论持续了半个多世纪。[89]
\subsubsection{2.9 数理生物学}
图灵在1951年39岁时转向数学生物学,最终在1952年1月出版了他的代表作《形态发生的化学基础》。他对形态发生感兴趣,即生物有机体中模式和形状的发展。除此之外,他想了解斐波那契叶序,即斐波那契数在植物结构中的存在。[90]他认为,一个化学物质相互反应并在空间中扩散的系统,称为反应扩散系统,可以解释“形态发生的主要现象”。[91]他使用偏微分方程系统来模拟催化化学反应。例如,如果发生某种化学反应需要催化剂A,并且如果该反应产生更多的催化剂A,那么我们说该反应是自催化的,并且存在可以通过非线性微分方程建模的正反馈。图灵发现,如果化学反应不仅能产生催化剂甲,还能产生抑制剂乙,从而减缓甲的生成,那么可以产生模式。如果甲和乙以不同的速度扩散通过容器,那么你可能会有甲占主导地位的区域和乙占主导地位的区域。为了计算这种程度,图灵需要一台功能强大的计算机,但是这些在1951年还不能免费获得,所以他不得不使用线性近似来手工求解方程。这些计算给出了正确的定性结果,例如,产生了均匀的混合物,奇怪的是,它有规则间隔的固定红点。俄罗斯生物化学家鲍里斯·贝鲁索夫(Boris Belousov)曾进行过类似结果的实验,但他的论文未能发表,因为当代的偏见认为,任何这样的事情都违反热力学第二定律。贝鲁索夫不知道图灵在《皇家学会哲学学报》上的论文。[92]

尽管在理解脱氧核糖核酸的结构和作用之前就发表了,图灵在形态发生方面的工作今天仍然有意义,被认为是数学生物学中的一项开创性工作。[93]图灵论文的早期应用之一是詹姆斯·穆雷解释猫皮毛上大大小小的斑点和条纹的工作。[94][95][96]该领域的进一步研究表明,图灵的工作可以部分解释“羽毛、毛囊、肺的分支模式,甚至是心脏位于胸部左侧的左右不对称”的生长,[97]2012年,谢特等人发现,在小鼠中,Hox基因的去除会导致手指数量的增加,而不会增加肢体的整体尺寸,这表明Hox基因通过调节图灵型机制的波长来控制手指的形成。[98]后来的论文直到1992年出版《图灵全集》才发表。[99]

\subsection{个人生活}
1941年,图灵向8号小屋的同事琼·克拉克( Joan Clarke)求婚,琼·克拉克是一位数学家兼密码分析学家,但他们的订婚时间很短。图灵向未婚妻承认了自己的同性恋行为,据报道,未婚妻对此事“并不担心”,图灵决定他不能继续这桩婚姻。[100]
\subsubsection{3.1 猥亵罪}
1952年1月,图灵39岁时开始与19岁的失业者阿诺德·默里交往。就在圣诞节前,图灵正沿着曼彻斯特的牛津路散步,就在皇家电影院外面遇见了默里,并邀请他吃午饭。1月23日,图灵的房子被盗。默里告诉图灵,他和窃贼相识,图灵向警方报案。在调查期间,他承认与默里有性关系。同性恋行为在当时的联合王国是刑事犯罪,[101]根据1885年《刑法修正案》第11条,这两个人都被指控犯有“严重猥亵罪”。[102]审判的初步交付审判程序于2月27日举行,在此期间图灵的律师“保留他的辩护”,即没有就指控进行辩论或提供证据。

图灵后来被他哥哥和他自己的律师的建议说服了,他认罪了。[103]里贾纳诉图灵和默里一案于1952年3月31日受审。[104]图灵被定罪,并在监禁和缓刑之间做出选择。他的缓刑将取决于他同意接受旨在降低性欲的荷尔蒙身体变化。他接受了注射当时被称为stilbestrol(现在被称为己烯雌酚或DES)的一种合成雌激素的选择;他身体的女性化持续了一年。这种治疗使图灵阳痿并导致乳房组织形成,[105]从字面意义上实现了图灵的预测,“毫无疑问,我将从这一切中变得完全不同,但我完全没有发现”。默里被有条件释放。[106][107] [108]

图灵的定罪导致他的安全许可被取消,并禁止他继续为英国信号情报机构政府通信总部(GCHQ)提供密码咨询服务,该机构于1946年从GC&CS演变而来,尽管他保留了他的学术工作。1952年定罪后,他被拒绝进入美国,但可以自由访问其他欧洲国家。图灵从未被指控从事间谍活动,但和所有在布莱切利公园工作的人一样,官方保密法禁止他讨论自己的战争工作。[109]
\subsubsection{3.2 死亡}
1954年6月8日,图灵的管家发现他死了;他前一天去世了。氰化物中毒被确定为死因。[110]当他的尸体被发现时,他的床边放了一个吃了一半的苹果,虽然苹果没有检测氰化物,[111]但据推测这是图灵摄入致命剂量的方法。调查确定他自杀了。安德鲁·霍奇斯和另一位传记作者大卫·莱维特都推测图灵正在重演沃尔特·迪斯尼(Walt Disney)电影《白雪公主和七个小矮人,Snow White and the Seven Dwarfs》(1937)中的一幕,这是他最喜欢的童话故事。两人都注意到(用莱维特的话来说),他“特别喜欢邪恶女王将苹果浸泡在有毒的啤酒中的场景”[112]图灵的遗体于1954年6月12日在沃金火葬场火化,他的骨灰撒在火葬场的花园里,就像他父亲的骨灰一样。[113][114]

哲学教授杰克·科普兰质疑验尸官历史判决的各个方面。他对图灵的死因提出了另一种解释:意外吸入一种用于将金电镀到勺子上的仪器中的氰化物烟雾。氰化钾被用来溶解黄金。图灵在他小小的备用房间里安装了这样一个设备。科普兰指出,尸检结果与吸入比摄入毒药更一致。图灵还习惯性地在睡觉前吃一个苹果,苹果被半途丢弃并不罕见。[115]此外,据报道图灵“心情愉快”地忍受了法律上的挫折和激素治疗(一年前已经停止),在他死前没有表现出沮丧的迹象。他甚至列出了假期周末后回到办公室时打算完成的任务清单。[115]图灵的母亲认为摄入是意外的,是因为她儿子不小心储存了实验室化学品。[116]传记作者安德鲁·霍奇斯理论上说,图灵安排运送设备是为了故意让他母亲对任何自杀指控做出可信的否认。[117]
\begin{figure}[ht]
\centering
\includegraphics[width=8cm]{./figures/9882857da2b22471.png}
\caption{目前在舍伯恩学校档案馆陈列的图灵的官佐勋章} \label{fig_ALTL_6}
\end{figure}
阴谋论者指出图灵是他去世时英国当局极度焦虑的原因。特勤局担心共产党会诱捕杰出的同性恋者,并利用他们收集情报。图灵仍然从事高度机密的工作,当时他也是一名在铁幕附近的欧洲国家度假的实践同性恋者。特勤局可能认为他的安全风险太大,并暗杀了他们雇佣的最聪明的人之一。[118]

图灵相信超感官知觉,[119][120]有人认为他对算命的信仰可能导致了他的抑郁情绪。图灵年轻时,一个吉普赛算命师告诉他,他会是个天才。[114]去世前不久,在和格林鲍姆一家去 St Annes-on Sea一日游期间,图灵再次决定去咨询算命师。格林鲍姆的女儿芭芭拉说:[114]

但是那天阳光明媚,艾伦心情愉快,我们就走了...然后他认为去布莱克浦的游乐场是个好主意。我们找到了一个算命的帐篷,艾伦说他想进去,所以我们在周围等他回来...这张阳光明媚、愉快的脸已经缩成一张苍白、颤抖、惊恐的脸。出事了。我们不知道算命先生说了什么,但他显然非常不高兴。我想那可能是我们听说他自杀之前最后一次见到他”[121]
\subsubsection{3.3 政府道歉}
2009年8月,英国程序员约翰•格雷厄姆•卡明斯开始请愿,敦促英国政府为图灵被控同性恋一事道歉。请愿书收到了30000多个签名。首相戈登•布朗承认了请愿书,并于2009年9月10日发表声明道歉,称图灵的遭遇“骇人听闻”:

成千上万的人聚集在一起,要求为艾伦•图灵伸张正义,并承认他受到的骇人听闻的待遇。虽然图灵是按照当时的法律处理的,我们不能让时光倒流,但他的待遇当然是完全不公平的,我很高兴有机会为他的遭遇向我和我们所有人表示深深的歉意...因此,我代表英国政府,以及所有因艾伦的工作而自由生活的人,非常自豪地说:对不起,你应该得到更好的待遇。
\subsubsection{3.4 女王赦免}
2011年12月,威廉姆•琼斯创建了一份请求英国政府赦免图灵“严重猥亵罪”的请愿书:

我们请求英国政府赦免艾伦•图灵的“严重猥亵罪”。1952年,他被判与另一名男子“严重猥亵”,并被迫接受所谓的“有机疗法”——化学阉割。两年后,他用氰化物自杀,年仅41岁。艾伦•图灵被他为拯救的国家逼到了绝望和早逝的境地。这仍然是英国政府和英国历史的耻辱。赦免可以在某种程度上治愈这种伤害。这可能会成为对许多其他同性恋者的道歉,而这些同性恋者并不像艾伦•图灵那样广为人知,他们受到了这些法律的约束。

请愿书收集了37000多个签名,并得到曼彻斯特议员约翰•利奇(John Leech)的支持,但司法部长麦克纳利勋爵(Lord McNally)劝阻了这项请求,他说:

死后赦免被认为是不合适的,因为艾伦•图灵被正确地判定犯有当时的刑事罪。他会知道他的罪行是违法的,他会被起诉。艾伦•图灵被判犯有一项现在看来既残酷又荒谬的罪行,这是一个悲剧——尤其是考虑到他对战争的杰出贡献,这一罪行令人痛心。然而,当时的法律要求起诉,因此,长期的政策是接受这种定罪,而不是试图改变历史背景,纠正不能纠正的地方,而是确保我们永远不再回到那个时代。

曼彻斯特威斯顿议员约翰·利奇(2005-15)向议会提交了几项法案,[122]并与琼斯一起争取赦免。利奇在下议院指出图灵对战争的贡献使他成为了民族英雄,而这种信念仍然存在“最终只是令人尴尬”。[123]利奇继续在议会通过该法案,并进行了几年的竞选活动,直到该法案获得通过。[124]

在一部以图灵生活为基础的电影《模仿游戏》的英国首映式上,制片人感谢利奇让公众关注这个话题,并获得图灵的赦免。[125]他的竞选活动转向获得对其他75000名犯有同样罪行的人的赦免。利奇的运动获得了包括斯蒂芬·霍金在内的主要科学家的公开支持。[126]

2012年7月26日,上议院提出了一项法案,对图灵犯下的1885年《刑法修正案》第11条所述罪行给予法定赦免,图灵于1952年3月31日被定罪。[127]今年晚些时候,物理学家斯蒂芬·霍金和其他10名签署人,包括天文学家皇家里斯勋爵、皇家学会主席保罗·纳斯爵士、特朗普顿夫人(战时为图灵工作)和夏基勋爵(该法案的发起人),在给《每日电讯报》的一封信中呼吁首相大卫·卡梅伦就赦免请求采取行动。[128]政府表示将支持该法案,[129][130][131]并于10月在上议院通过了三读。[132]

2013年11月29日,在下议院对该法案进行二读时,保守党议员克里斯托弗·乔普(Christopher Chope)反对该法案,推迟了该法案的通过。该法案本应于2014年2月28日返回下议院,[133]但在下议院辩论该法案之前,[134] 政府选择根据皇家赦免特权行事。2013年12月24日,伊丽莎白二世女王签署了图灵“严重猥亵罪”的赦免令,立即生效。[135]宣布赦免时,大法官克里斯·格雷林说图灵应该“因为他对战争的巨大贡献而被铭记和认可”,而不是因为他后来的刑事定罪。[136][137]图灵是自第二次世界大战结束以来第四次皇家赦免。[138][139]赦免通常只在该人在技术上是无辜的,并且该家庭或其他利益方提出请求时才给予;图灵的定罪没有满足两个条件。[140]

人权倡导者彼得·塔切尔(Peter Tatchell)在给首相大卫·卡梅伦(David Cameron)的一封信中批评了选择图灵的决定,这是因为他的名声和成就,而根据同一法律被定罪的其他数千人却没有得到赦免。[141]塔切尔还呼吁对图灵的死亡进行新的调查:

一项新的调查早就应该进行了,即使只是为了消除对他死因的任何怀疑——包括他被安全部门(或其他部门)谋杀的猜测。我认为国家特工谋杀是不可能的。没有已知证据表明有任何此类行为。然而,这种可能性从未被考虑或调查,这是一个重大的失败。

2016年9月,政府宣布打算将这一追溯免责扩大到被判犯有类似历史猥亵罪的其他男子,这被称为"艾伦·图灵法"。[142][143] 艾伦·图灵法现在是联合王国法律的非正式术语,载于《2017年治安和犯罪法》,该法作为大赦法追溯赦免根据禁止同性恋行为的历史立法被警告或定罪的男子。该法律适用于英格兰和威尔士。[144]
\subsubsection{3.5 英国情报机构道歉}
2016年4月16日,英国三大情报机构之一——政府通信总部(GCHQ )主管罗伯特·汉尼根(Robert Hannigan)在会议中表示,对该情报机构在上世纪50年代错误地对待“人工智能之父”艾伦·麦席森·图灵(Alan Mathison Turing)表示道歉。

汉尼根说,政府通信总部对待图灵等天才的方法有错:他们遭受折磨,是我们的损失,也是国家的损失,我们应该为此道歉

\subsection{奖项荣誉}
\begin{figure}[ht]
\centering
\includegraphics[width=8cm]{./figures/69ff3dc5b4989df2.png}
\caption{2008年曼彻斯特大学的艾伦图灵大楼} \label{fig_ALTL_7}
\end{figure}
1926年,图灵考入英国著名的谢伯恩公学,在中学时就获得了国王爱德华六世数学金盾奖章。

1932年,荣获英国著名的史密斯数学奖。

1946年,由于他在二战中为破译德军密码做出的巨大贡献,获得“不列颠帝国勋章”,这是英国皇室授予为国家和人民做出巨大贡献者的最高荣誉勋章。[145]

1951年,他还被选为皇家学会会员。[145]

以他名字命名的:

艾伦·图灵研究所

丘奇——图灵论文

GOOD–图灵频率估计

图灵完备性

图灵度

图灵不动点组合器

图灵研究所

图灵讲座

图灵机

图灵模式

图灵归约

图灵开关

图灵测试

5 后世纪念编辑

\subsection{后世纪念}
\subsubsection{5.1 追悼纪念}
各种机构都以图灵的名字来纪念他,包括:

图灵的母校剑桥国王学院的计算机室被称为图灵室。

爱丁堡大学信息学院的图灵室收藏了爱德华多·保罗齐的图灵半身像和一套图灵版画(2000年,第42/50号)。

萨里大学的主广场上有一尊图灵雕像,工程和物理科学学院的一座建筑以他的名字命名。

伊斯坦布尔比尔基大学组织了一个名为“图灵日”的计算理论年度会议。

奥斯汀的德克萨斯大学有一个名为图灵学者的荣誉计算机科学项目。

20世纪60年代初,斯坦福大学将波利亚大厅数学大楼唯一的演讲室命名为“艾伦·图灵礼堂”。

计算机科学系的圆形剧场之一(法国北部里尔大学的LIFL以艾伦·图灵的名字命名)(另一个圆形剧场以库尔特·哥德尔的名字命名)。

华盛顿大学有一个以图灵命名的计算机实验室。

牛津布鲁克斯大学曼彻斯特大学、开放大学、伍尔弗汉普顿大学和奥尔胡斯大学(位于丹麦奥尔胡斯)都有以图灵命名的建筑。

萨里研究公园的艾伦图灵路和曼彻斯特内环路的艾伦图灵路。拉夫堡的艾伦·图灵路是以图灵的名字命名的。

卡内基梅隆大学有一个花岗岩长凳,位于洪博斯特购物中心,上面刻着“上午图灵”的名字,“读”在左腿上,“写”在另一条腿上。

俄勒冈大学在计算机科学大楼德舒特大厅一侧有一尊图灵半身像。

洛桑联邦理工学院有一条路和一个以图灵命名的广场。

斯洛伐克布拉迪斯拉发斯洛伐克理工大学信息学和信息技术系有一个名为“图灵礼堂”的演讲室。

巴黎狄德罗大学有一个名为“图灵学院”的教室。

维尔茨堡大学数学和计算机科学系有一个名为“图灵·赫萨尔”的演讲厅。

图卢兹的保罗·萨巴捷大学有一个教室,名叫“图灵教室”。

阿姆斯特丹科学园最大的会议厅被命名为图灵扎尔。

伦敦大学国王学院自然和数学科学学院授予艾伦·图灵百年奖。

肯特大学在坎特伯雷校区以他的名字命名了图灵学院。

巴黎理工大学的校园有一座以图灵命名的建筑;这是一个研究中心,其办公场所由法国理工学院和微软共享。

多伦多大学在1982年开发了图灵编程语言,以图灵命名。

巴西坎皮纳斯州立大学的校园有一条大道,是以图灵命名的最大的大道之一。

智利宗座天主教大学计算机科学系、布宜诺斯艾利斯大学、波多黎各理工大学、哥伦比亚波哥大洛斯安第斯大学、国王学院、剑桥、威尔士班戈大学、比利时蒙斯大学、都灵大学(都灵大学)、波多黎各胡马凯大学、基尔大学和AGH科技大学计算机科学、电子和电信系都有以图灵命名的建筑物。

根特大学以图灵的名字命名了他们计算机科学和应用数学系的一个计算机室。

英伟达推出了基于图灵微体系结构的GeForce显卡系列,该系列又以图灵命名。该架构引入了第一款能够实时光线追踪的消费产品,这是计算机图形行业的长期目标。

图灵死后不久,皇家学会出版的传记记载道,当时他的战时工作仍受官方保密法的约束:

战前就写了三篇关于三个不同数学主题的杰出论文,展示了如果他在关键时刻着手解决某个大问题,可能会产生的工作质量。由于他在外交部的工作,他被授予出勤奖。

1998年6月23日,也就是图灵86岁生日,他的传记作者安德鲁·霍奇斯( Andrew Hodges)在他的出生地伦敦沃林顿新月酒店,后来的柱廊酒店,揭开了官方的英国遗产蓝色匾。[145][146]为了纪念他逝世50周年,2004年6月7日,在他位于柴郡威尔姆斯洛的故居霍利梅德,一个纪念牌匾揭幕。[147]
\begin{figure}[ht]
\centering
\includegraphics[width=8cm]{./figures/ed508d9ab306acaf.png}
\caption{图灵在柴郡威尔姆斯洛的家的蓝色牌匾} \label{fig_ALTL_9}
\end{figure}
2000年3月13日,圣文森特和格林纳丁斯(Saint Vincent and the Grenadines)发行了一套纪念20世纪最伟大成就的邮票,其中一张邮票上有图灵的肖像,背景是重复的0和1,标题是:“1937年:艾伦·图灵的数字计算理论”。2003年4月1日,图灵在布莱切利公园的工作被命名为IEEE里程碑。[148]2004年10月28日,约翰·米尔斯(John W. Mills)雕刻的图灵铜像在吉尔福德萨里大学揭幕,纪念图灵逝世50周年;它描绘了他背着书穿过校园。[149]

图灵在曼彻斯特获得了各种荣誉,他在曼彻斯特工作到生命的最后。1994年,一段A6010公路(曼彻斯特市中间环路)被命名为“艾伦·图灵路”(Alan Turing Way)。一座承载着这条路的桥被拓宽了,并被命名为艾伦·图灵桥。图灵的雕像于2001年6月23日在曼彻斯特的萨克维尔公园揭幕,位于惠特沃思街的曼彻斯特大学大楼和运河街之间。这座纪念雕像描绘了“计算机科学之父”坐在公园中心位置的长凳上。图灵拿着一个苹果。铸造青铜长凳上浮雕着“艾伦·麦席森·图灵1912-1954”的文字和“计算机科学的创始人”的座右铭,如果用英格玛机器编码的话可能会出现:“IEKYF ROMSI ADXUO KVKZC GUBJ”。然而,编码信息的含义有争议,因为“计算机”中的“u”与“ADXUO”中的“u”相匹配。由于一个由英格玛机编码的字母不能像它自己一样出现,代码背后的实际信息是不确定的。[150]
\begin{figure}[ht]
\centering
\includegraphics[width=8cm]{./figures/ba85766bee6d8458.png}
\caption{位于曼彻斯特萨克维尔公园的图灵纪念雕像牌匾} \label{fig_ALTL_10}
\end{figure}
雕像脚下的一块匾写着“计算机科学之父、数学家、逻辑学家、战时破译者、偏见的受害者”。还有一句伯特兰·罗素(Bertrand Russell)的名言:“正确看待数学,它不仅拥有真理,而且拥有至高无上的美——一种冰冷而简朴的美,就像雕塑一样。”雕塑家把他自己的旧Amstrad电脑埋在基座下,以此纪念“所有现代电脑的教父”。[151]

1999年,《时代》杂志将图灵命名为20世纪100位最重要的人物之一,并指出,“事实仍然是,每个敲键盘、打开电子表格或文字处理程序的人都在研究图灵机的化身。”[152]

2002年,在英国广播公司对英国100位最伟大的人进行的民意调查中,图灵名列第21位。[152]2006年,英国作家兼数学家伊万·詹姆斯(Ioan James)选择图灵作为他的著作中的20人之一,书中描写了一些可能具有阿斯伯格综合症特征的著名历史人物。[153]2010年,演员兼剧作家杰德·埃斯特万·埃斯特拉达(ade Esteban Estrada)在独唱音乐剧《偶像:世界男女同性恋史》第4卷中扮演图灵。2011年,在《卫报》的“我的英雄”系列中,作家艾伦·加纳(Alan Garner)选择图灵作为他的英雄,并描述了他们在20世纪50年代初慢跑时是如何相遇的。加纳记得图灵“风趣幽默”,并说他“无休止地交谈”。[154]2006年,图灵被网上资源命名为同性恋者、双性恋者和变性者历史月图标。[155]2006年,波士顿骄傲命名图灵为他们的荣誉大元帅。[156]
\begin{figure}[ht]
\centering
\includegraphics[width=8cm]{./figures/2e58d76e0cda7d23.png}
\caption{曼彻斯特萨克维尔公园的图灵纪念雕像} \label{fig_ALTL_11}
\end{figure}
苹果公司的标志经常被错误地称为对图灵的致敬,咬痕则是对他死亡的一种暗示。[157]标志的设计者[158]和公司都否认设计中有任何对图灵的敬意。[159][160]史蒂芬·弗莱(Stephen Fry)讲述了他问史蒂夫·乔布斯(Steve Jobs)这个设计是否是有意的,他说乔布斯的回答是,“上帝,我们希望是这样。”[161]2011年2月,图灵第二次世界大战的论文被国家遗产纪念基金在第11个小时的竞标中收购,允许他们留在布莱切利公园。[162]

2012年,图灵被引入Legacy Walk,这是一个户外公共展示,庆祝同性恋、双性恋和变性者的历史和人们。[163][164]

法语歌手兼词曲作者萨尔瓦托勒·阿达莫( Salvatore Adamo)的歌曲《阿兰与波美》是对图灵的致敬。[165]图灵的生活和工作在英国广播公司关于著名科学家的儿童节目《迪克和多姆的绝对天才》中播出,该节目于2014年3月12日首次播出。

2014年5月17日,世界上第一件承认图灵是同性恋的公共艺术作品在布莱切利被委托创作,就在布莱切利公园附近,他战时的作品就是在那里完成的。该委员会宣布纪念反对同性恋恐惧症和变性恐惧症国际日。该作品于2014年6月23日图灵生日的仪式上揭幕,并被放置在繁忙的沃特琳街旁,沃特琳街是通往伦敦的老主干道,图灵本人曾多次路过这里。2014年10月22日,图灵被引入美国国家安全局荣誉厅。[166][167]

2019年2月,在英国广播公司八集系列《图标:20世纪最伟大的人》中,图灵被观众选为最伟大的人。[168]
\subsubsection{5.2 图灵奖}
为了纪念他对计算机科学的巨大贡献,由美国计算机协会(ACM)于1966年设立一年一度的图灵奖。图灵奖每年都由计算机械协会颁发,以表彰其对计算界的技术或理论贡献。它被广泛认为是计算世界的最高荣誉,也被誉为“计算机界的诺贝尔奖”。[169]
\subsubsection{5.3 百年庆典}
\begin{figure}[ht]
\centering
\includegraphics[width=8cm]{./figures/137811cc978e2c5d.png}
\caption{David Chalmers出席2012年3月27日在德拉萨大学举行的艾伦图灵年会议} \label{fig_ALTL_12}
\end{figure}
为了纪念图灵诞生100周年,图灵百年咨询委员会(TCAC)协调了艾伦图灵年,这是一个为期一年的全球活动计划,旨在纪念图灵的一生和成就。TCAC由巴里·库珀担任主席,图灵的侄子约翰·德莫特·图灵(S. Barry Cooper)爵士担任名誉主席,他与曼彻斯特大学的教职员工以及剑桥大学和布莱切利公园的众多人士合作。

2012年6月23日,谷歌展示了一个互动涂鸦,访问者必须改变图灵机的指令,所以当运行时,磁带上的符号将与提供的序列相匹配,以博多-默里代码中的“谷歌”为特色。[170]

布莱切利公园信托基金与赢棋公司合作出版了艾伦图灵版的棋盘游戏《大富翁》。游戏的方块和卡片已经被修改以讲述图灵的一生,从他在迈达谷的出生地到布莱切利公园的8号小屋。[171]该游戏还包括图灵导师马克斯·纽曼的儿子威廉·纽曼创作的原始手绘棋盘的复制品,图灵曾在20世纪50年代玩过这个游戏。[172]

在菲律宾,马尼拉德拉萨勒大学哲学系于2012年3月27日至28日主办了图灵2012年会,这是一次关于哲学、人工智能和认知科学的国际会议,以纪念印度图灵·马杜赖诞辰100周年,[173][174]举办了庆祝活动,有6000名学生参加。[175]

6月份在曼彻斯特举行了为期三天的会议,艾伦·图灵百年纪念会议,在旧金山举行了为期两天的会议,由美国计算机学会组织,在剑桥的国王学院和剑桥大学举行了生日聚会和图灵百年纪念会议,后者由欧洲可计算性协会组织。[176]

伦敦科学博物馆在2012年6月推出了一个免费展览,专门展示图灵的生活和成就,展览将持续到2013年7月。[177]2012年2月,皇家邮政发行了一枚图灵邮票,作为其“杰出英国人”系列的一部分。[178]2012年6月23日晚,即图灵诞辰100周年之际,伦敦2012奥运会火炬在曼彻斯特萨克维尔花园图灵雕像前传递。

2012年6月22日,曼彻斯特市议会与男女同性恋基金会合作,启动了艾伦·图灵纪念奖,该奖将表彰为打击曼彻斯特同性恋恐惧症做出重大贡献的个人或团体。[179]

牛津大学开设了计算机科学和哲学新课程,以纪念图灵诞生一百周年。[180]

此前的活动包括由英国逻辑座谈会和英国数学史学会于2004年6月5日在曼彻斯特大学举办的图灵生平和成就庆祝活动。[181]