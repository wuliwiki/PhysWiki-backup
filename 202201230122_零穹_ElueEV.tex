% 欧拉方程(变分学)
% 欧拉方程|正规点

\pentry{变分的变换(变分学)\upref{VarCha}}
由属于 $C_1$ 类的所有过给定点 $a,b$ 的曲线构成的可取曲线族\autoref{DesCur_sub1}~\upref{DesCur}中,给出泛函 
\begin{equation}
J(y)=\int_a^bF(x,y,y')\dd x
\end{equation}
的相对弱极值\autoref{AbPol_sub1}~\upref{AbPol}的可取曲线满足的方程便是\textbf{欧拉方程}.其具有下面的形式
\begin{equation}\label{ElueEV_eq1}
F'_y-\dv{}{x}F'_{y'}=0
\end{equation}
在引出欧拉方程的过程中,我们也将得到在变分的基本定理\upref{VarDef}一节中开头提到但未详细说明的结论,其可归为下面的定理.
\begin{theorem}{}
设 $F(x,y,y')$ 及其二阶偏微商对于 $a\leq x\leq b$ 及任意的 $y,y'$ 连续,泛函
\begin{equation}
J(y)=\int_a^b F(x,y,y')\dd x
\end{equation}
的可取曲线族由属于 $C_1$ 类的所有过点 $a,b$ 的曲线构成.若 $y=y(x)$ 给出 $J(y)$ 的相对弱极值,则函数 $y(x)$ 满足欧拉方程\autoref{ElueEV_eq1} .
并在 $F_{y'y'}\neq0$ 的一切 $x$ 值上, $y''(x)$ 存在且连续.

\end{theorem}
满足欧拉方程的曲线便称为\textbf{极值曲线}.
\subsection{证明}
由\autoref{PolReq_the1}~\upref{PolReq}, $C_1$ 类的过 $a,b$ 的函数 $y(x)$ 给出泛函 $J(y)$ 极值的必要条件是 $\delta J=0$,即\autoref{Varia_eq5}~\upref{Varia}
\begin{equation}
\delta J=\int_a^b (F'_y-\dv{}{x}F'_{y'})\delta y\dd x=0
\end{equation}
