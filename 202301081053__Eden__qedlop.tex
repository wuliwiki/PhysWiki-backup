% QED的重整化理论(单圈修正)
% 重整化|单圈修正|量子电动力学

\pentry{QED的费曼规则\upref{qedfey}}

裸的拉氏量为
\begin{equation}
\begin{aligned}
\mathcal{L}=
\bar{\psi}_0 (i\not\partial -m_0)\psi_0 -\frac{1}{4}(F_0^{\mu\nu})^2 - e_0\bar\psi_0 \gamma_\mu\psi_0 A_0^\mu
\end{aligned}
\end{equation}
为了消除圈图的紫外发散,采取一定的正规化方案(例如我们最常使用的维数正规化方案),然后再调整裸参数使得可观测量的计算结果与实验相符,即满足一定的重整化条件。或者我们采取 OS 方案或 $\overline{MS}$ 方案,不同的重整化方案实际上是对场量进行了平移和缩放,我们可以写出重整化的拉氏量。
\begin{equation}
\begin{aligned}
\mathcal{L} = \bar\psi (iZ_2\not\partial - Z_m m)\psi -\frac{1}{4}Z_3F^{\mu\nu}F_{\mu\nu}-Z_1 e\bar\psi \gamma^\mu \psi A_\mu
\end{aligned}
\end{equation}
提取出其中自由场的部分后,微扰的拉氏量就是
\begin{equation}
\mathcal{L}_1 = - Z_1 e \bar\psi \gamma^\mu\psi A_\mu + \mathcal{L}_\text{CT},\quad \mathcal{L}_\text{CT} = \bar\psi (i\delta_2 \not\partial - \delta_m m)\psi - \frac{1}{4}\delta_3 F^{\mu\nu}F_{\mu\nu}
\end{equation}
