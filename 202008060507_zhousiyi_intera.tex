% 引力波和测试质量的相互作用
这一节我们讲引力波和探测器的相互作用。在理想情况下,我们把探测器看作一系列测试质量。在广义相对论里面,数学上选择一个规范对应于物理上选择一个特定的观测者。在前面的学习中我们已经知道了在TT规范下,引力波具有非常简单的形式,所以现在我们想要懂得TT规范对应的是什么参考系。我们也将学习到,在探测器共动参考系下,描述探测器上的引力波将更加直观。因此,在我们讨论引力波和探测器的相互作用的时候,知道我们使用的是哪个参考系是非常重要的事情。

理解一个特定的规范下的物理意义,我们需要使用两个工具:测地线方程和测地线偏移方程。

\subsection{测地线方程和测地线偏移}
考虑一个含参数$\lambda$的曲线$x^\mu(\lambda)$.两个相距$d\lambda$的点的时空间隔是
\begin{equation}
\begin{aligned}
ds^2 & = g_{\mu\nu}  dx^\mu dx^\nu \\
& = g_{\mu\nu } \frac{dx^\mu}{d\lambda} \frac{dx^\nu}{d\lambda} d \lambda^2~.
\end{aligned}
\end{equation}
沿着类空曲线,我们有$ds^2>0$,所以我们可以直接开平方
\begin{equation}
ds = (g_{\mu\nu } dx^\mu dx^\nu )^{1/2}~. 
\end{equation}
这是曲线固有距离的一个量度。一个类时的曲线有$ds^2<0$。在这种情况下,我们可以定义固有时间$\tau$
\begin{equation}
c^2 d \tau^2  = - ds^2 = - g_{\mu\nu} dx^\mu dx^\nu ~. 
\end{equation}



