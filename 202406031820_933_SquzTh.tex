% 夹逼定理
% keys 极限|函数|数列|序列
% license Usr
% type Tutor

\begin{issues}
\issueDraft
\end{issues}
\pentry{极限定义 \nref{nod_Lim},极限运算法则 \nref{nod_LimOp}}{nod_014c}
\textbf{夹逼定理}(Squeeze Theorem,或称夹挤定理)

之前在学习重要极限(参见\autoref{ex_FunLim_1}~\upref{FunLim})时,只是形象地将这个概念给出,事实上,它可以由夹逼定理得到。
\begin{example}{证明:$\lim_{x\to0}{\sin(x)\over x}=1$}
构造函数$g(x)=\cos(x)$,$f(x)={\sin(x)\over x}$,$h(x)=1$,
\end{example}

\subsection{数列极限}

\subsection{函数极限}

若$g(x)\leq f(x)\leq h(x)$,且$\lim _{x\to x_0}h(x)=\lim _{x\to x_0}g(x)=a$,则
\begin{equation}
\lim _{x\to x_0}f(x)=a~.
\end{equation}

\begin{lemma}{不等式一端为0时的极限}\label{lem_SquzTh_1}
\begin{equation}
0\leq f(x)\leq h(x),\lim _{x\to x_0}h(x)=0\implies\lim _{x\to x_0}f(x)=0~.
\end{equation}

证明:

由函数极限定义,$\forall\varepsilon>0$,$\exists\delta>0$,$0<|x-x_0|<\delta\implies|h(x)|<\varepsilon$。

由于$0\leq f(x)\leq h(x)$,所以对$0<|x-x_0|<\delta$,有$|f(x)|\leq |h(x)|<\varepsilon$,即$|f(x)|<\varepsilon$。根据极限定义可知:$\lim _{x\to x_0}f(x)=0$。

\end{lemma}

证明:

由$g(x)\leq f(x)\leq h(x)$,有$0\leq f(x)-g(x)\leq h(x)-g(x)$。为了求取$\lim _{x\to x_0}(f(x)-g(x))$,需要使用\autoref{lem_SquzTh_1} ,下面求$\lim _{x\to x_0}(h(x)-g(x))$。
$$
\begin{align*}
\lim _{x\to x_0}(h(x)-g(x))&\overset{\mathrm{1}}{=}\lim _{x\to x_0}h(x)-\lim _{x\to x_0}g(x)\\
&=a-a\\
&=0~.
\end{align*}
$$

因此,由\autoref{lem_SquzTh_1} 可知$\lim _{x\to x_0}(f(x)-g(x))=0$。现在已经求得了$\lim _{x\to x_0}(f(x)-g(x))$,又已知$\lim _{x\to x_0}g(x)$,剩下的就是:
$$
\begin{align*}
\lim _{x\to x_0}f(x) &= \lim _{x\to x_0}[(f(x)-g(x))+g(x)] \\ 
&\overset{\mathrm{1}}{=} \lim _{x\to x_0}(f(x)-g(x))+\lim _{x\to x_0}g(x)\\ 
&= 0+a\\ 
&=a~.\end{align*}
$$
上面的推导过程中,所有等号1的地方都使用了极限运算法则的\autoref{the_LimOp_1}~\upref{LimOp}。

证毕。

\subsection{使用技巧}

在面对不太容易通过极限运算得到结果的极限时,尤其是涉及到复合、求和等场景或存在比较显然的不等关系时,记得使用夹逼定理,可以收获奇效。事实上,在使用夹逼定理时,其实是把复杂的对极限的运算转移到了构造不等式上来。而且,就像乱拳打死老师傅,由于不需要考虑中间的极限是否存在,也能够避免一些需要讨论存在性的场景,避免出错(说的就是你,洛必达法则)。

当然,这里对于不等式的构造会有一定的要求,所以下面会提供一些常用的不等关系,一起服用,效果极佳。

\begin{itemize}
\item $\sin x < x < \tan x, \quad x \in \left(0, \frac{\pi}{2}\right)$
\item $\sin x \leq x, \quad x \in (0, +\infty)$
\item $\arctan x \leq x \leq \arcsin x, \quad x \in [0, 1] $
\item $x + 1\leq e^x$
\item $\ln x\leq x - 1 , \quad x \in (0, +\infty)$
\item $\frac{1}{1 + x} \leq \ln\left(1 + \frac{1}{x}\right) \leq \frac{1}{x}, \quad x \in (0, +\infty)$
\item $\sqrt{xy} \leq \frac{x + y}{2} \leq \sqrt{\frac{x^2 + y^2}{2}}, \quad (x, y > 0)$
\end{itemize}

\begin{example}{\addTODO{应用上述不等式的例题}}

\end{example}
另外,在处理求和的极限时,将与n相关的分子或分母进行放大或缩小,使其能够消去或易于计算,也是一个好方法。
\begin{example}{$\lim_{n\to \infty}\sum_{i=1}^n{i\over n^2+n+i}$}
\addTODO{求和的例题}
\end{example}