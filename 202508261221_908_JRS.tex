% 卷绕数(综述)
% license CCBYSA3
% type Wiki

本文根据 CC-BY-SA 协议转载翻译自维基百科\href{https://en.wikipedia.org/wiki/Winding_number}{相关文章}。

\begin{figure}[ht]
\centering
\includegraphics[width=6cm]{./figures/801932f041ea8a31.png}
\caption{这条曲线相对于点 $p$ 的绕数是2。} \label{fig_JRS_1}
\end{figure}
在数学中,闭合曲线相对于平面中某一点的绕数或绕线指数是一个整数,表示该曲线绕该点逆时针方向环绕的总次数,也就是曲线的“转数”。对于某些非闭合的平面曲线,其绕数可能是非整数。绕数依赖于曲线的方向:如果曲线沿顺时针方向绕点运动,则绕数为负数。

绕数是代数拓扑中的基础研究对象,并且在向量分析、复分析、几何拓扑、微分几何以及物理学(例如弦理论)中都扮演着重要角色。
\subsection{直观描述}
沿着红色曲线运动的一个物体,会绕位于原点的人逆时针转两圈。
假设我们有一条位于 $xy$ 平面上的**闭合、有方向的曲线**。我们可以把这条曲线想象成某个物体的运动轨迹,而曲线的方向表示物体运动的方向。这样,这条曲线的**绕数**就等于该物体**绕原点逆时针转的总圈数**。

在计算总圈数时,**逆时针**的运动计为**正数**,而**顺时针**的运动计为**负数**。例如,如果一个物体先绕原点逆时针转了四圈,然后又绕原点顺时针转了一圈,那么这条曲线的总绕数就是**3**。

按照这个规则,一条完全没有绕过原点的曲线,其绕数为**0**;而一条绕原点**顺时针**运动的曲线,其绕数为**负数**。因此,曲线的绕数可以是**任意整数**。下面的图示展示了绕数从 **−2 到 3** 的不同曲线。
\begin{figure}[ht]
\centering
\includegraphics[width=6cm]{./figures/f02fc89a7a934eb5.png}
\caption{} \label{fig_JRS_2}
\end{figure}