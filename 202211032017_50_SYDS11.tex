% 中山大学 2011 年913专业基础(数据结构)考研真题

\subsection{一、单项选择题(每题2分,共40分)}

1.算法复杂度通常是表达算法在最坏情况下所需要的计算量,$O(1)$的含义是( ) \\
(A).算法执行1步就完成 \\
(B).算法执行1秒钟就完成 \\
(C).解决执行常数步就完成 \\
(D).算法执行可变步数就完成

2.在数据结构中,按逻辑结构可把数据结构分为( ) \\
(A).静态结构和动态结构 \\
(B).线性结构和非线性结构 \\
(C).顺序结构和链式结构 \\
(D).内部结构和外部结构

3.在数据结构中,可用存储顺序代表逻辑顺序的数据结构为( ) \\
(A). Hash表 \\
(B).二叉搜索树 \\
(C).链式结构 \\
(D).顺序结构

4. 对链式存储的正确描述是( ) \\
(A).结点之间是连续存储的 \\
(B).各结点的地址由小到大 \\
(C).各结点类型可以不一致 \\
(D).结点内单元是连续存储的

5. 在下列关于“串”的陈述中,正确的说明是( ) \\
(A).串是一种特殊的线性表 \\
(B).串中元素只能是字母 \\
(C).串的长度必须大于零 \\
(D).空串就是空白串

6.关于堆栈的正确描述是( ) \\
(A). FILO \\
(B). FIFO \\
(C).只能用数组来实现 \\
(D).可以修改栈中元素的数据

7. 假设循环队列的长度为QSize. 当队列非空时,从其队列头取出数据后,其队头下标Front的变化为() \\
(A). Front = Front+ 1 \\
(B). Front = (Front + 1) \% 100 \\
(C). Front = (Front+ 1) \% QSize \\
(D). Front = Front \% Qsize + 1

8.假设Head是带头结点单向循环链的头结点指针,判断其为空的条件是( )
(A). Head.next = NULL
(B). Head~>next == Head
(C). Head->next = NULL
(D). Head = NULL
