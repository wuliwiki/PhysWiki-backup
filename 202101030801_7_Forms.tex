% 微分形式
% keys 微分形式|微分几何|外微分|form|1-形式|2-形式|张量|流形|切向量|余切向量|切空间|余切空间

\pentry{对偶空间\upref{DualSp},流形上的切空间\upref{tgSpa}}

\subsection{1-形式}

在流形$M$上给定一个光滑函数$f$和点$p\in M$处的一个切向量$\bvec{v}$,则由切向量的道路定义容易计算出$\bvec{v}f(p)$,这个值不依赖于图的选择.不过为了方便理解,既然这是一点处的性质,我们不妨取$p$所在的一个图来考虑,这样就把问题范围从任意流形$M$化为任意欧几里得空间了.

在欧几里得空间$\mathbb{R}^n$中,$\bvec{v}f(p)=\bvec{v}\cdot\nabla f$.我们可以看成是给定$f$,任意的向量$\bvec{v}$都会被映射为$\bvec{v}\cdot\nabla f$,这是一个线性映射.不同的$f$有可能对应相同的$\nabla f$,只要它们全微分的形式是相同的就可以——由此诞生了“微分形式”的概念.

我们用$\mathbb{R}$上的光滑函数举一个例子.考虑函数$f(x)=x+1$和$g(x)=\E^x$.在$x=0$处,它们的函数值是相同的,斜率也相同,如图\autoref{Forms_fig1} 所示.这样一来,在这个点附近,忽略高阶无穷小的情况下,这两个函数是无法进行区分的.如果两个函数表示两个斜面,那么当你站在$x=0$这一点上的时候,无论是你手表上的高度计读数还是倾斜的感觉,都无法告诉你自己到底站在函数$f$上还是$g$上.

\begin{figure}[ht]
\centering
\includegraphics[width=12cm]{./figures/Forms_1.pdf}
\caption{函数$f=x+1$和$g=\E^x$的例子.} \label{Forms_fig1}
\end{figure}





