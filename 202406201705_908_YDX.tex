% 运动学
% license CCBYSA3
% type Wiki

(本文根据 CC-BY-SA 协议转载自原搜狗科学百科对英文维基百科的翻译)

\textbf{运动学}是经典力学的一个分支,描述了点、体(对象)和体系统(对象组)的运动,而不考虑引起运动的力。[1][2][3]运动学作为一个研究领域,通常被称为“运动的几何”,偶尔也被视为数学的一个分支。[4][5][6]运动学问题首先描述系统的几何形状,并声明系统内任何已知点的位置、速度和/或加速度值的初始条件。然后,使用几何参数,可以确定系统任何未知部分的位置、速度和加速度。对力如何作用于物体的研究属于动力学范畴,而不是运动学。有关更多详细信息,请参见分析动力学。

运动学在天体物理学中用于描述天体的运动和这些天体的集合。在机械工程、机器人学和生物力学中[7]运动学用于描述由连接部件(多连杆系统)组成的系统的运动,例如发动机、机械臂或人体骨骼。

几何变换,也称为刚性变换,用于描述机械系统中部件的运动,简化了运动方程的推导。它们也是动态分析的核心。

运动学分析是测量用于描述运动的运动学量的过程。例如,在工程中,运动学分析可用于找到给定机构的运动范围,并反向工作,使用运动学综合来设计所需运动范围的机构。[8] 此外,运动学将代数几何学应用于研究机械系统或机构的机械效益。

\subsection{词源}



\subsection{非旋转参照系中粒子轨迹的运动学}



\subsubsection{2.1 速度和速度}



\subsubsection{2.2 加速度}



\subsubsection{2.3 相对位置向量}



\subsubsection{2.4 相对速度}



\subsubsection{2.5 相对加速度}



\subsection{恒定加速度下的粒子轨迹}



\subsection{圆柱-极坐标中的粒子轨迹}



\subsubsection{4.1 恒定半径}



\subsubsection{4.2 平面圆形轨迹}



\subsection{物体在平面内运动的点轨迹}



\subsubsection{5.1 位移和运动}



\subsubsection{5.2 矩阵表示}



\subsection{纯平移}



\subsection{ 物体围绕固定轴的旋转}



\subsection{物体三维运动的点轨迹}



\subsubsection{8.1 位置}



\subsubsection{8.2 速度}



\subsubsection{8.3 加速度}



\subsection{运动约束}



\subsubsection{9.1 运动耦合}



\subsubsection{9.2 滚动而不打滑}



\subsubsection{9.3 不可拉伸的绳索}



\subsubsection{9.4 运动副}



\textbf{较低副}



\textbf{较高副}



\subsubsection{9.5 运动链}



\textbf{实例}



\subsection{参考文献}

[1]
^Edmund Taylor Whittaker (1904). A Treatise on the Analytical Dynamics of Particles and Rigid Bodies. Cambridge University Press. Chapter 1. ISBN 0-521-35883-3..

[2]
^Joseph Stiles Beggs (1983). Kinematics. Taylor & Francis. p. 1. ISBN 0-89116-355-7..

[3]
^Thomas Wallace Wright (1896). Elements of Mechanics Including Kinematics, Kinetics and Statics. E and FN Spon. Chapter 1..

[4]
^Russell C. Hibbeler (2009). "Kinematics and kinetics of a particle". Engineering Mechanics: Dynamics (12th ed.). Prentice Hall. p. 298. ISBN 0-13-607791-9..

[5]
^Ahmed A. Shabana (2003). "Reference kinematics". Dynamics of Multibody Systems (2nd ed.). Cambridge University Press. ISBN 978-0-521-54411-5..

[6]
^P. P. Teodorescu (2007). "Kinematics". Mechanical Systems, Classical Models: Particle Mechanics. Springer. p. 287. ISBN 1-4020-5441-6.。.

[7]
^A. Biewener (2003). Animal Locomotion. Oxford University Press. ISBN 019850022X..

[8]
^J.M. McCarthy和G. S. Soh,2010年,连杆的几何设计,纽约斯普林格。.

[9]
^Ampère, André-Marie. Essai sur la Philosophie des Sciences. Chez Bachelier..

[10]
^Merz, John (1903). A History of European Thought in the Nineteenth Century. Blackwood, London. p. 5..

[11]
^O. Bottema & B. Roth (1990). Theoretical Kinematics. Dover Publications. preface, p. 5. ISBN 0-486-66346-9..

[12]
^Harper, Douglas. "cinema". Online Etymology Dictionary..

[13]
^https://web.archive.org/web/20221025164404/https://www.youtube.com/watch?v=jLJLXka2wEM 速成物理.

[14]
^https://web.archive.org/web/20221025164404/https://www.youtube.com/watch?v=jLJLXka2wEM 速成物理积分.

[15]
^https://web.archive.org/web/20221025164404/https://duckduckgo.com/?q =面积+的+a+矩形& ampatb=v92-4_g&amp。ia DuckDuckGo.

[16]
^https://web.archive.org/web/20221025164404/http://www . mathsisfun . com/代数/trig-area-triangle-with-not-right-angle . html没有直角的三角形面积.

[17]
^https://web.archive.org/web/20221025164404/https://www 4 . uwsp . edu/phys tar/kmen ning/phys 203/eqs/运动学. gif.

[18]
^几何学:对给定元素在特定变换下保持不变的性质的研究。"Definition of geometry". Merriam-Webster on-line dictionary..

[19]
^Paul, Richard (1981). Robot manipulators: mathematics, programming, and control : the computer control of robot manipulators. MIT Press, Cambridge, MA. ISBN 978-0-262-16082-7..

[20]
^R. Douglas Gregory (2006). Chapter 16. Cambridge, England: Cambridge University. ISBN 0-521-82678-0..

[21]
^William Thomson Kelvin & Peter Guthrie Tait (1894). Elements of Natural Philosophy. Cambridge University Press. p. 4. ISBN 1-57392-984-0..

[22]
^William Thomson Kelvin & Peter Guthrie Tait (1894). Elements of Natural Philosophy. p. 296..

[23]
^M. Fogiel (1980). "Problem 17-11". The Mechanics Problem Solver. Research & Education Association. p. 613. ISBN 0-87891-519-2..

[24]
^Irving Porter Church (1908). Mechanics of Engineering. Wiley. p. 111. ISBN 1-110-36527-6..

[25]
^Morris Kline (1990). Mathematical Thought from Ancient to Modern Times. Oxford University Press. p. 472. ISBN 0-19-506136-5..

[26]
^Phillips, Jack (2007). Freedom in Machinery, Volumes 1–2 (reprint ed.). Cambridge University Press. ISBN 978-0-521-67331-0..

[27]
^Tsai, Lung-Wen (2001). Mechanism design:enumeration of kinematic structures according to function (llustrated ed.). CRC Press. p. 121. ISBN 978-0-8493-0901-4..