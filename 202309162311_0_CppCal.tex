% C++ 的可调用类型
% license Usr
% type Tutor


\textbf{可调用类型(Callable)}

最经典的是 C 语言的函数指针\upref{CNote}。 C++ 还引入了函数引用,用法和指针类似(\verb|*| 换成 \verb|&| 即可)。

\subsection{lambda 函数}
lambda 函数是 C++11 引入的。 除了语法更简洁, 好处主要在于可以修改函数的接口, 预先填装一些参数或者改变参数的顺序或类型。 例如一个积分函数 \verb|integral(T integrand, double start, doube end)| 若要求 \verb|integrand| 只接收一个 \verb|double| 返回一个 \verb|double|, 但我们想要积分的函数还需要其他若干参数。 这时就可以直接 capture: \verb|auto integral = [&param1, &param2](double x) { return fun1(param1, param2, x); }|。 其中 \verb|&| 是 capture by reference。 也就是说 lambda 定义以后,仍然可以改变它们的值。 不加 \verb|&| 就是 capture by value, 复制定义时的值。

\begin{itemize}
\item lambda 函数的 \verb|sizeof()| 通常是函数指针的函数, 但如果有 capture, 就会相应增加每个 capture 变量的 \verb|sizeof|。
\item 普通 capture 的参数默认都是 \verb|const|, 如果想改变可以用 \verb|[...](...) mutable {...}| 但这不会影响被 capture 参数在函数外的值。 capture by reference 的参数, 在 lambda 函数中可以修改, 修改后也会影响它在 lambda 外面的值。
\item lambda 的返回值只能有一种类型, 它会自动推导返回值的类型。 如果要明确指定返回类型, 用 \verb|[...](...) -> 类型 {...}|, 此时返回值会试图 cast 成该类型。
\item 从 C++14 开始, lambda 的参数类型可以是 \verb|auto| 或者 \verb|auto &|。 这样即使是同一个 lambda 对象, 返回值的类型也可以由不同的输入值决定。 相当于普通的函数模板声明返回 \verb|auto|。
\end{itemize}

\subsection{functor}
functor 也可以达到 lambda 一样的效果。 可以把 \verb|param1, param2| 都用数据成员储存, 在 \verb|operator()| 中调用。 但比起 lambda 还是略嫌麻烦。

\begin{itemize}
\item 函数中可以定义 class, 且 class 的成员函数可以访问函数中的 \verb|static| 变量。
\end{itemize}
