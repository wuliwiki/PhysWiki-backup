% 量子力学诠释
% license CCBYSA3
% type Wiki

(本文根据 CC-BY-SA 协议转载自原搜狗科学百科对英文维基百科的翻译)

所谓\textbf{量子力学的解释}是试图用数学的方式将量子力学与现实“对应”起来。尽管量子力学在广泛的实验中经受住了严格和极其精确的考验(所有量子力学的预测都与实验相符),但是对于量子力学的解释存在着许多不同的思想流派。这些流派在诸如量子力学是确定的还是随机的、量子力学的哪些量可以被认为是“真实的”、测量的本质是什么等基本问题的观点上有所不同。

尽管进行了近一个世纪的辩论和实验,物理学家和哲学家们还没有就哪种解释最能“代表”现实达成共识。

\subsection{历史}

\begin{figure}[ht]
\centering
\includegraphics[width=6cm]{./figures/3ef8c507a5bfcd47.png}
\caption{薛定谔} \label{fig_QMinet_1}
\end{figure}

\begin{figure}[ht]
\centering
\includegraphics[width=6cm]{./figures/33a6c91157a45fe2.png}
\caption{玻姆} \label{fig_QMinet_2}
\end{figure}

量子力学解释中具有影响力的人物

薛定谔
玻姆
量子物理学家所使用的术语的定义,如\textbf{波函数}和\textbf{矩阵力学},经历了许多变化阶段。例如,埃尔温·薛定谔最初认为电子的波函数是它在空间中的电荷密度,而马克斯·玻恩将波函数的绝对值的平方重新解释为电子在空间中的概率密度。

如尼尔斯·玻尔和维尔纳 ·海森堡这样的早期量子力学先驱的观点,经常被归为“哥本哈根解释”,尽管物理学家和物理史学家认为这个术语掩盖了观点之间的差异。虽然哥本哈根式的想法从未被普遍接受,但对哥本哈根正统观念的挑战在20世纪50年代随着大卫·玻姆的先导波解释和休·艾弗雷特三世的多世界解释的兴起而日益受到关注。[2][3] 此外,试图回避区分解释学派的严苛主义立场受到了可证伪实验的挑战,这些实验可能有朝一日可以利用诸如人工智能测量或量子计算来区分不同的解释学派。[4] [5]

物理学家大卫·梅尔明曾经打趣道:“新的解释学派每年都会出现。但是没有哪个解释学派会消失。”作为20世纪90年代至21世纪初主流观点发展的粗略指南,可以参考一下施洛斯豪尔等人在2011年7月“量子物理和现实本质”会议上的民意调查中收集的“快照”。[6]作者引用了马克斯·泰格马克在1997年8月“量子中的基本问题”会议上进行的一项类似的非正式投票调查。作者的主要结论是,“除了哥本哈根的解释仍然是占据很大优势,在他们的投票中获得最多的选票(42\verb|%| ),多世界解释也得到了显著上升:

“哥本哈根解释在这里仍然是有着巨大的优势,特别是如果我们把它和以此产生相关解释(如基于信息的解释和量子贝叶斯解释)算在一起。在泰格马克的民意测验中,埃弗雷特的解释获得了17\verb|%|的选票,这与我们民意测验中的票数(18\verb|%|)相似。"
值得注意的是,只有克莱默在1986年发表的事务性解释,才为梅克斯·玻恩的断言赋予物理基础,即波函数的绝对平方是概率密度。[7]

\subsection{本质}

或多或少,量子力学的所有解释都有两个特质:

\begin{enumerate}
\item  解释的方式都具有——有一系列的方程和规则,以及通过对其输入初始条件来进行预测——这种\textbf{形式}。
\item 解释都是——有一系列的观察结果,包括通过实证研究获得的观察结果和非正式获得的观察结果,例如人类在这个客观世界中获得的经验——这样的\textbf{唯象学}。
\end{enumerate}

不同的解释有两种类型:

\begin{enumerate}
\item 本体论——主张世界上\textbf{存在}什么东西,例如类别和实体
\item 认知论——主张对世界相关知识的可能性、范围和手段的\textbf{认识}
\end{enumerate}

在科学哲学中,认识与现实的区别被称为\textbf{认知论}与\textbf{本体论}。普遍的规律是结果(认知)的\textbf{规律性},而因果机制可以\textbf{调节}结果(本体)。可以从本体角度解释一个现象,也可以从认知角度解释一个现象。例如,不确定性可能归因于人类观察和感知的局限性(认知),或者也许是一个宇宙中的真实存在\textbf{可能}编码的结果(本体)。混淆认知论和本体论,就像假设一个一般性法则实际上“支配”了结果——但是对于规律的表述却具有因果机制的效应——是一个范畴性错误。

从广义上讲,科学理论可以看作是科学实在论——近乎真实地描述或解释自然世界——或者可以看作为反实在论。实在论的立场寻求认知论和本体论,而反实在论的立场则寻求认知论而非本体论。在20世纪上半叶,反实在论基本上是逻辑实证主义,它试图将科学理论中现实中不可观察的方面排除出去。

自20世纪50年代以来,反实在论变得更加温和,某种程度上可以算是工具论,允许谈论不可观察的方面,但最终抛弃了实在论中的这个关键问题,并把科学理论看作一种帮助人类做出预测工具,而不是获得对世界的形而上学理解。工具论的观点是由大卫·梅明的名言“闭嘴,计算”所承载的,这句名言经常被误认为是理查德·费曼所言。[8]

为了解决一些概念问题,其他解释方法引入了新的数学形式,因此也发展出了其他的理论及其解释。一个例子就是波希米亚力学,它与三种标准形式——薛定谔的波动力学、海森堡的矩阵力学和费曼的路径积分——都是等价的,并且已被验证。

\subsection{挑战}

\begin{enumerate}
\item 抽象,量子场论的数学本质:量子力学的数学结构在数学上是抽象的,没有对其变量的明确定义。
\item 明显不确定和不可逆过程的存在:在经典场论中,在场中给定位置的物理性质很容易推导出来。在量子力学的大多数数学公式中,测量在理论中扮演了不同寻常的角色,因为它是唯一能导致状态不均匀、不可逆演化的过程。
\item 观察者在结果中的决定作用:哥本哈根解释中波函数是一个计算工具,并且仅代表在测量之后那一刻的现实,可以由观察者得到;埃弗里特的解释承认所有的可能性都是真实的,并且可测量相互作用会导致有效的分支过程。[9]
\item 远距离物体之间意想不到的关联:纠缠量子系统,如EPR悖论所示,服从统计却破坏局域因果关系。[10]
\item 描述的互补性:互补原理认为没有哪一组经典物理概念可以同时适用于量子系统的所有性质。例如,波形式的描述A和粒子形式的描述B可以各自描述量子系统S,但不能同时描述。这意味着当使用命题连接词时,S的物理性质的组成不遵守经典命题逻辑的规则。与量子互文性类似,“互补原理的起源在于所描述的量子对象的算符的非对易性”(Omnès 1999) 。
\item 随着系统规模的增加,复杂性迅速上升,远远超过了人类目前的计算能力:因为量子系统的态空间发展到子系统是数量上是指数增长的,所以很难得到经典近似。
\item 系统的定域互文行为:量子互文性证明了系统的物理量有确定的值,这与它们的测量方式无关的这种经典经验直觉即使对定域系统也是失败的。此外,物理原理,如莱布尼茨的不可分辨的同一性原理,也不再适用于量子领域,这说明大多数经典直觉对量子世界可能是不正确的。
\end{enumerate}
