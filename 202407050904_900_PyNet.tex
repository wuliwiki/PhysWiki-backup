% Python 网络编程笔记
% license Xiao
% type Note

\begin{itemize}
\item 参考\href{https://www.tutorialspoint.com/python/python_networking.htm}{这个教程}。
\item 基本概念: Socket Programming
\item Sockets are the endpoints of a bidirectional communications channel. Sockets may communicate within a process, between processes on the same machine, or between processes on different continents.
\item 安装 python \verb`socket` 模块: \verb`pip3 install sockets` (注意后面有 s)
\end{itemize}

\subsection{一个简单的 python 网络程序}
\begin{itemize}
\item 以阿里云为例, 在服务器上, 先用 \verb`ifconfig` 查看 ip 是多少, 这个 ip 未必是公网 ip
\item 运行 \verb`python3 -m http.server --bind 上面的ip地址 端口号如9000`
\item 现在就可以在任意电脑浏览器输入 \verb`公网ip:端口号`, 浏览用户文件夹的文件了。
\end{itemize}

在服务器自己写一个简单的 server.py, 注意这里的 ip 必须是 ifconfig 里面显示的, 而不是 127.0.0.1 或者公网 ip。 用 \verb`python3 server.py` 运行。
\begin{lstlisting}[language=python, caption=server.py]
import socket               # Import socket module

s = socket.socket()         # Create a socket object
host = "172.19.190.150"   # socket.gethostname() # Get local machine name
port = 12345                # Reserve a port for your service.
s.bind((host, port))        # Bind to the port

s.listen(5)                 # Now wait for client connection.
while True:
   c, addr = s.accept()     # Establish connection with client.
   print('Got connection from', addr)
   msg = 'Thank you for connecting'
   c.send(msg.encode())
   c.close()                # Close the connection
\end{lstlisting}

再在自己电脑写一个简单的 client.py, 用 \verb`python3 client.py` 运行。
\begin{lstlisting}[language=python, caption=client.py]
import socket               # Import socket module

s = socket.socket()         # Create a socket object
host = "47.254.67.252" # socket.gethostname() # Get local machine name
port = 12345                # Reserve a port for your service.

s.connect((host, port))
print(s.recv(1024).decode())
s.close()
\end{lstlisting}

这时服务器的命令行就会显示 \verb`Got connection from ('120.79.212.166', 45286)`, 自己的电脑会显示 \verb`Thank you for connecting`。

% 已经在阿里云和 Bandwagon 测试成功。
