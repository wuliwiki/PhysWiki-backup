% 泡利方程
% keys Pauli 方程|泡利方程|薛定谔方程|电磁场

\pentry{自旋 1/2 粒子的非相对论波函数\upref{scheq2},电磁场中的薛定谔方程及规范变换\upref{QMEM}}
\subsection{从自旋 $1/2$ 粒子的非相对论方程到泡利方程}
自旋 $1/2$ 粒子的非相对论性波函数具有 $4$ 个分量,可以表示为 $\psi=\pmat{\phi\\\chi}$,其中 $\phi,\chi$ 为双分量的波函数,且满足方程\autoref{scheq2_eq3}~\upref{scheq2}:
\begin{equation}
\begin{aligned}
&\qty[\hat H-\frac{(\bvec \sigma\cdot \hat{\bvec P})^2}{2m}]\phi=0\\
&
\qty[\hat H-\frac{(\bvec \sigma\cdot \hat{\bvec P})^2}{2m}]\chi=0
\end{aligned}
\end{equation}
在电磁场中,需要对 $\hat H,\hat{\bvec P}$ 作如下的替换:
\begin{equation}
\hat H\rightarrow \hat H-e\phi, \hat{\bvec P}\rightarrow \hat{\bvec P}-e\bvec A
\end{equation}
其中 $\phi$ 为电势,$\bvec A$ 为磁矢势.那么
\begin{equation}
\qty[\bvec \sigma\cdot(\hat{\bvec P}-e\bvec A)]^2=(\hat{\bvec P}-e\bvec A)^2+i\bvec \sigma\cdot[(\hat{\bvec P}-e\bvec A)\times (\hat{\bvec P}-e\bvec A)]
\end{equation}
注意这里的 $\hat \sigma,\hat{\bvec P},\bvec A$ 都应当被视为作用于 Hilbert 空间上的算符,这个 Hilbert 空间中的“矢量”是如同 $\phi,\chi$ 那样的二分量波函数.因此算符间不一定满足交换律,所以右边的两个算符的“叉积”不为 $0$.计算可得
\begin{equation}
\begin{aligned}
\qty[\bvec \sigma \cdot (\hat{\bvec P}-e\bvec A)]^2
&=(\hat{\bvec P}-e\bvec A)^2+i\bvec \sigma\cdot[-e(\hat{\bvec P}\times \bvec A- \bvec A\times \hat{\bvec P})]\\
&=(\hat{\bvec P}-e\bvec A)^2+i\bvec \sigma\cdot[ei\hbar \nabla \times  \bvec A]\\
&=(\hat{\bvec P}-e\bvec A)^2-e\hbar\cdot \bvec B
\end{aligned}
\end{equation}
于是有泡利方程
\begin{equation}
\qty[(\hat H-e\phi)-\frac{1}{2m}(\hat{\bvec P}-e\bvec A)^2+\frac{e}{2m}\hbar\bvec \sigma\cdot \bvec B]
\end{equation}
