% 线性变换与矩阵的代数关系
% keys 线性变换|矩阵|代数同构

在本页中,我们将所有 $m$ 行 $n$ 列的实矩阵组成的集合记为 $\mathbb{R}^{m\times n}.$

并记从 $\mathbb{R}^{n}$ 到 $\mathbb{R}^{m}$ 的一切线性变换全体构成的空间为 $L(\mathbb{R}^{n},\mathbb{R}^{m}).$
\begin{itemize}
\item $\mathbb{R}^{m\times n}$ 中的矩阵与 $L(\mathbb{R}^{n},\mathbb{R}^{m})$
中的线性变换是代数同构的关系. 矩阵可以看成线性变换, 线性变换在取定一组基底后, 可以表示为矩阵. 
\end{itemize}

\begin{definition}{(矩阵的秩)}
\textbf{矩阵的秩} 既可以定义为其行向量组的秩 (称为\textbf{行秩}), 也可以定义为其列向量组的秩 (称为\textbf{列秩}),
这是因为矩阵的行秩和列秩被证明是相等的. \\

\textsl{注}:矩阵行向量组形成的空间称为\textbf{行空间}, 行向量组的秩也正是行空间的维数;
列向量组形成的空间称为\textbf{列空间}, 列向量组的秩也正是列空间的维数.
\end{definition}



\begin{definition}{(线性变换的秩)}
\textbf{线性变换的秩} 定义为其像空间的维数.
\end{definition} 
\verb| |

下面的定理表明,矩阵与线性变换的秩在某种意义下是等同的.
\begin{theorem}{}
 如果将矩阵看成线性变换, 那么该变换的像空间的维数, 恰是矩阵的列空间的维数.\\

\textsl{ 证明}:设 $A\in\mathbb{R}^{m\times n}$, $A$ 也可以看成从 $\mathbb{R}^{n}$ 到 $\mathbb{R}^{m}$
的线性变换, 那么取定 $\mathbb{R}^{n}$ 的标准基 $(e_{1},e_{2},\ldots,e_{n})$ 后,
$A$ 的像空间就由 $\{Ae_{1},Ae_{2},\ldots,Ae_{n}\}$ 张成, 因此像空间的维数就是 $\{Ae_{1},Ae_{2},\ldots,Ae_{n}\}$
的极大无关组的个数, 也就是 $\{Ae_{1},Ae_{2},\ldots,Ae_{n}\}$ 的秩; 而每个 $Ae_{i}$
正好又是矩阵 $A$ 的第 $i$ 列, 这就是说明了 $A$ 的像空间的维数等于它的列空间的维数. 
\end{theorem}

\textsl{注}:上述证明过程表明,对于看成线性变换的矩阵来说,其像空间正好就是其列空间. 

\begin{itemize}
\item 矩阵是行满秩的, 当且仅当其线性变换是满射. (\textbf{行秩等于列秩,列空间等于像空间})
\end{itemize}

\begin{itemize}
\item 矩阵是列满秩的, 当且仅当其线性变换是单射. (\textbf{列空间等于像空间,维数定理,核空间为零})
\end{itemize}

\begin{itemize}
\item 方阵是满秩的, 当且仅当它是可逆的, 当且仅当其线性变换是双射. \end{itemize}
\verb| |

\begin{theorem}{}
若 $A\in\mathbb{R}^{n\times m}$, $B\in\mathbb{R}^{m\times n}$, 且 $AB=I_n$, 则 $A$作为线性算子是满射,$B$作为线性算子是单射.

\textsl{证明}:先证 $A$ 是满射. 由已知得 $B\in L(\mathbb{R}^{n},\mathbb{R}^{m})$, $A\in L(\mathbb{R}^{m},\mathbb{R}^{n})$.
若 $A$ 不满, $\mathrm{Im}A$ 真包含于 $\mathbb{R}^{n}$ 中, 但是 $AB=I_{n}$
说明 $\dim(\mathrm{Im}A)=n$, 矛盾. 

再证 $B$ 是单射. 若 $Bx=0$, 则 $x=I_{n}x=(AB)x=A(Bx)=A0=0$, 说明 $\mathrm{Ker}B=\{0\}$.
证毕. 
\end{theorem}



\begin{itemize}
\item 矩阵有右逆, 当且仅当其线性变换是满射. \end{itemize}

\begin{itemize}
\item 矩阵有左逆, 当且仅当其线性变换是单射. 
\end{itemize}

未完待续