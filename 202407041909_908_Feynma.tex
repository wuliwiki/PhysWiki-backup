% 理查德·费曼
% license CCBYSA3
% type Wiki

(本文根据 CC-BY-SA 协议转载自原搜狗科学百科对英文维基百科的翻译)

理查德·菲利普斯·费曼(1918年5月11日-1988年2月15日),美国理论物理学家,以在量子力学的路径积分表述、量子电动力学理论、液体过冷氦的超流态物理学以及在粒子物理学领域提出的部分子模型等工作而闻名。由于对量子电动力学发展的贡献,费曼于1965年与朱利安·施温格和朝永振一郎共同获得诺贝尔物理学奖。

费曼为描述亚原子粒子行为的数学表达式开发了被广泛使用的图形表示方案,即后来的费曼图。在他一生中,费曼是世界上最著名的科学家之一。英国杂志 《物理世界》(Physics World )在1999年对全球130名顶尖物理学家进行了一项民意调查,费曼被列为有史以来十大物理学家之一。[1]

费曼在第二次世界大战期间协助设计了原子弹,并在1980年代作为罗杰斯委员会(调查航天飞机“挑战者号”事故的专家小组)的成员而广为人知 。随着在理论物理方面的工作,费曼被认为是量子计算领域的先驱,他还提出了纳米技术的概念。费曼在加州理工学院获得了理查德·C·托尔曼理论物理学教授职位。

费曼通过出书和讲座,包括1959年关于自上而下纳米技术,名为 There's Plenty of Room at the Bottom(底部有足够的空间)的演讲,以及他的三卷大学本科讲义《费曼物理学讲义》而成为物理学的热心普及者。除此之外,费曼还通过他的半自传体书籍《别闹了,费曼先生! 》、《你在乎别人怎么想吗? 》以及关于他的书籍如拉尔夫·莱顿的《 要么是图瓦要么是半身像!》 和詹姆斯·格雷克为他写的传记《天才:理查德·费曼的生活和科学 》而为人们所知。

\subsection{早年生活}
费曼于1918年5月11日出生在纽约皇后区,[2]母亲露西尔·菲利普斯是一位家庭主妇,父亲梅尔维尔·阿瑟·费曼[3]是一位来自白俄罗斯明斯克[4] (当时是俄罗斯帝国的一部分)的销售经理,两人都是立陶宛犹太人。[5] 费曼说话很晚,直到三岁生日后才学会说话。成年后,他说话还带有纽约口音[6][7] 以至于被认为十分做作和夸张[8][9]——这使得他的朋友沃尔夫冈·泡利和汉斯·贝特曾评论说费曼说话像个“流浪汉”。[8] 年轻的费曼深受父亲的影响,父亲鼓励他提出问题来挑战正统思想,并且总是准备教费曼一些新东西。而从母亲那里,他则获得了贯穿其一生的幽默感。费曼在孩童时代就很有工程天赋,他在家里有一个实验室,他还很喜欢修理收音机。费曼小学期间,在父母外出办事的时候,他还发明了一个家庭防盗报警系统。[10]

费曼五岁时,他的母亲生了弟弟亨利·菲利普斯,但仅存活了四周就夭折了。[11] 四年后,理查德的妹妹琼出生,一家人搬到了皇后区的法洛克威。[3] 尽管相差九岁,琼和费曼还是很亲近,他们都共同对这个世界有着好奇心。尽管母亲认为女性缺乏对天文学的理解能力,费曼还是鼓励琼对天文学的兴趣,琼最终成为了一名天体物理学家。[12]

\subsubsection{宗教}
费曼的父母没有宗教信仰,年轻时的费曼形容自己是“公开的无神论者”。[13]许多年后,在给蒂娜·莱维特的一封信中,他拒绝了蒂娜为写关于犹太人诺贝尔奖获得者的书而向他咨询信息的请求,他在信中表明,“为了认可来自某些所谓犹太人遗传性的特殊因素,要为种族理论上的各种谬论打开大门”,以及“13岁时,我不仅转向别的宗教观点,而且也不再相信犹太人在任何方面是‘天选之子’”。[14]晚年,在参观犹太神学院时,费曼第一次遇到了《塔木德》(犹太古法典),并评论该书包含了一种中世纪的思辨,是一本很好的书。[15]

\subsection{教育经历}
费曼就读于皇后区法洛克威的一所学校——法洛克威高中,和他同为诺贝尔奖获得者的伯顿·里克特和巴鲁克·塞缪尔·布隆伯格也曾就读于该校。 从高中开始,费曼很快被提拔到一个高等数学班。 根据其传记作者詹姆斯·格雷克的说法,一项由高中管理的智商测试估计他的智商为125——很高,但“还算不错”。 费曼的妹妹琼在这方面表现得更好,因此她声称自己比费曼更聪明。几年后,费曼拒绝加入门萨国际俱乐部,自称智商太低。 物理学家史蒂夫·许对这项测试的描述如下:

我怀疑这个测试强调的是语言能力,而不是数学能力。费曼在臭名昭著的普特南数学竞赛考试中获得了美国最高分......他在普林斯顿大学的数学/物理研究生入学考试中也获得了最高分......费曼的认知能力可能有点不平衡......我记得看过费曼在大学时保存的笔记本摘录......里面有许多拼写错误和语法错误。我怀疑费曼对这些事情其实十分关心。

费曼15岁时自学了三角学、高等代数、无穷级数、解析几何以及微积分。[16] 在进入大学之前,他已经在用自己的记数法实验并推导出数学课题,如半导数。[17] 他为对数、正弦、余弦和正切函数创造了特殊的符号,使其看起来不像三个变量相乘,他还为导数创造了特殊的符号。[18][19] 作为Arista荣誉协会的成员,费曼在高中的最后一年获得了纽约大学数学冠军。[20] 他直接描述事物的习惯有时会让更传统的思想家感到困惑,例如在学习猫科动物解剖学时,他的问道:“你有猫的地图吗?”(指解剖图)。[21]

费曼申请了哥伦比亚大学,但由于学校犹太人录取名额的限制而未被录取。这反而使他进入了麻省理工学院并在那里加入了Pi Lambda Phi 同好会。 尽管费曼最初主修数学,但后来却转到了电气工程专业,因为他认为数学太抽象了。由于觉得自己“走得太远”,他后来又投身物理,并声称物理“介于两者之间”。 费曼在大学本科期间就在《物理评论》杂志上发表了两篇论文。其中一篇是与曼努埃尔·瓦拉塔合作的,题为《银河系恒星的宇宙射线散射》。

瓦拉塔让他的学生(费曼)了解到师生共同发表论文的一个潜规则:这位资深科学家导师的名字必须排在第一位。几年后,费因曼对此进行了报复,海森堡在自己关于宇宙射线的书的结尾中说道:“根据瓦拉塔和费因曼的说法,这种影响很难预料。”当师生二人再次见面时,费曼高兴地问瓦拉塔是否看过海森堡的书。瓦拉塔知道费曼为什么对此发笑。“是的,”他回答。“你是宇宙射线中的最后一个词。”

另一篇是费曼的高级论文,关于“分子中的力”,[22] 该论文的发表是源于约翰·斯莱特的一个想法,他对这篇论文印象深刻,故将其发表。今天,它被称为“海尔曼-费曼定理”。[23]

1939年,费曼获得了学士学位,[24] 并获得普特南研究员的职称。[25] 他在普林斯顿大学的物理研究生入学考试中获得了满分,这是前所未有的壮举。同时他在数学方面也取得了优异的成绩,但历史和英语部分表现不佳。当时普林斯顿的物理系系主任亨利·史密斯有一个忧虑,他写信给菲利普·莫尔斯问:“费曼是犹太人吗?我们对犹太人没有明确的规定,但由于安置犹太人的困难性,我们不得不将他们在我们系的比例保持在很小的范围内。”[26] 莫尔斯承认费曼的确是犹太人,但让史密斯放心,费曼的“外貌和举止并没有显示出这种特征”。[26]

费曼参加的第一次研讨会是关于经典的惠勒-费曼吸收理论,与会者包括阿尔伯特·爱因斯坦、沃尔夫冈·泡利和约翰·冯·诺依曼。泡利预言这个理论极难量化,爱因斯坦认为人们可以尝试将这种方法应用于广义相对论中的引力。基于此,弗雷德·霍伊尔和贾扬·纳利卡尔展开了大量的研究,形成了后来的霍伊尔-纳利卡尔引力理论。 费曼于1942年在普林斯顿获得博士学位,他的论文导师是约翰·阿奇博尔德·惠勒。费曼的博士论文题为“量子力学中的最小作用量原理”。 出于对量子化电动力学中的惠勒-费曼吸收体理论的兴趣,费曼将静止作用原理应用于量子力学问题,并为路径积分表述和费曼图奠定了基础。其中一个重要观点是,正电子的行为就像电子在时间上向后移动。詹姆斯·格雷克写道:

这是理查德·费曼的巅峰时期。在他二十三岁时......现在地球上可能没有一个物理学家能像费曼一样对理论科学的既有知识如此掌握和了解。这些科学理论不仅仅是数学方面的才能(尽管人们已经很清楚......惠勒-费曼合作中出现的数学机制已经超出了惠勒自身的能力)。费曼似乎对方程背后的物质感到异常轻松,就像同龄的爱因斯坦,就像苏联物理学家列夫·兰道——但其他人很少能做到。

费曼获得普林斯顿大学奖学金的条件之一是他不能结婚,但他还是继续与高中时的恋人阿琳·格林鲍姆见面,尽管费曼知道她患有严重的肺结核,他还是决定一旦获得博士学位就娶她,。肺结核在当时是一种不治之症,人们预计她不会活过两年。1942年6月29日,他们乘渡船去斯塔滕岛并在那里的市政府办公室结婚。婚礼仪式既没有家人也没有朋友参加,仅有一对陌生人见证了仪式,婚礼上费曼也只能亲吻阿琳的脸颊。仪式结束后,费曼带阿琳去了黛博拉医院进行治疗,并在此后每到周末就去医院看望她。[27][28]

\subsection{曼哈顿计划}
\begin{figure}[ht]
\centering
\includegraphics[width=6cm]{./figures/a7213a5af965b6cb.png}
\caption{费曼在Los Alamos实验室的身份证。} \label{fig_Feynma_1}
\end{figure}
1941年,随着第二次世界大战在欧洲肆虐,但此时美国尚未涉入战争,费曼花了一个夏天在宾夕法尼亚州弗兰克福兵工厂研究弹道学问题。[29][30] 珍珠港事件使美国卷入战争后,费曼被罗伯特·威尔逊招募,他当时正在研究制造原子弹用浓缩铀的方法,这将成为曼哈顿计划的一部分。[31][32] 威尔逊在普林斯顿的团队正在研究一种名为isotron的同位素分离器装置,旨在电磁分离铀-235和铀-238。这与威尔逊的前导师欧内斯特·劳伦斯(Ernest O. Lawrence)在加州大学辐射实验室的一个团队开发的calutron电磁型同位素分离器完全不同。理论上,isotron的效率比calutron高很多倍,但费曼和保罗·奥卢姆很难确定它是否可行。最终,在劳伦斯的建议下,isotron项目还是被放弃了。[33]

就在此时,1943年初,罗伯特·奥本海默正在建立 Los Alamos实验室,这是一个位于新墨西哥州的方山,用来设计和制造原子弹的秘密实验室,普林斯顿大学的研究团队被重新部署到了该实验室。“像一群职业军人一样,”威尔逊后来回忆说,“我们一起报名去Los Alamos实验室。”[34] 费曼很快就像许多其他年轻物理学家一样被魅力非凡的奥本海默迷住了,奥本海默从芝加哥打长途电话给费曼,告诉费曼他在新墨西哥州的阿尔伯克基为阿琳找到了一个疗养院。他们于1943年3月28日乘火车离开,是第一批前往新墨西哥州的人。铁路公司为阿琳提供了轮椅,费曼为她支付了额外的私人房间费用。[35]

在Los Alamos实验室,费曼被分配到了汉斯·贝特的理论部门,[36] 他给贝特留下了深刻的印象并被任命为组长。[37] 他和贝特在罗伯特·塞伯先前工作的基础上发展了计算裂变式原子弹产量的贝特-费曼公式。[38] 作为一名年轻的物理学家,费曼还不是这个项目的核心,他主要负责管理理论部门的人类计算机计算组。费曼与斯坦利·弗兰克尔和尼古拉斯·梅特罗波利斯一起,帮助建立了一个使用国际商用机器公司(IBM)穿孔卡进行计算的系统。[39] 他还发明了一种新的计算对数的方法,后来用在连接机上。[40][41] 费曼在Los Alamos实验室的其他工作包括计算实验室小型核反应堆锅炉的中子方程,以测量裂变材料组件与临界状态的接近程度。[42]

费曼在完成这项工作后被派往位于田纳西州橡树岭的克林顿工程师工厂,曼哈顿计划在那里拥有铀浓缩设施。费曼帮助那里的工程师设计材料储存的安全程序,以避免临界事故的发生,特别是当浓缩铀与水接触时,水充当中子慢化剂,此时很容易发生事故。费曼还坚持给普通人做关于核物理的讲座,使他们能够居安思危。[43] 他解释说,虽然无论多少非浓缩铀都可以安全储存,但浓缩铀则必须小心处理。费曼还为不同等级的浓缩制定了一系列安全建议。[44] 他被告知,如果橡树岭的人对他的建议有任何异议,费曼会告诫他们Los Alamos实验室“否则无法对他们的安全负责”。[45]
\begin{figure}[ht]
\centering
\includegraphics[width=10cm]{./figures/29367573878adfd8.png}
\caption{1946年在Los Alamos实验室举行的“超级”座谈会上。费曼在第二排,左起第四位,奥本海默旁边。} \label{fig_Feynma_2}
\end{figure}
费曼回到Los Alamos实验室后被任命为对提出的氢化铀炸弹进行理论工作和计算的负责人,并最终证明这是不可行的。[37][46] 物理学家尼尔斯·玻尔曾找到他进行一对一的讨论。费曼后来发现了玻尔会来找他的原因:大多数其他物理学家对玻尔太敬畏了,不敢和他争论,而费曼不受这些限制,他极力指出自己认为玻尔思想中有缺陷的任何东西。费曼表示他和其他人一样尊重玻尔,但一旦有人让他谈论物理,他就会变得十分专注,乃至忽略社交细节。也许正因如此,玻尔从未对费曼产生好感。[47][48]

在出于安全考虑而被与世隔绝的Los Alamos实验室,费曼通过调查物理学家们的柜子和桌子上的密码锁来自娱自乐。他经常发现同事们喜欢把锁的密码留在工厂设备上,写下组合,或者使用像日期这样容易猜测的密码。[49] 通过尝试他认为物理学家可能使用的数字,费曼找到了一个柜子密码(经证实是自然对数的基数之后的27–18–28, e = 2.71828...),并发现同事保存研究笔记的三个文件柜都有相同的密码。他打开柜子后在里面留下纸条当做恶作剧,吓得他的同事弗雷德里克·德·霍夫曼一度以为间谍已经盯上他们了。[50]

费曼每月380美元的工资大约仅是他微薄的生活费和阿琳医疗费的一半,因此他们被迫动用了阿琳3300美元的积蓄。[51] 周末的时候费曼会开着从朋友克劳斯·福克斯那里借来的车去阿尔伯克基看阿琳。[52][53] 当被问及实验室谁最有可能是间谍时,福克斯提到费曼曾破解过实验室文件柜密码以及频繁的阿尔伯克基之行,[52] 不过福克斯本人后来承认在为苏联从事间谍活动。[54] 联邦调查局(FBI)汇编了一份关于费曼的庞大文件。[55]
\begin{figure}[ht]
\centering
\includegraphics[width=10cm]{./figures/4b4fa48b40702640.png}
\caption{曼哈顿计划期间,费曼(中)和罗伯特·奥本海默(费曼右侧第一位)在Los Alamos实验室的社交集会上。} \label{fig_Feynma_3}
\end{figure}
在得知阿琳即将离世时,费曼开车去阿尔伯克基,坐着陪了她几个小时,直到阿琳于1945年6月16日去世。[56] 随后,费曼就全身心地投入到曼哈顿项目的工作中,并致力于三位一体核试验。费曼声称自己是唯一一个不戴墨镜或焊工眼镜就能看到爆炸的人,他认为透过卡车挡风玻璃看是安全的,因为这样可以屏蔽有害的紫外线辐射。爆炸的巨大亮度使他不得不躲到卡车地板上,在那里他看到了一个暂时的“紫色斑点”余像。[57]

\subsection{康奈尔大学时期}
费曼名义上在威斯康星大学麦迪逊分校被任命为物理学助理教授,但在参与曼哈顿计划期间休了无薪假。[58] 1945年,他收到了文理学院院长马克·英格拉哈姆的一封信,信中要求他在下一学年回到大学任教。在费曼拒绝之后,他便不再在该学校任职。几年后在威斯康星大学麦迪逊分校的一次演讲中,费曼戏谑道:“回到唯一一所基于正确判断力而解雇了我的大学真是太棒了。”[59]

早在1943年10月30日,贝特就写信给他所在大学康奈尔大学的物理系的系主任,建议聘用费曼。1944年2月28日,这一建议得到了罗伯特·巴彻的认可,[60] 罗伯特·巴彻同样来自康奈尔大学,[61] 是Los Alamos实验室最资深的科学家之一。[62] 1944年8月,费曼接受了康奈尔大学的职位。奥本海默也曾希望招聘费曼加入加州大学,但物理系的负责人雷蒙德·T·比尔奇并不是很愿意。后来他在1945年5月表示愿意给费曼提供职位,但遭到了费曼的拒绝。康奈尔大学为费曼提供了与其相匹配的3900美元年薪。[60] 1945年10月,费曼成为Los Alamos实验室首批离开前往纽约州伊萨卡的小组领导者之一。[63]

由于费曼不再在Los Alamos实验室工作,他也无法再免除兵役。在费曼的入伍体检中,军队精神科医生诊断出费曼患有心理疾病,军队以此为由给予他4级豁免。[64][65] 费曼的父亲于1946年10月8日突然去世,此后费曼就一直患有抑郁症。[66] 1946年10月17日,他给阿琳写信表达了自己对她深深的爱和心痛,这封密封的信直到费曼死后才被公诸于世。“请原谅我没有寄出这封信,”信的结尾写道,“但是我不知道你的新地址。”[67] 由于无法专注于科研,费曼开始着手解决一些物理问题,不是为了实际应用,而是为了自我满足。[66] 费曼研究的内容之一涉及分析旋转中的物理学、章动圆盘在空气中运动,其灵感来自于他在康奈尔自助餐厅时见到有人将餐盘抛向空中。[68] 费曼阅读了威廉·罗恩·汉密尔顿爵士关于四元数的著作,并试图用它们来表述电子的相对论,但没有获得成功。费曼在此期间使用旋转方程来表达不同的旋转速度的工作,最终被证明对他获得诺贝尔奖的工作具有重要意义。但是由于他筋疲力尽,后来把注意力转移到不太直接的实际问题。当其他著名大学如普林斯顿高等研究院、加州大学洛杉矶分校和加州大学伯克利分校向他抛来教授职位的橄榄枝时,费曼甚至感到十分惊讶。[66]
\begin{figure}[ht]
\centering
\includegraphics[width=6cm]{./figures/f7fef9704a1adbc4.png}
\caption{表示电子/正电子湮灭的费曼图。} \label{fig_Feynma_4}
\end{figure}
费曼不是战后初期唯一碰壁的理论物理学家。当时量子电动力学深受微扰理论中无穷积分的影响,这些都是理论中明显的数学缺陷,费曼和惠勒试图解决这些问题,但未能成功。[69] 对此,默里·盖尔曼指出,“这是理论物理学家之耻。”[70] 1947年6月,美国主要物理学家在牛尾洲会议上会面。对费曼来说,这是他“第一次与大人物举行大型会议......和平时期我从参与过这样的会议。"[71] 科学家们在会上讨论了困扰量子电动力学界的问题,但是理论物理学家完全被实验物理学家的巨大成就所掩盖,实验物理学家们报道了发现兰姆位移,测量了电子的磁矩,以及提出了罗伯特·马沙克的双介子假说。[72]

贝特率先完成了汉斯·克拉默斯的工作,并为兰姆位移导出了重整化的非相对论量子方程,下一步是创建相对论量子方程。费曼觉得自己可以完成这一任务,但当他带着解决方案找到贝特时,两人的答案并不统一。[73] 费曼运用他在论文中使用的路径积分表述,再次仔细研究了这一问题。像贝特一样,他通过应用截止项使积分有限,最终得到了与贝特一致的结果。[74][75] 费曼在1948年的波科诺会议上向同行们介绍了自己的工作,但是进展并不顺利。朱利安·施温格对他在量子电动力学方面的工作做了长篇的介绍,费曼随后提出了他的版本,题为“量子电动力学的替代表述”。第一次被提出的陌生的费曼图令费曼的听众们迷惑不解,费曼也未能令他人理解自己的观点,而保罗·狄拉克、爱德华·泰勒和尼尔斯·玻尔都对此提出了反对意见。[76][77]

对弗里曼·戴森而言,至少有一件事是明确的:即使没有其他人明白,朝永振一郎、施温格和费曼也明白他们在说什么,尽管他们对此没有发表任何看法。他确信费曼的表述更容易理解,并最终说服奥本海默事实就是如此。[78] 戴森在1949年发表了一篇论文,在费曼的论文中增加了关于如何实现重整化的新规则。[79] 这促使费曼在20世纪60年代,三年多的时间内于杂志《物理评论》上发表了一系列论文来表述自己的观点。[80] 费曼在1948年发表的关于“经典电动力学的相对论截止点”的论文试图解释他在波科诺会议上没能理解的东西,[81] 在1949年发表的关于“正电子理论”的论文讨论了薛定谔方程和狄拉克方程,并介绍了现在所谓的费曼传播子。[82] 最后,在1950年关于“电磁相互作用的量子理论的数学表述”和1951年关于“应用于量子电动力学中的算符微积分”的论文中,他发展了自己这些想法的数学原理,推导出了熟悉的公式并提出了新的公式。[83]

人们最初都是引用施温格的论文,但是1950年开始出现了引用费曼论文并使用费曼图的论文,并且这一现象越来越广泛。[84] 学生们开始学习并使用费曼创造的有力的新工具,计算机程序后来也被用于计算费曼图,提供了一个前所未有的强大工具。费曼图用形式语法构成了一种形式语言,这使得编写这样的程序成为可能。马克·卡克提供了历史下求和的形式证明,表明抛物型偏微分方程可以重新表示为不同历史下的和(即期望算子),这就是现在所知的费曼-卡克公式,其使用范围已经超出了物理学,扩展到关于随机过程的诸多应用中。[85] 但对施温格来说,费曼图是“教育学,而不是物理学”。[86]

到1949年,费曼在康奈尔大学变得情绪焦躁不安。“直到这些安排变得性不稳定”,他才定居在固定的房子或公寓里,住在招待所或学生公寓里,或者和已婚朋友住在一起。[87] 费曼喜欢和大学生约会,雇佣妓女,和朋友的妻子睡觉。[88] 他不喜欢伊萨卡寒冷的冬季天气,渴望暖和的气候。[89] 最重要的是,在康奈尔大学,费曼一直生活在在汉斯·贝特的阴影下。[87] 尽管如此,费曼还是对他在康奈尔生涯中的居住了大部分时间的Telluride House进行了做了亲切的回顾。在一次采访中,他将这所房子描述为“一群男孩,因为他们的学识、聪明、头脑或其他原因,而被特别遴选出来,提供免费食宿等等。”他喜欢这套房子的便利,并说“正是在那里我做了一些基础工作”,而这帮助他获得了诺贝尔奖。[90][91]

\subsection{加州理工学院时期}
\subsubsection{5.1 个人生活和政治生活}
1949年7月,费曼在里约热内卢度过了几个星期。[92] 同年,苏联引爆了第一颗原子弹,引发了反对共产主义的疯狂热潮。[93] 福克斯在1950年作为苏联间谍被捕后,联邦调查局来询问贝特,福克斯的好友费曼是否忠于美国。[94] 物理学家戴维·玻姆也于1950年12月4日被捕[95] 并于1951年10月移居巴西。[96] 一位女性朋友告诉费曼,他也应该考虑搬到南美去。[93] 费曼于1951-1952年来到巴西休假,[97] 在此期间他还在巴西莱罗中心授课。在巴西,费曼对桑巴音乐印象十分深刻,他还学会了演奏金属打击乐器 frigideira。[98] 他还是一名热情的业余邦戈鼓和康加鼓演奏者,经常在音乐剧的露天管弦乐队中演奏。[99][100] 他和朋友波姆在里约呆了一段时间,但是波姆未能说服费曼去研究自己的物理思想。[101]

费曼后来再也没有回到康奈尔。当年曾帮助费曼来康奈尔大学的巴赫尔,又诱使他去了加州理工学院(Caltech),两人此次交易的一部分是他可以在巴西度过第一年休假。[102][87] 费曼后来被来自堪萨斯州新德沙的姑娘玛丽·路易斯·贝尔迷住了,两人是在康奈尔的一家自助餐厅第一次见面,当时玛丽正在康奈尔学习墨西哥艺术和纺织品史。费曼后来带着玛丽去了加州理工学院,并在那里开展了一个讲座。费曼在巴西期间,玛丽在密歇根州立大学教授家具和室内设计史课程。费曼从里约热内卢通过邮件向玛丽求了婚,在他回来后不久,两人于1952年6月28日在爱达荷州的博伊西结婚。但婚后两人经常吵架,玛丽被他暴躁脾气吓坏了。另外他们的政见也不一样,尽管费曼登记并经投票成为共和党人,但玛丽更保守,她在1954年的奥本海默安全听证会(“有烟就有火”)上的意见冒犯了费曼。两人于1956年5月20日分居。1956年6月19日,以“极端残忍”为由,他们签署了一项中间离婚判决,并最终于于1958年5月5日正式离婚。[103][104]

经过1957年的史普尼克危机之后,美国政府对科学的兴趣一度上升。费曼一度被考虑去总统科学顾问委员会任职,但最终没有被任命。此时联邦调查局(FBI)采访了一位与费曼关系密切的女性,可能是玛丽·楼,她于1958年8月8日向FBI的埃德加·胡佛发送了一份书面声明:

我不知道——但我认为理查德·费曼要么是共产主义者,要么非常亲共产主义——因此存在非常明确的安全风险。在我看来,这个人是一个极其复杂和危险的人,一个在公众信任的位置上非常危险的人...在阴谋诡计方面,我认为理查德·费曼非常聪明——事实上他是个天才——而且我进一步认为,他是完全无情的,不受道德、伦理或宗教的阻碍——并且会不惜一切代价达到他的目的。

他一醒来就开始在脑中处理微积分问题。他无论开车时,坐在客厅时,晚上躺在床上时,都在处理微积分的问题。

来自玛丽·路易斯·贝尔的离婚投诉。 [105]
尽管如此,政府还是派费曼去日内瓦参加了1958年9月的原子促进和平会议。在日内瓦湖的湖滨上,费曼遇到了格温内·豪沃斯,她来自约克郡的里彭登,是一名在瑞士工作的互惠换工生。费曼离婚后,爱情生活一直不安定,前女友带着他的阿尔伯特·爱因斯坦奖奖章离开了他,她还在另一个前女友的建议下,假装怀孕以勒索费曼,让他支付堕胎费用,然后用这笔钱买家具。当费曼发现豪沃斯的月薪只有25美元时,他每周给她20美元让她做和自己同住的女佣。费曼知道根据曼恩法案该行为是非法的,所以找了朋友马修·桑兹作为她的担保人。豪沃斯表示自己已经有两个男朋友了,但她依然决定接受费曼的邀请,并于1959年6月抵达加州阿尔塔迪纳。在加州,豪沃斯特意想找别的男人约会,但是费曼在1960年初向她求婚了。两人于1960年9月24日在帕萨迪纳的亨廷顿酒店结婚,并在1962年有了一个儿子卡尔,随后又于1968年领养了一个女儿米歇尔。[106][107] 除了他们在阿尔塔迪纳的家,两人在下加利福尼亚还有一栋海滩别墅,是用费曼诺贝尔奖的钱买的。[108]

费曼在约翰·莉莉著名的感觉剥夺管中尝试用大麻和氯胺酮作为研究意识的一种方法。[109][110] 当他开始表现出模糊的早期酗酒迹象时就戒酒了,因为他不想做任何可能损害他大脑的事情。[111] 尽管费曼对幻觉很好奇,但他也不愿意尝试迷幻药。[111]

\subsubsection{5.2 物理学}
\begin{figure}[ht]
\centering
\includegraphics[width=6cm]{./figures/3985f48f3cf80385.png}
\caption{1984年,理查德·费曼在马萨诸塞州沃尔瑟姆的罗伯特·泰德·佩恩庄园工作。} \label{fig_Feynma_5}
\end{figure}
在加州理工学院,费曼研究了液体过冷氦的物理学,该状态下氦流动时似乎完全没有粘性。 费曼为苏联物理学家列夫·朗道的超流体理论提供了量子力学解释。[112] 将薛定谔方程应用于该问题的结果表明,超流体显示了宏观尺度上可观察到的量子力学行为,这一发现有助于解决超导问题,但费曼并没有找到解决办法。[113] 直到1957年,由约翰·巴丁、利昂·尼尔·库珀和约翰·施里弗提出的超导体BCS理论才解决了该问题。[112]

费曼受到来自于对电动力学中惠勒-费曼吸收体理论的量子化的期望所激励,为路径积分表述和费曼图奠定了基础。[114]

费曼和默里·盖尔曼一起发展了一个弱衰变模型,该模型表明过程中的电流耦合是矢量电流和轴向电流的组合(弱衰变的一个例子是中子衰变为电子、质子和反中微子)。尽管乔治·苏达山和罗伯特·马沙克几乎同时发展了这个理论,但费曼与默里·盖尔曼的合作由于巧妙地用矢量流和轴向流描述了弱相互作用,被认为是具有开创性的。该模型由此结合了1933年恩里科·费米的$\beta$衰变理论和对宇称不守恒的解释。[115]

费曼试图通过一种叫做部分子模型的理论来解释核散射的强相互作用。部分子模型是对由盖尔曼提出的夸克模型的补充,这两种模式之间的关系模糊不清,盖尔曼因此嘲笑费曼的部分子模型是“骗局”。20世纪60年代中期,物理学家认为夸克只是用来记录对称数字的工具,而不是真实粒子。如果把$\omega$-负粒子解释为三个相同的奇异夸克结合在一起,假设夸克是真的,那么$\omega$-负粒子似乎不可能统计到。[116][117]

20世纪60年代后期,SLAC国家加速器实验室的深度非弹性散射实验表明,核子(质子和中子)包含可以散射电子的点状粒子。很自然可以用夸克来鉴定这些,但是费曼的部分子模型却试图以不引入额外假设的方式来解释实验数据。例如,数据显示大约45\%的能量动量由核子中的电中性粒子携带,这些电中性粒子现在被认为是在夸克之间传递力的胶子,它们的三值色量子数解决了$\omega$-负粒子的问题。费曼没有质疑夸克模型,例如当第五夸克在1977年被发现时,费曼立即向他的学生指出,这一发现意味着第六夸克的存在,果然在他死后十年,人们发现了第六夸克。[116][118]

在量子电动力学取得成功之后,费曼又转向了量子引力的研究。通过与自旋为1的光子进行类比,他研究了自由且无质量,自旋为2的场的结果,并推导出广义相对论的爱因斯坦场方程,也取得了一点新的进展。费曼当时发现的重力的计算设备“幽灵”,即他图表内部的“粒子”,在自旋和统计之间有“错误”的联系,这一点在解释杨-米尔斯理论的量子粒子行为方面是极其重要的,例如解释量子色动力学和电弱理论。[119] 费曼研究了自然的所有四种力量:电磁力、弱力、强力和重力。约翰和玛丽·格里宾在他们关于费曼的书中指出,“没有其他人(像费曼一样)对这四种相互作用的研究做出如此有影响力的贡献”。[120]

费曼为他在纳米技术领域的两项挑战设立了1000美元的奖金,在一定程度上是也为了宣传物理学的进展,一项是威廉·麦克勒朗提出的,另一项是汤姆·纽曼提出的。[121] 费曼也是第一批设想量子计算机可能性的科学家之一。[122][123] 在1984-1986年,他发展了一种近似计算路径积分的变分法,提出了一种将发散微扰展开转变为收敛强耦合展开(变分微扰理论)的有力方法,并由此测量出了卫星实验中最精确的临界指数。[124][125]

\subsubsection{5.3 教育学}
20世纪60年代初,费曼接受了加州理工学院“改善”本科生教学的任务。投入这项任务三年后,他创作了一系列讲义,即后来的 《费曼物理学讲义》。他想要一张撒有粉末的鼓膜的照片,以显示讲义开篇讲到的振动的模式。尽管书的前言中显示了费曼击鼓的照片,但由于担心这张照片可能与毒品和摇滚乐有关,出版商将还是将书封面改成了纯红色。几年来,另外两位物理学家——罗伯特·B·雷顿和马修·桑兹也一直参与了《费曼物理学讲义》的兼职合著。尽管这些书没有被大学采用为教科书,但该书对物理学有着深刻理解,仍然很畅销。[126] 费曼的许多讲义和各种各样的演讲稿后来都被编著成书,包括 《物理定律的本质》《量子电动力学(QED): 光和物质的奇妙理论》《 统计力学》《引力讲义》,以及《费曼计算讲义》。[127]

费曼记录下了他在巴西教物理系本科生的经历。学生们的学习习惯和葡萄牙语教科书缺乏上下文连贯性或信息应用,在费曼看来学生们根本没有学到物理知识。年底,费曼被邀请就他的教学经历发表演讲,他同意了,但前提是要求自己可以实话实说,当然费曼也的确是这么做的。 费曼反对死记硬背或不加思考的记忆,以及其他强调形式胜于功能的教学方法,清晰的思维和展示是引起他注意力的基本前提。学生即使是毫无准备地接近费曼也是危险的,他也会很清楚地记得这些人当中的傻瓜和假装的人。1964年,他在加州课程委员会工作,该委员会负责批准加州学校使用的教科书,费曼对自己在此期间的生活并没有很深刻的印象。 作为“新数学”的一部分,许多数学教科书只涵盖纯数学家使用的内容,小学生就已经被被教授了集合的相关知识,但是:

研究过这些教科书的大多数人可能会惊讶地发现,代表集合的并集和交集的符号∪或∩,以及括号的特殊使用{ }等,这些书中给出的关于集合的所有复杂符号几乎从未出现在理论物理、工程、商业计算、计算机设计或其他应用到数学的领域的任何著作中。我认为没有必要也没有理由在学校解释或教授这一切。这不是一种表达自我的有效方式,也不是一个有说服力和简单的方法。这据说很精确,但精确的目的是什么?

1966年4月,费曼在美国国家科学教师协会发表演讲,他在演讲中建议如何让学生像科学家一样思考,保持开放的思想和好奇新,尤其是学会质疑。在演讲过程中,他给出了科学的定义,他认为这是分几个阶段进行的。地球上智能生命的进化——例如猫这样的生物是通过玩耍并从经验中学习。而人类的进化则是通过使用语言将知识从一个个体传递到另一个个体,这样当一个个体死亡时,知识也不会丢失。但不幸的是,错误的和正确的知识一样会被传递下去,所以需要采取另外的步骤。伽利略等人怀疑传承下来的事实并根据经验从头开始进行调查,以了解真实的情况——这才是科学。[128]

1974年,费曼发表了加州理工学院的毕业典礼演讲,主题为 cargo cult science,它有着科学的表象,但仅仅是伪科学,因为一部分这样的科学家缺乏“一种科学诚信,一种对应着完全诚实的科学思想原则”。他告诉毕业班,“第一个原则是你不能欺骗自己——你是最容易被欺骗的人。所以你必须非常小心。在你没有欺骗自己之后,不欺骗其他科学家是很容易的。之后,你只需要以传统的方式做一个诚实的人。”[129]

费曼还担任了31名学生的博士生导师。[130]

\subsubsection{5.4 《别闹了,费曼先生!》}
20世纪60年代,费曼开始考虑写自传,并接受历史学家的采访。20世纪80年代,他与拉尔夫·莱顿(罗伯特·莱顿的儿子)合作,在录音磁带上录制了拉尔夫转录的章节。《别闹了,费曼先生!》这本书出版于1985年 ,并成为畅销书。然而这本书的出版引发了一股新的抗议费曼对女性态度的热潮,在1968年和1972年都有人抗议费曼所谓的性别歧视。然而这些抗议无济于事,1969年被聘为加州理工学院第一位女教授的珍妮乔·拉贝尔,在1974年依然被拒绝成为终身教授。珍妮乔向美国同等就业机会委员会提起诉讼,该机构于1977年做出了不利于加州理工学院的裁决,她还补充道自己的薪酬低于男性同事。珍妮乔·拉贝尔最终于1979年获得了终身教授职位。费曼的许多同事对他站在拉贝尔的一边感到颇为惊讶,但通过此事,费曼开始了解了拉贝尔,对她既喜欢又钦佩。[131][132]

费曼在《弱相互作用》一书中的描述让盖尔曼感到不安,他威胁要起诉费曼,这导致在费曼在该书以后的版本中不得不加入了更正。[133] 这一事件只是这两位科学家几十年来不愉快感情的最新挑衅。费曼受到关注经常让盖尔曼感到不愉快,[134] 他说:“费曼是一位伟大的科学家,但他花了大量的精力来创造关于自己的轶事。”[135] 他还指出费曼的怪癖包括拒绝刷牙,甚至在国家电视台上建议其他人不要刷牙。[135]

\subsubsection{5.5 “挑战者号”事故}
\begin{figure}[ht]
\centering
\includegraphics[width=6cm]{./figures/f69ce2d6c5cd7531.png}
\caption{1986年,“挑战者号”事故。} \label{fig_Feynma_6}
\end{figure}
当费曼被邀请加入罗杰斯委员会时来调查“挑战者号”事故时,他其实是十分犹豫的。他对妻子说,将要去的首都“对我来说是一个巨大的神秘世界,有着强大的力量。”[136] 但是妻子却说服费曼去,并认为他可能会发现其他人忽略的东西。费曼由于毫不犹豫地将灾难归咎于NASA(美国国家航空航天局)而与前国务卿威廉·罗杰斯发生了冲突。在一次听证会的休息期间,罗杰斯告诉委员会成员尼尔·阿姆斯特朗,“费曼渐渐变成了一个讨厌鬼。”[137] 在一次电视听证会上,费曼将航天飞机O形环中使用的材料样品压缩在夹具中,然后浸入冰冷的水中以说明在寒冷的天气中该材料的弹性会变差[138] 委员会最终确定此次事故是由卡纳维拉尔角异常寒冷的天气中主O形环密封不当造成的。[139]

费曼在自己的书《你在乎别人怎么想吗? 》的后半部分记载了他在罗杰斯委员会的经历,与他通常简单轻松的叙事惯例不同,他对这部分的叙述十分详细和庄重。费曼的叙述揭示了NASA工程师和高管之间联系的脱节,而这种脱节远超出了费曼的预期。他对NASA高管的采访揭示了这些高层对于事故中基本概念的巨大误区。例如,NASA管理人员声称航天飞机上灾难性故障的概率仅为10万分之一,但是费曼发现NASA的工程师估计灾难的概率接近200分之一。他的结论是,NASA管理层对航天飞机可靠性的估计完全不现实,令他尤为愤怒的是,NASA还用它来招募克丽斯塔·麦考利夫加入太空教师( Teacher-in-Space)计划。费曼在委员会报告的附录(这部分附录在费曼威胁拒绝签署报告后才勉强被收录在报告内)中警告说,“一项成功的技术,其现实性必须优先于公关,因为自然真理是不会被愚弄的。”[140]

\subsubsection{5.6 表彰与奖励}
费曼的工作第一次得到公众认可是在1954年,时任原子能委员会主席的路易斯·斯特劳斯通知他,他赢得了价值15000美元的阿尔伯特·爱因斯坦奖和一枚金牌。由于斯特劳斯剥夺了对奥本海默的安全许可,费曼并不愿意接受奖项,但伊西多·艾萨克·拉比告诫费曼:“你决不能把一个人的慷慨当成攻击他的利剑。一个人如果拥有美德,即使他有诸多缺点,这些缺点也不应该被当做反对他的工具。”[141] 费曼接着又获得了1962年的AEC欧内斯特·奥兰多·劳伦斯奖。[142] 施温格、朝永振一郎和费曼共同获得了1965年的诺贝尔物理学奖,“因为他们在量子电动力学方面的基础性工作,对基本粒子物理学产生了深远的影响”。[143] 费曼于1965年当选为英国皇家学会的外籍会员,[2][144] 于1972年获得了奥斯特奖章奖,[145] 于1979年获得了美国国家科学奖章。[146] 他还当选为美国国家科学院的会员,但最终辞职[147][148] 并不再被他们列出。[149]

\subsection{死亡}
1978年,费曼因腹部疼痛而求医,并被诊断为脂肪肉瘤,这是一种罕见的癌症。外科医生切除了一个足球大小的肿瘤,肿瘤压伤了费曼的一个肾脏和脾脏。1986年10月和1987年10月医生为他进行了进一步的手术。[150] 1988年3月,费曼再次进入加州大学洛杉矶分校医疗中心住院治疗。十二指肠溃疡破裂导致了他的肾衰竭,但费曼拒绝接受透析治疗,尽管透析可能会延长他几个月的生命。费曼在妻子格温内、妹妹琼和堂弟弗朗西斯·卢因的照料下,最终于1988年2月15日去世,享年69岁。[151]

费曼在将死之时,问丹尼·希利斯为何如此难过,希利斯回答说,他感觉费曼将要离世了。费曼回答道,死亡的确有时会令自己不安,但他补充说,当一个人变得和自己一样老,只要给人们留下了很多故事,即使死了,也不会完全消失。[152]

费曼临终前还曾试图访问俄罗斯的图瓦苏维埃社会主义自治共和国,但这个梦想被冷战时期的官僚主义问题所驳回。直到他死后的第二天,费曼才收到苏联政府授权这次旅行的信,女儿米歇尔后来代替他踏上了旅程。[153]

费曼的葬礼在加州阿尔塔迪纳的山景公墓举行。[154] 他的遗言是:“我讨厌死去两次,这太无聊了。”[153]

\subsection{公众遗产}
费曼生活的方方面面被各种媒体描述得绘声绘色。1996年,马修·布罗德里克拍摄了传记电影《无限》来记录费曼的生活。[155] 演员艾伦·艾尔达委托剧作家彼得·帕内尔写了一部关于费曼生命中一个虚构日子的双人剧,故事发生在费曼去世两年前。这部剧叫做《QED》,于2001年在洛杉矶的Mark Taper论坛首映,后来又在百老汇薇薇安·博蒙特剧院上演,两次演出都由奥尔达主演,饰演理查德·费曼。[156] 实时歌剧在2005年6月的诺福克室内音乐节上首演了歌剧《费曼》 。[157] 2011年,费曼成为一部同名传记漫画小说的主题,小说作者为吉姆·奥塔维亚尼,插图由利兰·杨梅克完成。[158] 2013年,费曼的故事在罗杰斯委员会的角色被英国广播公司(BBC)改编成剧《挑战者号》 (美国名为《“挑战者号”事故》),由威廉·赫特扮演费曼。[159][160][161] 在2016年出版的书《创意创造者:关于一些名人生活和创意的个人观点》中提到的费曼经常说的事情之一是“心灵的平静是创造性工作最重要的先决条件。”费曼觉得一个人应该尽一切可能实现内心的平静。[162]

人们以各种方式纪念费曼。2005年5月4日,美国邮政管理局发行了一套“美国科学家”纪念邮票,共有四枚37美分的不干胶邮票。该系列邮票中包括的科学家有理查德·费曼、约翰·冯·诺依曼、芭芭拉·麦克林托克和乔赛亚·威拉德·吉布斯等。费曼的邮票是棕褐色的,上面有一张30多岁的费曼的照片和八张费曼小图。[163] 这些邮票是维克托·斯塔宾在卡尔·赫尔曼的艺术指导下设计的。[164] 为纪念费曼,费米国立加速器实验室的计算部门主楼被命名为“费曼计算中心”。[165] 苹果公司于1997年为他们的“非同凡想”广告活动制作了一系列海报,而费曼作报告的照片也被选为了海报的一部分。[166] 美剧《生活大爆炸》中的谢尔顿·库珀这一角色也是个费曼迷,谢尔顿经常演奏邦戈鼓来模仿费曼。[167] 2016年1月27日,比尔·盖茨写了一篇名为《我从未有过的最好的老师》的文章,把费曼描述为一名很有才华的老师,这也启发盖茨创建了“图瓦计划”将费曼的信使讲座《物理定律的本质》视频放于网上供大众观看。2015年,为了纪念费曼获得1965年诺贝尔奖50周年,同时也为了响应加州理工学院关于学习费曼思想的要求,盖茨制作了一个视频,讲述为什么他认为费曼很特别。[168]

\subsection{相关文献}
\subsubsection{8.1 精选科学著作}
\begin{itemize}
\item Feynman, Richard P. (1942). Laurie M. Brown, ed. The Principle of Least Action in Quantum Mechanics (PDF). PhD Dissertation, Princeton University. World Scientific (with title Feynman's Thesis: a New Approach to Quantum Theory) (published 2005). ISBN 978-981-256-380-4.
\item Wheeler, John A.; Feynman, Richard P. (1945). "Interaction with the Absorber as the Mechanism of Radiation". Reviews of Modern Physics. \textbf{17} (2–3): 157–181. Bibcode:1945RvMP...17..157W. doi:10.1103/RevModPhys.17.157.
\item Feynman, Richard P. (1946). A Theorem and its Application to Finite Tampers. Los Alamos Scientific Laboratory, Atomic Energy Commission. doi:10.2172/4341197. OSTI 4341197.
\item Feynman, Richard P.; Welton, T. A. (1946). Neutron Diffusion in a Space Lattice of Fissionable and Absorbing Materials. Los Alamos Scientific Laboratory, Atomic Energy Commission. doi:10.2172/4381097. OSTI 4381097.
\item Feynman, Richard P.; Metropolis, N.; Teller, E. (1947). Equations of State of Elements Based on the Generalized Fermi-Thomas Theory. Los Alamos Scientific Laboratory, Atomic Energy Commission. doi:10.2172/4417654. OSTI 4417654.
\item Feynman, Richard P. (1948). "Space-time approach to non-relativistic quantum mechanics". Reviews of Modern Physics. \textbf{20} (2): 367–387. Bibcode:1948RvMP...20..367F. doi:10.1103/RevModPhys.20.367.
\item Feynman, Richard P. (1948). "A Relativistic Cut-Off for Classical Electrodynamics". Physical Review. 74 (8): 939–946. Bibcode:1948PhRv...74..939F. doi:10.1103/PhysRev.74.939.
\item Feynman, Richard P. (1948). "Relativistic Cut-Off for Quantum Electrodynamics". Physical Review. \textbf{74} (10): 1430–1438. Bibcode:1948PhRv...74.1430F. doi:10.1103/PhysRev.74.1430.
\item Wheeler, John A.; Feynman, Richard P. (1949). "Classical Electrodynamics in Terms of Direct Interparticle Action" (PDF). Reviews of Modern Physics. \textbf{21} (3): 425–433. Bibcode:1949RvMP...21..425W. doi:10.1103/RevModPhys.21.425.
\item Feynman, Richard P. (1949). "The theory of positrons". Physical Review. 76 (6): 749–759. Bibcode:1949PhRv...76..749F. doi:10.1103/PhysRev.76.749.
\item Feynman, Richard P. (1949). "Space-Time Approach to Quantum Electrodynamic". Physical Review. \textbf{76} (6): 769–789. Bibcode:1949PhRv...76..769F. doi:10.1103/PhysRev.76.769.
\item Feynman, Richard P. (1950). "Mathematical formulation of the quantum theory of electromagnetic interaction". Physical Review. \textbf{80} (3): 440–457. Bibcode:1950PhRv...80..440F. doi:10.1103/PhysRev.80.440.
\item Feynman, Richard P. (1951). "An Operator Calculus Having Applications in Quantum Electrodynamics". Physical Review. \textbf{84} (1): 108–128. Bibcode:1951PhRv...84..108F. doi:10.1103/PhysRev.84.108.
\item Feynman, Richard P. (1953). "The λ-Transition in Liquid Helium". Physical Review. \textbf{90} (6): 1116–1117. Bibcode:1953PhRv...90.1116F. doi:10.1103/PhysRev.90.1116.2.
\item Feynman, Richard P.; de Hoffmann, F.; Serber, R. (1955). Dispersion of the Neutron Emission in U235 Fission. Los Alamos Scientific Laboratory, Atomic Energy Commission. doi:10.2172/4354998. OSTI 4354998.
\item Feynman, Richard P. (1956). "Science and the Open Channel". Science (published February 24, 1956). \textbf{123} (3191): 307. Bibcode:1956Sci...123..307F. doi:10.1126/science.123.3191.307. PMID 17774518.
\item Cohen, M.; Feynman, Richard P. (1957). "Theory of Inelastic Scattering of Cold Neutrons from Liquid Helium". Physical Review. \textbf{107} (1): 13–24. Bibcode:1957PhRv..107...13C. doi:10.1103/PhysRev.107.13.
\item Feynman, Richard P.; Vernon, F. L.; Hellwarth, R. W. (1957). "Geometric representation of the Schrödinger equation for solving maser equations" (PDF). J. Appl. Phys. \textbf{28} (1): 49. Bibcode:1957JAP....28...49F. doi:10.1063/1.1722572.
\item Feynman, Richard P. (1959). "Plenty of Room at the Bottom". Presentation to American Physical Society. Archived from the original on February 11, 2010.
\item Edgar, R. S.; Feynman, Richard P.; Klein, S.; Lielausis, I.; Steinberg, C. M. (1962). "Mapping experiments with r mutants of bacteriophage T4D". Genetics (published February 1962). \textbf{47} (2): 179–86. PMC 1210321. PMID 13889186.
\item Feynman, Richard P. (1968) [1966]. "What is Science?" (PDF). The Physics Teacher. \textbf{7} (6): 313–320. Bibcode:1969PhTea...7..313F. doi:10.1119/1.2351388. Retrieved December 15, 2016. Lecture presented at the fifteenth annual meeting of the National Science Teachers Association, 1966 in New York City
\item Feynman, Richard P. (1966). "The Development of the Space-Time View of Quantum Electrodynamics". Science (published August 12, 1966). 153 (3737): 699–708. Bibcode:1966Sci...153..699F. doi:10.1126/science.153.3737.699. PMID 17791121.
\item Feynman, Richard P. (1974a). "Structure of the proton". Science (published February 15, 1974). \textbf{183} (4125): 601–610. Bibcode:1974Sci...183..601F. doi:10.1126/science.183.4125.601. JSTOR 1737688. PMID 17778830.
\item Feynman, Richard P. (1974). "Cargo Cult Science" (PDF). Engineering and Science. \textbf{37} (7).
\item Feynman, Richard P.; Kleinert, Hagen (1986). "Effective classical partition functions". Physical Review A (published December 1986). \textbf{34} (6): 5080–5084. Bibcode:1986PhRvA..34.5080F. doi:10.1103/PhysRevA.34.5080. PMID 9897894.
\item Feynman, Richard P. (1986). Rogers Commission Report, Volume 2 Appendix F – Personal Observations on Reliability of Shuttle. NASA.
\item Feynman, Richard P. (2000). Laurie M. Brown, ed. Selected Papers of Richard Feynman: With Commentary. 20th Century Physics. World Scientific. ISBN 978-981-02-4131-5.
\end{itemize}

\subsubsection{8.2 教材和课堂讲义}
\begin{figure}[ht]
\centering
\includegraphics[width=6cm]{./figures/2ef6e4e7a920d3c7.png}
\caption{费曼物理学讲义,包括费曼物理学小贴士:最终版和扩展版(2005年第2版)。} \label{fig_Feynma_7}
\end{figure}
费曼物理学讲义》也许是费曼最为物理爱好者所熟知的著作,该书汇编了费曼1961-1964年间在加州理工学院的本科生课程讲稿。随着费曼的讲课逐渐广为人知,甚至有专业物理学家和研究生慕名来听课。后来费曼的同事及合著者罗伯·莱顿和马修·桑兹编辑了他上课的讲稿,并将其著成了书。这项编写工作一直持续,并且依然有用。这套书在2005年又被重新编辑和补充,补充的内容被命名为《费曼物理学小贴士:费曼物理学讲义的问题解决补充》,其作者是迈克尔·戈特利布和拉尔夫·莱顿(罗伯特·莱顿之子),此次编辑和补充得到了基普·索恩等其他物理学家的大力支持。
\begin{itemize}
\item Feynman, Richard P.; Leighton, Robert B.; Sands, Matthew (2005) [1970]. The Feynman Lectures on Physics: The Definitive and Extended Edition (2nd ed.). Addison Wesley. ISBN 0-8053-9045-6. 包含《费曼的物理秘诀》 (与迈克尔·戈特利布和拉尔夫·莱顿合作),其中包括四个先前未发行的关于解决问题的讲座、罗伯特·莱顿和罗歇斯·沃格特的练习题以及马修·桑兹的一篇历史论文。共三卷,最初分别于1964年和1966年出版。
\item Feynman, Richard P. (1961). Theory of Fundamental Processes. Addison Wesley. ISBN 0-8053-2507-7.
\item Feynman, Richard P. (1962). Quantum Electrodynamics. Addison Wesley. ISBN 978-0-8053-2501-0.
\item Feynman, Richard P.; Hibbs, Albert (1965). Quantum Mechanics and Path Integrals. McGraw Hill. ISBN 0-07-020650-3.
\item Feynman, Richard P. (1967). The Character of Physical Law: The 1964 Messenger Lectures. MIT Press. ISBN 0-262-56003-8.
\item Feynman, Richard P. (1972). Statistical Mechanics: A Set of Lectures. Reading, Mass: W. A. Benjamin. ISBN 0-8053-2509-3.
\item Feynman, Richard P. (1985b). QED: The Strange Theory of Light and Matter. Princeton University Press. ISBN 0-691-02417-0.
\item Feynman, Richard P. (1987). Elementary Particles and the Laws of Physics: The 1986 Dirac Memorial Lectures. Cambridge University Press. ISBN 0-521-34000-4.
\item Feynman, Richard P. (1995). Brian Hatfield, ed. Lectures on Gravitation. Addison Wesley Longman. ISBN 0-201-62734-5.
\item Feynman, Richard P. (1997). Feynman's Lost Lecture: The Motion of Planets Around the Sun (Vintage Press ed.). London: Vintage. ISBN 0-09-973621-7.
\item Feynman, Richard P. (2000). Tony Hey and Robin W. Allen, ed. Feynman Lectures on Computation. Perseus Books Group. ISBN 0-7382-0296-7.
\end{itemize}

\subsubsection{8.3 通俗作品}
\begin{itemize}
\item Feynman, Richard P. (1985). Ralph Leighton, ed. Surely You're Joking, Mr. Feynman!: Adventures of a Curious Character. W. W. Norton & Co. ISBN 0-393-01921-7. OCLC 10925248.
\item Feynman, Richard P. (1988). Ralph Leighton, ed. What Do You Care What Other People Think?: Further Adventures of a Curious Character. W. W. Norton & Co. ISBN 0-393-02659-0.
\item No Ordinary Genius: The Illustrated Richard Feynman, ed. Christopher Sykes, W. W. Norton & Co, 1996, ISBN 0-393-31393-X.
\item Six Easy Pieces: Essentials of Physics Explained by Its Most Brilliant Teacher, Perseus Books, 1994, ISBN 0-201-40955-0. Listed by the Board of Directors of the Modern Library as one of the 100 best nonfiction books.[169]
\item Six Not So Easy Pieces: Einstein's Relativity, Symmetry and Space-Time, Addison Wesley, 1997, ISBN 0-201-15026-3.
\item Feynman, Richard P. (1998). The Meaning of It All: Thoughts of a Citizen Scientist. Reading, Massachusetts: Perseus Publishing. ISBN 0-7382-0166-9.
\item Feynman, Richard P. (1999). Robbins, Jeffrey, ed. The Pleasure of Finding Things Out: The Best Short Works of Richard P. Feynman. Cambridge, Massachusetts: Perseus Books. ISBN 0-7382-0108-1.
\item Classic Feynman: All the Adventures of a Curious Character, edited by Ralph Leighton, W. W. Norton & Co, 2005, ISBN 0-393-06132-9. Chronologically reordered omnibus volume of Surely You're Joking, Mr. Feynman! and What Do You Care What Other People Think?, with a bundled CD containing one of Feynman's signature lectures.
\end{itemize}

\subsubsection{8.4 音像记录}
\begin{itemize}
\item 保险箱窃贼的套房 (夹杂着费曼讲述的轶事和一些击鼓声)
\item 仰望洛斯阿拉莫斯 (音频,费曼于1975年2月6日在圣巴巴拉的演讲)
\item 六个简单的部分 (同名书作所参考的原始讲座)
\item 六个不容易的部分 (同名书作所参考的原始讲座)
\item 费曼物理学讲义:完整的音频收藏
\item 费曼击鼓、吟诵和演讲的样本被收集于专辑《回到图瓦的未来,冒险继续》中的歌曲Tuva Groove (Bolur Daa-Bol, Bolbas Daa-Bol)和Kargyraa Rap (Dürgen Chugaa)中, 作者为Kongar-ool Ondar。该专辑的隐藏曲目还包括没有音乐背景的讲座摘录。
\item 1964年在康奈尔大学举办的信使讲座,费曼在讲座中解释了物理学的基本主题。可在图瓦计划中免费获得。[170] (参见图书《物理定律的本性》)
\item 换个角度看世界——与理查德·费曼的录像:胡氏影视(1972)
\item The Douglas Robb Memorial Lectures,四次公开演讲,其中四章转录于图书《QED:光和物质的奇妙理论》中。(1979年)
\item The Pleasure of Finding Things Out, BBC地平线系列插曲(1981)(勿与后来出版的同名书籍混淆)
\item Richard Feynman: Fun to Imagine CollectionBBC存档的费曼六部短片,该短片以通俗易懂的方式讲述了不同的事件背后共同的物理原理。(1983年)
\item 基本粒子和物理定律 (1986年)
\item 微型机器:费曼谈纳米技术 (视频,1984年)
\item 由里而外谈计算机 (视频)
\item 现实中的量子力学观点:Esalen研讨会(视频,1983)
\item 特质思维研讨会(视频,1985)
\item 零碎杂录——来自理查德的生活和时代(视频,1988年)
\item 陌生度减三(视频), BBC地平线系列,1964年)
\item 不是普通的天才 (录像,克里斯托夫·赛克斯纪录片)
\item 理查德·费曼——The Best Mind Since Einstein (视频、纪录片)
\item 行星围绕太阳的运动(音频,又名为“费曼失落的演讲”)
\item 物质的本质(音频)[171]
\end{itemize}

\subsection{笔记}
\begin{enumerate}
\item Tindol, Robert (December 2, 1999). "Physics World poll names Richard Feynman one of 10 greatest physicists of all time" (Press release). California Institute of Technology. Archived from the original on March 21, 2012. Retrieved December 1, 2012.
\item "Richard P. Feynman – Biographical". The Nobel Foundation. Retrieved April 23, 2013.
\item J. J. O'Connor; E. F. Robertson (August 2002). "Richard Phillips Feynman". University of St. Andrews. Retrieved April 23, 2013.
\item  p. 231.
\item "Richard Phillips Feynman". Timeline of Nobel Prize Winners. Retrieved April 23, 2013.
\item  p. 34.
\item  p. 58.
\item  p. 54.
\item  p. 231.
\item  p. 8.
\item  pp. 25–26.
\item Hirshberg, Charles (May 2002). "My Mother, the Scientist". Popular Science. Archived from the original on June 20, 2016. Retrieved April 23, 2013 – via www.aas.org.
\item , p. 25.
\item Harrison, John. "Physics, bongos and the art of the nude". The Daily Telegraph. Retrieved April 23, 2013.
\item  pp. 284–287.
\item  p. 374.
\item "Richard Feynman – Biography". Atomic Archive. Retrieved July 12, 2016.
\item  p. 24.
\item  p. 15.
\item  p. 41.
\item  p. 72.
\item Feynman, R. P. (August 1939). "Forces in Molecules". Physical Review. American Physical Society. 56 (4): 340–343. Bibcode:1939PhRv...56..340F. doi:10.1103/PhysRev.56.340.
\item  pp. 71–78.
\item  p. 56.
\item "Putnam Competition Individual and Team Winners". Mathematical Association of America. Retrieved March 8, 2014.
\item  p. 84.
\item  pp. 66–67.
\item  pp. 150–151.
\item  pp. 63–64.
\item  pp. 99–103.
\item  pp. 64–65.
\item  pp. 107–108.
\item  pp. 141–145.
\item  p. 59.
\item  pp. 158–160.
\item  pp. 165–169.
\item  pp. 157–159.
\item  p. 183.
\item  p. 14.
\item  p. 78.
\item  pp. 125–129.
\item  pp. 403–407.
\item  pp. 407–409.
\item Wellerstein, Alex (June 6, 2014). "Feynman and the Bomb". Restricted Data. Retrieved December 4, 2016.
\item  p. 122.
\item  pp. 414–422.
\item  p. 257.
\item  pp. 95–96.
\item  pp. 188–189.
\item  pp. 147–149.
\item  p. 99.
\item  p. 184.
\item  p. 96.
\item  pp. 296–297.
\item Michael Morisy; Robert Hovden (June 6, 2012). J Pat Brown, ed. "The Feynman Files: The professor's invitation past the Iron Curtain". MuckRock. Retrieved July 13, 2016.
\item  pp. 200–202.
\item  p. 134.
\item  p. 101.
\item Robert H. March (2003). "Physics at the University of Wisconsin: A History". Physics in Perspective. 5 (2): 130–149. Bibcode:2003PhP.....5..130M. doi:10.1007/s00016-003-0142-6.
\item  pp. 161–164, 178–179.
\item  pp. 47–52.
\item  p. 316.
\item  p. 205.
\item  p. 225.
\item  pp. 162–163.
\item  pp. 171–174.
\item "I love my wife. My wife is dead". Letters of Note. February 15, 2012. Retrieved April 23, 2013.
\item  pp. 227–229.
\item  pp. 213–214.
\item  p. 232.
\item  p. 217.
\item  pp. 218–219.
\item  pp. 223–228.
\item  pp. 229–234.
\item "Richard P. Feynman – Nobel Lecture: The Development of the Space-Time View of Quantum Electrodynamics". Nobel Foundation. December 11, 1965. Retrieved July 14, 2016.
\item  pp. 246–248.
\item  pp. 256–258.
\item  pp. 267–269.
\item Dyson, F. J. (1949). "The radiation theories of Tomonaga, Schwinger, and Feynman". Physical Review. 75 (3): 486–502. Bibcode:1949PhRv...75..486D. doi:10.1103/PhysRev.75.486.
\item  pp. 271–272.
\item  pp. 251–252.
\item  pp. 271–272.
\item  pp. 301–302.
\item  pp. 275–276.
\item Kac, Mark (1949). "On Distributions of Certain Wiener Functionals". Transactions of the American Mathematical Society. 65 (1): 1–13. doi:10.2307/1990512. JSTOR 1990512.
\item  p. 276.
\item  p. 277.
\item  p. 287.
\item  pp. 232–233.
\item Feynman, Richard. Richard Feynman – Session III (访谈者 : Charles Weiner). American Institute of Physics. March 5, 1966 [June 19, 2016].
\item  p. 191.
\item  p. 333.
\item  p. 278.
\item  p. 296.
\item  p. 98.
\item  p. 120.
\item  p. 331.
\item  pp. 283–286.
\item  pp. 322–327.
\item "Calisphere: Richard Feynman playing the conga drum [full photo]". Calisphere (in 英语). Retrieved 2019-05-13.
\item  pp. 125–127.
\item  pp. 233–236.
\item  pp. 291–294.
\item Wellerstein, Alex (July 11, 2014). "Who smeared Richard Feynman?". Restricted Data. Retrieved July 15, 2016.
\item  p. 168.
\item  pp. 339–347.
\item  pp. 151–153.
\item "A Weekend at Richard Feynman's House". It's Just A Life Story. Retrieved July 15, 2016.
\item  pp. 405–406.
\item  pp. 330–337.
\item  pp. 204–205.
\item  pp. 299–303.
\item Pines, David (1989). "Richard Feynman and Condensed Matter Physics". Physics Today. 42 (2): 61. Bibcode:1989PhT....42b..61P. doi:10.1063/1.881194.
\item  pp. 92–101.
\item  pp. 330–339.
\item  pp. 387–396.
\item  pp. 507–514.
\item  pp. 516–519.
\item  pp. 505–507.
\item  p. 189.
\item  p. 170.
\item West, Jacob (June 2003). "The Quantum Computer" (PDF). Archived from the original (PDF) on March 15, 2015. Retrieved September 20, 2009.
\item , pp. 57–61.
\item Kleinert, Hagen (1999). "Specific heat of liquid helium in zero gravity very near the lambda point". Physical Review D. 60 (8): 085001. arXiv:hep-th/9812197. Bibcode:1999PhRvD..60h5001K. doi:10.1103/PhysRevD.60.085001.
\item Lipa, J. A.; Nissen, J.; Stricker, D.; Swanson, D.; Chui, T. (2003). "Specific heat of liquid helium in zero gravity very near the lambda point". Physical Review B. 68 (17): 174518. arXiv:cond-mat/0310163. Bibcode:2003PhRvB..68q4518L. doi:10.1103/PhysRevB.68.174518.
\item  pp. 357–364.
\item  pp. 12–13.
\item  pp. 184–185.
\item Feynman, Richard P. (June 1974). "Cargo Cult Science" (PDF). Engineering and Science. California Institute of Technology. 37 (7): 10–13. ISSN 0013-7812. Retrieved June 15, 2016.
\item Van Kortryk, T (May 2017). "The doctoral students of Richard Feynman". Physics Today. doi:10.1063/PT.5.9100.
\item "Interview with Jenijoy La Belle" (PDF). Caltech. Retrieved June 15, 2016.
\item  pp. 409–411.
\item  p. 411.
\item Johnson, George (July 2001). "The Jaguar and the Fox". The Atlantic. Retrieved July 16, 2016.
\item YouTube上的Murray Gell-Mann talks about Richard Feynman in January 12, 2012
\item "Mr. Feynman Goes to Washington". The Attic. Retrieved July 14, 2018.
\item  p. 423.
\item  p. 151.
\item James Gleick (February 17, 1988). "Richard Feynman Dead at 69; Leading Theoretical Physicist". The New York Times. Retrieved April 23, 2013.
\item Richard Feynman. "Appendix F – Personal observations on the reliability of the Shuttle". Kennedy Space Center. Retrieved September 11, 2017.
\item  pp. 295–296.
\item "Award Laureates". United States Department of Energy. Retrieved July 15, 2016.
\item "The Nobel Prize in Physics 1965". The Nobel Foundation. Retrieved July 15, 2016.
\item Mehra, J. (2002). "Richard Phillips Feynman 11 May 1918 – 15 February 1988". Biographical Memoirs of Fellows of the Royal Society. 48: 97–128. doi:10.1098/rsbm.2002.0007.
\item "The Oersted Medal". American Association of Physics Teachers. Retrieved July 15, 2016.
\item "The President's National Medal of Science: Recipient Details". National Science Foundation. Retrieved July 15, 2016.
\item Toumey, Chris (2005). "SPT v8n3 - Reviews - Feynman Unprocessed". vt.edu. Virginia Tech. Archived from the original on 2019-03-19.
\item Feynman, Richard; Feynman, Michelle (2005). Perfectly reasonable deviations from the beaten track : the letters of Richard P. Feynman. New York: Basic Books. ISBN 0738206369. OCLC 57393623.
\item  p. 13.
\item  pp. 600–605.
\item  pp. 437–438.
\item "Video of Danny Hillis Speaking about his conversation with Feynman about his dying". The Long Now. Retrieved December 13, 2016.
\item  pp. 257–258.
\item Rasmussen, Cecilia (June 5, 2005). "History Exhumed Via Computer Chip". Los Angeles Times. Retrieved January 30, 2017.
\item Holden, Stephen (4 October 1996). "A Man, a Woman and an Atomic Bomb". The New York Times. Retrieved 18 June 2017.
\item 百老汇网路资料库(IDBD)上《QED》的资料(英文)
\item "Real Time Opera productions". Real Time Opera. Retrieved January 30, 2017.
\item .
\item "The Challenger". BBC. Retrieved March 18, 2013.
\item "The Challenger". BBC Two. Retrieved March 19, 2013.
\item Goldberg, Lesley (September 26, 2012). "William Hurt to Star in Science Channel/BBC Challenger Docu-Drama (Exclusive)". The Hollywood Reporter. Retrieved January 30, 2017.
\item p. 8.
\item "Who is Richard Feynman?". feynmangroup.com. Archived from the original on November 5, 2011. Retrieved December 1, 2012.
\item "American Scientists Series Slideshow". beyondtheperf.com. Archived from the original on May 23, 2013. Retrieved December 1, 2012.
\item "Fermilab Open House: Computing Division". fnal.gov. Retrieved December 1, 2012.
\item Great Mind Richard Feynman Birthday | Manhattan Project and Challenger Disaster | Quantum Electrodynamics | Biography. Techie-buzz.com (May 10, 2011). Retrieved on May 6, 2012.
\item Miller, Anthony (March 13, 2013). "Big Bang Theory: Sheldon's Top 5 Moments". Los Angeles Magazine. Retrieved January 31, 2017.
\item Gates, Bill. "The Best Teacher I Never Had". The Gates Notes. Retrieved January 29, 2016.
\item "100 Best Nonfiction". Modern Library. Retrieved November 12, 2016.
\item "Richard Feynman Messenger Lectures (1964)". Cornell University. Retrieved November 12, 2016.
\item Feynman, Richard. "Richard Feynman's Poignant Letter to His Departed Wife Arline: Watch Actor Oscar Isaac Read It Live Onstage". OpenCulture. Retrieved December 2, 2016.
\end{enumerate}

\subsection{参考文献}
[1]
^Tindol, Robert (December 2, 1999). "Physics World poll names Richard Feynman one of 10 greatest physicists of all time" (Press release). California Institute of Technology. Archived from the original on March 21, 2012. Retrieved December 1, 2012..

[2]
^"Richard P. Feynman – Biographical". The Nobel Foundation. Retrieved April 23, 2013..

[3]
^J. J. O'Connor; E. F. Robertson (August 2002). "Richard Phillips Feynman". University of St. Andrews. Retrieved April 23, 2013..

[4]
^Oakes 2007, p. 231..

[5]
^"Richard Phillips Feynman". Timeline of Nobel Prize Winners. Retrieved April 23, 2013..

[6]
^Chown 1985, p. 34..

[7]
^Close 2011, p. 58..

[8]
^Sykes 1994, p. 54..

[9]
^Friedman 2004, p. 231..

[10]
^Henderson 2011, p. 8..

[11]
^Gleick 1992, pp. 25–26..

[12]
^Hirshberg, Charles (May 2002). "My Mother, the Scientist". Popular Science. Archived from the original on June 20, 2016. Retrieved April 23, 2013 – via www.aas.org..

[13]
^Feynman 1988, p. 25..

[14]
^Harrison, John. "Physics, bongos and the art of the nude". The Daily Telegraph. Retrieved April 23, 2013..

[15]
^Feynman 1985, pp. 284–287..

[16]
^Schweber 1994, p. 374..

[17]
^"Richard Feynman – Biography". Atomic Archive. Retrieved July 12, 2016..

[18]
^Feynman 1985, p. 24..

[19]
^Gleick 1992, p. 15..

[20]
^Mehra 1994, p. 41..

[21]
^Feynman 1985, p. 72..

[22]
^Feynman, R. P. (August 1939). "Forces in Molecules". Physical Review. American Physical Society. 56 (4): 340–343. Bibcode:1939PhRv...56..340F. doi:10.1103/PhysRev.56.340..

[23]
^Mehra 1994, pp. 71–78..

[24]
^Gribbin & Gribbin 1997, p. 56..

[25]
^"Putnam Competition Individual and Team Winners". Mathematical Association of America. Retrieved March 8, 2014..

[26]
^Gleick 1992, p. 84..

[27]
^Gribbin & Gribbin 1997, pp. 66–67..

[28]
^Gleick 1992, pp. 150–151..

[29]
^Gribbin & Gribbin 1997, pp. 63–64..

[30]
^Feynman 1985, pp. 99–103..

[31]
^Gribbin & Gribbin 1997, pp. 64–65..

[32]
^Feynman 1985, pp. 107–108..

[33]
^Gleick 1992, pp. 141–145..

[34]
^Hoddeson et al. 1993, p. 59..

[35]
^Gleick 1992, pp. 158–160..

[36]
^Gleick 1992, pp. 165–169..

[37]
^Hoddeson et al. 1993, pp. 157–159..

[38]
^Hoddeson et al. 1993, p. 183..

[39]
^Bashe et al. 1986, p. 14..

[40]
^Hillis 1989, p. 78..

[41]
^Feynman 1985, pp. 125–129..

[42]
^Galison 1998, pp. 403–407..

[43]
^Galison 1998, pp. 407–409..

[44]
^Wellerstein, Alex (June 6, 2014). "Feynman and the Bomb". Restricted Data. Retrieved December 4, 2016..

[45]
^Feynman 1985, p. 122..

[46]
^Galison 1998, pp. 414–422..

[47]
^Gleick 1992, p. 257..

[48]
^Gribbin & Gribbin 1997, pp. 95–96..

[49]
^Gleick 1992, pp. 188–189..

[50]
^Feynman 1985, pp. 147–149..

[51]
^Gribbin & Gribbin 1997, p. 99..

[52]
^Gleick 1992, p. 184..

[53]
^Gribbin & Gribbin 1997, p. 96..

[54]
^Gleick 1992, pp. 296–297..

[55]
^Michael Morisy; Robert Hovden (June 6, 2012). J Pat Brown, ed. "The Feynman Files: The professor's invitation past the Iron Curtain". MuckRock. Retrieved July 13, 2016..

[56]
^Gleick 1992, pp. 200–202..

[57]
^Feynman 1985, p. 134..

[58]
^Gribbin & Gribbin 1997, p. 101..

[59]
^Robert H. March (2003). "Physics at the University of Wisconsin: A History". Physics in Perspective. 5 (2): 130–149. Bibcode:2003PhP.....5..130M. doi:10.1007/s00016-003-0142-6..

[60]
^Mehra 1994, pp. 161–164, 178–179..

[61]
^Hoddeson et al. 1993, pp. 47–52..

[62]
^Hoddeson et al. 1993, p. 316..

[63]
^Gleick 1992, p. 205..

[64]
^Gleick 1992, p. 225..

[65]
^Feynman 1985, pp. 162–163..

[66]
^Mehra 1994, pp. 171–174..

[67]
^"I love my wife. My wife is dead". Letters of Note. February 15, 2012. Retrieved April 23, 2013..

[68]
^Gleick 1992, pp. 227–229..

[69]
^Mehra 1994, pp. 213–214..

[70]
^Gleick 1992, p. 232..

[71]
^Mehra 1994, p. 217..

[72]
^Mehra 1994, pp. 218–219..

[73]
^Mehra 1994, pp. 223–228..

[74]
^Mehra 1994, pp. 229–234..

[75]
^"Richard P. Feynman – Nobel Lecture: The Development of the Space-Time View of Quantum Electrodynamics". Nobel Foundation. December 11, 1965. Retrieved July 14, 2016..

[76]
^Mehra 1994, pp. 246–248..

[77]
^Gleick 1992, pp. 256–258..

[78]
^Gleick 1992, pp. 267–269..

[79]
^Dyson, F. J. (1949). "The radiation theories of Tomonaga, Schwinger, and Feynman". Physical Review. 75 (3): 486–502. Bibcode:1949PhRv...75..486D. doi:10.1103/PhysRev.75.486..

[80]
^Gleick 1992, pp. 271–272..

[81]
^Mehra 1994, pp. 251–252..

[82]
^Mehra 1994, pp. 271–272..

[83]
^Mehra 1994, pp. 301–302..

[84]
^Gleick 1992, pp. 275–276..

[85]
^Kac, Mark (1949). "On Distributions of Certain Wiener Functionals". Transactions of the American Mathematical Society. 65 (1): 1–13. doi:10.2307/1990512. JSTOR 1990512..

[86]
^Gleick 1992, p. 276..

[87]
^Gleick 1992, p. 277..

[88]
^Gleick 1992, p. 287..

[89]
^Feynman 1985, pp. 232–233..

[90]
^Feynman, Richard. Richard Feynman – Session III (访谈者 : Charles Weiner). American Institute of Physics. March 5, 1966 [June 19, 2016]..

[91]
^Feynman 2005, p. 191..

[92]
^Mehra 1994, p. 333..

[93]
^Gleick 1992, p. 278..

[94]
^Gleick 1992, p. 296..

[95]
^Peat 1997, p. 98..

[96]
^Peat 1997, p. 120..

[97]
^Mehra 1994, p. 331..

[98]
^Gleick 1992, pp. 283–286..

[99]
^Feynman 1985, pp. 322–327..

[100]
^"Calisphere: Richard Feynman playing the conga drum [full photo]". Calisphere (in 英语). Retrieved 2019-05-13..

[101]
^Peat 1997, pp. 125–127..

[102]
^Feynman 1985, pp. 233–236..

[103]
^Gleick 1992, pp. 291–294..

[104]
^Wellerstein, Alex (July 11, 2014). "Who smeared Richard Feynman?". Restricted Data. Retrieved July 15, 2016..

[105]
^Krauss 2011, p. 168..

[106]
^Gleick 1992, pp. 339–347..

[107]
^Gribbin & Gribbin 1997, pp. 151–153..

[108]
^"A Weekend at Richard Feynman's House". It's Just A Life Story. Retrieved July 15, 2016..

[109]
^Gleick 1992, pp. 405–406..

[110]
^Feynman 1985, pp. 330–337..

[111]
^Feynman 1985, pp. 204–205..

[112]
^Gleick 1992, pp. 299–303..

[113]
^Pines, David (1989). "Richard Feynman and Condensed Matter Physics". Physics Today. 42 (2): 61. Bibcode:1989PhT....42b..61P. doi:10.1063/1.881194..

[114]
^Mehra 1994, pp. 92–101..

[115]
^Gleick 1992, pp. 330–339..

[116]
^Gleick 1992, pp. 387–396..

[117]
^Mehra 1994, pp. 507–514..

[118]
^Mehra 1994, pp. 516–519..

[119]
^Mehra 1994, pp. 505–507..

[120]
^Gribbin & Gribbin 1997, p. 189..

[121]
^Gribbin & Gribbin 1997, p. 170..

[122]
^West, Jacob (June 2003). "The Quantum Computer" (PDF). Archived from the original (PDF) on March 15, 2015. Retrieved September 20, 2009..

[123]
^Deutsch 1992, pp. 57–61..

[124]
^Kleinert, Hagen (1999). "Specific heat of liquid helium in zero gravity very near the lambda point". Physical Review D. 60 (8): 085001. arXiv:hep-th/9812197. Bibcode:1999PhRvD..60h5001K. doi:10.1103/PhysRevD.60.085001..

[125]
^Lipa, J. A.; Nissen, J.; Stricker, D.; Swanson, D.; Chui, T. (2003). "Specific heat of liquid helium in zero gravity very near the lambda point". Physical Review B. 68 (17): 174518. arXiv:cond-mat/0310163. Bibcode:2003PhRvB..68q4518L. doi:10.1103/PhysRevB.68.174518..

[126]
^Gleick 1992, pp. 357–364..

[127]
^Gleick 1992, pp. 12–13..

[128]
^Feynman 1999, pp. 184–185..

[129]
^Feynman, Richard P. (June 1974). "Cargo Cult Science" (PDF). Engineering and Science. California Institute of Technology. 37 (7): 10–13. ISSN 0013-7812. Retrieved June 15, 2016..

[130]
^Van Kortryk, T (May 2017). "The doctoral students of Richard Feynman". Physics Today. doi:10.1063/PT.5.9100..

[131]
^"Interview with Jenijoy La Belle" (PDF). Caltech. Retrieved June 15, 2016..
[132]
^Gleick 1992, pp. 409–411..

[133]
^Gleick 1992, p. 411..
[134]
^Johnson, George (July 2001). "The Jaguar and the Fox". The Atlantic. Retrieved July 16, 2016..
[135]
^YouTube上的Murray Gell-Mann talks about Richard Feynman in January 12, 2012.
[136]
^"Mr. Feynman Goes to Washington". The Attic. Retrieved July 14, 2018..
[137]
^Gleick 1992, p. 423..
[138]
^Feynman 1988, p. 151..
[139]
^James Gleick (February 17, 1988). "Richard Feynman Dead at 69; Leading Theoretical Physicist". The New York Times. Retrieved April 23, 2013..
[140]
^Richard Feynman. "Appendix F – Personal observations on the reliability of the Shuttle". Kennedy Space Center. Retrieved September 11, 2017..
[141]
^Gleick 1992, pp. 295–296..
[142]
^"Award Laureates". United States Department of Energy. Retrieved July 15, 2016..
[143]
^"The Nobel Prize in Physics 1965". The Nobel Foundation. Retrieved July 15, 2016..
[144]
^Mehra, J. (2002). "Richard Phillips Feynman 11 May 1918 – 15 February 1988". Biographical Memoirs of Fellows of the Royal Society. 48: 97–128. doi:10.1098/rsbm.2002.0007..
[145]
^"The Oersted Medal". American Association of Physics Teachers. Retrieved July 15, 2016..
[146]
^"The President's National Medal of Science: Recipient Details". National Science Foundation. Retrieved July 15, 2016..
[147]
^Toumey, Chris (2005). "SPT v8n3 - Reviews - Feynman Unprocessed". vt.edu. Virginia Tech. Archived from the original on 2019-03-19..
[148]
^Feynman, Richard; Feynman, Michelle (2005). Perfectly reasonable deviations from the beaten track : the letters of Richard P. Feynman. New York: Basic Books. ISBN 0738206369. OCLC 57393623..
[149]
^Feynman 1999, p. 13..
[150]
^Mehra 1994, pp. 600–605..
[151]
^Gleick 1992, pp. 437–438..
[152]
^"Video of Danny Hillis Speaking about his conversation with Feynman about his dying". The Long Now. Retrieved December 13, 2016..
[153]
^Gribbin & Gribbin 1997, pp. 257–258..
[154]
^Rasmussen, Cecilia (June 5, 2005). "History Exhumed Via Computer Chip". Los Angeles Times. Retrieved January 30, 2017..
[155]
^Holden, Stephen (4 October 1996). "A Man, a Woman and an Atomic Bomb". The New York Times. Retrieved 18 June 2017..
[156]
^百老汇网路资料库(IDBD)上《QED》的资料(英文).
[157]
^"Real Time Opera productions". Real Time Opera. Retrieved January 30, 2017..
[158]
^Ottaviani & Myrick 2011..
[159]
^"The Challenger". BBC. Retrieved March 18, 2013..
[160]
^"The Challenger". BBC Two. Retrieved March 19, 2013..
[161]
^Goldberg, Lesley (September 26, 2012). "William Hurt to Star in Science Channel/BBC Challenger Docu-Drama (Exclusive)". The Hollywood Reporter. Retrieved January 30, 2017..
[162]
^Wolfram 2016, p. 8..
[163]
^"Who is Richard Feynman?". feynmangroup.com. Archived from the original on November 5, 2011. Retrieved December 1, 2012..
[164]
^"American Scientists Series Slideshow". beyondtheperf.com. Archived from the original on May 23, 2013. Retrieved December 1, 2012..
[165]
^"Fermilab Open House: Computing Division". fnal.gov. Retrieved December 1, 2012..
[166]
^Great Mind Richard Feynman Birthday | Manhattan Project and Challenger Disaster | Quantum Electrodynamics | Biography. Techie-buzz.com (May 10, 2011). Retrieved on May 6, 2012..
[167]
^Miller, Anthony (March 13, 2013). "Big Bang Theory: Sheldon's Top 5 Moments". Los Angeles Magazine. Retrieved January 31, 2017..
[168]
^Gates, Bill. "The Best Teacher I Never Had". The Gates Notes. Retrieved January 29, 2016..
[169]
^"100 Best Nonfiction". Modern Library. Retrieved November 12, 2016..
[170]
^"Richard Feynman Messenger Lectures (1964)". Cornell University. Retrieved November 12, 2016..
[171]
^Feynman, Richard. "Richard Feynman's Poignant Letter to His Departed Wife Arline: Watch Actor Oscar Isaac Read It Live Onstage". OpenCulture. Retrieved December 2, 2016..