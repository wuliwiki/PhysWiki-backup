% 格奥尔格·康托尔(综述)
% license CCBYSA3
% type Wiki

本文根据 CC-BY-SA 协议转载翻译自维基百科\href{https://en.wikipedia.org/wiki/Georg_\%E5\%BA\%B7\%E6\%89\%98\%E5\%B0\%94}{相关文章}。

\begin{figure}[ht]
\centering
\includegraphics[width=6cm]{./figures/409c043d8a08ce28.png}
\caption{} \label{fig_Canto_1}
\end{figure}
乔治·费迪南德·路德维希·菲利普·康托尔(Georg Ferdinand Ludwig Philipp 康托尔,/ˈkæntɔːr/ KAN-tor;德语发音:[ˈɡeːɔʁk ˈfɛʁdinant ˈluːtvɪç ˈfiːlɪp ˈkantoːɐ̯];1845年3月3日(旧历2月19日)-1918年1月6日)是一位数学家,他在集合论的创建中发挥了关键作用,集合论已成为数学中的一项基本理论。康托尔确立了两个集合成员之间一对一对应的重要性,定义了无限集合和良序集合,并证明了实数比自然数更多。康托尔证明该定理的方法意味着存在无数个不同大小的无限集合。他定义了基数和序数以及它们的算术运算。康托尔的工作在哲学上具有重大意义,他对此非常清楚。

最初,康托尔的超无限数理论被认为是反直觉的——甚至是震惊的。这使得它遭遇了数学界 contemporaries 的抵制,如利奥波德·克罗内克和亨利·庞加莱[3],以及后来的赫尔曼·外尔和L·E·J·布劳威尔,而路德维希·维特根斯坦则提出了哲学上的反对意见;参见康托尔理论的争议。康托尔是一个虔诚的路德宗基督徒[4],他认为这一理论是上帝传达给他的[5]。一些基督教神学家(尤其是新经院哲学家)认为康托尔的工作挑战了上帝本质中绝对无限的独特性[6]——曾有一次将超无限数理论与泛神论等同起来[7]——这一命题被康托尔坚决拒绝。然而,并非所有神学家都反对康托尔的理论;著名的新经院哲学家康斯坦丁·古特贝尔特支持这一理论,而枢机主教约翰·巴普蒂斯特·弗朗泽林在康托尔做出一些重要澄清后也接受了这一理论作为有效理论[8]。

对康托尔工作的反对有时非常激烈:利奥波德·克罗内克的公开反对和个人攻击包括将康托尔描述为“科学江湖医生”、“叛徒”和“青年堕落者”[9]。克罗内克反对康托尔证明代数数是可数的,以及超越数是不可数的,这些结果如今已被纳入标准数学课程中。维特根斯坦在康托尔去世几十年后写道,他感叹数学“完全被集合论的有害习语所支配”,并将其斥为“完全的胡说八道”,“可笑”且“错误”[10]。从1884年到他生命的尽头,康托尔反复遭遇抑郁症,这被归咎于许多同时代人对他的敌对态度[11],尽管也有人将这些症状解释为双相情感障碍的可能表现[12]。

激烈的批评与后来的赞誉相匹配。1904年,皇家学会授予康托尔西尔维斯特奖,这是它能授予数学工作者的最高荣誉[13]。大卫·希尔伯特为其辩护,宣称:“没有人能够将我们从康托尔创造的乐园中驱逐出去”[14][15]。
\subsection{传记}  
\subsubsection{青年时期与学业}
\begin{figure}[ht]
\centering
\includegraphics[width=6cm]{./figures/b8db46daef6392b0.png}
\caption{康托尔,约1870年} \label{fig_Canto_2}
\end{figure}
乔治·康托尔,1845年出生于俄罗斯帝国圣彼得堡,直到十一岁时一直在该市成长。他是六个孩子中的长子,被认为是一位出色的小提琴手。他的祖父弗朗茨·伯姆(Franz Böhm,1788–1846)(小提琴家约瑟夫·伯姆的兄弟)是俄罗斯帝国乐团著名的音乐家和独奏家[16]。康托尔的父亲曾是圣彼得堡证券交易所的一名成员;当他生病时,家人于1856年搬到德国,首先定居在威斯巴登,随后又迁往法兰克福,寻求比圣彼得堡温和的冬季。1860年,康托尔以优异的成绩从达姆施塔特的实科学校毕业;他的数学天赋,尤其是在三角学方面,得到了认可。1862年8月,他从达姆施塔特的“高等职业学校”毕业,现为达姆施塔特工业大学[17][18]。同年,他进入瑞士联邦理工学院苏黎世校区学习。1863年6月,父亲去世后,康托尔获得了一笔可观的遗产,他转学到柏林大学,听取了利奥波德·克罗内克、卡尔·魏尔斯特拉斯和恩斯特·库默的讲座。他于1866年夏季在哥廷根大学度过,那时和后来哥廷根是数学研究的中心。康托尔是一位优秀的学生,并于1867年获得博士学位[19][20]。
\subsubsection{教师与研究员}  
康托尔于1867年在柏林大学提交了他的关于数论的博士论文。在柏林的一所女子学校短暂任教后,他在哈雷大学获得了职位,并在那里度过了他的一生职业生涯。他凭借关于数论的论文获得了必要的资格认证,并于1869年在哈雷大学任职时提交了该论文[20][21]。

1874年,康托尔与瓦莉·古特曼结婚,他们共有六个孩子,最小的一个(鲁道夫)出生于1886年。尽管康托尔的学术工资 modest,他依靠父亲的遗产得以养活家庭。在蜜月期间,康托尔与理查德·德德金德进行了许多数学讨论,他们在两年前在瑞士因度假而相识,当时康托尔在哈茨山脉度假[22]。

康托尔于1872年晋升为特聘教授,并于1879年成为正教授[20][19]。在34岁时获得这一职位是一个值得注意的成就,但康托尔渴望能够在更有声望的大学担任教职,尤其是在当时德国领先的大学——柏林大学。然而,他的工作遭遇了过多的反对,使得这一目标难以实现[23]。克罗内克是柏林大学数学系的负责人,直到1891年去世,他对康托尔作为同事的前景感到越来越不安[24],认为他是“堕落青年”的人,因为康托尔将自己的观点教授给年轻一代数学家[25]。更糟糕的是,克罗内克,这位在数学界已经确立地位并曾是康托尔的教授,自从故意推迟康托尔的第一篇重要论文于1874年发表以来,就从根本上不同意康托尔工作的核心内容[20]。克罗内克,现被视为数学中构造性观点的奠基人之一,不喜欢康托尔的大部分集合论,因为它宣称存在满足特定性质的集合,却没有提供具体的集合实例,证明其成员确实满足这些性质。每当康托尔申请柏林的职位时,他总是被拒绝,而这一过程通常涉及克罗内克[20],因此康托尔开始相信,克罗内克的立场将使他永远无法离开哈雷[26]。

1881年,康托尔在哈雷的同事爱德华·海内去世。哈雷大学接受了康托尔的建议,提议将海内留下的职位依次提供给德德金德、海因里希·M·韦伯和弗朗茨·梅尔滕斯,但每位教授在被邀请后都拒绝了这个职位。最终,弗里德里希·万格林被任命,但他与康托尔并未建立亲近关系[27]。

1882年,康托尔与德德金德的数学通信结束,显然是因为德德金德拒绝了哈雷大学的职位[28]。康托尔还开始了另一重要的通信,与瑞典的Gösta Mittag-Leffler,并很快开始在Mittag-Leffler的期刊《数学年刊》(Acta Mathematica)上发表文章。但在1885年,Mittag-Leffler对康托尔提交给《数学年刊》的论文中的哲学性质和新术语感到担忧[29]。他要求康托尔在论文校样阶段撤回该论文,并写道:“……大约晚了一百年。”康托尔同意了,但随后缩短了与Mittag-Leffler的关系和通信,他写信给第三方说:“如果Mittag-Leffler有他的意愿,我就得等到1984年,而对我来说那简直是个过大的要求!……不过当然,我再也不想知道关于《数学年刊》的任何事情了”[30]。

康托尔在1884年5月首次出现已知的抑郁症发作[19][31]。对他工作的批评让他感到沉重:他在1884年写给Mittag-Leffler的五十二封信中,每一封都提到了克罗内克。其中一封信中的一段话揭示了康托尔自信心受损的情况:

“…我不知道何时才能继续我的科学工作。目前我根本做不下去,只能做最基本的讲座职责;如果我有足够的心力,我多么希望能够继续科学上的工作,哪怕只是活跃于此。[32]”

这次危机促使他申请讲授哲学,而不是数学。他还开始深入研究伊丽莎白时代的文学,认为可能有证据表明弗朗西斯·培根写了那些被归于威廉·莎士比亚的戏剧(见莎士比亚作者问题);最终,这导致了两本小册子的出版,分别是在1896年和1897年[33]。

康托尔很快恢复了,随后做出了更多重要的贡献,包括他的对角线论证和定理。然而,尽管克罗内克于1891年12月29日去世[20],他再也未能达到1874年至1884年期间那些杰出论文的高度。他最终寻求并达成了与克罗内克的和解。然而,他们之间的哲学分歧和困难依旧存在。

1889年,康托尔在成立德国数学学会方面发挥了重要作用[20],并在1891年主持了该学会的第一次会议,会议上他首次介绍了他的对角线论证;尽管克罗内克反对他的工作,康托尔的声誉足够强大,以确保他当选为该学会的首任会长。抛开克罗内克对他的敌意,康托尔邀请他在会议上发言,但克罗内克由于当时妻子在滑雪事故中受伤致死,未能如愿。乔治·康托尔还在1897年推动了第一届国际数学家大会的成立,该大会在瑞士苏黎世举行[20]。
\subsubsection{晚年与去世}  
在康托尔1884年住院后,直到1899年没有记录显示他再次住进任何疗养院。[31] 在那次第二次住院后,康托尔的小儿子鲁道夫于12月16日突然去世(康托尔当时正在讲授关于培根理论与莎士比亚的演讲),这一悲剧使康托尔失去了大部分对数学的热情。[34] 康托尔在1903年再次住院。一年后,他对尤利乌斯·科尼希在第三届国际数学家大会上所发表的论文感到愤怒和激动。该论文试图证明超限集合理论的基本原则是错误的。由于论文是在他的女儿和同事面前宣读的,康托尔感到自己遭受了公开的羞辱。[35] 尽管恩斯特·泽梅洛在不到一天后证明科尼希的证明失败了,康托尔依然感到震惊,甚至短暂地对上帝产生了疑问。[13] 康托尔在余生中一直患有慢性抑郁症,因此他在几次教学中被豁免,并且多次被送往各个疗养院。1904年的事件导致了他之后每隔两三年就住院一次。[36] 然而,他并未完全放弃数学,在1903年,他向德国数学家协会会议讲授集合论悖论(布拉利-福尔蒂悖论、康托尔悖论和拉塞尔悖论),并于1904年参加了海德堡的国际数学家大会。

1911年,康托尔成为受邀参加苏格兰圣安德鲁斯大学成立500周年庆典的杰出外国学者之一。康托尔参加了此次庆典, hoping to meet Bertrand Russell, whose newly published *Principia Mathematica* repeatedly cited 康托尔's work, but the encounter did not come about. 第二年,圣安德鲁斯大学授予康托尔荣誉博士学位,但由于康托尔的身体状况,他未能亲自接受该学位。

康托尔于1913年退休,并在第一次世界大战期间生活贫困,营养不良。[37] 由于战争,他的70岁生日庆祝活动被取消。1917年6月,康托尔最后一次进入疗养院,并不断给妻子写信,要求允许他回家。乔治·康托尔于1918年1月6日在他度过生命最后一年的疗养院中因心脏病发作去世。[19]
\subsection{数学工作}  
康托尔在1874年到1884年之间的工作是集合论的起源。[38] 在这之前,集合的概念是一个相当初步的概念,从数学的起源开始就被隐含地使用,追溯到亚里士多德的思想。没有人意识到集合论有任何非平凡的内容。在康托尔之前,只有有限集合(这些集合容易理解)和“无限”概念(被视为哲学讨论的主题,而非数学讨论的主题)。通过证明无限集合有(无限多种)不同的大小,康托尔确立了集合论并非平凡,且需要进行研究。集合论已经成为现代数学的基础理论之一,意味着它通过一个统一的理论来解释关于数学对象(例如数和函数)的命题,涵盖所有传统的数学领域(如代数、分析学和拓扑学),并提供了一套标准的公理来证明或反驳这些命题。集合论的基本概念现在已广泛应用于整个数学领域。[39]

在他的一篇早期论文中,[40] 康托尔证明了实数集合比自然数集合“更多”——这是首次证明存在不同大小的无限集合。他也是第一个意识到一一对应(以下简称“1对1对应”)在集合论中重要性的人。他利用这个概念定义了有限集和无限集,并将后者细分为可列(或可数无限)集合和不可列集合(不可数无限集合)。[41]

康托尔在拓扑学中发展了重要的概念,并探讨了这些概念与基数的关系。例如,他证明了康托尔集(由亨利·约翰·斯蒂芬·史密斯于1875年发现)在任何地方都不稠密,但其基数与实数集合相同,而有理数则处处稠密,但却是可数的。他还证明了所有没有端点的可数稠密线性序列都与有理数是顺序同构的。

康托尔在集合论中引入了基础的构造方法,例如集合A的幂集,它是A的所有可能子集的集合。他后来证明了幂集A的大小严格大于A的大小,即使A是一个无限集合;这个结果很快被称为康托尔定理。康托尔发展了一个完整的无限集合的理论和算术,称为基数和序数,它扩展了自然数的算术。他为基数所用的符号是希伯来字母ℵ(ℵ,aleph),并附有自然数下标;而序数则使用希腊字母ω(ω,omega)。这种符号至今仍在使用。

由康托尔提出的连续统假设在1900年巴黎国际数学大会上被大卫·希尔伯特作为他二十三个未解问题中的第一个问题提出。康托尔的工作也受到了希尔伯特著名赞美之外的广泛关注。[15] 美国哲学家查尔斯·桑德斯·皮尔士赞扬了康托尔的集合论,并且在康托尔于1897年在苏黎世举行的第一次国际数学大会上发表公开讲座后,阿道夫·赫尔维茨和雅克·阿达马也都表达了他们的钦佩。在那次大会上,康托尔与德德金德重拾了友谊和通信。从1905年起,康托尔开始与他的英国崇拜者和翻译菲利普·乔丹就集合论的历史以及康托尔的宗教思想进行通信。这些信件后来被出版,还有一些他的阐述性作品也相继出版。
\subsubsection{数论、三角级数与序数}  
康托尔的前十篇论文涉及数论,这是他的博士论文主题。在哈雷大学教授爱德华·海涅的建议下,康托尔转向了分析学。海涅建议康托尔解决一个未解的问题,这个问题曾让彼得·古斯塔夫·勒让·狄利克雷、鲁道夫·利普希茨、伯恩哈德·黎曼和海涅本人都无法解决:通过三角级数表示函数的唯一性。康托尔于1869年解决了这个问题。正是在研究这个问题时,他发现了超有限序数,这些序数作为指数\(n\)出现在三角级数零点集\(S\)的第\(n\)阶导集\(S_n\)中。给定一个三角级数\(f(x)\),其零点集为\(S\),康托尔发现了一种方法可以得到另一个三角级数,其零点集为\(S_1\),其中\(S_1\)是\(S\)的极限点集。如果\(S_{k+1}\)是Sk的极限点集,那么他可以构造一个三角级数,其零点为\(S_{k+1}\)。因为这些集合\(S_k\)是封闭的,所以它们包含了自己的极限点,而无限递减序列集合\(S\)、\(S_1\)、\(S_2\)、\(S_3\),…的交集形成了一个极限集,我们现在称之为\(S_\omega\),随后他注意到\(S_\omega\)也必然有一个极限点集\(S_{\omega+1}\),依此类推。他得到了一些可以无限延续的例子,因此出现了一个自然发生的无限数列:\(\omega\), \(\omega+1\),\(\omega+2\), …[43]

在1870年至1872年间,康托尔发表了更多关于三角级数的论文,还发表了一篇定义无理数为收敛有理数序列的论文。康托尔在1872年与德德金德建立了友谊,德德金德后来在同年引用了康托尔的这篇论文,并在他的论文中首次提出了著名的实数定义——德德金分割。尽管通过其革命性的无限基数概念扩展了数的概念,康托尔却出人意料地反对他同时代的奥托·斯托尔茨和保罗·杜·布瓦-雷蒙的无穷小理论,称它们既是“可憎的”,又是“数学中的霍乱菌”。[44]康托尔还发表了一个错误的“证明”,声称无穷小的内在矛盾。[45]
\subsubsection{集合论}
\begin{figure}[ht]
\centering
\includegraphics[width=6cm]{./figures/65e841d34d6944de.png}
\caption{这是康托尔对不可数集合存在性进行对角线论证的示意图。[46] 底部的序列无法出现在上面无限序列的任何位置。} \label{fig_Canto_3}
\end{figure}
集合论作为数学分支的开端通常被标志为康托尔于1874年发表的论文《Ueber eine Eigenschaft des Inbegriffes aller reellen algebraischen Zahlen》(《关于所有实代数数集合的一个性质》)的发表。[38][47] 这篇论文首次提供了一个严格的证明,证明了存在不止一种类型的无穷大。在此之前,所有的无穷集合都被隐含地认为是等势的(即“大小相同”或包含相同数量的元素)。[48] 康托尔证明了实数集合和正整数集合不是等势的。换句话说,实数集合是不可数的。他的证明不同于他在1891年给出的对角线论证。[49] 康托尔的文章还包含了一种构造超越数的新方法。超越数最早是由Joseph Liouville于1844年构造的。[50]

康托尔通过两种构造法建立了这些结果。他的第一种构造法展示了如何将实代数数[51]写成一个序列\(a_1\), \(a_2\), \(a_3\), … 换句话说,实代数数是可数的。康托尔的第二种构造法从任何实数序列开始。利用这个序列,他构造了嵌套区间,这些区间的交集包含一个不在序列中的实数。由于每个实数序列都可以用来构造一个不在该序列中的实数,因此实数不能被写成一个序列——即,实数是不可数的。通过将他的构造法应用于实代数数序列,康托尔得到了一个超越数。康托尔指出,他的构造法证明了更多内容——即,它们为Liouville定理提供了一个新证明:每个区间都包含无穷多个超越数。[52] 康托尔的下一篇文章包含了一种构造法,证明了超越数集合的“势”(见下文)与实数集合相同。[53]

在1879年到1884年之间,康托尔在《Mathematische Annalen》期刊上发表了一系列六篇文章,这些文章共同构成了他集合论的介绍。同时,针对康托尔的观点,反对声音逐渐增多,主要由Leopold Kronecker领导。Kronecker只接受那些可以通过有限步骤从自然数构造出来的数学概念,而他认为自然数是直观上给定的。对于Kronecker来说,康托尔的无限阶层是不可接受的,因为接受实际无穷大的概念将打开通向悖论的大门,挑战数学整体的有效性。[54] 康托尔在这一时期还引入了康托尔集。

这一系列的第五篇论文\textbf{《Grundlagen einer allgemeinen Mannigfaltigkeitslehre》}(《一般集合理论的基础》),发表于1883年,[55] 是六篇论文中最重要的一篇,并且也作为独立的专著出版。该论文包含了康托尔对批评者的回应,并展示了超有限数是如何系统地扩展自然数的。论文首先定义了良序集合。然后引入了序数作为良序集合的顺序类型。接下来,康托尔定义了基数和序数的加法和乘法。1885年,康托尔扩展了他的顺序类型理论,使得序数成为顺序类型的一个特例。

1891年,他发表了一篇论文,提出了优雅的“对角线论证”,证明了不可数集合的存在。他应用相同的思想证明了康托尔定理:集合\(A\)的幂集的基数严格大于\(A\)的基数。这一结果确立了无限集合的层次结构,以及康托尔所定义的基数和序数的算术。这个论证在解决停机问题和证明Gödel第一不完全性定理中具有基础性作用。康托尔在1894年还曾写过关于哥德巴赫猜想的论文。
\begin{figure}[ht]
\centering
\includegraphics[width=6cm]{./figures/a0b50c3cf3b9f84d.png}
\caption{乔治·康托尔文章中的定义集合的段落} \label{fig_Canto_4}
\end{figure}
在1895年和1897年,康托尔在Felix Klein编辑的《Mathematische Annalen》上发表了两篇论文,这是他关于集合理论的最后几篇重要论文。[56] 第一篇论文从定义集合、子集等开始,这些定义在今天基本上仍然是可以接受的。论文回顾了基数和序数的算术。康托尔原本希望第二篇论文包括连续统假设的证明,但最终只能阐述他对良序集合和序数的理论。康托尔试图证明,如果集合\(A\)和\(B\)满足\(A\)与\(B\)的一个子集等价,\(B\)与\(A\)的一个子集等价,那么\(A\)和\(B\)是等价的。恩斯特·施罗德早些时候就提出了这个定理,但他的证明以及康托尔的证明都存在缺陷。费利克斯·伯恩斯坦在1898年的博士论文中提供了正确的证明,因此这个定理被称为康托尔-伯恩斯坦-施罗德定理。

\textbf{一一对应}

\begin{figure}[ht]
\centering
\includegraphics[width=6cm]{./figures/83a671629215d8a8.png}
\caption{双射函数} \label{fig_Canto_5}
\end{figure}
康托尔的1874年《Crelle》论文首次提出了“一一对应”这一概念,尽管他并没有使用这个词语。随后,他开始寻找单位正方形的点与单位线段的点之间的一一对应关系。在1877年写给理查德·德德金的一封信中,康托尔证明了一个更强的结果:对于任何正整数\(n\),存在单位线段与n维空间中所有点之间的一一对应关系。关于这个发现,康托尔写信给德德金说:“Je le vois, mais je ne le crois pas!”(“我看到了,但我不相信!”)这个结果令他感到非常惊讶,它对几何学和维度的概念产生了深远的影响。

在1878年,康托尔向《Crelle's Journal》提交了另一篇论文,他在其中精确定义了一一对应的概念,并引入了“幂”(他从Jakob Steiner那里借用的术语)或“等价”集合的概念:如果存在一一对应关系,两个集合是等价的(有相同的幂)。康托尔将可数集合(或可列集合)定义为能够与自然数建立一一对应关系的集合,并证明有理数是可列的。他还证明了n维欧几里得空间\(R^n\)与实数\textbf{R}具有相同的幂,任何可列无限的\textbf{R}的副本的乘积也具有相同的幂。尽管他在概念上频繁使用了可数性,但直到1883年才正式写出“可数”一词。康托尔还讨论了自己对维度的思考,强调他在单位区间和单位正方形之间的映射并不是连续的。

这篇论文让克罗内克尔不满,康托尔曾想撤回它;然而,德德金劝说他不要这样做,卡尔·魏尔斯特拉斯也支持它的发表。尽管如此,康托尔从此再也没有向《Crelle》提交任何论文

\textbf{连续统假设}

康托尔是第一个提出后来被称为连续统假设(CH)的人:不存在一个集合,它的势比自然数的势大,但比实数的势小(或者等价地,实数的基数恰好是阿列夫一,而不是至少是阿列夫一)。康托尔认为连续统假设为真,并且尝试了很多年去证明它,但都未成功。他未能证明连续统假设给他带来了相当大的焦虑。[11]

康托尔在证明连续统假设方面的困难,已经被数学领域的后续发展所强调:库尔特·哥德尔在1940年的一个结果和保罗·科恩在1963年的一个结果共同表明,连续统假设既不能通过标准的泽尔梅洛–弗兰克尔集合论加选择公理(合称为“ZFC”)证明,也不能否定。[59]

\textbf{绝对无限、良序定理与悖论}

1883年,康托尔将无限分为超限和绝对无限两类。[60]

超限是可以在大小上增加的,而绝对无限则是无法增加的。例如,一个序数\(\alpha\)是超限的,因为它可以增加到\(\alpha+1\)。另一方面,序数形成了一个绝对无限的序列,因为没有更大的序数可以添加到其中,从而无法在大小上增加。[61] 1883年,康托尔还提出了良序原则“每个集合都可以良序”,并声明它是“一条思维法则”。[62]

康托尔通过在证明中使用绝对无限,扩展了他关于绝对无限的工作。大约在1895年,他开始将自己的良序原则视为一个定理,并试图证明它。1899年,他向德德金德(Dedekind)发送了关于等价阿列夫定理的证明:每个无限集合的基数都是阿列夫。[63] 首先,他定义了两种类型的多重性:一致的多重性(集合)和不一致的多重性(绝对无限的多重性)。接着,他假设序数形成一个集合,证明这会导致矛盾,最终得出结论,序数形成一个不一致的多重性。他利用这一不一致的多重性来证明阿列夫定理。[64] 1932年,泽梅洛(Zermelo)批评了康托尔证明中的构造。[65]

康托尔通过认识到存在两种类型的多重性,避免了悖论。在他的集合论中,当假设序数形成一个集合时,结果的矛盾仅仅意味着序数形成了一个不一致的多重性。与此相对,伯特兰·罗素将所有集合都视为集合,这导致了悖论。在罗素的集合论中,序数形成了一个集合,因此由此产生的矛盾意味着理论是不一致的。从1901年到1903年,罗素发现了三个悖论,暗示他的集合论是不一致的:布拉利-福尔蒂悖论(即前文提到的)、康托尔悖论和罗素悖论。[66] 尽管布拉利和康托尔都不认为他们发现了悖论,但罗素还是以他们的名字命名了这些悖论。[67]

1908年,泽梅洛发布了他的集合论公理体系。他发展这个公理体系有两个动机:消除悖论并确保他对良序定理的证明。[68] 泽梅洛在1904年已经使用选择公理证明了这个定理,但他的证明因多种原因受到批评。[69] 他对批评的回应包括他的公理体系和对良序定理的新证明。他的公理支持这个新证明,并通过限制集合的构造消除了悖论。[70]

1923年,约翰·冯·诺依曼发展了一个公理体系,通过使用类似于康托尔的方法——即识别不是集合的集合,并对其进行不同处理,从而消除了悖论。冯·诺依曼指出,如果一个类可以与所有集合的类建立一一对应关系,那么它太大,不能是一个集合。他将集合定义为某个类的成员,并提出了这个公理:如果且仅如果存在一个一一对应关系将一个类与所有集合的类相联系,那么该类不是集合。这个公理意味着这些“大类”不是集合,因为它们不能成为任何类的成员,这消除了悖论。[71] 冯·诺依曼还利用他的公理证明了良序定理:像康托尔一样,他假设序数形成一个集合。结果的矛盾意味着所有序数的类不是集合。然后,他的公理为这个类与所有集合的类之间提供了一一对应关系。这个对应关系良序化了所有集合的类,这就意味着了良序定理。[72] 1930年,泽梅洛定义了满足冯·诺依曼公理的集合论模型。[73]
\subsection{哲学、宗教、文学与康托尔的数学}
实际无穷的存在概念是数学、哲学和宗教领域中一个重要的共同关注点。尽管这种关系的正统性并不像康托尔的批评者所持的观点那样形成,但保持上帝与数学之间关系的正统性长期以来一直是康托尔关心的问题。[74] 他在《一般集合论基础》的导言中直接讨论了这些学科之间的交集,在其中他强调了他对无穷的观点与哲学观点之间的联系。[75] 对康托尔来说,他的数学观点与其哲学和神学的意义密切相关——他将绝对无穷与上帝等同,[76] 并认为他对超限数的研究是上帝直接传达给他的,认为上帝选择了康托尔来向世界揭示这些概念。[5] 他是一个虔诚的路德宗教徒,明确的基督教信仰塑造了他的科学哲学。[77] 约瑟夫·道本追溯了康托尔的基督教信仰对超限集合论发展的影响。[78][79]

数学家们的辩论源于数学哲学中关于实际无穷的本质的对立观点。一些人坚持认为无穷是一个抽象概念,在数学上不具备合法性,并否认其存在。[80] 来自三个主要思潮的数学家(构造主义及其两个分支:直觉主义和有限主义)在这一问题上反对康托尔的理论。对于像克罗内克尔这样的构造主义者来说,拒绝实际无穷源于他们与非构造性证明(如康托尔的对角线论证)是否足以证明某物存在的根本分歧,而是坚持认为需要构造性证明。直觉主义也否定了实际无穷是任何现实的表达这一观点,但他们通过与构造主义不同的路径得出这一结论。首先,康托尔的论证依赖于逻辑来证明超限数作为一种实际数学实体的存在,而直觉主义者认为数学实体不能简化为逻辑命题,而是源自人类心灵的直觉。[81] 其次,直觉主义本身就不允许将无穷视为现实的表现,因为人类的心灵无法直观地构造一个无限集合。[82] 像L. E. J. 布劳威尔和特别是亨利·庞加莱这样的数学家采纳了直觉主义的立场,反对康托尔的工作。最后,维特根斯坦的攻击是有限主义的:他认为康托尔的对角线论证混淆了基数或实数集合的内涵与其外延,从而将生成集合的规则与实际集合的概念混淆了。[10]

一些基督教神学家认为康托尔的工作对上帝绝对无穷的独特性构成挑战。[6] 特别是新托马斯主义的思想家们认为,存在一个由非上帝构成的实际无穷,会危及“上帝对至高无穷的独占权”。[83] 康托尔坚信,这种观点是对无穷的误解,他深信集合论可以帮助纠正这个错误:[84] “……超限种类就像有限数一样,完全属于造物主的意图和他绝对无边的意志。”[85] 知名的新斯多亚学派德国哲学家康斯坦丁·古特贝尔特支持这种理论,认为它并不与上帝的本质相对立。[8]

康托尔还认为,他的超限数理论与唯物主义和决定论相悖——当他意识到自己是哈雷唯一不持有决定论哲学信仰的教员时,他感到震惊。[86]

康托尔认为,他的哲学提供了对自然的“有机解释”,在他1883年的《集合论基础》中,他表示这样的解释只能通过借鉴斯宾诺莎和莱布尼茨的哲学资源来实现。[87] 在提出这些观点时,康托尔可能受到了F·A·特伦德伦堡的影响,他曾在柏林听过特伦德伦堡的讲座,而康托尔后来还写了一本关于斯宾诺莎《伦理学》第一书的拉丁文评论。特伦德伦堡也是康托尔的《资格论文》考官。[88][89]

1888年,康托尔发表了他与几位哲学家关于其集合论哲学意义的通信。在广泛试图说服其他基督教思想家和权威接受他观点的过程中,康托尔与诸如蒂尔曼·佩施和约瑟夫·洪特海姆等基督教哲学家,以及神学家如约翰·巴普蒂斯特·弗朗茨林枢机等人进行了通信。[90] 弗朗茨林曾回答说,将超限数理论与泛神论等同起来。[7] 尽管后来这位枢机接受了该理论的有效性,这要归功于康托尔的一些澄清。[8] 康托尔甚至亲自给教皇利奥十三世写了一封信,并向他寄送了几本小册子。[84]

康托尔关于数字本质的哲学观点使他坚信,数学有自由提出和证明概念的权利,超越物理现象的领域,作为内部现实中的表现形式。对这个形而上学系统的唯一限制是:所有数学概念必须没有内在的矛盾,并且它们必须从现有的定义、公理和定理中推导出来。这个信念可以总结为他所说的“数学的本质是它的自由。”[91] 这些观点与爱德蒙·胡塞尔的观点相似,康托尔曾在哈雷遇见过胡塞尔。[92]

与此同时,康托尔自己坚决反对无穷小,称其为“可憎的”和“数学中的霍乱分枝杆菌”。[44]

康托尔在1883年的论文中表明,他非常清楚自己的观点所遇到的反对:“……我意识到,在这个工作中,我把自己置于与广泛持有的关于数学无限的观点及常常捍卫的数字本质的意见的对立面。”[93]

因此,他花了大量篇幅来为自己早期的工作辩护,主张只要数学概念不包含矛盾,并且能够用以前接受的概念来定义,就可以自由地引入这些概念。他还引用了亚里士多德、勒内·笛卡尔、乔治·伯克利、戈特弗里德·莱布尼茨和伯纳德·博尔扎诺关于无限的论述。相反,他始终强烈拒绝伊曼努尔·康德的哲学,无论是在数学哲学领域还是形而上学领域。他认同伯特兰·罗素的座右铭“康德或康托尔”,并称康德为“那个远离数学的诡辩的庸人”。[94]
\subsection{康托尔的家族背景}
\begin{figure}[ht]
\centering
\includegraphics[width=6cm]{./figures/056d4cf215af6872.png}
\caption{纪念 plaque 上的标题(俄文):“伟大的数学家和集合论创始人乔治·康托尔于1845年至1854年间出生并生活在这座建筑内”,位于圣彼得堡的瓦西里岛。} \label{fig_Canto_6}
\end{figure}
康托尔的父方祖父母来自哥本哈根,他们在拿破仑战争的动荡中逃亡到俄罗斯。关于他们的直接信息非常少。[95] 康托尔的父亲,乔治·瓦尔德马尔·康托尔,在圣彼得堡的路德宗传教士学校接受教育,他与儿子的通信表明他们都是虔诚的路德宗信徒。关于乔治·瓦尔德马尔的出生地和教育背景,几乎没有确凿的资料。[96] 康托尔的母亲,玛丽亚·安娜·贝姆,是一位出生在圣彼得堡的奥匈帝国人,受洗为罗马天主教徒;她在结婚后改宗为新教。然而,康托尔的哥哥路易斯写给母亲的一封信中有这样的内容:

“即使我们十次都是犹太人的后裔,我原则上仍然支持希伯来人平等的权利,但在社会生活中,我更喜欢基督徒……”[96]

(“即使我们是从十代犹太人传下来的,尽管原则上我完全支持犹太人的平等权利,但在社会生活中,我更喜欢与基督徒交往……”)这句话可能暗示她有犹太血统。[97]

根据传记作家埃里克·坦普尔·贝尔(Eric Temple Bell)的说法,康托尔有犹太血统,尽管他的父母都已受洗。[98] 在1971年的一篇名为《走向乔治·康托尔传记》的文章中,英国数学史学家艾弗·格雷顿-吉尼斯(Ivor Grattan-Guinness)提到(《科学年鉴》27卷,345-391页,1971年),他未能找到康托尔有犹太血统的证据。(他还提到康托尔的妻子瓦利·古特曼(Vally Guttmann)是犹太人)。

在1896年写给保罗·坦纳里的信中(保罗·坦纳里,《科学回忆录》第13篇,1934年巴黎高蒂尔-维拉尔出版,306页),康托尔指出,他的父亲是来自哥本哈根塞法迪犹太社区的人。具体来说,康托尔在描述他的父亲时写道:“Er ist aber in Kopenhagen geboren, von israelitischen Eltern, die der dortigen portugisischen Judengemeinde....” (“他出生在哥本哈根,父母是来自当地葡萄牙犹太社区的犹太人(直译:‘以色列人’)。”)[99] 此外,康托尔的外曾叔叔约瑟夫·博姆(Josef Böhm),一位匈牙利小提琴家,被描述为犹太人,[100] 这可能意味着康托尔的母亲至少部分源自匈牙利犹太社区。[102]

在写给伯特兰·罗素的一封信中,康托尔描述了他的家族背景和自我认知:

“我的父母都不是德国血统,父亲是丹麦人,出生在哥本哈根;母亲则是奥匈帝国血统。先生,您必须知道,我并不完全是德国人,我生于1845年3月3日,在俄罗斯的圣彼得堡,但我在1856年与父母、兄弟和姐妹一起,年仅11岁时,搬到德国。”[103]

在1930年代,有文件性声明质疑康托尔是否具有犹太血统:

“更常讨论的问题是乔治·康托尔是否有犹太血统。1937年,丹麦哥本哈根的家谱研究所曾发布一则声明,涉及他的父亲:‘特此证明,乔治·沃尔德马尔·康托尔,生于1809年或1814年,并未出现在犹太社区的登记册上,毫无疑问他不是犹太人...’”[96]
\subsection{传记}  
直到1970年代,关于康托尔的主要学术出版物是两部简短的专著,分别由阿图尔·莫里茨·舍恩弗利(1927年)编写——主要是与米塔格-莱弗勒的通信——和弗雷恩克尔(1930年)。两者都是间接资料,且都没有涉及康托尔的私人生活。这个空白在很大程度上被埃里克·坦普尔·贝尔的《数学人物》(1937年)填补,一位现代康托尔的传记作者形容它是“可能是最广泛阅读的现代数学历史书籍”;也被称为“最糟糕的书之一”。[104] 贝尔将康托尔与父亲的关系描述为俄狄浦斯式的,将康托尔与克罗内克的分歧描述为两位犹太人之间的争执,并将康托尔的疯狂归因于他未能使自己的数学获得接受的浪漫式绝望。格拉特恩-金尼斯(1971年)发现这些说法都不属实,但它们出现在许多随后的书籍中,因为缺乏其他叙述。在贝尔的影响之外,还有一些其他的传说——包括将康托尔的父亲标记为弃婴,被不知名的父母送到圣彼得堡的说法。[105] 乔瑟夫·道本的传记中批评了贝尔的书。[106] 道本写道:

“康托尔把一些最激烈的信件以及《贡献》中的一部分,专门用于攻击他曾一度形容为‘数学的无穷小霍乱分枝菌’的东西,这种思想从德国通过托马、杜·布瓦-雷蒙德和斯托尔茨的工作传入,进而感染了意大利数学……任何接受无穷小的思想,必然意味着他自己关于数的理论是不完整的。因此,接受托马、杜·布瓦-雷蒙德、斯托尔茨和维罗内塞的工作,就等于否认康托尔自己创造的数学的完美性。可以理解,康托尔发起了一场彻底的运动,竭尽全力贬低维罗内塞的工作。”[107]
\subsection{另见}
\begin{itemize}
\item 绝对无限  
\item 阿列夫数  
\item 连续统的基数  
\item 康托尔奖——由德国数学家协会(Deutsche Mathematiker-Vereinigung)为纪\item 念乔治·康托尔设立的奖项  
\item 基数  
\item 连续统假设  
\item 可数集  
\item 导出集(数学)  
\item ε数(数学)  
\item 阶乘数系统  
\item 配对函数  
\item 超越数  
\item 以乔治·康托尔命名的事物列表
\end{itemize}
\subsection{注释}
\begin{enumerate}
\item 1. Grattan-Guinness 2000, 第351页。
2. 本文中的传记材料主要来自Dauben 1979。Grattan-Guinness 1971,以及Purkert和Ilgauds 1985是有用的补充来源。
3. Dauben 2004, 第1页。
4. Dauben, Joseph Warren (1979). 《Georg Cantor:他的数学与无限的哲学》. 普林斯顿大学出版社, 第引言部分。ISBN 9780691024479。
5. Dauben 2004, 第8, 11, 12–13页。
6. Dauben 1977, 第86页;Dauben 1979, 第120, 143页。
7. Dauben 1977, 第102页。
8. Dauben 1979, 第6章。
9. Dauben 2004, 第1页;Dauben 1977, 第89页,第15n。
10. Rodych 2007。
11. Dauben 1979, 第280页:“……由Arthur Moritz Schönflies所流行的传统将Cantor反复发作的抑郁症归咎于Kronecker的持续批评以及Cantor无法确认其连续统假设。”
12. Dauben 2004, 第1页。文本中包括来自精神病学家Karl Pollitt的1964年引用,他是Cantor在哈雷神经病院的主治医生之一,提到Cantor的精神疾病为“周期性躁郁症”。

\end{enumerate}
\subsection{参考文献}  
\begin{itemize}
\item Dauben, Joseph W. (1977). "Georg Cantor and Pope Leo XIII: Mathematics, Theology, and the Infinite". *Journal of the History of Ideas*. 38 (1): 85–108. doi:10.2307/2708842. JSTOR 2708842.  
\item Dauben, Joseph W. (1979). [Unavailable on archive.org] *Georg Cantor: his mathematics and philosophy of the infinite*. Boston: Harvard University Press. ISBN 978-0-691-02447-9.  
\item Dauben, Joseph (2004) [1993]. *Georg Cantor and the Battle for Transfinite Set Theory* (PDF). *Proceedings of the 9th ACMS Conference* (Westmont College, Santa Barbara, Calif.). pp. 1–22. Archived (PDF) from the original on 23 January 2018. Internet version published in *Journal of the ACMS 2004*. Note, though, that Cantor's Latin quotation described in this article as a familiar passage from the Bible is actually from the works of Seneca and has no implication of divine revelation.  
\item Ewald, William B., ed. (1996). *From Immanuel Kant to David Hilbert: A Source Book in the Foundations of Mathematics*. New York: Oxford University Press. ISBN 978-0-19-853271-2.  
Grattan-Guinness, Ivor (1971). "Towards a Biography of Georg \item Cantor". *Annals of Science*. 27 (4): 345–391. doi:10.1080/00033797100203837.  
\item Grattan-Guinness, Ivor (2000). *The Search for Mathematical Roots: 1870–1940*. Princeton University Press. ISBN 978-0-691-05858-0.  
\item Hallett, Michael (1986). *Cantorian Set Theory and Limitation of Size*. New York: Oxford University Press. ISBN 978-0-19-853283-5.  
\item Moore, Gregory H. (1982). *Zermelo's Axiom of Choice: Its Origins, Development & Influence*. Springer. ISBN 978-1-4613-9480-8.  
\item Moore, Gregory H. (1988). "The Roots of Russell's Paradox". *Russell: The Journal of Bertrand Russell Studies*. 8: 46–56. doi:10.15173/russell.v8i1.1732 (inactive 5 November 2024).  
\item Moore, Gregory H.; Garciadiego, Alejandro (1981). "Burali-Forti's Paradox: A Reappraisal of Its Origins". *Historia Mathematica*. 8 (3): 319–350. doi:10.1016/0315-0860(81)90070-7.  
\item Purkert, Walter (1989). "Cantor's Views on the Foundations of Mathematics". In Rowe, David E.; McCleary, John (eds.). *The History of Modern Mathematics, Volume 1*. Academic Press. pp. 49–65. ISBN 978-0-12-599662-4.  
\item Purkert, Walter; Ilgauds, Hans Joachim (1985). *Georg Cantor: 1845–1918*. Birkhäuser. ISBN 978-0-8176-1770-7.  
\item Suppes, Patrick (1972) [1960]. *Axiomatic Set Theory*. New York: Dover. ISBN 978-0-486-61630-8. Although the presentation is axiomatic rather than naive, Suppes proves and discusses many of Cantor's results, which demonstrates Cantor's continued importance for the edifice of foundational mathematics.  
\item Zermelo, Ernst (1908). "Untersuchungen über die Grundlagen der Mengenlehre I". *Mathematische Annalen*. 65 (2): 261–281. doi:10.1007/bf01449999. S2CID 120085563.  
\item Zermelo, Ernst (1930). "Über Grenzzahlen und Mengenbereiche: neue Untersuchungen über die Grundlagen der Mengenlehre" (PDF). *Fundamenta Mathematicae*. 16: 29–47. doi:10.4064/fm-16-1-29-47. Archived (PDF) from the original on 28 June 2004.  
\item van Heijenoort, Jean (1967). *From Frege to Godel: A Source Book in Mathematical Logic, 1879–1931*. Harvard University Press. ISBN 978-0-674-32449-7. 
\end{itemize} 
\subsection{参考书目}  
关于Cantor生平的早期资料应谨慎对待。请参阅上述“传记”部分。  
\subsubsection{英文原始文献}  
\begin{itemize}
\item Cantor, Georg (1955) [1915]. Philip Jourdain (ed.). *Contributions to the Founding of the Theory of Transfinite Numbers*. New York: Dover Publications. ISBN 978-0-486-60045-1.  
\end{itemize}
\subsubsection{德文原始文献 } 
\begin{itemize}
\item Cantor, Georg (1874). "Ueber eine Eigenschaft des Inbegriffes aller reellen algebraischen Zahlen" (PDF). *Journal für die Reine und Angewandte Mathematik*. 1874 (77): 258–262. doi:10.1515/crll.1874.77.258. S2CID 199545885. Archived (PDF) from the original on 7 October 2017.  
\item Cantor, Georg (1878). "Ein Beitrag zur Mannigfaltigkeitslehre". *Journal für die Reine und Angewandte Mathematik*. 1878 (84): 242–258. doi:10.1515/crelle-1878-18788413.  
\item Georg Cantor (1879). "Ueber unendliche, lineare Punktmannichfaltigkeiten (1)". *Mathematische Annalen*. 15 (1): 1–7. doi:10.1007/bf01444101. S2CID 179177510.  
\item Georg Cantor (1880). "Ueber unendliche, lineare Punktmannichfaltigkeiten (2)". *Mathematische Annalen*. 17 (3): 355–358. doi:10.1007/bf01446232. S2CID 179177438.  
\item Georg Cantor (1882). "Ueber unendliche, lineare Punktmannichfaltigkeiten (3)". *Mathematische Annalen*. 20 (1): 113–121. doi:10.1007/bf01443330. S2CID 177809016.  
\item Georg Cantor (1883). "Ueber unendliche, lineare Punktmannichfaltigkeiten (4)". *Mathematische Annalen*. 21 (1): 51–58. doi:10.1007/bf01442612. S2CID 179177480.  
\item Georg Cantor (1883). "Ueber unendliche, lineare Punktmannichfaltigkeiten (5)". *Mathematische Annalen*. 21 (4): 545–591. doi:10.1007/bf01446819. S2CID 121930608. Published separately as: *Grundlagen einer allgemeinen Mannigfaltigkeitslehre*.  
\item Georg Cantor (1884). "Ueber unendliche, lineare Punktmannichfaltigkeiten (6)". *Mathematische Annalen*. 23 (4): 453–488. doi:10.1007/BF01446598. S2CID 179178052.  
\item Georg Cantor (1891). "Ueber eine elementare Frage der Mannigfaltigkeitslehre" (PDF). *Jahresbericht der Deutschen Mathematiker-Vereinigung*. 1: 75–78. Archived (PDF) from the original on 1 January 2018.  
\item Cantor, Georg (1895). "Beiträge zur Begründung der transfiniten Mengenlehre (1)". *Mathematische Annalen*. 46 (4): 481–512. doi:10.1007/bf02124929. S2CID 177801164. Archived from the original on 23 April 2014.  
\item Cantor, Georg (1897). "Beiträge zur Begründung der transfiniten Mengenlehre (2)". *Mathematische Annalen*. 49 (2): 207–246. doi:10.1007/bf01444205. S2CID 121665994.  
\item Cantor, Georg (1932). Ernst Zermelo (ed.). "Gesammelte Abhandlungen mathematischen und philosophischen inhalts". Berlin: Springer. Archived from the original on 3 February 2014. Almost everything that Cantor wrote. Includes excerpts of his correspondence with Dedekind (p. 443–451) and Fraenkel's Cantor biography (p. 452–483) in the appendix.  
\end{itemize}
\subsubsection{次要文献}  
\begin{itemize}
\item Aczel, Amir D. (2000). *The Mystery of the Aleph: Mathematics, the Kabbala, and the Search for Infinity*. New York: Four Walls Eight Windows Publishing. ISBN 0-7607-7778-0. 这是一本关于无穷大的流行书籍,书中多次提到Cantor。  
\item Dauben, Joseph W. (June 1983). "Georg Cantor and the Origins of Transfinite Set Theory". *Scientific American*. 248 (6): 122–131. Bibcode:1983SciAm.248f.122D. doi:10.1038/scientificamerican0683-122.  
\item Ferreirós, José (2007). *Labyrinth of Thought: A History of Set Theory and Its Role in Mathematical Thought*. Basel, Switzerland: Birkhäuser. ISBN 3-7643-8349-6. 本书详细讨论了Cantor和Dedekind在集合论中的贡献。  
\item Halmos, Paul (1998) [1960]. *Naive Set Theory*. New York & Berlin: Springer. ISBN 3-540-90092-6.  
\item Hilbert, David (1926). "Über das Unendliche". *Mathematische Annalen*. 95: 161–190. doi:10.1007/BF01206605. S2CID 121888793.  
\item Hill, C. O.; Rosado Haddock, G. E. (2000). *Husserl or Frege? Meaning, Objectivity, and Mathematics*. Chicago: Open Court. ISBN 0-8126-9538-0. 本书有三章和18个关于Cantor的索引条目。  
\item Meschkowski, Herbert (1983). *Georg Cantor, Leben, Werk und Wirkung* (Georg Cantor, Life, Work and Influence, 德文版). Vieweg, Braunschweig.  
\item Newstead, Anne (2009). "Cantor on Infinity in Nature, Number, and the Divine Mind", *American Catholic Philosophical Quarterly*, 83 (4): 532–553, [链接](https://doi.org/10.5840/acpq200983444). 本文承认Dauben的开创性历史研究,并进一步深入讨论Cantor与斯宾诺莎和莱布尼茨哲学的关系,以及他在“泛神论争议”中的参与。简要提到Cantor从F.A. Trendelenburg那里学到的东西。  
\item Penrose, Roger (2004). *The Road to Reality*. Alfred A. Knopf. ISBN 0-679-77631-1. 第16章阐述了Cantor思想如何吸引一位当代著名的理论物理学家。  
\item Rucker, Rudy (2005) [1982]. *Infinity and the Mind*. Princeton University Press. ISBN 0-553-25531-2. 处理与Aczel相似的主题,但深入探讨了这些主题。  
\item Rodych, Victor (2007). "Wittgenstein's Philosophy of Mathematics". 在Edward N. Zalta(编). *The Stanford Encyclopedia of Philosophy*. Metaphysics Research Lab, Stanford University.  
\item Leonida Lazzari, *L'infinito di Cantor*. Editrice Pitagora, Bologna, 2008.
\end{itemize}
\subsection{外部链接}
\begin{itemize}
\item 与Georg Cantor相关的引用,见Wikiquote
\item 与Georg Cantor相关的媒体,见Wikimedia Commons
\item 关于Georg Cantor的作品,见互联网档案馆
\item O'Connor, John J.; Robertson, Edmund F., "Georg Cantor",MacTutor数学史档案,圣安德鲁斯大学
\item O'Connor, John J.; Robertson, Edmund F., "集合论的历史",MacTutor数学史档案,圣安德鲁斯大学,主要讨论Cantor的成就。
\item Georg Cantor,Britannica网站
\item 斯坦福哲学百科全书:集合论,作者Thomas Jech。集合论的早期发展,作者José Ferreirós。
\item “Cantor的无穷大”,分析Cantor的1874年文章,BibNum(点击“à télécharger”查看英文版)。该分析存在错误。它正确陈述了Cantor的定理1:代数数是可以计数的。然而,它错误地陈述了定理2:实数无法计数。然后,它说:“Cantor注意到,定理1和定理2结合起来可以重新证明非代数实数的存在……”这种存在证明是非构造性的。定理2正确表述应为:给定一个实数序列,可以确定一个不在该序列中的实数。结合定理1和该定理2,可以得到一个非代数数。Cantor还使用定理2证明实数无法被计数。详情请见Cantor的第一篇集合论文章或《Georg Cantor与超越数》,已于2022年1月21日归档在Wayback Machine。
\end{itemize}