% latex

设$$A$$是一个$$m \times n$$的实数矩阵,则$$A$$可以进行奇异值分解为:

$$A = U \Sigma V^T$$

其中$$U$$是一个$$m \times m$$的正交矩阵,其列向量是$$A A^T$$的特征向量;$$V$$是一个$$n \times n$$的正交矩阵,其列向量是$$A^T A$$的特征向量;$$\Sigma$$是一个$$m \times n$$的对角矩阵,其对角线上的元素就是奇异值,满足:

$$\sigma_1 \geq \sigma_2 \geq \cdots \geq \sigma_r > 0, \quad r = rank(A)$$

则有:

•  $$A A^T = U \Sigma V^T V \Sigma^T U^T = U \Sigma \Sigma^T U^T$$

•  $$A^T A = V \Sigma^T U^T U \Sigma V^T = V \Sigma^T \Sigma V^T$$

由于正交矩阵满足$$U^T U = I_m, V^T V = I_n$$,所以可以得到:

•  $$A A^T$$和$$U$$有相同的特征向量,其特征值为奇异值的平方:$$\lambda_i = \sigma_i^2, i = 1, 2, ..., r$$

•  $$A^T A$$和$$V$$有相同的特征向量,其特征值也为奇异值的平方:$$\lambda_i = \sigma_i^2, i = 1, 2, ..., r$$

因此,奇异值就是原矩阵或其转置与自身乘积后得到的方阵(称为格拉姆矩阵)的特征值的平方根。