% 指数函数(综述)
% license CCBYSA3
% type Wiki

本文根据 CC-BY-SA 协议转载翻译自维基百科\href{https://en.wikipedia.org/wiki/Exponential_function}{相关文章}。

在数学中,\textbf{指数函数}是唯一一个将零映射为一且其导数在所有点上都等于自身的实函数。变量 $x$ 的指数函数记作 $\exp x$ 或 $e^x$,这两种记法可以互换使用。之所以称其为“指数”,是因为其自变量可以看作是某个常数 $e \approx 2.718$(即底数)的幂指数。指数函数有多种定义方式,尽管它们的形式迥异,但在数学意义上是等价的。

指数函数可以将加法转化为乘法:它将加法的单位元 0 映射为乘法的单位元 1,并且满足加法转乘法的性质,即$\exp(x + y) = \exp x \cdot \exp y$。它的反函数是自然对数函数,记作 $\ln$ 或 $\log$,它则将乘法转化为加法:$\ln(x \cdot y) = \ln x + \ln y$。

指数函数有时被称为自然指数函数,以对应自然对数的名称,用以区别于其他也常被称作“指数函数”的一些函数。这些函数包括形如$f(x) = b^x$的函数,即以固定底数 $b$ 的幂函数。更一般地,尤其在实际应用中,形如$f(x) = a b^x$的函数也被称为指数函数。这些函数被称为“指数增长”或“指数衰减”,是因为当 $x$ 增加时,函数 $f(x)$ 的变化速率与它当前的取值成正比。

指数函数可以推广到接受复数作为自变量。这一推广揭示了复数乘法、复平面中的旋转以及三角函数之间的内在联系。欧拉公式$\exp(i\theta) = \cos\theta + i\sin\theta$正是这些关系的集中表达和总结。

指数函数甚至还可以进一步推广到其他类型的自变量,比如矩阵以及李代数中的元素。
\subsection{图像}
函数 $y = e^x$ 的图像是一个向上增长的曲线,并且其增长速度快于任何幂函数 $x^n$ 的增长速度。\(^\text{[1]}\) 该图像始终位于 $x$ 轴之上,但在 $x$ 取非常大的负值时会无限接近于 $x$ 轴,因此 $x$ 轴是其一条水平渐近线。方程$\frac{d}{dx}e^x = e^x$意味着该图像上任一点的切线斜率,等于该点的函数值(即该点的 $y$ 坐标)。
\subsection{定义与基本性质}
指数函数有多种形式迥异但在数学上等价的定义方式。
\subsubsection{微分方程}
\begin{figure}[ht]
\centering
\includegraphics[width=6cm]{./figures/ad33dd2da36528ba.png}
\caption{指数函数的导数等于其函数值。由于导数表示切线的斜率,这意味着图中的所有绿色直角三角形的底边长度为 1。} \label{fig_ZShs_1}
\end{figure}
最简单的定义之一是:指数函数是唯一一个可导函数,其导数等于自身,并且在变量为 0 时取值为 1。

这个“概念性”定义需要证明存在性和唯一性,但它使得指数函数的主要性质可以很容易地推导出来。

唯一性:若 $f(x)$ 和 $g(x)$ 是两个满足上述定义的函数,那么根据商法则,函数 $f/g$ 的导数在所有点上都为零。这意味着 $f/g$ 是一个常数函数;又因为 $f(0) = g(0) = 1$,所以这个常数为 1,即 $f(x) = g(x)$。

存在性将在接下来的两个小节中分别加以证明。
