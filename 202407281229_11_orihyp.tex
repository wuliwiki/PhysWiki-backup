% 超平面的定向
% license Usr
% type Tutor


\begin{issues}
\issueTODO
\end{issues}
\addTODO{需新增条目:流形上的定向}
假设现在我们有一$n$维光滑流形$M$,超平面则是其余维数为1的子流形。为了得到超平面上的定向,我们首先需要设计一种能把$n$形式映射到超平面上$n-1$形式的方法。缩并映射就是符合我们需求的线性映射。
\subsection{缩并}
\begin{definition}{}
设$V$为有限维线性空间,$X\in V$,定义线性映射$i_X:\Gamma^n V\rightarrow\Gamma^{n-1}V$,使得
\begin{equation}
i_X\omega(Y_1,\ldots,Y_{k-1})=\omega(X,Y_1,\ldots,Y_{k-1})~,
\end{equation}
则称$i_X$为内乘或缩并映射(interior multiplication or contraction),有时为了表示简洁,也用$X\lrcorner$表示。
\end{definition}
其线性是显而易见的,我们还可以证明缩并映射满足以下性质:
\begin{lemma}{}
$x\in V$,$V$为有限维线性空间,则
\begin{enumerate}
\item $i_X\circ i_X=0$.
\item $i_X$有与外代数类似的反对称性质。若$\omega$是$k$阶余切向量,$\eta$是$l$阶余切向量,则
\begin{equation}
i_X(\omega\wedge\eta)=(i_X\omega)\wedge\eta+(-1)^k\omega\wedge(i_X\eta)~.
\end{equation}
\end{enumerate}
\end{lemma}
\textbf{证明:}

第一条利用交错张量的性质即可得。


