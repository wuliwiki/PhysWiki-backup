% 黎曼度量与伪黎曼度量
% 度量|metric|流形|微分几何|相对论|闵可夫斯基时空|relativity|minkowski|riemann|riemannian metric|洛伦兹度量|lorentz
\pentry{内积\upref{InerPd},切丛\upref{TanBun},余切丛(未完成),张量场\upref{TenMan}}

本节采用爱因斯坦求和约定.

黎曼度量(Riemannian metric)或伪黎曼度量(pseudo-Riemannian metric)是黎曼几何或伪黎曼几何所要求的基本结构.赋予黎曼度量/伪黎曼度量的微分流形被称为黎曼流形/伪黎曼流形,它们既是几何学研究的对象,也是广义相对论得以展开的舞台.

这里将一直设 $M$ 是一个 $n$ 维实微分流形.

\subsection{黎曼度量}
\begin{definition}{黎曼度量}
$M$ 上的一个黎曼度量 $g$ 是指丛 $T^*(M)\otimes T^*(M)$ 的一个对称的正定截面.等价地,给出黎曼度量 $g$,就相当于在每一点 $p$ 的切空间 $T_pM$ 上指定一个内积 $g_p$.指定了黎曼度量的微分流形称为黎曼流形.有时也会把黎曼度量记为 $\langle\cdot,\cdot\rangle_p$.
\end{definition}
给定局部坐标系 $\{x^i\}$ 后,黎曼度量 $g$ 的局部表达式是
$$
g_{ij}(x)dx^i\otimes dx^j,
$$
这里 $g_{ij}(x)=g_{ji}(x)$,而且对于任何向量 $(X^i)_{i=1}^n\neq0$ 都有
$$
g_{ij}X^iX^j>0.
$$

有了黎曼度量,就可以谈论诸如长度/面积/体积之类的度量性质了.例如,设 $\gamma:[a,b]\to M$ 是黎曼流形 $(M,g)$ 上的一条道路,则定义其长度为
$$
L[\gamma]:=\int_{a}^b \sqrt{g_{\gamma(t)}(\gamma'(t),\gamma'(t))}dt.
$$
如果用局部坐标系给出曲线的局部参数方程 $x(t)=(x^i(t))_{i=1}^n$,则
$$
L[\gamma]:=\int_{a}^b \sqrt{g_{ij}(x(t))(\dot x^i(t),\dot x^j(t))}dt.
$$
容易验证:曲线的长度不依赖于其参数化的方式.

\subsection{洛伦兹度量}