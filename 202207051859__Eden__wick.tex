% Wick 定理
% wick定理|费曼图|编时格林函数

\subsection{导语}
我们先以标量场 $\phi(x)$ 为例研究它的编时格林函数.
量子场论的重要研究对象就是散射过程的费曼矩阵元或 S 矩阵元,它给出了散射过程发生的概率,从而与实验中可直接观测到的散射截面、衰变率等物理量相联系.而根据相互作用场论的相关知识,LSZ 理论将 S 矩阵与编时格林函数联系了起来.为了求出相互作用场论的编时格林函数,我们又可以利用 Gell-Mann-Low 定理,将相互作用场论的真空态 $|\Omega\rangle$ 用自由场论的真空态 $|0\rangle$ 表示:
\begin{equation}
|\Omega\rangle = \lim\limits_{T_0\rightarrow \infty(1-i\epsilon)}\qty(e^{-iE_0(t_0-(-T_0))}\langle \Omega|0\rangle )^{-1}U(t_0,-T_0)|0\rangle
\end{equation}
从而可以得到以下公式
\begin{equation}
\bra\Omega T[\phi(x)\phi(y)]\ket\Omega = \lim\limits_{T_0\rightarrow \infty(1-i\epsilon)}\frac{\bra 0 T(\phi_I(x)\phi_I(y)\exp[])}{}
\end{equation}

于是最终问题变成了,如何求自由场论的编时格林函数.