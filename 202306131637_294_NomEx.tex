% 正规扩张
% 分裂域|多项式|共轭|compositum|composite|合成域

\pentry{纯不可分扩张\upref{PInsEx}}

\autoref{the_SpltFd_2}~\upref{SpltFd}揭示了\textbf{有限扩张}情况下,分裂域和正规扩张的等价性,但并没有说到一般情况。本节的焦点集中在正规扩张本身上,讨论一般情况。


\begin{definition}{共轭}\label{def_NomEx_1}
设$\mathbb{F}$是一个域,$\overline{\mathbb{F}}$是其代数闭包。

对于$\alpha\in\overline{\mathbb{F}}$,其关于$\mathbb{F}$的\textbf{共轭元素(conjugate)}定义为其在$\overline{\mathbb{F}}$的某个保$\mathbb{F}$自同构的像。

对于$\mathbb{F}$的代数扩张$\mathbb{K}$,其关于$\mathbb{F}$的\textbf{共轭域(conjugate)}定义为其在$\overline{\mathbb{F}}$的某个保$\mathbb{F}$自同构的像。
\end{definition}

显然,无论对元素还是域,共轭都是一种等价关系。由于共轭元素和共轭域的定义都依赖同一个自同构,因此,两个共轭域中的元素彼此对应共轭。

类比\autoref{the_SpltFd_3}~\upref{SpltFd}的证明思路,可知两元素共轭的充要条件是,它们是同一个不可约多项式的根。

我们也可以用共轭的语言来描述正规扩张:

\begin{theorem}{}\label{the_NomEx_1}
一个代数扩张$\mathbb{K}/\mathbb{F}$是正规的,当且仅当$\mathbb{K}$关于$\mathbb{F}$的共轭只有它自己,当且仅当任意$a\in\mathbb{K}$的共轭元素仍然在$\mathbb{K}$中。
\end{theorem}



\subsection{正规扩张的性质}

现在我们讨论的是,给定域上正规扩张集合的结构。

\begin{theorem}{}\label{the_NomEx_6}
设$\mathbb{K}/\mathbb{F}$是正规扩张,且存在中间域$\mathbb{M}$,则$\mathbb{K}/\mathbb{M}$也是正规扩张。
\end{theorem}

\textbf{证明}:

% $\opn{ch}\mathbb{F}=p\neq 0$时,正规扩张等价于分裂域。如果$\mathbb{K}=\mathbb{F}(a_1, a_2, \cdots)$,那么$\mathbb{K}=\mathbb{M}(a_1, a_2, \cdots)$。由此得证。

% $\opn{ch}\mathbb{F}=p=0$时,

由于$\mathbb{F}\subseteq\mathbb{M}$,故$\mathbb{K}$的保$\mathbb{M}$自同构一定是保$\mathbb{F}$的,故$\mathbb{K}$关于$\mathbb{M}$的共轭必是关于$\mathbb{F}$的共轭。由\autoref{the_NomEx_1} 则得证。

\textbf{证毕}。





\begin{definition}{合成}
设$\mathbb{F}_i$是$\mathbb{K}$的一族子域,则记$\prod_{i}\mathbb{F}_i$为包含全体$\mathbb{F}_i$的最小的子域,称为族$\{\mathbb{F}_i\}$的\textbf{合成(composite或compositum)}。

$\prod_{i}\mathbb{F}_i$也可以记为$\mathbb{F}_1\mathbb{F}_2\cdots$。
\end{definition}


合成也可以看成是一种扩域,$\mathbb{K}\mathbb{F}=\mathbb{K}(\mathbb{F})=\mathbb{F}(\mathbb{K})$。


\begin{theorem}{}\label{the_NomEx_7}
设$\mathbb{K}/\mathbb{F}$是正规扩张,子域$\mathbb{E}\subseteq\overline{\mathbb{F}}$。如果合成域$\mathbb{EK}$存在,那么$\mathbb{EK}/\mathbb{EF}$是正规扩张。
\end{theorem}

\textbf{证明}:


由于$\mathbb{K}/\mathbb{F}$正规,故任取$\overline{\mathbb{F}}$的$\mathbb{F}$-自同构$\sigma$,$\sigma(\mathbb{K})=\mathbb{K}$。

任取$\overline{\mathbb{F}}$的$\mathbb{EF}$-自同构$\tau$,则$\tau(\mathbb{E})=\mathbb{E}$\footnote{更准确地,$\tau\mid_{\mathbb{E}}=\opn{id}_\mathbb{E}$。}。

由于$\mathbb{F}\subseteq\mathbb{EF}$,故$\overline{\mathbb{F}}$的$\mathbb{EF}$-自同构$\tau$必是$\mathbb{F}$-自同构,从而$\tau(\mathbb{K})=\mathbb{K}$。

综上,$\tau(\mathbb{EK})=\tau(\mathbb{E})\tau(\mathbb{K})=\mathbb{EK}$。

\textbf{证毕}。






\begin{theorem}{}
设$\mathbb{K}_i/\mathbb{F}$是正规扩张,则$\mathbb{K}_1\mathbb{K}_2/\mathbb{F}$是正规扩张。
\end{theorem}


\textbf{证明}:

任取$\overline{\mathbb{F}}$上的$\mathbb{F}$-自同构$\sigma$,则$\sigma(\mathbb{K}_i)=\mathbb{K}_i\implies \sigma(\mathbb{K}_1\mathbb{K}_2)=\sigma(\mathbb{K}_1)\sigma(\mathbb{K}_2)=\mathbb{K}_1\mathbb{K}_2$。

\textbf{证毕}。


\begin{theorem}{}\label{the_NomEx_2}
设$\mathbb{K}_i/\mathbb{F}$是正规扩张,则$\bigcap_{i}\mathbb{K}_i/\mathbb{F}$是正规扩张。
\end{theorem}

\textbf{证明}:

正规扩张的定义:任取$f\in\mathbb{F}[x]$,若其有一根在$\mathbb{K}_i$中,则其所有根都在$\mathbb{K}_i$中。由定义直接得证。

\textbf{证毕}。


考虑到共轭的定义及其性质,即\autoref{the_NomEx_1} ,我们可以得到下面这个性质:

\begin{theorem}{}\label{the_NomEx_3}
设$\mathbb{K}/\mathbb{F}$是一个代数扩域。$\mathbb{F}$的全体包含$\mathbb{K}$的正规扩张之交集,是$\mathbb{K}$关于$\mathbb{F}$的全体共轭域之\textbf{合成}。
\end{theorem}

\textbf{证明}:

由于正规扩张包含所有的根,共轭域之间的元素也对应共轭,且共轭元素是同一个不可约多项式的根,故$\mathbb{F}$的每一个包含$\mathbb{K}$的正规扩张,都包含$\mathbb{K}$关于$\mathbb{F}$的全体共轭域。

下证全体共轭域的合成是正规扩张。

$\mathbb{K}$关于$\mathbb{F}$的全体共轭域之合成记为$\mathbb{H}$。设$\mathbb{K}$关于$\mathbb{F}$的全体共轭域的并集为$S$,则$\mathbb{H}=\mathbb{F}(S)$。

任取$\overline{\mathbb{F}}$的保$\mathbb{H}$自同构$\sigma$,则由共轭域的定义,$\sigma(S)=S$。于是$\sigma(\mathbb{F}(S))=\sigma(\mathbb{F})(\sigma(S))=\mathbb{F}(\sigma(S))=\mathbb{F}(S)$。即,$\sigma(\mathbb{H})=\mathbb{H}$。

由\autoref{the_NomEx_1} 即得证。

\textbf{证毕}。


\begin{theorem}{}\label{the_NomEx_4}
域$\mathbb{F}$的有限扩张,总包含在$\mathbb{F}$的某个有限正规扩张里;$\mathbb{F}$的可分扩张,总包含在$\mathbb{F}$的某个可分正规扩张里。
\end{theorem}

\textbf{证明}:

先证明有限扩张的情况:

设$\mathbb{K}/\mathbb{F}$是有限扩张,那么作为$\mathbb{F}$上的线性空间,$\mathbb{K}$的基只有有限多个元素。$\mathbb{K}$的任何保$\mathbb{F}$自同构,由于是同构,因此只取决于基向量映射到哪里。由于有限扩张必是代数扩张,故每个基向量都是代数元素,故每个基向量的共轭元素是有限多的。综上,$\mathbb{K}$关于$\mathbb{F}$的共轭域只能是有限多个。

据\autoref{the_NomEx_3} ,取$\mathbb{K}$关于$\mathbb{F}$的共轭域之合成。由于每个共轭域在$\mathbb{F}$都是有限维线性空间,则其合成也是有限维的\footnote{各共轭域取一基向量求积,所得的集合即是合成域的基。因此,如果每个共轭域的维数是$n$,一共$k$个共轭域,则其合成的维数不超过$n^k$。}。

再证明可分扩张的情况:

设$\mathbb{K}/\mathbb{F}$是有限扩张,取$\mathbb{K}$关于$\mathbb{F}$的共轭域之合成,记为$\mathbb{L}$。由\textbf{可分扩张的传递性}\autoref{cor_SprbE2_3}~\upref{SprbE2},可知$\mathbb{L}$上的任意元素都是可分元素,进而是可分扩张。

\textbf{证毕}。

把\autoref{the_NomEx_4} 的两个情况组合起来,也能得到自然推论:有限可分扩张总包含在有限可分正规扩张里。


\subsection{纯不可分扩张}



留意\autoref{the_PInsEx_2}~\upref{PInsEx},纯不可分元素与其在域自同构下的像的数目息息相关。和正规扩张结合起来,这一性质可以延伸出下列性质:

\begin{theorem}{}\label{the_NomEx_5}
设$\mathbb{K}/\mathbb{F}$是\textbf{正规扩张},记
\begin{equation}\label{eq_NomEx_1}
\mathbb{S}=\{\alpha\in\mathbb{K}\mid \sigma\alpha = \alpha, \sigma\text{是域}\overline{\mathbb{F}}\text{的任意保}\mathbb{F}\text{自同构}\}~,
\end{equation}
则$\mathbb{S}/\mathbb{F}$是纯不可分扩张,$\mathbb{K}/\mathbb{S}$是可分扩张。
\end{theorem}

\textbf{证明}:

首先要证明$\mathbb{S}$确实是一个域\footnote{非常显然,建议能自己想就跳过本段说明。}:任取$\alpha, \beta, \gamma\in\mathbb{S}$,则有$\sigma(\alpha\beta+\gamma)=\sigma(\alpha)\sigma(\beta)+\sigma(\gamma)=\alpha\beta+\gamma$,所以$\mathbb{S}$的元素之间相加、相乘是封闭的;由于$\sigma(\alpha^{-1})=(\sigma(\alpha))^{-1}=\alpha^{-1}$和$\sigma(-\alpha)=-(\sigma(\alpha))/=-\alpha/$,所以取逆运算也封闭。

然后,据\autoref{the_PInsEx_2}~\upref{PInsEx}直接可得$\mathbb{S}/\mathbb{F}$是纯不可分扩张。

接下来证明$\mathbb{K}/\mathbb{S}$是可分扩张,只考虑$\opn{ch}\mathbb{F}$是素数$p$的情况,因为特征为$0$的域必是完美域\footnote{见\textbf{可分扩张}\upref{SprbEx}。}。

任取$\alpha\in\mathbb{K}$。取$\mathbb{F}(\alpha)$的\textbf{全体}保$\mathbb{F}$单同态$\varphi_1, \cdots, \varphi_n:\mathbb{F}\to\overline{\mathbb{F}}$,\textbf{互不相同},其中$\varphi_1$是恒等映射。由\autoref{the_FldExp_5}~\upref{FldExp},可将每个$\varphi_i$开拓为$\overline{\mathbb{F}}$的保$\mathbb{F}$自同构$\phi_i$。

由于$\mathbb{K}/\mathbb{F}$是正规扩张,而域同态总是把多项式的根映射为另一根,故$\varphi_i\mathbb{K}\subseteq\mathbb{K}$。因为$\mathbb{F}(\alpha)$是由$\alpha$生成的,可知不同的$\varphi_i$将$\alpha$映入不同的$\varphi_i\alpha$,并且因为$\{\varphi_i\}$便历所有可能性,$\{\varphi_i\alpha\}$就是$\opn{irr}(\alpha, \mathbb{F})$的\textbf{全体根}集合。

构造$\mathbb{K}[x]$上的多项式:
\begin{equation}
f(x) = (x-\varphi_1\alpha)(x-\varphi_2\alpha)\cdots(x-\varphi_n\alpha)~,
\end{equation}
则由于已设定$\varphi_1$是恒等映射,可知$f(\alpha)=0$;由于各$\varphi_i\alpha$互不相同,可知$f$是可分多项式\footnote{注意,这里没法证明$f=\opn{irr}(\alpha, \mathbb{F})$,因为无法保证$f\in\mathbb{F}[x]$。最多只能确定$\opn{irr}(\alpha, \mathbb{F})=f^k$,其中$k$是正整数。换句话说,没法证明$f$的系数都在$\mathbb{F}$上。}。

下证$f$的系数都在$\mathbb{S}$上。

任取$\overline{\mathbb{F}}$的保$\mathbb{F}$自同构$\sigma$,则$\sigma\mid_{\{\varphi_i\alpha\}}$是一个$\{\varphi_i\alpha\}$($\opn{irr}(\alpha, \mathbb{F})$的全体根集合)上的置换。于是有
\begin{equation}\label{eq_NomEx_2}
\begin{aligned}
\sigma(\prod_i \varphi_i\alpha) &= \prod_i \varphi_i\alpha~,\\
\sigma(\sum_{j}\frac{\prod_i \varphi_i\alpha}{\varphi_j\alpha}) &= \sum_{j}\frac{\prod_i \varphi_i\alpha}{\varphi_j\alpha}~,\\
\sigma(\sum_{j, k}\frac{\prod_i \varphi_i\alpha}{\varphi_j\alpha\times\varphi_k\alpha}) &= \sum_{j, k}\frac{\prod_i \varphi_i\alpha}{\varphi_j\alpha\times\varphi_k\alpha}~,\\
&\vdots\\
\sigma(\sum_i\varphi_i\alpha) &= \sum_i\varphi_i\alpha~,\\
\sigma(1) &= 1~.
\end{aligned}
\end{equation}
式中第$k$行左边的括号里和右边是$f(x)$的第$k-1$次项系数。

由\autoref{eq_NomEx_2} ,$f$的每一项系数都满足\autoref{eq_NomEx_1} ,故$f$的系数都在$\mathbb{S}$上。

故$\opn{irr}(\alpha, \mathbb{S})\mid f$,因此也是可分的。

故$\alpha$是$\mathbb{S}$上的可分元素。由$\alpha$的任意性,得证$\mathbb{K}/\mathbb{S}$是可分扩张。

\textbf{证毕}。

注意\autoref{the_NomEx_5} 和\autoref{the_PInsEx_3}~\upref{PInsEx}的描述,恰好是对偶的:后者是先进行可分扩张再进行纯不可分扩张,前者则反了过来。




% 注释掉的原因:本来想引入gtm 242的几个定理,却发现已经在可分扩张\upref{SprbEx}里讲清楚了。毕竟本书思路和gtm 242有所不同,这方面比gtm 242要更细。
% \subsection{完美域}


% 我们已经在\textbf{可分扩张}\upref{SprbEx}中简单讨论过完美域,这里添加一些其和可分性的联系。

% \begin{lemma}{}
% 完美域的纯不可分扩张只有它自己。
% \end{lemma}

% \textbf{证明}:

% 由\autoref{the_SprbEx_5}~\upref{SprbEx},完美域的代数扩张全都是可分扩张,从而由\autoref{def_PInsEx_2}~\upref{PInsEx}直接得证。

% \textbf{证毕}。

% 注意,此证明同时适用于完美域的特征为$0$或者素数$p$的情况。


















