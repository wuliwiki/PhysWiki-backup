% 2018 年计算机学科专业基础综合全国联考卷
% 2018 年计算机学科专业基础综合全国联考卷

\subsection{一、单项选择题}
第1~40 小题,每小题2 分,共80 分.下列每题给出的四个选项中,只有一个选项最符合试题要求.

1.若栈S1 中保存整数,栈S2 中保存运算符,函数F() 依次执行下述各步操作:
( 1)从S1 中依次弹出两个操作数a 和b;
( 2)从S2 中弹出一个运算符op;
( 3)执行相应的运算b op a;
( 4)将运算结果压人S1 中.
假定S1 中的操作数依次是5, 8, 3, 2( 2 在栈顶),S2 中的运算符依次是*, - , + (+在栈顶).调
用3 次F() 后, S1 栈顶保存的值是.
A . -15 B. 15 C. -20 D. 20
2. 现有队列Q 与栈S,初始时Q 中的元素依次是1, 2, 3, 4, 5, 6( 1 在队头),S 为空.若仅允
许下列3 种操作:①出队并输出出队元素;②出队并将出队元素人栈;③出栈并输出出栈元素,
则不能得到的输出序列是.
A . 1, 2, 5, 6, 4, 3 B. 2, 3, 4, 5, 6, 1
C. 3, 4, 5, 6, 1, 2 D. 6, 5, 4, 3, 2, 1
3. 设有一个 12×12 的对称矩阵 M,将其上三角部分的元素 mi, j(1≤i ≤j ≤)按12行优先存人 C
语言的一维数组N 中,元素m6, 6 在N 中的下标是.
A . 50 B. 51 C. 55 D. 66
4. 设一棵非空完全二叉树T 的所有叶结点均位于同一层,且每个非叶结点都有2 个子结点.
若T 有k 个叶结点,则T 的结点总数是.
A . 2k-1 B. 2k C. k2 D. 2k-1
5. 已知字符集{a, b, c, d, e, f} ,若各字符出现的次数分别为6, 3, 8, 2, 10, 4 ,则对应字符集中
各字符的哈夫曼编码可能是.
A . 00, 1011, 01, 1010, 11, 100 B. 00, 100, 110, 000, 0010, 01
C. 10, 1011, 11, 0011, 00, 010 D. 0011, 10, 11, 0010, 01, 000
6. 已知二叉排序树如下图所示,元素之间应满足的大小关系是.