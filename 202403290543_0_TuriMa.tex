% 图灵机到真实计算机
% license Usr
% type Tutor

\begin{issues}
\issueDraft
\end{issues}

\textbf{图灵机要素}:无限长纸带、有限字符集、有限的状态集、根据状态和当前字符决定下一步行为(写入、移动)、初始状态、停机状态。

图灵可以完成任何计算机可以完成的事情。如果一个编程语言可以做图灵机的任何事情(除了纸带不是无限长),那么他就是图灵完备的。

\subsubsection{真实计算机}
\textbf{纸带被划分为几个部分}:\textbf{代码段}(储存程序指令,一般只读)、\textbf{数据段}(分为只读和读写,只读部分是 literal,读写部分是全局变量和 static 变量)、\textbf{栈}(函数局部变量)、\textbf{堆}(动态分配内存)。


