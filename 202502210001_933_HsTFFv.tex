% 函数视角下的三角函数(高中)
% keys 函数|三角函数|周期|性质
% license Usr
% type Tutor

\begin{issues}
\issueDraft
\end{issues}

\pentry{三角函数\nref{nod_HsTrFu},函数\nref{nod_functi},函数的性质\nref{nod_HsFunC},导数\nref{nod_HsDerv}}{nod_5a43}

在前面的内容中,已经接触过三角函数的定义,并基于这些定义推导出了诱导公式及同角三角函数之间的关系。这些推导主要依赖于任意角和三角函数的几何定义。然而,三角函数不仅仅是几何分析的工具,它们本质上也是一种函数,并具备一般函数的基本性质,如周期性、单调性和对称性。因此,本文将从函数的角度进一步分析三角函数,考察它们的性质、图像、变化趋势等。需要注意的是,这些视角本质上是等价的,它们都在描述同一数学对象。无论是几何定义还是函数分析,最终指向的都是相同的数学结构。这种多重视角的统一性,正是三角函数作为数学工具的强大之处。它不仅能够通过直观的几何形式展现对称性和变换规律,也能在函数的框架下揭示更广泛的性质,为各种数学应用提供坚实的基础。

另外,在三角函数的介绍中,有一个广为流传的动画:一个点在单位圆上运动,表示角度的变化,同时,在单位圆的右侧和上侧,将角度与对应的线段长度映射到另一坐标系,从而自然引出各个三角函数的图像。尽管这种动画能够直观展示三角函数的变化过程,更理想的方式是能够在脑海中主动演练这一过程。看到函数图像时,能够自动联想到单位圆上的点如何旋转;反之,观察圆周运动时,能够迅速在脑海中构建出相应的函数图像。这种能力不仅有助于理解三角函数的本质,也将在更深入的数学学习中提供帮助。本文内容主要关注正弦、余弦与正切函数,其余三角函数由于与它们存在倒数关系,将适当涉及,但不会展开详细推导。

\subsection{定义域与值域}

由于三角函数的自变量是任意角,所以理论上所有三角函数的定义域都应该是实数域,然而在讨论三角函数的集合含义时曾经\aref{提及}{eq_HsTrFu_13}过,$\tan x$在$\displaystyle x\in\{x|x=\frac{\pi}{2}+k\pi,(k\in\mathbb{Z})\}$时无意义,因此:
\begin{itemize}
\item $\sin x,\cos x$的定义域为$\mathbb{R}$;
\item $\tan x,\sec x$的定义域为$\displaystyle\{x|x\neq\frac{\pi}{2}+2k\pi,k\in\mathbb{Z}\}$;
\item $\cot x,\csc x$的定义域为$\displaystyle\{x|x\neq k\pi,k\in\mathbb{Z}\}$。
\end{itemize}

关于值域,正弦和余弦对应的都是单位圆上的横纵坐标,显然范围是在$[-1,1]$。而$\tan x$的分析稍微麻烦一些。根据几何定义中正切所对应的线段,

研究 $\cos x$ 在 $\frac{\pi}{2}$ 附近的变化
	•	当 $x \to \frac{\pi}{2}^-$(即 $x$ 从左侧逼近 $\frac{\pi}{2}$),$\cos x$ 逐渐趋近于 $0$ 且 $\cos x > 0$。
	•	而 $\sin x \to 1$,保持正值。
	•	因此,$\tan x = \frac{\sin x}{\cos x}$ 由于 $\cos x$ 在正数方向上趋近于 $0$,导致 $\tan x$ 迅速增大,趋于 $+\infty$。

类似地:
	•	当 $x \to -\frac{\pi}{2}^+$(即 $x$ 从右侧逼近 $-\frac{\pi}{2}$),$\cos x$ 依然趋向 $0$,但此时 $\cos x < 0$。
	•	$\sin x \to -1$,保持负值。
	•	由于 $\tan x = \frac{\sin x}{\cos x}$,$\sin x$ 为负,$\cos x$ 也是趋近于 $0$ 的负数,所以 $\tan x$ 迅速变小,趋向 $-\infty$。

3. 解释 $\tan x$ 在 $(-\frac{\pi}{2}, \frac{\pi}{2})$ 内覆盖 $\mathbb{R}$
	•	$\tan x$ 在 $(-\frac{\pi}{2}, \frac{\pi}{2})$ 内是 严格递增 的:
$$
(\tan x)’ = \sec^2 x > 0
$$
说明它是单调函数,不会有间断的跳跃值。
	•	$\tan x$ 在这个区间的极限行为表明:
$$
\lim\limits_{x \to -\frac{\pi}{2}^+} \tan x = -\infty, \quad \lim\limits_{x \to \frac{\pi}{2}^-} \tan x = +\infty
$$
	•	既然 $\tan x$ 是单调递增的,并且极限可以分别趋近于 $-\infty$ 和 $+\infty$,那么在 $(-\frac{\pi}{2}, \frac{\pi}{2})$ 内,$\tan x$ 必然遍历整个 $\mathbb{R}$,即它的值域是 $\mathbb{R}$。

\subsection{周期性}

正弦函数、余弦函数、正切函数的都是周期函数,根据定义易得,正弦函数和余弦函数,周期为 $2k\pi(k\in Z,k\neq0)$,正切函数的周期为 $k\pi(k\in Z,k\neq0)$.


\subsection{图像}
根据前面的推导,可以得到基本三角函数的图像如下图。
\begin{figure}[ht]
\centering
\includegraphics[width=14.25cm]{./figures/14fd66d8d1e6e0b5.png}
\caption{$\sin x$和$\cos x$} \label{fig_HsTFFv_1}
\end{figure}
可以看出正弦函数和余弦函数是定义域为 $R$ 值域为 $[-1,1]$ 最小正周期 $T = 2\pi$ 的周期函数。

\begin{figure}[ht]
\centering
\includegraphics[width=14.25cm]{./figures/6f97182187b36e36.png}
\caption{$\tan x$和$\cot x$} \label{fig_HsTFFv_3}
\end{figure}

作为扩展,下面也给出正割函数与余割函数的函数图像,他们的性质均可通过与正弦和余弦的关系分析得到,此处不予赘述。

\begin{figure}[ht]
\centering
\includegraphics[width=14.25cm]{./figures/56f93ee1a7fb0faa.png}
\caption{$\sec x$和$\csc x$} \label{fig_HsTFFv_2}
\end{figure}

\subsection{从三角函数推广得到的其他三角函数}

\subsection{导数}