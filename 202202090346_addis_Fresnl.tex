% 菲涅尔公式、布儒斯特角、临界角、内反射与外反射
% 菲涅尔|折射|反射|偏振|布儒斯特角

\begin{issues}
\issueTODO
\end{issues}

\pentry{麦克斯韦方程组(介质)\upref{MWEq1}}

\subsection{菲涅尔公式}
\begin{figure}[ht]
\centering
\includegraphics[width=14cm]{./figures/Fresnl_1.pdf}
\caption{菲涅尔公式} \label{Fresnl_fig1}
\end{figure}
 
利用具体的电磁场的边界条件 % 链接未完成
\begin{itemize}
\item $\div \bvec D = 0$ 和$\div \bvec B = 0$  分别对应 $\epsilon \bvec E_\bot = \epsilon' \bvec E'_\bot$ 和 $\epsilon \bvec B_\bot = \epsilon' \bvec B'_\bot$.

\item $\div \bvec E = 0$ 和 $\div \bvec H = 0$ 分别对应 $\bvec E_{//} = \bvec E'_{//}$ 和 $\bvec B_{//}/\mu = \bvec B'_{//}/\mu'$.
\end{itemize}

现在分两种情况讨论
\begin{enumerate}
\item 极化方向垂直于入射面(\autoref{Fresnl_fig1} 右)

\begin{equation}
\frac{E_R^{(s)}}{E_I^{(s)}} =  \frac{m_1\cos{\theta_i} - m_2\cos\theta_t}{m_1\cos\theta_i + m_2\cos\theta_t}
\qquad
\frac{E_T^{(s)}}{E_I^{(s)}} = \frac{2 m_1\cos\theta_i}{m_1\cos\theta_i + m_2\cos\theta_t}
\end{equation}

\item 极化方向平行于入射面(\autoref{Fresnl_fig1} 左)
\begin{equation}\label{Fresnl_eq2}
\frac{E_R^{(p)}}{E_I^{(p)}} =  \frac{m_2\cos\theta_i - m_1\cos\theta_t}{m_2 \cos\theta_i + m_1\cos\theta_t}
\qquad
\frac{E_T^{(p)}}{E_I^{(p)}} =  \frac{2 m_1\cos\theta_i}{m_2\cos\theta_i + m_1\cos\theta_t}
\end{equation}
\end{enumerate}
其中 $m_1=n_1/\mu_1 = c\sqrt{\epsilon_1/\mu_1}$, 一般情况下介质的磁导率与真空区磁导率的区别可忽略,即可以把 $m_1$ 替换为折射率 $n_1$. 另外注意菲涅尔公式包含相位信息,即以上的 $E$ 可以是复振幅.

这两个表达式是菲涅尔方程中的两个,它们是极其普遍的陈述,适用于任何线性、各向同性的均匀介质.%未完成:均匀介质的定义
当电介质的磁导率 $\mu_1\approx\mu_2\approx\mu_0=1$ 时,$n/\mu \approx n$, 菲涅尔公式可以简化为:

\begin{equation}\label{Fresnl_eq3}
r_s \equiv \left(\frac{E_R}{E_I}\right)_s = \frac{n_1\cos{\theta_i} - n_2\cos\theta_t}{n_1\cos\theta_i + n_2\cos\theta_t}
\end{equation}
\begin{equation}\label{Fresnl_eq4}
t_s \equiv \left(\frac{E_T}{E_I}\right)_p =  \frac{2 n_1\cos\theta_i}{n_1\cos\theta_i + n_2\cos\theta_t}
\end{equation}
其中,$r_s$表示\textbf{振幅反射系数},$t_s$表示\textbf{振幅透射系数}.
同理:
\begin{equation}\label{Fresnl_eq5}
r_p \equiv \left(\frac{E_R}{E_I}\right)_s = \frac{n_2\cos{\theta_i} - n_1\cos\theta_t}{n_1\cos\theta_t + n_2\cos\theta_i}
\end{equation}
\begin{equation}\label{Fresnl_eq6}
t_p \equiv \left(\frac{E_T}{E_I}\right)_p =  \frac{2 n_1\cos\theta_i}{n_1\cos\theta_t + n_2\cos\theta_i}
\end{equation}

应用斯涅尔定律\upref{Snel},可以进一步使记号简化.别害怕,它们只是看起来有些复杂:
\begin{equation}\label{Fresnl_eq7}
r_s = -\frac{\sin(\theta_i - \theta_t)}{\sin(\theta_i + \theta_t)}
\end{equation}
\begin{equation}\label{Fresnl_eq8}
r_p = +\frac{tan(\theta_i - \theta_t)}{tan(\theta_i + \theta_t)}
\end{equation}
\begin{equation}\label{Fresnl_eq9}
t_s = -\frac{2\sin\theta_t\cos\theta_i}{\sin(\theta_i + \theta_t)}
\end{equation}
\begin{equation}\label{Fresnl_eq10}
t_p = +\frac{2\sin\theta_t\cos\theta_i}{\sin(\theta_i + \theta_t)\cos(\theta_i - \theta_t)}
\end{equation}
这里请大家注意,推导菲涅尔方程时,场的方向(更精确地说是相位)是相当任意地选择的.因此,为了避免混乱,必须把菲涅尔方程与导出它们的特定的场的方向联系起来.

\subsection{内反射与外反射}
\begin{definition}{外反射}
设入射介质与出射介质的折射系数分别为 $n_1$ 和 $n_2$,若 $n_1<n_2$,则称为\textbf{外反射(external reflection)};反之,为\textbf{内反射(internal reflection)}.
\end{definition}

\subsection{布儒斯特角}
我们这里考虑常见的 $n_2>n_1$ 且 $\mu_1 = \mu_2$ 情况.由\autoref{Fresnl_eq8} 容易证明当 $\theta_i + \theta_t = \pi/2$时, $r_p = 0$, 反射光的平行分量消失,反射光为线偏振光. 此时,入射角称为\textbf{布儒斯特角(Brewster's angle)} ,记为 $\theta_B$. 代入斯涅尔公式 $n_i\sin\theta_i = n_t\sin\theta_t$可得
\begin{equation}
\theta_B = \arctan (n_2/n_1)
\end{equation}

\subsection{临界角}
\begin{definition}{}
当 $\theta_t  = \frac{\pi}{2}$ 时,此时的入射角 $\theta_i$ 称为 \textbf{临界角(critical angle)},记为 $\theta_c$.
\end{definition}

\begin{exercise}{}
类比 $\theta_B$ 的求法,给出 $\theta_c$ 表达式.
\end{exercise}

易知, $\theta_c = \arcsin(n_2/n_1)$
% 未完成: 画图!
\addTODO{线偏振光} 


