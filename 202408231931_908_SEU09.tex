% 东南大学 2009 年 考研 量子力学
% license Usr
% type Note

\textbf{声明}:“该内容来源于网络公开资料,不保证真实性,如有侵权请联系管理员”

\textbf{1.(15 分)}以下叙述是否正确:(1)在定态下,任意不是含的力学量的平均值均不随时间变化:(2)若厄密算符与对易,则它们必有共同本征态:(3)一维谐振子的所有能级均是非简并的:(4)厄密算符的本征值必为正数(5)时间反演对称性导致能量守恒

\textbf{2.(15 分)}质量为 $m$ 的粒子作一维运动,几率守恒定理为
\[
\partial \rho/\partial t + \partial j/\partial x = 0,~
\]
其中,$\rho(x,t) = |\psi|^2$, $j(x,t) = -(i\hbar/2m)(\psi^*\partial \psi/\partial x - \psi \partial \psi^*/\partial x)$。

\begin{enumerate}
    \item 若粒子处于定态 $\psi = \phi(x) \exp(-iEt/\hbar)$,试证 $j = c$(与 $z,t$ 无关的常数);
    \item 若自由粒子处于动量本征态 $\psi(x,t) = \exp(ipx/\hbar - iEt/\hbar)$,试证 $j = p/m$。
\end{enumerate}

\textbf{3.(15 分)}试在坐标表象中写出:

\begin{enumerate}
    \item 位置算符 $\hat{x}$ 的本征函数;
    \item 动量算符 $\hat{p}_z$ 的本征函数;
    \item $\{\hat{x},\hat{y}, \hat{p}_z\}$ 的共同本征函数。
\end{enumerate}

\textbf{4.(15 分)}已知 $[\hat{A}, \hat{B}] = \hat{A}\hat{B} - \hat{B}\hat{A}$,$[\hat{A}, \hat{B}]_+ = \hat{A}\hat{B} + \hat{B}\hat{A}$,验证:

\begin{enumerate}
    \item $[A, BC] = A[B, C]_{+} - [A, C]_{+}B$;
    \item $[A, BC] = [A, B]C - B[A, C]$。
\end{enumerate}