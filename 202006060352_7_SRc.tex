% 约化光速

\subsection{约去量纲}

初高中物理常用一个计算技巧,在公式各处单位制统一的情况下,可以把单位制和数值分开计算.比如说,$(1\opn{cm}+2\opn{cm}=3\opn{cm})$中,类似“合并同类项”的操作.

相同的物理量有相同的量纲,只不过单位制可能不一样.同样是表示长度,量纲都是长度,单位却可能选取厘米、英寸、天文单位、光年等.只有相同的物理量可以相加减,不同的物理量之间可以相互乘除从而得到新的物理量.比如说,长度和时间相除,可以得到速度.

有了以上约束,合法的带单位多项式中,相加减的必然是相同的物理量.如果我们给每个物理量都选择统一的单位制,那么就可以通过合并同类项,把物理量集中放在一起,进行乘除,得到最终的单位;数值部分计算后和最终单位放在一起,就是最终的表达.

\begin{example}{约去量纲的例子}
不考虑相对论效应.一辆火车以速度$3.6\opn{km/h}$向东行驶,车上有一个小朋友以速度$4\opn{m/s}$向东奔跑,那么小朋友相对铁轨的速度就是$3.6\times\frac{\opn{km}}{\opn{h}}+4\times\frac{\opn{m}}{\opn{s}}=3.6\times\frac{1000\cdot \opn{m}}{3600\opn{s}}+4\times\frac{\opn{m}}{\opn{s}}=(3.6\times\frac{1000}{3600}+4)\frac{\opn{m}}{\opn{s}}=5\frac{\opn{m}}{\opn{s}}=5\opn{m/s}$.
\end{example}

\subsection{约去光速}

如果取长度单位为$\opn{m}$,时间单位为$\opn{\tau}=299792458\opn{s}$,那么光速就可以写为$1\opn{m/\tau}$.用$\opn{\tau}$取代$\opn{s}$作为时间单位,那么一切涉及光速的等式中,我们都可以把光速的\textbf{数值}写为$1$,大大简化计算,而光速的量纲$[\opn{m/s}]=[\opn{m/\tau}]$是独立于数值进行计算的.在这种写法中,$0.1$倍光速就可以写为$0.1\opn{m/\tau}$.




