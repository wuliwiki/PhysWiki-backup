% 矢量的模和度量的关系
% 矢量|模|度量|关系

\pentry{度量空间\upref{Metric},埃尔米特矢量空间(酉空间)\upref{HVorUV}}

在带有内积的矢量空间(实数域 $\mathbb R$ 或复数域 $\mathbb C$ 上的)中,矢量的模的性质可用来定义度量,使得矢量空间可看成一个度量空间.反过来,满足一定性质的度量也可用来定义矢量的模.

先列出矢量\textbf{模} $\norm{v}=\sqrt{\braket{v}{v}}$ 的性质(设矢量空间为 $V$):
\begin{enumerate}
\item 
\begin{equation}
\norm{x}\geq 0\quad \forall x\in V
\end{equation}
且 $\norm{x}=0\Rightarrow x=0$.
\item \begin{equation}
\norm{\lambda v}=\abs{\lambda}\norm{v},\quad \lambda\in V,\quad\forall\lambda\in\mathbb C,v\in V
\end{equation}
\item 
\begin{equation}
\norm{x+y}\leq\norm{x}+\norm{y},\quad \forall x,y\in V
\end{equation}
\end{enumerate}

而度量是满足以下三个条件的函数 $d:E\times E\rightarrow\mathbb R$( $E$ 为任意的集合):
\begin{enumerate}
\item 正定性:$d(u, v) \geq 0$,且 $d(u, v)=0$ 当且仅当 $u=v$
\item 对称性:$d(u, v) = d(v, u)$
\item 三角不等式:$d(u, v) \leqslant d(u, w) + d(w, v)$
\end{enumerate}
\subsection{从矢量模到度量}
通过模 $\norm{x}$ 取
\begin{equation}
d(x,y):=\norm{x-y}
\end{equation}
显然由此定义的 $d$ 满足度量定义中的3个条件.
\begin{example}{}
在 $V=C_2(a,b)$ (\autoref{EuVS_ex1}~\upref{EuVS})中
\begin{equation}
d(f,g)=\sqrt{\int_a^b\abs{f(x)-g(x)}^2\dd x}
\end{equation}

\end{example}