% 矩阵对易与共同本征矢

% 待删除!
% 和 “算符对易与共同本征函数” 重复了!!!!!!!!!

由于矩阵乘法满足结合律(\autoref{Mat_eq1}\upref{Mat}), 算符对易意味着, 对于任意一个列矢量 $\bvec v$(看成单列矩阵), 先将 $\mat A$ 对其作用再将 $\mat B$ 对其作用, 等于以相反的顺序作用. 即
\begin{equation}\label{OpComu_eq1}
\mat B(\mat A \bvec v) = \mat A (\mat B \bvec v)
\end{equation}

为什么要讨论两个矩阵是否对易? 因为对易与矩阵的本征问题紧密关联

\begin{theorem}{对易与共同本征矢}
令 $\mat A$ 和 $\mat B$ 为 $N$ 维矢量空间中的厄米矩阵, 以下两个命题互为充分必要条件
\begin{enumerate}
\item $[\mat A, \mat B] = \mat 0$
\item 空间中存在一组正交基底, 同时是 $\mat A$ 和 $\mat B$ 的共同本征矢
\end{enumerate}
\end{theorem}

\subsubsection{证明}
首先用条件 2 证明条件 1. 记这组共同本征矢为 $\bvec v_i \ \ (i = 1, \dots , N)$, 且令本征值为 $a_i, b_i$, 即
\begin{equation}
\begin{cases}
\mat A \bvec v_i = a_i \bvec v_i\\
\mat B \bvec v_i = b_i \bvec v_i
\end{cases}
\end{equation}
那么对所有 $i = 1, \dots, N$ 都有
\begin{equation}
\mat B (\mat A \bvec v_i) = \mat B (a_i \bvec v_i) = a_i \mat B \bvec v_i = a_i b_i \bvec v_i
\end{equation}
\begin{equation}
\mat A (\mat B \bvec v_i) = \mat A (b_i \bvec v_i) = b_i \mat A \bvec v_i = a_i b_i \bvec v_i
\end{equation}
可见\autoref{OpComu_eq1} 对每个基底都成立, 所以对空间中的任意矢量也成立(因为任意矢量可以表示成基底的线性组合, 而算符都是线性的), 所以 $\mat A \mat B = \mat B \mat A$.

再来用条件 1 证明条件 2. 这要更复杂一些. 我们解 $\mat A$ 的本征方程
\begin{equation}\label{OpComu_eq2}
\mat A \bvec v_i = a_i \bvec v_i
\end{equation}
由于 $\mat A$ 是厄米矩阵, 我们必定可以得到一组($N$ 个)两两正交的本征矢量 $\bvec v_i$ 和本征值 $a_i$. 由于两个算符对易, 有
\begin{equation}\label{OpComu_eq3}
\mat A (\mat B \bvec v_i) = \mat B (\mat A \bvec v_i) = \mat B (a_i \bvec v_i) = a_i (\mat B \bvec v_i)
\end{equation}
观察等式两端, 这说明 $\mat B \bvec v_i$ 矢量同样也是本征值 $a_i$ 对应的本征矢.

现在分两种情况讨论. 第一种 $\mat A$ 的本征矢不存在简并(注意我们并不在乎 $\mat B$ 是否有简并). 这样每个 $a_i$ 对应一个一维子空间, 空间中的所有矢量都共线, 所以 $\mat B \bvec v_i$ 和 $\bvec v_i$ 共线, 即后者乘以常数等于前者. 令该常数为 $b_i$, 得
\begin{equation}
\mat B \bvec v_i = b_i \bvec v_i
\end{equation}
这就证明了 $\bvec v_i$ 同样是 $\mat B$ 的本征矢.

更复杂的是如果 $\mat A$ 的某个本征值 $a_i$ 存在简并, 那它的所有本征矢可以组成 $n_i > 1$ 维的子空间. 如果 $\bvec v_i$ 属于这个子空间, 那么\autoref{OpComu_eq2} 和\autoref{OpComu_eq3} 未必说明 $\mat B \bvec v_i$ 与 $\bvec v_i$ 共线, 而是只能说明 $\mat B \bvec v_i$ 也在这个子空间中. 或者说 $\mat B$ 在该子空间中具有\textbf{封闭性}. 可以证明在该子空间中, 矩阵 $\mat B$ 必定有 $n_i$ 个正交归一的本征矢, 它们既是 $\mat A$ 的本征矢(因为该子空间中的任何矢量都是), 也是 $\mat B$ 的本征矢.

% 例题未完成
