% 堆排序
% 堆排序|算法|排序|C++

前文讲到了堆\upref{heap}这个数据结构,这里来讲一下堆排序这个排序算法.

堆排序是用二叉堆这种数据结构实现的排序算法,从小到大排序的话是实现小根堆,而从大到小则相反.这里以实现从小到大排序为例.

具体做法是每次假设堆已经建好了,由于是小根堆,所以只需每次输出堆顶元素,再把堆顶删除,涉及到了三种操作:

\begin{enumerate}
\item 建堆;
\item 输出堆顶;
\item 删除堆顶.
\end{enumerate}

由于只有删除堆顶这个操作,所以只需要实现 down 操作.那如何来建堆呢?朴素方法是一个一个往堆中插入,但这种方法太慢了,有一个 $\mathcal{O}(n)$ 的建堆方式,就是从 $\dfrac{n}{2}$ down 到 $1$ 就可以了. 

为什么从 $\dfrac{n}{2}$ 开始 down 呢?假设堆中共有 $n$ 个结点,$n$ 结点的下标最大,$\dfrac{n}{2}$ 这个结点是有子结点的最大值,显然叶结点一定满足堆的性质,所以只需从 $\dfrac{n}{2}$ 开始建堆就能把堆建好.

为什么建堆的时间复杂度是 $\mathcal{O}(n)$ 呢?这里简单的地来证明一下.

\textbf{证明:}

一棵完全二叉树上有 $\left\lceil{\log_2 n}\right\rceil$ 层,叶子结点没有结点了,所以叶子结点不需要 down,所以从 $\dfrac{n}{2}$ 开始 down,所以除了叶子结点,上面的所有结点的个数为 $\dfrac{n}{2}$,除去叶子结点,上面的最后一层结点就是 $\dfrac{n}{4}$.

所以:

\begin{equation}
\dfrac{n}{2} \times 1 + \dfrac{n}{4} \times 2 + \cdots + \dfrac{n}{2^{\left\lceil{\log_2 n}\right\rceil}} \times \left\lfloor{\log_2 n}\right\rfloor \\
= \left\lfloor{\log_2 h}\right\rfloor \sum ^ {\left\lceil{\log_2 n}\right\rceil} _ {h=1} \dfrac{n}{2^h} \\
= n(\dfrac{1}{2^2} + \dfrac{1}{2^3} + \dfrac{1}{2^4} + \cdots +\dfrac{1}{2^{\left\lceil{\log_2 n}\right\rceil}}) \\
= n \sum^{\left\lceil{\log_2 n}\right\rceil} _ {h = 2} 2^h\\
\end{equation}

那么上面的公式是不是 $\mathcal{O}(n)$ 的呢?

\begin{equation}
s = \dfrac{1}{2^2} + \dfrac{1}{2^3} + \dfrac{1}{2^4} + \cdots +\dfrac{1}{2^{\left\lceil{\log_2 n}\right\rceil}} \\
, \ 2s = \dfrac{1}{2} + \dfrac{2}{2^2} + \dfrac{3}{2^3} + \cdots + \dfrac{\left\lceil{\log_2 n}\right\rceil}{2^{\left\lceil{\log_2 n}\right\rceil}} \\
, \ 2s - s = s= \dfrac{1}{2} + \dfrac{1}{2^2} + \dfrac{1}{2^3} + \cdots + \dfrac{1}{2^{\left\lceil{\log_2 n}\right\rceil}}
\end{equation}


因为 s 是小于 $1$ 的,所以 $\sum^{\left\lceil{\log_2 n}\right\rceil} _ {h = 2} 2^h < 1$,所以时间复杂度为 $\mathcal{O}(n)$.因此堆排序的时间复杂度为 $\mathcal{O}(n \log_2 n)$.

证毕.

down 操作在堆\upref{heap}这篇文章讲了,这里不再赘述.

 由于堆排序并没有设计到随机删除或修改,所以直接交换堆中的元素即可,不需要使用复杂度特殊的堆交换方式.

数组模拟堆排序:
\begin{lstlisting}[language=cpp]
const int N = 1e5 + 10;
int n, cnt;
int h[N];

void down(int u)
{
    int t = u;
    if ((u << 1) <= cnt && h[u << 1] < h[t]) t = u << 1;
    if ((u << 1 | 1) <= cnt && h[u << 1 | 1] < h[t]) t = u << 1 | 1;
    if (u != t)
    {
        swap(h[u], h[t]);
        down(t);
    }
}

int main()
{
    cin >> n;
    for (int i = 1; i <= n; i ++ ) cin >> h[i];
    cnt = n;
    
    for (int i = n / 2; i; i -- ) down(i);
    
    while (n -- )
    {
        cout << h[1] << ' ';
        h[1] = h[cnt -- ];
        down(1);
    }
    
    return 0;
}
\end{lstlisting}

由于堆排序并没有涉及到随机删除或修改,因此可以用 C++ STL 库里的优先队列.

\begin{lstlisting}[language=cpp]
int main()
{
    cin >> n;
    priority_queue<int, vector<int>, greater<int>> heap;  // 定义小根堆方式
    
    for (int i = 0; i < n; i ++ )
    {
        int x;
        cin >> x;
        heap.push(x);
    }
    
    for (int i = 0; i < n; i ++ )
    {
        cout << heap.top() << ' ';  // 堆顶就是最小值
        heap.pop();  // 删除堆顶
    }
    
    return 0;
}
\end{lstlisting}