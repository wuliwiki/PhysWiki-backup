% 泡利矩阵(综述)
% license CCBYSA3
% type Sum

本文根据 CC-BY-SA 协议转载翻译自维基百科\href{https://en.wikipedia.org/wiki/Pauli_matrices}{相关文章}。

在数学物理和数学中,泡利矩阵是一组三个$2\times2$ 的复矩阵,它们具有迹为零、厄米特、自反和酉的性质。这些矩阵通常用希腊字母$\sigma$(σ)表示,在涉及同位旋对称性时,有时也用$\tau$(τ)表示。
$$
\sigma_1 = \sigma_x = 
\begin{pmatrix}
0 & 1 \\
1 & 0
\end{pmatrix},
\quad
\sigma_2 = \sigma_y = 
\begin{pmatrix}
0 & -i \\
i & 0
\end{pmatrix},
\quad
\sigma_3 = \sigma_z = 
\begin{pmatrix}
1 & 0 \\
0 & -1
\end{pmatrix}.~
$$
这些矩阵因物理学家 沃尔夫冈·泡利而得名。在量子力学中,它们出现在泡利方程中,用于描述粒子的自旋与外部电磁场相互作用的情况。它们还可以用来表示两种偏振滤光片的相互作用状态,例如水平/垂直偏振、45 度偏振(左/右)以及圆偏振(左/右)的状态。
