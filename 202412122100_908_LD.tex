% 零点能量(综述)
% license CCBYSA3
% type Wiki

本文根据 CC-BY-SA 协议转载翻译自维基百科\href{https://en.wikipedia.org/wiki/Zero-point_energy}{相关文章}。

\textbf{零点能(ZPE)}是量子力学系统可能具有的最低能量。与经典力学不同,量子系统即使在最低能量状态下也会不断波动,这可以通过海森堡不确定性原理来描述[1]。因此,即使在绝对零度下,原子和分子也会保持某些振动运动。除了原子和分子外,真空的空旷空间也具有这些性质。根据量子场论,宇宙可以被看作不仅是孤立的粒子,而是连续波动的场:物质场,其量子是费米子(即轻子和夸克),以及力场,其量子是玻色子(例如光子和胶子)。所有这些场都有零点能[2]。这些波动的零点场导致了一种在物理学中重新引入以太的现象[1][3],因为某些系统可以探测到这种能量的存在[需要引用]。然而,如果这个以太要保持洛伦兹不变性,以保证与爱因斯坦的相对论没有矛盾,那么它就不能被视为一种物理介质[1]。

零点能的概念对宇宙学也非常重要,然而,物理学目前缺乏一个完整的理论模型来理解宇宙学中的零点能;特别是理论上与观测到的宇宙真空能量之间的差异,成为了一个重大争议问题[4]。然而,根据爱因斯坦的广义相对论,任何这种能量都会引起引力,而来自宇宙膨胀、暗能量和Casimir效应的实验证据表明,任何这种能量都极其微弱。一个试图解决这一问题的提案是认为费米子场具有负的零点能,而玻色子场具有正的零点能,因此这些能量会以某种方式相互抵消[5][6]。如果超对称是自然界的精确对称性,这个想法是成立的;然而,欧洲核子研究中心的大型强子对撞机至今未找到支持这一理论的证据。此外,已知如果超对称是有效的,它最多也只是一个破缺的对称性,仅在极高的能量下才成立,目前没有人能够展示一个低能宇宙中发生零点能抵消的理论[6]。这一差异被称为宇宙学常数问题,是物理学中最大的未解之谜之一。许多物理学家认为,“真空是理解自然的关键”[7]。