% 公理化集合论
% keys 集合|公理|公理系统
% license Usr
% type Tutor
\begin{issues}
\issueTODO
\end{issues}

\pentry{集合\upref{Set}}
\subsection{产生原因}
通常来说,我们都采取朴素集合论的观点,认为集合是一个最基本的数学概念,不需要严格的定义。但是,\textbf{罗素悖论(antinomy of Russell)}使得这一观念受到挑战。罗素悖论可以叙述为:已知对于任何一个集合\footnote{这里把集合也看成元素。},都能判定自己是否属于这个集合。所以,我们可以将不属于自己的集合收集起来构成一个集合。因此就应该存在一个集合 $\mathcal{A}=\{A|A\notin A\}$,其中 $A$ 是一个集合。但是,如果认为 $\mathcal{A}\in\mathcal{A}$,那么根据定义,$\mathcal{A}\notin\mathcal{A}$;反过来,如果认为 $\mathcal{A}\notin\mathcal{A}$,根据定义又有 $\mathcal{A}\in\mathcal{A}$。这就构成一个悖论。

罗素悖论有许多通俗版本:
\begin{example}{理发师悖论}
一个小镇里的理发师乐于助人,他说:“我会给所有不给自己理发的人理发。”

请问:理发师是否可以给自己理发?
\end{example}

罗素悖论的存在使人们认识到,朴素集合论并不像他们想象中的那么严谨。为了解决罗素悖论,人们利用公理对集合进行一系列限制,从而使得在新的公理系统中无法构造出罗素悖论中的集合。

\subsection{ZF公理系统}
ZF 公理系统中认为所有的元素都可以看作是集合。区分它们可以用一个简单的方法,看“$\in$” “$\notin$”符号,左边应看作元素,右边应看作集合。

下面是 ZF 公理系统的全部公理\footnote{由于 ZF 公理系统认定所有的元素都是集合,所以下面存在一些容易引起混淆的地方,比如集族也是集合。}:

\textbf{公理 1(外延公理)} 一个集合完全由其元素决定。如果两集合所有元素相等,则这两个集合相等。表述为:给定任意集合 $A$ 和集合 $B$,$A=B$ 当且仅当:给定任意集合中的元素 $x$,$x \in A$ 当且仅当 $x \in B$。这又被成为容积公理。

\textbf{公理 2(无序对公理)} 对于任意两个集合 $x,y$,存在一个集合 $A$,使得对于任意 $w\in A$,$w=x$ 或 $w=y$。这又被成为配对公理。

\textbf{公理 3(并集公理)}对于任意一个集合 $A$,存在一个集合 $B$ 使 $x\in B$ 当且仅当存在一个集合 $y\in A$ 使 $x\in y$。

\textbf{公理 4(幂集公理)} 对任意集合 $x$,存在集合 $y$,使 $z\in y$ 当且仅当对 $z$ 的所有元素 $w$,$w\in x$。

\textbf{公理 5(无穷公理)}存在一个集合 $\mathbb{N}$,使得 $\varnothing\in\mathbb{N}$,且对任意 $x\in \mathbb{N}$,$x\cup\{x\}\in\mathbb{N}$。

把这个集合命名为 $\mathbb{N}$ 是因为根据皮亚诺公理,它就是自然数集合。

\textbf{公理 6(正则公理)}对于任意非空集合 $A$,存在一个 $x\in A$ 使 $x\cap A=\varnothing$。

其中$\varnothing$的存在性可以利用下面的公理证明。

接下来的公理实际上是无数条公理的结合\footnote{一阶逻辑禁止量化谓词,所以只能算是无数条公理。},被称为公理模式:

\textbf{公理 7(分离公理模式)}对任意集合 $A$ 和任意对 $A$ 的元素有定义的逻辑谓词 $P(z)$,存在集合 $B$ ,使 $z\in B$ 当且仅当 $z\in A$ 而且 $P(z)$ 为真。

也就是说,我们可以构造一个集合$\{z\in A | P(z)\}$。

\textbf{公理 8(替换公理模式)}对任意集合 $A$ 和任意对$A$的元素有定义的(逻辑)公式$F(z)$,存在集合 $B$,使 $y\in B$ 当且仅当存在 $z\in A$ 而且 $F(z)=y$. 这又被称为置换公理模式。

\subsubsection{罗素悖论的解决}
\begin{theorem}{}
在ZF公\label{the_SetAxi_1}理系统中,不存在所有集合的集合。
\end{theorem}

本定理是正则公理的直接推论。

证明:假如这个集合存在,那么利用分离公理模式,我们可以构造所有非空集合的集合,设这个集合为 $A$ ,那么对任意 $x\in A$ , 由于存在集合$y\in x$,故至少有$y\in x\cap A$,也就是说$x\cap A\neq\varnothing$。这与正则公理矛盾。证毕。

\begin{lemma}{}\label{lem_SetAxi_1}
不存在一个集合$A$,使得它的唯一元素为$A$本身。
\end{lemma}

留作习题。

\begin{theorem}{}\label{the_SetAxi_2}
对于任何集合$A$,$A\notin A$。
\end{theorem}

证明:如不然,利用分离公理模式可以构造一个 \autoref{lem_SetAxi_1} 所述的集合。

\begin{theorem}{}
不存在集合$\{A|A\notin A\}$.
\end{theorem}

由 \autoref{the_SetAxi_1} 和 \autoref{the_SetAxi_2} 可得。

\subsection{选择公理和ZFC公理系统}

\subsection{连续统假设}



