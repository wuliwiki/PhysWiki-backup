% 东南大学 2015 年 考研 量子力学
% license Usr
% type Note

\textbf{声明}:“该内容来源于网络公开资料,不保证真实性,如有侵权请联系管理员”

\textbf{1.}以下对称性是否导致一个守恒量,如果是,请指出相应的守恒量(1)空间反演对称性:(2)空间平移对称性;(3)空间转动对称性;(4)时间反演对称性;(5)时间平移对称性.

\textbf{2.判断题}
\begin{enumerate}
\item 一维线性谐振子的是子态空间是无穷维的
\item 全同 bose子系统的波函数具有交换反对称性
\item Hermie 算符$\hat{A}$与$\hat{B}$不对易, 则$\hat{A}$与$\hat{B}$一定无共同本征态
\item 三维各向同性谐振子的所有能级均是简并的.
\item 中心力场无自旋的角动量一定是守恒量
\end{enumerate}

\textbf{3.}证明 Bloch 函数

\[
\psi_k(r) = \exp(ik \cdot r)\phi_k(r), \quad \phi_k(r) = \phi_k(r + a),
\]
是平移算符 $\hat{D}(a) = \exp\left(-\frac{ia \cdot \hat{p}}{\hbar}\right)$ 的本征态。相应的本征值为 $\exp(-ik \cdot a)$。


\textbf{4.}

\textbf{5.}

\textbf{6.}

\textbf{7.}

\textbf{8.}