% 线性变换与矩阵的代数关系
% keys 线性变换|矩阵|代数同构

在本页中,我们将所有 $m$ 行 $n$ 列的实矩阵组成的集合记为 $\mathbb{R}^{m\times n}.$

并记从 $\mathbb{R}^{n}$ 到 $\mathbb{R}^{m}$ 的一切线性变换全体构成的空间为 $L(\mathbb{R}^{n},\mathbb{R}^{m}).$
\begin{itemize}
\item $\mathbb{R}^{m\times n}$ 中的矩阵与 $L(\mathbb{R}^{n},\mathbb{R}^{m})$
中的线性变换是代数同构的关系. 矩阵可以看成线性变换, 线性变换在取定一组基底后, 可以表示为矩阵. 
\end{itemize}

为明确起见,本页所说的线性变换都是指有限维空间之间的线性变换,即 $L(\mathbb{R}^{n},\mathbb{R}^{m})$ 中的映射.

\subsection{1.矩阵的秩与线性变换的联系}

\begin{definition}{(矩阵的秩)}
\textbf{矩阵的秩} 既可以定义为其行向量组的秩 (称为 \textbf{行秩}), 也可以定义为其列向量组的秩 (称为 \textbf{列秩}),
这是因为矩阵的行秩和列秩被证明是相等的. \\

\textsl{注}:矩阵行向量组形成的空间称为 \textbf{行空间}, 行向量组的秩也正是行空间的维数;
列向量组形成的空间称为 \textbf{列空间}, 列向量组的秩也正是列空间的维数.
\end{definition}



\begin{definition}{(线性变换的秩)}
\textbf{线性变换的秩} 定义为其像空间的维数.
\end{definition} 
\verb| |

下面的定理表明,矩阵与线性变换的秩在某种意义下是等同的.
\begin{theorem}{(像空间 $=$ 列空间)}\label{linmat_the1}
 如果将矩阵看成线性变换, 那么该变换的像空间的维数, 恰是矩阵的列空间的维数.\\

\textsl{ 证明}:设 $A\in\mathbb{R}^{m\times n}$, $A$ 也可以看成从 $\mathbb{R}^{n}$ 到 $\mathbb{R}^{m}$
的线性变换, 那么取定 $\mathbb{R}^{n}$ 的标准基 $(e_{1},e_{2},\ldots,e_{n})$ 后,
$A$ 的像空间就由 $\{Ae_{1},Ae_{2},\ldots,Ae_{n}\}$ 张成, 因此像空间的维数就是 $\{Ae_{1},Ae_{2},\ldots,Ae_{n}\}$
的极大无关组的个数, 也就是 $\{Ae_{1},Ae_{2},\ldots,Ae_{n}\}$ 的秩; 而每个 $Ae_{i}$
正好又是矩阵 $A$ 的第 $i$ 列, 这就是说明了 $A$ 的像空间的维数等于它的列空间的维数. 
\end{theorem}

\textsl{注}:上述证明过程表明,对于看成线性变换的矩阵来说,其像空间正好就是其列空间. 

\begin{itemize}
\item 矩阵是行满秩的, 当且仅当其线性变换是满射. (\textbf{行秩等于列秩,列空间等于像空间})
\end{itemize}

\begin{itemize}
\item 矩阵是列满秩的, 当且仅当其线性变换是单射. (\textbf{列空间等于像空间,维数定理,核空间为零})
\end{itemize}

\begin{itemize}
\item 方阵是满秩的, 当且仅当它是可逆的, 当且仅当其线性变换是双射. \end{itemize}
\verb| |

在介绍线性方程组的解的结构性定理时, 很多书是采用初等行变换来证明的, 这显然是有利于初学者接受的, 但是\textsl{不够优雅}. 这里我们仅需巧妙地运用 \autoref{linmat_the1} 就能导出线性方程组的解的结构,
并且能对齐次和非齐次线性方程组做统一的处理. 

\begin{theorem}{(线性方程组的解的结构性定理)}\label{linmat_the2}
设 $A\in\mathbb{R}^{m\times n},$ $b\in\mathbb{R}^{n\times1}$ 且 $b\neq0$.

① $Ax=0$ 有唯一解 (即零解) $\Leftrightarrow$ $\ker A=\{0\}$ $\Leftrightarrow$
$A$ 是单射 $\Leftrightarrow$ $A$ 列满秩 $\Leftrightarrow$ $\mathrm{rank}$
$A=n.$ 

② $Ax=0$ 有无穷个解 (即有非零解) $\Leftrightarrow$ $A$ 不是单射 $\Leftrightarrow$
$A$ 列不满秩 $\Leftrightarrow\mathrm{rank}$ $A<n.$ 

③ $Ax=b$ 无解 $\Leftrightarrow$ $b\notin\mathrm{Im}A$ $\Leftrightarrow$
$b\notin$\{$A$ 的列空间\} $\Leftrightarrow$ $b$ 无法被 $A$ 的列向量组线性表出
$\Leftrightarrow$ $\mathrm{rank}$ $A<$ $\mathrm{rank}$ $(A:b).$

④ $Ax=b$ 有解 $\Leftrightarrow$ $b\in\mathrm{Im}A$ $\Leftrightarrow$
$b\in$\{$A$ 的列空间\} $\Leftrightarrow$ $b$ 可以被 $A$ 的列向量组线性表出 $\Leftrightarrow$
$\mathrm{rank}$ $A=$ $\mathrm{rank}$ $(A:b).$

⑤ $Ax=b$ 有唯一解 $\Leftrightarrow$ $b\in\mathrm{Im}A$ 且 $A$ 是单射 $\Leftrightarrow$
$\mathrm{rank}$ $A=$ $\mathrm{rank}$ $(A:b)=n$ (结合 ①、④)

⑥ $Ax=b$ 有无穷多解 $\Leftrightarrow$ $b\in\mathrm{Im}A$ 且 $A$ 不是单射
$\Leftrightarrow$ $\mathrm{rank}$ $A=$ $\mathrm{rank}$ $(A:b)<n$
(结合 ②、④)
\end{theorem}

\textsl{注}:以上结果展示了将矩阵和线性变换视作 \textbf{代数同构} 的强大作用.

\verb| |

\begin{theorem}{}
若 $A\in\mathbb{R}^{n\times m}$, $B\in\mathbb{R}^{m\times n}$, 且 $AB=I_n$, 则 $A$作为线性算子是满射,$B$作为线性算子是单射.

\textsl{证明}:先证 $A$ 是满射. 由已知得 $B\in L(\mathbb{R}^{n},\mathbb{R}^{m})$, $A\in L(\mathbb{R}^{m},\mathbb{R}^{n})$.
若 $A$ 不满, $\mathrm{Im}A$ 真包含于 $\mathbb{R}^{n}$ 中, 但是 $AB=I_{n}$
说明 $\dim(\mathrm{Im}A)=n$, 矛盾. 

再证 $B$ 是单射. 若 $Bx=0$, 则 $x=I_{n}x=(AB)x=A(Bx)=A0=0$, 说明 $\mathrm{Ker}B=\{0\}$.
证毕. 
\end{theorem}

上面的定理有如下两个推论:

\begin{itemize}
\item 矩阵有右逆, 当且仅当其线性变换是满射. \end{itemize}

\begin{itemize}
\item 矩阵有左逆, 当且仅当其线性变换是单射. 
\end{itemize}




\subsection{2.矩阵与线性变换的范数}
线性空间上的范数是指其上的一个满足 \textbf{非负正定性}、\textbf{正齐次性}、\textbf{三角不等式} 的实值函数, 记作 $\left\Vert \cdot\right\Vert ~~$ 
(参见《范数、赋范空间》的 \autoref{NormV_def1}~\upref{NormV}). 

\begin{definition}{(线性变换的范数)}
设 $A\in L(\mathbb{R}^{n},\mathbb{R}^{m})$, 定义
\[
\left\Vert A\right\Vert =\max_{|x|=1}|A(x)|=\max_{x\neq0}{\displaystyle \frac{|A(x)|}{x}}.
\]
\end{definition}
\textsl{注}:可以证明 $L(\mathbb{R}^{n},\mathbb{R}^{m})$ 中的这个范数不仅满足非负正定性、正齐次性、三角不等式,
还满足

$(4)$ 向量范数相容性: $\left\Vert A(x)\right\Vert \leqslant\left\Vert A\right\Vert \left|x\right|$;

$(5)$ 复合运算相容性: $\left\Vert B\circ A\right\Vert \leqslant\left\Vert B\right\Vert \left\Vert A\right\Vert $.



\begin{definition}{(矩阵的范数)}\label{linmat_def1}
设 $A\in\mathbb{R}^{m\times n}$, 定义

\[
\left\Vert A\right\Vert =\max_{|x|=1}|Ax|=\max_{x\neq0}{\displaystyle \frac{|Ax|}{x}}.
\]

\end{definition}
\textsl{注}:可以证明矩阵的这个范数不仅满足非负正定性、正齐次性、三角不等式, 还满足

$(4)$ 向量范数相容性: $\left\Vert Ax\right\Vert \leqslant\left\Vert A\right\Vert \left|x\right|$;

$(5)$ 乘法运算相容性: $\left\Vert BA\right\Vert \leqslant\left\Vert B\right\Vert \left\Vert A\right\Vert $.



\begin{theorem}{}
矩阵 $A$ 的范数 $\left\Vert A\right\Vert $ 可由以下公式计算: 
\[
\left\Vert A\right\Vert =\sqrt{\mu_{1}},
\]
 其中 $\mu_{1}$ 为矩阵 $A^{T}A$ 的最大特征值. 
\end{theorem}

\textsl{注}:矩阵按 \autoref{linmat_def1} 所定义的范数称为 \textbf{矩阵的标准范数}. 除此之外, 矩阵 $A\in\mathbb{R}^{m\times n}$ 还常用以下范数:
矩阵按 ({*}) 定义的范数称为矩阵的标准范数. 除此之外, 矩阵 $A\in\mathbb{R}^{m\times n}$ 常用以下范数:
\[
\left\Vert A\right\Vert _{1}=\max_{1\leqslant j\leqslant m}\{\sum_{i=1}^{n}|a_{ij}|\},\quad\left\Vert A\right\Vert _{\infty}=\max_{1\leqslant i\leqslant n}\{\sum_{j=1}^{n}|a_{ij}|\},\quad\left\Vert A\right\Vert _{F}=\left(\sum_{i,j=1}^{n}|a_{ij}|^{2}\right)^{\frac{1}{2}}
\]
分别称为矩阵的 \textbf{列和范数}、\textbf{行和范数}、\textbf{Frobenius范数} (简称 \textbf{F范数}). 

 可以证明, 矩阵的范数自然会满足 (\textsl{也应该满足}) 非负正定性、正齐次性、三角不等式, 但是却不一定要满足 $(4)$和$(5)$. 比如,矩阵的列和范数与行和范数都不满足 $(4)$, 但满足$(5)$; 而矩阵的 F范数则既满足 $(4)$, 也满足 $(5)$. 



\begin{theorem}{}
任意两个矩阵的范数是相互等价的. 

\textsl{注}:等价范数的定义参见《范数、赋范空间》一节\upref{NormV}.
\end{theorem}
\verb| |

\begin{theorem}{}
设 $A$ 是 $n\times n$ 方阵, 且 $\left\Vert A\right\Vert <1$, 则矩阵 $I-A$
可逆, 且 
\[
I-A=\sum_{k=0}^{\infty}A^{k},
\]
即 ${\displaystyle \lim_{k\rightarrow\infty}}\left\Vert I-A-{\displaystyle \sum_{i=0}^{k}A^{i}}\right\Vert =0.$
并且此时成立不等式
\[
\left\Vert (I-A)^{-1}\right\Vert \leqslant{\displaystyle \frac{1}{1-\left\Vert A\right\Vert }.}
\]
\end{theorem}
\textsl{注}:这在形式上与等比级数非常像,比如
\[
{\displaystyle \frac{1}{1-x}=1+x+x^{2}+\cdots+x^{k}+\cdots=\sum_{k=0}^{\infty}A^{k},\quad\quad|x|<1}\]


\begin{corollary}{}
可逆矩阵的小扰动仍是可逆矩阵;等价地, 可逆线性变换的小扰动仍是可逆线性变换.
\end{corollary}

\textsl{注}: 若矩阵 $A\in\mathbb{R}^{n\times n}$ 可逆, 取矩阵 $P\in\mathbb{R}^{n\times n}$
使其范数小于 $\frac{1}{\left\Vert A^{-1}\right\Vert }$, 则
\[
\left\Vert A^{-1}P\right\Vert \leqslant\left\Vert A^{-1}\right\Vert \left\Vert P\right\Vert <\left\Vert A^{-1}\right\Vert \cdot{\displaystyle \frac{1}{\left\Vert A^{-1}\right\Vert }=1,}
\]
那么根据定理 1 知 $I+A^{-1}P$ 可逆, 从而
\[
A+P=A(I+A^{-1}P)
\]
也可逆. 

\begin{corollary}{}
可逆矩阵函数在连续点附近仍是可逆的.
\end{corollary} 

(by R.Stern)