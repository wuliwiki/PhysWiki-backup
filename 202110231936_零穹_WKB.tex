% WKB 近似

\begin{issues}
\issueDraft
\end{issues}
在某些定态问题中,WKB近似方法可以比较容易地求解薛定谔方程.该方法基于将波函数按 $\hbar$ 作幂级数展开,就其本身而言,有两个基本问题:1.在远离转折点处的近似解;2.在转折点处的连接条件.并通过这两个问题求解薛定谔方程.

经典区域
\begin{equation}
\psi(x) \approx \frac{C}{\sqrt{p(x)}} \exp(\pm \I \int p(x) \dd{x})
\end{equation}

隧道区域
\begin{equation}
\psi(x) \approx \frac{C}{\sqrt{\abs{p(x)}}} \exp(\pm \int p(x) \dd{x})
\end{equation}

在二者的转折点, 假设势能为线性函数, 其解是艾里函数 $\opn{Ai}$, 详见 “线性势能的定态薛定谔方程\upref{LinPot}”.

从经典到非经典区域
\begin{equation}
\psi(x) = \leftgroup{
&\frac{B}{\sqrt{p(x)}} \exp(\I \int_{-x}^{x_0} p(x')\dd{x'}) + \frac{C}{\sqrt{p(x)}} \exp(-\I \int_{-x}^{x_0} p(x')\dd{x'}) \qquad &(x < x_0)\\
&\frac{D}{\sqrt{\abs{p(x)}}} \exp(-\int_{x_0}^x \abs{p(x')} \dd{x'})  \qquad &(x > x_0)
}\end{equation}
衔接以后
\begin{equation}
\psi(x) = \leftgroup{
&\frac{2D}{\sqrt{p(x)}} \sin(\int_{x}^{x_0} p(x')\dd{x'} + \frac{\pi}{4}) \quad &(x < x_0)\\
&\frac{D}{\sqrt{\abs{p(x)}}} \exp(-\int_{x_0}^x \abs{p(x')} \dd{x'}) \quad &(x > x_0)
}\end{equation}
\subsection{证明}
\subsubsection{WKB近似解}
薛定谔方程
\begin{equation}\label{WKB_eq2}
\I\hbar\pdv{\psi}{t}=-\frac{\hbar^2}{2m}\Delta\psi+V(\bvec{r})\psi
\end{equation}
的解一般总能写成如下形式
\begin{equation}\label{WKB_eq1}
\psi(\bvec r,t)=A\E^{\I W(\bvec r,t)/\hbar}
\end{equation}
\autoref{WKB_eq1} 代入\autoref{WKB_eq2} 得到 $W$ 满足的方程
\begin{equation}\label{WKB_eq3}
\pdv{W}{t}+\frac{1}{2m}(\nabla W)^2+V-\frac{\I\hbar}{2m}\Delta W=0
\end{equation}
在经典极限($\hbar\rightarrow 0$)下,\autoref{WKB_eq3} 等同 $W$ (称为主函数)的哈密顿方程
\begin{equation}
\pdv{W}{t}+\frac{1}{2m}(\nabla W)^2+V=0
\end{equation}
如果 $\psi$ 是能量本征函数 $u(\bvec E)\E^{-\I Et/\hbar}$,则 $W$ 可写成
\begin{equation}
W(\bvec r,t)=S(\bvec r)-Et
\end{equation}
在这种情况下,我们有
\begin{equation}
u(\bvec r)=A\E^{\I S(\bvec r)}
\end{equation}
且
\begin{equation}
\frac{1}{2m}(\nabla S)^2-\qty[E-V(\bvec r)]-\frac{\I\hbar}{2m}\Delta S=0
\end{equation}

在一维情况下,WKB 方法得到 $S$ 的按 $\hbar$ 幂展开式的头两项可以清楚的给出.下面以一维定态为例.

一维薛定谔方程可写为
\begin{equation}\label{WKB_eq4}
\begin{aligned}
&\dv[2]{u}{x}+\qty[k(x)]^2u=0\\
&k(x)\equiv\left\{\begin{aligned}
&\qty[\frac{2m}{\hbar^2}\qty(E-V(x))]^{1/2},\quad \text{when}\; E>V(x)\\
&-\I\kappa(x)\equiv-\I\qty[\frac{2m}{\hbar^2}(V(x)-E)]^{1/2},\quad \text{when}\; E<V(x)
\end{aligned}\right.
\end{aligned}
\end{equation}
我们寻求\autoref{WKB_eq4} 如下形式的解
\begin{equation}
u(x)=A\E^{\I S(x)/\hbar}
\end{equation}
将上式代入\autoref{WKB_eq4} ,得
\begin{equation}\label{WKB_eq6}
\I\hbar S''-S'^2+\hbar^2\qty[k(x)]^2=0
\end{equation}
其中,撇号表示对 $x$ 求导.将 $S(x)$ 按 $\hbar$ 之幂作级数展开
\begin{equation}\label{WKB_eq5}
S(x)=\sum_{n=0}\hbar^nS_n(x)
\end{equation}
\autoref{WKB_eq5} 代入\autoref{WKB_eq6} 
\begin{equation}
\sum_{n=0}\qty(\I S_n''-\sum_{i=0}^{n+1}S_{n+1-i}S_i)\hbar^{n+1}+\hbar^2\qty[k(x)]^2
\end{equation}


\addTODO{Griffiths 例题 Potential well with one vertical wall.}
