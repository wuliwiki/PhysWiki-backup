% 世界坐标系
% keys 世界坐标系 全局坐标系


一个三维场景中通常都不会只有一个物体.我们真正需要的是把我们建立的物体按照我们所需要的形式摆放在场景之中.每个物体分布在场景的适当的位置上.整个场景的坐标系就称为\textbf{世界坐标系(world coordinate, world frame)}.

从建模坐标系到世界坐标系有一个坐标变换,即建模变换(modeling transformation)或称模型矩阵(model matrix).建模变换通过平移(translate)、旋转(rotate)和缩放(scale)将物体摆放在场景中的适当位置上.

世界坐标系最常使用的是标准三维笛卡尔坐标系.笛卡尔坐标系是正交坐标系.其中三条坐标轴两两垂直,相交于一点.图形学软件包OpenGL通常是采用右手笛卡尔坐标系(详见右手定则\upref{RHRul}).


\begin{figure}[ht]
\centering
\includegraphics[width=5cm]{./figures/Worcod_1.png}
\caption{右手坐标系} \label{Worcod_fig1}
\end{figure}


参考文献:
\begin{enumerate}
\item Donald Hearn, Pauline Baker, Carithers著, 蔡士杰, 杨若瑜译. 计算机图形学[M]. 电子工业出版社. 2014
\item \href{https://learnopengl.com/Getting-started/Coordinate-Systems}{https://learnopengl.com/Getting-started/Coordinate-Systems}
\item \href{https://learnopengl-cn.readthedocs.io/zh/latest/01\%20Getting\%20started/08\%20Coordinate\%20Systems/}{https://learnopengl-cn.readthedocs.io/zh/latest/01\%20Getting\%20started/08\%20Coordinate\%20Systems/}
\end{enumerate}