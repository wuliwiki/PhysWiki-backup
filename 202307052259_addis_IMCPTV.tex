% 理想混合物的热力学量


\pentry{相简介(热力学)\upref{PHS}, 理想气体混合的熵变\upref{IGME}}
\begin{figure}[ht]
\centering
\includegraphics[width=8 cm]{./figures/46ac3ca623e0be6b.pdf}
\caption{一个多元相是各组分的混合物} \label{fig_IMCPTV_1}
\end{figure}

\footnote{本文参考了刘俊吉等人的《物理化学》,采用 本文采用 CC-BY 协议发布。}一个多元相仅仅是各种纯组分的简单堆砌吗?非也,在多组分形成一个均匀的相时,这些组分将需要被\textsl{混合}。因此,一个多元相是多组分的混合物。

实际的混合是一个复杂的过程,往往伴随着熵的改变(我们已经在理想气体混合\upref{IGME}中探讨了一些)、体积的改变、能量的改变等等等等。

由于实际的混合过程过于复杂,我们先裁弯取直,探讨一种最简单的理想混合模型:
\begin{itemize}
\item 混合后没有能量变化
\item 混合后没有体积变化
\item 混合后熵的变化是理想的
\end{itemize}
尽管很少有实际混合物遵循这几个假设,但这个足够简单的模型有助于我们初步理解混合物的热力学性质。

\subsubsection{内能、体积、焓}
由于混合前后没有能量变化,因此混合物的总内能是各组分内能的和:
$$
U = \sum_i n_i u_i^*~,
$$
其中$u_i^*$是各组分以纯物质形式存在时的摩尔内能。

同理,由于没有体积变化,因此混合物的总体积是各组分体积的和:
$$
V = \sum_i n_i v_i^*~.
$$
根据焓的定义,
$$
\begin{aligned}
H=U+pV&=\sum_i n_i u_i^*+p\sum_i n_i v_i^*\\
&=\sum_i n_i (u_i^*+p v_i^*)\\
&=\sum_i n_i h_i^*~,\\
\end{aligned}
$$
其中 $h^*_i = u_i^* + p v_i^*$ 是各组分纯物质的摩尔焓。

根据理想混合的假设,这三个量都只是各组分纯物质时的摩尔量的简单加和。

\subsubsection{熵}
熵会复杂一些。混合物的熵不仅是各组分“固有”熵之和,还要再加上混合熵。在理想混合的情况下,混合熵的形式恰好与理想气体的相同\upref{IGME} 。%为什么?可以补充一个论证过程...不过我觉得这样的论证会相当繁琐...
$$S = \sum_i n_i s_i^* + \Delta_{mix} S = \sum_i n_i s_i^* - R \sum_i n_i \ln (x_i)~. $$

\subsubsection{Gibbs 自由能}

\begin{figure}[ht]
\centering
\includegraphics[width=8cm]{./figures/8ee3645fb00dc120.pdf}
\caption{理想混合后系统的Gibbs自由能比未混合的低。$G_{unmixed}=\sum_i n_i \mu_i^*$,$G_{mixed}= \sum_i n_i \mu_i$。假设系统中有两种组分,且总$\Si{mol}$数是$N=\sum_i n_i =1 \Si{mol}$。} \label{fig_IMCPTV_2}
\end{figure}
由于Gibbs自由能\upref{GibbsG} 与熵有关,因此Gibbs能的表达也会复杂一些。直接根据Gibbs能的定义:
$$
\begin{aligned}
G = H-TS &= \sum_i n_i h_i^* - T \sum_i n_i s_i^* + RT \sum_i n_i \ln (x_i)\\ 
&= \sum_i n_i (h_i^* - T s_i^*) + RT \sum_i n_i \ln (x_i)\\
& = \sum_i n_i \mu_i^* + RT \sum_i n_i \ln (x_i)\\
&=\sum_i n_i (\mu_i^* + RT \ln (x_i))\\
&= \sum_i n_i \mu_i~.\\
\end{aligned}
$$
另一种说法是,混合物的Gibbs能等于未混合物的Gibbs能加上混合Gibbs变。其中未混合物的Gibbs能相当于各组分“固有”Gibbs能的简单加和,而混合Gibbs变则有关混合熵。无论如何,最终结果是相同的:
$$
\begin{aligned}
G_{unmixed}&=\sum_i n_i \mu_i^*\\
G_{mixed} &= G_{unmixed} + \Delta_{mix} G\\ 
&=G_{unmixed} -  T \Delta_{mix} S\\
& = \sum_i n_i \mu_i^* + RT \sum_i n_i \ln (x_i)\\
&= \sum_i n_i \mu_i~.\\
\end{aligned}
$$

其中
$$
\mu_i = \mu_i^* + RT \ln (x_i)~,
$$
反映了混合物中各组分的化学势$\mu_i$与相应纯物质化学势$\mu_i^*$的关系。%化学势需要建立一个更好的词条

虽然我们此处得到的化学势 $\mu_i = \mu_i^* + RT \ln (x_i)$ 似乎只与这第$i$种物质有关,但别忘了,根据摩尔分数的定义 $x_i = \frac{n_i}{n_1+n_2+...}$,即使$n_i$不变,其余物质的物质的量改变也会导致$x_i$的变化。因此,某物质的化学势与各物质的物质的量(分数)均有关。这也回应了之前我们对化学势的论述。
