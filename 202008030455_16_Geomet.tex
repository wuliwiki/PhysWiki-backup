% 引力波的几何描述
引力的作用量分为爱因斯坦作用量和物质的作用量$S=S_E+S_M$其中
\begin{align}
S_E = \frac{c^3}{16\pi G} \int d^4 x \sqrt{-g} R~.
\end{align}
能量动量张量的定义如下
\begin{align}
\delta S_M = \frac{1}{2 c} \int d^4 x \sqrt{-g} T^{\mu\nu} \delta g_{\mu\nu} ~. 
\end{align}
对作用量求变分,我们得到了如下的爱因斯坦方程
\begin{align}
R_{\mu\nu} - \frac{1}{2} g_{\mu\nu} R = \frac{8\pi G}{c^4} T_{\mu\nu} ~. 
\end{align}
广义相对论在很大的对称群下都是不变的,比如下面的坐标变换
\begin{align}
x^\mu \rightarrow x'^\mu (x) ~. 
\end{align}
在上面的坐标变换下,度规按照如下规则变换
\begin{align}
g_{\mu\nu} (x) \rightarrow g'_{\mu\nu}  = \frac{\partial x^\rho}{\partial x'^\mu} \frac{\partial x^\sigma}{\partial x'^\nu} g_{\rho\sigma} (x) ~. 
\end{align}
我们把这种对称性叫做广义相对论的规范对称性。现在我们把度规按照如下的方式展开
\begin{equation}\label{Geomet_eq1}
g_{\mu\nu} = \eta_{\mu\nu} + h_{\mu\nu}~, \quad |h_{\mu\nu}| \ll 1 ~. 
\end{equation} 
考虑如下的平移坐标变换
\begin{align}
x^\mu \rightarrow x'^\mu = x^\mu + \xi^\mu (x) ~. 
\end{align}
在这样的坐标变换下,$h_{\mu\nu}$按照如下规则变换
\begin{align}
h_{\mu\nu} (x) \rightarrow h'_{\mu\nu} (x) - (\partial_\mu\xi_\nu+\partial_\nu \xi_\mu) ~.
\end{align}
现在考虑如下的洛伦兹变换
\begin{equation}
x^\mu \rightarrow \Lambda^\mu_\nu x^\nu ~.
\end{equation}
矩阵$\Lambda^\mu_\nu$满足
\begin{equation}
\Lambda^\rho_\mu\Lambda^\sigma_\nu \eta_{\rho\sigma} = \eta_{\mu\nu} ~. 
\end{equation}
在洛伦兹变换下,矩阵$g_{\mu\nu}$的变换如下
\begin{align}\nonumber
g_{\mu\nu} \rightarrow g'_{\mu\nu} (x') & = \Lambda^\rho_\mu \Lambda^\sigma_\mu g_{\rho\sigma} (x) \\\nonumber
& = \Lambda^\rho_\mu \Lambda^\sigma_\nu [\eta_{\rho\sigma} + h_{\rho\sigma}(x)] \\
& = \eta_{\mu\nu} + \Lambda^\rho_\mu \Lambda^\sigma_\nu h_{\rho\sigma} (x) ~.
\end{align}
因此,在洛伦兹变换下,$h_{\mu\nu}$的变换规则如下
\begin{equation}
h_{\mu\nu}'(x) = \Lambda^\rho_\mu\Lambda^\sigma_\nu h_{\rho\sigma} (x) ~. 
\end{equation}
这说明$h_{\mu\nu}$在洛伦兹变换下是张量。转动变换无论如何不会破坏$|h_{\mu\nu}|\ll 1$这个条件。但是对于boost我们必须只做总是使得$h_{\mu\nu}\ll 1$能够满足的boost的变换。这样的理论叫做线性理论。
另外,值得注意的有以下几点
\begin{itemize}
\item $h_{\mu\nu}$在有限的平移变换$x^\mu\rightarrow x'^\mu = x^\mu + a^\mu$和洛伦兹变换$x^\mu\rightarrow \Lambda^\mu_\nu x^\nu$下不变。
\item $h_{\mu\nu}$在无穷小的局域变换$x^\mu \rightarrow x'^\mu = x^\mu + \xi^\mu (x)$下不变。
\end{itemize}
与此形成鲜明对比的是,广义相对论没有庞加莱对称性。这是因为在广义相对论的框架下,平坦空间并没有什么特殊性,而庞加莱对称群是平坦空间里面的对称群。但是,广义相对论拥有全部的坐标变换的不变性,而不仅仅是无穷小的坐标变换的不变性。
在线性阶,黎曼张量是
\begin{equation}
R_{\mu\nu\rho\sigma} = \frac{1}{2}  (\partial_\nu\partial_\rho h_{\mu\sigma} + \partial_\mu\partial_\sigma h_{\nu\rho} - \partial_\mu\partial_\rho h_{\nu\sigma} - \partial_\nu\partial_\sigma h_{\mu\rho}  ) ~.
\end{equation} 
在我们感兴趣的物理情形下,总是存在一个参考系使得\autoref{Geomet_eq1}成立。我们定义如下的两个量 
\begin{align}
h = \eta^{\mu\nu} h_{\mu\nu} ~, \quad \bar h_{\mu\nu} = h_{\mu\nu} - \frac{1}{2} \eta_{\mu\nu} h~.
\end{align}
同样的,我们可以定义另外一个量
\begin{align}
\bar h \equiv \eta^{\mu\nu} \bar h_{\mu\nu} = h - 2 h = -h~.
\end{align}
我们可以反解出$h_{\mu\nu}$关于$\bar h_{\mu\nu}$的表达式。
\begin{equation}
h_{\mu\nu} = \bar h_{\mu\nu} - \frac{1}{2} \eta_{\mu\nu} \bar h ~. 
\end{equation}
线性化的爱因斯坦方程如下
\begin{equation}
\Box \bar h_{\mu\nu} + \eta_{\mu\nu} \partial^\rho \partial^\sigma \bar h_{\rho\sigma} - \partial^\rho \partial_\nu \bar h_{\mu\rho} - \partial^\rho\partial_\mu \bar h_{\nu\rho} = - \frac{16\pi G}{c^4} T_{\mu\nu} ~. 
\end{equation}








