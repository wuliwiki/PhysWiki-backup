% Make 简介
% Makefile | Linux | 自动编译 | make
\pentry{Linux 基础\upref{Linux}}

Make 通常被用于自动编译程序. 当程序由许多文件构成或使用许多其他的程序库时, 编译的命令往往会变得非常复杂, 如果每次编译都手动输入这些命令会非常麻烦且容易出错. 如果每次输入的编译命令都相同, 一个简单的解决方法是直接把这些命令写进一个 shell script, 每次编译调用即可.

但这样做会遇到另一个问题, 当程序文件(如 .cpp 文件)较多时, 如果在上次编译以后只修改了某些文件, 我们会希望只将这些文件重新编译(生成对应的 .o 文件)并再次 link, 这样可以大大节约编译时间.

Linux 系统中, \lstinline|make| 就是为了解决这个问题产生的. \lstinline|make| 并不是为某个编程语言而设计的, 也不会去分析任何文件的内容. 用户需要写一个名为 \lstinline|Makefile| 的文件告诉 \lstinline|make| 该怎么做. \lstinline|Makefile| 中主要描述了两件事情, 第一是每个文件需要依赖哪些文件才能开始生成, 第二就是每个文件需要用哪些命令生成. 下面举例说明, 为了方便, 这个例子只使用 Linux 常用命令而不涉及任何编程语言.


\subsection{简单的例子}
我们先在某个目录创建一个简单的 \lstinline|Makefile| 文件, 内容如下
\begin{lstlisting}
a :
	echo "file a" > a

b : a
	echo "text in a {" > b
	cat a >> b
	echo "}" >> b
\end{lstlisting}
这个 Makefile 指定了两个要生成的文本文件 \lstinline|a| 和 \lstinline|b|. 先来看含有冒号的两行, 冒号表示依赖关系, 左边是需要生成的文件名, 右边是依赖的文件名. 可见 \lstinline|a| 没有任何依赖, 可以直接生成, 而文件 \lstinline|b| 依赖于 \lstinline|a|, 需要等 \lstinline|a| 生成后才能开始生成.

在每个含有冒号的行后, 可以添加若干行命令, 当 \lstinline|make| 决定生成某个文件时, 就执行这些命令. 注意每个命令前必须有一个制表符(用键盘的 \lstinline|tab| 键插入). 所以生成 \lstinline|a| 文件的命令就是 \lstinline|echo "file a" > a|, 即新建文件 \lstinline|a|, 内容为 \lstinline|file a|. 同理, 生成 \lstinline|b| 的命令就是新建文件, 内容为 \lstinline|text in a {...}| 其中 \lstinline|...| 为 \lstinline|a| 中的文本. 如果 \lstinline|a| 文件不存在, 第 6 行命令就无法执行, 所以必须指定 \lstinline|a| 为 \lstinline|b| 的依赖.

每个含有冒号的行以及接下来的所有命令构成一个 \textbf{规则(rule)}. 冒号左边的文件叫做 \textbf{目标(target)}, 右边的文件叫做 \textbf{依赖(dependency)}. 如果有多个依赖文件, 用空格隔开即可.

现在可以在命令行先 \lstinline|cd| 到 \lstinline|Makefile| 所在的目录, 然后执行 \lstinline|make| 命令. 如果一切正常, 程序将会在命令行依次输出执行的命令(\lstinline|Makefile| 中第 2 行)以生成 \lstinline|a|
\begin{lstlisting}
echo "file a" > a
\end{lstlisting}
生成文件 \lstinline|a| 的内容为 \lstinline|file a|.

但为什么不生成 \lstinline|b| 呢? 因为 \lstinline|make| 默认情况下只生成第一个出现的目标. 如果想指定生成 \lstinline|b|, 执行 \lstinline|make b| 即可. 执行过程中命令行显示 \lstinline|Makefile| 的 5-7 行被执行了. 但注意第 2 行没有被执行. 这是因为 \lstinline|a| 文件已经存在了且上一次 \lstinline|make| 后没有被修改过, 所以就不需要再重复生成一次了. \lstinline|b| 中的内容如下
\begin{lstlisting}
text in a {
file a
}
\end{lstlisting}

现在两个文件都生成了, 且没有被修改, 所以若再次执行 \lstinline|make b|, 则会显示
\begin{lstlisting}
make: 'b' is up to date.
\end{lstlisting}
执行 \lstinline|make| 或 \lstinline|make a| 结果类似.

接下来我们把 \lstinline|a|, \lstinline|b| 两个文件都删除, 执行一次 \lstinline|make b|, 会发现第 2 行和 5-7 行都被执行了, 因为当 \lstinline|make| 要生成 \lstinline|b| 时发现它的依赖文件 \lstinline|a| 并不存在, 所以就会先生成 \lstinline|a|. 两文件中内容与之前相同.

我们来手动把 \lstinline|a| 中的内容修改为 \lstinline|modified file a|, 再次运行 \lstinline|make b|, 发现 \lstinline|Makefile| 的 5-7 行被执行, 且 \lstinline|b| 中的内容变为
\begin{lstlisting}
text in a {
modified file a
}
\end{lstlisting}
这是因为, \lstinline|make| 在生成 \lstinline|b| 前, 不但要检查它的所有依赖都存在还要检查他们在上次 \lstinline|make| 后是否有更新. 如果任何依赖有更新, \lstinline|b| 将会被重新生成.

\subsection{伪目标}
注意 rule 的目标除了是具体的文件, 也可以是\textbf{伪目标(phony tarket)}, 伪目标可以有任意名字. 伪目标的 rule 中可以有命令也可以没有.

一个特殊的伪目标是 \lstinline|goal|. 在上面的 \lstinline|Makefile| 中如果我们想让 \lstinline|make| 默认生成 \lstinline|b|, 可以在文件开始插入 \lstinline|goal| 作为第一个目标
\begin{lstlisting}
goal : b

a :
	echo "file a" > a

b : a
	echo "text in a {" > b
	cat a >> b
	echo "}" >> b
\end{lstlisting}
现在执行 \lstinline|make| 或者 \lstinline|make goal| 就相当于 \lstinline|make b| 了.

另一个常用的伪目标是 \lstinline|clean|, 通常用于清除所有被生成的文件. 修改 \lstinline|Makefile| 如下
\begin{lstlisting}
goal : b

a :
	echo "file a" > a

b : a
	echo "text in a {" > b
	cat a >> b
	echo "}" >> b
    
clean :
	rm -f a b
\end{lstlisting}
现在执行 \lstinline|make clean| 就可以清除两个生成的文件. 使用 \lstinline|-f| 选项的好处是, 即使某些被删除的文件不存在, \lstinline|rm| 也不会警告.
