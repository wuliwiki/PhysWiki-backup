% 科学计数法、数量级、单位(高中)

\begin{issues}
\issueDraft
\end{issues}

\subsection{科学计数法}
\textbf{科学计数法}通常表示为
\begin{equation}
x \e{n}~.
\end{equation}
其中 $x$ 是一个小数, 满足 $1 \leqslant \abs{x} < 10$。 $n$ 是一个整数($n=0$ 时 $10^{0} = 1$ 可以省略不写)。

例如 $1.23\e{4} = 12300$, $1.23\e{-4} = 0.000123$。

\begin{itemize}
\item $\e{n}$ 中的 $n$ 可以看作小数点移动的位数。 $n > 0$ 则向右移动, $n < 0$ 则向左移动, $n=0$ 则不移动($10^{0} = 1$)。
\item 小技巧: $x\e{-n}$ ( $1 < \abs{x} < 10$, $n > 0$)前面一共有 $n$ 个零。
\end{itemize}

\subsection{数量级}
\footnote{参考 Wikipedia \href{https://en.wikipedia.org/wiki/Order_of_magnitude}{相关页面}。}数量级。

\subsection{单位}

\subsubsection{单位换算}
