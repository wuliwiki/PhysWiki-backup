% Minimax 搜索和 Alpha-Beta 剪枝
% keys 极小化极大|minimax|alpha-beta|alpha beta|博弈论
% license Usr
% type Tutor

\subsection{Minimax 搜索(极小化极大搜索)}
在双方博弈类型的搜索中,例如棋类的最优化搜索,如果我们是 AI,我们会希望,走下这一步棋后,我们的最大的失败可能最小。为什么这里是“最大的失败可能”而失败可能不是一个定值呢?因为对方的可能下法不止一种,对于每一种对方的下法,我们也都有一个“失败的可能性”(实际上这里是一个最小值,我们选择下最优的下法让我们失败的可能性最小),我们需要使用这些失败的可能性的总体的最大值来评估这步棋的好坏。

所以可以发现,这是一个使得某个最大值(失败的可能性)最小的算法,所以叫做\textbf{极小化极大搜索(Minimax 搜索)}。

那么实际上搜索进行的是,”你一步我一步“的方法。也就是,循环以下过程:
\begin{enumerate}
\item AI(“我”)先选择最优的方法下棋;
\item 对方(“你”)再选对于你来说最优、也就是对我来说最差的一个方法下棋。
\end{enumerate}
考虑一个经过一些常规剪枝技巧的、搜索树为二叉树的对弈。我们模拟搜索并且步骤也遵循递归的步骤,通过模拟的过程来学习 Minimax 搜索方法。

假若现在应当 AI 下棋,我们向后搜索 $4$ 步棋,绘制出搜索树并且标出叶子节点的一个“估价”(也就是 AI,“我”,的胜率):
\begin{figure}[ht]
\centering
\includegraphics[width=14cm]{./figures/5db090b0210640e6.png}
\caption{模拟搜索树(仅叶子结点)} \label{fig_mmsab_1}
\end{figure}
其中,圆角方框为 AI 需要下棋,圆形为对方需要下棋。

那么,在对方下棋(也就是圆形的节点)进行搜索的时候,对方如果是“足够聪明的”,会选择让我们的胜率最小,也就是失败率最大的下法。也就是对所有这些可能胜率求最小值,模拟搜索过程进行更新后得到新的搜索树:
\begin{figure}[ht]
\centering
\includegraphics[width=14cm]{./figures/67d902d43a2ada71.png}
\caption{模拟搜索树(更新一层后)} \label{fig_mmsab_3}
\end{figure}

之后向上一步,轮到 AI 下棋,AI 会选择使自己胜率最大的,也就是对所有可能胜率求最大值,这次更新后为:
\begin{figure}[ht]
\centering
\includegraphics[width=14cm]{./figures/2cdc1f0260c41898.png}
\caption{模拟搜索树(更新两层后)} \label{fig_mmsab_4}
\end{figure}

现在这轮,又轮到对方下棋,对方这时候下棋仍然会选择使我们胜率最低的下法,也就是求最小值,这一次更新后的搜索树为:
\begin{figure}[ht]
\centering
\includegraphics[width=14cm]{./figures/71f8c442d8898127.png}
\caption{模拟搜索树(更新三层后)} \label{fig_mmsab_5}
\end{figure}

最后回到了搜索的根节点,这一步是 AI 下棋,选择使得自己胜率最大的