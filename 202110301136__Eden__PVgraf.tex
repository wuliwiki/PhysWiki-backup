% 压强体积图
% 理想气体|做功|压强|体积|状态方程

\begin{issues}
\issueDraft
\end{issues}

\pentry{功\upref{Fwork}}

\subsection{压强体积图与等温线}
对于一定量气体,在同一温度下,不同的压强对应不同的体积.为了直观地表现它们的关系,我们可以借助压强体积图,在上面画出一条曲线——曲线上每个点表示在当前温度下不同压强所对应的不同体积.
\begin{figure}[ht]
\centering
\includegraphics[width=10cm]{./figures/PVgraf_4.png}
\caption{理想气体的等温线} \label{PVgraf_fig4}
\end{figure}
\begin{figure}[ht]
\centering
\includegraphics[width=10cm]{./figures/PVgraf_3.png}
\caption{范德瓦尔斯气体的等温线} \label{PVgraf_fig3}
\end{figure}

等温线帮助我们分析压强、体积与温度的关系.例如\autoref{PVgraf_fig4} 中等温线是反比例函数,意味着理想气体在同一温度下压强与体积成反比.

\addTODO{还可以举几个等体线、等压线、绝热线的图片作为例子}
\subsection{压强体积图与做功}

\subsubsection{体积、压强与做功}
现有一个\textbf{体积 $V$ 可变}的容器,在容器壁上可以安装探测器来测量内部气体的\textbf{压强 $p$}.只要气体处于平衡态\upref{TherEq},气体的压强就处处相等.现在我们考虑这样一个问题——改变容器的体积,需要对容器做多少功(或者容器对外界做多少功)?

假设容器是长方体,横截面是 $S$,那么这块面上受气体压强的力为 $p\cdot S$.如果改变长方体的体积,使它长度增加 $\Delta x$(假设在这个过程中压强变化不大),那么气体对外做功为 $F\cdot \Delta x=p\cdot S\Delta x=p\Delta V$.于是我们得出结论 $\Delta W=p\Delta V$,这不仅对长方体容器成立,对任意形状的容器都成立.
\subsubsection{压强体积图上任意曲线的做功}
由于容器在改变体积的过程中,气体的压强可以发生变化,在一个复杂过程中做功不能用简单的 $p\Delta V$ 概括,但我们却可以划分为无穷多个小过程,每一个小过程中压强变化不大,做功近似地可以看成是 $p\Delta V$.这样一来,总的做功就是 $p\dd V$ 的积分了.

图像可以帮助我们方便地分析这个问题.对于 $p-V$ 图中的一条曲线,设容器从点 $1$ 沿着曲线变化到点 $2$——曲线上每一个点表明了气体在当时的体积和压强.
\begin{figure}[ht]
\centering
\includegraphics[width=5cm]{./figures/PVgraf_1.pdf}
\caption{$p-V$ 图中气体做的功} \label{PVgraf_fig1}
\end{figure}

如\autoref{PVgraf_fig1} 所示,做功
\begin{equation}\label{PVgraf_eq1}
W = \int_{V_1}^{V_2}P(V) \dd{V}
\end{equation}

由积分关系式可得,做功就是 $p-V$ 曲线下方的面积,也就是图中标注的阴影部分的面积. 

\begin{figure}[ht]  
\centering
\includegraphics[width=5cm]{./figures/PVgraf_2.pdf}
\caption{$p-V$ 图中闭合路径气体做的功} \label{PVgraf_fig2}
\end{figure}

如\autoref{PVgraf_fig2} 所示,如果是一个闭合路径,在气体状态沿着它走的过程中,对外做功就是闭合路径围出的面积(顺时针为正).

\begin{example}{理想气体在等温压缩需要做的功}
一个装有理想气体的体积可变的导热容器,放在温度为 $T$ 的恒温箱中,保证其温度恒为 $T$.现在缓慢地将它从体积 $V_0$ 压缩成 $V_1$,求这个过程中需要对容器壁做的功为多大.

解:气体对外界做功 $p\Delta V$,则外界对气体做功为 $-p\Delta V$.由理想气体状态方程\autoref{PVnRT_eq1}~\upref{PVnRT}可得 $p=\frac{nRT}{V}$.所以
\begin{equation}
\begin{aligned}
\int_{V_0}^{V_1} -p\dd V&=\int_{V_0}^{V_1} -\frac{nRT}{V} \dd V \\
&=nRT \ln\qty(\frac{V_0}{V_1})
\end{aligned}
\end{equation}

\end{example}
