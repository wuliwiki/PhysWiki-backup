% 单纯剖分(三角剖分)
% triangulation|triagulate|多面体|剖分|代数拓扑|拓扑|同调|homology|topology
\begin{issues}
\issueOther{应当合并到单纯形与单纯复形\upref{SimCom}}
\end{issues}


\pentry{单纯形与复形\upref{SimCom}}

\begin{definition}{多面体}
给定复形 $K$,则 $\bigcup\limits_{A\in K}A$ 称为 $K$ 的\textbf{多面体(polyhedron)},记为 $\abs{K}$.此时称 $K$ 为 $\abs{K}$ 的一个\textbf{三角剖分(triangulation)},也称\textbf{单纯剖分}.
\end{definition}

该定义是先有了复形的概念,再在此基础上得到多面体的概念.反过来,我们也可以认为多面体是欧几里得空间中一种可以表示为若干单纯形之并的子集,而由于单纯形就是各个维度的“三角形”,这种表示自然被称为三角剖分.

许多常见的拓扑空间都可以进行三角剖分.

\begin{example}{有界圆柱面}

考虑一个圆柱体的表面,即集合 $\{(x, y, z)\in\mathbb{R}^3|x^2+y^2=1, z\in [0, 1]\}$,它可以表示为 $6$ 个\textbf{规则相处}的单纯形的并集,\autoref{Traglt_fig1} 展示的是其展开的样子.


\begin{figure}[ht]
\centering
\includegraphics[width=8cm]{./figures/Traglt_1.pdf}
\caption{有界圆柱面的单纯剖分示意图.图中展示的是圆柱面沿着单形 $[1, 2]$ 剪开并铺平后的样子,方便观察.注意图中点 $1$ 和点 $2$ 都出现了两次,还原圆柱面的时候要把它们对应粘接起来.圆柱面被剖分成三角形 $[1,2,4], [1,3,4], [3,4,5], [4,5,6], [2,5,6], [1,2,5]$ 的并集.} \label{Traglt_fig1}
\end{figure}


这一剖分使用了六个三角形,看起来有些多余,似乎只用四个三角形也可以,即抛弃图中的 $[1,2,5]$ 和 $[2,5,6]$,而将 $[1, 2]$ 和 $[5, 6]$ 对应粘接,其中 $1$ 粘到 $5$,$2$ 粘到 $6$.然而这样四个三角形的粘接并不满足“规则相处”条件,因为这样一来 $[1, 2, 4]$ 和 $[5, 6, 4]$ 的交集就变成了 $[4]\cup[1, 2]$,即一个点和一条边,也即一个 $0$ 维面和一个 $1$ 维面,而规则相处要求任意两个三角形的交集是“一个面”.为了满足规则相处条件,我们不得不增添两个三角形,形成图中的剖分方法.



有界圆柱面进行单纯剖分最少需要这六个三角形,当然也可以使用八个、十个、一百个等.

\end{example}


\begin{example}{克莱因瓶}

克莱因瓶的三角剖分如图所示.

\begin{figure}[ht]
\centering
\includegraphics[width=5cm]{./figures/Traglt_3.pdf}
\caption{克莱因瓶的三角剖分示意图.注意点 $4$ 和点 $7$ 的位置.} \label{Traglt_fig3}
\end{figure}

\end{example}


\begin{exercise}{莫比乌斯带}
请尝试给出莫比乌斯带的三角剖分.
\end{exercise}


\begin{example}{射影平面}

射影平面的单纯剖分如\autoref{Traglt_fig2} 所示.

\begin{figure}[ht]
\centering
\includegraphics[width=8cm]{./figures/Traglt_2.pdf}
\caption{射影平面的单纯剖分示意图.射影平面是用一个单位圆,将其对径点粘合而成的拓扑空间.图中使用了方形,以方便进行三角剖分的表示.} \label{Traglt_fig2}
\end{figure}

\end{example}

\begin{example}{双环面}

考虑双环面,即两个环面 $S^1\times S^1$ 各挖去一块连通区域后,把挖去区域的两个边界对应粘连起来得到的一个整体.双环面的单纯剖分如\autoref{Traglt_fig4} 所示.

\begin{figure}[ht]
\centering
\includegraphics[width=12cm]{./figures/Traglt_4.pdf}
\caption{双环面的单纯剖分示意图.注意,由顶点 $5, 6, 8, 9$ 构成的方形不在环面上,它就是被挖去的那部分.这个单纯剖分被分为两部分来展示,分别是两个环面,挖去中间部分后对应粘连.} \label{Traglt_fig4}
\end{figure}


\end{example}






