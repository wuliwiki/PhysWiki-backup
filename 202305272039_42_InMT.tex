% 积分中值定理
% 中值定理

\begin{issues}
\issueDraft
\end{issues}

积分中值定理可以将积分号去掉,或者将复杂的被积函数化为相对简单的被积函数,从而使问题简化,其应用相当的广泛。

\subsection{积分中值定理}
\begin{theorem}{积分中值定理}
如果 $f(x)$ 在 $[a,b]$ 上可积,并且在该区间上恒有$m<f(x)<M$
则
\begin{equation}
\int_a^b f(x)\dd x=\mu(b-a),
\end{equation}
其中 $m\leq\mu\leq M.$
\end{theorem}

\textbf{证明:}
设 $a<b$,则由\autoref{the_DIntP_3}~\upref{DIntP},
\begin{equation}
m(b-a)\leq\int_a^b f(x)\dd x\leq M(b-a)
\end{equation}
故有
\begin{equation}
m\leq\frac{1}{b-a}\int_a^b f(x)\dd x\leq M
\end{equation}
令
\begin{equation}
\frac{1}{b-a}\int_a^b f(x)\dd x=\mu
\end{equation}
即可得所需求等式。

对 $a>b$ ,
\begin{equation}
\int_a^b f(x)\dd x=-\int_b^a f(x)\dd x=\mu(b-a),
\end{equation}
而 $a=b$ 时定理显然。 

\textbf{证毕!}

\begin{theorem}{推广积分中值定理}\label{the_InMT_1}
设 $f(x),g(x)$ 在 $[a,b]$ 上可积,并且在该区间上恒有$m\leq f(x)\leq M$
, 且 $g(x)$ 不变号,则
\begin{equation}\label{eq_InMT_3}
\int_a^b f(x)g(x)\dd x=\mu\int_a^b g(x)\dd x
\end{equation}
其中 $m\leq\mu\leq M.$
\end{theorem}
\textbf{证明:}
设 $g(x)\geq 0$,且 $a<b$,则
\begin{equation}
mg(x)\leq f(x)g(x)\leq Mg(x).
\end{equation}
由\autoref{cor_DIntP_1}~\upref{DIntP}
\begin{equation}\label{eq_InMT_1}
m\int_a^b g(x)\dd x\leq\int_a^b f(x)g(x)\dd x\leq M\int_a^b g(x)\dd x
\end{equation}
由 $g(x)\geq0$,得
\begin{equation}\label{eq_InMT_2}
\int_a^b g(x)\dd x\geq0
\end{equation}
\autoref{eq_InMT_1} 同除于 \autoref{eq_InMT_2} ,并令
\begin{equation}
\mu=\frac{\int_a^b f(x)g(x)\dd x}{\int_a^b g(x)\dd x}
\end{equation}
就有\autoref{eq_InMT_3} 和
\begin{equation}
m\leq\mu\leq M
\end{equation}
对于 $g(x)\leq0$,只需用 $-g(x)$ 代替 $g(x)$ 就能转到刚才的情形,同理 $a>b$ 的情形容易由 $a<b$ 得到,$a=b$ 则定理恒成立。

\textbf{证毕!}

\begin{example}{连续函数中值定理}
若\autoref{the_InMT_1} 的 $f(x)$ 连续,则在 $[a,b]$ 上,必存在某一点 $c$,使得 $f(c)=\mu$(\autoref{sub_conff_4}~\upref{conff})。于是此时\autoref{the_InMT_1} 的\autoref{eq_InMT_3} 可写为
\begin{equation}
\int_a^b f(x)g(x)\dd x=f(c)\int_a^b g(x)\dd x,
\end{equation}
其中,$c\in[a,b].$
\end{example}