% 乔治·比德尔·艾里(综述)
% license CCBYSA3
% type Wiki

本文根据 CC-BY-SA 协议转载翻译自维基百科 \href{https://en.wikipedia.org/wiki/George_Biddell_Airy}{相关文章}。

\begin{figure}[ht]
\centering
\includegraphics[width=6cm]{./figures/f2f2a06f3fcb14e2.png}
\caption{} \label{fig_AL_1}
\end{figure}
乔治·比德尔·艾里爵士(Sir George Biddell Airy,/ˈɛəri/;1801年7月27日-1892年1月2日)是一位英国数学家和天文学家,曾于1826年至1828年担任卢卡斯数学教授,并于1835年至1881年担任第七任皇家天文学家。他的诸多成就包括对行星轨道的研究、测定地球平均密度、提出二维固体力学问题的解法,以及在担任皇家天文学家期间,确立格林尼治作为本初子午线的所在地。
\subsection{传记}
艾里出生于诺森伯兰郡的安尼克,出身于一个世代相传的艾里家族,其家族可追溯至14世纪居住在西摩兰郡肯特米尔的同姓家族。他所属的这一支系因英国内战而遭受打击,遂迁至林肯郡并以务农为生。艾里的早期教育先是在赫里福德的初等学校完成,随后进入科尔切斯特皇家文法学校就读。\(^\text{[1]}\)艾里是个性格内向的孩子,但因擅长制作豌豆射手而在同学中颇受欢迎。\(^\text{[2]}\)

从13岁起,艾里经常住在他位于萨福克郡普莱福德的叔叔亚瑟·比德尔家中。比德尔将艾里介绍给了他在普莱福德庄园的朋友、废奴主义者托马斯·克拉克森。克拉克森是剑桥大学的数学硕士,先是考察了艾里的古典学水平,随后又请剑桥三一学院的一位研究员对他进行数学能力的考核。\(^\text{[3][4]}\)结果,艾里于1819年以“资助生”的身份进入三一学院学习,这意味着他缴纳较低的学费,但必须通过承担部分服务工作来抵消差额。\(^\text{[5]}\)在那里,他的学术表现十分出色,几乎立刻就被认为是当届最优秀的学生。1822年他当选为三一学院奖学金获得者,次年以“高级状元”身份毕业,并获得史密斯一等奖。

1824年10月1日,他被选为三一学院研究员;1826年12月,接替托马斯·特顿担任卢卡斯数学教授。此职位他仅担任了一年多,随后于1828年2月当选为普卢米安天文学教授,并出任新成立的剑桥天文台台长。\(^\text{[1]}\)1836年,他当选为皇家学会院士;1840年当选为瑞典皇家科学院外籍院士;1859年又成为荷兰皇家艺术与科学学院外籍院士。\(^\text{[6]}\)
\subsubsection{研究}
\begin{figure}[ht]
\centering
\includegraphics[width=6cm]{./figures/de664d7acdba7a8b.png}
\caption{乔治·比德尔·艾里} \label{fig_AL_2}
\end{figure}
在这些早期岁月中,艾里在数学和物理领域的写作活动之频繁,可从以下事实略窥一二:在他被任命之前,他已为《皇家学会哲学汇刊》撰写了三篇重要论文,为剑桥哲学学会撰写了八篇论文。在剑桥天文台,艾里很快展现出其出色的组织能力。接手时,该天文台唯一的望远镜是过境仪,他便将全部精力投入其上。通过建立一套规范的工作系统,并采取细致的观测数据整理计划,他能够使观测工作始终保持最新,并每年按时发表观测结果,这种准时的作风令当时的同行惊叹不已。不久之后,一台壁圈仪被安装起来,并于1833年开始定期使用进行观测。同年,诺桑伯兰公爵为剑桥天文台捐赠了一块口径为12英寸的优质物镜,该物镜的安装由艾里亲自设计并监督施工,尽管其建成已是他于1835年调任格林尼治之后。\(^\text{[1]}\)

艾里在这一时期的著作主要分为数学物理和天文学两类。前者大多源自他教授课程中与光学理论相关的问题,其中尤为重要的是他关于《圆孔物镜的衍射》一文,以及他对彩虹完整理论的阐述。因这些研究成果,他于1831年获得了皇家学会的科普利奖章。在这一时期的天文学著作中,最重要的包括:他对木星质量的研究、为英国科学促进会撰写的关于19世纪天文学进展的报告,以及他关于《地球和金星运动中一项长期不等项》的研究。\(^\text{[7]}\)

他的这篇精彩而富有启发性的报告中有一节专门探讨了“英国与其他国家在天文学进展方面的比较”,并指出英国在这方面的明显劣势。后来,这一批评在很大程度上因他本人的努力而得以改善。\(^\text{[8]}\)

\textbf{地球平均密度的测定}

艾里最杰出的研究之一是他对地球平均密度的测定。1826年,他萌生了一个想法:通过在一座深井的顶部和底部进行摆钟实验来解决这个问题。他于当年在康沃尔的多尔科斯矿(Dolcoath mine)首次尝试,但因其中一个摆钟发生意外而失败。第二次尝试在1828年,由于矿井被水淹没再次告吹。之后多年未能再次开展实验。直到1854年,他终于在英格兰北部南希尔兹附近的哈顿矿井(Harton pit)进行了实验。实验直接得出的结果是:在井底测得的重力比井顶大出其总量的 1/19286,矿井深度为383米(1,257英尺)。据此,他推算出地球的比密度为6.566。\(^\text{[9]}\)尽管这一数值远高于通过其他方法得出的早期结果,艾里仍坚信自己经过严谨细致的观测与论证所得出的这一结果“至少有资格与其他结果同台竞争”。\(^\text{[8]}\)目前公认的地球平均密度值为5.5153克/立方厘米。

\textbf{参考大地水准面}

1830年,艾里利用在英国境内进行的测量数据,计算出了地球的极半径和赤道半径。尽管他所使用的半径数值后来被更精确的数值所取代(如 GRS 80 和 WGS84 所采用的数值),但他的“艾里大地水准面”(严格来说是一种参考椭球体,称为 OSGB36)至今仍被英国测绘局(Ordnance Survey)用于英格兰、苏格兰和威尔士的地图绘制工作,因为它与当地的海平面拟合得更好(大约比全球平均海平面低80厘米)。\(^\text{[10][11]}\)

\textbf{行星摄动不等式}

艾里在金星与地球运动中的一个新不等式的发现,在某些方面是他最卓越的成就之一。在修正德朗布尔(Delambre)所编太阳表的轨道要素时,他开始怀疑该表的编制者忽略了某项不等项。这一原因他很快就找到了:金星平均运动的13倍几乎等于地球平均运动的8倍,两者之间的差异仅是地球平均运动的一小部分。尽管这一差值本身极小,但在对微分方程进行积分时,这一项会获得一个约为 2,200,000 的乘子,由此艾里推断出一个影响显著、周期长达 240 年的不等式(见《哲学汇刊》第122卷,第67页)。这项研究很可能是当时在行星理论领域中最为繁复的工作,也代表了自万有引力理论建立以来,英国在太阳表方面做出的第一项明确改进。

为表彰这项成就,英国皇家天文学会于1833年授予他金质奖章\(^\text{[8]}\)(他在1846年再次获得该奖)。

\textbf{艾里斑}
\begin{figure}[ht]
\centering
\includegraphics[width=6cm]{./figures/e39cb804ab2c10e1.png}
\caption{计算机生成的模拟艾里斑图像} \label{fig_AL_3}
\end{figure}
光学仪器的分辨率受衍射限制。因此,即使是最完美的透镜也无法在焦点处生成一个真正的点像,而是会形成一个明亮的中央图样,现在称为“艾里斑”,其周围环绕着一圈圈同心的光环,称为“艾里图样”。艾里斑的大小取决于光的波长和光圈的尺寸。约翰·赫歇尔曾描述过这一现象,\(^\text{[12]}\)但艾里是第一个对其进行理论解释的人。\(^\text{[13]}\)

这成为反驳绝对地心说最后几个残留论据之一的关键证据:即所谓的“巨星论点”。第谷·布拉赫和乔瓦尼·巴蒂斯塔·里乔利曾指出,未能观测到恒星视差表明恒星距离极其遥远。然而,肉眼以及早期光圈较小的望远镜似乎显示恒星具有一定大小的圆盘状外观。这就意味着这些恒星的体积将远大于太阳(当时人们还不知道超巨星或特超巨星的存在,但有些恒星据推算甚至比当时所估算的整个宇宙还大)。然而,这种“圆盘”只是虚像:人们实际上并没有看到恒星的真实图像,而是看到了艾里斑。使用现代望远镜,即便是放大倍率最高的望远镜,几乎所有恒星的图像仍然只是一个光点,这才是正确的表现。

\textbf{皇家天文学家}
\begin{figure}[ht]
\centering
\includegraphics[width=6cm]{./figures/00059e3408dbcf52.png}
\caption{艾里的子午仪位于皇家格林尼治天文台的子午仪厅中;从1884年到1984年的一百年间,它标志着世界本初子午线——更准确地说,本初子午线是由它所标示或衍生而来的。} \label{fig_AL_4}
\end{figure}
1835年6月,艾里接替约翰·庞德被任命为皇家天文学家,开启了他在国家天文台的漫长职业生涯,也正是这一角色为他赢得了最主要的声誉。他上任时天文台的状况堪忧,第一海军大臣奥克兰勋爵甚至认为“应该将其彻底清空”,而艾里自己也承认“它的状态非常糟糕”。他一如既往地充满干劲,立即着手全面重组天文台的管理体系。他重新整理了观测记录册,使图书馆运作规范,安装了新的(谢普香克斯)赤道仪,并组建了一个新的地磁观测站。

1847年,他设计并安装了一台高度—方位仪,使得对月球的观测不再局限于子午线,而是可以在其出现在天际的任何时候进行。\(^\text{[14]}\)1848年,艾里发明了反射天顶望远镜,以替代此前使用的天顶象限仪。1850年底,一台口径203毫米(8英寸)、焦距3.5米(11英尺6英寸)的大型子午仪被安装起来,至今仍是天文台同类仪器中的主力设备。1859年,一台口径330毫米(13英寸)的赤道仪完成安装,艾里在当年的日记中写道:“天文台中现已没有任何一个人或一台仪器是庞德时代留下的。”这一彻底的转变又因1868年引入的光谱观测和1873年启动的太阳黑子摄影记录而画上了圆满的句号。\(^\text{[8]}\)
\begin{figure}[ht]
\centering
\includegraphics[width=6cm]{./figures/f47cb17756837120.png}
\caption{格林尼治的本初子午线(1884–1984);左上方展示的望远镜位于玻璃后方。} \label{fig_AL_5}
\end{figure}
在艾里被任命为皇家天文学家时,他已经开始监督整理格林尼治自1750年至1830年间积累的行星观测数据这一艰巨任务。不久之后,他又承担了另一项繁重的工作,即整理同一时期在詹姆斯·布拉德雷、内森尼尔·布利斯、内维尔·马斯克林和约翰·庞德等人主持下于格林尼治完成的大量月球观测数据。为支付这一工作的费用,财政部拨出了一大笔资金。最终,共有不下8,000条月球观测记录得以从湮没中“拯救”出来,并于1846年以便于直接与理论对比并改进月球运行表的形式,提供给天文学家使用。

艾里因此项工作于1848年获得英国皇家天文学会的表彰,这项成果直接促使彼得·安德烈亚斯·汉森发现了月球运行中的两个新不等式。在完成这些数据整理后,艾里在着手进行相关理论研究前,先了解是否已有其他数学家正在研究该问题。当得知汉森在丹麦国王资助下已着手研究,但因国王去世导致资金中断而面临放弃风险时,他便代表汉森向英国海军部申请必要的资助。他的请求立刻得到批准,从而促成了汉森将其著名的《月球表》献给“其大不列颠及爱尔兰女王陛下之海军部”。

1851年,艾里在格林尼治建立了一条新的本初子午线。这是第四条“格林尼治子午线”,并于1884年成为国际公认的本初子午线。1984年,该线被IERS参考子午线取代,后者位于其以东约102米处。

\textbf{海王星的探索}
\begin{figure}[ht]
\centering
\includegraphics[width=6cm]{./figures/a6193fabb85e1713.png}
\caption{乔治·比德尔·艾里在《名利场》1875年11月期中被Ape(笔名)讽刺漫画化} \label{fig_AL_6}
\end{figure}
1846年6月,艾里开始与法国天文学家乌尔班·勒维耶通信,讨论后者关于天王星运动异常可能是由尚未观测到的天体引起的预测。得知剑桥天文学家约翰·卡奇·亚当斯也曾提出类似的预测后,艾里于7月9日敦促詹姆斯·查利斯展开系统性的搜寻,希望能为英国赢得发现的荣耀。然而,最终由勒维耶推动、约翰·戈特弗里德·伽勒在柏林展开的竞争性搜索取得了优先发现的胜利。

尽管艾里因未能更迅速地采纳亚当斯的建议而“遭到英法两国最激烈的抨击”,但也有观点认为亚当斯的通信内容含糊拖沓,\(^\text{[17]}\)另有意见指出,搜寻新行星的任务本就不应由皇家天文学家承担。\(^\text{[19]}\)

\textbf{以太拖曳实验}

1871年,艾里使用装满水的望远镜进行实验,以验证“以太拖曳假说”是否会导致恒星光行差的变化[20] 与所有其他试图检测以太漂移或拖曳的实验一样,艾里的结果也是阴性的。

\textbf{月球理论}

1872年,艾里萌生了用一种全新方式处理月球运动理论的想法。71岁高龄的他开始投入这项艰巨的工作。关于他方法的一般性描述,可见于《皇家天文学会月报》第34卷第3期。其方法的核心是采用夏尔-欧仁·德洛奈最终得出的关于月亮经度、纬度和视差的数值表达式,并为每个数字附加一个符号项,该符号项的具体数值将通过代入月球运动方程来确定。\(^\text{[8]}\)

在这种处理方式下,项的顺序是按数值排列的。尽管工作量巨大,足以让年轻人望而却步,但每一步的计算相对简单,大部分工作都可以交给“一位普通的计算员”来完成。\(^\text{[8][a]}\)

这项工作于1886年出版,当时艾里已85岁。在此之前一段时间,他便开始担心某些计算中可能存在错误,因此着手进行修正。然而他的精力已大不如前,最终未能彻底完成复查。1890年,他在笔记中写道,早期的一个关键步骤中存在严重错误,并感慨地补充说:“我的研究热情被打击了,从那以后我再也没有真正继续推进这项工作。”\(^\text{[8]}\)

\textbf{工程力学}

\textbf{应力函数法}

1862 年,艾里提出了一种用于确定梁内部应变和应力场的新方法。\(^\text{[21]}\)这种方法有时被称为艾里应力函数法,可用于求解固体力学中许多二维问题(参见维基学院)。例如,H.M. 韦斯特加德曾使用此方法来确定裂纹尖端周围的应力和应变场,从而为断裂力学的发展作出了重要贡献。\(^\text{[22]}\)

\textbf{泰桥灾难}
\begin{figure}[ht]
\centering
\includegraphics[width=8cm]{./figures/3ca3b98c43168636.png}
\caption{从北侧拍摄的原始泰桥} \label{fig_AL_7}
\end{figure}
在19世纪70年代末,为北英格兰铁路设计的福斯吊桥项目中,托马斯·鲍奇曾就该桥可能遇到的风速和风压向艾里咨询。艾里认为,最大风压大约不会超过每平方英尺10磅(约合500帕斯卡)。鲍奇误以为这一判断也适用于当时正在建造的第一座泰铁路桥。然而,在严重风暴中,实际风压可能远远超过这一数值。艾里随后被召唤出席泰桥灾难的官方调查,并因其建议而受到批评。然而,当时人们对大型结构的风阻问题了解甚少,最终成立了一个风压皇家委员会以研究这一问题。\(^\text{[23]}\)

\textbf{争议}
\begin{figure}[ht]
\centering
\includegraphics[width=8cm]{./figures/d4aa7916a606b697.png}
\caption{倒塌后的原始泰桥} \label{fig_AL_8}
\end{figure}
艾里的讣告由英国皇家学会发表,其中称他为“一位强硬的对手”,并流传下不少他与其他科学家意见不合的故事。弗朗西斯·罗纳兹在担任邱园天文台首任名誉台长期间,发现艾里将自己视为敌人,因为艾里认为邱园是对格林尼治天文台的竞争者。\(^\text{[24][25]}\)艾里与查尔斯·巴贝奇及詹姆斯·索斯爵士的矛盾也有详细记载。\(^\text{[26][27]}\)
\subsubsection{私生活}
1824年7月,艾里在德比郡徒步旅行时邂逅了“美貌出众”的理查达·史密斯(1804–1875)。他后来写道:“我们的目光相遇……我的命运已注定……我感到我们必须结合在一起。”两天后他便向她求婚。但理查达的父亲——牧师理查德·史密斯——认为艾里的经济条件尚不足以成家。直到1830年,艾里在剑桥获得稳定职位后,才获准成婚。\(^\text{[17][28][29]}\)
\begin{figure}[ht]
\centering
\includegraphics[width=6cm]{./figures/6946c87a7734b592.png}
\caption{} \label{fig_AL_9}
\end{figure}
艾里一家共有九个孩子,其中前三个在年幼时夭折。
\begin{itemize}
\item 伊丽莎白·艾里(生于1833年)于1852年因肺痨(结核病)去世。
\item 最早活到成年的孩子是威尔弗里德(1836–1925),他为“上校”乔治·汤姆林设计并建造了奥威尔公园天文台。\(^\text{[30][31]}\)威尔弗里德的女儿是艺术家安娜·艾里。\(^\text{[31]}\)安娜的母亲在她出生后不久去世,她由未婚的姑妈克里斯塔贝尔和安诺特(见下文)抚养长大。
\item 乔治·艾里的儿子休伯特·艾里(1838–1903)是一名医生,是偏头痛研究的先驱。\(^\text{[32]}\)艾里本人也患有此病。
\item 艾里的大女儿希尔达(1840–1916)于1864年嫁给了数学家爱德华·鲁斯。\(^\text{[33]}\)
\item 克里斯塔贝尔(1842–1917)终身未婚,接下来的妹妹安诺特(1843–1924)也是如此。
\item 艾里最小的孩子是奥斯蒙德(1845–1929)。艾里于1872年6月17日被授予爵位。\(^\text{[34]}\)
\end{itemize}