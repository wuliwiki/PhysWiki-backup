% 向量空间的张量积
% license Usr
% type Wiki

\begin{issues}
\issueDraft
\end{issues}
% 我可能不d

\pentry{多线性映射\upref{MulMap}}
% 此处只引用多线性映射的第一子节

\addTODO{存在性}

\addTODO{万有性质}
\begin{definition}{张量积}\label{def_vecTsr_1}
设 $V_1, V_2,T$ 是域 $\mathbb F$ 上的向量空间,$\sigma: V_1 \times V_2 \rightarrow T$ 是个双线性映射。若对域 $\mathbb F$ 上任意向量空间 $U$ ,任一双线性映射 $\varphi:V_1\times V_2\rightarrow U$, 都存在唯一的线性映射 $\psi:T\rightarrow U$ 使 $\varphi=\psi\sigma$,即如下交换图
\begin{equation}
\begin{CD}
V_1 \times V_2 @>{\sigma}>> T \\
@| @V{\psi}VV\\
V_1 \times V_2 @>{\varphi}>> U
\end{CD}~
\end{equation}
那么对 $(\sigma, T)$ 就称为 $V_1$ 与 $V_2$ 的\textbf{张量积},$T$ 记为 $V_1 \otimes V_2$,$\sigma(v_1, v_2)$ 记为 $v_1 \otimes v_2$\footnote{在张量的张量积\upref{TsrPrd}一文中,在有限维度的情况下,我们定义了向量间的张量积,这里是对它的推广}。

我们还可以定义向量空间 $V$ 的 $k$ 次\textbf{张量幂},$V^{\otimes k}: = \underbrace{V \otimes \dots \otimes V}_k$。
\end{definition}
\addTODO{张量积的维度}
\addTODO{张量积与对偶空间}