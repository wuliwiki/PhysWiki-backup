% 棣莫弗公式(综述)
% license CCBYSA3
% type Wiki

本文根据 CC-BY-SA 协议转载翻译自维基百科\href{https://en.wikipedia.org/wiki/Laplace\%27s_equation}{相关文章}。

在数学中,德摩根公式(也称为德摩根定理或德摩根恒等式)表明,对于任何实数\( x \)和整数\( n \),有
\[
(\cos x + i \sin x)^n = \cos(nx) + i \sin(nx),~
\]
其中\( i \)是虚数单位(\( i^2 = -1 \))。该公式以亚伯拉罕·德摩根的名字命名[1],尽管他在自己的著作中并未明确提出该公式[2]。表达式\( \cos x + i \sin x \)有时简写为\( \text{cis} \, x \)。

该公式非常重要,因为它将复数和三角学联系起来。通过展开左侧表达式,并在假设\( x \)为实数的情况下比较实部和虚部,可以推导出\( \cos(nx) \)和 \( \sin(nx) \)的有用表达式,形式为\( \cos x \)和\( \sin x \)的函数。

如所写,该公式对非整数幂\( n \)无效。然而,存在该公式的广义形式,适用于其他指数。这些广义形式可用于给出统一根的显式表达式,即使得\( z^n = 1 \)的复数\( z \)。

使用正弦和余弦函数对复数的标准扩展,该公式即使在\( x \)为任意复数时也成立。
