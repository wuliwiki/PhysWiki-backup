% 张成空间
% 矢量空间|张成|集合|span|基底

\pentry{线性相关性\upref{LinInd}}

矢量的线性组合,使得我们有可能用少量矢量来表示更多的矢量. 例如 $N$ 维几何矢量空间\upref{GVec}空间中的一组基底可以通过线性组合得到整个矢量空间中的任意矢量, 我们就说这组基底\textbf{张成(span)}了这个矢量空间. 我们来看 “张成” 更一般的定义.

\subsection{张成空间}

\begin{definition}{张成空间}
给定若干矢量构成的集合 $S=\{\bvec{v_\alpha}\}$,可以是无穷集合, 也可以线性相关, 那么从$S$中任意地选择有限个矢量进行线性组合所得到的集合 $\{\sum_i^N c_i\bvec{v_i}|N\in\mathbb{Z}, c_i\in\mathbb{R}, \bvec{v}_i\}$ 称为 $S$ 所\textbf{张成(span)的空间},记为 $\opn{span} S$ 或者 $\ev{\{\bvec v_\alpha\}}$.
\end{definition}
容易证明, 定义中的 $\opn{span} S$ 必定符合矢量空间\upref{LSpace}的定义.

张成空间的概念,最直观的例子就是子空间.比如说,三维的几何矢量空间$V$中,任意选定两个向量$\bvec{v}$和$\bvec{u}$,这两个向量的所有线性组合构成的几何,是$V$中的一个过原点的平面.这样,$\opn{span}\{\bvec{v}, \bvec{u}\}$构成了一个线性空间,同时还是$V$的子集,因此又被称作$V$的一个子空间.

\begin{exercise}{}
给定一个三维线性空间,把其中的每个向量表示成一个行向量$(x,y,z)$,其中$x,y,z\in\mathbb{R}$.取三个向量$\bvec{v}_1=(1,2,0)$,$\bvec{v}_2=(3,2,4)$和$\bvec{v}_3=(1,1,1)$,那么这三个向量所张成的空间是一个点、一条线、一个平面还是整个三维空间本身?三个向量是否线性相关?它们的线性相关性和张成空间的样子有什么关系?
\end{exercise}

实际上,我们完全可以摆脱“用给定空间的向量来张成一个子空间”,而形式化地定义张成的概念.取任何一个集合$S$,我们可以用$S$在域$\mathbb{K}$上直接构造一个线性空间$V=\{\sum a_\alpha s_\alpha|a_\alpha\in\mathbb{K}, s_\alpha\in S\}$.这里的$a_\alpha s_\alpha$表示把数字$a_\alpha$和元素$s_\alpha$组合在一起,变成一个不属于$S$的新元素\footnote{只有一个情况例外,那就是当$a_\alpha=1$时,将$1s_\alpha$认为就是$s_\alpha$本身,从而这时并没有得到新元素.},而$\sum$表示这些新元素组合起来又得到了新的元素,组合符号用加号$+$表示.这时回过头来,我们可以把每个$s_\alpha$看成是$V$的一个向量,它们张成了$V$这个线性空间.
