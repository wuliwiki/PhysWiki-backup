% 矢量算符运算法则
% keys 矢量算符|微积分|叉乘|点乘|内积
% license Xiao
% type Tutor

\begin{issues}
\issueOther{每个等式给出一些百科中应用的链接}
\end{issues}

\pentry{梯度\nref{nod_Grad}, 旋度\nref{nod_Curl}, 列维—奇维塔符号\nref{nod_LeviCi}}{nod_dd84}

\footnote{参考 \cite{GriffE} 相关章节。}这里列举一些常用的矢量算符恒等式, 证明见文末。

\subsection{一阶导数}
\subsubsection{梯度}
\begin{equation}\label{eq_VopEq_2}
\grad (fg) = f\grad g + g \grad f~,
\end{equation}

\begin{equation}
\grad (\bvec A \vdot \bvec B) = \bvec A \cross (\curl \bvec B) + \bvec B \cross (\curl \bvec A) + (\bvec A \vdot \grad) \bvec B + (\bvec B \vdot \grad) \bvec A~.
\end{equation}

\subsubsection{散度}

\begin{equation}\label{eq_VopEq_1}
\div (f \bvec A) = f (\div \bvec A) + \bvec A \vdot (\grad f)~,
\end{equation}

\begin{equation}\label{eq_VopEq_4}
\div (\bvec A \cross \bvec B) = \bvec B \vdot (\curl \bvec A) - \bvec A \vdot (\curl \bvec B)~.
\end{equation}

\subsubsection{旋度}
\begin{equation}\label{eq_VopEq_3}
\curl(f\bvec A) = f (\curl\bvec A) + (\grad f) \cross\bvec A~,
\end{equation}

\begin{equation}\label{eq_VopEq_6}
\curl (\bvec A \cross \bvec B) = (\bvec B \vdot \grad) \bvec A - (\bvec A \vdot \grad) \bvec B + \bvec A (\div \bvec B) - \bvec B (\div \bvec A)~.
\end{equation}

\subsection{二阶导数}

\begin{equation}
\div (\grad f) = \laplacian f~,
\end{equation}

\begin{equation}\label{eq_VopEq_5}
\laplacian (f \bvec v) = (\laplacian f)\bvec v + f \laplacian \bvec v + 2 (\grad f \vdot \grad) \bvec v~,
\end{equation}

\begin{equation}
\curl (\grad f) = \bvec 0~,
\end{equation}

\begin{equation}
\div (\curl \bvec v) = 0~,
\end{equation}

\begin{equation}\label{eq_VopEq_7}
\curl(\curl \bvec v) = \grad(\div\bvec v) - \laplacian \bvec v~.
\end{equation}

\subsection{复合函数求导}
使用链式法则\upref{PChain}容易证明以下关系
\begin{equation}
\grad f[g(\bvec r)] = f'[g(\bvec r)] \grad g(\bvec r)~,
\end{equation}
\begin{equation}
\div \bvec f[g(\bvec r)] = \bvec f'[g(\bvec r)] \vdot \grad g(\bvec r)~,
\end{equation}
\begin{equation}
\curl \bvec f[g(\bvec r)] = \qty[\grad g(\bvec r)] \cross \bvec f'[g(\bvec r)]~,
\end{equation}
\begin{equation}
\laplacian f[g(\bvec r)] = f''[g(\bvec r)] (\grad g)^2 + f'[g(\bvec r)] \laplacian g(\bvec r)~.
\end{equation}

\subsection{证明}
理论上, 我们可以直接根据定义, 将各个矢量记为分量的形式证明, 但直接写出来非常繁琐。 一种简单的记号是使用克罗内克 delta 函数\upref{Kronec} $\delta_{i,j}$ 和 Levi-Civita 符号\upref{LeviCi} $\epsilon_{i,j,k}$, 再结合爱因斯坦求和约定\upref{EinSum}证明。

我们另外推荐一种不需要写出分量的推导方法, 见 “一种矢量算符的运算方法\upref{MyNab}”。
