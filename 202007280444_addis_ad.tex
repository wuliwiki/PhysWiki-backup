% latex

\usepackage{slideSFOCS}

\title{\LaTeX: Slides \& Posters}
\author{Donnell Benitez and Ines Scott}
\date{Summer 2020}

%
% presentation specific
%
\definecolor{do}{rgb}{0.5,0.8,.5}
\colorlet{dont}{red!65}

% do  dont frame
\newenvironment{framedd}[1][]{
	\setbeamertemplate{background}{\color{do}\rule{0.5\paperwidth}{\paperheight}\color{dont}\rule{0.5\paperwidth}{\paperheight}}
	\setslidecolor{fg=white,bg=do}
	\setbeamercolor{local structure}{fg=white}
	\begin{frame}[environment=framedd,#1]
		}{
	\end{frame}
}

\newcommand{\lcmd}[1]{{\tt \textbackslash #1}}

%\colortheme{green!48!black}
%\colortheme{orange!80!white}
%\colortheme{blue!45!white}
%\colortheme{black!55!white}

\begin{document}

\maketitle

\mindtoc

\section{Content}

\begin{framef}{side}

	\pause
	\vspace{.225cm}
	\Huge \center See things from the other side
	
\end{framef}

\begin{frame}{Preparing for a presentation}

	Before designing the presentation think of the four ``W'':
	\begin{itemize}\bigsep
		\item \alert{W}ho will be in the audience?
		\item \alert{W}hat is the key message?
		\item \alert{W}hen will the presentation take place?
		\item \alert{W}hy is the message important?
	\end{itemize}

\end{frame}

\begin{frame}{Constructing the presentation}
	General strategy:
	\begin{enumerate}\bigsep 
	\item Select the content based on the four ``W''
	\item Include any background information relevant to the audience
	\item Prepare the outline of the presentation
	\item Carefully tailor the conclusion and introduction
\end{enumerate}

\end{frame}

%\setslidecolor[fg=blue,bg=green]{fg=red,bg=orange}

\begin{frame}{Slides content}
	
	\centering {\em Advertise a work in a visual and oral fashion}
	\bigskip

	\raggedright 
 
	Basic expectations for slides:
	\begin{itemize}\bigsep 
		\item Good oratory skills, i.e. voice and gestures
		\item Clarify the problem and the global strategy
		\item Expose high level explanations on the methodology
		\item Clearly explain the results 
		\item Emphasize the significance of the work
	\end{itemize}

\end{frame}

%\resetslidecolor

\begin{frame}{Poster content}

	\centering {\em Advertise a work in a visual and silent fashion}
	\bigskip

	\raggedright 

	Basic expectations for a poster:
	\begin{itemize}\bigsep
		\item Appealing title, striking pictures with a few explanatory words 
		\item Simple background setup 
		\item Present a high level idea of the work 
		\item Emphasize the significance of the work
	\end{itemize}
	\pause\medskip

	\center \em 
	A poster must be self-explanatory

\end{frame}

\section{Design}

\begin{frame}{Color choice}

The choice of color is of a major importance for a presentation:
\begin{itemize}
	\item Black or dark color: formal, highly technical
	\item Blue or green: peace, tranquility, nature
	\item Gray: conservative, security
	\item Red or orange: passion, warmth, excitement 
	\item Yellow: optimism, happiness
	\item White: purity, simplicity
\end{itemize}
\pause\medskip

\centering 
{\em Ensure the content is easy to read and looks appealing}

\end{frame}

\begin{frame}{Quality figures}

	Depending on the type of figure to display:
	\begin{multicols}{2}
		\begin{itemize}\bigsep
			\item MATLAB
			\item Image manipulation software 
			\item Gnuplot
			\item Spreadsheet 
		\columnbreak
			\item Xfig
			\item Inkscape
			\item Ipe
			\item Ti{\em k}Z \& PGF
		\end{itemize}
	\end{multicols}
	\bigskip

	\centering
	{\em Avoid lossy format such as jpeg, prefer pdf, eps, or png instead}

\end{frame}

\begin{framedd}{Slide design}

	\begin{columns}
		\column{\halfwidth}
			\centering {\large\bf Good}\bigskip

			\raggedright 
			\begin{itemize}\itemsep .75cm
				\item Bullet points
				\item Different colors and fonts 
				\item Add spacing 
			\end{itemize}

		\column{\halfwidth}
			\centering {\large\bf Bad}\bigskip

			\raggedright 
			\begin{itemize}\itemsep .75cm
				\item Whole sentences
				\item Too many different styles
				\item More than $4$-$6$ bullet points
			\end{itemize}
	\end{columns}

\end{framedd}


\begin{framenl}{Slide animations}

	\center
	\includegraphics<beamer:1|handout:0>[scale=.65]{officers1-b}

	\includegraphics<beamer:2|handout:0>[scale=.65]{officers2-b}

	\includegraphics<beamer:3|handout:1>[scale=.65]{officers3-b}

\end{framenl}

\begin{framedd}{Poster design}

	\begin{columns}
		\column{\halfwidth}
		\centering {\large\bf Good}\medskip

		\raggedright 
		\begin{itemize}\bigsep 
			\item Bullet points 
			\item Short introductory sentences 
			\item Different colors and fonts 
			\item Large figures and tables
			\item Boxes to add some structure
		\end{itemize}

		\column{\halfwidth}
		\centering {\large\bf Bad}\medskip 

		\raggedright
		\begin{itemize}\bigsep 
			\item Long sentences
			\item Whole paragraphs
			\item Too many styles
			\item Complicated equations
			\item Follow a very rigid structure
		\end{itemize}
	\end{columns}

\end{framedd}

\section{Beamer}

\begin{frame}{Theming beamer}

	Beamer defines five types of theme:
	\begin{itemize}
		\item Color: colors for the various elements
		\item Font: fonts and font attributes for all the text elements
		\item Inner: elements inside the frame, e.g. enumerate, block, theorem 
		\item Outer: elements outside the frame, e.g. headline, footline, sidebar 
	\item Presentation: predefined set of color, font, inner, and outer themes
	\end{itemize}
	\pause\bigskip

	\centering
	{\em Predefined themes can be  previewed in the~\href{https://www.hartwork.org/beamer-theme-matrix/}{beamer theme matrix}}
	
\end{frame}

\begin{frame}{Basic customisations}

	Adjusting the appearance of the slides:
	\begin{itemize}\bigsep 
		\item Selecting a theme: 
			\begin{multicols}{2}
				\begin{itemize}
			\item \lcmd{usetheme\{name\}}
			\item \lcmd{usecolortheme\{name\}}
			\item \lcmd{useinnertheme\{name\}}
			\item \lcmd{useoutertheme\{name\}}
			\item \lcmd{usefonttheme\{name\}}	
				\end{itemize}
			\end{multicols}
	\item Removing beamer navigation bar: \lcmd{beamertemplatenavigationsymbolsempty}
	\item Providing information on the progress of the presentation: {\small \lcmd{setbeamertemplate\{navigation~symbols\}
		\{\lcmd{insertframenumber}/\lcmd{inserttotalframenumber}\}}}
	\end{itemize}
\end{frame}

\begin{frame}{Basic customisations}

	Colors and background:
	\begin{itemize}\bigsep 
		\item Defining new colors: 
			\lcmd{definecolor\{chocolate\}\{RGB\}\{33,33,33\}} 
		\item Changing the background color:\\
			\lcmd{setbeamercolor\{background canvas\}\{bg=col1, fg=col2\}}
		\item Using a picture as background:
			\begin{itemize}
				\item Image possibly cropped: \lcmd{setbeamertemplate\{background\} \{$\backslash$includegraphics[width=\lcmd{paperwidth},height=\lcmd{paperheight}, keepaspectratio] \{backgroundpicture\}\}}
				\item More flexible solution: use Ti{\em k}Z
		\end{itemize}
			
	\end{itemize}
\end{frame}

\begin{frame}{More advanced customizations}

	New theme templates can be defined:
	\begin{itemize}
		\item All beamer elements can be individually themed
		\item Predefine an option for the template {\tt element}: \lcmd{defbeamertemplate\{element\}\{tname\}\{definition\}}
	\item Set a beamer template: 
		\begin{itemize}
			\item Use a predefined template: \lcmd{setbeamertemplate\{element\}[tname]}
			\item Define a new template: \lcmd{setbeamertemplate\{bname\}\{definition\}}
		\end{itemize}
	\item Insert the content of the {\tt bname} template: \lcmd{usebeamertemplate\{bname\}}
	\end{itemize}
		
\end{frame}


\begin{frame}{Posters with beamer}

	Useful packages:
	\begin{itemize}\bigsep 
		\item Beamerposter: beamer with support for large canvas
		\item Tcolorbox: fancy and very flexible boxes
		\item Ti{\em k}Z \& PGF: basic and advanced \LaTeX~drawing 
		\item Textpos: absolute positioning of text on the canvas
	\end{itemize}
	\pause\bigskip 

	\center \em
	A poster is large, use vector graphics

\end{frame}


\thankframe

\end{document}

