% 平面上的平移、旋转与缩放
% license Usr
% type Tutor

主动理解:

三个变换的顺序不同,只会导致最终结果相差一个平移。例如:

“平移,旋转,缩放” 为
\begin{equation}
s\mat R (\bvec r + \bvec d) = s\mat R\bvec r + s\mat R\bvec d~.
\end{equation}

“旋转,平移,缩放” 为
\begin{equation}
s(\mat R\bvec r + \bvec d) = s\mat R\bvec r + s\bvec d~.
\end{equation}

“缩放,平移,旋转” 为
\begin{equation}
\mat R(s\bvec r + \bvec d) = s\mat R\bvec r + \mat R\bvec d~.
\end{equation}
可见第一项都是相同的,只有第二个常数项不同。

