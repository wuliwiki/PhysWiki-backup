% 级数(极简微积分)
% keys series|级数
% license Xiao
% type Tutor

\begin{issues}
\issueDraft
\end{issues}

\pentry{数列的极限(极简微积分)\nref{nod_Lim0}}{nod_a0d0}

对于一个数列 $a_1,a_2,\dots$,形式表达式
\begin{equation}\label{eq_SerCal_1}
\sum_{n=1}^\infty a_n~
\end{equation}
称为以 $a_n$ 为一般项的\textbf{(无穷)级数 (series)}。 其中前 $N$ 项的和
\begin{equation}
S_N:=\sum_{n=1}^N a_n~
\end{equation}
称为级数 $\sum_{n=1}^\infty a_n$ 的\textbf{部分和 (partial sum)}. 如果部分和序列 $S_1,S_2,\dots$ 有极限, 也就是说存在实数 $S$ 使得
\begin{equation}
\lim_{N\to\infty}\sum_{n=1}^N a_n=S~,
\end{equation}
则称级数\autoref{eq_SerCal_1} 收敛到 $S$ \textbf{(converges to $S$)}, $S$ 称为它的\textbf{和(sum)}。 如果部分和序列不存在极限, 则称级数\textbf{不收敛}或\textbf{发散(divergent)}。

\addTODO{可以看看\enref{级数(分析)}{Series}里还有什么东西可以搬过来}

\subsection{级数的应用}

% Giacomo:为啥要引用这个?
% 小角极限(极简微积分)\nref{nod_LimArc}, 复数\nref{nod_CplxNo}

\pentry{自然对数底(极简微积分)\nref{nod_E}}{nod_f32c}

自然常数 $\E$ 有一个等价定义为
\begin{equation}
\E \equiv \sum_{n=0}^\infty \frac{1}{n!} = 1 + 1 + \frac{1}{2} + \frac{1}{6} + \frac{1}{24} +\dots~,
\end{equation}
这是一个收敛的级数,因为
