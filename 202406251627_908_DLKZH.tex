% 狄拉克之海
% license CCBYSA3
% type Wiki

(本文根据 CC-BY-SA 协议转载自原搜狗科学百科对英文维基百科的翻译)

\begin{figure}[ht]
\centering
\includegraphics[width=8cm]{./figures/21cd607e8e795fc4.png}
\caption{狄拉克之海为一个巨大的粒子。 • 粒子, • 反粒子。} \label{fig_DLKZH_1}
\end{figure}

狄拉克之海是真空的理论模型,它是一个由负能量粒子组成的无限海洋。它首先是由英国物理学家保罗狄拉克在1930年[1] 为了解释狄拉克方程的相对论电子解中出现的反常的负能量态而提出的真空理论假设。[2] 正电子,电子的反物质粒子,在1932年实验发现之前,就被认为是狄拉克海中的一个洞。

在求解3动量平面波动解的自由狄拉克方程时,
$$i \hbar \frac{\partial \Psi}{\partial t} = (c \alpha \cdot \hat{p} + mc^2 \beta) \Psi~,$$
人们发现
$$\Psi_{p \lambda} = N \left( \begin{array}{c}    U \\    \frac{(c \hat{\sigma} \cdot \mathbf{p})}{mc^2 + \lambda E_p} U \end{array} \right)\frac{\exp\left[ i ( \mathbf{p} \cdot \mathbf{x} - \epsilon t ) / \hbar \right]}{\sqrt{2 \pi \hbar}^3}~,$$

其中

$$\epsilon = \pm E_p, \quad E_p = +c \sqrt{\mathbf{p}^2 + m^2 c^2}, \quad \lambda = \text{sgn} \, \epsilon~$$

这是相对论能量动量关系的直接结果

$$E^2 = p^2 c^2 + m^2 c^4~$$



