% 组合
% 高中|排列

我们可以从排列来推导组合,在组合中认为${a,b}$,${b,a}$是等效的,在排列中认为两者是不等的,现在我们有$n$个元素,我们要从中选取$m$个元素,即$C_n^m$.
我们先根据组合的知识可以知道 $A_n^m$ 表示 $n$ 个元素,从中选取 $m$ 个再对选中的 $m$ 个元素进行全排,我们就可以得到如下公式
\begin{equation}
A_n^m = C_n^m A_m^m
\end{equation}
对等式进行变换,可得
\begin{equation}\label{combin_eq1}
 C_n^m = \frac {A_n^m}{A_m^m}
\end{equation}
对于这个公式,我们可以理解为,从排列中排除组合中认为等效的组合.
我们将\autoref{combin_eq1}展开,可得
\begin{equation}\label{combin_eq2} 
C_n^m = \ frac{n(n -1) \ cdots(n -m + 1)}{m(m-1)\ cdots 1}
\ end{equation}

我们将\ autoref{combin_eq3}中的$m$用$n-(n-m) $代换,可得组合的\ textbf{性质1}
\ begin{equation}
C_n^m = \ frac{n!} {(n -m)![n-(n-m)]!} = C_n^{n-m}
\ end{equatio
对于性质1我们可以用一种直观的方式理解,到我们取m个元素时,剩余的元素本身就是取$n-m$个元素时的组合

当我们的总元素个数从$n$编委