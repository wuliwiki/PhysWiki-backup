% 氦原子中的对易算符

\begin{issues}
\issueDraft
\end{issues}

\pentry{类氢原子的束缚态\upref{HWF}}
要思考是否对易, 除了直接计算以外, 也可以思考\autoref{Commut_the1}~\upref{Commut} 中的其他等效条件, 例如是否存在一组共同本征矢, 又例如 $A$ 是否在 $B$ 的本征子空间闭合, 也就是 $A$ 是否会耦合 $B$ 的不同本征值的本征矢($\mel{b_i}{A}{b_j} \ne 0$).

氦原子中总哈密顿算
\begin{equation}
H = H_1 + H_2 + V_{12}~,
\qquad H_i = K_i + \frac{L_i^2}{2r_i^2}
\end{equation}
$K_i$ 是纯径向算符(只和 $r_i$ 有关), 所有的 $L$ 都是纯角向算符(只和角度有关\upref{SphAM}), 纯径向算符和纯角向算符可对易. 不同 $i$ 的任意算符可对易.

所以 $L_1^2, L_2^2, L^2, L_z$ 两两对易, $L_1^2, L_2^2, L_{1z}, L_{2z}$ 也两两对易, $L_z$ 和 $L_{iz}$ 对易, 但 $L^2$ 和单个 $L_{iz}$ 不对易.

比较复杂的是 $V_{12}$, 既有径向也有角向, 且耦合两个电子(\autoref{HeTDSE_eq5}~\upref{HeTDSE})
\begin{equation}
[V_{12}, L^2] = [V_{12}, M] = 0~,
\qquad
[V_{12}, L_i^2] \ne 0~,
\qquad
[V_{12}, L_{iz}] \ne 0
\end{equation}
从经典力学的角度来看这是成立的.

已经数值验证: $\mel*{l'_1,l'_2,L',M'}{\mathcal Y_{l,l}^{0,0}}{l_1,l_2,L,M} = \delta_{L,L'}\delta_{M,M'}$. 说明 $H$ 只会耦合不同的 $l_1,l_2$ 而不会耦合不同的 $L,M$. $H$ 的其他部分不会耦合任何不同的分波. 根据\autoref{Commut_the1}~\upref{Commut} 这说明 $H,L^2,L_z$ 两两对易, 可以在每个 $L,M$ 本征子空间中分别求解能量本征值. 这可以用于束缚态能量求解.

另外容易证明宇称算符 $\Pi$ 和 $H$ 对易, 和 $L^2$、$L_z$ 也对易(用\autoref{GenYlm_eq2}~\upref{GenYlm}\autoref{GenYlm_eq6}~\upref{GenYlm}). 所以 $H,L^2,L_z,\Pi$ 两两对易.
