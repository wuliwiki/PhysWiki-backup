% 等式与不等式(高中)
% keys 方程|不等式|代数基本定理
% license Xiao
% type Tutor

\begin{issues}
\issueDraft
\end{issues}

表示数学运算的符号,如加、减、乘、除以及各种函数等,称作\textbf{运算符}。表示两个数学元素关系的符号,称作\textbf{关系符},比如$>,<,\leq,\geq,\neq$称作\textbf{不等号},$=$称作\textbf{等号}。其他的关系符还有集合中的包含、属于,几何中的平行、垂直,逻辑中的等价、蕴含等。

由数字、变量以及运算符组成的数学符号串,称作\textbf{表达式}。表达式可以看作是对某个值或状态的描述,关键点在于它不需要求解,它只是对某种数学数量的表示。对表达式可以进行简化、合并同类项,或者在特定情况下给出某个变量的值进行替代计算。


% 什么是等式和不等式
用等号连接两个表达式,用于描述相等关系,称作等式。用不等号连接两个表达式,用于描述不等关系,称作不等式。%举例。


\subsection{等式方程与不等式方程}

含有未知数的等式
方程是对于所涉及变量的某些值成立的数学陈述。
另一方面,恒等式是对于所涉及变量的所有允许值都成立的方程。无论其变量的值如何,恒等式都保持相同的相等性。

广泛使用的方程的定义是“含有未知数的等式”,但这样的定义会使人困惑如$x=3$是否也是一个方程。

方程是包含未知数的表达两个代数表达式相等的数学关系等式。

\subsection{解}

方程和不等式只在解或解集中成立。

方程的解可以分为两大类:解析解和数值解。如果方程的解可以通过有限次的常见运算(如加、减、乘、除等)得到,这种解称为\textbf{解析解(Analytical Solution)}。这时,解的表达式可以用代数形式清晰地表示出来。有些复杂的方程很难找到解析解,甚至解析解根本不存在。在这种情况下,可以使用数值分析方法,如二分法、牛顿法等,通过迭代和近似计算来求解方程。此时得到的解称为\textbf{数值解(Numerical Solution)}。数值解通常通过计算机来计算,能够为复杂问题提供高精度的近似解。

总的来说,解析解是精确的,但不总是存在;数值解是近似的,却总是能提供实用的近似结果。在高中阶段,一般只涉及解析解,但存在大量的方程无法获得解析解,或难以获得解析解。

\subsubsection{解与零点}

\begin{definition}{代数学基本定理}
任何一个 $n$ 次多项式函数在复数域上都有 $n$ 个零点(重数计入)。
\end{definition}
这意味着在复数范围内,可以找到所有多项式方程的解。


\subsection{恒成立}

\subsubsection{恒等式}

恒成立的等式称作\textbf{恒等式}。

\subsection{常见恒成立不等式}

与等式的情况相同,存在某些不等式对任意变量值都成立,此时称\textbf{不等式恒成立}。下面介绍一些恒成立的不等式。

\subsubsection{基本不等式}

\begin{equation}
{a^2+b^2\over2}\geq ab~.
\end{equation}

\subsubsection{*柯西不等式}

柯西不等式(Cauchy-Schwarz Inequality)为两个向量或数列的内积与它们的模长之间建立了不等关系。

高中常用的二维模式如下:

\begin{equation}
\left( a^2 + b^2\right) \left(c^2 + d^2 \right) \geq \left( ac+bd \right)^2~.
\end{equation}

多维模式如下:

\begin{equation}
\left( \sum_{i=1}^{n} a_i^2 \right) \left( \sum_{i=1}^{n} b_i^2 \right) \geq \left( \sum_{i=1}^{n} a_i b_i \right)^2~.
\end{equation}

证明:在几何上,柯西不等式可以通过向量内积和向量模的关系得到解释。设 $\mathbf{a}$ 和 $\mathbf{b}$ 是两个向量,则它们的内积可以表示为:

$$\mathbf{a} \cdot \mathbf{b} = \|\mathbf{a}\| \|\mathbf{b}\| \cos \theta~.$$

其中 $\theta$ 是两个向量之间的夹角。根据 $\cos \theta$ 的取值范围,显然有:

$$|\mathbf{a} \cdot \mathbf{b}| \leq \|\mathbf{a}\| \|\mathbf{b}\|~.$$

这就是柯西不等式的向量形式。等号成立的条件是当 $\theta = 0$ 或 $\theta = \pi$ 时,即两个向量平行。

\subsubsection{*排序不等式}

假设有两列数字$a_1,a_2,\cdots,a_n;b_1,b_2,\cdots,b_n$。若$\forall 1\leq i<j\leq n,a_i\leq a_j\land b_i\leq b_j$则称二者为顺序排列,也就是两个序列的增长方向相同。若$\forall 1\leq i<j\leq n,a_i\leq a_j\land b_i\geq b_j$,则称二者为逆序排列,也就是两个序列的增长方向相反。其他排列方式称为乱序排列。排序不等式表示:逆序和$\leq$乱序和$\leq$顺序和,规范表述如下:

\begin{theorem}{排序不等式}
设两列数字 $a_1, a_2, \cdots, a_n$ 和 $b_1, b_2, \cdots, b_n$,两列数分别满足 $a_1 \leq a_2 \leq \cdots \leq a_n$ 和 $b_1 \leq b_2 \leq \cdots \leq b_n$。那么有如下不等式:
\begin{equation}
\sum_{i=1}^n a_i b_i \geq \sum_{i=1}^n a_i b_{\sigma(i)} \geq \sum_{i=1}^n a_i b_{n+1-i}~,
\end{equation}
其中 $\sigma$ 表示对 ${1, 2, \cdots, n}$ 的非顺序、逆序的任意排列,当且仅当$\forall 1\leq i,j\leq n,a_i=a_j$或$\forall 1\leq i,j\leq n,b_i=b_j$时,等号成立。
\end{theorem}

排序不等式可以证明基本不等式、柯西不等式等其他恒成立的不等式。