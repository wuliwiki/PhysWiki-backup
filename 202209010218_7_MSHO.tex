% 多自由度简谐振子(经典力学)
% keys 简谐振子|mode|模

\pentry{简谐振子(经典力学)\upref{SHO}}

\subsubsection{二自由度简谐振子}

考虑轻质弹簧与两个光滑滑块构成的体系,如\autoref{MSHO_fig1} 所示.

\begin{figure}[ht]
\centering
\includegraphics[width=10cm]{./figures/MSHO_1.pdf}
\caption{两个滑块和两根弹簧构成的简谐振子.} \label{MSHO_fig1}
\end{figure}

和\textbf{简谐振子(经典力学)}\upref{SHO}的情况一样,弹簧的原长不重要,因此图中没给出.我们用$x_1$和$x_2$分别表示滑块$m_1$和$m_2$的位置,其中$x_1$表示弹簧$k_1$的长度减去其原长,$x_2$表示弹簧$k_2$的长度减去其原长.显然,当$x_1=x_2=0$时两根弹簧都处于原长.

则这个体系的运动方程为
\begin{equation}
\leftgroup{
    m_1\ddot{x} &= -k_1x_1+k_2x_2\\
    m_2\ddot{x} &= -k_2x_2
}
\end{equation}

当然,我们也可以在













