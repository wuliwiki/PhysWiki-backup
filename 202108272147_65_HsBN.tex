% 二项式定理(高中)
% keys 高中|二项式定理

\begin{issues}
\issueDraft
\end{issues}
\pentry{排列(高中)\upref{Sample}}\pentry{组合(高中)\upref{HsCb}}
\subsection{定义}
\begin{equation}
(a + b)^n = C_n^0a^n + C_n^1a^{n- 1}b + C_n^2a^{n- 2}b^2 + \cdots + C_n^ra^{n-r}b^r + \cdots + C_n^nb^n \qquad (n\in N_{+})
\end{equation}

这个公式所表示的规律叫做\textbf{二项式定理},
等式右边的多项式叫做 $(a+b)^n$ 的\textbf{二项展开式},它一共有 $n+1$项,其中各项系数 $C_n^r(r = 0, 1, \cdots, n)$ 叫做展开式的\textbf{二项式系数}.展开式中的 $C_n^ra^{n-r}b^r$ 项叫做二项展开式的\textbf{通项},通项是展开式的第 $r+1$ 项.

\textsl{注意:二项式系数不是项的系数.}

\subsection{推导}
我们在初中时就学过平方和公式, $(a+b)^2 = a^2 + 2ab + b^2$ 显然这就是一个二项式,我们先从这里开始研究.
\begin{equation}\label{HsBN_eq2}
\begin{aligned}
(a+b)^2 &= a(a + b) + b(a + b)\\
&= a^2 + ab + ab + b^2\\
&= a^2 + 2ab + b^2
\end{aligned}
\end{equation}

我们将\autoref{HsBN_eq2} 的过程图形化.

(图形)

我们会发现在同一深度的次数相同,那二项式系数的问题就转换成了组合问题, $a^rb^{n-r}$ $a$ 和 $b$ 有多少种组合方式.