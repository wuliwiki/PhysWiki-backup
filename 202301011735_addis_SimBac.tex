% 简单稳定的多版本文件备份方案(含 python 实现)

\subsection{最原始的多版本备份方法}
为了防止文件被误删、误改、硬盘损坏等带来的文件丢失, 最普通原始的多版本方法大概要数复制粘贴了。 假设你笔记本上所有重要的文件都在一个文件夹 \verb|我的文件| 中。 为了安全起见你买了一个甚至多个移动硬盘, 每隔一段时间把它复制到硬盘中, 用不同的版本号(例如日期)命名为: \verb|我的文件夹v20230101|, \verb|我的文件夹v20230108| 等等。 但这样做的缺点是大量重复的文件会浪费移动硬盘空间, 写入这些文件也同样会浪费许多时间。 这时你很可能会发现很多支持增量备份的软件。

在进一步讲解各种不同的增量备份方法之前, 我们需要知道计算机文件的构成, 以及如何检查其内容的完整性。

\subsection{计算机文件}
计算机的硬盘中会有不同的文件系统, 例如 Windows 系统盘的 NTFS, Linix 系统常见的 Ext4 等。 它们可能


\subsection{用网盘增量备份}
现在大部分网盘都支持所谓的
