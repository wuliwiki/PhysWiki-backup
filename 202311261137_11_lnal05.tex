% 正交矩阵(高等线性代数)
% license Xiao
% type Tutor


\begin{issues}
\issueTODO
\end{issues}

注:本文参考Jie Peter的《代数学基础》

从矩阵论里我们已经学过,欧几里得空间中正交矩阵的定义:$A^T A=E$。并有充要条件为:该矩阵的列(行)向量组为标准正交基。既是充要条件,出于实用角度,我们可以把正交矩阵直接理解为由标准正交基构成的矩阵,并把“定义”理解为“性质”。本文解释为何欧几里得正交矩阵是如此定义的,以及在其他空间的情况。
\subsection{欧几里得空间}
正交矩阵是正交线性变换的表示。顾名思义,我们希望一组正交基经由线性变换后依然保持正交的性质。其次,我们希望正交矩阵最好构成一个子群,即若干个正交矩阵相乘后依然是正交矩阵。现在我们来探讨一下为何要求起始基和变换后的基都是“标准”的。
所谓标准,即该正交向量组满足内积运算为$e_ie_j=\delta_{ij}$.这组基对应的二次型为$\mathrm {diag}(1,1,1)$。如果我们默认起始的基是标准的,则意味着在矩阵运算时可以省略基的代入。如果不是标准的,那么每次做矩阵运算后都要乘以相应二次型的系数。比如在二次型$\mathrm {diag}(2,1,3)$下,$(1\,0\,0)^T (1\,0\,0)=2$,这是相当麻烦的。


如果允许正交矩阵映射到非标准正交基上,两个正交矩阵相乘结果可能并不是正交矩阵。比如以下矩阵$A,B$。它们都是正交矩阵,然而$AB$并不是。

\begin{equation}
A=\begin{pmatrix}
  1&  1&  0   \\
  1&  -1&  0  \\
  0& 0 & 1   \\
\end{pmatrix},
B=\begin{pmatrix}
  0&  1&  1   \\
  1&  &    \\
  0& 1 & -1   \\
\end{pmatrix}~,
\end{equation}

\subsection{其他空间}