% 轨道方程、比耐公式
% keys 轨道方程|比耐公式

\pentry{中心力场问题\upref{CenFrc}, 二阶常系数非齐次微分方程\upref{Ode2N}}
\subsection{比耐公式}
\footnote{参考 Wikipedia \href{https://en.wikipedia.org/wiki/Binet_equation}{相关页面}。}我们来看 “中心力场问题\upref{CenFrc}” 中得到的两条运动方程(\autoref{CenFrc_eq5} 和\autoref{CenFrc_eq4})
\begin{equation}
\ddot{r} - r \dot\theta^2 = F(r)/m \label{Binet_eq1}~,
\end{equation}
\begin{equation}
mr^2\dot \theta = L \label{Binet_eq2}~.
\end{equation}
为了得到极坐标中 $r(\theta)$ 的微分方程(\textbf{轨道方程}), 我们以下用\autoref{Binet_eq2} 消去\autoref{Binet_eq1} 中的 $t$。 首先可以把 $r$ 看做复合函数 $r[\theta(t)]$, 再用链式法则\upref{ChainR}处理\autoref{Binet_eq1} 的第一项
\begin{equation}\label{Binet_eq10}\ali{
\ddot{r} & = \dv{t} \qty( \dv{r}{t} ) = \dv{t} \qty( \dv{r}{\theta} \dv{\theta}{t} ) = \dv{\theta}\qty( \dv{r}{\theta} ) \qty( \dv{\theta}{t} )^2 + \dv{r}{\theta}\dv[2]{\theta}{t}\\
& = \dv[2]{r}{\theta} \qty( \dv{\theta}{t} )^2 + \dv{r}{\theta}\dv{\theta} \qty( \dv{\theta}{t} )\dv{\theta}{t}~,
}\end{equation}
然后把\autoref{Binet_eq2} 代入\autoref{Binet_eq1} 消去所有 $\dot\theta = \dv*{\theta}{t}$, 得到 $r$ 关于 $\theta$ 的微分方程
\begin{equation}
\dv[2]{r}{\theta} \qty( \frac{L}{r^2} )^2 + \dv{r}{\theta}\dv{\theta} \qty( \frac{L}{r^2} )\frac{L}{r^2} - r \qty( \frac{L}{r^2} )^2 =  m F(r)~.
\end{equation}
即
\begin{equation}\label{Binet_eq5}
\dv[2]{r}{\theta} + r^2\dv{r}{\theta}\dv{\theta} \qty( \frac{1}{r^2} ) - r =  \frac{m r^4}{L^2} F(r)~,
\end{equation}
这就是轨道方程。 这个方程比较复杂, 但可以通过换元法% 未完成:介绍微分方程的换元
化为十分简洁的形式。令
\begin{equation}\label{Binet_eq13}
u \equiv \frac{1}{r}~.
\end{equation}
代入\autoref{Binet_eq5},  得到 $u$ 关于 $\theta $ 的微分方程
\begin{equation}\label{Binet_eq15}
\dv[2]{u}{\theta} + u = -\frac{m}{L^2 u^2} F\qty(\frac 1u)~.
\end{equation}
这是一个阶常系数非齐次微分方程\upref{Ode2N}, 被称为\textbf{比耐公式(Binet equation)}。

在开普勒问题\upref{CelBd}中,相互作用势为 $V(\rho)=-GMm/r$, 可以证明此时轨道的形状是圆锥曲线的一种, 详见 “开普勒第一定律的证明\upref{Keple1}”。

\subsection{变形为一阶方程}
将\autoref{Binet_eq15} 两边同乘 $\dv*{u}{\theta}$,再对 $\theta$ 作积分,可以得到一阶的微分方程:
\begin{equation}\label{Binet_eq3}
\qty(\frac{\dd u}{\dd \theta})^2+u^2=-\frac{2m}{L^2}V\qty(\frac{1}{u})+\frac{2 m E}{L^2}~.
\end{equation}
其中最后一项 $2mE/L^2$ 是积分过程中产生的常量,并且可以验证 $E$ 就是系统的总能量。有了一阶微分方程之后,就可以分离变量法进行积分,求解 $u$ 关于 $\theta$ 的函数,即求解轨道形状。
