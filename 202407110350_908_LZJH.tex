% 粒子聚合
% license CCBYSA3
% type Wiki

(本文根据 CC-BY-SA 协议转载自原搜狗科学百科对英文维基百科的翻译)

颗粒凝聚是指在悬浮液中形成的集合体的过程,代表了导致胶体系统不稳定的一种机制。在此过程中,分散在液相中的颗粒相互粘附,并自发形成不规则的颗粒团、絮状物或聚集体。这种现象也被称为凝结或絮凝,这种悬浮液也被称为不稳定物。颗粒聚集可以通过添加盐类或另一种称为凝结剂或絮凝剂的化学物质来引起。[1] 当凝聚物是由聚合物或聚电解质的加入引起时被称为絮凝,而凝结这一概念包含更广泛的含义。

粒子聚集通常是一个不可逆的过程。一旦颗粒聚集体形成,它们就不会轻易分解。在聚集过程中,聚集物的尺寸会增大,因此它们最终可能会沉淀到容器底部,这种现象被称为沉淀。或者,胶体凝胶可以在浓缩悬浮液中形成,从而改变其流变性质。胶溶是一种向的过程,指颗粒聚集体作为单个颗粒被破坏和分散,这种过程很难自发发生,但可以在搅拌或受剪切力断裂下引起。

胶体颗粒也可能长时间在液体中保持分散,这种分散状态可以保持几天甚至几年。这种现象被称为胶体的稳定性,而这种悬浮液被称为稳定物。稳定的悬浮液可以存在于低盐浓度下,或通过添加稳定剂或有稳定性的化学】品获得。胶体或其他颗粒的稳定性最常用ζ电位来测量。该参数量化了粒子间的斥力测量,这种斥力是抑制颗粒聚集的关键因素。

类似的聚合过程也发生在其他分散的系统中。在乳状液中,粒子也可能与液滴聚结匹配,导致沉淀和乳状液分层。在气溶胶中,同样地,空气中的颗粒也可能聚集并形成更大的团簇,烟灰就是一个例子。

\subsection{早期阶段}
分散良好的胶体悬浮液由单独分离的颗粒组成,颗粒间相互的排斥力使其稳定。当排斥力因为凝结剂的添加而减弱,甚至成为吸引力时,颗粒开始聚集。刚开始,两个单独的$A1$粒子根据以下公式形成$A2$成对物[2]
$A 1 + A 1 \to A 2$
在聚集过程的早期,悬浮液主要含有颗粒单体和一些二聚体。该反应的速率由聚集速率系数$k$表示,因为成对是二阶速率过程,所以该系数的单位是,因为颗粒浓度等于单位体积内()的颗粒数。由于绝对聚集率很难被测量,人们通常参考无量纲稳定比值$W = /k$,其中是快速状态下的聚集率系数,$k$是特定条件下的系数。稳定比在快速状态下接近于1,在慢速状态下增大,在悬浮液稳定时变得非常大。

当粒子之间的相互作用的势能只有纯粹的吸引力时,聚集过程仅受粒子相互扩散(或布朗运动)的限制,又被称作是高速、高频或扩散受限的聚集(DLA)。当相互作用势表现出中间介质的阻止作用时,聚集会因需要多次尝试来克服这类阻止作用而减慢,这又被称作缓慢或反应受限的聚集(RLA)。通过改变盐的浓度、酸碱度或其他添加剂,聚集的速度可以从快到慢进行调节。因为从快速聚集到慢速聚集的转变发生在狭窄的浓度范围内,人们将该范围称为临界凝结浓度(CCC)。
\begin{figure}[ht]
\centering
\includegraphics[width=6cm]{./figures/693158a2989f1f6c.png}
\caption{以上是胶体悬浮液的稳定性与盐浓度关系示意图。} \label{fig_LZJH_1}
\end{figure}
通常,胶体颗粒悬浮在水中。在这种情况下,电荷在粒子的表面累积,并在每个粒子周围形成双电层。[3] 两个逐渐靠近的粒子的扩散层相互重叠,导致了双层排斥力作用的相互势能,最终使得粒子具有稳定性。当悬浮液中加入盐类时,双电层排斥力被屏蔽,范德华引力便占优势,并引起粒子的快速聚集。图中的右边部分显示了典型的稳定比W与电解质浓度的相关性,并由此标出了缓慢和快速聚集的状态。

下表总结了不同负离子净电荷的CCC范围。[4]电荷以单个电荷为基本单位。这种依赖性反映了舒尔茨-哈代法则,[5][6] 该法则指出CCC随着负离子电荷的倒数第六次幂而变化。CCC也多少取决于离子的类型,即使它们携带相同的电荷。这种依赖性可以反映不同的颗粒性质或对颗粒表面的不同离子亲和力。由于大部分粒子带负电荷,因此高价金属阳离子可作为高效凝结剂。
\begin{table}[ht]
\centering
\caption\label{LZJH}
\begin{tabular}{|c|c}
\hline
\textbf{电荷数} & \textbf{CCC(×103摩尔/升)}\\
\hline
1 & 50-300\\
\hline
2 & 2-30\\
\hline
3 & 0.03-0.5\\
\hline
\end{tabular}
\end{table}
带相反电荷的物质的吸附(例如质子,典型的可吸附离子、表面活性剂或聚电解质)可以通过电荷中和,使颗粒悬浮液不稳定,或者通过电荷积累使其稳定,这导致粒子在电荷中和点附近快速聚集,而在远离电荷中和点缓慢聚集。

胶体稳定性的定量解释最初是在DLVO理论中提出的。[2] 这一理论证实了缓慢和快速聚集状态的存在,尽管在缓慢聚集状态下,对盐浓度的依赖性通常被预测为比实验所观察到的强得多。舒尔茨-哈代法则也可以从DLVO理论中推导出来。

胶体的稳定性在其他的机制下也有相同的可能性发生,尤其是涉及聚合物的机制。吸附或接枝聚合物可在颗粒周围形成保护层,产生空间排斥力,并导致空间稳定。聚羧酸酯醚(PCE)就是这种情况,它是最新一代专门设计的化学定制超塑化剂,旨在提高混凝土的可加工性,同时降低其含水量,以改善其性能和耐久性。当聚合物链松散地吸附到颗粒上时,聚合物链可以桥接两个颗粒,并产生桥接力。这种情况被称为桥接絮凝。

当粒子聚集仅由扩散驱动时,被称为围动力学聚集。聚集可以通过分解切应力(例如搅拌)来增强,被称为同向移动的动力学聚集。

\subsection{后期阶段}
\begin{figure}[ht]
\centering
\includegraphics[width=6cm]{./figures/f522f7a5ad033977.png}
\caption{形成的较大聚集体的结构可能会不同。在快速聚集区或DLCA区,聚集区更分散,而在慢速聚集区或RLCA区,聚集区更紧密。} \label{fig_LZJH_2}
\end{figure}
随着聚合过程的继续,会形成更大的集群。这种增长主要是通过不同集群之间的相遇来实现的,因此我们称之为集群和集群的聚合过程。这样得到的聚类是不规则的,但在数据上是自相似的。它们是质量分形的例子,由此它们的质量随着典型尺寸成比例增长,可以被旋转半径的$d$次幂[2]描述:
