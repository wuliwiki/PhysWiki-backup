% 数列的极限(简明微积分)
% keys 微积分|极限|数列极限|函数极限|无穷小

微积分的核心概念是极限,而极限最基础的情形是数列的极限.数列是离散的,比较容易理解,而所有与极限有关的概念也都可以从数列的极限拓展得到. 先来看一个数列的例子.

\begin{example}{}\label{Lim0_ex1}
我们都知道 $\pi$ 是一个无理数,所以 $\pi$ 的小数部分是无限多的.目前用计算机,已经可以将 $\pi$ 精确地计算到小数点后数亿位.然而在实际应用中,往往只用取前几位小数的近似即可.下面给出一个数列,定义第 $n$ 项是 $\pi$ 的前 $n$ 位小数近似(不考虑四舍五入),即
\begin{equation}\label{Lim0_eq1}
a_0 = 3,\,\, a_1 = 3.1,\,\, a_2 = 3.14,\,\, a_3 = 3.141,\,\dots.
\end{equation}

这个数列显而易见的性质,就是当 $n$ \textbf{趋于无穷}时,$a_n$ \textbf{趋于} $\pi$. \textbf{无穷}用符号 $\infty$ 来表示. 我们把这类过程叫做\textbf{极限}. 以上这种情况,用极限符号表示,就是
\begin{equation}
\lim_{n \to \infty } {a_n} = \pi 
\end{equation}
这里 $\lim$ 是极限(limit)的意思,下方用箭头表示某个量变化的趋势. $\lim\limits_{n \to \infty }$ 在这里相当于一个“操作”,叫\textbf{算符(operator)}, 它作用在数列 $a_n$ 上, 输出一个数, 即极限值 $\pi$.
\end{example}

不要误以为这条式子是说当 $n = \infty$ 时, $a_n=\pi$. 有两个理由可以说明这种理解不正确: 首先, $\infty$ 不是一个整数或实数, 不存在 $n=\infty$ 的说法, 这里的$n\to\infty$只是表示 $n$ 的不断增加的过程. 其次, 按照定义每个 $a_n$ 都是有理数(因为他们都只有有限位小数),而 $\pi$ 是无理数, 所以不应该有任何一个 $a_n=\pi$. 类比函数 $\sin x = y$ 并不是说 $x = y$, 而是说 $x$ 经过正弦函数作用后等于 $y$.

所以从概念上来说,极限中的 “趋于” 和“等于” 是不同的.趋于是数列整体的性质,而不是某一个项性质.

容易知道, 数列的极限和前面有限项的数值都无关, 例如把\autoref{Lim0_eq1} 中的前三项改成 $0$, 那么该数列的极限仍然是 $\pi$.

当然, 数列极限也并不要求 $n\to \infty$ 时数列的项不能等于极限, 例如数列
\begin{equation}
b_0 = 3.3,\,\, b_1 = 3.2, \,\, b_2 = 3.1, \,\, b_n = \pi \,\, (n > 2)
\end{equation}
当 $n > 2$ 时所有的项都等于 $\pi$, 那么他的极限显然也是 $\pi$, 尽管这样的极限十分无趣.

要给数列的极限一个一般的定义, 必须要体现 “$n$ 越大, 后面的项就越接近极限值” 这一思想. 把数列记为 $\{a_n\}$, 极限值记为实数 $A$, 那么每一项 $a_n$ 和极限值 $A$ 的接近程度, 也就是距离, 可以用绝对值 $\abs{a_n - A}$ 来衡量. 例如在\autoref{Lim0_eq1} 中, 当 $n \ge 2$ 就可以保证 $\abs{a_n - A} < 0.01$, 当 $n \ge 3$ 就可以保证 $\abs{a_n - A} < 0.001$, 等等. 不管我们要求这个距离多么近, 总可以找到某个具体的 $N$, 当 $n > N$ 时, 后面所有的项和 $A$ 的距离都会满足这个要求. 我们把要求的距离记为 $\epsilon$, 那么就给出正式的定义了.

\begin{definition}{数列的极限}\label{Lim0_def2}
考虑数列 $\{a_n\}$. 若存在一个实数 $A$,使得对于\textbf{任意}给定的\textbf{正实数} $\varepsilon > 0$(无论它有多么小),总存在正整数 $N_\epsilon$, 使得对于所有编号 $n>N_\epsilon$ ,都有 $\abs{a_n - A} < \varepsilon$ ($A$ 为常数) 成立,那么数列 $a_n$ 的极限就是 $A$.

如果一个数列$\{a_n\}$不存在极限,就称它是\textbf{发散(divergent)}的.如果$\{a_n\}$存在极限,则称它是\textbf{收敛(convergent)}的.
\end{definition}
由于以上讨论中 $\lim$ 作用的对象是数列,那么箭头右边只能是 $\infty$ (准确来说应该是正无穷 $+\infty$, 但是由于数列的项一般是正的,所以正号省略了). 可以证明\autoref{Lim0_ex1} 的极限符合数列极限的定义.

我们来看几个简单的例题,加深一下印象. 注意不是任何数列都存在极限.

\begin{example}{}
可以证明 $\lim\limits_{n\to\infty}1/2^n = 0$.
\end{example}

\begin{example}{}\label{Lim0_exe1}
考虑数列 $a_n=(-1)^n$. 显然这个数列存在极限不存在.
\end{example}
