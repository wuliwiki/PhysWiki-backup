% 中山大学 2012 年913专业基础(数据结构)考研真题
% keys 中山大学 2012 年913专业基础(数据结构)考研真题


\subsection{一、单项选择题(每题2分,共40分)}

1.算法复杂度通常是表达算法在最坏情况下所需要的计算量,0(1)的含义是( ) \\
(A).算法执行1步就完成 \\
(B).算法执行1秒钟就完成 \\
(C).解决执行常数步就完成 \\
(D).算法执行可变步数就完成

2.在数据结构中,按逻辑结构可把数据结构分为( ) \\
(A).静态结构和动态结构 \\
(B).线性结构和非线性结构 \\
(C).顺序结构和链式结构 \\
(D).内部结构和外部结构

3.在数据结构中,可用存储顺序代表逻辑顺序的数据结构为( ) \\
(A). Hash表 \\
(B).二叉搜索树 \\
(C).链式结构 \\
(D).顺序结构

4.对链式存储的正确描述是( ) \\
(A).结点之间是连续存储的 \\
(B).各结点的地址由小到大 \\
(C).各结点类型可以不一-致 \\
(D).结点内单元是连续存储的

5.在下列关于“串”的陈述中,正确的说明是( ) \\
(A).串是一种特殊的线性表 \\
(B).串中元素只能是字母 \\
(C).串的长度必须大于零 \\
(D).空串就是空白串

6.关于堆栈的正确描述是( ) \\
(A). FILO \\
(B). FIFO \\
(C).只能用数组来实现 \\
(D).可以修改栈中元素的数据

7. 假设循环队列的长度为QSize. 当队列非空时,从其队列头取出数据后,其队头下标Front的变化为() \\
(A). Front = Front+1 \\
(B). Front= (Front+ 1)\% 100 \\
(C). Front= (Front + 1) \% QSize \\
(D). Front = Front \% Qsize + 1

8.假设Head是带头结点单向循环链的头结点指针,判断其为空的条件是( ) \\
(A). Head.next = NULL \\
(B). Head->next == Head \\
(C). Head->next == NULL \\
(D). Head == NULL

9.设A[n][n]为一个对称矩阵,数组下标从[0][0]开始.为了节省存储,将其下三角部分按行存放在一维数组B[0..m-1],m=n(n+1)/2, 对下三角部分中任一元素$A_{ij}(i\geqslant j)$它在一-维数组B的下标k值是( ) \\
(A). (-1)/2+j \\
(B). (-1)2+)-1 \\
(C). (+1)2+j-1 \\
(D). i(i+1)/2+j

10.假设二又树的根结点为第0层,那么,其第i层(20)的结点数最多为( ) \\
(A). $2i$ \\
(B). $2^i$ \\
(C). $2^{i+1}-1$ \\
(D). $2^{i+1}$

11.若一棵二叉树的后序和中序序列分别是dbefca和dbacef,则其先序序列是( ) \\
(A). adbefc \\
(B). abdcfe \\
(C). adbcef \\
(D). abdcef

12.用一维数组来存储满二叉树,若数组下标从0开始,则元素下标为k的右子结点下标是( )(不考虑数组下标的越界问题) \\
(A). 2k+1 \\
(B). 2k+2 \\
(C). $\lfloor k/2 \rfloor$ \\
(D). $\lceil k/2 \rceil$

13. 假设LTree和RTree是二叉搜索树Tree的左右子树,H(T)表示树T的高度.若树Tree是AVL树,则( ) \\
(A). H(LTree)-H(RTree)=0
(B). H(LTree)- H(RTree)<1
(C). H(LTree) - H(RTree)<= 1
(D). |H(LTree)- HRTree)| <= 1

14. 对n个结点和e条边的无向图(无环),其邻接矩阵中零元素的个数为( ) \\
(A).$e$ \\
(B).$2e$ \\
(C).$n^2-e$ \\
(D).$n^2-2e$

IS.用邻接矩阵存储有n个顶点和e条边的有向閔,则删除与某个顶点相邻的所有边的时间复杂度是() \\
(A). $O(m)$ \\
(B). $O(e)$ \\
(C). $O(n+e)$ \\
(D). $O(ne)$

16. 下列播序算法中,时间复杂度最差的是( ) \\
(A).选择排序 \\
(B).桶(基數)排序 \\
(C).快速排序 \\
(D).堆排序

17. 基于比较的排序算法对n个数进行排序的比较次数下界为( ) \\
(A). $O(logn)$ \\
(B). $O(m)$ \\
(C). $O(nlogn)$ \\
(D). $O(n^2)$

18. 在下列存储条件下,( )是最适合使用折半查找算法来进行查找操作. \\
(A).顺序存储 \\
(B).链式存储 \\
(C).散列存储 \\
(D),数据有序且顺序存储

I9. 在下列算法中,求图最小生成树的算法是( ) \\
(A). DFS算法 \\
(B). KMP算法 \\
(C). Prim算法 \\
(D). Dijkstra算法

20. 若结点的存储地址与其关键字之间存在某种映射关系,则称这种存储结构为( ) \\
(A).顺序存储结构 \\
(B).链式存储结构 \\
(C).散列存储结构 \\
(D).索引存储结构

\subsection{二、解答题(每题10分,共50分)}

1.假设有如图1所示的图 \\
\begin{figure}[ht]
\centering
\includegraphics[width=5cm]{./figures/SYDS12_1.png}
\caption{第二1题图} \label{SYDS12_fig1}
\end{figure}
(1)写出图1的邻接矩阵; \\
(2)根据邻接矩阵从顶点a出发进行宽度(或广度)优先遍历,面出相应的宽度优先遍历树(同一个结点的邻接结点按结点序号大小为序).


2.简单描述求图最小生成树的Kruskal算法(克鲁斯卡尔算法)的基本思想,并按步骤列出图2的最小生成树的求解过程.
\begin{figure}[ht]
\centering
\includegraphics[width=8cm]{./figures/SYDS12_2.png}
\caption{第二2题图} \label{SYDS12_fig2}
\end{figure}

3.简单叙述快速排序的思想,在“第一个元素为支点”前提下按步骤列出下列序列的排序过程. \\
待排序的数值序列: 45 12 56 87 34 78

4.已知有下列13个元索的散列表:
\begin{figure}[ht]
\centering
\includegraphics[width=12cm]{./figures/SYDS12_3.png}
\caption{第二4题图} \label{SYDS12_fig3}
\end{figure}
其散列函数为K(key) = key \% m (m= 13).处理冲突的方法为双重散列法,探查序列为:  \\
$h_i=(h(key)+i*h(key))\%m$ $\qquad$ $i=0,1,...,m-1$  其中: $h(key)=key\%11+1$ \\
问:对表中关键字35进行查找时,所需进行的比较次数为多少?依次写出每次的计算公式和值.

5.假设设在通信中,字符a, b,c,d, e,,g出现的频率如下: \\
a: 20\% b: 7\% c:16\% d: 27\% e:7\% f 10\% g: 13\% \\
(1)根据Huffinan算法(赫夫曼算法)画出其赫夫曼树: \\
(2)给出每个字母所对应的赫夫曼编码,规定:结点左分支边上标0,右分支边上标1; \\
(3)计算其加权路径的长度WPL.

\subsection{三、阅读理解题,按空白编号填写相应的C语言语句,以实现函数功能.(每空2分,每题10分,共30分)}

1.排队是日常生活中常见的一种现象,比如:在商店排队付款.当第- -位顾客完成付款离开后,其他顾客依次前移.下面用数据结构中的队列来模拟这种排队现象. \\
\begin{lstlisting}[language=cpp]
# define QUEUE
struct Queue {
    int queue[QUEUE];
    int Rear;    // Rear记录队列尾
};
\end{lstlisting}
(1)初始化队列Q
\begin{lstlisting}[language=cpp]
void InitQueue(Queue *Q)
{
    Q->Rear=-1;
}
\end{lstlisting}

(2)入队操作EnQueue(Q, dat):若队列Q已满,返回0, 否则,把数据data加入队列Q,并返回1
\begin{lstlisting}[language=cpp]
int EnQueue(Queue *Q, int *data)
{
    if(___(1)___) return 0;
    ____(2)____;
    Q->queue[Rear] = data;
    return 1;
}
\end{lstlisting}

(3)出队操作DeQueue(Q, dat);若队列Q为空,则返回0,否则,把队头元素存入地址参数data,然后从队列Q中去除该队头元素,并返回1
\begin{lstlisting}[language=cpp]
int DeQueue(Queue *Q, int *data)
{
    if(Q->Rear==-1) return0;
    *data=____(3)____;
    for(i=0;i<Q->Rear;i++)  ____(4)____ ;
    ____(5)____;
    return 1;
}
\end{lstlisting}
这二个堆栈在某个时刻的状态如下图所示.
\begin{figure}[ht]
\centering
\includegraphics[width=12cm]{./figures/SYDS12_4.png}
\caption{第三1题图} \label{SYDS12_fig4}
\end{figure}
(1)初始化堆栈
\begin{lstlisting}[language=cpp]
void InitStacks(Stacks *stack)
{
    stack->Top1 =___(1)___;
    stack->Top2 =___(2)___;
}
\end{lstlisting}
(2)堆栈l(左堆栈)压栈操作
\begin{lstlisting}[language=cpp]
int push1(Stacks *stack, int data)
{
    if( __(3)__ )return 0;
    stack->Top1++;
    Elements[stack->Top1]= data;
    return 1;
}
\end{lstlisting}
(3)堆栈2(右堆栈)出栈操作,并把栈顶元素的值赋给指针变量data所指向的存储单元
\begin{lstlisting}[language=cpp]
BOOL pop2(Stacks *stack, int *data)
{
    if(___(4)___)return 0;
    *data = Elements[stack-> Top2];
    ____(5)____;
    return 1;
\end{lstlisting}
