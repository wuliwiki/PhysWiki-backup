% 三重积
% keys 向量三重积|标量三重积|triple product
% license Usr
% type Tutor

\pentry{几何向量的点乘\nref{nod_Dot},几何向量的叉乘\nref{nod_Cross}}{nod_744a}




在\textbf{三维欧几里得空间}中,有一类常用的特殊运算,称为\textbf{三重积(triple product)},就是选三个向量进行点乘或叉乘。三重积分为两类:标量三重积的结果是标量,向量三重积的结果是向量。




\subsection{标量三重积}


取三个向量$\bvec{a}, \bvec{b}, \bvec{c}$,定义它们的标量三重积为$\bvec{a}\cdot\qty(\bvec{b}\times\bvec{c})$。$\bvec{b}\times\bvec{c}$是向量$\bvec{b}$和向量$\bvec{c}$的叉乘,结果是向量;于是$\bvec{a}\cdot\qty(\bvec{b}\times\bvec{c})$是向量$\bvec{a}$和向量$\bvec{b}\times\bvec{c}$的点乘,结果是标量。

如果将三个向量分别展开为直角坐标,如下:
\begin{equation}
\bvec{a}\sim\pmat{a_1\\a_2\\a_3}, \bvec{b}\sim\pmat{b_1\\b_2\\b_3}, \bvec{c}\sim\pmat{c_1\\c_2\\c_3}, ~
\end{equation}
那么根据叉乘和点乘的计算公式可得
\begin{equation}
\begin{aligned}
\bvec{a}\cdot\qty(\bvec{b}\times\bvec{c}) \sim{}& \pmat{a_1\\a_2\\a_3}\cdot\qty(\pmat{b_1\\b_2\\b_3}\times \pmat{c_1\\c_2\\c_3})\\
={}& \pmat{a_1\\a_2\\a_3}\cdot\pmat{b_2c_3-b_3c_2\\b_3c_1-b_1c_3\\b_1c_2-b_2c_1}\\
={}& \pmat{a_1b_2c_3-a_1b_3c_2\\a_2b_3c_1-a_2b_1c_3\\a_3b_1c_2-a_3b_2c_1}. 
\end{aligned}~
\end{equation}

标量三重积有直观的几何意义。考虑一个平行六面体,取其一个顶点作为原点,过此顶点的三条边作为三个向量,如\autoref{fig_trpprd_1} 所示。




\begin{figure}[ht]
\centering
\includegraphics[width=8cm]{./figures/a3cc16a5159b6752.pdf}
\caption{标量三重积示意图。图中$c_\parallel$是向量$\vec{c}$垂直于$\vec{a}, \vec{b}$平面的分量。} \label{fig_trpprd_1}
\end{figure}

$\color{NavyBlue}\vec{a}\color{Black}\times\color{Red}\vec{b}\color{Black}$是垂直于$\color{NavyBlue}\vec{a}\color{Black}, \color{Red}\vec{b}\color{Black}$平面的向量,其长度恰为以$\color{NavyBlue}\vec{a}\color{Black}, \color{Red}\vec{b}\color{Black}$为边的平行四边形的面积。$\abs{\qty(\color{NavyBlue}\vec{a}\color{Black}\times\color{Red}\vec{b}\color{Black})\cdot\color{OliveGreen}\vec{c}\color{Black}}$等于$\color{NavyBlue}\vec{a}\color{Black}\times\color{Red}\vec{b}\color{Black}$的长度乘以$\color{OliveGreen}\vec{c}\color{Black}$的长度,





\subsection{向量三重积}




\begin{figure}[ht]
\centering
\includegraphics[width=12cm]{./figures/23207c1117f04b94.pdf}
\caption{向量三重积示意图。} \label{fig_trpprd_2}
\end{figure}

















