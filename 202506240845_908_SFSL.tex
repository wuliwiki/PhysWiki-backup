% 索菲斯·李(综述)
% license CCBYSA3
% type Wiki

本文根据 CC-BY-SA 协议转载翻译自维基百科 \href{https://en.wikipedia.org/wiki/Sophus_Lie}{相关文章}。

\begin{figure}[ht]
\centering
\includegraphics[width=6cm]{./figures/a7728e1c8b169d85.png}
\caption{} \label{fig_SFSL_1}
\end{figure}
马留斯·索福斯·李(Marius Sophus Lie,/liː/,挪威语:[liː];1842年12月17日-1899年2月18日)是一位挪威数学家。他在连续对称性理论方面做出了奠基性的贡献,并将其应用于几何和微分方程的研究中。他还在代数学的发展中做出了重要贡献。
\subsection{生平与职业生涯}
马留斯·索福斯·李于1842年12月17日出生在挪威小镇诺尔德菲尤尔。他是路德教牧师约翰·赫尔曼·李与其妻子所生的六个孩子中最小的一个,母亲出身于特隆赫姆的一个知名家族。[1]

他在挪威东南部的莫斯接受了初等教育,之后在奥斯陆(当时称为克里斯蒂安尼亚)读高中。高中毕业后,他原本希望投身军事事业,但由于视力不佳而被军队拒绝,于是改而进入皇家弗雷德里克大学(即今日奥斯陆大学)就读。

索福斯·李的第一篇数学论文《平面几何中虚数的表示》于1869年由克里斯蒂安尼亚科学院与《克雷勒期刊》共同发表。同年,他获得了一项奖学金前往柏林,自9月起在当地停留至1870年2月。在那里,他结识了费利克斯·克莱因,两人迅速成为挚友。离开柏林后,李前往巴黎,克莱因也于两个月后与他会合。在巴黎,他们结识了卡米耶·若尔当与加斯顿·达布。然而,1870年7月19日,普法战争爆发,克莱因因身为普鲁士人而不得不迅速离开法国。李则前往枫丹白露,却被误认为是德国间谍而遭到逮捕,这在挪威为他带来了一定的名声。在达布的干预下,李被关押一个月后获释。[2]

李在1871年于皇家弗雷德里克大学(今奥斯陆大学)获得博士学位,其博士论文题为《一类几何变换》(挪威语:Over en Classe geometriske Transformationer,英文:On a Class of Geometric Transformations)。[3] 这篇论文后来被达布称为“现代几何中最优美的发现之一”。次年,挪威议会为他特别设立了一个教授职位。同年,李拜访了正在埃尔朗根任教的克莱因,后者当时正在发展著名的“埃尔朗根纲领”。

1872年,李与彼得·路德维希·迈德尔·西洛一起合作,花了八个月时间编辑并出版了挪威同胞尼尔斯·亨里克·阿贝尔的数学著作。

1872年底,索福斯·李向当时年仅18岁的安娜·比奇求婚,并于1874年结婚。这对夫妇育有三个孩子:玛丽(Marie,生于1877年)、达格妮(Dagny,生于1880年)和赫尔曼(Herman,生于1884年)。

从1876年起,他与医生雅各布·沃姆-穆勒以及生物学家乔治·奥西安·萨尔斯共同编辑《数学与自然科学档案》期刊。

1884年,在克莱因和阿道夫·迈耶(当时二人均为莱比锡大学教授)的支持下,弗里德里希·恩格尔来到克里斯蒂安尼亚(今奥斯陆)协助李。恩格尔帮助李撰写了他最重要的著作《变换群理论》,该书分三卷于1888年至1893年间在莱比锡出版。数十年后,恩格尔也成为李全集的两位编辑之一。

1886年,李接替前往哥廷根任教的克莱因,成为莱比锡大学的教授。1889年11月,李突发精神崩溃,被迫住院治疗,直到1890年6月才出院。出院后他虽然重返岗位,但随着贫血症状的加重,他最终不得不返回故乡。1898年5月,他递交辞呈,并于当年9月离开莱比锡回国。次年,即1899年,李因恶性贫血去世,享年56岁。这种疾病是由于维生素B12吸收障碍所致。

李在其生前获得多项荣誉:1878年被授予伦敦数学学会名誉会员,1892年成为法国科学院通讯院士,1895年被选为英国皇家学会外籍会员,同年亦当选为美国国家科学院外籍院士。
\begin{figure}[ht]
\centering
\includegraphics[width=6cm]{./figures/a2c3f983d47e5724.png}
\caption{《变换群理论》第一卷,1888年版。} \label{fig_SFSL_2}
\end{figure}
\begin{figure}[ht]
\centering
\includegraphics[width=6cm]{./figures/2929c736d07ed25e.png}
\caption{《变换群理论》标题页} \label{fig_SFSL_3}
\end{figure}
\begin{figure}[ht]
\centering
\includegraphics[width=6cm]{./figures/fe4fa35fd71ead25.png}
\caption{《变换群理论》前言} \label{fig_SFSL_4}
\end{figure}
\subsection{遗产}
Lie 的主要工具,也是他最伟大的成就之一,是他发现连续变换群(后来以他的名字命名为 Lie 群)可以通过“线性化”来更好地理解,即研究对应的生成向量场(即所谓的无穷小生成元)。这些生成元服从群律的线性化形式,即现在所称的对易括号,从而构成了我们今天所说的 Lie 代数的结构。[4][5]

Hermann Weyl 在 1922 年和 1923 年的论文中使用了 Lie 的群论研究成果,而 Lie 群今天在量子力学中发挥着重要作用。[5] 然而,今天对 Lie 群的研究与 Sophus Lie 当年的研究方向已有很大不同,正如有人所说:“在 19 世纪的数学大师中,Lie 的工作无疑是今天最不为人所熟知的。”[6]

Sophus Lie 积极倡导设立阿贝尔奖。他受到以 Fridtjof Nansen 命名的南森基金启发,并注意到诺贝尔奖未设立数学奖项,因此他致力于推动设立一个表彰纯数学杰出工作的奖项。[7]

Lie 指导了许多后来成为著名数学家的博士生。Élie Cartan 被广泛认为是 20 世纪最伟大的数学家之一。Kazimierz Żorawski 的工作在多个领域被证明具有重要意义。Hans Frederick Blichfeldt 对多个数学领域做出了贡献。
\subsection{书籍}
\begin{itemize}
\item Lie, Sophus(1888年),《变换群理论 I》(德语),莱比锡:B. G. Teubner 出版社。由 Friedrich Engel 协助撰写。有英文翻译版:Joël Merker 编辑并从德语翻译并撰写前言,详见 ISBN 978-3-662-46210-2 及 arXiv:1003.3202。
\item Lie, Sophus(1890年),《变换群理论 II》(德语),莱比锡:B. G. Teubner 出版社。由 Friedrich Engel 协助撰写。
\item Lie, Sophus(1891年),《已知无穷小变换的微分方程讲义》(德语),莱比锡:B. G. Teubner 出版社。由 Georg Scheffers 协助撰写。[8]
\item Lie, Sophus(1893年),《连续群讲义》(德语),莱比锡:B. G. Teubner 出版社。由 Georg Scheffers 协助撰写。[9]
\item Lie, Sophus(1893年),《变换群理论 III》(德语),莱比锡:B. G. Teubner 出版社。由 Friedrich Engel 协助撰写。
\item Lie, Sophus(1896年),《接触变换几何学》(德语),莱比锡:B. G. Teubner 出版社。由 Georg Scheffers 协助撰写。[10]
\item Lie, Sophus;Engel, Friedrich;Heegaard, Poul(编辑),《Lie 论文集》(德语),莱比锡:Teubner 出版社;共7卷,出版时间为1922–1960年。[11][12]
\end{itemize}
\subsection{另见}
\begin{itemize}
\item Lie 导数
\item 单纯 Lie 群列表
\item 以 Sophus Lie 命名的事物列表
\end{itemize}