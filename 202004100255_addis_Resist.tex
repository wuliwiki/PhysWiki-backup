% 电阻
% 电阻|导体|电流|电压|欧姆定律

\pentry{电流\upref{I}, 电势 电势能\upref{QEng}}

\begin{equation}
U = IR
\end{equation}


\subsection{电阻的简单模型}
平行板电容器, 中间有某种均匀的导电材料, 该材料中自由电子的电荷密度 $\rho < 0$ 为定值, 每个电子受到的阻力与电子速度成正比, 比例常数 $\alpha > 0$. 即
\begin{equation}
f = -\alpha v
\end{equation}
当我们在电容器两板上施加电压时, 内部会产生匀强电场, 使电子受到电场力
\begin{equation}
F = -Ee
\end{equation}
电子在该电场力下加速(由于电子质量很小, 这个过程很快可以认为是一瞬间完成的), 直到阻力等于电场力时加速停止, 进行匀速运动. 于是有
\begin{equation}
Ee = \alpha v
\end{equation}
所以电阻内电流密度大小为
\begin{equation}
j = \rho v = \frac{\rho Ee}{\alpha}
\end{equation}
电流为
\begin{equation}
I = jS = \frac{\rho EeS}{\alpha}
\end{equation}
电阻两端电压为
\begin{equation}
U = EL
\end{equation}
带入上式得
\begin{equation}
U = I \frac{\alpha L}{\rho eS}
\end{equation}
我们定义\textbf{电阻率}为
\begin{equation}
\varrho = \frac{\alpha}{\rho e}
\end{equation}
这是一个和材料性质(可能会随温度,压强,光照,等环境因素变化)有关的常数. 定义电阻为
\begin{equation}
R = \frac{\varrho L}{S}
\end{equation}
可见电阻和电阻率


电容率 $\rho$, 
\begin{equation}
R = \rho L / S
\end{equation}
