% 等距变换
% license Usr
% type Tutor

\pentry{伴随映射\nref{nod_Sample}}{nod_592f}\pentry{正交空间与辛空间\nref{nod_OrSp}}{nod_2cce}
\begin{issues}
\issueDraft 向量
\end{issues}


\begin{definition}{}
$(V,f),(V',f')$是两个正交空间或辛空间。若线性映射$\sigma:(V,f)\rightarrow (V',f')$是双射且保内积不变,即\begin{equation}
f(x,y)=f'(\sigma x,\sigma y)~,
\end{equation}
则称$\sigma $是\textbf{等距映射}(isometry)。若$\sigma:(V,f)\rightarrow (V,f)$,则称之为等距变换。
\end{definition}
由于等距变换是线性映射,因此保加法和数乘运算,是全体向量的自同构映射。由定义式可知,等距变换复合依然是等距变换。因此全体等距变换的集合可记作$\opn{Aut}(V,f)$,表明其自同构成群和保内积的性质。从正交变换的\autoref{def_ortho_1}~\upref{ortho}可知,正交变换是度量矩阵为$E$的等距变换,因此我们可以拓展正交变换的定义为\textbf{非退化}正交空间的等距变换,并称\textbf{非退化}辛空间的等距变换为\textbf{辛变换}。

等距变换的保内积性质意味着其必然对度量矩阵作出限制。设$\sigma,f$的矩阵表示分别为$A$和$G$,任意向量$x,y\in V$,则有$x^{T}Gy=(Ax)^TG(Ay)$,也就是说:
\begin{equation}
A^TGA=G~.
\end{equation}

若$(V,f)$是非退化的正交空间和辛空间,则其上的线性变换都可以诱导出伴随变换,所以若设等距变换的伴随变换为$\sigma':V\rightarrow V$,则有$f(\sigma(x),y)=f(x,\sigma'(y))$,因此其伴随变换的形式必然受度量矩阵约束。
\begin{theorem}{}
设$(V,f)$为n维非退化正交空间或辛空间,$\sigma$是双射线性变换,$\sigma$是其伴随变换。则$\sigma$是等距变换当且仅当$\sigma'\sigma=1$
\end{theorem}
\textbf{证明:}

设$x,y$是任意两个向量,由定义得:$f(\sigma(x),\sigma(y))=f(x,\sigma'\sigma(y))=f(x,y)$。

因此,$f(x,\sigma'\sigma(y)-y)=0$对任意$x$成立。由于非退化线性空间的根只有零向量,所以$\sigma'\sigma(y)=y$,得证。