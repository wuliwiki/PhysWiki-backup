% 金属的变形(科普)

\subsection{变形}
正如我们拉伸一根弹簧,弹簧会变形一样,当我们拉伸一根金属棒时,金属棒也会变形.(只不过由于金属棒的“弹性系数”很大,以正常人的手劲一般拉不出可以明显察觉的变形)

\begin{example}{}
\begin{figure}[ht]
\centering
\includegraphics[width=12cm]{./figures/MetDfm_1.png}
\caption{框架结构}} \label{MetDfm_fig1}
\end{figure}
事实上,与弹簧类似,金属的支持力也源自金属的细微变形...只要在安全的范围内.
\end{example}

根据变形的机制,一般把变形分为两类:弹性变形与塑形变形.顾名思义,弹性变形指撤去外力后金属的形状能恢复原样;而塑形变形后即使撤去外力,金属的形状也不能恢复.

塑形变形与弹性变形不是非此即彼,而是相辅相成的.变形可以既包括弹性形变也包括塑形形变.但一般而言,塑形形变只在外力超过一定限度时发生.
\begin{figure}[ht]
\centering
\includegraphics[width=5cm]{./figures/MetDfm_2.png}
\caption{请添加图片描述} \label{MetDfm_fig2}
\end{figure}