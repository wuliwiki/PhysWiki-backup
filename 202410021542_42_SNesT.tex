% 球套定理
% keys 完备性|球套
% license Usr
% type Tutor
\pentry{柯西序列、完备度量空间\nref{nod_cauchy}}{nod_55e7}
在分析学中,所谓区间套定理(\autoref{the_RCompl_3})被广泛的应用。在度量空间理论中,本节所谓的球套定理也其中类似的作用。

首先需要明确度量空间 $(X,d)$ 的球 $B$ 是指:以某点 $x\in X$ 为中心,正实数 $r\in\mathbb R^+$ 为半径的集合 
\begin{equation}

\end{equation}
B_{r_i}(x_0):=B(x_0,r_i):=\{x|d(x,x_0)\leq(or <)r_i,x\in X\}$
\begin{definition}{球套}
设 $X$ 是度量空间,序列 $\{B_n\}$ 中的每一个 $B_i$ 都是 $X$ 中的球。
\end{definition}

\begin{theorem}{球套定理}
度量空间 $X$ 是完备的充要条件是:$R$ 中半径趋于0的闭球套的任一序列有非空的交。
\end{theorem}