% 狭义相对论与洛伦兹对称性
% keys 洛伦兹变换|参考系变换|四矢量
% license Xiao
% type Tutor

\pentry{洛伦兹变换\upref{SRLrtz},世界线和固有时\upref{wdline}}

约定使用东海岸度规 $\eta_{\mu\nu}=\rm{diag}(-1,1,1,1)$ 和自然单位制 $c=1$。
\subsection{不变距离与光速不变原理}
\subsubsection{不变距离与度规}
在狭义相对论中,两个事件之间的 $\dd s^2=\dd t^2-\dd x^2-\dd y^2-\dd z^2$ 为不变距离,即在任何惯性参考系下 $\dd s=\sqrt{-\eta_{\mu\nu}\dd x^\mu \dd x^\nu}$ 都是不变的量, $\eta_{\mu\nu}$ 为闵氏时空的度规。 在欧式时空中,度规为对角矩阵 $\mathrm{diag}(1,\cdots,1)$ 代表任意两点之间的欧式距离可以用 $\sqrt{\sum_i x_i^2}$ 来计算。闵氏时空的度规为 $\mathrm{diag}(-1,1,1,1)$,闵氏时空的时空维数为 $d=4$,其中时间维度为 $s=1$ 为度规场正则形式中 $-1$ 的数量,空间维数 $d-s=3$ 为 $+1$ 的数量,这种形式的度规场被称为洛伦兹度规。定义
\begin{equation}
A_\mu = \eta_{\mu\nu}A^\nu~.
\end{equation}
为四矢量 $A^\mu$ 的指标下降后得到的矢量,它被叫做是\textbf{逆变矢量}。那么不变距离可以写为 $\dd s=\sqrt{-\dd x_\mu \dd x^\mu}$。洛伦兹对称性要求任意两个四矢量 $A^\mu, B^\mu$ 的“内积”
\begin{equation}
\eta_{\mu\nu} A^\mu B^\nu~.
\end{equation}
是洛伦兹不变的。

有的时候也会采用另一种号差规定,被称为西海岸度规: $g_{\mu\nu}=\mathrm{diag}(1,-1,-1,-1)$,这时不变距离就可以写作 $\dd s = \sqrt{g_{\mu\nu}\dd x^\mu \dd x^\nu}$。在西海岸度规的号差约定下,四矢量的上升下降指标的定义也要相应地发生变化。
\subsubsection{光速不变原理与因果律}

考虑在惯性系 $S_1$ 中,从 $(t_1,\bvec x_1)$ 发射一束光,运动到 $(t_2,\bvec x_2)$。由于是光速,$\Delta t=\Delta \bvec x$,所以两点之间的不变距离为 $\dd s=\sqrt{(t_2-t_1)^2-|\bvec x_2-\bvec x_1|^2)}=0$。那么如果参考系变换到 $S_2$,两个事件的时空坐标变为 $x'_1,x'_2$,$\dd s$ 仍然要保持不变,即 $\dd s=\sqrt{(t'_2-t'_1)^2-|\bvec x'_2-\bvec x'_1|^2)}=0$,所以在新的惯性系中,光从 $(t'_1,\bvec x'_1)$ 传播到 $(t'_2,\bvec x'_2)$ 的速度仍然是光速 $c=1$(自然单位制),这就是\textbf{光速不变原理}。

在世界线和固有时\upref{wdline} 中提到了,如果曲线上的任意一点都有 $\eta_{\mu\nu}V^\mu V^\nu<0$,那么曲线是\textbf{类时}的。
如果曲线上的任意一点都有 $\eta_{\mu\nu}V^\mu V^\nu>0$,那么曲线是\textbf{类空}的。如果曲线上的任意一点都有 $\eta_{\mu\nu}V^\mu V^\nu=0$,即粒子总是以光速运动,这样的世界线被称为\textbf{类光}的。由于在洛伦兹变换下两个四矢量的内积是洛伦兹不变的量,所以曲线的类空、类光性质在任意惯性参考系中都是不变的,这使得任意两个事件之间的因果关系不会因为参考系变换而发生变化。