% 进动:旋转的陀螺为什么不会倒(科普)
% keys 陀螺|章动|进动|旋转|经典力学
% license Usr
% type Art


不旋转的陀螺,尖端朝下放在桌上无法立住,因为几乎不可能让陀螺完美对称且完美竖直,于是在重力矩作用下就会倾倒。一旦陀螺旋转起来,尽管重力矩依然要拉着陀螺倾倒,但陀螺似乎只是在不停改变旋转轴的方向,一直不倾倒。

考虑一个空塑料瓶,称从它的瓶盖到瓶底的轴为其“长轴”。将塑料瓶扔到空中,使得其长轴有翻转的角速度,你会发现塑料瓶的长轴大致在一个平面内不停翻转,瓶盖的指向变化范围也差不多有$360^\circ$。但如果扔的同时让塑料瓶本身还绕着长轴旋转,那么塑料瓶的长轴就会在一个双圆锥面内反转,瓶盖变化范围显著小于$360^\circ$。

上述两个现象有一个共同的名字:进动。简单来说,进动就是指,旋转的刚体在旋转轴方向变化的时候会受到一个回转力矩,从而改变其旋转轴变化的方向。我们通过受力分析就能理解进动的成因。


\subsection{术语准备}


为了方便接下来的讨论,我需要定义一些概念,这样能大大简化描述和理解的难度。


首先确定研究对象:一个固定在轻质杆上的均质圆环\footnote{这只是简化模型,请不要纠结杆为什么没有质量以及圆环怎么固定在杆上的。把这个简化模型研究清楚以后,结论可以套在真实的复杂模型上。},如\autoref{fig_PrecPS_1} 所示。圆环的半径为$r$,总质量为$M$;杆的两个端点分别取名为$a$和$b$,它们到圆环的几何中心的距离相等;称圆环的几何中心为$O$点。设$A$、$B$间的距离为$2d$。

\begin{figure}[ht]
\centering
\includegraphics[width=6cm]{./figures/649e4f73fbad7bc3.pdf}
\caption{陀螺模型示意图。} \label{fig_PrecPS_1}
\end{figure}


要描述$ab$轴的方向变化,可以用$a$和$b$的相对速度。比如说,如果陀螺在平移,那么两个点的相对速度为$0$,$ab$轴的方向不会改变;但两个点只要有非零的相对速度,那么$ab$轴的方向就一定会改变。为方便计,我们称$ab$轴的方向改变为陀螺的偏转。


以点$O$为参考点,定义陀螺受到的力矩。具体来说,如果陀螺在某点$P$处受到力$\bvec{F}$作用,那么陀螺受到的力矩就是$\bvec{\tau}=\bvec{r}\times \bvec{F}$,其中$\bvec{r}$是从$O$到$P$的位移向量。$\bvec{\tau}$的方向可以用右手定则确定:伸出手掌,四指指向$\bvec{r}$的方向,调整手掌的方向使得$\bvec{F}$的方向垂直从手背穿入、手心传出,此时竖起大拇指,拇指所指方向即为$\bvec{\tau}$的方向。


当陀螺受到沿着$ab$轴的力矩的时候,并不会因此偏转。如果受到的力矩垂直于$ab$轴,则陀螺会偏转,但$a$相对$b$的\textbf{加速度}却并不是沿着力矩的方向,而是垂直于力矩的方向。举个例子,从你的视角看,如果对点$a$和$b$分别施加一个水平向左和水平向右的等大的力,那么陀螺受到的力矩是水平指向你的,但陀螺因此得到的偏转方向、或者说$a$相对$b$的加速度,是水平向左的。



但无论如何,力矩的效果是给$a$提供一个相对$b$的加速度,而我们描述的正是$a$相对$b$的运动状态,因此接下来讨论力矩时只讨论它提供的相对加速度。称$a$相对$b$的速度为陀螺的偏转速度,加速度为偏转加速度。

另外请注意,力矩和力一样都是向量,因此可以分解为垂直$ab$轴和沿着$ab$轴方向的分量,其中只有第一个分量影响陀螺的偏转,第二个分量影响的是圆环自转的加速度。




\subsection{受力分析}



当圆环本身没有自转的时候,如果有偏转速度,则$ab$轴会一直在一个平面内以恒定的角速度转动,此时偏转速度虽然一直在改变,但没有脱离以$d$为半径的圆周。

但如果圆环绕着$ab$轴转起来,




























