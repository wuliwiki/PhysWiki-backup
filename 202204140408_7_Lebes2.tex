% Lebesgue积分的一些补充性质
% keys Levi定理|Lebesgue基本定理|Fatou引理

\pentry{Lebesgue 积分\upref{Lebes1},子列极限、上极限与下极限\upref{SubLim}}

由描述Lebesgue积分几何意义的\autoref{Lebes1_the3}~\upref{Lebes1},容易推得以下定理:

\begin{theorem}{Levi定理}

设$\{f_k\}$是可测集$E$上的一列\textbf{单调不减}的\textbf{非负可测}函数,则有
\begin{equation}
\int_E  \qty[\lim\limits_{k\to \infty} f_k(x)] \dd x = \lim\limits_{k\to \infty}\int_E f_k(x) \dd x
\end{equation}

\end{theorem}

简单来说,Levi定理就是指单调不减的非负可测函数列具有“可极限换序”的性质.

由Levi定理可以直接得到以下推论:

\begin{corollary}{Lebesgue基本定理}

设$\{f_k\}^\infty_{k=1}$是可测集$E$上的一列\textbf{非负可测}函数,则有

\begin{equation}
\int_E  \qty[\sum\limits_{k=1}f_k(x)] \dd x = \sum\limits_{k=1} \int_E f_k(x) \dd x
\end{equation}

\end{corollary}

回忆\textbf{上极限}与\textbf{下极限}的\autoref{SubLim_def1}~\upref{SubLim},我们可以把这个定义推广到函数列上去.为了方便阅读,我们在这里简单解释一下上下极限的含义,已熟悉的读者可直接跳过,需要更详细解释的读者请跳转到\textbf{子列极限、上极限与下极限}\upref{SubLim}.

\begin{definition}{数列的上、下极限}
给定实数列$\{a_n\}$.

定义
\begin{equation}
\lim\limits_{n\to\infty} \sup\{a_i\}_{i=n}^\infty = \overline{\lim\limits_{n\to\infty}} a_n
\end{equation}
为$\{a_n\}$的\textbf{上极限(upper limit)}.

类似地,定义
\begin{equation}
\lim\limits_{n\to\infty} \inf\{a_i\}_{i=n}^\infty = \underline{\lim\limits_{n\to\infty}} a_n
\end{equation}
为$\{a_n\}$的\textbf{下极限(lower limit)}.
\end{definition}














