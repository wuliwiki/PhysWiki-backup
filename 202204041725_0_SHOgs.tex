% 简谐振子中的高斯波包(Matlab)



\begin{lstlisting}[language=matlab]
% 高斯波包在简谐振子势阱中运动
function SHOqm
% === 参数 ===
m = 1;
w = 1;
xmin = -8; xmax = 8; Nx = 500;
tmin = 0; tmax = 2*pi/w; Nt = 150;
Nn = 70;
x0 = 0; sig_x = 1;
p0 = 5;
t0 = 0;
ax = [xmin, xmax, -1, 1];
% ===========
x = linspace(xmin, xmax, Nx);
psi_bound = zeros(Nn, Nx);

%% 计算并画前几个束缚态
figure;
for n = 0:Nn-1
    psi_bound(n+1,:) = psi_SHO(m, w, n, x);
    if n <= 4
        plot(x, psi_bound(n+1,:)); hold on; grid on;
    end
end
axis([-5, 5, -0.7, 0.8]);
xlabel x; title '\psi_n';
legend({'n=0','n=1','n=2','n=3','n=4'});

%% 计算高斯波包的初始展开系数
C = zeros(1, Nn);
psi_gs = @(x) 1/(2*pi*sig_x^2)^0.25/sqrt(1 + 1i*t0/(2*m*sig_x^2))...
      *exp(-(x-x0-p0*t0/m).^2/(2*sig_x)^2/(1 + 1i*t0/(2*m*sig_x^2)))...
      .*exp(1i*p0*(x-p0*t0/(2*m)));
plot(x, real(psi_gs(x)));

for n = 0:Nn-1
    psi_sho = @(x) psi_SHO(m, w, n, x);
    C(n+1) = integral(@(x) psi_sho(x).*psi_gs(x), xmin, xmax);
end
figure; plot(abs(C).^2);

%% 含时动画
figure; set(gcf, 'Unit', 'Normalized', 'Position', [0.1, 0.1, 0.4, 0.3]);
t = linspace(tmin, tmax, Nt);
for it = 1:Nt
    psi = zeros(1, Nx);
    for n = 0:Nn-1
        En = (0.5 + n)*w;
        psi = psi + C(n+1)*exp(-1i*En*t(it))*psi_bound(n+1,:);
    end
    clf;
    plot(x, real(psi)); hold on;
    plot(x, imag(psi)); axis(ax);
    xlabel x; title(['t = ' num2str(t(it), '%.2f')]);
    set(gca, 'FontSize', 12);
    saveas(gcf, [num2str(it), '.png']);
end
end

function psi = psi_SHO(m, w, n, x)
alpha = sqrt(m*w); u = alpha * x;
psi = 1/sqrt(2^n*factorial(n)) *...
    (alpha^2/pi)^0.25 * hermiteH(n, u) .* exp(-u.^2/2);
end
\end{lstlisting}
