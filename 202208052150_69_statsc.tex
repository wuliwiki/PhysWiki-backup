% 统计物理·微观与宏观之间的桥梁
% keys 统计物理|科普|微观
\subsection{微观世界与宏观世界}
物理学史上,人们对大自然的探索主要有两条路线.其中之一,是对物质的结构不断“细分”,通过对\textbf{微观世界}的研究来探索最根本的自然法则.这个思想源自于古希腊时期德谟克里特的“原子论”,随后被发扬光大,对物理学、化学等学科都产生了深远的影响.而在二十世纪\textbf{相对论}与\textbf{量子力学}\footnote{这一时期诞生了狭义相对论\upref{SpeRel}、广义相对论,以及量子力学.在这之后量子场论的蓬勃发展极大地推动了人们对粒子物理的认识.}的革命之后,伴随着加速器技术的发展,人们发现了重子、介子、轻子等形形色色的粒子,在许许多多物理学家的共同努力下,“粒子物理标准模型”诞生,人们对微观世界的认识达到了一个崭新的高度.这一条探索路线上发生过无数次物理学的“革命”,带来了各种各样的惊喜.

另一条路线则与之不同,人们不是仅仅着眼于那些自然法则,而是去挑战\textbf{复杂系统}所特有的物理学性质,从我们所处的宇宙,到我们身边的空气、河流……这纷繁的世界之间似乎有一种遥远的相似性,似乎存在着许许多多重要的关于复杂系统的规律,独立于那些最根本的自然法则.具有代表性的成果是\textbf{热力学和统计物理}\footnote{可以参考热力学与统计力学导航\upref{StatMe}.}.在卡诺、焦耳、克劳修斯等科学家的努力下,人们建立起一套完备的自洽的热力学体系,这是以热力学第零定律到热力学第三定律\footnote{事实上还包括其他的一些假设,例如广延量和强度量的约定等等}为基本假设建立的一套关于多粒子系统\footnote{或者说,是“大量微观粒子组成的宏观客体”.这其实暗示了,在微观物理定律与宏观物质性质之间有大量的物理可以挖掘.}的物理理论.而基于这个物理理论,人们得到了许许多多美妙的结果.我们能够运用流体力学分析大气、海洋,能够运用连续介质力学研究地壳的运动,运用等离子体物理研究太阳风、地球磁层,运用固体物理研究金属、半导体等材料的性质…… 对宏观世界和复杂系统的研究与科学技术的发展、与人类的生产生活方式息息相关.而正因为 "\textbf{More is different}"\footnote{出自 Anderson 的著名论文 More is different,可以看作是凝聚态物理的《独立宣言》.},关于复杂系统有数不清的无限多的难题等着科学家们去解决.

微观与宏观、基本定律与复杂系统之间的研究是相辅相成的.宇宙学标准模型的建立离不开广义相对论和热力学的发展,凝聚态系统中的准粒子(其实质是场的激发)也启发着人们对量子场论的理解.微观物理和宏观物理的联系是如此微妙又神奇.当玻尔兹曼\footnote{玻尔兹曼是现代物理学奠基人,对统计物理的发展有巨大的贡献.}写下 $S=k\ln \Omega$\footnote{玻尔兹曼墓碑上的公式,阐释了热力学熵的统计涵义.}时,他一定不会想到统计力学在物理学史上具有如此重要的地位——它掀起了近代物理的革命,使人们对事物的认识达到了一个崭新的层次.
\subsection{统计物理的思想}
19世纪末,以麦克斯韦、玻尔兹曼为代表的物理学家们开始对大量粒子系统的微观与宏观之间的联系产生兴趣.人们开始思考并尝试解构一些经典的热力学概念,例如温度、内能、热传导…… 得益于对原子、分子的认识,我们可以将宏观物质放大再放大,并尝试通过微观的粒子的运动、它们之间的相互作用来“还原”出宏观的一些物理量.

想象你有一瓶空气,这瓶空气中有许许多多的空气分子,可以是氮气分子、水分子、氧气分子等等.这里为了简单起见,我们考虑一瓶水蒸气,即只有水分子.一小瓶水蒸气所包含的分子的数量是巨大的,人们常用阿伏伽德罗常数($1 {\rm N_A}\approx 6.02\times 10^{23}$)为单位来刻画粒子的数量,$1 {\rm N_A}$ 的粒子记为 $1 {\rm mol}$.分子的大小非常小,以至于相邻气体分子之间的距离达到分子半径的数十倍.在这个距离上,气体分子之间的相互作用如此之弱,以至于我们暂时可以忽略它\footnote{如果考虑范德瓦尔斯力,则得到的热力学性质将与理想气体不同,参考范德瓦尔斯气体\upref{Vand}词条.}.

\begin{figure}[ht]
\centering
\includegraphics[width=5cm]{./figures/statsc_1.png}
\caption{一小瓶气体分子示意图,图中的小球是气体分子,可以有更复杂的内部结构.} \label{statsc_fig1}
\end{figure}
当然,气体分子无时无刻都在运动.英国植物学家布朗发现悬浮在水中的微粒在做不规则运动,这被称为“布朗运动”.同样地,如果我们标记瓶子中的某一个气体分子,称它为“小明”,那么他的运动轨迹也将是无规则的,这是由于小明常常与其他的气体分子发生碰撞.人们用 $\lambda$ 来衡量小明在相邻两次碰撞间走过的平均的距离,这被称为分子的平均自由程.为了方便讨论,我们先假定\footnote{事实上物理学家们常常通过约定一些假设来简化问题;而最后当物理学家们对该理论有更深刻了解时,再回过头来看这些假设,对它们进行修改、简化,或者揭示这些假设的本质.所谓的非弹性碰撞实际上意味着气体分子的内部自由度:自旋角动量、轨道角动量、振动等因素也被考虑到碰撞过程中.而在系综理论中,我们能够充分地考虑这些自由度,从而更细致地描写系统.}分子间的碰撞全是弹性碰撞,并忽视气体分子的内部的复杂结构,将它们看作是刚性的小球——这被称为理想气体\footnote{对理想气体的更细致的定义参考理想气体\upref{Igas}词条.}.模型的简化给我们带来了很多的方便,例如我们暂时地撇去了分子内部结构因素,从而可以将该模型运用到对各种各样气体的讨论.我们还得到了重要的守恒定律:能量守恒.对于理想气体而言,这个能量是指微观系统的一切分子的动能与在外场\footnote{可以是重力场、电场或是其他形式的场.}下势能的和\footnote{由于我们忽略了分子内部自由度,所以分子的振动、转动可以不被算在这个能量中}.

热力学对宏观物理量的研究告诉我们,能量是守恒的,即热力学第一定律:$\dd E=\delta Q+\delta W$.为了