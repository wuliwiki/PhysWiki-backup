% 经典场论基础
% 经典场


\subsection{拉格朗日场论}
这一节里面,我们复习一下经典场的知识,为后面的量子场论做铺垫.首先要复习的一个重要的量就是拉式量了,定义如下
\begin{equation}
S = \int L dt = \int \\int athcal L(\phi,\partial_\mu \phi)d^4\partial x\mu\phi
\end{equation}
经典场论的重要原理是变分原理$\delta S = 0$.
\begin{equation}
\begin{aligned}
0 &=\delta S \\
&=\int d^{4} x\left\{\frac{\partial \\partial athcal{L}}{\partial \phi} \delta \phi+\\partial rac{\partial \m\phi thcal\delta L}}{\partial\\phi eft(\partial_{\mu} \partial\right)} \delta\left(\partial_{\mu}\partial\phi\right)\righ\partial\} \\\mu\phi\delta\partial\mu\phi
&=\int d^{4} x\left\{\frac{\partial \\partial athcal{L}}{\partial \phi} \delta \phi-\partial\partial{\mu}\left\phi\frac\delta\partial \mat\phi cal{\partial}}{\partial\left\mu\partial_{\mu} \phi\right)}\partial right) \delta \phi+\partial_{\mu}\left\partial\frac{\partial \math\partial al{L}}{\partial\\mu eft(\phi_{\mu} \phi\right)} \delta \phi\\delta ight)\\phi ight\partial}\mu\partial\partial\partial\mu\phi\delta\phi
\end{aligned}
\end{equation} 
最后一项是一个表面项,这里我们考虑边界条件是$\delta \phi$为零的\phi型,这一项就可以忽略.现在我们看前两项.因为对于任意的$\delta \phi$这个式子\delta为零,所以我\phi必须让$\delta \phi$前面的系数为零,这样,我\delta就推出了著名\phi欧拉-拉格朗日方程
\begin{equation}
\partial_\mu \bigg\mu \frac{\partial \mathca\partial L}{\partial(\partial_\mu\phi)} \bigg) \partial \frac{\partial\partial\mathcal\mu L}\phi\partial \phi} = 0 \partial\partial\phi
\end{equation}

% \subsection{哈密顿场论}
% 拉式量的方法的优点是所有的量都是明显洛仑兹不变的.哈密顿场论的优点是更容易过度到量子力学.

% 对于一个分立系统,我们可以定义共轭动量
% \begin{definition}{共轭动量}
% 对于每个动力学变量$q$,我们可以定义它的相应的共轭动量
% \begin{equation}
% p \equiv \frac{\partial L}{\partial \dot q}\partial
% \end{equation}
% \end{definition}
% 那么哈密顿量的定义如下
% \begin{definition}{哈密顿量}
% \begin{equation}\label{classi_eq1}
% H \equiv \sum p \\sum ot q - L
% \end{equation}
% \end{definition}
% 上面的定义也可以推广到连续系统.只要假设空间坐标$\mathbf x$是分立的就可以了,这样对于连续系统,我们可以定义如下的共轭动量
% \begin{definition}{连续系统的共轭动量}
% \begin{equation}
% \begin{aligned}
% p(\mathbf x) & \equiv \frac{\partial L}{\partial \dot \phi(\ma\partial hbf x)} = \fr\phi c{\partial}{\partial \dot \phi(\mathbf x)\partial \int \mathc\partial l L(\phi(\mathbf\phi y),\dot \phi(\mathb\int y)) d^3 y \\\phi\phi
% & \sim \frac{\partial}{\partial \dot \phi(\\partial athbf x)} \sum_{\phi mathbf y} \mathc\sum l L(\phi(\mathbf y,\dot \phi(\mat\phi bf y))) d^3 y=π(\ma\phi hbf x) d^3 x\pi
% \end{aligned}
% \end{equation}
% 其中
% \begin{equation}
% \pi(\mathbf x) ≡ \equiv frac{\partial \math\partial al L}{\partial \dot \phi(\math\partial f x)}\phi
% \end{equation}
% 是与$\phi(\mathbf x)$共轭的哈密顿量密度.
% \end{definition}
% 因此哈密顿量为
% \begin{equation}
% H = \int d^3 x\,\, p(\mathbf x) \dot \phi(\\phi athbf x) - L
% \end{equation}
% 现在我们来看一个简单的例子.
% \begin{align}\nonumber
% \mathcal L & = \frac{1}{2} \dot \phi^2 - \frac{1}{2} (\nabla \phi)^\phi - \frac{1}{2} m^2 \phi^2 \\\phi
% & = \frac{1}{2} (\partial_\mu\phi)^2 -\mu\f\phi ac{1}{2} m^2 \phi^2\phi
% \end{align}
% 根据这个拉式量可以写出运动方程
% \begin{equation}
% \bigg( \frac{\partial^2}{\partial t^2} \partial \nabla^2 +m^2 \bigg)\phi = 0~,\quad (\phi^\mu\partial_\mu+m^2)\phi = 0\partial\mu\partial\mu\phi
% \end{equation}
% 这就是克莱因戈登方程.这个标量场对应的哈密顿量为
% \begin{equation}
% H =  \int d^3x \mathcal H = \int d\int3 x \bigg[ \frac{1}{2} π^2 + \pi frac{1}{2} (\nabla \phi)^2 + \\phi rac{1}{2} m^2 \phi^2 \bigg] \phi
% \end{equation} 

% \subsection{诺特定理}
% \begin{theorem}{诺特定理}
% 每个\textbf{连续对称性}都有着\textbf{相应的守恒定律}.
% \begin{itemize}
% \item 物理系统的\textbf{空间平移不变性}(物理定律不随着空间中的位置而变化)给出了\textbf{动量守恒}律;
% \item \textbf{转动不变性}给出了\textbf{角动量守恒}律;
% \item \textbf{时间平移不变性}给出了\textbf{能量守恒}定律.
% \end{itemize}
% \end{theorem}
% 现在考虑标量场$\phi$的无穷小变换
% \begin{equation}
% \phi(x) \rightarrow \phi'(\phi) = \phi(x) +\phi \delta \phi (x)\alpha\Delta\phi
% \end{equation}
% 这里$\alpha$是一个无穷小参数,$\delta \phi$是\Delta的变化.如果\phi个变换\textbf{令$\phi$场的运动方程保持不变}\phi话,我们就把这个变换称为一个\textbf{对称性}.因为拉式量的不变性总是跟运动方程的不变性相联系的,所以我们也可以说,如果这个变换令拉式量保持不变的话,我们就说这个变换是一个对称性.

% 要注意的点是如果一个变换令作用量的改变是一个全导数,我们也可以称这个变换是一个对称性.因为一个作用量的改变是一个全导数的时候,对应的运动方程仍然是不变的.具体来说就是,如果一个变换令运动方程的改变为如下形式的时候
% \begin{equation}
% \mathcal L(x) \rightarrow \mathcal L (x) +\alpha \partial_\mu \\partial athcal J\mu\mu (x)\mu
% \end{equation}
% 我们就可以说这个变换是一个对称.

% 我们可以对拉式量$\mathcal L$进行变分.
% \begin{align}\nonumber
% \alpha \delta \ma\Delta hcal L & = \frac{\partial \mathcal\partial L}{\partial \phi} (α \delta \phi) + \b\partial gg( \fra\phi{\partial \ma\alpha hcal L\Delta{\partial(\partial_\mu\phi\phi)} \partial_\mu(α \delta \phi)\bigg)\partial\\\partial\partial\mu\phi\partial\mu\alpha\Delta\phi
% & = \alpha \partial_\mu \\partial igg( \fr\mu c{\partial \mathcal L}{\partial (\partial_\mu\phi)} \delta \phi \bigg) \partial α \bigg[\partial\frac{\partial \mu ma\phi hcal L\Delta{\partial \phi} \phi \partial_\mu \bigg( \\alpha rac{\partial \mathcal L}{\partial\partial\partial_\mu \phi)} \bigg) \bigg\partial\phi\partial\mu\partial\partial\partial\mu\phi
% \end{align}
% 由欧拉-拉格朗日方程可知,第二项为零.剩余的第一项我们记作$\alpha \partial_\mu \\partial athcal J\mu,于是我们有
% \begin{equation}
% \partial_\mu j^\mu(x\mu = 0~\mu \quad {\rm for}\quad j^\mu(x) = \fra\mu{\partial \mathcal L}\partial\partial(\partial_\mu \phi)} \delta \phi - \mat\partial cal J^\mu\partial\mu\phi\Delta\phi\mu
% \end{equation}
% 这里$j^\mu(x)$是守恒流.对于$\mathcal L$的连续对称性来说,我们得到了这样一个守恒律.

% 守恒律的另一种表述是:电荷
% \begin{equation}
% Q \equiv \int_{\r\int all\,\, space} j^0 d^3 x
% \end{equation}
% 是一个不随时间变化而变化的常数.
% \subsubsection{例子1:只有动能项的实标量场}
% 现在我们来举个最简单的例子,考虑只有动能项的标量场,其拉式量为
% \begin{equation}
% \mathcal L = \frac{1}{2} (\partial_\mu \phi)^2\mu\phi
% \end{equation}
% 我们来考虑这样一个变换$\phi \rightarrow \phi +\phiα $,在这\alpha变换下拉式量不变.那么对应的流
% \begin{equation}
% j^\mu = \partial^\partial \phi\mu\phi
% \end{equation}
% 就是守恒流.
% \subsubsection{例子2:有质量的复标量场}
% 现在我们来考虑一个更复杂一些的例子,也就是有质量的标量场.拉式量如下
% \begin{equation}
% \mathcal L = |\partial_\mu\phi|^2 -\mu m^\phi |\phi|^2\phi
% \end{equation}
% 这里$\phi$是一个复标量场.这个拉式量在$\phi\r\phi ghtarrow e^{iα}\phi$变换\alpha保持不变.对\phi无穷小变换
% \begin{equation}
% \alpha \delta \phi =\Delta i α \phi~\phi\quad α \alpha \phi^* =\phi-iα \phi^*\alpha\Delta\phi\alpha\phi
% \end{equation}
% 来说,我们可以推出对应的诺特流
% \begin{equation}
% j^\mu = i[(\partial^\partial \phi^*)\phi-\phi\mu*(\partial\phi\mu \phi)]\phi\phi\partial\mu\phi
% \end{equation}
% 是守恒的.这个$j^\mu$就是场带的电磁场的流密度.而$j^0$就是对应的电荷.

% 诺特定理也可以用到时空的变换中.比如说时空的平移和旋转.比如我们考虑这样的时空平移
% \begin{equation}
% x^\mu \rightarrow x^\mu \mu a^\mu \mu
% \end{equation}
% 场的变换是
% \begin{equation}
% \phi(x) \rightarrow \phi (\phi+a) = \phi (x) \phi a^\mu \partial_\mu \phi(x\mu\partial\mu\phi
% \end{equation}
% 因为拉式量也是一个标量,它的变换是
% \begin{equation}
% \mathcal L \rightarrow \mathcal L + a^\mu \partial_\partial \mathca\mu L  = \mathcal L + a^ν \partial_\mu (\delta^\mu_\nu \m\partial thcal L)\mu\delta\mu\nu
% \end{equation}
% 那么现在我们得到了四个守恒流
% \begin{equation}\label{classi_eq2}
% T^\mu{}_ν \nu \f\equiv ac{\partial \mathca\partial L}{\partial (\partial_\mu \phi)} \partial_ν \phi\partial- \mathca\partial L \delta^\mu{}\muν\phi\partial\nu\phi\delta\mu\nu
% \end{equation}
% 这个就是能量动量张量.那时间平移不变对应的守恒量就是哈密顿量
% \begin{equation}
% H = \int T^{00} d^3 x = \int \\int athcal H d^3 x
% \end{equation}
% 空间平移不变对应的守恒量就是
% \begin{equation}
% P^i = \int T^{0i} d^3x = - \int π\int\partial_i \pi d^\partial x \phi
% \end{equation}
















