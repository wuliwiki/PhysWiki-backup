% 恩里科·费米(综述)
% license CCBYSA3
% type Wiki

本文根据 CC-BY-SA 协议转载翻译自维基百科 \href{https://en.wikipedia.org/wiki/Enrico_Fermi}{相关文章}。

恩里科·费米(意大利语:[enˈriːko ˈfermi],1901年9月29日-1954年11月28日)是一位意大利裔、后归化为美国公民的物理学家,以建造世界上第一座人工核反应堆——芝加哥一号堆而闻名,并曾是曼哈顿计划的重要成员。他被誉为“核时代的建筑师”以及“原子弹之父”。\(^\text{[1]}\)他是极少数在理论物理和实验物理两个领域都卓有成就的物理学家之一。费米因其在中子轰击引发放射性方面的研究以及对超铀元素的发现而获得1938年诺贝尔物理学奖。他与同事们共同申请了多项与核能应用相关的专利,所有这些专利最终都被美国政府接管。他在统计力学、量子理论、核物理和粒子物理的发展中都作出了重要贡献。

费米的第一个重大贡献是在统计力学领域。1925年,沃尔夫冈·泡利提出了著名的泡利不相容原理,随后费米发表了一篇论文,将该原理应用于理想气体,发展出一种统计方法,如今被称为费米–狄拉克统计。今天,那些遵守不相容原理的粒子被称为“费米子”。后来,泡利为了解释β衰变中能量守恒的问题,提出了在电子发射的同时还会发射一种不带电的不可见粒子这一假设。费米接纳了这个想法,并构建了一个理论模型,纳入了这一假想粒子,并将其命名为“中微子”。他的这一理论后来被称为“费米相互作用”,现今称为“弱相互作用”,是自然界四种基本相互作用之一。在用新发现的中子进行诱导放射性实验时,费米发现慢中子比快中子更容易被原子核俘获,并据此发展出描述该过程的“费米年龄方程”。在用慢中子轰击钍和铀的实验中,费米认为自己合成了新的元素。尽管他因这一发现获得了诺贝尔奖,但后来证实这些“新元素”其实是核裂变的产物。1938年,为了躲避影响其犹太妻子劳拉·卡蓬的意大利新种族法,费米离开意大利,移民美国。在第二次世界大战期间,他参与了“曼哈顿计划”。在芝加哥大学,费米领导的团队设计并建造了“芝加哥堆-1”,该堆于1942年12月2日首次实现了人类制造的、自持的核链式反应。他还在田纳西州橡树岭的X-10石墨反应堆于1943年达到临界状态时在场,次年又见证了华盛顿州汉福德基地的B反应堆启动。在洛斯阿拉莫斯国家实验室,他领导F部门,其中一部分致力于爱德华·泰勒的热核“超级炸弹”项目。他还亲历了1945年7月16日的“特立尼蒂试验”,即首次核弹爆炸测试,并使用著名的“费米估算法”评估了炸弹的当量。

战后,费米协助创建了芝加哥的核研究所,并在J·罗伯特·奥本海默担任主席的总顾问委员会中任职,为美国原子能委员会提供核事务建议。1949年8月苏联成功引爆第一颗裂变原子弹后,费米从道德和技术两方面都强烈反对研制氢弹。1954年,在导致奥本海默失去安全许可的听证会上,费米也是为奥本海默作证的科学家之一。

费米在粒子物理领域也做出了重要贡献,尤其是在与介子(如π介子和μ子)相关的研究方面。他还推测宇宙射线的产生是由于星际空间中的磁场加速物质所致。许多奖项、概念和机构都以费米的名字命名,包括费米一号(快中子增殖反应堆)、恩里科·费米核发电站、恩里科·费米奖、恩里科·费米研究所、费米国家加速器实验室、费米伽马射线太空望远镜、“费米悖论”,以及人造元素“镄”,这使他成为仅有的十六位拥有化学元素以自己命名的科学家之一。
\subsection{早年生活}
恩里科·费米于1901年9月29日出生在意大利罗马。\(^\text{[3]}\)他是铁路部司局长阿尔贝托·费米与小学教师伊达·德·加蒂斯的第三个孩子。\(^\text{[3][4][5]}\)他的姐姐玛丽亚比他大两岁,哥哥朱利奥大他一岁。两位男孩幼年时被送到乡下由乳母抚养,直到恩里科两岁半时才返回罗马与家人团聚。\(^\text{[6]}\)尽管按照祖父母的意愿他接受了天主教洗礼,但他的家庭并不虔诚;恩里科成年后一直是无神论者。\(^\text{[7]}\)童年时期,他和哥哥朱利奥有着相同的兴趣爱好,比如制作电动机、玩电动和机械玩具。\(^\text{[8]}\)1915年,朱利奥因喉部脓肿手术不幸去世;玛丽亚则于1959年在米兰附近的一场空难中遇难。\(^\text{[9][10]}\)

在罗马的鲜花广场集市上,费米发现了一本物理书,名为《Elementorum physicae mathematicae》,全书900页,由耶稣会士、罗马学院教授安德烈亚·卡拉法神父用拉丁文编写。书中介绍了当时(1840年出版)对数学、经典力学、天文学、光学和声学的理解。\(^\text{[11][12]}\)在一位同样对科学感兴趣的朋友恩里科·佩尔西科的陪伴下,\(^\text{[13]}\)费米开始了诸如制作陀螺仪、测量地球重力加速度等实验项目。\(^\text{[14]}\)

1914年,费米常常在父亲下班后到办公室门口与其会合。有一次,他遇见了父亲的一位同事阿道夫·阿米代伊。恩里科得知阿道夫对数学和物理感兴趣,便趁机向他请教几何问题。阿道夫意识到小费米问的是射影几何,随后送给他一本由特奥多尔·雷耶所著的相关书籍。两个月后,费米将书还给了阿道夫,声称自己已完成了书后所有的习题,其中一些题目连阿道夫都认为很难。阿道夫核实后惊叹道费米“至少在几何方面是个天才”,并开始更加系统地指导他,给他提供更多关于物理与数学的书籍。阿道夫还指出,费米的记忆力极好,读完书后就能记住全部内容,因此通常读完便归还书籍。\(^\text{[15]}\)
