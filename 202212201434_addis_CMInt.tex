% 经典力学笔记(科普)

\begin{issues}
\issueDraft
\end{issues}

\subsection{基础}
\begin{itemize}
\item \textbf{经典力学}一般是指\textbf{牛顿力学}, 但在一些较现代的文献中也包含相对论(我们不采用)。
\item 经典力学的适用范围是宏观低速弱引力场。
\item 经典力学中物体受力后的运动规律由牛顿三定律描述。
\item 牛顿定律不讨论物体受到的力是如何产生的。
\item 物理中, \textbf{粒子}也叫\textbf{质点}, 是指在当前问题的一般尺度下大小可以忽略不记的物体。 我们假设它们不存在转动, 只存在平移。
\item 物理中, 在研究物体的运动规律时往往先研究粒子的运动规律, 再把所有物体都看作由许多粒子组成的系统。
\item 牛顿三定律本身描述的对象只是质点。 经过一些推导才可以把它应用到一般的物体的运动上。
\item \textbf{几何矢量}的两个要素: 长度和方向(不包含起始位置)。
\item 相对于矢量, 一个数字\footnote{准确来说是实数或复数。}叫做\textbf{标量}。
\item 质点的\textbf{位置矢量}、 \textbf{位移}、 \textbf{速度}、 \textbf{加速度}、 以及受到的每个\textbf{力}都是矢量。
\item 粗略而言, 匀速直线运动中, 若一秒内速度增加了 $1\Si{m/s}$, 它的加速度的数值就是 $1$。
\item 加速度的国际单位是 $1\Si{(m/s)/s}$, 或者说 $1\Si{m/s^2}$(米每秒的平方)。
\item 地球表面的重力加速度约为 $9.8\Si{m/s^2}$, 会随地理位置和高度发生稍微变化。
\item 真空中(地球表面的)自由落体的加速度就是(当地的)\textbf{重力加速度}。 与物体的大小、材料、质量等都无关。
\item \textbf{路程}(标量)和\textbf{位移}的区别。
\item 几何矢量的加法: 位移、速度的叠加。
\item \textbf{力的叠加}: 一个质点可能同时受若干个力, 力会改变物体的运动。 把这些力(矢量)相加得到该质点的\textbf{合力}。 若只有该合力作用在质点上, 作用效果与合成前的若干个力是一样的。
\item \textbf{牛顿第一定律}:质点合力为零时, 静止或者匀速直线运动。
\item \textbf{牛顿第二定律}:质点的加速度与合力成正比, 和它的质量成反比。
\item \textbf{牛顿第三定律}:两个质点间的相互作用力大小相等, 方向相反。
\item 注意牛顿第二个第三定律都是矢量之间的关系, 因为力和加速度都是矢量。 两个矢量相等, 要求大小和方向都相等。
\item \textbf{惯性系}: 牛顿第一定律成立的参考系。
\item 任何两个惯性系之间匀速直线运动。
\end{itemize}

\subsection{万有引力}
\begin{itemize}
\item \textbf{万有引力定律}: 任何两个质点之间都存在万有引力, 即两质点相互吸引, 相互作用力等大反向。 万有引力的大小和两物体的质量乘积成正比, 和距离的平方成反比。
\end{itemize}
\addTODO{补充}
