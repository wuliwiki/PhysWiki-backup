% 李群的李代数
% 李群|李代数|Lie group|Lie algebra|切空间|tangent space|向量场|vector field|李括号|Lie bracket

\pentry{李群\upref{LieGrp},李代数\upref{LieAlg}}

\subsection{光滑向量场}

我们先回顾光滑向量场的两条性质.

\begin{lemma}{提升映射与向量场的交换性}\label{LieGA_lem1}
设$F:M\to N$是一个\textbf{微分同胚},那么对于任意光滑函数$f\in C^{\infty}(M)$和光滑向量场$X\in\mathfrak{X}(M)$,都有:
\begin{equation}
(Xf)\circ F^{-1}=F_*(X)(f\circ F^{-1})
\end{equation}
其中$F_*:TM\to TN$是$F$的微分.
\end{lemma}

\autoref{LieGA_lem1} 根据“向量场对光滑函数作用”的定义就可以证出.

\begin{lemma}{微分和李括号的交换性}
设$F:M\to N$是一个\textbf{微分同胚},那么对于任意光滑向量场$X, Y\in\mathfrak{X}(M)$,有$F_*([X, Y])=[F_*(X), F_*([X, Y])]$.
\end{lemma}

\textbf{证明}:

我们只需要证明$F_*(XY)(f\circ F^{-1})=F_*(X)F_*(Y)(f\circ F^{-1})$对于任意$f\in C^{\infty}(M)$成立即可.

由\autoref{LieGA_lem1} ,
\begin{equation}
\begin{aligned}
F_*(XY)(f\circ F^{-1})&=(XYf)\circ F^{-1}\\
&=F_*(X)(Yf\circ F^{-1})\\
&=F_*(X)(F_*(Y)(f\circ F^{-1}))\\
&=F_*(X)F_*(Y)(f\circ F^{-1})
\end{aligned}
\end{equation}

\textbf{证毕}.

\subsection{李群上的左不变向量场}

\begin{definition}{左不变向量场}
给定李群$G$,对于任意$g\in G$,定义映射$l_g:G\to G$为\textbf{左平移映射},即对于任意$x\in G$,都有$$
\end{definition}



