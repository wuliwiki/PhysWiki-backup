题目:$\mathrm{若五阶行列式的展开式中项}a_{13}a_{2k}a_{34}a_{42}a_{5l}\mathrm{带负号},\mathrm 则k,l\mathrm{的值分别为}(\;).$

A.$k=1,l=5$ $\quad$ B.$k=5,l=1$ $\quad$ C.$k=2,l=5$ $\quad$ D.$k=5,l=2$

答案解析:

$\begin{array}{l}\begin{array}{l}\mathrm{根据行列式的定义},k,l\mathrm{只能取}1或5,\\若k=5,l=1,则\tau(35421)=8,\end{array}\\\begin{array}{l}若k=1,l=5,则\tau(31425)=3,\\\mathrm{所以}k=1,l=5.\\\end{array}\end{array}$



题目:$\mathrm{在四阶行列式的展开式中},\mathrm{下列各项中带正号的是}(\;).$

A.$a_{13}a_{24}a_{32}a_{41}$ $\quad$ B.$a_{14}a_{23}a_{32}a_{41}$ $\quad$ C.$a_{14}a_{22}a_{33}a_{41}$ $\quad$ D.$a_{11}a_{23}a_{32}a_{44}$

答案解析:

$a_{14}a_{23}a_{32}a_{41}\mathrm{的列标的逆序数为}1+2+3=6,\mathrm{故此项前边带有正号}.$



题目:$\mathrm{下列哪一个不是五阶行列式的展开式中的项}(\;).$

A.$a_{21}a_{13}a_{34}a_{55}a_{42}$ $\quad$ B.$a_{21}a_{13}a_{24}a_{55}a_{42}$ $\quad$ C.$a_{11}a_{23}a_{34}a_{55}a_{42}$ $\quad$ D.$a_{13}a_{34}a_{51}a_{42}a_{25}$

答案解析:

$a_{21}a_{13}a_{24}a_{55}a_{42}\mathrm{在第二行取了两个元素},\mathrm{与行列式定义不符}.$



题目:$在n\;\mathrm{阶行列式中},\mathrm{关于主对角线与元素}a_{ij}\mathrm{对称的元素为}(\;).$

A.$a_{ij}$ $\quad$ B.$a_{ji}$ $\quad$ C.$a_{(i-1)(j-1)}$ $\quad$ D.$a_{(j-1)(i-1)}$

答案解析:

$\mathrm{由行列式定义知},\mathrm{与元素}a_{ij}\mathrm{关于主对角线对称的元素行列刚好互换},\mathrm{即为}a_{ji}.$



题目:$若a_{1i}a_{23}a_{35}a_{4j}a_{54}\mathrm{为五阶行列式的展开式中带正号的一项},则\;i,j\;\mathrm{分别为}(\;).$

A.$i=1,j=2$ $\quad$ B.$i=3,j=2$ $\quad$ C.$i=2,j=1$ $\quad$ D.$i=2,j=3$

答案解析:

$\begin{array}{l}\begin{array}{l}\mathrm{由题意可知},i,j\mathrm{只能分别取值}1,2\\\mathrm{不妨设}i=1,j=2,\mathrm{则排列}13524\mathrm{的逆序数为}3,\mathrm{故此排列为奇排列},\mathrm{对应项带负号},\mathrm{不合题意},\end{array}\\\mathrm{因此}i=2\;,j=1.\end{array}$



题目:$若-a_{32}a_{1r}a_{25}a_{4s}a_{53}\mathrm{是五阶行列式展开式中的一项},则\;r,s\;\mathrm{的值分别为}(\;).$

A.$r=1,s=1$ $\quad$ B.$r=1,s=4$ $\quad$ C.$r=4,s=1$ $\quad$ D.$r=4,s=4$

答案解析:

$\begin{array}{l}r和s\mathrm{只能取值为}1,4\\当r=4,s=1时,\mathrm{判断}a_{32}a_{14}a_{25}a_{41}a_{53}即a_{14}a_{25}a_{32}a_{41}a_{53}\mathrm{的列标的逆序数为}7(\mathrm{奇数}),\mathrm{故其项的符号为负}.\end{array}$



题目:$若a_{13}a_{2k}a_{34}a_{42}a_{5l}\mathrm{是五阶行列式的展开式中带有正号的项},则\;k,l\mathrm{分别为}(\;).$

A.$k=1,l=5$ $\quad$ B.$k=5,l=1$ $\quad$ C.$k=1,l=4$ $\quad$ D.$k=4,l=1$

答案解析:

$\mathrm{根据行列式的定义},k\;,l\mathrm{只能取值}1或5,若k=5,l=1,则\tau(\;35421)=8,若k=1,l=5,则\tau(\;31425)=3,\mathrm{所以}k=5,l=1.$



题目:$\mathrm{六阶行列式的展开式中共有}(\;)项.$

A.$36$ $\quad$ B.$720$ $\quad$ C.$240$ $\quad$ D.$160$

答案解析:

$\mathrm{由行列式的定义可知}n\mathrm{阶行列式表示所有取自不同行不同列}n\mathrm{个元素乘积的代数和},即n!\mathrm{项的代数和}.$



题目:$\mathrm{在四阶行列式的展开式中},\mathrm{下列带负号的是}(\;).$

A.$a_{13}a_{24}a_{32}a_{41}$ $\quad$ B.$a_{14}a_{23}a_{32}a_{41}$ $\quad$ C.$a_{12}a_{24}a_{33}a_{41}$ $\quad$ D.$a_{13}a_{21}a_{32}a_{44}$

答案解析:

$a_{13}a_{24}a_{32}a_{41}\mathrm{的列标的逆序数为}2+3=5,\mathrm{所以此项前面带有负号}.$



题目:$\mathrm{四阶行列式展开式中带负号且包含元素}a_{12}和a_{21}\mathrm{的项为}(\;).$

A.$a_{12}a_{21}a_{33}a_{44}$ $\quad$ B.$a_{12}a_{21}a_{34}a_{43}$ $\quad$ C.$a_{12}a_{21}a_{32}a_{44}$ $\quad$ D.$a_{12}a_{21}a_{33}a_{41}$

答案解析:

$\mathrm{根据行列式的定义},\mathrm{四阶行列式中包含元素}a_{12}和a_{21}\mathrm{的项为}a_{12}a_{21}a_{33}a_{44}或a_{12}a_{21}a_{34}a_{43},而a_{12}a_{21}a_{33}a_{44}\mathrm{带有负号}.$



题目:$n\mathrm{阶行列式}\begin{vmatrix}a_1&0&\cdots&0\\0&a_2&\cdots&0\\\cdots&\cdots&\cdots&0\\0&0&\cdots&a_n\end{vmatrix}=(\;).$

A.$0$ $\quad$ B.$1$ $\quad$ C.$a_1+a_2+\cdots+a_n$ $\quad$ D.$a_1a_2\cdots a_n$

答案解析:

$\mathrm{根据行列式的定义},\begin{vmatrix}a_1&0&\cdots&0\\0&a_2&\cdots&0\\\cdots&\cdots&\cdots&0\\0&0&\cdots&a_n\end{vmatrix}=a_1a_2\cdots a_n.$



题目:$\mathrm{四阶行列式}\begin{vmatrix}0&0&2&0\\-1&0&3&0\\6&0&0&-3\\2&2&5&4\end{vmatrix}=().$

A.$0$ $\quad$ B.$4$ $\quad$ C.$12$ $\quad$ D.$-12$

答案解析:

$\mathrm{由行列式的定义可知}\begin{vmatrix}0&0&2&0\\-1&0&3&0\\6&0&0&-3\\2&2&5&4\end{vmatrix}=(-1)^{\tau(3142)}\;\;2\times(-1)\times(-3)\times2=-12.$



题目:$\mathrm{设多项式}f(x)=\begin{vmatrix}x&x&0&2\\3&x&1&1\\2&1&x&0\\5&4&2&x\end{vmatrix},则f(x)中x^4\mathrm{的系数为}(\;).$

A.$1$ $\quad$ B.$3$ $\quad$ C.$2$ $\quad$ D.$4$

答案解析:

$\mathrm{根据行列式定义可以得出},含x^4\mathrm{的项只有主对角线元素的乘积},\mathrm{因此系数为}1.$



题目:$n\mathrm{阶行列式}\begin{vmatrix}1&0&0&...&0\\0&2&0&...&0\\0&0&3&...&0\\...&...&...&...&...\\0&0&0&...&n\end{vmatrix}=(\;).$

A.$n!$ $\quad$ B.$0$ $\quad$ C.$n$ $\quad$ D.$-n!$

答案解析:

$\mathrm{原式}=1\times2\times3....\times n=n!.$



题目:$\mathrm{四阶行列式}D=\begin{vmatrix}a_{11}&a_{22}&a_{32}&a_{13}\\a_{21}&a_{12}&a_{31}&a_{14}\\a_{33}&a_{41}&a_{24}&a_{34}\\a_{44}&a_{43}&a_{42}&a_{23}\end{vmatrix},\mathrm{下列哪一个不是}D\mathrm{的展开式中的项}().$

A.$a_{11}a_{22}a_{33}a_{44}$ $\quad$ B.$a_{32}a_{12}a_{44}a_{34}$ $\quad$ C.$-a_{21}a_{22}a_{23}a_{24}$ $\quad$ D.$a_{11}a_{12}a_{23}a_{24}$

答案解析:

$a_{11}a_{22}a_{33}a_{44}\mathrm{不是}D\mathrm{中的项},\mathrm{因为}a_{11},a_{22}\mathrm{在同一行}.$



题目:$设\begin{vmatrix}-1&1&1\\1&-1&x\\1&1&-1\end{vmatrix}\mathrm{是关于}x\mathrm{的一次多项式},\mathrm{该多项式一次项系数为}(\;).$

A.$-1$ $\quad$ B.$1$ $\quad$ C.$-2$ $\quad$ D.$2$

答案解析:

$含x\mathrm{的项为}x(-1)^{2+3}\begin{vmatrix}-1&1\\1&1\end{vmatrix}=2x,\mathrm{故一次项系数为}2.$



题目:$\mathrm{五阶行列式的展开式中},\mathrm{带负号的项是}(\;).$

A.$a_{12}a_{25}a_{31}a_{43}a_{54}$ $\quad$ B.$a_{12}a_{31}a_{25}a_{54}a_{43}$ $\quad$ C.$a_{11}a_{22}a_{33}a_{44}a_{55}$ $\quad$ D.$a_{12}a_{24}a_{31}a_{43}a_{55}$

答案解析:

$\mathrm{对于项}a_{12}a_{24}a_{31}a_{43}a_{55},\tau(\;24135)=3.$



题目:$\mathrm{四阶行列式}\begin{vmatrix}0&0&1&0\\0&1&0&0\\0&0&0&1\\1&0&0&0\end{vmatrix}=().$

A.$0$ $\quad$ B.$1$ $\quad$ C.$-1$ $\quad$ D.$-4$

答案解析:

$\begin{vmatrix}0&0&1&0\\0&1&0&0\\0&0&0&1\\1&0&0&0\end{vmatrix}=(-1)^{\tau(3241)}a_{13}a_{22}a_{34}a_{41}=1.$



题目:$\mathrm{四阶行列式}\begin{vmatrix}0&0&0&a\\0&0&b&0\\0&c&0&0\\d&0&0&0\end{vmatrix}=(\;).$

A.$0$ $\quad$ B.$a$ $\quad$ C.$abcd$ $\quad$ D.$-abcd$

答案解析:

$\begin{vmatrix}0&0&0&a\\0&0&b&0\\0&c&0&0\\d&0&0&0\end{vmatrix}=(-1)^{\tau(4321)}abcd=(-1)^6abcd=abcd.$



题目:$\mathrm{设多项式}f(x)=\begin{vmatrix}2x&3&1&2\\x&x&0&1\\2&1&x&4\\x&2&1&4x\end{vmatrix},则f(x)中x^4\mathrm{系数为}(\;).$

A.$2$ $\quad$ B.$4$ $\quad$ C.$8$ $\quad$ D.$6$

答案解析:

$\mathrm{根据行列式定义可以得出},含x^4\mathrm{的项只能是主对角线上元素的乘积},\mathrm{因此系数为}2\times4=8.$



题目:$\mathrm{四阶行列式}\begin{vmatrix}a&0&0&0\\0&0&b&0\\0&c&0&0\\0&0&0&d\end{vmatrix}=(\;).$

A.$0$ $\quad$ B.$1$ $\quad$ C.$abcd$ $\quad$ D.$-abcd$

答案解析:

$\begin{vmatrix}a&0&0&0\\0&0&b&0\\0&c&0&0\\0&0&0&d\end{vmatrix}=(-1)^{\tau(1324)}abcd=-abcd.$



题目:$\mathrm{四阶行列式的展开式中含有因子}a_{11}a_{23}\mathrm{的项为}(\;).$

A.$-a_{11}a_{23}a_{32}a_{44}和a_{11}a_{23}a_{34}a_{42}$ $\quad$ B.$a_{11}a_{23}a_{32}a_{44}和a_{11}a_{23}a_{34}a_{42}$ $\quad$ C.$-a_{11}a_{23}a_{32}a_{44}和-a_{11}a_{23}a_{34}a_{42}$ $\quad$ D.$a_{11}a_{23}a_{32}a_{44}和-a_{11}a_{23}a_{34}a_{42}$

答案解析:

$\begin{array}{l}\mathrm{由定义知},\mathrm{四阶行列式的一般项为}:(-1)^\tau a_{1p_1}a_{2p_2}a_{3p_3}a_{4p_4},\mathrm{其中}\tau 为p_1p_2p_3p_4\mathrm{的逆序数},\mathrm{由于}p_1=1,p_2=3\mathrm{已固定},p_1p_2p_3p_4\mathrm{只能形如}13xx,即1324或1342,\\其\tau\mathrm{分别为}0+0+1+0或0+0+0+2=2,\mathrm{所以}-a_{11}a_{23}a_{32}a_{44}和a_{11}a_{23}a_{34}a_{42}\mathrm{为所求}.\\\end{array}$



题目:$\mathrm{多项式}f(x)=\begin{vmatrix}x&3&1&2\\x&2x&0&1\\2&1&x&4\\0&2&1&x\end{vmatrix},则f(x)中x^4\mathrm{系数为}(\;).$

A.$2$ $\quad$ B.$1$ $\quad$ C.$4$ $\quad$ D.$8$

答案解析:

$\mathrm{由于行列式是不同行不同列元素乘积的和},\mathrm{故含有}x^4\mathrm{项只有行列式主对角线上元素的乘积},\mathrm{即系数为}2.$



题目:$若n\mathrm{阶行列式中有}n^2-n\mathrm{个以上的元素为零},\mathrm{则该行列式为}(\;).$

A.$0$ $\quad$ B.$n^2-n$ $\quad$ C.$n$ $\quad$ D.$\mathrm{无法确定}$

答案解析:

$\begin{array}{l}\mathrm{如果}n\mathrm{阶行列式中有}n^2-n\mathrm{个以上元素为零},\mathrm{则至多有}n-1\mathrm{个不为零元素}\\\mathrm{由于}n\;\mathrm{阶行列式的每一项为}n\mathrm{个不同元素的乘积},\mathrm{从而每一项均为零},\mathrm{故该行列式为为零}.\end{array}$



题目:$\mathrm{四阶行列式}\begin{vmatrix}0&0&1&0\\0&1&0&0\\0&0&0&1\\1&0&0&0\end{vmatrix}和\begin{vmatrix}1&1&1&0\\0&1&0&1\\0&1&1&1\\0&0&1&0\end{vmatrix}\mathrm{的值分别为}(\;).$

A.$1,0$ $\quad$ B.$0,0$ $\quad$ C.$1,1$ $\quad$ D.$0,1$

答案解析:

$\begin{array}{l}(1)\begin{vmatrix}0&0&1&0\\0&1&0&0\\0&0&0&1\\1&0&0&0\end{vmatrix}=(-1)^{\tau(3241)}=1,\\(2)\begin{vmatrix}1&1&1&0\\0&1&0&1\\0&1&1&1\\0&0&1&0\end{vmatrix}=(-1)^{\tau(1243)}+(-1)^{\tau(1423)}=0.\end{array}$



题目:$设\begin{vmatrix}x&x&1\\2x&x&1\\3&2&x\end{vmatrix},\mathrm{则展开式中}x^2\;\mathrm{的系数为}(\;).$

A.$1$ $\quad$ B.$-1$ $\quad$ C.$0$ $\quad$ D.$2$

答案解析:

$\begin{vmatrix}x&x&1\\2x&x&1\\3&2&x\end{vmatrix}=\begin{vmatrix}x&x&1\\0&-x&-1\\3&2&x\end{vmatrix}=\begin{vmatrix}x&0&1\\0&-x&-1\\3&-1&x\end{vmatrix}\mathrm{展开式不会出现}x^2.$



题目:$\mathrm{设多项式}f(x)=\begin{vmatrix}3&-1&x\\x&2&5\\1&4&x\end{vmatrix},则f(x)是()\mathrm{次多项式}.$

A.$0$ $\quad$ B.$1$ $\quad$ C.$2$ $\quad$ D.$3$

答案解析:

$f(x)=\begin{vmatrix}3&-1&x\\x&2&5\\1&4&x\end{vmatrix}=\begin{vmatrix}3&-1&x\\x&2&5\\-2&5&0\end{vmatrix}取a_{13}a_{21}a_{32}\mathrm{就会出现}x\mathrm{的最高次数}2.$



题目:$\mathrm{用行列式的定义计算五阶行列式}\begin{vmatrix}a_{11}&a_{12}&a_{13}&a_{14}&a_{15}\\a_{21}&a_{22}&a_{23}&a_{24}&a_{25}\\a_{31}&a_{32}&0&0&0\\a_{41}&a_{42}&0&0&0\\a_{51}&a_{52}&0&0&0\end{vmatrix}=(\;).$

A.$1$ $\quad$ B.$a_{11}a_{12}$ $\quad$ C.$0$ $\quad$ D.$a_{11}a_{22}a_{33}a_{44}a_{55}$

答案解析:

$\begin{array}{l}\mathrm{由于在五阶行列式的展开式中每一项的五个元素均分布在不同的行和不同的列},\mathrm{显然任意的这样五个元素中都至少有一个为零}\\\mathrm{从而行列式的每一项为零},\mathrm{故所求行列式为零}.\end{array}$



题目:$\mathrm{用行列式的定义计算五阶行列式}\begin{vmatrix}0&a_{12}&a_{13}&0&0\\a_{21}&a_{22}&a_{23}&a_{24}&a_{25}\\a_{31}&a_{32}&a_{33}&a_{34}&a_{35}\\0&a_{42}&a_{43}&0&0\\0&a_{52}&a_{53}&0&0\end{vmatrix}=(\;\;\;).$

A.$0$ $\quad$ B.$a_{11}a_{22}a_{33}a_{44}a_{55}$ $\quad$ C.$-a_{11}a_{22}a_{33}a_{44}a_{55}$ $\quad$ D.$a_{22}a_{33}$

答案解析:

$\begin{array}{l}设D\mathrm{中第}1,2,3,4,5\mathrm{行的元素分别为}a_{1p_1},a_{2p_2},a_{3p_3},a_{4p4},a_{5p_5},\mathrm{则由}D\mathrm{中第}1,2,3,4,5\mathrm{行可能的非零元素分别得到}p_1=2,3;p_2=1,2,3,4,5;p_3=1,2,3,4,5\\p_4=2,3;p_5=2,3;\;p_1,p_2,p_3,p_4,p_5\mathrm{在上述可能的代码中},\mathrm{一个}5\mathrm{元排列也不能组成},故D=0.\\\\\end{array}$



题目:$\mathrm{设多项式}f(x)=\begin{vmatrix}3&-1&x\\x&2&5\\1&4&x\end{vmatrix},则f(x)\mathrm{一次项的系数为}(\;).$

A.$-2$ $\quad$ B.$-4$ $\quad$ C.$2$ $\quad$ D.$4$

答案解析:

$\mathrm{一次项为}(-1)^{\tau(321)}x\times2\times1+3\times2\times x=4x,\mathrm{故系数为}4.$



题目:$\mathrm{按行列式的定义计算}n\mathrm{阶行列式}\begin{vmatrix}0&0&...&0&a_{1n}\\0&0&...&a_{2(n-1)}&a_{2n}\\...&...&...&...&...\\0&a_{(n-1)2}&...&a_{(n-1)(n-1)}&a_{(n-1)n}\\a_{n1}&a_{n2}&...&a_{n(n-1)}&a_{nn}\end{vmatrix}=(\;\;\;).$

A.$0$ $\quad$ B.$-a_{1n}a_{2(n-1)}a_{3(n-2)}....a_{(n-1)2}a_{n1}$ $\quad$ C.$a_{1n}a_{2(n-1)}a_{3(n-2)}....a_{(n-1)2}a_{n1}$ $\quad$ D.$(-1)^{\textstyle\frac{n(n-1)}2}{\textstyle{}^{}}a_{1n}a_{2(n-1)}...a_{(n-1)2}a_{n1}$

答案解析:

$\mathrm{原式}=(-1)^{\tau((n-1)....123)}a_{1n}a_{2(n-1)}...a_{(n-1)2}a_{n1}=(-1)^\frac{n(n-1)}2a_{1n}a_{2(n-1)}...a_{(n-1)2}a_{n1}.$



题目:$n\mathrm{阶行列式}\begin{vmatrix}0&0&...&0&1\\0&0&...&2&0\\...&...&...&...&...\\0&n-1&...&0&0\\n&0&...&0&0\end{vmatrix}=(\;).$

A.$n!$ $\quad$ B.$-n!$ $\quad$ C.$(-1)^{{\textstyle\frac12}n(n-1)}n!$ $\quad$ D.$(-1)^{\textstyle(n-1)}n!$

答案解析:

$\begin{array}{l}\mathrm{行列式的展开式中除了}(-1)^{\textstyle\tau(n(n-1)....321)}n!外,\mathrm{其余各项都为零},\mathrm{由于}(-1)^{\textstyle\tau(n(n-1)....321)}n!=(-1)^{{\textstyle\frac12}n(n-1)}n!,\mathrm{故原行列式}\\\mathrm{等于}(-1)^{\textstyle\frac12n(n-1)}n!.\end{array}$



题目:$\mathrm{设多项式}f(x)=\begin{vmatrix}2x&x&1&2\\1&2x&1&-1\\3&2&x&1\\1&1&1&x\end{vmatrix},则f(x)中x^4与x^3\mathrm{的系数分别为}(\;).$

A.$4,1$ $\quad$ B.$4,-1$ $\quad$ C.$1,2$ $\quad$ D.$1,-2$

答案解析:

$\begin{array}{l}含x^4\mathrm{的项只能由}a_{11}a_{22}a_{33}a_{44}\mathrm{组成},\mathrm{其系数为}4,\\含x^3\mathrm{的项只能由}a_{12}a_{21}a_{33}a_{44}\mathrm{组成},\mathrm{其系数为}-1.\end{array}$



题目:$n\mathrm{阶行列式}\begin{vmatrix}a_{11}&...&...&a_{1(n-1)}&a_{1n}\\a_{21}&...&...&a_{2(n-1)}&0\\a_{31}&...&...&0&0\\...&...&...&...&...\\a_{n1}&...&...&0&0\end{vmatrix}=(\;\;).$

A.$(-1)^{{\textstyle\frac12}n(n-1)}a_{1n}a_{2(n-1)}....a_{n1}$ $\quad$ B.$a_{1n}a_{2(n-1)}....a_{n1}$ $\quad$ C.$-a_{1n}a_{2(n-1)}....a_{n1}$ $\quad$ D.$(-1)^na_{1n}a_{2(n-1)}....a_{n1}$

答案解析:

$\begin{array}{l}\mathrm{根据行列式的定义},\mathrm{行列式中唯一可能不为零的项为}(-1)^{\tau(n(n-1)....321)}a_{1n}a_{2(n-1)}....a_{n1}\\\tau(n(n-1)....321)=\frac{n(n-1)}2,\mathrm{故原行列式为}(-1)^\frac{n(n-1)}2a_{1n}a_{2(n-1)}....a_{n1}.\end{array}$



题目:$\mathrm{行列式}D=\begin{vmatrix}0&0&...&0&1&0\\0&0&...&2&0&0\\...&...&...&...&...&...\\0&2015&...&0&0&0\\2016&0&...&0&0&0\\0&0&...&0&0&2017\end{vmatrix}=(\;\;\;).$

A.$-2017!$ $\quad$ B.$2017!$ $\quad$ C.$0$ $\quad$ D.$2017$

答案解析:

$\begin{array}{l}\mathrm{第一行的非零元素只有}a_{1,2016},故p_1\mathrm{只能取}2016,\mathrm{同理第},3,4,...2016\mathrm{可知}p_2=2001,p_3=2000...j_{2016}=1,p_{2017}=2017,\mathrm{于是在可能取的数码中},\\\mathrm{只能组成一个}2017\mathrm{级排列},\mathrm{故中非零项只有一项},即\;\;\;D=(-1)^{\tau(2016\;2015....2\;1\;2017)}\;a_{1,2016}a_{2,2015}\;.....a_{2016,1}a_{2017,2017}=2017!\;.\;\;\;\;\;\end{array}$



题目:$\mathrm{四阶行列式}\begin{vmatrix}-1&0&x&1\\1&1&-1&-1\\1&-1&1&-1\\1&-1&-1&1\end{vmatrix}\;,\mathrm{则展开式中}x\mathrm{的系数为}(\;)\;.$

A.$-4$ $\quad$ B.$4$ $\quad$ C.$2$ $\quad$ D.$-2$

答案解析:

$\begin{array}{l}x\mathrm{的系数应为}(-1)^{1+3}\begin{vmatrix}1&1&-1\\1&-1&-1\\1&-1&1\end{vmatrix}=-4.\\\end{array}$



题目:$n\mathrm{阶行列式}D_n=\begin{vmatrix}0&0&...&0&1&0\\0&0&...&2&0&0\\...&...&...&...&...&...\\n-1&0&...&0&0&0\\0&0&...&0&0&n\end{vmatrix}=(\;\;\;).$

A.$(-1)^\frac{(n-1)(n-2)}2\;\;\;n!$ $\quad$ B.$n!$ $\quad$ C.$(-1)^nn!$ $\quad$ D.$(-1)^\frac{(n+1)(n+2)}2\;\;n!$

答案解析:

$\begin{array}{l}D_n=(-1)^\tau a_{1(n-1)}a_{2(n-2)}...a_{nn}=(-1)^\tau n!,\mathrm{其中}\tau=\tau((n-1)(n-2)...21n)=0+1+2+....+(n-2)+0=\frac{(n-1)(n-2)}2\\故D_n=(-1)^\frac{(n-1)(n-2)}2n!.\end{array}$



题目:$n\mathrm{阶行列式}\begin{vmatrix}a_{11}&&&&a_{1n}\\&a_{22}&&&\\&&...&&\\&&&...&\\0&&&&a_{nn}\end{vmatrix}=\begin{vmatrix}a_{11}&&&&0\\&a_{22}&&&\\&&...&&\\&&&...&\\a_{n1}&&&&a_{nn}\end{vmatrix}=(\;).$

A.$0$ $\quad$ B.$a_{11}a_{22}...a_{nn}$ $\quad$ C.$-a_{11}a_{22}...a_{nn}$ $\quad$ D.$1$

答案解析:

$\begin{vmatrix}a_{11}&&&&a_{1n}\\&a_{22}&&&\\&&...&&\\&&&...&\\0&&&&a_{nn}\end{vmatrix}=\begin{vmatrix}a_{11}&&&&0\\&a_{22}&&&\\&&...&&\\&&&...&\\a_{n1}&&&&a_{nn}\end{vmatrix}=a_{11}a_{22}...a_{nn}.$



题目:$n\mathrm{阶行列式}\begin{vmatrix}0&&&&a_{1n}\\&&&a_{2(n-1)}&\\&&...&&\\&...&&&\\a_{n1}&&&&0\end{vmatrix}=(\;).$

A.$a_{1n}a_{2(n-1)}.....a_{n1}$ $\quad$ B.$-a_{1n}a_{2(n-1)}.....a_{n1}$ $\quad$ C.$(-1)^{\textstyle\frac{n(n-1)}2}a_{1n}a_{2(n-1)}.....a_{n1}$ $\quad$ D.$(-1)^{\textstyle\frac n2}a_{1n}a_{2(n-1)}.....a_{n1}$

答案解析:

$\begin{vmatrix}0&&&&a_{1n}\\&&&a_{2(n-1)}&\\&&...&&\\&...&&&\\a_{n1}&&&&0\end{vmatrix}=(-1)^{\tau(n(n-1)....1)}a_{1n}a_{2(n-1)}.....a_{n1}=(-1)^{\textstyle\frac{n(n-1)}2}a_{1n}a_{2(n-1)}.....a_{n1}.$



题目:$\mathrm{双曲面}x^2-y^2/4-z^2/9=1\mathrm{与平面}y=4\mathrm{交线为}(\;).$

A.$\mathrm{双曲线}$ $\quad$ B.$\mathrm{椭圆}$ $\quad$ C.$\mathrm{抛物线}$ $\quad$ D.$\mathrm{一对相交直线}$

答案解析:

$将y=4\mathrm{代入到双曲面方程得}x^2-z^2/9=5,\mathrm{所以相交的曲线为双曲线}.$



题目:$\mathrm{若五阶行列式的展开式中项}a_{13}a_{2k}a_{34}a_{42}a_{5l}\mathrm{带负号},\mathrm 则k,l\mathrm{的值分别为}(\;).$

A.$k=1,l=5$ $\quad$ B.$k=5,l=1$ $\quad$ C.$k=2,l=5$ $\quad$ D.$k=5,l=2$

答案解析:

$\begin{array}{l}\begin{array}{l}\mathrm{根据行列式的定义},k,l\mathrm{只能取}1或5,\\若k=5,l=1,则\tau(35421)=8,\end{array}\\\begin{array}{l}若k=1,l=5,则\tau(31425)=3,\\\mathrm{所以}k=1,l=5.\\\end{array}\end{array}$



题目:$\begin{array}{l}\mathrm{二阶行列式}\begin{vmatrix}\cos\;\alpha&-\sin\;\alpha\\\sin\alpha&\cos\;\alpha\end{vmatrix}=(\;).\;\\\end{array}$

A.$-1$ $\quad$ B.$1$ $\quad$ C.$2\sin^2\alpha$ $\quad$ D.$2\cos^2\alpha$

答案解析:

$\begin{vmatrix}\cos\;\alpha&-\sin\;\alpha\\\sin\alpha&\cos\;\alpha\end{vmatrix}=\cos^2\alpha-(-\sin^2\alpha)=\cos^2\alpha+\sin^2\alpha=1.$



题目:$\mathrm{若三阶行列式}\begin{vmatrix}1&2&5\\1&3&-2\\2&5&x\end{vmatrix}=0,则x\mathrm{的值为}().$

A.$-3$ $\quad$ B.$-2$ $\quad$ C.$2$ $\quad$ D.$3$

答案解析:

$\begin{vmatrix}1&2&5\\1&3&-2\\2&5&x\end{vmatrix}=3x-8+25-30+10-2x=0,x=3.$



题目:$\mathrm{三阶行列式}\begin{vmatrix}1&2&3\\4&0&5\\-1&0&6\end{vmatrix}\mathrm{的值为}().$

A.$0$ $\quad$ B.$-58$ $\quad$ C.$1$ $\quad$ D.$-48$

答案解析:

$\begin{vmatrix}1&2&3\\4&0&5\\-1&0&6\end{vmatrix}=1\times0\times6+2\times5\times(-1)+3\times0\times4-3\times0\times(-1)-2\times4\times6-1\times0\times5=-10-48=-58.$



题目:$\mathrm{三阶行列式}\begin{vmatrix}a&b&c\\b&c&a\\c&a&b\end{vmatrix}\mathrm{的值为}().$

A.$3abc-a^3-b^3-c^3$ $\quad$ B.$3abc+a^3+b^3+c^3$ $\quad$ C.$abc-a^3-b^3-c^3$ $\quad$ D.$abc+a^3+b^3+c^3$

答案解析:

$\begin{vmatrix}a&b&c\\b&c&a\\c&a&b\end{vmatrix}=acb+bac+cba-bbb-aaa-ccc=3abc-a^3-b^3-c^3.$



题目:$\mathrm{三阶行列式}\begin{vmatrix}a&b&b\\b&a&b\\b&b&a\end{vmatrix}\mathrm{的值为}(\;).$

A.$(a-b)^3$ $\quad$ B.$a^3-2b^3-3ab^2$ $\quad$ C.$a^3+2b^3-3ab^2$ $\quad$ D.$a^2+2b^3+3ab^3$

答案解析:

$\begin{vmatrix}a&b&b\\b&a&b\\b&b&a\end{vmatrix}=a^3+b^3+b^3-ab^2-b^2a-b^2a=a^3+2b^3-3ab^2.$



题目:$若a,b\in R,\begin{vmatrix}a&b&0\\-b&a&0\\100&0&-1\end{vmatrix}=0,则a,b\mathrm{的值分别为}().$

A.$a=0,b=1$ $\quad$ B.$a=0,b=0$ $\quad$ C.$a=1,b=0$ $\quad$ D.$a=1,b=1$

答案解析:

$\begin{vmatrix}a&b&0\\-b&a&0\\100&0&-1\end{vmatrix}=-a^2-b^2=0,a^2+b^2=0,a=b=0.$



题目:$\mathrm{三阶行列式}\begin{vmatrix}a&1&1\\0&-1&0\\4&a&a\end{vmatrix}<0\mathrm{的充要条件是}(\;\;\;).$

A.$a<2$ $\quad$ B.$a>-2$ $\quad$ C.$\left|a\right|>2$ $\quad$ D.$\left|a\right|<2$

答案解析:

$\begin{vmatrix}a&1&1\\0&-1&0\\4&a&a\end{vmatrix}=-a^2+4<0,\left|a\right|>2.$



题目:$\mathrm{三阶行列式}\begin{vmatrix}1&2&3\\4&0&5\\7&0&6\end{vmatrix}\mathrm{的值为}().$

A.$23$ $\quad$ B.$-24$ $\quad$ C.$22$ $\quad$ D.$-25$

答案解析:

$\begin{vmatrix}1&2&3\\4&0&5\\7&0&6\end{vmatrix}=1\times0\times6+2\times5\times7+3\times0\times4-3\times0\times7-2\times4\times6-1\times0\times5=22.$



题目:$\mathrm{若三阶行列式}\begin{vmatrix}1&2&5\\3&7&x\\6&9&0\end{vmatrix}=0,则x=(\;).$

A.$-25$ $\quad$ B.$-30$ $\quad$ C.$25$ $\quad$ D.$30$

答案解析:

$\begin{vmatrix}1&2&5\\3&7&x\\6&9&0\end{vmatrix}=12x+135-210-9x=3x-75=0,x=25.$



题目:$\mathrm{三阶行列式}\begin{vmatrix}a&-1&1\\0&1&0\\9&a&a\end{vmatrix}>0\mathrm{的充要条件是}(\;).$

A.$a<3$ $\quad$ B.$a>-3$ $\quad$ C.$\left|a\right|>3$ $\quad$ D.$\left|a\right|<3$

答案解析:

$\begin{vmatrix}a&-1&1\\0&1&0\\9&a&a\end{vmatrix}=a^2-9>0,则\left|a\right|>3.$



题目:$\mathrm{三阶行列式}\begin{vmatrix}1&2&0\\0&3&1\\3&0&2\end{vmatrix}\mathrm{的值为}(\;).$

A.$-18$ $\quad$ B.$18$ $\quad$ C.$12$ $\quad$ D.$-12$

答案解析:

$\begin{vmatrix}1&2&0\\0&3&1\\3&0&2\end{vmatrix}=1\times2\times3+1\times2\times3-0=12.$



题目:$\mathrm{三阶行列式}\begin{vmatrix}2&1&0\\1&0&2\\0&2&1\end{vmatrix}\mathrm{的值为}(\;).$

A.$45$ $\quad$ B.$-45$ $\quad$ C.$9$ $\quad$ D.$-9$

答案解析:

$\begin{vmatrix}2&1&0\\1&0&2\\0&2&1\end{vmatrix}=-1^3-2^3=-9.$



题目:$\mathrm{三阶行列式}\begin{vmatrix}1&0&-1\\3&5&0\\0&4&1\end{vmatrix}\mathrm{的值为}(\;).$

A.$7$ $\quad$ B.$-7$ $\quad$ C.$5$ $\quad$ D.$-5$

答案解析:

$\begin{vmatrix}1&0&-1\\3&5&0\\0&4&1\end{vmatrix}=1\times5\times1+0\times0\times0+(-1)\times3\times4-(-1)\times5\times0-3\times0\times1-1\times4\times0=-7.$



题目:$\mathrm{三阶行列式}\begin{vmatrix}2&0&0\\1&-4&-1\\-1&8&3\end{vmatrix}\mathrm{的值为}(\;).$

A.$-5$ $\quad$ B.$-8$ $\quad$ C.$-4$ $\quad$ D.$3$

答案解析:

$\begin{vmatrix}2&0&0\\1&-4&-1\\-1&8&3\end{vmatrix}=2\times(-4)\times3-2\times8\times(-1)=-24+16=-8.$



题目:$\mathrm{三阶行列式}\begin{vmatrix}0&a&0\\b&0&c\\0&d&0\end{vmatrix}\mathrm{的值为}(\;\;\;).$

A.$0$ $\quad$ B.$abcd$ $\quad$ C.$-abcd$ $\quad$ D.$acd$

答案解析:

$\begin{vmatrix}0&a&0\\b&0&c\\0&d&0\end{vmatrix}=0\times0\times0+b\times d\times0+0\times c\times a-0\times0\times0-a\times b\times0-0\times d\times c=0.$



题目:$\mathrm{三阶行列式}\begin{vmatrix}1&1&1\\a&b&c\\a^2&b^2&c^2\end{vmatrix}\mathrm{的值为}(\;).$

A.$(a-b)(b-c)(c-a)$ $\quad$ B.$(a-b)(b-c)(a-c)$ $\quad$ C.$(a-1)(b-1)(c-1)$ $\quad$ D.$(1-b)(1-c)(1-a)$

答案解析:

$\begin{vmatrix}1&1&1\\a&b&c\\a^2&b^2&c^2\end{vmatrix}=bc^2+ca^2+ab^2-ac^2-ba^2-cb^2=(a-b)(b-c)(c-a)\mathrm{或者利用范德蒙行列式可得}.$



题目:$\mathrm{三阶行列式}\begin{vmatrix}1&1&1\\3&1&4\\0&0&5\end{vmatrix}\mathrm{的值为}(\;\;\;).$

A.$8$ $\quad$ B.$5$ $\quad$ C.$15$ $\quad$ D.$-10$

答案解析:

$\begin{vmatrix}1&1&1\\3&1&4\\0&0&5\end{vmatrix}=1\times1\times5-3\times1\times5=-10.$



题目:$\mathrm{三阶行列式}\begin{vmatrix}2&0&1\\1&-4&-1\\-1&0&3\end{vmatrix}\mathrm{的值为}(\;).$

A.$8$ $\quad$ B.$-24$ $\quad$ C.$16$ $\quad$ D.$-28$

答案解析:

$\begin{vmatrix}2&0&1\\1&-4&-1\\-1&0&3\end{vmatrix}=2\times(-4)\times3-1\times(-4)\times(-1)=-28.$



题目:$\mathrm{二阶行列式}\begin{vmatrix}x-1&1\\x^3&x^2+x+1\end{vmatrix}\mathrm{的值为}(\;\;\;).$

A.$0$ $\quad$ B.$1$ $\quad$ C.$x^2+x$ $\quad$ D.$-1$

答案解析:

$\begin{vmatrix}x-1&1\\x^3&x^2+x+1\end{vmatrix}=(x-1)(x^2+x+1)-x^3=-1.$



题目:$\mathrm{二阶行列式}\begin{vmatrix}a+b&b+d\\a+c&c+d\end{vmatrix}\mathrm{的值为}(\;\;\;).$

A.$(a-d)(c-b)$ $\quad$ B.$(a-b)(c-d)$ $\quad$ C.$(a-d)(c+b)$ $\quad$ D.$(a-b)(c+d)$

答案解析:

$\begin{vmatrix}a+b&b+d\\a+c&c+d\end{vmatrix}=(a+b)(c+d)-(a+c)(b+d)=ac+bc+ad+bd-ab-bc-ad-cd=ac-ab+bd-cd=a(c-b)-d(c-b)=(a-d)(c-b).$



题目:$\mathrm{二阶行列式}\begin{vmatrix}a^2&ab\\b&b\end{vmatrix}\mathrm{的值为}(\;\;\;).$

A.$a^3b-b^2$ $\quad$ B.$b^2-a^3b$ $\quad$ C.$a^2b-ab^2$ $\quad$ D.$a^2b+ab^2$

答案解析:

$\begin{vmatrix}a^2&ab\\b&b\end{vmatrix}=a^2b-ab^2.$



题目:$\mathrm{二阶行列式}\begin{vmatrix}a+b&b\\a+c&c\end{vmatrix}\mathrm{的值为}(\;\;\;).$

A.$ac-ab$ $\quad$ B.$ac+b$ $\quad$ C.$ac-b$ $\quad$ D.$ac+bc$

答案解析:

$\begin{vmatrix}a+b&b\\a+c&c\end{vmatrix}=(a+b)c-(a+c)b=ac+bc-ab-bc=ac-ab.$



题目:$\mathrm{二阶行列式}\begin{vmatrix}1+\sqrt2&2-\sqrt3\\2+\sqrt3&1-\sqrt2\end{vmatrix}\mathrm{的值为}(\;\;\;).$

A.$\sqrt2$ $\quad$ B.$\sqrt3$ $\quad$ C.$2$ $\quad$ D.$-2$

答案解析:

$\begin{vmatrix}1+\sqrt2&2-\sqrt3\\2+\sqrt3&1-\sqrt2\end{vmatrix}=(1+\sqrt2)(1-\sqrt2)-(2+\sqrt3)(2-\sqrt3)=1-2-4+3=-2.$



题目:$\mathrm{二阶行列式}\begin{vmatrix}a_{11}&a_{12}\\a_{21}&a_{22}\end{vmatrix}\mathrm{的值为}(\;\;\;).$

A.$a_{11}a_{12}-a_{22}a_{21}$ $\quad$ B.$a_{11}a_{12}+a_{22}a_{21}$ $\quad$ C.$a_{11}a_{22}-a_{12}a_{21}$ $\quad$ D.$a_{11}a_{22}+a_{12}a_{21}$

答案解析:

$\begin{vmatrix}a_{11}&a_{12}\\a_{21}&a_{22}\end{vmatrix}=a_{11}a_{22}-a_{12}a_{21}.$



题目:$\mathrm{二阶行列式}\begin{vmatrix}a&b\\a^2&b\end{vmatrix}\mathrm{的值为}(\;\;\;).$

A.$ab^2$ $\quad$ B.$a^2b$ $\quad$ C.$ab(1-a)$ $\quad$ D.$ab(a-b)$

答案解析:

$\begin{vmatrix}a&b\\a^2&b\end{vmatrix}=ab-a^2b=ab(1-a).$



题目:$\mathrm{二阶行列式}\begin{vmatrix}x-1&1\\x^2&x^2+x+1\end{vmatrix}\mathrm{的值为}(\;).$

A.$1$ $\quad$ B.$-1$ $\quad$ C.$x^3-x^2-1$ $\quad$ D.$x^3-x^2+1$

答案解析:

$\begin{vmatrix}x-1&1\\x^2&x^2+x+1\end{vmatrix}=(x-1)(x^2+x+1)-1\times x^2=x^3-x^2-1.$



题目:$\mathrm{二阶行列式}\begin{vmatrix}1-t^2&2t\\-2t&1-t^2\end{vmatrix}\mathrm{的值为}(\;\;\;).$

A.$(1+t^2)^2$ $\quad$ B.$1+t^2$ $\quad$ C.$(1-t^2)^2$ $\quad$ D.$1$

答案解析:

$\begin{vmatrix}1-t^2&2t\\-2t&1-t^2\end{vmatrix}=(1-t^2)(1-t^2)-(-2t)\times2t=1+t^4+2t^2=(1+t^2)^2.$



题目:$\mathrm{三阶行列式}\begin{vmatrix}0&2&3\\3&0&2\\2&3&0\end{vmatrix}\mathrm{的值为}(\;).$

A.$12$ $\quad$ B.$15$ $\quad$ C.$35$ $\quad$ D.$21$

答案解析:

$\begin{vmatrix}0&2&3\\3&0&2\\2&3&0\end{vmatrix}=2\times2\times2+3\times3\times3=35.$



题目:$\mathrm{三阶行列式}\begin{vmatrix}3&1&x\\4&x&0\\1&0&x\end{vmatrix}\neq0,\mathrm{则有}(\;).$

A.$x\neq0$ $\quad$ B.$x\neq2$ $\quad$ C.$x\neq0且x\neq2$ $\quad$ D.$x\neq0或x\neq2$

答案解析:

$\begin{vmatrix}3&1&x\\4&x&0\\1&0&x\end{vmatrix}=3\times x\times x+0+0-x\times x\times1-0-1\times4\times x=3x^2-x^2-4x=2x(x-2)\neq0.$



题目:$\mathrm{已知}\omega^3=1,则\begin{vmatrix}\omega&\omega^2&1\\\omega^2&1&\omega\\1&\omega&\omega^2\end{vmatrix}=().$

A.$1$ $\quad$ B.$-1$ $\quad$ C.$0$ $\quad$ D.$\omega$

答案解析:

$\begin{vmatrix}\omega&\omega^2&1\\\omega^2&1&\omega\\1&\omega&\omega^2\end{vmatrix}=\omega^3+\omega^3+\omega^3-1-\omega^3-\omega^6=0.$



题目:$\mathrm{三阶行列式}\begin{vmatrix}b^2&ab&ac\\ab&c^2&bc\\ca&cb&a^2\end{vmatrix}\mathrm{的值为}(\;\;\;).$

A.$a^2b^2c^2$ $\quad$ B.$2a^2b^2c^2$ $\quad$ C.$3a^2b^2c^2-a^2c^4-b^2a^4-c^2b^4$ $\quad$ D.$0$

答案解析:

$\begin{vmatrix}b^2&ab&ac\\ab&c^2&bc\\ca&cb&a^2\end{vmatrix}=3a^2b^2c^2-a^2c^4-b^2a^4-c^2b^4.$



题目:$\mathrm{三阶行列式}\begin{vmatrix}1&2&3\\4&5&6\\3&2&1\end{vmatrix}\mathrm{的值为}(\;\;\;).$

A.$-1$ $\quad$ B.$-2$ $\quad$ C.$0$ $\quad$ D.$-4$

答案解析:

$\begin{vmatrix}1&2&3\\4&5&6\\3&2&1\end{vmatrix}=5+24+36-45-8-12=0.$



题目:$\mathrm{三阶行列式}\begin{vmatrix}1&2&3\\2&3&4\\3&4&5\end{vmatrix}\mathrm{的值为}(\;).$

A.$0$ $\quad$ B.$1$ $\quad$ C.$2$ $\quad$ D.$3$

答案解析:

$\begin{vmatrix}1&2&3\\2&3&4\\3&4&5\end{vmatrix}=15+24+24-27-20-16=0.$



题目:$\begin{vmatrix}a&b\\c&d\end{vmatrix}+\begin{vmatrix}0&b&a\\1&a&b\\0&d&c\end{vmatrix}=(\;).$

A.$ad-cd$ $\quad$ B.$2(ad-bc)$ $\quad$ C.$0$ $\quad$ D.$ad+bc$

答案解析:

$\begin{vmatrix}a&b\\c&d\end{vmatrix}+\begin{vmatrix}0&b&a\\1&a&b\\0&d&c\end{vmatrix}=(ad-bc)+(ad-bc)=2(ad-bc).$



题目:$\begin{vmatrix}0&b&a\\1&a&b\\0&d&c\end{vmatrix}-\begin{vmatrix}a&b\\c&d\end{vmatrix}=(\;).$

A.$ad-dc$ $\quad$ B.$2(ad-dc)$ $\quad$ C.$0$ $\quad$ D.$1$

答案解析:

$\begin{vmatrix}0&b&a\\1&a&b\\0&d&c\end{vmatrix}-\begin{vmatrix}a&b\\c&d\end{vmatrix}=(ad-bc)-(ad-bc)=0.$



题目:$\mathrm{二阶行列式}\begin{vmatrix}1&\log_ba\\\log_ab&1\end{vmatrix}\mathrm{的值为}(\;).$

A.$1$ $\quad$ B.$0$ $\quad$ C.$-1$ $\quad$ D.$与a,b\mathrm{有关}$

答案解析:

$\begin{vmatrix}1&\log_ba\\\log_ab&1\end{vmatrix}=1\times1-\log_ba\times\log_ab=0.$



题目:$\mathrm{方程}\begin{vmatrix}1&1&1\\2&3&x\\4&9&x^2\end{vmatrix}=0\mathrm{的解为}(\;).$

A.$x=2或x=3$ $\quad$ B.$x=2$ $\quad$ C.$x=3$ $\quad$ D.$\mathrm{无解}$

答案解析:

$\begin{vmatrix}1&1&1\\2&3&x\\4&9&x^2\end{vmatrix}=3x^2+4x+18-12-9x-2x^2=x^2-5x+6,则x=2或x=3.$



题目:$\begin{array}{l}\mathrm{下列计算正确的个数有}(\;)个.\\(1)\begin{vmatrix}1&0&-1\\3&5&0\\0&4&1\end{vmatrix}=-7\;\;\;(2)\begin{vmatrix}2&0&1\\1&-4&-1\\-1&8&3\end{vmatrix}=4\;\;\;(3)\begin{vmatrix}0&a&0\\b&0&c\\0&d&0\end{vmatrix}=0\end{array}$

A.$0$ $\quad$ B.$2个$ $\quad$ C.$1个$ $\quad$ D.$3个$

答案解析:

$\begin{array}{l}(1)\begin{vmatrix}1&0&-1\\3&5&0\\0&4&1\end{vmatrix}=1\times5\times1+0\times0\times0+(-1)\times3\times4-(-1)\times5\times0-3\times0\times1-1\times4\times0=-7,\\\;(2)\begin{vmatrix}2&0&1\\1&-4&-1\\-1&8&3\end{vmatrix}=2\times(-4)\times3+0\times(-1)\times(-1)+1\times1\times8-0\times1\times3-2\times(-1)\times8-1\times(-4)\times(-1)=-24+8+16-4=-4,\\\;(3)\begin{vmatrix}0&a&0\\b&0&c\\0&d&0\end{vmatrix}=0\times0\times0+0\times b\times d+0\times a\times c-0\times0\times0-b\times a\times0-0\times d\times c=0.\end{array}$



题目:$\mathrm{三阶行列式}D_1=\begin{vmatrix}1&3&1\\2&2&3\\3&1&5\end{vmatrix},D_2=\begin{vmatrix}\lambda&0&1\\0&\lambda-1&0\\1&0&\lambda\end{vmatrix},若D_1=D_2,则\lambda\mathrm{取值为}(\;).$

A.$0,1$ $\quad$ B.$0,2$ $\quad$ C.$1,-1$ $\quad$ D.$2,-1$

答案解析:

$\begin{array}{l}D_1=10+2+27-6-30-3=0,\\D_2=(\lambda+1)(\lambda-1)^2,\\若D_1=D_2,则(\lambda+1)(\lambda-1)^2=0,\mathrm{解得λ}=1\mathrm{或者}\lambda=-1.\end{array}$



题目:$\mathrm{三阶行列式}\begin{vmatrix}x&y&x+y\\y&x+y&x\\x+y&x&y\end{vmatrix}\mathrm{的值为}(\;).$

A.$-2(x^3+y^3)$ $\quad$ B.$2(x^3+y^3)$ $\quad$ C.$-(x^3+y^3)$ $\quad$ D.$(x^3+y^3)$

答案解析:

$\begin{array}{l}\begin{vmatrix}x&y&x+y\\y&x+y&x\\x+y&x&y\end{vmatrix}=x(x+y)y+yx(x+y)+(x+y)yx-y^3-(x+y)^3-x^3\\=3xy(x+y)-y^3-3x^2y-3y^2x-x^3-y^3-x^3=-2(x^3+y^3).\end{array}$



题目:$\mathrm{四阶行列式}\begin{vmatrix}0&0&0&2\\0&0&-1&1\\0&3&0&-1\\2&0&2&1\end{vmatrix}=().$

A.$12$ $\quad$ B.$-11$ $\quad$ C.$-12$ $\quad$ D.$11$

答案解析:

$\begin{vmatrix}0&0&0&2\\0&0&-1&1\\0&3&0&-1\\2&0&2&1\end{vmatrix}\overset{r_1\leftrightarrow r_4}{\overset{r_{2\;}\leftrightarrow r_3}=}\begin{vmatrix}2&0&2&1\\0&3&0&-1\\0&0&-1&1\\0&0&0&2\end{vmatrix}=2\times3\times\left(-1\right)\times2=-12.$



题目:$\mathrm{四阶行列式}\begin{vmatrix}0&0&0&2\\0&2&-1&0\\0&3&1&0\\2&0&0&0\end{vmatrix}=().$

A.$10$ $\quad$ B.$-20$ $\quad$ C.$30$ $\quad$ D.$-40$

答案解析:

$\begin{vmatrix}0&0&0&2\\0&2&-1&0\\0&3&1&0\\2&0&0&0\end{vmatrix}=-\begin{vmatrix}2&0&0&0\\0&2&-1&0\\0&3&1&0\\0&0&0&2\end{vmatrix}=-\begin{vmatrix}2&0&0&0\\0&2&-1&0\\0&0&\frac52&0\\0&0&0&2\end{vmatrix}=-2\times2\times\frac52\times2=-20.$



题目:$\mathrm{四阶行列式}\begin{vmatrix}a_{11}&a_{12}&a_{13}&a_{14}\\a_{21}&a_{22}&a_{23}&0\\a_{31}&a_{32}&0&0\\a_{41}&0&0&0\end{vmatrix}=().$

A.$-a_{14}a_{23}a_{32}a_{41}$ $\quad$ B.$a_{14}a_{23}a_{32}a_{41}$ $\quad$ C.$a_{14}a_{13}a_{12}a_{11}$ $\quad$ D.$0$

答案解析:

$\begin{vmatrix}a_{11}&a_{12}&a_{13}&a_{14}\\a_{21}&a_{22}&a_{23}&0\\a_{31}&a_{32}&0&0\\a_{41}&0&0&0\end{vmatrix}=\begin{vmatrix}a_{14}&a_{13}&a_{12}&a_{11}\\0&a_{23}&a_{22}&a_{21}\\0&0&a_{32}&a_{31}\\0&0&0&a_{41}\end{vmatrix}=a_{14}a_{23}a_{32}a_{41}.$



题目:$设\;\begin{vmatrix}x&1&1\\1&x&1\\1&1&x\end{vmatrix}=0,则x=().$

A.$0或1$ $\quad$ B.$0\;或\;-2$ $\quad$ C.$1或\;-2$ $\quad$ D.$0或-1$

答案解析:

$\begin{vmatrix}x&1&1\\1&x&1\\1&1&x\end{vmatrix}=\begin{vmatrix}x+2&1&1\\x+2&x&1\\x+2&1&x\end{vmatrix}=\left(x+2\right)\begin{vmatrix}1&1&1\\1&x&1\\1&1&x\end{vmatrix}=\left(x+2\right)\begin{vmatrix}1&1&1\\0&x-1&0\\0&0&x-1\end{vmatrix}=\left(x+2\right)\left(x-1\right)\;^2=0.$



题目:$设\begin{vmatrix}x&2&2\\2&x&2\\2&2&x\end{vmatrix}=0,则\;x=().$

A.$0或2$ $\quad$ B.$0\;或\;-4$ $\quad$ C.$-4或\;2$ $\quad$ D.$0或4$

答案解析:

$\begin{vmatrix}x&2&2\\2&x&2\\2&2&x\end{vmatrix}=\begin{vmatrix}x+4&2&2\\x+4&x&2\\x+4&2&x\end{vmatrix}=\left(x+4\right)\begin{vmatrix}1&2&2\\1&x&2\\1&2&x\end{vmatrix}=\left(x+4\right)\begin{vmatrix}1&2&2\\0&x-2&0\\0&0&x-2\end{vmatrix}=\left(x+4\right)\left(x-2\right)\;^2=0.$



题目:$设\begin{vmatrix}x&3&3\\3&x&3\\3&3&x\end{vmatrix}=0,则x=().$

A.$0或3$ $\quad$ B.$0\;或\;-6$ $\quad$ C.$-6或\;3$ $\quad$ D.$0或6$

答案解析:

$\begin{vmatrix}x&3&3\\3&x&3\\3&3&x\end{vmatrix}=\begin{vmatrix}x+6&3&3\\x+6&x&3\\x+6&3&x\end{vmatrix}=\left(x+6\right)\begin{vmatrix}1&3&3\\1&x&3\\1&3&x\end{vmatrix}=\left(x+6\right)\begin{vmatrix}1&3&3\\0&x-3&0\\0&0&x-3\end{vmatrix}=\left(x+6\right)\left(x-3\right)^2=0.$



题目:$设\begin{vmatrix}2x&2&2\\2&2x&2\\2&2&2x\end{vmatrix}=0,则\;x=().$

A.$0或1$ $\quad$ B.$0\;或\;-2$ $\quad$ C.$1或\;-2$ $\quad$ D.$0或-1$

答案解析:

$\begin{vmatrix}2x&2&2\\2&2x&2\\2&2&2x\end{vmatrix}=8\begin{vmatrix}x&1&1\\1&x&1\\1&1&x\end{vmatrix}=8\begin{vmatrix}x+2&1&1\\x+2&x&1\\x+2&1&x\end{vmatrix}=8(x+2)\begin{vmatrix}1&1&1\\1&x&1\\1&1&x\end{vmatrix}=8(x+2)\begin{vmatrix}1&1&1\\0&x-1&0\\0&0&x-1\end{vmatrix}=8(x+2)(x-1)^2=0.$



题目:$设\begin{vmatrix}x&1&1&1\\1&x&1&1\\1&1&x&1\\1&1&1&x\end{vmatrix}=0,则x=().$

A.$-3或0$ $\quad$ B.$0或1$ $\quad$ C.$-3或1$ $\quad$ D.$1或3$

答案解析:

$\begin{vmatrix}x&1&1&1\\1&x&1&1\\1&1&x&1\\1&1&1&x\end{vmatrix}=\begin{vmatrix}x+3&1&1&1\\x+3&x&1&1\\x+3&1&x&1\\x+3&1&1&x\end{vmatrix}=\left(x+3\right)\begin{vmatrix}1&1&1&1\\1&x&1&1\\1&1&x&1\\1&1&1&x\end{vmatrix}=\left(x+3\right)\begin{vmatrix}1&1&1&1\\0&x-1&0&0\\0&0&x-1&0\\0&0&0&x-1\end{vmatrix}=\left(x+3\right)\left(x-1\right)^3=0.$



题目:$设\begin{vmatrix}x&2&2&2\\2&x&2&2\\2&2&x&2\\2&2&2&x\end{vmatrix}=0,则x=().$

A.$-6或0$ $\quad$ B.$0或2$ $\quad$ C.$-6或2$ $\quad$ D.$1或6$

答案解析:

$\begin{vmatrix}x&2&2&2\\2&x&2&2\\2&2&x&2\\2&2&2&x\end{vmatrix}=\begin{vmatrix}x+6&2&2&2\\x+6&x&2&2\\x+6&2&x&2\\x+6&2&2&x\end{vmatrix}=(x+6)\begin{vmatrix}1&2&2&2\\1&x&2&2\\1&2&x&2\\1&2&2&x\end{vmatrix}=(x+6)\begin{vmatrix}1&2&2&2\\0&x-2&0&0\\0&0&x-2&0\\0&0&0&x-2\end{vmatrix}=\left(x+6\right)\left(x-2\right)^3=0.$



题目:$设\begin{vmatrix}x&3&3&3\\3&x&3&3\\3&3&x&3\\3&3&3&x\end{vmatrix}=0,则x=().$

A.$-9或0$ $\quad$ B.$0或3$ $\quad$ C.$-9或3$ $\quad$ D.$3或9$

答案解析:

$\begin{vmatrix}x&3&3&3\\3&x&3&3\\3&3&x&3\\3&3&3&x\end{vmatrix}=\begin{vmatrix}x+9&3&3&3\\x+9&x&3&3\\x+9&3&x&3\\x+9&3&3&x\end{vmatrix}=(x+9)\begin{vmatrix}1&3&3&3\\1&x&3&3\\1&3&x&3\\1&3&3&x\end{vmatrix}=(x+9)\begin{vmatrix}1&3&3&3\\0&x-3&0&0\\0&0&x-3&0\\0&0&0&x-3\end{vmatrix}=\left(x+9\right)\left(x-3\right)^3=0.$



题目:$\mathrm{三阶行列式}\begin{vmatrix}x+1&2&3\\1&x+2&3\\1&2&x+3\end{vmatrix}=(\;\;\;\;).\;$

A.$\left(x+6\right)x^2$ $\quad$ B.$\left(x+6\right)$ $\quad$ C.$\left(x+6\right)x$ $\quad$ D.$x^2$

答案解析:

$\begin{vmatrix}x+1&2&3\\1&x+2&3\\1&2&x+3\end{vmatrix}=\begin{vmatrix}x+6&2&3\\x+6&x+2&3\\x+6&2&x+3\end{vmatrix}=\left(x+6\right)\begin{vmatrix}1&2&3\\1&x+2&3\\1&2&x+3\end{vmatrix}=\left(x+6\right)\begin{vmatrix}1&2&3\\0&x&0\\0&0&x\end{vmatrix}=\left(x+6\right)x^2.$



题目:$\mathrm{四阶行列式}\begin{vmatrix}1&1&1&1\\-1&2&1&1\\-1&-1&3&1\\-1&-1&-1&4\end{vmatrix}=(\;\;\;\;).\;$

A.$40$ $\quad$ B.$50$ $\quad$ C.$60$ $\quad$ D.$30$

答案解析:

$\begin{vmatrix}1&1&1&1\\-1&2&1&1\\-1&-1&3&1\\-1&-1&-1&4\end{vmatrix}=\begin{vmatrix}1&1&1&1\\0&3&2&2\\0&0&4&2\\0&0&0&5\end{vmatrix}=1\times3\times4\times5=60.$



题目:$\mathrm{下列哪个不是方程}D\left(x\right)=\begin{vmatrix}1&1&1&1\\1&x&2&2\\2&2&x&3\\3&3&3&x\end{vmatrix}=0\mathrm{的解}(\;\;\;).$

A.$1$ $\quad$ B.$2$ $\quad$ C.$3$ $\quad$ D.$0$

答案解析:

$D\left(x\right)=\begin{vmatrix}1&1&1&1\\1&x&2&2\\2&2&x&3\\3&3&3&x\end{vmatrix}=\begin{vmatrix}1&1&1&1\\0&x-1&1&1\\0&0&x-2&1\\0&0&0&x-3\end{vmatrix}=\left(x-1\right)\left(x-2\right)\left(x-3\right)=0,即x_1=1,x_2=2,x_3=3.$



题目:$\mathrm{四阶行列式}\begin{vmatrix}1&1&1&1\\-1&1&1&1\\-1&-1&1&1\\-1&-1&-1&1\end{vmatrix}=(\;\;\;\;).\;$

A.$1$ $\quad$ B.$4$ $\quad$ C.$8$ $\quad$ D.$0$

答案解析:

$\begin{vmatrix}1&1&1&1\\-1&1&1&1\\-1&-1&1&1\\-1&-1&-1&1\end{vmatrix}=\begin{vmatrix}1&1&1&1\\0&2&2&2\\0&0&2&2\\0&0&0&2\end{vmatrix}=8.$



题目:$\mathrm{四阶行列式}\begin{vmatrix}4&1&2&4\\1&2&0&2\\10&5&2&0\\0&1&1&7\end{vmatrix}=(\;\;\;).$

A.$0$ $\quad$ B.$112$ $\quad$ C.$4$ $\quad$ D.$1$

答案解析:

$\begin{vmatrix}4&1&2&4\\1&2&0&2\\10&5&2&0\\0&1&1&7\end{vmatrix}=-\begin{vmatrix}1&2&0&2\\4&1&2&4\\10&5&2&0\\0&1&1&7\end{vmatrix}=-\begin{vmatrix}1&2&0&2\\0&-7&2&-4\\0&-15&2&-20\\0&1&1&7\end{vmatrix}=\begin{vmatrix}1&2&0&2\\0&1&1&7\\0&-15&2&-20\\0&-7&2&-4\end{vmatrix}=\begin{vmatrix}1&2&0&2\\0&1&1&7\\0&0&17&85\\0&0&9&45\end{vmatrix}=\begin{vmatrix}1&2&0&2\\0&1&1&7\\0&0&17&85\\0&0&0&0\end{vmatrix}=0.$



题目:$\mathrm{四阶行列式}\begin{vmatrix}1&1&1&1\\-1&1&1&1\\0&-1&1&1\\0&0&-1&1\end{vmatrix}=().$

A.$8$ $\quad$ B.$-8$ $\quad$ C.$4$ $\quad$ D.$-4$

答案解析:

$\begin{vmatrix}1&1&1&1\\-1&1&1&1\\0&-1&1&1\\0&0&-1&1\end{vmatrix}=\begin{vmatrix}1&1&1&1\\0&2&2&2\\0&-1&1&1\\0&0&-1&1\end{vmatrix}=2\begin{vmatrix}1&1&1&1\\0&1&1&1\\0&-1&1&1\\0&0&-1&1\end{vmatrix}=2\begin{vmatrix}1&1&1&1\\0&1&1&1\\0&0&2&2\\0&0&-1&1\end{vmatrix}=4\begin{vmatrix}1&1&1&1\\0&1&1&1\\0&0&1&1\\0&0&-1&1\end{vmatrix}=4\begin{vmatrix}1&1&1&1\\0&1&1&1\\0&0&1&1\\0&0&0&2\end{vmatrix}=8.$



题目:$\mathrm{四阶行列式}\begin{vmatrix}1&1&1&0\\1&2&3&4\\1&3&6&10\\0&4&10&20\end{vmatrix}=().$

A.$1$ $\quad$ B.$0$ $\quad$ C.$2$ $\quad$ D.$3$

答案解析:

$\mathrm{原式}=\begin{vmatrix}1&1&1&0\\0&1&2&4\\0&2&5&10\\0&4&10&20\end{vmatrix}=0.$



题目:$\mathrm{三阶行列式}\begin{vmatrix}1&2&3\\2&3&1\\0&1&2018\end{vmatrix}=(\;\;\;).$

A.$-2018$ $\quad$ B.$2013$ $\quad$ C.$2018$ $\quad$ D.$-2013$

答案解析:

$\begin{vmatrix}1&2&3\\2&3&1\\0&1&2018\end{vmatrix}=\begin{vmatrix}1&2&3\\0&-1&-5\\0&1&2018\end{vmatrix}=\begin{vmatrix}1&2&3\\0&-1&-5\\0&0&2013\end{vmatrix}=-2013.$



题目:$\mathrm{四阶行列式}\begin{vmatrix}1&1&1&1\\1&-1&1&1\\1&1&-1&1\\1&1&1&-1\end{vmatrix}=(\;\;\;).\;$

A.$-1$ $\quad$ B.$-4$ $\quad$ C.$-8$ $\quad$ D.$1$

答案解析:

$D\overset{}=\begin{vmatrix}1&1&1&1\\0&-2&0&0\\0&0&-2&0\\0&0&0&-2\end{vmatrix}=-8.$