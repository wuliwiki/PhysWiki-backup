% 拉普拉斯算符
% keys 梯度|散度|拉普拉斯方程

\begin{issues}
\issueDraft
\end{issues}

\pentry{梯度、梯度定理\upref{Grad}}

\footnote{参考 Wikipedia \href{https://en.wikipedia.org/wiki/Laplace_operator}{相关页面}.}我们令一个标量函数 $u(x, y, z)$ 的梯度的散度为它的\textbf{拉普拉斯(Laplacian)}, 合成的算符(类比复合函数)叫做\textbf{拉普拉斯算符}, 记为 $\laplacian$.
\begin{equation}
\laplacian u = \div (\grad u) = \pdv[2]{u}{x} + \pdv[2]{u}{y} + \pdv[2]{u}{z}
\end{equation}
也可以记
\begin{equation}
\begin{aligned}
\laplacian &= \Nabla \vdot \Nabla = \qty(\uvec x\pdv{x} + \uvec y\pdv{y} + \uvec z\pdv{z})^2\\
&= \pdv[2]{u}{x} + \pdv[2]{u}{y} + \pdv[2]{u}{z}
\end{aligned}
\end{equation}
这些定义也容易拓展到 $N = 1, 2, \dots$ 元函数上.

柱坐标系中的拉普拉斯算符柱坐标系中的矢量算符见\autoref{CylNab_eq4}~\upref{CylNab}, 
