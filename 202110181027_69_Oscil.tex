% 小振动
% keys 小振动

\pentry{拉格朗日方程\ref{Lagrng}}

\subsection{一维小振动}
一维谐振子的拉格朗日量为
\begin{equation}
L=\frac{1}{2}m\cdot{x}^2-\frac{1}{2}kx^2
\end{equation}
其运动方程为 $m\ddot {x}+kx=0$,或写作 $\ddot{x}+\omega^2 x=0$,其中 $\omega=\sqrt{k/m}$ 为该系统的固有频率.

受迫振动的拉格朗日量为 
\begin{equation}
L=\frac{1}{2}m\dot{x}^2-\frac{1}{2}kx^2+xF(t)
\end{equation}
其运动方程为$m\ddot{x}+kx=F(t)$,或写作 $\ddot{x}+\omega^2 x=\frac{F(t)}{m}$.
当 $F(t)=f\cos(\gamma t+\beta)$ 时,解为(齐次方程的通解+特解)
\begin{equation}
x(t)=a\cos(\omega t+\alpha)+\frac{f\cos(\gamma t+\beta)}{m(\omega^2-\gamma^2)}
\end{equation}
特别地,当 $\omega=\gamma$,上式不再成立,解为
\begin{equation}
x(t)=a \cos(\omega t+\alpha)+\frac{ft \sin(\omega t+\beta)}{2m\omega}
特解部分的振幅随时间线性增长.这就是共振现象.
\end{equation}