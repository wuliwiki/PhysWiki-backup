% 巨正则系综(综述)
% license CCBYSA3
% type Wiki

本文根据 CC-BY-SA 协议转载翻译自维基百科\href{https://en.wikipedia.org/wiki/Grand_canonical_ensemble}{相关文章}。

在统计力学中,巨正则系综(grand canonical ensemble,亦称宏观正则系综)是一种用于描述与热库处于热力学平衡(热平衡与化学平衡)的粒子力学系统可能状态的统计系综\(^\text{[1]}\)。该系综中的系统具有开放性特征——系统可与热库交换能量和粒子,因此系统的不同微观态可能具有不同的总能量与总粒子数。系统的体积、形状及其他外部坐标在所有可能微观态中保持恒定。

巨正则系综的热力学变量为化学势(符号:\(\mu\))与绝对温度(符号:\(T\))。该系综的特性还受体积(符号:\(V\))等力学变量的影响,这些变量决定了系统内部状态的性质。因此,该系综常被称为\(\mu VT\)系综,因为这三个物理量均为该系综的固定参数。
\subsection{基础概念}
在统计力学中,巨正则系综的基础概念可表述如下:该系综对每个微观态赋予的概率\(P\)由以下指数形式给出:
\[
P = e^{(\Omega + \mu N - E)/(kT)},~
\]
式中,\(N\)表示微观态中的粒子数,\(E\)为微观态的总能量,\(k\)为玻尔兹曼常数。  

物理量Ω称为巨势,对特定系综而言为常数。当选择不同的\(\mu,V,T\)参数时,概率分布和\(\Omega\)值将发生改变。巨势Ω具有双重作用:其一作为概率分布的归一化因子(所有微观态的概率之和必须为1);其二可通过函数\(\Omega(\mu,V,T)\)直接计算许多重要的系综平均值。  

当系统允许多种粒子数变化时,概率表达式推广为:  
\[
P = e^{(\Omega + \mu_1N_1 + \mu_2N_2 + \ldots + \mu_sN_s - E)/(kT)},~
\]  
其中,\(\mu_1\)表示第一种粒子的化学势,\(N_1\)为对应微观态中该种粒子数,依此类推(\(s\)为不同粒子种类总数)。需注意粒子数的精确定义(参见下文关于粒子数守恒的注释)。  

部分学者将巨正则系综的分布称为广义玻尔兹曼分布\(^\text{[2]}\)。

巨正则系综特别适用于描述以下系统:系统的几何形状固定,但因与热库接触(例如导体中的电子对应电接地,空腔中的光子对应吸光表面),其能量和粒子数可发生显著涨落。该系综为严格推导无相互作用量子粒子系统的费米-狄拉克统计或玻色-爱因斯坦统计提供了自然框架(参见下文示例)。 

\textbf{公式注记} 

该概念的另一表述形式将概率写为:\(P = \frac{1}{\mathcal{Z}} e^{(\mu N - E)/(kT)},\)  
其中通过巨配分函数\(\mathcal{Z} = e^{-\Omega/(kT)}\)替代了巨势Ω。本文中以巨势为基础的公式,均可通过简单的数学变换,利用巨配分函数重新表述。  
