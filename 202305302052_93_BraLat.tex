% 布拉伐格子、基元与原胞
% 布拉伐格子、基元与原胞

晶体最主要的特征是具有周期性重复的规则结果,可以看成是由不同原子组成的最小单元以某种方式在空间周期性重复平移的结果。这里有两个主要方面:
一是重复单元,二是重复排列的方式。只要确定了这两点,我们就可以得到整个晶体。

\subsection{布拉伐格子 基元}
早在1866年,布拉伐首先提出用\textbf{晶格点阵}来表示晶体中原子周期排列的方式,用\textbf{基元}表示最小重复单元。晶格点阵简称为\textbf{晶格},又称为\textbf{布拉伐格子}。关于晶体、基元与晶格点阵的关系,我们可以这样理解:晶体 = 基元 + 晶格点阵。
% 插入图片
\begin{figure}[ht]
\centering
\includegraphics[width=11cm]{./figures/1cd2e9434a0b3394.png}
\caption{ 基元与布拉伐格子} \label{fig_BraLat_1}
\end{figure}

通过布拉伐格子与基元的简化,我们抓住了晶体结构中最主要的特征————即晶体中原子周期性的排列,体现了其所具有的平移对称性。注意,布拉伐格子是一个无限延展的理想点阵。因为严格来说,有限大小的晶格无法满足周期结构的平移对称性。因此,对于宏观晶体,我们引入周期性边界条件。同时,它忽略了实际晶体中的缺陷,如原子在其平衡位置的微小偏移等;\textbf{不是真实的原子},是抽象的\textbf{点},是基元所在的位置。

打个比方,见过超市里的盒装鸡蛋吗?一个个鸡蛋整齐规律地垒在凹槽里,相当于一个个“基元”处在规律的“布拉伐格子”上。

注意,布拉伐格子具有\textbf{平移前后晶格重叠}和\textbf{无限延展}的特点。那么如何描述一个个基元的位置,即布拉伐格点的位置?我们引入\textbf{格矢}:
\begin{equation}
\bvec R(n_1, n_2, n_3) = n_1 \bvec a_1 + n_2 \bvec a_2 + n_3 \bvec a_3,~ 
\end{equation}
其中$\bvec a_1,~\bvec a_2,~\bvec a_3$ 是\textbf{基矢},$a_1,~ a_2,~\ a_3$ 是\textbf{晶格常数}。
\subsection{原胞}
