% 拓扑空间(综述)
% license CCBYSA3
% type Wiki

本文根据 CC-BY-SA 协议转载翻译自维基百科\href{https://en.wikipedia.org/wiki/Topological_space}{相关文章}

在数学中,拓扑空间可以粗略地理解为一种定义了“接近性”但不一定能用数值距离来度量的几何空间。更具体地说,拓扑空间是一个集合,其元素称为点,并配备一种称为拓扑的附加结构。这个拓扑可以定义为:每个点都对应一组满足特定公理的邻域集合,这些公理用来形式化“接近性”的概念。拓扑有多种等价的定义方式,其中最常用的是通过开集来定义。

拓扑空间是最一般的数学空间形式,能够支持极限、连续性以及连通性等概念的定义\(^\text{[1][2]}\)。常见的拓扑空间类型包括欧几里得空间、度量空间以及流形。

尽管拓扑空间的概念非常抽象和宽泛,但它是数学中的一个基础性概念,几乎存在于现代数学的各个分支中。专门研究拓扑空间本身的学科被称为一般拓扑学,也叫点集拓扑学。
\subsection{历史}
大约在1735 年,莱昂哈德·欧拉发现了公式:$V - E + F = 2$其中 $V$ 表示顶点数,$E$ 表示边数,$F$ 表示面数,该公式刻画了凸多面体(以及平面图)的数量关系。这一公式的研究与推广,尤其是由柯西(Augustin-Louis Cauchy, 1789–1857)和吕利耶(Simon Antoine Jean L'Huilier, 1750–1840) 的深入研究,极大地推动了拓扑学的发展。在1827 年,卡尔·弗里德里希·高斯发表了《曲面的一般研究》。在第三节中,他以类似现代拓扑的方式定义了曲面:“如果从曲面上一点 A 向与 A 无穷接近的曲面点引出的所有直线的方向,相对于通过 A 的同一平面发生无穷小的偏转,那么该曲面在 A 点就被称为具有连续曲率。\(^\text{[3]}\)”

然而,“直到黎曼在19 世纪 50 年代初的工作之前,曲面一直是从局部角度(作为参数曲面)来研究的,拓扑问题从未被考虑过。”莫比乌斯和乔尔当似乎是最早意识到拓扑学的核心问题是寻找不变量(最好是数值不变量),以此来判断两个曲面是否等价,即判断它们是否同胚。\(^\text{[4]}\)

费利克斯·克莱因在其“埃尔兰根纲领”(Erlangen Program, 1872中明确提出:研究任意连续变换下的不变量是一种新的几何学,即拓扑学的雏形。“拓扑学”一词由约翰·本尼迪克特·李斯廷于1847 年首次提出,尽管他在几年前的通信中已经使用过该术语来代替当时常用的“位置分析”。昂利·庞加莱将这一学科扩展到任意维度的空间,并奠定了现代拓扑学的基础。他关于该主题的第一篇论文发表于1894 年。\(^\text{[5]}\)在1930 年代,詹姆斯·沃德尔·亚历山大二世和哈斯勒·惠特尼首次提出了这样的观点:曲面是一种局部类似欧几里得平面的拓扑空间。

拓扑空间的现代定义最早由费利克斯·豪斯多夫于 1914 年在其奠基性著作 《集合论的原理》中提出。度量空间则更早在1906 年由莫里斯·弗雷歇定义,但“度量空间”(德语:metrischer Raum)这一术语的普及归功于豪斯多夫\(^\text{[6][7]}\)。
\subsection{定义}
拓扑这一概念的实用性在于:这种数学结构有多种等价的定义,可以根据具体应用选择最合适的公理化形式。最常用的定义是基于开集的方式,但基于邻域的定义可能更直观,因此通常先给出这一形式。
\subsubsection{基于邻域的定义}
这种公理化方法最早由费利克斯·豪斯多夫提出。设 $X$ 是一个(可能为空的)集合,$X$ 的元素通常称为点,但实际上它们可以是任何数学对象。令 $\mathcal{N}$ 是一个函数,为 $X$ 中的每个点 $x$ 赋予一个非空的子集族$\mathcal{N}(x)$,这些子集都属于 $X$ 的子集。$\mathcal{N}(x)$ 中的元素称为点 $x$ 的邻域(neighbourhoods of $x$,或简称邻域)。如果以下公理成立,那么函数 $\mathcal{N}$ 被称为一个邻域拓扑,此时 $X$ 与 $\mathcal{N}$ 构成一个拓扑空间。

公理
\begin{enumerate}
\item 点属于其每个邻域,若 $N$ 是 $x$ 的一个邻域(即 $N \in \mathcal{N}(x)$),则:$x \in N$换句话说,集合 $X$ 中的每个点都属于它的每一个邻域。
\item 邻域的超集仍是邻域,若 $N \subseteq X$,且 $N$ 包含了 $x$ 的某个邻域,则:$N \in \mathcal{N}(x)$也就是说,点 $x$ 的邻域的任意超集仍然是 $x$ 的邻域。
\item 邻域的交集仍是邻域,若 $N_1, N_2 \in \mathcal{N}(x)$,则:$N_1 \cap N_2 \in \mathcal{N}(x)$
\item 局部一致性,对于点 $x$ 的任意邻域 $N$,总存在另一个邻域 $M \subseteq N$ 使得:$\forall y \in M, \; N \in \mathcal{N}(y)$换句话说,点 $x$ 的任意邻域中都包含一个更小的邻域,这个更小的邻域中的每个点也都“共享”原来的邻域。
\end{enumerate}
邻域的前三条公理有着直观的含义,而第四条公理在理论结构中具有非常重要的作用,它用于将空间$X$中不同点的邻域联系起来。

一个标准的邻域系统示例是实数轴 $\mathbb{R}$:当 $\mathbb{R}$ 的一个子集$N$包含某个实数$x$附近的一个开区间时,就称 $N$ 是 $x$ 的一个邻域。

在这种结构下,如果集合 $U \subseteq X$ 对于其中的每个点都是该点的邻域,则称集合 $U$ 是开集。这些开集满足下一个拓扑空间定义中的开集公理。反过来,当我们已知一个拓扑空间的开集后,也可以重新得到满足上述邻域公理的邻域系统:如果集合$N$包含一个开集$U$且 $x \in U$,则称 $N$ 是点 $x$ 的邻域\(^\text{[9]}\)。
\subsubsection{通过开集定义}
一个集合$X$上的拓扑可以定义为一个子集族 $\tau$,称为开集,并且满足以下公理\(^\text{[10]}\):
\begin{enumerate}
\item 空集与全集.空集 $\varnothing$ 和全集 $X$ 都属于 $\tau$。
\item 任意并集封闭.$\tau$ 中任意子集族(无论有限或无限)的并集仍然属于 $\tau$。
\item 有限交集封闭.$\tau$ 中任意有限多个子集的交集仍然属于 $\tau$。
\end{enumerate}
由于这种定义是最常用的,因此集合$\tau$也通常被称为集合 $X$ 上的拓扑。

一个子集$C \subseteq X$称为在$(X, \tau)$中闭集,当且仅当它的补集$X \setminus C$是开集。由此可以推出:空集 $\varnothing$ 和全集$X$既是开集又是闭集,因为它们互为补集,并且每个本身都是开集。一般地,任何在$X$中同时是开集又是闭集的子集,称为闭开集。

\textbf{拓扑的示例}
\begin{figure}[ht]
\centering
\includegraphics[width=6cm]{./figures/a3edbfe6b3752ab3.png}
\caption{令 $\tau$ 用圆圈表示,下面展示了在三元素集合$\{1, 2, 3\}$上的四个拓扑示例和两个非拓扑示例。左下角的示例不是拓扑,因为集合 $\{2\}$ 与 $\{3\}$ 的并集(即 $\{2, 3\}$)缺失。右下角的示例也不是拓扑,因为集合 $\{1, 2\}$与$\{2, 3\}$的交集(即 $\{2\}$)缺失。} \label{fig_TPKJ_1}
\end{figure}
\begin{enumerate}
\item 设 $X = \{1, 2, 3, 4\}$:平凡拓扑 / 不分化拓扑.族$\tau = \{\{\}, \{1, 2, 3, 4\}\} = \{\varnothing, X\}$仅包含拓扑公理要求的两个子集,即空集与全集,这样的 $\tau$ 构成了 $X$ 上的一个拓扑。
\item 另一种拓扑.族$\tau = \{\varnothing, \{2\}, \{1, 2\}, \{2, 3\}, \{1, 2, 3\}, X\}$包含 6 个子集,这同样构成了 $X$ 上的另一个拓扑。
\item 离散拓扑.若取 $X$ 的幂集$\tau = \wp(X)$即所有可能的子集,则得到的是离散拓扑,此时的拓扑空间 $(X, \tau)$ 被称为离散空间。
\item 一个非拓扑的例子.设 $X = \mathbb{Z}$(整数集),令 $\tau$ 为“所有整数的有限子集,加上 $\mathbb{Z}$ 本身”。这个集合族不是一个拓扑,因为例如“所有不包含 0 的有限子集的并集”不是有限集,因此不在该集合族中;而这个并集也不是 $\mathbb{Z}$ 本身,所以它也不满足拓扑公理要求。
\end{enumerate}
\subsection{通过闭集定义}
利用德摩根律,前面对开集的公理可以转化为闭集的等价公理:
\begin{enumerate}
\item 空集 $\varnothing$ 和全集$X$都是闭集。
\item 任意族闭集的交集仍然是闭集。
\item 有限多个闭集的并集仍然是闭集。
\end{enumerate}
根据这些公理,可以从另一个角度定义拓扑空间:拓扑空间是一个集合$X$,以及一个闭子集族 $\tau$。此时,$\tau$中的集合就是闭集,而它们在$X$中的补集就是开集。
\subsubsection{其他定义}
拓扑空间还有许多等价的定义方式。换句话说,邻域、开集或闭集的概念都可以从其他出发点重新构造,并满足相应的公理。

另一种定义拓扑空间的方法是利用Kuratowski闭包公理,它将闭集定义为集合 $X$ 幂集上的一个算子的不动点。

此外,网的概念是序列的推广。若对集合 $X$ 中的每一个网,其所有聚点都被指定出来,那么该拓扑也就被完全确定了。
\subsection{拓扑的比较}
在一个集合上可以定义许多不同的拓扑,从而形成不同的拓扑空间。当一个拓扑 $\tau_1$ 的每一个开集在另一个拓扑 $\tau_2$ 中也都是开集时,就称$\tau_2$ 比 $\tau_1$ 更细,相应地,$\tau_1$ 比 $\tau_2$ 更粗。如果一个证明仅依赖于某些集合是开集,那么它对任何更细的拓扑都成立;同样地,如果一个证明依赖于某些集合不是开集,那么它对任何更粗的拓扑也都适用。有时,人们用“更大”和“更小”来替代“更细”和“更粗”。也有文献中使用“更强”和“更弱”这样的术语,但其具体含义并不统一,因此阅读时需要特别注意作者的约定。

在一个给定集合$X$上,所有可能的拓扑的集合构成一个完备格:若$F = \{\tau_\alpha : \alpha \in A\}$是 $X$ 上的一族拓扑,则:交是$F$中所有拓扑的交集;并是所有包含$F$中每个拓扑的拓扑的交集。
\subsection{连续函数}
设$f: X \to Y$是两个拓扑空间之间的一个函数,如果对于每个 $x \in X$ 以及 $f(x)$ 的每一个邻域 $N$,都存在 $x$ 的一个邻域 $M$,使得:$f(M) \subseteq N$,那么$f$ 称为连续函数。这个定义与分析学中常见的连续性概念密切相关。等价地,函数$f$连续当且仅当每个开集的逆像仍然是开集【11】。这样的定义试图刻画函数“没有跳跃或断裂”的直观概念。一个同胚是指一个双射函数,它本身连续且其逆函数也是连续的。如果两个空间之间存在同胚映射,则称这两个空间是同胚的。从拓扑学的角度来看,同胚空间本质上是完全相同的【12】。

在范畴论中,一个基本的范畴是Top,表示拓扑空间的范畴,其中对象是拓扑空间,态射是连续函数。尝试通过不变量对该范畴的对象进行分类,推动了多个研究领域的发展,例如同伦理论、同调理论以及K-理论。
