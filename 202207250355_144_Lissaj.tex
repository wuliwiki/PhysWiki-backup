% 利萨茹曲线
% 简谐振子|示波器|周期

\pentry{简谐振子\upref{SHO}}

\footnote{参考 Wikipedia \href{https://en.wikipedia.org/wiki/Lissajous_curve}{相关页面}.}

\begin{figure}[ht]
\centering
\includegraphics[width=11cm]{./figures/Lissaj_1.pdf}
\caption{利萨茹曲线 $A = B = a = b = 1$, $\phi = \pi n/6$($n = 0,\dots,6$)}
\end{figure}
\begin{figure}[ht]
\centering
\includegraphics[width=5cm]{./figures/Lissaj_3.png}
\caption{请添加图片描述} \label{Lissaj_fig3}
\end{figure}
\textbf{利萨茹曲线(Lissajous curve)}是平面上一点 $P$ 在两个垂直的方向 $x, y$ 分别做相简谐运动形成的轨迹, 常见于示波器上. 莉萨如曲线可以用以下参数方程表示
\begin{equation}\label{Lissaj_eq1}
\leftgroup{
x &= A\sin(a t + \phi)\\
y &= B\sin(b t)
}\end{equation}
通常来说, 我们讨论的是闭合的利萨茹曲线, 此时要求频率之比 $a/b$ 是一个有理数, 通常是两个较小的整数之比. 曲线的周期为两个方向周期 $2\pi/a$ 和 $2\pi/b$ 的最小公倍数. 式中 $\phi$ 被称为\textbf{相位差}, 顾名思义就是两个简谐振动的相位之差. 当曲线闭合时, 把\autoref{Lissaj_eq1} 中的 $\sin$ 都换成 $\cos$ 曲线形状也不变, 因为这相当于给 $x, y$ 同时加上 $\pi/2$ 相位. 给 $x, y$ 同时加上任意相同的相位都不会改变曲线形状.

下文中,我们假设 $\phi \in (-\pi, \pi]$. 不难证明, 给 $\phi$ 取相反数同样不改变曲线的形状, 唯一的区别只是点 $P$ 的运动方向相反.

\subsection{频率相同}
频率相同时,两个方向上的频率之比固定为1,故利萨茹曲线必然闭合.
当 $\phi = 0$ 时,曲线是延 $y=(B/A)x$ 的线段; $\phi = \pi$ 时,是延 $y=-(B/A)x$ 的线段; 当 $\phi = \pm\pi/2$ 时,轨迹是一个椭圆, 椭圆的长短两个轴与 $x,y$ 坐标轴平行. 特殊地, 如果 $\phi = \pm\pi/2$ 且 $A = B$ 则轨迹是一个圆.

当 $\phi$ 是其他值时, 曲线是一个斜置的椭圆. 当 $A = B$ 时, 椭圆长轴与 $x$ 轴的夹角为 $\pm\pi/4$.
\subsubsection{证明}
频率相同,即 $a=b$ 时,\autoref{Lissaj_eq1} 可以化为
\begin{equation}\label{Lissaj_eq2}
\frac xA-\frac yB\cos\phi-\frac{\sqrt{B^2-y^2}}{B}\sin\phi=0
\end{equation}
当 $\phi=0$ 时,\autoref{Lissaj_eq2} 化为 $x/A-y/B=0$,即 $y=(B/A)x$,表示一条斜率为 $B/A$ 的直线.当 $\phi=\pi$ 时,\autoref{Lissaj_eq2} 化为 $y=-(B/A)x$ ,表示斜率为 $-B/A$ 的直线.

当 $\phi=\pm\pi/2$ 时,\autoref{Lissaj_eq2} 可以化为 $x/A=\pm\sqrt{B^2-y^2}/B$,两边平方即可化为\autoref{Elips3_eq3}~\upref{Elips3},表示一个长短轴分别沿 $x$、$y$ 轴的椭圆.又若 $A=B$ ,则表示为一个圆.

当 $A=B$ 时,\autoref{Lissaj_eq2} 可以化为
\begin{equation}\label{Lissaj_eq3}
x^2+y^2-2xy\cos\phi-A^2\sin^2\phi
\end{equation}
设由\autoref{Elips3_eq3}~\upref{Elips3}所表示的椭圆旋转 $\theta$ 角度后得到的椭圆方程为
\begin{equation}
\frac{(x\cos\theta-y\sin\theta)^2}{a^2}+\frac{(x\sin\theta+y\cos\theta)^2}{b^2}=1
\end{equation}
将其展开为一般形式
\begin{equation}\label{Lissaj_eq4}
(a^2\sin^2\theta+b^2\cos^2\theta)x^2 + (a^2\cos^2\theta+b^2\sin^2\theta)y^2+2(a^2-b^2)(
\sin\theta\cos\theta)xy-a^2b^2=0
\end{equation}
对比\autoref{Lissaj_eq3} 和\autoref{Lissaj_eq4} 各项系数,可得
\begin{equation}\label{Lissaj_eq5}
\leftgroup{
& a^2\sin^2\theta+b^2\cos^2\theta = 1\\
& a^2\cos^2\theta+b^2\sin^2\theta = 1\\
& 2(a^2-b^2)\sin\theta\cos\theta = -2\cos\phi\\
& a^2b^2=A^2\sin^2\phi
}\end{equation}
将\autoref{Lissaj_eq5} 前两式相减,可得 $a^2\cos(2\theta)=b^2\cos(2\theta)$ ,这说明若 $\cos(2\theta) \neq 0$ ,则 $a^2 = b^2$.可由第三式,$\phi$ 取其他值时有 $a^2 \neq b^2$ ,所以必然有 $\cos(2\theta) = 0$,即 $\theta = \pm\pi/4$.证毕.

一般而言, 我们可以令 $t = 0$, 那么曲线与 $x$ 轴的截线长度为 $A\sin(\phi)$, 它与曲线的 $x$ 坐标最大值 $A$ 之比为 $\sin\phi$, 由此可判断相位差.

\begin{figure}[ht]
\centering
\includegraphics[width=11cm]{./figures/Lissaj_1.pdf}
\caption{利萨茹曲线 $A = B = a = b = 1$, $\phi = \pi n/6$($n = 0,\dots,6$)} \label{Lissaj_fig1}
\end{figure}

画图代码如下, 读者可以试着使用参数运行.
\begin{lstlisting}[language=matlab, caption=lissajous.m]
% 画利萨茹曲线
A = 1; B = 1; % 振幅
a = 1; b = 1; % 频率
ph = linspace(0, pi, 7);
t = linspace(0, 2*pi, 1000);
figure; axis equal; hold on; grid on;
xlabel x; ylabel y;
for i = 1:numel(ph)
    x = A*sin(a*t+ph(i));
    y = B*sin(b*t);
    plot(x, y);
end
\end{lstlisting}

\subsection{频率不同}
对于频率不同的情况,若频率之比为无理数,则两个方向的周期之比也为无理数,因此不存在这两个周期的最小公倍数(或最小公倍数为无穷),此时周期为无穷大,利萨茹曲线永不闭合.此时,随着时间的增加,曲线将会趋于遍布整个定义域 $-A\leqslant x \leqslant A$,$-B \leqslant y \leqslant B$ 内所有的点(或称曲线在该区域内稠密).若频率之比为有理数,则周期之比也为有理数,此时一定存在二者的最小公倍数,即存在一个有限的周期,利萨茹曲线闭合.

对于曲线闭合的情况,可以从一个完整周期内的曲线图样推知频率之比.因为在一个完整周期内,某个方向上达到极大值的次数与该方向上的频率呈正比.于是,只要数出一个周期内两个方向的峰值各出现的次数,他们的比值即为两个方向上的频率之比.
\begin{figure}[ht]
\centering
\includegraphics[width=14.25cm]{./figures/Lissaj_2.pdf}
\caption{利萨茹曲线(左:$A=B=1$,$a=2$,$b=3$,$\phi=0$;右:$A=B=1$,$a=\sqrt 2$,$b=\sqrt 3$,$\phi=0$)} \label{Lissaj_fig2}
\end{figure}
