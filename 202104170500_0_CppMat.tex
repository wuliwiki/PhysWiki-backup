% C++ 中矩阵类的实现方式

\pentry{C++ 中类的定义和继承}

容器指的就是像 \verb|std::vector| 这样的可以储存多个元素的对象, 下面介绍的矢量,矩阵,高维数组等都是容器.

C/C++ 的数组是储存于内存的 stack (栈)中的, 而 stack 一般只有几个 Mb, 一旦数组太大就会产生 stack overflow 错误. 第二是 stack 中的数据的大小都只能在编译时确定(即 constant expression, 例如 literal 或者宏定义), 所以我们不能在运行是确定数组的长度(例如从文件中读取, 或通过运行时的计算得到).
\begin{lstlisting}[language=cpp]
// 用 C/C++ array 的嵌套定义矩阵
#define N1 5
#define N2 5
double a[N1][N2]; // 定义并在栈中分配内存
// 全部元素赋值为 0
for (long i = 0; i < N1; ++i)
    for (long j = 0; j < N2; ++j)
        a[i][j] = 0;
\end{lstlisting}

所以在一般来说容器都会 allocate (分配)到内存的 heap (堆)中. 在 C++ 中, 在 heap 中分配内存用 \verb|new|, 释放则用 \verb|delete|.

\verb|std::vector| 可以说是使用最广的矢量容器, 理论上我们可以用 \verb|vector| 的 \verb|vector| 定义一个 heap 中的矩阵.
\begin{lstlisting}[language=cpp]
// 用 std::vector 的嵌套定义矩阵
vector<vector<double>> a;
// 分配内存
long N1 = 10, N2 = 10;
a.resize(N1);
for (long i = 0; i < N1; ++i)
    a[i].resize(N2);
// 全部元素赋值为 0
for (long i = 0; i < N1; ++i)
    for (long j = 0; j < N2; ++j)
        a[i][j] = 0;
\end{lstlisting}
注意由于用了多次 \verb|new|, 数据在内存中不一定是连续的, 甚至不一定按 \verb|i| 的顺序排列. 这样不是很好, 为了性能和语法上的考虑, 一般需要给每个容器分配一块连续的内存(例如我们会想要用单个连续索引来获得矩阵元).

最后一种方法就是本书 SLISC 库\upref{SliMat}中的底层实现方法, 即在堆中分配一段连续的内存, 并把双索引转换乘单索引来获取矩阵元.
\begin{lstlisting}[language=cpp]
Long N1 = 10, N2 = 10;
double *a;
new double a[N1*N2];
// 全部元素赋值为 0
for (long i = 0; i < N1; ++i)
    for (long j = 0; j < N2; ++j)
        a[N2*i + j] = 0; // 行主序
        // a[i + N1*j] = 0; // 列主序
// 或者直接使用单索引赋值更快
for (long i = 0; i < N1*N2; ++i)
    a[i] = 1;
delete a[];
\end{lstlisting}
前面两个例子中双索引都是行主序\upref{MatSto}的, 即第二个索引增加 1, 内存地址也增加 1. 而这个例子中我们既可以使用行主序也可以使用列主序(第一个索引增加 1, 内存地址增加 1), 还可以使用单索引. 可见这种方式要灵活得多.

注意这只是一个底层的原理, 我们需要把这个功能封装到一个矩阵类 \verb|Mat| 中, 在使用的时候可以达到例如以下效果
\begin{lstlisting}[language=cpp]
Mat a(10, 10); // 类的 constructor 会执行 new
// 全部元素赋值为 0
for (long i = 0; i < a.n1(); ++i) // n1() n2() 用于获取行数列数
    for (long j = 0; j < a.n2(); ++j)
        a(i, j) = 0;
// 使用单索引
for (long i = 0; i < a.size(); ++i) // size() 用于获取总长度
    a[i] = 1;

// 改变尺寸, 先用 delete 再用 new 重新分配, 数据丢失
a.resize(20, 20);
\end{lstlisting}
为了区分行主序和列主序, 我们可以设计两个类似的矩阵类 \verb|Cmat| 是列主序矩阵, \verb|Mat| 是行主序矩阵. 我们来看一个简单的 \verb|Cmat| 定义
