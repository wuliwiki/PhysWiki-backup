% 模
% keys 代数|环|线性空间|向量|module

\pentry{环\upref{Ring},矢量空间\upref{LSpace}}

模是对线性空间的一种推广,相当于要求进行数乘时所用的“数”不再是一个域的元素,而只要求是一个环.也就是说,线性空间是模的一种,但模不一定是线性空间.

\begin{definition}{左模}
给定一个环$(R, \times)$和一个\texbf{阿贝尔}群$(M, +)$,将$R$中的每个元素都定义为$M$上的一个变换,其中对于$r\in R, m\in M$,记$rm\in M$为$r$对$m$进行变换的结果.

如果所定义的变换满足:
\begin{enumerate}
\item 对于任意$r_i\in R, m\in M$,有:$r_2(r_1m)=(r_2\times r_1)m$;
\item 对于任意$r_i\in R, m\in M$,有:$r_2m+r_1m=(r_2+r_1)m$;
\item 对于任意$r\in R, m_i\in M$,有:$r(m_1+m_2)=rm_1+rm_2$;
\item 对于$R$的乘法单位元$1_R$和任意$m\in M$,有:$1_Rm=m$.
\end{enumerate}

那么我们说环$R$、群$M$配合给定的变换定义,构成一个\textbf{左}$R$\textbf{-模(left} $R$\textbf{-modeule)}.
\end{definition}

类似地,也可以定义$r_i$从右边作用于$m_i$,所得的结构就是\textbf{右}$R$\textbf{-模(right} $R$\textbf{-modeule)}.

线性空间的数乘并没有左右的区分,而模的“数乘”,即以上定义的变换,是有的.这是因为域的乘法必然是交换的,而环的则不一定,导致$r_1mr_2$的定义不明确.但是如果$R$是交换环,那么我们就可以良好地定义$r_1mr_2=(r_1\times r_2)m=(r_2\times r_1)m=m(r_1\times r_2)=m(r_2\times r_1)$.对于交换环$R$的模,左模和右模是一样的,统称为$R$-模.

\begin{definition}{线性空间}
域上的模,称为\textbf{线性空间(linear space)}.
\end{definition}

我们在线性代数中已经熟悉了线性空间的概念.一般的模有很多和线性空间相似的性质,但是模不一定能找到









