% 流体力学守恒方程
% keys 流体力学|守恒方程

\pentry{流体运动的描述方法\upref{fluid1},重积分、面积分、体积分(简明微积分)\upref{IntN}}

人们一般用密度 $\rho$,速度 $\bvec v$ 等物理量描述流体中每个位置流体微元的性质,并列出所谓的流体力学方程来描述每个流体微元随时间的演化.它们之间应当满足一定的守恒方程,例如\textbf{连续性方程,动量守恒方程,能量守恒方程}等.在这一节中,我们将对流体力学的守恒方程进行简要的推导和介绍.
\subsection{连续性方程(流守恒方程)}
让我们对物质体(Material Volumn)进行分析,取一个随流体一起运动的被闭曲面 $S$ 包围的体积 $V$,对其中的密度进行积分得到这一个物质体的质量.物质体在随流体一起运动的过程中,其质量应当是不变的,那么我们有
\begin{equation}
\frac{\dd}{\dd t}\int \rho \dd V=0
\end{equation}
进一步将它拆成两部分,一部分是每个位置密度的偏导数随时间的变化,另一部分是闭合曲面 $S$ 的变化带来的质量变化.我们得到
\begin{equation}
\int \pdv{\rho}{t}\
\end{equation}
