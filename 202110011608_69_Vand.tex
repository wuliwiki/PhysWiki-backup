% 范德瓦尔斯气体
% keys 范式方程|范德瓦尔斯气体

范德瓦尔斯方程是理想气体向真实气体的推广,架起了微观图像与宏观测量之间的桥梁.

范德瓦尔斯对理想气体作了两点修正:1、真实气体占据一定体积;2、真实气体间有分子间作用势(Lennard-Jones 势是一个很好的近似).范德瓦尔斯方程展现出惊人的威力——从它的图像上可以看出气液相变线,可以找到临界点…… 1910年诺贝尔物理学奖授予范德瓦尔斯,以表彰他为气体和液体状态方程所作的工作.

\subsection{范式方程}

\begin{equation}
\left(p+\frac{a}{V_m^2}\right)(V_m-b)=RT
\end{equation}

$b$ 是因为真实气体分子总占据一定体积而做的修正.$a$ 是考虑分子间作用力(主要是吸引力)而做的修正.

设气体分子的有效直径为 $d$,分子原本能达到的空间体积为 $V_0=\frac{1}{6}\pi d^3$,当考虑它与另一分子的碰撞时,它所能达到的空间体积减少了 $\frac{4}{3}\pi d^3$.$1\rm mol$ 气体含有 $N_A$ 个气体分子,从一个粒子的角度看,它面对 $N_A-1$ 个排斥球,而每个排斥球只有一面可能对它产生排斥,体积只能算一半:
\begin{equation}
b=\frac{1}{2}(N_A-1)(\frac{4}{3}\pi d^3)=4N_A V_0
\end{equation}

由于分子间作用力(主要是吸引力),碰撞容受到朝向容器内的吸引力而动量减小:所以要引入内压强 $\Delta p$.内压力正比于单位时间内碰撞器壁的粒子数,又正比于粒子数密度(影响吸引力的大小),所以 $\Delta p$ 正比于 $\frac{1}{V_m^2}$,所以设这个修正量为 $a/V_m^2$,$a$ 与相互作用势有关.器壁的分子设分子间作用力在 $r>d$ 时为林纳德琼斯势,当 $r\le d$ 时为钢球势(势能趋向于无穷大).当 $r>d$ 时,有 $\phi(r)=-\epsilon_0(d/r)^6$,这样可以

