% 图的顶点度
% keys 顶点度
% license Usr
% type Tutor

\pentry{图\nref{nod_Graph}}{nod_2e68}

图的某一顶点的度是指与它关联的边数。由于图论的早期主要研究的是无向图,因此顶点度的概念往往专门用于无向图。然而,为了一般化起见我们将它定义在任一图上。

\begin{definition}{顶点度}
设 $G=(V,E,\varphi)$ 是图,$v\in V$。则 $G$ 中与 $v$ \aref{关联}{def_Graph_1}的边数称为 $v$ 在 $G$ 中的\textbf{度}(degree),记作 $d_G(v)$。$G$ 中以 $v$ 为起点的有向边数称为 $v$ 在 $G$ 中的\textbf{出度}(out degree),记作 $d_G^+(v)$。$G$ 中以 $d$ 为终点的有向边数称为 $v$ 在 $G$ 中的\textbf{入度}(in degree),记作 $d_G^-(v)$。
\end{definition}

\begin{exercise}{}
设 $G$ 是有向图,试证明 $d_G(v)=d_G^+(v)+d_G^-(v)-r$,其中 $r$ 是过 $v$ 的环数。

\textbf{证明:}设 $E^+$ 是 $G$ 中以 $v$ 为起点的边集,$E^-$ 是 $G$ 中以 $v$ 为终点的边集。则 $E^+\cap E^-$ 是过 $v$ 的环集。由于
\begin{equation}
E(v)=
\end{equation}

\end{exercise}







