% 开普勒问题数值模拟(Matlab)

\begin{issues}
\issueDraft
\issueOther{应该计算含时运动方程}
\issueOther{推导并引用相关公式}
\end{issues}

\pentry{Matlab 画图\upref{MatPlt}, 开普勒问题\upref{CelBd}, 圆锥曲线的极坐标方程\upref{Cone}}

以下给出计算开普勒问题中轨道的代码.
\begin{figure}[ht]
\centering
\includegraphics[width=11cm]{./figures/KepNum_1.pdf}
\caption{在同一位置同一速度以不同仰角发射的轨道, 虚线为参考圆(例如地球表面).} \label{KepNum_fig1}
\end{figure}

所有符号沿用 “开普勒问题\upref{CelBd}”: 若 $F(r)$ 是平方反比的力(斥力为正引力为负), 即
\begin{equation}
F(r) = \frac{k}{r^2}
\end{equation}
其中 $k$ 为非零实数. $k>0$ 表示斥力, $k<0$ 表示引力. 由于质点的运动与质点的质量无关, 所以我们令 $m=1$. 令发射位置用极坐标表示为 $(R,\alpha)$, 初速度为 $v_0$, 发射仰角为 $\beta$ (顺时针为正, $\beta=0$ 时速度延逆时针方向).

所以质点机械能为动能加势能 $E = v_0^2/2 + k/R$, 角动量(逆时针为正)为 $L = v_0 R\cos\beta$. 于是离心率 $e$ 可由\autoref{CelBd_eq2}~\upref{CelBd} 计算, 半通径 $p$ 由\autoref{CelBd_eq3}~\upref{CelBd}计算. 然后代入圆锥曲线的极坐标方程(\autoref{Cone_eq5}~\upref{Cone}). 即可. 然而由于发射点已经选好, 实际的轨道还需要逆时针旋转一个角度 $\theta_0$, 即
\begin{equation}
r(\theta)  = \frac{p}{1 - e\cos (\theta-\theta_0)}
\end{equation}
要求 $\theta_0$, 可以代入初始条件, 即 $r(\alpha) = R$. 得
\begin{equation}
\theta_0 = \alpha \pm \arccos(\frac{1-p/R}{e})
\end{equation}
这里有两个解是因为圆锥曲线和半径为 $R$ 的圆必定有两个交点. 我们需要根据发射角度的 $\cos\beta$ 值的正负号来判断. 

\begin{lstlisting}[language=matlab]
% 已知初始位置、发射速度、发射方向, 求轨道以及运动方程
function kepler
% === params ===
k = -1; % -GMm, (令 m=1)
R = 1; alpha = 0; % 初始位置(极坐标)
v0 = 1.1; % 初速度(支持行矢量)
beta = linspace(0, 4*pi/11, 6); % 发射仰角(支持行矢量)
% ==============

if isscalar(v0)
    v0 = repmat(v0, size(beta));
elseif isscalar(beta)
    beta = repmat(beta, size(v0));
elseif ~all(size(v0)==size(beta))
    error('v0 和 beta 的尺寸必须一样,或者其中之一为标量');
end

% 画参考圆
close all; figure;
tmp = linspace(0, 2*pi, 500);
plot(R*cos(tmp), R*sin(tmp), 'k--'); hold on;
scatter(0, 0); axis equal;
xlabel x; ylabel y;

for i = 1:6
    kepler1(k, R, alpha, v0(i), beta(i)); hold on;
end
end

function kepler1(k, R, alpha, v0, beta)
E = v0^2/2 + k/R; % 轨道总机械能
L = v0*cos(beta)*R; % 角动量
e = sqrt(1 + 2*E*L^2/k^2); % 离心率
p = L^2/abs(k); % 半通径

% 画图
dth = pi/20; Nth = 1000;
if e < 1
    th = linspace(0, 2*pi, Nth);
elseif e == 1
    th = linspace(dth, 2*pi-dth, Nth);
else
    th1 = atan(sqrt(e^2-1));
    th = linspace(th1+dth, 2*pi-th1-dth, Nth);
end

% 计算圆锥曲线需要旋转的角度 th0
if e == 0
    th0 = 0;
elseif cos(beta) >= 0
    th0 = alpha + acos((1-p/R)/e);
else
    th0 = alpha - acos((1-p/R)/e);
end
th0 = real(th0);
scatter(R*cos(alpha), R*sin(alpha));

th = th + th0;
r = p./(1 - e*cos(th-th0));
x = r .* cos(th); y = r .* sin(th);
plot(x, y); axis equal;
end
\end{lstlisting}
