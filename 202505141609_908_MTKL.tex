% 蒙特卡罗方法(综述)
% license CCBYSA3
% type Wiki

本文根据 CC-BY-SA 协议转载翻译自维基百科\href{https://en.wikipedia.org/wiki/Monte_Carlo_method}{相关文章}。

蒙特卡罗方法,又称蒙特卡罗实验,是一类依赖重复随机抽样以获得数值解的广义计算算法。其基本思想是利用随机性来解决一些本质上可能是确定性的问题。该名称源自摩纳哥的蒙特卡罗赌场,该方法的主要发展者、数学家斯坦尼斯瓦夫·乌拉姆受到他叔叔赌博习惯的启发而命名。

蒙特卡罗方法主要用于三类问题:优化问题,数值积分,从概率分布中抽样生成样本。此外,它还可以用于对输入存在高度不确定性的现象进行建模,例如评估核电站故障的风险。蒙特卡罗方法通常通过计算机模拟实现,能为那些无法解析求解或过于复杂的问题提供近似解。

蒙特卡罗方法广泛应用于多个科学与工程领域,包括:物理、化学、生物学、统计学、人工智能、金融、密码学,也被应用于社会科学,如:社会学、心理学与政治学。它被认为是20世纪最重要且最具影响力的思想之一,并推动了众多科学与技术突破的实现。

不过,蒙特卡罗方法也存在一些局限性和挑战,例如:精度与计算成本之间的权衡,维度灾难,随机数生成器的可靠性,结果的验证与确认。
\subsection{概述}
蒙特卡罗方法虽然形式多样,但通常遵循以下基本步骤:
\begin{enumerate}
\item 定义可能输入的范围(即问题的定义域);
\item 从该范围内的概率分布中随机生成输入样本;
\item 对每个输入进行确定性的计算,得到输出结果;
\item 汇总(聚合)所有结果,以得到最终的估计或解。
\end{enumerate}
\begin{figure}[ht]
\centering
\includegraphics[width=6cm]{./figures/db49926bb4b1422c.png}
\caption{蒙特卡罗方法用于近似计算\(\pi\)的值} \label{fig_MTKL_1}
\end{figure}
例如,考虑一个内接于单位正方形的象限圆扇形。由于它们面积的比值是 $\frac{\pi}{4}$,因此可以使用蒙特卡罗方法来近似计算 $\pi$ 的值:\(^\text{[1]}\)
\begin{enumerate}
\item 画一个正方形,并在其中内接一个象限圆(四分之一圆);
\item 在正方形内均匀随机撒布一定数量的点;
\item 统计落在象限圆内的点的数量(即其到原点的距离小于1的点);
\item 象限内点数与总点数之比是两者面积比 $\frac{\pi}{4}$ 的估计值。
将该比值乘以4,即可得到 $\pi$ 的估计值。
\end{enumerate}
在上述过程中,输入的定义域是包围该象限的正方形。可以通过在正方形内随机撒点(如撒沙粒)来生成输入,然后对每个输入执行一次计算,以判断它是否落在象限圆内。将结果汇总后,就可以得到最终的估计值——对 $\pi$ 的近似。

这里有两个重要注意事项:
\begin{enumerate}
\item 如果点不是均匀分布的,则估计结果将不准确;
\item 随机点的数量越多,估计的精度就越高。
\end{enumerate}
蒙特卡罗方法的应用需要大量的随机数,因此它的发展极大地受益于伪随机数生成器的出现,这些生成器比早期所用的随机数表要高效得多。
\subsection{应用}
蒙特卡罗方法常用于物理和数学问题,尤其在其他方法难以或无法使用时最为有用。蒙特卡罗方法主要应用于三类问题:\(^\text{[2]}\):优化问题,数值积分,从概率分布中生成样本(抽样)。

在与物理相关的问题中,蒙特卡罗方法非常适用于模拟具有多个耦合自由度的系统,例如流体、无序材料、强耦合固体以及细胞结构等(参见:细胞 Potts 模型、相互作用粒子系统、McKean–Vlasov 过程、气体动力学模型)。

其他应用示例包括:建模输入存在重大不确定性的现象,例如商业风险评估;在数学中,用于求解具有复杂边界条件的多维定积分;在系统工程问题中(如航天、石油勘探、飞机设计等),蒙特卡罗方法在预测失效概率、成本超支和进度延误方面,往往优于人类直觉或其他“软性”方法。\(^\text{[3]}\)

从原理上讲,任何具有概率解释的问题都可以使用蒙特卡罗方法求解。根据大数定律,由某随机变量期望值所定义的积分可以通过该变量的独立样本的经验平均值(即样本均值)来近似。当所研究的随机变量具有参数化的概率分布时,数学家通常会使用马尔可夫链蒙特卡罗方法(MCMC)。\(^\text{[4][5][6]}\) 其核心思想是构造一个具有指定平稳分布的合适马尔可夫链模型。也就是说,在极限下,由 MCMC 方法生成的样本将来自目标分布。\(^\text{[7][8]}\)根据遍历定理,该平稳分布可以通过 MCMC 采样器生成的随机状态的经验分布来近似。

在其他类型的问题中,目标是从满足非线性演化方程的一系列概率分布中生成样本。这些概率分布的演化流总可以解释为某个马尔可夫过程的随机状态分布,其转移概率依赖于当前状态的分布(参见:McKean–Vlasov 过程、非线性滤波方程)。\(^\text{[9][10]}\)在另外一些情形下,会出现采样复杂度不断上升的概率分布流,例如:随时间增长的路径空间模型,与降温参数相关的玻尔兹曼–吉布斯测度,以及其他类似模型。这些模型也可以看作是非线性马尔可夫链的随机状态分布的演化过程。\(^\text{[10][11]}\)一种模拟此类复杂非线性马尔可夫过程的自然方法是:对该过程生成多个独立副本,用样本的经验分布来替代演化方程中未知的状态分布。与传统蒙特卡罗或 MCMC 方法不同,这些被称为平均场粒子方法的方法依赖于一组连续相互作用的样本。术语“平均场”指的是,每一个样本(也称为粒子、个体、行者、代理、实体或表型)都与过程的经验分布发生相互作用。当系统规模趋于无穷大时,这些随机的经验测度将收敛到非线性马尔可夫链中随机状态的确定性分布,从而粒子之间的统计相互作用也将消失。
\subsection{简单蒙特卡罗方法}
假设我们希望知道某个总体的期望值 $\mu$(且已知该期望值存在),但无法通过公式直接计算。简单蒙特卡罗方法通过运行 $n$ 次模拟并对结果取平均,来对 $\mu$ 进行估计。

该方法对模拟输入的概率分布没有限制,只要求以下几点:输入是随机生成的,输入彼此相互独立,$\mu$存在。只要 $n$ 足够大,所得估计值 $m$ 就能**任意接近于**真实值 $\mu$。更正式地说,对于任意 $\epsilon > 0$,都可以有$|\mu - m| \leq \epsilon$\(^\text{[12]}\)

通常用于计算 $m$ 的算法如下:
\begin{figure}[ht]
\centering
\includegraphics[width=14.25cm]{./figures/99fbcdd21ea48ed3.png}
\caption{} \label{fig_MTKL_2}
\end{figure}
\subsubsection{一个示例}
假设我们想知道:掷三个八面骰子,总点数至少为 $T$,我们期望需要掷多少次才能满足这一条件。我们知道该期望值是存在的。骰子的每次投掷是随机的且相互独立的,因此可以应用简单蒙特卡罗方法:
\begin{figure}[ht]
\centering
\includegraphics[width=14.25cm]{./figures/679fe4eb3bb18f45.png}
\caption{} \label{fig_MTKL_3}
\end{figure}
如果 $n$ 足够大,则对于任意 $\epsilon > 0$,估计值 $m$ 将会落在真实值 $\mu$ 的 $\epsilon$ 范围内。
\subsection{确定足够大的 $n$}

\textbf{通用公式}

设$\epsilon = |\mu - m| > 0$,其中 $\mu$ 是真实期望值,$m$ 是蒙特卡罗估计值。首先,选择所需的置信水平——也就是希望在蒙特卡罗算法运行完成后,估计值$m$落在$\mu$的$\epsilon$ 范围内的概率。设对应该置信水平的标准正态$z$-分数为 $z$。

设$s^2$为估计方差(有时称为“样本方差”),它是从一个相对较小的样本数量 $k$ 次模拟中计算得到的结果的方差。选择一个$k$值;Driels 和 Shin 指出:“即使样本量比所需数量小一个数量级,这个估算结果通常也相当稳定。”\(^\text{[13]}\)

以下算法可以单遍计算 $s^2$,同时最小化累积数值误差导致的错误结果风险:\(^\text{[12]}\)
\begin{figure}[ht]
\centering
\includegraphics[width=14.25cm]{./figures/8dcfec3f3fc4e736.png}
\caption{} \label{fig_MTKL_4}
\end{figure}
请注意,当算法运行结束时,$m_k$是从 $k$ 次模拟中得到的平均值。

当满足以下条件时,$n$ 被认为是足够大的:
$$
n \geq \frac{s^2 z^2}{\epsilon^2},^\text{[12][13]}~
$$
如果$n \leq k$,那么$m_k = m$,说明已有的 $k$ 次模拟已足以保证 $m_k$ 在误差 $\epsilon$ 范围内逼近 $\mu$。如果$n > k$,那么可以选择:重新“从头开始”运行 $n$ 次模拟,或因为已完成了 $k$ 次模拟,只需再运行 $n - k$ 次模拟,并将它们的结果加入原来的样本统计中即可:
\begin{figure}[ht]
\centering
\includegraphics[width=14.25cm]{./figures/e52479c12430563f.png}
\caption{} \label{fig_MTKL_5}
\end{figure}
\textbf{当模拟结果有界时的估算公式}

在所有模拟结果都有**上下界**的特例中,可以使用另一种估算公式。

选择一个 $\epsilon$ 值,使其为 $\mu$ 与 $m$ 之间最大允许误差的两倍。设置信心水平 $\delta$,满足

$$
0 < \delta < 100,
$$

并以百分比表示。

假设每次模拟结果

$$
r_1, r_2, \ldots, r_i, \ldots, r_n
$$

都满足

$$
a \leq r_i \leq b,
$$

其中 $a$ 和 $b$ 是有限实数。

若希望以至少 $\delta\%$ 的置信度满足

$$
|\mu - m| < \frac{\epsilon}{2},
$$

则应选择满足以下条件的 $n$:

$$
n \geq \frac{2(b - a)^2 \ln\left(\frac{2}{1 - \delta/100}\right)}{\epsilon^2}
$$

例如,若 $\delta = 99\%$,则有:

$$
n \geq \frac{2(b - a)^2 \ln(2 / 0.01)}{\epsilon^2} \approx \frac{10.6(b - a)^2}{\epsilon^2}
$$

\[12]
