% 体积形式
% 微分几何|微分形式|体积|微积分|流形|calculus|manifold|differential form|differential geometry


\pentry{微分形式\upref{Forms}}

\addTODO{加入目录}
\addTODO{内容还可完善}



\subsection{基本概念和例子}


体积形式是微分流形上一种微分形式,可以用于计算切平面附近区域的大小,进而实现积分运算。

\begin{definition}{体积形式}
给定$n$维微分流形$M$,则其上一个处处非零的$n$-形式即为一个\textbf{体积形式(volume form)}。
\end{definition}

\begin{example}{二维欧几里得空间}\label{ex_VolFom_1}
设$M$是二维欧几里得空间$\mathbb{R}^2$,给定两个\textbf{坐标函数}以确定一个(局部)坐标系。设在这个坐标系中,欧几里得度量被表示为$g_{ab}$。

这里举一个坐标函数与坐标系的例子:给定一点$o$和从这一点出发的一根射线$R$,再规定一个逆时针方向。任取$p\in M$,定义第一个坐标函数$r(p)$为$p$到给定点的距离,第二个坐标函数$\theta(p)$为射线$R$沿着规定方向转到与射线$op$重合时所经过的弧度,取值范围为$[0, 2\pi)$。上述定义的坐标函数所得到的坐标系,就是我们熟知的极坐标系。在极坐标系中,欧几里得度量为
\begin{equation}
g_{ab}=\pmat{
    1&0\\
    0&r^2
}~.
\end{equation}


如果用以坐标函数$r, \theta$确定的余切向量$\dd r$和$\dd \theta$构成余切空间的基,那么切空间的对偶基由$\partial_r=\cos\theta\partial_x+\sin\theta\partial_y$和$\partial_\theta=\frac{-\sin\theta}{r}\partial_x+\frac{\cos\theta}{r}\partial_y$构成。



设坐标函数$x$和$y$诱导出的是我们熟知的\textbf{标准正交基},其中$x$轴的非负半轴与上述极坐标轴重合。





而$2$-形式$\dd r\wedge\dd \theta$对这两个切向量的作用为\footnote{推导留给读者,注意按定义,$\dd f(\partial_i)=\partial_i f$,其中$i\in\{x, y\}$,$f\in\{r, \theta\}$。余切向量的外积定义见\autoref{sub_ExtAlg_1}~\upref{ExtAlg}。}
\begin{equation}
\dd r\wedge\dd \theta(\partial_x, \partial_y) = \frac{1}{r}~,
\end{equation}

我们使用通常的面积(二维体积)的定义,即给定\textbf{无穷小量}$\varepsilon_x$和$\varepsilon_y$,截取切向量$\partial_x$的代表曲线上$t\in[0, \varepsilon_x)$上的一段和$\partial_y$上$t\in[0, \varepsilon_y)$上的一段,这两段近似张成的平行四边形面积应为
\begin{equation}\label{eq_VolFom_1}
\varepsilon_x\varepsilon_y = r\dd r\wedge\dd \theta(\varepsilon_x\partial_x, \varepsilon_y\partial_y)~.
\end{equation}


\autoref{eq_VolFom_1} 中的$r\dd r\wedge\dd \theta$即为体积形式,可以看出如何用它配合切向量算出体积。


\end{example}



\begin{example}{黎曼流形}
\autoref{ex_VolFom_1} 是黎曼流形的一个特例。在一般的$n$维黎曼流形上,任意给定一组坐标函数$\{x^\mu\mid \mu=1, 2, \cdots, n\}$,其对应的余切向量分别为$\{\dd x^\mu\}$,对应的切空间的基为$\{\partial_\mu\}$,度量张量为$g_{\mu\nu}$。

则体积形式为
\begin{equation}\label{eq_VolFom_2}
\sqrt{\abs{g}}\dd x^1\wedge \dd x^2\wedge \cdots \wedge \dd x^n~,
\end{equation}
其中$g=\det g_{\mu\nu}$。

在伪黎曼流形上\autoref{eq_VolFom_2} 同样成立。
\end{example}



不是所有流形都可以找到处处存在的体积形式,莫比乌斯带就是一个反例。体积形式的存在性依赖于流形的可定向性:


\begin{theorem}{}\label{the_VolFom_1}
当且仅当流形可定向时,流形上存在处处非零的体积形式。
\end{theorem}

\autoref{the_VolFom_1} 的证明思路是,体积形式在某一点处总能导出一个法向量,而某点处的法向量总能导出一个体积形式。



\subsection{与测度的关系}

\addTODO{需要流形上的微积分、测度作为预备。}

流形$M$上,给定的体积形式$\omega$能导出一个测度$\mu_{\omega}$:
\begin{equation}
\mu_\omega(U) = \int_U \omega~.
\end{equation}
















