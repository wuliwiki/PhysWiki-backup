% 一维散射态的归一化

\pentry{一维散射(量子)\upref{Sca1D}}
写作参考\href{https://chaoli.club/index.php/4541/last}{这篇帖子}. 类似于平面波, 一维散射态也有不同的归一化方式, 但情况要更为复杂. 

为了方便先假设 $V(x)$ 关于原点对称, 且 $V(x)$ 只在区间 $[-L,L]$ 内不为零. 由于 $V(x)$ 的对称性, 我们必定能找到实值的奇函数和偶函数两种解. 令 $k = \sqrt{2mE} > 0$, 在区间 $[-L,L]$ 外, 波函数就是正弦函数加上一个相移
\begin{equation}
\psi_k(x) \propto \sin(kx + \phi) \qquad (x > L)
\end{equation}
其中$\phi$ 是 $k$ 的函数, 称为\textbf{相移(phase shift)}. 为方便书写下文把 $\phi(k),\phi(k')$ 分别记为 $\phi, \phi'$.

令对称和反对称散射态分别为实函数 $\psi_{k,e}(x)$ 和 $\psi_{k,o}(x)$ 我们希望通过添加适当的归一化系数后, 波函数能满足归一化条件(\autoref{EngNor_eq3}~\upref{EngNor})% 链接未完成
\begin{equation}
\int_{0}^{+\infty} \psi_{k',i'}(x)^* \psi_{k,i}(x) \dd{x} = \delta_{i',i}\delta(k' - k) \qquad (k > 0, i = e, o)
\end{equation}
虽然笔者不会证明, 但大量实例表明无论 $V(x)$ 的具体形式, 该关系是成立的, 且归一化系数和简谐波(\autoref{EngNor_eq5}~\upref{EngNor})一样都是 $1/\sqrt{\pi}$.


首先已知
\begin{equation}
\int_{0}^{+\infty} \sin(k'x)\sin(kx)\dd{x} = \frac{\pi}{2}\delta(k'-k)
\end{equation}
现在添加相位 $\phi(k)$, 使用 Wolfram Alpha 得不定积分
\begin{equation}
\int \sin(k'x+\phi')\sin(kx+\phi) \dd{x} = \frac{\sin[(k'-k)x + (\phi'-\phi)]}{2(k'-k)}
- \frac{\sin[(k'+k)x+(\phi'+\phi)]}{2(k'+k)}
\end{equation}
在 $(0,n)$ 积分取极限 $n\to\infty$ 后发现添加相位后多出了两项
\begin{equation}
\int_{0}^{+\infty} \sin(k'x+\phi')\sin(kx+\phi) \dd{x} = \frac{\pi}{2}\delta(k'-k)
+ \frac{\sin(\phi'+\phi)}{2(k'+k)} - \frac{\sin(\phi'-\phi)}{2(k'-k)}
\end{equation}
如果 $[-L,L]$ 中有 $V(x) \ne 0$, 导致相移 $\phi(k)$ 需要添加该区间中的修正项
\begin{equation}
I(k,k') = \int_0^L \psi_{k'}(x)^* \psi_k(x) \dd{x}
-\frac{1}{\pi}\int_{0}^{L} \sin(k'x+\phi')\sin(kx+\phi) \dd{x}
\end{equation}
使内积为
\begin{equation}
\begin{aligned}
\braket{\psi_{k'}}{\psi_k} &= \frac{2}{\pi}\int_{0}^{+\infty} \sin(k'x+\phi')\sin(kx+\phi) \dd{x} + 2I(k,k')\\
&= \delta(k'-k) + \frac{\sin(\phi'+\phi)}{\pi(k'+k)} - \frac{\sin(\phi'-\phi)}{\pi(k'-k)} + 2I(k,k')
\end{aligned}
\end{equation}
如果能证明最后的三项抵消, 那么我们就证明了正交归一关系. 笔者不会证, 但相信这对任意偶函数 $V(x)$ 都成立. 一个具体的例子是方势垒\upref{SqrPot}.


\subsection{能量归一化}
================ 回收 ====================

想要能量归一化, 需要
\begin{equation}
\int_{-\infty}^{+\infty} \psi_{E'}^*(x) \psi_E(x) \dd{x}  = \delta (E - E')
\end{equation}
能不能在动量归一化的波函数基础上修改, 得到能量归一化的本征函数呢?
动量归一化的要求是
\begin{equation}
\int_{-\infty}^{+\infty} \psi_{k'}^*(x) \psi_k(x) \dd{x}  = \delta(k - k')
\end{equation}
满足
\begin{equation}\ali{
\int_{-\infty}^{+\infty} \psi_k^*(x) \psi(x) \dd{x}
= \int_{-\infty}^{+\infty} \psi_k^*(x) \qty(\int_{-\infty}^{+\infty} c(k')\psi_{k'}(x) \dd{k'}) \dd{x}
= c(k)
}\end{equation}
另外注意 $E = \hbar ^2 k^2/(2m)$, $E' = \hbar^2k'^2/(2m)$. 
\begin{equation}
\delta \qty[ \frac{\hbar ^2}{2m}(k^2 - k'^2)]
\end{equation}
根据 $\delta $ 函数的性质, 若 $x_0$ 是 $f(x)$ 的一个零点
\begin{equation}
\delta[f(x)] = \frac{1}{f'(x)}\delta (x - x_0)
\end{equation}
所以
\begin{equation}
\begin{aligned}
\delta (E - E') & = \delta \qty[\frac{\hbar^2}{2m}(k^2 - k'^2)] = \frac{m}{\hbar ^2 k}\delta (k - k') \\
&= \frac{m}{\hbar^2 k}\int_{-\infty }^{+\infty } \psi_k^*(x) \psi_k(x)\dd{x} 
\end{aligned}
\end{equation}
可得
\begin{equation}
\psi_E (x) = \frac{1}{\hbar} \sqrt{\frac{m}{k}} \psi_k(x)
\end{equation}


=============== 草稿 ================



\subsection{杂}
\begin{equation}
\int_0^\alpha\alpha \E^{\I kx} \dd{x} = \frac{1}{\I k} (\E^{\I k\alpha} - 1)
= \frac{2}{ k} \sin(k\alpha/2) \E^{\I k\alpha/2}
\end{equation}
这并不是一个 delta 函数. 因为最右边的 $\sin^2$ 会使虚部在不包含原点的有限区间积分在极限 $\alpha \to +\infty$ 下不为零.
