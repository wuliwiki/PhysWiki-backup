% 等效原理
% keys 等效原理
% license Usr
% type Tutor

等效原理是广义相对论的根基,它声称:在足够小的时空区域内,没有实验能够验证处于其中的物体是在引力场中还是加速参考系中。本节将通两个思维实验来理解Einstein是如何获得这一思想的。

\subsection{下落的物体}
让我们思考这样一个愚人节恶作剧:趁我们其中一个朋友睡着的时候,把他放在一个进行精心设计的盒子里,这个盒子装饰得和这个伙计睡觉的地方一模一样。然后我们从很高的飞机上丢下这个盒子。

当我们的朋友醒来的时候,他认为他正处于他的房间里。由于他和他周围的所有东西都以盒子同样的速率向下加速,相对于他的四周来说,他感觉不到他在向下坠落。他轻轻的一跳,他发现他飘向了天花板。然而对于外面的观测者来说:我们的朋友,只不过是通过踩在地板上,降低了他的下落速度,同时增加了盒子的下落速度。他认为他是飘向天花板的,然而实际上他的坠落正在和之前一样的速率向下加速。

事实上,这一可怕且不道德的愚人节恶作剧已经被实验过了:我们的宇航员被放在一个叫做宇宙飞船的盒子里,然后从天空之外丢下它(宇航员返回地球)。人性化起见,总是给盒子一个向前的运动,以便坠入地面时和地面有个好的弯曲度感受。

为更详细的了解引力,然我们再次思考愚人节恶作剧。为了让这一恶作剧奏效,关键是要所有物体精确地以相同的速率下落。相反,若盒子比我们的朋友下落的快,那么我们的朋友将会发现自己被钉在天花板上,他可能会解释为存在一个力将他往上拉。若盒子比他下落的慢,则他会感到一个力把他拉在地面上。所有的物体以相同的速率下落而和它们的组成成分无关,这和日常经验是相反的,但是Galileo猜测我们的日常经验被空气阻力给扭曲了。

\subsubsection{引力的普遍性}
一个坠落的人不知道他是下落的,因为他周围的所有东西都以同样的速率下落。


\subsection{远离引力场的加速系}






