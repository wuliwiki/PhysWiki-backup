% 空间旋转矩阵
% 线性代数|矩阵|平面旋转矩阵|空间旋转矩阵

% 先解释 平面旋转矩阵中的矩阵元是 x*x' 内积得出的, 任何直角坐标的变换都可以用内积完成.
% 顺便可以解释为什么如果所有列矢量正交归一, 所有行矢量也会正交归一.

\pentry{平面旋转矩阵\upref{Rot2D}, 自由度\upref{DoF}}

类比平面旋转矩阵\upref{Rot2D},空间旋转矩阵是三维坐标的旋转变换,所以应该是 $3 \cross 3$ 的方阵.不同的是平面旋转变换只有一个自由度 $\theta $, 而空间旋转变换除了转过的角度还需要考虑转轴的方向, 三维空间中的方向有两个自由度, 所有三维旋转矩阵共有 3 个自由度.

若已经知道空间直角坐标系中三个单位正交矢量
\begin{equation}
\uvec x=(1,0,0)\Tr \quad \uvec y=(0,1,0)\Tr \quad \uvec z=(0,0,1)\Tr
\end{equation}
经过三维旋转矩阵变换以后变为另外三个正交归一矢量, 令坐标变为 
\begin{equation}\label{Rot3D_eq1}
({a_{11}},{a_{21}},{a_{31}})\Tr \quad ({a_{12}},{a_{22}},{a_{32}})\Tr \quad ({a_{13}},{a_{23}},{a_{33}})\Tr
\end{equation}
类比平面旋转矩阵\upref{Rot2D}, 可以得到旋转矩阵为
\begin{equation}
\mat R_3 = \begin{pmatrix}
{a_{11}}&{a_{12}}&{a_{13}}\\
{a_{21}}&{a_{22}}&{a_{23}}\\
{a_{31}}&{a_{32}}&{a_{33}}
\end{pmatrix}\end{equation}
这 9 个矩阵元只有 3 个是独立的, 因为我们有 6 个条件: 每个列矢量模长等于 1(3 个等式), 且两两间正交(3 个等式).

除了通过三个单位矢量构建旋转矩阵, 我们可以通过由转轴的方向和旋转的角度来计算每个矩阵元, 参考 “绕轴旋转矩阵\upref{RotA}” 和 “四元数\upref{QuatN}”.

\subsection{逆矩阵}
如果我们把\autoref{Rot3D_eq1} 中的三个正交归一基底记为 

% 未完成 三维的旋转旋转矩阵与二维旋转矩阵具有许多相似的性质.

% 未完成: 和平面旋转一样, 两种理解: 矢量旋转和基底变换. 后者是很重要的
