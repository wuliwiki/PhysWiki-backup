% 常微分方程简介
% keys ODE|ordinary differential equation

\pentry{微积分/数学分析}

方程最初是“等式”的同义词,发展到现在,又特指“含未知量的等式”.解方程的过程,就是通过摆弄方程,求出未知量.

我们从小学就开始接触到的方程,通常又被叫做“代数方程”.代数方程的未知量就是数字,我们通过在方程两边加减乘除乘方开方,或者在多个方程之间进行加减乘除乘方开方等操作,计算出这个(些)数字到底是什么,这样就解出了一个代数方程.由于解方程时所用到的操作全都是代数运算,所以称之为代数方程.

微分方程的未知量则是函数,而方程中会涉及对这些函数求微分.解微分方程要做的是,得到具体的函数表达式.

举例而言,如果我们知道一个物体在空气中受到的阻力正比于其速度,那么这个物体在重力作用下在空气中下落时,该怎么求出其每时每刻的速度大小呢?这里,未知量就是速度函数,其自变量是时间.如果用$v$来表示物体竖直向下的速度大小,那么空气对它造成的加速度就可以写成$-kv$,重力对它造成的加速度则是$g$,其中$k, g$都是正实数.我们自然可以根据问题描述,写下如下关系:
\begin{equation}
\frac{\dd v}{\dd t}=-kv+g
\end{equation}

求解这个微分方程,就是得到类似$v=\frac{g}{k}- C\E^{-kt}$的结果,其中$C$是一个待定常数,由该物体的初速度决定.

在各个领域中,我们常常需要用函数来描述各种现象,比如人口的增长;而我们通常能获取到的信息是这些函数的变化率等和微分有关的信息,比如人口增长率.解微分方程的方法就能处理这些问题.

微分方程通常被分为两个类型,\textbf{常微分方程(ordinary differential equation,ODE)}和\textbf{偏微分方程(partial differential equation,PDE)}.常微分方程是指未知函数只有一个自变量的微分方程,而偏微分方程的未知函数则含有多个未知量,使得偏微分方程的表达式中有很多偏微分,由此得名.


















