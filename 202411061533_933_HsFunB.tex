% 函数回顾(高中)
% keys 初中|函数|正比例|反比例|二次
% license Xiao
% type Tutor

\begin{issues}
\issueDraft
\end{issues}

在初中阶段,函数的概念初步展示了变量之间的关系。初中接触的函数主要包括正比例函数、反比例函数和一次函数二次函数。

\subsection{正比例函数}

对应的是一条过原点的直线,

斜率


\subsection{反比例函数}

对应的是两条双曲线。你还学过如何根据反比例函数的表达式,通过已知的点来求解函数的值。

中心对称性

矩形面积相同


\subsection{一次函数}

截距

\subsection{二次函数}

二次函数的图像可能与  x  轴有两个交点,并且具有一个对称轴和一个最低点。

轴对称性:图像上对称点到对称轴的距离相等,且连线与对称轴垂直。

关于二次函数和一元二次方程的关系参见\enref{因式分解与一元二次方程}{quasol}。