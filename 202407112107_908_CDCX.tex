% 超对称性
% license CCBYSA3
% type Wiki

(本文根据 CC-BY-SA 协议转载自原搜狗科学百科对英文维基百科的翻译)

在粒子物理学中,超对称性(SUSY)是一个原理,它提出了基本粒子之间两种基本类别的关系:玻色子具有整数值自旋和费米子具有半整数自旋。[1][2] 超对称性是时空对称的一种类型,是未被发现的粒子物理学的一个可能的候选者,如果被证实是正确的,它被认为是当前粒子物理学中许多问题的一个优秀的解决方案,可以解决当前理论被认为是不完整的各个领域。对标准模型的超对称扩展将通过保证扰动理论中所有阶的二次发散被抵消来解决规范理论中的主要层次问题。

在超对称性中,来自一组中的每一个粒子在另一组中都有一个关联的粒子,称为它的超对称伙伴,其自旋相差半个整数。这些超对称伙伴将是新的未被发现的粒子。例如,有一种粒子叫做“超电子”(超伙伴电子),是电子的玻色子伙伴。在最简单的超对称性理论中,由于完美的“不间断”超对称性,每对超对称伙伴有相同的质量和内部量子数。因为我们希望用现在的设备找到这些“超对称伙伴”,如果超对称性存在,那么它由自发破缺的对称性组成,允许超对称伙伴在质量上有所不同。[3][4][5] 自发破缺超对称性可以解决粒子物理中的许多神秘问题,包括等级问题。

目前没有证据表明超对称性是否正确,或者对当前模型的其他扩展可能更准确。从某种程度上说,这是因为自大约2010年,专门为研究标准模型之外的物理而设计的粒子加速器才开始运行,而且还不知道具体在哪里寻找,也不知道成功搜索所需的能量。

物理学家支持超对称性的主要原因是,目前的理论已知是不完整的,而且它们的局限性是公认的,超对称性将是一些主要问题的有吸引力的解决方案。直接确认将需要在对撞机实验中生产超对称伙伴,例如大型强子对撞机(LHC)。大型强子对撞机(LHC)的第一次运行除了希格斯玻色子之外没有发现以前未知的粒子,希格斯玻色子已经被怀疑是标准模型的一部分,因此也没有超对称性的证据。[6][7] 间接方法包括在已知的标准模型粒子中寻找永久电偶极矩(EDM),这可能发生在标准模型粒子与超对称粒子相互作用时。目前对电子电偶极矩的最佳约束是使其小于10-28 e cm,相当于对TeV标度下新物理的灵敏度,与目前最好的粒子对撞机匹配。[8] 任何基本粒子中的永久电偶极矩都指向违反时间反转的物理现象,因此由CPT定理指向CP不对称性。这种电偶极矩实验也比传统的粒子加速器具有更大的可扩展性,并且随着加速器实验变得越来越昂贵和维护起来越来越复杂,它为检测超出标准模型的物理成分提供了一种实用的替代方法。

这些发现让许多物理学家失望,他们认为超对称性(以及其他依赖它的理论)是迄今为止“新”物理学最有希望的理论,并希望这些实验能有意想不到的结果。[9][10] 前热情支持者米哈伊尔·希夫曼(Mikhail Shifman)甚至敦促理论界寻找新的想法,并接受超对称性是一个失败的理论。[11] 然而,也有人认为这种“自然”危机还为时过早,因为各种计算对质量极限过于乐观,而质量极限将得到基于超对称性的解决方案。[12][13]

\subsection{动机}
超对称性在电弱尺度附近有许多现象学动机,以及任何尺度的超对称性的技术动机。
\subsubsection{1.1 等级问题}
接近电弱尺度的超对称性改善了困扰标准模型的等级问题。[14] 在标准模型中,电弱尺度得到巨大的普朗克尺度量子修正。观察到的弱电标度和普朗克标度之间的等级必须通过非常精细的调谐来实现。另一方面,在超对称理论中,在伙伴和超伙伴之间的普朗克尺度的量子校正相互抵消(由于费米子环有负号)。弱电标度和普朗克标度之间的等级是以自然的方式实现的,没有奇迹般的调谐。