% 共晶、共析相图
共晶转变:由一个液相生成两个固相 $L \rightarrow \alpha+\beta$

共析转变:由一个固相生成两个固相 $\gamma \rightarrow \alpha+\beta$

共晶转变与共析转变具有很多相似之处,因此本文主要介绍共晶转变,以Pb
-Sn合金为例

在热力学中,往往只关心转变前后的相变化;但由于动力学因素,各相晶粒往往形成一定的有序组织结构.本文一并简要讨论.

\subsection{共晶相图}
\begin{figure}[ht]
\centering
\includegraphics[width=12cm]{./figures/EUTECT_1.png}
\caption{典型的共晶系合金相图.注意固体部分不完全是固溶体.} \label{EUTECT_fig1}
\end{figure}
根据具体的成分不同,共晶系合金的转变过程可分为以下几类.

\subsubsection{共晶合金}
\begin{figure}[ht]
\centering
\includegraphics[width=14cm]{./figures/EUTECT_2.png}
\caption{共晶合金} \label{EUTECT_fig2}
\end{figure}

共晶合金的成分正好等于共晶点(i点)所对应的成分 ($\omega_{Sn}=61.9$).

全程的相转变:$L \rightarrow \alpha+\beta$

全程的组织转变:$L \rightarrow (\alpha+\beta)_{eutectic}$

i处,发生共晶转变,由一个液相生成两个固相,并形成一定组织结构 $L \rightarrow (\alpha+\beta)_{eutectic}$
\begin{itemize}
\item 共晶转变是恒成分转变,即共晶转变全程中,先后结晶部分的成分一致.例如,共晶转变全程中,α相中Sn浓度始终为$18.3\%$.
\item 共晶转变是恒温转变: 相转变时,系统自由度f=2-3+1=0,因此相转变温度是定值,在相图上体现为三相区是直线.
\end{itemize}

随后发生脱熔转变 $\alpha \rightarrow \beta_{II}, \beta \rightarrow \alpha_{II}$
\begin{itemize}
\item 随温度降低,α相溶解Sn、β相溶解Pb的能力均减弱.Sn将以β相固溶体的形式从α中析出,而Pb将以α相固溶体的形式从β中析出,称为二次相 $\alpha_{II},\beta_{II} $.
\item 脱熔转变生成的$\alpha_{II},\beta_{II} $结构与性质与$\alpha, \beta$完全相同.
\item 但是,αII,βII难以与共晶组织中α,β区分,可以不标出
\end{itemize}

\subsubsection{亚共晶合金}
\begin{figure}[ht]
\centering
\includegraphics[width=14cm]{./figures/EUTECT_3.png}
\caption{亚共晶合金} \label{EUTECT_fig3}
\end{figure}

亚共晶合金的成分小于共晶点所对应的成分,但大于Sn在$\alpha$中的最大固溶度 ($18.3<\omega_{Sn}<61.9$).

全程的相转变:$L \rightarrow \alpha+\beta$

全程的组织转变:$L \rightarrow \alpha_{primary}+\beta_{II}+(\alpha+\beta)_{eutectic}$

k处,部分液体先发生匀晶转变生成α相,此部分先生成的α相称为初生α相$L \rightarrow \alpha_{primary}$.α相中Pb的浓度高于液相浓度;由于Pb进入了初生α相,剩余液体中Pb浓度下降,Sn浓度上升;随着α相不断生成,液体的成分逐渐与共晶成分相同.

m处,剩余液体的发生共晶转变并生成共晶组织 $L \rightarrow (\alpha+\beta)_{eutectic}$

随后发生脱溶转变$\alpha \rightarrow \beta_{II}, \beta \rightarrow \alpha_{II}$.只有在α相中的βII脱熔相容易被观察到.

\subsubsection{过共晶合金}
过共晶合金的成分大于共晶点所对应的成分,但小于Pb在$\beta$中的最大固溶度 ($61.9<\omega_{Sn}<97.8$).

过共晶合金的转变过程与亚共晶合金类似.

\subsubsection{端部固溶体}
\begin{figure}[ht]
\centering
\includegraphics[width=10cm]{./figures/EUTECT_4.png}
\caption{端部固溶体} \label{EUTECT_fig4}
\end{figure}
端部固溶体的成分小于Sn在$\alpha$中的最大固溶度($\omega_{Sn}<18.3$),或大于Pb在$\beta$中的最大固溶度$\omega_{Sn}>97.8$).由于二者类似,此处只讨论前者.

全程的相转变:$L \rightarrow \alpha+\beta$

全程的组织转变:$L \rightarrow \alpha+ \beta_{II}$

端部固溶体的转变比较简单,全程不涉及共晶转变.

e处发生匀晶转变$L \rightarrow \alpha$

g处发生脱熔转变$\alpha \rightarrow \beta_{II}$