% 欧几里得环
% Euclid Ring|Euclid环|整环|多项式|因式分解|欧几里得整环|Euclid Domain

\pentry{真因子树\upref{FctTre}}

\begin{definition}{欧几里得环}
给定整环$R$,如果存在映射$\delta:R\to\mathbb{Z}^+\cup\{0\}$\footnote{即给每一个环中元素赋予一个非负整数.},使得对于任意$a, b\in R$且$b\not=0$,都存在$q, r\in R$使得$a=qb+r$,且$\delta(r)<\delta(b)$,则称$R$为一个\textbf{欧几里得环(Euclid Ring)},或者\textbf{欧几里得整环(Euclid Domain)}.
\end{definition}



直观来说,欧几里得环就是“可以做辗转相除法的环”.这从定义就可以看出来:任取$a$,再用任意的$b$去尝试除以它,总能得到$qb+r$的形式,其中$r$相当于除法的余数.虽然任意环中的元素都可以这么做分解,而欧几里得环就特殊在它还关系到一个非负整数赋值,使得余数的赋值总是小于非零除数$b$的,这样就使得我们可以多次分解,也就是进行辗转相除.

为了严格证明以上说法成立,我们还需要讨论欧几里得环的一个性质:

\begin{theorem}{}\label{EuRing_the1}
设$R$是一个欧几里得环,那么对于$r\in R$,有:

$r=0\iff \delta(r)=0$.
\end{theorem}

\textbf{证明}:

任选$a\in R$且$a\not=0$,那么$a=1a+0$,故由欧几里得环的定义可推知,对于任意$a\in R-\{0\}$,必有$\delta(0)<\delta(a)$,由此得证.

\textbf{证毕}.

有了\autoref{EuRing_the1} , 我们还可以补上定义中没有说明的一点:当$\delta(a)\leq\delta(b)$的时候,我们可以取$q=0$而$r=a$来满足$a=qb+r$且符合欧几里得环的性质.因此,一般只讨论$\delta(a)>\delta(b)$的情形.

我们来看几个典型的欧几里得环.

\begin{example}{整数环}
对于$n\in\mathbb{Z}$,令$\delta(n)=n$,则容易看出$\mathbb{Z}$是一个欧几里得环.
\end{example}

\begin{example}{多项式环}
任给域$\mathbb{K}$,其多项式环$\mathbb{K}[x]$是一个欧几里得环.取$\delta(f(x))=2^{\opn{deg}(f)}$即可.
\end{example}

$\opn(deg)(f)$是$f$的次数,也就是系数非零的最高项的幂次.这里$\delta$要取$2$的指数是因为$\opn{deg}(0)=-\infty$,而我们希望$\delta(0)=0$.

\begin{example}{高斯整数环}
高斯整数环是复平面上的全体坐标为整数的点的集合$\mathbb{Z}[\I]$构成的环,并且是一个欧几里得环.证明见下.
\end{example}

\textbf{证明}:

对于任意$z\in\mathbb{Z}[\I]$,令$\delta(z)=\abs{z}^2$.

任给$a, b\in\mathbb{Z}[\I]$,不妨设$\delta(a)>\delta(b)$.于是必有$u, v\in\mathbb{Q}$,使得$\frac{a}{b}=(u+v\I)$.

于是必有$c, d\in\mathbb{Z}$,使得$\abs{c-u}\leq\frac{1}{2}$

\textbf{证毕}.
