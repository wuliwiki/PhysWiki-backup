% 康托尔定理(综述)
% license CCBYSA3
% type Wiki

本文根据 CC-BY-SA 协议转载翻译自维基百科\href{https://en.wikipedia.org/wiki/Cantor\%27s_theorem}{相关文章}。

\begin{figure}[ht]
\centering
\includegraphics[width=8cm]{./figures/28b3b283de600fc9.png}
\caption{集合 $\{x, y, z\}$ 的基数是 3,而它的幂集中有 8 个元素($3 < 2^3 = 8$),如下图按包含关系排列。} \label{fig_KTDL_1}
\end{figure}
在数学的集合论中,康托尔定理是一个基本结论,它指出:对于任意集合 $A$,其幂集(即包含 $A$ 所有子集的集合)具有严格大于$A$ 本身的基数。

对于有限集合,康托尔定理可以通过对子集数量的直接枚举来验证。将空集也算作一个子集,若一个集合含有 $n$ 个元素,则它共有 $2^n$ 个子集。而由于对所有非负整数都有 $2^n > n$,因此定理在有限情形下成立。

更为重要的是,康托尔发现了一种适用于任意集合的论证方法,表明该定理对无限集合同样成立。因此,实数集的基数(它与整数集的幂集具有相同基数)严格大于整数集的基数;详细内容见“连续统的基数”。

该定理以格奥尔格·康托尔命名,他在19世纪末首次提出并证明了这一命题。康托尔定理对数学哲学产生了直接而深远的影响。例如,通过对一个无限集合不断取幂集并应用康托尔定理,可以得到一个无尽的无限基数层级,每一层的基数都严格大于前一层。因此,该定理意味着:不存在“最大的”基数(通俗地说,就是“没有最大的无穷大”)。
\subsection{证明}
康托尔的论证既优雅又极其简洁。完整的证明如下,随后将附上详细解释。

\textbf{定理(康托尔)} — 设$f$是一个从集合$A$到其幂集$\mathcal{P}(A)$的映射,即$
f: A \to \mathcal{P}(A)$,那么$f$不是满射。因此,对于任意集合$A$,都有:
$\operatorname{card}(A) < \operatorname{card}(\mathcal{P}(A))$即集合 $A$ 的基数严格小于其幂集的基数。
