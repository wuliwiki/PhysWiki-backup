% 古斯塔夫·赫兹(综述)
% license CCBYSA3
% type Wiki

本文根据 CC-BY-SA 协议转载翻译自维基百科\href{https://en.wikipedia.org/wiki/Gustav_Ludwig_Hertz}{相关文章}。

古斯塔夫·路德维希·赫兹(德语:[ˈɡʊs.taf ˈluːt.vɪç hɛʁt͡s] ⓘ;1887年7月22日-1975年10月30日)\(^\text{[2]}\)是德国的原子物理学家,他与詹姆斯·弗兰克共同获得了1925年诺贝尔物理学奖,“因其发现了电子与原子碰撞的规律”。
\subsection{生平}
赫兹出生于汉堡,父母分别是奥古斯特(婚前姓阿尔宁)和律师古斯塔夫·西奥多·赫兹(1858-1904)\(^\text{[2]}\),他是海因里希·鲁道夫·赫兹的兄弟。他在汉堡的约翰修道院学校接受教育,之后在以下大学学习:1906年至1907年间,哥廷根的乔治-奥古斯特大学;1907年至1908年间,慕尼黑的路德维希·马克西米利安大学;以及1908年至1911年间,柏林的洪堡大学。1911年,他在海因里希·鲁本斯教授指导下获得博士学位。\(^\text{[3][4]}\)

从1911年到1914年,赫兹是鲁本斯教授在柏林大学的助手。在这段时间里,赫兹与詹姆斯·弗兰克一起进行了关于气体中非弹性电子碰撞的实验,通常称为弗兰克-赫兹实验,\(^\text{[5]}\)因这一实验,他们共同获得了1925年诺贝尔物理学奖。\(^\text{[6]}\)

第一次世界大战期间,赫兹自1914年起服役。1915年,他加入了弗里茨·哈伯的部队,该部队负责使用有毒的氯气作为武器。\(^\text{[7]}\)赫兹在1915年受重伤。1917年,他返回柏林大学担任私人讲师。1920年,他在埃因霍温的飞利浦白炽灯厂担任研究物理学家,直到1925年。\(^\text{[2]}\)
\subsubsection{职业生涯}
1925年,赫兹成为马丁·路德大学哈雷-威滕贝格分校物理学研究所的正教授和所长。1928年,他成为柏林工业大学(现为柏林工业大学)实验物理学的正教授及物理学研究所所长。在此期间,他开发了一种通过气体扩散分离同位素的技术。

由于赫兹在第一次世界大战期间曾服役,他曾一度免受纳粹政策和《恢复公务员法》的影响,但随着政策和法律变得更加严格,到1934年底,他被迫辞去柏林工业大学的职务,因为他被归类为“二级部分犹太人”(他的祖父古斯塔夫·费尔迪南德·赫兹(原名大卫·古斯塔夫·赫兹)(1827-1914)童年时期曾是犹太人,直到全家在1834年皈依路德教)。随后,他在西门子公司担任第二研究实验室的所长。在那里,他继续从事原子物理学和超声波的研究,但最终停止了同位素分离方面的工作。他一直担任这一职务,直到1945年前往苏联。\(^\text{[4][2][9]}\)
\subsubsection{在苏联}
\textbf{“叛逃协议”}

赫兹对自己的安全感到担忧,与他同为诺贝尔奖得主的詹姆斯·弗兰克一样,他正在寻求前往美国或其他任何德国以外的地方。因此,他与三位同事达成了协议:曼弗雷德·冯·阿尔登,私人实验室“电子物理学研究实验室”主任;彼得·阿道夫·蒂森,柏林洪堡大学的正教授及凯瑟·威廉姆物理化学与电化学研究所(KWIPC)在柏林达赫姆的所长;以及马克斯·福尔默,柏林工业大学物理化学研究所的正教授和所长。\(^\text{[10]}\)该协议的内容是,谁先与苏联取得联系,谁就代表其他人发声。协议的目标有三个:(1)防止他们的研究所遭受掠夺,(2)尽量减少工作中断,(3)保护他们免受因过去政治行为的起诉。\(^\text{[11]}\)在第二次世界大战结束前,蒂森是纳粹党成员,并有共产主义联系。\(^\text{[12]}\)
