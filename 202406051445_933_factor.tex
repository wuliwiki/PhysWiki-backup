% 阶乘(高中)
% keys 整数|gamma 函数|pi 符号
% license Xiao
% type Tutor

\begin{issues}
\issueDraft
\end{issues}

\pentry{\enref{求积符号(累乘)}{ProdSy}}{nod_fe64}

对自然数 $n$\footnote{拓展知识:阶乘的定义本身只限于自然数。\enref{Gamma 函数}{Gamma}实现了对阶乘的解析延拓,即:1.保证它的函数值与阶乘对应;2.保证自变量取其他实数值(如:部分负数、分数、和无理数等)时也可以有结果;3.函数的性质满足特定条件。因此,随着研究深入经常也会用Gamma函数来代替需要表达“阶乘”概念的场景。}, \textbf{阶乘(factorial)}定义为所有小于等于 $n$ 的正整数的乘积,即
\begin{equation}
n! := \prod_{i = 1}^n i =1 \cdot 2 \cdot 3 \dots (n - 2) (n - 1)n~.
\end{equation}
其中,$\prod_{i = 1}^n i$为\enref{求积符号}{ProdSy}。特殊地,约定
\begin{equation}
0! := 1~.
\end{equation}

\begin{example}{求5的阶乘}
小于等于5的正整数有:1,2,3,4,5,因此:
$$5! = 1\times 2\times 3\times 4\times 5 = 120.~$$
\end{example}

