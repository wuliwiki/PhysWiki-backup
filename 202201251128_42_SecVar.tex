% 二次变分
% 二次变分

\begin{issues}
\issueTODO
\end{issues}

\pentry{变分\upref{Varia}}
在数学分析中,一次(或阶)微分为0是函数取极值的必要条件,为了获得极值是极大值或极小值的信息,可以研究二次微分.同样的,在泛函中,一次变分为0是泛函取极值的必要条件,而研究二次变分可以获得极值是极大值或极小值的信息.

泛函 $J(y)=\int_a^bF(x,y,y')\dd x$ 的\textbf{二次变分} $\delta^2J$ 是指
\begin{equation}
\delta^2J=\frac{1}{2}\int_a^b(F_{aa}\delta y^2+2F_{yy'}\delta y\delta y'+F_{y'y'}\delta y'^2)\dd x
\end{equation}
\subsection{二次变分的引入}
设 
\begin{equation}
J(y)=\int_a^bF(x,y,y')\dd x
\end{equation}
为定义在有固定端点的 $C_1$ 类曲线上的泛函.

应用泰勒公式并引用符号\autoref{Varia_eq6}~\upref{Varia}
\begin{equation}
\delta y(x)=\overline{y}(x)-y(x)
\end{equation}
此处 $\delta y(a)=\delta y(b)=0$.则
\begin{equation}
\begin{aligned}
J(\overline{y})-J(y)&=\int_a^b[F(x,\overline{y},\overline{y}')-F(x,y,y')]\dd x\\
&=\int_a^b\qty[F'_y\delta y+F'_{y'}\delta y'+\frac{1}{2}\qty(\overlinr{F}''_{yy}\delta y^2+2\overline{F}''_{yy'}\delta y\delta y'+\overline{F}''_{y'y'}\delta y'^2)]\dd x
\end{aligned}
\end{equation}
