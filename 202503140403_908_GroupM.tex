% 群(综述)
% license CCBYSA3
% type Wiki

本文根据 CC-BY-SA 协议转载翻译自维基百科\href{https://en.wikipedia.org/wiki/Group_(mathematics)}{相关文章}。

\begin{figure}[ht]
\centering
\includegraphics[width=6cm]{./figures/3d977276c842b7f1.png}
\caption{魔方的操作构成了魔方群。} \label{fig_GroupM_1}
\end{figure}
在数学中,群是一个具有二元运算的集合,并满足以下约束条件:该运算是结合的,它具有单位元,并且集合中的每个元素都有逆元。  

许多数学结构都是具有其他性质的群。例如,整数在加法运算下构成一个无限群,该群由一个称为\textbf{1}的单一元素生成(这些性质以独特的方式刻画了整数)。

群的概念被提出,以统一方式处理许多数学结构,例如数、几何形状和多项式的根。由于群的概念在数学内外的多个领域中无处不在,一些作者将其视为当代数学的核心组织原则之一。  

在几何学中,群自然地出现在对对称性和几何变换的研究中:一个物体的对称性构成一个群,称为该物体的\textbf{对称群},而某种特定类型的变换构成一个更一般的群。\textbf{李群}在几何中的对称群中出现,也出现在粒子物理学的\textbf{标准模型}中。\textbf{庞加莱群}是一个李群,包含狭义相对论中时空的对称性。\textbf{点群}则用于描述分子化学中的对称性。

群的概念起源于对多项式方程的研究,最早由埃瓦里斯特·伽罗瓦在 1830 年代提出,他使用 群(法语:groupe)这一术语来描述方程根的对称群,这一概念如今被称为\textbf{伽罗瓦群}。随着来自数论、几何等其他领域的贡献,群的概念得到了推广,并在1870 年左右被正式确立。\textbf{现代群论}是一个活跃的数学学科,它研究群本身的性质。为了探索群,数学家引入了各种概念,以便将群分解为更小、更易理解的部分,例如子群、商群和单群。除了研究群的抽象性质之外,群论学者还研究群的具体表现方式,包括表示论(即群的表示)和计算群论的方法。对于有限群,已经发展出一整套理论,并最终在2004年完成了有限单群的分类。自 20世纪80年代中期以来,\textbf{几何群论}这一分支迅速发展,它将\textbf{有限生成群}视为几何对象进行研究,成为群论中的一个活跃领域。
\subsection{定义与示例}  
\subsubsection{第一个例子:整数}
一个常见的群是整数集
\[
\mathbb{Z} = \{\ldots, -4, -3, -2, -1, 0, 1, 2, 3, 4, \ldots\}~
\]
配备\textbf{加法运算} \((+)\) 。对于任意两个整数 \(a\) 和 \(b\),它们的和 \(a + b\) 仍然是整数;这个封闭性表明加法是整数集 \(\mathbb{Z}\) 上的一个二元运算。  

整数加法的以下性质构成了群的基本公理,并在下面的定义中得到了推广:  
\begin{itemize}
\item 结合律(Associativity)对于所有整数 \(a, b, c\),有:\((a + b) + c = a + (b + c)\)这意味着,无论是先将 \(a\) 与 \(b\) 相加再加上 \(c\),还是先将 \(b\) 与 \(c\) 相加再加上 \(a\),最终的结果相同。  
\item 单位元(Identity Element)对于任意整数 \(a\),有:\(0 + a = a \quad \text{且} \quad a + 0 = a\) \textbf{0}被称为\textbf{加法的单位元},因为它与任意整数相加都不改变该整数的值。  
\item 逆元(Inverse Element) 对于任意整数 \(a\),存在一个整数 \(b\),使得: \(a + b = 0 \quad \text{且} \quad b + a = 0\)这个整数 \(b\) 称为 \(a\) 的\textbf{加法逆元},通常记作 \(-a\)。  
\end{itemize}
整数集 \(\mathbb{Z}\) 连同加法运算构成了一个数学结构,该结构属于一类具有相似性质的更广泛的代数对象。为了更系统地理解这类结构,下面给出正式的定义。
\subsubsection{定义} 
一个群是一个\textbf{非空集合 }\( G \),配备一个\textbf{二元运算}(在此记作“\( \cdot \)”),该运算将 \( G \) 中的任意两个元素 \( a \) 和 \( b \) 组合,得到仍属于 \( G \) 的元素 \( a \cdot b \)。此外,该运算必须满足以下三个被称为\textbf{群公理}的条件:[5][6][7][a]  

结合律(Associativity) 

对于所有 \( a, b, c \in G \),有:\((a \cdot b) \cdot c = a \cdot (b \cdot c)\)这意味着运算的计算顺序不会影响最终结果。  

单位元(Identity Element) 

存在一个元素 \( e \in G \),使得对于任意 \( a \in G \),有:\(e \cdot a = a \quad \text{且} \quad a \cdot e = a\)该元素 \( e \) 是唯一的(见下文),称为单位元(或中性元)。  

逆元(Inverse Element)

对于 \( G \) 中的每个元素 \( a \),存在一个元素 \( b \in G \),使得:\(a \cdot b = e \quad \text{且} \quad b \cdot a = e\)其中 \( e \) 是单位元。对于每个 \( a \),这个元素 \( b \) 是唯一的(见下文),称为 \( a \) 的逆元,通常记作 \( a^{-1} \)。
\subsubsection{符号与术语} 
从形式上看,一个群是一个\textbf{有序对},由一个集合和该集合上的二元运算组成,并满足群公理。这个集合称为\textbf{群的底层集合},而该运算称为群运算或群律。  

因此,群和它的底层集合是两个不同的数学对象。为了避免繁琐的符号表示,通常会滥用符号,用同一个符号表示二者。这种做法也反映了一种非正式的思维方式:即认为群只是该集合的一个“扩展版本”,其附加的结构由群运算提供。  

例如,考虑\textbf{实数集} \( \mathbb{R} \),它配备了加法运算 \( a + b \) 和乘法运算 \( ab \):形式上,\( \mathbb{R} \)只是一个集合,\( (\mathbb{R}, +) \)是一个加法群,\( (\mathbb{R}, +, \cdot) \)是一个域(field)。然而,在实际使用中,通常直接用 \( \mathbb{R} \) 来表示这三种对象之一,而不作区分。  

在域 \( \mathbb{R} \) 中,加法群(additive group)是以 \( \mathbb{R} \) 为底层集合,并以加法 \( + \) 作为运算的群,即 \( (\mathbb{R}, +) \)。乘法群记作 \( \mathbb{R}^{\times} \),其底层集合是去掉零的实数集\( \mathbb{R}\setminus \{0\} \),其运算是乘法\( \cdot \),即 \( (\mathbb{R}^\times, \cdot) \)。

当群的运算使用加法表示时,通常称之为加法群。在这种情况下,单位元通常记作\textbf{0},元素 \( x \) 的逆元记作\(-x\)。同样,当群的运算使用乘法表示时,通常称之为乘法群。在这种情况下,单位元通常记作 \textbf{1},元素 \( x \) 的逆元记作 \( x^{-1} \)。在乘法群中,运算符号通常被省略,即直接用并置表示运算,例如 \( ab \) 代替 \( a \cdot b \)。  

群的定义并不要求对所有元素 \( a, b \in G \) 都满足:\(a \cdot b = b \cdot a\) 如果满足这个额外条件,则称该运算是交换的,并称该群为\textbf{阿贝尔群}。通常的惯例是:阿贝尔群可以使用加法记号(\( + \))或乘法记号(\( \cdot \))。非阿贝尔群仅使用乘法记号(\( \cdot \))。

对于元素不是数的群,常见的还有其他记号。例如:当群的元素是函数时,运算通常是函数复合,记作:\(f \circ g\)在这种情况下,单位元通常记作\textbf{id}(即恒等函数)。在更具体的情况中,例如几何变换群、对称群、置换群和自同构群,运算符号 \( \circ \)通常被省略,类似于乘法群的记法。此外,群的记号可能会有许多其他变体,具体取决于应用领域和数学上下文。
\subsubsection{第二个例子:对称群 }
在平面上,如果一个图形可以通过旋转、反射和平移的组合变换成另一个图形,则它们是全等的。任何图形都与自身全等。然而,一些图形不仅与自身全等,而且有多种不同的方式与自身全等,这些额外的全等变换称为对称。  

例如,一个正方形有八种对称变换,它们包括:
\begin{figure}[ht]
\centering
\includegraphics[width=14.25cm]{./figures/f32a15365d3f3335.png}
\caption{} \label{fig_GroupM_2}
\end{figure}
\begin{itemize}
\item 恒等变换:保持所有内容不变,记作\textbf{id}。  
\item 旋转:以正方形的中心为旋转点,顺时针旋转90°,记作\( r_1 \) 180°,记作 \( r_2 \)270°,记作 \( r_3 \) 反射:  
\item 关于水平轴和垂直轴的反射,分别记作 \( f_h \) 和 \( f_v \);关于两条对角线的反射,分别记作 \( f_d \) 和 \( f_c \)。
\end{itemize}
这些对称变换本质上是函数,它们将正方形上的一个点映射到对称变换后的对应点。例如:\( r_1 \):将一个点顺时针旋转90°,围绕正方形的中心。\( f_h \):将一个点关于水平中轴线反射。组合两个这样的对称变换,仍然得到另一个对称变换。因此,这些对称变换构成了一个群,称为四阶二面体群,记作\( D_4 \)。该群的\textbf{底层集合}就是上述八个对称变换,而群运算是函数复合。两个对称变换的组合是按照函数复合的方式进行的,即:先应用第一个变换 \( a \) ,再应用第二个变换 \( b \)。记作:\(b \circ a\)这表示先执行对称变换 \( a \),然后对结果再应用对称变换 \( b \)。这里采用的记号是从右到左的顺序,这是函数复合的标准记法。

凯莱表列出了所有可能的对称变换组合的结果。例如:先顺时针旋转 270°(\( r_3 \)),再进行水平反射(\( f_h \)),其结果与沿对角线反射(\( f_d \))的效果相同。用上述符号表示,在凯莱表中(通常以蓝色高亮):  
\[
f_h \circ r_3 = f_d~
\]
给定这组对称变换及其运算方式,可以如下理解群公理:  

二元运算(Binary Operation):函数复合是一种二元运算,即对于任意两个对称变换 \( a \) 和 \( b \),它们的复合运算 \( a \circ b \) 仍然是一个对称变换。例如:  
 \[
 r_3 \circ f_h = f_c~
\]  
这表示先进行水平反射 \( f_h \),然后顺时针旋转 270°(\( r_3 \)),其结果等同于沿副对角线反射 \( f_c \)。实际上,任意两个对称变换的组合仍然是某种对称变换,这可以通过凯莱表(Cayley table)进行验证。

结合律(Associativity):结合律公理处理的是多个对称变换的组合。从 \( D_4 \) 群中的三个元素 \( a \)、\( b \) 和 \( c \) 开始,存在两种可能的方式按顺序使用这三个对称变换来确定正方形的对称性:先将 \( a \) 和 \( b \) 组合成一个对称变换,然后再将该对称变换与 \( c \) 组合。先将 \( b \) 和 \( c \) 组合成一个对称变换,然后再将该对称变换与 \( a \) 组合。这两种方法必须始终给出相同的结果,即:  
\[
(a \circ b) \circ c = a \circ (b \circ c)~
\]

例如,\((f_d \circ f_v) \circ r_2 = f_d \circ (f_v \circ r_2)\)可以通过凯莱表进行验证:  
\[
(f_d \circ f_v) \circ r_2 = r_3 \circ r_2 = r_1~
\]
\[
f_d \circ (f_v \circ r_2) = f_d \circ f_h = r_1~
\]  
单位元(Identity Element):单位元是\( \text{id} \),因为它与任何对称变换 \( a \) 组合(无论在左侧还是右侧)都不改变该变换的结果。

逆元(Inverse Element): 每个对称变换都有一个逆元:\( \text{id} \)、反射变换\( f_h \)、\( f_v \)、\( f_d \)、\( f_c \)和180° 旋转 \( r_2 \)都是它们自己的逆元,因为执行两次这些变换将正方形恢复到其原始的朝向。旋转变换\( r_3 \)和\( r_1 \)是彼此的逆元,因为先旋转 90° 再旋转 270°(或反之)得到的结果是旋转 360°,这会让正方形保持不变。这可以在凯莱表中轻松验证。

与上面提到的整数群不同,在\( D_4 \)群中,运算的顺序是重要的。例如:\(
f_h \circ r_1 = f_c \quad \text{但} \quad r_1 \circ f_h = f_d\)换句话说,\( D_4 \)不是阿贝尔群(abelian group)。
\subsection{历史}  
现代抽象群的概念源自多个数学领域的发展。[9][10][11] 群论的最初动机是寻求高于四次的多项式方程的解法。19世纪的法国数学家埃瓦里斯特·伽罗瓦在保罗·鲁菲尼和约瑟夫-路易·拉格朗日之前工作的基础上,提出了通过多项式方程根的对称群来判定方程是否可解的标准。这样的\textbf{伽罗瓦群}的元素对应于根的某些置换。最初,伽罗瓦的思想遭到当时学者的拒绝,并且仅在他去世后才得以发表。[12][13] 更一般的置换群被奥古斯丁·路易·柯西等人进一步研究。阿瑟·凯利在其著作《群论的理论》(On the Theory of Groups, as Depending on the Symbolic Equation \( \theta^n = 1 \),1854年)中,首次给出了有限群的抽象定义。[14]

几何学是群论被系统应用的第二个领域,特别是对称群,作为费利克斯·克莱因1872年\textbf{埃尔朗根计划}的一部分。[15] 在双曲几何和射影几何等新几何学出现后,克莱因利用群论将这些几何学以更一致的方式组织起来。进一步发展这些思想,索福斯·李于1884年创立了李群的研究。[16]

群论的第三个贡献领域是数论。在卡尔·弗里德里希·高斯的数论著作《算术研究》(Disquisitiones Arithmeticae,1798年)中,某些阿贝尔群结构已经被隐式使用,而莱奥波德·克罗内克则更加明确地应用了这些结构。[17] 在1847年,恩斯特·库默通过发展描述质数分解的群体,开始尝试证明\textbf{费尔马最后定理}。[18]

这些不同来源的汇聚形成了群论的统一理论,始于卡米尔·乔丹的《置换与代数方程论》(1870年)。[19]沃尔特·冯·迪克(1882年)提出了通过生成元和关系来指定群的概念,并且是第一个给出“抽象群”公理化定义的人(当时的术语)。进入20世纪后,群论通过费迪南德·乔治·弗罗贝纽斯和威廉·伯恩赛德的开创性工作(他们研究了有限群的表示论)、理查德·布劳尔的模表示理论和伊萨伊·舒尔的论文获得了广泛的认可。[21]李群理论,更广泛地说,局部紧群的研究由赫尔曼·维尔、埃利·卡尔坦等人进行。[22]其代数对应物——代数群理论,最初由克劳德·谢瓦雷(从1930年代末期开始)奠定,后来由阿尔芒·博雷尔和雅克·蒂茨的工作进一步发展。[23]

芝加哥大学的1960–61群论年汇聚了群论学者,如丹尼尔·戈伦斯坦、约翰·G·汤普森和沃尔特·费特,为一个合作奠定了基础,这个合作在许多其他数学家的贡献下,最终导致了有限单群的分类,最后一步由阿什巴赫和史密斯在2004年完成。这个项目在数学历史上超越了以往的工作,其规模无论是在证明的长度还是研究人员的数量上都是前所未有的。关于这个分类证明的研究仍在进行中。[24] 群论仍然是一个高度活跃的数学分支,[b] 对许多其他领域产生了影响,以下例子便展示了这一点。
\subsection{群公理的基本推论}  
从群公理直接得到的所有群的基本事实通常归纳于初等群论中。例如,结合性公理的反复应用表明,式子 \( a \cdot b \cdot c = (a \cdot b) \cdot c = a \cdot (b \cdot c) \) 的明确性可以推广到三个以上的因子。因为这意味着括号可以在这样的项序列中随意插入,所以通常省略括号。
\subsubsection{单位元的唯一性 } 
群公理暗示单位元是唯一的;也就是说,只有一个单位元:群中的任何两个单位元 \( e \) 和 \( f \) 是相等的,因为群公理暗示 \( e = e \cdot f = f \)。因此,通常会称之为群的单位元。
\subsubsection{逆元的唯一性}  
群公理还暗示每个元素的逆元是唯一的。设群中元素 \( a \) 具有两个逆元 \( b \) 和 \( c \),则:
\[
\begin{aligned}
b &= b \cdot e && (\text{\( e \) 是单位元}) \\
&= b \cdot (a \cdot c) && (\text{\( c \) 与 \( a \) 互为逆元}) \\
&= (b \cdot a) \cdot c && (\text{结合律}) \\
&= e \cdot c && (\text{\( b \) 是 \( a \) 的逆元}) \\
&= c && (\text{\( e \) 是单位元,且 \( b = c \)}) 
\end{aligned}~
\]
因此,通常称之为元素的逆元。
\subsubsection{除法} 
对于群 \( G \) 中的任意元素 \( a \) 和 \( b \),方程\(a \cdot x = b\)在 \( G \) 中有唯一解 \( x \),即\(x = a^{-1} \cdot b\)因此,对于每个 \( a \in G \),映射\(G \to G, \quad x \mapsto a \cdot x\)是一个双射,这个映射被称为左乘法或左平移由 \( a \) 作用。

类似地,对于任意 \( a, b \in G \),方程 \(x \cdot a = b\)的唯一解是\(x = b \cdot a^{-1}\)因此,对于每个\( a \in G \),映射\(G \to G, \quad x \mapsto x \cdot a\)也是一个双射,被称为右乘法或右平移由 \( a \) 作用。
\subsubsection{等价定义与放宽公理} 
群的单位元与逆元公理可以“弱化”,仅要求左单位元和左逆元的存在。从这些\textbf{单侧公理}出发,可以证明左单位元也是右单位元,而左逆元也是相应元素的右逆元。因此,这些公理实际上定义了与群完全相同的结构,因此整体而言,这些公理并不更弱。  

具体而言,假设群运算满足结合律,并且存在左单位元 \( e \)(即 \( e \cdot f = f \))以及每个元素 \( f \) 具有左逆元 \( f^{-1} \)(即 \( f^{-1} \cdot f = e \)),则可以证明每个左逆元也是相应元素的右逆元。  

证明如下:  
\[
\begin{aligned}
f \cdot f^{-1} &= e \cdot (f \cdot f^{-1}) && (\text{左单位元}) \\
&= ((f^{-1})^{-1} \cdot f^{-1}) \cdot (f \cdot f^{-1}) && (\text{左逆元}) \\
&= (f^{-1})^{-1} \cdot ((f^{-1} \cdot f) \cdot f^{-1}) && (\text{结合律}) \\
&= (f^{-1})^{-1} \cdot (e \cdot f^{-1}) && (\text{左逆元}) \\
&= (f^{-1})^{-1} \cdot f^{-1} && (\text{左单位元}) \\
&= e && (\text{左逆元})
\end{aligned}~
\]
同样,左单位元也是右单位元: 
\[
\begin{aligned}
f \cdot e &= f \cdot (f^{-1} \cdot f) && (\text{左逆元}) \\
&= (f \cdot f^{-1}) \cdot f && (\text{结合律}) \\
&= e \cdot f && (\text{右逆元}) \\
&= f && (\text{左单位元})
\end{aligned}~
\]
这些证明需要满足所有三个公理(结合律、左单位元的存在和左逆元的存在)。对于定义更宽松的结构(如半群),可能会出现左单位元不一定是右单位元的情况。  

同样的结论也可以仅假设\textbf{右单位元的存在和右逆元的存在}来获得。

然而,仅假设左单位元的存在和右逆元的存在(或反之)并不足以定义一个群。例如,考虑集合\(G = \{ e, f \}\)及其运算 \( \cdot \) 满足以下规则:\(e \cdot e = f \cdot e = e\)和\(e \cdot f = f \cdot f = f\)该结构确实具有左单位元(即 \( e \)),并且每个元素都有右逆元(对于两个元素来说都是 \( e \))。此外,该运算是结合的,因为任意多个元素的乘积始终等于最右侧的元素,无论计算顺序如何。然而,该结构 \((G, \cdot)\) 不是一个群,因为它缺少右单位元。
\subsection{基本概念}  
以下章节使用数学符号,例如\(X = \{x, y, z\}\)表示一个包含元素 \( x \)、\( y \) 和 \( z \) 的集合 \( X \),或者\(x \in X\)表示 \( x \) 是 \( X \) 的一个元素。符号\(f: X \to Y\)表示 \( f \) 是一个函数,它将 \( X \) 中的每个元素映射到 \( Y \) 中的某个元素。  

在研究集合时,会使用诸如子集、函数和按等价关系取商等概念。而在研究群时,则使用对应的子群、同态映射和商群等概念。这些概念是对集合论概念的推广,以考虑群的结构。
\subsubsection{群同态}  
群同态是保持群结构的函数,它们可用于建立两个群之间的联系。一个从群 \((G, \cdot)\) 到群 \((H, *)\) 的同态是一个函数\(\varphi: G \to H\)
满足  
\[
\varphi(a \cdot b) = \varphi(a) * \varphi(b), \quad \forall a, b \in G.~
\]
自然地,我们可能希望同态函数也保持单位元和逆元,即\(\varphi(1_G) = 1_H\)\(\varphi(a^{-1}) = \varphi(a)^{-1}, \quad \forall a \in G.\)然而,这些额外的条件不必包含在同态的定义中,因为它们已经被保持群运算的要求所蕴含。

群\(G\)的恒等同态是映射\(\iota_G: G \to G\)它将\( G\)中的每个元素映射到身。一个同态\(\varphi:G\to H\)的逆同态是一个映射\(\psi:H\to G\)满足\(\psi \circ\varphi = \iota_G,\quad \varphi \circ \psi = \iota_H,\)即对于所有\( g \in G \) 和 \( h \in H \)都有:\(\psi(\varphi(g)) = g, \quad \varphi(\psi(h)) = h\).如果一个同态存在逆同态,则称其为同构;等价地,一个同构就是一个双射同态。如果存在一个同构\( \varphi: G \to H \),则称群\(G\)和\( H \)同构。在这种情况下,可以通过函数 \( \varphi \) 将 \( G \) 的元素重命名得到\( H \),并且对\( G \)成立的所有命题,在适当重命名元素后,对于\(H\)也成立。

所有群及其之间的同态的集合构成了一个范畴,即群的范畴。  

一个单射同态\(\phi:G'\to G\)可以规范地分解为一个同构后跟一个包含映射:\(G' \; {\stackrel{\sim}{\to}} \; H \hookrightarrow G\)其中 \( H \) 是 \( G \) 的一个子群。单射同态是群的范畴中的单态射。
\subsubsection{子群}  
非正式地说,子群是一个包含在更大群\( G \)中的群\( H \):它是\( G \)元素的一个子集,并且使用相同的群运算。[32] 具体来说,这意味着 \( G \) 的单位元必须包含在 \( H \)中,并且每当 \( h_1 \) 和 \( h_2 \)都属于\( H \) 时,\(h_1 \cdot h_2 \)和\( h_1^{-1} \) 也必须在 \( H \) 中,因此,\( H \) 中的元素,配合在 \( G \)上限制到\( H \)的群运算,确实构成一个群。在这种情况下,包含映射\( H \to G \)是一个同态。

在正方形的对称性例子中,单位元和旋转构成了一个子群\(R = \{\text{id}, r_1, r_2, r_3\}\)在示例的凯利表格中以红色标出:任意两个旋转组合仍然是旋转,并且旋转可以通过互补的旋转来取消(即,互为逆元素):90°的旋转通过270°的旋转取消,180°的旋转通过另一个180°的旋转取消,270°的旋转通过90°的旋转取消。子群检验提供了一个充分且必要的条件,来判定群 \( G \) 的一个非空子集 \( H \) 是否为子群:只需要检查对于所有 \( g \) 和 \( h \) 属于 \( H \),有 \( g^{-1} \cdot h \in H \)。了解一个群的子群对于理解整个群非常重要。