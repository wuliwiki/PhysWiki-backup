% 共轭算子
% keys 共轭算子
% license Usr
% type Tutor

\pentry{共轭空间与代数共轭空间\nref{nod_ConSpa},拓扑线性空间中的线性算子\nref{nod_TLinO}}{nod_0325}

设 $E,E_1$ 是拓扑线性空间, $A$ 是 $E$ 到 $E_1$ 的\enref{线性连续算子}{TLinO},$g$ 是 $E_1$ 上的\enref{线性连续泛函}{LinCon},即 $g\in$ \enref{$ E_1^*$}{ConSpa},则由映射连续和线性的传递性,$gA$ 是 $E$ 上的线性连续泛函,即 $gA\in E^*$。因此,在线性连续算子 $A$ 的作用下,每一 $E_1$ 上的线性连续泛函 $g\in E_1^*$ 都对应一个泛函 $gA\in E^*$。即得从 $E_1^*$ 到 $E^*$ 的某个算子,这就是所谓的算子 $A$ 的共轭算子。

\begin{definition}{共轭算子}\label{def_ConOp_1}
设 $A$ 是线性拓扑空间 $E$ 到 $E_1$ 的线性连续算子,则称映射 $E_1^*\rightarrow  E^*:g\mapsto gA$ 为 $A$ 的\textbf{共轭算子},记作 $A^*$。
\end{definition}

\begin{definition}{符号约定}
设 $f$ 是泛函,则记 $(f,x):=f(x)$。
\end{definition}

\begin{theorem}{}
设 $A$ 是线性连续算子,则
\begin{equation}\label{eq_ConOp_1}
(g,Ax)=(A^* g,x).~
\end{equation}
\end{theorem}

\textbf{证明:}由共轭算子\autoref{def_ConOp_1} ,有
\begin{equation}
A^*g=gA.~
\end{equation}
于是 
\begin{equation}
(A^*g,x)=(gA,x)=g(Ax)=(g,Ax).~
\end{equation}

\textbf{证毕!}
 
 
\autoref{eq_ConOp_1} 往往被用作共轭算子的定义。


\begin{example}{有限维空间的共轭算子}
设 $A:\mathbb R^n\rightarrow\mathbb R^m$ 是线性连续算子,则 
\begin{equation}
A(E_1 X)=A(E_1)X=E_2 \bar AX.~
\end{equation}
其中 $E_1=(e_1,\cdots,e_n)$ 是 $\mathbb R^n$ 的基, $E_2=(e'_1,\cdots,e'_n)$ 是 $\mathbb R^m$ 的基,$X=(x_1,\cdots,x_n)^T$ 是矢量 $E_1X\in\mathbb R^n$ 在基 $E_1$ 下的坐标,$\bar A$ 为算子 $A$ 在两基底下对应的矩阵。设 $f$ 是 $\mathbb R^m$ 上的线性连续泛函,则由共轭算子定义,
\begin{equation}
A^*f=fA.~
\end{equation}
因此
\begin{equation}
\begin{aligned}
(A^*f)(E_1)X&=A^*f(E_1X)=(fA)(E_1X)\\
&=f(E_2 \bar AX)=f(E_2)\bar AX
\end{aligned}.~
\end{equation}
从而 $(A^*f)(E_1)=f(E_2)\bar A=\bar A^T f(E_2)^T$,注意 $f(E_2)^T$ 是 $f$ 在 ${\mathbb R^m}^*$ 中的坐标,因此该式表明在 $A^*$ 的作用下,$f$ 的坐标变换由矩阵 $\bar A^T$ 作用,即 $A^*$ 对应的矩阵是 $A$ 矩阵的转置。

\end{example}


\subsection{性质}

\begin{theorem}{}
设 $A,B$ 是连续线性算子,则成立:
\begin{enumerate}
\item $A^*$ 是线性的;
\item $(A+B)^*=A^*+B^*$;
\item $(\alpha A)^*=\alpha A^*$。
\end{enumerate}

\end{theorem}

\textbf{证明:}由\autoref{def_ConOp_1} ,
\begin{equation}
\begin{aligned}
A^*(\alpha f+\beta g)&=(\alpha f+\beta g)A\\
&=\alpha fA+\beta gA\\
&=\alpha A^*f+\beta A^*g.
\end{aligned}~
\end{equation}
\begin{equation}
\begin{aligned}
(A+B)^*(f)&=f(A+B)=fA+fB\\
&=A^*f+B^*f=(A^*+B^*)f.
\end{aligned}~
\end{equation}

\begin{equation}
\begin{aligned}
(\alpha A)^*(f)&=f(\alpha A)=\alpha fA=\alpha A^*f.
\end{aligned}~
\end{equation}
其中用到了 $f(\alpha A+\beta B)=\alpha fA+\beta fB$,这是因为
\begin{equation}
\begin{aligned}
f(\alpha A+\beta B)(x)&=f(\alpha A(x)+\beta B(x))\\
&=\alpha fA(x)+\beta fB(x)\\
&=(\alpha fA+\beta fB)(x).
\end{aligned}~
\end{equation}

\textbf{证毕!}



\begin{theorem}{}
若 $A$ 是
\end{theorem}



