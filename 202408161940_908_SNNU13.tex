% 陕西师范大学 2013 年 考研 量子力学
% license Usr
% type Note

\textbf{声明}:“该内容来源于网络公开资料,不保证真实性,如有侵权请联系管理员”

\subsection{填空(每小题3分,共30分)}
\begin{enumerate}
    \item 根据德布罗意假设,对于一定能量 $E$ 和一定动量 $p$ 的粒子,与它相联系的是频率为 $\nu$,波长为 $\lambda$ 的平面波,它们之间的关系为 $E = \underline{\hspace{2cm}}$,$p = \underline{\hspace{2cm}}$。
    
    \item 泡利算符的对易关系是 $\hat \sigma_x \hat \sigma_y - \hat \sigma_y \hat \sigma_x = \underline{\hspace{2cm}}$,泡利算符的反对易关系是 $\hat \sigma_x \hat \sigma_y + \hat \sigma_y \hat \sigma_x = \underline{\hspace{2cm}}$。
    
    \item 氢原子态函数波函数 $\psi_{311} = R_{31}(r) Y_{11}(\theta, \varphi)$ 所描述的态,其轨道角动量的长度为 $\underline{\hspace{2cm}}$,轨道角动量在 z 轴上的投影为 $\underline{\hspace{2cm}}$。
    
    \item 力学量 $F$ 的本征值方程 $\hat F\psi = F\psi$ 中,如果对 $\hat F$ 的一个本征值,有 $f$ 个相互独立(线性无关)的本征函数数,我们就说该本征值 $F$ 是 $\underline{\hspace{2cm}}$ 的,简并度为 $\underline{\hspace{2cm}}$。
    
    \item 在动力学表达式中,动量算符的本征方程表示为 $\underline{\hspace{4cm}}$,坐标算符的本征方程表示为 $\underline{\hspace{4cm}}$。
    
    \item $n$ 个全同粒子组成的体系中,不能有两个或两个以上的粒子处于 $\underline{\hspace{2cm}}$,这就是泡利不相容原理。

    \item 湮灭算符$\hat a$ 作用于谐振子的第 $n$ 个本征态上的结果为 $\hat a\psi_n = \underline{\hspace{3cm}}$。\\
    湮灭算符$\hat a = \left( \frac{\mu \omega}{2\hbar} \right)^{2} \left(\hat x + \frac{i}{\mu \omega} \hat{p} \right)$ 的共轭算符是 $\underline{\hspace{6cm}}$。

    \item 不考虑相对论效应及自旋轨道耦合,氢原子的能级只依赖于 $\underline{\hspace{1cm}}$量子数;考虑相对论效应及自旋轨道耦合后(不考虑实验值修正),氢原子的能级依赖于 $\underline{\hspace{1cm}}$ (填a或b或c)。\\
   (a) $n, l$;  (b) $n, l$;  (c) $n, j$。
    \item 对任意两个波函数 $\psi, \phi$,如算符 $\hat{F}$ 满足下述关系 $\underline{\hspace{7cm}}$ 则算符 $\hat{F}$ 称为厄米算符。

    \item 当粒子能量低于势垒高度时,粒子仍有可能透过势垒,这种现象称为$\underline{\hspace{2cm}}$ 。
\end{enumerate}
\subsection{(20 分)}
设一维线性谐振子在 $t = 0$ 时的状态为
\[
\psi(t=0) = \frac{1}{\sqrt{6}}
\begin{pmatrix}
1 \\
2 \\
0 \\
0 \\
\end{pmatrix}~
\]

\begin{enumerate}
    \item [(1)] 计算 $t = 0$ 时,能量的平均值 $E$。
    
    \item [(2)] 在 $t$ 时刻求 $P(t) = \langle \psi(t) | \hat{P} | \psi(t) \rangle$,已知在能量表象中,
    \[
    \hat{P} = \frac{1}{\sqrt{2}} \left(\frac{\hbar \omega}{2}\right)^{1/2}
    \begin{pmatrix}
    0 & -\sqrt{1} & 0 & 0 & \dots \\
    -\sqrt{1} & 0 & -\sqrt{2} & 0 & \dots \\
    0 & -\sqrt{2} & 0 & -\sqrt{3} & \dots \\
    0 & 0 & -\sqrt{3} & 0 & \dots \\
    \vdots & \vdots & \vdots & \vdots & \ddots \\
    \end{pmatrix}~
    \]
\end{enumerate}
