% 实数集的拓扑

\pentry{实数\upref{ReNum} 上确界与下确界\upref{SupInf}}

\subsection{开集与闭集}

首先给出如下定义.

\begin{definition}{开集与闭集}
设$x$是实数. 任意包含$x$的开区间$U(x)$都称作$x$的一个开邻域 (open neighbourhood). 如果将点$x$挖去, 得到的集合称为$x$的去心邻域 (deleted neighbourhood), 常记为$\mathring U(x)$.

实数集$\mathbb{R}$的子集$U$称为开集 (open set), 如果对于任意$x\in U$, 都存在$x$的开邻域$V(x)$使得$V(x)\subset U$. 

实数集$\mathbb{R}$的子集$C$称为闭集 (closed set), 如果$\mathbb{R}\setminus C$是开集.

规定空集既是开集也是闭集.
\end{definition}

容易证明如下性质:

\begin{theorem}{开集和闭集的运算}
任意多个开集的并集仍然是开集. 有限多个开集的交集仍然是开集.

等价地, 任意多个闭集的交集仍然是闭集. 有限多个闭集的并集仍然是闭集.
\end{theorem}

\begin{exercise}{}
证明这个定理. 提示: 设$\{U_\alpha\}_{\alpha\in A}$是一族开集, 那么若$x\in \cup_{\alpha\in A}U_\alpha$, 则必定有一$\alpha$使得$x\in U_\alpha$. 如果$U_1,...,U_n$是有限多个开集, $x\in\cap_{k=1}^nU_k$, 而$V^k(x)$是$x$的包含在$U^k(x)$中的开邻域, 那么$\cap_{k=1}^nV^k(x)$还是$x$的开邻域.
\end{exercise}

\begin{exercise}{}
在证明"有限多开集的交集还是开集"时, "有限"这个条件究竟被用在哪里? 可以参考下面的反例.
\end{exercise}

\begin{example}{一些反例}
无限多个开集的交集不一定是开集. 例如, 设开区间$U_k=(-1/k,1/k)$, 那么$\cap_{k=1}^\infty=\{0\}$. 相应地, 无限多个闭集的并集也不一定是闭集, 例如, 设闭区间$I_k=[0,1-1/k]$, 则$\cup_{k=1}^\infty I_k=[0,1)$, 它不是开集也不是闭集.
\end{example}

粗略地说, 在一个集合上给定拓扑, 就是给定一个衡量元素之间的"远近关系"的尺度. 在实数集$\mathbb{R}$中, 一个给定的实数$x$的全体开邻域就划定了距离这个实数的"远近关系". 如上定义的开集的全体符合抽象的拓扑的定义; 详见词条拓扑空间\upref{Topol}.

\subsection{开集的结构}
在实数集$\mathbb{R}$中, 开集的结构可以被清楚地刻画出来. 首先引入一个定义: 包含于非空开集$G\subset\mathbb{R}$中的开区间$(a,b)$称为一个分支 (component), 如果端点$a,b\notin G$. 容易看出, 任何非空开集中的两个分支一定不相交.

\begin{theorem}{实数集中开集的结构}
每一个非空开集$G\subset\mathbb{R}$都是至多可数个分支的并集. 
\end{theorem}
\textbf{证明.} 首先注意到任何一点$x\in\mathbb{R}$都一定属于某个分支$U$: 这个分支是所有包含在$G$中且包含$x$的开区间的并集. 为了说明它符合分支的定义, 首先注意到$U$当然是个区间. 进一步, 可以反设, 例如, $U$的左端点$a\in G$; 那么有$a$的开邻域$U(a)\subset G$, 从而$U(a)\cup U$也是包含$x$的区间, 但它严格包含了$U$, 同$U$的定义相违背.

接下来, 按照这个推理, 注意到$G$中的任何有理数$r$都属于某个分支$U_r$. 这些分支或者不相交, 或者重合. 由此, $G$的全体分支被$G$中所包含的有理数所标记, 从而分支的个数一定是至多可数的. \textbf{证毕.}

\subsection{距离, 接触点与闭包}
实数集$\mathbb{R}$是一个度量空间 (metric space). 关于一般的度量空间理论, 详见词条度量空间\upref{Metric}. 在实数集上, 最自然的度量是绝对值函数$d(x,y)=|x-y|$, 它显然满足如下三条性质:

\begin{itemize}
\item $|x-y|=0$当且仅当$x=y$.
\item $|x-y|=|y-z|$.
\item 三角不等式: 对于$x,y,z\in\mathbb{R}$, 总有
$$
|x-z|\leq|x-y|+|y-z|.
$$
\end{itemize}

显然, 开区间$(x_0-\delta,x_0+\delta)$恰好等同于集合
$$
\{x:|x-x_0|<\delta\}.
$$
在这种情况下, 我们说实数集的拓扑是由这个度量诱导得到的.

给定点集$E\subset\mathbb{R}$, 函数
$$
\text{dist}(x;E):=\inf_{y\in E}|x-y|
$$
称为到点集$E$的距离函数 (distance function). 与$E$的距离为零的点称为$E$的接触点 (contact point). $x$是$E$的接触点, 当且仅当$x$的任何邻域都与$E$有非空交集. 实际上, $\inf_{y\in E}|x-y|=0$等价于如下命题: 任何$\delta>0$都不是数集$\{|x-y|:y\in E\}$的下界, 或者换句话说, 对于任何一个$\delta>0$, 都存在$y_\delta\in E$使得$|x-y_\delta|<\delta$. 这恰好等价于"$x$的任何邻域都与$E$有非空交集".

$E$的接触点的全体称为$E$的闭包 (closure), 常常记为$\bar E$. 有如下定理:
\begin{theorem}{}
$E$的闭包是包含$E$的所有闭集之交.
\end{theorem}
实际上, 如果$C$是包含$E$的闭集, 那么$\mathbb{R}\setminus C$是包含于$\mathbb{R}\setminus E$的开集, 因此任意一点$x\in\mathbb{R}\setminus C$都有开邻域$U(x)\subset\mathbb{R}\setminus C$, 而这开邻域显然同$E$不相交. 这表示任意的$x\in\mathbb{R}\setminus C$都不是$E$的接触点, 或者反过来说, $E$的接触点集必然包含于$C$. 

有如下显然的推论:
\begin{corollary}{}
集合$E\subset\mathbb{R}$为闭集, 当且仅当它的闭包等于它自己.
\end{corollary}