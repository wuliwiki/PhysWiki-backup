% 电磁场标势和矢势
% 法拉第电磁感应|麦克斯韦方程组

\begin{issues}
\issueDraft
\end{issues}

\pentry{法拉第电磁感应\upref{FaraEB}, 磁矢势\upref{BvecA}}

用标势和矢势表示电磁场
\begin{equation}\label{EMPot_eq1}
\bvec E = -\grad \varphi - \pdv{\bvec A}{t}~.
\end{equation}
\begin{equation}\label{EMPot_eq2}
\bvec B = \curl \bvec A
\end{equation}

\subsection{推导}
首先定义 $\bvec A$, 则由法拉第电磁感应定律(\autoref{MWEq_eq2}~\upref{MWEq})
\begin{equation}
\curl \qty(\bvec E + \pdv{\bvec A}{t}) = \curl \bvec E + \pdv{\bvec B}{t} = \bvec 0
\end{equation}
这说明括号中的矢量可以表示为一个标量函数的梯度,即标势 $\varphi$, 负号是为了在静电场的情况下使得标势等于电势。

\subsection{标势和矢势的麦克斯韦方程组}

将\autoref{EMPot_eq1} 和\autoref{EMPot_eq2} 代入麦克斯韦方程组可以得到两条与麦克斯韦方程组等效的方程
\begin{equation}\label{EMPot_eq4}
\laplacian \varphi + \pdv{t} (\div \bvec A) = -\frac{\rho}{\epsilon_0}
\end{equation}
\begin{equation}\label{EMPot_eq5}
\qty(\laplacian \bvec A - \mu_0\epsilon_0 \pdv[2]{\bvec A}{t}) - \grad \qty(\div \bvec A + \mu_0\epsilon_0 \pdv{\varphi}{t}) = -\mu_0\bvec J
\end{equation}

