% 矢量空间
% 线性代数|矢量|几何矢量|矢量空间|线性空间|集合|交换律|结合律|分配率|多项式|线性相关|线性无关|基底|n维空间|行矢量|列矢量|子空间|内积

% “抽象”的释义是否需要单独开词条阐释?
% 修改进行中,在取舍重复内容

\pentry{集合\upref{Set}, 几何矢量\upref{GVec}}

矢量的线性组合,使得我们有可能用少量矢量来表示更多的矢量.

给定若干矢量构成的集合$S=\{\bvec{v_\alpha}\}$,可以是无穷集合,那么从$S$中任意地选择有限个矢量进行线性组合,所得到的集合$\{\sum_i^N c_i\bvec{v_i}|N\in\mathbb{Z}, c_i\in\mathbb{R}, \bvec{v}_i\}$称为$S$所\textbf{张成(span)的空间},记为 $\ev{S}$ 或者 $\ev{\{\bvec v_\alpha\}}$.由于$\bvec v_\alpha$的线性组合的线性组合还是$\bvec v_\alpha$的线性组合,因此在这个集合里,任何有限个矢量相加的结果还在集合里.

\begin{example}{线性相关组的表示不唯一}
把实数轴上各点$x$都看成一个向量$\bvec{x}$,起点是$0$,终点是$x$.向量组$\{\bvec{1}, \bvec{2}\}$是线性相关的,因为两个向量可以互相表示:$2\bvec{1}=\bvec{2}$.

对于任意一个向量$\bvec{x}$,如果用这两个向量来表示,那就有无穷多种组合.比如说,$\bvec{3}$可以表示成$3\bvec{1}$,也可以表示成$2\bvec{1}+0.5\bvec{2}$,还可以表示成$1\bvec{1}+\bvec{2}$.
\end{example}


“抽象(abstract)” 一词, 意为 “抽出一个对象的部分特性做研究而忽略其它特性”或者“抽出若干对象的共同特征做研究而抛弃它们的特有特征”,其中“象”取“类比”、“相似”之意.当你在脑海中想象“树”的概念时,只关注树共有的一些关键特质,比如有树叶有枝干有根系等,至于具体多少树叶、根系如何分布则被忽略了,因此抽象的结果往往是不能画出来的,因为画出来的树都有了具体的特征了,不再是抽象的树.抽象的对立面,是具象.abstract一词是由ab-(向外的)和-tract(拉、拔)构成的,词义和抽象完全一样.当我们说一个事物更抽象时,我们其实也在说它更为“一般”或者更“广义”.

\textbf{矢量空间(vector space)} 也叫\textbf{向量空间}或\textbf{线性空间(linear space)},是一种矢量的集合,但不是任意的集合;一个矢量空间必须满足,在其中选择任意两个矢量,它们的线性组合仍然在这个空间中.进行归纳后易得,这个条件等价于“任意有限个矢量的线性组合仍然在这个空间中”.这个性质被称为矢量线性组合的\textbf{封闭性}.这里的“矢量” 是一个广义的概念,是几何矢量的抽象;反过来,几何矢量是矢量的具象.一个矢量,不一定具有长度和方向.

\subsection{定义}
一个\textbf{矢量空间}是满足特定条件的集合, 有无穷多个元素, 每个元素叫做一个\textbf{矢量}. 除此之外,矢量空间的定义还必须依赖一个\textbf{域}\footnote{见词条\upref{field},简单来说,域就是能进行加减乘除的一个集合.},比如实数域和复数域;这个域本身被称为该矢量空间的\textbf{标量域(scalar field)}或\textbf{标域},它的元素被称为矢量空间的\textbf{标量(scalor)},它们不是矢量空间的元素,但是可以用来和矢量进行数乘. 任意矢量空间内必须定义两个矢量的\textbf{加法(addition)}(用 “+” 表示)和标量与矢量之间的\textbf{数乘(scalar multiplication)} 两种运算, 得到的结果也必须在同一空间中. 我们把这样的运算叫做\textbf{闭合(closed)}的. 两种运算的必须满足如下性质, 其中 $\bvec u,\bvec v,\bvec w$ 为空间中任意三个矢量, $a,b$ 为任意两个标量. 若$a,b$的选择范围是域$\mathbb{F}$,则称以下定义的矢量空间是\textbf{域}$\mathbb{F}$\textbf{上的}.通常选择的域就是实数域$\mathbb{R}$和复数域$\mathbb{C}$.

一个矢量空间,就是定义了矢量加法运算和数乘运算的集合,两个运算分别满足以下条件:

\subsubsection{加法运算}
\begin{enumerate}
\item 满足加法交换律 $\bvec u + \bvec v = \bvec v + \bvec u$.
\item 满足加法结合律 $(\bvec u + \bvec v) + \bvec w = \bvec u + (\bvec v + \bvec w)$.
\item 存在零矢量,使得 $\bvec v + \bvec 0 = \bvec v$.
\item 空间中任何矢量 $\bvec v$ 存在逆矢量 $-\bvec v$,使得 $\bvec v + (-\bvec v) = \bvec 0$.
\end{enumerate}

\subsubsection{数乘运算}
\begin{enumerate}
\item 乘法分配律 $a(\bvec u + \bvec v) = a\bvec u + a\bvec v$ 
\item 乘法分配律 $(a + b)\bvec v = a\bvec v + b\bvec v$
\item 乘法结合律 $a (b \bvec v) = (ab) \bvec v$
\end{enumerate}

作为一个非几何矢量的例子, 我们来看由多项式构成矢量空间.

\begin{example}{多项式}\label{LSpace_ex1}
所有不大于 $n$ 阶的多项式 $c_n x^n + c_{n-1} x^{n-1} + \dots + c_1 x + c_0$ 可以构成一个实数矢量空间或复数矢量空间.定义矢量加法为两多项式相加, 满足
\begin{itemize}
\item 封闭性:任意两个不大于 $n$ 阶的多项式相加仍然为不大于 $n$ 阶的多项式.
\item 交换律:多项式相加显然满足交换律.
\item 零矢量:常数 0 可以看做一个 0 阶多项式, 任何多项式与之相加都不改变.
\item 逆矢量:把任意多项式乘以 $-1$ 就得到它的逆矢量, 任意多项式与其逆矢量相加等于 0.
\end{itemize}
定义矢量数乘为多项式乘以常数, 显然也满足数乘的各项要求, 不再赘述.
\end{example}

\begin{exercise}{几何矢量}
证明 1,2,3 维空间中的所有几何矢量各自构成一个实数矢量空间.
\end{exercise}

另一个极为重要的矢量空间,是\textbf{函数空间}.

\begin{example}{函数空间}
实数到实数的全体函数的集合$F$构成一个线性空间,称为$\mathbb{R}$\textbf{函数空间}.函数空间中两个向量的加法定义为,对于任何实数$x$和函数(即向量)$f, g\in F$,有$(f+g)(x)=f(x)+g(x)$;数乘定义为,对于任何实数$a, x$和函数$f\in F$,有$(af)(x)=af(x)$.

类似地,复数到复数、实数到复数等的函数都可以构成线性空间;把函数限制在连续函数、可导函数等条件下也依然构成线性空间.特别地,复数域上的归一化可导函数,构成了复数域上的希尔伯特空间,这是一种无穷维的特殊“线性空间”,是量子力学的基础概念,我们将会在将来详细讨论.
\end{example}

注意矢量空间的定义并不需要包含内积(点乘) 的概念, 但我们可以在其基础上额外定义内积, 这样的空间叫做\textbf{内积空间}\upref{InerPd}, 留到以后介绍. 除了内积, 我们可以把 “几何矢量\upref{LSpace}” 中介绍的概念都拓展到一般的矢量空间中.

而展开系数就是\textbf{坐标}(再次注意, 复数矢量空间中, 矢量的坐标可以是复数).
\begin{equation}
\bvec v = \sum_{i=1}^N c_i \bvec \beta_i
\end{equation}

\begin{exercise}{}
三维几何矢量空间中, 建立直角坐标系, 基底为 $\uvec x, \uvec y, \uvec z$. 请证明直角坐标(即关于基底 $\uvec x, \uvec y, \uvec z$ 的坐标)为 $(2, 1, 1)$, $(1, 3, 1)$, $(1, 1, 4)$ 的三个矢量线性无关, 并用这三个矢量作为基底, 求直角坐标为 $(1, 1, 1)$ 的矢量关于这组基底的坐标.
\end{exercise}

\begin{exercise}{复数列矢量}
我们把 $N$ 个复数 $c_1, \dots, c_N$ 按顺序排成一列(或一行, 下同), 叫做\textbf{列矢量}(\textbf{行矢量}, 下同). 给它们定义通常意义的加法和数乘运算, 这样所有列矢量可以构成一个 $N$ 维矢量空间. 注意由于我们使用了复数, 即使 $N \leqslant 3$ 时我们也无法将这些矢量与几何矢量对应起来.

如果我们将基底取为\footnote{上标 $\mathrm T$ 表示转置, 这里是为了排版方便} $(1, 0, \dots, 0)\Tr$, $(0, 1, 0, \dots, 0)\Tr$, …, $(0, \dots, 0, 1)\Tr$, 那么显然任意列矢量 $(c_1, \dots, c_N)\Tr$ 的坐标就是 $c_1, \dots, c_N$. 但我们也可以取其他基底, 这时坐标就会改变.
\end{exercise}

\begin{exercise}{}
证明\autoref{LSpace_ex1} 中多项式空间是 $n+1$ 维空间, $x^k$ ($k = 0, \dots, n$) 是一组基底(提示: 证明它们线性无关, 可以表示空间中的任意矢量).
\end{exercise}
