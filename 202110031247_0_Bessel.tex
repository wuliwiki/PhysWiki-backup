% 贝塞尔函数
% keys 贝塞尔方程|贝塞尔函数|汉克尔函数|修正贝塞尔函数

\pentry{二阶齐次变系数线性微分方程的幂级数解法\upref{ODE2P}}

\textbf{贝塞尔微分方程(Bessel's differential equation)}也被称为\textbf{柱谐函数}、\textbf{圆柱函数}或\textbf{圆柱谐波}, 通常在使用分离变量法求解柱坐标中的拉普拉斯方程\upref{CylLap}时得到:
\begin{equation}\label{Bessel_eq1}
x\dv{x}\qty(x\dv{y}{x}) + (x^2 - l ^2)y = 0
\end{equation}
其中 $l$ 叫做\textbf{阶数(order)},一般来说可以是任意实数,但半整数和整数较为常见. 两个线性无关的解分别是\textbf{第一类贝塞尔函数(Bessel function of the first kind)} $J_l(x)$ 和\textbf{第二类贝塞尔函数(Bessel function of the second kind)} $Y_l(x)$. 这里只讨论 $x > 0$ 且 $l$ 为整数或半整数的情况\footnote{更详细的讨论请参见\textbf{二阶齐次变系数线性微分方程的幂级数解法}\upref{ODE2P}词条中的\autoref{ODE2P_ex4}~\upref{ODE2P}到\autoref{ODE2P_ex5}~\upref{ODE2P}部分.}.

\begin{figure}[ht]
\centering
\includegraphics[width=13cm]{./figures/Bessel_1.pdf}
\caption{第一类和第二类贝塞尔函数(非负整数阶)} \label{Bessel_fig1}
\end{figure}

$J_l(x)$ 的级数形式为
\begin{equation}\label{Bessel_eq3}
J_l(x) = \sum_{m=0}^\infty \frac{(-1)^m}{m!\Gamma(m+l+1)} \qty(\frac x2)^{2m+l}
\end{equation}
其中使用了 $\Gamma$ 函数\upref{Gamma}. $Y_l(x)$ 可以通过 $J_l(x)$ 来定义
\begin{equation}
Y_l(x) = \frac{J_l(x)\cos(l\pi) - J_{-l}(x)}{\sin(l\pi)}
\end{equation}
容易证明
\begin{equation}
J_{1/2}(x) = \sqrt{\frac{2}{\pi x}} \sin x = -Y_{-1/2}(x)
\end{equation}
\begin{equation}
J_{-1/2}(x) = \sqrt{\frac{2}{\pi x}} \cos x = -Y_{1/2}(x)
\end{equation}

\subsection{变量替换}
令 $u(x) = \sqrt{x} y(x)$, 对\autoref{Bessel_eq1} 替换变量可得更简洁的方程, 使一阶导数项消失
\begin{equation}
x^2\dv[2]{u}{x} + \qty(x^2 - l^2 + \frac{1}{4}) u = 0
\end{equation}
显然 $u(x)$ 的两组解就是 $\sqrt{x}J_l(x)$ 和 $\sqrt{x}Y_l(x)$.

\subsection{汉克尔函数}
贝塞尔方程的两个线性无关解也可以用第一类和第二类汉克尔函数来表示\footnote{可类比欧拉公式 $\exp(\pm\I x) = \cos x \pm \I\sin x$ (\autoref{CExp_eq2}~\upref{CExp}).}
\begin{equation}
H_l ^{(1)}(x) = J_l(x) + \I Y_l(x)
\qquad
H_l ^{(2)}(x) = J_l(x) - \I Y_l(x)
\end{equation}

\subsection{常用性质}
令 $Z$ 为 $J, Y, H^{(1)}, H^{(2)}$ 的任意一种, 则
\begin{equation}
Z_{-l}(x) = (-1)^l Z_l(x)
\end{equation}
递推关系
\begin{equation}
\frac{2l}{x} Z_l(x) = Z_{l -1}(x) + Z_{l+1}(x)
\end{equation}
一阶导数
\begin{equation}
\dv{Z_l}{x} = \frac12 [Z_{l  - 1}(x) - Z_{l +1}(x)]
\end{equation}
渐进形式($x \gg 1$)
\begin{equation}
J_l(x) \to \sqrt{\frac{2}{\pi x}} \cos\qty(x - \frac{l\pi}{2} - \frac{\pi}{4})
\end{equation}
\begin{equation}
Y_l(x) \to \sqrt{\frac{2}{\pi x}} \sin\qty(x - \frac{l\pi}{2} - \frac{\pi}{4})
\end{equation}
正交关系
\begin{equation}
\int_0^1 J_l (u_{l ,m} x) J_l (u_{l ,n} x) x \dd{x} = \frac{\delta_{m,n}}{2}[J_{l + 1} (u_{l ,m})]^2
\end{equation}
其中 $u_{l, m}$ 是 $J_l(x)$ 的第 $m$ 个根.

连续正交关系(容易从渐进形式得到)
\begin{equation}
\int_0^\infty \sqrt{k_1} J_l (k_1 x) \sqrt{k_2}J_l (k_2 x) x \dd{x} = \delta(k_2 - k_1)
\end{equation}

\subsection{修正贝塞尔函数}
修正贝塞尔方程为
\begin{equation}
x^2 \dv[2]{y}{x} + x \dv{y}{x} - (x^2 + l ^2)y = 0
\end{equation}
其解为两个\textbf{修正贝塞尔函数(Modified Bessel Function)},第一类为 $I_l(x)$,  第二类为 $K_l(x)$,  与贝塞尔函数的关系为
\begin{equation}\label{Bessel_eq2}
I_l(x) = \I^{-l} J_l(\I x)
\qquad
K_l(x) = \frac{\pi}{2} \I^{l  + 1} H_l ^{(1)}(\I x)
\end{equation}

\autoref{ODE2P_ex1}~\upref{ODE2P} \autoref{ODE2P_ex2}~\upref{ODE2P} \autoref{ODE2P_ex4}~\upref{ODE2P} \autoref{ODE2P_ex5}~\upref{ODE2P}
