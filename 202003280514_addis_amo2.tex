% AMO Seminar 2

\begin{equation}
\I \pdv{t}\Psi(r_1, r_2, t) = (H_1 + H_2 + V_{12} + H_{F1} + H_{F2})\Psi(r_1, r_2, t)
\end{equation}
\begin{equation}
V_{12} = \frac{1}{\abs{\bvec r_2 - \bvec r_1}}
\end{equation}
\begin{equation}
H_{Fi} = \bvec E(t) \vdot \bvec r_i
\end{equation}

\begin{equation}
H = H_f + H_{12}
\end{equation}

\begin{equation}
\exp(-\I H \Delta t) = \exp(-\I H_0 \Delta t/2) \exp(-\I H_{int}\Delta t) \exp(-\I H_0 \Delta t/2) + \order{\Delta t^3}
\end{equation}

No error for
\begin{equation}
\exp(-\I H_0 \frac{\Delta t}{2}) = \exp(-\I H_{1} \frac{\Delta t}{2}) \exp(-\I H_2 \frac{\Delta t}{2})
\end{equation}

\begin{equation}
H_{int} = V_{12} + F_1 + F_2
\end{equation}

\begin{equation}
\exp(-\I H\Delta t) = \exp(-\I H_{1} \frac{\Delta t}{2}) \exp(-\I H_2 \frac{\Delta t}{2}) \exp\qty[-\I (V_{12} + F_1 + F_2) \frac{\Delta t}{2}] \exp(-\I H_{1} \frac{\Delta t}{2}) \exp(-\I H_{2} \frac{\Delta t}{2}) + \order{\Delta t^3}
\end{equation}

\begin{equation}
\psi_{l_1,l_2}^{L,M}(r_1, r_2, t) = \Sigma_{i,j} c_{i,j}(t) u_i(r_1) u_j(r_2)
\end{equation}

\begin{equation}
u_i(r_{i'}) = 0 \qquad (i \ne i')
\end{equation}

Orthornormal basis
\begin{equation}
\int u_i(r)u_j(r) \dd{r} = \delta_{i,j}
\end{equation}

$\exp(-\I H_{int}\Delta t)$

$90\Si{eV}, 10^{14} \Si{W/cm^2}, 1\Si{fs}$

$\vec k$

$\Re [\psi_{\bvec k}(\bvec r)]$

$\theta_1 \theta_2$

$10^{-7}$

$10^{-11}$

$M = 0, \quad\max{l_1} = 5, \quad\max{l_2} = 8, \quad\max{L} = 5$

$\Delta k = 0$

$\Psi(\bvec r_1, \bvec r_2, t) = \Psi(\bvec r_2, \bvec r_1, t)$

Single ionization analysis
\begin{equation}
\phi_{n_1, l_1, m_1}(\bvec r) = R_{n_1,l_1}(r)Y_{l_1,m_1}(\uvec r)
\end{equation}

\begin{equation}
\phi_{l_2, m_2}(k, \bvec r) = \frac{1}{r}\sqrt{\frac{2}{\pi}} F_{l_2}(k, r) Y_{l_2, m_2} (\uvec r)
\end{equation}

\begin{equation}
\phi_{n_1, l_1, m_1, l_2, m_2}(\bvec r_1, \bvec r_2) = \frac{1}{\sqrt{2}} [\phi_{n_1, l_1, m_1}(\bvec r_1)\phi_{l_2, m_2}(k_2, \bvec r_2) + \phi_{n_1, l_1, m_1}(\bvec r_2)\phi_{l_2, m_2}(k_2, \bvec r_1)]
\end{equation}

\begin{equation}
t = [230, 340] \Si{a.u.}
\end{equation}

\begin{equation}
r_2 = [0,400] \Si{a.u.}
\end{equation}

\begin{equation}
P(\bvec r_2)
\end{equation}

\begin{equation}
a_2^3
\end{equation}

\begin{equation}
f(\theta_2) = \int P(\bvec k_1, \bvec k_2) k_2^2 \dd k_2
\end{equation}

\begin{equation}
P(\bvec r_2) = 2\int \abs{\Psi(\bvec r_1, \bvec r_2)}^2 \dd[3]{r_1}
\end{equation}

\begin{equation}
\tau_{tot} = \tau_{EWS} + \tau_{CLC} + \tau_{ISLC}
\end{equation}


\begin{equation}
E(t) = A_{ir} \sin(\omega_{ir} t) \cos^2\qty(\frac{\pi t}{\Delta t_{ir}})
+ A_{xuv} \sin(\omega_{xuv} t) \cos^2\qty[\frac{\pi(t-\tau)}{\Delta t_{xuv}}]
\end{equation}
