% 经典力学笔记(科普)

\begin{issues}
\issueDraft
\end{issues}

\begin{itemize}
\item \textbf{经典力学}一般是指\textbf{牛顿力学}, 但在一些较现代的文献中也包含相对论。
\item 经典力学中物体受力后的运动规律一般由牛顿三定律描述。
\item 牛顿定律不讨论物体受到的力是如何产生的。
\item 物理中, \textbf{粒子}也叫\textbf{质点}, 是指在当前问题的一般尺度下大小可以忽略不记的物体。 我们假设它们不存在转动, 只存在平移。
\item 物理中, 在研究物体的运动规律时往往先研究粒子的运动规律, 再把所有物体都看作由许多粒子组成的系统。
\item 牛顿三定律本身描述的对象只是质点。 经过一些推论才可以把它拓展到一般的物体。
\item \textbf{几何矢量}的两个要素: 长度和方向(不包含起始位置)。
\item 质点的\textbf{位置矢量}、 \textbf{位移}、 \textbf{速度}、 \textbf{加速度}、 以及受到的每个\textbf{力}都是矢量。
\item 几何矢量的加法: 位移、速度的叠加。
\item 一个质点可能同时受若干个力。 力是矢量, 可以通过矢量加法合成一个等效的合力。 二者对质点运动的效果是一样的。
\item \textbf{牛顿第一定律}:质点合力为零时, 静止或者匀速直线运动。
\item \textbf{牛顿第二定律}:质点的加速度与合力成正比, 和它的质量成反比。
\item \textbf{牛顿第三定律}:两个质点间的相互作用力大小相等, 方向相反。
\end{itemize}
