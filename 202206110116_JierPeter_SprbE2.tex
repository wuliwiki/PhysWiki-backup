% 可分元素的单扩张是可分扩张
% 可分扩张|分裂域|域单同态|域嵌入

\pentry{待定}

\subsection{定理的描述}

本词条是专门用来证明下述\autoref{SprbE2_the1} 的,其自然语言的描述即是本词条的标题.

\begin{theorem}{}\label{SprbE2_the1}
域$\mathbb{F}$上的不可约可分多项式$f(x)$的分裂域$\mathbb{K}$是$\mathbb{F}$的可分扩张.
\end{theorem}

考虑\autoref{SpltFd_cor2}~\upref{SpltFd},以及可分扩张的定义(所有元素都是可分元),我们还可以得到上述\autoref{SprbE2_the1} 的等价描述:

\begin{theorem}{}\label{SprbE2_the2}
$\mathbb{F}(a)/\mathbb{F}$是可分扩张$\iff$ $a$是$\mathbb{F}$的可分元素.
\end{theorem}

本节我们就要证明\autoref{SprbE2_the2} .




\subsection{定理的证明\footnote{该证明的思路取自University of Connecticut的Keith Conrad教授的讲义,但细节思路会有不同.这是因为本书有自己的逻辑体系,会大量引用小时百科中的内容来完成证明.}}
%证明思路来源:https://kconrad.math.uconn.edu/blurbs/galoistheory/separable1.pdf

\begin{lemma}{}\label{SprbE2_lem1}
设$\mathbb{L}/\mathbb{K}$是一个域扩张,且$[\mathbb{L}:\mathbb{K}]=n$.设$\sigma:\mathbb{K}\to\mathbb{F}$是一个域单同态.

则:

1. $\sigma$开拓而得的域单同态$\mathbb{L}\to\mathbb{F}$的数量\textbf{小于等于}$n$.

2. 如果$\mathbb{L}/\mathbb{K}$是\textbf{不可分}扩张,则$\sigma$开拓而得的域单同态$\mathbb{L}\to\mathbb{F}$的数量\textbf{小于}$n$.

3. 如果$\mathbb{L}/\mathbb{K}$是\textbf{可分}扩张,则存在扩域$\mathbb{F}'/\mathbb{F}$,使得$\sigma$开拓而得的域单同态$\mathbb{L}\to\mathbb{F}'$的数量\textbf{等于}$n$.

\end{lemma}




\subsubsection{\autoref{SprbE2_lem1} 中1. 的证明}

% 我们用数学归纳法来处理.

% 显然,当$n=1$时,$\mathbb{L}=\mathbb{K}$,定理成立.下设$n>1$,且对于任意域扩张$\mathbb{L}'/\mathbb{K}$,只要扩张次数小于$n$则定理成立.

注意,单同态就是定义域和象的同构.

取$a\in\mathbb{L}-\mathbb{K}$,$a$在$\mathbb{K}$上的最小多项式是$f$.则$f(x)\in\mathbb{K}[x]$的分裂域是$\mathbb{K}(a)\subseteq\mathbb{L}$.考虑\autoref{SpltFd_the3}~\upref{SpltFd},可知$\sigma$开拓为$\mathbb{K}(a)$到$\mathbb{F}$的同态后,同态象最多只有一种可能.也就是说,$\mathbb{F}$中最多只有一个子域同构于$\mathbb{K}(a)$.

于是,如果$\sigma$开拓为$\mathbb{K}(a)\to\mathbb{F}$的同态存在,那么每个不同的开拓都对应一个$\mathbb{K}(a)$(或者$\sigma(\mathbb{K}(a))$)到自身的保$\mathbb{K}$(或者保$\sigma(\mathbb{K})$)自同构.

又据\autoref{SpltFd_the1}~\upref{SpltFd},可知由$\sigma$开拓而来的域同态$\sigma:\mathbb{K}(a)\to\mathbb{F}$最多只有$[\mathbb{K}(a):\mathbb{K}]=\opn{deg}f$个.

上述讨论说明,定理对单扩张情况成立.再考虑\autoref{FldExp_the3}~\upref{FldExp}和\textbf{中间域升链}\autoref{FldExp_cor3}~\upref{FldExp},则得证.



\subsubsection{\autoref{SprbE2_lem1} 中2. 的证明}

由\textbf{可分扩张的传递性}%\addTODO{等写完本原元素与单代数扩张的这个定理后引用.}
,再据\textbf{中间域升链}\autoref{FldExp_cor3}~\upref{FldExp},可知$\mathbb{L}/\mathbb{K}$的中间域中,必有相邻的两个构成的扩张,比如记为$\mathbb{K}_2/\mathbb{K}_1$,是\textbf{不可分单扩张}.

不可分多项式的根的数目,小于其次数.但是据\autoref{FldExp_the1}~\upref{FldExp},$[\mathbb{K}_2:\mathbb{K}_1]$正等于其次数.因此$\sigma$在一步步开拓的过程中,到了$\mathbb{K}_2/\mathbb{K}_1$这一步,开拓出的同态数目就要小于扩张次数.

由此得证.


\subsubsection{\autoref{SprbE2_lem1} 中3. 的证明}

同上一条的证明,可知$\mathbb{L}/\mathbb{K}$的中间域中,任意相邻两个域之间都是\textbf{单可分}扩域的关系.

也就是说,每一步扩张都是一个\textbf{无重根的不可约多项式}的分裂域,因此据\autoref{SpltFd_the1}~\upref{SpltFd},每一步中$\sigma$的开拓的数目都恰为扩张的次数.

除了一种情况:那就是某一步分裂域扩域无法映射入$\mathbb{F}$,那就在$\mathbb{F}$上取这个分裂域的多项式在$\mathbb{F}$上的映射的分裂域,即可\footnote{这么说很绕口.但如果采用“同构就是同一个”和“单同态就是定义域和象的同构”的理解,就会直白得多.}.这也是为什么定理中会说存在一个扩域$\mathbb{F}'/\mathbb{F}$.




\subsubsection{正式证明\autoref{SprbE2_the2} }

注意,\autoref{SprbE2_lem1} 的证明中,只出现了“可分单扩张”和“不可分单扩张”,但没有说\textbf{不可分单扩张}所用的元素是不是可分元素(即下面证明要讨论的),所以没有构成循环论证.

$\Rightarrow$:

由可分扩张的定义,显然.

$\Leftarrow$:

设$a$是域$\mathbb{F}$的可分代数元,其在$\mathbb{F}$上的最小多项式是$\opn{irr}(a, \mathbb{F})=f(x)\in\mathbb{F}[x]$.


















