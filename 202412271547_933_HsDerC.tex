% 导数的性质与构造(高中)
% keys 导数|性质|构造|恒等
% license Usr
% type Tutor

\begin{issues}
\issueDraft
\end{issues}

\pentry{导数\nref{nod_HsDerv},函数的性质\nref{nod_HsFunC}}{nod_139b}

\subsection{近似代替}

在导数的\aref{几何含义}{sub_HsDerv_1}中就提到过“以直代曲”。

\begin{equation}
f(x_0+\Delta x)\approx f(x_0)+f'(x_0)\Delta x~.
\end{equation}

\subsection{单调性和极值点}
\subsubsection{单调性}
在介绍\aref{导函数}{sub_HsDerv_2}时,提及区间的中函数的增减与导函数的符号相关。
\begin{theorem}{单调性与导数的关系}
导函数  $f'(x)$  代表了原函数  f(x)  图像在每一点的切线斜率。
\begin{itemize}
\item 在$f'(x)>0$的区间上,原函数的图像单调递增。
\item 在$f'(x)<0$的区间上,原函数的图像单调递减。
\item 在$f'(x)=0$的区间上,原函数的图像是水平的。
\end{itemize}
\end{theorem}


\subsubsection{极值点}

$f'(x) = 0$的点表示函数的输出值停止增加或减少的点,被称为\textbf{驻点}。在该点是水平的,可能是极值点。

严格的来说,在点x0某一邻域内,f(x0)>=f(x)或f(x0)<=f(x),x0才是极值点。
f’(x)=0的点并不一定是极值点,在该点附近f’(x)必须要变号,如y=x^3,在x=0处一阶导数为0,但两侧不变号,就不是极值点
极值点与该点的一阶导数是否存在无关,只要两侧的f’(x)变号,就是极值点,如y=|x|,在x=0处就是极值点,但在x=0处一阶导数不存在。

极值点

\subsection{高阶导数}

导函数作为原函数,则又可以求得它的导函数,这也被称为高阶导数。

凹凸性

连续曲线的凹弧与凸弧的交界点。(在该点的二阶导并不一定有定义)
f"(x)=0,且该点 两侧 二阶导数变号,那么该点就是极值点。
当然,可能在一点x0处,二阶导数并不存在,在x0左侧的二阶导数趋于正无穷,右侧的二阶导数趋于负无穷,该点也是拐点。

\subsection{常用构造}

\subsubsection{逆向使用求导法则}

逆用乘法公式:
\begin{enumerate}
\item 
\item $f'(x) + f(x) \geq 0$,构造 $F(x) = \E^x f(x)$,$[e^x f(x)]' = e^x [f'(x) + f(x)]$
\item $x^n f'(x) + f(x) \geq 0$,构造 $F(x) = x^n f(x)$,$[x^n f(x)] = x^n f'(x) + nx^{n-1} f(x) = x^{n-1} [x f'(x) + nf(x)]$。特别地,$n=1$时x f'(x) + f(x) \geq 0$,构造 $F(x) = x f(x)$,$[x f(x)]' = x f'(x) + f(x)$
\item $f'(x) + k f(x) \geq 0$,构造 $F(x) = \E^{kx} f(x)$,$[e^{kx} f(x)]' = e^{kx} [f'(x) + kf(x)]$
\end{enumerate}

(3)(4)为(1)(2)的一般形式。