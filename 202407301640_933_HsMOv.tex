% 【导航】高中数学
% keys 高中|数学|概述
% license Xiao
% type Map

\begin{issues}
\issueDraft
\end{issues}

% Giacomo:首先要确定目标
小时百科面向的主要群体是对物理、数学、计算机等感兴趣或在科研上存在需求的本科生及研究生,但事实上,有很多的知识是在高中阶段(甚至小学或初中)就有涉及的。有很多的读者反映,当前高中数学学习过程中存在一些问题或困惑,比如:
\begin{itemize}
\item 由于教学目标、进度或其他原因导致的,学生不能深入或透彻理解当前教材或课堂上的教学内容,最后变成“听不懂、记不住、做不会”;
\item 高考改革,不仅要求在当前缩减的大纲下保证知识体系的构建和联通,还需要能够快速理解一些陌生概念与已有概念的关系。这要求学生在高中阶段的学习中:一方面,提前去接触一些陌生的新的概念,避免畏惧;另一方面,知识体系稳固,分析、理解新概念与已有知识的关系;
\item 由于课纲修改造成的部分重要知识的调整,使得尽管高中自成一派,但与大学的知识衔接不上,造成本科时学习的障碍。很多知识如果没有高中阶段没有接触过,产生一些比较直观的理解或印象,那么,在后期的深入接触中,由于知识体系存在缺位,导致可能会存在一些理解上的障碍。
\end{itemize}

因此,这部分的主要目标包括三点:
\begin{enumerate}
\item 应对高中身份,在语言、记号、图片、例题上给予更多的形象解释;
\item 立足高中内容,参考高中数学课本,力求建立稳固的知识体系;
\item 增加一定的深度,尽可能展现一些当前高中教材中不涉及的,但又对高中学生理解而言不过于艰难的内容。
\end{enumerate}
以期帮助高中学生更好的学习备考

,祝各位备考顺利!

\subsection{数列与函数}

\subsection{三角函数}

\subsection{排列、组合和概率与统计}

\subsection{解方程}

\subsection{几何向量}

线性代数的研究对象是向量和矩阵,而我们最早认识的向量就是\textbf{几何向量},这里我们回顾几何向量的相关概念。

几何向量的存在与坐标系无关,它是一些有长度有方向的箭头。我们把(二维)平面中的向量称为平面向量,(三维)空间中的向量称为空间向量;在高中数学的语境下,我们把(一维)直线上的向量称为标量,但这是不严谨的。

% 对于讨论问题的不同,我们有时仅需要处于同一平面(\textbf{二维空间})的所有几何向量,有时需要\textbf{三维空间}中的所有几何向量,最简单的情况下只需要沿某条线(\textbf{一维空间})的所有几何向量(这时我们可以规定一个正方向,且仅使用几何向量的模长加正负号来表示几何向量以简化书写)。

\addTODO{高中数学中平面向量和空间向量的链接}
% Giacomo:是不是应该把高中数学/物理,改成中学数学/物理?

几何向量有起点(箭尾)和终点(箭头),但我们对几何向量的绝对位置不感兴趣,我们只在乎起点和终点的相对位置,即两个几何向量如果有相同的方向和长度就被视为同一个向量。

\enref{几何向量的一些基本运算}{GVec} 同样不需要有任何坐标系的概念,\textbf{几何向量相加}按照三角形法则或平行四边形法则即可。
\textbf{几何向量数乘}就是把几何向量的模长乘以一个实数,若乘以正数,方向不变,若乘以负数,取相反方向。 \textbf{几何向量的线性组合}是把若干几何向量分别乘以一个实数再相加得到新的几何向量。

几何向量的\enref{内积}{Dot}等于一个几何向量在另一个几何向量上的投影长度乘以另一个几何向量的模长得到一个实数,几何向量的\textbf{模长}等于几何向量与自身内积再开方,把几何向量除以自身模长使模长变为单位长度的过程叫做\textbf{归一化}。若两几何向量内积为零,这两个几何向量相互\textbf{正交}\footnote{对于几何向量,正交就是方向垂直,不加区分。}。

三维欧几里得空间中,两几何向量\enref{叉乘}{Cross}得到的几何向量垂直于两几何向量,模长为一个几何向量在另一个几何向量垂直方向的投影长度乘以另一个几何向量的模长。

为了方便描述几何向量之间的关系,我们选取一些\textbf{线性无关}的几何向量作为所有几何向量的\textbf{基底},使空间中的任何几何向量可以用这些基底的唯一一种线性组合来表示,$N$ 维空间需要 $N$ 个基底向量。一般来说,基底不必互相正交。我们先把这些基底排序,任意几何向量表示成它们的线性组合时,把式中的 $N$ 个系数按照顺序排列,就是该几何向量的\textbf{坐标},通常用列几何向量表示。由于线性组合的唯一性,每个几何向量的坐标是唯一的。

为了方便计算任意几何向量的坐标,往往取\enref{正交归一的基底}{OrNrB}(所有基底模长为1,任意两基底互相正交)。这样,任意向量的坐标都可以通过与基底的内积得到。

\addTODO{添加相关文章的链接}

\subsubsection{几何向量的线性变换}
\addTODO{科普版本的线性映射}

我们可以设计一种规则把某个空间的任意几何向量\textbf{变换}(\textbf{映射})到另一个空间的几何向量;如果任意几何向量线性组合的变换等于这些几何向量分别变换再线性组合,这个变换就被称为\enref{线性变换}{LTrans}。

\addTODO{具体例子旋转变换}



\addTODO{链接待处理,文本待处理}

\subsection{平面几何与立体几何}
