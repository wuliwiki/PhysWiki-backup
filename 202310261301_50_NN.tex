% 神经网络
% keys 神经网络 人工神经网络
% license Xiao
% type Tutor

\textbf{神经网络}(Neural network, NN),准确地说,是\textbf{人工神经网络}(Artificial neural network, ANN),在机器学习领域中是指“由具有适应性的简单单元组成的广泛并行互联的网络,其组织能够模拟生物神经系统对真实世界物体所做出的交互反应”[1]。

神经网络是机器学习中广泛使用的一种基本方法。该方法具有较好的曲线拟合能力,能够从数据中学习离散型、连续型或者向量型函数。

\subsection{动机}

神经网络最初是受到生物神经系统结构的启发,而提出的机器学习模型。生物的神经系统,比如人脑,从结构上讲,是由大量的基本单位——神经元通过各种复杂的互相连接而构成。从功能的角度,神经系统中的每个神经元都可以接收别的神经元传来的信号,然后做出处理,将处理后的结果,通过信号发送给其它与之连接的神经元。大量神经元能够同时协调工作,从而使得整个神经系统具有对各种环境刺激做出反应的能力,即\textbf{智能}(Intelligence)。

由此,人们受到启发,如果能够模拟生物神经系统的结构,并赋予其类似的信息传送和处理机制,则可以定义一个能够具有一定智能的数学模型。值得注意的是,虽然人工神经网络最初的想法是受到生物学的启发,但是在其后续实际研究过程中,站在计算机科学家的角度上来说,并不追求在每一个细节上都模拟生物神经系统。例如,人工神经元输出单一不变的值,然而生物神经元输出的是复杂的时序脉冲[2]。

\subsection{基本结构}

\subsubsection{1.神经元}

生物神经系统的基本单位是神经元,能够接收、处理和发送信号。人们将生物神经元抽象出一个简单模型,即\textbf{人工神经元}(Artificial neuraon),在机器学习领域内,通常就称\textbf{神经元}(Neuron)。在该模型中,神经元接收到其它多个神经元传来的输入信号,这些输入信号通过带有权重的连接进行传递,神经元接收到的总输入值将与神经元的阈值进行比较,然后通过\textbf{激活函数}\upref{ActFun}处理以产生神经元的输出[3]。
\begin{figure}[ht]
\centering
\includegraphics[width=10cm]{./figures/4181cdc1351396c0.png}
\caption{神经元示意图} \label{fig_NN_1}
\end{figure}
神经元的基本结构如图1所示。其中,$x_1, x_2, ..., x_i, ..., x_n$表示神经元的输入,$w_1, w_2, ..., w_i, ..., w_n$表示每个输入所对应的权重,$g$为激活函数,$w_0$为偏移量,$y$为神经元的输出值。输出和输入的关系是
\begin{equation}
y=g(w_1x_1+w_2x_2+...+w_ix_i+...+w_nx_n+w_0).~
\end{equation}
也可以写成向量形式:
\begin{equation}
y=g(\bvec w \bvec x).~
\end{equation}
其中,$\bvec w=(w_0, w_1, w_2, ..., w_i, ..., w_n)$,$\bvec x=(x_0, x_1, x_2, ..., x_i, ..., x_n)$,$x_0=1$.


\subsubsection{2.感知机}
\textbf{感知机}(Perceptron)是有两层神经元构成的最简单的神经网络。其结构中主要包含输入层和输出层。输入层能够接收外界传送来的输入信号,并传递给输出层。输出层就是一个神经元,功能是接收来自输入层传递来的信号,然后做出处理,并输出结果。


感知机能够表示所有的原子布尔函数:与(AND)、或(OR)、与非(NAND)、或非(NOR)[2]。



\subsubsection{参考文献}
\begin{enumerate}
\item T. Kohonen, “An introduction to neural computing,” Neural Networks, vol. 1, no. 1, pp. 3–16, 1988.
\item T. M. Mitchell, Machine learning. 1997.
\item 周志华. 机器学习[M]. 北京:清华大学出版社. 2016: 97
\end{enumerate}