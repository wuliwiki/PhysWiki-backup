% B 树

\begin{issues}
\issueDraft
\end{issues}

\pentry{二叉树\upref{tree}}

\textbf{B-树(B-tree)}, 是一种常见的自平衡树数据结构, 它允许进行高效的插入、删除和搜索操作。 它常用于数据库和文件系统,因是因为 B-树可以最小化数据的读写操作。

B-树与二叉搜索树的不同之处在于,它每个节点可以有两个以上的子节点,并针对读取和写入大块数据的系统进行了优化。 “平衡” 一词指的是所有叶节点都在同一深度的特性,这确保了低高度,从而能快速执行操作。

\subsubsection{B-树结构}
一个\textbf{阶数(degree 或 order)}为 $m$ 的 B-树具有以下特性:
\begin{itemize}
\item 一棵树中的所有键都是唯一的。
\item 根节点至少有 1 个键。 其他节点至少有 $m-1$ 个键。
\item 每个节点(包括根节点)最多有 $2m-1$ 个键。
\item 每个节点中的键是递增排列的。
\item 每个节点的子节点数等于它的键数加 1。 也就是说每个节点最多有 $2m$ 个子节点。
\item 每个节点的第 $i$ 个子节点中的键值必须介于该节点的第 $i-1$ 个和第 $i$ 键值之间。
\item 所有叶节点在同一层级。
\item B-树在根节点增长和收缩。
\item 搜索、插入和删除节点的复杂度都是 $\order{\log n}$。
\end{itemize}

\begin{lstlisting}[language=cpp]
struct Node {
    int n;
    int key[MAX_KEYS];
    Node* child[MAX_CHILDREN];
    bool leaf;
};
 
Node* BtreeSearch(Node* x, int k) {
    int i = 0;
    while (i < x->n && k >= x->key[i]) {
        i++;
    }
    if (i < x->n && k == x->key[i]) {
        return x;
    }
    if (x->leaf) {
        return nullptr;
    }
    return BtreeSearch(x->child[i], k);
}
\end{lstlisting}


========= 以下为草稿 ============


在 B-树上的操作

1. 搜索
B-树中的搜索操作与二叉搜索树中的二分搜索非常相似。从根开始,我们移动到适当的子节点,直到找到键或者达到空位。

2. 插入
插入的开始与搜索类似。然而,如果键应该插入的节点已满,我们必须分裂节点。

分裂涉及到:

将节点中的中间键上移至其父节点。
将剩余的键分裂成两个新的节点,并将它们作为新移动键的子节点链接。
如果父节点已满并且发生了分裂,那么这个过程会继续向上,直到找到一个非满父节点或者创建一个新的根节点。

3. 删除
在B-树中,删除是最复杂的操作。目标是移除一个键并仍然保持B-树的属性。

B-树操作示例
为了简化,我们将使用阶数为 m=3 的B-树(也称为2-3树),其中每个节点最多可以有3个子节点。

插入
让我们按顺序插入键1, 2, 3, 4, 和5。

插入1:树为空,所以1成为根。

插入2:根没有满,所以2被添加进去。

Copy code
 1 2
插入3:根已满,所以我们必须分裂它。

markdown
Copy code
     2
   /   \
 1       3
插入4:从根开始,转到右子节点(3),这个节点没有满,所以4被添加进去。

markdown
Copy code
     2
   /   \
 1      3 4
插入5:从根开始,转到右子节点(3 4),这个节点已满,所以必须分裂它。父节点没有满,所以可以接受另一个键。

markdown
Copy code
     2 4
   /   |   \
 1      3    5
删除
我们删除2:

找到2:从根开始,这包含2。

如果2有子节点,我们需要用后继或前驱键替换它,但在这种情况下,它没有。所以,2可以简单地被删除:

markdown
Copy code
     4
   /   \
 1      3 5
这就是如何在B-树中插入和删除键!


B-树有变体,如B+树和B*树,它们有自己的特定用途和优化。