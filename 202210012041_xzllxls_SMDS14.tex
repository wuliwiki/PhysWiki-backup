% 上海海事大学 2014 年数据结构
% 上海海事大学 2014 年数据结构

\subsection{一.判断题(本题10分,每小题1分)}

1、若某顺序表采用顺序存储结构,每个元素占$10$个存储单元,首地址为$200$,则下标为$11$(第$12$个)的元素的存储起始地址为$320$.

2、若对线性表进行的主要操作不是插入和删除,则该线性表宜采用顺序存储结构.

3、对一个空栈按$a,b,c,d,e,f,g$顺序依次读入,经过多次入栈和出栈的操作后,能得到按$f,e,g,d,a,c,b$顺序的出栈序列.

4、假定在顺序表中每个位置插入的概率相同,向一个有$64$个元素的顺序表中插入一个新元素并保持原来顺序不变,平均要移动$33$个元素.

5、含有$3$个结点(元素值均不相同)的二叉排序树共有$30$种.

6、$n$个顶点的连通图至少有$n-1$条边.

7、在无向图$G$的邻接矩阵$A$中,若$A[i][j]$等于$1$,则$A[j][i]$等于$0$.

8、采用顺序检索法在一个有$123$个元素的有序顺序表中查找,若每个元素的查找概率相等,则成功检索的平均查找长度$ASL$为$61$.

9、在散列存储中,装载因子$a$的值越大,发生冲突的可能性就越大.

10、快速排序是一种稳定的排序方法.

