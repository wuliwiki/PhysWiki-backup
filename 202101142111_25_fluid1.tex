% 流体运动的描述方法
% keys 欧拉法|拉格朗日法

\subsection{流体与固体的区别}
我们从流体与固体的区别引入流体运动的描述方法,固体无论处于静止还是运动都可以通过有限的静变形承受剪切力,其形状不易变化,在运动学中,可以只从几何角度来描述物体的位置随时间的变化.

流体在静止时无法通过有限的静变形承受剪切力,在运动状态下虽能产生剪切力,剪切力却不能维持流体内部各质点位置的有序,反而使其产生连续不断的变形,流体内部各质点位置变化很大,单从几何角度描述其运动将复杂而困难,需要专门的处理方法.

固体力学中,只着眼于需要描述的物体,物体之外的都叫做环境,某一物体在环境中运动,与环境之间发生力的作用从而改变物体运动状态,这种方法称为拉格朗日法.

流体当然也可以采拉格朗日法,但因为几何形状的变化,需要进行复杂的坐标变换.变化的形状难以描述,我们考虑是不是可以研究一个特定不变的空间呢,着眼于流体经过这个空间时发生的变化以及与这个空间的相互作用,是不是也能全面描述流体的运动呢,实践证明这也是可行的,这种方法称为欧拉法.

\subsection{拉格朗日法}
我们先尝试使用拉格朗日法描述流体的运动,考虑一个无穷小的不规则流体微团,,为研究其运动,我们跟随此单个微团在时间和空间中运动,从而描述微团的空间位置随时间变化的轨迹.可以想象我们是坐在船上,跟随河水向前运动.

运动过程中,微团形状不断变化,独立的变量为时间$t$与变动的空间坐标$s$,拉格朗日法这样描述流体微团的运动:

“在$t$时刻,微团A的空间位置为...”,用公式表示为

“在$t$时刻,微团A的速度为...”,用公式表示为
\begin{equation}
V
\end{equation}

“在$t$时刻,微团A的加速度为...”,用公式表示为

\subsection{欧拉法}