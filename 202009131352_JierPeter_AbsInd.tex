% 抽象指标

\begin{issues}
\issueTODO
\end{issues}


\pentry{爱因斯坦求和约定\upref{EinSum}}

在欧几里得空间中,我们常常可以把一个向量看成若干个标量的组合,类似地也可以把一个向量场看成若干个标量场的组合.这是因为在欧几里得空间中我们已经默认了一组标准正交基,使得向量都可以表示成坐标形式,每个坐标都可以单独看成一个标量.

欧几里得空间中可以唯一地把向量看成其分量的组合,这使得问题的讨论具体而直观.比如说,当我们讨论向量和对偶向量的乘积时,就可以说是各对应分量的乘积之和;按照爱因斯坦求和约定,如果有对偶向量$\bvec{v}=\pmat{v_x, v_y, v_z}$和向量$\bvec{u}=\pmat{u^x, u^y, u^z}$,那么这两个向量的内积就是$v_iu^i$.

在微分几何中,我们研究的主要对象是流形.流形是局部同胚于欧几里得空间的对象,这样就可以按照这个同胚在流形上的每个点附近定义一个局部坐标系.但是每个点上,可以选择不同的欧几里得空间来进行同胚,也就有了不同的坐标系.当我们讨论流形上一点的切空间时,选择不同的局部坐标系会导致切空间的基的选择有所不同.这样一来,流形上一个切向量的坐标分量就是不确定的,必须给定了局部坐标系才可以讨论其分量.

在流形上讨论向量、对偶向量乃至一般的张量时,只有给定了具体的局部坐标系才可以像欧几里得空间里那样拆成坐标形式来直观讨论.总是要选定局部坐标系会让讨论变得繁琐,因此,为了在保留坐标表示的优势的同时避免选定坐标系的烦恼,我们引入了抽象指标的表示方法.




\subsection{抽象指标}

对于流形上一点的切向量$\bvec{v}$,我们用抽象指标记号将其记为$\bvec{v}^a$.$a$是这个切向量的一个抽象指标,意味着当我们选定任意一个局部坐标系以确定切空间的基后,这个$\bvec{v}$在这组基下的坐标表示是由一个指标来确定的,也就是说,它的分量是$v^1$,$v^2$,$v^3$等等.





