% 浙江大学 2014 年硕士入学量子力学考试试题
% keys 浙江大学|考研|物理|量子力学|2014年
% license Copy
% type Tutor



\textbf{声明}:“该内容来源于网络公开资料,不保证真实性,如有侵权请联系管理员”

\subsubsection{第一题}
(1)一维问题的能级的最大简并度最大是多少?
(2)什么是量子力学中的守恒量,它们有什么性质。
(3)什么是受激辐射?什么是光电效应?
(4)试写出非简并微扰论的能级修正公式(到二阶)。
(5)由正则对易关系$[\uvec x,\uvec p]=ih$导出角动量的三个分量$\displaystyle L_x=y\pdv{}{z}-z\pdv{}{y}  \quad L_y=z\pdv{}{x}-x\pdv{}{z} \quad L_z=x\pdv{}{y}-y\pdv{}{x}$
的对易关系。
\subsubsection{第二题}
原子序数较大的原子的最外层电子感受到的原子核和内层电子的总位势可以表示为$\displaystyle V(r)=-\frac{e^2}{r}-\lambda\frac{e^2}{r^2},\lambda=1$试求其基态能量。
\subsubsection{第三题}
设电子以给定的能量$E=\frac{h^2}{}$自左入射,遇到一个方势阱2m
x<0,x>aV(x)0≤x≤a
求反射系数和透射系数;(a
给出发生共振隧穿的条件:
考虑到电子有自旋(自旋向下或向上),你能否借用上面的结果,设计一个量子调控装置,使反射回来的只有自旋向上的电子而没有自旋向下的电子?
第四题:(25 分)
试求屏蔽库伦场V(r)=ピe-/ 的微分散射截面,(提示:可直接用中心势散射的玻恩近似公式的化简形式)a@)- [-rna其中K=2ksin号K
第五题:(25 分)
A+a许多物理问题可以化成两能级系统,如=+'=其中a,b为实B+a
并且远小于 A-B数,
(a)试求能级的精确值: