% 亚伯拉罕·棣莫弗(综述)
% license CCBYSA3
% type Wiki

本文根据 CC-BY-SA 协议转载翻译自维基百科\href{https://en.wikipedia.org/wiki/Abraham_de_Moivre}{相关文章}。

\begin{figure}[ht]
\centering
\includegraphics[width=6cm]{./figures/a68a19e0be08cf7f.png}
\caption{约瑟夫·海默绘《德·莫弗像》,1736年} \label{fig_YBLH_1}
\end{figure}
亚伯拉罕·德·莫弗尔爵士(Abraham de Moivre FRS,法语发音:[abʁaam də mwavʁ],1667年5月26日-1754年11月27日)是一位法国数学家,以“德·莫弗公式”闻名——这是一个将复数与三角函数联系起来的公式。他还因在正态分布和概率论方面的工作而享有盛誉。

由于法国对胡格诺派的新教徒进行宗教迫害,特别是在1685年《枫丹白露敕令》颁布后达到高潮,他年少时迁居英格兰。他是艾萨克·牛顿、埃德蒙·哈雷和詹姆斯·斯特林的朋友。在英格兰的胡格诺派流亡者中,他也是编辑和翻译家皮埃尔·德·梅佐的同僚。

德·莫弗尔撰写了一本关于概率论的著作《机会论》,据说广受赌徒欢迎。他最早发现了比奈公式,这是一个将黄金分割数 φ 的 n 次幂与第 n 个斐波那契数联系起来的闭式表达式。他同样是最早提出中心极限定理的人之一,该定理是现代概率论的基石。
\subsection{生平}
\begin{figure}[ht]
\centering
\includegraphics[width=6cm]{./figures/832799eab7c34141.png}
\caption{} \label{fig_YBLH_2}
\end{figure}