% 晶核的长大

\pentry{经典形核理论\upref{NCLT}}
\footnote{本文参考了刘智恩的《材料科学基础》与Callister的 Material Science and Engineering An Introduction} 在形核理论 \upref{NCLT} 中我们已经探讨了形核的必要条件。在本文中我们将处理晶核最终长大为晶粒的问题。

\subsection{长大动力学}
晶核整体的长大速率仍然受到两方面因素的制约。首先,晶核长大的前提是要有晶核,所以长大速率与形核速率有关;其次,长大过程是原子从液相扩散到固相的过程,因此长大速率还与原子的扩散速率有关。这两者是确定长大速率的主要因素。
\begin{figure}[ht]
\centering
\includegraphics[width=8 cm]{./figures/dc67abf0ad7d0ec1.pdf}
\caption{长大速率示意图} \label{fig_GGRW_1}
\end{figure}

我们已经在形核理论 \upref{NCLT} 中探讨了形核率与过冷度的联系:随过冷度增高,形核率会先升高再降低;与此同时,低温将会抑制原子的扩散,因此随过冷度增高,扩散速率会降低。这二者的共同结果是,随过冷度上升,长大速率会上升再减小。在简单的情况下,在长大速率呈现下降前液体就已经凝固完成,因此可以简单地认为随过冷度升高,长大速度将升高。

\subsubsection{过冷度与晶粒尺寸}
\begin{figure}[ht]
\centering
\includegraphics[width=8 cm]{./figures/815e75cba9e05772.pdf}
\caption{过冷度与晶粒尺寸} \label{fig_GGRW_fig2}
\end{figure}
更值得注意的是,过冷度对晶粒尺寸的影响。在较高的过冷度下,由于形核率较高而扩散速率较低,主导的机制是形核,因此最终形成的晶粒将更多也更小(很显然,晶粒的数量和大小成反比,因为可供结晶的物质的总量是一定的);反而言之,在较低的过冷度下,形核率较低而扩散速率较高,主导的机制是扩散,最终形成的晶粒也将更大也更少。
