% 引力的弱场近似
% keys 广义相对论|relativity|引力|gravity|弱场|weak field|测地线|geodesic|牛顿力学|闵可夫斯基时空|Minkowski spacetime|时空|spacetime|流形|manifold|闵可夫斯基度规|Minkowski metric

\pentry{测地线\upref{geodes}}

广义相对论的革命性创见在于将引力解释为时空的几何效应。广义相对论将引力视为非力作用,并假设不受力的物质的运动轨迹是测地线,其参数取该物质的本征时间。从测地线一节中的\autoref{eq_geodes_1}~\upref{geodes}可知,非平坦的度量在给定图中可能有非零的Christoffel符号,使得测地线方程的解不再是这个图中的一条匀速直线,这就启发了我们,如果把时空看成是流形,那么引力可能通过影响联络来造成表面上的“扭转”物质轨迹。当然,由于曲率是由联络来定义的,也可以说是引力改变了时空的曲率。

以上只是定性说法,那么我们是否真的可以用这种方法来描述引力作用呢?由于牛顿的引力论在低速、弱场且静态的情况下已经被实验反复证实,我们可以从此下手,在低速弱场静态近似下尝试解一个质点的测地线方程,看能不能回归到牛顿的引力方程上\footnote{思路取自Carroll的广义相对论讲义\cite{CarrollGR},第4章\textsl{Gravitation}中的讨论。}。

在接下来的讨论中,对于真指标,用希腊字母表示在 $\{0, 1, 2, 3\}$ 中遍历指标,用拉丁字母表示在 $\{1, 2, 3\}$ 中遍历指标。也就是说,拉丁字母特指空间分量。

\subsection{近似假设}

\textbf{低速}近似意味着,在所讨论的参考系里,质点的四速度非常接近 $\pmat{1&0&0&0}$,也就是说,其轨迹的\textbf{三个空间坐标}$x^i$ 满足下式:
\begin{equation}\label{eq_WeakG_1}
\frac{\dd}{\dd \tau}x^i\ll \frac{\dd}{\dd \tau}x^0\approx 1
\end{equation}

\textbf{弱场}近似意味着,引力的作用非常微弱,也就是说对度量的影响很小。记 $\eta_{\mu\nu}$ 为该参考系中的Minkowski度规,那么引力作用下的度规就是 $g_{\mu\nu}=\eta_{\mu\nu}+h_{\mu\nu}$,其中 $\abs{h_{\mu\nu}}\ll 1$。考虑到度规的指标升降法则要求 $g^{ai}g_{bi}=\delta^a_b$,结合 $\abs{h_{\mu\nu}}\ll 1$,可以计算出 $g^{\mu\nu}=\eta^{\mu\nu}-h^{\mu\nu}$\footnote{你可以尝试验证:$(\eta_{ai}+h_{ai})(\eta^{bi}-h^{bi})=\eta_{ai}\eta^{bi}+h_{ai}\eta^{bi}-h^{bi}\eta_{ai}-h_{ai}h^{bi}$,其中右边的中间两项由度量的对称性以及 $h^{ab}=h_{ij}\eta^{ia}\eta^{jb}$,可以抵消掉,于是只剩下 $\delta^b_a-h_{ai}h^{bi}$ 项,后者是更高阶的小量,弱场近似下就被忽略了。}。

\textbf{静态}近似意味着,引力场不随时间变化,因此 $\frac{\dd}{\dd t}g_{\mu\nu}=0$。用另一种记号来写的话,就是 $\partial_0g_{\mu\nu}=0$。

\subsection{推导测地线方程}

假设质点的坐标为 $x^\mu$,它们是自身固有时间的函数,也是给定参考系(图)中的时间的函数。当然,$x^0$ 就是参考系中的时间参数本身。

参考\textbf{测地线}\upref{geodes}一节可知,测地线方程为\autoref{eq_geodes_1}~\upref{geodes}:
\begin{equation}
\frac{\mathrm{d}^2}{\dd\tau^2}x^\mu+(\frac{\dd}{\dd\tau}x^a)(\frac{\dd}{\dd\tau}x^b)\Gamma^\mu_{ab}=0
\end{equation}

注意这里求导用的参数是固有时间 $\tau$,因为我们的假设就是“质点的轨迹为\textbf{按固有时间为参数的测地线}”。

由低速近似的\autoref{eq_WeakG_1} ,我们可以把测地线方程简化为
\begin{equation}\label{eq_WeakG_3}
\frac{\mathrm{d}^2}{\dd\tau^2}x^\mu+(\frac{\dd}{\dd\tau}x^0)^2\Gamma^\mu_{00}=0
\end{equation}

也就是说,我们只需要关注 $\Gamma^\mu_{00}$ 的表达。

由\textbf{Christoffel符号}\upref{CrstfS}词条中的\autoref{eq_CrstfS_3} 可得
\begin{equation}
\Gamma^\mu_{00}=\frac{1}{2}g^{k\mu}(\partial_0g_{0k}+\partial_0g_{0k}-\partial_{k}g_{00})~.
\end{equation}

结合弱场近似 $\partial_0g_{\mu\nu}\approx\partial_0\eta_{\mu\nu}$,上式化简为
\begin{equation}\label{eq_WeakG_2}
\Gamma^\mu_{00}=-\frac{1}{2}g^{k\mu}\partial_{k}g_{00}\approx-\frac{1}{2}\eta^{k\mu}\partial_{k}h_{00}
\end{equation}

将\autoref{eq_WeakG_2} 代回测地线方程\autoref{eq_WeakG_3} ,得到
\begin{equation}\label{eq_WeakG_4}
\frac{\mathrm{d}^2}{\dd\tau^2}x^\mu\approx\frac{1}{2}(\frac{\dd}{\dd\tau}x^0)^2\eta^{k\mu}\partial_{k}h_{00}
\end{equation}

考虑 $\mu=0$ 项,结合静态近似,再考虑到 $\eta_{\mu\nu}$ 的坐标,则\autoref{eq_WeakG_4} 化为
\begin{equation}
\frac{\mathrm{d}^2}{\dd\tau^2}x^0\approx\frac{1}{2}(\frac{\dd}{\dd\tau}x^0)^2\eta^{k0}\partial_{k}h_{00}=0
\end{equation}

因此 $\frac{\dd}{\dd\tau}x^0$ 是一个常数。

我们将\autoref{eq_WeakG_4} 进行移项,得到
\begin{equation}\label{eq_WeakG_5}
\frac{\mathrm{d}^2}{\dd t^2}x^i\approx\frac{1}{2}\eta^{ki}\partial_kh_{00}
\end{equation}

由于 $\eta^{ij}=\delta^{ij}$\footnote{注意这里用的是拉丁字母,也就是限制在空间坐标上,那么 $\eta_{ij}$ 就是通常的欧几里得度量。},\autoref{eq_WeakG_5} 进一步化简为
\begin{equation}\label{eq_WeakG_6}
\frac{\mathrm{d}^2}{\dd t^2}x^i\approx\frac{1}{2}\partial^ih_{00}
\end{equation}

这就是\textbf{低速弱场静态近似}下,质点的测地线方程。

\subsection{与牛顿引力论比较}

牛顿引力论认为引力是由分布在时空中的一个光滑函数 $\Phi$ 决定的,这个函数就称为\textbf{引力势}。引力对质点的作用被体现为以下式子:
\begin{equation}\label{eq_WeakG_7}
\frac{\mathrm{d}^2}{\dd t^2}x^i=-\partial^i\Phi
\end{equation}

比较\autoref{eq_WeakG_6} 和\autoref{eq_WeakG_7} 可知,只需要令 $h_{00}=-2\Phi$,以上结果在低速、弱场、静态的情况下就可以\textbf{近似}回归到牛顿引力论上。这意味着 $h^{\mu\nu}$ 的其它部分并不影响近似结果,是冗余的自由度。








