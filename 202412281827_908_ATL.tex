% 埃托雷·马约拉纳(综述)
% license CCBYSA3
% type Wiki

本文根据 CC-BY-SA 协议转载翻译自维基百科\href{https://en.wikipedia.org/wiki/Arthur_Compton}{相关文章}。

\begin{figure}[ht]
\centering
\includegraphics[width=6cm]{./figures/7dc439ef645ec5c1.png}
\caption{马约拉纳在1930年代} \label{fig_ATL_1}
\end{figure}
埃托雷·马约拉纳(Ettore Majorana,/maɪəˈrɑːnə/,[2] 意大利语:[ˈɛttore majoˈraːna];1906年8月5日出生——可能在1959年或之后去世)是意大利理论物理学家,曾研究中微子质量。1938年3月25日,他在购买了从那不勒斯到巴勒莫的船票后神秘失踪。

马约拉纳方程和马约拉纳费米子以他的名字命名。2006年,为了纪念他,设立了马约拉纳奖。
\subsection{生活与工作}
1938年,恩里科·费米曾这样评价马约拉纳:“世界上有几类科学家;第二或第三流的科学家尽最大努力,但永远不会走得太远。然后是第一流的科学家,他们做出了对科学进步至关重要的发现。但还有一些天才,比如伽利略和牛顿。马约拉纳就是其中之一。”