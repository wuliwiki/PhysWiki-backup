% 查尔斯·巴贝奇(综述)
% license CCBYSA3
% type Wiki

本文根据 CC-BY-SA 协议转载翻译自维基百科\href{https://en.wikipedia.org/wiki/Charles_Babbage}{相关文章}。

\begin{figure}[ht]
\centering
\includegraphics[width=6cm]{./figures/2f69cba1b5206a32.png}
\caption{巴贝奇在1860年} \label{fig_CRSBQ_1}
\end{figure}
查尔斯·巴贝奇(Charles Babbage,1791年12月26日-1871年10月18日)是一位英国博学家。[1] 他是一位数学家、哲学家、发明家和机械工程师,巴贝奇提出了数字可编程计算机的概念。[2]

巴贝奇被一些人认为是“计算机之父”。[2][3][4][5] 他被认为发明了第一台机械计算机——差分机,这为更复杂的电子设计奠定了基础,尽管现代计算机的所有基本思想都可以在他的分析机中找到,该机器是通过一个明确借鉴自雅卡尔织机的原则来编程的。[2][6] 除了计算机相关工作外,巴贝奇在他1832年出版的《制造与机械经济学》一书中还涉及了广泛的兴趣领域。[7] 他是伦敦社交圈中的重要人物,并且以其举办的周六晚会而闻名,被认为将“科学晚会”从法国引入英国。[8][9] 他在其他领域的多样工作使他被描述为其世纪中“最杰出”的博学家之一。[1]

巴贝奇虽然未能完成许多设计的成功工程实现,包括他的差分机和分析机,但他在计算机理念的提出上依然是一个重要人物。他未完成的部分机械装置如今被展示在伦敦的科学博物馆。1991年,一台根据原始设计图纸构建的功能性差分机完成了建造。按照19世纪可以实现的公差制造,最终完成的差分机成功运转,证明了巴贝奇的机器本应能够正常工作。