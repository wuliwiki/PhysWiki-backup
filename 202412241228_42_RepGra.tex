% 图的矩阵表示
% keys 矩阵表示|图|邻接矩阵|关联矩阵
% license Usr
% type Tutor

\pentry{图\nref{nod_Graph},矩阵及其运算\nref{nod_Mat}}{nod_3f6e}
将\enref{图}{Graph}的点集和边集的关系用矩阵来体现,就是所谓的图的矩阵表示。体现点与点相邻关系的矩阵称为邻接矩阵;体现点与边关联关系的矩阵称为关联矩阵。当对某个图选定了权函数成为\enref{赋权图}{GraSpa}时,体现权函数的矩阵称为赋权矩阵。

\subsection{邻接矩阵}
邻接矩阵是用来体现点点相邻关系的,具体地,设图 $G$ 的点集为 $V(G)=\{v_1,\cdots,v_n\}$。若 $G$ 是无向图,则邻接矩阵的第 $(i,j)$ 个元反映了连接 $v_i,v_j$ 的边数。若 $G$ 是有向图,则邻接矩阵的第 $(i,j)$ 个元反映了以 $v_i$ 为起点,$v_j$ 为终点的有向边数。
\begin{definition}{邻接矩阵}\label{def_RepGra_1}
设 $G$ 是图,$V(G)=\{v_1,\cdots,v_n\}$,$A=(a_{ij})$ 是 $n$ 阶矩阵。若 $G$ 是无向图,$a_{ij}$ 等于连接 $v_i,v_j$ 的边数;若 $G$ 是有向图,$a_{ij}$ 等于以 $v_i$ 为起点,$v_j$ 为终点的有向边数。则称 $A$ 为 $G$ 的\textbf{邻接矩阵}(adjacency matrix)。
\end{definition}

\begin{theorem}{}
设 $A$ 是有向图 $G$ 的邻接矩阵,$V(G)=\{v_1,\cdots,v_n\}$。则 $A^k$ 中的第 $(i,j)$ 元是 $G$ 中长度为 $k$ 的从 $v_i$ 到 $v_j$ 的有向\enref{链}{PatCyc}的数目。
\end{theorem}
\textbf{证明:}
使用数学归纳法证明如下:当 $k=1$ 是,由邻接矩阵 $A$ 的定义(\autoref{def_RepGra_1} )和链长的定义(\autoref{def_PatCyc_4}),可知定理成立;

设 $k=l$ 时定理成立,那么当 $k=l+1$ 时,设 $a_{ij}^{(l)},a_{ij}^{(l+1)}$ 分别是 $A^{l},A^{l+1}$ 的 $(i,j)$ 元。由于 $A^l=A^{l-1}A$,所以
\begin{equation}
a_{ij}^{(l+1)}=\sum_{m=1}^na_{im}^{(l)}a_{mj}.~
\end{equation}
由假设,$a_{im}^{(l)}$ 是从 $v_i$ 到 $v_m$ 的长为 $l$ 的有向链数, $a_{mj}$ 是从 $v_m$ 到 $v_j$ 的长为 $l$ 的有向链数,于是 $a_{im}^{(l)}a_{mj}$ 是从 $V_i$ 到 $v_m$ 再到 $v_j$ 的长为 $l+1$ 的有向链数。而所有从 $v_i$ 到 $v_j$ 的长为 $l+1$ 的有向链,都是从 $v_i$ 到某个 $v_x$ 的长为 $l$ 的有向链与 $v_x$ 到 $v_j$ 的长为1的有向链连接构成。

因此若记 $A_m$ 是从 $V_i$ 到 $v_m$ 再到 $v_j$ 的长为 $l+1$ 的所有有向链的全体,而 $A$ 记为从 $V_i$ 到 $v_j$ 的长为 $l+1$ 的所有有向链的全体,那么
\begin{equation}
A=\bigcup_{m=1}^n A_m.~
\end{equation}
显然 $A_a\cap A_b=\varnothing,a\neq b$,因此
\begin{equation}
\abs{A}=\sum_{m=1}^n \abs{A_m}.~
\end{equation}
而 $\abs{A_m}=a_{im}^{(l)}a_{mj}$。因此定理得证。


\textbf{证毕!}

这一定理对无向图也是成立的,证明类似。
\subsection{关联矩阵}

关联矩阵是用来体现点边的关联关系的,具体地,设图 $G$ 的点集和边集分别为 
\begin{equation}\label{eq_RepGra_1}
V(G)=\{v_1,\cdots,v_n\},E(G)=\{e_1,\cdots,e_m\}.~
\end{equation}
若 $G$ 是无向图,则关联矩阵的第 $(i,j)$ 个元反映了 $v_i,e_j$ 是否关联。若 $G$ 是有向图,则关联矩阵的第 $(i,j)$ 个元反映了以 $v_i$ 是 $e_j$ 的起点还是终点。为方便使用Mathematica软件进行图论计算起见,我们这里使用Mathematica关于图的关联矩阵定义
\begin{definition}{关联矩阵}\label{def_RepGra_2}
设 $G$ 是图,$V(G)=\{v_1,\cdots,v_n\},E(G)=\{e_1,\cdots,e_m\}$,$A=(a_{ij})$ 是 $n\times m$ 的矩阵。若
\begin{equation}
a_{ij}=\left\{\begin{aligned}
&0,\quad &&v_i\notin e_j,\\
&1, &&e_j=(*,v_i) \quad or\quad {v_i,v_k},i\neq k\\
&-1, &&e_j=(v_i,*) ,\\
&2, &&e_j=(v_i,v_i)
\end{aligned}\right.~
\end{equation}
则称 $A$ 为 $G$ 的\textbf{关联矩阵}(incidence matrix)。
\end{definition}
\begin{example}{}
设有图 $G=(V,E,\varphi)$,其中
 \begin{equation}
 \begin{aligned}
  &V=\{v_1,v_2,v_3\},E=\{e_1,e_2,e_3,e_4,e_5\},\\
 &\varphi(e_1)=(v_1,v_1), \varphi(e_2)=(v_1,v_1), \\
 &\varphi(e_3)=(v_1,v_2), \varphi(e_4)=(v_2,v_1),\varphi(e_5)=\{v_1,v_2\}.
 \end{aligned}~
 \end{equation}
 其中我们定义的 $e_1,e_2$ 尽管都是 $(v_1,v_2)$,但是在mathematica中可以将它表示为有向环和无向环\footnote{在某些现实的意义上其不是等价的,比如一条以 $a$ 为端点的环,规定只能逆时针走,那么它就是单向的,而若双向可走则才等价于无向环,此时应当由表示顺时针和逆时针的记号进行方向的标记,比如"-",即 $(-a,a)$ 表示逆时针,$(a,-a)$ 表示逆时针。这都是定义的问题,仅仅为了一一对应我们关心的问题,没有标准的表示,只要符合表达了对应关系即可,但是为了方便往往会约定某一种习惯定义,正如我们习惯了集合用花括号定义},因此可以用它们演示mathematica中它们在关联矩阵中对应的元是相同的。
 该图具有如下的表示
\begin{figure}[ht]
\centering
\includegraphics[width=8cm]{./figures/5d2761ac8abb9f58.png}
\caption{例子图的表示} \label{fig_RepGra_2}
\end{figure} 
那么由关联矩阵\aref{定义}{def_RepGra_2},该图的关联矩阵为
\begin{equation}
\begin{pmatrix}
2&2&-1&1&1\\
0&0&1&-1&1\\
0&0&0&0&0
\end{pmatrix}.~
\end{equation}

该例的Mathematica代码如下,其表明关联矩阵和我们的定义一致。
\begin{figure}[ht]
\centering
\includegraphics[width=14.25cm]{./figures/ab6e9336b570e8cc.png}
\caption{代码及结果展示} \label{fig_RepGra_1}
\end{figure}

\end{example}

\begin{theorem}{}
设 $M$ 是有向图的关联矩阵,$M_i$ 是从 $M$ 的终点除掉第 $i$ 行后得到的矩阵。则 $MM^T$ 中所有元素的代数余子式都为 $|M_iM_i^T|$。

\end{theorem}
\textbf{证明:}
设 $N=MM^T$,由于当对某个 $k$ 而言, $M_{ik}M_{jk}$ 在 $v_i\neq v_j$ 为 $e_k$ 的两端点时为 -1,当 $v_i=v_j$ (即 $i=j$)为 $e_k$ 的端点且 $e_k$ 不是环时为1,是环时为4。因此
\begin{equation}
n_{ij}=\left\{\begin{aligned}
&d_G(v_i)=d_G^+(x_i)+d_G^{-1}(x_i)-c,\quad &i=j,\\
&-\mu(x_i,x_j)-\mu(x_j,x_i),\quad &i\neq j.
\end{aligned}\right.~
\end{equation}


\textbf{证毕!}


\subsection{赋权矩阵}

赋权矩阵是用来体现赋权函数的,其有不同的定义方式。例如,设图 $G$ 的点集为 $V(G)=\{v_1,\cdots,v_n\}$,则赋权矩阵是一个 $n$ 阶矩阵。若 $G$ 为有向图,则其第 $(i,j)$ 个元反映了以 $v_i$ 为起点 $v_j$ 为终点的所有边的权值之和;若 $G$ 为无向图,则其第 $(i,j)$ 个元反映了连接 $v_i,v_j$ 的所有边的权值之和。
\begin{definition}{赋权矩阵}
若 $G$ 是图,$\omega$ 是其上的权函数。则给出权函数的矩阵称为\textbf{赋权矩阵}。
\end{definition}









