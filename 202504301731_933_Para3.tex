% 抛物线(高中)
% keys 极坐标系|直角坐标系|圆锥曲线|抛物线
% license Xiao
% type Tutor

\begin{issues}
\issueDraft
\end{issues}

\pentry{解析几何\nref{nod_JXJH},圆\nref{nod_HsCirc},双曲线\nref{nod_Hypb3}}{nod_7c17}

不知道读者在初次接触双曲线时,是否产生了一种似曾相识的感觉:它的一支看起来与初中阶段学习过的二次函数图像——抛物线,非常相似。二者都不封闭、有一个开口、略微弯曲、向无限延伸,甚至也拥有一条对称轴。相信一些读者可能早已不禁在心中将双曲线的一支等同于抛物线,认为双曲线不过是“两个抛物线”的组合而已。

这种误解并不罕见,难以否认,抛物线与双曲线在形状上确实有些相像。但从本质上看,二者在几何定义、解析式结构以及性质上都有明显区别。之所以容易混淆,很大程度上与初中学习的重点有关。当时的教学更侧重于二次函数的代数表达与图像之间的关系,例如判断开口方向、对称轴位置、顶点坐标等。这些内容有助于建立对抛物线的基本印象,但主要仍停留在函数视角,对抛物线作为几何图形本身的理解较为有限。

大多数人对抛物线的印象,往往停留在现实生活中物体被抛出后所形成的轨迹。在理想状态下,这类轨迹正是一条抛物线,这也正是“抛物线”名称的来源。然而,随着人们的进一步研究发现,抛物线并不仅仅出现在物理运动中,它还具有独特的几何性质,在许多实际工程中发挥着重要作用。例如,雷达天线的反射面通常设计成抛物面结构,原因就在于抛物线具备一种精确的聚焦特性:来自远处的平行电磁波在抛物面上反射后,会准确地汇聚到焦点;而从焦点出发的信号,也能被反射成方向一致的平行波。这一聚焦能力,使抛物面非常适合实现能量的集中与传输,使抛物线广泛出现于雷达、卫星通信设备、汽车大灯以及太阳能灶等场景中。

\subsection{抛物线的几何定义}

由于初中阶段已经花了很多力气研究二次函数,并且基本了解了抛物线的图像特征,因此此处不再从函数的角度展开,而是直接进入几何定义的探讨。

在研究椭圆和双曲线的过程中,曾尝试将圆的定义加以推广。圆可以看作是满足条件 $|O_1P| = |O_2P| = r$ 的点$P$的集合,其中 $O_1$ 与 $O_2$ 是重合的点,二者之间的距离 $d(O_1,O_2)=0$。如果放宽这个限制,即允许 $d(O_1,O_2)$ 不为零,并且打开第一个等号,就可以得到椭圆和双曲线这两种新曲线,这在之前已经探究过了。

那么,如果不打开第一个等号,而是改为打开第二个等号,只要求 $|O_1P| = |O_2P|$呢?之前提到过,在这种设定下,点 $P$ 的轨迹是所有关于两个定点等距的点,也就是两点连线的垂直平分线。感觉似乎这样修改之后没有什么可改动的空间了。不过,不妨换个思路,改变一下条件中的几何元素呢?也就是说,保留其中一个点 $O_1$,将另一个点 $O_2$ 替换为一条固定的直线 $l$,那么问题就变成了:考虑所有满足“到某一固定点与到某一固定直线的距离相等”的点 $P$ 的轨迹,会构成怎样的图形?显然,这样修改之后还需要修改$d$,令其满足 $d(O_1,l)\ne 0$。否则如果直线 $l$ 正好通过定点 $O_1$,唯一满足条件的点就只有 $O_1$,轨迹会退化成一个点。


回顾研究椭圆和双曲线的定义时,都曾尝试从圆的定义进行推广。圆的定义可以看作是:对于平面上任意一点 $P$,满足 $|O_1P| = |O_2P| = r$,其中 $O_1$ 与 $O_2$ 满足二者距离$d(O_1,O_2)=0$。通过放松定义中的约束条件,可以构造出不同的新曲线。椭圆和双曲线就是通过调整$d$之后,再打开第一个等号而得到的。

那么,如果不去动第一个等号,而是打开第二个等号,也就是只要求 $|O_1P| = |O_2P|$,又会得到什么?之前提到过,此时,点 $P$ 的轨迹变成了所有点关于两个定点的垂直平分线的集合。看上去,好像不能再进行什么其他的变换了。

不过,换个思路,改变一下元素呢?如果不再是两个定点之间作比较,而是保留其中一个点不变,把另一个点$O_2$换成一条直线$l$,会发生什么?换句话说,考虑满足“点到某一固定点与到某一固定直线距离相等”的所有点的轨迹,这样是否能构成新的图形?

分析一下,经过这样的改动后,与之前类似,也需要先调整$d(O_1,l)\neq0$这个条件,毕竟,若$l$恰好经过定点$O_1$,则满足条件的只有$O_1$一个点,结果非常平凡。当$d(O_1,l)\neq0$但若这条直线和定点之间保持一定距离,那么情况就变得有趣起来。


在研究椭圆和双曲线的定义时,都尝试从圆的定义出发推广,对定义的表达方式进行调整。核心就是调整 $|O_1P| = |O_2P| = r$ 这个限定条件中的一些定价的表达。不论是椭圆还是双曲线,都是打开了第一个等号。下面考虑打开第二个等号,之前提过,只保证$|O_1P| = |O_2P|$可以得到所有点关于两个定点的垂直平分线的集合。看上去似乎没有什么其他可改动的空间了。不过,如果改变一下元素呢?如果原本只是改成了两个点,现在把其中的一个点改成一条直线会得到什么呢?当然,这条直线如果和点重合,那么除了这个点,没有任何其他点能满足条件,结论很平凡,没什么意思。如果让直线和点分开一些呢?



标准定义:平面上到定点(焦点)和定直线(准线)距离相等的点的轨迹
这就是 “\enref{圆锥曲线的极坐标方程}{Cone}” 中对抛物线的定义。
\begin{figure}[ht]
\centering
\includegraphics[width=4.2cm]{./figures/c89771dd2fef516e.pdf}
\caption{抛物线的定义} \label{fig_Para3_1}
\end{figure}

在 $x$ 轴正半轴作一条与准线平行的直线 $L$, 则抛物线上一点 $P$ 到其焦点的距离 $r$ 与 $P$ 到 $L$ 的距离之和不变。

如\autoref{fig_Para3_1}, 要证明由焦点和准线定义的抛物线满足该性质, 只需过点 $P$ 作从准线到直线 $L$ 的垂直线段 $AB$, 由于 $r$ 等于线段 $PA$ 的长度, 所以 $r$ 加上 $PB$ 的长度等于 $AB$ 的长度, 与 $P$ 的位置无关。 证毕。


\subsection{抛物线的方程}

\begin{theorem}{抛物线的标准方程}

\end{theorem}
	•	顶点在原点,轴为 $y$ 轴的标准式:$x^2=2py$
	•	讨论参数 $p$ 的意义(焦点到顶点的距离)
\begin{theorem}{抛物线的参数方程}
	•	用参数表示抛物线上的点,如 $x=pt^2,,y=2pt$ 等(视教学安排可选讲)
\end{theorem}

\subsection{抛物线的几何性质}
	•	对称性(关于轴对称)
	•	顶点、焦点、准线的定义和关系
	•	开口方向与参数正负有关
	•	通用式推导(顶点在 $(h,k)$ 时的方程)
    反射性质(光线从焦点发出反射后平行于轴)
\subsubsection{切线}
	•	给定抛物线方程和点,求切线方程
	•	切线的几何意义(过点,与焦点、准线的关系)
