% 一阶不含时微扰理论

\begin{issues}
\issueDraft
\issueOther{这甚至不是量子力学问题, 直接作为厄米矩阵本征值的微扰理论来讲可能会更好, 放到线性代数相关章节}
\end{issues}

\pentry{矩阵对易与共同本征矢\upref{OpComu}}

\footnote{参考 \cite{GriffQ} \cite{Shankar} \cite{Sakurai} \cite{Merzbacher} 相关章节.}不含时微扰理论.

本征方程
\begin{equation}\label{TIPT_eq9}
H \psi_n = E_n \psi_n
\end{equation}
其中
\begin{equation}\label{TIPT_eq3}
H = H_0 + \lambda H^1
\end{equation}
\begin{equation}
E_n = E_n^0 + \lambda E_n^1 + \lambda^2 E_n^2 + \dots
\end{equation}
令 $\psi_n^0$ 是 $H_0$ 的任意一组完备正交归一基底.
\begin{equation}
\psi_n = \psi_n^0 + \lambda\psi_n^1 + \lambda^2 \psi_n^2 + \dots
\end{equation}
代入后本征方程\autoref{TIPT_eq9} 变为
\begin{equation}
(H_0 + \lambda H^1)(\psi_n^0 + \lambda\psi_n^1 + \dots) = (E_n^0 + \lambda E_n^1 + \dots)(\psi_n^0 + \lambda\psi_n^1 + \dots)
\end{equation}
拆括号后, 若分别令等式两边 $\lambda = 1,2,3\dots$ 的项相等即可保证本征方程成立. 若我们忽略所有 $\lambda > n$ 的项并解出本征方程, 就把这种近似称为 $n$ 阶微扰.

\subsection{一阶微扰}
来考虑一阶微扰, 令 $\lambda \to 0$, 忽略 $\order{\lambda^2}$, 有
\begin{equation}
H^0\psi_n^1 + H^1 \psi_n^0 = E_n^0 \psi_n^1 + E_n^1 \psi_n^0
\end{equation}
要使其恒成立, 就要求投影到任意 $\psi_m^0$ 上都成立:
\begin{equation}\label{TIPT_eq1}
\mel{\psi_m^0}{H^0}{\psi_n^1} + \mel{\psi_m^0}{H^1}{\psi_n^0} = E_n^0 \braket{\psi_m^0}{\psi_n^1} + E_n^1 \braket{\psi_m^0}{\psi_n^0}
\end{equation}
第一项利用厄米算符的性质
\begin{equation}
\mel{\psi_m^0}{H^0}{\psi_n^1} = \braket{H^0\psi_m^0}{\psi_n^1} = E_m^0\braket{\psi_m^0}{\psi_n^1}
\end{equation}
所以\autoref{TIPT_eq1} 化简为
\begin{equation}\label{TIPT_eq2}
\mel{\psi_m^0}{H^1}{\psi_n^0} = (E_n^0 - E_m^0) \braket{\psi_m^0}{\psi_n^1} + E_n^1 \delta_{m,n}
\end{equation}
下面分为 $H_0$ 是否简并来具体讨论. 简并是指 $H_0$ 的一个本征值可能对应多个线性无关的本征矢, 它们的任意线性组合也是本征矢, 每个本征值对应的所有本征矢的集合构成\textbf{本征子空间}(\autoref{HerEig_sub1}~\upref{HerEig}).

\subsection{非简并情况}
先看简单情况, $H_0$ 非简并时, 当 $m\ne n$ 必有 $E_n^0 \ne E_m^0$. 所以右边第一项对角线元素全为零. 所以考虑对角线上的元素($m = n$)有
\begin{equation}\label{TIPT_eq6}
E_n^1 = \mel{\psi_n^0}{H^1}{\psi_n^0}
\end{equation}
对 $m \ne n$ 即 $E_m^0 \ne E_n^0$ 的元素有
\begin{equation}\label{TIPT_eq4}
\braket{\psi_m^0}{\psi_n^1} = \frac{\mel{\psi_m^0}{H^1}{\psi_n^0}}{E_n^0 - E_m^0} \qquad (E_m^0 \ne E_n^0)
\end{equation}
于是, 要满足\autoref{TIPT_eq2}, 只需要令
\begin{equation}\label{TIPT_eq5}
\ket{\psi_n^1} = \sum_m^{E_m^0 \ne E_n^0} \braket{\psi_m^0}{\psi_n^1} \ket{\psi_m^0}
\end{equation}
即可. 注意\autoref{TIPT_eq5} 再加上任意 $c \psi_n^0$ 同样能使\autoref{TIPT_eq2} 成立, 说明\autoref{TIPT_eq2} 的解不止一个. 注意无论是能量还是波函数修正都和微扰哈密顿 $H^1$ 成正比.

以上两式之所把条件写成 $E_m^0 \ne E_n^0$ 而不是 $m \ne n$ 是因为前者对下面介绍的简并情况同样适用. 对非简并情况, 两个条件是等价的.

\subsubsection{验证正交归一性}
正交性要求 $\braket{\psi_m}{\psi_n} = \delta_{m,n}$, 即
\begin{equation}\label{TIPT_eq7}
\braket{\psi_m^0 + \lambda \psi_m^1 + \dots}{\psi_n^0 + \lambda \psi_n^1 + \dots} = \delta_{m,n}
\end{equation}
忽略 $\order{\lambda^2}$ 得
\begin{equation}\label{TIPT_eq8}
\braket{\psi_m^0}{\psi_n^1} + \braket{\psi_n^0}{\psi_m^1}\Cj = 0
\end{equation}
把\autoref{TIPT_eq4} 代入会发现 $m \ne n$ 时必成立. 当 $m = n$ 时上式变为
\begin{equation}
\Re{\braket{\psi_n^0}{\psi_n^1}} = 0
\end{equation}
\autoref{TIPT_eq5} 同样满足该要求.
\addTODO{但上面的任意常数 $c$ 可以是一个…… 非零纯虚数?}

\subsection{简并情况}
$H_0$ 简并时, 当 $m\ne n$ 也未必有 $E_n^0 \ne E_m^0$, 这使得\autoref{TIPT_eq4} 和\autoref{TIPT_eq5} 失效(分母为零), 所以要从\autoref{TIPT_eq2}
\begin{equation}
\mel{\psi_m^0}{H^1}{\psi_n^0} = (E_n^0 - E_m^0) \braket{\psi_m^0}{\psi_n^1} + E_n^1 \delta_{m,n}
\end{equation}
重新推导一次. 以下假设 $\psi_n^0$ 按照不同本征值来分段, 那么矩阵 $\mel{\psi_m^0}{H^1}{\psi_n^0}$ 也可以相应划分为分块矩阵, 对角块 $(i,j)$ 代表 $H^1$ 对第 $i$ 和 $j$ 个本征子空间之间的耦合.

我们只需要通过某种方法找到上式的某个解即可. 观察等式右边可以发现右边第一项对角块都为零($E_m = E_n$), 第二项是对角矩阵, 所以两项相加后矩阵的对角块都一定是对角矩阵, 非对角块则未必(图未完成). 所以等式左边的矩阵也需要满足同样的要求, 但 $H^1$ 是给定好的无法改变, 所以左边是否满足该要求只取决于基底 $\psi_n^0$ 的选取.

当 $H_0$ 非简并时, $\psi_n^0$ 是唯一确定的(除了整体相位). 但现在 $H_0$ 简并, $\psi_n^0$ 就并不完全确定: 每个本征子空间中的正交归一基底可以任意选取. 所以我们在每个本征子空间中再对 $H^1$ 进行角化即可满足上述要求. 注意这并不要求 $[H^0, H^1] = 0$ 对易, 因为对易要求整个 $\mel{\psi_m^0}{H^1}{\psi_n^0}$ 矩阵可以被彻底对角化而不只是每个对角块对角化. 这样通常就能唯一地确定一组 $\psi_n^0$, 但不唯一也关系不大.

这样得到的 $\psi_n^0$ 通常叫做\textbf{好本征态(good eigen states)}, $n$ 叫做\textbf{好量子数(good quantum number)}. 好量子数的这个定义有些模糊, 它强调的不是 $n$ 的数值, 而是能够唯一确定 $\psi_n$ 的某个物理量.

现在对\autoref{TIPT_eq2} 考虑对角块, 有 $E_n^0 = E_m^0$, 所以同样有\autoref{TIPT_eq6}, 只是现在这就是对角块中的对角元, 也就是 $H^1$ 在该简并子空间中的本征值. 在同一个简并子空间中, 为了简洁我们也可以令 $\braket{\psi_m^0}{\psi_n^1} = 0$.

在非对角子空间中, 有 $E_n^0 \ne E_m^0$, 所以同样有\autoref{TIPT_eq4} 和\autoref{TIPT_eq5}. 但注意此时 $E_n^0 \ne E_m^0$ 只是 $m \ne n$ 的充分非必要条件, 也就是更强的条件.

同样容易验证正交归一性\autoref{TIPT_eq7} 成立. 另外给\autoref{TIPT_eq5} 再加上若干 $c_m \psi_m^0$ ($E_m = E_n$) 同样可以, 但新增的系数需要符合正交归一化要求(\autoref{TIPT_eq8}).

\subsection{渐进近似}
接下来考虑一个应用问题: 含时薛定谔方程中如果开始时波函数处于任意一个态 $\psi$, 然后非常缓慢地增加 $H'$ 的强度, $\psi$ 会如何变化? 是否先分解为好量子态的线性组合, 然后再对修正后的好量子态做同样的线性组合?

渐进近似(链接未完成)告诉我们, 如果 $\psi$ 是一个好量子态, 那么这是可以的(参考\cite{GriffQ}). 那么既然好量子态是完备的, 含时薛定谔方程又是线性的, 上面的猜测的确成立.
