% 中国科学院 2014 年考研普通物理
% 中科院|考研|普通物理

\subsection{选择题}
1.地面上有一固定点电荷 $A$, $A$ 的正上方有另一带同种电荷的质点 $B$,在重力和库仑排斥力作用下, $B$ 在 $A$ 的正上方 $h$ 到 $h/2$ 高度间往返运动.则 $B$ 的最大运动速率为\\
A. $(\sqrt{2}+1) \sqrt{g h}\quad$
B. $(\sqrt{2}+1) \sqrt{2 g h}\quad$
C. $(\sqrt{2}-1) \sqrt{g h}\quad$
D. $(\sqrt{2}-1) \sqrt{2 g h}$

2.如\autoref{CAS14_fig7} 所示,不计质量的细杆组成一个等腰直角三角形,各顶点上固定一个质量为 $m$ 的小球,此三角形的直边边长为 $l$ ,则该系统对过质点且与三角形平面垂直的固定轴的转动惯量为\\
A.$\frac{4}{3} m l^{2}\quad$
B.$\frac{11}{9} \mathrm{~m} l^{2}\quad$
C.$\frac{10}{9} m l^{2}\quad$
D.$ml^{2}$
\begin{figure}[ht]
\centering
\includegraphics[width=4cm]{./figures/CAS14_7.pdf}
\caption{选择题第1题图示} \label{CAS14_fig7}
\end{figure}
3.一个人在大而滑的墙前,手里拿着一个频率 $500\mathrm{Hz}$ 的音叉,以速度 $1\mathrm{m/s}$ 向墙壁前进,他同时听到直接由音叉发出的声音和由墙壁反射回来的声音.在听觉上会感到音量有周期性的强弱变化.这一强一弱的现象称为拍.如果空气中的声速为 $334\mathrm{m/s}$ .问拍的频率为?\\
A.$1.5\mathrm{Hz}\quad$ B.$3\mathrm{Hz}\quad$ C.$5\mathrm{Hz}\quad$ D.$10\mathrm{Hz}$

4.一卡诺热机在温度为 $T_1$ 的高温热源和温度为 $T_2$ 的低温热源之间工作.当对环境做功为 $W$ 时,系统从高温热源吸收的热量记为 $Q_1$ ,像低温热源放出的热量记为 $Q_2$,下面的表达式正确的是:\\

A.$Q_1=\frac{T_1W}{T_1+T_2}\quad$ B.$Q_1=\frac{T_2W}{T_1+T_2}\quad$ C.$Q_2=\frac{T_1W}{T_1-T_2}\quad$ D.$Q_2=\frac{T_2W}{T_1-T_2}$

5.一电导率为 $\sigma$ 的立方体导体,初始时间均匀分布着体密度为 $\rho$ 的电荷.若忽略电磁感应效应,经过一段时间稳定后,下面说法不正确的是\\
A.导体内部电荷随时间减少,直至为零\\
B.导体表面电荷分布不均匀\\
C.导体外部电场强度不变\\
D.整个系统的静电能变小

6.指出在不同介质界面上,关于电场强度 $\bvec{E}$ 和电位移矢量 $\bvec{D}$ 的静电边界条件正确的是\\
A.$\bvec E$ 在界面两侧垂直于界面方向上的分量相等\\
B.$\bvec{D}$ 在界面两侧平行于界面方向上的分量相等\\
C.$\bvec E$ 在界面两侧平行于界面方向上的分量相等\\
D.$\bvec E$ 和 $\bvec D$ 在界面两侧平行于界面方向上的分量都相等

7.某原子的某一能态在磁场中分裂为三条,该能态的总角动量 $J$ 的量子数是\\
A.$1\quad $ B.$2 \quad$ C.3 $\quad$ D.4

8.磁通量 $\Phi_B$ 在国际单位制中的量纲是\\
A.$\mathrm{LMT^{-2}I^{-2}}\quad$ B.$\mathrm{L^{2}MT^{-2}I^{-1}}\quad$ C.$\mathrm{L^{2}MT^{-1}I^{-1}}\quad$ D.$\mathrm{LMT^{-1}I^{-1}}$

\subsection{简答题}
1.如\autoref{CAS14_fig6} 所示,一只圆桶中有一重物,其上、下两端都连有轻质弹簧,弹簧分别与桶的上下底相连,圆桶的质量与重物相等.开始时,桶和重物都静止不动地放置在一个高台上,然后将高台突然撤掉.使系统自由下落,不计空气阻力,问此后观察到地圆桶如何运动?并画出下落高度随时间变化的示意图.
\begin{figure}[ht]
\centering
\includegraphics[width=3cm]{./figures/CAS14_6.pdf}
\caption{简答题第1题图示} \label{CAS14_fig6}
\end{figure}

2.用恒流电源给一个线圈充电.若突然断电后,请问磁场会瞬间消失吗?并解释原因.


3.请说明夫琅禾夫衍射和菲涅尔衍射所使用的实验方案有何不同?


\subsection{解答题}
1.如\autoref{CAS14_fig5} 所示,离地面高 $H$ 处有一质量为 $M$ ,半径为 $R$ 的匀质飞轮,以角速度 $\omega_0$ 绕其中心竖直轴无摩擦旋转.某时刻,有一质量为 $m$ 的小 碎片从飞轮边缘飞出.\\
(1) 计算碎片飞出前后,飞轮的角速度、角动量和动能的改变量.\\
(2) 计算碎片落点距竖直轴的距离.
\begin{figure}[ht]
\centering
\includegraphics[width=3.3cm]{./figures/CAS14_5.pdf}
\caption{解答题第1题图示} \label{CAS14_fig5}
\end{figure}

2.如\autoref{CAS14_fig4} 所示,一块厚度为 $d$ 的匀质平板长为 $l$ ,放在一个半径为 $R$ 的固定圆柱上,当板水平放置时,其在重心刚好在圆柱中心轴的正上方.已知平板和圆柱间的摩擦系数为 $\mu$,且 $I>\pi R$\\
(1) 当板受到小扰动后,即板的倾角 $\beta$ 非常小.求板的振动周期.\\
(2) 当板在逆时针方向的力偶矩作用下维持逆时针倾斜状态时,求板能维持该状态的倾斜角 $\beta_0$ 需要满足的条件.\\
提示:力偶就是作用于同一物体上的一对大小相等、方向相反、但不共线的一对平行力.力偶矩就是力偶产生的力矩.
\begin{figure}[ht]
\centering
\includegraphics[width=5cm]{./figures/CAS14_4.pdf}
\caption{解答题第2题图示} \label{CAS14_fig4}
\end{figure}

3.如\autoref{CAS14_fig3} 所示,真空中一均匀带电圆环,总电量为 $Q(Q>0)$,圆环半径为 $R$,厚度忽略不计,通过圆环中心且垂直于环面的坐标轴为 $z$ 轴.圆环中心为原点 $O$.\\
(1) 求 $z$ 轴上各点的电势分布和电场强度分布.\\
(2) 如果该圆环绕 $z$ 轴匀速转动,角速度为 $\omega$ .求 $z$ 轴上各点磁感应强度.\\
(3) 设圆环如题 (2) 的速度转动时,一带电为 $-q(q>0)$ ,质量为 $m$ 的粒子从 $z$ 轴无穷远处静止释放后,忽略电磁辐射,问所能达到的最大速度是多少?(忽略电磁力以外的其他作用力)
\begin{figure}[ht]
\centering
\includegraphics[width=3cm]{./figures/CAS14_3.pdf}
\caption{解答题第3题图示} \label{CAS14_fig3}
\end{figure}

4.如\autoref{CAS14_fig2} 所示,已知一根中空长直的同轴电缆由半径为 $r_1$ 的实心圆柱导体和半径为 $r_2$ 的导体薄圆桶组成($r_1<r_2$).在同轴电缆的实心圆柱导体一端接电阻 $R$.另一端加一电压 $U$ 构成一个回路.忽略电缆的电阻.设电缆内介电常数和磁导率分别为 $\epsilon_0$ 和 $\mu_0$.\\
(1) 求电阻内部区域 $r_1<r<r_2$ 的磁感应强度 $\bvec{B}$ 和电场强度 $\bvec{E}$.用 $U,r_1,r_2,R$ 表示.
(2) 取同轴电缆一段长为 $l$ 的区域,求该段内区域 $r_1<r<r_2$ 中的电场能量和磁场能量.
\begin{figure}[ht]
\centering
\includegraphics[width=5cm]{./figures/CAS14_2.pdf}
\caption{解答题第4题图示} \label{CAS14_fig2}
\end{figure}

5.制造半导体元件时,常常要精确测量硅片 (Si) 上二氧化硅 ($\mathrm{SiO_2}$) 的厚度,可以用等厚条纹测其厚度,具体方法是:把二氧化硅薄膜的一部分腐蚀掉,是其形成劈尖,如下\autoref{CAS14_fig1} 所示.已知硅的折射率为 3.42 ,二氧化硅的折射率为 1.5 ,空气的折射率为1.如果入射光边长为 $500\mathrm{nm}$ ,观测到 5 条暗纹.问二氧化硅 ($SiO_2$) 薄膜的厚度 $h$ 是多少?
\begin{figure}[ht]
\centering
\includegraphics[width=4cm]{./figures/CAS14_1.pdf}
\caption{解答题第5题图示} \label{CAS14_fig1}
\end{figure}