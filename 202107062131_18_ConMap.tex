% 巴拿赫不动点定理

\pentry{
度量空间中的概念\upref{Metri2}
完备度量空间\upref{cauchy}
}

\textbf{巴拿赫不动点定理 (Banach fixed point theorem)} 又称作\textbf{压缩映像原理 (contraction mapping principle)}. 它是完备度量空间理论中的基本定理, 在分析数学的诸多分支中均有应用.

\subsection{两个著名例子}
\subsubsection{落在地面上的地图}
这是数学科普中常见的命题: \textbf{将一座公园的地图铺开在公园地面上, 则地面上恰有唯一一点与地图上对应的点重合.} 借助一点线性代数知识, 这个命题是不难验证的.

\subsubsection{函数的迭代}

\begin{theorem}{巴拿赫不动点定理}
设$(X,d)$是完备度量空间, $T:X\to X$是连续映射. 如果存在一个数$q\in(0,1)$使得$d(Tx,Ty)\leq qd(x,y)$对于任意的$x,y\in X$都成立, 那么
\end{theorem}
