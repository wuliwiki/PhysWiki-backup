% 普朗克常数(综述)
% license CCBYSA3
% type Wiki

本文根据 CC-BY-SA 协议转载翻译自维基百科\href{https://en.wikipedia.org/wiki/Planck_constant}{相关文章}。

普朗克常数,记作\( h \),是量子力学中一个具有基础性重要性的物理常数:一个光子的能量等于其频率与普朗克常数的乘积,物质波的波长等于普朗克常数除以相关粒子的动量。与之密切相关的约化普朗克常数,等于\( \frac{h}{2\pi} \),通常用\( \hbar \)表示,在量子物理方程中广泛使用。

这个常数由马克斯·普朗克在1900年提出,作为解释实验黑体辐射所需的比例常数。\(^\text{[2]}\)普朗克后来将这个常数称为“作用量子”。\(^\text{[3]}\)1905年,阿尔伯特·爱因斯坦将“量子”或能量的最小元素与电磁波本身联系起来。马克斯·普朗克因其“通过发现能量量子对物理学发展的贡献”而获得了1918年诺贝尔物理学奖。

在计量学中,普朗克常数与其他常数一起,用来定义千克,即质量的国际单位制(SI)单位。\(^\text{[4]}\)国际单位制的定义方式使得当普朗克常数以国际单位制单位表示时,其精确值为\( h = 6.62607015 \times 10^{-34} \, \text{J} \cdot \text{Hz}^{-1} \)。\(^\text{[5][6]}\)
\subsection{历史}  
\subsubsection{常数的起源}
\begin{figure}[ht]
\centering
\includegraphics[width=6cm]{./figures/ef10d79012cb91de.png}
\caption{柏林洪堡大学的铭牌:“马克斯·普朗克,作用量子\(h\)的发现者,从1889年到1928年曾在这座建筑中授课。”} \label{fig_PLKCS_1}
\end{figure}
普朗克常数是作为马克斯·普朗克成功努力的一部分提出的,普朗克试图推导出一个数学表达式,准确预测从封闭炉子(黑体辐射)中观察到的热辐射谱分布。\(^\text{[7]}\)这个数学表达式现在被称为普朗克定律。

在19世纪的最后几年,马克斯·普朗克正在研究黑体辐射的问题,这个问题大约40年前由基尔霍夫首次提出。每个物体都会自发地、持续地发射电磁辐射。然而,观察到的辐射光谱的整体形状并没有表达式或解释。当时,维恩定律能够很好地拟合短波长和高温的数据,但对于长波长则失败了。\(^\text{[7]: 141}\) 同时,尽管普朗克尚不知情,雷利勋爵已经理论推导出了一个公式,现在称为雷利-简定律,该公式能够合理地预测长波长,但在短波长处表现得非常不准确。
\begin{figure}[ht]
\centering
\includegraphics[width=10cm]{./figures/90dffea43d212d35.png}
\caption{黑体辐射的光强度。每条曲线表示不同体温下的行为。普朗克常数 \( h \) 用于解释这些曲线的形状。} \label{fig_PLKCS_2}
\end{figure}
普朗克试图找到一个数学表达式,既能重现维恩定律(适用于短波长),又能重现经验公式(适用于长波长)。这个表达式包括一个常数\(h\),这个常数被认为是 Hilfsgröße(辅助量),\(^\text{[8]}\)并随后成为了普朗克常数。普朗克所提出的表达式表明,某物体在绝对温度\(T\)下,对于频率\(\nu\)的频率单位辐射光谱辐照度为:
\[
B_{\nu}(\nu, T) d\nu = \frac{2h\nu^3}{c^2} \frac{1}{e^{\frac{h\nu}{k_{\mathrm{B}}T}} - 1} d\nu,~
\]
其中\(k_{\text{B}}\)是玻尔兹曼常数,\(h\)是普朗克常数,\(c\)是介质中的光速,无论是物质还是真空。\(^\text{[9][10][11]}\)

普朗克很快意识到他的解并不是唯一的。存在几种不同的解,每种解给出的振荡器熵值都不同。\(^\text{[2]}\)为了挽救他的理论,普朗克采用了当时颇具争议的统计力学理论,\(^\text{[2]}\)他将其描述为“一种绝望的尝试”。他的一个新边界条件是:

将\( U_N \)[‘N个振荡器的振动能量’] 解释为不是一个连续的、可以无限分割的量,而是作为一个由有限数量相等部分组成的离散量。我们称每一部分为能量元素\(\varepsilon\);

——普朗克,《关于正常谱中能量分布的定律》\(^\text{[2]}\)

通过这一新条件,普朗克对振荡器的能量进行了量子化,正如他自己所说,“这只是一个纯粹的形式假设……实际上我并没有多想”,\(^\text{[13]}\)但这个假设却将彻底改变物理学。将这一新方法应用于维恩位移定律,表明“能量元素”必须与振荡器的频率成正比,这就是现在有时被称为“普朗克-爱因斯坦关系”的第一个版本:
\[
E = hf.~
\]
普朗克能够通过黑体辐射的实验数据计算出普朗克常数\(h\)的值:他的结果为\( 6.55 \times 10^{-34} \, \text{J} \cdot \text{s}\),与当前定义的值相差仅1.2\%。\(^\text{[2]}\) 他还通过相同的数据和理论首次确定了玻尔兹曼常数\(k_{\text{B}}\)的值。\(^\text{[14]}\)
\subsubsection{发展与应用}
\begin{figure}[ht]
\centering
\includegraphics[width=10cm]{./figures/cac866e2a45269bc.png}
\caption{} \label{fig_PLKCS_3}
\end{figure}