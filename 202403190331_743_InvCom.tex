% 原码、反码、补码
% keys 编码|位运算|ALU|计算机组成原理
% license Usr
% type Wiki
\begin{issues}
\issueDraft
\end{issues}

\pentry{数字电路_运算器\nref{nod_Sample}}{nod_eb6a}

\subsection{原码(True form)}

原码、反码、补码都是二进制编码方式。一般的编码都包括符号位和数值位。通常地,符号位使用最高位表示,其中0代表正数,1代表负数。

原码即“未经更改”的码,是指一个二进制数左边加上符号位后所得到的码,且当二进制数大于0时,符号位为0;二进制数小于0时,符号位为1;二进制数等于0时,符号位可以为0或1(+0/-0)。

使用n位原码表示\textbf{有符号数}时,范围是 $-(2^{n-1}-1)\sim +(2^{n-1}-1)$  

\subsubsection{优点}

由于其直观的表示方法,原码易于人类理解和计算,但在进行算术运算时会

\subsubsection{缺点}
遇到一些问题,如正负零的歧义和电路复杂性。

\subsection{反码}

反码同样用于表示整数,特别是在计算机系统内部处理负数时。在反码表示法中,正数的反码与其原码相同;负数的反码则是将原码(除符号位外)的每一位取反。反码解决了一些原码在运算上的问题,但仍然存在如负零的表示以及加法运算中的进位问题。

\subsection{补码}

\subsubsection{优点——为什么要使用补码}






% 参考 https://www.cnblogs.com/zhangziqiu/archive/2011/03/30/computercode.html remove by lzq
