% 2012 年考研数学试题(数学一)
% keys 考研|数学
% license Copy
% type Tutor
\subsection{选择题}
\begin{enumerate}
\item 曲线 $\displaystyle y=\frac{x^2+x}{x^2-1}$ 的渐近线的条数为 ($\quad$)\\
(A) $0$\\
(B) $1$\\
(C) $2$\\
(D) $3$
\item 设函数 $f(x)=(e^x-1)(e^{2x}-2)\dots(e^{nx}-n)$ ,其中 $n$ 为正整数,则 $f'(0)$=($\quad$)\\
(A)$(-1)^{n-1}(n-1)!$\\
(B)$(-1)^n(n-1)!$\\
(C)$(-1)^{n-1}n!$\\
(D) $(-1)^n n!$
\item 如果函数 $f(x,y)$ 在点$ (0,0) $处连续,那么下列命题正的是($\quad$)\\
(A)若极限 $\displaystyle \lim_{\substack {x\to0 \\ y\to 0}}\frac{f(x,y)}{\abs{x}+\abs{y}}$ 存在,则 $f(x,y)$ 在点  $(0,0)$ 处可微。\\
(B)若极限 $\displaystyle \lim_{\substack {x\to0 \\ y\to 0}}\frac{f(x,y)}{x^2+y^2}$ 存在,则 $f(x,y)$ 在点  $(0,0)$ 处可微。\\
(C)若 $f(x,y)$ 在点  $(0,0)$ 处可微,则极限 $\displaystyle \lim_{\substack {x\to0 \\ y\to 0}}\frac{f(x,y)}{\abs{x}+\abs{y}}$ 存在,\\
(D)若 $f(x,y)$ 在点  $(0,0)$ 处可微。


(A)
(B)
(C)
(D)


(A)
(B)
(C)
(D)

(A)
(B)
(C)
(D)
\end{enumerate}
