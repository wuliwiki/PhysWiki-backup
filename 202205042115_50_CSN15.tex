% 2015 年计算机学科专业基础综合全国联考卷
% keys 2015 计算机 考研 真题 全国卷

\subsection{一、单项选择题}
\textbf{1~40小题,每小题2分,共80分.下列每题给出的四个选项中.只有一个选项符合题目要求.}

1.已知程序如下: \\
\begin{lstlisting}[language=cpp]
int S(int n)
{
    return(n<=0)?0:s(n-1)+n;
}

void main()
{ 
    cout<<S(1);
}
\end{lstlisting}
程序运行时使用栈来保存调用过程的信息,自栈底到栈顶保存的信息依次对应的是 \\
A.main( )→S(1)→S(0) $\quad$ B.S(0)→S(1)→main( ) \\
C.main( )→S(0)→S(1) $\quad$ D.S(1)→S(0)→main( )

2.先序序列为a,b,c,d的不同二叉树的个数是 \\
A.13 $\quad$ B.14 $\quad$ C.15 $\quad$ D.16

3.下列选项给出的是从根分别到达两个叶结点路径上的权值序列,能属于同一棵哈夫曼树的是 \\
A.24,10,5和24,10,7 $\quad$ B.24,10,5和24,12,7 \\
C.24,10,10和24,14,11 $\quad$ D.24,10,5和24,14,6

4.现有一棵无重复关键字的平衡二叉树(AVL树),对其进行中序遍历可得到一个降序序列.下列关于该平衡二叉树的叙述中,正确的是 \\
A.根结点的度一定为2 $\quad$ B.树中最小元素一定是叶结点 \\
C.最后插入的元素一定是叶结点 $\quad$ D.树中最大元素一定无左子树

5.设有向图G=(V,E),顶点集V={v0,v1,v2,v3},边集E:{<v0,v1>,<v0,v2>,<v0,v3>,<v1,v3>}.若从顶点v0.开始对图进行深度优先遍历,则可能得到的不同遍历序列个数是 \\
A.2 $\quad$ B.3 $\quad$ C.4 $\quad$ D.5

6.求下面带权图的最小(代价)生成树时,可能是克鲁斯卡尔(Kruskal)算法第2次选中但不.是普里姆(Prim)算法(从v4开始)第2次选中的边是
\begin{figure}[ht]
\centering
\includegraphics[width=10cm]{./figures/CSN15_1.png}
\caption{第6题图} \label{CSN15_fig1}
\end{figure}
A.(v1,v3) $\quad$ B.(v1,v4) $\quad$ C.(v2,v3) $\quad$ D.(v3,v4)

7.下列选项中\textbf{不能}构成折半查找中关键字比较序列的是 \\
A.500,200,450,180 $\quad$ B.500,450,200,180 \\
C.180,500,200,450 $\quad$ D.180,200,500,450

8.已知字符串s为“abaabaabacacaabaabcc”,模式串t为“abaabc5’.采用KMP算法进行匹配,第一次出现“失配”(s[i]≠t[j])时,i=j=5,则下次开始匹配时,i和j的值分别是 \\
A.i=1,j=0 $\quad$ B.i=5,j=0 \\
C.i=5,j=2 $\quad$ D.i=6,j=2

9.下列排序算法中,元素的移动次数与关键字的初始排列次序无关的是 \\
A.直接插入排序 $\quad$ B.起泡排序 \\
C.基数排序 $\quad$ D.快速排序

10.已知小根堆为8,15,10,21,34,16,12,删除关键字8之后需重建堆,在此过程中,关键字之间的比较次数是 \\
A.1 $\quad$ B.2 $\quad$ C.3 $\quad$ D.4

11.希尔排序的组内排序采用的是 \\
A.直接插入排序 $\quad$ B.折半插入排序 \\
C.快速排序 $\quad$ D.归并排序

12.计算机硬件能够直接执行的是 \\
Ⅰ.机器语言程序 $\quad$ Ⅱ.汇编语言程序 $\quad$ Ⅲ.硬件描述语言程序 \\
A.仅Ⅰ $\quad$ B.仅Ⅰ、Ⅱ \\
C.仅Ⅰ、Ⅲ $\quad$ D.Ⅰ、Ⅱ、Ⅲ

13.由3个“1”和5个“0”组成的8位二进制补码,能表示的最小整数是 \\
A.-126 $\quad$ B.-125 \\
C.-32 $\quad$ D.-3

14.下列有关浮点数加减运算的叙述中,正确的是 \\
Ⅰ.对阶操作不会引起阶码上溢或下溢 \\
Ⅱ.右规和尾数舍入都可能引起阶码上溢 \\
Ⅲ.左规时可能引起阶码下溢 \\
Ⅳ.尾数溢出时结果不一定溢出 \\
A.仅Ⅱ、Ⅲ $\quad$ B.仅Ⅰ、Ⅱ、Ⅳ \\
C.仅Ⅰ、Ⅲ、Ⅳ $\quad$ D.Ⅰ、Ⅱ、Ⅲ、Ⅳ

15.假定主存地址为32位,按字节编址,主存和Cache之间采用直接映射方式,主存块大小为4个字,每字32位,采用回写(Write Back)方式,则能存放4K字数据的Cache的总容量的位数至少是 \\
A.146K $\quad$ B.147K \\
C.148K $\quad$ D.158K

16.假定编译器将赋值语句“x=x+3;”转换为指令“add xaddr,3”,其中,xaddr是x对应的存储单元地址.若执行该指令的计算机采用页式虚拟存储管理方式,并配有相应的TLB,且Cache使用直写(Write Through)方式,则完成该指令功能需要访问主存的次数至少是 \\
A.0 $\quad$ B.1 $\quad$ C.2 $\quad$ D.3

17.下列存储器中,在工作期间需要周期性刷新的是 \\
A.SRAM $\quad$ B.SDRAM $\quad$ C.ROM $\quad$ D.FLASH

18.某计算机使用4体交叉编址存储器,假定在存储器总线上出现的主存地址(十进制)序列为8005,8006,8007,8008,8001,8002,8003,8004,8000,则可能发生访存冲突的地址对是 \\
A.8004和8008 $\quad$ B.8002和8007 \\
C.8001和8008 $\quad$ D.8000和8004

19.下列有关总线定时的叙述中,错误的是 \\
A.异步通信方式中,全互锁协议的速度最慢 \\
B.异步通信方式中,非互锁协议的可靠性最差 \\
C.同步通信方式中,同步时钟信号可由各设备提供 \\
D.半同步通信方式中,握手信号的采样由同步时钟控制

20.若磁盘转速为7200转/分,平均寻道时间为8ms,每个磁道包含1000个扇区,则访问一个扇区的平均存取时间大约是 \\
A.8.1 ms $\quad$ B.12.2 ms \\
C.16.3 ms $\quad$ D.20.5 ms

21.在采用中断I/O方式控制打印输出的情况下,CPU和打印控制接口中的I/O端口之间交换的信息不可能是 \\
A.打印字符 $\quad$ B.主存地址 \\
C.设备状态 $\quad$ D.控制命令

22.内部异常(内中断)可分为故障(fault)、陷阱(trap)和终止(abort)三类.下列有关内部异常的叙述中,错误的是 \\
A.内部异常的产生与当前执行指令相关 \\
B.内部异常的检测由CPU内部逻辑实现 \\
C.内部异常的响应发生在指令执行过程中 \\
D.内部异常处理后返回到发生异常的指令继续执行

23.处理外部中断时,应该由操作系统保存的是 \\
A.程序计数器(PC)的内容 $\quad$ B.通用寄存器的内容 \\
C.快表(TLB)中的内容 $\quad$ D.Cache中的内容

24.假定下列指令已装入指令寄存器,则执行时\textbf{不可能}导致CPU从用户态变为内核态(系统态)的是 \\
A.DIV R0,R1 ;(R0)/(R1)→R0 \\
B.INT n ; 产生软中断 \\
C.NOT R0 ;寄存器R0的内容取非 \\
D.MOV R0,addr;把地址addr处的内存数据放入寄存器R0中

25.下列选项中,会导致进程从执行态变为就绪态的事件是 \\
A.执行P(wait)操作 $\quad$ B.申请内存失败 \\
C.启动I/O设备 $\quad$ D.被高优先级进程抢占

26.若系统S1采用死锁避免方法,S2采用死锁检测方法.下列叙述中,正确的是 \\
Ⅰ.S1会限制用户申请资源的顺序,而S2不会 \\
Ⅱ.S1需要进程运行所需资源总量信息,而S2不需要 \\
Ⅲ.S1不会给可能导致死锁的进程分配资源,而S2会 \\
A.仅Ⅰ、Ⅱ $\quad$ B.仅Ⅱ、Ⅲ $\quad$ C.仅Ⅰ、Ⅲ $\quad$ D.Ⅰ、Ⅱ、Ⅲ

27.系统为某进程分配了4个页框,该进程已访问的页号序列为2,0,2,9,3,4,2,8,2,4,8,4,5.若进程要访问的下一页的页号为7,依据LRU算法,应淘汰页的页号是 \\
A.2 $\quad$ B.3 $\quad$ C.4 $\quad$ D.8

28.在系统内存中设置磁盘缓冲区的主要目的是 \\
A.减少磁盘I/O次数 $\quad$ B.减少平均寻道时间 $\quad$ C.提高磁盘数据可靠性 $\quad$ D.实现设备无关性

29.在文件的索引节点中存放直接索引指针10个,一级和二级索引指针各1个.磁盘块大小为1KB,每个索引指针占4个字节.若某文件的索引节点已在内存中,则把该文件偏移量(按字节编址)为1234和307400处所在的磁盘块读入内存,需访问的磁盘块个数分别是 \\
A.1、2 $\quad$ B.1、3 $\quad$ C.2、3 $\quad$ D.2、4

30.在请求分页系统中,页面分配策略与页面置换策略不.能组合使用的是 \\
A.可变分配,全局置换 $\quad$ B.可变分配,局部置换 \\
C.固定分配,全局置换 $\quad$ D.固定分配,局部置换

31.文件系统用位图法表示磁盘空间的分配情况,位图存于磁盘的32~127号块中,每个盘块占1024个字节,盘块和块内字节均从0开始编号.假设要释放的盘块号为409612,则位图中要修改的位所在的盘块号和块内字节序号分别是 \\
A.81、1 $\quad$ B.81、2 $\quad$ C.82、1 $\quad$ D.82、2

32.某硬盘有200个磁道(最外侧磁道号为0),磁道访问请求序列为:130,42,180,15,199,当前磁头位于第58号磁道并从外侧向内侧移动.按照SCAN调度方法处理完上述请求后,磁头移过的磁道数是 \\
A.208 $\quad$ B.287 $\quad$ C.325 $\quad$ D.382

33.通过POP3协议接收邮件时,使用的传输层服务类型是 \\
A.无连接不可靠的数据传输服务 \\
B.无连接可靠的数据传输服务 \\
C.有连接不可靠的数据传输服务 \\
D.有连接可靠的数据传输服务

34.使用两种编码方案对比特流01100111进行编码的结果如下图所示,编码1和编码2分别是
\begin{figure}[ht]
\centering
\includegraphics[width=12.5cm]{./figures/CSN15_2.png}
\caption{第34题图} \label{CSN15_fig2}
\end{figure}
A.NRZ和曼彻斯特编码 $\quad$ B.NRZ和差分曼彻斯特编码 \\
C.NRZI和曼彻斯特编码 $\quad$ D.NRZI和差分曼彻斯特编码

35.主机甲通过128 kbps卫星链路,采用滑动窗口协议向主机乙发送数据,链路单向传播延迟为250ms,帧长为1000字节.不考虑确认帧的开销,为使链路利用率不小于80%,帧序号的比特数至少是 \\
A.3 $\quad$ B.4 $\quad$ C.7 $\quad$ D.8

36.下列关于CSMA/CD协议的叙述中,\textbf{错误}的是 \\
A.边发送数据帧,边检测是否发生冲突 \\
B.适用于无线网络,以实现无线链路共享 \\
C.需要根据网络跨距和数据传输速率限定最小帧长 \\
D.当信号传播延迟趋近0时,信道利用率趋近100%

37.下列关于交换机的叙述中,正确的是 \\
A.以太网交换机本质上是一种多端口网桥 \\
B.通过交换机互连的一组工作站构成一个冲突域 \\
C.交换机每个端口所连网络构成一个独立的广播域 \\
D.以太网交换机可实现采用不同网络层协议的网络互联

38.某路由器的路由表如下表所示:
\begin{table}[ht]
\centering
\caption{第38题表}\label{CSN15_tab1}
\begin{tabular}{|c|c|c|}
\hline
目的网络 & 下一跳 & 接口 \\
\hline
169.96.40.0/23 & 176.1.1.1 & S1 \\
\hline
169.96.40.0/25 & 176.2.2.2 & S2 \\
\hline
169.96.40.0/27 & 176.3.3.3 & S3 \\
\hline
0.0.0.0/0 & 176.4.4.4 & S4 \\
\hline
\end{tabular}
\end{table}
若路由器收到一个目的地址为169.96.40.5的IP分组,则转发该IP分组的接口是 \\
A.S1 $\quad$ B.S2 $\quad$ C.S3 $\quad$ D.S4

39.主机甲和主机乙新建一个TCP连接,甲的拥塞控制初始阈值为32 KB,甲向乙始终以MSS=1 KB大小的段发送数据,并一直有数据发送;乙为该连接分配16 KB接收缓存,并对每个数据段进行确认,忽略段传输延迟.若乙收到的数据全部存入缓存,不被取走,则甲从连接建立成功时刻起,未发生超时的情况下,经过4个RTT后,甲的发送窗口是 \\
A.1 KB $\quad$ B.8KB $\quad$ C.16KB $\quad$ D.32KB

40.某浏览器发出的HTTP请求报文如下: \\
\begin{lstlisting}
GET/index.html HTTP/1.1
Host: www.test.edu.cn
Connection : Close
Cookie: 123456
\end{lstlisting}
下列叙述中,\textbf{错误}的是 \\
A.该浏览器请求浏览index.html \\
B.index,html存放在www.test.edu.cn上 \\
C.该浏览器请求使用持续连接 \\
D.该浏览器曾经浏览过www.test.edu.cn

\subsection{二、综合应用题}
\textbf{41~47小题,共70分.}

41.(15分)用单链表保存m个整数,结点的结构为:data,link,且|data|≤n(n为正整数).现要求设计一个时间复杂度尽可能高效的算法,对于链表中data的绝对值相等的结点,仅保留第一次出现的结点而删除其余绝对值相等的结点.例如,若给定的单链表head如下:
\begin{figure}[ht]
\centering
\includegraphics[width=14.25cm]{./figures/CSN15_3.png}
\caption{第41题图1} \label{CSN15_fig3}
\end{figure}
则删除结点后的head为:
\begin{figure}[ht]
\centering
\includegraphics[width=12.5cm]{./figures/CSN15_4.png}
\caption{第42题图2} \label{CSN15_fig4}
\end{figure}
要习之: \\
(1)给出算法的基本设计思想. \\
(2)使用C或C++语言,给出单链表结点的数据类型定义. \\
(3)根据设计思想,采用C或C++语言描述算法,关键之处给出注释. \\
(4)说明你所设计算法的时间复杂度和空间复杂度.

42.(8分)已知含有5个顶点的图G如下图所示. \\
请回答下列问题. \\
(1)写出图G的邻接矩阵A(行、列下标均从0开始).
\begin{figure}[ht]
\centering
\includegraphics[width=10cm]{./figures/CSN15_5.png}
\caption{第42题图} \label{CSN15_fig5}
\end{figure}
(2)求 $A^2$,矩阵 $A^2$ 中位于 $0$ 行 $3$ 列元素值的含义是什么? \\
(3)若已知具有 $n(n\geqslant2)$ 个顶点的图的邻接矩阵为 $B$,则 $B^m$ $(2\leqslant m\leqslant n)$ 中非零元素的含义是什么?

43.(13分)某16位计算机的主存按字节编址,存取单位为16位;采用16位定长指令字格式;CPU采用单总线结构,主要部分如下图所示.图中R0~R3为通用寄存器;T为暂存器;SR为移位寄存器,可实现直送(mov)、左移一位(left)和右移一位(right)3种操作,控制信号为SRop,SR的输出由信号SRout控制;ALU可实现直送A(mova)、A加B(add)、A减B(sub)、A与B(and)、A或B(or)、非A(not)、A加l(ine)7种操作,控制信号为ALUop. \\
请回答下列问题. \\
(1)图中哪些寄存器是程序员可见的?为何要设置暂存器T? \\
(2)控制信号ALUop和SRop的位数至少各是多少?
\begin{figure}[ht]
\centering
\includegraphics[width=14.25cm]{./figures/CSN15_6.png}
\caption{第43题图} \label{CSN15_fig6}
\end{figure}
(3)控制信号SRout所控制部件的名称或作用是什么? \\
(4)端点①~⑨中,哪些端点须连接到控制部件的输出端? \\
(5)为完善单总线数据通路,需要在端点①~⑨中相应的端点之间添加必要的连线.写出连线的起点和终点,以正确表示数据的流动方向. \\
(6)为什么二路选择器MUX的一个输入端是2?

44.(10分)题43中描述的计算机,其部分指令执行过程的控制信号如题44图a所示.
\begin{figure}[ht]
\centering
\includegraphics[width=14.25cm]{./figures/CSN15_7.png}
\caption{第44题图a:部分指令的控制信号} \label{CSN15_fig7}
\end{figure}
该机指令格式如题44图b所示,支持寄存器直接和寄存器间接两种寻址方式,寻址方式位分别为0和1,通用寄存器R0~R3的编号分别为0、1、2和3.
\begin{figure}[ht]
\centering
\includegraphics[width=14.25cm]{./figures/CSN15_8.png}
\caption{第44题图b:指令格式} \label{CSN15_fig8}
\end{figure}
请回答下列问题.
(1)该机的指令系统最多可定义多少条指令? \\
(2)假定inc、shl和sub指令的操作码分别为01H、02H和03H,则以下指令对应的机器代码各是什么? \\
①inc R1 ;(R1)+1→R1 \\
②sh1 R2,R1 ;(R1)<<1→R2 \\
③sub R3,(R1),R2 ;((R1))-(R2)→R3 \\
(3)假设寄存器x的输入和输出控制信号分别记为Xin和Xout,其值为1表示有效,为0表示无效(例如,PCout=1表示PC内容送总线);存储器控制信号为MEMop,用于控制存储器的读(read)和写(write)操作.写出题44图a中标号①~⑧处的控制信号或控制信号取值. \\
(4)指令“sub R1,R3,(R2)”和“inc R1”的执行阶段至少各需要多少个时钟周期?

45.(9分)有A、B两人通过信箱进行辩论,每个人都从自己的信箱中取得对方的问题,将答案和向对方提出的新问题组成一个邮件放人对方的信箱中.假设A的信箱最多放M个邮件,B的信箱最多放Ⅳ个邮件.初始时A的信箱中有x个邮件(0<x<M),B的信箱中有y个邮件(0<y<N).辩论者每取出一个邮件,邮件数减1.A和B两人的操作过程描述如下: \\
\begin{lstlisting}[language=cpp]
CoBegin
A {
  while(TRUE) {
    从A的信箱中取出一个邮件;
    回答问题并提出一个新问题;
    将新邮件放人B的信箱;
  }
}
B {
  while(TRUE) {
    从B的信箱中取出一个邮件;
    回答问题并提出一个新问题;
    将新邮件放人A的信箱;
  }
}
CoEnd
\end{lstlisting}
$\qquad$ 当信箱不为空时,辩论者才能从信箱中取邮件,否则等待.当信箱不满时,辩论者才能将新邮件放入信箱,否则等待.请添加必要的信号量和P、V(或wait、signal)操作,以实现上述过程的同步.要求写出完整的过程,并说明信号量的含义和初值.


\subsection{参考答案}
\subsubsection{一、单项选择题}
1.A $\qquad$ 2.B $\qquad$ 3.D $\qquad$ 4.D $\qquad$ 5.D \\
6.C $\qquad$ 7.A $\qquad$ 8.C $\qquad$ 9.C $\qquad$ 10.C \\
11.A $\qquad$ 12.A $\qquad$ 13.B $\qquad$ 14.D $\qquad$ 15.C \\
16.B $\qquad$ 17.B $\qquad$ 18.D $\qquad$ 19.C $\qquad$ 20.B \\
21.B $\qquad$ 22.D $\qquad$ 23.B $\qquad$ 24.C $\qquad$ 25.D \\
26.B $\qquad$ 27.A $\qquad$ 28.A $\qquad$ 29.B $\qquad$ 30.C \\
31.C $\qquad$ 32.C $\qquad$ 33.D $\qquad$ 34.A $\qquad$ 35.B \\
36.B $\qquad$ 37.A $\qquad$ 38.C $\qquad$ 39.A $\qquad$ 40.C

\subsubsection{二、综合应用题}

41.【答案要点】 \\
(1)算法的基本设计思想 \\
算法的核心思想是用空间换时间.使用辅助数组记录链表中已出现的数值,从而只需对链表进行一趟扫描. \\
$\quad$ 因为|data|≤n,故辅助数组q的大小为n+1,各元素的初值均为0.依次扫描链表中的各结点,同时检查q[|data|]的值,如果为0,则保留该结点,并令q[|data|]=1;否则,将该结点从链表中删除. \\
$\quad$ (2)使用C语言描述的单链表结点的数据类型定义 \\
\begin{lstlisting}[language=cpp]
typedef struct node {
  int data:
  struct node *link:
} NODE;
typedef NODE*PNODE;
\end{lstlisting}


