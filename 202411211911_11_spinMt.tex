% 自旋角动量矩阵
% keys 自旋|角动量|矩阵|算符|轨道角动量|升降算符
% license Xiao
% type Tutor

\pentry{自旋, 简谐振子升降算符归一化\nref{nod_QSHOnr}}{nod_886d}%未完成: 自旋文章

我们已知 $1/2$ 自旋的矩阵可以用 $\hbar/2$ 乘以泡利矩阵得到。 以下我们试图计算任意 $n/2$ 自旋粒子的三个自旋分量算符 $\hat S_x$, $\hat S_y$ 和 $\hat S_z$ 对应的矩阵。 这些矩阵一般使用 $\hat S_z$ 的本征态 $\ket{s, m}$ 作为基底(另外两组基底同理可得)。 注意当 $s$ 为整数时, 本文的结论同样适用于轨道角动量(把算符 $S$ 和量子数 $s$ 分别替换为 $L$ 和 $l$ 即可)。

基本思路是先求出升降算符 $\hat S_\pm = \hat S_x \pm \I \hat S_y$ 的矩阵。 再把它们分别相加和相减得到 $\hat S_x$ 和 $\hat S_y$ 的矩阵。 $\hat S_z$ 算符在其本征态基底下是本征值组成的对角矩阵。

已知归一化系数为
\begin{equation}
\hat S_\pm \ket{s,m} = \hbar \sqrt{s(s + 1) - m(m \pm 1)} \ket{s,m\pm1}~,
\end{equation}
所以
\begin{equation}
\mel{s,m}{\hat S_\pm}{s,m'} = \delta_{m, m'\pm1} \hbar \sqrt{s(s + 1) - mm'}~.
\end{equation}
可见 $\hat S_+$ 只有下方的子对角线不为零, $\hat S_-$ 只有上方的子对角线不为零。

最后得三个矩阵为
\begin{equation}
\mel{s,m}{\hat S_x}{s,m'} = \frac12(\delta_{m, m'+1} + \delta_{m, m'-1})\hbar \sqrt{s(s + 1) - mm'} ~,
\end{equation}
\begin{equation}
\mel{s,m}{\hat S_y}{s,m'} = \frac{1}{2\I}(\delta_{m, m'+1} - \delta_{m, m'-1}) \hbar \sqrt{s(s + 1) - mm'}~,
\end{equation}
\begin{equation}
\mel{s,m}{\hat S_z}{s,m'} = \delta_{m,m'} m\hbar ~.
\end{equation}
可以验证当 $s = 1/2, m = \pm1/2$ 时, 我们就得到了在$\hat S_z$表象下的自旋角动量分量矩阵。如下所示:
\begin{equation}\label{eq_spinMt_1}
\hat S_x=\frac{\hbar}{2}\pmat{0&1\\1&0},\quad\hat S_y=\frac{\hbar}{2}\pmat{0&-\I\\\I&0},\quad\hat S_z=\frac{\hbar}{2}\pmat{0&1\\-1&0}~.
\end{equation}

\subsubsection{自旋算符的“态矢表示法”}
为了计算方便,有时候我们也会用“态矢”组合来表示自旋分量矩阵。设
\begin{equation}
\ket{+}=\pmat{1\\0},\ket{-}=\pmat{0\\1}~,
\end{equation}
显然这是$\hat \sigma_z$的两个本征向量。则二阶矩阵的基可以表示为
\begin{equation}
\begin{aligned}
\ket{+}\bra{-}&=\pmat{1\\0}\pmat{0&1}=\pmat{0&1\\0&0},\ket{-}\bra{+}=\pmat{0\\1}\pmat{1&0}=\pmat{0&0\\1&0}\\
\ket{+}\bra{+}&=\pmat{1\\0}\pmat{1&0}=\pmat{1&0\\0&0},\ket{-}\bra{-}=\pmat{0\\1}\pmat{0&1}=\pmat{0&0\\0&1}~.
\end{aligned}
\end{equation}
利用基向量表示自旋分量矩阵如下,
\begin{equation}
\hat S_{x}=\frac{\hbar}{2}(|+\rangle\langle-|+|-\rangle\langle+|), \quad \hat S_{y}=\frac{i \hbar}{2}(-|+\rangle\langle-|+|-\rangle\langle+|), \quad 
\hat S_{z}=\frac{\hbar}{2}(|+\rangle\langle+|-|-\rangle\langle-|)~.
\end{equation}
相应的,我们也可以用态矢表示$\hat S_x,\hat S_y$的特征向量。由\autoref{eq_spinMt_1} 可知在归一化后,$\hat S_x,\hat S_y$的特征向量为分别为:
\begin{equation}
\begin{aligned}
\ket {\hat S_x;\pm}&=\frac{1}{\sqrt{2}}\pmat{1\\1}=\frac{1}{\sqrt{2}}(\ket{+}\pm \ket{-}),\\
\ket{S_y;\pm}&=\frac{1}{\sqrt{2}}\pmat{1\\ \pm \I}=\frac{1}{\sqrt 2}(\ket{+}\pm \I \ket{-})~.
\end{aligned}
\end{equation}

