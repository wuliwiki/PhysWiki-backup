% Julia 数组的拷贝
% Julia 数组 拷贝

本文授权转载自郝林的 《Julia 编程基础》. 原文链接:\href{https://github.com/hyper0x/JuliaBasics/blob/master/book/ch10.md}{第 10 章 容器:数组(下)}.


\subsubsection{10.3 数组的拷贝}

在 Julia 中,任何的值都可以通过两种方式进行拷贝.一种方式是浅拷贝,另一种方式是深拷贝.数组也不例外.

函数\verb|copy|的功能是浅拷贝一个值.它只会拷贝原值的外层结构,然后把原结构中的各个内部对象原封不动地塞到这个新的结构中.也就是说,新的结构加上原有的内部对象就形成了新的值.

这些原有的内部对象可能是原语类型的值,也可能是复合类型的值,另外还可能是更加复杂的值,比如容器.就拿数组来说,它的内部对象通常指的是其中的元素值.至于它们有多复杂,就要看数组的元素类型是什么了.

对于元素类型属于原语类型的数组来说,从表面上看,浅拷贝与深拷贝并没有什么区别.例如:

\begin{lstlisting}[language=julia]
julia> array_orig1 = [2, 4, 6, 8, 10];

julia> array_copy1 = copy(array_orig1);

julia> array_copy1[1] = 12; show(array_copy1)
[12, 4, 6, 8, 10]

julia> show(array_orig1)
[2, 4, 6, 8, 10]

julia> 
\end{lstlisting}

我利用\verb|copy|函数得到了数组\verb|array_orig1|的复本,并将其赋给了\verb|array_copy1|变量.然后,我又使用索引表达式为这个复本中的第一个元素赋予了新的值.可以看到,即使我没有采用深拷贝,这种改变也不会影响到原数组\verb|array_orig1|.

再来看元素类型属于复合类型的数组.我们对其复本中的元素值进行替换仍然不会影响到原数组.不过,有一些改变却会给这类的原数组带来实质性的影响.请看下面的代码:

\begin{lstlisting}[language=julia]
julia> mutable struct Person
           name::String
           age::UInt8
           extra
       end

julia> p1 = Person("Robert", 37, "Beijing");
   p2 = Person("Andres", 32, "Madrid"); p3 = Person("Eric", 28, "Paris");

julia> array_orig2 = Person[p1, p2, p3]
3-element Array{Person,1}:
 Person("Robert", 0x25, "Beijing")
 Person("Andres", 0x20, "Madrid") 
 Person("Eric", 0x1c, "Paris")    

julia> array_copy2 = copy(array_orig2)
3-element Array{Person,1}:
 Person("Robert", 0x25, "Beijing")
 Person("Andres", 0x20, "Madrid") 
 Person("Eric", 0x1c, "Paris")    

julia> 
\end{lstlisting}

数组\verb|array_orig2|的元素类型是\verb|Person|,是一个可变的复合类型.该数组包含了 3 个元素值,分别是我刚刚定义的\verb|p1|、\verb|p2|和\verb|p3|.数组\verb|array_copy2|是\verb|array_orig2|的复本,同样是由\verb|copy|函数创建的.

我将要把\verb|array_copy2|中的第三个元素值替换掉.这显然会改变\verb|array_copy2|,但却不会对\verb|array_orig2|造成影响.如:

\begin{lstlisting}[language=julia]
julia> array_copy2[3] = Person("Mark", 45, "New York")
Person("Mark", 0x2d, "New York")

julia> show(array_orig2)
Person[Person("Robert", 0x25, "Beijing"),
   Person("Andres", 0x20, "Madrid"), Person("Eric", 0x1c, "Paris")]

julia> 
\end{lstlisting}

实际上,无论数组的元素类型是什么,通过索引表达式替换元素值的操作都不会对原数组产生任何的影响.但是,如果我们改变的是某个元素值的内部对象,那么结果就截然不同了.就像下面这样:

\begin{lstlisting}[language=julia]
julia> array_copy2[1].age = 38; Int(array_orig2[1].age)
38

julia> 
\end{lstlisting}

我修改了\verb|array_copy2|中的第一个元素值,把它的字段\verb|age|的值改成了\verb|38|.可以看到,这一操作使得\verb|array_orig2|中对应元素值里的\verb|age|字段的值也发生了同样的改变.

其实,\verb|array_copy2|中的第一个元素值就是\verb|array_orig2|中的第一个元素值本身,也就是说它们完全是同一个值.所以,我对前者的修改实际上就是在修改后者.而且,对于其中任何在对应位置上的元素值来说都会是如此.其根本原因就是,数组\verb|array_copy2|是经由对数组\verb|array_orig2|的浅拷贝而得出的复本.\verb|copy|函数只拷贝了原数组的外层结构给新数组,却在新数组上直接沿用了原数组中的所有元素值.

不但如此,我们对原\verb|Person|类型值的修改,也会立即反应到这两个数组中相应的元素值上.例如:

\begin{lstlisting}[language=julia]
julia> p1.age = 25
25

julia> Int(array_orig2[1].age), Int(array_copy2[1].age)
(25, 25)

julia> 
\end{lstlisting}

这再次印证了 Julia 中的值总是以共享的方式被传递的(passed by sharing).数组\verb|array_orig2|和\verb|array_copy2|中的第一个元素值实际上都是\verb|p1|的值本身.\verb|p2|和\verb|p3|的情况也是这样.

如果我们在元素类型属于容器类型的数组上做浅拷贝,显然也会发生类似的事情.比如:

\begin{lstlisting}[language=julia]
julia> a1 = [1, 3, 5]; a2 = [2, 4, 6];

julia> array_orig3 = [a1, a2]; array_copy3 = copy(array_orig3);

julia> a1[2] = 30; array_orig3[1][2], array_copy3[1][2]
(30, 30)

julia> 
\end{lstlisting}

那么,与浅拷贝相比,深拷贝有着怎样的不同呢?

深拷贝除了会拷贝原值的外层结构,还会把原结构中的所有内部对象都复制一遍,无论这些内部对象藏得有多么深.也就是说,新的结构和新的内部对象共同形成了全新的值.如此一来,复本与原值就可以做到相互独立、毫不相干了.示例如下:

\begin{lstlisting}[language=julia]
julia> array_deepcopy3 = deepcopy(array_orig3);

julia> a1[2] = 60; array_orig3[1][2], array_deepcopy3[1][2]
(60, 30)

julia> 
\end{lstlisting}

函数\verb|deepcopy|用于对一个值进行深拷贝.我利用它得到了数组\verb|array_orig3|的复本\verb|array_deepcopy3|.我虽然可以通过\verb|a1[2] = 60|修改数组\verb|array_orig3|中的元素值,但无法通过同样的方式对数组\verb|array_deepcopy3|进行修改.注意,在前一个例子执行完之后,\verb|array_orig3|的值已变为\verb|[[1, 30, 5], [2, 4, 6]]|,而\verb|array_deepcopy3|正是基于那时的它的复本.因此,索引表达式\verb|array_deepcopy3[1][2]的结果值才会是\verb|30|.

我们来看一个更复杂一些的例子:

\begin{lstlisting}[language=julia]
julia> a3 = [7, 9, 11]; a4 = [8, 10, 12];

julia> array_orig4 = [[a1, a3], [a2, a4]]
2-element Array{Array{Array{Int64,1},1},1}:
 [[1, 60, 5], [7, 9, 11]]
 [[2, 4, 6], [8, 10, 12]]

julia> array_deepcopy4 = deepcopy(array_orig4);

julia> 
\end{lstlisting}

数组\verb|array_orig4|中的每一个元素值都分别是一个数组,并且在这些数组之中还有数组.也就是说,\verb|array_orig4|是数组的数组的数组,即一个拥有三层结构的数组.而数组\verb|array_deepcopy4|则产生自对\verb|array_orig4|的深度拷贝.下面我们来看看\verb|deepcopy|函数是否已将其中的所有内部对象全都拷贝了出来.代码如下:

\begin{lstlisting}[language=julia]
julia> a4[1] = 80; array_orig4[2][2][1], array_deepcopy4[2][2][1]
(80, 8)

julia> array_orig4[2][2][1] = 82; array_deepcopy4[2][2][1]
8

julia> 
\end{lstlisting}

显然,无论内部对象有多么复杂,由\verb|deepcopy|函数产生的数组复本都是完全独立于原数组的.

好了,现在我们知道了,拷贝像数组这样的对象很容易,调用一个函数就可以了.\verb|copy|函数用于浅拷贝,它只会拷贝原数组的外层结构,并沿用其所有的元素值.\verb|deepcopy|函数用于深拷贝,它不仅会拷贝原数组的外层结构,而且还会拷贝其所有的元素值以及更深层次的内部对象(如果有的话).