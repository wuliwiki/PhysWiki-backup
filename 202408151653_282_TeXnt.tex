% LaTeX 笔记
% license Usr
% type Note

\begin{issues}
\issueDraft
\end{issues}

\subsection{排版细节}
\begin{itemize}
\item 在 \verb`\begin{document}` 之前, \verb`\hyphenation{exam-ple hy-phen-a-tion}` 可以设置自动换行时如果一个单词被拆开, \verb`-` 出现的位置。
\end{itemize}

\subsection{BiTex}

\begin{lstlisting}[language=none, caption=BibTeX 的例子]
@article{sampleArticle,
  author  = {Author Name},
  title   = {Title of the Article},
  journal = {Journal Name},
  year    = {2024},
  volume  = {10},
  number  = {2},
  pages   = {123--130}
}
\end{lstlisting}

\begin{itemize}
\item \verb`author` 是作者列表, 多个作者用 \verb`and` 隔开, 如果姓名中间有逗号,那逗号前面就是姓,后面是名。 如果没有逗号则姓在后面。
\item \verb`number` 和 \verb`issue` 是一回事, issue 一般是一年 12 个,每个 volume 中 issue 一般从 1 开始。
\item \verb`pages` 是起始页码或者页码范围, \verb`--` 是推荐的格式,但单个 \verb`-` 有时候也行。
\end{itemize}
