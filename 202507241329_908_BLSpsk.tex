% 布莱斯·帕斯卡(综述)
% license CCBYSA3
% type Wiki

本文根据 CC-BY-SA 协议转载翻译自维基百科\href{https://en.wikipedia.org/wiki/Blaise_Pascal}{相关文章}。

布莱兹·帕斯卡(Blaise Pascal,\(^\text{[a]}\) 1623年6月19日-1662年8月19日)是一位法国数学家、物理学家、发明家、哲学家及天主教作家。

帕斯卡是神童,由担任鲁昂税务官的父亲亲自教育。他最早的数学研究是投影几何,16岁时便撰写了一篇重要的关于圆锥曲线的论文。后来,他与皮埃尔·费马通信探讨概率论,对现代经济学与社会科学的发展产生了深远影响。1642年,他开始从事计算机的先驱性研究,发明了后来被称为“帕斯卡计算器”或“帕斯卡机”的装置,使他成为机械计算器的最早两位发明人之一\(^\text{[6][7]}\)。

与同时代的勒内·笛卡尔一样,帕斯卡也是自然科学和应用科学的先驱。他撰文为科学方法辩护,并提出了若干颇具争议的研究成果。他在流体研究方面作出了重要贡献,推广伊万杰利斯塔·托里拆利的研究成果,澄清了压力和真空的概念。国际单位制中压力单位“帕斯卡”正是以他命名的。1647年,他继托里拆利和伽利略之后,驳斥了亚里士多德与笛卡尔等人所持的“自然界厌恶真空”的观点。

他也被誉为现代公共交通的发明者,因为他在1662年去世前不久创立了“五苏之马车”,这是历史上第一种现代公共交通服务\(^\text{[8]}\)。

1646年,他与妹妹雅克琳一同接受了天主教内部一个被批评者称为詹森主义的宗教运动\(^\text{[9]}\)。1654年末经历一次宗教体验后,他开始撰写有深远影响的哲学与神学作品。他最著名的两部著作都诞生于这一时期:《省函集》和《思想录》。《省函集》以詹森主义者与耶稣会士之间的冲突为背景;而《思想录》中包含了著名的“帕斯卡赌注”,原名为《论机器的演说》\(^\text{[10][11]}\),这是一个以信仰主义为基础、具有概率论性质的论证,主张人应当相信上帝的存在。同年,他还撰写了一部关于“算术三角形”的重要论文。1658至1659年间,他又研究了摆线及其在求解立体体积中的应用。在多年疾病折磨之后,帕斯卡于39岁时在巴黎去世。
