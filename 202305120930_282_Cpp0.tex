% C++ 基础
% c++|cpp|语法

Matlab 和 Python 等动态语言虽然用起来方便, 但缺点是运行较慢, 对于一些计算量大的项目不适合。 目前在高性能计算中广泛使用的只有两种语言即 C++ 和 Fortran。 虽然 Fortran 普遍被认为是一个过时的语言, 但在计算物理中, 许多人仍然在使用, 一是因为以前遗留下的 Fortran 代码比较多, 二是一些年纪较大的学者只会 Fortran。

一本在数值算法中很有名的书是 Numerical Recipes, 这本书第三版以前都使用 Fortran 或 C, 而第三版却只有 C++, 这也是本书选择介绍 C++ 而不是 Fortran 的原因之一。 本书将从 Numerical Recipes 中借鉴许多代码上的风格和算法。

C++ 的特征实在多不胜数, 事实上无论是什么语言, 做计算物理的研究者大多会倾向于只选择一些最基础的语法来使用。
我们在这里列出本书使用的 C++ 特性。

在学习基本 C++ 语法时, 强烈建议在 Jupyter Notebook 中使用 cling 解释器\upref{Cling}。 这将节约大量的编译时间。

\subsection{基础语法}
\begin{itemize}
\item 基本类型 (\verb|bool|, \verb|char|, \verb|int|, \verb|long|, \verb|long| \verb|long|, \verb|float|, \verb|double|, \verb|long double|) \verb|char, short, int| 几乎总是 8, 16, 32 bit 的, 虽然标准没这么规定。
\item C 和 C++ 的整数(笔记)\upref{cppInt}
\item 基本算符 (\verb|=, +, -, *, /, %, ++, --, +=, -=, *=, /=, ?:|) 以及优先级
\item 求值顺序没有定义: 例如 \verb|f1() + f2() + f3()| 等效于 \verb|(f1() + f2()) + f3()|, 但是三个函数被调用的顺序是没有定义的。
\item 算符 \verb|%| 的定义使 \verb|(m / n) * n + m % n == m| 成立。
\item 比特算符(\verb|&|, \verb+|+, \verb|^|, \verb|~|, \verb|<<|, \verb|>>|)
\item 格式输出(\verb|printf|)
\item scope (scope 内的名字在 scope 外没有定义, scope 内可以定义与 scope 外相同的名字并覆盖, 类的 destructor 会在 scope 结束时自动调用)
\item 判断(\verb|if|, \verb|else if|, \verb|else|)
\item 循环 (\verb|for(;;)|, \verb|while|, \verb|do while|, \verb|break|, \verb|continue|)
\item \verb|goto|, \verb|label:| 谨慎使用, 可用于跳出多重循环。
\item 函数(函数名重载, 变量默认值, 算符函数, \verb|inline| 函数, \verb|static| 变量)
\item \verb|const|
\item 指针, reference, pass by reference/value
\item 引用
\item 头文件机制 (相当于原地插入头文件中的代码) \verb|#include <>| 和 \verb|""| 的区别
\item 多文件编译
\item one definition rule (ODR)
\item 栈(stack)和堆(heap)
\item \verb|typedef|
\item 数组: 不指定个数的数组作为函数参数和指针是完全等效的。
\item 动态内存管理 \verb|new|, \verb|delete|。
\item 异常处理 \verb|throw|, \verb|try|, \verb|catch|, \verb|std::runtime_error|, \verb|out_of_range| 等。
\item \textbf{undefined behavior} 和 \textbf{implementation defined behavior} 有什么不同? 前者编译器不需要诊断不需要有确定的行为,任何事情都可能发生(包括程序整个 crash)。 后者需要编译器或者操作系统等有自己的统一定义, 需要写入文档并遵守。 作为程序员, 前者坚决不要使用, 后者也最好不要使用。
\end{itemize}

\subsubsection{explicit 类型转换}
\begin{itemize}
\item \verb|static_cast<新类型>(变量)| 和 \verb|(新类型)变量| 效果相同, 但有一些转换不允许。
\item \verb|reinterpret_cast<新类型*>(变量指针)| 和 \verb|reinterpret_cast<新类型&>(变量)| 并不会检查 \textbf{strict aliasing rule}, 需要用户自己保证。
\item \textbf{strict aliasing rule}: 不允许不同类型的指针或索引指向同一个对象, 可能导致优化时出错。 但是有例外, 例如用 \verb|char*| 和 \verb|(un)signed char*| 指向任何地址, 用 base type 的指针指向对象等。
\item standard-layout 类型 (e.g., structs without virtual functions or private members) 的指针可以互相转换, 如果他们前几个成员是一样的。
\end{itemize}

\subsection{标准库}
\begin{itemize}
\item \verb|cmath|
\item \verb|complex|
\item \verb|vector| (包括内存管理机制、 改变长度可能使指针失效)
\item \verb|string|, \verb|string32|
\item \verb|iostream| (\verb|cin|, \verb|cout|, \verb|<<| 算符, \verb|>>| 算符, \verb|cin| 可以放在判断语句中, 如 \verb|if (cin >> a)|。)
\item \verb|ifstream|, \verb|ofstream|
\item 如何判断 \verb|cin| 或 \verb|fin| 读到了文件末尾(文件末尾允许任意多空格/空行)? 不要用 \verb|cin.eof| 判断, 例如读 int, 可以先初始化为 \verb|INT_MAX|, 然后读完以后判断是否仍为 \verb|INT_MAX|。
\item 换行: \verb|cin.ignore(numeric_limits<int>::max(), '\n');|
\item C 语言的 \verb|#include <climits>| 中的常数(\verb|INT_MIN|, \verb|INT_MAX|, \verb|DBL_MAX|, \verb|DBL_EPSILON|), 以及 \verb|std::numeric_limits<>::max()| 等。
\end{itemize}

\subsubsection{高级}
\begin{itemize}
\item \verb|__cplusplus| 用于检测 cpp 版本。
\item linux 中用 \verb|<bits/stdc++.h>| 快速 include 所有 C 头文件和 STL 头文件。
\item \verb|pair|
\item \verb|unordered_set|, \verb|set|
\item \verb|unordered_map|, \verb|ordered_map|
\item \verb|stack|
\item \verb|queue|, \verb|deque|, \verb|priority_que|
\end{itemize}

\subsection{较高级的功能}
\begin{itemize}
\item \verb|for(auto &e : v)|, 需要定义 \verb|v.begin()|, \verb|v.end()|, iterator 需要支持 \verb|++|, \verb|!=|, \verb|*p|。
\item 类(constructor, destructor, \verb|public|, \verb|private|, 数据成员, 函数成员, \verb|operator+, -, *, /, (), []|)
\item POD (plain old data), \verb|类名() = default;| (可以有其他 constructor), 没有 destructor, 没有 virtual function。 可以用 \verb|std::is_pod<类名>::value| 来判断。
\item 宏 (\verb|#include|, \verb|#define|, \verb|#if|, \verb|#ifdef|, \verb|#ifndef|, \verb|#else|, \verb|#endif|)
\item 宏函数, \verb|var|(变量), \verb|#var|(字符串), 以及 \verb|...##...| 用于连接变量\footnote{例如 \verb|#define MYFUN1(x) x*x|, 那么 \verb|MYFUN1(abcd)| 就展开为 \verb|abcd*abcd|。 \verb|#define MYFUN2(x) #x|, 那么 \verb|MYFUN2(abcd)| 就展开为 \verb|"abcd"|。\verb|#define MYFUN3(x,y) a##x##y##f|, 那么 \verb|MYFUN3(bc,de)| 展开为 \verb|abcdef|。 注意作为宏函数自变量的宏名不会\textbf{先}展开, 例如 \verb|#define NUM 5|, 那么 \verb|MYFUN1(NUM)| 先展开为 \verb|NUM*NUM| 再进一步展开为 \verb|5*5|。 又例如 \verb|MYFUN2(NUM)| 会展开为 \verb|"NUM"|, 但字符串不会作为宏进一步展开。 宏函数体中双引号内的东西都不会展开, 例如 \verb|#define MYFUN4(x) "x"|, 那么 \verb|MYFUN4(abc)| 仍然展开为 \verb|"x"|}。
\item \verb|#define ... do {} while(0)| 避免多个语句不加花括号带来的错误。
\item \verb|namespace|
\item \verb|constexpr|
\item \verb|decltype| (\verb|double x; decltype(x) y;|)
\item 类的继承
\item lambda 表达式
\end{itemize}

\subsection{高级功能}
\begin{itemize}
\item \verb|extern "C"| \href{https://www.geeksforgeeks.org/extern-c-in-c/}{参考}
\item \verb|#ifdef __cplusplus| 用于判断是否在用 C++ 编译器, 还可以判断版本: \verb|#if __cplusplus >= 201103L|, \verb|201402L|, \verb|201703L|
\item 类模板
\item 函数模板
\item SFINAE\upref{SFINAE}
\end{itemize}
