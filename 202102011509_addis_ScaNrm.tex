% 一维散射态的正交归一化

\begin{issues}
\issueDraft
\end{issues}

\pentry{一维散射(量子)\upref{Sca1D}}
本文使用原子单位制\upref{AU}.\footnote{写作参考\href{https://chaoli.club/index.php/4541/last}{这篇帖子}.}类似于平面波的归一化\upref{EngNor}, 一维散射态也有不同的归一化方式, 但情况要更为复杂. 

为了方便先假设 $V(x)$ 关于原点对称, 且 $V(x)$ 只在区间 $[-L,L]$ 内不为零. 由于 $V(x)$ 的对称性, 我们必定能找到实值的奇函数和偶函数两种解. 令 $k = \sqrt{2mE} > 0$, 在区间 $[-L,L]$ 外, 波函数就是正弦函数加上一个相移
\begin{equation}\label{ScaNrm_eq3}
\psi_k(x) = A\sin(kx + \phi) \qquad (x > L)
\end{equation}
其中$\phi$ 是 $k$ 的函数, 称为\textbf{相移(phase shift)}. 为方便书写下文把 $\phi(k),\phi(k')$ 分别记为 $\phi, \phi'$.

令奇函数和偶函数散射态分别为实函数 $\psi_{k,1}(x)$ 和 $\psi_{k,2}(x)$ 我们希望通过添加适当的归一化系数后, 波函数能满足归一化条件(\autoref{EngNor_eq3}~\upref{EngNor})
\begin{equation}\label{ScaNrm_eq1}
\int_{-\infty}^{+\infty} \psi_{k',i}(x)^* \psi_{k,i}(x) \dd{x} = \delta(k' - k) \qquad (k > 0, i = 1, 2)
\end{equation}

\begin{theorem}{}
\autoref{ScaNrm_eq1} 对所有性质良好的 $V(x)$ 都成立, 且\autoref{ScaNrm_eq3} 中\textbf{归一化系数和简谐波相同}, 即 $A = 1/\sqrt{\pi}$(\autoref{EngNor_eq5}~\upref{EngNor}).
\end{theorem}

\subsubsection{部分证明}
对于奇偶性不同的两个函数, 他们显然式正交的. 首先已知
\begin{equation}
\int_{0}^{+\infty} \sin(k'x)\sin(kx)\dd{x} = \frac{\pi}{2}\delta(k'-k)
\end{equation}
现在添加相位 $\phi(k)$ 后, 有不定积分
\begin{equation}
\int \sin(k'x+\phi')\sin(kx+\phi) \dd{x} = \frac{\sin[(k'-k)x + (\phi'-\phi)]}{2(k'-k)}
- \frac{\sin[(k'+k)x+(\phi'+\phi)]}{2(k'+k)}
\end{equation}
在 $(0,n)$ 做定积分取极限 $n\to\infty$ 后发现比 $\delta(x)$ 多了两项
\begin{equation}
\int_{0}^{+\infty} \sin(k'x+\phi')\sin(kx+\phi) \dd{x} = \frac{\pi}{2}\delta(k'-k)
+ \frac{\sin(\phi'+\phi)}{2(k'+k)} - \frac{\sin(\phi'-\phi)}{2(k'-k)}
\end{equation}
所以在区间 $[0, +\infty)$ 上 $\sin(kx+\phi)$ 并不正交\footnote{但在 $(-\infty,\infty)$ 上却正交}.

使用归一化系数 $1/\sqrt{2}$, \autoref{ScaNrm_eq1} 的积分为(利用波函数的奇偶性)
\begin{equation}\label{ScaNrm_eq2}
\begin{aligned}
\braket{\psi_{k'}}{\psi_k} &= 2\int_{0}^{+\infty} \frac{1}{\sqrt{\pi}}\sin(k'x+\phi')\frac{1}{\sqrt{\pi}}\sin(kx+\phi) \dd{x} + 2I(k,k')\\
&= \delta(k'-k) + \frac{\sin(\phi'+\phi)}{\pi(k'+k)} - \frac{\sin(\phi'-\phi)}{\pi(k'-k)} + 2I(k,k')
\end{aligned}
\end{equation}
其中 $2I(k,k')$ 修正了 $[-L,L]$ 区间实际波函数和 $\sin(kx+\phi)$ 的不同
\begin{equation}
I(k,k') = \int_0^L \psi_{k'}(x)^* \psi_k(x) \dd{x}
-\int_{0}^{L} \frac{1}{\sqrt{\pi}}\sin(k'x+\phi') \frac{1}{\sqrt{\pi}}\sin(kx+\phi) \dd{x}
\end{equation}
如果能证明对于任意 $V(x)$, \autoref{ScaNrm_eq2} 的最后的三项之和都为零, 那么我们就证明了\autoref{ScaNrm_eq1} 的正交归一关系. 读者可以尝试用一些具体的例子证明, 如方势垒\upref{SqrPot}.

\subsection{不对称势能}
\subsubsection{归一化}
由于势能 $V(x)$ 不对称, 我们无法保证上文的 $\psi_{k,1},\psi_{k,2}$ 的奇偶性. 但通过适当的操作仍然能使\autoref{ScaNrm_eq1} 成立. 先给出结论: 令区间 $[-L,L]$ 外的波函数为
\begin{equation}\label{ScaNrm_eq9}
\psi_{k,i}(x) = \leftgroup{
    &A_{+,i} \sin(kx + \phi_{+,i}) &\quad &(x > L)\\
    &A_{-,i} \sin(kx - \phi_{-,i}) &&(x < -L)
} \quad (i = 1,2)
\end{equation}
那么, 归一化系数需要满足
\begin{equation}\label{ScaNrm_eq4}
\frac{1}{2}\qty(\abs{A_-}^2 + \abs{A_+}^2) = \frac{1}{\pi}
\end{equation}
当 $A_+ = A_-$ 时就有上文的归一化系数 $1/\sqrt{\pi}$.

\textbf{部分证明}: 把正交化积分划分为正负半轴两部分进行($\psi_{k}$ 取 $\psi_{k,1}, \psi_{k,2}$ 中一个)
\begin{equation}
\begin{aligned}
\braket{\psi_{k'}}{\psi_{k}} &= \abs{A_-}^2\int_{-\infty}^0 \sin(kx + \phi_{+})\sin(kx + \phi_{+}) \dd{x} + I_-(k,k')\\
&\qquad+ \abs{A_+}^2\int_0^{+\infty} \sin(kx + \phi_{+})\sin(kx + \phi_{+}) \dd{x}   + I_+(k,k')\\
&= \abs{A_-}^2\frac{\pi}{2}\delta(k'-k) + \abs{A_+}^2 \frac{\pi}{2} \delta(k'-k)
\end{aligned}
\end{equation}
把\autoref{ScaNrm_eq4} 代入有 $\braket{\psi_{k'}}{\psi_{k}} = \delta(k'-k)$.

\subsubsection{正交化}
和上文类似, 我们可以保证 $i = i'$ 时\autoref{ScaNrm_eq1} 成立. 但由于我们缺失了函数的奇偶性, 一般无法保证 $i \ne i'$ 时也成立. 可以证明定态薛定谔方程必定存在一对满足以下正交条件的 $\psi_{k,1}, \psi_{k,2}$
\begin{equation}\label{ScaNrm_eq7}
\lim_{n\to\infty}\frac{1}{n}\int_{-n}^{+n} \psi_{k,1}(x)^* \psi_{k,2}(x) \dd{x} = 0
\end{equation}
显然上文的奇函数和偶函数根据该定义是正交的.

若我们得到两个不正交的线性无关解 $\psi_{k,1}, \psi'_{k,2}$, 这时我们可以使用一个类似于施密特正交化\upref{SmdtOt}的操作. 先把 $\psi_{k,1}$ 按照\autoref{ScaNrm_eq4} 归一化, 那么
\begin{equation}
\psi_{k,2}(x) = \psi_{k,2}'(x) - \alpha \psi_{k,1}(x)
\end{equation}
\begin{equation}\label{ScaNrm_eq5}
\begin{aligned}
\alpha &= \lim_{n\to\infty} \frac{\pi}{n} \int_{-n}^n \psi_{k,1}(x)^* \psi'_{k,2}(x) \dd{x}\\
&= \frac{\pi}{2} \qty[A_{-,1}A_{-,2}\cos(\phi_{-,1}-\phi_{-,2}) + A_{+,1}A_{+,2}\cos(\phi_{+,1}-\phi_{+,2})]
\end{aligned}
\end{equation}
这相当于把 $\psi'_{k,2}$ 中与 $\psi_{k,1}$ 不正交的部分减去. 最后再对 $\psi_{k,2}$ 归一化使其满足\autoref{ScaNrm_eq4} 即可使 $\psi_{k,1}, \psi_{k,2}$ 正交归一.

\textbf{部分证明}: 假设对于任意给定 $k$ 必定存在满足\autoref{ScaNrm_eq1} 和\autoref{ScaNrm_eq7} 的一对正交归一解 $\psi_{k,1}, \psi_{k,2}$. 它们是 2 维解空间的一组基底, 那么 $\psi'_{k,2}$ 必定可以表示为
\begin{equation}
\psi'_{k,2} = \alpha\psi_{k,1} + \beta\psi_{k,2}
\end{equation}
所以
\begin{equation}\label{ScaNrm_eq8}
\begin{aligned}
&\lim_{n\to\infty}\frac{1}{n}\int_{-n}^{+n} \psi_{k,1}(x)^*\psi'_{k,2}(x)\dd{x}
=\\
&\alpha\lim_{n\to\infty}\frac{1}{n}\int_{-n}^{+n} \psi_{k,1}(x)^*\psi_{k,1}(x)\dd{x}
+ \beta\lim_{n\to\infty}\frac{1}{n}\int_{-n}^{+n} \psi_{k,1}(x)^*\psi_{k,2}(x)\dd{x}
\end{aligned}
\end{equation}
其中由\autoref{ScaNrm_eq4} 得($[-L,L]$ 的积分在极限中消失)
\begin{equation}\label{ScaNrm_eq6}
\lim_{n\to\infty}\frac{1}{n}\int_{-n}^{+n} \psi_{k,1}(x)^*\psi_{k,1}(x)\dd{x} = \frac{1}{2}\qty(\abs{A_-}^2 + \abs{A_+}^2) = \frac{1}{\pi}
\end{equation}
把\autoref{ScaNrm_eq6} 和\autoref{ScaNrm_eq7} 代入\autoref{ScaNrm_eq8} 得\autoref{ScaNrm_eq5}. 证毕.

\subsection{行波的归一化}
\addTODO{通过 2 乘 2 的酋矩阵变换即可.或者, 令规定 $\psi_{k,1}$ 在负半轴为向右的行波除以 $1/\sqrt{2\pi}$, 然后对 $\psi_{k,2}$ 做施密特正交归一化.}

\begin{equation}
\psi_{k,i} = \leftgroup{
    &A_i\exp(\I kx) + B_i\exp(-\I kx) &\quad& (x < -L)\\
    &C_i\exp(\I kx) + D_i\exp(-\I kx) && (x > L)
}
\end{equation}
和上文类似的方法得归一化条件为
\begin{equation}
\abs{A_i}^2 + \abs{B_i}^2 + \abs{C_i}^2 + \abs{D_i}^2= \frac{1}{\pi}
\end{equation}
对于平面波, 显然有 $A = C = 1/\sqrt{2\pi}$, $B = D = 0$. 和 “平面波的的正交归一化\upref{EngNor}” 中结论相同.

一种常用的行波边界条件为 $D_1 = 0$, 它的物理意义是粒子从左边入射, 发生反射和投射. 我们令 $\psi_1$ 满足该条件, 然后通过施密特正交化得到与之正交归一的 $\psi_2$. 我们先令不一定正交的 $\psi'_2$ 满足边界条件 $A_2 = 0$, 即粒子从右边入射, 投影得
\begin{equation}
\alpha = \pi (B_1^* B_2 + C_1^*C_2)
\end{equation}
于是(未归一化的) $\psi_2$ 为
\begin{equation}
\psi_2 = \psi'_2 - \alpha \psi_1
\end{equation}

\subsubsection{变换矩阵}
\begin{equation}
\psi_{k,i}(x) = \leftgroup{
    &A_{+,i} \sin(kx + \phi_{+,i}) &\quad &(x > L)\\
    &A_{-,i} \sin(kx - \phi_{-,i}) &&(x < -L)
} \quad (i = 1,2)
\end{equation}
令变换矩阵为 $\pmat{C_{1} & C_{2}\\ C_{2}^* & -C_{1}^*}$, 满足 $\abs{C_1}^2 + \abs{C_2}^2 = 1$, 有
\begin{equation}
\psi_{k,a}(x) = C_1\psi_{k,1}(x) + C_2\psi_{k,2}(x)
\end{equation}
\begin{equation}
\psi_{k,b}(x) = C_{2}^*\psi_{k,1}(x) - C_{1}^*\psi_{k,2}(x)
\end{equation}
令 $\psi_{k,a}$ 满足
\begin{equation}
\psi_{k,a}(x) \propto \exp(\I kx) \qquad (x > L)
\end{equation}
得到条件
\begin{equation}
C_{1}A_{+,1}\E^{-\I\phi_{+,1}} + C_{2}A_{+,2}\E^{-\I\phi_{+,2}} = 0
\end{equation}
这就可以求出 $C_1, C_2$ (乘以一个不确定的相位因子\footnote{要规定这个因子, 只需要令入射波 $\exp{\I kx}$ 的系数为实数即可}), 以及 $\psi_{k,a}, \psi_{k,b}$.
