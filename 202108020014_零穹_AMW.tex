% 不对称势阱
\pentry{有限深方势阱\upref{FSW}}
\begin{figure}[ht]
\centering
\includegraphics[width=7cm]{./figures/AMW_1.pdf}
\caption{一维不对称势阱} \label{AMW_fig1}
\end{figure}
本节我们来解下面的一维不对称势阱的离散谱(束缚态),\autoref{AMW_fig1} .
\begin{equation}
V(x)=\left\{\begin{aligned}
&V_1\quad(x<0)\\
&0\\
&V_2\quad(x>L)
\end{aligned}\right.
\end{equation}

\textbf{解:}对于离散谱,能量 $E$ 需小于无穷远处势能,故 $E<V_1$. 

在 $x<0$ 区域内的其薛定谔方程为
\begin{equation}
\frac{\hbar^2}{2m}\dv[2]{\psi(x)}{x}+(E-V_1)\psi(x)=0
\end{equation}
波函数为
\begin{equation}
\psi=C_1 e^{\kappa_1 x},\quad \kappa_1=\frac{1}{\hbar}\sqrt{2m(V_1-E)}
\end{equation}

在 $x>L$ 区域内的其薛定谔方程为
\begin{equation}
\frac{\hbar^2}{2m}\dv[2]{\psi(x)}{x}+(E-V_2)\psi(x)=0
\end{equation}
波函数为
\begin{equation}
\psi=C_2 e^{-\kappa_2 x},\quad \kappa_2=\frac{1}{\hbar}\sqrt{2m(V_2-E)}
\end{equation}

在阱内 $0 < x < L$ ,薛定谔方程为
\begin{equation}
\frac{\hbar^2}{2m}\dv[2]{\psi(x)}{x}+E\psi(x)=0
\end{equation}
$\psi$ 可取以下形式:
\begin{equation}
\psi=C\sin(kx+\delta),\quad k=\frac{\sqrt{2mE}}{\hbar}
\end{equation}

根据势阱边上 $\psi'/\psi$ 的连续性条件,得
\begin{equation}
k\cot\delta=\kappa_1=\sqrt{\frac{2m}{\hbar^2}V_1-k^2},\quad k\cot(Lk+\delta)=-\kappa_2=-\sqrt{\frac{2m}{\hbar^2}V_2-k^2}
\end{equation}
或
\begin{equation}
\sin\delta=\frac{k\hbar}{\sqrt{2mV_1}},\quad\sin(kL+\delta)=-\frac{k\hbar}{\sqrt{2mV_2}}
\end{equation}
消去 $\delta$ 后,得下列超越方程:
\begin{equation}\label{AMW_eq1}
kL=n\pi-\arcsin\frac{k\hbar}{\sqrt{2mV_1}}-\arcsin\frac{k\hbar}{\sqrt{2mV_2}}
\end{equation}
其中,$k=1,2,3,\cdots$,反正弦函数取值介于 $0$ 到 $\pi/2$ 之间.上式之根确定了能级 $E=k^2\hbar^2/2m$.对每一个 $n$ 一般来讲只有一个根;$n$ 值按能级的递增次序编号.

由于反正弦函数宗量不能超过1,$k$ 值显然只能介于 $0$ 到 $\sqrt{V_1}/\hbar$ 之间.\autoref{AMW_eq1} 左边随 $k$ 单调增加,右边却随 $k$ 单调减小.因此\autoref{AMW_eq1} 有根的必要条件为 $k=\sqrt{2mV_1}/\hbar$ 时,\autoref{AMW_eq1} 右边小于左边.特别是,由 $n=1$ 所得的下列不等式
\begin{equation}
L\frac{\sqrt{2mV_1}}{\hbar}\geq\frac{\pi}{2}-\arcsin\sqrt{\frac{V_1}{V_2}}
\end{equation}
给出了阱中至少存在一个能级的条件.由此可见,给定了不xiang'deng