% 小时百科编辑器简介
% keys 小时物理|LaTeX|编辑器
% license Xiao
% type Tutor

\pentry{LaTeX 结构简介\nref{nod_latxIn}}{nod_f276}

\subsection{文件与目录}

小时百科使用统一的 \href{https://github.com/MacroUniverse/PhysWiki}{LaTeX 模板}, 主文件是根目录下的 \verb`main.tex`, 每个\textbf{文章}(即 \verb`\section`)占一个文件, 保存在 \verb`contents` 子目录中 (如 \verb`contents/Sample.tex`), 我们把它们叫做\textbf{文章文件}, 在编译 pdf 时文章文件会被插入 \verb`main.tex`。

在线编辑器新建的文件不是主文件, 而是被插入主文件的文章文件。 从编辑器打开文件时, 除第一项 \verb`main.tex` 外的所有文件也都是文章文件。 以下是一篇文章文件示例(文件名 \verb`gougu.tex`)

\begin{lstlisting}[language=latex]
% 勾股定理

令 $a$, $b$ 和 $c$ 分别为直角三角形的两条直角边和斜边, 有
\begin{equation}
a^2 + b^2 = c^2
\end{equation}
\end{lstlisting}
保存(ctrl+s)后, 编辑器会自动给出网页预览, 如\autoref{fig_latxIn_1} 
\begin{figure}[ht]
\centering
\includegraphics[width=11cm]{./figures/c4a74b64207c3c05.png}
\caption{网页预览} \label{fig_editIn_1}
\end{figure}

注意文章文件中并不需要 \verb`\section{勾股定理}`, 因为它会被直接插入主文件。 但为了查看方便, 我们规定每篇文章第一行必须注释文章标题。

完成编写文章后, 我们需要在主文件 \verb`main.tex` 中插入该文章。 在编辑器中打开主文件, 在适当的章节中插入 \verb`\entry{勾股定理}{gougu}` 即可\footnote{\verb`\entry` 是模板定义的, 相当于 \verb`\section{勾股定理}\label{gougu}\input{./contents/gougu}`}。 保存后, 我们就可以在\href{https://wuli.wiki/changed}{百科目录}中看到这篇文章了\footnote{如果你在使用个人笔记功能, 则会显示到个人笔记目录中}, 如果编译 pdf 也可以在 pdf 的目录中看到。

注意新建和修改的文章页面(html)会暂存在草稿区(\href{https://wuli.wiki/changed/}{wuli.wiki/changed/}), 只有有权限的用户发布后才会转移到正式目录 (\href{https://wuli.wiki/online/}{wuli.wiki/online/})。
