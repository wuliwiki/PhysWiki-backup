% 马约拉纳方程(综述)
% license CCBYSA3
% type Wiki

本文根据 CC-BY-SA 协议转载翻译自维基百科\href{https://en.wikipedia.org/wiki/Majorana_equation}{相关文章}。

在物理学中,\textbf{Majorana 方程}是一种相对论波动方程。它以意大利物理学家埃托雷·马约拉纳(Ettore Majorana)的名字命名,他于1937年提出这一方程用于描述那些自身即为反粒子的费米子。依据这一方程的粒子被称为\textbf{Majorana 粒子}。然而,如今这个术语涵盖了更广泛的意义,指任何(可能是非相对论的)自身为反粒子的费米子,因此这些粒子必然是电中性的。

有理论提议认为,具有质量的中微子可以用 Majorana 粒子来描述;标准模型的各种扩展允许这种可能性。关于 Majorana 粒子的文章中包含了实验搜索的最新进展,包括中微子的相关细节。而本文则主要关注该理论的数学发展,特别是其离散和连续对称性。离散对称性包括\textbf{电荷共轭}、\textbf{宇称变换}和\textbf{时间反演};连续对称性为\textbf{洛伦兹不变性}。

电荷共轭在其中扮演了重要角色,这是使得 Majorana 粒子能够被描述为电中性的关键对称性。一个特别值得注意的特性是,电中性允许对左右手螺旋场的全局相位进行自由选择。这意味着,在没有显式限制这些相位的情况下,Majorana 场天然是 CP 破坏的。电中性带来的另一个特性是,左右手螺旋场可以被赋予不同的质量。换句话说,\textbf{电荷}是洛伦兹不变量,同时也是运动常数;而\textbf{手性}则是洛伦兹不变量,但对于具有质量的场来说不是运动常数。因此,电中性的场受到的约束比带电场更少。在电荷共轭作用下,这两个自由的全局相位出现在质量项中(因为它们是洛伦兹不变量),因此 Majorana 质量被描述为一个复矩阵,而不是一个单一数值。

简而言之,Majorana 方程的离散对称性远比 Dirac 方程复杂。在 Dirac 方程中,电荷 
\( U(1) \) 对称性约束并消除了这些自由度,而在 Majorana 方程中,这些自由度得以保留。