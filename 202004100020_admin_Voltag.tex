% 电压

\pentry{电势 电势能\upref{QEng}}

\textbf{电压(voltage)} 就是电势差的同义词, 通常在讨论电路时使用, 本词条只讨论电路中的电压. 电势差的定义为(\autoref{QEng_eq1}\upref{QEng})
\begin{equation}\label{Voltag_eq1}
U_{21} = V(\bvec r_2) - V(\bvec r_1) = - \int_{\bvec r_1}^{\bvec r_2} \bvec E_0(\bvec r) \vdot \dd{\bvec r}
\end{equation}
在 “电势 电势能\upref{QEng}” 中, 我们强调了要定义电荷的电势能或电势, 我们必须要使用无旋的电场(保守场), 而电路中一般来说既包含无旋场也包含有旋场. 所以我们规定\autoref{Voltag_eq1} 中的 $\bvec E_0(\bvec r)$ 只包含电路中的净电荷产生的电场, 线积分的路径也只能取电路的一部分.

\begin{exercise}{磁生电}
假设我们有一个 $N$ 匝的不闭合线圈, 两端接在理想电压表上. 若线圈中存在变化的磁场, 磁通量为 $\Phi = \alpha t$, 那么根据高中的知识我们知道电压表会显示读数.

注意这并不是一个静电学问题, 变化的磁场会沿着线圈产生涡旋电场, 这是一个旋度场, 所以在计算电压时我们不能将这部分电场算入.

那为什么线圈两端还会产生电压呢? 因为整个导线(忽略电阻)作为一个导体, 其内部电场必须为零%链接未完成
, 所以当线圈放在涡旋电场中, 线圈就会自动调整其净电荷分布, 使得线圈中处处存在与涡旋电场反向的电场, 而根据定义, \autoref{Voltag_eq1} 的积分需要考虑这部分电场.
\end{exercise}
