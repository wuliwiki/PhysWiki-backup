% 可微映射的导数
% keys 导数|可微映射
% license Xiao
% type Tutor

\pentry{常微分方程的几何图像\upref{GofODE}}
可微映射(\autoref{def_GofODE_2}~\upref{GofODE})$f:U\rightarrow V$ 将 $\mathbb R^n$ 空间的区域 $U$ 映射到 $\mathbb R^m$ 空间区域 $V$,于是就将 $U$ 上的曲线(\autoref{sub_GofODE_1}~\upref{GofODE})$\varphi$ 映射到 $V$ 上的曲线 $f(\varphi)$,可微性意味着这一对应是一一的。而切向量是曲线的等价类(\autoref{def_GofODE_3}~\upref{GofODE}),于是曲线 $\varphi,f(\varphi)$ 各自对应一切向量 $\dv{\varphi}{t},\dv{f(\varphi)}{t}$。这就是说在可微映射 $f$ 作用下,$U$中的切向量 $\dv{\varphi}{t}$ 和 $V$ 中的切向量 $\dv{f(\varphi)}{t}$ 对应,这一对应是一一的,因为若 $\dv{\varphi_1}{t}=\dv{\varphi_2}{t}$,则
\begin{equation}\label{eq_DoDifM_1}
\dv{f(\varphi_1(t))}{t}=\sum_i\pdv{f}{x^i}\dv{\varphi^i_1}{t}=\sum_i\pdv{f}{x^i}\dv{\varphi^i_2}{t}=\dv{f(\varphi_2(t))}{t}~.
\end{equation}
描述由可微映射 $f$ 导致的 $U$ 中的切向量和 $V$ 中的切向量的这一对应关系的双射称为 $f$ 的导数,记为 $f_*$。由于 $\dv{\varphi}{t}\in TU_{\varphi(t)} ,\dv{f(\varphi)}{t}\in TV_{f(\varphi(t))}$ (\autoref{def_GofODE_4}~\upref{GofODE}),所以 $TU_{\varphi(t)}$ 和 $TV_{f(\varphi(t))}$ 就是 $f$ 的导数在点 $\varphi(t)$ 的定义域和值域。

\begin{definition}{可微映射的导数}
设 $f:U\rightarrow V$ 是可微映射,称映射 $f_*|_x:TU_x\rightarrow TU_{f(x)}$ 为映射 $f$ 在点 $x$ 的\textbf{导数},若 $f_*|_x$ 把过点 $x$ 的任意曲线 $\varphi$ ($\varphi(0)=x$)的速度向量(\autoref{def_GofODE_5}~\upref{GofODE})映到过点 $f(x)$ 的曲线 $f\circ\varphi$ 的速度向量,即
\begin{equation}
f_*|_x(\dv{\varphi}{t}|_{t=0})=\dv{}{t}\Bigg|_{t=0}(f\circ\varphi)~.
\end{equation}
\end{definition}

\begin{theorem}{导数是线性映射}
可微映射的导数是线性映射。
\end{theorem}
\textbf{证明:}
\begin{equation}
\begin{aligned}
&f_*|_x(a\dv{\varphi_1}{t}|_{t=0}+b\dv{\varphi_2}{t}|_{t=0})\\
&=\dv{}{t}\Bigg|_{t=0}(f\circ(a\varphi_1+b\varphi_2))\\
&=\sum_i\pdv{f}{y^i}\Bigg|_{x}\qty(a\dv{\varphi^i_1}{t}\Bigg|_{t=0}+b\dv{\varphi^i_2}{t}\Bigg|_{t=0})\\
&=a\sum_i\pdv{f}{y^i}\Bigg|_{x}\dv{\varphi^i_1}{t}\Bigg|_{t=0}+b\sum_i\pdv{f}{y^i}\Bigg|_{x}\dv{\varphi^i_2}{t}\Bigg|_{t=0}\\
&=a\dv{}{t}\Bigg|_{t=0}(f\circ\varphi_1)+b\dv{}{t}\Bigg|_{t=0}(f\circ\varphi_2)\\
&=af_*|_x(\dv{\varphi_1}{t}|_{t=0})+bf_*|_x(\dv{\varphi_2}{t}|_{t=0})~.
\end{aligned}
\end{equation}



\textbf{证毕!}