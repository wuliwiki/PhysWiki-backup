% BEC 超流
% 玻色爱因斯坦|凝聚态|超流

首先,我们回顾一下理想玻色气体的BEC. 假设这种玻色子的色散是 $\epsilon_{\bvec k} = \hbar^2{\bvec k}^2/2m$ (实际上长什么样在我们这里一般的分析中并没有什么关系). 我们可以看到, $n_{\bvec k}=1/(\E^{\beta(\epsilon_{\bvec k}-\mu)}-1)$. 那么,在热力学极限下可以发现如果 $\mu<0$,换句话说温度处于 $T_c$, $N\to\infty$, $n_{\bvec k}/N\to 0$. 而相变的时候 $T_c: \mu=0$,这时候物理图像是de Brogilie波长 $\approx$ 粒子间距. 在 $<T_c$ 时,
\begin{equation}
N=N_0+\frac{V}{2\pi^3}\int \text{d}^3{\bvec k}\frac{1}{\E^{\beta\epsilon_{\bvec k}}-1},\quad N_0/N\neq0
\end{equation}
呈现宏观占据. 看得出来我们虽然很熟悉什么是理想玻色气体的BEC,但是对于一般的情况,如何规定它是否是处于凝聚的态呢?

首先,我们可以写出某个多体($N$ 体)波函数 $\Psi({\bvec r}_1,{\bvec r}_2,\cdots,{\bvec r}_N)$ 密度矩阵 $\rho$. 它的矩阵元有 $2N$ 个指标, $\langle x_1,\cdots,x_N|\rho|y_1,\cdots,y_N\rangle$. 把它trace掉 $N-1$ 个指标,得到
\begin{equation}
\rho({\bvec r},{\bvec r'}) = N\int\sum_s p_s\psi^*({\bvec r},{\bvec r}_2,\cdots,{\bvec r}_N)p_s\psi({\bvec r'},{\bvec r}_2,\cdots,{\bvec r}_N)\text{d}^3{\bvec r}_2\cdots\text{d}^3{\bvec r}_N
\end{equation}
二参数函数可以按照某一个基展开,
\begin{equation}
\rho({\bvec r},{\bvec r'}) = \sum_i N_i \varphi_i^*({\bvec r})\varphi_i({\bvec r'})
\end{equation}
对于 $N_i$,如果所有 $N_i/N=0$, 那么就称这个态是normal phase. 如果有一个不等于 $0$ 的,那就是一般的BEC,如果有多个 $N_i\neq0$ 叫 fragment BEC. 后两者称之存在非对角长程序(off-diagonal long-range order,ODLRO).

对于正常的BEC的情况, $\exists N_0\gg N_{i\neq0}$,如果我们忽略两体相互作用的细节,我们就可以写出
\begin{equation}
\rho({\bvec r,\bvec r'}) \approx N_0\varphi^*({\bvec r})\varphi({\bvec r'})
\end{equation}
其中,满足要求的系统的态是
\begin{equation}\label{BECSup_eq5}
\Psi({\bvec r}_1,{\bvec r}_2,\cdots,{\bvec r}_N) = \prod_{\otimes}\varphi({\bvec r}_i)
\end{equation}

得到
\begin{equation}
{\bvec v}_i\cdot{\bvec q}-c|{\bvec q}| = \frac{{\bvec q}^2}{2m}
\end{equation}
可见右边等式大于 $0$,如果 ${\bvec v}_i<c$,则左边无法满足也为正,即无法发生散射,也就是真正意义的“超流”(不会产生摩擦作用),这个临界速度也就是声速 $c$.
