% 维尔纳·海森堡(综述)
% license CCBYSA3
% type Wiki

本文根据 CC-BY-SA 协议转载翻译自维基百科\href{https://en.wikipedia.org/wiki/Werner_Heisenberg}{相关文章}。

\textbf{维尔纳·卡尔·海森堡}(Werner Karl Heisenberg,/ˈhaɪzənbɜːrɡ/;[2] 德语:[ˈvɛʁnɐ ˈhaɪzn̩bɛʁk] ⓘ;1901年12月5日-1976年2月1日)[3] 是德国理论物理学家,量子力学理论的主要奠基人之一,也是二战期间纳粹核武器计划中的核心科学家。他于1925年发表了他的《重新诠释》论文,对旧量子理论进行了重要的重新解释。同年,他与马克斯·玻恩和帕斯库尔·约尔丹合写了一系列论文,详细阐述了他的矩阵量子力学表述。他以1927年发表的不确定性原理而闻名。海森堡因“创立量子力学”而获得1932年诺贝尔物理学奖。[4][a]  

海森堡还对湍流流体动力学、原子核、铁磁性、宇宙射线和亚原子粒子的理论作出了贡献。他在1957年协助规划了卡尔斯鲁厄的第一个西德核反应堆,以及慕尼黑的一个研究反应堆。  

二战后,他被任命为\textbf{威廉皇帝物理研究所}(Kaiser Wilhelm Institute for Physics)所长,不久后该研究所更名为马\textbf{克斯·普朗克物理研究所}。他一直担任所长,直到研究所于1958年迁至慕尼黑。1960年至1970年间,他担任\textbf{马克斯·普朗克物理与天体物理研究所}所长。  

此外,海森堡还担任\textbf{德国研究委员会}主席、\textbf{原子物理委员会}\textbf{主席}、\textbf{核物理工作组}主席以及\textbf{洪堡基金会}主席。[1]  
\subsection{早年生活与教育 }
\subsubsection{早年}  
维尔纳·卡尔·海森堡(Werner Karl Heisenberg)于德国维尔茨堡出生,父亲是卡斯帕·恩斯特·奥古斯特·海森堡(Kaspar Ernst August Heisenberg),母亲是安妮·维克莱因(Annie Wecklein)。[6] 他的父亲是一名古典语言的中学教师,后来成为德国大学体系中唯一一位中世纪和现代希腊研究的正式教授(ordinarius professor)。[7]  

海森堡在信仰上是一名路德宗基督徒。[8] 在青少年晚期,他曾在巴伐利亚阿尔卑斯山徒步旅行时阅读柏拉图的《蒂迈欧》。他回忆起在慕尼黑、哥廷根和哥本哈根接受科学训练时,与同学和老师进行的关于理解原子的哲学对话。[9] 海森堡后来表示:“我的思想是通过学习哲学、柏拉图以及类似的内容形成的。”[10] 他还说:“现代物理学明确支持柏拉图的观点。实际上,物质的最小单位并不是普通意义上的物理对象;它们是形式、理念,只能通过数学语言清晰地表达。”[11]  

1919年,海森堡作为自由军团(Freikorps)的一员来到慕尼黑,参与对一年前成立的巴伐利亚苏维埃共和国的斗争。五十年后,他回忆起那段日子,说那是年轻人的乐趣,就像“玩警察抓小偷一类的游戏;完全没有任何严肃的意义。”[12] 他的职责仅限于从“红色”行政建筑中没收自行车或打字机,以及看守被怀疑的“红色”囚犯。[13]  
\subsubsection{大学学习}
\begin{figure}[ht]
\centering
\includegraphics[width=6cm]{./figures/481ce511fb21e6cb.png}
\caption{1924年的海森堡} \label{fig_Heisen_1}
\end{figure}
从1920年至1923年,海森堡在慕尼黑的路德维希-马克西米利安大学师从阿诺德·索末菲和威廉·维恩学习物理和数学,同时在哥廷根的乔治-奥古斯特大学师从马克斯·玻恩和詹姆斯·弗兰克**学习物理,并跟随大卫·希尔伯特学习数学。他于1923年在慕尼黑大学获得索末菲指导的博士学位。  

1922年6月,索末菲带海森堡前往哥廷根参加玻尔节(Bohr Festival),因为索末菲对学生非常关心,并了解海森堡对尼尔斯·玻尔原子物理理论的兴趣。在活动中,玻尔作为客座讲师发表了一系列关于量子原子物理的全面演讲,海森堡首次见到了玻尔,这对他产生了深远的影响。[14][15][16]  

海森堡的博士论文由索末菲建议题目,研究湍流问题;[17] 论文讨论了层流的稳定性和湍流的性质。他通过**奥尔-索末菲方程**(Orr-Sommerfeld equation)研究稳定性问题,这是一种描述层流微小扰动的四阶线性微分方程。二战后,他曾短暂地回到这一课题的研究中。[18]  

在哥廷根,海森堡在马克斯·玻恩指导下完成了关于反常塞曼效应的资格论文(Habilitationsschrift),并于1924年取得资格证书。[19][3][20][21]  

青年时期,海森堡是新探路者运动(Neupfadfinder,一支德国童子军协会,也是德国青年运动的一部分)的成员和童子军领袖。[22][23][24] 1923年8月,罗伯特·霍泽尔与海森堡共同组织了一个来自慕尼黑的童子军团队前往芬兰旅行。[25]  
\subsubsection{个人生活}  
海森堡热爱古典音乐,是一位出色的钢琴家。[3] 他对音乐的兴趣促成了与未来妻子的相遇。1937年1月,海森堡在一场私人音乐会中结识了伊丽莎白·舒马赫(Elisabeth Schumacher,1914–1998)。伊丽莎白是柏林一位著名经济学教授的女儿,她的兄弟是《小即是美》的作者、经济学家E. F. 舒马赫。海森堡于同年4月29日与她结婚。  

1938年1月,他们的双胞胎玛丽亚(Maria)和沃尔夫冈(Wolfgang)出生。沃尔夫冈·泡利因此祝贺海森堡的“对生成”(pair creation),这是一种来自基本粒子物理学的术语“对产生”的文字游戏。此后12年间,他们又育有五个孩子:芭芭拉(Barbara)、克里斯蒂娜(Christine)、尤赫恩(Jochen)、马丁(Martin)和弗蕾娜(Verena)。[26][27]  

1939年,海森堡在德国南部瓦尔兴湖的乌尔费尔德为家人购置了一处夏季住宅。  

海森堡的一个儿子马丁·海森堡(Martin Heisenberg)成为维尔茨堡大学的神经生物学家,另一个儿子尤赫恩·海森堡(Jochen Heisenberg)则成为新罕布什尔大学的物理学教授。[28]  
\subsection{学术生涯}  
\subsubsection{哥廷根、哥本哈根与莱比锡}  
从1924年至1927年,海森堡在哥廷根担任私人讲师(Privatdozent),意味着他具备独立教学和考试的资格,但没有教授职位。1924年9月17日至1925年5月1日期间,海森堡获得国际教育委员会洛克菲勒基金会奖学金,前往哥本哈根大学理论物理研究所,与所长尼尔斯·玻尔开展研究。他的重要论文《关于运动学与力学关系的量子理论重新诠释》(Über quantentheoretische Umdeutung kinematischer und mechanischer Beziehungen),即所谓的重新诠释论文(Umdeutung paper),于1925年9月发表。[29]  

他随后返回哥廷根,与马克斯·玻恩和帕斯库尔·约尔丹合作,在大约六个月内发展了量子力学的矩阵力学表述。1926年5月1日,海森堡开始在哥本哈根担任大学讲师,并成为玻尔的助理。1927年,海森堡在哥本哈根工作期间,研究量子力学的数学基础,提出了著名的不确定性原理。2月23日,他在写给物理学家沃尔夫冈·泡利的信中首次描述了这一新原理。[30] 在关于该原理的论文中,[31] 海森堡使用了“Ungenauigkeit”(意为“模糊”或“不精确”),而非“不确定性”来描述这一概念。[3][32][33]  

1927年,海森堡被任命为莱比锡大学理论物理学的正式教授(ordentlicher Professor)兼物理系主任。他于1928年2月1日发表了就职演讲。在莱比锡发表的第一篇论文中,[34] 海森堡利用泡利不相容原理解决了铁磁性的谜题。[3][20][32][35]  

25岁的海森堡成为德国最年轻的全职教授,并担任莱比锡大学理论物理研究所所长。[36] 他讲授的课程吸引了包括爱德华·泰勒和罗伯特·奥本海默在内的学生,[36] 后者后来参与了美国的曼哈顿计划。[37]  

在海森堡于莱比锡任职期间,与他一起学习和工作的博士生、研究生以及研究合作者的高水平可见一斑,从他们后来获得的广泛赞誉便可证明。他的学生和合作者包括:埃里希·巴格(Erich Bagge)、费利克斯·布洛赫(Felix Bloch)、乌戈·法诺(Ugo Fano)、齐格弗里德·弗吕格(Siegfried Flügge)、威廉·维尔米利恩·休斯顿(William Vermillion Houston)、弗里德里希·洪德(Friedrich Hund)、罗伯特·S·穆利肯(Robert S. Mulliken)、鲁道夫·佩耶尔斯(Rudolf Peierls)、乔治·普拉茨克(George Placzek)、伊西多·艾萨克·拉比(Isidor Isaac Rabi)、弗里茨·绍特(Fritz Sauter)、约翰·C·斯莱特(John C. Slater)、爱德华·泰勒(Edward Teller)、约翰·哈斯布鲁克·范弗莱克(John Hasbrouck van Vleck)、维克托·弗雷德里克·韦斯科普夫(Victor Frederick Weisskopf)、卡尔·弗里德里希·冯·魏茨泽克(Carl Friedrich von Weizsäcker)、格雷戈尔·温策尔(Gregor Wentzel)以及克拉伦斯·齐纳(Clarence Zener)。[38]  

1929年初,海森堡和泡利共同发表了两篇奠定相对论量子场论基础的论文中的第一篇。[39] 同年,海森堡展开了讲学之旅,访问了中国、日本、印度和美国。[32][38] 1929年春,他在芝加哥大学担任访问讲师,讲授量子力学课程。[40]  

1928年,英国数学物理学家保罗·狄拉克(Paul Dirac)推导出了他的相对论量子力学波动方程,这一方程预示了正电子的存在。1932年,美国物理学家卡尔·大卫·安德森(Carl David Anderson)通过宇宙射线的云室照片,确认了一条轨迹是由正电子产生的。1933年中期,海森堡提出了他的正电子理论,并在两篇论文中详细阐述了他对狄拉克理论的思考和该理论的进一步发展。第一篇论文《对狄拉克正电子理论的评论》(Bemerkungen zur Diracschen Theorie des Positrons)于1934年发表,[41] 第二篇《狄拉克正电子理论的推论》(Folgerungen aus der Diracschen Theorie des Positrons)于1936年发表。[32][42][43]  

在这些论文中,海森堡首次将狄拉克方程重新解释为描述任意自旋为 ħ/2 的点粒子的“经典”场方程,并将其置于涉及反对易子的量子化条件下。通过这一重新诠释,海森堡将狄拉克方程作为一种准确描述电子的(量子)场方程,将物质与电磁学置于同一理论基础上:均由允许粒子产生和湮灭的相对论量子场方程所描述。(赫尔曼·外尔(Hermann Weyl)早在1929年的一封写给爱因斯坦的信中已经描述了这一点。)
\subsubsection{《矩阵力学与诺贝尔奖》}
海森堡的《重新解释》论文,奠定了现代量子力学的基础,这篇论文让物理学家和历史学家都感到困惑。他的方法假定读者已经熟悉克拉默-海森堡过渡概率的计算。论文中的主要新思想——非对易矩阵——仅通过拒绝不可观测的量来得到证明。它通过基于对应原理的物理推理,引入了矩阵的非对易乘法,尽管海森堡当时并不熟悉矩阵的数学理论。麦金农(MacKinnon)已经重构了通向这些结果的路径,艾奇森(Aitchison)及其合著者则完成了详细的计算。

在哥本哈根,海森堡和汉斯·克拉默斯合作撰写了一篇关于色散的论文,讨论了波长大于原子尺寸的辐射与原子的散射。他们证明了克拉默斯早期发展出的成功公式不能基于波尔轨道,因为过渡频率是基于不恒定的能级间距。相比之下,经典锐化轨道的傅里叶变换中的频率是等间隔的。但这些结果可以通过半经典的虚拟态模型来解释:入射辐射使得价电子或外层电子激发到一个虚拟态,随后从这个虚拟态衰减。在后来的论文中,海森堡展示了这个虚拟振荡器模型也能解释荧光辐射的偏振现象。

这两项成功,以及波尔-索末菲模型在解释异常塞曼效应的突出问题上不断失败,促使海森堡利用虚拟振荡器模型尝试计算光谱频率。尽管这个方法在应用于现实问题时过于复杂,海森堡转而考虑一个更简单的例子,即非谐振荡器。

偶极振荡器由一个简单的谐振荡器组成,可以将其视为一个充电粒子在弹簧上振荡,受到外部力的扰动,例如外部电荷。振荡电荷的运动可以用振荡器频率的傅里叶级数来表示。海森堡通过两种不同的方法解决了量子行为问题。首先,他采用虚拟振荡器方法处理该系统,计算由外部源产生的能级间的跃迁。

然后,他通过将非谐势项视为对谐振荡器的扰动,并使用他和玻恩开发的扰动方法来解决同样的问题。这两种方法得出了相同的结果,尤其是对于第一项和非常复杂的二阶修正项。这表明,在这些复杂的计算背后,存在一个一致的方案。

因此,海森堡着手在不依赖虚拟振荡器模型的情况下,公式化这些结果。为此,他用矩阵代替了空间坐标的傅里叶展开,这些矩阵对应于虚拟振荡器方法中的跃迁系数。他通过引用波尔的对应原理和泡利的观点——即量子力学必须局限于可观察量——来为这种替代方法提供理论依据。

7月9日,海森堡将这篇论文交给玻恩审阅并提交出版。当玻恩阅读论文时,他认出了这种公式化方法可以被转录并扩展为矩阵的系统语言,而这正是他在布雷斯劳大学跟随雅各布·罗萨内斯学习时所学的内容。玻恩在他的助手兼前学生帕斯夸尔·乔丹的帮助下,立即开始进行转录和扩展,并将结果提交出版;这篇论文在海森堡论文提交后的60天内就被接收了。三位作者在年底前共同提交了后续论文。

直到此时,物理学家们很少使用矩阵;它们被认为属于纯数学的领域。古斯塔夫·米耶(Gustav Mie)在1912年关于电动力学的论文中使用过矩阵,玻恩在1921年关于晶体格理论的研究中也使用过矩阵。尽管在这些案例中使用了矩阵,但矩阵的代数运算和乘法并未像在量子力学的矩阵形式化中那样发挥作用。

1928年,阿尔伯特·爱因斯坦提名海森堡、玻恩和乔丹为诺贝尔物理学奖的候选人。1932年诺贝尔物理学奖的宣布推迟到了1933年11月。那时,宣布海森堡获得了1932年诺贝尔物理学奖,“因其创造了量子力学,其应用,尤其是,导致了氢的同素异形体形式的发现”。
\subsubsection{量子理论的解释}
量子力学的发展及其在“现实”方面的看似矛盾的含义,具有深刻的哲学影响,特别是关于科学观察真正意义的讨论。与阿尔伯特·爱因斯坦和路易·德布罗意等现实主义者不同,他们相信粒子在任何时刻都有客观存在的动量和位置(即使无法同时测量这两个量),海森堡则是一位反现实主义者,他认为直接了解“现实”是什么超出了科学的范围。在他的著作《物理学家对自然的理解》中,海森堡认为,归根结底,我们只能谈论描述粒子某些特征的知识(如表格中的数字),但我们永远无法获得对粒子本身的“真实”接触:

“我们再也不能独立于观察过程谈论粒子的行为。最终的结果是,量子理论中数学形式化的自然法则不再处理基本粒子本身,而是处理我们对它们的认知。我们也不再可能问这些粒子是否在空间和时间中客观存在……当我们谈论当今时代精确科学中的自然图景时,我们所指的不再是自然的图景,而是我们与自然之间关系的图景……科学不再作为客观观察者面对自然,而是把自己视为人类与自然之间互动中的一个行动者。科学分析、解释和分类的方法已经意识到它的局限性,这些局限性源于这样一个事实:科学通过介入改变并重塑了研究对象。换句话说,方法和对象不再能够分开。”
\subsubsection{SS调查}
在詹姆斯·查德威克于1932年发现中子后不久,海森堡提交了关于他核子中子-质子模型的三篇论文中的第一篇。1933年,阿道夫·希特勒上台后,海森堡在媒体中被攻击为“白犹太人”(即表现得像犹太人的雅利安人)。支持德国物理学(也称雅利安物理学)的团体对包括阿诺德·索末菲和海森堡在内的领先理论物理学家发起了激烈的攻击。从1930年代初起,反犹太主义和反理论物理学运动“德国物理学”开始关注量子力学和相对论理论。政治因素在大学环境中的影响超过了学术能力,尽管该运动的两位最著名支持者是诺贝尔物理学奖得主菲利普·伦纳德和约翰内斯·斯塔克。

曾有许多失败的尝试,希望海森堡能被任命为多所德国大学的教授。他尝试接替阿诺德·索末菲的职位未果,因为遭到了德国物理学运动的反对。1935年4月1日,海森堡的博士导师、著名的理论物理学家索末菲在慕尼黑大学获得了名誉教授职称。然而,索末菲在选举继任者的过程中继续担任其职位,这一过程直到1939年12月1日才结束。这个过程之所以如此漫长,是因为慕尼黑大学教职工的选拔与德国教育部及德国物理学支持者之间存在学术和政治上的分歧。

1935年,慕尼黑大学教职工委员会拟定了替代索末菲担任理论物理学常任教授和理论物理学研究所所长的候选人名单。三位候选人都曾是索末菲的学生:海森堡,曾获得诺贝尔物理学奖;彼得·德拜,1936年获得诺贝尔化学奖;以及理查德·贝克尔。慕尼黑大学教职工委员会坚定支持这些候选人,将海森堡列为首选。然而,德国物理学的支持者和帝国教育部(REM)内部的部分力量有自己的候选人名单,这场争斗持续了四年多。在此期间,海森堡受到了德国物理学支持者的激烈攻击。其中一次攻击被刊登在SS的报纸《黑色军团》上,该报由海因里希·希姆莱领导。在文章中,海森堡被称为“白犹太人”,应该让他“消失”。这些攻击被严肃对待,因为犹太人当时遭到了暴力攻击和监禁。海森堡通过一篇社论和一封给希姆莱的信进行反击,试图解决这一问题并恢复自己的名誉。

有一次,海森堡的母亲拜访了希姆莱的母亲。这两位妇女相识,因为海森堡的外祖父和希姆莱的父亲是巴伐利亚登山俱乐部的成员,并且曾分别担任过教堂牧师。最终,希姆莱通过两封信解决了海森堡的争议,分别于1938年7月21日向SS集团军指挥官海德里希和海森堡本人发出了信件。在给海德里希的信中,希姆莱表示德国不能失去或沉默海森堡,因为他对培养一代科学家有重要价值。在给海森堡的信中,希姆莱说,这封信是根据海森堡家庭的推荐发出的,并警告海森堡要区分专业物理学研究成果与相关科学家的个人和政治态度。

威廉·穆勒取代索末菲,成为慕尼黑大学的教授。穆勒并非理论物理学家,未曾在物理学期刊上发表过论文,也不是德国物理学会的成员。他的任命被视为一种荒谬的做法,并且对理论物理学的教育造成了不利影响。

三位领导SS调查海森堡的调查员都具备物理学背景。实际上,海森堡曾参与其中一位调查员在莱比锡大学的博士考试。三人中最有影响力的是约翰内斯·朱尔夫。在调查过程中,他们转而支持海森堡,并支持他反对德国物理学运动在理论物理学和学术界的意识形态政策。”
\subsection{德国核武器计划}    
\subsubsection{战前的物理学工作}  
1936年中期,海森堡在两篇论文中提出了他的宇宙射线暴理论。[73] 接下来两年里,又发表了四篇论文[74][75][76][77]。[32][78]

1938年12月,德国化学家奥托·哈恩和弗里茨·斯特拉斯曼向《自然科学》杂志提交了一篇稿件,报告称他们在用中子轰击铀时发现了元素钡,这使得哈恩得出铀核爆裂的结论;[79] 同时,哈恩将这些结果传达给了他的朋友莉泽·迈特纳(Lise Meitner),她在同年7月已逃亡到荷兰,随后又去了瑞典。[80] 迈特纳和她的侄子奥托·罗伯特·弗里施正确地将哈恩和斯特拉斯曼的结果解释为核裂变。[81] 弗里施于1939年1月13日进行了实验确认。[82]

1939年6月和7月,海森堡前往美国,拜访了密歇根大学的塞缪尔·亚伯拉罕·古德斯密特(Samuel Abraham Goudsmit)。然而,海森堡拒绝了移民美国的邀请。直到六年后,二战结束时,海森堡才再次见到古德斯密特,当时古德斯密特是美国阿尔索斯行动(Operation Alsos)的首席科学顾问。[32][83][84]
\subsubsection{乌兰协会成员资格}  
德国核武器计划,称为乌兰协会(Uranverein),于1939年9月1日成立,即第二次世界大战在欧洲爆发的那一天。德国陆军兵器局(Heereswaffenamt,HWA)将德国研究委员会(Reichsforschungsrat,RFR)挤出德国教育部(Reichserziehungsministerium,REM),并在军事支持下启动了正式的德国核能项目。该项目的首次会议于1939年9月16日召开。会议由HWA顾问库尔特·迪布纳(Kurt Diebner)组织,并在柏林举行。受邀者包括沃尔特·博特(Walther Bothe)、齐格弗里德·弗吕格(Siegfried Flügge)、汉斯·盖格(Hans Geiger)、奥托·哈恩(Otto Hahn)、保罗·哈特克(Paul Harteck)、赫尔曼·霍夫曼(Gerhard Hoffmann)、约瑟夫·马图赫(Josef Mattauch)和乔治·斯特特(Georg Stetter)。不久后召开了第二次会议,海森堡、克劳斯·克卢修斯(Klaus Clusius)、罗伯特·多佩尔(Robert Döpel)和卡尔·弗里德里希·冯·维茨泽克(Carl Friedrich von Weizsäcker)也参加了该会议。位于柏林达赫勒姆的凯瑟·威廉物理研究所(Kaiser-Wilhelm Institut für Physik,KWIP)被置于HWA的管理下,由迪布纳担任行政主任,核研究开始受到军事控制。[85][86][87] 在迪布纳负责管理KWIP并在HWA计划下运行的期间,迪布纳与海森堡的核心团队之间,包括卡尔·维尔茨(Karl Wirtz)和卡尔·弗里德里希·冯·维茨泽克(Carl Friedrich von Weizsäcker),产生了相当大的个人和职业敌对情绪。[32][88]

在1942年2月26日至28日的凯瑟·威廉物理研究所(Kaiser Wilhelm Institute for Physics)举办的科学会议上,海森堡向帝国官员就从核裂变中获取能量做了报告。此次会议由陆军兵器局(Army Weapons Office)召集。海森堡的报告题为《Die theoretischen Grundlagen für die Energiegewinnung aus der Uranspaltung》(“从铀裂变中获得能量的理论基础”)。正如海森堡在第二次世界大战后写给塞缪尔·古兹密特(Samuel Goudsmit)的信中所说,这篇讲座“根据帝国部长的智力水平进行了调整”,这是向外行展示复杂且前沿的科学概念时常做的处理方式。

海森堡在报告中讲解了核裂变的巨大能量潜力,指出通过原子核的裂变可以释放2.5亿电子伏特的能量。海森堡强调,要实现链式反应,必须获得纯净的铀-235。他探讨了获取纯铀-235同位素的不同方法,包括铀浓缩和另一种将普通铀与中子慢化剂层叠放置的方式。他指出,这种装置可以在实际中用于为车辆、船只和潜艇提供燃料。海森堡还强调了陆军兵器局在这项科学工作中的财务和物资支持的重要性。

接着召开了第二次科学会议。会议上讨论了对国家防御和经济具有决定性重要性的现代物理学问题。会议由科学、教育和国家文化部长伯恩哈德·鲁斯特(Bernhard Rust)出席。在会议上,鲁斯特部长决定将核项目从凯瑟·威廉学会(Kaiser Wilhelm Society)转移到帝国研究委员会(Reich Research Council)手中。

1942年4月,军方将物理学研究所归还给凯瑟·威廉学会,并任命海森堡为该研究所的所长。彼得·德拜(Peter Debye)仍然是该研究所的所长,但他在军方接管凯瑟·威廉物理学研究所行政控制权时,因拒绝成为德国公民而前往美国休假。海森堡依然在莱比锡大学拥有他的物理学系,罗伯特·德佩尔(Robert Döpel)及其妻子克拉拉·德佩尔(Klara Döpel)曾为乌兰计划(Uranverein)做过研究。

1942年6月4日,海森堡被召见,向德国军备部长阿尔伯特·斯佩尔(Albert Speer)报告将乌兰计划的研究转向开发核武器的前景。在会议上,海森堡告诉斯佩尔,要在1945年前制造出原子弹是不可能的,因为这需要大量的资金和人员。

在乌兰计划被转交给帝国研究委员会领导后,研究重心转向了核能生产,因此仍保持了“战争重要”(kriegswichtig)的地位,军方的资助得以继续。核能项目被分为以下几个主要领域:铀和重水的生产、铀同位素的分离以及铀机器(即核反应堆)。该项目随即在多个研究所之间进行分配,各个研究所的负责人主导了研究并设定了各自的研究议程。1942年,军方放弃了对德国核武器项目的控制,成为该项目人员规模的顶点。大约70名科学家为该项目工作,其中约40人将超过一半的时间用于核裂变研究。1942年之后,从事应用核裂变研究的科学家人数急剧减少。许多没有在主要研究所工作的科学家停止了核裂变研究,并将精力转向了更为紧迫的与战争相关的工作。

1942年9月,海森堡提交了关于散射矩阵(或S矩阵)的三篇论文系列中的第一篇,涉及初级粒子物理学。前两篇论文于1943年发表,第三篇论文则于1944年发表。S矩阵仅描述了碰撞过程中入射粒子的状态、从碰撞中出现的粒子的状态以及稳定的束缚态;其中不涉及中间态的参考。这与他在1925年所采用的先例相同,那时他通过仅使用可观测量,提出了量子力学矩阵公式的基础。

1943年2月,海森堡被任命为弗里德里希-威廉大学(今天的洪堡大学)理论物理学教席教授。同年4月,他当选为普鲁士科学院院士。同月,由于盟军对柏林的轰炸日益加剧,他将家人迁移到位于乌尔费尔德的度假胜地。夏季,他将凯瑟·威廉物理学研究所的第一批工作人员派往黑森林边缘的赫兴根和邻近的海格尔洛赫镇,出于同样的原因。10月18日至26日,他前往德国占领的荷兰。1943年12月,海森堡访问了德国占领的波兰。

1944年1月24日至2月4日,海森堡前往占领下的哥本哈根,此时德国军队已经没收了波尔的理论物理学研究所。他在4月进行了短暂的返程。12月,海森堡在中立的瑞士做了演讲。美国战略服务局(OSS)派遣了特工莫·伯格(Moe Berg)参加演讲,携带手枪,并指示他如果海森堡的演讲表明德国接近完成原子弹的研制,就开枪射击海森堡。

1945年1月,海森堡和大部分工作人员从凯瑟·威廉物理学研究所搬到了黑森林的设施。
\subsection{第二次世界大战后} 
\subsubsection{1945年:阿尔索斯行动}
\begin{figure}[ht]
\centering
\includegraphics[width=6cm]{./figures/329f469449542c9c.png}
\caption{在海格尔洛赫被俘并拆解的德国实验性核反应堆复制品} \label{fig_Heisen_2}
\end{figure}
阿尔索斯行动是盟军的一项努力,目的是确定德国是否拥有原子弹计划,并利用德国的原子能相关设施、研究、物资资源和科学人员,服务于美国的利益。该行动的人员通常会进入刚刚被盟军控制的区域,但有时也会在仍由德军控制的区域内行动。[102][103][104] 柏林曾是许多德国科学研究设施的所在地。为了减少人员伤亡和设备损失,这些设施在战争后期被分散到其他地方。凯瑟·威廉物理研究所(KWIP,凯瑟·威廉物理研究所)曾遭到轰炸,因此大部分设施在1943年和1944年搬到了黑森林边缘的黑兴根和邻近的海格洛赫,这些地方最终被纳入法国占领区。这使得阿尔索斯行动的美国特遣队能够拘留大量与核研究相关的德国科学家。[105][106]

1945年3月30日,阿尔索斯行动小组到达海德堡,[107] 在那里,他们捕获了包括瓦尔特·博特、理查德·库恩、菲利普·莱纳德和沃尔夫冈·根特纳在内的重要科学家。[108] 他们的审讯揭示了奥托·哈恩正在他的塔伊芬根实验室,而海森堡和马克斯·冯·劳厄则在海森堡位于黑兴根的实验室中,并且海森堡团队在柏林建造的实验性天然铀反应堆已被转移到海格洛赫。此后,阿尔索斯行动的主要焦点转向了维尔特堡地区的这些核设施。[37] 1945年5月3日,海森堡通过一项阿尔卑斯山区的行动从乌尔费尔德被秘密转移出来,所在区域仍由精英德军控制。他被带到海德堡,并于5月5日与戈茨密特见面,这是自1939年在安阿伯访问以来的第一次见面。德国两天后宣布投降。海森堡在接下来的八个月内再也没有见到他的家人,因为他被转移到法国和比利时,并于1945年7月3日飞往英国。[109][110][103]
\subsubsection{1945年:对广岛的反应}  
作为Uranverein成员的九位德国著名科学家曾在《核物理研究报告》上发表过报告[111],他们在阿尔索斯行动中被捕,并在“Epsilon行动”下被关押在英国[112]。包括海森堡在内的十位德国科学家被关押在英格兰的法姆霍尔。该设施曾是英国外国情报局MI6的安全屋。在他们被拘留期间,他们的对话被记录下来。被认为有情报价值的对话被转录并翻译成英文。转录本于1992年公开[113][114]。1945年8月6日,法姆霍尔的科学家通过媒体报道得知美国在日本广岛投下了原子弹。最初,他们对制造和投掷原子弹感到难以置信。在接下来的几周里,德国科学家讨论了美国是如何制造原子弹的。[115]

法姆霍尔的转录本显示,海森堡与其他被关押在法姆霍尔的物理学家们,包括奥托·哈恩和卡尔·弗里德里希·冯·维茨泽克,都很高兴盟军赢得了第二次世界大战。[116] 海森堡告诉其他科学家,他从未考虑过制造原子弹,只是考虑过为产生能源而建造原子堆。他们还讨论了为纳粹制造原子弹的道德问题。只有少数科学家对核武器的前景表示真正的恐惧,而海森堡本人在讨论这一问题时十分谨慎。[117][118] 关于德国核武器计划未能制造出原子弹,海森堡说:“我们在1942年春天是不会有道德勇气向政府建议,应该投入12万人只为建造这个东西的。”[119]

1992年,转录本解密后,德国物理学家曼弗雷德·波普分析了这些转录本以及Uranverein的相关文件。当德国科学家听说广岛爆炸后,海森堡承认他以前从未计算过原子弹的临界质量。后来,他在尝试计算该质量时犯了严重的计算错误。爱德华·泰勒和汉斯·贝特看过这些转录本后得出结论,认为海森堡是第一次做这种计算,因为他和他们当初犯了类似的错误。仅仅一周后,海森堡就做了关于原子弹物理学的讲座。他正确地认识到许多重要方面,包括原子弹的效率,尽管他仍然低估了其效率。波普认为,这证明了海森堡在战争期间并没有花时间研究核武器;相反,他甚至避免去思考它。[120][121]
\subsection{战后研究生涯}
\subsubsection{德国研究机构的领导职位} 
\begin{figure}[ht]
\centering
\includegraphics[width=6cm]{./figures/735aed46898ea4b4.png}
\caption{海森堡晚年肖像雕像,展出在位于慕尼黑附近加尔兴的马克斯·普朗克学会校园。} \label{fig_Heisen_3}
\end{figure} 
1946年1月3日,十名“Epsilon行动”被拘留者被运送至德国的Alswede。海森堡定居在哥廷根,该地位于英国占领区。[122] 海森堡立即开始推动德国的科学研究。在凯瑟·威廉社会被盟军控制委员会解散后,英国占领区成立了马克斯·普朗克学会,海森堡成为了马克斯·普朗克物理研究所的所长。马克斯·冯·劳厄被任命为副所长,而卡尔·维尔茨、卡尔·弗里德里希·冯·魏茨泽克和路德维希·比尔曼则加入了海森堡,帮助他建立该研究所。海因茨·比灵于1950年加入,推动电子计算机的发展。该研究所的核心研究方向是宇宙辐射。研究所每周六上午都会举行学术研讨会。[123]

海森堡与赫尔曼·雷因(Hermann Rein)共同在建立研究委员会(Forschungsrat)方面发挥了关键作用。海森堡设想这个委员会旨在促进新成立的德意志联邦共和国与以德国为基地的科学界之间的对话。[123] 海森堡被任命为研究委员会的主席。1951年,该组织与德国科学紧急协会(Notgemeinschaft der Deutschen Wissenschaft)合并,并在同年更名为德国研究基金会(Deutsche Forschungsgemeinschaft)。合并后,海森堡被任命为主席团成员。[32]

1958年,马克斯·普朗克物理学研究所(Max-Planck-Institut für Physik)迁至慕尼黑,进行了扩展,并更名为马克斯·普朗克物理学与天体物理学研究所(Max-Planck-Institut für Physik und Astrophysik,MPIFA)。在此期间,海森堡与天体物理学家路德维希·比尔曼(Ludwig Biermann)共同担任MPIFA的共同所长。海森堡还成为慕尼黑路德维希·马克西米连大学(Ludwig-Maximilians-Universität München)的普通教授。从1960年到1970年,海森堡是MPIFA的唯一所长。海森堡于1970年12月31日辞去MPIFA所长职务。[20][32]
\subsubsection{国际科学合作的推动}  
1951年,海森堡同意成为德意志联邦共和国在联合国教科文组织(UNESCO)会议上的科学代表,目的是建立一个欧洲核物理实验室。海森堡的目标是建造一个大型粒子加速器,利用西方集团内科学家的资源和技术力量。1953年7月1日,海森堡代表德意志联邦共和国签署了建立欧洲核子研究中心(CERN)的公约。尽管他曾被邀请成为CERN的创始科学主任,但他拒绝了这一职位。相反,他被任命为CERN科学政策委员会主席,并负责确定CERN的科学项目。[124]

1953年12月,海森堡成为亚历山大·冯·洪堡基金会(Alexander von Humboldt Foundation)的主席。[124] 在他担任主席期间,来自78个国家的550名洪堡学者获得了科学研究资助。海森堡在去世前不久辞去了主席职务。[125]
\subsubsection{研究兴趣}  
1946年,德国科学家海因茨·波塞(Heinz Pose),俄布宁斯克实验室V组组长,写信邀请海森堡前往苏联工作。信中赞扬了苏联的工作条件和可用资源,以及苏联对德国科学家的友好态度。这封日期为1946年7月18日的招聘信由信使亲自送达海森堡,海森堡礼貌地婉拒了邀请。[126][127]1947年,海森堡在剑桥、爱丁堡和布里斯托尔做了讲座。海森堡在1947年和1948年分别发表了三篇关于超导现象的论文,[128] 其中一篇与马克斯·冯·劳厄(Max von Laue)合作。[32][131]  

在二战后的短暂时期内,海森堡再次研究了他的博士论文课题——湍流问题。1948年他发表了三篇相关论文,[132][133][134] 1950年又发表了一篇。[18][135]在战后,海森堡继续关注宇宙射线的现象,特别是介子多重产生的研究。1949年他发表了三篇论文,[136][137][138] 1952年发表了两篇,[139][140] 1955年又发表了一篇。[141][142]

1955年底到1956年初,海森堡在苏格兰圣安德鲁斯大学进行了吉福德讲座,讲授物理学的思想史。讲座后来以《物理学与哲学:现代科学的革命》一书出版。[143]在1956年和1957年,海森堡担任德国原子能委员会(DAtK)“研究与发展”专业委员会(Fachkommission II “Forschung und Nachwuchs”)核物理工作组(Arbeitskreis Kernphysik)的主席。1956年和1957年,核物理工作组的其他成员包括:瓦尔特·博特(Walther Bothe)、汉斯·科普费尔曼(Hans Kopfermann,副主席)、弗里茨·博普(Fritz Bopp)、沃尔夫冈·根特纳(Wolfgang Gentner)、奥托·哈克塞尔(Otto Haxel)、威利巴尔德·延茨基(Willibald Jentschke)、海因茨·迈尔-莱布尼茨(Heinz Maier-Leibnitz)、约瑟夫·马图赫(Josef Mattauch)、沃尔夫冈·里茨勒(Wolfgang Riezler)、威廉·瓦尔彻(Wilhelm Walcher)和卡尔·弗里德里希·冯·魏茨泽克(Carl Friedrich von Weizsäcker)。沃尔夫冈·保罗(Wolfgang Paul)在1957年也成为该小组成员之一。[144]

1957年,海森堡是《哥廷根宣言》的签署人之一,公开表态反对联邦德国武装核武器。海森堡和帕斯库尔·乔丹(Pascual Jordan)一样认为,政治家们会忽视核科学家的这一声明。但海森堡认为,《哥廷根宣言》将“影响公众舆论”,而政治家们必须考虑这一点。他写信给瓦尔特·格拉赫(Walther Gerlach)说:“由于公众舆论可能会松懈,我们可能需要长时间不断地在公众面前讨论这个问题。”[145]  

1961年,海森堡与卡尔·弗里德里希·冯·魏茨泽克(Carl Friedrich von Weizsäcker)和路德维希·赖瑟(Ludwig Raiser)召集的一群科学家一起签署了《图宾根备忘录》[146],并与政治家展开了公开讨论。[147] 随着著名政治家、作家和社交名流参与核武器辩论,备忘录的签署人们也对“全职的知识界非从众者”提出了反对意见。[148]  

自1957年起,海森堡对等离子体物理学和核聚变过程产生了兴趣。他还与日内瓦国际原子物理研究所合作,是该研究所科学政策委员会的成员,并且曾担任该委员会主席多年。[3] 他是《图宾根备忘录》的八位签署人之一,该备忘录呼吁承认奥得—尼斯线为德国与波兰之间的正式边界,并反对西德可能的核武装。[149]  

1973年,海森堡在哈佛大学做了一次关于量子理论概念历史发展的讲座。[150] 1973年3月24日,海森堡在巴伐利亚天主教学院发表演讲,接受罗马诺·瓜尔迪尼奖。他的演讲英文版以《科学与宗教的真理》为题出版,其中一段引用将在本文章的后续部分出现。[151]  
\subsection{哲学与世界观}
海森堡钦佩东方哲学,并认为它与量子力学之间存在相似之处,他自称“完全同意”《物理学的道》一书的观点。海森堡甚至表示,在与拉宾德拉纳特·泰戈尔(Rabindranath Tagore)讨论印度哲学之后,“一些看似疯狂的想法突然变得更加有意义。”[152]  关于自然法则,他提到:“‘自然法则’的概念不可能完全客观,‘法则’这个词是一个纯粹的人类原则。”[153]  

关于路德维希·维特根斯坦(Ludwig Wittgenstein)的哲学,海森堡不喜欢《逻辑哲学论》(Tractatus Logico-Philosophicus),但他非常喜欢维特根斯坦后期的思想,尤其是他关于语言的哲学。[154]

海森堡是一位虔诚的基督徒,[155][156] 他在给爱因斯坦的最后一封信中写道:“我们可以安慰自己,上帝会知道这些[亚原子]粒子的位置,因此他会让因果律继续有效。”[157] 爱因斯坦则始终坚持量子物理学必然是不完整的,因为它意味着宇宙在基本层面上是不可确定的。[158]

在1950年代的讲座中,海森堡曾提出,科学进步正在导致文化冲突。他指出,现代物理学是“一个历史过程的一部分,这个过程倾向于统一并拓宽我们现有的世界观。”[159]  

当海森堡在1974年接受罗马诺·瓜尔迪尼奖(Romano Guardini Prize)时,他发表了一篇演讲,后来将其出版,标题为《科学与宗教的真理》。他沉思道:

“自从伽利略的著名审判以来,科学史上屡次有人声称科学真理无法与宗教的世界观相协调。尽管我现在确信科学真理在其领域内是不可动摇的,但我从未认为可以将宗教思想的内容简单地视为人类意识中一种过时的阶段,认为这一部分是我们今后必须放弃的。因此,在我的一生中,我多次不得不思考这两个思想领域之间的关系,因为我从未能够怀疑它们所指向的现实。”

—— 海森堡 1974年,213页[160]  

海森堡把大自然称为“上帝的第二本书”(第一本是《圣经》),并认为“物理学是对创造的神圣理念的反思;因此,物理学是神圣的服务”。这是因为“上帝按照他创世的理念创造了世界”,而人类之所以能理解这个世界,是因为“人是按照上帝的精神形象创造的”。[161]
\subsection{自传与去世}
在六十多岁时,海森堡撰写了面向大众市场的自传。1969年,这本书在德国出版,1971年初以英文版出版,随后几年也被翻译成多种语言。海森堡于1966年开始这个项目,当时他的公开讲座越来越多地转向哲学和宗教话题。他曾将关于统一场论的教科书手稿提交给Hirzel Verlag和John Wiley & Sons出版。他在给其中一位出版商的信中写道,这本手稿是他自传的准备工作。他将自传按主题结构,涵盖以下内容:1)精确科学的目标,2)原子物理学中语言的问题,3)数学与科学中的抽象,4)物质的可分性或康德的反命题,5)基本对称性及其证实,6)科学与宗教。

海森堡将自己的回忆录写成一连串的对话,讲述了自己的一生。这本书取得了大众成功,但被科学史学者认为是有问题的。在序言中,海森堡写道,他已经删减了历史事件,以使其更加简洁。在出版时,保罗·福尔曼在《科学》杂志上对这本书进行了评价,他评论道:“这是一本以理性重构对话形式写的回忆录。而正如伽利略所深知的那样,对话本身是一个非常狡猾的文学手段:生动、娱乐性强,特别适合于在避免对观点负责的同时暗中渗透个人意见。”

虽然科学回忆录的出版不多,但康拉德·洛伦茨和阿道夫·波特曼曾写过畅销书,将学术成果传播给广大读者。海森堡在慕尼黑的Piper Verlag出版社出版了这本书。他最初拟定的书名是《原子物理学的对话》(Gespräche im Umkreis der Atomphysik)。最终自传以《部分与整体》(Der Teil und das Ganze)为书名出版。1971年英文版出版时,书名为《物理学与超越:邂逅与对话》(Physics and Beyond: Encounters and Conversations)。
\begin{figure}[ht]
\centering
\includegraphics[width=6cm]{./figures/4adfd093dff7039c.png}
\caption{海森堡家族的墓地位于慕尼黑Waldfriedhof墓地,墓中包括奥古斯特·海森堡(1869–1930)、安妮·海森堡(1879–1945)、维尔纳·海森堡(1901–1976)和伊丽莎白·海森堡(1914–1998)。} \label{fig_Heisen_4}
\end{figure}
海森堡于1976年2月1日因肾癌在家中去世。[166] 次日晚,他的同事和朋友们从物理学研究所步行到他家,点燃蜡烛并将其放在他家门前,以示纪念。[167] 海森堡葬于慕尼黑的Waldfriedhof墓地。[168]

1980年,他的遗孀伊丽莎白·海森堡出版了《一个非政治人的政治生活》(Das politische Leben eines Unpolitischen),在书中她将海森堡描述为“首先是一个自发的人,其次是一个杰出的科学家,再其次是一个极具天赋的艺术家,只有在第四位,出于责任感,才是一个‘政治人’(homo politicus)”。[169]
\subsection{荣誉与奖项}  
海森堡获得了许多荣誉:[3]
\begin{itemize}
\item 布鲁塞尔大学、卡尔斯鲁厄理工大学和厄尔特大学的名誉博士学位。  
\item 巴伐利亚功绩勋章  
\item 罗马诺·瓜尔迪尼奖[151]  
\item 联邦服务大十字勋章  
\item Pour le Mérite(民事类)  
\item 1937年当选为美国哲学学会国际会员[170]  
\item 1955年当选为英国皇家学会外籍会员(ForMemRS)[1]  
\item 1958年当选为美国艺术与科学院国际荣誉会员[171]  
\item 会员:哥廷根、巴伐利亚、萨克森、普鲁士、瑞典、罗马尼亚、挪威、西班牙、荷兰(1939年)、罗马(教皇)、德国自然科学学者莱奥波尔迪纳学会(哈雷)、林采学会(罗马)和美国科学院等学术院所。  
\item 1932年诺贝尔物理学奖 "因创造量子力学,应用此理论发现了氢的同素异形体"。[54]  
\item 1933年德国物理学会马克斯·普朗克奖章
\end{itemize}
\subsection{核物理研究报告}  
以下报告发表于《核物理研究报告》(Kernphysikalische Forschungsberichte),这是德国铀协(Uranverein)的内部出版物。这些报告被分类为“绝密”,发行范围非常有限,作者不得保留副本。报告在盟军的阿尔索斯行动(Operation Alsos)中被没收,并被送交美国原子能委员会进行评估。1971年,报告被解密并返回德国。这些报告可以在卡尔斯鲁厄核研究中心和美国物理学研究所查阅。[174][175]
\begin{itemize}
\item Werner Heisenberg 《从铀裂变中获取技术能源的可能性》 G-39(1939年12月6日)  
\item Werner Heisenberg 《关于从铀裂变中获取技术能源的可能性(II)》 G-40(1940年2月29日)  
\item Robert Döpel、K. Döpel 和 Werner Heisenberg 《测定热中子在重水中的扩散长度》 G-23(1940年8月7日)  
\item Robert Döpel、K. Döpel 和 Werner Heisenberg 《测定热中子在样品38中的扩散长度》 G-22(1940年12月5日)  
\item Robert Döpel、K. Döpel 和 Werner Heisenberg 《D2O 和38的层次排列实验》 G-75(1941年10月28日)  
\item Werner Heisenberg 《关于使用238同位素产生能源的可能性》 G-92(1941年)  
\item Werner Heisenberg 《关于在柏林达赫伦的凯瑟尔·威廉物理研究所进行样品38与石蜡的层次排列实验的报告》 G-93(1941年5月)  
\item Fritz Bopp、Erich Fischer、Werner Heisenberg、Carl-Friedrich von Weizsäcker 和 Karl Wirtz 《使用铀金属和石蜡的新层次排列的研究》 G-127(1942年3月)  
\item Robert Döpel 《关于处理铀金属时发生的事故的报告》 G-135(1942年7月9日)  
\item Werner Heisenberg 《关于1.5吨D2O和3吨38金属的半技术实验计划的评论》 G-161(1942年7月31日)  
\item Werner Heisenberg、Fritz Bopp、Erich Fischer、Carl-Friedrich von Weizsäcker 和 Karl Wirtz 《对铀金属和石蜡的38金属层次排列的测量》 G-162(1942年10月30日)  
\item Robert Döpel、K. Döpel 和 Werner Heisenberg 《在D2O和铀金属的球形层次系统中有效中子增殖的实验验证》 G-136(1942年7月)  
\item Werner Heisenberg 《从原子核裂变中获取能源》 G-217(1943年5月6日)  
\item Fritz Bopp、Walther Bothe、Erich Fischer、Erwin Fünfer、Werner Heisenberg、O. Ritter 和 Karl Wirtz 《关于使用1.5吨D2O和铀及40厘米石墨回弹屏的实验报告(B7)》 G-300(1945年1月3日)  
\item Robert Döpel、K. Döpel 和 Werner Heisenberg 《在D2O-38金属层次系统中的中子增殖》 G-373(1942年3月)
\end{itemize}
\subsection{其他研究出版物}  
\begin{itemize}
\item \textbf{索末菲, A.; 海森堡, W.} (1922). *"关于相对论性X射线双线和线宽的一个注记"*. **Z. Phys.** 10 (1): 393–398. [Bibcode:1922ZPhy...10..393S](https://ui.adsabs.harvard.edu/abs/1922ZPhy...10..393S), [DOI:10.1007/BF01332582](https://doi.org/10.1007/BF01332582), [S2CID 123083509](https://scite.ai/reports/123083509).  
\item \textbf{索末菲, A.; 海森堡, W.} (1922). *"多重光谱线及其塞曼效应分量的强度"*. **Z. Phys.** 11 (1): 131–154. [Bibcode:1922ZPhy...11..131S](https://ui.adsabs.harvard.edu/abs/1922ZPhy...11..131S), [DOI:10.1007/BF01328408](https://doi.org/10.1007/BF01328408), [S2CID 186227343](https://scite.ai/reports/186227343).  
\item \textbf{玻恩, M.; 海森堡, W.} (1923). *"关于玻尔原子和分子模型中的相位关系"*. **Z. Phys.** 14 (1): 44–55. [Bibcode:1923ZPhy...14...44B](https://ui.adsabs.harvard.edu/abs/1923ZPhy...14...44B), [DOI:10.1007/BF01340032](https://doi.org/10.1007/BF01340032), [S2CID 186228402](https://scite.ai/reports/186228402).  
\item \textbf{玻恩, M.; 海森堡, W.} (1923). *"激发态氦原子的电子轨道"*. **Z. Phys.** 16 (9): 229–243. [Bibcode:1924AnP...379....1B](https://ui.adsabs.harvard.edu/abs/1924AnP...379....1B), [DOI:10.1002/andp.19243790902](https://doi.org/10.1002/andp.19243790902).  
\item \textbf{玻恩, M.; 海森堡, W.} (1924). *"分子量子理论"*. **Annalen der Physik.** 74 (4): 1–31. [Bibcode:1924AnP...379....1B](https://ui.adsabs.harvard.edu/abs/1924AnP...379....1B), [DOI:10.1002/andp.19243790902](https://doi.org/10.1002/andp.19243790902).  
\item \textbf{玻恩, M.; 海森堡, W.} (1924). *"离子的可变形性对光学和化学常数的影响. I"*. **Z. Phys.** 23 (1): 388–410. [Bibcode:1924ZPhy...23..388B](https://ui.adsabs.harvard.edu/abs/1924ZPhy...23..388B), [DOI:10.1007/BF01327603](https://doi.org/10.1007/BF01327603), [S2CID 186220818](https://scite.ai/reports/186220818).  
\item \textbf{海森堡, W. }(1924). *"关于流体稳定性与湍流的研究(博士论文)"*. **Annalen der Physik.** 74 (4): 577–627. [Bibcode:1924AnP...379..577H](https://ui.adsabs.harvard.edu/abs/1924AnP...379..577H), [DOI:10.1002/andp.19243791502](https://doi.org/10.1002/andp.19243791502).  
\item \textbf{海森堡, W.} (1924). *"关于量子理论形式规则在异常塞曼效应问题中的修正"*. **Z. Phys.** 26 (1): 291–307. [Bibcode:1924ZPhy...26..291H](https://ui.adsabs.harvard.edu/abs/1924ZPhy...26..291H), [DOI:10.1007/BF01327336](https://doi.org/10.1007/BF01327336), [S2CID 186215582](https://scite.ai/reports/186215582).  
\item \textbf{海森堡, W.} (1925). *"关于运动学与力学关系的量子理论重新解释"*. **Zeitschrift für Physik.** 33 (1): 879–893. [Bibcode:1925ZPhy...33..879H](https://ui.adsabs.harvard.edu/abs/1925ZPhy...33..879H), [DOI:10.1007/BF01328377](https://doi.org/10.1007/BF01328377), [S2CID 186238950](https://scite.ai/reports/186238950).该论文于1925年7月29日提交。[英文翻译](https://ui.adsabs.harvard.edu/abs/1968vanderWaerden...12) 收录于 **van der Waerden 1968**,标题为《量子理论对运动学与力学关系的重新解释》。这是著名的三部曲论文中的第一篇,奠定了量子力学矩阵力学表述的基础。
\item \textbf{玻恩, M.; 乔丹, P.} (1925). *"关于量子力学"*. **Zeitschrift für Physik** 34 (1): 858–888. [Bibcode:1925ZPhy...34..858B](https://ui.adsabs.harvard.edu/abs/1925ZPhy...34..858B), [DOI:10.1007/BF01328531](https://doi.org/10.1007/BF01328531), [S2CID 186114542](https://scite.ai/reports/186114542). 该论文于1925年9月27日提交。[英文翻译](https://ui.adsabs.harvard.edu/abs/1968vanderWaerden) 收录于 **van der Waerden 1968**,标题为《论量子力学》(On Quantum Mechanics)。这是著名的三部曲论文中的第二篇,奠定了量子力学矩阵力学表述的基础。  
\item \textbf{玻恩, M.; 海森堡, W.; 乔丹, P.} (1926). *"关于量子力学 II"*. **Zeitschrift für Physik** 35 (8–9): 557–615. [Bibcode:1926ZPhy...35..557B](https://ui.adsabs.harvard.edu/abs/1926ZPhy...35..557B), [DOI:10.1007/BF01379806](https://doi.org/10.1007/BF01379806), [S2CID 186237037](https://scite.ai/reports/186237037). 该论文于1925年11月16日提交。[英文翻译](https://ui.adsabs.harvard.edu/abs/1968vanderWaerden) 收录于 **van der Waerden 1968**,标题为《论量子力学 II》(On Quantum Mechanics II)。这是著名的三部曲论文中的第三篇,奠定了量子力学矩阵力学表述的基础。  
\item \textbf{海森堡, W.} (1927). *"关于量子理论运动学与力学的直观内容"*. **Zeitschrift für Physik** 43 (3–4): 172–198. [Bibcode:1927ZPhy...43..172H](https://ui.adsabs.harvard.edu/abs/1927ZPhy...43..172H), [DOI:10.1007/BF01397280](https://doi.org/10.1007/BF01397280), [S2CID 122763326](https://scite.ai/reports/122763326).  
\item \textbf{海森堡, W.} (1928). *"铁磁理论"*. **Zeitschrift für Physik** 49 (9–10): 619–636. [Bibcode:1928ZPhy...49..619H](https://ui.adsabs.harvard.edu/abs/1928ZPhy...49..619H), [DOI:10.1007/BF01328601](https://doi.org/10.1007/BF01328601), [S2CID 122524239](https://scite.ai/reports/122524239).  
\item \textbf{海森堡, W.; 泡利, W.} (1929). *"关于波动场的量子动力学"*. **Zeitschrift für Physik** 56 (1): 1–61. [Bibcode:1929ZPhy...56....1H](https://ui.adsabs.harvard.edu/abs/1929ZPhy...56....1H), [DOI:10.1007/BF01340129](https://doi.org/10.1007/BF01340129), [S2CID 121928597](https://scite.ai/reports/121928597).  
\item \textbf{海森堡, W.; 泡利, W.}(1930). *"关于波动场的量子理论 II"*. **Zeitschrift für Physik** 59 (3–4): 168–190. [Bibcode:1930ZPhy...59..168H](https://ui.adsabs.harvard.edu/abs/1930ZPhy...59..168H), [DOI:10.1007/BF01341423](https://doi.org/10.1007/BF01341423), [S2CID 186219228](https://scite.ai/reports/186219228).  
\item \textbf{海森堡, W.} (1932). *"关于原子核结构 I"*. **Zeitschrift für Physik** 77 (1–2): 1–11. [Bibcode:1932ZPhy...77....1H](https://ui.adsabs.harvard.edu/abs/1932ZPhy...77....1H), [DOI:10.1007/BF01342433](https://doi.org/10.1007/BF01342433), [S2CID 186218053](https://scite.ai/reports/186218053).  
\item \textbf{海森堡, W.} (1932). *"关于原子核结构 II"*. **Zeitschrift für Physik** 78 (3–4): 156–164. [Bibcode:1932ZPhy...78..156H](https://ui.adsabs.harvard.edu/abs/1932ZPhy...78..156H), [DOI:10.1007/BF01337585](https://doi.org/10.1007/BF01337585), [S2CID 186221789](https://scite.ai/reports/186221789).  
\item \textbf{海森堡, W.} (1933). *"关于原子核结构 III"*. **Zeitschrift für Physik** 80 (9–10): 587–596. [Bibcode:1933ZPhy...80..587H](https://ui.adsabs.harvard.edu/abs/1933ZPhy...80..587H), [DOI:10.1007/BF01335696](https://doi.org/10.1007/BF01335696), [S2CID 126422047](https://scite.ai/reports/126422047).
\item \textbf{海森堡, W.} (1934). *"关于狄拉克正电子理论的注释"*. **Zeitschrift für Physik** 90 (3–4): 209–231. [Bibcode:1934ZPhy...90..209H](https://ui.adsabs.harvard.edu/abs/1934ZPhy...90..209H), [DOI:10.1007/BF01333516](https://doi.org/10.1007/BF01333516), [S2CID 186232913](https://scite.ai/reports/186232913). 该论文于1934年6月21日提交,作者当时在莱比锡。  
\item \textbf{海森堡, W.} (1936). *"宇宙射线‘暴’现象"*. **Forsch. Fortscher.** 12: 341–342.  
\item \textbf{海森堡, W.; 欧拉, H.} (1936). *"狄拉克正电子理论的推论"*. **Zeitschrift für Physik** 98 (11–12): 714–732. [Bibcode:1936ZPhy...98..714H](https://ui.adsabs.harvard.edu/abs/1936ZPhy...98..714H), [DOI:10.1007/BF01343663](https://doi.org/10.1007/BF01343663), [S2CID 120354480](https://scite.ai/reports/120354480).该论文于1935年12月22日提交,作者当时在莱比锡。该论文由 **W. Korolevski 和 H. Kleinert** 翻译,译文可见于 [arXiv:physics/0605038v1](https://arxiv.org/abs/physics/0605038v1)。  
\item \textbf{海森堡, W.} (1936). *"关于高空辐射‘暴’的理论"*. **Zeitschrift für Physik** 101 (9–10): 533–540. [Bibcode:1936ZPhy..101..533H](https://ui.adsabs.harvard.edu/abs/1936ZPhy..101..533H), [DOI:10.1007/BF01349603](https://doi.org/10.1007/BF01349603), [S2CID 186215469](https://scite.ai/reports/186215469).  
\item \textbf{海森堡, W.} (1937). *"高能粒子穿过原子核"*. **Die Naturwissenschaften** 25 (46): 749–750. [Bibcode:1937NW.....25..749H](https://ui.adsabs.harvard.edu/abs/1937NW.....25..749H), [DOI:10.1007/BF01789574](https://doi.org/10.1007/BF01789574), [S2CID 39613897](https://scite.ai/reports/39613897).  
\item \textbf{海森堡, W.} (1937). *"关于超高能辐射的理论研究"*. **Verh. Dtsch. Phys. Ges.** 18: 50.  
\item \textbf{海森堡, W.} (1938). *"高空辐射中穿透成分的吸收"*. **Annalen der Physik** 425 (7): 594–599. [Bibcode:1938AnP...425..594H](https://ui.adsabs.harvard.edu/abs/1938AnP...425..594H), [DOI:10.1002/andp.19384250705](https://doi.org/10.1002/andp.19384250705).  
\item \textbf{海森堡, W.} (1938). *"高能粒子穿过原子核"*. **Nuovo Cimento** 15 (1): 31–34. [Bibcode:1938NCim...15...31H](https://ui.adsabs.harvard.edu/abs/1938NCim...15...31H), [DOI:10.1007/BF02958314](https://doi.org/10.1007/BF02958314), [S2CID 123209538](https://scite.ai/reports/123209538).  
\item \textbf{海森堡, W.} (1938). *"高能粒子穿过原子核"*. **Verh. Dtsch. Phys. Ges.** 19 (2).  
\item \textbf{海森堡, W.} (1943). *"基本粒子理论中的可观测量 I"*. **Zeitschrift für Physik** 120 (7–10): 513–538. [Bibcode:1943ZPhy..120..513H](https://ui.adsabs.harvard.edu/abs/1943ZPhy..120..513H), [DOI:10.1007/BF01329800](https://doi.org/10.1007/BF01329800), [S2CID 120706757](https://scite.ai/reports/120706757).  
\item \textbf{海森堡, W.} (1943). *"基本粒子理论中的可观测量 II"*. **Zeitschrift für Physik** 120 (11–12): 673–702. [Bibcode:1943ZPhy..120..673H](https://ui.adsabs.harvard.edu/abs/1943ZPhy..120..673H), [DOI:10.1007/BF01336936](https://doi.org/10.1007/BF01336936), [S2CID 124531901](https://scite.ai/reports/124531901).  
\item \textbf{海森堡, W.} (1944). *"基本粒子理论中的可观测量 III"*. **Zeitschrift für Physik** 123 (1–2): 93–112. [Bibcode:1944ZPhy..123...93H](https://ui.adsabs.harvard.edu/abs/1944ZPhy..123...93H), [DOI:10.1007/BF01375146](https://doi.org/10.1007/BF01375146), [S2CID 123698415](https://scite.ai/reports/123698415).  
\item \textbf{海森堡, W.} (1947). *"关于超导理论"*. **Forsch. Fortschr.** 21/23: 243–244.  
\item \textbf{海森堡, W.}(1947). *"关于超导理论"*. **Zeitschrift für Naturforschung A** 2 (4): 185–201. [Bibcode:1947ZNatA...2..185H](https://ui.adsabs.harvard.edu/abs/1947ZNatA...2..185H), [DOI:10.1515/zna-1947-0401](https://doi.org/10.1515/zna-1947-0401).
\item \textbf{海森堡, W.} (1948). *"超导体的电动力学行为"*. **Zeitschrift für Naturforschung A** 3a (2): 65–75. [Bibcode:1948ZNatA...3...65H](https://ui.adsabs.harvard.edu/abs/1948ZNatA...3...65H), [DOI:10.1515/zna-1948-0201](https://doi.org/10.1515/zna-1948-0201).  
\item \textbf{海森堡, W.; 范·劳厄, M.} (1948). *"超导材料制成的巴洛斯轮"*. **Zeitschrift für Physik** 124 (7–12): 514–518. [Bibcode:1948ZPhy..124..514H](https://ui.adsabs.harvard.edu/abs/1948ZPhy..124..514H), [DOI:10.1007/BF01668888](https://doi.org/10.1007/BF01668888), [S2CID 121271077](https://scite.ai/reports/121271077).  
\item \textbf{海森堡, W.}(1948). *"关于湍流的统计理论"*. **Zeitschrift für Physik** 124 (7–12): 628–657. [Bibcode:1948ZPhy..124..628H](https://ui.adsabs.harvard.edu/abs/1948ZPhy..124..628H), [DOI:10.1007/BF01668899](https://doi.org/10.1007/BF01668899), [S2CID 186223726](https://scite.ai/reports/186223726).  
\item \textbf{海森堡, W.} (1948). *"关于统计湍流和各向同性湍流的理论"*. **Proceedings of the Royal Society A** 195 (1042): 402–406. [Bibcode:1948RSPSA.195..402H](https://ui.adsabs.harvard.edu/abs/1948RSPSA.195..402H), [DOI:10.1098/rspa.1948.0127](https://doi.org/10.1098/rspa.1948.0127).  
\item \textbf{海森堡, W.} (1948). *"湍流问题的注释"*. **Zeitschrift für Naturforschung A** 3a (8–11): 434–437. [Bibcode:1948ZNatA...3..434H](https://ui.adsabs.harvard.edu/abs/1948ZNatA...3..434H), [DOI:10.1515/zna-1948-8-1103](https://doi.org/10.1515/zna-1948-8-1103), [S2CID 202047340](https://scite.ai/reports/202047340).  
\item \textbf{海森堡, W.} (1949). *"介子暴的产生"*. **Nature** 164 (4158): 65–67. [Bibcode:1949Natur.164...65H](https://ui.adsabs.harvard.edu/abs/1949Natur.164...65H), [DOI:10.1038/164065c0](https://doi.org/10.1038/164065c0), [PMID 18228928](https://pubmed.ncbi.nlm.nih.gov/18228928), [S2CID 4043099](https://scite.ai/reports/4043099).  
\item \textbf{海森堡, W.} (1949). *"在多重过程中的介子产生"*. **Nuovo Cimento** 6 (补刊): 493–497. [Bibcode:1949NCim....6S.493H](https://ui.adsabs.harvard.edu/abs/1949NCim....6S.493H), [DOI:10.1007/BF02822044](https://doi.org/10.1007/BF02822044), [S2CID 122006877](https://scite.ai/reports/122006877).  
\item \textbf{海森堡, W.} (1949). *"多重过程中的介子生成"*. **Zeitschrift für Physik** 126 (6): 569–582. [Bibcode:1949ZPhy..126..569H](https://ui.adsabs.harvard.edu/abs/1949ZPhy..126..569H), [DOI:10.1007/BF01330108](https://doi.org/10.1007/BF01330108), [S2CID 120410676](https://scite.ai/reports/120410676).  
\item \textbf{海森堡, W.} (1950). *"层流的稳定性"*. **Proc. International Congress Mathematicians** II: 292–296.  
\item \textbf{海森堡, W.} (1952). *"关于介子多重生成理论的注释"*. **Die Naturwissenschaften** 39 (3): 69. [Bibcode:1952NW.....39...69H](https://ui.adsabs.harvard.edu/abs/1952NW.....39...69H), [DOI:10.1007/BF00596818](https://doi.org/10.1007/BF00596818), [S2CID 41323295](https://scite.ai/reports/41323295).  
\item \textbf{海森堡, W.} (1952). *"介子生成作为冲击波问题"*. **Zeitschrift für Physik** 133 (1–2): 65–79. [Bibcode:1952ZPhy..133...65H](https://ui.adsabs.harvard.edu/abs/1952ZPhy..133...65H), [DOI:10.1007/BF01948683](https://doi.org/10.1007/BF01948683), [S2CID 124271377](https://scite.ai/reports/124271377).  
\item \textbf{海森堡, W.} (1955). *"在非常高能碰撞中的介子生产"*. **Nuovo Cimento** 12 (补刊): 96–103. [Bibcode:1955NCim....2S..96H](https://ui.adsabs.harvard.edu/abs/1955NCim....2S..96H), [DOI:10.1007/BF02746079](https://doi.org/10.1007/BF02746079), [S2CID 121970196](https://scite.ai/reports/121970196).  
\item \textbf{海森堡, W.} (1975). *"量子理论历史中概念的发展"*. **American Journal of Physics** 43 (5): 389–394. [Bibcode:1975AmJPh..43..389H](https://ui.adsabs.harvard.edu/abs/1975AmJPh..43..389H), [DOI:10.1119/1.9833](https://doi.org/10.1119/1.9833). 本文的内容由海森堡于哈佛大学的讲座中呈现。
\end{itemize}
\subsection{出版书籍}
\begin{itemize}
\item \textbf{海森堡, W.} (1949) [1930]. *《量子理论的物理原理》*. 翻译:Eckart, Carl; Hoyt, F.C. Dover. ISBN 978-0-486-60113-7.
\item \textbf{海森堡, W.} (1953). *《核物理学》*. Philosophical Library.
\item \textbf{海森堡, W.} (1955). *《当代物理学的自然观》*. Rowohlts Enzyklopädie. 第8卷. Rowohlt.
- **\textbf{海森堡, W.} (1958). *《物理学与哲学》*. Harper & Rowe.
- **\textbf{海森堡, W.} (1966). *《核科学的哲学问题》*. Fawcett.
- **\textbf{海森堡, W.} (1971). *《物理学与超越:遭遇与对话》*. Harper & Row. ISBN 9780061316227.
- **\textbf{海森堡, W.} (1971). *《物理学与超越:遭遇与对话》*.
- **\textbf{海森堡, W.} (1977). *《科学中的传统:演讲与论文》*. 慕尼黑:Piper.
- **\textbf{海森堡, W.; Busche, Jürgen}(1979). *《量子理论与哲学:讲座与论文》*. Reclam. ISBN 978-3-15-009948-3.
- **\textbf{海森堡, W.} (1979). *《量子物理学的哲学问题》*. Ox Bow. ISBN 978-0-918024-14-5.
- **\textbf{海森堡, W.} (1983). *《科学中的传统》*. Seabury Press.
- **\textbf{海森堡, W.} (1988). *《物理学与哲学:世界视野》*. Ullstein Taschenbuchvlg.
- **\textbf{海森堡, W}. (1989). *《与爱因斯坦的遭遇:以及关于人物、地点与粒子的其他文章》*. 普林斯顿大学出版社. ISBN 978-0-691-02433-2.
- **\textbf{海森堡, W.; Northrop, Filmer}(1999). *《物理学与哲学:现代科学的革命》(伟大的思想家系列)* . Prometheus.
- **\textbf{海森堡, W.} (2002). *《部分与整体:围绕原子物理学的对话》*. Piper. ISBN 978-3-492-22297-6.
- **\textbf{海森堡, W.; Rechenberg, Helmut} (1992). *《德国物理学与犹太物理学》*. Piper. ISBN 978-3-492-11676-3.
- **\textbf{海森堡, W.} (2007). *《物理学与哲学:世界视野》*. Hirzel.
- **\textbf{海森堡, W.} (2007). *《物理学与哲学:现代科学的革命》*. Harper Perennial Modern Classics(再版版)。HarperCollins. ISBN 978-0-06-120919-2. (1958年版本的全文)
\end{itemize}