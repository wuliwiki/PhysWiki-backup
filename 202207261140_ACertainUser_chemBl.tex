% 化学反应平衡

\begin{issues}
\issueDraft
\end{issues}

\pentry{化学势, 态函数\upref{statef}}

对于一个多元系统,可以写出
\begin{equation}
G=G(p,T,n_1,n_2,...)
\end{equation}
那么,
\begin{equation}
\dd G=-S \dd T +V \dd P + \sum \mu_B \dd n_B
\end{equation}
在等温等压下,p,T均为定值
\begin{equation}
\dd G=-\sum \mu_B \dd n_B
\end{equation}

代入化学反应进度 $\dd \xi=\dd n_B/\nu_b$,得
\begin{equation}
\dd G=-\sum \nu_B \mu_B \dd \xi
\end{equation}
$\nu$: 化学反应系数

或写为导数形式
\begin{equation}
\dv{G}{\xi}=-\sum \nu_B \mu_B
\end{equation}

定义摩尔Gibbs变
\begin{equation}
\Delta _r G_M = \dv{G}{\xi}=-\sum \nu_B \mu_B
\end{equation}
含义为“化学反应进度再多进行一点,系统的Gibbs能变”.

\subsection{热力学判据}
根据吉布斯自由能判据\upref{GibbsG},
若 $\Delta _r G_M<0$,则反应会继续进行;若$\Delta _r G_M = 0$,则反应达到平衡.

\subsection{活度积}
注意到 
$\mu_B=\mu_B^*+RT \ln x_B$,那么
\begin{equation}
\Delta _r G_M =-\sum \nu_B \mu_B = -\sum \nu_B (\mu_B^*+RT\ln _B)=-\sum \nu_B \mu_B^* -\sum \nu_B RT \ln _B
\end{equation}
