% 引力量子化
 
\begin{issues}
\issueNeedCite
\issueMissDepend
\end{issues}
 
 广义相对论包含引力波。他们带有 $J = 2$ 的轨道角动量。对引力波做量子化能得到自旋2的引力子。弦理论自然包含了引力子,因此也是量子引力的有力候选者之一。但是引力子的圈图会有发散。经过计算,我们得到引力子圈图
\begin{equation}
I \sim \int d^4 p \rightarrow \infty ~,
\end{equation}
这意味着引力是不可重整的。也就是说,把引力纳入量子场论会产生巨大的问题。弦理论处理这个问题的办法是,用一个一维的弦来代替零维的点粒子。这样的化相互作用就不会在一个点发生了。我们看如下的不确定性关系
\begin{equation}
\Delta x \Delta p \sim \hbar  
\end{equation}
我们可以看出发散的真正原因是因为 $\Delta x \rightarrow 0$ 导致的 $\Delta p \rightarrow  \infty  $。  为了解决发散问题,在弦理论中,不确定性关系有如下修正
\begin{equation}
\Delta x  = \frac{\hbar}{\Delta p} + \alpha' \frac{\Delta p}{\hbar }
\end{equation}
其中 $\alpha'$ 跟弦的张力之间的关系是
\begin{equation}
\alpha ' = \frac{1}{2 \pi T_s} 
\end{equation}
弦理论的最小距离单位是
\begin{equation}
x_{\rm min} \sim 2 \sqrt{\alpha'}
\end{equation}
