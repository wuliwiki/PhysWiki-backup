% 热力学量的统计表达式(玻尔兹曼分布)
% 热力学量|统计力学|配分函数|玻尔兹曼分布

\pentry{玻尔兹曼分布(统计力学)\upref{MBsta}}

\subsection{配分函数}

满足经典极限\footnote{在玻尔兹曼分布\upref{MBsta} 词条中谈到了玻色分布和费米分布的表达式,式中如果 $e^\alpha\gg 1$,那么将过度到经典情况的玻尔兹曼分布.我们称这个条件为经典极限.}的大量粒子组成的系统中,粒子遵从玻尔兹曼分布.我们可以试图用统计力学中配分函数来推出一切热力学量.

配分函数表达式为:
\begin{equation}
Z_1=\sum_l \omega_l e^{-\beta \epsilon_l}
\end{equation}

式中 $\omega_l$ 为能级的简并度.根据玻尔兹曼分布,每个能级上的粒子数为 $e^{-\alpha-\beta\epsilon_l}$.于是有
\begin{equation}
\begin{aligned}
&N=\sum_l \omega_l e^{-\alpha-\beta\epsilon_l}=e^{-\alpha} Z_1\\
&E=\sum_l \epsilon_l \omega_l e^{-\alpha-\beta\epsilon_l}=-e^{-\alpha}\frac{\partial Z_1}{\partial \beta}=-\frac{N}{Z_1}\frac{\partial Z_1}{\partial \beta}=-N\frac{\partial \ln Z_1}{\partial \beta}
\end{aligned}
\end{equation}

对于一个粒子数 $N$ 和内能 $E$ 确定的系统,$\alpha,\beta$ 可以由上面两个表达式确定.

由热力学第一定律,系统可以通过功和热量两种方式与外界交换能量.例如在可逆过程中做功可以写为 $Y\dd y$,$Y$ 为广义力,$y$ 为广义位移\footnote{例如,$Y$ 取 $-P$,$y$ 取 $V$ 对应体积压缩做功;$Y$ 取电场强度 $E$,$y$ 取电极化强度 $P$ 可以表示外界使介质极化需要做的功}.当系统发生广义位移,能级也会发生变化.外界对能级 $\epsilon_l$ 熵一个粒子的力为 $\frac{\partial \epsilon_l}{\partial y}$.因此可以表示出 $Y$:
\begin{equation}
\begin{aligned}
Y&=\sum_l a_l\frac{\partial \epsilon_l}{\partial y}=\sum_l \frac{\partial \epsilon_l}{\partial y}\omega_le^{-\alpha-\beta\epsilon_l}\\
&=e^{-\alpha}\qty(-\frac{1}{\beta}\frac{\partial }{\partial y})Z_1\\
&=-\frac{N}{\beta}\frac{\partial }{\partial y}\ln Z_1
\end{aligned}
\end{equation}
例如将上面的 $y$ 替换成 $V$,可以得到压强 $P$ 的表达式.

假设系统发生一个 $\dd y$ 的广义位移的变化.则外界对系统做的功为
\begin{equation}
Y\dd y=\sum_l a_l\dd \epsilon_l=-\frac{N}{\beta}\frac{\partial }{\partial y}\ln Z_1
\end{equation}
这时内能的改变为
\begin{equation}
\dd U=\sum_l a_l\dd \epsilon_l+\sum_l\epsilon_l\dd a_l=\delta W+\delta Q=Y\dd y+\delta Q
\end{equation}
