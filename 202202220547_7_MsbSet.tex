% 可测集合
% 实变函数|Lebesgue积分

\pentry{集合的测度(实变函数)\upref{SetMet}}


在\textbf{集合的测度(实变函数)}\upref{SetMet}中我们提到,Lebesgue积分是对值域作分划.但是值域作分划以后,有可能出现\autoref{SetMet_ex1}~\upref{SetMet}那样病态的两个集合,它们互不相交,但其并集的测度小于各自测度之和.这就好比对定义域作分划的Riemann积分中,两根柱子的总面积小于单个柱子的面积之和,非常不符合直觉,而且会导致无法建立自洽的理论.

因此,讨论Lebesgue积分时,我们要对集合的性质作出限制,只讨论所谓的\textbf{可测集合(measurable set)},以及进而导出的\textbf{可测函数(measurable function)}.













