% 自治系统解的特点
% keys 自治系统|解的性质
% license Xiao
% type Tutor

\pentry{基本知识(常微分方程)\upref{ODEPr}}
本节将说明自治系统(\autoref{def_ODEPr_2}~\upref{ODEPr})解的两个性质,其中一个性质在物理上相当于说不同场线不相交。一般的自治系统的对应的标准方程组(\autoref{def_ODEPr_1}~\upref{ODEPr})可写为
\begin{equation}\label{eq_AuSy_1}
y'_i=f_i(y_1,\cdots,y_n),\quad i=1,\cdots,n~.
\end{equation}
采用矢量写法为
\begin{equation}\label{eq_AuSy_4}
y'=f(y)~,
\end{equation}
其中 $y=(y_1,\cdots,y_n),f=(f_1,\cdots,f_n)$。与“基本知识(常微分方程)\upref{ODEPr}”中类似,我们总假定 $f_i(y_1,\cdots,y_n)$ 及其一阶偏导数在其定义区间(记为 $\Delta$)上连续。约定:当出现关于指标 $i$ 的表达式而不指出其取值范围时,就代表对每个 $i$ 所能取的值表达式均成立。

\begin{theorem}{}\label{the_AuSy_1}
若 $y_i=\varphi_i(x)$ 是方程组\autoref{eq_AuSy_1} 的解,则 $y_i=\varphi^*_i(x)=\varphi_i(x+c)$ 也是方程组\autoref{eq_AuSy_1} 的解,其中 $c$ 是任意常数。
\end{theorem}
\textbf{证明:}由复合函数求导法则\upref{ChainR},成立
\begin{equation}\label{eq_AuSy_2}
\begin{aligned}
{\varphi^*_i}'(x)&=\dv{\varphi^*_i(x)}{x}=\dv{\varphi_i(x+c)}{x}\\
&=\dv{\varphi_i(x+c)}{(x+c)}\dv{(x+c)}{x}=\varphi'_i(x+c)~.
\end{aligned}
\end{equation}
由于 $y_i=\varphi_i(x)$ 是解,所以
\begin{equation}
\begin{aligned}
\varphi'_i(x+c)=f_i(\varphi_1(x+c),\cdots,\varphi_n(x+c))~,
\end{aligned}
\end{equation}
恒成立。\autoref{eq_AuSy_2} 带入上式,即得
\begin{equation}
{\varphi^*_i}'(x)=f_i(\varphi^*_1(x),\cdots,\varphi^*_n(x))~.
\end{equation}

\textbf{证毕!}

显然,若解 $\varphi_1(x)$ 有最大存在区间\upref{ODEPr} $(m_1,m_2)$,则 $\varphi^*(x)$ 有最大存在区间 $(m_1-c,m_2-c)$.

自治系统(\autoref{eq_AuSy_1})的解定义在 $n$ 维空间上,随 $x$ 的变化 $y^i$ 在空间描出一条曲线,这条曲线称为\textbf{轨线}。若把解看成轨线而非运动过程,则需要指出在轨线上指出运动的方向,因为相反的运动方向\autoref{eq_AuSy_1} 右边差一负号。

\begin{theorem}{不同轨线不相交}\label{the_AuSy_2}
若两轨线 $\varphi_i,\psi_i$ 有一公共点,即 $\varphi_i(x_1)=\psi_i(x_1)$,则
\begin{equation}\label{eq_AuSy_3}
\psi_i(x)=\varphi_i(x+x_1-x_2)~.
\end{equation}
也就是说两轨线相合。
\end{theorem}
\textbf{证明:}由\autoref{the_AuSy_1} ,$\varphi^*_i(x)=\varphi_i(x+x_1-x_2)$ 也是解。由于 $\varphi^*_i(x_2)=\varphi_i(x_1)=\phi_i(x_2)$,知 $\varphi^*_i,\phi_i$ 有相同初始值 $x_2$。根据唯一性定理,它们是相同的。即\autoref{eq_AuSy_3} 成立。

\textbf{证毕!}

在物理中,物理系统的哈密顿函数 $H$ 定义了物理系统在相空间的矢量场 $(\pdv{H}{p_i},-\pdv{H}{q_i})$ , 物理系统满足正则方程(\autoref{eq_HamCan_2}~\upref{HamCan}):
\begin{equation}
\dot q_i=\pdv{H}{p_i},\quad \dot o_i=-\pdv{H}{q_i}.
\end{equation}
若令 $x=t,y=(q_1,\cdots,q_n,p_1,\cdots,p_n),f=(\pdv{H}{p_1},\cdots,\pdv{H}{p_n},-\pdv{H}{q_1,\cdots,-\pdv{H}{q_n}})$,则上式成为常微分方程组\autoref{eq_AuSy_4} ,正则方程的解也称为物理系统的场线(由于正则方程和场方程,亦即欧拉拉格朗日方程等价)。\autoref{the_AuSy_2} 表明物理系统不同的场线不相交。

人们一般熟悉电磁场的场线的概念,其上点的切线方向和电磁场在该点的方向相同,若用 $\bvec A$ 表示电磁场(电场或磁场),那么其场线满足方程
\begin{equation}
\dv{x}{A_x}=\dv{y}{A_y}=\dv{z}{A_z}~,
\end{equation}
上式等价于
\begin{equation}
\dv{x}{s}=A_x,\quad\dv{y}{s}=A_y,\quad\dv{z}{s}=A_z,
\end{equation}
其中 $s$ 代表场线的参数,也可以想象为弧长参量,可以发现,它和
