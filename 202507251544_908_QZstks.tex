% 乔治·斯托克斯(综述)
% license CCBYSA3
% type Wiki

本文根据 CC-BY-SA 协议转载翻译自维基百科\href{https://en.wikipedia.org/wiki/Sir_George_Stokes,_1st_Baronet}{相关文章}。


乔治·加布里埃尔·斯托克斯爵士,第一代从男爵(/stoʊks/;1819年8月13日-1903年2月1日),是爱尔兰数学家和物理学家。斯托克斯出生于爱尔兰斯莱戈郡,在剑桥大学度过了整个职业生涯,并在1849年至1903年去世期间担任卢卡斯数学教授长达54年,是该职位任期最长的持有者。

作为物理学家,斯托克斯在流体力学领域作出了开创性的贡献,包括纳维-斯托克斯方程;在物理光学方面,他的研究涵盖偏振和荧光等现象。作为数学家,他普及了矢量微积分中的“斯托克斯定理”,并对渐近展开理论作出了贡献。斯托克斯与菲利克斯·霍普-塞勒一道,首次揭示了血红蛋白的携氧功能,并展示了血红蛋白溶液通气后所产生的颜色变化。

1889年,斯托克斯被英国君主封为从男爵。1893年,他因“在物理科学领域的研究与发现”获得当时全球最负盛名的科学奖项——皇家学会的科普利奖章。他曾于1887年至1892年在英国下议院担任剑桥大学选区的议员,隶属保守党。斯托克斯还于1885年至1890年担任皇家学会会长,并曾短暂出任剑桥大学彭布罗克学院院长。由于他的大量通信往来以及担任皇家学会秘书期间的工作,他被称为维多利亚时代科学的大门守卫者,其贡献远远超越了他发表的论文本身\(^\text{[1]}\)。
\subsection{传记}
乔治·斯托克斯是加布里埃尔·斯托克斯牧师(1834年去世)的幼子。加布里埃尔是爱尔兰圣公会的牧师,担任斯莱戈郡斯克林的教区牧师;其母为伊丽莎白·霍顿,是约翰·霍顿牧师的女儿。斯托克斯的家庭生活深受父亲福音派新教信仰的影响,他的三个兄弟也都进入教会,其中最杰出的是阿马郡副主教约翰·惠特利·斯托克斯\(^\text{[2]}\)。斯托克斯自幼对新教信仰虔诚,他在斯克林的童年经历也对他后来的研究方向产生了重大影响,尤其是他选择流体力学作为研究领域\(^\text{[3]}\)。他的女儿伊莎贝拉·汉弗莱斯曾写道,她父亲“告诉我,他小时候在斯莱戈海岸游泳时差点被一股大浪卷走,这第一次引起了他对波浪的兴趣”\(^\text{[4]}\)。

约翰与乔治兄弟情深,乔治在都柏林上学期间就住在约翰家中。在所有家庭成员中,他与妹妹伊丽莎白关系最为亲近。家族中回忆其母亲是“美丽但非常严厉”的。斯托克斯在斯克林、都柏林和布里斯托尔的学校接受教育,1837年,他进入剑桥大学彭布罗克学院。四年后,他以“高级数学状元”和史密斯一等奖毕业,凭此成就他被选为该学院研究员\(^\text{[5]}\)。

根据当时学院章程,斯托克斯于1857年结婚时必须辞去研究员职务。十二年后,在新章程下,他重新当选为研究员,并一直担任此职直到1902年。在他83岁生日的前一天,他被选为彭布罗克学院院长。然而他担任院长时间不长,于次年(1903年)2月1日在剑桥去世,并被安葬于米尔路公墓\(^\text{[6]}\)。他在西敏寺北侧走廊也有一块纪念碑\(^\text{[7]}\)。
\subsubsection{职业生涯}
1849年,乔治·斯托克斯被任命为剑桥大学卢卡斯数学教授(Lucasian Professor of Mathematics),他一直担任这一职位直到1903年去世。1899年6月1日,剑桥大学为庆祝斯托克斯就任卢卡斯教授五十周年举行了纪念典礼,众多来自欧洲和美国的大学代表参加了此次活动。剑桥大学校长向斯托克斯颁发了纪念金质奖章;哈莫·索尼克罗夫特(Hamo Thornycroft)为斯托克斯雕刻的大理石半身像,被开尔文勋爵正式赠送给彭布罗克学院和剑桥大学收藏。斯托克斯担任卢卡斯数学教授的时间长达54年,为历史上最长。

斯托克斯于1889年获封从男爵(Baronet),并在1887年至1892年期间代表剑桥大学选区担任英国议会议员,为母校继续服务。此外,他还在1885年至1890年间担任英国皇家学会会长,在此之前自1854年起便担任该学会秘书。由于他同时还是卢卡斯数学教授,斯托克斯也成为第一个同时担任这三个职位的人;牛顿虽然也曾担任过这三个职务,但并非同时\(^\text{[6]}\)。

斯托克斯是与詹姆斯·克拉克·麦克斯韦和开尔文勋爵齐名的三位自然哲学家中最年长的一位,他们三人是剑桥数学物理学派在19世纪中叶声名显赫的主要奠基人。

斯托克斯的原创性研究始于1840年左右,其成果在数量和质量上都堪称卓越。根据英国皇家学会的科学论文目录,他在1883年之前已发表逾百篇论文。这些作品中有些只是简短的笔记,有些是简要的争论性或纠正性陈述,但也有大量是长篇且详尽的论文\(^\text{[8]}\)。
\subsection{对科学的贡献}
在研究范围上,斯托克斯的工作涵盖了广泛的物理问题,但正如玛丽·阿尔弗雷德·科尔努在1899年的瑞德讲座中指出的那样,\(^\text{[9]}\)他的大部分研究都集中在波动及其在穿越不同介质时所经历的变化上。\(^\text{[10]}\)
\subsubsection{流体动力学}
斯托克斯最早发表的论文出现在1842年和1843年,内容是关于不可压缩流体的稳态运动以及某些流体运动情形的分析。\(^\text{[11][12]}\)随后,他于1845年发表了一篇关于流体在运动中所产生的摩擦力以及弹性固体的平衡与运动的论文,\(^\text{[13]}\)1850年又发表了一篇探讨流体内摩擦对钟摆运动影响的文章。\(^\text{[14]}\)他还对声学理论作出了一些贡献,包括讨论风对声音强度的影响,\(^\text{[15]}\)以及气体性质如何影响声波强度的解释。\(^\text{[16]}\)这些研究共同为流体动力学奠定了新的基础,不仅解释了许多自然现象(如云悬浮于空中、水面涟漪与波浪的消退),也为诸多实际问题的解决提供了关键理论支持,例如河流与渠道中水流的行为、以及船体的表面阻力问题。\(^\text{[10]}\)
