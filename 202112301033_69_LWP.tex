% 李纳-维谢尔势
% 李纳-维谢尔势|带电粒子的辐射

\pentry{电磁场推迟势\upref{RetPt0},狄拉克 delta 函数\upref{Delta}}

对于一个给定的电荷电流分布(作为关于时空坐标 $x^\mu$ 的函数),可以根据\textbf{推迟势}算出电磁势$V,\bvec A$(作为关于时空坐标 $x^\mu$ 的函数).
\begin{equation}
V(\bvec r,t)=\frac{1}{4\pi\epsilon_0}\int \frac{\rho(\bvec r',t-|\bvec r-\bvec r'|/c)}{|\bvec r-\bvec r'|}\dd V'
\end{equation}

\begin{equation}
\bvec A(\bvec r,t)=\frac{\mu_0}{4\pi}\int \frac{\bvec J(\bvec r',t-|\bvec r-\bvec r'|/c)}{|\bvec r-\bvec r'|}\dd V'
\end{equation}

或者写成四维协变形式:设四维电磁势 $A^\mu=(V/c,\bvec A)$,四维电荷电流密度 $J^\mu=(\rho c,\bvec J)$,那么
\begin{equation}
A^\mu(\bvec r,t)=\frac{1}{4\pi\epsilon_0c^2}\int \frac{J^\mu(\bvec r',t-|\bvec r-\bvec r'|/c)}{|\bvec r-\bvec r'|}\dd V'
\end{equation}

在自然单位制下,上式为
\begin{equation}
A^\mu(\bvec r,t)=\int \frac{J^\mu(\bvec r',t-|\bvec r-\bvec r'|)}{|\bvec r-\bvec r'|}\dd V'
\end{equation}

如果要考察一个带电粒子的辐射,带电粒子的电荷密度为 $\rho(\bvec r,t)=q\delta(\bvec r-\bvec r''(t))$,电流密度为 $q\bvec v''(t) \delta(\bvec r-\bvec r''(t))$.此时可以对推迟势进行化简,得到\textbf{李纳-维谢尔势}:
\subsection{李纳-维谢尔势}

对于电荷密度为 $\rho(\bvec r,t)=q\delta(\bvec r-\bvec r''(t))$,电流密度为 $q\bvec v''(t) \delta(\bvec r-\bvec r''(t))$ 的带电粒子,它产生的电磁势可以由下式给出
\begin{equation}
A^\mu(\bvec r,t)=\int \frac{J^\mu(\bvec r,t)}{|\bvec r-\bvec r'|}
\end{equation}