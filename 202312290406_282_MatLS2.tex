% 秩-零化度定理
% keys 线性变换|线性映射|补空间|线性空间|零空间|值空间
% license Xiao
% type Tutor

\begin{issues}
\issueTODO %用更规范的语言描述这个定义。
\end{issues}

\pentry{补空间\upref{DirSum},线性映射\upref{LinMap}}

\addTODO{需要在《线性代数》part中写一个矩阵版本}
% prt_LA1

如果线性映射是单射\upref{map}的(即定义域空间到值域空间\upref{LinMap}是双射的), 那么它的结构简单明了, 没有太多值得讨论的。 我们现在来看一个揭示多对一线性映射的结构的重要定理。 考虑线性映射 $A:X\to Y$, 线性空间 $X$ 的零空间(\autoref{def_LinMap_2}~\upref{LinMap})$X_0$ 中的每个矢量都被 $A$ 映射到 $Y$ 空间中的零矢量, 即 $A(X_0) = \qty{0}$。

\begin{figure}[ht]
\centering
\includegraphics[width=9cm]{./figures/704f87dae0252e2f.pdf}
\caption{两个 $X$ 子空间的线性映射: $X_0$ 是零空间, $X_1$ 是 $X_0$ 在 $X$ 中的补空间} \label{fig_MatLS2_1}
\end{figure}
\addTODO{“一一对应”放到箭头上面,}

\begin{theorem}{}\label{the_MatLS2_1}
令 $A:X \to Y$ 的零空间为 $X_0$, $X_1$ 为 $X_0$ 在 $X$ 中任意一个补空间(即 $X_0\oplus X_1 = X$, \autoref{def_DirSum_1}~\upref{DirSum}), 令值空间 $Y_1 = A(X)$, 那么映射 $A:X_1\to Y_1$ 是一一对应的(即双射)。
\end{theorem}
证明见下文。 我们可以形象地把\autoref{the_MatLS2_1} 用\autoref{fig_MatLS2_1} 表示, 图中每个三角形代表一个矢量空间, 由于 $X_0, X_1$ 是互补的, 它们只相交于零矢量。 注意对于给定的映射 $A$, $X_0$ 和 $Y_1$ 是确定的, 而 $X_0$ 的补空间 $X_1$ 可以随意选取。

特殊地, 若 $X_0$ 是零维空间即 $X_0 = \qty{0}$, 则补空间 $X_1 = X$。 此时可以把\autoref{fig_MatLS2_1} 中表示 $X_0$ 的三角形去掉。

该定理可以更好地帮助我们理解线性方程组解的结构\upref{LinEq}: 当 $\mat A \bvec x = \bvec b$ 的解集等于一个特解加齐次解集, 齐次解。 $\bvec b$ 是 $Y_1$ 中的某向量, 特解是 $X_1$ 中映射到 $\bvec b$ 的向量, 而齐次解就是零空间 $X_0$。

我们记

\begin{corollary}{秩-零化度定理}\label{cor_MatLS2_1}
线性映射 $f: X \to Y$ 满足
\begin{equation}\label{eq_MatLS2_1}
\dim(X) = \opn{nullity}(f) + \opn{rank}(f)~.
\end{equation}
其中 $\opn{nullity}(f), \opn{rank}(f)$ 分别是 $f$ 的零化度(\autoref{def_LinMap_2}~\upref{LinMap})和秩(\autoref{def_LinMap_3}~\upref{LinMap})。
\end{corollary}
证明: 见\autoref{eq_MatLS2_1}, 由于 $X_1$ 与 $Y_1$ 一一对应, $X_1$ 的维数也是 $\opn{rank}(A)$, 又因为 $X_0\oplus X_1 = X$, $X$ 的维数等于 $X_0, X_1$ 维数之和(\autoref{cor_DirSum_1}~\upref{DirSum})。 证毕。

\begin{corollary}{}
线性映射 $A:X\to Y$ 中值空间 $A(X) \subseteq Y$ 的维数小于或等于定义域空间 $X$ 的维数。
\end{corollary}
证明: \autoref{eq_MatLS2_1} 中 $\dim(X), \opn{nullity}(A), \opn{rank}(A) > 0$, 所以有 $\opn{rank}(A) \leqslant \dim(X)$。

\begin{corollary}{}
线性映射 $A:X\to Y$ 是单射(即 $A:X\to A(X)$ 是双射)当且仅当零空间 $X_0$ 中只有零向量。
\end{corollary}
说明:零空间 $X_0$ 只有零矢量一个元素当且仅当\autoref{eq_MatLS2_1} 中 $\dim(X) = \opn{rank}(A)$ 或 $\opn{nullity}(A) = 0$。

证明: 前者证明后者: 使用\autoref{cor_LinMap_1}~\upref{LinMap} 以及单射的定义, 可得 $0\in X$ 是零空间中的唯一向量。 后者证明前者: $\opn{nullity}(A) = 0$ 即\autoref{the_MatLS2_1} 中的 $X = X_1$, 与值空间 $Y_1 = A(X)$ 有一一对应关系。 证毕。

\subsection{证明}
要证明\autoref{the_MatLS2_1}, 首先证明 $A(X_1) = A(X)$。 任意 $x\in X$ 可以表示为 $x = x_0 + x_1$, 其中 $x_0\in X_0$, $x_1\in X_1$。 $Ax = A x_0 + A x_1 = A x_1$。 这说明任意 $y_1 \in Y_1 = A(X)$ 都能找到 $x_1$ 使 $A x_1 = y_1$, 所以 $A(X_1) = Y_1$。

现在证明 $X_1$ 和 $Y_1$ 一一对应: 令 $u, v \in X_1$, 我们要证明如果 $Au = Av$ 那么 $u = v$。 算符 $A$ 是线性的, 所以 $A(u-v) = 0$, 所以 $u - v \in X_0$。 由封闭性, $u - v \in X_1$。 由于补空间只交于零矢量 $X_0 \cap X_1 = \qty{0}$, 所以 $u - v = 0$, 即 $u = v$。 证毕。

\addTODO{线性方程组词条\upref{LinEq}中说明: 线性方程组的解空间就是特解加上齐次解 $X_0$。}
