% Doxygen 笔记
% license Xiao
% type Note

\begin{issues}
\issueDraft
\end{issues}

\begin{itemize}
\item \href{https://www.doxygen.nl/}{官方网站}。
\item \verb`@brief`: Provides a brief description of the function, class, or variable (including data member).
\item \verb`@class`: Documents a class.
\item \verb`@param`: Describes a function parameter.
\item \verb`@return`: Describes the return value of a function.
\item \verb`@details`: Provides a detailed description (optional).
\item \verb`@note`: Adds a note or additional information.
\item \verb`@warning`: Adds a warning message.
\item \verb`@see`: References other related functions, classes, or documentation.
\item \verb`@code` and \verb`@endcode`: Inserts a code block in the documentation.
\item \verb`@enum`: Documents an enumeration.
\item \verb`@file`: Documents a file.
\item \verb`@namespace`: Documents a namespace.
\end{itemize}

例子
\begin{lstlisting}[language=cpp,caption=factorial.hpp]
/**
 * @brief Calculates the factorial of a number.
 *
 * @details This function calculates the factorial of a non-negative integer
 * using a recursive approach.
 *
 * @note The function does not handle negative inputs.
 *
 * @param n The number to calculate the factorial for.
 * @return The factorial of the number.
 *
 * @code
 * int result = factorial(5); // result will be 120
 * @endcode
 *
 * @see https://en.wikipedia.org/wiki/Factorial
 */
inline int factorial(int n);

inline int factorial(int n) {
    if (n <= 1) return 1;
    return n * factorial(n - 1);
}
\end{lstlisting}

\subsection{安装使用}
\begin{itemize}
\item Ubuntu 可以直接 \verb`sudo apt install doxygen`。
\item 在源码目录下, 用 \verb`doxygen -g` 生成配置文件 \verb`Doxyfile`。 It contains many configuration options. You can edit it to customize the documentation generation process. Here are some common settings you might want to change:
\item \verb`PROJECT_NAME`: The name of your project.
\item \verb`PROJECT_BRIEF`: A brief description of your project.
\item \verb`OUTPUT_DIRECTORY`: The directory where the documentation will be generated.
\item 配置完成后,用 \verb`doxygen Doxyfile` 即可生成文档。 但上面 \verb`factorial.cpp` 中的说明不会生成,因为还需要手动在 Doxyfile 中设置:
\item \verb`EXTRACT_ALL = YES`: Set this to YES to ensure all entities (including undocumented ones) are included in the documentation.
\item \verb`EXTRACT_STATIC = YES`: Set this to YES to document static functions and variables.
\item \verb`RECURSIVE = YES`: If your code is in subdirectories, set this to YES to scan recursively.
\item 再用 \verb`doxygen Doxyfile` 生成就有了!
\end{itemize}
