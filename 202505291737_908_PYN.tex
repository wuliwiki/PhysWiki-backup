% 皮亚诺公理(综述)
% license CCBYSA3
% type Wiki

本文根据 CC-BY-SA 协议转载翻译自维基百科 \href{https://en.wikipedia.org/wiki/Peano_axioms}{相关文章}。

在数学逻辑中,皮亚诺公理(Peano axioms,/piˈɑːnoʊ/,[peˈaːno]),也称为德德金–皮亚诺公理或皮亚诺假设,是一组用于描述自然数的公理体系,由19世纪意大利数学家朱塞佩·皮亚诺提出。这些公理在多项元数学研究中几乎未加改动地被广泛采用,其中包括关于数论是否一致与完备等基本问题的研究。

由皮亚诺公理所提供的算术公理化体系,通常被称为皮亚诺算术。

将算术进行形式化的重要性,在赫尔曼·格拉斯曼的工作之前并未被广泛重视。格拉斯曼在19世纪60年代指出,算术中的许多事实可以从关于后继运算和数学归纳的更基本事实中推导出来。1881年,查尔斯·桑德斯·皮尔士提出了一种自然数算术的公理化体系。1888年,理查德·德德金又提出了另一种自然数的公理体系,而皮亚诺则在1889年出版的著作《以新方法阐述的算术原理》(拉丁文:Arithmetices principia, nova methodo exposita)中,将其加以简化并作为一组公理发表。

皮亚诺公理共包括三类陈述:1. 第一条公理断言自然数集合中至少存在一个元素;2. 接下来的四条是关于等号的一般性陈述,在现代的处理方式中,这些往往被看作是“基础逻辑”的一部分,而非皮亚诺公理本身;3. 再接下来的三条公理是关于自然数及其后继运算的基本性质的一阶逻辑陈述;4. 第九条、也是最后一条公理则是一个二阶逻辑陈述,它表达了对自然数进行数学归纳法的原理,正是这一点使得原始皮亚诺公理体系接近二阶算术。如果显式引入加法和乘法两个运算符号,并将第九条二阶归纳公理替换为一阶公理模式,就可以得到一个较弱的一阶系统。术语“皮亚诺算术”有时特指这一限制后的系统。
\subsection{历史上的二阶表述}
当皮亚诺提出他的公理时,数理逻辑的语言还处于起步阶段。他为了表达这些公理而创造的逻辑符号体系并未广泛流行,尽管它成为现代集合成员关系符号(∈,源自皮亚诺的 ε)的起点。皮亚诺明确区分数学符号和逻辑符号,这在当时的数学中还不常见;这种区分最早由哥特洛布·弗雷格在其 1879 年出版的《概念文字》中引入。[7] 然而,皮亚诺并不知道弗雷格的工作,而是独立地基于布尔和施罗德的研究重建了自己的逻辑体系。[8]

皮亚诺公理定义了自然数的算术性质,通常表示为集合 $N$ 或 $\mathbb{N}$。这些公理中的非逻辑符号包括一个常数符号 0 和一个一元函数符号 S(表示“后继”)。

第一条公理声明常数 0 是一个自然数:

1.0 是一个自然数。

皮亚诺最初在其公理表述中使用 1 而非 0 作为“第一个”自然数,[9] 而他在《数学公式集》中的公理则包括了零。[10]
