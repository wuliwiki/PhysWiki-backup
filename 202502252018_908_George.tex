% 乔治·布尔(综述)
% license CCBYSA3
% type Wiki

本文根据 CC-BY-SA 协议转载翻译自维基百科\href{https://en.wikipedia.org/wiki/George_Boole#}{相关文章}。

\begin{figure}[ht]
\centering
\includegraphics[width=6cm]{./figures/228c41af0dd53eca.png}
\caption{布尔的肖像,摘自《插图伦敦新闻》,1865年1月21日} \label{fig_George_1}
\end{figure}
乔治·布尔(George Boole,发音:/buːl/ 布尔,1815年11月2日-1864年12月8日)是一位主要自学成才的英国数学家、哲学家和逻辑学家,他的大部分短暂职业生涯都在爱尔兰科克的女王学院担任数学教授。他在微分方程和代数逻辑领域有所研究,最著名的作品是《思维的法则》(The Laws of Thought,1854),该书中包含了布尔代数。布尔逻辑对计算机编程至关重要,并被认为为信息时代奠定了基础。

布尔是一个鞋匠的儿子。他接受了初等教育,并通过各种方式学习了拉丁语和现代语言。16岁时,他开始教授工作以养家糊口。19岁时,他创办了自己的学校,后来在林肯经营了一所寄宿学校。布尔积极参与当地社团活动,并与其他数学家合作。1849年,他被任命为爱尔兰科克女王学院(现为科克大学)首任数学教授,在那里他遇见了未来的妻子玛丽·埃弗雷斯特。他继续参与社会事业,并保持与林肯的联系。1864年,布尔因患肺炎引发的胸膜积液而去世。

布尔一生发表了约50篇文章和几本单独的著作。他的一些关键作品包括关于早期不变性理论的论文和《逻辑的数学分析》(The Mathematical Analysis of Logic),该书引入了符号逻辑。布尔还写了两部系统性的专著:《微分方程论》和《有限差分法则论》。他对线性微分方程理论和有理函数的留数和研究作出了贡献。1847年,布尔发展了布尔代数,这一二进制逻辑的基本概念为逻辑代数传统奠定了基础,并构成了数字电路设计和现代计算机科学的基石。布尔还试图在概率论中发现一种通用方法,重点研究如何通过逻辑连接给定概率的事件来确定其后果概率。

布尔的工作得到了许多学者的拓展,如查尔斯·桑德斯·皮尔士和威廉·斯坦利·杰文斯等。布尔的思想后来得到了实际应用,当克劳德·香农和维克托·谢斯塔科夫利用布尔代数优化机电继电器系统的设计时,推动了现代电子数字计算机的发展。他对数学的贡献为他赢得了各种荣誉,包括皇家学会的首个数学金奖、基思奖章以及都柏林大学和牛津大学的名誉学位。科克大学在2015年庆祝布尔诞辰200周年,强调了他对数字时代的重大影响。

\begin{figure}[ht]
\centering
\includegraphics[width=6cm]{./figures/ae46c5bae0377e72.png}
\caption{3号波特门街的房子和学校} \label{fig_George_2}
\end{figure}
布尔于1815年出生在英格兰林肯市,父亲是鞋匠约翰·布尔(John Boole Snr,1779-1848),母亲是玛丽·安·乔伊斯(Mary Ann Joyce)。他接受了初等教育,并从父亲那里获得了教学,但由于家庭生意的严重衰退,他没有接受更多的正式和学术教育。林肯的书商威廉·布鲁克(William Brooke)可能帮助他学习拉丁语,他也可能在托马斯·贝恩布里奇(Thomas Bainbridge)学校学习过拉丁语。他在现代语言方面是自学成才的。事实上,当当地报纸刊登了他翻译的拉丁诗歌时,一位学者指控他抄袭,声称他不可能取得这样的成就。16岁时,布尔成为家庭的经济支柱,负责养活父母和三个年幼的弟妹,并在唐卡斯特的海厄姆学校担任初级教师。他还曾在利物浦短暂任教。