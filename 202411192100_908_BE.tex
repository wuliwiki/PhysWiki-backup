% 马克斯·玻恩(综述)
% license CCBYSA3
% type Wiki

本文根据 CC-BY-SA 协议转载翻译自维基百科\href{https://en.wikipedia.org/wiki/Max_Born}{相关文章}。

\textbf{马克斯·玻恩}(Max Born,FRS FRSE)(德语发音:[ˈmaks ˈbɔʁn];1882年12月11日-1970年1月5日)是一位德裔英国物理学家和数学家,对量子力学的发展起到了关键作用。他还在固态物理学和光学领域做出了重要贡献,并在20世纪20年代和30年代指导了许多著名物理学家的研究工作。玻恩因其“对量子力学的基础研究,特别是对波函数统计解释的贡献”于1954年获得了诺贝尔物理学奖。[1]

玻恩于1904年进入哥廷根大学,在那里他结识了三位著名的数学家:费利克斯·克莱因(Felix Klein)、大卫·希尔伯特(David Hilbert)和赫尔曼·闵可夫斯基(Hermann Minkowski)。他以弹性丝和带的稳定性为主题撰写了博士论文,并因此获得了该校哲学系奖项。1905年,他开始与闵可夫斯基研究狭义相对论,随后完成了以汤姆森原子模型为主题的资格论文(habilitationsschrift)。1918年,他在柏林偶然遇到弗里茨·哈伯(Fritz Haber),讨论了一种金属与卤素反应生成离子化合物的过程,这一过程今天被称为\textbf{玻恩–哈伯循环}(Born–Haber cycle)。