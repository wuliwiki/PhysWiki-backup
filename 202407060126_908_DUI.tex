% 堆(综述)
% license CCBYSA3
% type Wiki

(本文根据 CC-BY-SA 协议转载自原搜狗科学百科对英文维基百科的翻译)

在计算机科学中,堆是一种特殊的基于树的数据结构,本质上是一棵几乎完全的[1]树, 它满足堆属性:在最大堆中,对于任意给定的节点C,如果P是C的父节点,那么P的键(值)大于或等于C的键。在最小堆中,P的键小于或等于C的键。堆“顶部”的节点(没有父节点)称为根节点。

堆是称为优先队列的抽象数据类型的最有效实现,事实上优先队列通常被称为“堆”,不管它们是如何实现的。在堆中,最高(或最低)优先级的元素总是存储在根节点。然而,堆不是排序结构,它可以被认为是部分有序的。当需要重复移除优先级最高(或最低)的对象时,堆是一种有用的数据结构。

堆的一个常见实现是二叉堆,其中树是二叉树(见图)。堆数据结构,特别是二叉堆,是由J. W. J. Williams在1964年引入的,作为堆排序算法的数据结构。[2] 堆在一些高效的图算法中也很重要,比如Dijkstra算法。当堆是一棵完全二叉树时,它有一个最小的可能高度——一个有N个节点的堆,对于每个节点,一个分支总是有loga N 的高度。

注意,如图所示,兄弟和堂兄弟之间没有隐含的排序,也没有有序遍历的隐含序列(例如,在二叉搜索树中)。上面提到的堆关系仅适用于节点和它们的父亲、祖父等之间。每个节点可以拥有的最大子节点数取决于堆的类型。

\subsection{操作}
涉及堆的常见操作有:

基础
\begin{itemize}
\item 查找最大值(或查找最小值):分别查找最大堆的最大项或最小堆的最小项(也称为peek)
插入:向堆中添加一个新键(也可称push[3])
\item 提取最大值(或提取最小值):从堆中移除节点后,返回最大堆的最大值[或最小堆的最小值](也称为pop)[4])
\item 删除-最大(或删除-最小):分别删除最大堆(或最小堆)的根节点
\item 替换:弹出根节点并压入新键。比先进行pop,后进行push更有效,因为只需要平衡一次,而不是两次,并且适合固定大小的堆。[5]
\end{itemize}

创建
\begin{itemize}
\item 创建堆:创建一个空堆
\item 堆化:用给定的元素数组创建一个堆
\item 合并(merge、union):连接两个堆,形成一个包含两个堆的所有元素的有效新堆,保留原始堆。
\item 合并(meld):连接两个堆,形成一个包含两个堆的所有元素的有效新堆,销毁原始堆。
\end{itemize}

检查
\begin{itemize}
\item 大小(size):返回堆中的项数。
\item 判空(is-empty):如果堆为空,则返回true,否则返回false。
\end{itemize}

内部的
\begin{itemize}
\item 增加键或减少键:分别更新最大堆或最小堆中的键
\item 删除:删除任意节点(随后移动最后一个节点并筛选以维护堆)
\item 向上筛选:根据需要在树中向上移动一个节点;用于在插入后恢复堆状态。之所以被称为“筛选”,是因为节点在树中向上移动,直到达到正确的水平,就像在筛子中一样。
\item 向下筛选:在树中向下移动一个节点,类似于向上筛选;用于在删除或替换后恢复堆状态。
\end{itemize}

\subsection{实现}
\begin{figure}[ht]
\centering
\includegraphics[width=10cm]{./figures/1e47b4130d0c5882.png}
\caption{一个完全二叉最大堆的例子,节点是从1到100之间的整数。图示了它将如何存储在数组中。} \label{fig_DUI_1}
\end{figure}
堆通常在数组(固定大小或动态数组)中实现,并且不需要元素之间的指针。元素插入堆或从堆中删除后,可能会违反堆性质,必须通过内部操作来平衡堆。

二叉堆可以用单独使用数组的非常节省空间的方式来表示。第一个(或最后一个)元素将包含根元素。数组接下来两个元素包含根元素的孩子。接下来的四个包含两个子节点的四个子节点,等等。因此,在位置n的节点的子节点将在以1为起点编号数组的2n和2n+1位置,或者“以0为起点编号数组的2n+1和2n+2位置。这允许通过简单的索引计算来向上或向下移动树。允许通过向上筛选或向下筛选操作(交换无序元素)来完成平衡堆。因为我们可以从一个数组构建一个堆,而不需要额外的内存(例如,对于节点),所以heapsort可以用来就地排序一个数组。

不同类型的堆以不同的方式实现操作,但值得注意的是,插入通常是通过在堆的末尾第一个可用空间中添加新元素来完成的。这通常会违反堆性质,因此元素会被向上筛选,直到堆性质被重新建立。类似地,删除根元素是通过移动根元素,然后将最后一个元素放入根并向下筛选以重新平衡。因此,替换是通过删除根并将新元素放入根并向下筛选来完成的,和先pop(向下筛选最后一个元素),再push(向上筛选新元素)后相比,避免了向上筛选步骤。

从给定的元素数组中构造二叉(或d叉)堆可以使用经典弗洛伊德算法在线性时间内执行,最差情况下的比较次数等于2N − 2s2(N) − e2(N) (对于二叉堆),其中 s2(N)是N的二进制表示的所有数字的总和,e2(N)是N的素数因子分解中的2的指数,[6] 这比连续插入原始空堆的序列更快,该操作是对数线性的。

\subsection{变体}
\begin{itemize}
\item 2–3堆
\item b堆
\item Beap
\item 二进制堆
\item 二项式堆
\item Brodal队列
\item d-ary堆
\item 斐波那契堆
\item 树叶堆
\item 左翼堆
\item 配对堆
\item 基数堆
\item 随机可融合堆
\item 倾斜堆
\item 软堆
\item 三元堆
\item Treap
\item 弱势群体
\end{itemize}

\subsection{变体理论界限的比较}
以下是各种堆数据结构的时间复杂度。 函数名假设一个最小堆。

In the following time complexities[7] O(f) is an asymptotic upper bound and Θ(f) is an asymptotically tight bound (see Big O notation). Function names assume a min-heap.
\begin{table}[ht]
\centering
\caption{辐射的类型}\label{AYST}
\begin{tabular}{|c|c|c|c|c|c|c|c|c|c}
\hline
\textbf{Operation} & \textbf{Binary[7]} & \textbf{Leftist} & \textbf{Binomial[7]} & \textbf{Fibonacci[7][8]} & \textbf{} & \textbf{} & \textbf{} & \textbf{} & \textbf{}\\