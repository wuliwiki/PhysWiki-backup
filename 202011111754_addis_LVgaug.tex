% 长度规范和速度规范
% 长度规范|速度规范|波函数|规范变换|薛定谔方程|麦克斯韦方程组

\begin{issues}
\issueDraft
\end{issues}

\pentry{电磁场中的单粒子薛定谔方程\upref{QMEM}, 偶极子近似} % 未完成

\subsection{速度规范}
当空间中存在静止的电荷分布时, 我们可以把标量势能分为 $V + \varphi$ 两部分. 前者由静止电荷根据库伦定律计算, 不参与规范变换, 在这里我们甚至可以不把它看成电磁力而只是某种一般的势能. 后者随时间变化, 但库仑规范下 $\varphi = 0$. 所以库伦规范下, 电磁场中带电粒子的哈密顿量为(\autoref{QMEM_eq4}~\upref{QMEM})
\begin{equation}
\begin{aligned}
H &= -\frac{1}{2m} \laplacian + \I \frac{q}{m} \bvec A \vdot \Nabla + \frac{q^2}{2m} \bvec A^2 + q\varphi\\
&= H_0 - \frac{q}{m} \bvec A \vdot \bvec p + \frac{q^2}{2m} \bvec A^2
\end{aligned}
\end{equation}
其中定义不含时哈密顿算符为
\begin{equation}
H_0 = \frac{\bvec p^2}{2m} + qV
\end{equation}
当我们用偶极子近似时, $\bvec A(t)$ 与位置无关而只是时间的函数. 我们可以利用这个性质方便地得到另外另种规范. 注意他们只有在偶极子近似时才成立.

规范变换为\autoref{QMEM_eq5}~\upref{QMEM}
\begin{equation}\label{LVgaug_eq1}
\bvec A = \bvec A' + \grad \chi
\qquad
\varphi = \varphi' - \pdv{\chi}{t}
\end{equation}
对库仑规范使用规范变换
\begin{equation}\label{LVgaug_eq3}
\Psi(\bvec r, t) = \exp(\I q\chi)\Psi^V(\bvec r, t)
\end{equation}
\begin{equation}\label{LVgaug_eq4}
\chi(t) = \frac{q}{2m} \int^t \bvec A^2(t') \dd{t'}
\end{equation}
注意\autoref{LVgaug_eq1} 中 $\grad \chi = \bvec 0$ 所以 $\bvec A' = \bvec A$. 利用 $\varphi$ 的变换就可以将 $\bvec A^2$ 项消去得\footnote{该式中省略了 $\bvec A', \varphi'$ 的瞥}
\begin{equation}
\I \pdv{t} \Psi^V = \qty(H_0 - \frac{q}{m} \bvec A \vdot \bvec p) \Psi^V
\end{equation}
这种规范叫做\textbf{速度规范(velocity gauge)}.

\subsection{长度规范}
要推导长度规范, 我们可以直接使用与规范无关的哈密顿量(\autoref{QMEM_eq2}~\upref{QMEM}). 其中我们同样把 $V_0$ 排除在 $\varphi$ 之外:
\begin{equation}
H = H_0 - \frac{q}{2m} (\bvec A \vdot \bvec p + \bvec p \vdot \bvec A)
+ \frac{q^2}{2m} \bvec A^2 + q \varphi
\end{equation}
相对于库仑规范的 $\Psi$, 我们令
\begin{equation}
\Psi(\bvec r, t) = \exp(\I q\chi)\Psi^L(\bvec r, t)
\end{equation}
\begin{equation}
\chi(\bvec r, t) = \bvec A(t) \vdot \bvec r
\end{equation}
利用 $-\pdv*{\bvec A}{t} = \bvec {\mathcal E}$(库伦规范+偶极子近似,引用未完成) $\grad \chi = \bvec A(t)$ 所以 $\bvec A' = \bvec 0$; 而另一方面 $\varphi' = \pdv*{\chi}{t} = -\bvec {\mathcal E} \vdot \bvec r$. 变换后薛定谔方程为
\begin{equation}
\I \pdv{t} \Psi^L = \qty(H_0 - q\bvec{\mathcal{E}} \vdot \bvec r) \Psi^L
\end{equation}
这种规范叫做\textbf{长度规范(length gauge)}.
