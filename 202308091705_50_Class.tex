% 分类
% keys 分类 分类器
% license Xiao
% type Wiki

\textbf{分类}(Classfication)是一种把实例或者对象划归到特定的类别中的操作。在机器学习中,分类指的是从给定特征来预测离散型输出值的学习任务。预测输出值就称为类别。与之对应的,预测连续型变量的操作是\textbf{回归}。

机器学习中的分类是通过建立一个从输入空间到输出空间的映射来实现的。该映射被称为分类模型或者\textbf{分类器}(Classifier)。

最常见分类任务的二分类(Binary classification),即类别只有两类的分类任务。通常,称两类中的一类为正类(Positive class),另一类为反类或负类(Negative class)。如果,一个分类任务的待预测类别数是三个或者三个以上的话,就称为多分类(Muti-class classification)任务。




\textbf{参考文献:}
\begin{enumerate}
\item 周志华. 机器学习[M]. 清华大学出版社. 2016
\end{enumerate}