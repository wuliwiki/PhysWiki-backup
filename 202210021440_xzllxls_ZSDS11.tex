% 浙江理工大学 2011 年数据结构
% 2011年浙江理工大学991数据结构考研真题

\subsection{一、单选题}
在每小题的四个备选答案中选出一个正确答案,每小题3分,共45分.

1. 若线性表最常用的操作是存取第$i$个元素及其前趋的值,则采用(  )存储方式节省时间. \\
A.单链表 $\qquad$ B.双链表 $\qquad$ C.单循环链表 $\qquad$ D.顺序表

2. 设输入序列为$1$、$2$、$3$、$4$,则借助栈所得到的输出序例不可能是(  ) \\
A.1、2、3、4 \\
B.4、1、2、3 \\
C.1、3、4、2 \\
D.4、3、2、1

3. 常对数组进行的两种基本操作是(  ). \\
A.建立与删除 \\
B.插入与修改 \\
C.查找与修改 \\
D.查找与插入

4. 数组$Q[n]$用来表示个循环队列,$f$为当前队列头元素的前一位置, $r$为队尾元素的位置,假定队列中元素的个数小于$n$ ,计算队列中元素的公式为(  ) \\
A. r-f \\
B. (n+f-r)\%n \\
c. n+r-f \\
D. (n+r-f)\%n

