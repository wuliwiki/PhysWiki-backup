% 磁化强度

为了表征磁介质磁化的程度,与讨论电介质时定义极化强度一样引进一个物理量,叫做\textbf{磁化强度(magnetization intensity)}. 在被磁化后的磁介质内,任取一体积元$\Delta V$. 在这体积元中所有分子固有磁矩的矢量和$\sum \mathbf{m}_{mole}$加上附加磁矩的矢量和$\sum \Delta\mathbf{m}_{mole}$与该体积元的比值,即单位体积内分子磁矩的矢量和,称为\textbf{磁化强度},用$\mathbf M$表示.即
\begin{equation}
\mathbf M=\frac{\sum \mathbf m_{mole}+\sum \Delta \mathbf m_{mole}}{\Delta V}
\end{equation}
对于顺磁质,$\sum \Delta\mathbf{m}_{mole}$可以忽略;对于抗磁质,$\sum \mathbf{m}_{mole}=0$;对于真空,$\mathbf M=0$.如果在介质中各点的$\mathbf M $相同,就称磁介质被均匀磁化.在国际单位制中,$\mathbf M$的单位是$\rm A/m$.

顺磁质经磁化后, M 的方向与该处的磁场B 一致,它在磁介质内所激发的附加磁场B' 的方向也与凡的方向相同抗磁质经磁化后, M 的方向与该处磁场B 相反,它在磁介质内所激发的附加磁场B' 的方向也与凡的方向相反磁介质的磁化悄况,可以用磁化强度M 来描述,也可以用磁化电流来反映.磁化电流实质上是分子电流的宏观表现,它与磁化强度M 之间必然存在一定的联系.下面,我们将用直观的方法找出能测定的宏观的磁化强度与磁化电流之间的关系.

为简单计,我们选一特例来讨论设有一"尤限长”的载流直螺线管,管内充满均匀磁介质,电流在螺线管内激发均匀磁场在此磁场中磁介质被均匀磁化,这时磁介质中各个分子电流平面将转到与磁场的方向相垂直,图8-50 表示磁介质内任一截而上分子电流排列的情况.从图8-50(c) 和(d) 中可以乔出,在磁介质内部任意一点处,总是有两个方向相反的分子电流通过,结果相且抵消;只有在截面边缘处,分子屯流未被抵悄,形成与截面边缘重合的圆屯流对磁介质的整体来说,未被抵消的分子电流是沿豹柱面流动的,称为安培表面电流(或叫磁化面电流)对顺磁性物质,安培表面电流和螺线管上异线中
的电流l 方向相同;对抗磁性物质,则两者方向相反.图8-50 所示的是顺磁质的情况设q 为圆柱形磁介质表面上单位长度的磁化面电流,即磁化面屯流的线密度, S 为磁介质的截面积, l 为所选取的一段磁介质的长度在l 长度[,表面电
流的总从伯为l 、=a.I, 因此在这段磁介质总体积SI 中的总磁矩为