% 勒内·笛卡尔(综述)
% license CCBYSA3
% type Wiki

本文根据 CC-BY-SA 协议转载翻译自维基百科\href{https://en.wikipedia.org/wiki/Ren\%C3\%A9_Descartes}{相关文章}。

勒内·笛卡尔(/deɪˈkɑːrt/ day-KART 或英国发音:/ˈdeɪkɑːrt/ DAY-kart;法语:[ʁəne dekaʁt] ⓘ;[注3][11] 1596年3月31日 – 1650年2月11日)[12][13]: 58  是法国哲学家、科学家和数学家,被广泛认为是现代哲学和科学兴起的奠基人物之一。数学在他的研究方法中至关重要,他将几何与代数相结合,创立了解析几何。笛卡尔的职业生涯大部分时间是在荷兰共和国度过的,最初在荷兰国军服役,后来成为荷兰黄金时代的核心知识分子。[14] 尽管他服务于一个新教国家,且后来被批评者视为自然神论者,笛卡尔实际上是罗马天主教徒。[15][16]

笛卡尔哲学的许多元素可以在晚期的亚里士多德主义、16世纪复兴的斯多葛主义或更早的哲学家如奥古斯丁的思想中找到前例。在他的自然哲学中,他在两个主要方面不同于当时的学派。首先,他拒绝将有形实质划分为质料和形式;其次,他拒绝在解释自然现象时诉诸于神或自然的终极目的。[17] 在神学中,他坚持神创造行为的绝对自由。笛卡尔拒绝接受前人哲学家的权威,常常将自己的观点与之前的哲学家区分开来。在《灵魂的激情》开篇中,这部早期现代情感论著中,笛卡尔甚至声称他将“仿佛从未有人写过这些问题一样”来论述该主题。他最著名的哲学陈述是“我思故我在”(拉丁语:cogito, ergo sum;法语:Je pense, donc je suis),出现在《方法谈》(1637年,以法语和拉丁语写成,1644年)和《哲学原理》(1644年拉丁语版,1647年法语版)中。[注4] 这一陈述要么被解释为逻辑三段论,要么被视为一种直觉思想。[18]

笛卡尔常被称为现代哲学之父,被广泛认为是17世纪对认识论关注增加的主要推动者。[19][注5] 他为17世纪大陆理性主义奠定了基础,后由斯宾诺莎和莱布尼茨提倡,但后来受到霍布斯、洛克、贝克莱和休谟组成的经验主义学派的反对。早期现代理性主义的兴起——作为历史上首次独立的系统性哲学学派——对现代西方思想产生了广泛影响,诞生了笛卡尔(笛卡尔主义)和斯宾诺莎(斯宾诺莎主义)两个理性主义哲学体系。正是17世纪的理性主义大师如笛卡尔、斯宾诺莎和莱布尼茨赋予了“理性时代”其名称和历史地位。莱布尼茨、斯宾诺莎[20] 和笛卡尔都在数学和哲学方面造诣颇深,笛卡尔和莱布尼茨还在多个科学领域有所贡献。[21] 尽管只有莱布尼茨被广泛认可为博学家,这三位理性主义者都在各自的著作中整合了不同的知识领域。[22]

笛卡尔的《第一哲学沉思》(1641年)至今仍是大多数大学哲学系的标准教材。笛卡尔在数学方面的影响同样显著,以他的名字命名了笛卡尔坐标系。他被誉为解析几何之父,这一数学分支后来用于微积分和数学分析的发现。笛卡尔也是科学革命的关键人物之一。
\subsection{生平}
\subsubsection{早年生活}
\begin{figure}[ht]
\centering
\includegraphics[width=6cm]{./figures/266c186e2fde6001.png}
\caption{笛卡尔家族的族徽。} \label{fig_DKE_1}
\end{figure}
笛卡尔出生的房子位于图赖讷的拉艾

勒内·笛卡尔于1596年3月31日出生在法国图赖讷省的拉艾(现今法国安德尔-卢瓦尔省的笛卡尔镇)。[23] 1597年5月,他的母亲让娜·布罗沙在生下一个死胎后几天去世。[24][23] 笛卡尔的父亲乔阿希姆是雷恩议会的成员。[25]: 22 勒内与祖母和叔祖一起生活。尽管笛卡尔一家是罗马天主教徒,但普瓦图地区当时由新教徒胡格诺派控制。[26] 1607年,由于体弱多病,笛卡尔晚入学,进入位于拉弗莱什的耶稣会皇家亨利-勒格朗学院。[27][28] 在那里,他接触了数学和物理学,包括伽利略的作品。[29][30] 在此期间,笛卡尔首次接触到赫尔墨斯神秘主义。1614年毕业后,他在普瓦捷大学学习了两年(1615-1616年),于1616年获得教会法和民法的学士和执照学位,[29] 这符合其父亲希望他成为律师的愿望。[31] 随后,他前往巴黎。

笛卡尔在普瓦捷大学的毕业注册记录,1616年

在《方法谈》中,笛卡尔回忆道:[32]: 20–21

我完全放弃了对书本的学习,决定不再追求任何除了可以在自己内心或伟大的‘世界之书’中找到的知识。我将余下的青春岁月用于旅行,拜访宫廷和军队,与不同性格和地位的人交往,积累各种经验,在命运带给我的各种情境中考验自己,并始终反思遇到的一切,以从中获得一些收获。
\begin{figure}[ht]
\centering
\includegraphics[width=6cm]{./figures/4849a80f640afc83.png}
\caption{笛卡尔出生于图赖讷的拉艾的房子。} \label{fig_DKE_2}
\end{figure}
\begin{figure}[ht]
\centering
\includegraphics[width=6cm]{./figures/02e60f46db774513.png}
\caption{笛卡尔在普瓦捷大学的毕业登记记录,1616年} \label{fig_DKE_3}
\end{figure}
\subsubsection{军队服役}
出于成为职业军官的志向,1618年笛卡尔以雇佣兵的身份加入了新教的荷兰国军,在拿骚的毛里茨指挥下于布雷达服役,[29] 并正式学习由西蒙·斯蒂文确立的军事工程学。[33] 因此,笛卡尔在布雷达得到了许多鼓励,来提升他的数学知识。[29] 通过这种方式,他结识了多德雷赫特学校的校长艾萨克·贝克曼,[29] 并为他写了《音乐概要》(1618年撰写,1650年出版)。[34]

自1619年起,笛卡尔为天主教的巴伐利亚公爵马克西米利安效力,[35] 并于1620年11月在布拉格附近参加了白山之战。[36][37]

根据阿德里安·巴耶的记述,1619年11月10日至11日夜间(圣马丁节),笛卡尔驻扎在多瑙河畔新堡,将自己锁在一个带有“火炉”的房间里(可能是一个陶瓷炉)[38] 以避寒。在那里,他做了三个梦,[39] 并认为一个神圣的灵感向他揭示了一种新的哲学。然而,有人推测笛卡尔所谓的第二个梦实际上是一次“爆头综合症”的发作。[40] 醒来后,他已经构思出了解析几何以及将数学方法应用于哲学的思想。他从这些异象中得出结论:科学的追求对他而言是通向真正智慧的道路,也是他一生工作的重要部分。[41][42] 笛卡尔还清楚地意识到所有真理彼此关联,因此找到一个基本真理并用逻辑推理就可以打开通往所有科学的大门。他很快就达到了这一基本真理:他著名的“我思故我在”。[43]
\subsubsection{职业生涯}

\textbf{法国}

1620年,笛卡尔离开了军队。他参观了洛雷托的圣殿圣母堂,然后游历了多个国家,之后回到法国,并在接下来的几年里在巴黎度过了一段时间。在巴黎,他撰写了第一篇关于方法的论文:《指导心智的规则》(Regulae ad Directionem Ingenii)。[43] 1623年,他抵达拉艾,将自己的全部财产出售,投资于债券,这为他余生提供了舒适的收入。[44][45]: 94 笛卡尔以观察者的身份参加了1627年黎塞留枢机围攻拉罗谢尔的行动。[45]: 128 在那里,他对黎塞留修建的大堤的物理特性产生了兴趣,并从数学角度研究了围攻期间所见的一切。他还结识了法国数学家吉拉尔·德萨尔格。[46] 同年秋天,他与梅森及其他学者一起前往教廷大使圭迪·迪·巴尼奥的住所,聆听炼金术士尼古拉·德·维利耶(西尔·德·尚杜)关于一种新哲学原则的讲座,[47] 枢机主教贝吕尔建议他在宗教裁判所势力范围之外的某地撰写他的新哲学的论述。[48]

\textbf{荷兰}
\begin{figure}[ht]
\centering
\includegraphics[width=6cm]{./figures/e7a65f692f782526.png}
\caption{在阿姆斯特丹,笛卡尔住在西市场6号(左侧的笛卡尔之家)。} \label{fig_DKE_4}
\end{figure}
笛卡尔于1628年返回荷兰共和国。[39] 1629年4月,他进入弗拉内克尔大学,师从阿德里安·梅蒂乌斯,住在一个天主教家庭或租住在沙尔德马城堡。次年,他以“普瓦图人”之名入读莱顿大学,这是一所新教大学。[49] 他跟随雅各布斯·戈利乌斯学习数学,后者向他介绍了帕普斯的六边形定理,并跟随马丁·霍腾修斯学习天文学。[50] 1630年10月,他与贝克曼发生冲突,指责对方剽窃了他的一些思想。在阿姆斯特丹,笛卡尔与一名女仆海伦娜·扬斯·范德斯特罗姆有过一段关系,并在1635年于代芬特尔生下了女儿弗朗辛。她接受了新教洗礼,[51][52] 但5岁时因猩红热去世。

与当时的许多道德家不同,笛卡尔并不贬低情感,反而为其辩护;[53] 在1640年弗朗辛去世时,他泪流满面。[54] 根据杰森·波特菲尔德最近的一部传记,“笛卡尔认为,成为男子汉并不意味着要抑制泪水。”[55] 拉塞尔·肖托推测,父亲身份的体验以及失去孩子的经历是笛卡尔工作中的一个转折点,使其研究重点从医学转向对普遍答案的追寻。[56]

尽管频繁搬家,[注6] 笛卡尔在荷兰的二十多年间完成了所有主要著作,开创了数学和哲学的革命。[注7] 1633年,伽利略被意大利宗教裁判所谴责,笛卡尔因此放弃了出版《世界论》的计划,这是他前四年的研究成果。然而,在1637年,他将该作品的一部分发表为三篇论文:[57]《气象》(Les Météores)、《屈光学》(La Dioptrique)和《几何学》(La Géométrie),并以他著名的《方法谈》(Discours de la méthode)作为引言。[57] 在书中,笛卡尔提出了四条思维规则,以确保我们的知识建立在坚实的基础上:[58]

首先,永远不要接受任何我不确定为真的事物;也就是说,要小心避免仓促和偏见,并且我的判断中不应包含任何未在我心中清晰而明确呈现且排除一切疑点的事物。

在《几何学》中,笛卡尔运用了他与皮埃尔·费马共同发现的成果。这后来被称为笛卡尔几何。[59]
\begin{figure}[ht]
\centering
\includegraphics[width=6cm]{./figures/36f19911beea68e2.png}
\caption{《哲学原理》(Principia philosophiae)标题页,1656年} \label{fig_DKE_5}
\end{figure}

笛卡尔在余生中继续发表有关数学和哲学的著作。1641年,他出版了一部形而上学论文《第一哲学沉思》(Meditationes de Prima Philosophia),用拉丁文写成,主要面向学术界。1644年,他出版了《哲学原理》(Principia Philosophiae),这是一种对《方法谈》和《第一哲学沉思》的综合。1643年,乌得勒支大学谴责了笛卡尔哲学,笛卡尔被迫逃往海牙,定居于埃蒙德-比宁。

1643年至1649年间,笛卡尔与他的女友住在埃蒙德-比宁的一家旅馆中。[60] 笛卡尔与卑尔根的主人安东尼·斯图德勒·范祖克交好,并参与了他的宅邸和庄园的设计。[61][62][63] 他还结识了数学家兼测量员德克·雷姆布兰茨·范尼罗普。[64] 他对范尼罗普的知识印象深刻,甚至向康斯坦丁·惠更斯和弗朗斯·范斯库滕推荐了他。[65]

克里斯蒂亚·默瑟认为,笛卡尔可能受到了西班牙作家兼天主教修女阿维拉的特蕾莎的影响,后者五十年前出版了《心灵城堡》,探讨了哲学反思在智力成长中的作用。[66][67]

笛卡尔通过荷兰军队中的意大利将军阿方索·波洛蒂与波希米亚公主伊丽莎白展开了长达六年的书信往来,主要涉及道德和心理学主题。[68] 与这段书信往来相关,1649年他出版了《心灵激情》(Les Passions de l'âme),并将其献给公主。由克洛德·皮科特神父翻译的《哲学原理》法语版于1647年出版,同样献给伊丽莎白公主。在法语版的前言中,笛卡尔称赞了通过哲学来获得智慧的途径。他指出了获得智慧的四个通常来源,并最终指出还有第五个更好且更安全的来源,即对第一原因的探寻。[69]

\textbf{瑞典}
\begin{figure}[ht]
\centering
\includegraphics[width=8cm]{./figures/f7b5996a40a40b4c.png}
\caption{笛卡尔与瑞典女王克里斯蒂娜在斯德哥尔摩的谈话} \label{fig_DKE_6}
\end{figure}
到1649年,笛卡尔已成为欧洲最著名的哲学家和科学家之一。[57] 当年,瑞典女王克里斯蒂娜邀请他到宫廷,协助创办一所新的科学学院,并教授她有关爱情的思想。[70] 笛卡尔接受了邀请,在冬季前往瑞典帝国。[71] 克里斯蒂娜对《心灵激情》很感兴趣,并激励笛卡尔将其出版。[72]

他居住在皮埃尔·沙努家中,住址位于斯德哥尔摩的Västerlånggatan,距离三皇冠城堡不到500米。在那里,沙努与笛卡尔使用托里拆利水银气压计进行观测。[70] 笛卡尔还向布莱兹·帕斯卡发起挑战,首次在斯德哥尔摩进行气压读数实验,以观察大气压力是否可以用于天气预报。[73]
\subsubsection{逝世}
笛卡尔安排在女王生日后每周三次在清晨5点给克里斯蒂娜女王授课,地点在她寒冷且通风的城堡中。然而,到了1650年1月15日,女王实际上只与笛卡尔会面了四五次。[70] 很快,双方明显地表现出不喜欢对方;她不感兴趣他的机械哲学,而他也对她对古希腊语言和文学的兴趣不以为然。[70] 1650年2月1日,笛卡尔感染肺炎,并于2月11日在沙努特家中去世。[74]

“昨天早上约凌晨四点,笛卡尔先生在法国大使沙努特先生的住所去世。据我所知,他因患胸膜炎病了几天。但由于他不愿服用药物,似乎出现了高烧。随后,他一天内进行了三次放血,但未失大量血。女王陛下对他的去世深表哀悼,因为他是如此博学之人。他的遗体被制成了蜡像。他并不打算死在这里,因为在去世前不久他决定在机会允许时返回荷兰。”[75]

根据沙努特的说法,笛卡尔死因是肺炎,而克里斯蒂娜的医生约翰·范·沃伦则认为是支气管肺炎,但他未获许可进行放血治疗。[76] (冬季似乎很温和,[77] 除了1月下半月的严寒,笛卡尔本人对此有所描述;然而,“此话可能既是笛卡尔对知识氛围的看法,也是对天气的描述。”)[72]
\begin{figure}[ht]
\centering
\includegraphics[width=7cm]{./figures/630c8d75edc0654b.png}
\caption{笛卡尔的墓(中间,带有铭文细节),位于巴黎圣日耳曼德佩修道院} \label{fig_DKE_7}
\end{figure}
\begin{figure}[ht]
\centering
\includegraphics[width=7cm]{./figures/38f749404f30d034.png}
\caption{笛卡尔纪念碑,建于1720年代,位于阿道夫·弗雷德里克教堂} \label{fig_DKE_8}
\end{figure}
E. Pies 基于医生范·沃伦的一封信对这一说法提出质疑;然而,笛卡尔拒绝了他的治疗,之后也出现了更多质疑这一说法真实性的观点。[78] 在2009年,德国哲学家西奥多·埃贝特提出,笛卡尔可能是被反对其宗教观点的天主教传教士雅克·维奥格毒死的。[79][80] 作为证据,埃贝特指出,笛卡尔的侄女凯瑟琳·笛卡尔在1693年撰写的《笛卡尔哲学家之死报告》中,提到她的叔叔在去世前两天接受“圣餐”时,似乎含蓄地提到了一次投毒。[81]

作为一个身处新教国家的天主教徒,[82][83][84] 笛卡尔被安葬在斯德哥尔摩的阿道夫·弗雷德里克教堂的墓地中,该教堂的墓地主要用于埋葬孤儿。他的手稿归沙努特的妹夫克劳德·克莱尔塞利耶所有,他是一位虔诚的天主教徒,开始通过选择性地删减、添加和发表笛卡尔的信件来将其“塑造成一位圣人”。[85][86]: 137–154  1663年,教皇将笛卡尔的作品列入禁书目录。1666年,在他去世16年后,笛卡尔的遗骸被运回法国,安葬在圣艾蒂安迪蒙教堂。1671年,路易十四禁止了所有笛卡尔主义的讲座。尽管1792年国民公会计划将他的遗骸迁至先贤祠,他最终于1819年被重新安葬在圣日耳曼德佩修道院,但失去了一个手指和头骨。[注8] 他据称的头骨现在在巴黎人类博物馆,[87] 但一些2020年的研究表明,这可能是伪造的。原始头骨可能在瑞典被分成若干部分并赠送给私人收藏者;其中一部分于1691年抵达隆德大学,并至今保存于该校。[88]
\subsection{哲学著作}
\begin{figure}[ht]
\centering
\includegraphics[width=6cm]{./figures/3f1202ed1566e010.png}
\caption{勒内·笛卡尔工作中} \label{fig_DKE_9}
\end{figure}
在《方法谈》中,笛卡尔尝试得出一套可以毫无疑问地被视为真实的基本原则。为此,他使用了一种被称为“极端/形而上学的怀疑”的方法,有时也称为“方法论怀疑”或“笛卡尔怀疑”:他拒绝任何可以被怀疑的想法,然后重新确立它们,以获得真正知识的坚实基础。[89] 笛卡尔从头开始构建他的思想,这在《第一哲学沉思》中得以体现。他将此与建筑进行类比:先将地表土移开,以建立新的建筑或结构。笛卡尔将他的怀疑称为“土壤”,将新的知识称为“建筑”。对笛卡尔而言,亚里士多德的基础主义是不完整的,而他的怀疑方法则增强了基础主义。[90]

起初,笛卡尔仅得出一个首要原则:他思考。这在《方法谈》中用拉丁语表达为“Cogito, ergo sum”(英语:“我思故我在”)。[91] 笛卡尔得出结论,如果他怀疑,那么必定有某物或某人正在进行怀疑;因此,怀疑本身证明了他的存在。“这句话的简单含义是,如果有人对存在持怀疑态度,这本身就是他存在的证明。”[92] 这两个首要原则——“我思考”和“我存在”——后来通过笛卡尔在《第一哲学沉思》第三沉思中的清晰而明确的感知得到了确认:笛卡尔推理道,他对这两个原则的清晰感知确保了它们的不可怀疑性。

笛卡尔得出结论,他可以确定自己存在,因为他在思考。但是什么样的形式?他通过感官感知到自己的身体;然而,感官之前曾是不可靠的。因此,笛卡尔确定唯一不可怀疑的知识是他是一个“思维的存在”。思考是他所做的事,其力量必然来自于他的本质。笛卡尔将“思维”(cogitatio)定义为“在我内发生的、使我对其立即有意识的事情,正如我对它有意识那样”。因此,思考是一个人对其即时有意识的每一项活动。[93] 他给出理由认为清醒的思维是可与梦境区分的,并且一个人的心智不可能被“恶魔”劫持,呈现出一个虚幻的外部世界。[90]

“因此,我以为自己是用眼睛看到的东西,实际上是通过我心智中的判断力来掌握的。”[94]: 109

通过这种方式,笛卡尔开始构建一个知识体系,舍弃了感知作为不可靠的途径,而只接受演绎作为方法。[95]
\subsubsection{心身二元论}
\begin{figure}[ht]
\centering
\includegraphics[width=6cm]{./figures/bc1738493f6191a7.png}
\caption{《人》(1664年)} \label{fig_DKE_10}
\end{figure}
笛卡尔受巴黎附近圣日耳曼昂莱城堡展示的自动机启发,研究了心灵与身体之间的联系及其相互作用。[96] 他心身二元论的主要影响来自神学和物理学。[97] 心身二元论是笛卡尔的标志性学说,并渗透到他提出的其他理论中。笛卡尔的心身分离理论被称为笛卡尔二元论(或心身二元论),对后来的西方哲学产生了影响。[98] 在《第一哲学沉思》中,笛卡尔试图证明上帝的存在以及人类灵魂与身体的区别。人类是心灵与身体的结合体;[99] 因此,笛卡尔的二元论包含了心灵与身体不同但密切相连的观点。尽管许多同时代的读者难以理解心灵与身体的区分,笛卡尔认为这一概念非常直接。他使用“样式”的概念,即物质存在的方式。在《哲学原理》中,笛卡尔解释道:“我们可以清晰地理解一种物质,独立于它的样式,而相反地,我们无法理解脱离物质的样式。”理解样式脱离物质需要智力抽象,[100] 笛卡尔对此解释如下:

“智力抽象在于我将思维从这个更丰富的观念的一部分内容转移开,以便更专注于另一部分。因此,当我考虑某种形状而不想到它所属于的物质或延展时,我在进行一种心理抽象。”[100]

根据笛卡尔的观点,当两个实体可以彼此独立存在时,它们才是真正独立的。因此,笛卡尔推论出上帝与人类是独立的,人类的身体与心灵也是相互独立的。[101] 他认为身体(一个有延展的事物)与心灵(一个无延展的、非物质的事物)之间的巨大差异使二者在本体上截然不同。根据笛卡尔的不可分割性论证,心灵是完全不可分割的,因为“当我考虑心灵,或我自己,只是作为一个思维的存在时,我无法在自己内部区分任何部分;我理解自己是一个完整且单一的存在。”[102]

此外,在《沉思》中,笛卡尔讨论了一块蜡,并阐述了笛卡尔二元论的核心学说:宇宙包含两种截然不同的实体——被定义为思维的心灵或灵魂,和被定义为物质且不具思维的身体。[103] 在笛卡尔时代的亚里士多德哲学中,宇宙被认为是具有目的性或终极目的的。所有发生的事情,不论是星体的运动还是树木的生长,据说都可以通过某种目的、目标或结果来解释。亚里士多德称之为“最终原因”,这些最终原因在解释自然的运作方式时不可或缺。笛卡尔的二元论支持了传统亚里士多德科学与开普勒和伽利略的新科学之间的区别,后者在解释自然时否认了神圣力量和“最终原因”的作用。笛卡尔的二元论为后者提供了哲学依据,通过将最终原因从物质宇宙(或延展之物)中移除,转而交给心灵(或思维之物)。因此,笛卡尔的二元论为现代物理学铺平了道路,同时也为关于灵魂不朽的宗教信仰打开了大门。[104]

笛卡尔的心物二元论暗含了一种关于人类的概念。根据笛卡尔的观点,人类是心灵与身体的复合体。笛卡尔将心灵置于优先地位,认为心灵可以独立于身体存在,但身体无法独立于心灵存在。在《沉思》中,笛卡尔甚至认为,心灵是实体,而身体仅由“偶性”组成。[105] 但他确实主张心灵与身体密切相连:[106]

“自然也通过痛苦、饥饿、口渴等感觉教导我,我并不仅仅是如同船长在船中那样存在于我的身体中,而是非常紧密地与它结合在一起,甚至与之交融,以至于我和身体构成一个整体。如果不是这样,我这个纯粹的思维存在在身体受伤时不会感到疼痛,而是会通过理智感知到损伤,就像水手通过视力察觉到船上的损坏一样。”[106]

笛卡尔关于身体化的讨论引发了他二元论哲学中最令人困惑的问题之一:一个人心灵与身体的结合关系究竟是什么?[106] 因此,笛卡尔的二元论在他去世后为心身问题的哲学讨论设定了议题。[107] 笛卡尔也是一位理性主义者,相信先天观念的力量。[108] 他提出了先天知识的理论,认为所有人类通过上帝的力量在出生时便拥有知识。这一先天知识的理论后来遭到经验主义哲学家约翰·洛克(1632–1704)的反驳。[109] 经验主义主张所有知识都通过经验获得。
\subsubsection{生理学和心理学}
在1649年出版的《灵魂的激情》中,[110] 笛卡尔讨论了当时普遍的观念,即人体内含有“动物灵”。这些动物灵被认为是轻而流动的液体,在神经系统中快速循环,往返于大脑与肌肉之间。人们认为这些动物灵会影响人类的灵魂或灵魂的激情。笛卡尔将激情分为六种基本类型:惊奇、爱、憎恨、欲望、喜悦和悲伤。他认为,这些激情是原始精神的不同组合,并影响灵魂去意图或追求某些行动。例如,他主张恐惧是一种促使灵魂在身体中产生反应的激情。与他关于灵魂和身体分离的二元论教义一致,他推测大脑的某个部分是灵魂与身体之间的连接点,并将松果体确定为这一连接部位。[111] 笛卡尔认为,信号通过动物灵从耳朵和眼睛传递到松果体。因此,松果体的不同运动引发了各种动物灵的活动。他认为松果体中的这些运动基于上帝的意志,人类应渴望和喜欢对自己有益的事物。但他也指出,身体内的动物灵可能会扭曲松果体的指令,因此人类需要学习如何控制自己的激情。[112]

笛卡尔提出了一个关于身体对外界事件自动反应的理论,对19世纪的反射理论产生了影响。他主张外部运动(如触摸和声音)到达神经末端,影响动物灵。例如,火的热量会作用于皮肤上的某个点,引发一连串反应,动物灵通过中枢神经系统到达大脑,随后动物灵传回肌肉,使手移开火源。[112] 通过这一反应链,身体的自动反应不需要思维过程。[108]

最重要的是,他是最早认为灵魂也应被纳入科学研究的科学家之一。他挑战了当时灵魂为神圣的观念,因此宗教权威视其书籍为危险。[113] 笛卡尔的著作成为了情感理论的基础,探讨了认知评估如何转化为情感过程。笛卡尔认为大脑类似于一台运作的机器,数学和力学可以解释其中复杂的过程。[114] 在20世纪,艾伦·图灵基于数学生物学创立了计算机科学,这受到了笛卡尔的启发。他的反射理论在他去世200多年后成为高级生理学理论的基础。生理学家伊万·巴甫洛夫是笛卡尔的热心崇拜者。[115]
\subsubsection{关于动物}
笛卡尔否认动物具备理性或智慧。[116] 他认为动物并非没有感知或知觉,但这些可以用机械原理来解释。[117] 尽管人类有灵魂或心智,能够感到痛苦和焦虑,动物由于没有灵魂则无法感受到痛苦或焦虑。如果动物表现出痛苦的迹象,那只是为了保护身体免受伤害,但它们缺乏必要的内在状态以真正体验痛苦。[118] 虽然笛卡尔的观点并未被普遍接受,但在欧洲和北美广为流行,使人类可以毫无顾忌地对待动物。认为动物完全不同于人类,仅是机器的观点,使虐待动物合法化并得到法律和社会规范的支持,直到19世纪中期才有所改变。[119]: 180–214 随后,查尔斯·达尔文的著作逐渐削弱了笛卡尔关于动物的观点。[120]: 37 达尔文主张人类与其他物种的连续性暗示了动物也可能会经历痛苦。[121]: 177