% 全同粒子
% 量子力学|交换算符|对易|哈密顿|玻色子|费米子
\pentry{多体量子力学简介% 未完成: 应该引用什么词条呢? 这应该是二级词条, 是否引用一级词条?
, 正交子空间\upref{OrthSp}}

\subsection{粒子交换算符}

定义\textbf{粒子交换算符}为(先不考虑自旋)
\begin{equation}
P_{1,2}\psi(\bvec r_1, \bvec r_2) = \psi(\bvec r_2, \bvec r_1)
\end{equation}
可以证明这是一个厄米算符, 即
\begin{equation}
\braket{\phi}{P_{1,2}\psi} = \braket{P_{1,2}\phi}{\psi}
\end{equation}
该算符有 $1$ 和 $-1$ 两个本征值, 对应两个正交子空间\upref{OrthSp}, 分别是对称波函数和反对称波函数(即满足下式)构成的空间.
\begin{equation}
\psi_\pm(\bvec r_2, \bvec r_1) = \pm\psi_\pm(\bvec r_1, \bvec r_2)
\end{equation}

这两个子空间外的波函数既非对称也非反对称.

对称波函数描述全同玻色子, 反对称波函数描述全同费米子.

\subsection{与哈密顿算符对易}

\pentry{守恒量(量子力学)\upref{QMcons}}
对于全同粒子, 交换算符与哈密顿算符对易\footnote{对非全同粒子则不成立, 例如两个质量不同的粒子的哈密顿算符与交换算符不对易}. 这保证了 $P$ 是一个守恒量. 也就是全同粒子的波函数在演化过程中将一直保持对称性或反对称性.

\subsection{本征态与测量}
两个全同粒子的本征态(或者其他任何态)也必须是必须是对称或反对称的, 例如位置本征态 $(\delta_{\bvec r_1} \delta_{\bvec r_2}\pm \delta_{\bvec r_2} \delta_{\bvec r_1})/\sqrt 2$ 只能告诉我们一个粒子在 $\bvec r_1$ 处另一个粒子在 $\bvec r_2$ 处, 仍然不能区分它们.

于是根据测量理论, 我们仍然要将总波函数投影到本征态上去. 例如位置分布为
\begin{equation}\label{IdPar_eq1}
\begin{aligned}
P(\bvec r_1, \bvec r_2) &= \abs{\frac{1}{\sqrt{2}}\int [\delta_{\bvec r_1}(\bvec r'_1) \delta_{\bvec r_2}(\bvec r'_2) \pm \delta_{\bvec r_2}(\bvec r'_1) \delta_{\bvec r_1}(\bvec r'_2)] \psi_\pm(\bvec r'_1,\bvec r'_2) \dd[3]{r'_1}\dd[3]{r'_2}}^2\\
&= \abs{\frac{1}{\sqrt{2}} \psi_\pm(\bvec r_1, \bvec r_2) \pm  \frac{1}{\sqrt{2}} \psi_\pm(\bvec r_2, \bvec r_1)}^2\\
&= \abs{\frac{1}{\sqrt{2}} \psi_\pm(\bvec r_1, \bvec r_2) +  \frac{1}{\sqrt{2}} \psi_\pm(\bvec r_1, \bvec r_2)}^2\\
&= 2\abs{\psi_\pm(\bvec r_1, \bvec r_2)}^2
\end{aligned}
\end{equation}
注意该式中 $P(\bvec r_1, \bvec r_2)$ 不区分 $\bvec r_1, \bvec r_2$ 的顺序. 也就是说 $P(\bvec r_1, \bvec r_2)$ 告诉我们一个粒子在 $\bvec r_1$, 另一个在 $\bvec r_2$ 的概率密度.

对应地, 在做归一化时, 一种方法是先对所有变量在全部范围积分再除以 $2$, 因为每一个 $P(\bvec r_1, \bvec r_2)$ 都重复计算了 $P(\bvec r_2, \bvec r_1)$, 归一化条件
\begin{equation}\label{IdPar_eq2}
\frac{1}{2}\int P(\bvec r_1, \bvec r_2) \dd[3]{r_1}\dd[3]{r_2} = \int \abs{\psi_\pm(\bvec r_1, \bvec r_2)}^2 \dd[3]{r_1}\dd[3]{r_2} = 1
\end{equation}

另一种方法是不除二, 而是选取上式积分范围的一半使得对任何 $P(\bvec r_1, \bvec r_2)$, $P(\bvec r_2, \bvec r_1)$ 不会被重复计算, 在一维运动情况下, 我们可以只对 $r_1, r_2$ 平面满足 $r_1 < r_2$ 的一半积分. 在多维运动时, 我们只需要选取任意一个坐标, 例如令 $y_1 < y_2$ 即可将整个积分范围划分为满足要求的两半\footnote{也可以使用任何可以区分粒子交换的条件, 例如 $x_1^2 + y_1^2 + z_1^2 < x_2^2 + y_2^2 + z_2^2$, 将角标 $1$ 和 $2$ 互换后, 两条不等式只能满足一条.}.

\begin{example}{单粒子位置分布}\label{IdPar_ex1}
若两个全同粒子的波函数为 $\psi(\bvec r_1,\bvec r_2)$, 求单个粒子的位置分布 $P(\bvec r)$. 注意我们并不能求 $P(\bvec r_1)$ 或者 $P(\bvec r_2)$.

我们只需要将\autoref{IdPar_eq1} 对其中一个 $\bvec r$ 全空间积分即可
\begin{equation}
P(\bvec r) = \int P(\bvec r, \bvec r') \dd[3]{r'} = 2\int \abs{\psi_\pm(\bvec r, \bvec r')}^2 \dd[3]{r'} 
\end{equation}
根据\autoref{IdPar_eq2}, $\int P(\bvec r) \dd[3]{r} = 2$ 而不是 $1$, 这并没有问题, 因为空间中共有两个粒子.
\end{example}

\subsubsection{一般投影}
类比\autoref{IdPar_eq1} 可以发现, 将任何对称或反对称的波函数投影到形式为
\begin{equation}
[\phi_1(\bvec r_1)\phi_2(\bvec r_2) \pm \phi_2(\bvec r_1)\phi_1(\bvec r_2)]/\sqrt{2}
\end{equation}
且具有相同对称性的波函数上, 都会得到相同的两项\footnote{如果投影到对称性相反的波函数上, 结果为零}. 所以为了方便计算, 我们可以只投影到第一项 $\phi_1(\bvec r_1)\phi_2(\bvec r_2)$ 上, 再乘以 $\sqrt{2}$ 即可.
