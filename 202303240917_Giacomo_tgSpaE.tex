% 曲面的切空间(古典微分几何)
% 切空间|切向量|道路|几何向量|导子|切从

\pentry{光滑映射\upref{SmthM}}

切空间是切向量的集合。从几何直观来说,切向量就是和某个平面或者超平面相切于一点的向量。

在线性代数中,我们将原点默认为所有向量的起点,这样只需要终点就可以表示向量了,因此向量被一一对应到点上。这种方式获得的向量,实际上是更广义的切向量中的一类,即从原点“发射”出去的向量。我们也可以空间中的其它点作为起点,来得到发射出去的向量。当然,如果把向量理解成“具有长度和方向的量”以及此概念的推广,那么向量的起点是无关紧要的;而切向量是区分了起点的。

为什么叫做切向量呢?我们将在讨论光滑流形在欧几里得空间中的嵌入时看到原因。%引用相关词条,写注释时尚未开始该词条,目前预计将该词条命名为“切空间”。

从一个点 $P$ 发射出去的向量,被称为点 $P$ 处的\textbf{切向量}。因此,线性代数中所研究的空间可以看成是原点处的切向量构成的空间,称作\textbf{原点处的切空间}。

如果一个欧氏空间被嵌入到更高维的一个空间中,比如说将 $\mathbb{R}^2$ 以抛物面的形状嵌入到 $\mathbb{R}^3$ 中,那么切空间的意义就非常直观了:点 $P$ 处的切空间就是 $\mathbb{R}^3$ 中的一个与该抛物面切于点 $P$ 的平面,取点 $P$ 作为该平面的原点,那么这个平面也可以看成一个二维实线性空间 $\mathbb{R}^2$。

\subsection{切空间的定义}

设 $S$ 是一个欧几里得空间中的曲面或者就是欧几里得空间本身,那么在 $S$ 的一点 $x$ 处的切空间被记为 $T_xS$,其定义如下所述。

\subsubsection{几何向量定义}

对于一个欧几里得空间 $\mathbb{R}^n$,我们可以简单地把某点 $x\in\mathbb{R}^n$ 处的切向量定义为以该点为起点的向量,而所有该点处的切向量构成一个 $x$ 上的切空间。曲面上的某一点 $x$ 的切空间也很容易定义:所有从 $x$ 出发且和曲面相切的向量构成的集合,比如三维空间里二维平面上的切空间,直观来看就是一个切平面。显然,切空间就是一个线性空间。

几何向量的定义在欧几里得空间里非常直观,但是当我们讨论曲面,或者更一般地说,流形上的切向量时,就不是那么方便了,因此我们会采用以下定义方式。

\subsubsection{道路定义}

如果在欧几里得空间 $\mathbb{R}^n$ 中有一个曲面 $S$,假设它是一个光滑的曲面,也就是说可以把它看成某个光滑函数的等值面\footnote{对于函数 $f$,任意数值 $a\in\mathcal{F}$,称 $f^{-1}(a)$ 是 $f$ 的一个等值面。},那么我们可以使用道路来求出曲面某一点处的\textbf{切向量(tangent vector)}。

取点 $p\in S$,令 $r:I\rightarrow S$ 是一条光滑\footnote{由于预备知识中光滑的定义依赖于欧氏空间,这里的光滑是指 $r$ 作为 $I\rightarrow\mathbb{R}^n$ 的映射而言的。}的道路,且满足 $r(0)=p$,那么 $r(t)$ 是从 $x$ 处出发的一条向量。从几何的角度来说,这个向量不一定是曲面 $S$ 的切向量,它有可能穿过 $S$;不过从微积分中我们知道,极限 $\lim\limits_{t\rightarrow 0}\frac{r(t)}{\abs{r(t)}}$ 是 $p$ 处的一个切向量。用这个极限计算出来的切向量涵盖了所有可能的方向,但都是\textbf{单位向量};为了讨论 $p$ 处的整个切空间,可以改用 $\lim\limits_{t\rightarrow 0}\frac{r(t)}{t}$ 来定义,这样所有长度的切向量都存在了,甚至包括无穷长度的切向量。

如果一条道路 $r$ 按以上定义导出的向量存在\footnote{光滑道路必然存在导出的向量;虽然我们只考虑光滑道路,但并不妨碍将道路导出切向量的思想应用到一切道路上,这样的话有可能有的道路并不存在导出的向量。}且有限长,那么我们称 $r$ 是\textbf{收敛}的。如果导出的向量不存在或者长度为无穷,那么称 $r$ 是\textbf{发散}的。如果道路 $r$ 导出切向量 $\bvec{v}$,我们也说 $r$\textbf{收敛于}$\bvec{v}$。

不同的道路可能收敛同一个向量。我们把收敛于同一个切向量的道路称为等价的。这样一来,我们就可以对道路进行等价划分,同一个等价类的道路收敛于同一个切向量,不同等价类的道路收敛于不同的切向量。于是,我们就可以把这样定义的道路等价类和切向量一一对应;或者说,\textbf{称道路等价类为向量}。对于道路 $r$,把 $[r]$ 记为所有和 $r$ 等价的道路之集合,也就是 $r$ 所在的等价类。

一个点处的切向量全体构成一个\textbf{切空间(tangent space)},这是一个线性空间。两个道路等价类的和该怎么计算呢?在两个道路等价类 $[r_1]$ 和 $[r_2]$ 中任取代表道路 $r_1$ 和 $r_2$,那么令 $r(t)=r_1(t)+r_2(t)$,得到和道路 $r$,那么可定义道路等价类的向量加法:$[r_1]+[r_2]=[r]$。类似地,可以定义数乘:$a[r]=[ar]$。

不仅是在 $S$ 上,在欧几里得空间 $\mathbb{R}^n$ 本身中也可以用道路定义切向量,这个定义和几何向量定义的切向量是一致的。

\subsubsection{导子定义}

考虑欧几里得空间 $\mathbb{R}^n$。对于该空间中的任意一个光滑函数 $f$,在给定点 $P$ 处,沿着道路 $p(t)=P+(t, 0, 0, \cdots, 0)$ 对 $f$ 求方向导数的操作,实际上等价于求 $\frac{\partial}{\partial x_1}f$。因此,我们可以把道路 $p(t)$ 和偏微分算子 $\frac{\partial}{\partial x_1}$ 等同起来。

一般地,道路 $p(t)=P+(a_1t, a_2t, a_3t, \cdots, a_nt)$ 对应于偏微分算子 $a_1\frac{\partial}{\partial x_1}+a_2\frac{\partial}{\partial x_2}+a_3\frac{\partial}{\partial x_3}+\cdots+a_n\frac{\partial}{\partial x_n}$。因此,我们也可以将切向量理解为偏微分算子,又叫\textbf{导子(derivation)}。



% 同样设在欧几里得空间 $\mathbb{R}^n$ 中有一个光滑曲面 $S$,取点 $p\in S\subseteq\mathbb{R}^n$。令 $\partial/\partial x_i$ 表示对 $\mathbb{R}^n$ 的第 $i$ 个坐标求偏导数,$\partial/\partial x_i|_{p}$ 表示在点 $p$ 处求这个偏微分。那么如果在 $p$ 的某邻域内 $S$ 的部分可以看成是一个函数 $f:\mathbb{R}^{n-1}\rightarrow\mathbb{R}$,则 $\partial/\partial x_i|_{p}f$ 是 $S$ 在 $p$ 处的一个切向量。全体形如 $\sum a_i\cdot\partial/\partial x_i|_{p}f$ 的向量构成了 $p$ 处的切空间。

% 进一步地抽象,不考虑具体的 $f$ 形式,则 $\partial/\partial x_i|_{p}$ 是 $S$ 在 $p$ 处的一个切向量。全体形如 $\sum a_i\cdot\partial/\partial x_i|_{p}$ 的向量构成了 $p$ 处的切空间。

$\partial/\partial x_i|_{p}f$ 表示的是一个偏导数的值,而 $\partial/\partial x_i|_{p}$ 是表示求偏导数这一操作,因而被称作算子。任何导子相加或乘以某个数字的结果还是导子。导子之间自然引出的向量加法和数乘使它们构成了一个线性空间。

导子和偏导数的关系,就像电场和静电力的关系一样。在电场中放入一个测试电荷,测试电荷就会受到电场施加的静电力,但我们通常研究的是电场本身,测试电荷只是方便描述其性质,并不是关键;同样地,取一个测试函数,用导子作用于测试函数就可以得到一个数值,但我们研究的是导子本身,要摆脱具体的测试函数。

% 在欧几里得空间 $\mathbb{R}^n$ 中用导子定义切向量,此时导子定义、道路定义和几何向量定义的切向量是一致的。

\subsubsection{三种定义的联系}

对于流形上的图 $(U,\varphi)$ 或流形上的某个嵌入 $i$,沿着相应道路对时间 $t\in I$ 求导,可以得到 $\varphi(U)$ 中的向量,即几何向量定义;也可以得到 $i(U)$ 上的切向量,这是几何向量定义难以清楚描述的。道路定义就是沿着道路求方向导数,但将直线道路取代成任意光滑道路,所求的方向导数是光滑道路在起点处切线的方向。这是因为我们对图的选择是任意的,一个图中的直线道路到另一个图中不一定还是直线,但一定是光滑道路。而导子本身就是在求方向导数。

今后,当我们提到切向量时,总是指道路定义和导子定义,有时会交替使用,读者应熟悉两种定义之间的等价性。

\subsection{切丛}
详见\textbf{切丛}\upref{TanBun}。

欧氏空间中任意一点处都有切空间,所有这些切空间构成的集合,被称为一个\textbf{切丛(tangent bundle)}。$\mathbb{R}^n$ 上每个点的切空间都同构于 $\mathbb{R}^n$,因此切丛可以看成集合 $\mathbb{R}^n\times\mathbb{R}^n=\mathbb{R}^{2n}$\footnote{这种看法仅限于欧几里得空间这样已经自带\textbf{联络}的情况。实际上\textbf{丛}并不能简单地看成两个拓扑空间的笛卡尔积,详见\textbf{纤维丛}\upref{Fibre}。}。

物理学中有一个很常见的切丛的实例,那就是切向速度空间。给定一个时空,一个质点在时空中每一个点处都可以取不同的瞬时速度,在每个点上可能的瞬时速度的集合,就构成了这个点上的一个切空间。所有点上所有可能的瞬时速度的集合,就构成了该时空中的一个切丛。









