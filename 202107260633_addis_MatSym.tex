% Matlab 符号计算和任意精度计算(笔记)

\pentry{Matlab 的程序调试及其他功能\upref{MatOtr}}

Matlab 的符号计算需要符号计算工具箱, 取决于你的证书类型, 可能需要额外购买. 另外需要提醒的是, 虽然 Matlab 和 Python 都有符号计算功能, 但在符号计算领域 Mathemtica\upref{Mma} 才更为主流, 其默认界面也更适合符号计算.

\subsection{符号变量和符号表达式}
\begin{itemize}
\item Matlab 中用于储存符号计算表达式的变量类型为 \verb|sym|. 可以用 \verb|syms 变量1 变量2 ...| 声明变量类型为 \verb|sym|. 例如 \verb|syms x y z;|. Matlab 的大部分自带算符和函数支持 \verb|sym| 类型的变量, 例如 \verb|x^2| 就是 \verb|sym| 类型的表达式 $x^2$, $x$ 并不是一个数值而是符号. 若此时令 \verb|expr = x^2|, 那么用 \verb|class(expr)| 可以验证 \verb|expr| 的类型也是 \verb|sym|.

\item 除了使用 \verb|syms| 一次声明几个符号变量, 也可以使用 \verb|sym(字符串)|. 例如 \verb|syms x; expr = x^2;| 得到的 \verb|expr| 和 \verb|expr = sym('x')^2;| 得到的 \verb|expr| 是等效的.

\item  对表达式求导如 \verb|diff(sym('x')^2)|. 若要求偏导, \verb|diff(sym('x')^2 * sym('y')^3)| 默认对 \verb|x| 求偏导, 结果是 $2x y^3$. 也可以声明对 \verb|y| 求偏导, 如 \verb|diff(sym('x')^2 * sym('y')^3), sym('y')|, 或者更简洁地, \verb|diff(sym('x')^2 * sym('y')^3), 'y'|. 这时因为, 如果函数的一个参数需要符号表达式, 但如果输入时使用了字符串或者数值, 那么 Matlab 就会自动将其用 \verb|sym()| 函数进行转换. 建议总是声明对哪个变量求偏导.

\item  要把一个数字作为符号, 可以使用例如 \verb|sym('2')| 或 \verb|sym('7/3')|, 但这仅限于分式, 不允许注入 \verb|sym('sqrt(2)')| 这样的用法.

\item 另一种方法是使用 \verb|sym(数值)|. 例如 \verb|sqrt(sym(2))| 的结果是表达式 $\sqrt 2$, 而不同于数值计算中的 \verb|sqrt(2) = 1.414...|. 由于符号计算是精确无误差的, 无理数 $\sqrt{2}$ 并不会被自动转换为小数形式. 另一个例子, \verb|d = 3.1; sqrt(sym(d))| 得到的是表达式 $961/100$. \verb|sym(数值)| 会自动猜测 \verb|数值| 所代表的根式, 例如 \verb|sym(0.866025403784439)| 的结果是表达式 $\sqrt{3}/2$, 又例如 \verb|sym(pi)| 的结果是圆周率 $\pi$. 注意在新版的 Matlab 中, \verb|sym('pi')| 将得到名为 \verb|pi| 的普通变量, 而不是圆周率.
\end{itemize}

\subsubsection{符号替换}
\begin{itemize}
\item \verb|subs(符号表达式,符号变量, 新表达式)| 可以把表达式中的所有 \verb|符号变量| 替换为 \verb|新表达式|. 例如 \verb|subs(sin(sym('x')), 'x', 'y')| 相当于 \verb|subs(sin(sym('x')), sym('x'), sym('y'))|, 结果是 $\sin y$. 又例如 \verb|subs(sin(sym('x')), 'x', pi/4)| 的结果是 $\sqrt 2/2$.
\item 如果要对一个符号表达式求数值近似, 那么用 \verb|eval()|, 例如 \verb|eval(sqrt(sym(2)))| 结果是 \verb|1.414...|, 是一个双精度数值, 误差就是双精度类型的相对误差, 约为 $2\e{-16}$.
\end{itemize}

\subsection{任意精度计算}
注意任意精度计算功能往往和符号计算一同使用, 但该功能本身却可以独立使用. 任意精度计算本质上还是数值浮点计算, 存在数值误差. 只不过我们可以规定每个变量的有效数字位任意值, 而不是统一使用双精度类型的 16 位有效数字.

\begin{itemize}
\item 和 \verb|eval()| 不同, \verb|vpa()| 可以计算符号表达式的任意精度结果, 例如 \verb|vpa(sqrt(sym('2')), 50)| 计算 $\sqrt{2}$ 的 50 位有效数字. 返回的值是一个 \verb|'sym'| 类型的对象而不是 \verb|'double'| 类型.
\item  特别注意表示成小数形式的 \verb|sym| 类型数字在计算过程中是存在误差的, 例如令 \verb|x = vpa(sqrt(sym('2')), 5)|
\end{itemize}

