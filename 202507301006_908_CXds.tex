% 抽象代数(综述)
% license CCBYNCSA3
% type Wiki

本文根据 CC-BY-SA 协议转载翻译自维基百科\href{https://en.wikipedia.org/wiki/Abstract_algebra}{相关文章}。

\begin{figure}[ht]
\centering
\includegraphics[width=6cm]{./figures/e5b4bfc6385c6286.png}
\caption{魔方的所有排列构成一个群,这是抽象代数中的一个基本概念。} \label{fig_CXds_1}
\end{figure}
在数学中,更具体地说,在代数学中,抽象代数或现代代数是研究代数结构的学科。代数结构是指带有特定运算作用于其元素的集合。\(^\text{[1]}\)代数结构包括群、环、域、模、向量空间、格以及域上的代数。抽象代数这一术语是在20世纪早期提出的,用来将其与代数学的旧分支区分开来,更具体地,是为了区别于初等代数,即使用变量来表示数进行计算和推理的部分。如今,抽象的代数观点已成为高等数学中如此根本的内容,以至于通常直接称为“代数”,而“抽象代数”这一术语除了在教学中很少再被使用。

代数结构及其相关的同态构成数学上的范畴。范畴论提供了一个统一的框架,用来研究各种结构中类似的性质和构造。

一般代数是一个相关学科,研究将不同类型的代数结构作为单一对象来对待。例如,在一般代数中,群的结构是一个单一的对象,被称为群的多类。
\subsection{历史}
在19世纪之前,代数被定义为对多项式的研究。\(^\text{[2]}\)随着更复杂的问题和解法的发展,抽象代数在19世纪逐渐产生。具体的问题和例子来自数论、几何、分析以及代数方程的解。如今被认为是抽象代数组成部分的大多数理论,最初只是来自数学各个分支的一些零散事实的集合,随后逐渐形成了一个共同的主题,作为核心将各种结果汇集起来,并最终在一套共同概念的基础上实现统一。这种统一发生在20世纪的前几十年,促成了对群、环、域等各种代数结构的形式公理化定义。\(^\text{[3]}\)这一历史发展过程几乎与流行教材中的处理方式相反,例如范德瓦尔登的 《现代代数》\(^\text{[4]}\),这些教材通常在每一章开头给出某种结构的形式定义,然后再给出具体的实例。\(^\text{[5]}\)
\subsubsection{初等代数}
对多项式方程或代数方程的研究有着悠久的历史。大约公元前 1700 年,巴比伦人已经能够解出以文字题形式给出的二次方程。这个“文字题”阶段被称为修辞代数,并且一直到 16 世纪都是主流方法。公元 830 年,花拉子米首次提出“algebra(代数)”一词,但他的工作完全属于修辞代数。完全符号化的代数直到弗朗索瓦·维埃特1591 年的《新代数》才出现,即便如此,其中仍有一些拼写出的词语,直到笛卡尔1637 年的《几何学》才被赋予统一的符号表示。\(^\text{[6]}\)对符号方程求解的正式研究促使莱昂哈德·欧拉在 18 世纪末接受当时被认为是“荒谬”的根,例如负数和虚数。\(^\text{[7]}\)然而,大多数欧洲数学家直到 19 世纪中叶仍然抗拒这些概念。\(^\text{[8]}\)

乔治·皮考克1830 年的《代数论》是第一次尝试将代数完全建立在严格的符号基础之上。他区分了新的符号代数与旧的算术代数。在算术代数中,$a - b$被限制为$a \geq b$,而在符号代数中,所有的运算规则在没有任何限制的情况下成立。利用这一点,皮考克能够证明类似$(-a)(-b) = ab$这样的法则:只需令$a = 0,\ c = 0$代入$(a - b)(c - d) = ac + bd - ad - bc$即可成立。皮考克使用他称为等价形式永久性原理来为他的论证辩护,但他的推理存在归纳法问题。\(^\text{[9]}\)例如,$\sqrt{a}\sqrt{b} = \sqrt{ab}$对于非负实数成立,但对一般复数却不成立。
