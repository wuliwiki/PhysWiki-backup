% 商空间(线性代数)
% keys 商集|商空间|向量空间
% license Xiao
% type Tutor

\begin{issues}
\issueDraft
\end{issues}

\pentry{二元关系\nref{nod_Relat},仿射集\nref{nod_AffSet}}{nod_f4b0}

\addTODO{用仿射集的语言重写}


\footnote{参考《物理学中的几何方法》} % Giacomo:改成cite

设 $W$ 是向量空间 $V$ 的一个子空间。我们想通过 $W$ 来定义 $V$ 中元素的一个等价关系,并由此得到 $V$ 的一个划分(商集)。对于 $v \in V$,$v + W$ 是 $V$ 的一个仿射集,全体(关于 $W$ 的)仿射集的集合构成里一个划分(商集),定义了一个等价关系——对于 $v_1, v_2\in V$,如果 $v_1 - v_2\in W$,则称它们是关于 $W$ 等价的,记为 $v_1 \sim v_2$。我们把这个划分(商集)记为$V / W$,包含 $v$ 的等价类记为 $[v]$.

% 不难证明这是一个等价关系。因此,它就确定了 $V $ 的一个分类。元素 $v $ 关于 $\bmod W $ 的相合类,即 $V $ 中所有与 $v $ 关于 $\bmod W $ 的相合的元的全体,用符号 $(v) $ 标记。这样,有相合类为元素的商集
% \begin{equation}
% V / W=\{(v) | v \in V\}~.
% \end{equation}

现在我们在商集 $V/W $ 中引入线性运算,使它也成为一个线性空间。当然这种线性运算应与 $V $ 中原有的线性运算要有联系。为此,对于相合类的加法和数乘,我们自然采用下列定义
\begin{equation}
\begin{array}{l}\left(v_{1}\right)+\left(v_{2}\right)=\left(v_{1}+v_{2}\right)~, \\ a(v)=(a v)\qquad (a \in K)~.
\end{array}
\end{equation}
这里的定义与相合类的代表的选取无关,故是有意义的,容易证明 $V/W $ 在这些运算下构成体 $K $ 上的向量空间。$V/W $ 称为 $V $ 关于 $\bmod W$ 的\textbf{商空间},$(0)$ 然是它的零元。

下面我们再来分析一下 $V,W$ 和 $V/W $ 三者基之间的关系。设 $u_1,u_2,\cdots,u_m$ 是 $W $ 的一个基。我们再补充 $n- m$ 个元素 ${u}_{m+1}, \cdots {u}_{n}$ 使 $u_{1}, u_{2}, \cdots, u_{n}$ 成为 $V$ 的一个基。于是对 $V $ 中任一元 $v=a^iu_i$ 所确定的相合类,有
\begin{equation}
(v)=\left(a^{i} u_{i}\right)=a^{i}\left(u_{i}\right)~.
\end{equation}
(未完成)
%因此任意相合类可用(U1) , (Uz) , …, (u.) 的线性组合表示。但是这并不是说,
%(u心(u心…, Cun) 构成V/W 的一个基因为它们是线性相关的。由于u,EW
%(i=l, …,m), 故, u,-OEW, 即u, (i= l, …, m) 与0 是关于mod W 相合,所以
%最后有
%(u,) = (0) , i = l , 2 ,\dots, m
%类似地分析可知
%(u,)\#-(0) ,j=m+l, \dots,n
%且容易证明(作为练习), (u1) (j = m + 1, \dots, n) 是线性无关的,它们构成V/W
%的一个基。作为推论,我们有dim V / W= n-m.
%这些概念和结构,我们将在讨论同调群和上同调群时用到。
