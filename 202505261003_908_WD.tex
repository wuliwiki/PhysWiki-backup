% 弗朗索瓦·韦达(综述)
% license CCBYSA3
% type Wiki

本文根据 CC-BY-SA 协议转载翻译自维基百科 \href{ https://en.wikipedia.org/wiki/Fran\%C3\%A7ois_Vi\%C3\%A8te}{相关文章}。

\begin{figure}[ht]
\centering
\includegraphics[width=6cm]{./figures/34ef13936a106163.png}
\caption{} \label{fig_WD_1}
\end{figure}
弗朗索瓦·韦达(法语:[fʁɑ̃swa vjɛt];1540年-1603年2月23日),拉丁文名为弗朗西斯库斯·维埃塔(Franciscus Vieta),是一位法国数学家。他在新代数学方面的研究是通向现代代数学的重要一步,原因在于他创新性地使用字母作为方程中的参数。他本职是一名律师,曾担任法国亨利三世和亨利四世的枢密顾问。
\subsection{生平}
\subsubsection{早年生活与教育}
韦达出生于现今法国旺代省的丰特奈勒孔特。他的祖父是来自拉罗谢尔的一位商人,父亲埃蒂安·韦达是丰特奈勒孔特的一名律师,并在勒比索担任公证人。他的母亲是法国天主教同盟势力上升时期,议会首任主席兼法官巴尔纳贝·布里松的姑妈。

韦达曾在方济各会学校就读,并于1558年前往普瓦捷学习法律,次年取得法学学士学位。一年后,他在家乡开始了律师生涯。起初他就被委以重任,办理了一些重要案件,包括替法国国王弗朗索瓦一世的遗孀在普瓦图地区处理租金问题,并代表苏格兰女王玛丽维护其在法国的利益。
\subsubsection{在帕尔特奈任职}
1564年,韦达进入苏比斯夫人安托瓦内特·多贝泰尔的服务。她是帕尔特奈-苏比斯的让五世的妻子,而让五世是胡格诺派(法国新教)主要的军事领袖之一。韦达随他前往里昂,收集有关他在前一年英勇保卫该城抵御萨伏依的雅克(第二代内穆尔公爵)军队的资料。

同年,韦达在现今法国旺代省穆尚普公社的帕尔克-苏比斯担任苏比斯夫人12岁女儿凯瑟琳·德·帕尔特奈的家庭教师。他教授她科学和数学,并为她撰写了多部关于天文学和三角学的论文,其中一些作品至今仍留存。在这些著作中,韦达使用了十进制数(比斯特文发表相关论文早20年),并注意到行星运动的椭圆轨道(这比开普勒早了40年,也早于布鲁诺之死20年)。

帕尔特奈的让五世向法国国王查理九世推荐了韦达。韦达撰写了帕尔特奈家族的谱系,并在让五世于1566年去世后为他写了传记。

1568年,苏比斯夫人安托瓦内特将女儿凯瑟琳嫁给了查尔斯·德·凯勒内克男爵,韦达随她前往拉罗谢尔。在那里,他接触到加尔文主义贵族的最高层,包括科利尼、孔代以及纳瓦拉的女王让娜·达尔布雷及其子纳瓦拉的亨利——即后来的法国国王亨利四世。

1570年,韦达拒绝为苏比斯夫人及其女儿在一桩臭名昭著的诉讼中出庭辩护。在那场官司中,她们指控凯勒内克男爵无法(或不愿)生育继承人。
\subsubsection{初到巴黎}
1571年,韦达在巴黎注册成为律师,并持续探望他的学生凯瑟琳。他经常居住在丰特奈勒孔特,担任一些市政职务。他开始出版《通用观察:论数学正割表的一部专著》,并在夜晚或闲暇时间从事新的数学研究。据他的朋友雅克·德·图记载,他常常连续三天专注于一个问题,手肘支在书桌上,几乎不挪动位置地进食。

1572年,韦达恰好身在巴黎,亲历了“圣巴托洛缪大屠杀”。当晚,凯勒内克男爵在试图营救阿德米拉尔·科利尼未果后被杀。同年,韦达结识了加尔纳什的夫人弗朗索瓦丝·德·罗昂,并成为她的顾问,协助她对抗内穆尔公爵雅克。

1573年,他被任命为雷恩高等法院的顾问。两年后,他说服安托瓦内特·多贝泰尔同意将凯瑟琳·德·帕尔特奈嫁给弗朗索瓦丝的弟弟——雷恩公爵勒内·德·罗昂。

1576年,雷恩公爵亨利·德·罗昂给予韦达特别庇护,并在1580年推荐他担任“申诉事务主管”。1579年,韦达完成了《通用观察》的印刷工作(由梅泰耶出版社出版),该书作为两部三角函数表的附录出版:《数学正割表,或关于三角形》,即《通用观察》标题中提到的“canon”),以及《有理边三角形表》。一年后,他被任命为巴黎高等法院的申诉事务主管,正式为国王服务。同年,他在内穆尔公爵与弗朗索瓦丝·德·罗昂的诉讼中成功为后者辩护,因而引来顽固的天主教同盟的怨恨。
\subsubsection{流亡丰特奈}
1583年至1585年间,在天主教同盟的劝说下,法国国王亨利三世将韦达罢免,因为他被指同情新教事业。在罗昂的推动下,纳瓦拉的亨利(即后来的亨利四世)于1585年3月3日和4月26日分别致信法国国王亨利三世,试图促使韦达重返原职,但未能成功。

随后,韦达与弗朗索瓦·德·罗昂一同退居丰特奈和博瓦尔叙梅尔。他潜心数学研究,度过了四年,并于1591年完成了其名作《新代数学》。
\subsubsection{为两位国王破译密码}
1589年,亨利三世在布卢瓦避难,并命令王室官员于当年4月15日前抵达图尔。韦达是最早返回图尔的人之一。他成功破译了天主教同盟及其他国王敌人的密信。后来,他与古典学者约瑟夫·朱斯特·斯卡利杰发生争论,并在1590年战胜了对方。

亨利三世去世后,韦达成为纳瓦拉的亨利(即法国国王亨利四世)的枢密顾问。\(^\text{[4]: 75–77  }\)国王非常欣赏韦达,钦佩他的数学才华。韦达被任命为图尔高等法院的顾问。1590年,韦达破译了一种由500多个字符组成的西班牙密码系统,这使得法国方面获得的所有西班牙文密电都能被轻易解读。\(^\text{[5]}\)

亨利四世公开了一封由指挥官莫雷奥写给西班牙国王的信件。这封信经韦达解读,揭露了法国天主教同盟首领夏尔·德·梅讷企图篡位,取代亨利四世登上法国王位。信件的公布成为法国宗教战争和解的重要转折点。西班牙国王甚至指控韦达使用了“魔法”来破译密码。

1593年,韦达发表了自己反驳斯卡利杰的论证。从1594年起,他被专门任命负责破译敌方的秘密密码。
\subsubsection{格里高利历改革}
1582年,教皇格里高利十三世发布通谕《Inter gravissimas》,命令天主教国家的国王执行从儒略历向格里高利历的变更。该历法改革基于卡拉布里亚医生阿洛伊修斯·利利乌斯(Aloysius Lilius,又名路易吉·利利奥或路易吉·吉利奥)的计算。在他去世后,其工作由教皇的科学顾问克里斯托弗·克拉维乌斯接手并推进。

1600年,韦达在一系列小册子中指责克拉维乌斯随意引入修正和中间日,并误解了其前任(利利乌斯)在月相周期计算中的原意。韦达提出了一个新的历法时间表,然而在他去世后,克拉维乌斯在其著作《解释》(Explicatio,1603年)中巧妙地予以反驳。\(^\text{[6]}\)

据说韦达是错的。毫无疑问,他自视甚高,正如数学史学家东布尔所说,他相信自己是某种“时间之王”。\(^\text{[7]}\)韦达确实对克拉维乌斯评价不高,正如雅克·德·图所记:

他曾说克拉维乌斯确实非常擅长解释数学原理,也能清晰地理解前人所发明的内容,并撰写了多种论文,将前人著作汇编整理而成,但未加引用来源。因此,他的作品虽然更有条理,但原始著作中的内容却被他重新组织、易于理解,却也掩盖了最初的出处。
\subsubsection{阿德里安·范·罗门的问题}
1596年,斯卡利杰在莱顿大学重新发起对韦达的攻击。次年,韦达作出了最终的回应。同年3月,阿德里安·范·罗门(Adriaan van Roomen)向欧洲顶尖数学家提出一个挑战:求解一个45次多项式方程。法国国王亨利四世从荷兰大使那里听到冷嘲热讽:大使声称法国根本没有数学家,并说之所以没有法国人被邀请解决这个问题,仅仅是因为荷兰数学家范·罗门从未请法国人出手。

韦达出场,看了一眼这个问题,然后在窗前站了几分钟,便解出了该题。这个方程实际上是 $\sin(x)$ 与 $\sin(x/45)$ 之间的关系。他立刻解决了它,并表示他可以(事实上第二天就)将另外22道类似难题一并解出并交给大使。他后来回忆道:“一看便解”。此外,韦达还给范·罗门出了一个新的问题,要求他用欧几里得工具(即直尺和圆规)来解出亚历山大时代数学家佩尔加的阿波罗尼乌斯所提出的一个早已失传的几何问题。范·罗门面对这个问题无力应对,不得不借助一个“巧计”才能解出(见下文详细说明)。
\subsubsection{晚年岁月}
1598年,韦达获准特别休假。然而,亨利四世仍委托他平息一场公证人叛乱——国王命令这些公证人退还所收费用,引发了他们的不满。由于长期操劳,身体欠佳,韦达于1602年12月辞去了王室职务。他获得了2万埃居的酬劳,这笔钱后来在他去世时被发现放在他的床边。

在去世前几周,他还撰写了一篇关于密码学问题的最后论文,该论文使当时所有现行的加密方法瞬间过时。他于1603年2月23日去世,如德·图所记。\(^\text{[8]}\)他留下了两个女儿:珍妮,其母为芭芭拉·科特罗;苏珊娜,其母为朱莉安娜·勒克莱尔。长女珍妮于1628年去世,曾嫁给布列塔尼议会顾问让·加布里奥;苏珊娜于1618年1月在巴黎去世。

韦达的死因不明。他的学生、著作出版人亚历山大·安德森称其为“praeceps et immaturum autoris fatum”(即“作者英年早逝,命途多舛”)。\(^\text{[5][9]}\)
\subsection{著作与思想}
\subsubsection{新代数学}
\textbf{背景}
\begin{figure}[ht]
\centering
\includegraphics[width=6cm]{./figures/ac6194bfc397d35e.png}
\caption{《著作集》,1646年} \label{fig_WD_2}
\end{figure}
16世纪末,数学仍处于希腊几何与阿拉伯方程求解法则的双重影响之下。在韦达所处的时代,代数学因此在两极之间摇摆:一方面是看似由规则列表组成的算术,另一方面是看上去更为严谨的几何学。与此同时,意大利数学家卢卡·帕乔利、斯奇皮奥内·德尔·费罗、尼科洛·丰塔纳·塔塔利亚、杰罗拉莫·卡尔达诺、洛多维科·费拉里,尤其是拉斐尔·邦贝利(1560年)等人,都发展了三次方程的解法,预示着一个新时代的到来。

另一方面,来自德国 Coss 学派的传统,以及威尔士数学家罗伯特·里科德(1550年)和荷兰人西蒙·斯特文(1581年)的努力,引入了早期的代数符号体系,如小数和指数的使用。然而,复数在当时最多只是哲学上的思考工具。几乎一个世纪后,笛卡尔才将它们视为“虚数”来使用。当时仍只考虑正实数解,几何证明仍是常规方法。

当时数学家的任务实际上是双重的:一方面,需要以更几何化的方式构建代数学,为其提供严谨的基础;另一方面,也要使几何更具代数特性,使平面上的解析计算成为可能。韦达与笛卡尔正是完成了这场“双重革命”的关键人物。

\textbf{韦达的符号代数学}

首先,韦达为代数学奠定了与几何学同等坚实的基础。他结束了以操作步骤为主的代数学传统(即“代数与相等法”——al-Jabr 与 al-Muqabala),创立了最早的符号代数学,并宣称凭借这种方法,“没有任何问题不能解决”。\(^\text{[10][11]}\)

在他为《导论》所写的献词中,韦达致凯瑟琳·德·帕尔特奈写道:

“凡是新的事物,在起初通常都显得粗糙而无形,必须由后世加以打磨与完善。看啊,我所呈现的这门艺术虽然是新的,但实际上却是古老的,只不过被野蛮人破坏与玷污得面目全非。因此,为了在其中引入一种全新的形式,我认为有必要重新构思并出版一套新的词汇体系,将所有伪术语清除掉……”\(^\text{[12]}\)

韦达尚未使用后来的“乘号”符号(由威廉·奥特雷德于1631年引入),也未使用等号“=”。这种缺失尤其引人注目,因为罗伯特·里科德早在1557年就采用了现代等号,而威廉姆斯·克西兰德自1575年起已使用平行竖线来表示等值。\(^\text{[5]}\)此外还值得注意的是,1572年拉斐尔·邦贝利曾用类似“u”的符号并在其上方标数字,来表示未知数的幂次。\(^\text{[13]}\)

韦达既缺乏时间,也没有得力的学生来精彩地展现他的方法。他出版作品耗时多年(因为他非常严谨),更重要的是,他做出了一个非常独特的选择:将未知数与参数明确区分,用元音字母表示未知数,用辅音字母表示参数。在这一记号体系上,他或许是借鉴了一些同时代的学者,比如彼得鲁斯·拉穆斯,后者在几何图形中以元音标示点位,仅在用尽后才用辅音字母如 R、S、T 等。\(^\text{[5]}\)这种选择在后来的数学家中并不受欢迎,包括笛卡尔在内的许多人更倾向于用字母表的前几个字母表示参数,后几个字母表示未知数。

韦达在某些方面也仍受制于其时代背景。首先,作为拉穆斯思想的继承者,他并未将线段长度视为“数”。他的书写体系严格遵循同类量的一致性,这使得表达更为繁琐难懂。他未能承认邦贝利提出的复数概念,并且常常需要借助几何构造来反复验证其代数解。尽管他完全明白自己的新代数学已经足以提供解法,这种“让步”仍在一定程度上损害了他的声誉。

尽管如此,韦达依然带来了诸多创新:他提出了二项式公式,后来被帕斯卡和牛顿进一步发展;他还首次建立了多项式的系数与其根的和与积之间的关系,即著名的“韦达公式”。

\textbf{几何代数学}

韦达擅长多种当时最新的数学技巧,致力于通过引入与原始未知量相关的新量来简化方程。他的另一部作品《几何构造的规范审查》具有现代特征,后来被称为“代数几何”——它是一系列使用直尺和圆规构造代数表达式的方法规则汇编。这些著作通俗易懂,因此具有极高的教学价值。然而,他最早提出的“同类性原则”(homogeneity principle)则超前于时代,以至于大多数读者未能真正理解这一思想。古典时代的希腊数学家曾经使用过这一原则,但在后来的数学家中,只有希罗(Hero)、丢番图(Diophantus)等少数人敢于将线段和面积视为纯粹的数值,并将其相加得到新的数值(即它们的和)。\(^\text{[5]}\)

这种关于“加和”的研究,在丢番图的作品中已有体现,或许正是促使韦达提出如下原则的动因:一个方程中的所有量必须是同类的——要么都是线段、要么都是面积、体积,甚至更高维的“超体积”;纯粹的数值间方程被视为不可接受。在韦达之后的几个世纪中,数学界对这一问题的看法经历了多次转变。现代数学家则倾向于将非同类的方程化为齐次(同类)形式,以获得更对称的表达。韦达本人尚未走得如此之远,然而,他间接地启发了这一方向。他还提出了解析求解二次、三次和四次方程的一般方法,这些方法不同于西皮奥内·达尔·费罗和洛多维科·费拉里的解法,而他当时并不知道他们的工作。他还设计了二次和三次方程的数值近似解法,虽然在他之前斐波那契可能已有相关尝试,但那种方法后来完全失传。\(^\text{[5]}\)

最重要的是,韦达是第一位不仅为“未知数”,也为“问题本身”引入符号表示的数学家。\(^\text{[10]}\)正因如此,他的代数学不再只是一套规则陈述,而是一种高效的“计算代数学”——在其中,各种运算直接作用于字母符号,最后只需简单地代入具体数值即可得解。这种方法,正是当代代数方法的核心,成为数学发展的关键一步。\(^\text{[14]}\)借此,韦达不仅为中世纪代数学(从花剌子密到斯特文)画上句号,也正式开启了代数学的现代时代。
\subsubsection{物种逻辑}
由于经济宽裕,韦达开始自费出版其系统的数学理论著作,赠予遍布欧洲各国的朋友与学者。他将这一理论称为“物种逻辑”(species logistic,其中 species 指“符号”),即“符号运算艺术”(1591年)。\(^\text{[15]}\)

他将解决数学问题的过程分为三个阶段:

\begin{itemize}
\item 第一步是将问题用方程的形式加以表达,韦达将此阶段称为“探求阶段”。在这一阶段,他用辅音字母(如 B、D 等)表示已知量,用元音字母(如 A、E 等)表示未知量。
\item 第二步是分析问题,他称之为“结论阶段”。在此阶段,数学家需分析方程并求解,得出问题的性质与特征(即“推论”——porisma),从而可以进入下一阶段。
\item 第三步是“解释性分析”,即回到原始问题,通过几何或数值构造来呈现解答,并以之前的 porisma 为基础加以验证。
\end{itemize}

韦达曾用这种方法处理过多类问题,包括完整解出形式为$X^2 + Xb = c$的二次方程,以及形式为$X^3 + aX = b$的三次方程(韦达将其化简为二次方程处理)。他了解方程的正根(在他那个时代,根通常只指正数根)与未知数各次幂的系数之间的联系(即韦达公式,以及其在二次方程中的应用)。他还发现了一个公式,可由已知角的正弦值推导出倍角的正弦值,并考虑到了正弦函数的周期性。据推测,韦达早在1593年便已知晓此公式。\(^\text{[5]}\)
\subsubsection{韦达公式}
1593年,韦达在几何推理的基础上,结合他精通的三角计算,首次在数学史上发现了一个无穷乘积,并以此给出了圆周率 π 的表达式,这个表达式如今被称为“韦达公式”:[16]
$$
\pi = 2 \times \frac{2}{\sqrt{2}} \times \frac{2}{\sqrt{2 + \sqrt{2}}} \times \frac{2}{\sqrt{2 + \sqrt{2 + \sqrt{2}}}} \times \frac{2}{\sqrt{2 + \sqrt{2 + \sqrt{2 + \sqrt{2}}}}} \times \cdots~
$$
他还使用阿基米德的方法,将一个正六边形反复倍增至边数为 $6 \times 2^{16} = 393,216$ 的多边形,借此计算出了$\pi$的前10位小数。
