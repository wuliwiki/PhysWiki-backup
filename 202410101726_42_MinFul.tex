% Minkowski 泛函
% keys Minkowski泛函
% license Usr
% type Tutor

\pentry{齐次凸泛函\nref{nod_ConFul},凸集和凸体\nref{nod_ConSet}}{nod_bf82}

\begin{definition}{Minkowski泛函}\label{def_MinFul_1}
设 $L$ 是任一线性空间,$A$ 是\enref{核}{ConSet} 包含 0 的 $L$ 中的凸体(\autoref{def_ConSet_1}),则称泛函
\begin{equation}\label{eq_MinFul_1}
p_A(x)=\inf\qty{r|\frac{x}{r}\in A,r>0}~
\end{equation}
为凸体 $A$ 的\textbf{Minkowski泛函}。

\end{definition}


\begin{theorem}{}
Minkowski泛函\autoref{eq_MinFul_1} 是齐次凸的与非负的。
\end{theorem}

\textbf{证明:}对每一 $x\in L$,如果 $r$ 充分大,则元素 $x/r$ 属于 $A$。因此\autoref{def_MinFul_1} 定义的 $p(x)$ 是非负有限的(由于是取下确界)。现在来证明正齐次性。

\textbf{正齐次性:}若 $t>0$,则
\begin{equation}
\begin{aligned}
p_A(tx)=&\inf \{r|tx/r\in A,r>0\}=\inf \{r|x/(r/t)\in A,r>0\}\\
=&
\end{aligned}~
\end{equation}





\textbf{证毕!}

\begin{theorem}{}
若 $p(x)$ 是线性空间 $L$ 上任意齐次凸非负泛函,$k$ 是正数,则
\begin{equation}\label{eq_MinFul_2}
A=\{x:p(x)\leq k\}~
\end{equation}
是凸体,其核是包含 $0$ 的集 $\{x:p(x)<k\}$。如果\autoref{eq_MinFul_2} 中 $k=1$,则原泛函 $p(x)$ 是 $A$ 的Minkowski泛函。
\end{theorem}

\textbf{证明:}




\textbf{证毕!}
