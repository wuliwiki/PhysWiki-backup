% 数字与函数回顾(高中)
% keys 初中|函数|正比例|反比例|二次|实数|坐标系
% license Xiao
% type Tutor

\begin{issues}
\issueDraft
\end{issues}

在初中阶段,函数的概念初步展示了变量之间的关系。初中接触的函数主要包括正比例函数、反比例函数和一次函数、二次函数。接下来会在介绍实数和坐标系的概念后,逐一回顾每种函数的特性和相关概念。

\subsection{实数与坐标系}

函数的基础是\textbf{数字(number)},而函数的图像则依赖于坐标系的表现。在介绍函数前,下面会先从“运算”的视角来重新审视所学过的数字,并介绍坐标系的相关概念。

最简单也最广为人知的数字,莫过于\textbf{自然数(natural number)}。自然数是每个人开对数字最开始的概念,自然数指的是从$0$开始\footnote{也有领域认为数字从$1$开始。}一个接一个地排列的数。这个排列的过程,也称作\textbf{递增(increment)}或\textbf{后继(successor)}。在后继的基础上,人们抽象出了加法运算,又从加法抽象出了乘法运算。而自然数对于加法和乘法都是\textbf{封闭的(closed)},也就是说任意两个自然数的和或积都是自然数。

随着对数字的进一步需求,会遇到表示“少于零”的情况,比如温度计上的零度以下温度,银行账户的负债等。这就引入了负数的概念。此时,减法也可视作与后者相反数的和。\textbf{整数(integer)}包括所有自然数和负数,也使得它不仅对加法和乘法封闭,对减法也封闭。这是一次数的扩充。

整数虽然很有用,但当需要表示更精确的数量,比如半个苹果或三分之一米时,仅用整数就不够了。这时,引入了分数\footnote{小数是一种分数的表示方法,尽管在小学就已经学习了,但小数本身很复杂,此处不讨论。}。此时,除法也可视作与后者倒数的积。当然,定义也要求$0$没有倒数\footnote{$0$有倒数也不是不可以,但是引入之后造成的麻烦会很大,因此,数学领域放弃了这个设定。},后面的讨论涉及到除法时,也均不包含$0$作除数的情况。\textbf{有理数(rational number)}\footnote{“有理数”这个词的翻译本身是有问题的,译者当初可能混淆了“合理”和“可比”的词义。更贴切的译法应是“可比数”,也符合表示成两数之比的含义。但由于此翻译已经广为使用,难以纠正。}包括所有整数和分数。而这时,有理数也可以统一表示成两个整数$m,n$之比$\displaystyle\frac{n}{m}\quad(m \neq 0)$。此时,数字又一次扩张,有理数不仅对加法、乘法和减法封闭,对除法也封闭。

随着数学的发展,要描述的事物逐渐增多,人们逐渐意识到,有些数其实无法用有理数来表示。例如,正方形的对角线长度和圆的周长与直径的比值。最初,数学家们期望这些数可以用有理数的形式来表达,即写成两个整数的比值。然而,深入研究后发现,这种方法只能得到近似值,而无法精确表示这些数的真实值。这一发现可以追溯到公元前400年,但直到17世纪,人们才普遍接受了这一事实。至于对这类数的明确定义,直到19世纪末才得以完成\footnote{当然,既然这件事是这么晚才搞清楚,在高中阶段一定不会涉及它的精确定义。}。最终,数学家们借助“极限(limit)”运算定义了\textbf{实数(real number)}。由于高中阶段不涉及极限的具体内容,仅需知道实数包含两类数:一类是有理数,另一类是无法用整数比表示的无理数。无理数对之前提到的加、减、乘、除运算都不封闭,但实数在这些运算下是封闭的。另外,实数还对极限运算封闭,一特性称为实数的\textbf{完备性(completeness)}\footnote{有些人会认为有些极限是无穷,而无穷不是实这数。但其实,极限为无穷只是一种极限不存在的情况。}。

除了上面的性质,实数还具备两个重要的性质:顺序性和稠密性。\textbf{顺序性(orderliness)}指的是任意两个实数都可以进行大小比较,因此实数形成了一个有序的数集。而\textbf{稠密性(density)}则意味着在任意两个实数之间,总能找到另一个实数。与此形成对比的是,后面会学习到的复数并不具备顺序性,而前面介绍的有理数和整数则不具备稠密性。稠密性使得实数能够与直线上的所有点建立一一对应关系,构成\textbf{数轴(number line)}。数轴的定义基于一个确定的原点、单位长度和正方向,这三个因素唯一地确定了数轴在几何中的位置和方向。法国数学家勒内·笛卡尔(René Descartes)在数学研究中,将两条数轴的原点重叠,并将其正交(即相互垂直)放置,创造了\textbf{坐标系(coordinate system)}。这就是初中阶段学习过的\textbf{笛卡尔坐标系(Cartesian coordinate system)},也称为\textbf{直角坐标系(rectangular coordinate system)}。

引入坐标系后,平面上的任何一点都可以通过一个\enref{有序数对}{CartPr} $(x, y)$ 来表示。借助这种表示法,几何形状可以通过数对来分析和研究,这一方式称为\enref{解析几何}{JXJH}。而当数对中的值对应于函数的变量及其结果时,几何图形就成为了函数的图像。因此,坐标系不仅为函数的图像提供了清晰的视觉表达,还使得人们可以通过几何图形直观地观察函数的性质,例如其变化趋势、最大值和最小值等。

通常,直角坐标系中,两条数轴称为$x$轴和$y$轴,且向右的方向为$x$轴的正方向,向上为$y$轴的正方向。数轴将平面分为四个区域,称为\textbf{象限(quadrant)}。其中,第一象限是两个坐标都为正的区域,之后按逆时针方向依次为第二、第三和第四象限。

\subsection{正比例函数与一次函数}

\begin{definition}{正比例函数}
形如
\begin{equation}
y = kx\quad(k\neq0)~.
\end{equation}
的函数称为\textbf{正比例函数(proportional function)}。
\end{definition}

这里只要求$k\neq0$,称为正比例是因为函数与自变量的关系形式,而非参数的正负。正比例函数的图像是一条经过原点的直线,且正比例函数的图像总是一条直线,与$k$的取值无关\footnote{有些函数的形状是与参数的取值有关的。}。由于 $x=0$ 时,$y$ 也必然为 $0$,所以图像一定会穿过原点。

参数$k$控制了直线的倾斜程度,或者说它量化了正比例函数的倾斜程度。这种对倾斜程度量化的值被抽象出来,称为斜率。

\begin{definition}{斜率}
对平面上相异两点,以两点坐标差之比,或两点连线与坐标轴夹角的正切,来表示两点所在直线关于某坐标轴(通常是$x$轴)倾斜程度的量,称为\textbf{斜率(Slope)},通常记作$k$或$m$,即:
\begin{equation}
k = \frac{\Delta y}{\Delta x}={y_2-y_1\over x_2-x_1}~.
\end{equation}
或
\begin{equation}
k =\tan\theta~.
\end{equation}
其中,$(x_1,y_1),(x_2,y_2)$为相异两点的坐标,$\theta$为直线与坐标轴夹角,称作\textbf{倾角(angle of inclination)}。
\end{definition}

从定义可以看出,斜率表示的是每单位的 $x$ 增加带来 $y$ 的变化。当 $k$ 为正时,表示直线向上倾斜,即$y$ 随着 $x$ 增加而增加;反之,当 $k$ 为负时,直线向下倾斜,$y$ 随着 $x$ 增加而减少。$|k|$ 决定了直线的倾斜程度。$|k|$ 越大,直线越陡峭, $x$ 的变化带来的$y$ 的变化越大;反之,$|k|$ 越小,直线越平缓,$x$ 的变化带来的$y$ 的变化越小。

在未来的学习中,\textbf斜率的概念将被推广为\textbf{梯度(gradient)},用于描述多维空间中函数的变化速率。

正比例函数相当于描述了三点要求:
\begin{enumerate}
\item 函数的图像是直线;
\item 函数要过原点;
\item 函数的斜率为$k$。
\end{enumerate}
如果打开第二个要求,或者说放松第二个要求,则得到了一次函数。

\begin{definition}{一次函数}
形如
\begin{equation}
y = kx+b\quad(k\neq0)~.
\end{equation}
的函数称为\textbf{一次函数(linear function,或线性函数)}。
\end{definition}

一次函数的图像是一条直线,它也是最常见的线性关系。参数$b$ 决定了直线与 $y$ 轴的交点,也可以认为它表现了变量$y$在没有变量$x$的影响时的默认状态,这种默认状态被称为截距。从图像和表达式可知,正比例函数是特殊的一次函数或特殊的线性关系,特殊性就体现在$b$的取值上。

\begin{definition}{截距}
截距就是直线与坐标轴相交的点所对应的坐标值。对于函数而言,一般y轴的截距简称为截距,而x轴的截距则称为零点。
\end{definition}

换句话说,它就是直线在X轴或Y轴上“停留”的地方。

截距的作用包括:

理解直线的位置: 截距可以快速判断直线在坐标系中的位置。

求解直线的方程: 一些常用的直线方程形式,比如斜截式,就利用了截距的信息。通过知道截距,我们可以方便地写出直线的方程。

初始条件:对于很多函数而言,$x$只能是非负数,因此,$x=0$所对应的函数值,也就是截距,意味着函数的初始条件。

某些模型,虽然$x=0$时没有意义,但截距项仍然是模型中的非常重要的一部分。假如模型中没有截距项,斜率系数很可能是一个错误的估计值,相当于强制要求回归线通过原点。


\subsection{反比例函数}

、\textbf{反比例函数(inversely proportional function)}
对应的是两条双曲线。你还学过如何根据反比例函数的表达式,通过已知的点来求解函数的值。

中心对称性

矩形面积相同

反比例函数描述了一种“此增彼减”的关系。

定义与表达式:反比例函数的标准形式为  y = \frac{k}{x} ,其中 k 是常数。当 x 增加时,y 减少;当 x 减少时,y 增加。这意味着 x 和 y 之间存在一种反向变化的关系。

图像特点:反比例函数的图像是一条双曲线,分别位于坐标轴的两侧,并且关于原点对称。这种对称性表明,x 值变号时,y 值也会相应地变号。

矩形面积不变性:反比例函数的一个特殊性质是,对于任意点 (x, y),x 和 y 的乘积总是等于常数 k。这类似于矩形的面积总是等于长乘以宽,即使长宽发生变化,只要乘积不变,面积就保持不变。


\subsection{二次函数}

\textbf{二次函数(quadratic function)}的图像可能与  x  轴有两个交点,并且具有一个对称轴和一个最低点。

轴对称性:图像上对称点到对称轴的距离相等,且连线与对称轴垂直。

关于二次函数和一元二次方程的关系参见\enref{因式分解与一元二次方程}{quasol}。

二次函数描述了一种曲线关系,常用于表示自然界中的抛物运动。

定义与标准形式:二次函数的标准形式为  y = ax^2 + bx + c ,其中 a, b, c 是常数。a 的正负决定了抛物线的开口方向:当 a 为正,开口向上;当 a 为负,开口向下。

图像特点:二次函数的图像是一条抛物线,具有明显的对称性。抛物线的对称轴可以帮助找到对称的两个点以及最高点或最低点。其顶点是抛物线的最高点(当 a < 0)或最低点(当 a > 0)。

二次函数与一元二次方程的关系:二次函数图像与 x 轴的交点是方程的解,即求得的 x 值可以使得 y 为零。这在很多实际问题中帮助找到最优值或临界点。
