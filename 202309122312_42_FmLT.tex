% 费马小定理
% keys 费马小定理
% license Xiao
% type Tutor
\pentry{剩余类环\upref{RRing},整环\upref{Domain},域}
费马小定理是法国律师费马于1636年发现的,其由欧拉在1736年出版的名为“一些与素数有关的定理的证明”的论文集中第一次给出证明。其描述如下
\begin{theorem}{费马小定理}
若 $p$ 是素数,$m$ 是一个不能被 $p$ 整除的整数,则有同余式(\autoref{def_RRing_1}~\upref{RRing})
\begin{equation}
m^{p-1}\equiv 1(p)~.
\end{equation}
\end{theorem}
翻译成自然语言,就是说若整数 $m$ 不能被素数 $p$ 整除,则 $p$ 除 $m^{p-1}$ 余数为1。本文将用群论的方法证明该定理。这依赖于这样一个引理:
\begin{lemma}{}\label{lem_FmLT_1}
即若 $p$ 是素数,则剩余类环\upref{RRing} $\mathbb Z_p$ 是个域(\autoref{def_field_4}~\upref{field})。
\end{lemma}
\subsection{\autoref{lem_FmLT_1} 的证明}

