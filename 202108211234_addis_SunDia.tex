% 日晷的计算

\subsection{水平日晷}
常见的水平日晷如\autoref{SunDia_fig2} 和\autoref{SunDia_fig1} 所示. 把图片打印后, 把直角三角形的底边垂直固定在半圆的 12 点方向, 然后把半圆水平放于地面, 12 点方向指向正北(地轴北)即可. 此时三角形的斜边, 也就是日晷的指针与地轴平行. 太阳绕日晷指针以每小时 15° (一天 360°)的角速度匀速转动.

\begin{figure}[ht]
\centering
\includegraphics[width=8cm]{./figures/SunDia_2.png}
\caption{水平日晷} \label{SunDia_fig2}
\end{figure}
该类型日晷的计算网站见\href{https://www.blocklayer.com/sundial.aspx}{这里}.
\begin{figure}[ht]
\centering
\includegraphics[width=10cm]{./figures/SunDia_1.png}
\caption{北纬 30° 的日晷} \label{SunDia_fig1}
\end{figure}

日晷圆盘上的刻度与所在纬度有关, 若将其放在北极或南极, 那么直角三角形的就变为一个和地面垂直的线段, 且表盘上的刻度是均匀的. 相反, 若把这种日晷放在赤道上, 那么三角形的斜边将会与 12 点的刻度共线, 此时这种日晷将失效.

刻度的具体计算并不复杂, 公式为
\begin{equation}
\beta = \tan^{-1}(\sin\alpha \tan \theta)
\end{equation}
其中 $\beta$ 是表盘上某个刻度到 12 点刻度的夹角, $\theta$ 是太阳当前位置和 12 点位置关于地轴(指针)的夹角, $\alpha$ 是当地维度. 可以验证在北极点处 $\alpha = \pi/2$ 时 $\beta = \phi$.

证明: 令\autoref{PrmSol_eq3}~\upref{PrmSol} 中 $\theta_2 = \pi/2$, $\theta_1 = \theta$ 即可.

我们给出水平日晷的 Matlab 画图\upref{MatPlt}代码, 结果类似\autoref{SunDia_fig1}.
\begin{lstlisting}[language=matlab, caption=sunDial.m]
alpha = pi/6; % 纬度
th = linspace(-pi/2, pi/2, 13);
beta = atan(sin(alpha).*tan(th));
th0 = linspace(-pi/2, pi/2, 1000);
figure; plot(cos(th0),sin(th0), 'k'); hold on;
for i = 1:13
    plot([0,cos(beta(i))], [0,sin(beta(i))], 'k');
end
axis equal; view(-90, 90);
axis([-0.1, 1.1, -1.1, 1.1]);
\end{lstlisting}

\subsection{赤道平面日晷}
另一种在任何纬度都适用的方案是让刻度盘始终与指针保持垂直, 而指针始终指向地轴. 这种日晷的结构相对更复杂, 但表盘上的刻度始终是均匀的, 秩序调整表盘与底座之间的夹角就能适用于不同纬度.
