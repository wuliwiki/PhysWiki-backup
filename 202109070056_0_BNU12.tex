% 北京师范大学 2012 年考研普通物理
% 北师大|考研|普通物理

1.质量为 $M$ 的滑块沿着光滑水平地面向右滑动, 一质量为 $\mathrm{m}$ 的小球水平向右飞行, 以速度 $\bvec{v_1}$ (对地)与滑块沿斜面相碰, 碰后竖直向上弹起, 速率为
$v_2$ (对地), 若碰撞事间为 $\Delta \mathrm{t}$, 试计算此 过程中滑块对地的平均作用力和滑块速
度增量的大小.
\begin{figure}[ht]
\centering
\includegraphics[width=5cm]{./figures/BNU12_1.pdf}
\caption{第1题图示} \label{BNU12_fig1}
\end{figure}

2.有一宽为 $I$ 的大河,河水由北向南流去.设河中心流速为 $\mathrm{u}_{0}$, 靠两岸的流 速为零.河中任意点的流速与河中心流速之差正比于河心至该点距离的平方.今有相 对于水的速度为 $\bvec{v_0}$ 的小船由西岸出发, 向东偏北 $45^{\circ}$ 方向航行,试求其航线的轨迹方 程以及到达东岸的地点.

3.一个水平转台可绕竖直固定轴线作光滑转动, 人站立在转台上, 身体的中 心轴线与转台竖直轴线重合, 两臂伸开各举一个哑铃, 当转台转动时,此人把两哑铃 水平地收缩到胸前, 在这一收缩过程中\\
(1)转台,人与哑铃以及地球组成的系统机械能是否守恒? 为什么?\\
(2) 转台, 人与哑铃组成的系统相对于转台竖直轴线角动量是否守恒? 为什么?
(3)每个哑铃的动量与动能是否守恒?为什么?

4.火箭以 $v_1$ 的速率相对于地球匀速运动, 火箭中的观察者测得火箭长度为$L$,从火箭尾部向火箭前端的靶发射一颗相对火箭以 $v_{2}$ 的速率匀速运动的高速子弹. 求: \\(1) 火箭中的观察者测得的子弹击中靶所用的时间, \\(2)地球上的观察者测得子弹的速度, \\(3)地球上的观察者测得的子弹击中靶所用的时间.

5.辨析题:请判断下列说法的正误, 并说明判断的理由.\\
(1)产生毛细现象和浸润现象的物理机制是完全不同的.\\
(2) 根据热力学第二定律, 制冷机的制冷系数一定小于 $1$\\
(3)空调的制热循环与制冷循环分别是正循环和逆頍环.\\
(4) 根据热力学第二定律,热机的工作效率一定小于 1.

6.如图, 1 摩尔非刚性双原子分子理想气体作两个等压过程与两个等容过程构成的 $\mathrm{A} \rightarrow \mathrm{D} \rightarrow \mathrm{C} \rightarrow \mathrm{B} \rightarrow \mathrm{A}$
制冷循环, $\mathrm{p}_{1}, \mathrm{p}_{2}, \mathrm{~V}_{1}, \mathrm{~V}_{2}$ 已知, 请推导制冷系数 $\mathrm{w}$ 的表达式.
\begin{figure}[ht]
\centering
\includegraphics[width=5cm]{./figures/BNU12_2.pdf}
\caption{第6题图示} \label{BNU12_fig2}
\end{figure}

7.电磁场的规律可以由麦克斯韦方程组描述, 请写出麦克斯韦方程组对应的 四个方程式.

8.相对介电常数为 $\varepsilon_{x}$ 的均匀电介质内有一球形空腔(腔内为真空), 介质 中有均匀电场 $\mathrm{E}_{0}$, 求球面上的极化电荷面密度在球心处激发的电场强度 $\mathrm{E}_{\circ}$
\begin{figure}[ht]
\centering
\includegraphics[width=4cm]{./figures/BNU12_3.pdf}
\caption{第8题图示} \label{BNU12_fig3}
\end{figure}

9.一个半径为 $\mathrm{R}$ 的均匀带点圆盘, 中心为 $0, X$ 轴过中心垂直圆盘, 已知电荷 密度为 $\sigma(>0)$
(1) 求 $X$ 轴线上距圆盘中心 0 距离为 $x$ 处的电场强度 $\mathrm{E} $\\
(2) 圆盘以角速度 $\omega$ 绕 $\mathrm{X}$ 轴作匀速转动, 求圆盘中心 0 处的磁场 B;\\
(3) 圆盘以角速度 $\omega$ 绕 $X$ 轴作匀速转动, 求该带点圆盘的磁矩 $M$ 的大小.
\begin{figure}[ht]
\centering
\includegraphics[width=5cm]{./figures/BNU12_4.pdf}
\caption{第9题图示} \label{BNU12_fig4}
\end{figure}

10.半径为 $\mathrm{R}$ 的无限长直螺线管, 单位长度上绕 $\mathrm{n}$ 匝线圈, 当通入的电流均匀
增加 $\left(\frac{\mathrm{dB}}{\mathrm{dt}}=\mathrm{c}\right.$ ) 时, 有一 $2 \mathrm{R}$ 长的金属棒如图放置在磁场中( $\mathrm{AB}=\mathrm{BC}=\mathrm{R}$ ), 求金属棒的感
应电动势大小.
\begin{figure}[ht]
\centering
\includegraphics[width=4.5cm]{./figures/BNU12_5.pdf}
\caption{第10题图示} \label{BNU12_fig5}
\end{figure}