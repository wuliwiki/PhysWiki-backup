% 约翰·伯努利(综述)
% license CCBYSA3
% type Wiki

本文根据 CC-BY-SA 协议转载翻译自维基百科\href{https://en.wikipedia.org/wiki/Johann_Bernoulli}{相关文章}。

\begin{figure}[ht]
\centering
\includegraphics[width=6cm]{./figures/040c46e497ce024e.png}
\caption{约翰·伯努利(约翰·鲁道夫·胡贝尔绘制的肖像,约1740年)} \label{fig_YHBNL_1}
\end{figure}
\textbf{约翰·伯努利}(Johann Bernoulli,也称为法语中的Jean或英语中的John;1667年8月6日(旧历7月27日)-1748年1月1日)是一位瑞士数学家,是伯努利家族中许多杰出数学家之一。他因在微积分学上的贡献以及在莱昂哈德·欧拉少年时期的教育而闻名。
\subsection{传记}  
\subsubsection{早期生活}  
约翰·伯努利出生于巴塞尔,父亲是药剂师尼古拉斯·伯努利,母亲是玛格丽特·施翁高尔。他在巴塞尔大学开始学习医学。他的父亲希望他学习商业,以便能够接管家族的香料贸易,但约翰·伯努利并不喜欢商业,最终说服父亲允许他改学医学。约翰·伯努利开始在课余时间与他的哥哥雅各布·伯努利一起学习数学。[5] 在巴塞尔大学的学习过程中,伯努利兄弟共同合作,花费大量时间研究新发现的微积分。他们是最早不仅学习和理解微积分,而且将其应用于各种问题的数学家之一。[6] 1690年,[7] 他完成了医学学位论文,[8] 由莱布尼茨审阅,[7] 论文题为《肌肉运动论与发酵与发泡现象的研究》[9]。
\subsubsection{成年生活} 
从巴塞尔大学毕业后,约翰·伯努利开始教授微分方程。后来,在1694年,他娶了巴塞尔市议员的女儿多萝西亚·法尔克纳,并很快接受了格罗宁根大学数学教授的职位。应岳父的要求,伯努利于1705年开始返回家乡巴塞尔的旅程。就在出发后不久,他得知哥哥因肺结核去世。伯努利原本打算回到巴塞尔大学担任希腊语教授,但他最终接管了哥哥的数学教授职位。作为莱布尼茨微积分学派的学生,伯努利在1713年支持莱布尼茨,在莱布尼茨与牛顿关于谁应为微积分的发现负责的争论中站在了莱布尼茨一方。伯努利通过展示莱布尼茨用其方法解决了牛顿未能解决的某些问题来为莱布尼茨辩护。伯努利还推崇笛卡尔的漩涡理论,而非牛顿的万有引力理论。这最终延迟了牛顿理论在欧洲大陆的接受。[10]

1724年,约翰·伯努利参加了由法国皇家科学学院举办的一个竞赛,竞赛题目是:

“一个完美坚硬的物体,在运动中,按照什么规律能使另一个同类物体(无论是静止还是运动)发生移动,而该物体可能在真空或充满介质的环境中与之相遇?”

在辩护莱布尼茨之前提出的观点时,他设想了一个无限的外部力,要求克服使物体坚硬的无限内部力,从而使物体具有弹性。因此,他被取消了竞赛资格,奖项最终被麦克劳林(Maclaurin)获得。然而,伯努利的论文在1726年被接受,当时法国皇家科学学院考虑了关于弹性体的论文,并授予皮埃尔·马齐埃尔(Pierre Mazière)该奖项。伯努利在这两个竞赛中都获得了荣誉提名。
\subsubsection{争议与纷争}
尽管约翰和他的兄弟雅各布·伯努利在约翰毕业于巴塞尔大学之前曾一起工作,但毕业后不久,两人之间便发展出一种嫉妒和竞争的关系。约翰嫉妒雅各布的地位,两人经常互相竞争,力图超越对方。在雅各布去世后,约翰的嫉妒心转向了自己才华横溢的儿子丹尼尔。1738年,父子二人几乎同时分别发表了关于流体力学的独立著作。约翰试图通过故意并虚假地将自己的工作发布日期提前到比儿子早两年,以此来让自己在学术上居于领先地位。[11][12]
\begin{figure}[ht] 
\centering
\includegraphics[width=6cm]{./figures/a2c425c7631019db.png}
\caption{《哲学与数学通信集》(1745年),是莱布尼茨与伯努利之间信件的合集。} \label{fig_YHBNL_2}
\end{figure}
伯努利兄弟经常共同研究同一个问题,但并非没有摩擦。他们之间最激烈的争执涉及到最速降线问题,或者说,关于在仅受重力作用下,一个粒子从一点到另一点所遵循的路径方程,即所用时间最短的路径。约翰在1696年提出了这个问题,并为解决方案提供了奖励。为了解决这一挑战,约翰提出了圆弧线(即,轮子上某点的轨迹),同时指出该曲线与光线通过不同密度层时所采取路径之间的关系。雅各布提出了相同的解决方案,但约翰的推导是错误的,他将兄弟雅各布的推导当作自己的成果展示出来。[13]

伯努利被吉约姆·德·洛皮塔尔雇用,负责数学辅导。伯努利与洛皮塔尔签订了合同,授予洛皮塔尔使用伯努利发现的权利。洛皮塔尔于1696年出版了《无限小分析与曲线理解》(Analyse des Infiniment Petits pour l'Intelligence des Lignes Courbes),这是第一本关于微积分的教科书,主要内容是伯努利的工作,包括现在被称为“洛皮塔尔法则”的部分。[14][15][16] 随后,在给莱布尼茨、瓦里农和其他人的信中,伯努利抱怨自己没有得到应有的贡献认可,尽管他书中的前言写道:

“我承认我欠伯努利先生们,特别是年轻的(约翰)教授,很多,当前他是格罗宁根的教授。我确实无所顾忌地使用了他们的发现,也使用了莱布尼茨先生的成果。正因如此,我同意他们可以尽可能地索取荣誉,而我将满足于他们同意留给我的部分。”
\subsection{著作}  
\begin{figure}[ht]
\centering
\includegraphics[width=6cm]{./figures/dfb34eb1d04b981c.png}
\caption{《De motu corporum gravium》插图,发表于《Acta Eruditorum》,1713年。} \label{fig_YHBNL_3}
\end{figure}
\begin{itemize}
\item 《De motu musculorum》(拉丁文)。威尼斯:Giovanni Antonio Pinelli & Almoro Pinelli,1721年。  
\item 《Recherches physiques et géométriques sur la question comment se fait la propagation de la lumière》(法文)。巴黎:Imprimerie Royale,1736年。  
\item 《[Opere]》(法文),第1卷。洛桑:Marc Michel Bousquet & C.,1742年。  
\item 《[Opere]》(法文),第2卷。洛桑:Marc Michel Bousquet & C.,1742年。  
\item 《[Opere]》(法文),第3卷。洛桑:Marc Michel Bousquet & C.,1742年。  
\item 《[Opere]》(法文),第4卷。洛桑:Marc Michel Bousquet & C.,1742年。  
\item 伯努利,约翰(1786)。《Analyse de l'Opus Palatinum de Rheticus et du Thesaurus mathematicus de Pitiscus》(法文)。巴黎:sn。于2015年6月18日检索。  
\item 伯努利,约翰(1739)。《Dissertatio de ancoris》(拉丁文)。莱比锡:sn。于2018年6月20日检索。
\end{itemize}
\begin{figure}[ht]
\centering
\includegraphics[width=6cm]{./figures/c2d56fe3890ceb19.png}
\caption{伯努利《1742年全才集》第一至四卷} \label{fig_YHBNL_4}
\end{figure}
\begin{figure}[ht]
\centering
\includegraphics[width=6cm]{./figures/c0a38a374598a70f.png}
\caption{《1742年全才集》第一卷中的伯努利肖像} \label{fig_YHBNL_5}
\end{figure}
\begin{figure}[ht]
\centering
\includegraphics[width=6cm]{./figures/dbca1984baec3f1b.png}
\caption{《全才集》第一卷的封面} \label{fig_YHBNL_6}
\end{figure}
\subsection{另见}  
\begin{itemize}
\item 大二生的梦 —— 伯努利的两个分析恒等式  
\item 部分分式分解
\end{itemize}  
\subsection{注释}  
a.英语发音:/bɜːrˈnuːli/ bur-NOO-lee;瑞士标准德语发音:[ˈjoːhan bɛrˈnʊli]
\subsection{参考文献}  
\begin{enumerate}
\item Bernoulli, Johannes (1690). *Dissertatio de effervescentia et fermentatione nova hypothesi fundata*. Switzerland: Basileae, Typis Iacobi Bertschii. doi:10.3931/e-rara-16316. 2018年8月14日检索。  
\item 发表于1690年,提交于1694年。  
\item Wells, John C. (2008). *Longman Pronunciation Dictionary* (第3版). Longman. ISBN 978-1-4058-8118-0.  
\item Mangold, Max (1990). *Duden — Das Aussprachewörterbuch*. 第3版. Mannheim/Wien/Zürich, Dudenverlag.  
\item Sanford, Vera (2008) [1958]. *A Short History of Mathematics* (第2版). Read Books. ISBN 978-1-4097-2710-1. OCLC 607532308.  
\item *The Bernoulli Family*, by H. Bernhard, Doubleday, Page & Company, (1938).  
\item Bernoulli, Johan; Paul G. J. Maquet; August Ziggelaar (1997). *Dissertatio de Effervescent Et Fermentatione*. *Transactions of the American Philosophical Society*. Vol. 87 (Part 3). American Philosophical Society. pp. 5–6. doi:10.2307/1006610. ISBN 9780871698735. ISSN 0065-9746. JSTOR 1006610. OCLC 185537598. 2021年7月16日检索。  
\item Smith, David Eugene (1917年7月1日). "Medicine and Mathematics in the Sixteenth Century". *Ann. Med. Hist.* 1 (2): 125–140. OCLC 12650954. PMC 7927718. PMID 33943138. (引用第133页)。  
\item *De mote musculorum, de effervescentia et fermentatione dissertations physico-mechanicae*: Account Petri Antoni Michelotti. Pinelli. 1721. OCLC 433236093. 2021年7月16日检索。  
\item Fleckenstein, Joachim O. (1977) [1949]. *Johann und Jakob Bernoulli* (德语) (第2版). Birkhäuser. ISBN 3764308486. OCLC 4062356.  
\item Darrigol, Olivier (2005年9月). *Worlds of Flow: A History of Hydrodynamics from the Bernoullis to Prandtl*. OUP Oxford. 第9页. ISBN 9780198568438.  
\item Speiser, David; Williams, Kim (2008年9月18日). *Discovering the Principles of Mechanics 1600-1800: Essays by David Speiser*. Springer. ISBN 9783764385644.  
\item Livio, Mario (2003) [2002]. *The Golden Ratio: The Story of Phi, the World's Most Astonishing Number* (第一版平装). New York City: Broadway Books. 第116页. ISBN 0-7679-0816-3.  
\item Maor, Eli (1998). *e: The Story of a Number*. Princeton University Press. 第116页. ISBN 0-691-05854-7. OCLC 29310868.  
\item Coolidge, Julian Lowell (1990) [1963]. *The mathematics of great amateurs* (第2版). Oxford: Clarendon Press. 第154–163页. ISBN 0-19-853939-8. OCLC 20418646.  
\item Struik, D. J. (1969). *A Source Book in Mathematics: 1200–1800*. Harvard University Press. 第312–316页. ISBN 978-0-674-82355-6.
\end{enumerate}
\subsection{外部链接}  
\begin{itemize}
\item 与Johann Bernoulli相关的媒体,见Wikimedia Commons  
\item Johann Bernoulli在数学家谱项目中的资料  
\item O'Connor, John J.; Robertson, Edmund F., "Johann Bernoulli", *MacTutor History of Mathematics Archive*, 圣安德鲁斯大学  
\item Golba, Paul, "Bernoulli, Johan"  
\item "Johann Bernoulli"  
\item Weisstein, Eric Wolfgang (编). "Bernoulli, Johann (1667–1748)",*ScienceWorld*  
\item Truesdell, C. (1958年3月). "The New Bernoulli Edition". *Isis*. 49 (1): 54–62. doi:10.1086/348639. JSTOR 226604. S2CID 143648596. 讨论了Bernoulli与de l'Hôpital之间的奇怪协议,见第59–62页。
\end{itemize}