% 矩阵的秩
% 矩阵|秩|满秩矩阵|方阵

\pentry{矩阵\upref{Mat}, 行列式\upref{Deter}}

我们定义矩阵的\textbf{列秩}等于其线性无关的列数, \textbf{行秩}等于线性无关的行数. 可以证明\upref{RCrank}, 对于任意尺寸的矩阵, 二者是相同的, 所以简称为矩阵的\textbf{秩}.

根据定义, 一个矩阵的秩必定小于或等于矩阵的行数以及列数(取较小者). 对于方阵, 若三者相等, 我们就称其为\textbf{满秩矩阵}. 判断满秩矩阵的一种常用方法时计算矩阵的行列式, 若结果不为零, 则矩阵是满秩的, 否则不是(\autoref{Deter_the2}\upref{Deter}). 注意非满秩的情况下行列式并不能判断秩具体是多少.

% 补充一道例题

% 未完成: 用高斯消元法判断秩的大小
\subsection{高斯消元法计算秩}
\pentry{高斯消元法\upref{GAUSS}}
要确定任意矩阵秩的大小, 我们可以先用高斯消元法将矩阵变换为梯形矩阵. 矩阵的秩数就是梯形矩阵中不为零的行数. 这是因为, 行变换并会不会改变秩, 而梯形矩阵中不为零的行都是线性无关的.
