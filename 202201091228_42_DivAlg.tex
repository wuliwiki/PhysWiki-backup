% 带余除法
% 余式|商式

\pentry{一元多项式\upref{OnePol}}
在一元多项式\upref{OnePol}的最后提到,系数在数域 $\mathbb{F}$ 上的全体一元多项式构成一元多项式环 $\mathbb{F}[x]$.在环 $\mathbb{F}[x]$ 中,可以做加、减、乘三种运算,运算的结果还是该多项式环 $\mathbb{F}[x]$ 中的元素(一元多项式).自然,我们会像在整数里那样,考虑除法运算.你将看到,和整数一样,在一元多项式里,任意两个多项式做除法运算的结果并不一定还是一个多项式,取而代之的是带余除法.
\begin{theorem}{带余除法}
设 $f(x)$ 与 $g(x)$ 为 $\mathbb{F}[x]$ 中的两个多项式,并且 $g(x)\neq 0$,则存在唯一的 $\mathbb{F}[x]$ 中的多项式 $q(x),r(x)$,使得
\begin{equation}
f(x)=q(x)g(x)+r(x)
\end{equation}
其中 $\mathbb{deg}r(x)<\mathbb{deg}g(x)$, $q(x),r(x)$ 分别称为 $g(x)$ 除 $f(x)$ 的\textbf{商式}和\textbf{除式}.并将这种算法称为\textbf{带余除法},有时称为\textbf{长除法}.
\end{theorem}
\subsection{证明}