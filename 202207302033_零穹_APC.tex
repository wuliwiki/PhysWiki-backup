% 布置、排列、组合
% 布置|排列|组合

\begin{issues}
\issueDraft
\end{issues}

\pentry{映射\upref{map},计数原理\upref{ProfPM}}
组合学的一个重要方面在与计数问题.实际上,在很长一段时间里,大多数数学工作者把组合学与“计数”当作一回事.“计数”往往是指找出进行某个确定的运算方法的个数,与之紧密联系的内容便是排列组合.

“排列”和“组合”这两个术语在高中部分已经熟悉.本节将给予这些概念于一个更完整、更直观的描述,这意味着要建立一些术语,而这些术语给出了这种直观的意义.排列组合在很多方面都有应用,本节的术语将使得这种应用更加直观.
\subsection{布置Arrangement}
排列组合的很多应用在于将一定数量的“物体”放置在一定数量的“房间”(盒子)中,下面的定义使得这种应用更加明显.
\begin{definition}{物体、房间、布置}\label{APC_def2}
设 $X,Y$ 是有限集,把 $X$ 的元素叫做\textbf{物体},把 $Y$ 的元素叫\textbf{房间}.映射 $f:X\rightarrow Y$ 称为\textbf{物体集合 $X$ 到房间集合 $Y$ 的一种\textbf{布置}(arrangement)}.
\end{definition}
物体集合 $X$ 到房间集合 $Y$ 的映射 $f:X\rightarrow Y$ 使得 $y_i$ 与 $X$ 的一些元素构成的集合
\begin{equation}
\{x|x\in X,f(x)=y_i\}
\end{equation}
相对应.称 $f$ 为布置暗示着 $f$ 将 $X$ 中的物体放置或分配到 $Y$ 中的房间内.

和很多命名一样,关于集合的元素是为了方便人们的记忆和应用.比如研究矢量的集合称为矢量空间,集合中的元素称为矢量;研究点集拓扑学的集合称为拓扑空间,其上的元素称为点.对应的,与组合数学对应的集合分别称为物体(定义域)和房间(值域),而物体到房间的映射称为布置.

当 $\abs{X}=n,\abs{Y}=m$ 时,每个函数 $f$ 和一个字符串“$f(x_1)\cdots f(x_n)$” 相对应.往往字符串代表着一个字,比如字 word 是由字母 “w”,“o”,“r”,“d” 构成的, 下面的定义给出了这种直观上的意义.
\begin{definition}{字母,字}\label{APC_def1}
设 $X,Y$ 分别是基数为 $\abs{X}=n,\abs{Y}=m$ 的有限集, $f:X\rightarrow Y$ 是 $X$ 到 $Y$ 上的映射,则称 $Y$ 的元素为\textbf{字母}, $f$ 是由 $Y$ 中的字母形成的长度为 $n$ 的一个\textbf{字} $f(x_1)\cdots f(x_n)$.
\end{definition}
上面定义中,$X$ 可看成给出了字的一个顺序.

通过这些概念,得到以下一些定理.

\begin{theorem}{}\label{APC_the1}
设 $X,Y$ 分别是基数为 $\abs{X}=n,\abs{Y}=m$ 的有限集,则映射 $f:X\rightarrow Y$ 的个数等于 $m^n$.
\end{theorem}
\textbf{证明:}由\autoref{APC_def1} ,$f$ 的个数等于由 $Y$ 中的字母拼成的长度为 $n$ 的字的个数.因为字的第一个字母有 $m$ 种选法,第二个,$\cdots$ ,第 $n$ 字母也有 $m$ 种选法.由乘法原则(\autoref{ProfPM_sub1}~\upref{ProfPM}),结果字的个数等于 $\underbrace{m\cdots m}_{n\text{个}}=m^n$.

\textbf{证毕!}

将上述定理换成\autoref{APC_def2} 的语言,得到:
\begin{theorem}{}
$n$ 个物体的集合在 $m$ 个房间的集合里的布置个数为 $m^n$.
\end{theorem}

\autoref{APC_the1} 就是不管集合 $X,Y$ 是否有限,所有的映射 $f:X\rightarrow Y$ 的集合常记作 $Y^X$ \upref{Topo8}的原因.

\begin{definition}{有序布置}
设物体集合 $X$ 被布置在房间集合 $Y$ 里,且每个房间可容纳 $X$ 中任意个物体,若改变房间中物体的顺序得到的布置不同,则称这样的布置叫作房间里的\textbf{有序布置(ordered arrangement)}.$n$ 个物体在 $m$ 个房间里有序布置的个数记作 $[m]^n$.
\end{definition}
\begin{theorem}{}
$n$ 个物体在 $m$ 个房间里有序布置的个数 $[m]^n$ 为
\begin{equation}
[m]^n=m(m+1)\cdots(m+n-1)
\end{equation}
\end{theorem}
\textbf{证明:}
记 $T_n$ 为 $n$ 个物体在 $m$ 个房间里的所有有序布置的集合. $T_n$ 中的每一有序布置都可以这样构造,先将前
\textbf{证毕!}
\subsection{排列}
\begin{theorem}{}
设 $X,Y$ 分别是基数为 $\abs{X}=n,\abs{Y}=m$ 的有限集,则单射 $f:X\rightarrow Y$ 的个数为 $m(m-1)\cdots(m-n+1)$.
\end{theorem}
\textbf{证明:}函数 $f$ 的单一性由字的字母是两两不同的这个事实表示.因为字的第一个字母有 $m$ 种选法,第二个字母有 $m-1$ 种选法,$\cdots$ ,第 $n$ 个字母有从剩下的 $m-n+1$ 个字母中有 $m-n+1$ 中不同选法.由乘法原则,由 $m$ 元集 $Y$ 的 $n$ 个不同字母构成的字的个数为 $m(m-1)\cdots(m-n+1)$.

\textbf{证毕!}

同样的,上述定理相当于:
\begin{theorem}{}\label{APC_the2}
将 $n$ 个物体的集合在 $m$ 个房间的集合里进行布置,要求每个房间至多包含一个物体的布置个数为 $m(m-1)\cdots(m-n+1)$.
\end{theorem}

\begin{definition}{排列,全排列}
从 $m$ 元集 $Y$ 的 $n$ 个不同字母构成的字叫作从集合 $Y$ 的 $m$ 个物体每次取 $n$ 个的\textbf{排列}(\textbf{permutation})(或 $m$ 元集 $Y$ 的 $n$ 排列),其个数记为 $[m]_n$ (或$P(m,n)$),其中 $m\geq n$.若 $m=n$,则 $[n]_n$ 也记作 $P_n$ 或 $n!=1\cdot2\cdots n$,此时的排列称 $n$元集 $Y$ 的\textbf{全排列}.
\end{definition}
由\autoref{APC_the2} 
\begin{equation}
P(m,n)=m(m-1)\cdots(m-n+1)
\end{equation}

