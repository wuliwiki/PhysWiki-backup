% 正交曲线坐标系
% keys 多元微积分|坐标系|柱坐标系|球坐标系|矢量|内积|内积|导数|偏导数|曲线坐标系|正交曲线坐标系

\pentry{柱坐标系\upref{Cylin}, 球坐标系\upref{Sph}, 矢量内积\upref{Dot}, 矢量的导数\ 求导法则\upref{DerV}}

\subsection{柱坐标系}
我们先来分析柱坐标系\footnote{由于极坐标系可以看做柱坐标系 $z = 0$ 的情况, 我们不单独讨论}, 位置矢量 $\uvec r$ 在直角坐标系中展开为
\begin{equation}\label{CurCor_eq1}
\bvec r(r, \theta, z) = r\cos\theta\, \uvec x + r\sin\theta\, \uvec y + z\uvec z
\end{equation}
柱坐标系中三个单位矢量 $\uvec r, \uvec \theta, \uvec z$ 的方向被定义为每个坐标增加时 $\bvec r$ 增加的方向, 即以下偏导数的方向
\begin{equation}\label{CurCor_eq2}
\leftgroup{
\pdv*{\bvec r}{r} &= \cos\theta\, \uvec x + \sin\theta\, \uvec y\\
\pdv*{\bvec r}{\theta} &= -r\sin\theta\, \uvec x + r \cos\theta\, \uvec y\\
\pdv*{\bvec r}{z} &= \uvec z
}\end{equation}
将这三个矢量归一化% 未完成:相关词条中可以给一道例题
, 就得到三个单位矢量
\begin{equation}\label{CurCor_eq3}
\begin{cases}
\uvec r = \cos\theta\, \uvec x + \sin\theta\, \uvec y\\
\uvec \theta = -\sin\theta\, \uvec x + \cos\theta\, \uvec y\\
\uvec z = \uvec z
\end{cases}
\end{equation}

可见柱坐标系和直角坐标系中的 $\uvec z$ 相同, 而 $\uvec r, \uvec \theta$ 分别是 $\uvec x, \uvec y$ 绕 $z$ 轴逆时针旋转 $\theta$ 角所得. 所以尽管柱坐标系中的三个单位矢量的方向取决于坐标, 但它们始终两两垂直.

我们把单位矢量始终保持两两垂直的坐标系叫做\textbf{正交曲线坐标系}, 或简称为\textbf{曲线坐标系}. 我们熟知的直角坐标系显然就是一个正交曲线坐标系, 稍后我们会看到球坐标系也是正交曲线坐标系.

现在我们可以将\autoref{CurCor_eq1} 和\autoref{CurCor_eq2} 用柱坐标中的三个单位矢量来表示.
\begin{equation}
\bvec r = r\uvec r + z\uvec z
\end{equation}
\begin{equation}
\pdv{\bvec r}{r} = \uvec r \qquad \pdv{\bvec r}{\theta} = r\uvec \theta \qquad \pdv{\bvec r}{z} = \uvec z\label{CurCor_eq5}
\end{equation}

与极坐标的情况\upref{DPol1} 类似, 将\autoref{CurCor_eq3} 对 $\theta$ 求偏导可以得到单位矢量的偏导
\begin{equation}
\pdv{\uvec r}{\theta} = \uvec \theta \qquad
\pdv{\uvec \theta}{\theta} = -\uvec r \qquad
\pdv{\uvec z}{\theta} = \bvec 0
\end{equation}
根据\autoref{CurCor_eq5} 和矢量函数的全微分%未完成: 链接
, 柱坐标系中一段微小位移可记为
\begin{equation}\label{CurCor_eq7}
\dd{\bvec r} = \pdv{\bvec r}{r}\dd{r} + \pdv{\bvec r}{\theta}\dd{\theta} + \pdv{\bvec r}{z}\dd{z} = \dd{r}\uvec r + r\dd{\theta} \uvec \theta + \dd{z} \uvec z
\end{equation}

\subsection{球坐标系}
球坐标系中, 位置矢量可以表示为
\begin{equation}
\bvec r = r \uvec r = r\sin\theta\cos\phi\,\uvec x + r\sin\theta\sin\phi\,\uvec y + r\cos\theta\uvec z
\end{equation}
同样, 球坐标系的三个单位矢量由三个坐标增加的方向确定
\begin{equation}\label{CurCor_eq9}
\leftgroup{
\pdv*{\bvec r}{r} &= \sin\theta\cos\phi\,\uvec x + \sin\theta\sin\phi\,\uvec y + \cos\theta\,\uvec z\\
\pdv*{\bvec r}{\theta} &= r\cos\theta\cos\phi\,\uvec x + r\cos\theta\sin\phi\,\uvec y - r\sin\theta\,\uvec z\\
\pdv*{\bvec r}{\phi} &= -r\sin\theta\sin\phi\,\uvec x + r\sin\theta\cos\phi\,\uvec y
}\end{equation}
归一化得三个单位矢量为
\begin{equation}\label{CurCor_eq10}
\begin{cases}
\uvec r = \sin\theta\cos\phi\,\uvec x + \sin\theta\sin\phi\,\uvec y + \cos\theta\,\uvec z\\
\uvec \theta = \cos\theta\cos\phi\,\uvec x + \cos\theta\sin\phi\,\uvec y - \sin\theta\,\uvec z\\
\uvec\phi = -\sin\phi\,\uvec x + \cos\phi\,\uvec y
\end{cases}
\end{equation}
不难验证这三个单位矢量两两间内积为零, 即两两垂直, 所以球坐标系也属于正交曲线坐标系. 对比\autoref{CurCor_eq9} 与\autoref{CurCor_eq10}, 有
\begin{equation}\label{CurCor_eq11}
\pdv{\bvec r}{r} = \uvec r \qquad
\pdv{\bvec r}{\theta} = r\uvec \theta \qquad
\pdv{\bvec r}{\phi} = r\sin\theta\,\uvec\phi
\end{equation}
所以位置矢量的微分可以表示为
\begin{equation}\label{CurCor_eq12}
\dd{\bvec r} = \dd{r}\uvec r + r\dd{\theta} \uvec \theta + r\sin\theta\dd{\phi}\uvec \phi
\end{equation}

