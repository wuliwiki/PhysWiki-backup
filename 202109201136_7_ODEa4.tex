% 一阶隐式常微分方程
% keys 隐式方程|ODE|differential euqation

\pentry{一阶常微分方程解法:常数变易法\upref{ODEa2},一阶常微分方程解法:恰当方程\upref{ODEa3}}

我们之前讨论的常微分方程都是显式写出导函数表达式的,即$\frac{\dd y}{\dd x}=F(x, y)$的形式.很多时候,一阶微分方程常被写为$F(x, y, \frac{\dd y}{\dd x})=0$ 的形式,如果这样的方程可以被改写为显式的形式,那么我们就可以尝试用预备知识中介绍过的方法来解方程;但如果难以改写或者解出来的形式极为复杂,那我们也可以尝试\textbf{换元}的方法.

本节介绍四种一阶隐式方程和它们的换元方法.

\subsection{第一种}

第一个要讨论的是形如
\begin{equation}\label{ODEa4_eq2}
y=f(t, \frac{\dd y}{\dd t})
\end{equation}
的方程.这里自变量用的是通常代表时间的$t$,为的是提示该怎么换元——如果$y$是位移,那$\dd y/\dd t$就是速度,这就是我们要的变换.

令$v=\frac{\dd y}{\dd t}$,则原方程变为$y=f(t, v)$.在方程两边同时对$t$求导,得到
\begin{equation}\label{ODEa4_eq1}
v=\frac{\partial f(t, v)}{\partial t}+\frac{\partial f(t, v)}{\partial v}\frac{\dd v}{\dd t}
\end{equation}

\autoref{ODEa4_eq1} 就是一个关于$t, v$的一阶微分方程,用我们之前讨论过的方法就可以解出,再将解出的$v$代回\autoref{ODEa4_eq2} 即可得到原方程的通解.

\begin{example}{}


考虑方程
\begin{equation}\label{ODEa4_eq3}
y=\qty(\frac{\dd y}{\dd x})^2+2x\frac{\dd y}{\dd x}
\end{equation}
令$v=\frac{\dd y}{\dd x}$,代入\autoref{ODEa4_eq3} ,并两端对$x$求导,则\autoref{ODEa4_eq3} 化为
\begin{equation}
v=2v\frac{\dd v}{\dd x}+2v+2x\frac{\dd v}{\dd x}
\end{equation}
整理一下,得
\begin{equation}
v\dd x+(2v+2x)\dd v=0
\end{equation}
这不是一个恰当方程\footnote{$\frac{\frac{\partial (2v+2x)}{\partial x}-\frac{\partial v}{\partial v}}{v}=\frac{1}{v}$是$v$的函数,因此我们可以为它找到一个积分因子$f(v)$.},不过我们可以给它添加一个积分因子$f(v)=\E^{\int 1/v \dd v}=v$,把它变成一个恰当方程
\begin{equation}
v^2\dd x+(2v^2+2xv)\dd v=0
\end{equation}

令$u(x, v)=v^2x+\frac{2}{3}v^3$,那么$\dd u=v^2\dd x+(2v^2+2xv)\dd v$.

因此,\autoref{ODEa4_eq3} 的通解为$u=C$,即
\begin{equation}
\leftgroup{
    \begin{aligned}
    v^2x+\frac{2}{3}v^3&=C\\
    v^2+2xv&=y
    \end{aligned}
}
\end{equation}



\end{example}


\begin{example}{}
考虑方程
\begin{equation}\label{ODEa4_eq7}
5\qty(\frac{\dd y}{\dd x})^2+5x\frac{\dd y}{\dd x}+x^2=y
\end{equation}
令$p=\dd y/\dd x$,两边对$x$求导,则原方程化为
\begin{equation}
10p\frac{\dd p}{\dd x}+5p+5x\frac{\dd p}{\dd x}+2x=p
\end{equation}
整理得
\begin{equation}\label{ODEa4_eq4}
10p\frac{\dd p}{\dd x}+5x\frac{\dd p}{\dd x}+4p+2x=0
\end{equation}

\autoref{ODEa4_eq4} 还可以进一步整理为
\begin{equation}
(\frac{5}{2}\frac{\dd p}{\dd x}+1)(4p+2x)=0
\end{equation}

由$\frac{5}{2}\frac{\dd p}{\dd x}+1=0$得通解
\begin{equation}
p=-\frac{2}{5}x+C
\end{equation}
代入$p=\dd y/\dd x$就得到原方程的\textbf{第一个通解}
\begin{equation}\label{ODEa4_eq5}
y=-\frac{1}{5}x^2+C_1x+C_2
\end{equation}
其中$C_1, C_2$都是积分常数.

但这个方程\textbf{还有一个通解}:取$4p+2x=0$,再代入$p=\dd y/\dd x$,得到
\begin{equation}\label{ODEa4_eq6}
y=-\frac{1}{4}x^2+C
\end{equation}

因此,\autoref{ODEa4_eq5} 和\autoref{ODEa4_eq6} 都是\autoref{ODEa4_eq7} 的通解.



\end{example}




% 出题的思路草稿
% % $(A\frac{\dd p}{\dd x}+B)(Cp+Dx)=(ACp\frac{\dd p}{\dd x}+ADx\frac{\dd p}{\dd x}+BCp+BDx)$

% % \begin{equation}
% % \leftgroup{
% %     AC=10\\
% %     AD=5\\
% %     BC=4\\
% %     BD=2\\
% % }
% % \end{equation}

% % $\frac{2X}{Y}=\frac{Y-1}{2}$
% 此处$X, Y$是原方程前两个的系数,$2X/Y=AC/AD=BC/BD=(Y-q)/2$
























