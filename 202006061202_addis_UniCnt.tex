% 一致连续
% 数学分析|连续函数|极限

\pentry{函数的连续性\upref{contin}}% 未完成

\subsection{一元函数}
\begin{definition}{一元函数的一致连续}
若函数 $f(x)$ 满足对于任意 $\epsilon > 0$, 存在 $\delta$, 当 $\abs{x_2 - x_1} < \delta$ 就有 $\abs{f(x_1) - f(x_2)} < \epsilon$, 那么它就是一致连续的.
\end{definition}

一致连续是比连续更强的条件, 一致连续的函数必定是连续的.

\begin{exercise}{连续但不一致连续的函数}
试证明 $1/x$ 在区间 $(0, +\infty]$ 以及 $x^2$ 在 $\mathbb R$ 都是连续的, 但不是一致连续的.
\end{exercise}
