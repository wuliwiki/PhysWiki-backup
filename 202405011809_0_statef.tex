% 态函数
% keys 态函数|状态量|内能|熵
% license Xiao
% type Tutor
\pentry{热力学第一定律\nref{nod_Th1Law},全微分\nref{nod_TDiff}}{nod_b00f}
%有些专业名称我拿捏的不是很准,可能需要规范一下表述%
\footnote{参考了朱文涛《简明物理化学》}


在热力学中,我们常常研究处于\textbf{平衡状态}的热力学系统,它的各个宏观物理量具有确定的值,仅由系统当前所处的状态所决定,与达到平衡态的过程无关。最简单的态函数有体积 $V$,温度 $T$,压强 $p$ 等,它们在实验上通常可以被直接测量,并且它们也有直观的物理意义,我们常常以这些物理量作为参量来研究系统。
\subsection{态函数}
\textbf{态函数}(有时候也称为\textbf{状态量})为描述平衡态热力学系统的宏观物理量。\textbf{这些量只和系统当前的状态有关},并可“度量”系统的某一性质。系统的状态通常可以由有限多个热力学参量来表示。

在经典热力学中,有不少描述系统热力学性质的状态量。最直观的状态量是体积 $V$,温度 $T$,压强 $p$。 更深刻一点,我们还使用内能 $U$(有时也用记号 $E$ 表示)来表征系统所具有的能量、熵$S$表征系统的混乱度。如果系统处于外场之中,我们还需要描述系统的磁化强度 $M$ 等。 在以后,我们还会涉及 焓 $H$、\enref{吉布斯能}{GibbsG} $G$、亥姆霍兹自由能 $F$\upref{HelmF}等状态量,这些状态量更为抽象、但在某些问题中十分好用(再算上内能,这四个具有能量量纲的量也被称为热力学势)。

在热力学中,我们常常对系统作各种变换(例如等压变换、等体积变换、等温变换、绝热变换……),在这一过程中态函数的变化常常有一定的规律。\textbf{但是不管变化过程如何,系统态函数的改变量只与起点和终点系统的状态有关,与系统是如何从一个状态变化至另一个状态的具体路径无关。} 热力学研究的就是热力学系统的态函数之间的关系。\textbf{一些态函数在实验上无法被直接测量,但可以通过它们的定义式或通过间接的方式测量和计算。}

\subsubsection{广延量与强度量}
我们可以将系统的状态量分为两类:广延量与强度量。广延量与“系统的规模”成正比,即与系统所含物质的量成正比;相反地,强度量与“系统的规模”不直接相关。例如,体积、内能是广延量,他们与系统的物质的量相关;而温度、压强是强度量。

举个例子,由于体积是广延量,如果 $1 \Si {mol}$的水的体积是$V$,那么(相同温度、压强下)$4 \Si {mol}$的水的体积就应该是$4 V$。
$$
V(p,T,4n) = 4V(p,T,n)~,
$$

\textsl{这就是那个古老的笑话:"把四杯$25 ^\circ C$的水倒在一起,你不会得到一杯$100 ^\circ C$的水,而只会得到一大杯$25 ^\circ C$的水"。}

\begin{table}[ht]
\centering
\caption{广延量与强度量}\label{tab_statef_1}
\begin{tabular}{|c|c|c|c|c|c|}
\hline
强度量 & 压强 $p$ & 温度 $T$ & & & \\
\hline
广延量 & 物质的量 $n$ & 体积 $V$ & 内能 $U$& 熵 $S$ & ...   \\
\hline
\end{tabular}
\end{table}


\subsection{状态方程}
一个系统的各状态量不是完全独立的,他们之间存在某种关联。也就是说,一旦系统的一些状态量确定了,他的另一些状态量也就随之确定、而不再能随意变动。

这种关联抽象地体现为“状态方程”("Equation of State, EOS")。在教材中,“状态方程”一般特指 $P, V, T, n$这四个量间的联系:
$$
f(P, V, T, n) = 0~.
$$

最简单的状态方程或许是理想气体的状态方程:
$$pV = nRT~,$$
这个状态方程告诉我们,气体的$T$ 可以由 $n, P,V$ 确定。当然,你也可以说 气体的$p$ 可以由 $n, V,T$ 确定,所谓“知三求一”。

由于固体、液体的性质相当复杂,而且各种物质的性质也不尽相同,因此我们很少显式地写出他们的“状态方程”。不过,正如我们上述所言,我们理论上有一些计算状态量的方法。

\subsection{"状态公理"}
%好像没有正式的表述?我感觉这么叫挺合适的(
我们可以进一步细化状态方程中的独立变量数。事实上,如果系统不处于外场\footnote{例如处于重力场下的大气,压强会随高度的变化而变化,情况肯定会有所不同,}之中、系统中的物质种类已知,那么少数几个宏观的热力学量就能完整地刻画一个\textbf{热力学平衡系统}。

\subsubsection{一元单相系统}
\begin{figure}[ht]
\centering
\includegraphics[width=5cm]{./figures/bfb4e1188eb70cf0.pdf}
\caption{一元单相系统} \label{fig_statef_1}
\end{figure}
对于一个一元单相平衡系统,只需两个热力学强度量(一般使用压强 $P$ 和温度 $T$ )与该物质的物质的量,便足以刻画这个系统的一切宏观特征,即原则上可得到U,S,H等其余热力学量。
\begin{equation}
X=f(T,P,n)~.
\end{equation}
% 在X是强度量(与系统的尺度无关,例如温度、压力、单位内能等)而非广度量(与系统的尺度有关,例如体积、内能、熵等)时,可省去物质的量条件。
% \begin{equation}
% X=f(T,P)~.
% \end{equation}

\begin{example}{E的P-T全微分形式}
对于一定量的物质(可以是气体、液体或固体),我们可以把内能 $E$ 写成 $T$ 和 $P$ 的函数 $E(T,P)$
\begin{equation}
E=E(T,P)~.
\end{equation}
因此可以写出 $E$ 的全微分\upref{TDiff}形式:
\begin{equation}
\dd E=\left(\frac{\partial E}{\partial T}\right)_p \dd T + \left(\frac{\partial E}{\partial P}\right)_T \dd P~.
\end{equation}
这个全微分形式刻画的是,在一个改变温度和压强的\textbf{微小的准静态过程}\upref{Quasta}中\footnote{也就是说,在这个过程中,系统每时每刻都是平衡态,$E$ 总是态函数。要注意的是,态函数的定义只对一个热力学平衡系统有效},$E$ 的改变量与 $\dd T$ 和 $\dd P$ 的线性关系,偏导数即为相应的比例系数。

如果我们能知道两个偏导数,利用全微分关系式,我们可以刻画任意准静态过程的内能改变量;或者更进一步,直接得到任意$T$,$P$下的$E=E(T,P)$(难过的是,实操上这一般不大可行)。
\end{example}

\begin{example}{E的S-V全微分形式}
\pentry{熵\nref{nod_Entrop}}{nod_a470}
熵是表征系统混乱程度的一个态函数,一个系统熵越小,则越有序,这会在熵的纯微观分析中\upref{IdeaS}会进行解释。熵的宏观定义是,可逆过程中 $\dd Q/T$ 的积分\upref{Entrop},也就是说 $\dd S=\dd Q/T$。所以,\textbf{等熵过程就是绝热过程\upref{Adiab}}。

我们把热力学第一定律中 $\dd Q$ 改写为 $T\dd S$,$\dd W$ 改写为 $p\dd V$(这里我们只考虑体积功)。那么有\footnote{
此处的全微分式是在可逆过程中成立的。由热力学第二定律\upref{Td2Law},对于不可逆的热力学过程,有 $\dd Q<T\dd S$。所以代入第一定律可以得到 $T\dd S\ge \dd E+\dd W=\dd E+P\dd V$,等号在可逆过程中成立。
}

\begin{equation}
\dd E=T\dd S-P\dd V~.
\end{equation}

于是 $E$ 可以看成是熵 $S$ 和体积 $V$ 的函数,由全微分式,可得 $T=\left(\frac{\partial E}{\partial S}\right)_V$,$P=-\left(\frac{\partial E}{\partial V}\right)_S$。第二个式子表明压强就是绝热过程中内能随体积增大而减小的量。

上面的结论也告诉我们一个神奇的事实,对一个过程量(例如做功,吸热),如果将它乘以某个函数,它可能能成为态函数。虽然 $\dd W$ 是过程量,但 $\dd W/P=\dd V$ 是全微分($V$ 是态函数)。$\dd Q/T=\dd S$ 也是全微分,其中 $S$ 为热力学熵\upref{Entrop}。
\end{example}

\subsubsection{一元多相系统}
\begin{figure}[ht]
\centering
\includegraphics[width=5cm]{./figures/6598647b8d0d82e7.pdf}
\caption{一元多相系统} \label{fig_statef_2}
\end{figure}
如果系统中包括多个相,由于各个相的状态量不一定相同,因此需要把每一个相分别视为一个子系统,并分别运用“状态公理”。同时,由于物质可以在相之间转移(“相变”),还需要强调各相的物质的量。
$$
\begin{aligned}
X^\alpha&=f(T^\alpha,P^\alpha,n^\alpha)~,\\
X^\beta&=f(T^\beta,P^\beta,n^\beta)~,\\
&...\\
\end{aligned}
$$
此处$\alpha, \beta, ...$上标表示相,不是指数。

\subsubsection{多元多相系统}
\begin{figure}[ht]
\centering
\includegraphics[width=5cm]{./figures/82cd03ba52e1bcdf.pdf}
\caption{多元多相系统} \label{fig_statef_3}
\end{figure}
若系统中包括多种物质,那么要分别写出每种物质的物质的量。
$$
\begin{aligned}
X^\alpha&=f(T^\alpha,P^\alpha,n_1^\alpha, n_2^\alpha, ...)~.\\
X^\beta&=f(T^\beta,P^\beta,n_1^\beta, n_2^\beta, ...)~,\\
&...\\
\end{aligned}
$$
下标$1,2,3,...$表示各物质的物质的量。
