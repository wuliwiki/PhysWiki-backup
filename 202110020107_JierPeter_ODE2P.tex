% 二阶齐次变系数线性微分方程的幂级数解法
% 幂级数|ODE|常微分方程|differential equation|二阶微分方程

\pentry{幂级数\upref{powerS},常系数线性齐次微分方程\upref{ODEb2}}

\addTODO{是不是还要加上差分方程作为预备知识?}

\subsection{从例子出发}

从微积分学中我们知道,许多函数是可以表示为幂级数的形式:$f(x)=a_0+a_1x+a_2x^2+\cdots$.幂级数良好的性质可以用于解二阶微分方程.

我们先看一个简单的实例.遵循微积分学的习惯,我们这里以$x$为自变量了.

\begin{example}{}\label{ODE2P_ex1}
考虑方程
\begin{equation}\label{ODE2P_eq4}
\frac{\mathrm{d}^2 y}{\dd x^2}-2x\frac{\dd y}{\dd x}-4y=0
\end{equation}
在初始条件
\begin{equation}\label{ODE2P_eq1}
\leftgroup{
    y(0)&=0\\
    y(1)&=1
}
\end{equation}
下的\textbf{特解}.

尝试设
\begin{equation}\label{ODE2P_eq3}
y(x)=a_0+a_1x+a_2x^2+\cdots=\sum\limits_{i=0}^\infty a_ix^i
\end{equation}

首先代入初值条件\autoref{ODE2P_eq1} ,得到$a_0=0, a_1=1$.

接着,考虑到
\begin{equation}\label{ODE2P_eq2}
\leftgroup{
    &\frac{\mathrm{d}^2 y}{\dd x^2}=2a_2+6a_3x+12a_4x^2+\cdots=\sum\limits_{k=0}^\infty (k+1)(k+2)a_{k+2}x^k\\
    &2x\frac{\dd y}{\dd x}=2a_1x+4a_2x^2+6a_3x^3+\cdots=\cdots=\sum\limits_{k=1}^\infty 2ka_kx^k
}
\end{equation}

将\autoref{ODE2P_eq2} 和\autoref{ODE2P_eq3} 代回\autoref{ODE2P_eq4} ,比较各$x^k$的系数,得到
\begin{equation}
(k+1)(k+2)a_{k+2}=(2k+4)a_k
\end{equation}
整理得
\begin{equation}
a_{k+2}=\frac{2}{k+1}a_k
\end{equation}

这是一个二阶\textbf{差分方程}.

由于$a_0=0$,故$a_{2k}=0$对所有$k$成立.我们只需要考虑奇数项即可.

令$b_k=a_{2k-1}$\footnote{反过来就是$a_k=b_{\frac{k+1}{2}}$.},则我们有$b_1=a_1=1$和$b_{\frac{k+3}{2}}=\frac{2}{k+1}b_{\frac{k+1}{2}}$;换个写法,就是$b_{k+1}=\frac{1}{k}b_k$.

因此,
\begin{equation}
b_k=\frac{1}{(k-1)!}
\end{equation}

进而
\begin{equation}\label{ODE2P_eq6}
\begin{aligned}
y&=b_1x+b_2x^3+b_3x^5+\cdots\\
 &=x\sum\limits_{k=1}^\infty b_kx^{2k-2}\\
 &=x\sum\limits_{k=1}^\infty \frac{x^{(2k-2)}}{(k-1)!}\\
 &=x\sum\limits_{k=0}^\infty \frac{x^{2k}}{k!}\\
 &=x\E^{x^2} 
\end{aligned}
\end{equation}

\end{example}

\autoref{ODE2P_ex1} 中“假设解为$x$的幂级数,通过比较系数来求出解”的方法,被称为\textbf{幂级数解法}.


\subsection{幂级数解法}

\autoref{ODE2P_ex1} 和我们之前所解的方程不一样,\autoref{ODE2P_eq4} 中的系数$2x$不再是一个常数,而是$x$的函数,这使得我们应对\textbf{常系数}方程的方法无效了.对于二阶变系数方程,幂级数解法是很有用的.

哪些方程能应用幂级数解法呢?幂级数解的收敛区间又是否能覆盖所要求解的区间呢?这些问题有完善的解答,但由于较为深入,我们在此只给出重要的结论.

我们所考虑的方程是形如
\begin{equation}\label{ODE2P_eq5}
\qty(\frac{\mathrm{d}^2}{\dd x^2}+p(x)\frac{\dd}{\dd x}+q(x))y(x)=0
\end{equation}
在$x=0$处的\textbf{特解}.

$x=x_0$处的特解,可以通过变量代换$t=x-x_0$,来化为$y(t)$的方程在$t=0$处的特解问题.

\begin{theorem}{}\label{ODE2P_the2}
如果\autoref{ODE2P_eq5} 中的$p(x)$和$q(x)$都可以写为$x$的幂级数形式,且它们在区间$\abs{x}<X$上收敛,那么方程有形如
\begin{equation}
y=\sum\limits_{k=0}^\infty a_nx^n
\end{equation}
的特解,且该特解也在$\abs{x}<X$上收敛.
\end{theorem}

\autoref{ODE2P_ex1} 中的系数为$-2x$和$-4$,它们都在整个实数轴上收敛,因此我们最终算出来的特解\autoref{ODE2P_eq6} 也在整个实数轴上收敛.

\begin{theorem}{}\label{ODE2P_the1}
如果$xp(x)$和$x^2q(x)$均能展成幂级数形式,并且都在$\abs{x}<X$上收敛,那么\autoref{ODE2P_eq5} 有形如
\begin{equation}\label{ODE2P_eq7}
y=x^\alpha\sum\limits_{k=0}^\infty a_kx^k
\end{equation}
的特解,其中$a_0\neq 0$,$\alpha$是一个待定常数,并且\autoref{ODE2P_eq7} 也在$\abs{x}<X$上收敛.


\end{theorem}


\subsection{若干例题}

二阶变系数线性微分方程在工程和物理中应用广泛,因此我们在此举出一些例题,以帮助读者熟悉其解法.

\begin{example}{}\label{ODE2P_ex2}
考虑方程
\begin{equation}\label{ODE2P_eq9}
\qty(\frac{\mathrm{d}^2}{\dd x^2}+\frac{1}{x}\frac{\dd}{\dd x}-\frac{1}{x})y(x)=0
\end{equation}
在
\begin{equation}\label{ODE2P_eq10}
\leftgroup{
    y(0)&= 1\\
    y'(0)&= 1\\
}
\end{equation}
下的特解.

设所求特解为
\begin{equation}\label{ODE2P_eq8}
\sum\limits_{k=0}^\infty a_kx^k
\end{equation}

这次我们先求通解,再代入初值条件求特解.

首先将\autoref{ODE2P_eq8} 代入\autoref{ODE2P_eq9} ,得到
\begin{equation}
\sum\limits_{k=0}^\infty (k+2)(k+1)a_{k+2}x^k+(k+2)a_{k+2}x^k-a_{k+1}x^k=0
\end{equation}

从而得到
\begin{equation}
a_{k+2}=\frac{a_{k+1}}{(k+2)^2}
\end{equation}

即
\begin{equation}\label{ODE2P_eq11}
a_k=\frac{a_{k-1}}{k^2}
\end{equation}

由初值条件\autoref{ODE2P_eq10} 得,
\begin{equation}
a_k=\frac{1}{(k!)^2}
\end{equation}

则
\begin{equation}
y=\sum\limits_{k=0}^\infty \frac{x^k}{(k!)^2}
\end{equation}
是所求的特解.


\end{example}


% 注释掉一个出得不好的题

% \begin{example}{}
% 考虑方程
% \begin{equation}\label{ODE2P_eq14}
% \qty(\frac{\mathrm{d}^2}{\dd x^2}+\frac{1}{x}\frac{\dd}{\dd x}-\frac{1}{x})y(x)=0
% \end{equation}
% 在
% \begin{equation}\label{ODE2P_eq12}
% \leftgroup{
%     y(0)&= 0\\
%     y'(0)&= 1\\
% }
% \end{equation}
% 下的特解.

% 这个例子的方程和\autoref{ODE2P_ex2} 一样, 但是初值条件不同.用\autoref{ODE2P_ex2} 的方法是求不出给定特解的,因为初值条件不满足\autoref{ODE2P_eq11} .

% 应用\autoref{ODE2P_the1} ,设所求特解为
% \begin{equation}\label{ODE2P_eq13}
% y=x^{-\alpha}\sum\limits^\infty_{k=0}a_kx^k=\sum\limits^\infty_{k=0}a_kx^{k+\alpha}
% \end{equation}

% 将\autoref{ODE2P_eq13} 代入\autoref{ODE2P_eq14} ,得到
% \begin{equation}\label{ODE2P_eq15}
% \begin{aligned}
% \sum\limits_{k=0}^\infty (k+\alpha)(k+\alpha-1)a_kx^{k+\alpha-2}+\\
% \sum\limits_{k=0}^\infty(k+\alpha)a_kx^{k+\alpha-2}-\\
% \sum\limits_{k=0}^\infty a_kx^{k+\alpha-1}\\
% =0
% \end{aligned}
% \end{equation}

% 重新排列一下,把$x$的对应次幂放到一起,将\autoref{ODE2P_eq15} 整理成
% \begin{equation}\label{ODE2P_eq16}
% \sum\limits_{k=1}^\infty\qty[(k+\alpha)^2a_k-a_{k-1}]x^{k+\alpha-2}+\alpha^2a_0x^{\alpha-2}=0
% \end{equation}

% 令\autoref{ODE2P_eq16} 左边各项系数为零,得到一系列代数方程:
% \begin{equation}\label{ODE2P_eq17}
% \leftgroup{
%     \alpha^2a_0&=0\\
%     (\alpha+1)^2a_1-a_0&=0\\
%     (\alpha+2)^2a_2-a_1&=0\\
%     &\vdots\\
%     (\alpha+k)^2a_k-a_{k-1}&=0\\
%     &\vdots\\
% }
% \end{equation}

% 初值条件决定了
% \begin{equation}
% \leftgroup{
%     a_0=0\\
%     \sum\limits_{k=0}^\infty
% }
% \end{equation}



% \end{example}


\begin{example}{$n$阶贝塞尔方程}

$n$阶贝塞尔方程形如
\begin{equation}\label{ODE2P_eq18}
x^2\frac{\mathrm{d}^2 y}{\dd x^2}+x\frac{\dd y}{\dd x}+(x^2-n^2)y=0
\end{equation}

首先把\autoref{ODE2P_eq18} 改写为
\begin{equation}
\frac{\mathrm{d}^2 y}{\dd x^2}+\frac{1}{x}\frac{\dd y}{\dd x}+(1-\frac{n^2}{x^2})y=0
\end{equation}

由\autoref{ODE2P_the1} ,它具有一个通解
\begin{equation}\label{ODE2P_eq19}
y=\sum\limits_{k=0}^\infty a_kx^{k+\alpha}
\end{equation}
其中$a_k, \alpha$是待定常数.

将\autoref{ODE2P_eq19} 代入\autoref{ODE2P_eq18} ,可得
\begin{equation}\label{ODE2P_eq20}
\begin{aligned}
&\sum\limits^\infty_{k=0}a_k(k+\alpha)(k+\alpha-1)x^{k+\alpha}+\\
&\sum\limits^\infty_{k=0}a_k(k+\alpha)x^{k+\alpha}+\\
&\sum\limits^\infty_{k=0}a_kx^{k+\alpha+2}-\\
&\sum\limits^\infty_{k=0}a_kn^2x^{k+\alpha}\\
&=0
\end{aligned}
\end{equation}

重新整理一下\autoref{ODE2P_eq20} ,将$x$的同次幂放在一起,得到
\begin{equation}
\begin{aligned}
\sum\limits^\infty_{k=0}[(k+\alpha)^2-n^2]a_kx^{k+\alpha}\\
+\sum\limits^\infty_{k=2}a_{k-2}x^{k+\alpha}=0
\end{aligned}
\end{equation}
令各项系数为$0$,则得到一系列代数方程:
\begin{equation}
\leftgroup{
    (\alpha^2-n^2)a_0&=0\\
    [(1+\alpha)^2-n^2]a_1&=0\\
    [(2+\alpha)^2-n^2]a_2+a_0&=0\\
    [(3+\alpha)^2-n^2]a_3+a_1&=0\\
    &\vdots
}
\end{equation}
由\autoref{ODE2P_the2} ,$


\end{example}
















