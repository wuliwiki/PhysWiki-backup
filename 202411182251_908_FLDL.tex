% 法拉第电磁感应定律(综述)
% license CCBYSA3
% type Wiki

本文根据 CC-BY-SA 协议转载翻译自维基百科\href{https://en.wikipedia.org/wiki/Faraday\%27s_law_of_induction}{相关文章}。


法拉第电磁感应定律(简称法拉第定律)是电磁学中的一条定律,用于预测磁场如何与电路相互作用以产生电动势(emf)。这种现象被称为电磁感应,是变压器、电感器以及许多类型的电动机、发电机和螺线管的基本工作原理。[2][3]

麦克斯韦-法拉第方程(作为麦克斯韦方程组之一)描述了一个事实,即空间变化的(也可能是时间变化的,具体取决于磁场随时间的变化情况)电场总是伴随着时间变化的磁场,而法拉第定律则表明,当通过由导电回路包围的表面的磁通量随时间变化时,导电回路中会产生电动势(即单位电荷沿回路运动一圈时电磁作用所做的功)。

法拉第定律被发现后,其一个方面(变压器电动势)被表述为麦克斯韦-法拉第方程。法拉第定律的方程可以通过麦克斯韦-法拉第方程(描述变压器电动势)和洛伦兹力(描述运动电动势)推导而来。麦克斯韦-法拉第方程的积分形式仅描述变压器电动势,而法拉第定律的方程同时描述变压器电动势和运动电动势。