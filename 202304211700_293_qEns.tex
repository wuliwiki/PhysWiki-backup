% 量子系综
% 量子系综|混态

\begin{issues}
\issueDraft
\issueTODO
\end{issues}

\pentry{量子比特\upref{Qubit}}


在传统的量子力学学习中,我们已经对密度矩阵\upref{denMat}和量子系综有了一定的了解。现在我们将会从信息的角度来重新理解这一概念。

\subsection{统计与系综}

我们不妨从Stern Gerlach实验中来引入量子系综的概念。

\pentry{Stern Gerlach实验\upref{SGExp}, 系综理论}
在对单比特纯态的讨论中,我们发现了一个重要的结论:对于任意的单比特纯态,总是存在着一个方向$\vec{n}\cdot\vec{\sigma}$,使得这个纯态是它的本征值为1的本征态。这意味着这样一个重要的结论:只要银原子的自旋态是纯态,那么总是存在着一个方向,只要将非均匀磁场调整到这个方向,那么这些银原子就只会在屏幕上产生一个斑点。

但是这与实验结论是不相符的。从炉子中产生的银原子满足这样的性质:不管将磁场调整到什么方向,屏幕上都会产生两个斑点。这一结论只能说明,这些银原子处于一种不能用纯态描述的状态。

什么状态能够描述这些银原子呢?经典统计物理能够给我们提供一些灵感。回想统计物理中,我们定义了系综这个概念,它指的是大量的,拥有相同宏观性质的系统(但微观状态不一定相同)的集合,从这样的大量系统的样本中进行取样,就可以确定系统代表点在相空间中的概率分布。在明确了这一点后,我们就可以把逻辑反过来:在给定了一个概率分布$\{p_i\}$和系统的一系列可能拥有(微观)状态$\{\text{态}_i\}$后,我们就可以定义一个系综为
\begin{equation}
\mathcal{E}:=\{p_i, \text{态}_i\}_{i\in\mathcal{I}}
\end{equation}
其中$\mathcal{I}$是指标集。在经典物理中,系统的状态就是相空间中的一个几何点,因此系综往往不用(1)中的记号来描述,而直接将其与相空间上的一个概率密度分布$\rho$建立等同。

在量子力学中,我们仍然希望将这一套方法仍然成立。幸运的是,只需要将(1)中的态从“相空间中的几何点”换成“希尔伯特空间中的射线”即可。也就是说,一个量子系综应当被定义为
\begin{equation}
\mathcal{E}:=\{p_i,\ket{\psi_i}\}_{i\in\mathcal{I}}
\end{equation}给定一些量子纯态$\psi_i$和一个概率分布,我们即可定义一个量子系综。

回到Stern-Gerlach实验中。由于自旋的方向对银原子在炉子里的能量没有影响,那么统计力学的经验会告诉我们,如果用系综来对炉子中的银原子进行描述的话,那么它关于自旋态是等概率分布的,也就是说$p_i$是一个均匀分布,而$\ket{\psi_i}$取遍整个Bloch球面。

\subsection{密度矩阵}

现在我们要考虑在一个量子系综上测量一个可观测量,并且获得它的平均值。容易得到对$\mathcal{E}$来说,这个平均值是\begin{equation}
\langle O\rangle_\mathcal{E}=\sum_i p_i\bra{\psi_i}O\ket{\psi_i}
\end{equation}
即先在每一个纯态$\ket{\psi_i}$上取期望,然后按照$p_i$进行混合。注意到利用trace的性质,(3)可以进一步写为
\begin{equation}
\langle O\rangle_\mathcal{E}=\sum_i p_i\operatorname{Tr}\left[\ket{\psi_i}\bra{\psi_i}O\right]=\operatorname{Tr}\left[\left(\sum_i p_i\ket{\psi_i}\bra{\psi_i}\right)O\right]
\end{equation}
这暗示着我们可以把小括号中的表达式$\sum_i p_i\ket{\psi_i}\bra{\psi_i}$看作是$\mathcal{E}$的一种等价表示。我们把形如这样的表达式叫做密度矩阵(density matrix)。容易得到密度矩阵既可以描述一个量子纯态(只需要仅让一个$p_i=1$即可),也可以描述量子系综。因此在量子信息科学中,我们更习惯使用密度矩阵来表示量子系统的状态。


\subsection{经典和量子概率}

在得到了量子系综和其密度矩阵描述之后,我们可以对量子力学中的概率获得更加深入的理解了。



