% (低阶)偏微分方程的分类与特征线
% keys 特征线|PDE|偏微分方程
% license Usr
% type Tutor

\begin{issues}
\issueMissDepend
\end{issues}

特征线法又被称为达朗贝尔法和行波法,一般适用于解初值问题的一阶 PDE,对二阶 PDE 的分类也颇有帮助。本文默认 PDE 的范围是 $-\infty < x, y, \dots < +\infty, t>0$,用 $u'_t$ 表示 $\pdv{u}{t}$,用 $u'_x$ 表示 $\pdv{u}{x}$。默认二阶导连续,即 $u''_{xy} = u''_{yx} = \pdv{}{y}\left(\pdv{u}{x}\right) = \pdv{}{x}\left(\pdv{u}{y}\right)$。

特征线有如下性质,
\begin{enumerate}
\item 对于 $n$ 维(这里包含时间维)的 PDE,其特征线总是 $n-1$ 维的。
\item 解的“间断”性质通过、且仅能通过特征线传播。(也就是说,不同时刻,同一特征线上的解相同,但有时数值会沿特征线衰减。)
\item 特征线上 PDE 的解是一样的,特征线的斜率是解“传播”的速度。
\end{enumerate}
其中性质 $2$ 对应着如果不存在特征线(特征线不是实的,类似一元二次方程没有实数根称为根不存在),那么 PDE 的解连续(这就是 elliptic PDE)。


































%=====================================================
\subsection{一维一阶 PDE}
\begin{theorem}{}
对于关于 $u(x, t)$ 的 PDE:
\begin{equation}
\pdv{u}{t} + A(x, t) \pdv{u}{x} + B(x, t) u = f(x, t) ~,
\end{equation}
初值条件 $u(x, 0) = \phi(x), -\infty < x < +\infty$。

设特征线族为 $x = x(t, \tau)$ 是下面 ODE 的解,
\begin{equation}
\left\{
\begin{aligned}
\dv{x}{t} &= A(x, t), \\
x(0) &= \tau ~.
\end{aligned}
\right .
\end{equation}
考虑 $v(t) = u(x(t), t)$,由全微分与偏导的关系,
$$\dv{v}{t} = \pdv{u}{x} \dv{x}{t} + \pdv{u}{t} \dv{t}{t} = \pdv{u}{x} \cdot A(x, t) + \pdv{u}{t} ~,$$
这便是特征线的解。若再要求解原方程组,又有:
\begin{equation}
\left\{
\begin{aligned}
\dv{v}{t} + B(x, t) v &= f(x(t), t) ~, \\
v(0) = u(x(0), 0) &= u(\tau, 0) = \phi(x) ~.
\end{aligned}
\right.
\end{equation}
\end{theorem}











%========================================================举例
下面利用两个一维一阶的 PDE 举例来说明如何求解 PDE 的特征线。
\begin{example}{}
求解这 PDE 的特征曲线:
\begin{equation}
\left \{ 
\begin{aligned}
u'_t + (x + t) u'_x + u &= x ~, \\
\eval{u}_{t=0} &= x ~.
\end{aligned}
\right .
\end{equation}

\textbf{解}:特征曲线 $x=x(t)$ 对应下面 ODE,
\begin{equation}
\left \{ 
\begin{aligned}
\dv{x}{t} &= x+t ~, \\
x(0) &= \tau ~.
\end{aligned}
\right .
\end{equation}
可以解得 $x(t) = e^t - t - 1 + \tau \cdot e^t$。
\end{example}









%========================================另一个方法
另一个方法则是考虑特征线下的坐标系 $(s, \tau)$,其中 $\tau$ 仍是某个常数,由选择的“哪条”特征线而定,而 $s \ge 0$ 是代表在某条特征线上的位置。
\begin{figure}[ht]
\centering
\includegraphics[width=12em]{./figures/0a3f5c2a3d19bed0.png}
\caption{$s-\tau$ 坐标系\cite{SJFPDE}} \label{fig_chacur_1}
\end{figure}


\begin{example}{}
通过特征线法求这 pde 的解:
$$
\left\{
\begin{aligned}
{u_t'} + {u_x'} &= -u, -\infty < x < +\infty, t > 0, \\
u(x, 0) &= \sin x, -\infty < x < +\infty .
\end{aligned}
\right.
~~
$$

\textbf{解}:考虑将原 PDE 转化到 $s-\tau$} 坐标系下的 ODE。由全微分与偏微分的关系:
$$
\dv{u}{s} = \pdv{u}{x} \dv{x}{s} + \pdv{u}{t} \dv{t}{s}~,
$$
则由 $u_t' + u_x' = -u$,因而 $u_t' + u_x' + u = 0$,就有
$$\dv{u}{s} + u = \pdv{u}{x} \dv{x}{s} + \pdv{u}{t} \dv{t}{s} + u = 0 ~.$$
对比系数,得到特征线方程:
$$\dv{x}{s} = 1, \dv{t}{s} = 1~, s>0$$
故 $x(s) = s+C_1, t(s) = s+C_2$。

令 $s=0$,得到 $x(0) = C_1, t(0) = C_2$。结合 $s-\tau$ 坐标系的意义,$x(0)$ 是由 $\tau$ 决定的初值 $x(0) = \tau$,$t(0)=0$。故 $x(s) = s+\tau, t(s)=s$。则特征线方程就是 $x = t+\tau$。

下面利用上面求出的特征线将原 pde 转化为 ode 求解:
$$
\left\{
\begin{aligned}
\dv{u}{s} +u &= 0, s>0, \\
\eval{u}_{s=0} &= u(\eval{x}_{s=0}, \eval{t}_{s=0}) = u(\tau, 0) = \sin \tau .
\end{aligned}    
\right.~~$$
故 $u(s,\tau) = \exp(-s) \sin \tau$。

下面将 $u(s, \tau)$ 转化回到 $x-t$ 坐标系下:
$x(s) = s + \tau, t(s) = s$,故 $u(x, t) = \exp(-t)\sin(x-t) $。

\begin{figure}[ht]
\centering
\includegraphics[width=10cm]{./figures/95de5a725503dc36.png}
\caption{解的示意} \label{fig_chacur_2}
\end{figure}
可以发现解沿着特征线有衰减的传播。图中画出了 $0\ge t\ge3$, $0\ge x \ge 4 \pi$ 的情况。
\end{example}



























%======================================
很多人都对偏微分方程为什么用圆锥曲线来分类有疑惑,一个偏微分方程为什么会跟平面上曲线的分类有关呢?对 PDE(偏微分方程)进行分类又对解其有何帮助?下面来讨论这些问题。

\subsection{二阶线性 PDE 的分类}
在一般的数理方程或介绍 PDE 的书籍中,会把二阶线性 PDE 作为着重点来讲并一般将之主要分为以下三类,分别是
\begin{enumerate}
\item 椭圆类(elliptic PDE),例如泊松方程 $\laplacian u = f$;
\item 抛物类(parabolic PDE),例如热传导方程 $k \laplacian T = \frac{\partial T}{\partial t}$;
\item 双曲类(hyperbolic PDE),例如波动方程 $\laplacian w = \frac{1}{c^2} \frac{\partial^2 w}{\partial t^2}$。
\end{enumerate}
这三类的解信息沿特征线传播,有性质与对应的物理问题:
\begin{enumerate}
\item 椭圆类:解的信息以无限速度传播,且解不应当存在间断。适用于\textbf{平衡}问题(静态或准静态,例如无源力场)。
\item 抛物类:解的信息以无限速度传播,适用于\textbf{扩散}问题(例如热传导)。
\item 双曲类:解的信息以有限速度传播,且能保留间断信息,适用于\textbf{波动}问题。
\end{enumerate}
另外还有一类 PDE 被称为 ultrahyperbolic PDE(超双曲类)。下面将讲解是如何分类的。

\subsubsection{二元二阶线性 PDE}
首先保持二阶导连续的假设,即 $\pdv{}{x}\pdv{u}{y} = \pdv{}{y}\pdv{y}{x}$。则偏微分方程具有一般形式:
$$Au''_{xx} + Bu''_{xy} + Cu''_{yy} + Du'_{x} + Eu'_{y} + F = 0 ~,$$
与一阶线性 PDE 对比,可以得到特征线应当满足方程:
$$
\left\{
\begin{aligned}
\dd{(u'_{x})} &= u''_{xx} \dd x + u''_{xy} \dd y,\\
\dd{(u'_{y})} &= u''_{xy} \dd x + u''_{yy} \dd y.
\end{aligned}
\right. ~~$$
联立这三式,令有三个矩阵:
\begin{equation}
\mat M= 
\begin{pmatrix}
A & B & C \\
\dd x & \dd y & 0 \\
0 & \dd x & \dd y 
\end{pmatrix}, 
\mat p= 
\begin{pmatrix}
u''_{xx} \\
u''_{xy} \\
u''_{yy}
\end{pmatrix},
\mat N = 
\begin{pmatrix}
-D u'_{x} - E u'_{y} - F\\
\dd{(u'_{x})}\\
\dd{(u'_{y})}
\end{pmatrix}
~~
\end{equation}
则这三式可转化为 $\mat{Mp}=\mat N$。
显然,$u$ 的各个二阶偏导项不是唯一确定的,所以  $\det \mat M = 0$,也就是需要:
\begin{equation}
A\left(\mathrm{d} y\right)^2 - B\left(\mathrm{d} x \mathrm{d} y\right) + C\left(\mathrm{d} x\right)^2 =0
~,
\end{equation}
也就是:
\begin{equation}
A\left(\dv{y}{x}\right)^2 - B\left(\dv{y}{x}\right) + C = 0 ~,
\end{equation}
发现是一个关于 $\dd y/\dd x$ 的二次方程,判别式 $\Delta = B^2-4AC$。对比一般利用二次表达式的圆锥曲线定义,可以将特征线分类如下。
\begin{enumerate}
\item $\Delta < 0$,存在虚特征线(即无特征线),称这种为\textbf{椭圆类 PDE};
\item $\Delta = 0$,存在一族特征线,称这种为\textbf{抛物类 PDE};
\item $\Delta > 0$,存在两族不同的特征线,称这种为\textbf{双曲类 PDE}。
\end{enumerate}
这就是分类的依据。

值得注意的是,PDE 可能在不同区域有不同分类。例如 $y u''_{xx} - u''_{yy} = 0$,在 $y> 0$ 时是双曲类;在 $y=0$ 时是抛物类,在 $y<0$ 时是椭圆类。
\subsubsection{多元二阶线性 PDE}
考虑线性微分算符 $\hat L$:
\begin{equation}
\hat L u = \sum_{i, j}\left(\mat A_{i, j} \cdot \pdv{}{x_i}\left(\pdv{u}{x_j}\right)\right) + \sum_{i}\left(\mat B_{i} \pdv{u}{x_i}\right) + F~,
\end{equation}
忽略低阶项(一阶的 $\mat B_{i}$ 与 $F$),考虑系数矩阵 $\mat A_{i,j}$ 的特征值,我们将 $\hat Lu=0$ 分类如下。
\begin{enumerate}
\item 当特征值均大于 $0$ 或均小于 $0$,即特征值全部同号且都非零,归类为\textbf{椭圆类};
\item 当特征值有一个为 $0$,其余均大于 $0$ 或均小于 $0$,即特征值除有一 $0$ 外均同号,归类为\textbf{抛物类};
\item 当特征值一个为正数,其他为负数;或一个为负数,其他为正数。即特征值均非 $0$,并且有仅有一非 $0$ 特征值与其他非 $0$ 特征值符号相反,归类为\textbf{双曲类};
\item 正特征值和负特征值的个数都均大于一,且特征值均非 $0$,归类为\textbf{超双曲类}。
\end{enumerate}
一般来说,对于一个有物理意义的 PDE,特征值有一个为 $0$ 的归类为抛物类,其余全同号为椭圆类,否则大概率为双曲类。
