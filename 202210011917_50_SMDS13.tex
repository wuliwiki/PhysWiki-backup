% 上海海事大学 2013 年数据结构
% keys 上海海事大学 2013 年数据结构

\subsection{一.判断题(本题10分,每小题1分)}

1.数据的逻辑结构说明数据元素之间的顺序关系,它依赖于计算机的存储结构.

2.两个栈共享-片连续内存空间时,为了提高内存利用率,减少溢出机会,应把两个栈的栈底分别设在这片内存空间的两端.

3.线性表若采用链式存储表示时所有结点之间的存储单元地址可连续也可不连续.

4.假定$n$和$m$为二叉树中的两个结点,$m$的层数大于$n$的层数,则前序遍历时$n$一定在$m$之后.

5.采用链地址法解决冲突时,若规定插入总是在链首,则插入任一个元素的时间是相同的.

6.在查找树(二叉排序树)中插入一个新结点,总是插入到叶子结点的下面.

7.用邻接矩阵存储一个图时,在不考虑压缩存储的情况下,所占用的存储空间大小与图中顶点的个数有关,而与图的边数无关.

8.排序方法是否稳定的,指的是该方法在各种情况下的时间效率是否相差不大.

9.对大小均为$n$的有序表和无序表进行顺序查找,在等概率查找的情况下,它们对于查找成功的平均查找长度是相同的,而对于查找失败的平均查找长度是不同的.

10.算法分析只要考虑算法的时间复杂度.

\subsection{二,填空题(本题30分,每空2分)}

1.分析下列程序段,其时间复杂度分别为:( (1) )、( (2) )
\begin{lstlisting}[language=cpp]
i=1;
while (i<=n*n)
    i=i+i;

int s=0;
for(i=0; i<n; i++)
    for(j=0; j<n; j++)
        s+=B[i][j];
sum=s;
\end{lstlisting}

2.在一个长度为$n$的顺序表的第$i$个元素(其中$1\leqslant i\leqslant n$)之前插入一个元素,需要后移( (3) )个元素.

3.设有一空栈, 现有输入序列1,2,3,4,5 ,经过push, push, pop, push, pop, push,push, pop, pop, pop后,输出序列是( (4) ).

4.深度为$k$的完全二叉树至少有( (5) )个结点,至多有( (6) )个结点,设根结点的层数为$1$.

5.广义表$A=((a),((b),c),(((d))))$的长度是( (7) )深度是( (8) ),取表头和表尾函数分别为$head()$和$tail()$,则$head(head(head(tail(A))))$=( (9) ) , 而从表中取出原子项$c$的运算为( (10)  ).

6.有一个二维数组A1.8]0..] ,每个数组元素占用4个存储单元,并且A[3][4]的存储地址为1080 , 若按行序为主序方式存储,数组元素A[2][3]的存储地址是__ (则)___ ;若按列序为主序方式存储,数组元素A[2][3]的存储地址是__ (2)

7.一个图的边集为{<a,c>,<a,e>,<c,f>,<d,c>,<e,b>,<e,d>} ,从顶点a出发进行深度优先搜索遍历得到的顶点序列为___ (3)__, 从项点a 出发进行广度优先搜索逼历得到的顶点序列为(4).,对该图进行拓扑排序得到的顶点序列为(5).
