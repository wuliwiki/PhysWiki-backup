% 点电荷模型与连续电荷模型的缺陷
% 点电荷模型|连续电荷模型|粗粒化|有效场论
\pentry{电场的能量\upref{EEng},电荷守恒、电流连续性方程\upref{ChgCsv}}

\subsection{点电荷模型与发散困难}
在计算电荷的固有能时,我们发现计算结果是发散的:
\begin{equation}\label{elecdb_eq1}
E=\int \dd[3]{\bvec x}\frac{\epsilon}{2} \left(\frac{e}{4\pi\epsilon_0 |\bvec x|^2}\right)^2=\frac{}{}
\end{equation}


\subsection{连续电荷模型与粗粒化}
在连续电荷模型中,我们采用点荷密度和电流密度分布的函数 $\rho,\bvec J$.这实际上是一个理想化的模型.例如当我们在考虑一个中性的材料,虽然在宏观角度上看它是不带电的(那么根据连续电荷模型,$\rho=0,\bvec J=0$,我们计算得到的电场能量\upref{EEng}应当为 $0$),然而当我们真正地去考察其微观结构,我们会发现材料是由许多带正电的离子、许多被离子束缚的电子、或自由电子组成的.当我们真正计算这些微观层面上电荷间的相互作用,我们会发现他们的累加不为 $0$.这一部分能量通常被我们称为\textbf{化学能},它是材料的一个属性.

因此在连续电荷模型中,我们是从粗粒化的角度去考察我们的物理系统,所以在距离很小的两点荷之间的相互作用的信息并没有被我们考虑进来.在粗粒化的过程中我们实际上\textbf{丢失了那些微观的物理信息},但尽管如此,许多物理量的计算仍然可以采用经典电磁学的公式,电荷守恒、电流连续性方程\upref{ChgCsv}、麦克斯韦方程组(介质)\upref{MWEq1}在\textbf{粗粒化的层面上}仍然成立.也就是说,我们不仅对电荷分布、电流分布粗粒化,还对电场、电极化强度、磁感应强度粗粒化.再考察这些宏观物理量之间的关系式,它们将满足介质中的麦克斯韦方程组,其形式与真空中的麦克斯韦方程组是相当类似的,只有几个系数上有差异.即使我们承认我们不知道那些微观的物理信息,我们仍然能够在宏观层面上得到有意义的物理结果,这正是\textbf{有效场论}的精髓.