% 分子轨道理论
% keys 分子轨道理论
% license Usr
% type Wiki

分子轨道是指,由于原子核足够接近后,原子轨道重合,各原子的轨道线性组合而成的轨道。成键遵循以下三大原则:
\begin{enumerate}
\item 对称性匹配:原子轨道(s,p,d,f...)关于点、线、面等的对称性在形成分子轨道的时候(尽量)不被破坏;
\item 能量接近:能量相近的原子轨道更容易形成分子轨道;
\item 最大重叠:轨道重叠的程度越大,最终形成的分子轨道能量越低。
\end{enumerate}

形成的线性组合轨道有以下三种:
\begin{itemize}
\item 总能量降低,使得分子更稳定的轨道,称为\textbf{成键轨道}(例如二原子分子中对应 $\psi = \psi_1 + \psi_2$);
\item 总能量升高,使得分子更不稳定,有排斥的交换作用,称为\textbf{反键轨道}(二原子分子中对应 $\psi = \psi_1 - \psi_2$);
\item 能量不变,称为\textbf{非键轨道}。
\end{itemize}

考察分子的对称性使用\textbf{分子点群}描述。处理多原子分子时,我们一般按照以下方法进行“\enref{对称性匹配的线性组合}{SALCs}(SALCs)”:
\begin{enumerate}
\item 确定分子所属的点群;
\item 根据分子的对称性与特点,拆分分子为若干小部分,分别对每个部分进行 SALCs;
\item 按照对称性匹配原则,将每一个部分的 SALCs 进一步线性组合,得到相应的分子轨道;
\item 预估能量高低,就可以通过合理安排得到分子轨道能级示意图。
\end{enumerate}

特别的,我们判断出分子的点群之后,若分子有对称中心(即对应有“i”对称操作),或分子是两个对称元素相较于一点,即可得到分子是\textbf{非极性分子}。换句话说,仅有以下几个分子点群:$C_1$,$C_s$,$C_n$ 和 $C_{nv}$ 才可能是极性分子。

同样的,只有分子不具有任何 $S_n$ 的对称操作时才可能有手性。


