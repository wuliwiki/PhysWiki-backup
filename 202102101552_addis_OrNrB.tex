% 正交归一基底
% 线性代数|矢量|正交归一基|单位正交基|克罗内克 \delta 函数|克罗内克 delta 函数

\pentry{几何矢量的基底和坐标\upref{Gvec2}, 几何矢量内积\upref{Dot}}

我们已经知道了矢量基底的概念, 如果 $N$ 维空间中一组矢量基底中的每个矢量模长都为 $1$ 且每两个矢量都正交, 则我们把这组基底称为\textbf{正交归一基(orthonormal bases)}, 也叫\textbf{单位正交基}. 若把这组正交归一基记为 $\uvec x_1,\uvec x_2\dots\uvec x_N$, 则正交归一可以用内积表示为
\begin{equation}\label{OrNrB_eq3}
\uvec x_i \vdot \uvec x_j = \delta_{ij} \qquad (i,j = 1,\dots, N)
\end{equation}
其中 $\delta_{ij}$ 是克罗内克 delta 函数\upref{Kronec}. 简单来说就是任意两个不同的基底点乘为零, 以及任意基底与自身的点成为 1.

单位正交基是某个空间的基底\upref{Gvec2}, 所以该空间中任意矢量 $\bvec v$ 在单位正交基上的展开
\begin{equation}\label{OrNrB_eq1}
\bvec v = \sum_{i = 1}^N c_i \,\uvec x_i \qquad (i = 1,\dots, N)
\end{equation}
其中 $c_i$ 就是 $\bvec v$ 的各个坐标. 由正交归一性可以证明(见下文)
\begin{equation}
c_i = \bvec v\vdot\uvec x_i \qquad (i = 1,\dots, N)
\end{equation}
或者把\autoref{OrNrB_eq1} 记为
\begin{equation}
\bvec v = \sum_{i = 1}^N (\bvec v\vdot\uvec x_i) \,\uvec x_i \qquad (i = 1,\dots, N)
\end{equation}

最常见的例子就是几何矢量在直角坐标系的 $\uvec x, \uvec y, \uvec z$ 三个单位正交矢量上的展开:
\begin{equation}
\bvec v = (\bvec v \vdot \uvec x)\,\uvec x + (\bvec v \vdot \uvec y)\,\uvec y + (\bvec v \vdot \uvec z)\,\uvec z = x \,\uvec x + y \,\uvec y + z \,\uvec z
\end{equation} 

\subsection{证明}
用 $\uvec x_k$ 乘以\autoref{OrNrB_eq1} 两边,得
\begin{equation}\label{OrNrB_eq2}
\bvec v \vdot \uvec x_k = \sum_{i = 1}^N  c_i \uvec x_i \vdot\uvec x_k = \sum_{i = 1}^N c_i \delta_{ik}  = c_k \qquad (k = 1, \dots, N)
\end{equation}
最后一步使用了\autoref{Kronec_eq1}~\upref{Kronec}. 所以\autoref{OrNrB_eq2} 中的系数有唯一确定的值 $c_k = \bvec v \vdot\uvec x_k$. 证毕.

\autoref{OrNrB_eq2} 的过程可以看作是用 “点成 $\uvec x_k$” 的操作把\autoref{OrNrB_eq2} 求和中需要的一项筛选出来. 注意只有使用正交归一基底才可以进行这样的筛选. 一个反例是斜坐标系(\autoref{Gvec2_ex1}~\upref{Gvec2})不能使用\autoref{OrNrB_eq1} 计算坐标.
