% 浙江理工大学 2013 年数据结构
% 浙江理工大学 2013 年数据结构

\subsection{一、 单选题(在每小题的四个备选答案中选出一一个正确答案.每小题2分,共20分.)}

1. 链表不具备的特点是 \\
A.可随机访问任一结点 \\
B .插入删除不需要移动元素 \\
C .不必事先估计存储空间 \\
D .所需空间与其长度成正比

2. 设线性表有n个元素,以下算法中,___在顺序表 上实现比在链表上实现效率更高. \\
A.交换第0个元素与第1个元素的值 \\
B .顺序输出这n个元素的值 \\
C .输出第i(Osisn-1)个元素值 \\
D .输出与给定值x相等的元素在线性表中的序号

3. 设输入序列为a、b、C、d ,则借助栈所得到的输出序列不可能是___. \\
A.a、b、c、d \\
B.d、c、b、a \\
C.a、c、d、b \\
D.d、a、b、c

4. 为解决计算机主机与打印机之间的速度不匹配问题,通常设计一个打印数据缓冲区, 主机将要输出的数据依次写入到该缓冲区,而打印机则依次从该缓冲区中取出数据.该缓冲区的逻辑结构应该是 \\
A.栈 $\qquad$ B.队列 $\qquad$ C.树 $\qquad$ D.图

5. 设哈夫曼树中的叶子结点总数为m ,若用二叉链表作为存储结构,则该哈夫曼树中总共有___个空指针域. \\
A.2m $\qquad$ B.4m $\qquad$ C . 2m+1 $\qquad$ D.2m-1

6. 二叉树若用顺序存储结构表示,则下列四种运算中____最容易实现. \\
A.先序遍历二叉树 $\qquad$ B.层次遍历二叉树 \\
C.中序遍历二叉树 $\qquad$ D.后序遍历二又树

7. 以下关于有向图的说法正确的是_ \\
A .强连通图是任何顶点到其他所有顶点都有边 \\
B .完全有向图一定是强连通图 \\
C .有向图中某顶点的入度等于出度 \\
D .有向图边集的子集和顶点集的子集可构成原有向图的子图

8. 若一个有向图中的顶点不能排成一一个拓扑结构序列,则可断定该有向图 \\
A.含有多个出度为0的顶点 \\
B.是个强连通图 \\
C.含有多个入度为0的顶点 \\
D.含有顶点数目大于1的强连通分量

9. 顺序查找法适合于存储结构为的线性表. \\
A.哈希存储 \\
B.压缩存储 \\
C.顺序存储或链式存储 \\
D.索引存储

10. 在所有排序方法中,关键字比较的次数与记录地初始排列次序无关的是_ \\
A.shell排序 \\
B.冒泡排序 \\
C.直接插入排序 \\
D.简单选择排序

\subsection{二、填空题(每空2分,共30分.)}

1. 下面程序段的时间复杂度是
\begin{lstlisting}[language=cpp]
for (i=0;i<n;i++)
  for (j=0j<mj++)
    A[i][j]=0;
\end{lstlisting}

2向一个不带头节点的栈指针为Ist的链式栈中插入一个*s所指节点时,则执行____ _和____.
3在二叉链表中判断某指针p所指结点为叶子结点的条件是
按_遍历一棵二叉排序树所得到的结点访问序列是一一个有序序列.
5广义表A=((a,b,c,d),( )的表尾是
6 }有一个10阶对称矩阵A ,采用压缩存储方式(以行序为主存储,且A[0][0]=1) ,则A[8][5]的地址是
7 }高度为h(>=0)的二=叉树,至少有___ 个结点,最多有_ 个结点.
8普里姆( PRIM )算法更适合于求边
_的网的最小生成树.
9在无向图G的邻接矩阵A中,若A[i][j]等于1 ,则A[]们]等于_
10在对一-组记录(54, 38, 96, 23, 15, 72, 60, 45 , 83)进行直接插入排序时,当把第7个记录60插入到
有序表时,为寻找插入位置需比较____次.
11若一组记录的排序码为( 46, 79, 56, 38, 40, 84) ,则利用堆排序的方法建立的初始堆为____.
12有一个长度为10的有序表,按折半查找法对该表进行查找,在表内各元素等概率情况下查找成功所需的平均比较次数为
