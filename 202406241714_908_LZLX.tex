% 量子力学
% license CCBYSA3
% type Wiki

(本文根据 CC-BY-SA 协议转载自原搜狗科学百科对英文维基百科的翻译)

量子力学(英语:quantum mechanics;或称量子论)是描述微观物质(原子、亚原子粒子)行为的物理学理论,量子力学是我们理解除万有引力之外的所有基本力(电磁相互作用、强相互作用、弱相互作用)的基础。

量子力学是许多物理学分支的基础,包括电磁学、粒子物理、凝聚态物理以及宇宙学的部分内容。量子力学也是化学键理论、结构生物学以及电子学等学科的基础。

量子力学主要是用来描述微观下的行为,所描述的粒子现象无法精确地以经典力学诠释。例如:根据哥本哈根诠释,一个粒子在被观测之前,不具有任何物理性质,然而被观测之后,依测量仪器而定,可能观测到其粒子性质,也可能观测到其波动性质,或者观测到一部分粒子性质一部分波动性质,此即波粒二象性。

量子力学始于20世纪初马克斯·普朗克和尼尔斯·玻尔的开创性工作,马克斯·玻恩于1924年创造了“量子力学”一词。因其成功的解释了经典力学无法解释的实验现象,并精确地预言了此后的一些发现,物理学界开始广泛接受这个新理论。量子力学早期的一个主要成就是成功地解释了波粒二象性,此术语源于亚原子粒子同时表现出粒子和波的特性。

\subsection{ 历史}

对光的波动性研究始于17世纪和18世纪,当时像罗伯特·胡克(Robert Hooke)、克里斯蒂安·惠更斯(Christiaan Huygens)和莱昂哈德·欧拉(Leonhard Euler)这样的科学家根据实验观察提出了光的波动说。[1] 1803年,英国博学家托马斯·杨( Thomas Young)在一篇题为 光和颜色的本质的论文中描述了他的著名的双缝实验。这个实验在人们普遍接受光的波动说教方面发挥了重要作用。

1838年,麦可·法拉第(Michael Faraday)发现了阴极射线。在此之后是1859年古斯塔夫·基尔霍夫(Gustav Kirchhoff)关于黑体辐射问题的陈述,1877年路德维希·玻尔兹曼(Ludwig Boltzmann)关于物理系统的能量状态可以是分立的设想,以及1900年马克斯·普朗克的量子假说。[2] 普朗克关于能量以分立的“量子”(或能量包)辐射和吸收的假设与观测到的黑体辐射模式精确匹配。

1896年,威廉·维恩凭(Wilhelm Wien)经验确定了黑体辐射的分配定律,[3] 以他的名字被称为维恩定律,从麦克斯韦方程组的角度考虑,路德维希·玻尔兹曼(Ludwig Boltzmann)也独立地得出了这个结果。然而,它只在高频部分有效,而低估了低频辐射。后来,普朗克利用玻尔兹曼对热力学的统计解释修正了这个模型,并提出了现在所谓的黑体辐射定律,这进一步推动了量子力学的发展。

继马克斯·普朗克于1900年解决了黑体辐射问题(1859年报道)之后,阿尔伯特·爱因斯坦提出了一个基于量子的理论来解释光电效应(1905年,1887年报道)的想法。 大约在1900-1910年,原子理论和光的微粒理论[4] 第一次被广泛接受为科学事实;后一种理论可以分别视为物质的量子理论和电磁辐射的量子理论。

最先研究自然界量子现象的是阿瑟·康普顿(Arthur Compton)、钱德拉塞卡拉·拉曼( C. V. Raman)和皮特·塞曼(Pieter Zeeman),他们每个人都有以他们名字命名的量子效应。罗伯特·安德鲁·密立根(Robert Andrews Millikan)通过实验研究了光电效应,阿尔伯特·爱因斯坦为此发展了一套理论。与此同时,欧内斯特·卢瑟福(Ernest Rutherford)通过实验发现了原子的核模型,为此,尼尔斯·玻尔(Niels Bohr)发展了他的原子结构理论,后来得到了亨利·莫塞莱(Henry Moseley)实验的证实。1913年,彼得·约瑟夫·威廉·德拜(Peter Debye)进一步发展了尼尔斯·玻尔的原子结构理论,引入了椭圆轨道,这一概念也是由阿诺·索末菲(Arnold Sommerfeld)提出的。[5] 这个阶段被称为旧量子论。

\begin{figure}[ht]
\centering
\includegraphics[width=6cm]{./figures/247a720099b3cf2f.png}
\caption{马克斯·普朗克被认为是量子理论之父} \label{fig_LZLX_1}
\end{figure}



















