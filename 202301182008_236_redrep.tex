% 表示的约化
% keys 等价表示|可约表示|不可约表示|完全可约表示

\begin{issues}
\issueDraft
\issueMissDepend
\end{issues}

\subsection{等价表示}
\begin{definition}{等价表示}
若$D(g)$与$D'(g)$同为群$G$的$n$维表示,且存在可逆的$n$维矩阵$C$满足于$\forall g\in G$,$C^{-1}D(g)C=D'(g)$,那么称$D(g)$与$D'(g)$为等价表示,记作$D(g)\cong D'(g)$。
\end{definition}

等价表示在本质上是对群空间中基的不同选取,定义中提到的变换矩阵$C$则恰好是不同基底之间的过渡矩阵矩阵。

\subsection{可约表示与不可约表示}

在给出可约表示的具体定义之前,先给出一个从形式上的理解:可约表示就是指那那些存在某个等价表示有如下形式的表示:

\begin{equation}
D(g)=\begin{pmatrix}
 D^1(g) & M\\
 0 & D^2(g)
\end{pmatrix}
\end{equation}

上式中的M可以为0。

验证乘法规则:

\begin{align}
D(g_1)D(g_2)&=
\begin{pmatrix}
 D^1(g_1) & M(g_1)\\
 0 & D^2(g_1)
\end{pmatrix}
\begin{pmatrix}
 D^1(g_2) & M(g_2)\\
 0 & D^2(g_2)
\end{pmatrix} \\
&=\begin{pmatrix}
 D^1(g_1)D^1(g_2) & D^1(g_1)M(g_2)+M(g_2)D^2(g_2)\\
 0 & D^2(g_1)D^2(g_2)
\end{pmatrix} 
&=
\end{align}
