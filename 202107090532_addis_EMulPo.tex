% 电多极展开
% 偶极子|多极子|球谐函数|电势

\pentry{球谐函数\upref{SphHar}}

若空间中的一个球内($r < a$) 存在静止的电荷分布 $\rho(\bvec r)$, 那么球外的电势 $V(\bvec r)$($r > a$)可以展开为径向函数和球谐函数之积的形式
\begin{equation}\label{EMulPo_eq2}
V(\bvec r) = \sum_{l = 0}^\infty \frac{1}{r^{l+1}}\sum_{m = -l}^l C_{l,m} Y_{l,m}(\uvec r)
\end{equation}
其中常数 $C_{l,m}$ 为
\begin{equation}
C_{l,m} = \frac{1}{(2l+1)\epsilon_0} \int \rho(\bvec r) r^l Y_{l,m}^*(\uvec r) \dd[3]{r}
\end{equation}

当 $m = 0$ 时, $l = 1$ 的项就是电单极子(各向同性), $l = 2$ 的项就是电偶极子\upref{eleDP2}($\propto\cos \theta$), $l = N$ 的项叫做电 $N$ 极子.

\subsubsection{外展开}
相反, 若 $\rho(\bvec r)$ 只存在于 $r > a$ 的球外空间, 那么可以把球内部的电势 $V(\bvec r)$ 展开位
\begin{equation}
V(\bvec r) = \sum_{l = 0}^\infty r^l \sum_{m = -l}^l C_{l,m} Y_{l,m}(\uvec r)
\end{equation}
其中
\begin{equation}
C_{l,m} = \frac{1}{(2l+1)\epsilon_0} \int \rho(\bvec r) \frac{1}{r^{l+1}} Y_{l,m}^*(\uvec r) \dd[3]{r}
\end{equation}



\subsection{推导}
首先我们给出单个点电荷势能的展开公式
\begin{equation}\label{EMulPo_eq1}
\frac{1}{\abs{\bvec r - \bvec r'}} = 4\pi \sum_{l=0}^{\infty} \frac{1}{2l+1} \frac{r_<^l}{r_>^{l+1}} \sum_{m = -l}^l Y_{l,m}^*(\uvec r') Y_{l,m}(\uvec r)
\end{equation}
其中 $r_< := \min\qty{r, r'}$, $r_> := \max\qty{r, r'}$.

若我们要求电荷必须在 $r' < a$ 的球内而 $V(\bvec r)$ 必须在球外计算, 那么 $r_> = r, r_< = r'$, 根据库仑势能有
\begin{equation}
V(\bvec r) = \frac{1}{4\pi\epsilon_0}\int \frac{\rho(\bvec r')}{\abs{\bvec r - \bvec r'}} \dd[3]{r'}
\end{equation}
把\autoref{EMulPo_eq1} 代入可得\autoref{EMulPo_eq2}.
