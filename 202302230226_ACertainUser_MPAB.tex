% 质点问题(摘要)

\begin{issues}
\issueDraft
\end{issues}

这是关于质点运动学与动力学问题的摘要与总结。

\subsection{参考系}
\begin{figure}[ht]
\centering
\includegraphics[width=5cm]{./figures/MPAB_1.pdf}
\caption{参考系=参考物+坐标系} \label{MPAB_fig1}
\end{figure}
我们往往需要选择一个参考系才能有效地探讨各类物理问题。参考系=参考物+坐标系。\upref{HSPM01}

\begin{figure}[ht]
\centering
\includegraphics[width=5cm]{./figures/MPAB_2.pdf}
\caption{参考系不同,对物理量的描述也可能不同} \label{MPAB_fig2}
\end{figure}
例如,“小明在教学楼往北30m的地方”这句话就包括了一个参考物“教学楼”、一个(一维的)坐标系“北”和一个单位标尺 “m”。而在另一人看来,小明可能“在食堂往东20m”。这两句话看似完全不一样,但都是正确的,只是描述时使用的参考系不同。

\subsubsection{惯性参考系}
牛顿第一定律(见下)定义了惯性参考系;相对于惯性参考系匀速运动的参考系还是惯性参考系。\upref{New3}

\subsection{质点运动学}
\begin{figure}[ht]
\centering
\includegraphics[width=5cm]{./figures/MPAB_3.pdf}
\caption{位矢、速度、加速度} \label{MPAB_fig3}
\end{figure}

质点运动学量描述了质点的位矢\upref{Disp}、速度、加速度\upref{VnA}等。
\begin{table}[ht]
\centering
\caption{质点运动学量}\label{MPAB_tab1}
\begin{tabular}{|c|c|}
\hline
名称 & 定义 \\
\hline
位矢 $\bvec r$ & (原点至质点的矢量) \\
\hline
速度 $\bvec v$& $\bvec v = \dv{\bvec r}{t}$ \\
\hline
加速度 $\bvec a$& $\bvec a = \dv{\bvec v}{t} = \dv[2]{\bvec r}{t}$ \\
\hline
\end{tabular}
\end{table}

反过来,可由加速度、速度求位矢。\upref{VnA}
$$
\bvec v = \int^{t}_0\bvec a \dd t + \bvec v_0
$$
$$
\bvec r = \int^{t}_0\bvec v \dd t + \bvec r_0
$$

$\bvec v_0, \bvec r_0$与初始条件有关。单由速度不能完全确定质点的位置。


\subsection{质点动力学}

\subsubsection{牛顿定律}
牛顿定律\upref{New3} 是经典物理中最伟大的发现(\textsl{可能没有之一})。由于太重要了,因此我不得不在这里复读一遍:

\begin{itemize}
\item \textbf{第一定律}:不受力或受合力为零的质点做匀速直线运动或静止。
\item \textbf{第二定律}:质点所受合外力等于质点的质量乘以加速度。
\item \textbf{第三定律}:两质点的相互作用力等大反向。
\end{itemize}

“力”又是一个令人费解的抽象概念,不过,你只暂时需要知道“力是物体之间的相互作用,它会改变物体的形状或运动状态;力的作用效果与力的大小、方向、作用点有关。”(反之,如果物体的形状或运动状态改变,那我们说有力作用在上面)

牛顿第二定律描述了力对物体运动状态的影响\footnote{力对物体形状的影响是个复杂得多的问题,包括弹性力学\upref{YoungM} 、塑性力学、流体力学...},联系了运动学与动力学。

\subsubsection{力的合成与分解:力的“加法”}
\begin{figure}[ht]
\centering
\includegraphics[width=8cm]{./figures/MPAB_4.pdf}
\caption{力的合成} \label{MPAB_fig4}
\end{figure}
力由矢量\upref{GVec}描述,并且符合矢量的加减法。我们通过合成或分解力来处理合力或力平衡的问题。\upref{Fdecom}

\subsubsection{质点动力学量与定理}
\begin{table}[ht]
\centering
\caption{质点系相关物理定律}\label{MPAB_tab2}
\begin{tabular}{|c|c|c|c|c|}
\hline
名称 & 公式 & 涉及的物理量1 & 涉及的物理量2 & 相应的守恒律\\
\hline
牛二 \upref{New3} & $\bvec F = m \bvec a$  & $\bvec a$是质点的加速度& $\bvec F=\sum \bvec F_i$ 是质点所受合力 & $\bvec F = \bvec 0 \Rightarrow \bvec a = \bvec 0 \Rightarrow \bvec v = \bvec v_0$ 质点所受合力为零时,保持静止或匀速运动。这就是判断质点受力平衡的重要依据。
\\
\hline
动量定理 \upref{PLaw1} & $\bvec F = \dv{\bvec p}{t}$  & $\bvec p = m \bvec v$ 质点动量是质点质量与速度的乘积 & $\bvec F=\sum \bvec F_i$ 是质点所受的合力。 & 若$\bvec F = \bvec 0$,则$\dv{\bvec P}{t} = 0$,即合力为零时,质点动量守恒。\\
\hline
角动量定理 \upref{AMLaw1} & $\bvec \tau = \dv{\bvec L}{t}$ & $\bvec L = \bvec r \times \bvec p$ 质点角动量是质点动量与位矢的叉乘 & $\bvec \tau = \sum \bvec \tau_i $是质点所受力矩和;$\bvec \tau_i= \bvec r \times \bvec F_i$ 力矩是力与位矢的叉乘\upref{Torque};& 力矩和为零时,质点角动量守恒\\
\hline
动能定理 \upref{KELaw1}& $W = \Delta E_k$  & $E_k = \frac{1}{2} m v^2$ 质点动能=1/2*质点质量*速度的平方 & 
$W = \sum W_i$是各力做的功之和;
$W_i = \bvec F_i \cdot \Delta \bvec r$ 功是力在质点位移上的“作用量”。\upref{Fwork} & 功之和为零时,质点动能守恒\\
\hline
\textsl{保守力的功-能关系} \upref{V}&$W_{CON} = -\Delta E_p$&$E_P$是势能&$W_{CON}$是保守力的功之和&$W_{C} = 0 \Rightarrow \Delta E_{P} = 0$\\
\hline
\textsl{机械能定理} \upref{ECnst}& $W_{NC} = \Delta E_{mech}$ & $E_{mech} = E_k+E_P$ 机械能是动能、势能之和 & $W_{NC}$是非保守力的功之和& $W_{NC} = 0 \Rightarrow \Delta E_{mech} = 0$
\\
\hline
\end{tabular}
\end{table}
\footnote{有些人(包括笔者)认为,“势能”源于质点和系统其余部分的相互作用,因此势能应属于整个系统,而非单独这个质点。}, 
\footnote{看起来,“合力的矩”与“力矩的和”$\bvec \tau = \bvec r \times (\sum \bvec F_i) = \sum (\bvec r \times \bvec F_i)$是一样的,但只有单个质点时才如此,在多质点体系(质点系与刚体)中只有“力矩的和”有意义。因此,本文避免了前者这种具有迷惑性的写法。}

\subsection{伽利略变换}
%或许应该先探讨惯性系间的变换...
\begin{figure}[ht]
\centering
\includegraphics[width=6cm]{./figures/MPAB_5.pdf}
\caption{伽利略变换} \label{MPAB_fig5}
\end{figure}
有时我们需要改变参考系来观察同一质点。伽利略变换\upref{GaliTr}告诉我们应该如何在不同参考系中相应地转换物理量。我们先处理惯性系之间的相互转换问题,非惯性系问题将在文末简要说明。
\begin{table}[ht]
\centering
\caption{伽利略变换及其推论}\label{MPAB_tab3}
\begin{tabular}{|c|c|}
\hline
物理量 & 公式 \\
\hline
位矢\upref{GaliTr} & $\bvec r_{K1} = \bvec r_{K2} + \bvec r_r$ \\
\hline
速度\upref{Vtrans} & $\bvec v_{K1} = \bvec v_{K2} + \bvec v_r$ \\
\hline
加速度 \upref{Iner}& $\bvec a_{K1} = \bvec a_{K2}$\\
% 加速度(惯性参考系间)\upref{AccTra}& $\bvec a_{K1} = \bvec a_{K2} $ \\
% \hline
% 加速度(参考系间仅平动)\upref{AccTra}& $\bvec a_{K1} = \bvec a_{K2} + \bvec a_r$ \\
% \hline
% 加速度(参考系间平动+转动)\upref{AccTra}& $\bvec a_{K1} = \bvec a_{K2} + \bvec a_r +2 \bvec \omega \times \bvec v_{K2}$ \\
\hline
\end{tabular}
\end{table}
注意到在任意惯性参考系中物体的加速度都一致。根据牛顿第二定律$\bvec F = m \bvec a$,这暗示了物体所受合力无关惯性参考系。\autoref{Iner_eq3}~\upref{Iner} 

\subsection{相对性原理}
在任意惯性系中,上述动力学定理始终成立。亦即在不同惯性参考系中可能观察到不同的物理量,但不会观察到不同的物理规律。\upref{RELTHM}

\subsection{非惯性参考系问题}
\begin{table}[ht]
\centering
\caption{请输入表格标题}\label{MPAB_tab4}
\begin{tabular}{|c|c|}
\hline
* & * \\
\hline
* & * \\
\hline
* & * \\
\hline
* & * \\
\hline
\end{tabular}
\end{table}
牛顿定律仅适用于惯性参考系。如果要在非惯性系中使用这些定律,则应该引入额外的假想力,例如惯性力\upref{Iner}、离心力\upref{Centri}、科氏力\upref{Corio}。

例如,在一个(相对于某一惯性参考系)向右加速的参考系看来,所有物体都似乎受一个向左的惯性力。

\subsection{一些你可能关心的其他问题}
这些结论并不是构建经典力学体系所\textsl{必需}的,但是他们很常用。

重力:$\bvec G = mgh \hat{\bvec r}$ 方向指向地面。地球引力的地表小范围近似。

引力\upref{Gravty}:$\bvec G = \frac{GMm}{r^2} \hat{\bvec r}$ 方向是两物体相互吸引的方向。

静电力:$\bvec F = \frac{kQq}{r^2} \hat{\bvec r}$ 同种电荷相互吸引、异种电荷相互排斥。\textsl{静电力与引力的相似引发了人们的无尽遐想。}

摩擦力: $\bvec f= - \mu N \hat{\bvec r}$ 指向物体(将要)运动的反方向。有时也使用 $\bvec f= - \alpha v^n \hat{\bvec v}$。

弹簧弹力与回复力: $\bvec F = - k \bvec x$ 方向指向平衡位置

支持力 $\bvec N$: 总是垂直于接触面的公切线。
