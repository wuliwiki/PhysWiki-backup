% 华罗庚(综述)
% license CCBYSA3
% type Wiki

本文根据 CC-BY-SA 协议转载翻译自维基百科\href{https://en.wikipedia.org/wiki/Hua_Luogeng}{相关文章}。

\begin{figure}[ht]
\centering
\includegraphics[width=6cm]{./figures/e19fa344575726af.png}
\caption{1956年的华罗庚} \label{fig_Luogen_1}
\end{figure}
华罗庚(Hua Luogeng 或 Hua Loo-Keng,1910年11月12日—1985年6月12日)是中国著名数学家和政治家,以在数论方面的重要贡献以及在中华人民共和国数学研究和教育方面的领导作用而闻名。他在发现和培养著名数学家陈景润方面发挥了重要作用,陈景润证明了“陈景润定理”,这是关于哥德巴赫猜想的最著名结果。此外,华罗庚后来在数学优化和运筹学方面的研究,对中国经济产生了巨大影响。他于1982年被选为美国国家科学院外籍院士。他还被选为第一至第六届全国人民代表大会常务委员会委员,第六届全国政协副主席(1985年4月),以及中国民主同盟副主席(1979年)。他于1979年加入中国共产党。

华罗庚没有接受过正式的大学教育。尽管获得了几次荣誉博士学位,但他从未获得任何大学的正式学位。实际上,他的正规教育仅限于六年的小学和三年的中学。因此,熊庆来在阅读了华罗庚的早期论文后,对华罗庚的数学天赋感到惊讶,于是1931年邀请他到清华大学学习数学。
\subsection{传记}
\subsubsection{早年(1910–1936)}  
华罗庚于1910年11月12日出生在江苏金坛。华罗庚的父亲是一位小商人。在中学时,华罗庚遇到了一位优秀的数学老师,这位老师早早发现了他的数学天赋,并鼓励他阅读一些高级的数学书籍。中学毕业后,华罗庚考入了上海的中华职业大学,并在那里通过赢得全国珠算比赛而表现突出。尽管该校的学费较低,但生活费用对于他的经济状况来说依然过高,最终华罗庚不得不提前一学期离校。未能在上海找到工作后,华罗庚于1927年返回家乡,帮助父亲经营商店。1929年,华罗庚患上了伤寒,卧床休养了半年。疾病的后果导致华罗庚的左腿部分瘫痪,从此他的行动受到严重限制,直到生命的尽头。[1]

中学毕业后,华罗庚继续自学数学,凭借他所拥有的几本书,学习了整个高中和初级大学的数学课程。当华罗庚回到金坛时,他已经开始从事独立的数学研究,他的第一篇论文《斯图尔姆定理的若干研究》发表于1929年12月的上海期刊《科学》上。次年,华罗庚在同一期刊上发表了一篇简短的文章,指出一篇声称已解决五次方程的1926年论文存在根本性缺陷。华罗庚清晰的分析引起了北京清华大学熊庆来教授的注意,1931年,尽管华罗庚没有正式的资格,而且清华的几位教职工对此也有所保留,华罗庚还是被邀请加入清华大学数学系。

在清华,华罗庚起初在图书馆担任文员,随后转为数学助教。到1932年9月,他已成为讲师,二年后,华罗庚发表了十几篇论文,晋升为讲师。

1935年至1936年,雅克·阿达马和诺伯特·维纳访问了清华大学,华罗庚热情地参加了他们的讲座,并给人留下了良好的印象。维纳随后访问了英国,并向G.H.哈代提到了华罗庚。于是,华罗庚收到了前往英国剑桥大学的邀请,并在那里停留了两年。


