% 华东师范大学 2007 年 考研 量子力学
% license Usr
% type Note

\textbf{声明}:“该内容来源于网络公开资料,不保证真实性,如有侵权请联系管理员”

\subsection{一、(每小题6分,共30分)简要回答下列问题}
\begin{enumerate}
\item 薛定谔方程是否由其它原理推导出来的?它是否适用于相对论性粒子?
\item 厄密算符具有那些特性?为什么猫写力学量的算特必须要求是厄密算符?
\item 力学量之间的对易关系是否具有传递性?即如$A$与$B$对易,且$B$与$C$对易,是否必有$A$与$C$对易?简单举例说明你的判断.
\item $a',a$分别是谐振子升降算符,$\lambda$是常数,试计算:$\left[ a,e^{\lambda a^+} \right]=?$
\item 简述斯特恩-盖拉赫(Stern-Gerlach)实验的实验现象,通过该实验有什么发现?
\end{enumerate}
\subsection{二、(10分)}
证明:对一个质量为$m$的粒子,有下式成立:
$$\frac{d}{dt} \overline{x^2} = \frac{1}{m} \overline{\left( {x} \hat{p}_x + \hat{p}_x {x}\right)}~$$
\subsection{三、(10分)}
已知一量子体系,只有两个互相正交的归一化能量本征态$|1\rangle$ 和 $|2\rangle$,若有某一可观测力学量算符$\hat{R}$,在 $|1\rangle$ 态下的几个平均值为:
$$\langle 1|\hat{R}|1\rangle = 1,\quad \langle 1|\hat{R}^2|1\rangle = \frac{5}{4},\quad \langle 1|\hat{R}^3|1\rangle = \frac{7}{4}~$$
试求定$\hat{R}$的本征值。
\subsection{(15 分)}
假设某原子中有两个价电子,同处于 $E_{nl}$ 上。按 LS 耦合方案,
$\vec{L} = \vec{l_1} + \vec{l_2}, \quad \vec{S} = \vec{s_1} + \vec{s_2}, \quad \vec{J} = \vec{L} + \vec{S}.$
    其中 $\vec{l_1}, \vec{l_2}, \vec{s_1}, \vec{s_2}$ 分别为两个电子的轨道角动量算符和自旋算符。
    
    讨论 $L, S, J$ 的可能取值,并证明 $L + S$ 必为偶数。
\subsection{(15 分)}
某体系的哈密顿算符为} $\hat{H} = K\hat{I}^2 + \omega \hat{I}_z + \lambda \hat{I}_y$, 其中 $K, \omega, \lambda $为正实数,而最后一项可视为微扰项,即 $\lambda \ll \omega, hK. \hat{I}$ 为角动量的失量算符。用微扰方法计算能级(至二级近似,不考虑偶然简并),
$$\hat{I}_x Y_{lm} = \hbar \sqrt{l(l+1) - m(m \pm 1)} Y_{l m \pm 1}, \quad \hat{I}_x = \hat{I}_x \pm i \hat{I}_y~$$.
