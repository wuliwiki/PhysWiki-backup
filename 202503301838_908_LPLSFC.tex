% 拉普拉斯方程(综述)
% license CCBYSA3
% type Wiki

本文根据 CC-BY-SA 协议转载翻译自维基百科\href{https://en.wikipedia.org/wiki/Laplace\%27s_equation}{相关文章}。

在数学和物理学中,拉普拉斯方程是一个二阶偏微分方程,得名于皮埃尔-西蒙·拉普拉斯,他在1786年首次研究了其性质。通常写作:
\[
\nabla^2 f = 0~
\]
或
\[
\Delta f = 0~
\]
其中\(\Delta = \nabla \cdot \nabla = \nabla^2\)是拉普拉斯算符,\(^\text{[注1]}\)\(\nabla\cdot\)是散度算符(也表示为“div”),\(\nabla\)是梯度算符(也表示为“grad”),\(f(x, y, z)\)是一个二次可微的实值函数。因此,拉普拉斯算符将标量函数映射到另一个标量函数。

如果右边指定为一个给定函数,\(h(x, y, z)\)则我们有
\[
\Delta f = h~
\]
这被称为泊松方程,是拉普拉斯方程的一种推广。拉普拉斯方程和泊松方程是椭圆型偏微分方程的最简单例子。拉普拉斯方程也是赫尔姆霍兹方程的特例。

拉普拉斯方程解的一般理论称为势理论。拉普拉斯方程的二次连续可微解是调和函数,\(^\text{[1]}\)这些函数在多个物理学分支中非常重要,特别是在静电学、引力学和流体动力学中。在热传导的研究中,拉普拉斯方程是稳态热方程。\(^\text{[2]}\)一般来说,拉普拉斯方程描述了平衡状态或那些不显式依赖于时间的情况。
\subsection{在不同坐标系中的形式} 
在直角坐标系中,\(^\text{[3]}\)
\[
\nabla^2 f = \frac{\partial^2 f}{\partial x^2} + \frac{\partial^2 f}{\partial y^2} + \frac{\partial^2 f}{\partial z^2} = 0.~
\]
在柱面坐标系中,\(^\text{[3]}\)
\[
\nabla^2 f = \frac{1}{r} \frac{\partial}{\partial r} \left( r \frac{\partial f}{\partial r} \right) + \frac{1}{r^2} \frac{\partial^2 f}{\partial \phi^2} + \frac{\partial^2 f}{\partial z^2} = 0.~
\]
在球坐标系中,使用\( (r, \theta, \varphi) \)约定,\(^\text{[3]}\)
\[
\nabla^2 f = \frac{1}{r^2} \frac{\partial}{\partial r} \left( r^2 \frac{\partial f}{\partial r} \right) + \frac{1}{r^2 \sin \theta} \frac{\partial}{\partial \theta} \left( \sin \theta \frac{\partial f}{\partial \theta} \right) + \frac{1}{r^2 \sin^2 \theta} \frac{\partial^2 f}{\partial \varphi^2} = 0.~
\]
更一般地,在任意曲线坐标系\( (\xi^i) \)中,
\[
\nabla^2 f = \frac{\partial}{\partial \xi^j} \left( \frac{\partial f}{\partial \xi^k} g^{kj}\right) + \frac{\partial f}{\partial \xi^j} g^{jm}\Gamma_{mn}^n = 0,~
\]
或
\[
\nabla^2 f = \frac{1}{\sqrt{|g|}} \frac{\partial}{\partial \xi^i} \left( \sqrt{|g|} g^{ij} \frac{\partial f}{\partial \xi^j} \right) = 0, \quad (g = \det \{g_{ij}\}),~
\]
其中\( g_{ij} \)是相对于新坐标的欧几里得度量张量,\( \Gamma \) 表示其克里斯托费尔符号。
\subsection{边界条件}
\begin{figure}[ht]
\centering
\includegraphics[width=6cm]{./figures/e07e8bea5b6b38d4.png}
\caption{在一个环形区域(内半径\( r = 2 \),外半径 \( R = 4 \)上的拉普拉斯方程,具有狄里克雷边界条件\( u(r=2) = 0 \) 和 \( u(R=4) = 4 \sin(5 \theta)\)} \label{fig_LPLSFC_1}
\end{figure}