% phi^4 理论的重整化(单圈修正)
% keys 重整化|单圈修正|phi^4理论|MS bar 方案|OS方案
\pentry{LSZ约化公式(标量场)\upref{LSZ},Wick 定理(标量场)\upref{wick}}
为了消除 $\phi^4$ 理论中圈图的紫外发散,我们采取一定的正规化方案(常见的有截断正规化、维数正规化、格点正规化方案,在这里我们采用的是维数正规化),并引入重整化的拉氏量:
\begin{equation}
\begin{aligned}
\mathcal{L}
&=\frac{1}{2}Z_\phi \partial_\mu \phi \partial^\mu \phi - \frac{1}{2} Z_m m^2\phi^2 - \frac{Z_\lambda \lambda}{4!}\phi^4\\
&=\frac{1}{2}(\partial_\mu\phi)^2 - \frac{1}{2}m^2\phi^2 - \frac{\lambda}{4!}\phi^4+\frac{1}{2}\delta_Z (\partial_\mu \phi)^2 - \frac{1}{2}\delta_m \phi^2 - \frac{\delta_\lambda}{4!}\phi^4
\end{aligned}
\end{equation}
其中 $\delta_Z=Z_\phi-1,\delta_m = (Z_m-1)m^2,\delta_\lambda = (Z_\lambda-1)\lambda$,它们为微扰论的 Feynman 图表示贡献了抵消项顶点。

\subsection{OS 重整化方案}
我们先来回顾一下如何从裸的拉氏量联系到重整化的拉氏量。不同的重整化方案其实时对裸场的不同的归一化。设裸场的拉氏量为 $\mathcal{L}=\frac{1}{2} (\partial_\mu \phi_0)^2 - \frac{1}{2} m_0^2 \phi^2 - \frac{\lambda_0}{4!}\phi_0^2$,对裸场而言,根据其 Lehmann-Kallen 谱表示,其两点编时格林函数在物理质量处的留数为 $Z$ 而不是 $1$,也就是说 $\bra{\Omega}\phi(x) \ket{k}=\sqrt{Z_\phi}e^{-ikx}$。而且为了消除两点函数的自能圈图的发散,裸参数 $m_0$ 也必然包含了发散的部分。
为了更利于使用微扰论求解,我们可以对裸场进行归一化得到重整化的场 $\phi_0 = \sqrt{Z_\phi}\phi$,满足 $\bra{\Omega}\phi(x)\ket{k} = e^{-ikx}$。这样做的好处是,波函数的两点格林函数在物理质量处的留数为 $1$,即波函数是良好归一化的,并且用 LSZ 约化公式计算散射振幅时每条外腿上无需再带上 $\sqrt{Z_\phi}$ 的因子。新的拉氏量为
\begin{equation}
\begin{aligned}
\mathcal{L} = \frac{1}{2}Z_\phi (\partial_\mu\phi)^2 - \frac{1}{2} Z_\phi m_0^2 \phi^2 - \frac{Z_\phi^2\lambda_0}{4!} \phi^4
\end{aligned}
\end{equation}
再经过一系列的变量替换,令 $Z_m = Z_\phi m_0^2/m^2$,$Z_\lambda = Z_\phi^2 \lambda_0/\lambda$,其中 $m,\lambda$ 为实验上在一定条件下测得的物理量,例如在这里,我们将 $m$ 取为标量粒子的物理质量,对应于两点函数的极点位置;再将$\lambda$ 取为两个非相对论极限下标量粒子相互作用的耦合常数,即 $s=4m^2,t=u=0$ 时两标量粒子的散射振幅。这种重整化方案被成为 \textbf{OS scheme}(\textbf{在壳重整化方案}),具体的重整化条件如下:

\subsubsection{重整化条件}
我们需要在维数正规化方案下,调整重整化常数来保证重整化条件的成立:
\begin{equation}
\begin{aligned}
&\text{two-point function:}\quad G^{(2)}(p^2)=\frac{i}{p^2-m^2+\epsilon}+(\text{terms regular at}\ p^2=m^2)\\
&\text{four-point function:}\quad G^{(4)}(s,t,u)_\text{amputated}=-i\lambda\quad (\text{at}\ s=4m^2,t=u=0)
\end{aligned}
\end{equation}
其中 $G^{(4)}_\text{amputated}$ 代表截肢的四点函数,即通过 LSZ 公式消去了外腿在壳时的极点行为,因此这个重整化条件的 $-i\lambda$ 正对应于实验中可观测的两个标量粒子的散射截面。通过第二个重整化条件可以确定 $\delta_\lambda$。$\delta_Z,\delta_m$ 的确定需要第一个重整化条件,通过计算自能 $M^2(p^2)$(所有单粒子不可约图的贡献为 $-iM^2(p^2)$),那么两点函数可以如下计算:
\begin{equation}
\begin{aligned}
G^{(2)}(p^2)&=\frac{i}{p^2-m^2+\epsilon}+\frac{i}{p^2-m^2+\epsilon} (-iM^2(p^2))\frac{i}{p^2-m^2+\epsilon}+\cdots\\
&=\frac{i}{p^2-m^2-M^2(p^2)+\epsilon}
\end{aligned}
\end{equation}
为了保证该函数在 $p^2=m^2$ 处有一阶极点的行为,且留数为 $1$,我们要求
\begin{equation}
\begin{aligned}
M^2(p^2)|_{p^2=m^2}=0,\quad \frac{\dd }{\dd p^2} M^2(p^2) |_{p^2=m^2}=0
\end{aligned}
\end{equation}
有了这两个方程以后,我们就可以确定 $\delta_Z$ 和 $\delta_m$。

\subsubsection{单圈修正}
计算自能到单圈:
\begin{equation}
\begin{aligned}
-iM^2(p^2)=-i\lambda \frac{1}{2}\int \frac{\dd[4]{k}}{(2\pi)^4} \frac{i}{k^2-m^2+i\epsilon} + i(p^2\delta_Z-\delta_m)
\end{aligned}
\end{equation}
这里有因子 $1/2$ ,是因为单圈自能图的对称因子为 $2$。$i(p^2\delta_Z-\delta_m)$ 为抵消项顶点的贡献。由于动量积分发散,我们采用维数正规化($\tilde{\mu}$ 是维数正规化过程中引入的一个带量纲的假参量,计算过程中忽略 $O(\epsilon)$):
\begin{equation}\label{phi4c1_eq2}
\begin{aligned}
-iM^2(p^2)&=-i\lambda \frac{1}{2} \tilde{\mu}^{4-d}\frac{1}{(4\pi)^{d/2}}\frac{\Gamma(1-d/2)}{(m^2)^{1-d/2}} + i(p^2\delta_Z-\delta_m)\\
(d\rightarrow 4)&= -i\lambda \frac{m^2}{2(4\pi)^2} \frac{1}{\epsilon/2\cdot(\epsilon/2-1)} \left(\frac{4\pi\tilde{\mu}^{2}}{m^2}\right)^{\epsilon/2}\Gamma(1+\epsilon/2) + i(p^2\delta_Z-\delta_m),\quad \epsilon=4-d\\
&=\frac{i\lambda m^2}{32\pi^2}\left(\frac{2}{\epsilon}+1+\log\left(\frac{4\pi\tilde{\mu}^2}{m^2e^{\gamma}}\right) \right)+ i(p^2\delta_Z-\delta_m),\quad \epsilon=4-d\\
&=\frac{i\lambda m^2}{16\pi^2}\left(\frac{1}{\epsilon}+\frac{1}{2}+\log\left(\frac{\mu}{m}\right) \right)+ i(p^2\delta_Z-\delta_m),\quad \epsilon=4-d
\end{aligned}
\end{equation}
这里我们定义了 $\mu^2=4\pi \tilde{\mu}^2 e^{-\gamma}$,并在 $\epsilon=4-d\sim 0$ 处进行洛朗展开得到了这一结果。

其中第一项与 $p^2$ 无关,因此我们可以在 $O(\lambda)$ 的修正下确定 $\delta_Z$ 和 $\delta_m$:
\begin{equation}
\begin{aligned}
\delta_Z &= 0\\
\delta_m &= -\frac{\lambda m^2}{2(4\pi)^{d/2}}\tilde{\mu}^{4-d} \frac{\Gamma(1-d/2)}{(m^2)^{1-d/2}}\\
&\quad (d\rightarrow 4)=\frac{\lambda m^2}{16\pi^2}\left(\frac{1}{\epsilon}+\frac{1}{2}+\log\left(\frac{\mu}{m}\right)\right)
\end{aligned}
\end{equation}
虽然在单圈图修正下 $\delta_Z=0$,但事实上,如果我们引入二圈图的修正以后,$\delta_Z=O(\lambda^2)$ 将不为零。

下面再来计算 $\delta_\lambda$,四点截肢的单圈图共有三种,分别对应 $u,s,t$ 通道的双粒子散射图(注意它们的对称因子为 $2$)。除了计算三个单圈图的贡献,还有一个抵消项顶点。所以
\begin{equation}\label{phi4c1_eq3}
\begin{aligned}
i\mathcal{M}(p_1p_2\rightarrow p_3p_4) &= -i\lambda + \frac{(-i\lambda)^2}{2} \int \frac{\dd[4]{k}}{(2\pi)^4} \frac{i}{k^2-m^2+i\epsilon} \frac{i}{(k+p_1+p_2)^2-m^2+i\epsilon}\\
&\quad\quad + (p_2\leftrightarrow -p_3)+(p_2\leftrightarrow -p_4)-i\delta_\lambda\\
&=-i\lambda + (-i\lambda)^2 [iV((p_1+p_2)^2)+iV((p_1-p_3)^2)+iV((p_1-p_4)^2)]-i\delta_\lambda\\
&=-i\lambda + (-i\lambda)^2 [iV(s)+iV(t)+iV(u)]-i\delta_\lambda
\end{aligned}
\end{equation}
因此为了满足 $s=4m^2,t=u=0$ 时的重整化条件,$\delta_\lambda$ 应取为
\begin{equation}\label{phi4c1_eq1}
\delta_\lambda = -\lambda^2[V(4m^2)+2V(0)]
\end{equation}
可以用维数正规化和费曼参数化技巧对 $V(p^2)$ 作具体的计算:
\begin{equation}
\begin{aligned}
V(p^2)&=\frac{i}{2}\tilde{\mu}^{4-d}\int_0^1 \dd x \int \frac{\dd[d]{k}}{(2\pi)^d} \frac{1}{[k^2+2xk\cdot p+xp^2-m^2+i\epsilon]^2}\\
&=\frac{i}{2}\tilde{\mu}^{4-d}\int_0^1 \dd x \int \frac{\dd[d]{k}}{(2\pi)^d} \frac{1}{[l^2+x(1-x)p^2-m^2+i\epsilon]^2}\\
&=\frac{i}{2} \cdot \tilde{\mu}^{4-d}\cdot i\cdot\frac{\Gamma(2-d/2)}{(4\pi)^{d/2}}\int_0^1 \dd x\frac{1}{(m^2-x(1-x)p^2)^{2-d/2}}\\
(d\rightarrow 4)
&=-\frac{1}{2(4\pi)^2}\int_0^1 \dd x 
\left(
    \frac{2}{\epsilon}-\gamma+\log(4\pi\tilde{\mu}^2)-\log[m^2-x(1-x)p^2]
    \right) ,\quad \epsilon =(4-d)\\
&=-\frac{1}{32\pi^2}\int_0^1 \dd x 
\left(
    \frac{2}{\epsilon}+\log\left(\frac{\mu^2}{m^2-x(1-x)p^2}\right)
    \right) ,\quad \epsilon =(4-d)
\end{aligned}
\end{equation}
将这个结果代入\autoref{phi4c1_eq1} 可以得到
\begin{equation}
\begin{aligned}
\delta_\lambda (d\rightarrow 4)=\frac{\lambda^2}{32\pi^2}\int_0^1\dd x \left(\frac{6}{\epsilon}
 +3\log(\mu^2) - \log[m^2-x(1-x)4m^2]
-2\log[m^2]
\right)
\end{aligned}
\end{equation}

对于其他给定的 $s,t,u$,利用以上结果可以求出两个标量粒子的散射振幅,由于所有的耦合常数都被设置得满足重整化条件,圈图中紫外发散的部分被抵消项相消,我们最终得到的 Feynman 矩阵元的结果是有限的:
\begin{equation}
\begin{aligned}
i\mathcal{M}(s,t,u)&=-i\lambda - \frac{i\lambda^2}{32\pi^2}\int_0^1 \dd x\left(\log\left(\frac{m^2-x(1-x)s}{m^2-x(1-x)4m^2}\right)\right.
\\
&\quad\quad\left.+
\log\left(\frac{m^2-x(1-x)t}{m^2}\right)
+
\log\left(\frac{m^2-x(1-x)u}{m^2}\right)
\right)
\end{aligned}
\end{equation}

\subsection{$\overline{MS}$重整化方案}

除去上面讨论的 OS 方案,可以有许多不同的重整化方案来确定重整化常数。$\overline{MS}$ 方案被称为修正的最小减除方案,“最小减除”意味着重整化常数包含着恰好抵消紫外发散的项。根据\autoref{phi4c1_eq2} 和\autoref{phi4c1_eq3} 的计算结果,在 $\overline{MS}$ 方案中重整化常数取为
\begin{equation}
\begin{aligned}
&Z^{\overline{MS}}_{\phi}=1+O(\lambda^2)\\
&Z^{\overline{MS}}_{m}=1+\delta^{\overline{MS}}_m/m^2 = 1+\frac{\lambda}{16\pi^2\epsilon}+O(\lambda^2),\\
&Z^{\overline{MS}}_{\lambda} = 1+\delta_\lambda^{\overline{MS}}/\lambda = 1+\frac{3\lambda}{16\pi^2\epsilon}+O(\lambda^2)
\end{aligned}
\end{equation}
需要注意的是,在 $\overline{MS}$ 方案中,两点编时格林函数在物理质量处的留数不为 $1$,而是 $Z=Z^{OS}_\phi/Z^{\overline{MS}}_\phi$,这可以从 $\phi^{\overline{MS}}$ 与裸场之间的关系看出:
\begin{equation}
\phi_0 = \sqrt{Z_\phi^{\overline{MS}}} \phi^{\overline{MS}} = \sqrt{Z_\phi^{OS}} \phi^{OS},\quad \Rightarrow \quad \phi^{\overline{MS}} = \sqrt{Z} \phi^{OS}
\end{equation}
因此在 LSZ 约化公式中,每个外腿将对应 $\sqrt{Z}$ 的因子。

接下来考虑 $\phi^4$ 理论在 $\overline{MS}$ 方案下的单圈修正。自能函数是
\begin{equation}
-iM^2(p^2)=\frac{i\lambda m^2}{32\pi^2}\left(1+2\log\left(\frac{\mu}{m}\right)\right)+O(\lambda^2)
\end{equation}
因此
\begin{equation}
m^2_\text{ph}=m^2-\frac{\lambda m^2}{32\pi^2}\left(1+2\log\frac{\mu}{m}\right)+O(\lambda^2)
\end{equation}
并且由于单圈修正下自能函数不是 $p^2$ 的函数,所以 $Z=(1-\frac{\dd}{\dd p^2}M^2(p^2))^{-1}= 1+O(\lambda^2)$。在单圈修正下 $Z=1$。此外我们可以看到 $m$ 是 $\mu$ 的函数。

再计算两体散射的 Feynman 振幅:
\begin{equation}
\begin{aligned}
i\mathcal{M}(s,t,u)=-i\lambda + \frac{i\lambda^2}{32\pi^2}\int_0^1 \dd x
\left(
\log\left(\frac{\mu^2}{m^2-x(1-x)s}\right)+
(s\leftrightarrow u)+(s\leftrightarrow t)+O(\lambda^3)
\right)
\end{aligned}
\end{equation}
由于 $|\mathcal{M}|^2$ 是可观测量,它应当不随重整化方案的变化而变化。这告诉我们不仅 $m$ 是 $\mu$ 的函数,$\lambda$ 也是 $\mu$ 的函数。我们通常取 $\mu^2\approx s^2$,其能标也就是散射过程的能标,于是 $\lambda$ 是随着能标 $\mu$ 跑动的,这被称为跑动耦合常数。