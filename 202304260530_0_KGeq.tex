% Klein-Gordon 方程
% keys 量子场论|高等量子力学|克莱因-戈登方程|克莱茵-戈登方程|d'Alembert operator|达朗贝尔算子|达朗贝尔算符

\addTODO{预备知识需要薛定谔量子力学的相关内容,但现在该部分还未整理好,不宜引用。}

\pentry{自然单位制、普朗克单位制\upref{NatUni},爱因斯坦求和约定\upref{EinSum}}

Klein-Gordon方程(以下简称“K-G方程”)的命名源自两位物理学家Oskar Klein和Walter Gordon,他们于1926年指出该方程能够描述狭义相对论中的电子。虽说实际上,电子这样有自旋的费米子应该用Dirac方程来描述,但K-G方程依然成功地描述了相对论性的无自旋复合粒子。

%Jier:有论文讨论K-G方程的四元数形式,不知是否有纳入百科的必要。我个人觉得这种形式实用性不咋地,也不简洁。


\subsection{问题的引入}

Schrödinger方程(以下以通译称“薛定谔方程”)在量子力学中的地位,就像牛顿三定律在经典力学中的地位一样,是描述理论结构的“公理”。因此,如果要了解量子力学的局限性,可以从研究薛定谔方程本身入手。

\subsubsection{质能关系问题}

回顾单粒子薛定谔方程的表达(注意这里使用了\textbf{自然单位制}\upref{NatUni}):
\begin{equation}\label{eq_KGeq_1}
\qty(-\frac{\nabla^2}{2m}+V)\psi = \I \partial_t \psi~.
\end{equation}

由于量子力学假设$\hat{\bvec{p}}=-\I\nabla$和$\hat{E}=\I\partial_t$分别是动量、能量算子,故\autoref{eq_KGeq_1} 左边体现的是经典力学中的\textbf{哈密顿量}:

\begin{equation}\label{eq_KGeq_2}
\begin{aligned}
\hat{H}=-\frac{\nabla^2}{2m}+V &= \frac{\hat{\bvec{p}}^2}{2m}+V ~,\\
&\updownarrow\\
H &= \frac{\bvec{p}^2}{2m}+V~.
\end{aligned}
\end{equation}

\autoref{eq_KGeq_2} 上下两部分含义完全不同\footnote{上面一行各项是算符,它们作为量子态之间线性变换的\textbf{特征值}才是能量、动量等可观测量;下面一行各项就是实数,本身即为能量、动量等可观测量。},但其描述的能量-动量-质量关系是一致的。因此薛定谔方程本质上是经典力学的推广,与经典时空观契合,但与相对论时空观矛盾。

\subsubsection{粒子数守恒问题}

回顾量子力学的概率守恒。取薛定谔方程的复共轭,得

\begin{equation}\label{eq_KGeq_3}
\qty(-\frac{\nabla^2}{2m}+V)\psi^* = -\I \partial_t \psi^*~.
\end{equation}

在\autoref{eq_KGeq_1} 上乘以$\psi^*$,再减去\autoref{eq_KGeq_3} 乘以$\psi$ ,得

\begin{equation}\label{eq_KGeq_4}
\begin{aligned}
\psi^*\qty(-\frac{\nabla^2}{2m}+V)\psi - \psi\qty(-\frac{\nabla^2}{2m}+V)\psi^* &= \I\qty(\psi^*\partial_t\psi + \psi\partial_t\psi^*)\\
-\psi^*\frac{\nabla^2}{2m}\psi + \psi\frac{\nabla^2}{2m}\psi^* &= \I \partial_t\qty(\psi\psi^*)\\
\partial_t\qty(\abs{\psi}^2) + \frac{-\I\nabla}{2m}\qty(\psi^*\nabla\psi-\psi\nabla\psi^*) &= 0\\
\partial_t \rho + \frac{\nabla}{2m} \qty(\psi^*\hat{\bvec{p}}\psi + (\psi^*\hat{\bvec{p}}\psi)^*) &= 0~.
\end{aligned}
\end{equation}
其中$\rho=\abs{\psi}^2$可以理解为粒子的空间位置分布,即粒子数密度。

对于动量本征态容易验证,\autoref{eq_KGeq_4} 相当于
\begin{equation}\label{eq_KGeq_5}
\partial_t\rho + \nabla\cdot(\rho \bvec{v}) = 0~,
\end{equation}
其中$\bvec{v}=\bvec{p}/m=\psi^*\hat{\bvec{p}}\psi/m$是粒子的速度。对于非本征态也有类似的阐释,因为任何量子态都是动量本征态的叠加。

\autoref{eq_KGeq_5} 意味着任意空间区域内粒子随时间增加的速率,恰为粒子从外部进入该区域的速率,即整个宇宙中粒子数守恒。于是,薛定谔方程无法描述粒子数变化的现象,如质子和电子结合成中子的过程中,质子和电子的数目减少,中子的数目增多。

特别要注意的是,在推导\autoref{eq_KGeq_4} 的过程中,我们假设$V$是\textbf{实数}。如果$V$可以取\textbf{复数},那么实际上能导出粒子数消失或产生的结果,这可以\textbf{唯象}地解释粒子数不守恒的情况,如对核反应的描述。



\subsection{Klein-Gordon方程}

相对论的成功以及自然界广泛存在的粒子数改变的现象,都表明我们必须改变薛定谔方程的形式,才能扩展量子理论的适用范围。Klein-Gordon方程即是一个良好的扩展。

\subsubsection{方程的导出}

我们考虑从质能关系切入。相对论中的质能关系为
\begin{equation}
E^2=\bvec{p}^2+m^2~.
\end{equation}
用它代替经典力学的$E=p^2/2m+V$,代入量子力学的算符假设$\hat{\bvec{p}}=-\I\nabla$和$\hat{E}=\I\partial_t$,得到一个方程:
\begin{equation}\label{eq_KGeq_6}
\qty(-\nabla^2+m^2)\psi = -\partial_t^2 \psi~.
\end{equation}

如果使用抽象指标来表示,取$\eta_{\mu\nu}=\opn{diag}(-1, 1, 1, 1)$,则\autoref{eq_KGeq_6} 也表达为
\begin{equation}\label{eq_KGeq_7}
\partial_\mu\partial^\mu \psi = m^2\psi~.
\end{equation}

如果取$\eta_{\mu\nu}=\opn{diag}(1, -1, -1, -1)$,则\autoref{eq_KGeq_6} 应表达为
\begin{equation}\label{eq_KGeq_8}
\partial_\mu\partial^\mu \psi = -m^2\psi~.
\end{equation}
这仅仅是\textbf{号差}选择的习惯问题。

\autoref{eq_KGeq_6} 及其抽象指标表达式\autoref{eq_KGeq_7} 和\autoref{eq_KGeq_8} 被称为闵可夫斯基时空中的\textbf{Klein-Gordon方程}。

我们也可以用\textbf{达朗贝尔算子(d'Alembert operator)}表示K-G方程。达朗贝尔算子是拉普拉斯算子$\nabla^2$在闵可夫斯基时空中的推广,其定义为
\begin{equation}
\square = \partial_t^2-\nabla^2~,
\end{equation}


于是\autoref{eq_KGeq_6} 又可以表示为
\begin{equation}
\square \psi + m^2\psi = 0~.
\end{equation}
注:使用正\textbf{四}边形是为了表示这是在\textbf{四}维时空中。以上是自然单位制的表述,回归国际单位制后应有
\begin{equation}
\square = \frac{1}{c}\partial_t^2-\nabla^2~,
\end{equation}
此外,由于质量项的算符量纲应为长度平方的倒数,所以回归国际单位制后为
\begin{equation}
m\rightarrow \frac{mc}{\hbar}
\end{equation}


达朗贝尔算子又称“波算子(wave operator)”,因为$\square \psi=0$正是经典的线性机械波方程。从这个视角看自然会发现$m^2\psi$是“多出来”的一项,可以认为是一种“势”$V(\psi)$,这样我们就可以把K-G方程拓展为更一般的形式
\begin{equation}
\square\psi + V(\psi) = 0~.
\end{equation}

除了$m^2\psi$以外,实标量场$\psi$在相互作用理论里也有一种常见的势:$V(\psi)=\frac{1}{2}m^2\psi^2+\lambda\psi^4$。





\subsubsection{连续性方程}

类似处理薛定谔方程的方法,我们给K-G方程左乘一个$\psi^*$,再取结果的复共轭,相减:
\begin{equation}\label{eq_KGeq_9}
\begin{aligned}
\psi^*\qty(-\nabla^2+m^2)\psi-\psi\qty(-\nabla^2+m^2)\psi^* &= -\psi^*\partial_t^2 \psi+\psi\partial_t^2 \psi^*\\
-\psi^*\nabla^2\psi+\psi\nabla^2\psi^* &= -\psi^*\partial_t^2\psi+\psi\partial_t^2\psi^*\\
\nabla\cdot\qty(\psi^*\nabla\psi-\psi\nabla\psi^*) &= \partial_t\qty(\psi^*\partial_t\psi-\psi\partial_t\psi^*)\\
\partial_\mu\qty(\psi^*\partial^\mu\psi-\psi\partial^\mu\psi^*) & =0~,
\end{aligned}
\end{equation}

\autoref{eq_KGeq_9} 的最后一步与$\eta_{\mu\nu}$的号差选择无关。

记$\rho=\frac{\I}{2m}\qty(\psi^*\partial_t\psi-\psi\partial_t\psi^*)$和$\bvec{j}=-\frac{\I}{2m}\qty(\psi^*\nabla\psi-\psi\nabla\psi^*)$,则\autoref{eq_KGeq_9} 还可以写成
\begin{equation}\label{eq_KGeq_10}
\partial_t\rho+\nabla\cdot\bvec{j} = 0~,
\end{equation}
这意味着$\bvec{j}$是$\rho$的流。

K-G方程关于变量$t$是二阶的,因此$\psi$和$\partial_t\psi$是相互独立的初值,因此可以自由选择$\rho$作为位置的函数是正值还是负值。因负值的存在,把$\rho$诠释为概率密度显然是不妥当的。

我们也可以用更统一的形式来表述\autoref{eq_KGeq_10} 。令$J^\mu=\frac{\I}{2m}\qty(\psi^*\partial^\mu\psi-\psi\partial^\mu\psi^*)$,则\autoref{eq_KGeq_10} 写为
\begin{equation}
\partial_\mu J^\mu = 0~.
\end{equation}



\subsubsection{自由解}

无外力作用下的自由K-G方程有两个线性无关的特解:
\begin{equation}
\psi(t, \bvec{x}) = \E^{\I Et-\I \bvec{p}\cdot\bvec{x}}~,
\end{equation}

其中$E=\pm\sqrt{\bvec{p}^2+m^2}$。





\subsubsection{拉格朗日形式}

\pentry{从分析力学到场论\upref{CFa1}}

K-G方程描述了场的运动(变化)规律,采用了微分语言进行描述,但也可以借助拉格朗日函数或者说拉格朗日作用量来描述。

一个质量为$M$的复标量场$\psi$的Klein-Gordon作用量为
\begin{equation}
\int {\dd}^4 x \frac{1}{2}\qty(\partial^\mu\psi\partial_\mu\psi^*-M^2\psi\psi^*)~.
\end{equation}

$\psi$的能动张量可以由拉格朗日密度算出:

\begin{equation}
T^{\mu\nu} = 2\partial^\mu\psi^*\partial^\nu\psi-\eta^{\mu\nu}\qty(\partial^\rho\psi^*\partial_\rho\psi-M^2\psi^*\psi)~.
\end{equation}
\addTODO{缺乏预备知识:如何用拉格朗日密度算出能动张量。考虑在《能动张量》中讲。}


\subsubsection{弯曲时空上的K-G方程}

\autoref{eq_KGeq_7} 和\autoref{eq_KGeq_8} 在形式上可以直接推广到任意时空流形上:
\begin{equation}
\nabla_\mu\nabla^\mu=m^2\psi
\end{equation}
和
\begin{equation}
\nabla_\mu\nabla^\mu=-m^2\psi~.
\end{equation}
其中$\nabla_\mu$是该时空中的联络,$\nabla^\mu=g^{\mu\nu}\nabla_\nu$





\subsection{Klein-Gordon场}\label{sub_KGeq_1}

本小节讨论一些满足K-G方程的场,即Klein-Gordon场。


























