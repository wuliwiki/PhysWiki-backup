% 傅科摆(科普)
% license Usr
% type Art

\begin{issues}
\issueOther{需要配图}
\end{issues}

\subsection{简介}
19 世纪中叶,人们对地球的自转有了初步的理解,但直接证明地球自转仍然是一个挑战。 1851年,莱昂·傅科在巴黎的巴拿赛神殿中挂起了一根长达 67 米的钢丝,其末端悬挂着一个重达 28 公斤的铅球。 当他让铅球沿一个初始方向摆动时,摆动的平面逐渐旋转,从而直接展示了地球自转的影响。这个简单而又直观的实验让傅科摆成为了地球物理学和天文学领域的一个重要里程碑。

傅科摆为什么能说明地球在自传? 最简单的方法是设想我们把傅科摆放在南极或北极, 摆动的平面会如何变化? 此时傅科摆的摆动平面将会在 24 小时内完成一整圈的旋转, 直接反映了地球自转的周期。 这是因为在北极(或南极),水平面与傅科摆的摆动平面垂直。

如果我们将傅科摆放置在赤道上,情况就完全不同了。 无论开始时摆动沿哪个方向,傅科摆的摆动平面实际上不会出现旋转现象。 这也是容易理解的,最简单的两种情况是南北和东西方向的摆动。

傅科摆最有趣的情况在于,如果我们把它放在介于南北极和赤道之间的某个纬度,它摆动方向相对于地面的转动是多快呢?物理学中把转动的快慢叫做\textbf{角速度}。不难猜测,这个角速度介于赤道和极点之间,也就是介于几乎不转动和每天转动一圈。 巧妙的是,它每天转动的圈数恰好是当地纬度的正弦值 $\sin\theta$。\footnote{公式 $\sin\theta$ 的理论计算中,假设了地球是完美的均匀球体,实际实验结果和该公式存在非常微小的偏差,本文予以忽略。}该公式给出的定量转速既符合理论计算,也符合实验结果,这更强有力说明地球是一个自转的近似球体。 因为如果只是简单观察到摆锤缓慢偏转,不能排除是微风或其他干扰造成的。 例如杭州的纬度大约是北纬 30°,那么在杭州的傅科摆将会每天转动 $\sin 30^\circ = 1/2$ 圈。 又例如哈尔滨大约是北纬 45°,那么每天将转动 $\sin 45^\circ = 1/\sqrt{2}\approx 0.71$ 圈。

在以上的讨论中,我们要区分两种转动,一种是单摆的位置沿地球纬线划出的圆周运动,在一天当中,该圆周运动必定会完成整整一圈的转动,该转动的转速和所在纬度无关。另一种转动是单摆的方向和水平地面之间的相对转动,这才是我们讨论傅科摆时关心的。

\subsection{转椅问题}
事实上,傅科摆的角速度公式本质上是一个球面上的几何问题,而不是一个单摆的动力学问题。为了说明这点,我们可以通过另外一个有趣的模型来推导傅科摆的角速度公式。

假设地球是一个坚硬光滑的球体且静止不动,而我们有一个办公室的转椅,那么当我们企图把转椅朝任意方向沿着 “直线” 推动的时候,它其实是在延球体的某个大圆运动,也就是 “测地线”。 测地线是一段圆弧,它所在的圆必定经过地球的球心。 例如在所有纬线中,只有赤道是大圆和测地线。 如果我们在转椅上放一个单摆,那么一个几乎显然的结论是,当转椅沿任意测地线运动时,无论单摆朝向如何,它相对于转椅表面都不会改变。而当我们原地转动转椅时,转动多少度,单摆相对转椅的朝向就改变同样的角度。

现在,我们需要近似沿着任意一根纬线推该转椅,但我们规定必须把该纬线看成若干段长度相等的测地线拼接而成,也就是用球面上的正多边形去近似这根纬线。当我们沿着每段测地线前进时,单摆和椅子表面不发生相对转动。 而每当我们到达多边形的拐角处,需要原地扭动转椅使其正面朝着下一个前进方向。而每当我们这么做的时候,单摆相对于转椅的相对角度也会改变同样的角度。 当这个正多边形的边无限缩短,这个正多边形也无限趋近于纬线。当转椅回到起点时,我们在每个拐角处扭动的角度之和,就是单摆相对于转椅转动的角度,可以证明该角度就是 $\sin\theta$。 具体的推导所需要的数学已经超出本文范围,详见傅科摆\upref{Fouclt}。

我们还是使用北极和赤道这两个极端的例子。 如果我们的纬线和多边形选在北极点附近,半径只有几米,那么这附近的地面几乎是一个平面,而我们知道平面上多边形的外角之和恒为 360°, 所以无论我们取几条边,在所有拐点处我们需要扭动的角度之和就是 360°。 而如果把纬线选在赤道, 那么无论我们使用几边形,这些边都属于同一条测地线,也就是说它每个 “拐角” 处需要原地转动的角度为零!
