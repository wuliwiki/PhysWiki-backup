% 磁场的能量
% keys 磁场|能量密度|麦克斯韦方程组
%未完成 不要写那么深! 先从两片面电流开始推导! (包括介质)

\begin{issues}
\issueTODO
\issueMissDepend
\end{issues}

\pentry{磁矢势\upref{BvecA}}

磁场 $\bvec B(\bvec r)$ 的能量密度为
\begin{equation}\label{BEng_eq1}
W = \frac{1}{2\mu_0} \int \bvec B^2 \dd{V}~.
\qquad 
\text{或}
\qquad
W = \frac12 \int \bvec A \vdot \bvec J \dd{V}
\end{equation} 
其中 $\mu_0$ 是真空中的磁导率, $\bvec A$ 是磁场矢势\upref{BvecA}, $\bvec J$ 是电流密度,积分是对全空间积分(或者对被积函数不为零的空间积分)。
\subsection{幼稚的推导}
\pentry{电感\upref{Induct}}
首先我们根据能量守恒的思想,假设给一个电感\upref{Induct} $l$ 充电的能量都以“磁场能”的形式储存起来,且每个点的能量密度只是磁场强度绝对值的函数。

在\autoref{AmpLaw_ex2}~\upref{AmpLaw}中, 我们知道螺线管中的磁场强度为
\begin{equation}\label{BEng_eq8}
B = \mu_0 nI
\end{equation}
该螺线管的电感为
\begin{equation}\label{BEng_eq9}
L = \mu_0n^2lS
\end{equation}
该螺线管的能量\upref{Induct}为
\begin{equation}\label{BEng_eq10}
E_B = \frac{1}{2}L I^2
\end{equation}
将\autoref{BEng_eq8} ,\autoref{BEng_eq9} 代入 \autoref{BEng_eq10} 
\begin{equation}
E_B=\frac{1}{2} \mu_0n^2lS (\frac{B}{\mu_0 n})^2=\frac{1}{2\mu_0} B^2 lS
\end{equation}
$lS$即为螺线圈的体积,写为积分形式,即为
\begin{equation}
W = \frac{1}{2\mu_0} \int \bvec B^2 \dd{V}
\end{equation}
这就是螺线圈中磁场的能量。

\subsection{简单的推导}
 我们首先考虑一个单匝线圈的磁场能量
\begin{figure}[ht]
\centering
\includegraphics[width=12cm]{./figures/BEng_1.pdf}
\caption{单匝线圈} \label{BEng_fig1}
\end{figure}

假设 $t = 0$ 时 $I = 0$, 此时没有磁场,磁场能量为零。

接下来将线圈接入外部电源,令 $I$ 随着 $t$ 慢慢增加。变化的电流激发变化的磁场\upref{AmpLaw},而变化的磁场又产生一个反向电动势\upref{FaraEB}。反向电动势为(定义与电流相同的方向为正)
\begin{equation}\label{BEng_eq2}
\varepsilon  =  - \dv{\Phi}{t} =  - \dv{t} \int \bvec B \vdot \dd{\bvec s} 
=  - \dv{t} \int \qty(\curl \bvec A) \vdot \dd{\bvec s} 
=  - \dv{t} \oint \bvec A \vdot \dd{\bvec l}
\end{equation}
电源克服反电动势的功率为
\begin{equation}\label{BEng_eq3}
\dv{W}{t} =  - \varepsilon I = I\dv{t} \int \bvec B \vdot \dd{\bvec s} = \int I \dv{\bvec B}{t} \vdot \dd{\bvec s}
\end{equation}
从能量守恒角度看,这部分功率对应的能量转换为了磁场的能量\footnote{自感对电流的阻碍作用或许让你联想到电阻的阻碍作用,但这两者有一些微妙的不同。在电阻中,电能转换为热能;而在自感线圈中,电能转换为了磁场能。}。由于磁场与电流成正比(见比奥萨伐尔定律\upref{BioSav})
,不妨设 $\bvec B = \bvec bI$ 。则
\begin{equation}\label{BEng_eq4}
I \dv{\bvec B}{t} = \bvec bI \dv{I}{t} = \frac{\bvec b}{2} \dv{I^2}{t} = \dv{t} (\frac12 I\bvec B)
\end{equation}
所以
\begin{equation}\label{BEng_eq5}
\dv{W}{t} = \dv{t} \qty(\frac12 \int I\bvec B \vdot \dd{\bvec s})
\end{equation}
注意两边都是对时间的导数。两边对时间积分,得
\begin{equation}\label{BEng_eq6}
W = \frac12 \int I\bvec B \vdot \dd{\bvec s}
\end{equation}
注意当 $I = 0$ 时 $ W = 0$, 所以积分常数为零。注意在上述过程中,并没有假设电流以什么样的函数随时间变化(只要是缓慢变化即可)。
\begin{equation}\label{BEng_eq7}
W = \frac{I}{2} \int \bvec B \vdot \dd{\bvec s}  = \frac{I}{2} \int \curl \bvec A \vdot \dd{\bvec s}  = \frac12 \oint I\bvec A \vdot \dd{\bvec l}
\end{equation}
\addTODO{推导}
 