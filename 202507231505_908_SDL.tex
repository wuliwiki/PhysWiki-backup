% 素数定理(综述)
% license CCBYSA3
% type Wiki

本文根据 CC-BY-SA 协议转载翻译自维基百科\href{https://en.wikipedia.org/wiki/Prime_number_theorem}{相关文章}。

在数学中,素数定理描述了素数在正整数中的渐近分布情况。它形式化地表达了一个直观的观点:随着数值的增大,素数变得越来越稀疏,并且精确地量化了这一稀疏现象发生的速度。

该定理由雅克·阿达马和夏尔·让·德拉瓦莱-普桑于1896年各自独立证明,所用的方法基于伯恩哈德·黎曼引入的一些思想,尤其是黎曼ζ函数。最早发现的分布形式为:$\pi(N) \sim \frac{N}{\log N}$其中,$\pi(N)$ 表示素数计数函数,即不超过 $N$ 的素数个数,$\log N$ 是 $N$ 的自然对数。这意味着对于足够大的 $N$,从不超过 $N$ 的整数中随机选一个数是素数的概率大约为 $1 / \log N$。换句话说,在前 $N$ 个整数中,相邻两个素数之间的平均间隔大约为 $\log N$。

因此,一个最多有 $2n$ 位数的随机整数(当 $n$ 足够大时)成为素数的可能性,大约是一个最多有 $n$ 位数的随机整数的一半。例如,在所有最多有 1000 位的正整数中,大约每 2300 个数中有一个是素数(因为 $\log(10^{1000}) \approx 2302.6$);而在最多有 2000 位的正整数中,大约每 4600 个数中才有一个是素数(因为 $\log(10^{2000}) \approx 4605.2$)。
\subsection{定理的表述}
\begin{figure}[ht]
\centering
\includegraphics[width=8cm]{./figures/17ef6d25124d34a0.png}
\caption{该图显示了素数计数函数 $\pi(x)$ 与其两个近似函数 $x / \log x$ 和 $\mathrm{Li}(x)$ 的比值。随着 $x$ 的增大(注意:横轴为对数刻度),这两个比值都趋近于 1。其中,$\pi(x) / (x / \log x)$ 的比值从上方非常缓慢地收敛,而 $\pi(x) / \mathrm{Li}(x)$ 的比值则从下方收敛得更快。} \label{fig_SDL_1}
\end{figure}
设 $\pi(x)$ 为素数计数函数,定义为不超过实数 $x$ 的素数个数,例如 $\pi(10) = 4$,因为有四个素数(2, 3, 5 和 7)不超过 10。

那么,素数定理指出:$\frac{x}{\log(x)}$ 是 $\pi(x)$ 的一个良好近似,其含义是,当 $x$ 趋于无穷大时,函数 $\pi(x)$ 与 $\frac{x}{\log(x)}$ 的商的极限为 1,即:
$$
\lim_{x \to \infty} \frac{\pi(x)}{\left[\frac{x}{\log(x)}\right]} = 1~
$$
这被称为素数分布的渐近法则。

使用渐近记号,该结果可重新表述为:
$$
\pi(x) \sim \frac{x}{\log x}~
$$
这里的渐近记号(以及该定理)并不表示 $\pi(x)$ 与 $\frac{x}{\log x}$ 的差在 $x$ 趋于无穷大时有极限,而是表示后者对前者的相对误差趋近于 0。
\begin{figure}[ht]
\centering
\includegraphics[width=8cm]{./figures/c7e868ce31e62b68.png}
\caption{对数-对数图显示了 $x / \log x$ 与 $\mathrm{Li}(x)$ 这两个对素数计数函数 $\pi(x)$ 的近似所产生的绝对误差。与比值不同的是,$\pi(x)$ 与 $x / \log x$ 之间的差值会随着 $x$ 的增大而无限增长。另一方面,$\mathrm{Li}(x) - \pi(x)$ 的符号会无限次地发生变化。} \label{fig_SDL_2}
\end{figure}
素数定理等价于如下陈述:第 $n$ 个素数 $p_n$ 满足:
$$
p_n \sim n \log(n)~
$$
这里的渐近记号同样表示:当 $n$ 趋于无穷大时,该近似的相对误差趋于 0。例如,第 $2 \times 10^{17}$ 个素数是 8512677386048191063,而 $(2 \times 10^{17}) \log(2 \times 10^{17})$ 的值为约 7967418752291744388,相对误差约为 6.4\%。
另一方面,以下渐近关系在逻辑上是等价的\(^\text{[5]: 80–82 }\):
$$
\lim_{x \to \infty} \frac{\pi(x) \log x}{x} = 1, \quad \text{以及} \quad \lim_{x \to \infty} \frac{\pi(x) \log \pi(x)}{x} = 1.~
$$
如下面将要概述的,素数定理也等价于以下公式:
$$
\lim_{x \to \infty} \frac{\vartheta(x)}{x} = \lim_{x \to \infty} \frac{\psi(x)}{x} = 1,~
$$
其中 $\vartheta(x)$ 和 $\psi(x)$ 分别是第一和第二切比雪夫函数。

还等价于:
$$
\lim_{x \to \infty} \frac{M(x)}{x} = 0,~
$$
其中:
$$
M(x) = \sum_{n \leq x} \mu(n)~
$$
是梅滕斯函数,$\mu(n)$ 是莫比乌斯函数\(^\text{[5]: 92–94 }\)。
\subsection{素数渐近分布定律的证明史}
根据安东·费尔克尔和尤里·维加编制的素数表,阿德里安-马里·勒让德在1797年或1798年猜想,素数计数函数 $\pi(a)$ 可由函数 $a / (A \log a + B)$ 近似,其中 A 和 B 是未具体说明的常数。在他1808年出版的第二版《数论》一书中,他提出了更精确的猜想,设定 $A = 1$,$B = -1.08366$。卡尔·弗里德里希·高斯大约在1792或1793年、年仅15或16岁时,也考虑过同样的问题(这是他1849年回忆中提到的)。1838年,彼得·古斯塔夫·勒热讷·狄利克雷提出了他自己的近似函数,即对数积分 $\mathrm{li}(x)$(以略有不同的级数形式,并与高斯通信交流)。勒让德与狄利克雷的公式都蕴含了前文所述的猜想:$\pi(x)$ 与 $x / \log(x)$ 渐近等价,尽管后来的研究发现,如果从差值(而非比值)角度来看,狄利克雷的近似要显著更好。

1848年和1850年,俄国数学家帕夫努季·切比雪夫发表了两篇论文,试图证明素数分布的渐近规律。他的工作值得注意之处在于引入了黎曼ζ函数 $\zeta(s)$(实数 $s$ 的情形),类似于莱昂哈德·欧拉早在1737年的用法。切比雪夫的论文早于黎曼1859年著名的论文,他成功地证明了一个稍弱的形式:如果极限 $\lim_{x \to \infty} \pi(x) / (x / \log x)$ 存在,那么该极限值必定等于1\(^\text{[7]}\)。此外,切比雪夫无条件地证明了该比值对充分大的 $x$ 被夹在两个常数之间,即:$0.92129 < \frac{\pi(x)}{x / \log x} < 1.10555$\(^\text{[7]}\)【8】【9】。尽管切比雪夫的论文尚未给出素数定理的完整证明,但他对 $\pi(x)$ 的估计已足以证明“伯特兰猜想”:对于任意整数 $n \geq 2$,在区间 $(n, 2n)$ 内必定存在至少一个素数。

