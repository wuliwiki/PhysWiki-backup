% 电磁场中的狄拉克方程
% keys 电磁场|狄拉克方程|狄拉克场

\pentry{狄拉克方程\upref{qed4},狄拉克场\upref{Dirac}}

\subsection{量子电动力学的拉格朗日量}
让我们从狄拉克场\upref{Dirac}的拉氏量出发:
\begin{equation}
\mathcal{L}=\bar\psi (i\gamma^\mu \partial_\mu - m)\psi 
\end{equation}
我们已经知道了该方程在整体 $U(1)$ 规范变换下是不变的,即 $\psi\rightarrow e^{i\alpha}\psi,\bar\psi \rightarrow \bar\psi e^{-i\alpha}$ 变换下拉氏量保持不变.根据诺特定理\footnote{可以参考经典场论基础\upref{classi}.},整体规范变换对称性导致狄拉克场的电荷守恒.现在我们希望进一步地引入\textbf{定域规范不变性}.即 $\psi\rightarrow e^{i\alpha(x)}\psi $,其中 $\alpha(x)$ 为时空坐标的函数.然而,引入这样的规范变换以后,拉氏量不再保持不变,会多出一项 $-\bar\psi \gamma^\mu \qty[\partial_\mu \alpha(x)]\psi$.所以为了抵消掉这一项,我们约定 $\partial_\mu$ 在定域规范变换下变为 $\partial_\mu-i[\partial_\mu \alpha(x)]$.

因此,我们定义协变导数 $D_\mu = \partial_\mu +iA_\mu(x)$,其中 $A_\mu(x)$ 是一个矢量场.