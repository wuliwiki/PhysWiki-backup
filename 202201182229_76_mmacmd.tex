% Mathematica 脚本模式
% keys Mathematica|命令行脚本|wolframscript

\begin{issues}
\issueDraft
\issueTODO
\end{issues}

\subsection{命令行二进制程序}

\begin{lstlisting}[language=bash]
file -L \verb`which wolframscript`
wolframscript --help
\end{lstlisting}

\subsection{命令行选项} 

\verb\verb`wolframscript --help` 的输出如下:

辅助:
\begin{itemize}
\item \verb\verb`-v, -verbose`; 在执行过程中打印附加信息.
\item \verb\verb`-h, -help` ; 打印帮助文本.
\item \verb\verb`-version`; 打印 WolframScript 的版本.
\end{itemize}
运行模式类:
\begin{itemize}
\item \verb\verb`-c, -code WL`; 给出要执行的 Wolfram Language 代码.
\item \verb\verb`-f, -file PATH`; 给出要执行的包含 Wolfram 语言代码的 \verb\verb`文件`.
\item \verb\verb`-a, -api URL|UUID`; 使用指定 \verb\verb`URL` 的 \verb\verb`API`, 或来自指定 \verb\verb`UUID` 的云或本地对象. 在'-args'之后, 以 \verb\verb`key=value` 的形式提供参数.
\item \verb\verb`-fun, -function WL`; 使用函数, 参数为使用 \verb\verb`-args` 给出的 \verb\verb`字符串`.并将参数解析为 \verb\verb`-signature` 给出的类型.
\item \verb\verb`--, -args ARGS...`; 与 \verb\verb`-api` 或 \verb\verb`-function` 一起使用, 以提供参数.
\item \verb\verb`-s, -signature TYPE...`; 与 \verb\verb`-function` 和 \verb\verb`-args` 一起使用, 为提供的参数指定解释器类型.
\end{itemize}
其他:
\begin{itemize}
\item \verb\verb`-charset ENCODING`; 使用\verb\verb`ENCODING`进行输出. 编码可以是 \verb\verb`None`, 用于输出原始字节,
或是 \verb\verb`$CharacterEncodings` 中的任何条目, 除了 \verb\verb`Unicode`. 默认情况继承终端的 \verb\verb`语言设置` 中的值.
\item \verb\verb`-format TYPE`; 指定输出的格式. 可以使用 \verb\verb`Export` 所理解的任何格式.
\item \verb\verb`-print [all]`; 当运行 \verb\verb`脚本` 时, 打印脚本 \verb\verb`最后一行` 的结果. 如果给了 \verb\verb`all` 参数, 则打印 \verb\verb`每一行`.
\item \verb\verb`-linewise`; 执行读取到的 \verb\verb`标准输入` 中的每行代码.
\item \verb\verb`-script ARGS...`; 与 \verb\verb`wolfram -script` 相对应, 作用是设置 \verb\verb`$ScriptCommandLine`.
\end{itemize}

\subsection{Wolfram内部预定义变量}

常用调用方式

建立后缀名为 \verb`.wl`/\verb`.wls` 的文件, 然后按平常写 \verb`Mathematica` 笔记本的语法编写脚本.
但最好用 \verb`字符输出函数` 代替那些, 炫酷的 \verb`Box` 输出格式的函数, 如 \verb`TableForm`.
运行时, 用 \verb`wolframscript` 唤起脚本(如 windows 为 \verb`wolframscript.exe`).

假如当前目录有脚本 \verb`test.wl`, 用类似下面的方式运行:

```bash
wolframscript -script test.wl  --Lbd-num '0.90' --Lbd-fit '0.90'  --ord '\$ord0'
```

其中 \verb`--Lbd-num '0.90' --Lbd-fit '0.90'  --ord '\$ord0'` 将作为参数提供给脚本 \verb`test.wl`, 可在脚本内部使用.

---

+ 注意 \verb`$ScriptCommandLine` 中的参数, 其类型均为 \verb`String`.
也就是命令行接收到的 \verb`参数`, 会被强制转码为 \verb`字符串`.
所以上面收到的参数列表为:

    ```mathematica
    {"./test.wl", "--Lbd-num", "0.90", "--Lbd-fit", "0.90", "--ord", "$ord0"}
    ```

    如果需要使用 \verb`数字` 或其他类型的 \verb`值`, 需要在脚本内部自行 \verb`ToExpression`.

+ 此外, 如果命令行参数中包含 \verb`$`, 需要用 \verb`单引号 + 反斜杠`转义(escape).
例如\verb`$ord` -> \verb`'\$ord0'`.

---

在 \verb`WolframScript` 开始执行时, 会预定义一些变量.
因此在脚本内部, 可以从这些 \verb`预设变量` 读取 \verb`shell` 中提供的参数.

+ \verb`$CommandLine`; \verb`字符串列表`, 给出运行 \verb`WolframKernel` 使用的 \verb`完整命令行`.
+ \verb`$ScriptCommandLine`; 为 \verb`正运行的脚本` 准备的 \verb`命令行参数列表`.
这些参数出现在 \verb`-option` 给出的选项之后.

+ \verb`$ScriptInputString`; 代表在 \verb`标准输入通道` 上对 \verb`原始操作系统命令` 的输入,
正运行的 Wolfram Language 实例即来自 \verb`此命令的调用`(invoke)
脚本迭代一次, \verb`-linewise` 选项就把 \verb`标准输入` 的一行载入该变量.
可在交互式脚本中使用.

在上面的例子中, 三个变量被填充如下, 不同平台可能稍有不同

```wolfram
$CommandLine->{/usr/local/Wolfram/Mathematica/12.2/SystemFiles/Kernel/Binaries/Linux-x86-64/WolframKernel,
-wlbanner,-script,test.wl,--,test.wl,--Lbd-num,0.90,--Lbd-fit,0.90,--ord,$ord0}

$ScriptCommandLine->{test.wl,--Lbd-num,0.90,--Lbd-fit,0.90,--ord,ord0}

$ScriptInputString->None
```

+ 也可用 \verb`wolfram` 命令运行脚本, 但是 \verb`wolfram` 不会设置 \verb`$ScriptCommandLine`, 只会填充 \verb`$CommandLine`:

```bash
wolfram -script test.wl  --Lbd-num '0.90' --Lbd-fit '0.90'  --ord '\$ord0'
```


### Unix

\verb`Unix` 还可以加上 \verb`Shebang` 行. 即在脚本文件 **首行** 添加

```wolfram
#!/usr/bin/env wolframscript [其他选项]
```

运行的时候, 不需要先输入 \verb`wolframscript`, 传递参数的方法和上面相同.

```bash
./test.wl para1 para2
wolframscript -file ./init.wl &> ~/test/log.txt & # 在后台运行, 把输出重定向到日志文件 log.txt
```

经常用到的\verb`mma`系统变量,参考

+ \verb`guide/SystemInformation` : mma 系统信息
+ \verb`guide/WolframSystemSetup`: 更一般的系统设置

+ \verb`$InputFileName`: 脚本的绝对路径.
+ \verb`$Notebooks`: 如果是用前端运行的, 则为\verb`True`.
+ \verb`$BatchInput`: 输入是否来自批处理
+ \verb`$BatchOutput`:如果在命令行中输出, 则为\verb`True`.
+ \verb`$CommandLine`: 唤醒环境变量所使用的命令行,
+ \verb`$ProcessID`:进程ID
+ \verb`$ParentProcessID`:
+ \verb`$Username`: 用户的登陆名
+ \verb`Environment["var"]`:操作系统的环境变量, 如\verb`Environment["HOME"]`