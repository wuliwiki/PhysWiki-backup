% OpenVPN 笔记
% license Xiao
% type Note

\begin{issues}
\issueDraft
\end{issues}

\subsection{搭建服务器}
\begin{itemize}
\item bandwagon 服务器亲测成功
\item 注意以下脚本需要访问 github, 如果在国内可能用不了(需要手动修改脚本)
\item 服务器上, 跑\href{https://github.com/MacroUniverse/openvpn-install/blob/master/openvpn-install.sh}{这个脚本}(\verb`./openvpn-install.sh`), 然后按提示做即可。 如果不确定, 就选默认。
\item 如果跑成功, 会生成一个 \verb`.ovpn` 配置文件(文本文件)。 在命令行中 \verb`cat` 出来, 然后复制下来创建新文件即可。
\item 在 Ubuntu 的 VPN 设置中, 导入该文件, 开关打开即可开心上网。可以在线搜索一下自己的 ip 地址是否是服务器的。
\item 在服务器检查 vpn 服务状态 \verb`sudo systemctl status openvpn`
\end{itemize}

\subsection{客户端}
\begin{itemize}
\item 在 Windows 上, 需要下载 OpenVPN 的客户端才可以导入文件。
\item Windows 客户端的设置文件默认路径为 \verb`C:\Program Files\OpenVPN\config\`
\item Ubuntu Desktop 设置中原生支持通过 \verb`.ovpn` 文件设置 VPN
\item Ubuntu 命令行(未测试):\verb`sudo apt install openvpn`, \verb`sudo openvpn --config your-vpn-file.ovpn`, 要断开连接直接 \verb`Ctrl+C` 即可。 这个命令只对当前的 ssh session 起作用,如果远程 ssh,断开后 vpn 也会自动断开。
\item 使用 ubuntu vpn clien 不会影响局域网 ssh 连接到这台电脑。
\end{itemize}
