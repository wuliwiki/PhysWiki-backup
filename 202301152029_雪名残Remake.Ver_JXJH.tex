% 解析几何
\begin{aligned}
解析几何初步
\end{aligned}
本章我将会在高中教材的基础上,探究在平面直角坐标系上,直线、圆、和圆锥曲线的一些性质,并解决部分中、高考问题
\begin{aligned}
1.一次函数与直线方程
\end{aligned}
\begin{example}{}
\begin{itemize}
\item 在平面直角坐标系中画出y=3x+1的图像
\item 用方程表示平面直角坐标系中过点(1,1),(3,2)的直线
\end{itemize}
\end{example}
初中课本曾用描点法刻画了一次函数的图像,在这里我将用另外的方法刻画一次函数图像(此时我们不知道一次函数图像是一条直线)

解:(1)令$y=0$得$x=-\frac{1}{3}$,因此图像过点A:($-\frac{1}{3}$,0)
设一点P:($x_0$,$y_0$)是图像上异于($-\frac{1}{3}$,0)的点,所以$y_0=3x_0+1\Rightarrow \frac{y_0}{x_0-\frac{1}{3}}=3$,当$y_0>0$时,$tan\angle PAO=3$,当$y_0<0$时,$tan\angle PAO=-3$由此可知,无论$x_0$取任何实数值,P都在一条固定的直线上,于是作过点($-\frac{1}{3}$,0)、(0,1)的直线即为y=3x+1的图像

(2)设点Q:($x_0$,$y_0$)在该直线上.点M:(1,1)、N:(3,2).由向量三点共线公式:$\overrightarrow{OQ}=\lambda \overrightarrow{OM}+(1-\lambda)\overrightarrow{ON}$
\\ 
$\left\{\begin{aligned}
   & x_0=\lambda+3(1-\lambda)\\
   & y_0=\lambda+2(1-\lambda)
\end{aligned}\right.$
\\ $\therefore 
\left\{\begin{aligned}
&x_0=3-2\lambda\\
&y_0=2-\lambda
\end{aligned}\right.$
\\ $\therefore y_0=\frac{1}{2}x_0+\frac{1}{2}$
\\由于Q可以是直线上任意一点,所以所有该直线上的点,都应满足方程:$y=\frac{1}{2}x+\frac{1}{2}$
\\ $\star$(图一、图二暂定)$\star$


上面我用了两种方法实现了平面直角坐标系中直线和方程的转化,除此之外,我们还可以用“变化率”(其实就是导数,这里只作简单论述)来解释,如(1)中y=3x+1,每当x增加$\Delta x$时,y都会同时增加$\Delta y=3\Delta x$,在我对于(1)的论证中则表现为“$y_0>0$时$tan\angle PAO=3$”,即倾斜角正切值(或斜率)为3.

下面介绍直线方程的不同形式
\begin{itemize}
\item 斜截式
y=kx+b.
\\式中的一次项系数“k”代表直线方程的斜率$^{^{(1)}}$,常数项b则是直线在y轴上的截距$^{^{(2)}}$ 
\\ $\star$(图三暂定)$\star$
\item 点斜式
$y-y_0=k(x-x_0)$
\\式中k代表直线方程的斜率,$x_0$、$y_0$代表直线经的点($x_0$,$y_0$).不难发现,当该点为(0,b)的时候,此式形同y=kx+b
\item 截距式
$\frac{x}{a}+\frac{y}{b}=1$
\item 一般式
$Ax+By+C=0$
此式代表的几何意义不那么明显,但是可以表示平面直角坐标中的所有直线
\end{itemize}
上述四个式子是中学最常见的表示直线的式子,还有一些不太常见的,如法线式、参数式、行列式等,这里不作介绍.
\begin{issues}
需要注意的是,这四个式子中,只有一般式可以代表平面内所有直线,一、二受到斜率限制,无法表示垂直于x轴的直线(那样的话k=$\pm \infty$),而三受分母的限制,无法表示过原点O的直线和垂直或平行于x轴的直线.
\end{issues}
$\star$存档线----------------------
\begin{exercise}{几种直线方程的利用}
1.求过点(6,2)且斜率为4的直线方程


2.求直线$\frac{x}{7}+\frac{y}{6}=1$斜率

3.两直线$l_1:x+my+6=0,l_2:(m-2)x+3y+2m=0,$求当m____时,$l_1$、$l_2$相交.
\end{exercise}
$\star$存档线---------

直线系方程
我们一般把具有同等性质的直线的集合(如过同一个点、斜率相同等)称为一个直线系