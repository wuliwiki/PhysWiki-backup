% 电子
% license CCBYSA3
% type Wiki

(本文根据 CC-BY-SA 协议转载自原搜狗科学百科对英文维基百科的翻译)

\textbf{电子}是一种亚原子粒子,符号是$e^-$或者$\beta^-$,其电荷量是负一个单位的基本电荷。[1]电子属于第一代轻子族,且通常被认为是基本粒子,因为它们没有已知的成分或亚结构。电子的质量大约是质子质量的1/1836。 电子的量子力学性质包括半整数值的内禀角动量(自旋),单位为约化普朗克常数ħ。根据泡利不相容原理,作为费米子,没有两个电子可以占据相同的量子态。像所有基本粒子一样,电子表现出波粒二象性:它们可以与其他粒子发生碰撞,也可以像光一样发生衍射。比其他粒子(例如中子和质子)更容易通过实验观察到,因为电子的质量更低,对于给定的能量电子拥有更大的德布罗意波长。

电子在许多物理现象中起着重要作用,如电学、磁学、化学和热导率,电子还参与重力、电磁相互作用和弱相互作用。[2]因为电子带有电荷,所以它周围有一个电场,如果电子相对于某个观察者发生运动,这个观察者将观察到它产生一个磁场。其他来源所产生的电磁场将影响电子的运动,这种影响由洛伦兹力定律所描述。当电子被加速时,它们以光子的形式辐射或吸收能量。实验室仪器能够通过使用电磁场囚禁单个电子和电子等离子体。特殊的望远镜可以探测外层空间的电子等离子体。电子涉及许多应用,例如电子学、焊接、阴极射线管、电子显微镜、放射疗法、激光器、气体电离探测器和粒子加速器。

电子与其他亚原子粒子的相互作用在化学和核物理等领域都很有意义。原子核内带正电荷的质子和没有原子核的负电荷电子之间的库仑力相互作用允许它们共同组成原子。电离或负电荷电子与正电荷原子核之间的电荷量差异改变了原子系统的结合能。两个或多个原子之间的电子交换或共享是化学键形成的主要原因。1838年,英国自然哲学家理查德·拉明首先假设了一个不可分割的电荷量的概念来解释原子的化学性质。爱尔兰物理学家乔治·约翰斯顿·斯通尼在1891年将这种电荷命名为“电子”,而约瑟夫·汤姆孙和他的英国物理学家团队在1897年将它确定为粒子。电子也可以参与核反应,例如恒星中的核合成,在其中电子被称为贝塔粒子。电子可以通过放射性同位素的β衰变以及高能碰撞产生,后者的一个例子发生在宇宙射线进入大气层的时候。电子的反粒子被称为正电子;它与电子拥有很多相同的性质,除了它携带有与电子相反的电荷以及其他的荷。当一个电子与正电子碰撞时,两个粒子可以发生湮灭,并产生$\gamma$射线光子。

\subsection{历史}
\subsubsection{1.1 电力效应的发现}
古希腊人注意到琥珀在摩擦皮毛后会吸引小物体。与闪电一样,这种现象是人类最早记录的电学经验之一。[3]在1600年的论文《De Magnete》中,英国科学家威廉·吉尔伯特创造了新拉丁语术语electrica来指代那些性质类似琥珀、摩擦后会吸引小物体的物质。[4]electric和electricity这两个词都源自拉丁语ēlectrum(这也是同名合金的名称由来),来源于希腊语中的“琥珀”ἤλεκτρον(ēlektron)。
\subsubsection{1.2 两种电荷的发现}
17世纪初,法国化学家Charles Franç ois du Fay发现如果带电的金箔被用丝绸摩擦过的玻璃所排斥,那么同样的带电金箔就会被用毛皮摩擦过的琥珀所吸引。从这个和其他类似实验的结果中,他得出结论:电由两种电流体所组成,分别是来自被丝绸摩擦过的玻璃的玻璃液体以及来自被毛皮摩擦过的琥珀的树脂液体。这两种流体混合时可以互相中和。[4][5]美国科学家埃比尼泽·金纳斯利后来也独立得出了同样的结论。[6]十年后本杰明·富兰克林提出电不是来自不同类型的电流体,而是表现出过量(+)或不足(-)的单一电流体。他将这两种情况分别命名为正和负,这也是现代的电荷命名规则。[7]富兰克林认为电荷载体是正的,但他没有正确地识别哪种情况是电荷载体的盈余、哪种情况是不足。[8]

1838年至1851年间,英国自然哲学家理查德·拉明发展了原子由被亚原子粒子包围的物质核心所构成的观点,这些包围核心的亚原子粒子具有单位数量的电荷。[9]从1846年开始,德国物理学家威廉·韦伯建立了电流由带正电荷和负电荷的流体组成的理论,它们的相互作用受平方反比定律支配。在1874年研究了电解现象后,爱尔兰物理学家乔治·约翰斯顿·斯通尼提出存在“单一确定量的电”,即单价离子的电荷。借助于法拉第电解定律,他能够估算出这个基本电荷的数值e。[10]然而,斯通尼认为这些电荷永久地附着在原子上并且无法移除。1881年,德国物理学家赫尔曼·赫尔姆霍茨认为正电荷和负电荷都被分成基本部分,每个都“表现得像电的原子”。[11]

斯通尼在1881年创造了术语electrolion。十年后,他转而用electron来描述这些基本电荷,他在1894年写道:“...有人估计了这个最引人注目的基本电学单位的实际量,后来我大胆地提出了这个名字电子(electron)”。1906年有人提议改用electrion一词,但这个提议没有成功,这是因为亨德里克·洛伦兹更愿意继续使用electron这个词。[11][12]electron这个词是单词electric和ion的组合。[13]来自electron的后缀-on现在被用来为其他亚原子粒子命名,如质子或中子。[14][15]
\subsubsection{1.3 物质外自由电子的发现}
约瑟夫·汤姆孙对电子的发现与许多物理学家几十年来对阴极射线的实验和理论研究密切相关。[11]在1859年研究稀薄气体的电导率时,德国物理学家朱利叶斯·普吕克尔观察到由阴极发射的辐射引起的磷光出现在阴极附近的管壁上,并且磷光区域可以通过施加磁场来移动。1869年,普吕克的学生希托夫发现置于阴极和磷光体之间的固体会在管的磷光区域上投下阴影。希托夫推断阴极发出笔直的射线,而磷光是由撞击管壁的射线所引起的。1876年,德国物理学家尤金·戈德斯坦证明这些射线被垂直于阴极表面发射出来,这区分了从阴极发射的射线和白炽光。戈德斯坦称这种射线为阴极射线。[16] [17]

19世纪70年代,英国化学家和物理学家威廉·克鲁克斯爵士开发了第一个内部具有高度真空的阴极射线管。[18]接着,他在1874年证明阴极射线可以转动放在它们的路径上的一个小桨轮。因此,他得出结论:这种射线带有动量。此外,通过施加磁场,他能够偏转这种射线,从而证明这种射线表现得好像带了负电荷。[17]1879年,他提出这些性质可以通过将阴极射线视为由处于第四种物质状态的带负电的气态分子组成来解释,在这里粒子的平均自由程太长以致于它们之间的碰撞可以被忽略。[16]

德国出生的英国物理学家阿瑟·舒斯特扩展了克鲁克斯的实验,他将金属板相对阴极射线平行放置,并在金属板之间施加电势。电场将射线偏转向带正电荷的板上,这进一步证明了这种射线带有负电荷。通过测量给定水平的电流的偏转量,舒斯特在1890年就能够估计出射线成分质荷比。然而,这产生了一个比预期大一千多倍的数值,所以当时他的计算很少得到认可。[17]

1892年,亨德里克·洛伦兹提出这些粒子(电子)的质量可能是它们电荷的结果。[19]

