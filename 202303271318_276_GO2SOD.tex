% 化一般常微分方程组为标准方程组(常微分方程)
% 一般常微分方程|标准方程组

\pentry{基本知识(常微分方程)\upref{ODEPr}}
如\autoref{ODEPr_sub1}~\upref{ODEPr}所说,存在及唯一定理是对标准常微分方程组所证明的,而一般常微分方程组都可化为这一方程组,本节就来给出这一转化程序。

\subsection{一个常微分方程化为标准微分方程组}
在仅含一个常微分方程的情形,即将一般常微分方程
\begin{equation}\label{GO2SOD_eq1}
F(x,y,y',\cdots,y^{(n)})=0~.
\end{equation}
化为具有标准常微分方程组\autoref{ODEPr_def1}~\upref{ODEPr}的形式:
\begin{equation}\label{GO2SOD_eq3}
y_i'=f_i(x,y_i,\cdots,y_m),\quad i=1,\cdots,m~.
\end{equation}

我们总可以从方程\autoref{GO2SOD_eq1} 的已解出最高阶导数的方程开始,即
\begin{equation}\label{GO2SOD_eq2}
y^{(n)}=f(x,y,y',\cdots,y^{(n-1)})~.
\end{equation}
这是因为,从\autoref{GO2SOD_eq1} 到\autoref{GO2SOD_eq2} 的过程不属于微分方程领域,而属于函数论领域。这一过程中,某些问题被归结为微分方程领域,比如\autoref{GO2SOD_eq1} 中关于 $y^{(n)}$ 是二次的,因此 $y^{(n)}$ 是其余变量的二值函数(假设这二值不同),实际上得到的是两个形如\autoref{GO2SOD_eq2} 的微分方程组 $y_1^{(n)},y_2^{(n)}$。但是,\autoref{GO2SOD_eq1} 将 $y_1,y_2$ 并成一个 $y$ 了,不能分解成两个形如\autoref{GO2SOD_eq2} 的方程组 ,这就需要讨论\autoref{GO2SOD_eq1} 。关于这种方程的研究,引向微分方程的\textbf{奇解}概念,而我们并不讨论这一问题。故这里只需记住,从\autoref{GO2SOD_eq1} 总能获得\autoref{GO2SOD_eq2} 。
\begin{theorem}{}\label{GO2SOD_the1}
方程
\begin{equation}\label{GO2SOD_eq4}
y^{(n)}=f(x,y,y',\cdots,y^{(n-1)})~.
\end{equation}
与标准方程组
\begin{equation}\label{GO2SOD_eq5}
\leftgroup{
&z_1'=z_2\\
&z_2'=z_3\\
&\cdots\\
&z_{n-1}'=z_n\\
&z_n'=f(x,z_1,\cdots,z_n)
}~,
\end{equation}
等价。

其中,
\begin{equation}\label{GO2SOD_eq6}
z_1=y,\;z_2=y',\;\cdots,\;z_n=y^{(n-1)}~.
\end{equation}
\end{theorem}
\textbf{证明:}
1。\autoref{GO2SOD_eq4} $\Rightarrow$ \autoref{GO2SOD_eq5} 

设 $y$ 满足\autoref{GO2SOD_eq4} ,对\autoref{GO2SOD_eq6} 求导,得到
\begin{equation}\label{GO2SOD_eq7}
z_i'=y^{(i)},\quad i=1,\cdots,n~.
\end{equation}
将\autoref{GO2SOD_eq6} 和\autoref{GO2SOD_eq4} 代入\autoref{GO2SOD_eq7} 右端,即得\autoref{GO2SOD_eq5}。

2。\autoref{GO2SOD_eq5} $\Rightarrow$ \autoref{GO2SOD_eq4} 

设 $z_1,\cdots,z_n$ 满足\autoref{GO2SOD_eq5} ,取 $z_1=y$,得
\begin{equation}
z_2=y'~.
\end{equation}
如此迭代入 \autoref{GO2SOD_eq5} ,就有
\begin{equation}\label{GO2SOD_eq8}
z_i=y^{(i-1)},\quad i=1,\cdots,n~.
\end{equation}
\autoref{GO2SOD_eq8} 代入\autoref{GO2SOD_eq5} 最后一式,即得\autoref{GO2SOD_eq4} 。

\textbf{证毕!}

\subsection{一般常微分方程组化为标准微分方程组}
基于同样的理由,对于一般的常微分方程,我们总能从解出最高阶导数的方程组出发,即
\begin{equation}
\begin{aligned}
&y_i^{(n_i)}=f_i\qty(x,y_1,y_1',\cdots,y_1^{(n_1-1)},\cdots,y_m,y_m',\cdots,y_m^{(n_m-1)}),\\
& i=1,\cdots m
\end{aligned}
\end{equation}
对一般的情形,有下面的定理成立
\begin{theorem}{}\label{GO2SOD_the2}
方程组
\begin{equation}\label{GO2SOD_eq9}
\begin{aligned}
&y_i^{(n_i)}=f_i\qty(x,y_1,y_1',\cdots,y_1^{(n_1-1)},\cdots,y_m,y_m',\cdots,y_m^{(n_m-1)}),\\
& i=1,\cdots m
\end{aligned}
\end{equation}
等价于标准方程组
\begin{equation}
\leftgroup{
&z_1'=z_2\\
&z_2'=z_3\\
&\cdots\\
&z_{n_1-1}'=z_{n_1}\\
&z_{n_1}'=f_1(x,z_1,\cdots,z_{\sum_{i=1}^{m}n_i})\\
&\cdots\\
&z_{\sum_{i=1}^{m-1}n_i+1}'=z_{\sum_{i=1}^{m-1}n_i+2}\\
&z_{\sum_{i=1}^{m-1}n_i+2}'=z_{\sum_{i=1}^{m-1}n_i+3}\\
&\cdots\\
&z_{\sum_{i=1}^{m}n_i-1}'=z_{\sum_{i=1}^{m}n_i}\\
&z_{\sum_{i=1}^{m}n_i}'=f_m(x,z_1,\cdots,z_{\sum_{i=1}^{m}n_i})
}~.
\end{equation}

其中
\begin{equation}\label{GO2SOD_eq10}
\leftgroup{
&z_1=y_1,\cdots,z_{n_1}=y_1^{(n_1-1)}\\
&z_{n_1+1}=y_2,\cdots,z_{n_1+n_2}=y_2^{(n_2-1)}\\
&\cdots\\
&z_{\sum_{i}^{m-1}n_i+1}=y_m,\cdots,z_{\sum_{i}^{m}n_i}=y_m^{(n_m-1)}
}~,
\end{equation}
\end{theorem}
证明和\autoref{GO2SOD_the1} 完全类似。

由常微分方程组阶的定义\autoref{ODEPr_sub2}~\upref{ODEPr},可知方程组\autoref{GO2SOD_eq9} 是 $m$ 个关于 $y_i$ 为 $n_i$ 阶的 $\sum\limits_{i=1}^{m}n_i$ 阶微分方程组,而\autoref{GO2SOD_eq10} 是 $\sum\limits_{i=1}^{m}n_i$ 个关于 $z_i$ 为1阶的 $\sum\limits_{i=1}^m n_i$ 阶的微分方程组。而这两方程组等价。