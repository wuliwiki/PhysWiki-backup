% 量子力学中的数学笔记
% 量子力学|有界算符|对称算符|自伴算符|希尔伯特空间

\begin{itemize}
\item 柯西完备: 如果所有柯西级数的极限仍然落在同一个矢量空间中, 这个矢量空间就是柯西完备的. 直觉上来说, 柯西完备的矢量空间中不应该有缺失的点, 例如有理数的集合中缺失了无理数的点.
\item Hilbert 空间定义: 有内积, 所以可以测量长度(度量空间)和角度, 而且是柯西完备的.
\item 所有平方可积函数并不是 Hilbert 空间, 而是 Hilbert 空间的一个特例. 这个特例中内积才有我们熟悉的定义.
\item 任何算符都可以表示为用 kernel 积分的形式(如果 kernel 是广义函数, 例如 $\delta$ 函数)
\item \textbf{有界算符}: 如果一个算符作用再 Hilbert 空间中任意模长为 1 的矢量上得到的矢量模长都为有限, 那么这个算符就是有界的.
\item \textbf{伴随算符}: $\braket{A^* \psi}{\phi} = \braket{\psi}{A \phi}$
\item \textbf{对称算符}: $\braket{A \psi}{\phi} = \braket{\psi}{A \phi}$
\item \textbf{自伴(随)算符}: 对称算符, 且 $A^*$ 与 $A$ 的定义域相同
\end{itemize}
