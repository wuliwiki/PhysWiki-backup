% 开普勒问题数值模拟(Matlab)

\begin{issues}
\issueDraft
\issueOther{应该计算含时运动方程}
\issueOther{推导并引用相关公式}
\end{issues}

\begin{figure}[ht]
\centering
\includegraphics[width=11cm]{./figures/KepNum_1.pdf}
\caption{运行结果} \label{KepNum_fig1}
\end{figure}

\begin{lstlisting}[language=matlab]
% 已知初始位置、发射速度、发射方向, 求轨道以及运动方程
function cel
% === params ===
k = -1; % -GMm, (令 m=1)
R = 1; % 初始半径
alpha = 0; % 初角度
v0 = 1.1; % 初速度
beta = linspace(0, 4*pi/11, 6); % 发射仰角
% ==============

if isscalar(v0)
    v0 = repmat(v0, size(beta));
elseif isscalar(beta)
    beta = repmat(beta, size(v0));
elseif ~all(size(v0)==size(beta))
    error('v0 和 beta 的尺寸必须一样,或者其中之一为标量');
end

% 画地球
close all; figure;
tmp = linspace(0, 2*pi, 500);
plot(R*cos(tmp), R*sin(tmp), 'k--'); hold on;
scatter(0, 0); axis equal;
xlabel x; ylabel y;

for i = 1:6
    cel1(k, R, alpha, v0(i), beta(i)); hold on;
end
end

function cel1(k, R, alpha, v0, beta)
E = v0^2/2 + k/R;
L = v0*cos(beta)*R;
e = sqrt(1 + 2*E*L^2/k^2); % 离心率
p = L^2/abs(k); % 半通径

% 画图
dth = pi/20; Nth = 1000;
if e < 1
    th = linspace(0, 2*pi, Nth);
elseif e == 1
    th = linspace(dth, 2*pi-dth, Nth);
else
    th1 = atan(sqrt(e^2-1));
    th = linspace(th1+dth, 2*pi-th1-dth, Nth);
end

% 计算圆锥曲线需要旋转的角度 th0
if e == 0
    th0 = 0;
elseif cos(beta) >= 0
    th0 = alpha + acos((1-p/R)/e);
else
    th0 = alpha - acos((1-p/R)/e);
end
th0 = real(th0);
scatter(R*cos(alpha), R*sin(alpha));

th = th + th0;
r = p./(1 - e*cos(th-th0));
x = r .* cos(th); y = r .* sin(th);
plot(x, y); axis equal;
end
\end{lstlisting}
