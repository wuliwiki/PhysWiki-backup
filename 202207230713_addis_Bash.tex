% Bash 编程笔记

\begin{issues}
\issueDraft
\end{issues}

\subsection{基础}
\begin{itemize}
\item shell script 就是拓展名为 \verb`.sh` 的文件, 也可以没有
\item 执行 shell script 可以用 \verb`./name.sh` 也可以用 \verb`source /path/name.sh`, 前者在一个新的 shell process 中运行, 后者在当前的 shell process 运行, 应该是相当于直接把文件中的内容复制粘贴到命令行执行.
\item shell script 后面可以输入若干 arguments, 还可以用 \verb`>` 把 shell 中的 stdout 导入文件.
\item 文件第一行可以用 shebang \verb`#!` 指定运行使用的 shell, 如 \verb`#!/bin/bash`
\item 退出时可以使用 \verb`exit 0` 说明运行成功
\item 用 \verb`#` 注释
\end{itemize}

\subsection{判断}

\subsection{循环}
\begin{lstlisting}[language=bash]
#!/bin/bash
for ((i = 1; i < 10; ++i)) # for i in {1,2,3,4}
do
	printf "\nfile${i}.txt\n"
done
\end{lstlisting}

\subsection{数组}
\begin{lstlisting}[language=bash]
#!/bin/bash
arr=(1 3 hello 13)
for ((i=0;i<4;++i))
do
    printf "arr[${i}] = ${arr[i]}\n"
done
\end{lstlisting}


\subsection{输入}
例子(确保文件最后由两个空格)

\begin{lstlisting}[language=bash]
#!/bin/bash
# use `find . -type f -name "*.matt" -exec ./convert_matt.sh {} \;`
# to convert subfolder
file=${1}
printf "${file}... "
c2=$(tail -c 2 ${file})
if [ "${c2}" == "  " ]
then
	printf "already in new format\n"
else
	printf "  " >> ${file}
	printf "done\n"
fi
\end{lstlisting}


\subsection{特殊变量}
\begin{itemize}
\item \verb`$0` 当前 shell script 的文件名
\item \verb`$@` 所有 shell arguments
\item \verb|$?| 上一个命令的 exit status
\end{itemize}
