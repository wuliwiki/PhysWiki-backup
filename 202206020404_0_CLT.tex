% 中心极限定理
% 统计|随机变量|方差|高斯分布|中心极限定理

\begin{issues}
\issueDraft
\end{issues}

\pentry{高斯分布\upref{GausPD}}

若 $N$ 个独立的连续随机变量 $x_i$ 的平均值为 $\mu$, 方差为 $\sigma^2$, 令 $X = \sum_i^N x_i$, 当 $N \to \infty$ 时, $X$ 满足高斯分布, 平均值为
\begin{equation}
\ev{X} = N\mu
\end{equation}
且方差为
\begin{equation}
\ev{X^2} = N\sigma^2
\end{equation}
即分布函数趋近于
\begin{equation}
f(X) = N(N\mu, N\sigma^2) = \frac{1}{\sigma\sqrt{2\pi N}} \exp[-\frac{(X-N\mu)^2}{2N\sigma^2}]
\end{equation}
证明略.

令平均值 $\bar x = X/N$, 那么 $N\to\infty$ 时 $\bar x\sim N(\mu, \sigma^2/N)$(\autoref{RandCV_ex1}~\upref{RandCV}).

\begin{example}{抛硬币}
我们都知道一枚公平的硬币, 若抛许多次, 那么正反的比例大约是 1:1. 那么若把某次实验的结果中, 前 $N$ 次中正面向上所占的比例记为 $f(N)$, 那么 $f(N)$ 是否满足以下极限呢?
\begin{equation}
\lim_{N\to\infty} f(N) = \frac{1}{2} \qquad (\text{误})
\end{equation}

\end{example}
