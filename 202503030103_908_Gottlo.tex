% 戈特洛布·弗雷格(综述)
% license CCBYSA3
% type Wiki

本文根据 CC-BY-SA 协议转载翻译自维基百科\href{https://en.wikipedia.org/wiki/Gottlob_Frege}{相关文章}。

\begin{figure}[ht]
\centering
\includegraphics[width=6cm]{./figures/49ab91a025d7e7b7.png}
\caption{弗雷格,约1879年} \label{fig_Gottlo_1}
\end{figure}
弗里德里希·路德维希·戈特洛布·弗雷格[7] (Friedrich Ludwig Gottlob Frege,1848年11月8日-1925年7月26日)是德国哲学家、逻辑学家和数学家。他曾担任耶拿大学的数学教授,被许多人视为分析哲学的奠基人,专注于语言哲学、逻辑学和数学哲学。尽管他在生前几乎未受到关注,但朱塞佩·皮亚诺(Giuseppe Peano,1858–1932)、伯特兰·罗素(Bertrand Russell,1872–1970)以及在某种程度上路德维希·维特根斯坦(Ludwig Wittgenstein,1889–1951)将他的工作介绍给了后来的哲学家。弗雷格被广泛认为是自亚里士多德以来最伟大的逻辑学家,也是有史以来最深刻的数学哲学家之一。[8] 

他的贡献包括在《概念文字》(Begriffsschrift)中发展了现代逻辑,以及在数学基础方面的工作。他的著作《算术基础》是逻辑主义项目的开创性文本,迈克尔·杜梅特(Michael Dummett)将其视为语言学转向的标志。弗雷格的哲学论文《论意义与指称》和《思想》也被广泛引用。前者论证了两种不同的意义类型和描述主义。在《算术基础》和《思想》中,弗雷格分别在关于数字和命题的问题上主张与心理主义或形式主义对立的柏拉图主义。

