% aaa
% \documentclass[UTF8]{ctexart}
% \usepackage{multirow}
% \usepackage{framed}
% \usepackage{listings}
% \usepackage{courier}
% \usepackage{color}
\title{CSP-J 普及训练赛 2}
\author{\textbf{\color{black}A\color{red}estas16}}
\begin{document}
	\maketitle
	\begin{center}
		\begin{tabular}{|p{3cm}|p{2.5cm}|p{2.5cm}|p{2.5cm}|}
			\hline
			题目名称 & & 追光 &    \\
			\hline
			源程序文件名 & .cpp & light.cpp & .cpp   \\
			\hline
			输入文件名 & .in & light.in & .in   \\
			\hline
			输出文件名 & .out & light.out & .out  \\
			\hline
			时间限制 & 1s & 1s & 1s \\ 
			\hline
			空间限制 & 256 MB & 256 MB & 256 MB\\
			\hline
			题目类型 & 传统 & 传统 & 传统 \\
			\hline
			测试点数量 & 20 & 20 & 20 \\
			\hline
		\end{tabular}
	\end{center}
	\subsection*{注意事项}
		0. 这个 pdf 文件是 ZPC\_233 做的,所以题面格式和上一次差不多.
	
		1. 本场测试时长 3 小时,请选手注意时间分配.
		
		2. 每道题目可能你没有办法拿到满分,但请尽自己能力拿到部分分.
		
		3. 测试请独立完成,不要互相讨论做法,或上网查找相关资料.
		
		4. 本场比赛使用文件输入输出,需要写 \texttt{freopen}.
		
		5. 部分题目数据读入量较大,请使用较快的数据读入方式.
		
		6. 祝大家能 AK 这场比赛!
	\newpage
	\section*{1 ()}
	\subsection*{题目描述}
	\subsection*{输入格式}
	\textbf{从 .in 读入输入数据.}
	\subsection*{输出格式}
	\textbf{将答案输出到 .out 中.}
	\subsection*{样例 1 输入}
	\subsection*{样例 1 输出}
	\subsection*{样例 1 解释}
	\subsection*{测试点约束}
	对于所有测试点,保证.
	\newpage
	\section*{2 追光 (light)}
	\subsection*{题目背景}
	\begin{center}
		\begin{framed}
			『人们追光,可光的速度是 $3 \times 10^5\ \texttt{km/s}$,又怎么能追得上啊』
			
			『所以啊,所谓追光,其实是光在等人啊』
		\end{framed}
	\end{center}
	\subsection*{题目描述}
	
	GFA 在追光.
	
	光和 GFA 在同一个坐标系中,光最初在原点,面向 $\texttt{x}$ 轴正半轴方向.
	
	\textbf{光出发,按照如下方式运动:}
	
	光有一个长度为 $n$ 的序列 $a$,它将要运动 $n \times m$ 个时刻.
	
	在第 $i$ 个时刻,光向当前方向移动 $a_p$ 个位置,然后顺时针旋转 $a_p \times 90$ 度 ($p = i \bmod n$).
	
	GFA 只能沿 $\texttt{x}$ 轴或 $\texttt{y}$ 轴运动,每个时刻运动 $1$ 个单位长度.
	
	光的速度是 $3 \times 10^5\ \texttt{km/s}$,GFA 追不上它,所以她只想知道她从原点出发,到达光所在的位置最少需要的时间.
	
	\subsection*{输入格式}
	\textbf{从 light.in 读入输入数据.}
	
	第一行包括两个正整数 $n,m$.

	第二行包括 $n$个用空格隔开的正整数 $a_i$.
	\subsection*{输出格式}
	\textbf{将答案输出到 light.out 中.}
	
	一行一个非负整数,即 GFA 从原点出发,到达光所在的位置最少需要的时间.
	\newpage
	\subsection*{样例 1 输入}
	\texttt{5 3}
	
	\texttt{1 2 3 4 5}
	\subsection*{样例 1 输出}
	\texttt{9}
	\subsection*{样例 2 输入}
	\texttt{10 100}
	
	\texttt{97 46 39 12 54 89 32 76 88 100}
	\subsection*{样例 2 输出}
	\texttt{0}
	\subsection*{测试点约束}
	
	对于 $60\%$ 的数据,保证 $1 \le n,m,a_i \le 500$.
	
	对于 $100\%$ 的数据,保证 $1 \le n,m,a_i \le 5 \times 10^5$.
	\subsection*{后记}
	GPA 追上了 FPS.
	\newpage
	\section*{3  ()}
	\subsection*{题目描述}
	\subsection*{输入格式}
	\textbf{从 .in 读入输入数据.}.
	\subsection*{输出格式}
	\textbf{将答案输出到 .out 中.}
	\subsection*{样例 1 输入}
	\subsection*{样例 1 输出}
	\subsection*{样例 1 解释}
	\subsection*{测试点约束}
	对于所有的数据,保证.
\end{document}