% 求和符号(高中)
% license Usr
% type Tutor

\begin{issues}
\issueDraft
\end{issues}

\begin{equation}
\sum_{i=m}^n a_i = a_m + a_{m+1} + \dots + a_n~.
\end{equation}
其中 $i$ 叫做求和指标。 为了区分不同指标也会经常使用 $j,k,l,m,n,p,q$ 等字母。

许多时候,如果已经在语境中明确了求和中 $i$ 取哪些值, 为了方便就可以直接写作 $\sum\limits_i a_i$。 在印刷排版中,行内公式也经常写成 $\sum_{i=m}^n a_i$,但我们尽量使用 $\sum\limits_{i=m}^n a_i$。

\subsection{指标换元}
例如要把指标替换为 $j=i+1$,则
\begin{equation}
\sum_{i=m}^n a_i = \sum_{j=m+1}^{n+1} a_{j-1} ~.
\end{equation}
计算方法是,先用 $j$ 表示 $i$ 得 $i=j-1$, 然后求和号内部的 $i$ 可以全部代入 $j-1$。 对于上下标,也可以直接把 $j-1$ 代入并移项,例如下标代入后得 $j-1=m$,即 $j=m+1$。

\subsection{乘法}
\begin{equation}
\sum_i C a_i = C\sum_i a_i~.
\end{equation}

\begin{equation}
\sum_i a_i \sum_j b_j = \sum_{i,j} a_i b_j~.
\end{equation}

\begin{equation}
\qty(\sum_i a_i)^2 = \sum_{i,j} a_i a_j = \sum_i a_i^2 + 2\sum_{i<j} a_i a_j~.
\end{equation}
可见在求和的相乘中,区分求和指标很重要,如果写成 $\sum\limits_{i,i} a_i a_i$ 将产生混乱。
