% 渐近展开
\pentry{泰勒展开\upref{Taylor}}

在数学中, \textbf{渐近展开 (asymptotic expansion)} 是用一列较简单的函数来逐次逼近给定的函数的办法. 它的形式定义如下:

\begin{definition}{渐近展开}
设自变量$x$趋于某点$a$ (有限或无限) 时, 函数序列$\{\phi_{n}(x)\}$满足
$$
\phi_{n+1}(x)=o(\phi_n(x)),x\to a.
$$
则对于给定的函数$f$, 称$f(x)$在$x\to a$时有渐近展开式
\begin{equation}\label{Asympt_eq1}
f(x)\simeq\phi_0(x)+\phi_1(x)+...+\phi_n(x)+...,x\to a.
\end{equation}
是指: 对于任何给定的$n$, 皆有
\begin{equation}\label{Asympt_eq2}
f(x)=\phi_0(x)+\phi_1(x)+...+\phi_n(x)+o(\phi_n(x)),x\to a.
\end{equation}
\autoref{Asympt_eq1}中的级数称为$x\to a$时$f(x)$的渐近级数.
\end{definition}

需要注意两点:

(1) 渐近展开一般来说不是唯一的: 当$x\to0$时, 函数$f(x)\equiv 0$和函数$g(x)=e^{-1/x^2}$都有渐近展开
$$
0+0x+0x^2+...
$$

(2) 渐近级数可以收敛也可以发散. \autoref{Asympt_eq1}只是一个形式等式, 它的真正含义是\autoref{Asympt_eq2}, 而\autoref{Asympt_eq2}只表示对于固定的$n$, 当$x\to a$时, $f(x)$与渐近级数的第$n$项部分和相差一个高阶无穷小. 