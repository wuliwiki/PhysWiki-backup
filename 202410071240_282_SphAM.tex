% 球坐标系中的轨道角动量算符
% keys 角动量|轨道角动量|球坐标|拉普拉斯算符|本征矢
% license Xiao
% type Tutor

\pentry{角动量(量子)\nref{nod_QOrbAM}}{nod_1276}

本文使用\enref{原子单位制}{AU}。 在量子力学中, 我们一般把角动量算符放在球坐标中表示。 把轨道角动量算符在直角坐标系中的定义(\autoref{eq_QOrbAM_2})通过\enref{链式法则}{PChain}用球坐标表示(留作习题)。
\begin{equation}
L_x = \I \qty(\sin\phi\pdv{\theta} + \cot\theta\cos\phi\pdv{\phi})~,
\end{equation}
\begin{equation}
L_y = \I \qty(-\cos\phi\pdv{\theta} + \cot\theta \sin\phi \pdv{\phi})~,
\end{equation}
\begin{equation}
L_z = -\I\pdv{\phi}~,
\end{equation}
\begin{equation}
L^2 = L_x^2 + L_y^2 + L_z^2 = -\frac{1}{\sin\theta}\pdv{\theta} \qty(\sin \theta \pdv{u}{\theta}) - \frac{1}{\sin^2 \theta} \pdv[2]{u}{\phi}~.
\end{equation}
注意 $L^2$ 恰好是球坐标系中拉普拉斯算符的角向部分(\autoref{eq_SphNab_3}) $\laplacian_\Omega$ 取负。
\begin{equation}\label{eq_SphAM_1}
L^2 = -\laplacian_\Omega~.
\end{equation}
这并不奇怪, 经典力学中球坐的哈密顿量可以记为(\autoref{eq_HamCan_3})
\begin{equation}
H = \frac{p_r^2}{2m} + \frac{L^2}{2mr^2} + V~,
\end{equation}
其中 $p_r = m\dot r$, $L = mr^2\dot\theta$。 而量子力学的哈密顿算符在球坐标中可以用\autoref{eq_SphNab_3} 分解为
\begin{equation}
H = -\frac{1}{2m}\laplacian + V = \frac{-\laplacian_r}{2m} +\frac{-\laplacian_\Omega}{2mr^2} + V~,
\end{equation}
这让我们很容易猜出 $p_r^2 = -\laplacian_r$ 和\autoref{eq_SphAM_1}。

类比动量算符 $\bvec p = -\I \grad$, 我们可以定义 $\grad_\Omega$ 满足
\begin{equation}
\bvec L = -\I \grad_\Omega~,
\end{equation}
于是 $\laplacian_\Omega$ 可以看作是两个 $\grad_\Omega$ 相乘而得。

\subsection{角动量算符的本征函数}
\pentry{球谐函数\nref{nod_SphHar}}{nod_e53e}
我们已经知道 $L^2, L_z$ 对易且具有共同本征矢 $\ket{l, m}$, 现在我们在球坐标中求解它的波函数。 来看本征方程
\begin{equation}
L_z \ket{l, m} = m\ket{l, m}~,
\end{equation}
\begin{equation}
L^2 \ket{l, m} = l(l+1)\ket{l, m}~,
\end{equation}
它的解就是球谐函数 $Y_{l,m}(\theta,\phi)$。 但本征波函数应该是三维的, 所以任意波函数 $R(r)Y_{l,m}(\theta, \phi)$ 都是 $L^2$ 和 $L_z$ 的共同本征波函数。
