% 双共轭梯度法解线性方程组

\pentry{正定矩阵\upref{DefMat}}

\footnote{本文参考 \cite{NR3}.}要求解对称正定矩阵(SPD) $\mat A$, 的线性方程组
\begin{equation}
\mat A \bvec x = \bvec b
\end{equation}
只需要令
\begin{equation}
f(\bvec x) = \frac{1}{2}\bvec x\Tr \mat A\bvec x - \bvec b\Tr \vdot \bvec x
\end{equation}
容易证明 $\grad f = \mat A \bvec x - \bvec b$. 注意 $f(x)$ 是一个凹二次函数, 所以取最小值当且仅当梯度为零. 这样, 解方程组的问题就转化为求函数极小值问题. 我们可以用牛顿法(链接未完成)来求最小值.

这种方法叫做\textbf{共轭梯度法}, 它可以拓展为\textbf{共轭梯度法}, 以适用于任意线性方程组. 该方法的优势在于用户只需要向解算器提供矩阵乘矢量的函数, 而不需要提供矩阵本身. 这样, 矩阵可以具有任意的数据结构, 例如各种稀疏矩阵\upref{SprMat}.

当矩阵 $\mat A$ 接近于单位矩阵时, 该方法收敛最快, 因此, 

