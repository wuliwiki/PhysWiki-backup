% 丢番图(综述)
% license CCBYSA3
% type Wiki

本文根据 CC-BY-SA 协议转载翻译自维基百科\href{https://en.wikipedia.org/wiki/Diophantus#Notes}{相关文章}。

亚历山大的丢番图(约公元200年–约214年出生;约公元284年–约298年去世)是一位希腊数学家,著有两部主要作品:《多边形数论》,该书现存不全,以及《算术学》,分为十三卷,大部分仍然存在,包含了一些通过代数方程求解的算术问题。

丢番图的《算术学》对阿拉伯数学的发展产生了影响,他的方程式也影响了现代抽象代数和计算机科学的研究。他的前五卷完全是代数性质的。此外,最近对丢番图作品的研究表明,他在《算术学》中教授的解题方法,与后来的中世纪阿拉伯代数在概念和整体程序上高度相似。

丢番图是最早认识到正有理数作为数字的数学家之一,通过允许系数和解为分数。他创造了术语 \textbf{παρισότης}(parisotēs)来表示近似等式。这个术语在拉丁语中翻译为 \textbf{adaequalitas},并成为皮埃尔·德·费尔马(Pierre de Fermat)发展出的“等近性”技术,用于求函数的最大值以及曲线的切线。

尽管《算术学》不是最早使用代数符号解决算术问题的作品,但它无疑是最著名的一个,这些问题来源于希腊古代,并且其中的一些问题激发了后来的数学家在分析学和数论领域的研究。现代使用中,丢番图方程指的是带有整数系数的代数方程,目标是寻找其整数解。丢番图几何和丢番图逼近是另外两个以他命名的数论子领域。
\subsection{传记}
丢番图出生于一个希腊家庭,并且已知他在罗马时代的公元200年至214年到284年或298年期间生活在埃及的亚历山大城。[6][8][9][a] 关于丢番图生平的大部分知识来自一部5世纪的希腊数字游戏和谜题选集,由梅特罗多罗斯(Metrodorus)编纂。书中有一个问题(有时被称为他的墓志铭)内容如下:

“此地安葬丢番图,令人惊叹。通过代数的艺术,石碑上述说他的年岁:‘上帝赋予他少年时期,生命的一六分之一;青春期更多,成长为胡须浓密的青年,一十二分之一;然后在婚姻前,又度过了七分之一;五年后,迎来了一个跳跃的儿子。可怜的是,这个父亲与智者的亲爱的孩子,在活到父亲生命的一半时,命运将他带走。四年后,他通过数字科学安慰自己的命运,最终结束了生命。’”

这个谜题意味着丢番图的年龄 x 可以表达为:

\[x = \frac{x}{6}+\fa + x/7 + 5 + x/2 + 4~\]

解得 x 的值为84岁。然而,这些信息的准确性无法确认。

在流行文化中,这个谜题出现在《雷顿教授与潘多拉的盒子》中,作为游戏中最难解的谜题之一,需要通过先解决其他谜题才能解锁。