% Stern-Gerlach实验
% keys 斯特恩-盖拉赫实验|狄拉克符号|自旋|坍缩|量子态

\pentry{量子力学的基本原理(量子力学)\upref{QMPrcp}}

Stern-Gerlach实验由O. Stern于1921年构想,由Stern和W. Gerlach于1922年在法兰克福完成\footnote{可参见Bretislav Fridrich和Dudley Herschbach发表的\textsl{Stern and Gerlach: How a Bad Cigar Helped Reorient Atomic Physics}, \textsl{Physics Today}, Dec. 2003.}.

引用樱井纯《现代量子力学》中的描述:“在某种意义上,Stern-Gerlach类型的双态系统是最少经典力学而最多量子力学的系统.对涉及双态系统问题的坚实理解将对任何认真学习量子力学的学生都是有益的.”\textbf{量子力学的基本原理(量子力学)}\upref{QMPrcp}词条中也建议配合本词条内容来理解抽象的概念.


\subsection{实验描述}

用一个炉子加热银原子,使之获得动能,从炉子上的一个小洞跑出来.出射的银原子会经过一个准直器,之后朝已经建立好的非均匀磁场飞去.这个磁场由如图所示的两磁极构成,其中一磁极有尖锐的边缘.之后,银原子会打到一块接收屏上,形成可观测的光斑.实验如\autoref{SGExp_fig1} 所示.


\begin{figure}[ht]
\centering
\includegraphics[width=14cm]{./figures/SGExp_1.pdf}
\caption{Stern-Gerlach实验示意图.} \label{SGExp_fig1}
\end{figure}




















