% Ubuntu 22.04 安装 NVIDIA 显卡驱动笔记
% license Usr
% type Note

如果你装机放了一张 Nvidia 显卡, 那么 Ubuntu 在安装时会自动提供一个 Nouveau 开源驱动。 但为了进行 GPU 编程或者提高显卡性能,需要安装官方显卡。 笔者使用 Ubuntu 22.04 和 GTX 1080 Ti 显卡进行测试。

=== ubuntu-toolchain-r 找不到 gcc 11.3.0 ===

参考\href{https://askubuntu.com/questions/1481354/install-a-specific-gcc-version-to-match-the-version-the-kernel-was-compiled-with}{这里}!

直接手动下载 deb 包! 分别三个文件:
\begin{itemize}
\item \href{http://mirrors.kernel.org/ubuntu/pool/main/g/gcc-11/gcc-11-base_11.3.0-6ubuntu1_amd64.deb}{gcc-11-base_11.3.0-6ubuntu1_amd64.deb}
\item \href{http://mirrors.kernel.org/ubuntu/pool/main/g/gcc-11/libgcc-11-dev_11.3.0-6ubuntu1_amd64.deb}{libgcc-11-dev_11.3.0-6ubuntu1_amd64.deb}
\item \href{http://mirrors.kernel.org/ubuntu/pool/universe/g/gcc-11/gcc-11_11.3.0-6ubuntu1_amd64.deb}{gcc-11_11.3.0-6ubuntu1_amd64.deb}
\end{itemize}


======= 最后还是安装官方 535 最新驱动, 编译成功, 但失败, 无法加载 kernel module ===

尝试安装 gcc 11.3.0, 因为内核使用这个版本编译的, 但是当前 gcc 版本是 11.2!

\begin{itemize}
\item reference: https://askubuntu.com/questions/815331/updating-to-latest-gcc-and-g-on-ubuntu-16-04
\item this worked for both g++-8 and g++-9
\begin{lstlisting}[language=none]
sudo add-apt-repository ppa:ubuntu-toolchain-r/test
sudo apt-get update
sudo apt-get install gcc-snapshot
# 这里可能会提示有 package 版本冲突, 用 aptitude 就可以了
sudo apt-get install gcc-9 g++-9
\end{lstlisting}
\item reference: https://askubuntu.com/questions/26498/how-to-choose-the-default-gcc-and-g-version
\item however, \verb`g++ --version` still gives the old version, because \verb`/usr/bin/gcc` and \verb`/usr/bin/g++` are symbolic links. To check the links, use
\begin{lstlisting}[language=none]
ls -la /usr/bin | grep gcc
ls -la /usr/bin | grep g++
\end{lstlisting}
Then remove and create new links
\begin{lstlisting}[language=none]
sudo rm /usr/bin/gcc
sudo rm /usr/bin/g++
sudo ln -s /usr/bin/gcc-9 /usr/bin/gcc
sudo ln -s /usr/bin/g++-9 /usr/bin/g++
\end{lstlisting}
now \verb`g++ --version` should return \verb`g++ 9`.
\end{itemize}



============= 失败 (但没有查一下为什么 kernel module 加载不了) ===============

失败原因好像是因为 apt 在设置 trigger 的时候同样也要运行 gcc

\verb`sudo add-apt-repository ppa:graphics-drivers/ppa`

\verb`-> Kernel module compilation complete.
ERROR: Unable to load the kernel module 'nvidia.ko'.  This happens most frequently when this kernel module was built against the wrong or improperly configured kernel sources, with a version of gcc that differs from the one used to build the target kernel, or if another driver, such as nouveau, is present and prevents the NVIDIA kernel module from obtaining ownership of the NVIDIA device(s), or no NVIDIA device installed in this system is supported by this NVIDIA Linux graphics driver release.`



参考\href{https://ubuntu.com/server/docs/nvidia-drivers-installation}{这个},推荐用 ubuntu-drivers tool 。

\begin{itemize}
\item 首先确定 Software & Updates 中的 Proprietary drivers for devices (restricted) 已经勾选
\item \verb`sudo apt update`
\item \verb`sudo ubuntu-drivers list`
\begin{lstlisting}[language=none]
nvidia-driver-470-server, (kernel modules provided by nvidia-dkms-470-server)
nvidia-driver-390, (kernel modules provided by nvidia-dkms-390)
nvidia-driver-470, (kernel modules provided by nvidia-dkms-470)
nvidia-driver-510-server, (kernel modules provided by nvidia-dkms-510-server)
nvidia-driver-450-server, (kernel modules provided by nvidia-dkms-450-server)
nvidia-driver-418-server, (kernel modules provided by nvidia-dkms-418-server)
nvidia-driver-510, (kernel modules provided by nvidia-dkms-510)
\end{lstlisting}
\item 可以自动安装系统认为合适的驱动 \verb`sudo ubuntu-drivers install --gpgpu`
\item \verb`sudo ubuntu-drivers install nvidia:510`, 就会开始安装。
\item 完成后(可能要重启)在 Software & Updates 中可以更改想要使用的驱动。
\end{itemize}
\begin{figure}[ht]
\centering
\includegraphics[width=14.25cm]{./figures/bee995b41e4efa8c.png}
\caption{} \label{fig_NvDrUb_1}
\end{figure}




========== 失败,因为 gcc 的版本是 11.2, 但 kernel 编译 gcc 版本是 11.3 =======

首先到 Nvidia 官网下载 Linux 驱动。 安装以后提示需要先禁止 nouveau 驱动才可以安装。 安装包可以自动写入一个 \verb`/etc/modprobe.d/nvidia-installer-disable-nouveau.conf` 文件来试图禁止。
\begin{lstlisting}[language=none,caption=nvidia-installer-disable-nouveau.conf]
# generated by nvidia-installer
blacklist nouveau
options nouveau modeset=0
\end{lstlisting}

注意写入该文件后, 如果显卡安装失败, 那么重启以后就可能导致屏幕没有任何反应。 但只要想办法把该文件删掉再次重启即可(例如用一个启动 U 盘)。

另外 Nvidia 驱动安装包的日志会写进 \verb`/var/log/nvidia-installer.log`。

安装包会提示这个
\begin{lstlisting}[language=none]
An alternate method of installing the NVIDIA driver was detected.
(This is usually a package provided by your distributor.) A      
  driver installed via that method may integrate better with your
  system than a driver installed by nvidia-installer.

  Please review the message provided by the maintainer of this alternate
  installation method and decide how to proceed:

Continue installation                        Abort installation 
\end{lstlisting}

ERROR: An error occurred while performing the step: "Building kernel modules". See /var/log/nvidia-installer.log for details.      
        
                                                                  OK 

