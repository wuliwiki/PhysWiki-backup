% 小振动
% 小振动

\pentry{拉格朗日方程\upref{Lagrng}}

\subsection{一维振动}
一维谐振子的拉格朗日量为
\begin{equation}
L=\frac{1}{2}m\cdot{x}^2-\frac{1}{2}kx^2
\end{equation}
其运动方程为 $m\ddot {x}+kx=0$,或写作 $\ddot{x}+\omega^2 x=0$,其中 $\omega=\sqrt{k/m}$ 为该系统的固有频率.

受迫振动的拉格朗日量为 
\begin{equation}
L=\frac{1}{2}m\dot{x}^2-\frac{1}{2}kx^2+xF(t)
\end{equation}
其运动方程为$m\ddot{x}+kx=F(t)$,或写作 $\ddot{x}+\omega^2 x=\frac{F(t)}{m}$.
当 $F(t)=f\cos(\gamma t+\beta)$ 时,解为(齐次方程的通解+特解)
\begin{equation}
x(t)=a\cos(\omega t+\alpha)+\frac{f\cos(\gamma t+\beta)}{m(\omega^2-\gamma^2)}
\end{equation}
特别地,当 $\omega=\gamma$,上式不再成立,解为
\begin{equation}
x(t)=a \cos(\omega t+\alpha)+\frac{ft \sin(\omega t+\beta)}{2m\omega}
\end{equation}
特解部分的振幅随时间线性增长.这就是共振现象.在实际力学系统中会有阻尼,则有新的运动方程.

\subsection{小振动体系的运动方程}
考虑 $n$ 个质点组成的\textbf{理想、完整、稳定}约束力学体系,主动力都是\textbf{保守力}.系统的广义坐标数与自由度数相同:$s=d$.取 $s$ 个广义坐标 $q_1,\cdots,q_s$,设坐标的变换方程为
\begin{equation}
x_i = x_i(q),\ \ \  i=1,2,\cdots,3n
\end{equation}

设整个体系在某个\textbf{平衡位置}附近运动,并取这个平衡位置为广义坐标的零点:$\bvec q=\bvec 0$.

由于主动力是保守力,可以用势能 $V(\bvec q)$ 描述.在平衡位置处对 $V(\bvec q)$ 作泰勒展开:

\begin{equation}
V(\bvec q)\approx V(0)+\left[\frac{\partial V}{\partial q_\alpha}\right]_\bvec {q=0} +\frac{1}{2}[\frac{\partial^2V}{\partial q_\alpha\partial q_\beta}]_\bvec{q=0}q_\alpha q_\beta
\end{equation}