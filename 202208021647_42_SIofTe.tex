% 张量的对称化和交错化
% 对称化|交错化

\begin{issues}
\issueTODO
\end{issues}

\pentry{张量的坐标\upref{CofTen},置换群\upref{Perm}}

张量可按选出来的共变或逆变指标的集合来讨论对称性和斜对称性.由于张量的共边和逆变指标的对称性(即共边和逆变或矢量和对偶矢量的区别仅在于原矢量空间的选择),在讨论张量的对称性和斜对称性时可直接将张量限制在 $(p,0)$ 型或 $(0,p)$ 型张量,而不减少一般性.比如:讨论张量 $T^{i_1 i_2\cdots i_p}$ 和 ${S_{i_1}}^{i_2\cdots i_p}$ 在指标 $\{i_1,i_2\}$ 上的对称性质没有任何区别,只需将 $T$ 的上指标 $i_1$ 当作下指标即可移值到张量 $S$.故在本词条里,我们总是将张量限制为 $(p,0)$ 型张量.并且讨论的是这 $p$ 个指标的对称性和斜对称性,而不是它的一部分(因为在讨论某些指标的(斜)对称性质的时候,只需将其它指标当作固定的就变成这里讨论的情形).
\subsection{张量的对称化}
设 $T\in \mathbb{T}_p^0(V)$ (\autoref{CofTen_def1}~\upref{CofTen}),即
\begin{equation}
T=\sum_{i_1,\cdots,i_p}T_{i_1,\cdots,i_p}e^{i_1}\otimes\cdots\otimes e^{i_p}
\end{equation}
而 $S_p$ 是作用在指标集合 $\{1,2,\cdots,p\}$ 上的 $p$ 阶置换群\upref{Perm}.

对任意置换 $\pi\in S_p$ ,定义映射:
\begin{equation}\label{SIofTe_eq1}
f_\pi(T)(x_1,x_2,\cdots,x_p):=T(x_{\pi1},\cdots,x_{\pi p})
\end{equation}
其中,$x_i$ 是以 $i$ 为指标的矢量.

\begin{theorem}{}
$\forall \pi,\sigma\in S_p$ 都有 
\begin{equation}
f_{\pi}(T)\in\mathbb{T}_p^0
\end{equation}
\begin{equation}\label{SIofTe_eq6}
f_\pi \circ f_\sigma=f_{\pi\sigma}
\end{equation}

\begin{equation}\label{SIofTe_eq2}
f_{\pi}(T)=\sum_{i_1\cdots i_p}T_{i_{\pi1}\cdots i_{\pi p}}e^{i_1}\otimes\cdots\otimes e^{i_p}
\end{equation}
上式等价于
\begin{equation}\label{SIofTe_eq3}
f_{\pi}(T)=\sum_{i_1\cdots i_p}T_{i_1\cdots i_p}e^{i_{\pi^{-1}1}}\otimes\cdots\otimes e^{i_{\pi^{-1}p}}
\end{equation}
\end{theorem}
\textbf{证明:}

1.$f_\pi(T)\in\mathbb{T}_p^0$

不失一般性,设 $\pi k=1$,那么由\autoref{SIofTe_eq1} 
\begin{equation}
\begin{aligned}
f_\pi(T)(\alpha x_1+\beta y_1,x_2,\cdots,x_p)&=T(x_{\pi1},\cdots,\alpha x_1+\beta y_1,\cdots,x_{\pi p})\\
&=\alpha T(x_{\pi1},\cdots,x_1,\cdots,x_{\pi p})+\beta T(x_{\pi1},\cdots,y_1,\cdots,x_{\pi p})\\
&=\alpha f_{\pi}(T)(x_1,x_2,\cdots,x_p)+\beta f_{\pi}(T)(y_1,x_2,\cdots,x_p)
\end{aligned}
\end{equation}

2. $f_\pi \circ f_\sigma=f_{\pi\sigma}$

由\autoref{SIofTe_eq1} 和函数的复合,有
\begin{equation}
\begin{aligned}
f_\pi\circ f_\sigma(T)(x_1,\cdots,x_p)&=f_\pi\circ T(x_{\sigma1},\cdots,x_{\sigma p})=f_\pi(T)(x_{\sigma1},\cdots,x_{\sigma p})\\
&=T(x_{\pi\sigma1},\cdots,x_{\pi\sigma p})=f_{\pi\sigma}(T)(x_1,\cdots,x_p)
\end{aligned}
\end{equation}


3.\autoref{SIofTe_eq2} ,\autoref{SIofTe_eq3} 的证明:

由张量坐标的定义(\autoref{CofTen_def2}~\upref{CofTen})
\begin{equation}
\begin{aligned}
f_{\pi}(T)_{i_1\cdots i_p}&=f_{\pi}(T)(e_{i_1},\cdots,e_{i_p})=T(e_{i_{\pi1}},\cdots,e_{i_{\pi p}})=T_{i_{\pi1}\cdots i_{\pi p}}
\end{aligned}
\end{equation}
于是由\autoref{CofTen_the1}~\upref{CofTen},即得\autoref{SIofTe_eq2} ,将\autoref{SIofTe_eq2} 指标中的 $i$ 换为  $\pi^{-1} i$,即得\autoref{SIofTe_eq3} .

\textbf{证毕!}

若设 $f_\pi(\alpha T+\beta T')=\alpha f(T)+\beta f_\pi(T')$,则任意 $\pi \in S_p$ 引导出非退化的线性算子\upref{LiOper} $f_\pi:\mathbb{T}_p^0\rightarrow\mathbb{T}_p^0$ .

\begin{definition}{对称,对称化}
称 $(p,0)$ 型张量 $T$ 是\textbf{对称的},如果 $\forall \pi\in S_p$ 都有 $f_{\pi}(T)=T$.称映射 
\begin{equation}
S=\frac{1}{p!}\sum_{\pi\in S_p} f_\pi:\mathbb{T}_p^0\rightarrow\mathbb{T}_p^0
\end{equation}
为 $\mathbb{T}_p^0(V)$ 上矢量的\textbf{对称化}映射.
\end{definition}

\begin{theorem}{张量的对称化}
每个张量经 $S$ 的作用后都是对称的,即 $\forall \pi\in S_p$,都有 $f_\pi(S(T))=S(T)$.
\end{theorem}
\textbf{证明:}
\begin{equation}
f_\pi(S(T))=\frac{1}{p!}\sum_{\sigma\in S_p} f_\pi(f_\sigma(T))=\frac{1}{p!}\sum_{\sigma\in S_p} f_{\pi\sigma}(T)
\end{equation}
由于 $S_p$ 是个群,则其上任一个元素 $\pi$ 都是其上的一个双射.即当 $\sigma$ 取遍 $S_p$ 时,$\pi\sigma$ 也取遍 $S_p$ .所以上式变为
\begin{equation}
f_\pi(S(T))=\frac{1}{p!}\sum_{\tau\in S_p} f_{\tau}(T)=S(T)
\end{equation}
\textbf{证毕!}

\begin{exercise}{}
试证明:$\mathbb{T}_p^0(V)$ 上所有对称张量构成一个子空间(矢量空间).
\end{exercise}
\begin{definition}{对称张量子空间}
$\mathbb{T}_p^0$ 中全体对称张量构成的子空间记作 $\mathbb{T}_p^+(V)$.
\end{definition}
\begin{exercise}{}
试证明: 若 $T\in\mathbb{T}_p^+(V)$,则 $S(T)=T$.
\end{exercise}
\subsection{张量的交错化}
\pentry{置换的奇偶性\upref{permu}}
\begin{definition}{斜对称张量}
称张量 $T$ 是\textbf{斜对称(反对称)}的,若 $\forall \pi\in S_p$,都有
\begin{equation}\label{SIofTe_eq4}
f_\pi(T)=\epsilon_\pi T
\end{equation}
其中 $\epsilon_\pi$ 是置换 $\pi$ 的(奇偶性)符号.
\end{definition}

\autoref{SIofTe_eq4} 等价为
\begin{equation}\label{SIofTe_eq5}
f_\tau(T)=-T
\end{equation}
其中,$\tau$ 是 $S_p$ 上的对换.
\begin{example}{}
\autoref{SIofTe_eq4} 和\autoref{SIofTe_eq5} 等价性的证明

\textbf{证明:}
根据\autoref{permu_the2}~\upref{permu},置换都可分解成轮换的乘积,而由\autoref{permu_the1}~\upref{permu},轮换都可分解成对换的乘积,而奇偶性不变,即最终置换都能写成对换的乘积,而奇偶性不变,而对换都是奇置换,那么由 \autoref{SIofTe_eq4} 可得\autoref{SIofTe_eq5} ;同样由\autoref{SIofTe_eq5} ,任意对换 $\tau$ 的符号 $\epsilon_\tau=-1$ ,那么对任意置换 $\pi$,有
\begin{equation}
\begin{aligned}
&\pi=\tau_1\cdots\tau_{2n}, \quad &\text{ $\pi$ 为偶置换}\\
&\pi=\tau_1\cdots\tau_{2n+1}, &\quad \text{ $\pi$ 为奇置换}
\end{aligned}
\end{equation}
于是由\autoref{SIofTe_eq6} 和\autoref{SIofTe_eq5} 
\begin{equation}
\begin{aligned}
&f_\pi(T)=f_{\tau_1}\cdots f_{\tau_{2n}}(T)=(-1)^{2n}T=\epsilon_\pi T,\quad&\text{$\pi$ 是偶置换}\\
&f_\pi(T)=f_{\tau_1}\cdots f_{\tau_{2n+1}}(T)=(-1)^{2n+1}T=\epsilon_\pi T,\quad&\text{$\pi$ 是奇置换}
\end{aligned}
\end{equation}
即由\autoref{SIofTe_eq5} 得到\autoref{SIofTe_eq4} 

\textbf{证毕!}

\end{example}






\begin{exercise}{}\label{SIofTe_exe1}
试证明:若 $T$ 是斜对称的,则 $T_{i_{\pi1}\cdots i_{\pi p}}=\epsilon_\pi T_{i_1\cdots i_p}$ ,特别的,$ T_{i_1\cdots i_p}$ 在任两指标相同时为0.
\end{exercise}

由\autoref{SIofTe_exe1} ,对于 $p$ 个数字构成的所有排列中,只要确定了一个排列下的坐标即可.而 $p$ 个指标中有两个相同则该坐标为0,所以独立的坐标个数相当于从 $n$ 个数中选择 $p$ 个不同的数的个数,即共有 $\left(\begin{aligned}
&n\\
&p
\end{aligned}\right)$
个独立的指标.

\begin{definition}{张量的交错化}
称映射
\begin{equation}\label{SIofTe_eq7}
A=\frac{1}{p!}\sum_{\pi\in S_p}\epsilon_\pi f_\pi:\mathbb{T}_p^0(V)\rightarrow\mathbb{T}_p^0(V)
\end{equation}
为 $\mathbb{T}_p^0(V)$ 上矢量的\textbf{交错化}映射.
\end{definition}
\begin{theorem}{}
$\mathbb{T}_p^0$ 上所有斜对称张量的集合构成 $\mathbb{T}_p^0$ 的一个子空间.
\end{theorem}
\textbf{证明:}结合律、单位元显然,封闭性证明如下:
\begin{equation}
\begin{aligned}
f_\pi P=\epsilon_\pi P,\quad&f_\pi R=\epsilon_\pi R\\
&\Downarrow\\
f_\pi(\alpha P+\beta R)&=\alpha f_\pi P+\beta f_\pi R\\
&=\alpha \epsilon_\pi P+\beta \epsilon_\pi R
\end{aligned}
\end{equation}
斜对称张量的逆元由 $f_\pi$ 的线性易证也是斜对称张量.

\textbf{证毕!}
\begin{definition}{斜对称张量子空间}
$\mathbb{T}_p^0$ 上所有斜对称张量构成的子空间记作 $\Lambda^p(V^*)$.
\end{definition}
\begin{theorem}{}
交错化映射 $A$ 是个线性算子,且满足:
\begin{enumerate}
\item $A^2=A$;
\item $\mathrm{Im} A=\Lambda^p(V^*)$;
\item $A(f_\sigma(T))=\epsilon_\sigma A(T)$
\end{enumerate}
\end{theorem}
\textbf{证明:}1.由\autoref{SIofTe_eq7} 
\begin{equation}
\begin{aligned}
A^2&=\frac{1}{(p!)^2}\sum_{\sigma,\pi\in S_p}\epsilon_\sigma\epsilon_\pi f_\sigma\circ f_\pi=\frac{1}{(p!)^2}\sum_{\sigma,\pi\in S_p}\epsilon_{\sigma\pi} f_{\sigma\pi}\\
\end{aligned}
\end{equation}
由于对任意 $\rho\in S_p$,任选 $\sigma\in S_p$ ,都有 $\pi=\sigma^{-1}\rho$,使得 $\pi$

\textbf{证毕!}