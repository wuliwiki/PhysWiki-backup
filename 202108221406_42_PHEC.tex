% 现象类
% keys 现象|现象类|数的等式

\pentry{量类和单位\upref{QCU}}
物理规律是物理量之间关系的反映.既然选定单位后每个量可用一个数代表,物理规律也就可用数的等式表示.反应物理规律的数的等式称为物理规律的\textbf{数值表达式},简称\textbf{数的等式}.

在此之前,我们只定义了同类量之间的一些运算,尚未对量的运算进行定义.所以,一个物理规律,比如牛顿第二定律 $f=ma$ 当中的每个字母应理解成一个数,物理规律应理解成数的等式,而非量