% 阶乘(高中)
% keys 整数|gamma 函数|pi 符号
% license Xiao
% type Tutor

\begin{issues}
\issueDraft
\end{issues}

对正整数 $N > 0$, \textbf{阶乘(factorial)}定义为小于等于 $N$ 且大于零的所有整数相乘
\begin{equation}\label{eq_factor_1}
N! \equiv 1 \cdot 2 \cdot 3 \dots (N - 2) (N - 1)N~.
\end{equation}
例如 $3! = 1\times 2\times 3 = 6$, 又例如 $5! = 1\times 2\times 3\times 4\times 5 = 120$。

特殊地, 我们定义
\begin{equation}
0! = 1~.
\end{equation}

\autoref{eq_factor_1} 也可以用\enref{求积符号}{ProdSy}表示为
\begin{equation}
N! \equiv \prod_{k = 1}^N k~.
\end{equation}
求积符号用法与\enref{求和符号}{SumSym} $\sum$ 相似, 只是把加法变为乘法:
\begin{equation}
\prod_{k = 1}^N a_k = a_1\times a_2\times \dots a_N~.
\end{equation}

学习微积分以后, 阶乘可以拓展到半整数甚至大部分实数, 见 “\enref{Gamma 函数}{Gamma}”(确切的说,Gamma 函数是阶乘的解析延拓,也就是我们通过找到一个函数使之满足原阶乘函数在各已知点处的取值而值域广于原来的阶乘函数,但不能简单的认为 Gamma 函数就是阶乘或者阶乘就是 Gamma 函数)。
