% 拉普拉斯变换

拉普拉斯变换是用来干什么的呢?拉普拉斯变换常用于初始值问题,即已知某个物理量在初始时刻$t=0$的值$f(0)$,而求解它在初始时刻之后的变化情况$f(t)$.至于它在初始时刻之前的值,我们就让它都等于$0$,也就是说
\begin{equation}
f(t)=0 \quad(t<0)
\end{equation}
为了获得较宽的变换条件,构造一个函数$g(t)$,
\begin{equation}
g(t)=\mathrm{e}^{-\sigma t} f(t)
\end{equation}
这里$e^{-\sigma t}$为收敛因子.我们需要选一个充分大的正实数$\sigma$,用来保证$g(t) $在区间$(-\infty,+\infty)$上绝对可积.于是,可以对$g(t) $做傅里叶变换:
\begin{equation}
G(\omega)=\frac{1}{2 \pi} \int_{-\infty}^{\infty} g(t) \mathrm{e}^{-\omega t} \mathrm{d} t=\frac{1}{2 \pi} \int_{0}^{\infty} f(t) \mathrm{e}^{-(\sigma+\mathrm{i} \omega) t} \mathrm{d} t
\end{equation}
将$\sigma+i \omega$记作$p$,并将$G(\omega)$改记作$\bar f(p) / 2 \pi$则
\begin{equation} \label{LapTra_eq1}
\bar{f}(p)=\int_{0}^{\infty} f(t) \mathrm{e}^{-p t} \mathrm{d} t
\end{equation}
其中积分
\begin{equation}
\int_{0}^{\infty} f(t) \mathrm{e}^{-p t} \mathrm{d} t
\end{equation}
称为\textbf{拉普拉斯积分}.$\bar f(p)$称为$f(t)$的\textbf{拉普拉斯函数}.\autoref{LapTra_eq1}称为\textbf{拉普拉斯变换}(简称\textbf{拉氏变换}),$\mathrm e^{-pt}$是拉普拉斯变换的\textbf{核}.

$G(\omega)$的傅里叶逆变换是:
\begin{equation}
g(t)=\int_{-\infty}^{\infty} G(\omega) \mathrm{e}^{\mathrm{i} \omega t} \mathrm{d} \omega=\frac{1}{2 \pi} \int_{-\infty}^{\infty} \bar{f}(\sigma+\mathrm{i} \omega) \mathrm{e}^{\mathrm{i} \omega t} \mathrm{d} \omega
\end{equation}
即
\begin{equation}
f(t)=\frac{1}{2 \pi} \int_{-\infty}^{\infty} \bar{f}(\sigma+\mathrm{i} \omega) \mathrm{e}^{(\sigma+\mathrm{i} \omega)} \mathrm{d} \omega
\end{equation}
由于$\sigma+\mathrm i\omega=p$,可得$\mathrm d\omega = 1/\mathrm i\mathrm dp$.所以有
\begin{equation}
f(t)=\frac{1}{2 \pi \mathrm{i}} \int_{\sigma-\mathrm{i} \infty}^{\sigma+\mathrm{i} \infty} \bar{f}(p) \mathrm{e}^{\mathrm{i} p} \mathrm{d} p
\end{equation}

$\bar f(p)$又称为\textbf{像函数},而$f(t)$称为\textbf{原函数}.它们之间的关系通常简单地用记号表示为:
\begin{equation}
\begin{array}{l}\bar{f}(p)=\mathscr{S}[f(t)] \\ f(t)=\mathscr{L}^{-1}[\bar{f}(p)]\end{array}
\end{equation}
或者也可以这么来表示:
\begin{equation}
\begin{array}{l}\bar{f}(p) \rightleftharpoons f(t) \\ f(t) \doteq \bar{f}(p)\end{array}
\end{equation}