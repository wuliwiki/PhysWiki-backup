% 线性方程组解的结构 2

\pentry{线性映射的结构\upref{MatLS2}}

下面我们从线性映射和向量空间的角度理解线性方程组 $\bvec A \bvec x = \bvec b$.

对线性映射 $A:X\to Y$ 以及给定 $b \in Y$, 求出所有满足
\begin{equation}
Ax = b
\end{equation}
的 $x \in X$ 的集合.

首先注意 $A$ 未必把 $Y$ 中的每个元素都射中, 所有被射中的元素的集合 $Y_1 = A(X) \subseteq Y$ 叫做线性映射的值空间\upref{LinMap}.

\begin{theorem}{线性方程}
线性映射 $A:X\to Y$ 中, 零值域为 $Y_1 = A(X)$, 那么任意 $b \in Y_1$ 的逆像\footnote{即所有满足 $Ax = b$ 的 $x$ 的集合. 详见 “映射\upref{map}”} $X_s = A^{-1}b$ 为
\begin{equation}\label{LinEq2_eq1}
X_s = X_0 + x_1
\end{equation}
其中 $x_1$ 为任意满足 $Ax_1 = b$ 的向量,  $X_0$ 为映射的零空间.
\end{theorem}
说明: \autoref{LinEq2_eq1} 表示把 $X_0$ 中的每一个向量与 $x_1$ 相加得到集合 $X_s$. 注意当 $x_1 \ne 0$ 时容易证明这不是一个向量空间(不存在零向量).

证明: 首先证明 $X_0 + x_1$ 中的元素满足 $Ax = b$.

要证明\autoref{MatLS2_the1}~\upref{MatLS2}
