% 解析函数与散度旋度
% license Xiao
% type Tutor

\pentry{复变函数的导数 柯西—黎曼条件\nref{nod_CauRie}}{nod_5ed6}

解析函数和矢量分析有紧密的关联, 如果把复平面对应到 $x$-$y$ 直角坐标系, 那么 $f(z)$ 的实部和虚部 $u(z), v(z)$ 看作矢量的 $x,y$ 分量, 那么 $f(z)$ 就定义了一个复平面上的矢量场\upref{Vfield}
\begin{equation}
\bvec f(x,y) = u(x,y)\uvec x + v(x,y)\uvec y~,
\end{equation}
而 $f(z)$ 的复共轭 $f^*(z)$ 对应的二维矢量场为
\begin{equation}
\bvec g(x,y) = u(x,y)\uvec x - v(x,y)\uvec y~.
\end{equation}

\begin{theorem}{}
函数 $f:\mathbb C\to\mathbb C$ 在某区域解析的充分必要条件是: $\bvec g(x,y)$ 是二维调和场, 即散度\upref{Divgnc}和旋度\upref{Curl}都为零。
\end{theorem}
证明留做习题。 由于 $\bvec g$ 的旋度为零, 必定存在势函数 $\phi$, 使
\begin{equation}
u = \pdv{\phi}{x} \qquad -v = \pdv{\phi}{y}~,
\end{equation}
且满足
\begin{equation}
\laplacian \phi = 0~.
\end{equation}
即 $\bvec g$ 必定可以表示为调和函数的梯度。

我们再来看 $\I f^*(z)$ 代表的矢量场 $\bvec h(x,y)$。
\begin{equation}
\I f^*(z) = \I (u - v\I) = v + u\I~,
\end{equation}
所以
\begin{equation}
\bvec h(x,y) = v(x,y)\uvec x + u(x,y)\uvec y~.
\end{equation}
由柯西公式易得 $\curl \bvec h = \bvec 0$。

\begin{theorem}{}
函数 $f:\mathbb C\to\mathbb C$ 在某区域解析的充分必要条件是: $\bvec g(x,y), \bvec h(x,y)$ 旋度都为零。
\end{theorem}
证明留做习题。
