% Weyl 旋量
% Weyl|旋量|费米子

洛仑兹群的狄拉克表示是可约的。我们可以构造两个二维表示
\begin{equation}
\psi = \begin{pmatrix}
\psi_L \\
\psi_R
\end{pmatrix}~.
\end{equation}
$\psi_L$ 被称为左手的Weyl旋量,$\psi_R$ 被称为右手的Weyl旋量。在无穷小转动 $\mathbf \theta$ 和boost $\mathbf \beta$ 下, 它们的变换规则为
\begin{align}
\psi_L \rightarrow (1-i \boldsymbol \theta \cdot \frac{\boldsymbol \sigma}{2} - \boldsymbol \beta \cdot \frac{\boldsymbol \sigma}{2})\psi_L ~, \\
\psi_R \rightarrow (1-i\boldsymbol \theta \cdot \frac{\boldsymbol \sigma}{2}+ \boldsymbol \beta \cdot \frac{\boldsymbol \sigma}{2})\psi_R ~.
\end{align}
下面这个恒等式很有用
\begin{equation}
\sigma^2\boldsymbol \sigma^* = - \boldsymbol \sigma \sigma^2~.
\end{equation}
不难证明 $\sigma^2\psi^*_L$ 像右手旋量一样变换。用 $\psi_L$ 和 $\psi_R$,我们可以把狄拉克方程写为如下的形式
\begin{equation}
(i\gamma^\mu\partial_\mu - m)\psi = \begin{pmatrix}
- m & i (\partial_0+\boldsymbol \sigma \cdot \boldsymbol \nabla) \\
i(\partial_0-\boldsymbol\sigma\cdot \boldsymbol\nabla) & -m 
\end{pmatrix} \begin{pmatrix}
\psi_L \\ \psi_R 
\end{pmatrix}=0~.
\end{equation}
从上式可以看出,两个洛仑兹群的表示 $\psi_L$ 和 $\psi_R$ 在狄拉克方程中通过质量项进行混合。如果我们把 $m$ 置成0,那么关于 $\psi_L$ 和 $\psi_R$ 的两个方程就分开了
\begin{align}
i(\partial_0 - \boldsymbol \sigma \cdot \boldsymbol \nabla) \psi_L = 0~, \\
i(\partial_0 + \boldsymbol \sigma \cdot \boldsymbol \nabla) \psi_R = 0~. 
\end{align}
这两个方程被称为Weyl方程。在处理\textbf{中微子物理}和\textbf{弱相互作用的物理}的时候尤为重要。定义
\begin{equation}
\sigma^\mu \equiv (1,\boldsymbol \sigma)~, \quad \bar \sigma^\mu \equiv (1,-\sigma)~.
\end{equation}
$\gamma$ 矩阵可以用刚才定义的这两个物理量来表示
\begin{equation}
\gamma^\mu = \begin{pmatrix}
0 & \sigma^\mu \\
\bar \sigma^\mu & 0 ~.
\end{pmatrix}
\end{equation}
用这样的记号,狄拉克方程可以写为
\begin{equation}
\begin{pmatrix}
-m & i \sigma \cdot \partial \\
i \bar \sigma \cdot \partial & -m 
\end{pmatrix}\begin{pmatrix}
\psi_L \\ \psi_R
\end{pmatrix} = 0~.
\end{equation}
Weyl方程可以写为
\begin{equation}
i \bar \sigma \cdot \partial \psi_L = 0~, \quad i \sigma \cdot \partial \psi_R = 0~.
\end{equation}

