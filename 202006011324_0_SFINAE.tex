% C++ 中的 SFINAE 技巧
% C++|SFINAE|模板

\subsection{限制模板范围}
例如我们有一个函数模板将两个对象 \verb|v1|, \verb|v2| 相加得到 \verb|v|
\begin{lstlisting}[language=cpp]
template <class T, class T1, class T2>
void Plus(T &v, const T1 &v1, const T2 &v2)
{...}
\end{lstlisting}
如果我们想要对不同类型的 T,T1,T2 执行不同的代码, 我强烈推荐下面要介绍的 SFINAE 技巧来实现.

我们先来定义一个宏函数 \verb|MY_IF()|(至于为什么要这么定义先不介绍),其中使用了 \verb|type_traits| 头文件中的 \verb|enable_if|. 这个宏用于输入一个 \verb|constexpr| 的 \verb|bool| 表达式, 由于表达式中有可能出现一个或多个逗号,所以形式上我们用了任意变量的宏函数.
\begin{lstlisting}[language=cpp]
#define MY_IF0(...) typename std::enable_if<(bool)(__VA_ARGS__), Int>::type
#define MY_IF(...) MY_IF0(__VA_ARGS__) = 0
\end{lstlisting}
这里先定义了 \verb|MY_IF0()|,看起来多此一举,但实际上在一些情况下我们也需要单独使用 \verb|MY_IF0()|.

另外, 我们假设存在一些函数模板用于判断 \verb|T, T1, T2| 的类型(具体怎么定义先不介绍), 例如如果 T 是一个矩阵,\verb|is_matrix<T>()| 就返回 \verb|true|, 而 \verb|is_scalar<T>()| 和 \verb|is_vector<T>()| 都返回 \verb|false|.
\begin{lstlisting}[language=cpp]
template <class T> bool is_scalar();
template <class T> bool is_vector();
template <class T> bool is_matrix();
\end{lstlisting}

现在我们用 \verb|MY_IF| 来区分不同版本的 Plus 函数. 标量相加的函数如下
\begin{lstlisting}[language=cpp]
template <class T, class T1, class T2,
MY_IF(is_scalar<T>() && is_scalar<T1>() && is_scalar<T2>())>
void Plus(T &v, const T1 &v1, const T2 &v2)
{ v = v1 + v2; }
\end{lstlisting}

注意在我们通过 \verb|MY_IF| 声明了这个模板什么时候有定义(只有 \verb|T, T1, T2| 为标量时有定义, 例如 \verb|int, double, complex| 等).

同样, 我们可以再写一个版本的 Plus 定义矢量相加
\begin{lstlisting}[language=cpp]
template <class T, class T1, class T2,
MY_IF(is_vector<T>() && is_vector<T1>() && is_vector<T2>())>
void Plus(T &v, const T1 &v1, const T2 &v2)
{
    for (int i = 0; i < v.size(); ++i) {
        v[i] = v1[i] + v2[i];
    }
}
\end{lstlisting}

我们还可以再定义矩阵相加,矩阵与标量相加,矢量与标量相加等等.

如果我们的函数需要先 declare 再 define(例如 class 的成员函数在 class 定义中 declare,然后在别的地方 define),那就需要在 declaration 中使用 \verb|MY_IF()|,而在 definition 中使用 \verb|MY_IF0()|,其他内容都一样.

使用 SFINAE 的好处是, 我们可以限制函数模板 instantiate 的条件, 使得一些不合法的使用变得没有定义(比如说我想要用 Plus 把矩阵和矢量相加,又比如复数矩阵相加得到实数矩阵). 另一个好处是无论我们定义多少个版本的 Plus, 只要 \verb|MY_IF| 中的条件总在唯一一个版本中为 \verb|true|, 编译器就不会抱怨无法判断使用哪个版本的 Plus 函数.

\subsection{类型判断}
我们下面来介绍这些函数如何实现.

首先, 它们必须是 constexpr 函数, 也就是说它们必须要能在编译阶段(而不是运行阶段)被调用并返回结果. 其次, 它们是模板函数, 因为类型 T 不可能作为函数参数, 而只能作为模板参数.

我们首先定义 \verb|is_same<T1, T2>()| 函数模板来判断 \verb|T1,  T2| 两个类型是否相同(这里用到了 标准库的 \verb|type_traits| 头文件)
\begin{lstlisting}[language=cpp]
template <class T1, class T2>
constexpr bool is_same()
{ return std::is_same<T1, T2>::value; }
\end{lstlisting}

注意在 \verb|type_traits| 中, \verb|is_| 开头的函数都是类模板而不是函数模板, 理论上我们可以直接拿 \verb|std::is_same| 来用, 但为了概念和使用上更简单, 我们重新定义了同名的函数模板.

有了 \verb|is_same|, 我们就可以很容易地实现 \verb|is_int<T>()|, \verb|is_double<T>()|, 等.
\begin{lstlisting}[language=cpp]
template <class T>
constexpr bool is_int()
{ return is_same<T, int>(); }

template <class T>
constexpr bool is_double()
{ return is_same<T, double>(); }
\end{lstlisting}

\verb|is_complex<T>()| 有所不同, 因为 \verb|std::complex| 本身也是一个类模板, 可以有不同的类. 这里为了简单, 姑且就假设我们只使用 \verb|std::complex<double>|. 如果需要支持任意类型的 \verb|std::complex<>|, 需要使用下文中定义 \verb|is_vector| 的方法.
\begin{lstlisting}[language=cpp]
template <class T>
constexpr bool is_complex()
{ return is_same<T, std::complex<double>>(); }
\end{lstlisting}

为了简单起见, 我们假设“标量”只包括 \verb|int, double, std::complex<double>| 三种, 于是可以定义用于判断标量的函数
\begin{lstlisting}[language=cpp]
template <class T>
constexpr bool is_scalar()
{ return is_int<T>() || is_double<T>() || is_complex<T>(); }
\end{lstlisting}

现在我们再来看如何定义 \verb|is_vector<T>()|. 对于矢量, 我们既可以使用 \verb|std::vector<>|, 也可以自己定义一个矢量类型. 我们通过 template specialization 来实现 \verb|is_vector<>|
\begin{lstlisting}[language=cpp]
template <class T> struct is_vector_imp : std::false_type {};
template <class T> struct is_vector_imp<vector<T>> : std::true_type {};
template<class T>
constexpr bool is_vector()
{
	return is_vector_imp<T>();
}
\end{lstlisting}

首先我们对一般的类型 \verb|T| 定义的 \verb|is_vector_imp|(imp 这里表示 implementation), 继承 \verb|std::false_type| (你只需要知道 \verb|falst_type| 是一个类, 它的对象可以自动转换为 false, \verb|true_type| 类的对象可以自动转换为 \verb|true|). 然后对凡是符合 \verb|is_vector_imp<vector<T>>| 格式的类进行不同的定义: 即继承 \verb|true_type|. 最后, 我们再把 \verb|is_vector_imp| 已经实现的功能封装成函数模板 \verb|is_vector<T>()|, 就大功告成了.

用于判断矩阵类型的 \verb|is_matrix<T>()| 也可以如法炮制
\begin{lstlisting}[language=cpp]
template <class T> struct is_matrix_imp : std::false_type {};
template <class T> struct is_matrix_imp<Matrix<T>> : std::true_type {};
template<class T>
constexpr bool is_matrix()
{
	return is_vector_imp<T>();
}
\end{lstlisting}
然而注意标准库中没有矩阵类型(据说其实有一个,后来烂尾了,几乎没人用), 所以我们一般自己定义矩阵类型 \verb|Matrix<T>|.
