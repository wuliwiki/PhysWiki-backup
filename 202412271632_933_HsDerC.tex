% 导数的性质与构造(高中)
% keys 导数|性质|构造|恒等
% license Usr
% type Tutor

\begin{issues}
\issueDraft
\end{issues}

\pentry{导数\nref{nod_HsDerv},函数的性质\nref{nod_HsFunC}}{nod_139b}

导数是用来研究函数变化规律的重要工具,它和原函数之间有很多有趣的联系,导函数就像原函数的素描画,虽然少了某些信息,但这些简化的信息足够掌握函数的主要特征,快速理解它的变化趋势。在一些简单的性质中,导函数的\textbf{奇偶性(parity)}通常与原函数相反,而它们的\textbf{周期性(periodicity)}却保持一致。这些基本性质为后续更深入的学习打下了基础。

通过导数,可以更清楚地了解函数的行为。例如,可以用导数判断函数在某个区间是递增还是递减,图像是“开口向上”还是“开口向下”,以及找出函数的最高点或最低点。这些特点就像是给函数贴上的标签,让人一目了然地知道它在不同位置的变化情况。


在解题过程中,导数的这些性质经常用于设计方法解决问题,主要涉及以下三种问题形式:
	1.	最值问题:通过分析导数为零的位置,确定函数的最高点或最低点。
	2.	寻找零点:借助导数符号变化的特性,结合\aref{零点存在定理}{the_HsFunC_1},确定函数在某区间内是否存在零点。
	3.	恒成立问题:利用导数的符号变化,分析函数在某区间的单调性和极值符号,判断是否满足恒成立条件。

具体解题方法的核心是找到导数为零的位置并分析导数的正负变化,从而大致刻画出函数的趋势和形态。例如,通过研究导数的符号,可以确定函数在哪些区间内递增或递减;通过分析区间上符号相异的函数值,可以结合零点存在定理推断函数零点的存在性。如何选取相异的函数值也需要细致考量,这是解题中的一个关键细节。


\subsection{近似代替}

在导数的\aref{几何含义}{sub_HsDerv_1}中就提到过“以直代曲”。

\begin{equation}
f(x_0+\Delta x)\approx f(x_0)+f'(x_0)\Delta x~.
\end{equation}

\subsection{单调性和极值点}
\subsubsection{单调性}
在介绍\aref{导函数}{sub_HsDerv_2}时,提及区间的中函数的增减与导函数的符号相关。
\begin{theorem}{单调性与导数的关系}
导函数  $f'(x)$  代表了原函数  f(x)  图像在每一点的切线斜率。
\begin{itemize}
\item 在$f'(x)>0$的区间上,原函数的图像单调递增。
\item 在$f'(x)<0$的区间上,原函数的图像单调递减。
\item 在$f'(x)=0$的区间上,原函数的图像是水平的。
\end{itemize}
\end{theorem}


\subsubsection{极值点}

$f'(x) = 0$的点表示函数的输出值停止增加或减少的点,被称为\textbf{驻点}。在该点是水平的,可能是极值点。

严格的来说,在点x0某一邻域内,f(x0)>=f(x)或f(x0)<=f(x),x0才是极值点。
f’(x)=0的点并不一定是极值点,在该点附近f’(x)必须要变号,如y=x^3,在x=0处一阶导数为0,但两侧不变号,就不是极值点
极值点与该点的一阶导数是否存在无关,只要两侧的f’(x)变号,就是极值点,如y=|x|,在x=0处就是极值点,但在x=0处一阶导数不存在。

极值点

\subsection{高阶导数}

导函数作为原函数,则又可以求得它的导函数,这也被称为高阶导数。

凹凸性

连续曲线的凹弧与凸弧的交界点。(在该点的二阶导并不一定有定义)
f"(x)=0,且该点 两侧 二阶导数变号,那么该点就是极值点。
当然,可能在一点x0处,二阶导数并不存在,在x0左侧的二阶导数趋于正无穷,右侧的二阶导数趋于负无穷,该点也是拐点。

\subsection{常用构造}

\pentry{恒等式与不等式恒成立\nref{nod_HsIden}}{nod_b069}

\subsubsection{逆向使用求导法则}

逆用积法则:

$x^n f'(x) + f(x) \geq 0$,构造 $F(x) = x^n f(x)$,$[x^n f(x)] = x^n f'(x) + nx^{n-1} f(x) = x^{n-1} [x f'(x) + nf(x)]$。特别地,当$n=1$时有$x f'(x) + f(x) \geq 0$,构造 $F(x) = x f(x)$,$[x f(x)]' = x f'(x) + f(x)$

$f'(x) + k f(x) \geq 0$,构造 $F(x) = \E^{kx} f(x)$,$[e^{kx} f(x)]' = e^{kx} [f'(x) + kf(x)]$。特别地,当$k=1$时有$f'(x) + f(x) \geq 0$,构造 $F(x) = \E^x f(x)$,$[e^x f(x)]' = e^x [f'(x) + f(x)]$

逆用商法则:

$xf'(x) - f(x) \geq 0$,构造 $F(x) = \frac{f(x)}{x}$,  
    $\therefore \left[\frac{f(x)}{x}\right]' = \frac{f'(x) \cdot x - f(x)}{x^2}$

$f'(x) - f(x) \geq 0$,构造 $F(x) = \frac{f(x)}{e^x}$,  
    $\therefore \left[\frac{f(x)}{e^x}\right]' = \frac{e^x \cdot f'(x) - e^x \cdot f(x)}{e^{2x}} = \frac{f'(x) - f(x)}{e^x}$

$x^n f'(x) - n f(x) \geq 0$,构造 $F(x) = \frac{f(x)}{x^n}$,  
    $\therefore \left[\frac{f(x)}{x^n}\right]' = \frac{x^n \cdot f'(x) - n x^{n-1} \cdot f(x)}{x^{2n}} = \frac{f'(x) - n f(x)}{x^{n+1}}$

$f'(x) - k f(x) \geq 0$,构造 $F(x) = \frac{f(x)}{e^{kx}}$,  
    $\therefore \left[\frac{f(x)}{e^{kx}}\right]' = \frac{e^{kx} \cdot f'(x) - k e^{kx} \cdot f(x)}{e^{2kx}} = \frac{f'(x) - k f(x)}{e^{kx}}$
