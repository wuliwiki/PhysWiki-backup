% 近自由电子模型
% 自由电子气体|布洛赫波|晶格|能带

\pentry{金属中的自由电子气体\upref{mfcgas},布洛赫理论\upref{Bloch}}

在自由电子模型中,我们将电子视作了满足周期性边界条件的平面波(理想气体单粒子能级密度\upref{IdED1}),也就是将电子气体视作了理想气体,来采用相应的能态密度 $D(\epsilon)$.在这样的近似下,我们得到了重要的结果,例如不同能量电子的分布,电子气体对金属热容的贡献等.但这种模型许多严重的缺陷,例如它不能解释为什么有些固体是金属,而有些固体是绝缘体;它不能解释固体丰富的热学、电学或光学性质.为此,我们需要的是\textbf{近自由电子模型},将固体中周期势场作为微扰项考虑进来,考察周期势场下的电子波函数的行为及能带关系.近自由电子模型几乎给出了关于金属中电子行为所有的定性问题的答案.
\subsection{微扰论与布洛赫函数}
\subsection{能带理论}