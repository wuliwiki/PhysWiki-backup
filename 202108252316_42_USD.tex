% 单位制和量纲
% keys 单位制|量纲|基本量类|导出量类

\begin{issues}
\issueTODO
\end{issues}

\pentry{现象类\upref{PHEC}}

\subsection{为什么要引入单位制?}
为回答这一问题,先来看一个例子.欧姆定律的数值表达式为
\begin{equation}\label{USD_eq1}
U=IR
\end{equation}
其中 $U,I,R$ 分别是以 \textbf{$\boldsymbol{V}$ }、\textbf{ $\boldsymbol{A}$ }和\textbf{ $\boldsymbol{\Omega}$ }为单位测量问题中的电压 $\boldsymbol{U},\boldsymbol{I},\boldsymbol{R}$ 所得的数.为明确起见,把\autoref{USD_eq1} 写为
\begin{equation}
U_{V}=I_{A}R_{\Omega}
\end{equation}
若以 \textbf{$\boldsymbol{mA}$} 测量电流,并把所得的数记作 $I_{mA}$ ,由\autoref{QCU_eq8}~\upref{QCU} 
\begin{equation}\label{USD_eq2}
I_{A}=\frac{\boldsymbol{mA}}{\boldsymbol{A}}I_{mA}=10^{-3}I_{mA}
\end{equation}
\autoref{USD_eq1} 代入 \autoref{USD_eq2} ,便得
\begin{equation}
U_{V}=10^{-3}I_{mA}R_{\Omega}
\end{equation}
为了简洁起见,通常都去掉下标,于是就有
\begin{equation}\label{USD_eq3}
U=10^{-3}IR
\end{equation}
\autoref{USD_eq1} 和\autoref{USD_eq3} 都称为欧姆定律,两者不同的原因在于采用不同的单位搭配.

通过上面的例子,不难想象,同一规律的各个数值表达式之间的差别仅体现在一个附加因子.因此,只需把式\autoref{USD_eq1} 改写为
\begin{equation}
U=kIR
\end{equation}
便能在任何单位搭配下成立.上式中 $k$ 依赖于式中各量所选的单位.

每一量类中的单位原则上可任选,但这会导致大量的数值表达式中的 $k$ 值复杂得难以记住.为克服这一困难,可用单位制来约束各个量类单位的选法.
\subsection{单位制}
一个\textbf{单位制}由以下3个要素构成:
\begin{enumerate}
\item 选定 $l$ 个量类 $\tilde{\boldsymbol{J}}_1,\cdots,\tilde{\boldsymbol{J}}_l$ 作为\textbf{基本量类}(个数和选法有相当任意性),其它量类一律称为\textbf{导出量类}.  
\item 对每一基本量类 $\tilde{\boldsymbol{J}}_i(i=1,\cdots,l)$ 任选一单位 $\hat{\boldsymbol{J}}_i$ ,称为\textbf{基本单位}.
\item 对每一导出量类 $\tilde{\boldsymbol{C}}$,利用一个适当的、涉及 $\tilde{\boldsymbol{C}}$ 的物理规律来定义它的单位,称为\textbf{导出单位}.
\end{enumerate}
\begin{example}{CGS制中速度量类 $\tilde{\boldsymbol{v}
}$ 单位的定义}
CGS单位制指定长度、质量和时间为基本量类,并选 $\boldsymbol{cm},\boldsymbol{g},\boldsymbol{s}$ 为基本单位.为定义速度 $\tilde{\boldsymbol{v}}$ (导出量类)的单位,考虑“质点做匀速直线运动”这一现象类\upref{PHEC}(它包括质点以各种不同速度做匀速直线运动这一现象).设质点在 $t\boldsymbol{s}$ 内走了 $l\boldsymbol{cm}$ ,以 $v$ 代表用任一速度单位 $\hat{\boldsymbol{v}}$ 测质点速度 $\boldsymbol{v}$ 所得的数,则有如下物理规律:
\begin{equation}
v=k\frac{l}{t}
\end{equation}
其中 $k$ 反应速度单位 $\hat{\boldsymbol{v}}$ 的任意性,与具体现象无关.指定 $k=1$ 便指定了一个确切的速度单位.具体说,$k=1$ 使上式简化为
\begin{equation}\label{USD_eq4}
v=\frac{l}{t}
\end{equation}
上式起到给速度的 CGS制单位 $\hat{\boldsymbol{v}}_{CGS}$ 下定义的作用,称为导出单位 $\hat{\boldsymbol{v}}_{CGS}$ 的\textbf{定义方程}.为看出这一 $\hat{\boldsymbol{v}}_{CGS}$ 是怎样一种速度,可在这一现象类中选出任一现象:质点在 $\boldsymbol{t}=t_0\boldsymbol{s}$ 内走了 $\boldsymbol{l}=l_0\boldsymbol{cm}$,代入\autoref{USD_eq4} 得 $v=l_0/t_0$ ,即有
\begin{equation}
\boldsymbol{v}=\frac{l_0}{t_0}\hat{\boldsymbol{v}}_{CGS}
\end{equation}
通常,我们令 $l_0=t_0=1$,这时 $\boldsymbol{v}=1\hat{\boldsymbol{v}}_{CGS}$ ,即 $\hat{\boldsymbol{v}}_{CGS}$ 代表每 $\boldsymbol{s}$ 走 $1\boldsymbol{cm}$ 这样一种速度.通常写为
\begin{equation}
\hat{\boldsymbol{v}}_{CGS}=\boldsymbol{cm/s}
\end{equation}
当然,任一现象都是可取的,因为对任一现象,都满足 $\boldsymbol{v}=v\hat{\boldsymbol{v}}$ ,而 $v=v/1$ 表示质点每 $\boldsymbol{s}$ 走 $v\boldsymbol{cm}$,即 $\boldsymbol{v}=v\boldsymbol{cm/s}$,与$\boldsymbol{v}=v\hat{\boldsymbol{v}}_{CGS}$ 结合,得 $\hat{\boldsymbol{v}}_{CGS}=\boldsymbol{cm/s}$.

\end{example}
上面例子中,需注意 "$\boldsymbol{cm/s}$" 只是 “每 $\boldsymbol{s}$ 走 $1\boldsymbol{cm}$ ”这样一种速度 的记号,切莫看成量的除法.
\begin{example}{CGS制中加速度量类$\tilde{\boldsymbol{a}}$单位的定义}
考虑 “质点从静止开始做匀加速直线运动”这一现象类.设 $t\boldsymbol{s}$ 末的速度为 $v\boldsymbol{cm/s}$ ,以 $a$ 代表用任一加速度单位 $\hat{\boldsymbol{a}}$ 测该质点加速度所得的数,则
\begin{equation}\label{USD_eq6}
a=k\frac{v}{t}
\end{equation}
其中 $k$ 反应 $\hat{\boldsymbol{a}}$ 的任意性.指定 $k=1$ 便指定了CGS制的加速度单位 $\hat{\boldsymbol{a}}_{CGS}$ .$k=1$ 使上式简化为
\begin{equation}\label{USD_eq5}
a=v/t
\end{equation}
当 $t=1$ 及 $v=1$ 时 $a=1$ ,可见 $\hat{\boldsymbol{a}}_{CGS}$ 是速度每 $\boldsymbol{s}$ 增加 $1\boldsymbol{m/s}$ 这样一种加速度,通常写成
\begin{equation}
\hat{\boldsymbol{a}}_{CGS}=\boldsymbol{cm/s}^2
\end{equation}
若将 $\hat{\boldsymbol{a}}_{CGS}$ 的定义方程\autoref{USD_eq5} 写成只涉及基本量类的形式,所得方程就称为  $\hat{\boldsymbol{a}}_{CGS}$ 的\textbf{终极定义方程}.与之区别,定义方程\autoref{USD_eq6} 称为\textbf{原始定义方程}.考虑现象类 “质点从静止开始开始做匀加速运动,然后以末速度做匀速运动”显然在匀加速阶段,加速度、时间和末速度有\autoref{USD_eq5} 的关系,而在匀速阶段,便有\autoref{USD_eq4} 的关系,两式联立得
\begin{equation}\label{USD_eq7}
a=l/t^2
\end{equation}
此即 $\hat{\boldsymbol{a}}_{CGS}$ 的终极定义方程.也可考虑现象类 “质点从静止开始做匀加速运动”,并设它在 $t\boldsymbol{s}$ 内走了 $l/2\boldsymbol{cm}$ ,也得到\autoref{USD_eq7} 
\end{example}
上面两个例子里,给出导出单位的方程时都说明了它所依托的现象类.事实上,对于每一导出单位都必须给出其所依托的现象类,因为同一定义方程配以不同的现象类可能定义出不同的导出单位.下面是一个例子
\begin{example}{}
许多单位制都选长度为一基本量类,选面积为导出量类,导出单位 $\hat{\boldsymbol{S}}$ 的定义方程为 $S=ab$,其中 $S$ 和 $a,b$ 依次是矩形面积和每个边长.如长度基本单位是 $\boldsymbol{m}$,则面积单位是 $\boldsymbol{m}^2$(\textbf{方米}).可见导出单位 $\boldsymbol{m}^2$ 的定义方程 $S=ab$ 所依托的是矩形现象类.然而,若改用三角形现象类,同样的定义方程 $S=ab$ 给出的是 \textbf{角米}.1\textbf{角米}=\textbf{方米}/2.
\end{example}
\subsection{量纲}
物理学中常常遇到改变单位制的情况,假定问题涉及两个单位制(分别称为“旧制”和“新制”),人们当然关心任一物理量类 $\tilde{\boldsymbol{Q}}$ 在旧、新两制的单位的比值,即 
\begin{equation}\label{USD_eq8}
\frac{\hat{\boldsymbol{Q'}}}{\hat{\boldsymbol{Q}}}
\end{equation}
其中 $\hat{\boldsymbol{Q}},\hat{\boldsymbol{Q'}}$ 分别表示 $\tilde{\boldsymbol{Q}}$ 在旧、新两制中的单位.这一比值\autoref{USD_eq8} 称为 $\tilde{\boldsymbol{Q}}$ 的\textbf{量纲}(dimension),记作 $\mathrm{dim}\tilde{\boldsymbol{Q}}$ ,即
\begin{equation}
\mathrm{dim}\tilde{\boldsymbol{Q}}=\frac{\hat{\boldsymbol{Q'}}}{\hat{\boldsymbol{Q}}}
\end{equation}
通常,记长度、质量、时间的量纲分别为 $\mathrm{L},\mathrm{M},\mathrm{T}$,即
\begin{equation}
\mathrm{L}\equiv\mathrm{dim}\tilde{\boldsymbol{l}},\quad\mathrm{M}\equiv\mathrm{dim}\tilde{\boldsymbol{m}},\quad
\mathrm{T}\equiv\mathrm{dim}\tilde{\boldsymbol{t}}
\end{equation}

所选单位制的基本量类的量纲称为该单位制的\textbf{基本量纲}.而 $\tilde{\boldsymbol{l}},\tilde{\boldsymbol{m}},\tilde{\boldsymbol{t}}$ 是国际制的基本量类,所以 $L,M,T$ 其实就是国际制的基本量纲.

由于导出单位由基本单位通过定义方程来定义,基本单位的改变自然会引起导出单位的改变.为描述导出量类的单位变换如何依从于基本量类的单位变换,需引入\textbf{量纲式}.
\begin{definition}{}
描述导出量类 $\hat{\boldsymbol{C}}$的量纲与基本量纲的 \textbf{量纲式}
\end{definition}

现在有个问题尚需解决,如果两个单位制连基本量类都不同,如何谈导出单位随基本单位的改变而改变?这需要引进单位制族的概念.
\begin{definition}{}
两个单位制称为\textbf{同族}的,若满足:
\begin{enumerate}
\item 基本量类相同 
\item 所有导出单位的定义方程(及其所依托的现象类)在两制中式相同的.
\end{enumerate}
\end{definition}

由量纲的定义,