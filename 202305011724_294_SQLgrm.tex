% SQL 入门语法

\begin{issues}
\issueDraft
\end{issues}

\subsection{SQL 基本语法}

\subsubsection{SQL 库操作}
\subsubsection{1.结构创建}
\begin{itemize}
\item CREATE 结构类型 结构名 结构描述;

\end{itemize}
\subsubsection{2.显示结构}
\begin{itemize}
\item 显示结构: show 结构类型(复数)
\item 显示结构创建详情: show create 结构类型 结构名;
\end{itemize}
\subsubsection{3.数据操作(数据表)}

\begin{itemize}
\item 新增数据: INSERT INTO 表名 VALUES
\begin{lstlisting}[language=bash]
例如:向学生表中插入学生数据:张三 231119001  男 23
INSERT INTO students
VALUES
('张三','231119001','男','北京');
\end{lstlisting}
\subsubsection{数据查询}
\begin{lstlisting}[language=bash]
//数据查询的一般格式为
SELECT [ALL|DISTINCT]<目标列表达式>[,<目标列表达式>]...
FROM <表名或视图名>[,<表名或视图名>..]|(<SELECT语句>)[AS]<别名>
[WHERE <条件表达式>]
[GROUP BY<列名 1> [HAVING <条件表达式> ]]
[ORDER BY<列名 2> [ASC|DESC]];
\end{lstlisting}
\item 查询指定列
\begin{lstlisting}[language=bash]
例如:查询全体学生的学号与姓名
SELECT Sno,Sname
FROM Stuent;

例如:从学生表查询名字为张三的学号
SELECT ID
FROM students
WHERE name='张三';



\end{lstlisting}
\item  查询全部列
\begin{lstlisting}[language=bash]
例如:查询全体学生的详细记录
SELECT *
FROM Student;
\end{lstlisting}
\item 查询满足条件的元组
\begin{table}[ht]
\centering
\caption{请输入表格标题}\label{tab_SQLgrm1}
\begin{tabular}{|c|c|}
\hline
查询条件& 谓词 \\
\hline
比较 & =,>,<,>=,<=,<>,!>,!<;NOT+上述比较运算符 \\
\hline
确定范围& BETWEEN AND,NOT BETWEEN AND \\
\hline
确定集合 & IN, NOT IN \\
\hline
字符匹配 & LIKE,NOT LIKE\\
\hline
空值& IS NULL<IS NOT NULL \\
\hline
多重条件 & AND,OR,NOT \\
\hline
\end{tabular}
\end{table}
1.比较大小
\begin{lstlisting}[language=bash]
例1:查询计算机科学系全体学生的名单
SELECT Sname
FROM Student
WHERE Sdept='计算机科学';

例2:查询所有年龄在20岁一下的学生姓名以及年龄
SELECT Sname,Sage
FROM Student
WHERE Sage<20;
\end{lstlisting}
2.确定范围
\begin{lstlisting}[language=bash]
例1:查询年龄在20-33岁(包括20岁和23岁)之间的学生姓名
SELECT Sname
FROM Student
WHERE Sage BETWEEN 20 AND  23;

例2:查询不在年龄在20-33岁之间的学生姓名
SELECT Sname
FROM Student
WHERE Sage NOT BETWEEN 20 AND  23;
\end{lstlisting}
3.确定集合
\begin{lstlisting}[language=bash]
例1:查询计算机科学系,数学系学生的姓名和性别
SELECT Sname,Ssex
FROM Student
WHERE Sdept IN('计算机系','数学系')

例2:查询不是计算机科学系,也不是数学系学生的姓名和性别
SELECT Sname,Ssex
FROM Student
WHERE Sdept NOT IN('计算机系','数学系')
\end{lstlisting}
4.字符匹配
\begin{lstlisting}[language=bash]
例1:从学生表查询姓名为张的学号
// % 表示任意多个字符
//_ 表示任意多个任意字符

SELECT ID
FROM students
WHERE  name like '张%' ;
或者
SELECT ID
FROM students
WHERE  name like '张_' ;

例2:查询名字中第二个字为“文”的学生的姓名和学号
SELECT Sname,Sno
FROM Student
WHERE Sname LIKE '_文%';
\end{lstlisting}

5.涉及空置的查询
\begin{lstlisting}[language=bash]
例如:某些学生选修课程后没有参加考试,所以有选课记录,但没有考试成绩。
查询缺少成绩的学生的学号和课程号

SELECT  Sno,Cno
FROM SC
WHERE Grade IS NULL;   //分数Grade 是空值
\end{lstlisting}
6.多重条件查询
\begin{lstlisting}[language=bash]
例如:查询计算机科学系年龄在20岁yi
\end{lstlisting}
\item 更新数据: UPDATE 表名 SET
\begin{lstlisting}[language=bash]
例如:在学生表中,将学生张三的id改为231119002
UPDATE students 
SET id='231119002'
WHERE name='张三';



另外,SET后面除了带文本之外,还可以是数学表达式

例如:将学生id为231119002的学生对应的年龄加2
UPDATE students 
SET age=age+2
WHERE id='231119019'
\end{lstlisting}

\item 删除数据:DELETE FROM 表名
\begin{lstlisting}[language=bash]
例如:删除students表中名字为张三的信息

DELETE FROM students 
WHERE name='张三';
\end{lstlisting}


\end{itemize}

\subsection{数据库操作}
\subsubsection{创建数据库:根据项目需求创建一个存储数据库的仓库}
\begin{itemize}

\item 使用create database 数据库名字创建
\item create database 数据库名字
\begin{lstlisting}[language=bash]
create database db_1
\end{lstlisting}

\end{itemize}
\begin{enumerate}
\item 数据库的创建是存储数据库的基础,数据库的创建通常是一次性的
\item 创建数据库的语法包含几个部分
\begin{itemize}
\item 关键字 create database
\item 数据库名字:自定义名字
\item  数字,字母和下划线组成
\item 数字不能开头
\end{itemize}
\item  创建好的数据库可以在数据存储指定地点看到



\end{enumerate}

\subsubsection{显示数据库:通过客户端指令来查看已有的数据库}
\begin{enumerate}
\item 数据库的查看是根据用户权限限定的
\item 数据库的查看分为两种查看方式:
\begin{itemize}
\item 查看全部数据库:show database;
\begin{lstlisting}[language=bash]
show  database;
\end{lstlisting}
\item 查看数据库创建指令:show crete database 数据库名字;
\begin{lstlisting}[language=bash]
show crete database db_1;
\end{lstlisting}
\end{itemize}

\end{enumerate}


\subsubsection{使用数据库}
\begin{enumerate}
\item 使用数据库的指令是:use 数据库名字;
\begin{lstlisting}[language=bash]
use db_1;
\end{lstlisting}
\item 使用数据库的目标
\begin{itemize}
\item 让系统知道后续SQL指令都是针对当前选择的数据库
\item  简化后续SQL指令的复杂度
\end{itemize}

\end{enumerate}

\subsubsection{修改数据库:修改数据库的相关库选项}
\begin{enumerate}
\item 数据库名字不可修改
\begin{itemize}
\item 先新增
\item 后迁移
\item 最后删除
\end{itemize}
\item 数据库修改分为两个部分
\begin{itemize}
\item 字符集
\item 校对集
\end{itemize}
\item 数据库修改指令:alter database数据库名字
\end{enumerate}
\subsubsection{删除数据库:将当前已有数据库删除}
\begin{itemize}
\item 删除数据库会删除数据库内所有的表和数据
\item 删除数据库操作要慎重(删前备份)
\item 删除数据库后,对应的存储文件夹就会消失
\item 删除语法:drop database 数据库名字
\begin{lstlisting}[language=bash]
drop database db_1
\end{lstlisting}
\end{itemize}

\subsection{数据定义}
SQL的数据定义语句





\subsubsection{模式的定义和删除}
\begin{enumerate}
\item 定义模式
\begin{lstlisting}[language=bash]
CREATE SCHEMA <模式名> AUTHORIZATION <用户名>;
\end{lstlisting}
如果没有指定 <模式名> ,那么 <模式名> 隐含为 <用户名>。

\begin{lstlisting}[language=bash]
例如:为用户zhang 创建一个模式 test,并且在其中定义一个表tab1
CREATE SCHEMA test AUTHORIZATION zhang
CREATE TABLE tab1(
    COL1 SMALLINT,
    COL2 INT,
    COL3 CHAR(20)
);

\end{lstlisting}
\item 删除模式
\begin{lstlisting}[language=bash]
DROP SCHEMA <模式名><CASCADE|RESTRICT>;
其中 CASCADE和RESTRICT两者必选其一。
\end{lstlisting}
\end{enumerate}
\subsubsection{基本表的定义,删除和修改}
\begin{enumerate}
\item 定义基本表
\begin{lstlisting}[language=bash]
CREATE TABLE <表名> (
<列名><数据类型>[列级完整性约束条件]
[,<列名><数据类型>[列级完整性约束条件]]
...
[,<表级完整性约束条件>]
);
\end{lstlisting}
\begin{lstlisting}[language=bash]
例如:建立一个”学生”表Student
CREATE TABLE Student(
    Sno CHAR(9) PRIMARY KEY, /*列级完整性约束条件,Sno是主码*/
    Sname CHAR(20)  UNIQUE,  /*Sname取唯一值*/ 
    Ssex CHAR(2),
    Sage SMALLINT,
    Sdept CHAR(20)
);

\end{lstlisting}
\item 数据类型
\begin{table}[ht]
\centering
\caption{数据类型}\label{tab_SQLgrm}
\begin{tabular}{|c|c|}
\hline
数据类型 & 含义 \\
\hline
CHAR(n),CHARACTER(n) & 长度为n的定长字符串\\
\hline
VARCHAR(n),CHARACTERVARYING(n) & 最大长度为n的边长字符串 \\
\hline
CLOB& 字符串大对象\\
\hline
BLOB & 二进制大对象 \\
\hline
INT,INTEGER & 长整型(4字节) \\
\hline
SMALLINT & 短整型(2字节) \\
\hline
BIGINT& 大整数(8字节) \\
\hline
\end{tabular}
\end{table}
\item 修改基本表
\begin{lstlisting}[language=bash]
ALTER TABLE<表明>
      //ADD子句用于增加新列,新的列级完整性约束条件和新的表级完整性约束条件
[ADD[COLUMN]<新列名><数据类型>[完整性约束]
[ADD<表级完整性约束>]
     //DROP COLUMN 子句用于删除表中的列
[DROP[COLUMN]<列名>[CASCADE|RESTRICT]]
     //DROP CONSTRAINT 子句用于删除指定的完整性约束条件
[DROP CONSTRAINT <完整性约束名>[RESTRICT|CASCADE]]
     //ALTER COLUMN子句用于修改原有的列定义,包括修改列名和数据类型
[ALTER COLUMN<列名><数据类型>];
\end{lstlisting}
\end{enumerate}
\subsubsection{索引的建立和删除}
\begin{enumerate}
\item 建立索引
\begin{lstlisting}[language=bash]
CREATE [UNIQUE] [CLUSTER] INDEX <索引名>
ON <表名>(<列名>[<次序>] [,<列名>[<次序>]]...)
\end{lstlisting}
\item 修改索引
\begin{lstlisting}[language=bash]
ALTER INDEX <旧索引名> RENAME TO <新索引名>
\end{lstlisting}
\item  删除索引
\begin{lstlisting}[language=bash]
DROP INDEX <索引名>
\end{lstlisting}
\end{enumerate}


