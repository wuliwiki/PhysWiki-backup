% 线性映射的张量积
% keys 对称积|反对称积|对称幂|反对称幂
% license Xiao
% type Wiki

\begin{issues}
\issueDraft
\issueOther{可以对照张量的对称化和交错化\upref{SIofTe}进行阅读}
\end{issues}
\pentry{向量空间的张量积\upref{TPofSp}}

\subsection{线性映射的张量积}

两个线性映射 $f: V \to W$,$g: V' \to W'$ 之间可以定义它们的张量积\footnote{更严格的写法应该是 $\sum_i v_i \otimes v_i' \mapsto \sum_i f(v_i) \otimes g(v_i')$,不过本文的所有映射都是线性映射,所以只需要定义一组基的线性变换即可。}
\begin{equation}
\begin{aligned}
f \otimes g: V \otimes V' &\to W \otimes W', \\
v \otimes v' &\mapsto f(v) \otimes g(v')~.
\end{aligned}
\end{equation}
此时 $f \otimes g \in L(V \otimes V', W \otimes W')$。

从另一个角度来说,全体 $V$ 到 $W$ 的线性映射的集合 $L(V, W)$ 是一个向量空间(参考\autoref{sub_LinMap_1}~\upref{LinMap}),因此可以定义两个线性映射空间之间的张量积 $L(V, W) \otimes L(V', W')$;用这种方法我们也可以定义线性映射的张量积,此时 $f \otimes g \in L(V, W) \otimes L(V', W')$。这两种定义并不完全等价。

要构造两种张量积之间的对应关系,我们需要考虑双线性映射
\begin{equation}
\begin{aligned}
L(V, W) \times L(V', W') &\to L(V \otimes V', W \otimes W')~, \\
(f, g) &\mapsto f \otimes_1 g~,
\end{aligned}
\end{equation}
其中 $f \otimes_1 g \in L(V \otimes V', W \otimes W')$,是第一种张量积。

\addTODO{张量积的万有性质}

由于这个映射是双线性的,根据张量积的万有性质(TODO)可以得到一个唯一的映射
\begin{equation}
\begin{aligned}
\iota: L(V, W) \otimes L(V', W') &\to L(V \otimes V', W \otimes W')~, \\
f \otimes_2 g &\mapsto f \otimes_1 g~,
\end{aligned}
\end{equation}
其中 $f \otimes_2 g \in L(V, W) \otimes L(V', W')$ 是第二种张量积。

\begin{theorem}{}
线性映射空间的张量积是张量积的线性映射空间的子集,即\footnote{用更严谨的说法是,它们之间存在一个(典范的)线性映射}
\begin{equation}
L(V, W) \otimes L(V', W') \subseteq L(V \otimes V', W \otimes W')~.
\end{equation}
如果 $V, W$ 是有限维度向量空间,那么上述包含关系相等。
\end{theorem}

这个定理中,我们把 $f \otimes_2 g$ 和 $\iota(f \otimes_2 g) = f \otimes_1 g$ 视作了相等,我们要证明的是 $\iota$ 是一个单射。

\textbf{证明:}
如果 $f \otimes_1 g = \iota(f \otimes_2 g) = 0$,这意味着对任意 $v \in V, v' \in V'$,$f(v) \otimes g(v') = (f \otimes_1 g)(v \otimes v') = 0$,即 $f(v)$ 或者 $g(v')$ 等于 $0$;假设 $f$ 不恒等于零,即存在 $v_0 \in V$ 使得 $f(v) \neq 0$,此时对任意的 $v' \in V'$ 我们都有 $g(v') = 0$,换言之 $g = 0$,我们得到了 $f$ 或者 $g$ 中至少有一个为零映射,因此 $f \otimes_2 g = 0$。证得 $\iota$ 是单射。

进一步的,假设 $V, V', W, W'$ 都是有限维度向量空间,维度分别为 $n, n', m, m'$,我们有
\begin{equation}
\dim(L(V, W) \otimes L(V', W')) = n n' m m' = \dim(L(V \otimes V', W \otimes W'))~
\end{equation}
根据 Rank–nullity 定理可得 $\iota$ 是一个同构映射。
\addTODO{张量积维度、Rank–nullity 定理链接}

\textbf{证明结束。}

\begin{example}{无限维度的反例}

\end{example}





\subsubsection{用矩阵表示张量积}

\addTODO{Kronecker product}

\subsubsection{张量积与映射的运算}

\addTODO{张量积的结合率与交换率}

\addTODO{直积、复合、}

\subsection{线性映射的对称/反对称幂}

\pentry{向量空间的对称/反对称幂\upref{vecSAS}}