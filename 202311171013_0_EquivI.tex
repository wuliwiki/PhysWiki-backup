% 惯性力:为什么月球不被太阳吸走
% license Usr
% type Tutor

为什么我们在分析月亮和卫星的运动时,可以不考虑太阳的引力? 事实上,太阳对月亮的引力比地球对月亮的引力要大!

你有兴趣的话,可以自行查找太阳、地球、月球的质量,以及他们的平均距离,然后用牛顿的万有引力公式计算:
\begin{equation}
F = \frac{Gm_1m_2}{r^2}~.
\end{equation}

如果你还记得高中的知识,也可以把地球中心作为坐标系的原点,假设太阳不存在,计算验证月球绕地球旋转的周期和轨道。 事实上你会发现这和真实月球真实的运动所差无几。

那么问题来了,为什么在计算月球绕地球旋转时不需要考虑太阳的引力?若硬要加上引力反而会得到 “月亮将被太阳吸走” 这样奇怪的结论。

在回答这个问题以前,我们先提出另一个问题。 这个问题是: 若把地球中心作为坐标系的原点,为什么地球没有被太阳吸走?难道换个参考系地球就不受太阳引力了吗?

这似乎有点循环论证的意思,我们知道运动是相对于观察者所在参考系的。 因为我们定义了地球中心是坐标原点,所以地球在这个坐标系中自然是不动的。 但为什么受了力的物体会没有加速度呢? 难道牛顿第二定律 $F = ma$ 在这里失效了? 如果太阳对地球的引力在这个参考系中对地球不起作用,你为什么会认为它对月球要起作用呢?

事实上许多人在牛顿定律时忽略了一个前提:\textbf{使用的参考系必须是惯性系}!

事实上牛顿第一定律的作用就是定义惯性系: 在某个参考系中,如果任何不受力(或者所有外力抵消)的物体都做匀速直线运动或静止不动,那么它就是一个惯性系。

所以在以上分析中,若要直接使用原汁原味的牛顿三定律和万有引力公式计算地球和月球的运动,必须要以太阳为参考系。当然了,严格来说太阳也在绕银河系做圆周运动,也并不完全是惯性系……但为了不把事情弄得太复杂,我们姑且人为假设太阳参考系是一个惯性系。

难道牛顿定律还有不那么原汁原味的版本?是的,因为我们经常用一些数学小技巧来让牛顿定律在惯性参考系中也变得可以使用。 我们先看一个更简单的例子。

假设地面参考系是惯性系, 地面上有一个以恒定加速度 $a$ 直线行驶的长方体车, 若它里面有一个质量为 $m$ 的小球在车的尾部,那么根据牛顿第二定律, 车的后壁对小球必须施加一个向前的压力 $N = ma$。 但如果我们以车作为参考系, 就会发现虽然小球受的合力 $F = N$ 不为零, 但小球却并没有动! 显然牛顿第二定律在车的参考系中失效了。 那么如何才能拯救 $F = ma$ 呢? 在车中 $a$ 为零, 所以合力 $F$ 也应该为零。 那么我们不妨假设小球额外受到一个假想的力,称为\textbf{惯性力},与 $N$ 相同但方向相反,不就可以把合力凑成 0 了吗? 我们把这个惯性力记为
\begin{equation}
F' = -ma~.
\end{equation}
这个惯性力的假设能彻底在车的参考系中拯救牛顿第二定律吗?我们可以考虑一些其他的情况。
