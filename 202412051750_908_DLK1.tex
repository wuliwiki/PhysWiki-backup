% 保罗·狄拉克(综述)
% license CCBYSA3
% type Wiki

本文根据 CC-BY-SA 协议转载翻译自维基百科\href{https://en.wikipedia.org/wiki/Paul_Dirac}{相关文章}。


\begin{figure}[ht]
\centering
\includegraphics[width=6cm]{./figures/887a811793832fb1.png}
\caption{1933年的狄拉克} \label{fig_DLK1_1}
\end{figure}
保罗·阿德里安·莫里斯·狄拉克(Paul Adrien Maurice Dirac,/dɪˈræk/;1902年8月8日 – 1984年10月20日)是英国的数学物理学家和理论物理学家,被认为是量子力学的奠基人之一。[6][7] 狄拉克为量子电动力学和量子场论的基础奠定了基础。[8][9][10][11] 他曾担任剑桥大学卢卡斯数学教授、佛罗里达州立大学物理学教授,并于1933年获得诺贝尔物理学奖。

狄拉克于1921年毕业于布里斯托大学,获得电气工程学的一级荣誉理学士学位,1923年获得数学的一级荣誉文学学士学位。[12] 随后,他于1926年从剑桥大学获得物理学博士学位,撰写了首篇关于量子力学的论文。[13]

狄拉克对量子力学和量子电动力学的早期发展作出了基础性贡献,并创造了后者的术语。[10] 其中,他在1928年提出了狄拉克方程,该方程描述了费米子的行为,并预测了反物质的存在,[14] 这一方程被认为是物理学中最重要的方程之一,[8] 并被一些物理学家视为“现代物理学的真正种子”。[15] 他在1931年撰写了一篇著名论文,[16] 进一步预测了反物质的存在。[17][18][14] 狄拉克与厄尔温·薛定谔共同分享了1933年诺贝尔物理学奖,以表彰他们“发现了原子理论的新生产性形式”。[19] 他是最年轻的诺贝尔理论物理学奖获得者,直到1957年T·D·李获奖。[20] 狄拉克还为广义相对论与量子力学的和解做出了巨大贡献。他的1930年专著《量子力学原理》是量子力学最具影响力的经典著作之一。[21]

狄拉克的贡献不仅限于量子力学。他还为“管道合金”项目作出了贡献,这是英国在二战期间研究和建造原子弹的计划。[22][23] 狄拉克对铀浓缩过程和气体离心机做出了基础性贡献,[24][25][26][23] 他的工作被认为是“可能是离心机技术中最重要的理论成果”。[27] 他还对宇宙学做出了贡献,提出了“大数假说”。[28][29][30][31] 狄拉克还在弦理论诞生之前预见到了弦理论,提出了如狄拉克膜、狄拉克-伯恩-因费尔德作用等工作,这些贡献对现代弦理论和规范理论至关重要。[32][33][34][35]

狄拉克被朋友和同事视为个性独特。在1926年给保罗·埃伦费斯特的信中,阿尔伯特·爱因斯坦写道:“我在费劲地研究狄拉克。这种在天才与疯狂之间的摇摆实在可怕。”在另一封关于康普顿效应的信中,他写道:“我完全不理解狄拉克的细节。”[36] 1987年,阿卜杜斯·萨拉姆宣称:“狄拉克无疑是这个世纪最伟大的物理学家之一……除爱因斯坦外,没有人能在如此短的时间内,对本世纪物理学的发展产生如此决定性的影响。”[37] 1995年,斯蒂芬·霍金表示:“狄拉克比任何人都做得更多,除了爱因斯坦之外,他推动了物理学的发展并改变了我们对宇宙的认识。”[38] 安东尼诺·齐基奇认为狄拉克对现代物理学的影响超过了爱因斯坦,[15] 斯坦利·德塞尔则评论道:“我们都站在狄拉克的肩膀上。”[39] 狄拉克被广泛认为与艾萨克·牛顿、詹姆斯·克拉克·麦克斯韦和爱因斯坦相提并论。[40][41][42]
\subsection{个人生活}  
\subsubsection{早年}
\begin{figure}[ht]
\centering
\includegraphics[width=6cm]{./figures/3f71a2576aff2af0.png}
\caption{克拉拉·埃瓦尔德(Clara Ewald)所绘的保罗·狄拉克肖像(1939年)} \label{fig_DLK1_2}
\end{figure}
保罗·阿德里安·莫里斯·狄拉克(Paul Adrien Maurice Dirac)于1902年8月8日出生在英国布里斯托尔市父母的家中,并在该市的比肖普斯顿区长大。他的父亲,查尔斯·阿德里安·拉迪斯拉斯·狄拉克(Charles Adrien Ladislas Dirac),是来自瑞士圣莫里茨的移民,具有法国血统,在布里斯托尔担任法语教师。他的母亲,弗洛伦斯·汉娜·狄拉克(Florence Hannah Dirac,娘家姓霍尔滕),出生在康沃尔的利斯基尔,是一个康沃尔卫理公会家庭的成员。她是由其父亲命名的,父亲是一位海军船长,在克里米亚战争期间曾与弗洛伦斯·南丁格尔(Florence Nightingale)相识。她在年轻时搬到布里斯托尔,并在那里担任布里斯托尔中央图书馆的图书管理员;尽管如此,她依然认为自己的身份是康沃尔人,而不是英国人。保罗有一个妹妹,比阿特丽斯·伊莎贝尔·玛格丽特(Béatrice Isabelle Marguerite),昵称贝蒂,还有一个哥哥,雷金纳德·查尔斯·费利克斯(Reginald Charles Félix),昵称费利克斯,后者在1925年3月自杀。狄拉克后来回忆道:“我的父母非常痛苦,我从未知道他们如此在意……我从来不知道父母应该如此关心孩子,但从那时起我知道了。” 

查尔斯和孩子们在1919年10月22日正式成为瑞士国籍,之前他们一直是瑞士公民。狄拉克的父亲为人严格且专制,尽管他不赞成体罚。狄拉克与父亲的关系紧张,甚至在父亲去世后,狄拉克写道:“我现在感到更加自由,我是我自己的人。”查尔斯强迫孩子们只能用法语与他交谈,以便他们能够学会这门语言。当狄拉克发现自己无法用法语表达想法时,他选择保持沉默。