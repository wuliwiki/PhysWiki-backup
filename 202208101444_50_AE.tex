% 自编码器
% keys 自编码器 auto encoder

\textbf{自编码器}(Autoencoder)是一种将输入数据映射为自身的神经网络模型.从输入和输出端来看,自编码器相当于是把输入数据复制到输出.

自编码器一般包含两个部分:一个编码器(Encoder)和一个解码器(Decoder).编码器将输入数据映射到一个内隐空间,数学表达式为$\boldsymbol{h}=f(\boldsymbol{x})$.内隐空间在网络结构中由一个内隐层实现.解码器的作用是数据重构,将内隐空间的特征重新映射回数据,数学表达式为$\boldsymbol{r}=g(\boldsymbol{h})$.整个自编码器,可以表示为:$g(f(\boldsymbol{x}))=\boldsymbol{x}$.

神经网络的自编码器推广了传统编码和解码的思想.传统编码解码是基于确定性函数的,而神经网络中的自编码器则是一种概率性关系.自编码器的目标是学习两个条件概率:编码器$p_{encoder}(h|x)$和解码器$p_{decoder}(x|h)$\cite{GDL}.

\begin{figure}[ht]
\centering
\includegraphics[width=5cm]{./figures/AE_1.png}
\caption{自编码器基本结构示意图} \label{AE_fig1}
\end{figure}


自编码器的类型主要有:欠完备自编码器(Undercomplete Autoencoders)、正则自编码器(Regularized Autoencoders)、稀疏自编码器(Sparse Autoencoders)等.其中,有一类应用较为广泛的自编码器变体,是变分自编码器(Variational Autoencoders).