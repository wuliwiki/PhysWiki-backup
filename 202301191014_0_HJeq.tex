% 哈密顿—雅可比方程
% keys 哈密顿雅可比方程

\begin{issues}
\issueTODO
\end{issues}


\pentry{端点可变的作用量\upref{EPAct}}
对作用量
\begin{equation}
S=\int_{t_1}^{t_2}L\dd t
\end{equation}
由\autoref{EPAct_eq1}~\upref{EPAct},就有
\begin{equation}\label{HJeq_eq1}
\pdv{S}{t^{(2)}}+H\qty(q^{(2)},p^{(2)},t^{(2)})=0
\end{equation}
同样,用\autoref{EPAct_eq1}~\upref{EPAct} 中的
\begin{equation}
p_i^{(2)}=\pdv{S}{{q^i}^{(2)}}
\end{equation}
代入\autoref{HJeq_eq1} ,就得到方程:
\begin{equation}
\pdv{S}{t^{(2)}}+H\qty({q^1}^{(2)},\cdots,{q^n}^{(2)};\pdv{S}{{q^1}^{(2)}},\cdots,\pdv{S}{{q^n}^{(2)}};t^{(2)})=0
\end{equation}

若去掉表示末时刻的上指标 $(2)$,而默认所有变量都对应末时刻的值,上式就写成
\begin{equation}\label{HJeq_eq2}
\pdv{S}{t}+H\qty({q^1},\cdots,{q^n};\pdv{S}{{q^1}},\cdots,\pdv{S}{{q^n}};t)=0
\end{equation}
这个关于 $S$ 的一阶偏微分方程,就称为\textbf{哈密顿-雅可比方程}。
\subsection{利用哈密顿-雅可比方程求解系统的运动}
\pentry{正则变换\upref{CanTra}}
\footnote{参考朗道《力学》}一阶偏微分方程的解(全积分)包含的独立常数之个数与独立变量的数目相同。由\autoref{HJeq_eq2} ,这里 $S$ 的独立变量是 $n$ 个坐标和1个时间,共 $n+1$ 个。而 $S$ 仅以其导数的形式出现在方程中,所以解中的任意常数中有一个是以相加的形式出现的,故哈密顿-雅可比方程解的全积分形式为
\begin{equation}
S=f(t,q^1,\cdots,q^n,\alpha_1,\cdots,\alpha^n)+A
\end{equation}
其中 $\alpha_1,\cdots,\alpha_n,A$ 是任意常数。

在相空间中进行正则变换,并以 $f(t,q,\alpha)$ 为母函数,而 $\alpha_1,\alpha_n$ 为新的动量,新的坐标以 $\beta^1,\cdots,\beta^n$ 表示。由\autoref{CanTra_eq8}~\upref{CanTra}
\begin{equation}
p_i=\pdv{f}{q^i},\quad \beta^i=\pdv{f}{\alpha_i},\quad H'=H+\pdv{f}{t}
\end{equation}
由于 $f$ 满足哈密顿-雅可比方程,所以
\begin{equation}
H'=H+\pdv{f}{t}=H+\pdv{S}{t}=0
\end{equation}
由于正则变换下新变量的运动方程任满足正则形式,而由上式,这时 $H'=0$,所以
\begin{equation}
\dot\alpha_i=0,\quad\dot\beta^i=0
\end{equation}
即 $\alpha_i,\beta^i$ 都为常数。

另外,利用 $n$ 个方程
\begin{equation}
\beta^i=\pdv{f}{\alpha_i}
\end{equation}
可以将 $n$ 个坐标 $q$ 用时间和 $2n$ 个常数 $\alpha,\beta$ 表示出来,这给出了运动方程的通积分。