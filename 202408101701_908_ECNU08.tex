% 华东师范大学 2008 年 考研 量子力学
% license Usr
% type Note

\textbf{声明}:“该内容来源于网络公开资料,不保证真实性,如有侵权请联系管理员”

\subsection{简答题(每题5分,共50分)}
\begin{enumerate}
\item 写出德布罗意(de Broglie)关系,并简述其物理含义,
\item 何为正常塞曼(2ccman)效应?其物理本质是什么?
\item 求对易关系[$x,xp$],其中$x$为位置算特,$p$为其共轭动量,
\item 如果粒子处在其动量的本征态上,对其位置进行观测将获得什么样的观测结果?
\item 可观测力学量对应的算符有什么特点?
\item 假定一体系的哈密顿量$H$不含时,且力学量$A$不显含时。问$A$与$H$满足什么关系时$A$为体系的守恒量?
\item 简述氢原子电子基态波函数的特点?
\item 力学量$A$与$B$有究备共同本征画数的必要条件是什么?
\item 粒子处在态e4叫e(x)(其中@和p为实雨数)上,在x,到x:*dx之间观测到粒子的几率为什么?
\item 举出一个说明光(或电磁辐射)具有粒子性的实验,简单给出理由。
\end{enumerate}
