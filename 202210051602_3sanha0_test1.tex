% 分块矩阵

\pentry{矩阵\upref{Mat}}

把一个矩阵分成若干个子矩阵,称为\textbf{矩阵的分块},将矩阵看作是由子矩阵组成的矩阵,这种矩阵称为\textbf{分块矩阵}.

构造分块矩阵的好处在于,使得矩阵的结构变得更加清楚,而且使得矩阵的运算通过它们的分块矩阵形式来进行,从而可以使有关矩阵的问题变得更加容易解决.
\begin{equation}
\mat M=
\begin{pmatrix}
\mat A & \mat B\\
\mat C & \mat D
\end{pmatrix}
\end{equation}

容易由矩阵的加法运算定义得到分块矩阵的加法运算规律

\begin{equation}
\mat A+\mat B=
\begin{pmatrix}
\mat A_1 & \mat A_2\\
\mat A_3 & \mat A_4
\end{pmatrix}
+
\begin{pmatrix}
\mat B_1 & \mat B_2\\
\mat B_3 & \mat B_4
\end{pmatrix}
=
\begin{pmatrix}
\mat A_1+\mat B_1 & \mat A_2+\mat B_2\\
\mat A_3+\mat B_3 & \mat A_4+\mat B_4
\end{pmatrix}
\end{equation}

由矩阵的数乘定义得到分块矩阵的数乘运算规律

\begin{equation}
k\mat A=
k\begin{pmatrix}
\mat A_1 & \mat A_2\\
\mat A_3 & \mat A_4
\end{pmatrix}
=
\begin{pmatrix}
k\mat A_1 & k\mat A_2\\
k\mat A_3 & k\mat A_4
\end{pmatrix}
\end{equation}
\textsl{其中$\mat A$是定义在数域$K$上的矩阵,$k\in{K}$}

分块矩阵的转置运算为

\begin{equation}
\mat A=
\begin{pmatrix}
\mat A_1 & \mat A_2\\
\mat A_3 & \mat A_4
\end{pmatrix}
,
\mat A'
=
\begin{pmatrix}
\mat A_1' & \mat A_3'\\
\mat A_2' & \mat A_4'
\end{pmatrix}
\end{equation}

\begin{theorem}{分块矩阵的乘法}
在数域$K$上,设矩阵$A\in{M_{s\times{n}}(K)},B\in{M_{n\times{m}}(K)}$.分别对两个矩阵分块,如果满足下面两个条件
\begin{enumerate}
\item $\mat A$分块的列组数等于$\mat B$分块的行组数(块数).
\item $\mat A$的每个列组所含的列数\textbf{对应}等于$\mat B$的每个行组所含的行数.
\end{enumerate}
则分块矩阵的乘法类似地满足矩阵乘法运算规律.
\begin{equation}
\mat A\mat B=
\begin{pmatrix}
\mat A_1 & \mat A_2\\
\mat A_3 & \mat A_4
\end{pmatrix}
\begin{pmatrix}
\mat B_1 & \mat B_2\\
\mat B_3 & \mat B_4
\end{pmatrix}
=
\begin{pmatrix}
\mat A_1\mat B_1+\mat A_2\mat B_3 & \mat A_1\mat B_2+\mat A_2\mat B_4\\
\mat A_3\mat B_1+\mat A_4\mat B_3 & \mat A_3\mat B_2+\mat A_4\mat B_4
\end{pmatrix}
\end{equation}
\end{theorem}
\textsl{和矩阵乘法规则的区别在于矩阵乘法的相对顺序不能改变,因为矩阵乘法没有交换律}

\textbf{证明}:
设$\mat C=\mat A\mat B$,则$C$的行数和列数分别为$s$和$m$.分块矩阵的乘积的行数为$s=s_1+s_2$,$m=m_1+m_2$,和$\mat C$的行数和列数相等.于是得到证明思路:证明任意$\mat C$的$(i,j)$元,就等于分块矩阵的$(i,j)$元.


