% 笛卡尔积
% keys 直积|有序数对|笛卡尔积
% license Usr
% type Tutor

\begin{issues}
\issueDraft
\end{issues}

在刚刚接触到乘法计算时,作为一个运算结果,“积”是作为与“乘法”相关的概念被引入的。后来,随着对向量的学习的深入,内积和外积逐渐也成为了熟悉的概念,二者分别与点乘($\cdot$)和叉乘($\times$)相对应。或许,“卷积”和“张量积”等概念也偶尔会出现在你的视野中。他们往往是与一个逐渐抽象的“乘法”相对应,说他逐渐抽象,是因为他与我们熟知的数的乘法的样子和计算方法相去甚远。而还称呼它是乘法,是因为某种程度上,它保留了乘法的一些特性。

下面会涉及到一点点的物理知识:在物理上,常常会有通过“乘法”来定义一个新的物理量,比如:功是力与位移的乘积,力矩是力与力臂的乘积,电路中功率是电流与电压的乘积(先忽略这个乘积具体的形式)等。这个新的物理量与原有的两个物理量之间都存在关系。而“加法”往往是在一个概念内部量的多少的计算,基本是不涉及其他概念的。

数学上有一个概念叫作\textbf{直积}(direct product),一般使用它来组合两个同类的已知对象,来定义新对象,比如:集合、群、模、拓扑空间等。而作用在两个集合上的直积便称为\textbf{笛卡尔积}(Cartesian product)。下面会先直接给出笛卡尔积的定义,然后就定义中的一些概念进行讲解。



\subsection{有序对}

在刚开始接触集合这个概念的时候,一定会知道,集合有这样一个性质——“集合内的元素的具有无序性”。而现实生活中,“序”这个概念又是必不可少的。有序对(ordered pair)的出现就是为了能够有概念来表示两个元素顺序。因此,定义时就希望这个概念能够满足两个性质:

\begin{enumerate}
\item 唯一性:每一个有序对是唯一定义的,即$(a, b) = (c, d)\implies (a = c) \land (b = d)$。
\item 顺序性:有序对中的元素顺序是固定的,即$(a, b)\neq (b, a)$,除非$a = b$。
\end{enumerate}

这样,“序”的概念就能在表达序的想法(顺序性)的同时,保持稳定(唯一性)。

\begin{definition}{有序对}
有序对有以下几种常见的定义方法:
\begin{itemize}
\item 维纳对(Wiener pair,1914年):$(a, b):= \{\{\emptyset,\{ a\}\}, \{\{b\}\}\} $
\item 豪斯多夫对(Hausdorff Pair,1914年):$ (a, b):= \{\{a, 1\}, \{b, 2\}\} $
\item 库拉托夫斯基对(Kuratoswki pair,1921年)$(a, b) := \{\{a\}, \{a, b\}\}$
\end{itemize}
\end{definition}
其中:
\begin{itemize}
\item 维纳对通常配合类型论使用。
\item 豪斯多夫对由于使用了数字作为序的描述,如果要考虑尚未定义数的场景,或需要研究数的序时,可能会造成循环论证。
\item 库拉托夫斯基对定义较为简洁,目前使用也最为广泛。\textbf{下面的描述中,会采取此定义}。
\end{itemize}

需要注意的是,每个定义在某些具体领域(如研究集合的理论)使用时,都存在各自的局限性。也存在一些新的定义,由于与内容关系不大没有在此给出。

定义能够比较好地描述之前提到的特性。
\begin{example}{唯一性证明}
根据定义有$(a, b) = \{\{a\}, \{a, b\}\} , (c, d) = \{\{c\}, \{c, d\}\} $,若(a, b)=(c,d),则有$\{\{a\}, \{a, b\}\}=\{\{c\}, \{c, d\}\}$。根据集合相等的定义,必有$\{a\}=\{c\}$或$\{a\}=\{c, d\}$。
\end{example}


有序对的概念在关系和函数、拓扑等领域都有应用。


\subsection{笛卡尔积}

\textbf{笛卡尔积}是一个集合领域的概念,经常通过对。

参考文献
Halmos, Paul R. *Naive Set Theory*. Springer, 1974.
Wikipedia: [Ordered Pair](https://en.wikipedia.org/wiki/Ordered_pair)