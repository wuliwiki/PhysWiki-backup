% 普朗克黑体辐射定律(综述)
% license CCBYSA3
% type Wiki

本文根据 CC-BY-SA 协议转载翻译自维基百科\href{https://en.wikipedia.org/wiki/Michael_Faraday}{相关文章}。

\begin{figure}[ht]
\centering
\includegraphics[width=8cm]{./figures/70b29e3c83575e69.png}
\caption{普朗克定律准确描述了黑体辐射。这里展示的是不同温度下的一组曲线。经典(黑色)曲线在高频率(短波长)下与观测到的强度偏离。} \label{fig_HTFS_1}
\end{figure}
在物理学中,普朗克定律(也称为普朗克辐射定律)描述了在给定温度T下,黑体在热平衡状态下发射的电磁辐射的谱密度,当黑体与其环境之间没有物质或能量的净流动时。

在19世纪末,物理学家无法解释为什么已经准确测量的黑体辐射谱在高频率处与现有理论预测的谱有显著的偏离。1900年,德国物理学家马克斯·普朗克通过启发式推导得出了一个公式,解释了观测到的谱,假设在含有黑体辐射的腔体中,假设的电荷振荡器只能以最小增量E改变其能量,该增量与其相关电磁波的频率成正比。虽然普朗克最初认为将能量分成增量的假设只是为了得到正确答案的数学技巧,但其他物理学家,包括阿尔伯特·爱因斯坦,在他的基础上进行了进一步发展,普朗克的洞察力现在被认为对量子理论具有根本性的重要性。
\subsection{定律}
每个物体都会自发且持续地发射电磁辐射,物体的谱辐射强度 \( B_\nu \) 描述了特定辐射频率下,每单位面积、每单位立体角和每单位频率的谱发射功率。普朗克辐射定律给出的关系表明,随着温度的升高,物体辐射的总能量增加,且辐射谱的峰值会向短波长方向移动。根据普朗克分布定律,在给定温度下,谱能量密度(单位体积每单位频率的能量)由下式给出:
\[
u_\nu(\nu, T) = \frac{8 \pi h \nu^3}{c^3} \cdot \frac{1}{\exp \left( \frac{h \nu}{k_{\mathrm{B}} T} \right) - 1}~
\]
或者,该定律可以表示为物体在绝对温度 \( T \) 下,频率为 \( \nu \) 的谱辐射强度:
\[
B_\nu(\nu, T) = \frac{2 h \nu^3}{c^2} \cdot \frac{1}{\exp \left( \frac{h \nu}{k_{\mathrm{B}} T} \right) - 1}~
\]
其中 \( k_{\mathrm{B}} \) 是玻尔兹曼常数,\( h \) 是普朗克常数,\( c \) 是介质中的光速,无论是物质还是真空。谱辐射强度 \( B_\nu \) 的 cgs 单位是 \( \text{erg} \cdot \text{s}^{-1} \cdot \text{sr}^{-1} \cdot \text{cm}^{-2} \cdot \text{Hz}^{-1} \)。术语 \( B \) 和 \( u \) 通过因子 \( \frac{4 \pi}{c} \) 相关,因为 \( B \) 与方向无关且辐射以光速 \( c \) 传播。谱辐射强度也可以按单位波长 \( \lambda \) 表示,而不是按单位频率。此外,该定律还可以用其他术语表示,例如在某个波长下发射的光子数,或辐射体积内的能量密度。

在低频极限(即长波长)下,普朗克定律趋近于雷leigh–Jeans定律,而在高频极限(即短波长)下,它趋近于维恩近似。

马克斯·普朗克于1900年提出了这个定律,定律中只有经验确定的常数,后来他证明,将其表示为能量分布时,它是热力学平衡辐射的唯一稳定分布。作为能量分布,它是热平衡分布族中的一个成员,包括玻色–爱因斯坦分布、费米–狄拉克分布和麦克斯韦–玻尔兹曼分布。
\subsection{黑体辐射}
\begin{figure}[ht]
\centering
\includegraphics[width=8cm]{./figures/21e7152f5833fb75.png}
\caption{太阳近似为一个黑体辐射源。它的有效温度约为5777 K。} \label{fig_HTFS_2}
\end{figure}
黑体是一个理想化的物体,能够吸收和发射所有频率的辐射。在接近热力学平衡的状态下,发射的辐射可以通过普朗克定律精确描述。由于其依赖于温度,普朗克辐射被称为热辐射,这意味着物体的温度越高,它在每个波长上发射的辐射越多。

普朗克辐射在一个与物体温度相关的波长处具有最大强度。例如,在室温下(约300 K),物体发出的热辐射主要是红外线,并且是不可见的。在较高的温度下,红外辐射的数量增加并可以感受到热量,同时发射更多的可见辐射,物体呈现可见的红色光。在更高的温度下,物体变为明亮的黄色或蓝白色,并发射大量短波长的辐射,包括紫外线甚至X射线。太阳的表面(约6000 K)发射大量的红外线和紫外线辐射,其发射在可见光谱中达到峰值。由于温度变化引起的这种变化被称为维恩位移定律。

普朗克辐射是任何处于热平衡状态的物体从其表面发射的最大辐射量,无论其化学成分或表面结构如何。辐射穿过介质之间的界面时,可以通过界面的发射率来表征(实际辐射强度与理论普朗克辐射强度的比值),通常用符号 \( \epsilon \) 表示。发射率通常依赖于化学成分、物理结构、温度、波长、传输角度和偏振状态。自然界面上,发射率总是在 \( \epsilon = 0 \) 和 \( \epsilon = 1 \) 之间。

一个与另一个介质相接触的物体,如果该介质的发射率 \( \epsilon = 1 \) 且能够吸收所有射入的辐射,则被称为黑体。黑体的表面可以通过在一个大密封容器的墙壁上开一个小孔来建模,这个容器在均匀温度下保持不透明的墙壁,且在每个波长上都不是完全反射的。在平衡状态下,这个容器内部的辐射由普朗克定律描述,离开小孔的辐射也由此定律描述。

就像麦克斯韦–玻尔兹曼分布是物质粒子气体在热平衡状态下的唯一最大熵能量分布一样,普朗克分布也是光子气体的最大熵分布。与物质气体不同,物质气体的质量和粒子数起着重要作用,而光子气体在热平衡状态下的谱辐射强度、压强和能量密度完全由温度决定。

如果光子气体不是普朗克分布的,热力学第二定律保证了交互作用(光子与其他粒子之间的相互作用,甚至在足够高的温度下,光子之间的相互作用)会导致光子能量分布的变化,并最终趋近于普朗克分布。在热力学平衡的过程中,光子会以正确的数量和能量被创造或湮灭,直到它们填充整个腔体,并且达到普朗克分布,直到达到平衡温度。就像气体是由多个子气体组成的,每个子气体对应一个波长范围,每个子气体最终都会达到共同的温度。

量 \( B_\nu(\nu, T) \) 是谱辐射强度,作为温度和频率的函数。它在国际单位制(SI)中的单位为 \( \text{W} \cdot \text{m}^{-2} \cdot \text{sr}^{-1} \cdot \text{Hz}^{-1} \)。一个无穷小的功率 \( B_\nu(\nu, T) \cos \theta \, dA \, d\Omega \, d\nu \) 被辐射到由角度 \( \theta \) 描述的方向,从表面法线处的无穷小面积 \( dA \) 向无穷小的立体角 \( d\Omega \) 以及无穷小频率带宽 \( d\nu \) 辐射,其中心频率为 \( \nu \)。辐射到任何立体角的总功率是 \( B_\nu(\nu, T) \) 对这三个量的积分,且由斯特凡–玻尔兹曼定律给出。普朗克辐射的谱辐射强度对于每个方向和偏振角度都是相同的,因此黑体被称为兰伯特辐射源。
\subsection{不同形式}
普朗克定律可以根据不同科学领域的惯例和偏好,以多种形式出现。下表总结了谱辐射强度定律的不同形式。左侧的形式通常在实验领域中遇到,而右侧的形式则通常在理论领域中遇到。
\begin{figure}[ht]
\centering
\includegraphics[width=14.25cm]{./figures/a6e4d32ae252241b.png}
\caption{} \label{fig_HTFS_3}
\end{figure}