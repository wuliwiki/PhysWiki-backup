% 张量(向量与矩阵)
% keys 张量|矩阵

\begin{issues}
\issueDraft
\end{issues}


\pentry{矩阵\upref{Mat}}

本文的难度分级为:科普.

% 在多元微积分中引入矩阵是为了将一元微积分中导数值的概念.具体而已,对于函数 $f: \mathbb{R} \to \mathbb{R}$,在某个点处(比如 $0$ 点)可以找到一条切线,我们可以用一个实数来表示切线的斜率;类似的,当我们考虑函数 $f: \mathbb{R}^n \to \mathbb{R}^m$ 的

直观上说,矩阵就是把数字排列成长方形.从这个角度来说,矩阵是一个纯粹的“组合数学”概念:不需要学习集合论、微积分、线性代数,只需要理解了数(整数,有理数,实数或者复数都可)的概念和上面的加、减、乘(可以没有除法)我们就可以理解矩阵.
\addTODO{什么是《组合数学》,可能需要一个词条来解释数学的子学科}
\addTODO{可以考虑移动到《矩阵》词条}

不过,长方形只是一个很简单的二维图形,抛开纸张(二维)和方(形状)的限制我们可以得到很多不同的“数字结构”,其中张量就是一种(不那么简单的)对方矩阵的推广.

在线性代数中,我们会学到用矩阵来理解(选定基之后的)线性算子\upref{LinMap},用张量来理解多线性变换\upref{MulMap}.

\subsection{列矩阵和行矩阵}
在这个词条中,我们把一个 $n$ 行 $n$ 列的矩阵记作\textbf{方矩阵}(省略掉 $n$), $1 \times n$ 矩阵记作\textbf{列矩阵}(列向量),$n \times 1$ 矩阵记作\textbf{行矩阵}(行向量),$1 \times 1$ 矩阵记作\textbf{数矩阵}(标量).根据矩阵的乘法\autoref{Mat_sub1}~\upref{Mat},我们有以下四种合法的乘法(注意顺序不能颠倒),

其中三种“简化”的乘法;
\begin{itemize}
\item 行矩阵 $\times$ 列矩阵 $\mapsto$ 数矩阵}
$$
\begin{pmatrix}
\square & \square & \square
\end{pmatrix}
\begin{pmatrix}
\square \\
\square \\
\square
\end{pmatrix}
=
\begin{pmatrix}
\square
\end{pmatrix}
$$
\item 行矩阵 $\times$ 方矩阵 $\mapsto$ 行矩阵
$$
\begin{pmatrix}
\square & \square & \square
\end{pmatrix}
\begin{pmatrix}
\square & \square & \square \\
\square & \square & \square \\
\square & \square & \square
\end{pmatrix}
=
\begin{pmatrix}
\square & \square & \square
\end{pmatrix}
$$
\item 方矩阵 $\times$ 列矩阵 $\mapsto$ 列矩阵
$$
\begin{pmatrix}
\square & \square & \square \\
\square & \square & \square \\
\square & \square & \square
\end{pmatrix}
\begin{pmatrix}
\square \\
\square \\
\square
\end{pmatrix}
=
\begin{pmatrix}
\square \\
\square \\
\square
\end{pmatrix}
$$
\end{itemize}

剩下一种“复杂化”的乘法,
\begin{itemize}
\item 列矩阵 $\times$ 行矩阵 $\mapsto$ 方矩阵
$$
\begin{pmatrix}
\square \\
\square \\
\square
\end{pmatrix}
\begin{pmatrix}
\square & \square & \square
\end{pmatrix}
=
\begin{pmatrix}
\square & \square & \square \\
\square & \square & \square \\
\square & \square & \square
\end{pmatrix}
$$
\end{itemize}


在研究张量的时候(本词条)只考虑“简化”的三种乘法,这之中我们发现,
\begin{itemize}
\item 行矩阵永远乘在左边
\item 列矩阵永远乘在右边
\end{itemize}
因此,对于一个矩阵(方矩阵或者行矩阵),我们可以把\textbf{作用一个列矩阵}(或者\textbf{输入一个列矩阵})定义为在右边乘上一个列矩阵;类似的对于一个矩阵(方矩阵或者列矩阵),我们可以把\textbf{作用一个行矩阵}(或者\textbf{输入一个行矩阵})定义为在左边乘上一个行矩阵.

就目前来说,上述行为显得很没有意义,因为总共只有三种情况,而我却用了 $2 \times 2 = 4$ 种情况来表述.但是,我们可以把这套语言进行推广,以此来得到张量.但在此之前我需要严格定义一下上述“简化”和“复杂化”的概念.

\subsection{矩阵的类型}
我们现在有四种矩阵,方矩阵、列矩阵与行矩阵、数矩阵.我们把数矩阵作为终止状态,那么如果想通过“作用列/行矩阵”来得到终止状态,我们有
\begin{itemize}
\item 数矩阵
\item 行矩阵\textbf{作用列矩阵}得到数矩阵
\item 列矩阵\textbf{作用行矩阵}得到数矩阵
\item 方矩阵\textbf{作用列矩阵}/\textbf{作用行矩阵}得到数矩阵
\end{itemize}
需要注意的是,对于方矩阵来说,作用列矩阵和作用行矩阵顺序不重要,因为一个是左乘一个是右乘.

我们把\textbf{不需要任何作用}的数矩阵称为 $(0,0)$ 型,把\textbf{需要一次列向量作用}的行矩阵称为 $(0,1)$ 型,把\textbf{需要一次行向量作用}的列矩阵称为 $(1,0)$ 型,把\textbf{需要一次列向量作用和一次行向量作用}方列矩阵称为 $(1,1)$ 型.

注:这里我不称呼它们为 $(s, t)$ 型矩阵是因为这很容易和 $s \times t$ 型矩阵搞混;如果读者自认不会混淆,可以称呼它们为 $(s, t)$ 型矩阵.之后我会称呼它们为$(s, t)$ 型张量.

很自然的我们就会考虑, $s$ 和 $t$ 如果不是 $0$ 或 $1$,我们能把 $(s, t)$ 型张量定义成\textbf{需要 $t$ 次列向量作用和 $s$ 次行向量作用}的\textbf{某种东西}吗?

答案是可以的,这就是张量的定义.

\subsection{张量}

\addTODO{未完待续:1. 矩阵的爱因斯坦记号,2. 张量的爱因斯坦记号,3. 爱因斯坦求和约定}
