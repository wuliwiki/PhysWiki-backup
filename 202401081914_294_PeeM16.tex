% 2016 年考研数学试题(数学一)
% keys 考研|数学
% license Copy
% type Tutor

\subsection{选择题}

\begin{enumerate}
\item 若反常积分 $\int_{0}^{+\infty}$ 收敛,则$(\quad)$\\
(A) $a<1$ 且 $b>1 \qquad$
(B) $a>1$ 且 $b<1$\\
(C) $a<1$ 且 $a+b>1 \quad$
(D) $a>1$ 且 $a+b>1$
\item 已知函数 $f(x)=\leftgroup{&2(x-1), &x<1\\ &\ln x, &x \ge 1}$ ,则 $f(x)$ 的一个原函数是  $(\quad)$\\
(A) $F(x)=\leftgroup{&(x-1)^2, &x<1\\&x(\ln x-1),&x \ge 1}$\\
(B) $F(x)=\leftgroup{&(x-1)^2, &x<1\\&x(\ln x+1)-1,&x \ge 1}$\\
(C)$F(x)=\leftgroup{&(x-1)^2, &x<1\\&x(\ln x+1)+1,&x \ge 1}$\\
(D) $F(x)=\leftgroup{&(x-1)^2, &x<1\\&x(\ln x-1)+1,&x \ge 1}$\\
\item 若  $y=(1+x^2)^2-\sqrt{1+x^2},y=(1+x^2)^2+\sqrt{1+x^2}$  是微分方程 $y'+p(x)y=q(x)$  的两个解,则 $q(x)= (\quad)$\\
(A) $3x(1+x^2) \quad$
(B) $-3x(1+x^2) \quad$
(C) $\displaystyle \frac{x}{1+x^2} \quad$
(D) $\displaystyle -\frac{x}{1+x^2} \quad$
\item 已知函数 $f(x)=\leftgroup{&x, &&x \le 0,\\&\frac{1}{n},&&\frac{1}{n+1}<x \le \frac{1}{n},n=1,2,\dots}$  则 $(\quad)$\\
(A) $x=0$ 是 $f(x)$  的第一类间断点\\
(B) $x=0$ 是 $f(x)$  的第二类间断点\\
(C) $f(x)$ 在 $x=0$ 处连续但不可导\\
(D) $f(x)$ 在 $x=0$ 处可导
\item  设 $\mat A,\mat B$  是可逆矩阵,且 $\mat A$ 与 $\mat B$相似,则下列结论错误的是 $(\quad)$ \\
(A)$\mat A \Tr$ 与 $\mat B \Tr$ 相似 \\
(B)$\mat A^{-1}$ 与 $\mat B^{-1}$ 相似\\
(C)$\mat A+\mat A \Tr$ 与 $\mat B+\mat B \Tr$ 相似\\
(D)$\mat A +\mat A^{-1}$ 与 $\mat B +\mat B^{-1}$ 相似
\item 设二次型  $f(x_1,x_2,x_3)=x_1^2+x_2^2+x_3^2+4x_1x_2+4x_1x_3+4x_2x_3$  ,则  $f(x_1,x_2,x_3)=2$  在空间直角坐标下表示的二次曲面为$(\quad)$\\
(A)单叶双曲面 $\qquad$
(B)双叶双曲面$\qquad$
(C)椭球面$\qquad$
(D) 柱面

\item 设随机变量 $X$~$N(\mu,\sigma^2)(\sigma>0)$  ,记  $p=P\{X \le \mu +\sigma ^2\}$ ,则 $(\quad)$\\
(A)$p$ 随着 $\mu$ 的增加而增加  $\quad$
(B)$p$ 随着 $\sigma$ 的增加而增加\\
(C)$p$ 随着 $\mu$ 的增加而减少 $\quad$
(D)$p$ 随着 $\sigma$ 的增加而减少
\item 随机试验 $E$ 有三种两两不相容的结果 $A_1,A_2,A_3$,且三种结果发生的概率均为$\frac1 3$,将试验 $E$ 独立重复做2次,X表示2次试验中结果 $A_1$ 发生的次数,Y表示2次试验中结果 $A_2$ 发生的次数,则 $(\quad)$\\
(A)$\frac1 3$ $\qquad$
(B)$\qquad$
(C)$\qquad$
(D) 

\end{enumerate}
