% 多项式的结式与判别式
% 多项式|代数|判别式

\pentry{一元多项式\upref{OnePol}}

结式和判别式都是由多项式的系数组合成的表达式,刻画了多项式的根的性质。其中结式可用于判断两个多项式有无公共根,而判别式可用于判断一个多项式有无重根。

介绍概念时,我们会先给出定义,然后再给例子来加深理解。

\subsection{结式}
\pentry{行列式的性质\upref{DetPro}}

\begin{definition}{结式}
设$f(x)=a(x-\alpha_1)(x-\alpha_2)\cdots(x-\alpha_m)$和$g(x)=b(x-\beta_1)(x-\beta_2)\cdots(x-\beta_n)$,且两多项式的系数都取自域$\mathbb{F}$\footnote{这里用“\textbf{域}\upref{field}”这个术语是为了尽可能全面,避免重复词条。在初等数学中,我们常取$\mathbb{F}$为\textbf{有理数域}$\mathbb{Q}$或\textbf{实数域}$\mathbb{R}$,也就是说多项式的系数取自有理数域和实数域。具体情况看具体情况下的声明。}。

则记
\begin{equation}\label{RDPly_eq1}
\begin{aligned}
\opn{Res}(f, g)&=a^nb^m(\alpha_1-\beta_1)(\alpha_2-\beta_1)\cdots(\alpha_m-\beta_n)\\
&=a^n\prod_{i} g(\alpha_i)\\
&=(-1)^{mn}b^m\prod_{i} f(\beta_i)~.
\end{aligned}
\end{equation}
称之为$f$和$g$的\textbf{结式(resultant)}

\end{definition}

结式有以下两个重要性质:

\begin{theorem}{}
设$f, g$是系数取自域$\mathbb{F}$的多项式。则$\opn{Res}(f, g)=0$当且仅当$f, g$(在$\mathbb{F}$的代数闭包中)有共同根。
\end{theorem}

\textbf{证明}:

由根的定义和\autoref{RDPly_eq1} ,显然$\opn{Res}(f, g)=0$当且仅当存在$\alpha_i=\beta_j$的情况。

\textbf{证毕}。


另一个性质式,能用多项式的系数来表示结式,于是,我们能从系数轻易判断两多项式有无公共根。



\begin{theorem}{}\label{RDPly_the2}
设$f(x)=a_mx^m+\cdots+a_0$,$g(x)=b_nx^n+\cdots+b_0$,其中各系数$a_i, b_j$都取自域$\mathbb{F}$,且$a_mb_m\neq 0$。则
\begin{equation}\label{RDPly_eq2}
\opn{Res}(f, g)=
\begin{vmatrix}
a_m&\cdots&\cdots&a_0\\
&\ddots& & \ddots\\
&&a_m\cdots&\cdots&a_0\\
b_n&\cdots&\cdots&b_0\\
&\ddots& & \ddots\\
&&b_n\cdots&\cdots&b_0\\
\end{vmatrix}
\end{equation}
其中\autoref{RDPly_eq2} 的前$n$行都是$a_i$,后$m$行都是$b_j$。

举例而言,$ax^2+bx+c$和$dx+e$的结式是
\begin{equation}
\vmat{
    a&b&c\\
    d&e&0\\
    0&d&e
}
=
ae^2+cd^2-bde
\end{equation}

\end{theorem}

证明过于长,此处暂略。



\subsection{判别式}

\begin{definition}{判别式}\label{RDPly_def1}
设$f(x)=a(x-\alpha_1)(x-\alpha_2)\cdots(x-\alpha_n)$的系数都在域$\mathbb{F}$中,且有根$\alpha_1, \cdots, \alpha_n$(重根也按重数分别计入)。

记
\begin{equation}\label{RDPly_eq3}
\opn{Dis}(f) = a^{2(n-1)}\prod_{1\leq i<j\leq n}(\alpha_i-\alpha_j)^2
\end{equation}
称之为$f$的\textbf{判别式(discriminant)}。
\end{definition}

\begin{example}{}
二次多项式$ax^2+b^x+c$又可以写为
\begin{equation}
(x-\frac{-b+\sqrt{b^2-4ac}}{2a})(x-\frac{-b-\sqrt{b^2-4ac}}{2a})
\end{equation}
因此其判别式为
\begin{equation}
\begin{aligned}
(\frac{-b+\sqrt{b^2-4ac}}{2a}-\frac{-b-\sqrt{b^2-4ac}}{2a})^2 &= (\frac{\sqrt{b^2-4ac}}{a})^2\\
&=\frac{b^2-4ac}{a^2}
\end{aligned}
\end{equation}
看起来,这个和中学教的$b^2-4ac$相差一个因子$a^2$。不过对于分辨是否有重根,这个因子没有影响,因为我们只看判别式是不是零。详见\autoref{RDPly_the1} 。
\end{example}


\begin{theorem}{}\label{RDPly_the1}
设$f(x)$的系数都在域$\mathbb{F}$中,且次数大于等于1。则$f$\textbf{无重根}的充要条件是$\opn{Dis}f\neq 0$。
\end{theorem}

\textbf{证明}:

充分性:

设$f(x)=a(x-\alpha_1)(x-\alpha_2)\cdots(x-\alpha_n)$有重根,即存在$i\neq j$使得$\alpha_i=\alpha_j$。于是根据\autoref{RDPly_eq3} ,$\opn{Dis}f = 0$。因此$\opn{Dis}f\neq 0$时$f$不能有重根。

必要性:

同样由\autoref{RDPly_eq3} ,当各$\alpha_i$互不相等时,$\opn{Dis}f\neq 0$。

\textbf{证毕}。


将\autoref{RDPly_the2} 和\autoref{RDPly_the1} 相结合,我们还能得到判别式的行列式表达法:

\begin{theorem}{}
设$f(x)=a_nx^n+\cdots+a_n$的系数$a_i$都在域$\mathbb{F}$中,且其次数大于等于2,则
\begin{equation}
\opn{Dis}(f) = (-1)^{n(n-1)/2}\frac{1}{a_n}
\begin{vmatrix}
a_n&\cdots&\cdots&a_0\\
&\ddots&&&\ddots\\
&&a_n&\cdots&\cdots&a_0\\
na_n&\cdots&\cdots&a_0\\
&\ddots&&&\ddots\\
&&na_n&\cdots&\cdots&a_0
\end{vmatrix}
\end{equation}
\end{theorem}

对于$f$有无重根的深入讨论,参见域论中的\textbf{可分扩张}\upref{SprbEx}词条。









