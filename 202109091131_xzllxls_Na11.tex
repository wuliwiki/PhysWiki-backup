% 2011年计算机学科专业基础综合全国联考卷
% 2011 | 计算机学科专业基础综合 | 全国联考卷

\subsection{一、单项选择题}
第1-40小题,每小题2分,共80分,下列每小题给出的四个选项中,只有一项符合题目要求的.请在答题卡上将所选项的字母涂黑.

1. 设$n$是描述问题规模的非负整数,下面程序片段的时间复杂度是  \\
$\qquad$ x=2;  \\
$\qquad$ while(x<n/2)  \\
$\quad$ $\qquad$ x=2*x;  \\
A. $O(log_2n)$ $\quad$ B. $O(n)$ $\quad$ C. $O(nlog_2n)$ $\quad$ D. $O(n^2)$

2. 元素$a$,$b$,$c$,$d$,$e$依次进入初始为空的栈中,若元素进栈后可停留、可出栈,直到所有元素都出栈,则在所有可能的出栈序列中,以元素$d$开头的序列个数是  \\
A. 3  $\quad$  B. 4  $\quad$  C. 5  $\quad$  D. 6

3. 已知循环队列存储在一维数组$A[0...n-1]$中,且队列非空时front和rear分别指向队头元素和队尾元素.若初始时队列为空,且要求第1个进入队列的元素存储在$A[0]$处,则初始时front和rear的值分别是 \\
A. $0$,$0$  $\quad$ B. $0$,$n-1$ $\quad$ C. $n-1$,$0$ $\quad$ D. $n-1$,$n-1$ 

4. 若一棵完全二叉树有768个结点,则该二叉树中叶结点的个数是 \\
A. 257 $\quad$ B. 258 $\quad$ C. 384 $\quad$ D. 385

5. 若一棵二叉树的前序遍历序列和后序遍历序列分别为1,2,3,4和4,3,2,1,则该二叉树的中序遍历序列不会是 \\
A.1,2,3,4 $\quad$ B.2,3,4,1 $\quad$ C.3,2,4,1 $\quad$ D.4,3,2,1

6. 已知一棵有2011个结点的树,其叶结点个数为116,该树对应的二叉树中无右孩子的结点个数是 \\
A.115 $\quad$ B.116 $\quad$ C.1895 $\quad$ D.1896

7. 对于下列关键字序列,不可能构成某二叉排序树中一条查找路径的序列是 \\
A.95,22,91,24,94,71 $\quad$ B.92,20,91,34,88,35 \\
C.21,89,77,29,36,38 $\quad$ D.12,25,71,68,33,34

8. 下列关于图的叙述中,正确的是 \\
Ⅰ. 回路是简单路径 \\
Ⅱ.存储稀疏图,用邻接矩阵比邻接表更省空间 \\
Ⅲ.若有向图中存在拓扑序列,则该图不存在回路 \\
A.仅Ⅱ $\quad$ B.仅Ⅰ、Ⅱ $\quad$  C.仅Ⅲ $\quad$ D.仅Ⅰ、Ⅲ

9. 为提高散列(Hash)表的查找效率,可以采取的正确措施是 \\
Ⅰ. 增大装填(载)因子 \\
Ⅱ.设计冲突(碰撞)少的散列函数 \\
Ⅲ.处理冲突(碰撞)时避免产生聚集(堆积)现象 \\
A.仅Ⅰ $\quad$ B.仅Ⅱ $\quad$ C.仅Ⅰ、Ⅱ $\quad$ D.仅Ⅱ、Ⅲ

10.为实现快速排序算法,待排序序列宜采用的存储方式是 \\
A.顺序存储 $\quad$ B.散列存储 $\quad$ C.链式存储 $\quad$ D.索引存储

11. 已知序列25,13,10,12,9是大根堆,在序列尾部插入新元素18,将其再调整为大根堆,调整过程中元素之间进行的比较次数是 \\
A.1 $\quad$  B.2 $\quad$ C.4 $\quad$ D.5

12.下列选项中,描述浮点数操作速度指标的是 \\
A.MIPS $\quad$ B.CPI $\quad$ C.IPC $\quad$ D.MFLOPS

13.float型数据通常用IEEE 754单精度浮点数格式表示.若编译器将float型变量$x$分配在一个$32$位浮点寄存器FR1中,且$x$=$-8.25$,则FR1的内容是 \\
A.C104 0000H $\quad$ B.C242 0000H $\quad$ C.C184 0000H $\quad$ D.C1C2 0000H

14.下列各类存储器中,不采用随机存取方式的是 \\
A.EPROM $\quad$ B.CDROM $\quad$ C.DRAM $\quad$ D.SRAM

15.某计算机存储器按字节编址,主存地址空间大小为64MB,现用4M×8位的RAM芯片组成32MB的主存储器,则存储器地址寄存器MAR的位数至少是 \\
A.22位 $\quad$ B.23位 $\quad$ C.25位 $\quad$ D.26位

16.偏移寻址通过将某个寄存器内容与一个形式地址相加而生成有效地址.下列寻址方式中,\textbf{不}属于偏移寻址方式的是 \\
A.间接寻址 $\quad$ B.基址寻址 $\quad$ C.相对寻址 $\quad$ D.变址寻址

17.某机器有一个标志寄存器,其中有进位/借位标志CF、零标志ZF、符号标志SF和溢出标志OF,条件转移指令bgt(无符号整数比较大于时转移)的转移条件是 \\
A.$CF+OF=1$ $\quad$ B.$SF+ZF=1$ $\quad$ C.$CF+ZF=1$ $\quad$ D.CFSF 1






\subsection{二、综合应用题}
第41~47小题,共70分.请将答案写在答题纸指定位置上.

41.(8分)已知有$6$个顶点(顶点编号为为$0$~$5$)的有向带权图$G$,其邻接矩阵$A$为上三角矩阵,按行为主序(行优先)保存在如下的一维数组中.
\begin{table}[ht]
\centering
\caption{第41题图}\label{Na11_tab1}
\begin{tabular}{|c|c|c|c|c|c|c|c|c|c|c|c|c|c|c|}
\hline
$4$ & $6$ & $\infty$ & $\infty$ & $\infty$ & $5$ & $\infty$ & $\infty$ & $\infty$ & $4$ & $3$ & $\infty$ & $\infty$ & $3$ & $3$ \\
\hline
\end{tabular}
\end{table}
要求:  \\
(1)写出图$G$的邻接矩阵$A$.  \\
(2)画出有向带权图$G$.  \\
(3)求图$G$的关键路径,并计算该关键路径的长度.




\subsection{参考答案}
\subsection{一、单项选择题}
1. 解答:A.程序中,执行频率最高的语句为“$x=2*x$”.设该语句执行了$t$次,则$2t+1=n/2$, 故$t=log2(n/2)-1=log2n-2= O(log2n)$.

2. 解答:B.出栈顺序必为d_c_b_a_,e的顺序不定,在任意一个“_”上都有可能.

3. 解答:B.插入元素时,front不变,rear+1.而插入第一个元素之后,队尾要指向尾元素,显然,rear初始应该为$n-1$,front为$0$.

4. 解答:C.叶结点数为$n$,则度为2的结点数为$n-1$,度为1的结点数为0或1,本题中为1(总结点数为偶数),故而即$2n=768$.

5. 解答:C.由前序和后序遍历序列可知3为根结点,故(1,2)为左子树,(4)为右子树,C不可能.或画图即可得出结果.

6. 解答:D.本题可采用特殊情况法解.设题意中的树是如下图所示的结构,则对应的二叉树中仅有前115个叶结点有右孩子.\\
(还有一个插图需要添加)

7. 解答:A.选项A中,当查到91后再向24查找,说明这一条路径之后查找的数都要比91小,后面的94就错了.

8. 解答:C.Ⅰ.回路对应于路径,简单回路对应于简单路径;Ⅱ.刚好相反;Ⅲ.拓扑有序的必要条件.故选C.

9. 解答:B.III错在“避免”二字.

10. 解答:A.内部排序采用顺序存储结构.

11. 解答:B.首先与10比较,交换位置,再与25比较,不交换位置.比较了二次.

12. 解答:D.送分题.

13. 解答:A.$x$的二进制表示为$-1000.01$﹦$-1.000 01$×$2^{11}$ 根据IEEE754标准隐藏最高位的“$1$”,又$E-127=3$,所以$E=130=1000 0010_{(2)}$数据存储为$1$位数符+$8$位阶码(含阶符)+$23$位尾数.
故FR1内容为1 10000 0010 0000 10000 0000 0000 0000 000,即1100 0001 0000 0100 0000 0000 0000 0000,即C104000H.

14. 解答:B.光盘采用顺序存取方式.

15. 解答:D.64MB的主存地址空间,故而MAR的寻址范围是64M,故而是26位.而实际的主存的空间不能代表MAR的位数.

16. 解答:A.间接寻址不需要寄存器,EA=(A).基址寻址:EA=A+基址寄存器内同;相对寻址:EA﹦A+PC内容;变址寻址:EA﹦A+变址寄存器内容.

\subsection{二、综合应用题}