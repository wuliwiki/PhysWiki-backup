% 小时百科参考项目
% license Xiao
% type Tutor

以下列出一些可供小时百科参考的项目。

\subsection{内容}
\subsubsection{综合}
\begin{itemize}
\item \href{https://en.wikipedia.org/wiki/Main_Page}{Wikipedia} 维基百科, 主要是英文版的内容。
\item \href{https://ncatlab.org/nlab/show/HomePage}{nLab} 专业高深的数学、物理和哲学百科, 面向科研, 强调从不同视角而不是中立视角看问题。
\item \href{https://openstax.org}{OpenStax} CC-BY 协议, 没有 SA。
\item \href{https://www.termonline.cn/index}{术语在线} 对照中英文术语
\item \href{https://open.163.com}{网易公开课}、\href{https://ke.qq.com}{腾讯课堂}、\href{https://edu.huaweicloud.com}{华为学院}
\item \href{https://www.edx.org}{edx}、\href{https://www.udacity.com}{udecity}、\href{https://www.khanacademy.org}{khan}、 \href{https://www.youtube.com/@3blue1brown}{3b1b}、 \href{https://www.coursera.org}{coursera}
\item \href{https://www.icourse163.org}{中国大学 Mooc}
\item \href{http://www.scholarpedia.org/
\end{itemize}

\subsubsection{数学}
\begin{itemize}
\item \href{https://math.fandom.com/}{Math Wiki} Fandom 上面一个偏教材的数学百科, 有上千个页面, CC BY-SA 协议。
\item \href{https://www.matongxue.com}{马同学} 高数、线代、概率统计、机器学习等付费教程 % 已购买账号
\item \href{https://www.shuxuele.com}{数学乐}(由\href{https://www.mathsisfun.com}{英文版}翻译): 高中及以下的数学和物理内容, 图文并茂。
\item \href{https://www.bananaspace.org/wiki}{香蕉空间} 中文数学社区(百科、讲义、讨论室) CC BY-SA 4.0
\item \href{https://brilliant.org}{Brilliant} 含有数学分析等内容的百科, 配有例题习题。 还有很多互动插件。
\item \href{https://dlmf.nist.gov}{NIST Digital Library of Mathematical Functions} 函数数据库
\item \href{https://fungrim.org}{The Mathematical Functions Grimoire}
\item \href{https://mathworld.wolfram.com}{Wolfram Mathworld}
\item \href{https://functions.wolfram.com/}{Mathematical Function Site} Wolfram 的函数网站,有 30 多万公式。
\item 数林广记公众号, 专业数学(JierPeter 推荐)
\end{itemize}

\subsubsection{物理}
\begin{itemize}
\item \href{http://gaozhongwuli.com}{高中物理网}: 高中物理知识点总结
\item \href{http://www.example.com}{Physics Wiki} (CC BY-SA)Fandom 上面的物理百科, 有 166 多个页面, 包括一些研究生资格考试的问题和解答, 以及一般性的文章。
\item \href{http://hyperphysics.phy-astr.gsu.edu}{Hyperphysics} 高中和大学物理百科
\item \href{http://www.astro.uvic.ca/~tatum/index.php}{Physics topics by Tatum} 含有一些天文内容
\item \href{http://www.myliushu.com}{刘叔物理} 高中物理网站
% \item 周思益书签里面有一些抖音快手知乎的科普号
\item \href{https://web.evanchen.cc/coursework.html}{Evan Chen 笔记和讲义}
\item \href{http://www.damtp.cam.ac.uk/user/tong/teaching.html}{David Tong 教授讲义}
\end{itemize}

\subsubsection{计算机}
\begin{itemize}
\item \href{https://www.w3schools.com}{w3schools} 互联网开发系列教程
\item \href{https://www.geeksforgeeks.org/}{GeeksforGeeks} 一个用户友好的计算机短教程网站, 页面很多覆盖面很广。
\item \href{https://cppreference.com}{cppreference} 权威的 C/C++ 标准参考文档。
\item \href{https://www.tutorialspoint.com/index.htm}{Tutorials Point} 编程系列教程
\item \href{http://mathmu.github.io/MTCAS/RecentChanges.html}{MaTHmu} 清华学生国产符号计算软件, 有很多数值/符号计算文档, 还有 \href{https://github.com/maTHmU/MTCAS}{Mathematica Theory of Computer Algebra System.pdf}
\item \href{https://oi-wiki.org}{信息学奥林匹克竞赛} 致力于成为一个免费开放且持续更新的编程竞赛知识整合站点
\item \href{https://pythonnumericalmethods.berkeley.edu/notebooks/Index.html}{Python Numerical Methods}
\item \href{https://www.qikegu.com}{奇客谷}
\item \href{http://www.biancheng.net/}{编程帮} 中文编程教程。 子域名:\href{http://c.biancheng.net}{C 语言中文网} 除了 C 还有 C++,\href{http://c.biancheng.net/python/}{Python 基础教程},\href{http://c.biancheng.net/java/}{Java 学习教程}。另一个子域名:\href{http://data.biancheng.net/}{数据结构与算法教程}。
\item \href{https://www.liaoxuefeng.com}{廖雪峰的博客}
\item \href{https://xiaolincoding.com}{小林coding}
\item \href{https://www.51zxw.net}{我要自学网}
\item \href{https://cloud.tencent.com/developer}{腾讯云开发者社区}
\end{itemize}

\subsection{互动演示}
\begin{itemize}
\item \href{https://phet.colorado.edu}{PhET} 老牌互动插件, 创始人诺奖得主 Carl Wieman
\item \href{https://ophysics.com}{ophysics} 基于 geogebra 的高中物理演示, 已部分翻译到百科
\item \href{https://www.nobook.com}{nobook} 国产 PhET 虚拟实验室
\item \href{https://www.falstad.com}{pfalstad} 比较硬核的网页模拟
\item \href{http://www.cs.cornell.edu/courses/cs5643/2010sp}{计算机 CG 课程}
\item \href{https://demonstrations.wolfram.com/topics.php?PhysicalSciences#5}{Wolfram Demonstration Project}
\item \href{https://github.com/3b1b/manim}{Manim} 3b1b 视频使用的视频制作 python 库
\end{itemize}

\subsection{科学计算软件和工具}
\subsubsection{云计算}
\begin{itemize}
\item \href{https://www.wolframalpha.com/}{Wolfram Alpha}
\item \href{http://www.drhuang.com}{黄博士网} 非常简单的百科和类似 Wolfram Alpha 的云符号计算
\item \href{https://uqs7z-qiaaa-aaaak-qacra-cai.raw.ic0.app/}{MatJ} 网页版 Matlab
\item \href{https://www.latexlive.com}{latexlive} 国产 latex 公式编辑器
\item \href{https://harrypotter.fandom.com}{Fandom} 使用 MediaWiki 的百科网站
\item \href{https://mathpix.com}{Mathpix} 从图片识别 latex 公式。 同时提供一个所见即所得的 latex 编辑器。
\end{itemize}

\subsubsection{在线编辑功能参考}
\begin{itemize}
\item \href{https://www.mathjax.org}{MathJax} 网页 LaTeX 公式渲染, 3.0 版本有性能提升以及更多拓展包(包括 physics)。
\item \href{https://mathpix.com/ocr?gclid=CjwKCAiAy_CcBhBeEiwAcoMRHKnSPGjROmFGBj5R_Ev3hke7lG7xNIzKDHv5FgkMFq62aXRpISt_LhoC47YQAvD_BwE}{Mathpix} 也提供一个所见即所得的 latex 编辑器, 支持 label 和 ref。
\item 编辑器 wysiwyg 参考\href{https://editoe.com}{投光的编辑器(已下线)}以及\href{https://www.wolai.com}{我来}以及 \href{https://github.com/Vanessa219/vditor}{Vditor}。
\item \href{https://pandoc.org/}{Pandoc} 支持很多不同类型文档之间的转换, 包括 LaTeX。
\item \href{https://github.com/michael-brade/LaTeX.js}{LaTeX.js} (小众)类似于 MathJax 但也可以处理非公式的部分。
\item \href{https://www.mediawiki.org/wiki/MediaWiki}{MediaWiki} 是 Wikipedia 的协同编辑器, 使用 PHP。
\item \href{https://www.gitbook.com/}{GitBook} 一个类似 Git 的协同编辑程序, 但并不是基于 Git。
\item \href{https://www.fandom.com/}{Fandom} 一个提供百科服务的网站,内容以娱乐为主, 基于 MediaWiki, 但有一些修改。
\item \href{https://obsidian.md/}{Obsidian} 笔记,有和小时百科非常相似的树状图功能(应该也是使用了 \href{https://d3js.org/}{d3} 库)。
\end{itemize}

\subsubsection{聊天机器人}
\begin{itemize}
\item OpenAI: GPT builder (可以现场去指定网站查内容)
\item Knowbo: 提供 url 或网站地图给它自动爬取网站内容
\item ChatBase: 提供 PDF 或 sitemap 给它学习
\end{itemize}

\textbf{国产}
\begin{itemize}
\item 百度:文新一言,飞桨
\item 腾讯:混元助手
\item 阿里:通义千问
\end{itemize}

\subsection{私有部署企业聊天协同}
\begin{itemize}
\item \href{https://xuanim.com/}{喧喧} 支持语音、视频、屏幕共享、文件共享等。
\item 
\end{itemize}

