% 一元隐函数的存在及可微定理
% 隐函数|存在|可微性|隐函数存在定理

\pentry{隐函数\upref{ImpFun}}
本节将建立保证隐函数单值连续及可微的条件.这由两个定理来保证
\begin{theorem}{存在定理}\label{ImFED_the1}
若:\begin{enumerate}
\item 函数 $F(x,y)$ 在以点 $(x_0,y_0)$ 为中心的长方形邻域
\begin{equation}
\mathcal{D}=[x_0-\Delta,x_0+\Delta;y_0-\Delta',y_0+\Delta']
\end{equation}
中有定义且连续;
\item $F(x_0,y_0)=0$ ;
\item 当 $x=x_0$ 时,函数 $F(x,y)$ 随着 $y$ 的增大而单调增大(或单调减小).
\end{enumerate}
那么:
\begin{enumerate}
\item 在点 $(x_0,y_0)$ 的某一邻域内,方程 $F(x,y)=0$ 确定 $y$ 为 $x$ 的单值函数: $y=f(x)$;
\item $f(x_0)=y_0$ ;
\item $f(x)$ 连续.
\end{enumerate}
\end{theorem}

\begin{theorem}{可微定理}
若:
\begin{enumerate}
\item 函数 $F(x,y)$ 在以点 $(x_0,y_0)$ 为中心的长方形领域
\begin{equation}
\mathcal{D}=[x_0-\Delta,x_0+\Delta;y_0-\Delta',y_0+\Delta']
\end{equation}
中有定义且连续;
\item 在 $\mathcal{D}$ 中偏导数 $F_x',F_y'$ 存在且连续;
\item $F(x_0,y_0)=0$;
\item $F_y'(x_0,y_0)\neq0$.
\end{enumerate}
那么:
除\autoref{ImFED_the1} 的结论1,2,3外,还可证明:\\
$\quad$4.函数 $f(x)$ 有连续导数.

\end{theorem}
\subsection{证明}
\subsubsection{存在定理的证明}
\begin{figure}[ht]
\centering
\includegraphics[width=14cm]{./figures/ImFED_1.pdf}
\caption{隐函数存在定理示意图} \label{ImFED_fig1}
\end{figure}
如\autoref{ImFED_fig1} ,沿着 $x=x_0$ 的竖直线,函数 $F(x,y)$ 变成一个变元 $y$ 的函数 $F(x_0,y)$.根据定理条件2:$F(x_0,y_0)=0$.根据定理条件3
\begin{equation}\label{ImFED_eq1}
F(B_0)=F(x_0,y_0+\Delta')>0,\quad F(A_0)=F(x_0,y_0-\Delta')<0
\end{equation}
沿着通过 $A_0,B_0$的两条水平直线,得到两个 $x$ 的函数:$F(x,y+\Delta'),F(x,y-\Delta')$.由\autoref{ImFED_eq1} 已看到,第一个函数有正值,第二个有负值.按条件1:这两函数时连续的.因此必有点 $x_0$ 的某一邻域 $(x_0-\delta,x_0+\delta),\quad(0<\delta\leq\Delta)$ ,使得这两函数保持自己的符号.于是当 $x\in(x_0-\delta,x_0+\delta)$ 时,
\begin{equation}
F(x,y+\Delta')>0,\quad F(x,y-\Delta')<0
\end{equation}
换而言之,在原矩形的上下底上,有以点 $A_0$ 及 $B_0$ 为中心而长为 $2\delta$ 的线段  $B_1B_2$ 及 $A_1A_2$ ,沿着这些线段,给定函数 $F(x,y)$ 在 $B_1B_2$ 上有 正值而在 $A_1A_2$ 上有负值.

在区间 $(x_0-\delta,x_0+\delta)$ 上任选一点 $\overline{x}$,其在矩形上下底的垂直对应点为 $\overline{B}$ 和 $\overline{A}$,则
\begin{equation}
F(\overline{A})<0,\quad F(\overline{B})>0
\end{equation}
由连续性,必有 $\overline{A}\overline{B}$ 间一点 $y=\overline{y}$,使得
\begin{equation}\label{ImFED_eq2}
F(\overline{x},\overline{y})=0
\end{equation}
由条件3,满足\autoref{ImFED_eq2} 的$\overline{y}$ 是唯一的.

这样,在点 $(x_0,y_0)$ 的邻域 $(x_0-\delta,x+\delta)$ 内,方程 $F(x,y)=0$ 确实确定 $y$ 为 $x$ 的单值函数\autoref{ImpFun_def1}~\upref{ImpFun}.

\subsubsection{可微定理的证明}