% 经典场论基础
% keys 经典场

这一节里面,我们复习一下经典场的知识,为后面的量子场论做铺垫.首先要复习的一个重要的量就是拉式量了,定义如下
\begin{equation}
S = \int L dt = \int \mathcal L(\phi,\partial_\mu \phi)d^4 x
\end{equation}
经典场论的重要原理是变分原理$\delta S = 0$.
\begin{equation}
\begin{aligned}
0 &=\delta S \\
&=\int d^{4} x\left\{\frac{\partial \mathcal{L}}{\partial \phi} \delta \phi+\frac{\partial \mathcal{L}}{\partial\left(\partial_{\mu} \phi\right)} \delta\left(\partial_{\mu} \phi\right)\right\} \\
&=\int d^{4} x\left\{\frac{\partial \mathcal{L}}{\partial \phi} \delta \phi-\partial_{\mu}\left(\frac{\partial \mathcal{L}}{\partial\left(\partial_{\mu} \phi\right)}\right) \delta \phi+\partial_{\mu}\left(\frac{\partial \mathcal{L}}{\partial\left(\partial_{\mu} \phi\right)} \delta \phi\right)\right\}
\end{aligned}
\end{equation} 
最后一项是一个表面项,这里我们考虑边界条件是$\delta \phi$为零的构型,这一项就可以忽略.现在我们看前两项.因为对于任意的$\delta \phi$这个式子都为零,所以我们必须让$\delta \phi$前面的系数为零,这样,我们就推出了著名的欧拉-拉格朗日方程
\begin{equation}
\partial_\mu \bigg( \frac{\partial \mathcal L}{\partial(\partial_\mu\phi)} \bigg) - \frac{\partial \mathcal L}{\partial \phi} = 0 
\end{equation}






