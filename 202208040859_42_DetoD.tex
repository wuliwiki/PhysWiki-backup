% 行列式唯一性定理
% keys 行列式|唯一性

\pentry{行列式的性质\upref{DetPro}}
在行列式的性质\upref{DetPro} 一节中,我们看到行列式 $\det \mat A$ 关于矩阵 $\mat A$ 的行是斜对称的(或反对称),即交换 $\mat A$ 的任意两行行列式变号; 且 $\det \mat A$ 是 $\mat A$ 的行的多重线性函数(下面解释),等等一些其它性质.事实上,任意函数只要满足斜对称性和多重线性性就天然具备其它性质,具有斜对称性和多重线性性的函数简称\textbf{斜对称多重线性函数}.

本节将证明:若 $\mathcal D$ 是矩阵 $\mat A$ 的斜对称多重线性函数,那么它将和 $\mat A$ 的行列式 $\det A$ 成比例,比例系数为 $\mathcal D(\mat E)$ ($\mat E$ 为单位矩阵),即 $\mathcal D(\mat A)=\mathcal D(\mat E)\cdot \det \mat A$.那么若函数 $\mathcal D$ 还满足条件 $\mathcal D(\mat E)=1$,则 $\mathcal D(\mat A)=\det\mat A$.

简言之,若矩阵 $\mat A$ 的斜对称多重线性函数 $\mathcal D$ 满足条件 $\mathcal D(\mat E)=1$ ,那么 $\mathcal D$ 就为矩阵的行列式函数 $\det$.这便是行列式唯一性定理.本定理将有助于与行列式有关的其它理解.
