% 施瓦西度规下时空的测地线
% keys 广义相对论|施瓦西度规
% license Usr
% type Tutor

\pentry{测地线\nref{nod_geodes},世界线与固有时\nref{nod_wdline},Christoffel 符号\nref{nod_CrstfS},自然单位制、普朗克单位制\nref{nod_NatUni}}{nod_6191}
\subsection{施瓦西度规下时空的测地线}
回顾施瓦西度规
\begin{equation}
\dd  s^2 = -\left(c^2 - \frac{2GM}{r}\right) \dd t^2+ \left(1-\frac{2GM}{r c^2}\right)^{-1}\dd r^2 + r^2(\dd \theta^2 + \sin^2 \theta \dd \varphi^2) ~,
\end{equation}
与其 Christoffel 符号(用 $x^0$ 代表时间坐标 $t$,$x^1$ 代表 $r$,$x^2$ 代表 $\theta$,$x^3$ 代表 $\varphi$,使用自然单位制 $G=1, c=1$)
\begin{equation}
\left\{
\begin{aligned}
\Gamma^0_{01} &= \Gamma^0_{10} = M(1-2M/r)^{-1}/r,\\
\Gamma^1_{00} &= M(1-2M/r)/r^2,\\ 
\Gamma^1_{11} &= -(1-2M/r)^{-1}/r^2,\\
\Gamma^1_{22} &= -r(1-2M/r),\\
\Gamma^1_{33} &= -r(1-2M/r)\sin^2 \theta,\\
\Gamma^2_{12} &= \Gamma^2_{21} = 1/r, \\
\Gamma^2_{33} &= -\sin \theta \cos \theta, \\
\Gamma^3_{13} &= \Gamma^3_{31} = 1/r, \\
\Gamma^3_{23} &= \Gamma^3_{32} = \cot \theta ~.
\end{aligned}\right. ~~
\end{equation}
以及度规降指标后的黎曼曲率张量 $R_{ijkm}$
\begin{equation}
\begin{aligned}
R_{0101} = -2M/r^3, R_{0202} = M(1-2M/r)/r, R_{0303} = M(1-2M/r)
\end{aligned}
\end{equation}

这指出在大天体 $M$ 附近的时空的情况。

而考虑对于一条类时的测地线 $\gamma(\tau)$,其中 $\tau$ 是固有时,若要求分量的参数表达式 $x^\mu(\tau)$ 则是要解测地线方程
\begin{equation}
\dv[2]{x^\mu}{\tau} + \Gamma^{\mu}_{\nu \lambda} \dv{x^\nu}{\tau} \dv{x^\lambda}{\tau} = 0~.
\end{equation}
其中 $\Gamma^\mu_{\nu \lambda}$ 是 Christoffel 符号。

而对于施瓦西度规下的时空,由于对称性,总可以选取某个坐标系使得 $\gamma(\tau)$ 的 $\theta$ 值恒为 $\pi/2$,即这类时测地线总在赤道面内。
