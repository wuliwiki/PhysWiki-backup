% x86-64 笔记

\begin{issues}
\issueDraft
\end{issues}

参考 Wikipedia \href{https://en.wikipedia.org/wiki/X86-64}{相关页面}。

x86 \textbf{指令集(instruction set)}, 或者\textbf{指令集架构(ISA, instruction set architecture)}, 分为 32 位和 64 位两个版本。

现在最常见的是 64 位, 称为 x86-64 (也叫 x64, x86_64, AMD64, 和 Intel 64)

比较老的 32 位版本, 称为 IA-32 (Intel Architecture, 32-bit, 也叫 i386)


\subsection{寄存器}
\begin{itemize}
\item 提供 16 个通用寄存器(general-purpose registers), 每个都是 16 bit。
\item \verb|RAX, RBX, RCX, RDX|: arithmetic and data manipulation
\item \verb|RSI, RDI, RBP, RSP|: storing data and addressing memory
\item \verb|R8|-\verb|R15|: additional arithmetic and data manipulation
\end{itemize}
