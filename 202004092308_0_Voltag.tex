% 电压

\pentry{电势 电势能\upref{QEng}}

广义来说, \textbf{电压}是指空间中任意两点的电势之差, 可以由电场的线积分\autoref{QEng_eq1}\upref{QEng}定义, 记为
\begin{equation}
U_{21} = V(\bvec r_2) - V(\bvec r_1) = - \int_{\bvec r_1}^{\bvec r_2} \bvec E(\bvec r) \vdot \dd{\bvec r}
\end{equation}

注意这个广义定义只在静电学中有意义, 即要求净电荷分布和电流分布不随时间变化.

但在当我们讨论电路时, 虽然可能不满足静电学的条件(例如交流电, 以及磁生电等情况), 但仍然可以用上式定义电路中任意两点之间的瞬时电压, 但是积分路径必须要沿着电路.

\begin{exercise}{磁生电}
假设我们有一个 $N$ 匝的不闭合线圈, 两端接在理想电压表上. 若线圈中有磁铁在上下运动, 使线圈中的磁通量随时间变化, 那么根据高中的知识我们知道电压表会显示读数.

然而这并不是一个静电学问题, 上下运动的磁铁会沿着线圈产生涡旋电场, 显然这种电场不是一个保守场, 所以一般来说我们无法在空间中给出\textbf{与路径无关}的电势的定义. 但如果我们只讨论导线
\end{exercise}