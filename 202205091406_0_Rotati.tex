% 轴角法进行旋转 (Matlab)

\begin{issues}
\issueDraft
\end{issues}

\addTODO{给出思路}

\pentry{罗德里格旋转公式、绕轴旋转矩阵\upref{RotA}}

计算点 $p$ 绕直线 $\frac{x-x_0}{k_x}=\frac{y-y_0}{k_y}=\frac{z-z_0}{k_z}$($\vec{k}$ 是法向单位矢量)按右手螺旋规则旋转 $\theta$ 角得到的新点

以下给出 Matlab 代码.


\begin{lstlisting}[language=matlab]
%main.m
clc;
clear all;
close all;


p=[2,0,0]; %旋转前的点
r0=[1,0,0]; %直线上的一点r0
x0=r0(1);
y0=r0(2);
z0=r0(3);
param=[0,1,0,pi/3];%分别是kx,ky,kz,theta
p_relative=p-r0;  %旋转前的点相对r0的矢量
p_rotated=Rotation(p_relative,param)+r0 %p_rotated就是结果

\end{lstlisting}



\begin{lstlisting}[language=matlab]
% Rotaion.m
function p1=Rotation(p,param)
[row,col]=size(param);
if row==1&col==4 
    ;
else
    warndlg('请检查Rotation函数的第二个参数param的size是否是1*4');
end
if norm(param(1,end-1))-1.0<1e-8
    ;
else
    warndlg('请检查Rotation函数的第二个参数param的kx,ky,kz构成的向量是否模长1')
end   
    
[row,col]=size(p);
if row==1&col==3
    ;
else 
    warndlg('请检查Rotation函数的第一个参数p的size是否是1*3');
end

kx=param(1);
ky=param(2);
kz=param(3);
theta=param(4);

K=[0 -kz ky;
   kz 0 -kx;
   -ky kx 0];
R = eye(3) + sin(theta)*K + (1- cos(theta))*K*K;    %罗德里格斯公式
p1= R*p';                           %利用罗德里格斯公式计算的旋转后的点
p1=p1';
end
\end{lstlisting}

\addTODO{demo}
