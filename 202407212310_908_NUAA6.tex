% 南京航空航天大学 2014 量子真题
% license Usr
% type Note

\textbf{声明}:“该内容来源于网络公开资料,不保证真实性,如有侵权请联系管理员”

\subsection{简答题 (本题 45 分,每小题 15 分)}
①写出氢原子、一维简谐振子、一维无限深势阱的能级,并用示意图表示。

②证明:定态波函数 $\psi(x)$总可以取作实数的。

③能量本征态有可能是角动量 $\hat{L}^2$ 的本征态吗?有可能是 $\hat{L}_z$ 的本征态吗?请回答为什么
并举例说明。

\subsection{二}
在一维无限深势阱中,一个粒子的初始波函数由前两个定态迭加而成:$\psi(x,0)=A[\psi_1 (x)
+\psi_2 (x)]$。为了简化计算可令 $\omega=\pi^2\hbar/2ma^2$

①归一化$\psi(x,0)$,并求 $\psi(x,t)$和$|\psi(x,t)|^2$,把后者用时间的正弦函数展开。

②计算〈$x$〉、〈$p$〉的值。它们是随时间振荡的,角频率是多少?振幅是多少?

③测量粒子的能量,可能得到什么值?得到各个值的几率是多少?求出 $\hat{H}$ 的期望值。并
与 $E_1$ 和 $E_2$ 比较。(本题 20 分)

\subsection{三}
质量为 $m$ 的粒子在一维线性谐振子势:$V(x)=m\omega^2x^2/2$ 中运动。在占有数表示中哈密顿量可写为$\hat{H} = \left(\hat{a}^\dagger \hat{a} + \frac{1}{2}\right)\hbar\omega$,这里 
$$\hat{a}^\dagger = \sqrt{\frac{m\omega}{2\hbar}} \left( \hat{x} - \frac{i}{m\omega} \hat{p} \right), \quad \hat{a} = \sqrt{\frac{m\omega}{2\hbar}} \left( \hat{x} + \frac{i}{m\omega} \hat{p} \right)~$$分别为升、降算符。已知谐振子基态波函数为:
$$\psi_0(x) = \sqrt[4]{\frac{m\omega}{\pi \hbar}} e^{-\frac{m\omega x^2}{2 \hbar}}~$$
① 利用升算符性质:$a^{\dagger} \psi_n(x) = \sqrt{n+1} \psi_{n+1}(x)$,求谐振子第一激发态的波函数;

② 假设粒子处在基态 $( \psi_0(x) )$,突然改变谐振子的“振动频率”为 $( \omega' = 2\omega )$,粒子新的基态能量是多少?新的基态波函数是什么?

③ 假设这时粒子波函数仍然保持 $( \psi_0(x) )$不变,此时测量粒子能量,发现粒子能量取新的基态能的几率是多少?(本题 25 分)

\subsection{四}
在 $t = 0 $ 时,氢原子的波函数$\Psi(r, 0)$ 为:$$\Psi(r, 0) = \frac{1}{\sqrt{10}} \left[ 2\psi_{100} + \psi_{210} + \sqrt{2}\psi_{211} + \sqrt{3}\psi_{21-1} \right]~$$
式中波函数的下标分别为量子数 $ n, l, m $ 的值,忽略自旋和辐射跃迁。

①  写出在 t 时刻的波函数;

② 在$ t = 0$ 时振子能量的平均值是多少?$t = 1$ 秒时呢?(本题 20 分)

\subsection{五}
电子静止在一振荡磁场 $\vec{B} = B_0 \cos(\omega t) {\vec{k}}$ 中,其哈密顿量写作 $\hat{H} = - \gamma \vec{B} \cdot \hat{\vec S}$,其中 $\mathbf{S}$ 为自旋角动量,$\gamma$(旋磁比),$B_0$(磁场振幅)和 $\omega$(振荡圆频率)为三个常数。

1. 构造这个体系的哈密顿矩阵。

2. 电子的初始态 $t = 0$ 时为处于 $x$ 轴方向向上的自旋态,即:$\chi (t) = \chi_{0}^{(+)}$。确定以后的任意时刻的 $\chi (t)$。

3. 如果测量 $S_x$,求出得到 $- \hbar / 2$ 的几率。

4. 迫使 $S_z$ 完全翻转所需要的最小磁场 $B_0$ 是多大?(本题 20 分)

\subsection{六}
粒子在二维无限深方势阱中运动,$ V = \begin{cases} 0, & 0 < x, y < a \\\\ \infty, & \text{其他}\end{cases} $。 加上微扰 $ H' = \lambda xy $。 求基态、第一激发态能级的一级微扰论修正。(本题 20 分)
