% 四川大学 2009 年研究生入学物理考试试题
% keys 四川大学|2009年|考研|物理
% license Copy
% type Tutor

\textbf{声明}:“该内容来源于网络公开资料,不保证真实性,如有侵权请联系管理员”


科目代码:947


适用专业:光学、无线电物理、物理电子学


可能用到的物理常数:真空中的介电常数:$8.85*10^{-12}F/m$,真空中的导磁率:$4\pi*10^{-7}H/m$,一个电子的电量:$1.62*10^{-19}C$
\subsection{填空题}
\begin{enumerate}
\item  处于静电平衡的理想导体,导体内部电场强度为 $(\qquad)$ 随曲率半径增大,导体外表面的电场强度$(\qquad)$ 
\item 利用万用表测量市电的交流电压,读数为220V,是指交流电压的$(\qquad)$ 值,对应的峰值电压为$(\qquad)$ V。
\item 相隔距离为 d 的等量同号点电荷+q 和+q,二者中点处的电势为$(\qquad)$ v,电场强度的大小为$(\qquad)$ V/m。
\item 两根无限长的均匀带电直线相互平行,相距为2a,线电荷密度分别为+$\rho$和-$\rho$则每单位长度上的带电直线受的库仑力为$(\qquad)$N,两根直线相
互$(\qquad)$。
\item 两块平行金属板间充满电容率为$\varepsilon_1=2\varepsilon_0$的均匀介质,当维持两块金属板上电压V不变,每块平行金属板的电荷为Q。1)如果将介质换为$\varepsilon_2=2\varepsilon_1$的介质则每块平行金属板的电荷为$(\qquad)$
。2)如果将介质$\varepsilon_1$ 移去,则每块平行金属板的电荷为$(\qquad)$。
\item 一波长为 550nm的黄绿光入射到间距为0.2mm的双缝上,则离缝2m远处观察屏上干涉条纹的间距是$(\qquad)$mm;若缝间距增大为2mm,则干涉条纹间距变为$(\qquad)$mm。
\item 一玻璃劈尖,折射率为1.52,波长 $\lambda=589.3nm$的鈉光垂直入射时,测得相邻条纹间距$L=5.0mm$,则劈尖夹角约为$(\qquad)$弧度。
\item 用迈克耳孙干涉仪测量光的波长过程中,测得可动反射镜移动了$\Delta L=0.3220mm$,等倾条纹在中心处缩进$1092.8$个条纹,则所采用的光波长$(\qquad)$nm。
\item 当人眼瞳孔直径为2mm 时,此时人眼的最小分辨角约为$(\qquad)$分(设照明的光波长为550nm)。
\item 自然光通过两个偏振化方向间成60°的偏振片,透射光强为$I_1$。今在这两个偏振片之间再插入另一偏振片,它的偏振化方向与前两个偏振片均为30°角,则透射光强为$(\qquad)I_1$
\end{enumerate}
\subsection{选择题}
\begin{enumerate}
\item 有一导体置于有一导体置于静电场中,如图所示,其中L>2a,位置1代表尖端处,位置 2代表柱体部分,位置3代表底面,请问该导体电位最高的位置是$(\qquad)I_1$\\
(A)位置1\\
(B)位置2\\
(C)位置3\\
(D)所有位置电位相同
\item 下列说法正确的是$(\qquad)I_1$\\
(A)理想导体表面电场强度为零\\
(B)理想导体的电容与形状无关\\
(C)两个理想导体所带的电荷量相同,则其电位也一定相同\\
(D)上述说法都不对
\item 下列关于超导体与理想导体的说法,哪一个不正确$(\qquad)I_1$
(A)超导体的电阻为零\\
(B)理想导体的电阻为零\\
(C)超导体内可以存在磁感应强度$\vec B$\\
(D)理想导体内可以有不随时间变化的磁感应强度$\vec B$
\item 如图所示,一封闭的导体壳A内有两个导体B和C,所带电荷为$Q_A=Q_0,Q_B=Q_c=0$。请问下列说法正确的是$(\qquad)I_1$
A.UU>Ua
\end{enumerate}