% 压缩映射
% keys 压缩映射
% license Xiao
% type Tutor
\pentry{映射\upref{map},度量空间\upref{Metric}}
首先,压缩映射是集合到自身的一个映射,其次,“压缩”是指对集合中任意两点,在压缩映射下两点的像点的“距离”比两原像更小。这就是说,压缩映射是定义在带有“距离”的集合上的。任意验证,满足上面定义的压缩映射 $A$ 的定义集合 $X$ 是个无限集。事实上,若 $X$ 有限,且 $d:X\times X\rightarrow \mathbb R$ 是距离函数\footnote{表示"距离"的集合的元素应都是可比较的,就是说它是个有序集\upref{OrdRel},不妨将这个集合取作实数集。}。我们构建两个序列
\begin{equation}
\{x,Ax,A^2x,\cdots\},\{y,Ay,A^2y,\cdots\}~,
\end{equation}
由于 $X$ 有限,所以必有$m_1<n_1,m_2<n_2$ 存在,使得 $A^{m_1}x=A^{n_1}x,A^{m_2}y=A^{n_2}y$ 。那么
\begin{equation}
A^{m_1+(n_1-m_1)(n_2-m_2)}x&=\underbrace{A^{(n_1-m_1)}\cdots A^{(n_1-m_1)}}_{(n_2-m_2)\text{个}}A^{m_1}x\\
&=\underbrace{A^{(n_1-m_1)}\cdots A^{(n_1-m_1)}}_{(n_2-m_2-1)\text{个}}A^{m_1}x~.
\end{equation}
