% 高斯散度定理(综述)
% license CCBYSA3
% type Wiki

本文根据 CC-BY-SA 协议转载翻译自维基百科\href{https://en.wikipedia.org/wiki/Divergence_theorem}{相关文章}。

在向量分析中,散度定理,又称高斯定理或奥斯特罗格拉德斯基定理[1],是一条将向量场通过闭合曲面的通量与该曲面所包围体积内场的散度联系起来的定理。

更准确地说,散度定理表明:一个向量场在闭合曲面上的曲面积分(即该曲面的“通量”)等于该曲面所包围区域内散度的体积分。直观地理解,这意味着“一个区域内所有场源的总和(将汇点视为负源)等于该区域向外的净通量”。

散度定理在物理和工程数学中有着重要地位,尤其是在静电学和流体力学领域。在这些领域中,它通常应用于三维情形。然而,该定理可以推广到任意维度。在一维情形下,它等价于微积分基本定理;在二维情形下,它等价于格林定理。
