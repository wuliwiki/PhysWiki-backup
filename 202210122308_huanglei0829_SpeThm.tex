% Sperner定理
% 组合数学|Sperner定理

先给出一些记号:
\begin{itemize}
\item 集合$\{1,2,\cdots,n\}$简记为$[n]$
\item 集合$X$的所有子集构成的集族记为$2^{X}$,称为$[n]$的幂集
\end{itemize}

\begin{definition}{独立族}
设$\mathcal{F}\subset 2^{[n]}$是$[n]$的一个子集族.我们说$\mathcal{F}$是\textbf{独立的}如果$\forall A,B\in \mathcal{F}$我们都有$A\not\subset B$和$B\not\subset A$(即$\mathcal{F}$中的任意两个集合互不包含).
\end{definition}

下面我们给出Sperner定理:
\begin{theorem}{Sperner's Theorem}
对于任意$[n]$的独立子集族$\mathcal{F}$,我们有:$|\mathcal{F}|\leq \pmat{n\\\lfloor \frac{n}{2}\rfloor}$

\end{theorem}
为给出定理1的证明,我们先引入\textbf{链}的定义:
\begin{definition}{链}
$[n]$的子集构成的一条\textbf{链}指一列$[n]$的子集$\{A_i\}_{i=1}^{k}$使得:\\
$A_1\subset A_2\subset A_3\subset \cdots \subset A_k$,其中$A_1,\cdots,A_k$互不相同.这里$k$称为链的长度.\\
$\qquad$
一条链称为\textbf{极大链}如果$k=n+1$.简单地我们知道,$[n]$的子集可以构成的极大链的个数为$n!$.
\end{definition}

\textbf{定理1}的证明:\\
\begin{enumerate}
\item 如果$\mathcal{F}$是独立族,那么每一条极大链$\mathcal{C}$中至多含有一个$\mathcal{F}$中的集合;
\item 算两次:$\sum\limits_{A\in \mathcal{F}}N_A=\sum\limits_{\mathcal{C}}N_{\mathcal{C}}$,其中$N_A$表示包含集合$A$的极大链的个数,$N_{\mathcal{C}}$表示极大链$\mathcal{C}$中含有的$\mathcal{F}$中的集合的个数,$\sum\limits_{\mathcal{C}}$表示对所有极大链求和;
\item 考察$N_A$,$N_A=|A|!\cdot (n-|A|)!=\frac{n!}{\pmat{n\\|A|}}\geq \frac{n!}{\pmat{n\\ \lfloor\frac{n}{2} \rfloor}}$;
\item 由1.,我们有:$N_{\mathcal{C}}\leq 1$;
\item 从而$n!=\sum\limits_{\mathcal{C}}1\geq \sum\limits_{\mathcal{C}}N_{\mathcal{C}}=\sum\limits_{A\in \mathcal{F}}N_A\geq \sum\limits_{A\in \mathcal{F}}\frac{n!}{\pmat{n\\ \lfloor\frac{n}{2} \rfloor}}=\frac{n!}{\pmat{n\\ \lfloor\frac{n}{2} \rfloor}}\cdot |\mathcal{F}|$,整理即得$|\mathcal{F}|\leqslant\pmat{n\\ \lfloor\frac{n}{2} \rfloor}$,证毕.
\end{enumerate}

Sperner定理(\textbf{定理1})的另一个证明用到如下引理:(以下讨论默认对$[n]$进行)

\begin{lemma}{$[n]$的幂集的对称链划分}
\begin{definition}{对称链}
一个链称为\textbf{对称链}如果$|A_1|=\frac{n+1-k}{2}$
\end{definition}

\end{lemma}


