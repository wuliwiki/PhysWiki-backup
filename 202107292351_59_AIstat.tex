% 概率密度函数与人工智能概论
% keys latex
\pentry{此文件保存为学习人工智能的监督学习和非监督学习算法的笔记页,概率密度函数是学习人工智能的基础函数.\upref{Sample}}

\begin{definition}{为什么求总似然的时候,要用正态分布的概率密度函数?}
由于中心极限定理,假设所有假设样本之间都为独立事件,并且误差变量随机产生,那样就服从正态分布!
\end{definition}

\begin{theorem}{为什么采用概率密度相乘而不是概率直接相乘?}
由于概率并不好求,所以找到了概率密度最大的时候也就相当于找到了概率最大的时候!
\end{theorem}

\begin{equation}
y^{\left(i\right)}=\theta^T x^{\left(i\right)} + \varepsilon^{\left(i\right)} %theta指的是W1->Wm每个W都是一个权重系数.对theta矩阵转置乘以影响因子x^i再与y^i相比较
\end{equation}

\begin{equation}
p\left(\epsilon^{\left(i\right)}\right)=\frac{1}{\sqrt{2\pi}\sigma}exp\left(-\frac{\left(\epsilon^{\left(i\right)}\right)^2}{2\sigma^2}\right)
\end{equation}

\begin{equation}
p\left(y^{\left(i\right)}|\; x^{\left(i\right)};\;\theta\right)=\frac{1}{\sqrt{2\pi}\sigma}exp\left(-\frac{\left(y^{\left(i\right)}-\theta^Tx^{\left(i\right)}\right)^2}{2\sigma^2}\right)
\end{equation}

\begin{equation}
\ali{
L\left(\theta\right)&=\prod_{i=1}^m p\left(y^{\left(i\right)} |\; x^{\left(i\right)};\;\theta\right)\\
&=\prod_{i=1}^m \frac{1}{\sqrt{2\pi}\sigma}exp\left(-\frac{\left(\epsilon^{\left(i\right)}\right)^2}{2\sigma^2}\right)
}
\end{equation}

\begin{definition}{取极值}
如果想要取得总似然的最大值,最好的模拟方法不是求得连乘的最大值,而是求得连加的最大值,要想得到这样的过程,我们应该对函数极值两侧取对数.
\end{definition}

\begin{equation}
\ali{
\boldsymbol{l}\left(\theta\right)&=ln\;L\left(\theta\right)\\
&=ln\frac{1}{\sqrt{2\pi}\sigma}exp\left(-\frac{\left(y^{\left(i\right)}-\theta^Tx^{\left(i\right)}\right)^2}{2\sigma^2}\right)\\
&=\sum_{i=1}^m ln\frac{1}{\sqrt{2\pi}\sigma}exp\left(-\frac{\left(y^{\left(i\right)}-\theta^Tx^{\left(i\right)}\right)^2}{2\sigma^2}\right)\\
&=m\,ln\frac{1}{\sqrt{2\pi}\sigma}-\frac{1}{\sigma^2}\cdot\frac{1}{2}\sum_{i=1}^m \left(y^{\left(i\right)}-\theta^Tx^{\left(i\right)}\right)^2
}
\end{equation}

\begin{equation}
{\color{Red}\boldsymbol{J}\left(\theta\right)=\frac{1}{2}\sum_{i=1}^m \left(h_\theta\left(x^{\left(i\right)}\right)-y^{\left(i\right)}\right)^2
}
\end{equation}

\begin{issues}
    加入一些我对人工智能的理解:为什么选取正态分布函数呢?或者是为什么不直接采用?
\begin{equation}\epsilon^2=0\end{equation}其实在哔哩哔哩上已经讲过,其实最后模拟完成之后带入的函数其实是Function_sigmoid(x)【现在已经被替换成ReLU(x)】进行优化.\\
\\
    但是我们这个函数的表达形式是因为:我们的x^i和y^i是固定模拟且随机的,所以这样结果所带来的epsilon是属于随机误差,而又根据概率论与数理统计的定理来看,一旦是随机的,那么应该处于正态分布的中间部分,其实也就是正太分布的极值点附近.只要能保证m个正态分布概率曲线相乘取到的值最大,也就说明了这样是最符合随机分布的.\\
\\
    要使L(theta)最大,就要使J(theta)最小.\\
\\
但是,这样的模型构建也有一定的问题,比如实际生活中并非所有的变量都是没有相互关联的,而且还可能出现两个正太分布峰值的情况.还有,在没有很多训练样本的情况下,不一定最大值就满足在正太分布的峰值的情况.
\end{issues}

\begin{equation}
\ali{
    \boldsymbol{J}\left(\theta\right)&=\frac{1}{2}\sum_{i=1}^m \left(h_\theta\left(x^{\left(i\right)}\right)-y^{\left(i\right)}\right)^2 \\
&=\frac{1}{2}\left(X\,\theta-y\right)^T\left(X\,\theta-y\right)\\
&{\color{Red}=\frac{1}{2}\left(\theta^T\,X^T-y^T\right)\left(X\,\theta-y\right)}
}
\end{equation}

再对上述式子进行展开

\begin{equation}
\ali{
\boldsymbol{J}\left(\theta\right)&=\frac{1}{2}\left(\theta^T\,X^T\,X\,\theta-y^T\,X\,\theta-\theta^T\,X^T\,y+y^T\,y\right)
}
\end{equation}

现在如果求其最小值,则需要求驻点,即:

\begin{equation}
\partial\boldsymbol{J}\left()
\end{equation}
