% 宇宙学的基本方程
% license Usr
% type Tutor

\subsection{弗里德曼方程和加速度方程}
\textbf{弗里德曼方程(Friedmann Equation)}是爱因斯坦方程的时间部分,即:
\begin{equation}G_{00}+\Lambda g_{00}=8\pi GT_{00}\quad\Rightarrow\quad\left(\frac{\dot{a}}{a}\right)^2=H^2=\frac{8\pi G}{3}\rho+\frac{\Lambda}{3}-\frac{k}{a^2}~,\end{equation}
可见,如果宇宙常数项足够大,$\dot a$将永不为0,宇宙将一直膨胀。而如果宇宙常数项很小,或者为负值,宇宙则有可能在将来经历停止膨胀,然后收缩的命运(比之如今的宇宙热历史理论,该方程是时间反演不变的。)\footnote{有一个理论是宇宙常数并非“常数”,而是缓慢变化的,称这种可能性为quintessence。}

对上式进行求导,便得到\textbf{加速度方程(acceleration equation)}:

\begin{equation}
\Rightarrow\quad\frac{\ddot{a}}{a}-\frac{\Lambda}{3}=-\frac{4\pi G}{3}(\rho+3P)~,
\end{equation}
又称作\textbf{雷乔杜里方程(Raychaudhuri equation)}。丛该方程上看,$\rho,P$的作用是使膨胀减速,可以理解为引力作用;宇宙常数项则能促进宇宙膨胀。

\subsection{连续性方程}
由爱因斯坦方程一节可知,我们可以把宇宙的一切组成成分看作没有粘度和热传递的理想流体,所遵循的四动量守恒方程在弯曲流形的拓展为
\begin{equation}
\nabla_{i}T^{ij}=0~.
\end{equation}
用Christoffel 符号表示其协变微分,便是

\begin{equation}\nabla_iT^{ij}=\partial_iT^{ij}+\Gamma_{ik}^iT^{kj}+\Gamma_{ik}^jT^{ik}=0~.
\end{equation}
代入具体联络的具体数值:
\begin{equation}\Gamma_{00}^0=0\quad;\quad\Gamma_{01}^1=\Gamma_{02}^2=\Gamma_{03}^3=\frac{\dot{a}}{a}~,\end{equation}
令$\nu=0$,便得到\textbf{连续性方程(continuity function)}——
\begin{equation}\label{eq_Frieq_1}
\dot{\rho}+3\frac{\dot{a}}{a}\left(\rho+\frac{p}{c^2}\right)=0~.
\end{equation}
\subsubsection{辐射主导时期与物质主导时期}
在把宇宙物质看作理想流体后,把体现压强与密度关系的$P(\rho)$称作\textbf{状态方程(state function)}。在宇宙学里,我们常用的关系为$p=k\rho,k\in \mathbb R$。应用到两种典型的理想流体上,便得到辐射主导情况(radiation-dominated)与物质主导情况(matter-dominated)。

所谓物质主导即宇宙这一集合可视作尘埃云,满足$p=0$。代入\autoref{eq_Frieq_1} 得:
\begin{equation}\dot{\rho}+3\frac{\dot{a}}{a}\rho=0\quad\Longrightarrow\quad\frac1{a^3}\dv{(\rho a^3)}{t}=0\quad\Longrightarrow\quad\dv{(\rho a^3)}{t}=0 ~,\end{equation}
显然,此时$\rho\propto 1/a^3$,这是只考虑三维体积膨胀效应的结果。

定义现今密度$\rho_0=\rho(t_0)$,则
\begin{equation}
\rho=\frac{\rho_0}{a^3},\quad ~.
\end{equation}

所谓辐射主导则指宇宙的成分集合以相对论性的粒子为主,此时$p=\rho/3$,代入\autoref{eq_Frieq_1} 后可解得:
\begin{equation}\dot{\rho}+4\frac{\dot{a}}{a}\rho=0\quad\Longrightarrow\quad\frac1{a^4}\dv{(\rho a^4)}{t}=0\quad\Longrightarrow\quad\dv{(\rho a^4)}{t} =0~,\end{equation}
显然可知,此时$\rho\propto  1/a^4$,这是空间膨胀效应和光子红移的共同结果。
