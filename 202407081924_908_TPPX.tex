% 拓扑排序
% license CCBYSA3
% type Wiki

(本文根据 CC-BY-SA 协议转载自原搜狗科学百科对英文维基百科的翻译)

在计算机科学中,有向图的拓扑排序或拓扑排序是其顶点的线性排序,使得对于每个从顶点u到顶点v的有向边,代表$u$在排序中位于$v$之前。例如,图的顶点可以表示要执行的任务,而边可以表示一个任务必须先于另一个任务执行的约束;在这个应用中,拓扑排序只是任务的有效序列。当且仅当图没有有向循环时,即当它是有向无环图时,才可能是拓扑排序。任何DAG都至少有一个拓扑排序,并且存在用于在线性时间内构造任何DAG的拓扑排序的算法。

\subsection{例子}
拓扑排序的典型应用是根据作业或任务的相关性来安排它们的顺序。作业由顶点表示,如果作业x必须在作业y开始之前完成,则从x到y有一条边(例如,洗衣服时,洗衣机必须在我们将衣服放入烘干机之前完成)。然后,拓扑排序给出了执行作业的顺序。拓扑排序算法的一个密切相关的应用是在20世纪60年代早期被首次研究的,其背景是在项目管理中用于调度的PERT技术(Jarnagin 1960);在这个应用中,图的顶点代表项目的决定性事件,边代表必须在一个决定性事件和另一个之间执行的任务。拓扑排序构成了寻找项目关键路径的线性时间算法的基础,一系列决定性事件和任务控制着整个项目进度的长度。

在计算机科学中,这种类型的应用出现在指令调度、在电子表格中重新计算公式值时对公式单元格求值的排序、逻辑合成、确定在makefiles中执行的编译任务的顺序、数据序列化以及解决链接器中的符号依赖性等方面。它还用于决定在数据库中以何种顺序加载带有外键的表。

\subsection{算法}
拓扑排序的常用算法的运行时间是节点数加上边数的线性渐进,