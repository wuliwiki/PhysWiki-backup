% 完备域
% complete domain|可分多项式|reducible polynomial|微商|形式微商|导数|多项式|重根|自同构|形式求导|形式微分

\pentry{分裂域\upref{SpltFd}}

由\autoref{SpltFd_the1}~\upref{SpltFd}可见,不可约多项式在其分裂域中有无重根,决定了该分裂域的自同构数量.一个域的全体自同构配合映射的复合,能构成一个群,域的许多性质都蕴含在这个群的结构中.群的元素数量自然是其重要性质之一.

综上所述,研究多项式的重根是非常重要的课题.

\subsection{可分多项式}

为了研究重根,我们借用微积分的知识,引入\textbf{形式微商}的概念

\begin{definition}{形式微商}\label{CmplD_def1}

设$\mathbb{F}$是一个域,$f\in\mathbb{F}[x]$.若$f$表达为
\begin{equation}
f(x) = \sum_{i=0}^n a_ix^i
\end{equation}
其中各$a_i\in\mathbb{F}$,那么定义\textbf{形式微商}算子$\opn{D}:\mathbb{F}[x]\to\mathbb{F}[x]$为:
\begin{equation}
\opn{D}f = \sum_{i=0}^{n-1} a_{i+1}x^i
\end{equation}

形式微商也可称为“形式求导”、“形式微分”等.

\end{definition}

\autoref{CmplD_def1} 形式微商就是直接套用微积分中的求导操作,只不过这里没有求导的概念,而是就进行多项式变换,其形式就是求导或者求导的推广,因此才叫\textbf{形式}微商.

\begin{example}{}
考虑域$\mathbb{Z}_3$上的多项式$f(x)=x^5+2x^2-x-2$,则
\begin{equation}
\opn{D}f(x) = 2x^4+x-1
\end{equation}

显然,这和真正的求导不同.一方面我们没有在域$\mathbb{Z}_3$上定义极限的概念,另一方面实数域上$f$的导函数应该是$5x^4+4x-1$.
\end{example}

\begin{example}{一个很抽象的例子}

考虑$\mathbb{R}$上的集合$S=\{\text{非零函数}\}\cup\{f|f(x)\equiv 0\}$,则$S$配上函数之间逐点相加和逐点相乘的运算,构成一个域,记为$\mathbb{S}$.

取$f, g\in\mathbb{S}$,构成多项式$F(y)=(f+g)y^2+(fg)y$.则
\begin{equation}
\opn{D}F(y) = (2f+2g)y+fg
\end{equation}

\end{example}

容易验证,形式微商有以下性质:

\begin{theorem}{形式微商的性质}\label{CmplD_the1}

给定域$\mathbb{F}$,$\opn{D}$是$\mathbb{F}[x]$上的形式微商算子.则:

1. 任取$a\in\mathbb{F}$,则作为多项式,$\opn{D}a=0$.

2. 当$\opn{ch}\mathbb{F}=0$\footnote{概念见\autoref{field_def2}~\upref{field}. }时,对于$f\in\mathbb{F}[x]$有:$\opn{D}f=0 \iff \opn{deg}f=0$.

3. 对于$f, g\in\mathbb{F}[x]$,$a, b\in\mathbb{F}$,有$\opn{D}(af+bg)=a\opn{D}f+b\opn{D}g$.

4. 对于$f, g\in\mathbb{F}[x]$,有$\opn{D}(fg) = (\opn{D}f)g+f\opn{D}g$.

\end{theorem}

上面这些性质都是由导数的性质自然启发而得的.但要注意的是,函数的导数和多项式的形式微商在概念上有所重叠,却不是互相包含的.函数的导数可以用来处理非多项式的函数,而多项式的形式微商又可以处理非实数域的多项式,所以请不要随意将二者混为一谈.





















