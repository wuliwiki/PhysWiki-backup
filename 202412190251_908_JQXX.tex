% 机器学习(综述)
% license CCBYSA3
% type Wiki

本文根据 CC-BY-SA 协议转载翻译自维基百科\href{https://en.wikipedia.org/wiki/Machine_learning}{相关文章}。

\textbf{机器学习(ML)}是人工智能的一个研究领域,关注于开发和研究能够从数据中学习并对未见过的数据进行泛化的统计算法,从而在没有明确指令的情况下执行任务。[1] 深度学习领域的进展使得神经网络在性能上超越了许多先前的方法。[2]

机器学习应用于许多领域,包括自然语言处理、计算机视觉、语音识别、电子邮件过滤、农业和医学。[3][4] 将机器学习应用于商业问题的领域被称为预测分析。

统计学和数学优化(数学编程)方法构成了机器学习的基础。数据挖掘是一个相关的研究领域,专注于通过无监督学习进行探索性数据分析(EDA)。[6][7]

从理论角度来看,可能大致正确(PAC)学习为描述机器学习提供了一个框架。
\subsection{历史}  
“机器学习”这一术语由IBM员工、计算机游戏和人工智能领域的先驱亚瑟·塞缪尔(Arthur Samuel)于1959年创造。[8][9] 在这一时期,“自我学习计算机”这个同义词也曾被使用。[10][11]

尽管最早的机器学习模型是在1950年代由亚瑟·塞缪尔发明的,该程序用于计算每方在跳棋中的获胜概率,但机器学习的历史可以追溯到几十年来人类对研究人类认知过程的渴望和努力。[12] 1949年,加拿大心理学家唐纳德·赫布(Donald Hebb)出版了《行为的组织》(The Organization of Behavior)一书,在书中他提出了通过神经元之间特定交互形成的理论神经结构。[13] 赫布关于神经元相互作用的模型为人工智能和机器学习算法在节点(或计算机用来传输数据的人工神经元)下如何工作奠定了基础。[12] 其他研究人类认知系统的学者也为现代机器学习技术做出了贡献,包括逻辑学家沃尔特·皮茨(Walter Pitts)和沃伦·麦卡洛克(Warren McCulloch),他们提出了早期的神经网络数学模型,旨在开发模拟人类思维过程的算法。[12]