% 可见光谱

可见光的波长范围在大约 400nm 到 700nm 之间. 各个波长对应的颜色如\autoref{VisSpt_fig1}, 图中我们进一步将可见光根据波长划分为红,橙,黄,绿,蓝,紫几个区间.

\begin{figure}[ht]
\centering
\includegraphics[width=14cm]{./figures/VisSpt_1.png}
\caption{电脑生成的标准可见光谱(图片来自维基百科)} \label{VisSpt_fig1}
\end{figure}

\subsection{人眼和显示器}
我们知道人眼可以识别不同的波长是因为我们视网膜上有三种不同的感光细胞, 它们分别主要感受红, 绿, 蓝三种波长的光. 而显示屏也根据人眼的这种特性对每个像素设置了红, 绿, 蓝三种颜色.

当电脑储存一张彩色图片时, 每个像素会被储存为 red, green, blue (或记为 RGB) 三个数, 每个数一般取 0 到 255 的整数, 或取 0 到 1 的实数.

\autoref{VisSpt_fig1} 中, RGB 三个分量的大小如

\begin{figure}[ht]
\centering
\includegraphics[width=14cm]{./figures/VisSpt_2.png}
\caption{RGB 三个分量} \label{VisSpt_fig2}
\end{figure}
