% 欧几里得矢量空间
% keys 欧几里得矢量空间

\pentry{正定二次型\upref{DeQua}}

\begin{definition}{欧几里得矢量空间}\label{EuVS_def1}
定义在实数域 $\mathbb R$ 上的矢量空间 $V$ ,若其带有一个对称的双线性型 $(\bvec x,\bvec y)\mapsto(\bvec x|\bvec y)$(\autoref{MulMap_def2}~\upref{MulMap}),且对应的二次型 $\bvec x\mapsto(\bvec x|\bvec x)$ 是正定的,则称空间 $V$ 是一个\textbf{欧几里得矢量空间}.
\end{definition}
通常,对称的双线性型 $(*|*)$ 在 $\bvec x,\bvec y$ 处的值称为它们的\textbf{纯量积}.我们用符号 $(*|*)$ 代替通常的 $f(\bvec x,\bvec y)$.这里我们不用 $(\bvec x,\bvec y)$ 和 $\langle\bvec x,\bvec y\rangle$ 代替纯量乘积,出于这样的考虑:已经有简单的矢量对 $(\bvec x,\bvec y)$ ,它是笛卡尔积 $V\times V$ 的元素(\autoref{Set_eq1}~\upref{Set});而 $\langle \bvec x,\bvec y\rangle $ 又是矢量 $\bvec x,\bvec y$ 生成的子空间(\autoref{VecSpn_def1}~\upref{VecSpn}).
再一次把纯量乘积的性质列出来:
\begin{enumerate}
\item \textbf{对称性:}$(\bvec x|\bvec y)=(\bvec y|\bvec x)$;
\item \textbf{线性:}$(\alpha\bvec x+\beta\bvec y|\bvec z)=\alpha(\bvec x|\bvec z)+\beta(\bvec y|\bvec z)$;
\item \textbf{正定性:}$(\bvec x|\bvec x)>0,\;\forall\bvec x\neq0((\bvec 0|\bvec x)=0)$.
\end{enumerate}
\begin{example}{}\label{EuVS_ex1}
次数 $\leq n-1$ 的多项式\upref{OnePol} (其域为实数域 $\mathbb R$)对通常的加法和数乘构成一个矢量空间 $V=P_n$.对任意两个矢量(多项式)$f,g\in P_n$ ,数 
\begin{equation}\label{EuVS_eq1}
(f|g)=\int_a^b f(x)g(x)\dd x
\end{equation}
给出 $P_n$ 上向量间的纯量乘积.此纯量乘积是用\autoref{EuVS_eq1} 在连续函数(区间 $[a,b]$ 上)的无穷维矢量空间 $C(a,b)$ 上给出的.相应的无穷维欧几里得空间则表示为 $C_2(a,b)$.
\end{example}
\begin{example}{}
欧几里得矢量空间 $V$ 的任一子空间 $U$ 本身也是欧几里得矢量空间,因为 $V$ 中纯量乘积在 $U$ 中的限制定义了双线性函数 $U\times U\rightarrow\mathbb R$,且保持纯量乘积的性质.特别的,域 $\mathbb R$ 本身可看成是个1维的实的矢量空间.
\end{example}
\begin{definition}{长度}\label{EuVS_def2}
在欧几里得矢量空间 $V$ 中,称非负实数
\begin{equation}
\abs{\abs{\bvec v}}=\sqrt{(\bvec v|\bvec v)}
\end{equation}
是任意矢量 $\bvec v\in V$ 的\textbf{长度}或 \textbf{模}.长度为1的矢量称为\textbf{标准的}.
\end{definition}
因为 $(\bvec v|\bvec v)\neq0$ ,所以任意矢量 $\bvec v$ 的长度都是完全确定的.

容易验证下面几个性质:
\begin{enumerate}
\item $\bvec v\neq0\Rightarrow \abs{\abs{\bvec v}}>0$.
\item $\abs{\abs{\lambda\bvec v}}=\abs{\lambda}\cdot\abs{\abs{\bvec v}}$
\end{enumerate}


\begin{example}{矢量的标准化}
任意矢量 $\bvec v$ 乘以它的长度的倒数$\frac{1}{\abs{\abs{\bvec v}}}$便可将其标准化,即
\begin{equation}
\abs{\abs{\frac{1}{\abs{\abs{\bvec v}}}\bvec v}}=1
\end{equation}
\end{example}
\begin{theorem}{柯西-布尼亚科夫斯基不等式}\label{EuVS_the1}
欧几里得向量空间中,对任意矢量 $\bvec x,\bvec y\in V$,成立不等式
\begin{equation}\label{EuVS_eq4}
\abs{(\bvec x|\bvec y)}\leq\abs{\abs{\bvec x}}\,\abs{\abs{\bvec y}}
\end{equation}
且等号仅在 $\bvec y=\lambda_0\bvec x,\;\lambda_0\in\mathbb R$ (即矢量共线)时成立.
\end{theorem}
\textbf{证明:}由纯量乘积的性质
\begin{equation}\label{EuVS_eq2}
\begin{aligned}
&(\lambda\bvec x-\bvec y|\lambda\bvec x-\bvec y)\\
&\underset{\text{线性}}{=}\lambda^2(\bvec x|\bvec x)-\lambda (\bvec x|\bvec y)-\lambda (\bvec y|\bvec x)+(\bvec y|\bvec y)\\
&\underset{\text{对称性}}{=}\lambda^2(\bvec x|\bvec x)-2\lambda (\bvec x|\bvec y)+(\bvec y|\bvec y)\\
&\underset{\text{正定性}}{\geq}0
\end{aligned}
\end{equation}
\autoref{EuVS_eq2} 最后一不等式可看成关于 $\lambda$ 的二次三项式,其判别式满足
\begin{equation}\label{EuVS_eq3}
(2(\bvec x|\bvec y))^2-4(\bvec x|\bvec x)(\bvec y|\bvec y)\leq0
\end{equation}
即
\begin{equation}
\begin{aligned}
(\bvec x|\bvec y)^2&\leq(\bvec x|\bvec x)(\bvec y|\bvec y)\\
&\Downarrow\\
\abs{(\bvec x|\bvec y)}&\leq\abs{\abs{\bvec x}}\,\abs{\abs{\bvec y}}
\end{aligned}
\end{equation}
当等号成立时,\autoref{EuVS_eq3} 应取等,即二次三项式仅有一根 $\lambda_0$.对照\autoref{EuVS_eq2} 有
\begin{equation}
(\lambda_0\bvec x-\bvec y|\lambda_0\bvec x-\bvec y)=0
\end{equation}
即 $\bvec y=\lambda_0\bvec x$.

\textbf{证毕!}
\begin{corollary}{三角不等式}
矢量 $\bvec x,\bvec y$ 与 $\bvec x+\bvec y$ 的长度满足不等式
\begin{equation}
\abs{\abs{\bvec x\pm\bvec y}}\leq\abs{\abs{\bvec x}}+\abs{\abs{\bvec y}}
\end{equation}
\end{corollary}
\textbf{证明:}由\autoref{EuVS_the1} 
\begin{equation}
\begin{aligned}
\abs{\abs{\bvec x\pm\bvec y}}^2&=\abs{\abs{\bvec x}}^2+\abs{\abs{\bvec y}}^2\pm2(\bvec x|\bvec y)\leq\abs{\abs{\bvec x}}^2+\abs{\abs{\bvec y}}^2+2(\bvec x|\bvec y)\\
&\leq \abs{\abs{\bvec x}}^2+\abs{\abs{\bvec y}}^2+2\abs{\abs{\bvec x}}\cdot\abs{\abs{\bvec y}}=(\abs{\abs{\bvec x}}+\abs{\abs{\bvec y}})^2
\end{aligned}
\end{equation}

\begin{example}{}
在空间 $C_2(a,b)$(\autoref{EuVS_ex1} ) 上,\autoref{EuVS_eq4} 就变为
\begin{equation}
\abs{\int_a^b f(x)g(x)\dd x}\leq\sqrt{\int_a^b f^2(x)\dd x}\cdot\sqrt{\int_a^b g^2(x)\dd x}
\end{equation}
\end{example}
\autoref{EuVS_the1} 意味着
\begin{equation}
-1\leq\frac{(\bvec x|\bvec y)}{\abs{\abs{\bvec x}}\,\abs{\abs{\bvec y}}}\leq1
\end{equation}
也就是说,存在一个角 $\varphi$ ,使得
\begin{equation}
\cos\varphi=\frac{(\bvec x|\bvec y)}{\abs{\abs{\bvec x}}\,\abs{\abs{\bvec y}}}
\end{equation}
\begin{definition}{夹角}\label{EuVS_def3}
欧几里得矢量空间中,由
\begin{equation}
\cos\varphi=\frac{(\bvec x|\bvec y)}{\abs{\abs{\bvec x}}\,\abs{\abs{\bvec y}}}
\end{equation}
确定的角 $\varphi$ 称之为矢量 $\bvec x$ 和 $\bvec y$ 的\textbf{夹角}. 
\end{definition}
\begin{definition}{正交}
如果矢量的夹角为 $\pi/2$,亦即 $(\bvec x|\bvec y)=0$,则称它们是\textbf{正交的},记作 $\bvec x\perp\bvec y$.
\end{definition}
\begin{theorem}{毕达哥拉斯定理}
\begin{equation}
\bvec x\perp\bvec y\Rightarrow\abs{\abs{\bvec x+\bvec y}}^2=\abs{\abs{\bvec x}}^2+\abs{\abs{\bvec y}}^2
\end{equation}
\end{theorem}
\begin{exercise}{}
试证明对两两正交的矢量 $\bvec v_1,\cdots,\bvec v_n$,成立
\begin{equation}
\abs{\abs{\bvec v_1+\cdots+\bvec v_n}}^2=\abs{\abs{\bvec v_1}}^2+\cdots+\abs{\abs{\bvec v_n}}^2
\end{equation}
\end{exercise}
\begin{example}{}
与给定矢量 $\bvec v$ 正交的所有矢量的集合是一个子空间,该空间称为 $\bvec v$ 的\textbf{正交补}.事实上,若
\begin{equation}
\bvec x\perp\bvec v,\quad\bvec y\perp\bvec v
\end{equation}
则
\begin{equation}
(\alpha\bvec x+\beta\bvec y|\bvec v)=\alpha(\bvec x|\bvec v)+\beta(\bvec y|\bvec v)=0,\quad \forall\alpha,\beta\in\mathbb R
\end{equation}
\end{example}
若矢量 $\bvec v$ 与子空间 $U$ 任意矢量都正交,则说 $\bvec v$ 正交于子空间 $U$,记作 $\bvec v\perp U$.
\begin{definition}{正交补}
设 $U$ 是 $V$ 的子空间,则集合
\begin{equation}
\{\bvec x|\bvec x\perp U\wedge\bvec x\in V\}
\end{equation}
也是一个子空间,称为 $U$ 的\textbf{正交补},记作 $U^{\perp}$ .
\end{definition}
\begin{definition}{正交基底}
称欧几里得空间 $V$ 的基底 $(\bvec e_1,\cdots,\bvec e_n)$ 是正交的,如果
\begin{equation}
(\bvec e_i|\bvec e_j)=0,\;i\neq j=1,\dots,n.
\end{equation}
若此外还有 $(\bvec e_i|\bvec e_i)=1,\;i=1,\cdots ,n$ ,则称此基底为\textbf{标准正交基底}.
\end{definition}
和所有的正定矩阵一样
\begin{exercise}{}
在标准正交基底下,矢量 $\bvec x$ 在基矢量 $\bvec e_i$ 上的坐标 $x_i$为
\begin{equation}
(\bvec x|\bvec e_i)=x_i
\end{equation}
\end{exercise}
\begin{definition}{投影}
称纯量乘积 $(\bvec x|\bvec e)$ 是矢量 $\bvec x$ 在直线 $\langle\bvec e\rangle_{\mathbb R}$ 上的\textbf{投影},其中 $\bvec e$ 是个长度为1的矢量.  
\end{definition}
如此,矢量 $\bvec x$ 在标准正交基底 $(\bvec e_1,\cdots,\bvec e_n)$ 下的坐标与 $\bvec x$ 在坐标轴 $\langle\bvec e_i\rangle_\mathbb{R}$ 上的投影一致. 