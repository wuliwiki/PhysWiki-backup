% 伊辛模型

\begin{issues}
\issueDraft
\issueMissDepend
\end{issues}

伊辛模型是用来描述铁磁现象的模型,其哈密顿量的形式为:

\begin{equation}
H=-J\sum\limits_{<i,j>}S_iS_j
\end{equation}

上式中的$J$为交换耦合常数$(J>0)$,$S_i$与$S_j$是两原子的自旋,求和中的$<i,j>$代表仅对最近邻的原子进行计算。

伊辛模型在一维和二维的情况下是可以严格求解的。

\subsection{一维伊辛模型求解}

构建一维伊辛模型,并附以周期性边界条件,在外加磁场$B$的情况下,模型的哈密顿量被写成:

\begin{equation}
H=-J\sum\limits_{i=1}^NS_iS_{i+1}-B\sum\limits_{i=1}^NS_i=-\sum\limits_{i=1}^N[JS_iS_{i+1}+B(S_i+S_{i+1})]
\end{equation}

上式中的$BS_i$项中应为$B\mu_i$,在此将玻尔磁子吸纳进$B$的定义中。

写出配分函数:

\begin{align}
Z&=\sum\limits_{S_1,S_2,...S_N}exp\{\sum\limits_{i=1}^N\beta[JS_iS_{i+1}+B(S_i+S_{i+1})]\} \\
&=\sum\limits_{S_1,S_2,...S_N}\prod\limits_{i=1}^N exp\{\beta[JS_iS_{i+1}+B(S_i+S_{i+1})]\}
\end{align}

考虑:

\begin{equation}
\bra{S_i}P\ket{S_j}=exp\{\beta[JS_iS_{j}+B(S_i+S_{j})]\}
\end{equation}

则有:

\begin{align}
Z&=\sum\limits_{S_1,S_2,...S_N}\prod\limits_{i=1}^N\bra{S_i}P\ket{S_i+1} \\
&=\sum\limits_{S_1,S_2,...S_N}\bra{S_1}P\ket{S_2} \bra{S_2}P\ket{S_3}\cdots\bra{S_{N-1}}P\ket{S_N} 
\end{align}

考虑:

\begin{equation}
\sum_S\ket{S}\bra{S}=\mathbb{1}
\end{equation}

则有:
\begin{equation}
Z=\sum\limits_{S_1}\bra{S_1}P^N\ket{S_1}=Tr(P^N)
\end{equation}

考虑P的矩阵元,由于$S\in\{-1,1\}$,有:

$$P=\begin{pmatrix}
1
\end{pmatrix}$$












\footnote{参考 Wikipedia \href{https://en.wikipedia.org/wiki/Ising_model}{相关页面}。}平均磁化率
\begin{equation}
\ev{M} = \frac{\sum_i M_i \E^{-M_i H \beta}}{\sum_i M_i \E^{-M_i H\beta}} = -kT \pdv{H} \ln Q
\end{equation}
平均能量
\begin{equation}
U = - \ev{M} H
\end{equation}
