% 欧拉方程(变分学)
% 欧拉方程|正规点

\pentry{变分的变换(变分学)\upref{VarCha},变分的基本定理(变分学)\upref{VarDef}}
由属于 $C_1$ 类的所有过给定点 $a,b$ 的曲线构成的可取曲线族\autoref{DesCur_sub1}~\upref{DesCur}中,给出泛函 
\begin{equation}
J(y)=\int_a^bF(x,y,y')\dd x
\end{equation}
的相对弱极值\autoref{AbPol_sub1}~\upref{AbPol}的可取曲线满足的方程便是\textbf{欧拉方程}.其具有下面的形式
\begin{equation}\label{ElueEV_eq1}
F'_y-\dv{}{x}F'_{y'}=0
\end{equation}
在引出欧拉方程的过程中,我们也将得到在变分的基本定理\upref{VarDef}一节中开头提到但未详细说明的结论,其可归为下面的定理.
\begin{theorem}{}
设 $F(x,y,y')$ 及其二阶偏微商对于 $a\leq x\leq b$ 及任意的 $y,y'$ 连续,泛函
\begin{equation}
J(y)=\int_a^b F(x,y,y')\dd x
\end{equation}
的可取曲线族由属于 $C_1$ 类的所有过点 $a,b$ 的曲线构成.若 $y=y(x)$ 给出 $J(y)$ 的相对弱极值,则函数 $y(x)$ 满足欧拉方程\autoref{ElueEV_eq1} .
并在 $F_{y'y'}\neq0$ 的一切 $x$ 值上, $y''(x)$ 存在且连续.

\end{theorem}
满足欧拉方程的曲线便称为\textbf{极值曲线}.
\subsection{证明}
可取曲线族过 $a,b$ 两点,表示任意的可取曲线族中的曲线 $y(x)$ 都满足 
\begin{equation}
\delta y(a)=\delta y(b)=0
\end{equation}
所以由\autoref{PolReq_the1}~\upref{PolReq}, $C_1$ 类的过 $a,b$ 的函数 $y(x)$ 给出泛函 $J(y)$ 极值的必要条件是\autoref{Varia_eq5}~\upref{Varia}
\begin{equation}\label{ElueEV_eq2}
\delta J=\int_a^b \left[F'_y(x,y,y')\delta y+F'_{y'}(x,y,y')\delta y' \right]  \,\mathrm{d}{x} =0
\end{equation}
由拉格朗日变换\autoref{VarCha_eq3}~\upref{VarCha},\autoref{ElueEV_eq2} 即
\begin{equation}
\delta J=\int_a^b \left(F'_y- \frac{\mathrm{d}{}}{\mathrm{d}{x}} F'_{y'} \right) \delta y \,\mathrm{d}{x} =0
\end{equation}
由\autoref{VarDef_the1}~\upref{VarDef},即得欧拉方程\autoref{ElueEV_eq1}.这样利用拉格朗日变换的和拉格朗日的变分基本定理得到的欧拉方程常被称为\textbf{欧拉-拉格朗日方程},甚至直接称为\textbf{拉格朗日方程}.

\textbf{但是},正如在拉格朗日变换\upref{VarCha}中提到的,对于$C_1$ 类的函数,拉格朗日变换不合法.所以,这是有缺陷的.

\textbf{幸运的是},对于 $C_1$ 类的函数,黎曼变换\autoref{VarCha_eq4}~\upref{VarCha}是合法的.即
\begin{equation}
\delta J=\int_a^b \left(F'_{y'}-N \right) \delta y' \,\mathrm{d}{x}=0  ,\quad where \;N=\int_a^xF'_y \,\mathrm{d}{x} 
\end{equation}
利用\autoref{VarDef_the2}~\upref{VarDef},有
\begin{equation}\label{ElueEV_eq3}
F'_{y'}-N=F_{y'}-\int_a^xF'_y\dd x=C
\end{equation}
其中, $C$ 是常数.这就是所谓的\textbf{欧拉方程的积分形式}.

因为 $N$ 可微,即 $N'=F'_y$,所以\autoref{ElueEV_eq3} ,$F'_{y'}$ 可微
\begin{equation}
\dv{}{x}F'_{y'}=N'=F'_y
\end{equation}
这便是欧拉方程的等价形式.

由多元函数的拉格朗日中值定理\autoref{MeanTh_the2}~\upref{MeanTh}
\begin{equation}
\delta F'_{y'}=\qty(\overline{F}_{xy'}+\frac{\Delta y}{\Delta x}\overline{F}_{yy'}+\frac{\Delta y'}{\Delta x}\overline{F}_{y'y'})\Delta x
\end{equation}
其中,$\overline{F}_{xy'},\overline{F}_{yy'},\overline{F}_{y'y'}$ 表示这些函数取
\begin{equation}
x+\theta_1\delta x,\quad y+\theta_2\delta y,\quad \quad y'+\theta_3\delta y',\quad (\abs{\theta_i}<1,i=1,2,3)
\end{equation}
时的值
