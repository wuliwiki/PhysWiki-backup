% 位力定理
位力定理是质点组力学在统计上的一个应用。在保守系下,该定理展示了“长时间”后系统的动能平均值及势能平均值的关系。作为牛顿力学的推论,位力定理可用于热力学中玻意耳定律的证明,可用于大尺度星系质量的估算。经典力学和量子力学的关系如此密切,你也很容易猜到,位力定理必然也会“出现”于量子力学中。(注,本文使用爱因斯坦求和约定,即$x_iy_i=\Sigma x_iy_i$)

设n个质点组成一质点系,我们有$r_i\cdot F_i=r_i\cdot p_i$。另设$G=r_i\cdot F_i$,由链式法则我们有:

\begin{equation}

\end{equation}

\begin{theorem}{位力定理}

\end{theorem}
