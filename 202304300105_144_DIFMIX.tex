% 简单的扩散与混合模拟器(MATLAB 绘图)

\subsection{混合模拟器}
本matlab/octave程序简易地模拟了容器中粒子的随机运动,直观地反映了统计力学中家喻户晓的基本问题:为什么微观上粒子的运动看似毫无规律,而宏观上物质的变化却体现出方向性?

本程序假设粒子的速度都是相同的,且粒子间没有相互作用。严格地说,这些简化的近似其实并不满足麦克斯韦分布与近独立系统假设。
\begin{lstlisting}[language=matlab]
clc
clear

global N;
global points;
global h
global dt;

function draw() %绘制粒子图像
  global N;
  global points;
  global h;

  hold on
  axis equal
  axis([-0.1 1.1 -0.1 1.1])

  for n = 1:N
    h(n)=scatter(points(1,n),points(2,n));
    set(h(n),'SizeData',16);
    if points(5,n)==1;
      set(h(n),'MarkerEdgeColor','r');
    else
      set(h(n),'MarkerEdgeColor','b');
    end
  end

  line([0 1], [0 0]) %绘制边框
  line([0 0], [0 1])
  line([0 1], [1 1])
  line([1 1], [0 1])

end

function update() %更新粒子图像
  global N;
  global points;
  global h;

  for n = 1:N
    set(h(n),'XData',points(1,n));
    set(h(n),'YData',points(2,n));
  end
end

function run() %更新粒子的位置与速度
  global N;
  global points;
  global dt

  for n = 1:N
    points(1,n) +=points(3,n)*dt; %位移=速度*时间
    points(2,n) +=points(4,n)*dt;

    if points(1,n) < 0 || points(1,n) > 1 %碰撞容器壁使速度反向
      points(3,n) = -points(3,n);
    end
    if points(2,n) < 0 || points(2,n) > 1
      points(4,n) = -points(4,n);
    end

  end
end


N=100;
points=zeros(5,N); %x,y,vx,vy,type
dt = 0.005;
 
for n = 1:N %随机创建粒子
  angle = 2*pi*rand(); 
  points(1,n)=rand();
  points(2,n)=rand();
  points(3,n)=cos(angle);
  points(4,n)=sin(angle);

  if points(2,n)>0.5 %将处于容器上方的粒子设为类型1,处于容器下方的粒子设为类型2.
    points(5,n)=1;
  else
    points(5,n)=2;
  end

end

draw();
n=1;

while true
  run();
  update();
  drawnow;
  pause(0.01);
end
\end{lstlisting}

\subsection{扩散模拟器}
将创建粒子的代码稍加修改,就可以模拟扩散现象:
\begin{lstlisting}[language=matlab]
points(2,n)= 0.5*rand() + 0.5; %所有粒子的y坐标都处于 (0.5, 1) 范围,即所有粒子初始时仅分布在上半容器。
\end{lstlisting}
