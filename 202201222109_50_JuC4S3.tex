% 两个特殊类型
% keys 特殊类型

本文授权转载自郝林的 《Julia 编程基础》. 原文链接:\href{https://github.com/hyper0x/JuliaBasics/blob/master/book/ch04.md}{第 4 章 类型系统}.


\subsubsection{4.3 两个特殊类型}

\textbf{4.3.1 Any 类型}

在Julia的类型图中,\verb|Any|是一个唯一的顶层类型.如果说超类型在上、子类型在下的话,那么它就处在类型图的最顶端.\verb|Any|类型是所有类型的直接或间接的超类型.也就是说,对于任意类型的变量\verb|x|,类型断言\verb|x::Any|都必定是成功的.

还记得吗?我们在前面定义第一个\verb|sum1|函数的时候,并没有为它的两个参数指定类型.然而,在这种情况下,这两个参数实际上都会有一个缺省的类型,即:\verb|Any|类型.这也是为什么我们可以用任何类型的值作为参数值调用这个\verb|sum1|函数的原因.

再比如,我们可以定义如下的原语类型(我们稍后会讲到这种类型):

\begin{lstlisting}[language=julia]
julia> primitive type MyWord 64 end

julia> 
\end{lstlisting}

注意,我们没有显式地指定它的超类型.然而,在这种情况下,\verb|MyWord|类型会有一个缺省的超类型,同样是\verb|Any|类型.也就是说,这个\verb|MyWord|类型是\verb|Any|类型的直接子类型.

更宽泛地讲,\verb|Any|类型会在很多情况下担当默认类型并发挥其作用.我们在后面还会遇到类似的情形.另外,\verb|Any|类型是一个抽象类型.因此它本身是不能被实例化的.但所有的值却都是它的实例.

\textbf{4.3.2 Union{} 类型}

在 Julia 的类型图中,还有一个与\verb|Any|完全相对的类型.它就是\verb|Union{}|类型.由于这个类型是所有类型的子类型,所以它是一个底层类型,并且也是唯一的一个.它处在类型图的最底端.也就是说,对于任意类型的变量\verb|x|,类型断言\verb|x::Union{}|都必定是失败的.另外,与\verb|Any|一样,\verb|Union{}|也是一个抽象类型.

从字面上我们就可以看出,\verb|Union{}|是一个被参数化的类型.它的源类型是\verb|Union{Types...}|类型,其中的\verb|Types...|代表任意个类型参数.如果这里有多个类型参数,那么它们之间需要用英文逗号分隔开.

这个\verb|Union{Types...}|类型有着一种很特殊的用途.我们可以利用它,让一个单一的类型字面量代表多个类型.换句话说,把多个类型联合在一起形成一个类型,并让后者作为前者的统一代表.因此,我们也可以把这个类型称为联合类型.而每一个类型参数的组合都可以代表一种联合类型.示例如下:

\begin{lstlisting}[language=julia]
julia> IntOrString = Union{Integer, AbstractString} 
Union{AbstractString, Integer}

julia> 2020::IntOrString
2020

julia> "2020"::IntOrString
"2020"

julia> 
\end{lstlisting}

类型\verb|Union{Integer, AbstractString}|表示的是\verb|Integer|类型和\verb|AbstractString|类型的联合.因此,任何\verb|Integer|类型或\verb|AbstractString|类型的实例都可以被视为这个联合类型的实例.这就是类型断言\verb|2020::IntOrString|和\verb|"2020"::IntOrString|可以成功的原因.

另外,由于Julia中的类型属于一类特殊的值(\verb|DataType|类型的值),所以上述的联合类型自然也就可以与标识符\verb|IntOrString|绑定在一起.这时,我们可以说\verb|IntOrString|是那个联合类型的别名(alias).

搞清楚了联合类型以及它的用途,我们就很容易理解“\verb|Union{}|类型处在类型图的最底端”的原因了.由于它的花括号中没有任何类型参数,所以这种联合类型也就代表不了任何类型,相当于一个“虚无”的类型.而任何类型都比“虚无”包含了更多的东西,所以它们都是这种联合类型的超类型.如果我们使用操作符\verb|<:|在这些类型之间做判断的话,就可以很形象地看到这种关系:

```julia
julia> Union{} <: Integer
true

julia> Union{} <: Union{Integer}
true

julia> 
```

此示例中的两个表达式的结果值都是`true`.这说明整数类型`Integer`和联合类型`Union{Integer}`都是“虚无”类型`Union{}`的超类型.

至此,我们已经较为充分地了解了 Julia 类型图中的两端,即:最顶端的`Any`和最底端的`Union{}`.下面,我们一起来看看在它们之间的类型都有哪些.