% 完整群(综述)
% license CCBYSA3
% type Wiki

本文根据 CC-BY-SA 协议转载翻译自维基百科\href{https://en.wikipedia.org/wiki/Holonomy}{相关文章}。

\begin{figure}[ht]
\centering
\includegraphics[width=6cm]{./figures/8ee31b3dfe98af26.png}
\caption{球面上沿分段光滑路径的平行移动。初始向量标记为 \(V\),它沿着曲线被平行移动,最终得到的向量标记为 \( \mathcal{P}_\gamma (V) \)。如果路径发生变化,平行移动的结果也会不同。} \label{fig_WZQ_1}
\end{figure}
在微分几何中,光滑流形上一个联络的\textbf{平行迁移群}(holonomy)描述的是:沿着闭合回路进行平行移动时,几何数据未被保持的程度。平行迁移群是联络曲率所导致的一种普遍几何效应。对于平坦联络,相关的平行迁移群是一种单值延拓(monodromy),并且本质上是一个全局概念。而对于曲率非零的联络,平行迁移群同时具有非平凡的局部和全局特征。

任何流形上的联络都会通过其平行移动映射引出某种\textbf{平行迁移群}(holonomy)的概念。最常见的平行迁移群形式是具有某种对称性的联络。重要的例子包括:黎曼几何中Levi-Civita联络的平行迁移群(称为黎曼平行迁移群),向量丛上联络的平行迁移群,Cartan联络的平行迁移群,以及主丛上联络的平行迁移群。在这些情形下,联络的平行迁移群都可以与某个李群(即\textbf{平行迁移群})对应起来。根据Ambrose-Singer定理,联络的平行迁移群与该联络的曲率密切相关。

对\textbf{黎曼平行迁移群}(Riemannian holonomy)的研究推动了许多重要的发展。\textbf{平行迁移群}最早由\textbf{埃利·嘉当}(Élie Cartan)于1926年引入,用于研究和分类对称空间。然而,直到很久之后,平行迁移群才被用来在更一般的背景下研究黎曼几何。

1952年,\textbf{乔治·德拉姆}(Georges de Rham)证明了\textbf{德拉姆分解定理},该定理通过将切丛分解为在局部平行迁移群作用下的不可约子空间,从而将黎曼流形分解为多个黎曼流形的笛卡尔积。随后在1953年,\textbf{马塞尔·贝尔热}(Marcel Berger)对可能出现的不可约平行迁移群进行了分类。

黎曼平行迁移群的分解和分类在物理学和弦理论中都有应用。
\subsection{定义} 
\subsubsection{向量丛上联络的平行迁移群}  
设\(E\)是光滑流形\(M\)上的一个秩为\(k\)的向量丛,\(\nabla\)是\(E\)上的一个联络。对于一个以\(M\)中点 \(x\) 为基点的分段光滑闭合路径\(\gamma : [0,1] \to M\),联络\(\nabla\)定义了一个\textbf{平行移动映射}\(P_\gamma : E_x \to E_x\)它作用在 \(E\)在点\(x\)处的纤维上。这个映射是线性的且可逆的,因此它定义了一个属于一般线性群\(GL(E_x)\)的元素。联络\(\nabla\)在基点\(x\)处的\textbf{平行迁移群}(holonomy group)定义为:
\[
\operatorname{Hol}_x(\nabla) = \{P_\gamma \in \mathrm{GL}(E_x) \mid \gamma \text{ 是以 } x \text{ 为基点的闭合路径}\}.~
\]
其中\(P_\gamma\) 是路径 \(\gamma\)对应的平行移动映射。基点\(x\)处的\textbf{限制平行迁移群}(restricted holonomy group)是子群:\(\operatorname{Hol}_x^0(\nabla)\)它由\textbf{可缩至点的闭合路径} \(\gamma\)所对应的平行移动映射组成。

如果 \(M\) 是连通的,那么平行迁移群(holonomy group)对于基点 \(x\) 的依赖仅体现在一般线性群 \(GL(k, \operatorname{R})\) 中的共轭关系上。具体来说,如果 \(\gamma\) 是从 \(x\) 到 \(y\) 的一条路径,那么:
\[
\operatorname{Hol}_y(\nabla) = P_\gamma \operatorname{Hol}_x(\nabla) P_\gamma^{-1}.~
\]
此外,选择不同的方式将纤维 \(E_x\) 与 \(\operatorname{R}^k\) 进行同构,也会得到共轭的子群。  

因此,有时,特别是在一般性的或非正式的讨论中(比如下面的内容),人们会省略对基点的引用,此时默认该定义仅确定到共轭的意义下。