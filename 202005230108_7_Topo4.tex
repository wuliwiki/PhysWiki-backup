% 道路连通性
\pentry{连续映射和同胚\upref{Topo1},连通性\upref{Topo3}}
连通性的概念还不够直观,因此我们还有一个更强的性质,道路连通性.连通性的定义思路是用不相交的开集来分割不连通的各部分,而道路连通性的定义思路是用一条连续的曲线来连接任意两点,如果不连通则这条曲线应该会被截断.

\subsection{道路连通性的概念}
\begin{definition}{道路}
在区间$I=[0,1]$上取通常的度量(子)拓扑.对于任意的拓扑空间$X$,如果存在一个连续映射$f: I\rightarrow X$,那么称该映射的像$f(I)$为$X$中的一条\textbf{道路(path)}.
\end{definition}

道路的定义要更直观一些,而且在别的数学分支也会出现,比如复变函数中会经常讨论复平面上的道路.需要注意的是,这里定义道路所用的区间是闭区间,也就是说它有起点和终点.可以把$I$看成是从$0$到$1$流逝的时间,而对应的$f(I)$就是这段时间里,一个点在$X$中连续运动的轨迹.这个点运动的“速度”不会是无穷大,否则就不是连续映射了;这样,在有限时间内,这个运动轨迹必然也是“有限”的.这里打双引号是因为“速度”和“有限”在一般情况下只是一个类比,只有在度量空间里面我们才可以讨论长度,进而有了“速度”的概念.

\begin{example}{道路的例子}
\begin{itemize}
\item 在度量空间$\mathbb{R}$中,$I$本身就是一条道路.任何闭区间都是一条道路.
\item 在$\sin{\frac{1}{x}}$\footnote{见\autoref{Topo3_ex3}.}空间中,对于任何$a, b>0$,从$x=a$到$x=b$的一段函数图像也构成道路
\item 还是在$\sin{\frac{1}{x}}=S$空间中,不过考虑的是$\bar{S}=S\cup\{(0, y)|y\in [-1,1]\}$.

\end{itemize}
\end{example}