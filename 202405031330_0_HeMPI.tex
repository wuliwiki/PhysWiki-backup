% 氦原子 TDSE MPI 演化笔记
% license Usr
% type Note

类似分块矩阵乘法,但其实有一个元素重叠。
\begin{figure}[ht]
\centering
\includegraphics[width=10cm]{./figures/f0baffec02af1adc.png}
\caption{径向 prop} \label{fig_HeMPI_1}
\end{figure}
另外其实不是一个 node 一个 block, 而是一个 node 多个 block。所有的相邻边界都要在两个 node 重复储存,且做完矩阵乘法后相加并同步(来回传递各一次)。

可以用 block 组成任意形状而不是规则形状。每个 node 有一个整数编号,需要一个描述文件来描述每个 node 的上下左右是谁。注意每个 block 不一定需要大小相同,只需要相邻边长匹配即可。

需要做的修改: 所有的 \verb`obd` 矩阵需要切割(\verb`H`, \verb`D`, \verb`D2` 等),每个 node 要有属于自己的两个径向 obd 矩阵。 我们可以规定 obd 矩阵每个 block 的右下角的矩阵元为零,除了最后一个。这样把边界相加时就不会重复计算。波函数 \verb`Psi` 显然也需要对每个 node 切割,边界处重复储存。

以上就是 MPI 算法中数据结构需要进行的所有修正!
