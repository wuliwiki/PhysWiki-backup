% 狄拉克方程的非相对论近似
% 狄拉克方程|泡利方程

\pentry{狄拉克方程\upref{qed4}}

词条狄拉克方程\upref{qed4}中给出了描述自由电子的方程\autoref{qed4_eq9}~\upref{qed4}):(这里我们没有采用自然但位置,所以需要带上 $\hbar,c$)
\begin{equation}
\begin{aligned}
&i\hbar \pdv{t} \psi = H\psi\\
&H=c\bvec \alpha\cdot \bvec p+mc^2\beta
\end{aligned}
\end{equation}
其中 $\bvec \alpha,\beta$ 是四维矩阵代数中的元素,满足一定的反对易关系.更常见地,上式也可以写成\autoref{qed4_eq22}~\upref{qed4} 的形式:
\begin{equation}
i\qty(\gamma^\mu \partial_\mu-\frac{mc}{\hbar})\psi(x)=0
\end{equation}
其中 $\partial_\mu=\qty(\pdv{(ct)},\pdv{x},\pdv{y},\pdv{z})$.

下面我们将从狄拉克方程出发,得到它的非相对论近似.设平面波解
\begin{equation}
\psi=\pmat{\varphi\\\chi}\exp(-imc^2t/\hbar)
\end{equation}
将它代入狄拉克方程.这里不妨采用 $\bvec \alpha,\beta$ 的标准表示\autoref{qed4_eq6}~\upref{qed4},则有
\begin{equation}
\begin{aligned}
i\hbar\pdv{t} \varphi=c\bvec \alpha\cdot \bvec p \chi\\
i\hbr
\end{aligned}
\end{equation}
