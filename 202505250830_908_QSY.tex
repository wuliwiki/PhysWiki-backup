% 乔赛亚·威拉德·吉布斯(综述)
% license CCBYSA3
% type Wiki

本文根据 CC-BY-SA 协议转载翻译自维基百科\href{https://en.wikipedia.org/wiki/Josiah_Willard_Gibbs}{相关文章}。

约西亚·威拉德·吉布斯(Josiah Willard Gibbs,/ɡɪbz/,1839年2月11日-1903年4月28日)是一位美国机械工程师和科学家,在物理学、化学和数学领域做出了基础性的理论贡献。他关于热力学应用的研究在将物理化学转变为一门严谨的演绎科学方面起到了关键作用。吉布斯与詹姆斯·克拉克·麦克斯韦和路德维希·玻尔兹曼一起创立了统计力学(该术语由他提出),将热力学定律解释为由大量粒子组成的物理系统可能状态集合的统计特性所导致的结果。吉布斯还研究了麦克斯韦方程在物理光学问题中的应用。作为数学家,他独立于英国科学家奥利弗·赫维赛德(后者在同一时期进行了类似的工作)创建了现代向量分析,并在傅里叶分析理论中描述了“吉布斯现象”。

1863年,耶鲁大学授予吉布斯美国首个工程学博士学位。在欧洲度过三年后,吉布斯将余生的职业生涯都奉献给了耶鲁大学,自1871年起担任数学物理学教授,直到1903年去世。他在相对孤立的环境中工作,成为美国最早获得国际声誉的理论科学家之一,曾被阿尔伯特·爱因斯坦称为“美国历史上最伟大的头脑”。

1901年,吉布斯因其在数学物理方面的贡献,获得了当时国际科学界最高荣誉——由伦敦皇家学会颁发的科普利奖章。

评论家和传记作家们曾指出,吉布斯宁静而孤独的新英格兰生活方式与其思想在国际上的巨大影响之间形成了鲜明对比。尽管他的研究几乎完全是理论性的,但随着20世纪上半叶工业化学的发展,吉布斯成果的实际价值逐渐显现。正如罗伯特·A·密立根所言,在纯科学领域,吉布斯“对于统计力学和热力学的贡献,就如拉普拉斯之于天体力学,麦克斯韦之于电动力学——他几乎将这个领域构建成一个完整的理论体系”。
\subsection{传记}
\subsubsection{家庭背景}
吉布斯出生于康涅狄格州纽黑文。他出身于一个古老的“洋基”家族,自17世纪以来,这个家族便不断涌现出杰出的美国牧师和学者。他是五个孩子中排行第四的孩子,也是父亲约西亚·威拉德·吉布斯与母亲玛丽·安娜(娘家姓范·克里夫,Mary Anna, née Van Cleve)唯一的儿子。在父系方面,他是塞缪尔·威拉德的后代,后者曾于1701年至1707年间担任哈佛学院代理校长。在母系方面,他的一位祖先是乔纳森·迪金森牧师,即新泽西学院(后来的普林斯顿大学)首任校长。“约西亚·威拉德”这个名字在他家族中代代相传,不仅他与父亲重名,家族中还有数位成员也叫这个名字,该名源自他的一位祖先——18世纪担任马萨诸塞湾省国务秘书的约西亚·威拉德。他的祖母默西·普雷斯科特·吉布斯(Mercy (Prescott) Gibbs)是丽贝卡·米诺特·普雷斯科特·舍曼的妹妹,而后者是美国开国元勋罗杰·舍曼的妻子。因此,吉布斯是这位开国元勋的第二代近亲。
