% Jupyter Notebook 和 IPython 笔记

\begin{issues}
\issueDraft
\end{issues}

\begin{itemize}
\item 如果你在 Jupyter Notebook 里面使用 Python, 那么实际上使用的不是 python REPL\footnote{python REPL 即在 linux 命令行输入 \verb|python| 或者 \verb|python3| 出现的界面。} 而是 IPython。 后者完全兼容前者, 但有更多功能。
\item Jupyter Notebook 前身叫做 IPython Notebook 仅支持 IPython, 但改名 Jupyter 以后, 还可以支持其他语言如 Julia, R, C++ 等。
\item 以下介绍 IPython 特有的而 python REPL 没有的功能。
\item Bash 中的 \verb|cd, ls, pwd, mkdir, rmdir, cat| 等命令可以直接使用。 但这不会污染命名空间, 如果对这些名字赋值或者用作函数名就会覆盖 bash 命令。
\item \verb|!命令 参数1 参数2| 可以直接用 Bash 执行命令而不是 IPython。 例如 IPython 并不支持 \verb|touch| 或者 \verb|echo|。 所以为了一致起见还是在 bash 命令前面都加上 \verb|!| 比较好。
\item \verb|变量?| 可以查看该变量的信息, 例如类型, 成员等
\item 用 Tab 键可以自动补全(可以方便查看对象的成员)。
\item \verb|%run 路径/文件名.py| 可以运行指定脚本。
\end{itemize}
