% 分裂域
% splitting field|正规扩张|regular extension


\pentry{域的扩张\upref{FldExp}}


本节我们要介绍一个在代数中非常基础且重要的概念:分裂域.简单来说,分裂域就是在一个域中添加某个多项式的全体根所得到的扩域.从分裂域出发,我们可以讨论代数扩域的自同构问题.




\subsection{分裂域的存在性}


\begin{definition}{分裂域}
给定域$\mathbb{F}$及其上一个多项式$f(x)$.若存在扩域$\mathbb{K}/\mathbb{F}$,使得$f(x)$在$\mathbb{K}$上可以分解为$f(x)=\prod_{i=1}^n (x-a_i)$,且$\mathbb{K}=\mathbb{F}(a_1, a_2, \cdots, a_n)$,则称$\mathbb{K}$是$f(x)\in \mathbb{F}[x]$上的\textbf{分裂域(splitting field)}.
\end{definition}

定义看起来有些绕口,先说$f$在$\mathbb{K}$中可以分解,也就是说每一个根都存在,再说$\mathbb{K}$可以看成用这些根对$\mathbb{F}$进行扩域的结果.这么定义是因为我们要先确定元素$a_i$都存在,而为此就需要先确定$\mathbb{K}$存在.但是定义中只说了“若$\mathbb{K}$存在”,这个假设到底成立与否呢?答案是肯定的.

\begin{theorem}{分裂域的存在性}
给定域$\mathbb{F}$及其上一个多项式$f(x)$,则$f(x)\in \mathbb{F}[x]$上的分裂域存在.
\end{theorem}

\textbf{证明}:

当$\opn{deg}f=1$时,定理自然成立,此时$f\in\mathbb{F}[x]$的分裂域就是其本身.

首先在环$\mathbb{F}[x]$上对元素$f(x)$进行因式分解\footnote{也就是画出它的一棵\textbf{真因子树}\upref{FctTre}.},得到其不可约因子.任选其中一个不可约因子$h(x)$,如果$\opn{deg} h = 1$,则跳过本段接下来的步骤.构造环$R_1=\mathbb{F}(x)/<h(x)>=\mathbb{F}[a_1]$,再取其分式域$\mathbb{F}_1$,则$\mathbb{F}_1$就是$\mathbb{F}$的单扩张$\mathbb{F}(a_1)$.

由\textbf{多项式环}\upref{RPlynm}的\autoref{RPlynm_the1}~\upref{RPlynm},$(x-a_1)|h(x)$,因此在$\mathbb{F}_1$上可以分解出$h_1(x)=h(x)/(x-a_1)$.如果$\opn{deg}h_1 = 1$,则跳过本段接下来的步骤.对$h_1(x)$进行相同的操作:构造环$R_2=\mathbb{F}(x)/<h_1(x)>=\mathbb{F}[a_2]$,再取其分式域$\mathbb{F}_2=\mathbb{F}_1(a_2)=\mathbb{F}(a_1, a_2)$.

以此类推,直到$h(x)$在$\mathbb{F}_{k_1}$上分解为一阶多项式之积.

接下来,取$f$在$\mathbb{F}_{k_1}$上的不可约因子$g(x)$,如果$\opn{deg} g = 1$,则跳过本段接下来的步骤.执行相同的扩域操作,直到得到$\mathbb{F}_{k_1+k_2}$,使得$g$在$\mathbb{F}_{k_1+k_2}$上分解为一阶多项式之积.

以此类推,最终可以得到$\mathbb{F}_k$,使得$f$在$\mathbb{F}_k$上可以分解为一阶多项式之积.则$\mathbb{F}_k$就是$f\in\mathbb{F}[x]$的分裂域.

\textbf{证毕}.

该证明过程的大体思路,就是看$f$的根是否在已知的域中.根$a$的最小多项式$h(x)$必是$f(x)$的一个不可约因子.如果$a$在已知的域中,那么$f$就可以因式分解出一阶多项式因子$(x-a)$;否则,就添加$a$进行一次单扩域,这次扩域至少能把$a$纳入,但也有可能把其它根一起纳入.因此我们容易得到以下推论:

\begin{corollary}{}
设$\mathbb{K}$是$f(x)\in \mathbb{F}[x]$上的分裂域,则$[\mathbb{K}:\mathbb{F}]\leq \opn{deg}f$.
\end{corollary}

\begin{corollary}{}
设$\mathbb{K}$是$f(x)\in \mathbb{F}[x]$上的分裂域,$\mathbb{M}$是$\mathbb{K}$和$\mathbb{F}$之间的\textbf{中间域},则$\mathbb{K}$也是$f(x)\in \mathbb{E}[x]$上的分裂域.
\end{corollary}

为了加深理解,我们讨论一个分裂域的例子.添加元素的过程中会遇到的主要情况在这里都出现了.

\begin{example}{分裂域的一个例子}


在有理数域$\mathbb{Q}$上有多项式$f(x)=(x^2-2)^2(x^2-3)(x^2-6)(x^2+1)$,其在$\mathbb{Q}$上有五个阶数大于$1$的不可约因子:$(x^2-2), (x^2-2), (x^2-3), (x^2-6), (x^2+1)$.

考虑因子$(x^2-2)$,得到扩域$\mathbb{Q}(\sqrt{2})$.在$\mathbb{Q}(\sqrt{2})$上,$f$有分解:
\begin{equation}
f(x)=(x+\sqrt{2})^2(x-\sqrt{2})^2(x^2-3)(x^2-6)(x^2+1)
\end{equation}

取其阶数大于$1$的不可约因子$x^2-3$,得到扩域$\mathbb{Q}(\sqrt{2}, \sqrt{3})$.

在$\mathbb{Q}(\sqrt{2}, \sqrt{3})$上,$f$有分解:

\begin{equation}
\begin{aligned}
f(x)=&(x+\sqrt{2})^2(x-\sqrt{2})^2(x+\sqrt{3})(x-\sqrt{3})\times\\
&(x+\sqrt{2}\sqrt{3})(x-\sqrt{2}\sqrt{3})(x^2+1)
\end{aligned}
\end{equation}

取其阶数大于$1$的不可约因子$x^2+1$,得到扩域$\mathbb{Q}(\sqrt{2}, \sqrt{3}, \I)$.

你可以验证,在最后这个扩域下,$f$可分解为一阶多项式之积.因此
\begin{equation}
\begin{aligned}
&\mathbb{Q}(\sqrt{2}, \sqrt{3}, \I)=\\
&\{a+A\I+(b+B\I)\sqrt{2}+(c+C\I)\sqrt{3}\\
&+(d+D\I)\sqrt{6} | a, A, b, B, c, C, d, D\in\mathbb{Q}\}
\end{aligned}
\end{equation}
就是$f\in\mathbb{Q}[x]$的分裂域.

\end{example}

\begin{corollary}{}
设$\mathbb{K}$是$f(x)\in \mathbb{F}[x]$上的分裂域,则$[\mathbb{K}:\mathbb{F}]\leq \opn{deg}f$.
\end{corollary}

\begin{corollary}{}
设$\mathbb{K}$是$f(x)\in \mathbb{F}[x]$上的分裂域,$\mathbb{M}$是$\mathbb{K}$和$\mathbb{F}$之间的\textbf{中间域},则$\mathbb{K}$也是$f(x)\in \mathbb{E}[x]$上的分裂域.

\end{corollary}


分裂域的构造也可以这样理解:给定一个域$\mathbb{F}$和其上一个\textbf{不可约}多项式$f$,则$f\in\mathbb{F}[x]$的分裂域就是$\mathbb{F}[x]/<f(x)>$的\textbf{分式域}\upref{FrcFld}.对于任意的$f\in\mathbb{F}[x]$,如果它能写为若干不可约多项式之积,如$f=h_1h_2\cdots h_n$,那么$f$的分裂域是:求$h_1\in\mathbb{F}[x]$的分裂域$\mathbb{F}_1$,再求$h_2\in\mathbb{F}_1[x]$的分裂域$\mathbb{F}_2$,以此类推,$\mathbb{F}_n$即为$f\in\mathbb{F}[x]$的分裂域.










\subsection{分裂域的唯一性}

从\textbf{开拓}(\autoref{FldExp_def6}~\upref{FldExp})的角度来说,如果存在域同构$\sigma:\mathbb{F}_1\to\mathbb{F}_2$,将其开拓为环同构$\sigma:\mathbb{F}_1[x]\to\mathbb{F}_2[x]$,任取$f\in\mathbb{F}_1[x]$,设$f\in\mathbb{F}_1[x]$的分裂域为$\mathbb{K}_1$,$\sigma(f)\in\mathbb{F}_2[x]$的分裂域为$\mathbb{K}_2$,则$\sigma$可以开拓为$\mathbb{K}_1\to\mathbb{K}_2$的同构.

上述开拓的角度或许有些绕,但考虑到“同构的域就是同一个域”,我们完全可以大大简化上述表达:

\begin{theorem}{}
给定域$\mathbb{F}$和其上一个多项式$f$以后,所构造出来的分裂域是唯一的,或者说构造出来的两个分裂域都是同构的.
\end{theorem}

证明非常简单,只需要利用多项式环和分式域的唯一性即可.多项式环的唯一性依赖于\autoref{RPlynm_the2}~\upref{RPlynm},分式域的唯一性由\autoref{FrcFld_the1}~\upref{FrcFld}得到.

再回到开拓的角度来思考:上面只说了$\sigma$可以开拓为$\mathbb{K}_1\to\mathbb{K}_2$的同构,那这种同构是不是唯一的呢?一般情况下不是的.继续用“同构就是同一个”的思维,我们有如下定理:

\begin{theorem}{}\label{SpltFd_the1}
给定域$\mathbb{F}$和其上一个多项式$f$,设$f\in\mathbb{F}[x]$的分裂域是$\mathbb{K}$,$\mathbb{K}$到自身的保$\mathbb{F}$自同构数量为$N$,那么$N\leq[\mathbb{K}:\mathbb{F}]$.

当且仅当$f$的每一个不可约因子$h$的不同根数目恰为$\opn{deg}h$时,等号成立.
\end{theorem}

\textbf{证明}:

定理中第二段的描述已经暗示了证明思路.

设$\sigma:\mathbb{K}\to\mathbb{K}$是自同构,且对于任意$a\in\mathbb{F}$有$\sigma(a)=a$.

对于任意$g\in\mathbb{F}[x]$和任意$k\in\mathbb{K}$,必有$\sigma(g(k))=g(\sigma(k))$.因此$k$和$\sigma(k)$的最小多项式相同.因此,$f$的任意不可约因子$h$的根,必须被$\sigma$映射为$h$的根.

\addTODO{没写完证明}




\textbf{证毕}.

\begin{example}{}

给定有理数域$\mathbb{Q}$及其上的多项式$f(x)=x^2-2$.则$f\in\mathbb{Q}[x]$的分裂域为$\mathbb{Q}(\sqrt{2})=\{a+b\sqrt{2}|a, b\in\mathbb{Q}\}$.

$\mathbb{Q}(\sqrt{2})$一共有两个保$\mathbb{Q}$自同构:第一个就是恒等映射,第二个$\sigma$则定义如下:
\begin{equation}
\sigma(a+b\sqrt{2})=a-b\sqrt{2}
\end{equation}
也就是说,$\sigma$把根$\pm\sqrt{2}$映射到根$\mp\sqrt{2}$.


而$[\mathbb{Q}(\sqrt{2}):\mathbb{Q}]=2$,因此这是一个\autoref{SpltFd_the1} 取等号的例子.

\end{example}


\begin{example}{}
给定实数域$\mathbb{R}$及其上的多项式$f(x)=x^2+x+1$,则$f\in\mathbb{R}[x]$的分裂域为$\mathbb{R}(\omega)=\{a+b\omega+c\omega^2 \mid a, b, c\in\mathbb{R}\}$,其中$\omega$是$1$的三次单位根$\frac{1}{2}(-1+\I\sqrt{3})$.

$\mathbb{R}(\omega)$一共有两个保$\mathbb{R}$自同构:第一个是恒等映射;第二个则是将$\omega$映射到$\omega^2$、$\omega^2$映射到$\omega$的映射.

$[\mathbb{R}(\omega):\mathbb{R}]=2$.这可以从\autoref{FldExp_the1}~\upref{FldExp}得到,也可以验证$\omega^2=-1-\omega$得到.
\end{example}





\begin{exercise}{}
求$x^3-2\in\sqrt{Q}[x]$的分裂域及其所有保$\mathbb{Q}$自同构.
\end{exercise}



\begin{exercise}{}
求$x^p-1\in\mathbb{Q}[x]$的分裂域及其所有保$\mathbb{Q}$自同构.这里$p$为素数.注意判断$x^p-1$是否为不可约多项式.
\end{exercise}


\begin{exercise}{}
求$x^6-1\in\mathbb{Q}[x]$的分裂域及其所有保$\mathbb{Q}$自同构.
\end{exercise}


\begin{corollary}{}\label{SpltFd_cor1}
设$\mathbb{K}$是多项式$f\in\mathbb{F}[x]$的分裂域,$\mathbb{E}$是$\mathbb{K}$的扩域.

则对于$\mathbb{E}$的任意保$\mathbb{F}$自同构$\sigma$,有$\sigma(\mathbb{K})=\mathbb{K}$.
\end{corollary}

\textbf{证明}:

设$f=(x-a_1)(x-a_2)\cdots(x-a_n)$,则$\mathbb{K}=\mathbb{F}(a_1, a_2, \cdots, a_n)$.

类比\autoref{SpltFd_the1} 证明中的讨论,$\sigma$必将$f$的根映射为另一根,也就是对$f$的根的置换.因此
\begin{equation}
\begin{aligned}
\sigma(\mathbb{K})&=\mathbb{F}(\sigma(a_1), \sigma(a_2), \cdots, \sigma(a_n))\\
&=\mathbb{F}(a_1, a_2, \cdots, a_n)\\
&=\mathbb{K}
\end{aligned}
\end{equation}

\textbf{证毕}.





\subsection{正规扩张}

分裂域的性质,其实对应的是一种非常重要的域扩张,它与代数方程的根式解问题息息相关.

\begin{definition}{正规扩张}
设$\mathbb{K}/\mathbb{F}$是一个\textbf{有限}扩域.如果对于$\mathbb{F}$上的任意不可约多项式$f$,要么$f$在$\mathbb{K}$中无根,要么就所有根都在$\mathbb{K}$中,则称$\mathbb{K}/\mathbb{F}$是一个\textbf{正规扩张}.
\end{definition}

实际上,正规扩张和分裂域是等价的概念,尽管它们表述差异很大.或者换句话说,分裂域的一个重要性质,就是正规性.


\begin{theorem}{正规扩张等价于分裂域}\label{SpltFd_the2}
设$\mathbb{K}/\mathbb{F}$是一个\textbf{有限}扩域,那么有:

$\mathbb{K}/\mathbb{F}$为正规扩张$\iff$ $\mathbb{K}$是某个多项式$f\in\mathbb{F}[x]$的分裂域.
\end{theorem}

\textbf{证明}:

$\Leftarrow$:

设$\mathbb{K}$是多项式$f\in\mathbb{F}[x]$的分裂域.取不可约的$h(x)\in\mathbb{F}[x]$且$\exists a\in\mathbb{K}$使得$h(a)=0$.我们要证明$h$的根都在$\mathbb{K}$中.


设$h\in\mathbb{K}[x]$的分裂域为$\mathbb{E}$,$\sigma:\mathbb{E}\to\mathbb{E}$是域自同构.则据\autoref{SpltFd_cor1} ,$\sigma(\mathbb{K})=\mathbb{K}$.于是,$\sigma(a)\in\mathbb{K}$.

由$\sigma$的任意性(即任意一个$\mathbb{E}$自同构,也即任意一个$h$的根的置换),知$h$的根都在$\mathbb{K}$中.



$\Rightarrow$:

由于$\mathbb{K}/\mathbb{F}$为有限扩张,故存在$a_1, a_2, \cdots, a_n\in \mathbb{K}$,使得$\mathbb{K}=\mathbb{F}(a_1, a_2, \cdots, a_n)$.

设$a_i$在$\mathbb{F}$上的最小多项式为$f_i(x)$,令$f(x)=\prod_{i=1}^n f_i(x)$.

由于$\mathbb{K}/\mathbb{F}$为正规扩张,而各$f_i$在$\mathbb{K}$上至少有一个根,故$f(x)$可以在$\mathbb{K}$上写为一次多项式的乘积:
\begin{equation}
f(x) = \prod_{i=1}^k (x-b_i)
\end{equation}
且各$a_i\in\{b_j\}$,各$b_i\in\mathbb{K}=\mathbb{F}(a_1, \cdots, a_n)$.

于是$f\in\mathbb{F}$的分裂域为$\mathbb{F}(b_1, \cdots, b_k)=\mathbb{F}(a_1, \cdots, a_n)=\mathbb{K}$.

\textbf{证毕}.


\begin{example}{正规扩张的反例}
$\mathbb{Q}(2^{1/3})$不是$\mathbb{Q}$的正规扩张.因为存在多项式$f(x)=x^3-2$,它在$\mathbb{Q}$上不可约,有一个根$2^{1/3}$在$\mathbb{Q}(2^{1/3})$上,但另外两个根都是复数,不在其中.

显然,另外两个根的模都是$2^{1/3}$,与正实轴的夹角分别为$\pm 2\pi/3$.
\end{example}

\begin{exercise}{}
求$x^3-2\in\mathbb{Q}[x]$的分裂域.
\end{exercise}













