% 函数的性质与变换(高中)
% keys 函数|变换|平移|旋转|伸缩|单调|对称|奇偶性|初等函数|周期
% license Xiao
% type Tutor

\begin{issues}
\issueDraft
\end{issues}

\pentry{函数(高中)\nref{nod_functi}}{nod_6f70}

在前面的学习中,我们已经接触了函数的概念,以及如何通过复合运算来将多个函数结合起来。现在,我们将深入探讨一个函数的各种性质,比如它的单调性、奇偶性和周期性。这些性质是理解函数行为和性质的关键,它们能够帮助我们更好地分析函数在不同情况下的表现。

然后,我们还会研究函数的变换,比如平移、缩放和对称等操作。当我们改变函数的形式,比如平移、拉伸或对称,它们的图像和性质也会跟着变化,就像镜子里的影像随手势而动。函数的变换不仅会改变函数的图像,还会影响它的性质和表达方式,这让我们可以用更灵活的方式去理解和处理函数。

最后,我们会大概介绍一下高中阶段需要掌握的几种主要函数类型。这些函数在数学和实际应用中扮演着重要角色,它们构成了函数世界的一大部分。这里会把它们作为两部分来整体介绍,后面的学习中会具体介绍每一个函数的细节。

\subsection{函数的变换}

在实际生活中,变换的概念无处不在。比如,调整照片的大小、改变音乐的速度、甚至是地图应用中缩放和旋转视图,这些操作都与函数变换的原理息息相关。

函数的变换本质上是空间的变换,也就是说函数本身的性质没有变,改变的是函数所在的坐标系。但就像驶远的汽车,从你的视角是汽车开远了,而从车的视角,假设车没动的话就是你在远离。坐标系的改变后,如果认为坐标系不变,那么就相当于是函数在进行变换了。尽管这与我们现实生活中的经验不太一样,毕竟我们想要移动一个画框在墙上的位置一般不会把整面墙移动走,但采用“变换的本质是变换坐标系”这个视角去看待函数的变换问题,不仅比较容易理解,而且在处理时也不用去记一些口诀,也更不容易出错。

\subsubsection{平移}

函数图像的平移变换是我们经常用到的一种操作。简单来说,就是将一个函数的图像按照某个方向进行移动,但不改变它的形状。尽管初中时就已经接触到了,或者说平时在生活中我们经常移动某个物品。为了更细致地理解这个过程,下面的内容可以拿两张纸自己比划一下试试,这个例子尽管简单,但对后面的其他变换的理解有好处。

\begin{example}{观察如何“向右平移函数”}
\begin{enumerate}
\item 你有两张纸——上面一张纸代表函数的图像,而下面一张纸代表坐标系。
\item 把上面的纸放在下面的纸上,这时候两张纸是对齐的,图像在坐标系的“原位置”。严格来讲,可以认为两张纸的左下角是原点。
\item 现在,如果想把函数图象向右移动,就需要把上面那张代表图像的纸向右移动(记得不要旋转或改变纸的形状),你会发现图像虽然位置变了,但它本身没有改变——形状还是一样的,只是相对于坐标系的位置发生了移动。
\item 此时,我们可以认为上面的那张纸代表的是原本的坐标系,而下面那张纸代表的是移动后的坐标系。于是,移动后的那张纸的原点出现在了在原本坐标系那张纸的原点的左侧,也就是说,图象向右移动,相当于坐标系向左移动。
\end{enumerate}
\end{example}

同理,如果想要向左、上、下平移函数图像,就需要把坐标系向右、下、上移动。如果要斜着移动,可以把移动的过程拆解成先水平方向移动,再垂直方向移动,也就是移动两次。

理解了“图像与坐标系相对移动”的关系后,接下来看看如何在函数表达式上反映出这种平移变换。

\begin{example}{用表达式描述“函数向右平移$c$个单位”}

假设$(X_0,Y_0)$和$(X_1,Y_1)$分别代表原坐标系和变换后坐标系上的某一点。根据上面的观察,移动后的原点在移动前的原点左侧$c$个单位。也就是说,旧的坐标原点的坐标,从$(0,0)$变成了$(c,0)$。而由于整张纸上的所有点的移动与原点一致,因此他们都符合这个关系,即$(X_0,Y_0)$变成了$(X_0+c,Y_0)$,而$(X_1,Y_1)$代表的就是移动后的点,于是:
\begin{equation}
\begin{cases}
X_1=X_0+c\\
Y_1=Y_0
\end{cases}\iff
\begin{cases}
X_0=X_1-c\\
Y_0=Y_1
\end{cases}~.
\end{equation}

现在假设原坐标系上的点  $(X_0, Y_0)$  在函数图像上,也就是说  $Y_0 = f(X_0)$ 。将  $X_0$  和  $Y_0$  的表达式带入平移后的关系式中,可以得到  $Y_1 = f(X_1 - c)$ 。这正好就是移动后的函数图像上对应的点。因此,我们得知,函数向右平移$c$个单位的表达式描述为:
\begin{equation}
y=f(x)\Rightarrow y=f(x-c)~.
\end{equation}
\end{example}

推荐你自己试一试,根据相同的方法可以得到其他情况下函数平移的表达式描述:
\begin{itemize}
\item 函数向左$c$个单位为:$y=f(x)\Rightarrow y=f(x+c)$
\item 函数向上$b$个单位为:$y=f(x)\Rightarrow y-b=f(x)\Rightarrow y=f(x)+b$
\item 函数向下$b$个单位为:$y=f(x)\Rightarrow y+b=f(x)\Rightarrow y=f(x)-b$
\end{itemize}

很多人会称上面的变换规律为“左加右减,上加下减”,这句话是在函数视角下的描述。一般情况下看似没有问题,但涉及到与伸缩及旋转等复合时,便可能引起错误。在坐标系视角下,应为“左加右减,下加上减”,水平方向上,对$x$进行操作,而垂直方向上,则对$y$进行操作。同时,可以观察到,如果认为$c$或$b$是可以取负值的,那么取负值时则意味着相反方向平移绝对值单位长度,这也与学习负数时的直觉一致。

\subsubsection{伸缩}

有了是上面的经验伸缩就容易得多了,如果我们希望把函数的图象在$y$方向上拉伸为$A$倍,那么就要把坐标系在$y$方向上压缩为$\displaystyle 1\over A$,即取$\displaystyle y_1={y\over A}$即可,代入整理后可以得到:
\begin{equation}
y_1=Af(x_1)~.
\end{equation}

\subsubsection{*旋转}

这一部分内容需要使用三角函数的知识,如果不太熟悉可以去\addTODO{添加初中部分}回顾。

就像复合函数可以多个函数一层层复合一样,这些变换也可以依次加在同一个函数上,最终会得到一个新的表达式。在处理时,如果不太熟悉可以把每一步独立考虑,最后再整理就可以了。下面试一试。
\begin{exercise}{把函数$y=f(x)$先在$x$方向上向右平移$a$个单位,再在$y$方向上向下平$b$个单位,先在$x$方向上压缩为原来的$1\over5$,求得到的新函数的表达式。}

\end{exercise}

\addTODO{抛物线的例子: 事实上抛物线只有一个形状,无论如何水平拉伸,竖直拉伸, 都等效于等比例缩放。这也是为什么抛物线的离心率只有一个值(离心率决定圆锥曲线的形状)。}
\subsubsection{复合变换}

注意顺序

其实函数的变换包含的范畴非常广泛:高中涉及到的函数变换主要针对的是函数图象而言的,包括平移、旋转、伸缩,它们统称线性变换;几何方面,相似变换、射影变换、仿射变换等变换用于分析和操作几何图形及其在空间中的关系;频域分析方面,傅里叶变换、拉普拉斯变换、Z变换等变换帮助我们将问题从时域转换到频域,从而揭示出信号和系统中隐藏的特征。当然现在主要先研究高中涉及的部分;学有余力的话或许也可以了解一番几何部分的变换,对理解解析几何的内容会很有帮助;而频域分析的部分在大学阶段才可能会接触到。
\subsection{函数的性质}\label{sub_HsFunC_1}

函数具有一些性质,有一些在高中会接触到,有一些不会接触到。
以后我们会看到一些用\enref{极限}{Lim}和\enref{导数}{Der}描述的性质。 例如 % \addTODO{链接}
, 可导。
\subsubsection{零点}

函数的零点是指使函数值为零的自变量 $x$,从几何角度来看,函数 $f(x)$ 的零点就是其图像与 $x$ 轴的交点。零点在许多实际问题中有着广泛的应用,比如解方程、找到物体的平衡点、以及工程问题中的最优设计或操作条件等。

\begin{definition}{零点}
对于函数 $f(x)$,使得 $f(x) = 0$ 成立的$x$的值,称为 $f(x)$ 的\textbf{零点(zero point)}。
\begin{equation}
x_0 \in \{ x \mid f(x) = 0 \}~.
\end{equation}
\end{definition}

一般情况下\footnote{确实有一些函数的零点是连续的,例如 $f(x) = 0$ 以及由此复合得到的函数。},零点是孤立的,而零点之外的点通常连续构成一个区间。在这些区间中,函数往往表现出某种特性(例如取正值或负值),而零点则作为这些区间的端点。零点不仅是 $f(x) = 0$ 的解,也是函数符号变化的分界线。通常,函数在零点的两侧会发生符号变化。因此,判断某个点是否属于某个区间时,通常需要将其与零点进行比较。事实上,函数的零点之所以具有特殊意义,正是因为它常常对应某种边界条件,标志着状态的转变或某种变化的界限。

函数的零点与方程有密切联系,零点沟通了函数与方程。任何方程总可以通过移项转换成形如 $f(x) = 0$ 的形式\footnote{高中阶段只涉及一元函数,因此此处指的是一元方程},而这个表达式所描述的,就是函数 $f(x)$ 的零点。这一点在初中学习时,相信你就已经感受过了。

两个函数的交点也可以用零点的形式来表示。假如要求解两个函数 $f(x)$ 和 $g(x)$ 的交点,其实就是寻找满足 $f(x) = g(x)$ 的点。通过设 $F(x) = f(x) - g(x)$,可以将问题转化为求 $F(x)$ 的零点,这样 $F(x)$ 的零点对应的就是 $f(x)$ 和 $g(x)$ 的交点。特别地,当函数 $f(x)$ 取某个固定值 $a$ 时,这个问题可以看作是 $g(x) = a$ 的特殊情况。此时我们设 $F(x) = f(x) - a$。根据函数的平移性质,这相当于将函数 $f(x)$ 向下平移 $a$ 个单位(如果 $a$ 是负值,就向上平移 $|a|$ 个单位)。因此,函数取某个值的点,其实就是平移后的零点。自然,$f(x)$ 的零点就是 $g(x) = 0$ 的特殊情况了。

如果一个零点在方程中出现多次,我们称之为重根。例如,对于 $f(x) = (x - 1)^2$,$x = 1$ 是零点,但它是一个重数为 2 的零点。重数的概念在多项式函数的分析中很重要,零点的重数还与函数在该点的图像行为有关,比如多项式曲线在某些重的重根处“接触”而不是“穿过” $x$ 轴。这是很关键的一点,在不等式部分会着重介绍。

3. 零点与导数的关系

零点与导数有紧密联系,特别是在讨论函数的极值和拐点时,导数的零点往往表示这些特性发生变化的地方。

	•	一阶导数的零点:如果 $f’(x) = 0$,那么 $x$ 可能是函数的极值点。通过结合导数符号的变化,可以进一步判断该点是极大值、极小值还是平稳点。
	•	二阶导数的零点:二阶导数的零点往往对应于函数的拐点,也就是曲率改变的点。讨论二阶导数零点可以帮助学生更好地理解函数图像的凹凸性。

4. 零点存在性定理

	•	介值定理:介值定理是高等数学中的一个基本定理,它在函数零点的存在性判断中起着重要作用。介值定理表明,如果连续函数 $f(x)$ 在区间 $[a, b]$ 上满足 $f(a)$ 和 $f(b)$ 符号相反,那么在 $[a, b]$ 上至少存在一个零点。这个定理为学生提供了一个强大的工具,帮助他们判断零点的存在。
	•	罗尔定理与拉格朗日中值定理:这些定理进一步说明了导数与零点的关系。罗尔定理说明,在连续且可导的函数中,如果函数在两个端点的值相等,那么在区间内部必有一个点,使得导数为零。这些定理可以帮助学生深入理解函数的局部性质。

5. 零点的数值求解

在实际应用中,很多函数的零点并不能通过简单的代数方法求出,此时我们可以使用数值求解方法。常用的数值方法包括:

	•	二分法:在区间 $[a, b]$ 上,如果 $f(a)$ 和 $f(b)$ 符号相反,那么通过不断将区间二分,逐步逼近零点。这种方法虽然简单,但很有效。
	•	牛顿法:通过线性逼近函数的导数,逐步迭代逼近零点。牛顿法收敛速度快,但需要选择合适的初始点,并且要求函数的导数存在。

6. 复数域上的零点

当讨论复数域上的函数时,零点的定义同样适用。特别是对于多项式函数,代数学基本定理告诉我们任何一个 $n$ 次多项式函数在复数域上都有 $n$ 个零点(重数计入)。这意味着在复数范围内,可以找到所有多项式方程的解。

	•	虚数零点的几何意义:虽然实数零点与 $x$ 轴的交点对应,但虚数零点并不在实际的二维平面上有直观的几何解释。然而,通过引入复平面,学生可以对虚数零点有更好的理解。


\subsubsection{单调性}



\begin{definition}{单调性}
设$f(x)$是定义在$D$上的函数,若在$I\subseteq D$上,对$\forall x_1,x_2\in I,x_1< x_2$,函数均满足:
\begin{equation}
f(x_1)<f(x_2)~.
\end{equation}
则称函数$f(x)$在区间$I$上\textbf{单调递增(monotonically increasing})\footnote{此处采取高中教材上的定义。其实,满足这个条件时称作\textbf{严格单调递增 (Strictly increasing)},而单调递增则是指函数满足$f(x_1)\leq f(x_2)$时。递减也相同。这个概念会在大学阶段区分,目前给出作为提醒。},或函数$f(x)$在区间$I$上是\textbf{增函数};若满足
\begin{equation}
f(x_1)>f(x_2)~.
\end{equation}
则称函数$f(x)$在区间$I$上\textbf{单调递减(monotonically decreasing}),或函数$f(x)$在区间$I$上是\textbf{减函数}。

若函数在定义域上单调,则称为\textbf{单调函数 (monotonic functions)}。
\end{definition}



\subsubsection{对称性}

对称性分为两种,一种是轴对称性,一种是中心对称性,这两个性质在初中就有接触过。

\begin{definition}{中心对称}
中心对称
\end{definition}

\begin{definition}{轴对称}
轴对称
\end{definition}

注意这里要区分轴对称性和反函数的区别。

有两个比较特殊的对称性称为奇偶性。

\begin{definition}{奇偶性}
设函数$f(x)$定义在$D$上,且$D$是关于$0$对称的。若对任意的$x\in D$,有:
\begin{itemize}
\item $f(x)=f(-x)$,则称$f(x)$是偶函数。
\item $f(x)=-f(-x)$或$-f(x)=f(-x)$,则称$f(x)$是奇函数。
\end{itemize}
\end{definition}

\subsubsection{周期性}

在生活中,很多事情都是有规律的,比如每天日出日落、四季轮回。数学中用周期性来描述这种“规律”。周期性的函数图象好比一首不断循环的旋律,它遵循着固定的步调,过一段时间就会“回到原点”,再继续以同样的方式变化。

\begin{definition}{周期}
设函数$f(x)$定义在$D$上,且满足
\begin{equation}
\forall x\in D,f(x+T)=f(x)~.
\end{equation}
则称,$T$是函数的一个周期。
\end{definition}


周期函数具有一些特性:
\begin{itemize}
\item 周期函数的图像会在每个周期$T$内重复。无论在$x$轴上平移多少个周期,函数的形状和取值都会保持不变。
\item 两个周期分别为$T_1,T_2$的周期函数,只有满足它们的周期之比$k={T_1\over T_2}$为有理数,即$k={p\over q},p,q\in\mathbb{N}^*$时,他们的和才是周期函数,周期为$T=pT_2=qT_1$。周期相同可以认为是二者周期之比为$1$的特殊情况,此时和仍然是周期函数且周期为原周期。
\item 周期函数在定义域内一定不是单调的,因此周期函数也没有反函数。
\end{itemize}

高中阶段涉及的周期函数主要是两类:一类是抽象函数,也就是不给出表达式,然后利用周期性的特性来等量替换;另一类是三角函数,这将在\enref{三角函数}{HsTrFu}的部分详细讲解。



还有一些性质是高中不会涉及到的,此处给出:
\begin{itemize}
\item \enref{连续性}{contin}, 一致连续
\end{itemize}



\subsection{特殊的函数}

在高中阶段会涉及到的两种特殊的函数包括初等函数和分段函数。

\subsubsection{初等函数}

高中研究的函数都是初等函数。初等函数指的是由基本初等函数经过基本运算(加减乘除)以及复合形成的函数。

初等函数之所以被称为初等函数就是因为它的性质很好,

基本初等函数:
\begin{itemize}
\item 常值函数
\item 幂函数
\item 指数函数
\item 对数函数
\item 三角函数
\end{itemize}

\subsubsection{分段函数}

绝对值函数

取整函数

狄利克雷函数