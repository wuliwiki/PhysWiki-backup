% 金属材料科普(草稿)

\begin{issues}
\issueDraft
\end{issues}

\subsection{原子与晶体}
如果你的视力足够好\footnote{可见光的波长(约为300-700nm)远大于此,因此凭光学显微镜是不可能看到如此细小的结构的.这也是为什么我们发明了电子显微镜)},可以看见纳米级别(大概$10^{-9}m =10^{-6} mm$)的金属微观结构,那么你会发现金属好像是由大量原子层层叠叠、有序堆积起来的.堆积的具体方式与金属的种类\footnote{有些金属有不止一种堆积方式,互称为同素异形体}有关.

\begin{figure}[ht]
\centering
\includegraphics[width=5cm]{./figures/MetInt_1.png}
\caption{晶体中原子的排列示意图} \label{MetInt_fig1}
\end{figure}

\begin{definition}{晶体}
原子(或分子)在三维空间按一定规律作周期性排列而形成的固体
\end{definition}

既然晶体中的原子排列是周期性重复的,我们自然就能找出其中最小的一个单元,以反映这种排列方式的特征.这种最小单元被称为晶胞.例如金属铁的晶胞是体心立方(BCC)结构,即正方体中心的一个原子与周围八个原子均相切,有点像这样:
\begin{figure}[ht]
\centering
\includegraphics[width=5cm]{./figures/MetInt_2.png}
\caption{铁的晶胞}} \label{MetInt_fig2}
\end{figure}
\begin{definition}{晶胞}
能够完全反应晶体几何特征的最小单元
\end{definition}

\subsection{缺陷}
如果金属中的原子都完全按这种理想的方式整整齐齐地排列,那材料科学也未免太无趣了(材料科学的书至少会薄半本!但同时,我们能用材料科学做的事也会少很多!).实际上,由于各种因素的影响,真实的金属晶体往往存在偏离理想结构的区域,称为缺陷.

金属中缺陷的占比虽然不多,但却对材料的性能起到了决定性影响.“缺陷”这个词往往让人以为缺陷对于材料性质都是不利的,但事实并非如此,有些缺陷对于材料(某一方面的性能)可以起到积极作用.

根据缺陷的空间尺度,缺陷一般被分为点缺陷、线缺陷与面缺陷.

\begin{definition}{缺陷}
实际金属中原子偏离理想排列而出现的不完整区域
\end{definition}

\subsubsection{点缺陷}
点缺陷指的是单独少数原子的错误排列.
\begin{figure}[ht]
\centering
\includegraphics[width=8cm]{./figures/MetInt_3.png}
\caption{空位缺陷与间隙缺陷} \label{MetInt_fig3}
\end{figure}
空位:原子离开了自己的理想位置,形成了一个空位

间隙:原子插入了本不应存在原子的位置,一般是晶体的间隙

有时,一些其他种类的原子也会混入到金属晶体之中
\begin{figure}[ht]
\centering
\includegraphics[width=8cm]{./figures/MetInt_4.png}
\caption{杂原子的置换与间隙} \label{MetInt_fig4}
\end{figure}

(杂原子)置换:其他种类的原子替换了格点上的原子

(杂原子)间隙:其他种类的原子插入了间隙之中

\subsubsection{线缺陷}
线缺陷又称为位错,可细分为刃位错、螺位错、以及二者的组合复合位错.

\subsubsection{面缺陷}
面缺陷包括金属的外表面(就是你能看到的那部分)、晶界、相界等等,后者将在下一节简要讨论.

\subsubsection{缺陷的能量}

\subsection{微结构}
\subsubsection{晶粒与晶界}
\subsubsection{相界}