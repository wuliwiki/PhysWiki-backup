% 国际单位制词头

% 未完成: 每个词头需要中文名

\pentry{国际单位制\upref{Consts}}

在使用国际单位时, 我们会在一些单位符号或全称前面加上一个表示数量级的前缀, 正式的名字叫做\textbf{国际单位制词头(metric prefix)}以方便书写. 如长度单位 $\Si{m}$ (米) 可以加不同的词头拓展成更小的单位 $\Si{cm}$(厘米, centimeter), $\Si{mm}$(毫米, millimeter), $\Si{km}$ (千米, kilometer).

国际单位制共有 20 个词头, 下面我们来看每个词头的全称和代表的数量级.

\begin{table}[ht]
\centering
\caption{词头列表}\label{UniPre_tab1}
\begin{tabular}{|c|c|c|c|c|c|}
\hline
全称 & 词头 & 数量级 & 全称 & 词头 & 数量级 \\
\hline
deci & d & $10^{-1}$ & deca & da & $10^1$ \\
\hline
centi & c & $10^{-2}$ & hecto & h & $10^2$ \\
\hline
milli & m & $10^{-3}$ & killo & k & $10^3$ \\
\hline
micro & μ & $10^{-6}$ & mega & M & $10^6$ \\
\hline
nano & n & $10^{-9}$ & giga & G & $10^9$ \\
\hline
pico & p & $10^{-12}$ & tera & T & $10^{12}$ \\
\hline
femto & f & $10^{-15}$ & peta & P & $10^{15}$ \\
\hline
atto & a & $10^{-18}$ & exa & E & $10^{18}$ \\
\hline
zepto & z & $10^{-21}$ & zetta & Z & $10^{21}$ \\
\hline
yocto & y & $10^{-24}$ & yotta & Y & $10^{24}$ \\
\hline
\end{tabular}
\end{table}
