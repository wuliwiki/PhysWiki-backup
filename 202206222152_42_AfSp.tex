% 仿射空间
% keys 仿射空间|坐标架

\begin{issues}
\issueTODO
\end{issues}

\pentry{矢量空间\upref{LSpace}}
在任一矢量空间 $V$ 中,与零矢量 $0$ 结合在一起的坐标原点总是起着特殊的作用,因为在空间 $V$ 的所有自同构 $f:V\rightarrow V$ 下,零矢量 $0$ 是不变的\autoref{LinMap_cor1}~\upref{LinMap},这意味着坐标原点具有突出于其它空间点的性质.然而,对身处3维空间中的我们来说,放置在不同位置的点线面显然不依赖于一个挑选出来的坐标原点.因此,为了满足我们的需求,我们希望把矢量空间的自同构群加以扩展,使得所有的空间点都等价.完成这一任务的属于\textbf{仿射空间}.
\subsection{仿射空间}
仿射空间的定义是基于矢量空间\upref{LSpace}的.
\begin{definition}{仿射空间}
设 $V$ 是域 $\mathbb F$ 上的矢量空间.且 $\mathbb A$ 是一个集合,其元素称着\textbf{点},并用 $\dot{p},\dot{q},\dot{r},\cdots$ 表示.称 $\mathbb A$ (或$(\mathbb A,V)$)是和 $V$ \textbf{相伴(连带的)的仿射空间},若给定笛卡尔积 $\mathbb A\times V$ 到 $\mathbb A$ 的映射:
\begin{equation}
f:(\dot p,v)\mapsto\dot p+v
\end{equation}
其具有性质:
\begin{enumerate}
\item \begin{equation}
\forall \dot p\in\mathbb A,\quad u,v\in V\Rightarrow\dot p+0=\dot p,(\dot p+u)+v=\dot p+(u+v)
\end{equation}
其中,0是空间 $V$ 的零矢量;
\item $\forall \dot p,\dot q\in\mathbb A$,有且仅有一个矢量 $v\in V$ 使得 $\dot p+v=\dot q$. 通常用 $\vec{pq}$ 或 $\dot q-\dot p$ 代表矢量 $v$.
\end{enumerate}
并把 $V$ 的维数 $n$ 称为 $\mathbb A$ 的\textbf{维数},有时记成 $\mathbb A^n$.
\end{definition}
由仿射空间的定义,可推出下面两个性质:
\begin{enumerate}
\item 每个点 $\dot p\in \mathbb A$ 都对应一个从 $V$ 到 $\mathbb A$ 的双射:
\begin{equation}
v\mapsto \dot p+v
\end{equation}

\item 有 $\mathbb A\rightarrow\mathbb A$ 上的双射:
\begin{equation}
t_v:\dot p\mapsto\dot p+v
\end{equation}
 称其为用矢量\textbf{$v$ 平移 $\mathbb A$}(或平行移动 $\mathbb A$).由定义中的1,2,有
 \begin{equation}
 t_u\cdot t_v=t_{u+v},\quad t_v\cdot t_{-v}=e
 \end{equation}
  其中 $e=t_0$ 是恒等映射.显然所有的平移构成一个群\upref{Group}.
\end{enumerate}
\begin{definition}{平移空间}
令
\begin{equation}
\alpha t_u+\beta t_v\equiv t_{\alpha u+\beta v}
\end{equation}
则所有平移的集合构成一个矢量空间,称为\textbf{平移空间},记作 $\mathbb A^{\#}$.
\end{definition}
\begin{example}{}
由仿射空间的定义,试证明:
\begin{equation}\label{AfSp_eq1}
\vec{pq}+\vec{qr}=\vec{pr},\quad \vec{pq}=-\vec{qp},\quad\vec{pp}=0
\end{equation}
亦即
\begin{equation}
(\dot q-\dot p )+(\dot r-\dot q)=\dot r-\dot p,\quad(\dot q-\dot p)=-(\dot p-\dot q),\dot p-\dot p=0
\end{equation}

\end{example}
\subsection{仿射空间的同构}
仿射空间的同构和一般的同构定义实质上相同,都是起说明两个集合结构相同的作用,即集合上元素之间运算,对应于另一集合对应元素之间的运算,这个“对应”是靠双射来完成的.
\begin{definition}{仿射映射,同构}
设 $\mathbb A,\mathbb A'$ 是同一域 $\mathbb F$ 上分别与矢量空间 $V,V'$ 相伴随的仿射空间.称映射 
\begin{equation}
f:\mathbb A\rightarrow\mathbb A' 
\end{equation}
是个\textbf{仿射映射}(或\textbf{线性仿射映射}),如果,所有的 $\dot p\in \mathbb A,v\in V$ 都满足
\begin{equation}
f(\dot p+v)=f(\dot p)+Df\cdot v
\end{equation}
 其中,$Df:V\rightarrow V'$ 是矢量空间上的线性映射,称 $Df$ 是映射 $f$ 的\textbf{线性部分}(或\textbf{微分}).若仿射映射 $f$ 为双射,则称 $\mathbb A$ 和 $\mathbb A'$ 是\textbf{同构的},若此外 $\mathbb A=\mathbb A'$ ,则称 $\mathbb A$ 借助 $f$ 实现\textbf{自同构}.
\end{definition}

\begin{theorem}{}\label{AfSp_the1}
仿射映射
\begin{equation}
f:\mathbb A\rightarrow\mathbb A' 
\end{equation}
是双射,当且仅当线性部分
\begin{equation}
Df:V\rightarrow V'
\end{equation}
是双射
\end{theorem}
\textbf{证明:}
\begin{enumerate}
\item $f\Rightarrow Df$\\

因为 $f$ 是双射,所以对 $v_1\neq v_2\Rightarrow \dot p+v_1\neq\dot p+v_2$,有
\begin{equation}
\begin{aligned}
f(\dot p)+Df\cdot v_1=f(\dot p+v_1)&\neq f(\dot p+v_2)=f(\dot p)+Df\cdot v_2\\
&\Downarrow\\
Df\cdot v_1&\neq Df\cdot v_2
\end{aligned}
\end{equation}
这就证明了 $Df$ 的单射性.  $\forall  v'\in V'$,对 $\forall f(\dot p)+v'$, $f$ 双射意味着 $\exists v$,使得 
\begin{equation}
f(\dot p)+v'=f(\dot p+v)=f(\dot p)+Df\cdot v
\end{equation}
即 $v'=Df\dot v$,这就证明了满射性.于是 $Df$ 双射得证.
\item $Df\Rightarrow f$\\
设 $\dot p\neq \dot q$,固定一点 $\dot o\in \mathbb A$,由仿射空间性质,$\exists \vec{op}\neq\vec{oq}\in V$,使得 
\begin{equation}
\dot p=\dot o+\vec{op},\quad\dot q=\dot o+\vec{oq} 
\end{equation}
由 $Df$ 的单射性
\begin{equation}
\begin{aligned}
Df\cdot \vec{op}&\neq Df\cdot \vec{oq}\\
&\Downarrow\\
f(\dot p)=f(\dot o)+Df\cdot \vec{op}&\neq f(\dot o)+Df\cdot \vec{oq}=f(\dot q)
\end{aligned}
\end{equation}
即有 $f(\dot p)\neq f(\dot q)$,这就证明了 $f$ 的单射性.\\

$\forall \dot p'\in\mathbb A'$,同样 $\exists\vec{o'p'}\in V'$ 使得 $\dot p'=o'+\vec{o'p'}$,其中 $\dot o'=f(\dot o)$,由 $Df$ 的满射性,
\begin{equation}
\exists v\in V,\quad st.\; Df\cdot v=\vec{o'p'}
\end{equation}
于是
\begin{equation}
f(\dot o+v)=f(\dot o)+Df\cdot v=\dot o'+\vec{o'p'}=\dot p'
\end{equation}
这就证明了满射性.
\end{enumerate}
\textbf{证毕!}

\begin{corollary}{}
具有相同维数的仿射空间 $(\mathbb A,V),(\mathbb A',V')$ 必同构.
\end{corollary}
\textbf{证明:} 由于维数相同的矢量空间必同构,所以存在双射 $F:V\rightarrow V'$ ,令其作为 $Df$ ,由\autoref{AfSp_the1} ,即\textbf{得证!}
\subsection{坐标}
\begin{definition}{坐标系(坐标架)}
称点 $\dot o\in\mathbb A$ 和 $V$ 的一个基底 $\{e_1,\cdots,e_n\}$ 的集合 $\{\dot o;e_1,\cdots,e_n\}$ 是 $n$ 维仿射空间 $(\mathbb A,V)$ 的一个\textbf{坐标系}(或\textbf{坐标架}).矢量 $\vec{op}$ 在基底 ${e_i}$ 下
\begin{equation}
\vec{op}=\sum_{i=1}^n x_ie_i
\end{equation}
的坐标就被认为是点 $\dot p$ 在坐标系 $\{\dot o;e_1,\cdots,e_n\}$ 之下的\textbf{坐标}.
\end{definition}
坐标系同样也可以用这样的 $n+1$ 个点 $\{\dot p_0;\dot p_1,\cdots,\dot p_n\}$ 给出,只要矢量 $\vec{p_0p_1},\cdots,\vec{p_0p_n}$ 构成空间 $V$ 的一个基底即可.
\begin{theorem}{}
设 $\{\dot p_0;\dot p_1,\cdots,\dot p_n\}$ 是空间 $V$ 的一个坐标系:
\begin{equation}
e_i:=\vec{p_0 p_i},\quad i=1,\cdots,n
\end{equation}
若 $\dot p,\dot q$ 在此坐标系中的坐标分别为 $x_1,\cdots,x_n$ 和 $y_1,\cdots,y_n$ ,那么:
\begin{enumerate}
\item 矢量 $\vec{pq}$ 在此基底之下的坐标为 $y_1-x_1,\cdots,y_n-x_n$;
\item 对任意矢量 $a=\sum\limits_{i=1}^n a_ie_i$ ,点 $\dot p+a$ 的坐标是 $x_1+a_1,\cdots,x_n+a_n$.
\end{enumerate}
\end{theorem} 
\textbf{证明:} 由\autoref{AfSp_eq1} :
\begin{equation}\label{AfSp_eq2}
\vec{pq}=\vec{pp_0}+\vec{p_0q}=\vec{p_0q}-\vec{p_0p}
\end{equation}
由矢量加法定义,即证得定理中的1.

设 $\dot q=\dot p+a$,于是 $a=\vec{pq}$,结合\autoref{AfSp_eq2} ,有
\begin{equation}
a=\vec{p_0q}-\vec{p_0p}\Rightarrow \vec{p_0q}=a+\vec{p_0p}
\end{equation}
由矢量加法和点的坐标定义,即证得定理中的2.

\textbf{证毕!}
\subsubsection{不同坐标系下坐标的转换关系}
设 $\{\dot o;e_1,\cdots,e_n\},\{\dot o';e'_1,\cdots,e'_n\}$ 是仿射空间 $\mathbb A$ 中的两个坐标系,