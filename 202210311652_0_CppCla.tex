% C++ 的 Class 笔记

\begin{issues}
\issueDraft
\end{issues}

\begin{itemize}
\item copy constructor 本身就是一个 constructor, 只有在用同类初始化, 或者函数 pass by value 和 return by value 的时候才用到.
\item 可以强制 constructor 或 \verb|operator=| 等函数使用 default 或者 delete. 格式为 \verb|fun() = default| 或 \verb|fun() = delete|.
\item Derived Class 继承 (inherit) Base Class 的所有数据, 以及所有函数 (除了 destructor, constructor, \verb|operator=|). 当 Derived Class 重新定义了 Base Class 的同格式函数时, 就把 Base Class 的函数覆盖了, 若要用 Base Class 的函数, 就在函数前面加 Base Class 名和 \verb|::|.
\item 声明 Derived Class 的 object 的时候, Base Class 的 Constructor 先被调用, 毁灭 object 的时候, Base Class 的 Destructor 后被调用.
\item 声明 Class 的时候不可以用空括号 (如 CBox mybox()), 把括号去掉即可.
\item class CCandyBox : CBox {...}; 这种情况下默认 class CCandyBox : private CBox {...};
\item protected 和 private 对当前的 class 来说效果是一样的, 只是继承的时候有所不同.
\item 一般情况下 Base class 中的 private members 在 Derived class 的函数中没有访问权限 (但 protected 有). 在 private 继承下, 所有继承的 member 都变为 private; 在 protected 继承下, 所有继承的 member 都变为 protected; 在 pubic 继承下, 继承的 member 权限不变.
\item 如果 Base Class 是一个 template Class, 那么任何 member 都默认不能被 Derived class 直接访问.
\item 如果 Derived Class 无法访问 Base Class member(包括如 \verb|operator=|), 可以在 member 前面加 this->, 或者在 Derived Class 内部的一开始声明如 using BaseClass::m_data; 这样做仍然不能获取 Base Class 的 private members.
\item 在 Derived class 的 constructor 的 Initializer list 中可以指定 Base Class 的 constructor. 如果不指定, 一律用默认 constructor. 就算指定了, list 中也还可以有其他变量.
\item Derived Class 的 constructor initializer list 中不可以有 Base class 中的变量?
\item 在写 Drived Class 的 copy constructor 的时候, 要处理 base class 的 copy, 例如直接在 initilization list 中调用 base class 的 copy constructor, 否则 base class 就会用默认的 connstrtor. move constructor 也一样!
\item 用 final 可以禁止被继承, 如 class CCandyBox final {};
\item 如果 class A 声明一个函数是 friend,那么这个函数必须要在 class 的前面有声明,如果函数是 class B 的成员函数,那么 class B 要在 class A 前声明,且这个成员函数要在 class A 之后定义,因为会用到 class A 中的数据.
\item 声明 friend class 如 friend Cbox; class 声明只是一厢情愿的.friend 声明不会被继承.
\item virtual function 第一是用来给 base class 的函数调用 derived class 的函数用的,另一种情况就是通过 base class 的指针(指向 derived class object)来调用 derived class 的函数.只要在最底层 base class 的被 call 函数声明 virtual,但为了可读性建议把所有的同名函数都声明 virtual.
\item 含有 virtual function 的 class 叫做 polymorphic class.
\item 为了确保 derived class 中的函数覆盖 base class 中的 virtual 函数,可以用 override,如 virtual double volume() const override
\item final 可以防止 base class 函数被 override,如 virtual double volume() const final
\item base class 的指针可以用来储存 derived class object 的地址.pointer 所指的 object 决定哪个同名 virtual function 被调用.
\item 如果没有 virtual function,通过指针调用的函数版本将由指针的类型决定(包括 delete 调用的 destructor),如果指针类型是 base class,存的却是 derived class 的地址,就可能会出问题.
\item 如果一个函数以 base class 为自变量, 该自变量也可以输入 derived class, 只有 base class 的部分进入函数内部? (private 继承的是 base class 能否被函数修改?)但 virtual function 仍然会调用的 drived class 的版本.
\item 如果一个函数以 base class 的 reference 为自变量, 那么在函数内部可以调用该变量的 virtual function.
\item pure virtual function 在 base class 中是 trivial 的,例如 virtual double volume() const = 0; 含有 pure virtual function 的 class 叫做 abstract class, 只能被继承, 不能声明 object, 但可以声明指针.
\item dynamic_cast<>() 可以将指针在 base class 和 derived class 间转换. 注意 dynamic_cast<>() 只能对 polymorphic class 使用.
\item 可以声明一个成员函数将一个 class type 转换成另一种 type,如 operator bool() {} 该函数的 return type 是 bool,将当前 class type 转换为 bool 类型.在前面加 explicit 可以阻止 implicit conversion,而只允许 explicit conversion.
\item 如果函数自变量是 base class reference, 而调用时输入的是 derived class reference,那么可以在函数内使用 static_cast<CDerived&>() 得到 derived class.例如 NR3 工程中
\begin{lstlisting}[language=cpp]
template <class T, class T1, class T2>
void myplus(NRbase<T> &v, NRbase<T1> &v1, NRbase<T2> &v2)
{
	// if v, v1, v2 are vectors
	/*auto& v_ = static_cast<NRvector<T>&>(v);
	auto& v1_ = static_cast<NRvector<T1>&>(v1);
	auto& v2_ = static_cast<NRvector<T2>&>(v2);*/

	// if v, v1, v2 are matrices
	auto& v_ = static_cast<NRmatrix<T>&>(v);
	auto& v1_ = static_cast<NRmatrix<T1>&>(v1);
	auto& v2_ = static_cast<NRmatrix<T2>&>(v2);
	// check v1 and v2 has the same size
	v_.resize(v1_);
	for (Long i = 0; i < v1.size(); ++i)
		v(i) = v1(i) + v2(i);
}
\end{lstlisting}

\item class 中的 \verb`typedef` 和 \verb`using` 同样由 \verb`private`, \verb`protected` 和 \verb`public` 之分, 和 data member 是一样的.
\item \verb`operator=` 不会被继承
\item 如果在 \verb`operator=` 自变量用 template, 那么默认的 \verb`operator=` 会首先被使用! 这种情况下若不想用默认的, 那么就自己写一个默认的 \verb`operator=(rhs)` 然后在函数体调用 template 版本即可: \verb`return operator=<T1>(rhs)`.
\item c++ 自带的 type 如果用 default constructor, 如 \verb`double x = double()`, 生成的就是 0 或等效的值. 注意不能是 \verb`double x()`, 这个什么都不会定义, 且不会报错.
\item 自定义类型的 temporary 是可以被赋值的! 例如 \verb`a + b = c` 对 \verb`complex<double> a, b, c` 是合法的! 如果想禁止这么做, 函数返回 \verb`const` 类型即可.
\item 如果 \verb`const T &` 指向 temporary (例如 \verb`T()`), 那么 temporary 的寿命会随 \verb`const T&` 延长. 例如函数可以返回了指向 temporary 的 \verb`const T &` 没有问题. 但这么做很危险, 因为如果 \verb`const T&` 再次作为自变量传递到另一个函数中(或者再次被返回?) temporary 就会失效.
\item template specilization 的时候, template argument 可以任意多.
\end{itemize}
