% 无穷阶差分矩阵
% keys 差分矩阵|无穷阶|傅里叶变换
% license Usr
% type Tutor

\pentry{导数近似之差分矩阵算法\nref{nod_DifMa}}{nod_3681}

\subsection{傅里叶变换回顾}
\subsubsection{连续区间上的傅里叶变换}
由\enref{正交函数系}{OFS}中的\autoref{ex_OFS_3} 可知,函数 $u(x),x\in\mathbb R$ 的傅里叶变换定义为
\begin{equation}
\hat u(k):=\int_{-\infty}^{\infty} e^{-\I kx}u(x)\dd x,\quad k\in\mathbb R.~
\end{equation}
而从 $\hat u$ 同构逆傅里叶变换可以重构 $u$:
\begin{equation}
u(x)=\frac{1}{2\pi}\int_{-\infty}^{\infty} e^{\I kx}\hat u(k)\dd k.~
\end{equation}
这被称为\textbf{傅里叶合成}(Fourier synthesis),变量 $x$ 称为\textbf{物理变量}(physical variable),$k$ 称为\textbf{傅里叶变量}(Fourier variable)或\textbf{波数}(wavenumber)。

\subsubsection{离散点上的傅里叶变换}
当限定 $x\in hZ$时,即此时 $x$ 只能取离散点,





















