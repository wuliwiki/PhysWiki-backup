% LSZ约化公式(矢量场)
% LSZ约化公式|量子电动力学|矢量场|S矩阵|费曼图


我们在这一节主要讨论的是旋量量子电动力学(Spinor electrodynamics,也简称旋量 QED),它是关于电子、正电子和光子的相互作用的量子场论,拉氏量为
\[
\mathcal{L}=\bar{\psi}(i\not D - m_0)\psi - \frac{1}{4}F_{\mu\nu}F^{\mu\nu}
\]
其中协变微商 $D_\mu=\partial_\mu+ie_0 A_\mu$,因此拉氏量的相互作用部分为 $\mathcal{L}_{int}=-e_0 A_\mu \bar\psi \gamma^\mu \psi=-e_0 A_\mu j^{\mu}$。自由场的拉氏量则包含了自由 Dirac 场和和自由电磁场这两个部分,它们分别描写了电子、正电子和光子。仿照之前的讨论,我们可以通过统计 Feynman 图的贡献来计算旋量 QED 的 $n$ 点编时格林函数。根据 Wick 定理,费米子线与光子线可以由自由场论的 Feynman 传播子得到,而相互作用顶点需要带上一个 $-ie_0 \gamma^\mu$ 的因子。
前面我们建立了自旋 $0,1/2$ 粒子(分别对应标量场和旋量场)的 LSZ 约化公式。为了描述入态和出态中有光子参与的散射过程,将 S-矩阵同相应的编时格林函数联系起来,我们还需要对自旋 $1$ 的矢量场(也就是电磁场)建立 LSZ 约化公式。
\subsubsection{矢量场的 LSZ 约化公式}
在讨论自由电磁场时,$\ket{\bvec k,\lambda}=a_{\bvec k}^{(\lambda)\dagger}\ket{0}$ 为归一化的单光子态(我们只考虑物理的偏振态,即不考虑纵向偏振和标量偏振的光子),它与自己的内积 $2E_{\bvec k}\delta^3(\bvec 0)$ 是一个洛伦兹不变量。对于相互作用场论也存在一系列由量子数 $\bvec k,\lambda$ 刻画的单粒子态,它们由洛伦兹变换相联系。我们有以下关系:
\[
\bra{\Omega}A_\mu(x)\ket{\bvec k,\lambda}=\sqrt{Z_3}\epsilon_\mu^{\lambda}(k)e^{-ik\cdot x}
\]
则在时间在 $\pm \infty$ 区域矢量场算符可以通过傅里叶变换约化为入射和出射的光子。(下面 $\int_1$ 和 $\int_3$ 分别代表在 $y^0$ 的积分区域在 $t>T^+$ 和 $t<T^-$)
\begin{equation}
\begin{aligned}
&\epsilon_\mu^{(\lambda)*}(\bvec k)g_{\lambda\lambda}\int_{ 1} \dd[4]{x} e^{ikx} \bra{\Omega}T[A^\mu(x)\cdots]\ket{\Omega}
\stackrel{k^0\rightarrow E_{\bvec k}}{\sim}
\frac{i\sqrt{Z_3}}{k^2+i\epsilon} \bra{\bvec k,\lambda} T[\cdots]\ket{\Omega}
\\
&
\epsilon_\mu^{(\lambda)}(\bvec k)g_{\lambda\lambda}
\int_{3} \dd[4]{y} e^{-iky}
\bra{\Omega}T[\cdots A^\mu(y)]\ket{ \Omega}
\stackrel{k^0\rightarrow E_{\bvec k}}{\sim}
\frac{i\sqrt{Z_3}}{k^2+i\epsilon}\bra{\Omega} T[\cdots]\ket{\bvec k,\lambda}
\end{aligned}
\end{equation}
经过波包调制后,$n$ 点格林函数的所有矢量场算符就可以被约化为入态和出态,最终就得到了矢量场的 LSZ 约化公式:
\begin{equation}
\begin{aligned}
&\prod_{i}\int \dd[4]{x} e^{ik'_ix_i} 
\left[\epsilon_{\mu_i}^{(\lambda'_i)*}(\bvec k'_i)g_{\lambda'_i\lambda'_i}\right] 
\prod_{j}\int \dd[4]{y} e^{-ik_jy_j}  \left[\epsilon_{\nu_j}^{(\lambda_j)}(\bvec k_j) g_{\lambda_j\lambda_j}\right]
\bra{\Omega}T[A^{\mu_1}(x_1)A^{\nu_1}(y_1)\cdots]\ket{\Omega}
\\
&\mapsto \prod_i\frac{i\sqrt{Z_3}}{k_i'^2+i\epsilon}\prod_j \frac{i\sqrt{Z_3}}{k_j^2+i\epsilon}\bra{\bvec k'_1,\lambda'_1;\cdots}S\ket{\bvec k_1,\lambda_1;\cdots}
\end{aligned}
\end{equation}
其中 $\mapsto$ 表示取所有的动量在壳,并且仅仅考察其中最为奇异的多极点部分的贡献。$Z_3$ 是矢量场的场强重整化因子。

将 $n$ 点格林函数微扰展开后每一项可以用一个连通的 Feynman 图表达。可以将外腿的两点函数与截肢图的贡献独立开来,以出射光子为例,外腿部分的两点函数为
\[
\epsilon_\mu^{(\lambda)*}(\bvec k)g_{\lambda\lambda}\int \dd[4]{x} e^{ikx} \bra{\Omega}T[A^\mu(x)A^\nu(0)]\ket{\Omega}= \epsilon^{\nu(\lambda)}(\bvec k)\frac{iZ_3}{k^2+i\epsilon}
\]
这与 LSZ 约化公式右侧的因子相抵消后,只剩下一个 $\sqrt{Z_3}\epsilon^{ (\lambda)\nu*}$ 的因子。我们将 $\epsilon^{(\lambda)\nu*}$ 记在 Feynman 图外线的贡献里,它代表出射光子的偏振方向;类似地,入射光子对应的外线要带上 $\epsilon^{(\lambda)\mu}$ 的因子。具体地有
\begin{enumerate}
\item 入射光子:$A^\mu(x) \ket{\bvec k,\lambda}=\epsilon^{(\lambda)\mu}(\bvec k)$。
\item 出射光子:$\bra{\bvec k',\lambda'}A^\nu(x)=\epsilon^{(\lambda)\nu*}(\bvec k')$。
\end{enumerate}
最后计算散射过程时只需要统计所有截肢的 Feynman 图的贡献,再带上 $(\sqrt{Z_3})^{n+m}$ 的因子即可($n+m$ 是光子外线的数量)。