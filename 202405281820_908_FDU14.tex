% 复旦大学 2014 量子真题
% license Usr
% type Note

\textbf{声明}:“该内容来源于网络公开资料,不保证真实性,如有侵权请联系管理员”

\subsection{一,简要回答下列问题 (40分)}

(1) 分别写出动量表示象 $p$ 和坐标表示象 $x$ 的薛定谔方程。

(2) 由正则对易关系 $\hat{x}, \hat{p}] = i\hbar$ 导出角动量的三个分量
$\begin{aligned}    L_x &= y \frac{\partial}{\partial z} - z \frac{\partial}{\partial y}, \\\\    L_y &= z \frac{\partial}{\partial x} - x \frac{\partial}{\partial z}, \\\\    L_z &= x \frac{\partial}{\partial y} - y \frac{\partial}{\partial x}\end{aligned}$
的对易关系。

(3) 氢原子中电子的状态必须用四个量子数才能确定。请问是哪四个量子数?它们的取值有何要求?它们分别决定了什么物理量?

(4) 一个体系由两个自旋量子数为 $3/2$ 的全同粒子组成,则体系对称的自旋波函数和反对称的自旋波函数各有几个?

(5) 两个力学量能够同时测准的条件是什么?试举例说明。

\subsection{二,}两个质量均为 $m$ 的粒子,在一维无限深势阱中
$V(x) = \begin{cases} \infty, & x < 0, x > a \\\\0, & 0 \leq x \leq a \end{cases}$

中运动,彼此间无相互作用。它们可在最低能级和第一激发态能级上以可能的方式进行填充。当这两个粒子是:

(1) 自旋为零的玻色粒子;

(2) 自旋为半的全同费米粒子 (自旋 $1/2$)。时,讨论该双粒子体系的能量本征值和相应的本征波函数的对称性部分。(20分)

\subsection{三,}设初始时刻一维自由粒子的波函数为

$\phi(x,0) = Ae^{ikx - \alpha x^2}$

其中 $k, \alpha$ 为常数,试求解以下问题:

(1) 先计算积分 $\int_{-\infty}^{+\infty} e^{-2 \alpha x^2} dx$,从而找到所需积分公式以求出归一化系数 $A$;

(2) 动力平均值 $\langle P \rangle$ 与动量分布函数 $c(p)$;

(3) t 时刻的波函数 $\phi(x,t)$。(30分)

\section*{四、设一个自旋为1,电荷为 e 的粒子处于磁场 $\mathbf{B} = B \hat{e}_z$ 中,其哈密顿量为}
$H = -\\mu \\cdot \\mathbf{B} = \frac{e}{mc} \mathbf{B} \cdot \mathbf{S}$

其中自旋为 1 的三个自旋矩阵为 (在 $S^2, S_z$ 表象)

$S_x = \frac{h}{\sqrt{2}}\begin{pmatrix}0 & 1 & 0 \\\\1 & 0 & 1 \\\\0 & 1 & 0 \end{pmatrix}, \quadS_y = \frac{h}{\sqrt{2}}\begin{pmatrix}0 & -i & 0 \\\\i & 0 & -i \\\\0 & i & 0 \end{pmatrix}, \quadS_z = h\begin{pmatrix}1 & 0 & 0 \\\\0 & 0 & 0 \\\\0 & 0 & -1 \end{pmatrix}$

设时间 t=0 时粒子的自旋在 x 轴上的投影为 $+h$,试求 t > 0 时:

(1) 粒子的自旋态 $z(t)$;

(2) 粒子自旋在 x 轴投影仍然为 $+h$ 的几率;

(3) 粒子自旋 $(S_x, S_y, S_z)$ 的平均值。(30分)
