% 丘成桐(综述)
% license CCBYSA3
% type Wiki

本文根据 CC-BY-SA 协议转载翻译自维基百科\href{https://en.wikipedia.org/wiki/Shing-Tung_Yau}{相关文章}。

\begin{figure}[ht]
\centering
\includegraphics[width=6cm]{./figures/fb16385d5284d04b.png}
\caption{} \label{fig_QCT_1}
\end{figure}
丘成桐(Shing-Tung Yau,发音:/jaʊ/;中文:丘成桐;拼音:Qiū Chéngtóng;1949年4月4日出生)是一位中美籍数学家。他是清华大学丘成桐数学科学中心的主任,同时是哈佛大学的名誉教授。直到2022年,丘成桐一直担任哈佛大学威廉·卡斯帕·格劳斯坦数学教授,之后他移居清华大学。

丘成桐1949年出生于汕头,年幼时移居英国香港,1969年移居美国。他因在偏微分方程、卡拉比猜想、正能量定理和蒙热–安培方程等方面的贡献而于1982年获得菲尔兹奖。丘成桐被认为是现代微分几何和几何分析发展的主要贡献者之一。他的工作在凸几何、代数几何、计数几何、镜像对称、广义相对论、弦理论等数学和物理领域产生了深远的影响,同时他的研究也涉及到应用数学、工程学和数值分析等领域。
\subsection{传记}  
丘成桐1949年出生于中华民国广东省汕头市,父母为客家人。[YN19] 他的祖籍是中国嘉应县。[YN19] 他的母亲梁玉兰来自中国梅县区;父亲丘镇英(Chen Ying Chiu)是中华民国国民党学者,涉猎哲学、历史、文学和经济学。[YN19] 他是家中八个孩子中的第五个。[4]

在中国大陆发生共产主义接管时,丘成桐还只有几个月大,他的家人移居到英国香港,并在那里接受教育(除了英语课外),他的学业完全用粤语,而不是父母的母语客家话。[YN19] 他直到1979年,才在华罗庚的邀请下回到大陆,那时中国大陆进入改革开放时代。[YN19] 他们最初住在元朗,1954年搬到沙田。[YN19] 由于失去了所有财产,他们家经济拮据,而他的父亲和第二个姐姐在他十三岁时相继去世。[YN19] 丘成桐开始阅读并欣赏父亲的书籍,变得更加专注于学业。完成培正中学学业后,他于1966至1969年间在香港中文大学学习数学,由于提前毕业,他未获得学位。[YN19] 他将课本留给了他的弟弟丘成栋,后者也决定主修数学。

丘成桐于1969年秋季前往加利福尼亚大学伯克利分校攻读数学博士学位。在寒假期间,他阅读了《微分几何学报》的第一期,并深受约翰·米尔诺(John Milnor)关于几何群体理论的论文启发。[5][YN19] 随后,他提出了普雷斯曼定理的一个推广,并在接下来的学期与布莱恩·劳森(Blaine Lawson)共同进一步发展了这一思想。[6] 基于这项工作,他于1971年获得博士学位,导师是陈省身(Shiing-Shen Chern)。[7]

他在普林斯顿高级研究院度过了一年后,于1972年加入石溪大学担任助理教授。1974年,他成为斯坦福大学的副教授。[8] 1976年,他在加州大学洛杉矶分校(UCLA)担任访问教授,并与物理学家郭玉云结婚,郭玉云是他在伯克利大学攻读研究生时认识的。[8] 1979年,他返回普林斯顿高级研究院,并于1980年成为该院教授。[8] 1984年,他接受了加州大学圣地亚哥分校的讲席教授职位。[9] 1987年,他搬到哈佛大学。[8][10] 2022年4月,丘成桐从哈佛大学退休,担任哈佛大学威廉·卡斯帕·格劳斯坦数学教授名誉教授。[8] 同年,他移居清华大学,担任数学教授。[8][2]

根据丘成桐的自传,他在1978年因英国领事馆撤销了他的香港居留权,而成为“无国籍人”,原因是他已拥有美国永久居留身份。[11][12] 关于1982年获得菲尔兹奖时的身份,丘成桐表示:“我很自豪地说,当我获得数学菲尔兹奖时,我没有任何国家的护照,应该被视为中国人。”[13] 丘成桐一直是“无国籍人”,直到1990年获得美国国籍。[11][14]

丘成桐与科学记者史蒂夫·纳迪斯(Steve Nadis)合作,写了几本书,包括一本非技术性的《卡拉比-丘流形与弦理论》介绍,[YN10][15] 一本哈佛大学数学系的历史,[NY13] 一本支持在中国建设环形电子正电子对撞机的书,[NY15][16][17] 一本自传,[YN19][18] 以及一本关于几何与物理关系的书。[NY24]
\subsection{学术活动}  
丘成桐对现代微分几何和几何分析的发展做出了重要贡献。正如威廉·瑟斯顿(William Thurston)在1981年所说:[19]

“我们很少有机会目睹一位数学家的工作在短短几年内影响整个研究领域的方向。在几何学领域,过去十年中最为显著的此类例子之一,就是丘成桐的贡献。”

他最广为人知的成果包括与郑守远共同解决的蒙热–安培方程的边值问题、与理查德·肖恩(Richard Schoen)共同在广义相对论的数学分析中取得的正能量定理、解决卡拉比猜想、最小曲面的拓扑理论(与威廉·米克斯(William Meeks)共同完成)、唐纳森–乌伦贝克–丘定理(与凯伦·乌伦贝克(Karen Uhlenbeck)共同完成)、以及与郑守远和李大潜(Peter Li)共同提出的丘–郑梯度估计和李–丘梯度估计(用于偏微分方程)。丘成桐的许多成果(除了其他人的成果)被编写成教科书,并与肖恩共同出版。[SY94][SY97]

除了他的研究工作,丘成桐还是多个数学研究所的创始人和主任,这些研究所大多位于中国。约翰·科茨(John Coates)评论道:“没有任何一位当代数学家能够与丘成桐在中国大陆和香港为数学活动筹款方面的成功相提并论。”[6] 在台湾国立清华大学的学术休假期间,丘成桐应高锟(Charles Kao)的邀请,在香港中文大学创建数学研究所。在几年筹款努力后,丘成桐于1993年建立了多学科的数学科学研究所,并邀请他常合作的作者郑守远担任副主任。1995年,丘成桐协助吕永祥(Yongxiang Lu)从罗尼·陈(Ronnie Chan)和陈戌源(Gerald Chan)的晨兴集团筹集资金,支持新成立的中国科学院晨兴数学中心。丘成桐还参与了浙江大学数学科学中心、清华大学数学科学中心、国立台湾大学数学中心以及三亚的数学研究所的建设。[20][21][22][23] 最近,在2014年,丘成桐筹集资金建立了哈佛大学的数学科学与应用中心(他是主任)、绿色建筑与城市中心以及免疫学研究中心。[24]

丘成桐以李政道和杨振宁曾组织的物理学会议为模型,提出了国际华人数学家大会,该大会现每三年举行一次。第一次大会于1998年12月12日至18日在晨兴数学中心举办。他还共同组织了《微分几何学报》和《数学的当前发展》年会。丘成桐是《微分几何学报》[25]、《亚洲数学杂志》[26]和《理论与数学物理进展》[27]的主编。截至2021年,他已指导超过七十位博士生。[7]

在香港,丘成桐在陈戌源的支持下设立了恒隆奖,奖励高中生。他还组织并参与了面向高中生和大学生的会议,例如2004年7月在杭州举行的“为什么学数学?向大师请教!”座谈会,以及2004年12月在香港举行的“数学的奇妙”讲座。丘成桐还共同发起了一系列关于大众数学的书籍《数学与数学人物》。

在2002年和2003年,格里戈里·佩雷尔曼(Grigori Perelman)向arXiv发布了预印本,声称证明了瑟斯顿几何化猜想,并作为特例,证明了著名的庞加莱猜想。尽管他的工作包含了许多新思想和结果,但他的证明在一些技术性论证上缺乏详细的推导。[28] 在接下来的几年里,几位数学家投入了大量时间,填补这些细节,并向数学界提供了佩雷尔曼工作的阐述。[29] 2006年8月,西尔维亚·纳萨尔(Sylvia Nasar)和大卫·格鲁伯(David Gruber)在《纽约客》上发表的一篇著名文章,将涉及丘成桐的几场职业争议公之于众。[13][14]

