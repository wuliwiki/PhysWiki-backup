% 仿射空间
% 仿射空间|坐标架

\pentry{矢量空间\upref{LSpace}}
在任一矢量空间 $V$ 中,与零矢量 $0$ 结合在一起的坐标原点总是起着特殊的作用,因为在空间 $V$ 的所有自同构 $f:V\rightarrow V$ 下,零矢量 $0$ 是不变的\autoref{LinMap_cor1}~\upref{LinMap},这意味着坐标原点具有突出于其它空间点的性质.然而,对身处3维空间中的我们来说,放置在不同位置的点线面显然不依赖于一个挑选出来的坐标原点.因此,为了满足我们的需求,我们希望把矢量空间的自同构群加以扩展,使得所有的空间点都等价.完成这一任务的属于\textbf{仿射空间}.
\subsection{仿射空间}
仿射空间的定义是基于矢量空间\upref{LSpace}的.
\begin{definition}{仿射空间}
设 $V$ 是域 $\mathbb F$ 上的矢量空间.且 $\mathbb A$ 是一个集合,其元素称着\textbf{点},并用 $\dot{p},\dot{q},\dot{r},\cdots$ 表示.称 $\mathbb A$ 是和 $V$ \textbf{相伴(连带的)的仿射空间},若存在映射 
\begin{equation}
f:\mathbb A\times V\rightarrow \mathbb A
\end{equation}
其具有性质:
\begin{enumerate}
\item \begin{equation}
\forall \dot p\in\mathbb A,\quad u,v\in V\Rightarrow\dot p+0=\dot p,(\dot p+u)+v=\dot p+(u+v)
\end{equation}
其中,0是空间 $V$ 的零矢量;
\item $\forall \dot p,\dot q\in\mathbb A$,有且仅有一个矢量 $v\in V$ 使得 $\dot p+v=\dot q$. 通常用 $\uparrow{pq}$ 或 $\dot q-\dot p$ 代表矢量 $v$.
\end{enumerate}



\end{definition}
