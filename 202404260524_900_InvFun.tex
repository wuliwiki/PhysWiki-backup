% 反函数(高中)
% keys 函数|映射|反函数
% license Usr
% type Tutor

\begin{issues}
\issueDraft
\end{issues}

\pentry{函数(高中)\nref{nod_functi}}{nod_f15b}
由于函数本质上是映射\upref{map}, 那么如果一个函数的映射存在逆映射, 那么我们就把逆映射对应的函数称为\textbf{反函数(inverse function)}。

我们知道不是所有的映射都有逆映射。 只有单射存在逆映射, 而多对一映射不存在逆映射。
\begin{equation}
x = f^{-1}[f(x)]~.
\end{equation}


我们往往可以通过缩小定义域的方式来使一个函数具有反函数。 例如 $\sin^{-1} x$ 是 $\sin(x)$ 在 $[-\pi/2, \pi/2]$ 区间上的反函数。

对于有反函数的函数,我们可以用反代法得出反函数,例如
\begin{equation}
y = kx + b \qquad (k \ne 0)~.
\end{equation}
我们用 $x$ 代换 $y$,用 $y$ 代换 $x$,可得
\begin{equation}
x = ky + b \qquad (k \ne 0)~,
\end{equation}
\begin{equation}
y = \frac{1}{k} \cdot x - \frac{b}{k}~.
\end{equation}

\subsection{图像}
如果一个实函数 $y = f(x)$($f: \mathbb R \to \mathbb R$) 可以使用图像描述, 那么反函数就是 $f(x)$ 关于直线 $y = x$ 的镜像对称。

例如 $x^2$ 和 $\sqrt{x}$ 在 $(0, \infty)$ 区间的函数图像。 关于 $y = x$ 对称。
