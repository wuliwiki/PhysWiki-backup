% Verlet算法的简单示例

% Verlet算法的简单实现

\begin{issues}
\issueNeedCite
\end{issues}

Verlet算法常用于数值计算一些经典动力学问题。一些论坛探讨可以参考https://www.zhihu.com/question/22531466/answer/27695021与https://www.zhihu.com/question/22531466/answer/763079874。

简而言之,Verlet算法认为,某时刻粒子的位置可由前两时刻粒子的位置与受力递推得到。假定时间步长为$\Delta t$.
\begin{equation}
\begin{aligned}
\bvec r (t) &= 2\bvec r(t-\Delta t) - \bvec r(t-2\Delta t) + \frac{\bvec f(t-\Delta t)}{m}(\Delta t)^2 \qquad t>0\\
\bvec r (0) &= \bvec r_0\\
\bvec r (- \Delta t) &= \bvec r_0 - \bvec v_0 \Delta t + \frac{\bvec f(0)}{2m}(\Delta t)^2\\
\end{aligned}
\end{equation}

其中,$\bvec r_0, \bvec v_0$分别为初始时刻$t=0$粒子的位置与速度,是需要给定的已知量;$\bvec f = \bvec f(t)$是$t$时刻粒子的受力。

以下是使用Verlet算法模拟单粒子圆周运动的octave/matlab代码。
\begin{lstlisting}[language=matlab]
clc
clear

function f = centriforce(r) %有心力,例如引力、点电荷的静电力
  mag = 1/(norm(r)^2);
  f = -mag*r/norm(r);
end

r0=[10;0]; %点粒子的初始位置
v0=[0;2]; %点粒子的初始速度。v0=[0;sqrt(10)]时为匀速率圆周运动
dt = 0.01;

R0 = r0 - v0*dt + 100*centriforce(r0)/2*dt^2;
R1 = r0;
R2 = [0;0];
i=0;

while 1
  R2 = 2*R1 - R0 + 100*centriforce(R1)*dt^2;
  if mod(i,5) == 0
    clf
    hold on
    scatter(0,0);
    scatter(R2(1),R2(2));
    axis equal
    axis([-10 10 -10 10])
    hold off
    pause(0.05);
    i = 0;
  endif

  R0 = R1;
  R1 = R2;
  R2 = [0;0];
  i=i+1;
end

\end{lstlisting}
