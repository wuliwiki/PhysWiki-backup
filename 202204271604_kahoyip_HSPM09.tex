% 机械振动(高中)
% 机械振动|弹簧振子|简谐运动|单摆|共振

\begin{issues}
\issueDraft
\issueTODO
\end{issues}

物体或物体的一部分在某个位置附近所做的往复运动叫做\textbf{机械振动},简称\textbf{振动}.

\subsection{简谐运动}

由弹簧和小球组成的系统,叫做\textbf{弹簧振子},其中的小球叫做\textbf{振子}.弹簧振子是一种理想模型,研究其运动时,小球被视为质点,并忽略弹簧的质量以及运动过程中的阻力.在弹簧振子的运动过程中,系统的机械能守恒.

\begin{figure}[ht]
\centering
\includegraphics[width=5cm]{./figures/HSPM09_1.png}
\caption{弹簧振子} \label{HSPM09_fig1}
\end{figure}

\autoref{HSPM09_fig1} 为安置在光滑水平面的弹簧振子,弹簧的一端被固定.弹簧处于自然状态时,振子静止,所受合力为零,此时振子所处的位置叫\textbf{平衡位置}.

当沿水平方向拉动(或推动)振子使其偏离平衡位置并释放,振子将在平衡位置的两侧做往复运动(振动).振动过程中,振子在竖直方向上所受合力为零,在水平方向上受到弹簧弹力 $\bvec F$ 的作用.弹力 $\bvec F$ 的方向与偏离平衡位置的位移 $\bvec x$ 的方向相反,总是指向平衡位置,其作用是将振子拉回平衡位置,这个力 $\bvec F$ 叫做\textbf{回复力}.以\autoref{HSPM09_fig1} 向右为正方向,根据胡克定律可知:
\begin{equation}\label{HSPM09_eq1}
F=-kx
\end{equation}
式中的负号表示回复力的方向与振子偏离平衡位置的位移方向相反.

物体在满足\autoref{HSPM09_eq1} 的回复力作用下发生的运动,叫做\textbf{简谐运动}(\textbf{简谐振动}),其位移与时间的关系遵循正弦或余弦函数的规律.简谐运动的位移—时间表达式为:
\begin{equation}\label{HSPM09_eq2}
x=A\cos(\omega t + \varphi_0)
\end{equation}
$A$ 为振幅,表示振子偏离平衡距离的最大距离;$\omega$ 为角频率,表示简谐运动物体振动的快慢;$\varphi_0$ 为初相位,表示 $t=0$ 时,简谐运动物体所处的状态.

由\autoref{HSPM09_eq2} 可知简谐运动是一个周期性往复运动.振子经历 $A$→$O$→$A'$→$O$→$A$ 这样一个完整的振动过程,叫\textbf{全振动}.

\begin{figure}[ht]
\centering
\includegraphics[width=7.2cm]{./figures/HSPM09_2.png}
\caption{弹簧振子运动过程的三个特殊位置} \label{HSPM09_fig2}
\end{figure}

\begin{table}[ht]
\centering
\caption{一次全振动的物理量变化规律}\label{HSPM09_tab1}
\begin{tabular}{|c|c|}
\hline
过程或位置 & 各物理量的变化情况 \\
\hline
$A$ & $\bvec x$ 向右,达最大值;$\bvec F$ 及 $\bvec a$ 向左,达最大值;$E_k$ 为零,$E_p$ 达最大值 \\
\hline
$A$→$O$ & $\bvec x$ 向右,减小;$\bvec F$ 及 $\bvec a$ 向左,减小;$\bvec v$ 向左,增大;$E_k$ 增大,$E_p$ 减小 \\
\hline
首次经过 $O$ & $\bvec x$、$\bvec F$ 及 $\bvec a$ 为零;$\bvec v$ 向左,达最大值;$E_k$ 达最大值,$E_p$ 为零 \\
\hline
$O$→$A'$ & $\bvec x$ 向左,增大;$\bvec F$ 及 $\bvec a$ 向右,增大;$\bvec v$ 向左,减小;$E_k$ 减小,$E_p$ 增大 \\
\hline
$A'$ & $\bvec x$ 向左,达最大值;$\bvec F$ 及 $\bvec a$ 向右,达最大值;$E_k$ 为零,$E_p$ 达最大值 \\
\hline
$A'$→$O$ & $\bvec x$ 向左,减小;$\bvec F$ 及 $\bvec a$ 向右,减小;$\bvec v$ 向右,增大;$E_k$ 增大,$E_p$ 减小 \\
\hline
第二次经过 $O$ & $\bvec x$、$\bvec F$ 及 $\bvec a$ 为零;$\bvec v$ 向右,达最大值;$E_k$ 达最大值,$E_p$ 为零 \\
\hline
$O$→$A$ & $\bvec x$ 向右,增大;$\bvec F$ 及 $\bvec a$ 向左,增大;$\bvec v$ 向右,减小;$E_k$ 减小,$E_p$ 增大 \\
\hline
\end{tabular}
\end{table}
$\bvec x$ 为振子偏离平衡位置的位移,$\bvec F$ 及 $\bvec a$ 分别为振子所受回复力及加速度,$\bvec v$ 为振子的运动速度,$E_k$ 和 $E_p$ 分别对应振子的动能和弹簧的势能.

\subsection{匀速圆周运动和简谐运动的联系}

质量为 $m$ 的质点绕点 $O$ 做半径为 $A$、角速度为 $\omega$ 的匀速圆周运动(\autoref{HSPM09_fig4} ),$t=0$ 时质点与点 $O$ 的连线与 $x$ 轴的夹角为 $\varphi_0$.质点相对圆心 $O$ 的位移在 $x$ 轴上的投影为 $x=A\cos(\omega t + \varphi_0)$,可见做匀速圆周运动的质点在直径上的投影的运动是简谐运动.

\begin{figure}[ht]
\centering
\includegraphics[width=7cm]{./figures/HSPM09_4.png}
\caption{匀速圆周运动} \label{HSPM09_fig4}
\end{figure}

质点在上述匀速圆周运动的向心加速度大小为 $a=\omega^2 A$,向心加速度在 $x$ 轴的投影为 $a_x=-\omega^2 A\cos(\omega t + \varphi_0)$,即为简谐运动的加速度,则简谐运动的回复力可表示为
\begin{equation}\label{HSPM09_eq3}
F_x=ma_x=-m\omega^2 A\cos(\omega t + \varphi_0)= -m\omega^2 x
\end{equation}

结合\autoref{HSPM09_eq1} 和\autoref{HSPM09_eq3} 可知,在简谐运动中有
\begin{equation}
\omega^2=\frac km
\end{equation}

简谐运动的振动周期
\begin{equation}\label{HSPM09_eq4}
T=\frac {2\pi}{\omega}=2\pi\sqrt{\frac mk}
\end{equation}


\subsection{单摆}

一根质量和伸缩量可以忽略不计的细绳(摆线),上端固定,下端系着一个看作质点的物体(摆球),就构成了一个\textbf{单摆}.单摆是实际摆的理想模型,摆球的运动是以摆线固定点为圆心的变速圆周运动,也是在平衡位置\footnote{单摆自然悬挂时摆球的位置,单摆运动过程中的最低点}两侧的周期性往复振动,运动过程中机械能守恒.

单摆运动时的受力分析如\autoref{HSPM09_fig3} 所示,摆球受到了重力和摆线拉力作用.摆线的拉力和摆球重力沿径向的分力的合力 $T-mg\cos\theta$ 提供了摆球做圆周运动的向心力,摆球重力沿圆弧切线方向的分力 $mg\sin\theta$ 提供了摆球振动的回复力.

\begin{figure}[ht]
\centering
\includegraphics[width=7cm]{./figures/HSPM09_3.png}
\caption{单摆受力分析} \label{HSPM09_fig3}
\end{figure}

当摆角 $\theta$ 很小时,有 $\sin\theta \approx \theta$\footnote{参考:小角正弦极限(简明微积分)\upref{LimArc}},摆球的位移大小 $x$ 近似于 $L\theta$,可得回复力 $F \approx -mgx/L$,符合\autoref{HSPM09_eq1} 的形式,此时可把单摆的摆动近似看成简谐运动,将 $k=mg/L$ 代入\autoref{HSPM09_eq4} 可得单摆做简谐运动的周期为
\begin{equation}
T=2\pi\sqrt{\frac Lg}
\end{equation}

\subsection{受迫振动和共振}

\addTODO{补充振动图像及共振曲线}

从上述内容可以知道,弹簧振子和单摆在不受外力时做简谐运动,其振动周期和频率只与自身的性质有关,这种振动系统不受外力的振动叫做\textbf{固有振动},对应的振动频率叫做\textbf{固有频率}.

实际的振动会受到阻力的作用,系统因克服阻力做功而消耗机械能,振幅逐渐减小.振幅随时间减小的振动,叫做\textbf{阻尼振动}.阻尼越大,振幅减小得越快,当阻尼过大时,系统将不能发生振动.

\begin{figure}[ht]
\centering
\includegraphics[width=8.5cm]{./figures/HSPM09_5.png}
\caption{阻尼振动曲线} \label{HSPM09_fig5}
\end{figure}


在实际的振动中,为了保持振幅不变,我们通常会给系统施加一个周期性的外界驱动力,由于外界驱动力的作用,系统机械能得以补充,并持续振动.系统在周期性的外界驱动力作用下的振动,叫做\textbf{受迫振动}.当系统所做的受迫振动达稳定后,其振动频率等于驱动力的频率,与固有频率无关.

当驱动力的频率等于系统的固有频率时,系统所做受迫振动的振幅最大,这种现象叫做\textbf{共振},如音叉共鸣实验所演示的就是声音的共振现象.