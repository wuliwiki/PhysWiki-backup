% “电池效率”

\footnote{本文参考了Schroeder的《热物理学导论》与刘俊吉等的《物理化学》。本文遵循CC-BY。}类似于将热能转换为功的热机,将化学能转换为电能的电池有没有效率问题呢?我们先思考一个简单的氢-氧电池模型。和卡诺热机一样,此处的电池模型是一个高度简化的模型,并不涉及任何技术细节。

\subsection{题设}
氢-氧电池中的化学反应是
$$H_2+ 1/2 O_2 \Rightarrow H_2O$$
,且假定每消耗$1 \Si{mol} H_2$ 与 $1/2 \Si{mol} O_2$并生成$1 \Si{mol} H_2O$ 时,系统相应的状态量变化是
$$\Delta U = -282 \Si{kJ}$$
$$\Delta S = -0.16 \Si{kJ/K}$$
(由于气体比液体无序,因此从气体变为液体的过程自然是熵降低的过程)
。系统的工作环境是
$$p=1 atm, T=303K$$。
为使问题更简单,我们假设系统整体与环境始终等温等压、且发生可逆的化学反应;此外,为了避免处理复杂的化学势问题,我们再假定系统足够大,以至于发生少量反应时系统的整体性质不变。

\subsection{分析}
\begin{figure}[ht]
\centering
\includegraphics[width=8cm]{./figures/BTEFF1_1.pdf}
\caption{能量的转换} \label{BTEFF1_fig1}
\end{figure}
我们先根据热力学第一定律与上述题设,列出电池反应前后能量的总体变化:
$$
\Delta U = \delta q - \delta w_\text{电} - \delta w_\text{气}
$$

\begin{itemize}
\item $\Delta U$是系统由于氢氧反应生成水而产生的内能变化,在上文已给出。
\item $\delta q$是系统对环境的放热。根据热力学第二定律$\Delta S = \delta q/T$,所以系统向环境放热 $\delta q = T \Delta S = -49 \Si{kJ}$。
\item $\delta w_\text{气}$是外界气压对系统做的功。在消耗气体、生成液体的过程中,系统体积减小,因此环境压力将对系统做压缩功 $\delta w_\text{气} = - p \Delta V = -\Delta n R T = -4\Si{kJ}$。负号是由于“环境对系统做功取为负”的符号规范。
\item 剩余的能量自然是系统对环境做的电功 $\delta w_\text{电} = \delta q - \delta w_\text{气} - \Delta U = 237 \Si{kJ}$.
\end{itemize}

这个反应后,系统的内能减少了 $\Delta U =  -282 \Si{kJ}$,而系统却只对环境做了 $w_\text{电} = 237 \Si{kJ}$ 的电功。可见,电池也不能将所有的能量(化学能)转换为电能,也和热机一样受到热力学第二定律的约束(\textsl{诅咒})。
唯象地说,这不可避免的转换损耗问题导致了电池的\textsl{内阻}。

如果我们认为“电池效率”是“对外做功”比上“内能消耗”\footnote{在实操上(见参考书目),我们一般使用$\Delta H = \Delta U + p\Delta V$作为分母,因为“环境对系统的压缩功是自发的”。在其余工程领域,“电池效率”可能不是这么定义的。},那么这个电池的效率是
$$\eta = \abs{\delta w_\text{电}/\Delta U} = 84\%$$

由于我们做出了大量理想假设,因此电池的实际效率不会比这个更高。

\subsubsection{另一个角度}
我们换一个角度,再次思考电池效率:\textsl{我们能不能使用相同的电池反应,但是改进电池的设计,以令电池不产热、并使所有的能量都变成电能呢?} 

答案显然是否定的,论证方法也就是反过来\textsl{复读}一遍上述的分析:如果系统不产热$\delta q = 0$,那么 $\delta q/ T = 0$。而根据热力学第二定律,任意过程后有$\Delta S \ge \delta q/ T = 0$。而由题设我们知道,这个反应的$\Delta S \le 0$。也就是说,这样的过程违反了热力学第二定律,也就不会发生。
