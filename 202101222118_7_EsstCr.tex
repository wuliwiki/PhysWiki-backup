% 爱森斯坦判别式
% 多项式|Eisenstein|不可约多项式
\pentry{素理想与极大理想\upref{Ideals}}

爱森斯坦判别式可以用来判断多项式是否可约.满足爱森斯坦判别式条件的多项式必然是不可约的,即无法表示为两个多项式的乘积.

最简单的多项式是整系数多项式,也被称为\textbf{整数环上的}多项式,因此我们先介绍此类多项式上的爱森斯坦判别式.这只是更一般的爱森斯坦判别式的特例.

\begin{theorem}{整数环上的爱森斯坦判别式}
设有整系数多项式$f(x)=f_0+f_1x+f_2x^2+\cdots+f_nx^n$,若存在素数$p$使得:
\begin{enumarate}
\item $p\nmid f_n$;
\item 对于$i\in\{0, 1, 2, \cdots, n-1\}$,都有$p|f_i$;
\item $p^2|f_0$.
\end{enumarate}
那么$f(x)$就是不可约多项式.
\end{theorem}

\textbf{证明}:

反设$f(x)$可约,则可以写成$f(x)=(h_0+h_1x+h_2x^2+\cdots+h_rx^r)(g_0+g_1x+g_2x^2+\cdots+g_sx^s)$.

由于$f(x)=f_0+f_1x+f_2x^2+\cdots+f_nx^n$,故\footnote{这种运算也被称为卷积.}\begin{equation}f_i=\sum\limits_{j+k=i}h_jg_k\end{equation}

注意到$f_0=h_0g_0$,而$p^2\nmid f_0=h_0g_0$,故$h_0$和$g_0$中有一个是不能被$p$整除的.不妨设$p\nmid h_0$.




\textbf{证毕}.







