% 芝诺时
% 芝诺
\subsection{芝诺时与普通时}

\subsubsection{佯谬}
在一些定性的结论和定理未曾创立前,人们没有办法去理解和解决实际上的一些问题,在一些人的助推波澜下,人们的思想越来越混乱.在微积分发明之前,古代哲学家\textbf{芝诺}曾经提出过一系列的佯谬其中最为著名的就是他关于时间的悖论.

\subsubsection{芝诺的论点和论据}
论点:阿克琉斯(希腊神话中的英雄)的速度是乌龟的十倍,乌龟在他之前100米,但是他永远追不上乌龟.
论据:因为在他们开始赛跑时,乌龟在阿克琉斯前面100米,那么当阿克琉斯跑了100米时到乌龟原来的位置时,乌龟已经前进了10米;当阿克琉斯再前进10米时,乌龟又在他前面1米;而当他再跑1米时;乌龟又向前运动了  0.1米…… 如此下去,直至无穷.因此,阿克琉斯永远追不上乌龟.

\subsubsection{实际上的解释}
在我们现在看来,这个结论是错误的.因此,错误一定处在论证过程上.而错在哪里呢?仔细思考后,不难发现,芝诺实际上采取了与我们平时所用的时间量度不同的另一种量度.在芝诺时间里,本来有限的时间被分成了无限多份.但是这并不意味着时间有无限的数量.也就是说,在芝诺的时间取到无穷后,还是有时间存在的.换句话说,一种到达无限的时间量度,在其他量度上,有可能是有限的.

\subsubsection{推导和证明}
在思考到芝诺时间实际上是另一种时间量度后,物理学家们思考的是时间量度之间的转换:我们平时使用的时间量度与芝诺时间之间.有什么样的转换关系呢?

我们在这里,设我们日常使用的普通时间为$t$,芝诺时间为为$t'$,阿克琉斯的速度为$v_1$,乌龟的速度为$v_2$