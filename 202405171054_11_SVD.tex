% 奇异值分解(SVD)
% keys SVD|singular value decomposition|奇异值分解|正交矩阵|正交变换|线性代数|对角矩阵
% license Xiao
% type Tutor

\begin{issues}
\issueTODO 引用酉矩阵对角化定理。
\end{issues}



本节默认在复数域下讨论。
\subsection{定理陈述}
\begin{theorem}{SVD}
给定复数域$\mathbb{C}$上的$n$行$m$列矩阵$\bvec{M}$,则存在$n$行$m$列对角矩阵$\bvec{D}$和酉矩阵$\bvec{P}$、$\bvec{Q}$,使得
\begin{equation}
\bvec{M} = \bvec{PDQ}~. 
\end{equation}
称$\bvec{D}$的对角元为$\bvec{M}$的\textbf{奇异值(singular value)},$\bvec{PDQ}$为$\bvec{M}$的\textbf{奇异值分解(singular value decomposition)},简称为\textbf{SVD}。
\end{theorem}


\subsection{证明过程}
我们先从\textbf{可逆方阵}证起,然后再过渡到不要求可逆的\textbf{任意复矩阵}。
\begin{theorem}{}
对于$n$阶\textbf{可逆方阵}$M$,存在\textbf{酉方阵}$P,Q$和\textbf{可逆对角矩阵}$D$使得
\begin{equation}\label{eq_SVD_1}
M=P^{\dagger}DQ~.
\end{equation}

\end{theorem}
\textbf{证明:}因为$M^{\dagger}M$是半正定厄米矩阵,必存在一组标准正交基使之对角化,设为$A$,对应的过渡矩阵为酉方阵$Q$,则该过程表示为$QM^{\dagger}MQ^{\dagger}=A$。左右取行列式可证$A$是可逆矩阵。在这组基下,对对角元计算后,我们可以得到$A^{\frac{1}{2}},A^{-\frac{1}{2}}$,显然互逆。

我们可以把$M$表示为$MQ^{\dagger}A^{-\frac{1}{2}}A^{\frac{1}{2}}Q$,和\autoref{eq_SVD_1} 比较,我们需要验证$MQ^{\dagger}A^{-\frac{1}{2}}$是酉方阵。

设$P^{\dagger}=MQ^{\dagger}A^{-\frac{1}{2}}$,则$P=A^{-\frac{1}{2}}QM^{\dagger}$,则
\begin{equation}
\begin{aligned}
P^{\dagger}P&=MQ^{\dagger}A^{-\frac{1}{2}}A^{-\frac{1}{2}}QM^{\dagger}\\
&=MQ^{\dagger}(Q^{\dagger})^{-1}(M)^{-1}(M^{\dagger})^{-1}Q^{-1}QM^{\dagger}\\
&=E~,
\end{aligned}
~.
\end{equation}
ding得证。


\begin{lemma}{}
对于任意矩阵$M$,存在酉矩阵$Q,S$和\textbf{满秩}矩阵$N$,使得
\begin{equation}
M=S^{\dagger}\left(\begin{array}{cc}
N & 0 \\
0 & 0
\end{array}\right) Q~.
\end{equation}
\end{lemma}

















