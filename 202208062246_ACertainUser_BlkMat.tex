% 分块矩阵

\begin{issues}
\issueDraft
\end{issues}
\footnote{本文参考了Steven J. Leon的 Linear Algebra with Applications.}

\subsection{分块矩阵 Partitioned Matrices}

有时,把一个大矩阵$\mat M$视为若干个小矩阵的“组合”可以帮助简化问题,例如:
\begin{equation}
\mat M = 
\begin{bmatrix}
\mat A & \mat B\\
\mat C & \mat D\\
\end{bmatrix}
\end{equation}
其中 $\mat A, \mat B, \mat C, \mat D$都是矩阵.这些子矩阵可以不是方阵,但这种划分必须“有意义”,即$\mat A$与$\mat C$的行数相同、$\mat A$与$\mat B$的列数相同,等等.

\subsubsection{分块矩阵的加法、乘法}
若相应的计算都有定义,那么分块矩阵的加法、乘法法则\upref{Mat}与普通矩阵形式上完全类似,例如:
\begin{equation}
\begin{bmatrix}
\mat A & \mat B\\
\mat C & \mat D\\
\end{bmatrix}
+
\begin{bmatrix}
\mat E & \mat F\\
\mat G & \mat H\\
\end{bmatrix}
=
\begin{bmatrix}
\mat A +\mat E & \mat B +\mat F\\
\mat C +\mat G & \mat D +\mat H\\
\end{bmatrix}
\end{equation}

\begin{equation}
\begin{bmatrix}
\mat A & \mat B\\
\mat C & \mat D\\
\end{bmatrix}
\begin{bmatrix}
\mat E & \mat F\\
\mat G & \mat H\\
\end{bmatrix}
=
\begin{bmatrix}
\mat A \mat E +\mat B \mat G & \mat A \mat F +\mat B \mat H \\
\mat C \mat E +\mat D \mat G  & \mat C \mat F +\mat D \mat H \\
\end{bmatrix}
\end{equation}

\begin{equation}
\mat M
\begin{bmatrix}
\mat A & \mat B\\
\mat C & \mat D\\
\end{bmatrix}
=
\begin{bmatrix}
\mat M \mat A & \mat M \mat B\\
\mat M \mat C & \mat M \mat D\\
\end{bmatrix}
\end{equation}

%相乘时可以把每块看成一个元素, 元素之间的乘法就是块的矩阵乘法.

\subsection{块对角矩阵}
%是这个吗?
定义块对角矩阵: 只有对角块为非零的矩阵,例如:
\begin{equation}
\mat M = 
\begin{bmatrix}
\mat A & \mat O\\
\mat O & \mat B\\
\end{bmatrix}
\end{equation}

一般讨论对称的块对角矩阵: $\bvec y_i$ 和 $\bvec x_i$ 的长度和划分相同.(各分块均为方阵)

在每个子空间中分别映射, 所以可以对每块分别处理. 例如计算本征问题时.

\subsubsection{块对角矩阵的特殊性质}
块对角矩阵有一些特别的性质,包括:
\begin{itemize}
\item $\mat M ^ {-1} = 
\begin{bmatrix}
\mat A^{-1} & \mat O\\
\mat O & \mat B^{-1}\\
\end{bmatrix}
$,可以直接运用逆矩阵\upref{InvMat}的定义证明
\item ...
\end{itemize}
