% 函数的极限
% keys 函数极限

\pentry{序列的极限\upref{SeqLim}}

回顾:函数是一个非空集合 $A$ 到另一个集合 $B$ 的**对应法则**.

本词条中,函数 $f$ 是指从 $\mathbb R$ 的某个非空子集 $X$ 到 $\mathbb{R}$ 的映射  

\subsection{函数的极限,左极限,右极限}
\begin{definition}{邻域和去心邻域}
  定义\textbf{邻域}: $U(x_0,\delta)=\{x\in \mathbb{R}:|x-x_0|<\delta\}$.

  \textbf{去心邻域}: $U_0(x_0,\delta)= U(x_0,\delta) \backslash \{x_0\}=\{x\in \mathbb{R}:0<|x-x_0|<\delta\}$.
\end{definition}
\begin{definition}{极限}
 设函数 $f(x)$ 在 $U_0(x_0,\delta_0)(\delta_0>0)$ 内有定义.

  若存在实数 $A$ ,使得对任意 $\epsilon >0$,存在 $\delta>0$,使得当 $x\in U_0(x_0,\delta)$时,有 $|f(x)-A|<\epsilon$,则称\textbf{当 $x$ 趋于 $x_0$,函数 $f(x)$ 以 $A$ 为极限},记为 $\large \lim\limits_{x\rightarrow x_0}f(x)=A$ 或 $f(x)\rightarrow A\ (x\rightarrow x_0)$.
\end{definition}
PS:实际上有更宽泛的定义,只要 $x_0$ 是函数 $f(x)$ 定义域的\textbf{聚点},就可以定义在该点处的极限.

同序列极限的性质类似,函数极限也具有唯一性:
\begin{theorem}{}
  若函数 $f(x)$ 在 $x_0$ 处极限存在,证明在 $x_0$ 处极限唯一.
\end{theorem}
\begin{exercise}{}
\begin{enumerate}
\item  $f(x)=\left\{\begin{aligned} 0\ \ &(x<1)\\ 1\ \ &(x\ge 1) \end{aligned}\right.$ ,判断 $f(x)$ 在 $x_0=1$ 处极限是否存在.
\item $f(x)=\left\{\begin{aligned} 0\ \ &(x<1)\\ 1\ \ &(x= 1)\\2\ \ &(x>1) \end{aligned}\right.$,判断 $f(x)$ 在 $x_0=1$ 处极限是否存在.
\item $f(x)=x\cdot \sin(1/x)$,判断 $f(x)$ 在 $x_0=0$ 处极限是否存在.
\item $f(x)=\sin(1/x)$,判断 $f(x)$ 在 $x_0=0$ 处极限是否存在.
\item $f(x)=e^x$,证明 $\large \lim\limits_{x\rightarrow a}=e^a,\forall a\in\mathbb{R}$.
\end{enumerate}
\end{exercise}


  ![pic2](E:\pku\Eden文集\Eden数学文集\course\数分\函数的极限与连续性\pic2.png)

  我们来看一个有趣的函数 $f(x)$,它的定义域为 $[0,1]$:
  $$
  f(x)=\left\{\begin{aligned} &1/q, &x=\frac{p}{q}\ (p,q\in \mathbb{N},\frac{p}{q}为既约真分数)\\
  &0,&x=0或x=1或x\notin \mathbb{Q}
  \end{aligned} \right.
  $$
  我们称它为**黎曼 (Riemann) 函数**.

  虽然在定义域内有无穷多个点的函数值不为 $0$,但 $f(x)$ 的极限却处处为 $0$,我们之后还将看到,$f(x)$ 在无理点处处连续,但 $f(x)$ 处处不可导.

  ##### Question:

  1. 证明:若函数 $f(x)$ 在 $U(a,\delta_0)(\delta_0>0)$ 上有定义,且满足 $\large\lim\limits_{x\rightarrow a}f(x)=f(a)$,那么对任意极限为 $a$ 的序列 $\{x_n\}$ ,序列 $\{f(x_n)\}$ 的极限也为 $f(a)$.
     上述命题反过来也成立.

  2. 对于任意给定的序列 $\{a_n\}(0<a_n<1)$,构造定义域为 $[0,1]$ 的函数 $f(x)$,满足 $\forall x\in \{a_n\},f(x)\neq 0;\ \forall x \in [0,1]\backslash \{a_n\},f(x)=0$,且 $f(x)$ 在定义域上极限处处为 $0$.

  

- #### 左右极限

  ![pic3](E:\pku\Eden文集\Eden数学文集\course\数分\函数的极限与连续性\pic3.png)

  如果把去心邻域 $U_0(x_0,\delta_0)$ 分成两块**单侧邻域**——

  左空心邻域:$U_0^+(x_0,\delta)=U_0(x_0,\delta)\cap (x_0,+\infty) \{x\in \mathbb{R} :x_0< x<x_0+\delta\}$

  右空心邻域:$U_0^-(x_0,\delta)=U_0(x_0,\delta)\cap (-\infty,x_0) \{x\in \mathbb{R} :x_0-\delta< x<x_0\}$

  那么就可以定义函数的左右极限:

  设 $f(x)$ 在 $U^+_0(x_0,\delta_0)(\delta_0>0)$ 上有定义.

  如果存在实数 $A$,使得对任意 $\epsilon >0$,存在 $\delta>0$,当 $x\in U_0^+(x_0,\delta)$ 时,有 $|f(x)-A|<\epsilon$,则称 $f(x)$ 在点 $x_0$ 的**右极限存在**,而称 $A$ 为 $f(x)$ 在点 $x_0$ 的**右极限**,记为 $\lim\limits_{x\rightarrow x_0^+}f(x)=A$ 或 $f(x_0^+)=A$.

  类似地可以定义**左极限存在**和**左极限**.

  ##### Question:

  1. $f(x)=\left\{\begin{aligned} 0\ \ &(x<1)\\ 1\ \ &(x= 1)\\2\ \ &(x>1) \end{aligned}\right.$,判断 $f(x)$ 在 $x_0=1$ 处的左极限与右极限.

  2. $f(x)=[x]$(取整函数),判断 $f(x)$ 在 $x_0=1$ 处的左极限与右极限.

  3. 设函数 $f(x)$ 在 $U_0(x_0,\delta_0)$ 上有定义,证明: $\lim\limits_{x\rightarrow x_0} f(x)=A$ 当且仅当 $\lim\limits_{x\rightarrow x_0^-}f(x)=\lim\limits_{x\rightarrow x_0^+}f(x)=A$.

     

- #### 极限的各种情况

  称集合 $\{x:|x|>h\}(h>0)$ 为 $\infty$ 的邻域,记为 $U(\infty,h)$ (这时就没有必要定义去心邻域了).

  同样的可以分成两块**单侧邻域**:

   $U^+(\infty,h)=U(\infty,h)\cup (0,\infty)=\{x:x>h\}$ 

   $U^-(\infty,h)=U(\infty,h)\cup (-\infty,0)=\{x:x<-h\}$ 

  这样就可以函数在自变量趋向于无穷大时的极限:

  设函数 $f(x)$ 在 $U^+(\infty,h_0)$ 上有定义.若存在实数 $A$,使得 $\forall \epsilon >0, \exist X\in U^+(\infty,h_0)$,当 $x>X$ 时,有 $|f(x)-A|<\epsilon$,则称当 $x$ 趋于 $+\infty$ 时 $f(x)$ 的**极限存在**,其极限为 $A$,记为 $\lim\limits_{x\rightarrow +\infty}f(x)=A$ 或 $f(x)\rightarrow A(x\rightarrow +\infty)$.

  类似地可以定义 $\lim\limits_{x\rightarrow -\infty}f(x)=A$ 和 $\lim\limits_{x\rightarrow \infty}f(x)=A$ .

  ##### Question:

  1. 设函数 $f(x)$ 在 $U(\infty,h_0)$ 上有定义,证明: $\lim\limits_{x\rightarrow \infty} f(x)=A$ 当且仅当 $\lim\limits_{x\rightarrow -\infty}f(x)=\lim\limits_{x\rightarrow +\infty}f(x)=A$.
  2. 设序列 $\{a_n\}$ 收敛于 $A$,定义函数 $f(x)=a_{|[x]+1|}$,证明:$\lim\limits_{x\rightarrow \infty} f(x)=A$.

- #### 广义极限

  如果自变量趋向于一个值时,函数趋向于无穷大,则可以定义广义极限:

  设 $f(x)$ 在 $U_0(x_0,\delta_0)(\delta_0>0)$ 上有定义,若 $\forall M>0,\exist \delta>0$,使得当 $x\in U_0(x,\delta)$ 时,有 $f(x)>M$,则称当 $x$ 趋于 $x_0$,$f(x)$ 趋于 $+\infty$,或称当 $x$ 趋于 $x_0$ 时,$f(x)$ 的**广义极限**为 $+\infty$.记为 $\lim\limits_{x\rightarrow x_0}f(x)=+\infty$ 或 $f(x)\rightarrow +\infty(x\rightarrow x_0)$.此时也称 $f(x)$ 为当 $x$ 趋于 $x_0$ 时的**正无穷大量**.

  同样地可以定义**负无穷大量**和**无穷大量**.

  对于函数极限而言,自变量有以下几种变化情况:
  $$
  x\rightarrow x_0;\ x\rightarrow x_0^+;\ x\rightarrow x_0^{-};\ x\rightarrow \infty;\ x\rightarrow +\infty;\ x\rightarrow -\infty
  $$
  对于这六种情况可以定义各自的**广义极限**:
  $$
  f(x)\rightarrow A;\ f(x)\rightarrow +\infty;\ f(x)\rightarrow -\infty;\ f(x)\rightarrow \infty
  $$
  因此一共有 $24$ 种可能的函数极限的情形.

  但由于它们之间有极大的相似之处,所以很容易进行记忆和想象.

  ##### Question:

  1. $f(x)=x^2\sin x$,判断它是不是当 $x\rightarrow \infty$ 时的无穷大量.

  2. $f(x)=1/(x-1)\ (x\neq 1)$,求 $f(1^-),f(1^+),\lim\limits_{x\rightarrow 1}f(x),\lim\limits_{x\rightarrow \infty}f(x)$.

     

### 2. 函数极限的性质

- #### 唯一性

  若函数 $f(x)$ 在 $x_0$ 处极限存在,则在 $x_0$ 处极限唯一.

- #### 局部保序性

  设函数 $f(x),g(x)$ 在 $x_0$ 处极限存在,若 $f(x)\le g(x), \forall x\in U_0(x_0,\delta_0)$,那么 $\lim\limits_{x\rightarrow x_0} f(x)\le \lim\limits_{x\rightarrow x_0}g(x)$.

- #### 局部保号性

  设函数 $f(x)$,若 $\lim\limits_{x\rightarrow x_0}f(x)=A>0$,

  那么存在 $x_0$ 的一个去心邻域 $U_0(x_0,\delta)$,满足 $\forall x\in U_0(x_0,\delta)$,$f(x)>0$.

- #### 局部有界性

  设函数 $f(x)$,$\lim\limits_{x\rightarrow x_0}f(x)$ 存在(不为无穷大量),

  那么存在 $x_0$ 的一个去心邻域  $U_0(x_0,\delta)$,满足 $\exist M>0$,$\forall x\in U_0(x_0,\delta)$,$|f(x)|<M$,即 $f(x)$ 在 $U_0(x_0,\delta)$ 上有界.

- #### 函数极限的四则运算

  设函数 $f(x),g(x)$,分别对于六种自变量的变化情况
  $$
  x\rightarrow x_0;\ x\rightarrow x_0^+;\ x\rightarrow x_0^{-};\ x\rightarrow \infty;\ x\rightarrow +\infty;\ x\rightarrow -\infty
  $$
  若 $f(x)\rightarrow A,\ g(x)\rightarrow B$,则容易证明:
  $$
  \begin{aligned}
  &h_1(x)=f(x)+g(x)\rightarrow A+B\\
  &h_2(x)=f(x)-g(x)\rightarrow A-B\\
  &h_3(x)=f(x)\cdot g(x)\rightarrow A\cdot B\ (A\neq 0,B\neq 0)\\
  &h_4(x)=f(x)/ g(x)\rightarrow A/B\ (B\neq 0)
  \end{aligned}
  $$
  若广义极限 $A,B$ 为无穷大量,则可以规定一些特殊的四则运算,例如 $(+\infty)+(+\infty)=+\infty,\ 
  (+\infty)\cdot (+\infty)=+\infty$ 等等.

  ##### Question:

  1. 设函数 $f(x)$,若 $\lim\limits_{x\rightarrow x_0}f(x)=A$,证明:对于任意 $r<A$,存在 $x_0$ 的一个去心邻域 $U_0(x_0,\delta)$,满足 $\forall x\in U_0(x_0,\delta)$,$f(x)>r$.(特别地,当 $r=0$ 时为局部保号性)
  2. 求 $\lim\limits_{x\rightarrow +\infty}(x^2+1)/(1-2x^2)$.
  3. 设函数 $f(x),g(x)$,若 $f(x)$ 在 $x_0=0$ 处极限为 $0$,而 $h(x)=f(x)/g(x)$ 在 $x_0=0$ 处极限为 $1$,证明 $g(x)$ 在 $x_0=0$ 处极限存在且也为 $0$.

- #### 夹逼收敛原理

  设函数 $f(x),h(x),g(x)$,

  若 $f(x)\le h(x)\le g(x),\ \forall x\in U_0(x_0,\delta_0)$,且 $\lim\limits_{x\rightarrow x_0}f(x)=\lim\limits_{x\rightarrow x_0} g(x)=A$,那么 $\lim\limits_{x\rightarrow x_0}h(x)=A$.

- #### 复合函数的极限?

  ##### Question:

  1. 函数 $f(x)$ 在 $U(a,\delta_0)$ 上有定义,序列 $\{x_n\}$ 收敛于 $a$,则什么情况下 $\lim\limits_{n\rightarrow \infty}f(x_n)=f(a)$ 成立?

  设 $f(x)=1/x,\ g(x)=x^2$

  容易证明 $\lim\limits_{x\rightarrow 4}f(x)=1/4,\lim\limits_{x\rightarrow 2}g(x)=4$.

  那么是否 $\lim\limits_{x\rightarrow 2}f(g(x))=f(\lim\limits_{x\rightarrow 2}g(x))=f(4)=1/4$ 呢?经验证是成立的.

  我们自然地就想到,是否对于任何函数 $f(x),g(x)$,都满足 $\lim\limits_{x\rightarrow a}f(g(x))=f(\lim\limits_{x\rightarrow a}g(x))$.然而答案是否定的.这种复合函数的极限运算需要满足一些限制条件,这将在下一节提到.

  ##### Question:

  1. $f(x)=[x],g(x)=1-x^2$,判断 $\lim\limits_{x\rightarrow 0}f(g(x)) $ 与 $f(\lim\limits_{x\rightarrow 0}g(x))$ 是否相等.