% 映射
% keys 单射|满射|函数|一一对应|定义域|值域|逆映射
% license Xiao
% type Tutor

\pentry{集合\nref{nod_Set}}{nod_70e1}

\begin{definition}{映射}
给定集合 $A$ 和 $B$,我们可以假想从 $A$ 中\textbf{每一个}元素上拉\textbf{一根}有方向的线连接到 $B$ 中的一个元素,这些线的连接方式就被称为一个从 $A$ 到 $B$ 的\textbf{映射(mapping)}, 也叫\textbf{算符(operator)}。将这个映射记为 $f$, $A$ 叫做 $f$ 的\textbf{定义域(domain)}, $B$ 叫做\textbf{到达域(codomain)}\footnote{也叫陪域、上域、目标集(target set)}。 对于 $A$ 的子集 $C$, $B$ 中被线连接到的元素的集合叫做 $C$ (关于 $f$)的\textbf{像(image)},记做$f(C)$; $f(A)$被称为 $f$ 的\textbf{值域(range)}\footnote{值域在一些文献中指的是到达域。}(有时也称为 $f$ 的\textbf{像(image)}), 记为 $\Im(f)$。
\end{definition}
% 没有人会把这么一般的映射叫做算符的!!!

我们一般将 “$f$ 是从 $A$ 到 $B$ 的映射” 记为
\begin{equation}
f: A \to B~,
\end{equation}
也就是说从 $A$ 的元素上拉线到 $B$ 的元素上。 有时候, 为了表示映射的定义域 $A$ 或到达域 $B$ 是另一个集合的 $S$ 的子集, 我们也会将映射记为
\begin{equation}
f: A\subseteq S \to B \quad\text{或}\quad f: A \to B \subseteq S~.
\end{equation}
注意映射是有方向区分的, $A$ 中每个元素都\textbf{有且只有}一根线拉出去, 但是 $B$ 中的元素可以同时被一根或多根线连接, 也可以没有连接(即不在值域中)。 换一种简单的说法, “一对一” 和 “多对一” 连线是允许的, 但 “一对多” 是不允许的。

\begin{example}{}
映射的典型例子是一元实函数 $y = f(x)$, 其中定义域和值域都是实数集 $R$ 的某个子集。
\end{example}

\subsection{映射的类型}
\begin{definition}{}\label{def_map_1}
如果映射 $f: A  \to B$ 中每个 $B$ 中元素只被 1 根或者 0 根线连接, 那么称 $f$ 是一个\textbf{单射(injection)}。 如果 $f: A \to B$ 中每个 $B$ 中元素都被至少1根线连接, 那么称 $f$ 是一个\textbf{满射(surjection)}。 如果 $f$ 既是单射又是满射,那么称它为一个\textbf{双射(bijection)}, 或者叫\textbf{一一对应(one-to-one correspondence)}。
\end{definition}

\begin{figure}[ht]
\centering
\includegraphics[width=8cm]{./figures/845ffbe6a2eb30b5.pdf}
\caption{映射的分类} \label{fig_map_1}
\end{figure}

如果 $f: A \to B$ 是一个双射,那么 $A$ 中每一个元素都唯一地连接到 $B$ 中某一个元素,并且 $B$ 中每一个元素也都唯一被 $A$ 中某一个元素所连接,因此很明显可以将这个过程可以反过来,从 $B$ 中向 $A$ 中拉连接线,即我们有逆映射 $f^{-1}: B \to A$。另外,如果 $A$ 和 $B$ 存在双射,意味着 $A$ 和 $B$ 的元素数量应该一致\footnote{小时百科中统一使用这种定义。 一些其他教材中也把我们的 “单射” 称为 “一一映射”, 把 “满射” 称为 “到上”, 把 “双射” 称为 “一一到上”, 需要特别小心。}。

实函数是一种常见的映射, 例如 $f(x) = 2x$ 可以看作映射 $f: \mathbb R \to \mathbb R$。 但是映射可以从任意集合到任意集合。 例如将整数映射到正多边形, 将函数的映射到函数或实数(一般把这种映射称为\textbf{算符})等。

注意当一个集合中有无限个元素时, 我们有可能在它的子集和它本身之间建立一一映射, 例如函数 $\tan(x)$ 可以从实轴的开区间 $(-\pi/2, \pi/2)$ 一一映射到整个实轴 $\mathbb R$, 又例如我们可以将全体整数 $\mathbb Z$ 乘以二后一一映射到全体偶数 $2\mathbb Z$ 上。 这时我们仍然认为这两个集合的元素一样多, 虽然直觉上可能不容易接受。

\textbf{Cantor-Bernstein 定理}显示,如果集合 $A$ 到集合 $B$ 上存在一个单射 $f$ 和一个满射 $g$,那么总可以利用 $f$ 和 $g$ 来构造出一个双射。

\subsection{多元运算}\label{sub_map_1}
有时候我们需要将两个集合 $A, B$ 中任意各取一个元素, 然后映射另一个集合 $C$ 中的元素, 称为\textbf{二元运算(binary operation)}。 我们可以使用笛卡尔积(\autoref{eq_Set_1}~\upref{Set})将这个映射表示为
\begin{equation}\label{eq_map_1}
A \times B \to C~.
\end{equation}
一个简单的例子就是两个实数的的加法减法或乘法可以表示为 $\mathbb R \times \mathbb R \to \mathbb R$ (或简记为 $\mathbb R^2 \to \mathbb R$), 但除法不可以, 因为除数的集合不是 $\mathbb R$ 而是 $\mathbb R$ 去掉 $0$。

\addTODO{定义笛卡尔积}

同理, \textbf{多元运算}可以用多个卡氏积(笛卡尔积)表示为
\begin{equation}
A_1 \times \dots \times A_N \to C~,
\end{equation}
例如含有 $N$ 个自变量的函数就是一个 $N$ 元运算。 特殊地, $N$ 个相同集合 $A$ 做卡氏积可以简单表示为 $A^N$, 例如 $N$ 个有序复数的集合为 $\mathbb C^N$。

\subsection{映射间的关系}
\begin{definition}{相等}
当映射(算符) $f: A\to B$ 和 $g: A \to B$ (定义域和到达域都相同),对任意 $x \in A$ 都有 $f(x) = g(x)$, 那么我们就说两个映射(算符)\textbf{相等}, 记为 $f = g$, 否则它们就\textbf{不相等}。
\end{definition}

\begin{definition}{限制与拓展}\label{def_map_3}
对于映射 $f: A \to B$,考虑子集 $C \subseteq A$,$D \subseteq B$,满足$f(C) \subseteq D$,我们可以定义一个新的映射 $g: C \to D$,满足对任意 $x \in C$,$g(x): = f(x)$,$g$ 被称为 $f$ 的\textbf{限制(restriction)},反过来 $f$ 被称为 $g$ 的\textbf{拓展(extension)};当 $B = D$ 时,我们记 $g$ 为 $f|_C$,称为 $f$ 在 $C$ 上的限制。

若 $A$ 是 $C$ 的子集($A\subseteq C$), 我们就说 $g$ 是 $f$ 的, 记为 $f \subseteq g$。 特殊地, 当 $A$ 是 $C$ 的真子集($A\subset C$), 就记为 $f \subset g$。
\end{definition}

\subsection{恒等映射}

\begin{definition}{恒等映射}
若一个集合到它自身的映射 $f: X\to X$ 把任意 $x\in X$ 映射到 $x$ 本身, 我们就叫它\textbf{恒等映射(identity map)}或者\textbf{单位算符(unit operator)}, 通常用 $I_X$($I$), $E_X$($E$) 或者 $\text{id}_X$($\text{id}$)表示。
\end{definition}
注意对不同集合 $X$, 它们的单位算符定义域并不相等, 所以它们的单位算符也不相等。

\subsection{复合映射}\label{sub_map_2}
\begin{definition}{}\label{def_map_4}
给定两个映射 $f: A \to B$ 和 $g: C \to D$, 如果 $f$ 的到达域 $B$ 是 $g$ 的定义域 $C$ 的一个子集($B \subseteq C$), 则可以定义\textbf{复合映射(composition of maps)} $g\circ f: A\to D$, 即先将 $A$ 中的元素通过 $f$ 映射到 $B \subseteq C$, 再通过 $g$ 映射到 $D$ 的元素,即对任意的 $x \in A$, $(g \circ f)(x): = g(f(x))$。
\end{definition}
注意“先 $f$ 后 $g$”记做 $g \circ f$,顺序与自然语言是相反的。

在没有歧义的情况下也可以将 “$\circ$” 省略, 尤其是将映射称为\textbf{算符}时。

复合映射常见的例子是复合函数, 令 $\mathbb R$ 上的函数 $f(x) = \sin x$, $g(x) = x^2$, 则复合函数 $g\circ f: \mathbb R \to [0, 1]$ 为 $(g\circ f)(x) = g(f(x)) = \sin^2 x$。

根据定义, 复合映射满足\textbf{结合律}, 令 $f, g, h$ 为映射, 则
\begin{equation}
h \circ (g \circ f) = (h \circ g) \circ f~.
\end{equation}

\subsection{映射的乘积}

\begin{definition}{}
给定两个映射 $f: A \to B$ 和 $g: C \to D$,则可以定义两个映射的\textbf{乘积(product)}为 $f\times g: A \times C \to B\times D$,其中对于任意 $a\in A, b\in B$ 有 $f\times g(a, b)=(f(a), g(b))$。
\end{definition}

当两个函数的定义域相等时,我们也可以只考虑到达域的乘积,

\begin{definition}{}\label{def_map_2}
给定两个映射 $f: A \to B$ 和 $g: A \to D$,则可以定义另一种\textbf{乘积(product)}为 $(f, g): A \to B\times D$,其中对于任意 $a\in A, b\in B$ 有 $(f, g)(a)=(f(a), g(a))$;实际上考虑 $A$ 的对角线映射$\Delta: A \to A \times A, a \mapsto (a, a)$,我们有
\begin{equation}
(f, g) = (f \times g) \circ \Delta~.
\end{equation}
\end{definition}

\subsection{逆映射}
若已知一个映射 $f: A\to B$, 如何构造一个逆映射? 注意我们要求\textbf{逆映射必须是一个映射}。 我们可以先试着把所有定义域和到达域互换, 然后把所有 “连线” 的方向逆转。 但一般情况下, 我们不能保证这样得到的关系符合映射的定义: 例如若 $f$ 是多对一映射, 那么方向逆转后, 就会出现 “一对多” 的情况, 而这是不允许的。 又例如 $B$ 中有些元素没被 $f$ 射到, 那么 $B$ 就不能作为逆映射的定义域, 因为定义域要求每个元素都要射出一条线。 加上最少的限制以后, 可以定义逆映射如下:

\begin{definition}{逆映射}
如果 $f: A\subseteq A' \to B$ 是单射, $A'$ 是 $A$ 的任意父集, 令 $R \subseteq B$ 为映射的值域, 那么可以把它的逆映射记为 $f^{-1}: R \to A'$, 把任意 $y \in R$, 映射到 $x \in A$, 并满足 $f(x) = y$。
\end{definition}
也就是说, 只有单射存在逆映射。 对于非单射, 我们可以先通过限制它的定义域找到一个单射, 再寻找逆映射。

\begin{example}{}
如果取正弦函数 $y = \sin x$ 的值域为 $R = [-1, 1]$ 如果取定义域为 $\mathbb R$,  那么它不是一个单射, 因为每一个 $y \in R$ 都对应无穷个 $x$, 所以不存在反函数。 但如果取定义域为 $[-\pi/2, \pi/2]$, 那么它是一个单射, 存在反三角函数 $\sin^{-1}: [-1, 1] \to [-\pi/2, \pi/2]$。

根据以上定义, $\sin^{-1} (\sin(x))$ 是定义在 $[-\pi/2, \pi/2]$ 上的恒等函数, 而 $\sin (\sin^{-1}(x))$ 是定义在 $[-1, 1]$ 上的恒等函数, 所以有 $\sin \circ \sin^{-1} \subseteq \sin^{-1} \circ \sin$。
\end{example}

\begin{corollary}{}
若映射 $f: A  \to R \subseteq B$ 存在逆映射 $f^{-1}: R \to A$, 那么复合映射 $f^{-1} \circ f: A \to A$ 是恒等映射。
\end{corollary}
根据定义可证。

% \addTODO{幂集最好另开文章, 这里只介绍单射的情况, 以及简单介绍一下逆像的概念}

给定集合 $A, B$,定义 $B^A$ 为“从 $A$ 到 $B$ 的所有可能的映射所构成的集合”。 如果 $B$ 是一个二元集合,即它只有两个元素,不妨记为 $B=\{0,1\}$,那么 $B^A$ 可以用来表示 $A$ 的幂集(\autoref{def_Set_1}~\upref{Set}),即由 $A$ 的所有子集所构成的集合。这是因为对于任意的 $f\in B^A$,我们可以把这个 $f$ 对应到 $A$ 的子集 $S$,其中 $S$ 的元素全都被 $f$ 映射到1上, $A-S$ 的元素全都被 $f$ 映射到 0 上。当然, 0 和 1 的地位反过来也可以,$\{0, 1\}$ 也可以被替换成任何一个二元集合。 由于这个特点,我们简单地把 $A$ 的幂集记为\footnote{当 $A$, $B$ 都是有限集的时候, $|B^A|=|B|^{|A|}$。特别地, $|2^A|=2^{|A|}$。}  $2^A$. 

利用映射 $f: A \to B$,可以导出一个映射 $f^{-1}:2^B \to 2^A$,称为映射 $f$ 的\textbf{逆像映射} 。对于 $B$ 的任意子集 $C$,有
\begin{equation}
f^{-1}(C) = \qty{x | x \in A, f(x) \in C}~.
\end{equation}
此时,$f^{-1}(C)$ 称为 $C$ 在 $f$ 下的\textbf{逆像(inverse image)}或\textbf{原像(preimage)}。

特别地,和 $f$ 的值域中不相交的 $C$ 被 $f^{-1}$ 映射到空集上,而空集也是 $A$ 的一个子集。如果 $f$ 是一个双射,那么对于任意 $y\in B$,单元素子集 $\{y\}$ 都被 $f^{-1}$ 映射在 $A$ 的某个单元素子集上,那么我们也可以认为此时 $f^{-1}$ 实际上是单个元素映射在单个元素上,也就是从 $B$ 到 $A$ 的映射。

如果 $f: A \to B$ 是\textbf{双射}, 那么 $f^{-1}$ 总是把单点集映射到单点集上,而且$B$任何点都有被映射到,因此这时我们可以定义 $f^{-1}:B \to A$,使得 $\forall x \in A$ 都有 $f^{-1}(f(x)) = x$,$\forall y \in A$,$f(f^{-1}(y)) = y$。特别地,此时我们将 $f^{-1}$ 称为 $f$ 的\textbf{逆映射(inverse map)}\footnote{注意“逆像映射”和“逆映射”的区别。\textbf{逆像映射}是子集到子集的映射,而\textbf{逆映射}是点到点的映射。任何映射都有\textbf{逆像映射},但是只有双射才有\textbf{逆映射}。},双射的逆映射是唯一确定的。
\addTODO{证明逆映射是唯一的}

从另一个方面来说,$f\circ f^{-1}$ 和 $f^{-1}\circ f$ 都是单位算符(恒等映射)。注意两者的定义域分别为 $A$ 和 $B$, 当 $A \ne B$ 时不能写成 $f\circ f^{-1} = f^{-1}\circ f$。如果把 $A$,$B$ 到自身的恒等映射分别记为 $I_A$ 和 $I_B$,那么 $f\circ f^{-1}=I_B$,$f^{-1}\circ f=I_A$。

对于一般的映射,我们不一定能定义逆映射,实际上

\begin{theorem}{双射的等价定义}
映射 $f: A \to B$,是一个双射等价于存在逆映射。
\end{theorem}
\addTODO{证明}

对于单射/满射,虽然不能定义逆映射,但我们可以定义更弱一点的“逆”映射。

\begin{definition}{左逆/右逆映射}
映射 $f: A \to B$,另一个映射 $g: B \to A$ 被称为 $f$ 的
\begin{itemize}
\item \textbf{左逆映射},如果 $g \circ f = I_A$;
\item \textbf{右逆映射},如果 $f \circ g = I_B$;
\end{itemize}
\end{definition}

\begin{theorem}{单射/满射的等价定义}
映射 $f: A \to B$,
\begin{itemize}
\item 是一个单射等价于存在左逆映射;
\item 是一个满射等价于存在右逆映射。
\end{itemize}
\end{theorem}
\addTODO{证明}

左/右逆映射可以理解成“相对的”,考虑 $g \circ f = I_A$,我们发现 $g: B \to A$ 是 $f: A \to B$ 的左逆映射,而同时 $f$ 又是 $g$ 的右逆映射,因此我们有推论:
\begin{corollary}{}
单射的左逆映射是满射,满射的左逆映射是单射。
\end{corollary}

\begin{example}{左逆是不唯一的}
考虑单射,
$$
\begin{aligned}
f: \{0, 1\} &\to \{a, b, c\} , \\
0 &\mapsto a , \\
1 &\mapsto b ,
\end{aligned}~
$$
我们有 $g_0, g_1: \{a, b, c\} \to \{0, 1\}$,两个左逆映射,它们都把$a \mapsto 0$,$b \mapsto 1$,但是
$$
\begin{aligned}
g_0: c \mapsto 0 , \\
g_1: c \mapsto 1 .
\end{aligned}~
$$
\addTODO{画图}
\end{example}





