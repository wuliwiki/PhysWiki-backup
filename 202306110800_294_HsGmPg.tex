% 等比数列(高中)
% 高中|等比数列

\pentry{数列的概念与函数特性(高中)\upref{HsSeFu}}

\begin{definition}{等比数列}
如果一个数列从第2项起,每一项与它前一项的比都等于同一个常数,那么这个数列叫作\textbf{等比数列},这个常数叫作等比数列的\textbf{公比},公比通常用字母 $q$ 表示 $(q\ne 0)$。
\end{definition}

\subsubsection{通项公式}
由定义可得,等比数列的\textbf{通项公式}
\begin{equation}
a_n = a_1 q^{n-1} \quad (a_1 \ne 0,\ q\ne 0,\ n=1,2,3\dots)~
\end{equation}

\subsubsection{等比中项}
与等差数列类似,如果在 $a$ 和 $b$ 中插入一个数 $G$,使得 $a,G,b$ 成等比数列,那么根据等比数列的定义, $\frac{G}{a} = \frac{b}{G},G^2 = ab,G = \pm \sqrt{ab}$。我们称 $G$ 为 $a,b$ 的\textbf{等比中项}。

易得,在等比数列中,首末两项除外,每一项都是它前后两项的等比中项。

\subsection{前 $n$ 项的和}
等比数列求和与等差数列求和有相似之处,
\begin{equation}\label{eq_HsGmPg_1}
S = a_1 + a_2 + \cdots + a_n~,
\end{equation}
\begin{equation}
qS = qa_1 + qa_2 + \cdots + qa_n~,
\end{equation}
\begin{equation}\label{eq_HsGmPg_2}
qS= a_2 + a_3 + \cdots + qa_n~.
\end{equation}
\autoref{eq_HsGmPg_1} 与 \autoref{eq_HsGmPg_2} 做差
\begin{equation}
\begin{aligned}
(1 - q)S &= a_1 - qa_n\\
&= a_1(1 - q^n)~,
\end{aligned}
\end{equation}
\begin{equation}
S = \frac{a_1(1-q^n)}{1-q} \quad (q\neq 1)~.
\end{equation}
则等比数列前 $n$ 项和为
\begin{equation}
S = 
\begin{cases}
\begin{aligned}
&na_1,(q = 1) \\
&\frac{a_1(1-q^n)}{1-q},\quad (q \neq 1)
\end{aligned}
\end{cases}~
\end{equation}

当等比数列无穷递缩时,即 $0<q<1$, $n\rightarrow \infty$。 这里的极限详见 “数列的极限(简明微积分)\upref{Lim0}”。
\begin{equation}
S = \frac{a_1}{1 - q}~.
\end{equation}
