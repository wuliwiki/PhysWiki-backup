% 自旋统计定理(综述)
% license CCBYSA3
% type Wiki

本文根据 CC-BY-SA 协议转载翻译自维基百科\href{https://en.wikipedia.org/wiki/Spin\%E2\%80\%93statistics_theorem}{相关文章}。

自旋-统计定理证明了粒子的内禀自旋(不源于轨道运动的角动量)与该类粒子集合的量子统计性质之间的关系是量子力学数学的必然结果。在以约化普朗克常数\( \hbar \)为单位的描述下,所有在三维空间中运动的粒子具有以下特性:整数自旋的粒子服从玻色-爱因斯坦统计;半整数自旋的粒子服从费米-狄拉克统计\(^\text{[1][2]}\)。
\subsection{自旋-统计关系} 
所有已知粒子都遵循费米-狄拉克统计或玻色-爱因斯坦统计。粒子的内禀自旋总是能够预测该类粒子集合的统计性质,反之亦然 \(^\text{[3]}\):  
\begin{itemize}
\item 整数自旋的粒子是玻色子,遵循玻色-爱因斯坦统计;  
\item 半整数自旋的粒子是费米子,遵循费米-狄拉克统计。  
\end{itemize}
自旋-统计定理证明了量子力学的数学逻辑预测或解释了这一物理结果\(^\text{[4]}\)。  

对于不可区分粒子的统计性质,其影响是最基本的物理效应之一。例如:泡利不相容原理 —— 规定每个占据的量子态中最多只能容纳一个费米子,决定了物质的形成。物质的基本组成部分,如质子、中子和电子,都是费米子。另一方面,介导物质粒子之间相互作用的粒子,如光子,都是玻色子\([\text{citation needed}]\)。自旋-统计定理试图解释这一基本的二分性的起源\(^\text{[5]: 4}\)。
\subsection{背景}
从直观上看,自旋作为粒子的内禀角动量属性,似乎与该类粒子集合的基本性质无关。然而,这些粒子是不可区分的,因此涉及多个不可区分粒子的任何物理预测在交换这些粒子时都不应发生变化。
\subsubsection{量子态与不可区分粒子}
在量子系统中,物理态由态矢量描述。如果两个态矢量之间仅相差一个整体相因子(忽略其他相互作用),则它们在物理上是等效的。对于一对不可区分粒子,它们只有一个物理态。这意味着:如果交换粒子的位置(即进行一个排列变换),不会产生新的物理态,而是得到与原始物理态相匹配的态。实际上,无法区分交换前后的哪个粒子处于哪个位置。  

尽管物理态在粒子交换后不变,但态矢量本身可能会因交换而改变符号。由于这种符号变化只是一个整体相因子,因此它不会影响物理态。  

证明自旋-统计关系的核心因素是相对论,即:物理定律在洛伦兹变换下保持不变。场算符在洛伦兹变换下的变换方式取决于它们所创造的粒子的自旋。  

此外,还需要引入一个关键假设:微因果性假设:类空分离的场算符要么对易,要么反对易。这一假设仅适用于相对论性理论(具有时间方向的理论),否则类空的概念将失去意义。然而,证明过程中需要采用欧几里得时空的方法,在该方法中,时间方向被视为一个空间方向,具体如下:  

洛伦兹变换包括:三维旋转 洛伦兹推进。洛伦兹推进将参考系转换为不同速度的惯性系,其数学表现类似于时间方向上的旋转。在量子场论的关联函数的解析延拓中,时间坐标可以变为虚数。此时:洛伦兹推进会变为旋转。这样得到的新“时空”仅具有空间方向,称为欧几里得时空。
\subsubsection{交换对称性或排列对称性}  
玻色子是具有在这种交换或排列下保持对称的波函数的粒子,因此如果交换这些粒子,波函数不会发生变化。费米子是具有在这种交换下呈反对称的波函数的粒子,因此在交换时,波函数会获得一个负号,这意味着两个相同的费米子占据同一状态的振幅必须为零。这就是泡利不相容原理:两个相同的费米子不能占据同一状态。而对于玻色子,这一规则并不适用。

在量子场论中,一个态或波函数由场算符作用于某个称为真空的基本态来描述。为了使这些算符能够投影出所创建波函数的对称或反对称分量,它们必须满足适当的对易关系。  

算符  
\[
\iint \psi (x,y)\phi (x)\phi (y)\,dx\,dy~
\]
(其中 \(\phi\) 是一个算符,\(\psi (x,y)\) 是一个具有复值的数值函数)创建了一个具有波函数 \(\psi (x,y)\) 的双粒子态,并且根据场的对易性质,只有反对称部分或对称部分才会起作用。

假设 \( x \neq y \),并且两个算符在同一时间作用;更一般地,它们可能具有类空分离,后文将对此进行解释。  

如果场算符对易,即满足以下关系:  
\[
\phi (x)\phi (y) = \phi (y)\phi (x)~
\]
那么只有波函数 \( \psi \) 的对称部分起作用,因此\(\psi (x,y) = \psi (y,x)\)
这意味着该场将产生玻色子粒子。  

另一方面,如果场算符反对易,即  
\[
\phi (x)\phi (y) = -\phi (y)\phi (x)~
\]
那么只有波函数 \( \psi \) 的反对称部分起作用,因此\(\psi (x,y) = -\psi (y,x)
\)这意味着所产生的粒子是费米子。
\subsection{证明} 
尽管自旋-统计定理的表述非常简单,但无法给出其初等解释。理查德·费曼在《费曼物理学讲义》中曾表示,这或许意味着我们对这一基本原理尚未完全理解。\(^\text{[3]}\) 

已经发表了许多著名的证明,它们各自具有不同的限制和假设。这些证明都是“否定性证明”,即它们证明了整数自旋场不能导致费米子统计,而半整数自旋场不能导致玻色子统计。\(^\text{[5]: 487 }\)

避免使用任何相对论性量子场论机制的证明存在缺陷。许多此类证明依赖于以下主张:  
\[
|\psi (\alpha _{1},\alpha _{2},\alpha _{3},\dots )|^{2} = |{\hat {P}}\psi (\alpha _{1},\alpha _{2},\alpha _{3},\dots )|^{2}~
\]
其中,算符 \( \hat{P} \) 交换坐标。然而,左侧的值表示粒子 1 处于 \( r_1 \)、粒子 2 处于 \( r_2 \) 等等的概率,因此对于不可区分粒子而言,在量子力学中是不成立的。\(^\text{[6]: 567  }\)

1939年,马克斯·菲尔兹,作为沃尔夫冈·泡利的学生,提出了第一个证明\(^\text{[7]}\);而在次年,泡利以更系统的方法重新推导了这一证明\(^\text{[8]}\)。在后来的总结中,泡利列出了相对论性量子场论中这些定理版本所必需的三个基本假设:
\begin{enumerate}
\item 任何具有粒子占据的状态,其能量都高于真空状态。
\item 空间上分离的测量不会相互干扰(它们满足对易关系)。
\item 物理概率必须为正(即希尔伯特空间的度量是正定的)。
\end{enumerate}
他们的分析忽略了除状态的对易/反对易之外的粒子相互作用。\(^\text{[9][5]: 374 }\)

1949 年,理查德·费曼提出了一种完全不同的证明方法\(^\text{ [10]}\),基于真空极化 ,但后来被泡利批评。\(^\text{[9][5]: 368}\)  泡利指出,费曼的证明明确依赖于他所使用的前两个公设,并且隐含地使用了第三个公设——即费曼最初允许负概率,但随后拒绝了概率大于 1 的场论结果。  

1950 年,朱利安·施温格提出了基于时间反演不变性的证明\(^\text{ [11]}\),而早在 1940 年,弗雷德里克·贝林方特就已经提出了一种基于电荷共轭不变性的证明。这些研究最终与CPT 定理相关联,并在 1955 年由泡利进一步发展。\(^\text{ [12]}\)这些证明因其复杂性而难以理解。\(^\text{ [5]: 393 }\)

阿瑟·怀特曼在量子场论公理化方面的研究导致了一个定理,该定理指出,两个场的乘积的期望值\(\phi (x)\phi (y)\)可以解析延拓到所有分离\((x - y)\)的情形。\(^\text{[5]: 425}\) (泡利时代的证明的前两个公设涉及真空态和不同位置的场。)这一新结果使得格哈特·吕德斯和布鲁诺·祖米诺\(^\text{[13]}\)以及彼得·伯戈因\(^\text{[5]: 393 }\)提出了更严格的自旋-统计定理证明。1957年,雷斯·约斯特通过自旋-统计定理推导出了CPT 定理,而伯戈因在1958年的证明无需对粒子相互作用或场论形式施加任何约束。这些结果是最严谨且最具实践意义的定理之一。\(^\text{ [14]: 529}\)   

尽管取得了这些成功,费曼在 1963 年的本科生讲座中讨论自旋-统计联系时仍表示:“我们很抱歉,无法给出一个初等的解释。”\(^\text{[3]}\)1994 年,纽恩施万德也提出了类似的疑问,询问是否取得了进展。\(^\text{[15]}\)这一问题激发了更多新的证明和相关书籍的出版。\(^\text{[5]}\)在2013年,纽恩施万德对自旋-统计联系的科普表明,简单的解释仍然难以找到。\(^\text{[16]}\)
\subsection{实验检验}
1987年,格林伯格和莫哈帕拉提出,自旋-统计定理可能存在微小的违背。\(^\text{[17][18]}\)戴拉米安、吉拉斯皮和凯莱赫借助对违反泡利不相容原理的氦原子态的精确计算结果,\(^\text{[19]}\)使用原子束光谱仪寻找氦的\(1s2s^1S_0\)态。实验未能找到该状态,并将其存在的上限约束为\(5\times10^{6}\)。
\subsection{与洛伦兹群表象理论的关系}  
洛伦兹群没有非平凡的有限维酉表示。因此,似乎不可能构造一个希尔伯特空间,使其中的所有态都具有有限、非零的自旋,并且具有正的、洛伦兹不变的范数。  

针对这一问题,不同自旋-统计类型的粒子采取了不同的解决方案:  

对于整数自旋的态,负范数态(称为“非物理极化”)被设为零,这使得规范对称性的使用变得必要。对于半整数自旋的态,可以通过费米子统计来规避这一问题。\(^\text{[21]}\)
\subsection{二维准粒子—任意子} 
1982 年,物理学家弗兰克·维尔切克发表了一篇研究论文,探讨了可能具有分数自旋的粒子,并将其称为“任意子”,因为它们能够取“任意”自旋。\(^\text{[22]}\)他指出,这些粒子在低维系统(即运动受限于三维以下的空间维度)中被理论预测存在。维尔切克描述它们的自旋统计为:“在玻色子和费米子两种情况之间连续插值”。\(^\text{[22]}\)这一效应后来成为理解分数量子霍尔效应的基础。\(^\text{[23][24]}\)
\subsection{参见 }
\begin{itemize}
\item 超统计
\item 任意子统计 
\item 辫群统计
\end{itemize}
\subsection{参考文献} 
\begin{enumerate}
\item Dirac, Paul Adrien Maurice(1981-01-01). The Principles of Quantum Mechanics. Clarendon Press. p. 149. ISBN 9780198520115.  
\item Pauli, Wolfgang (1980-01-01). General Principles of Quantum Mechanics. Springer-Verlag. ISBN 9783540098423.  
\item Feynman, Richard P.; Leighton, Robert B.; Sands, Matthew (1965). The Feynman Lectures on Physics. Vol. 3. Addison-Wesley. p. 4.1. ISBN 978-0-201-02118-9.  
\item Sudarshan, E. C. G. (May 1968). "The Fundamental Theorem on the Relation Between Spin and Statistics". Proceedings of the Indian Academy of Sciences - Section A*. 67 (5): 284–293. doi:10.1007/BF03049366. ISSN 0370-0089.  
\item Duck, Ian; Sudarshan, Ennackel Chandy George; Sudarshan, E. C. G. (1998). Pauli and the Spin-Statistics Theorem* (1st reprint ed.). Singapore: World Scientific. ISBN 978-981-02-3114-9.  
\item Curceanu, Catalina; Gillaspy, J. D.; Hilborn, Robert C. (2012-07-01). "Resource Letter SS–1: The Spin-Statistics Connection". American Journal of Physics. 80 (7): 561–577. doi:10.1119/1.4704899. ISSN 0002-9505.  
\item Fierz, Markus (1939). "Über die relativistische Theorie kräftefreier Teilchen mit beliebigem Spin". *Helvetica Physica Acta (in German). 12 (1): 3–37. Bibcode:1939AcHPh..12....3F. doi:10.5169/seals-110930.  
\item Pauli, Wolfgang (15 October 1940). "The Connection Between Spin and Statistics". Physical Review. 58 (8): 716–722. Bibcode:1940PhRv...58..716P. doi:10.1103/PhysRev.58.716.  
\item Pauli, Wolfgang (1950). "On the Connection Between Spin and Statistics". Progress of Theoretical Physics. 5 (4): 526–543. Bibcode:1950PThPh...5..526P. doi:10.1143/ptp/5.4.526.  
\item Feynman, Richard (1961). "The Theory of Positrons". Quantum Electrodynamics. Basic Books. ISBN 978-0-201-36075-2. (*A reprint of Feynman's 1949 paper in Physical Review*).
\end{enumerate}