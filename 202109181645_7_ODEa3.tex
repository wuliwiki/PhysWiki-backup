% 一阶常微分方程解法:恰当方程
% keys 恰当方程|微分方程|ODE|ordinary differential equation|exact equation|积分因子|integration factor

\pentry{常微分方程简介\upref{ODEint}}

\subsection{恰当方程的概念}

考虑一个二元函数$u(x, y)$,其全导数为
\begin{equation}
\dd u=\frac{\partial u}{\partial x}\dd x+\frac{\partial u}{\partial y}\dd y
\end{equation}

由多元微积分知识可知,$\frac{\partial^2 u}{\partial x\partial y}=\frac{\partial^2 u}{\partial y\partial x}$.因此,如果一个形如
\begin{equation}\label{ODEa3_eq1}
M(x, y)\dd x+N(x, y)\dd y=0
\end{equation}
的常微分方程满足
\begin{equation}\label{ODEa3_eq2}
\frac{\partial M}{\partial y}=\frac{\partial N}{\partial x}
\end{equation}
那么就可以存在一个$u(x, y)$,使得$M=\partial u/\partial x$和$N=\partial u/\partial y$.

这样一来,\autoref{ODEa3_eq1} 就相当于
\begin{equation}
\dd u=0
\end{equation}
其解就是$u=C$,$C$为积分常数.

也就是说,对于这样的方程,我们只需要求出$u$就能求解.

\begin{definition}{恰当方程}
将形如\autoref{ODEa3_eq1} 且满足\autoref{ODEa3_eq2} 的方程,称为\textbf{恰当方程(exact equation)}.
\end{definition}

我们研究一个例子,看看恰当方程是怎么解的.

\begin{example}{}\label{ODEa3_ex1}
考虑方程$\frac{\dd y}{\dd x}=\frac{y}{3y^2-x}$.

移项后得到$y\dd x+(x-3y^2)\dd y=0$.

记$M=y$,$N=x-3y^2$,则容易验证$\partial M/\partial y= 1 =\partial N/\partial x$,因此这是一个恰当方程.

我们希望找出一个$u$,使得$\partial u/\partial x=M$且$\partial u/\partial y=N$.

先用$M$关于$\dd x$积分,因为这样积分出来的结果再对$x$求偏微分就能得到$M$:
\begin{equation}\label{ODEa3_eq3}
u=\int M\dd x=\int y\dd x=xy+C_1(y)
\end{equation}
其中$C_1(y)$是一个关于$x$的常数——它完全可以是一个关于$y$的非常数函数,在对$x$求偏微分的时候不影响结果.

再用$N$关于$\dd y$积分,得到
\begin{equation}\label{ODEa3_eq4}
u=\int N\dd y=\int (x-3y^2)\dd y=xy-y^3+C_2(x)
\end{equation}
同样,$C_2(x)$是一个关于$y$的常数,它最多只和自变量$x$有关.

比较\autoref{ODEa3_eq3} 和\autoref{ODEa3_eq4} ,可见$C_1(y)=C_2(x)-y^3$,因此$C_2(x)$必须是一个和$x$也无关的常数,记为$C$,进而有$C_1(y)=C-y^3$.

代回\autoref{ODEa3_eq3} 或\autoref{ODEa3_eq4} ,得
\begin{equation}
u=xy-y^3+C
\end{equation}

于是方程的解就是$xy-y^3=K$,其中$K$为积分常数.

当$K\not=0$时,还可以写成$x-y^2=K$.

\end{example}


\subsection{积分因子}

如果给定的方程是恰当的,那按照\autoref{ODEa3_ex1} 的步骤就能很容易解出来.然而我们实际上遇到的方程往往乍一看不是恰当方程.不过,很多时候我们可以将一个非恰当方程转化为恰当方程,最常用的就是积分因子法.

\begin{definition}{积分因子}
对于\textbf{非恰当}常微分方程$M(x, y)\dd x+N(x, y)\dd y=0$,如果存在一个函数$f(x, y)$,使得
\begin{equation}\label{ODEa3_eq5}
f(x, y)M(x, y)\dd x+f(x, y)N(x, y)\dd y=0
\end{equation}
是一个恰当方程,那么称$f(x, y)$是原方程的一个\textbf{积分因子(integration factor)}.
\end{definition}

既然\autoref{ODEa3_eq5} 是一个恰当方程,那就有
\begin{equation}
\frac{\partial (fM)}{\partial y}=\frac{\partial (fN)}{\partial x}
\end{equation}
展开后有
\begin{equation}\label{ODEa3_eq6}
f\qty(\frac{\partial M}{\partial y}-\frac{\partial N}{\partial x})=N\frac{\partial f}{\partial x}-M\frac{\partial f}{\partial y}
\end{equation}

\autoref{ODEa3_eq6} 是一个关于未知


















