% 集合(高中)
% keys 高中|集合

\subsection{概述}
集合语言是现代数学的基本语言,这种语言可以简洁、准确的表达数学内容.在高中数学中,集合的内容相对简单,但集合是高中数学的基础,请重视集合的学习.

\subsection{定义}
一般地,指定的某些对象的全体称为\textbf{集合(set)}.集合常用大写字母 $A,B,C,D,\cdots$ 标记.集合中的每个对象叫作这个集合的\textbf{元素(element)}.常用小写字母 $a,b,c,d,\cdots$ 表示集合中的元素.

若 $a$ 在集合 $A$ 中,就说 $a$ \textbf{属于(belong to)}集合 $A$ ,记作 $a \in A$.若 $a$ 不在集合 $A$ 中,就说 $a$ \textbf{不属于}集合 $A$,记作 $a\notin A$.

数的集合简称\textbf{数集(number set)}.\\
自然数组成的集合简称\textbf{自然数集},记作 $N$;\\
正整数组成的集合简称\textbf{正整数集},记作 $N^{*}$ 或 $N^{+}$;\\
整数组成的集合简称\textbf{整数集},记作 $Z$;\\
有理数组成的集合简称\textbf{有理数集},记作 $Q$;\\
实数组成的集合简称\textbf{实数集},记作 $R$.

一般地,我们把含有有限个元素的集合叫\textbf{有限集},含有无限个元素的集合叫\textbf{无限集}.

我们将不含有任何元素的集合叫作\textbf{空集(empty set)},记作 $\varnothing$.

\subsection{表示}
\textbf{列举法}是把集合中的元素一一列举出来写在大括号内的方法.
符号表示为 $\begin{Bmatrix} ,\cdots, \end{Bmatrix}$,如 $\begin{Bmatrix} x_1,x_2, \cdots ,x_n \end{Bmatrix}$.

用确定的条件表示某些对象属于一个集合并写在大括号内的方法叫\textbf{描述法},符号表示为 $\begin{Bmatrix} | \end{Bmatrix}$,如 $\begin{Bmatrix} x\in A|p(x) \end{Bmatrix}$.

\subsection{性质}
\begin{enumerate}
\item 集合中的元素是\textbf{互异}的.
\item 集合中的元素是\textbf{无序}的.
\end{enumerate}

\subsection{集合的基本关系}
一般地,对于两个集合 $A$ 与 $B$,如果集合 $A$ 中的任何一个元素都是集合 $B$ 中的元素,即若 $a\in A$,则 $a\in B$,我们就说集合 $A$ \textbf{包含于}集合 $B$,记作
\begin{equation}
A \subseteq B
\end{equation}
也可以记作
\begin{equation}
B \supseteq A
\end{equation}
这时我们说集合 $A$ 是集合 $B$ 的\textbf{子集(subset)}.

显然,\textbf{任何一个集合都是它本身的子集},即
\begin{equation}
A \subseteq A
\end{equation}

对于两个集合 $A$ 与 $B$,如果集合 $A$ 中任何元素都是集合 $B$ 中的元素,同时集合 $B$ 中的任何一个元素都是集合 $A$ 中的元素,这时,我们就说集合 $A$ 与 集合 $B$ \textbf{相等(equality)},记作
\begin{equation}
A=B
\end{equation}

对于两个集合,$A$ 与 $B$,如果 $A\subseteq B$ ,并且 $A \ne B$,我们就说集合 $A$ 是集合 $B$ 的\textbf{真子集},记作
\begin{equation}
A \subset B
\end{equation}
也可记作
\begin{equation}
B \supset A
\end{equation}
\textsl{注:这个符号在高中课本中写作} “⫋”

当集合 $A$ 不包含于集合 $B$ 或集合 $B$ 不包含集合 $A$ 时,记作
\begin{equation}
A \nsubseteq B
\end{equation}
也可记作
\begin{equation}
B\footnote{原为小写 b 纠正为 B aluo} \nsupseteq A
\end{equation}

我们规定:\textbf{空集是任何集合的子集}.也就是说,对于任何一个集合 $A$ 都有
\begin{equation}
\varnothing \subseteq A
\end{equation}

\subsection{集合的基本运算}
一般地,由既属于集合 $A$ 又属于集合 $B$ 的所有元素组成的集合叫做 $A$ 与 $B$ 的\textbf{交集(intersection set)},记作 $A \cap B$(读作“A交B”),即
\begin{equation}
A\cap B = \begin{Bmatrix} x|x\in A \wedge x\in B \end{Bmatrix}
\end{equation}

由属于集合 $A$ 或属于集合 $B$ 的所有元素组成的集合,叫作 $A$ 与 $B$ 的\textbf{并集(union set)},记作 $A\cup B$(读作“A并B”),即
\begin{equation}
A\cup B = \begin{Bmatrix}x|x\in A \vee x\in B\end{Bmatrix}
\end{equation}

根据交集定义,可得
\begin{equation}
\begin{aligned}
&A\cap B = B\cap A \\ 
&A\cap B \subseteq A \\
&A\cap B \subseteq B \\
&A\cap A = A \\
&A\cap \varnothing = \varnothing
\end{aligned}
\end{equation}

根据并集的定义,可得
\begin{equation}
\begin{aligned}
&A\cup B = B\cup A \\
&A\subseteq A\cup B \\
&A\cup A = A \\
&A\cup \varnothing = A
\end{aligned}
\end{equation}

在研究某些集合的时候,这些集合往往是某个给定集合的子集,这个给定的集合叫做\textbf{全集(universal set)},常用符号 $U$ 表示.全集含有我们所要研究的这些集合的全部元素.

设 $U$ 是全集,$A$ 是 $U$ 的一个子集(即$A\subseteq U$),则由 $U$ 中所有不属于 $A$ 的元素组成的集合,叫作 $U$ 中子集 $A$ 的\textbf{补集(complementary set)}(或\textbf{余集}),记作$\complement_UA$,即
\begin{equation}
\complement_UA = \begin{Bmatrix}x|x\in U \wedge \notin A\end{Bmatrix}
\end{equation}

由补集定义可得,
\begin{equation}
\begin{aligned}
&A\cup (\complement_UA) = U \\
&A\cap (\complement_UA) = \varnothing
\end{aligned}
\end{equation}