% Liapunov 函数(稳定性直接法)
% keys Liapunov 函数|V函数
% license Usr
% type Tutor

\pentry{Liapunov 稳定性(常微分方程)\nref{nod_ODELia}}{nod_fcab}
Liapunov 对于非线性的问题又提出了“V 函数法”,又称 Liapunov 函数法、Liapunov 直接法。这个方法借助一个辅助函数直接从微分方程的动力性质研究,所以若知道这辅助函数之后研究稳定性将比较简单,只不过对于给定系统构造一个这样的函数是比较难的。



\subsection{函数“符号”}
\begin{definition}{}
对于一个函数 $ V(\bvec x)$ 在原点 $O$ 的某一邻域内有连续的一阶偏导数,同时 $ V(\bvec 0 ) = 0$,定义:
\begin{enumerate}
\item 若存在 $h>0$,当 $\Vert \bvec x \Vert < h$ 时,$V(\bvec x) \ge 0$($\le 0$),则称 $V$ 是常正(常负)函数,统称常号函数;
\item 若存在 $h > 0$,当 $0 < \Vert \bvec x \Vert < h$ 时,$V(\bvec x) > 0$($< 0$),则称 $V$ 是定正(定负)函数,统称定号函数;
\item 若无论 $h > 0$ 多么小,当 $\Vert \bvec x \Vert < h$ 时,$V(\bvec x)$ 都既可以取到正值、又可以取到负值,就称 $V$ 是变号函数。
\end{enumerate}

\end{definition}

例如,$V(x_1, x_2) = (x_1 + x_2)^2$ 是常正函数,$V(x_3, x_4) = 2 x_3^2 + x_4^2$ 是定正函数,而 $V(x_5, x_6) = x_5^2 - x_6^2$ 是变号函数。

\subsection{$V$ 函数法}
考虑一个非线性定常系统 
\begin{equation}\label{eq_ODELi2_1}
\dv{t} \bvec x = \bvec f(\bvec x) ~.
\end{equation}
约定 $\bvec f(x)$ 在包含坐标原点的某区域 $G \subseteq \mathbb R^n$ 内有连续的一阶偏导数,且 $\bvec f(0) = 0$。其中 $\bvec x = (x_1, x_2, \dots, x_n)$,$\bvec f = (f_1, f_2, \dots, f_n)$。

我们研究 $V$ 函数关于时间的全导数:
\begin{equation}
\dv{V(\bvec x(t))}{t} = \sum_{i=1}^n \pdv{V(\bvec x(t))}{x_i} \dv{x_i}{t} = \sum_{i=1}6n \pdv{V(\bvec x(t))}{x_i} f_i(\bvec x(t)) ~,
\end{equation}
就称 
\begin{equation}
\sum \pdv{V(\bvec x)}{x_i} f_i(\bvec x) ~~
\end{equation}
是 $V$ 关于系统 \autoref{eq_ODELi2_1} 对于 $t$ 的全导数。直接记
\begin{equation}
\dv{V}{t} = \sum \pdv{V(\bvec x)}{x_i} f_i(\bvec x) ~.
\end{equation}


