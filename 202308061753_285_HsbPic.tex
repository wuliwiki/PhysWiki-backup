% 薛定谔绘景和海森堡绘景
% license Xiao
% type Tutor



\pentry{时间演化算符(量子力学)\upref{TOprt},转移矩阵\upref{TransM}}

% \footnote{参考 Wikipedia \href{https://en.wikipedia.org/wiki/Heisenberg_picture}{相关页面}。}
薛定谔方程\upref{TDSE}通常使用的是动量表象\upref{moTDSE}和\textbf{薛定谔绘景}, 在海森堡绘景中, 波函数(态矢)不随时间改变, 而测量量的算符随时间改变。 海森堡绘景相当于在薛定谔绘景的基础上做了一个基底变换, 类似于位置和动量表象\upref{moTDSE}的关系。



作为一个物理理论,量子力学关心的是可观测量,包括本征值、概率、期望等,因此我们关心的不是量子态本身如何,而是量子态在可观测量的本征矢下展开的系数。随着时间流逝,这些系数会变化,而薛定谔绘景和海森堡绘景就是两种不同的解释系数变化的方法。

薛定谔绘景认为,可观测量不变,但是态矢量会随着时间变化,导致展开系数变化;海森堡绘景则认为,态矢量不变,但可观测量会变化,造成其本征矢量变化,从而导致态矢量的展开系数变化\footnote{严格来说,薛定谔绘景中也有可观测量会变化的情况,比如磁场在变化,那么哈密顿量$\bvec{S}\cdot \bvec{B}$就可能变化。所以这里说的是薛定谔绘景中不随时间变化的算符,到了海森堡绘景就会随时间变化。如果一个算符在薛定谔绘景中随时间变化,那么到了海森堡绘景只需要再叠加一个基底变换即可,正如本文开头所说,细节则请参见下面的小节。}。

这和线性代数里情况一模一样。一个矩阵可以解释为线性变换,它把一个向量变为另一个向量,导致这个向量的坐标变化了;也可以解释为\textbf{转移矩阵}\upref{TransM},向量本身没变,但是基变了,同样导致坐标变化。

你可以用这样一副图像来抽象地理解两个绘景的关系:薛定谔绘景中,坐标系不变,但向量在顺时针旋转;海森堡绘景中,向量不变,但坐标系在以相同的角速度逆时针旋转。这副图像已经能说明为什么海森堡绘景可以看作是基向量反向演化。

本文中,角标 $H$ 代表海森堡绘景, 角标 $S$ 代表薛定谔绘景。 例如波函数分别记为 $\psi_H(\bvec r)$ 和 $\psi_S(\bvec r, t)$, 后者不是时间的函数, 它的定义是
\begin{equation}
\psi_H(\bvec r) = \psi_S(\bvec r, 0)~.
\end{equation}

同时,本文使用$\hbar=1$的单位制,时间演化算符按定义为$\mathcal{U}(t)=\E^{-\I H t}$。

% \addTODO{演化子是什么?链接} %Jier: 就是时间演化算符\upref{TOprt},该词条我已经更新了

% 使用演化子(propagator) $U(t)$, 波函数之间的关系为
% \begin{equation}
% \psi_H(\bvec r, t) = U(t) \psi_H(\bvec r, 0) = U(t) \psi_S(\bvec r)
% \end{equation}
% 薛定谔方程在海森堡绘景中可以记为
% \begin{equation}
% H(U\psi_H) = \I\hbar \pdv{t} (U\psi_H)
% \end{equation}
% 由于 $\psi_H$ 不含时, 两边抵消, 得演化子满足方程
% \begin{equation}
% H U(t) = \I\hbar \pdv{t} U(t)
% \end{equation}

% 定义海森堡绘景中的算符为
% \begin{equation}
% Q_H(t) = U\Her(t) Q_S(t) U(t)
% \end{equation}
% 对时间求导得
% \begin{equation}
% \dv{t}Q_H = \frac{\I}{\hbar} [H_H, Q_H(t)] + \qty(\pdv{Q_S}{t})_H
% \end{equation}
% 平均值公式仍然和薛定谔绘景相同
% \begin{equation}
% \mel{\psi_H}{Q_H}{\psi_H} = \mel{\psi_S}{Q_S}{\psi_S}
% \end{equation}
% 证明:
% \begin{equation}
% \mel{\psi_S}{Q_S}{\psi_S} = \mel{\psi_H}{U\Her(t)Q_SU(t)}{\psi_H} = \mel{\psi_H}{Q_H}{\psi_H}
% \end{equation}
% 证毕。

\subsection{薛定谔绘景}

薛定谔绘景,简而言之,是固定算符不变,研究态矢量的演化。这也是我们在\textbf{量子力学的基本原理(量子力学)}\upref{QMPrcp}中使用的描述。

在薛定谔绘景中,算符是恒定的,从而其对应的本征矢量也是恒定的。实际的量子态$\ket{s}$则随着时间$t$演化为$\mathcal{U}(t)\ket{s}$。这个过程可以理解为,映射$\mathcal{U}(t)$作用在矢量$\ket{s}$上,导致$\ket{s}$变化,于是其关于各可观测量的本征态的基底展开系数变化——这些系数的模方就是测量后得到对应本征态的概率(概率密度),因此我们观测到的概率就会变化。


薛定谔绘景下,初态为$\ket{s}$的量子态,其可观测量$X$的期望值随时间演化:
\begin{equation}
\langle X \rangle(t) = \bra{s}\mathcal{U}^\dagger(t)X\mathcal{U}(t)\ket{s}~.
\end{equation}

\subsubsection{薛定谔方程}

薛定谔绘景下,态右矢对时间的导数为
\begin{equation}\label{eq_HsbPic_4}
\ali{
    \frac{\dd}{\dd t}\mathcal{U}(t)\ket{s} &= \frac{\dd}{\dd t}\E^{-\I H t}\ket{s}\\
    &= -\I H \mathcal{U}(t)\ket{s}~,
}
\end{equation}
如果记$\ket{s, t}=\mathcal{U}(t)\ket{s}$,那\autoref{eq_HsbPic_4} 也就是
\begin{equation}\label{eq_HsbPic_5}
\I \frac{\dd}{\dd t}\ket{s, t} = H\ket{s, t}~,
\end{equation}
这就是\textbf{薛定谔方程}。








\subsection{海森堡绘景}

海森堡绘景是另一种描述量子力学的框架,量子态本身不变,但可观测量的算符以及对应的本征态则随时间变化,由此造成量子态的基底展开系数变化。海森堡绘景下计算得到的可观测量的演化规律和薛定谔绘景相同,即两个绘景下给定量子态在给定算符的特征基下展开系数恒等。

注意,时间演化算符$\mathcal{U}(t)=\exp(-\I Ht)$是一个\textbf{幺正算符},即$\mathcal{U}^\dagger(t)=\mathcal{U}^{-1}(t)$。

\subsubsection{离散情况}



设$X^{(S)}(t)$是薛定谔绘景下的可观测量,,$X^{(H)}(t)$是在海森堡绘景下的同一个观测量,$\ket{a; t}$是任意一组基。为了方便,特别地令$X^{(H)}(0)=X^{(S)}(t)$。

% 设能量算子$H^{(S)}$的本征值$E_a$的本征矢为$\ket{a}$,其中$a$是正整数;$H^{(H)}(t)$的对应本征值$E_a$的本征矢则记为$\ket{a; t}$。规定在$t=0$时,$\ket{a; 0}$。

海森堡绘景要求,态右矢$\ket{s}$不变,是算符的改变导致$\ket{s}$在给定算符的特征基下系数变化。虽然薛定谔绘景下算子可能随时间变化,导致基也变化,但是态矢量同时也变化,记$t$时刻薛定谔绘景下$\ket{s}$变为$\ket{s(t)}$,其中$\ket{s(0)}=\ket{s}$,于是$\ket{s(t)}=\E^{-\I H t}\ket{s}$。



设$X^{(H)}(t)$的特征基为$\{\ket{a_i;t}\}$,$X^{(S)}(t)$的特征基为$\{\ket{b_i; t}\}$,其中各$\ket{a_i; t}$和$\ket{b_i; t}$关于$t$连续,且$\ket{a_i; 0}=\ket{b_i; 0}$。

则展开系数恒等意味着
\begin{equation}
\braket{a_i; t}{s} = \braket{b_i; t}{s(t)} = \bra{b_i; t}\E^{-\I H t}\ket{s}~. 
\end{equation}

因此
\begin{equation}\label{eq_HsbPic_3}
\ket{a_i; t} = \E^{\I H t}\ket{b_i; t}~, 
\end{equation}
即海森堡绘景下基的演化,与薛定谔绘景下态矢量的演化是反向的。这直观上很好理解和记忆:考虑一个二维实线性空间,如果说薛定谔绘景下态矢量逆时针旋转、坐标系不变,那么海森堡绘景下就应该是态矢量不变、坐标系顺时针旋转。




考虑到$X^{(H)}(t)$和$X^{(S)}(t)$的特征值一一对应且对应相等,故可设$\ket{a_i; t}$关于$X^{(H)}(t)$和$\ket{b_i; t}$关于$X^{(S)}(t)$的特征值都是$\lambda_i$。用基向量展开算子即得\textbf{测量算符随时间的演化}为
\begin{equation}\label{eq_HsbPic_2}
\begin{aligned}
X^{(H)}(t) = \sum_i \lambda_i \ket{a_i; t}\bra{a_i; t} ={}& \E^{\I H t}\qty(\sum_i \lambda_i \ket{b_i; t}\bra{b_i; t})\E^{-\I H t}\\
={}& \E^{\I H t}X^{(S)}(t)\E^{-\I H t}~.
\end{aligned}
\end{equation}



% 这是因为
% \begin{equation}
% \ali{
%     X^{(H)}(t)\mathcal{U}(t)^\dagger\ket{a_i; 0} &= X^{(H)}(t)\ket{a_i; t} \\
%     &= a_i\ket{a_i; t} \\
%     &= a_i\mathcal{U}(t)^\dagger\ket{a_i; 0}\\
%     &= \mathcal{U}(t)^\dagger a_i\ket{a_i; 0}\\
%     &= \mathcal{U}(t)^\dagger X^{(H)}(0)\ket{a_i; 0}\\
%     \implies X^{(H)}(t)\mathcal{U}(t)^\dagger &= \mathcal{U}(t)^\dagger X^{(H)}(0)
% }
% \end{equation}



\subsubsection{连续情况}

连续情况的讨论和离散情况完全相同,只是要求指标$a$的取值范围为指定范围的实数,并将$\sum_a$都替换为$\int \dd a$。\autoref{eq_HsbPic_2} 和\autoref{eq_HsbPic_3} 依然成立。


%已完成
%\addTODO{时间相关算符的海森堡绘景?}



\subsubsection{海森堡方程}

同\autoref{eq_HsbPic_5} 的导出思路相同,我们可以推导海森堡绘景下算符\footnote{这里的算符$X$在薛定谔绘景下不随时间变化,且注意$X$不一定和$H$对易。}的运动方程:
\begin{equation}\label{eq_HsbPic_6}
\ali{
    \frac{\dd}{\dd t} \mathcal{U}(t)^\dagger X \mathcal{U}(t) &= \mathcal{U}'(t)^\dagger X \mathcal{U}(t)+\mathcal{U}(t)^\dagger X \mathcal{U}'(x)\\
    &= \I H \E^{\I H t}X\E^{-\I H t}-\E^{\I H t}X\I H\E^{-\I H t}\\
    &= \I [H, \E^{\I H t}X\E^{-\I H t}]~,
}
\end{equation}
其中$[*, *]$是李括号。

记$\mathcal{U}(t)^\dagger X \mathcal{U}(t)=X(t)$,整理一下\autoref{eq_HsbPic_6} ,就得到\textbf{海森堡方程}:
\begin{equation}\label{eq_HsbPic_9}
\I\frac{\dd}{\dd t}X(t) = [X(t), H]~.
\end{equation}


利用\textbf{算符对易性(量子力学)}\upref{ComOpQ},可算出
\begin{equation}\label{eq_HsbPic_8}
    [\bvec{x}, \bvec{p}^2] = 2\I \bvec{p}~
\end{equation}
和
\begin{equation}\label{eq_HsbPic_7}
[\bvec{p}, V(\bvec{x})] = -\I\nabla V(\bvec{x})~.
\end{equation}
注意\autoref{eq_HsbPic_7} 右边的$\nabla$已经作用在$V$上了,整体是一个函数而非微分算子。

将\autoref{eq_HsbPic_8} 和\autoref{eq_HsbPic_7} 代入\autoref{eq_HsbPic_9} ,令$H=\frac{\bvec{p}^2}{2m}+V(x)$,则有
\begin{equation}\label{eq_HsbPic_10}
\leftgroup{
    \frac{\dd}{\dd t} \bvec{x} &= \frac{\bvec{p}}{m}\\
    \frac{\dd}{\dd t} \bvec{p} &= -\nabla V(x)
}~,
\end{equation}


\autoref{eq_HsbPic_10} 就是常用的海森堡运动方程,在狄拉克的\textsl{The Principles of Quantum Mechanics}中写为

\begin{equation}\label{eq_HsbPic_11}
\leftgroup{
    \frac{\dd}{\dd t} q_r &= \frac{\partial H}{\partial p_r}\\
    \frac{\dd}{\dd t} p_r &= -\frac{\partial H}{\partial q_r} 
}~.
\end{equation}
其中$q_r$是广义坐标,$p_r$是对应的共轭动量。












