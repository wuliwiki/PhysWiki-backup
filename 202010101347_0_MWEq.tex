% 麦克斯韦方程组
% 麦克斯韦方程组|电磁感应|高斯定理|电位移|安培定律

\begin{issues}
\issueDraft
\end{issues}

麦克斯韦方程组描述了经典电磁理论中电荷如何影响电磁场, 以及电磁场变化的规律.

麦克斯韦方程组共有四条方程
\begin{align}
\label{MWEq_eq1}
&\div \bvec E = \frac{\rho}{\epsilon_0}\\
\label{MWEq_eq2}
&\curl \bvec E = -\pdv{\bvec B}{t}\\
\label{MWEq_eq3}
&\div \bvec B = 0 \\
\label{MWEq_eq4}
&\curl \bvec B = \mu_0 \bvec J + \mu_0\epsilon_0 \pdv{\bvec E}{t}
\end{align}
其中\autoref{MWEq_eq1} 到\autoref{MWEq_eq4} 分别是电场的高斯定理证明\upref{EGausP},法拉第电磁感应定律\upref{FaraEB},磁场的高斯定理\upref{MagGau}, 安培环路定理\upref{AmpLaw}(加位移电流).%链接未完成

电场和磁场不是完全对称的, 可以通过引入磁单极子的概念使它们完全对称.

\subsection{磁单极子假设}
若定义磁荷的单位为 $\Si{Am}$(安培—米), 则
\begin{align}
&\div \bvec E = \frac{\rho}{\epsilon_0}\\
&\curl \bvec E = -\pdv{\bvec B}{t}\\
&\div \bvec B = 0 \\
&\curl \bvec B = \mu_0 \bvec J + \mu_0\epsilon_0 \pdv{\bvec E}{t}
\end{align}
