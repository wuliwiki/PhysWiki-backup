% 斜面应力公式

\pentry{应力\upref{STRESS}}

\begin{figure}[ht]
\centering
\includegraphics[width=5cm]{./figures/CHYSTR_1.pdf}
\caption{斜切面$\bvec n$上的应力$\bvec \sigma_\nu$。$\bvec n$是面的法向量} \label{CHYSTR_fig1}
\end{figure}

在应力\upref{STRESS}中,我们已经处理了微元体上的应力问题。那么新的问题来了,假设我们已经在某个坐标系中算出了微元体的各个应力,如果我们在微元体中取一个切面,那么这个切面上的应力要如何计算?

\textsl{跨界大牛}柯西给出了一个漂亮的回答,因此该公式也称柯西应力公式。若写成矩阵乘法的形式,那么:
\begin{equation}
\begin{pmatrix}
\sigma_{\nu,1}\\
\sigma_{\nu,2}\\
\sigma_{\nu,3}\\
\end{pmatrix}
=
\begin{pmatrix}
\sigma_{11} & \tau_{12} & \tau_{13} \\
\tau_{21} & \sigma_{22} & \tau_{23} \\
\tau_{31} & \tau_{32} & \sigma_{33} \\
\end{pmatrix}
\begin{pmatrix}
n_1\\
n_2\\
n_3\\
\end{pmatrix}
\end{equation}
还可以写成更紧凑的张量乘法形式,
\begin{equation}
\sigma_\nu = \nu \cdot \sigma
\end{equation}
注意到,切面上的应力$\sigma_\nu$不一定垂直于该平面。该公式的具体论证方法大致是在三角形台中运用力的平衡方程,此处按下不表。

\begin{figure}[ht]
\centering
\includegraphics[width=5cm]{./figures/CHYSTR_2.pdf}
\caption{$\sigma_\nu$中各个分量的含义} \label{CHYSTR_fig2}
\end{figure}

那么,我们写出的$\sigma_\nu=
\begin{pmatrix}
\sigma_{\nu,1}\\
\sigma_{\nu,2}\\
\sigma_{\nu,3}\\
\end{pmatrix}$中各个数值的具体含义是什么?显然(但在实操中又有点让人困惑),这是基于原先的坐标系给出的。