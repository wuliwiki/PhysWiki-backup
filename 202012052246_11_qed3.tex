% 洛伦兹群覆盖群SL(2,C)的不可约表示
\pentry{洛伦兹群\upref{qed1},不可约表示\upref{qed1}}
\subsection{李代数的重新推导}
洛伦兹群的李代数是
\begin{equation}
\begin{aligned}
\left[J_{i}, J_{j}\right] &=i \epsilon_{i j k} J_{k} \\
\left[J_{i}, K_{j}\right] &=i \epsilon_{i j k} K_{k} \\
\left[K_{i}, K_{j}\right] &=-i \epsilon_{i j k} J_{k}
\end{aligned}
\end{equation}
洛伦兹群元表达为
\begin{equation}
L=e^{i \mathbf{J} \cdot \theta+i \mathbf{K} \cdot \phi}
\end{equation}
引入新算符
\begin{equation}
N_{i}^{\pm}=\frac{1}{2}\left(J_{i} \pm i K_{i}\right)
\end{equation}
经过计算,新的生成元的对易关系如下所示
\begin{equation}
\begin{aligned}
\left[N_{i}^{+}, N_{j}^{+}\right] &=i \epsilon_{i j k} N_{k}^{+} \\
\left[N_{i}^{-}, N_{j}^{-}\right] &=i \epsilon_{i j k} N_{k}^{-} \\
\left[N_{i}^{-}, N_{j}^{+}\right] &=0
\end{aligned}
\end{equation}
显然,这与$SU(2)$群的李代数一致,也就是说洛伦兹群包含了两份SU(2)的李代数.然而洛伦兹群不是单连通群 118 ,李群理论告诉我们,对于非单连通群,不存在李代数的不可约表示和群的表示之间的一一映射 .通过推
导洛伦兹群李代数的不可约表示,可以导出洛伦兹群的覆盖群的表示.可以证明,其覆盖群为$SL(2,C)$.所以$SL(2,C)=SU(2)\oplus SU(2)$.$SU(2)$的每一不可约表示都可以用$SU(2)$的Casimir元对应的标量\textbf{j}来标记.(回忆一下,Casimir算符是群元中与群的所有生成元都对易的算符.本征值在群元所作变换下不变所以可以拿来标记群表示.)设这两份$SU(2)$的$j$为$j_1,j_2$,则SL(2,C)可以以($j_1,j_2$)作为群表示的标记.
\subsection{SL(2,C)与SO(3,1)关系的简单证明}
令坐标矢量为$(x_0,x_1,x_2,x_3)$,$SO(3,1)$的群元为$\Lambda$.$x'^{\nu}=\Lambda_{v}^{\mu} x^{\nu}$,洛伦兹变换使得该时空距离不变,即$x^\nu x^\nu=x'^\nu x'^\nu$.
\\一般的厄米矩阵可以表示为
\begin{equation}
\chi=\left[\begin{array}{cc}
x_0+x_3 & x_1-\mathrm{i} x_2 \\
x+\mathrm{i} _2 & x_0-x_3
\end{array}\right]
\end{equation}
\subsection{SL(2,C)的几个表示}
\subsubsection{($0,0$)表示}
\subsubsection{( $\frac{1}{2},0$)表示与($0,\frac{1}{2}$)表示}
\subsubsection{( $\frac{1}{2},\frac{1}{2}$)表示}