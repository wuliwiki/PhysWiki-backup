% 磁单极子
% keys 磁单极子|麦克斯韦方程组|高斯单位
% license Xiao
% type Tutor

\begin{issues}
\issueDraft
\end{issues}

\pentry{麦克斯韦方程组\upref{MWEq}, 洛伦兹力\upref{Lorenz}}

\addTODO{狄拉克弦,狄拉克量子化条件,Wu-Yang 单极子}
\footnote{参考 Wikipedia \href{https://en.wikipedia.org/wiki/Magnetic_monopole}{相关页面}。}\textbf{磁单极子(magnetic monopole)}是电动力学中的一类假想的粒子, 至今未被真实观测到。
\subsection{磁单极子的磁场与磁矢势}
假设一个静止在原点的磁单极子,磁荷为 $e_M$,则磁场满足类似于电场高斯定律的方程,可以写出磁场的散度与磁荷成正比的方程:
\begin{equation}
\div \bvec B(\bvec x)=e_M\delta^3(0)~.
\end{equation}

\subsection{麦克斯韦方程组}
 在麦克斯韦方程组\upref{MWEq} 出现后, 人们注意到若假设磁单极子存在并能像带电粒子产生电场那样产生磁场以及像电流产生磁场那样产生电场, 那么电场 $\bvec E$ 和磁场 $\bvec B$ 的地位就完全平等了。

磁单极子是一类假想粒子的名称, 类似于把质子和电子等微观带电粒子称为 “电单极子”。 类比电荷, 我们说磁单极子中带有\textbf{磁荷(magnetic charge)}。 磁荷的国际单位是 $\Si{Am}$(安培·米), 以下把磁荷记为 $q_m$, \textbf{磁荷密度(magnetic charge density)}记为 $\rho_m$, 则麦克斯韦方程组变为
\begin{equation}
\begin{aligned}
&\div \bvec E = \frac{\rho}{\epsilon_0}~,\\
&\curl \bvec E = - \mu_0 \bvec j_m -\pdv{\bvec B}{t}~,\\
&\div \bvec B = \mu_0 \rho_m~,\\
&\curl \bvec B = \mu_0 \bvec j + \mu_0\epsilon_0 \pdv{\bvec E}{t}~.
\end{aligned}
\end{equation}
电荷和磁荷的总洛伦兹力(\autoref{eq_Lorenz_1}~\upref{Lorenz})变为
\begin{align}
\bvec F = q \qty(\bvec E + \bvec v \cross \bvec B) +
q_m \qty(\bvec B - \bvec v \cross \bvec E)~.
\end{align}


\subsubsection{高斯单位}
高斯单位制\upref{GaussU}下, 麦克斯韦方程组和洛伦兹力具有更对称的形式
\begin{equation}\label{eq_BMono_1}
\begin{aligned}
&\div \bvec E = 4\pi\rho~,\\
&\curl \bvec E = -\frac{1}{c}\pdv{\bvec B}{t}  - \frac{4\pi}{c}\bvec j_m~,\\
&\div \bvec B = 4\pi\rho_m~,\\
&\curl \bvec B = \frac{1}{c}\pdv{\bvec E}{t} + \frac{4\pi}{c} \bvec j~.
\end{aligned}
\end{equation}
\begin{equation}
\bvec F = q \qty(\bvec E + \frac{\bvec v}{c}\cross \bvec B) + q_m \qty(\bvec B - \frac{\bvec v}{c}\cross \bvec E)~.
\end{equation}