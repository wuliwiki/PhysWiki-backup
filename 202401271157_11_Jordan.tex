% Jordan(若尔当)标准形
% license Usr
% type Tutor


\begin{issues}
\issueDraft 全文替换x
\end{issues}
\pentry{零化多项式\upref{nullpl}}
尽管复数域可以保证$n$阶矩阵的特征多项式都有$n$个解,但依然不是所有矩阵都有$n$个线性无关的特征向量从而可以对角化。为了简化问题,我们需要“简化”矩阵(找一组基,使得矩阵在该基下有比较简单的形式,比较多$0$)。\autoref{the_nullpl_1}~\upref{nullpl}保证我们线性变换都有分块对角矩阵的形式。更进一步,通过在根子空间的相似变换,我们可以把每一个“对角块”上三角化。

与之比较,本节介绍的Jordan标准形是更加简化的形式。
\subsection{幂零变换的循环子空间分解}
设矩阵$B$为线性空间$V$上的幂零矩阵,即对于任意\textbf{非零}向量$x$,总存在非负整数$k$使得$B^{k}(x)=\bvec 0$且$B^{k-1}(x)\neq \bvec 0$。可以证明,$\{x,Bx,B^2x,...B^{k-1}x\}$线性无关。
\begin{exercise}{}
证明:$\opn{Span}\{x,Bx,B^2x,...B^{k-1}x\}$张成线性空间,并证明这是$B$的不变子空间。
\end{exercise}
设$W=\opn{Span}\{x,Bx,B^2x,...B^{k-1}x\}$,将幂零变换$B$限制在该不变子空间上,记为$B|_W$。则每一列可表示为:$(Bx,B^2x,...B^{k-1}x,B^kx)$,即:
\begin{equation}
\left(\begin{array}{cccccc}
0 & 0 & 0 & \cdots & 0 & 0 \\
1 & 0 & 0 & \cdots & 0 & 0 \\
0 & 1 & 0 & \cdots & 0 & 0 \\
\vdots & \vdots & \vdots & \ddots & \vdots & \vdots \\
0 & 0 & 0 & \cdots & 1 & 0
\end{array}\right)~.
\end{equation}m

假设该线性空间为4维。输入$(1\,0\,0\,0)$,得到$(0\,1\,0\,0)$。输入$(0\,1\,0\,0)$,得到$(0\,0\,1\,0)$.$B$可以对基进行循环,无怪乎把这种形式下的基向量组$\opn{Span}\{x,Bx,B^2x,...B^{k-1}x\}$称为循环基。$W$则是$B$的循环子空间

\subsection{Jordan 标准形}