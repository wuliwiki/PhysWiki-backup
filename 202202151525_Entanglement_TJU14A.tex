% 天津大学 2014 年考研量子力学答案
% 考研|天津大学|量子力学|2014|答案

\begin{issues}
\issueTODO
\issueDraft
\end{issues}

\subsection{ }
\begin{enumerate}
\item 对于$\psi(r,\theta,\varphi) = \frac{1}{\sqrt{5}}\psi_{310} + \frac{2}{\sqrt{5}}\psi_{211} $,主量子数$n$可能等于$3,2$.\\
\begin{table}[ht]
\centering
\caption{$\hat{L}_{z},\hat{L}^{2}$的可能值与几率}\label{TJU14A_tab1}
\begin{tabular}{|c|c|c|}
\hline
$\hat{L}_z$ 的可能值 & 0 & $\hbar$  \\
\hline
$\hat{L}^2$ 的可能值 & $2\hbar^{2}$ & $2\hbar^{2}$  \\
\hline
相应几率 & $\frac{1}{5}$ & $\frac{4}{5}$  \\
\hline
\end{tabular}
\end{table}
$\hat{L}_{z}$和$\hat{L}^{2}$的平均值为:\\
\begin{align}\label{TJU14A_eq1}
& \overline{\hat{L}_{z}} = 0 \times \frac{1}{5} + \hbar \times \frac{4}{5} = \frac{4\hbar}{5} \\
& \overline{\hat{L}^{2}} = 2\hbar^{2} \times \frac{1}{5} + 2\hbar^{2} \times \frac{4}{5} = 2\hbar^{2}
\end{align}
\item (1) 光的波动性:光的干涉现象,光的衍射现象.光的粒子性:光电效应,康普顿效应.\\
(2)戴维孙—革末实验,即电子衍射实验除了证实电子具有粒子性之外也具有波动性.
\item 设均匀磁场方向沿$x$方向则:
\begin{equation}
\begin{aligned}
\hat{H}=& -\vec{\mu}\vdot \vec{B} \\
=& g_{n}\vec{S}_{x}\vdot \vec{B} \\
=& gB\hat{S}_{x}
\end{aligned}
\end{equation}

在$(\hat{S}^{2},\hat{S}_{z})$表象中:
\begin{equation}
\hat{H}=\frac{gB\hbar}{2}\bmat{0&1\\1&0}
\end{equation}

设体系的波函数为$\bmat{a\\b}$,能量为$E$则有:
\begin{equation}
\frac{gB\hbar}{2}\bmat{0&1\\1&0}\bmat{a\\b}=E\bmat{a\\b}
\end{equation}

其久期方程为:
\begin{equation}
\vmat{-E&\frac{gB\hbar}{2}\\\frac{gB\hbar}{2}&-E}=0
\end{equation}

解得$E_{1}=\frac{gB\hbar}{2}$,$E_{2}=-\frac{gB\hbar}{2}$.故电子的能级可能为$\frac{gB\hbar}{2}$,$-\frac{gB\hbar}{2}$.
\end{enumerate}
\subsection{ }
\begin{enumerate}
\item 对于二维谐振子,其势能为:
\begin{equation}
V(x,y)=\frac{1}{2}m\Omega^{2}(x^{2}+y^{2})
\end{equation}
哈密顿量为:
\begin{equation}
\hat{H}=-\frac{\hbar}{2m}(\dv[2]{x}+\dv[2]{y})+\frac{1}{2}m\Omega^{2}(x^{2}+y^{2})
\end{equation}
设其波函数为$\psi(x,y)$,能量为$E$,则
\begin{equation}
\qty[ -\frac{\hbar^{2}}{2m}(\dv[2]{x} + \dv[2]{y})+\frac{1}{2}m\Omega^{2}(x^{2}+y^{2})]\psi(x,y)=E\psi(x,y)
\end{equation}
分离变量$\psi(x,y)=\phi(x)\phi(y)$,然后在等式两边同时除以$\phi(x)\phi(y)$,则有:
\begin{equation}
\frac{\qty[-\frac{\hbar^{2}}{2m}\dv[2]{x}+\frac{1}{2}m\Omega x^{2}]\phi(x)}{\phi(x)}+\frac{\qty[-\frac{\hbar^{2}}{2m}\dv[2]{y}+\frac{1}{2}m\Omega y^{2}]\phi(y)}{\phi(y)}=E
\end{equation}
因此可得:
\begin{align}
&\qty[-\frac{\hbar^{2}}{2m}\dv[2]{x}+\frac{1}{2}m\Omega^{2}x^{2}]\phi{x}=E_{x}\phi(x)\\
&\qty[-\frac{\hbar^{2}}{2m}\dv[2]{y}+\frac{1}{2}m\Omega^{2}y^{2}]\phi{y}=E_{y}\phi(y)
\end{align}
解得:
\begin{align}
&\phi(x)=N_{n_{x}}e^{-\frac{\alpha^{2}x^{2}}{2}}H_{n_{x}}(\alpha x),\quad E_{x}=(n_{x}+\frac{1}{2})\hbar \Omega,\quad n_{x}=0,1,2\cdots \\
&\phi(y)=N_{n_{y}}e^{-\frac{\alpha^{2}y^{2}}{2}}H_{n_{y}}(\alpha y),\quad E_{y}=(n_{y}+\frac{1}{2})\hbar \Omega,\quad n_{y}=0,1,2\cdots
\end{align}
因此有:
\begin{equation}
\psi(x,y)=\phi(x)\phi(y)=N_{n_{x}}N_{n_{y}}e^{-\frac{\alpha^{2}(x^{2}+y^{2})}{2}}H_{n_{x}}(\alpha x)H_{n_{y}}(\alpha y)
\end{equation}
\begin{equation}
E=(n_{x}+n_{y}+1)\hbar \Omega=(N+1)\hbar \Omega,\quad N=0,1,2,\cdots
\end{equation}
\item 当处于基态时,非简并.当处于第$n$激发态时,其简并度为$n+1$.
\item 因为
\begin{align}
&H_{n_x}(-\alpha x)=(-1)^{n_x}H_{n_x}(\alpha x) \\
&H_{n_y}(-\alpha y)=(-1)^{n_y}H_{n_y}(\alpha y)
\end{align}
故而
\begin{equation}
H_{n_x}(-\alpha x)H_{n_y}(-\alpha y)=(-1)^{n_x + n_y}H_{n_x}(\alpha x)H_{n_y}(\alpha y)
\end{equation}
答:所以当$N$为奇数时,奇宇称;当$N$为偶数时,偶宇称.
\end{enumerate}
\subsection{ }
\begin{enumerate}
\item $\psi=\sqrt{\frac{1}{3}}\psi_{0}(x)+\sqrt{\frac{2}{3}}\psi_{1}(x)$
对于一维谐振子,其本征能量为:
\begin{equation}
E_{n}=(n+\frac{1}{2})\hbar \omega,\quad n=0,1,2,\cdots
\end{equation}
因此任意时间的波函数为:
\begin{equation}
\psi(x,t)=\sqrt{\frac{1}{3}}\psi_{0}(x)e^{-\frac{i}{\hbar}E_{0}t}+\sqrt{\frac{2}{3}}\psi_{1}(x)e^{-\frac{i}{\hbar}E_{1}t}
\end{equation}
\end{enumerate}