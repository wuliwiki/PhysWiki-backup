% 仿射变换在解析几何中的应用
% 仿射变换 圆锥曲线
\begin{definition}{仿射变换}
设椭圆\,\(\frac{x^2}{a^2}+\frac{y^2}{b^2}=1\),其中\,\(a>b>0\),置变换:
$$x'=\frac{x}{a},y'=\frac{y}{b}$$
则椭圆化为单位圆\,\(C:x'^2+y'^2=1\)
\end{definition}
届时,我们可以就可以抛开繁琐的代数计算,运用几何性质解决问题.此前,我们先介绍仿射变换的几个性质.
\begin{lemma}{}
变换后,平面内任意一条直线的斜率变为原来的\,\(\frac{a}{b}\)
\end{lemma}
\begin{lemma}{}
变换后,平面上任意区域的面积变为原来的\,\(\frac1{ab}\)
\end{lemma}
\begin{lemma}{}
变换后,线段中点依然是线段中点;关于坐标轴对称的元素依然关于坐标轴对称;平面区域的重心保持不变
\end{lemma}
\begin{lemma}{}
变换前后,平行关系保持不变
\end{lemma}
\begin{lemma}{}

\end{lemma}
已知椭圆 
:
,过  的直线  交  于  两点,且  ,求  斜率

解:作变换  ,则椭圆化为单位圆 

由性质5,仍有  ,记  ,于是 

由切割线定理,  ,其中  为单位圆与  轴的两交点

即 

接着:

由垂径定理, 

整理,由性质1得:

