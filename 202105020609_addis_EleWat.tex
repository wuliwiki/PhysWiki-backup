% 电路和水路的形象对比

\pentry{电压和电动势\upref{Voltag}}

\addTODO{图}

形象来说, 我们可以把电路中的导线类比为水平面上的水管, 管道中充满某种不可压缩的非粘稠流体, 即该流体不会因自身的摩擦损耗能量. 导线中自由电子的电荷可以类比流体的质量, 电流则类比液体单位时间流经管道横截面的质量. 没有电阻的理想导线类比为内壁无摩擦的管道.

在该模型中, 电势可以理解为管道内某点处的绝对水压. 事实上把整个管道系统的绝对水压处处增加一个常数并不会影响水流动的方式和快慢, 所以我们主要关心的是管道两点之间的绝对压强之差, 正如在电路中我们关心的不是绝对电势而是电势差(电压).

在电路中, 当移动的速度其实非常慢. 我们可以认为一般电路中的电流所对应的水流的速度极小, .

\subsubsection{电阻}
在以上类比中, 电路中的电阻\upref{Resist}可以看成在管道中塞入一块海绵用于阻碍水流. 海绵越密对水的阻力就越大, 只有通过增加海绵两侧的压强差才能让水流变快.

\subsection{电源}
在该类比中, 电源可以看成一个水泵, 如果




下面我们会看到

某处离水平地面的高度, 而电压/电势差可以理解为管道不同两点的高度只差.
