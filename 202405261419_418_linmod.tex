% 线性同余
% keys 线性同余
% license Usr
% type Tutor

\pentry{同余与剩余类\nref{nod_modres},最大公约数与最小公倍数\nref{nod_gcdlcm}}{nod_8e3a}

类比普通代数的解方程,同余式也有类似方程的问题,例如:
\begin{equation}
ka \equiv ka' \xrightarrow{?} a \equiv a'  ~.
\end{equation}
显然这不一定是成立的,例如 $2 \times 2 \equiv 2 \times 4 \pmod 4$,但 $2 \not \equiv 4 \pmod 4$。线性同余就是讨论类似这样的问题。

\begin{theorem}{}\label{the_linmod_1}
若 $(k, m) = d$,则
\begin{equation}
ka \equiv ka' \pmod m \Leftrightarrow a \equiv a' \pmod{\frac{m}{d}} ~.
\end{equation}
\end{theorem}
\textbf{证明}:考虑 $k = k_0 d$ 而 $m = m_0 d$,则 $(k_0, m_0) = 1$。由
\begin{equation}
\frac{ka - ka'}{m} = \frac{k_0(a - a')}{m_0} ~,
\end{equation}
而 $(k_0, m_0) = 1$,故 $m | (k a - k a')$ 当且仅当 $m_0 | (a - a')$,证毕!

\begin{theorem}{}\label{the_linmod_2}
若 $a_1, a_2, \dots, a_m$ 是模 $m$ 的一个完全剩余系,且整数 $k$ 满足 $(k, m) =1$,则 $ka_1, ka_2, \dots, ka_m$ 也是模 $m$ 的一个完全剩余系。
\end{theorem}
\textbf{证明}:对\autoref{the_linmod_1} 取 $d=1$ 时的情况,则由 $ka_i - ka_j \equiv 0 \pmod m$ 可以得到 $a_i - a_j \equiv 0 \pmod m$。而这仅当 $i = j$ 时才可能成立,故各 $ka_i$ 与 $ka_j$ 当 $i \neq j$ 时模 $m$ 不同余,而一共有 $m$ 个数,故必然构成一个 $m$ 的完全剩余系,证毕!

\pentry{整数模与裴蜀定理\nref{nod_mobezt}}{nod_72e0}
\begin{theorem}{}
若 $(k, m) = d$,则同余方程 $kx \equiv l \pmod m$ 有解,当且仅当 $d | l$。且有解时,若考虑同余的解是相同的一组解,则其恰有 $d$ 组不同的解。
\end{theorem}
\textbf{证明}:这同余式 $kx \equiv l \pmod m$ 等价于解 $kx - my = l$,而由裴蜀定理可以知道其有解当且仅当 $d | l$。

现在,我们考虑解的个数。若 $d=1$,则可以直接由刚才的\autoref{the_linmod_2} 得到,仅有一组同余的解。若 $d > 1$,考虑有解时,则 $d|l$,设 $m = m_0 d$,$k = k_0 d$,$l = l_0 d$,则此时 $(k_0, m_0) = 1$,且 $kx \equiv l \pmod m$ 等价于解(直接运用刚才的\autoref{the_linmod_1} 即可得到)
\begin{equation}
k_0 x \equiv l_0 \pmod {m_0} ~.
\end{equation}
而此时 $(k_0, m_0) = 1$,故这仅有一组解。考虑设这组解是 $x \equiv x_0 \pmod {m_0}$,则 $\forall c \in \mathbb Z$,$x = x_0 + c m_0$ 都是解。
