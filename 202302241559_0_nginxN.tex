% nginx 笔记

\begin{issues}
\issueDraft
\end{issues}

\begin{itemize}
\item 一个\href{https://zhuanlan.zhihu.com/p/80600540}{知乎教程}
\item \href{https://nginx.org/en/docs/}{官方文档}
\item \verb|sudo apt install nginx|
\item 要重启 nginx 服务用 \verb|sudo systemctl restart nginx|
\item 要查看是否连接成功, 用 \verb|curl localhost| 或者 \verb|curl http://localhost|(默认访问 80 接口)。 如果打印出一个 html 文本, 包含 \verb|Welcome to nginx!|, 就是成功了(当然也可以用浏览器访问, 只是有时候只有命令行)。
\item 如果要限制 nginx 只监听某个网卡, 编辑配置文件 \verb|sudo vim /etc/nginx/sites-enabled/default|, 然后在 \verb|listen 80 default_server;| 的 \verb|80| 改成 \verb|ip地址:80|, 然后重启 \verb|nginx| 服务即可生效。
\item 事实上不光是本机, 监听的网卡所在的所有机器访问该网卡的 ip 的 80 端口都会收到
\end{itemize}

\subsection{静态网站}
\begin{itemize}
\item 配置文件: \verb|/etc/nginx/nginx.conf|。 在 \verb|http| section 里面加入
\begin{lstlisting}[language=none]
server {
    listen ip地址:80;
    server_name ip地址或者域名:80;
    
    location / {
        root /静态网页根目录;
    }
}
\end{lstlisting}
\item 要特别注意 \verb|静态网页根目录| 以及它的所有上层目录需要可以被 nginx 的用户 \verb|www-data| 读取和执行, 里面的文件也一样。如果权限不对访问网页会出现错误 \verb|403 forbidden|。
\item 在非 ubuntu 系统中 nginx 可能会有别的用户名。  要查看具体的用户名, 用 \verb`ps aux | grep nginx`, 看第一列中除了 \verb|root| 都有哪些用户。
\item 要访问静态, 在本机或者其他机器的命令行用 \verb|curl ip地址或者域名|, 如果 \verb|ip| 不是公网 ip 就只能在局域网的机器上访问。 如果机器上有 GUI 浏览器, 也可以直接在网址栏输入 \verb|ip地址或者域名|。
\item 如果只是从本机访问, 那么 \verb|ip地址或者域名| 中可以使用任何域名不需要注册。
\end{itemize}

\subsubsection{配置 https}
\begin{itemize}
\item 首先要申请一个 \textbf{SSL/TLS 证书}: 比较著名的证书颁发机构如 \href{https://letsencrypt.org/}{Let's Encrypt}。 我们以它为例。
\item 一般最好有一个域名, 因为 SSL/TLS 证书是颁发给域名而不是 ip 的。 确保 \verb|80| 端口可以访问: \verb|http://yourwebsite.com|。
\item 安装 \href{https://certbot.eff.org/instructions?ws=nginx&os=ubuntufocal}{Certbot}, 用于自动获取以及更新证书:
\item 首先更新 snapd: \verb|sudo snap install core; sudo snap refresh core|
\item 卸载老版本: \verb|sudo apt remove certbot|
\item 安装 cerbot: \verb|sudo snap install --classic certbot|
\item \verb|sudo ln -s /snap/bin/certbot /usr/bin/certbot|
\item 获取证书,指定 nginx: \verb|sudo certbot certonly --nginx|, 这时会互动提示输入域名等信息。
\item 测试自动更新: \verb|sudo certbot renew --dry-run|
\item 现在就可以访问域名: \verb|https://yourwebsite.com/|
\item 申请证书用子域名也是可以的。
\end{itemize}
