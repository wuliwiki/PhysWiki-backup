% 牛顿第二定律的矢量形式
% 牛顿定律|矢量|加速度|速度|曲线运动

\begin{issues}
\issueTODO
\end{issues}

\pentry{圆周运动(高中)\upref{HSPM05}, 几何矢量的运算\upref{GVecOp}}

我们高中已经熟悉的是一维直线运动的牛顿第二定律\upref{HSPM03}, 也就是常见的标量形式 $F = ma$。 高中物理已经明确过力和加速度都是\textbf{矢量}, 本书中矢量用黑体和正体表示, 即 $\bvec F$ 和 $\bvec a$。 在大学物理中, 我们几乎总是使用矢量形式的牛顿第二定律。 我们知道, 矢量既有长度也有方向, 但是在一维情况下, 矢量只有两个方向, 我们把其中一个定义为\textbf{正方向}, 那么另一个就是\textbf{反方向}。

物理中有一个简单的约定, 一维的矢量可以用\textbf{标量}也就是一个实数表示, 对于正方向的矢量, 就使用正数表示, 反方向的矢量就使用负数。 所以\textbf{在一维情况下, 矢量与标量可以一一对应}。 例如, 标量 $3$ 可以对应指向正方向且长度为 $3$ 的矢量, 而 $-2$ 则代表指向反方向且长度为 $2$ 的矢量。 所以标量形式的牛顿第二定律只是矢量形式牛顿第二定律在一维直线运动时的特殊情况。

在更高维空间的运动中, 也就是二维平面和三维空间中, 牛顿第二定律适用于质点的任意曲线运动, 质点\textbf{速度矢量}的方向和大小都可以随时间任意连续变化。 牛顿第二定律的矢量形式是
\begin{equation}
\bvec F = m\bvec a~.
\end{equation}
它比标量形式所包含的意义要丰富得多。 等式两边的两个矢量相等, 就要求它们的长度和方向都相等。 右边的标量 $m$ 乘以矢量 $\bvec a$ 的意义是保持 $\bvec a$ 的方向不变, 而长度乘以 $m$, 详见 “几何矢量的运算\upref{GVecOp}” 中 “\textbf{数乘}” 的定义。 所以由该式可知 $\bvec F$ 和 $\bvec a$ 的方向时刻相同, 而长度总是成正比, 比例常数为 $m$。

一个典型的曲线运动例子就是在匀速圆周运动\upref{HSPM05}中, 质点和圆心之间的距离没有改变, 速度在向心方向的分量为零, 但却存在向心加速度。 根据牛顿第二定律, 做该运动的质点必须要受一个向心力才能保持匀速圆周运动。 相信在高中一些课堂中老师会用所谓的 “微元法” 推导一次向心加速度, 但大部分学生过一段时间以后仅仅只是记得向心加速度的公式 $a = v^2/r = \omega^2 r$, 却已经忘记这个加速度从何而来, 为什么会出现。 而在另一些课堂上, 这个公式仅作为实验结论而不做任何推导。

乍看之下, 圆周运动半径不变, 连向心速度都恒为零, 何来加速度? 问题的关键就在于如何定义一般的矢量加速度 $\bvec a$。 我们先来看\textbf{直线运动}中 “标量版加速度” 的定义\upref{HSPM01}
\begin{equation}\label{eq_New2_1}
a(t) = \lim_{\Delta t\to 0}\frac{v(t+\Delta t) - v(t)}{\Delta t}~.
\end{equation}
这里的 $\lim_{\Delta t\to 0}$ 表示 $\Delta t$ 需趋近于 0 时后面的表达式取极限\upref{FunLim}。 该式的意义是在一段微小时间段 $[t,\ t+\Delta t]$ 内, 速度随时间变化的快慢。 注意该式仅限于直线运动, 所以一旦速度方向发生改变, 那么这个定义将立即不适用。 千万不要误以为在曲线运动时, 该式中 $a$ 代表 $\bvec a$ 的大小, $v$ 代表 $\bvec v$ 的大小——这是非常严重的错误。 例如在匀速圆周运动中, 若这么认为, 那么由于速度大小不变, 将得到加速度为零的结论。 \textbf{矢量加速度不是速度大小的变化率, 而是速度矢量的变化率}。 后者的定义是:
\begin{equation}\label{eq_New2_2}
\bvec a(t) = \lim_{\Delta t\to 0}\frac{\bvec v(t+\Delta t) - \bvec v(t)}{\Delta t}~.
\end{equation}
该式和\autoref{eq_New2_1} 中最关键的不同就是, 当两个速度相减时, 使用的是\textbf{矢量相减}而不是模长相减(\autoref{fig_New2_1})。 矢量的减法需要用三角形法则, 这在高中数学和物理中经常出现, 简单来说就是先把两矢量\textbf{起点平移到一起}, 再从第二个矢量的末端指向第一个矢量的末端, 详见 “几何矢量的运算\upref{GVecOp}”。

\begin{figure}[ht]
\centering
\includegraphics[width=9cm]{./figures/4a8edc4f9c579934.pdf}
\caption{若质点延黑色曲线运动, $t$ 时刻的速度为 $\bvec v(t)$, $t+\Delta t$ 的速度为 $\bvec v(t+\Delta t)$。 要计算速度矢量的增量, 把两矢量的起点平移到一起后使用三角形法则做矢量减法即可。 把 $\Delta \bvec v$ 的长度除以 $\Delta t$, 并令 $\Delta t$ 为无穷小, 即可得到加速度矢量。} \label{fig_New2_1}
\end{figure}

我们需要记住, 几何矢量只包含方向和长度两个信息, 所以任意矢量平移后都仍然是同一矢量。 一般将速度矢量的起点画在质点处的做法容易产生误导, 让人以为速度矢量的起点也是重要信息(质点的位置需要用位置矢量来描述)。 所以如果在\autoref{fig_New2_1} 中未经过平移就把两个矢量的末端相连是错误的。

为什么要用\autoref{eq_New2_2} 来定义加速度而不是其他定义? 因为无数实验表明只有这样定义的加速度才符合矢量形式的牛顿第二定律。 我们研究加速度的主要原因就是为了使用牛顿第二定律。

如果你还是不理解加速度矢量和速度矢量的关系, 不妨看看速度矢量和位置矢量的关系——二者在数学上是一模一样的!
\begin{equation}
\bvec v(t) = \lim_{\Delta t\to 0}\frac{\bvec r(t+\Delta t) - \bvec r(t)}{\Delta t}~.
\end{equation}
如\autoref{fig_New2_2}, 速度矢量的定义比加速度矢量容易理解是因为位置矢量的起点本来就都画在一起, 所以我们不需要经过 “平移” 这步就可以直接把它们相减。
\begin{figure}[ht]
\centering
\includegraphics[width=6cm]{./figures/551fc75349507ec5.pdf}
\caption{若质点延黑色曲线运动, $t$ 时刻质点的位置矢量为 $\bvec r(t)$,$t+\Delta t$ 时刻位置矢量为 $\bvec r(t+\Delta t)$, 由于位置矢量的起点都是坐标原点, 可以直接连接两矢量的末端求出它们的矢量差。 把 $\Delta \bvec r$ 的长度除以 $\Delta t$, 并令 $\Delta t$ 为无穷小, 即可得到速度矢量。} \label{fig_New2_2}
\end{figure}

为了进一步进行类比, 我们可以在质点运动的过程中不但画出它位置矢量末端划出的轨迹, 也同时画出它速度矢量末端(先把起点平移到同一个位置)划出的轨迹, 前者的速度就是速度矢量, 而后者的 “速度” 就是加速度矢量。
\addTODO{如果这个也能做成动画就好了}

这个过程还能一直继续下去, 例如画出加速度端点的轨迹, 它的 “速度” 就是\textbf{加加速度}。 所有的这些在数学上看就是一个矢量函数的\textbf{ $n$ 阶导数}: \textbf{位置矢量(关于时间的)一阶导数是速度, 二阶导数是加速度}, 三阶导数是加加速度…… 但为什么我们一般不需要比加速度更高的高阶导数呢? 因为牛顿第二定律中只出现了加速度, 所以更高阶的加速度几乎不被使用。 由于牛顿第二定律中的加速度是位置矢量的二阶导数, 我们说牛顿第二定律是一个二阶微分方程。

\begin{example}{匀速圆周运动}
\addTODO{动画: 左右两幅图, 左边画位置空间, 右边画速度空间。 另外可以参考 “圆周运动的加速度\upref{CMAD}”}
\end{example}

\subsection{加速度的垂直分解}

以二维矢量为例, 矢量可以投影到 $x$ 和 $y$ 两个垂直的方向, 投影的长度分别称为 $x, y$ 分量。 通过两个分量就可以唯一地确定原来的矢量。

从坐标的角度来看, 矢量的减法就是矢量坐标分别相减, 所以矢量加速度的定义(\autoref{eq_New2_2}) 也可以写为两个分量的形式。 假设 $t$ 时刻速度的两个坐标为 $v_x(t), v_y(t)$, 加速度的两个坐标为 $a_x(t), a_y(t)$, 那么\autoref{eq_New2_2} 就可以等效地写为
\begin{equation}
\begin{aligned}
&a_x(t) = \lim_{\Delta t\to 0}\frac{v_x(t+\Delta t) - v_x(t)}{\Delta t}~,\\
&a_y(t) = \lim_{\Delta t\to 0}\frac{v_y(t+\Delta t) - v_y(t)}{\Delta t}~.
\end{aligned}
\end{equation}
可见 $a_x,a_y$ 分别只由 $v_x,v_y$ 定义, 这就是为什么高中物理在研究抛物运动时\upref{HSPM04}, 可以把运动分解到两个方向上各自独立讨论。 由于理想抛体运动只在 $y$ 方向存在加速度, 这么做就可以使问题更简单。

事实上圆周运动也可以分解到 $x,y$ 两个方向上, 形成两个简谐振动。 然而相比于抛体运动, 圆周运动两个方向都有随时间变化的加速度, 这么做并不会使问题变得更简单, 反而是匀速旋转的向心加速度矢量更容易理解。
