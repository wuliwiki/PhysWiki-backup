% 埃尔米特矢量空间(酉空间)
% keys 埃尔米特矢量空间|酉空间|施瓦茨不等式

\begin{issues}
\issueTODO
\end{issues}

\pentry{欧几里得矢量空间\upref{EuVS}}
$\mathbb{R}$ 上的欧几里得矢量空间的度量关系完全可由纯量积来刻画,这成为在复数域 $\mathbb{C}$ 上的矢量空间中引入纯量乘积的刺激因素.然而,在复矢量空间的情形,使用标准双线性型 $s(\bvec x,\bvec y)=\sum_{i}x_iy_i\;x_i,y_i\in\mathbb{C}$ 作为纯量积并不能胜任此任务,因为对 $\norm{x}\geq0$,模
\begin{equation}
\norm{\I\bvec x}=\sqrt{s(\I\bvec x,\I\bvec x)}=-\norm{\bvec x}\leq0
\end{equation}
要是想利用直观的矢量长度的概念,上面的纯量乘积定义显然是不能接受的.但是,使用正定的埃米尔特型(\autoref{HeFor_def1}~\upref{HeFor})作为纯量乘积的定义却是适合的.
\subsection{埃尔米特矢量空间(酉空间)}
\begin{definition}{}
域 $\mathbb{C}$ 上一个有限维矢量空间 $V$ 配备一个正定埃尔米特型$(\bvec x|\bvec y):=f(\bvec x|\bvec y)$,则称为\textbf{埃尔米特矢量空间}(或\textbf{酉空间}).复数 $(\bvec x|\bvec y)$ 称为是矢量 $\bvec x,\bvec y\in V$ 的\textbf{纯量积}(或\textbf{内积}).
\end{definition}
列出纯量积的性质:
\begin{enumerate}
\item $(\bvec x|\bvec y)=\overline{(\bvec x|\bvec y)}$;
\item $(\alpha\bvec x+\beta\bvec y|\bvec z)=\alpha^*(\bvec x|\bvec z)+\beta(\bvec y|\bvec z)$;
\item $(\bvec x|\bvec x)\geq0$,仅当 $\bvec x=\bvec 0$ 时等式成立.
\end{enumerate}

这里,仍用 $(*|*)$ 表示纯量乘积,是因为将矢量空间限制在实数域上以上性质仍成立,并不会发生任何矛盾.

有了纯量乘积的定义以后,和欧几里得矢量空间情形完全类似的,在埃尔米特空间,有
\begin{definition}{模}
称 $\norm{\bvec v}=\sqrt{(\bvec v|\bvec v)}$ 为矢量 $\bvec v$ 的\textbf{模}.
\end{definition} 
\begin{theorem}{施瓦茨不等式}
\begin{equation}
\abs{(\bvec x|\bvec y)}\leq \norm{\bvec x}\cdot \norm{\bvec y}
\end{equation}
仅当 $\bvec y=\lambda\bvec x,\lambda\in\mathbb{C}$ 时,等式成立.
\end{theorem}
\textbf{证明:}将复数记为指数形式,即 $(\bvec x|\bvec y)=\abs{(\bvec x|\bvec y)}\E^{\I \varphi}$.那么由纯量乘积的正定性,对 $\forall t\in\mathbb{R}$
\begin{equation}
\begin{aligned}
&(\bvec x t+\bvec y\E^{\I\varphi}|\bvec x t+\bvec y\E^{\I\varphi})\\
&=\norm{\bvec x}^2t^2+\qty((\bvec x|\bvec y)\E^{\I\varphi}+\overline{(\bvec x|\bvec y)}\E^{\I\varphi})t+\norm{\bvec y}^2
\end{aligned}
\end{equation}
