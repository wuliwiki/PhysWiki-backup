% 吸热过程
% license CCBYSA3
% type Wiki

(本文根据 CC-BY-SA 协议转载自原搜狗科学百科对英文维基百科的翻译)

“\textbf{吸热过程}” 是指任何需要或吸收周围环境能量的过程,通常是以热的形式存在。它可能是一个化学过程,如将硝酸铵溶解在水中,也可能是一个物理过程,如冰块的融化。这个术语是由马尔塞林·贝特洛从希腊词根“'endo-” 中创造出来的,其中词“endon”(ἔνδον)意为“内在”,而词“therm”(θεμ-),意为“热的”或“温暖的”,意思是一个反应的进行取决于是否吸收热量。与吸热过程相反的是放热过程,放热过程以热的形式释放“释放”能量。 因此,在每一个术语(吸热和放热)中,前缀是指反应发生时热量的去向,尽管实际上它只指能量的去向,而不是指热量的形式。

\subsection{细节}
