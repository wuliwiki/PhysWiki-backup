% 基本型
% 微分几何|面积|积分|二次型|fundamental form|quadratic form

\pentry{三维空间中的曲面\upref{RSurf}}
\addTODO{草稿阶段}
\subsection{第一基本型}

曲面上的第一基本型,本质上就是用局部坐标系的正定二次型来定义一个切空间上的内积。由于古典微分几何依赖 $\mathbb{R}^3$,我们可以把第一基本型简单理解为,曲面的几何切平面上各切向量的\textbf{长度}。

设 $\bvec{x}$ 是曲面 $S$ 上的某个局部坐标系,那么我们就可以把 $S$ 在这个坐标系内的一条曲线表示为 $\alpha(t)=\bvec{x}(u(t), v(t))$。这里 $(u(t), v(t))$ 可以理解为曲线在时间 $t$ 时的坐标,经 $\bvec{x}$ 映射后成为 $S$ 上的一个点。

再次回忆:曲线就是道路,也就是向量。那么我们可以计算这个向量的长度。首先,这个曲线对应的向量是什么呢?记这个向量为 $\bvec{v}_\alpha\in\mathbb{R}^3$,那么我们有:
\begin{equation}
\begin{aligned}
\bvec{v}_\alpha&=\lim\limits_{t\to 0}\frac{\alpha(t)}{t}=\lim\limits_{t\to 0}\frac{\bvec{x}(u(t), v(t))}{t}\\
&=\bvec{x}_uu'+\bvec{x}_vv'~,
\end{aligned}
\end{equation}
其中 $\bvec{x}_u$ 是 $\frac{\partial\bvec{x}}{\partial u}$、$u'$ 是 $\frac{\dd u}{\dd t}$ 的意思,将 $u$ 换成 $v$ 同理。

这么一来就可以将向量 $\bvec{v}_\alpha$ 的第一基本型定义清楚了。

\begin{definition}{第一基本型}\label{FForm_def1}
给定正则曲面 $S$ 上一点 $p$,切空间 $T_pS$ 处的第一基本型是一个 $T_pS\to \mathbb{R}$ 的光滑映射 $\mathrm{I}_p$,使得对于 $T_pS$ 中的切向量 $\bvec{v}_\alpha$,取其对应的任意道路 $\alpha(t)=\bvec{x}(u(t), v(t))$,有:$$\mathrm{I}_p(\bvec{v}_\alpha)=E\cdot u'^2+G\cdot v'^2+ 2F\cdot u'v'$$
其中
\begin{itemize}
\item $E=\bvec{x}_u^2$
\item $G=\bvec{x}_v^2$
\item $F=\bvec{x}_u\cdot\bvec{x}_v$
\end{itemize}
\end{definition}

由于我们通常是在坐标系里研究曲面的性质,第一基本型的三个实数 $E$、$G$ 和 $F$ 就方便我们直接在坐标系里计算曲面上切向量的长度。

如果 $\{\bvec{x}_u, \bvec{x}_v\}$ 是一个标准正交基,那么第一基本型的形式就和勾股定理一致。如果了解了\textbf{斜坐标系}\upref{ObSys}的概念,你会发现当第一基本型长得不像勾股定理时,都是因为这个基不是标准正交的。

因此,第一标准型可以用作勾股定理的延伸,或者说内积定义的延伸,从而得到现代微分几何中\textbf{黎曼度量}的概念,我们会在后续相关词条中阐述。如果再进行一步延伸,允许三个实数出现负值,那么还可能得到伪黎曼度量,其中一种伪黎曼度量是相对论时空的核心结构。

利用第一基本型,可以方便地计算曲面上某区域的面积。

\begin{theorem}{曲面上的面积}
给定曲面 $S$ 和它的一个局部坐标系 $\bvec{x}:U\to V\subseteq S$。任取 $U$ 的一个子集 $A$,那么曲面上 $\bvec{x}(A)$ 的面积就是
\begin{equation}
\iint_{\bvec{x}(A)}\abs{\bvec{x}_u\times\bvec{x}_v}\dd u\dd v=\iint_{A}\sqrt{EG-F^2} \dd u\dd v
\end{equation}
\end{theorem}


\subsection{第二基本型,以及更多}
\pentry{高斯映射\upref{GMap}}

\begin{definition}{第二基本型}
给定可定向正则曲面 $S$ 和它的一个定向 $N$,将 $N$ 视为 $S\to S^2$ 的\textbf{高斯映射},则 $\dd N_p(\bvec{v}_\alpha)\cdot \bvec{v}_\alpha$ 就是 $\bvec{v}_\alpha$ 的第二基本型,记为 $\mathrm{II}_p(\bvec{v}_\alpha)$。
\end{definition}

