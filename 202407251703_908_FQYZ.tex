% 反氢原子
% license CCBYSA3
% type Wiki

(本文根据 CC-BY-SA 协议转载自原搜狗科学百科对英文维基百科的翻译)

\textbf{反氢(H)}是对应元素氢的反物质。普通的氢原子由电子和质子组成,而反氢原子则由正电子和反质子组成。科学家们希望能够通过研究反氢来揭示在可观测到的宇宙中物质比反物质更多的原因,这也就是众所周知的重子不对称问题。[1] 反氢是在粒子加速器中人工合成的。1999年,美国国家航空航天局(NASA)估计每克反氢物质的成本为62.5万亿美元(相当于今天的94万亿美元),使其成为生产成本最高的材料。[2] 这是由于每次实验的产量极低,而使用粒子加速器的机会成本却很高。

\subsection{实验历史}
20世纪90年代,首次在加速器中检测到热的反氢。ATHENA项目在2002年对冷氢进行了研究。2010年,反氢首次被欧洲核子研究中心 (CERN)[3][4] 的Antihydrogen Laser Physics Apparatus (代号:ALPHA)研究团队捕获,该团队随后测量了反氢的结构和其他重要性质。[5] ALPHA、AEGIS和GBAR计划进一步地降低温度、研究反氢原子。
\subsubsection{1.1 1S–2S跃迁测量}
2016年,ALPHA实验测量了反氢的两个最低能级1S–2S之间的量子跃迁。在实验分辨率范围内,这些结果与氢的结果相同,支持了物质-反物质的对称性和CPT对称性等观点。[6]

磁场存在时,从1S到2S的能级跃迁分裂成两个频率略有不同的超精细跃迁。该团队计算了限制体积中磁场存在下正常氢的跃迁频率,如下所示:

$f_{dd} =2 466 061 103 064 (2) kHz$

$f_{cc} =2 466 061 707 104 (2) kHz$

量子选择规则禁止S态之间的单光子跃迁,因此为了将基态正电子提升到2S能级,通过限制空间,并使其被调谐到计算跃迁频率一半的激光照射,以激发允许双光子吸收。

被激发到2S能态的反氢原子可以以几种方式之一进化:
\begin{itemize}
\item 可以通过发射两个光子,直接返回基态
\item 可以吸收另一个光子,使原子电离
\item 可以发射单个光子,并经由2P态返回基态——在这种情况下,正电子自旋可能翻转或保持不变。
\end{itemize}
电离和自旋翻转的结果都会导致原子脱离束缚。该研究团队计算出,假设反氢的行为与正常氢相似,与无激光的情况相比,大约一半的反氢原子会在共振频率暴露过程中丢失。当激光源调谐到低于一半跃迁频率200 kHz时,计算出的损耗与无激光情况下的损耗基本相同。

ALPHA研究团队制造了多批反氢原子,将其保持600秒,然后在1.5秒内逐渐缩小限制场,同时计算有多少反氢原子湮灭。该实验在三种不同的实验条件下进行:
\begin{itemize}
\item 共振:–将受限的反氢原子暴露在激光源下,在两次跃迁中,每次跃迁时间为300秒,激光源的调谐频率恰好为跃迁频率的一半。
\item 无共振:–将受限的反氢原子暴露在低于两个共振频率200 kHz的激光源下各300秒,
\item 无激光:–在没有任何激光照射的情况下限制反氢原子。
\end{itemize}
需要无共振和非激光这两种控制手段来确保激光照射本身不会导致湮灭,或者可以通过从限制容器表面释放正常原子,然后使这些原子与反氢结合来实现。

该团队对三个案例进行了11次测试,发现非共振和无激光测试之间没有显著差异,但是共振测试后检测到的事件数量下降了58\%。他们还能够在运行期间计算湮灭事件,并在共振运行期间发现更高的水平,同样在非共振运行和无激光运行之间没有显著差异。该结果与基于正常氢的预测非常一致,可以被“理解为为在200 ppt的精度下进行的CPT对称性测试”。[7]