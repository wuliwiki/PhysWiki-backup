% 四川大学 2009 年研究生入学物理考试试题
% keys 四川大学|2009年|考研|物理
% license Copy
% type Tutor

\textbf{声明}:“该内容来源于网络公开资料,不保证真实性,如有侵权请联系管理员”


科目代码:947
适用专业:光学、无线电物理、物理电子学
可能用到的物理常数:真空中的
\begin{enumerate}
\item  处于静电平衡的理想导体,导体内部电场强度为
。随曲率半径增大
导体外表面的电场强度
\item 利用万用表测量市电的交流电压,读数为 220V,是指交流电压的
值,
对应的峰值电压为V。
\item 相隔距离为 d 的等量同号点电荷+q 和+q,二者中点处的电势为
电场强度的大小为__V/m。
\item 两根无限长的均匀带电直线相互平行,相距为 2a,线电荷密度分别为+p和-p
则每单位长度上的带电直线受的库仑力为
N,两根直线相
互
\item 两块平行金属板间充满电容率为g =2,的均匀介质,当维持两块金属板上电
压V不变,每块平行金属板的电荷为 Q。1)如果将介质换为6,=2 的介质
则每块平行金属板的电荷为
。2)如果将介质8 移去,则每块平
行金属板的电荷为
\end{enumerate}