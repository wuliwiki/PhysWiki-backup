% 向量丛
% keys纤维丛|丛空间
\addTODO{需补充引用,GTM 275第7章.}
\addTODO{与“纤维丛”词条内容重复}
\pentry{纤维丛\upref{Fibre},流形\upref{Manif}}

向量丛是纤维丛的特例,即纤维都是向量空间的情况.

\begin{definition}{向量丛}
给定拓扑空间 $B$ 和线性空间 $V$,如果存在一个拓扑空间 $E$ 和一个连续满射 $\pi:E\rightarrow B$,使得对于任意的 $x\in B$,都有 $\pi^{-1}(x)\cong V$,那么称这个结构 $(E, V, B, \pi)$ 为一个\textbf{向量丛(vector bundle)}.
\end{definition}

向量丛之间也有丛映射:

\begin{definition}{丛映射}
给定向量丛 $(E, V_E, M, \pi_E)$ 和 $(F, V_F, N, \pi_F)$,其中 $M$ 和 $N$ 是实流形.我们定义一个“\textbf{光滑丛映射($C^\infty$ bundle map)}”为 $E\rightarrow F$ 的映射偶 $\varphi: E\rightarrow F$ 和 $\overline{\varphi}: M\rightarrow N$,使得:
\begin{equation}
\overline{\varphi}\circ\pi_E=\pi_F\circ\varphi
\end{equation}
且在任意 $p\in M$ 处,$\varphi|_p$\footnote{即只考虑 $p$ 处纤维的映射 $\varphi$.}是从 $p\times V_E$ 到 $\overline{\varphi}(p)\times V_F$ 的映射,并且是一个线性映射.
\end{definition}

在\textbf{纤维丛}\upref{Fibre}词条中我们强调过,一个向量丛 $(E, V, B, \phi)$ 不能简单等同于 $B\times V$,不过 $B\times V$ 本身也是一个纤维丛,称之为\textbf{平凡(trivial)}的纤维丛.




