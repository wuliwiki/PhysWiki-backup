% 南京航空航天大学 2005 量子真题
% license Usr
% type Note

\textbf{声明}:“该内容来源于网络公开资料,不保证真实性,如有侵权请联系管理员”

1. 一粒子在一维势 $
V(x)$ 中运动,\\试use求package确定{束ams缚math态能级的方程。
$$V(x) =}\\begin{document} \\1begin.{ 一cases粒}子 U_0, & x < -a \\\\在0一,维 &势 - $aV \\(x)$ 中运动,试求确定束缚态能le级q的 x < 0方程。

\\[ \\ V\\(x\\)in =fty 
\\begin{,cases &} x 
 \\Uge_q0, &  x0 < 
 -\\end{a \\\\
0,cases}
$$(30 & -a \\le分q)2.  x证明 <,在  $(0l \\_x\\^\\2in,fty l,_y &)$ x 的共同 \\本ge征q态 下0 $I _x\\ =end I{_ycases =} 0$,并求 $(\\Delta l_x\\)^]

2(\$30 和 分 $()

\\2. 证明,在 $(l_x^2, lDelta l_y)^2$。
(30分)

_y3^.2 一)$维 的运动共同粒本征态下 $I_x = I_y = 0$,并求子的状态是 
$$\\psi(x $()\\ =Delta I_x)^2 \\$begin 和{ $(cases\\}Delta A Ixe_y^{)^-\\2lambda$ x。(30}, 分 &) x3 \\geq. 一维 运动0粒 \\子的\\0状态是 \\[ \\psi, & x < 0 
(x\\)end ={ cases\\}begin{cases} A$$
其中 $\\lambda >xe^{-\\lambda x}, 0$,求 & x \\geq :(1) 粒子动0量 \\的\\
几0率,分 &布 x函数 <; (02 
)\\end{cases}
\\ 粒子的平均动量]

。
其中( $\\20lambda分 >)

 0$,4求.:

 粒(子在一维势 $U(x)1 =) \\ 粒begin子的{几率分布函数;

cases} 
0, &(2) 粒子的 x \\le平均q动 量0。

(20 \\\\
\\frac{1 分)

4.}{2} \\mu 粒子在一维势 \\omega^2 x^ $U(x) = 
2, & x > \\begin{cases} 
0 
\\end{cases0, & x \\}$
中运动。leq 0 \\\\
 (1) 不解方\\frac{1}{2程,写出粒子}\\mu\\omega^2能级与波函数的 x^2, &表达式 (设已 x > 0 
\\归一化),并说明end{cases}
$ 中理由;(2) 加运动。

(1) 不入微扰 $H解方程,写' = \\beta \\cos \\lambda x$,其中 $\\beta$ 为常数出粒子能级与波函数的表达式(设已归一化),并说明理由,$\\lambda \\ll;

(2) 加 1$,求能入微扰 $H'级至二级修正 = \\beta \\cos \\,波函数至一级修lambda x$,其中 $\\正。
(30分)
```beta$ 为常数,$\\lambda < 1$,求能级至二级修正,波函数至一级修正。

(30 分)

\\end{document}

