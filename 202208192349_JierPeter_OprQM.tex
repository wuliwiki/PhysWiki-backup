% 量子力学中的基本算符
% 动量算符|角动量算符|哈密顿算符|哈密顿量|能量算符|生成元
\addTODO{加入目录.}

\pentry{经典力学,量子力学的基本原理(量子力学)\upref{QMPrcp}}

本文中,$\hbar=1$.

在\textbf{量子力学的基本原理(量子力学)}的\autoref{QMPrcp_ex1}~\upref{QMPrcp} 和\autoref{QMPrcp_ex2} 中,我们不加证明地给出了动量、能量(哈密顿)、角动量算符在给定表象下的形式,相当于进行了定义.本文将讨论如何从经典力学中导出这几个算符的定义.

\subsection{无穷小算符}

考虑无穷小算符总是有益的,因为微分线性近似,而线性的东西很简单.

如果要求一个算符在时间趋于$0$的时候趋于恒等算符,即时间上的连续性,那么这个算符形如
\begin{equation}
1-
\end{equation}


























