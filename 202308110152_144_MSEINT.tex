% 材料科学 Intro
% keys 
% license CCBY4
% type Wiki

\pentry{金属的变形(科普)\upref{MetDfm},金属材料结构(科普)\upref{MetInt}}

\subsection{不仅是单一成分…}
按照经典的比喻,如果材料是一栋大楼,那么他的成分就相当于建造大楼的一块块砖头。因此,大部分介绍材料科学的书都是从成分这个古老而深刻的话题入手。

早在幼儿园二年级,懵懂无知的我们就开始好奇地打量生活中的各种物品、初探材料世界。也是从那时起,我们知道了那些基本材料的名称:金属、陶瓷、玻璃、塑料…

为何这几种材料就能组成千变万化的世界?到了小学二年级,我们觉察事情并不如此简单。所谓金属、塑料,其实只是一大类材料的总称,而并非单一一种物质。譬如说,金属可以继续细分为铁、铜、铝等等等等,而塑料则可以分为热塑型塑料与热固型塑料。

到了\textsl{人生知识的巅峰、中二病发的}中学二年级,我们了解到,即使是铁、铝这些名词,仍然不是故事的全部。我们使用的金属材料往往不是单一成分的纯净金属,而是包括了多种元素的合金。常说的“钢铁”就至少包括了铁与碳这两种元素;塑料的情况则\textsl{更糟糕},根据塑料中构成高分子链的小分子种类,我们开始苦恼于聚乙烯PE、聚丙烯PP、聚氯乙烯PVC、聚乳酸PLA…

到了大学二年级,如果我们\textsl{有志在干饭、电竞之余}继续探索材料世界,就一定会感慨“吾生也有涯,而知也无涯”,\textsl{所谓知道得越多就发现自己知道的越少}。在铁与碳之外,“钢铁”还往往包括其余数(十)种元素:例如,常听说的304不锈钢就具有大量的铬Cr、镍Ni等元素,这些元素有助于提升不锈钢的性能;同时,为了适应不同的应用环境,各种不同成分的钢也被广泛地开发、应用,因此“钢铁”仍然还是一大类材料的代称,304不锈钢也只是钢铁中的一种。别忘了,世界上不只有钢铁(铁合金),还有铝合金、铜合金在摩拳擦掌…

而塑料更为复杂,塑料中除了高分子链本身,还含有催化剂(用于在塑料生产过程中将小分子聚合为高分子)、塑化剂(相当于分子层面的润滑剂,提高塑料的可变性性、“塑性”)、色素(顾名思义)等其余大量成分…就连高分子链本身也变幻莫测,例如有些塑料的高分子链还由多种小分子聚合而来,这种情况下称为共聚物,比较经典的是ABS塑料。

\begin{table}[ht]
\centering
\caption{我们认识越来越细化的材料成分}\label{tab_MSEINT1}
\begin{tabular}{|c|c|}
\hline
金属 & 铁、铝… & 铁合金、铝合金... & 各种铁合金、各种铝合金…\\
\hline
塑料 & 热塑塑料、热固塑料 & PE、PP、PVC… & 高分子链+催化剂、塑化剂… \\
\hline
\end{tabular}
\end{table}

在\textsl{云里雾里了一通}后,我们甚至还没有开始涉及“各个成分之间的含量是什么”,光是“有什么成分”的问题就已经空前复杂。总而言之,我想说的是,材料的成分是非常复杂的问题,在生活中没有太多材料是“纯净物”。(或许钻石、石墨是个例外)
 
\subsection{也不只是一种结构}
成分说明了“材料中含有哪些物质”,而结构反映了“这些物质是如何组成材料的”。“结构”的内涵十分丰富,不同的细化学科会着重研究不同含义下的结构。我们姑且认为,结构包括原子间的排列(晶胞)与原子团之间的排列(晶粒、高分子链间的结合)等\upref{MetInt}。

首先,元素组成肯定是影响结构的关键因素。不同元素的金属原子将形成不同的晶胞\upref{MetInt} 结构。例如,室温下,铁Fe形成体心立方BCC晶胞,铜Cu形成面心立方FCC晶胞,而钛Ti形成最密六方HCP晶胞。\footnote{一些书本和网站会给出大多数元素与不少化合物的晶胞结构。}
\begin{figure}[ht]
\centering
\includegraphics[width=14 cm]{./figures/6e90a1e16bc27fbd.png}
\caption{从左往右分别是体心立方BCC、面心立方FCC与最密六方HCP} \label{fig_MSEINT_1}
\end{figure}
元素的含量比例也会影响结构。例如,在铁碳合金中,当碳元素含量较低时,碳可以进入铁BCC晶胞的间隙中,形成固溶体;而如果碳含量较高,铁的空隙已经不足以容纳所有的碳,那么C与Fe就倾向于形成结构复杂的Fe3C\upref{MetInt}。这是Fe-C的两种相态,铁碳合金往往是这两相的混合物。

目前的一切还算符合生活直觉,那请问“成分能唯一确定结构吗?”你或许会觉得这句话很有道理,但是考虑到\textsl{写文章半路抛出个似是而非的问题后必没有好事的套路},又迟疑了起来。答案是,或许有点震惊的,\textbf{不是}。就像使用相同的积木能够搭出不同的房子一样,使用相同的元素也可以构造出不同结构的材料。

最耳熟能详的例子是碳C,它有两种结构的固体:石墨与钻石。并且,据我们所知,石墨与钻石的性质相差非常大!
\autoref{fig_PHS_3}~\upref{PHS}

与此同时,成分还也不能完全确定原子团(晶粒)的结构。即使成分相同,晶粒的尺寸、形状等仍可以不同。譬如说,晶粒的尺寸可小可大、形态也可以各异。
\begin{figure}[ht]
\centering
\includegraphics[width=10cm]{./figures/7616163670ea26d3.pdf}
\caption{大晶粒与小晶粒示意图。\textsl{画的有点渣,见笑}} \label{fig_MSEINT_2}
\end{figure}

复相材料中相的分布方式更丰富多彩,可以是等轴的、层片状的或者网状的。网状结构意味着,(少量)第二相聚集分布在第一相的晶界边上。
\begin{figure}[ht]
\centering
\includegraphics[width=10 cm]{./figures/e0dd673cbc0e159d.png}
\caption{从左往右是等轴、层片、网状结构的示意图。$\alpha, \beta$代表两种相。} \label{fig_MSEINT_3}
\end{figure}

如果我们考察高分子的结构,就会遇到更多全新的术语与概念。为简明起见,我们就举一个简单的例子:支链。对于聚乙烯PE这种“简单“的高分子,我们预期他似乎应该是一根直链。然而,因为\textsl{一些我也快忘了的}高分子化学原因,在合成高分子链时,往往会产生一些支链,这使PE高分子链更像一条错综复杂的、带有各种支流的河流。\textsl{如果你找找身边的塑料袋,应该能发现不少聚乙烯塑料袋,他们的塑料编号是2和4。如今为了环保,也有不少塑料袋改用可降解的聚乳酸PLA。}
\begin{figure}[ht]
\centering
\includegraphics[width=8 cm]{./figures/38f8cf25a1de617c.pdf}
\caption{支链多的高分子链形成的塑料一般更疏松} \label{fig_MSEINT_4}
\end{figure}
\footnote{塑料中的高分子链往往是卷曲起来的,不会如此平直。}如果一个PE链的支链多,那么链之间就很难凑在一起,使得链的排列很松散,宏观上看密度低,称为低密度聚乙烯LDPE;而如果一个PE链的支链少,那么链的排列更为紧密,宏观上看密度高,称为高密度聚乙烯HDPE。这就有点像你收拾衣柜时,如果你一通乱塞,那么衣柜可能放不进几件衣服;而如果你叠好衣物(或者用真空泵压缩)再放进衣柜,那么衣柜就能装不少衣服了。

综上所述,成分确实能影响结构,但是成分并不能唯一确定结构。即使成分确定,材料的结构仍有很大的变化空间。
 
\subsection{不同的结构、不同的性能}

除了对知识的满腔热情,还有什么动力驱动着我们去分析材料的结构呢?答案是,有用。相同成分、不同结构的材料具有截然不同的性能。换而言之,结构会影响性能。

那么结构如何影响材料的性能?我们以晶界为例。我们已经知道了一个重要结论,金属的变形往往关乎位错的运动\upref{MetDfm};而此处我们再给出另一个重要结论:晶界就像一堵墙,能挡住位错的去路、阻碍位错的运动。

为什么晶界能一夫当关万夫莫开呢?说来话长啊,容我简述一二:一方面,晶界本身是一个结构相对无序的区域,原子无序堆叠导致内力场,而内力场将排斥位错的靠近;另一方面,晶界两边晶粒取向不同,位错想要通过晶界就得改变自身的滑移系,而这是相当困难的。

总结这两个结论,倘若一种材料有大量晶界,那么在这种材料中位错将更难运动,材料也就更难变形,或者说,材料的强度更高(需要更大的外力才能使材料变形)。那么什么样的材料具有大量晶界?或许你已经猜到,答案是那些具有更多、更小晶粒的材料:晶粒的数量多了、晶界就自然增加了。

捋顺一下思路:小晶粒材料中,晶界数量更多、位错运动更被阻碍、变形更难发生、材料强度也更高。此外,由于晶粒之间的相互联系,晶粒的变形被相互制约,这也有利于提高小晶粒材料的强度。这就是材料学术语“细晶强化”的含义及由来。

晶粒尺寸小、数量多->大量的晶界->阻碍位错运动->抑制金属变形->提高材料强度

但这还只是硬币的一面,另一面是,既然晶界堵住了位错的去路,位错自然得堆积在晶界面前;而我们知道,位错作为缺陷的一种,自身也要产生应力场;大量位错塞积产生的大应力场是十分危险的,将可能使材料断裂。在小晶粒材料中,这个问题还不算太棘手,因为位错被分散在大量晶粒中,单处的位错塞积不明显,总体而言位错的塞积甚至还被缓解,材料的韧性也被增强(在材料断裂前,材料的变形程度更大);然而,在网状结构的材料中(见上文),位错在晶界(相界)的塞积十分明显,这使材料很容易断裂。因此,我们往往不喜欢具有网状结构的材料。

或许你还记得上文提及的PE支链问题。我们知道,支链多的 PE,链之间排列很松散,形成LDPE;而支链少的 PE,链之间排列更紧密,形成HDPE。由于HDPE相对紧密的链排序,一般而言,HDPE的强度、韧性甚至熔点都高于LDPE。

\subsection{如何调整结构}

我们既然已经知道了结构会影响材料的性能,那么自然要关注如何控制材料的结构。然而,目前没有非常好的方法能精细地控制材料结构(相比之下,任何一个细胞都可以在原子尺度上精确控制物质的结构),但是可以介绍一些简单而粗糙的方法。

在铸造过程中(就是我们熟知的,把材料加热至融化,再在容器中冷却至室温,从而得到特定形状的材料的过程),调整晶胞大小的最简单的方法,莫过于改变冷却的速率。

为什么调整冷却速率就能影响最终的晶胞大小?这要先从材料的凝固原理讲起。假设我们知道,材料的凝固是一个形核-长大过程:液体内将先形成一堆小晶核,然后这些晶核将逐渐长大、直到材料完全凝固为固体。如果冷却速度足够快,那么有利于形成大量的晶核,同时这些晶核将没有时间充分长大,最终会得到大量小晶粒;相反,如果冷却速度比较慢,那么晶核将能充分长大,从而得到少量大晶粒。更具体的说明可以参考“经典形核理论\upref{NCLT}”与“晶核的长大\upref{GGRW}”。

可见,通过调整冷却速率,我们就可以控制晶胞的大小,从而影响材料的力学性能!工业上用“退火”、“正火”与“淬火”来形容不同的冷却速度。当然实际问题更为复杂,单纯的快速冷却并不是\textsl{灵丹妙药、包制百材料}\upref{GGRW},还得具体问题具体分析。

再比如说,如上节与上上节所述,我们一般不喜欢性能不良的网状结构,那么我们怎么消去网状结构呢?我们知道,相界也是一种缺陷,而网状组织具有大面积的相界,本身是热力学不稳定的,只不过由于室温固态环境下,原子的排列已经很紧密,也就很难重新排序为热力学上更稳定的结构。\textsl{就像在水里跑步比在空气来跑步费力多了一样!}为给原子\textsl{加油鼓气},我们选择再加热材料一段时间:在较高的温度之下,原子就能获得足够的能量来克服阻碍、重新排布为更稳定的结构。因此,网状结构将在高温下自行分解,材料的性能也得以改善。这种方法称为“回火“,比较典型的例子是回火处理高碳钢以消除网状渗碳体。

让我们再思考一下念念不忘的PE支链问题。如何控制PE的支链数量?在传统的催化合成方法中,由于催化剂本身的性质,支链的形成在所难免,很难合成支链足够少的PE,所以一般只能做出LDPE;要合成HDPE,就得运用“新”()的表面催化方法,这能大幅降低支链的形成量。

以上我们讨论了一些传统而经典的控制结构的方法,或许简单得让你有点失望。不过,这也是只加工与结构故事的冰山一角,更多的方法需要你更深入的学习。

% \subsection{无言的环境因素}

% 在本文的最后,让我们来讨论一下环境与材料性能的关系。在之前的论述中,我们都心照不宣地假定了材料工作在室温室压的环境。然而,并非所有的材料都能如此幸运。