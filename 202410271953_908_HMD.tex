% 哈密顿力学(综述)
% license CCBYSA3
% type Wiki

本文根据 CC-BY-SA 协议转载翻译自维基百科\href{https://en.wikipedia.org/wiki/Hamiltonian_mechanics}{相关文章})

\begin{figure}[ht]
\centering
\includegraphics[width=6cm]{./figures/7110c2a74929e25b.png}
\caption{威廉·罗恩·哈密顿爵士} \label{fig_HMD_1}
\end{figure}
在物理学中,哈密顿力学是拉格朗日力学的重新表述,起源于1833年。由威廉·罗恩·哈密顿爵士提出【1】,哈密顿力学用(广义)动量替代了拉格朗日力学中使用的(广义)速度 \( \dot{q}^i \)。这两种理论都提供了对经典力学的解释,并描述了相同的物理现象。

哈密顿力学与几何学(特别是辛几何和泊松结构)有密切关系,并且作为经典力学与量子力学之间的纽带。