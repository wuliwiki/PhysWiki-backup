% 量子力学导航

\begin{issues}
\issueTODO
\end{issues}
\subsection{量子力学的诞生背景}
学习一门新的理论,最重要的是了解其历史发展背景以及明确理论研究的动机(motivation)。量子力学于 19 世纪初诞生,是许许多多物理学家智慧的结晶,许多成果的提出离不开统计力学、原子物理等诸多领域的发展。学习量子力学的第一步是了解量子力学的诞生背景 \upref{QMborn}。

\subsection{量子力学的数学基础:线性代数}
严格来说, 量子力学的数学基础是泛函分析而不是线性代数。 因为线性代数只研究有限维向量空间中的问题, 而泛函分析将研究对象拓展到了无穷维空间。 量子力学中的波函数一般是无穷维希尔伯特空间中的向量。 但是, 对于绝大部分物理系学生来说, 泛函分析都不是必修内容, 在物理系的量子力学的教学中, 绝大部分也只是采用 “有限维空间类比到无穷维空间” 的模糊说法, 并通过一些实例来讲解离散本征值和连续本征值的区别。 考虑到这仍然是无法轻易改变的现状, 我们在初学量子力学时同样避免泛函分析。 在量子力学中泛函分析相对于线性代数的关系, 就好比数学分析之于微积分。 虽然泛函分析能让数学严格化, 但对物理意义的启发并不算大。

线性代数对理解量子力学至关重要, 在笔者看来脱离了线性代数和矢量空间来讲解量子力学, 就好比脱离了微积分讲解牛顿力学, 是不切实际的。 

要用线性代数理解量子力学, 矢量空间\upref{LSpace}(也叫线性空间)是基本概念。 事实上欧美国家的线性代数课程大部分都是首先讲解矢量空间而不是例如行列式的计算方法。 应用到量子力学中, 就是把(波)函数看成矢量空间中的矢量, 而且在其中定义了内积\upref{InerPd}, 存在正交和模长的概念, 是一个赋范空间\upref{NormV}, 实际存在的波函数需要归一化使范数为 1。 在量子力学中常用狄拉克符号\upref{braket}表示矢量。 另外需要明白为什么矩阵可以表示有限维矢量空间之间的线性映射\upref{LinMap}, 也就是\textbf{算符}。 厄米矩阵的本征问题\upref{HerEig} 是非常重要的概念, 因为量子力学中的测量量都有各自的厄米算符。 对易厄米矩阵与共同本征矢\upref{Commut} 也非常重要, 因为两个算符是否对易决定了它们是否可以同时测量, 即是否存在不确定性原理。

需要理解什么是子空间\upref{SubSpc}, 为什么矩阵的本征方程的解集构成子空间。 需要掌握如何把矢量投影\upref{projOp}到子空间。 因为量子力学中的测量就是一个投影。

要学习量子力学中单个粒子的自旋问题, 首先需要明白张量积空间\upref{DirPro}, 空间波函数和自旋态的相乘就是一个张量积, 粒子的总态矢存在于张量积空间中。 多粒子的量子力学同样需要使用张量积空间\upref{DirPro}, 多粒子的波函数就是处于这样一个矢量空间中。 全同粒子\upref{IdPar}假设使波函数只能存在于张量积空间中的对称或反对称子空间。
\addTODO{讨论连续本征值的正交归一化, 把傅里叶变换搬到这里。}

\subsection{量子力学的基本原理}

在量子力学建立以前, 玻尔原子模型\upref{BohrMd}成功解释了氢原子的能级。 这并不是真正的量子力学, 而是量子力学早期的一个半经典模型。 甚至可以说它算出的氢原子能级能符合实验是一个巧合。 它除了能给出正确的能级和玻尔半径外, 其他方面几乎没有正确之处。

要学习真正的量子力学, 首先我们通过一篇量子力学科普\upref{QM0} 简单概括量子力学的基本原理, 然后为了避免直接解薛定谔方程, 我们先看 “量子力学与矩阵\upref{QMmat}”。 然后通过 “算符和本征问题\upref{QM1}” 进一步明确量子力学的基本假设。 但这并不是所有的基本假设, 其中并没有讲解如何处理连续本征值和散射态, 没有涉及自旋和全同粒子\upref{IdPar}假设。

量子力学中有 3 种不同的维数要加以区分, 一是物理空间的维数, 也就是单粒子位置矢量的维数, 在初学时, 我们通常讨论一维情况, 也就是粒子延直线运动。 在讨论原子时,我们则需要三维的物理空间。 第二是波函数除时间外自变量的个数, 例如沿直线运动的两个不同粒子的波函数为 $\Psi(x_1,x_2,t)$, 那么我们就说这个波函数是 2 维的; 又例如三维物理空间中的两个粒子的波函数为 $\Psi(\bvec r_1, \bvec r_2, t)$, 那么这个波函数就是 6 维的。 最后是波函数或者态矢所在的希尔伯特空间中的维数,除了自旋角动量\upref{Spin}外, 其他态矢所处的希尔伯特空间几乎都是无穷维的。

\subsubsection{束缚态}
接下来学习定态薛定谔方程\upref{SchEq}, 该方程就是能量算符(哈密顿算符)的本征方程, 可以解出能量的本征态, 分为束缚态和散射态。 在讲解量子力学原理的过程中会伴随一些基础的势能例子, 这部分内容会有一些枯燥,因为主要是在解微分方程。 物理系的同学在学习的过程中暂时不需要对具体的解方程技巧有多熟练, 只需重点看能级结构和波函数等结论即可。 常见的一维势能如无限深势阱\upref{ISW}, 有限深方势阱\upref{FSW}, 方势垒的定态波函数\upref{SqrPot}, 简谐振子\upref{SHO}, 一维自由粒子(量子)\upref{FreeP1}以及高斯波包\upref{GausPk}。 多维势能如无限深圆形势阱\upref{CirISW}, 三维量子简谐振子(球坐标系)\upref{SHOSph}。

\subsubsection{散射态}
对于大部分势阱, 除了束缚态以外还会有散射态, 对于势垒则只有散射态。 我们可以把所有的束缚态和散射态共同看成一组正交归一的基底\upref{OrNrB}, 可用于展开任意波函数。 散射态波函数正交归一化的数学工具是狄拉克 delta 函数\upref{Delta}。 散射态最简单的粒子就是平面波\upref{PWave}(未完成:引用动量算符本征方程), 将不同平面波叠加出任意波函数的过程实际上就是傅里叶变换\upref{FTExp}。 从线性代数(泛函分析)的角度, 傅里叶变换相当于把无穷维希尔伯特空间中的函数展开到正交归一的平面波基底上\upref{COrNoB}。 事实上通过变换到傅里叶变换中的平面基底\upref{FTvec}, 我们可以把位置表象的波函数切换到动量表象\upref{moTDSE}。 我们可以认为: 位置表象或者动量表象的波函数并不是态矢本身, 而分别是态矢在两组不同的基底下展开后的系数。

\subsubsection{含时薛定谔方程}
详见含时薛定谔方程\upref{TDSE}。 由薛定谔方程就可以解出任意波函数在已知势能中的演化。 在知道波函数如何根据薛定谔方程演化后, 我们开始探索测量量: 要知道量子力学如何计算一个物理量的平均值\upref{QMavg} 以及什么是守恒量\upref{QMcons}。

接下来学习最简单的原子模型:类氢原子的定态波函数\upref{HWF}。 氢原子是唯一具有解析解的原子, 其重要性可见一斑。 氢原子一般需要在球坐标中求解, 这就涉及球坐标中用分离变量法解薛定谔方程\upref{RYTDSE}, 该方法典型的产物就是球谐函数\upref{SphHar}以及重要的径向薛定谔方程。

在普通的量子力学课程中, 电磁场通常是不进行量子化的, 他们仍然是电磁学中的连续场。 例如一个恒电场在薛定谔方程中体现为在势能项中加上 $-q\bvec E \vdot \bvec r$。 所以在经典的量子力学的方程中不会明确出现 “光子” 的概念。 光子作为一种没有质量的玻色子, 不能像电子一样直接用普通的薛定谔方程来描述, 而是需要\textbf{量子电动力学}。

\subsection{近似理论}
在量子力学中,存在解析解的问题极少, 所以需要使用许多近似方法, 如不含时微扰理论\upref{TIPT} 和含时微扰理论\upref{TDPT}。 以及 WKB 近似等。

\subsection{原子物理}
量子力学的一些最重要的应用就是研究原子分子以及凝聚态(固体,液体,等离子体等)的各种性质。 固体液体由大量原子分子组成, 而分子由若干原子组成, 所以其中原子是量子力学最基础的应用。

\subsubsection{散射}
在实验中, 单个原子看不见摸不着, 应该怎么研究呢? 一种常见的方法就是用一些粒子如电子和原子核去轰击要研究的原子, 然后在各个方向分析轰击后的产物, 进而通过这些产物的角向分布以及动能等观测量反推出原子的性质。 这个过程叫做散射\upref{ParWav}。 研究原子结构的早期著名实验之一就是卢瑟福散射\upref{RuthSc}。

为了学习处理散射的数学方法, 我们先学习一维势阱/势垒的散射, 就像\autoref{fig_QM0_2}~\upref{QM0} 那样。
\addTODO{串联相关词条并引用}
\addTODO{弹性散射和非弹性散射(后者更有趣且复杂)}

\subsubsection{原子与电磁场的相互作用}
我们高中所熟悉的光电效应就是\textbf{光电离}, 这个过程和散射很相似, 又被称为半散射, 因为入射的是电磁波, 而出射的是电离出的电子, 称为光电子。 光本质上是电磁波, 对于原子而言, 当某个波形的光入射到原子上, 电子会在电磁场的作用下运动并吸收能量, 如果这种能量足以使电子脱离原子核的束缚, 那么该电子就会被电离出来。 在实验中,通过研究光电子的能量谱和动量谱,以及角向分布等, 就可以反推出原子和光如何作用。
\addTODO{串联相关词条并引用}
\addTODO{新增数值模拟: 一维势阱中电子的单光子光电效应: $\omega\hbar = I_p + E_k$。 在光强足够大的情况下, 还会出现多光子的多个峰(ATI): $n\omega\hbar =  I_p + E_{k, n}$, 这将会非常有趣。这说明即使薛定谔方程中不出现光子的概念, 电子从场中吸收的能量依然约等于光子能量的整数倍。}
\addTODO{激发态, 能级跃迁, 不含时微扰, 含时微扰理论\upref{TDPT}, 氢原子的选择定则\upref{SelRul}, 受激吸收, 受激辐射, 自发辐射, Stark 效应(抛物线坐标系)\upref{HStrk2}, 隧道电离\upref{Htunnl}。}
