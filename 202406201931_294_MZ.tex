% μ子
% license CCBYSA3
% type Wiki


(本文根据 CC-BY-SA 协议转载自原搜狗科学百科对英文维基百科的翻译)
\subsection{μ子}

$\mu$子(/ˈmjuːɒn/;来自用于表示它的希腊字母 mu($\mu$))是一种类似于电子的基本粒子,电荷量为−1e,自旋为 1/2,但是$\mu$子拥有比电子更大的质量。$\mu$子被归类为轻子。与其他轻子一样,子不被认为有任何亚结构——也就是说,它不被认为由任何更简单的粒子组成。

$\mu$子是一种不稳定的亚原子粒子,其平均寿命为2.2 $\mu$s,这比许多其他亚原子粒子都要更长。如同非基本粒子的中子的衰变一样(中子寿命约为15分钟),$\mu$子的衰变(按亚原子标准)发生得很慢,这是因为$\mu$子的衰变是完全由弱相互作用引起的(而不是更强大的强相互作用者电磁相互作用);且因为$\mu$子和它的衰变产物间的质量差很小,所以它的衰变只有很少的动力学自由度。$\mu$子的衰变几乎总是产生至少三个粒子,其中必须定括一个与$\mu$子拥有相同电荷的电子以及两个不同类型的中微子。

像所有基本粒子一样,$\mu$子也有相应的反粒子,后者拥有与前者相反的电荷(+1e),但两者的质量和自旋则是相同的,这个反粒子就是反$\mu$子(也称为正$\mu$子)。$\mu$子被表示为$\mu$−,反$\mu$子则是$\mu$+。$\mu$子以前被称为$\mu$介子,但现代粒子物理学家不再将其归类为介子,因此物理界也就不再使用$\mu$介子这个名称了。

$\mu$子的质量为105.66 MeV/c2,这大约是电子的207倍。由于$\mu$子的质量更大,当$\mu$子遇到电磁场时,其加速不会那么快,也不会发射那么多轫致辐射(减速辐射)。这使得拥有给定能量的$\mu$子在物质中可以比电子穿透得更深,因为电子和$\mu$子的减速主要是由于轫致辐射机制所导致的能量损失。例如,由宇宙射线撞击大气层所产生的所谓“次级$\mu$子”可以穿透到地表,甚至深入矿井。

因为$\mu$子的质量和能量比起放射性过程的衰变能量来说非常大,所以$\mu$子从来都不会由放射性衰变产生。然而,它们可以大量地产生于正常物质中的高能相互作用、涉及到强子的某些粒子加速器实验或者宇宙线与物质间自然发生的相互作用中。这些相互作用通常先产生$\pi$介子,$\pi$介子通常会衰变成$\mu$子。

与其他带电轻子一样,$\mu$子有一种相对应的$\mu$子中微子,后者被表示为$v_\mu$。$\mu$子中微子不同于电子中微子,它们不参与相同的核反应介子,$\pi$介子通常会衰变成$\mu$子。

与其他带电轻子一样,$\mu$子有一种相对应的$\mu$子中微子,后者被表示为$v_\mu$。$\mu$子中微子不同于电子中微子,它们不参与相同的核反应

\subsection{历史}
$\mu$子是由卡尔·安德森和Seth Neddermeyer于1936年在加利福尼亚理工学院研究宇宙辐射时发现的。安德森注意到一些粒子穿过磁场时所产生的弯曲路径与电子和其他已知粒子都不相同。它们带负电,在速度相同的情况下,它们在磁场中的路径的弯曲程度不如电子,同时又比质子要更加弯曲。$\mu$子所带的负电荷量被默认与电子相同,因次它们路径曲率的差异就意味着$\mu$子的质量大于电子但小于质子。安德森最初称这种新粒子为mesotron,采用了希腊词前缀meso-,意思就是“中间”。$\mu$子的存在于1937年被J. C. Street和E. C. Stevenson的威尔逊云雾室实验所证实。[1]

理论物理学家汤川秀树在发现介子之前就已经预测了质量在介子范围内的粒子:[2]

用以下方式修改海森堡和费米理论似乎很自然。重粒子从中子态状态到质子状态的转变并不总是伴随着轻粒子的发射。这种转变有时会被另一个重粒子占据。

由于其质量,$\mu$子最初被认为是汤川秀树所预测的粒子,但后来被证明拥有错误的性质。汤川预测的粒子(即$\pi$介子)最终在1947年被确定(再次来自宇宙射线相互作用),并被证明不同于早期发现的$\mu$子,$\pi$介子具有介导核力的正确性质。

已知两种具有中间质量的粒子以后,更一般的术语“介子”被用来指代那些质量位于电子和核子之间的粒子。此外,为了在发现第二种介子后区分两种不同类型的介子,先被发现的那种介子被重命名为$\mu$介子(希腊字母$\mu$(mu),对应于m),1947年才被发现的那种介子(汤川粒子)则被命名为$\pi$介子。

随着后来更多类型的介子在加速器实验中被发现,物理学家最终发现$\mu$介子不仅与(质量大约相同的)$\pi$介子之间存在显著差异,而且与所有其他类型的介子也非常不同。部分区别在于$\mu$介子不像$\pi$介子那样参与核力相互作用(这是汤川理论所要求的)。新发现的介子在核相互作用中表现得像$\pi$介子,但不像$\mu$介子。此外,$\mu$介子的衰变产物包括一个中微子和一个反中微子,而不是像在其他带电介子的衰变中观察到的那样只有一个中微子或一个反中微子。

在成形于1970年代的粒子物理学标准模型中,除了$\mu$介子之外的所有其他介子都被理解为强子。也就是说,这些粒子都是由夸克组成的,因此都受制于核力。在夸克模型中,介子不再由质量来定义(因为已经发现存在质量非常大的介子,这些介子的质量比核子还要大),介子被认为是由正好两个夸克(夸克和反夸克)组成的粒子,这一点与重子不同,后者是由三个夸克组成的(质子和中子是最轻的重子)。然而,$\mu$介子已经被证明是像电子一样的基本粒子(轻子),而且没有夸克结构。因此,对于基于粒子结构夸克模型的“介子”这个术语的新意义和新用法来说,$\mu$介子根本不是介子。

随着定义的改变,$\mu$介子这个术语已经被抛弃了,现在物理学家都尽可能用现代术语$\mu$子来代替$\mu$介子,$\mu$介子这个名称仅剩下了历史意义。在新的夸克模型中,其他类型的介子有时会被继续用较短的术语来表示(例如,$\pi$介子简称为pion),但$\mu$子保留了较短的名称muon,旧术语“$\mu$介子”再也没有被提及了。

最终承认“$\mu$介子”$\mu$子是一个简单的“重电子”并且在核相互作用中没有任何作用在当时似乎是如此的不协调和令人惊讶,以至于诺贝尔奖获得者拉比有这么一句名言:“谁点的?”[3]

在1941年的罗西-霍尔实验中,$\mu$子首次被用来观察狭义相对论所预测的时间膨胀(或者长度收缩)。
\subsection{μ子源}
到达地表的$\mu$子是宇宙射线与地球大气层粒子碰撞的衰变产物。[4]

在地球表面每分钟每平方厘米大约有10,000个$\mu$子到达。这些带电粒子是宇宙射线撞击大气层上层中的分子的副产物。以相对论性速度运动的$\mu$子在被其他原子吸收或偏移而衰减之前能够穿透几十米厚的岩石和其他物质。[5]

当宇宙射线质子撞击高层大气中的原子核时会产生π介子。这些π介子在相对较短的距离(若干米)内便衰变为$\mu$子(它们所偏好的衰变产物)和$\mu$子中微子。来自这些高能宇宙射线的$\mu$子通常以接近光速的速度沿着与初始质子大约相同的方向继续前进。虽然在没有相对论效应的情况下这些$\mu$子的寿命所允许的半存活距离最多只有456米($2.197\mu s * \ln(2) * 0.9997 * c$),但狭义相对论的时间膨胀效应(从地球参照系的观点)允许宇宙射线次级$\mu$子在飞往地表的飞行中存活,因为在地球参照系中$\mu$子由于其速度而具有更长的半衰期。从$\mu$子的角度来看(惯性坐标系),则是狭义相对论的长度收缩效应使得这种穿透能够实现,因为在$\mu$子的生命周期不受影响,但长度收缩导致通过大气层和地球的距离远短于地球静止参考系中所测量到的距离。这两种效应都是解释快速$\mu$子不寻常的长距离存在的有效方法。

由于$\mu$子能够不同寻常地穿透普通物质,像中微子一样,它们也可以在地下深处(地下700米处的Soudan 2探测器)和水下检测到,它们形成自然背景电离辐射的一个主要部分。如前所述,像宇宙射线一样,这种次级$\mu$子辐射也是定向的。

粒子物理学家使用与上述相同的核反应(即强子-强子碰撞产生π介子束,然后π介子束在短距离内快速衰变为$\mu$子束)来产生$\mu$子束,例如用于$\mu$子g-2实验的$\mu$子束。[6]

\subsection{μ子衰变}
$\mu$子是不稳定的基本粒子,比电子和中微子重,但比所有其他物质粒子轻。它们通过弱相互作用发生衰变。因为轻子族数在没有极不可能立即出现的中微子振荡的情况下是守恒的,$\mu$子衰变的产物之一必须是$\mu$子型中微子,另一个必须是电子型反中微子(反$\mu$子衰变则产生相应的反粒子,如下所述)。因为电荷必须守恒,$\mu$子衰变的产物之一总是与$\mu$子具有相同电荷的电子(如果是正$\mu$子,产物则是正电子)。因此,所有$\mu$子至少衰变为一个电子和两个中微子。有时,除了这些必要的产物之外,还会产生其他没有净电荷和自旋为零的粒子(例如,一对光子或电子-正电子对)。

主导的$\mu$子衰变模式(有时被以路易斯·米歇尔之名命名为米歇尔衰变)是最简单的可能:$\mu$子衰变为电子、电子型反中微子和$\mu$子中微子。反$\mu$子以镜像方式最常衰变成相应的反粒子:正电子、电子中微子和$\mu$子型反中微子。用公式化的术语来说,这两种衰变就是:
\begin{equation}
\mu^- \to e^-+v_e+v_\mu~
\end{equation}\begin{equation}
\mu^+ \to e^+ +v_e+v_\mu~
\end{equation}
正负$\mu$子的平均寿命(τ=ħ/Γ)是(2.1969811±0.0000022) $\mu$s。[7]$\mu$子和反$\mu$子寿命的相等已经被确定到优于1/104的精度。
\begin{figure}[ht]
\centering
\includegraphics[width=8cm]{./figures/9f0c24095708e039.png}
\caption{μ子最常见的衰变} \label{fig_MZ_1}
\end{figure}
\subsubsection{3.1 被禁止的衰变}
某些无中微子的衰变模式在运动学上是允许的,但出于所有实际目的,在标准模型中是被禁止的,甚至在中微子具有质量和存在振荡的情况下也是如此。轻子味守恒所禁止的例子有:
\begin{equation}
\mu^- \to e^- +
\end{equation}