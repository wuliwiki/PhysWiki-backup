% 集合论(综述)
% license CCBYSA3
% type Wiki

本文根据 CC-BY-SA 协议转载翻译自维基百科\href{https://en.wikipedia.org/wiki/Set_theory}{相关文章}。

\begin{figure}[ht]
\centering
\includegraphics[width=6cm]{./figures/590e7a061f3044c8.png}
\caption{一个维恩图,展示了两个集合的交集。} \label{fig_Set1_1}
\end{figure}
\textbf{集合论}是数学逻辑的一个分支,研究集合,集合可以非正式地描述为对象的集合。尽管任何类型的对象都可以组成一个集合,但集合论——作为数学的一个分支——主要关注那些与整个数学相关的集合。

现代集合论的研究始于19世纪70年代,由德国数学家理查德·德德金德和乔治·康托尔发起。特别是,乔治·康托尔通常被认为是集合论的创始人。在这个早期阶段研究的非形式化系统被称为朴素集合论。在朴素集合论中发现悖论(如罗素悖论、康托尔悖论和布拉利-福尔蒂悖论)之后,20世纪初提出了各种公理化系统,其中泽梅洛–弗兰克尔集合论(无论是否包含选择公理)仍然是最著名和最研究的。

集合论通常被用作整个数学的基础系统,特别是以泽梅洛–弗兰克尔集合论与选择公理的形式。除了其基础性作用外,集合论还提供了一个框架,用于发展数学中的无穷大理论,并在计算机科学(如关系代数理论)、哲学、形式语义学和进化动力学等领域有着广泛的应用。它的基础性吸引力、与悖论的关系、以及对无穷大的概念及其多重应用的影响,使得集合论成为逻辑学家和数学哲学家关注的主要领域之一。当代集合论的研究涵盖了广泛的主题,从实数线的结构到大基数的一致性研究。
\subsection{历史}  
\subsubsection{早期历史}
\begin{figure}[ht]
\centering
\includegraphics[width=6cm]{./figures/d1970bff40e0e3fa.png}
\caption{普尔科修斯(Purchotius)于1730年所绘的波尔斐里树,展示了亚里士多德的《范畴》。} \label{fig_Set1_2}
\end{figure}
基本的对象分组概念自至少在数字的出现以来就已存在,而将集合视为自身对象的概念至少自公元3世纪的《波尔斐里树》以来就存在。集合的简单性和普遍性使得很难确定现在在数学中使用的集合的起源,然而,伯纳德·博尔扎诺的《无穷悖论》(*Paradoxien des Unendlichen*,1851年)通常被认为是集合首次严格引入数学的工作。在他的著作中,他(除了其他内容外)扩展了伽利略的悖论,并引入了无限集合的一对一对应,例如通过关系$5y = 12x$,在区间$[0, 5]$和$[0, 12]$之间建立了对应。然而,他拒绝称这些集合是等势的,他的工作通常被认为在他那个时代的数学中没有产生影响。\(^\text{[1][2]}\)

在数学集合论出现之前,无穷的基本概念被认为完全属于哲学领域(参见:无穷(哲学)和无穷 § 历史)。自公元前5世纪起,从西方的希腊哲学家芝诺开始(以及东方的早期印度数学家),数学家们就一直在与无穷的概念作斗争。随着17世纪末微积分的发展,哲学家们开始普遍区分实际无穷与潜在无穷,其中数学仅涉及后者。\(^\text{[3]}\) 卡尔·弗里德里希·高斯 famously 说道:“无穷不过是一个修辞,帮助我们讨论极限。完成的无穷这一概念不属于数学。”\(^\text{[4]}\)

数学集合论的发展受到了几位数学家的启发。伯恩哈德·黎曼在《几何基础上的假设》(1854年)的讲座中提出了关于拓扑学的新思想,并关于将数学(特别是几何学)建立在集合或流形的基础上,以类的方式(他称之为Mannigfaltigkeit),这现在被称为点集拓扑学。该讲座于1868年由理查德·德德金德出版,同时也出版了黎曼关于三角级数的论文(该论文提出了黎曼积分),后者成为实分析领域中研究“严肃的”不连续函数的起点。年轻的乔治·康托尔进入了这一领域,这引导他研究点集。大约在1871年,受黎曼的影响,德德金德开始在他的出版物中使用集合,这些出版物非常清晰和精确地处理了等价关系、集合的划分和同态。由此,20世纪数学中许多常见的集合论程序可以追溯到他的工作。然而,他直到1888年才出版了关于集合论的正式解释。
\subsubsection{朴素集合论}
\begin{figure}[ht]
\centering
\includegraphics[width=6cm]{./figures/6209bd2e6720530a.png}
\caption{乔治·康托尔,1894年} \label{fig_Set1_3}
\end{figure}
集合论,现代数学家所理解的集合论,通常被认为是由乔治·康托尔于1874年发表的论文《关于所有实代数数集合的一个性质》奠定的基础。\(^\text{[5][6][7]}\)在这篇论文中,他发展了基数的概念,通过一一对应来比较两个集合的大小。他的“革命性发现”是,所有实数的集合是不可数的,也就是说,不能将所有实数列在一张列表中。这个定理是通过康托尔的第一次不可数性证明来证明的,这与更为人熟知的使用对角线法的证明有所不同。

康托尔引入了集合论中的基本构造,如集合$A$的幂集,它是$A$的所有可能子集的集合。他后来证明,幂集$A$的大小严格大于$A$的大小,即使$A$是一个无限集合;这一结果很快被称为康托尔定理。康托尔发展了一种超限数的理论,称为基数和序数,扩展了自然数的算术。他为基数使用的符号是希伯来字母$\aleph$(ℵ,aleph)带有自然数下标;对于序数,他使用希腊字母$\omega$(ω,omega)。

集合论开始成为新“现代”数学方法的一个重要组成部分。最初,康托尔的超限数理论被认为是违反直觉的——甚至是令人震惊的。这导致它遭遇了数学当代人物如利奥波德·克罗内克和亨利·庞加莱的反对,后来也遭遇了赫尔曼·韦尔和L·E·J·布劳威尔的抵制,而路德维希·维特根斯坦提出了哲学上的异议(参见:康托尔理论的争议)。\(^\text{[a]}\)德德金德的代数风格直到1890年代才开始找到追随者。
\begin{figure}[ht]
\centering
\includegraphics[width=6cm]{./figures/81a7f93f8b63fcd5.png}
\caption{戈特洛布·弗雷格,大约1879年} \label{fig_Set1_4}
\end{figure}
尽管存在争议,康托尔的集合论在20世纪初得到了显著的发展,得益于几位著名数学家和哲学家的贡献。理查德·德德金德在同一时期开始在他的出版物中使用集合,并以著名的德德金德切割法构造实数。他还与朱塞佩·皮亚诺合作,发展了皮亚诺公理,这些公理使用集合论的思想形式化了自然数算术,并引入了用于集合成员关系的epsilon符号。可能最为重要的是,戈特洛布·弗雷格开始发展他的《算术基础》。

在他的著作中,弗雷格试图通过逻辑公理来构建所有数学,使用康托尔的基数概念。例如,“马棚里有四匹马”这句话意味着四个对象属于“马”这一概念。弗雷格尝试通过基数(“...的数量”,或$Nx: Fx$)来解释我们对数字的理解,依赖于休谟原则。
\begin{figure}[ht]
\centering
\includegraphics[width=6cm]{./figures/25b899959d5b0a2d.png}
\caption{伯特兰·罗素,1936年} \label{fig_Set1_5}
\end{figure}
然而,弗雷格的工作很短命,因为伯特兰·罗素发现他的公理导致了一个矛盾。具体而言,弗雷格的基本法则V(现在被称为无限制理解公理模式)。根据基本法则V,对于任何足够明确的属性,都存在一个集合,包含所有且仅有具备该属性的对象。矛盾,称为罗素悖论,证明过程如下:

令$R$为所有不包含自身的集合的集合(这个集合有时被称为“罗素集合”)。如果$R$不是它自己的成员,那么它的定义意味着它是它自己的成员;然而,如果它是它自己的成员,那么它就不是它自己的成员,因为它是所有不包含自身的集合的集合。由此产生的矛盾就是罗素悖论。用符号表示:

令  
$R = \{x \mid x \notin x\}$,那么$R \in R \iff R \notin R$  

这出现在多个悖论或违反直觉的结果发生的时期。例如,平行公理无法被证明,存在无法计算或明确描述的数学对象,存在无法通过皮亚诺算术证明的算术定理。其结果是数学的基础危机。

