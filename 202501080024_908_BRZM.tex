% 路德维希·玻尔兹曼(综述)
% license CCBYSA3
% type Wiki

本文根据 CC-BY-SA 协议转载翻译自维基百科\href{https://en.wikipedia.org/wiki/Ludwig_Boltzmann}{相关文章}。

\begin{figure}[ht]
\centering
\includegraphics[width=6cm]{./figures/ed66f21add0c5482.png}
\caption{1902年的玻尔兹曼} \label{fig_BRZM_1}
\end{figure}
\textbf{路德维希·爱德华·玻尔兹曼}(Ludwig Eduard Boltzmann,1844年2月20日-1906年9月5日)是奥地利的物理学家和哲学家。他的最大成就包括统计力学的发展和热力学第二定律的统计解释。1877年,他提出了当前的熵定义:\(S = k_{\rm B}\ln\Omega \)其中,Ω是系统能量等于宏观系统能量的微观状态数,解释为衡量系统统计无序度的一个指标。马克斯·普朗克将常数 \( k_B \) 命名为玻尔兹曼常数。

统计力学是现代物理学的基石之一。它描述了宏观观测(如温度和压力)如何与围绕平均值波动的微观参数相关。它将热力学量(如比热容)与微观行为联系起来,而在经典热力学中,唯一可用的方式是为不同材料测量并列出这些量。