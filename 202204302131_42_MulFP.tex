% 多元函数泛函的极值
% 多元函数|泛函极值|Euler-Ostrogradsky方程

一元函数的泛函极值问题已经研究过,现在研究 $n$ 元函数的泛函极值问题,并推导 $n$ 元函数泛函极值问题中的Euler-Ostrogradsky方程.“一元”在于泛函的宗量(泛函的“自变量”)为一元函数(即可取曲线\upref{DesCur}),“ $n$ 元” 在于泛函的宗量为 $n$ 元函数(这时也可称为可取曲线,只不过变元为 $n$ 个).

在 $n$ 元函数的情形,我们仍然限定泛函 $J[\varphi]$ 的宗量 $\varphi(x_1,\cdots,x_n)$ 属于 $C_1$ 类函数,即在它的定义域上连续,并有关于所有变元 $x_i$ 的连续偏导数 $\partial_i\varphi=\pdv{\varphi}{x_i}=\varphi'_{x_i}$.

具体来说,我们的问题可归结为:在所有定义在 $n$ 维空间中有界区域 $Q$ 上的 $C_1$ 类函数中,以 $\overline{C_1}$ 表示那些在 $Q$ 的边界 $\Omega$ 上取给定值的属于 $C_1$ 类函数的全体,即若 $\varphi\in\overline{C_1}$,则 $\varphi(\Omega)=f(\Omega)$ 而 $f(\Omega)$ 是给定的.在 $\overline{C_1}$ 中找出一个函数 $\varphi$,使它给出泛函
\begin{equation}
J[\varphi]=\int\cdots\int\limits_Q F(x_i,\varphi,\partial_i\varphi)\dd x_1\cdots\dd x_n
\end{equation}
的极值.其中,$\partial_i\varphi$ 是 $\{\partial_i\varphi|i=1\cdots,n\}$ 的简写,$F$ 是其 $2n+1$ 个变元 $x_i,\varphi,\partial_i\varphi$ 的连续函数,并且对于一切变元有三阶以内的偏微商.

若函数 $\varphi$ 给出泛函 $J[\varphi]$ 的极值,则函数 $\varphi$ 在区域 $Q$ 上满足偏微分方程
\begin{equation}
F_\varphi-\sum_{i=1}^n\partial_i F'_{\varphi_i}=0
\end{equation}
这称为\textbf{Euler-Ostrogradsky方程}.

先给出一些必要的概念.
\begin{definition}{$C_1$类函数的距离及邻区}
若 $\varphi,\phi\in C_1$ ,则
\begin{equation}
\begin{aligned}
r(\varphi,\phi)=&\mathrm{max}\{\abs{\varphi(x_1,\cdots,x_n)-\phi(x_1,\cdots,x_n)},\\
&\abs{\partial_i\varphi(x_1,\cdots,x_n)-\partial_i\phi(x_1,\cdots,x_n)}\}\quad (i=1,\cdots,n)
\end{aligned}
\end{equation}
为 $C_1$ 类函数 $\varphi,\phi$ 的\textbf{距离}.满足 $r(\varphi,\phi)<\epsilon$ 的那些函数 $\phi$ 的全体构成 $\varphi$ 的 $\epsilon$ \textbf{邻区}. 
\end{definition}

