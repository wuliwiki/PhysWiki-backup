% 函数的极限(极简微积分)
% license Xiao
% type Tutor

\pentry{数列的极限(极简微积分)\nref{nod_Lim0}, 充分必要条件\nref{nod_SufCnd}, 函数(高中)\nref{nod_functi}}{nod_c4ea}

\subsection{引入}
我们先通过简单的例子初步了解函数的极限。
\begin{example}{}\label{ex_FunLim_1}

\begin{figure}[ht]
\centering
\includegraphics[width=10cm]{./figures/65fc12a7dfd67e6c.png}
\caption{$f(x)=\frac{\sin(x)}{x}$的图像} \label{fig_FunLim_1}
\end{figure}
思考一下$f(x)=\frac{\sin(x)}{x}$这一经典函数在原点附近的值。

众所周知,当$x=0$时,由于分母为$0$,该分数没有意义;但当$x$ \textbf{趋近于} $0$, 如分别令 $x=0.1,x=0.01,...$ 而不等于 $0$ 时,有趣的事情发生了: 如\autoref{fig_FunLim_1} 和\autoref{tab_FunLim_1} 所示,此时分数的值似乎趋于一个确定的值$1$.

\begin{table}[ht]
\centering
\caption{x与f(x)}\label{tab_FunLim_1}
\begin{tabular}{|c|c|c|c|}
\hline
$x$ & $0.1$ & $0.01$ & $0.001$ \\
\hline
$f(x)$ & $0.9983$ & $0.99998$ & $0.9999998$ \\
\hline
\end{tabular}
\end{table}
看起来,尽管我们不能定义$f(x)$在零点处的值,但是我们知道,当$x$趋近于$0$时,$f(x)$趋近于$1$. 因此,我们说$\lim\limits_{x\to0}\frac{\sin(x)}{x}=1$
\end{example}

\begin{example}{}
\begin{figure}[ht]
\centering
\includegraphics[width=10cm]{./figures/089eecead76d5e6b.png}
\caption{$f(x)=1/x$的图像 (x>0)} \label{fig_FunLim_2}
\end{figure}
然后,我们再看看$f(x)=1/x$另一经典函数的图像。我们还是知道,两个正数相除始终大于零;但当$x$\textbf{足够大}时,$1/x$会\textbf{足够小}以至趋于$0$. 因此,我们说$\lim\limits_{x\to+\infty}1/x=0$。
\end{example}

\subsection{自变量趋于无穷的极限}
实函数 $f(x)$ 可以看成是一种 “连续” 的数列, 只不过把元素编号从离散的 $n$ 改为连续的 $x$。 类比数列的极限, 我们也可以定义\textbf{函数趋于正无穷的极限}。

\begin{definition}{函数趋于正无穷的极限}\label{def_FunLim_1}
考虑实函数 $f(x)$。 若无论要求 $f(x)$ 和一确定实数 $A$ 的距离 $\epsilon$ 有多小(但 $\epsilon>0$), 都存在实数 $X$ ,使得所有 $x>X$ 都满足 $\abs{f(x)-A}<\epsilon$, 那么我们说 $A$ 是函数 $f(x)$ 在 $x$ 趋于正无穷时的极限, 记为
\begin{equation}
\lim\limits_{x\to +\infty} f(x) = A~.
\end{equation}
\end{definition}

可以看到该定义和数列极限的定义(\autoref{def_Lim0_2})非常相似, 只是简单做了替换。不过,函数并不是简单地把数列的概念拓展到连续的情况。 数列的编号只能朝着一个方向增大, 但函数的自变量 $x$ 既可以趋近正无穷也可以奔向负无穷, 

%\addTODO{画图, 画出函数曲线, 距离要求就是两条直线之间的范围, 等等}
\begin{figure}[ht]
\centering
\includegraphics[width=12cm]{./figures/0e045b4dd308915e.pdf}
\caption{对于任意一个$\epsilon$,都存在对应的$X$。仿自\cite{Thomas}} \label{fig_FunLim_5}
\end{figure}

\begin{exercise}{}
请仿照\autoref{def_FunLim_1} 给出函数趋于负无穷时极限的定义
\begin{equation}
\lim\limits_{x\to -\infty} f(x) = A~.
\end{equation}
\end{exercise}

注意 $\lim\limits_{x\to\infty} f(x) = A$ 仅表示正无穷的极限而不是两个方向的极限都是 $A$。

\subsection{自变量趋于一点的的极限}
另外, 由于 $x$ 是连续取值的, 也可以考察自变量 $x$ 不断趋近某一点 $x_0$ 的极限, 即 $x\to x_0$。如何描述 “自变量趋于一个给定的实数 $x_0$” 呢? 只需要取自变量 $x$ 使得二者间的距离 $\abs{x-x_0}$ 越来越接近 $0$ 即可。

\begin{definition}{函数在某点的极限}\label{def_FunLim_3}
考虑实函数 $f(x)$。 若无论要求 $f(x)$ 和确定实数 $A$ 的距离 $\epsilon>0$ 有多小, 都存在一个自变量的取值半径 $\delta>0$,使得对任意满足 $\abs{x-x_0} < \delta$的实数$x$,都有 $\abs{f(x)-A}<\epsilon$,
% 能通过不等式 $\abs{x-x_0} < \delta$ ($\delta$ 是一确定实数)使要求成立, 
那么我们说 $A$ 是函数 $f(x)$ 在 $x$ 趋于 $x_0$ 时的极限, 记为
\begin{equation}
\lim\limits_{x\to x_0}f(x)=A~.
\end{equation}
\end{definition}

\begin{figure}[ht]
\centering
\includegraphics[width=12cm]{./figures/63de5f309b49cf12.pdf}
\caption{对于任意一个$\epsilon$,都存在对应的$\delta$.仿自\cite{Thomas}} \label{fig_FunLim_8}
\end{figure}

\begin{example}{简单技巧}
求一些简单的函数在某个值处的极限时, 通常可以直接代入数值计算(如果存在的话), 如
\begin{equation}
\lim_{x\to 1} 2x + 1 = 3 ~,\qquad \lim_{x\to 2}\frac{x + 1}{x + 2} = \frac34~.
\end{equation}

当无穷大与常数相加时, 可以忽略常数, 如
\begin{equation}
\lim_{x\to +\infty} \frac{x + 1}{2x + 2} = \lim_{x\to +\infty} \frac{x}{2x} = \frac12~.
\end{equation}

我们可以换个角度计算上面的极限,
\begin{equation}
\lim_{x\to +\infty} \frac{x + 1}{2x + 2} = \lim_{x\to +\infty} \frac{1 + 1/x}{2 + 2/x} = \frac12~.
\end{equation}
$1/x$ 和 $2/x$ 是“无穷小”,因此可以忽略。

z

如果你想要处理更复杂的极限问题,那你可以参考求极限的一些方法\upref{ChaLim}。不过,其中部分方法已经超出了初学者的(以及 “极简微积分” 的)知识范围。
\end{example}

\subsubsection{左、右极限}
我们还可以区分函数在某点的\textbf{左极限(left limit)}和\textbf{右极限(right limit)}。 简而言之就是 $x$ 分别从左边和右边两个方向趋近 $x_0$ 时的极限, 具体定义留做思考。 左右极限记为
\begin{equation}
\lim_{x\to x_0^-} f(x) = A_- ~,\qquad \lim_{x\to x_0^+} f(x) = A_+~.
\end{equation}

\begin{theorem}{}
函数在某点存在极限的充分必要条件是它左右极限都存在并相等。
$$\lim_{x\to x_0} f(x) = A \Leftrightarrow \lim_{x\to x_0^-} f(x) = \lim_{x\to x_0^+} f(x) = A ~.$$

也就是说,若左(或右)极限不存在,或者左右极限存在但不相等,那此处的极限就不存在。
\end{theorem}

\begin{example}{}
\begin{figure}[ht]
\centering
\includegraphics[width=8cm]{./figures/17fd59550e160143.pdf}
\caption{函数$\theta(x)$的图像} \label{fig_FunLim_3}
\end{figure}
函数
\begin{equation}
\theta(x) = \leftgroup{
0 \qquad (x < 0)\\
1 \qquad (x \ge 0)
}~.\end{equation}

计算左极限$\lim\limits_{x\to x_0^-} \theta(x)$时,假定x从左侧不断接近$0$($x=-0.1,x=-0.01,...$), 但从不超过(也不等于)$0$。此时总有$x<0$,因此$\lim\limits_{x\to x_0^-} \theta(x) = 0$. 

同理,$\lim\limits_{x\to x_0^+} \theta(x) = 1~.$

由于左右极限不相同,因此$\theta(x)$在$x=0$处的\textbf{极限不存在}。
\end{example}

\subsubsection{某点处函数的极限值与函数值}
新手最常犯的错误莫过于\textbf{过度纠结}某处的函数极限值$\lim\limits_{x\to x_0} f(x)$与函数值$f(x_0)$的联系。事实上,这两者之间没有\textbf{必要的关联}\footnote{对于连续函数,才有$\lim\limits_{x\to x_0} f(x)=f(x_0)$。然而,大多数常见的函数都是连续函数,这使得这个问题更具迷惑性}。$f(x_0)$可以不等于$\lim\limits_{x\to x_0} f(x)$,$f(x)$甚至可以在$x_0$处没有定义。总之,某处的函数极限值并不依赖于该点处的函数值。

这是因为定义中只考虑 $x$ 慢慢接近 $x_0$ 的过程, 而不考虑 $x = x_0$ 的情况。 即使我们把这点从函数定义域中挖去, 极限是否存在, 以及极限值是多少都不会被改变。 例如在\autoref{ex_FunLim_1} 与在 “\enref{小角极限}{LimArc}” 中会看到, 虽然 $\sin x/ x$ 在 $x = 0$ 处没有定义, 但其极限却等于 $1$。

\begin{example}{可去间断点}
\begin{figure}[ht]
\centering
\includegraphics[width=10cm]{./figures/0953dfc9743819e5.png}
\caption{函数f(x)的图像} \label{fig_FunLim_4}
\end{figure}
函数
\begin{equation}
f(x) = \leftgroup{
x \qquad (x \ne 1)\\
1.5 \qquad (x = 1)
}~.\end{equation}
计算$\lim\limits_{x\to 1} f(x)$时,由于只考虑$x=1$附近的情况、而不考虑$x=1$本身的情况,因此$\lim\limits_{x\to 1} f(x)$的结果与$f(1)$的值无关。在本例中,$\lim\limits_{x\to 1} f(x)=1$, 而$f(1)=1.5$。
\end{example}

\subsection{极限不存在}
和数列的极限一样,如果一个函数 $f(x)$ 的某种极限不存在,就说该极限\textbf{不收敛}。 但不收敛的情况也有不同的细分。

如果在一个极限中,函数值趋于正无穷或负无穷\footnote{如果要严格定义\autoref{eq_FunLim_1},就用 $\delta$-$\epsilon$ 语言,例如:对任意给定的 $A > 0$ 总存在……,当…… 就有 $f(x) > A$。},则记为
\begin{equation}\label{eq_FunLim_1}
\lim\limits_{x\to \square} f(x) = \pm\infty~.
\end{equation}
其中 $+\infty$ 可以简记为 $\infty$。 注意术语上不能说该极限\textbf{存在}且等于无穷,因为该极限是不存在的,存在就意味着等号右边是一具体的数。

\begin{example}{}
\autoref{eq_FunLim_1} 并不是极限不存在的唯一一种情况,例如 $\lim\limits_{x\to\pm\infty}\sin x$ 和 $\lim\limits_{x\to 0}\sin(1/x)$ 的极限同样不存在,但不满足\autoref{eq_FunLim_1}。
\end{example}
