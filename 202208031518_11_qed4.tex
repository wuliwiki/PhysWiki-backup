% 狄拉克方程
% dirac function

\pentry{Klein-Gordon方程\upref{KGeq}}

注:本篇内容采取自然单位制\upref{NatUni},即 $c=\hbar=1$.
\subsection{背景}
薛定谔方程并不能描述高速运动的微观粒子,而且由于不满足洛伦兹协变性,无法保证在别的惯性系成立.自然人们希望有满足狭义相对论及量子力学的理论出现.1927年, Klein-Gordon 方程被顺势提出.该方程满足狭义相对论中的质壳条件($E^{2}=\boldsymbol{p}^{2}+m^{2}$),描述的是无自旋的自由粒子:
\begin{equation}\label{qed4_eq1}
\left(\frac{\partial^{2}}{\partial t^{2}}-\nabla^{2}+m^{2}\right) \phi(\boldsymbol{x}, t)=0
\end{equation}
然而它产生两个无法忽视的问题:

\begin{enumerate}
\item \textbf{负能解}
\autoref{qed4_eq1} 的平面波解为 $\phi_{\boldsymbol{p}}(\boldsymbol{x}, t)=\frac{1}{\sqrt{V}} e^{- \I\left(E_{\boldsymbol{p}} t-\boldsymbol{p} \cdot \boldsymbol{x}\right)}$.质壳条件并没有对能量的正负作出约束,其本征值为
$E_{\boldsymbol{p}}=\pm \sqrt{\boldsymbol{p}^{2}+m^{2}}$,然而自由粒子不应具有负能量.
\item \textbf{负概率密度}
K-G方程的荷密度为 $\rho \equiv  \I\left(\phi^{*} \frac{\partial \phi}{\partial t}-\phi \frac{\partial \phi^{*}}{\partial t}\right)$,代入平面波函数解,显然荷密度可以取为负值.然而荷密度表示的是粒子在$t$时刻,某区域体积$V$内发现的动量为$p$的概率,所以负值是没有物理意义的.
\end{enumerate}
\subsection{狄拉克方程}
K-G方程无法克服的疑难是源于约束条件仅仅是一个对时间和空间的二阶微分方程,所以为了解决以上两个问题,同时粒子需要继续满足K-G方程,那么寻找一个能量的一阶微分方程表达式是一个自然的选择.
狄拉克提出,在三维空间中运动的自由粒子应满足:
\begin{equation}\label{qed4_eq2}
E=\boldsymbol{\alpha} \cdot \boldsymbol{p}+\beta m=\alpha_{x} p_{x}+\alpha_{y} p_{y}+\alpha_{z} p_{z}+\beta m
\end{equation}
其中,$\boldsymbol {\alpha}$和$\beta$是待定系数,它们一定不能是常数,否则不能满足K-G方程,假设这两个待定系数是矩阵.\autoref{qed4_eq2} 平方后有
\begin{equation}\label{qed4_eq3}
\begin{aligned}
E^{2} &=\alpha_{x}^{2} p_{x}^{2}+\alpha_{y}^{2} p_{y}^{2}+\alpha_{z}^{2} p_{z}^{2}+\beta^{2} m^{2} \\
&+\left(\alpha_{x} \alpha_{y}+\alpha_{y} \alpha_{x}\right) p_{x} p_{y}+\left(\alpha_{y} \alpha_{z}+\alpha_{z} \alpha_{y}\right) p_{y} p_{z}+\left(\alpha_{x} \alpha_{z}+\alpha_{z} \alpha_{x}\right) p_{x} p_{z} \\
&+\left(\alpha_{x} \beta+\beta \alpha_{x}\right) p_{x} m+\left(\alpha_{y} \beta+\beta \alpha_{y}\right) p_{y} m+\left(\alpha_{z} \beta+\beta \alpha_{z}\right) p_{z} m
\end{aligned}
\end{equation}
若质壳关系依然成立,需要:
\begin{equation}\label{qed4_eq4}
\begin{aligned}
\beta^{2} &=1 & & \\
\alpha_j^{2} &=1 & &(j=1,2,3) \\
\alpha_j \alpha_k+\alpha_k \alpha_j &=0 & &(j, k=1,2,3, j \neq k) \\
\alpha_j \beta+\beta \alpha_j &=0 & &(j=1,2,3)
\end{aligned}
\end{equation}
显然,$\boldsymbol{\alpha}, \beta$ 不能是数,必须(至少)为四阶矩阵.$\alpha_j, \beta$ 满足反对易关系.

又因为 $E$ 为厄米算符,即 $E=\boldsymbol{\alpha} \cdot \boldsymbol{p}+\beta m=\boldsymbol{\alpha}^\dagger \cdot \boldsymbol{p}+\beta^\dagger m$,所以
\begin{equation}\label{qed4_eq5}
\alpha_j^{\dagger}=\alpha_j, \quad \beta^{\dagger}=\beta
\end{equation}
同时满足\autoref{qed4_eq4} 和\autoref{qed4_eq5} 的不同表示\textbf{(representation)}之间通过相似变换联系().
常用的表示有以下两种:

\textbf{标准表示(standard representation):}
\begin{equation}\label{qed4_eq6}
\alpha^{j}=\left(\begin{array}{cc}
0 & \sigma^{j} \\
\sigma^{j} & 0
\end{array}\right), \quad \beta=\left(\begin{array}{rr}
I & 0 \\
0 & -I
\end{array}\right)
\end{equation}
其中 $\sigma^{j}$ 为泡利矩阵
\begin{equation}\label{qed4_eq7}
\sigma^{1}=\left(\begin{array}{ll}
0 & 1 \\
1 & 0
\end{array}\right), \quad \sigma^{2}=\left(\begin{array}{rr}
0 & -i \\
i & 0
\end{array}\right), \quad \sigma^{3}=\left(\begin{array}{rr}
1 & 0 \\
0 & -1
\end{array}\right)
\end{equation}

\textbf{威尔/手征表示(Weyl/Chiral representation):}
\begin{equation}\label{qed4_eq8}
\alpha^{j}=\left(\begin{array}{cc}
-\sigma^{j} & 0 \\
0 & \sigma^{j}
\end{array}\right), \quad \beta=\left(\begin{array}{cc}
0 & I \\
I & 0
\end{array}\right)
\end{equation}
对\autoref{qed4_eq2} 进行一次量子化.动量被动量算符代替,由含时薛定谔方程得:\begin{equation}\label{qed4_eq9}
\boxed{ \I \frac{\partial}{\partial t} \psi(\boldsymbol{x}, t)=\left[- \I \sum_{j} \alpha^{j} \frac{\partial}{\partial x^{j}}+\beta m\right] \psi(\boldsymbol{x}, t), \quad j=1,2,3}
\end{equation}
这就是狄拉克方程的一般形式.由于两个表示的 $\alpha,\beta$ 都是四阶矩阵,所以狄拉克方程的解为具有四个分量的波函数,该解又称为\textbf{Dirac旋量}
\begin{equation}\label{qed4_eq10}
\psi=\left(\begin{array}{l}
\psi_{1} \\
\psi_{2} \\
\psi_{3} \\
\psi_{4}
\end{array}\right)
\end{equation}
与量子力学中的表象概念类似,不同的表示并不影响方程背后的物理内涵.具有物理意义的本征值不变,证明如下:

\textbf{标准表示}
采取静止坐标系,此时粒子动量为0,\autoref{qed4_eq9} 变为
\begin{equation}\label{qed4_eq11}
 \I \frac{\partial}{\partial t} \psi= \beta m\psi
\end{equation}
令 $\psi=\left(\begin{array}{l}\psi_{+} \\ \psi_{-}\end{array}\right)$,代入 $\beta$ 得:
\begin{equation}\label{qed4_eq14}
 \I \frac{\partial}{\partial t} \psi_{+}=+m \psi_{+}
\end{equation}
\begin{equation}\label{qed4_eq15}
 \I \frac{\partial}{\partial t} \psi_{-}=-m \psi_{-}
\end{equation}
以上两式表明 $\psi_{+}$ 与 $\psi_{-}$ 分别对应能量本征值为 $m$ 与 $-m$ 的波函数解.

\textbf{手征表示} 同样采取静止坐标系,令 $\psi=\left(\begin{array}{l}\psi_{L} \\ \psi_{R}\end{array}\right)$,代入该表示中的 $\beta$ 于\autoref{qed4_eq11} ,则:
\begin{equation}\label{qed4_eq12}
 \I\frac{\partial}{\partial t} \psi_{L}=m \psi_{R}
\end{equation}
\begin{equation}\label{qed4_eq13}
 \I\frac{\partial}{\partial t} \psi_{R}=m \psi_{L}
\end{equation}
上式耦合了 $\psi_{L}$ 与 $\psi_{R}$,然而只要令
\begin{equation}
\psi_{+}=\frac{1}{\sqrt{2}}\left(\psi_{L}+\psi_{R}\right)
\end{equation}
\begin{equation}
\psi_{-}=\frac{1}{\sqrt{2}}\left(\psi_{L}-\psi_{R}\right)
\end{equation}
代入\autoref{qed4_eq12} 与\autoref{qed4_eq13} ,并进行相加减,最后可得到\autoref{qed4_eq14} 与\autoref{qed4_eq15} ,意味着不同表示并没有改变能量本征值与对应的本征态.



\subsubsection{描述对象}
\textbf{狄拉克方程可以描述有质量的自旋为 $1/2$ 的正反费米子.}下面我们将从两种角度进行证明:\\
\textbf{1.守恒量}

将轨道角动量 $L$ 代入算符的运动方程有:
\begin{equation}\label{qed4_eq16}
\frac{d {L}}{d t}=- \I[L,H]
\end{equation}

由于狄拉克方程描述的自由粒子角动量守恒,令上式 为0,代入狄拉克方程的哈密顿量后有:
\begin{equation}\label{qed4_eq18}
- \I[L,H]=
 \I\left[\boldsymbol{\alpha} \cdot \boldsymbol{p},\boldsymbol{r} \times \boldsymbol{p}\right]
= \I[\alpha_ip_i,i\epsilon_{ijk}r_ip_j]=\boldsymbol{\alpha} \times \boldsymbol{p}=0
\end{equation}
显然不成立,所以总角动量必须引入其他自由度.现在引入新算符
\begin{equation}
\boldsymbol{\Sigma} \equiv\left(\begin{array}{cc}
\boldsymbol{\sigma} & 0 \\
0 & \boldsymbol{\sigma}
\end{array}\right)
\end{equation}
其中 $\boldsymbol{\sigma}$ 为二阶泡利矩阵.现在考察该算符与哈密顿量的对易关系:
\begin{equation}\label{qed4_eq17}
[H, \mathbf{\Sigma}]=[\boldsymbol{\alpha} \cdot \boldsymbol{p}+m\beta , \boldsymbol{\Sigma}]=[\boldsymbol{\alpha} \cdot \boldsymbol{p}, \boldsymbol{\Sigma}]+[m\beta , \boldsymbol{\Sigma}]
\end{equation}
由于\begin{equation}
[\beta m, \mathbf{\Sigma}]=\left(\begin{array}{cc}
I & 0 \\
0 & -I
\end{array}\right)\left(\begin{array}{cc}
\boldsymbol{\sigma} & 0 \\
0 & \boldsymbol{\sigma}
\end{array}\right)-\left(\begin{array}{cc}
\boldsymbol{\sigma} & 0 \\
0 & \boldsymbol{\sigma}
\end{array}\right)\left(\begin{array}{cc}
I & 0 \\
0 & -I
\end{array}\right)=0
\end{equation}
所以式21 化为
\begin{equation}
[H, \Sigma]=[\boldsymbol{\alpha} \cdot \boldsymbol{p}, \boldsymbol{\Sigma}]
\end{equation}
对 $x$ 分量进行计算,有:
\begin{equation}
\left[H, \Sigma_{x}\right]=\left[\alpha_{x} p_{x}+\alpha_{y} p_{y}+\alpha_{z} p_{z}, \Sigma_{x}\right]=p_{x}\left[\alpha_{x}, \Sigma_{x}\right]+p_{y}\left[\alpha_{y}, \Sigma_{y}\right]+p_{z}\left[\alpha_{z}, \Sigma_{z}\right]
\end{equation}
分别代入两个算符,得:
\begin{equation}
\left[\alpha_{x}, \Sigma_{x}\right] = \left(\begin{array}{cc}
0 & \sigma_{x} \\
\sigma_{x} & 0
\end{array}\right)\left(\begin{array}{cc}
\sigma_{x} & 0 \\
0 & \sigma_{x}
\end{array}\right)-\left(\begin{array}{cc}
\sigma_{x} & 0 \\
0 & \sigma_{x}
\end{array}\right)\left(\begin{array}{cc}
0 & \sigma_{x} \\
\sigma_{x} & 0
\end{array}\right) = 0
\end{equation}

\begin{equation}
\left[\alpha_{y}, \Sigma_{x}\right] = \left(\begin{array}{cc}
0 & \sigma_{y} \\
\sigma_{y} & 0
\end{array}\right)\left(\begin{array}{cc}
\sigma_{x} & 0 \\
0 & \sigma_{x}
\end{array}\right)-\left(\begin{array}{cc}
\sigma_{x} & 0 \\
0 & \sigma_{x}
\end{array}\right)\left(\begin{array}{cc}
0 & \sigma_{y} \\
\sigma_{y} & 0
\end{array}\right)
\end{equation}
\begin{equation}
\begin{array}{l}
=\left(\begin{array}{cc}
0 & \sigma_{y} \sigma_{x}-\sigma_{x} \sigma_{y} \\
\sigma_{y} \sigma_{x}-\sigma_{x} \sigma_{y} & 0
\end{array}\right)=\left(\begin{array}{cc}
0 & -2 i \sigma_{z} \\
-2 i \sigma_{z} & 0
\end{array}\right) \\
=-2 i \alpha_{z}
\end{array}
\end{equation}
整理一下:
\begin{equation}
\left[H, \Sigma_{x}\right]=-2  \I \alpha_{z} p_{y}+2 i \alpha_{y} p_{z}=2  \I(\alpha \times \boldsymbol{p})_{x}
\end{equation}
实际上,另外两个分量也有类似的表达式,最后我们有:
\begin{equation}\label{qed4_eq19}
[H, \mathbf{\Sigma}]=2  \I \boldsymbol{\alpha} \times \boldsymbol{p}
\end{equation}
对比式19 和式29,我们可以得到:
\begin{equation}
\left[H, \boldsymbol{L}+\frac{1}{2} \boldsymbol{\Sigma}\right]=[H, \boldsymbol{L}+\boldsymbol{S}]=[H, \boldsymbol{J}]=0
\end{equation}
由于 $S$ 的不同分量满足角动量的对易关系,所以 $S$ 为该自由粒子的内禀角动量,即自旋.
又因为
\begin{equation}
S^{2}=\frac{3}{4}\left(\begin{array}{ll}
I & 0 \\
0 & I
\end{array}\right), \quad S_{z}=\frac{1}{2}\left(\begin{array}{cc}
\sigma_{z} & 0 \\
0 & \sigma_{z}
\end{array}\right)
\end{equation}
通过计算,可知 $S^2$ 的本征值为 $\frac{3}{4}$,$S_{z}$ 的本征值为 $\frac{1}{2}$,所以 $S$ 描写的是自旋为1/2的粒子.

\textbf{2.洛伦兹协变性}证明思路如下:根据狄拉克方程的洛伦兹不变性,可以推出旋量的洛伦兹变换形式,进而得到自旋为 $\frac{1}{2}$.为了讨论方便,一般对式9狄拉克方程进行改写.引入 $\gamma^{\mu}$:
\begin{equation}
\begin{array}{l}
\gamma^{0}=\beta \\
\gamma^{i}=\beta \alpha^{i} \quad(i=1,2,3)
\end{array}
\end{equation}
可以证明, $\gamma^{\mu}$ 满足反对易关系:
\begin{equation}
\left\{\gamma^{\mu}, \gamma^{\nu}\right\}=2 \eta^{\mu \nu}
\end{equation}
$\eta^{\mu \nu}$ 为闵可夫斯基空间的度规张量.狄拉克方程此时可以写为:
\begin{equation}
\left(\mathrm{i} \gamma^{\mu} \frac{\partial}{\partial x^{\mu}}-m\right) \psi(x)=0
\end{equation}
若引入费曼斜线标记,即 $\not A$ 表示 $\gamma^{\mu}A_{\mu}$,狄拉克方程可以更加简便地写为:
\begin{equation}\label{qed4_eq21}
( \not p-m) \psi=0
\end{equation}
\subsubsection{拉氏量}
狄拉克方程作为描述旋量的经典场方程,理应能从拉格朗日方程或由哈密顿原理导出,以此可作为拉氏量的一种验证方式.方程的洛伦兹协变性要求拉氏量也具有相同的协变性.在无相互作用的条件下,Dirac场的拉氏密度如下:
\begin{equation}\label{qed4_eq20}
\mathcal{L}=\mathrm{\I} \bar{\psi} \gamma^{\mu} \partial_{\mu} \psi-m \bar{\psi} \psi \quad\left(\bar{\psi}=\psi^{\dagger} \gamma^{0}\right)
\end{equation}
对 $\bar\psi$ 进行变分可得到Dirac旋量的运动方程,即式34.

\subsection{狄拉克方程的解}
现在我们在手征表象下,对式35进行求解.由于狄拉克方程可变换为K-G方程,所以狄拉克方程的解亦是平面波的叠加,即 $\Psi(x)=u(p)e^{-ipx}+v(p)e^{ipx}$,其中 $u(p)$ 和 $v(p)$ 都是四分量旋量.将正能部分和负能部分分别代入狄拉克方程后可得:
\begin{equation}
(\not p-m) u(p)=0 
\end{equation}
\begin{equation}
(\not p+m) v(p)=0
\end{equation}

我们先对正能部分进行讨论.为了方便,对粒子采取静止坐标系,则 $p^\mu=(m,0,0,0)$.(非静坐标系下的解可通过对静坐标系下的解作boost得到).式37在手征表象下为:
\begin{equation}
\left(m \gamma^{0}-m\right) u\left(p\right)=m\left(\begin{array}{rr}
-1 & 1 \\
1 & -1
\end{array}\right) u\left(p\right)=0
\end{equation}
解得\begin{equation}
u\left(p\right)=\sqrt{m}\left(\begin{array}{l}
\xi \\
\xi
\end{array}\right)
\end{equation}
$\xi$ 是任意的二分量旋量 
\subsection{狄拉克方程对K-G方程疑难的解释}
\subsubsection{负能量}
对于解决负能量疑难,狄拉克提出了空穴理论,通过创立狄拉克海模型和空穴概念,把相对论量子力学拓展到多粒子体系.该理论的主要观点如下:
\begin{itemize}
\item  负能态是最稳定的状态,即真空态.所有$E<0$的负能态都被电子占据,电子遵循泡利不相容原理,均匀排列在这片“狄拉克海”中.
\item 狄拉克海不可被观测,对于身处其中的电子,只有跃迁到正能量区域,才会被观测到(因为我们的实际测量值只是和基态的差值).此时狄拉克海中该电子的缺位可看作被“空穴”占据.空穴表现为一个质量与电子相同,但是电荷、动量方向和自旋方向与电子相反的正能量的粒子,这就是正电子.那么正电子的力学量取值为:\begin{eqnarray}
E & = & m_{e}>0, \quad Q & = & -Q_{e} & = & e, \quad \boldsymbol{p} & = & -\boldsymbol{p}_{e}, \quad S & = & -\boldsymbol{S}_{e}
\end{eqnarray}
此时,正能区的电子力学量为:
\begin{eqnarray}
E & = & -m_{e}<0, \quad Q & = & Q_{e} & = & -e, \quad \boldsymbol{p} & = & \boldsymbol{p}_{e}, \quad S & = & S_{e}
\end{eqnarray}

\end{itemize}
狄拉克所预言的正电子在1933年被Anderson实验发现.然而空穴理论并不是完美的,它只能解释费米子的负能解,而不适用于玻色子的情况,所以自旋为0的K-G玻色子也没有迎来春天.在之后,Strückelberg 和 Feynman重新解释了相对论方程中的负能解.回顾上文,平面波解为$\phi_{\boldsymbol{p}}(\boldsymbol{x}, t)=\frac{1}{\sqrt{V}} e^{-\mathrm{i}\left(E_{\boldsymbol{p}} t-\boldsymbol{p} \cdot \boldsymbol{x}\right)}$.负能解实际上对应的是时间方向相反的正能量反粒子.具体的,他们认为:相对论性运动方程的负能量解表示一个逆着时间方向运动的正能量反粒子,反粒子与粒子有相同的质量,相反的荷(电荷、奇异数、粲数等量子数),以及相反的动量方向和相反的自旋方向
.
\subsubsection{概率密度}
为了得到狄拉克方程中概率密度的表达式,我们需要对狄拉克方程稍作变换,得到四维矢量微分的方程式,并与连续性方程比较.为此,我们先对狄拉克方程取厄米共轭,有:
\begin{equation}
\left(\left(\mathrm{i} \gamma^{\mu} \frac{\partial}{\partial x^{\mu}}-m\right) \psi(x)\right)^{\dagger}=-i \partial_{\mu} \psi^{\dagger} \gamma^{0} \gamma^{\mu} \gamma^{0}-m \psi^{\dagger}=0
\end{equation}
以上利用了$\gamma^{\mu \dagger}=\gamma^{0} \gamma^{\mu} \gamma^{0}$.然后右乘$\gamma ^0$,利用 $(\gamma^{0})^2=0$,我们有
\begin{equation}
\begin{aligned}
-i \partial_{\mu} \psi^{\dagger} \gamma^{0} \gamma^{\mu}-m \psi^{\dagger} \gamma^{0}&=0\\
\therefore\quad i\left(\partial_{\mu} \bar{\psi}\right) \gamma^{\mu}+m \bar{\psi}&=0
\end{aligned}
\end{equation}
$\bar\psi\times$





