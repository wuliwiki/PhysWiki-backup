% 平均值(量子力学)
% 量子力学|平均值|本征值|本征态

\pentry{测量理论\upref{QMPos}}

\subsection{离散本征态}

我们先来回顾测量理论. 假设某个测量量 $q$ 对应的算符为 $Q$, 有离散的本征态(可以是无穷多个) $\ket{\phi_i}$ ($i = 1,2\dots$), 对应的本征值为 $q_i$, 满足
\begin{equation}\label{QMavg_eq3}
Q\ket{\phi_i} = q_i \ket{\phi_i}
\end{equation}
且满足正交归一条件
\begin{equation}\label{QMavg_eq4}
\braket{\phi_i}{\phi_j} = \delta_{i,j}
\end{equation}
记粒子处于 $\ket{\psi}$ 状态, 可表示为离散本征态的线性组合
\begin{equation}\label{QMavg_eq2}
\ket{\psi} = \sum_i c_i \ket{\phi_i}
\end{equation}
对其测量 $Q$, 得到第 $q_i$ 的概率为
\begin{equation}
P_i = \abs{c_i}^2 = \abs{\braket{\phi_i}{\psi}}^2
\end{equation}
现在我们可以习惯定义 $Q$ 的平均值为
\begin{equation}\label{QMavg_eq1}
\ev{Q} = \sum_i q_i P_i = \sum_i q_i \abs{c_i}^2
\end{equation}
这意味着, 如果我们取大量处于 $\ket{\psi}$ 状态的系统, 分别测量 $Q$ 再取平均, 结果就是该式.

\subsubsection{另一个种形式}
沿用上面的符号, 平均值的另一种等效形式是
\begin{equation}
\ev{Q} = \mel{\psi}{Q}{\psi}
\end{equation}
由于 $Q$ 是线性算符, 把\autoref{QMavg_eq2} 代入即可证明和\autoref{QMavg_eq1} 等效.

\begin{example}{一维简谐振子}
谐振子内容详见——量子简谐振子(升降算符法)\upref{QSHOop}

频率为 $\omega$ 的一维线性谐振子,状态为 $\psi(x)=c_1 \psi_0(x)+c_3\psi_3(x)$ ,其中 $\psi_n(x)$ 为第 $n$ 个能量本征态.给出能量的平均值.

首先应当保证波函数归一化,不难写出归一化后的波函数为
\begin{equation}
\psi(x)=\frac{c_1}{\sqrt{c_1^2+c_3^2}}\psi_0(x)+\frac{c_3}{\sqrt{c_1^2+c_3^2}}\psi_3(x)
\end{equation}

由\autoref{QMavg_eq1}
\begin{equation}
\ev{E}=\frac{c_1^2}{c_1^2+c_3^2} E_0+\frac{c_3^2}{c_1^2+c_3^2} E_3
\end{equation}

其中 $E_n$ 为第 $n$ 个能量本征值:$E_n=\hbar \omega(n+\frac{1}{2})$.
\end{example}

平均值还有一个更常见的公式, 与\autoref{QMavg_eq1} 等效.
\begin{equation}\label{QMavg_eq5}
\ev{Q} = \mel{\psi}{Q}{\psi}
\end{equation}
要验证, 可以将\autoref{QMavg_eq2} 代入该式, 得
\begin{equation}
\begin{aligned}
\ev{Q} &= \qty(\sum_i c_i^* \bra{\phi_i}) Q \qty(\sum_j c_j \ket{\phi_j})\\
&= \sum_{i,j} c_i^* c_j \bra{\phi_i} Q \ket{\phi_j}
\end{aligned} 
\end{equation}
再代入\autoref{QMavg_eq3} 和\autoref{QMavg_eq4} 得
\begin{equation}
\ev{Q} = \sum_{i,j} c_i^* c_j q_i \braket{\phi_i}{\phi_j}
= \sum_{i,j} c_i^* c_j q_i \delta_{i,j} = \sum_i \abs{c_i}^2 q_i
\end{equation}
证毕.

\subsection{连续本征态}
若 $Q$ 的本征态是离散的, 记为 $\ket{q}$ ($q\in \mathbb R$), 且满足正交归一条件(链接未完成)
\begin{equation}
\braket{q'}{q} = \delta_{q,q'}
\end{equation}
那么波函数的展开变为
\begin{equation}
\ket{\psi} = \int c(q) \ket{q} \dd{q}
\end{equation}
其中系数 $c(q)$ 是关于 $q$ 的复值函数. 而 $\abs{c(q)}^2$ 则是测得 $q$ 的概率密度函数. 这时可以定义平均值为
\begin{equation}

\end{equation}



\begin{example}{波函数的坐标、动量平均值}
假设一粒子波函数为
\begin{equation}
\psi(x)=
\begin{cases}
C(a^2-x^2) &(-a\leqslant x\leqslant a)\\
0 &(x<-a \ \text{或}\ x>a) 
\end{cases}
\end{equation}
\begin{figure}[ht]
\centering
\includegraphics[width=7cm]{./figures/QMavg_2.pdf}
\caption{粒子波函数示意图} \label{QMavg_fig2}
\end{figure}
其中 $C$ 已归一化.求粒子的坐标、动量平均值.

由\autoref{QMavg_eq5} 
\begin{equation}\label{QMavg_eq6}
\ev{x}= \int_{-a}^a C^*(a^2-x^2)xC(a^2-x^2)\dd{x} = 0
\end{equation}

同理,将\autoref{QMavg_eq6} 中的算符 $x$ 用动量算符 $p$ 替换,可以得到动量平均值
\begin{equation}
\begin{aligned}
\ev{p} &= \int_{-a}^a C^*(a^2-x^2)(-i\hbar\frac{\dd}{\dd{x}})C(a^2-x^2)\dd{x}\\
&= -i\hbar |C|^2\int_{-a}^a (a^2-x^2)(-2x)\dd{x}\\
&= 0
\end{aligned}
\end{equation}
\end{example}
% 未完成: 无限深势阱中的三角波包, 用两种方法求能量的平均值






