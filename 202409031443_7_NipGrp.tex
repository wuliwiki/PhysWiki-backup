% 幂零群
% keys nilpotent group|正规序列|次正规序列|normal series|subnormal series|可解群
% license Usr
% type Tutor


\pentry{换位子群\nref{nod_CmtGrp}}{nod_3aa9}



\autoref{the_CmtGrp_2} 说明,给定群$G$,则求换位子群的运算在$G$的正规子群集合上封闭,因此我们可以认为“求换位子群”运算是定义在全体正规子群集合上的。

从运算的角度,我们可以定义$G^1=G$,而对任意正整数$k$,$G^{k+1}=[G, G^{k}]$,像是用求换位子群运算对$G$反复乘方,求其幂次。当然,这只是一个类比,求换位子群运算和乘法差别很大。


\begin{definition}{幂零群}
给定群$G$,若存在正整数$k$使得$G^k=\{e\}$,则称$G$是\textbf{幂零的(nilpotent)}。
\end{definition}

一个显而易见的性质是:


\begin{theorem}{}
幂零群的子群和商群都幂零。
\end{theorem}

\textbf{证明}:

设$G$是幂零群,$H$是其任意子群,$N$是其任意正规子群。

由于$H\subseteq G$,故$[H, H^{k}]\subseteq [G, H^{k}]$;又因为$H^1\subseteq G^1$,故可归纳得$H^k\subseteq G^k$对任意正整数$k$成立。由此易证$H$幂零。

由于$G$幂零,故存在正整数$n$使得$G^n\subseteq N$。这意味着,任取$g_1, g_2, \cdots, g_n\in G$,则
\begin{equation}
[\cdots[g_{n-1}, g_n]\cdots]~
\end{equation}



\textbf{证毕}。



































