% Arpack++2 大型本征方程库

\begin{issues}
\issueDraft
\end{issues}

\pentry{C++ 基础\upref{Cpp0}, 密矩阵\upref{MatSto}, 稀疏矩阵\upref{SprMat}}

\href{https://www.caam.rice.edu/software/ARPACK/}{Arpack} (ARnoldi PACKage)最初是一个 Fortran 语言编写的, 用于求解大型本征方程的程序库, 其主要算法是 Arnoldi 循环. 下面是它的一些特性:
\begin{itemize}
\item 可以求解广义本征问题(General) $Ax = \lambda Bx$ 或者标准本征问题(Standard) $Ax = \lambda x$.
\item 矩阵可以表示为密矩阵\upref{MatSto}, 系数的 CSC 矩阵\upref{SprMat}. 也可以不需要提供矩阵本身而是提供矩阵—矢量乘法. 也就是你只需要提供一个函数 \verb|v = F(u)| 给 Arpack 即可, 其中 \verb|v| 就是矩阵乘以 \verb|u| 的结果.
\item 它可以求出本征值最小(大)的指定个数的本征矢, 或者本征值的绝对值最小(大)的指定个数的本征矢.
\item shift 模式: 可以求出指定值附近的若干本征值和本征矢. 但这可能需要提供矩阵的逆(见下文).
\end{itemize}

\href{http://www.ime.unicamp.br/~chico/arpack++/}{Arpack++} 是 Arpack 的 C++ 模板库, 但实际上只是一个 C++ 接口, 仍然调用 Fortran 版的动态链接库. 官网提供的版本已经多年没有维护(最后更新于 2000 年, 版本 1.2), 用目前的 g++ 编译器无法正常编译. 麻省理工在 GitHub 上维护了一个新版本: \href{https://github.com/m-reuter/arpackpp}{Arpack++2}. 本文使用的是当前最新的 release 2.3.

\subsection{Ubuntu/Debian 安装}
Ubuntu/Debian 中可以直接用 apt 安装, 如果没有装 gfortran 要先 \verb|apt install gfortran|. 然后安装 \verb|apt install libarpack++2-dev| (当前最新是 2.3 版), 其实我们只需要它安装的二进制文件和 dependency. 注意 dependency 中会自动安装 OpenBlas, 如果已经装了其他版本的 Blas 如 CBLAS 可以将其手动卸载\footnote{OpenBlas 和 CBLAS 的接口不完全相同, 前者的函数使用 \lstinline|double *| 输入复数, 而后者使用 \verb|void *|, 兼容性更强. 例如本书的 SLISC0 库用 OpenBlas 就会编译出错(待更新).}: \verb|sudo apt purge libopenblas-dev|.

安装好后, 可以用 \verb|dpkg -L libarpack++2-dev| 查看安装的文件, 其中头文件和文档可以删除, 使用 github 上的头文件和文档. doc 中的 pdf 文档是重要的参考. 例如要编译 \verb|examples/areig/nonsym/simple.cc| 中的例子, 在该目录运行 \verb|make simple| 即可编译. 用 \verb|./simple| 运行.

\subsection{使用}
第三章讲解了 Classes that require user-defined matrix-vector products, 即不直接提供矩阵而是提供矩阵—矢量乘法. Real symmetric standard problems 讲解了具体怎么用 \verb|ARSymStdEig| 类. 如果要求解(数值或绝对值)最大或者最小的本征值, 那么只需要矩阵—矢量乘法即可, 但如果要算某个值 $\sigma$ 附近的本征值, 那么需要提供 $(A - \sigma)^{-1}$ 的乘法. 这有时候会大大增加难度, 甚至不可能做到.

Arpack 解决的本征问题分为两种, 一种是 General 本征问题, 就是 $Ax = \lambda Bx$, 例如例程 \verb|symg...cc| 代表对称矩阵的 General 本征问题, 没有 \verb|g| 代表是 Standard 本征问题.

\verb|example/product/sym| 文件夹中是通过提供矩阵—矢量乘法来解对称矩阵的方程. 其中没有 \verb|g| 的例子有两个, 分别是 \verb|symreg.cc| 和 \verb|symshft.cc|, 使用的都是 \verb|ARSymStdEig| 类. 其中 \verb|Sym| 代表对称, \verb|Std| 代表是非广义的本征值问题. 使用方法:
\begin{lstlisting}[language=cpp]
#include "arssym.h"
// 求 4 个绝对值最小的本征值的本征矢.
ARSymStdEig<T, SymMatrixA<T>>
    dprob(A.ncols(), 4, &A, &SymMatrixA<T>::MultMv, "SM");
// 求 4 个最接近 1 的本征值的本征矢.
ARSymStdEig<T, SymMatrixB<T>>
    dprob(A.ncols(), 4, &A, &SymMatrixB<T>::MultMv, 1.);
\end{lstlisting}
其中 \verb|SymMatrixB<ART>::MultMv(ART* u, ART* v)| 应该是可以自己随便写的. \verb|MultMv(u, v)| 输入是 \verb|u|, 输出是 \verb|v|.
