% 原码、反码、补码
% keys 编码|位运算|计算机组成原理
% license Usr
% type Wiki
\begin{issues}
\issueDraft
\end{issues}

\pentry{数字电路_运算器\nref{nod_Sample}}{nod_eb6a}

\subsection{原码}

原码、反码、补码都是针对整数的二进制编码方式。一般包括一个符号位和数值位。符号位通常使用最高位表示,其中0代表正数,1代表负数;剩余的位表示数字的绝对值。由于其直观的表示方法,原码易于理解,但在进行算术运算时会遇到一些问题,如正负零的歧义和运算复杂性。

\subsection{反码}

反码同样用于表示整数,特别是在计算机系统内部处理负数时。在反码表示法中,正数的反码与其原码相同;负数的反码则是将原码(除符号位外)的每一位取反。反码解决了一些原码在运算上的问题,但仍然存在如负零的表示以及加法运算中的进位问题。

\subsection{补码}

\subsection{为什么要使用补码}

\subsubsection{原码或反码}

如果使用原码或反码存储数据,

\subsubsection{补码}


% 参考 https://www.cnblogs.com/zhangziqiu/archive/2011/03/30/computercode.html remove by lzq
