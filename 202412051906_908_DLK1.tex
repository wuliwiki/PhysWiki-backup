% 保罗·狄拉克(综述)
% license CCBYSA3
% type Wiki

本文根据 CC-BY-SA 协议转载翻译自维基百科\href{https://en.wikipedia.org/wiki/Paul_Dirac}{相关文章}。


\begin{figure}[ht]
\centering
\includegraphics[width=6cm]{./figures/887a811793832fb1.png}
\caption{1933年的狄拉克} \label{fig_DLK1_1}
\end{figure}
保罗·阿德里安·莫里斯·狄拉克(Paul Adrien Maurice Dirac,/dɪˈræk/;1902年8月8日 – 1984年10月20日)是英国的数学物理学家和理论物理学家,被认为是量子力学的奠基人之一。[6][7] 狄拉克为量子电动力学和量子场论的基础奠定了基础。[8][9][10][11] 他曾担任剑桥大学卢卡斯数学教授、佛罗里达州立大学物理学教授,并于1933年获得诺贝尔物理学奖。

狄拉克于1921年毕业于布里斯托大学,获得电气工程学的一级荣誉理学士学位,1923年获得数学的一级荣誉文学学士学位。[12] 随后,他于1926年从剑桥大学获得物理学博士学位,撰写了首篇关于量子力学的论文。[13]

狄拉克对量子力学和量子电动力学的早期发展作出了基础性贡献,并创造了后者的术语。[10] 其中,他在1928年提出了狄拉克方程,该方程描述了费米子的行为,并预测了反物质的存在,[14] 这一方程被认为是物理学中最重要的方程之一,[8] 并被一些物理学家视为“现代物理学的真正种子”。[15] 他在1931年撰写了一篇著名论文,[16] 进一步预测了反物质的存在。[17][18][14] 狄拉克与厄尔温·薛定谔共同分享了1933年诺贝尔物理学奖,以表彰他们“发现了原子理论的新生产性形式”。[19] 他是最年轻的诺贝尔理论物理学奖获得者,直到1957年T·D·李获奖。[20] 狄拉克还为广义相对论与量子力学的和解做出了巨大贡献。他的1930年专著《量子力学原理》是量子力学最具影响力的经典著作之一。[21]

狄拉克的贡献不仅限于量子力学。他还为“管道合金”项目作出了贡献,这是英国在二战期间研究和建造原子弹的计划。[22][23] 狄拉克对铀浓缩过程和气体离心机做出了基础性贡献,[24][25][26][23] 他的工作被认为是“可能是离心机技术中最重要的理论成果”。[27] 他还对宇宙学做出了贡献,提出了“大数假说”。[28][29][30][31] 狄拉克还在弦理论诞生之前预见到了弦理论,提出了如狄拉克膜、狄拉克-伯恩-因费尔德作用等工作,这些贡献对现代弦理论和规范理论至关重要。[32][33][34][35]

狄拉克被朋友和同事视为个性独特。在1926年给保罗·埃伦费斯特的信中,阿尔伯特·爱因斯坦写道:“我在费劲地研究狄拉克。这种在天才与疯狂之间的摇摆实在可怕。”在另一封关于康普顿效应的信中,他写道:“我完全不理解狄拉克的细节。”[36] 1987年,阿卜杜斯·萨拉姆宣称:“狄拉克无疑是这个世纪最伟大的物理学家之一……除爱因斯坦外,没有人能在如此短的时间内,对本世纪物理学的发展产生如此决定性的影响。”[37] 1995年,斯蒂芬·霍金表示:“狄拉克比任何人都做得更多,除了爱因斯坦之外,他推动了物理学的发展并改变了我们对宇宙的认识。”[38] 安东尼诺·齐基奇认为狄拉克对现代物理学的影响超过了爱因斯坦,[15] 斯坦利·德塞尔则评论道:“我们都站在狄拉克的肩膀上。”[39] 狄拉克被广泛认为与艾萨克·牛顿、詹姆斯·克拉克·麦克斯韦和爱因斯坦相提并论。[40][41][42]
\subsection{个人生活}  
\subsubsection{早年}
\begin{figure}[ht]
\centering
\includegraphics[width=6cm]{./figures/3f71a2576aff2af0.png}
\caption{克拉拉·埃瓦尔德(Clara Ewald)所绘的保罗·狄拉克肖像(1939年)} \label{fig_DLK1_2}
\end{figure}
保罗·阿德里安·莫里斯·狄拉克(Paul Adrien Maurice Dirac)于1902年8月8日出生在英国布里斯托尔市父母的家中,并在该市的比肖普斯顿区长大。他的父亲,查尔斯·阿德里安·拉迪斯拉斯·狄拉克(Charles Adrien Ladislas Dirac),是来自瑞士圣莫里茨的移民,具有法国血统,在布里斯托尔担任法语教师。他的母亲,弗洛伦斯·汉娜·狄拉克(Florence Hannah Dirac,娘家姓霍尔滕),出生在康沃尔的利斯基尔,是一个康沃尔卫理公会家庭的成员。她是由其父亲命名的,父亲是一位海军船长,在克里米亚战争期间曾与弗洛伦斯·南丁格尔(Florence Nightingale)相识。她在年轻时搬到布里斯托尔,并在那里担任布里斯托尔中央图书馆的图书管理员;尽管如此,她依然认为自己的身份是康沃尔人,而不是英国人。保罗有一个妹妹,比阿特丽斯·伊莎贝尔·玛格丽特(Béatrice Isabelle Marguerite),昵称贝蒂,还有一个哥哥,雷金纳德·查尔斯·费利克斯(Reginald Charles Félix),昵称费利克斯,后者在1925年3月自杀。狄拉克后来回忆道:“我的父母非常痛苦,我从未知道他们如此在意……我从来不知道父母应该如此关心孩子,但从那时起我知道了。” 

查尔斯和孩子们在1919年10月22日正式成为瑞士国籍,之前他们一直是瑞士公民。狄拉克的父亲为人严格且专制,尽管他不赞成体罚。狄拉克与父亲的关系紧张,甚至在父亲去世后,狄拉克写道:“我现在感到更加自由,我是我自己的人。”查尔斯强迫孩子们只能用法语与他交谈,以便他们能够学会这门语言。当狄拉克发现自己无法用法语表达想法时,他选择保持沉默。
\subsubsection{教育背景}  
狄拉克首先在比肖普路小学(Bishop Road Primary School)接受教育,随后进入了全男生的商人冒险家技术学院(Merchant Venturers' Technical College,后来的科瑟姆学校,Cotham School),他的父亲曾在该校担任法语教师。该校隶属于布里斯托大学,校区和师资共享。学校强调技术学科,如砖砌、制鞋和金属加工,以及现代语言。这在当时英国的中等教育仍以古典学科为主的背景下,显得非常与众不同,狄拉克后来对此表示了感激。

狄拉克凭借布里斯托大学的奖学金在该校工程学院学习电气工程,该学院与商人冒险家技术学院共用设施。在完成学位前不久,即1921年,他参加了剑桥大学圣约翰学院的入学考试。他顺利通过并获得了70英镑的奖学金,但这笔钱不足以支付在剑桥的生活和学费。尽管他以一等荣誉学位毕业于电气工程学,但由于战后经济萧条,他未能找到工程师工作。于是,他接受了一个在布里斯托大学攻读数学文学学士学位的机会,并且免去了第一年的课程,因为他已获得工程学学位。在彼得·弗雷泽的影响下,狄拉克称他为最好的数学教师,并对投影几何学产生了浓厚兴趣,开始将其应用于明可夫斯基发展出的相对论的几何版本。

1923年,狄拉克再次以一等荣誉学位毕业,并获得了科学与工业研究部门提供的140英镑奖学金。再加上圣约翰学院提供的70英镑奖学金,这使得他能够在剑桥大学生活。在剑桥,狄拉克继续深入研究广义相对论(这一兴趣源自他在布里斯托大学时期的学习)和新兴的量子物理学领域,师从拉尔夫·福勒(Ralph Fowler)。1925至1928年,他获得了1851年皇家博览会研究奖学金。1926年6月,他完成了第一篇量子力学博士论文,这是世界上首篇提交的量子力学博士论文。随后,他在哥本哈根和哥廷根继续进行研究。1929年春季,他曾在威斯康星大学麦迪逊分校担任访问教授。
\subsubsection{家庭}
\begin{figure}[ht]
\centering
\includegraphics[width=7cm]{./figures/e3efba3af8241e4f.png}
\caption{保罗和曼西·狄拉克(Paul and Manci Dirac)于1963年7月在哥本哈根} \label{fig_DLK1_3}
\end{figure}
1937年,狄拉克与玛吉特·维格纳(Margit Wigner)结婚,玛吉特是物理学家尤金·维格纳(Eugene Wigner)的妹妹,并且是离婚妇女。狄拉克将玛吉特的两个孩子——朱迪丝和加布里埃尔——视如己出抚养。保罗和玛吉特·狄拉克还共同育有两个女儿,玛丽·伊丽莎白和弗洛伦斯·莫妮卡。

玛吉特,外号曼奇(Manci),于1934年从她的故乡匈牙利到新泽西州的普林斯顿探望哥哥,在安纳克斯餐厅的晚餐时遇见了“坐在隔壁桌的孤独男子”。这位韩国物理学家金裕成(Y. S. Kim)曾见过并受狄拉克影响,他在回忆中也提到:“对于物理学界来说,玛吉特照顾我们的尊敬的保罗·A·M·狄拉克是非常幸运的。在1939到1946年间,狄拉克发表了十一篇论文。狄拉克能够保持正常的研究生产力,完全是因为玛吉特负责了其他一切事务。”
\subsection{个性} 
狄拉克以其精确和沉默寡言的性格在同事中闻名。他在剑桥的同事们开玩笑地定义了一个单位——“狄拉克”(dirac),代表每小时说一个单词。[77] 当尼尔斯·玻尔抱怨自己写科学文章时不知道如何结束一个句子时,狄拉克回答道:“我在学校时学到的规则是,永远不要在不知道句子的结尾时开始写它。”[78] 他还批评物理学家J.罗伯特·奥本海默对诗歌的兴趣:“科学的目标是以更简单的方式让复杂的事情变得可以理解;而诗歌的目标是以无法理解的方式表达简单的事情。这两者是不相容的。”[79]  

狄拉克在研究生时期的日记中写道,他专注于自己的研究,只在星期天停下来独自散步。[80]  

一个轶事提到,1929年8月,维尔纳·海森堡和狄拉克乘坐远洋班轮前往日本参加一个会议。两人都二十多岁,未婚,性格截然不同。海森堡是个喜欢调情和跳舞的风流人物,而狄拉克——如其传记作者格雷厄姆·法梅洛(Graham Farmelo)所称,是一位“爱德华时代的怪才”——如果被迫参与任何形式的社交或闲聊,都会感到痛苦。狄拉克问海森堡:“你为什么跳舞?”海森堡回答:“当有漂亮的女孩时,这是一种乐趣。”狄拉克思索了一会儿,然后突然说:“但,海森堡,你怎么提前知道女孩是漂亮的?”[81]  

玛吉特·狄拉克在20世纪60年代告诉乔治·伽莫夫和安东·卡普里,她的丈夫曾对一位来访者说:“允许我介绍一下维格纳的妹妹,她现在是我的妻子。”[82][83]  

另一个关于狄拉克的故事是,他第一次在一次会议上见到年轻的理查德·费曼时,沉默了很久后说道:“我有一个方程。你也有一个吗?”[84]  

在一次会议上发表演讲后,一位同事举手说:“我不理解黑板右上角的那个方程。”经过一段长时间的沉默,主持人问狄拉克是否想回答这个问题,狄拉克回答道:“那不是一个问题,而是一个评论。”[85][86]  

狄拉克还以个人的谦逊著称。他将自己首次写下的关于量子力学算符时间演化的方程称为“海森堡运动方程”。大多数物理学家将描述半整数自旋粒子的统计称为费米-狄拉克统计,而整数自旋粒子的统计称为玻色-爱因斯坦统计。然而,狄拉克晚年在授课时始终坚持称前者为“费米统计”,后者为“玻色统计”,他解释说这是出于“对称性”的原因。[87]
\subsubsection{对宗教的看法}  
海森堡回忆起1927年索尔维会议期间,沃尔夫冈·泡利、海森堡和狄拉克关于爱因斯坦和普朗克宗教观点的讨论。狄拉克的发言是批评宗教的政治目的,当海森堡后来向博尔提及此事时,博尔认为他的解释非常清晰。[88] 海森堡想象狄拉克可能会说:

“我不知道为什么我们要讨论宗教。如果我们诚实——科学家必须诚实——我们必须承认,宗教是一些虚假的主张,没有任何现实基础。上帝的观念是人类想象的产物。很容易理解,为什么原始人会在对自然力量的恐惧和颤抖中将这些力量人格化。可是如今,当我们了解了许多自然过程时,我们不再需要这些解决方案。我无法理解全能的上帝假设到底能帮助我们什么。我看到的是,这种假设导致了诸如上帝为何允许如此多的苦难和不公,贫富之间的剥削,以及所有他本可以阻止的恐怖事件这样的无益问题。如果宗教依然在被传授,那并非因为它的思想仍能说服我们,而是因为某些人想要让下层阶级安静。安静的人比那些喧嚣和不满的人更容易管理。他们也更容易被剥削。宗教是一种鸦片,让一个国家沉浸在美好的幻想中,从而忘记正在对人民施加的种种不公。因此,国家和教会这两大政治力量有着密切的联系。它们都需要那种幻觉——即上帝是仁慈的,奖励所有未反抗不公、默默履行职责、不抱怨的人——无论是在天国还是在地球上。这正是为什么那种诚实的主张——上帝只是人类想象的产物——会被视为最严重的罪孽。”[89]  

海森堡的看法则较为宽容。泡利在一开始有所评论后保持沉默,但当他被要求发表看法时,说:“好吧,我们的朋友狄拉克有一种宗教,而它的指导原则是‘没有上帝,狄拉克是他的先知。’”所有人,包括狄拉克,都哄堂大笑。[90][91]

在晚年,狄拉克在1963年5月的《科学美国人》上发表了一篇提到上帝的文章,他写道:

“似乎自然界的一个基本特征是,基本的物理法则可以用一种非常美丽和强大的数学理论来描述,而要理解这些法则,需要相当高水平的数学能力。你可能会想:为什么自然是以这种方式构建的?我们只能回答:我们目前的知识表明,自然确实是以这种方式构建的。我们只能接受这一点。或许可以这样描述这种情况:上帝是一个非常高阶的数学家,他在构建宇宙时使用了非常先进的数学。我们微弱的数学尝试让我们能够理解宇宙的一部分,而随着我们发展出越来越高级的数学,我们可以期望更好地理解宇宙。”[92]  

1971年,在一次会议上,狄拉克阐述了他对上帝存在的看法。[93] 狄拉克解释说,只有在过去发生了一个极不可能的事件时,上帝的存在才可以被证明:

“或许开始生命是极其困难的。可能开始生命是如此困难,以至于它只发生在所有行星中一次……让我们考虑一下,假设生命在我们有适当物理条件时开始的机会是10^-100。我没有任何逻辑理由提出这个数字,我只是希望你们把它当作一种可能性。在这种情况下……几乎可以肯定生命不会开始。而我觉得,在这种情况下,必须假设上帝的存在才能启动生命。因此,我想要建立这样一个联系:如果物理法则是这样的,开始生命的机会极小,以至于仅凭盲目的机会生命不会开始,那么就必须有一个上帝,而这样的上帝可能会通过后来的量子跃迁来显示他的影响。另一方面,如果生命可以非常容易地开始,并且不需要任何神的影响,那么我会说没有上帝。”[94]  

狄拉克并没有对这一问题作出明确的立场,但他描述了科学上回答上帝问题的可能性。[94]
\subsection{职业生涯}
这个方程引入了一个显著的概念,即每个费米子粒子都有一个反粒子,例如正电子是电子的反粒子。这一概念来源于他的方程。他被认为是量子场论的创始人,量子场论是现代所有关于亚原子或“基本”粒子的理论工作的基础,这些工作对于我们理解自然力至关重要,同时他还创建了量子电动力学并首次提出了这一术语。[9][11] 他提出并研究了磁单极子的概念,这是一种尚未在实验证明的物体,旨在为詹姆斯·克拉克·麦克斯韦的电磁学方程带来更高的对称性。狄拉克还创造了“费米子”和“玻色子”这两个术语。[95]  

在他的职业生涯中,狄拉克受到数学美学原则的驱动,[96] 彼得·戈达德曾指出:“狄拉克将数学美学作为选择理论物理研究方向的最终标准。”[97] 狄拉克因其出色的数学才能而广受认同,在大学期间,学术界一致认为狄拉克具备“在数学物理学领域的最高水平的才能”,[98] 埃比尼泽·坎宁安曾表示:“狄拉克是我在数学物理学领域遇到的最具原创性的学生。”[99] 因此,狄拉克以其“惊人的物理直觉和发明新数学以创造新物理的能力”著称。[17] 在他的职业生涯中,狄拉克对数学领域做出了许多重要贡献,包括狄拉克δ函数、狄拉克代数和狄拉克算符。

\subsubsection{量子理论}  
狄拉克在1925年9月末迈出了进入新量子理论的第一步。 Ralph Fowler,他的研究导师,收到了一篇由维尔纳·海森堡(Werner Heisenberg)撰写的探索性论文,该论文基于老的量子理论(博尔和索末菲框架)。海森堡大量依赖博尔的对应原理,但改变了方程式,使它们直接涉及可观测的量,从而导出了量子力学的矩阵形式。福勒将海森堡的论文转交给狄拉克,要求他仔细研究这篇论文。

狄拉克被海森堡建立的一个神秘数学关系所吸引,初看时难以理解。几周后,当狄拉克回到剑桥时,他突然意识到这个数学形式与经典粒子运动的泊松括号(Poisson brackets)具有相同的结构。当时,他对泊松括号的记忆相当模糊,但他发现E. T. 惠特克(E. T. Whittaker)的《粒子与刚体的分析力学》对他有所启发。通过新的理解,狄拉克发展出了一种基于非对易动力学变量的量子理论。这使得他提出了迄今为止最深刻和最重要的量子力学的普遍公式。他通过使用狄拉克括号的创新方法,成功以一种新颖且更具启发性的方式得出了量子化规则。凭借这项工作(发表于1926年),狄拉克获得了剑桥大学的博士学位。这为费米–狄拉克统计学奠定了基础,适用于由许多相同的自旋1/2粒子(即遵循泡利不相容原理的粒子)组成的系统,如固体和液体中的电子,特别在半导体导电领域中具有重要意义。

狄拉克以不太关心量子理论解释问题而著称。事实上,在他为自己所写的纪念文章中,他写道:“量子力学的解释已经有许多作者讨论过,我不想在这里讨论这个问题。我想讨论更基础的东西。”然而,在1964年,他写了一篇关于基于海森堡图像的量子场论解释的短文;他在文中的主要观点是,薛定谔模型并不适用于此目的。