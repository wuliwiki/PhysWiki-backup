% 空间偶和带基点空间
% 不交并|空间偶|带基点空间|基点|压缩积|拓扑空间

%需要制图来demonstrate
\pentry{映射空间\upref{Topo8}}

\begin{definition}{不交并}
给定集合$A$和$B$.如果在对这两个集合取并时特别区分两个集合中的元素,强行认为它们没有交集,这样取出来的并集被称为$A$和$B$的\textbf{不交并},记为$A\amalg B$.
\end{definition}

\begin{example}{不交并的例子}
\begin{itemize}
\item 给定集合$A=\{1,2,3\}$,$B=\{2,3,4\}$,给两集合中的元素打上各自的标签以进行区分,得到$A'=\{1_A, 2_A, 3_A\}$和$B'=\{2_B, 3_B, 4_B$,那么并集$A'\cup B'=\{1_A, 2_A, 3_A, 2_B, 3_B, 4_\}=\{1, 2_A, 3_A, 2_B, 3_B, 4\}$也就是不交并$A\amalg B$.
\item 给定两个一维球(圆)$A=\{(x, y)|x^2+y^2=1\}$和$B=\{(x, y)|(x-1)^2+y^2=3\}$,那么$A\cup B$是两个粘在一起的圆,而$A\amalg B$是两个分开的圆.由于在拓扑意义上,任何两个圆都是同胚的,取不交并时也强行将其分开了,我们就可以把任何圆简单记为$S^1$再作不交并.一般地,$S^n$表示任意一个$n$维球面.
\end{itemize}
\end{example}

\begin{definition}{空间偶}
\begin{itemize}
\item 把拓扑空间$X$和它的一个子空间$A$绑定,记为$(X, A)$,称其为一个\textbf{空间偶}.
\item 如果一个连续映射$f:X\rightarrow Y$还满足$f(A)\subseteq B$,即限制在$A$上时可以认为$f:A\rightarrow B$,那么我们也称这是一个空间偶之间的映射$f:(X, A)\rightarrow(Y, B)$.
\item 空间偶的映射空间 $(Y, B)^{(X, A)}$ 是所有满足 $f(A)\subseteq B$ 的连续映射$f:X\rightarrow Y$的集合.显然这是$Y^X$的子集,因此我们定义其拓扑为$Y^X$的子拓扑.
\item 乘积空间偶$(X, A)\times(Y,B)$定义为$(X\times Y, (X\times B)\cup(A\times Y))$.特别地,$X\cong(X, \varnothing)$,所以由$X\times(Y, B)=(X\times Y, X\times B)$.


\end{itemize}
\end{definition}

\begin{definition}{带基点空间}

取拓扑空间$X$中的一个点$x_0$.称空间偶$(X, \{x_0\})=(X, x_0)$是一个\textbf{带基点空间},称$x_0$为这个带基点空间的\textbf{基点}.

\end{definition}

\begin{definition}{带基点空间的一点并和压缩积}\label{Topo9_def1}
\begin{itemize}
\item 给定带基点空间$(X, x_0)$和$(Y, y_0)$.记$(X\amalg Y)/(x_0\sim y_0)=(X, x_0)\vee(Y, y_0)=(X\vee Y, x_0\vee y_0)$,称为$(X, x_0)$和$(Y, y_0)$的一点并.换句话说,一点并就是将两个带基点空间进行不交并,然后再把基点粘在一起.

\item 记$(X\land Y, x_0\land y_0)=(X, x_0)\land(Y, y_0)=X\times Y/\sim$,其中$\sim$定义为$\forall x\in X, y\in Y$有$ (x, y_0)\sim(x_0, y)\sim(x_0, y_0)$.

\end{itemize}
\end{definition}
