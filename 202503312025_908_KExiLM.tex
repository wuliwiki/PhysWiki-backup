% 柯西-黎曼方程(综述)
% license CCBYSA3
% type Wiki

本文根据 CC-BY-SA 协议转载翻译自维基百科\href{https://en.wikipedia.org/wiki/Cauchy\%E2\%80\%93Riemann_equations}{相关文章}。

\begin{figure}[ht]
\centering
\includegraphics[width=8cm]{./figures/9b579a327747d777.png}
\caption{一个向量 \( X \) 在一个域中被复数 \( z \) 乘以后再通过函数 \( f \) 映射,和先通过函数 \( f \) 映射再乘以 \( z \) 的情况进行的视觉对比。如果这两种情况对于所有 \( X \) 和 \( z \) 都导致点最终落在相同的位置,那么函数 \( f \) 满足柯西-黎曼条件。} \label{fig_KExiLM_1}
\end{figure}
在数学的复分析领域,柯西–黎曼方程以奥古斯丁·柯西和伯恩哈德·黎曼命名,由一组二阶偏微分方程组成,这些方程为复变量的复函数可复微分的必要与充分条件。

这些方程是:
\[
\frac{\partial u}{\partial x} = \frac{\partial v}{\partial y} \quad (1a)~
\]

和
\[
\frac{\partial u}{\partial y} = -\frac{\partial v}{\partial x} \quad (1b)~
\]
其中,\(u(x, y)\)和\(v(x, y)\)是实值的二元可微函数。

通常,\(u\) 和\(v\)分别是复值函数\(f(x+iy)=f(x,y)=u(x,y)+iv(x,y)\)的实部和虚部,其中\( z = x + iy \)是单一复变量,且\( x \) 和 \( y \)是实变量;\( u \)和\( v \)是实变量的实值可微函数。如果\( f \)在某个复数点可复微分,当且仅当\( u \) 和\( v \)的偏导数在该点满足柯西–黎曼方程时,\( f \)才是复微分的。

全纯函数是指在复平面某个开子集的每一点都可微分的复函数。已经证明,全纯函数是解析的,而解析的复函数是复微分的。特别地,全纯函数是无限次可复微分的。

微分性与解析性之间的等价性是所有复分析的起点。
\subsection{历史}  
柯西–黎曼方程首次出现在让·勒·朗·达朗贝尔的研究中。\(^\text{[1]}\)后来,莱昂哈德·欧拉将这一系统与解析函数联系起来。\(^\text{[2]}\)随后,柯西使用这些方程构建了他的函数理论。\(^\text{[3]}\)黎曼关于函数理论的博士论文于1851年发表。\(^\text{[4]}\)
\subsection{简单示例}  
假设\( z = x + iy \)。复值函数\( f(z) = z^2 \)在复平面上的任意点\( z \)都是可微分的。  
\[
f(z) = (x + iy)^2 = x^2 - y^2 + 2ixy~
\]
实部\( u(x, y) \) 和虚部 \( v(x, y) \)分别为  
\[
u(x, y) = x^2 - y^2, \quad v(x, y) = 2xy~
\]
它们的偏导数是  
\[
u_x = 2x, \quad u_y = -2y, \quad v_x = 2y, \quad v_y = 2x~
\]
我们可以看到,柯西–黎曼方程确实成立,\(u_x = v_y \quad \text{和} \quad u_y = -v_x~
\]