% 浙江大学 2002 年 考研 量子力学
% license Usr
% type Note

\textbf{声明}:“该内容来源于网络公开资料,不保证真实性,如有侵权请联系管理员”

\subsection{第一题:从下面四题中任选三题(15分)}

(1)试说明光电效应实验中的“红限”现象,为何光电效应实验中有所谓截止频率的概念?

(2)如何从黑体辐射实验的Planck公式中推出 Stefan 公式?(只要求给出思路)。根据该
公式,能否做出什么测温仪器?

(3)你认为Bohr的量子理论有哪些成功之处?有哪些不成功的地方?试举一例说明

(4)你能从固体与分子的比热问题中得出哪些量子力学的概念?
\subsection{第二题(20 分):}
设氢原子处于状态:
\[\Psi(r,  0, \varphi) = \frac{1}{4} R_{21}(r) Y_{11}(0, \varphi) - \frac{\sqrt{7}}{4} R_{21}(r) Y_{10}(0, \varphi) + \frac{1}{\sqrt{2}} R_{31}(r) Y_{1-1}(0, \varphi)~\]

(1) 测量该原子的能量,测得的可能值为多少?相应的几率为多少?

(2) 测量该原子的角动量平方 $\hat{L}_z^2$ ,测得的可能值为多少?相应的几率为多少?

(3) 测得的角动量分量 $L_z$ 的可能值和相应几率为多少?
\subsection{第三题:(20分)}
一质量为 $m$ 的粒子处于势场 $V(x)$ 中运动,若

(1) 
\[V(x) = \begin{cases} \infty, & |x| > a \\\\0, & |x| \leq a \end{cases}~\]
则该粒子的本征能量为多少?

(2) 
\[V(x) = a \delta(x), \quad a < 0 \text{ 为已知常数, 则该粒子的本征能量为多少?特征长度为多少?}~\]

(3) 
\[V(x) = \begin{cases} V_0^0(x), & |x| < a \\\\\infty, & |x| \geq a \end{cases}, \quad V_0 > 0~\]
是一个给定的常数,则该粒子满足的方程为何?

(4) 
能量为 $E$ 的平行粒子束,以入射角 $\theta$ 的射向平面 $x = 0$, 在区域 $x < 0$, $V = 0$, 在区域 $x > 0$, $V = -V_0$. 试从量子力学的角度,分析粒子的反射及折射规律。(用 $\theta$ 及$n = \left( 1 + \frac{V_0}{E} \right)^{\frac{1}{2}}$
表示反射几率 $R$ 及折射几率 $D$ 。)