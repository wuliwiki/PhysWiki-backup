% 线段树
% 线段树|数据结构|C++|高级数据结构

线段树(Segment tree)是一种二叉树形的数据结构,用以存储区间或线段,并且可以在 $O(\log N)$ 的时间复杂度查询区间最大值、最小值、总和等属性.

\textbf{线段树的存储:}

线段树除了最后一层节点外是一棵满二叉树,因此可以用堆\upref{heap}的存储方式来存储线段树.
具体来说就是开一个一维数组,根节点的编号为 $1$,编号为 $x$ 的结点的左子节点的编号为 $x \times 2$,右子节点的编号为:$x \times 2 + 1$,父节点的编号为 $\left\lfloor\dfrac{x}{2}\right\rfloor$.

因此我们可以用一个结构体来存储线段树,线段树除了最后一层结点外是一棵满二叉树,除了最后一层结点外的结点个数为:$N + \dfrac{N}{2} + \dfrac{N}{4} + \cdots + 2 + 1 = 2N - 1$,最后一层的结点个数最坏情况下是 $2N$ 个结点,所以数组大小应不小于 $4N$ 才能保持不越界.

\begin{figure}[ht]
\centering
\includegraphics[width=14.25cm]{./figures/STree_1.png}
\caption{二叉树视角} \label{STree_fig1}
\end{figure}

\begin{figure}[ht]
\centering
\includegraphics[width=14.25cm]{./figures/STree_2.png}
\caption{区间视角} \label{STree_fig2}
\end{figure}


可以看出,线段树的每个结点都代表一个区间,叶结点的区间长度都为 $1$,对于每个区间结点 $[l, r]$,左子结点为 $[l, mid]$,右子结点为 $[mid + 1, r]$,$mid = \left\lfloor\dfrac{l+r}{2}\right\rfloor$.

\textbf{线段树的建树}: