% 高斯波包
% keys 高斯分布|波包|光学|量子力学
% license Xiao
% type Tutor

\pentry{波包\upref{WvPck}, 高斯分布\upref{GausPD}}

\begin{figure}[ht]
\centering
\includegraphics[width=14.25cm]{./figures/98b376107396a32e.pdf}
\caption{高斯波包(\autoref{eq_GausPk_1} ),蓝色为实部,红色为虚部, $x_0 = 0$, $A = 1$, $a = 1/20$, $k_0 = 5$。} \label{fig_GausPk_1}
\end{figure}

\footnote{参考 Wikipedia \href{https://en.wikipedia.org/wiki/Wave_packet}{相关页面}。}\textbf{高斯波包(Gaussian wave packet)}是指轮廓为高斯分布的波包, 在光学和量子力学中有重要应用。 高斯波包用复函数表示为($A$ 为复数)
\begin{equation}\label{eq_GausPk_1}
f(x) = A \E^{-a x^2}\E^{\I k_0 x}~,
\end{equation}

FWHMI (光强半高宽)是多少? 即 $f(x)^2$ 等于其峰值一半时的宽度。令
\begin{equation}
\E^{-2a\Delta x^2} = 1/2~,
\end{equation}
得
\begin{equation}
\mathrm{FWHMI} = 2\Delta x = \sqrt{\frac{2\ln 2}{a}}~.
\end{equation}

$f(x)$ 的积分为(用于求电场矢势)
\begin{equation}\label{eq_GausPk_3} % Mathematica 已验证
\int A \E^{-a x^2}\E^{\I k_0 x} \dd{x} = -\I\frac{A}{2} \sqrt{\frac{\pi}{a}} \exp(-\frac{k^2}{4a}) \erfi\qty(\frac{k + 2 \I a x}{2 \sqrt{a}}) + C~.
\end{equation}
$C$ 前面的部分在 $x = \pm \infty$ 处分别为 $\pm\frac{A}{2} \sqrt{\frac{\pi}{a}} \exp(-\frac{k^2}{4a})$。% 已数值验证

\subsection{频谱}
\pentry{傅里叶变换(指数)\upref{FTExp}}
(未完成:哪里有介绍频谱的概念?)

要求\autoref{eq_GausPk_1} 的傅里叶变换 $g(k)$, 由\autoref{ex_FTExp_1}~\upref{FTExp}以及傅里叶变换性质\autoref{eq_FTExp_4}~\upref{FTExp}和\autoref{eq_FTExp_7}~\upref{FTExp}得
\begin{equation} % Mathematica 已验证
g(k) = A\sqrt{\frac{\pi}{a}} \exp[-\frac{(k-k_0)^2}{4a}]~.
\end{equation}

\subsection{cos2 波包}
$\cos^2$ 波包也叫 $\sin^2$ 波包,比起高斯波包,它的优点是存在明确的范围。 它的函数形式为
\begin{equation}\label{eq_GausPk_5}
f(x) = \leftgroup{
&A\cos^2\qty(\frac{\pi x}{L}) \E^{\I k_0x} && (\abs{x} < L/2)\\
&0 && (\text{otherwise})~.
}\end{equation}
其中 $L$ 是波包的总长度。 它的 FWHMI 为
\begin{equation}\label{eq_GausPk_4} % 已数值验证
\text{FWHMI} = \frac{2}{\pi}\opn{acos}(2^{-1/4}) L \approx 0.3640567 L~.
\end{equation}

积分为(令 $a = \pi/L$)
\begin{equation}\label{eq_GausPk_2} % Mathematica 已验证
\begin{aligned}
\int A\cos^2\qty(a x) \E^{\I k_0x} \dd{x} &= -\I\frac{A}{4}\E^{\I kx}\qty(\frac{2}{k} +\frac{\E^{2\I ax}}{2a+k} -\frac{\E^{-2\I ax}}{2a-k}) + C\\
&= \frac{A}{4} \qty(\frac{2\sin(k x)}{k} +\frac{\sin[(2 a+k)x]}{2 a+k} +\frac{\sin[(2 a-k)x]}{2 a-k})\\
&+\I\frac{A}{4} \qty(-\frac{2\cos(k x)}{k} -\frac{\cos[(2 a+k)x]}{2 a+k} +\frac{\cos[(2 a-k)x]}{2 a-k}) + C~.
\end{aligned}
\end{equation}
$C$ 前面的部分在 $x = \pm\infty$ 处分别为 $-2\I A\frac{a^2}{k(4 a^2 -k^2)}\E^{-\I{\pi k}/{(2a)}}$。

傅里叶变换:
\begin{equation} % Mathematica 已验证
\tilde f(k) = \frac{1}{\sqrt{2\pi}} \int_{-\infty}^{+\infty} f(x) \E^{-\I k x} \dd{x}
= \frac{\sqrt{2} \pi ^{3/2} A L}{4 \pi ^2-L^2 (k-k_0)^2} \sinc[L(k-k_0)/2]~.
\end{equation}
零点的位置为
\begin{equation}
k = k_0 \pm 2n\pi/L \qquad (n=2,3,\dots)~.
\end{equation}
标准差约为 $2.92544/L$, 方差 $13.15947/L$, FWHMI $9.05144/L$。

画图对比如下(代码见文末):
\begin{figure}[ht]
\centering
\includegraphics[width=13cm]{./figures/925904dd88cd4821.pdf}
\caption{高斯波包和 cos2 波包的对比} \label{fig_GausPk_2}
\end{figure}

\subsection{附:Matlab 画图代码}
\begin{lstlisting}[language=matlab, caption=sin2\_gaussian\_compare.m]
% plot Gaussian vs cos2 profile

% gaussian
FWHMI = 1;
a = iFWHMIexp(FWHMI);
x = linspace(-2*FWHMI, 2*FWHMI, 1000);
field_gauss = exp(-a.*x.^2);

% cos2
field_cos2  = zeros(size(x));
dur_cos2 = FWHMI / FWHMIsin2;
mark = abs(x) < dur_cos2/2;
field_cos2(mark) = cos((pi/2)*x(mark)/(dur_cos2/2)).^2;

% plot field profile
figure;
subplot(2, 1, 1); hold on;
axis([min(x), max(x), 0, 1.1]);
plot_vert(-FWHMI/2, 'c--');
plot_vert(FWHMI/2, 'c--');
plot_hori(sqrt(1/2), 'c--');
plot(x, field_gauss, 'r');
plot(x, field_cos2, 'b--');
legend({'', '', '', 'Gaussian', 'cos2'});
% xlabel('t [FWHM]');
ylabel('field');
title('Gaussian vs cos2 profile (lines show FWHMI)');

% plot intensity profile
subplot(2, 1, 2); hold on;
axis([min(x), max(x), 0, 1.1]);
plot_vert(-FWHMI/2, 'c--');
plot_vert(FWHMI/2, 'c--');
plot_hori(1/2, 'c--');
plot(x, field_gauss.^2, 'r');
plot(x, field_cos2.^2, 'b--');
xlabel('t [FWHM]');
ylabel('intensity');
\end{lstlisting}
