% 戈特弗里德·莱布尼茨(综述)
% license CCBYSA3
% type Wiki

本文根据 CC-BY-SA 协议转载翻译自维基百科\href{https://en.wikipedia.org/wiki/Gottfried_Wilhelm_Leibniz}{相关文章}。

“戈特弗里德·威廉·莱布尼茨或莱布尼茨[a](1646年7月1日 [旧历6月21日] – 1716年11月14日)是一位德国博学家,活跃于数学家、哲学家、科学家和外交官等多个身份,与艾萨克·牛顿爵士一起被认为是微积分的发明者,此外还在二进制算术和统计学等数学分支做出了许多贡献。由于其在不同领域的知识和技能,以及随着工业革命的到来和专业化劳动的普及,像他这样的人变得越来越少,莱布尼茨被称为“最后的通才”。[15]他在哲学史和数学史上都是一个重要人物。他撰写了关于哲学、神学、伦理学、政治学、法律、历史、语文学、游戏、音乐和其他学科的作品。莱布尼茨还对物理学和技术做出了重大贡献,并预见了后来在概率论、生物学、医学、地质学、心理学、语言学和计算机科学中出现的概念。

此外,他在沃尔芬比特尔的赫尔佐格·奥古斯特图书馆工作时,设计了一种目录系统,为欧洲许多大型图书馆提供了指导。[16][17] 莱布尼茨在众多领域的贡献散见于各种学术期刊、成千上万封信件以及未发表的手稿中。他用多种语言写作,主要是拉丁语、法语和德语。[18][b]

作为一位哲学家,他是17世纪理性主义和唯心主义的主要代表之一。作为数学家,他的主要成就是独立于艾萨克·牛顿的同时代发展而提出了微积分的核心思想。[20] 数学家们一直偏爱莱布尼茨的符号法,因为它被认为是微积分的惯用且更为精确的表达方式。[21][22][23]

在20世纪,莱布尼茨关于连续性法则和超越同质性法则的概念通过非标准分析得到了一致的数学表述。他也是机械计算器领域的先驱。在为帕斯卡计算器添加自动乘法和除法功能的过程中,他于1685年首次描述了齿轮计算器,并发明了莱布尼茨轮,后来被用于首台大规模生产的机械计算器——算筹器中。

在哲学和神学方面,莱布尼茨以其乐观主义最为著名,即他得出的结论是,我们的世界在某种限定意义上是上帝所能创造的最好的世界,这一观点有时被其他思想家讽刺,例如伏尔泰在其讽刺小说《老实人》中就调侃了这一观点。莱布尼茨与勒内·笛卡尔和巴鲁赫·斯宾诺莎一起,是三位具有影响力的早期现代理性主义者。他的哲学还吸收了经院哲学传统的元素,尤其是假设通过从第一原理或先验定义推理,可以获得对现实的实质性知识。莱布尼茨的工作预示了现代逻辑的发展,并且仍然影响着当代分析哲学,例如采用“可能世界”这一术语来定义模态概念。
\subsection{传记}
\subsubsection{早年生活}
戈特弗里德·莱布尼茨于1646年7月1日[旧历:6月21日]出生在萨克森的莱比锡,父母是弗里德里希·莱布尼茨(1597–1652)和卡塔琳娜·施穆克(1621–1664)。[25] 他在两天后在莱比锡的圣尼古拉教堂受洗,他的教父是路德教神学家马丁·盖尔。[26] 在他六岁时,父亲去世,莱布尼茨由母亲抚养长大。[27]

莱布尼茨的父亲曾是莱比锡大学的道德哲学教授,也曾担任哲学系主任。莱布尼茨继承了父亲的私人图书馆。从七岁起,他就被允许自由使用这些书籍,正是在父亲去世后不久。他的学校教育主要集中在少量经典权威著作的学习上,但父亲的图书馆使他能够研究各种高级的哲学和神学著作——这些书籍本来他要到大学时才能接触到。[28] 父亲的图书馆大部分是用拉丁文写成的,这也帮助他在12岁时就掌握了拉丁语。13岁时,他在学校的一次特别活动中用拉丁文在一个早晨写了300行六步格诗。[29]

1661年4月,14岁的莱布尼茨进入了父亲曾任职的大学。[30][8][31] 在那里,他受到包括雅各布·托马修斯在内的导师的指导,托马修斯曾是弗里德里希的学生。莱布尼茨于1662年12月完成了哲学学士学位。1663年6月9日[旧历:5月30日],他为自己的论文《个体化原则的形而上学争论》(Disputatio Metaphysica de Principio Individui)进行了答辩,[32] 该论文探讨了个体化原则,提出了单子实在论的早期版本。1664年2月7日,莱布尼茨获得哲学硕士学位。1664年12月,他发表并为论文《从法律收集的哲学问题样本》(Specimen Quaestionum Philosophicarum ex Jure collectarum)进行了答辩,[32] 主张哲学与法律之间的理论和教学关系。经过一年的法律学习,他于1665年9月28日获得了法学学士学位。[33] 他的论文题为《论条件》(De conditionibus)。[32]

1666年初,年仅19岁的莱布尼茨写了他的第一本书《组合艺术》(De Arte Combinatoria),其第一部分也是他的哲学资格论文,并于1666年3月进行了答辩。[32][34] 《组合艺术》受拉蒙·卢尔的《大艺术》(Ars Magna)启发,包含了一个基于运动论证的几何形式的上帝存在性证明。

他的下一个目标是获得法律执业许可证和法学博士学位,这通常需要三年的学习时间。1666年,莱比锡大学拒绝了莱布尼茨的博士申请,并拒绝授予他法学博士学位,这很可能是由于他年纪尚轻。[35][36] 随后,莱布尼茨离开了莱比锡。[37]

莱布尼茨接着在阿尔特多夫大学注册,并很快提交了一篇论文,他很可能早在莱比锡时就已经在研究这篇论文。[38] 他的论文题为《法律中疑难案例的就职争论》(Disputatio Inauguralis de Casibus Perplexis in Jure)。[32] 莱布尼茨于1666年11月获得了法律执业许可证和法学博士学位。接下来,他拒绝了阿尔特多夫大学提供的教职,表示“我的想法完全朝向另一个方向”。[39]

成年后,莱布尼茨经常自称为“戈特弗里德·冯·莱布尼茨”。许多他去世后出版的作品在标题页上将他的名字标为“冯·莱布尼茨男爵G. W.”(Freiherr G. W. von Leibniz)。然而,没有发现任何当代政府的文件表明他被授予任何形式的贵族身份。[40]
\subsubsection{1666–1676}
\begin{figure}[ht]
\centering
\includegraphics[width=6cm]{./figures/f6003f86e6fa77fd.png}
\caption{戈特弗里德·威廉·莱布尼茨} \label{fig_LBNC_1}
\end{figure}
莱布尼茨的第一个职位是纽伦堡一家炼金术协会的薪酬秘书。[41] 当时他对该领域了解甚少,但表现得学识渊博。他很快遇到了约翰·克里斯蒂安·冯·博伊内堡(1622–1672),后者是美因茨选帝侯约翰·菲利普·冯·舍恩博恩的前任首席大臣。[42] 冯·博伊内堡雇佣莱布尼茨作为助手,不久之后他与选帝侯和解,并将莱布尼茨介绍给选帝侯。莱布尼茨随即将一篇法律论文献给选帝侯,希望借此获得职位。这一策略奏效了;选帝侯要求莱布尼茨协助重新起草选侯国的法律法规。[43] 1669年,莱布尼茨被任命为上诉法院的评审官。尽管冯·博伊内堡于1672年底去世,莱布尼茨仍然受雇于他的遗孀,直到1674年被解雇。[44]

冯·博伊内堡极大地提升了莱布尼茨的声誉,后者的备忘录和信件开始引起人们的关注。在莱布尼茨为选帝侯服务之后,很快就承担了一个外交角色。他以一个虚构的波兰贵族的笔名发表了一篇文章,主张(但未成功)支持德国候选人争夺波兰王位。在莱布尼茨成年期间,欧洲地缘政治的主要力量是法国国王路易十四的野心,由法国的军事和经济力量支持。同时,三十年战争使德语区的欧洲疲惫不堪,分裂且经济落后。莱布尼茨提出通过以下方式来保护德语区欧洲:将法国的注意力引向埃及,以此作为最终征服荷兰东印度群岛的跳板。作为交换,法国同意不干扰德国和荷兰。该计划获得了选帝侯的谨慎支持。1672年,法国政府邀请莱布尼茨前往巴黎讨论,[45] 但这一计划很快因法荷战争的爆发而失去意义。拿破仑1798年对埃及的失败入侵可以视为莱布尼茨计划的一种不自觉的迟来实施,而此时欧洲对东半球的殖民统治权已经从荷兰转移到了英国。

因此,莱布尼茨于1672年前往巴黎。到达后不久,他遇到了荷兰物理学家兼数学家克里斯蒂安·惠更斯,并意识到自己在数学和物理方面的知识还不够全面。在惠更斯的指导下,他开始了自学,并很快在这两个领域做出了重大贡献,包括发现了他版本的微积分。他结识了当时法国的主要哲学家尼古拉·马勒伯朗士和安托万·阿尔诺,并研究了笛卡尔和帕斯卡的著作,包括未发表和已发表的作品。[46] 他还与德国数学家埃伦弗里德·瓦尔特·冯·奇尔恩豪斯成为朋友,二人保持了终生的通信联系。
\begin{figure}[ht]
\centering
\includegraphics[width=6cm]{./figures/05d4c37c7cb9457d.png}
\caption{阶梯计算器} \label{fig_LBNC_2}
\end{figure}
当法国显然不会执行莱布尼茨的埃及计划时,选帝侯派遣他的侄子,由莱布尼茨陪同,于1673年初前往伦敦执行与英国政府相关的任务。[47] 在那里,莱布尼茨结识了亨利·奥尔登堡和约翰·柯林斯。他与皇家学会会面,展示了他自1670年起设计并建造的计算机。这台机器能够执行加、减、乘、除四种基本运算,学会很快将他列为外部会员。

当选帝侯去世的消息(1673年2月12日)传来时,这次任务突然中止。莱布尼茨迅速返回巴黎,而不是按计划返回美因茨。[48] 在同一个冬天,他的两位赞助人相继去世,这意味着莱布尼茨需要寻找新的事业基础。

在这方面,1669年布伦瑞克的约翰·弗里德里希公爵曾邀请莱布尼茨访问汉诺威,这一邀请后来被证明是命运的转折点。莱布尼茨最初拒绝了邀请,但从1671年起与公爵开始了通信。1673年,公爵向莱布尼茨提供了顾问的职位。直到两年后,当巴黎或哈布斯堡帝国宫廷中没有其他工作机会时,莱布尼茨才勉强接受了这一职位,尽管他非常享受巴黎的智力氛围。[49]

1675年,他试图以外国名誉会员的身份加入法国科学院,但由于科学院认为已有足够多的外国成员,因此未向他发出邀请。他于1676年10月离开巴黎。
\subsubsection{汉诺威家族,1676–1716}
\begin{figure}[ht]
\centering
\includegraphics[width=6cm]{./figures/bb0fe18b15719881.png}
\caption{汉诺威公共图书馆藏戈特弗里德·威廉·莱布尼茨肖像,1703年} \label{fig_LBNC_3}
\end{figure}
莱布尼茨设法推迟了前往汉诺威的时间,直到1676年底才抵达。在此之前,他又短暂前往伦敦,期间牛顿指责他提前看过自己未发表的微积分研究成果。[50] 这一指控被认为是后来的窃取微积分争议的证据,几十年后,有人指控莱布尼茨从牛顿那里窃取了微积分。在从伦敦前往汉诺威的旅途中,莱布尼茨在海牙停留,遇见了微生物发现者列文虎克。他还与斯宾诺莎进行了几天的深入讨论,后者刚刚完成了(但尚未出版)他的代表作《伦理学》。[51] 莱布尼茨拜访后不久,斯宾诺莎便去世了。

1677年,莱布尼茨应自己的请求被提升为司法枢密顾问,这一职位他保持至终生。莱布尼茨在勃伦瑞克家族的三位连续统治者手下服务,担任历史学家、政治顾问,最重要的是担任公爵图书馆的图书馆员。从此,他参与了涉及勃伦瑞克家族的各种政治、历史和神学事务,这些文件成为该时期宝贵的历史记录。

莱布尼茨开始推动一个利用风车改进哈尔茨山脉采矿作业的项目。该项目对采矿改进几乎没有成效,最终于1685年被恩斯特·奥古斯特公爵关闭。[49]

在北德接受莱布尼茨的人中为数不多,包括汉诺威选帝侯夫人索菲亚(1630–1714)、她的女儿汉诺威的索菲亚·夏洛特(1668–1705)——普鲁士的王后以及莱布尼茨公开的信徒,以及她的外孙、未来的乔治二世的妻子安斯巴赫的卡罗琳。对于这些女性来说,莱布尼茨既是通信者、顾问,也是朋友。她们对莱布尼茨的认可都超过了她们的配偶及未来的英国国王乔治一世。[52]

汉诺威的人口仅约一万人,其地方性最终让莱布尼茨感到不满。然而,成为勃伦瑞克家族的重要侍臣仍是一项殊荣,尤其是考虑到该家族在莱布尼茨的协助下声望的迅速上升。1692年,勃伦瑞克公爵成为神圣罗马帝国的世袭选帝侯。1701年的《英国王位继承法》指定选帝侯夫人索菲亚及其后代为英国王室成员,以应对国王威廉三世及其妯娌兼继任者安妮女王去世后的王位继承问题。莱布尼茨在该法案的倡议和谈判中发挥了作用,但并非总是有效。例如,他在英国匿名发表的一篇旨在促进勃伦瑞克家族利益的文章被英国议会正式谴责。

勃伦瑞克家族容忍了莱布尼茨在履行侍臣职责之外投入于知识追求的巨大精力,例如完善微积分,撰写数学、逻辑、物理和哲学的文章,并维持广泛的通信联系。他从1674年开始研究微积分;现存笔记中首次使用的证据出现在1675年。到1677年,他已拥有一套完整的体系,但直到1684年才出版。他最重要的数学论文发表在1682年至1692年间,通常刊载于他与奥托·门克于1682年创办的《学者纪事》期刊。该期刊在提升他的数学和科学声誉方面发挥了关键作用,这反过来也增强了他在外交、历史、神学和哲学领域的地位。
\begin{figure}[ht]
\centering
\includegraphics[width=7cm]{./figures/54d47600cd26d10a.png}
\caption{莱布尼茨论文的页面,收藏于波兰国家图书馆} \label{fig_LBNC_4}
\end{figure}
选帝侯恩斯特·奥古斯特委托莱布尼茨编写一部勃伦瑞克家族的历史,追溯至查理大帝时期甚至更早,希望这本书能有助于他的家族野心。从1687年到1690年,莱布尼茨在德国、奥地利和意大利广泛旅行,寻找并收集与此项目相关的档案材料。几十年过去了,历史书仍未完成;下一任选帝侯对莱布尼茨的拖延表现颇为不满。莱布尼茨从未完成这一项目,部分原因在于他在许多其他领域的大量创作,另一个原因则是他坚持编写一本基于档案资料、研究详尽且博学的书籍,而他的资助人其实只希望得到一本简短的通俗书籍,或许只是带有评注的家族世系,在三年或更短的时间内完成。他们从未知道,莱布尼茨实际上已经完成了相当一部分任务:当他为勃伦瑞克家族历史所撰写和收集的材料最终在19世纪出版时,共计三卷。

1691年,莱布尼茨被任命为下萨克森沃尔芬比特尔的赫尔佐格·奥古斯特图书馆馆长。

1708年,约翰·基尔在皇家学会的期刊上撰文,在牛顿的默许下指控莱布尼茨抄袭了牛顿的微积分。[53] 由此开始了微积分优先权之争,这场争议笼罩了莱布尼茨余生。莱布尼茨要求撤回这一指控,皇家学会随之进行了一次正式调查(牛顿是未被承认的参与者),最终维持了基尔的指控。自1900年左右以来,数学史学家们倾向于为莱布尼茨辩护,指出他与牛顿的微积分版本存在重要差异。

1712年,莱布尼茨在维也纳开始了为期两年的居住期,期间他被任命为哈布斯堡帝国宫廷顾问。1714年,安妮女王去世,根据1701年的《王位继承法》,选帝侯乔治·路易斯成为了英国国王乔治一世。尽管莱布尼茨为这一令人欣喜的事件做出了许多努力,但这并未成为他的荣耀时刻。尽管威尔士王妃安斯巴赫的卡罗琳为他求情,乔治一世仍禁止莱布尼茨前往伦敦,直到他完成至少一卷其父近30年前委托的勃伦瑞克家族历史。此外,乔治一世若将莱布尼茨纳入伦敦宫廷,被认为会对牛顿造成侮辱,因牛顿被视为微积分优先权之争的胜利者,其在英国官方圈中的地位无以复加。最终,他的挚友和捍卫者、选帝侯夫人索菲亚于1714年去世。1716年,俄国沙皇彼得大帝在北欧旅行途中经过巴特皮尔蒙特,与莱布尼茨会面。自1708年起,莱布尼茨便对俄国事务产生兴趣,并于1711年被任命为顾问。[54]
\subsubsection{死亡}
莱布尼茨于1716年在汉诺威去世。当时,他的地位如此低落,以至于乔治一世(当时恰好在汉诺威附近)和除了他私人秘书之外的任何宫廷成员都没有出席他的葬礼。尽管莱布尼茨是皇家学会和柏林科学院的终身会员,这两个机构却都未认为有必要纪念他的去世。他的墓地在50多年内没有标记。然而,他在巴黎的法国科学院得到了纪念,该院于1700年接纳他为外籍会员。悼词由丰特奈尔撰写,作于选帝侯夫人索菲亚的侄女奥尔良公爵夫人的请求之下。
\subsubsection{个人生活}
莱布尼茨终身未婚。他在50岁时向一位身份不明的女性求婚,但当对方考虑时间过长时,他改变了主意。[55] 他偶尔抱怨缺钱,但他留下的一笔可观遗产给了他姐姐的继子,这证明勃伦瑞克家族给予了他相当不错的报酬。在他的外交活动中,有时表现出一定的不择手段,这在当时的职业外交官中并不少见。在几次场合中,莱布尼茨将个人手稿倒签日期或进行修改,这些行为在微积分争议期间使他名誉受损。[56]

他性格迷人,举止得体,富有幽默感和想象力。[57] 他在欧洲各地有许多朋友和崇拜者。他被认为是一位新教徒和哲学有神论者。[58][59][60][61] 莱布尼茨终其一生都忠于三位一体基督教信仰。[62]
\subsection{哲学}
莱布尼茨的哲学思想看似零散,因为他的哲学著作主要由大量短文组成:期刊文章、死后多年才发表的手稿,以及写给通信者的信件。他只写过两部篇幅较长的哲学著作,其中仅1710年的《神义论》(*Théodicée*)是在他生前发表的。

莱布尼茨将自己的哲学起点追溯到1686年的《形而上学论辩》(\textbf{Discourse on Metaphysics}),该文是他对尼古拉·马勒伯朗士与安托万·阿尔诺之间一场持续争论的评论。这篇文章促成了他与阿尔诺的广泛通信;[63] 然而,《形而上学论辩》及这些通信直到19世纪才得以出版。1695年,莱布尼茨通过期刊文章《物质本性与交互作用的新体系》(\textbf{New System of the Nature and Communication of Substances})正式进入欧洲哲学领域。[64] 在1695年至1705年间,他撰写了《人类理解新论》(\textbf{New Essays on Human Understanding}),这是一部对约翰·洛克1690年出版的《人类理解论》的详细评论,但得知洛克于1704年去世后,莱布尼茨失去了出版的兴趣,因此《新论》直到1765年才出版。1714年完成的《单子论》(\textbf{Monadologie})由90条格言组成,在他死后出版。

莱布尼茨还撰写了一篇简短的文章《第一真理》(\textbf{Primae veritates}),该文首次由路易·库图拉于1903年发表(第518–523页),[65] 总结了他对形而上学的观点。这篇文章没有标注日期,但在1999年,研究者确定他是在1689年维也纳期间写下这篇文章的,当时正在进行的批判性版本终于出版了莱布尼茨1677年至1690年间的哲学著作。[66] 库图拉对这篇文章的解读影响了20世纪对莱布尼茨思想的研究,尤其是分析哲学家们。然而,经过对1688年前莱布尼茨哲学著作的详细研究(结合1999年批判性版本中的新增内容),默瑟(2001年)不同意库图拉的解读。

1676年,莱布尼茨与巴鲁赫·斯宾诺莎会面,阅读了斯宾诺莎的一些未发表的著作,并受到了他部分思想的影响。尽管莱布尼茨与斯宾诺莎建立了友谊,并钦佩他的卓越才智,但他对斯宾诺莎的结论感到不安,[67] 尤其是当这些结论与基督教正统信仰相矛盾时。

与笛卡尔和斯宾诺莎不同,莱布尼茨接受过大学哲学教育。他受到其莱比锡导师雅各布·托马修斯的影响,后者还指导了他的哲学学士论文。[9] 莱布尼茨还阅读了西班牙耶稣会士弗朗西斯科·苏亚雷斯的著作,苏亚雷斯在路德宗大学中也备受尊敬。尽管莱布尼茨对笛卡尔、惠更斯、牛顿和波义耳的新方法和结论深感兴趣,但他受教育时接触的传统哲学思想影响了他对这些新研究成果的看法。
\subsubsection{原则}
莱布尼茨在不同场合引用了七条基本哲学原则:[68]

\begin{itemize}
\item \textbf{同一性/矛盾律} 如果一个命题为真,则其否定为假,反之亦然。
\item \textbf{不可辨识者同一性}  
   两个不同的事物不可能具有完全相同的性质。如果实体x所拥有的每个性质都被实体y拥有,反之亦然,那么x和y是同一的;假设两个不可辨别的事物等同于将同一事物用两个名称表示。“不可辨识者同一性”在现代逻辑和哲学中经常被引用,但它引发了最激烈的争议和批评,特别是在微粒哲学和量子力学领域。其逆命题通常被称为莱布尼茨定律或“同一物不可辨性”,这一点大体上没有争议。
\item \textbf{充足理由律} 
   “任何事物的存在、任何事件的发生、任何真理的成立都必须有充分的理由。”[69]
\item \textbf{预定和谐}  
   “[每种实体的]适当本性导致了一个实体的发生与其他所有实体的发生相一致,尽管它们并未直接相互作用。”(《形而上学论辩》第十四节)一个掉落的玻璃杯破碎,因为它“知道”自己已经落地,而不是因为与地面的撞击“迫使”玻璃裂开。[70]
\item \textbf{连续性法则}  
   \textbf{Natura non facit saltus}(字面意思为“自然不跳跃”)。[71]
\item \textbf{乐观主义}  
   “上帝必然总是选择最好的。”[72]
\item \textbf{充实原则}  
   莱布尼茨认为,所有可能世界中最好的世界将实现每一种真正的可能性。他在《神义论》中论证道,这个最好的世界将包含所有可能性,我们对永恒有限的体验不足以怀疑自然的完美。[73]
\end{itemize}
莱布尼茨有时会对某一特定原则进行理性辩护,但更多时候他将其视为理所当然的。[74]
\subsubsection{单子}
\begin{figure}[ht]
\centering
\includegraphics[width=6cm]{./figures/334336c7412c6bcc.png}
\caption{《单子论》莱布尼茨手稿的一页} \label{fig_LBNC_5}
\end{figure}
\textbf{莱布尼茨在形而上学方面最著名的贡献是其单子论理论},这一理论在《单子论》(*Monadologie*)中进行了阐述。他提出宇宙由无限数量的简单实体构成,这些实体被称为单子。[75] 单子可以与勒内·笛卡尔及其他人机械哲学中的微粒相比。这些简单实体或单子是“自然界中存在的最终单位”。单子没有部分,但通过其拥有的性质而存在。这些性质会随着时间不断变化,每个单子都是独特的。单子不受时间的影响,仅受创造和消灭的支配。[76] 

单子是力量的中心;实体是力量,而空间、物质和运动仅仅是现象。莱布尼茨反对牛顿的观点,主张空间、时间和运动是完全相对的:[77] “至于我的观点,我不止一次地说过,我认为空间仅仅是相对的,就像时间一样,我认为空间是一种共存的秩序,就像时间是一种继起的秩序。”[78] 爱因斯坦自称为“莱布尼茨主义者”,他在马克斯·贾默的《空间概念》(\textbf{Concepts of Space})一书的序言中写道,莱布尼茨主义优于牛顿主义,如果当时技术工具不那么落后,莱布尼茨的思想本可以压倒牛顿的观点;约瑟夫·阿加西则认为,莱布尼茨为爱因斯坦的相对论铺平了道路。[79]

莱布尼茨对上帝的证明可以总结在《神义论》中。[80] 理性受到矛盾律和充足理由律的支配。根据这些原则,莱布尼茨得出结论:万物的第一原因是上帝。[80] 我们所见和所经历的一切都是可变的,世界的偶然性可以通过空间和时间中世界的不同排列可能性来解释。偶然的世界必须有某种必要的原因来解释其存在。莱布尼茨用一本几何书作为例子说明他的推理。如果这本书是从无限链条中的副本中复制出来的,那么一定有某种原因决定了书的内容。[81] 莱布尼茨因此得出结论:必定存在一个“单子之单子”(\textbf{monas monadum}),即上帝。

单子的本体论本质在于其不可约的简单性。与原子不同,单子不具有任何物质或空间特性。它们还因其完全的相互独立性而不同于原子,因此单子之间的相互作用只是表象。根据预定和谐原则,每个单子都遵循一套预先设定的、独属于自身的“指令”,因此单子“知道”在每个时刻应该做什么。由于这些内在的指令,每个单子就像宇宙的一面小镜子。

单子不一定是“微小的”;例如,每个人类个体都可以被视为一个单子。在这种情况下,自由意志的问题就变得复杂起来。

单子被认为解决了以下问题:  
\begin{itemize}
\item 笛卡尔体系中心灵与物质之间相互作用的困境;  
\item 斯宾诺莎体系中固有的个体化缺失问题,该体系将个体生物仅仅视为偶然的存在。
\end{itemize}
\subsubsection{神义论与乐观主义} 
《神义论》(*Theodicy*)试图通过以下观点来为世界的表面缺陷辩护:即这个世界是所有可能世界中最优的。它必须是最好的、最平衡的世界,因为这是由全能且全知的上帝所创造的,而如果上帝能够知道或者可能创造一个更好的世界,他就不会选择创造一个不完美的世界。实际上,在这个世界中可以识别出的表面缺陷必须存在于每一个可能的世界中,因为否则上帝会选择创造一个排除了这些缺陷的世界。[83]  

莱布尼茨主张,神学(宗教)和哲学的真理不能相互矛盾,因为理性与信仰都是“上帝的恩赐”,它们的冲突将暗示上帝在与自己作对。《神义论》是莱布尼茨试图将他个人的哲学体系与他对基督教教义的诠释调和的尝试。[84] 这个项目的部分动机来自于莱布尼茨的信念,这一信念也被启蒙时期许多哲学家和神学家所共享,即基督教宗教本质上是理性且启蒙的。他还相信,如果人类以正确的哲学和宗教作为指引,人性是可以完善的。此外,他认为形而上学的必然性必须有一个理性或逻辑的基础,即使这种形而上学的因果性在物理必然性(科学所识别的自然法则)的层面上显得难以解释。

在莱布尼茨看来,由于理性和信仰必须完全和谐,任何无法用理性辩护的信仰教义都必须被摒弃。莱布尼茨随后探讨了对基督教有神论的一个核心批评:[85] 如果上帝全善、全智、全能,恶又是如何进入世界的?根据莱布尼茨的回答,尽管上帝确实在智慧和力量上是无限的,但他的造物——人类——作为被造物,在智慧和意志(行动的能力)上都是有限的。这种有限性使人类在运用自由意志时容易产生错误的信念、错误的决策和无效的行为。上帝并非随意地让人类遭受痛苦和苦难;相反,他允许道德恶(罪)和物理恶(痛苦和苦难)作为形而上学恶(不完美)的必然结果,这是一种使人类能够识别并纠正其错误决策的方式,同时作为与真正善良的对比。[86]  

此外,尽管人类的行为源于上帝中最终起源的先因,并因此作为形而上学的确定性被上帝所知,但个体的自由意志是在自然法则之内得以运用的。在这种环境中,选择仅仅是偶然的必然性,由一场“奇妙的自发性”来决定,这种自发性为个体提供了从严格的宿命论中逃脱的可能性。
\subsubsection{《形而上学论辩》}
对于莱布尼茨而言,“上帝是一个绝对完美的存在。”他在第六节中描述了这种完美,认为它是某种事物最简单的形式,却有最实质的结果(第六节)。基于此,他宣称每种类型的完美“都以最高程度属于上帝”(第一节)。尽管他并未具体定义这些完美的类型,但莱布尼茨强调了证明上帝完美的关键:“如果一个人以低于他能力的完美程度行事,那么他就是在不完美地行事”,而因为上帝是完美的存在,他不可能以不完美的方式行事(第三节)。因此,上帝在世界上所作的决定必然是完美的。  

莱布尼茨安慰读者,指出由于上帝已经将一切做到最完美的程度,那些爱他的人不会受到伤害。然而,爱上帝并非易事,因为莱布尼茨认为,我们“并不倾向于希望上帝所希望的”,这是因为我们有能力改变自己的倾向(第四节)。因此,许多人以叛逆的态度行事,但莱布尼茨说,我们真正爱上帝的唯一方式是对“根据他的意志来到我们身上的一切”感到满足(第四节)。

由于上帝是“一个绝对完美的存在”(第一节),莱布尼茨认为,如果上帝以低于其能力的完美程度行事,他就是在不完美地行事(第三节)。他的推理最终得出结论:上帝以所有方式完美地创造了世界。这也影响了我们对上帝及其意志的看法。莱布尼茨指出,鉴于上帝的意志,我们必须明白上帝“是所有主人中最好的”,他知道他的善行何时成功,因此我们必须以与他的善意一致的方式行事——或尽我们所知去理解并行动(第四节)。  

在我们对上帝的看法中,莱布尼茨声明,我们不能仅因为造物主而赞美造物,否则这将玷污荣耀,并可能在这样做时错误地爱上帝。相反,我们应该因为造物主所完成的工作而赞美他(第二节)。莱布尼茨明确表示,如果我们仅仅因为上帝的意志而认为地球是好的,而不是根据某些善的标准,那么按照这种定义,如何为上帝的作为赞美他,而与之相反的作为也可能被赞美(第二节)。  

莱布尼茨接着断言,不同的原则和几何学不能仅仅来自上帝的意志,而必须遵循他的理解。[87]

莱布尼茨写道:“为什么有‘某物’而不是‘无’?充足的理由……存在于一种实体之中,这种实体是一个必要的存在,其存在的理由在于自身。”[88] 马丁·海德格尔称这个问题为“形而上学的基本问题”。[89][90]
\subsubsection{符号化思维与争端的理性解决}

莱布尼茨认为,人类的大部分推理可以归结为某种形式的计算,而这种计算能够解决许多意见分歧的问题:

“纠正我们推理的唯一方法是使其像数学家的推理那样具体化,以便我们能够一眼发现错误;当人与人之间发生争执时,我们可以简单地说:让我们计算一下,无需多言,看看谁是对的。”[91][92][93] 

莱布尼茨的理性计算器(*calculus ratiocinator*)类似于符号逻辑,可以被视为使这种计算变得可行的一种方法。莱布尼茨撰写了一些备忘录,[94] 这些备忘录如今被视为他尝试启动符号逻辑——以及他的计算器——的摸索性努力。这些著作直到卡尔·伊曼纽尔·格哈特(Carl Immanuel Gerhardt)于1859年编辑的一部分出版后才得以问世。路易·库图拉于1901年出版了另一部分选集;到那时,现代逻辑的主要发展已经由查尔斯·桑德斯·皮尔士和戈特洛布·弗雷格完成。

莱布尼茨认为符号对于人类理解至关重要。他高度重视良好符号系统的发展,并将自己在数学上的所有发现归功于此。他在微积分中的符号体系就是他在这一方面能力的一个例子。莱布尼茨对符号和符号系统的热情,以及他认为这些对于完善的逻辑和数学至关重要的信念,使他成为符号学的先驱之一。[95]

莱布尼茨将他的推测推进得更远。他将\textbf{符号}定义为任何书写的标志,并进一步将“真实符号”定义为直接代表一个想法的符号,而不仅仅是体现这一想法的词语。一些真实符号,如逻辑符号,仅仅是为了促进推理服务。在他那个时代广为人知的许多符号,包括埃及象形文字、汉字以及天文学和化学的符号,被他认为不是真实符号。[96]  

相反,他提出创造一种\textbf{通用符号系统}(\textbf{characteristica universalis}),基于一个人类思想的字母表,其中每个基本概念都用一个独特的“真实”符号表示:

“显而易见,如果我们能够找到适合清晰而精确地表达我们所有思想的符号或标记,就像算术表达数字或几何表达线条那样清晰精确,我们就能够在所有受推理支配的事物中做到算术和几何所能做到的一切。因为所有依赖推理的研究都可以通过重新排列这些符号和一种计算方法来完成。”[97]
 
复杂的思想可以通过组合代表简单思想的符号来表达。莱布尼茨意识到,素数分解的唯一性暗示了素数在通用符号系统中的核心作用,这预示了后来哥德尔编号的概念。然而,用素数为任何一组基本概念编号并没有直观或记忆上的便利性。

由于莱布尼茨在最初提出通用符号系统时还是一位数学新手,他起初并未将其设想为一种代数,而是作为一种通用语言或文字。直到1676年,他才设想出一种类似“思想代数”的体系,借鉴了传统代数及其符号表示法。这种符号系统包括逻辑演算、一部分组合数学、代数、他提出的“位置分析”(\textbf{analysis situs},即几何关系的分析)、一种通用概念语言,以及其他内容。

莱布尼茨实际所指的通用符号系统(*characteristica universalis*)和理性计算器(*calculus ratiocinator*),以及现代形式逻辑在多大程度上实现了他的设想,可能永远无法确定。[98] 莱布尼茨通过通用符号和计算进行推理的想法,令人惊叹地预示了20世纪形式系统的重大进展,例如图灵完备性(\textbf{Turing completeness}),在这种理论中,计算被用来定义等价的通用语言(参见图灵度)。
\subsubsection{形式逻辑} 
莱布尼茨被认为是亚里士多德与戈特洛布·弗雷格之间最重要的逻辑学家之一。[99] 他阐明了我们如今所称的合取、析取、否定、同一性、集合包含以及空集的基本性质。莱布尼茨逻辑的基本原则,并且可以说是他整个哲学的原则,可以归纳为两条:  
\begin{enumerate}
\item 我们的所有观念都由极少数简单观念组合而成,这些简单观念构成人类思想的字母表。  
\item 复杂观念通过这些简单观念的统一且对称的组合产生,这种组合类似于算术的乘法。 
\end{enumerate} 
20世纪初兴起的形式逻辑至少需要一元否定和在某个论域内量化的变量作为基础。

莱布尼茨在生前从未发表任何关于形式逻辑的著作;他在这一领域的大部分作品都是工作草稿。在《西方哲学史》中,伯特兰·罗素甚至声称,莱布尼茨在其未发表的著作中对逻辑的研究达到了后来200年后才实现的水平。

罗素对莱布尼茨的主要研究发现,莱布尼茨一些最令人惊叹的哲学思想和主张(例如,每个基本单子都映射整个宇宙)逻辑上源自他有意识地选择否认事物之间的关系是“真实”的。他将这种关系视为事物的(真实)性质(莱布尼茨只接受一元谓词):对他而言,“玛丽是约翰的母亲”描述的是玛丽和约翰各自的独立性质。这种观点与后来的关系逻辑(如德·摩根、皮尔士、施罗德和罗素本人提出的逻辑)形成鲜明对比,这种关系逻辑如今是谓词逻辑中的标准方法。值得注意的是,莱布尼茨还宣称,空间和时间本质上是关系性的。[100]

1690年,莱布尼茨发现了他的概念代数[101][102](在推理上等价于布尔代数)[103],以及与之相关的形而上学,这些发现如今在计算形而上学领域仍有重要意义。[104]
\subsection{数学}
尽管数学中的函数概念在他那个时代的三角函数表和对数表中已隐含存在,莱布尼茨在1692年和1694年首次明确使用了这一概念,用来表示从曲线中导出的几何概念,如横坐标、纵坐标、切线、弦和垂线(参见函数概念的历史)。[105] 到18世纪,“函数”逐渐失去了这些几何关联。莱布尼茨还是精算科学的先驱之一,他计算了终身年金的购买价格以及国家债务的清偿方法。[106]

莱布尼茨对形式逻辑的研究也与数学密切相关,这已在前述章节中讨论。关于莱布尼茨在微积分方面著作的最佳概述可见于Bos(1974年)。[107]

莱布尼茨是最早发明机械计算器的人之一,他对计算的看法是:“优秀的人花费数小时在计算的劳作上就像奴隶一样,这是不值得的。如果使用机器,这些工作可以安全地交由他人完成。”[108]
\subsubsection{线性系统}
莱布尼茨将线性方程组的系数排列成一个矩阵(即如今所称的矩阵),以寻找该系统的解(如果存在的话)。[109] 这一方法后来被称为高斯消元法。莱布尼茨奠定了行列式的基础和理论,尽管日本数学家关孝和也独立于莱布尼茨发现了行列式。[110][111] 他的著作显示,他通过余子式计算行列式。[112] 使用余子式计算行列式的方法被称为莱布尼茨公式。然而,对于较大的矩阵 \( n \) 来说,这种方法在实践中不太可行,因为它需要计算 \( n! \) 个乘积以及 \( n \) 个排列的数量。[113]  

他还通过行列式解决线性方程组问题,这一方法现在被称为克莱姆法则(Cramer's Rule)。基于行列式解决线性方程组的方法实际上是莱布尼茨于1684年发现的(而加布里尔·克莱姆在1750年发表了他的研究结果)。[111] 尽管高斯消元法需要 \( O(n^3) \) 的算术操作,线性代数教材仍然在讲解LU分解之前教授余子式展开法。[114][115]
\subsubsection{几何学}
莱布尼茨的圆周率公式表示为:
\[
1 - \frac{1}{3} + \frac{1}{5} - \frac{1}{7} + \cdots = \frac{\pi}{4}~
\]
莱布尼茨写道,圆“可以通过这一级数最简洁地表达,即分数交替相加和相减的总和”。[116] 然而,这个公式只有在使用大量项时才能获得足够的准确度。例如,需要使用10,000,000项才能将 \(\frac{\pi}{4}\) 的值精确到小数点后8位。[117]  

莱布尼茨还尝试在试图证明平行公设的过程中为直线提供一个定义。[118] 大多数数学家将直线定义为两点之间的最短距离,而莱布尼茨认为这仅仅是直线的一个性质,而不是其定义。[119]
\subsubsection{微积分}
莱布尼茨与艾萨克·牛顿一起被认为是微积分(包括微分与积分)的发现者。据莱布尼茨的笔记本记载,一次关键性突破发生在1675年11月11日,他首次使用积分计算找到函数 \( y = f(x) \) 图形下的面积。[120] 他引入了几种至今仍在使用的符号,例如积分符号 \( \int \)(如 \( \int f(x) \, dx \)),其形状是拉长的 S,来源于拉丁词 *summa*(总和),以及微分符号 \( d \)(如 \( \frac{dy}{dx} \)),来源于拉丁词 *differentia*(差异)。莱布尼茨直到1684年才发表关于微积分的任何内容。[121]  

莱布尼茨在其1693年的论文《几何量度学补遗》(*Supplementum geometriae dimensoriae*)中,通过图形表示了积分与微分的反关系,这后来被称为微积分基本定理。[123] 然而,詹姆斯·格里高利在几何形式中首次提出了这一定理,艾萨克·巴罗证明了一个更广义的几何版本,而牛顿则发展了支撑这一理论的相关概念。通过莱布尼茨的形式化和新符号的引入,这一概念变得更加透明和易于理解。[124]  

微分法中的乘积法则至今仍被称为“莱布尼茨法则”。此外,积分符号下的微分规则,即关于何时以及如何在积分符号下进行微分的定理,被称为“莱布尼茨积分法则”。

莱布尼茨在发展微积分时利用了无穷小量,并以某种方式操作这些量,暗示它们具有悖论的代数性质。乔治·贝克莱在其著作《分析家》(*The Analyst*)和《论运动》(*De Motu*)中对此提出了批评。然而,最近的一项研究表明,莱布尼茨的微积分并无矛盾之处,并且其理论基础比贝克莱的经验主义批评更加扎实。[125]  

从1711年到去世,莱布尼茨与约翰·基尔、牛顿及其他人陷入争议,争论莱布尼茨是否独立于牛顿发明了微积分。

卡尔·魏尔施特拉斯的追随者对数学中使用无穷小量持批评态度,[126] 但无穷小量通过称为微分的基本计算工具,在科学、工程以及严格的数学中得以保留。从1960年开始,亚伯拉罕·罗宾逊在超实数领域的模型论背景下,为莱布尼茨的无穷小量建立了严密的理论基础。由此产生的非标准分析可以被视为对莱布尼茨数学推理的迟来辩护。罗宾逊的转移原理(*transfer principle*)是莱布尼茨连续性启发法则的数学实现,而标准部分函数(*standard part function*)实现了莱布尼茨的超越同质性法则。
\subsubsection{拓扑学}
莱布尼茨是第一个使用“位置分析”(*analysis situs*)这一术语的人,[127] 该术语在19世纪被用来指代如今被称为拓扑学的领域。关于这一术语的意义,有两种不同的观点:

一方面,Mates引用雅各布·弗罗因登塔尔1954年的一篇德文论文,认为:

“尽管对于莱布尼茨来说,一组点的‘位置’完全由它们之间的距离决定,如果这些距离发生改变,位置也会随之改变,但他的崇拜者欧拉在1736年解决柯尼斯堡七桥问题及其推广的著名论文中,以一种位置在拓扑变形下保持不变的意义使用了‘几何位置’(*geometria situs*)这一术语。他错误地认为莱布尼茨是这一概念的起源者。……然而,有时人们没有意识到,莱布尼茨使用这一术语的意义完全不同,因此他几乎不能被视为这一数学领域的创始人。”[128]

但Hideaki Hirano则提出了不同的观点,他引用曼德尔布罗特的说法:[129]

“阅读莱布尼茨的科学作品是一次引人深思的体验。除了微积分以及其他已经完成的思想外,他提出的大量预示性的观点种类繁多、数量惊人。我们在‘打包问题’中看到了例子……我对莱布尼茨的痴迷进一步加深,因为发现他一度重视几何尺度。在《欧几里得的原理》中……他试图强化欧几里得的公理,其中提到:‘我对直线有不同的定义。直线是一条曲线,其任何部分与整体相似,并且它独有这一特性,不仅在曲线中如此,在所有集合中也是如此。’这一主张今天可以得到证明。”[130]

因此,曼德尔布罗特所倡导的分形几何借鉴了莱布尼茨关于\textbf{自相似性}和\textbf{连续性原则}(\textbf{Natura non facit saltus},“自然不跳跃”)的思想。[71] 我们还可以看到,当莱布尼茨从形而上学的角度写道“直线是一条曲线,其任何部分与整体相似”时,他实际上预示了拓扑学的发展,时间早于这一领域诞生两个多世纪。  

至于“打包”问题,莱布尼茨告诉他的朋友兼通信者德·博斯,想象一个圆,然后在其中内接三个具有最大半径的全等圆;通过相同的方法,这三个较小的圆中还可以填入三个更小的圆。这一过程可以无限继续,从而产生了对\textbf{自相似性}的良好理解。莱布尼茨对欧几里得公理的改进也包含了相同的概念。
\subsection{科学与工程}  
莱布尼茨的著作如今不仅因其预见性和尚未被完全认识的可能发现而受到讨论,还被视为推动当今知识进步的一种途径。他关于物理学的大部分著作收录在格哈特编辑的《数学著作》中。
\subsubsection{物理学}  
莱布尼茨对当时兴起的静力学和动力学做出了不少贡献,且经常与笛卡尔和牛顿的观点不一致。他基于动能和势能提出了一种新的运动理论(动力学),认为空间是相对的,而牛顿坚信空间是绝对的。莱布尼茨成熟的物理思想的一个重要例子是1695年的《动力学范例》(*Specimen Dynamicum*)。[131]  

直到亚原子粒子和量子力学被发现之前,莱布尼茨关于那些无法简化为静力学和动力学的自然现象的许多推测性观点显得难以理解。例如,他预见了阿尔伯特·爱因斯坦的观点,反对牛顿的绝对观念,主张空间、时间和运动是相对的:“就我的观点而言,我不止一次地说过,我认为空间仅仅是相对的,就像时间一样,我认为空间是一种共存的秩序,而时间是一种继起的秩序。”[78]  

莱布尼茨持有一种空间和时间的关系性观念,反对牛顿的实体主义观点。[132][133][134] 根据牛顿的实体主义,空间和时间本身就是独立于事物而存在的实体。而莱布尼茨的关系主义则描述空间和时间为物体之间关系的系统。广义相对论的兴起以及后来的物理学史研究使莱布尼茨的观点受到了更为积极的评价。

莱布尼茨的一个项目是将牛顿的理论重新表述为涡旋理论。[135] 然而,他的研究超越了涡旋理论,因为其核心是试图解释物理学中最困难的问题之一,即物质凝聚力的起源。[135]  

在现代宇宙学中,充足理由律被引用,而他的不可辨识者同一性原则则在量子力学中得到了体现,甚至有人认为他在某种意义上预见了量子力学。除了关于现实本质的理论外,莱布尼茨对微积分发展的贡献也对物理学产生了重大影响。

\textbf{活力(Vis Viva)}

莱布尼茨的“活力”(\textbf{vis viva},拉丁语意为“活力”)表达式为 \( mv^2 \),是现代动能的两倍。他意识到,在某些机械系统中,总能量是守恒的,因此他将其视为物质的一种内在动力特性。[136] 这一思想也引发了一场遗憾的民族主义争论。莱布尼茨的“活力”被认为与英国牛顿、法国笛卡尔和伏尔泰倡导的动量守恒相竞争;因此,这些国家的学者往往忽视莱布尼茨的观点。莱布尼茨知道动量守恒的有效性。实际上,在封闭系统中,能量和动量都是守恒的,因此这两种方法都是正确的。
\subsubsection{其他自然科学}
通过提出地球具有熔融核心的观点,莱布尼茨预见了现代地质学的发展。在胚胎学中,他是一个“前成论”支持者,但同时也提出,生物体是无数可能的微观结构及其能力相结合的结果。在生命科学和古生物学中,他展现了令人惊叹的转化论直觉,这种直觉受到他对比较解剖学和化石研究的启发。他在这一主题上的主要著作之一《地球起源》(*Protogaea*),在他生前未出版,最近首次被翻译成英文。他还提出了一种原始有机体理论。[137]  

在医学领域,他敦促当时的医生(并取得了一些成果)将其理论建立在详细的比较观察和经验证实验的基础之上,并要求严格区分科学观点与形而上学观点。
\subsubsection{心理学}
心理学一直是莱布尼茨的核心兴趣之一。[138][139] 他被认为是一个“未被充分认可的心理学先驱”。[140] 他撰写了许多如今被视为心理学领域的主题,例如注意力与意识、记忆、学习(联想)、动机(即“追求”的行为)、个体性的产生、发展的总体动力学(进化心理学)。他在《人类理解新论》和《单子论》中的讨论经常基于日常观察,如狗的行为或海浪的声音,并且他发展了直观的类比(如钟表的同步运行或钟表的平衡弹簧)。  

他还提出了适用于心理学的公设和原则:从未被察觉的*微小知觉*(*petites perceptions*)到清晰的、自我意识的*领悟知觉*(*apperception*)的连续体,以及从因果性和目的性角度讨论的心理物理平行论:“灵魂根据最终原因的规律行动,通过愿望、目标和手段;身体根据效率原因的规律行动,即运动规律。这两个领域——效率原因领域和最终原因领域——彼此协调一致。”[141] 这一思想涉及心身问题,指出心灵与大脑并不相互作用,而是各自独立行动,同时和谐一致。[142] 然而,莱布尼茨并未使用“心理学”(*psychologia*)一词。[143]  

莱布尼茨的认识论立场——反对约翰·洛克及英国经验主义(感官主义)——非常明确:“*Nihil est in intellectu quod non fuerit in sensu, nisi intellectu ipse.*”——“没有任何东西在智力中存在,而没有首先在感官中存在,除非是智力本身。”[144] 在人类的知觉和意识中,可以识别出并不源于感官印象的原则:逻辑推理、思维范畴、因果性原则和目的性原则(目的论)。

莱布尼茨的思想在心理学领域找到了最重要的阐释者——心理学作为一门学科的创始人威廉·冯特。冯特在1862年出版的《感觉知觉理论贡献》(*Beiträge zur Theorie der Sinneswahrnehmung*)的扉页上引用了莱布尼茨的名言“... *nisi intellectu ipse*”(“除非是智力本身”),并撰写了一部关于莱布尼茨的详尽而富有抱负的专著。[145]  

冯特将莱布尼茨提出的“领悟知觉”(*apperception*)概念发展为一种实验心理学基础上的领悟知觉心理学,这一研究包含了神经心理学的建模——这是一个哲学家的概念如何激发心理学研究计划的典范。莱布尼茨思想中的一个原则在其中发挥了重要作用:“独立但对应的视角平等原则。”冯特将这种思维方式(观点主义,*perspectivism*)描述为“彼此补充的视角,同时也可能表现为相反的观点,只有在更深的考虑中才能得到解决。”[146][147]  

莱布尼茨的许多研究对心理学领域产生了深远影响。[148] 他认为存在许多“微小知觉”(*petites perceptions*),即我们感知到但却未意识到的小知觉。他相信,根据现象在自然中默认是连续的原则,从有意识状态到无意识状态之间很可能存在中间步骤。[149] 如果这一理论成立,那么我们在任何时候都会有一部分思维是我们未察觉的。他关于意识与连续性原则的理论可以被视为关于睡眠阶段的早期理论。由此,莱布尼茨的知觉理论可以被看作通向无意识思想概念的众多理论之一。  

莱布尼茨直接影响了恩斯特·普拉特纳(Ernst Platner),后者最早提出了“无意识”(*Unbewußtseyn*)这一术语。[150] 此外,潜意识刺激(*subliminal stimuli*)的概念也可以追溯到他关于小知觉的理论。[151] 莱布尼茨关于音乐和音调知觉的观点进一步影响了冯特的实验室研究。[152]
\subsubsection{社会科学}
在公共卫生领域,莱布尼茨主张建立一个拥有流行病学和兽医学管理权的医疗行政机构。他致力于建立一个以公共卫生和预防措施为导向的系统化医学培训项目。在经济政策方面,他提出了税收改革和国家保险计划,并探讨了贸易平衡问题。他甚至提出了一些类似于后来出现的博弈论的概念。在社会学方面,他为传播理论奠定了基础。
\subsubsection{技术}
1906年,加兰出版了一卷关于莱布尼茨众多实用发明和工程工作的著作。迄今为止,其中很少有被翻译成英文的。然而,人们普遍认为,莱布尼茨是一位严肃的发明家、工程师和应用科学家,对实践生活充满敬意。他遵循“理论与实践结合”(\textbf{theoria cum praxi})的座右铭,主张将理论与实际应用相结合,因此被誉为应用科学之父。

他设计了风力驱动的螺旋桨和水泵、用于提取矿石的采矿机器、液压机、灯具、潜水艇、时钟等设备。他还与丹尼斯·帕平共同研制了一台蒸汽机。他甚至提出了一种淡化水的方法。从1680年到1685年,他致力于克服困扰哈尔茨山脉公爵银矿的长期洪灾问题,但未能成功。[153]

\textbf{计算}

莱布尼茨或许是最早的计算机科学家和信息理论家。[154] 他在早年间记录了二进制数字系统(以2为基数),并在整个职业生涯中不断研究这一系统。[155] 在比较自己的形而上学观点与其他文化时,莱布尼茨接触到了中国古籍《易经》。他将其中的阴阳图解释为与0和1相对应。[156] 更多信息可见于中学研究(Sinophology)部分。莱布尼茨与胡安·卡拉穆埃尔·洛博科维茨(Juan Caramuel y Lobkowitz)和托马斯·哈里奥特(Thomas Harriot)有相似之处,这两人独立开发了二进制系统,而莱布尼茨熟悉他们在二进制系统方面的研究。[157] 胡安·卡拉穆埃尔·洛博科维茨广泛研究了对数,包括以2为基数的对数。[158] 托马斯·哈里奥特的手稿中包含二进制数字及其符号的表格,展示了任何数字都可以用二进制系统表示。[159]  

尽管如此,莱布尼茨简化了二进制系统,并明确了逻辑性质,如合取、析取、否定、同一性、包含关系和空集。[160] 他预见了拉格朗日插值法和算法信息论。他的理性计算器(*Calculus Ratiocinator*)预示了通用图灵机的一些特性。1961年,诺伯特·维纳建议将莱布尼茨视为控制论的守护圣人。[161] 维纳曾说:“实际上,计算机的总体理念无非是对莱布尼茨理性计算器的机械化。”[162]

1671年,莱布尼茨开始发明一种可以执行加、减、乘、除四则运算的机器,并在接下来的数年中逐步改进。这台被称为“阶梯计算器”(*stepped reckoner*)的机器引起了相当大的关注,并成为他1673年被选为皇家学会会员的基础。在他任职汉诺威期间,一位工匠在他的监督下制造了几台这样的机器。然而,这些机器并未完全成功,因为它们未能完全实现进位操作的机械化。库图拉报告发现了一份莱布尼茨1674年写的未发表笔记,其中描述了一台能够执行某些代数运算的机器。[163]  

莱布尼茨还设计了一种密码机,该机已被复原,由尼古拉斯·雷舍于2010年重新发现。[164] 在1693年,莱布尼茨描述了一种机器的设计,该机器理论上可以用于求解微分方程,他将其称为“积分仪”(\textbf{integraph})。[165]

莱布尼茨在探索硬件和软件概念,这些概念后来由查尔斯·巴贝奇和艾达·洛芙莱斯发展得更加完善。1679年,在思考他的二进制算术时,莱布尼茨设想了一台机器,其中二进制数字用弹珠表示,并由一种初级形式的打孔卡片控制。[166][167] 现代电子数字计算机用移位寄存器、电压梯度和电子脉冲代替了莱布尼茨设想的依靠重力移动的弹珠,但在其他方面,它们的运行方式大致与莱布尼茨在1679年设想的相同。
\subsubsection{图书馆员}
在莱布尼茨职业生涯的后期(冯·博伊内堡去世后),他移居巴黎,并接受了不伦瑞克-吕讷堡公爵约翰·弗里德里希的汉诺威宫廷图书馆馆员职位。[168] 莱布尼茨的前任托比亚斯·弗莱舍已为公爵图书馆创建了一套编目系统,但这一尝试显得笨拙。在该图书馆工作时,莱布尼茨更多地关注于推动图书馆的发展,而不是仅仅专注于编目。例如,他在接任该职位后一个月内,就制定了一项扩大图书馆的全面计划。他是最早考虑为图书馆发展核心馆藏的人之一,并认为“用来炫耀的图书馆是一种奢侈品,甚至是多余的,而一个藏书丰富且组织良好的图书馆对人类各领域的活动都至关重要,应当被视为与学校和教堂同等重要。”[169]  

然而,莱布尼茨缺乏以这种方式发展图书馆的资金。在这家图书馆工作之后,到1690年底,莱布尼茨被任命为沃尔芬比特尔奥古斯特图书馆(Bibliotheca Augusta)的枢密顾问和图书馆馆长。这是一家藏书丰富的图书馆,至少有25,946册印刷书籍。[169] 在该图书馆工作期间,莱布尼茨致力于改进图书目录。他未被允许对现有的封闭目录进行彻底变更,但可以对其进行改进,因此他立即开始了这项任务。他创建了按作者字母顺序排列的目录,还设计了其他未被实施的编目方法。

在汉诺威和沃尔芬比特尔的公爵图书馆担任馆长期间,莱布尼茨实际上成为了图书馆学的奠基者之一。他显然非常重视主题分类,倾向于构建涵盖众多主题和兴趣的平衡馆藏。例如,他在《汉诺威闲暇集》(*Otivm Hanoveranvm Sive Miscellanea*,1737年)中提出了以下分类系统:[170][171]
\begin{itemize}
\item 神学  
\item 法学  
\item 医学 
\item 理性哲学  
\item 想象哲学或数学  
\item 感性哲学或物理学  
\item 语言学或语言  
\item 政治历史  
\item 文学历史与图书馆学  
\item 综合与杂项
\end{itemize}
他还设计了一种图书索引系统,但当时他并不知道当时唯一存在的类似系统是牛津大学博德利图书馆的系统。他还呼吁出版商以标准格式分发每年出版的新书摘要,以便于索引。他希望这一摘要项目最终涵盖从他那个时代一直追溯到古腾堡的所有印刷品。这些提议在当时并未取得成功,但类似的做法在20世纪成为英语出版界的标准惯例,由美国国会图书馆和英国图书馆推动。[需要引用]  

他提议创建一个经验数据库,以推动所有科学的发展。他的通用符号系统(*characteristica universalis*)、理性计算器(*calculus ratiocinator*)以及“思想共同体”——旨在促进欧洲的政治和宗教统一——可以被视为对人工语言(如世界语及其竞争语言)、符号逻辑甚至万维网的无意中的遥远预见。
\subsubsection{科学学会的倡导者}
莱布尼茨强调研究是一项协作性的事业。因此,他热情提倡按照英国皇家学会和法国皇家科学院的模式建立国家科学学会。他在通信和旅行中特别推动在德累斯顿、圣彼得堡、维也纳和柏林创建这样的学会。其中只有一个项目取得了成果:1700年,柏林科学院成立。莱布尼茨起草了该学会的首批章程,并在余生中担任其首任会长。该学会后来发展成为德国科学院,并负责出版他作品的持续批判性版本。[172]
\subsection{法律与道德}

莱布尼茨关于法律、伦理和政治的著作[173] 长期以来未被英语学者重视,但这一状况最近有所改变。[174]

虽然莱布尼茨并不像霍布斯那样为绝对君主制辩护,也不支持任何形式的暴政,但他也没有赞同其同时代人约翰·洛克的政治和宪政观点。洛克的观点在18世纪的美国以及后来的其他地区被用来支持自由主义。以下是莱布尼茨1695年致巴伦·J.C.博伊内堡之子菲利普的一封信中透露其政治观点的摘录:

“至于……君主权力及其臣民对其所负的服从义务这一重大问题,我通常会说,让君主相信其人民有权反抗他们,而让人民反过来相信对君主的被动服从,这将是有益的。然而,我完全赞同格劳秀斯的意见,即通常情况下应该服从,因为革命的危害远远大于导致革命的那些弊端。然而,我也承认,一个君主可能会达到如此极端的地步,将国家的福祉置于如此危险的境地,以至于忍受的义务不再存在。然而,这种情况极为罕见,而在这种借口下授权使用暴力的神学家应当小心避免过度;因为过度比不足危险得多。”[175]

1677年,莱布尼茨呼吁建立一个由欧洲国家组成的联邦,由一个代表各国的议会或参议院治理,其成员可以自由投票表达他们的良知。[176] 这一提议有时被视为对欧盟的预见。他还认为,欧洲将采用一种统一的宗教。他在1715年重申了这些建议。

同时,莱布尼茨提出了一个跨宗教和多元文化的项目,旨在创建一个普遍的司法体系。这一项目需要他采取广泛的跨学科视角。为此,他结合了语言学(尤其是汉学)、道德与法律哲学、管理学、经济学和政治学。[177]
\subsubsection{法律}
莱布尼茨接受过法律学术训练,但在笛卡尔主义支持者埃哈德·魏格尔(Erhard Weigel)的指导下,他已经尝试通过理性主义的数学方法解决法律问题(魏格尔的影响在《从法学中收集的哲学问题范例》(*Specimen Quaestionum Philosophicarum ex Jure collectarum*)中表现得最为明显)。例如,《困惑案件的就职论辩》(*Inaugural Disputation on Perplexing Cases*)使用了早期的组合学方法来解决一些法律争端,而1666年的《组合艺术论》(*Dissertation on the Combinatorial Art*)通过简单的法律问题进行了说明。[178][179]  

使用组合方法解决法律和道德问题似乎受到阿塔那修斯·基彻(Athanasius Kircher)和丹尼尔·施文特(Daniel Schwenter)的影响,可以追溯到雷蒙·吕尔(Ramón Llull)的灵感:吕尔试图通过他认为是普遍的组合推理模式(*mathesis universalis*)来解决宗教争议。[180]  

1660年代晚期,美因茨开明的亲王主教约翰·菲利普·冯·舍恩博恩宣布对法律体系进行审查,并提供了一个职位以支持其现任法律专员。莱布尼茨尚未获得这一职位,就离开弗兰肯前往美因茨。在抵达法兰克福时,莱布尼茨写下了《教授和学习法律的新方法》(*The New Method of Teaching and Learning the Law*)作为申请的一部分。[181] 这篇文章提出了法律教育的改革,具有典型的综合性,融合了托马斯主义、霍布斯主义、笛卡尔主义以及传统法学的各个方面。莱布尼茨主张,法律教学的功能不是像训练狗一样灌输规则,而是帮助学生发现自己的公共理性。这一观点显然打动了冯·舍恩博恩,使莱布尼茨成功获得了该职位。

莱布尼茨下一次重要的尝试是在法律中找到普遍的理性核心,从而奠定一种法律的“正义科学”(*science of right*),[182] 这是他在1667年至1672年间于美因茨工作期间进行的。最初,他从霍布斯的机械主义权力学说出发,但后来重新转向逻辑-组合方法,试图定义正义。[183] 随着莱布尼茨所谓的《自然法原理》(*Elementa Juris Naturalis*)的推进,他引入了关于权利(可能性)和义务(必然性)的模态概念,这或许是他在规范性框架内最早阐述可能世界理论的地方。[184] 尽管《自然法原理》最终未发表,莱布尼茨直到去世仍在不断完善其草稿,并向通信者推广这些思想。
\subsubsection{普世主义}
莱布尼茨投入了大量的智力和外交努力,致力于如今被称为**普世主义的事业**,即寻求调和罗马天主教会和路德宗教会。在这方面,他追随了早期资助者冯·博伊内堡男爵和约翰·弗里德里希公爵的榜样——他们都是出生时为路德宗信徒,成年后改信天主教,并尽其所能推动两大信仰的统一,并热情支持他人进行的类似努力。(勃伦瑞克家族保持了路德宗信仰,因为公爵的子女并未追随其父亲的信仰转换。)  

这些努力包括与法国主教雅克-贝尼涅·博叙埃(Jacques-Bénigne Bossuet)的通信,也使莱布尼茨卷入了一些神学争议。他显然认为,彻底应用理性足以弥合宗教改革所造成的裂痕。
\subsection{语言学}
作为语言学家的莱布尼茨对语言充满热情,他热衷于研究任何有关词汇和语法的信息。1710年,他在一篇短文中将渐变论和均变论的思想应用于语言学。[185] 他驳斥了当时基督教学者普遍持有的“希伯来语是人类原始语言”的观点。同时,他也否认了语言之间毫无关联的理论,并认为所有语言都有一个共同的来源。[186] 他还反驳了当时瑞典学者提出的“原始瑞典语是日耳曼语言祖先”的主张。他对斯拉夫语系的起源感到困惑,并对古典汉语充满兴趣。莱布尼茨还是梵语领域的专家。[187]

他出版了中世纪晚期《霍尔施泰因编年史》(\textbf{Chronicon Holtzatiae})的\textbf{初版}(现代意义上的首次出版),这是一部关于霍尔施泰因伯国的拉丁语编年史。

\begin{figure}[ht]
\centering
\includegraphics[width=6cm]{./figures/4c7bfd52f3dce971.png}
\caption{一幅《易经》六十四卦的图示,由若望·布韦(Joachim Bouvet)寄给莱布尼茨。阿拉伯数字是由莱布尼茨添加的。[188]} \label{fig_LBNC_6}
\end{figure}