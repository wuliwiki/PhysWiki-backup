% 亥姆霍兹定理
% keys 散度|旋度|调和场|亥姆霍兹

\begin{issues}
\issueDraft
\end{issues}

参考 \cite{GriffE}

任何矢量场都可以分解为一个无旋场 $\bvec F_{div}$ 和一个无散场 $\bvec F_{curl}$, 也可以选择性地添加一个调和场 $\bvec h$.
\begin{equation}
\bvec F(\bvec r) = \bvec F_{div}(\bvec r) + \bvec F_{curl}(\bvec r) + \bvec h(\bvec r)
\end{equation}
无旋场总能表示为某个势能函数的梯度, 而无散场总能表示为另一个场的旋度(为何?), 所以
\begin{equation}
\bvec F_{div} = \grad V
\end{equation}
\begin{equation}
\bvec F_{curl} = \curl \bvec A
\end{equation}
调和场既没有散度也没有旋度, 所以可以合并到前两项中任意一个中. 所以亥姆霍兹分解可以记为
\begin{equation}
\bvec F = \grad V + \curl \bvec A
\end{equation}

但具体如何分解呢? 一种方法是
\begin{equation}
\bvec F_{div} = \int \frac{(\div \bvec F)\bvec R}{4\pi R^3} \dd{V'}
\end{equation}
\begin{equation}
\bvec F_{curl} = \int \frac{(\curl \bvec F)\cross \bvec R}{4\pi R^3} \dd{V'}
\end{equation}
类比静电静磁学中的库伦定律和比奥萨法尔定律, 他们显然满足无散无旋的条件. 但是还需要证明 $\bvec F - \bvec F_{div} - \bvec F_{curl}$ 是一个调和场.
(未完成)
