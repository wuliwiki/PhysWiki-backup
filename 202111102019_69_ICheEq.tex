% 理想气体化学平衡条件
% keys 化学反应|化学平衡条件|理想气体

\pentry{吉布斯自由能\upref{GibbsG}}

\addTODO{未完成}

设在一组化学反应中,$i$ 组元的物质的量变动为 $\delta n_i$,它们之间满足一定的比例关系,简写为以下方程:
\begin{equation}
\delta n_i=\nu_i \delta n
\end{equation}

$\nu_i>0$ 表示生成物,$\nu_i<0$ 表示反应物.这一般用来区分化学方程式的左侧和右侧.

在等温等压条件下,化学平衡要求平衡态的吉布斯函数最小,即\textbf{吉布斯判据}(\autoref{GibbsG_eq2}~\upref{GibbsG}).
\begin{equation}
\delta G=\sum_i \mu_i\delta n_i=\delta n\sum_i \nu_i\mu_i
\end{equation}

于是我们得到了化学平衡条件.
\begin{equation}
\sum_i\nu_i\mu_i=0
\end{equation}

要注意的是,因为化学反应的系统是多元系\upref{mulTh},这里组元 $i$ 的化学势 $\mu_i$ 是偏摩尔吉布斯函数,它是温度、压强以及各组员摩尔分数的函数.

\subsection{理想气体化学平衡}