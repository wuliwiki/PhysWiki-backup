% 磁场中闭合电流的合力
% 安培力|磁场|回路|曲线积分

% 未完成: 感觉这个推导太复杂了, 如果用匀强磁场 
\pentry{安培力\upref{FAmp}, 斯托克斯定理\upref{Stokes}, 静磁场的高斯定律\upref{MagGau}, 矢量算符运算法则\upref{VopEq}}

假设空间中有任意磁场 $\bvec B(\bvec r)$, 无限细的闭合电流回路 $L$ 中有电流 $I$, 则其受到的安培力可以用线积分表示为 $\bvec F = \oint I \dd{\bvec l} \cross \bvec B$。 积分方向为电流方向。 若磁场是匀强磁场, 则立即得到 $\bvec F = I(\oint \dd{\bvec l}) \cross \bvec B = \bvec 0$
若磁场是任意的, 那么
\begin{equation}\ali{
\bvec F &= \oint_L I \dd{\bvec l} \cross \bvec B
= \uvec x I\oint_L \dd{\bvec l} \cross \bvec B  \vdot \uvec x + \uvec y I\oint_L \dd{\bvec l} \cross \bvec B \vdot \uvec y + \uvec z I\oint_L \dd{\bvec l} \cross \bvec B  \vdot \uvec z\\
&= \uvec x I\oint_L (\bvec B \cross \uvec x) \vdot \dd{\bvec l}  + \uvec y I\oint_L (\bvec B \cross \uvec y) \vdot \dd{\bvec l}  + \uvec z I\oint_L (\bvec B \cross \uvec z) \vdot \dd{\bvec l}\\
&= \uvec x I \int_\Sigma  \curl (\bvec B \cross \uvec x) \vdot \dd{\bvec s}  + \uvec y I\int_\Sigma  \curl (\bvec B \cross \uvec y) \vdot \dd{\bvec s}  + \uvec z I\int_\Sigma  \curl (\bvec B \cross \uvec z) \vdot \dd{\bvec s}~.
}\end{equation}
其中用到了斯托克斯定理, $\Sigma $ 是以闭合曲线 $L$ 为边界的曲面。上式中
\begin{equation}
\curl (\bvec B \cross \uvec x) = \bvec B (\div \uvec x) + (\uvec x\vdot\bvec\nabla )\bvec B - \uvec x (\div \bvec B) - (\bvec B \vdot\bvec\nabla)\uvec x = (\uvec x\vdot\bvec\nabla)\bvec B = \pdv{\bvec B}{x}~,
\end{equation} 
这里用到了 $\uvec x$ 的任意微分为 0 以及 $\div \bvec B = 0$ 的性质。 对称地, 将上式中的 $\uvec x$ 替换成 $\uvec y$ 和 $\uvec z$ , 等式也成立。 所以
\begin{equation}
\bvec F = \uvec x I\int_\Sigma  \pdv{\bvec B}{x} \vdot \dd{\bvec s} + \uvec y I\int_\Sigma  \pdv{\bvec B}{y} \vdot \dd{\bvec s} + \uvec z I\int_\Sigma \pdv{\bvec B}{z} \vdot \dd{\bvec s}~.
\end{equation} 
写成分量的形式, 就是
\begin{equation}
F_x = I\int_\Sigma  \pdv{\bvec B}{x} \vdot \dd{\bvec s}~, \qquad
F_y = I\int_\Sigma  \pdv{\bvec B}{y} \vdot \dd{\bvec s} ~,\qquad
F_z = I\int_\Sigma  \pdv{\bvec B}{z} \vdot \dd{\bvec s}~.
\end{equation}










