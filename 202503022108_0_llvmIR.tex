% LLVM IR 笔记
% license Xiao
% type Note

\pentry{LLVM 笔记\nref{nod_LLVMnt}}{nod_e723}

\textbf{LLVM IR (Intermediate Representation)} 是 LLVM 的一个 portable 的中间语言。 使用 LLVM 的编译器会先把高级语言编译成 IR, 然后再做进一步优化。 LLVM IR 的语法和汇编语言类似。

一个完整的例子, 可以编译运行
\begin{lstlisting}[language=none,caption=test.ll]
; Declare the main function
define i32 @main() {
  entry:
    ; Allocate stack space for a single 32-bit integer in the stack memory
    ; `var` is a pointer
    %var = alloca i32

    ; Store the value 42 into the variable on memory stack
    ; * means pointer
    store i32 42, i32* %var

    ; Load the value of the variable into a [virtual] register
    %val = load i32, i32* %var

    ; Return the value of the register
    ret i32 %val
}
\end{lstlisting}

编译:
\begin{lstlisting}[language=bash]
llvm-as test.ll -o test.bc # 编译成 bitcode
clang test.bc -o test # 编译成可执行文件
./test # 执行
echo $? # 查看返回值(exit status)
\end{lstlisting}

\begin{itemize}
\item \textbf{Static Single Assignment (SSA)}:每个变量只能赋值一次!
\item  在使用一个变量之前, 通常需要把它加载到 register 中, 而不是直接 dereference。
\item 例如算数以及 return 都需要先加载到 register 中.
\end{itemize}

以下是一个函数例子: 计算两个整数相加。
\begin{lstlisting}[language=none]
define i32 @add(i32 %a, i32 %b) {
  entry:
    %sum = add i32 %a, %b
    ret i32 %sum
}
\end{lstlisting}

调用该函数
\begin{lstlisting}[language=none]
define i32 @main() {
  entry:
    ; set a=10, b=20 in registers
    %a = add i32 0, 10
    %b = add i32 0, 20
    %sum = call i32 @add(i32 %a, i32 %b)
    ret i32 %sum
}
\end{lstlisting}

其中对 \verb`a, b` 的赋值也可以用
\begin{lstlisting}[language=none]
%a = or i32 10, 0
%b = or i32 20, 0
\end{lstlisting}

第三种方法是从内存 stack 中加载到 register
\begin{lstlisting}[language=none]
define i32 @main() {
  entry:
    %a = alloca i32
    %b = alloca i32
    store i32 10, i32* %a
    store i32 20, i32* %b
    %a_val = load i32, i32* %a
    %b_val = load i32, i32* %b
    %sum = call i32 @add(i32 %a_val, i32 %b_val)
    ret i32 %sum
}
\end{lstlisting}

\subsubsection{另一些例子}
栈和堆上的矩阵:
\begin{lstlisting}[language=none]
define i64 @my_func(i64 signext %"my_var") #0 {
  top:
    ; Allocate an array of 10 doubles on the stack
    %array = alloca [10 x double], align 8

    ; Declare the malloc function
    declare i8* @malloc(i64)

    ; Allocate an array of 10 doubles on the heap
    ; Calculate the size needed: 10 doubles * 8 bytes per double
    %size = mul i64 10, 8
    %malloc_result = call i8* @malloc(i64 %size)
    %array = bitcast i8* %malloc_result to double*

    ; Store a value into the array
    %value = fadd double 0.0, 3.14  ; Example value to store
    ; Get pointer to the 3rd element
    %ptr = getelementptr inbounds [10 x double], \
      [10 x double]* %array, i64 0, i64 2
    store double %value, double* %ptr, align 8

    ; Load a value from the array
    %loaded_value = load double, double* %ptr, align 8
}
\end{lstlisting}

\subsection{常识}
\begin{itemize}
\item 合法的本地变量(寄存器)名字如 \verb`%x, %var1, %my_var, %func.entry, %tmp-1`,也可以用数字开头如 \verb`%0, %1a`,甚至用字符串 \verb`%"my var", %"var@1", %"func.entry#1"`。
\item 函数返回用 \verb`ret` 不是 \verb`return`
\item \verb`@名字` 用于声明全局函数或变量
\item \verb`signext`: 指定入参实参应该 sign-extended 到 64 位。一些机器上,如惨可能通过小于 64 位的寄存器上
\item 函数定义的第一行中, \verb`#0` 是函数属性组。 \verb`#0` 代表一系列在模块中其他地方定义的属性,例如优化提示或调用规范。
\item \verb`标签名:` 定义用于跳转的标签。
\item 缩进不重要
\item 名字(变量、函数、label)里面可以包含英文句号 “\verb`.`”
\item 变量名以 \verb`%` 开始
\item 注释用 \verb`;`
\item 对本地变量,不存在类似 C 语言的 \verb`const`。
\end{itemize}

\subsection{判断}
\begin{itemize}
\item \verb`br i1 条件, label %真标签, label %假标签` 用于判断, \verb`i1` 是一个 1bit 整数,也就是 bool。 如果条件为真, 就跳到标签 \verb`真标签`, 否则跳到 \verb`假标签`
\item \verb`br label %标签` 相当于 \verb`goto`
\item 例如, \verb`%cond = icmp eq i32 %x, %y`, 然后 \verb`br i1 %cond, label %equals, label %notequals`
\item label 的定义: \verb`标签名:`
\item 判断 $x < y$: \verb`icmp slt i32 %x, %y` (signed less than), 或者 \verb`sgt`(signed greater than)
\item \verb`ult, ugt` (unsigned less/greater than)
\item 与: \verb`%and_result = and i1 %a, %b`
\item 或: \verb`%result = or i1 %cond1, %cond2`
\item phi 函数用于判断变量来自哪个分支
\end{itemize}

\begin{lstlisting}[language=cpp]
int foo(int x) {
    int y;
    if (x > 0) {
        y = 10;
    } else {
        y = 20;
    }
    return y;
}
\end{lstlisting}

\begin{lstlisting}[language=none]
define i32 @foo(i32 %x) {
  entry:
    %cmp = icmp sgt i32 %x, 0  ; Compare x > 0
    br i1 %cmp, label %if.then, label %if.else

  if.then:
    br label %merge

  if.else:
    br label %merge

  merge:
    ; Select value based on predecessor
    %y = phi i32 [ 10, %if.then ], [ 20, %if.else ]
    ret i32 %y
}
\end{lstlisting}


\subsection{循环}
一个什么都不做的空循环,注意这里循环变量 \verb`%i`
\begin{lstlisting}[language=none]
define void @countUpTo(i32 %limit) {
  entry:
    ; Initialize loop variable
    %i = alloca i32
    store i32 0, i32* %i

    ; Jump to the loop condition check
    br label %loop.cond

  loop.cond:
    ; Load loop variable
    %i.val = load i32, i32* %i

    ; Check if the loop variable is less than the limit
    %cond = icmp slt i32 %i.val, %limit
    br i1 %cond, label %loop.body, label %loop.end

  loop.body:
    ; Increment the loop variable
    %next.val = add i32 %i.val, 1
    store i32 %next.val, i32* %i

    ; Jump back to the loop condition
    br label %loop.cond

  loop.end:
    ; Exit point of the loop
    ret void
}
\end{lstlisting}

生成数组 \verb`[1,2,3,4]`
\begin{lstlisting}[language=none]
define void @assignArray() {
  entry:
    ; Allocate an array of 4 integers on the stack
    %array = alloca [4 x i32]

    ; Loop counter
    %i = alloca i32
    store i32 0, i32* %i

    ; Jump to the loop condition
    br label %loop.cond

  loop.cond:
    ; Load the loop counter
    %i.val = load i32, i32* %i

    ; Check if the loop counter is less than 4
    %cond = icmp slt i32 %i.val, 4
    br i1 %cond, label %loop.body, label %loop.end

  loop.body:
    ; Calculate the address of the current array element
    %addr = getelementptr [4 x i32], [4 x i32]* %array, i32 0, i32 %i.val

    ; Assign the corresponding value (i+1) to the array element
    %val = add i32 %i.val, 1
    store i32 %val, i32* %addr

    ; Increment the loop counter
    %next.i = add i32 %i.val, 1
    store i32 %next.i, i32* %i

    ; Jump back to the loop condition
    br label %loop.cond

  loop.end:
    ; Exit point of the loop
    ret void
}
\end{lstlisting}

要在 heap 中分配内存,取决于所在系统,一般调用 C 语言的 \verb`malloc`
\begin{lstlisting}[language=none]
declare i8* @malloc(i32) ; 声明 malloc

define i32* @createArray() {
  entry:
    ; Calculate the size of the array (4 integers)
    %size = mul i32 4, 4 ; size for 4 integers (assuming i32 is 4 bytes)

    ; Call malloc to allocate memory on the heap
    %arrayPtr = call i8* @malloc(i32 %size)

    ; Cast the returned i8* (generic pointer) to i32* (pointer to integers)
    %typedArrayPtr = bitcast i8* %arrayPtr to i32*

    ; Return the pointer to the allocated memory
    ret i32* %typedArrayPtr
}
\end{lstlisting}
