% python符号计算
% 符号计算|python

\subsection{什么是符号计算?}
符号计算又称计算机代数,通俗地说就是用计算机推导数学公式,如对表达式进行因式分解、化简、微分、积分、解代数方程、求解常微分方程等.在SciPy 数值微分与积分\upref{SciPy}部分我们已经介绍了如何利用\verb|python|实现相关数值计算,这里面我们将进一步介绍符号计算在\verb|python|中的实现.那么数值计算与符号计算有什么区别与联系呢?个人认为:首先在数值计算过程中,所有出现的\verb|变量|或者\verb|参数|在使用之前必须给定具体取值,并且计算结果大多数是近似的;相反的是,在符号计算过程中,变量可以预先不给定取值,计算结果是准确的,解析的.不太准确的表述为:符号计算就是对表达式进行的操作;数值计算是对数据进行的操作.

\subsection{\verb|scipy|库}
在\verb|python|中,专门进行符号计算的库是\verb|sympy|(symbol python的简写).利用这个库可以进行符号表达式的加减乘除等四则运算、符号化简、求导、积分、极限、解方程(组)、解微分方程(组)等等.下面我们将进行逐一介绍.

\subsubsection{符号变量的定义}
我们首先导入\verb|sympy|库:
\begin{lstlisting}[language=python]
import sympy as sy
\end{lstlisting}
为了代码简洁,后面默认在代码之前均引入了这个库.例如
定义单个变量
\begin{lstlisting}[language=python]
x0 = sy.symbols('x0')
\end{lstlisting}
同时定义多个变量
\begin{lstlisting}[language=python]
x0 = sy.symbols('x0')
x1 = sy.symbols('x1')
\end{lstlisting}
这样定义理论上没有问题,但是显得繁琐.上述梁行代码可以改写为
\begin{lstlisting}[language=python]
x0, x1 = sy.symbols('x0, x1')
\end{lstlisting}
对于多个变量的定义方法类似,这里不再赘述.需要说明的是\verb|sy.symbols('x0, x1')|中的\verb|,|用空格隔开也是可以的.即\verb|sy.symbols('x0, x1')=sy.symbols('x0 x1')|.

当定义多个连续变量时,也可以这样
\begin{lstlisting}[language=python]
 x, y, z = sy.symbols('x:z')
x4, x5, x6, x7 = sy.symbols('x4:8')
\end{lstlisting}
注意不包括\verb|x8|.

\subsubsection{符号常量与函数}
\verb|sympy|库中预置了许多常量:圆周率$\pi$,自然常数$e$,无穷大$\infty$,虚数单位$i$等等.它们分别为
\begin{lstlisting}[language=python]
sy.pi sy.E sy.I sy.oo
\end{lstlisting}
里面也有许多常用函数,比如$sin(x), arcsin(x), sinh(x), e^x,
log(x),\sqrt{x}$.对应\verb|python|代码为:
\begin{lstlisting}[language=python]
sy.sin() sy.asin() sy.sinh() sy.exp() sy.log() sy.sqrt()
\end{lstlisting}
上面只是列举了部分函数与常量,\verb|sympy|库只能还有非常多,感兴趣的读者可以参考官方文档\href{https://docs.sympy.org/latest/index.html#}{https://docs.sympy.org/latest/index.html}
另外需要注意的是,\verb|sympy|中的符号不能与\verb|numpy,scipy|中的复合混用,否则会报错,例如
\begin{lstlisting}[language=python]
>>> import numpy as np
>>> x = sy.symbols('x')
>>> np.exp(x) # 尝试用numpy来计算exp(x),注意此处x是符号变量.
\end{lstlisting}
此处返回错误
\begin{lstlisting}[language=python]
Traceback (most recent call last):
File "<stdin>", line 1, in <module>
AttributeError: 'Symbol' object has no attribute 'exp'
\end{lstlisting}
正确的用法为
\begin{lstlisting}[language=python]
>>> sy.exp(x) # Use SymPy's version instead.
exp(x)
\end{lstlisting}
我们在看这个例子
\begin{lstlisting}[language=python]
>>> x = sy.symbols('x')
>>> (2/3) * sy.sin(x) # 2/3 返回一个浮点数而不是有理数.
0.666666666666667*sin(x)
>>> sy.Rational(2, 3) * sy.sin(x) #为了使2/是一个有理数需要使用Rational
2*sin(x)/3
\end{lstlisting}
为了巩固上面符号计算的一些细节我们来做一个练习.
\begin{example}{}
用\verb|sympy|库的函数写一个函数返回如下表达式:
\begin{align}
\frac{2}{5}e^{x^2-y}cosh(x+y)+\frac{1}{2}log(xy+1)
\end{align}


对应代码如下
\begin{lstlisting}[language=python]
def cal()
   x, y  = sm.symbols('x,y')
   express = sy.Rational(2,5)*sm.exp(x**2-y)*\
   sm.cosh(x+y)+Rational(1,2)*sm.log(x*y+1)
   return express

\end{lstlisting}
\end{example}

\subsubsection{符号加与符号乘运算}
利用\verb|sympy|计算$\sum_{i=1}^4{x+iy}$与$\prod_{i=1}^5{x+iy}$