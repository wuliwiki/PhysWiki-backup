% 正规扩张
% 分裂域|多项式|共轭|compositum|composite|合成域

\pentry{分裂域\upref{SpltFd}}

\autoref{SpltFd_the2}~\upref{SpltFd}揭示了分裂域和正规扩张的等价性,所以本词条讨论的对象本质上和\textbf{分裂域}\upref{SpltFd}是一样的,但会更加深入和全面.


\begin{definition}{共轭}
设$\mathbb{F}$是一个域,$\overline{\mathbb{F}}$是其代数闭包.

对于$\alpha\in\overline{\mathbb{F}}$,其关于$\mathbb{K}$的\textbf{共轭元素(conjugate)}定义为其在$\overline{\mathbb{F}}$的某个保$\mathbb{F}$自同构的像.

对于$\mathbb{F}$的代数扩张$\mathbb{K}$,其关于$\mathbb{K}$的\textbf{共轭域(conjugate)}定义为其在$\overline{\mathbb{F}}$的某个保$\mathbb{F}$自同构的像.
\end{definition}

显然,无论对元素还是域,共轭都是一种等价关系.由于共轭元素和共轭域的定义都依赖同一个自同构,因此,两个共轭域中的元素彼此对应共轭.

类比\autoref{SpltFd_the3}~\upref{SpltFd}的证明思路,可知两元素共轭的充要条件是,它们是同一个不可约多项式的根.

我们也可以用共轭的语言来描述正规扩张:

\begin{theorem}{}\label{NomEx_the1}
一个代数扩张$\mathbb{K}/\mathbb{F}$是正规的,当且仅当$\mathbb{K}$关于$\mathbb{F}$的共轭只有它自己,当且仅当任意$a\in\mathbb{K}$的共轭元素仍然在$\mathbb{K}$中.
\end{theorem}



\subsection{正规扩张的性质}

现在我们讨论的是,给定域上正规扩张集合的结构.

\begin{theorem}{}
设$\mathbb{K}/\mathbb{F}$是正规扩张,且存在中间域$\mathbb{M}$,则$\mathbb{K}/\mathbb{M}$也是正规扩张.
\end{theorem}

\textbf{证明}:

正规扩张等价于分裂域.如果$\mathbb{K}=\mathbb{F}(a_1, a_2, \cdots)$,那么$\mathbb{K}=\mathbb{M}(a_1, a_2, \cdots)$.由此得证.

\textbf{证毕}.


\begin{definition}{合成}
设$\mathbb{F}_i$是$\mathbb{K}$的一族子域,则记$\prod_{i}\mathbb{F}_i$为包含全体$\mathbb{F}_i$的最小的子域,称为族$\{\mathbb{F}_i\}$的\textbf{合成(composite或compositum)}.

$\prod_{i}\mathbb{F}_i$也可以记为$\mathbb{F}_1\mathbb{F}_2\cdots$.
\end{definition}


合成也可以看成是一种扩域,$\mathbb{K}\mathbb{F}=\mathbb{K}(\mathbb{F})=\mathbb{F}(\mathbb{K})$.


\begin{theorem}{}
设$\mathbb{K}/\mathbb{F}$是正规扩张,子域$\mathbb{E}\subseteq\overline{\mathbb{F}}$.如果合成域$\mathbb{EK}$存在,那么$\mathbb{EK}/\mathbb{EF}$是正规扩张.
\end{theorem}

\textbf{证明}:

取$f\in\mathbb{F}[x]\subseteq\mathbb{EF}[x]$,使得$\mathbb{K}$是$f\in\mathbb{F}[x]$的分裂域,那么$\mathbb{EK}$是$f\in\mathbb{EF}[x]$的分裂域.

\textbf{证毕}.






\begin{theorem}{}
设$\mathbb{K}_i/\mathbb{F}$是正规扩张,则$\mathbb{K}_1\mathbb{K}_2/\mathbb{F}$是正规扩张.
\end{theorem}


\textbf{证明}:

设$\mathbb{K}_i$是$f_i\in\mathbb{F}[x]$的分裂域,那么$\mathbb{K}_1\mathbb{K}_2$是$f_1f_2\in\mathbb{F}[x]$的分裂域.

\textbf{证毕}.


\begin{theorem}{}\label{NomEx_the2}
设$\mathbb{K}_i/\mathbb{F}$是正规扩张,则$\bigcap_{i}\mathbb{K}_i/\mathbb{F}$是正规扩张.
\end{theorem}

\textbf{证明}:

正规扩张的定义:任取$f\in\mathbb{F}[x]$,若其有一根在$\mathbb{K}_i$中,则其所有根都在$\mathbb{K}_i$中.由定义直接得证.

\textbf{证毕}.


考虑到共轭的定义及其性质,即\autoref{NomEx_the1} ,我们可以得到下面这个性质:

\begin{theorem}{}\label{NomEx_the3}
设$\mathbb{K}/\mathbb{F}$是一个代数扩域.$\mathbb{F}$的全体包含$\mathbb{K}$的正规扩张之交集,是$\mathbb{K}$关于$\mathbb{F}$的全体共轭域之\textbf{合成}.
\end{theorem}

\textbf{证明}:

由于正规扩张包含所有的根,共轭域之间的元素也对应共轭,且共轭元素是同一个不可约多项式的根,故$\mathbb{F}$的每一个包含$\mathbb{K}$的正规扩张,都包含$\mathbb{K}$关于$\mathbb{F}$的全体共轭域.

下证全体共轭域的合成是正规扩张.

$\mathbb{K}$关于$\mathbb{F}$的全体共轭域之合成记为$\mathbb{H}$.设$\mathbb{K}$关于$\mathbb{F}$的全体共轭域的并集为$S$,则$\mathbb{H}=\mathbb{F}(S)$.

任取$\overline{\mathbb{F}}$的保$\mathbb{H}$自同构$\sigma$,则由共轭域的定义,$\sigma(S)=S$.于是$\sigma(\mathbb{F}(S))=\sigma(\mathbb{F})(\sigma(S))=\mathbb{F}(\sigma(S))=\mathbb{F}(S)$.即,$\sigma(\mathbb{H})=\mathbb{H}$.

由\autoref{NomEx_the1} 即得证.

\textbf{证毕}.


\begin{theorem}{}
域$\mathbb{F}$的有限扩张,总包含在$\mathbb{F}$的某个有限正规扩张里;$\mathbb{F}$的可分扩张,总包含在$\mathbb{F}$的某个可分正规扩张里.
\end{theorem}

\textbf{证明}:

先证明有限扩张的情况:

设$\mathbb{K}/\mathbb{F}$是有限扩张,那么作为$\mathbb{F}$上的线性空间,$\mathbb{K}$的基只有有限多个元素.$\mathbb{K}$的任何保$\mathbb{F}$自同构,由于是同构,因此只取决于基向量映射到哪里.由于有限扩张必是代数扩张,故每个基向量都是代数元素,故每个基向量的共轭元素是有限多的.综上,$\mathbb{K}$关于$\mathbb{F}$的共轭域只能是有限多个.

据\autoref{NomEx_the3} ,取$\mathbb{K}$关于$\mathbb{F}$的共轭域之合成.由于每个共轭域在$\mathbb{F}$都是有限维线性空间,则其合成也是有限维的\footnote{如果每个共轭域的维数是$n$,一共$k$个共轭域,则其合成的维数不超过$n^k$.}.

\textbf{证毕}.













