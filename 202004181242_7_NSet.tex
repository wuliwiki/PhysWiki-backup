% 朴素集合论
\subsection{公理,定义和定理}
我们是不可能证明所有的命题的,所以任何理论必须有一个出发点,也就是一些基础命题.这些基础命题本身不可证明,它们决定了理论的样貌,理论中一切其它命题都是由这些命题根据逻辑推演得到的.这样的基础命题被称为一个理论的\textbf{公理}.

有了公理系统以后,我们还需要明确所讨论的对象是什么.比如我用了皮亚诺公理来定义小学四则运算,那么为了讨论“1+1等于几”这样的问题,我首先需要明确“1”和“+”具体指什么.用来明确概念的陈述句,被称为\textbf{定义}.如果说公理系统是创建了一个宇宙的基础参数的话,那么定义就是在给这个宇宙里已经自然存在的事物进行命名,这样才能讨论这些事务.

最后,任何一个理论的绝大部分内容都是在使用基础命题来进行推演,看哪些命题能成立.这些成立的命题,就叫做\textbf{定理}. 有时候,根据定理作用的不同,我们也可能称其中一些为\textbf{引理}、\textbf{推论}等.所有定理加在一起就构成了整个理论.

不同的公理系统可能推演出相同的命题,也可能推演出彼此矛盾的命题,更可能存在一些无法判断是否成立的命题.一个公理系统中所无法判断是否成立的命题,就叫做独立于这个公理系统的命题.如果两个公理系统能够推演出完全一样的命题(定理),那么这两个公理系统就是等价的.如果公理系统A能够推演出公理系统B的一切定理,但是B不能推出A中的一切定理,即A能推演出的一些定理实际上是独立于B的命题,那么可以认为是公理系统A包含了公理系统B.\textbf{在阅读本段话时,请注意命题和定理的区别:定理是在给定公理体系下能被推演出来的命题.}

以上表述是数学的表达方式.在物理学中,\textbf{公理}和\textbf{定理}可以分别被翻译成\textbf{定律}和\textbf{现象}. 

\subsection{集合}

对于物理学习而言,集合论没必要从公理角度来严格理解,所以在此给出的是朴素集合论的解释.

集合是由元素组成的.任何事物和概念都可以成为元素,任何不同的元素都可以放在一起,构成一个集合.可以说,如果我们划定一个讨论的范围,那么这个范围就是一个集合,范围涉及到的事物和概念就是这个集合当中的元素.

表达一个集合的方式有多种,最简单的方式是列出所有集合中的元素.在数学中规定的语法规范是用大括号“\{\}”来列举集合中的一切元素,以逗号“,”隔开彼此.比如,{猪,牛,狗,羊,猫}构成了一个具有五个元素的集合,{1,2,3,4,$\cdots$}则是全体自然数的集合.


