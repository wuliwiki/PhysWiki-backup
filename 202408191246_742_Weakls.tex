% 弱引力透镜
% license Usr
% type Tutor


今天,关于星系团尺度上暗物质存在的最引人注目的证据之一来自对一对碰撞星系团的观测,这些星系团被称为子弹星系团,位于3.7G年远的地方,目录名为1E0657-558(或1E0657-56),首次于2006年进行了详细观测,以及类似的系统。子弹星系团中大部分重子质量以热气体的形式存在,其分布可以通过X射线发射来追踪。总质量的分布,包括可见和暗物质,是通过弱透镜独立测量的。子弹星系团系统的特殊之处在于,可见物质和暗物质在空间上是分离的,见\autoref{fig_Weakls_1} 。\begin{figure}[ht]
\centering
\includegraphics[width=14cm]{./figures/95028d589604d526.png}
\caption{子弹星系团} \label{fig_Weakls_1}
\end{figure}
在过去,这两个星系团是普通的系统,可见物质和暗物质混合在一起。这两个物体在1.5亿年前发生了碰撞。可见物质与自身有显著的相互作用,以至于两个星系团的热气体经历了碰撞性冲击波。另一方面,暗物质与自身和正常物质的碰撞可以忽略不计,以至于两个系统中的暗物质云仅仅是相互穿过。这导致了目前可见和暗物质组分的分离\footnote{详细研究重建了两个星系团碰撞前大约3000 km/s的初始相对速度。有人声称这个速度异常高:根据ΛCDM宇宙学中速度分布的尾部,观察到这种事件的概率声称太低(假设合理的物质非均匀性,大约是10^-5)。因此,子弹星系团在相对速度的这个特定方面被用作反对暗物质的证据。后来的研究对此提出了质疑,并发现与ΛCDM宇宙学一致的概率。}。在观察子弹星系团之后,许多类似的系统已经被研究。Harvey等人(2015)报告了对72个类似系统的观测结果,并得出结论,暗物质的存在可以以超过7σ的显著性建立。这种观测对那些用修改引力代替暗物质的替代解释施加了严重的压力。这样的修改不能从正常物质中空间分离出来(除非它们也引入了实际上表现为暗物质的东西),因此异常的透镜信号将不会跟随可见物质的分布.

子弹星系团和类似的系统也提供了有关暗物质粒子物理属性的信息。事实上,与无碰撞轨迹相比,暗物质晕没有减速,这意味着在系统典型尺度上,一个暗物质粒子与另一个暗物质粒子散射的概率$\varphi$有一个上限,大约是$\varphi \sim \sigma n r_{cl}\sim 1$,其中σ$$是暗物质自相互作用截面,n是暗物质的数密度,rcl是星系团的大小。在子弹星系团的特定情况下,数据显示rcl ∼ 150 kpc,ρ ∼ 0.5 GeV/cm³。观测提供了关于暗物质密度ρ = Mn的信息(而不是直接关于暗物质数密度n的信息),因此可以得到σ/M的界限,其中M是暗物质的质量。不同的研究找到了略有不同的值[13, 16],大致在(1.5)式中所示的范围内,即σ/M ≲ 1.8 mb GeV,当假设点相互作用时。作为比较,50 mb是典型的强相互作用截面,例如,质子-质子散射。因此,如果暗物质由比质子稍重和/或相互作用比质子稍弱的粒子组成,则满足(1.5)式中的界限。暗物质/物质的截面同样受到天体物理学和宇宙学的限制,大致在M ∼ 1 GeV时比强相互作用截面小,尽管还有许多其他更严格的限制适用(见第5.1.2节的完整讨论)。

宇宙剪切
在星系团和宇宙尺度之间的某些长度尺度上,宇宙剪切的探测也提供了暗物质的证据[17]。宇宙剪切指的是由于前景质量集中的引力吸引,非常远的星系的光被偏转。这些质量集中并不是以星系或星系团的暗物质晕的形式存在(就像在星系-星系透镜和星系团碰撞的情况中讨论的那样),而是以更大、更分散的结构形式存在,如庞大的丝状结构和松散的团块。测量是在多波长的大规模巡天上,对数百万计的远星系进行的,这些巡天必须非常深(探测非常远的星系)并且非常广(探索天空的大部分地区)。由于剪切,远星系的形状被扭曲,改变了主轴与次轴比率大约2\%,这可以在统计基础上被探测到。观测确定了ΩDM ≈ 0.25。宇宙剪切的重要性远远超出了仅仅为暗物质的存在提供额外证据。通过沿视线重建物质分布,我们实际上是在回顾为宇宙提供支撑的大型暗物质结构的形成历史,我们将在下一节中讨论。

CMB透镜效应
与宇宙剪切类似的现象还有所谓的CMB透镜效应,它涉及寻找由于前景质量集中造成的CMB图的扭曲。从某种意义上说,这是利用弱引力透镜作为探测暗物质工具的终极应用。由于CMB是最早可观测到的光,它为我们提供了一个独特的机会,来研究在高红移(z ≳ 2)处的极大暗物质结构。CMB透镜效应已经在统计上得到了高度显著的探测,无论是在Planck的数据中还是在其他实验的数据中(见[18]的综述和最新的Planck结果),这证实了暗物质的证据。透镜效应的作用是重新洗牌CMB的扰动。值得注意的是,它在大的ℓ处平滑了CMB功率谱的峰值,并影响了CMB光子的偏振,从而在E模式中创造了所谓的B模式。实际上,对于对初级CMB感兴趣的人来讲,透镜效应是一种污染物,必须被减去(即,CMB谱必须被“去透镜化”)。

