% 基本型
% 微分几何|面积|积分|二次型|fundamental form|quadratic form

\pentry{三维空间中的曲面\upref{RSurf}}

\subsection{第一基本型}

曲面上的第一基本型,本质上就是用局部坐标系的正定二次型来定义一个切空间上的内积.由于古典微分几何依赖$\mathbb{R}^3$,我们可以把第一基本型简单理解为,曲面的几何切平面上各切向量的\textbf{长度}.

设$\bvec{x}$是曲面$S$上的某个局部坐标系,那么我们就可以把$S$在这个坐标系内的一条曲线表示为$\alpha(t)=\bvec{x}(u(t), v(t))$.这里$(u(t), v(t))$可以理解为曲线在时间$t$时的坐标,经$\bvec{x}$映射后成为$S$上的一个点.

再次回忆:曲线就是道路,也就是向量.那么我们可以计算这个向量的长度.首先,这个曲线对应的向量是什么呢?记这个向量为$\bvec{v}_\alpha\in\mathbb{R}^3$,那么我们有:
\begin{equation}
\bvec{v}_\alpha=\lim\limits_{t\to 0}\frac{\alpha(t)
\end{equation}
