% 理想气体的内能
% 状态方程|理想气体|内能|动能

\begin{issues}
\issueDraft
\end{issues}

\pentry{理想气体状态方程\upref{PVnRT}, 能均分定理\upref{EqEng}}

理想气体的动能为(\autoref{PVnRT_eq5}~\upref{PVnRT})
\begin{equation}
E_k = \frac32 Nk_B T = \frac{3}{2}nRT
\end{equation}
分子运动中, 三个方向的动能占三个自由度, 而对于多原子分子, 还可能出现转动和振动等自由度. 令自由度为 $i$, 则内能(即总能量)为
\begin{equation}\label{IdgEng_eq1}
E = \frac{i}{2}Nk_B T = \frac{i}{2}nRT
\end{equation}
令平动自由度为$t$,转动自由度为$r$,振动自由度为$s$.则
\begin{equation}
i=t+r+s
\end{equation}
在三维空间中,平动自由度$t=3$(即:分子具有3个独立的速度分量),而转动自由度为$r$取决于分子的形状.对于一般的非线型分子,取$r=3$(即:分子绕$x,y,z$三个方向旋转),而对线型分子,有$r=2$(分子对于沿分子所在直线的轴旋转对称,故少1).而振动自由度$s$在较低温度(如:常温)下不激发(即:$s=0$),在高温下可由下式决定:
\begin{equation}
s=3N-t-r
\end{equation}
其中$N$为分子中的总原子数.则总自由度为
\begin{equation}
i=
\end{equation}
