% 阿瑟·凯莱(综述)
% license CCBYSA3
% type Wiki

本文根据 CC-BY-SA 协议转载翻译自维基百科\href{https://en.wikipedia.org/wiki/Arthur_Cayley}{相关文章}。

\begin{figure}[ht]
\centering
\includegraphics[width=6cm]{./figures/54015c4f6823f2b0.png}
\caption{} \label{fig_Cayley_1}
\end{figure}
阿瑟·凯利(Arthur Cayley,FRS,1821年8月16日-1895年1月26日)是英国数学家,主要从事代数方面的研究。他帮助创立了现代英国纯数学学派,并在剑桥大学三一学院担任教授长达35年。

他提出了如今被称为凯利-哈密顿定理的观点——即每个方阵都是其自身特征多项式的根,并验证了2阶和3阶矩阵的情况。[1]他是第一个定义抽象群概念的人,抽象群是满足某些运算规律的集合,[2]区别于艾瓦里斯特·伽罗瓦(Évariste Galois)对置换群的定义。在群论中,凯利表、凯利图以及凯利定理都以他命名,而在组合数学中,也有凯利公式以纪念他。
