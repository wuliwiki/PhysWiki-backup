% 普朗克常数(综述)
% license CCBYSA3
% type Wiki

本文根据 CC-BY-SA 协议转载翻译自维基百科\href{https://en.wikipedia.org/wiki/Planck_constant}{相关文章}。

普朗克常数,记作\( h \),是量子力学中一个具有基础性重要性的物理常数:一个光子的能量等于其频率与普朗克常数的乘积,物质波的波长等于普朗克常数除以相关粒子的动量。与之密切相关的约化普朗克常数,等于\( \frac{h}{2\pi} \),通常用\( \hbar \)表示,在量子物理方程中广泛使用。

这个常数由马克斯·普朗克在1900年提出,作为解释实验黑体辐射所需的比例常数。\(^\text{[2]}\)普朗克后来将这个常数称为“作用量子”。\(^\text{[3]}\)1905年,阿尔伯特·爱因斯坦将“量子”或能量的最小元素与电磁波本身联系起来。马克斯·普朗克因其“通过发现能量量子对物理学发展的贡献”而获得了1918年诺贝尔物理学奖。

在计量学中,普朗克常数与其他常数一起,用来定义千克,即质量的国际单位制(SI)单位。\(^\text{[4]}\)国际单位制的定义方式使得当普朗克常数以国际单位制单位表示时,其精确值为\( h = 6.62607015 \times 10^{-34} \, \text{J} \cdot \text{Hz}^{-1} \)。\(^\text{[5][6]}\)
\subsection{历史}  
\subsubsection{常数的起源}
\begin{figure}[ht]
\centering
\includegraphics[width=6cm]{./figures/ef10d79012cb91de.png}
\caption{柏林洪堡大学的铭牌:“马克斯·普朗克,作用量子\(h\)的发现者,从1889年到1928年曾在这座建筑中授课。”} \label{fig_PLKCS_1}
\end{figure}
普朗克常数是作为马克斯·普朗克成功努力的一部分提出的,普朗克试图推导出一个数学表达式,准确预测从封闭炉子(黑体辐射)中观察到的热辐射谱分布。\(^\text{[7]}\)这个数学表达式现在被称为普朗克定律。

在19世纪的最后几年,马克斯·普朗克正在研究黑体辐射的问题,这个问题大约40年前由基尔霍夫首次提出。每个物体都会自发地、持续地发射电磁辐射。然而,观察到的辐射光谱的整体形状并没有表达式或解释。当时,维恩定律能够很好地拟合短波长和高温的数据,但对于长波长则失败了。\(^\text{[7]: 141}\) 同时,尽管普朗克尚不知情,雷利勋爵已经理论推导出了一个公式,现在称为雷利-简定律,该公式能够合理地预测长波长,但在短波长处表现得非常不准确。
\begin{figure}[ht]
\centering
\includegraphics[width=10cm]{./figures/90dffea43d212d35.png}
\caption{黑体辐射的光强度。每条曲线表示不同体温下的行为。普朗克常数 \( h \) 用于解释这些曲线的形状。} \label{fig_PLKCS_2}
\end{figure}
普朗克试图找到一个数学表达式,既能重现维恩定律(适用于短波长),又能重现经验公式(适用于长波长)。这个表达式包括一个常数\(h\),这个常数被认为是 Hilfsgröße(辅助量),\(^\text{[8]}\)并随后成为了普朗克常数。普朗克所提出的表达式表明,某物体在绝对温度\(T\)下,对于频率\(\nu\)的频率单位辐射光谱辐照度为:
\[
B_{\nu}(\nu, T) d\nu = \frac{2h\nu^3}{c^2} \frac{1}{e^{\frac{h\nu}{k_{\mathrm{B}}T}} - 1} d\nu,~
\]
其中\(k_{\text{B}}\)是玻尔兹曼常数,\(h\)是普朗克常数,\(c\)是介质中的光速,无论是物质还是真空。\(^\text{[9][10][11]}\)

普朗克很快意识到他的解并不是唯一的。存在几种不同的解,每种解给出的振荡器熵值都不同。\(^\text{[2]}\)为了挽救他的理论,普朗克采用了当时颇具争议的统计力学理论,\(^\text{[2]}\)他将其描述为“一种绝望的尝试”。他的一个新边界条件是:

将\( U_N \)[‘N个振荡器的振动能量’] 解释为不是一个连续的、可以无限分割的量,而是作为一个由有限数量相等部分组成的离散量。我们称每一部分为能量元素\(\varepsilon\);

——普朗克,《关于正常谱中能量分布的定律》\(^\text{[2]}\)

通过这一新条件,普朗克对振荡器的能量进行了量子化,正如他自己所说,“这只是一个纯粹的形式假设……实际上我并没有多想”,\(^\text{[13]}\)但这个假设却将彻底改变物理学。将这一新方法应用于维恩位移定律,表明“能量元素”必须与振荡器的频率成正比,这就是现在有时被称为“普朗克-爱因斯坦关系”的第一个版本:
\[
E = hf.~
\]
普朗克能够通过黑体辐射的实验数据计算出普朗克常数\(h\)的值:他的结果为\( 6.55 \times 10^{-34} \, \text{J} \cdot \text{s}\),与当前定义的值相差仅1.2\%。\(^\text{[2]}\) 他还通过相同的数据和理论首次确定了玻尔兹曼常数\(k_{\text{B}}\)的值。\(^\text{[14]}\)
\subsubsection{发展与应用}
\begin{figure}[ht]
\centering
\includegraphics[width=10cm]{./figures/cac866e2a45269bc.png}
\caption{在不同温度下观察到的普朗克曲线,以及理论上的雷利-简斯(黑色)曲线与在5000 K下观察到的普朗克曲线的偏离。} \label{fig_PLKCS_3}
\end{figure}
黑体问题在1905年再次被讨论,当时雷利勋爵和詹姆斯·简斯(共同)以及阿尔伯特·爱因斯坦独立证明了经典电磁学永远无法解释观察到的光谱。这些证明通常被称为“紫外灾难”,这个名称是保罗·爱伦费斯特在1911年提出的。它们(连同爱因斯坦关于光电效应的工作)极大地促使物理学家相信,普朗克关于量子化能级的假设不仅仅是一个数学形式主义。1911年第一次索尔维会议专门讨论了“辐射和量子理论”。\(^\text{[15]}\)
\subsubsection{光电效应}  
光电效应是指当光照射到表面时,表面释放出电子(称为“光电子”)。它首次由亚历山大·埃德蒙·贝克雷尔于1839年观察到,尽管通常将荣誉归于海因里希·赫兹,\(^\text{[16]}\)他于1887年发表了第一次全面的研究。另一个特别详细的研究由菲利普·莱纳德于1902年发表。\(^\text{[17]}\)爱因斯坦1905年关于光量子的论文\(^\text{[18]}\)讨论了这一效应,这使得他在1921年获得了诺贝尔奖,\(^\text{[16]}\)这一奖项是在罗伯特·安德鲁·米利肯的实验工作验证了爱因斯坦的预测之后颁发的。诺贝尔委员会将奖项颁给他在光电效应方面的工作,而非相对论,部分原因是存在偏见,认为纯理论物理学如果没有基于发现或实验的支持,就不值得授奖,同时委员会成员对相对论是否真实的证明存在分歧。\(^\text{[20][21]}\)

在爱因斯坦的论文之前,电磁辐射,如可见光,被认为表现为波的性质:因此,使用“频率”和“波长”来描述不同类型的辐射。波在给定时间内传递的能量称为其强度。剧院聚光灯的光比家庭灯泡的光更强烈;也就是说,聚光灯在单位时间和单位空间内释放的能量比普通灯泡多(因此消耗更多电力),即使光的颜色可能非常相似。其他波,如声音或冲击海岸线的海浪,也有其强度。然而,光电效应的能量解释似乎与光的波动描述不一致。

由于光电效应产生的“光电子”具有一定的动能,这些动能可以被测量。每个光电子的动能与光的强度无关,\(^\text{[17]}\)但与频率呈线性关系;\(^\text{[19]}\)如果频率过低(对应的光子能量小于材料的功函数),则根本不会发射任何光电子,除非多个光子,其能量之和大于光电子的能量,几乎同时作用(多光子效应)。\(^\text{[22][23]}\)假设频率足够高以引发光电效应,光源强度的增加会导致更多的光电子以相同的动能被发射出来,而不是以更高的动能发射相同数量的光电子。\(^\text{[17]}\)

爱因斯坦对这些观察结果的解释是,光本身是量子化的;光的能量不是像经典波那样连续传递的,而是以小“包”或量子的形式传递。这些“能量包”的大小,后来被命名为光子,将与普朗克的“能量元素”相同,从而得出了普朗克-爱因斯坦关系的现代版本:
\[
E = hf.~
\]
爱因斯坦的假设后来通过实验得到了验证:入射光的频率 \( f \) 与光电子的动能 \( E \) 之间的比例常数被证明等于普朗克常数 \( h \)。\(^\text{[19]}\)
\subsubsection{原子结构}  
\begin{figure}[ht]
\centering
\includegraphics[width=8cm]{./figures/6e0ad2f7a0370bdb.png}
\caption{氢原子波尔模型的示意图。图中从 n = 3 能级到 n = 2 能级的跃迁产生了656 nm(红色)波长的可见光,正如该模型所预测的那样。} \label{fig_PLKCS_4}
\end{figure}
在1912年,约翰·威廉·尼科尔森开发了\(^\text{[24]}\) 一种原子模型,并发现模型中电子的角动量与\(h/2\pi\)相关。\(^\text{[25][26]}\)尼科尔森的核量子原子模型影响了尼尔斯·波尔原子模型的发展,\(^\text{[27][28][26]}\)波尔在他的1913年关于波尔模型的论文中引用了他。\(^\text{[29]}\)波尔的模型超越了普朗克的抽象谐振子概念:波尔原子中的电子只能具有某些特定的能量\( E_n \),由以下公式定义:
\[
E_n = -\frac{hcR_{\infty}}{n^2},~
\]
其中\( c \)是真空中的光速,\( R_{\infty} \)是一个通过实验确定的常数(瑞德伯常数),\( n \in \{1, 2, 3, \dots\} \)。这种方法还使波尔能够解释瑞德伯公式,这是氢的原子光谱的经验描述,并能通过其他基本常数来解释瑞德伯常数\(R_{\infty}\)的值。在讨论模型中电子的角动量时,波尔引入了量\(\frac{h}{2\pi} \),现在被称为约化普朗克常数,作为角动量的量子。\(^\text{[29]}\)
\subsubsection{不确定性原理}  
普朗克常数也出现在维尔纳·海森堡不确定性原理的表述中。对于多个处于相同状态的粒子,它们的位置不确定性\( \Delta x \)和动量不确定性\( \Delta p_x \)满足如下关系:
\[
\Delta x \, \Delta p_x \geq \frac{\hbar}{2},~
\]
其中,不确定性被定义为测量值与期望值之间的标准差。还有几个其他类似的物理可测共轭变量对,它们也遵循类似的规则。一个例子是时间与能量。两个共轭变量之间的不确定性反比关系在量子实验中强迫进行权衡,因为更精确地测量一个量会导致另一个量变得不精确。