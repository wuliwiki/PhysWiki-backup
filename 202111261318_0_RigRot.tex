% 刚体定轴转动 转动惯量
% 转动惯量|角动量定理|刚体|绕轴转动|角速度|角加速度

\pentry{刚体\upref{RigBd}, 角动量定理\upref{AMLaw}}

\subsection{刚体的绕轴转动}
若刚体绕固定轴转动, 那么刚体的位置只需一个变量即可完全确定(一个自由度), 我们令该变量为转角 $\theta$. $\theta$ 关于时间 $t$ 的导数就是刚体绕轴旋转的角速度 $\omega$. 我们还可以定义角速度 $\omega$ 关于时间的导数(即 $\theta$ 关于时间的二阶导数)为\textbf{角加速度(angular acceleration)}, 记为 $\alpha$
\begin{equation}
\alpha = \dv[2]{\theta}{t}
\end{equation}

我们可以把刚体的绕轴转动类比质点的直线运动, 把 $\theta$, $\omega$ 和 $\alpha$ 分别类比为直线运动中的位置 $x$, 速度 $v$ 和 加速度 $a$, 因为后三个变量之间的数学关系是完全相同的. 于是我们可以立即得到匀变速转动(即 $\alpha$ 为常数)的一些公式, 如
\begin{gather}
\theta = \theta_0 + \omega t + \frac12 \alpha t^2\\
\omega_1^2 - \omega_0^2 = 2\alpha \theta
\end{gather}
也可以同样得到类似牛顿第二定律的转动公式
\begin{equation}
\tau_z = I \alpha
\end{equation}
其中 $\tau_z$ 是外力对系统的力矩在延轴方向的分量, $I$ 是一个和质量分布有关的量角转动惯量.

在以上三个标量的基础上, 我们可以定义它们的矢量形式 $\bvec \theta$, $\bvec \omega$ 和 $\bvec \alpha$, 令它们的方向为转轴的方向, 用右手定则\upref{RHRul} 来判断.

要判断刚体上任意一点的速度, 使用\autoref{CMVD_eq5}~\upref{CMVD} 即可(见\autoref{RigRot_fig2})
\begin{equation}
\bvec v = \bvec \omega \cross \bvec r = \bvec \omega \cross \bvec r_\bot
\end{equation}

\begin{figure}[ht]
\centering
\includegraphics[width=4cm]{./figures/RigRot_2.pdf}
\caption{刚体绕轴旋转时任意一点的线速度} \label{RigRot_fig2}
\end{figure}

\subsection{角动量的延轴分量与转动惯量}
要讨论刚体的定轴转动和力矩之间的关系, 我们需要先来看\textbf{角动量矢量在延轴方向的分量}\footnote{矢量 $\bvec L$ 与矢量 $\bvec \omega$ 的关系见 “惯性张量\upref{ITensr}” 词条.}. 我们把转轴的某个正方向定义为 $z$ 方向. 对于刚体上的单个质点, 角动量在 $z$ 方向的分量为
\begin{equation}
L'_{z} = \bvec L \vdot \uvec z = (\bvec r \cross \bvec p) \vdot \uvec z
\end{equation}
首先把质点的位矢在 $z$ 方向和垂直 $z$ 方向分解(称为水平方向), $\bvec r = \bvec r_z + \bvec r_ \bot$. 由于 $\bvec p$ 一直沿水平方向, 根据叉乘的几何定义, $\bvec r_z \cross \bvec p$ 也是沿水平方向, 只有 $\bvec r_ \bot \cross \bvec p$ 沿 $z$ 方向.另外, 在圆周运动中, 半径始终与速度垂直, 所以 $\bvec r_ \bot$ 始终与 $\bvec p$ 垂直.得出结论
\begin{equation}
L'_z = \abs{\bvec r_\bot} \abs{\bvec p} = m r_ \bot v = mr_ \bot ^2\omega 
\end{equation}

现在讨论刚体的角动量, 若把刚体分成无数小块, 每小块的质量分别为 $m_i$, 离轴的距离 $r_{\bot i} = \sqrt{x_i^2 + y_i^2} $, 角动量延轴分量为 $L_{iz}$ 则刚体的总角动量 $z$ 分量为
\begin{equation}
L_z = \sum_i L_{iz} = \omega \sum_i m_i r_{ \bot i}^2
\end{equation}
用定积分写成
\begin{equation}
L_z = \omega \int r_ \bot ^2 \dd{m} = \omega \int r_ \bot ^2\rho(\bvec r)  \dd{V} 
\end{equation}

定义刚体绕固定轴旋转的\textbf{转动惯量(moment of inertia)}为
\begin{equation}
I = \int r_ \bot ^2 \dd{m} = \omega \int r_ \bot ^2\rho(\bvec r) \dd{V} 
\end{equation}
注意角动量的大小不仅取决于刚体的质量分布, 还取决于转轴的位置和方向. 最后得刚体沿轴方向的角动量分量为
\begin{equation}\label{RigRot_eq5}
L_z = I \omega 
\end{equation}
可见 $L_z$ 和旋转角速度成正比.

\subsection{角加速度与力矩}
现在来看“角动量定理\upref{AMLaw}” 的\autoref{AMLaw_eq1}, 注意等号两边是矢量, 所以各个分量必须相等, 我们有
\begin{equation}\label{RigRot_eq6}
\tau_z = \dv{L_z}{t}
\end{equation}
将\autoref{RigRot_eq5} 代入\autoref{RigRot_eq6}, 并利用角加速度的定义得
\begin{equation}\label{RigRot_eq7}
\tau_z = I\alpha
\end{equation}
这就是刚体绕轴转动的动力学方程, 其形式可类比质点做直线运动时的牛顿第二定律\upref{New3} $F = ma$, $\tau_z$ 可以类比力 $F$, $I$ 类比质量 $m$, $\alpha$ 类比加速度 $a$.

\subsubsection{角动量延轴分量守恒}
在定轴转动的情况下, 根据\autoref{RigRot_eq6}, \textbf{当系统外对系统力矩的延轴分量 $\tau_z$ 为零时, 角动量的延轴分量 $L_z$ 守恒}.  这意味着刚体的角加速度为零(\autoref{RigRot_eq7}), 也就是\textbf{刚体做匀速转动或静止}. 类比到质点的直线运动就是当 $F = ma$ 中外力 $F = 0$ 时, 加速度为零, 质点做匀速运动, 动量守恒.

可见和 “匀速运动不需要外力维持” 一样, 匀速转动也并不需要延轴方向的外力矩维持. 然而在现实生活中, 正如水平直线轨道上具有初速度的滑块会由于与轨道的摩擦力, 空气阻力等各种外力下最终停止运动, 具有初始角速度的物体也会在轴承的摩擦、空气阻力等外力矩的作用下最终停止转动, 这与延轴角动量守恒并不矛盾.

\subsubsection{例题}

\begin{example}{物理摆}\label{RigRot_ex1}
如\autoref{RigRot_fig1}, 已知质量为 $M$ 的薄片绕某点的转动惯量为 $I$, 转轴到刚体质心的长度为 $r_c$, 转轴和质心的连线与竖直方向夹角为 $\theta$, 求刚体的运动方程.
\begin{figure}[ht]
\centering
\includegraphics[width=4.2cm]{./figures/RigRot_1.pdf}
\caption{物理摆} \label{RigRot_fig1}
\end{figure}

首先我们把刚体看做质点系, 以转轴为原点计算刚体的合力矩为(由于这是一个平面问题, 力矩必然垂直于该平面)
\begin{equation}\ali{
\bvec \tau &= \sum_i \bvec r_i \cross (m_i \bvec g)
= \qty(\sum_i m_i \bvec r_i) \cross \bvec g
= M \bvec r_c \cross \bvec g\\
&= Mg r_c \sin\theta
}\end{equation}
这就说明, 刚体所受力矩相当于质量为 $M$, 长度为 $r_c$ 的单摆所受的力矩. 代入\autoref{RigRot_eq7} 得刚体摆的运动方程为
\begin{equation}
I\ddot \theta = Mg r_c \sin\theta
\end{equation}
可以验证当刚体的质量全部集中在质心时($I = Mr_c^2$)我们就得到了单摆的运动方程\autoref{Pend_eq4}~\upref{Pend}.

若已知初始角度和角速度, 由刚体定轴转动的动能定理可以求得任何角度时的角速度, 详见\autoref{RigEng_ex1}~\upref{RigEng}.
\end{example}

\begin{example}{箱子倾倒}
如\autoref{RigRot_fig3} ,一长方体箱子初始时倾角为 $\theta_0$,以初速度 0 无滑动倾倒,求其运动方程.
\begin{figure}[ht]
\centering
\includegraphics[width=2.1cm]{./figures/RigRot_3.pdf}
\caption{箱子倾倒} \label{RigRot_fig3}
\end{figure}
这和\autoref{RigRot_ex1} 属于同一模型,只需把质量为 $M$ 的长方体看成质量为 $M$ 的长方形即可.事实上,长方体箱子的旋转运动和其截面长方形的旋转一致,那么只需将\autoref{RigRot_ex1} 的不规则的薄片看成规则的长方形即可.
%\addTODO{假设初始时倾角为 $\theta_0$, 本质上这和上面的刚体摆是同一个模型, 假设长方体箱子以一个底边为转轴倾倒, 不发生摩擦, 画图, 简单说明}
\end{example}

\begin{exercise}{陀螺进动的角速度}\label{RigRot_exe1}
在“角动量定理\upref{AMLaw}” 的\autoref{AMLaw_ex2} 中, 如果除 $r_0, m, g$ 外, 还知道陀螺的转动惯量为 $I$ 和陀螺的角速度 $\omega$, 试证明陀螺进动的角速度为
\begin{equation}
\Omega = \frac{mgr_0}{I\omega}
\end{equation}
注意进动角速度与陀螺倾角 $\theta$ 无关.
\end{exercise}

\subsection{垂直轴的角动量}
以上的讨论中, 我们有意避免讨论垂直轴方向的角动量分量 $L_x, L_y$. 一般情况下, 我们不能保证他们是守恒的. 在一些特殊情况下, 例如刚体的形状和质量分布关于转轴呈轴对称, 那么容易证明刚体(关于任意固定点)的总角动量 $\bvec L$ 只可能延转轴方向, 即 $L_x = L_y = 0$ 守恒. % 未完成: 证明留到惯性张量? 轴处于本征矢量的方向.

在一些不对称的情况下, 例如一个倾斜的细杆绕转轴旋转, % 链接未完成
转轴就需要对细杆施加一个不停旋转的力矩, 细杆也会对轴施加一个反力矩, 这类似于作用力和反作用力, 详见“刚体定轴转动 2\upref{RBrot2} ”.

