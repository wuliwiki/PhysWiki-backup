% test
% keys test|测试|编辑器
% license Xiao
% type Tutor

\pentry{原子单位的第二个节点(自动跳到预备知识 2)\nref{nod_a8e9}, 原子单位的默认节点(依赖整篇文章,相当于依赖最后一个节点)\nref{nod_AU},原子单位第一个节点(变灰是因为是多余的,多余是因为第二个节点默认依赖第一个节点)\nref{nod_48f1},}{nod_af87}

% 这里的 nod_af87 是本文第一个节点的 label,可以在别的文章的 \pentry{\nref{...}} 中引用

\subsubsection{一些符号}
参考 \href{https://docs.julialang.org/en/v1/manual/unicode-input/}{Julia 符号表}和 \href{https://oeis.org/wiki/List_of_LaTeX_mathematical_symbols}{LaTeX 符号表}, 以及 \href{http://www.onemathematicalcat.org/MathJaxDocumentation/TeXSyntax.htm}{MathJax 符号表}。

\begin{equation}
\cap, \bigcap, \cup, \bigcup, \vee, \wedge, \int, \iint, \iiint, \oint~
\end{equation}
\begin{equation}
\diamond, \ominus, \triangleleft, \triangleright, \Longleftarrow, \Longrightarrow, \iff, \leftrightarrow, \updownarrow, \cdots~
\end{equation}
\begin{equation}
\ddots, \top, \bot, \measuredangle~
\end{equation}

交换图
\begin{equation}
\begin{CD}
A @>f>> B \\
@VVV @VVV \\
C @>>g> D
\end{CD}~.
\end{equation}

\begin{figure}[ht]
\centering
\includegraphics[width=5cm]{./figures/a520b3ef78f5e3af.png}
\caption{小时百科} \label{fig_test_1}
\end{figure}

\subsubsection{化学式}
编辑器预览的 MathJax 3 开始支持化学式了。 但是网站用的 2 还不支持。 也不确定 LaTeX 是否支持。
\begin{equation}
\ce{SO4^2- + Ba^2+ -> BaSO4 v}~
\end{equation}

\subsubsection{公式中的链接}
公式里面居然可以用 \verb|\href|, 这个功能很强大, 但不知道 texlive 是否支持。
\begin{equation}
\href{https://wuli.wiki/online}{a}^2 + b^2 = c^2~
\end{equation}

\subsubsection{付费内容}
我们要学习的公式为
\begin{equation}\label{eq_test_1}
a = 1~
\end{equation}

\begin{example}{}
请问 $a$ 为多少?
\pay

实际答案就是
\begin{equation}\label{eq_test_2}
a + 1 = 2~
\end{equation}
\paid
\end{example}

\begin{equation}\label{eq_test_3}
b = 3~
\end{equation}


引用\autoref{eq_test_1}, 以及隐藏的\autoref{eq_test_2}。 注意\autoref{eq_test_3} 的序号在隐藏后不会改变。

引用外部公式 \autoref{eq_AU_6}~\upref{AU}
