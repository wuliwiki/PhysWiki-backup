% 牛顿求根公式(牛顿迭代法)(Octave/matlab 绘图)
% license Pub
% type Note


\begin{figure}[ht]
\centering
\includegraphics[width=8 cm]{./figures/ef3a3fca036b7c2e.pdf}
\caption{牛顿迭代法示意图} \label{fig_NWTNRT_1}
\end{figure}

Newton迭代法是求解方程(或函数零点)$f(x)=0$的经典方法。它的迭代公式是
\begin{equation}
x_{k+1} = x_k-\frac{f(x_k)}{f'(x_k)}~. \qquad k=0,1,2,...
\end{equation}
$x_0$是我们对解的初始猜测。如果迭代收敛,那么迭代的结果将收敛于方程$f(x)=0$的零点$x$。
$$\lim_{k\to\infty} x_k = x~.$$

以下的Octave/matlab代码演示了Newton法的迭代过程:
\begin{lstlisting}[language=matlab]
clc
clear

f = @(x) 0.2*x.^2; %待求解的 f(x)=0
x(1) = 4;  %初始猜测。由于octave/matlab数组从1开始,因此我们用x1表示x0

hold on
axis equal

for i = 1:100
  if i ~=1
    dx = 0.001;
    fx = (f(x(i-1)+dx) - f(x(i-1)))/dx; %数值近似函数的导数值。更好的办法是使用三点牛顿法。

    x(i) = x(i-1) - f(x(i-1))/fx;
    line([x(i) x(i-1)],[0 f(x(i-1))])
  end

  line([x(i) x(i)],[0 f(x(i))],'color','r')
  scatter(x(i),0);

  if i~=1 && abs(x(i) - x(i-1)) < 0.01
    break;
  endif
end

##axis([0 5 0 5])
if x(i) < x(1)
  _x = x(i)-0.5:0.01:x(1)+0.5;
else
  _x = x(1)-0.5:0.01:x(i)+0.5;
end

_y = f(_x);
plot(_x, _y,'k')
\end{lstlisting}
