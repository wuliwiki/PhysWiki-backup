% 高斯波包
% keys 高斯分布|波包|光学|量子力学
% license Xiao
% type Tutor

\begin{issues}
\issueDraft
\end{issues}

\pentry{波包\upref{WvPck}, 高斯分布\upref{GausPD}}

\begin{figure}[ht]
\centering
\includegraphics[width=14.25cm]{./figures/98b376107396a32e.pdf}
\caption{高斯波包(\autoref{eq_GausPk_1} ),蓝色为实部,红色为虚部, $x_0 = 0$, $A_0 = 1$, $a = 1/20$, $k_0 = 5$。} \label{fig_GausPk_1}
\end{figure}

\footnote{参考 Wikipedia \href{https://en.wikipedia.org/wiki/Wave_packet}{相关页面}。}\textbf{高斯波包(Gaussian wave packet)}是指轮廓为高斯分布的波包, 在光学和量子力学中有重要应用。 高斯波包用复函数表示为($A_0$ 为复数)
\begin{equation}\label{eq_GausPk_1}
f(x) = A_0 \E^{-a(x-x_0)^2}\E^{\I k_0 x}~,
\end{equation}

FWHMI (光强半高宽)是多少? 即 $f(x)^2$ 大于峰值一半时的宽度。令
\begin{equation}
\E^{-2a\Delta x^2} = 1/2~,
\end{equation}
得
\begin{equation}
\mathrm{FWHMI} = 2\Delta x = \sqrt{\frac{2\ln 2}{a}}~.
\end{equation}

\subsection{高斯和 cos2 波包比较}
$\cos^2$ 波包也叫 $\sin^2$ 波包,比起高斯波包,它的优点是存在明确的范围。
\begin{figure}[ht]
\centering
\includegraphics[width=13cm]{./figures/925904dd88cd4821.pdf}
\caption{高斯波包和 cos2 波包的对比} \label{fig_GausPk_2}
\end{figure}

\subsection{频谱}
\pentry{傅里叶变换(指数)\upref{FTExp}}
(未完成:哪里有介绍频谱的概念?)

要求\autoref{eq_GausPk_1} 的傅里叶变换 $g(k)$, 由\autoref{ex_FTExp_1}~\upref{FTExp}以及傅里叶变换性质\autoref{eq_FTExp_4}~\upref{FTExp}和\autoref{eq_FTExp_7}~\upref{FTExp}得
\begin{equation}
g(k) = A_0\sqrt{\frac{\pi}{a}} \exp[-\frac{(k-k_0)^2}{4a}]~.
\end{equation}
