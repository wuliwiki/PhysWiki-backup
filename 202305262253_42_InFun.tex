% 可积函数
% keys 可积函数|性质|种类
\pentry{定积分存在条件\upref{Rieman}}
可积函数的定义已经在\autoref{def_DInt_3}~\upref{DInt}给出,严格来说,这里的可积函数称为黎曼可积函数。本词条将给出一些常见的可积函数和可积函数的一些性质。
\subsection{一些可积函数}
\begin{theorem}{连续函数必可积}
若 $f(x)$ 是区间 $[a,b]$ 上的连续函数,则 $f(x)$ 可积。
\end{theorem}
\textbf{证明:}
由\autoref{sub_conff_1}~\upref{conff}的康托尔定理,连续函数在闭区间上必一致连续(\autoref{sub_conff_2}~\upref{conff})。即对任意 $\epsilon>0$ 可找到 $\delta>0$,使得当 $[a,b]$ 分成长度 $\Delta x_i<\delta$ 的若干部分时,所有的 $\omega_i<\epsilon$( $\omega_i$ 是对应区间上的振幅(\autoref{def_Rieman_1}~\upref{Rieman})),由此
\begin{equation}
\sum_{i=0}
\end{equation}


\textbf{证毕!}

\begin{theorem}{有限间断点的连续函数必可积}
若 $f(x)$ 是区间 $[a,b]$ 上除在有限个点外都连续的函数,则 $f(x)$ 可积。
\end{theorem}

\begin{theorem}{单调有界函数必可积}
若 $f(x)$ 是区间 $[a,b]$ 上的单调有界函数,则 $f(x)$ 可积。
\end{theorem}