% 复旦大学 2010 量子真题
% license Usr
% type Note

\textbf{声明}:“该内容来源于网络公开资料,不保证真实性,如有侵权请联系管理员”


1. 取无限深方势阱的中心为坐标原点,势阱宽为$a$,求粒子的能级及波函数。

2. 估算一维谐振子的基态能量

3. 估算类氢原子的基态能量

4. 利用 $[a,a^+]=1$, \quad $[a,a]=[a^+,a^+]=0$, \quad $a \left| 0 \right\rangle =0$ 证明 $\left| n \right\rangle = \frac{(a^+)^n}{\sqrt{n!}} \left| 0 \right\rangle$

5. 两个自旋为$\frac{1}{2}$,质量为$m$的全同粒子,自旋平行,处于一个边长为$a>b>c$的长方形盒子中,粒子间的相互作用势为 $V=A\delta (\mathbf{r}_1 - \mathbf{r}_2)$;体系处于与下列条件相容的最低能级,试用一级微扰论计算体系能量

\begin{itemize}
    \item 1)两个粒子是自旋$\frac{1}{2}$的全同粒子
    \item 两个粒子是自旋$\frac{1}{2}$的非全同粒子
    \item 两个粒子自旋为零
\end{itemize}

6. 有一个自旋$\frac{1}{2}$,磁距$\mu$,电荷0的粒子,置于磁场$B$中,开始时($t=0$)磁场沿z方向,$\mathbf{B}= \left( 0, 0, B_0 \right)$,粒子处于$\sigma_z$的本征态$\left| \uparrow \right\rangle$,即$\sigma_z = 1$,$t>0$时,再加上沿x方向的较弱的磁场$\mathbf{B}_1 = \left( B_1, 0, 0 \right)$,从而$\mathbf{B} = \mathbf{B} + \mathbf{B}_1 = \left( B_1, 0, B_0 \right)$

求t>0时粒子的自旋态,以及测得自旋“向上”($\sigma_z=1$)的概率

