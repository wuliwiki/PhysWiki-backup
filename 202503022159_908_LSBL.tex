% 罗素悖论(综述)
% license CCBYSA3
% type Wiki

本文根据 CC-BY-SA 协议转载翻译自维基百科\href{https://en.wikipedia.org/wiki/Russell\%27s_paradox}{相关文章}。

在数学逻辑中,罗素悖论(也称为罗素反义命题)是由英国哲学家和数学家伯特兰·罗素于1901年提出的一个集合论悖论。罗素悖论表明,任何包含不受限制的理解原理的集合论都会导致矛盾。根据不受限制的理解原理,对于任何足够明确定义的属性,都存在一个集合,包含所有且仅包含具有该属性的对象。设\(R\)为所有不属于自身的集合的集合(这个集合有时被称为“罗素集合”)。如果\(R\)不属于自身,则根据其定义,它必须属于自身;然而,如果它属于自身,那么它就不属于自身,因为它是所有不属于自身的集合的集合。由此产生的矛盾就是罗素悖论。用符号表示如下:

设 
\[
R = \{ x \mid x \notin x \}~
\]
那么
\[
R \in R \iff R \notin R~
\]
罗素还展示了该悖论的一个版本可以在德国哲学家和数学家戈特洛布·弗雷格所构建的公理化系统中推导出来,从而破坏了弗雷格试图将数学归约为逻辑的尝试,并对逻辑主义的程序提出了质疑。避免该悖论的两种有影响力的方式都在1908年提出:罗素的类型理论和泽梅洛集合论。特别是,泽梅洛的公理限制了无限理解原理。随着亚伯拉罕·弗兰克尔的进一步贡献,泽梅洛集合论发展成了现在标准的泽梅洛–弗兰克尔集合论(当包含选择公理时,通常称为ZFC)。罗素和泽梅洛解决悖论的主要区别在于,泽梅洛修改了集合论的公理,同时保持了标准的逻辑语言,而罗素则修改了逻辑语言本身。ZFC的语言,在托拉夫·斯科勒姆的帮助下,最终被证明是第一阶逻辑的语言。[4]

该悖论早在1899年就由德国数学家恩斯特·泽梅洛独立发现。[5]然而,泽梅洛并未发表这一想法,这一想法仅为大卫·希尔伯特、埃德蒙德·胡塞尔和哥廷根大学的其他学者所知。在1890年代末,现代集合论的创始人乔治·康托尔就已经意识到他的理论会导致矛盾,他通过信件告诉希尔伯特和理查德·德德金德。[6]
\subsection{非正式的呈现}  
大多数常见的集合都不是它们自己的成员。我们称一个集合为“正常的”,如果它不是它自己的成员;如果它是它自己的成员,则称其为“非正常的”。显然,每个集合必须是正常的或非正常的。例如,考虑平面上的所有正方形的集合。这个集合本身不是平面上的一个正方形,因此它不是它自己的成员,因此是正常的。相比之下,包含所有非正方形的补集本身不是平面上的正方形,因此它是它自己的成员,因此是非正常的。

现在我们考虑所有正常集合的集合\(R\),并尝试确定R是正常的还是非正常的。如果\(R\)是正常的,那么它将包含在所有正常集合的集合中(即它自身),因此它是非正常的;另一方面,如果\(R\)是非正常的,它就不包含在所有正常集合的集合中(即它自身),因此它是正常的。这导致了一个结论:\(R\)既不是正常的,也不是非正常的:这就是罗素悖论。

\subsection{正式的呈现}  
“天真集合论”这个术语有多种用法。在一种用法中,天真集合论是一个形式理论,使用一阶语言并具有二元非逻辑谓词  
\(\in\),包括外延公理:
\[
\forall x\,\forall y\,(\forall z\,(z \in x \iff z \in y) \implies x = y)~
\]
以及不受限制的理解公理模式:
\[
\exists y \forall x (x \in y \iff \varphi(x))~
\]
对于任何谓词\(\varphi\),其中\(x\)是\(\varphi\)中的自由变量。将  
\(x \notin x\)代入\(\varphi(x)\)得到:
\[
\exists y \forall x (x \in y \iff x \notin x)~
\]
然后通过存在性实例化(重用符号\(y\))和全称实例化,我们得到:
\[
y \in y \iff y \notin y~
\]
这是一个矛盾。因此,这种天真集合论是不一致的。[7]
\subsection{哲学意义}  
在罗素悖论(以及同时期发现的其他类似悖论,如布拉利-福尔蒂悖论)之前,关于集合的一个普遍观念是“外延集合的概念”,正如冯·诺伊曼和莫根斯坦所叙述的:

集合是一个任意的对象集合,对于这些对象的性质和数量没有任何限制,这些对象是集合中的元素。集合的元素构成并决定了集合本身,元素之间没有任何顺序或关系。

特别地,集合和适当类作为对象的集合之间没有区分。此外,集合中每个元素的存在被视为该集合存在的充分条件。然而,像罗素悖论和布拉利-福尔蒂悖论这样的悖论通过一些对象集合的例子,展示了这种集合观念的不可行性,尽管所有这些对象都是存在的,但它们并不能形成集合。
\subsection{集合论的回应}  
根据经典逻辑的爆炸原理,从矛盾中可以证明任何命题。因此,像罗素悖论这样的矛盾在公理化集合论中的存在是灾难性的;因为如果任何公式都可以被证明为真,它就破坏了传统的真与假的意义。此外,由于集合论被视为所有其他数学分支公理化发展的基础,罗素悖论威胁到了整个数学的基础。这促使了20世纪初大量研究,旨在发展一种一致的(无矛盾的)集合论。

1908年,恩斯特·泽梅洛提出了一种集合论的公理化方法,通过用较弱的存在公理(如他的分离公理,Aussonderung)替换任意集合理解,从而避免了天真集合论中的悖论。(避免悖论并非泽梅洛的初衷,他的目的是记录他在证明良序定理时所使用的假设。)1920年代,亚伯拉罕·弗兰克尔、托拉夫·斯科勒姆和泽梅洛本人对这一公理化理论进行修改,最终得出了称为ZFC的公理化集合论。自泽梅洛的选择公理不再具有争议后,这一理论得到了广泛接受,ZFC一直保持到今天,成为公认的标准公理化集合论。

ZFC并不假设对于每一个属性,都存在一个包含所有满足该属性的事物的集合。相反,它断言,给定任何集合\(X\),\(X\)的任何子集,只要使用一阶逻辑可定义,就存在。由上文罗素悖论定义的对象\(R\)无法作为任何集合\(X\)的子集构造,因此在ZFC中它不是一个集合。在ZFC的某些扩展中,特别是在冯·诺伊曼–伯奈斯–哥德尔集合论中,像\(R\)这样的对象被称为适当类。

ZFC 对类型没有明确说明,尽管其累积层次结构有类似于类型的层次概念。泽尔梅洛本人从未接受斯科勒姆使用一阶逻辑语言来表述 ZFC。正如 José Ferreirós 所指出的,泽尔梅洛坚持认为,“用于区分子集的命题函数(条件或谓词)以及替代函数,可以是‘完全任意的’[ganz beliebig]”;现代对这一声明的解释是,泽尔梅洛希望引入高阶量化,以避免斯科勒姆悖论。大约在 1930 年,泽尔梅洛还引入了(显然是独立于冯·诺依曼的)基础公理,从而——正如 Ferreirós 所观察到的——“通过禁止‘循环’和‘无根’集合,它 [ZFC] 融入了 TT [类型理论] 的一个关键动机——即‘论证类型的原则’”。泽尔梅洛偏好的这个包括基础公理的二阶 ZFC 允许了丰富的累积层次。Ferreirós 写道:“泽尔梅洛的‘层次’本质上与哥德尔和塔尔斯基提出的简单类型理论 [TT] 当代版本中的类型相同。我们可以将泽尔梅洛发展其模型的累积层次结构描述为一个允许无限类型的累积类型理论的宇宙。(一旦我们采用无定义的观点,放弃类是构造出来的这个想法,接受无限类型就不再是反自然的了。)因此,简单类型理论和 ZFC 现在可以看作是‘谈论’本质上相同的预期对象的系统。主要区别在于,类型理论依赖于强大的高阶逻辑,而泽尔梅洛使用了二阶逻辑,ZFC 也可以用一阶形式表述。累积层次结构的一阶‘描述’要弱得多,这一点通过可数模型的存在(斯科勒姆悖论)得到了体现,但它也享有一些重要的优势。”[10]