% 朴素集合论
\subsection{公理,定义和定理}
我们是不可能证明所有的命题的,所以任何理论必须有一个出发点,也就是一些基础命题.这些基础命题本身不可证明,它们决定了理论的样貌,理论中一切其它命题都是由这些命题根据逻辑推演得到的.这样的基础命题被称为一个理论的\textbf{公理}(axiom).

有了公理系统以后,我们还需要明确所讨论的对象是什么.比如我用了皮亚诺公理来定义小学四则运算,那么为了讨论“1+1等于几”这样的问题,我首先需要明确“1”和“+”具体指什么.用来明确概念的陈述句,被称为\textbf{定义}(definition).如果说公理系统是创建了一个宇宙的基础参数的话,那么定义就是在给这个宇宙里已经自然存在的事物进行命名,这样才能讨论这些事务.

最后,任何一个理论的绝大部分内容都是在使用基础命题来进行推演,看哪些命题能成立.这些成立的命题,就叫做\textbf{定理}(theorem). 有时候,根据定理作用的不同,我们也可能称其中一些为\textbf{引理}(lemma)、\textbf{推论}(corollary)等.所有定理加在一起就构成了整个理论.

不同的公理系统可能推演出相同的命题,也可能推演出彼此矛盾的命题,更可能存在一些无法判断是否成立的命题.一个公理系统中所无法判断是否成立的命题,就叫做独立于这个公理系统的命题.如果两个公理系统能够推演出完全一样的命题(定理),那么这两个公理系统就是等价的.如果公理系统A能够推演出公理系统B的一切定理,但是B不能推出A中的一切定理,即A能推演出的一些定理实际上是独立于B的命题,那么可以认为是公理系统A包含了公理系统B.\textbf{在阅读本段话时,请注意命题和定理的区别:定理是在给定公理体系下能被推演出来的命题.}

以上表述是数学的表达方式.在物理学中,\textbf{公理}和\textbf{定理}可以分别被翻译成\textbf{定律}和\textbf{现象}. 

\subsection{集合}

对于物理学习而言,集合论没必要从公理角度来严格理解,所以在此给出的是朴素集合论的解释.

集合(set)是由元素(element)组成的.任何事物和概念都可以成为元素,任何不同的元素都可以放在一起,构成一个集合.可以说,如果我们划定一个讨论的范围,那么这个范围就是一个集合,范围涉及到的事物和概念就是这个集合当中的元素.

表达一个集合的方式有多种,最简单的方式是列出所有集合中的元素.在数学中规定的语法规范是用大括号“\{\}”来列举集合中的一切元素,以逗号“,”隔开彼此.比如,\{猪,牛,狗,羊,猫\}构成了一个具有五个元素的集合,$\{1,2,3,4,\cdots\}$则是全体正整数的集合.第二个例子并没有显然地列举出所有正整数,只是用省略号表达了这个意思;也就是说,表达一个集合的方式并没有死板的规定,只要能让读者理解就可以了.

另一种常见的表达集合的方式是确定一个规则,语法规范是“\{$x|x$需要满足的条件\}”.比如全体正整数的集合,也可以写为\{$x|x$是一个正整数\}.如果有多个条件,也可以列在一起,比如全体正整数的集合:\{$x|x$是一个正数,且$x$是一个整数\}.特别地,如果某条规则是“x属于某集合”,我们通常会将这个条件写到单竖线的前面,如全体正整数的集合:\{$x\in\mathbb{Z}|x$是一个正数}. 这里,$\in$是一个简写的符号,$A\in B$等于说“A是B的元素”.

如果集合A的元素都是集合B的元素,那么称A是B的子集.一切集合都是自身的子集.如果A是B的子集但又和B不同,也就是说A没有包含B的所有元素,那么称A是B的真子集.

\subsection{属于和包含}
为了简化表达,数学家把集合论中常用的动词表示成简略的形式.

$A\in B$或者$B\ni A$等价B于“A是B的元素”,表达“属于”关系. 

$A\subset B$ 或者 $B\supset A$等价于“A是B的子集”,表达“包含”关系. 

注意区分这两个情况,前一个情况中A是B的元素,后一个情况中A是B的子集.另外,集合本身也可以是别的集合的元素,元素的概念没有限定,任何事物和概念都可以成为元素,包括集合.

其它形式的子集符号,如$\subseteq,\supseteq$等,在不同的文献中可能有不同的含义,所以一般没有特殊说明时不会使用.


\subsection{集合运算}

集合间可以互相操作,生成新的集合,这种操作被称为集合间的运算(operation).$\cap$表示两个集合的交,意思是将两个集合中共有的元素提取出来,组成一个新的集合.比如说,$\mathbb{N^+}$表示全体自然数的集合,$\mathbb{R^+}$表示全体正实数的集合,$\mathbb{Z}$表示全体整数的集合,那么显然我们可以有$\mathbb{N^+}=\mathbb{R^+}\cap\mathbb{Z}$. 多个集合$A_i$的交集,可以写为$A_0\cap A_1\cap A_2\cap A_3\cdots$,也可以用一个大号的交集符号简记为$\bigcap A_i$,表示“所有形式为$A_i$的集合的交集”.

类似地,将两个集合中都有的元素提取出来,组成一个新的集合的操作,被称为集合的并,用符号$\cup$, 和$\bigcup$表示.注意,如果两个元素中有相同元素,那么在并集中这个元素只出现一次.这是因为我们关心的是每个元素是否出现在集合中,计算集合元素数量时也不会重复计算同一个元素.这是一个并集的例子:{猪,牛,狗,羊,猫\}$\cup\mathbb{N^+}$=$\{$猪,狗,猫,牛,羊, 1,2,3,4,$\cdots\}$. 注意,列举时元素的顺序也不影响集合的本质.

交和并是最基础的集合运算.
