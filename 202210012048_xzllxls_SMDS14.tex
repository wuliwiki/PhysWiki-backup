% 上海海事大学 2014 年数据结构
% 上海海事大学 2014 年数据结构

\subsection{一.判断题(本题10分,每小题1分)}

1、若某顺序表采用顺序存储结构,每个元素占$10$个存储单元,首地址为$200$,则下标为$11$(第$12$个)的元素的存储起始地址为$320$.

2、若对线性表进行的主要操作不是插入和删除,则该线性表宜采用顺序存储结构.

3、对一个空栈按$a,b,c,d,e,f,g$顺序依次读入,经过多次入栈和出栈的操作后,能得到按$f,e,g,d,a,c,b$顺序的出栈序列.

4、假定在顺序表中每个位置插入的概率相同,向一个有$64$个元素的顺序表中插入一个新元素并保持原来顺序不变,平均要移动$33$个元素.

5、含有$3$个结点(元素值均不相同)的二叉排序树共有$30$种.

6、$n$个顶点的连通图至少有$n-1$条边.

7、在无向图$G$的邻接矩阵$A$中,若$A[i][j]$等于$1$,则$A[j][i]$等于$0$.

8、采用顺序检索法在一个有$123$个元素的有序顺序表中查找,若每个元素的查找概率相等,则成功检索的平均查找长度$ASL$为$61$.

9、在散列存储中,装载因子$a$的值越大,发生冲突的可能性就越大.

10、快速排序是一种稳定的排序方法.

\subsection{二.填空题(本题30分,每空2分)}

1.分析下列程序段,其时间复杂度分别为:( (1) )、( (2) ).
\begin{lstlisting}[language=cpp]
i=m=0;
while (m<n) {
    i++; s+=i;
}

m=0;
for(i=1; i<=n; i++)
    for(j=2*i; j<=n; j++)
        m++;
\end{lstlisting}

2.广义表$A=(a, (a, b), (i,j),k),d,e)$的长度是( (3) ),深度是( (4) ),取表头和表尾函数分别为head()和tail(),则head(tail(head(tail(A))))=( (5) ), 而从表中取出原子项$j$的运算为( (6) ).

3.有一个二维数组A.0.[2..9], 每个数组元素占用8个存储单元,并且A[2][5]的存储地址为2080,若按行序为主序方式存储,数组元素A[4][6]的存储地址是_ (7)

4.一棵完全二叉树有$600$个结点,则它的深度是( (8) ).

5.已知一个图采用邻接矩阵表示,计算第$i$个结点的入度的方法是( (9) ).

6.一个图的边集为{<A, B>,<A,C>,<A,E>,<B,C>, <B,D>, <C,D>, <E,B>,<E, D>},从顶点A出发进行深度优先搜索遍历访问顶点顺序为( (10) ), 从顶点A出发进行广度优先搜索遍历访问顶点顺序为( (11) ), 对该图进行拓扑排序得到的顶点序列为( (12) ).

7.对$12$个元素的序列进行直接插入排序时,最少的比较次数为( (13) ).

8.( (14) )排序方法采用的是二分法思想,在( (15) )情况下最不利于发挥其长处.

