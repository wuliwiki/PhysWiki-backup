% 拉莫尔进动
% 自旋1|磁场中的电子2

\textbf{拉莫尔(Larmor)进动}假定有一个自旋为$1/2$的粒子静止处于均匀磁场:
\begin{equation}
\bvec B =B_0\hat{z}
\end{equation}
静止在磁场中的带点自旋粒子的哈密顿为:
\begin{equation}
H = -\gamma\bvec B\cdot S
\end{equation}
其中$\gamma$为磁旋比 \upref{BohMag},由此可得:
\begin{equation}
\bvec H = -\gamma B_0S_z=-\frac{\gamma B_0\hbar}{2}\pmat{1&0\\0&-1}
\end{equation}
注意到$\bvec H$有着和$S_z$相同的本征矢:
\begin{equation}
\chi_+ = \pmat{1\\0}:E_+=-(\gamma B_0\hbar)/2; \ \chi_- = \pmat{0\\1}:E_-=+(\gamma B_0\hbar)/2; 
\end{equation}
显然和经典情况一致,在偶极矩平行于磁场时能量时最低的.

由于哈密顿量和时间无关,因此含时薛定谔方程为:
\begin{equation}
\I \hbar \pdv{\chi}{t}=\bvec H\chi
\end{equation}
它的一般解可以被表示为定态的线性组合:
\begin{equation}
\chi(t)=a\chi_+\E^{-\I E_+t/\hbar}+b\chi_-\E^{-\I E_-t/\hbar}=\pmat{a\E^{\I \gamma B_0t/2}\\b\E^{-\I \gamma B_0t/2}}
\end{equation}
其中的常数$a,b$是由初始条件所确定的:
\begin{equation}
\chi(0)=\pmat{a\\b}
\end{equation}
注意到这里$|a|^2+|b|^2=1$,为不失一般性,我们将:
\begin{equation}
a=\cos(\frac{\alpha}{2});\ b= \sin(\frac{\alpha}{2})
\end{equation}
