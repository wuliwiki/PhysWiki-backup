% 字符串哈希
% 字符串哈希|哈希|数据结构|C++

字符串哈希和普通的哈希算法类似,字符串哈希是把一个很长的字符串变成一个整数,这样的好处是:如果想比较两个很长的字符串是否相等时,普通算法是遍历一遍整个字符串,如果其中一个字符串的字符和另一个字符串的字符不等,则两个字符串不一样.时间复杂度为 $O(N)$.而用字符串哈希的话可以直接比较两个字符串的哈希值是否相同,时间复杂度为 $O(1)$.下面介绍一种哈希方式可以把任意一个字符串变成一个非负整数,并且哈希冲突的概率几乎为 $0$.

举个例子:
先把一个字符串 $\text{abcfea}$ 变成 $p$ 进制数,将字符 $a \sim z$ 映射成 $1 \sim 26$,所以原来的字符串就变成了:$(123651)_p$,转化为十进制就为:\begin{equation}

\end{equation}