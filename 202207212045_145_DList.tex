% 双链表
% 双链表|数据结构|链表|C++

本文对应了上一篇文章,这里讲解\textbf{双链表}.

双链表在竞赛中用的不多,通常是因为需要优化某些问题而使用双链表.

\begin{figure}[ht]
\centering
\includegraphics[width=14.25cm]{./figures/DList_1.png}
\caption{双链表示意图} \label{DList_fig1}
\end{figure}

双链表上的每个结点都有 $4$ 个值,$\mathtt{Lnext}$ 指针表示它左边的结点的下标,$\mathtt{R-next}$ 指针表示它右边的结点的下标,其他的数组和变量和单链表\upref{List}的存储代表一个意思.

用 C++ 的指针和结构体来实现双链表:
\begin{lstlisting}[language=cpp]
struct Double_List
{
    int value;
    Double_List *prev, *next;  // 左指针和右指针
};
\end{lstlisting}

这里来详细的讲解一下如何使用数组来模拟双链表.

数组模拟双链表需要这么几个数组和变量:
\begin{lstlisting}[language=cpp]
int value[N], Lnext[N], Rnext[N], idx;
\end{lstlisting}

双链表的基本操作有:\begin{enumerate}
\item 在第 $k$ 个插入的数左侧插入一个数;
\item 在第 $k$ 个插入的数右侧插入一个数;
\item 将第 $k$ 个插入的数删除
\end{enumerate}

$\text{insert}$ 操作: