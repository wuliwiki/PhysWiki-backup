% 金属的变形(科普)

\subsection{变形}
正如我们拉伸一根弹簧,弹簧会变形一样;当我们拉伸一根金属棒时,金属棒也会变形.只不过由于金属棒的“弹性系数”很大,以正常人的手劲一般拉不出看得见的变形.

\begin{example}{}
\begin{figure}[ht]
\centering
\includegraphics[width=12cm]{./figures/MetDfm_1.png}
\caption{框架结构}} \label{MetDfm_fig1}
\end{figure}
与弹簧类似,金属结构提供的支持力也源自金属的细微变形...只要在安全的范围内.
\end{example}

根据变形的性质,变形一般分为两类:弹性变形与塑性变形.顾名思义,弹性变形后,撤去外力后金属的形状能恢复原样;而塑性变形后,即使撤去外力,金属的形状也不能恢复.塑性变形只在外力大到超过一定境界时才发生.

弹性变形与塑性变形不是非此即彼,而可以相辅相成.一次变形可能既包括弹性变形也包括塑性变形.

\begin{figure}[ht]
\centering
\includegraphics[width=10cm]{./figures/MetDfm_2.png}
\caption{弹性变形与塑性变形示意图} \label{MetDfm_fig2}
\end{figure}

那么,为什么会有两种类型的变形呢?这就涉及到变形的微观原理了.大体而言,弹性变形时原子间的“键”被拉伸,但原子并没有运动到新的位置,因而撤去外力后原子可以回到原位,体现为形状恢复原样;
\begin{figure}[ht]
\centering
\includegraphics[width=8cm]{./figures/MetDfm_11.png}
\caption{弹性变形示意图} \label{MetDfm_fig11}
\end{figure}
而塑性变形后,原子间原本的键已经被破坏、原子运动到了新的位置,并形成了新的键.因此,塑性变形后金属的形状发生永久改变.
\begin{figure}[ht]
\centering
\includegraphics[width=8cm]{./figures/MetDfm_12.png}
\caption{塑形变形示意图} \label{MetDfm_fig12}
\end{figure}
\subsection{塑性变形机制}
\pentry{金属材料结构(科普)\upref{MetInt}}
%接下来,我们更细致地探讨一下塑性变形.此处简要探讨\textsl{滑移},这是塑性形变的主要机制之一.本文以单晶体为例,即金属中只有一个硕大的晶粒,原子的排列方向都一致.

\subsubsection{位错的运动}
%或许你还记得位错\upref{MetInt}的概念.金属的滑移变形与位错有着密不可分的关系;位错理论的提出正是为了解释金属的塑性变形.
现在,我们来更具体地思考一下,塑性变形时金属经历了什么.假设你有一块完整的晶体,现在你要施加外力使其塑性变形.看起来,为了使原子运动到新位置,你得大力出奇迹、破坏一整面原子间的键.
\begin{figure}[ht]
\centering
\includegraphics[width=14cm]{./figures/MetDfm_13.png}
\caption{塑性变形...?} \label{MetDfm_fig13}
\end{figure}
毫无疑问,这需要非常大的能量(\textsl{与手劲})!可事实上,现实中的金属强度远低于此(大概是按照这种理论计算的1/100至1/1000).怎么回事呢?

或许你还记得位错\upref{MetInt}的概念,而位错理论的提出正是为了解释金属的塑性变形.假定金属中存在一个位错,这时,奇迹发生了:由于位错的存在,现在上下部分相对运动时,只需断一列的键而不是一面的键.这大大降低了原子运动的难度.当位错运动离开晶体时,就会在晶体边缘留下一个台阶.
\begin{figure}[ht]
\centering
\includegraphics[width=14cm]{./figures/MetDfm_3.png}
\caption{位错的运动1} \label{MetDfm_fig3}
\end{figure}

\begin{figure}[ht]
\centering
\includegraphics[width=8cm]{./figures/MetDfm_4.png}
\caption{位错的运动2} \label{MetDfm_fig4}
\end{figure}

在\href{https://5b0988e595225.cdn.sohucs.com/q_70,c_zoom,w_640/images/20191101/b54fbd26dc14497d8607965e6b395c96.gif}{这里}可以看到一张动图,\href{https://m.sohu.com/a/350972084_120056486/}{原文}(站外链接)

这有点像蠕动的毛毛虫,或者地板上鼓起包的地毯.铺平了的地毯很难被直接拖动,但地毯上鼓起的凸包就容易运动地多.
\begin{figure}[ht]
\centering
\includegraphics[width=14cm]{./figures/MetDfm_5.png}
\caption{位错的运动3}} \label{MetDfm_fig5}
\end{figure}

看起来,位错大大降低了滑移的难度!事实上,位错运动是滑移机制的主要因素.

\subsubsection{位错运动的方向:滑移系}
那么,位错能够在任意的面或方向上运动吗?答案显然又是...否定的.\textbf{一般而言,位错只在它偏好的运动方向与运动面上滑移.}这里的“面”指的是一系列法向量特定的相互平行的面,不是专指具体的某一个面.

\begin{theorem}{滑移系,滑移面,滑移方向}
一般而言,滑移在特定的面上,沿着特定的方向进行.

这些特定的面与方向分别称为滑移面与滑移方向;一种滑移面与其上的一种滑移方向称为一个滑移系.

\end{theorem}
\begin{figure}[ht]
\centering
\includegraphics[width=6cm]{./figures/MetDfm_6.png}
\caption{一种可能的滑移系示意图,切面为滑移面,黑色线为滑移方向} \label{MetDfm_fig6}
\end{figure}

那么,位错偏好哪一些滑移面或滑移方向呢?一般而言,\textbf{滑移面是晶体的密排面,而滑移方向是晶体的密排方向}(密排面可以理解为晶体中原子堆积得最密的面).具体的滑移系与晶胞的种类有关\footnote{滑移系还与温度有关,有些额外的滑移系将在高温下被激活.因此高温时,晶体的变形能力更好.}.当然,由于一种晶体中密排面、方向不止一种,因此滑移系也不止一种.

\begin{example}{钛强了}
\begin{figure}[ht]
\centering
\includegraphics[width=5cm]{./figures/MetDfm_10.png}
\caption{密排六方晶胞示意图,\href{https://www.geogebra.org/m/xrzejabt}{一个可交互的模型}(站外链接)} \label{MetDfm_fig10}
\end{figure}
为什么常听说金属钛的强度高?一部分因为钛是密排六方(HCP)结构,而密排六方的滑移系很少,因而位错难以运动.
\end{example}

实际上可以观察到,单晶金属变形时,往往沿着特定方向形成台阶状结构.这些台阶状结构(图中所示“滑移带”)就是一系列位错滑移后留下的小台阶之和.
\begin{figure}[ht]
\centering
\includegraphics[width=14cm]{./figures/MetDfm_7.png}
\caption{金属的宏观塑性变形} \label{MetDfm_fig7}
\end{figure}

\subsubsection{位错运动的条件:临界分切应力}
即使位错大幅降低了滑移的难度,也得有足够大的外力才能驱动位错运动;更准确的说,是\textbf{位错滑移方向上的力}得足够大.满足这一条件的最小分力被称为临界分切应力.

\begin{definition}{临界分切应力}
临界分切应力指晶体恰好开始滑移时,滑移方向上的分切应力.
\end{definition}

\begin{figure}[ht]
\centering
\includegraphics[width=12cm]{./figures/MetDfm_8.png}
\caption{外力的分解.注意只有滑移方向上的分力是“有效”的} \label{MetDfm_fig8}
\end{figure}

如图所示,外力F在滑移方向上的分力\footnote{$\lambda$与$\phi$之和不一定为$90^\circ$,他们是相对独立的变量:)} \footnote{在材料科学中,往往更关注\textsl{应力},即“一点处的力的大小”,类似于压强的概念,因此要除以力作用面的面积.正应力(垂直于截面)常记为$\sigma=\dv{F}{A}$,切应力(平行于截面)记为$\tau=\dv{F}{A}$.对于均匀拉伸,可以认为拉力在截面是均匀分布的,因此$\sigma=\frac{F}{A}$. \textsl{霍金曾经说过,文章每多一个公式,就少一半读者,但我没霍金那么厉害,我只能迫不得已地在这里放一个公式(}}为
\begin{equation}
\tau=\frac{F \cos \lambda}{A/{\cos \phi}}=\frac{F}{A}{\cos \lambda}{\cos \phi}
\end{equation}
若要达到临界切分应力为$\tau_k$,那么外力至少为 $F_s=\frac{\tau_k A}{\cos \lambda \cos \phi} $ 
%若写为$\sigma_s=\frac{F_s}{A}$为材料的(屈服)强度或弹性限度\footnote{咬文嚼字地说,有一些细微的不同},指材料在塑性变形前能承受的最大应力.

其中,$(\cos \lambda \cos \phi)$被称为取向因子,与外力的作用方向、晶体中原子的排列方向、位错运动的滑移系等有关.可见,外力的最小大小与取向因子紧密关联,这强烈明示了晶体材料的\textbf{各向异性},即沿不同的方向,晶体的性质不同.\footnote{由于现实材料多为多晶材料,原子排列方式不完全一致,因此各向异性性质被削弱.}

\begin{figure}[ht]
\centering
\includegraphics[width=4cm]{./figures/MetDfm_9.png}
\caption{晶体的各向异性示意图} \label{MetDfm_fig9}
\end{figure}
