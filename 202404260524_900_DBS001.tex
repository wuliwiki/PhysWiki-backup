% 初步认识数据库系统
% keys 数据,数据库,数据库系统
% license Xiao
% type Tutor

\begin{issues}
\issueDraft
\issueOther{本文搬运的内容可能存在版权问题}
\end{issues}

参考资料:\href{http://t.csdn.cn/SFcuR}{《数据库系统概论》第五版学习笔记}

% \subsection{四大基本概念}
% \subsubsection{数据–Data:}
% 数据(Data)是数据库中存储的基本对象\\
% \begin{enumerate}
% \item 数据的定义\\
% 描述事物的符号记录\\
% \item 数据的种类\\
% 文本、图形、图像、音频、视频、学生的档案记录等\\
% \item 数据的特点\\
% 数据与其语义是不可分的\\
% \end{enumerate}
% \subsubsection{数据库–Database}
% \begin{enumerate}
% \item 数据库的定义\\
% 数据库(Database,简称DB)是长期储存在计算机内、有组织、可共享的大量数据的集合。\\
% \item 数据库的基本特征\\
% \begin{itemize}
% \item 数据按一定的数据模型组织、描述和储存
% \item 可为各种用户共享
% \item 冗余度较小
% \item 数据独立性较高
% \item 易扩展
% \end{itemize}
% \end{enumerate}

% \subsubsection{数据库管理系统–DataBase Management System}
% \begin{enumerate}
% \item DBMS的概念
% DBMS是位于用户与操作系统之间的一层数据管理软件。是基础软件,是一个大型复杂的软件系统

% \item DBMS的用途
% 科学地组织和存储数据、高效地获取和维护数据

% \item DBMS的主要功能
% \begin{itemize}
% \item 数据定义功能\\
% 提供数据定义语言(DDL)\\
% 定义数据库中的数据对象\\

% \item 数据组织、存储和管理\\
% 分类组织、存储和管理各种数据\\
% 确定组织数据的文件结构和存取方式\\
% 实现数据之间的联系\\
% 提供多种存取方法提高存取效率\\

% \item 数据操纵功能\\
% 提供数据操纵语言(DML)\\
% 实现对数据库的基本操作 (查询、插入、删除和修改)\\

% \item 数据库的事务管理和运行管理\\
% 数据库在建立、运行和维护时由DBMS统一管理和控制\\
% 保证数据的安全性、完整性、多用户对数据的并发使用\\
% 发生故障后的系统恢复\\

% \item 数据库的建立和维护功能(实用程序)\\
% 数据库初始数据装载转换\\
% 数据库转储\\
% 介质故障恢复\\
% 数据库的重组织\\
% 性能监视分析等\\

% \item 其它功能\\
% DBMS与网络中其它软件系统的通信\\
% 两个DBMS系统的数据转换\\
% 异构数据库之间的互访和互操作\\
% \end{itemize}
% \end{enumerate}

% \subsubsection{数据库系统–Database System}
% \begin{enumerate}
% \item 数据库系统(Database System,简称DBS)的概念\\
% 在计算机系统中引入数据库后的系统构成

% \item 数据库系统的构成\\
% \begin{itemize}
% \item 数据库 Database
% \item 数据库管理系统(及其开发工具)Database Management System
% \item 应用系统
% \item 数据库管理员 Database Administrator
% \end{itemize}

% \item 数据库系统的特点

% \begin{itemize}
% \item 数据结构化\\
% 数据彼此之间产生联系,发生关系\\
% 数据结构化是数据库系统与文件系统的根本区别\\

% \item 数据的共享性高,冗余度低,易扩充\\
% 数据是面向整体的,可以被多个用户、多个应用程序共享使用\\
% 减少数据冗余,节约存储空间,避免数据之间的不相容性与不一致性\\

% \item 数据独立性高\\
% 数据独立性包括数据的\textbf{物理独立性}和\textbf{逻辑独立性}\\
% 物理独立性是指数据在磁盘上的数据库中如何存储是由DBMS管理的,用户程序不需要了解,应用程序要处理的只是数据的逻辑结构。\\
% 逻辑独立性是指用户的应用程序与数据库的逻辑结构是相互独立的,也就是说,数据的逻辑结构改变了, 用户程序也可以不改变。\\

% \item 数据由DBMS统一管理和控制\\

% 数据库的共享是并发的共享\\
% 多个用户可以同时存取数据库中的数据\\
% DBMS必须提供以下几方面的数据控制功能:\\

% 数据的安全性保护(security)\\
% 保护数据,以防止不合法的使用造成的数据的泄密和破坏。\\

% 数据的完整性检查(integrity)\\
% 将数据控制在有效的范围内,或保证数据之间满足一定的关系。\\

% 数据库的并发访问控制(concurrency)\\
% 对多用户的并发操作加以控制和协调,防止相互干扰而得到错误的结果。\\

% 数据库的故障恢复(recovery)\\
% 将数据库从错误状态恢复到某一已知的正确状态。\\

% \end{itemize}
% \end{enumerate}

% \subsection{数据独立性}
% \subsubsection{数据抽象} 
% 数据抽象(data abstraction)是将现实世界映射到计算机世界的过程
% \begin{figure}[ht]
% \centering
% \includegraphics[width=10cm]{./figures/5754eb41cf5ca203.png}
% \caption{数据抽象映射过程} \label{fig_DBS001_1}
% \end{figure}
% \begin{itemize}
% \item 现实世界:张三、李四、数学系、物理系、高等数学、大学物理...\\
% \item 信息世界:实体、属性、联系、约束...\\
% \item 计算机世界:记录、域、引用...\\
% \end{itemize}
% \subsubsection{数据模型}\\
% 数据模型(data model)是完成数据抽象的工具 \\
% 数据模型的三要素:
% \begin{itemize}
% \item 用于描述数据库的结构的一系列概念
% \item 用于操纵数据结构的一系列操作
% \item 数据库应当服从的约束条件
% \end{itemize}
% 数据模型的分类:
% \begin{itemize}
% \item 概念数据模型(conceptual data model):该模型提供的概念最接近用户理解数据的方式,用于将现实世界映射到信息世界。
% \item 物理数据模型(physical data model):该模型提供的概念用于描述数据库在计算机中的存储细节
% \item 实现数据模型(implementation data model):该模型处在概念数据模型和物理数据模型之间,在实现DBMS时使用
% \end{itemize}

% 参考资料:\href{http://t.csdn.cn/SFcuR}{《数据库系统概论》第五版学习笔记}

