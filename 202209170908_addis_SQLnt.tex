% SQL 基础笔记

\begin{issues}
\issueDraft
\end{issues}

\begin{itemize}
\item 本文参考 \href{https://www.w3schools.com/sql/default.asp}{w3schools}
\item 以下用 SQLite 做测试.
\item SQL(Structured Query Language) 是操作数据库的语言
\item MySQL 是具体的管理数据库的软件, 是 \textbf{RDBMS (relational database management system)}
\item 开发者需要掌握 SQL 语言来使用 MySQL. 不同的 RDBMS 基本都是用 SQL 语言, 虽然有微小差异, 但都兼容 SQL standard.
\item dump 操作常常用于备份数据库(而不是直接拷贝数据库文件). 数据库文件一般是加密的, 但是 dump 不会加密.
\item 一个 database 里面可以有许多 table, 每个 field 就是 table 的一列, 每个 record 就是一行
\item 显示数据库 \verb`SHOW DATABASES;`
\end{itemize}

\subsection{SQL Basics}
\begin{itemize}
\item 用 \verb|--| 注释直到行末, 用 \verb|/*   */| 注释块
\item 大部分操作都通过 \textbf{SQL statement} 完成, 如 \verb`SELECT * FROM 表名;`
\item 命令可以随意换行,两个命令之间必须用 \verb|;| 隔开
\item SQL 关键字, 表格名, 列名都不区分大小写,一般关键词用大写
\item 数据中的字符串用单引号,大部分数据库双引号也行,字符串区分大小写
\item 一些最重要的关键字如
\item \verb`SELECT` - extracts data from a database
\item \verb`UPDATE` - updates data in a database
\item \verb`DELETE` - deletes data from a database
\item \verb`INSERT INTO` - inserts new data into a database
\item \verb`CREATE DATABASE` - creates a new database
\item \verb`ALTER DATABASE` - modifies a database
\item \verb`CREATE TABLE` - creates a new table
\item \verb`ALTER TABLE` - modifies a table
\item \verb`DROP TABLE` - deletes a table
\item \verb`CREATE INDEX` - creates an index (search key)
\item \verb`DROP INDEX` - deletes an index
\end{itemize}

\subsubsection{SELECT}
\begin{itemize}
\item \verb`SELECT 列1, 列2, ... FROM 表格名;` 获取的数据表叫做 \textbf{result-set}. 如果要选取所有列, 用 \verb`*` 即可 e.g. \verb`SELECT * FROM 表格名;` 显示名为 \verb|表格名| 的表格中全部内容
\item \verb`SELECT DISTINCT 列1, 列2, ... FROM 表格名;` 仅列出不完全相同的行, 例如 \verb`SELECT DISTINCT Country FROM 表格名;`
\item \verb`SELECT COUNT(DISTINCT Country) FROM 表格名;` 显示表格中有多少不同的国家. (这个命令在 Microsoft Access 里面没用), 除了 \verb|COUNT| 还有 \verb|AVG| 和 \verb|SUM|
\end{itemize}

\subsubsection{WHERE}
\begin{itemize}
\item 如果没有任何满足条件的行, 那么返回一个空表格, 不报错.
\item \verb`SELECT 列1, 列2, ... FROM 表格名 WHERE condition;`
\item \verb`SELECT * FROM 表格名 WHERE Country='Mexico';`
\item \verb`SELECT * FROM 表格名 WHERE CustomerID=1;`
\item \verb`WHERE` 后面可以是 \verb`=`, \verb`>`, \verb`<`, \verb`>=`, \verb`<=`, \verb`<>`(不等于,有时候 \verb`!=` 也行), \verb`BETWEEN` \verb`LIKE` \verb`IN` (在几个可能的值之中)
\item \verb`SELECT * FROM Products WHERE Price BETWEEN 50 AND 60;`
\item \verb`WHERE` 后面的条件可以用 \verb`AND`, \verb`OR`, \verb`NOT`
\item \verb`SELECT 列1, 列2, ... FROM 表格名 WHERE NOT condition;`
\item \verb`SELECT * FROM 表格名 WHERE City='Berlin' OR City='München';`
\item \verb`SELECT OCTET_LENGTH(列名) FROM 表名 WHERE ...;` 可以返回 blob 的大小
\end{itemize}

\subsubsection{ORDER BY}
\begin{itemize}
\item 用 \verb`ORDER BY` 来进行排序 \verb`SELECT 列1, 列2, ... FROM 表格名 ORDER BY 列1, 列2, ... ASC|DESC;`
\item \verb`SELECT * FROM 表格名 ORDER BY Country DESC;`
\end{itemize}

\subsubsection{INSERT INTO}
\begin{itemize}
\item \verb`INSERT INTO 表格名 (列1, 列2, 列3, ...) VALUES (值1, 值2, 值3, ...);` 插入行, 如果每列都插入, 那么 \verb`(列1, 列2,...)` 可以省略
\item 对所有列, \verb`INSERT INTO 表格名 VALUES (值1, 值2, 值3, ...);`
\item 第一列总是数字(无论 field 叫做什么)? 插入时可以指定,  如果不指定会自动更新. 不能指定已经存在的数字.
\item 如果某列时 optional 的, 那么插入一行时可以不输入这列的信息, 这列的值就是 \verb`NULL`
\item \verb`SELECT 列 FROM 表格名 WHERE 列 IS NULL;`
\item \verb`SELECT 列 FROM 表格名 WHERE 列 IS NOT NULL;`
\end{itemize}

\subsubsection{UPDATE}
\begin{itemize}
\item 改变已经存在的 record. where 可以选择一条或多条 record
\item 如果不用 \verb`WHERE`, 所有行的指定列都会被更新
\item \verb`UPDATE 表格名 SET 列1 = 值1, ..., 列n = 值n WHERE condition;`
\item 例如 \verb`UPDATE 表格名 SET ContactName = 'Alfred Schmidt', City= 'Frankfurt' WHERE CustomerID = 1;`
\item \verb|UPDATE 表格名 SET 列名 = 列名 + 1 WHERE ...| 数值加 1.
\end{itemize}

\subsubsection{DELETE}
\begin{itemize}
\item \verb`DELETE FROM 表格名 WHERE condition;` 删除符合条件的行
\item e.g. \verb`DELETE FROM 表格名 WHERE CustomerName='Alfreds Futterkiste';`
\end{itemize}

\subsubsection{ALTER}
\begin{itemize}
\item 给表格重命名 \verb|ALTER TABLE 旧名字 RENAME TO 新名字;|
\item 从表格中删除一列 \verb|ALTER TABLE 表名 DROP COLUMN 列名;|
\end{itemize}

\subsubsection{CREATE}
\begin{itemize}
\item \verb|CREATE TABLE Person (列1名称 类型 选项1 选项2 ... , 列2名称 类型 选项1 选项2 ... , ...)|
\item 数据类型详见\href{https://www.w3schools.com/sql/sql_datatypes.asp}{这里}.
\item 选项 \verb|PRIMARY KEY| 可以使表格的某一列不出现重复的值且不能为 \verb|NULL|. 一个表格中只能有一个 \verb|PRIMARY KEY|, 但可以是多行.
\item \verb|AUTO_INCREMENT| 选项可以在创建时(如果不指定)自动比该列当前的最大值加 1, 例如 ID 号. 在 sqlite 中没有下划线且只能出现在 \verb|INTEGER PRIMARY KEY AUTOINCREMENT| 中.
\item 如果要在一个 table 中的某列指定另一个 table 中的某个 \verb|PRIMARY KEY|, 那就使用 \verb|FOREIGN KEY|. 例如在 \verb|Orders| 表中链接 \verb|Persons| 表中名为 \verb|PersonID| 的 primary key:
\begin{lstlisting}
CREATE TABLE Orders (
    OrderID int NOT NULL,
    OrderNumber int NOT NULL,
    PersonID int,
    PRIMARY KEY (OrderID),
    FOREIGN KEY (PersonID) REFERENCES Persons (PersonID)
);
\end{lstlisting}
\end{itemize}

\subsubsection{DROP}
\begin{itemize}
\item \verb|DROP TABLE 表名| 删除表.
\end{itemize}

