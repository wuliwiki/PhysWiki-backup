% 罗素悖论(综述)
% license CCBYSA3
% type Wiki

本文根据 CC-BY-SA 协议转载翻译自维基百科\href{https://en.wikipedia.org/wiki/Russell\%27s_paradox}{相关文章}。

在数学逻辑中,罗素悖论(也称为罗素反义命题)是由英国哲学家和数学家伯特兰·罗素于1901年提出的一个集合论悖论。罗素悖论表明,任何包含不受限制的理解原理的集合论都会导致矛盾。根据不受限制的理解原理,对于任何足够明确定义的属性,都存在一个集合,包含所有且仅包含具有该属性的对象。设R为所有不属于自身的集合的集合(这个集合有时被称为“罗素集合”)。如果R不属于自身,则根据其定义,它必须属于自身;然而,如果它属于自身,那么它就不属于自身,因为它是所有不属于自身的集合的集合。由此产生的矛盾就是罗素悖论。用符号表示如下:

设 
\[
R = \{ x \mid x \notin x \}~
\]
那么
\[
R \in R \iff R \notin R~
\]