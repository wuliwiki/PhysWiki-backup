% 暗物质密度分布
% license Usr
% type Tutor

 
我们现在回顾一下文献中通常被认为是最可信的银河系暗物质密度分布 \(\rho(r)\)。所有这些分布都假设球对称性,其中 \(r\) 是从银河中心 (GC) 测量的径向坐标。其中一些函数可以方便地重构为具有三个参数 \(\alpha, \beta, \gamma\) 的集体公式,也被称为“双幂律”公式:

\begin{figure}[ht]
\centering
\includegraphics[width=14cm]{./figures/24bd16cf1ad84078.png}
\caption{} \label{fig_DMdens_1}
\end{figure}
\begin{equation}\label{eq_DMdens_1}
\rho_{\alpha\beta\gamma}(r) =  \frac{\rho_s}{ \left(\frac{r}{r_s}\right)^\gamma \left[1 + \left(\frac{r}{r_s}\right)^\alpha\right]^{\frac{\beta-\gamma}{\alpha}}  } ~.
\end{equation}
参数 \(\alpha, \beta,\) 和 \(\gamma\) 控制从银河中心不同径向距离处暗物质密度分布的形状。具体来说,\(\alpha\) 控制内坡和外坡过渡的锐度,\(\beta\) 影响外坡,而 \(\gamma\) 决定分布的内坡。这些函数的动机如下:

纳瓦罗、弗兰克和怀特 (NFW) 密度轮廓是最常见的基准选择,它是由N体模拟激发的。密度在接近银河中心时以 \(r^{-1}\) 发散。参数固定为 \(\alpha = 1, \beta = 3\),\(\gamma\) 作为自由参数的版本,有时被称为“广义NFW (gNFW)”配置文件。当 \(\gamma > 1\) 时,中心的斜率比标准NFW更陡:在这种情况下,该密度轮廓被称为“收缩NFW (cNFW)”。例如,摩尔和他的合作者发现 \(\gamma = 1.16\)。这样的收缩密度轮廓出现在一些早期的数值模拟中,这些模拟包括了重子。

Einasto密度轮廓作为对更近期数值模拟的更好拟合而出现。Einasto密度轮廓的密度比NFW更“丰满”。Einasto幂律指数随着 \(r\) 的变化而连续变化,由形状参数 \(\alpha_{\text{Ein}}\) 控制。\(\alpha_{\text{Ein}} = 0.17\) 的值代表了不同N体模拟建议的不同值的合理平均值。

具有核心的密度轮廓,如“具有核心”(“截断”)的等温密度轮廓或Burkert密度轮廓,具有恒定的中心密度。它们是出于一些观测到的银河旋转曲线的动机,这些曲线可能指向核心的存在,以及数值模拟发现重子反馈降低了中心密度。

Di Cintio等人提出了一个密度轮廓,它在\autoref{eq_DMdens_1} 中具有双幂律形式,但参数 \(\alpha, \beta, \gamma\) 并不适用于所有星系,而是依赖于每个星系的恒星-晕质量比。然后,暗物质配置文件有效地从有尖点的变为有核心的,适应每个星系的恒星和暗物质内容。这是基于包括重子的模拟的结果。

在考虑的密度轮廓中,当 \(r \to 0\) 时,\(\rho(r)\) 会发散,然而,在所有情况下 \(r^2\rho(r) \to 0\)(除非 \(\gamma\) 不切实际地大),这样银河中心区域只包含少量的暗物质。

虽然各种密度分布模型在距离银河中心几kpc以上的地方给出相似的结果,包括地球附近,但它们在更小的距离上的差异相当大—相差数个数量级。在银河中心附近,没有关于暗物质分布的观测数据。此外,数值模拟的分辨率不允许我们观测到比大约1kpc更小的尺度。因此,\( \rho(r) \) 的行为在 \( r \to 0 \) 时简单地由假定的渐近函数形式控制。结果,来自银河内部区域的间接暗物质信号,如来自银河中心周围几度区域的伽马射线通量,强烈依赖于不确定的暗物质分布。这与许多其他暗物质信号形成对比,后者取决于远离银河中心的暗物质分布:暗物质直接探测信号取决于地球位置的暗物质密度;一些暗物质信号探测了地球附近的局部银河区域(例如,最高产生于地球几kpc外的高能正电子通量),以及探测远离银河中心区域(例如,来自高纬度的伽马射线)的暗物质信号。

\autoref{fig_DMdens_2} 展示了暗物质分布(左)和表中列出的暗物质分布模型参数(右)。确定等温分布参数的过程与其他模型不同。表中提供了 \( r_s \)(\( \rho_s \))的2位(3位)有效数字,这对于大多数计算已经足够精确。然而,在特定情况下,例如为了精确重现围绕银河中心的小角度区域的 \( J \) 因子,需要更精确的输入。

\begin{figure}[ht]
\centering
\includegraphics[width=14cm]{./figures/474f0d8783a86c10.png}
\caption{} \label{fig_DMdens_2}
\end{figure}

接下来,需要确定进入暗物质分布 \( \rho(r) \) 的参数 \( r_s \)(典型尺度半径)和 \( \rho_s \)(典型尺度密度)。这可以通过不同的方式完成,例如,从银河系类光环的数值模拟中提取它们的值,或者以某种方式从对银河系或类似外部星系的观测中确定它们。一种方法是要求银河系的暗物质分布满足以下一组约束条件:

A) 太阳位置的暗物质密度 \( \rho_\odot \)。许多研究小组使用不同的技术研究了这个量,特别是全局方法,它拟合了整个银河系的旋转曲线,如上文所讨论的,或者依赖于研究局部恒星运动学(特别是垂直方向的恒星运动)来确定局部引力作用,从而确定局部暗物质密度的局部方法。全局方法可以提供 \( \rho_\odot \) 的精确测定,但对银河系重子组分的不确定建模很敏感。局部方法不够精确,并且一方面受到由于特殊情况(例如,北南银河半球的不对称性)引起的系统误差的影响,另一方面受到分析中的简化(例如,是否包括所谓的“倾斜项”,它关联了径向和垂直的恒星运动)的影响。

最近的 \( \rho_\odot \) 测定指向

\[ \rho_\odot = \rho(r_\odot) = 0.40 \text{ GeV/cm}^3 \approx 0.0106 M_\odot/\text{pc}^3 ~. \]

**表2.2** 展示了太阳位置的暗物质密度。最上面的一组是“历史性”的研究。最下面的一组是在盖亚数据发布后发表的作品选择(尽管不一定使用它们)。中间的一组列出了最近大约15年的选定结果。注意,对于所有作者来说,太阳的位置可能并不完全相同,使用的方法不同,关键假设可能不同,可能需要相对重新标准化。参考文献在[50]。

典型的相关误差是 ±0.10 GeV/cm³,但可能的误差范围高达 0.2 至 0.8 GeV/cm³(有时被称为“一个因子的2”)经常被采用。从表2.2中列出的选定结果中可以看出,ρ⊙ 的不同测定结果仍然没有达成一致。从历史上看,自90年代以来文献中常用的传统值是 ρ⊙ = 0.3 GeV/cm³。这个值在2009年到2011年之间进行的几项研究之后有所增加,这些研究发现了稍微高的中心值。另一方面,其他研究发现了较小的中心值。

此外,相关的误差仍然是激烈争论的话题。盖亚任务的数据激发了确定 ρ⊙ 的新活动,但情况尚未解决,而且更精确地确定 ρ⊙ 似乎很难实现。

与此相比,太阳周围几百秒差距范围内的总物质密度(包括普通物质)平均估计值比暗物质密度(2.11)大约一个数量级,4.5 GeV/cm³ ≈ 0.120 M⊙/pc³(2.12)。

方程(2.11)意味着,在海王星轨道内(约30天文单位)包含的暗物质质量大约是 \(10^{-13} M_\odot\),或者等价地说,太阳的质量对应于包含在半径约2.8秒差距(约9.2光年)球体内的所有暗物质的质量,与最近恒星的距离(约4光年)相当。这说明了暗物质在局部是相当次要的。人们可能会想,尽管如此,暗物质在太阳系的存在是否仍然可以通过它对行星和密切监测的小行星的轨道的影响来检测(例如,累积的暗物质将略微增加外天体看到的太阳的有效质量)。然而,直接的天文观测虽然准确,但仍然不是特别有限制性。由此得出的太阳系中暗物质密度的界限是 ρ < 5000 GeV/cm³,这比预期的 ρ⊙ 大约15000倍。换句话说,暗物质对太阳系中行星的轨道没有显著影响。一个相关问题是,方程(2.11)中 ρ⊙ 的平均值是否会因太阳系中天体的存在而显著改变,例如,是否会因太阳的引力吸引而增加,或者因对通过行星(特别是木星,但还有金星和地球本身,就像对传统小行星发生的情况一样)的暗物质粒子的弹弓效应而减少。对局部暗物质密度和速度分布的任何显著改变将对解释暗物质直接探测结果(见第5章)以及预测来自太阳或地球捕获的暗物质粒子的中微子通量(见6.9节)特别重要。一系列详细的研究发现了许多微妙的效应,这些效应在大多数情况下相互抵消,导致对暗物质粒子的净效应几乎可以忽略不计,至少对于那些只与天体弱相互作用的暗物质粒子是这样[54]。

还要注意,暗物质的能量密度是 ρ⊙ ∼ 20 kW/m²,大约是太阳辐射在地球表面的能量密度的14倍(如果这种能量有一天被证明是可用的形式,将我们的宇宙变成一个巨大的电池组,这将是一个好主意)。

B) 银河系中包含的总暗物质质量。这个量也被许多小组使用不同的技术研究过[55]。这些方法包括研究亮卫星(如矮星系或麦哲伦云)的运动学、恒星流的运动学、附近流浪星的逃逸速度以及使用各种追踪器。对于MDM 200,我们使用“标准”值

\[ MDM_{200} \equiv (1.0 \pm 0.25) \times 10^{12} M_\odot ~. \]

然而,值得注意的是,盖亚任务的精确天体测量数据的出现导致最近很多活动,2023年的一些研究得出了MDM 200的值比之前低几倍。对于银河系,r200约为200到300 kpc,因此可以将MDM 200视为以银河系中心为中心,半径为200 kpc的大球体内的总暗物质质量。

相比之下,使用类似技术推断出的恒星质量约为 \(8 \times 10^{10} M_\odot\)。地球到大麦哲伦云的距离约为48.5 kpc。

10 使用c=1单位。可以将其视为如果所有暗物质静止质量被吸收,以光速移动的观测者看到的能量通量。由于暗物质与物质的相互作用非常微弱(如果有的话),暗物质实际被吸收的概率非常小,这使得利用这种能量来源变得极其困难。

\subsection{矮星系} 

银河系周围环绕着许多卫星星系,即通过引力与主星系晕绑定的天体系统。除了大麦哲伦云和小麦哲伦云足够大可以用肉眼看到之外,大多数卫星星系由矮星系组成。其中大部分属于矮球状(dSph)类型,只有少数是矮不规则(dIrr)类型。历史上,雕塑家星系(Sculptor)是第一个被发现的(1938年),通常被认为是典型的dSph。直到1990年代,又发现了七颗矮星系(Fornax、Leo I、Leo II、Ursa Minor、Draco、Carina和Sextans)。这八颗矮星系统称为经典矮星系。随着数字巡天的发展,如Sdss、Subaru超高清相机(Hsc)(覆盖北天)、Pan-Starrs、Des(覆盖南天)和Gaia(从太空运行),发现了许多更多的矮星系:这些被称为超微弱矮星系。目前总数接近50个,包括一些尚未确认其性质和属性的星系。随着巡天变得更加敏感并覆盖更大的天区,例如即将到来的Rubin天文台(以前称为Lsst),这个名单可能会继续增加。还有一个特殊的射手座dSph,它在1994年被发现,由于它被认为受到银河系晕的潮汐破坏,因此被单独列出。"矮星系"的分类定义并不明确,即,没有普遍接受的质量、大小和光度值范围来定义一个小星系为矮星系类型。一般来说,可以将矮星系视为具有大约10^6太阳质量的暗物质主导质量,而实际的矮星系可以与这个值相差几个数量级:例如,Fornax估计为56×10^6太阳质量,而Segue 2为$0.23\times 10^6$太阳质量。半光度半径rHL定义为包含来自星系一半光度的圆的半径,可以从几十秒差距到几百秒差距不等。根据模拟,矮星系的暗物质晕的维里半径r200往往比rHL大约一个数量级。被识别为属于矮星系的恒星数量可以少至十几个或多达数千个。

矮星系是暗物质搜索的有希望的目标,因为它们只包含很少的恒星,并且被认为主要由暗物质支配。可能的原因是,与更大的星系相比,矮星系构成了相对较浅的势阱,这有助于早期超新星将重子气体驱逐出去,而随后的恒星形成率很低。估计的总质量与光度之比Mtot/L可以达到几十太阳质量/太阳光度(例如,Sculptor)到几百太阳质量/太阳光度(例如,Draco),甚至可以达到几千太阳质量/太阳光度。14矮星系中暗物质的丰富当然是预期暗物质信号强度的好消息。此外,恒星和其他活跃天体对象的稀缺意味着信号比其他系统受到的前景辐射污染更少。然而,这是以只有少量追踪器可以用来重建暗物质内容和暗物质密度分布的代价。此外,这些追踪器通常只是恒星:这些系统中没有发现像那些在追踪螺旋星系的旋转曲线中发挥关键作用的氢云。

由于恒星的微弱和复杂的动力学,确定dSphs中的暗物质密度函数是具有挑战性的,并且引起了激烈的争论。例如,在观测上,文献中似乎没有共识,恒星运动学数据是否指向中心dSphs中的尖点或核心暗物质分布。一些测量,特别是在2000年代初,曾表明倾向于dSphs中心的核心。随后的观测显示,至少在像Sculptor这样的特定案例中,尖点的NFW配置文件同样适合数据。另一方面,数值模拟的结果似乎支持中等大小的矮星系形成核心,而低质量的矮星系似乎发展出尖点。矮星系的核-尖性质也可能取决于恒星形成的历程:那些在6十亿年前停止形成恒星的矮星系通过不受阻碍的收缩发展出尖点,而那些继续形成恒星的矮星系则因为恒星爆发而“加热”了它们的暗物质含量,因此发展出了核心。文献中的一种常见实用方法是使用第2.2节中讨论的配置文件,其参数以多种不同的方式确定。然后从这些预测相应的暗物质信号。

\subsection{星系团}

星系团是宇宙中最大的受引力束缚的系统。它们包含数百到数千个星系,延伸到几个Mpc的大小,并且具有高达$10^{14}$−几$10^{15}$太阳质量的典型总质量。在某些方面,星系团可以被视为星系的放大版本,这为理解它们的主要特征提供了基本框架。然而,星系团和星系之间也存在显著差异。

星系团包含重子物质和暗物质,它们的比例通常是1到10。星系团的暗物质晕的形成可以初步描述为初始过密区域的非线性球形坍缩,这在方式上与2.1.2节中描述的银河系尺度上的过程完全类似。对于更精确的建模,采用数值模拟,可能只包括暗物质或同时包括暗物质和重子物质。这些模拟表明,NFW和Einasto配置文件非常适合星团暗物质分布。这强调了这样的配置文件大致是通用函数,即它们适用于独立于其质量的坍缩系统。像星系一样,透镜观测似乎也显示出一些星系团中心区域的密度分布斜率比预测的要平缓。

也许星系和星系团之间最重要的定性区别是,星系团中的大多数重子物质(和宇宙中一样)以弥散气体的形式存在,而不是恒星。这意味着...

14 这里,\( L_\odot = 3.828 \times 10^{26} \) W 是太阳的名义光度,而 \( M_\odot = 1.9984 \times 10^{30} \) kg 是已经遇到的太阳质量。

...以弥散气体的形式存在,而不是恒星。这意味着星系团的大部分重子物质可以通过X射线观测来追踪,因为这些气体被加热到数百万度,发出X射线。相比之下,星系中的重子物质大部分以恒星的形式存在,只能通过它们的光来观测。星系团中的气体通常占据晕的中心,而不是像在星系中那样集中在一个平面上。此外,星系团中的气体质量可以与暗物质的质量相当,而在星系中,重子物质(以恒星形式)的质量通常只占总质量的一小部分。

星系团的暗物质晕的密度分布可以通过两种主要方式确定:通过观测星系团的引力透镜效应,以及通过数值模拟。引力透镜效应是由于星系团的总质量(包括暗物质)对背景星系的光线产生弯曲造成的。通过测量这种光线的弯曲,可以推断出星系团的质量分布。数值模拟则可以提供对星系团暗物质晕形成的更深入理解,包括其密度分布和演化。

星系团的暗物质晕通常被假设为具有与星系类似的密度分布,例如NFW或Einasto模型。然而,观测和模拟都表明,星系团中心的密度分布可能比这些模型预测的要平坦。这可能表明星系团的暗物质晕在中心区域经历了某种程度的“加热”,导致密度分布变得更加均匀。

星系团的暗物质晕的另一个特点是它们可能包含多个子结构,如子晕或丝状结构。这些子结构的形成和演化可以通过数值模拟来研究,它们对星系团的整体性质有重要影响。

总的来说,星系团提供了研究暗物质在宇宙大尺度结构中分布和演化的重要窗口。通过对星系团的详细观测和模拟,我们可以更好地理解暗物质的性质和宇宙的演化历程。 

\subsection{晕质量函数} 

我们计算晕质量函数 \( \frac{dn(t)}{dM} \):在给定时间 \( t \) 下,质量为 \( M \) 的受引力束缚结构(如星系、星系团、暗物质晕等)的平均宇宙学数量密度 \( n \)。确定这一量值本身就很有趣,因为它是结构形成理论的一个基本预测。同时,了解 \( \frac{dn(t)}{dM} \) 对于预测暗物质湮灭的宇宙学信号也是必要的。晕质量函数 \( \frac{dn(t)}{dM} \) 最准确地从数值N体模拟中提取。然而,历史上,它也是通过近似的解析处理来确定的,结果出人意料地很好地再现了数值结果,尽管存在一些可疑的假设和剧烈的简化。以下,我们将回顾解析推导,并提到它们如何被改进以更好地与模拟结果一致。

我们回顾了在密度对比 \( \delta(x, t) \ll 1 \) 极限下的结构形成分析理论,其中 \( \delta \) 是密度对比。在这个近似下,宇宙仍然几乎是均匀的。关注在物质/辐射平等之后(这是与结构形成相关的时期),所有在视界内部的密度扰动模式都以相同的方式演化,使得 \( \delta(x, t) = D(t/t')\delta(x, t') \),其中 \( x \) 是共动坐标,\( D \) 是所谓的增长因子。下面,我们将这些结果外推到结构开始形成的后期时间。这可能是值得怀疑的,因为当 \( \delta \) 接近1时,即在轻度非线性区域,受引力束缚的物体开始形成。更准确地说,无论是解析计算还是数值模拟都表明,当 \( \delta \) 大于临界密度对比 \( \delta_c \) 时结构形成,对于球对称情况,\( \delta_c \) 的临界值等于 \( \delta_c \approx 1.686 \)。将小 \( \delta \) 区域外推到 \( \delta \) 接近 \( \delta_c \) 对我们的目的来说已经足够了,因为 \( \delta_c \) 大约是1。我们将假设密度对比 \( \delta \) 遵循高斯分布。(实际上,由于密度是正的,\( \delta \) 至少是 \( -1 \),因此需要截断分布。)密度对比分布的方差可以表示为:
\begin{equation}
\sigma^2 = \int \frac{d^3x}{(2\pi)^3} \delta^2(x) = \int \frac{d^3k}{(2\pi)^3} P(k) = \int dk \Delta^2(k)~.
\end{equation}
其中 \( \Delta^2(k) = \frac{k^3 P(k)}{2\pi^2} \) 是无量纲方差,\( P(k) \)是功率谱。方差 \( \sigma^2 \) 发散,因为小尺度(即大 \( k \))上的涨落占主导地位。通过将 \( \delta(x) \) 替换为 \( \delta_R(x) \) 来解决这个问题,\( \delta_R(x) \) 是在 \( R \) 阶尺度上平滑的密度对比。为了进行平滑,我们引入一个“窗口函数” \( W_R(x) \),它仅在原点大约 \( R \) 的距离内显著。然后将其与 \( \delta \) 卷积以获得:
\begin{equation}
\delta_R(x) = \int d^3x' \delta(x') W_R(x - x'), \quad \text{i.e.,} \quad \delta_R(k) = W(kR) \delta_k~. 
\end{equation}
这里,\( W(kR) = \int d^3x W_R(x) e^{-ik \cdot x} \) 是 \( W_R(x) \) 的傅里叶变换。窗口函数归一化,使得 \( \int d^3x W_R(x) = 1 \)。常见的选择是在 \( k \) 空间中进行锐截止(\( W(kr) = 0 \) 在 \( k < 1/R \) 之外,内部恒定),或在 \( x \) 空间中进行锐截止(\( W_R(x) = 0 \) 在 \( x > R \) 之外,内部恒定);或者在两个空间中使用高斯函数(\( W_R \propto e^{-r^2/2R^2}, W = e^{-k^2R^2/2} \))。按照方程(2.14)类似地计算平滑不均匀性的方差:
\begin{equation}
\sigma_R^2(t) = \int dk \Delta^2(k, t) W^2(kR)~.
\end{equation}
是有限且在小 \( R \) 时增加的。这种平滑有一个副作用,即在大的过密区域内,任何位于其中的低密区域都被计为坍缩了。接下来,遵循Press和Schechter (PS) [61],我们有些可疑地采用以下假设:在时间 \( t \) 时,宇宙中已经坍缩并形成了半径为 \( R \) 的结构的体积分数 \( F \),是 \( \delta > \delta_c \) 的体积的两倍。后者等于 \( \delta > \delta_c \) 的概率。由于 \( \delta \) 被假设为具有 \( \sigma_R^2 \) 的方差的高斯分布,我们因此有:
\begin{equation}
F(R) = p(\delta > \delta_c|R) = 2 \sqrt{\frac{2}{\pi}\sigma_R} \int_{\delta_c}^\infty d\delta \, e^{-\delta^2/2\sigma_R^2} = \text{erfc} \left(\frac{\delta_c}{\sqrt{2}\sigma_R}\right)~.
\end{equation}
2的因子是人为添加的,以便在大不均匀性的极限下,所有的体积(而不是一半的体积)都坍缩。

最后,我们得到了晕质量函数。我们通过估计半径 \( R \) 大致对应于包含质量 \( M = 4\pi R^3 \bar{\rho}/3 \),其中 \( \bar{\rho}(t) \) 是平均暗物质密度,并通过识别在相应质量下的质量和半径的方差:\( \sigma_M = \sigma_R(M) \) 来从体积转换为质量。总质量中通过引力束缚在质量在 \( M \) 和 \( M + dM \) 之间的结构中的分数由 \( dF = (dF/dM)dM \) 给出。然后通过将上述结果乘以这些对象的平均数量密度 \( \bar{\rho}/M \) 来获得质量在 \( M \) 和 \( M+dM \) 之间的对象的数量:\( \frac{dn}{dM} = \left(\frac{\bar{\rho}}{M}\right) \frac{dF}{dM} \)。因此,晕质量函数是:
\begin{equation}
\frac{dn}{dM} = \frac{\bar{\rho}}{M^2} f_{\text{PS}}(\nu) \frac{d\ln \sigma_M}{d\ln M}, \quad \text{其中} \quad f_{\text{PS}}(\nu) = \frac{\nu}{2\pi} e^{-\nu^2/2} \quad \text{且} \quad \nu \equiv \frac{\delta_c}{\sigma_M(t)}~.
\end{equation}
数量密度 \( \frac{dn}{dM} \) 对于小质量 \( M \) 大致按 \( 1/M^2 \) 缩放,与N体模拟一致。在足够大的质量上,使得 \( \sigma_M \gtrsim \delta_c \),\( \frac{dn}{dM} \) 大致以 \( e^{-\nu^2/2} \) 指数抑制。

数值模拟给出的结果与Press-Schechter近似略有不同:它们预测有更多的重晕,更少的小晕。这被理解为模拟的坍缩往往是椭圆形而不是球形。当球对称近似放松时,典型临界过密区域的恒定值被一个更 







