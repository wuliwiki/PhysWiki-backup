% 隔板法(排列组合)
% license Xiao
% type Tutor

\pentry{组合\nref{nod_combin}}{nod_b71e}

\footnote{参考 Wikipedia \href{https://en.wikipedia.org/wiki/Stars_and_bars_(combinatorics)}{相关页面}。}在排列组合问题中, \textbf{隔板法(stars and bars)}常用来解决以下问题:

\begin{example}{}
把 $n$ ($n = 1,2,\dots$)个不加区分的小球放进 $m$($1\leqslant m\leqslant n$) 个有编号的盒子, 每个盒子至少有一个小球, 有多少种不同的方法?
\begin{figure}[ht]
\centering
\includegraphics[width=8cm]{./figures/cedf0adc23fa4f59.pdf}
\caption{题目示意图} \label{fig_BarCom_1}
\end{figure}
\end{example}

我们可以想象这些小球排成一列后被 $m-1$ 个隔板隔开, 每一组被隔开的小球就相当于装进一个盒子中。 小球之间一共有 $n-1$ 个空隙可以插入隔板, 一个空隙最多插一个隔板, 所以不同的情况数就是组合 $C_{n-1}^{m-1}$。
\begin{figure}[ht]
\centering
\includegraphics[width=6cm]{./figures/f7571353fd2e7670.pdf}
\caption{隔板法示意图} \label{fig_BarCom_2}
\end{figure}

综上,可以发现隔板法的应用有以下 $3$ 个要求:
\begin{enumerate}
\item 要求每个盒子至少分得 $a$ 个物品,$a$ 必须等于 $1$;
\item 物品之间无差异;
\item 盒子之间有差异(盒子有顺序)。
\end{enumerate}


\subsection{盒子可以为空}
% \pentry{范德蒙恒等式\nref{nod_ChExpn}}{nod_adeb}

% \addTODO{应当是高中内容,不需要引入范德蒙恒等式才对: 另外开文章吧}

\begin{example}{}
把 $n$ ($n=1,2\dots$)个不加区分的小球放进 $m$ ($m=1,2\dots$)个有编号的盒子, 盒子可为空, 有多少种不同的方法?
\end{example}

% int256: 以下是,原来内容,如果后续需要新开文章可以复制,使用快捷键“Shift+/”一键取消复制。

% 把所有的情况根据非空盒子的个数分类。 非空盒子个数可能为 $i=1$ 个( $n$ 个小球都在里面), $i=2$ 个, 一直到 $i=\min\qty{m,n}$ 个(若 $n\leqslant m$ 则每个小球都在不同的盒子)。 首先从 $m$ 个盒子里面选择 $i$ 个非空盒子会有 $C_m^i$ 种情况。 再考虑这 $i$ 个有编号盒子装 $n$ 个小球(不为空)又有几种情况: 用隔板法得到共有 $C_{n-1}^{i-1}$ 种。 所以一个 $i$ 对应 $C_m^i C_{n-1}^{i-1}$ 种情况。 最后把所有不同 $i$ 的情况数加在一起, 得出所有情况的总数为
% \begin{equation}
% \sum_{i = 1}^{\min\qty{m,n}} C_m^i C_{n-1}^{i-1} = \sum_{i=1}^{\min\qty{m,n}}  C_m^i C_{n-1}^{n-i}~.
% \end{equation}
% 这里使用了 $C_a^b = C_a^{a-b}$ (\autoref{eq_combin_4}~\upref{combin})。 又由范德蒙恒等式\upref{ChExpn}, 有
% \begin{equation}
% \sum_{i=1}^{\min\qty{m,n}}  C_m^i C_{n-1}^{n-i} = C_{m+n-1}^n = \frac{(m+n-1)!}{n!(m - 1)!}~.
% \end{equation}
% 这就是最后的答案。

对于盒子可以空的方法,为了满足条件 $1$,我们可以先增加 $m$ 个球(这 $m$ 个球将分给各个盒子,使得满足条件 $1$),而对于每种分法的实际情况对应于减去这 $m$ 个虚拟球。

即:若在现在这 $n+m$ 个球、$m$ 个盒子的前提下任一盒子分得了 $k$ 个球,则在原题条件下应分得 $k-1$ 个球。这样现在的问题情形是求 $n+m$ 个球、$m$ 个盒子的标准隔板法。故答案就为 $C_{n+m-1}^{m-1}$。
% 使用范德蒙德不等式会得到 $C_{m+n-1}^n$ 的结果与这里一致。

\subsection{要求各个盒子至少分得 $k > 1$ 个球}
模仿刚才要求盒子可以为空的方法,我们先从总共的 $n$ 个球中拿出 $(k-1)m$ 个(注意这里是 $k-1$,不是 $k$。因为标准的隔板法要求每组还至少分得一个),先钦定给各个盒子分别 $k-1$ 个。这样剩余的 $n-(k-1)m$ 个球、$m$ 个盒子的标准隔板法。故这样的问题的方法数是 $C_{n-(k-1)m-1}^{m-1}$。注意要求存在一个方法,即 $n-(k-1)m-1 \le m-1$。

\subsection{要求混合}
前两种情况也可以混合,例如下面这道例题。
\begin{example}{}
把 $11$ 个球分给 $4$ 个盒子。各个盒子要求如下:
\begin{enumerate}
\item $1$ 号盒子无要求,可以没有球也可以有球;
\item $2$ 号盒子要求至少分到 $3$ 个球;
\item $3$ 号盒子要求至少分得 $1$ 个球;
\item $4$ 号盒子无要求(同 $1$ 号盒子)。
\end{enumerate}
求不同的方法数。
\end{example}
我们先创建两个“虚拟球”,这两个虚拟球分别在 $1$ 号和 $4$ 号盒子中,使得他们变为标准的隔板法情形。这里使得 $11$ 个球变为 $11+2 = 13$ 个球。

接下来考虑 $2$ 号盒子,需要 $3$ 个球,故先拿走 $3-1=2$ 个球固定分配给 $2$ 号盒子。这使得 $2$ 号盒子也变为标准的隔板法的情形。而 $3$ 号盒子就是标准的隔板法,无需对其进行任何处理。故实际是要对一个 $13-2=11$ 个球、$4$ 个盒子的情形进行标准隔板法(巧妙地,可以直接带入原题数据进标准隔板法的“$C_{n-1}^{m-1}$” 而得到正确结果)。故答案就是 $C_{11-1}^{4-1} = C_{10}^3$。