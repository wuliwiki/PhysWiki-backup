% 完整群(综述)
% license CCBYSA3
% type Wiki

本文根据 CC-BY-SA 协议转载翻译自维基百科\href{https://en.wikipedia.org/wiki/Holonomy}{相关文章}。

\begin{figure}[ht]
\centering
\includegraphics[width=6cm]{./figures/8ee31b3dfe98af26.png}
\caption{球面上沿分段光滑路径的平行移动。初始向量标记为 \(V\),它沿着曲线被平行移动,最终得到的向量标记为 \( \mathcal{P}_\gamma (V) \)。如果路径发生变化,平行移动的结果也会不同。} \label{fig_WZQ_1}
\end{figure}
在微分几何中,光滑流形上一个联络的平行迁移群(holonomy)描述的是:沿着闭合回路进行平行移动时,几何数据未被保持的程度。平行迁移群是联络曲率所导致的一种普遍几何效应。对于平坦联络,相关的平行迁移群是一种单值延拓(monodromy),并且本质上是一个全局概念。而对于曲率非零的联络,平行迁移群同时具有非平凡的局部和全局特征。

任何流形上的联络都会通过其平行移动映射引出某种平行迁移群(holonomy)的概念。最常见的平行迁移群形式是具有某种对称性的联络。重要的例子包括:黎曼几何中Levi-Civita联络的平行迁移群(称为黎曼平行迁移群),向量丛上联络的平行迁移群,Cartan联络的平行迁移群,以及主丛上联络的平行迁移群。在这些情形下,联络的平行迁移群都可以与某个李群(即平行迁移群)对应起来。根据Ambrose-Singer定理,联络的平行迁移群与该联络的曲率密切相关。