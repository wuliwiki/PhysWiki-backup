% 华罗庚(综述)
% license CCBYSA3
% type Wiki

本文根据 CC-BY-SA 协议转载翻译自维基百科\href{https://en.wikipedia.org/wiki/Hua_Luogeng}{相关文章}。

\begin{figure}[ht]
\centering
\includegraphics[width=6cm]{./figures/e19fa344575726af.png}
\caption{1956年的华罗庚} \label{fig_Luogen_1}
\end{figure}
华罗庚(Hua Luogeng 或 Hua Loo-Keng,1910年11月12日—1985年6月12日)是中国著名数学家和政治家,以在数论方面的重要贡献以及在中华人民共和国数学研究和教育方面的领导作用而闻名。他在发现和培养著名数学家陈景润方面发挥了重要作用,陈景润证明了“陈景润定理”,这是关于哥德巴赫猜想的最著名结果。此外,华罗庚后来在数学优化和运筹学方面的研究,对中国经济产生了巨大影响。他于1982年被选为美国国家科学院外籍院士。他还被选为第一至第六届全国人民代表大会常务委员会委员,第六届全国政协副主席(1985年4月),以及中国民主同盟副主席(1979年)。他于1979年加入中国共产党。

华罗庚没有接受过正式的大学教育。尽管获得了几次荣誉博士学位,但他从未获得任何大学的正式学位。实际上,他的正规教育仅限于六年的小学和三年的中学。因此,熊庆来在阅读了华罗庚的早期论文后,对华罗庚的数学天赋感到惊讶,于是1931年邀请他到清华大学学习数学。
\subsection{传记}
\subsubsection{早年(1910–1936)}  
华罗庚于1910年11月12日出生在江苏金坛。华罗庚的父亲是一位小商人。在中学时,华罗庚遇到了一位优秀的数学老师,这位老师早早发现了他的数学天赋,并鼓励他阅读一些高级的数学书籍。中学毕业后,华罗庚考入了上海的中华职业大学,并在那里通过赢得全国珠算比赛而表现突出。尽管该校的学费较低,但生活费用对于他的经济状况来说依然过高,最终华罗庚不得不提前一学期离校。未能在上海找到工作后,华罗庚于1927年返回家乡,帮助父亲经营商店。1929年,华罗庚患上了伤寒,卧床休养了半年。疾病的后果导致华罗庚的左腿部分瘫痪,从此他的行动受到严重限制,直到生命的尽头。[1]

中学毕业后,华罗庚继续自学数学,凭借他所拥有的几本书,学习了整个高中和初级大学的数学课程。当华罗庚回到金坛时,他已经开始从事独立的数学研究,他的第一篇论文《斯图尔姆定理的若干研究》发表于1929年12月的上海期刊《科学》上。次年,华罗庚在同一期刊上发表了一篇简短的文章,指出一篇声称已解决五次方程的1926年论文存在根本性缺陷。华罗庚清晰的分析引起了北京清华大学熊庆来教授的注意,1931年,尽管华罗庚没有正式的资格,而且清华的几位教职工对此也有所保留,华罗庚还是被邀请加入清华大学数学系。

在清华,华罗庚起初在图书馆担任文员,随后转为数学助教。到1932年9月,他已成为讲师,二年后,华罗庚发表了十几篇论文,晋升为讲师。

1935年至1936年,雅克·阿达马和诺伯特·维纳访问了清华大学,华罗庚热情地参加了他们的讲座,并给人留下了良好的印象。维纳随后访问了英国,并向G.H.哈代提到了华罗庚。于是,华罗庚收到了前往英国剑桥大学的邀请,并在那里停留了两年。
\subsubsection{中年早期(1936–1950)}  
在剑桥大学期间,华罗庚致力于将哈代–利特伍德圆法应用于数论问题。他在解决瓦林问题(Waring's problem)方面的开创性工作奠定了他在国际数学界的声誉。1938年,随着中日战争的全面爆发,华罗庚决定返回中国,回到清华大学,尽管他没有任何学位,但仍被任命为教授。当时,随着大面积的中国领土被日本占领,清华大学、北京大学和南开大学合并为位于云南昆明的西南联合大学。尽管面临贫困、敌机轰炸和相对孤立的学术环境,华罗庚仍然继续创作一流的数学成果。在昆明的八年中,华罗庚研究了维诺格拉多夫(Vinogradov)关于估计三角和的开创性方法,并将其重新表述为更为尖锐的形式,这一成果现在被称为维诺格拉多夫均值定理。这一著名结果是改进版希尔伯特–瓦林定理的核心,并在黎曼ζ函数的研究中具有重要应用。华罗庚将这项工作写成了名为《素数的加法理论》的小册子,该书早在1940年就被接受在俄罗斯出版,但由于战争原因,直到1947年才作为斯捷克洛夫研究所的专著正式出版。在昆明时期的最后几年,华罗庚将兴趣转向代数和分析,并很快开始做出原创性贡献。

战后,华罗庚于1946年春季应伊凡·维诺格拉多夫(Ivan Vinogradov)的邀请,在苏联逗留了三个月,随后前往普林斯顿大学高等研究院。在普林斯顿,华罗庚从事了矩阵理论、多复变函数和群论的研究。此时,中国正处于内战之中,交通极为困难,为了“方便旅行”,中国当局为华罗庚在护照上授予了将军的军衔。

1948年春,华罗庚接受了伊利诺伊大学香槟分校的全职教授职务。然而,他在伊利诺伊的停留时间非常短暂。1949年10月,中华人民共和国成立,华罗庚希望成为新时代的一部分,尽管他已经在美国安定下来,但仍决定与妻子和孩子一起返回中国。
\subsubsection{在中国的后期事业(1950–1985)}  
回到中国后,华罗庚投身于教育改革,并积极组织研究生层次的数学活动,推动学校和新兴工业中的工人数学教育。1952年7月,中国科学院数学研究所成立,华罗庚被任命为首任所长。次年,他作为中国科学院26人代表团的一员前往苏联,旨在建立与俄罗斯科学界的联系。之后,他成为中国科学技术大学(USTC)数学系的首任主任,并担任副校长。中国科学技术大学是由中国科学院于1958年创建的一所新型大学,旨在培养适应经济发展、国防和科技教育所需的高技能研究人员。

尽管承担了许多教学和行政职责,华罗庚依然活跃于研究工作,并继续撰写学术论文,不仅是他之前涉及的主题,还包括一些新领域或此前仅浅尝辄止的领域。1956年,他的大作《数论引论》问世,后来该书被Springer出版社出版成英文版。1958年,《经典领域中多复变函数的调和分析》出版,并于同年被翻译成俄文,随后在1963年由美国数学学会出版了英文版。

在纯数学之外,华罗庚于1952年首次提出了中国电子计算机的发展构想,并于1953年初,在华罗庚的领导下,中国科学院数学研究所组建了该项目的初步研究团队。

1958年大跃进的开始带来了对纯数学和知识分子的激烈攻击,这促使华罗庚转向应用数学。华罗庚与王元共同开展了广泛的线性规划、运筹学和多维数值积分的研究。与最后一项研究相关,他们研究了蒙特卡洛方法和均匀分布的角色,并发明了一种基于代数数论思想的替代性确定性方法。他们的理论最终在1978年出版的《数论在数值分析中的应用》中得以阐述,1981年该书由Springer出版社出版英文版。对于应用数学的新兴趣带领他在1960年代,伴随着一队助手,走遍中国各地,向各种工人展示如何将推理能力应用于车间和日常问题的解决。无论是在工厂的临时问题解决会议还是在露天授课中,他都以数学的精神打动了听众,甚至成为了全国英雄,并且收到了毛泽东的未请求的表彰信,这在动荡时期为他提供了宝贵的保护。华罗庚拥有强大的气场、和蔼的个性以及将事物简单化的绝妙方式,他的旅行不仅传播了数学的魅力,也使他声名远扬。

文化大革命后,华罗庚恢复了与西方数学家的联系。1980年,华罗庚成为中国的文化大使,负责重新建立与西方学术界的联系。在接下来的五年里,他广泛访问了欧洲、美国和日本。1979年,他成为英国科学研究委员会的访问研究员,在伯明翰大学工作;1983至1984年,他是加利福尼亚理工学院的谢尔曼·费尔柴尔德杰出学者。他在1985年6月12日的东京讲座结束时因心脏病发作去世。

位于江苏金坛的华罗庚公园以他的名字命名。

