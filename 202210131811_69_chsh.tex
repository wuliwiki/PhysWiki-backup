% Bell 不等式与 CHSH 不等式
% keys EPR佯谬|定域隐变量理论|Bell 不等式|CHSH 不等式

\pentry{EPR 佯谬与定域隐变量理论\upref{EPR}}

1964 年,Bell 给出了基于定域隐变量假定得出 Bell 不等式,从而把 EPR 佯谬所引发的一些思辨和哲学性的问题定量化和具体化,得到一个能够通过实验验证的判据.在这之后,1969 年 John Clauser,Michael Horne,Abner Shimony 和 Richard Holt 四人提出了 CHSH 不等式,使得 Bell 检验的实行更容易. 由 CHSH 不等式也可以导出 Bell 不等式,从而得到 Bell 定理,即\textbf{局域隐变量理论}\footnote{爱因斯坦等人于 EPR 佯谬的论文中提出的一个不同于正统量子力学的理论,基于局域实在论的假设,并假设一个完备的理论需要隐变量去刻画.}不能再现量子力学纠缠的某些结果.因此下面将直接给出 CHSH 不等式的证明.

\subsection{Bell 实验与定域隐变量理论}
我们仍然以 Bohm 表述下的 Bell 实验为例,粒子源产生的两粒子 $A,B$ 处于自旋单态:

\begin{equation}
\ket{\Psi^-}=\frac{1}{\sqrt{2}}(\ket{\uparrow}\ket{\downarrow}-\ket{\downarrow}\ket{\uparrow})
\end{equation}

假设 Alice 通过用测量方式 $x$ 得到粒子 $A$ 沿 $\bvec n_x$ 方向的自旋.Bob 通过测量方式 $y$ 得到粒子 $B$ 沿 $\bvec n_y$ 方向的自旋.假设 Alice 得到的测量结果为 $a$,Bob 得到的测量结果为 $b$.

\begin{figure}[ht]
\centering
\includegraphics[width=12cm]{./figures/chsh_1.png}
\caption{Bell 实验} \label{chsh_fig1}
\end{figure}

为了为了简化讨论,假设测量结果取值为 $\{-1,1\}$,因此相应的物理量测量算符可以表达为 $\bvec \sigma \cdot \bvec n$(注意到自旋 $\bvec S=\bvec \sigma/2$ 在各个方向上的本征值为 $\pm 1/2$).

回忆定域隐变量的基本假定:在粒子源产生两粒子时,$A,B$ 两粒子携带着隐变量 $\lambda$,它可能包含了粒子源产生两粒子时周围的一切环境因素和其他人们不知道和无法测量的信息.总之隐变量 $\lambda$ 在集合 $\Lambda$ 中以 $q(\lambda)$ 的概率分布函数被随机采样.我们有以下基本关系式:

\begin{equation}
\begin{aligned}
&P(a|x)=\int \dd \lambda q(\lambda)P(a|x,\lambda),\\
&P(b|y)=\int \dd \lambda q(\lambda)P(b|y,\lambda),\\
&P(ab|xy)=\int \dd \lambda q(\lambda)P(ab|xy,\lambda)
\end{aligned}
\end{equation}
其中 $P(a|x,\lambda)$ 代表由测量 $x$ 和隐变量 $\lambda$ 导致得到测量结果 $a$ 的概率.根据定域论,$A,B$ 两粒子的测量结果在统计上应当是独立的,即:
\begin{equation}
P(ab|xy,\lambda)=P(a|x,\lambda)P(b|y,\lambda)
\end{equation}
\subsection{CHSH 不等式}
可以进行多次 Bell实验,不改变测量方式 $x$ 或 $y$,再对结果进行统计,取其平均值,那么得到的结果就是\textbf{数学期望值}.例如 Alice 测得结果 $a$ 的期望就是:
\begin{equation}
\langle a(x) \rangle =\int \dd \lambda \cdot q(\lambda) \sum_{a=\pm 1}a P(a|x,\lambda) = a P(a|x)
\end{equation}
上式是根据定域隐变量理论的基本假定得出的.根据正统量子力学的预言,结果 $a$ 的期望就是两粒子自旋单态的 $\bvec \sigma_A\cdot \bvec n_x$ 算符期望值:
\begin{equation}
\langle a(x)\rangle = \bra{\Psi^-} \bvec \sigma_A \cdot \bvec n_x \ket{\Psi^-} = 0
\end{equation}
这是因为我们产生的自旋单态中某一个粒子沿任意方向的 $\pm 1$ 结果的概率应当是相等的,因而期望值就是 $0$.

为了进一步研究两粒子自旋单态的纠缠性质,我们可以研究,当测量方式 $x,y$ 确定时,$a$ 与 $b$ 乘积的期望值:
\begin{equation}\label{chsh_eq1}
\begin{aligned}
\langle a(x)b(y)\rangle &= \sum_{a,b=\pm 1} ab P(ab|xy)=\int \dd\lambda \cdot q(\lambda) \sum_{a,b=\pm 1}abP(a|x,\lambda)P(b|y,\lambda)\\
&=\int \dd\lambda \cdot q(\lambda) \left(\sum_{a=\pm 1}aP(a|x,\lambda)\right)\left(\sum_{a=\pm 1}bP(b|y,\lambda)\right)\\
&\equiv \int \dd\lambda\cdot q(\lambda) A(x,\lambda) B(y,\lambda)
\end{aligned}
\end{equation}
其中我们定义了 $A(x,\lambda)=\sum_a aP(a|x,\lambda),B(x,\lambda)=\sum_b bP(b|y,\lambda)$.根据正统量子力学,对于一个自旋单态系统,上式是可以被精确计算的:
\begin{equation}
\begin{aligned}
\langle a(x)b(y)\rangle &= \bra{\Psi^-} (\bvec \sigma_A\cdot \bvec n_x)(\bvec \sigma_B\cdot \bvec n_y)\ket{\Psi^-}\\
&=-\bra{\Psi^-} (\bvec \sigma_A\cdot \bvec n_x)(\bvec \sigma_A\cdot \bvec n_y)\ket{\Psi^-}\\
&=-\bvec n_x\cdot \bvec n_y
\end{aligned}
\end{equation}
其中第二行运用了自旋单态的性质:$\bvec \sigma_B\ket{\Psi^-}=-\bvec \sigma_A\ket{\Psi^-}$ \footnote{即 $\ket{\Psi^-}$  是 $\bvec S=\bvec S_A+\bvec S_B$ 的本征值为 $0$ 的态.}.第三行用了泡利矩阵所满足的恒等式 $(\bvec \sigma \cdot \bvec x)(\bvec \sigma\cdot \bvec y)=\bvec x\cdot \bvec y+i\bvec \sigma\cdot (\bvec x\times \bvec y)$.

现在我们从\autoref{chsh_eq1} 出发,导出一些关系式.定域隐变量理论为了保护“定域性”而引入隐变量,实际上是一个很强的假定,相比于正统量子力学中对定域性的破坏,定域隐变量理论则是严格的遵守定域论的.这导致对纠缠态测量的统计结果很可能与正统量子力学相矛盾.让我们稍加推导:
\begin{equation}\label{chsh_eq2}
\begin{aligned}
\langle a(x) b(y)\rangle - \langle a(x)b(y')\rangle &= \int \dd\lambda q(\lambda) [A(x,\lambda)B(y,\lambda)-A(x,\lambda)B(y',\lambda)]\\
&=\int \dd\lambda q(\lambda)A(x,\lambda)B(y,\lambda)[1\pm A(x',\lambda)B(y',\lambda)]\\&\quad -\int \dd\lambda q(\lambda)A(x,\lambda)B(y',\lambda)[1\pm A(x',\lambda)B(y,\lambda)]
\end{aligned}
\end{equation}
注意到 $A(x,\lambda)=\sum_a a\cdot P(a|x,\lambda)$ 是 $x,\lambda$ 确定时 $a$ 的期望值,$a$ 的取值只有 $\pm 1$,因此有 $-1\le A\le 1$.类似地,$-1\le B\le 1$,$-1\le AB\le 1$.因此可以对\autoref{chsh_eq2} 的左右两侧取绝对值再进行放缩:
\begin{equation}
|\langle a(x) b(y)\rangle - \langle a(x)b(y')\rangle|\le \int \dd\lambda q(\lambda) [1\pm A(x',\lambda)B(y',\lambda)]+\int \dd\lambda q(\lambda) [1\pm A(x',\lambda)B(y',\lambda)]
\end{equation}
