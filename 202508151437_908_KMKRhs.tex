% 卡迈克尔函数(综述)
% license CCBYSA3
% type Wiki

本文根据 CC-BY-SA 协议转载翻译自维基百科\href{https://en.wikipedia.org/wiki/Carmichael_function}{相关文章}。

在数论(数学的一个分支)中,正整数 $n$ 的卡迈克尔函数$\lambda(n)$ 定义为满足下列条件的最小正整数 $m$:
$$
a^{m} \equiv 1 \pmod{n}~
$$
其中 $a$ 为任意与 $n$ 互素的整数。从代数的角度来看,$\lambda(n)$ 是模 $n$ 的整数乘法群的指数。由于这是一个有限阿贝尔群,必然存在一个元素,其阶等于该指数 $\lambda(n)$。这样的元素被称为模 $n$ 的本原 $\lambda$-根(primitive $\lambda$-root modulo $n$)。
\begin{figure}[ht]
\centering
\includegraphics[width=14.25cm]{./figures/133eee57c8a8a0bb.png}
\caption{卡迈克尔 λ 函数:$1 \le n \le 1000$ 时的 $\lambda(n)$(与欧拉 φ 函数对比)} \label{fig_KMKRhs_1}
\end{figure}