% 单链表
% 链表|单链表|数据结构

链表是一种用于存储数据的链式数据结构,形如一条链子一样来连接元素,通常用于存储树和图.

与数组不同的是:数组是一种支持随机访问,但不支持在任意位置插入或删除元素的数据结构.但链表支持在任意位置插入或删除,但只能按顺序依次访问其中的元素.

\textbf{单链表:}

\begin{figure}[ht]
\centering
\includegraphics[width=14.25cm]{./figures/List_1.png}
\caption{单链表示意图} \label{List_fig1}
\end{figure}

可以看到,链表上每个结点都有三个值,分别为:下标、值和 $\text{next}$ 指针.

下标为 $3$ 的结点的下一个结点为空结点,所以 $\text{next}$ 值为 $-1$.

如果用 C++ 的指针和结构体来写链表的话,长成这样子:
\begin{lstlisting}[language=cpp]
struct List
{
    int value;
    Node *next;
};
\end{lstlisting}

一般在竞赛中很少用上面这种方式来实现链表,因为这种写法效率很低,所以这里我们来讲一下如何使用数组来模拟链表.

数组模拟单链表需要这么几个数组和变量:
\begin{lstlisting}[language=cpp]
int value[N], Next[N], head, idx;
// N 表示数组的大小
// head 表示头节点指向的下一个值
// idx 表示插入的第几个数,后续模拟一下链表就懂了
// value[i] 表示结点 i 的值
// next[i] 表示结点 i 的 next 指针是多少
\end{lstlisting}

单链表通常有这么几个操作:
\begin{enumerate}
\item 向链表头插入一个数;
\item 删除第 $k$ 个插入的数后面的数;
\item 在第 $k$ 个插入的数后插入一个数.
\end{enumerate}

我们来模拟一个样例来更好的理解单链表,如要进行如下这些操作:

\begin{enumerate}
\item 先在链表头插入一个数 $9$
\item 在第 $1$ 个插入的数后面再插入一个数 $1$
\item 删除第一个插入的数的后面的一个数
\item 删除头结点后面的数
\item 在链表头插入一个数 $6$
\item 在第 $3$ 个插入的数后面再插入一个数 $6$
\item 在第 $4$ 个插入的数后面再插入一个数 $5$
\item 在第 $4$ 个插入的数后面再插入一个数 $5$
\item 在第 $3$ 个插入的数后面再插入一个数 $4$
\item 删除第 $6$ 个插入的数的后面一个数
\end{enumerate}

我们来借助几张图片来模拟一下上面的 $10$ 个操作

\begin{figure}[ht]
\centering
\includegraphics[width=14.25cm]{./figures/List_2.png}
\caption{执行 $1\sim3$ 次操作} \label{List_fig2}
\end{figure}

\begin{figure}[ht]
\centering
\includegraphics[width=14.25cm]{./figures/List_3.png}
\caption{执行 $4\sim7$ 次操作} \label{List_fig3}
\end{figure}

\begin{figure}[ht]
\centering
\includegraphics[width=14,25cm]{./figures/List_4.png}
\caption{执行 $8\sim9$ 次操作} \label{List_fig4}
\end{figure}

经过模拟了 $10$ 次单链表的操作,就可以理解单链表的执行过程了,让我们来看看这些操作具体怎么写.

\textbf{第一个操作}:\textbf{向头结点插入一个数}

代码如下:

\begin{lstlisting}[language=cpp]
void add_to_head(int x) 
{
    value[idx] = x;
    Next[idx] = head;
    head = idx;
    idx ++ ;
}
// 熟练掌握了之后就可以写成一行了.

value[idx] = x, Next[idx] = head, head = idx ++ ;
\end{lstlisting}

\begin{figure}[ht]
\centering
\includegraphics[width=14.25cm]{./figures/List_5.png}
\caption{向头结点插入一个数} \label{List_fig5}
\end{figure}



\textbf{第二个操作}:在第 $k$ 个插入的数后插入一个数


代码如下:

\begin{lstlisting}[language=cpp]
void add(int k, int x)
{
    value[idx] = x, Next[idx] = ne[k], ne[k] = idx ++ ;
}
\end{lstlisting}

插入方式和\textbf{插入一个数到头结点的后面}类似,这里就再不详讲了.

\textbf{第三个操作}:删除一个点的后面一个点

代码如下:

\begin{lstlisting}[language=cpp]
void remove(int k)
{
    Next[k] = Next[Next[k]];
}
\end{lstlisting}

