% 理想气体混合的熵变

\pentry{理想气体分压定律\upref{PartiP}, 熵的宏观表达式\upref{MacroS}}

\begin{issues}
\issueDraft
\end{issues}

\subsection{同种气体的混合}
\begin{figure}[ht]
\centering
\includegraphics[width=5cm]{./figures/ed271e2113ff3534.pdf}
\caption{同种气体的混合} \label{fig_IGME_1}
\end{figure}
首先,先让我们思考一种最简单的混合过程:同种气体的混合。当我们拿开挡板之后会发生什么?你觉得可能\textsl{什么都不会发生}。既然两侧的气体都是完全一样的,那有和没有挡板有什么区别吗?事实正是如此的简单(伴随着些许的诡异):在这种情况下,同种气体的混合并不会导致熵变。
\begin{equation}
\Delta S = 0
\end{equation}

\subsection{异种气体的混合}
\begin{figure}[ht]
\centering
\includegraphics[width=5cm]{./figures/f6a63cf8de808fbc.pdf}
\caption{异种气体的混合 } \label{fig_IGME_fig2}
\end{figure}