% 格林函数法解非齐次偏微分方程

\begin{issues}
\issueDraft
\end{issues}

\pentry{狄拉克 delta 函数\upref{Delta}}

例子: 一根两端固定的弦两端固定在 $x$ 轴上, 区间为 $[0, L]$, 张力为 $T$, 形状为 $y(x)$, 边界条件为 $y(0) = y(L) = 0$. 在弦上有 $y$ 方向的连续受力分布, 若弦的受力密度函数为 $f(x)$, 即单位长度受到的 $y$ 方向的力, 那么当 $\abs{f(x)} \ll T$ 时有方程(过程类比 “一维波动方程\upref{WEq1D}”)
\begin{equation}\label{GreenF_eq1}
-T y'' = f(x)
\end{equation}

虽然该方程可以直接对两边积分两次得到解, 但为了jiao'xu

我们用格林函数法, 先令格林函数 $G(x_0, x)$ 满足
\begin{equation}\label{GreenF_eq2}
-T G''(x_0, x) = \delta(x - x_0) \qquad (0 < x_0 < L)
\end{equation}
方程右边是狄拉克 $\delta$ 函数\upref{Delta}. 且同样有边界条件 $G(x_0, a) = G(x_0, b) = 0$. 这相当于弦上只有一点 $x_0$ 受大小为 $F = 1$ 的力.

解出格林函数后, $f(x)$ 可以分解为许多不同位置的 $\delta$ 函数的线性组合(积分)
\begin{equation}
f(x) = \int_0^L f(x_0) \delta(x - x_0) \dd{x_0}
\end{equation}
由于\autoref{GreenF_eq1} 的方程是线性的, 那么把 $G(x_0, x)$ 做同样的线性组合就是满足边界条件的解
\begin{equation}\label{GreenF_eq3}
y(x) = \int_0^L f(x_0) G(x_0, x) \dd{x_0}
\end{equation}

现在来解\autoref{GreenF_eq2}, 事实上我们可以直接从受力分析上得出格林函数 $G(x_0, x)$ 是一个三角形, 顶点的位置为 $x = x_0$, 令高为 $h = y(x_0)$ 由受力分析可得
\begin{equation}
T\frac{h}{x_0} + T\frac{h}{L - x_0} = F = 1
\end{equation}
即
\begin{equation}
h = \frac{x_0 (L - x_0)}{LT}
\end{equation}
即格林函数为
\begin{equation}
G(x_0, x) = \leftgroup{
&? \qquad (0 < x \le x_0)\\
&? \qquad (x_0 < x)
}\end{equation}

代入\autoref{GreenF_eq3} 得
\begin{equation}
y(x) = ?
\end{equation}
可以验证, $y(x)$ 符合微分方程.
