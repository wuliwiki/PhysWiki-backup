% 柱坐标中的亥姆霍兹方程
% keys 柱坐标|亥姆霍兹

\begin{issues}
\issueDraft
\end{issues}

\pentry{亥姆霍兹方程, 柱坐标系中的拉普拉斯方程\upref{CylLap}}

将柱坐标系中的拉普拉斯方程右边加上一项得到亥姆霍兹方程.
\begin{equation}
\frac{1}{r} \pdv{r} \qty(r\pdv{u}{r}) + \frac{1}{r^2} \pdv[2]{u}{\theta} + \pdv[2]{u}{z} = -k^2 u
\end{equation}
使用分离变量法, 令 $u = R(r) \Phi(\theta) Z(z)$, 代入方程得
\begin{equation}
\frac{1}{rR}\pdv{r} \qty(r\pdv{R}{r}) + \frac{1}{r^2 \Phi} \pdv[2]{\Phi}{\theta} + \frac{1}{Z} \pdv[2]{Z}{z} = -k^2
\end{equation}
$\Theta(\theta)$ 和 $Z(z)$ 的常微分方程和解与拉普拉斯方程中的相同, 径向方程变为
\begin{equation}
123
\end{equation}

