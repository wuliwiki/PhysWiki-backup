% 抽象代数(综述)
% license CCBYNCSA3
% type Wiki

本文根据 CC-BY-SA 协议转载翻译自维基百科\href{https://en.wikipedia.org/wiki/Abstract_algebra}{相关文章}。

\begin{figure}[ht]
\centering
\includegraphics[width=6cm]{./figures/e5b4bfc6385c6286.png}
\caption{魔方的所有排列构成一个群,这是抽象代数中的一个基本概念。} \label{fig_CXds_1}
\end{figure}
在数学中,更具体地说,在代数学中,抽象代数或现代代数是研究代数结构的学科。代数结构是指带有特定运算作用于其元素的集合。\(^\text{[1]}\)代数结构包括群、环、域、模、向量空间、格以及域上的代数。抽象代数这一术语是在20世纪早期提出的,用来将其与代数学的旧分支区分开来,更具体地,是为了区别于初等代数,即使用变量来表示数进行计算和推理的部分。如今,抽象的代数观点已成为高等数学中如此根本的内容,以至于通常直接称为“代数”,而“抽象代数”这一术语除了在教学中很少再被使用。

代数结构及其相关的同态构成数学上的范畴。范畴论提供了一个统一的框架,用来研究各种结构中类似的性质和构造。

一般代数是一个相关学科,研究将不同类型的代数结构作为单一对象来对待。例如,在一般代数中,群的结构是一个单一的对象,被称为群的多类。
\subsection{历史}
在19世纪之前,代数被定义为对多项式的研究。\(^\text{[2]}\)随着更复杂的问题和解法的发展,抽象代数在19世纪逐渐产生。具体的问题和例子来自数论、几何、分析以及代数方程的解。如今被认为是抽象代数组成部分的大多数理论,最初只是来自数学各个分支的一些零散事实的集合,随后逐渐形成了一个共同的主题,作为核心将各种结果汇集起来,并最终在一套共同概念的基础上实现统一。这种统一发生在20世纪的前几十年,促成了对群、环、域等各种代数结构的形式公理化定义。\(^\text{[3]}\)这一历史发展过程几乎与流行教材中的处理方式相反,例如范德瓦尔登的 《现代代数》\(^\text{[4]}\),这些教材通常在每一章开头给出某种结构的形式定义,然后再给出具体的实例。\(^\text{[5]}\)
\subsubsection{初等代数}
对多项式方程或代数方程的研究有着悠久的历史。大约公元前 1700 年,巴比伦人已经能够解出以文字题形式给出的二次方程。这个“文字题”阶段被称为修辞代数,并且一直到 16 世纪都是主流方法。公元 830 年,花拉子米首次提出“algebra(代数)”一词,但他的工作完全属于修辞代数。完全符号化的代数直到弗朗索瓦·维埃特1591 年的《新代数》才出现,即便如此,其中仍有一些拼写出的词语,直到笛卡尔1637 年的《几何学》才被赋予统一的符号表示。\(^\text{[6]}\)对符号方程求解的正式研究促使莱昂哈德·欧拉在 18 世纪末接受当时被认为是“荒谬”的根,例如负数和虚数。\(^\text{[7]}\)然而,大多数欧洲数学家直到 19 世纪中叶仍然抗拒这些概念。\(^\text{[8]}\)

乔治·皮考克1830 年的《代数论》是第一次尝试将代数完全建立在严格的符号基础之上。他区分了新的符号代数与旧的算术代数。在算术代数中,$a - b$被限制为$a \geq b$,而在符号代数中,所有的运算规则在没有任何限制的情况下成立。利用这一点,皮考克能够证明类似$(-a)(-b) = ab$这样的法则:只需令$a = 0,\ c = 0$代入$(a - b)(c - d) = ac + bd - ad - bc$即可成立。皮考克使用他称为等价形式永久性原理来为他的论证辩护,但他的推理存在归纳法问题。\(^\text{[9]}\)例如,$\sqrt{a}\sqrt{b} = \sqrt{ab}$对于非负实数成立,但对一般复数却不成立。
\subsubsection{早期群论}
数学的几个领域共同促成了对群的研究。拉格朗日1770 年对五次方程解的研究,促成了多项式的伽罗瓦群的诞生。高斯1801 年对费马小定理的研究引出了模 $n$ 的整数环、模 $n$ 的整数乘法群,以及更一般的循环群和阿贝尔群的概念。克莱因1872 年的埃尔朗根纲领研究几何学,并引出了对称群,例如欧几里得群和射影变换群。1874 年,李引入了李群理论,旨在发展“微分方程的伽罗瓦理论”。1876 年,庞加莱和克莱因引入了莫比乌斯变换群,以及它的子群,例如模群和Fuchs 群,其基础是分析中自守函数的研究。[10]

**群的抽象概念**在 19 世纪中叶缓慢发展起来。1832 年,伽罗瓦首次使用“群(group)”一词\[11],表示在复合运算下封闭的一个置换集合。\[12] 阿瑟·凯莱(Arthur Cayley)1854 年的论文《论群论》(*On the theory of groups*)将群定义为一个带有结合性复合运算和单位元 \$1\$ 的集合,这在今天被称为**幺半群(monoid)**。\[13] 1870 年,克罗内克(Kronecker)定义了一种抽象的二元运算,该运算是封闭的、交换的、结合的,并且具有左消去性质

$$
b \neq c \to a \cdot b \neq a \cdot c,
$$

\[14] 类似于有限阿贝尔群的现代运算规律。\[15] 韦伯(Heinrich Weber)1882 年对群的定义是一个封闭的二元运算,它满足结合律并具有左右消去性。\[16] 1882 年,沃尔特·冯·迪克(Walther von Dyck)首次要求\*\*逆元(inverse elements)\*\*必须作为群定义的一部分。\[17]

一旦抽象群的概念形成,各种结果就被重新用这种抽象框架加以表述。例如,西罗(Sylow)定理在 1887 年被弗罗贝尼乌斯(Frobenius)直接根据有限群的运算规律重新证明,尽管弗罗贝尼乌斯指出该定理可由关于置换群的柯西定理(Cauchy's theorem)以及“每个有限群都是某个置换群的子群”这一事实推出。\[18]\[19] 奥托·赫尔德(Otto Hölder)在这一领域尤为高产:他在 1889 年定义了**商群(quotient groups)**,1893 年定义了**群自同构(group automorphisms)**以及**单群(simple groups)**,并且完成了**约旦–赫尔德定理(Jordan–Hölder theorem)**。德德金(Dedekind)和米勒(Miller)独立地刻画了**哈密顿群(Hamiltonian groups)**并引入了两个元素的**换子(commutator)**的概念。伯恩赛德(Burnside)、弗罗贝尼乌斯(Frobenius)和莫利安(Molien)在 19 世纪末建立了有限群的**表示论(representation theory)**。\[18]

J. A. de Séguier 1905 年的专著《抽象群论要素》(*Elements of the Theory of Abstract Groups*)以抽象、一般的形式呈现了其中许多成果,把“具体的”群 relegated 到附录中,尽管它仅限于有限群。第一本关于有限与无限抽象群的专著是 O. K. Schmidt 1916 年的《抽象群论》(*Abstract Theory of Groups*)。\[20]

---

👉 需要我继续翻译 **“Rings and fields”** 或 **“Vector spaces and modules”** 等后续章节吗?
