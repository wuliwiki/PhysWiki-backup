% 克利福德代数(综述)
% license CCBYSA3
% type Wiki

本文根据 CC-BY-SA 协议转载翻译自维基百科\href{https://en.wikipedia.org/wiki/Clifford_algebra}{相关文章}。

在数学中,克利福德代数[a] 是由具备二次型的向量空间生成的一类代数。它是一个带有单位元的结合代数,同时具有一个特定子空间这一附加结构。作为 $K$-代数,它们推广了实数、复数、四元数以及若干其他超复数体系。[1][2] 克利福德代数理论与二次型理论及正交变换理论紧密相关。克利福德代数在几何学、理论物理以及数字图像处理等诸多领域中具有重要应用。该名称源自英国数学家威廉·金登·克利福德(William Kingdon Clifford, 1845–1879)。

在所有克利福德代数中,最为常见的是正交克利福德代数,它们亦称为(伪)黎曼克利福德代数,以区别于辛克利福德代数。[b]

\subsection{引言与基本性质}
克利福德代数是一个包含并由向量空间 $V$ 生成的有单位元的结合代数,其中 $V$ 定义在一个域 $K$ 上,并配备有一个二次型$Q: V \to K$。克利福德代数 $\mathrm{Cl}(V, Q)$ 是满足如下条件的“最自由”[c] 有单位元结合代数:
$$
v^{2} = Q(v) \cdot 1 \quad \text{对所有 } v \in V,~
$$
其中左侧的乘积是代数内部的乘法,而右侧的 $1$ 表示代数的乘法单位元(需注意,不同于域 $K$ 的乘法单位元)。这里“最自由”或“最一般”的含义,可以通过泛性质的概念形式化地加以刻画,如下所述。

当 $V$ 是有限维实向量空间且 $Q$ 非退化时,$\mathrm{Cl}(V, Q)$ 可以记作 $\mathrm{Cl}_{p,q}(\mathbf{R})$。这表示 $V$ 存在一个正交基,其中 $p$ 个基向量满足 $e_i^2 = +1$,而 $q$ 个基向量满足 $e_i^2 = -1$;$\mathbf{R}$ 表明这是定义在实数上的克利福德代数,即代数元素的系数均为实数。该基底可通过正交对角化方法获得。

由向量空间 $V$ 生成的自由代数可写作张量代数$\bigoplus_{n \geq 0} V^{\otimes n}$,即所有 $n$ 重张量积的直和。于是,克利福德代数可表述为该张量代数关于双边理想的商,其中理想由以下形式的元素生成:$v \otimes v - Q(v) \cdot 1, \quad \forall v \in V$.在商代数中,由张量积诱导的乘法通常用并置表示(例如 $uv$)。其结合性直接源于张量积的结合性。

克利福德代数具有一个**特定子空间** $V$,即嵌入映射的像。然而,一般来说,给定一个与克利福德代数同构的 $K$-代数,并不能唯一确定这样的子空间。

若底域 $K$ 中的 2 可逆,则上述基本恒等式可以改写为
$$
uv + vu = 2 \langle u, v \rangle \cdot 1 \quad \forall u, v \in V,~
$$
其中
$$
\langle u, v \rangle = \tfrac{1}{2}\big(Q(u+v) - Q(u) - Q(v)\big)~
$$
是通过极化恒等式由二次型 $Q$ 所关联的对称双线性型。

在此方面,特征为 2 的情形构成一个特殊情况。特别地,当 $\mathrm{char}(K) = 2$ 时,二次型并不一定能够唯一地决定一个满足$Q(v) = \langle v, v \rangle$的对称双线性型。[3] 因此,本文中的许多叙述均附带条件:底域的特征不为 2;若去除此条件,结论即不再成立。
\subsubsection{作为外代数的量子化}
克利福德代数与外代数密切相关。事实上,当 $Q = 0$ 时,克利福德代数 $\mathrm{Cl}(V, Q)$ 正是外代数 $\bigwedge V$。当底域 $K$ 中的 $2$ 可逆时,存在一个典范线性同构,将 $\bigwedge V$ 与 $\mathrm{Cl}(V, Q)$ 联系起来。也就是说,它们作为向量空间是天然同构的,但其乘法结构不同(在特征为 2 的情况下,它们依然是向量空间同构的,只是该同构不再是天然的)。克利福德代数中的乘法结合其特定子空间,比外代数的外积更加丰富,因为它利用了二次型 $Q$ 所提供的额外信息。

克利福德代数是一个滤过代数;其相关的分次代数正是外代数。

更准确地说,克利福德代数可以被看作是外代数的一种量子化(参见量子群),正如魏尔代数是对称代数的量子化一样。

魏尔代数与克利福德代数还具备进一步的 *-代数结构,它们可以统一地理解为超代数的偶次项与奇次项,这一点在 CCR 与 CAR 代数的研究中已有论述。
\subsection{泛性质与构造}
设 $V$ 是定义在域 $K$ 上的一个向量空间,且 $Q : V \to K$ 是 $V$ 上的一个二次型。在大多数感兴趣的情形中,域 $K$ 是实数域 $\mathbf{R}$、复数域 $\mathbf{C}$,或某个有限域。

一个克利福德代数 $\mathrm{Cl}(V, Q)$ 是一对 $(B, i)$[d][4],其中 $B$ 是一个定义在 $K$ 上的有单位元结合代数,$i : V \to B$ 是一个线性映射,并满足$i(v)^2 = Q(v) \cdot 1_B, \quad \forall v \in V$,这里 $1_B$ 表示代数 $B$ 的乘法单位元。其定义依赖于以下泛性质:若给定任意一个定义在 $K$ 上的有单位元结合代数 $A$,以及任意线性映射 $j : V \to A$,满足
$$
j(v)^2 = Q(v) \cdot 1_A, \quad \forall v \in V,~
$$
其中 $1_A$ 表示代数 $A$ 的乘法单位元,则存在唯一的代数同态$f : B \to A$使得下列图表交换(即 $f \circ i = j$)。
\begin{figure}[ht]
\centering
\includegraphics[width=6cm]{./figures/c396abc3eb93542f.png}
\caption{} \label{fig_KLFds_1}
\end{figure}
二次型 $Q$ 可以被(不必对称的[5])双线性型 $\langle \cdot, \cdot \rangle$ 取代,只要它满足$\langle v, v \rangle = Q(v), \quad \forall v \in V$.此时对 $j$ 的等价要求为
$$
j(v) j(v) = \langle v, v \rangle \cdot 1_A, \quad \forall v \in V.~
$$
当域的特征不为 2 时,该条件又可等价改写为:
$$
j(v) j(w) + j(w) j(v) = \big(\langle v, w \rangle + \langle w, v \rangle\big)\cdot 1_A, \quad \forall v, w \in V,~
$$
其中,双线性型在不失一般性的情况下还可以进一步要求为对称双线性型。

如上所述,克利福德代数总是存在的,其构造方法如下:从包含 $V$ 的最一般代数开始,即张量代数 $T(V)$,然后通过取适当的商来强制满足基本恒等式。在此,我们取 $T(V)$ 中由以下元素生成的双边理想 $I_Q$:
$$
v \otimes v - Q(v) \cdot 1, \quad \forall v \in V,~
$$
并定义克利福德代数为该商代数:
$$
\operatorname{Cl}(V, Q) = T(V) / I_Q.~
$$
该商代数所继承的环乘法有时被称为克利福德乘法[6],以区别于外积和数量积。

随后可直接验证,$\operatorname{Cl}(V, Q)$ 包含 $V$,并满足上述的泛性质,因此 $\operatorname{Cl}(V, Q)$ 在唯一同构意义下是唯一的;这也是通常称为“克利福德代数 $\operatorname{Cl}(V, Q)$”的原因。此外,该构造还蕴含了嵌入映射 $i$ 是单射。通常在表述时会省略 $i$,直接将 $V$ 视为 $\operatorname{Cl}(V, Q)$ 的一个线性子空间。

克利福德代数的泛性质刻画表明,$\operatorname{Cl}(V, Q)$ 的构造本质上是函子性的。换言之,$\operatorname{Cl}$ 可以被视为一个函子:从“带二次型的向量空间”范畴(其态射为保持二次型的线性映射)到“结合代数”范畴。泛性质保证了:在保持二次型的前提下,向量空间之间的线性映射可以唯一扩张为相应克利福德代数之间的代数同态。
\subsection{基与维数}
由于向量空间 $V$ 配备了一个二次型 $Q$,在特征不等于 $2$ 的情况下,$V$ 总是存在正交基。正交基的定义是:对于某个对称双线性型,满足
$$
\langle e_i, e_j \rangle = 0, \quad i \neq j,~
$$
以及
$$
\langle e_i, e_i \rangle = Q(e_i).~
$$
克利福德基本恒等式蕴含了对于正交基有:
$$
e_i e_j = - e_j e_i, \quad i \neq j,~
$$
并且
$$
e_i^2 = Q(e_i).~
$$
这使得对正交基向量的操作大为简化。若取一组互不相同的正交基向量的乘积$e_{i_1} e_{i_2} \cdots e_{i_k}$,则总可以通过有限次两两交换将其排成标准顺序,而总体符号由所需交换次数的奇偶性(即置换的符号)决定。

设 $V$ 在域 $K$ 上的维数为 $n$,并且 $\{e_1, \ldots, e_n\}$ 是 $(V, Q)$ 的一个正交基,则克利福德代数 $\operatorname{Cl}(V, Q)$ 在 $K$ 上是自由的,其一组基可以写作:
$$
\big\{\, e_{i_1} e_{i_2} \cdots e_{i_k} \;\big|\; 1 \leq i_1 < i_2 < \cdots < i_k \leq n, \; 0 \leq k \leq n \,\big\}.~
$$
其中,当 $k = 0$ 时的“空乘积”被定义为代数的乘法单位元。对每一个固定的 $k$,基元素的数量为$\binom{n}{k}$,因此克利福德代数的总维数为:
$$
\dim \operatorname{Cl}(V, Q) = \sum_{k=0}^n \binom{n}{k} = 2^n.~
$$
\subsection{例子:实克利福德代数与复克利福德代数}
最重要的克利福德代数是那些定义在实向量空间与复向量空间上的,并且配备有非退化二次型的情形。

每一个代数 $\mathrm{Cl}_{p,q}(\mathbf{R})$ 与 $\mathrm{Cl}_{n}(\mathbf{C})$ 都同构于 $A$ 或 $A \oplus A$,其中 $A$ 是一个满矩阵环,其元素取自 $\mathbf{R}$、$\mathbf{C}$ 或 $\mathbf{H}$(四元数)。关于这些代数的完整分类,可参见 克利福德代数的分类。
