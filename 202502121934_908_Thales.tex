% 泰勒斯定理(综述)
% license CCBYSA3
% type Wiki

本文根据 CC-BY-SA 协议转载翻译自维基百科\href{https://en.wikipedia.org/wiki/Thales\%27s_theorem}{相关文章}。

\begin{figure}[ht]
\centering
\includegraphics[width=6cm]{./figures/7d4f4b21f0b3ad05.png}
\caption{塔勒斯定理:如果AC是直径,B是直径圆上的一点,则角∠ABC是直角。} \label{fig_Thales_1}
\end{figure}
在几何学中,塔勒斯定理指出,如果A、B和C是圆上的不同点,其中AC是直径,则角∠ABC是直角。塔勒斯定理是内切角定理的一个特例,并作为欧几里得《几何原本》第三卷第31命题的一部分被提及和证明。[1] 通常认为该定理归功于米利都的塔勒斯,但有时也归功于毕达哥拉斯。
\subsection{历史}
\begin{figure}[ht]
\centering
\includegraphics[width=6cm]{./figures/290ee9c3032feba1.png}
\caption{“Non si est dare primum motum esse o se del mezzo cerchio far si puote triangol sì c'un recto nonauesse. – 但丁《神曲》天堂篇,第13章,100–102行Non si est dare primum motum esse, Or if in semicircle can be made Triangle so that it have no right angle. – 英文翻译:朗费罗”} \label{fig_Thales_2}
\end{figure}
巴比伦数学家在希腊数学家证明之前,就已经知道这个定理的特殊情况。[2]

米利都的泰勒斯(公元前6世纪初)传统上被认为是第一个证明该定理的人;然而,到了公元前5世纪,泰勒斯的著作已不复存在,后来的历史学家根据听说和推测,将许多发明和思想归于泰勒斯和毕达哥拉斯等智慧人物。[3][4] 普罗克鲁斯(公元5世纪)和狄奥根尼·拉厄尔修(公元3世纪)都提到了泰勒斯,并且记录了帕姆菲拉(公元1世纪)的话,称泰勒斯“是第一个在圆内刻画直角三角形的人”。[5]

据说泰勒斯曾前往埃及和巴比伦,在那里他学到了几何学和天文学的知识,并将这些知识带回希腊,途中发明了几何证明的概念,并证明了各种几何定理。然而,没有直接证据支持这些说法,它们很可能是后来的推测性合理化。现代学者认为,希腊的演绎几何学(如欧几里得《几何原本》中的几何学)直到公元前4世纪才发展起来,而泰勒斯所掌握的任何几何知识很可能只是观察性的。[3][6]

该定理出现在欧几里得《几何原本》第三卷(公元前300年左右)的第31命题中:“在圆中,半圆内的角是直角,较大弧段中的角小于直角,较小弧段中的角大于直角;此外,较大弧段的角大于直角,较小弧段的角小于直角。”

但丁·阿利吉耶里的《神曲·天堂篇》(第13歌,101-102行)在一段演讲中提到了泰勒斯定理。
\subsection{证明}  
\subsubsection{第一次证明}  
以下事实被使用:三角形内角和等于180°,等腰三角形的底角相等。
\begin{figure}[ht]
\centering
\includegraphics[width=6cm]{./figures/595a6672f1f0b6ab.png}
\caption{假设\(\overline{AC}\)是直径,则B点处的角度始终是直角(90°)。} \label{fig_Thales_3}
\end{figure}
\begin{figure}[ht]
\centering
\includegraphics[width=6cm]{./figures/55c56086a84a1af1.png}
\caption{证明的图示。} \label{fig_Thales_4}
\end{figure}
由于 \(\overline{OA} = \overline{OB} = \overline{OC}\),△OBA 和 △OBC 是等腰三角形,并且根据等腰三角形底角相等的性质,∠OBC = ∠OCB 且 ∠OBA = ∠OAB。

设 \(\alpha = \angle BAO\) 和 \(\beta = \angle OBC\)。△ABC三角形的三个内角分别是 \(\alpha\)、\((\alpha + \beta)\) 和 \(\beta\)。由于三角形内角和等于180°,我们有:
\[
\alpha + (\alpha + \beta) + \beta = 180^\circ~
\]
\[
2\alpha + 2\beta = 180^\circ~
\]
\[
2(\alpha + \beta) = 180^\circ~
\]
\[
\therefore \alpha + \beta = 90^\circ.~
\]
证毕。
\subsubsection{第二次证明}  
该定理也可以通过三角学来证明:设 \( O = (0, 0) \),\( A = (-1, 0) \),\( C = (1, 0) \)。然后 \( B \) 是单位圆上的一点 \((\cos \theta, \sin \theta)\)。我们将通过证明 \(\overline{AB}\)和 \(\overline{BC}\)是垂直的来展示 △ABC 形成直角——即它们的斜率的乘积等于 -1。我们计算 \(\overline{AB}\)和 \(\overline{BC}\) 的斜率:
\[
m_{AB} = \frac{y_B - y_A}{x_B - x_A} = \frac{\sin \theta}{\cos \theta + 1}~
\]
\[
m_{BC} = \frac{y_C - y_B}{x_C - x_B} = \frac{-\sin \theta}{-\cos \theta + 1}~
\]
然后我们证明它们的乘积等于 -1:
\[
m_{AB} \cdot m_{BC} = \frac{\sin \theta}{\cos \theta + 1} \cdot \frac{-\sin \theta}{-\cos \theta + 1}~
\]
\[
= \frac{-\sin^2 \theta}{-\cos^2 \theta + 1}~
\]
\[
= \frac{-\sin^2 \theta}{\sin^2 \theta}~
\]
\[
= -1~
\]
注意使用了毕达哥拉斯三角恒等式:\(\sin^2 \theta + \cos^2 \theta = 1\).
\subsubsection{第三次证明}