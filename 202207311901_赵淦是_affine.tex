% 仿射变换在解析几何中的应用
% 仿射变换 圆锥曲线
\begin{definition}{仿射变换}
设椭圆\,\(\frac{x^2}{a^2}+\frac{y^2}{b^2}=1\),其中\,\(a>b>0\),置变换:
$$x'=\frac{x}{a},y'=\frac{y}{b}$$
则椭圆化为单位圆\,\(C:x'^2+y'^2=1\)
\end{definition}
届时,我们可以就可以抛开繁琐的代数计算,运用几何性质解决问题.此前,我们先介绍仿射变换的几个性质.
\begin{lemma}{}
变换后,平面内任意一条直线的斜率变为原来的\,\(\frac{a}{b}\)
\end{lemma}
\begin{lemma}{}
变换后,平面上任意区域的面积变为原来的\,\(\frac1{ab}\)
\end{lemma}
\begin{lemma}{}
变换后,线段中点依然是线段中点;关于坐标轴对称的元素依然关于坐标轴对称;平面区域的重心保持不变
\end{lemma}
\begin{lemma}{}
变换前后,平行关系保持不变
\end{lemma}
\begin{lemma}{}
变换前后,平行线段的长度比保持不变
\end{lemma}
例 1 设椭圆 
:
 ,直线  交椭圆于点  和 ,点  为线段  的中点,求直线斜率


解:作变换  ,则椭圆化为单位圆 ,

故 

由性质3,  为  的中点

在圆中,由垂径定理, 

而  ,得 

由上述性质 1, 

后面还有很多,今天就先更到这里吧,同志们晚安!

我又来了~

例 2 设椭圆 
:
 ,直线  切椭圆于点 ,求直线斜率

解:作变换  ,则椭圆化为单位圆 ,

故 

在圆中,由切线定理, 

而  ,得 

由上述性质 1, 

例 3 设椭圆 
:
 ,过点   作椭圆的切线  和  分别切椭圆于点  和  ,求直线  斜率


解:作变换  ,则椭圆化为单位圆 ,

由圆的切线长定理, 

联立: 

得 

由性质1,  , 

例 4(椭圆第三定义) 设椭圆 
:
,  、 和  为椭圆上的点,点  和  关于原点对称,求证:  为一定值


证明:作变换  ,则椭圆化为单位圆 

则  为圆  的直径,所以  ,

由性质1, 

证毕

例 5 设椭圆 
:
,  和  为椭圆上的点,点  为  的中点,求证:  为一定值


证明:作变换  ,则椭圆化为单位圆 

由性质3,  为  的中点,由垂径定理,  , 

由性质1, 

证毕

TO BE CONTINUED and GOOD NIGHT

例 5 设椭圆 
:
,  和  为椭圆上的点,直线  交  轴于点  ,在  轴上求一点  ,使  轴平分 


解:作变换  ,则椭圆化为单位圆 ,则 

由性质3,  依然被  轴平分,记  交圆  于另一点 

易见 

又 

所以 

又 

故 

又 

故 

进而 


即 

由性质1, 



习题 11-13 已知点  在椭圆 
:
 内,过点  的直线  与椭圆  相交于  和  两点,且点  是线段  的中点,  为坐标原点.求  的最大值





答案明天公布,今后我们每天更新 1 至 2 道习题,并在次日公布答案

Sweet Dreams!

昨日习题答案:

令  ,则  , 


由性质3,  为  的中点

由垂径定理, 

故 



由性质2, 





习题 11-14 已知椭圆  的左焦点为  ,离心率为  .

(1)求椭圆  的标准方程;

(2)设  为直线  上一点,过点  作  的垂线交椭圆于  两点,当四边形  是平行四边形时,求四边形  的面积.



(椭圆的标准方程是  ,但我相信你会自己去算而不是直接抄答案,对吧~)



SEE U TOMORROW!



昨日习题答案:


令  ,则  , 

由性质4,  仍为平行四边形,又  ,  为菱形

故 

记 
轴
 于 

因为 

所以 

故 

得 

由垂径定理, 

则 
菱
形

由性质2, 
四
边
形



习题 11-15 已知椭圆 
:
 ,过左焦点  的直线  交椭圆于  两点,  为  的中点,  为坐标原点. 若  是以  为底边的等腰三角形,求直线  的方程



重榜消息

今天,我又发现了一条有趣的性质:

例 5 过椭圆 
:
 上任一点  作两条倾斜角互补的直线交椭圆于点  ,求证:  为一定值

(时间有点晚了,我先把图贴在这里,明天再来更证明过程吧,对不起!)




SEE YOU TOMORROW!

昨日习题答案:


令 ,则 

由性质3,  仍旧关于铅垂线对称

在圆中,由垂径定理, 

故  为等腰直角三角形, 

设  ,于是 

解得  ,由性质1,  , 



昨天例5的证明过程:


证明:作变换  ,则椭圆化为单位圆  , 

由性质3,  仍然关于铅垂线对称,故 

同弧所对的圆周角等于圆心角的一半,故 

又  ,等腰三角形三线合一,所以  , 

又因为  ,所以 

而  ,因此 

由性质1, 

证毕





明天我们将要讲解有关椭圆内接、外切三角形的性质,敬请期待!

ONE MORE DAWN,ONE MORE DAY,ONE DAY MORE!

例 6 求椭圆  内接三角形的最大面积和外切三角形的最小面积


解:作变换  ,则椭圆化为单位圆 

由切比雪夫不等式, 
内
外

由性质2, 
内
外



例 7 求以  为重心的椭圆  的内接三角形  的面积


解:作变换  ,则椭圆化为单位圆 

由性质3,  仍为  的重心,故  ,  为等边三角形

故  ,由性质3, 



下面我们引入性质5:

性质 5 变换前后,平行线段的长度比保持不变
例 8 已知椭圆 
:
,过  的直线  交  于  两点,且  ,求  斜率

解:作变换  ,则椭圆化为单位圆 

由性质5,仍有  ,记  ,于是 

由切割线定理,  ,其中  为单位圆与  轴的两交点

即 

接着:

由垂径定理, 

整理,由性质1得:

我愿称这玩意为“软解定理”

SWEET DREAMS!



例 9 设椭圆 
:
 , 为第三象限内椭圆上的一点,  分别是椭圆的上顶点、右顶点,直线  与  轴交于点  ,直线  与  轴交于点  ,求证:四边形  的面积为定值


证明:作变换  ,则椭圆化为单位圆 

于是 

于是


由性质2, 

证毕

习题 11-18 设椭圆 
:
,  分别是椭圆的焦点以及同侧准线与  轴的交点,椭圆  以  为长轴且与  相似,  相交于  ,证明:  与  相切



周末愉快 !
例 10(2017年全国 I 卷,理数)


解:(1)易求得椭圆的方程为 

(2)


作变换  ,则椭圆化为单位圆 

记如图所示的  ,则 

 ,由性质1,  ,即 

移项,得 

进而 

展开,得 

所以 

整理,得 

得 

由万能公式, 

所以  恒过定点 

所以  恒过定点 

