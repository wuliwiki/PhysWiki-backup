% 相机的定位
% 三棱锥|相机|焦点

\pentry{相机模型\upref{CamMdl}}

若我们已知世界系中若干点 $\bvec p_i$ ($i = 1, \dots, N$)的坐标(假设任意 3 个 $\bvec p_i$ 不在一条直线上), 而一个相机同时拍摄了这些点, 我们在一定条件下可以根据得到的图片确定相机的位置和朝向.

若 $N = 3$ 且 , 这个问题一般有 3 个解, 但若 $N = 4$ 时, 就只有一个解. 当 $N$ 更大时, 可以用多余的点进一步减小计算误差.

我们可以通过计算得到图片上任意两点的夹角, 所以当 $N = 3$ 时, 我们相当于在解三棱锥: 已知三棱锥三条底边的长度以及顶点的三个顶角, 求顶点的坐标. 如果把三条棱的长度作为未知数, 我们会得到非线性的三元二次方程组, 这个方程组最多有三个解.

(方程组未完成)

另一种方法是使用长方形, 见“长方形相机定位法\upref{RecCam}”.
