% 量子纠缠 2
% keys 纠缠|entanglement|施密特秩
% license Xiao
% type Tutor

\pentry{约化密度矩阵\nref{nod_partra}}{nod_31fd}

我们研究一个二元纯态 $\ket{\psi}_{AB}$ 的子系统 $A$,假设整个大的孤立系统的 Hilbert 空间可以表示为两个子 Hilbert 空间的张量积:$\mathcal{H}_A\otimes \mathcal{H}_B$,其中 $\mathcal A$ 是待研究的子系统的 Hilbert 空间,设约化密度算符为 $\rho_A=\text{tr}_B \ket{\psi}\bra{\psi}$。在\enref{约化密度矩阵}{partra}文章中,我们证明了 $A,B$ 处于纠缠态的一个判据 \autoref{eq_partra_2}。

\begin{equation}
\text{tr} \rho^2 <1~.
\end{equation}

它意味着约化密度算符在某个正交完备基下对角化以后,表现为 $\mathcal{H}_A$ 中若干个纯态(大于一个)组成的系综,系综中每个纯态有 $p_a<1$ 的概率出现。这也意味着\textbf{当且仅当} $\rho_A$ 的\textbf{施密特秩}\footnote{类似于\enref{矩阵的秩}{MatRnk},可以将这一概念推广到任意线性算符。}大于 $1$, $A,B$ 处于纠缠态。在此处我们考察的约化密度算符是正定算符,因此施密特秩等于正的本征值的个数。若本征值个数 $>1$,体系处于纠缠态,我们称 $A,B$ 之间具有\textbf{量子相关性}。

如果施密特秩为 $1$,那么约化密度算符可以表示为 ${}_A\ket{\varphi}\bra{\varphi}_A,\varphi_A\in \mathcal{H}$,此时 $A$ 与 $B$ 之间是不纠缠的,或者被称为\textbf{可分的}(seperable)。此时二元纯态 $\ket{\psi}$ 实际上可以表示为两个子系统的量子态的张量积:
\begin{equation}
\ket{\psi}_{AB}=\ket{\varphi}_A\otimes \ket{\varphi}_B~.
\end{equation}

在量子力学中,纠缠态是量子系统的一种特殊状态,它表明系统的各个子部分之间存在非经典的关联。下面我们将进一步探讨纠缠态,并介绍一个重要的量化纠缠度量工具——纠缠熵。

二元纯态与约化密度算符
我们研究一个二元纯态 $\ket{\psi}_{AB}$ 的子系统 $A$,假设整个大的孤立系统的 Hilbert 空间可以表示为两个子 Hilbert 空间的张量积:$\mathcal{H}_A \otimes \mathcal{H}_B$,其中 $\mathcal{H}_A$ 是待研究的子系统的 Hilbert 空间,设约化密度算符为