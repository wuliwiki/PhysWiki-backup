% 矩阵的秩
% 矩阵|秩|满秩矩阵|方阵

\pentry{行列式\upref{Deter}}

\begin{definition}{矩阵的秩}
定义矩阵\upref{Mat}的\textbf{列秩(column rank)}等于其线性无关\upref{linDpe}的列数, \textbf{行秩(row rank)}等于线性无关的行数. 由于任意的矩阵的行秩和列秩相等, 可以直接称为矩阵的\textbf{秩(rank)}.
\end{definition}
行秩和列秩相等的证明见\autoref{RCrank_cor1}~\upref{RCrank}.

\begin{example}{}
矩阵
\begin{equation}\label{MatRnk_eq1}
\pmat{1 & 1 & 4\\ 1 & 0 & 2\\ 0 & 1 & 2}
\end{equation}
中, 如果我们把矩阵看做是三个列矢量 $\bvec v_1, \bvec v_2, \bvec v_3$ 组成, 那么 $\bvec v_1, \bvec v_2$ 显然是线性无关的(不共线), 而 $\bvec v_3$ 可以表示为 $\bvec v_1, \bvec v_2$ 的线性组合, 即
\begin{equation}
\bvec v_3 = 2\bvec v_1 + 2\bvec v_2
\end{equation}
所以它们之中只有两个矢量线性无关, 该矩阵的秩为 2.

当然,我们也可以认为 $\bvec v_2, \bvec v_3$ 线性无关, 排除 $\bvec v_1$, 同样得到秩为 2. 一般来说, 若有(\autoref{linDpe_eq2}~\upref{linDpe})
\begin{equation}
\sum_i c_i \bvec v_i = \bvec 0 \qquad (c_i \ne 0)
\end{equation}
我们就可以把任意一个 $\bvec v_i$ 排除, 再次求解上式, 直到上式无解, 那么可以确定剩下的矢量就是线性无关的, 他们的数量就是矩阵的秩. 这种方法计算量过大, 下文我们会介绍更简单的方法.
\end{example}

根据定义, 一个矩阵的秩 $R$ 必定小于或等于矩阵的行数 $M$ 以及列数 $N$(取较小者). 对于方阵, 若三者相等, 即 $M = N = R$, 我们就称其为\textbf{满秩矩阵(full rank matrix)}. 满秩矩阵意味着矩阵中所有行(列)都线性无关.

判断一个方阵是否为满秩矩阵的一种常见方法是计算方阵的行列式\upref{Deter}, 若结果不为零, 则矩阵是满秩的, 否则不是(\autoref{DetPro_the2}~\upref{DetPro}). 注意非满秩的情况下行列式并不能判断秩具体是多少. 另一种更简单的方法是使用下面的高斯消元法.

\subsection{高斯消元法计算秩}
\pentry{高斯消元法\upref{GAUSS}}
要确定任意矩阵秩的大小, 我们可以先用高斯消元法将矩阵变换为梯形矩阵. 矩阵的秩数就是梯形矩阵中不为零的行数. 这是因为行变换不会改变矩阵的秩.

\textbf{证明}: 如果通过行变换可以把矩阵的某行变为零, 那么就说明该行必定可以表示为其他行的线性组合; 而梯形矩阵中每一个不为零的行都无法通过行变换变为零, 所以他们都是线性无关的(具体证明留作习题). 证毕.

\begin{example}{}
我们用高斯消元法计算\autoref{MatRnk_eq1} 的秩, 该矩阵经过行变换 $\bvec r_2 \leftrightarrow \bvec r_1$, $\bvec r_2 - \bvec r_1$, $\bvec r_3 - \bvec r_2$ 后变为梯形矩阵
\begin{equation}
\pmat{1 & 0 & 2\\ 0 & 1 & 2\\ 0 & 0 & 0}
\end{equation}
有两个不为零的行, 所以矩阵的秩为 2.
\end{example}

同理,我们也可以把矩阵先做转置再用高斯消元法计算其线性无关的列. 这么做可以验证给定矩阵的行秩等于列秩.
