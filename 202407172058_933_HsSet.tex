% 集合(高中)
% keys 高中|集合
% license Usr
% type Tutor

\begin{issues}
\issueDraft
\end{issues}

\subsection{前言}

学习初中数学时,你很有可能感受到了,每个部分都有它自己的一套体系和语言(比如,几何证明时的“平行”$\mathbin{\!/\mkern-5mu/\!}$就不会出现在解方程的过程中,而解方程时的“未知数”$x$也不会成为几何证明的要素),每个部分原本就这样各自发展着,从未想过彼此之间有什么可以联系起来的可能。但经过一代代数学家们的不断努力,现代数学展现在了世人面前,它最重要的一个特点就是将整座数学大厦建立在了集合论的基础之上,从此各部分不再是孤立的。同时,集合也能够简洁而准确地表达其他数学内容。这使得,集合语言不但成为了所有数学分支的通用语,也使得不论母语是什么的研究者都可以用相同的语言来展现和理解某个数学思想。可以这么说,不能掌握集合,数学之路就寸步难行。

因此,高中数学理所当然地以“集合”作为一切后面学习的开始,同时,高中阶段也并不要求对集合进行过深的探索。下文会首先介绍集合和元素这两个概念、表示方法以及性质,然后再分别介绍集合与元素的关系以及集合之间的关系。

提醒一下:
\begin{itemize}
\item 这篇文章会涵盖不少与集合相关的新概念(当然也包括集合本身),它和你以往的数学经验不太会有直接的关系,但请不要被这些概念的数量吓倒,它只是名字比较新,事实上却与生活经验联系密切,每个完成高中阶段学习的同学都会认为“集合”是最简单的。
\item 请务必理解这部分的内容,哪怕暂时不熟悉这些新概念的名称,也一定要理解他们的意思。两个很有帮助的理解方向是:为什么需要这个概念?它和其他概念之间的有哪些差异?在文中也会尽可能指明它们来帮助理解。如果学有余力,借助这个机会可以感受一下高中数学思考方式的不同,它是会延伸到高等教育中数学乃至其他学科中。由于这些概念很常用,在使用时再逐步熟悉名称就可以,不必因为名称太多记不住而紧张。
\end{itemize}

\subsection{集合与元素}

\begin{example}{塑料袋与水果}
我给了你一个塑料袋,求你帮我把它送给另一个人。塑料袋嘛,就是用来装东西的,里面装了三种水果:苹果、香蕉、橙子。至于多少个,我没说你也没看,反正不轻。

你拎着这些东西到了他那里。他随口问了一声:“这个塑料袋里,装没装苹果?”你打开一看,有,就回答他“装了。”你刚想放下,他又问:“这个塑料袋里,装没装西瓜?”你只能又打开一看,没有,就回答他“没装。”他又要张嘴问你,奈何你的手已经不堪重负,于是你把塑料袋放地上了,跟他说:“东西我都放这里了,都在袋子里面我也没动,挺多的你自己看吧。”
\end{example}

数学上,这个塑料袋称为集合,而里面的水果就称为元素。就像刚才的问问题的那个人只关注塑料袋里有没有某种水果一样,研究集合时,也只关注一个问题,那就是这个集合里面有没有某个元素。至于水果好不好吃、塑料袋会不会破、你的手疼不疼,那都不是集合需要关注的问题。

相信你现在已经对这个塑料袋大概有印象了,请记住塑料袋这个例子,每次遇到跟集合相关的问题时,用塑料袋来理解都蛮好用。现在我们回到数学上来,给出一个集合的概念\footnote{这里的集合概念是朴素的,之后因为引发了一些问题,导致数学家们又创立了新的定义,这个问题直到现在还莫衷一是。但就如一开始所说,高中阶段对集合的要求并不这么艰深,毕竟这事现在也没说太准。},不过其实跟刚才说的塑料袋是一样的,它跟塑料袋的区别也不过就是用了书面语罢了。

\begin{definition}{集合}
一定范围内,某些能够确定的(well-defined,也称良定义的)、不同的对象(object)构成的整体(collection)称为集合\textbf{集合}(set),而构成集合的每个对象则称为该集合的\textbf{元素}(element)。

集合常记作大写拉丁字母 $A,B,C,D,\cdots$,而元素则常记作小写拉丁字母 $a,b,c,d,\cdots$ 。
\end{definition}

数的集合简称\textbf{数集}(number set),由于数在数学领域有特别的地位,一般对这些特殊的集合采用一些特定的记号\footnote{在高等数学领域,这些数集的记号为$\mathbb{N,N^+,Z,Q,R}$,既表示他们的地位特殊,同时这些集合的定义都是广泛明确的,使用这个记号会方便交流。}:
\begin{itemize}
\item 自然数组成的集合简称\textbf{自然数集},记作 ${\rm \mathbf{N}}$
\item 正整数组成的集合简称\textbf{正整数集},记作 ${\rm \mathbf{N^+}}$ 或 ${\rm \mathbf{N^*}}$; 
\item 整数组成的集合简称\textbf{整数集},记作${\rm \mathbf{Z}}$
\item 有理数组成的集合简称\textbf{有理数集},记作 ${\rm \mathbf{Q}}$
\item 实数组成的集合简称\textbf{实数集},记作 ${\rm \mathbf{R}}$
\end{itemize}

如果集合里元素的个数有限叫\textbf{有限集},如果集合里元素的个数无限叫\textbf{无限集}。

\subsection{性质}
\begin{enumerate}
\item 集合中的元素是\textbf{确定}的。
\item 集合中的元素是\textbf{互异}的。
\item 集合中的元素是\textbf{无序}的。
\end{enumerate}

每个元素在集合中都是独一无二的。换句话说,如果你已经把一个苹果放进袋子里,再次放入一个苹果是没有意义的
在塑料袋里,物品的顺序是不重要的。你可以先放苹果再放香蕉,也可以先放香蕉再放苹果,袋子里的内容都是一样的。
\subsection{集合的表示方法}

\subsubsection{枚举法}

由于只要确定了集合内的所有元素,集合就能确定下来了。\textbf{枚举法}就是把集合中的元素全部写在大括号内,用来表示集合,例如:用$\{1,2,3\}$表示由数字1,2,3三个元素构成的集合。特别地,由于有些集合的元素比较多,全都写出来的话比较麻烦,因此在不产生歧义的前提下,一般可以将大括号内部过多的元素用“$\cdots$”代替,例如想要表示集合A是从0到50的整数构成的集合就可以这样:

\begin{equation}
A=\{0,1, \cdots ,49,50\}.~
\end{equation}


\subsubsection{描述法}

用确定的条件表示某些对象属于一个集合并写在大括号内的方法叫\textbf{描述法},符号表示为 $\begin{Bmatrix} | \end{Bmatrix}$,如 $A=\begin{Bmatrix} x|p(x) \end{Bmatrix}$.

\subsubsection{图示法}

\begin{figure}[ht]
\centering
\includegraphics[width=10cm]{./figures/e449e54347ae8e24.png}
\caption{Veen图} \label{fig_SufCnd_1}
\end{figure}

\textbf{韦恩图(Veen diagram)}是一种用圆圈来表示集合的一种草图。它直观易于理解,通常用来表示集合间的大致关系。在进行集合间关系的分析时,非常有效。但它的缺点也很明显,就是不够严谨。也正因如此,也因为他是草图,所以在平时证明时只能用于自己理解的辅助出现在草纸上,而不能作为理由直接出现在证明、计算过程或试卷上。


\subsection{元素与集合的关系}

\begin{definition}{属于与不属于}
若 $a$ 在集合 $A$ 中,就说 $a$ \textbf{属于(belong to)}集合 $A$ ,记作 $a \in A$.若 $a$ 不在集合 $A$ 中,就说 $a$ \textbf{不属于}集合 $A$,记作 $a\notin A$。
\end{definition}

就像一个完全空的塑料袋一样,数学中也有一个特殊的集合叫做空集

\begin{definition}{空集}
如果一个集合不含有任何元素,称为\textbf{空集}(empty set),记作 $\varnothing$,即:
\begin{equation}
\forall a,a\notin\varnothing.~
\end{equation}

\end{definition}

\subsection{集合与集合的关系}
\begin{definition}{集合相等}
对集合A、B,若他们的元素完全相同,则称他们\textbf{相等}(equal),记作:
\begin{equation}
A=B.~
\end{equation}
\end{definition}


如果你有两个塑料袋,一个袋子装的物品全部都在另一个袋子里,那么我们说前一个袋子是后一个袋子的子集。
\begin{definition}{子集}
对两个集合$A,B$,若$A$ 的所有元素都属于集合$B$,即$\forall a\in A\implies a\in B$,则称集合 $A$ 是集合 $B$ 的\textbf{子集}(subset),或者说集合 $A$ \textbf{包含于}集合 $B$,记作
\begin{equation}
A \subseteq B~\text{或}B \supseteq A~.
\end{equation}
\end{definition}

显然,\textbf{任何一个集合都是它本身的子集},即
\begin{equation}
A \subseteq A~.
\end{equation}

\begin{definition}{真子集}
对于两个集合,$A$ 与 $B$,如果 $A\subseteq B$ ,并且 $A \ne B$,我们就说集合 $A$ 是集合 $B$ 的\textbf{真子集},记作\footnote{这里给出的记法是人教版高中课本上的,在高中阶段请只使用这种写法。事实上,还有$A\subset B,B\supset A$和$A\subsetneq B,B\supsetneq A$两种写法用来表示“$A$是$B$的真子集”,且前一种更常用。}
\begin{equation}
A \subsetneqq B~,
\end{equation}
也可记作
\begin{equation}
B \supsetneqq A~.
\end{equation}
\end{definition}



当集合 $A$ 不包含于集合 $B$ 或集合 $B$ 不包含集合 $A$ 时,记作
\begin{equation}
A \nsubseteq B~,
\end{equation}
也可记作
\begin{equation}
B \nsupseteq A~.
\end{equation}

我们规定:\textbf{空集是任何集合的子集}。也就是说,对于任何一个集合 $A$ 都有
\begin{equation}
\varnothing \subseteq A~.
\end{equation}

在研究某些集合的时候,这些集合往往是某个给定集合的子集,这个给定的集合叫做\textbf{全集(universal set)},常用符号 $U$ 表示。全集含有我们所要研究的这些集合的全部元素。


