% 热力学与统计力学导航

\begin{issues}
\issueDraft
\end{issues}

% 这一段是 bing ai 写的%
热力学是从宏观角度研究物质的热运动性质及其规律的学科 。它主要关注物质系统在平衡状态或准平衡态下的能量转化和物态变化,以及系统与外界之间的能量传递和转换。热力学不涉及物质的微观结构,而只用少数几个可直接感受和可观测的宏观状态量,如温度、压强、体积、浓度等,来描述和确定系统所处的状态。热力学基于实验事实总结出了四条基本定律,即零、一、二、三定律,它们分别揭示了系统状态的唯一性、能量守恒、熵增加和绝对零度不可达等重要原理。热力学在物理学、化学、生物学等领域有广泛的应用,也是工程技术中不可缺少的基础理论。




在物理学本科的专业课程中, 往往会先学习热力学再学习统计力学, 后者高级难度更大。 热力学几乎直接开始研究系统的宏观性质如温度,体积,压强等, 而统计力学更强调如何由微观定律如经典力学和量子力学, 用统计的方法导出宏观性质。 统计力学对数学工具以及物理基础知识的要求上比热力学要更高, 但也更能揭示宏观现象的本质。

\subsection{热力学 Ⅰ}
热力学最经典的入门内容莫过于\textbf{理想气体及其状态方程}\upref{PVnRT}。 通过该模型, 我们可以定义温度和温标\upref{tmp}。 状态方程只能决定系统达到平衡后的状态, 于是下一步就是学习系统宏观性质随时间缓慢变化的过程, 也就是\textbf{准静态过程}。 在该过程中, 在每一个时刻都可以把系统视为热平衡的。 于是我们可以研究理想气体的一些常见过程: 等压过程\upref{EqPre}, 等体过程\upref{EqVol}, 等温过程\upref{EqTemp}, 绝热过程\upref{Adiab}。这些过程的背后都有一个基本定律支撑——\textbf{热力学第一定律}\upref{Th1Law},也就是热力学中的能量守恒定律,它描绘了能量、吸热与做功之间的关系。

为了用简洁的公式对气体的任意过程作分析计算,\textbf{理想气体}\upref{Igas}是非常好的模型。通过理想气体状态方程\upref{PVnRT}和理想气体内能公式\upref{IdgEng}和第一定律的运用,我们可以分析理想气体的任意过程以及循环,例如重要的\textbf{卡诺热机}\upref{Carnot}。我们将发现,热机的效率与最高温度和最低温度的比值有关,而且效率总是小于 $1$。对卡诺热机的研究为\textbf{热力学第二定律}\upref{Td2Law}埋下了基础——我们对理想气体和可逆过程的研究启发了我们对不可逆过程的探索,而第二定律也意味着第二类永动机是不可能的。

在热力学第二定律中,我们第一次提出了 \textbf{熵} 的概念。这是一个极其重要的概念。随后我们要学习的是熵的宏观定义\upref{Entrop}和微观定义 \upref{IdeaS}(我们将在统计力学中再次学习微观定义),这揭示了熵与系统吸热、系统混乱度的联系。熵与混乱度、时间箭头的联系使我们对宏观世界产生无穷的联想。

\subsection{热力学 Ⅱ}

当我们有了热学基础以后,我们可以开始从理想气体过渡到真实气体或液体,真实气体比理想气体要复杂的多。为了对真实气体的不同状态有更深入的了解,我们也从热力学平衡态状态入手。我们需要重新审视\textbf{态函数}\upref{statef}的概念,回顾关于\textbf{热平衡}\upref{TherEq}的种种事实。\textbf{热力学第零定律}\upref{TherEq}不再是无关紧要的事实,而是联系不同物质系统的桥梁。有了态函数的概念,我们可以通过定义\textbf{焓}、\textbf{亥姆霍兹自由能}\upref{HelmF}、\textbf{吉布斯自由能}\upref{GibbsG},来研究不同条件下热力学平衡的特征,即热动平衡判据\upref{equcri}。在学习过程中,我们经常要用到各种和偏导数的公式,我们必须要了解和全微分\upref{TDiff}、泰勒展开\upref{NDtalr}有关的数学知识。\textbf{麦克斯韦关系}\upref{MWRel} 是对态函数的偏导数进行分析而得出的重要热力学关系,有了麦克斯韦关系,我们可以对热力学平衡系统的状态作充分的计算。有了重要的数学工具和热力学关系,我们可以重新审视态函数熵,得出理想气体的熵关于温度和压强的具体表达式\upref{MacroS}。

有了吉布斯函数、亥姆霍兹自由能和化学势的概念,以及重要的热动平衡判据\upref{equcri},我们可以开启复相体系、多元体系的研究。
……


\subsection{统计力学}

在学习宏观性质的另一面,我们还要对热学物理量的\textbf{微观图景}有一定了解。我们需要掌握一定的概率知识,以对气体分子的速度分布\upref{VelPdf}的图景有一定把握。我们需要学习气体分子对容器壁的压强在微观上是怎样的\upref{MolPre},分子平均碰壁数的简要推导\upref{AvgHit},以及重要的\textbf{麦克斯韦——玻尔兹曼分布}\upref{MxwBzm}。
\addTODO{微观块缺少大量词条,有待补充}

统计力学方面的导航:统计物理·微观与宏观之间的桥梁\upref{statsc}。
