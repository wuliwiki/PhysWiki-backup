% 指数有限度量空间
% keys 指数有限度量|迷向矢量|伪欧空间|闵可夫斯基空间

\begin{issues}
\issueDraft
\end{issues}

\pentry{正定二次型\upref{DeQua},矢量空间\upref{LSpace}}
带有内积的矢量空间是指在矢量空间\upref{LSpace} $V$ 上配备一个固定的二次型\upref{QuaFor}
\begin{equation}
q(x)=f(x,x)=\sum_{i,j}a_{ij}x_ix_j~,
\end{equation}
 即内积空间是一个二元组 $(V,q)$。例如,欧几里得矢量空间配备的是一个正定二次型(\upref{EuVS}),而埃尔米特矢量空间配备的是正定的埃尔米特型(\upref{HVorUV})。此外,配备一个不定型的矢量空间同样起着重要的作用,这样的空间称之\textbf{指数有限度量矢量空间}。

 \begin{definition}{指数有限度量矢量空间}
 矢量空间 $V$ 配备一个不定型 $q$ 构成的二元组 $(V,q)$ 称之为\textbf{指数有限度量矢量空间}。
 \end{definition}
定义中的“指数”是指惯性指数(或正惯性指数)(\autoref{RQuaF_def1}~\upref{RQuaF})。

配备了二次型后的矢量空间 $V$,完全可对照欧几里得矢量空间建立一系列的概念。

设矢量空间 $V$ 定义在实数域 $\mathbb R$ 上,于是非退化二次型 $q$ 可取标准形式($n$ 为 $V$ 的维数)
\begin{equation}\label{EFSp_eq1}
q(x)=\sum_{i=1}^s x_i^2-\sum_{i=s+1}^n x_i^2~.
\end{equation}
事实上,每个二次型 $q$ 都有一配极的双线性型 $f$(\autoref{QuaFor_def2}~\upref{QuaFor}),$f$ 是个对称的。二次型非退化意味着对称双线性 $f$ 非退化,而 $f$ 对称性保证了它可对角化(\autoref{GuaOQu_the1}~\upref{GuaOQu}), 进而可选择基底使得 $q$ 标准化为\autoref{EFSp_eq1} (\autoref{RQuaF_the1}~\upref{RQuaF})。

于是,$V$ 上的\textbf{内积}为:
\begin{equation}
(x|y)=\sum_{i=1}^{s}x_iy_i-\sum_{i=s+1}^{n}x_{i}y_i~.
\end{equation}

此时,为了停留在处理实数的领域,只讨论矢量 $x$ 的模的平方 $\norm{x}^2=(x|x)$,在 $1\leq s\leq n-1$ 时,它可正可负。若 $\norm{x}^2=0$,则说矢量 $x$ 是\textbf{迷向的}。

欧几里得矢量空间中的二次型建立了其上的度量,同样的,这里仍将二次型 $(x|x)$ 称为 $V$ 上的\textbf{度量型}。
\begin{definition}{指数有限度量空间}
 仿射空间 $\mathbb E$ 配备一个指数有限度量矢量空间 $V$ 构成的三元组 $(\mathbb E,V,\rho)$ 称为\textbf{指数有限度量空间}。其中 $\rho$ 是其上的距离函数。
 \end{definition}
 和欧几里得空间\upref{EucSp}完全类似,可建立一系列的概念。同样也把 $\rho^2(\dot p,\dot q)$ 称为仿射空间$\mathbb E$ 上的\textbf{度量型},并且为了停留在实数域,只讨论 $\rho^2(\dot p,\dot q)$。

\begin{definition}{伪欧几里得空间、闵可夫斯基空间}\label{EFSp_def1}
设 $\mathbb E$ 是一个指数有限度量空间,$s$ 是其正惯性指数。当 $1\leq s\leq n-1$ 时,称 $\mathbb E$ 是\textbf{伪欧几里得空间}。当 $s=1$ 时,则称它是\textbf{闵可夫斯基空间}(当 $s=n-1$ 时,令 $q=-q$ ,则也是一个闵可夫斯基空间,所以可把 $s=1$ 和 $s=n-1$ 当作同一情形对待)。
\end{definition}
\subsection{伪欧几里得运动}
在欧几里得空间中,运动是使得点点距离不变的变换。即若 $f$ 是运动,则 $\rho(f(\dot p),f(\dot q))=\rho(\dot p,\dot q)$ 。然而在伪欧几里得空间中,由于只考虑 $\rho^2$ ,所以伪欧几里得空间中的运动是使得
\begin{equation}
\rho^2(f(\dot p),f(\dot q))=\rho^2(\dot p,\dot q)
\end{equation}
成立的变换 $f$。
\begin{definition}{伪欧几里得空间运动}
设 $(\mathbb E,V,\rho)$ 是伪欧几里得点空间,变换 $f:\mathbb E\rightarrow\mathbb E$ 称为其上的\textbf{运动}(或\textbf{保内积映射}),如果对 $\forall \dot p,\dot q\in\mathbb E$ ,都有
\begin{equation}
\rho^2(f(\dot p),f(\dot q))=\rho^2(\dot p,\dot q)~.
\end{equation}
\end{definition}
在伪欧几里得空间,运动也有和欧氏空间相似的结论成立,即:运动和一仿射映射等价,该仿射映射线性部分对应矩阵 $\mat F$ 满足 $F^TI_sF=I_s$ ( $I_s$ 是伪欧空间中标准二次型 $q$ 对应的双线性型对应的矩阵);任一运动可由平移和保持某个固定点的仿射映射(其线性部分对应矩阵满足 $F^TI_sF=I_s$ )复合而成。具体来说,下面定理成立:
\begin{theorem}{}

\end{theorem}