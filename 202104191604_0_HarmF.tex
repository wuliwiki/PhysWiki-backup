% 调和场(无散无旋场)
% 散度|旋度|调和场|拉普拉斯

\begin{issues}
\issueDraft
\end{issues}

\pentry{拉普拉斯方程与调和函数\upref{LapEq}}

我们把散度和旋度都为零的场称为\textbf{无散无旋场}或\textbf{调和场}. 注意后者并不是一个很常用的数学名词, 笔者只在个别中文教材中见过. 如果这只在一个空间的一定区域内成立, 那么就说它在这个区域内是调和场.

由于调和场旋度为零, 积分与路径无关, 必定可以定义势函数 $u(\bvec r)$ (包含一个任意常数项), 而调和场就是其梯度
\begin{equation}\label{HarmF_eq1}
\bvec f(\bvec r) = \div u
\end{equation}
要保证散度 $\div \bvec f$ 为零, \autoref{HarmF_eq1} 就要求 $u$ 是一个调和函数:
\begin{equation}
\laplacian u = 0
\end{equation}
所以调和场的充分必要条件是它可以表示为一个调和函数的梯度, 当且仅当给调和函数加一个任意常数时, $\bvec f(\bvec r)$ 不会改变.

\begin{theorem}{}
调和场的各个分量都是调和函数.
\end{theorem}
证明: 根据定义, 由于偏微分的顺序可以任意改变, 易证. 证毕.

与调和函数类似, 调和场的一个显著特点是其在无穷远处不为零, 由于无散无旋, 它的 “场线” (例如电场线)没有起点也没有终点, 而是从无穷远来, 到无穷远去. 我们可以把它想象为某种不可压缩流体的速度场或流密度\upref{CrnDen}场.

\begin{theorem}{最大值定理}\label{HarmF_the1}
$\mathbb R^N$ 的有限区域内的调和场的模长最大值必定出现在该区域的边界处.
\end{theorem}
\addTODO{证明: 可以计算梯度模方的 laplacian 然后使用最大值原理. 计算 laplacian 可以看出来是次调和函数, 次调和函数都有最大值原理.}

\begin{corollary}{}
如果 $\mathbb R^N$ 上的调和场 $\bvec f(\bvec r)$ 满足 $\lim_{\abs{\bvec r}\to \infty} \abs{\bvec f(\bvec r)}  = 0$, 那么 $\bvec f(\bvec r) \equiv \bvec 0$.
\end{corollary}
证明: 可以先选择一个半径为 $r$ 的圆/球作为\autoref{HarmF_the1} 的区域, 然后令半径区域无穷即可.

\addTODO{如果 $\mathbb R^N$ 上的调和场是有界的, 能否得出它是一个常矢量?}
