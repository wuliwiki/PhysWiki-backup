% 艾兹赫尔·戴克斯特拉(综述)
% license CCBYSA3
% type Wiki

本文根据 CC-BY-SA 协议转载翻译自维基百科\href{https://en.wikipedia.org/wiki/Edsger_W._Dijkstra}{相关文章}。

\begin{figure}[ht]
\centering
\includegraphics[width=6cm]{./figures/b98db333d3a6a579.png}
\caption{2002年的Dijkstra。} \label{fig_ACHR_1}
\end{figure}
Edsger Wybe Dijkstra(/ˈdaɪkstrə/ DYKE-strə;荷兰语:[ˈɛtsxər ˈʋibə ˈdɛikstraː] ⓘ;1930年5月11日-2002年8月6日)是荷兰计算机科学家、程序员、软件工程师、数学家和科学散文家。

Dijkstra出生于荷兰鹿特丹,他学习了数学和物理学,之后在莱顿大学攻读理论物理学。阿德里安·范·温格尔丹(Adriaan van Wijngaarden)曾邀请他担任荷兰第一位计算机程序员,在阿姆斯特丹的数学中心工作,任职时间为1952年到1962年。他在1956年提出并解决了最短路径问题,并于1960年与同事贾普·A·宗内费尔德(Jaap A. Zonneveld)共同开发了编程语言ALGOL 60的首个编译器。1962年,他搬到埃因霍温,随后又移居到纽嫩,在埃因霍温技术大学数学系担任教授。20世纪60年代末,他构建了THE多任务操作系统,该系统通过使用基于软件的分页虚拟内存,影响了后续操作系统的设计。1973年8月,Dijkstra加入了Burroughs公司,成为唯一的研究员。在Burroughs的这段时间里,他的研究文章产出最为丰盛,撰写了近500篇《EWD》系列文件,其中大部分为技术报告,供特定小组私下传阅。

1984年,Dijkstra接受了德克萨斯大学奥斯汀分校计算机科学系的施伦伯杰百年讲席,并在奥斯汀工作,直到1999年11月退休。退休后,他和妻子返回原住地纽嫩,直至2002年8月6日因长期与癌症斗争去世。

Dijkstra因其在开发结构化编程语言方面的基础性贡献而获得了1972年的图灵奖。临终前,他因在程序计算自稳定性方面的研究获得了ACM PODC影响力论文奖,这项奖项在次年以“Dijkstra奖”命名,以纪念他。
\subsection{生活与工作} 
\subsubsection{早年经历}  
Edsger W. Dijkstra 出生于鹿特丹。他的父亲是一位化学家,曾担任荷兰化学学会会长,并在一所中学教授化学,后来成为该校的校长。母亲是一位数学家,但没有正式的职业。  

Dijkstra 曾考虑过从事法律职业,并希望能代表荷兰出任联合国代表。然而,在1948年中学毕业后,在父母的建议下,他选择了在莱顿大学学习数学和物理,之后继续攻读理论物理学。  

在1950年代初期,电子计算机还是一种新鲜事物。Dijkstra 偶然进入了计算机领域,并通过他的导师Johannes Haantjes教授,遇到了Adriaan van Wijngaarden(阿姆斯特丹数学中心计算部门主任),后者给了他一份工作。1952年3月,他正式成为荷兰的第一位“程序员”。  

Dijkstra 曾一段时间致力于物理学,他在莱顿每周有三天时间从事物理工作。然而,随着对计算机的接触增加,他的兴趣开始发生转变。他回忆道:  

“在编程了大约三年后,我与当时是我的老板的A. van Wijngaarden进行了讨论,这个讨论我将永远感激他。问题是,我当时应该在莱顿大学同时攻读理论物理学,而我发现这两项活动越来越难以兼顾,于是我必须做出选择,要么放弃编程,成为一个真正的、受人尊敬的理论物理学家,要么以最低的努力完成物理学学习,只做个名义上的毕业生,去做……做什么呢?做程序员?但那是一个体面的职业吗?毕竟,什么是编程呢?有没有一个坚实的知识体系,能够支持它成为一门 intellectually respectable(受人尊敬的学科)?我记得很清楚,当我面对那个问题时,我嫉妒我的硬件同事们,因为当他们被问及他们的专业能力时,至少可以指出他们了解真空管、放大器等硬件,而我觉得,当我面对这个问题时,我会空手而归。满怀疑虑,我敲响了Van Wijngaarden的办公室门,问他是否能‘和我谈一谈’,当我几小时后离开时,我已经是另一个人。因为在耐心倾听了我的问题后,他同意到目前为止编程的学科确实不多,但他接着平静地解释说,自动计算机是不可避免的,它们将会存在下去,而我们现在只是在起步阶段,难道我不能成为那些人之一,帮助将编程发展为未来几年里受人尊敬的学科吗?这一刻是我生命中的转折点,我迅速完成了我的物理学学习。”

—— Edsger Dijkstra,《谦逊的程序员》(EWD340),《ACM通讯》  

Dijkstra 于1957年与Maria "Ria" C. Debets结婚,婚礼中他需要声明自己的职业。当时他声明自己是程序员,这在当时荷兰是无法接受的,因为荷兰当时并没有这个职业。  

1959年,他获得了阿姆斯特丹大学的博士学位,论文题目为《与自动计算机的通信》,该论文详细描述了为荷兰第一台商用计算机——Electrologica X1 设计的汇编语言。他的论文导师是Van Wijngaarden。
\subsubsection{阿姆斯特丹数学中心(Mathematisch Centrum)}
从1952年到1962年,Dijkstra在阿姆斯特丹的数学中心工作,在那里他与Bram Jan Loopstra和Carel S. Scholten紧密合作,他们被聘用来建造一台计算机。他们的互动方式非常有纪律性:首先,他们会决定硬件与软件之间的接口,并通过编写编程手册来确定这些接口。然后,硬件设计人员必须忠实于他们的部分,而作为程序员的Dijkstra则为尚不存在的机器编写软件。从这段经历中,Dijkstra学到了两课:清晰的文档是至关重要的,并且通过精心设计,程序调试可以大大避免。

Dijkstra在1956年为ARMAC计算机的正式启用展示中,提出并解决了最短路径问题。由于当时没有专门用于自动计算的期刊,他直到1959年才发布这一成果。

在数学中心,Dijkstra和他的同事Jaap Zonneveld在1960年8月开发了第一个用于编程语言ALGOL 60的编译器,比其他团队生产的编译器早了超过一年。ALGOL 60被认为是结构化编程发展的一个重要进展。
\subsubsection{埃因霍温理工大学}
\begin{figure}[ht]
\centering
\includegraphics[width=8cm]{./figures/a5f21dff870daf56.png}
\caption{埃因霍温理工大学位于荷兰南部的埃因霍温市,迪杰斯特拉曾在该校数学系担任教授,任期从1962年到1984年。} \label{fig_ACHR_2}
\end{figure}
1962年,Dijkstra搬到了埃因霍温,后来又搬到了荷兰南部的纽嫩,在那里他成为了埃因霍温理工大学数学系的教授。该大学没有单独的计算机科学系,数学系的文化也不特别适合他。Dijkstra尝试建立一个计算机科学家的团队,以便合作解决问题。这对于数学系来说是一种不寻常的研究模式。

在1960年代末,他开发了THE操作系统(该系统以当时称为“Technische Hogeschool Eindhoven”的大学命名),该系统通过使用基于软件的分页虚拟内存,影响了后续操作系统的设计。
\subsubsection{伯罗克公司}  
迪杰斯特拉于1973年8月加入伯罗克公司——当时以生产基于创新硬件架构的计算机而闻名,担任其研究员。他的职责包括每年几次访问公司的研究中心,并进行自己的研究,他的研究地点是在位于努恩的家中的二楼书房。实际上,迪杰斯特拉是伯罗克公司唯一的研究员,并且主要在家工作,偶尔会前往公司在美国的分支机构。因此,他将大学的职务减至每周一天。那一天,周二,很快就成为著名的“周二下午俱乐部”会议日,在这个研讨会上,他与同事们讨论科学文章,审视所有方面:符号、组织、呈现、语言、内容等。1984年他搬到美国德克萨斯大学奥斯汀分校后,位于奥斯汀的“周二下午俱乐部”分会也随之成立。

在伯罗克公司的岁月中,他的研究成果最为丰硕。他在EWD系列中写了近500篇文献,其中大部分是技术报告,这些报告主要在一个特定小组内进行私人流通。
\subsubsection{德克萨斯大学奥斯汀分校}
迪杰斯特拉于1984年接受了德克萨斯大学奥斯汀分校计算机科学系的施伦伯杰百年讲座教授职位。
\subsubsection{最后岁月}  
Dijkstra 在奥斯汀工作直到 1999 年 11 月退休。为纪念这一时刻并庆祝他四十多年对计算机科学的开创性贡献,计算机科学系于 2000 年 5 月 他的 70 岁生日那天组织了一场研讨会。[6]  

Dijkstra 和他的妻子从奥斯汀返回荷兰的原居所,得知他只剩下几个月的生命。他曾表示希望在奥斯汀退休,但希望在荷兰去世。Dijkstra 于 2002 年 8 月 6 日因长期与癌症斗争去世。[3][12][13][14] 他和妻子留下了三个孩子:Marcus、Femke 和计算机科学家 Rutger M. Dijkstra。[15]
\begin{figure}[ht]
\centering
\includegraphics[width=8cm]{./figures/115f17324448499e.png}
\caption{德克萨斯大学奥斯汀分校,Dijkstra 从 1984 年到 1999 年在该校计算机科学系担任施伦贝谢百年讲席教授。} \label{fig_ACHR_3}
\end{figure}