% 线性引力
% keys 线性近似|线性爱因斯坦方程|弱引力场

\begin{issues}
\issueMissDepend
\issueDraft
\end{issues}



\subsection{线性引力理论}
爱因斯坦场方程\footnote{在广义相对论中,常常采用几何单位制\upref{NatUni},也即是$c=G=k_B=1$}如下

\begin{equation}\label{LinGra_eq2}
G_{\mu \nu} = R_{\mu \nu} - \frac{1}{2}g_{\mu\nu}R = 8\pi T_{\mu\nu}
\end{equation}

由于方程采用几何语言描述,十分简洁,但它包含着一系列复杂的非线性微分方程.一方面,寻求严格满足爱因斯坦场方程的解是一个漫长而艰难的过程,许多数学天才也投入其中.另一方面,我们试图简化爱因斯坦场方程,因为在大多数情况中引力场都很微弱,所以我们可以采用近似处理的方式使爱因斯坦场方程线性化,简而言之,就是要对时空进行一阶线性微扰.


\subsection{广义相对论中的时空微扰}

对已知的某一背景时空进行微扰的核心观点非常简单,假设我们可以用下列展开来近似描述真实的物理时空:

\begin{equation}
g_{\mu\nu}=g^{(0)}_{\mu\nu} + \epsilon g^{(1)}_{\mu\nu} + \frac{1}{2!}\epsilon^2 g^{(2)}_{\mu\nu}+\cdots
\end{equation}

其中,$\epsilon $仅作为计算过程中判断各小量阶数的指标,必要的时候可以令 $\epsilon = 1 $,而$g^{(0)}_{\mu\nu} $ 是已知的背景时空,一阶项$g^{(1)}_{\mu\nu}$是线性微扰项.在本节中我们只讨论到一阶线性微扰部分.


\subsection{闵氏时空}

我们先考虑背景时空为平直的闵氏时空,线性微扰项记为$h_{\mu\nu}$. 在微扰的情景下,我们约定使用背景时空的度规进行指标升降,因此我们有

\begin{equation}
g_{\mu\nu} = \eta_{\mu\nu} + h_{\mu\nu}, \quad \abs{h_{\mu\nu}}<<1 
\end{equation}

以及

\begin{equation}\label{LinGra_eq1}
g^{\mu\nu} = \eta^{\mu\nu} - h^{\mu\nu} , \quad \abs{h^{\mu\nu}}<<1 
\end{equation}



例如,对于太阳系来说,$\abs{h_{\mu\nu}} \sim 10^{-6}$. 而\autoref{LinGra_eq1} 来自于度规定义需要满足$gg^{-1}=\delta$:

\begin{equation}
g_{\mu\nu}g^{\mu\rho} = (\eta_{\mu\nu} + h_{\mu\nu})(\eta^{\mu\rho} - h^{\mu\rho}) = \delta^{\rho}_{\nu} - \mathcal{O}(h^2)
\end{equation}

首先计算\textbf{克里斯托夫符号}(Christoffel Symbol,\autoref{CrstfS_def1}~\upref{CrstfS}),并保留到一阶项\footnote{以下计算同样均保留到一阶项.}

\begin{equation}
\begin{aligned}
\Gamma^{\mu}{}_{\alpha\beta}&=\frac{1}{2}g^{\mu\nu}(\partial_\alpha g_{\nu\beta} + \partial_\beta g_{\alpha\nu} - \partial_\nu g_{\alpha\beta})\\
&=\frac{1}{2}(\eta^{\mu\nu} - h^{\mu\nu})(\partial_\alpha h_{\nu\beta} + \partial_\beta h_{\alpha\nu} - \partial_\nu h_{\alpha\beta})\\
&=\frac{1}{2}\eta^{\mu\nu} (\partial_\alpha h_{\nu\beta} + \partial_\beta h_{\alpha\nu} - \partial_\nu h_{\alpha\beta})
\end{aligned}
\end{equation}



再计算黎曼张量
\begin{equation}
R_{\alpha \mu \beta \nu}=\frac{1}{2}\left(h_{\alpha \nu, \mu \beta}+h_{\mu \beta, \nu \alpha}-h_{\mu \nu, \alpha \beta}-h_{\alpha \beta, \mu \nu}\right)
\end{equation}


以及里奇张量,
\begin{equation}
\begin{aligned}
R_{\mu\nu} &= \Gamma^{\alpha}{}_{\mu\nu,\alpha} - \Gamma^{\alpha}_{\mu\alpha,\nu}\\
&=\frac{1}{2}\left(h^{\alpha}{}_{\mu, \nu \alpha} + h_{\nu}{ }^{\alpha}{ }_{, \mu \alpha}-h_{\mu \nu, \alpha}{ }^{\alpha}-h_{, \mu \nu}\right)
\end{aligned}
\end{equation}


其中,$h\equiv h^{\mu}{}_{\mu}=\eta^{\mu\nu}h_{\mu\nu}$.

进一步计算里奇标量$R = g^{\mu\nu}R_{\mu\nu}\approx \eta^{\mu\nu}R_{\mu\nu}$,将带其入爱因斯坦场方程\autoref{LinGra_eq2},便可得到线性爱因斯坦场方程.

\begin{equation}
\begin{aligned}
h_{\mu \alpha, \nu}{ }^{\alpha} &+h_{\nu \alpha, \mu}{ }^{\alpha}-h_{\mu \nu, \alpha}{ }^{\alpha}-h_{, \mu \nu} -\eta_{\mu \nu}\left(h_{\alpha \beta}{ }^{, \alpha \beta}-h_{, \beta}{ }^{\beta}\right)= 16 \pi T_{\mu \nu} .
\end{aligned}
\end{equation}

我们可以定义如下变量

\begin{equation}
\bar{h}_{\mu\nu} \equiv h_{\mu\nu} - \frac{1}{2}\eta_{\mu\nu}h.
\end{equation}

来化简,并且用横线代表对其他对称张量采取同样的操作,例如

\begin{align}
G_{\mu\nu} &= \bar{R}_{\mu\nu} = R_{\mu\nu}- \frac{1}{2}\eta_{\mu\nu}R\\
\bar{\bar{h}}_{\mu\nu} &= \bar{h}_{\mu\nu} - \frac{1}{2}\eta_{\mu\nu}\bar{h} = \bar{h}_{\mu\nu} - \frac{1}{2}\eta_{\mu\nu} (h-\frac{1}{2}\eta^{\alpha\beta}\eta_{\alpha\beta}h)= h_{\mu\nu}\\
h_{\mu\nu} &= \bar{h}_{\mu\nu} - \frac{1}{2}\eta_{\mu\nu} \bar{h}
\end{align}

因此可以得到\textbf{线性引力场方程}为
\begin{equation}
-\box{\bar{h}_{\mu \nu}}-\eta_{\mu \nu} \bar{h}_{\alpha \beta,}{ }^{\alpha \beta} + \bar{h}^{\alpha}{ }_{\mu,\alpha\nu} + \bar{h}^{\alpha}{ }_{\nu,\alpha\mu}=16 \pi T_{\mu \nu}
\end{equation}




我们可以应用相应的\textbf{规范条件(gauge conditions)},这里采用洛伦兹规范(Lorentz),

\begin{equation}
\bar{h}
\end{equation}

于是场方程最终就变成了

\begin{equation}
\bar{h}
\end{equation}

这边是洛伦兹规范下的线性引力的基本方程.


\subsection{史瓦西时空}

我们可以将史瓦西时空看作对于平直闵氏时空的微扰.


\subsection{规范不变性}


\subsection{推广到一般时空}

背景时空的选择其实是任意的,我们同样可以对其进行线性微扰.

