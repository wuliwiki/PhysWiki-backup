% 数列的极限(简明微积分)
% 微积分|极限|数列极限|函数极限|无穷小

微积分的核心概念是极限,而极限最基础的情形是数列的极限.数列是离散的,比较容易理解,而所有与极限有关的概念也都可以从数列的极限拓展得到. 先来看一个数列的例子.  注意只有无穷数列具有极限.

\begin{example}{}\label{Lim0_ex1}
我们知道 $\pi$ 是一个无理数,所以 $\pi$ 的小数部分是无限多的.目前用计算机,已经可以将 $\pi$ 精确地计算到小数点后数亿位.然而在实际应用中,往往只用取前几位小数的近似即可.下面给出一个数列,定义第 $n$ 项是 $\pi$ 的前 $n$ 位小数近似(不考虑四舍五入),即
\begin{equation}\label{Lim0_eq1}
a_0 = 3,\,\, a_1 = 3.1,\,\, a_2 = 3.14,\,\, a_3 = 3.141,\,\dots.
\end{equation}

这个数列显而易见的性质,就是当 $n$ \textbf{趋于无穷}时,$a_n$ \textbf{趋于} $\pi$. \textbf{无穷(infinity)}用符号 $\infty$ 来表示. 我们说该数列的\textbf{极限值}是 $\pi$, 也可以简称\textbf{极限(limit)}. 以上这种情况,用极限符号表示,就是
\begin{equation}
\lim_{n \to \infty } {a_n} = \pi 
\end{equation}
这里 $\lim$ 是极限的数学符号,下方用箭头表示某个量 “无止境” 变化的过程. 对于数列而言, 唯一的 “无止境” 变化就是项标 $n$ 不断增加, 所以数列只有一种极限.
\end{example}

$\lim\limits_{n \to \infty }$ 在这里相当于一个“操作”,叫\textbf{算符(operator)}. 它作用在数列上,并输出一个数, 也就是数列的极限值. 这有些类似于函数输入一个自变量并输出一个因变量, 只不过自变量从数换成了数列. 所以不要误以为这条式子是说当 $n = \infty$ 时,$a_n=\pi$ \footnote{有两个理由可以说明这种理解不正确:首先,按定义,每个$a_n$都是有理数,而$\pi$是无理数,所以不应该有任何一个$a_n=\pi$;其次,$\infty$不是一个实数,不存在$n=\infty$的说法.这里的 $n\to\infty$ 只是表示$n$的增大是没有限制的.},而要理解成数列 $a_n$ 经过算符 $\lim\limits_{n \to \infty }$ 的作用以后,得出的结果是 $\pi$. 类比函数 $\sin x = y$,并不是说 $x=y$, 而是说 $x$ 经过正弦函数作用后等于 $y$. 所以从概念上来说,极限中的 “趋于” 和“等于” 是不同的.趋于是数列整体的性质,而不是某一个项性质.

我们可以总结出以上数列的一个性质, 并把它作为数列极限的一般定义. 具有极限的数列最显著的特征是, 随着 $n$ 增加, 后面的所有项都越来越接近极限值. 可是难点在于如何定义 “越来越接近”. 先看一个错误的理解: 考虑数列
\begin{equation}\label{Lim0_eq2}
a_1 = 3.21,\ a_2 = 3.201,\ a_3 = 3.2001,\ a_4 = 3.20001, \dots
\end{equation}
这个数列也同样越来越接近 $\pi$, 但直觉告诉我们, 它的极限是 $3.2$ 而不是 $\pi$.

既然要讨论有多接近, 那就要定义距离. 我们可以把第 $a_n$ 和极限值 $\pi$ 的\textbf{距离}用绝对值定义为 $\abs{a_n - \pi}$. 对\autoref{Lim0_eq1} 的数列, 可以发现无论我们要求这个距离有多小, 当 $n$ 大到一定程度的时候就总能满足这个要求. 例如要求 $\abs{a_n - \pi} < 10^{-2}$, 容易发现 $n > 1$ 时就总能满足; 又例如要求 $\abs{a_n - \pi} < 10^{-10}$, 容易发现 $n > 9$ 时就总能满足. 就可以保证第 $N$ 项后面所有的项都满足要求. 一般地如果要求 $\abs{a_n - \pi} < 1\e{-q}$ ($q$ 为整数), 只要 $n > q-1$ 就总能满足. 这就意味着 $\lim\limits_{n \to \infty } a_n = \pi$.

有了这个定义, 我们就可以轻易地判断\autoref{Lim0_eq2} 的极限是 $3.2$ 而不是 $\pi$. 因为无论 $n$ 为多大, 总是有 $\abs{a_n - \pi} > 3.2 - \pi$. 所以如果要求一个比这更小的距离, 那么就没有任何 $n$ 可以满足, 更不用说给出条件 $n > N$ 了.

\begin{definition}{数列的极限}\label{Lim0_def2}
考虑实数数列 $a_1, a_2, \dots, a_n, \dots$, 若无论要求项 $a_n$ 和确定实数 $A$ 的距离 $\abs{a_n - A}$ 有多小, 都能给出一个相应的条件 $n > N$ 来满足, 那么该数列的极限就是 $A$.
\end{definition}

我们来看几个简单的例题,加深一下印象.

\begin{example}{}
考虑数列 $a_n= {(-1)^n}/{2^n}$.根据定义可证明 $\lim\limits_{n\to\infty}a_n=0$.
\end{example}

一些数列不存在极限:
\begin{example}{}\label{Lim0_exe1}
考虑数列 $a_n = n$ 以及 $a_n=(-1)^n$. 它们存在极限吗?
\end{example}

\begin{definition}{数列的敛散性}\label{Lim0_def4}
如果一个数列不存在极限, 就称它是\textbf{发散(divergent)}的. 如果存在极限, 则称它是\textbf{收敛(convergent)}的.
\end{definition}

\begin{example}{}
容易知道, 数列的极限和前面有限项的数值都无关, 例如把\autoref{Lim0_eq1} 中的前三项改成 $0$, 那么该数列的极限仍然是 $\pi$.
\end{example}

\begin{example}{}
当然, 数列极限也并不要求 $n\to \infty$ 时数列的项不能等于极限值, 例如数列
\begin{equation}
b_0 = 3.3,\,\, b_1 = 3.2, \,\, b_2 = \pi, \,\, b_3 = \pi, \,\, b_n = \pi \;\; (n \ge 2)
\end{equation}
当 $n \ge 2$ 时所有的项都等于 $\pi$, 那么根据\autoref{Lim0_def2} 他的极限显然也是 $\pi$, 尽管这样的极限十分无趣.
\end{example}
