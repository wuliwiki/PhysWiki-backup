% 圆周运动:为什么月球和卫星不会掉下来(科普)
% license Usr
% type Tutor

\subsection{牛顿大炮}
牛顿有一个很形象的思想实验: 若不考虑大气阻力,大炮在水平方向发射的炮弹开始时近似做抛物线,但当炮弹的初速度足够大时,它将绕地球做圆周运动永远不落到地面。

那么使用所谓的微元法\footnote{高中经常把微积分中不断细分的思想称为 “微元法”,但要严格地证明这样的方法是对的,应该学习微积分。},我们是否可以认为当炮弹做圆周运动时,它在每一小段可以近似看作是在做平抛运动(水平发射的抛物线)呢?答案是可以的,且这些小段分割得越短,这个近似就越准确。

\begin{figure}[ht]
\centering
\includegraphics[width=6cm]{./figures/f5e66bd1fac10d50.png}
\caption{假设炮弹做平抛运动,当飞出一段距离后,稍微调整重力的方向,并以此时速度为初速度再次做平抛运动。} \label{fig_CMintr_1}
\end{figure}

\begin{figure}[ht]
\centering
\includegraphics[width=6cm]{./figures/3646d689f4b24526.png}
\caption{当平抛运动重复多次,我们就得到了近似的圆周运动。 每次抛物运动的时间越短结果就越精确。} \label{fig_CMintr_2}
\end{figure}

这就是为什么月亮和人造卫星不会掉下来。 从某种意义上他们每时每刻都在往下掉!但这要相对于直线运动而言——如果没有地球的引力,他们将做直线运动飞出地球。

通过这种方式,我们甚至可以推导出圆周运动的向心加速度,过程见最后。


\subsection{多边形运动推导圆周运动}
另一种用微元法解释圆周运动的方式是使用线段代替圆。 这也是古代人在计算圆周率时使用的 “割圆术”。

想象看似做圆周运动的质点实际上实在沿着一个正 $N$ 边形做匀速运动。显然,在每个线段中匀速运动时,根据牛顿第一定律,质点不需要受任何外力。 但在拐点处,我们可以想象它和


\subsection{由平抛运动推导向心加速度}
通过这种方式,我们甚至可以推导出圆周运动的向心加速度。且这个加速度就是平抛运动中物体向下的加速度。

每个抛物运动中,若时间为 $\Delta t$,则转过的角度为
\begin{equation}
\Delta\theta = \tan^{-1}\frac{g\Delta t}{v} \approx \frac{g\Delta t}{v}~.
\end{equation}
当 $\Delta\theta$ 变小时, 式中的近似会越来越精确。

每段抛物线的长度近似为
\begin{equation}
\Delta l \approx v \Delta t~.
\end{equation}
同样,当 $\Delta\theta$ 变小时, 近似会越来越精确。

注意 $\Delta\theta$ 不但是炮弹转过的角度,也是炮弹绕圆心旋转的角度。

另外考虑到圆的弧长公式 $\Delta l = R\Delta\theta$,那么
\begin{equation}
R = \frac{\Delta l}{\Delta\theta} \approx \frac{v^2}{g}~.
\end{equation}
也就是说
\begin{equation}
g \approx \frac{v^2}{R}~.
\end{equation}
而 $g$ 就是重力加速度,也就是圆周运动的向心加速度。这就是我们高中熟悉的圆周运动向心加速度的公式
\begin{equation}
a = \frac{v^2}{R}~.
\end{equation}