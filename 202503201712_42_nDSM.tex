% n 维球的度规
% keys n维球|高维球|度规
% license Usr
% type Tutor

\pentry{度规张量与指标升降\nref{nod_TofEuc}}{nod_2f31}
$n$ 维球是 $n+1$ 维空间中的超球面,其中的“超”字在数学上定义为 $n$ 维空间中的 $n-1$ 为曲面。因此,$n$ 维球作为三维空间球面的推广,代表着 $n$ 维空间中到某一(称为球心的)点距离恒定的所有点的全体。通常球心取为坐标原点。本节将推导球坐标下 $n$ 维球的\enref{度规}{TofEuc}。

\subsection{$n+1$ 维空间的球坐标}
\begin{definition}{笛卡尔坐标}
在 $n+1$ 空间中,若在坐标 $(x^1,\cdots,x^{n+1})$ 下,线元 $\dd s^2$ 可写为
\begin{equation}
\dd s^2=\dd x^i+\cdots+\dd x^{n+1},~
\end{equation}
 则称坐标 $(x^1,\cdots,x^{n+1})$ 为\textbf{笛卡尔坐标}。
\end{definition}

高维空间中的球坐标可以通过三维空间的球坐标推广得到。在三维空间中,笛卡尔坐标 $(x,y,z)$ 和球坐标 $(r,\theta,\varphi)$ 的关系具有这样的几何图像:$r$ 代表(由坐标原点指向对应点的矢量)对应点径矢的长度, $\theta$ 是径矢与 $z$ 轴的夹角,$\varphi$ 是径矢在 $x-y$ 平面的投影与 $x$ 轴的夹角。由此得到两坐标系统的转换关系
\begin{equation}
\begin{aligned}
&x=r\sin\theta\cos\varphi,\\
&y=r\sin\theta\sin\varphi,\\
&z=r\cos\theta.\\
\end{aligned}~
\end{equation}
推广到 $n+1$ 维空间中,则 $n+1$ 维球坐标 $(r,\theta^1,\cdots,\theta^{n})$ 和笛卡尔坐标 $(x^1,\cdots,x^{n+1})$ 具有这样的联系:$r$ 代表点径矢的


















