% 巴拿赫-塔斯基定理(综述)
% license CCBYNCSA3
% type Wiki

本文根据 CC-BY-SA 协议转载翻译自维基百科\href{https://en.wikipedia.org/wiki/Banach\%E2\%80\%93Tarski_paradox}{相关文章}。

\begin{figure}[ht]
\centering
\includegraphics[width=10cm]{./figures/a4f2dc0381172744.png}
\caption{} \label{fig_BTS_1}
\end{figure}
巴拿赫–塔尔斯基悖论是集合论几何中的一个定理,其内容如下:给定三维空间中的一个实心球体,存在一种将该球体分解为有限个不相交子集的方式,这些子集可以以不同的方式重新组合,从而得到两个与原球体完全相同的副本。实际上,重新组合过程仅涉及移动和旋转这些部分,而不改变它们的原始形状。然而,这些部分本身并不是传统意义上的“固体”,而是无穷多个点的散布。重构可以通过最少五个部分来实现。\(^\text{[1]}\)

定理的另一种形式表述为:给定任意两个“合理的”固体物体(例如一个小球和一个巨大球),这两个物体的切割部分可以重新组合成对方。这通常被非正式地表述为“一个豌豆可以被切割并重新组合成太阳”,并被称为\textbf{“豌豆和太阳悖论”}。

该定理是一个真实悖论:它与基本的几何直觉相矛盾,但并不是错误的或自相矛盾的。通过将球体分割成部分并通过旋转和平移来移动它们,而没有任何拉伸、弯曲或添加新点,“将球体翻倍”似乎是不可能的,因为所有这些操作从直觉上讲都应该保持体积不变。这样的操作保持体积的直觉并不是数学上荒谬的,甚至它也包含在体积的正式定义中。然而,这里不适用这种直觉,因为在这种情况下,无法定义所考虑子集的体积。重新组合它们会产生一个具有体积的集合,而这个体积恰好与开始时的体积不同。

与大多数几何定理不同,这个结果的数学证明在关键的方面依赖于集合论公理的选择。它可以通过选择公理来证明,该公理允许构造不可测量的集合,即没有普通意义上体积的点集,而其构造需要不可数次的选择。\(^\text{[2]}\)

2005年曾证明,分解中的各个部分可以以一种方式选择,使它们可以连续地移动到适当的位置,而不会相互碰撞。\(^\text{[3]}\)

正如Leroy\(^\text{[4]}\)和Simpson\(^\text{[5]}\)独立证明的那样,Banach–Tarski悖论在使用局部空间而不是拓扑空间时不会违反体积。在这种抽象的设置中,可能存在没有点但仍然非空的子空间。悖论分解的部分在局部空间的意义上相互交叠得很厉害,以至于这些交集中的一些应该被赋予正的质量。允许将这种隐藏的质量考虑在内,局部空间理论使得欧几里得空间的所有子集(甚至所有子局部空间)都能得到令人满意的度量。
\subsection{Banach和Tarski的论文} 
在 1924 年发表的一篇论文中,\(^\text{[6]}\)斯特凡·巴拿赫和阿尔弗雷德·塔尔斯基给出了一个这种悖论分解的构造,基于朱塞佩·维塔利关于单位区间的早期工作,以及费利克斯·豪斯多夫对球体的悖论分解,并讨论了与欧几里得空间中各种维度的子集分解相关的若干问题。他们证明了以下更一般的命题,即 Banach-Tarski 悖论的强形式:

给定任何两个欧几里得空间中至少三维的有界子集\( A \)和\( B \),且这两个子集都有非空的内部,则\( A \)和\( B \)可以被分解成有限个不相交的子集,\(A = A_1 \cup \cdots \cup A_k\),\(B = B_1 \cup \cdots \cup B_k \)(其中\(k\)为某个整数),使得对于每一个\( i \)(\( i \)在 1 到\( k \)之间的整数),集合\( A_i \)和\( B_i \)是全等的。  

现在让\( A \)为原始的球体,\( B \)为原始球体的两个平移副本的并集。则这一命题的意思是,原始的球体\( A \)可以被分割成若干个部分,然后通过旋转和平移,使得结果是包含两个\( A \)副本的整个集合\( B \)。

Banach-Tarski 悖论的强形式在一维和二维中是错误的,但巴拿赫和塔尔斯基表明,如果允许可数多个子集,类似的命题仍然成立。维度 1 和 2 与 3 及更高维度之间的差异,源于三维欧几里得运动群\( E(n) \)的更丰富结构。对于\( n = 1, 2 \),该群是可解的,但对于\( n \geq 3 \),它包含一个有两个生成元的自由群。约翰·冯·诺依曼研究了使得悖论分解成为可能的等价群的性质,并引入了可容许群的概念。他还在平面中发现了一个悖论的形式,使用保持面积的仿射变换代替通常的全等变换。

塔尔斯基证明了可容许群恰好是那些不存在悖论分解的群。由于在巴拿赫-塔尔斯基悖论中只需要自由子群,这导致了长期存在的冯·诺依曼猜想,该猜想在 1980 年被反驳。
\subsection{形式处理}  
巴拿赫-塔尔斯基悖论指出,普通欧几里得空间中的一个球体可以通过仅使用将其分割成子集、用全等的集合替换一个集合以及重新组合的操作来“翻倍”。其数学结构通过强调欧几里得运动群的作用以及引入等分解集合和悖论集合的概念得到了极大阐明。假设\( G \)是一个作用于集合\( X \)的群。在最重要的特例中,\( X \)是一个\( n \)-维欧几里得空间(对于整数\( n \)),而 \( G \)由所有\( X \)的等距变换组成,即保持距离的\( X \) 到自身的变换,通常表示为\( E(n) \)。两个几何图形如果可以相互转换,则称它们是全等的,这一术语将扩展到一般的\( G \)-作用。若集合\( A \)和\( B \)可以被分割成相同个数的、分别与\( G \)全等的部分,则称\( A \)和\( B \) 是 \( G \)-等分解的,或称与\( G \)相对于等分解。这个定义了\( X \)中所有子集之间的一个等价关系。正式地,如果存在非空集合\(A_1, \dots, A_k, \quad B_1, \dots, B_k\)使得
\[
A = \bigcup _{i=1}^{k}A_{i}, \quad B = \bigcup _{i=1}^{k}B_{i},~
\]
\[
A_i \cap A_j = B_i \cap B_j = \emptyset \quad \text{for all } 1 \leq i < j \leq k,~
\]
并且存在元素\( g_i \in G \)使得
\[
g_i(A_i) = B_i \quad \text{for all } 1 \leq i \leq k,~
\]
则可以说\( A \)和\( B \)是用\( k \)个部分\( G \)-等分解的。如果集合\( E \)有两个不相交的子集\( A \)和 \( B \),使得\( A \)和\( E \),以及\( B \)和\( E \)都是\( G \)-等分解的,则称\( E \)是悖论的。

使用这些术语,巴拿赫-塔尔斯基悖论可以重新表述为:

三维欧几里得球体与其两个副本是等分解的。  
事实上,由 Raphael M. Robinson 提出了一个明确的结果:\(^\text{[7]}\)将球体翻倍可以用五个部分完成,而少于五个部分将不足以完成。