% Egoroff定理
% keys 叶戈洛夫定理|实变函数|一致收敛

在讨论函数极限的相关问题时,我们常需要函数列具有\textbf{一致收敛}的性质.反过来,观察一个不一致收敛的函数列,比如$\{f_n(x)=x^n\}$在区间$[0, 1]$上就不一致收敛,我们会发现如果把区间挖掉长度$\epsilon$\textbf{任意}小的一部分,那么$\{f_n\}$在$[0, 1-\epsilon]$上总是一致收敛的.这提示我们研究,任意收敛函数列是否可以去掉一个小部分以后是一致收敛的?

答案是肯定的,这就是本节要讨论的Egoroff定理.

\subsection{Egoroff定理}

\begin{theorem}{Egoroff定理}



\end{theorem}











