% 反常霍尔效应
% keys 晶体|电子|霍尔效应
\pentry{电子运动的准经典模型\upref{cryele}}
\begin{issues}
\issueDraft
\end{issues}
一般霍尔效应的产生需要磁场,并且满带不出现霍尔效应。但是反常霍尔效应不需要这些条件。
\subsection{介绍}
我们知道一个布洛赫态可以写成:
\begin{equation}
\psi_{n,\bvec{k}}=e^{i\bvec{k}\cdot\bvec{r}}u_{n,\bvec{k}}~,
\end{equation}
其哈密顿量为:$\hat H_0=-\frac{\hbar^2}{2m}\nabla^2+V(\bvec{r})$,对应的能量是$E_{n,\bvec{k}}$。其中$V(\bvec{r})$是一个周期函数,有$V(\bvec{r}+\bvec{R})=V(\bvec{r})$,$\bvec{R}$是任意一个格矢。

外力作用下哈密顿量变成 $\hat H =\hat H_0 -\bvec{F}\cdot\bvec{r}$,则dt时间后,布洛赫态变成:
\begin{equation}
\begin{aligned}
\psi(\bvec{r},dt)&=e^{-\frac{i\,\hat H\,dt}{\hbar}}\psi_{n,\bvec{k}} \approx \psi_{n,\bvec{k}}-\frac{idt}{\hbar}(\hat H_0 -\bvec{F}\cdot\bvec{r})\psi_{n,\bvec{k}}\\
&=(1-i\frac{E_{n,\bvec{k}}dt}{\hbar}+i\frac{\bvec{F}\cdot\bvec{r}dt}{\hbar})e^{i\bvec{k}\cdot\bvec{r}}u_{n,\bvec{k}}
\approx e^{-\frac{i}{\hbar}E_{n,\bvec{k}}dt}e^{i(\bvec{k}+\frac{\bvec{F}dt}{\hbar})\cdot \bvec{r}}u_{n,\bvec{k}}~.
\end{aligned}
\end{equation}
在推导电子运动的准经典模型时,我们忽略了$u_{n,\bvec{k}}$到$u_{n,\bvec{k}+\frac{\bvec{F}dt}{\hbar}}$之间的变化(包络近似),从而得出了$d\bvec{k}=\frac{\bvec{F}dt}{\hbar}$的结论。现在我们不忽略它的变化,从而推导出反常霍尔效应来。

有:
\begin{equation}
\begin{aligned}
\psi(\bvec{r},dt)&\approx e^{-\frac{i}{\hbar}E_{n,\bvec{k}}dt}e^{i(\bvec{k}+\frac{\bvec{F}dt}{\hbar})\cdot \bvec{r}}u_{n,\bvec{k}}\\
&=e^{-\frac{i}{\hbar}E_{n,\bvec{k}}dt}e^{i(\bvec{k}+\frac{\bvec{F}dt}{\hbar})\cdot \bvec{r}}(u_{n,\bvec{k}}-u_{n,\bvec{k}+\frac{\bvec{F}}{\hbar}dt}+u_{n,\bvec{k}+\frac{\bvec{F}}{\hbar}dt})\\
&=e^{-\frac{i}{\hbar}E_{n,\bvec{k}}dt}e^{i(\bvec{k}+\frac{\bvec{F}dt}{\hbar})\cdot \bvec{r}}u_{n,\bvec{k}+\frac{\bvec{F}}{\hbar}dt}-e^{-\frac{i}{\hbar}E_{n,\bvec{k}}dt}e^{i(\bvec{k}+\frac{\bvec{F}dt}{\hbar})\cdot \bvec{r}}(u_{n,\bvec{k}+\frac{\bvec{F}}{\hbar}dt}-u_{n,\bvec{k}})\\
&=e^{-\frac{i}{\hbar}E_{n,\bvec{k}}dt}\psi_{n,\bvec{k}+\frac{\bvec{F}dt}{\hbar}}-e^{-\frac{i}{\hbar}E_{n,\bvec{k}}dt}e^{i(\bvec{k}+\frac{\bvec{F}dt}{\hbar})\cdot \bvec{r}}(\nabla_{\bvec{k}} u\cdot \frac{\bvec{F}}{\hbar}dt)\\
&\approx e^{-\frac{i}{\hbar}E_{n,\bvec{k}}dt}\psi_{n,\bvec{k}+\frac{\bvec{F}dt}{\hbar}}-e^{i\bvec{k}\cdot\bvec{r}}\nabla_{\bvec{k}} u\cdot \frac{\bvec{F}}{\hbar}dt~,
\end{aligned}
\end{equation}
其中$\nabla_{\bvec{k}} u$表示函数$u$在$\bvec{k}$空间的梯度。最后一条等式的前一项即为经典理论的结果,后一项的导出只保留了dt的一阶小量。
\subsection{深入}
现在我们来研究一下$\nabla_{\bvec{k}} u$。由上可知,有:
\begin{equation}
\hat H_0 e^{i\bvec{k}\cdot\bvec{r}}u_{n,\bvec{k}}(\bvec{r})
=E_{n,\bvec{k}}e^{i\bvec{k}\cdot\bvec{r}}u_{n,\bvec{k}}(\bvec{r})~,
\end{equation}
所以有:
\begin{equation}\label{eq_anHe_1}
e^{-i\bvec{k}\cdot\bvec{r}} \hat H_0 e^{i\bvec{k}\cdot\bvec{r}}u_{n,\bvec{k}}(\bvec{r})
=E_{n,\bvec{k}}u_{n,\bvec{k}}(\bvec{r})~,
\end{equation}
即 $\hat H_{\bvec k} =e^{-i\bvec{k}\cdot\bvec{r}} \hat H_0 e^{i\bvec{k}\cdot\bvec{r}}$的本征函数是$u_{n,\bvec{k}}(\bvec{r})$。代入$\hat H_0 =-\frac{\hbar^2}{2m}\nabla^2+V(\bvec{r})$,对于任意一个函数$f(\bvec{r})$,即有:
\begin{equation}
\begin{aligned}
\hat H_{\bvec k} f(\bvec{r})&=e^{-i\bvec{k}\cdot\bvec{r}}(-\frac{h^2}{2m}\nabla^2+V(\bvec{r}))e^{i\bvec{k}\cdot\bvec{r}}f(\bvec{r})\\
&=V(\bvec{r})f(\bvec{r})+e^{-i\bvec{k}\cdot\bvec{r}}(-\frac{\hbar^2}{2m})\nabla(i\bvec{k}e^{i\bvec{k}\cdot\bvec{r}}f(\bvec{r})+e^{i\bvec{k}\cdot\bvec{r}}\nabla f(\bvec{r}))\\
&=V(\bvec{r})f(\bvec{r})+e^{-i\bvec{k}\cdot\bvec{r}}(-\frac{\hbar^2}{2m})(-k^2e^{i\bvec{k}\cdot\bvec{r}}+2i\bvec{k}e^{i\bvec{k}\cdot\bvec{r}}\nabla f(\bvec{r})+e^{i\bvec{k}\cdot\bvec{r}}\nabla^2 f(\bvec{r}))\\
&=V(\bvec{r})f(\bvec{r})-\frac{\hbar^2}{2m}(-k^2+2i\bvec{k}+\nabla^2)f(\bvec{r})~,
\end{aligned}
\end{equation}
即 $\hat H_{\bvec k}$ 的具体形式为
\begin{equation}
\hat H_{\bvec k} =\frac{\hbar^2}{2m}(-i\nabla+\bvec{k})^2+V(\bvec{r})~.
\end{equation}
当$\bvec{k}$变化$\delta\bvec{k}$时,即有:
\begin{equation}
\hat H_{\bvec{k}+\delta\bvec{k}} =\frac{\hbar^2}{2m}(-i\nabla+\bvec{k})^2+V(\bvec{r})+\frac{\hbar^2}{2m}((\bvec{k}+\delta{\bvec{k}})^2-k^2)+\frac{\hbar^2}{m}\delta\bvec{k}\cdot(-i\nabla)~.
\end{equation}
等式右边第1、2项即为原先的 $\hat H_{\bvec k}$, 第3、4项可以视为微扰。其中第3项是能量微扰项,不作用在波函数上,第四项为$\bvec{k}\cdot\bvec{p}$微扰项,与动量相关(此处还说明了晶格动量$\hbar\bvec{k}$与电子动量$\bvec{p}$不是同一个东西)。
