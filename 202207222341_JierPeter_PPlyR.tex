% 本原多项式(线性代数)
% 本原多项式|有理系数多项式|整系数多项式|线性代数|高等代数|多项式|primitive polynomial

任何一个有理系数多项式乘以一个整数,总能得到一个整系数多项式,且二者的根完全一样;类似地,任何整系数多项式,如果其系数有公共整数因子,那也可以用这个因子去除该多项式,得到的还是整系数多项式,且根不变.

综上,研究有理系数多项式的根时,可以把焦点完全集中在下述“本原多项式”上.


\begin{definition}{本原多项式}\label{PPlyR_def1}

若整系数多项式$f(x)$的各系数之最大公因子是$1$,则称$f(x)$为一个\textbf{本原多项式(primitive polynomial)}.

\end{definition}

显然,每个有理系数多项式都唯一对应一个本原多项式.首项系数为$1$的多项式(简称为首一多项式)必为本原多项式.

下面给出有关本原多项式基本性质的两个引理:

\begin{lemma}{}\label{PPlyR_lem1}
若$f$和$g$都是本原多项式,则$fg$也是本原多项式.
\end{lemma}

\textbf{证明}:

设$f(x) = \sum_{i=0}^n a_ix^i$,$g(x) = \sum_{j=0}^m b_jx^j$.由\autoref{PPlyR_def1} ,任取正整数$k>1$,都存在$i_0, j_0$使得$k\not\mid a_{i_0}b_{j_0}$.

考虑
\begin{equation}\label{PPlyR_eq1}
\begin{aligned}
f(x)g(x) &= a_0b_0 + \sum_{i+j=1}a_ib_jx + \sum_{i+j=2}a_ib_jx^2+\cdots +a_nb_mx^{n+m}
\end{aligned}
\end{equation}

对于上面任取的正整数$k$,$k\mid a_0b_0$一共带来两种情况:$k$整除$a_0, b_0$中的一个但不整除另一个,或$k$同时整除$a_0, b_0$.我们看看这两个情况分别导致什么结果:

假设$k\mid a_0$,且$k\not\mid b_0$.则当$k\mid a_1b_0+a_0b_1$时,必有$k\mid a_1$.于是,如果$k\mid\sum_{i+j=d}a_ib_j$对于$d<i_0+j_0$都成立,那么可推出$k\mid a_d$对于$d<i_0+j_0$都成立.

假设$k\mid a_0$且$k\mid b_0$.则当$k\mid a_2b_0+a_1b_1+a_0b_2$时,必有$k\mid a_1b_1$.于是问题又回到原点.

这么一来,如果各$k\mid\sum_{i+j=d}a_ib_j$对于$d<i_0+j_0$都成立,则要么所有满足$i<i_0+j_0-1$的$a_i$都能被$k$整除,要么要么所有满足$j<i_0+j_0-1$的$b_j$都能被$k$整除.

现在考虑\autoref{PPlyR_eq1} 右边的第$i_0+j_0$项系数.由上一段的结论,该系数求和式中,除去$a_{i_0}b_{j_0}$以外的所有$a_ib_j$都能被$k$整除.由于已知$k\mid a_{i_0}b_{j_0}$,故知$k$不整除这一项.

由$k$的任意性,知\autoref{PPlyR_eq1} 各项系数的最大公因子是$1$.于是$fg$是本原多项式.



\textbf{证毕}.



\begin{lemma}{}
设$f(x)=g(x)h(x)$,$g$是本原多项式.

如果$f$是整系数多项式,那么$h$是整系数多项式.

如果$f$是本原多项式,那么$h$是本原多项式.
\end{lemma}

\textbf{证明}:

设$h(x)$的本原多项式是$kh(x)$,则由\autoref{PPlyR_lem1} ,$kf(x)$是本原多项式.

先设$f$是整系数多项式,$g$是本原多项式.



\textbf{证毕}.












