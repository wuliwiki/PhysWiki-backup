% 拉莫尔进动
% 自旋1|磁场中的电子2
\pentry{自旋角动量\upref{Spin}}
\textbf{拉莫尔(Larmor)进动}是围绕其外部磁场的磁矩所做的进动。
\begin{equation}
\bvec \tau = \bvec\mu\cross\bvec B=\gamma\bvec J\cross \bvec B
\end{equation}
其中 $\gamma$ 为磁旋比 \upref{BohMag},$\bvec\tau$ 为力矩,$\bvec \mu$ 为磁偶极矩,$\bvec J$ 是角动量矢量。

\subsection{自旋粒子的拉莫尔进动}

假定有一个自旋为 $1/2$ 的粒子静止处于均匀磁场:
\begin{equation}
\bvec B =B_0\hat{z}~.
\end{equation}
静止在磁场中的带点自旋粒子的哈密顿为:
\begin{equation}
H = -\gamma\bvec B\cdot S
\end{equation}
由此可得:
\begin{equation}
\bvec H = -\gamma B_0S_z=-\frac{\gamma B_0\hbar}{2}\pmat{1&0\\0&-1}
\end{equation}
注意到 $\bvec H$ 有着和 $S_z$ 相同的本征矢:
\begin{equation}
\chi_+ = \pmat{1\\0}:E_+=-(\gamma B_0\hbar)/2; \ \chi_- = \pmat{0\\1}:E_-=+(\gamma B_0\hbar)/2; 
\end{equation}
显然和经典情况一致,在偶极矩平行于磁场时能量时最低的。

由于哈密顿量和时间无关,因此含时薛定谔方程为:
\begin{equation}
\I \hbar \pdv{\chi}{t}=\bvec H\chi
\end{equation}
它的一般解可以被表示为定态的线性组合:
\begin{equation}\label{eq_Larmor_1}
\chi(t)=a\chi_+\E^{-\I E_+t/\hbar}+b\chi_-\E^{-\I E_-t/\hbar}=\pmat{a\E^{\I \gamma B_0t/2}\\b\E^{-\I \gamma B_0t/2}}
\end{equation}
其中的常数 $a,b$ 是由初始条件所确定的:
\begin{equation}
\chi(0)=\pmat{a\\b}
\end{equation}
注意到这里 $|a|^2+|b|^2=1$,为不失一般性,我们将:
\begin{equation}\label{eq_Larmor_2}
a=\cos(\frac{\alpha}{2});\ b= \sin(\frac{\alpha}{2})
\end{equation}
其中的 $\alpha$ 代表着一个固定的角度。这样结合 \autoref{eq_Larmor_1} 和 \autoref{eq_Larmor_2} 可得:
\begin{equation}
\chi(t)=\pmat{\cos(\frac{\alpha}{2})\E^{\I \gamma B_0t/2}\\\sin(\frac{\alpha}{2})\E^{-\I \gamma B_0t/2}}
\end{equation}
为了让我们对这一态有直观的认知,我们来计算 $\bvec S$ 的期待值和时间的关系:
\begin{equation}
\langle S_x\rangle = \chi(t)^\dagger \bvec S_x \chi(t)=\left(\cos(\frac{\alpha}{2})\E^{\I \gamma B_0t/2} \ \ \\sin(\frac{\alpha}{2})\E^{-\I \gamma B_0t/2}\right)\times\frac{\hbar}{2}\pmat{1&0\\0&-1}\pmat{\cos(\frac{\alpha}{2})\E^{\I \gamma B_0t/2}\\\sin(\frac{\alpha}{2})\E^{-\I \gamma B_0t/2}}
\end{equation}
化简可得:
\begin{equation}
\langle S_x\rangle = \frac{\hbar}{2}\sin(\alpha)\cos(\gamma B_0t)
\end{equation}
类似的我们还有:
\begin{equation}
\langle S_y\rangle = -\frac{\hbar}{2}\sin(\alpha)\sin(\gamma B_0t);\ \langle S_z\rangle = \frac{\hbar}{2}\cos(\alpha)
\end{equation}
显然 $\langle\bvec S\rangle$ 在 $z$ 轴之间的夹角 $\alpha$ 是不变的。并且和经典情况一样 $\langle\bvec S\rangle$ 绕着磁场的方向以拉莫尔频率 $\omega$ 进动
\begin{equation}
\omega=\gamma B_0
\end{equation}
如下图所示:
\begin{figure}[ht]
\centering
\includegraphics[width=12cm]{./figures/1dc307dce03d4f62.pdf}
\caption{$\langle\bvec S\rangle$ 在均匀磁场中做进动} \label{fig_Larmor_1}
\end{figure}
