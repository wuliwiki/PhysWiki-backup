% 南京理工大学 普通物理 B(845)模拟五套卷 第四套
% license Usr
% type Note

\textbf{声明}:“该内容来源于网络公开资料,不保证真实性,如有侵权请联系管理员”

\subsection{一、 填空题 I(24 分,每空 2 分)}
1.一质点作圆周运动,设半径为 $R$,运动方程$s=v_0t-\frac{1}{2}bt^2$ ,其中 $s$ 为弧长,$v_0$为初速度,$b$ 为常数。则任一时刻 $t$ 质点的法向加速度为_____,切向加速度为__________。

2. 质量为 4.25Kg 的质点,在合力 (N)的作用下由静止从原点运动到(m)时,合力所做的功为_________;此时质点的运动速度大小为_______________。

3. 花样滑冰运动员绕通过自身的竖直轴转动,开始时两臂伸开,转动惯量为 J,角速度为 ,然后她将两臂收回,使转动惯量减少为 J/2,这时她转动的角速度变为____________。

4. 质量为 2kg 的质点,按方程 x=0.2sin[5t-(π/6)]沿着 x 轴振动,则 t=0 时,作用于质点的力的大小为__________;作用于质点的力的最大值为________,此时质点的位置________。