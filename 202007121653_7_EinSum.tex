% 爱因斯坦求和约定

爱因斯坦求和约定是一种物理学中常用的表达方式,用于简化线性代数的表示.

用一句话来总结爱因斯坦求和约定,就是:\textbf{当式子中任何一个角标出现了两次,并且一次是上标、一次是下标时,那么该式表示的实际上是对这个角标一切可能值的求和}.换言之,如果角标$i$作为上标和下标各出现了一次,那么式子相当于添加了一个关于$i$的求和符号.

我们举例来说明:

\begin{example}{线性函数}
从张量\upref{Tensor}中我们知道,一个$1$-线性函数可以表示为
\end{example}