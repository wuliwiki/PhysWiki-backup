% 连续正交归一基底与傅里叶变换
% keys 正交归一|delta 函数|内积

\pentry{傅里叶变换(指数)\upref{FTExp}}

\subsection{离散的函数基底}
本文使用狄拉克符号\upref{braket}. 在 “傅里叶级数(三角)\upref{FSTri}” 中, 我们介绍了正交归一函数基底的概念, 即把满足一定条件的一元函数的集合看作一个矢量空间\upref{LSpace}, 两个函数(矢量)的内积\upref{InerPd}定义为
\begin{equation}
\braket{f}{g} = \int_{-\infty}^{+\infty} f(x)^* g(x) \dd{x}
\end{equation}
其中 $*$ 表示复共轭, 如果空间中的函数都是实函数则可忽略.

该空间中的一组正交归一基底用狄拉克符号\upref{braket}表示为 $\ket{x_i}$ ($i = 1, 2,\dots$), 基底的个数可以是有限个或无限个, 空间的维数就是基底的个数.

基底满足正交归一条件(\autoref{OrNrB_eq3}~\upref{OrNrB})
\begin{equation}\label{COrNoB_eq2}
\braket{x_i}{x_j} = \delta_{i,j}
\end{equation}
若这组正交归一基底是完备的, 那么任何函数都可以分解为它们的线性组合:
\begin{equation}\label{COrNoB_eq6}
\ket{f} = \sum_j c_j\ket{x_j}
\end{equation}
两边左乘 $\bra{x_i}$, 则有
\begin{equation}\label{COrNoB_eq10}
\braket{x_i}{f} = \sum_j c_j\braket{x_i}{x_j} = \sum_j c_j \delta_{i,j} = c_i
\end{equation}
即
\begin{equation}\label{COrNoB_eq5}
c_i = \braket{x_i}{f}
\end{equation}
\autoref{COrNoB_eq10} 的过程相当于用正交归一性把 $\ket{x_i}$ 项从\autoref{COrNoB_eq6} 的求和中筛选了出来. 我们得到几何矢量中一个熟悉的结论: 一个矢量关于一组正交归一基底的坐标等于它在每个基底上的投影.

\subsection{连续的函数基底}
我们接下来用类似的方法来理解傅里叶变换(\autoref{FTExp_eq6}~\upref{FTExp}).
\begin{align}\label{COrNoB_eq4}
g(k) &= \frac{1}{\sqrt{2\pi }} \int_{-\infty }^{+\infty } f(x)\E^{-\I kx} \dd{x} \\
\label{COrNoB_eq3}f(x) &= \frac{1}{\sqrt{2\pi }} \int_{-\infty }^{+\infty } g(k)\E^{\I kx} \dd{k}
\end{align}
我们令所有可以做傅里叶变换的函数构成的空间为 $X$, 从傅里叶变换的公式, 我们猜想该空间的正交归一 “基底” 为
\begin{equation}\label{COrNoB_eq1}
\ket{k} = \frac{1}{\sqrt{2\pi}} \E^{\I kx} \qquad (k \in \mathbb R)
\end{equation}
严格来说, $X$ 空间的函数必须要满足 $\braket{x}{x}$ 为有限值, 而\autoref{COrNoB_eq1} 中的函数显然不满足这点, 所以它们并不属于 $X$ 空间, 而是一个包含 $X$ 的更大的空间, 所以这个 “基底” 只是一个形象的说法, 需要加上引号.

显然, \autoref{COrNoB_eq1} 中的任意两个 “基底” 的内积都不收敛, 而且 $k$ 的取值是\textbf{连续}的, 所以我们不可能用\autoref{COrNoB_eq2} 表示它们的正交归一关系. 但通过(\autoref{Delta_eq8}~\upref{Delta})
\begin{equation}
\int_{-\infty}^{+\infty} \E^{\I kx}\dd{x} = 2\pi \delta(k)
\end{equation}
可以得到一个和\autoref{COrNoB_eq2} 类似的关系
\begin{equation}
\braket{k'}{k} = \int_{-\infty}^{+\infty} \frac{\E^{-\I k'x}}{\sqrt{2\pi}} \frac{\E^{\I kx}}{\sqrt{2\pi}}\dd{x}
= \frac{1}{2\pi}\int_{-\infty}^{+\infty} \E^{\I (k'-k)x}\dd{x}
= \delta(k' - k)
\end{equation}
即
\begin{equation}\label{COrNoB_eq8}
\braket{k'}{k} = \delta(k' - k)
\end{equation}
这可以看作是\textbf{连续基底的正交归一条件}.

现在, 把\autoref{COrNoB_eq4}  和\autoref{COrNoB_eq3} 用狄拉克符号表示为
\begin{align}\label{COrNoB_eq9}
&g(k) = \braket{k}{f}\\
\label{COrNoB_eq7}&f(x) = \int_{-\infty}^{+\infty} g(k') \ket{k'} \dd{k'}
\end{align}
它们可以分别看作是把\autoref{COrNoB_eq5} 和 \autoref{COrNoB_eq6} 拓展到连续基底的情况. 根据定义, 任何能做傅里叶(反)变换的 $f(x)$ 必定能展开成\autoref{COrNoB_eq7} 的形式. 再来证明\autoref{COrNoB_eq9}, 过程和\autoref{COrNoB_eq10} 类似: \autoref{COrNoB_eq7} 两边左乘 $\bra{k}$, 使用 $\delta$ 函数的性质\autoref{Delta_eq7}~\upref{Delta} 把积分中 $\ket{k}$ 基底的系数 “筛选” 出来
\begin{equation}
\begin{aligned}
\braket{k}{f} &= \int_{-\infty}^{+\infty} g(k) \braket{k}{k'} \dd{k} = \int_{-\infty}^{+\infty} g(k) \delta(k-k') \dd{k'}\\
&= g(k)
\end{aligned}
\end{equation}
证毕.

注意该证明并不仅限于傅里叶变换一种情况, 任何连续的基底 $\ket{k}$ 只要满足\autoref{COrNoB_eq8} , 且可以展开某函数 $f(x)$, 就都能使\autoref{COrNoB_eq9} 成立.
