% 分析力学(综述)
% license CCBYSA3
% type Wiki

(本文根据 CC-BY-SA 协议转载翻译自维基百科\href{https://en.wikipedia.org/wiki/Analytical_mechanics}{相关文章})

在理论物理和数学物理中,分析力学或理论力学是一系列紧密相关的经典力学表述。分析力学利用表示整个系统的标量运动性质——通常是其动能和势能。运动方程通过某种标量变化的基本原理从标量量推导出来。

分析力学是在牛顿力学之后,于18世纪及以后由许多科学家和数学家发展起来的。牛顿力学考虑运动的矢量量,特别是系统组成部分的加速度、动量、力等,因此也可以称为矢量力学。标量是一个数量,而矢量则由数量和方向表示。这两种方法的结果是等价的,但分析力学在处理复杂问题时具有许多优势。

分析力学利用系统的约束条件来解决问题。这些约束限制了系统的自由度,并可以用于减少求解运动所需的坐标数。这种形式适合任意选择的坐标,在此语境下称为广义坐标。系统的动能和势能用这些广义坐标或广义动量表示,运动方程可以轻松建立,因此,分析力学比完全矢量化的方法能够更高效地解决许多力学问题。然而,对于非保守力或如摩擦力等耗散力,分析力学并不总是有效,此时可以回归到牛顿力学。

分析力学的两个主要分支是拉格朗日力学(在构型空间中使用广义坐标及其对应的广义速度)和哈密顿力学(在相空间中使用坐标及对应的动量)。两种表述通过广义坐标、速度和动量上的勒让德变换互相等价,因此它们包含相同的信息来描述系统的动力学。还有其他表述方法,如哈密顿-雅可比理论、劳斯力学和阿佩尔运动方程。任意形式的粒子和场的运动方程都可以从广泛适用的最小作用量原理推导而出。其中一个结果是诺特定理,它将守恒定律与其相关对称性联系起来。

分析力学并没有引入新的物理概念,也不比牛顿力学更为普遍。它是一系列等效的形式,具有广泛的应用。事实上,相同的原理和形式可以用于相对论力学和广义相对论,并在经过一些修正后用于量子力学和量子场论。

分析力学广泛应用于基础物理学和应用数学,尤其是在混沌理论中。

分析力学的方法适用于离散粒子系统,每个粒子具有有限的自由度。它们可以被修改以描述具有无限自由度的连续场或流体。这些定义和方程与力学中的定义和方程有着密切的类比。