% 自旋与有限转动
% license Xiao
% type Tutor

\begin{issues}
\issueMissDepend
\issueTODO
\end{issues}

\subsubsection{空间转动与角动量生成元}
在物理里,常有系统“主动旋转”与坐标系“被动旋转”之分。如下图所示,系统$P$和附着在坐标系上的$Q$点到原点的距离相同。因此,若系统要到达$Q$点,可以绕原点顺时针转动$\phi$角,或者坐标系逆时针转动$\phi$。
\begin{figure}[ht]
\centering
\includegraphics[width=6cm]{./figures/41ed15d994cd1884.png}
\caption{} \label{fig_spinqm_1}
\end{figure}
以右手定则确定$z$轴,现在我们将系统$P$逆时针转动$\phi$。设原坐标为$\bvec v=a\bvec i+b\bvec j+c\bvec k$,新坐标为$\bvec {v'}=a\bvec {i'}+b\bvec {j'}+c\bvec k$。稍加计算可知$\bvec {i'}=\pmat{\cos \phi &\sin {\phi}}^T,\bvec {j'}=\pmat{-\sin \phi &\cos {\phi}}^T$。所以
\begin{equation}\label{eq_spinqm_2}
\bvec{v'}\equiv R_z(\phi)\bvec v=\pmat{\cos\phi&-\sin\phi&0\\\sin\phi&\cos \phi&0\\0&0&1}\bvec v~,
\end{equation}
同理可得
\begin{equation}
R_x(\phi)=\pmat {1&0&0\\0&\cos\phi&-\sin\phi\\0&\sin\phi&\cos\phi},R_y(\phi)=\pmat {\cos\phi&0&\sin\phi\\0&1&0\\-\sin\phi&0&\cos\phi}~.
\end{equation}
设绕$x,y,z$轴转动的生成元分别为$J_x,J_y,J_z$,满足$\E^{-\I J_i\phi}=R_i(\phi)$,则可计算得到三个生成元分别为:
\begin{equation}
\begin{aligned}
J_x&=\I \dv{R_x(\phi)}{\phi}|_{\phi=0}=\I \pmat{0&0&0\\0&0&-1\\0&1&0}\\
J_y&=\I \dv{R_y(\phi)}{\phi}|_{\phi=0}=\I\pmat{0&0&1\\0&0&0\\-1&0&0}\\
J_z&=\I \dv{R_z(\phi)}{\phi}|_{\phi=0}=\I\pmat{0&-1&0\\1&0&0\\0&0&0}~.
\end{aligned}
\end{equation}
读者可验证,生成元满足对易关系$[J_i,J_j]=\I\epsilon_{ijk}J_k$,与量子力学的角动量算符对易关系相同。


\subsubsection{自旋与有限转动}
\pentry{四元数\nref{nod_Quat}}{nod_be4b}
\addTODO{修改旋转算符一节或者合并}

从\autoref{eq_rotOp_1} 可知,轨道角动量算符的$\E$指数关系($\E^{-\I \hat L_i\phi/\hbar}$)可以诱导波函数绕$i$轴逆时针转动$\phi$角。于是很自然的一个猜想是:自旋角动量算符是否可以诱导自旋部分的态矢“转动”? 尽管这并非是空间转动,但满足“旋转”后的“态矢”模方不变。

现在我们利用四元数来证明这一点。已知单位四元数$Q\cong SU(2)$,设线性同构映射$f$为:

\begin{equation}
1\rightarrow\pmat{1&0\\0&1}~,\uvec i\rightarrow-\I\sigma_1=-\I\pmat{0&1\\1&0}~,\uvec j\rightarrow -\I\sigma_2=-\I\pmat{0&-\I\\ \I&0}~,\uvec k\rightarrow -\I\sigma_3=-\I\pmat{1&0\\0&-1}~.
\end{equation}

设$q$为任意四元数,则
\begin{equation}\label{eq_spinqm_1}
\begin{aligned}
f(q)&=f(\cos(\frac{\phi}{2})+\uvec k \sin\frac{\phi}{2})\\
&=\pmat{\cos(\frac{\phi}{2})-\I \sin (\frac{\phi}{2})&0\\0&\cos(\frac{\phi}{2})+\I\sin (\frac{\phi}{2})}\equiv\E^{-\I \hat S_z \phi /\hbar}~,
\end{aligned}
\end{equation}
解得生成元为
\begin{equation}
\hat S_z=\I\hbar \dv{f(q)}{\phi}|_{\phi=0}=\frac{\hbar}{2}\pmat{1&0\\0&-1}~,
\end{equation}
确实是我们熟知的$\hat S_z$。同理,用$\uvec j,\uvec i$代替$\uvec k$,可以推导得到$\hat S_y,\hat S_x$,也是$\hat S_z$表象下的自旋角动量分量形式。因此$\hat S_x,\hat S_y,\hat S_z$是$SU(2)$的生成元,可以诱导对复二维列向量的特殊酉变换。

由上述旋转算符的幺正性可知,特殊酉变换对自旋态矢作用后能保总概率密度不变,然而这和空间旋转不是等价的。以\autoref{eq_spinqm_1} 为例,可知当$\phi=4\pi$时,才对应恒等变换,才能让自旋态矢复位。而对于\autoref{eq_spinqm_2} ,$\phi=2\pi$即可让空间向量复位。
\begin{exercise}{}
设系统态矢为$\ket{\beta}=\pmat{a^1&a^2}^T$。如果系统“绕$z$轴转动$\phi$”,写出转动后的态矢。
\end{exercise}
\subsubsection{自旋进动}
以自旋$1/2$的粒子为例,其自旋期望值为$(\overline{\hat S_x},\overline{\hat S_y},\overline{\hat S_z})$。设该粒子的初始态矢为$\ket{a}$,态矢绕$z$轴“转动”后变为$\mathrm e^{- \I\hat S_z\phi}\ket{a}$。

则期望值变化为:

\begin{equation}
\bra{a}\hat S_i\ket{a}\rightarrow \bra{a}\mathrm e^{ \I\hat S_z\phi}\hat S_i\mathrm e^{- \I\hat S_z\phi}\ket{a}~.
\end{equation}
在$\hat S_z$表象下计算$\mathrm e^{ \I\hat S_z\phi}\hat S_x\mathrm e^{- \I\hat S_z\phi }$得:

\begin{equation}
\begin{aligned}
\mathrm e^{ \I\hat S_z\phi}\hat S_x\mathrm e^{- \I\hat S_z\phi}&=\mathrm e^{ \I\hat S_z\phi}\left(\frac{1}{2}(\ket{-}\bra{+}+\ket{+}\bra{-})\right)\mathrm e^{- \I\hat S_z\phi}\\
 &=\frac{1}{2}\left(\mathrm e^{-\mathrm i t}\ket{-}\bra{+}+\ket{+}\bra{-}\mathrm e^{\mathrm i t}\right)\\
 &=\frac{1}{2}\left[\cos(\phi)(\ket{-}\bra{+}+\ket{+}\bra{-})+\mathrm i\sin(\phi)(\ket{+}\bra{-}-\ket{-}\bra{+})\right]\\
 &=\cos(\phi) \hat S_x-\sin(\phi) \hat S_y~.
\end{aligned}
\end{equation}
因此,$\hat S_x$的期望值变化为:
\begin{equation}
\overline{\hat S_x}\rightarrow  \overline{\hat S_x}\cos(\phi)-\overline{\hat S_y}\sin(\phi)~.
\end{equation}
同理可以计算出其他分量的期望值变化:
\begin{equation}
\overline{\hat S_y}\rightarrow \overline{\hat S_y}\cos(\phi)+\overline{\hat S_x}\sin(\phi)~,
\end{equation}
\begin{equation}
\overline{\hat S_z}\rightarrow \overline{\hat S_z}~.
\end{equation}
因此,自旋期望值可看作经典矢量,态矢绕自旋$z$分量“旋转”相当于该矢量绕自旋$z$分量“旋转”:
\begin{equation}
\begin{pmatrix}
 \cos(\phi) &-\sin(\phi)  &0 \\
  \sin(\phi) & \cos(\phi)  & 0\\
  0& 0 &1
\end{pmatrix}
\begin{pmatrix}
 \overline{\hat S_x}\\
  \overline{\hat S_y}\\
 \overline{\hat S_z}
\end{pmatrix}
=
\begin{pmatrix}
  \overline{\hat S'_x}\\
  \overline{\hat S'_y}\\
 \overline{\hat S'_z}
\end{pmatrix}~.
\end{equation}
可以利用贝克-豪斯多夫(Baker-Hausdorff)公式计算$\mathrm e^{ \I\hat S_z\phi}\hat S_x\mathrm e^{- \I\hat S_z\phi}$。
计算过程表明自旋期望值的变化适用于任意角动量期望值的变化(即也适用于轨道角动量算子期望值)。



