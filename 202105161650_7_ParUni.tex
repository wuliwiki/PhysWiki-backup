% 单位分割
% partition of unity|bump function|光滑函数|光滑延拓|流形|支撑集|support set|support|supp

\pentry{流形\upref{Manif}}

单位分割是现代微分几何中的重要理论工具.我们首先直接给出单位分割的概念.

\begin{definition}{单位分割}
给定\textbf{拓扑空间}$X$.如果存在一组函数$f_\alpha: X \to [0, 1]$,使得对于任意$x\in X$,有:
\begin{enumerate}
\item 只有有限多个$f_\alpha(x)$不为零;
\item 全体不为零的$f_\alpha(x)$之和为$1$.
\end{enumerate}
则称这组函数是$X$上的一个\textbf{单位分割(partition of unity)}
\end{definition}

“单位分割”这一术语不难理解:“单位”就是指$1$这个数字,“分割”就是把它拆分开.单位分割中的所有函数加起来,就得到一个$X$上恒等于$1$的函数.

定义中强调“全体不为零的$f_\alpha(x)$之和”是出于严谨考虑,因为我们并没有定义“不可数多个数字之和”这一运算,而单位分割完全可以包含任意多个函数.当然,第一条限制了只有有限多个函数值不为零,我们确实可以定义全体函数值之和就是全体非零函数值之和,但那还是需要“定义”这一过程的,不如直接按以上定义里的表述来了.

\begin{definition}{支撑集}
给定任意\textbf{集合}$S$和其上一个函数$f:S\to\mathbb{R}$,记$\opn{supp} f = \{s\in S| f(s)\not=0\}$,称其为函数$f$的\textbf{支撑集(support)},也可以简称\textbf{支集}.
\end{definition}







