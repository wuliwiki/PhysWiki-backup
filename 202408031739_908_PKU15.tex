% 北京大学 2015 年 考研 量子力学
% license Usr
% type Note

\textbf{声明}:“该内容来源于网络公开资料,不保证真实性,如有侵权请联系管理员”

\subsection{一}
给定一维无限深势阱的波函数 $\psi_n$。求:
\begin{enumerate}
    \item $E_n$;
    \item $\langle p \rangle$,$\langle p^2 \rangle$;
    \item $\psi_n$的 $\Delta p$ 是否为 0?说明理由。
    \item 动量为 $p$ 的概率密度。
\end{enumerate}
\subsection{二}
在一维无限深势阱中有 2 个粒子,一个处于基态,一个处于第一激发态。按以下三种情况求 $\langle p_1 + p_2 \rangle$ 和 $\langle (p_1 + p_2)^2 \rangle$:
\begin{enumerate}
    \item 粒子可分辨;
    \item 全同玻色子;
    \item 全同费米子。
\end{enumerate}
\subsection{三}
$\vec{r} = \vec{x} \cos \theta + \vec{y} \sin \theta$,$\hat s_r = \hat{s} \cdot \vec{r}$,给定 $\hat s_x$,$\hat s_y$,$\hat s_z$ 的表达式。
\begin{enumerate}
    \item 求 $ \hat\sigma_r$ 的表达式(矩阵),本征态和本征值。
    \item 某个态(自旋单态)测得 $\hat s_z$,$\hat s_r$ 均为 +$\hbar/2$ 的概率。
    \item 求自旋单态的 $\langle \hat s_z ,\hat s_r \rangle$,以及 $\langle \hat s_r ,\hat s_z ?\rangle$。
\end{enumerate}
\subsection{四}
已给出H原子波函数表示式 $\psi_{nlm}$: $\psi_{100}$、$\psi_{200}$、$\psi_{210}$、$\psi_{21\pm 1}$。对于态:
$\psi = \psi_{200}/\sqrt{2} + \psi_{210}/{2} + \frac{\psi_{211}}{2}
$

求:
\begin{enumerate}
    \item $E$ 的测量值和概率;
    \item $L_z$ 的测量值和概率;
    \item $L_x$、$L_y$ 的测量值和概率,并证明 $\Delta L_x \Delta L_y \geq \frac{\hbar}{2} \langle L_z \rangle$~。
\end{enumerate}