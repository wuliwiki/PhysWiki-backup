% 无穷 Galois 扩张与Krull定理
% keys 域论|域扩张|Galois群|伽罗华|伽罗瓦|Krull拓扑|Krull定理|无限Galois扩张|伽罗华扩张

\pentry{Galois扩张\upref{GExt},拓扑空间\upref{Topol}}

\addTODO{尚未完成}

\textbf{Galois扩张}\upref{GExt}中除了Galois扩张和Galois群的基本性质,剩下的重点内容全是\textbf{有限}Galois扩张的情况,见\autoref{GExt_sub1}~\upref{GExt}.作为提醒,再总结一次:有限Galois扩张都是单代数扩张,且为分裂域.

本节介绍的是无限Galois扩张中的性质,将有限扩张的\textbf{Galois理论基本定理}(\autoref{GExt_the10}~\upref{GExt})进行拓展,得到\textbf{Krull}定理.Krull的工作亮点,在于给Galois群赋予了一个拓扑结构.


为了得到Krull拓扑,我们要先观察Galois扩域的一些性质.注意,接下来我们不再限定为有限扩张了.


由\autoref{GExt_the6}~\upref{GExt},$\mathbb{K}/\mathbb{M}$是Galois扩张,因此$\opn{Gal}(\mathbb{K}/\mathbb{M})$存在.有了这一点,我们就可以讨论下面两条引理:

\begin{lemma}{}\label{GExInf_lem2}
设$\mathbb{K}/\mathbb{F}$是Galois扩张,且存在中间域$\mathbb{M}$.

则$\mathbb{M}/\mathbb{F}$是Galois扩张 $\iff$ $\mathbb{M}/\mathbb{F}$是正规扩张 $\iff$ $\opn{Gal}(\mathbb{K}/\mathbb{M})\triangleleft \opn{Gal}(\mathbb{K}/\mathbb{F})$ .
\end{lemma}

\autoref{GExInf_lem2} 实际上就是\autoref{GExt_the8}~\upref{GExt},因此证明参见该定理.


\begin{lemma}{}\label{GExInf_lem1}
设$\mathbb{K}/\mathbb{F}$是Galois扩张,且存在中间域$\mathbb{M}$.则$[\opn{Gal}(\mathbb{K}/\mathbb{F}):\opn{Gal}(\mathbb{K}/\mathbb{M})]=[\mathbb{M}:\mathbb{F}]$.


\end{lemma}


\textbf{证明}:

取$f, g\in\opn{Gal}(\mathbb{K}/\mathbb{F})$,则$f$和$g$模$\opn{Gal}(\mathbb{K}/\mathbb{M})$同余\textbf{当且仅当}$f^{-1}g\in\opn{Gal}(\mathbb{K}/\mathbb{M})$,或者说$f$和$g$限制在$\mathbb{M}$上是相同的.

因此,$\opn{Gal}(\mathbb{K}/\mathbb{M})$的每个左陪集对应一个$\mathbb{M}/\mathbb{F}$的映射.

\textbf{证毕}.


这条引理很像\autoref{GExt_the9}~\upref{GExt},只不过不再要求是有限扩张了.这显得\autoref{GExt_the9}~\upref{GExt}似乎没有存在的必要,然而我们依然将它保留了,体现“有限Galois扩张就是单扩张”的思路.




\begin{theorem}{}\label{GExInf_the1}
设$\mathbb{K}/\mathbb{F}$是Galois扩域,$\mathcal{M}=\{\opn{Gal}(\mathbb{K}/\mathbb{M})\mid \mathbb{M}/\mathbb{F}\text{是有限扩张}\}$\footnote{注意,由\autoref{GExt_the6}~\upref{GExt},$\mathbb{K}/\mathbb{M}$必是Galois扩张.但$\mathbb{M}/\mathbb{F}$则不一定,取决于它是否正规.}.则有:

1.$\forall H\in\mathcal{M}$,$[\opn{Gal}(\mathbb{K}/\mathbb{F}): H]$\footnote{即群指数,子群$H$在$\opn{Gal}(\mathbb{E}/\mathbb{K})$中陪集的数量.}是有限的;

2.$\bigcap_{H\in\mathcal{M}}=\{e\}$.这里$e$是群的单位元,即$\mathbb{K}$上的恒等映射$\opn{id}_{\mathbb{K}}$.

3.$\forall H_1, H_2\in\mathcal{M}$,有$H_1\cap H_2\in\mathcal{M}$;

4.$\forall H\in\mathcal{M}$,$\exists N\triangleleft\opn{Gal}(\mathbb{K}/\mathbb{F})$,且$N\subseteq H$;

\end{theorem}

\textbf{证明}:

1.

由\autoref{GExInf_lem1} 直接可得.

2.

任取$\sigma\in\opn{Gal}(\mathbb{K}/\mathbb{F})$,只要$\sigma\not=\opn{id}_{\mathbb{K}}$,那么就存在$\alpha\in\mathbb{K}-\mathbb{F}$使得$\sigma(\alpha)\neq\alpha$.取$\mathbb{M}=\mathbb{F}(\alpha)$,则$\sigma\not\in\opn{Gal}(\mathbb{K}/\mathbb{F})$.故$\sigma\not\in \bigcap_{H\in\mathcal{M}}$.

3.

由\autoref{GExt_lem1}~\upref{GExt},$H_1\cap H_2$是$\mathbb{M}_1\mathbb{M}_2$的Galois群.

有限域的合成可以看成$\mathbb{M}_1$用$\mathbb{M}_2$的元素反复进行有限次单扩张的结果.由于是Galois扩张,故这些单扩张全都是代数扩张,从而是有限扩张,从而$\mathbb{M}_1\mathbb{M}_2/\mathbb{F}$是有限扩张.

4.

由\autoref{GExInf_lem2} ,只需要证明存在$\mathbb{M}'$,使得$\mathbb{M}'\supseteq\mathbb{M}$且$\mathbb{M}'/\mathbb{F}$是正规扩张.

取$\mathbb{M}'$为$\mathbb{M}$关于$\mathbb{F}$的所有共轭域之合成即可.

\textbf{证毕}.

你可能会想到,\autoref{GExInf_the1} 的第4条完全可以取$N=\{e\}$来证明,也就是取$\mathbb{M}'=\mathbb{K}$,这就导致情况过于平凡,似乎定理第4条没有存在的必要.但我们实际采用的证明过程说明非平凡的情况也是存在的.





\subsection{Krull 定理}\label{GExInf_sub1}


\subsubsection{Krull拓扑}

考虑任意集合$X$和$Y$,令$M\subseteq X^Y$.任取$f\in M$,以及$X$的\textbf{有限}子集$S$,令
\begin{equation}
V(f, S) = \{g\in M\mid g(s)=f(s), \forall s\in S\}
\end{equation}
即$V(f, S)$是全体属于$M$且限制在$S$上与$f$相等的映射的集合.

任取$h\in V(f, S)\cap V(g, T)$,则易得$V(f, S)\cap V(g, T) = V(h, S\cup T)$\footnote{这是因为,任取$V(f, S)\cap V(g, T)$中的元素$c$,则$c$和$f$在$S$上相等,故和$h$在$S$上相等;同理可得$c$和$h$在$T$上相等,从而由$c$的任意性知,$V(f, S)\cap V(g, T)\subseteq V(h, S\cup T)$.反过来,任取$c\in V(h, S\cup T)$,则也可以推知$c$和$f$在$S$上相等、和$g$在$T$上相等,从而$V(h, S\cup T)\subseteq V(f, S)\cap V(g, T)$.}.于是,全体$V(f, S)$的集合对于有限交封闭,从而据\autoref{Topol_sub1}~\upref{Topol}的讨论知,全体$V(f, S)$的集合是一个\textbf{拓扑基}(\autoref{Topol_def2}~\upref{Topol}).

这样全体$V(f, S)$的集合就定义了$M$上的一个拓扑,称之为$M$上的\textbf{有限拓扑(finite topology)}.



\begin{theorem}{}
考虑Galois扩张$\mathbb{K}/\mathbb{F}$.令
\begin{equation}
\begin{aligned}
\mathcal{N} &= \{\sigma N\mid N\in\mathcal{M}, \sigma\in\opn{Gal}(\mathbb{K}/\mathbb{F}), N\triangleleft\opn{Gal}(\mathbb{K}/\mathbb{F})\}\\
\mathcal{L} &= \{\sigma H\mid H\in\mathcal{M}, \sigma\in\opn{Gal}(\mathbb{K}/\mathbb{F})\}\\
\mathcal{R} &= \{H\sigma \mid H\in\mathcal{M}, \sigma\in\opn{Gal}(\mathbb{K}/\mathbb{F})\}
\end{aligned}
\end{equation}
其中$\mathcal{M}d$的定义见\autoref{GExInf_the1} .

则$\mathcal{N}$、$\mathcal{L}$和$\mathcal{R}$都是$\opn{Gal}(\mathbb{K}/\mathbb{F})$上\textbf{有限拓扑}的拓扑基.
\end{theorem}

\textbf{证明}:

只需证明$\mathcal{N}$的情况即可,因为$\mathcal{N}\subseteq \mathcal{L}\cap\mathcal{R}$.

任取$\mathbb{K}/\mathbb{M}$的中间域$\mathbb{M}$,使得$\mathbb{M}/\mathbb{F}$是有限扩张.考虑$\mathcal{M}$的定义,以及“有限可分扩张都是单扩张”(\autoref{PrmtEl_cor2}~\upref{PrmtEl}),可知存在$\alpha$使得$\mathbb{M}=\mathbb{F}(\alpha)$.因此,如果$\sigma_i\in\opn{Gal}(\mathbb{K}/\mathbb{F})$使得$\sigma_1(\alpha)=\sigma_2(\alpha)$,那么$\sigma_1\sigma_2^{-1}\in\opn{Gal}(\mathbb{K}/\mathbb{M})$.

换句话说,$\mathbb{K}$的两个自同构关于$\mathbb{M}$同余,当且仅当它们把$\alpha$映射为同一个元素.于是,取$\mathbb{K}$的\textbf{有限子集}$\{\alpha\}$,有
\begin{equation}
V(\sigma, \{\alpha\}) = \sigma \opn{Gal}(\mathbb{K}/\mathbb{M})
\end{equation}

\textbf{证毕}.



































