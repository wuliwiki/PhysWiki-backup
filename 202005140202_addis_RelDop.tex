% 光的多普勒效应
% 相对论|光速|多普勒|频率|波长

\pentry{多普勒效应(一维匀速)\upref{Dople1}, 洛伦兹变换\upref{LornzT}}

我们只讨论真空中光源和接收者在同一直线做匀速运动的情况. 和机械波的最大的区别在于, 光的传播不需要介质, 任何参考系中真空中的光速都是常数 $c$. % 链接未完成
\begin{equation}
\frac{\omega_2}{\omega_1} = \sqrt{\frac{c - u}{c + u}}
\end{equation}
其中 $u$ 是两点间的相对速度, 远离为正, 靠近为负.

\subsection{推导}
令光源所在参考系为 $S_1$, 接收者所在参考系为 $S_2$.

我们令 $S_1$ 系的两个事件分别为 $x_1 = 0$, $t_1 = 0$, $x_1 = \lambda$, $t_1 = T$. 分别对应一个波节经过原点和一周期之后该节点的位置. 使用洛伦兹变换, 在 $S_2$ 系中, 第一个事件仍然是 $(0, 0)$, 第二个事件为
\begin{equation}
x_2
\end{equation}

