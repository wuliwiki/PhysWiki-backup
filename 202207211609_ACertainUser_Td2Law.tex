% 热力学第二定律
% 热力学|第二定律|熵

\begin{issues}
\issueDraft
\end{issues}

\pentry{熵\upref{Entrop}}

虽然许多基本物理规律(比如牛顿定律,麦克斯韦方程)是可逆的,但许多\textbf{唯象定律}(比如摩擦,传热方程,扩散方程)是不可逆的.克劳修斯首先看出,有必要在热力学第一定律之外建立一条独立的定律来概括自然界的不可逆现象.

\subsection{被动表述}
开尔文表述:不可能从单一热源吸热做功而无环境影响.

克劳修斯表述:热量不能从低温传到高温而无环境影响.

奥斯特瓦德表述:不存在第二类永动机.

可以利用卡诺热机证明,这三种表述是等价的.
\addTODO{证明等价性}

\subsection{数学表述——卡诺定理}

在相同的高温热源和相同的低温热源之间工作的一切\textbf{可逆热机的效率都相等},与工作物质无关;在相同的高温热源和相同的低温热源之间工作的一切\textbf{不可逆热机的效率 $\eta'$ 都小于可逆热机的效率 $\eta$}:
\begin{equation}
\eta'=1-\frac{Q_2'}{Q_1}\le 1-\frac{T_2}{T_1}=\eta
\end{equation}

\addTODO{证明等价性}
\subsection{熵的表述}
\footnote{本节参考自朱文涛《简明物理化学》}

\begin{theorem}{热力学第二定律}
封闭系统任意过程前后,总有
\begin{equation}
\Delta S \ge \int \frac{\delta q}{T}
\end{equation}
否则该过程不可能发生.过程可逆时取等号.
\end{theorem}

\begin{corollary}{熵增原理}
对于绝热或孤立系统,任意过程前后,均有$Q=0$,因此$\Delta S \ge 0$

即\textbf{绝热或孤立系统的熵永不减少.}
\end{corollary}

从这个定理可以看出,孤立系统热平衡的判据可以设为:熵取极大值.

由于热力学熵和玻尔兹曼熵之间的关系(宏观熵 $=$ 微观熵),热力学第二定律有其统计意义:孤立系统的自发过程(不可逆过程)总是从有序向无序过渡,即从出现概率小的宏观状态向出现概率大的宏观状态过渡.也就是说,虽然基本物理定律是有时间反演不变性,但从统计意义上看,宏观规律却能体现时间箭头的方向!
\addTODO{证明等价性}
