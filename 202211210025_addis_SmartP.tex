% C++ 的 smart pointer

\begin{issues}
\issueDraft
\end{issues}

\subsection{unique_ptr}
\begin{itemize}
\item \href{https://en.cppreference.com/w/cpp/memory/unique_ptr}{文档}.
\item \verb|unique_ptr<类> p(类指针)|
\item \verb|unique_ptr<类,删除函数类> p(类指针, 删除函数)|
\item \verb|删除函数类| 和 \verb|删除函数| 可以省略, 省略后默认用 \verb|delete 类指针|.
\item unique_ptr 的 destructor 被调用时, 删除函数被调用.
\item 必须初始化, 不存在 \verb|operator=|, 不可以复制.
\end{itemize}

\subsection{shared_ptr}
\begin{itemize}
\item 可以使用 \verb|operator=| 复制, 只有所有复制的 shared_ptr 的 destructor 都被调用, 才会调用删除函数.
\end{itemize}
