% 二维无限深方势阱
% keys 薛定谔方程|无限深势阱|束缚态|分离变量法|算符|傅里叶级数
% license Xiao
% type Tutor

\pentry{无限深势阱\nref{nod_ISW}, 分离变量法解偏微分方程\nref{nod_SepVar}}{nod_17f9}

\footnote{本文使用原子单位}现在我们来看一个粒子的二维运动。 定态薛定谔方程为
\begin{equation}\label{eq_ISW2D_1}
H \Psi(x, y) = E \Psi(x, y)~.
\end{equation}
其中哈密顿算符为
\begin{equation}
H = -\frac{1}{2} \laplacian + V(x, y) =  -\frac{1}{2} \pdv[2]{x} -\frac{1}{2} \pdv[2]{y}  + V(x, y)~.
\end{equation}
前两项分别为 $x$ 和 $y$ 方向的动能算符, $V(x, y)$ 为势能(算符)
\begin{equation}
V(x, y) =
\begin{cases}
0  & (0 \leqslant x \leqslant a,\: 0 \leqslant y \leqslant b)\\
+\infty & (\text{其他情况})
\end{cases}~.
\end{equation}
$V(x, y)$ 可以看成 $V_x(x)$ 和 $V_y(y)$ 两个函数相加, 他们的定义为
\begin{equation}
V_x(x) =
\begin{cases}
0  & (0 \leqslant x \leqslant a)~,\\
+\infty & (\text{其他情况})
\end{cases}
\end{equation}
\begin{equation}
V_y(y) =
\begin{cases}
0  & (0 \leqslant y \leqslant b)~.\\
+\infty & (\text{其他情况})
\end{cases}
\end{equation}
于是总哈密顿算符可以记为两部分, 每部分是一个一维简谐振子的哈密顿算符
\begin{equation}
H = H_x + H_y~,
\end{equation}
\begin{equation}
\leftgroup{
H_x &= -\frac{1}{2} \pdv[2]{x} + V_x(x)\\
H_y &= -\frac{1}{2} \pdv[2]{y} + V_y(y)
}~.
\end{equation}

为什么这样做呢? 这使我们可以使用分离变量法。 令
\begin{equation}
\Psi(x, y) = \Psi_x(x)\Psi_y(y)~.
\end{equation}
代入薛定谔方程两边同除以 $\Psi_x(x)\Psi_y(y)$ 得
\begin{equation}
\frac{H_x \Psi_x(x)}{\Psi_x(x)} + \frac{H_y \Psi_y(y)}{\Psi_y(y)} = E~.
\end{equation}
由于第一项只和 $x$ 有关, 第二项只和 $y$ 有关, 所以它们都是常数, 令
\begin{equation}
\begin{cases}
H_x \Psi_x(x) = E_x \Psi_x(x)\\
H_y \Psi_y(y) = E_y \Psi_y(y)
\end{cases}~,
\end{equation}
他们分别是一维简谐振子的定态薛定谔方程。 他们的各个束缚态波函数(\autoref{eq_ISW_1})分别为
\begin{equation}
\Psi_{x, i}(x) = \sqrt{\frac{2}{a}} \sin(\frac{n\pi}{a} x) \quad (i = 1,2,...)~,
\end{equation}
\begin{equation}
\Psi_{y, j}(y) = \sqrt{\frac{2}{b}} \sin(\frac{n\pi }{b} y) \quad (j = 1,2,...)~.
\end{equation}
对应的能量分别为
\begin{equation}
E_{x, i} = \frac{\pi^2}{2 a^2} i^2~,
\qquad
E_{y, j} = \frac{\pi^2}{2 b^2} j^2~.
\end{equation}
则总哈密顿算符的束缚态可以记为它们的乘积
\begin{equation}
\Psi_{i, j}(x, y) = \Psi_{x, i}(x) \Psi_{y, j}(y) = \frac{2}{\sqrt{ab}} \sin(\frac{i \pi}{a} x) \sin(\frac{j \pi}{b} )~.
\end{equation}
对应的能量为
\begin{equation}
E_{i, j} = E_{x,i} + E_{y,j} = \frac{\pi^2}{2} \qty(\frac{i^2}{a^2} + \frac{j^2}{b^2})~.
\end{equation}
另外, 由于 $\Psi_{x, i}(x)$ 和 $\Psi_{y, j}(y)$ 都是完备的, 它们的乘积也是完备的, 即二维势阱中任意波函数可以表示为
\begin{equation}
\Psi(x, y) = \sum_{i,j} C_{i, j} \Psi_{x, i}(x) \Psi_{y, j}(y)~.
\end{equation}
事实上, 这就是二维傅里叶级数。 % \addTODO{链接}。 
% 未完成: 说明一维无限深势阱其实就是傅里叶级数
