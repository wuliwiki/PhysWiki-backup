% Clifford 代数
% keys 环|二次型
% license Xiao
% type Tutor


如果要构建有限维结合代数,我们需要对basis的结合作出约束。例如,Grassmann代数要求反对称性:$\mathrm {e_1e_2}=-\mathrm {e_2e_1}$。实际上,Grassmann代数是Clifford代数的一个平凡特例,或者也可以理解为一般Clifford代数上的内嵌结构。本节先给出线性空间上的Clifford代数的定义,再用集合语言将其拓展为更一般的Clifford代数。
\subsection{几何代数}
\begin{theorem}{线性空间的理想}
给定域$\mathbb F$上的线性空间$V$,其上有一二次型$B_q(B_q(v,w)=\frac{1}{2}(q(v+w)-q(v)-q(w)))$.令$\mathcal T(V)$为$V$上的张量代数,那么如下定义的$\mathcal {I}_q(V)$是它的理想:
\begin{equation}
\mathcal{I}_{q}(V)=\left\{\sum A_{k} \otimes(v_k \otimes v_k-q(v_k)) \otimes B_{k} \mid v \in V, A_{k}, B_{k} \in \mathcal{T}(V)\right\}~,
\end{equation}
\end{theorem}
proof.
环理想首先是加法子群,其次对乘法有“吸收律”。该定理可以简化成一个更简单的形式。即对于环$R$上的一个非空子集$S$,我们可以证明该子集生成的理想为
\begin{equation}
\mathcal {I}_S=\left\{\sum _k a_k s_kb_k|k\in \mathbb N ,a_k,b_k\in R,s_k\in S\right\}~,
\end{equation}
检查理想的定义,该集合确实构成加法子群。其次,无论是左乘还是右乘环元素,都能表示为该形式,因而是理想。对于张量代数,乘法为张量积。

\begin{definition}{几何代数}
给定域$\mathbb F$上的线性空间$V$,其上有一二次型$B_q$.理想同上定义,商代数则为几何代数(\textbf{geometric algebra}),即
\begin{equation}
\mathcal{G}(V, q) \stackrel{\text { def }}{=} \mathcal{T}(V) / \mathcal{I}_{q}(V)~,
\end{equation}
另外,称$V$为$\mathcal{G}(V, q)$的底空间(\textbf{base space})。
\end{definition}
划分等价类后,把$\mathcal{G}(V, q)$上的向量积称为\textbf{几何积},或者\textbf{Clifford积},符号可以用$\cdot$或者不写。

观察等价类,我们会发现一项特殊的等价关系。即
\begin{equation}
v\cdot v=q(v)~,
\end{equation}

理想首先是正规子群。回想对正规子群求商集时,若$a,b$等价,即属于同一左陪集,那么$a^{-1}b$属于该正规子群。因而,上式的等价关系实际上指的是
$v \otimes v-q(v)$在理想里,显然这是成立的。该等价关系得以让我们把重复的$\mathrm {e_i}$约掉。如果$\{\mathrm{e_i}\}$是线性空间中的正交基,由于
\begin{equation}
\begin{aligned}
q(\mathrm{e_i+e_j,e_i+e_j})&=q(\mathrm{e}_i)+q(\mathrm{e}_j)+2 B_q(\mathrm{e}_i,\mathrm{e}_j)\\
&=\mathrm{e}_i\mathrm{e}_i+\mathrm{e}_j\mathrm{e}_j+\mathrm{e}_i\mathrm{e}_j+\mathrm{e}_j\mathrm{e}_i~,
\end{aligned}
\end{equation}

因而对于正交基有$\mathrm{e}_i\mathrm{e}_j=-\mathrm{e}_j\mathrm{e}_i$,满足交换反对称关系。对于$V$中任意的两个向量$v,w$,由$q(v+w,v+w)$的展开式得到Clifford代数下的交换反对称关系:
\begin{equation}
vw+wv=2B_q(v,w)~
\end{equation}

\subsection{Clifford代数的形式化定义}
几何代数的基域为线性空间,而\textbf{Clifford代数}的基域是交换幺环。\textbf{Clifford代数}实际上是交换环上的模。
\begin{definition}{Clifford代数}
给定任意指标集合$X$,任意给定交换幺环$R$,任意函数$s:X\rightarrow {R}$作为$X$上的符号。幂集$2^X$可以生成如下自由R-模:
\begin{equation}
\mathrm{Cl}(X,R,s)=\bigoplus_{2^X}R~,
\end{equation}
称之为$(X,R,s)$上的\textbf{Clifford代数}。
\end{definition}

例如,给定整数环,指标集为$\{1,2,3\}$。那么$Cl(X,R,s)$上的一个元素为$2\{1\}+3\{2,3\}+4\{\}$


$Cl(X,R,s)$上的乘法是$R$-线性的,并且对于$A,B\in 2^X$有
\begin{equation}\label{eq_clf01_1}
AB=\tau(A,B)A\Delta B~,
\end{equation}
其中映射$2^X \times 2^X \rightarrow R$定义为
\begin{equation}
\left\{\begin{aligned}
\tau(\{x\},\{x\}) & =s(x), \quad \forall x \in X, \\
\tau(\{x\},\{y\}) & =-\tau(\{y\},\{x\}), \quad \forall x, y \in X, x \neq y, \\
\tau(A, B) & \in\{1,-1\}, \quad \forall A, B \subseteq X, A \cap B=\varnothing, \\
\tau(A, \varnothing) & =\tau(\varnothing, A)=1, \quad \forall A \subseteq X, \\
\tau(A, B) \tau(A \Delta B, C) & =\tau(B, C) \tau(A, B \Delta C), \quad \forall A, B, C \subseteq X~,
\end{aligned}\right.
\end{equation}


从上式我们可以发现其与定义1的联系。
\begin{enumerate}
\item 幂集是张量积空间基底的拓展。因此$\tau(\{x\},\{x\})$是二次型的拓展。当环为域时可以清晰地看见这种联系:$\tau(\{e_i\},\{e_i\})=q(e_i)=s(e_i)$
\item 同样的,上式第二条是正交基的反对称性。实际上是$\mathrm {e_ie_j=-e_je_i}$。也就是说,如果集合定义要与定义1兼容,各单点集需要构成正交基。
\item 第三条定义是为了和第二条自洽
\item 第四条是$1\times e_A=e_A\times 1$的拓展
\item 最后一条实际上是结合律$(AB)C=A(BC)$,再次强调这里的$A,B,C$是指标集合$X$的子集(默认该子集的指标按从小到大排列),各自相当于张量积空间的基底。即$\mathrm {(e_Ae_B)e_C=e_A(e_Be_C),e_A=e_i/e_ie_j/...}$
\item \autoref{eq_clf01_1} 实际上说的是$\mathrm {e_A e_B\propto e_{A\Delta B}}$。比如$\mathrm {(e_{1}e_{2}e_{3})(e_{2}e_{3}e_{4}e_{5})=-q(e_2)q(e_3)e_1 e_4 e_5}$
\end{enumerate}

下面证明第四条实际上是结合律。
\begin{equation}
\begin{aligned}
\left(e_A e_B\right) e_C & =\tau(A, B) e_{A \Delta B} e_C \\
& =\tau(A, B) \tau(A \Delta B, C) e_{(A \Delta B) \Delta C} \\
& =\tau(B, C) \tau(A, B \Delta C) e_{A \Delta(B \Delta C)} \\
& =\tau(B, C) e_A e_{B \Delta C} \\
& =e_A\left(e_B e_C\right) .
\end{aligned}~,
\end{equation}
从第二行到第三行的推导使用$(A \Delta B) \Delta C=A \Delta(B \Delta C)$
\subsubsection{分次结构}
\begin{definition}{}
给定Clifford代数$\mathrm {Cl(X,R,s)}$。定义
\begin{equation}
\mathrm{Cl}^k(X,R,s)=\bigoplus_{A \in 2^X:|A|=k}R~,
\end{equation}
称之为$\mathrm {Cl(X,R,s)}$的k次部分。该子空间中的元素称之为\textbf{k-向量}
\end{definition}
\begin{definition}{}
\begin{equation}
\begin{array}{l}
\mathrm{Cl}^{+}(X, R, s)=\bigoplus_{k \text { 为偶数 }} \mathrm{Cl}^k(X, R, s) \\
\mathrm{Cl}^{-}(X, R, s)=\bigoplus_{k \text { 为奇数 }} \mathrm{Cl}^k(X, R, s)
\end{array}~,
\end{equation}
\end{definition}






