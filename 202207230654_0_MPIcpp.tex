% MPI 笔记(C++)

\begin{issues}
\issueDraft
\end{issues}

\begin{itemize}
\item intel MPI 和 MKL 一样是免费的, google 一下, 下载安装包即可, 运行 \verb`sudo ./install.sh` 安装, 过程和 MKL 差不多
\item 如果已经安装了 MKL, 可能会提示目录已经存在(不确定 MKL 是否已经包含 MPI)
\item 安装好以后同样需要在 \verb`~/.bashrc` 中添加路径, 即在文件最后加入命令 \verb`source /opt/intel/compilers_and_libraries_2020.1.217/linux/mpi/intel64/bin/mpivars.sh`
\item 重启一下 shell, 运行 \verb`mpicxx --help`, 如果进入帮助页面就说明成功了


\item 入门教程页面 https://people.sc.fsu.edu/~jburkardt/cpp_src/hello_mpi/hello_mpi.html
\item 尝试直接使用 Intel 的 MPI
\item \verb`mpicxx` 或者 \verb`mpigxx` 好像都是一样的是 g++ 的一个 wraper, 和 g++ 一样使用即可
\end{itemize}
