% 零化多项式
% license Xiao
% type Tutor


\begin{issues}
\issueDraft
\end{issues}



\begin{definition}{}
设$f(x)$是以域$\mathbb F$中元素为系数的一元多项式,$A$是线性空间$V$上的线性变换。若$f(A)=0$(即零变换),则称$f$是$A$的一个\textbf{零化多项式}(null polynomial)。
\end{definition}
可以验证,若$f$是线性变换$A$的零化多项式,那么也是其任意基下矩阵所对应的零化多项式,即无论选取什么基底,代入多项式的最终结果为零矩阵。这是因为若$A=Q^{-1}BQ$,我们有$f(A)=Q^{-1}f(B)Q$。


下面一条定理及其推论表明了寻找零化多项式的重要意义。
\begin{theorem}{}
给定域$\mathbb F$上的线性空间$V$,域上的多项式$h$可以分解为互素多项式:$h=fg$。对于定义在该线性空间的线性变换$A$,我们有:
\begin{equation}
\opn{ker}h(A)=\opn{ker}f(A)\oplus\opn{ker}g(A)~.
\end{equation}
\end{theorem}
Proof.

首先证明$\opn{ker}f\cap \opn{ker}g=0$。由互素得,存在多项式$u,v$使得$uf+vg=I$。设$\bvec x\in \opn{ker}f\cap \opn{ker}g$,则$(uf+vg)\bvec x=0=\bvec x$。矛盾,因而$f,g$无交集,和为直和。

下证$\opn{ker}h=\opn{ker}f\oplus\opn{ker}g$。
第一步,先证$\opn{ker}h\subset\opn{ker}f\oplus\opn{ker}g$。设$\{\bvec x_i\}$和$\{\bvec y_i\}$分别是$\opn{ker}f$和$\opn{ker}g$的基,则$fg(a^i\bvec x_i+b^i \bvec y_i)=a^igf(\bvec x_i)+b^ifg(\bvec y_i)=0$,第一步得证。

第二步证明$\opn{ker}h\supset\opn{ker}f\oplus\opn{ker}g=\opn{ker}f+\opn{ker}g$。设$\bvec x\in \opn{ker}h$,由之前的证明过程知:$\bvec x=(uf+vg)\bvec x$,只要把这两项分配给$f,g$的核即可。显然,$g(uf\bvec x)=f(vg\bvec x) =0$,得证。

\begin{corollary}{}\label{cor_nullpl_1}
$\mathbb F,V,A,h$同上设。$h$可以分解为\textbf{两两互素}的多项式乘积:$h=h_1h_2...h_n$,则:
\begin{equation}
\opn{ker}h(A)=\opn{ker}h_1(A)\oplus\opn{ker}h_2(A)\oplus...\oplus\opn{ker}h_n(A)~.
\end{equation}
\end{corollary}

若$h(A)=0$,则$\opn{ker} h=V$,该条定理意味着若我们找到任意线性变换$A$的零化多项式,则可以把线性空间分解为互素多项式的核。

若$f$是$A$的多项式,且$\bvec x\in \opn{ker}f(A)$,则$f(A)A x=0$,即\textbf{任意多项式的核都是对应线性变换的不变子空间。}综合前文,这意味着如果任意线性变换都有零化多项式,那么我们可以把$A$分解为\textbf{块对角矩阵},此时线性空间是$A$的不变子空间之直和。
为了达到这个目的,我们还需要两个准备工作:找到一条路径能让我们快速确定任意线性变换对应的零化多项式,以及把多项式分解为互素项的简便方法。

Cayley-Hamilton定理告诉我们,线性变换的特征多项式就是一个零化多项式。
\begin{theorem}{Cayley-Hamilton定理}
给定复线性空间$V$上的线性变换$A$,若$f(\lambda)=\opn{det}(A-\lambda I)$为其特征多项式,则$f(A)=0$
\end{theorem}
证明思路是利用复线性空间的任意矩阵都可相似于上三角矩阵,上三角矩阵的零化多项式即特征多项式,以及零化多项式不随基的改变而改变($f(A)=Q^{-1}f(B)Q=0$)。现在只证明第二点,其余读者可自证。

设线性变换$\mathcal A$在基向量组$\{\bvec e_1,\bvec e_2,...,\bvec e_n\}$下的表示为$n$阶上三角矩阵$A=(a^i_j)$,由矩阵表示可知:$\mathcal A(\bvec e_k)=\opn{span}(\bvec e_1,\bvec e_2...\bvec e_k)$。具体来说是:
\begin{equation}
\begin{aligned}
\mathcal A(\bvec e_1)&=a^1_1 \bvec e_1,\\
\mathcal A(\bvec e_2)&=a^1_2 \bvec e_1+a^2_2\bvec e_2,\\
&...\\
\mathcal A(\bvec e_n)&=a^1_n \bvec e_1+a^2_n\bvec e_2+...a^n_n\bvec e_n,\\
\end{aligned}~
\end{equation}
因此,对于$A$的特征多项式,代入$A$后$f(A)=(A-\lambda_1)(A-\lambda_2)...(A-\lambda_n)=(A-a^1_1)(A-a^2_2)...(A-a^n_n)$\footnote{上三角矩阵的对角元为其特征值},利用多项式对易,可证该线性空间中的任意矢量被该多项式作用后结果都为$0$。因此,上三角矩阵的特征多项式为其零化多项式。
\begin{theorem}{线性空间第一分解定理}\label{the_nullpl_1}
设$V$为\textbf{复数域}上的线性空间,$A$为该空间中的线性变换,且特征多项式可以写为:
\begin{equation}
f(\lambda)=(\lambda-\lambda_1)^{k_1}(\lambda-\lambda_2)^{k_2}...(\lambda-\lambda_m)^{k_m}~,
\end{equation}
若设$f_i(A)=(A-\lambda_i)^{k_i}$,则有:
\begin{equation}
V=\opn{ker}f(A)=\opn{ker}f_1(A)\oplus\opn{ker}f_2(A)...\oplus\opn{ker}f_m(A)~.
\end{equation}
\end{theorem}
Proof.

由于$f_i$的$\lambda_i$彼此不同,因此互素,由\autoref{cor_nullpl_1} 得:
\begin{equation}
\opn{ker}f_i(A)\oplus \opn{ker}f_j(A)=\opn{ker}(f_i(A)f_j(A))~.
\end{equation}
所以
\begin{equation}
\begin{aligned}
\opn{ker}f_1(A)\oplus\opn{ker}f_2(A)...\oplus\opn{ker}f_m(A)&=\opn{ker}(f_1(A)f_2(A))..\oplus\opn{ker}f_m(A)\\
&=\opn{ker}(f_1(A)f_2(A)f_3(A))..\oplus\opn{ker}f_m(A)\\
&=\opn{ker}f(A)=V
\end{aligned}
~.
\end{equation}
称$\opn{ker}f_i(A)$是$A$的属于对应特征值的根子空间,该分解方式称为$A$的\textbf{根子空间分解。}

总结一下,任意线性变换的特征多项式都是其零化多项式,由于特征多项式可以分解为互素项,因此任意线性空间都可以分解为任意线性变换的不变子空间之直和,此时线性变换是分块对角矩阵。

