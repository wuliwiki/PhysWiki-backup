% 多线性映射
% keys 线性映射|线性型
% license Xiao
% type Tutor

\begin{issues}
\issueTODO
\end{issues}


\pentry{线性映射\upref{LinMap}}

\subsection{多线性映射}
\begin{definition}{多线性映射}
设 $V_1,\cdots,V_p, U$ 为域\upref{field} $\mathbb{F}$ 上的向量空间\upref{LSpace}。映射
\begin{equation}
f:V_1\times V_2\times\cdots\times V_p\rightarrow U~
\end{equation}
称为\textbf{多线性}、\textbf{$p$-线性}(或者\textbf{多重线性映射}),如果对任意指数 $i=1,\cdots,p$ 及任意固定的向量 $\bvec{a}_j \in V_j\quad(1 \leq j \leq p,j\neq i)$ ,映射
\begin{equation}
f_i:\bvec{v}\mapsto f(\bvec{a}_1, \cdots, \bvec{a}_{i-1}, \bvec{v},\bvec{a}_{i+1}, \cdots,\bvec{a}_p)~
\end{equation}
都是线性的。
\end{definition}
所有 $p$-线性的映射构成的集合记为 $\mathcal{L}(V_1,\cdots,V_p;U)$。

\begin{theorem}{}
$p$-线性映射的集合 $\mathcal{L}(V_1,\cdots,V_p;U)$ 构成 $\mathbb{F}$-向量空间,它的线性组合(加法和数乘)定义为
\begin{equation}
(\mu f+\nu g)(\bvec a_1,\cdots,\bvec a_p)=\mu f(\bvec a_1,\cdots,\bvec a_p)+\nu g(\bvec a_1,\cdots,\bvec a_p)~
\end{equation}
其中,$\forall \mu,\nu \in\mathbb{F}~, \bvec a_i \in V_i~, f, g\in \mathcal{L}(V_1,\cdots,V_p;U)$。
\end{theorem}
\textbf{证明:} 即证明,任意两个 $p$-线性映射的线性组合 $\alpha f+\beta g$ 也是一个 $p$-线性映射。

令 $h=\alpha f+\beta g$ ,且对任意指数 $i=1,\cdots,p$ 及任意固定的向量 $\bvec{a}_j \in V_j\quad(1\leq j\leq p,j\neq i)$ ,记
\begin{equation}
h_i:\bvec{v}\mapsto h(\bvec{a}_1,\cdots,\bvec{a}_{i-1};\bvec{v},\bvec{a}_{i+1},\cdots,\bvec{a}_p)~.
\end{equation}
则对 $\forall \bvec x,\bvec y\in V_i,\quad a,b\in\mathbb{F}$
\begin{equation}
\begin{aligned}
h_i(a \bvec x+b\bvec y)&=(\alpha f_i+\beta g_i)(a \bvec x+b\bvec y)\\
&=\alpha f_i(a \bvec x+b\bvec y)+\beta g_i(a \bvec x+b\bvec y)\\
&=a\alpha f_i(\bvec x)+b\alpha f_i(\bvec y)+a\beta g_i(\bvec x)+b\beta g_i(\bvec y)\\
&=a\qty(\alpha f_i(\bvec x)+\beta g_i(\bvec x))+b(\alpha f_i(\bvec y)+\beta g_i(\bvec y))\\
&=ah_i(\bvec x)+bh_i(\bvec y)~,
\end{aligned}
\end{equation}
这显然满足 $p$-线性映射的定义。证毕!

% \subsubsection{多线性型}

\begin{definition}{多线性型}\label{def_MulMap_2}
任意 $V_1\times V_2\times\cdots\times V_p$ 到 $\mathbb{F}$ 的多线性映射称为 $V_1\times V_2\times\cdots\times V_p$ 上的\textbf{多线性型}(\textbf{multilinear form})或者\textbf{多线性$\mathbb{F}$-值函数},简称\textbf{多线性函数}。
\end{definition}

% \begin{example}{}
% 设
% \begin{equation}
% l^i:\bvec{v_i}\mapsto l^i(\bvec{v_i})\qquad (i=1,\cdots,p)~.
% \end{equation}
% 是 $V_i$ 上的线性函数(型),那么用关系式
% \begin{equation}
% f(\bvec{v_1},\cdots,\bvec{v_p})=l^1(\bvec{v_1})\cdots l^p(\bvec{v_p})~
% \end{equation}
% 定义的函数 $f$ 就是 $V_1\times V_2\times\cdots\times V_p$ 上的多线性函数。称它为线性函数(型)$l^1,\cdots,l^p$ 的\textbf{张量积}且表成 $f=l^1\otimes l^2\otimes\cdots\otimes l^p$ 或简记为 $l^1l^2\cdots l^p$(有序)。
% \end{example}

% Giacomo: 这种写法容易产生误解
% 
% 当 $V_1=\cdots=V_p=V$ 时, $V^p=V\times\cdots\times V$ (集合 $V$ 的 $p$ 个元素的笛卡尔积),此时,记
% \begin{equation}
% \mathcal{L}_p(V,\mathbb{F})=\mathcal{L}(V,\cdots,V;\mathbb{F})~
% \end{equation}
% 是很方便的。

\subsection{对称/反对称多线性映射}

\addTODO{群作用重写}

回忆\autoref{sub_vecSAS_1}~\upref{vecSAS}中我们定义的 $S_n$ 置换作用,
\begin{equation}
\begin{aligned}
\rho(\sigma): V^{\times n} &\to V^{\times n}~, \\
(v_1, \cdots, v_n) &\mapsto (v_{\sigma(1)}, \cdots, v_{\sigma(n)})~,
\end{aligned}
\end{equation}
我们称一个多线性映射 $f: V^{\times n} (= V \times \cdots \times V) \to W$ 为\textbf{对称多线性映射}(简称\textbf{对称映射})如果它是 $(S_n, \rho)$-不变的(参考\autoref{def_GrpRep_1}~\upref{GrpRep}),即对任意置换 $g \in S_n$
\begin{equation}
f(\rho(g) v) = f(v)~,
\end{equation}
由于对称群的性质,我们可以把它简化为,交换任意两项 $v_i, v_j$ 后函数值保持不变:
\begin{equation}
f(\cdots, v_j, \cdots, v_i, \cdots) = f(\cdots, v_i, \cdots, v_j, \cdots)~.
\end{equation}

\addTODO{将 $G$-不变映射的定义改写到 $G$-空间}

\begin{definition}{对称多线性型}
$V^{\times n}$ 上的多线性型 $f$ 称为\textbf{对称的},如果它是一个对称多线性映射。
% Giacomo: 对称/反对称多线性型没有特殊之处。
%
% 如果对任意 $\bvec{v}_1,\cdots\bvec{v}_p \in V$ 及任意置换\upref{Perm} $\pi\in S_p$ ,都有
% \begin{equation}
% f(\bvec{v}_{\pi(1)},\bvec{v}_{\pi(2)},\cdots,\bvec{v}_{\pi(p)})=f(\bvec{v}_1,\bvec{v}_2,\cdots,\bvec{v}_p)~,
% \end{equation}
% 而 $f$ 称为\textbf{斜对称的}(或反对称)。如果
% \begin{equation}
% f(\bvec{v}_{\pi(1)},\bvec{v}_{\pi(2)},\cdots,\bvec{v}_{\pi(p)})=\epsilon_\pi f(\bvec{v}_1,\bvec{v}_2,\cdots,\bvec{v}_p)~,
% \end{equation}
% 其中 $\epsilon_\pi$ 是置换的符号(偶置换取1,奇置换取负)\upref{permu}。
\end{definition}

类似的我们可以定义的\textbf{反对称多线性映射} (简称\textbf{反对称映射})$f: V^{\times n} \to W$ 如果它满足 $g \in S_n$
\begin{equation}
f(\rho(g) v) = \opn{sgn}(g) f(v)~,
\end{equation}
其中对于偶置换$\opn{sgn}{\sigma} = 1$、奇置换$\opn{sgn}{\sigma} = -1$;由于对称群的性质,我们可以把它简化为,交换任意两项 $v_i, v_j$ 后函数值相反:
\begin{equation}
f(\cdots, v_j, \cdots, v_i, \cdots) = - f(\cdots, v_i, \cdots, v_j, \cdots)~.
\end{equation}



\begin{definition}{反对称性多线性型}\label{def_MulMap_1}
$V^{\times n}$ 上的多线性型 $f$ 称为\textbf{反对称的}(或者\textbf{斜对称}、\textbf{交错的}),如果它是一个反对称多线性映射。
\end{definition}
