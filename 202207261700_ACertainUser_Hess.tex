% 盖斯定律与设计路径
% 赫斯 赫士 盖斯 Hess 路径 状态量变化

\pentry{态函数\upref{statef},状态量和过程量\upref{StaPro},准静态过程\upref{Quasta}}

\footnote{参考了朱文涛的《简明物理化学》,安宇等的《大学物理》课程,李俊杰的《物理化学》课程}

\subsection{盖斯定律}
\begin{theorem}{盖斯定律}
(等压或等容过程中)一个反应,不管是一步完成的还是多步完成的,其热效应总是相同.
\end{theorem}
盖斯定律最早源于实验观察,是能量守恒在热力学中的另一种体现.

\begin{example}{碳的氧化}
\begin{figure}[ht]
\centering
\includegraphics[width=8cm]{./figures/Hess_2.png}
\caption{C的氧化} \label{Hess_fig2}
\end{figure}
一定量的C被氧化为CO2.无论是直接被氧化为CO2,还是先被不完全氧化生成CO、再由CO被氧化为CO2,二者放出的总热量应相同.
\end{example}

\subsection{盖斯定律的物理内涵}
事实上,盖斯定律有着更为基本、深刻的物理含义.\textbf{我们知道,根据状态量的特性,系统状态量的改变与路径无关\upref{StaPro}\upref{statef}}.也就是说,无论C是直接被氧化为CO2,还是先被不完全氧化生成CO、再由CO被氧化为CO2,由于系统的始末状态一致,因此系统始、末状态的焓H分别相同,因此系统的焓变也相同$$\Delta H=H_2-H_1=Constant$$,又因为等压,$$Q=\Delta H$$,系统的放热自然相同.

可见,盖斯定律正是状态量性质的一种体现.

\subsection{“设计路径”}
\begin{figure}[ht]
\centering
\includegraphics[width=12cm]{./figures/Hess_1.png}
\caption{设计路径} \label{Hess_fig1}
\end{figure}

根据系统状态量的改变与路径无关的特性,我们可以\textbf{在两个状态间人为设计一条路径,以求解任意过程前后状态量的变化}.这可以理解为“广义盖斯定律”.

设计的路径一般为等压、等容或等温的可逆(准静态)路径;且可以包括多个步骤,每一步只改变一个变量.这样方便运用现成结论.

此处的状态量不再局限于焓.只要能找到适合的路径,原则上可以计算任意状态量(U,S,G,H...)的变化.然而,由于过程量(Q,W)的变化与路径有关,一般不可设计路径计算过程量的变化\footnote{除了少数 条件特殊 的情况}.

\begin{example}{烧水热力学}
\begin{figure}[ht]
\centering
\includegraphics[width=7cm]{./figures/Hess_3.png}
\caption{烧水} \label{Hess_fig3}
\end{figure}
一大气压下,使用500K的恒温烤炉加热一杯1mol的水,使其从室温273K加热至沸点373K(先不考虑相变).求该过程中水的熵变.

这个问题粗看无从下手.热力学第二定律\upref{Td2Law}阐明的$dS=\delta q/T$只适用于可逆(准静态)过程.但是,视水为系统,根据系统状态量的改变与路径无关的特性,熵变应只和水的初始温度与末温度有关\footnote{别忘了,压力恒为一个大气压}.这启发我们设计一条可逆(准静态)路径,即令热水器的温度与水的温度时时相同.

\begin{figure}[ht]
\centering
\includegraphics[width=7cm]{./figures/Hess_4.png}
\caption{准静态烧水} \label{Hess_fig4}
\end{figure}
现在,我们可以运用热力学第二定律计算熵变了.
$$\Delta S=\int \delta q/T$$

如果忽略热容随温度的变化,则 
$$\Delta S=C_p \ln \frac{T}{T_0}$$

那么,我们可以计算这个系统的吸热吗?原则上,Q是过程量,不能设计路径;但这个问题有点特殊.由于这是一个等压过程,\textsl{刚好}有$Q=\Delta H$. $\Delta H$是很典型的可设计路径求解的量,$\Delta H = \int C_p \dd T$.
\end{example}
