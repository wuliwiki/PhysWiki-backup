% 北京大学 2000 年 考研 量子力学
% license Usr
% type Note

\textbf{声明}:“该内容来源于网络公开资料,不保证真实性,如有侵权请联系管理员”

1. (20分) 质量为的粒子, 在位势
$$V(x) = a \delta (x) + V'(x), (a < 0)~$$
$$V'(x) = \begin{cases} 0 & x < 0, \\\\ V_0 & x > 0,\quad (V_0 > 0)\end{cases}~$$
中运动,\\
(a) 试给出在束缚态的条件,并给出其能量本征值和相应的本征函数。\\
(b) 给出粒子处于$ x > 0 $区域的几率。它是大于1/2, 还是小于1/2, 为什么?\\

2. (10分) 态$|\alpha\rangle$和$|\beta\rangle$是复原子的定态 (电子和质子的相互作用为库仑作用, 并计及电子的自旋-轨道耦合项)\\
(a)] 给出$|\alpha\rangle$和$|\beta\rangle$态的守恒量完全集。\\
(b)] 若$\langle \beta | f({r}) \hat{\mathbf{s}} \cdot \vec{r} | \alpha \rangle \neq 0$, 则$|\alpha\rangle$和$|\beta\rangle$态的哪些量子数可能是不同的, 为什么?\\
(注: $f({r})$是$r$的径向函数, $\hat{s},\vec{r}$为电子的自旋和坐标矢量)\\

3. (16分) 三个自旋为1/2的粒子,在$(s_{1z}, s_{2z})$表象中的表示为$\begin{pmatrix} \alpha_1 \\ \beta_1 \end{pmatrix} \begin{pmatrix} \alpha_2 \\ \beta_2 \end{pmatrix}$, 其中,$\left|\alpha_i\right|^2$是第$i$粒子自旋向上的几率,$\left|\beta_i\right|^2$是第$i$粒子自旋向下的几率。\\\\
(a) 求哈密顿量 
$$\hat{H} = V_0 (\sigma_{1x} \sigma_{2y} - \sigma_{1y} \sigma_{2z})~$$
    的本征值和本征函数 ( $V_0$为一常数)\\\\
(b) 在$t=0$时,体系处于态$\alpha_1 = \beta_2 = 1, \alpha_2 = \beta_1 = 0$,求此时刻发现体系在态$\alpha_1 = \beta_2 = 0, \alpha_2 = \beta_1 = 1$的几率。
    (注:$\sigma_{1x}, \sigma_{1y}$为第一个粒子泡利算符的$x,y$分量)

4.考虑一维谐振子,其哈密顿量
$$\hat{H} = \hbar \omega \left( \hat{a}^\dagger \hat{a} + \frac{1}{2} \right)~$$
    而$[a, a] = [a^\dagger, a^\dagger] = 0, [a, a^\dagger] = 1$。

(a)若$|0\rangle$是归一化的基态矢($(a|0\rangle = 0)$),则第 $n$ 个激发态为
        \[        |n\rangle = N_n (a^\dagger)^n |0\rangle        \]
        求归一化因子 \\(N_n\\);
若外加一微扰,$\hat{H}' = g a^\dagger a a a$,试求第 $n$ 个激发态的能量本征值(准至 \g\\) 一级)。
    \\end{enumerate}
\\end{enumerate}
