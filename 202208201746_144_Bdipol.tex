% 磁偶极矩
% 磁偶极矩|静磁学|电流|磁场

\begin{issues}
\issueDraft
\end{issues}
完全类似于电偶极子\upref{eleDpl},我们可以定义磁偶极子.在经典电动力学中,磁偶极子可以理解为一个带恒定电流的环形回路.
\begin{figure}[ht]
\centering
\includegraphics[width=5cm]{./figures/Bdipol_1.png}
\caption{磁偶极子示意图} \label{Bdipol_fig1}
\end{figure}

定义他的磁偶极矩$\bvec m$为
\begin{equation}
\bvec m = I \bvec s
\end{equation}
其中$I$是这个回路的电流,$\bvec s = ab\bvec n$是面积元矢量,数值上等于回路的面积,方向由电流方向即右手法则确定.

\begin{equation}
\bvec B(\bvec r) = \frac{\mu_0}{4\pi} \frac{1}{r^3} [3(\bvec m \vdot \uvec r) \uvec r - \bvec m]
\end{equation}
