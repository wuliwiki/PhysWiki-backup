% 旋转矩阵的导数
% 矩阵|导数|瞬时转轴|圆周运动|旋转矩阵

\pentry{刚体的瞬时转轴、角速度的矢量相加\upref{InsAx}, 叉乘的矩阵形式\upref{CrosMt}}

我们来证明含参数 $t$ 的旋转矩阵\footnote{$t$ 不一定表示时间, 可以是任意参数.} $\mat R(t)$ 的导数 $\dot{\mat R}$ 满足
\begin{equation}\label{RotDer_eq4}
\dot{\mat R} = \mat \Omega \mat R
\end{equation}
其中
\begin{equation}
\mat \Omega = \pmat{
0 & -\omega_z & \omega_y\\
\omega_z & 0 & -\omega_x\\
-\omega_y & \omega_x & 0
}
\end{equation}
其中 $(\omega_x, \omega_y, \omega_z)$ 是旋转的瞬时转轴的坐标\footnote{如果使用主动理解, 即旋转矩阵将同一个坐标系中的一个矢量旋转后得到另一个矢量. 如果使用被动理解, 即旋转矩阵将同一个矢量在两个坐标系之间进行变换, 则矢量 $(\omega_x, \omega_y, \omega_z)$ 取反方向即可.}.% 未完成: 主动理解和被动理解哪里有提?

\subsection{证明}
我们可以把一个任意的不随 $t$ 变化的列矢量 $\bvec r_0$ 用旋转矩阵\upref{Rot3D}旋转得
\begin{equation}\label{RotDer_eq1}
\bvec r(t) = \mat R(t) \bvec r_0
\end{equation}
关于 $t$ 求导得
\begin{equation}\label{RotDer_eq3}
\dot{\bvec r} = \dot{\mat R} \bvec r_0
\end{equation}
令列矢量 $\bvec \omega = (\omega_x, \omega_y, \omega_z)\Tr$, 由\autoref{CMVD_eq5}~\upref{CMVD} 得
\begin{equation}\label{RotDer_eq5}
\dot{\bvec r} = \bvec\omega \cross \bvec r
\end{equation}
将叉乘表示为矩阵乘法(\autoref{CrosMt_eq3}~\upref{CrosMt})得
\begin{equation}\label{RotDer_eq2}
\dot{\bvec r} = \mat\Omega \bvec r
\end{equation}
其中
\begin{equation}
\mat \Omega = \pmat{
0 & -\omega_z & \omega_y\\
\omega_z & 0 & -\omega_x\\
-\omega_y & \omega_x & 0
}\end{equation}
将\autoref{RotDer_eq1} 代入\autoref{RotDer_eq2} 右边, 再对比\autoref{RotDer_eq3}, 可得\autoref{RotDer_eq4}. 证毕.


=== 回收 ===

另外, $\bvec \omega$ 和 $\bvec R$ 之间的关系就像速度和位移的关系, 假设体坐标系中固定在刚体上的任意一点坐标为 $\bvec r$ (不随时间变化), 变换到实验室坐标系中为 $\mat R \bvec r$. 对时间求导得该点在实验室坐标系的速度为
\begin{equation}
\bvec v = \dv{\mat R}{t} \bvec r
\end{equation}
而角速度和速度之间有 $\bvec v = \bvec \omega \cross (\mat R \bvec r)$(??). 我们可以把叉乘用矩阵乘法表示??为
\begin{equation}
\bvec v = \mat\Omega \mat R \bvec r
\end{equation}
$\mat\Omega$ (??)是一个反对称矩阵, 即
\begin{equation}
\mat \Omega\Tr = -\mat \Omega
\end{equation}
由于 $\bvec r$ 是任意的, 对比?? 和?? 得??.
