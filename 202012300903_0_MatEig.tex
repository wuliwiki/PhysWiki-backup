% 矩阵的本征方程
% keys 矩阵|本征问题|本征矢|本征值|线性方程组|对角线

\begin{issues}
\issueOther{简并子空间应该写到哪里?}
\end{issues}

\pentry{线性方程组与矢量空间\upref{LinEq}, 子空间\upref{SubSpc}}

若已知矩阵 $\mat A$, 我们把线性方程组
\begin{equation}\label{MatEig_eq1}
\mat A \bvec v = \lambda \bvec v
\end{equation}
称为矩阵 $\mat A$ 的\textbf{本征方程}. 式中 $\mat A$ 是已知的, 而 $\lambda$ 和 $\bvec v$ 是未知的. 显然, 当 $\bvec v = \bvec 0$ 时方程恒成立, 所以我们通常只对非零解感兴趣. 也就是说, 我们希望找到一些非零矢量 $\bvec v$, 使得矩阵 $\mat A$ 乘以该矢量以后方向不变\footnote{“方向” 只是从几何矢量\upref{GVec}中沿用过来的一个习惯说法, 注意\autoref{MatEig_eq1} 中的所有量都可以是复数. 两个矢量方向相同意味着一个矢量乘以标量(包括复数)可以得到另一个.}. 对于每个这样的矢量, 我们用一个标量 $\lambda$ 来描述其模长的改变. 我们把这些矢量叫做\textbf{本征矢量(eigen vector)}, 把对应的 $\lambda$ 叫做\textbf{本征值(eigen value)}.

若令 $\mat I$ 为 $N\times N$ 的单位矩阵\footnote{即对角线上的元为 1, 其他元为 0, 见“矩阵\upref{Mat}”}, 则本征方程本质上是一个齐次方程组
\begin{equation}\label{MatEig_eq2}
(\mat A - \lambda\mat I)\bvec v = \bvec 0
\end{equation}
括号中的矩阵相当于把矩阵 $\mat A$ 的对角线上的元都减去 $\lambda$ 得到的方阵. 要确保方程有非零解, 只需令系数矩阵 $\mat A - \lambda\mat I$ 不是满秩的, 即行列式为零
\begin{equation}
\abs{\mat A - \lambda\mat I} = 0
\end{equation}
这是一个关于 $\lambda$ 的 $N$ 阶多项式, 必有 $N$ 个复数根(包括重根),% 链接未完成
记为 $\lambda_i$ ($i = 1, 2\dots N$). 将它们依次代入\autoref{MatEig_eq2}, 就可以分别解出对应的本征矢. 考虑到\autoref{MatEig_eq2} 是一个齐次方程, 所以 $\mat A - \lambda_i\mat I$ 的零空间中所有矢量都是本征矢, 且零空间至少是一维的. 我们把这个空间叫做 $\lambda_i$ 的\textbf{本征矢空间}, 是 $\bvec v$ 所在的矢量空间的子空间.

令 $\lambda_i$ 的本征矢空间的维度是 $n_i$, 若 $n_i = 1$, 我们说 $\lambda_i$ 是\textbf{非简并(non-degenerate)}的, 若 $n_i > 1$ 就说 $\lambda_i$ 是 $n_i$ 重\textbf{简并(degenerate)}的, 把 $n_i$ 叫做\textbf{简并数(degeneracy)}.

% 未完成: 举例
