% 极限的运算法则
% license Usr
% type Tutor

\subsection{极限的基本性质与运算法则}
下面列出一些函数极限的定理, 从直觉上来看它们是显然的, 证明略, 感兴趣的读者可以尝试自己证。
\begin{theorem}{极限的四则运算}\label{the_LimOp_1}
若两个函数分别存在极限 $\lim\limits_{x\to a} f(x)$ 和 $\lim\limits_{x\to a} g(x)$ ($a$ 可取 $\pm \infty$), 那么有
\begin{equation}
\lim_{x\to a} [f(x) \pm g(x)] = \lim_{x\to a}f(x) \pm  \lim_{x\to a} g(x)~,
\end{equation}
\begin{equation}
\lim_{x\to a} [f(x) g(x)] = \lim_{x\to a}f(x) \lim_{x\to a} g(x)~,
\end{equation}
\begin{equation}
\lim_{x\to a} [f(x)/g(x)] = \lim_{x\to a}f(x)/\lim_{x\to a} g(x) \qquad (\lim_{x\to a} g(x) \ne 0)~.
\end{equation}
注意,可以四则运算的前提是参与运算的各个极限均存在。
\end{theorem}

\begin{theorem}{局部保号性、保序性}
%我对着我的笔记写的,定理的表述可能有一点不严谨;不过对于“入门”启发,应该是够了(
局部保号性:
\begin{equation}
\lim_{x\to x_0}f(x)>0\Rightarrow \exists \mathring{U} (x_0), \forall x \in \mathring{U} (x_0), f(x)>0~.
\end{equation}

局部保序性:
\begin{equation}
\lim_{x\to x_0}f(x)=A>0 \Rightarrow \exists \mathring{U} (x_0), \forall x \in \mathring{U} (x_0), f(x)>A/2~.
\end{equation}

% Giacomo: 要不要把 $A/2$ 改成 $A - \epsilon$?

$\mathring{U} (x_0)$指$x_0$附近的一个小区间,但不包括$x_0$自身,也称去心区间。\footnote{“去心”这个词让我想到“比干挖心”的传说故事}

这是一组简单的、但却\textbf{有点难以理解}的结论。在处理一些刁钻的问题时,局部保号、保序性偶尔会派上用场。通俗地说,这意味着若$\lim\limits_{x\to x_0}f(x)=A$,则$x$在$x_0$附近时,$f(x)$的函数值也会收缩到$A$附近。

\begin{figure}[ht]
\centering
\includegraphics[width=10cm]{./figures/94672fbdd10053ba.pdf}
\caption{在$(x_0-\delta, x_0+\delta)$区间内,$f(x)>0$. 仿自\cite{Thomas}} \label{fig_FunLim_9}
\end{figure}
局部保号性的一个幼稚“证明”:如\autoref{fig_FunLim_9} 所示(其实就是\autoref{fig_FunLim_8}),我们总能取一个$\varepsilon_1 \in (0,A)$,极限的定义保证了我们总能找到对应的$\delta_1$。显然,在$(x_0-\delta_1, x_0+\delta_1)$这个小去心区间\footnote{严格来说,或许应该写为$(x_0-\delta_1,x_0)\cup(x_0, x_0+\delta_1)$,不过在正文中这么写实在太繁琐了,也不利于把握重点}(即$\mathring{U} ({x_0})$)内,有$f(x)>0$。

同理,将选取$\varepsilon$的区间改为$(0,A/2)$,我们就能找到相应的、使$f(x)>A/2$的区间,即说明了局部保序性。原则上,由此可导出更广义的局部保序性,即总存在$f(x)>A/3,A/4,A/5,...$的区间。

\end{theorem}


\addTODO{如果在 $x_0$ 附近(不包括) $f(x) > 0$,那么 $\lim_{x \to x_0} f(x) \geq 0$}
