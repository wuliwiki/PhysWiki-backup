% 数的观念
% license Usr
% type Art

(本文根据 CC-BY 协议转载自季燕江的《量子序曲》, 进行了重新排版和少量修改)

\subsection{万物皆数}

数是抽象的观念,它并不在自然中对我们现身。当我们说水的时候,我们知道水是什么,当我们说红的时候也知道红对我们意味着什么,但数是什么呢?当我们说一的时候,是什么意思呢?

“一”是个动作,是我用手指向某物,但我为何要做这动作呢?我是做给他者(另外一个我)看的,“瞧,此物”。这就意味着一种整全性,我无法用手指同时指向0-1之间所有的实数,我指向的是一个整全的对象,我并没有指向部分,除非你要求我澄清。

弗雷格在《算术基础》中问1+1=2意味什么?它不可能是两个月亮相加。世界上也不可能找到两个相同的月亮。

那么1、和1是什么呢?

1是用手指,1是再用手指,1就是count这个动作。

据说在某些原始部落,人们没有超过数字3的概念,他们没法像我们这样数数,他们只能这样:

“1,2,3,3,3,……”

但,假如我们去和这些原始部落中的人做交换,我们能糊弄他们吗?

我们用玻璃珠和他们换珍珠,我们能用3颗玻璃珠,换来比如100颗珍珠吗?

这当然不可能。原始人只是缺乏对数字的命名,没有像我们那样定下加法口诀表,但这并不意味着他们没有算术技术。

比如他们可以用一个玻璃珠和一个珍珠配对,当1 vs 1都配好对了,我们拿走珍珠,他们拿走玻璃珠。

当然这是极其原始的算术技术,但把石头、米粒或小木棍摆放在地上确实给数字一个直观的印象。

必须用不同的东西count,或者空间分离,或者在时间的序列上间隔。count就是数数,数数是用不同的东西数,把石头一个个摆开就是在数数,虽然没有命名,但已经在数了。

1+1是count, count 这个动作,我们对count, count的命名是2,这就是1+1=2. 而count, count, count 我们命名为3。这背后的基础是生活,我们过某种合作的生活导致我们发明了count, count, ... 这种计数技术。它可能用于交换,拿走一个果子,我们就摆一块石头,再拿走一个,再摆一块,....

\subsubsection{自然数}

小孩学数学的第一步是背诵,1, 2, 3, 4,…这就是对count, count,...的命名。n+1= n+1 表示count n次后,再count一次。n+1对n而言是唯一确定的,而且n+1不同于之前任何一个count, 这里我们需要不同的命名,如此定义的对象将像“不闭合的珠链”一样无尽伸展出去。

最简单的数是自然数,0,1,2,3……,从学习的角度,我们是这么掌握自然数的:

1.首先是背诵,先是背熟10以内的自然数:

\begin{equation}
\text{0,1,2,3,4,5,6,7,8,9,10}~
\end{equation}

这可以借助10个手指头。

然后还是背诵,背熟20以内的:

\begin{equation}
\text{……11,12,13,14,15,16,17,18,19,20}~
\end{equation}

这里已经涉及个位和10位的问题了,好在我们有科学计数法,大于10但小于20的自然数可写为$1n$,这里$n$是0到9中间的一个数:

\begin{equation}
1n  = 1 \times 10 + n~
\end{equation}

严格来说,我们现在还没有定义加法,从0,1,2,3……开始到18,19,20的罗列仅仅是个罗列,是个有方向的一一罗列,好比是20来个好伙伴手拉手列成一队,从左到右,我一一清点他们,记熟他们的名字和位置。

进一步地,我们还可以背熟0到100的数字,这其实就是对一个链式结构的命名,命名并把它们记熟。这个过程是没头的,中国古代有五数,“一、十、百、千、万”,每逢10进1,一而十,十而百……

有“一、十、百、千、万”,日常生活中碰到的数字足够表达了。

2.把从0到20的自然数背熟,就是在把握里面的数学结构,我们也可以通过定义运算把这里面的数学结构说清楚。

运算就是对数字的操作,比如我们可以定义加法:

$m + n$,先从0开始数,数到$m$,然后$m+1$,$m+2$,……,一直到$m+n$,因为数字已经背熟了,我们发现$m+n$就是我们背熟的数字序列中的某个数。

这个就是所谓“掰手指头”,由三开始再掰两个就是五,记作:

\begin{equation}
3+2 = 5~
\end{equation}

但实际上算术不是这么学的,我们依然是背诵,画出加法表,然后把它们背下来。

\begin{figure}[ht]
\centering
\includegraphics[width=6cm]{./figures/60d67a1288bd3ab1.png}
\caption{加法表就是对加法的定义} \label{fig_QMPre6_1}
\end{figure}

加法满足这样的一些性质:

\begin{itemize}
    \item [(a)] 交换律: (3 + 2 = 2 + 3)
    \item [(b)] 结合律: ((3 + 2) + 1 = 3 + (2 + 1))
\end{itemize}

对这些性质的理解都可以回到链式结构:

\begin{equation}
0, 1, 2, 3, 4, 5, 6, ..., n, n+1, ...~
\end{equation}

这样的结构里,但这种结构并非是唯一可能的代数结构。

比如,我们可以设想一个环形的结构:

\begin{equation}
0, 1, 2, 3, 4, 5, 6, 7, 8, 9, 0, 1, 2, ...~
\end{equation}

即每经过周期10,数字就会重新开始,在这个结构下也可以定义加法,在局部它拥有和链式结构一样的算术,比如:

\begin{equation}
3 + 2 = 5~
\end{equation}

但对足够大数字的加法,则要考虑以10为周期这样一个特征:

\begin{align}
5+5 & =  0\\
5+6 & =  1 ~
\end{align}

需要注意的是在这样一个算术系统里,10和11都是不存在的,$5+5$只能是0。

交换律也并非在所有的运算中都成立,比如对物体在空间中的转动,如果我们把围绕不同转轴的转动看作是被运算、操作的对象的话,交换律就不再满足。


\subsubsection{分数}

我们可以用两个数($m,n$)表示分数:

\begin{equation}
\frac{n}{m}~
\end{equation}

首先把1(这个1可以是单位长度、单位面积,也可以是单位重量等)分成$m$份,然后再从这$m$份中拣选出$n$份来,这意味着$m > n$。

这个动作和分配有关,比如井田制中,公田占$\frac{1}{9}$,剩下的$\frac{8}{9}$分给8家是私田,每家1份,耕作的时候要先8家一起耕公田,公田忙完后才能治私田。

当然也可以$m < n$,只要分母$m$不为0,分数的定义就是有意义的。它表示对$\frac{1}{m}$累积了$n$次。

假设我们有1把尺子,比如汉尺的长度是$23.1$厘米,然后去量尖碑影子的长度,在经历了几个整数的长度后,也许还剩一点,这剩下的一点不足1尺,同时又不可以忽略,这时我们就可以用分数的概念了,把1尺分成$m$份,然后拣选出$n$份来。

比如中国古代用分数来表示圆周率$\pi$\footnote{圆周率被定义为圆周长$L$和圆直径$D$的比值:$\pi = \frac{L}{D}$},比如祖冲之用分数$\frac{22}{7}$来表示对圆周率的粗略近似,而用$\frac{355}{113}$来表示对圆周率的精确近似。

分数也叫有理数,是不是所有的数都是有理数呢?或是不是所有的数都可表示为两个数字($m,n$)的组合$\frac{n}{m}$呢?

\subsubsection{$\sqrt{2}$}

%%%%%%%%

古希腊的毕达哥拉斯坚信,宇宙万物的形状都可表示为数,这里的数指的是整数或整数的组合(比如分数)。这就是所谓“万物皆数”。但很快,毕达哥拉斯本人或毕达哥拉斯学派中的某个弟子就构造出了一个反例。这个反例也利用到了毕达哥拉斯的另外一项成就,毕达哥拉斯定理。

毕达哥拉斯定理说:

\begin{equation}
\text{直角三角形两直角边的平方和等于直角三角形斜边的平方。}~
\end{equation}

\begin{equation}
a^2 + b^2 = c^2~
\end{equation}

现在考虑一个等腰直角三角形,直角边的长度是1,斜边的长度是多少呢?
