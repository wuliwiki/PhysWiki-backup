% 高阶线性微分方程的降阶
% 线性微分方程组|常微分方程|ode|ordinary differential equation

\pentry{一阶常系数线性微分方程组(常微分方程)\upref{ODEb3},拉普拉斯变换与常系数线性微分方程\upref{ODELap}}

在\textbf{拉普拉斯变换与常系数线性微分方程}\upref{ODELap}中,我们讨论了高阶常系数线性微分方程的解法,通过拉普拉斯变换,将高阶微分方程化为代数方程进行求解.但是很多时候,单个的高阶微分方程也是可以化为多个一阶方程的,从而可以应用\textbf{一阶常系数线性微分方程组(常微分方程)}\upref{ODEb3}中的方法来进行求解.

一般地,$n$阶微分方程可以写为
\begin{equation}\label{ODEb4_eq1}
F(t, x(t), \frac{\dd}{\dd t}x(t), \cdots, \frac{\mathrm{d}^n}{\dd t^n}x(t))=0
\end{equation}
接下来,我们介绍几种可以降阶的方程.

\begin{example}{}
如果\autoref{ODEb4_eq1} 不显含$x$的$k$次及以下导函数,即方程形式为
\begin{equation}
F(t, \frac{\mathrm{d}^{k+1}}{\dd t^n}x(t), \frac{\dd}{\dd t}x(t), \cdots, \frac{\mathrm{d}^n}{\dd t^n}x(t))=0
\end{equation}
\end{example}





















