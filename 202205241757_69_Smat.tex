% 散射理论与S矩阵
% keys 相互作用场论|散射|S矩阵|费曼散射振幅
\pentry{相互作用表象\upref{Ipic}}

现在我们在相互作用表象中讨论形式散射理论,方便起见,设参考时间 $t_0=0$.在一般的散射问题中,入射粒子和出射粒子实际上是一系列波包,当 $t\rightarrow \pm\infty$ 时,我们假设不同的粒子间相互远离而导致它们之间的相互作用趋于 $0$.\footnote{在量子场论中问题会更加复杂,即使一个粒子与其他粒子完全远离,它与自己仍然存在自相互作用.只要允许相应的虚过程,单粒子与自身的相互作用是无法通过“绝热的方法”消去的.我们在后面会进一步讨论这个问题.}

设 $|\phi_{in/out,\alpha}\rangle=|\psi_\alpha^{\pm}(\mp \infty)\rangle^I$ 是一系列自由单粒子态的直积,分别代表入射粒子和出射粒子,$E_\alpha$ 是它们对应的自由哈密顿量 $H_0$ 的本征值,并且所有的 $|\phi_\alpha\rangle$ 构成 Hilbert 空间的一组完备基矢.在相互作用表象下 $|\phi_{in/out,\alpha}\rangle$ 随时间分别演化到  $|\psi_\alpha^{(\pm)}\rangle=|\psi_\alpha^{(\pm)}(0)\rangle^I$.
我们称 $|\psi_\alpha^{+}\rangle$ 为入态,$|\psi_\alpha^{-}\rangle$ 为出态. 它们之间有以下的关系:

\begin{equation}
\begin{aligned}
&|\psi_\alpha^{+}(-\infty)\rangle^I=|\phi_{in,\alpha}\rangle,|\psi_\alpha^{+}(0)\rangle^I=|\psi_\alpha^{+}\rangle\\
&|\psi_\alpha^{-}(+\infty)\rangle^I=|\phi_{out,\alpha}\rangle,|\psi_\alpha^{-}(0)\rangle^I=|\psi_\alpha^{-}\rangle\\
&|\psi_\alpha^+\rangle=U_I(0,-\infty)|\phi_{in,\alpha}\rangle = \Omega^{(+)} |\phi_{in,\alpha}\rangle,
|\psi_\alpha^-\rangle=U_I(0,+\infty)|\phi_{out,\alpha} \rangle = \Omega^{(-)} |\phi_{out,\alpha}\rangle
\end{aligned}
\end{equation}