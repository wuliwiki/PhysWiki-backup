% 积分中值定理
% 中值定理

\begin{issues}
\issueDraft
\end{issues}

积分中值定理可以将积分号去掉,或者将复杂的被积函数化为相对简单的被积函数,从而使问题简化,其应用相当的广泛。

\subsection{积分中值定理}
\begin{theorem}{积分中值定理}
如果 $f(x)$ 在 $[a,b]$ 上可积,并且在该区间上恒有$m<f(a)<M$
则
\begin{equation}
\int_a^b f(x)\dd x=\mu(b-a),
\end{equation}
其中 $m\leq\mu\leq M.$
\end{theorem}

\textbf{证明:}
设 $a<b$,则由\autoref{the_DIntP_3}~\upref{DIntP},
\begin{equation}
m(b-a)\leq\int_a^b f(x)\dd x\leq M(b-a)
\end{equation}
故有
\begin{equation}
m\leq\frac{1}{b-a}\int_a^b f(x)\dd x\leq M
\end{equation}
令
\begin{equation}
\frac{1}{b-a}\int_a^b f(x)\dd x=\mu
\end{equation}
即可得所需求等式。

\textbf{证毕!}

\begin{theorem}{推广积分中值定理}
设 $f(x),给$ 在 $[a,b]$ 上可积,并且在该区间上恒有$m<f(a)<M$
则
\end{theorem}