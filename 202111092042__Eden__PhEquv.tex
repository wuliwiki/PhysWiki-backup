% 相变平衡条件
% 相平衡|单元系|多元系
\pentry{热动平衡判据\upref{equcri},吉布斯自由能\upref{GibbsG}}
%需要增加一个介绍相变的词条%
\subsection{单元系相变平衡条件}
单元系的意思是系统只有一种化学组分,但有多个相.例如冰水混合物是一元二相系,水与水蒸气组成的体系是一元二相系.水的三相图($P$-$V$ 图)中,有最主要的三条线区分开三个区域——固态、液态、气态(见\autoref{PhEquv_fig1},图片来源自维基百科\href{https://en.wikipedia.org/wiki/Triple_point}{相关页面}.).
\begin{figure}[ht]
\centering
\includegraphics[width=14cm]{./figures/PhEquv_1.png}
\caption{水的相图}} \label{PhEquv_fig1}
\end{figure}

两相分界线处是两相共存的平衡态.对于单元单相系,相变平衡条件有
\begin{equation}
p^{\alpha}=p^{\beta}, 
T^{\alpha}=T^{\beta}
\end{equation}
同一种化学组分的摩尔吉布斯函数是压强和体积的函数.
\subsection{多元系相变平衡条件}