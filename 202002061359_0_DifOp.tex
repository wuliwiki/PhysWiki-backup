% 算符

\textbf{算符}(也叫\textbf{算子})可以理解为 “函数的函数”, 即一个函数经过算符作用可以得到另一个算符. 例如将一个函数乘以一个数 $\lambda$ 或另一个函数, 又例如对一个函数求或偏导, 又或是依次进行若干个不同的操作.

我们可以更形式化一些, 将算符与被算符作用的函数分离开并用一个符号表示, 例如另
\begin{equation}
\Q A = \qty(\dv{x} + 1)
\end{equation}
\begin{equation}
\Q A f(x) = \qty(\dv{x} + 1)f(x) = \dv{f}{x} + f(x)
\end{equation}

\begin{equation}
\qty(\dv{x} + 1)^2 = \dv[2]{x} + 2\dv{x} + 1
\end{equation}

\begin{equation}
\qty(\pdv{x} + \pdv{y})^2 = \pdv[2]{x} + \pdv[2]{y} + 2\pdv{}{x}{y}
\end{equation}

\begin{equation}

\end{equation}