% 静电场与静磁场(摘要)

\begin{issues}
\issueDraft
\end{issues}


\subsection{电荷守恒}

\subsection{静电场与静磁场}
\begin{table}[ht]
\centering
\caption{静电场与静磁场}\label{tab_estfid1}
\begin{tabular}{|c|c|c|}
\hline
* & 电场 $\bvec E$ \upref{Efield} & 磁场 $\bvec B$\upref{MagneF} \\
\hline
场源 & 电荷 $q$ \upref{Efield}& 电流(运动的电荷) $\bvec j$ \upref{I}\\
\hline
场源产生的场 & $$\dd \bvec E (\bvec r) = \frac{1}{4 \pi \epsilon_0} \frac{\dd q}{R^2} \bvec {\hat R}~ $$
其中 $\bvec r$是场点,$\bvec r'$是场源位置,$\bvec R = \bvec r - \bvec r'$是场源指向场点的矢量,$\bvec{\hat R}$是相应的单位矢量。\upref{Efield}
& $$\dd \bvec B(\bvec r) = \frac{\mu_0}{4\pi} \frac{I \dd{\bvec r'} \cross \uvec R}{R^2}~$$ 比奥萨伐尔定律\upref{BioSav} \footnote{不同于静电场中可以任意摆放电荷,在静磁场中我们不能“任意摆放”电流。假如设计的“电路”不成环,那么根据电荷守恒\upref{ChgCsv},区域内的电荷密度必定变化,从而不再是静场问题。这也是为什么这个公式实际上不能准确描述“单个运动电荷的磁场”。}\\
\hline
散度方程 & 
$$\oint \bvec E \vdot \dd{\bvec s} = \frac{1}{\epsilon_0}\int \rho \dd{V} = \frac{Q}{\epsilon_0}~$$
$$\div \bvec E = \frac{\rho}{\epsilon_0}~$$ 电场的高斯定律\upref{EGauss}
&
$$\oint \bvec B \vdot \dd{\bvec s} = 0~$$
$$\div \bvec B = 0~$$ 磁场的高斯定律\upref{MagGau}\\
\hline
旋度方程 & 
$$ \oint \bvec E \vdot \dd{\bvec l} = \bvec 0~$$
$$ \curl \bvec E = \bvec 0 ~$$ 静电场的环路定理\upref{ELECLD}
 &
$$\oint \bvec B \vdot \dd{\bvec l} = \mu_0 \int \bvec j \vdot \dd{\bvec s}=\mu_0 I ~$$ 
$$\curl \bvec B = \mu_0 \bvec j~$$ 静磁场的环路定理(专业术语:安培环路定律) \upref{AmpLaw}\\
\hline
势 & $$\varphi~$$ $$\bvec E = -\grad \varphi~$$ 电势\upref{QEng}& $$\bvec A~$$ $$\bvec B = \curl \bvec A~$$ 磁矢势\upref{BvecA}\\
\hline 
\end{tabular}
\end{table}
