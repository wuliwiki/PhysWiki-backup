% 拉回映射
% license Xiao
% type Tutor

\pentry{伴随映射\nref{nod_AdjMap}}{nod_712c}

\begin{issues}
\issueDraft 
1.预备知识需新增余切向量的外积。2.最后一条定理待证。
\end{issues}

设$f:M\rightarrow N$是光滑映射,则可以定义切空间之间前推映射:
\begin{equation}
f_*: T_p M\rightarrow T_{f(p)}N~,
\end{equation}
切空间是线性空间,因此可以诱导其对偶空间上的伴随映射(或称对偶映射)。简写该伴随映射为$f^*$:
\begin{equation}
f^*\equiv(f_*)^*:\,T^*_f(p)N\rightarrow T_p^*M~,
\end{equation}
使得其满足伴随映射的基本定义,即对于任意$X\in T_p M,\eta\in T^*_{f(p)}N$有:
\begin{equation}
f^*(\eta)X=f_*(X)\eta~.
\end{equation}
一般称如上定义的伴随映射为伴随$f$的\textbf{拉回(pull-back)},以示其把$N$上的余切向量“拉回”到$M$上这一特点。

既然拉回映射是把余切向量映射为余切向量,自然也可以拉回$N$上的余切向量场。
\begin{lemma}{}\label{lem_PullBk_1}
设$g\in C^{\infty }N,\sigma$是$N$上的光滑余切场,则有
\begin{enumerate}
\item $f^*(dg)=d(g\circ f).$
\item $f^*(g\sigma)=(g\circ f)f^*\sigma$
\end{enumerate}
\end{lemma}
\textbf{proof.}

设任意$X\in T_pM$,由拉回定义得:$f^*(dg)X=dgf_*(X)$。又因为$f_*(X)$是$N$上的切向量,则由函数微分及前推的定义得:$dgf_*(X)=f_*(X)g=X(g\circ f)=d(g\circ f)X$,第一条得证;

同理可证第二条如下:
\begin{equation}
\begin{aligned}
f^*(g\sigma)|_pX&=(g\sigma)|_{f(p)}f_*(X)\\
&=g|_{f(p)}f^*(\sigma)X\\
&=(g\circ f)f^*(\sigma)X
\end{aligned}
~,
\end{equation}

\begin{theorem}{}
伴随\textbf{光滑映射}的拉回映射把\textbf{光滑余切场}映射为光滑余切场。
\end{theorem}
\textbf{proof.}

设$f:M\rightarrow N$为光滑映射,$\sigma$为$N$上的光滑余切场,$\{dy^i\}$为$N$上的局部余切基\footnote{$y^i:V\subset N\rightarrow R$是对应坐标卡的坐标函数}。则可以将光滑余切场表示为$\sigma=\sum_i\sigma_idy^i $,分量均为光滑函数。
由\autoref{lem_PullBk_1} 得:
\begin{equation}
f^*(\sigma)=f^*(\sum_i\sigma_idy^i)=\sum_i(\sigma_i\circ f)f^*(dy^i)=\sum_i(\sigma_i\circ f)d(y^i\circ f)~,
\end{equation}
因为光滑映射的复合是光滑的,所以$f^*(\sigma)$是光滑的,得证。
\subsubsection{张量场的拉回}
类比余切向量场,我们可以定义张量场的拉回。如果流形间的映射和张量场都是光滑的,通过对应的拉回映射,我们可以得到另一个光滑张量场。
\begin{definition}{}
设$f:M\rightarrow N$为光滑映射,$\sigma$是$N$上的$k$阶光滑张量场。定义$f^*\sigma$使得:
\begin{equation}
f^*\sigma(X_1,X_2...X_k)=\sigma|_{f(p)}(f_*X_1,f_*X_2...,f_*X_k)~,
\end{equation}
称$f^*\sigma$为$\sigma$的拉回。
\end{definition}
\begin{exercise}{}
设$F:M\rightarrow N$和$G:N\rightarrow P$都是光滑映射,$\sigma,\tau$分别是$N$上的$k$阶,$l$阶光滑张量场。证明下述结论:
\begin{enumerate}
\item 拉回映射是$\mathcal R,$
\end{enumerate}
\end{exercise}
\subsubsection{微分形式的拉回}
作为交错张量,微分形式也是可以被拉回的。
\begin{definition}{}
设$f:M\rightarrow N$是光滑映射,$\omega$是$N$上的$k$次微分形式,则可以定义微分形式的拉回,使之也是微分形式,满足对于任意$M$上的一组切向量$(X_1,X_2...X_k)$有:
\begin{equation}
f^*(\omega)(X_1,X_2...X_k)=\omega(f_*(X_1),f_*(X_2)...f_*(X_k))~.
\end{equation}
\end{definition}
由于微分形式的外积依然是微分形式,因此也可以被“拉回”。
\begin{theorem}{}\label{the_PullBk_1}
若$f:M\rightarrow N$是光滑的,$\omega,\eta$是$N$上的微分形式。那么有:
\begin{enumerate}
\item $f^*(\omega\wedge \eta)=f^*(\omega)\wedge f^*(\eta)$,即保微分形式的外代数同态。
\item 给定一个$N$上的一个局部余切基$\{dy^{i_1}\}$,定义$k$形式$\omega$为$\omega=\sum_I\omega_I dy^{i_1}\wedge dy^{i_2}...\wedge dy^{i_k}$,则有:
\begin{equation}
f^*(\omega)=\sum_I (\omega_I\circ f)d(y^{i_1}\circ f)\wedge...d(y^{i_k}\circ f)~.
\end{equation}
\end{enumerate}
\end{theorem}
\textbf{proof.}

因为楔积是双线性的,因此我们只需要证明第一条对微分形式的基成立即可。设$\omega$是$k$形式,对应的基为$\{\varepsilon^I\},I=\{i_1,i_2...i_k\}$,$\eta$是$l$形式,对应的基为$\{\varepsilon^J\},J=\{j_1,j_2...j_l\}$。简化左侧表达式为:
\begin{equation}
f^*(\varepsilon^I\wedge \varepsilon^J)(X_{1},X_{2}...X_{k+l})=\varepsilon^I\wedge \varepsilon^J(f_*(X_{1}),f_*(X_{2})...f_*(X_{k+l}))~.
\end{equation}
根据张量的外积定义,可简化右侧的表达式为:
\begin{equation}
\begin{aligned}
f^*(\varepsilon^I)\wedge f^*(\varepsilon^J)(X_{1},X_{2}...X_{k+l})&=\frac{(k+l)!}{k!l!}\opn{Alt}(f^*(\varepsilon^I)\otimes f^*(\varepsilon^J))(X_{1},X_{2}...X_{k+l})\\
&=\frac{1}{k!l!}\opn{sgn}\sigma^I\opn{sgn}\sigma^Jf^*(\varepsilon^I)(X_1,X_2...X_k)f^*(\sigma^J)(X_{k+1},X_{K+2}...X_{k+l})\\
&=\opn{Alt}\varepsilon^I(f_*(X_1),f_*(X_2)...f_*(X_k))\opn{Alt}\varepsilon^J(f_*(X_1),f_*(X_2)...f_*(X_{k+l}))\\
&=\varepsilon^I\wedge\varepsilon^J(f_*(X_{1}),f_*(X_{2})...f_*(X_{k+l}))~.
\end{aligned}
\end{equation}
现证第二条。由第一条得:
\begin{equation}
f^*(\omega)=\sum_If^*(\omega_I dy^{i_1} )\wedge f^*(dy^{i_2})...\wedge f^*(dy^{i_k})~.
\end{equation}

因为$dy^{i_k}$是光滑余切场,由\autoref{lem_PullBk_1} 得:$f^*(\omega_I)dy^{i_1}=(\omega\circ f)d(y^{i_1}\circ f)$
且$f^*(dy^{i_k})=d(y^{i_k}\circ f)$,代入上式即可得证。

\begin{theorem}{}
令$f:M\rightarrow N$是n维微分流形之间的光滑映射。设$\{x_i\}$为$U\subset M$上的局部坐标系,$\{y_i\}$是$V\subset N$上的局部坐标系,$u$是$V$上的光滑函数。那么在$U\cap F^{-1}(V)$上有:
\begin{equation}
f^*\left(u d y^1 \wedge \cdots \wedge d y^n\right)=(u \circ f) \operatorname{det}\left(\frac{\partial f^j}{\partial x^i}\right) d x^1 \wedge \cdots \wedge d x^n~.
\end{equation}
\end{theorem}
\textbf{proof.}

由\autoref{the_PullBk_1} 得:
\begin{equation}\label{eq_PullBk_1}
f^*\left(u d y^1 \wedge \cdots \wedge d y^n\right)=(u \circ f) df^1\wedge df^2\wedge ...\wedge  df^n~,
\end{equation}
又因为$df^i=\frac{\partial y^i}{\partial x^j}dx^j$,结合楔积的反对称性可知:
\begin{equation}
df^1\wedge df^2\wedge ...\wedge  df^n=\operatorname{det}\left(\frac{\partial f^j}{\partial x^i}\right) d x^1 \wedge \cdots \wedge d x^n~.
\end{equation}
代入\autoref{eq_PullBk_1} 得证。