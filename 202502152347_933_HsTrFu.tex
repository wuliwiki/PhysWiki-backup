% 三角函数(高中)
% keys 高中|三角函数
% license Usr
% type Tutor
\pentry{弧度制与任意角\nref{nod_HsAngl},几何与解析几何初步\nref{nod_HsGeBa},函数\nref{nod_functi},函数的性质\nref{nod_HsFunC},导数\nref{nod_HsDerv}}{nod_1829}
\begin{issues}
\issueDraft
\end{issues}

在初中阶段,\textbf{三角函数(trigonometric functions)}通常是在直角三角形的背景下\aref{引入}{eq_HsGeBa_1}的。在学习时,相信读者尚未学习“函数”这一概念,自然仅仅将其作为名称接受,并未意识到它与数学上的函数有何关联。而现在,在了解了函数表示的是输入与输出之间的确定关系的基础上,回顾初中的学习内容,可以发现:无论直角三角形的边长如何变化,只要其中一个锐角固定,三边之间的任意两者的比例始终不变。这一现象正是函数关系的体现——每个角都对应着唯一的比例,这也解释了三角函数名称的由来。

三角函数是一个历史悠久的数学主题,它的独特性在于,它不仅直接关联于几何图形,同时也符合函数的数学定义。这种双重属性使其从一开始就展现出复杂性。然而,复杂性往往伴随着强大的应用能力——事实上,几乎所有的周期函数都可以用三角函数表示。这一特性催生了一门重要的数学分支——\textbf{\enref{傅里叶分析}{FSTri}(Fourier analysis)},它构成了现代电子信息技术、信号处理等领域的基础,建立了时间与频率之间的数学联系。

\subsection{任意角下的三角函数}

最初,三角函数(如 $\sin$、$\cos$ 等)被用于衡量三角形的边长关系,但这种定义仅适用于 $0^\circ$ 到 $90^\circ$ 之间的锐角,因为直角三角形的定义要求其中一角固定为 $90^\circ$,从而限制了角的取值范围。随着角度的推广至任意角,为了突破这一限制,使三角函数适用于任意角。三角函数的定义也需要从直角三角形推广到符合任意角的更一般的形式,以涵盖所有角度。

\begin{figure}[ht]
\centering
\includegraphics[width=10cm]{./figures/fc460d9041b1fc1b.png}
\caption{任意角示意图} \label{fig_HsTrFu_5}
\end{figure}

如\autoref{fig_HsTrFu_5} 所示,既然\aref{任意角}{def_HsAngl_1}可以在坐标系中由角 $\alpha$ 与圆 $O$ 的交点 $P(u,v)$ 确定,自然可以想到,三角函数的定义也应与圆 $O$ 和点 $P(u,v)$ 相关。此外,由于相差 $2\pi$ 或 $360^\circ$ 的任意角具有相同的终边,显然这些角的三角函数值应该相同。换句话说,所有基本三角函数都应是周期函数,并且 $T=2\pi$ 是它们的一个周期。

接下来的问题是如何在单位圆中构造包含角 $\alpha$ 的直角三角形。这样做的好处在于,既能利用原本基于直角三角形的三角函数定义来推导新的定义,又能保证新定义在锐角情况下与原定义的结果一致,从而实现平滑地扩展。

在坐标系中,一个能把已知条件都利用上的直角三角形构造方式是过点 $P$ 向 $x$ 轴作垂线。此时,斜边即圆的半径 $r$,角 $\alpha$ 的对边等于点 $P$ 的纵坐标 $v$,邻边等于其横坐标 $u$。由于这三个量对于任意角 $\alpha$ 都是确定的,因此可以自然地推广初中学过的三角函数定义:

\begin{gather}
\sin\alpha = \frac{v}{r}~.\\
\cos\alpha = \frac{u}{r}~.\\
\tan\alpha = \frac{v}{u}~.
\end{gather}

需要注意,尽管在几何问题中,三角函数通常仍然使用角度作为单位,但从函数的视角来看,为了使三角函数真正成为“函数”,并与当前学习的实数函数体系结合,通常需要引入弧度制,使角成为三角函数的自变量。弧度制的引入使得三角函数的性质更加优良,例如,在弧度制下,$\sin x$ 在 $x=0$ 处的导数为 $1$,这一性质使得微积分中的运算更加简洁自然。

此外,如之前介绍任意角和弧度制时所述,通常对三角函数的研究也放在单位圆中,这样做有一个重要的优势:由于单位圆的半径恒为 $1$,定义中不会因半径的不同而影响正弦和余弦的值(而正切 $\tan\alpha$ 的表达式中本身不涉及 $r$,因此它与半径无直接关系)。这一推广方式不仅使三角函数的定义更加一般化,也更有利于后续的函数分析和计算。

\begin{definition}{三角函数}
取单位圆上一点 $P(u,v)$,令 $OP$ 与 $x$ 轴夹角为$\alpha$ ,则有:
\begin{itemize}
\item \textbf{正弦(sine)}
\begin{equation}
\displaystyle\sin \alpha = v~.
\end{equation}
\item \textbf{余弦(cosine)}
\begin{equation}
\displaystyle\cos \alpha = u~.
\end{equation}
\item \textbf{正切(tangent)}
\begin{equation}
\displaystyle\tan \alpha = \frac{v}{u}~.
\end{equation}
\end{itemize}
\end{definition}

这里的三角函数在定义方式上,与之前学习的幂函数、指数函数和对数函数有所不同。尽管它仍然是一个运算,但这一运算并非直接的代数运算,而是涉及几何图形、长度等概念。在高中阶段,这种区别并不会带来太大影响,但在高等数学中,这样的定义可能会引发某些循环论证的问题,因此在更深入的数学学习中,三角函数的定义会进一步调整。然而,无论如何调整,这些改进仍然是基于最初直角三角形中的三角函数概念,只是为了逻辑上的严密性进行扩展。

事实上,许多数学概念的发展都遵循类似的过程。最初的概念往往是直观且朴素的,但为了扩大其适用范围或增强其严谨性,数学家会对其重新定义。这时,理解概念的关键在于两点:一是掌握最初的直观概念,二是理解重新定义时所采用的思路。任何数学概念的提出都不是凭空产生的,而是基于已有的知识体系逐步演化而来的。

值得注意的是,由于点 $P$ 位于单位圆上,其坐标 $u$ 和 $v$ 可能为负值,因此三角函数的取值也可能为负。例如,第二象限角的余弦值为负,第三象限角的正弦值为负,这超出了原本直角三角形中三角函数仅表示边长比的概念。毕竟,在直角三角形中,边长始终是非负的。然而,这种推广方式仍然是合理的。一方面,在锐角范围内,它与初中阶段的三角函数定义完全一致;另一方面,它揭示了三角函数更本质的数学关系,即不仅仅是边长的比例,而是结合坐标方向的数值表达。实际上,原定义中没有负值的情况只是直角三角形模型下的特殊表现,而推广至任意角后,负值的引入是自然且必要的,使得三角函数的定义更加普遍和完备。

除却以上介绍的三种三角函数,还有三种三角函数在高中不涉及,包括\footnote{其实还包括两个已经被弃用的概念:

\textbf{正矢(versine)}\begin{equation}
\mathrm{versin }\alpha=\displaystyle{r-y\over r}~.
\end{equation}

\textbf{余矢(vercosine)}
\begin{equation}
\mathrm{covers }\alpha=\displaystyle{r-x\over r}~.
\end{equation}
为方便探索,并没有把上面的表达式放在单位圆中。在\autoref{fig_HsTrFu_3} 中,他们分别对应的就是$P$在$y$轴对应的点和$(0,1)$之间以及在$x$轴对应的点和$(1,0)$之间的部分,以及它们在早期的三角函数表中出现,本意是避免因正弦或余弦值过小造成的误差,但随着计算机的发展,这两个三角函数由于与其他三角函数关系不那么密切逐渐被弃用。}:

\begin{definition}{三角函数}
取单位圆上一点 $P(u,v)$,令 $OP$ 与 $x$ 轴夹角为$\alpha$ ,则有:
\begin{itemize}
\item \textbf{余切(cotangent)}
\begin{equation}
\displaystyle\cot \alpha = \frac{u}{v}~.
\end{equation}
\item \textbf{正割(secant)}
\begin{equation}
\displaystyle\sec \alpha = \frac{1}{u}~.
\end{equation}
\item \textbf{余割(secant)}
\begin{equation}
\displaystyle\csc \alpha = \frac{1}{v}~.
\end{equation}
\end{itemize}
\end{definition}

可以看到,这六个三角函数的定义都与圆密切相关,它们既可以看作是角 $\alpha$ 的变化,也可以看作是点 $P$ 在单位圆上的运动。因此,三角函数也被称为 圆函数(circular functions)。在初中阶段,三角函数的背景始终限定在某个直角三角形中,而在扩展后的定义中,可以这样理解:三角函数出现的场景必然涉及某个点在圆上运动,从而使得其定义更加广泛和统一。

可以看到,这六个三角函数的定义都离不开圆,反应的既可以说是角$\alpha$的变化,也可以说是点$P$的运动。因此,三角函数也被称为\textbf{圆函数(circular functions)}。初中时三角函数的背景一定是在某个直角三角形中,现在扩充之后,可以这样说,三角函数出现的场景一定有某一个点在圆上运动。

\begin{figure}[ht]
\centering
\includegraphics[width=14.25cm]{./figures/e9ea4e779f1c67a4.png}
\caption{任意角的三角函数(锐角)} \label{fig_HsTrFu_3}
\end{figure}

\subsection{*同角三角函数的基本关系}

根据现在给出的三角函数定义,很容易得到下面三组同角三角函数的恒等关系:
\begin{itemize}
\item 倒数关系:
\begin{equation}\label{eq_HsTrFu_1}
\begin{split}
\tan \alpha \cdot \cot \alpha = 1\\
\sin \alpha \cdot  \csc \alpha = 1\\
\sec \alpha  \cdot \cos \alpha = 1
\end{split}~.
\end{equation}
\item 乘积关系:
\begin{equation}\label{eq_HsTrFu_2}
\begin{split}
\tan \alpha \cdot\cos \alpha= \sin \alpha\\
\sin \alpha \cdot\cot \alpha= \cos \alpha\\
\cos \alpha \cdot\csc \alpha= \cot \alpha\\
\cot \alpha \cdot\sec \alpha= \csc \alpha\\
\csc \alpha \cdot\tan \alpha= \sec \alpha\\
\sec \alpha \cdot\sin \alpha= \tan \alpha\\
\end{split}~.
\end{equation}
\item 平方关系:
\begin{equation}\label{eq_HsTrFu_3}
\begin{split}
\sin ^{2} \alpha + \cos ^{2}\alpha =1\\
\tan  ^{2} \alpha + 1 =\sec ^{2}\alpha\\
1 + \cot ^{2}\alpha =\csc ^{2}\alpha\\
\end{split}~.
\end{equation}
\end{itemize}
上面的公式太多了,不好记怎么办?有人总结了如\autoref{fig_HsTrFu_4} 所示的方法来辅助记忆:
\begin{itemize}
\item 六边形对角线上的函数互为倒数。
\item 六边形顶点上的函数等于相邻两顶点乘积。
\item 三个倒立的三角(黄色标记)上方两顶点的平方和等于下方顶点。
\end{itemize}
\begin{figure}[ht]
\centering
\includegraphics[width=12cm]{./figures/6390d1e662067a9b.png}
\caption{同角三角函数的基本关系} \label{fig_HsTrFu_4}
\end{figure}

在上面的同角三角函数的恒等关系中,可以发现同一个数值往往有多种不同的表达方式。例如,$1$ 既可以表示为 $\sin^2\alpha+\cos^2\alpha$,也可以表示为 $\sec^2\alpha-\tan^2\alpha$。这种性质表明,由三角函数构成的表达式与代数中熟悉的幂表达式不同,其表示形式通常并不唯一。这种灵活性是三角函数的一个重要特点,也是高中阶段经常需要处理的问题之一。

其实,由于高中阶段不涉及$\cot,\sec,\csc$,因此,上面的关系中,常常使用的只有:

\begin{gather}
\sin ^{2} \alpha + \cos ^{2}\alpha =1~.\\
\tan \alpha= \frac{\sin \alpha}{\cos \alpha}~.
\end{gather}

前者是大名鼎鼎的勾股定理,后者则是初中时学习的$\tan\alpha$的定义。这里之所以给出这么多,一则是为了知识的完整,二则是在一些推导过程中,利用恒等关系可以化简避免错误。如果觉得记忆困难,完全可以放弃,提升熟练度一样可以。

\subsection{三角函数的性质}
易得,正弦函数和余弦函数的\textbf{定义域为全体实数},正切函数的定义域为 $\begin{Bmatrix}\alpha|\alpha \neq \frac{\pi}{2}+k\pi,k\in Z\end{Bmatrix}~.$
\subsubsection{周期性}
正弦函数、余弦函数、正切函数的都是周期函数,根据定义易得,正弦函数和余弦函数,周期为 $2k\pi(k\in Z,k\neq0)$,正切函数的周期为 $k\pi(k\in Z,k\neq0)$.
\subsubsection{导数}

\subsubsection{图像}

\begin{figure}[ht]
\centering
\includegraphics[width=14.25cm]{./figures/bb986656153a547d.png}
\caption{正弦函数} \label{fig_HsTrFu_1}
\end{figure}
\begin{figure}[ht]
\centering
\includegraphics[width=14.25cm]{./figures/d0ae1e0d9c167f1e.png}
\caption{余弦函数} \label{fig_HsTrFu_2}
\end{figure}
可以看出正弦函数和余弦函数是定义域为 $R$ 值域为 $[-1,1]$ 最小正周期 $T = 2\pi$ 的周期函数。
