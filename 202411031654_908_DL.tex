% 动量(综述)
% license CCBYSA3
% type Wiki

本文根据 CC-BY-SA 协议转载翻译自维基百科\href{https://en.wikipedia.org/wiki/Momentum#Conservation}{相关文章})

\begin{figure}[ht]
\centering
\includegraphics[width=6cm]{./figures/4be4a8b3d4641396.png}
\caption{台球白球的动量在碰撞后传递给排列好的球。} \label{fig_DL_1}
\end{figure}
在牛顿力学中,动量(复数:momenta 或 momentums;更具体地称为线动量或平动动量)是物体质量与速度的乘积。它是一个矢量量,具有大小和方向。如果物体的质量为 \( m \),速度为 \( \mathbf{v} \)(也是矢量量),则该物体的动量 \( \mathbf{p} \)(来自拉丁语 pellere,意为“推动”)表示为:\(\mathbf{p} = m \mathbf{v}\)在国际单位制(SI)中,动量的单位是千克米每秒(kg⋅m/s),在量纲上等同于牛顿秒。

牛顿的第二运动定律指出,物体动量的变化率等于作用在其上的合力。动量依赖于参考系,但在任何惯性系中都是守恒量,意味着如果一个封闭系统不受外力影响,其总动量保持不变。动量在狭义相对论中也守恒(采用修正的公式),并在电动力学、量子力学、量子场论和广义相对论中以修正的形式得到保留。它表达了时空的基本对称性之一:平移对称性。

经典力学的高级表述,如拉格朗日力学和哈密顿力学,允许选择包含对称性和约束的坐标系。在这些系统中,守恒量是广义动量,而一般情况下这不同于上述的动能动量。广义动量的概念被延续到量子力学中,在那里它成为作用于波函数的算符。动量算符和位置算符通过海森堡不确定性原理相联系。

在电磁场、流体动力学和可变形体等连续系统中,可以定义动量密度,即每单位体积的动量(体积特定的量)。动量守恒的连续体版本导致了诸如流体的纳维-斯托克斯方程或可变形固体或流体的柯西动量方程等方程。
\subsection{经典力学}
动量是一个矢量量,具有大小和方向。由于动量具有方向性,因此可以用它来预测物体碰撞后的运动方向和速度。下面描述的是动量在一维中的基本性质。矢量方程与标量方程几乎相同(见多维情况)。
\subsubsection{单个粒子}
粒子的动量通常用字母 \( p \) 表示,它是粒子质量(用字母 \( m \) 表示)和速度(\( v \))的乘积:[1]
\[
p = mv.~
\]
动量的单位是质量和速度单位的乘积。在国际单位制(SI)中,如果质量以千克(kg)为单位,速度以米每秒(m/s)为单位,那么动量的单位就是千克米每秒(kg⋅m/s)。在厘米-克-秒制(cgs)单位中,如果质量以克(g)为单位,速度以厘米每秒(cm/s)为单位,那么动量的单位就是克厘米每秒(g⋅cm/s)。

作为一个矢量,动量具有大小和方向。例如,一个质量为 1 kg 的模型飞机,以 1 m/s 的速度正北方向平稳飞行,其相对于地面的动量为 1 kg⋅m/s,方向为正北。