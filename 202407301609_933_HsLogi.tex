% 数理逻辑(高中)
% keys 逻辑|高中
% license Usr
% type Tutor

\begin{issues}
\issueDraft
\end{issues}
\pentry{集合\nref{nod_HsSet}}{nod_fc7f}

尽管高中shu x教材中在逐渐弱化
数理逻辑是什么,以及它在数学和其他学科中的关键作用。

\subsection{定义}

\subsection{命题}
介绍命题的概念及其分类(真命题和假命题)。
\subsubsection{布尔值}
这是值,而非变量,“真、假”或者“${\rm T,F}$”或者“$1,0$”。注意不要斜体。
\subsection{逻辑连接词}
\subsubsection{且}
\textbf{且}(and,也称为同)记号为$A\land B$。
\subsubsection{或}
\textbf{或}(or,也称为或者)记号为$A\lor B$。


\subsubsection{非}
\textbf{非}(not,也称为同)记号为$\lnot A$。
\subsection{量词}
\subsubsection{全称量词}
$\forall$
\subsubsection{存在量词}
$\exists$

\subsection{条件}