% 主理想整环
% keys PID|主理想|整环|因式分解
% license Xiao
% type Tutor

\pentry{唯一析因环\nref{nod_UFD}}{nod_887f}

在寻找唯一析因环的例子时,一个非常容易识别的类别就是本节介绍的的主理想整环。

可以证明,任意多个理想的交集依然是理想,因而可以用该方法来构造环上的理想。对于环$R$的\textbf{非空子集}$S$,用$<S>$表示环上所有包含该集合的理想之交,称为由$S$\textbf{生成}的理想。易证<$S$>是包含该集合的最小理想。
\begin{example}{}
设$R$为幺环,则有
\begin{equation}
\langle S\rangle=\left\{\sum_{i=1}^n x_i a_iy_i \mid n \in \mathbb{N}, x_i,y_i \in R, a_i \in S, i=1,2, \cdots, n\right\}~.
\end{equation}
\end{example}
\begin{definition}{主理想}
单个元素生成的理想,称为一个\textbf{主理想(principal ideal)},并称该元素为主理想上的一个\textbf{生成元}。
对于整环 $R$,任取 $a\in R$,可证 $\langle a \rangle=\{ra|r\in R\}$ 。
\end{definition}

\begin{definition}{主理想环}
对于\textbf{环}$R$,如果它的理想全都是主理想,那么称它为一个\textbf{主理想环(principal ideal ring)},常简称PIR。

对于\textbf{整环}$R$,如果它是PIR,那么称它为一个\textbf{主理想整环(principal ideal domain)},常简称PID。
\end{definition}







