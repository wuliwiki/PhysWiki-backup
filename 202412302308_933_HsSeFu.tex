% 数列(高中)
% keys 高中|数列的概念|数列的函数特性
% license Usr
% type Tutor

\begin{issues}
\issueDraft
\end{issues}
\pentry{函数\nref{nod_functi}}{nod_7191}

在之前的学习中,已经介绍过函数的概念。函数是一种描述变量之间关系的数学工具,高中阶段接触到的函数,其定义域通常是连续的,比如实数集 $\mathbb{R}$ 或区间,因此函数的图像往往是一条连续的曲线。然而,数学世界中并非所有的关系都具有连续性。一些仅定义在自然数集或其子集上的特殊函数,被称为数列。数列是非常古老的数学内容,在某些方面,古代数学家们已经做了很深入的研究。最初,数列是人们将观测到的对象按顺序排列而形成的一种表示方式,那时甚至还没有函数的概念。尽管数列作为独立的数学领域有着丰富的研究内容和独特的性质,但从函数的视角观察数列,会发现它实际上继承了函数的许多特点。在学习数列时,既要关注其作为离散排列的特殊性,也要结合函数的视角来探索它的规律。

\subsection{数列}

小学时,经常会出现类似的题目:$1,3,7,(),31$,然后找规律填出数字。尽管这种题目似乎给出了标准答案,但事实上填任何数都可以讲出规律来。数列重点就是想体现这个规律,并希望用一个统一的表达式来表示出来。上面的题目,很可能想要表达的规律就是$2^n-1$,这里$n$就代表了第几个数。

把一个月中的每天,按照星期的数字标记上,就得到了一个数列,这列数(大概描述一下性质)。

一般地,按一定次序排列的一列数叫做\textbf{数列(sequence)}或\textbf{序列},数列中的每一个数叫作这个数列的\textbf{项}。数列的一般形式可以写成
\begin{equation}
a_1,a_2,a_3,\cdots,a_n,\cdots~
\end{equation}
简记为数列 $\begin{Bmatrix} a_n \end{Bmatrix}$,其中数列的第1项 $a_1$,也称\textbf{首项};$a_n$ 是数列的第 $n$ 项,也叫数列的\textbf{通项}。

项数有限的数列,称为\textbf{有限数列};项数无限的数列,称为\textbf{无穷数列}。

如果数列 $\begin{Bmatrix} a_n \end{Bmatrix}$ 的第 $n$ 项 $a_n$ 与 $n$ 之间的函数关系可以用一个式子表示成 $a_n = f(n)$,那么这个式子就叫作这个数列的\textbf{通项公式},数列的通项公式就是相应函数的解析式。

\textsl{注意:不是所有数列都能写出通项公式。}

\subsection{数列和}

无穷数列的数列和也称为级数,事实上


数列的基本性质和特点

数列的规律性
每个数列都有独特的规律,例如:
等差数列的公差:相邻两项之差为 $d$。
等比数列的公比:相邻两项之比为 $r$。

常见问题
如何根据数列规律求通项公式?
如何计算数列的前 $n$ 项和?
递推公式是数列的另一种表示方法,通过已知项推导出下一项。例如:
斐波那契数列的递推公式:$a_n = a_{n-1} + a_{n-2}$。

\subsection{函数特性}

\subsection{增减性}
一般地,一个数列 $\begin{Bmatrix} a_n \end{Bmatrix}$,如果从第2项起,每一项都大于前一项,即 $a_{n+1}>a_n$,那么这个数列叫作\textbf{递增数列}。

如果从第2项起,每一项都小于前一项,即 $a_{n+1}<a_n$,那么这个数列叫作\textbf{递减数列}。

如果数列 $\begin{Bmatrix} a_n \end{Bmatrix}$ 的各项都相等,那么这个数列叫作\textbf{常数列}。

如果数列 $\begin{Bmatrix} a_n \end{Bmatrix}$ 从第2项起,有些项大于它的前一项,有些项小于它的前一项,这样的数列叫\textbf{摆动数列}。


在高中阶段,数列的研究相对狭隘,主要集中在古代早期就被发现的等差数列和等比数列。然而,建议读者不要因此局限视野,数列的概念在更高层次的数学学习中,将与许多重要概念密切相关,例如实数的构建、级数以及分析学中的广泛应用。因此,在学习数列时,需要多关注其规律和研究方法,而不仅仅停留于记忆一些常见公式。理解这些公式背后的原理尤为重要,这不仅有助于掌握数列的变化规律,也能培养更深层次的数学思维,扎实当前的知识体系,为未来更复杂的数学学习奠定坚实的基础。