% 带余除法
% keys 余式|商式

\begin{issues}
\issueOther{需添加例题}
\end{issues}

\pentry{一元多项式\upref{OnePol}}
\footnote{吴群.矩阵分析[M].上海:同济大学出版社}在一元多项式\upref{OnePol}的最后提到,系数在数域 $\mathbb{F}$ 上的全体一元多项式构成一元多项式环 $\mathbb{F}[x]$.在环 $\mathbb{F}[x]$ 中,可以做加、减、乘三种运算,运算的结果还是该多项式环 $\mathbb{F}[x]$ 中的元素(一元多项式).自然,我们会像在整数里那样,考虑除法运算.你将看到,和整数一样,在一元多项式里,任意两个多项式做除法运算的结果并不一定还是一个多项式,取而代之的是带余除法.
\begin{theorem}{带余除法}\label{DivAlg_the1}
设 $f(x)$ 与 $g(x)$ 为 $\mathbb{F}[x]$ 中的两个多项式,并且 $g(x)\neq 0$,则存在唯一的 $\mathbb{F}[x]$ 中的多项式 $q(x),r(x)$,使得
\begin{equation}
f(x)=q(x)g(x)+r(x)
\end{equation}
其中 $\mathrm{deg}\;r(x)<\mathrm{deg}\;g(x)$, $q(x),r(x)$ 分别称为 $g(x)$ 除 $f(x)$ 的\textbf{商式}和\textbf{除式}.并将这种算法称为\textbf{带余除法},有时称为\textbf{长除法}.
\end{theorem}
\subsection{证明}
1.首先证明 $q(x)$,$r(x)$ 的存在性.

当 $\mathrm{deg}\;f(x)<\mathrm{deg}\;g(x)$ 时,取 $q(x)=0,r(x)=f(x)$ 即可;

假设当 $\mathrm{deg}\;f(x)-\mathrm{deg}\;g(x)\leq k$ 时,$q(x),r(x)$ 存在,那么当 $\mathrm{deg}\;f(x)-\mathrm{deg}\;g(x)=k+1$ 时,设
\begin{equation}
f(x)=\sum_{i=0}^n a_i x^i\quad g(x)=\sum_{i=0}^m b_ix^i
\end{equation}
取
\begin{equation}
f_1(x)=f(x)-\frac{a_n}{b_m}x^{n-m}g(x)
\end{equation}
注意 $f_1(x)$ 的 $m$ 次项系数为0,由\autoref{OnePol_eq3}~\upref{OnePol}
\begin{equation}
\mathrm{deg}\;f_1(x)-\mathrm{deg}\;g(x)\leq f(x)-1-\mathrm{deg}\;g(x)=k
\end{equation}
由归纳假设知,存在 $q_1(x),r(x)$ 使得
\begin{equation}
f_1(x)=q_1(x)g(x)+r(x)
\end{equation}
其中 $\mathrm{deg}\;r(x)<\mathrm{deg}\;g(x)$,于是
\begin{equation}
f(x)=f_1(x)+\frac{a_n}{b_m}x^{n-m}g(x)=\qty(\frac{a_n}{b_m}x^{n-m}+q_1(x))g(x)+r(x)
\end{equation}
记 $q(x)=\frac{a_n}{b_m}x^{n-m}+q_1(x)$,即有 $q(x),r(x)$ 在 $\mathrm{deg}\;f(x)-\mathrm{deg}\;g(x)=k+1$ 时存在.有数学归纳法,存在性得证.

2.下面证明 $q(x),r(x)$ 的唯一性.设另有多项式 $q_1(x),r_1(x)$ 使
\begin{equation}
f(x)=q_1(x)g(x)+r_1(x)
\end{equation}
其中, $\mathrm{deg}\;r_1(x)<\mathrm{deg}\;g(x)$,于是
\begin{equation}
q_1(x)g(x)+r_1(x)=q(x)g(x)+r(x)
\end{equation}
即
\begin{equation}
\qty(q_1(x)-q(x))g(x)=r(x)-r_1(x)
\end{equation}
若 $q(x)\neq q_1(x)$,又 $g(x)\neq 0$,那么 $r(x)-r_1(x)\neq 0$,由\autoref{OnePol_eq4}~\upref{OnePol}
\begin{equation}\label{DivAlg_eq1}
\mathrm{deg}\;(q_1(x)-q(x))+\mathrm{deg}\;g(x)=\mathrm{deg}\;\qty(r(x)-r_1(x))
\end{equation}
但
\begin{equation}
\mathrm{deg}\;g(x)>\mathrm{deg}\;\qty(r(x)-r_1(x))
\end{equation}
所以\autoref{DivAlg_eq1} 不能成立,这就证明了
$q(x)=q_1(x)$,因此 $r(x)=r_1(x)$.
\subsection{推论}
利用\autoref{DivAlg_the1} ,立即得到下面的推论
\begin{corollary}{余数定理}\label{DivAlg_cor1}
设 $f(x)$ 为 $\mathbb{F}[x]$ 中的任意一个的多项式,并且 $c\in\mathbb{F}$,则存在 $\mathbb{F}[x]$ 中的多项式,使得
\begin{equation}
f(x)=q(x)(x-c)+f(c)
\end{equation}
并且 $q(x)$ 是唯一的.$f(c)$ 称为 多项式 $f(x)$ 除以 $x-c$ 的余数.
\end{corollary}
\begin{definition}{零点(根)}
设 $f(x)$ 为 $\mathbb{F}[x]$ 中的多项式,$c\in\mathbb{F}$,若 $f(c)=0$,则称 $c$ 为 $f(x)$ 的\textbf{根}或\textbf{零点}.
\end{definition}
由\autoref{DivAlg_cor1} 很容易得到下面关于多项式根的推论.
\begin{corollary}{}
设 $f(x)$ 为 $\mathbb{F}[x]$ 中的多项式,$c\in\mathbb{F}$,则 $c$ 是 $f(x)$ 的根当且仅当存在 $\mathbb{F}[x]$ 中的多项式 $q(x)$,使得
\begin{equation}
f(x)=q(x)(x-c)
\end{equation}

\end{corollary}