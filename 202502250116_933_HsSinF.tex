% 正弦型函数(高中)
% keys 正弦型函数|五点法作图|相位|角频率|信号
% license Usr
% type Tutor

\begin{issues}
\issueDraft
\end{issues}

\pentry{函数视角下的三角函数\nref{nod_HsTFFv},三角恒等变换\nref{nod_HsAnTf}}{nod_373d}

前面所介绍的三角函数的基本图像是理解和应用的重要基础,需要熟记。这些图像源于对各个自变量对应函数值的计算,凝聚了数学家的长期探索成果。

此前,在研究幂函数等函数时,通常通过分析关键点和整体趋势来绘制其图像。同样的方法也适用于三角函数。正弦函数在所有三角函数中具有特殊地位,许多涉及三角函数的函数在化简和推导后,都可以表示为正弦函数的某种变形。其实在前面的介绍中已经接触过这种例子了,根据诱导公式$\cos x$可以表示为:
\begin{equation}\label{eq_HsSinF_1}
\cos x=\sin(x+{\pi\over2})~.
\end{equation}
因此,接下来将以正弦函数为例,探讨如何绘制并理解相关的任意图像。

\subsection{正弦型函数}

目前研究的三角函数均为 $\sin x$ 形式,即未涉及额外参数。为了进一步拓展应用,需要引入正弦型函数,以更一般的形式刻画这些变形。正弦型函数是对基本正弦函数的扩展,它通过调整振幅、频率和相位来适应不同的周期性变化。

\begin{definition}{正弦型函数}
形如
\begin{equation}
f(x) = A\sin(\omega x + \varphi)~.
\end{equation}
的函数称为\textbf{正弦型函数(sinusoidal function)},其中 $A, \omega, \varphi$ 为常数,且满足 $A\omega \neq 0$。其中:
\begin{itemize}
\item $|A|$ 称为\textbf{振幅(amplitude)};
\item $\omega x + \varphi$ 称为\textbf{相位(phase)}。
\end{itemize}
\end{definition}

如 \autoref{eq_HsSinF_1} 所示,$\cos x$ 可以视为正弦型函数的一种特殊形式,其中 $\displaystyle A = \omega = 1, \varphi = \frac{\pi}{2}$。此外,通过适当的变形,可以将参数调整至更规范的范围。利用 $\sin(-x) = -\sin x$ 和 $-\sin x = \sin(x + \pi)$,可以确保 $A$ 和 $\omega$ 取正值,而所有的符号变化都体现在 $\varphi$ 的取值变化上\footnote{需要注意的是,这并不意味着仅仅改变 $\varphi$ 的符号,具体情况将在后续讨论中说明}。因此,在规范化的表达中,参数满足 $A \in (0, +\infty)$,$\omega \in (0, +\infty)$。教科书中通常使用 $|A|$ 和 $|\omega|$ 进行表示。同样,相位  $\phi $ 的取值范围一般是任意实数,但在实际应用中,通常约定它的范围在$[0, 2\pi)$  或  $(-\pi, \pi]$ ,以避免冗余描述。由于三角函数的周期性,如果不在此范围内,可以利用 $\sin x = \sin(x + 2k\pi)$ 进行调整,使其化为符合规范的形式。为了简化讨论,后续内容均默认正弦型函数已转换为上述标准形式。

一下子引入多个参数可能会让人感到眼花缭乱,但它们的核心作用是刻画正弦型函数相较于标准正弦函数 $\sin x$ 的变化。这些参数的设置旨在描述明确函数的变换规律,使其与 $\sin x$ 的对应关系更加清晰。

\subsection{相位}

在这些新概念中,相位 $\omega x + \varphi$ 尤其值得关注。与以往的参数不同,它不是单独作为一个数值出现,而是作为整体引入,从而提供了一种新的视角来理解三角函数的变化。具体而言,由于 $\sin x$ 是一个非线性函数,直接分析其变化规律并不直观。因此,可以借鉴指数函数的处理方式,将 $\sin(\omega x + \varphi)$ 视为一个复合函数,其中 $\omega x + \varphi$ 对 $x$ 进行线性变换,作为 $\sin$ 的输入,而 $\sin$ 仅在最后起到非线性映射的作用,将输入限制在 $[-1,1]$ 之间,而不直接影响 $x$ 的变换过程。

这种分解方式将三角函数的周期性的非线性行为与输入变量的线性变化分离,使得分析更加直观,并有助于理解各参数对函数图像的影响。在数学建模中,许多非线性问题,如神经网络或回归分析,也常采用类似的方法:先处理线性部分,再通过非线性映射得到最终结果。这种思路不仅简化了分析过程,还在广泛的数学和工程领域中发挥了重要作用。

这里要注意的是,尽管在英语中使用相同的单词,但\textbf{相(phase)} 和 \textbf{相位(phase)}却指向两个不同的概念。理解这两个概念的区别,对于准确把握相位的意义至关重要。

先想象这样一个过程:垂直向上抛出一个球,当球达到某个高度时,它可能处于上升阶段,也可能处于下降阶段。仅凭高度本身无法判断球的运动趋势,必须结合其运动方向的信息。“相”指的是周期性变化过程中某个特定的状态,例如某一时刻球的高度和运动方向的组合,就像一张记录了该瞬间所有信息的特殊照片。而“相位”则标识了该状态在整个周期中的位置,类似于给照片附加的时间戳,使其明确对应于周期内的哪个时刻。

同样,在 $\sin x$ 的周期内,虽然同一个 $y$ 值通常对应两个不同的 $x$ 值,但这两个点的导数符号相反,意味着它们的运动趋势不同。因此,$(y, y')$ 这对信息可以唯一地确定周期运动中的某个状态,即“相”,而 $x$ 则是指向该状态的唯一“相位”。在正弦型函数中,$x$ 的位置由 $\omega x + \varphi$ 代替,因此确定 $\omega x + \varphi$ 也就等同于确定 $\sin x$ 在周期内的具体位置,从而确保状态信息的完整性。在相位的表达式中:
\begin{itemize}
\item $\omega$ 称作\textbf{角频率(angular frequency,也称圆频率)},它控制输入值的增长速度,即 $x$ 变化时 $\omega x$ 变化的速率,从而决定了函数的变化频率。
\item $\varphi$ 称为\textbf{初相(initial phase)},它设定了初始相位,即 $x=0$ 时,决定了函数图像相处于标准 $\sin x$ 的哪一个相。
\end{itemize}

\subsection{正弦型函数中参数的含义}

接下来依次分析 $f(x)$ 中的各个参数,并探讨它们对函数行为的具体影响。

$A$ 决定了 $f(x)$ 的值域 $\left[-A, A\right]$。如果将正弦函数视为一种振动,那么 $A$ 代表的是函数的最大偏离值,这也是“振幅”一词的来源。换句话说,$A$ 控制了函数图像在垂直方向上的伸缩,最高点与最低点的差为 $2A$。

$\omega$ 决定了函数变化的快慢,即图像在水平方向上的压缩或拉伸程度。$\omega$ 越大,周期越短,函数振荡得越快。\textbf{周期(cycle)} $T$ 与 $\omega$ 之间的关系为:
\begin{equation}
    T = \frac{2\pi}{\omega}~.
\end{equation}
在实际应用中,有时直接用 $T$ 代替 $\omega$ 来描述函数的周期性:
\begin{equation}
    f(x) = A\sin\left(\frac{2\pi}{T} x + \varphi\right)~.
\end{equation}

从圆周运动的视角来理解。角频率 $\omega$ 是圆周运动特有的量,表示单位时间内转过的弧度。在日常生活中,与此类似的另一个常见的概念是\textbf{频率(frequency)},它频率表示单位时间内完成的完整振荡次数,在圆周运动中,它对应于物体转过的圈数。由于一个完整的圆周对应 $2\pi$ 弧度,因此频率 $f$ 与角频率 $\omega$、周期 $T$ 的关系为:
\begin{equation}
    f = \frac{1}{T} = \frac{\omega}{2\pi}~.
\end{equation}
这一关系表明,角频率和频率本质上是同一现象的不同刻画方式。值得注意的是,这里的分析基于圆周运动的概念,这也正体现了“三角函数”也称为“圆函数”这件事。

$\varphi$ 的作用主要体现在比较具有相同角频率的两个正弦型函数时。在给定 $\omega$ 的情况下,两个正弦型函数的形状完全相同,但如果它们具有不同的初相 $\varphi$,则它们的图像将在 $x$ 轴方向上发生平移。两个函数的相位便始终保持固定的\textbf{相差(phase difference)},因此得知初相便可计算不同函数的相差,进而清晰地描述两个周期信号之间的相对位置。

相差在信号处理、波动分析以及同步系统中具有重要意义,例如,在交流电路中,相差决定了电压与电流的相对关系,从而影响功率的计算和能量的传输。下面来看相差如何具体影响周期信号的行为。设有两个具有相同角频率 $\omega$的正弦型函数:
\begin{equation}
f_0\left(x\right)=A_0\sin\left(\omega x+\varphi_0\right),\quad f_1\left(x\right)=A_1\sin\left(\omega x+\varphi_1\right)~.
\end{equation}
它们的相差定义为:
\begin{equation}
\Delta \varphi = \varphi_1 - \varphi_0~.
\end{equation}
若将$f_0,f_1$视作信号,则$\Delta \varphi$ 直接决定了两个信号在时间轴上的相对位置:
\begin{itemize}
\item 若 $\Delta \varphi = 0$,则两个信号完全同步,峰值和零点对齐。
\item 若 $\Delta \varphi > 0$,则 $f_1(x)$ 相对于 $f_0(x)$ 向左平移,即 $f_2(x)$ 领先于 $f_1(x)$。
\item 若 $\Delta \varphi < 0$,则 $f_1(x)$ 相对于 $f_0(x)$ 向右平移,即 $f_1(x)$ 滞后于 $f_0(x)$。
\end{itemize}

另外,当计算两个正弦型函数的和,即将两个信号叠加时,相差将直接影响合成信号的幅度和形态:
\begin{equation}
\begin{split}
f(x) &= A_0\sin(\omega x + \varphi_0) + A_1\sin(\omega x + \varphi_1)\\
&=(A_0\cos\varphi_0 + A_1\cos\varphi_1) \sin \omega x +
(A_0\sin\varphi_0 + A_1\sin\varphi_1) \cos \omega x~.
\end{split}
\end{equation}
令:
\begin{equation}
C = A_0\cos\varphi_0 + A_1\cos\varphi_1, \quad D = A_0\sin\varphi_0 + A_1\sin\varphi_1~.
\end{equation}

则合成振幅  A  和相位  \varphi  由以下关系确定:
\begin{equation}
A = \sqrt{C^2 + D^2}, \quad \tan\varphi = \frac{D}{C}~.
\end{equation}

最终,
\begin{equation}
f(x) = A \sin(\omega x + \varphi)~.
\end{equation}

可见,两个相同圆频率的正弦函数的和仍然是同一圆频率的正弦函数,只是振幅和相位发生了变化。相位差不仅影响数学上的波形关系,在物理应用中也决定了系统的同步性、信号的叠加效果以及能量传输的效率。

另外,利用余弦和差公式:$\cos(\varphi_0 - \varphi_1) = \cos\varphi_0 \cos\varphi_1 + \sin\varphi_0 \sin\varphi_1$三角恒等式计算得到
\begin{equation}
\begin{split}
A^2 &= C^2+D^2\\
&=(A_0\cos\varphi_0 + A_1\cos\varphi_1)^2+(A_0\sin\varphi_0 + A_1\sin\varphi_1)^2\\
&=A_0^2 + A_1^2 + 2 A_0 A_1 (\cos\varphi_0 \cos\varphi_1 + \sin\varphi_0 \sin\varphi_1)\\
&=A_0^2 + A_1^2 + 2 A_0 A_1 \cos(\varphi_0 - \varphi_1)~.
\end{split}
\end{equation}

由于$\cos x$是偶函数,而$A_0,A_1$是定值,所以$A^2$或$|A|$的大小取决于$|\varphi_0 - \varphi_1|$的大小。
\begin{itemize}
\item 当 $\varphi_0 = \varphi_1$ 时,即$f_0(x),f_1(x)$\textbf{同相},$\cos(\varphi_0 - \varphi_1) = 1$, $A$  取得最大值  $A_0 + A_1$ 。
\item 当 $|\varphi_0 - \varphi_1| = \pi$时 ,即$f_0(x),f_1(x)$\textbf{反相},$\cos(\varphi_0 - \varphi_1) = -1$, $A$  取得最小值  $|A_0 - A_1|$ 。
\end{itemize}
当两个正弦信号的相位相同时,它们的波峰和波谷完全对齐;而当它们的相位相差 $\pi$ 时,它们处于完全相反的状态,即一个波峰对应另一个的波谷。


\subsection{五点法作图}

五点作图法是一种简便绘制三角函数图像的方法,主要用于 **正弦函数 $\sin x$、余弦函数 $\cos x$ 和正切函数 $\tan x$**。它利用函数的周期性和对称性,在一个基本周期内选取五个关键点,从而大致勾勒出函数的图像。这些点可以帮助快速绘制正弦或余弦函数的平滑曲线。这种方法简洁高效,适用于手绘和计算机绘图中的初步拟合。

对于更一般形式的三角函数:
\begin{equation}
f(x) = A \sin(\omega x + \varphi) + b~.
\end{equation}
注意这里在后面加了另一个参数b,这是从线性变换的视角来看了。另外如前所述,内部是线性的,因此经过$\sin x$映射后,仍然会呈现出sinx的形态,外部的A和b是对结果进行第二次线性变换。内外的变化都是线性的,只有sin是非线性的,由于三角函数都是周期函数,因此选取一个周期内的一些点来作为代表就可以大概地刻画函数的图像形态。在幂函数、指数函数等函数的研究时也大概地利用过这种方法,即说函数过定点,然后在按照参数的取值恢复不同的形态。由于三角函数的形态比较统一,不论取什么参数都是类似的振动的样子,因此,就像直线需要两个点来确定位置一样,为了描述sinx的非线性通常选取在一个周期内的五个关键点作为限定。

之前提到过,周期$\displaystyle T = \frac{2\pi}{|\omega|}$

相位平移
\begin{equation}
\omega x + \varphi=\alpha\implies x={\alpha-\varphi\over \omega}~.
\end{equation}

\begin{table}[ht]
\centering
\caption{正弦函数的五个关键点}\label{tab_HsSinF1}
\begin{tabular}{|c|c|c|c|c|c|}
\hline
$\alpha$ & $0$ &$\displaystyle {T\over4},\left(\frac{1}{2\omega}\pi\right)$& $\displaystyle {T\over2},\left(\frac{1}{\omega}\pi\right)$ & $\displaystyle {3\over4}T,\left(\frac{3}{2\omega}\pi\right)$ & $\displaystyle T,\left(\frac{2}{\omega}\pi\right)$\\
\hline
$\sin \alpha$ & $0$&$1$&$0$&$-1$&$0$ \\
\hline
$x=\displaystyle{\alpha-\varphi\over \omega}$&$\displaystyle-{\varphi\over \omega}$&$\displaystyle{1\over 2\omega^2}\pi-{\varphi\over \omega}$&$\displaystyle{1\over \omega^2}\pi-{\varphi\over \omega}$&$\displaystyle{3\over 2\omega^2}\pi-{\varphi\over \omega}$&$\displaystyle{2\over \omega^2}\pi-{\varphi\over \omega}$\\
\hline
$f(x)$ & $b$&$b+A$&$b$&$b-A$&$b$ \\
\hline
\end{tabular}
\end{table}
这五个点包含了一个周期内的零点(对称中心)和极值点(对称轴)的信息,利用周期性扩展基本就能获得完整的函数图像。

五点作图法的核心是 **找到一个周期内的五个等分关键点**,然后利用这些点的函数值来绘制光滑的曲线:**$\sin x$ 和 $\cos x$**:取 **一个周期的五等分点**。

初学者容易混淆此处$\alpha=0$与之前介绍初相时说$\varphi$对应着$x=0$的含义
\addTODO{这块要重新写一下。就是平移和拉伸先后的问题,是两种思考方式。}


需要注意的是,相位与之前学习的的函数的平移、伸缩变换的视角略有不同,初学者容易在此混淆。例如,对于 $\displaystyle\sin(2x + \frac{\pi}{3})$,可以从两种视角来分析:
\begin{itemize}
\item 相位视角:将 $\displaystyle2x + \frac{\pi}{3}$ 视为整体,表示函数的输入值变化方式。$\omega = 2$ 使得输入值增长速度变快,而 $\displaystyle\varphi = \frac{\pi}{3}$ 使得初始位置发生偏移。如果要用平移和伸缩来表示的话,理解为先拉伸后平移。
\item 平移+拉伸视角:可以将 $\displaystyle\sin(2x + \frac{\pi}{3})$ 变形为 $\displaystyle\sin\left(2(x + \frac{\pi}{6})\right)$,此时可认为先对 $x$ 进行水平平移 $\displaystyle\frac{\pi}{6}$,再进行 $2$ 倍的水平缩放。
\end{itemize}

这两种分析方式在数学上是等价的,但从直觉上来看,平移+拉伸视角更符合传统的变换思路,相位视角能够更直接地解释 $\omega$ 和 $\varphi$ 对图像的影响,也避免了伸缩变换不保持距离的影响。掌握这两种不同的理解方式,有助于更灵活地分析正弦型函数的变化。

相位  \phi  与  \omega  无关,它只决定了波形的平移量,而  \omega  决定的是波的频率(或波长)。但如果写成 错位的表达式  \sin(\omega(x + \phi)) ,那相位的物理意义会被改变,使得 平移量依赖于频率,这就错误了。
此时  \phi  不再是简单的平移角度,而是被频率放大了,导致波形的偏移量随着  \omega  增大而增大,这不符合物理意义。
\addTODO{1}
另外关于正切,重要的就是找到渐近线和对称点。