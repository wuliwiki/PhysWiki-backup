% 群的自由积
\pentry{自由群\upref{FreGrp}}

将自由群的概念推广,即可得到两个群之间的自由积的概念.

\subsection{自由积的构造}

给定两个群$G$和$H$,取集合$G\cup H$上的自由群$F(G\cup H)$,则$F(G\cup H)$的元素形如$x_1x_2\cdots x_k$的有限长字符串,其中$k$是某个正整数,各$x_i$都是$G\cup H$的元素.

在$F(G\cup H)$上定义一个等价关系:如果字符串$g_1g_2\cdots g_k$中各$g_i\in G$,那么令$g_1g_2\cdots g_k\sim g_1\cdot g_2\cdot\cdots\cdot g_k$,即把该字符串等同于各字母在群$G$中运算的结果;同样地把字母都是$H$中元素的字符串等同于这些元素在群$H$中的运算结果;把$G$和$H$的单位元等同于空词.比如说,在整数加法群$\mathbb{Z}$中,把字符串$123$等同于数字$1+2+3$所代表的字符串,即只有一个字母$6$的字符串.

称商群$F(G\cup H)/\sim$为群$G$和群$H$的\textbf{自由积(free product)},记为$G*H$.

