% 小时百科感谢墙说明

\begin{issues}
\issueDraft
\issueOther{写一个简单的 svg 教程}
\end{issues}

小时百科感谢墙的创意源自 \href{http://www.milliondollarhomepage.com}{milliondollarhomepage.com}。

\subsection{具体规则}

\begin{figure}[ht]
\centering
\includegraphics[width=8cm]{./figures/1131f8b4944872fb.pdf}
\caption{两个相邻的最小单元格} \label{fig_thanks_1}
\end{figure}

\begin{itemize}
\item 感谢墙和小时百科共存,在百科首页放置链接,永不删除。 遭遇不可抗力或网站关闭除外。
\item 感谢墙由若干方形的 svg 图片上下无缝衔接组成, 每张等分为 $100\times 100$ 的\textbf{单元格}。 若一张图片填满则向下添加另一张。
\item 同一个用户可以根据捐款数额获取一个或多个单元格的编辑权, 数量为捐款金额除以 $100$ 并向下取整。 该数值可能根据通货膨胀或紧缩按比例调整。
\item 感谢墙上线以前的捐款者,经过核实也可以按同样规则获取若干单元格,先到先得。
\item 小时百科的突出贡献者,经内部讨论,可赠与一定数量的单元格。
\item 禁止使用违反法律法规或公序良俗的内容,禁止擦边涉黄或时政等敏感内容。
\item 禁止交易炒作编辑权,一经证实删除该用户所有编辑,并视情况进行封号等措施。
\item 单元格位置一经选定无法移动。 同一用户的单元格位置必须连通(即使多次捐款)。 两单元格对角连接不属于连通。 因其他用户的单元格阻挡而不能连通的除外。
\item 每个用户平均每个单元格的数据不能超过 1000 字节。
\item 允许在每个单元格内设置一个超链接。
\item 允许在可编辑范围内插入数量合理的 png 或 jpg 文件。
\item 每个单元格发布后仅允许修改 2 次,视觉或链接上的改变视为修改。
\item 单元格位置一经选定必须填充内容,不填充的,内容视为空白。
\item 可以申请清空一个或多个单元格, 清空无法恢复, 可以被新的用户占用。
\item 原始模板中可以提供一些公共的 \verb`<style>` 帮助用户节约代码长度, 后来的用户可以使用其他用户定义的 \verb`<style>`。
\end{itemize}
