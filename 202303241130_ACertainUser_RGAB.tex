% 刚体的平面运动(摘要)

\begin{issues}
\issueDraft
\issueOther{还没画图}
\end{issues}

\subsection{刚体}
刚体可视作一种特殊的质点系:质点之间存在约束并从不相对运动;同时,刚体中质量的分布是连续的而非离散的。质点系的所有结论仍然适用于刚体。

\subsubsection{自由度}
自由度可以理解为“确定系统状态所需的独立物理量的个数”。

单一质点有$3$个自由度,即$x,y,z$坐标(质点自身不能旋转,因此没有旋转角度的自由度);而具有$N$个质点的质点系有$3N$个自由度,即各个质点的坐标(质点系中,各质点的状态或位置没有直接明确的约束)。

作为对比,由于刚体内质点间的刚性约束,刚体只有$6$个自由度,即质心的$x,y,z$坐标与关于$x,y,z$轴的旋转角度。在平面运动(见下文)中,刚体有$4$个($x,y,z$坐标与关于转轴的旋转角度)或$3$个自由度(假定刚体在$z=z_0$平面内运动,那么自由度只剩下 $x,y$坐标与关于转轴的旋转角度)。

\begin{table}[ht]
\centering
\caption{自由度}\label{RGAB_tab3}
\begin{tabular}{|c|c|c|c|}
\hline
 & 质点 & 质点系 & 刚体 & 刚体(平面运动)\\
\hline
自由度 & $3$ & $3N$ & $6$ & $4$或$3$ \\
\hline
\end{tabular}
\end{table}

\subsection{刚体运动学}
\footnote{本节参考了张娟的《理论力学》与谢传峰等的《理论力学》。}
\subsubsection{平动}
刚体平动前后,刚体中任意两点的连线依然平行。同时,刚体内各质点的位移、速度、加速度等都相同。

\subsubsection{定轴转动}
刚体定轴转动前后,刚体中任意两点的连线不再平行,各质点的速度、加速度等也不再相同,但角速度、角加速度等圆周运动量还是相同。由圆周运动的知识可以得出各质点的运动状态。

\begin{example}{平动还是转动}
如图所示,如果不考虑地球的自转,那么地球的“公转”是平动还是转动?有点违反直觉的是,地球的“公转”还是平动,只是他的平动轨迹是一条圆弧线。
\end{example}
作为对比,由于质点没有体积、形状,因此质点只存在“平动”问题,而不存在“转动”问题。

\subsubsection{平面运动=同质心的平动+关于质心轴的定轴转动}
刚体的任意平面运动都可分解为同质心的平动与质心系中关于质心轴的定轴转动。也就是说,如果你跟随刚体的质心一起平动,那么你会看到刚体的其余部分正关于质心轴转动。

根据伽利略变换,我们可以得出如下基本结论:
\begin{itemize}
\item (某参考系中)刚体上任意点的位矢=(该参考系中)质心的位矢+质心参考系中该点的位矢。
\item (某参考系中)刚体上任意点的平面运动速度=(该参考系中)质心平动速度+质心参考系中该点的圆周运动速度。
\item (某参考系中)刚体上任意点的平面运动加速度=(该参考系中)质心平动加速度+质心参考系中该点的圆周运动切向+法向加速度。
\end{itemize}

\begin{exercise}{}
如果一刚体绕着一非质心轴做定轴转动,那么该如何用以上思路解释这种运动?
\end{exercise}

事实上,除质心之外,我们可以选取任意一点作为基点,并将刚体的平面运动分解为同该基点的平动与固连于基点的参考系中刚体关于过该基点的转轴的定轴转动。与此同时,不论如何如何选取基点,你都能得到相同的角速度与角加速度(但不一定能得到相同的速度或加速度)!这为解决一些问题带来了极大的方便。

但是,下文的柯尼希定理\textbf{没有}这么好的性质,要想使用柯尼希定理,你必须选取质心作为基点。

\subsubsection{运动学公式}
以上内容可总结为如下公式:
\begin{table}[ht]
\centering
\caption{刚体运动学方程}\label{RGAB_tab2}
\begin{tabular}{|c|c|c|c|}
\hline
 & 位矢 & 速度  & 加速度 \\
\hline
平动 & $\bvec r_A = \bvec r_C + \bvec r_{AC}$, $\bvec r_{AC}$是常数 & $\bvec v_A = \bvec v_C$ & $\bvec a_A = \bvec a_C$ \\
\hline
定轴转动(转轴在原点) & $\bvec r_A = \bvec r_{AC}$, $\bvec r_{AC}$模长不变 $$\theta =\theta (t) $$ & $\bvec v_A = \bvec \omega \times \bvec r_{AC}$ & $\bvec a_A = \bvec a_t + \bvec a_n$\\
\hline
平面运动(质心为基点) & $\bvec r_A = \bvec r_C + \bvec r_{AC}$ $$\theta =\theta (t) $$& $\bvec v_A = \bvec v_C + \bvec \omega \times \bvec r_{AC}$ & $$\bvec a_A = \bvec a_C + \bvec a_t + \bvec a_n$$ \\
\hline
\end{tabular}
\end{table}
其中,
\begin{itemize}
\item $\bvec r_A, \bvec v_A, \bvec a_A$ 是某参考系中点$A$的位矢、速度、加速度。
\item $\bvec r_C, \bvec v_C, \bvec a_C$ 是某参考系中质心的位矢、速度、加速度。
\item $\bvec r_{AC} = \bvec r_A -\bvec r_C $ 是该点$A$与质心的相对位矢
\item $\bvec \omega, \bvec \alpha$是质心系中刚体转动的角速度与角加速度
\item $\bvec a_t, \bvec a_n$是质心系中点$A$圆周运动的切向与法向加速度。$a_t=r_{AC}\alpha, a_n=r_{AC}\omega^2$
\end{itemize}

\subsection{刚体动力学}
\subsubsection{转动惯量}
类似于质量衡量了物体的“(平动)惯性”,转动惯量衡量了物体的“转动惯性”。

我们先从定轴转动的角动量出发,引入转动惯量的概念。如图所示,刚体做定轴转动。系统的角动量是各个质点的角动量之和
$$L = \sum r_\perp p = \sum r_\perp \Delta m v = \sum r_\perp \Delta m \omega r_\perp = \omega \sum r_\perp^2 \Delta m$$

我们发现其中$\sum r_\perp^2 \Delta m$是无关转动角速度的量(但是有关刚体几何形状、质量分布、转轴位置等)。我们定义转动惯量(由于刚体的质量是连续分布的,因此常写为积分形式)
$$
I =\sum r_\perp^2 \Delta m = \int_V r_\perp^2 \dd m
$$
那么
$$ L = I \omega$$

同理,从定轴转动的转动动能出发,也容易得到相应的结果。
$$E_k = \sum \frac{1}{2} \Delta m v^2 = \frac{1}{2} \sum \Delta m (\omega r_\perp)^2 = \frac{1}{2} \omega^2 \sum r_\perp^2 \Delta m = \frac{1}{2} I \omega^2$$

\begin{example}{转动惯量与生活}
当你推一扇大而厚重的门(例如防盗门)时,你会发现门很难被推动,也很难被停下。这就是因为这扇门关于门轴的转动惯量很大。如上文所述,门的转动惯量与质量与几何形状都有关。\textsl{\textbf{非常不建议}你为了感受转动惯量而用力推门,除非你是Suzume并有拯救世界的重任在身。}
\end{example}

\subsubsection{柯尼希定理:动能=平动动能+转动动能}
我们知道柯尼希定理:某参考系中质点系的动能=该参考系中质心的动能+质心参考系中系统的动能。在刚体这个特殊的质点系中,柯尼希定理依然成立,并更有物理意义:某参考系中刚体的平面运动动能=该参考系中质心的平动动能+质心参考系中刚体的转动动能:
$$E_k = \frac{1}{2}Mv_c^2 + \frac{1}{2} I_c \omega^2$$
其中$M$是刚体的质量,$v_c$是刚体(质心)的平动速率,$I_c$是刚体关于质心轴的转动惯量,$\omega$是刚体的转动角速度。

你\textbf{不能}选取质心之外的点,然后试图套用柯尼希定理!

\subsubsection{平面运动 动力学方程}
\begin{table}[ht]
\centering
\caption{平面运动 动力学方程}\label{RGAB_tab1}
\begin{tabular}{|c|c|c|c|}
\hline
名称 & 公式 & 相应的物理量1 & 相应的物理量2 \\
\hline
动量定理 & $\sum \bvec F_i = \dv{\bvec P}{t} = M \bvec a_c$ & $\sum \bvec F_i$是外力和 & $\bvec P$是刚体的动量,$M$是刚体的质量,$\bvec a_c$是质心加速度 \\
\hline
角动量定理 & $\sum (\bvec r - \bvec r_c) \times \bvec F_i = \dv{\bvec L}{t} = I \bvec \alpha$ 

\footnote{
这个公式的背后存在一些\textsl{复杂的哲学问题}。
我们说“定轴转动”时,已经认为角动量增量总在某一个方向上$\dv{\bvec L}{t} = \dv{L}{t} \hat{\bvec z}$;
但从公式本身看,我们似乎不能保证外力矩都恰好在这个方向上,$\sum (\bvec r - \bvec r_c) \times \bvec F_i$ 的方向似乎是任意的。
理解这个问题的两种思路:
1.刚体是被某种刚体之外的物理转轴钉死,转轴总会提供一个反向的力矩,来平衡那些不在转轴方向上的外力矩,因此刚体只能在转轴上转动;
2.我们施加的外力矩并不是“任意的”,而是精心选择的、只在转轴方向上的。
} 

& $\sum (\bvec r - \bvec r_c)\times \bvec F_i$是关于质心的外力矩和 & $\bvec L$是刚体的角动量,$I_c$是关于质心轴的转动惯量,$\bvec \alpha$是刚体的转动加速度 \\
% \hline
% 动能定理 & * & * & * \\
%baike上相应内容未完成...我大概知道一点但不是很了解。
\hline
\end{tabular}
\end{table}
\subsection{刚体静力学}
刚体的动量定理与角动量定理启发我们,如果刚体所受的外力和\textbf{与}外力矩和均为零\footnote{即使外力和为零,外力矩和也不一定为零!参考例子},那么刚体保持静止或做匀速平面运动。这也是我们分析刚体静力平衡的重要依据。


