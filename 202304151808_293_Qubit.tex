% 量子比特
% keys 未完成

\begin{issues}
\issueDraft
\end{issues}

\pentry{量子力学基本原理\upref{QMPrcp}}

在经典信息学中,一个比特(bit)代表着一个取值为0或者1的随机变量。比如一个电容器的状态可以离散表示为一个比特。当电容器处于高电平的时候,我们将其状态记为1,否则则记为0。在基于经典物理的信息论中,我们认为0和1这两种状态是可以被准确无误地区分开的。

在量子信息处理中,量子比特(qubit)是比特这个概念的量子对应。它描述了一个由$\ket{0},\ket{1}$表示的二能级量子系统的状态。我们仍然希望两种“量子状态”是可以被准确无误地区分开的,这自然要求着$\braket{0}{1}=0$。

和经典比特不同的是,量子比特可以处于两种状态的叠加态上

\subsection{布洛赫(Bloch)球表示}

\subsection{物理实现}