% g++ 编译器笔记

\begin{issues}
\issueDraft
\end{issues}

\pentry{在 Linux 上编译 C/C++ 程序\upref{linCpp}}

\begin{itemize}
\item \verb|-I <dir>| 选项可以声明 .h 所在的目录
\item \verb|-c| 选项只编译不 link
\item 使用其他目录下的 cpp 文件如 \verb|g++ -c <other flags>  <dir1>/1.cpp <dir2>/2.cpp 3.cpp 4.cpp.....|
\item 编译器会自动搜索默认目录 \verb|/usr/local/include/| 下的头文件
\item \verb|-D 宏| 定义宏
\item \verb|-O3| 是最优化, \verb|-g| 是调试, 如果不调试, 需要手动定义 \verb|-D NDEBUG|
\item \verb|-g| 不包含一些信息如宏定义, 需要用 \verb|-g3|
\item link 时候, \verb|-l| 选项需要放在所有 \verb|.o| 文件后面
\item 检查内存泄露可以用 Address Sanitizer: \verb|g++ main.cpp -fsanitize=address -static-libasan -g|
\item 检查 signed 整数加减乘的溢出可以用 \verb|-ftrapv|, 亲测支持 \verb|int, long, long long| (其他类型貌似不行), 运行时会终止程序, 可以用 gdb 找到位置, 不能在程序中 \verb|catch|.如果想支持所有整数类型, 可以用 \verb|SafeInt| 库.
\end{itemize}

\subsection{linker (ld)}
\begin{itemize}
\item \verb|-L| 指定添加 \verb|-l| 的搜索路径
\end{itemize}

\subsection{preprocessor}
\begin{itemize}
\item \verb|g++ -E -P xxx.cpp > out.txt| 会显示 preprocessor 的输出, 其中 \verb|-P| 会删掉 linkmarker (用于显示代码在头文件中来源)
\end{itemize}
