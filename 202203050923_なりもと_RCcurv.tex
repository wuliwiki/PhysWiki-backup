% 电容—电阻电路充放电曲线
% 电容|电阻|微分方程

%\begin{issues}
%\issueDraft
%\end{issues}

\pentry{电容\upref{Cpctor}, 一阶线性微分方程\upref{ODE1}}
\begin{figure}[ht]
\centering
\includegraphics[width=5cm]{./figures/RCcurv_1.pdf}
\caption{电容电阻串联} \label{RCcurv_fig1}
\end{figure}
回路中有直流电源 $U$ , 电阻 $R$ 和电容 $C$. 当开关拨向1时,接通电源,电容器充电;当开关拨向2时,断开电源,电容器放电.对于充放电过程有如下方程
\begin{equation}
IR+U_C=
\begin{cases}
U,& \quad\text{充电}\\
0,& \quad\text{放电}
\end{cases}
\end{equation}
设电容器极板带电量为 $Q$ ,由电流的定义\autoref{I_eq1}~\upref{I}和电容的定义\autoref{Cpctor_eq2}~\upref{Cpctor}
\begin{equation}
I = \dv{Q}{t}=C\dv{U_C}{t}
\end{equation}
代入得
\begin{equation}
RC\dv{U_C}{t}+U_C =
\begin{cases}
U,& \quad\text{充电}\\
0,& \quad\text{放电}
\end{cases}
\end{equation}
这是一个一阶线性常微分方程\upref{ODE1}. 初始条件为
\begin{equation}
U_C=\begin{cases}
0,&\quad \text{充电}\\
U,&\quad \text{放电}
\end{cases}
\end{equation}
 解得
\begin{equation}
U_C(t) = 
\begin{cases}
U\qty(1 - \E^{-t/(RC)}),&\quad\text{充电}\\
U\E^{-t/(RC)},&\quad\text{放电}
\end{cases}
\label{result}
\end{equation}
可以看到当 $t \to \infty$ 时,对于充电过程:$U_C = U$ ;而对放电过程:$U_C = 0$.

记 $\tau =RC$ 为该电路的时间常数.由式\upref{result}

作出 $RC$ 电路的充放电曲线如\autoref{RCcurv_fig2} 
\begin{figure}[ht]
\centering
\includegraphics[width=8cm]{./figures/RCcurv_2.pdf}
\caption{充电过程(左)和放电过程(右)} \label{RCcurv_fig2}
\end{figure}

