% 生成对抗网络
% 生成 对抗 无监督 深度学习

\textbf{生成对抗网络}(Generative Adversarial Network, GAN)是基于神经网络结构的生成模型,是深度学习中的一种主流方法.该模型在各种问题场景,比如数据生成、艺术造作、图像修复、图像风格转换、语音合成、文本图像互相转换等中均有十分广泛的应用.

生成对抗网络模型主要包含两个网络结构:一是捕获数据分布的生成模型,也称为生成器(通常用$G$表示),二是估计来自训练数据(而不是$G$)样本的概率的判别模型,也称为判别器(通常用$D$表示).生成器$G$的训练步骤是要最大化判别器$D$做出错误判断的概率.这个框架对应于最小最大化两人博弈.在生成器$G$和判别器$D$的任意函数的解空间中,存在一个独一无二的解,生成器还原训练数据分布,判别器处处等于$1/2$.

\begin{figure}[ht]
\centering
\includegraphics[width=14.25cm]{./figures/GAN_2.png}
\caption{生成对抗网络示意图} \label{GAN_fig2}
\end{figure}

生成器在数据$x$上的分布记为$p_g$,输入生成器的随机变量,即噪音,记为$z$,其分布为$p(z)$.生成器本质上是一个从潜在空间(latent space),即输入的随机噪音,到数据空间(data space)的映射,是由多层神经网络表示的带有参数$\theta_g$的可微分函数,记为$G(z;\theta_g)$,参数$\theta_g$的值是模型训练时确定的,也就是学习出来的.
生成器和判别器玩的是双人最小最大游戏,其价值函数为:
\begin{equation}
\mathop{\min}\limits_G \mathop {\max }\limits_D V(D,G)=E_{x\sim p_{data}(x)}[logD(x)]+E_{z\sim p_z(z)}[log(1-D(G(z)))]
\end{equation}

% Copied From MathType
% \begin{equation}
% \[\mathop {\min }\limits_G \mathop {\max }\limits_D V(D,G) = {E_{x\~{p_{data}}(x)}}[\log (D(x)] + {E_{z\~{p_z}(z)}}[\log (1 - D(G(z)))]\]
% \end{equation}

判别器的训练目标是尽可能分辨出数据是来自真实数据集,还是由生成器伪造的.通常会将真实数据的标签设置为$1$,生成器产生的数据的标签设置为$0$.生成器训练的目标是尽可能迷惑判别器,以使其无法准确辨别数据是真实的,还是生成器产生的.

训练生成对抗网络时,会将生成器和判别器连接成一个网络.判别器参数保持不变,更新生成器参数;生成器参数保持不变,更新判别器参数.生成器和判别器连续不断地交替更新,直至训练结束.
\begin{figure}[ht]
\centering
\includegraphics[width=14.25cm]{./figures/GAN_1.png}
\caption{网络演化过程} \label{GAN_fig1}
\end{figure}

随着人们研究的深入,生成对抗网络又出现多种变体.主要有:条件生成对抗网络(Conditional Generative Adversarial Nets, cGAN)和循环生成对抗网络(Cycle Generative Adversarial Network, CycleGAN).




\textbf{参考文献:}
\begin{enumerate}
\item I. Goodfellow et al., “Generative adversarial nets,” in Advances in neural information processing systems, 2014, pp. 2672–2680.
\end{enumerate}