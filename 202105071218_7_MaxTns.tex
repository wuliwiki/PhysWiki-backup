% 麦克斯韦应力张量
% keys 电磁|电动力学|张量|动量

\addTODO{应用中各定理的证明需要稍后补充.}

\pentry{坡印廷矢量\upref{EBS},爱因斯坦求和约定\upref{EinSum}}

麦克斯韦应力张量可以用来电动力学中的力学运算.要注意的是,麦克斯韦张量和电磁场张量不是一个东西,后者是用来表示电磁场的具体取值的.和所有张量一样,麦克斯韦张量的坐标也依赖于坐标系的选择,而坐标系就等价于参考系,可以参见\textbf{电磁场张量}\upref{EMFT} 中的讨论.

\begin{definition}{麦克斯韦应力张量}
\textbf{麦克斯韦应力张量(Maxwell stress tensor)}是空间中的一个二阶张量场,记为$T^{ij}$.如果在某个观察者眼中,电场为$\pmat{E^x, E^y, E^z}\Tr$,磁场为$\pmat{B^x, B^y, B^z}\Tr$,那么在这个观察者眼中,麦克斯韦应力张量的坐标值为
\begin{equation}
T^{ij}=\epsilon(E^iE^j-\frac{1}{2}\sigma^{ij}E^2)+\frac{1}{\nu^0}(B^iB^j-\frac{1}{2}\sigma^{ij}B^2)
\end{equation}
其中$E^2=E^iE_i={E_x}^2+{E_y}^2+{E_z}^2$,且$B^2=B^iB_i=B_x^2+B_y^2+B_z^2$.
\end{definition}


\subsection{应用}

\begin{theorem}{电荷受力密度}
设空间中存在某一电荷分布和某一电磁场分布,电荷可能运动.那么每一个点处的电荷受力密度为
\begin{equation}
f^i=\nabla_jT^{ij}-\epsilon_0\nu_0\frac{\partial\bvec{S}}{\partial t}
\end{equation}
其中
\end{theorem}






