% 2016 年计算机学科专业基础综合全国联考卷
% keys 2016 计算机 考研 全国联考

\subsection{一、单项选择题}
1~40小题,每小题2分,共80分.下列每题给出的四个选项中.只有一个选项符合试题要求.

1.已知表头元素为c的单链表在内存中的存储状态如下表所示.
\begin{table}[ht]
\centering
\caption{第1题表}\label{CSN16_tab1}
\begin{tabular}{|c|c|c|}
\hline
地址 & 元素 & 链接地址 \\
\hline
1000H & a & 1010H \\
\hline
1004H & b & 100CH \\
\hline
1008H & c & 1000H \\
\hline
100CH & d & NULL \\
\hline
1010H & e & 1004H \\
\hline
1014H &   &  \\
\hline
\end{tabular}
\end{table}
现将f存放于1014H处并插入到单链表中,若f在逻辑上位于a和e之间,则a,e,f的“链接地址”依次是
