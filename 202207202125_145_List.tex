% 链表
% 链表|单链表|数据结构

链表是一种用于存储数据的链式数据结构,形如一条链子一样来连接元素,通常用于存储树和图.

与数组不同的是:数组是一种支持随机访问,但不支持在任意位置插入或删除元素的数据结构.但链表支持在任意位置插入或删除,但只能按顺序依次访问其中的元素.

\textbf{单链表:}

\begin{figure}[ht]
\centering
\includegraphics[width=14.25cm]{./figures/List_1.png}
\caption{单链表示意图} \label{List_fig1}
\end{figure}

可以看到,链表上每个结点都有三个值,分别为:下标、值和 $\text{next}$ 指针.

下标为 $3$ 的结点的下一个结点为空结点,所以 $\text{next}$ 值为 $-1$.

如果用 C++ 的指针和结构体来写链表的话,长成这样子:
\begin{lstlisting}[language=cpp]
struct List
{
    int value;
    Node *next;
};
\end{lstlisting}

一般在竞赛中很少用上面这种方式来实现链表,因为这种写法效率很低,所以这里我们来讲一下如何使用数组来模拟链表.