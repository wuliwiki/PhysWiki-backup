% 常系数线性微分方程
% keys 复数|complex number|微分方程|高阶微分方程|ordinary differential equation|解法|特征方程|欧拉方程

\pentry{指数函数(复数)\upref{CExp},线性微分方程的一般理论\upref{ODEb1}}

\subsection{实数轴上的复值函数}

本小节首先介绍一个很有用的概念,复值函数.

复值函数和复变函数是不一样的.复变函数中的“变”指变量,因此复变函数是复数域到复数域的映射;但复值函数的自变量还是实数,它只是实数域到复数域的映射而已.

由于任何复数都可以写成$a+\I b$的形式,其中$a, b\in\mathbb{R}$,因此一个复值函数可以看成是两个实值函数的组合.

设$f(x), g(x)$是某给定区间上的连续实值函数,那么复值函数$z(x)=f(x)+\I g(x)$的极限被定义为$\lim\limits_{x\to x_0}z(x)=\lim\limits_{x\to x_0}f(x)+\I\lim\limits_{x\to x_0}g(x)$.进一步,$z(x)$的导函数是$z'(x)=f'(x)+\I g'(x)$.

复数的指数由欧拉公式定义:对于实数$a, b$,有
\begin{equation}
\E^{a+\I b}=\E^a(\cos b+\I\sin b)
\end{equation}

如果$K$是一个常数复数,那么$\E^{Kt}$就是一个关于实变量$t$的复值函数.这个函数继承了实值函数$\E^{at}$的很多优良性质.

\begin{theorem}{}
设$K, K_i$是复常数,$z(t), z_i(t)$是区间$[a, b]$上可导的复值函数,那么我们有:
\begin{itemize}
\item $\frac{\dd }{\dd t}(K_1z_1(t)+)$
\end{itemize}
\end{theorem}























