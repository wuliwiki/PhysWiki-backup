% Berkeley-ECS 方法

\footnote{参考 Renate Pazourek Thesis Eq 3.36}要把最后的波函数 $\ket{\Psi(t_0)}$ 投影到精确散射态 $\ket{\varphi_{\alpha,E}}$ 上, 而避免计算精确散射态, 可以先解
\begin{equation}\label{BerECS_eq1}
(E-H)\ket{\Psi_{SC}(E)} = \ket{\Psi(t_0)}
\end{equation}
根据\autoref{LipSch_eq2}~\upref{LipSch}, 形式上这相当于计算
\begin{equation}
\ket{\Psi_{SC}(E)} = G^+(E)\ket{\Psi(t_0)}
\end{equation}
但这并不重要, 数值上仍然直接解非齐次线性方程组\autoref{BerECS_eq1} 即可.

下一步, 把末态投影到精确散射态上得
\begin{equation}\label{BerECS_eq2}
\begin{aligned}
&\quad \braket{\varphi_{\alpha,E}}{\Psi(t_0)} = \mel{\varphi_{\alpha,E}}{E-H}{\Psi_{SC}(E)}\\
&= \mel*{\varphi_{\alpha,E}}{\frac{\laplacian}{2}}{\Psi_{SC}(E)} + \mel*{\varphi_{\alpha,E}}{E-V}{\Psi_{SC}(E)}
\end{aligned}
\end{equation}
由于 $(-\laplacian/2+V)\varphi_{\alpha,E} = E\varphi_{\alpha,E}$
\begin{equation}
\mel*{\varphi_{\alpha,E}}{E-V}{\Psi_{SC}(E)} = -\braket*{\frac{\laplacian}{2} \varphi_{\alpha,E}}{\Psi_{SC}(E)}
\end{equation}

根据格林第二恒等式
\begin{equation}
\int (u\laplacian v - v\laplacian u)\dd{V} = \oint (u\grad v - v\grad u) \dd{\bvec S}
\end{equation}
可以把\autoref{BerECS_eq2} 写成面积分(注意 $\varphi_{\alpha,E}$ 是一个实函数)
\begin{equation}
\braket{\varphi_{\alpha,E}}{\Psi(t_0)} = \frac{1}{2} \oint [\varphi_{\alpha,E} \grad \Psi_{SC} - \Psi_{SC} \grad \varphi_{\alpha,E}] \dd{\bvec S}
\end{equation}
注意这里的矢量都是 6 维的.

另外由于这个面积分距离原子核很远, 可以用 $\mathrm{He}^+$ 束缚态和库仑平面波的对称化乘积来代替.
