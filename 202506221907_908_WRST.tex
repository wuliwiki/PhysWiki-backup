% 魏尔施特拉斯函数(综述)
% license CCBYSA3
% type Wiki

本文根据 CC-BY-SA 协议转载翻译自维基百科 \href{https://en.wikipedia.org/wiki/Weierstrass_function}{相关文章}。

\begin{figure}[ht]
\centering
\includegraphics[width=10cm]{./figures/a5c32c67ac8fb68b.png}
\caption{} \label{fig_WRST_1}
\end{figure}
在数学中,魏尔施特拉斯函数以其发现者卡尔·魏尔施特拉斯的名字命名,是一个实值函数的典型例子,它在所有点上都是连续的,但却在任何一点都不可导。它也是分形曲线的一个例子。

魏尔施特拉斯函数在历史上被视为一种“反常函数”,它是第一个被明确构造出来、用于挑战“每个连续函数除了在一些孤立点外都可导”这一观念的例子(发表于1872年)\(^\text{[1]}\)。魏尔施特拉斯通过这个函数展示了连续性并不意味着“几乎处处可导”,这一点颠覆了数学界的传统看法,也推翻了许多依赖几何直觉和模糊“光滑性”定义的证明。彼时的数学家普遍不喜欢这类函数:查尔斯·埃尔米特在发现自己研究的一类函数也具备这种性质时,曾称其为“令人悲哀的灾难”\(^\text{[2]}\)。在计算机出现之前,这类函数难以直观地想象,其结果也未被广泛接受,直到后来出现了实际应用场景,例如对布朗运动的建模,这类“无限锯齿状”的函数才变得必要(如今称为分形曲线)\(^\text{[3]}\)。
\subsection{构造方法}
在魏尔斯特拉斯的原始论文中,这个函数被定义为一个傅里叶级数:
$$
f(x) = \sum_{n=0}^{\infty} a^n \cos(b^n \pi x)~
$$
其中
$0 < a < 1$、
$b$ 是一个正的奇整数,且满足条件:
$$
ab > 1 + \frac{3}{2}\pi~
$$
对于满足上述条件的 $0 < a < 1$,使得不等式成立的最小整数 $b$ 是 $b = 7$。这一构造方式以及证明该函数在任意点处都不可导的论证,最早由魏尔斯特拉斯于 1872 年 7 月 18 日在普鲁士皇家科学院提交的论文中提出\(^\text{[4][5][6]}\)。
