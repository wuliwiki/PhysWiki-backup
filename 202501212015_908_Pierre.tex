% 皮埃尔·德·费马(综述)
% license CCBYSA3
% type Wiki

本文根据 CC-BY-SA 协议转载翻译自维基百科\href{https://en.wikipedia.org/wiki/Pierre_de_Fermat}{相关文章}。

\begin{figure}[ht]
\centering
\includegraphics[width=6cm]{./figures/645e6146de629a4b.png}
\caption{《皮埃尔·德·费马》,17世纪未知作者画作} \label{fig_Pierre_1}
\end{figure}
皮埃尔·德·费马(法语:[pjɛʁ də fɛʁma];1601年8月17日 – 1665年1月12日)是法国数学家,他因早期的研究成果而被认为是微积分学的奠基人之一,其中包括他使用的“等值法”(adequality)。特别地,他因发现了一种原始的方法,用于求解曲线的最大和最小纵坐标,这一方法与当时未知的微分学相类似。此外,他还对数论进行了深入研究,并在解析几何、概率论和光学方面作出了重要贡献。他最著名的贡献包括光传播的费马原理和费马大定理,后者是他在一份《丢番图算术》副本的空白页上所写的一条注释中提出的。他还是法国图卢兹议会的律师。