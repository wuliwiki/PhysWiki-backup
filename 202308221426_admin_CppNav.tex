% C++ 导航
% license Usr
% type Map


\begin{issues}
\issueDraft
\end{issues}

\subsection{C++ 可以做什么}
C++ 属于较为底层的语言。
\addTODO{一般来说,相当长一段时间内你都做不出任何图形界面,也就是你打交道的都是命令行输入和输出。}

\subsection{环境设置}

要学习 C++,首先需要选择一个用于编译和运行你的 C++ 程序的环境。 而选择环境的第一步是选择一个操作系统。 对于小时百科的绝大部分使用 C++ 的场景(主要是科学计算)来说,强烈推荐你使用 Linux\upref{Linux} 系统。 如果你一般使用 Windows 且不想装一个完整的操作系统,那么也可以在 Windows 中安装 Mingw-w64\upref{Mingw} 或者 WSL2\upref{WSLnt}, 又或者使用 VirtualBox\upref{VirBox} 虚拟机安装 Linux 系统。 三者所占用的资源从少到多,兼容性也从弱到强。

\subsubsection{简单环境}
一个强烈推荐的学习 C++ 的工具是 Cling\upref{Cling}。 它可以让你像使用 Python 一样互动地运行一些简单的 C++ 代码,可以逐行运行。 它可以在 Jupyter Notebook\upref{jupyNb} 中运行, 也同时提供一个命令行版本。 你可以通过 Jupyter Notebook 的\href{https://jupyter.org/}{官网}直接使用也可以在本地安装\upref{Cling}。

如果你不想在自己电脑上部署环境,那么网上也有一些在线的 C++ 编译器, 例如 \href{https://wandbox.org/}{Wandbox} 可以直接在服务器上编译运行简单的 C++ 代码。 你也可以使用一些算法题网站如 \href{https://leetcode.com/}{LeetCode} 或者\href{https://www.nowcoder.com/}{牛客网},它们甚至还提供逐行调试功能(下面会介绍)。但这种网站只能运行一些简单的测试代码,一般不支持安装和使用编译好的第三方库。

\subsubsection{Linux 命令行环境}
如果你属于相对硬核的用户,那么你使用 C++ 的整个过程中需要的就只需要一个文本编辑器(如 Visual Studio Code\upref{VScode} 甚至 vim\upref{Vim})以及一个类 Linux 命令行窗口(一般是 bash\upref{Bash})。 该做法的一个入门教程见 “在 Linux 上编译 C/C++ 程序\upref{linCpp}”, 你需要在上面安装一个 C++ 编译器, 比较流行的有 g++\upref{gpp}, clang++\upref{clangp} 和 icpc\upref{icpcNt}(前两个开源,icpc 闭源)。

\subsubsection{IDE:集成开发环境}
如果你还想要更多的功能,例如变量和函数的跳查,用\textbf{图形界面(GUI)}进行调试, 那么你需要一个 \textbf{IDE (Integrated Development Environment, 集成开发环境)}:

Visual Studio 是 Windows 的官方 IDE, 无法在其他系统中使用, 且社区版的 Visual Studio 免费。 老版本的 Visual Studio 只能使用 Windows 专有的 Visual C++ 编译器进行 C++ 开发, 但新版本的 Visual Studio 支持使用 WSL 中的其他编译器。 由于不能跨平台,我们不推荐使用。

另一个流行的 IDE 是 JetBrain 推出的 \href{https://www.jetbrains.com/clion/}{CLion}。 CLion 是跨平台的,在 Windows,Mac,Linux 上都能使用。 CLion 功能较多较完善,学生可以免费使用,但对其他用户收费。 我们推荐 CLion。

另外也有一些完全免费的 IDE 如 Eclipse,NetBeans,Code::Blocks,Qt Creator 等。 事实上不同的 IDE 在基本功能上使用起来大同小异。

IDE 的部署通常要比直接使用命令行麻烦得多。 它们通常需要通过 “工程” 来进行项目管理。

\subsubsection{调试程序}
如果你选择简单的文本编辑器和命令行的环境, 那么命令行调试器几乎是必备的, 详见 “调试 C++ 程序\upref{gdbcpp}”。 当然有许多不那么专业的用户选择完全不使用调试器。 g++ 和 icpc 编译器对应的调试器是 gdb\upref{gdbNt}, clang++ 编译器对应的是 lldb\upref{LLDB}, 二者的用法几乎一样。

\subsection{C++ 语法}
有了基本的环境以后, 就可以着手学习 C++ 语法了。 “C++ 基础\upref{Cpp0}” 中列出了小时百科最常用的一些语法。
\addTODO{。。。}

\subsection{构建工具}
\addTODO{Makefile 简介\upref{Make0}, Makefile 笔记\upref{Make},CMake 笔记\upref{CMakeN}}
