% 恒等式与恒成立不等式(高中)
% keys 恒等式|恒成立|不等式恒成立
% license Xiao
% type Tutor

\pentry{等式与不等式 \nref{nod_HsEquN}}{nod_1bc6}

\begin{issues}
\issueDraft
\end{issues}

在\enref{等式与不等式}{HsEquN}中已经介绍过等式和不等式的概念,本文将专注处理恒成立的等式和不等式。恒成立指的是在给定的变量取值范围中,不论取哪个值,等式(或不等式)均成立。尽管普通的不等式方程也可说在他们的解集上恒成立,但一般来讲对不等式而言恒成立要求的取值范围更广泛一些。

\subsection{恒等式}\label{sub_HsIden_1}

恒成立的等式称作\textbf{恒等式(identities)},某些时刻为了强调恒等关系会采用$\equiv$来代替$=$。

在高中阶段,常见的恒等式包括\enref{三角恒等式}{HsAnTf}、\enref{组合恒等式}{combin}以及代数运算中的恒等式,包括\textbf{二项式定理(Binomial theorem)}和\textbf{等幂和差公式(Sum and difference of powers)}。

\begin{theorem}{二项式定理}
对两个表达式$a$与$b$和正整数$n$有:
\begin{equation}
\displaystyle(a+b)^n=\sum_{k=0}^n{\mathrm C}_n^ka^kb^{n-k}~.
\end{equation}
其中,${\mathrm C}_n^k$称作\enref{组合数}{HsCb},${\mathrm C}_n^k={n!\over k!(n-k)!}$,$\sum$为\enref{求和符号}{SumSym}。两个常见的特例为:
\begin{itemize}
\item 完全平方:$(a\pm b)^2=a^2\pm2ab+b^2$
\item 完全立方:$(a\pm b)^2=a^3\pm 3a^2b+3ab^2\pm b^2$
\end{itemize}
\end{theorem}

完全平方和完全立方公式的$\pm$符号,可以认为是$a+(\pm b)$,这样就能解释拆开后的$\pm$出现的规律——它们都出现在$b$为奇次的项前面。

\begin{theorem}{等幂和差公式}
两个表达式$a$与$b$的幂次均为$n$($n\in\mathbb{N}$),则它们的差具有如下关系:
\begin{equation}
a^n - b^n = (a - b)(a^{n-1} + a^{n-2}b + \cdots + b^{n-1})~.
\end{equation}
若$n$为奇数,则它们的和具有如下关系:
\begin{equation}
a^n + b^n = (a + b)(a^{n-1} - a^{n-2}b + a^{n-3}b^2 - \cdots + b^{n-1})~.
\end{equation}
以下是两个常见特例:
\begin{itemize}
\item $n=2$时,称作\textbf{平方差公式}$a^2-b^2=(a+b)(a-b)$
\item $n=3$时,称作\textbf{立方和、差公式}$a^3\pm b^3=(a\pm b)(a^2\mp ab+b)$
\end{itemize}
\end{theorem}

这个公式在因式分解中相当常用,另外,偶数次的和没有分解方式。

\subsubsection{恒等式与公式、定义}

\textbf{公式(formula)}和\textbf{定义(式)(definition)}是两个常见且容易与恒等式混淆的概念。实际上,二者都属于恒等式的范畴,但侧重点和应用场景不同。

恒等式的本质在于揭示数学中的普遍规律,是一种真理性陈述。具体来说,定义式通过明确的等号,将数学对象与其性质或关系结合在一起。如果定义式不成立,就意味着相关定义已经发生改变。例如,小学时$\pi$被定义为圆的周长与直径之比,这一关系必须恒成立,才能使$\pi$的定义具有一致性。

相比之下,公式更关注变量之间的联系及其在特定情境中的应用。公式通常在约定的范围内成立,且一般应用的条件或范围也并不苛刻。以圆的面积公式$A = \pi r^2$为例,这不仅揭示了圆的面积与半径之间的关系,也为具体计算提供了有力工具。公式的主要作用在于描述变量间的依赖关系,并广泛应用于问题解决的过程中。

除了在数学中,公式在其他学科中同样具有重要价值。例如:
\begin{itemize}
\item 在力学中,牛顿第二定律公式$F = ma$描述了力、质量与加速度之间的关系;
\item 在电学中,欧姆定律$V = IR$揭示了电压、电流和电阻的依赖关系;
\item 在计算机科学中,算法复杂度公式$T(n) = an^2 + bn + c$用于衡量算法的时间效率。
\end{itemize}

综上所述,定义式提供了理论基础,确保数学对象的概念清晰和一致;而公式则更注重描述关系、解决问题的工具性。二者的不同点在物理领域将非常凸显。

\subsubsection{与函数相关的恒等式}

当说两个函数相等时,由于相等中包含定义域、对应关系和值域,因此,这时所说的相等是指恒等。
例如$f(x)=g(x)$,如果指恒等,那么这个操作是平凡的。而它一般就指的是条件等式,即方程,等式成立的条件就是两函数的交点。

恒等式意味着对定义区间上的任何点上的行为都是完全相同的,因此恒等式两侧进行求导、积分等运算均仍能保持恒等关系。而一般的方程(即条件等式)不可以通过这种方法进行运算。

恒成立的代数方程两侧对应参数相等。

当讨论两个函数相等时,需明确“相等”的具体含义。在数学中,函数相等通常意味着它们在定义域、对应关系以及值域上完全一致。若强调这种严格的一致性,即函数在定义域内每一点的对应值都相同,即恒等。

例如,
而对于 f(x) 和 g(x),若 f(x) = g(x) 是指恒等关系,那么这一表述是平凡的,因为它意味着 f(x) 和 g(x) 在定义域的每一点都完全一致。然而,数学中更常见的情况是涉及\textbf{条件等式(conditional equation)}。条件等式仅在特定条件或某些点上成立,此时,它通常指代一个\textbf{方程(equation)},即通过求解找出使等式成立的点,这些点通常对应于两个函数的交点。

恒等式的一个重要性质是,它在定义域的每一点都成立,因此可以对等式两侧进行求导、积分等运算,而不改变恒等关系。然而,对于一般的条件等式(即方程),这样的操作是不可行的,因为条件等式仅在特定点上成立,运算可能会破坏其成立的条件。

例如,若一个\textbf{代数方程(algebraic equation)}是恒成立的,那么方程两侧的对应参数在代数关系上也需要完全一致。这是恒等式的本质属性。


\subsection{不等式恒成立的条件}

与等式的情况相同,存在某些不等式对任意变量值都成立,此时称\textbf{不等式恒成立}。由于不等式恒成立经常做为题目背景给出,下面会在最值概念的基础上,解释一下常见的表述方式,以防看到某个不等式恒成立时,读者无法理解它背后想要传递的内容。

\subsubsection{最值}

\textbf{最值(extreme value)}是生活中一个极为常见的概念,用来描述一组可以比较的量的范围。无论是物理上的高度、温度,还是其他可量化的数值,确定其最值可以帮助更快地掌握这些量的变化或分布情况。在数学中,最值通常分为\textbf{最大值(maximum)}和\textbf{最小值(minimum)}。对于不同的对象,最值有多种定义和表示方法,但无论形式如何,其核心目标始终是为了确定这些量的范围。例如,两个数的最值运算是最基本的情况。生活中经常遇到类似的场景,例如比较两个朋友的身高。若想知道谁更高,需要求最大值,而若想知道谁更矮,则需要求最小值。

\begin{definition}{最值运算}
对于两个实数$a,b$,最大值运算$\max(a,b)$的结果定义为二者更大的值:
\begin{equation}
\max(a,b)=\begin{cases}
a,\qquad a\geq b\\
b,\qquad b>a
\end{cases}~.
\end{equation}
最小值运算$\min(a,b)$的结果定义为二者更小的值:
\begin{equation}
\min(a,b)=\begin{cases}
a,\qquad a\leq b\\
b,\qquad b<a
\end{cases}~.
\end{equation}
\end{definition}

关于对两个数进行的最值运算,有个蛮有趣的通用计算规则,或许可以帮助你理解它为什么是一个运算:
\begin{equation}
\begin{array}{c} 
\displaystyle\max(a, b) = \frac{a + b + |a - b|}{2} \\  
\displaystyle\min(a, b) = \frac{a + b - |a - b|}{2} \\  
\end{array}~.
\end{equation}

这个计算规则,利用了绝对值隐含的不等式性质。在最值运算中,可以将多个需要比较的项用括号括起来,比如$\max(1,2,3,4)$。然而,这种方式对项目较多的情形来说显得繁琐而冗长。为了解决这个问题,可以借助集合的概念,将多个元素集合化,然后直接对集合进行最值运算,表述为$\max\{a, b, c, d\}$,从而更直观地表达“从一组数中找最大值”的含义。于是有下面针对集合的最值。

\begin{definition}{集合的最值}
对于一个\textbf{全序(total order)}集$A$\footnote{\textbf{全序}指整个集合中,任意两个元素都可以比较大小。},集合的最值指的是最大或最小的元素,分别记作:
\begin{equation}
\max A\qquad\text{和}\qquad\min A~.
\end{equation}
\end{definition}

对于有限集合,最值的确定相对简单,可以通过比较直接得到。而当集合包含无限多个元素时,讨论最值的范围超出了高中数学的要求,但可以借助区间的概念进行直观理解。

如果一个集合是闭区间,例如$[a, b]$,由于端点$a$和$b$属于集合,因此集合在这两个方向上都存在最值,最小值为$a$,最大值为$b$。相反,对于开区间$(a, b)$,端点$a$和$b$并不属于集合。根据\aref{实数的稠密性}{sub_HsFunB_1},可以在任意给定的最值与端点之间找到其他满足条件的最值,因此无法明确确定一个最值。这种情况下,称集合的最值不存在。\footnote{值得注意的是,对于开区间$(a, b)$,虽然最值不存在,但端点$a$和$b$在数学上被称为区间的\textbf{确界(supremum和infimum)}确界是闭区间和开区间的统一描述,用于研究集合的上界与下界性质。确界并不一定属于集合,但它是所有集合元素的上界或下界中“最紧密的界”。例如,对于$(a, b)$,$a$是其下确界(infimum),$b$是其上确界(supremum)。上面的概念会在大学接触。}

除了区间之外的集合,通常会由函数给出,于是得到了函数的最值。

\begin{definition}{函数的最值}
对定义在$D$上函数$f(x)$,若存在$x_0\in D$使得$f(x_0)$是$f(x)$的所有值中最大的,则称$f(x_0)$为函数的最大值,记作:
\begin{equation}
f(x_0) = \max_{x \in D} f(x)~.
\end{equation}
反之,若是最小的,则称$f(x_0)$为函数的最小值,记作:
\begin{equation}
f(x_0) = \min_{x \in D} f(x)~.
\end{equation}
\end{definition}

二者其实对应的就是函数图像中的最高点和最低点。关于函数的最值,更多的研究会在函数部分展开。

\subsubsection{不等式恒成立条件}

若给定集合$A$,在$A$上函数满足:
\begin{equation}
\displaystyle\max_{x\in A} f(x)\leq 0\qquad\text{或}\qquad\min_{x\in A} f(x)\geq 0~.
\end{equation}
,则称$f(x)\leq 0$或$f(x)\geq 0$在$A$上恒成立。可以这么理解,如果一组数里面最大的都是负数,那么剩下的也都是负数,而反过来,如果一组数里面最小的都是正数,那么剩下的也都是正书。

$f(x)\leq g(x)$或$f(x)\geq g(x)$成立的充要条件是函数$F(x)=f(x)-g(x)$满足之前的恒成立条件。如果想要各自考虑$f(x),g(x)$,而不是上面的方法。那么$\max f(x)\leq\min g(x)$或$\min f(x)\geq\max g(x)$对于恒成立而言是充分不必要条件。以第一个为例,成立的原因是$f(x)\leq\max f(x)\leq\min g(x)\leq g(x)$,这是一种严格的成立,就像二者中间有一个区域,谁都不越雷池一步。
\addTODO{说明图,一个是彼此交错但不超过,一个是中间严格不大于。}

这里容易搞混符号,注意,$\min f(x)\leq\min g(x)$或$\max f(x)\geq\max g(x)$对于恒成立而言既不充分也不必要,因为这只描述了一个值的关系,可能只有一部分点满足了大小关系,而其余则相反。

\addTODO{说明图,瘦高矮胖的正态分布}

另外如果$g(x)=c$是常数,则之前的成立条件也可认为是:若在$A$上函数满足:
\begin{equation}
\displaystyle\max_{x\in A} f(x)\leq a\qquad\text{或}\qquad\min_{x\in A} f(x)\geq b~.
\end{equation}
,则称$f(x)\leq a$或$f(x)\geq b$在$A$上恒成立。

一般出现在题目中时,往往会给出一个含参的函数,如$f(x;a,b,c)=ax^2+bx+c$等。这时需要根据题目的条件,利用上面的成立条件来推导。下面给出一些常用的不同的函数或运算满足的恒成立的不等式,请在使用时注意他们成立的前提条件和取等条件:
\begin{table}[ht]
\centering
\caption{常见的恒成立不等式}\label{tab_HsIden1}
\begin{tabular}{|c|c|c|}
\hline
成立前提 & 不等式 & 取等条件 \\
\hline
$x\in\mathbb{R}$&$|x|\geq0$&$x=0$ \\
\hline
$x\in\mathbb{R}$&$x^{2n}\geq0\quad(n\in\mathbb{Z})$&$x=0$\\
\hline
$x\in[0,+\infty)$&$x^{1\over2n}\geq0\quad(n\in\mathbb{Z})$&$x=0$\\
\hline
$x\in\mathbb{R}$,$b^2-4ac\leq0$且$a>0$\footnote{大部分情况下也可以由$c>0$来代替$a>0$进行判定。}&$ax^2+bx+c\geq0$&$b^2-4ac=0$且$\displaystyle x=-{2a\over b}$\\
\hline
$x\in\mathbb{R}$,$b^2-4ac\leq0$且$a<0$\footnote{大部分情况下也可以由$c<0$来代替$a<0$进行判定。}&$ax^2+bx+c\leq0$&$b^2-4ac=0$且$\displaystyle x=-{2a\over b}$\\
\hline
$x\in\mathbb{R}$&$a^x>0\quad(a\in\mathbb{R}^+)$&-\\
\hline
$x\in\mathbb{R}$&$|\sin x|\leq 1$&$\displaystyle x={\pi\over2}+k\pi\quad (k\in\mathbb{Z})$\\
\hline
$x\in\mathbb{R}$&$|\cos x|\leq 1$&$x=k\pi\quad (k\in\mathbb{Z})$\\
\hline
$\displaystyle x\in[0,{\pi\over2})$&$\sin x\leq x\leq\tan x$&$x=0$\\
\hline
\end{tabular}
\end{table}

% $\displaystyle x\in[-{3\pi\over4}+2k\pi,{\pi\over4}+2k\pi]\quad (k\in\mathbb{Z})$&$\sin x\leq\cos x$&$\displaystyle x={\pi\over4}+k\pi\quad (k\in\mathbb{Z})$\\
% \hline
% $\displaystyle x\in[{\pi\over4}+2k\pi,{5\pi\over4}+2k\pi]\quad (k\in\mathbb{Z})$&$\sin x\geq\cos x$&$\displaystyle x={\pi\over4}+k\pi\quad (k\in\mathbb{Z})$\\

还有一种常见的情况,选取$g(x)$为$f(x)$的某条切线,此时,切点往往成为取等的临界条件。比如$\forall x\in\mathbb{R},e^x\geq x+1$当且仅当$x=0$时成立,又如$\forall x\in\mathbb{R}^+,\ln x\leq x-1$当且仅当$x=1$时成立。这往往会成为考题的考点。

\subsection{基本不等式}

\textbf{基本不等式}是基于$(a-b)^2\geq0$得到的一组不等关系。由于$(a-b)^2\geq0$在实数域恒成立,因此由此推广得到的不等式也是在实数域恒成立的,而在复数域则不具备这种关系。推广的过程是通过向$a,b$赋值实现的,而推广后由于不等号两边的表达式均为某种均值的形式,因此基本不等式也称为\textbf{均值不等式},而由于存在多种均值,也有人称其为\textbf{(均值)不等式链}。

\subsubsection{基本不等式的简易形式}

下面给出的是高中要求的常见形式:

\begin{theorem}{基本不等式(简易)}
\begin{equation}
{a^2+b^2\over2}\geq ab\qquad\text{或}\qquad{a+b\over2}\geq \sqrt{ab}~.
\end{equation}
当且仅当$a=b$时取等。
\end{theorem}
证明过程是显然的,将$(a-b)^2\geq0$打开并移项就可以得到,或后的写法也只是将$a,b$替换成了$\sqrt{a},\sqrt{b}$。但他背后的意义是深远的,一般称$\displaystyle a+b\over2$为\textbf{算术平均数(Arithmetic mean,或算数均值)},也就是通常生活中提到平均值时所指的数,称$\sqrt{ab}$为\textbf{几何平均数(Geometric mean,或几何均值)}。几何平均数通常在计算增长率时使用,比如投资某个项目,第一年的收益率是$4\%$,第二年是$10\%$,则两年的平均收益率并非$\displaystyle {104\%+110\%\over2}-1=7\%$,而是$\sqrt{1.04\times1.10}-1=6.96\%$。根据基本不等式,如果直接计算算术平均数,则总会高估实际的收益率。

同时,根据$a^2+b^2\geq2ab$可以得到$2a^2+2b^2\geq a^2+2ab+b^2=(a+b)^2$,两侧同时开方则有:
\begin{equation}
\sqrt{a^2+b^2\over2}\geq{a+b\over2}~.
\end{equation}
而将$a^2+b^2\geq2ab$中的$a,b$替换成$\displaystyle\sqrt{1\over a},\sqrt{1\over b}$,则有$\displaystyle{1\over a}+{1\over b}\geq{2\over\sqrt{ab}}$,即:
\begin{equation}
{2\over\displaystyle{1\over a}+\displaystyle{1\over b}}\leq\sqrt{ab}~.
\end{equation}
其中,$\displaystyle \sqrt{a^2+b^2\over2}$称为\textbf{平方平均数(Quadratic mean,或平方均值)},它的特点是越大的值影响越大,或者说给予较大的值更高的权重。通常用在一些$0$附近的数上,来反映波动程度。$\displaystyle{2\over\displaystyle{1\over a}+\displaystyle{1\over b}}$称为\textbf{调和平均数(Harmonic mean,或调和均值)},通常用来计算等量变化时的速率或比率,比如一段路一半以$60{\rm km/h}$的速度运动,另一半以$30{\rm km/h}$的速度运动,那么平均速度是$\displaystyle \frac{2}{\displaystyle\frac{1}{60} + \frac{1}{30}} = 40{\rm km/h}$,而非$\displaystyle{60+30\over2}=45{\rm km/h}$。

注意,本节出现的$a,b$都是变量,因此可以给其赋予某个数或某种函数,例如取$\displaystyle a=x,b={1\over x}$,此时有:$\displaystyle x+{1\over x}\geq2,\quad(x>0)$


\subsubsection{*不等式链}

下面的内容高中目前不再涉及,作为开阔视野。这四种均值都有广泛的应用\footnote{此处只给出这四种,其实有很多种均值的定义,它们也都可纳入下面的均值不等式中。}。事实上,上面的四种平均数都可以扩展至多个数,对于一列数$a_1,a_2,\cdots,a_n$,有如\autoref{tab_HsIden2} 所示的均值形式。

\begin{table}[ht]
\centering
\caption{常见平均数(或均值)}\label{tab_HsIden2}
\begin{tabular}{|c|c|c|c|}
\hline 名称 & 常用字母 & 表达式 & 与数列的关系 \\
\hline
平方平均数\footnote{也称作\textbf{均方根(root mean square,RMS)}}&$Q$ & $\displaystyle \sqrt{\frac{a_1^2 + a_2^2 + \cdots + a_n^2}{n}}$ & $\displaystyle \sum_{i=1}^nQ^2= \sum_{i=1}^na_i^2$ \\
\hline
算术平均数&$A$ & $\displaystyle \frac{a_1 + a_2 + \cdots + a_n}{n}$ &$\displaystyle \sum_{i=1}^nA= \sum_{i=1}^n{a_i}$ \\
\hline
几何平均数&$G$ & $\displaystyle \sqrt[n]{a_1 \cdot a_2 \cdot \cdots \cdot a_n}$ &$\displaystyle \prod_{i=1}^nG= \prod_{i=1}^n{a_i}$ \\
\hline
调和平均数&$H$ &  $\displaystyle \frac{n}{\frac{1}{a_1} + \frac{1}{a_2} + \cdots + \frac{1}{a_n}}$ & $\displaystyle \sum_{i=1}^n\frac{1}{H}= \sum_{i=1}^n\frac{1}{a_i}$ \\
\hline
\end{tabular}
\end{table}

在\autoref{tab_HsIden2} 中,有一列展示了不同均值与数列的关系。通过分析这一列的表达式,可以清晰地理解每种均值的具体形式是如何得出的。这些表达式揭示了每种均值的计算方法和背后的数学结构,同时也提示了不同均值适用的情境。\autoref{the_HsIden_1} 是从之前的简单形式推广得到的,此处不加证明地给出。

\begin{theorem}{均值不等式}\label{the_HsIden_1}
对于一列数$a_1,a_2,\cdots,a_n$,有
\begin{equation}
Q\geq A\geq G\geq H~.
\end{equation}
即
\begin{equation}
\displaystyle\sqrt{\frac{\displaystyle\sum_{i=1}^na_i^2}{n}}\geq \frac{\displaystyle\sum_{i=1}^na_i}{n}\geq \sqrt[n]{\prod_{i=1}^na_i}\geq \frac{n}{\displaystyle\sum_{i=1}^n\frac{1}{a_i}}~.
\end{equation}
当且仅当$\forall 0\leq i,j\leq n,a_i=a_j$时取等。
\end{theorem}


\subsection{*排序不等式}

假设有两列数字$a_1, a_2, \dots, a_n$和$b_1, b_2, \dots, b_n$。当满足条件$\forall 1 \leq i < j \leq n,  a_i \leq a_j \land b_i \leq b_j$时,称这两列为\textbf{顺序排列(increasingly ordered)},即两个序列的增长方向一致。当满足$\forall 1 \leq i < j \leq n,  a_i \leq a_j \land b_i \geq b_j$时,则称为\textbf{逆序排列(decreasingly ordered)},表示两个序列的增长方向相反。若不满足以上任一条件的排列方式,则称为\textbf{乱序排列(unordered)},即增长方向无规则。

基于上述定义,可以使用排序不等式来比较不同排列的和的大小关系。排序不等式表明:逆序和 $\leq$ 乱序和 $\leq$ 顺序和。这一不等式的证明基于数学归纳法和两非正数之积非负的性质,因此排序不等式的使用范围也与均值不等式相同。事实上,可以使用排序不等式证明基本不等式、柯西不等式、切比雪夫不等式等其他恒成立的不等式。它的规范表述如下:

\begin{theorem}{排序不等式}
设两列数字 $a_1, a_2, \cdots, a_n$ 和 $b_1, b_2, \cdots, b_n$,两列数分别满足 $a_1 \leq a_2 \leq \cdots \leq a_n$ 和 $b_1 \leq b_2 \leq \cdots \leq b_n$。那么有如下不等式:
\begin{equation}
\sum_{i=1}^n a_i b_i \geq \sum_{i=1}^n a_i b_{\sigma(i)} \geq \sum_{i=1}^n a_i b_{n+1-i}~,
\end{equation}
其中 $\sigma$ 表示对 ${1, 2, \cdots, n}$ 的非顺序、逆序的任意排列,当且仅当$\forall 1\leq i,j\leq n,a_i=a_j$或$\forall 1\leq i,j\leq n,b_i=b_j$时,等号成立。
\end{theorem}

由于证明方法只使用了高中常见的方法,下面给出证明过程。

\begin{lemma}{若$a\geq b,m\geq n$,则$am+bn\geq an+bm$}\label{lem_HsEquN_1}
证明:

取$a=b+k,m=n+p$,则$k,p\geq0$,代入有:
\begin{equation}
\begin{split}
am+bn&= (b+k)m+(a-k)n \\ 
&= bm+an+kp \\
&\geq an+bm
\end{split}~.
\end{equation}
或者由$a\geq b,m\geq n$有$a-b\geq 0,m- n\geq0$,从而$(a-b)(m- n)\geq0$,打开整理即可得到结论。上面两种方法都可以知道,原不等式当且仅当$a=b$或$m=n$时取等。
\end{lemma}

下面使用数学归纳法证明排序不等式中顺序和不小于乱序和,逆序部分同理:

1. 当$n=1$时显然成立;

2. 假设当$n$时,任意乱序$\sigma(i),(1\leq i\leq n)$时不等式均成立。当取$n+1$时,$\forall 1\leq i\leq n$有$a_{n+1}>a_i,b_{n+1}>b_i$,从而代入假设有:

\begin{equation}
\begin{split}
\sum_{i=1}^{n+1} a_i b_i&=\sum_{i=1}^n a_i b_i+a_{n+1}b_{n+1}\\
&\overset{\mathrm{1}}{\geq} \sum_{i=1}^n a_i b_{\sigma(i)} +a_{n+1}b_{n+1} \\ 
&\overset{\mathrm{2}}{\geq} \sum_{i=1}^{n+1} a_i b_{\sigma(i)} \\
\end{split}~.
\end{equation}
其中:第一个$\geq$根据假设,成立,对应$1$种乱序的情况;第二个$\geq$是交换了某个$b_{\sigma(i)}$和$b_{n+1}$,此时根据\autoref{lem_HsEquN_1} 可知,必有$a_i b_{\sigma(i)}+a_{n+1}b_{n+1}\geq a_ib_{n+1} +a_{n+1}b_{\sigma(i)}$,分别对应$n$个乱序的情况。综上,对某个$\sigma(i)$乱序的前提下,添加$a_{n+1}b_{n+1}$后新生成的$n+1$种乱序情况均成立。由于假设任意乱序$\sigma(i),(1\leq i\leq n)$均成立,因此对任意的新乱序$\sigma(i),(1\leq i\leq n+1)$不等式均成立。

由数学归纳法,排序不等式恒成立。

证毕。

\subsection{*柯西不等式}

\pentry{向量\nref{nod_GVec},几何向量的点乘\nref{nod_Dot}}{nod_fdd1}

\textbf{柯西-施瓦茨不等式(Cauchy-Schwarz Inequality)},也称\textbf{柯西不等式},为两个向量的内积(inner product)与它们的模长(norm)之间建立了重要的不等关系。柯西不等式是根据“内积”这种运算的定义推知的基本属性,因此所有满足“内积”定义要求的运算,或者说可以称为“内积”的运算,都会满足这个不等式\footnote{关于内积更详细的内容可以参考\enref{内积、内积空间}{InerPd}。}。

\begin{theorem}{柯西-施瓦茨不等式}
对于具备内积运算的任意两个向量$\bvec{u},\bvec{v}$,有
\begin{equation}
|\bvec{u} \cdot \bvec{v}| \leq (\bvec{u}\cdot\bvec{u})(\bvec{v}\cdot\bvec{v})=\|\bvec{u}\|\cdot\|\bvec{v}\|~.
\end{equation}
,当且仅当两个向量平行时取等。其中$\|\bvec{u}\|$是与内积对应的\textbf{范数(norm)}。
\end{theorem}

鉴于这部分是为开阔视野,上面采用了比较严谨的表述,但涉及到很多高中未曾了解的知识。不必担心,在高中阶段由于只涉及到几何向量的运算,这时的内积指的就是$\bvec{u}\cdot\bvec{v}=|\bvec{u}|\cdot|\bvec{v}|\cos\theta$,而所谓对应的“范数”指的就是向量的“模长”$|\bvec{u}|$。其实这里根据 $\cos \theta$ 的取值范围,显然可以得到结论。等号成立的条件是当 $\theta = 0$ 或 $\theta = \pi$ 时,即此时两个向量平行。

如果以高中常见的二维向量为例,利用坐标将两个向量表示出来,即$\bvec{u}=(a,b),\bvec{v}=(c,d)$,就得到了下面的表达方法:
\begin{equation}
|ac+bd|\leq\sqrt {a^2 + b^2}\cdot\sqrt{c^2 + d^2}\quad\text{或} \quad(ac+bd)^2\leq(a^2 + b^2)\cdot(c^2 + d^2)~.
\end{equation}
这种表示方法在高中的题目中更为常见。

对于多维情况,即向量的坐标是由一列数构成的时,有:

\begin{equation}
\left( \sum_{i=1}^{n} a_i b_i \right)^2\leq\left( \sum_{i=1}^{n} a_i^2 \right) \left( \sum_{i=1}^{n} b_i^2 \right) ~.
\end{equation}

这个不等式不仅在几何中起作用,在数学分析、线性代数、概率等领域都非常重要。
