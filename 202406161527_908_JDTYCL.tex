% 经典统一场论
% license CCBYSA3
% type Wiki

(本文根据 CC-BY-SA 协议转载自原搜狗科学百科对英文维基百科的翻译)

自19世纪以来,一些物理学家,特别是阿尔伯特·爱因斯坦,一直试图创建一个单一的理论框架来解释自然的所有基本作用力——一个统一的场论。经典统一场论是在经典物理基础上建立统一场论的尝试。特别是,在两次世界大战之间的几年里,几位物理学家和数学家积极追求引力和电磁学的统一。这项工作刺激了微分几何的纯数学发展。

\subsection{概观}

创建统一场论的早期尝试始于广义相对论的黎曼几何,并试图将电磁场纳入更一般的几何,因为普通的黎曼几何似乎无法表述电磁场的性质。爱因斯坦并不是唯一一个试图统一电磁和引力的人;包括赫尔曼·韦勒、阿瑟·爱丁顿和西奥多·卡鲁扎在内的大量数学家和物理学家也试图开发能够统一这些相互作用的方法。[1][2] 这些科学家探索了几种推广途径,包括扩展几何基础和增加额外的空间维度。

\subsection{早期工作}

米氏在1912年和恩斯特·赖欣巴赫在1916年首次尝试提供统一的理论。[3][4] 然而,这些理论并不令人满意,因为它们没有包含广义相对论,因为广义相对论尚未形成。连同鲁道夫·福斯特的这些尝试都未成功,包括把度规张量(以前被认为是对称的和实值的)变成不对称的和/或复值的张量,他们还试图为物质创造场论。

\subsection{微分几何和场论}

从1918年到1923年,场论有三种截然不同的方法:韦勒的规范理论、卡鲁扎的五维理论和爱丁顿对仿射几何的发展。爱因斯坦与这些研究人员通信,并与卡鲁扎合作,但尚未完全参与统一的工作。

\subsection{韦勒的微分几何}

为了将电磁学纳入广义相对论的几何范畴,赫尔曼·韦勒致力于推广广义相对论所依据的黎曼几何。他的想法是创造一个更一般的微分几何。他指出,除了度量场之外,在流形中两点之间的路径上还可能有额外的自由度,他试图通过引入一种基本方法来利用这一点,依据规范场来比较沿着这一路径的局部尺寸测量值。这个几何推广了黎曼几何,因为除了公制$g$之外,还有一个向量场$Q,它们共同产生电磁场和引力场。这个理论尽管很复杂,在数学上却是合理的,但却导致了困难的高阶场方程。这个理论中的关键数学成分,拉格朗日元和曲率张量,是由韦勒和他的同事们计算出来的。然后韦勒就其物理有效性与爱因斯坦和其他人进行了广泛的通信,最终发现该理论在物理上是不合理的。然而,韦勒的规范不变性原理后来以一种修正的形式应用于量子场论。

\subsection{卡鲁扎的第五维度}

卡鲁扎统一的方法是将时空嵌入一个由四维空间和一维时间组成的五维圆柱世界。与韦勒的方法不同,黎曼几何被保留下来,额外的维度允许电磁场矢量结合到几何中。尽管这种方法在数学上相对优雅,但在爱因斯坦和他的助手格罗姆的合作下,我们确定了这种理论不允许非奇异的静态球对称解。这一理论确实对爱因斯坦后来的工作产生了一些影响,后来克莱因进一步发展了这一理论,试图将相对论纳入量子理论,即现在所说的卡鲁扎-克莱因理论。