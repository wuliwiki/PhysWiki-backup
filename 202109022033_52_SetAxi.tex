% 集合(公理化)
% 集合|公理|公理系统
\begin{issues}
\issueTODO
\end{issues}

\pentry{集合\upref{Set}}
\subsection{产生原因}
通常来说,我们都采取朴素集合论的观点,认为集合是一个最基本的数学概念,不需要严格的定义.但是,\textbf{罗素悖论(antinomy of Russell)}使得这一观念受到挑战.罗素悖论可以叙述为:已知对于任何一个集合,都能判定自己是否属于这个集合.所以,我们可以将不属于自己的集合收集起来构成一个集合.因此就应该存在一个集合$\mathcal{A}=\{A|A\notin A\}$,其中$A$是一个集合.但是,如果认为$\mathcal{A}\in\mathcal{A}$,那么根据定义,$\mathcal{A}\notin\mathcal{A}$;反过来,如果认为$\mathcal{A}\notin\mathcal{A}$,根据定义又有$\mathcal{A}\in\mathcal{A}$.这就构成一个悖论.

罗素悖论的存在使人们认识到,朴素集合论并不像他们想象中的那么严谨.为了解决罗素悖论,人们利用公理对集合进行一系列限制,从而使得在新的公理系统中无法构造出罗素悖论中的集合.

\subsection{ZF公理系统}
ZF公理系统中认为所有的元素都可以看作是集合.区分它们可以用一个简单的方法,看“$\in$” “$\notin$”符号,左边应看作元素,右边应看作集合.

下面是ZF公理系统的全部公理\footnote{由于ZF公理系统认定所有的元素都是集合,所以下面存在一些容易引起混淆的地方,比如集族也被称为集合.}:

\textbf{公理1(外延公理)} 一个集合完全由其元素决定.如果两集合所有元素相等,则这两个集合相等.

\textbf{公理2(空集合存在公理)} 存在一个集合$\varnothing$,使得对于任何元素$x$,$x\notin\varnothing$.

\textbf{公理3(无序对公理)} 对于任意两个集合$x,y$,存在一个集合$A$,使得对于任意$w\in A$,$w=x$或$w=y$.

\textbf{公理4(并集公理)}对于任意一个集合$A$,存在一个集合$B$使对于任意$x\in B$,存在一个集合$y\in A$使$x\in y$.

\textbf{公理4(幂集公理)} 对任意集合$x$,存在集合$y$,使$z∈y$当且仅当对$z$的所有元素$w$,$w\in x$.

\subsection{选择公理和ZFC公理系统}

\subsection{连续统假设}



