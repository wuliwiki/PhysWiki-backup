% Linux 网络基础笔记

\begin{issues}
\issueDraft
\end{issues}

\begin{itemize}
\item \verb`sudo ifconfig` 显示网卡信息, 以及 ip 地址 (inet addr)
\item \verb`ifconfig 网卡名 up` 启动网卡, \verb`ifconfig 网卡名 down` 关闭网卡, 可以用这两个命令重启网卡
\item \verb`netplan` 是 ubuntu 18.04 的默认管理网络设置的程序, 比如设置 hdcp, 静态 ip, 掩码, 网关等, 设置完成以后用 \verb`sudo netplan apply` 可以生效. 但有时候还需要重启网卡才能成功.
\item \verb`sudo hpclient` 显示 ip 地址
\item \verb`sudo /etc/init.d/networking restart` 重启网络服务 
\item 要从某个网址下载文件, 只需安装 wget 软件即可, 使用方法如 \verb`wget http://...` 文件会下载到当前文件夹
\item \verb`ping` 可以用来检查是否有网络连接, 比如 \verb`ping google.com` 也可以用来查看某个域名的 ip 地址, 也可以直接使用 ip 地址如 \verb`ping 8.8.8.8` (8.8.8.8 是谷歌的主要 DNS 服务器)
\item 如果连不上网, 可以参考\href{https://upcloud.com/community/tutorials/troubleshoot-network-connectivity-linux-server/}{这篇文章}的步骤调试
\item 如果想 ping 某个端口, 用 \verb|telnet IP 端口号|
\end{itemize}

\subsection{iptables}
\begin{itemize}
\item \verb|iptables| 是一个命令行防火墙
\item table 的种类有 \verb|FILTER|, \verb|NAT|, \verb|MANGLE|, 每种又有不同的 chain, 例如 \verb|INPUT|, \verb|OUTPUT|, \verb|FORWARD|,  \verb|PREROUTING|, \verb|POSTROUTING| 等.
\item \verb|systemctl start iptables|, \verb|systemctl stop iptables|, \verb|systemctl restart iptables|
\item 查看所有防火墙规则 \verb|iptables -L -n -v|
\item 查看某个 table 的规则用 \verb|-t|, 如 \verb|iptables -t nat -L -v -n|
\item 屏蔽某个 ip: \verb|iptables -A INPUT -s xxx.xxx.xxx.xxx -j DROP|
\item 屏蔽某个 ip 的 tcp traffic: \verb|iptables -A INPUT -p tcp -s xxx.xxx.xxx.xxx -j DROP|
\item 取消屏蔽 ip: \verb|iptables -D INPUT -s xxx.xxx.xxx.xxx -j DROP|
\item 屏蔽一个 OUTPUT 端口: \verb|iptables -A OUTPUT -p tcp --dport xxx -j DROP|
\item 允许一个 INPUT 端口:  \verb|iptables -A INPUT  -p tcp --dport xxx -j ACCEPT|
\item \verb|iptables -A INPUT  -p tcp -m multiport --dports 22,80,443 -j ACCEPT|
\item \verb|iptables -A OUTPUT -p tcp -m multiport --sports 22,80,443 -j ACCEPT|
\end{itemize}
