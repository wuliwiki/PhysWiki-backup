% 用狄拉克 delta 函数表示点电荷的散度
% 点电荷|散度|无穷大|狄拉克 delta 函数

\pentry{电场的高斯定律证明\upref{EGausP}, 多元狄拉克 delta 函数\upref{deltaN}}

\footnote{参考 \cite{GriffE}.}在电场的高斯定律证明\upref{EGausP}中, 我们看到在通常意义下, 如果取一个包含点电荷的闭合高斯面, 那么散度定理的积分形式(\autoref{Divgnc_eq1}~\upref{Divgnc})并不能直接使用, 因为在点电荷处电场的散度没有定义, 而在曲面内的其他位置电场的散度处处为零.

但是我们可以先把每个点电荷替换成一个半径为 $R$ 的均匀带电小球, 然后令 $R\to 0$ 即可. 考虑一个在原点的均匀带电小球, 半径为 $R$, 电荷为 $Q$, 电荷密度为常数 $\rho = Q/V$, $V = 4\pi R^3/3$. 我们已知小球之外电场散度处处为零, 而小球内部根据高斯定律的微分形式(\autoref{EGauss_eq1}~\upref{EGauss}) 有
\begin{equation}\label{CEfDiv_eq1}
\div \bvec E = \frac{\rho}{\epsilon_0} = \frac{Q}{\epsilon_0 V} = \frac{3Q}{4\pi \epsilon_0 R^3}
\end{equation}
可见电场散度在小球内为常量, 所以散度在包含小球的高斯面内的体积分就是 $\div \bvec E$ 乘以球体体积 $V$ 得 $Q/\epsilon_0$, 这就验证了高斯定律的积分形式(\autoref{EGauss_eq2}~\upref{EGauss}).

接下来当我们令 $R\to 0$ 的时候, 小球的半径越来越小, 而电荷密度却越来越大. 所以取极限时, 电荷密度并不是一个通常意义的函数, 而狄拉克 delta 函数\upref{deltaN}就是专门用来描述这种密度极限的数学工具. 由于三维的狄拉克函数的体积分等于 1($\int \delta(\bvec r)\dd{V} = 1$), 而\autoref{CEfDiv_eq1} 的体积分等于 $Q/\epsilon_0$, 所以我们可以令
\begin{equation}
\div \bvec E = \frac{Q\delta(\bvec r)}{\epsilon_0}
\end{equation}
或者
\begin{equation}
\rho = Q\delta(\bvec r)
\end{equation}
这样, 在点电荷所在处, 高斯定律的微分形式就同样 “成立” 了.

若我们把电场直接写出来, 有
\begin{equation}
\div \bvec E = \frac{Q}{4\pi\epsilon_0}\div \frac{\uvec r}{r^2} = \frac{Q\delta(\bvec r)}{\epsilon_0}
\end{equation}
两边消去 $Q/\epsilon_0$, 有
\begin{equation}\label{CEfDiv_eq2}
\div \frac{\uvec r}{r^2} = 4\pi\delta(\bvec r)
\end{equation}
这样, 我们就得到了矢量场 ${\uvec r}/{r^2}$ 在 $\bvec r = \bvec 0$ 处的, 原本没有定义的散度. 如果仅看其本身, 我们只能模糊地说矢量场的散度在原点处一个无穷小的体积内无穷大, 并没有太大意义. 该式的意义在于对它做体积分可以得到确切的值.
