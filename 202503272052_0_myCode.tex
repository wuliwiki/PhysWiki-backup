% 小时百科服务器开发框架
% license Xiao
% type Tutor

\subsection{概述}
网页版百科分为两个界面:
\begin{itemize}
\item \textbf{原界面} \verb`wuli.wiki/online`,以html格式存储的文章,每篇文章都有唯一的ID,例如:文章 \verb`wuli.wiki/online/AU.html` 中的`AU`。 
\item \textbf{新界面} \verb`wuli.wiki/book`,使用数据库存储的文章,并添加了评论、搜索等功能。文章内容全部来自原界面,定时爬取。文章在新界面的链接为 \verb`wuli.wiki/book/AU`。
\item \textbf{文章编辑器} \verb`wuli.wiki/editor`,一个LaTeX编辑器,文章作者使用它来写作文章,编辑器会将文章转换为 HTML 格式,并发布到原界面。
\end{itemize}
数据流向:文章编辑器=>原界面=>新界面

所有代码通过 GitHub 账号 \href{https://github.com/wuliwiki}{wuliwiki} 管理(更新,现在主要通过 \href{https://git.wuli.wiki}{自建 gitlab})。 以下是各个仓库的功能
\begin{itemize}
\item \href{https://github.com/wuliwiki/littleshi.cn}{littleshi.cn}(不公开): 网站中所有静态页面, 服务器路径 \verb`/var/www/littleshi.cn`, 其中的 \verb`online` 子目录就是所有文章的静态 html 放置的地方, 通过 url \verb`wuli.wiki/online/xxx.html` 直接访问。
\item \href{https://github.com/wuliwiki/PhysWiki}{PhysWiki}(不公开): 百科文章的源代码, 使用 LaTeX 语言, 支持用 TeXlive 直接编译 pdf, 服务器路径 \verb`/var/www/PhysWiki`。 编辑器 \verb`wuli.wiki/editor` 编辑文章后, 把 LaTeX 源码保存到 \verb`PhysWiki/contents/xxx.tex` 文件, 其中 \verb`xxx` 就是每篇文章的 id。 然后编辑器调用一个 C++ 命令行程序 \verb`PhysWikiScan` 把 tex 文件转换为 \verb`online` 目录的 html 文件。
\item \href{https://github.com/wuliwiki/PhysWikiScan}{PhysWikiScan}(公开,GPL-3): 一个后台 C++ 命令行程序, 用于把 PhysWiki 中的 LaTeX 源码文件转为 littleshi.cn 中的静态的百科 html 页面, 可以单个转换也可以批量转换。服务器路径 \verb`/var/www/PhysWikiScan`
\item \href{https://github.com/wuliwiki/littleshi.cn-server}{littleshi.cn-server}(不公开): 百科编辑器 \verb`wuli.wiki/editor` 的源码, 调用 PhysWikiScan 程序。 服务器路径 \verb`/var/www/editor`, 使用 node.js 开发, 已封装到 docker 中运行。
\item \href{https://github.com/wuliwiki/PhysWiki-backup}{PhysWiki-backup}(公开): 用于存放百科文章备份,每篇文章页面根据该备份记录给作者排序。服务器路径 \verb`/var/www/PhysWiki-backup`
\item \href{https://github.com/wuliwiki/wuliwiki-web}{wuliwiki-web}(不公开):网站中除了文章编辑器外的所有动态页面,以及后台。 当前在服务器中用 docker 运行。
\item \href{https://github.com/wuliwiki/wuliwiki-app}{wuliwiki-app}(不公开): 百科移动端 app, 用 flutter 开发, 支持安卓、iPhone、iPad。 经费不足已经停止开发。
\end{itemize}

\subsubsection{框架}
\begin{itemize}
\item 编辑器和论坛前端都是 React 的 MUI
\item 后端 Python/Django/Redis/PostgreSQL/RESTful/RabbitMQ
\end{itemize}

\subsection{阿里云全站加速(DCDN)}
\begin{itemize}
\item 加速区域:全球
\item 端口:443;
\item DNS 解析:CNAME 记录添加,A 记录暂停;
\item 回源配置:全部关闭
\item 静态文件类型:全部选中
\item 自适应缓存:开
\item 缓存配置:无
\item https 配置:见下文
\item 访问控制:无
\item 性能优化: html,css,js 全开, Gzip 开, Brotli 关
\item Websocket: 开
\item 静态路径:
\begin{lstlisting}[language=none]
/apps/*
/assets/*
/images/*
/MathJax3/*
/media/*
/online/*
/statics/*
/tree/*
\end{lstlisting}
\end{itemize}

\subsubsection{https 配置}
\begin{itemize}
\item Nginx 配置(\verb`/etc/nginx/sites-available/wuli.wiki/`)
\item 【可能过时了】从 Nginx 配置可以查到证书路径:当前使用 Lets Encrypt (\verb`/etc/letsencrypt/live/wuli.wiki/*.pem`)(每三个月更新,需要手动 \verb`nginx -s reload`)
\item 【新】更新证书按照 \enref{Nginx 笔记}{nginxN}里面的方法更新即可(\verb`sudo certbot certonly --nginx`)。
\item \verb`cert.pem`, \verb`chain.pem`, \verb`fullchain.pem` 都是证书, \verb`privkey.pem` 是私钥。 \verb`fullchain.pem` 是 \verb`cert.pem` 和 \verb`chain.pem` 的合并, 一般情况都是使用 \verb`fullchain.pem`(例如配置 DCDN 的 https 时)。
\item 要查看当前证书过期时间,用 \verb`sudo openssl x509 -in fullchain.pem -noout -enddate`。
\item 当每三个月证书更新后, DCDN 也需要重新设置新的证书。把 \verb`fullchain.pem` 和 \verb`privkey.pem` 的文本手动复制粘贴到 https 设置里面即可。如果提示证书已存在,就把自定义的名字改一下即可。
\end{itemize}
\begin{figure}[ht]
\centering
\includegraphics[width=6cm]{./figures/1144311ea893817a.png}
\caption{更新证书(实测成功): 1.\verb`nginx -s reload`。 2. 按图中的修改配置,第一个框粘贴 \verb`fullchain.pem`,第二个框粘贴 \verb`privkey.pem`,3.改一下证书名否则会提示已存在} \label{fig_myCode_1}
\end{figure}
