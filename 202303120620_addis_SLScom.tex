% SLISC 的编译和测试

\begin{issues}
\issueDraft
\end{issues}

\pentry{SLISC 库概述\upref{SLISC}, 在 Linux 上编译 C/C++ 程序\upref{linCpp}}

\subsection{直接使用头文件}
和 C++ 标准库一样, 如果你不想修改 SLISC 源码, 只需要把 \verb|SLISC| 文件夹复制到您的项目, 并在 C++ 源码中使用相应的头文件即可。

下载 SLISC 源码(例如, 创建 “SLISC 库概述\upref{SLISC}” 中的 \verb|intro.cpp|。 注意如果 \verb|intro.cpp| 如果和 \verb|SLISC| 文件夹不在同一个目录, 那么你需要修改 \verb|#include| 中的相对路径, 也可以用绝对路径。

要编译, 用 \verb|g++ -D SLS_USE_INT_AS_LONG intro.cpp -o intro|。 其中 \verb|-D SLS_USE_INT_AS_LONG| 用于预定义宏, 声明我们使用 4 字节的 \verb|Long|。 等效地, 你也可以在 \verb|intro.cpp| 最开始插入 \verb|#define SLS_USE_INT_AS_LONG|。

编译成功后, 运行程序就得到输出结果
\begin{lstlisting}[language=none]
u = 
0   1   2   

a 有 2 行和 3 列, 共计 6 个元素。
1.1   0   4.4   
0   9.9   5.5
\end{lstlisting}
以及两个输出文件 \verb|data0.matb| 和 \verb|data.matb|。

\subsection{编译测试}
为了测试代码在你使用的 CPU 架构和操作系统中能正常运行, 强烈建议编译测试, 即 \verb|SLISC0/main.cpp|。 现在运行 \verb|make -j8| 即可, 然后运行使用 \verb|./main.x|。
