% Gamma 函数
% 微积分|定积分|Γ 函数| Gamma 函数|阶乘|半整数

\pentry{分部积分法\upref{IntBP}}

\footnote{参考 Wikipedia \href{https://en.wikipedia.org/wiki/Gamma_function}{相关页面} 以及 \cite{Arfken} 相关章节。}$\Gamma$ 函数\textbf{(gamma function)}可以看成是阶乘在实数或复数域的拓展, 该函数有多种定义方法, 这里先讨论实数域上的定积分\upref{DefInt}定义。 该方法可以定义 $(-1, \infty)$ 区间的阶乘
\begin{equation}\label{eq_Gamma_1}
x! \equiv \Gamma (x + 1) = \int_0^{+\infty} t^x \E^{-t} \dd{t} \qquad (x > -1)
\end{equation}
即 $\Gamma$ 函数在区间 $(0,\infty)$ 被定义为
\begin{equation}\label{eq_Gamma_3}
\Gamma(x) \equiv \int_0^{+\infty} t^{x-1} \E^{-t} \dd{t} \qquad (x > 0)
\end{equation}
可以证明新定义的阶乘的递推关系仍为
\begin{equation}\label{eq_Gamma_2}
(x+1)!=(x+1)x! \qquad (x>-1)
\end{equation}
且 $0! = 1$。 所以当 $x$ 取正整数 $N$ 时, \autoref{eq_Gamma_1} 的结果仍然是熟悉的 $N! = N(N-1)\dots, 1$。

另外能证明 $(-1/2)!=\sqrt{\pi}$, 由此我们可以直接写出半整数的阶乘为
\begin{equation}
\frac{N}{2}! = \frac{N}{2} \qty(\frac{N}{2}-1) \dots \frac12 \sqrt{\pi} \qquad (N > 0)
\end{equation}

\subsection{推导}

首先
当 $x \leqslant 0$ 时该积分在 $x=0$ 处不收敛,以下仅讨论 $x$ 为正实数的情况\footnote{事实上,自变量为负实数(非整数)时,$\Gamma$ 函数有另一种定义,这里不讨论。}。

我们现在验证当 $x$ 取正整数时,新定义的阶乘 $x! = \Gamma(x+1)$ 与原来的定义 $x! = x(x-1)\dots 1$ 相同。首先
\begin{equation}\label{eq_Gamma_4}
0! = \Gamma(1) = \int_0^{+\infty} \E^{-t} \dd{t} = 0 - (-1) = 1
\end{equation}

使用分部积分法\upref{IntBP},令 $t^x$ 为“求导项”, $\E^{-t}$ 为“积分项”, 可得递推公式\footnote{该证明仅对 $x>0$ 适用, 这样才有 $0^x \E^{-0} = 0$, 使第四个等号成立。}(\autoref{eq_Gamma_2})
\begin{equation}\label{eq_Gamma_5}
\begin{aligned}
x! &= \Gamma(x+1) = \int_0^{+\infty} t^x \E^{-t} \dd{t} =  - \eval{t^x \E^{-t}}_{0}^{+\infty} + \int_0^{+\infty} x t^{x-1} \E^{-t} \dd{t} \\
&= x\int_0^{+\infty} t^{x-1} \E^{-t} \dd{t} = x\Gamma (x) = x(x-1)!
\end{aligned} \end{equation} 
由递推\autoref{eq_Gamma_5} 和初值\autoref{eq_Gamma_4}, 对任意正整数 $n$ 有
\begin{equation}
n! = n(n-1)! = n(n-1)(n-2)!... = n(n-1)...1
\end{equation}

再来看半整数的阶乘,我们讨论范围内的最小半整数的阶乘为 
\begin{equation}
\qty(-\frac12) ! = \int_0^{+\infty} \frac{\E^{-x}}{\sqrt x}\dd{x} = 2\int_0^{+\infty} \E^{-t^2} \dd{t} = \int_{-\infty}^{+\infty} \E^{-t^2} \dd{t} = \sqrt{\pi}
\end{equation}
其中使用了换元法令 $x = t^2$ 将定积分变为高斯积分\upref{GsInt}。

\subsection{渐近公式}
对于大的$x$, 有\textbf{斯特林公式(Stirling formula)}:
$$
\Gamma(x+1)
  =\sqrt{2\pi x}\left({x\over \E}\right)^x
  \left(
   1
   +{1\over12x}
   +{1\over288x^2}
   -{139\over51840x^3}
   -{571\over2488320x^4}
   + \cdots
  \right).
$$
这是一个渐近展开, 右边的级数是发散的。 它的推导可见拉普拉斯方法\upref{LapAsm}

\subsubsection{Matlab 代码}
Matlab 中的默认 \verb|gamma(x)| 函数接受一个实数 double 类型 \verb|x|, 但如果你安装了符号工具箱 Matlab 符号计算和变精度计算\upref{MatSym}, 可以用 \verb|gamma(vpa(x))|。
\begin{lstlisting}[language=matlab, caption=gamma\_sym.m]
% symbolic implementation of gamma function
% input and output are 'double' (supports complex and matrix)
% efficiency: ~6e-4 [s/eval]
function z = gamma_sym(z)
if issym(z), error('argument must be double'); end
old = digits(17);
z = double(gamma(vpa(z)));
if any(isnan(z) | isinf(z))
    warning('bad output !');
end
digits(old);
end
\end{lstlisting}
