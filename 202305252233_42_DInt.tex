% 定积分
% keys 定积分|可积函数|性质

\pentry{极限\upref{Lim}}
\subsection{和 $\sum_{i=0}^{n-1}f(\xi_i)\Delta x_i$ 的极限定义}
在给出定积分定义之前,我们先给出下面的和
\begin{equation}\label{eq_DInt_1}
\sigma=\sum_{i=0}^{n-1}f(\xi_i)\Delta x_i
\end{equation}
的极限的两种等价定义。
\autoref{eq_DInt_1} 的级数和所表示的意义理解如下:

设函数 $f(x)$ 在区间 $[a,b]$ 上有定义。用任意方法在该区间上插入分点
\begin{equation}
x_0=a<x_1<\cdots<x_i<\cdots<x_n=b.
\end{equation}
用 $\lambda$ 表示差 $\Delta x_i=x_{i+1}-x_i(i=0,1,\cdots,n-1)$ 中最大的一个,即
\begin{equation}
\lambda=\max\qty{\Delta x_i|i=0,\cdots,n-1}
\end{equation}
从每一个区间 $[x_i,x_{i+1}]$ 上任取一点 $x=\xi_i\quad (i=0,\cdots,n-1)$,并做出和
\begin{equation}
\sigma=\sum_{i=0}^{n-1}f(\xi_i)\Delta x_i.
\end{equation}
其几何意义可用下图表出。
\begin{figure}[ht]
\centering
\includegraphics[width=8cm]{./figures/c36fca0b6f2f5aca.pdf}
\caption{\autoref{eq_DInt_1} 的几何意义} \label{fig_DInt1}
\end{figure}
可以证明,以下关于 $\sigma$ 的极限的两种定义等价。
\begin{definition}{第一种定义}
称和
\begin{equation}
\sigma=\sum_{i=0}^{n-1}f(\xi_i)\Delta x_i.
\end{equation}
当 $\lambda\rightarrow0$ 时有(有限)\textbf{极限} $I$ ,如果对每个数 $\epsilon>0$ 可以找到这样的数 $\delta>0$,使得,只要 $\lambda<\delta$ ,不等式
\begin{equation}
\abs{\sigma-I}<\epsilon
\end{equation}
在数 $\xi_i$ 的任意选择之下皆成立。并记作
\begin{equation}
I=\lim_{\lambda\rightarrow0}\sigma
\end{equation}
\end{definition}
上面的定义可看作“$\epsilon-\delta$语言”。 

下面用“序列语言”进行定义。在此之前,先引入一些概念。
\begin{definition}{基本区间分划序列}
设用不同的方法将区间 $[a,b]$ 进行分划,记第 $i$ 种分划对应的 $\lambda$ 为 $\lambda_i$,如果分划对应的序列 $\lambda_1,\lambda_2,\cdots$ 收敛到0,则称这样的区间分划序列叫做\textbf{基本区间分划序列}。
\end{definition}

\begin{definition}{第二种定义}
不论 $\xi_i$ 如何选取,若对任一基本区间分划序列对应的和 $\sigma$ 的序列,恒有极限 $I$,则称 $I$ 为和 $\sigma$ 的极限。
\end{definition}