% 量子引力(综述)
% license CCBYSA3
% type Wiki

本文根据 CC-BY-SA 协议转载翻译自维基百科\href{https://en.wikipedia.org/wiki/Quantum_gravity}{相关文章}。
\begin{figure}[ht]
\centering
\includegraphics[width=8cm]{./figures/687a04fa76585299.png}
\caption{cGh立方体的图示} \label{fig_LZYL_1}
\end{figure}
\begin{figure}[ht]
\centering
\includegraphics[width=8cm]{./figures/c1a2696f0c41f657.png}
\caption{“以维恩图表示”} \label{fig_LZYL_2}
\end{figure}
\textbf{量子引力(QG)}是理论物理学的一个领域,旨在根据量子力学的原理描述引力。它处理的是在引力效应和量子效应都不能忽视的环境下的问题,如黑洞或类似的紧凑天体附近,以及宇宙大爆炸后早期阶段的时空。[1][2]

自然界的四种基本相互作用中,除了引力,其他三种相互作用——电磁相互作用、强相互作用和弱相互作用——已经在量子力学和量子场论的框架中得到了描述。引力是唯一尚未完全适应量子力学框架的相互作用。当前对引力的理解基于阿尔伯特·爱因斯坦的广义相对论,这一理论结合了特殊相对论,并深刻修改了时间和空间等概念的理解。虽然广义相对论因其优雅和精确性受到高度评价,但它也有其局限性:例如,黑洞内部的引力奇点、暗物质的假设以及暗能量和宇宙常数的关系,这些都是目前关于引力未解之谜之一,[3] 这些都表明广义相对论在不同尺度上的崩溃,并突显了需要一种能够进入量子领域的引力理论。在接近普朗克长度的尺度上,如黑洞中心附近,时空的量子波动预计会发挥重要作用。[4] 最后,真空能量的预测值与观测值之间的差异(根据考虑的不同,差异可能达到60或120个数量级)[5][6] 强调了量子引力理论的必要性。

量子引力领域正在积极发展,理论学者正在探索多种解决量子引力问题的方法,其中最受欢迎的是M理论和环量子引力。[7] 所有这些方法都旨在描述引力场的量子行为,尽管这并不一定包括将所有基本相互作用统一成一个单一的数学框架。然而,许多量子引力的途径,如弦理论,试图开发一个描述所有基本相互作用的框架。这样的理论通常被称为“万物理论”。一些方法,如环量子引力,并不试图做出这种统一尝试;相反,它们努力量子化引力场,同时将引力与其他相互作用分开。其他一些较不为人知,但同样重要的理论包括因果动力学三角化、非交换几何和扭曲理论。[8]

制定量子引力理论的一个难点是,量子引力效应的直接观测预计只会在接近普朗克尺度(约10^-35米)的长度尺度上出现,这一尺度远小于目前高能粒子加速器所能达到的能量水平,因此,物理学家缺乏实验数据来区分已提出的竞争性理论。[n.b. 1][n.b. 2]

已提出了一些思维实验的方法,作为测试量子引力理论的工具。[9][10] 在量子引力领域有若干悬而未决的问题——例如,尚不清楚基本粒子的自旋如何源引力,思维实验可能为探索这些问题的可能解决方案提供路径,[11] 即使没有实验室实验或物理观察。

21世纪初,出现了新的实验设计和技术,这些技术表明,间接测试量子引力的方法在未来几十年内可能成为可行的。[12][13][14][15] 这一研究领域被称为现象学量子引力。
\subsection{概述}
\begin{figure}[ht]
\centering
\includegraphics[width=10cm]{./figures/ef4bc1a90c9f4da4.png}
\caption{图示展示量子引力在物理学理论层级中的位置} \label{fig_LZYL_3}
\end{figure}
将这些理论在所有能量尺度上结合起来的困难,主要来自于这些理论对宇宙运作方式的不同假设。广义相对论将引力建模为时空的弯曲:用约翰·阿奇博尔德·惠勒的话来说,“时空告诉物质如何运动;物质告诉时空如何弯曲。”另一方面,量子场论通常是在特殊相对论中使用的平直时空中进行表述的。目前没有任何理论能够成功地描述物质动力学(用量子力学建模)如何影响时空的弯曲的普遍情况。如果试图将引力视为另一种量子场,得到的理论就无法进行重整化。即便是在时空弯曲事先固定的简单情况下,发展量子场论也变得更加数学上具有挑战性,许多物理学家在平直时空中使用的量子场论思想不再适用。

人们普遍希望量子引力理论能够帮助我们理解非常高能和非常小空间尺度上的问题,例如黑洞的行为以及宇宙的起源。

一个主要的障碍是,在具有固定度规的弯曲时空中,量子场论中的玻色子/费米子算符场对于时空分离的点是超对易的。(这是一种施加局部性的原则。)然而,在量子引力中,度规是动态的,因此两个点是否时空分离取决于状态。实际上,它们可以处于量子叠加态,既是时空分离的,也不是时空分离的。