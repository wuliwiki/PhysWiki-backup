% 南京理工大学 2007 年 研究生入学考试试题 普通物理(A)
% license Usr
% type Note

\textbf{声明}:“该内容来源于网络公开资料,不保证真实性,如有侵权请联系管理员”

\subsection{填空题(26分,每空2分)}
1.一小轿车作直线运动,刹车时$(t=0)$,速度为$v_0$,刹车后其加速度与速度成正比而反向,即$a=k-v,k$为已知常数。为已知常数。则任一时刻 $a(t) = \underline{\hspace{2cm}}$, $v(t) = \underline{\hspace{2cm}}$。

2.一质量为 $2 \\, \text{kg}$ 初速为零的物体在水平推力 $F = 3t^2$ (N) 的作用下, 在光滑的水平面上作直线运动, 则在第 1 秒内物体获得的冲量大小为 $\underline{\hspace{2cm}}$, 在第 2 秒末物体的速度大小为 $\underline{\hspace{2cm}}$, 在第 3 秒内的动能增量为 $\underline{\hspace{2cm}}$。

3. $1 \\, \text{mol}$ 37℃ 的氧气分子的平均速率为 $\underline{\hspace{2cm}}$, 分子的平均动能为 $\underline{\hspace{2cm}}$。

4. 如图, 已知 $1 \\, \text{mol}$ 某理想气体 $C_V = \frac{3}{2} R$, 状态 1 温度为 $T_1$, 图中 $p_2 = 2p_1$, $V_2 = 2V_1$, 则气体在(1-4-2)过程中从外界吸收的热量为  $\underline{\hspace{2cm}}$, 在(1-3-2)过程中从外界吸收的热量为  $\underline{\hspace{2cm}}$ (用 $R$, $T_1$ 表示)。
\begin{figure}[ht]
\centering
\includegraphics[width=6cm]{./figures/d3f2217212acbdfe.png}
\caption{} \label{fig_NJU07_1}
\end{figure}

5. 一音叉置于反射面 $S$ 和观察者 $R$ 之间, 音叉的频率为 $v_0$。现在 $R$ 静止, 而音叉以速度 $v_1$ 向反射面 $S$ 运动, 则 $R$ 处接收到的拍频 $\underline{\hspace{2cm}}$。 $\Delta v = \underline{\hspace{2cm}}$。设声速 $u$已知。。
\begin{figure}[ht]
\centering
\includegraphics[width=6cm]{./figures/231e6ded4ad475e2.png}
\caption{} \label{fig_NJU07_2}
\end{figure}

6. 同一媒质中的两波源 $A$, $B$, 相距为 $x_B - x_A = 10m$, 它们的振幅均为 $15$ 厘米, 频率都是 $50$Hz, 相位差为 $\pi$, 波速为 $400m\cdot s^{-1}$。$t = 0$ 时, $A$ 处质点的位移为 $15$ 厘米。若以 $A$ 点为坐标原点, 则 $A$ 处质点由波动方程(即波函数)为 $\underline{\hspace{2cm}}$, $B$ 处质点的波动方程为 $\underline{\hspace{2cm}}$。$A$, $B$ 连线上的干涉静止点的各点的位置为 $\underline{\hspace{2cm}}$。

7.已知一维无限深势井中粒子的波函数为:$\psi=\sqrt{\frac{2}{a}}\sin \frac{n \pi }{a}x$
\subsection{填空题(30分,每空2分)}
1. 半径为 $R$ 的金属球球心与点电荷 $q_2$ 相距为 $d$,金属球带电 $q_1$,则金属球 $O$ 处的电场强度 $E= \underline{\hspace{3cm}}$ ,电势 $U= \underline{\hspace{3cm}}$ ,金属球接地后,$O$ 处的电场强度 $E= \underline{\hspace{3cm}}$。

  2. 半径为 $R$,长为 $L (R \ll L)$,匝数为 $N$ 的均匀密绕直螺线管的自感系数为 $\underline{\hspace{3cm}}$; 若螺线管通以电流$I$,则螺线管内的磁感应强度大小为 $\underline{\hspace{3cm}}$;储存的磁场能量为 $\underline{\hspace{3cm}}$。

  3. 在真空中,一平面电磁波的电场 $B=B_t=B_0 \cos(\alpha (t - \frac{x}{c})) (T)$,则该电磁波的传播方向为 $\underline{\hspace{3cm}}$,电场强度的可表示为 $\underline{\hspace{3cm}}$。

  4. 两个偏振片平行放置,若其偏振化方向相互垂直,就组合成一正交偏振片。现有光强为 $I_0$ 的一束自然光垂直射入该正交偏振片,则透射光强为 $\underline{\hspace{3cm}}$。若在两偏振片之间放入第三块偏振片,其偏振化方向与第一个偏振片的偏振化方向夹角为 $30^\circ$,则透射光强为 $\underline{\hspace{3cm}}$。

  5. 下图仪标出等倾干涉和等倾环的两级明条纹,分别用 $k_1, k_2$ 表示。对于干涉环,有 $k_1 = \underline{\hspace{1cm}}$,对于等倾干涉条纹,有 $k_1 = \underline{\hspace{1cm}}$ (填 $>, <, =$ )。
\begin{figure}[ht]
\centering
\includegraphics[width=6cm]{./figures/15ebfe4fcda41f1a.png}
\caption{} \label{fig_NJU07_3}
\end{figure}
6. 介子 $\pi_0$ 相对静止时测得其平均寿命为 $r_0=1.8 \times 10^{-8} \\ s$,若使其以 $v = 0.8c$ 的速度穿行加速器,则从实验室观测,$\pi$介子的平均寿命为 $\underline{\hspace{3cm}}$,飞行的平均距离为 $\underline{\hspace{3cm}}$。
\subsection{(12分)}
设唱机的转盘绕着通过盘心的固定竖直轴以角速度$\omega$匀速转动,且保持不变,唱片可以看成是半径为$R$、质量为 $m$ 的均匀圆盘,唱片放上转盘后将受转盘的摩擦力矩作用下而随转盘转动。已知唱片和转盘之间的滑动摩擦系数为$\mu$,求

1)唱片刚放上去时受到的摩擦力矩

2)唱片达到角速度$\mu$所需的时间
\subsection{(12分)}
某平面简谐波$t=0$时的波形如图示。波速$u=340m\cdot s^{-1}$,求其波动方程并画出$x=1$处质点的振动图线:
\begin{figure}[ht]
\centering
\includegraphics[width=6cm]{./figures/e25c91f5f069a905.png}
\caption{} \label{fig_NJU07_4}
\end{figure}
\subsection{(12分)}
(12分)空气标准奥托循环由下述四个过程组成

(1)$a-b$,绝热:(2)$b-c$,等体吸热:

(3)$c-d$,绝热;(4)$d-a$,等体放热

求此循环的效率;
\begin{figure}[ht]
\centering
\includegraphics[width=6cm]{./figures/599e63fb2dee3499.png}
\caption{} \label{fig_NJU07_5}
\end{figure}