% 路德维希·玻尔兹曼(综述)
% license CCBYSA3
% type Wiki

本文根据 CC-BY-SA 协议转载翻译自维基百科\href{https://en.wikipedia.org/wiki/Ludwig_Boltzmann}{相关文章}。

\begin{figure}[ht]
\centering
\includegraphics[width=6cm]{./figures/ed66f21add0c5482.png}
\caption{1902年的玻尔兹曼} \label{fig_BRZM_1}
\end{figure}
\textbf{路德维希·爱德华·玻尔兹曼}(Ludwig Eduard Boltzmann,1844年2月20日-1906年9月5日)是奥地利的物理学家和哲学家。他的最大成就包括统计力学的发展和热力学第二定律的统计解释。1877年,他提出了当前的熵定义:\(S = k_{\rm B}\ln\Omega \)其中,Ω是系统能量等于宏观系统能量的微观状态数,解释为衡量系统统计无序度的一个指标。马克斯·普朗克将常数 \( k_B \) 命名为玻尔兹曼常数。

统计力学是现代物理学的基石之一。它描述了宏观观测(如温度和压力)如何与围绕平均值波动的微观参数相关。它将热力学量(如比热容)与微观行为联系起来,而在经典热力学中,唯一可用的方式是为不同材料测量并列出这些量。
\subsection{传记}  
\subsubsection{童年与教育}  
尔兹曼出生在维也纳的郊区厄尔德贝格(Erdberg),来自一个天主教家庭。他的父亲路德维希·乔治·玻尔兹曼(Ludwig Georg Boltzmann)是一名税务官员。他的祖父从柏林迁至维也纳,是一位钟表制造商,而玻尔兹曼的母亲凯瑟琳·保尔恩芬德(Katharina Pauernfeind)则来自萨尔茨堡。玻尔兹曼在家中接受教育,直到十岁才开始正式上学,之后在上奥地利州的林茨市读高中。15岁时,玻尔兹曼的父亲去世。

1863年起,玻尔兹曼在维也纳大学学习数学和物理学。他于1866年获得博士学位,并于1869年获得讲授资格(venia legendi)。玻尔兹曼与物理学研究所所长约瑟夫·斯特凡(Josef Stefan)密切合作,正是斯特凡将玻尔兹曼引入了麦克斯韦的研究成果。
\subsubsection{学术生涯} 
1869年,玻尔兹曼在25岁时,凭借约瑟夫·斯特凡的推荐信,[9] 被任命为格拉茨大学(位于施蒂利亚省)数学物理学全职教授。1869年,他在海德堡与罗伯特·本森(Robert Bunsen)和莱奥·凯尼茨贝格(Leo Königsberger)合作工作了几个月,随后在1871年与古斯塔夫·基尔霍夫(Gustav Kirchhoff)和赫尔曼·冯·亥姆霍兹(Hermann von Helmholtz)在柏林合作。1873年,玻尔兹曼加入维也纳大学,担任数学教授,并在此工作直到1876年。