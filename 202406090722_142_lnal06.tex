% 乘积空间
% license Xiao
% type Tutor

\begin{issues}
\issueTODO
\end{issues}
% Giacomo:我把这篇文章的补空间,商空间部分全部移走了

\pentry{直和与补空间(线性空间)\nref{nod_DirSum},}{nod_b6c4}

我们可以在向量空间的笛卡尔积上规定向量的数乘和加法,使得它也是一个向量空间,称为\textbf{乘积空间}。

\begin{definition}{乘积空间}
给定域$\mathbb F $上的线性空间$U$与$V$,定义$U\times V=\{(u, v)|u \in U, v \in V\}$上的数乘和加法运算为:
\begin{equation}
\left\{\begin{aligned}
a \cdot (u, v) &= (a \cdot u, a \cdot v), \quad \forall a \in \mathbb F\\
(u_1, v_1) + (u_2, v_2) &= (u_1 + u_2, v_1 + v_2)
\end{aligned}\right.~
\end{equation}
\end{definition}
根据该定义,我们容易验证积空间在数乘和加法下封闭。
若令$\{x_i\}^r_{i=1}$和$\{y_i\}^s_{i=1}$分别为$U$与$V$的基,我们也容易验证积空间的基为$\{(x_i, 0)\}^r_{i=1}\cup \{(0, y_i)\}^s_{i=1}$。

% Giacomo: 和shang
% 
% 从群的角度上看,线性空间是一个加法群,则其任意一个子空间都是正规子群。因此线性空间可以对任意一个子空间求商,商群上的运算为\textbf{向量加法}。由于线性空间在域上运算,所以我们需要证明这样的定义是合理的,即:
% \begin{enumerate}
% \item 该商群需要额外满足加法分配性,我们才可以称其为“商空间”;
% \item 加法分配性对等价类无影响。比如设$V$商去子空间$V_0$后,左陪集$a + V_0=b + V_0$,则$k(a + V_0)=k(b + V_0)$。

% 这两点是显而易见的,读者可自证。
% \end{enumerate}
% \begin{definition}{商空间}
% 给定域$F$上的线性空间$V$及其子空间$V_0$,则对任意$v \in V$,定义左陪集$v + V_0=\{v + v_0|v_0 \in V_0\}$。

% 数乘定义为:$\forall a \in \mathbb F, a(v + V_0)=av + V_0$
% \end{definition}


利用商空间的概念也能证明线性空间的任意子空间都有补空间。
\begin{theorem}{}\label{the_lnal06_1}
给定域 $\mathbb F$ 上的线性空间 $V$,设 $V_1$ 是其子空间,则 $V_1$必有补空间。
\end{theorem}
从$V_1$的左陪集${S_{\alpha}}$里各选一个元素$v_a$,$\opn{Span} \{v_{\alpha}\}$构成商空间$V/V_1$的一组基\footnote{尽管$kv_\alpha$为不同于$v_\alpha$的等价类,但这里取张成,不影响后续定理2这一推论},张成的也是$V_1$的补空间。由等价类划分可知,$\opn{Span} \{v_{\alpha}\}$与$V_1$没有交集,因此我们只需要证明所有元素都可以表示为$b_{\alpha}v_{\alpha} + V_1$即可。

设$v$为任意元素,则有$v + V_1=b_{\alpha} S_{\alpha}$,由于$v_{\alpha} \in S_{\alpha}$,所以$b_{\alpha} v_{\alpha} \inv + V_1$。也就是说,总可以找到一个元素$v_1 \in V_1$,使得$v=bv_{\alpha} + v_1$,由于左陪集与子空间互不包含,因而和是直和,证毕。
\begin{exercise}{例子}
设$V$是一个平面,$V_1$是$x$轴,那么$V/V_1$是什么呢?该商空间的元素,即每一个等价类为无数平行于该直线的直线。设$x$和$y$等价,由等价条件知$x-y \in V_1$。
\end{exercise}
\autoref{the_lnal06_1} 的证明过程同时明示着商空间维数和子空间的关系。

% 移动到了商空间