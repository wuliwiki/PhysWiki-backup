% 磁旋比、玻尔磁子
% 磁矩|磁旋比|玻尔磁子

%未完成: 将开头内容移动到 “磁矩\upref{MagMom}” 中

\pentry{转动惯量\upref{RigRot}, 磁矩\upref{MagMom}}

\footnote{参考 Wikipedia \href{https://en.wikipedia.org/wiki/Gyromagnetic_ratio}{相关页面}。}当一个物体做定轴转动时, 磁矩\upref{MagMom}的和角动量\upref{AngMom}的大小之比就叫做\textbf{磁旋比(gyromagnetic ratio)}。 该物理量常在量子力学中出现,但对于宏观的带电物体也可以定义。

\subsection{均匀带电体的磁旋比}
一个任意形状的带电刚体, 如果质量密度 $\rho_m$ 和电荷密度 $\rho_q$ 成正比,当它定轴转动时,令角速度矢量为 $\bvec\omega$, 则角动量\upref{CM2}为
\begin{equation}
\bvec L = I\bvec \omega  = \bvec \omega \int r_\bot^2 \rho_m \dd{V}~.
\end{equation}
磁矩定义为
\begin{equation}
\bvec \mu  = Ia \uvec\omega = \int \frac{\dd{Q}}{2\pi/\omega}  \pi r^2 \uvec \omega
= \frac12 \bvec \omega \int r_\bot^2\rho_q \dd{V} ~,
\end{equation}
其中 $r$ 为质量元到转轴的距离。两式比较,得
\begin{equation}
\bvec \mu  = \frac{q}{2m} \bvec L~.
\end{equation}
其中系数 $q/(2m)$ 就是该物体的磁旋比。

把上面的推导稍加改变可以发现, 若电荷分布在物体的外表面, 那么磁旋比。

\subsection{基本粒子的磁旋比}
但对基本粒子(例如电子)的实验中,发现上式还需要一个修正因子, 称为 \textbf{g 因子(g-factor)}
\begin{equation}
\bvec \mu  = g\frac{q}{2m}\bvec L~.
\end{equation}
定义\textbf{磁旋比(gyromagnetic ratio)}\footnote{有时也叫 \textbf{magnetogyric ratio}}为
\begin{equation}
\gamma  = \frac{gq}{2m}~.
\end{equation}


对于粒子的自旋, $L = \hbar \sqrt{l(l + 1)} $。 所以 $\mu = \sqrt{l (l + 1)} \hbar gq/(2m)$。 
对于电子,实验测得
\begin{equation}
g_e \approx 2.0023193043617(15) \approx 2~.
\end{equation}
所以
\begin{equation}
\bvec \mu_e  = g_e \frac{e\hbar}{2 m_e} \sqrt{\frac12 \qty(1 + \frac12)}  = \frac{\sqrt 3}{2} g_e \mu_B \approx \sqrt 3 \mu_B~,
\end{equation}
其中 $\mu_B$ 为\textbf{玻尔磁子(Bohr magneton)},定义为
\begin{equation}
\mu_B = \frac{e\hbar}{2 m_e} \approx 5.788\e{-5} \Si{eV/T}~.
\end{equation}
它可以看作是磁矩的一个单位。
