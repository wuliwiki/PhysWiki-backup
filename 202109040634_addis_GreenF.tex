% 格林函数法解非齐次偏微分方程

\begin{issues}
\issueDraft
\end{issues}

\pentry{狄拉克 delta 函数\upref{Delta}}

区间 $(a,b)$ 的非齐次线性微分方程可记为
\begin{equation}
\Q Q y(x) = f(x)
\end{equation}
其中 $\Q Q$ 是线性微分算符. 令格林函数为 $G(x', x)$ 满足
\begin{equation}
\Q Q G(x', x) = \delta(x - x') \qquad (a < x' < b)
\end{equation}
那么方程的解为
\begin{equation}
f(x) = \int_a^b f(x') G(x', x) \dd{x'}
\end{equation}

\subsection{例子: }
例子: 一根两端固定的弦两端固定在 $x$ 轴上, 区间为 $[0, L]$, 张力为 $T$, 形状为 $y(x)$, 边界条件为 $y(0) = y(L) = 0$. 在弦上有 $y$ 方向的连续受力分布, 若弦的受力密度函数为 $f(x)$, 即单位长度受到的 $y$ 方向的力, 那么当 $\abs{f(x)} \ll T$ 时有方程(过程类比 “一维波动方程\upref{WEq1D}”)
\begin{equation}\label{GreenF_eq1}
-T y'' = f(x)
\end{equation}

虽然该方程可以直接对两边积分两次得到解(两个积分常数由边界条件确定)
\begin{equation}
y(x) = -\frac{1}{T}\iint f(x) \dd{x}\dd{x}
\end{equation}
但为了教学我们用格林函数法. 先令格林函数 $G(x', x)$ 满足
\begin{equation}\label{GreenF_eq2}
-T G''(x', x) = \delta(x - x') \qquad (0 < x' < L)
\end{equation}
方程右边是狄拉克 $\delta$ 函数\upref{Delta}. 且同样有边界条件 $G(x', a) = G(x', b) = 0$. 这相当于弦上只有一点 $x'$ 受大小为 $F = 1$ 的力.

解出格林函数后, $f(x)$ 可以分解为许多不同位置的 $\delta$ 函数的线性组合(积分)
\begin{equation}
f(x) = \int_0^L f(x') \delta(x - x') \dd{x'}
\end{equation}
由于\autoref{GreenF_eq1} 的方程是线性的, 那么把 $G(x', x)$ 做同样的线性组合就是满足边界条件的解
\begin{equation}\label{GreenF_eq3}
y(x) = \int_0^L f(x') G(x', x) \dd{x'}
\end{equation}

现在来解\autoref{GreenF_eq2}, 事实上我们可以直接从受力分析上得出格林函数 $G(x', x)$ 是一个三角形, 顶点的位置为 $x = x'$, 令高为 $h = y(x')$ 由受力分析可得
\begin{equation}
T\frac{h}{x'} + T\frac{h}{L - x'} = F = 1
\end{equation}
\addTODO{画图}
\addTODO{数学上的方法: 如何通过积分得到斜率在 $x'$ 处的增量?}

即
\begin{equation}
h = \frac{x' (L - x')}{LT}
\end{equation}
即格林函数为
\begin{equation}
G(x', x) = \leftgroup{
&? \qquad (0 < x \le x')\\
&? \qquad (x' < x)
}\end{equation}
代入\autoref{GreenF_eq3} 得
\begin{equation}
y(x) = \int_0^x ? + \int_x^L ?
\end{equation}
