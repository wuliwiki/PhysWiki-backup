% 南京理工大学 2007 年 研究生入学考试试题 普通物理(A)
% license Usr
% type Note

\textbf{声明}:“该内容来源于网络公开资料,不保证真实性,如有侵权请联系管理员”

\subsection{填空题(26分,每空2分)}
1.一小轿车作直线运动,刹车时$(t=0)$,速度为$v_0$,刹车后其加速度与速度成正比而反向,即$a=k-v,k$为已知常数。为已知常数。则任一时刻 $a(t) = \underline{\hspace{2cm}}$, $v(t) = \underline{\hspace{2cm}}$。

2.一质量为 $2 \\, \text{kg}$ 初速为零的物体在水平推力 $F = 3t^2$ (N) 的作用下, 在光滑的水平面上作直线运动, 则在第 1 秒内物体获得的冲量大小为 $\underline{\hspace{2cm}}$, 在第 2 秒末物体的速度大小为 $\underline{\hspace{2cm}}$, 在第 3 秒内的动能增量为 $\underline{\hspace{2cm}}$。