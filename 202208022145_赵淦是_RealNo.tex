% 实数域的戴德金构造
% 实数域|戴德金公理

假设我们把 $\mathbb{Q}$ 划分为左右两部分,两部分不相交,两部分之并为 $\mathbb{Q}$

从逻辑上讲,只有以下四种可能:

\begin{enumerate}
\item “左”中有最大数,“右”中无最小数
\item “左”中无最大数,“右”中右最小数
\item “左”中无最大数,“右”中无最小数
\item “左”中有最大数,“右”中有最小数
\end{enumerate}
我们发现,只要将 1 中“左”中的最大数移入“右”中,即得到 2 ,因此 1 和 2 是等价的,其割点均为有理数

假设 4 是合理的,记“左”中的最大数为 $a$ ,“右”中的最小数为 $b$ ,由有理数的稠密性可知,存在 $c\in\mathbb{Q}$ 使得 $\displaystyle{a<c<b}$ 于是 $c$ 既不在“左”中,又不在“右”中,这与两部分之并为 $\mathbb{Q}$ 矛盾,故 4 不合理

我们要证明 3 是合理的,只需考虑 $A=\{x\in\mathbb{Q}\mid x<\sqrt{2}\}$ 和 $B=\{x\in\mathbb{Q}\mid x>\sqrt{2}\}$

于是,产生了有理数的“间隙”,因此,我们需要定义一类新数来填补空隙

\begin{definition}{戴德金分割}
有理数集 $\mathbb{Q}$ 的一个非空真子集 $\alpha$  称作一个戴德金分割,若满足:
\begin{enumerate}
\item 对于 $r\in \alpha$ ,任给 $s\in\mathbb{Q}$ ,若 $s<r$ 则 $s\in\alpha$ 
\item 对于 $r\in \alpha$ ,都存在 $s\in\alpha$ 使得 $s>r$ 
\end{enumerate}
我们把每个分割称为一个实数,全体实数的集合称作实数集,记作 $\mathbb{R}$ 
\end{definition}
戴德金分割实际上就是对引言中的“左”的刻画, 1 刻画了“左”向左无限延伸, 2 刻画了“左”中无最大数.其几何意义很明显:向左无限延伸,向右无限逼近某个点却无法达到这个点
我们所定义的分割,既包括引言所述的情况 1 或 2 中的“左”,其对应有理数;又包括情况 4 中的“左”,其对应产生新数,我们称之为无理数.于是,我们就实现了数系的扩充

\begin{definition}{界 确界}
设 $F$ 为一有序域, $D$ 为其非空子集,若存在 $a\in  F$ 使得任意 $x\in D$ 都有 $x\leq a$ ,则称 $a$ 为 $D$ 的一个上界;若存在 $b\in  F$ 使得任意 $x\in D$ 都有 $x\geq b$ ,则称 $b$ 为 $D$ 的一个下界

若 $a$ 为 $D$ 的一个上界,且对于任意 $\varepsilon>0$ 都存在 $x\in D$ 使得 $x>a-\varepsilon$ ,则称 $a$ 为 $D$ 的上确界,记作:
$$a=\sup D$$ 

若 $b$ 为 $D$ 的一个下界,且对于任意 $\varepsilon>0$ 都存在 $x\in D$ 使得 $x<b+\varepsilon$ ,则称 $b$ 为 $D$ 的上确界,记作:
 $$b=\inf D$$ 
\end{definition}
\begin{lemma}{}
上确界是上界中的最小者,下确界是下界中的最大者
\end{lemma}

\begin{definition}{序关系}
$\subseteq$ 就是 $\mathbb{R}$ 上的序关系
\end{definition}
为什么能够这样定义?下面我们来证明 $\subseteq$  满足序关系的基本性质:
\begin{itemize}
\item 自反性: $\forall \alpha\in\mathbb{R}$ , $\alpha\subseteq\alpha$
\end{itemize}
\begin{itemize}
\item 
反对称性: $\forall \alpha,\beta\in\mathbb{R}$ ,若 $\alpha\subseteq\beta$ 且 $\beta\subseteq\alpha$ ,则 $\alpha=\beta$ 
\end{itemize}
\begin{itemize}
\item 传递性: $\forall \alpha,\beta,\gamma \in \mathbb{R}$ ,若 $\alpha\subseteq\beta$ 且 $\beta\subseteq\gamma$ ,则 $\alpha\subseteq\gamma$ 
\end{itemize}

因此, $\subseteq$ 是 $\mathbb{R}$ 上的序关系,而 $\subseteq$ 又对 $\mathbb{R}$ 满足:

\begin{itemize}
\item 完全性: $\forall \alpha,\beta \in\mathbb{R}$ , $\alpha\subseteq \beta$  与 $\beta\subseteq \alpha$  必有一项成立
\end{itemize}

\textbf{证明:}若 $\alpha\neq\beta$ ,又因为 $\alpha,\beta\neq\emptyset$ ,故 $\alpha-\beta\neq \emptyset$ 与 $\beta-\alpha \neq \emptyset$ 必有一项成立

若 $\alpha-\beta\neq\emptyset$ ,则存在 $r\in\mathbb{Q}$ 使得:
$$(r\in\alpha)\wedge (r\notin\beta)$$

因此 $\forall s\in\beta$ ,必有 $s<r$ ,从而 $s\in\alpha$ ,因此 $\beta\subset \alpha$

若 $\beta-\alpha\neq\emptyset$ ,同理可得 $\alpha\subset \beta$

综上,$\alpha\subseteq \beta$   与 $\beta\subseteq \alpha$  必有一项成立

于是,$\subseteq$ 还是 $\mathbb{R}$ 上的全序关系



\begin{definition}{加法}
$\mathbb{R}$ 上的加法指的是映射 $\mathbb{R}\times\mathbb{R}\to\mathbb{R}$ ,对于 $\alpha,\beta\in\mathbb{R}$ :
$$\alpha+\beta=\{r+s\mid(r\in\alpha)\wedge(s\in\beta)\}$$ 
\end{definition}
我们应保证 $\alpha+\beta$ 仍为实数,易见 $\alpha+\beta\neq \emptyset$ ,考察:
$$\alpha+\beta\ni p=r+s$$

其中 $r\in\alpha,s\in\beta$

首先,对于任意 $\mathbb{Q}\ni q<p$ ,两端同时减去 $q$ 得:
$$q-s< p-s=r$$

因此 $q-s\in\alpha$,而 $s\in\beta$ ,且有:
$$q=(q-s)+s$$

故 $q\in\alpha+\beta$

其次,由定义知存在 $t\in\alpha$ 使得 $t>r$ ,故也有 $t+s\in\alpha+\beta$ 使得:
$$t+s>r+s=p$$

综上所述, $\alpha+\beta$ 仍为实数,这样定义加法是合理的

加法具有我们熟悉的性质:
 \begin{theorem}{交换律}
 $\forall\alpha,\beta\in\mathbb{R}$ , $\alpha+\beta=\beta+\alpha$ 
 \end{theorem}
\begin{theorem}{结合律}
$\forall \alpha,\beta,\gamma\in\mathbb{R}$ , $(\alpha+\beta)+\gamma=\alpha+(\beta+\gamma)$
\end{theorem} 
\begin{theorem}{零元存在}
存在唯一的 $O\in\mathbb{R}$ ,使得 $\forall \alpha\in\mathbb{R}$ 有 $\alpha+O=\alpha$

\textbf{证明:}我们考虑全体负有理数的集合 $\mathbb{Q}^-=\{p\in\mathbb{Q}\mid p<0\}$ ,容易断定其为实数.

考察任意 $\alpha\in\mathbb{R}$

首先,对于 $\forall r\in\alpha,s\in\mathbb{Q}^-$ ,显而易见有:
$$r+s< r$$

因此 $r+s\in\alpha$ ,因而 $\alpha+\mathbb{Q}^-\subseteq\alpha$

其次,对于 $\forall r\in\alpha$ ,存在 $t\in\alpha$ 使得 $t>r$

于是 $r-t<0\in\mathbb{Q}^-$ ,而:
$$r=(r-t)+t$$

故 $r\in\alpha+\mathbb{Q}^-$ ,即 $\alpha\subseteq\alpha+\mathbb{Q}^-$

综上所述,$\alpha+\mathbb{Q}^-=\alpha$,零元被找到
\end{theorem}
\begin{theorem}{逆元存在}
\forall\alpha\in\mathbb{R} ,存在唯一的 \beta\in\mathbb{R} 使得 \alpha+\beta=O 
证明:对于 \alpha\in\mathbb{R} ,我们考虑 \beta=\{p\in\mathbb{Q}\mid\exists r\in\mathbb{Q}^+(-p-r\notin\alpha)\} 
容易知道 \beta\neq \emptyset ,考察 p\in\beta ,由定义知:
\exists r\in\mathbb{Q}^+(-p-r\notin\alpha)\\ 
首先,对于任意 \mathbb{Q}\ni q<p ,可知:
-p-r<-q-r\\ 
因而 -q-r\notin\alpha ,满足定义,故 q\in\beta 
其次,我们找到 \frac{r}{2}\in\mathbb{Q}^+ 使得:
-\left(p+\frac{r}{2}\right)-\frac{r}{2}=-p-r\notin\alpha\\ 
而 \frac{r}{2}>0 ,故 p+\frac{r}{2}>p ,即存在 p+\frac{r}{2}\in\beta 使得 p+\frac{r}{2}>p 
综上所述, \beta 是实数,这样定义加法逆元是合理的
我们取 \alpha+\beta\ni u=p+q ,其中 p\in\alpha,q\in\beta ,由定义知:
\exists r\in\mathbb{Q}^+(-p-r\notin\alpha)\\ 
而 -q>-p-r ,故 -q\notin \alpha ,而 p\in\alpha ,故:
-q>p\\ 
可得 u=p+q<0 ,所以 u\in\mathbb{Q}^- ,因而 \alpha+\beta\subseteq\mathbb{Q}^- 
对于任意 r\in\mathbb{Q}^- ,我们取 s=-\frac{r}{2} ,于是 s>0 
由有理数的阿基米德性质可知,存在 n\in\mathbb{N} 使得:
n\cdot s\in\alpha\,\,\text{but}\,\,(n+1)\cdot s\notin \alpha\\ 
取 p=-(n+2)\cdot s=\frac{(n+2)r}2 ,则:
-p-s=(n+1)\cdot s\notin\alpha\\ 
于是 p\in\beta 而 n\cdot s\in\alpha ,且:
r=-2s=n\cdot s-(n+2)\cdot s=n\cdot s+p\\ 
故 r\in\alpha+\beta ,因而 \mathbb{Q}^-\subseteq\alpha+\beta 
综上所述, \alpha+\beta=O ,逆元被找到
\end{theorem}
5. 序与加法的关系: \forall\alpha,\beta,\gamma\in\mathbb{R} ,若 \alpha\subseteq\beta 则 \alpha+\gamma\subseteq\beta+\gamma 
