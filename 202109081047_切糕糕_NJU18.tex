% 南京大学 2018 年考研普通物理
% 普通物理|普物|南大
\addTODO{画图}
\subsection{力学}
1. 在竖直平面内有一光滑的轨道,轨道左边是光滑的弧线,右边是足够长的水平直线.现有 $\mathrm{A}$、$\mathrm{B}$ 两个质点,质量分别 $m_{A}$ 和 $m_B$.质点 $B$ 静止于水平轨道上.将质点 $A$ 置于弧形轨道上,无初速释放.假定质点 $\mathrm{A}$、$\mathrm{B}$ 间的碰撞是完全弹性的.求 $\mathrm{A}$、$\mathrm{B}$ 至少能发生两次碰撞的条件.

2. 在光滑的水平桌面上开有一个小孔,一条不可伸长的轻绳穿过小孔.绳的两头各系一个小球.置于桌面上的小球以速率 $v_{0}$ 绕小孔作匀速圆周运动.桌面下的小球悬在空中,保持静止.假定桌面下面的小球的质量是桌面上小球的2倍.\\
(1)求桌面上绳子的长度 $l_{0}$;\\
(2)若给桌面上小球一个径向的小动量,则桌面下的小球将作上下小振动,求振动周期.
\subsection{热学}
1. 已知 $1 \mathrm{~mol}$ 某气体的定压膨胀系数
\begin{equation}
\alpha=\frac{1}{V}\left(\frac{\partial V}{\partial T}\right)_{P}=\frac{R}{P V}
\end{equation}
等温压缩系数 
\begin{equation}
\kappa=-\frac{1}{V}\left(\frac{\partial V}{\partial P}\right)_{T}=\frac{1}{P}+\frac{a}{V}
\end{equation}
其中 $R$、$a$ 是常数.求$1\opn{mol}$该气体的物态方程, 以及在温度 $T$ 时该气体的定压热容量随压强变化的关系.
\subsection{电磁学}
1. 高斯单位制中以厘米、克、秒分别作为长度、质量、时间的单位.求国际单位制与高斯单位制下电场强度的换算关系.已知国际单位制中 $1/(4 \pi \varepsilon_{0})=9 \times 10^{9} \opn{N \cdot m^{2}/C^{2}}$ 求精细结构常数 $\alpha$ 的数值.已知高斯单位制下 $\alpha=e^{2}/(\pi c)$.

2. 有一无限大金属平板,其一端接地,一带电量为 $Q$ 的小球与其垂直距离为 $d$,求金属板上感应电荷密度.
\subsection{光学}
1. 一油膜滴在玻璃上缓慢伸展扩大,一束 $\lambda=576 \opn{nm}$ 的光垂直照在玻璃上.已知,油膜最高点距玻璃 $d=857 \mathrm{~nm}$.油滴的折射率 $n_{1}=1.6$ ,玻璃折射率 $n_{2}=1.5$ 求\\
(1)可以看到多少条亮纹;\\
(2)各级亮纹的油膜厚度;\\
(3)随油膜伸展开,亮纹如何变化.
