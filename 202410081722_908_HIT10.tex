% 哈尔滨工业大学 2010 年 考研 量子力学
% license Usr
% type Note

\textbf{声明}:“该内容来源于网络公开资料,不保证真实性,如有侵权请联系管理员”

\subsection{共50分,每小题10分}
\begin{enumerate}
\item 证明:若$\hat{A}$和$\hat{B}$均为厄米算符,则$i[\hat{A},\hat{B}]$也为厄米算符
\item 设氢原子在$t=0$时出于状态\\\\
求其能量、角动最平方及角动量$Z$分量的的可能取值
\item 若一个算符与角动量算符$\hat{j}$的两个分量对易,则其必与$\hat{j}$的另一个分量对易。
\item 
\item 
\end{enumerate}
\subsection{(15分)}
对于一个系统,力学量算符$\hat{A}$与哈密顿算符$\hat{H}$(不易含时间$t$)不对易,已知$\hat{A}$的两个本征值为$a_1$和$a_2$,相应的本征函数分别为:
$$\psi_1=\frac{1}{\sqrt{2}}()~$$