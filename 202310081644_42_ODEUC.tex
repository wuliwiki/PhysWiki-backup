% 常微分方程解的存在、唯一及对参数的连续依赖定理
% keys 存在|唯一|连续依赖|常微分方程
% license Xiao
% type Tutor

\pentry{皮卡映射\upref{PicMap},基本知识(常微分方程)\upref{ODEPr}}
本节证明常微分方程的解的存在、唯一、及对参数的连续依赖定理。所谓 “解对参数的连续依赖” 是指微分方程
\begin{equation}\label{eq_ODEUC_2}
\dot x=v(x,t)~,
\end{equation}
 的解 $\varphi$ 也是某些参数 $\mu=(\mu_1,\cdots,\mu_m)$ 的函数,即 $\varphi(\mu,t)$。于是 \autoref{eq_ODEUC_2} 右边也因写成 $v(x,\mu,t)$。 即需要证明形如
\begin{equation}\label{eq_ODEUC_1}
\dot x=v(x,\mu,t)~
\end{equation}
的微分方程的存在唯一且对参数 $\mu$ 的连续依赖定理。然而,可以验证,对\autoref{eq_ODEUC_2} 证明存在唯一性和对初始点的连续依赖性,等价于证明\autoref{eq_ODEUC_1} 的存在唯一性和对参数 $\mu$ 的连续依赖性。事实上:
若微分方程\autoref{eq_ODEUC_2} 解 $\varphi$ 存在唯一且对初始点 $x$ 的连续依赖。记 
\begin{equation}
y=(x,\mu),\quad f=(v_x,v_{\mu})=(v,v_{\mu})~,
\end{equation}
则 \autoref{eq_ODEUC_1} 等价于(初始 $\mu$ 分量为 $\mu$ 的)微分方程
\begin{equation}\label{eq_ODEUC_3}
\begin{aligned}
\dot y=f(y,t)=(v(y,t),0)~.
\end{aligned}
\end{equation}
由假设,\autoref{eq_ODEUC_3} 的解 $\varphi(t)$存在唯一,且连续依赖于起始点 $y$(即解可写为 $\varphi(y,t)\equiv\varphi(t)$ 且 $\varphi(y,t_0)=y$),于是解也就连续依赖于 $\mu$;
反之,如果对\autoref{eq_ODEUC_1} 的微分方程存在唯一及对参数 $\mu$ 连续依赖的定理成立,则由 $\mu$ 是参数,可令 $v_\mu(x,t)=\equiv v(x,\mu,t)$ 则\autoref{eq_ODEUC_1} 等价于
\begin{equation}\label{eq_ODEUC_4} 
\dot x=v_{\mu}(x,t)~.
\end{equation}
由假定,其解存在唯一且对参数 $\mu$ 连续。于是设取参数 $\mu$ 对应起始点 $x_0$ 的情形,于是微分方程\autoref{eq_ODEUC_4} 的解存在唯一且对起始点连续。

一般的常微分方程都可以写为\autoref{eq_ODEUC_2} 的形式,并通过上面考虑,我们只需要证明对形为\autoref{eq_ODEUC_2} 的微分方程的解存在唯一且连续依赖于起始点即可。

\subsection{存在、唯一及对参数的连续依赖定理}
\begin{theorem}{存在、唯一及对参数的连续依赖定理}
设微分方程
\begin{equation}
\dot x=v(x,t)~
\end{equation}
中向量场 $v$ 在扩张相空间(\autoref{def_PSaPF_2}~\upref{PSaPF})中的一区域 $U$ 上有定义,则对任一点 $(t_0,x_0)\in U$,存在相空间中点 $x_0$ 的一邻域 $M$,使得任一给定的 $x\in M$,都存在一个 $t_0$ 的邻域区间,使得唯一存在满足初始条件 $\varphi(t_0)=x$ 的微分方程
\begin{equation}\label{eq_ODEUC_5}
\dot x=v(x,t)~
\end{equation}
的解 $\varphi(t)$ ,而且这个解连续依赖于初始点 $x$。即若将解 $\varphi$ 写成包含初始点 $(t_0,x)$ 的形式 $g(x,t)$,则 $g$ 对变量 $x$ 连续。
\end{theorem}
\textbf{证明:}由\autoref{the_PicMap_2}~\upref{PicMap},对任一点 $(t_0,x_0)$,都有要求的 $x_0$ 的领域 $M$(对应 $\abs{x-x_0}\leq b'$),使得 $\forall x\in M$ ,都有  $t_0$ 的一领域(对应 $\abs{t-t_0}\leq a'$)存在,使得皮卡映射(\autoref{def_PicMap_1}~\upref{PicMap})
\begin{equation}\label{eq_ODEUC_6}
(A\varphi)(x,t)\equiv x+\int_{t_0}^{t}v(\varphi(x,\tau),\tau) \,\mathrm{d}{\tau} ~
\end{equation}
是 $M$ 中的压缩映射($M$ 的定义见\autoref{the_PicMap_2}~\upref{PicMap})。由于压缩映射都有不动点(\autoref{the_ConMap_1}~\upref{ConMap}),而皮卡映射\autoref{eq_ODEUC_6} 的不动点 $g(x,t)$ 就是微分方程\autoref{eq_ODEUC_5} 的解(\autoref{the_PicMap_1}~\upref{PicMap}),由 $M$ 的完备性(\autoref{exe_PicMap_1}~\upref{PicMap}),该不动点对应的曲线也在 $M$ 中。而 $M$ 中的曲线都是连续的,故解的存在性和对初值点 $x$ 的连续性得证!

下面证明唯一性:设 $g_1(x,t),g_2(x,t)$ 都是初值条件 $g(x,t_0)=x$ 的微分方程\autoref{eq_ODEUC_5} 的解。由于 $g_1,g_2\in M$,所以可对其实行皮卡映射。由于在解存在的区域里皮卡映射是压缩映射,故成立(为何度量变成范数可类比线性算子度量空间\upref{ONorm},这里 $\norm{g}=\max\limits_{x,t}\abs{g(x,t)}$)
\begin{equation}\label{eq_ODEUC_7}
\norm{A g_1-A g_2}\leq\lambda\norm{g_1-g_2},\quad\lambda<1~.
\end{equation}
由于\autoref{the_PicMap_1}~\upref{PicMap},$g_1,g_2$ 都是 $A$ 的不动点,故\autoref{eq_ODEUC_7} 变成
\begin{equation}
\norm{g_1-g_2}\leq\lambda\norm{g_1-g_2},\quad\lambda<1~.
\end{equation}
上式只能在 $\norm{g_1-g_2}=0$ 时成立,这意味着唯一性。

\textbf{证毕!}