% 库尔特·哥德尔(综述)
% license CCBYSA3
% type Wiki

本文根据 CC-BY-SA 协议转载翻译自维基百科\href{https://en.wikipedia.org/wiki/Kurt_G\%C3\%B6del}{相关文章}。

\begin{figure}[ht]
\centering
\includegraphics[width=6cm]{./figures/6b68f02159236857.png}
\caption{哥德尔,大约在1926年} \label{fig_KRT_1}
\end{figure}
库尔特·弗里德里希·哥德尔(Kurt Friedrich Gödel,1906年4月28日-1978年1月14日)是一位逻辑学家、数学家和哲学家。与亚里士多德和戈特洛布·弗雷格一起,被认为是历史上最重要的逻辑学家之一,哥德尔深刻影响了20世纪的科学和哲学思维(当时,伯特兰·罗素、阿尔弗雷德·诺斯·怀特海德和大卫·希尔伯特正在利用逻辑和集合论研究数学基础),并在弗雷格、理查德·德德金德和乔治·康托尔的早期工作基础上进行扩展。

哥德尔在数学基础方面的发现导致了他在1929年通过其维也纳大学博士论文证明的完备性定理,并且两年后在1931年发表了哥德尔不完备性定理。第一个不完备性定理指出,对于任何足够强大、能够描述自然数算术的 ω-一致递归公理系统(例如,佩亚诺算术),存在一些关于自然数的命题,这些命题既无法从公理中证明,也无法被反驳。为了证明这一点,哥德尔发展了一种现在被称为哥德尔编号的技术,该技术将形式表达式编码为自然数。第二个不完备性定理从第一个定理中推导出来,指出该系统不能证明其自身的一致性。

哥德尔还证明了,在接受的策梅洛–弗伦克尔集合论(Zermelo–Fraenkel set theory)中,选择公理和连续统假设无法被反驳,前提是其公理是一致的。前一个结果为数学家们在其证明中假设选择公理打开了大门。他还通过澄清经典逻辑、直觉主义逻辑和模态逻辑之间的联系,对证明论做出了重要贡献。