% React (前端框架)笔记
% license Xiao
% type Note

\begin{issues}
\issueDraft
\end{issues}

\pentry{JavaScript 入门笔记\nref{nod_JS}}{nod_77a8}

\begin{itemize}
\item \href{https://react.dev/}{React} 是一个流行的 JavaScript 库, 用于制作单页网页 SPA 应用。
\item 本地开发环境一般是 \enref{Node.JS}{NodeJS} + VS code
\item 安装好 nodejs 和 npm 以后, 找到一个目录 \verb`npx create-react-app my-app` 然后 \verb`cd my-app`, 然后 \verb`npm start`。现在 \verb`my-app` 的开发服务器就开启了。
\item \verb`npx` 命令是 npm 提供的,用于临时运行程序而无需在电脑中永久安装包,也可以避免包的冲突。 例如 \verb`npx cowsay "Hello, world\!"` 在命令行输出一只牛,说这句话。
\item \verb`npm install ...` 会在当前文件夹安装某个包, 如果要直接安装到操作系统,就用 \verb`npm -g install ...`
\item 调试就用 Chrome 的开发环境(F12)。 可以先安装插件 \href{https://chromewebstore.google.com/detail/react-developer-tools/fmkadmapgofadopljbjfkapdkoienihi?pli=1}{React Developer Tools}。
\item 也有一些人会用 \verb`yarn` 而不是 \verb`npm` 作为包管理, \verb`npm install -g yarn
yarn create react-app my-new-app`
\item 常用辅助工具: \verb`npm install --save-dev eslint prettier`, \verb`npm install --save-dev typescript`
\end{itemize}
