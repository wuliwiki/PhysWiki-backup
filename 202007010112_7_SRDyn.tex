% 相对论动力学
\pentry{相对论加速度变换\upref{SRAcc},时空的四维表示\upref{SR4Rep}}


\subsection{什么是动力学 常见误解的辟谣}
初学牛顿力学时,你也许注意到了,课本将它分为两大部分:运动学和动力学.

运动学的研究范畴,基本上是纯数学的,只讨论了什么是位移、速度、加速度等概念,以及这些概念之间的数学联系.在运动学中有一个看起来很自然的假设,在了解相对论之前常常被忽略,那就是在不同参考系中这些概念和它们的联系是如何变化的.

动力学的研究范畴,则加上了“力”的概念,讨论力是如何影响物体的运动状态、力有什么性质等的.在牛顿力学中,力的性质由牛顿三定律决定,由此可以引申出更深刻的动量守恒和能量守恒等定律.

在狭义相对论中,参考系的变换导致的运动学变换复杂了很多,力的作用也不再像牛顿力学中那样简单.许多科普读物和视频会使用光子钟等模型来“推导”出爱因斯坦的明星方程,$E=mc^2$,或者用我们约定的单位制,$E=m$.这些推导的思路是用牛顿第二定律$F=m\dd x/\dd t$来定义“力”,然后利用洛伦兹变换计算在不同惯性系中,物体加速度的变化.当然,在使用伽利略变换的牛顿力学中,任何惯性系下物体的加速度都一样,因此质量


从相对论加速度变换\upref{SRAcc}词条中可以看到,物体