% 泡利方程
% keys Pauli 方程|泡利方程|薛定谔方程|电磁场
% license Xiao
% type Tutor

\pentry{电磁场中的薛定谔方程及规范变换\nref{nod_QMEM}}{nod_b3d9}

下文中所采用的电磁单位制为\enref{高斯单位制}{GaussU}或洛伦兹-亥维赛单位制(Lorentz–Heaviside Units)\footnote{参考:\href{https://en.wikipedia.org/wiki/Heaviside\%E2\%80\%93Lorentz_units}{维基百科}}。

泡利方程可以写成以下的形式
\begin{equation}
\boxed{
i\hbar\pdv{t} \varphi = \qty[\frac{(-i\hbar\nabla - (e/c)\bvec A)^2}{2m}+e\phi-\frac{e}{mc}\bvec S\cdot \bvec B]\varphi}~.
\end{equation}
其中 $\varphi$ 是一个二分量的波函数,$e=-|q_e|$ 代表电子的电荷量。它可以有效地描述的非相对论情形下电磁场中自旋 $1/2$ 粒子的行为。
\subsection{从狄拉克方程到泡利方程}
\pentry{狄拉克方程的非相对论近似\nref{nod_DiracB},电磁场中的狄拉克方程\nref{nod_DiracE}}{nod_a656}
从\autoref{eq_DiracB_3}~\upref{DiracB}出发:
\begin{equation}
i\hbar\pdv{t} \varphi = \frac{1}{2m}\qty[\bvec \sigma \cdot\qty(\bvec P - \frac{q}{c}\bvec A)]^2 \varphi + e\phi  \varphi~.
\end{equation}
并利用\autoref{eq_Pauli_1} 开始的一系列推导对它进行化简,可以直接得到泡利方程。这种推导方式简单粗暴,直接取相对论性狄拉克方程的非相对论极限得到最终的解。

其中 $\frac{e}{mc} \bvec S\cdot \bvec B$ 是自旋与磁场耦合项。代表电子磁矩与外磁场之间的相互作用。如果将这一项舍去将可以得到非相对论情形下电磁场中无自旋带点粒子的薛定谔方程 \autoref{eq_QMEM_1}~\upref{QMEM}。
\subsection{从自旋 $1/2$ 粒子的非相对论方程到泡利方程}
\pentry{自旋 1/2 粒子的非相对论波函数\nref{nod_scheq2}}{nod_90f1}
或者我们也可以换一个角度,这种思路直接从非相对论的 Galilean 变换出发\cite{刘觉平},依据非相对论情形下的时空对称性得到量子力学的算符以及电子波函数所满足的非相对论方程。虽然这种推导看起来拐弯抹角,但其实和狄拉克方程的推导是完全一致的。大部分的推导已经在\enref文章 \upref{scheq2} 中展现过,我们这里直接采用\upref{scheq2} 中的结果。对于自旋 $1/2$ 粒子,描述它的非相对论性波函数具有 $4$ 个分量,可以表示为 $\psi=\pmat{\varphi\\\chi}$,其中 $\varphi,\chi$ 为双分量的波函数,且满足方程\autoref{eq_scheq2_3}~\upref{scheq2}:
\begin{equation}
\begin{aligned}
&\qty[\hat H-\frac{(\bvec \sigma\cdot \hat{\bvec P})^2}{2m}]\varphi=0~,\\
&
\qty[\hat H-\frac{(\bvec \sigma\cdot \hat{\bvec P})^2}{2m}]\chi=0~.
\end{aligned}
\end{equation}
在电磁场中,需要对 $\hat H,\hat{\bvec P}$ 作如下的替换:
\begin{equation}
\hat H\rightarrow \hat H-e\phi, \hat{\bvec P}\rightarrow \hat{\bvec P}-\frac{e}{c}\bvec A~.
\end{equation}
其中 $\phi$ 为电势,$\bvec A$ 为磁矢势。那么
\begin{equation}\label{eq_Pauli_1}
\qty[\bvec \sigma\cdot(\hat{\bvec P}-\frac{e}{c}\bvec A)]^2=(\hat{\bvec P}-\frac{e}{c}\bvec A)^2+i\bvec \sigma\cdot[(\hat{\bvec P}-\frac{e}{c}\bvec A)\times (\hat{\bvec P}-\frac{e}{c}\bvec A)]~.
\end{equation}
注意这里的 $\hat \sigma,\hat{\bvec P},\bvec A$ 都应当被视为作用于 Hilbert 空间上的算符,这个 Hilbert 空间中的“矢量”是如同 $\phi,\chi$ 那样的二分量波函数。因此算符间不一定满足交换律,所以右边的两个算符的“叉积”不为 $0$。计算可得
\begin{equation}
\begin{aligned}
\qty[\bvec \sigma \cdot (\hat{\bvec P}-\frac{e}{c}\bvec A)]^2
&=(\hat{\bvec P}-\frac{e}{c}\bvec A)^2+i\bvec \sigma\cdot[-\frac{e}{c}(\hat{\bvec P}\times \bvec A- \bvec A\times \hat{\bvec P})]\\
&=(\hat{\bvec P}-\frac{e}{c}\bvec A)^2+i\bvec \sigma\cdot[\frac{e}{c}i\hbar \nabla \times  \bvec A]\\
&=(\hat{\bvec P}-\frac{e}{c}\bvec A)^2-\frac{e}{c}\hbar\bvec \sigma\cdot \bvec B~.
\end{aligned}
\end{equation}
于是有泡利方程
\begin{equation}
\qty[(\hat H-e\phi)-\frac{1}{2m}(\hat{\bvec P}-\frac{e}{c}\bvec A)^2+\frac{e}{2mc}\hbar\bvec \sigma\cdot \bvec B]\varphi=0~,
\end{equation}
或者写为
\begin{equation}
i\hbar\pdv{t} \varphi = \qty[\frac{(-i\hbar\nabla - (e/c)\bvec A)^2}{2m}+e\phi-\frac{e}{mc}\bvec S\cdot \bvec B]\varphi~,
\end{equation}
$\bvec S$ 为自旋算符。注意到自旋 $1/2$ 的粒子所满足的泡利方程比自旋为 $0$ 的粒子在电磁场中的薛定谔方程要多出一项 $-\bvec \mu\cdot \bvec B$,其中 $\bvec \mu=e\bvec S/m$ 为粒子的自旋磁矩。
