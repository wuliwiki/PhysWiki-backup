% 凸函数
% 凸函数|凹凸性|琴生不等式

\pentry{导数(数学分析)\upref{Der2}}

设函数 $f(x)$ 在区间 $I$ 上有定义,若对任意的 $x_1,x_2\in I$,对任意 $t\in (0,1)$,都有
\begin{equation}
f(tx_1+(1-t)x_2)\le tf(x_1)+(1-t)f(x_2)
\end{equation}
那么称 $f(x)$ 为 $I$ 上的\textbf{凸函数}.如果将 $\le$ 改为 $<$,那么称 $f(x)$ 为\textbf{严格凸函数}.类似地可以定义\textbf{凹函数}和\textbf{严格凹函数}.
\begin{theorem}{}\label{ConvFu_the1}
设 $f(x)\in C[a,b]$.$f(x)$ 满足
\begin{enumerate}
\item 在 $(a,b)$ 上可导.则 $f(x)$ 是(严格)凸函数的充分必要条件是 $f'(x)$ 在 $(a,b)$ 内(严格)单调递增.
\item 在 $(a,b)$ 上二阶可导.则 $f(x)$ 是凸函数的充分必要条件是 $f''(x)\ge 0$;$f(x)$ 是严格凸函数的充分必要条件是 $f''(x)\ge 0$ 且 $f''(x)$ 在 $(a,b)$ 的任意子区间上都不恒等于 $0$.
\end{enumerate}
\end{theorem}
凸函数的以上几条充分必要条件可以从图像中直观地理解,事实上,画图常常有助于解决与凸函数性质相关的问题.
\begin{exercise}{}
证明两个 $I$ 上的凸函数相加仍然是 $I$ 上的凸函数.
\end{exercise}
提示:由于给定函数在区间 $I$ 上不一定可导,所以不能用  
\autoref{ConvFu_the1} 证明,必须回归到定义.
\begin{exercise}{}
设 $f(x)$ 是区间 $(a,b)$ 上的凸函数,证明 $f(x)$ 在 $(a,b)$ 内的任意闭区间 $[\alpha,\beta]$ 上都满足 Lipschitz 条件(也就是说存在 $L>0$,对任意 $x_1,x_2\in [\alpha,\beta]$,都有 $|f(x_1)-f(x_2)|\le L|x_1-x_2|$).

\textbf{提示:}如果将 $[\alpha,\beta]$ 改为开区间,则很容易找到反例,所以在证明这个命题时要充分利用好闭区间的特性,尝试将 $L$ 的上界量化.取充分小的 $h$,使 $[\alpha-h,\beta+h]\subset (a,b)$,令 $x_3=x_2+h$,根据凸函数的定义(事实上可以从图像上很快得出)$(f(x_2)-f(x_1))/(x_2-x_1)<(f(x_3)-f(x_2))/(x_3-x_2)\le (M-m)/h$,其中 $M,m$ 是区间 $[\alpha-h,\beta+h]$ 上函数的最大、最小值(可以证明,闭区间上的凸函数一定有最大最小值).于是可以令 $L=(M-m)/h$.
\end{exercise}
