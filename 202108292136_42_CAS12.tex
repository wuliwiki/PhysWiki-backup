% 中科院2012年普物
% 中科院|2012|普通物理
\subsection{选择题}
1. 如\autoref{CAS12_fig1} 所示, 两个固定小球质量分别是 $m_{1}$ 和 $m_{2}$, 在它们的连线上总可以找到一
点 $p$, 使得质量为 $m$ 的质点在该点所受到的万有引力的合力为零, 则质点在 $p$
点的万有引力势能\\
(A) 与无穷远处的万有引力势能相等;\\
(B) 与该质点在两小球连线间其它各点处相比势能最大;\\
(C) 与该质点在两小球连线间其它各点处相比势能最小;\\
(D) 无法判定.
\begin{figure}[ht]
\centering
\includegraphics[width=3.5cm]{./figures/CAS12_1.pdf}
\caption{选择题1图示} \label{CAS12_fig1}
\end{figure}
2.两个全同的均质小球 $\mathrm{A}$ 和 $\mathrm{B}$ 都放置在光滑水平面上, 球 $\mathrm{A}$ 静止.在某一时刻 球 $\mathrm{B}$ 与 $\mathrm{A}$ 发生完全弹性斜碰撞(即碰撞时 $\mathrm{A}$、 $\mathrm{~B}$ 的质心连线方向与球 $\mathrm{B}$ 的速 度方向不同),则碰撞后两球的速度方向\\
(A) 相同;\\
(B) 夹角为锐角;\\
(C)相垂直;\\
(D) 夹角为钝角.