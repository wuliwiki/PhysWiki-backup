% Python 语法笔记
% license Xiao
% type Note

\begin{issues}
\issueDraft
\end{issues}

\addTODO{参考 GitHub 的 \href{https://github.com/MacroUniverse/MyPythonLibrary}{MyPythonLibrary}, 还有以前 computational physics 课的作业等}

\subsection{常识}
\begin{itemize}
\item 严格区分大小写
\item 不需要声明变量
\item 每一个 block 前面的空格数需要一致(缩进一般用四个空格)
\item 缩进的个数很重要, python 用缩进来区分不同的 scope。 一般建议用四个空格
\item 用 \verb`#` 号进行注释
\item 用 \verb`\` 把一行长代码转移到下一行. (相当于Matlab里面的省略号)
\item 逗号赋值 \verb`a,b,c=1,2,3`。 这其实是 tuple 赋值, 即 \verb`(a,b,c)=(1,2,3)`。 等号右边也可以是一个 list。
\item 连等 \verb`x = y = z = "Orange"`
\item 调用没有自变量的函数需要保留括号  \verb`MyFun()`
\item \verb`locals()` 和 \verb`globals()` 查看所有变量
\item \verb`if 'a' in locals()` 用于查询变量 a 是否存在
\item 上一次输出储存在变量 \verb`_` 中
\item \verb`import sys`, 可以获得命令行的 arguments 数组(list of string), \verb`sys.argv[0]` 是调用程序的命令, 剩下的是跟随的 arguments
\item 见\enref{路径处理笔记}{PyPath}。
\item \verb`from math import *` 用这个格式可以在代码中直接输入 math 包里面的函数名, sin, cos, 而不用 \verb`math.sin`
\item \verb`print()` 相当于 Matlab 的 \verb`disp()`, \verb`print(string,end=' ')` 可在输出后面用空格隔离, 默认为回车. 
\end{itemize}

\subsection{变量类型}
\begin{itemize}
\item 用 \verb`type()` 可以检查任意变量的类型
\item 有一类变量叫做 \verb`type`, 例如 float, bool, int 等.
\item 强制类型转换 \verb`int()`, \verb`float()` 等等
\item \verb`float` 就是 C 的双精度 \verb`double`, 有 16 位有效数字
\item \verb`int` 类型可以储存任意位数的整数, 不会溢出
\item \verb`1+2j` 或者 \verb`1+2J` 表示复数,不能用 \verb`i`
\item 在最外层 scope 定义的变量就是全局变量。 例如在函数外定义的变量可以在函数内读取。
\item 一旦试图在函数内修改全局变量, 它马上就会在整个函数内部视为另一个局部变量,如果修改之前试图读取就会报错。
\item 函数内用 \verb`global 变量1, 变量2` 可以声明需要修改的全局变量, 也就是函数外定义的变量。 如果只用于读取则无需声明
\end{itemize}

\subsection{字符串类 (str)}
\begin{itemize}
\item \verb`'...'` 和 \verb`"..."` 都是字符串,只是里面对 \verb`'` 和 \verb`"` 的转义不同。
\item 可以用加号 \verb`+` 合并两个字符串。
\item raw string:\verb`r'...'`,\verb`r"..."`,\verb`r"""..."""` 可以把字符串原样输出,不做任何转义。 \verb`r"""..."""` 内支持换行。
\item str 类型, \verb`a='word';` 也可以 \verb`a="""word"""` 后者可以输入多行字符
\item \verb`str*10`, 就是 \verb`'strstrstrstrstrstrstrstrstrstr'`.
\item string 类型不是 list,不支持索引修改
\item \verb`str(a)` 可以把其他类型变为字符串
\item \verb`sep='00'; sep.join(['1','2','3'])` 可以生成 \verb`'1002003'`
\item \verb`str.find('key', ind)` 可以从某个位置开始搜索下一个 \verb`key`。 搜不到返回 \verb`-1`。 例如 \verb`str[:str.find('.')]` 可以获取句号之前的内容。
\item \verb`str.replace('key', 'new_key')` 把所有的 \verb`key` 替换成 \verb`new_key` 并输出。 \verb`str` 本身不会被改变。
\end{itemize}

\subsection{逻辑类 (bool)}
\begin{itemize}
\item 逻辑变量是 \verb`True` 和 \verb`False`, 注意大写
\item 逻辑与是 \verb`and`, 或是 \verb`or`, 非是 \verb`not`, 异或是 \verb`^`
\item \verb`NoneType`, 当函数没有输出的时候返回.
\end{itemize}

\subsection{Math}
\begin{itemize}
\item 牢记指数是 \verb`**` 不是 \verb`^` !!
\item 有些版本中 \verb`3/4=0`, 因为 \verb`int` 相除输出 \verb`int`, 向下取整. 这时把任意一个转换成 float 即可. \verb`3.0/4=0.75`
\item 整除是 \verb`//`,总是向负无穷取整(C 语言向 0 取整)。 如果其中一个参数是浮点数,那么结果也会是浮点数。 如果两边都是整数输出才是整数。
\item 求余是 \verb`%`
\end{itemize}

\subsection{判断}
\begin{itemize}
\item \verb`if <条件语句> :`
\item \verb`if a in 容器:` 判断   一定记住加冒号!
\item \verb`elseif` 要写成 \verb`elif`
\item \verb`if` 后面加非零整数是 \verb`false`,加零是 \verb`true`,单元素的空 list 是 false,否则是 true
\item 如果某个分支中定义了某个变量且被执行, 分支结束后它仍有定义
\end{itemize}

\subsection{循环}
\begin{itemize}
\item while 循环的例子  \verb`while a > b :`
\item for 循环的例子 \verb`for i in range(inta, intb):`, 注意 \verb`i` 最大值是 \verb`intb-1`
\item 循环结束后, \verb`i` 仍然有定义, 值为最后一次循环中的值。
\item 循环体中新定义的其他变量在循环结束后也同样还有定义
\item \verb`for i in range(0,5):` 或 \verb`for i in range(5):` 和 \verb`for i in [0,1,2,3,4]:` 等效
\item \verb`range(inta,intb)` 实际上不是数组, 输出的数据类型叫做 \verb`range`.
\item \verb`for i, arg in enumerate(容器): ...` 可以在遍历容器的同时添加一个计数变量 \verb`i`。 也可以用 \verb`enumerate(容器, start=N)` 指定第一个循环中的计数
\item 用 \verb`for key in 字典: ...` 可以只遍历 \verb`key`, \verb`for key,val in 字典.values(): ...` 可以遍历值,\verb`for key,val in 字典.items(): ...` 可以遍历 \verb`(key, val)`。
\item \verb`break` 跳出任意循环 (for/while), \verb`continue` 跳到下一个循环。
\item \verb`pass` 是一个用来填充格式的语句, 不做任何事情, 可以放在循环中或函数定义中。
\item \verb`for` 或 \verb`while` 下面可以加上一个 \verb`else`, 如果循环没有被 \verb`break` 打断, 则会执行 \verb`else`。
\item \verb`range()` 函数可以有 1 到 3 个变量, \verb`range(n)` 相当于 \verb`[0, … , n-1]`, \verb`range(n1,n2,step)` 相当于 \verb`[n1, …, n2-1]` 步长为 step 
\item \verb`for ... else ...` 结构中, 如果循环中没有出发 \verb`break`, \verb`else` 就会被执行。
\end{itemize}

\subsection{数组(list)}
\begin{itemize}
\item list 的每一个元素可以是不同的变量类型
\item 数组索引中 \verb`i:j` 实际表示从 \verb`i` 到 \verb`j-1` 的元素, 另外和 C 语言一样, 数组 index 都是从 0 开始的
\item 直接定义数组 \verb`a=[1,2,3,4], b=[[1,2],[3,4,[7,8],5],[6]]`, 可以任意数量任意嵌套.
\item \verb`a = [x*2 for x in range(0,3)]` 的结果是 \verb`a = [0,2,4]`
\item \verb`a = [[ii+jj for ii in range(0,3)] for jj in range(0,3)]` 的结果是 \verb`a = [[0,1,2],[1,2,3],[2,3,4]]`
\item \verb`len()` 获取数组最外层长度
\item 数组索引可以用负数,-1 代表最后一个元素,-2 代表倒数第二个元素等。
\item 拼接两个数组:\verb`[*[1,2],3,*[4,5,6]]=[1,2,3,4,5,6]`
\item 使用 \verb`a[ind1:ind2]` 超出范围时并不会出错。
\item 二维数组 \verb`a = [[0,1,2],[1,2,3],[2,3,4]]`, \verb`a[i][j]` 索引数组中的元素,也可以用 \verb`a[:,0:1]` 提取第 0 到第 1 列的所有元素。
\item 如果用 \verb`B = A` 复制数组,python 会把用两个变量名视为等效,数组还是同一个。
\item 要真正复制数组, 用 \verb`B = A.copy()`, \verb`B = A[:]`, 或者 \verb`B = list(A)`
\item \verb`a == b` 判断它们的值是否相等, \verb`a is b` 判断它们是否指向同一个对象
\item \verb`list` 或者 \verb`nparray` 在函数中如果给自变量元素进行赋值,那么函数结束以后自变量也会改变!python 这是什么鬼啊!所以在函数中千万不要变动原数组,除非在函数外也不需要了。
\item list 的不同元素可以装不同的变量类型!
\item 用小括号的 list 叫做 tuple
\end{itemize}

\subsection{常用容器}
\begin{itemize}
\item list 相当于数组, \verb`listA + listB` 操作合并两个 \verb`list`。 \verb`list.append()` 可以把输入的东西变成一个元素加入到 list 最后。 \verb`list.reverse()` 可以颠倒 list 的顺序。
\item \verb`listA.append(listB)`
\item dictionary 类.  \verb`dict={"a":"b", "c":"d", "e":"f"}`
\item unordered set: \verb`my_set = {"a", b, 2}`。 空集用 \verb`my_set = set()`
\item \verb`my_set.add(元素)` 可以添加元素, \verb`my_set.remove(元素)` 删除元素
\item \verb`if key in 容器:` 可以用于检查 \verb`key` 是否在容器中, 几乎所有容器都适用
\end{itemize}

\subsection{匿名函数}
\begin{itemize}
\item 定义格式 \verb`f = lambda x, y: a*(x**2+y**2)` 相当于 Matlab 中的 \verb`f=@(x,y) a*(x^2+y^2)`
\item 注意参数是动态变化的! 如果定义了 lambda 以后再改动 \verb`a`, 那么再次调用 lambda 时 \verb`a` 会更新.
\item 在函数定义内可以使用函数外定义的 lambda 函数, 但是不能更改其参数.
\end{itemize}

\subsection{函数}

\begin{itemize}
\item 基本格式
\begin{lstlisting}[language=python]
def FunctionName(a,b,c):
""" 函数的描述, 会出现在documentation中 (详见P25)
     三个引号可以占用多行 """
    ...
    ...
   return d,e,f,g
\end{lstlisting}

函数中任何return都会跳出函数.

\item 调用函数
\verb`d,e,f,g=FunctionName(a,b,c)`
\verb`d,e,f,g=Function Name(b=1,a=2,c=3)`  这样可以不按顺序输入
\item \verb`func(*list)` 可以把 list 中的变量按顺序输给 function 做自变量.

\item 自定义函数的类是 \verb`function`, 可以把函数名赋值给另一个变量.
\item 系统函数的类是 \verb`builtin_function_or_method`

\item 内嵌函数
内嵌函数可以自动读取母函数的所有变量, 但只要函数中有对该变量名进行赋值, 就不能从母函数中读取, 相当于子函数的一个本地新变量.

\item 定义函数自变量的默认值  \verb`def func(a,b=1,c='str'):` 调用时必须按顺序输入
\item 4类输入类型: positional arguments, keyword arguments, \verb`*name`, \verb`**name`. 定义函数时用 \verb`*name` 可以接受任意数量的 positional argument, \verb`**name` 可以接受任意数量的 keyword argument.
\end{itemize}

\subsection{常用函数}
\begin{itemize}
\item \verb`dir()` 可以获取当前的所有对象
\end{itemize}

\subsection{杂}
\begin{itemize}
\item Python 完全是面对对象的
\item \verb`os.system('clear')` 或 Windows 中 \verb`os.system('cls')` 可以清空 REPL
\item \enref{Spyder}{spyder} 的强项是科学计算
\item Anaconda 清洁地装在一个单独的目录, 不需要系统权限, 不影响其他目录下的安装
\item 常用的 IDE 有, Spyder, Wing IDE, PyCharm, Python Tools for Visual Studio.
\item 其实 Jupiter 并不比 Spyder 要慢, 前者具有格式优势, 后者具有调试优势.
\end{itemize}
