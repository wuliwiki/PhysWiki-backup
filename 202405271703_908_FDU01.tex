% 复旦大学 2001 量子真题
% license Usr
% type Note


\textbf{声明}:“该内容来源于网络公开资料,不保证真实性,如有侵权请联系管理员”

复旦大学 2001年招收攻读硕士研究生入学考试试题

1. 质量为 $m$ 的粒子在一维对称势场 $V(x) = Ax^2$ 中运动,请用量子化条件求粒子能量 $E$ 的可能取值。(20分)

2. 质量为 $m$ 的粒子处在一维谐振子势 $V_0(x) = \frac{\alpha^4 x^2}{2m}$ 的基态,波函数函数为
$$ varphi_0(x) = \left(\frac{\alpha}{\pi}\right)^{1/4} e^{-\frac{\alpha x^2}{2}}~,$$ 
若势能 $V_0(x)$ 突然变为 $V_1(x) = \frac{\beta^4 x^2}{2m}$,问在某后任意时刻,粒子处在基态的几率。若势能是非常缓慢地从 $V_0(x)$ 变为 $V_1(x)$,结果如何?(20分)

3. 分析、讨论或求解粒子在势阱
$$  V(x) = \begin{cases} \infty, & x < 0; \\\\ \\frac{1}{2} m \omega^2 x^2, & x \geq 0 \end{cases}~,$$ 
中的能量。(20分)
中运动的能级 (20分)

4. 设体系处在 \\( w = c_1 I_1 + c_2 I_2 ) 状态,求

(1) \\( I_1 ) 的可能测值及平均值;

(2) \\( I_2 ) 的可能测值及相应几率;

(3) \\( I_1 ) 的可能测值及相应几率。 (20 分)

5. 对于一维谐振子,取基态试探波函数形状为 $( e^{-\lambda x^2} )$,$(\lambda)$ 为参数。用变分法来
    估定基态能量和波函数,并与严格解比较。(20 分)
