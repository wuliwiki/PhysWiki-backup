% X 射线衍射技术
% license CCBYSA3
% type Wiki

(本文根据 CC-BY-SA 协议转载自原搜狗科学百科对英文维基百科的翻译)

\textbf{X射线衍射法}是一种揭示材料和薄膜的晶体结构、化学成分和物理性质的无损分析技术。这些技术是基于观察X射线照射到样品上的散射强度,它是入射和散射角度、偏振、波长或能量的函数。

需要注意的是,X射线衍射现在通常被认为是X射线散射的的一个子集,其中散射是弹性的,而散射对象是晶体,因此所得图案包含由 X射线晶体学分析的尖锐斑点(如图所示)。然而,散射和衍射都是相关的普遍现象,它们之间并不总是存在区别。因此,1963年Guinier的经典著作[1]《晶体、不完美晶体和非晶中的X射线衍射》”,所以“衍射”在当时显然并不局限于晶体。

\subsection{散射技术}
\subsubsection{1.1 弹性散射}
\begin{itemize}
\item $X$射线衍射或者更具体地说是广角$X$射线衍射(WAXD)
\item 小角$X$射线散射(SAXS)通过测量散射角$2\theta$接近0°时的散射强度来探测纳米到微米范围内的结构。
\item $X$射线反射率是一种分析技术,用于确定单层和多层薄膜的厚度、粗糙度和密度。
\item 广角$X$射线散射 (WAXS),一种专注于散射角$2\theta$大于5°的技术。
\end{itemize}
\subsubsection{1.2 非弹性x光散射(IXS)}
在IXS中,对非弹性散射X射线的能量和角度进行监测,给出动态结构因数 $S(q, \omega)$ 。由此可以获得材料的许多性质,具体性质取决于能量转移的尺度。下表列出了技术,改编自[2]。非弹性散射的X射线具有中间相位,因此原则上对X射线晶体学没有用处。实际上,由于弹性散射,能量传输较小的X射线包含在衍射斑点中,而能量传输较大的X射线对衍射条纹的背景噪声有贡献。

\begin{table}[h]
    \centering
    \begin{tabular}{|c|c|c|c|}
        \hline
        技术 & 典型入射能量 (keV) & 能量转移范围 (eV) & 以下信息 \\
        \hline
        康普顿散射 & 100 & 1000 & 贵米表面形状 \\
        \hline
        共振IXS (RIXS) & 4-20 & 0.1 - 50 & 电子结构和激发 \\
        \hline
        非共振IXS & 10 & 0.1 - 10 & 电子结构和激发 \\
        \hline
        X射线散射剪裁 & 10 & 50 - 1000 & 吸收边结构,键,价态 \\
        \hline
        高分辨率IXS & 10 & 0.001 - 0.1 & 原子动力学,声子色散 \\
        \hline
    \end{tabular}
    \caption{IXS技术及其应用}
\end{table}