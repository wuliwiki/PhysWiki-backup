% Euler-Maclaurin 求和公式
% keysEuler-Macluarin求和公式|Bernoulli数|Bernoulli多项式
\pentry{反常积分\upref{impro} 复变函数的积分\upref{CpxInt} 傅里叶级数(三角)\upref{FSTri}  渐近展开\upref{Asympt}} 

由于自然数任何形式下的求和公式中的和函数一般并不容易求出,因此数学家Euler利用原函数来拟合和函数.下面我们先介绍一个较为简单的Euler求和公式,它在一般情况下已经是一个较好的结果.而完整的Euler-Macluarin求和公式分别由Euler和Maclaurin独立发现,前者主要应用于求和函数,后者利用它来处理数值积分.

\subsubsection{Euler求和公式}

\begin{theorem}{}
设函数 $f:X\to Y$ 在区间$[0,+\infty)$上连续可微,则有如下Euler求和公式成立	
        \begin{equation}\label{EMSum_eq1} 
            \sum_{k=1}^{n}f(k)=\int_{0}^{n}f(x)\,\dd x
            +\frac{f(n)-f(0)}{2}+\int_{0}^{n}\psi(x)f'(x)\dd x
        \end{equation}
    其中 $\psi(x)=x-\lfloor x \rfloor-1/2=\{x\}-1/2$
\end{theorem}
证明如下:将区间$[0,n]$划分为长度为$1$的小区间$[k-1,k](k=1,2,\cdots,n)$,则
​\[
    \int_{k-1}^{k}\psi(x)f'(x)\,\dd x
    =\frac{f(k)+f(k-1)}{2}-\int_{k-1}^{k}f(x)\,\dd x
\]
​对$k$从1到$n$求和,于是
​\[
    \int_{0}^{n}\psi(x)f'(x)\,\dd x
    =\sum_{k=1}^{n}f(k)-\frac{f(n)-f(0)}{2}-\int_{0}^{n}f(x)\,\dd x
\]
这样就证明了Euler求和公式.注意到\autoref{EMSum_eq1}  右端积分比较复杂,一般情况下仅仅估计其上界便可得到较为不错的结果.设函数 $f\in{C^1[0,+\infty)}$ ,再设函数$\varphi(x)=\displaystyle{\int_{0}^{x}\psi(t)\,\dd t}$,易知该函数周期为$1$,且
 $-1/8\leqslant\varphi(x)\leqslant 0$,因此
\[
    \left|\int_{0}^{n}\psi(x)f'(x)\,\dd x\right|
    =\left|\int_{0}^{n}\varphi(x)f''(x)\mathrm{d}x\right|
    \leqslant\frac{|f'(n)-f'(0)|}{8}
\]
\begin{figure}[ht]
\centering
\includegraphics[width=10cm]{./figures/EMSum_1.pdf}
\caption{函数$\varphi(x)$} \label{EMSum_fig1}
\end{figure}
在实际情况中,更为常见的是函数 $f$ 在$[1,+\infty)$ 上可微,则补充定义$ f\equiv 0,x\in[0,1)$.在\autoref{EMSum_eq1}  中令$n=1$,从而
\[
    \int_{0}^{1-\delta}\psi(x)f'(x)\mathrm{d}x
    =\frac{f(1-\delta)+f(0)}{2}-\int_{0}^{1-\delta}f(x)\,\dd x=0
\]
由于令$\delta\to0+0$时,极限存在且为$0$,则对于在区间$[1,+\infty)$连续可微函数$f$就有
\begin{equation}\label{EMSum_eq2}
    \sum_{k=1}^{n}f(k)
    =\int_{1}^{n}f(x)\,\dd x+\frac{f(n)+f(1)}{2}+\int_{1}^{n}\psi(x)f'(x)\,\dd x
\end{equation}
在\autoref{EMSum_eq2}  式中,若积分$\displaystyle{\int_{1}^{\infty}\psi(x)f'(x)\,\dd x}$收敛,则
\begin{equation}\label{EMSum_eq3}
    \sum_{k=1}^{n}f(k)=\int_{1}^{n}f(x)\,\dd x
    +\frac{f(n)}{2}-\int_{n}^{\infty}\psi(x)f'(x)\,\dd x
    +\int_{1}^{\infty}\psi(x)f'(x)\,\dd x+\frac{f(1)}{2}
\end{equation}
\begin{example}{}
Euler-Mascheroni常数\upref{Masche}由调和级数与自然对数的差值的极限所给出,利用\autoref{EMSum_eq3} 和该常数可以给出类调和级数前$n$项和较好的计算公式,即
\begin{equation}
    H_{n}=\sum_{k=1}^{n}\frac{1}{k}=\log n+\frac{1}{2n}+C_{2}
    +\int_{n}^{\infty}\frac{\psi(x)}{x^2}\,\dd x
\end{equation}
其中
\begin{equation}
    C_{2}=\frac{1}{2}-\int_{1}^{\infty}\frac{\psi(x)}{x^2}\,\dd x
\end{equation}
由Dirichlet定理可知, 积分存在,因此将 结合
Euler-Mascheroni常数代入 就有$C_{2}=\gamma$,从而
\begin{equation}
    H_{n}=\log n+\frac{1}{2n}+\gamma+\int_{n}^{\infty}\frac{\psi(x)}{x^2}\,\dd x
\end{equation}
\end{example}

\subsection{Bernoulli数与Bernoulli多项式}

前面我们已经得到了一个较为不错的求和式,为了得到更精确的结果,我们需要深入分析\autoref{EMSum_eq1} 右端积分的一些性质,为此引入Bernoulli数和Bernoulli多项式.

\begin{definition}{}
设Bernoulli数为 $B:\mathbb{N}\to\mathbb{R} $,它有多种定义方式,其生成函数定义为
\begin{equation}
    \frac{z}{e^{z}-1}=\sum_{n=0}^{\infty}B_{n}\frac{z^{n}}{n!}
    \quad(|z|<2\pi,z\in{\mathbb C})
\end{equation}
\end{definition}
由Taylor公式和Cauchy乘积,对照系数就有
\begin{equation}\label{EMSum_eq4}
    \sum_{k=0}^{n}\dbinom{n+1}{k}B_{k}=0\quad(n\in{\mathbb N^{+}})
\end{equation}
依据\autoref{EMSum_eq4} ,计算前10项Bernoulli数(由生成函数定义就有$B_{0}=1$).列举如下:
\begin{table}[ht]
\centering
\caption{Bernoulli数}
\begin{tabular}{|c|c|c|c|c|c|c|c|c|c|c|}
\hline
$ n $ & 0 & 1 & 2 & 3 & 4 & 5 & 6 & 7 & 8 & 9\\ 
\hline
$ B_{n} $ & $1$ & $-\dfrac{1}{2}$ & $ \dfrac{1}{6}$ & $0$ & $-\dfrac{1}{30}$ & $0$ & $\dfrac{1}{42}$ & $0$ & $-\dfrac{1}{30}$ & $0$\\
\hline
\end{tabular}
\end{table}
事实上,通过一定的计算可以发现似乎对于$n\in\mathbb N^+$,有$B_{2n+1}=0$.下面我们借助复分析给出一个较为简洁的证明.\\
不妨设函数
\[
f(z)=\frac{z}{e^{z}-1}=\sum_{k=0}^{\infty}B_{k}\frac{z^{k}}{k!}\quad|z|<2\pi
\]
设圆周 $\gamma:|z|=1$, 可知 $f(z)$ 在 $|z|\leq 1$ 内解析,因此函数$\dfrac{f(z)}{z^{2n+2}}$在$|z|\leq 1$的洛朗展开式为
\[
\frac{f(z)}{z^{2n+2}}=\sum_{k=0}^{\infty}\frac{B_{k}}{k!}{z^{k-2n-2}}\quad n\in\mathbb N
\]
由留数定理可得
\[
\oint_{\gamma}\frac{f(z)}{z^{2 n+2}}\,\dd z=2\pi\mathrm i\cdot\frac{B_{2 n+1}}{(2n+1)!}
\]
于是就有
\[
B_{2n+1}=\frac{(2n+1)!}{2\pi\mathrm i}\oint_{\gamma}\frac{f(z)}{z^{2 n+2}}\,\dd z
\]
应用Euler公式作代换 $z=\E^{\mathrm i\theta}$,则有
$$
\begin{aligned}
B_{2n+1}&=\frac{(2n+1)!}{2\pi\mathrm i}\int_{0}^{2\pi}\frac{\E^{\mathrm i\theta}}{\E^{\E^{\mathrm i\theta}}-1}\E^{-2n\mathrm i\theta-2\mathrm i\theta}\mathrm i\E^{i \theta}\,\dd\theta
=\frac{(2n+1)!}{2\pi}\int_{0}^{2\pi}\frac{\E^{-2n\mathrm i\theta}}{\E^{\mathrm i\theta}-1}\,\dd\theta\\
&=\frac{(2n+1)!}{2\pi}\int_{0}^{\pi}\frac{\E^{-2n\mathrm i\theta}}{\E^{\E^{\mathrm i\theta}}-1}\,\dd \theta+\frac{(2n+1)!}{2\pi}\int_{0}^{\pi} \frac{\E^{-2n\mathrm i\theta}}{\E^{\E^{-\mathrm i\theta}}-1}\,\dd \theta\\
&=\frac{(2 n+1)!}{2\pi}\int_{0}^{\pi}-\E^{-2n\mathrm i\theta}\,\dd\theta
\end{aligned}
$$
可以看出对于任意的$n\in\mathbb N^+$,有$B_{2n+1}=0$.
\begin{definition}{}
设Bernoulli多项式为 $B_{n}:X\to Y$,利用生成函数可定义为
\begin{equation}{}
    \frac{ze^{xz}}{e^{z}-1}=\sum_{n=0}^{\infty}B_{n}(x)\frac{z^{n}}{n!}
    \quad(|z|<2\pi,z\in{\mathbb C})
\end{equation}	
\end{definition}
利用Cauchy乘积,并比较系数可得
\begin{equation}\label{EMSum_eq5}
    B_{n}(x)=\sum_{k=0}^{n}\dbinom{n}{k}B_{k}x^{n-k}
\end{equation}
对\autoref{EMSum_eq5} 求导,故有
\begin{equation}
    \frac{\dd}{\dd x}(B_{n+1}(x))=\sum_{k=0}^{n}\dbinom{n+1}{k}(n-k+1)B_{k}x^{n-k}=(n+1)B_{n}(x)
\end{equation}
两边同时积分,因此
\begin{equation}
    B_{n+1}(x)-B_{n+1}=(n+1)\int_{0}^{x}B_{n}(t)\,\dd t
\end{equation}
不妨记$\bar{B}_{n}(x)=B_{n}(\{x\})$,注意到$\bar{B}_{n}(x)$是一个周期为1的函数,则
\begin{equation}\label{EMSum_eq6}
    \int_{0}^{x}\bar{B}_{n}(t)\mathrm{d}t
    =\lfloor x\rfloor\int_{0}^{1}B_{n}(t)\mathrm{d}t+\int_{0}^{\{x\}}B_{n}(t)\mathrm{d}t
    =\frac{\bar{B}_{n}(x)-B_{n+1}}{n+1}
\end{equation}
应用式,计算前5项Bernoulli多项式,列举如下:
\begin{table}[ht]
\centering
\caption{Bernoulli多项式}
\begin{tabular}{|c|c|c|c|c|c|}
\hline
$n$ & 0 & 1 & 2 & 3 & 4\\
\hline
$B_{n}(x)$ & $1$ & $x-\dfrac{1}{2}$ & $x^{2}-x+\dfrac{1}{6}$ & $x^{3}-\dfrac{3}{2}x^{2}+\dfrac{1}{2}x$ & $x^{4}-2x^{3}+x^{2}-\dfrac{1}{30}$\\
\hline
\end{tabular}
\end{table}
考虑在区间$[0,1]$上将$B_{2n}(x),n\in{\mathbb N^{+}}$展开为Fourier级数:
\[
    B_{2n}(x)=\frac{a_{0}}{2}+\sum_{m=1}^{\infty}[a_{m}\cos(2m\pi x)+b_{m}\sin(2m\pi x)]
\]
可求得其系数为
\begin{equation}
\begin{cases}
    a_{0}=0\\
    a_{m}=(-1)^{n+1}\dfrac{2(2n)!}{(2m\pi)^{2n}}\\
    b_{m}=0
\end{cases}
\end{equation}
于是就有
\begin{equation}
    B_{2n}(x)=2(-1)^{n+1}(2n)!\sum_{k=1}^{\infty}\frac{\cos(2m\pi x)}{(2m\pi)^{2n}},0\leqslant x\leqslant1
\end{equation}
令$x=1$,可推出
\begin{equation}
    \zeta(2n)=\frac{(-1)^{n+1}B_{2n}2^{2n-1}\pi^{2n}}{(2n)!}
\end{equation}
可以看到,当$n=2$时
\[
    \sum_{n=1}^{\infty}\frac{1}{n^2}=\frac{\pi^2}{6}
\]
这就是著名的Basel问题的精确解答.同时,借助Bernouolli数和 可以较容易地得到自然数幂求和公式.
下面将介绍本篇的求和最终公式,它能得到常见初等函数的求和极为精确的结果.

\subsection{Euler-Maclaurin求和公式}

Euler-Maclaurin求和公式是有关定积分的一种数值计算公式,它建立了函数的积分与其导数的联系.在数值积分理论与级数求和法中,Euler-Maclaurin求和公式是一个极有用的工具.

\begin{theorem}{}
(Euler-Maclaurin求和公式)设$f:X\to Y$在区间$[0,+\infty)$上至少$2K(K\in{\mathbb N})$阶连续可微,则对于任意给定的$n\in\mathbb N$有
\begin{equation}\label{EMSum_eq7}
\begin{aligned}
    \sum_{k=1}^{n}f(k)&=\int_{0}^{n}f(x)\mathrm{d}x+\frac{f(n)-f(0)}{2}\\
    &+\sum_{k=1}^{K}\frac{B_{2k}}{(2k)!}[f^{(2k-1)}(n)-f^{(2k-1)}(0)]\\
    &+R_{2K+1}(n)
\end{aligned}
\end{equation}
其中
\begin{equation}
    R_{2K+1}(n)=\frac{1}{(2K+1)!}\int_{0}^{n}\bar{B}_{2K+1}(x)f^{(2k+1)}\,\dd x
\end{equation}
\end{theorem}
由\autoref{EMSum_eq6} 可知
\begin{equation}
    \int_{0}^{n}\bar{B}_{k}(x)f^{(\alpha)}\,\dd x
    =\frac{B_{k+1}}{k+1}[f^{(\alpha)}(n)-f^{(\alpha)}(0)]
    -\frac{1}{k+1}\int_{0}^{n}\bar{B}_{k+1}(x)f^{(\alpha+1)}\,\dd x
\end{equation}
 代入\autoref{EMSum_eq1} 即可得证.
\begin{theorem}{}
设$f:X\to Y$在区间$[0,+\infty)$无穷阶可微且各阶导函数在其上一致有界,则对于任意给定的$n\in\mathbb N$有
\begin{equation}\label{EMSum_eq8}
    \sum_{k=1}^{n}f(k)=\int_{0}^{n}f(x)\mathrm{d}x+\frac{f(n)-f(0)}{2}
    +\sum_{k=1}^{\infty}\frac{B_{2k}}{(2k)!}[f^{(2k-1)}(n)-f^{(2k-1)}(0)]
\end{equation}
\end{theorem}
同样地,\autoref{EMSum_eq7} 和\autoref{EMSum_eq8} 可如同\autoref{EMSum_eq2} 进行类似地改写.
只须证明$\lim\limits_{K\to\infty}R_{2K+1}(n)=0$对于任意给定的$n\in{\mathbb N^{+}}$都成立.
可设$|f^{(\alpha)}|\leqslant M(\alpha\in{\mathbb N})$,结合 ,就有
\begin{equation}
        |R_{2K+1}(n)|<\frac{2M(n+2)}{(2\pi)^{2K+2}}
\end{equation}
从而
\[
    \lim\limits_{K\to\infty}R_{2K+1}(n)=0	
\]
注意当$K$较大时,相应的余项的绝对值也会非常大,因此在近似计算时不宜将 的项数取的过多.
\begin{lemma}{}
Wallis公式是关于圆周率的无穷乘积的公式,其内容如下
		\begin{equation}\label{EMSum_eq10}
			\lim_{n\to\infty}\frac{1}{2n+1}\left[\frac{(2n)!!}{(2n-1)!!}\right]^2=\frac{\pi}{2}
		\end{equation}
\end{lemma}
考虑如下积分
\begin{equation}
    I_{n}=\int_{0}^{\pi/2}\sin^{n}x\mathrm{d}x\quad(n\in{\mathbb{N}})
\end{equation}
计算可得
\[
    nI_{n}=(n-1)I_{n-2}\Rightarrow I_{n}=\frac{(n-1)!!}{n!!}
    \left(\frac{\pi}{2}\right)^{\frac{1+(-1)^n}{2}},n\geqslant2
\]
在$0<x<\pi/2$时,$0<\sin^{2n+1}x<\sin^{2n}x<\sin^{2n-1}x(n\in{\mathbb{N}^{+}})$,积分就有
\[
    I_{2n+1}<I_{2n}<I_{2n-1}
\]
即
\[
    \alpha_{n}=\frac{1}{2n+1}\left[\frac{(2n)!!}{(2n-1)!!}\right]^2<\frac{\pi}{2}<\frac{1}{2n}\left[\frac{(2n)!!}{(2n-1)!!}\right]^2=\beta_{n}
\]
从而
\[
    0<\frac{\pi}{2}-\alpha_{n}<\beta_{n}-\alpha_{n}=\frac{\alpha_{n}}{2n}<\frac{\pi}{4n}
\]
由迫敛性即可得证.
\begin{example}{Stirling公式}
\begin{equation}
    \log n!=\left(n+\frac{1}{2}\right)\log n-n+\frac{1}{2}\log(2\pi)
    +O\left(\frac{1}{n}\right)\quad n\to\infty
\end{equation}
现在给出的Stirling公式的一种推导方法.应用\autoref{EMSum_eq3} 就有
\begin{equation}\label{EMSum_eq9}
    \log n!=\sum_{k=1}^{n}\log k=\left(n+\frac{1}{2}\right)\log n-n+C_{1}+R_{n}
\end{equation}
其中
\[
    C_{1}=1+\int_{1}^{\infty}\frac{\psi(x)}{x}\,\dd x\quad
    R_{n}=-\int_{n}^{\infty}\frac{\psi(x)}{x}\,\dd x
\]
将\autoref{EMSum_eq9} 代入\autoref{EMSum_eq10} 可得
\[
    \frac{\pi}{2}=\lim_{n\to\infty}\frac{ne^{4R_n-2R_{2n}}}{2(2n+1)}e^{2C_{1}}
    =\frac{e^{2C_{1}}}{4}\Rightarrow C_{1}=\frac{1}{2}\log(2\pi)
\]
这就得到了Stirling公式.
\end{example}
