% 原子(科普)

\begin{issues}
\issueDraft
\end{issues}

\begin{itemize}
\item 日常中的绝大部分物质(固体,液体,气体等)都是由原子组成的。
\item 原子的大小一般在几个 “埃”, 一埃是 $1\e{-10}\Si{m}$ \footnote{$1\e{-10}$ 是科学计数法, 也就是 $0.0\dots01$ 中共有 10 个零, 或者说把 $1.0$ 的小数点向左移动 10 位。}。 % 未完成:链接到科学计数法。 % 未完成: 数量级是什么
\item 原子由一个原子核以及一个或多个电子组成, 电子围绕原子旋转\footnote{这只是一个模糊的说法, 在原子的尺度下, 电子的运动由量子力学描述, 不存在运动轨迹的概念。}。
\item 原子核和电子之间通过库仑力(正负电荷之间的相互吸引力)束缚在一起。 原子核带正电荷, 电子带负电荷。
\item 原子核的尺寸一般在 $1\e{-15}\Si{m}$ 的数量级, 电子还要更小, 所以原子内部是很空旷的。
\end{itemize}
