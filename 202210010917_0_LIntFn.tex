% Lebesgue 可积的函数
% keys Lebesgue 积分|实变函数

\pentry{Lebesgue积分的一些补充性质\upref{Lebes2}}

\autoref{Lebes1_def1}~\upref{Lebes1}中已经给出了任意可测集上任意可测函数的Lebesgue积分之定义.为方便,我们将此定义再次誊抄如下:

\begin{definition}{Lebesgue积分}
设 $f$ 是可测集 $E\subseteq\mathbb{R}^n$ 上的可测函数,$f^+$ 和 $f^-$ 分别是其正部与负部.如果 $\int_E f^+(x) \dd x$ 和 $\int_E f^-(x) \dd x$ 中\textbf{至少有一个是有限的},则可定义其Lebesgue积分为:
\begin{equation}
\int_E f(x) \dd x = \int_E f^+(x) \dd x - \int_E f^-(x) \dd x
\end{equation}
\end{definition}

由定义直接可得,如果 $f$ 在 $E$ 上可积,那么必有 $\int_E -f(x) \dd x = -\int_E f(x) \dd x$.另外,也易得,$f$ 可测当且仅当 $\abs{f}$ 可测;如果 $E$ 测度有限且 $f$ 有界,那么 $f$ 的正部和负部的积分都是有限的,故 $f$ 可积.

% 考虑到 $\abs{\int_E f(x) \dd x}=\abs{\int_E f^+(x) \dd x - \int_E f^-(x) \dd x}$ 和 $\int_E \abs{f(x)} \dd x = \int_E f^+(x) \dd x + \int_E f^-(x) \dd x$,又可得\textbf{柯西不等式}:

% \begin{equation}
% \abs{\int_E f(x) \dd x} \leq \int_E \abs{f(x)} \dd x
% \end{equation}
%注释说明:搬到Lebesgue积分的一些补充性质\upref{Lebes2}词条去了.

Riemann积分中对定义域进行的分划,也可以当作Lebesgue积分中的可测分划,于是可以结合Riemann可积的定义(Riemann上和和Riemann下和之差可以任意小),得知Riemann可积的函数也Lebesgue可积.再注意到,在\textbf{Lebesgue 积分}\upref{Lebes1}中的\autoref{Lebes1_the3}~\upref{Lebes1},我们证明了可积函数的Lebesgue积分实际上就是其下方图形的测度,而测度又是面积的推广;因此,如果区间 $[a, b]$ 上的函数 $f$ 是Riemann可积的,那么它也是Lebesgue可积的,且两个积分值相等.

因此,Lebesgue积分是Riemann积分的推广.那么Lebesgue积分是和Riemann积分等价呢,还是能处理比Riemann可积函数范围更广的函数呢?答案是后者,我们以一个例子来说明:

\begin{example}{Dirichlet函数}

Dirichlet函数
\begin{equation}
D(x)=
\leftgroup{
    1, x\in\mathbb{Q}\\
    0, x\not\in\mathbb{Q}
}
\end{equation}
是Lebesgue可积的,其积分是 $0$(因为有理数集是可数的,从而是零测的),但它不是Riemann可积的.

\end{example}
















