% 流体、固体和气体
% keys 流体力学|剪应力|正应力

我们都知道物质的三态有液体、固体和气体\footnote{当然有一些物质无法归为这三类形态,比如果冻、凝胶等等,这些物质也会有复杂的性质},这三者在我们生活的世界里无处不在.我们脚底下的土地,我们呼吸的空气,我们喝的水、吹的风、踏过的河流…… 如何在物理定义上给出这三种物质形态的一个划分呢?

\subsection{应力的角度}
在流体力学中我们一般从应力的角度来划分这三种物质形态.\textbf{流体}是这样的一类物质,它能够在\textbf{剪应力}的作用下连续地发生形变.

\begin{figure}[ht]
\centering
\includegraphics[width=10cm]{./figures/SLG_1.png}
\caption{几种不同的施加应力的方式}} \label{SLG_fig1}
\end{figure}

图中的“剪切”和“扭曲”过程都有剪应力的参与.剪应力是指施加在流体的某个单位面积元上的\textbf{平行于面积元方向的力},是应力张量的非对角元部分.它体现了剪应力与正应力的明显的区分.
\addTODO{应力张量词条}

\textbf{固体}是指在外力的影响下只会发生微小的形变(几乎没有),但撤去外力以后就会恢复原形.这意味着固体的形状不会轻易地发生变化(在一些特殊例子中,木头会发生断裂,金属会被拉伸而延展).虽然固体不易形变,但总会有微小的形变,只是我们的肉眼很难观察的到.正因为它的形变及恢复力,固体才会发生振动,及声波在固体内部传播.于是我们才会听到鼓声敲响,才会感受到地震波的来临.