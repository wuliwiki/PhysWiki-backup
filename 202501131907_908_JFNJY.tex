% 约翰·福布斯·纳什(综述)
% license CCBYSA3
% type Wiki

本文根据 CC-BY-SA 协议转载翻译自维基百科\href{https://en.wikipedia.org/wiki/John_Forbes_Nash_Jr.}{相关文章}。

\begin{figure}[ht]
\centering
\includegraphics[width=6cm]{./figures/1ee7155e152489e2.png}
\caption{纳什在2000年代} \label{fig_JFNJY_1}
\end{figure}
约翰·福布斯·纳什 Jr.(1928年6月13日—2015年5月23日),以约翰·纳什为人所知,是一位美国数学家,对博弈论、实代数几何、微分几何和偏微分方程作出了重要贡献。[1][2] 纳什与博弈论学者约翰·哈萨尼和赖因哈德·塞尔滕共同获得了1994年诺贝尔经济学奖。2015年,他与路易斯·尼伦贝格因在偏微分方程领域的贡献而获得阿贝尔奖。

作为普林斯顿大学数学系的研究生,纳什引入了许多概念(包括纳什均衡和纳什议价解),这些概念如今被认为是博弈论及其在各学科应用中的核心内容。在1950年代,纳什通过求解一组出现在黎曼几何中的非线性偏微分方程,发现并证明了纳什嵌入定理。这项工作也引入了纳什-莫泽定理的初步形式,后来被美国数学学会授予勒罗伊·P·斯蒂尔奖,以表彰其对研究的开创性贡献。恩里奥·德·乔尔吉与纳什通过不同的方法发现了一些成果,为系统地理解椭圆和抛物型偏微分方程铺平了道路。他们的德·乔尔吉-纳什定理解决了希尔伯特第十九问题,即变分学中的正则性问题,这个问题已经成为一个著名的悬而未解的难题近六十年。

1959年,纳什开始显现出精神疾病的明显迹象,并在精神病医院接受了多年的精神分裂症治疗。1970年后,他的病情逐渐好转,使他能够在1980年代中期重新投入学术工作。[3]

纳什的生平成为了西尔维娅·纳萨尔于1998年出版的传记《美丽心灵》的主题,书中讲述了他与病魔的斗争以及他如何恢复健康,这一过程也成为了同名电影的基础,电影由朗·霍华德执导,拉塞尔·克劳饰演纳什。[4][5][6]
\subsection{早年生活与教育}
约翰·福布斯·纳什 Jr.于1928年6月13日出生在西弗吉尼亚州的蓝田(Bluefield)。他的父亲和同名的祖父约翰·福布斯·纳什Sr.是阿巴拉契亚电力公司的电气工程师。母亲玛格丽特·维吉尼亚(原姓马丁)·纳什在结婚前曾是一名学校教师。他在圣公会教堂接受了洗礼。[7] 他有一个妹妹,玛莎(生于1930年11月16日)。[8]

纳什上过幼儿园和公立学校,他的父母和祖父母为他提供了大量书籍供他阅读和学习。[8] 纳什的父母尽力为他提供更多教育机会,在他高中最后一年时,为他安排了在附近的蓝田学院(现为蓝田大学)参加高级数学课程的机会。他通过乔治·韦斯汀豪斯奖学金(George Westinghouse Scholarship)全额资助,进入卡内基技术学院(后来成为卡内基梅隆大学),最初主修化学工程。后来他转向化学专业,并最终在他的导师约翰·莱顿·辛格(John Lighton Synge)的建议下,改学数学。1948年,他以数学学士和硕士学位毕业后,接受了普林斯顿大学的奖学金,继续攻读数学和科学的研究生学位。[8]

纳什的导师、前卡内基教授理查德·达芬(Richard Duffin)为纳什申请普林斯顿大学时写了一封推荐信,信中称他为“数学天才”。[9][10] 纳什也收到了哈佛大学的录取通知。然而,普林斯顿数学系的主任所罗门·莱夫谢茨(Solomon Lefschetz)为他提供了约翰·S·肯尼迪奖学金,成功说服纳什普林斯顿更看重他。[11] 此外,他也更倾向于选择普林斯顿,因为该校距离他在蓝田的家较近。[8] 在普林斯顿,他开始研究均衡理论,后来这个理论被称为纳什均衡。[12]