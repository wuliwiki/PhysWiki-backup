% 首都师范大学 2010 年硕士考试试题
% keys 首都师范大学|考研|2010年|物理
% license Copy
% type Tutor
\begin{enumerate}
\item 半圆型光滑凹槽竖直固定在水平地面上,一质量为m的质点从凹槽的任一端最高处由静止开始无摩擦地滑下。求:在质点运动过程中,凹槽给与质点的支撑力的变化情况。
\item 将一轻质弹簧放在两个木块之间,并与它们固定连接,构成一个系统。木块的质量分别为 m和 M,放在光滑水平面上。初始,系统静止,弹簧处于自然状态。后给予质量为m的木块一冲量,使其获得沿弹簧压缩方向的水平速率$v_0$。在系统的质心系中试求:\\
(1)两木块各做何种运动?\\
(2)当弹簧处于原长状态时,系统的总能量在两个木块之间是如何分配的?
\item 一根长度为L、质量为M的均匀细直杆,其一端悬挂在一光滑、水平轴上,杆可以绕轴在竖直面内做自由摆动。另一完全同样的杆与其并排悬挂(两水平轴平行)。初始两杆静止,且刚刚可以接触。后,将其中的一杆(令其为杆1)拉起一定角度$\theta_0$后放手,使两杆发生完全弹性碰撞,求:\\
(1)碰撞后,被碰撞杆(即杆2)摆起的角度$\theta$;\\
(2)两杆发生两次碰撞的时间间隔。
\item 已知地球表面的重力加速度为$9.8ms^{-2}$,围绕地球的大圆周长为$4*10^7$m,月球与地球的直径及质量之比分别是$\frac{D_m}{D_e}$=0.27和$\frac{M_m}{M_e}$=0.0123。试计算从月球表面逃离月球引力场所必需的最小速度。
\item 半径为R的圆平面上,分布着电荷面密度为σ的均匀正电荷,过圆平面的中心,做垂直于平面的ox轴,P点在轴线上,与原点相距为x处,如图。求:\\
(1)该带电圆平面在P点处产生的电场强度$\vec E_p$;\\
(2)若P点沿x轴远移,当移到足够远时,该带电圆平面可以被看做点电荷模型,证明E,在x→>∞时可表示成点电荷的电场表达式。
\end{enumerate}
