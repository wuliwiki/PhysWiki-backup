% 主应力

%书本不在手边,这篇文章主要是按个人理解和笔记写的,可能有所疏漏,需要dalao指正.
\begin{issues}
\issueDraft
\end{issues}
\pentry{应力\upref{STRESS}}

\subsection{主应力}
假设我们有一个受力状况相对复杂的二维微元体.我们能不能改变划分微元体的方式,从而简化他的受力?
\begin{figure}[ht]
\centering
\includegraphics[width=5cm]{./figures/PRSTR_1.png}
\caption{二维微元体.仿自P. Beer的Mechanics of Materials} \label{PRSTR_fig1}
\end{figure}

答案难得的是...可以的.在一点处,我们总能够找到一种选取微元体的方式,使其只受正应力而不受切应力.这个结论同时适用于二维与三维的微元体.
\begin{figure}[ht]
\centering
\includegraphics[width=6cm]{./figures/PRSTR_2.png}
\caption{通过改变选取微元体的方式,使其只受正应力而不受切应力} \label{PRSTR_fig2}
\end{figure}
在这种情况下,这些正应力也被称为主应力(Principal Stress). 可见,应力的大小与微元体的选取方式有关,而主应力的大小则与之无关.因此,某种意义上,主应力比单纯的应力更具有代表性.

同时,主应力减少了变量个数:在二维情况下,由应力的3个分量减少为了主应力的2个;而三维情况下,由6个分量减少为了3个.

\subsection{计算主应力}
\addTODO{需要补充证明}

那么,你现在想问的问题\textsl{大概一定}是:如果已知了微元体的受应力情况,如何计算出他的主应力呢?事实上,主应力就是应力矩阵\upref{STRESS}的本征值\upref{EigVM}.计算一个应力矩阵本征值,就能得到他的主应力.因此,主应力的求解是公式化的、不需要太多\textsl{灵性}.

\begin{example}{二维微元体的主应力}
对于二维情况,主应力就是方程
$$\sigma_p^2+(\sigma_x+\sigma_y)\sigma_p+\sigma_x\sigma_y-\tau^2=0$$
的两个根,即
$$\sigma_p=\frac{\sigma_x+\sigma_y \pm \sqrt{(\sigma_x-\sigma_y)^2+4\tau^2}}{2}$$
$\sigma_x,\sigma_y,\tau$是相应微元体的应力.
\end{example}

\begin{example}{三维微元体的主应力}
对于三维情况,主应力是方程 
$$
\sigma_p^3-I_1\sigma_p^2-I_2\sigma_p-I_3=0
$$
的三个根,其中
$$
\begin{aligned}
I_1&=tr(\sigma)\\
I_2&=\frac{1}{2} ({\sum \sigma_{ij} \sigma_{ij} -I_1^2})\\
I_3&=det(\sigma)\\
\end{aligned}
$$
$\sigma$是相应的应力矩阵
\end{example}
