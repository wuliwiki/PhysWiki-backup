% 电流密度
% 电流|流密度|电流密度|电磁学

\pentry{电流\upref{I}, 流密度\upref{CrnDen}}

电流某时刻在空间中的分布情况可以用\textbf{电流密度}(矢量) $\bvec j(\bvec r)$ 来描述, 其方向与 $\bvec r$ 处的电荷运动方向相同, 详见 “流密度\upref{CrnDen}”.

\begin{equation}\label{Idens_eq1}
I = \int_S \dd I= \int_S \bvec j \vdot \dd{\bvec S}
\end{equation}
我们可以这样理解上式:若作一个垂直于电流方向的横截面 $\dd S$,且穿过这一横截面的电流为 $\dd I$(这意味着单位时间 $\Delta t$ 内有 $\dd I\cdot \Delta t$ 的电荷经过这个横截面),那么该横截面的电流密度 $\bvec j$ 可以用 $\dd{\bvec I}/\dd S$ 来估计,它的方向与电流是一致的.它衡量了单位时间内单位横截面通过的电流量.现在考虑,如果作一个横截面\textbf{不垂直于}电流方向,或者说将原来的那个横截面倾斜一个角度 $\theta$;假设通过它的电流仍然是 $I$,那么可以预料到该横截面的大小变为原来的 $1/\cos\theta$ 倍;并且法向矢量 $\dd{\bvec S}$ 与电流 $\bvec j$ 不再平行,而是呈一个 $\theta$ 的夹角,它们的点乘就会贡献一个 $\cos\theta$,这与前面的 $1/\cos\theta$ 相抵消.这意味着 $I=\int \bvec j\cdot \dd{\bvec S}$ 是良定义的.

另外还要注意的一点时,\autoref{Idens_eq1} 中对曲面 $S$ 上电流的面积分是有方向性的.在面积微元 $\dd S$ 处,当法线方向 $\dd{\bvec S}$ 与 $\bvec j$ 的夹角小于 $90^\circ$ 时,该区域对电流 $I$ 的贡献大于 $0$,否则小于 $0$.在这里我们所考察的电流 $I$ 可正可负,代表了从曲面 $S$ 的\textbf{内侧}到\textbf{外侧}所通过的电流(单位时间的电荷).\textbf{外侧}的意思是法线所指代的方向,即对应着曲面积分的定向.

上面我们从一个经典的宏观世界的角度考察了电流密度的定义.下面让我们回到电流的微观定义\upref{I}.假设介质中 $n$ 为载流子的数密度,$\bvec v$ 为介质中某一点载流子的平均运动速度.在垂直于 $\bvec v$ 方向画一个横截面 $\dd S$,容易写出电流 $I$ 的关系式:$I=ne\bvec v \cdot \dd{\bvec S}$.再根据电流密度的定义,我们有:
\begin{equation}
\bvec j = \rho \bvec v = ne \bvec v
\end{equation}
其中 $\rho=ne$ 是载流子的体电荷密度,$n$是载流子的数密度,$e$是单个载流子的电荷量.要注意 $\bvec v$ 是载流子的平均运动速度,因为再微观层面上各个载流子的运动方向其实是不确定的,它们在外场的作用下具有了沿一个方向上的平均运动速度的分量,才产生了电流.也就是说,电流密度、电流这些概念在宏观层面和充分多载流子的统计意义上才能够成立\cite{GriffE}.
