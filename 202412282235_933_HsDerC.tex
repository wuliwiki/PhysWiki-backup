% 导数的性质与构造(高中)
% keys 导数|性质|构造|恒等
% license Usr
% type Tutor

\begin{issues}
\issueDraft
\end{issues}

\pentry{导数\nref{nod_HsDerv},函数的性质\nref{nod_HsFunC}}{nod_139b}

导数是分析函数变化规律的关键工具,它揭示了函数随变量变化时的趋势和行为。可以把导函数看作原函数的一张“素描画”,虽然简化了某些信息,但保留了所有的局部信息,从而它可以描述函数的几乎所有性质。例如,导数可以用来判断函数在某一区间内是上升还是下降,还可以分析图像的弯曲方向是“开口向上”还是“开口向下”,甚至帮助找到最高点和最低点的位置。这种功能类似于在地图上标注道路的坡度,通过这些标记可以快速了解路段的起伏情况。同样,导函数为函数的变化贴上了清晰的“标签”,使得人们能够一目了然地把握它在不同区间的行为。

在解题过程中,导数相关的典型问题主要包括以下三类:

\begin{itemize}
\item \textbf{最值问题}:通过确定导数为零的位置,分析函数的变化趋势,找到函数的最高点或最低点。
\item \textbf{零点问题}:利用导数得到单调区间,结合区间两端函数值的符号情况,利用\aref{零点存在定理}{the_HsFunC_1},判断函数在某区间内的零点情况。
\item \textbf{恒成立问题}:通过导数符号的变化,分析函数在某区间内的单调性和极值的符号,判断函数是否满足某些恒成立条件。
\end{itemize}

解题的核心是找出导数为零的点,并分析导数在不同区间内的正负变化,从而勾勒出函数的变化趋势和形态。在此过程中,构造适当的函数并进行求导,以及合理选取区间和相异的函数值,是解决这类问题的两个主要难点。本文会对此提供一些常见的方法。

\subsection{近似代替}

在导数的\aref{几何含义}{sub_HsDerv_1}中,曾经提到了“以直代曲”的思想。简单来说,就是用直线来近似描述曲线的变化。在某一点 $x_0$  处的\aref{切线方程}{eq_HsDerv_1}可以实现这样的替代:如果选取的另一个点  $x_0 + \Delta x$  离  $x_0$  足够近,也就是$\Delta x$比较小,那么函数在这两个点上的值差距很小,近似可以用切线上的值代替函数的值,即:

\begin{equation}\label{eq_HsDerC_1}
y-f(x_0)=f'(x_0)(x-x_0)\overset{x=x_0+\Delta x}{\implies} y=f(x_0)+f'(x_0)\Delta x\approx f(x_0+\Delta x) ~.
\end{equation}

虽然这样的近似会带来一定的误差,但在实际问题中,这种误差通常可以接受。因为切线是一个线性函数(形状简单、参数少、容易计算),将一个难以研究的非线性函数转换成容易处理的线性函数,可以显著降低计算难度。换句话说,“以直代曲”是一种非常高效的思维方式和解题手段。而由于线性近似的性质,这种替代被认为是没有系统性偏差的,也就是说,它在大多数情况下都能很好地反映函数在点$x_0$附近的变化。类似的研究,在大学阶段会更加深入。

\begin{example}{估计$\sqrt{26}$}

答:

已知 $\sqrt{25} = 5$,可以将 $26$ 视为接近 $25$ 的值,设基准点 $x_0 = 25$,并利用函数 $f(x) = \sqrt{x}$来计算,根据\aref{求导公式}{tab_HsDerB2}有$\displaystyle f'(x) = \frac{1}{2\sqrt{x}}$。在 $x_0 = 25$ 处有$\displaystyle f'(25) = \frac{1}{2\sqrt{25}} = \frac{1}{10}$。

取 $\Delta x = 26 - 25 = 1$,代入\aref{线性近似公式}{eq_HsDerC_1},有:

\begin{equation}
\sqrt{26} \approx \sqrt{25} + \frac{1}{10} \cdot 1 = 5 + 0.1 = 5.1~.
\end{equation}

实际用计算器计算 $\sqrt{26} \approx 5.099$(保留三位小数)。近似值 $5.1$ 与真实值的误差仅为 $0.001$,可以满足多数情况下的计算需求。
\end{example}

\begin{figure}[ht]
\centering
\includegraphics[width=12cm]{./figures/a3b16a176d00425e.png}
\caption{100以内的估计值与标准值对比} \label{fig_HsDerC_1}
\end{figure}

事实上,上面例子中的计算方法就是一种手动开方的估计方法,它的精度很高,而且误差会随着被开方数的增加而降低,是“以直代曲”的典型应用。

\begin{figure}[ht]
\centering
\includegraphics[width=12cm]{./figures/8ff4847b1960b05d.png}
\caption{估计值与标准值的误差} \label{fig_HsDerC_2}
\end{figure}

\subsection{单调性}\label{sub_HsDerC_1}

在介绍\aref{导函数}{sub_HsDerv_2}时,提及区间内函数的单调性与导函数的符号密切相关。这一关系可以类比为“指南针”:导函数的符号如同指南针的指针,指示着函数在某个区间内的增减方向。事实上,这种关系可以通过导数的定义直接推导出来。

以函数的增区间为例,分析如下:

根据导数的定义,$x_1$处的导数为:
\begin{equation}
f'(x_1) = \lim_{x_2 \to x_1} \frac{f(x_2) - f(x_1)}{x_2 - x_1}~.
\end{equation}


假设 $x_2 > x_1$,则分母 $x_2 - x_1 > 0$。在这种情况下,$f'(x_1)$ 的符号与分子 $f(x_2) - f(x_1)$ 的符号一致。也就是说,当 $f'(x_1) > 0$ 时,有 $f(x_2) > f(x_1)$,表明函数在 $x_1$ 处向右增长。

虽然上述讨论基于两个非常接近的自变量值,但由于不等式的传递性,可以推广到整个区间。如果某个区间上的导函数 $f'(x)$ 始终为正,则对于该区间内任意 $x_1, x_2$(且 $x_1 < x_2$),有 $f(x_1) \leq f(x_2)$,这恰好符合函数在该区间内单调递增的定义。

由此,可以总结函数单调性与导函数符号的关系如下:

\begin{theorem}{单调性与导数的关系}
对于函数 $f(x)$ 及其导函数 $f'(x)$,有以下结论:
\begin{itemize}
\item 若 $f'(x) > 0$ 在某区间内成立,则 $f(x)$ 在该区间上单调递增。
\item 若 $f'(x) < 0$ 在某区间内成立,则 $f(x)$ 在该区间上单调递减。
\item 若 $f'(x) = 0$ 在某区间内成立,则 $f(x)$ 在该区间上为常值,其图像为水平直线。
\end{itemize}
\end{theorem}

这一结论为研究函数图像的单调性提供了明确的判断依据,同时也表明导函数在解析函数变化趋势中的核心作用。

在介绍\aref{导函数}{sub_HsDerv_2}时,提及区间的中函数的增减与导函数的符号相关。看上去在分析函数的单调性时,可以将导数的符号看作一种“指南针”,指示函数的增长或下降方向。这个关系其实是可以从导数的定义上推理出来的,以函数增区间为例:

根据导数的定义,$x_1$处的导数为$\displaystyle f'(x_1)=\lim_{\Delta x_2\to x_1}{f(x_2)-f(x_1)\over x_2-x_1}$。假设$x_2>x_1$,则$f'(x_1)$的符号与$f(x_2)-f(x_1)$的符号相同,即若$f'(x_1)>0$则$f(x_2)>f(x_1)$。尽管刚才讨论的是两个相隔很近的自变量值,却因为不等号具有传递性,使得对于某个区间如果导数值一直为正,那么,对于区间上的任意 $x_1, x_2$ ,若 $x_1 < x_2$,总能推出有 $f(x_1) \leq f(x_2)$。而这正满足了函数增区间的定义。

\begin{theorem}{单调性与导数的关系}
对于$f(x)$和它的导函数$f'(x)$:
\begin{itemize}
\item 在$f'(x)>0$的区间上,原函数单调递增。
\item 在$f'(x)<0$的区间上,原函数单调递减。
\item 在$f'(x)=0$的区间上,原函数为不变的常值,图像是水平的。
\end{itemize}
\end{theorem}

利用导数判断单调性的步骤如下:
\begin{enumerate}
\item 确定函数定义域
\item 计算导数:通过求导得到 $f'(x)$ 的表达式,并分析其符号。
\item 找出导数变化的关键点:解 $f'(x) = 0$找到导数为零的点,分析导数不存在的点。
\item 根据导数符号划分单调性区间:根据导数符号变化,将定义域划分为若干区间,并分析每个区间的单调性。
\end{enumerate}

\subsection{驻点与极值点}

前面提到过,在讨论导数相关的问题时,最值问题是一个核心内容。如果函数是一座山峰的地图,那么整个地图中最高的山峰或最低的洼地则称为\textbf{最值},而一个局部的“高峰”或“低谷”被称作\textbf{极值}。整体对应的\aref{最值}{def_HsFunC_4}在之前就已经作了详细介绍。本章将重点讨论极值。

虽然高中教科书主要围绕极值和最值展开讨论,但还有一个重要的概念被忽略了,那就是\textbf{驻点}。驻点与极值点有着密切的联系。然而,驻点却并不一定是极值点,极值点也并不一定是驻点。在高中阶段,由于教学内容的简化,驻点与极值点的区分往往不够明确,常用极值点的概念代替驻点。然而,随着对函数性质理解的加深,认识到两者的区别显得尤为重要。为了便于理解,接下来的内容将从驻点这一相对简单的概念入手展开讨论。

\begin{definition}{驻点}
对于函数$y=f(x)$,如果某点$x_0$满足$f'(x_0)=0$,即$x_0$是$f'(x)$的零点,则称$x_0$为$f(x)$的\textbf{驻点(stationary point)}。
\end{definition}

驻点表示的是函数值暂时停止变化的点,或者说它是函数的水平切点。

极值点是针对某个局部范围而言的,而这个局部在数学上被称为\textbf{邻域(neighborhood)}。具体来说,对于一个点 $x_0$,邻域可以表示为 $\left( x_0 - \delta, x_0 + \delta \right)$,其中 $\delta > 0$\footnote{也常记作${U}(x_0, \delta)$。}。这意味着在点 $x_0$ 的左右各延伸 $\delta$ 的范围内的所有点都属于 $x_0$ 的邻域。这就像一个人在自己家附近的范围内活动,这个范围可以由一定的距离(类似于 $\delta$)决定。邻域概念可以保证在研究函数时聚焦于某一点的周围情况,而不必考虑整个区域的性质。而\textbf{去心邻域(deleted neighbourhood)}指的是在$x_0$的邻域$U$中去掉$x_0$的集合,也就是$\left( x_0 - \delta,x_0)\cup(x_0, x_0 + \delta \right)$,记作$\mathring{U}(x_0,\delta)$。

\begin{definition}{极值点}
于函数 $y=f(x)$及其定义域内一点 $x_0$,若存在$x_0$的某个去心邻域$\mathring{U}(x_0,\delta)$,使得$U$中的任意点$x$满足:
\begin{itemize}
\item $f(x) < f\left(x_0\right)$,则称 $x_0$ 是$f(x)$的\textbf{极大值点(maximum point)}。
\item $f(x) > f\left(x_0\right)$,则称 $x_0$ 是$f(x)$的\textbf{极小值点(minimum point)}。
\end{itemize}
极大值点和极小值点合称为\textbf{极值点(local extremum point)}。
\end{definition}

极值点的定义与驻点密切相关,但两者并不等价。根据定义,非所有驻点都是极值点。例如,对于函数 $y = x^3$,在 $x = 0$ 处满足 $f'(x) = 0$,但在该点两侧,函数值既有增大也有减小,因此 $x = 0$ 并不是极值点。

值得注意的是,极值点的判定并不依赖于函数在这一点的导数是否存在。只要在某个邻域内满足极值的定义,该点就可以被视为极值点。例如,对于函数 $y = |x|$,在 $x = 0$ 处,函数的一阶导数不存在,因此$x = 0$并非驻点,但由于在任意的区间 $(-\delta, \delta)$ 上都有 $|0|$ 最小,因此 $x = 0$ 是一个极值点。

实际上,极值点的定义可以等价为某点两侧导数符号的变化,而这一点的可导性并非必要条件。如果导数从负变正,则该点是极小值点;如果从正变负,则该点是极大值点。由于驻点的导数为零,因此驻点经常与导数符号变化对应,从而容易引起概念上的混淆。类比来说,驻点就像一个检测站,只有确认车辆(函数值)的行驶方向在该点发生变化,才能确定它是山顶(极大值点)或山谷(极小值点)。

不过特别地,如果函数是可导函数那么它的极值点必定是它的驻点。在高中阶段的考察中,通常不涉及不可导点的极值问题,因此在实际求解最值问题时,可以简化为以下步骤:
\begin{enumerate}
\item 求驻点:通过解 $f'(x) = 0$ 找到函数的驻点。
\item 判定导数符号变化:通过观察每个驻点左右的导数符号,判定该点是否为极值点。
\item 确定极值类型:根据导数符号的变化方向,判定驻点是极大值点还是极小值点。
\item 比较得出最值:比较所有极值点和边界点的函数值,从中确定最大值和最小值。
\end{enumerate}


\subsection{高阶导数}

导函数作为原函数,则又可以求得它的导函数,这也被称为高阶导数。

凹凸性

连续曲线的凹弧与凸弧的交界点。(在该点的二阶导并不一定有定义)
f"(x)=0,且该点 两侧 二阶导数变号,那么该点就是极值点。
当然,可能在一点x0处,二阶导数并不存在,在x0左侧的二阶导数趋于正无穷,右侧的二阶导数趋于负无穷,该点也是拐点。

\subsection{常用构造}

\pentry{恒等式与不等式恒成立\nref{nod_HsIden}}{nod_b069}

\subsubsection{逆向使用求导法则}

逆用积法则:

$x^n f'(x) + f(x) \geq 0$,构造 $F(x) = x^n f(x)$,$[x^n f(x)] = x^n f'(x) + nx^{n-1} f(x) = x^{n-1} [x f'(x) + nf(x)]$。特别地,当$n=1$时有$x f'(x) + f(x) \geq 0$,构造 $F(x) = x f(x)$,$[x f(x)]' = x f'(x) + f(x)$

$f'(x) + k f(x) \geq 0$,构造 $F(x) = \E^{kx} f(x)$,$[e^{kx} f(x)]' = e^{kx} [f'(x) + kf(x)]$。特别地,当$k=1$时有$f'(x) + f(x) \geq 0$,构造 $F(x) = \E^x f(x)$,$[e^x f(x)]' = e^x [f'(x) + f(x)]$

逆用商法则:

$xf'(x) - f(x) \geq 0$,构造 $F(x) = \frac{f(x)}{x}$,  
    $\therefore \left[\frac{f(x)}{x}\right]' = \frac{f'(x) \cdot x - f(x)}{x^2}$

$f'(x) - f(x) \geq 0$,构造 $F(x) = \frac{f(x)}{e^x}$,  
    $\therefore \left[\frac{f(x)}{e^x}\right]' = \frac{e^x \cdot f'(x) - e^x \cdot f(x)}{e^{2x}} = \frac{f'(x) - f(x)}{e^x}$

$x^n f'(x) - n f(x) \geq 0$,构造 $F(x) = \frac{f(x)}{x^n}$,  
    $\therefore \left[\frac{f(x)}{x^n}\right]' = \frac{x^n \cdot f'(x) - n x^{n-1} \cdot f(x)}{x^{2n}} = \frac{f'(x) - n f(x)}{x^{n+1}}$

$f'(x) - k f(x) \geq 0$,构造 $F(x) = \frac{f(x)}{e^{kx}}$,  
    $\therefore \left[\frac{f(x)}{e^{kx}}\right]' = \frac{e^{kx} \cdot f'(x) - k e^{kx} \cdot f(x)}{e^{2kx}} = \frac{f'(x) - k f(x)}{e^{kx}}$
