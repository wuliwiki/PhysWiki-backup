% 卡方分布

\begin{issues}
\issueDraft
\end{issues}

\begin{figure}[ht]
\centering
\includegraphics[width=7cm]{./figures/Chi2_1.pdf}
\caption{卡方概率密度函数(\autoref{Chi2_eq1}), 来自 Wikipedia} \label{Chi2_fig1}
\end{figure}

\begin{equation}\label{Chi2_eq1}
f_k(x) = \frac{1}{2^{k/2}\Gamma(k/2)}x^{k/2-1}\E^{-x/2} \qquad (x > 0)
\end{equation}
期望值 $k$, 方差 $2k$.

累计分布函数为(Matlab 的 \verb|gammainc(x/2,k/2)|)
\begin{equation}
F_k(x) = \frac{\gamma(k/2, x/2)}{\Gamma(k/2)}
\end{equation}
其中 $\gamma$ 为下不完全 $\Gamma$ 函数\upref{IncGam}.


若把 要检验 $N$ 次试验中, 
\begin{equation}
\sum_{i=1}^k C_i \qty(\frac{f_i}{N} - p_i)^2
\end{equation}
令 $C_i = N/p_i$. 于是 Pearson's cumulative test statistic 为
\begin{equation}
\chi^2 = \sum_{i=1}^k \frac{N}{p_i} \qty(\frac{f_i}{N} - p_i)^2 = \sum_{i=1}^k \frac{f_i^2}{Np_i} - N
\end{equation}
当 $N$ 足够大, 上式近似服从 $\chi^2(k-1)$ 分布.

对于\textbf{显著水平(significance level)} $\alpha$, 当 $\chi^2 \geqslant \chi_{\alpha+}^2(k-1)$ 或者 $\chi^2 \leqslant \chi_{\alpha-}^2(k-1)$ 时就拒绝. 其中 $\chi_{\alpha\pm}^2(k-1)$ 满足
\begin{equation}
\int_{\chi_{\alpha+}^2(k-1)}^\infty f_{k-1}(x) \dd{x} = \alpha
\end{equation}
\begin{equation}
\int_0^{\chi_{\alpha-}^2(k-1)} f_{k-1}(x) \dd{x} = \alpha
\end{equation}

\addTODO{如何检测两个随机变量是否独立?}
