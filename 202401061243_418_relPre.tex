% 相对论效应造成的近日进动
% keys 相对论|进动
% license Usr
% type Wiki

\begin{issues}
\issueDraft
\end{issues}

\pentry{开普勒问题\upref{CelBd},狭义相对论,分析力学}



对于开普勒问题,还需要考虑由于太阳的强引力对轨道产生的影响。这里仅讨论狭义相对论,不考虑广义相对论造成的修正。

\subsection{观测值与其他星体造成的影响理论值不符}
对于水星近日点的总进动值(约每世纪 $570''$),其他行星对水星的影响约仅有观测值的 $93 \%$,特别是木星、金星与地球(占约 $91\%$),人们发现有微小偏差,广义相对论的修正约是每世纪进动 $43''$,与其他星体造成的影响合并后恰好符合观测值。

特别的,广义相对论的修正包含了狭义相对论的修正,狭义相对论的修正结果只有广义相对论的约 $1/6$。

\subsection{狭义相对论修正的近日点进动}
这个问题一般被称为\textbf{相对论性开普勒问题},在经典力学中可以严格求解。第一个解决该问题的是索墨菲(Sommerfeld)。

\subsubsection{分析力学求近似解}
为简化计算,采用速度满足光速 $c=1$ 的单位制。

在狭义相对论中,粒子的拉氏量忽略常数 $-mc^2 = -m$ 后为:

$$L = -mc^2 \sqrt{1- \frac{{\bvec v}^2}{c^2}} = \frac{m}{2} {\bvec v}^2 \left( 1+\frac{1}{4} {\bvec v}^2 + \cdots \right) + \frac{\alpha}{r} ~.$$

其中省略的内容是 ${\bvec v}^2$ 的高次项,都是更高阶的相对论修正。现仅考虑近似解,故将其忽略。

开普勒问题仍然满足在一个二维平面上,采用极坐标表示速度 $\bvec v$,则:

$${\bvec v}^2 = \dot r^2 + r^2 \dot \phi^2 ~.$$

令 $E$ 为扣除粒子静止能量 $mc^2 = m$ 之后的能量。那么相对论性开普勒问题的系统能量 $E$、角动量 $p_\phi \equiv J$ 仍然为守恒量。对于角动量满足:

$$p_\phi = \frac{\partial L}{\partial {\dot \phi}} \equiv \mathcal J \approx mr^2 \dot \phi \left( 1+ {\bvec v}^2/2 \right)~.$$

而对于 $r$,其共轭动量为:

$$p_r = \frac{\partial L}{\partial {\dot r}} \approx m \dot r \left( 1 + {\bvec v}^2/2 \right)~.$$

那么粒子的能量可以写作:

\begin{equation}
\begin{aligned}
E &= \dot r p_r + \dot \phi p_\phi - L \\
&\approx m {\bvec v}^2(1+{\bvec v}^2/2)-L \\
&= \frac{1}{2} m {\bvec v}^2 \left( 1 + \frac{3}{4} {\bvec v}^2 \right) - \frac{\alpha}{r} ~.
\end{aligned}
\end{equation}

我们仅关心轨道形状,而非与时间依赖关系,因此考虑将上式用 $\dd{r}/\dd{\phi}$ 表示。其中 $\phi$ 考虑用守恒量 $\mathcal J$ 表示,这样不显含时。
$$\dot \phi \approx \frac{\mathcal J}{mr^2} \left( 1-{\bvec v}^2/2 \right) ~.$$

实际上这个公式并没有完全求解,因为 $\bvec v$ 中还含有 $\dot \phi$ 项。但仍可以将其考虑为一个近似解。其中相对论修正的部分应当用原来问题的零级近似,也就是无相对论修正时的开普勒问题的解代入。

由于:

$$\frac{\dd r}{\dd t} = \frac{\dd r}{\dd \phi} \frac{\dd \phi} {\dd t}~.$$

也就是 $\dot r = \dot \phi \cdot \dd{r}/\dd{\phi}$。仍然采用开普勒问题常用的代换 $u = 1/r$ 可以得到:

\begin{equation}
\dot r \approx - \frac{\mathcal J}{m} \frac{\dd u}{\dd \phi} \left(1-{\bvec v}^2/2\right)
~.\end{equation}


将 $\dot \phi$ 和 $\dot r$ 的表达式代入速度的表达式便可以得到粒子动能的近似表达式:

\begin{equation}
\frac{1}{2} m \bvec{v}^2 \approx \frac{\mathcal J^2}{2m} \left[\left(\frac{\dd u}{\dd \phi}\right) + u^2\right] (1-\bvec{v}^2)~.
\end{equation}

与粒子能量的表达式合并,可以得到粒子总能量的表达式:

\begin{equation}
E \approx \frac{\mathcal J^2}{2 m} \left[ \left(\frac{\dd u}{\dd \phi}\right) + u^2 \right] - \frac{1}{8} m \left(\bvec{v}^2\right)^2 -\alpha u ~.
\end{equation}

可以发现,其中仅有第二项为修正项,仍考虑用零级近似的 $\bvec{v}^2$ 来替代,利用:

$$\frac{1}{2} m \bvec{v}^2 \approx E + \alpha / r = E + \alpha u ~,$$

来替换相对论修正得到的 $\bvec{v}^2$,就得到了狭义相对论修正的:

\begin{equation}
-\frac{m}{8} (\bvec v)^2 \approx - \frac{1}{2m} \left(E + \alpha u\right)^2 ~.
\end{equation}

zh

\subsubsection{哈密顿力学求精确解}
