% 路径积分(量子力学)
% keys 传播子|跃迁振幅|关联振幅|概率振幅|高等量子力学

\pentry{传播子(量子力学)\upref{PpgtQM},薛定谔绘景和海森堡绘景\upref{HsbPic}}

% 定义传播子$C_i(t) = \bra{\bvec{x}_{i+1}}\E^{-\I Ht}\ket{\bvec{x}_i}$.




\subsection{概念的引入}

为了方便,我们考虑二维时空的情况,即空间只有一维.

在初始时刻$t=0$时,一个粒子处于$x_0$位置,将它的态记为$\ket{x_0}$,其在位置空间的波函数为$\psi_0(x)=\braket{x}{x_0}=\delta(x-x_0)$.

时间过去$t_1$后,我们在$x_1$位置测量,发现粒子的概率振幅为$\bra{x_1}\E^{-\I H t_1}\ket{x_0}$.因此我们可以说,粒子在时间$t_1$后“出现”在\footnote{我们也可以说“传播”到.}$x_1$的概率密度是$\abs{\bra{x_1}\E^{-\I H t_1}\ket{x_0}}^2$.

同样地,时间过去$t_2>t_1$后,粒子在$x_2$位置的概率振幅为$\bra{x_2}\E^{-\I H t_2}\ket{x_0}$.

注意到$\int \ket{x_1}\bra{x_1}\dd x_1=1$,即恒等变换(其矩阵总是单位矩阵),因此我们可以把这个积分插入到任何位置,比如:
\begin{equation}\label{PIntQM_eq1}
\ali{
    \bra{x_2}\E^{-\I H t_2}\ket{x_0} &= \bra{x_2}\int \ket{x_1}\bra{x_1}\dd x_1 \E^{-\I H t_2}\ket{x_0}\\
    &= \int \braket{x_2}{x_1}\bra{x_1} \E^{-\I H t_2}\ket{x_0}\dd x_1\\
    &= \int \bra{x_2}\E^{-\I H( t_2-t_1)}\ket{x_1}\bra{x_1} \E^{-\I H t_1}\ket{x_0}\dd x_1\\
}
\end{equation}

\autoref{PIntQM_eq1} 数学上成立,但它有什么物理意义呢?

每个$\bra{x_2}\E^{-\I H( t_2-t_1)}\ket{x_1}\bra{x_1} \E^{-\I H t_1}\ket{x_0}$表达的是,粒子从$x_0$开始,$t_1$后出现在$x_1$的振幅,乘以从$x_1$开始,再过$t_2-t_1$后出现在$x_2$的振幅.而\autoref{PIntQM_eq1} 是对这个表达式关于$x_1$遍历全空间求积分.

综上,\autoref{PIntQM_eq1} 表达的是:求粒子从$x_0$出发、经过$t_2$后出现在$x_2$的概率振幅,等于先求出粒子经过$t_1$后传播到$x_1$后再从$x_1$经过$t_2-t_1$后传播到$x_2$的振幅,然后把所有可能的$x_1$遍历一遍,把得到的所有\textbf{路径}的振幅求积分.这个过程如\autoref{PIntQM_fig1} 所示:

\begin{figure}[ht]
\centering
\includegraphics[width=10cm]{./figures/PIntQM_1.pdf}
\caption{路径积分的示意图.如图,计算粒子从$x_0$经过时间$t_2$后传播到$x_2$的概率振幅,相当于图中各路径的振幅关于$x_1$遍历整个空间求积分.也就是说,$x_1$取遍所有可能性,得到类似图中三条路径的所有路径,所有这些路径的振幅积分,即为所求.} \label{PIntQM_fig1}
\end{figure}



同样地,我们可以把时间分成多段,产生更多的路径可能性,则我们所求的振幅$\bra{x_n}\mathcal{U}(t)\ket{x_1}$同样是所有这些可能路径的振幅之积分,如\autoref{PIntQM_fig2} 所示.

\begin{figure}[ht]
\centering
\includegraphics[width=10cm]{./figures/PIntQM_2.pdf}
\caption{将时间分成多段后,得到更多可能路径.} \label{PIntQM_fig2}
\end{figure}


积分表达式为:
\begin{equation}
\ali{
    &\bra{x_n}\E^{-\I H t_n}\ket{x_0}\\
    =& \int \bra{x_n}\E^{-\I H( t_n-t_1 )}\ket{x_1}\bra{x_1} \E^{-\I H t_1}\ket{x_0}\dd x_1\\
    =& \int\int \bra{x_n}\E^{\I H (t_3-t_2)}\ket{x_2}\bra{x_2}\E^{-\I H( t_2-t_1 )}\ket{x_1}\bra{x_1} \E^{-\I H t_1}\ket{x_0}\dd x_1\dd x_2\\
    =& \cdots\\
    =&\int\int\cdots\int \dd x_1\dd x_2\cdots \dd x_{n-1}\times \\
    &\qty (\bra{x_n}\E^{-\I H( t_n-t_{n-1} )}\ket{x_{n-1}}\cdots\bra{x_2}\E^{-\I H( t_2-t_1 )}\ket{x_1}\bra{x_1} \E^{-\I H t_1}\ket{x_0})
}
\end{equation}



最后,当我们给时间所分的段数趋于无穷时,能得到所有可能的路径,如\autoref{PIntQM_fig3} 所示.同样地,所有这些路径的振幅之积分就是我们所求的$\bra{x_n}\E^{-\I H t_n}\ket{x_0}$.



\begin{figure}[ht]
\centering
\includegraphics[width=10cm]{./figures/PIntQM_4.pdf}
\caption{全体可能的路径.} \label{PIntQM_fig3}
\end{figure}


\subsection{路径积分的计算}


\subsubsection{无穷小路径的概率振幅与经典作用量}

我们回到\autoref{PIntQM_fig1} 的例子.任取$t_1$时刻一个给定的位置$x_1$,得到一条给定的折线路径.粒子沿着\textbf{这条路径}演化到$x_2$的概率振幅是\footnote{你可以理解为,在$t_1$时刻整个宇宙盖上了吸收粒子的材料,只有$x_1$处例外,然后立刻把材料取走,那么在$t_2$时刻在$x_2$找到粒子的概率振幅就是\autoref{PIntQM_eq2} .这可以和单缝衍射作类比.}
\begin{equation}\label{PIntQM_eq2}
\bra{x_2}\E^{\I H (t_2-t_1)}\ket{x_1}\bra{x_1}\E^{\I H t_1}\ket{x_1}
\end{equation}
这是两条直线段路径的概率振幅之\textbf{积}.

% 应用上述事实,让时间分割趋于无穷,可以推知,一条任意路径的概率振幅,可以视为各\textbf{无穷小段}的概率振幅之积.于是,我们又要考虑无穷小近似了.

% \subsubsection{无穷小路径的概率振幅}

由于概率振幅是个数字,即路径的泛函,这让人想到一个重要的泛函,经典作用量:$S=\int_{\text{给定路径}}L\dd t$,其中拉格朗日函数$L=\frac{m\dot{x}^2}{2}-V(x)$.

折线段的作用量,是各直线段组分的作用量之\textbf{和};而\autoref{PIntQM_eq2} 中,折线段的概率振幅,是各直线段组分的作用量之\textbf{积}.因此,我们可以取作用量的指数,来对应概率振幅.

对于每条折线段(以\autoref{PIntQM_fig2} 为例),记其$t_{i-1}$和$t_i$之间直线段的作用量为
\begin{equation}
S(t_{i}-t_{i-1})=\int^{t_i}_{t_{i-1}}\dd t L(x, \dot{x})
\end{equation}
给定\textbf{无穷小段}的作用量记为$S(\dd t)$.

我们猜测
\begin{equation}
\ali{
    \bra{x_1}\E^{\I H \dd t}\ket{x_0} &= \qty(\frac{1}{\omega(\dd t)})\exp\qty(\I S(\dd t))\\
    %&= \qty(\frac{1}{\omega(\Delta t)})\qty(1+\frac{\I}{\hbar}S(\dd t))
}
\end{equation}
其中$\omega$是比例和量纲的修正项.下面就讨论该猜想的合理性.

我们对自由粒子应用上述猜想.注意到自由粒子的拉格朗日函数为$L=mv^2/2$,而无穷小段可以近似于匀速直线运动,故无穷小段的作用量为(时间差$\Delta t$是小量,位置$x_0$到$x_1=x_0+v \Delta t$):
\begin{equation}
\ali{
    S(\Delta t) &= \int^{\Delta t}_0 \dd t\qty(\frac{mv^2}{2})\\
    &= \frac{mv^2}{2}\Delta t\\
    &= \frac{m}{2}\qty(\frac{x_1-x_0}{\Delta t})^2\Delta t
}
\end{equation}
于是
\begin{equation}\label{PIntQM_eq3}
\ali{
    \exp\qty(\I S(\Delta t)) &= \exp\qty(\frac{\I m(x_1-x_0)^2}{2\Delta t})
}
\end{equation}
\autoref{PIntQM_eq3} 右边的指数恰为自由粒子传播子\autoref{PpgtQM_eq9}~\upref{PpgtQM}的指数部分!

因此,对于一维自由粒子,其无穷小段路径的概率振幅为
\begin{equation}\label{PIntQM_eq4}
\bra{x_1}\E^{\I H \dd t}\ket{x_0} = \sqrt{\frac{m}{2\pi \dd t\I}}\exp\qty(\I S(\dd t))
\end{equation}
其中$S$是该路径的经典作用量.

类似地,对于三维自由粒子,其无穷小段路径的概率振幅为
\begin{equation}
\bra{x_1}\E^{\I H \dd t}\ket{x_0} = \qty(\frac{m}{2\pi \dd t\I})^{3/2}\exp\qty(\I S(\dd t))
\end{equation}



\subsubsection{一维自由粒子的路径积分}

由\autoref{PIntQM_eq4} 易得,对于\autoref{PIntQM_fig2} 中任意\textbf{给定路径}$\Gamma$,其起点与终点分别是$(x_0, t_0)$和$(x_n, t_n)$,则粒子沿着这条路径的概率传播的振幅为
%未完































