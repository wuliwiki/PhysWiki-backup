% 热力学第一定律
% 热力学第一定律|能量守恒|做功|传热|内能

\begin{issues}
\issueDraft
\end{issues}

\pentry{压力体积图\upref{PVgraf}, 理想气体内能\upref{IdgEng}}

热力学第一定律是能量守恒在热力学中的形式,外部对系统传递的热量等于系统对外做功加上系统的内能增加:
\begin{equation}\label{Th1Law_eq1}
\Delta Q = W + \Delta U
\end{equation}

对于容器中的气体,设其压强为 $p$,体积为 $V$,那么对外做功可以写成积分形式:
\begin{equation}
W = \int p \dd{V}
\end{equation}
热力学第一定律写成微分形式是
\begin{equation}
\dd U=\delta Q-\delta W=\delta Q-p\dd V
\end{equation}
$Q$ 和 $W$ 前用的是 $\delta$ 符号而不是全微分符号,是因为 $Q$ 和 $W$ 和系统变化的过程本身有关.$U$ 前面用的是全微分符号,是因为 $U$ 本身代表气体系统的内能函数,是一个态函数,而 $\Delta U$ 只与初始和最终的系统状态有关.$U$ 是系统状态的函数,我们称它为 \textbf{态函数},而 $W$ 和 $Q$ 并不是系统状态的函数,它们用来描述在一个系统变化过程中功和热量的传递,是一个 \textbf{过程量}.

\subsection{内能和态函数}
对理想气体, 令分子自由度为 $i$, 有
\begin{equation}
U = \frac{i}{2}n RT
\end{equation}
