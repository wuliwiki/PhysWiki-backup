% 级数(数学分析)
% license Xiao
% type Tutor

\pentry{级数(简明微积分)\nref{nod_SerCal},极限存在的判据、柯西序列\nref{nod_CauSeq}}{nod_af31}

\subsection{基本定义回顾}

% Giacomo: 我把一部分移动到级数(简明微积分)\upref{SerCal}里面了

设 $\{a_n\}_{n\in\mathbb{N}}$ 是一个复数序列, $\sum_{n=1}^\infty a_n$ 为 $a_n$ 的级数 (series)。假设级数是收敛的,它满足
$$
\sum_{n=1}^\infty a_n = \lim_{N \to \infty} \sum_{n=1}^N a_n~.
$$

记级数 $\sum_{n=1}^\infty a_n$ 的部分和为
\[
S_N := \sum_{n=1}^N a_n~.
\]

根据极限运算的简单性质, 容易看出部分和序列 $\{S_N\}$ 若有极限, 则必定有 $\lim_{n\to\infty}a_n=0$, 由此即得到级数收敛的一个简单的必要条件。 但它并不是充分条件。 一般级数收敛性的唯一充分必要条件是
\begin{theorem}{级数收敛的柯西判据}
级数
\begin{equation}
\sum_{n=1}^\infty a_n~
\end{equation}



收敛, 当且仅当任给 $\varepsilon>0$, 都存在正整数 $N_\varepsilon$, 使得当 $N'>N>N_\varepsilon$ 时有
$$
\left|\sum_{n=N}^{N'} a_n\right|<\varepsilon~,
$$
\end{theorem}
这是序列收敛的柯西判据的直接推论。

\begin{exercise}{}
利用极限的四则运算法则证明: 级数 $\sum_{n=1}^\infty a_n$ 收敛, 当且仅当序列 $\{a_n\}$ 的实部和虚部组成的级数都收敛。
\end{exercise}

\begin{example}{发散级数}
根据定义, 一般项不趋于零的级数当然发散。 但一般项趋于零也并不能保证收敛性。 一个著名的例子是调和级数
$$
\sum_{n=1}^\infty\frac{1}{n}~,
$$
有许多种办法能证明它发散。 例如, 对于正整数 $N$, 和式
$$
\sum_{n=N}^{2N}\frac{1}{n}~
$$
总共有 $N+1$ 项, 除了最后一项之外, 每一项都大于 $1/2N$, 所以
$$
\sum_{n=N}^{2N}\frac{1}{n}>\frac{N+1}{2N}>\frac{1}{2}~,
$$
按照柯西判据, 这个级数是发散的。

通过积分的办法, 可以得到一些有用的渐近公式:
$$
\sum_{n=1}^{N}\frac{1}{n}=\ln N+O(1)\quad (N\to\infty)~,
$$
如果 $0<\alpha<1$, 那么
$$
\sum_{n=1}^{N}\frac{1}{n^\alpha}=\frac{N^{1-\alpha}}{1-\alpha}+O(1)~.
$$
\end{example}

\subsection{重要的收敛级数}

这里列出一些最重要的收敛级数。

$p$-级数是指
$$
\sum_{n=1}^\infty\frac{1}{n^p}~,
$$
如果 $p>1$ 则级数收敛, $p\leq1$ 则级数发散。 最方便的证明是积分判别法; 它还能给出收敛和发散的速度估计。

$p$-对数级数是指
$$
\sum_{n=2}^\infty\frac{1}{n(\ln n)^p}~,
$$
如果 $p>1$ 则级数收敛, $p\leq1$ 则级数发散。 最方便的证明仍然积分判别法。

几何级数是指
$$
\sum_{n=0}^\infty q^n~,
$$
当 $q=1$ 时, 它的部分和 $\sum_{n=0}^N q^n=N+1$. 当 $q\neq1$ 时, 按照等比数列的求和公式, 可以算出它的部分和:
$$
\sum_{n=0}^N q^n=\frac{1-q^{N+1}}{1-q}~.
$$
由此可见, 几何级数在 $|q|<1$ 时收敛, 在 $|q|\geq1$ 时发散。

还有一类例子是所谓的裂项级数, 也称为望远镜级数。 设 $\{b_n\}$ 是极限为 $b$ 的复数序列, 则级数
$$
\sum_{n=1}^\infty(b_{n+1}-b_n)~
$$
是收敛的。 实际上, 它的部分和是
$$
\sum_{n=1}^N(b_{n+1}-b_n)=b_{N+1}-b_1~,
$$
当 $N\to\infty$ 时它的极限等于 $b-b_1$.

\subsection{定义辨析}
对于收敛的级数 $\sum_{n=1}^\infty a_n$, 有时也可以用这个形式表达式来代表它的和。 但是需要注意, 形式等式 $\sum_{n=1}^\infty a_n=S$ 的真正含义是 $\lim_{N\to\infty}S_N=S$, 必须要在这个意义下来理解它。 由此, 级数的和与有限个实数的和有所不同。 

根据极限的四则运算法则, 两个收敛级数的和与差都仍然是收敛级数。 但对于改变收敛级数的求和顺序, 以及两个收敛级数的乘积, 则需要多加留意。 后续文章\enref{绝对收敛与条件收敛}{Convg} 将详细解释这里可能出现的问题。

如果对级数进行四则运算时涉及到不收敛的级数, 则需要更加留意。 来看一个著名的例子。

\begin{example}{}
考虑序列 $a_n=(-1)^n$ 组成的几何级数。 这个序列的部分和在 $-1$ 和 $0$ 之间来回跳跃: 当 $N$ 为奇数时 $S_N=-1$, 当 $N$ 为偶数时 $S_N=0$. 因此级数 $\sum_{n=1}^\infty(-1)^n$ 是发散的。 

如果强行定义 $S=\sum_{n=1}^\infty(-1)^n$, 并对这个 $S$ 像有限和那样进行四则运算, 那么立刻就能导出矛盾。 例如, 将级数的和铺开并进行错位相加:
$$
\begin{aligned}
1 & \quad-1 & 1 & \quad-1 & 1 &...\\
  & \quad\quad1 & -1 & \quad\quad1 & -1 &...
\end{aligned}~
$$
除了最开头的一项之外, 其余项都消去了。 由此将得到"等式"$2S=1$, 或者 $S=1/2$. 但如果换一种错位相加的方式, 例如错两位进行相加, 那么得到的又将是"等式"$S=0$. 造成这种矛盾的唯一原因在于 $S$ 实际上不能代表任何实数, 因此实数的四则运算法则对于它不适用。

然而, 以上的形式操作尽管导出逻辑矛盾, 却并非没有意义。 由此引发出的概念叫做"发散级数的广义求和"(generalized summation of divergent series). 顾名思义, 这种操作方式与通常意义下的求和 -- 取部分和的极限 -- 是不同的。 它在20世纪的分析数学中占据着重要的地位。
\end{example}
