% 傅里叶变换
% keys 微积分|傅里叶级数|傅里叶变换|三角傅里叶变换|指数傅里叶变换
% license Xiao
% type Tutor

\pentry{傅里叶级数(指数)\nref{nod_FSExp},傅里叶变换(三角)\nref{nod_FTTri}}{nod_8af4}

用三角傅里叶变换\upref{FTTri} 中同样的方法可把指数傅里叶级数\upref{FSExp}的区间长度 $l$ 取极限后拓展为指数傅里叶变换
\begin{equation}\label{eq_FTExp_6}
g(k) = \frac{1}{\sqrt{2\pi }} \int_{-\infty }^{+\infty } f(x)\E^{-\I kx} \dd{x}~,
\end{equation}
\begin{equation}\label{eq_FTExp_1}
f(x) = \frac{1}{\sqrt{2\pi }} \int_{-\infty }^{+\infty } g(k)\E^{\I kx} \dd{k}~.
\end{equation}
当 $f(x)$ 为实函数时,$g(k)$ 的实部是偶函数,虚部是奇函数。

\subsection{实数函数的傅里叶变换}
上述傅里叶变换中, $g(k)$ 需要满足什么条件才能使 $f(x)$ 是实函数呢?实函数的充要条件是 $f(x) = f^*(x)$, 代入\autoref{eq_FTExp_6} 得
\begin{equation}
g(k) = \frac{1}{\sqrt{2\pi }} \int_{-\infty }^{+\infty } f^*(x)\E^{-\I kx} \dd{x}~.
\end{equation}
两边取复共轭, 得
\begin{equation}\label{eq_FTExp_5}
g^*(k) = g(-k)~.
\end{equation}
所以对于实数函数的傅里叶变换, 往往只需要 $k$ 的正半轴。 上式就是 $f(x)$ 为实函数的充要条件, 要证明充分性, 将其代入\autoref{eq_FTExp_1} 可得 $f(x) = f^*(x)$。

\begin{example}{高斯分布的傅里叶变换}\label{ex_FTExp_1}
要计算高斯函数
\begin{equation}
f(x) = \E^{-ax^2} \quad (a > 0)~
\end{equation}
的傅里叶变换, 代入\autoref{eq_FTExp_6} 并使用\autoref{ex_Erf_2}~\upref{Erf}有
\begin{equation}
g(k) = \frac{1}{\sqrt{2a}} \exp(-\frac{k^2}{4a})~,
\end{equation}
一个方便的记忆法是 $x^2$ 前的系数乘以 $k^2$ 前的系数相乘等于 $1/4$。
\end{example}

\begin{example}{}\label{ex_FTExp_2}
\begin{equation}
f(x) = \begin{cases}
\exp(\I k_0 x) \cos^2(ax) & (\abs{x} < \frac{\pi}{2a})~.\\
0 & (\text{其他})
\end{cases}
\end{equation}
则傅里叶变换为
\begin{equation}\label{eq_FTExp_11} % 已数值验证
g(k) = \frac{\sqrt{2\pi}a}{4a^2 - (k - k_0)^2} \sinc\qty[\frac{\pi (k - k_0)}{2a}]~,
\end{equation}
其中 $\sinc$ 函数见相关页面\upref{sinc}。
\end{example}

\begin{example}{方波}
区间 $[-l,l]$, 高为 $1$ 的单个方波,
\begin{equation}
g(k) = \sqrt{\frac{2}{\pi}} \frac{\sin(lk)}{k}~.
\end{equation}
\end{example}

\subsection{证明}
\pentry{狄拉克 delta 函数\nref{nod_Delta}}{nod_debb}

以下的证明可以用矢量空间和基底的概念得到更深刻的理解, 详见 “傅里叶变换与连续正交归一基底\upref{COrNoB}”。

我们把\autoref{eq_FTExp_1} 看作定义, 用狄拉克 $\delta$ 函数\upref{Delta}来证明\autoref{eq_FTExp_6}, 反之同理。 把\autoref{eq_FTExp_1} 代入\autoref{eq_FTExp_6} 得
\begin{equation}
g(k) = \frac{1}{2\pi} \int_{-\infty}^{+\infty} \qty[ \int_{-\infty}^{+\infty} g(k') \E^{\I k' x} \dd{k'}] \E^{-\I k x} \dd{x}~.
\end{equation}
这就是我们需要证明的。 我们把无穷的积分上下限改写为极限,即
\begin{equation}\ali{
g(k) = \frac{1}{2\pi} \lim_{m\to\infty}\int_{-m}^{m} \qty[\lim_{n\to\infty} \int_{-n}^{n} g(k') \E^{\I k' x} \dd{k'}] \E^{-\I k x} \dd{x}~.
}\end{equation}
如果内极限可以移动到两个积分外,有
\begin{equation}\ali{
g(k) = \frac{1}{2\pi} \lim_{m\to\infty} \lim_{n\to\infty} \int_{-m}^{m}\int_{-n}^{n} g(k') \E^{\I (k'-k) x} \dd{k'}\dd{x}~.
}\end{equation}
我们假设 $g(k')$ 在 $[-n,n]$ 内绝对值可积, 那么\upref{Fubin0}有限区间的重积分可以交换顺序
\begin{equation}
g(k) = \frac{1}{2\pi} \lim_{m\to\infty} \lim_{n\to\infty} \int_{-n}^{n} g(k') \int_{-m}^{m}\E^{\I (k'-k) x} \dd{x} \dd{k'}~.
\end{equation}
由\autoref{ex_Delta_1}~\upref{Delta}, 里面的积分变为 $2\pi\delta(k'-k)$, 现在只需证明
\begin{equation}
g(k) = \int_{-\infty}^{+\infty} \delta(k'-k)  g(k')\dd{k'}~,
\end{equation}
而这就是\autoref{eq_Delta_7}~\upref{Delta}。 证毕。

\subsection{性质}
为了书写方便我们用\textbf{算符} $\mathcal F$ 和 $\mathcal F^{-1}$ 表示傅里叶变换和反变换, 即 $\mathcal F f = g$ 以及 $\mathcal F^{-1} g = f$。 算符在这里可以看作 “函数的函数”, 即自变量和函数值都是函数。

平移性质:
\begin{equation}\label{eq_FTExp_4}
\mathcal F [f(x) \E^{\I k_0 x}] = g(k - k_0)~.
\end{equation}
\begin{equation}\label{eq_FTExp_7}
\mathcal F[f(x - x_0)] = g(k) \E^{-\I k x_0}~.
\end{equation}
也就是说, 给函数乘以 $\E^{\I k_0 x}$ 因子再做傅里叶变换, 等于先对函数做傅里叶变换, 再向右平移 $k_0$; 给函数再向右平移 $x_0$ 再做反傅里叶变换, 等于先对函数做傅里叶变换, 再乘以 $\E^{-\I x_0 k}$。 证明留做习题。

变换前后模长不变:
\begin{equation}\label{eq_FTExp_2}
\int_{-\infty}^{+\infty} g(k)^* g(k) \dd{k} = \int_{-\infty}^{+\infty} f(x)^* f(x) \dd{x}~.
\end{equation}

函数拉伸的傅里叶变换:
\begin{equation}
\mathcal F[f(ax)] = \frac{1}{a} g\qty(\frac{k}{a})~.
\end{equation}
也就是说把函数在 $x$ 方向压缩 $a$ 倍后, 各个频率都变大 $a$ 倍, 所以傅里叶变换会在 $k$ 方向拉伸 $a$ 倍, 另外归一化不变性易得系数 $1/a$。

导数的傅里叶变换:
\begin{equation}\label{eq_FTExp_3}
\mathcal F [f'(x)] = \I k g(k)~,
\end{equation}
同理
\begin{equation}\label{eq_FTExp_10}
\mathcal F^{-1} [g'(k)] = -\I x f(x)~.
\end{equation}

作为\autoref{eq_FTExp_2} 的拓展, 有
\begin{equation}
\int_{-\infty}^{+\infty} f_1(x)^* f_2(x) \dd{x} = \int_{-\infty}^{+\infty} g_1(k)^* g_2(k) \dd{k}~.
\end{equation}
这可以理解为傅里叶变换不改变内积, 所以是一个无穷维空间中的幺正变换\upref{Unitar}。

如果 $f_1(x)$ 可以在 $x = 0$ 泰勒展开, 有
\begin{equation}\label{eq_FTExp_8}
\mathcal{F}[f_1(x) f_2(x)] = f_1\qty(\I \pdv{k}) g_2(k)~.
\end{equation}
如果 $g_1(k)$ 可以在 $k = 0$ 泰勒展开, 有
\begin{equation}\label{eq_FTExp_9}
\mathcal{F}^{-1}[g_1(k) g_2(k)] = g_1\qty(-\I \pdv{x}) f_2(x)~.
\end{equation}
注意\autoref{eq_FTExp_3} 和\autoref{eq_FTExp_10} 是该性质的特殊情况(令\autoref{eq_FTExp_8} 中 $f_1(x) = x$, \autoref{eq_FTExp_9} 中 $g_1(k) = k$)。 记忆方法: 在傅里叶变换外面的 $\I\pdv*{k}$ 相当于傅里叶变换里面的 $x$, 反傅里叶变换外面的 $-\I\pdv*{x}$ 相当于反傅里叶变换里面的 $k$。

平均值:
\begin{equation}
\ev{k} = \int_{-\infty}^{+\infty} k\abs{g(k)}^2 \dd{x} = \int_{-\infty}^{+\infty} f^*(x)\qty(-\I \dv{x}) f(x) \dd{x}~.
\end{equation}
\begin{equation}
\ev{x} = \int_{-\infty}^{+\infty} x\abs{f(x)}^2 \dd{x} = \int_{-\infty}^{+\infty} g^*(k)\qty(\I \dv{k}) g(k) \dd{k}~.
\end{equation}
推导参考 “平均值(量子力学)\upref{QMavg}”。

不确定性原理:
\begin{equation}
\sigma_x \sigma_k \geqslant \frac{1}{2}~.
\end{equation}
其中 $\sigma_x, \sigma_k$ 分别是 $f(x), g(k)$ 的标准差。 推导参考量子力学的 “不确定性原理\upref{Uncert}”。

\subsection{性质的证明}

\textbf{证明}\autoref{eq_FTExp_2} : 把傅里叶变换看成傅里叶级数在 $l \to \infty$ 时的极限, 使用\autoref{eq_FSExp_5}~\upref{FSExp}, 右边的求和在极限下变为积分即可证明。 详细过程留做习题。

\textbf{证明}\autoref{eq_FTExp_3} (\autoref{eq_FTExp_10} 同理): 对\autoref{eq_FTExp_1} 关于 $x$ 求导得
\begin{equation}
f'(x) = \frac{1}{\sqrt{2\pi }} \int_{-\infty }^{+\infty } [\I kg(k)]\E^{\I kx} \dd{k}~.
\end{equation}
把方括号看作一整个 $k$ 的函数, 那么上式对应的反变换为
\begin{equation}
\I kg(k) = \frac{1}{\sqrt{2\pi }} \int_{-\infty }^{+\infty } f'(x) \E^{-\I kx} \dd{x} = \mathcal F [f'(x)]~.
\end{equation}
其中 $g(k) = F [f(x)]$, 证毕。

\textbf{证明}\autoref{eq_FTExp_8} (\autoref{eq_FTExp_9} 同理): 过程和\autoref{eq_FTExp_3} 类似, 等式右边为
\begin{equation}
\begin{aligned}
& \quad \frac{1}{\sqrt{2\pi }} f_1\qty(\I \pdv{k}) \int_{-\infty }^{+\infty } f_2(x)\E^{-\I kx} \dd{x}\\
&= \frac{1}{\sqrt{2\pi }} \int_{-\infty }^{+\infty } f_2(x) f_1\qty(\I \pdv{k})\E^{-\I kx} \dd{x}\\
&= \frac{1}{\sqrt{2\pi }} \int_{-\infty }^{+\infty } f_1(x) f_2(x) \E^{-\I kx} \dd{x}~.
\end{aligned}
\end{equation}
