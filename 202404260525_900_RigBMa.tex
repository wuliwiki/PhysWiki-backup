% 刚体的几何运算(Matlab)
% license Xiao
% type Tutor

\begin{issues}
\issueDraft
\end{issues}

\begin{itemize}
\item \verb|intersect| 决定两个多边形重合部分的多边形
\item \verb|alphaShape| 确定 2D 或 3D 刚体的形状
\item \verb|inShape| 判断一点是否在某个 \verb|alphaShape| 的内部
\end{itemize}

\begin{figure}[ht]
\centering
\includegraphics[width=10cm]{./figures/e47d03b1a196f8f8.png}
\caption{\lstinline|intersect| 测试结果} \label{fig_RigBMa_1}
\end{figure}

\begin{lstlisting}[language=matlab]
p1 = polyshape([0.425 0.284 0.335 0.482 0.388 0.425], ...
    [0.715 0.610 0.371 0.453 0.584 0.715]);
p2 = polyshape([0.439 0.411 0.389 0.408 0.466 0.439], ...
    [0.621 0.639 0.610 0.524 0.541 0.621]);
figure; subplot(1,2,1); plot(p1); hold on; plot(p2);
p3 = intersect(p1, p2);
subplot(1,2,1); plot(p3);
\end{lstlisting}
\verb|p1| 的类型是 \verb|polyshape|, \verb|p1.Vertices| 是 \verb|N*2| 尺寸的顶点坐标。 一个 \verb|polyshape| 可以表示多个多边形, 个数为 \verb|p3.NumRegions|。 \verb|p3.Vertices| 中用一行 \verb|[nan, nan]| 隔开不同的多边形。 如果 \verb|p3| 是空的, 那么 \verb|p3.Vertices| 是空矩阵。


\begin{figure}[ht]
\centering
\includegraphics[width=10cm]{./figures/7e561244e6712f29.png}
\caption{\lstinline|alphaShape| 测试结果} \label{fig_RigBMa_2}
\end{figure}

\begin{lstlisting}[language=matlab]
[X,Y,Z] = sphere(20);
shp = alphaShape(X(:),Y(:),Z(:));
figure; subplot(1,2,1); plot(shp);
shp.Alpha = 2.5;
subplot(1,2,2); plot(shp);
inShape(shp, [1.01,0,0]); % 判断是否在内部
\end{lstlisting}
