% 摩尔量与偏摩尔量

\begin{issues}
\issueDraft
\end{issues}

\pentry{相简介(热力学)\upref{PHS}}
\footnote{本文参考了朱文涛的《简明物理化学》与Schroeder的《热物理学导论》}

\subsection{多元多相系统中的状态量}
\pentry{态函数\upref{statef}}
\begin{figure}[ht]
\centering
\includegraphics[width=8cm]{./figures/e5aa9c6040573e79.pdf}
\caption{多元多相系统} \label{fig_ParMol_1}
\end{figure}

在多元多相系统中,我们需要扩展状态量与“状态公理”的含义。

在多元多相系统中,由于不同相的热力学性质不同,因此需要分别写出每一个相的状态量。\textsl{此处上标的含义是相,而不是指数!}
$$U^\alpha, U^\beta,...$$

同时,由于物质可在各相间流动、每一相中物质的含量可以变化,因此强调相状态量也与各物质的物质的量有关。
$$U^\alpha = U^\alpha (p^\alpha, T^\alpha, n_1^\alpha,n_2^\alpha,...)~.$$

\subsection{单组份系统:摩尔量}
我们先来思考一种最简单的单组份系统,即系统中只包括一种物质。以下我们先讨论一个相中的摩尔量,因此省略相的角标。

我们先以内能为例:
$$U = U (p, T, n)~.$$
由于内能是广延量,因此内能应当与相中物质的量成正比。比如,如果物质的量增多到原来的$4$倍,那么内能也将变为原来的$4$倍。
$$U(p, T, 4 n) = 4 U (p, T, n)~,$$
这是广延量的一个非常好的性质:根据齐次方程欧拉定理\footnote{关于这个神奇的定理可以参考\href{https://zhuanlan.zhihu.com/p/61111531}{https://zhuanlan.zhihu.com/p/61111531}(站外链接)},我们可以将其化为以下形式
\footnote{严格地论证步骤是基于$U=U(p,T,n)$定义偏摩尔量的微分形式$u = \left( \pdv{U}{n} \right)_{p,T}$ (见下),再从齐次方程欧拉定理论证 $U = n \cdot u $}
:
$$U(p, T, n) = n \cdot u (p, T)~.$$
可见, $u (p, T)$ 不再与相的规模相关,而成为了某种强度量。我们将其定义为\textbf{摩尔内能},意味着每摩尔物质所具有的内能。摩尔量有时也通过在相应的状态量下加下标$m$表示,例如$U_m$。

同理,对于其他广延量,我们可以定义相应的摩尔量:
$$
\begin{aligned}
S(p, T, n) &= n \cdot s (p, T)\\
V(p, T, n) &= n \cdot v (p, T)\\
G(p, T, n) &= n \cdot \mu (p, T)\\
&...\\
\end{aligned}
$$
Gibbs自由能的摩尔量写法上比较特别,一般写为$\mu$,也称为化学势。以后你会知道为什么人们\textsl{对Gibbs的摩尔量搞特殊}。

对于纯物质的摩尔量,可加$*$上标与混合物质中的区分(见下),例如$v^*, \mu^*$。

\subsubsection{摩尔量的微分定义}
另一方面,我们对
$$U(p, T, n) = n \cdot u (p, T)~.$$
关于$n$求偏导,那么
$$u = \left( \pdv{U}{n} \right)_{p,T}~,$$
这是摩尔量的另一种定义。这启发了我们摩尔势的另一层含义:某一状态下保持$p,T$不变,再往系统中加入少量物质后,相应热力学量的变化。这个定义在下文的偏摩尔量中更为常用。

\subsubsection{摩尔量的热力学基本关系}
\pentry{热力学关系式\upref{MWRel}}

我们先写出化学势(摩尔Gibbs自由能)的定义
$$G = n \cdot \mu~.$$
对其求微分
$$\dd G = n \cdot \dd \mu +  \mu \cdot\dd n~.$$
要如何理解这个公式?做一个简单的类比:商店收益的上升(“$\dd G$”)可以来自于商品价格的增长("$n \cdot \dd \mu$"),或商品销量的上升("$\mu \cdot \dd n$")

假设系统中的物质数不发生变动,即$\dd n = 0$,那么
$$\dd G = n \cdot \dd \mu~.$$

同时,根据热力学基本关系式\upref{MWRel} :
$$\dd G = -S \dd T + V \dd P~.$$
因此
$$-S \dd T + V \dd P = n \cdot \dd \mu~,$$
或者
$$
\dd \mu = -s \dd T + v \dd P~,
$$
这就是化学势的微分等式。可见,摩尔量的微分与热力学关系式 中的相同,只是把广延量换为了相应的摩尔量。

将其代回上式,得多元系统中的热力学基本关系式:
$$\dd G = -S \dd T + V \dd P + \mu \dd n~.$$

可见,相应的热力学基本关系式只是补充了额外有关化学势与物质的量的一项。可以证明其余的热力学关系式也可以这么修正,不过证明过程比较繁琐:
$$
\begin{aligned}
&\dd U = T \dd S - P \dd V + \mu \dd n~,\\
&\dd H = T \dd S + V \dd P + \mu \dd n~,\\
&\dd A = -S \dd T - P \dd V + \mu \dd n~.\\
\end{aligned}
$$

\subsubsection{不同环境下的摩尔量}
\pentry{盖斯定律与设计路径\upref{Hess}}
假如我们知道了 某一状态下物质的摩尔量$\mu(T_0,p_0)$,怎么计算另一状态下的摩尔量 $\mu(T_0,p_1)$?答案还是设计路径\upref{Hess} :
$$\mu(T_0,p_1) = \mu(T_0,p_0) + \int_{p_0}^{p_1} \left( \pdv{\mu}{p} \right)_{T_0} \dd p~.$$
根据上述的热力学基本关系,$\left( \pdv{\mu}{p} \right)_{T_0} = v$
因此,$$\mu(T_0,p_1) = \mu(T_0,p_0) + \int_{p_0}^{p_1} v \dd p~.$$

许多热力学实验量都是在特定环境(例如,$T = 298 \Si{K}, p = 101325\Si{Pa}$)下测量的,因此你需要使用这些公式来换算。

\begin{exercise}{}
写出 $\mu(T_1,p_1)$ 的表达式。
\end{exercise}

\subsection{多组分系统:偏摩尔量}
对于一个多组分的均匀热力学系统,其各个热力学量可以由温度 $T$,压强 $p$,各组分的物质的量 $n_1,n_2,\cdots $ 来决定。

以内能为例,一个多元均匀体系的内能可以被表示为一个函数:
\begin{equation}
U = U (p, T, n_1, n_2, ...)~.
\end{equation}
内能为\textbf{广延量},它应当具有广延性质。例如,当将整个多元体系复制成 4 份后,所有组分的物质的量都变为原先的 4 倍,那么内能也应当变为原先的 4 倍:
$$ U (p, T, 4n_1, 4n_2, ...) = 4U (p, T, n_1, n_2, ...)~.$$
注意这里的“广延性质”与单组份的情况不大一样:只有当所有组分的含量都翻4倍,总的内能才会上升到原来的4倍。

更一般地,
\begin{equation}
U(p,T,tn_1,tn_2,\cdots )=t U(p,T,)
\end{equation}


如法炮制,我们可以得到
$$ U (p, T, n_1, n_2, ...) = \sum_i n_i u_i(p, T, x_1, x_2, ...)~. $$
其中,每一个 $u_i$被称为该组元的\textsl{偏摩尔内能},该公式也称偏摩尔量的集合公式。偏摩尔量有时也通过在相应的状态量下加下标$B$或者$m,B$表示,$B$代表物质种类,例如$U_B$。

特别要注意的是,相比与摩尔量,偏摩尔量有一些相当独特的性质:
\begin{itemize}
\item 尽管偏摩尔量与相的规模无关,但是\textbf{与相中各物质的比例}有关。当系统中物质比例改变时,偏摩尔量也可能改变。
\item 某一物质的偏摩尔量与相应纯物质的摩尔量一般\textbf{不相同}。
\item 尽管集合公式是一个漂亮的累加,但是偏摩尔量\textbf{不适合}理解为“每摩尔物质具有的热力学量”。例如,在某些溶液中,溶质的溶解反而能降低溶液的总体积。此时,溶质的偏摩尔体积$v_B$的值是负数。显然,物质的“体积”不能是负数。
\end{itemize}

\begin{example}{}
例如,某一条件$(p,T)$下系统中具有$4 \Si{mol} A$物质与$2 \Si{mol} B$物质,那么系统的内能
$$U = 4 u_{A} +  2 u_B~. $$
如果系统中具有$8 \Si{mol} A$物质与$4 \Si{mol} B$物质,那么
$$U = 8 u_A +  4 u_B~.$$
但是,如果系统中具有$8 \Si{mol} A$物质与$2 \Si{mol} B$物质,由于物质的比例不再是2:1,因此偏摩尔量改变。
$$U \ne 8 u_A +  2 u_B~.$$
\end{example}

同理,我们可以定义其他广延量相应的偏摩尔量
$$
\begin{aligned}
V &= \sum_i  n_i \cdot v_i ~,\\
G &= \sum_i  n_i \cdot \mu_i~.\\
&...\\
\end{aligned}
$$

\subsubsection{偏摩尔量的微分定义}
我们还是以内能为例:
$$ U (p, T, n_1, n_2, ...) = \sum_i n_i u_i(p, T, x_1, x_2, ...)~. $$
对集合公式求关于$n_i$的偏导数\footnote{这个做法是不严谨的,因为$x_i$其实也是隐含$n_i$的,但阴差阳错间竟能得到正确结果。严谨的做法是先从$U=U(p,T,n_1, n_2,...)$定义偏摩尔量为 ${u_i} = \left(\pdv{U}{n_i}\right)_{p,T,n_j, j \neq i}$,然后再根据齐次方程欧拉定理得到上述的集合公式 $ U = \sum_i n_i u_i $},
$$
{u_i} = \left(\pdv{U}{n_i}\right)_{p,T,n_j, j \neq i} ~.
$$
这告诉我们偏摩尔量的另一层含义:$u_i \dd n_i$ 是某一状态下保持$(p,T)$与其余物质的量不变,再往系统中加入少量物质i,系统内能的变化。

\subsubsection{偏摩尔量与摩尔量的关系}
我们在上文中已经提到,偏摩尔量$\mu_i$与相应纯物质的摩尔量$\mu_i^*$是不同的。那么他们之间到底是怎么联系的呢?我们一直避而不谈这个问题的原因在于,这个问题\textsl{太复杂了},事实上这并不是经典热力学能有效处理的。

不过,如果我们认为多元相是一种理想混合物\upref{IMCPTV},那么我们就能联系偏摩尔量与摩尔量。\textsl{尽管事实上没什么物质混合得如此理想。}
% \subsubsection{Gibbs-Duhem 公式}
% $$
% \sum n_B \dd Z_B = 0
% $$

% 推导:对集合公式两端求导,$dZ=\sum n_B \dd Z_B + \sum {Z_B}  \dd n_B$,即得证。

% 特别地,对于二元混合物,
% $$
% n_1 \dd {Z_1} = - n_2 \dd {Z_2}
% $$
% 或
% $$
% \dd {Z_1} = -\frac{n_2}{n_1} \dd {Z_2}
% $$

