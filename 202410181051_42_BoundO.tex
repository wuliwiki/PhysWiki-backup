% 有界算子
% keys 有界性|算子
% license Usr
% type Tutor

\begin{issues}
\issueDraft
\end{issues}

\pentry{有界集\nref{nod_BudSet},拓扑线性空间中的线性算子\nref{nod_TLinO}}{nod_8722}
拓扑线性空间的集可以定义有界性,相应的拓扑线性空间中的线性算子也可以定义有界性。

\begin{definition}{有界算子}
设 $E,E_1$ 是两个拓扑线性空间, $A:D_A\rightarrow E_1$ 是 $E$ 到 $E_1$ 的\enref{线性算子}{TLinO}。若 $A$ 把 $E$ 的属于 $D_A$ 的每一有界集都映到 $E_1$ 的 有界集,则称 $A$ 是\textbf{有界的}。
\end{definition}

线性算子的有界性和连续性有着密切的联系,这可以由下面的定理看出。

\begin{theorem}{}
1.线性连续算子必有界;

2.若 $A:E\rightarrow E_1$ 是线性有界算子,且 $E$ 满足第一可数性公理,则 $A$ 连续。
\end{theorem}
\textbf{证明:}1. 令 $A:D_A\rightarrow E_1$ 是 $E$ 到 $E_1$ 上的线性连续算子。 我们利用反证法证明:设 $M\subset D_A$ 是有界集,而 $AM\subset E_1$ 无界。那么存在 $E_1$ 中的零邻域 $V$,使得任意 $n>0$,都有 $\abs{\lambda}\geq n,AM\nsubseteq\lambda V$ (\autoref{def_BudSet_1})。即存在 $x\in M$,而 $Ax\nsubseteq \lambda V$。选取收敛于无穷大的正数序列 $\{m_i\}$ ,则存在 $\abs{\lambda_i}\geq m_i,x_i\in M$,使得 
\begin{equation}
Ax_i\nsubseteq \lambda_iM\quad\Rightarrow\quad A\qty(\frac{1}{\lambda_i}x_i)\nsubseteq V. ~
\end{equation}
因为 $\{\frac{\lambda_i}\}$ 是收敛于0的正数列,而 $M$ 有界,所以 $\{\frac{1}{\lambda_i}x_i\}$ 收敛于0(\autoref{the_BudSet_1} 第一点)。而 $A$ 连续,所以 
\begin{equation}
\lim_{i\rightarrow\infty} A\qty(\frac{1}{\lambda_i}x_i)=A(0)=0.~
\end{equation}


\textbf{证毕!}


