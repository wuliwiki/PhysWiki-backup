% 小时百科 App 设计

\begin{issues}
\issueDraft
\end{issues}

\subsection{主页}

(先不做)
\begin{figure}[ht]
\centering
\includegraphics[width=8cm]{./figures/947be9fb88d8961a.png}
\caption{主页设计} \label{fig_app_3}
\end{figure}

\subsection{百科}
\begin{figure}[ht]
\centering
\includegraphics[width=9cm]{./figures/57a147fccbfd858a.png}
\caption{百科 iPhone 页面草图, 顶部图标: 退出,其他功能。 底部图标从左到右: 返回,多页面,搜索,目录,点赞,收藏,评论}\label{fig_app_1}
\end{figure}

\subsubsection{按钮说明}
\begin{itemize}
\item 退出:退出百科到 app 首页。
\item 其他功能(右上角省略号):1. 作者以及贡献占比(从备份里面通过 git diff 决定)。 点击每个作者可进入其主页,选择关注等。 2. 分享功能, 转发给微信或 qq 好友, 或者复制链接。 3. 查看和编辑收藏夹, 打开时保持上次的展开状态和滚动位置。 收藏夹本质上是一个自定义的目录。 4. 显示:字体大小、 背景色(可以设置护眼的米黄色)。
\item 返回:返回上一次链接跳转之前的位置。 链接跳转可能是页面内的跳转也可能是不同页面的。
\item 多页面: 列出最近打开的若干词条,最近浏览的在最上面,可以拖动调整顺序,每行右边设置关闭按钮。 点击后打开词条中上次阅读的位置。
\item 搜索(包括最近浏览): 显示搜索框, 可以选择搜索本页,搜索百科标题,以及搜索百科全文。
\item 目录:支持每一级的向下展开,记住上次的位置
\item 点赞:点赞和显示数量。
\item 收藏:类似知乎的,可以自己建立不同类别,可以选择公开或者不公开。 若收藏了则图标改变。
\item 评论:类似知乎 app 的评论页面(\autoref{fig_app_2}), 可选择按照 “默认” 和 “最新” 排序。 评论顶部设置点赞功能并显示点赞数。
\end{itemize}

\begin{figure}[ht]
\centering
\includegraphics[width=9cm]{./figures/05e838d222e1cd65.png}
\caption{类似知乎 app 的评论页面} \label{fig_app_2}
\end{figure}

\subsubsection{操作说明}
\begin{itemize}
\item 在屏幕中间点一下,上下菜单弹出,再点一下,上下菜单隐藏。
\item 从屏幕左边向右滑动,跳到上一个词条。 从屏幕右边向左滑动,跳到上一个词条。
\item 点击蓝色小标题, 可以折叠内容, 直到下一个小标题或文末。
\end{itemize}

\subsection{以后的功能}
\begin{itemize}
\item 支持 pdf 阅读器, 阅读百科完整版。
\end{itemize}
