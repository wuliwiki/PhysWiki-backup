% 路径积分(量子力学)
% keys 传播子|跃迁振幅|关联振幅|概率振幅|高等量子力学

\pentry{传播子(量子力学)\upref{PpgtQM},薛定谔绘景和海森堡绘景\upref{HsbPic}}

% 定义传播子$C_i(t) = \bra{\bvec{x}_{i+1}}\E^{-\I Ht}\ket{\bvec{x}_i}$.




\subsection{概念的引入}

为了方便,我们考虑二维时空的情况,即空间只有一维.

在初始时刻$t=0$时,一个粒子处于$x_0$位置,将它的态记为$\ket{x_0}$,其在位置空间的波函数为$\psi_0(x)=\braket{x}{x_0}=\delta(x-x_0)$.

时间过去$t_1$后,我们在$x_1$位置测量,发现粒子的概率振幅为$\bra{x_1}\E^{-\I H t_1}\ket{x_0}$.因此我们可以说,粒子在时间$t_1$后“出现”在\footnote{我们也可以说“传播”到.}$x_1$的概率密度是$\abs{\bra{x_1}\E^{-\I H t_1}\ket{x_0}}^2$.

同样地,时间过去$t_2>t_1$后,粒子在$x_2$位置的概率振幅为$\bra{x_2}\E^{-\I H t_2}\ket{x_0}$.

注意到$\int \ket{x_1}\bra{x_1}\dd x_1=1$,即恒等变换(其矩阵总是单位矩阵),因此我们可以把这个积分插入到任何位置,比如:
\begin{equation}\label{PIntQM_eq1}
\ali{
    \bra{x_2}\E^{-\I H t_2}\ket{x_0} &= \bra{x_2}\int \ket{x_1}\bra{x_1}\dd x_1 \E^{-\I H t_2}\ket{x_0}\\
    &= \int \braket{x_2}{x_1}\bra{x_1} \E^{-\I H t_2}\ket{x_0}\dd x_1\\
    &= \int \bra{x_2}\E^{-\I H( t_2-t_1)}\ket{x_1}\bra{x_1} \E^{-\I H t_1}\ket{x_0}\dd x_1\\
}
\end{equation}

\autoref{PIntQM_eq1} 数学上成立,但它有什么物理意义呢?

每个$\bra{x_2}\E^{-\I H( t_2-t_1)}\ket{x_1}\bra{x_1} \E^{-\I H t_1}\ket{x_0}$表达的是,粒子从$x_0$开始,$t_1$后出现在$x_1$的振幅,乘以从$x_1$开始,再过$t_2-t_1$后出现在$x_2$的振幅.而\autoref{PIntQM_eq1} 是对这个表达式关于$x_1$遍历全空间求积分.

综上,\autoref{PIntQM_eq1} 表达的是:求粒子从$x_0$出发、经过$t_2$后出现在$x_2$的概率振幅,等于先求出粒子经过$t_1$后传播到$x_1$后再从$x_1$经过$t_2-t_1$后传播到$x_2$的振幅,然后把所有可能的$x_1$遍历一遍,把得到的所有\textbf{路径}的振幅求积分.这个过程如\autoref{PIntQM_fig1} 所示:

\begin{figure}[ht]
\centering
\includegraphics[width=8cm]{./figures/PIntQM_1.pdf}
\caption{路径积分的示意图.如图,计算粒子从$x_0$经过时间$t_2$后传播到$x_2$的概率振幅,相当于图中各路径的振幅关于$x_1$遍历整个空间求积分.也就是说,$x_1$取遍所有可能性,得到类似图中三条路径的所有路径,所有这些路径的振幅积分,即为所求.} \label{PIntQM_fig1}
\end{figure}

积分表达式为:
\begin{equation}
\ali{
    \bra{x_2}\E^{-\I H t_2}\ket{x_0} &= \int \bra{x_2}\E^{-\I H( \Delta t_2)}\ket{x_1}\bra{x_1} \E^{-\I H \Delta t_1}\ket{x_0}\dd x_1\\
}
\end{equation}

同样地,我们可以把时间分成多段,产生更多的路径可能性,则我们所求的振幅$\bra{x_n}\mathcal{U}(t)\ket{x_1}$同样是所有这些可能路径的振幅之积分,如\autoref{PIntQM_fig2} 所示.

\begin{figure}[ht]
\centering
\includegraphics[width=8cm]{./figures/PIntQM_2.pdf}
\caption{将时间分成多段后,得到更多可能路径.} \label{PIntQM_fig2}
\end{figure}

最后,当我们给时间所分的段数趋于无穷时,能得到所有可能的路径,如\autoref{PIntQM_fig3} 所示.同样地,所有这些路径的振幅之积分就是


\begin{figure}[ht]
\centering
\includegraphics[width=8cm]{./figures/PIntQM_3.pdf}
\caption{全体可能的路径.} \label{PIntQM_fig3}
\end{figure}














