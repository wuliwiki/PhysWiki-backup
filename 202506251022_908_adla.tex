% 阿德里安-马里·勒让德(综述)
% license CCBYSA3
% type Wiki

本文根据 CC-BY-SA 协议转载翻译自维基百科\href{https://en.wikipedia.org/wiki/Adrien-Marie_Legendre}{相关文章}。

\begin{figure}[ht]
\centering
\includegraphics[width=6cm]{./figures/ce46ca8eff284986.png}
\caption{} \label{fig_adla_1}
\end{figure}
阿德里安-马里·勒让德(Adrien-Marie Legendre,/ləˈʒɑːndər, -ˈʒɑːnd/\(^\text{[2]}\);法语发音:[adʁiɛ̃ maʁi ləʒɑ̃dʁ];1752年9月18日-1833年1月9日)是一位法国数学家,对数学做出了众多贡献。以他命名的重要概念包括勒让德多项式和勒让德变换。他还因在最小二乘法上的贡献而著称,尽管卡尔·弗里德里希·高斯在他之前已发现这一方法,勒让德却是第一个正式发表该方法的人。\(^\text{[3][4]}\)
\subsection{生平}
阿德里安-马里·勒让德于1752年9月18日出生在巴黎的一个富裕家庭。他在巴黎马扎兰学院接受教育,并于1770年在物理与数学方面完成了论文答辩。1775年至1780年间,他在巴黎的军事学院任教,1795年起又在国立高等师范学院任教。同时,他还隶属于法国经度局。1782年,柏林科学院因其关于阻力介质中抛体运动的论文授予勒让德奖项,这篇论文也使他引起了拉格朗日的注意。\(^\text{[5]}\)

法国科学院于1783年任命勒让德为副成员,1785年成为正式院士。1789年,他被选为英国皇家学会会士。\(^\text{[6]}\)

他参与了英法联合测量项目(Anglo-French Survey, 1784–1790),该项目旨在通过三角测量法精确计算巴黎天文台与格林尼治天文台之间的距离。为此,他于1787年与多米尼克·卡西尼伯爵和皮埃尔·梅尚一同前往多佛和伦敦。他们三人还拜访了天王星的发现者威廉·赫歇尔。

勒让德在1793年法国大革命期间失去了他的私人财产。同年,他与玛格丽特-克洛迪娜·库安结婚,后者帮助他理顺了财务事务。1795年,勒让德成为重组后的法国科学院(当时改名为“国家科学与艺术研究院”Institut National des Sciences et des Arts)数学部的六位成员之一。1803年,拿破仑对该研究院进行了重组,勒让德成为几何学部的成员。

1799年至1812年间,勒让德担任巴黎军事学院毕业炮兵生的数学考官;1799年至1815年,他担任巴黎综合理工学院常任数学考官。\(^\text{[7]}\)1824年,因他拒绝在国家研究院投票支持政府候选人,勒让德失去了来自军事学院的退休金。1831年,他被授予荣誉军团勋章军官级别。\(^\text{[5]}\)
\begin{figure}[ht]
\centering
\includegraphics[width=6cm]{./figures/7f9e43d88374097d.png}
\caption{} \label{fig_adla_2}
\end{figure}