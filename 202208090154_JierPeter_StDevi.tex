% 标准差与方差
% 期望


\addTODO{本词条处于草稿阶段.}

%如果本词条定位是高中,那么建议不使用求和符号$\sum$.我没有细写,只是顺手写了一些别的词条要用到的知识.by Jier

标准差和方差用于衡量一组数据的\textbf{离散程度}.

直观地,如果所有数据都是相等的,我们就认为这组数据没有一点离散;但若数据平均值不变,各数据离平均值越远,我们就认为数据离散程度越高.因此,我们可以给出衡量离散程度的式子:

\begin{definition}{方差}
设有一组数据$\{x_i\}_{i=1}^n$,它们的平均值是$\mu=\frac{x_1+x_2+\cdots+x_n}{n}$.则定义这组数据的\textbf{方差}为
\begin{equation}
\sigma^2 = \frac{(x_1-\mu)^2+(x_2-\mu)^2+\cdots+(x_n-\mu)^2}{n}
\end{equation}
\end{definition}

定义方差时使用平方,是为了避免向不同方向离开平均值的数据相互抵消.

\begin{definition}{总体标准差}
设有一组数据$\{x_i\}_{i=1}^n$,它们的平均值是$\mu=\frac{x_1+x_2+\cdots+x_n}{n}$.则定义这组数据的\textbf{总体标准差}为
\begin{equation}
\sigma = \sqrt{\frac{(x_1-\mu)^2+(x_2-\mu)^2+\cdots+(x_n-\mu)^2}{n}}
\end{equation}
\end{definition}

\begin{definition}{标准误差}
数据的\textbf{标准误差}定义为$\sigma_n=\sigma/\sqrt{n}$.
\end{definition}



%解释一下为什么会有样本标准差.


\begin{definition}{样本标准差}
设有一组数据$\{x_i\}_{i=1}^n$,它们的平均值是$\mu=\frac{x_1+x_2+\cdots+x_n}{n}$.则定义这组数据的\textbf{样本标准差}为
\begin{equation}
S = \sqrt{\frac{(x_1-\mu)^2+(x_2-\mu)^2+\cdots+(x_n-\mu)^2}{n-1}}
\end{equation}
\end{definition}























