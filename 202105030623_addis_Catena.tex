% 悬链线

\footnote{参考 Wikipedia \href{https://en.wikipedia.org/wiki/Catenary}{相关页面}.}\textbf{悬链线(catenary)}

\begin{figure}[ht]
\centering
\includegraphics[width=7cm]{./figures/Catena_1.pdf}
\caption{悬链线} \label{Catena_fig1}
\end{figure}

微分方程, 假设原点处的张力为 $T$, 绳的线密度为 $\lambda$, 那么区间 $[0, x]$ 的曲线长度为
\begin{equation}
L(x) = \int_0^x \sqrt{1 + y'(x')^2} \dd{x'}
\end{equation}
区间 $[0, x]$ 所受重力为 $G = \lambda L g$. 根据受力分析, $x$ 点的斜率为 $[0, x]$ 所受重力除以水平拉力 $T$, 这样就得到了悬链线 $y(x)$ 的微分—积分方程
\begin{equation}
y' = \frac{g\lambda}{T} \int_0^x \sqrt{1 + y'^2} \dd{x'}
\end{equation}
两边对 $x$ 再次求导得二阶微分方程
\addTODO{如何对积分上限求导?放链接}
\begin{equation}
y'' = \frac{g\lambda}{T} \sqrt{1 + y'^2}
\end{equation}
注意这是一个非线性二阶常微分方程. 可以验证双曲余弦(\autoref{TrigH_eq1}~\upref{TrigH}) $A\cosh(kx)$ 就是它的通解.
