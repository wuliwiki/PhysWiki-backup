% 球坐标和柱坐标中的定态薛定谔方程
% 球坐标|柱坐标|定态薛定谔方程|拉普拉斯算子|径向动量

\pentry{定态薛定谔方程\upref{SchEq}, 球坐标中的拉普拉斯方程\upref{SphLap}}

\subsection{球坐标}

使用球坐标的拉普拉斯算子\autoref{SphNab_eq4}~\upref{SphNab} 可以将单粒子的哈密顿算符表示为
\begin{equation}
H = -\frac{\hbar^2}{2m} \laplacian + V(\bvec r) =  K_r + \frac{L^2}{2mr^2} + V(\bvec r)
\end{equation}
其中径向动量算符为
\begin{equation}
K_r =-\frac{\hbar^2}{2m} \laplacian_r =  - \frac{\hbar^2}{2m} \qty(\pdv[2]{r} + \frac2r \pdv{r}) = -\frac{\hbar^2}{2m}\frac{1}{r^2} \dv{r} \qty(r^2\pdv{r})
\end{equation}
角动量平方算符为%未完成: 在 wikipedia 上, \bvec L 也是有极坐标的表达式的. 研究一下. https://en.wikipedia.org/wiki/Angular_momentum_operator
\begin{equation}\label{RadSE_eq3}
L^2 = -\hbar^2 r^2 \laplacian_\Omega = -\hbar^2 \qty[ \frac{1}{\sin\theta} \pdv{\theta} \qty(\sin \theta \pdv{u}{\theta}) + \frac{1}{\sin^2 \theta} \pdv[2]{u}{\phi}]
\end{equation}
注意角动量算符不含 $r$.

定态薛定谔方程为
\begin{equation}
\qty(K_r + \frac{L^2}{2mr^2} + V - E)\Psi(\bvec r) = 0
\end{equation}
我们假设势能函数只与粒子到原点的距离有关, 即 $V = V(r)$. 两边乘以 $r^2$ 可以将 $r$ 与角向变量 $\theta, \phi$ (简写为 $\uvec r$)分离, 令 $\Psi = R(r)Y(\uvec r)$.

解得 $Y(\uvec r)$ 为球谐函数 $Y_{l,m}(\uvec r)$ 满足
\begin{equation}
L^2 Y_{l,m}(\uvec r) = \hbar^2 l(l+1) Y_{l,m}(\uvec r)
\end{equation}
分离变量后 $R(r)$ 满足的方程一般被称为\textbf{径向薛定谔方程}
\begin{equation}\label{RadSE_eq6}
K_r R_l(r) + \qty[V(r) + \frac{\hbar^2 l(l+1)}{2mr^2} ]R_l(r) = ER(r)
\end{equation}
我们可以通过变量替换将其化为更简洁的形式.

定义 \textbf{Scaled Radial Wave Function} 为
\begin{equation}
u_l(r) = r R_l(r)
\end{equation}
代入\autoref{RadSE_eq6}, 第一项变为
\begin{equation}
K_r R_l =  - \frac{\hbar^2}{2m} \qty(\dv[2]{R_l}{r} + \frac2r \dv{R_l}{r}) = -\frac{\hbar^2}{2m}\frac{1}{r^2} \dv{r} \qty(r^2\dv{R_l}{r}) =  - \frac{\hbar^2}{2m}\frac1r \dv[2]{u_l}{r}
\end{equation}
所以\autoref{RadSE_eq6} 化简为
\begin{equation}\label{RadSE_eq1}
-\frac{\hbar^2}{2m} \dv[2]{u}{r} + \qty[V(r) + \frac{\hbar^2}{2m}\frac{l(l + 1)}{r^2}]u = Eu
\end{equation}
这就是径向方程, 方括号中可以看作\textbf{一维等效势能}(类比经典力学的情况\autoref{CenFrc_eq6}~\upref{CenFrc}). 该方程的解取决于 $V(r)$ 的具体形式.

% 未完成, 讨论一下束缚态和连续态
% 令束缚态的解为 $u_{n,l}(r)$, 对应能量(本征值)为 $E_n$

总波函数体积分要求
\begin{equation}
\int \abs{R}^2 \abs{Y}^2 r^2 \dd{\Omega}\dd{r}  = 1
\end{equation}
球谐函数已经满足 $\int \abs{Y}^2 \dd{\Omega} = 1$,  所以, 要求
\begin{equation}
\int \abs{R}^2 r^2 \dd{r}  = 1
\end{equation}
正交条件类似. 根据定义
\begin{equation}
\int \abs{u}^2 \dd{r}  = 1
\end{equation}

\subsection{柱坐标}
\begin{equation}
u(r) = \sqrt r R(r)
\end{equation}
\begin{equation}
 H = K_r + \frac{L_z^2}{2m r^2}
\end{equation}
\begin{equation}
K_r R = -\frac{\hbar^2}{2m} \frac1r \dv{r} \qty(r \dv{R}{r}) =  - \frac{\hbar^2}{2m} \frac{1}{\sqrt r} \qty(\dv[2]{u}{r} + \frac{u}{4 r^2})
\end{equation}
\begin{equation}
\frac{L_z^2}{2m r^2}\psi  = \frac{\hbar^2}{2m} \frac{m_z^2}{r^2}\psi 
\end{equation}
所以径向方程为
\begin{equation}
- \frac{\hbar^2}{2m} \dv[2]{u}{r} + \qty[V(r) + \frac{\hbar^2}{2m} \qty(\frac{m_z^2}{r^2} - \frac{1}{4 r^2})]u = Eu
\end{equation}

