% 混淆矩阵
% keys 分类|机器学习|统计学
% license Xiao
% type Tutor

\pentry{分类\nref{nod_Class}}{nod_554a}

在机器学习中,\textbf{混淆矩阵}(Confusion Matrix)是一种通过表格可视化的方式呈现分类模型性能的常用工具,能够显示出模型预测值与实际标签之间的对应关系。顾名思义,混淆矩阵能够方便地看出模型是否将两个不同的类混淆了(比如把一个类错误地判定为另一类)以及混淆的数量有多少。

对于二分类问题而言,混淆矩阵包含两行、两列,一共四个单元格。列(行)分别表示分类器预测的值,行(列)分别表示实际的值。要弄清楚混淆矩阵,首先必须了解以下基本概念。

\begin{enumerate}
\item True Positive(TP):真正类。样本的真实类别是正类,并且模型也将其判定为正类。
\item False Negative(FN):假负类。样本的真实类别是正类,但模型将其判定为负类。
\item False Positive(FP):假正类。样本的真实类别是负类,但模型将其判定为正类。
\item True Negative(TN):真负类。样本的真实类别是负类,并且模型将其判定为负类。 
\end{enumerate}

\begin{table}[ht]
\centering
\caption{混淆矩阵基本模式}\label{tab_ConMat_1}
\begin{tabular}{|c|c|c|}
\hline
 & 预测为阳性 & 预测为阴性 \\
\hline
实际为阳性 & 真阳性的数量 & 假阴性的数量 \\
\hline
实际为阴性 & 假阳性的数量 & 真阴性的数量 \\
\hline
\end{tabular}
\end{table}
举个例子,现在有一个训练好的二元分类器,用于判断给定图片上的动物是马还是羊。假设,有一个图片数据集,一共14张图片,其中9只为羊,5只为马。现在用训练好的分类器来做判断,有可能产生下面的结果。
\begin{table}[ht]
\centering
\caption{混淆矩阵例子}\label{tab_ConMat_2}
\begin{tabular}{|c|c|c|}
\hline
 & 预测为马 & 预测为羊 \\
\hline
实际为马 & 5 & 3 \\
\hline
实际为羊 & 2 & 4 \\
\hline
\end{tabular}
\end{table}
