% 域上的代数
% 代数|环|域|乘法|群

\pentry{矢量空间\upref{LSpace},环\upref{Ring}}

“代数(algebra)”一词,有两个含义。

第一个含义是指一个数学分支,代数学。代数学研究集合中各种各样的运算结构,我们从小学就开始涉及了。一次方程的移项、两边同乘等操作都是代数学研究的性质;理工科到了本科还必须研究线性代数(或称高等代数),作为大量理工学科的数学基础;此外,本部分“抽象代数”则研究了更为基础的一些代数学,但抽象代数本身也只是代数学这门广博学科的起点。

第二个含义,是代数学中研究的一种数学对象,代数。简单来说,代数就是一种定义了向量乘法的向量空间。当然,乘法的性质要具体讨论。



\begin{definition}{狭义的代数定义}\label{AlgFie_def1}
设 $A$ 是域 $\mathbb{K}$ 上的向量空间,若在 $A$ 中再定义代数乘法 $\times$,使得 $(A,+,\times)$ 成为环,并且 $\forall a\in \mathbb{K}, \bvec{u}, \bvec{v}\in A$ 有
\begin{equation}
a(\bvec{u} \times \bvec{v})=(a \bvec{u}) \times \bvec{v}=\bvec{u}{\times}(a \bvec{v})
\end{equation}
则称 $A$ 为\textbf{域 $\mathbb{K}$ 上的代数},简称\textbf{(结合)代数}。

\end{definition}

如果没有特别说明,一个代数一般是指狭义的结合代数,也就是说,是在向量空间里定义向量的乘法,使得它还能构成一个环。当然,乘法只是一种运算,它只要求对加法满足\textbf{分配律}\footnote{分配律有时也被表示为线性性,在\autoref{AlgFie_def1} 中由于我们要求 $(A, +, \times)$ 成环,故已经默认了具有分配律。},不一定是结合的。如果所定义的乘法不是结合的,那么我们称这样的结构是一个\textbf{非结合代数}。


\subsection{代数的例子}
下面我们来看几个例子,加深对代数这一概念的理解。

\subsubsection{结合代数}

\begin{example}{矩阵代数}
域 $\mathbb{K}$ 上的 $n\times n$ 的矩阵的全体 $\mathrm{GL}(n, \mathbb{K})$ (\autoref{Group_ex5}~\upref{Group})在矩阵的加法,标量与矩阵的相乘运算下构成一个线性空间,在矩阵的加法和矩阵的乘法运算下构成一个环,因此它是一个结合代数。
\end{example}

\begin{example}{}
对于域 $\mathbb{K} $ 及有限群 $G=\{g_1,g_2,\cdots,g_n\}$,我们可构成\textbf{群代数}$\displaystyle A(G)=\left\{u | u=\sum a^{i} g_{i}, a^{i} \in \mathbb{K}\right\}$,其中加法为
\begin{equation}
u+v=\sum a^{i} g_{i}+\sum b^{i} g_{i}=\sum\left(a^{i}+b^{i}\right) g_{i}~,
\end{equation}
数乘为
\begin{equation}\label{AlgFie_eq2}
a u=\sum\left(a a^{i}\right) g_{i}, a \in \mathbb{K}
\end{equation}
而代数乘法为
\begin{equation}
u \times v=\left(\sum a^{i} g_{i}\right) \times\left(\sum b^{i} g_{i}\right)=\sum_{i, j} a^{i} b^{j}\left(g_{i} g_{j}\right)
\end{equation}
这是一个结合代数。
\end{example}

\subsubsection{李代数}

\begin{example}{三维实线性李代数}
在 $3$ 维实向量空间中,以两个向量的叉积 $\mathbf A\times \mathbf B$ 来定义它们的代数乘法运算,则它们构成一个代数,此时有
\begin{equation}
\begin{aligned}
&\mathbf{A} \times \mathbf{A}=0\\
&\mathbf{A} \times \mathbf{B}=-\mathbf{B} \times \mathbf{A}\\
&\mathbf{A} \times(\mathbf{B}+\mathbf{C})=\mathbf{A} \times \mathbf{B}+\mathbf{A} \times \mathbf{C}\\
&(\bvec B+\mathbf{C}) \times \mathbf{A}=\mathbf{B} \times \mathbf{A}+\mathbf{C} \times \mathbf{A}
\end{aligned}
\end{equation}
以及
\begin{equation} \label{AlgFie_eq1}
(\mathbf{A} \times \mathbf{B}) \times \mathbf{C}+(\mathbf{B} \times \mathbf{C}) \times \mathbf{A}+(\mathbf{C} \times \mathbf{A}) \times \mathbf{B}=0
\end{equation}
$3$ 维向量空间配上向量外积作为乘法,得到一个非结合代数,三维实线性李代数。
\end{example}

通常把\autoref{AlgFie_eq2}  称为\textbf{雅可比恒等式}。 具有这种性质的代数称为\textbf{李代数(Lie Algebra)}。我们会在将来详细讨论李代数。

\begin{example}{}
$\mathrm{gl}(n, \mathbb C)$ 除矩阵的加法及数乘外,再定义代数乘法 $\mathbf A\times \mathbf B=[\mathbf A, \mathbf B] = \mathbf A \mathbf B - \mathbf B\mathbf A, \mathbf A, \mathbf B\in \mathrm{gl}(n, \mathbb C)$。显然 $\mathrm{gl}(n,\mathbb C)$ 构成一个李代数。
\end{example}


下面一个例子需要一些分析力学中的内容。如果不知道分析力学也没关系,知道即可。
\begin{example}{}
在分析力学中,正则变量记为 $p_i,q_i, i=1,2,\cdots, s$。函数 $u(p,q)$ 和 $v(p, q)$ 的\textbf{泊松括号}定义为\begin{equation}
\{u, v\}=\sum_{i=1}^{s}\left(\frac{\partial u}{\partial q_{i}} \frac{\partial v}{\partial p_{i}}-\frac{\partial u}{\partial p_{i}} \frac{\partial v}{\partial q_{i}}\right)
\end{equation}
于是我们得到了一个李代数。

\addTODO{也许可以引用李代数中使用李代数反推出结合代数的方法来描述泊松括号对应的结合代数。}



\end{example}

\subsubsection{外代数}
\pentry{直和\upref{DirSum}}

\begin{example}{外代数}\label{AlgFie_ex1}
给定域 $\mathbb{K}$ 上的 $n$ 维线性空间 $V$,任取其一组基 $\{\bvec{e}_i\}^n_{1}$。我们进行以下构造:
\begin{itemize}
\item 对于每两个元素 $\bvec{v}, \bvec{u}\in V$,我们定义一个新的元素 $\bvec{v}\wedge \bvec{u}$,只要求它满足\textbf{线性性}和\textbf{反对称性}\footnote{线性性即 $\forall a_1\bvec{v}_1+a_2\bvec{v}_2,{u}\in V$,必有 $(a_1\bvec{v}_1+a_2\bvec{v}_2)\wedge\bvec{u}=a_1\bvec{v}_1\wedge\bvec{u}+a_2\bvec{v}_1\wedge\bvec{u}$。反对称性即 $\forall \bvec{v}, \bvec{u}\in V$,有 $\bvec{v}\wedge\bvec{u}=-\bvec{u}\wedge\bvec{v}$。}。这样,对于两个非零且不相等的 $\bvec{v},\bvec{u}\in V$,$\bvec{v}\wedge \bvec{u}$ 是一个不在 $V$ 中的新元素;当 $\bvec{v}=\bvec{u}$ 或其中之一为 $0$ 时,$\bvec{v}\wedge \bvec{u}=0\in V$。由于线性性,我们用集合 $\{\bvec{e}_i\wedge \bvec{e}_j\}_{i\not=j}$ 作为基底,在域 $\mathbb{K}$ 上构造一个新的线性空间,记为 $A^2(V)$。由于该运算得到的元素要么是 $0$,要么是 $V$\textbf{以外}的元素,因此我们把它称为\textbf{外积(exterior product)}。

\item 将以上步骤推广,对于每 $k$ 个元素 $\bvec{v}_i\in V$,我们定义一个新的元素 $\bvec{v}_1\wedge \bvec{v}_2\wedge\cdots\wedge \bvec{v}_k$,要求它满足\textbf{线性性}、\textbf{反对称性}和\textbf{结合性}\footnote{结合性即对于任意三个元素 $\bvec{v}_1,\bvec{v}_2,\bvec{v}_3$,都有 $(\bvec{v}_1\wedge \bvec{v}_2)\wedge \bvec{v}_3=\bvec{v}_1\wedge(\bvec{v}_2\wedge \bvec{v}_3)$,这样就可以摆脱括号而直接将这三个元素的外积表示为 $\bvec{v}_1\wedge \bvec{v}_2\wedge \bvec{v}_3$。}。由于反对称性,$k$ 个 $\{e_i\}$ 的外积中只有 $C^k_n$ 个不为零,用这些不为零的外积作为基底在 $\mathbb{K}$ 上构造一个线性空间,记为 $A^k(V)$。特别地,$V=A^1(V)$。

\item 同样由于反对称性,若干元素 $\bvec{v}_i$ 中只要有两个相等,它们的外积就为零。这就意味着我们最多可以取到 $A^n(V)$,而对于 $m>n$,都有 $A^m(V)=\{0\}$。这样,将各 $A^k(V)$ 视为互相不同的线性空间,其中只有 $n+1$ 个非平凡,它们作为向量空间的直和 $\bigoplus\limits_{k=1}^n A^k(V)$ 也是一个向量空间,记为 $A(V)$。由前面的步骤可知,任取 $\bvec{v}, \bvec{u}\in A(V)$,都有 $\bvec{v}\wedge\bvec{u}\in A(V)$,因此外积在 $A(V)$ 上是\textbf{封闭的}。
\item 将各 $A_k(V)$ 的加法都记为 $+$,它们的直和 $A(V)$ 的加法也记为 $+$,那么用外积作为乘法,$(A(V), +, \wedge)$ 构成一个代数,称为线性空间 $V$ 上的\textbf{外代数(exterio algebra)},又称\textbf{格拉斯曼代数(Grassmann algebra)}。
\end{itemize}

\end{example}

$n$ 维线性空间 $V$ 上的 $k$ 次外代数可用于表示 $V$ 上的 $k$ 维“体积”,而当 $k=n$ 时就是我们所熟悉的行列式:取定线性空间 $V$ 的基 $\{\bvec{e}_i\}$,那么 $n$ 个元素的外积 $\sum\limits_{i}(a_{1i}\bvec{e}_i)\wedge\sum\limits_{i}(a_{2i}\bvec{e}_i)\wedge\cdots\wedge\sum\limits_{i}(a_{ni}\bvec{e}_i)=\vmat{\{a_{ij}\}}\bvec{e}_1\wedge\cdots\wedge\bvec{e}_n$。

外代数最重要的例子是外微分\upref{ExtDer},它是以\textbf{微分形式}的集合构建的线性空间的外代数。
