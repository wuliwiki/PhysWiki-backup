% 约瑟夫·约翰·汤姆逊
% license CCBYSA3
% type Wiki

(本文根据 CC-BY-SA 协议转载自原搜狗科学百科对英文维基百科的翻译)

\begin{figure}[ht]
\centering
\includegraphics[width=6cm]{./figures/f6768fb3594b1005.png}
\caption \label{fig_Joseph_8}
\end{figure}

约瑟夫·约翰·汤姆逊是一名英国的物理学家和诺贝尔物理学奖获得者,他发现并且鉴定了电子, 这是第一个被发现的亚原子粒子。

在1897年,Thomson表明,阴极射线由以前未知的带负电粒子(现称为电子)组成,经过计算,他得出这个粒子一定比原子小得多且具有非常大的电荷质量比。 Thomson在1913年发现了稳定(非放射性)元素同位素存在的第一个证据,这是他对极隧射线 (正离子)的组成的研究的一部分。他和Francis William Aston一起进行的确定带正电粒子性质的实验是质谱分析的首次应用,并导致了质谱仪的发展。

Thomson因在气体导电方面的研究获得1906年诺贝尔物理学奖。

\subsection{教育以及个人生活}
Joseph John Thomson于1856年12月18日出生于英格兰兰开夏郡,曼彻斯特的Cheetham Hill。他的母亲Emma Swindells来自当地的一个纺织世家。他的父亲Joseph James Thomson经营一家由曾祖父创办的古董书店。他有一个比他小两岁的弟弟,Frederick Vernon Thomson。 J. J. Thomson是一个保守而虔诚的英国国教徒.[1][2][3]

他的早期教育是在小型私立学校,在那里他表现出了杰出的天赋和对科学的兴趣。1870年,他在14岁的时候被曼彻斯特(现曼彻斯特大学)的欧文斯学院录取。他的父母原计划让他去机车制造商--Sharp-Stewart & Co公司当见习工程师,但在他父亲于1873年去世后,这些计划被取消了.

1876年,他转到剑桥大学三一学院。1880年,他获得了数学文学学士学位(剑桥优等考试的榜眼[4] 和史密斯奖的第二名)。[5] 1881年,他申请并成为了三一学院的一员。[6] Thomson于1883年获得文学硕士学位(亚当斯奖)。[5]

\subsection{家庭}
1890年,Thomson娶了Rose Elisabeth Paget, 她是Thomson曾经的一个学生,[7]George Edward Paget爵士,KCB的女儿,她是一名医生,以及在小圣玛丽教堂的剑桥大学医学钦定讲座教授。 他们有一个儿子George Paget Thomson,还有一个女儿Joan Paget Thomson。

\subsection{职业生涯与从事的研究}
\subsubsection{3.1 总概}
1884年12月22日,Thomson被任命为剑桥大学卡文迪什物理学教授。[8] 鉴于Osborne Reynolds or Richard Glazebrook等其他候选人的年龄较大且在实验室的工作方面更有经验,所以对Thomson的这项任命让大家都很吃惊。Thomson以其数学家的工作而闻名,在这个领域里他被认为是一位杰出的天才。[8]

他在1906年被授予诺贝尔奖,“以表彰他在气体传导的理论和实验研究方面的巨大成就。” 1908年,他被授予爵士爵位,1912年被授予荣誉勋章。1914年,他在牛津大学举办了关于“原子论”的Romanes讲座。1918年,他成为剑桥三一学院的硕士,直到去世。Joseph John Thomson于1940年8月30日去世;他的骨灰安息在Westminster大教堂, 靠近Isaac Newton爵士和他曾经的学生Ernest Rutherford的坟墓。[9]

Thomson对现代科学的最大贡献之一就是他是一个极具天赋的老师。Ernest Rutherford是他的学生之一,后来接替他成为了卡文迪许的物理教授. 除了Thmoson他自己, 他的研究助理中的其中六位 (Charles Glover Barkla, Niels Bohr, Max Born, William Henry Bragg, Owen Willans Richardson and Charles Thomson Rees Wilson) 获得了诺贝尔物理学奖, 还有两位 (Francis William Aston and Ernest Rutherford)获得了诺贝尔化学奖。此外,由于证明了电子的波状特性,Thomson的儿子(George Paget Thomson)获得1937年诺贝尔物理学奖。

\subsubsection{3.2 早期著作}
Thomson一篇获奖的大师之作,《论涡旋环的运动》,显示出了他早期对于原子结构的兴趣。[10] 在这篇文章里, Thomson用数学方法描述了 William Thomson原子涡旋理论的运动。[8]

Thomson发表了许多涉及电磁学的数学和实验问题的论文。他研究了James Clerk Maxwell的电磁理论,引入带电粒子的电磁质量的概念,并证明了运动的带电体的质量会明显增加。[8]

许多他对于化学过程的数学建模工作可以被认为是早期的计算化学。[8] 在进一步的工作:1888年出版的《动力学在物理和化学中的应用》一书中,Thomson从数学和理论的角度论述了能量的转化,认为所有的能量都可能是动力学的。[8] 他的下一本书《电磁学研究近况》(1893年),是建立在Maxwell关于电磁学的论述之上的,有时还被称为“Maxwell的第三卷”。[10] 在书中,汤姆森强调了物理方法和实验,并囊括了大量的数据和仪器图表,包括一个电流通过气体的传导数据。[8] 他的第三本书《电磁数学理论的基本原理》(1895)[10] 是一本内容广泛、可读性强的入门书,并且作为教科书获得了相当大的欢迎。[8]

Thomson在1896年访问普林斯顿大学时所作的四次讲座,后来被出版为《气体放电》(1897)。1904年Thomson还在耶鲁大学作了六次演讲。[10]

\subsubsection{3.3 电子的发现}
一些科学家,如William Prout and Norman Lockyer,曾提出原子是由一个更基本的单位构成的,但他们设想这个单位是最小的原子氢的大小。Thomson在1897年首次提出这个基本单位比原子小1000倍以上,这表明亚原子粒子的存在(现在被称为电子)。Thomson通过对阴极射线性质的研究发现了这一点。1897年4月30日,Thomson发现阴极射线(当时被称为勒纳德射线)在空气中的传播距离比原子大小的粒子要远得多,于是他提出了这个设想。[11]他通过测量阴极射线击中热结点时产生的热量,并将其与射线的磁偏转进行比较,从而估算出阴极射线的质量。他的实验表明,阴极射线不仅比氢原子轻1000多倍,而且无论阴极射线来自哪种原子,它们的质量都是相同的。他的结论是,这些射线由非常轻的带负电荷的粒子组成,而这些粒子是原子的普遍组成部分。他把这些粒子称为“微粒”,但后来科学家们更喜欢George Johnstone Stoney在1891年提出的电子这个名字,那时汤姆逊还没有真正发现电子。[12]

1897年4月,Thomson只有早期发现的一些迹象表明阴极射线可以被电偏转(之前的研究人员,如Heinrich Hertz,认为阴极射线不可能被电偏转)。在Thomson宣布这个微粒的存在的一个月后,他发现,如果把放电管排空到一个非常低的压力下,他可以通过电场使射线偏转。通过比较阴极射线束在电场和磁场作用下的偏转,他得到了更可靠的质量电荷比测量值,证实了他之前的估量。[13]这成为测量电子荷质比的经典方法。 (直到1909年Robert A. Millikan的滴油实验,电荷自身才被测量出来)

Thomson 认为这些微粒来自于阴极射线管中微量气体的原子。他由此得出结论,原子是可以被分割的,微粒是它们的组成部分。 1904年,汤姆森提出了一个原子模型,假设它是一个由正电荷构成的球体,其中静电力决定了微粒的位置。[8] 为了解释原子的整体中性电荷,他提出微粒分布在均匀的正电荷海洋中。在这个“李子布丁”模型中,电子被视为像李子布丁中的李子一样嵌入正电荷中(尽管在Thomson模型中,它们不是静止的,而是快速地绕轨道运行)。[14][15]

\subsubsection{3.4 同位素与质谱分析}
\begin{figure}[ht]
\centering
\includegraphics[width=6cm]{./figures/e8bdd73ba76432a3.png}
\caption{在这张照相底片的右下角有两种氖同位素的标记:氖-20和氖-22。} \label{fig_Joseph_1}
\end{figure}
在1912年,作为他对正电荷粒子流(当时称作极隧射线)的组成的研究的一部分,Thomson和他的研究助理F. W. Aston通过磁场和电场引导了一股氖离子,并通过在其路径上放置一个照相板来测量其偏转。[16] 他们在照相板上观察到有两束光(见右图),这表明有两种不同的偏折抛物线,并得出结论,氖是由两种不同原子质量的原子(氖-20和氖-22)组成的,也就是说是由两种同位素组成的。[16][17] 这是稳定元素存在同位素的第一个证据;Frederick Soddy先前曾提出同位素的存在来解释某些放射性元素的衰变。

J. J. Thomson通过质量分离氖同位素是质谱分析的第一个例子,后来被F. W. Aston和A. J. Dempster改进并发展成一种通用方法。[8]

\subsubsection{3.5 阴极射线实验}
早些时候,物理学家们争论,阴极射线是像光一样的非物质,或者,引用Thomson的观点,是“实际上完全是物质,并且...标记了带负电的物质粒子的路径”。[13] 以太假说是含糊的,[13] 但粒子假说是明确的,使得Thomson可以去检验。

\textbf{磁偏转}

Thomson首先研究了阴极射线的磁偏转。阴极射线在仪器左侧的侧管中产生,并通过阳极进入主钟形罩,在那里它们被磁铁偏转。Thomson通过玻璃罐中正方形屏幕上的荧光探测到它们的路径。他发现,无论阳极的材料是什么,瓶中的气体是什么,射线的偏转都是相同的,这表明,无论射线的来源是什么,它们的形式都是相同的.[18]

\textbf{电荷}

\begin{figure}[ht]
\centering
\includegraphics[width=6cm]{./figures/5688ff542503218d.png}
\caption{J.J.Thomson通过阴极射线管证明阴极射线可以被磁场偏转,并且它们的负电荷不是一个单独的现象。} \label{fig_Joseph_2}
\end{figure}
虽然以太理论的支持者接受了在Crookes管中产生负电荷粒子的可能性,但他们认为负电荷只是副产品,阴极射线本身是非物质的。Thomson开始研究他是否能把电荷从射线中分离出来。

Thomson在阴极射线的直射路径外构造了一个带有静电计的克鲁克斯管。Thomson可以通过观察射线击中管子表面时产生的磷光斑来追踪射线的路径。Thomson观察到,静电计只有在用磁铁使阴极射线偏转时才会记录电荷。他得出结论,负电荷和射线在本质上是一样的。[11]

\textbf{电偏转}

\begin{figure}[ht]
\centering
\includegraphics[width=10cm]{./figures/dc2bd58d80f6bfb6.png}
\caption{此为Thomson的Crookes管的插图,他用它观察了阴极射线在电场作用下的偏转(后来测量了它们的质量电荷比)。阴极射线从阴极C发射,通过狭缝A(阳极)和B(接地),然后通过板D和E之间产生的电场,最后撞击远端的表面。} \label{fig_Joseph_3}
\end{figure}
在1897年的5月到六月,Thomson研究了这些射线是否会被电场偏转。[16] 以前的实验者没有观察到这一点,但汤姆森认为他们的实验是有缺陷的,因为他们的管子含有太多的气体。

Thomson构建了一个具有更好真空度的Crookes管。在管子的开头是阴极,射线从阴极射出。光线被两个金属缝隙锐化成一束,其中第一个缝隙兼作阳极,第二个缝隙接地。然后,光束在两块平行的铝板之间通过,当它们连接到一块电池时,两块铝板之间会产生电场。管子的末端是一个大球体,在那里光束会撞击玻璃,形成一个发光斑。Thomson在这个球体的表面粘贴了一个刻度来测量光束的偏转。注意,任何电子束都会与Crookes管内的一些残余气体原子碰撞,从而使它们电离并在管内产生电子和离子(空间电荷); 在以前的实验中,这种空间电荷屏蔽了外加电场。然而,在Thomson的Crookes管中,残留原子的密度非常低,以至于电子和离子的空间电荷不足以电屏蔽外部施加的电场,这使得Thomson能够成功地观察到电偏转。
\begin{figure}[ht]
\centering
\includegraphics[width=10cm]{./figures/6202ddb6f03f6d25.png}
\caption{阴极射线(蓝线)被电场(黄色)偏转。} \label{fig_Joseph_4}
\end{figure}
当上极板连接到电池的负极,下极板连接到电池的正极时,发光斑向下移动,当极性反转时,发光斑向上移动。

\textbf{质荷比测量}

\begin{figure}[ht]
\centering
\includegraphics[width=6cm]{./figures/a5cc3fd3176de270.png}
\caption{请添加图片标题} \label{fig_Joseph_5}
\end{figure}
在他的经典实验中,Thomson通过测量阴极射线受磁场偏转的程度,并将其与电偏转进行比较,来测量阴极射线的质量电荷比。 他使用的仪器和之前的实验一样,但是把放电管放在一个大型电磁铁的两极之间了。他发现其质量电荷比比氢离子($H+$)低一千倍以上,这表明这些粒子要么非常轻,要么带电量非常高。[13] 值得注意的是,来自每个阴极的射线产生相同的质量电荷比。这与阳极射线(现在已知由阳极发射的正离子产生)形成对比,在阳极射线中,质量电荷比因阳极而异。汤姆森本人对他的研究成果仍持批评态度,他在诺贝尔奖获奖感言中提到的是“微粒”而不是“电子”。

汤姆逊的计算可以总结如下(注意,我们在这里重现了汤姆逊的原始符号,用$F$代替$E$表示电场,用$H$代替$B$表示磁场):

电偏转由下式给出:$\Theta = Fel/mv^2$其中$\Theta$是角电偏转,$F$是施加的电场强度,$e$是阴极射线粒子的电荷,$l$是电板的长度,$m$是阴极射线粒子的质量,$v$是阴极射线粒子的速度。磁偏转由下式给出:$\phi = Hel/mv$其中$\phi$是角度磁偏转,$H$是施加的磁场强度。

磁场会被变化,直到磁偏转和电偏转相同,即当$\Theta = \phi$时,$Fel/mv^2 = Hel/mv$这可以简化为$m/e = H^2 l/F \Theta$单独测量电偏转以得到$\Theta$,并且已知$H$,$F$和$l$,因此可以计算$m / e$。

\textbf{结论}

当阴极射线带着负电荷,被静电力偏转,好像它们本身就是负电荷一样,被磁力作用,就像这个力作用在沿着这些射线运动的负电荷物体上一样,我认为它们是由物质粒子携带的负电荷。

——J. J. Thomson [13]

至于这些粒子的来源,Thomson认为它们来自阴极附近的气体分子。

如果,在阴极附近非常强烈的电场中,气体分子被解离并被分裂,而不是分解成普通的化学原子,而是分裂成这些原始原子,为简洁起见,我们将其称为小体; 如果这些微粒充电并通过电场从阴极投射,它们的行为就会和阴极射线完全一样。

——J. J. Thomson [19]

Thomson认为原子是由这些在正电荷海洋中运行的微粒组成的,这就是他的李子布丁模型。后来,当他的学生Ernest Rutherford证明正电荷集中在原子核中时,这个模型被证明是不正确的。

\subsubsection{3.6 其他成就}
1905年,Thomson发现了钾的天然放射性.[20]

1906年,Thomson证明氢原子每个只有一个电子。以前的理论允许原子内含有不同数量的电子。[21][22]

\subsubsection{3.7 奖项和荣誉}
\begin{figure}[ht]
\centering
\includegraphics[width=6cm]{./figures/7113a25bd2f7e702.png}
\caption{剑桥旧卡文迪什实验室外,纪念J. J. Thomson发现电子的牌匾。} \label{fig_Joseph_7}
\end{figure}
Thomson于1884年当选为 皇家学会会员 (FRS)[23][23]并且被任命为剑桥大学,卡文迪许实验室的实验物理学的卡文迪许教授。[8] Thomson在其职业生涯中赢得了无数奖项和荣誉,包括:
\begin{itemize}
\item 亚当斯奖 (1882)
\item 家奖章 (1894)
\item 伦敦皇家协会休斯奖 (1902)
\item 霍奇金斯奖章 (1902)
\item 诺贝尔物理学奖 (1906)
\item Elliott Cresson Medal (1910)
\item 科普利奖章 (1914)
\item 富兰克林奖章 (1922)
\end{itemize}
汤姆森于1884年6月12日当选为皇家协会会员[23],并于1915年至1920年担任皇家学会会长。

\textbf{遗赠荣誉}

1991年,为了纪念他,thomson(符号:Th)被提议作为质谱中测量质荷比的单位。[24]

J J Thomson大道位于剑桥大学西剑桥校区,以Thomson的名字命名。[25]

1927年11月,J. J. Thomson在剑桥 莱斯学校开设了以他的名字命名的汤姆森大楼。[26]

\subsection{参考文献}
[1]
^Peter J. Bowler, Reconciling Science and Religion: The Debate in Early-Twentieth-Century Britain (2014). University of Chicago Press. p. 35. ISBN 9780226068596. "Both Lord Rayleigh and J. J. Thomson were Anglicans.".

[2]
^Seeger, Raymond. 1986. "J. J. Thomson, Anglican," in Perspectives on Science and Christian Faith, 38 (June 1986): 131-132. The Journal of the American Scientific Affiliation. ""As a Professor, J.J. Thomson did attend the Sunday evening college chapel service, and as Master, the morning service. He was a regular communicant in the Anglican Church. In addition, he showed an active interest in the Trinity Mission at Camberwell. With respect to his private devotional life, J.J. Thomson would invariably practice kneeling for daily prayer, and read his Bible before retiring each night. He truly was a practicing Christian!" (Raymond Seeger 1986, 132).".

[3]
^Richardson, Owen. 1970. "Joseph J. Thomson," in The Dictionary of National Biography, 1931-1940. L. G. Wickham Legg - editor. Oxford University Press..

[4]
^Grayson, Mike (May 22, 2013). "The Early Life of J.J. Thomson: Computational Chemistry and Gas Discharge Experiments". Profiles in Chemistry. Chemical Heritage Foundation. Retrieved 11 February 2015..

[5]
^"Thomson, Joseph John (THN876JJ)". A Cambridge Alumni Database. University of Cambridge..

[6]
^The Victoria University Calendar for the Session 1881-2. 1882. p. 184. Retrieved 11 February 2015. [缺少ISBN].

[7]
^The Biographical Dictionary of Women in Science: L-Z by By Marilyn Bailey Ogilvie and Joy Dorothy Harvey, Taylor & Francis, p.972.

[8]
^"Joseph John "J. J." Thomson". Science History Institute. June 2016. Retrieved 20 March 2018..

[9]
^Westminster Abbey. "Sir Joseph John Thomson"..

[10]
^"J.J. Thomson - Biographical". The Nobel Prize in Physics 1906. The Nobel Foundation. Retrieved 11 February 2015..

[11]
^Thomson, J.J. (1897). "Cathode Rays". The Electrician. 39: 104..

[12]
^Falconer, Isobel (2001). "Corpuscles to electrons" (PDF). In Buchwald, J. Z.; Warwick, A. Histories of the Electron. MIT Press. p. 77–100. ISBN 9780262024945..

[13]
^Thomson, J. J. (7 August 1897). "Cathode Rays". Philosophical Magazine. 5. 44 (269): 293. doi:10.1080/14786449708621070. Retrieved 4 August 2014..

[14]
^Mellor, Joseph William (1917), Modern Inorganic Chemistry, Longmans, Green and Company, p. 868, According to J. J. Thomson's hypothesis, atoms are built of systems of rotating rings of electrons..

[15]
^Dahl(1997), p. 324: "Thomson's model, then, consisted of a uniformly charged sphere of positive electricity (the pudding), with discrete corpuscles (the plums) rotating about the center in circular orbits, whose total charge was equal and opposite to the positive charge.".

[16]
^Davis & Falconer, J.J. Thomson and the Discovery of the Electron.

[17]
^J.J. Thomson (1913) "Rays of positive electricity," Proceedings of the Royal Society A, 89: 1–20..

[18]
^Thomson, J. J. (8 February 1897). "On the cathode rays". Proceedings of the Cambridge Philosophical Society. 9: 243..

[19]
^Thomson, J. J. (1897). "Cathode rays". Philosophical Magazine. 44: 293..

[20]
^Thomson, J. J. (1905). "On the emission of negative corpuscles by the alkali metals". Philosophical Magazine. Series 6. 10 (59): 584–590. doi:10.1080/14786440509463405..

[21]
^Hellemans, Alexander; Bunch, Bryan (1988). The Timetables of Science. Simon & Schuster. p. 411. ISBN 0671621300..

[22]
^Thomson, J. J. (June 1906). "On the Number of Corpuscles in an Atom". Philosophical Magazine. 11 (66): 769–781. doi:10.1080/14786440609463496. Archived from the original on 19 December 2007. Retrieved 4 October 2008..

[23]
^Rayleigh (1941). "Joseph John Thomson. 1856-1940". Obituary Notices of Fellows of the Royal Society. 3 (10): 586–609. doi:10.1098/rsbm.1941.0024..

[24]
^Cooks, R. G.; A. L. Rockwood (1991). "The 'Thomson'. A suggested unit for mass spectroscopists". Rapid Communications in Mass Spectrometry. 5 (2): 93..

[25]
^"Cambridge Physicist is streets ahead". 2002-07-18. Retrieved 2014-07-31..

[26]
^"Opening of the New Science Building: Thomson". 2005-12-01. Archived from the original on 2015-01-11. Retrieved 2015-01-10..