% 连续统假设(综述)
% license CCBYSA3
% type Wiki

本文根据 CC-BY-SA 协议转载翻译自维基百科\href{https://en.wikipedia.org/wiki/Continuum_hypothesis}{相关文章}。

在数学中,特别是集合论中,连续统假设(缩写为CH)是一个关于无限集合可能大小的假设。它陈述了:

没有一个集合,其基数严格介于整数和实数之间。

或者等价地:

实数的任何子集要么是有限的,要么是可数无限的,要么具有实数的基数。

在包含选择公理的泽梅洛–弗兰克尔集合论(ZFC)中,这与以下的阿列夫数方程等价:\()
2^{\aleph_0} = \aleph_1\)或者用贝斯数表示更简洁地写作:\(\beth_1 = \aleph_1\)

连续统假设由格奥尔格·康托尔于1878年提出,\(^\text{[1]}\)并且确定其真伪是1900年希尔伯特提出的23个问题中的第一个。这个问题的答案与ZFC独立,因此可以将连续统假设或其否定作为公理加入到ZFC集合论中,且如果ZFC是一致的,则所得到的理论是一致的。1963年,保罗·科恩证明了这一独立性,补充了1940年库尔特·哥德尔的早期工作。\(^\text{[2]}\)

该假设的名称来源于实数的“连续统”一词。