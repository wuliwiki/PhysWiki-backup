% 科学计数法、数量级(高中)
% license Xiao
% type Tutor

\begin{issues}
\issueDraft
\end{issues}

\subsection{科学计数法}
\textbf{科学计数法}通常表示为
\begin{equation}\label{eq_OrdMag_1}
x \e{n}~.
\end{equation}
其中 $x$ 是一个小数, 满足 $1 \leqslant \abs{x} < 10$。 $n$ 是一个整数($n=0$ 时 $10^{0} = 1$ 可以省略不写)。

例如 $1.23\e{4} = 12300$, $1.23\e{-4} = 0.000123$。

\autoref{eq_OrdMag_1} 中的 $n$ 可以看作 $x$ 的小数点需要移动的位数。 $n > 0$ 则向右移动, $n < 0$ 则向左移动, $n=0$ 则不移动($10^{0} = 1$)。

小技巧:若保证 $1 < \abs{x} < 10$, $x\e{-n}$ 前面一共有 $n$ 个零。

在计算机领域中, $\times 10^\square$ 一般简单表示为 \verb`e` (

\subsection{数量级}
\footnote{参考 Wikipedia \href{https://en.wikipedia.org/wiki/Order_of_magnitude}{相关页面}。}数量级。
