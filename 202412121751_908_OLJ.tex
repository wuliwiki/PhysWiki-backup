% 欧拉角(综述)
% license CCBYSA3
% type Wiki

本文根据 CC-BY-SA 协议转载翻译自维基百科\href{https://en.wikipedia.org/wiki/Euler_angles}{相关文章}。
\begin{figure}[ht]
\centering
\includegraphics[width=8cm]{./figures/f7361f2b5f088c22.png}
\caption{} \label{fig_OLJ_1}
\end{figure}
欧拉角是由莱昂哈德·欧拉提出的三个角度,用于描述刚体相对于固定坐标系的方向。[1]

它们也可以表示物理学中运动参考系的方向,或三维线性代数中一般基的方向。

经典欧拉角通常采用倾斜角度的方式,其中零度表示垂直方向。后来,由彼得·古思里·泰特(Peter Guthrie Tait)和乔治·H·布赖恩(George H. Bryan)提出了替代形式,主要用于航空学和工程学中,其中零度表示水平位置。
\subsection{链式旋转等价性}
欧拉角可以通过元素几何或旋转组合(即链式旋转)来定义。几何定义表明,三个元素旋转(绕坐标系的轴旋转)总是足够将物体定向到任何目标参考系。

这三个元素旋转可以是外在旋转(绕原始坐标系xyz轴旋转,假设坐标系保持静止),也可以是内在旋转(绕旋转坐标系XYZ轴旋转,该坐标系与运动体固连,在每次元素旋转后,物体相对于外部参考系的方向会发生变化)。

在下面的各节中,带有撇号标记的轴(例如,z″)表示元素旋转后的新轴。

欧拉角通常用 α、β、γ 或 ψ、θ、φ 来表示。不同的作者可能会使用不同的旋转轴集来定义欧拉角,或者使用不同的名称来表示相同的角度。因此,任何涉及欧拉角的讨论都应该首先明确它们的定义。

在不考虑使用两种不同约定来定义旋转轴(内在或外在)的情况下,旋转轴有十二种可能的旋转顺序,可以分为两组:

\begin{itemize}
\item \textbf{正确的欧拉角}(\(z-x-z,x-y-x,y-z-y,z-y-z,x-z-x,y-x-y\))
\item \textbf{泰特-布赖恩角}(\(x-y-z,y-z-x,z-x-y,x-z-y,z-y-x,y-x-z\))。
\end{itemize}

泰特-布赖恩角也被称为卡尔丹角、航海角、航向、仰角和倾斜角,或偏航、俯仰和滚转角。有时,这两类旋转顺序都被称为“欧拉角”。在这种情况下,第一组旋转顺序被称为正确的或经典的欧拉角。
\subsection{经典欧拉角} 
欧拉角是瑞士数学家莱昂哈德·欧拉(1707–1783)引入的三个角度,用于描述刚体相对于固定坐标系统的方向。
\subsubsection{几何定义}  
\begin{figure}[ht]
\centering
\includegraphics[width=10cm]{./figures/1c8b217760d79068.png}
\caption{左:一个万向架组合,展示了 z-x-z 旋转序列。外部坐标系显示在底座中,内部坐标轴以红色表示。右:一个简单的图示,展示了类似的欧拉角。} \label{fig_OLJ_2}
\end{figure}
原始坐标系的轴表示为 \(x\)、\(y\)、\(z\),旋转后坐标系的轴表示为 \(X\)、\(Y\)、\(Z\)。几何定义(有时称为静态定义)首先定义节点线(N)为平面 \(xy\) 和 \(XY\) 的交线(也可以定义为轴 \(z\) 和 \(Z\) 的公垂线,然后表示为向量积 \(N = z \times Z\))。基于这个定义,三个欧拉角可以如下定义:

\(\alpha\)(或 \(\varphi\))是 \(x\) 轴与 \(N\) 轴之间的带符号角度(\(x\)-惯例——也可以定义为 \(y\) 轴与 \(N\) 轴之间的角度,称为 \(y\)-惯例)。  
\(\beta\)(或 \(\theta\))是 \(z\) 轴与 \(Z\) 轴之间的角度。  
\(\gamma\)(或 \(\psi\))是 \(N\) 轴与 \(X\) 轴之间的带符号角度(\(x\)-惯例)。  

只有当两个参考系具有相同的手性时,才能定义这两个参考系之间的欧拉角。

\subsubsection{内在旋转的约定}
内在旋转是发生在附着于运动物体的坐标系统$XYZ$的轴上的元素旋转。因此,它们在每次元素旋转后会改变其方向。$XYZ$系统会旋转,而$xyz$系统保持固定。从$XYZ$与$xyz$初始重合开始,三个内在旋转的组合可以用来达到$XYZ$的任何目标方向。

欧拉角可以通过内在旋转来定义。旋转后的坐标系$XYZ$可以想象为最初与$xyz$对齐,然后经历由欧拉角表示的三个元素旋转。其连续的方向可以表示如下:
\begin{itemize}
\item $x-y-z$ 或 $x_0-y_0-z_0$(初始)  
\item $x^\prime-y^\prime-z^\prime$ 或 $x_1-y_1-z_1$(第一次旋转后)  
\item $x"-y"-z"$ 或 $x_2-y_2-z_2$(第二次旋转后)  
\item $X-Y-Z$ 或 $x_3-y_3-z_3$(最终)
\end{itemize}
对于上述列出的旋转序列,节点线 $N$ 可以简单地定义为第一次元素旋转后 $X$ 的方向。因此,$N$ 可以简单地表示为 $x^\prime$。此外,由于第三次元素旋转是围绕 $Z$ 轴进行的,它不会改变 $Z$ 的方向。因此,$Z$ 与 $z"$ 重合。这使得我们可以简化欧拉角的定义如下:
\begin{itemize}
\item $\alpha$(或$\varphi$)表示围绕 $z$ 轴的旋转,  
\item $\beta$(或 $\theta$)表示围绕 $x^\prime$ 轴的旋转,  
\item $\gamma$(或 $\psi$)表示围绕 $z"$ 轴的旋转。
\end{itemize}
\subsubsection{外在旋转的约定}

外在旋转是关于固定坐标系统$xyz$轴的元素旋转$XYZ$系统旋转,而$xyz$系统保持不变。从$XYZ$与$xyz$重合开始,通过三次外在旋转的组合可以实现任何目标方向。欧拉角或泰特–布莱恩角($\alpha,\beta,\gamma$)表示这些元素旋转的幅度。例如,目标方向可以按以下步骤实现(注意欧拉角应用的顺序是反向的):
\begin{enumerate}
\item $XYZ$ 系统绕$z$轴旋转角度$\gamma$。此时$X$轴与$x$轴的夹角为 γ。
\item $XYZ$ 系统再次旋转,这次绕$x$轴旋转角度 $\beta$。此时 $Z$ 轴与$z$轴的夹角为$\beta$。
\item $XYZ$ 系统第三次旋转,绕$z$轴再旋转角度$\alpha$。
\end{enumerate}
总之,三次元素旋转依次发生在$z,x$ 和$z$轴上。实际上,这个旋转序列通常表示为$z-x-z$(或 3-1-3)。与欧拉角和泰特–布莱恩角相关的旋转轴集合通常使用这种符号表示(具体细节请参见上文)。

如果每一步旋转都作用于旋转坐标系统$XYZ$,那么该旋转为内在旋转($Z-X^\prime-Z"$)。内在旋转也可以表示为 3-1-3。
\subsubsection{符号、范围和约定}
角度通常按照右手法则定义。即,当角度表示的旋转在沿轴正方向观察时呈顺时针方向时,其值为正;当旋转呈逆时针方向时,值为负。相反的约定(左手法则)较少采用。

关于范围(使用区间表示法):
\begin{itemize}
\item 对于$\alpha$和$\gamma$ ,范围定义为模$2\pi$弧度。例如,一个有效的范围可以是$[ -\pi,\pi]$。
\item 对于$\beta$,范围覆盖π弧度(但不能称其为模$\pi$的)。例如,它可以是$[0,\pi]$或$[-\pi/2,\pi/2]$。
\end{itemize}
角度 $\alpha,\beta$和$\gamma$唯一确定,除了一个特殊情况,即当$xy$平面和$XY$平面重合时,即当$z$轴和$Z$轴方向相同或相反时。实际上,如果$z$轴和 $Z$轴方向相同,则$\beta =0$,只有$(\alpha+\gamma)$被唯一确定(而不是各自的值);同样,如果$z$轴和$Z$轴方向相反,则 $\beta=\pi$,只有$(\alpha-\gamma)$被唯一确定(而不是各自的值)。这些模糊性在应用中被称为“万向节锁”。

有六种选择旋转轴的方式来确定欧拉角。在所有情况下,第一和第三个旋转轴是相同的。这六种可能的顺序是:
\begin{enumerate}
\item $z_1-x'-z_2"$(内旋转)或 $z_2-x-z_1$(外旋转)  
\item $x_1-y'-x_2"$(内旋转)或 $x_2-y-x_1$(外旋转)  
\item $y_1-z'-y_2"$(内旋转)或 $y_2-z-y_1$(外旋转)  
\item $z_1-y'-z_2"$(内旋转)或 $z_2-y-z_1$(外旋转)  
\item $x_1-z'-x_2"$(内旋转)或 $x_2-z-x_1$(外旋转)  
\item $y_1-x'-y_2"$(内旋转)或 $y_2-x-y_1$(外旋转)
\end{enumerate}
\subsubsection{进动、章动和内旋转}
\begin{figure}[ht]
\centering
\includegraphics[width=6cm]{./figures/93b7f1ef9f9be958.png}
\caption{地球的欧拉基本运动。内旋转(绿色),进动(蓝色)和章动(红色)} \label{fig_OLJ_3}
\end{figure}
进动、章动和内旋转(自旋)被定义为通过改变其中一个欧拉角,同时保持其他两个角度不变所得到的运动。这些运动不是通过外部坐标系来表示,也不是通过与物体共同旋转的坐标系来表示,而是两者的混合。它们构成了一个混合轴旋转系统,其中第一个角度使节点线绕外部轴$z$旋转,第二个角度绕节点线$N$旋转,第三个角度则是绕$Z$轴的内旋转,$Z$轴是一个固定在物体上的轴,随着物体的运动而改变。

静态定义意味着:
\begin{itemize}
\item $\alpha$(进动)表示绕$z$轴的旋转,
\item $\beta$(章动)表示绕$N$或$x'$轴的旋转,
\item $\gamma$(内旋转)表示绕$Z$或 $z"$轴的旋转。
\end{itemize}
如果$\beta$为零,则没有绕$N$轴的旋转。因此$Z$与$z$重合,$\alpha$ 和$\gamma$表示绕同一轴($z$)的旋转,最终的方向可以通过绕$z$轴的单一旋转来获得,旋转角度等于$\alpha+\gamma$。

例如,考虑陀螺仪。陀螺仪绕其自身的对称轴旋转;这对应于它的内旋转。它还绕其支撑轴旋转,质心绕支撑轴做轨道运动;这种旋转是进动。最后,陀螺仪可能会上下摆动;这个倾斜角度就是章动角度。类似的例子也可以用地球的运动来说明。

虽然所有三种运动都可以通过某个坐标系中的旋转算子表示,并且这些算子的系数是常数,但它们不能同时通过这些算子来表示。给定一个参考框架,最多只有一个旋转可以没有系数。一般来说,只有进动可以在空间基底中作为矩阵来表达,而不依赖于其他角度。

这些运动也表现得像一个万向节系统。给定一组框架,每个框架都能根据一个角度相对于前一个框架旋转,就像万向节一样,那么将会存在一个外部固定框架,一个最终框架和两个中间框架,这些中间框架被称为“中间框架”。这两个中间框架就像两个万向节环,使得最后一个框架能够达到空间中的任何方向。
\subsection{泰特–布莱恩角}
第二种形式主义称为泰特–布莱恩角(Tait–Bryan angles),以苏格兰数学物理学家彼得·古思里·泰特(Peter Guthrie Tait,1831–1901)和英国应用数学家乔治·H·布莱恩(George H. Bryan,1864–1928)的名字命名。它是航空航天应用中通常使用的约定,其中零度仰角表示水平姿态。泰特–布莱恩角表示飞机相对于世界坐标系的方向。当涉及到其他类型的飞行器时,可能采用不同的轴约定。
\begin{figure}[ht]
\centering
\includegraphics[width=6cm]{./figures/5eb189b9d48b8484.png}
\caption{泰特–布莱恩角。z-y′-x″ 顺序(内旋转;N 与 y′ 重合)。角度旋转顺序为 ψ、θ、φ。注意,在这种情况下,ψ > 90°,而 θ 是一个负角度。} \label{fig_OLJ_4}
\end{figure}
\subsubsection{定义}
\begin{figure}[ht]
\centering
\includegraphics[width=6cm]{./figures/1828c65a65e6e29a.png}
\caption{泰特–布莱恩角。z-x′-y″ 顺序(内旋转;N 与 x′ 重合)} \label{fig_OLJ_5}
\end{figure}
泰特–布莱恩角的定义和符号与上述的正欧拉角(几何定义、内旋转定义、外旋转定义)类似。唯一的区别是,泰特–布莱恩角表示绕三个不同的轴旋转(例如$x-y-z$,或 $x-y'-z"$),而正欧拉角在第一和第三个元素旋转时使用相同的轴(例如 $z-x-z$,或$x-y'-z"$)。

这意味着在几何构造中,节点线的定义不同。在正欧拉角的情况下,节点线被定义为两个同类笛卡尔平面的交线(当欧拉角为零时,平面平行;例如$xy$和$XY$)。而在泰特–布莱恩角的情况下,节点线被定义为两个非同类平面的交线(当欧拉角为零时,平面垂直;例如$xy$和$YZ$)。
\subsubsection{约定}
\begin{figure}[ht]
\centering
\includegraphics[width=6cm]{./figures/dd81cf57dc981f94.png}
\caption{对于一架飞机,使用机载 ENU 坐标系来表示航向角、仰角和滚转角(Z-Y′-X″),既用于机载系统也用于地面跟踪站。固定参考框架 x-y-z 代表地面跟踪站。机载坐标系中的 Y 轴和 Z 轴未显示。X 轴以绿色表示。根据右手法则(RHS),所示的 y 轴是负的。} \label{fig_OLJ_6}
\end{figure}
三种基本旋转可以发生在原始坐标系的轴上,该坐标系保持不动(外旋转),也可以发生在旋转坐标系的轴上,旋转坐标系在每次基本旋转后都会改变其方向(内旋转)。

对于泰特–布莱恩角,选择旋转轴有六种可能。六种可能的顺序如下:
\begin{itemize}
\item $x-y'-z"$(内旋转)或 $z-y-x$(外旋转)
\item $y-z'-x"$(内旋转)或 $x-z-y$(外旋转)
\item $z-x'-y"$(内旋转)或 $y-x-z$(外旋转)
\item $x-z'-y"$(内旋转)或 $y-z-x$(外旋转)
\item $z-y'-x"$(内旋转)或 $x-y-z$(外旋转):内旋转通常称为:偏航(yaw)、俯仰(pitch)和滚转(roll)
\item $y-x'-z"$(内旋转)或 $z-x-y$(外旋转)
\end{itemize}
\subsubsection{符号和范围}
\begin{figure}[ht]
\centering
\includegraphics[width=8cm]{./figures/1cc6806c7ebed73e.png}
\caption{根据航空标准 DIN 9300,飞机的主轴。请注意,固定坐标系和移动坐标系在角度为零时必须重合。因此,这一标准也会强制在参考系统中采用兼容的轴约定。} \label{fig_OLJ_7}
\end{figure}
泰特–布莱恩约定在工程中被广泛应用,具有不同的目的。实际中有多种选择移动轴和固定轴的轴约定,这些约定决定了角度的符号。因此,在每种情况下必须仔细研究符号。

角度$\Psi$和$\varphi$的范围覆盖$2\pi$弧度。对于$\theta$,范围覆盖$\pi$弧度。
\subsubsection{替代名称}  
这些角度通常在外部参考框架中取一个(航向角、方位角),在内旋转的移动框架中取一个(滚转角),并在一个中间框架中取一个,表示相对于水平面的仰角或倾斜角,这对于这个目的来说等同于节点线。
\subsubsection{作为链式旋转}
\begin{figure}[ht]
\centering
\includegraphics[width=8cm]{./figures/723aeb5d6b8bfa4e.png}
\caption{记忆角度名称的助记法} \label{fig_OLJ_8}
\end{figure}
对于一架飞机,如果按照正确的顺序并从与参考框架重合的框架开始,可以通过绕其主轴的三次旋转来获得这些角度。
\begin{itemize}
\item 进动(yaw)得到方位角,
\item 俯仰(pitch)得到仰角,
\item 滚转(roll)得到滚转角。
\end{itemize}
因此,在航空航天领域,它们有时被称为偏航角、俯仰角和滚转角。请注意,如果旋转按照其他顺序应用,或如果飞机的轴线不从与参考框架等效的位置开始,则这种方法将无法奏效。

泰特–布莱恩角,按照$z-y'-x"$(内旋转)约定,也被称为航海角,因为它们可以用来描述船只或飞机的方向,或者被称为卡尔达诺角,以纪念意大利数学家和物理学家杰罗拉莫·卡尔达诺(Gerolamo Cardano),他首次详细描述了卡尔达诺悬挂系统和卡尔达诺接头。
\subsection{给定坐标系的角度}
\begin{figure}[ht]
\centering
\includegraphics[width=6cm]{./figures/7c7c0e8964f9ef70.png}
\caption{Z 向量的投影} \label{fig_OLJ_9}
\end{figure}
一个常见的问题是求给定坐标系的欧拉角。获得欧拉角的最快方法是将三个给定的向量写成矩阵的列,并将其与理论矩阵的表达式进行比较(见后面的矩阵表格)。因此,可以计算出三个欧拉角。然而,使用基本几何而不依赖矩阵代数也可以得到相同的结果。这里我们展示了两种最常用的约定的结果:$ZXZ$用于正欧拉角,$ZYX$用于泰特–布莱恩角。请注意,任何其他约定都可以通过更改轴的名称来获得。
\subsubsection{正欧拉角}
\begin{figure}[ht]
\centering
\includegraphics[width=6cm]{./figures/f22872a0ad356f04.png}
\caption{Y 向量的投影} \label{fig_OLJ_10}
\end{figure}
假设一个坐标系,单位向量 (X, Y, Z) 由其坐标给出,如主图所示,可以得出:
\[
\cos(\beta) = Z_3.~
\]
由于
\[
\sin^2 x = 1 - \cos^2 x,~
\]
对于 \( 0 < \beta < \pi \),我们有:
\[
\sin(\beta) = \sqrt{1 - Z_3^2}.~
\]
由于 \( Z_2 \) 是单位向量的双重投影,
\[
\cos(\alpha) \cdot \sin(\beta) = -Z_2,~
\]
所以
\[
\cos(\alpha) = -Z_2/\sqrt{1 - Z_3^2}.~
\]
对于 \( Y_3 \),我们有类似的构造,首先将其投影到由轴 z 和节点线定义的平面上。由于两个平面之间的角度是 \( \pi/2 - \beta \),并且\(\cos\left(\pi/2 - \beta\right) = \sin(\beta),\)
这导致:
\[
\sin(\beta) \cdot \cos(\gamma) = Y_3,~
\]
所以
\[
\cos(\gamma) = Y_3/\sqrt{1 - Z_3^2}.~
\]
最后,使用反余弦函数,我们可以得到:
\[
\alpha = \arccos\left(-Z_2/\sqrt{1 - Z_3^2}\right),~
\]

\[
\beta = \arccos(Z_3),~
\]

\[
\gamma = \arccos\left(Y_3/\sqrt{1 - Z_3^2}\right).~
\]
\subsubsection{泰特–布莱恩角}
假设一个坐标系,单位向量 (X, Y, Z) 由其坐标给出,如此新图所示(注意角度 \( \theta \) 是负的),可以得出:
\[
\sin(\theta) = -X_3.~
\]
如同之前的推导,
\[
\cos^2 x = 1 - \sin^2 x,~
\]
对于 \( -\pi/2 < \theta < \pi/2 \),我们有:
\[
\cos(\theta) = \sqrt{1 - X_3^2}.~
\]
与前面的推导类似:
\[
\sin(\psi) = X_2/\sqrt{1 - X_3^2},~
\]

\[
\sin(\phi) = Y_3/\sqrt{1 - X_3^2}.~
\]
寻找与前面类似的表达式:
\[
\psi = \arcsin\left(X_2/\sqrt{1 - X_3^2}\right),~
\]

\[
\theta = \arcsin(-X_3),~
\]

\[
\phi = \arcsin\left(Y_3/\sqrt{1 - X_3^2}\right).~
\]
\subsubsection{最后的备注}  
请注意,反正弦和反余弦函数会为其参数提供两个可能的值。在这个几何描述中,只有一个解是有效的。当欧拉角定义为旋转序列时,所有解都可能有效,但只有一个解位于角度范围内。这是因为,如果范围没有事先定义,那么到达目标坐标系的旋转序列不是唯一的。[2]

为了计算的方便,使用 \( \text{atan2}(y, x) \) 来表示角度可能会很有用。例如,在正欧拉角的情况下:
\[
\alpha = \operatorname{atan2}(Z_1, -Z_2),~
\]

\[
\gamma = \operatorname{atan2}(X_3, Y_3).~
\]
\subsection{转换为其他方向表示法} 
主条目:三维旋转公式 § 旋转公式之间的转换公式  

欧拉角是一种表示方向的方法。还有其他方法,可以在不同的约定之间进行转换。描述三维欧几里得空间中的方向总是需要三个参数。它们可以通过多种方式给出,欧拉角是其中之一;有关其他方法,请参见SO(3)的图表。

最常见的方向表示法有旋转矩阵、轴角表示法和四元数(也称为欧拉–罗德里格斯参数),它们提供了另一种表示三维旋转的机制。这等同于特殊酉群的描述。

将三维旋转表示为单位四元数而不是矩阵有一些优点:
\begin{itemize}
\item 连接旋转的计算更快,数值更稳定。
\item 提取旋转角度和旋转轴更简单。
\item 插值更直观。例如,参考球面线性插值(slerp)。
\item 四元数不像欧拉角那样会遇到万向节锁死(gimbal lock)问题。
\end{itemize}
无论如何,旋转矩阵的计算是获得其他两种表示法的第一步。
\subsubsection{旋转矩阵}  
任何方向都可以通过从已知标准方向开始,组合三个基本旋转来实现。等效地,任何旋转矩阵 \( R \) 都可以分解为三个基本旋转矩阵的乘积。例如:
\[
R = X(\alpha) Y(\beta) Z(\gamma)~
\]
这是一个旋转矩阵,可以用来表示关于 $z,y,x$ 轴的外旋转的组合(按此顺序),或者表示关于 $x-y'-z"$ 轴的内旋转的组合(按此顺序)。然而,基本旋转矩阵 \( X \)、\( Y \)、\( Z \) 的定义以及它们的乘法顺序取决于用户对旋转矩阵和欧拉角定义的选择(例如,参见旋转矩阵定义中的歧义)。不幸的是,不同的约定集在不同的上下文中被用户采用。以下表格是根据这一约定集构建的:
\begin{enumerate}
\item 每个矩阵都意味着通过左乘列向量  
\[
\begin{bmatrix} x \\ y \\ z \end{bmatrix}~
\]  
(参见旋转矩阵定义中的歧义)
\item 每个矩阵都表示一个主动旋转(组成矩阵和被组成矩阵假定作用于在初始固定参考系中定义的向量坐标,并给出旋转后向量在同一参考系中的坐标)。
\item 每个矩阵主要表示内旋转的组合(绕旋转参考系的轴旋转),其次表示外旋转的组合(三个外旋转的乘积,表示通过三个真正的基本矩阵的乘法按相反顺序构造得到 \( R \) 矩阵)。
采用右手坐标系,并使用右手定则来确定角度 \( \alpha \)、\( \beta \)、\( \gamma \) 的符号。
\end{enumerate}
为了简化,以下矩阵乘积表使用了以下命名法:
\begin{enumerate}
\item \( X \)、\( Y \)、\( Z \) 是表示绕固定坐标系的 \( x \)、\( y \)、\( z \) 轴旋转的基本矩阵(例如,\( X_\alpha \) 表示绕 \( x \) 轴旋转角度 \( \alpha \))。
\item \( s \) 和 \( c \) 分别表示正弦和余弦(例如,\( s_\alpha \) 表示 \( \alpha \) 的正弦)。
\end{enumerate}

\begin{figure}[ht]
\centering
\includegraphics[width=14.25cm]{./figures/c03f2c3d6141a97a.png}
\caption{} \label{fig_OLJ_11}
\end{figure}

这些表格结果可以在许多教科书中找到。[3] 对于每一列,最后一行表示最常用的约定。

要更改被动旋转的公式(或找到反向主动旋转),只需转置矩阵(此时每个矩阵将初始坐标系中的向量坐标转换为在旋转参考系中测量的相同向量的坐标;相同的旋转轴,相同的角度,但现在是坐标系旋转,而不是向量旋转)。

下表包含了从旋转矩阵元素 \( R \) 中得到的角度 \(\alpha\),\(\beta\) 和 \(\gamma\) 的公式。[4]

\begin{table}[ht]
\centering
\caption\label{OLJ}
\begin{tabular}{|c|c||c|c|}
\hline
&\textbf{正规欧拉角}& &\textbf{泰特–布赖恩角}\\


\hline 
\end{tabular}
\end{table}