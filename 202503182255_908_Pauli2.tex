% 沃尔夫冈·泡利(综述)
% license CCBYSA3
% type Wiki

本文根据 CC-BY-SA 协议转载翻译自维基百科\href{https://en.wikipedia.org/wiki/Wolfgang_Pauli}{相关文章}。

\begin{figure}[ht]
\centering
\includegraphics[width=6cm]{./figures/7d0315d83d1af784.png}
\caption{泡利于1945年} \label{fig_Pauli2_1}
\end{figure}
沃尔夫冈·恩斯特·泡利(Wolfgang Ernst Pauli,/ˈpɔːli/,[5] 德语:[ˈvɔlfɡaŋ ˈpaʊli];1900年4月25日—1958年12月15日)是一位奥地利物理学家,量子力学的先驱。1945年,在阿尔伯特·爱因斯坦的提名下,[6] 泡利因其“通过发现一条新的自然法则——泡利不相容原理(Pauli Exclusion Principle)所做出的决定性贡献”而获得诺贝尔物理学奖。这一发现涉及自旋理论,该理论是物质结构理论的基础。为了保持\(\beta\)衰变中的能量守恒,他提出了一种质量极小、电中性的粒子,其后被恩里科·费米命名为中微子。该粒子最终于1956年被探测到。

泡利就读于维也纳的多布林格文理中学,并于1918 年以优异成绩毕业。两个月后,他发表了第一篇论文,内容涉及阿尔伯特·爱因斯坦的广义相对论。随后,他进入慕尼黑大学学习,并在阿诺德·索末菲指导下开展研究。[1]1921年7月,他因其关于电离双原子氢(\(H_2^+\))的量子理论的论文获得博士学位。[2][9]
\subsection{职业生涯}  
索末菲邀请泡利为《数学科学百科全书》撰写相对论综述。在获得博士学位后的两个月内,泡利便完成了这篇长达 237 页的文章。爱因斯坦对此给予高度评价;该文章后来以专著形式出版,并至今仍是该领域的重要参考文献。[10]
\begin{figure}[ht]
\centering
\includegraphics[width=6cm]{./figures/5713325387bd5c08.png}
\caption{沃尔夫冈·泡利用于讲授(1929年)} \label{fig_Pauli2_2}
\end{figure}
泡利曾在哥廷根大学担任马克斯·玻恩的助理一年,随后一年在哥本哈根理论物理研究所(后来的 尼尔斯·玻尔研究所)工作。[1] 从1923 年到 1928 年,他在汉堡大学任教。[11] 在此期间,泡利对现代量子力学理论的发展发挥了关键作用,特别是提出了泡利不相容原理以及非相对论性自旋理论。

1928 年,泡利被任命为瑞士苏黎世联邦理工学院的理论物理教授。[1]1930 年,他获得洛伦兹奖章。[12] 他还曾在1931 年担任密歇根大学客座教授,并于1935 年在普林斯顿高等研究院担任访问教授。
\subsubsection{荣格}  
**1930 年底**,就在**提出中微子假说**后不久,泡利经历了一场**个人危机**,当时他**刚刚离婚**,并且他的**母亲自杀**。**1932 年 1 月**,他向居住在苏黎世附近的**心理学家和心理治疗师** **卡尔·荣格**(Carl Jung)寻求帮助。荣格立刻开始分析泡利**充满原型意象的梦境**,并将其纳入自己的研究。泡利随后**从科学角度批判性地探讨了荣格理论的认识论基础**,这些讨论帮助**澄清了荣格的某些概念**,尤其是**共时性**(synchronicity)的理论。他们的讨论记录保存在**泡利—荣格通信中,后来以**《原子与原型》(Atom and Archetype)**出版。而荣格对泡利**400 多个梦境的详细分析**,则记录在**《心理学与炼金术》(Psychology and Alchemy)**中。

**1933 年**,泡利出版了他的**物理学著作《物理手册》(Handbuch der Physik)**的**第二部分**,该书被认为是**量子物理学领域的权威著作**。罗伯特·奥本海默(Robert Oppenheimer)称其为**“唯一一本真正成熟的量子力学入门书”**。[13]