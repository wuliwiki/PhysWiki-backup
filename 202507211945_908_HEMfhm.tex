% 赫尔曼·冯·亥姆霍兹(综述)
% license CCBYSA3
% type Wiki

本文根据 CC-BY-SA 协议转载翻译自维基百科\href{https://en.wikipedia.org/wiki/Hermann_von_Helmholtz}{相关文章}。

赫尔曼·路德维希·费迪南德·冯·亥姆霍兹(Hermann Ludwig Ferdinand von Helmholtz,/ˈhɛlmhoʊlts/;德语:[ˈhɛʁman fɔn ˈhɛlmˌhɔlts];1821年8月31日-1894年9月8日,自1883年起冠以“冯(von)”的贵族头衔)是一位德国物理学家和医生,在多个科学领域作出了重要贡献,尤其以流体动力学稳定性理论而闻名\(^\text{[2]}\)。以他命名的亥姆霍兹协会是德国最大的科研机构联合体\(^\text{[3]}\)。

在生理学和心理学领域,亥姆霍兹以其关于眼睛的数学研究、视觉理论、空间视觉感知的观点、色觉研究、音调感觉与听觉感知理论,以及对感知生理学中经验主义的探讨而著称。在物理学中,他以能量守恒定律、电双层理论、电动力学、化学热力学,以及热力学的力学基础研究而闻名。尽管能量守恒原则的发展也归功于尤利乌斯·冯·迈尔、詹姆斯·焦耳和丹尼尔·伯努利等人,但亥姆霍兹被认为是第一个以最一般形式提出能量守恒原理的人\(^\text{[4]}\)。

作为哲学家,亥姆霍兹以其科学哲学、关于知觉规律与自然规律之间关系的见解、美学科学思想,以及关于科学的文明力量等理念而受到关注。到19世纪末,亥姆霍兹发展出一种广义的康德方法论,包括对知觉空间中可能取向的先验确定,这不仅激发了对康德的新解读\(^\text{[4]}\),也对现代后期的新康德主义哲学运动产生了重要影响\(^\text{[5]}\)。
\subsection{生平}
\subsubsection{早年经历}
亥姆霍兹出生于波茨坦,是当地文理中学校长费迪南德·亥姆霍兹的儿子。父亲曾学习古典语言学和哲学,是出版人兼哲学家伊曼努尔·赫尔曼·费希特(的密友。亥姆霍兹的研究受到约翰·戈特利布·费希特和伊曼努尔·康德哲学思想的影响,他尝试在诸如生理学等经验领域中追溯这些理论的体现。

年轻时,亥姆霍兹对自然科学兴趣浓厚,但父亲希望他学习医学。1842年,亥姆霍兹在柏林的腓特烈-威廉医学外科研究院获得医学博士学位,并在夏里特医院完成为期一年的实习\(^\text{[6]}\)(因为医学专业提供财政资助)。

虽然主要接受的是生理学训练,亥姆霍兹却在许多其他主题上都有著述,从理论物理学到地球年龄的估计,再到太阳系的起源等问题。
\subsubsection{大学任职}
1848年,亥姆霍兹的第一份学术职务是在柏林艺术学院担任解剖学教师\(^\text{[7]}\)。随后,他于1849年在普鲁士的哥尼斯堡大学被任命为生理学副教授。1855年,他接受了波恩大学的全职解剖学与生理学教授职位。然而,他在波恩并不特别满意,三年后调任至巴登的海德堡大学,担任生理学教授。1871年,他接受了最后一个大学职位,在柏林的腓特烈·威廉大学(今柏林洪堡大学)担任物理学教授。
\subsection{研究工作}
\subsubsection{亥姆霍兹}
力学

亥姆霍兹的第一项重要科学成就,是他于1847年撰写的一篇关于能量守恒的论文。这项工作是在其医学研究和哲学背景的语境中完成的。他对能量守恒的研究起初源于对肌肉代谢的探索,试图证明肌肉运动过程中并没有能量的损失,其动机在于说明肌肉的运动不需要任何“生命力”来驱动。这是对当时在德国生理学中占主导地位的自然哲学(Naturphilosophie)和生命力论等投机哲学传统的直接否定。他反对一些生命力论者提出的观点——即“生命力”可以无限地驱动一台机器\(^\text{[4]}\)。

在此前萨迪·卡诺、贝努瓦-保罗·埃米尔·克拉佩龙和詹姆斯·普雷斯科特·焦耳等人的研究基础上,亥姆霍兹提出了一个假设,认为力学、热、光、电和磁都是单一“力”的表现形式——用今天的术语来说,即“能量”。他在其著作《论力的守恒》(Über die Erhaltung der Kraft, 1847)中发表了这一理论\(^\text{[8]}\)。

在19世纪50至60年代,亥姆霍兹与威廉·汤姆森(后来的开尔文勋爵)及威廉·兰金基于前者的出版物,共同推广了“宇宙热寂”这一概念。

在流体动力学方面,亥姆霍兹也作出多项贡献,包括在无粘性流体中提出的“亥姆霍兹涡旋定理”。
