% 拉普拉斯变换的性质

\pentry{拉普拉斯变换\upref{LapTra}}

前面我们介绍了拉普拉斯变换,那么这个变换到底给我们带来了什么特别的好处呢?我们下面就来看一下它到底具有哪些性质.

\subsection{线性性质}


若$\mathscr L[f_1(t)] = \bar f_1(p),\mathscr L[f_2(t)] = \bar f_2(p)$,则
\begin{equation}
\mathscr L[c_1f_1(t)+c_2f_2(t)] = c_1\bar f_1(p) + c_2\bar f_2(p)
\end{equation}

证明起来也十分容易.
\begin{equation}
\begin{aligned} \mathscr L[c_{1} f_{1}(t)+c_{2} f_{2}(t)] & = \int_{0}^{\infty}\left[c_{1} f_{1}(t)+c_{2} f_{2}(t)\right] \mathrm{e}^{-\rho t} \mathrm{d} t \\ &=\int_{0}^{\infty} c_{1} f_{1}(t) \mathrm{e}^{-p t} \mathrm{d} t+\int_{0}^{\infty} c_{2} f_{2}(t) \mathrm{e}^{-\rho t} \mathrm{d} t \\ &=c_{1} \bar{f}_{1}(p)+c_{2} \bar{f}_{2}(p) \end{aligned}
\end{equation}

它能让我们干什么呢?我们可以把未知的函数拆成已知函数的线性组合,相加它们的拉普拉斯变换来得到所要的未知函数的拉普拉斯变换.例如:
\begin{example}{线性性质的应用}
求$\mathscr L[\sin \omega t]$,$\omega$为常数.

我们知道
\begin{equation}
\sin \omega t=\frac{1}{2 \mathrm{i}}\left(\mathrm{e}^{\mathrm{i} \omega t}-\mathrm{e}^{-\mathrm{i} \omega t}\right)
\end{equation}
所以有
\begin{equation}
\begin{aligned} \mathscr{L}[\sin \omega t] &=\mathscr{L}\left[\frac{1}{2 \mathrm{i}}\left(\mathrm{e}^{\mathrm{i} \omega t}-\mathrm{e}^{-\mathrm{i} \omega t}\right)\right]=\frac{1}{2 \mathrm{i}} \mathscr{L}\left[\mathrm{e}^{\mathrm{i} \omega t}\right]-\frac{1}{2 \mathrm{i}}\mathscr{L}\left[\mathrm{e}^{-\mathrm{i} \omega t}\right] \\ &=\frac{1}{2 \mathrm{i}}\left[\frac{1}{p-\mathrm{i} \omega}-\frac{1}{p+\mathrm{i} \omega}\right] \\ &=\frac{\omega}{p^{2}+\omega^{2}} \quad(\Re p>0) \end{aligned}
\end{equation}

同理可得
\begin{equation}
\mathscr{L}[\cos \omega t]=\frac{p}{p^{2}+\omega^{2}} \quad(\operatorname{Re} p>0)
\end{equation}
\end{example}

\subsection{导数定理}

导数定理是说:
\begin{equation}
\mathscr L[f'(t)]=p\bar f(p)-f(0)
\end{equation}

证明同样直接积分就可以了.

\begin{equation}
\begin{aligned} \mathscr L[f^{\prime}(t)] & = \int_{0}^{\infty} f^{\prime}(t) \mathrm{e}^{-p t} \mathrm{d} t=\int_{0}^{\infty} \mathrm{e}^{-p t} \mathrm{d} f \\ &=\left[\mathrm{e}^{-p t} f(t)\right]_{0}^{\infty}-\int_{0}^{\infty} f(t) \mathrm{d}\left(\mathrm{e}^{-p t}\right) \end{aligned}
\end{equation}
取$\Re p>\sigma$,有$\lim_{t\to\infty}\mathrm{e}^{-pt}f(t)=0$.于是可得
\begin{equation}
\begin{aligned} \mathscr L[f^{\prime}(t)] & = -f(0)-\int_{0}^{\infty} f(t) \mathrm{d}\left(\mathrm{e}^{-\rho t}\right)=p \int_{0}^{\infty} f(t) \mathrm{e}^{-p t} \mathrm{d} t-f(0) \\ &=p \bar{f}(p)-f(0) \quad\left(\operatorname{Re} p>\sigma_{0}\right) \end{aligned}
\end{equation}

我们还可以推广到高阶导数的情形:
\begin{equation}
\mathscr L[f^{(n)}(t)] = p^{n} f(p)-p^{n-1} f(0)-p^{n-2} f^{\prime}(0)-\cdots-p f^{(n-2)}(0)-f^{(n-1)}(0)
\end{equation}