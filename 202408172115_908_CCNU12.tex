% 华中师范大学 2012 年 考研 量子力学
% license Usr
% type Note

\textbf{声明}:“该内容来源于网络公开资料,不保证真实性,如有侵权请联系管理员”

\subsection{选择题(共 18分,每小题3分)}
\begin{enumerate}
\item  在给定的状态 $\\psi(x,t)$ 中 “测量” 粒子的坐标(\\ \\ \\ \\ )
    \begin{enumerate}
    \item 测量” 使得 $\\psi(x,t)$ 不再按照薛定谔方程演化
    \item “测量” 使得微观粒子不在任何位置
    \item “测量” 使得微观粒子的坐标越精确动量就越精确
    \item “测量” 使得 $\\psi(x,t)$ 突然和不连续的坍塌
    \end{enumerate}
    
\\item  坐标对时间导数的算符是(\\ \\ \\ \\ )
    \\begin{enumerate}
    \\item 坐标算符 
    \\item 动量算符 
    \\item 速度算符 
    \\item 角动量算符
    \\end{enumerate}
    
\\item  设质量为 $m$ 粒子的两个本征函数分别是 $\\psi_1(x) = c_1e^{-ax^2/2}$,$\\psi_2(x) = c_2(x^2+b)e^{-ax^2}$,则粒子这两状态的能级间隔为(\\ \\ \\ \\ )
    \\begin{enumerate}
    \\item $-\\frac{\\hbar^2}{mb}$
    \\item $-\\frac{\\hbar^2}{(mb)^2}$
    \\item $-\\frac{\\hbar}{mb}$
    \\item $\\frac{\\hbar^2}{(mb)^2}$
    \\end{enumerate}

\\item  对于任意的 $\\mathbf{a}$,若 $(\\mathbf{a}|\\mathbf{b}) = (\\mathbf{a}|\\mathbf{c})$,则(\\ \\ \\ \\ )
    \\begin{enumerate}
    \\item $\\mathbf{b} \\ne \\mathbf{c}$
    \\item $\\mathbf{b} = \\mathbf{c}$
    \\item $(\\mathbf{a}| \\mathbf{b}) = (\\mathbf{a}| \\mathbf{c})$
    \\item $(\\mathbf{a}| = \\mathbf{b})$
    \\end{enumerate}

\\item  在一维情况下,若 $U(x) \\text{连续}, U(\\pm \\infty) = 0$ 且 $U(x) < 0$,则该体系(\\ \\ \\ \\ )
    \\begin{enumerate}
    \\item 两个束缚态
    \\item 无束缚态
    \\item 一个束缚态
    \\item 至少存在一个束缚态
    \\end{enumerate}
    
\\item  微观体系存在任意态 $|\\mathbf{n}\\rangle$ 中,能量的平均值 $\\bar{E}$(\\ \\ \\ \\ )
    \\begin{enumerate}
    \\item 体系的基态能量 $E$
    \\item 没有确定值
    \\item $\\geq$ 体系的基态能量 $E$
    \\item = 体系的基态能量 $E$
    \\end{enumerate}

\\end{enumerate}
