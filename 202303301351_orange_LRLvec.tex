% 拉普拉斯—龙格—楞次矢量
% LRL 矢量|守恒证明

\pentry{开普勒问题\upref{CelBd}, 柱坐标系\upref{Cylin}}

在开普勒问题% 未完成:什么是开普勒问题?
中, 我们定义\textbf{拉普拉斯—龙格—楞次矢量(Laplace-Runge-Lenz Vector)} (缩写为 \textbf{LRL 矢量})为
\begin{equation}\label{LRLvec_eq1}
\bvec A = \bvec p \cross \bvec L - mk \uvec r~.
\end{equation}
其中 $\bvec p$ 为质点动量, $\bvec L$ 为轨道角动量, $\cross$ 表示矢量叉乘\upref{Cross}, $k$ 是一个常数(对万有引力\upref{Gravty} $k = GMm$, 对库仑力\upref{ClbFrc} $k = -Qq/(4\pi\epsilon_0)$。 $\uvec r$ 为质点位矢 $\bvec r$ 的单位矢量。 在开普勒问题% 未完成:什么是开普勒问题?
中, 可以证明 $\bvec A$ 是一个守恒量。

\subsection{守恒证明}
我们下面证明 $\dot{\bvec A} = \bvec 0$。 对\autoref{LRLvec_eq1} 求时间导数, 考虑到中心力场中质点角动量 $\bvec L$ 守恒, 有
\begin{equation}\label{LRLvec_eq2}
\dot{\bvec A} = \dot{\bvec p}\cross \bvec L  - mk\dot{\uvec r}~.
\end{equation}
其中由牛顿第二定律\upref{New3} 和万有引力定律/库仑力, 有
\begin{equation}\label{LRLvec_eq3}
\dot{\bvec p} = \bvec F = - \frac{k}{r^2}\uvec r~,
\end{equation}
又由“极坐标中单位矢量的偏导\upref{DPol1}” 得
\begin{equation}\label{LRLvec_eq4}
\dot{\uvec r} = \pdv{\uvec r}{\theta} \dv{\theta}{t} = \dot\theta\uvec \theta~.
\end{equation}
最后由\autoref{CenFrc_eq4}~\upref{CenFrc}, 极坐标系中的角动量等于($\uvec z$ 是垂直于轨道平面的单位矢量, 由右手定则\upref{RHRul}决定, 参考柱坐标系\upref{Cylin})
\begin{equation}\label{LRLvec_eq5}
\bvec L = mr^2\dot \theta \uvec z~.
\end{equation}
将\autoref{LRLvec_eq3} 至\autoref{LRLvec_eq5} 代入\autoref{LRLvec_eq2} 得
\begin{equation}
\dot{\bvec A} = -\frac{k}{r^2}\uvec r \cross (mr^2\dot\theta\uvec z) - mk\dot\theta\uvec\theta
=-mk\dot\theta (\uvec r\cross \uvec z + \uvec \theta)
= \bvec 0~.
\end{equation}
最后一个等号成立是因为 $\uvec r\cross\uvec z = -\uvec\theta$, 可以类比直角坐标系中的 $\uvec x\cross\uvec z = -\uvec y$。 证毕。

% 未完成: 习题:证明二体问题中
% \bvec A_B = \frac{mu}{m_B} \bvec A
% \bvec A_A = -\frac{mu}{m_A} \bvec A
% \bvec A = \bvec A_B - \bvec A_A
% 其中 \bvec A 是等效天体(质量为 mu)的 LRL 矢量。

% 未完成: LRL 矢量的模长和方向?
