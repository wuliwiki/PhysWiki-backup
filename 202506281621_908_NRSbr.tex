% 尼尔斯·玻尔(综述)
% license CCBYSA3
% type Wiki

本文根据 CC-BY-SA 协议转载翻译自维基百科\href{https://en.wikipedia.org/wiki/Niels_Bohr}{相关文章}。

\begin{figure}[ht]
\centering
\includegraphics[width=6cm]{./figures/2f702d956c83f882.png}
\caption{1922 年的玻尔} \label{fig_NRSbr_3}
\end{figure}
尼尔斯·亨里克·戴维·玻尔(Niels Henrik David Bohr,美国发音:/boʊr/,英国发音:/bɔː/,\(^\text{[2]}\)丹麦语:[ˈne̝ls ˈpoɐˀ];1885年10月7日-1962年11月18日)是一位丹麦理论物理学家,他在原子结构和量子理论的理解方面作出了奠基性贡献,并因此于1922年获得诺贝尔物理学奖。玻尔同时也是一位哲学家和科学研究的推动者。

玻尔提出了著名的玻尔原子模型,他提出电子的能级是离散的,电子围绕原子核在稳定轨道上运行,但可以从一个能级(或轨道)跃迁到另一个能级(或轨道)。尽管玻尔模型已被其他模型取代,但其基本原理仍然有效。他提出了互补性原理:事物可以从相互矛盾的属性中被分别分析,例如表现为波动或粒子流的行为。这一互补性概念贯穿了玻尔在科学和哲学领域的思考。

玻尔在哥本哈根大学创立了理论物理研究所(现称尼尔斯·玻尔研究所),该研究所于1920年开放。玻尔指导并与多位物理学家合作,包括汉斯·克拉默斯、奥斯卡·克莱因、乔治·德·赫维希和沃尔夫冈·海森堡。他预测了一种类似锆的新元素的特性,该元素因在哥本哈根被发现而以哥本哈根的拉丁名称命名为“铪”。后来,合成元素“𬬻”因玻尔在原子结构领域的开创性工作而以他的名字命名。

在20世纪30年代,玻尔帮助了逃离纳粹主义的难民。丹麦被德国占领后,他会见了已成为德国核武器项目负责人的海森堡。1943年9月,玻尔得知德国人即将逮捕他,于是他逃往瑞典。从那里,他被空运到英国,加入了英国的“合金管”核武器项目,并作为英国代表团成员参与了曼哈顿计划。战争结束后,玻尔呼吁在核能领域开展国际合作。他参与了欧洲核子研究中心(CERN)和丹麦原子能委员会下属的里瑟研究机构的建立,并于1957年成为北欧理论物理研究所的首任主席。
\subsection{早年生活}
尼尔斯·亨里克·戴维·玻尔于1885年10月7日出生在丹麦哥本哈根,是克里斯蒂安·玻尔和妻子埃伦(娘家姓阿德勒,Ellen née Adler)的三个孩子中的老二。\(^\text{[3][4]}\)其父克里斯蒂安是哥本哈根大学的生理学教授,母亲埃伦出身于一个富裕的犹太银行世家。\(^\text{[5]}\)他有一个姐姐珍妮和一个弟弟哈拉尔。\(^\text{[3]}\)珍妮后来成为教师,\(^\text{[4]}\)而哈拉尔成为数学家和足球运动员,曾代表丹麦国家队参加1908年在伦敦举行的夏季奥运会。尼尔斯本人也是一名热情的足球运动员,两兄弟曾一起为位于哥本哈根的学术足球俱乐部效力,尼尔斯担任守门员。\(^\text{[6]}\)

玻尔七岁时进入加梅尔霍姆拉丁学校就读。\(^\text{[7]}\)1903年,玻尔进入哥本哈根大学本科就读,主修物理学,师从当时该校唯一的物理学教授克里斯蒂安·克里斯蒂安森。此外,他还在托瓦尔·蒂勒教授指导下学习天文学和数学,并在其父的朋友哈拉尔·霍夫丁教授指导下学习哲学。\(^\text{[8][9]}\)
\begin{figure}[ht]
\centering
\includegraphics[width=6cm]{./figures/3736fdffe67b90d2.png}
\caption{年轻时的玻尔} \label{fig_NRSbr_1}
\end{figure}
1905 年,丹麦皇家科学院举办了一项金质奖章竞赛,题目是研究测量液体表面张力的方法,该方法最初由瑞利勋爵于 1879 年提出。这项研究需要测量水射流半径振动的频率。玻尔在大学里利用他父亲的实验室进行了一系列实验;当时大学本身并没有物理实验室。为了完成实验,他不得不自己制作玻璃器皿,吹制出具有所需椭圆形横截面的试管。他不仅完成了原先的任务,还在瑞利的理论和方法上进行了改进,他考虑了水的黏滞性,并使用有限振幅而非仅限于无穷小振幅进行实验。他在最后一刻提交的论文赢得了这项奖章。他随后将改进后的论文提交给伦敦皇家学会,在《皇家学会哲学汇刊》上发表。\(^\text{[10][11][9][12]}\)

哈拉尔德是玻尔兄弟中第一个获得硕士学位的人,他于 1909 年 4 月获得数学硕士学位。尼尔斯又花了九个月时间,于同年完成了关于金属电子理论的硕士论文,这一课题是由他的导师克里斯琴森布置的。随后,玻尔将硕士论文扩展成了篇幅更大的博士论文。他调研了该领域的文献,最终选择了保罗·德鲁德提出并由亨德里克·洛伦兹完善的模型,该模型认为金属中的电子表现得像气体一样。玻尔在洛伦兹模型的基础上进行了扩展,但仍无法解释霍尔效应等现象,最终他得出结论:电子理论无法完全解释金属的磁性特性。论文于 1911 年 4 月被接收,\(^\text{[13]}\)玻尔于 5 月 13 日进行了正式答辩。哈拉尔德在前一年已获得博士学位。\(^\text{[14]}\)

玻尔的论文具有开创性,但由于当时哥本哈根大学要求论文必须用丹麦语撰写,因此在斯堪的纳维亚以外地区鲜有人关注。1921 年,荷兰物理学家亨德里卡·约翰娜·范·李文独立推导出了玻尔论文中的一个定理,今天被称为玻尔–范·李文定理。\(^\text{[15]}\)
\begin{figure}[ht]
\centering
\includegraphics[width=6cm]{./figures/3f69c50087df3760.png}
\caption{玻尔与玛格丽特·讷鲁恩于 1910 年订婚时合影} \label{fig_NRSbr_2}
\end{figure}
1910 年,玻尔结识了数学家尼尔斯·埃里克·讷鲁恩的妹妹玛格丽特·讷鲁恩。\(^\text{[16]}\)玻尔于 1912 年 4 月 16 日退出丹麦国教会,并于同年 8 月 1 日在斯莱厄瑟市政厅与玛格丽特举行了民事婚礼。多年后,他的弟弟哈拉尔德在结婚前也同样退出了教会。\(^\text{[17]}\)

玻尔和玛格丽特育有六个儿子。\(^\text{[18]}\)长子克里斯蒂安于 1934 年在一次划船事故中去世,\(^\text{[19]}\)另一位儿子哈拉尔德有严重智力障碍,在四岁时被送到离家较远的机构安置,六年后因儿童脑膜炎去世。\(^\text{[20][18]}\)阿格·玻尔成为了一名成功的物理学家,并于 1975 年获得了与父亲相同的诺贝尔物理学奖。阿格的儿子维尔赫姆·A·玻尔是一位科学家,供职于哥本哈根大学\(^\text{[21]}\) 和美国国家衰老研究所。\(^\text{[22]}\)

汉斯(Hans [da])成为医生;埃里克(Erik [da])成为化学工程师;欧内斯特成为律师。\(^\text{[23]}\)与他的叔叔哈拉尔德一样,欧内斯特·玻尔也成为奥运运动员,曾代表丹麦参加 1948 年伦敦夏季奥运会曲棍球比赛。\(^\text{[24]}\)
\subsection{物理学}
\subsubsection{玻尔模型}
1911 年 9 月,玻尔在卡尔斯伯基金会的奖学金资助下前往英国,当时关于原子和分子结构的大部分理论工作都在英国进行。\(^\text{[25]}\)他拜访了剑桥大学三一学院和卡文迪许实验室的 J.J. 汤姆孙,听取了詹姆斯·金斯和约瑟夫·拉默关于电磁学的讲座,并做了一些阴极射线的研究,但未能给汤姆孙留下深刻印象。[26][27] 他在与年轻物理学家,如澳大利亚的威廉·劳伦斯·布拉格\(^\text{[28]}\)和新西兰的欧内斯特·卢瑟福的交流中取得了更大的收获。卢瑟福在 1911 年提出的原子小而集中的原子核模型对汤姆孙 1904 年提出的“葡萄干布丁模型”提出了挑战。\(^\text{[29]}\)卢瑟福邀请玻尔到曼彻斯特维多利亚大学进行博士后研究,\(^\text{[30]}\)在那里玻尔结识了乔治·德·赫维希和查尔斯·高尔顿·达尔文(玻尔称其为“真正达尔文的孙子”)。\(^\text{[31]}\)

玻尔于 1912 年 7 月回到丹麦准备婚礼,并在英格兰和苏格兰度蜜月。返回后,他成为哥本哈根大学的私人讲师,讲授热力学课程。马丁·克努森提名玻尔担任讲师一职,该提名于 1913 年 7 月获得批准,玻尔随后开始为医学生授课。\(^\text{[32]}\)他那三篇后来被称为“玻尔三部曲”的论文\(^\text{[30]}\),于当年 7 月、9 月和 11 月发表于《哲学杂志》上。\(^\text{[33][34][35][36]}\)他将卢瑟福的核结构与马克斯·普朗克的量子理论结合起来,创立了著名的玻尔原子模型。\(^\text{[34]}\)

行星式原子模型并非玻尔首创,但玻尔的处理方式具有创新性。\(^\text{[37]}\)他以 1912 年达尔文关于电子在 $\alpha$粒子与原子核相互作用中作用的论文为起点,\(^\text{[38][39]}\)提出了电子围绕原子核在量子化“稳定状态”轨道中运行以维持原子稳定的理论,但直到 1921 年的论文中他才展示出各元素的化学性质在很大程度上取决于其原子外层轨道电子数量。\(^\text{[40][41][42][43]}\)他提出电子可以从高能轨道跃迁到低能轨道,并在此过程中发射出一个离散能量量子的观点。这一观点成为后来被称为“旧量子论”的基础。\(^\text{[44]}\)
\begin{figure}[ht]
\centering
\includegraphics[width=6cm]{./figures/da391b4f7d2afe03.png}
\caption{玻尔氢原子模型:一个带负电的电子被限制在原子轨道上,围绕一个小而带正电的原子核旋转;当电子在轨道之间发生量子跃迁时,会伴随着一定量的电磁辐射的发射或吸收。} \label{fig_NRSbr_4}
\end{figure}
1885 年,约翰·巴尔末提出巴尔末系,用于描述氢原子可见光谱线:
$$
\frac{1}{\lambda} = R_H \left( \frac{1}{2^2} - \frac{1}{n^2} \right) \quad \text{其中 } n = 3,4,5,\ldots~
$$
其中,$\lambda$ 是被吸收或发射光的波长,$R_H$ 是 里德伯常数。\(^\text{[45]}\)
巴尔末公式得到了更多光谱线发现的验证,但在接下来的 30 年里,没有人能解释它为何成立。

在其“三部曲”中的第一篇论文中,玻尔能够从其模型推导出该公式:
$$
R_Z = \frac{2 \pi^2 m_e Z^2 e^4}{h^3}~
$$
其中,$m_e$ 是电子质量,$e$ 是电子电荷,$h$ 是普朗克常数,$Z$ 是原子序数(氢原子时 $Z=1$)。\(^\text{[46]}\)
\begin{figure}[ht]
\centering
\includegraphics[width=10cm]{./figures/f4f3923f5134291e.png}
\caption{20 世纪原子模型的演变:汤姆孙、卢瑟福、玻尔、海森堡/薛定谔} \label{fig_NRSbr_5}
\end{figure}
该模型面临的第一个障碍是皮克林系,这些光谱线并不符合巴尔末公式。当阿尔弗雷德·福勒就此质疑玻尔时,玻尔回答说,这些光谱线是由电离氦(即仅有一个电子的氦原子)产生的。事实证明,玻尔模型对这种离子同样适用。\(^\text{[46]}\)许多年长的物理学家,如汤姆孙、瑞利和亨德里克·洛伦兹并不喜欢这部“三部曲”,但包括卢瑟福、大卫·希尔伯特、阿尔伯特·爱因斯坦、恩里科·费米、马克斯·玻恩和阿诺德·索末菲在内的年轻一代视其为一次重大突破。\(^\text{[47][48]}\)爱因斯坦称玻尔的模型是“思想领域中最高形式的音乐性”。\(^\text{[49]}\)“三部曲”之所以被接受,完全是因为它能够解释其他模型无法解决的现象,并能够预测随后通过实验得到验证的结果。\(^\text{[50][51]}\)如今,玻尔的原子模型虽然已被更先进的模型取代,但仍是最知名的原子模型,因为它经常出现在高中物理和化学教材中。\(^\text{[52]}\)

玻尔并不喜欢给医学生授课。他后来承认自己并不是一名优秀的讲师,因为他需要在清晰与真实之间取得平衡。\(^\text{[53]}\)他决定返回曼彻斯特,那里卢瑟福为他提供了一份讲师职位,以接替任期已满的达尔文。玻尔接受了这一邀请,并向哥本哈根大学申请了休假。他首先与哥哥哈拉尔和姑妈汉娜·阿德勒在蒂罗尔度假期间开始休假。在那里,他拜访了哥廷根大学和慕尼黑的路德维希-马克西米利安大学,期间见到了索末菲,并就“三部曲”开展了研讨会。第一次世界大战在他们身处蒂罗尔期间爆发,这极大地增加了返回丹麦及玻尔和玛格丽特前往英国旅程的复杂性。他们于 1914 年 10 月抵达英国,并一直停留到 1916 年 7 月,此时玻尔已被任命为哥本哈根大学专为他设立的理论物理学讲席教授。与此同时,他的副教授职位被撤销,因此他仍然需要为医学生教授物理课程。新教授会被正式介绍给丹麦国王克里斯蒂安十世,国王在见到这位著名的足球运动员时表示非常高兴。\(^\text{[54]}\)
\subsubsection{物理研究所}
1917 年 4 月,玻尔开始筹建理论物理研究所。他获得了丹麦政府和嘉士伯基金会的支持,工业界和许多私人捐助者(其中很多是犹太人)也提供了大量资助。确立该研究所的立法于 1918 年 11 月通过。该研究所现被称为尼尔斯·玻尔研究所,于 1921 年 3 月 3 日正式开放,玻尔任所长。他的家人搬入了研究所一楼的公寓。\(^\text{[55][56]}\)在 1920 和 1930 年代,玻尔的研究所成为量子力学及相关领域研究者的中心,当时世界上大多数著名理论物理学家都曾在玻尔的研究所短暂停留。早期来访者包括来自荷兰的汉斯·克拉默斯、来自瑞典的奥斯卡·克莱因、来自匈牙利的乔治·德·海维西、来自波兰的沃伊切赫·鲁比诺维茨,以及来自挪威的斯文·罗斯兰。\(^\text{[57][58]}\) 玻尔被广泛赞誉为他们的平易近人的东道主和杰出的同行。克莱因和罗斯兰在研究所正式开放之前就已经发表了研究所的首篇出版物。\(^\text{[56]}\)
\begin{figure}[ht]
\centering
\includegraphics[width=6cm]{./figures/d77e022cbd31c107.png}
\caption{} \label{fig_NRSbr_6}
\end{figure}
玻尔模型在氢和单电子离子化氦中表现良好,这给爱因斯坦留下了深刻印象\(^\text{[59][60]}\),但无法解释更复杂元素的情况。到 1919 年,玻尔开始放弃电子绕核运转的想法,转而发展用于描述电子的启发式方法。稀土元素因化学性质极为相似而成为化学家分类中的一大难题。1924 年,沃尔夫冈·泡利发现泡利不相容原理,为玻尔的模型奠定了坚实的理论基础,这是一个重要的发展。随后玻尔能够宣称,当时尚未被发现的第 72 号元素并非稀土元素,而是具有与锆相似化学性质的元素。(自 1871 年以来,元素便是通过化学性质被预测并发现的\(^\text{[61]}\))。法国化学家乔治·乌尔班立即对玻尔提出挑战,声称自己已发现第 72 号稀土元素,并将其命名为“celtium”。在哥本哈根的研究所,德克·科斯特和乔治·德·海维西接下挑战,决心证明玻尔正确、乌尔班错误。由于一开始就对未知元素的化学性质有清晰认识,大大简化了搜索过程。他们在哥本哈根矿物学博物馆的样品中寻找具有锆特性的元素,并很快找到了它。这种元素被命名为铪(Hafnium,Hafnia 是哥本哈根的拉丁名),结果发现它比黄金更常见。\(^\text{[62][63]}\)

1922 年,玻尔因“在研究原子结构及其辐射方面的贡献”而被授予诺贝尔物理学奖。\(^\text{[64]}\)这一奖项既认可了他的“三部曲”论文,也承认了他在量子力学这一新兴领域的早期领先工作。在诺贝尔演讲中,玻尔向听众全面概述了当时关于原子结构的已知内容,其中包括他提出的对应原理。该原理指出,用量子理论描述的系统在量子数趋于无穷大的极限下会再现经典物理学的行为。\(^\text{[65]}\)

1923 年,阿瑟·康普顿发现康普顿散射,使大多数物理学家相信光由光子组成,并且在电子与光子的碰撞中能量和动量守恒。1924 年,玻尔、克拉默斯和在哥本哈根研究所工作的美国物理学家约翰·C·斯莱特提出了玻尔–克拉默斯–斯莱特理论(BKS 理论)。这更像是一项计划,而非完整的物理理论,因为其中提出的想法并未被量化地详细展开。BKS 理论成为最后一次试图在旧量子理论框架下理解物质与电磁辐射相互作用的尝试,在该框架中,量子现象是通过对电磁场的经典波动描述施加量子限制来处理的。\(^\text{[66][67]}\)

使用“虚振子”在吸收和发射频率(而非玻尔轨道上的(不同)表观频率)下对入射电磁辐射下的原子行为进行建模,促使马克斯·玻恩、维尔纳·海森堡和克拉默斯探索了不同的数学模型,并最终导致矩阵力学的发展,这也是现代量子力学的第一个形式。BKS 理论还引发了对旧量子理论基础困难的新讨论和关注。\(^\text{[68]}\)BKS 理论中最具争议的部分——动量和能量并不一定在每次相互作用中守恒,而只是统计意义上守恒——很快被瓦尔特·玻特和汉斯·盖革的实验证明与实验结果相冲突。\(^\text{[69]}\)鉴于这些结果,玻尔告诉达尔文:“除了尽可能光荣地为我们的革命性努力举办葬礼,别无他法。”\(^\text{[70]}\)
\subsubsection{量子力学}
1925 年 11 月,乔治·乌伦贝克和塞缪尔·古兹米特引入自旋(spin)是一个重要的里程碑。次月,玻尔前往莱顿,参加庆祝亨德里克·洛伦兹获得博士学位 50 周年的活动。当玻尔乘坐的火车在汉堡停靠时,沃尔夫冈·泡利和奥托·斯特恩来见他,询问他对自旋理论的看法。玻尔指出,他对电子与磁场之间的相互作用存在疑虑。当他到达莱顿后,保罗·埃伦费斯特和阿尔伯特·爱因斯坦告诉玻尔,爱因斯坦已利用相对论解决了这一问题。玻尔随后让乌伦贝克和古兹米特将这一修正纳入他们的论文中。因此,当玻尔在返回途中在哥廷根见到维尔纳·海森堡(和帕斯夸尔·约尔当时,用他自己的话说,他已经成为“电子磁性福音的传道者”。\(^\text{[71]}\)
\begin{figure}[ht]
\centering
\includegraphics[width=10cm]{./figures/b3ea22118f7ed16c.png}
\caption{1927 年索尔维会1927 年 10 月,比利时布鲁塞尔索尔维会议合影。玻尔位于中排右侧,站在马克斯·玻恩旁边。} \label{fig_NRSbr_7}
\end{figure}
海森堡首次来哥本哈根是在 1924 年,随后于 1925 年 6 月返回哥廷根,并不久后发展出了量子力学的数学基础。当他在哥廷根将结果展示给马克斯·玻恩时,玻恩意识到这些结果最好使用矩阵来表达。这项工作引起了英国物理学家保罗·狄拉克的关注,\(^\text{[72]}\)狄拉克于 1926 年 9 月来到哥本哈根停留了六个月。奥地利物理学家埃尔温·薛定谔也在 1926 年到访,他试图通过波动力学用经典术语解释量子物理,这给玻尔留下了深刻印象,玻尔认为这“极大地提高了数学上的清晰性与简洁性,使其相较于以往任何形式的量子力学都代表了一次巨大的进步”。\(^\text{[73]}\)

当克拉默斯于 1926 年离开研究所,前往乌得勒支大学担任理论物理学教授时,玻尔安排海森堡回到哥本哈根,接替克拉默斯在哥本哈根大学的讲师职位。\(^\text{[74]}\)海森堡于 1926 年至 1927 年期间在哥本哈根担任大学讲师及玻尔的助理。\(^\text{[75]}\)

玻尔逐渐确信光既表现为波也表现为粒子,且在 1927 年,实验证实了德布罗意的假设,即物质(如电子)也表现出波动性。\(^\text{[76]}\)他提出了互补性哲学原理:事物在不同实验框架下可能具有看似互相排斥的属性,例如既是波又是粒子流。\(^\text{[77]}\)他认为专业哲学家并未完全理解这一原理。\(^\text{[78]}\)

1927 年 2 月,海森堡提出了不确定性原理的初版,并以通过伽马射线显微镜观测电子的思想实验进行展示。玻尔对海森堡的论证并不满意,因为其论证仅要求测量扰动已存在的属性,而不是更激进地认为电子的属性离开测量情境无法讨论。在 1927 年 9 月科莫会议上提交的论文中,玻尔强调海森堡的不确定性关系可以从有关光学仪器分辨率的经典考虑中导出。\(^\text{[79]}\)玻尔认为,要真正理解互补性的含义,需要“更深入的探讨”。\(^\text{[80]}\)爱因斯坦偏好经典物理的决定论,而非他本人亦参与奠基的概率性新量子物理。量子力学新颖特性引发的哲学问题成为广泛讨论的热点话题。爱因斯坦与玻尔在这些问题上终生保持着善意的争论。\(^\text{[81]}\)

1914 年,嘉士伯啤酒厂继承人卡尔·雅各布森将其宅邸(嘉士伯荣誉官邸,现称嘉士伯学院)捐赠,作为终身荣誉住所(丹麦语:Æresbolig),供在科学、文学或艺术领域对丹麦贡献最卓著者使用。哲学家哈拉尔·霍夫丁是第一位入住者,他于 1931 年 7 月去世后,丹麦皇家科学院将此住所分配给了玻尔。玻尔和家人在 1932 年搬入此处。\(^\text{[82]}\)1939 年 3 月 17 日,他当选为丹麦皇家科学院院长。\(^\text{[83]}\)

到 1929 年,β 衰变现象促使玻尔再次提出应放弃能量守恒定律,但沃尔夫冈·泡利提出的假想中微子以及 1932 年中子的发现提供了另一种解释。这促使玻尔在 1936 年提出了复合核理论,用以解释中子如何被原子核俘获。在该模型中,原子核可以像液滴一样发生变形。他与丹麦物理学家弗里茨·卡尔卡合作开展此项研究,但卡尔卡于 1938 年突然去世。\(^\text{[84][85]}\)

1938 年 12 月,奥托·哈恩发现核裂变(莉泽·迈特纳随后做出理论解释)在物理学界引发了强烈关注。玻尔将这一消息带到美国,并于 1939 年 1 月 26 日与费米共同主持了第五届华盛顿理论物理会议的开幕。\(^\text{[86]}\)当玻尔告诉乔治·普拉切克这一发现解决了超铀元素的所有谜团时,普拉切克告诉他仍有一个未解之谜:铀的中子俘获能与其衰变不匹配。玻尔思考了几分钟后,便对普拉切克、莱昂·罗森菲尔德和约翰·惠勒宣布:“我已经完全理解了。”\(^\text{[87]}\)基于其液滴模型,玻尔得出结论,主要由铀-235同位素而非更丰富的铀-238在热中子作用下引发裂变。1940 年 4 月,约翰·R·邓宁证明了玻尔的结论正确无误。\(^\text{[86]}\)与此同时,玻尔与惠勒发展了理论处理方法,并在 1939 年 9 月发表了题为《核裂变机制》的论文。\(^\text{[88]}\)
\subsection{哲学}
海森堡曾说玻尔“首先是位哲学家,而非物理学家”。\(^\text{[89]}\)玻尔曾阅读19世纪丹麦基督教存在主义哲学家索伦·克尔凯郭尔的著作。理查德·罗德斯在《制造原子弹》中认为玻尔是通过霍夫丁受到克尔凯郭尔的影响。[90] 1909年,玻尔将克尔凯郭尔的《人生道路上的几个阶段》作为生日礼物寄给弟弟,并在随信中写道:“这是我唯一能寄回家的东西;但我不认为能轻易找到比它更好的……我甚至觉得这是我读过的最令人愉悦的作品之一。”玻尔喜欢克尔凯郭尔的语言和文学风格,但曾提到自己对克尔凯郭尔的哲学有一些不同意见。[91] 玻尔的一些传记作者认为,这种分歧源于克尔凯郭尔对基督教的倡导,而玻尔是无神论者。[92][93][94]

关于克尔凯郭尔对玻尔哲学与科学影响的程度存在一些争议。大卫·法弗霍尔特认为克尔凯郭尔对玻尔的工作影响甚微,并直观接受玻尔声称自己与克尔凯郭尔观点不合的说法,[95] 而扬·法耶则认为,一个人可以不同意某一理论的具体内容,但仍接受其总体前提和结构。[96][91]

玻尔曾担任哲学丛书《世界视野》编委会成员,该系列出版了多部哲学著作。[97]
