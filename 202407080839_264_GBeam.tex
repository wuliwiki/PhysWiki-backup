% 高斯光束 3
% keys 光学|激光
% license Usr
% type Art

\begin{issues}
\issueDraft
\end{issues}

你可能在很多地方都听说过高斯光束,例如在光学课上,或者激光器的说明书里,但它们不会完整介绍高斯光束,因为这是高等光学的内容,即使高斯光束的应用非常普遍。本文将介绍高斯光束最重要的特性,也就是经过透镜等光学器件后,光斑的空间分布维持不变的这一特性,涉及近轴近似和惠更斯定理。%本文的主要参考是Svelto、Orazio等人的著作《Principles of Lasers》,以及汪凯戈的著作《高等物理光学》。

\subsubsection{前言}
光斑当然是应该有空间分布的。一开始学习的平面波的表达式是$\bvec E(x,y,z,t)=\bvec {E_0}\E^{-\I\omega t}$,你可能觉得$\bvec E_0$可能是个常数,这显然是错误的,因为根据\enref{电场的能量}{EEng},单位空间内平面波的能量是$u=\frac{1}{2}\epsilon_0 \bvec E_0^{\;2}$,这是个常数,那么对$u$全空间积分之后,就能得知该平面波携带了无穷大的能量,这显然是不合理的。那么$\bvec E_0$会不会只局限在某个局域呢?比如光斑只在一个圆形的截面里,就像我们日常看到的那样?这也是不行的,因为光会衍射,如果某个截面的光是局域的,那么就相当于在光路上放一个小孔,光就会在孔后形成衍射图样,当然,衍射图样是全空间的。这说明真实情况下光一定会有一个全空间分布。

\subsection{近轴近似和ABCD矩阵}
在谈论高斯光束之前需要先谈到一般情况下光线如何在光学器件上传播。为简单起见,这里只涉及几何光学的内容,也就是先忽略光斑的空间分布,把光看成是一条光线。如\autoref{fig_GBeam_1} 是一个任意的光学器件,例如是凸透镜,玻璃片或者什么都没有。
\begin{figure}[ht]
\centering
\includegraphics[width=12cm]{./figures/d7d39c0c7c232184.png}
\caption{光学器件的定义。光线在$z_1$平面以$\bvec r_1$角度入射,在$z_2$平面以$\bvec r_2$角度出射。} \label{fig_GBeam_1}
\end{figure}
我们定义了几个参数以表示这些光学器件对光线的作用,也就是图中的$r_1,\theta_1$和$r_2,\theta_2$.

\subsection{近轴近似下的衍射}
占位符。

\subsection{低模高斯光束}
占位符。

\subsection{低模高斯光束的传播}
占位符。

\subsection{高模高斯光束}
占位符。