% 基数
% 集合论|cardinal number|映射|对应|数量|元素|number|element|map|set|元素数量

本文节选自《小时百科系列教材》中的《代数学》,为了良好的阅读体验和完整的,建议阅读原文.

\subsection{引言}

当我们讨论问题的时候,一定是在围绕某些对象、概念进行的,这些对象或概念就可以构成一个集合.讨论如何计算销售苹果的数量的时候,我们是在整数的集合上讨论运算;绘画一张地图的时候,我们是在坐标点的集合上讨论各点的性质,如地势高低、地形特征等等.原始的集合概念几乎在数学诞生之初就模糊地存在于人们的脑海中了,只是一直以来都局限于原始的直觉,通常是有限的集合,而无穷集合则被认为是不在严谨的数学框架中的.人们一直都没有意识到“集合”这一概念有什么深入研究的意义,直到康托尔(Cantor)迈出了探索无穷的第一步.

如何研究无穷呢?要回答这个问题,我们首先要回答另一个问题:我们可以对无穷提出什么样的问题?最基本的一个问题自然是:什么叫做无穷?

有限的集合,就是所包含的元素数量是有限的.如果$A$是一个有限集合,那么不管$A$含有几个元素,一个,两个,五个,十个,一千零二十四个,我们总能找到一个非负整数来对应$A$的元素数量.也可以这么想象:从$A$中依次拿出元素,拿出的元素不放回;从零开始,每拿一个计数增加一个;当元素全部拿出来的时候,计数的结果就是$A$的元素数量.如果拿的顺序不一样,计数结果会有变化吗?为什么呢\footnote{结果当然不会有变化,但“为什么”似乎没那么容易回答.当然,如果你熟悉ZFC公理的推演或者熟悉了本节的内容,应该可以给出严谨的解释,不过我们目前可以默认这是成立的.}?

我们自然会推想,无限集合就是指元素数量无限.但是我们并没有定义“无限”这样一个数字,就没法像有限集合一样赋予一个非负整数作为它的元素数量.怎么严格描述什么是无限呢?方法其实很简单:不论你选择哪一个非负整数$k$,从集合$A$中拿走$k$个元素以后仍然有剩下的元素,那么我们就说$A$是一个无限集合\footnote{有的理论里也会定义“无限”这一存在,比如复平面的无穷远点,但那个实际上是拓扑空间的一点紧化的结果.在打基础的阶段,应该扎实理解“无限”是一种性质而不是一个存在,比如此处,无限集合是指具有“拿走有限多元素后都还有剩余”这一性质的集合.对这一基本概念理解扎实以后,才能更准确地涉及“无穷”的各种概念,包括“无穷远点”和“一点紧化”等.}.

为了方便讨论,我们把集合所包含的元素数量称作“\textbf{基数(cardinal)}”.用这个术语来说,比较集合的大小就是在比较它们的基数的大小.

有限集合非常容易比较大小:给定有限集合$A$和$B$,同时从两个集合中拿走元素,每次分别拿一个,那么先被拿完的那个集合就比较小;如果同时被拿完,那么两个集合一样大.这也就是整数比大小的原理.

从集合中一个一个拿走元素的行为,可以叫做“\textbf{计数}”、“\textbf{数数}”等.这种方式可以很好地比较甚至计算有限集合的元素数量,但对于无限集合却束手无策,因为按照无限集合的定义,无论你数了多少个元素出来,依然会有剩下的元素,换句话说,怎么数都数不完,那该怎么办呢?

康托尔意识到我们可以用“一一对应”的思想来比较集合的数量.这其实就是计数的延伸:如果两个集合之间的元素可以一一对应,即每一个元素都对应对方的唯一一个元素,那么就说这两个集合是相等的.如果无论怎么构造和集合$A$和$B$的对应,总是无法让每个$B$的元素都对应到$A$上,也就是说$B$总有多余元素,那么我们就说$B$是大于$A$的.

上述描述看起来很绕口,因此我们引出了“\textbf{映射}\upref{map}”的概念,来专门描述集合间的元素对应关系.



\subsection{可数基数}


有了映射的概念,我们就可以比较两个集合的元素数量了:如果集合$A$和$B$之间\textbf{能够}建立双射,那么我们就说$A$和$B$的基数相同.对于有限集合,由于我们可以给每个元素进行编号,很容易就能看出“两个有限集合的基数相同,当且仅当它们的元素数量相同”.我把这段话写成如下定义:

\begin{definition}{}
如果$A$是有限集合,那么称$A$的元素数量为其\textbf{基数(cardinal number)},记为$\abs{A}$.对于任意集合$A$和$B$,如果$A$到$B$之间\textbf{存在一个}双射,那么定义$\abs{A}=\abs{B}$.
\end{definition}


对于有限集合,基数相同的时候,单射必然是双射——无论你怎么从一个集合到另一个集合进行元素对应,只要保持“被映射过的元素不再参与映射”,那么最后总能涉及到所有元素,也就是说只要保持单射,最后总能得到满射.

但是,对于无限集合,情况就不一样了.考虑全体正整数的集合$\mathbb{Z}^+$,以及全体正偶数的集合$2\mathbb{Z}^+$,如果把映射$f:\mathbb{Z}^+\rightarrow2\mathbb{Z}^+$定义为$f(n)=2n$,那么容易验证$f$是一个双射\footnote{检查一下,每个正偶数都被映射到了(满射),并且只被映射了一次(单射).},也就是说,$\abs{\mathbb{Z}^+}=\abs{2\mathbb{Z}^+}$,即全体正整数和正偶数的数量是一样的,尽管前者是后者的真子集!这就是无限集合所特有的性质,部分可以等于整体.

\begin{definition}{}
如果集合$A$到集合$B$存在一个单射,那么记$\abs{A}\leq\abs{B}$,称作“$A$的基数小于等于$B$的基数”;如果$A$到$B$有一个满射,那么记$\abs{A}\geq\abs{B}$,称作“$A$的基数大于等于$B$的基数”.\footnote{如果$A$到$B$存在一个单射,那么反过来$B$到$A$一定存在一个满射.方法很简单,如果$f:A\rightarrow B$是单射,那么可以构造$g:B\rightarrow A$,使得对于$b\in f(A)$,有$g(b)=f^{-1}(b)$,然后随便挑一个$a\in A$,让剩下的所有$B-f(A)$中的元素都映射到$a$上,这样$g$就是满射了.同样地,如果$A$到$B$存在一个满射,那么$B$到$A$存在一个单射;不过对于任意集合,这个单射的存在性依赖于\textbf{选择公理}.}
\end{definition}

根据以上定义,如果$\abs{A}\leq\abs{B}$且$\abs{A}\geq\abs{B}$,那么就有$\abs{A}=\abs{B}$.

\begin{definition}{}
记全体正整数的集合$\mathbb{Z}^+$的基数为$\aleph_0$,且称任何基数\textbf{小于等于}$\aleph_0$的集合为\textbf{可数集(countable set)}.
\end{definition}

可数集这一定义比较顾名思义.按照可数集、基数以及单射的定义,对于任意可数集$A$,可以给每个元素挨个分配一个正整数,就好像给它们排序后报数一样,虽然不一定报得完,但是每个元素总可以报一次的.也就是说,可数集是“可以计数”的集合,尽管并不一定能数出全部元素,但是只要数下去,任何元素都可以数到.更准确地说,对于可数集$A$,可以给定数数的规则(从$\mathbb{Z}+$到$A$的满射),使得若随机点出一个$A$的元素,这个元素总能被数到至少一次.

我们特地定义了可数集,说明存在不可数的集合.以下所说的幂集就可以用来构造不可数集.















