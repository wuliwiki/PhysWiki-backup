% 亥姆霍兹自由能
% 自由能

\pentry{热力学第一定律\upref{Th1Law}理想气体的内能\upref{IdgEng}}

亥姆霍兹自由能,简称自由能,常用 $A$ 或 $F$ 表示(本文将用 $F$ 表示).自由能是一个热力学态函数,\textbf{对于一个恒温恒容的封闭热力学系统,自由能的增量总是小于等于外界对系统做功的量.}在可逆过程中,自由能的增量等于外界对系统做功的量;也就是说如果外界对系统不做功,在恒温恒容下任意可逆过程都不改变系统的自由能.所以我们有如下定义.

我们将 $F$ 定义为 $U-TS$,$U$ 为系统的内能,$S$ 为熵.由热力学第一定律(\upref{Th1Law}\autoref{Th1Law_eq2})可得
\begin{equation}
\dd F=-S\dd T-p\dd V
\end{equation}

假如考虑表面张力\upref{sftens},则液体的自由能可以写作
\begin{equation}
\dd F=-S\dd T-p \dd V+\sigma \dd S
\end{equation}
\subsection{自由能判据}
由前面的分析可知,对于恒温恒容系统,当外界对系统不做功时,系统总是趋向自由能减小的状态.所以我们可以得出稳定平衡状态的判据.

等温等容系统处在稳定平衡状态的必要和充分条件为:
\begin{equation}
\Delta F>0
\end{equation}
这里的 $\Delta F$ 指的是在等温等容条件下系统可能发生的任何虚变动引起的自由能改变量.

如果对 $F$ 作泰勒展开,准确到二级,则有 $\Delta F=\delta F+\frac{1}{2}\delta^2 F$.此时平衡条件为 $\delta F=0$,稳定性条件为 $\delta^2 F>0$.这一点与熵判据\upref{equcri}的原理一样.