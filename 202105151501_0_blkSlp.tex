% 滑块和运动斜面问题
% keys 滑块|斜面|人船模型|加速度|动量守恒

\begin{issues}
\issueDraft
\end{issues}

\pentry{人船模型} % 链接未完成

在滑块斜面问题的基础上, 如果我们假设斜面质量为 $M$, 与地面无摩擦, 那么滑块相对斜面的加速是多少呢?

令 $x, y$ 为滑块水平方向和竖直方向移动的距离. $X$ 为斜面水平方向移动的距离.
\begin{equation}\label{blkSlp_eq2}
x = \frac{M}{M + m}l\cos\theta
\qquad
y = -l\sin\theta
\qquad
X = -\frac{m}{M + m}l\cos\theta
\end{equation}
以下介绍三种方法, 都可以解得滑块相对斜面的加速度为
\begin{equation}\label{blkSlp_eq1}
a = \ddot l = \frac{g\sin\theta(M+m)}{M + m\sin^2\theta}
\end{equation}

\subsection{受力分析法}
\addTODO{图}
\begin{figure}[ht]
\centering
\includegraphics[width=10cm]{./figures/blkSlp_1.pdf}
\caption{受力分析} \label{blkSlp_fig1}
\end{figure}
\addTODO{使用高中的方法, 用 “拉格朗日方程法” 中的变量 $x, y, X$ 列方程}

\subsection{非惯性系法}
\pentry{惯性力\upref{Iner}}
这是最简单的方法. 在斜面的参考系, 滑块会受到向右的惯性力 $-m\ddot X$, 所以沿斜面向下使用牛顿第二定律\upref{New3}得
\begin{equation}
-m\ddot X\cos\theta + mg\sin\theta = m\ddot l
\end{equation}
把\autoref{blkSlp_eq2} 代入解得\autoref{blkSlp_eq1}.

\subsection{拉格朗日方程法}
\pentry{拉格朗日方程\upref{Lagrng}}
考虑动量守恒, 这个系统只有一个自由度, 即一个广义坐标 $l$. 拉格朗日量等于
\begin{equation}
\begin{aligned}
L = T - V &= \frac12 m(\dot x^2 + \dot y^2) + \frac12 M \dot X^2 - mgy\\
&= \frac{1}{2}m \qty(1 - \frac{m\cos^2\theta}{M + m}) \dot l^2 + mg\sin\theta \cdot l
\end{aligned}
\end{equation}
代入拉格朗日方程(\autoref{Lagrng_eq1}~\upref{Lagrng})
\begin{equation}
\dv{t} \pdv{L}{\dot l} = \pdv{L}{l}
\end{equation}
解得\autoref{blkSlp_eq1}.
