% 2019 年计算机学科专业基础综合全国联考卷
% 2019 年计算机学科专业基础综合全国联考卷

\subsection{一、单项选择题}
1~40小题, 每小腿2分, 共80分.下列每题输出的四个选项中,只有一个选项符合试题要求.

1.设$n$是描述问题规模的非负整数,下列程序段的时间复杂度是
\begin{lstlisting}[language=cpp]
x=0;
while(n>=(x+1)*(x+1))
    x=x+1;
\end{lstlisting}
A. $O(logn)$  $\quad$  B.$O(n)$  $\quad$   C.$O(n)$  $\quad$  D.$O(n^2)$

2. 若将一棵树$T$转化为对应的二叉树$BT$,则下列对$BT$的遍历中,其遍历序列与$T$的后根遍历序列相同的是 \\
A.先序遍历  $\quad$  B.中序遍历  $\quad$  C.后序遍历  $\quad$ D.按层遍历

3. 对$n$个互对不相同的符号进行哈夫曼编码.若生成的哈夫曼树共有115个结点,则$n$的值 \\
A.56  $\quad$  B.57  $\quad$  C. 58  $\quad$  D.60

4. 在任意一棵非平衡二叉树(AVL树)$T_1$中,删除某结点$v$之后形成平衡二叉树$T_2$,再将$v$插入$T_2$形成平衡二叉树$T_3$. 下列关于$T_1$与$T_3$的叙述中,正确的是 \\
I .若$v$是$T_1$的叶结点,则$T_1$与$T_3$可能不相同 \\
II .若$v$不是$T_1$的叶结点,则$T_1$与$T_3$定不相同  \\
III.若$v$不是$T_1$的叶结点,则$T_1$与$T_3$一定相同 \\
A.仅I  $\quad$  B.仅H  $\quad$ C.仅I , II  $\quad$  D.仅I 、III

5.下图所示的AOE网表示一项包含8个活动的工程.活动d的最早开始时间和最迟开始时间分别是
\begin{figure}[ht]
\centering
\includegraphics[width=12.5cm]{./figures/CSN19_1.png}
\caption{第5题图} \label{CSN19_fig1}
\end{figure}
A.1  $\quad$  B.2  $\quad$  C.3  $\quad$  D.4

6.用有向无环图描述表达式$(x+y)*(x+y)/x$,需要的顶点个数至少是 \\
A.5  $\quad$  B. 6  $\quad$  C.8  $\quad$  D.9



11.设外存上有120个初始归并段,进行12路归并时,为实现最佳归
④
g-6
20
并,需要补充的虚段个数是
b-4
c=6
A.1
B.2
C.3
D.4
h=9
12.下列关于冯.诺依曼结构计算机基本思想的叙述中,错误的是
c=8
f=10
A.程序的功能都通过中央处理器执行指令实现
A.3和7
B.12和12
C.12和14.
D.15和15.
B.指令和数据都用二进制表示,形式上无差别
6. 用有向无环图描述表达式(x+y) * ((x+y)/x) ,需要的顶点个数至
C.指令按地址访问,数据都在指令中直接给出
少是
D.程序执行前,指令和数据需预先存放在存储器中
A. 5
B. 6
C.8
D.9
13.考虑以下C语言代码:
7.选择一个排序算法时.除算法的时空效率外,下列因素中,还需要考
unsigned short usi = 65535;
虑的是
short si = usi;
1.数据的规模
II.数据的存储方式
执行上述程序段后,si的值是
m.算法的稳定性
IV.数据的初始状态
A. -1
B. -32767
C. -32768
D. -65535
A.仅M
B.仅1、I
14.下列关于缺页处理的叙述中,错误的是
C.仅1I、山、IV
D. I、I、W、IV
A.缺页是在地址转换时CPU检测到的-种异常
8.现有长度为11且初始为空的散列表HT,散列的数是H(hey)= hey
B.缺页处理由操作系统提供的缺页处理程序来完成
% 7,采用线性探查(线性探测再散列)法解决冲突将关键字序列
c.缺页处理程序根据页故障地址从外存读人所缺失的页
87 ,40 ,30,6,11,22,98 ,20依次插人到HT后,HT查找失败的平均查
D).缺页处理完成后回到发生缺页的指令的下一条指令执行
找长度是
15. 某计算机采用大端方式,按字节编址.某指令中操作数的机器数
A.4
B.5.25
C.6
D.6.29
为1234 FFOOH,该操作数采用基址寻址方式,形式地址(用补码表
9.设主串T=“abaabaabeabaabe",模式申S=" abaabe" ,采用KMP算法
示)为FFI2H,基址寄存器内容为F000000,则该操作数的LSB
进行模式匹配,到匹配成功时为止,在匹配过程中进行的单个字符
(最低有效字节)所在的地址是
间的比较次数是
A. F000FFI2H
B. F000 FF15H
A.9
B.10
C.12
D. 15
C. EFFF FF12H
D. EFFF FFI5H
10.排序过程中,对尚未确定最终位置的所有元索进行一遍处理称为
16.下列有关处理器时钟脉冲信号的叙述中,错误的是
一“趟".下列序列中,不可能是快速排序第二趟结果的是
A.时钟脉冲信号由机器脉冲源发出的脉冲信号经整形和分频后
A.5,2,16,12,28 ,60,32,72
形成
B.2,16,5,28,12 ,60,32,72
B.时钟脉冲信号的宽度称为时钟周期,时钟周期的倒数为机器
C.2,12,16,5,28,32,72,60
主频
D.5,2,12,28,16,32,72 ,60
C.时钟周期以相邻状态单元间组合逻辑电路的最大延迟为基准
