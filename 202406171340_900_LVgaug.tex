% 速度规范
% keys 长度规范|速度规范|波函数|规范变换|薛定谔方程|麦克斯韦方程组
% license Xiao
% type Tutor

\pentry{长度规范\nref{nod_LenGau},库仑规范(量子力学)\nref{nod_CouGau}}{nod_5928}

\footnote{本文参考 \cite{Bransden}。}本文使用\enref{原子单位制}{AU}。和长度规范中的思路一样, 我们只在使用\enref{偶极子近似}{DipApr}时讨论\textbf{速度规范(velocity gauge)}。 用角标 $V$ 表示速度规范, 先从规范不变的哈密顿算符(\autoref{eq_QMEM_1})出发
\begin{equation}\label{eq_LVgaug_2}
H = \frac{\bvec p^2}{2m} - \frac{q}{2m} (\bvec A \vdot \bvec p + \bvec p \vdot \bvec A)
+ \frac{q^2}{2m} \bvec A^2 + q \varphi + V(\bvec r)~,
\end{equation}
由于偶极子近似下 $\bvec A$ 与位置无关,$\bvec p\vdot\bvec A = \bvec A\vdot\bvec p$。 长度规范的思路是把上式中的 $\bvec A^2$ 消去。 对库仑规范使用规范变换(\autoref{eq_QMEM_5})
\begin{equation}\label{eq_LVgaug_3}
\Psi_C(\bvec r, t) = \exp(\I q\chi_V)\Psi_V(\bvec r, t)~,
\end{equation}
\begin{equation}\label{eq_LVgaug_4}
\chi_V(t) = -\frac{q}{2m} \int_{-\infty}^t \bvec A_C^2(t') \dd{t'}~,
\end{equation}
得
\begin{equation}\label{eq_LVgaug_1}
\bvec A_V = \bvec A_C - \grad \chi_V = \bvec A_C~.
\end{equation}
可见\textbf{速度规范下的矢势和库仑规范的相同}, 以下统一记为 $\bvec A$。 这使得广义动量(\autoref{eq_QMEM_6})也和库仑规范的相同, 
\begin{equation}
\bvec p_V = \bvec p_C =  m \bvec v + q\bvec A = -\I \grad~.
\end{equation}
再看标势的变换:
\begin{equation}\label{eq_LVgaug_5}
\varphi_V = \varphi_C + \pdv{\chi_V}{t} = - \frac{q}{2m} \bvec A^2~,
\end{equation}
可见于库仑规范相比, 速度规范的标势 $\varphi_V$ 与位置无关,只随时间变化。

\autoref{eq_LVgaug_1} 和\autoref{eq_LVgaug_5} 带入\autoref{eq_LVgaug_2} 可以消去 $\bvec A^2$ 项得
\begin{equation}
H_V = \frac{\bvec p^2}{2m} - \frac{q}{m} \bvec A \vdot \bvec p + V(\bvec r)~.
\end{equation}
薛定谔方程为
\begin{equation}\label{eq_LVgaug_7}
H_V \Psi_V = \I \pdv{t} \Psi_V~.
\end{equation}


对比\autoref{eq_LVgaug_3} 和\autoref{eq_LenGau_1}  得长度规范与速度规范中的波函数转换关系为
\begin{equation}\label{eq_LVgaug_6}
\Psi_V = \exp[\I q(\chi_L - \chi_V)]\Psi_L = \exp[\I q \bvec A\vdot \bvec r + \I\frac{q^2}{2m}\int_{-\infty}^t \bvec A^2\dd{t'}] \Psi_L~.
\end{equation}

速度规范在数值解薛定谔方程时有一定的优势,若带电粒子的波包在外电场中加速, 其动量增量和 $-q\bvec A$ 增量成正比, 会产生 $\exp(-\I q\bvec A\vdot\bvec r)$ 相位因子, 而\autoref{eq_LVgaug_6} 恰好可以将其抵消,使波包的空间频率减小,便于用较少的空间格点表示波函数。
