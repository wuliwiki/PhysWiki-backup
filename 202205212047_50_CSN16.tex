% 2016 年计算机学科专业基础综合全国联考卷
% keys 2016 计算机 考研 全国联考

\subsection{一、单项选择题}
1~40小题,每小题2分,共80分.下列每题给出的四个选项中.只有一个选项符合试题要求.

1.已知表头元素为c的单链表在内存中的存储状态如下表所示.
\begin{table}[ht]
\centering
\caption{第1题表}\label{CSN16_tab1}
\begin{tabular}{|c|c|c|}
\hline
地址 & 元素 & 链接地址 \\
\hline
1000H & a & 1010H \\
\hline
1004H & b & 100CH \\
\hline
1008H & c & 1000H \\
\hline
100CH & d & NULL \\
\hline
1010H & e & 1004H \\
\hline
1014H &   &  \\
\hline
\end{tabular}
\end{table}
现将f存放于1014H处并插入到单链表中,若f在逻辑上位于a和e之间,则a,e,f的“链接地址”依次是 \\
A.1010H,1014H,1004H $\qquad$ B.1010H,1004H,1014H \\
C.1014H,1010H,1004H $\qquad$ D.1014H,1004H,1010H

2.已知一个带有表头结点的双向循环链表L,结点结构为 \\
\begin{table}[ht]
\centering
\caption{第2题表}\label{CSN16_tab2}
\begin{tabular}{|c|c|c|}
\hline
prev & data & next \\
\hline
\end{tabular}
\end{table}
其中,prev和next分别是指向其直接前驱和直接后继结点的指针.现要删除指针p所指的结点,正确的语句序列是 \\
A. p->next->prev=p->prev; p->prev->next=p->prev; free (p); \\
B. p->next->prev=p->next; p->prey-> next=p->next; free (p); \\
C. p->next->prev=p->next; p->prev->next=p->prev; free (p); \\
D. p-> next-> prey=p->prey; p->prev->next=p->next; free (p);

3.设有如下图所示的火车车轨,入口到出口之间有n条轨道,列车的行进方向均为从左至右,列车可驶入任意一条轨道.现有编号为1~9的9列列车,驶入的次序依次是8,4,2,5,3,9,1,6,7.若期望驶出的次序依次为1~9,则n至少是 \\
\begin{figure}[ht]
\centering
\includegraphics[width=14.25cm]{./figures/CSN16_1.png}
\caption{第3题图} \label{CSN16_fig1}
\end{figure}
A. 2 $\qquad$ B.3 $\qquad$ C.4 $\qquad$ D.5

4.有一个$100$阶的三对角矩阵$M$,其元素$m_{i,j}$($1$≤$i$≤$100$,$1$≤$j$≤$100$)按行优先次序压缩存入下标从$0$开始的一维数组Ⅳ中.元素$m_{30}$,$30$在$N$中的下标是 \\
A.86 $\qquad$ B.87 $\qquad$ C.88 $\qquad$ D.89

5.若森林F有15条边、25个结点,则F包含树的个数是 \\
A.8 $\qquad$ B.9 $\qquad$ C.10 $\qquad$ D.11

6.下列选项中,不.是下图深度优先搜索序列的是 \\
\begin{figure}[ht]
\centering
\includegraphics[width=5cm]{./figures/CSN16_2.png}
\caption{请添加图片描述} \label{CSN16_fig2}
\end{figure}
A.$V_1$,$V_5$,$V_4$,$V_3$,$V_2$ \\
B.$V_1$,$V_3$,$V_2$,$V_5$,$V_4$ \\
C.$V_1$,$V_2$,$V_5$,$V_4$,$V_3$ \\
D.$V_1$,$V_2$,$V_3$,$V_4$,$V_5$

7.若将n个顶点e条弧的有向图采用邻接表存储,则拓扑排序算法的时间复杂度是 \\
A.$O(n)$ $\qquad$ B.$O(n+e)$ $\qquad$ C.$O(n^2)$ $\qquad$ D.$O(n\times e)$

8.使用迪杰斯特拉(Dijkstra)算法求下图中从顶点1到其他各顶点的最短路径,依次得到的各最短路径的目标顶点是 \\
A.5,2,3,4,6 $\qquad$ B.5,2,3,6,4 \\
C.5,2,4,3,6 $\qquad$ D.5,2,6,3,4

9.在有n(n>1000)个元素的升序数组A中查找关键字x.查找算法的伪代码如下所示. \\
\begin{lstlisting}[language=cpp]
k=0;
while(k<n且A[k]<x)k=k+3;
if(k<n且A[k]==x)查找成功;
else if(k-1<n且A[k-1]==x)查找成功;
  else if(k-2<n且A[k-2]==x)查找成功;
    else查找失败;
\end{lstlisting}
本算法与折半查找算法相比,有可能具有更少比较次数的情形是 \\
A.当x不在数组中 $\qquad$ B.当x接近数组开头处 \\
C.当x接近数组结尾处 $\qquad$ D.当x位于数组中间位置

10.$B^+$树\textbf{不同}于B树的特点之一是 \\
A.能支持顺序查找 \\
B.结点中含有关键字 \\
C.根结点至少有两个分支 \\
D.所有叶结点都在同一层上

11.对10TB的数据文件进行排序,应使用的方法是 \\
A.希尔排序 $\qquad$ B.堆排序 \\
C.快速排序 $\qquad$ D.归并排序

12.将高级语言源程序转换为机器级目标代码文件的程序是 \\
A.汇编程序 $\qquad$ B.链接程序 \\
C.编译程序 $\qquad$ D.解释程序

13.有如下C语言程序段: \\
\begin{lstlisting}[language=cpp]
short si=-32767;
unsigned short usi=si;
\end{lstlisting}
执行上述两条语句后,usi的值为
A.-32767 $\qquad$ B.32767 $\qquad$ C.32768 $\qquad$ D.32769

14.某计算机字长为32位,按字节编址,采用小端(Little Endian)方式存放数据.假定有一个double型变量,其机器数表示为1122 3344 5566 7788H,存放在0000 8040H开始的连续存储单元中,则存储单元0000 8046H中存放的是 \\
A.22H $\qquad$ B.33H $\qquad$ C.66H $\qquad$ D.77H

15.有如下C语言程序段: \\
\begin{lstlisting}[language=cpp]
for(k=0;k<1000;k++)
  a[k]=a[k]+32;
\end{lstlisting}
若数组a及变量k均为int型,int型数据占4 B,数据Cache采用直接映射方式、数据区大小为1 KB、块大小为16 B,该程序段执行前Cache为空,则该程序段执行过程中访问数组a的Cache缺失率约为 \\
A.1.25% $\qquad$ B.2.5% $\qquad$ C.12.5% $\qquad$ D.25%

16.某存储器容量为64 KB,按字节编址,地址4000H~5FFFH为ROM 区,其余为RAM区.若采用8 K×4位的SRAM芯片进行设计,则需要该芯片的数量是 \\
A.7 $\qquad$ B.8 $\qquad$ C.14 $\qquad$ D.16

17.某指令格式如下所示.
\begin{table}[ht]
\centering
\caption{第17题表}\label{CSN16_tab3}
\begin{tabular}{|c|c|c|c|}
\hline
OP & M & I & D \\
\hline
\end{tabular}
\end{table}
其中M为寻址方式,I为变址寄存器编号,D为形式地址.若采用先变址后间址的寻址方式,则操作数的有效地址是 \\
A.I+D $\qquad$ B.(I)+D $\qquad$ C.((I)+D) $\qquad$ D.((I))+D

18.某计算机主存空间为4 GB,字长为32位,按字节编址,采用32位定长指令字格式.若指令按字边界对齐存放,则程序计数器(PC)和指令寄存器(IR)的位数至少分别是 \\
A.30、30 $\qquad$ B.30、32 $\qquad$ C.32、30 $\qquad$ D.32、32

19.在无转发机制的五段基本流水线(取指、译码/读寄存器、运算、访存、写回寄存器)中,下列指令序列存在数据冒险的指令对是 \\
I1:add R1, R2, R3 ;(R2)+(R3)→R1 \\
I2:add R5, R2, R4 ;(R2)+(R4)→R5 \\
I3:add R4, R5, R3 ;(R5)+(R3)→R4 \\
I4:add R5, R2, R6 ;(R2)+(R6)→R5 \\
A.I1和I2 B.I2和I3 C.I2和I4 D.I3和I4

20.单周期处理器中所有指令的指令周期为一个时钟周期.下列关于单周期处理器的叙述中,\textbf{错误}的是 \\
A.可以采用单总线结构数据通路 \\
B.处理器时钟频率较低 \\
C.在指令执行过程中控制信号不变 \\
D.每条指令的CPI为1

21.下列关于总线设计的叙述中,错误的是 \\
A.并行总线传输比串行总线传输速度快 \\
B.采用信号线复用技术可减少信号线数量 \\
C.采用突发传输方式可提高总线数据传输率 \\
D.采用分离事务通信方式可提高总线利用率

22.异常是指令执行过程中在处理器内部发生的特殊事件,中断是来自处理器外部的请求事件.下列关于中断或异常情况的叙述中,\textbf{错误}的是 \\
A.“访存时缺页”属于中断 \\
B.“整数除以0”属于异常 \\
C.“DMA传送结束”属于中断 \\
D.“存储保护错”属于异常

23.下列关于批处理系统的叙述中,正确的是 \\
Ⅰ.批处理系统允许多个用户与计算机直接交互 \\
Ⅱ.批处理系统分为单道批处理系统和多道批处理系统 \\
Ⅲ.中断技术使得多道批处理系统的I/O设备可与CPU并行工作 \\
A.仅Ⅱ、Ⅲ B.仅Ⅱ C.仅Ⅰ、Ⅱ D.仅Ⅰ、Ⅲ

24.某单CPU系统中有输入和输出设备各1台,现有3个并发执行的作业,每个作业的输入、计算和输出时间均分别为2 ms、3 ms和4 ms,且都按输入、计算和输出的顺序执行,则执行完3个作业需要的时间最少是 \\
A.15 ms $\qquad$ B.17 ms $\qquad$ C.22 ms $\qquad$ D.27 ms

25.系统中有3个不同的临界资源R1、R2和R3,被4个进程p1、p2、p3及p4共享.各进程对资源的需求为:p1申请R1和R2,p2申请R2和R3,p3申请R1和R3,p4申请R2.若系统出现死锁,则处于死锁状态的进程数至少是 \\
A.1 $\qquad$ B.2 $\qquad$ C.3 $\qquad$ D.4

26.某系统采用改进型CLOCK置换算法,页表项中字段A为访问位,M为修改位.A=0表示页最近没有被访问,A=1表示页最近被访问过.M=0表示页没有被修改过,M=1表示页被修改过.按(A,M)所有可能的取值,将页分为四类:(0,0)、(1,0)、(0,1)和(1,1),则该算法淘汰页的次序为 \\
A.(0,0),(0,1),(1,0),(1,1) \\
B.(0,0),(1,0),(0,1),(1,1) \\
C.(0,0),(0,1),(1,1),(1,0) \\
D.(0,0),(1,1),(0,1),(1,0)

27.使用TSL(Test and Set Lock)指令实现进程互斥的伪代码如下所示. \\
\begin{lstlisting}[language=cpp]
do {
  ……
  while(TSL(&lock));
  critical section;
  lock=FALSE;
  ……
} while(TRUE);
\end{lstlisting}
下列与该实现机制相关的叙述中,正确的是 \\
A.退出临界区的进程负责唤醒阻塞态进程 \\
B.等待进入临界区的进程不会主动放弃CPU \\
C.上述伪代码满足“让权等待”的同步准则 \\
D.while(TSL(&lock))语句应在关中断状态下执行

28.某进程的段表内容如下所示. \\
\begin{table}[ht]
\centering
\caption{第28题表}\label{CSN16_tab4}
\begin{tabular}{|c|c|c|c|c|}
\hline
段号 & 段长 & 内存起始地址 & 权限 & 状态 \\
\hline
0 & 100 & 6000 & 只读 & 在内存 \\
\hline
1 & 200 & - & 读写 & 不在内存 \\
\hline
2 & 300 & 4000 & 读写 & 在内存 \\
\hline
\end{tabular}
\end{table}
当访问段号为2、段内地址为400的逻辑地址时,进行地址转换的结果是 \\
A.段缺失异常 $\qquad$ B.得到内存地址4400 \\
C.越权异常 $\qquad$ D.越界异常

29.某进程访问页面的序列如下所示.
\begin{figure}[ht]
\centering
\includegraphics[width=14.25cm]{./figures/CSN16_3.png}
\caption{第29题图} \label{CSN16_fig3}
\end{figure}
若工作集的窗口大小为6,则在£时刻的工作集为 \\
A.{6,0,3,2} $\qquad$ B.{2,3,0,4}
c.{0,4,3,2,9} $\qquad$ D.{4,5,6,0,3,2}

30.进程P1和P2均包含并发执行的线程,部分伪代码描述如下所示. \\
\begin{lstlisting}[language=cpp]
//进程P1
  int x=0:
  Thread1( )
  { int a:
    a=1; x+=1;
  }
  Thread2( )
  { int a:
    a=2;x+=2;
  }
\end{lstlisting}

\begin{lstlisting}[language=cpp]
//进程P2
  int x=0:
  Thread3( )
  { int a;
    a=x; x+=3;
  }
  Thread4( )
  { int b;
    b=x;x+=4;
  }
\end{lstlisting}
下列选项中,需要互斥执行的操作是 \\
A.a=1与a=2 $\qquad$ B.a=x与b=x \\
c.x+=1与x+=2 $\qquad$ D.x+=1与x+=3

31.下列关于SPOOLing技术的叙述中,\textbf{错误}的是 \\
A.需要外存的支持 \\
B.需要多道程序设计技术的支持 \\
C.可以让多个作业共享一台独占设备 \\
D.由用户作业控制设备与输入/输出井之间的数据传送

32.下列关于管程的叙述中,\textbf{错误}的是 \\
A.管程只能用于实现进程的互斥 \\
B.管程是由编程语言支持的进程同步机制 \\
C.任何时候只能有一个进程在管程中执行 \\
D.管程中定义的变量只能被管程内的过程访问

题33~41均依据题33~41图回答.

33.在OSI参考模型中,R1、Switch、Hub实现的最高功能层分别是 \\
A.2、2、1 $\qquad$ B.2、2、2 $\qquad$ C.3、2、1 $\qquad$ D.3、2、2

34.若连接R2和R3链路的频率带宽为8 kHz,信噪比为30 dB,该链路实际数据传输速率约为理论最大数据传输速率的50%,则该链路的实际数据传输速率约是 \\
A.8 kbps $\qquad$ B.20 kbps $\qquad$ C.40 kbps $\qquad$ D.80 kbps
\begin{figure}[ht]
\centering
\includegraphics[width=14.25cm]{./figures/CSN16_4.png}
\caption{第33~41题图} \label{CSN16_fig4}
\end{figure}

35.若主机H2向主机H4发送1个数据帧,主机H4向主机H2立即发送一个确认帧,则除H4外,从物理层上能够收到该确认帧的主机还有 \\
A.仅H2 $\qquad$ B.仅H3 $\qquad$ C.仅H1、H2 $\qquad$ D.仅H2、H3

36.若Hub再生比特流过程中,会产生1.535μs延时,信号传播速度为200 m/μs,不考虑以太网帧的前导码,则H3与H4之间理论上可以相距的最远距离是 \\
A.200 m $\qquad$ B.205 m $\qquad$ C.359 m $\qquad$ D.512 m

37.假设R1、R2、R3采用RIP协议交换路由信息,且均已收敛.若R3检测到网络201.1.2.0/25不可达,并向R2通告一次新的距离向量,则R2更新后,其到达该网络的距离是 \\
A.2 $\qquad$ B.3 $\qquad$ C.16 $\qquad$ D.17

38.假设连接R1、R2和R3之间的点对点链路使用201.1.3.x/30地址,当H3访问Web服务器S时,R2转发出去的封装HTTP请求报文的IP分组的源IP地址和目的IP地址分别是 \\
A.192.168.3.251,130.18.10.1 $\qquad$ B.192.168.3.251,201.1.3.9 \\
C.201.1.3.8,130.18.10.1 $\qquad$ D.201.1.3.10,130.18.10.1

39.假设H1与H2的默认网关和子网掩码均分别配置为192.168.3.1和255.255.255.128,H3与H4的默认网关和子网掩码均分别配置为192.168.3.254和255.255.255.128,则下列现象中可能发生的是 \\
A.H1不能与H2进行正常IP通信 \\
B.H2与H4均不能访问Internet \\
C.H1不能与H3进行正常IP通信 \\
D.H3不能与H4进行正常IP通信

40.假设所有域名服务器均采用迭代查询方式进行域名解析.当H4访问规范域名为www.abc.xyz.com的网站时,域名服务器201.1.1.1在完成该域名解析过程中,可能发出DNS查询的最少和最多次数分别是 \\
A.0,3 $\qquad$ B.1,3 $\qquad$ C.0,4 $\qquad$ D.1,4

\subsection{二、综合应用题}

41~47小题,共70分.

41.(9分)假设题33~41图中的H3访问Web服务器S时,S为新建的TCP连接分配了20 KB(K=1 024)的接收缓存,最大段长MSS=1 KB,平均往返时间RTT=200 ms.H3建立连接时的初始序号为100,且持续以MSS大小的段向S发送数据,拥塞窗口初始阈值为32 KB;S对收到的每个段进行确认,并通告新的接收窗口.假定TCP连接建立完成后,S端的TCP接收缓存仅有数据存入而无数据取出.请回答下列问题. \\
(1)在TCP连接建立过程中,H3收到的S发送过来的第二次握手TCP段的SYN和ACK标志位的值分别是多少?确认序号是多少? \\
(2)H3收到的第8个确认段所通告的接收窗口是多少?此时H3的拥塞窗口变为多少?H3的发送窗口变为多少? \\
(3)当H3的发送窗口等于0时,下一个待发送的数据段序号是多少?H3从发送第1个数据段到发送窗口等于0时刻为止,平均数据传输速率是多少(忽略段的传输延时)? \\
(4)若H3与S之间通信已经结束,在t时刻H3请求断开该连接,则从t时刻起,S释放该连接的最短时间是多少?

42.(8分)如果一棵非空k(k≥2)叉树T中每个非叶结点都有k个孩子,则称T为正则后k树.请回答下列问题并给出推导过程. \\
(1)若T有m个非叶结点,则T中的叶结点有多少个? \\
(2)若T的高度为h(单结点的树h=1),则T的结点数最多为多少个?最少为多少个?

43.(15分)已知由$n(n \geqslant 2)$个正整数构成的集合$A=\{a_k|0\leqslant k<n\}$,将其划分为两个不相交的子集$A_1$和$A_2$,元素个数分别是$n_1$和$n_2$,$A_1$和$A_2$中元素之和分别为$S_1$和$S_2$.设计一个尽可能高效的划分算法,满足|$n_1-n_2$|最小且|$S_1-S_2$|最大.要求: \\
(1)给出算法的基本设计思想. \\
(2)根据设计思想,采用C或C++语言描述算法,关键之处给出注释. \\
(3)说明你所设计算法的平均时间复杂度和空间复杂度.

44.(9分)假定CPU主频为50 MHz,CPI为4.设备D采用异步串行通信方式向主机传送7位ASCII字符,通信规程中有1位奇校验位和1位停止位,从D接收启动命令到字符送入I/O端口需要0.5 ms.请回答下列问题,要求说明理由. \\
(1)每传送一个字符,在异步串行通信线上共需传输多少位?在设备D持续工作过程中,每秒钟最多可向I/0端口送入多少个字符? \\
(2)设备D采用中断方式进行输入/输出,示意图如下:
\begin{figure}[ht]
\centering
\includegraphics[width=14.25cm]{./figures/CSN16_5.png}
\caption{第45题图} \label{CSN16_fig5}
\end{figure}
请回答下列问题. \\
(1)图中字段A~G的位数各是多少?TLB标记字段B中存放的是什么信息? \\
(2)将块号为4099的主存块装入到Cache中时,所映射的Cache 组号是多少?对应的H字段内容是什么? \\
(3)Cache缺失处理的时间开销大还是缺页处理的时间开销大?为什么? \\
(4)为什么Cache可以采用直写(Write Through)策略,而修改页面内容时总是采用回写(Write Back)策略?

46.(6分)某进程调度程序采用基于优先数(priority)的调度策略,即选择优先数最小的进程运行,进程创建时由用户指定一个nice作为静态优先数.为了动态调整优先数,引入运行时间cpuTime和等待时间waitTime,初值均为0.进程处于执行态时,cpuTime定时加1,且waitTime置0;进程处于就绪态时,cpuTime置0,waitTime定时加1.请回答下列问题. \\
(1)若调度程序只将nice的值作为进程的优先数,即priority=nice,则可能会出现饥饿现象,为什么? \\
(2)使用nice、cpuTime和waitTime设计一种动态优先数计算方法,以避免产生饥饿现象,并说明waitTime的作用.

47.(9分)某磁盘文件系统使用链接分配方式组织文件,簇大小为4 KB.目录文件的每个目录项包括文件名和文件的第一个簇号,其他簇号存放在文件分配表FAT中. \\
(1)假定目录树如下图所示,各文件占用的簇号及顺序如下表所示,其中dir、dir1是目录,file1、file2是用户文件.请给出所有目录文件的内容. \\
\begin{figure}[ht]
\centering
\includegraphics[width=14.25cm]{./figures/CSN16_6.png}
\caption{第47题图} \label{CSN16_fig6}
\end{figure}
(2)若FAT的每个表项仅存放簇号,占2个字节,则FAT的最大长度为多少字节?该文件系统支持的文件长度最大是多少? \\
(3)系统通过目录文件和FAT实现对文件的按名存取,说明file1的106、108两个簇号分别存放在FAT的哪个表项中. \\
(4)假设仅FAT和dir目录文件已读入内存,若需将文件dir/dir1/file1的第5000个字节读入内存,则要访问哪几个簇?

\subsection{参考答案}
\subsubsection{一、单项选择题}
1. D  $\qquad$ 2. D $\qquad$ 3. C $\qquad$ 4.B $\qquad$ 5. C \\
6. D $\qquad$ 7. B $\qquad$ 8. B $\qquad$ 9. B $\qquad$ 10.A \\
11. D $\qquad$ 12. C $\qquad$ 13. D $\qquad$ 14. A $\qquad$ 15. C \\
16. C $\qquad$ 17. C $\qquad$ 18. B $\qquad$ 19. B $\qquad$ 20. A \\
21. A $\qquad$ 22. A $\qquad$ 23. A $\qquad$ 24. B $\qquad$ 25. C \\
26. A $\qquad$ 27. B $\qquad$ 28. D $\qquad$ 29. A $\qquad$ 30. C \\
31. D $\qquad$ 32. A $\qquad$ 33. C $\qquad$ 34. C $\qquad$ 35. D \\
36. B $\qquad$ 37. B $\qquad$ 38. D $\qquad$ 39. C $\qquad$ 40. C

\subsubsection{二、综合应用题}
41.【答案要点】 \\
(1)第二次握手TCP段的SYN=1,(1分)ACK=1;(1分)确认序号是101.(1分) \\
(2)H3收到的第8个确认段所通告的接收窗口是12 KB;(1分)此时H3的拥塞窗口变为9 KB;(1分)H3的发送窗口变为9 KB.(1分) \\
(3)当H3的发送窗口等于0时,下一个待发送段的序号是20 K+101=20×1024+101=20581;(1分)H3从发送第1个段到发送窗口等于0时刻为止,平均数据传输速率是20 KB/(5×200 ms)=20 KB/s=20.48 kbps.(1分) \\
(4)从t时刻起,S释放该连接的最短时间是:1.5×200 ms=300 ms.(1分)

42.【答案要点】 \\
(1)根据定义,正则$k$叉树中仅含有两类结点:叶结点(个数记为$n_0$)和度为$k$的分支结点(个数记为$n_k$).树$T$中的结点总数$n=n_0+n_k=n_0+m$.树中所含的边数$e=n-1$,这些边均为$m$个度为$k$的结点发出的,即$e=m\times k$.整理得:$n_0+m=m\times k+1$,故$n_0=(k-1)\times m+1$.(3分) \\
(2)高度为$h$的正则$k$叉树$T$中,含最多结点的树形为:除第$h$层外,第$1$到第$h-1$层的结点都是度为$k$的分支结点,而第$h$层均为叶结点,即树是“满”树.此时第j(1≤j≤h)层结点数为kj-1,结点总数M1为:
