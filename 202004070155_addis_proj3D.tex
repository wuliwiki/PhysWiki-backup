% 三维投影
% 立体几何|投影|透视|画图

% 参考 https://en.wikipedia.org/wiki/3D_projection

当我们在平面上画三维物体时, 我们需要某种投影算法把物体上的每个点对应到平面上的一点. 以下介绍两种常用的方法, 一种是\textbf{平行投影(parallel projection)}, 另一种是\textbf{透视投影(perspective projection)}.

% 首先来比较两个图(图未完成: 左图是长方体的平行投影, 右图是长方体的透视投影)
% 参考 https://construct3.ideas.aha.io/ideas/C3-I-754

\subsection{平行投影}
顾名思义, 平行投影是指在空间中指定一个方向(如图未完成), 将三维物体上的每一点沿着该方向投影到与该方向垂直的平面上. 工程制图中的正视图, 侧视图等都属于平行投影. 这种投影的特点是, 空间中的任意两条平行线的投影仍然是平行线. 平行投影是线性的, 即一个线段若伸长若干倍, 那么它的投影也会按照同样的比例伸长.

\subsection{透视投影}
当人眼或相机观察一个三维物体时, 使用的是透视投影.

(用一张图介绍原理)

我们在平面后方取一个固定点 $F$ 称为\textbf{焦点}, 要把物体上任意一点 $P$ 投影到平面上, 先作直线 $PF$, 该直线与平面的交点 $P'$ 就是投影后的点. 我们在平面上建立直角坐标系, 把平面上离焦点最近的点定义为平面坐标的原点. 透视投影不是线性的, 但可以保证直线的投影仍然为直线.

为什么透视投影要这样定义? 我们先思考一种非常简单的相机, 即小孔成像相机. 根据光的直线传播, 物体上的某点发出的(或反射的)光线只有经过小孔才能投影到孔后面的平面. 如果按照这个模型, 我们在计算透视投影时应该把焦点定义在屏幕之前, 然而这么做有一个缺点就是投影后平面上的像是倒像. 所以为了方便起见, 我们保持焦点不变, 但是把屏幕平移到焦点之前(焦距也不变), 容易看出这样做的唯一改变就是把倒像变为正像, 即点 $P'$ 的两个平面坐标分别取相反数. % (未完成)需要一张图比较焦点在前和焦点在后

如果我们将相机上的小孔改为焦距为 $f$ 的小凸透镜, 并假设物距远大于焦距, 那么根据成像公式, 凸透镜和平面(即底片)的距离就是焦距 $f$. 根据凸透镜成像原理, 物体一点在底片上的像仍然会过凸透镜的中点% 图未完成
, 所以使用凸透镜的相机和使用小孔成像的相机得到的投影是相同的. 至于人眼, 人眼成像的原理和相机基本一致, 虽然视网膜并不是一个平面, 得到的成像是扭曲的, 但大脑在处理图象是会自动纠正这种扭曲(例如我们看到的直线仍然是直的).

\subsection{计算方法}
无论是哪种投影, 我们通常建立两个坐标系, 一个是\textbf{世界坐标系(world frame)} $S$, 一个是\textbf{相机坐标系}\upref{CamMdl}\textbf{(camera frame)} $S'$. 一个基本的问题就是将一个系里面的坐标变换到另一个系中的坐标. 这可以通过空间旋转矩阵\upref{Rot3D}, 再加上一个平移完成(平移的矢量是两个坐标系原点之间的位移).

把所有要投影的点 $P_i$ ($i = 1, \dots, N$) 的在世界系中的坐标(假设已知)记为一个 3 行 $N$ 列矩阵, 矩阵的第 $i$ 列就是列矢量 $\bvec P_i = (x_i, y_i, z_i)\Tr$. 将 3 乘 3 的旋转矩阵左乘该矩阵, 再给每行加上常数进行平移即可. 完成后, 我们就得到了相机系中的坐标矩阵, 每一列是 $\bvec P_i' = (x_i', y_i', z_i')\Tr$

当我们做平行投影时, 可以令相机系的 $x$-$y$ 平面为投影平面, 投影方向为 $-\uvec z$. 即 $(x_i', y_i', z_i')$ 投影后为 $(x_i', y_i')$.

当我们做透视投影时, 可以令原点为焦点, $(0, 0, f)$ 为平面的中点, 平面与 $z$ 轴垂直. 相机系中的某点 $(x_i', y_i', z_i')$ 投影后变为 $(x_i' f/z_i', y_i' f/z_i')$, 即每个分量乘以 $f/z_i'$, 使得 $z$ 分量等于 $f$.

(未完成: 另外开一篇文章分享平行投影和透视投影的 Matlab 代码)

\subsection{3D 艺术画}
见 3D 艺术画\upref{art3D}.
