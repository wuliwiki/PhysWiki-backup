% 国际单位制

% 物理常数|国际单位|测量
% 未完成: 只是写了基本单位的定义, 还要提及 “物理单位前缀\upref{UniPre}”

\footnote{参考 Wikipedia \href{https://en.wikipedia.org/wiki/International_System_of_Units}{相关页面}.}\textbf{国际单位制(SI Units)}是一套国际通用的单位标准.

本文采用 2019 年 5 月 20 日开始生效的新国际单位标准, 该标准中的数值也被称为 2018 CODATA 推荐值(Committee on Data for Science and Technology, 科学技术数据委员会). 在新标准中, 所有基本单位都可以通过 7 个预定义的物理常数来测定(\autoref{Consts_tab1}~\upref{Consts}). 以下的数值除了有特殊说明, 都是精确值(无限位小数用省略号表示), 不存在误差.

\subsubsection{时间:秒(s)}
一秒等于铯(Cs)原子 133 基态的超精细能级之间的跃迁辐射的电磁波周期的 $9,192,631,770$ 倍. 

说明: 我们知道原子中的电子具有不同的能级, 当电子从一个能级跃迁到一个更低的能级时, 会放出一个光子. 光子的频率为 $\nu  = \varepsilon /h$,   其中 $\varepsilon $ 是光子的能量, $h$ 为普朗克常数.

\subsubsection{长度:米(m)}
真空中, 光在 $1/299792458$ 秒内传播的距离.

说明: 由于真空中的光速是物质和信息能传播的最快速度(见狭义相对论\upref{SpeRel} 相关内容), 且在任何参考系中都相同, 所以可以作为一个精确的标准. 结合秒的定义, 就可以通过实验得到一米的长度. 根据米的定义, 一秒中光可以在真空中传播
\begin{equation}
c = 299792458 \Si{m/s}
\end{equation}

\subsubsection{质量:千克(kg)}
千克的定义需要使得普朗克常数精确等于 $h = 6.62607015\times10^{-34}\Si{Js}$.

说明: 2019 年 5 月开始, 千克根据普朗克常数定义(见量子力学\upref{QMIntr}相关内容). 这个定义可以类比“米”的定义(使光速精确地等于 $299792459\Si{m/s}$): $h$ 的单位 $\Si{Js}$ 也可以表示为 $\Si{kg\cdot m^2/s}$, 我们已经定义了 “米” 和 “秒”, 所以通过测量普朗克常数, 我们就可以定义千克.

历史上, 千克最初在 1795 年被定义为一升水的质量, 但在实际操作中会遇到许多困难使结果不太精确. 1799 年使用国际公斤原器的质量来定义, 并复制若干份分别存放, 但经过长时间后被发现和复制品存在细微误差.

\subsubsection{力:牛顿(N)}
等效于 $\Si{kg \cdot m s^{-2}}$. 等于使 $1\Si{kg}$ 物体获得 $1\Si{m/s^2}$ 加速度的力.

说明: 该定义符合牛顿第二定律(\autoref{New3_eq1}~\upref{New3}).

\subsubsection{压强:帕斯卡(Pa)}
使 $1\Si{m^2}$ 面积受力为 $1\Si{N}$ 的压强.

\subsubsection{能量:焦耳(J)}
等效于 $\Si{kg\cdot ms^{-2}}$. $1\Si{N}$ 的恒力将受力物体沿力的方向移动 $1\Si{m}$, 做功为 $1\Si{J}$.

\subsubsection{电荷:库仑(C)}
库伦的定义使得每个电子的电荷精确等于 $1.602176634 \times 10^{-10} \Si{C}$.

\subsubsection{电势/电压:伏特(V)}
等效于 $\Si{J/C}$ 或者 $\Si{kg\cdot m^2 s^{-2} C^{-1}}$. 伏特的定义使得 $1\Si{C}$ 的电荷增加 $1\Si{V}$ 电势, 需要 $1\Si{J}$ 的能量.

\subsubsection{电场}
等效于 $\Si{V/m}$ 或 $\Si{N/C}$ 或 $\Si{kg\cdot m s^{-2} C^{-1}}$.

\subsubsection{电容:法拉(F)}
等效于 $C/V$ 或者 $\Si{C^2 s^2 kg^{-1} m^{-2}}$. 法拉是电容量的单位, 一个 $1\Si{F}$ 的电容器两端施加 $1\Si{V}$ 电压, 可以储存 $1\Si{C}$ 净电荷.

\subsection{电容率}
单位 $\Si{F/m}$.

\subsubsection{电流:安培(A)}
每秒钟流过某截面的的静电荷等于一库仑库仑.

\subsubsection{磁场:特斯拉(T)}
等效于 $\Si{kgC^{-1}s^{-1}}$. 可以由洛伦兹力(\autoref{Lorenz_eq1}~\upref{Lorenz})或安培力(\autoref{FAmp_eq1}~\upref{FAmp})来定义.

\subsubsection{电感:亨利(H)}
(未完成) 电感的单位.

\subsection{磁导率}
单位 $\Si{H/m}$.

\subsubsection{温度:开尔文(K)}
开尔文温度的定义应该使得玻尔兹曼常数精确等于
\begin{equation}
k_B = 1.3806505\times 10^{-23} \Si{J/K}
\end{equation}
例如, 理想气体中分子平均动能为 $\ev{E_k} = 3 k_B T/2$, 理论上 我们可以根据测其动能来定义温度.
