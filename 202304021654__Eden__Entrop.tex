% 熵
% 热力学|玻尔兹曼|熵|统计力学

\pentry{理想气体状态方程\upref{PVnRT}}
在热力学和统计力学中, \textbf{熵(entropy)}用于描述系统的无序程度, 是一个状态量, 通常记为 $S$。 例如若已知理想气体的 $P, V, n, T$ 等状态量, 就可以确定它的熵\upref{MacroS}。

\subsection{回顾卡诺循环}

卡诺循环\upref{Carnot}由两个等温过程,两个绝热过程构成。在 $P$-$V$ 图上由两条等温线和绝热线组成。对于卡诺热机(卡诺制冷机有类似的结论),系统经历 $P$-$V$ 图上一个顺时针的过程。在这个过程中系统对外做功为 $P$-$V$ 图上曲线围成的面积,在等温膨胀过程中系统吸热 $Q_1$,在等温压缩过程中系统放热 $-Q_2$(注意这里我们取负号,来表示系统在放出热量),由热力学第一定律\upref{Th1Law},有
\begin{equation}
Q_1+Q_2=W
\end{equation}
由卡诺循环的公式\autoref{eq_Carnot_4}~\upref{Carnot},可得\footnote{这里与卡诺热机词条\upref{Carnot}中不同的地方在于,系统吸热用正号表示,放热用符号表示,所以符号要做相应的调整。}
\begin{equation}
\frac{Q_1}{T_1}+\frac{Q_2}{T_2}=0
\end{equation}
绝热过程中 $\delta Q=0$,而等温过程中温度不变。上式可以转化为积分形式:
\begin{equation}
\oint \frac{\delta Q}{T}=0
\end{equation}

\subsection{宏观定义}

熵是一个系统的状态参量,它的增量为
\begin{equation}
\mathrm{d} S = \left . \frac{\delta Q}{T}\right |_{\text{可逆}}
\end{equation}
其中,$\delta Q$ 是可逆吸热,也就是说系统应无限地接近平衡态。

根据这个定义,经过任一卡诺循环\upref{Carnot},墒均回到初始值:
\begin{equation} \label{eq_Entrop_1}
\sum{\text{d}S_i=0}
\end{equation}
然而,要成为真正意义上的「状态参量」的桂冠,\autoref{eq_Entrop_1} 应该对任何循环都成立,而不是只对卡诺循环成立。换言之,我们想要
\begin{equation}
\oint \dd S =\oint \frac{\delta Q}{T}=0
\end{equation}
对于所有循环均成立。这恰恰是正确的。这个证明基于这样的事实: 热不能全部变为功而不产生其他效果。因为它意味着 $\eta=1$,违背了卡诺的结论。

\begin{figure}[ht]
\centering
\includegraphics[width=8cm]{./figures/a8353c8d9a836dc3.pdf}
\caption{系统进行准静态循环} \label{fig_Entrop_1}
\end{figure}

系统在 $i$ 段吸热 $\Delta Q_i$,该热量来自工作于系统现在的温度与热库 $T_0$ 之间的卡诺制冷机。制冷机所需要的功是 $\Delta W_i$。若 $\Delta Q_i<0$,例如 $j$ 点处,热量由卡诺热机出系统井向热库 $T_0$ 放出该热量。

现在,我们证明,对于任意循环都有 $\oint \mathrm d S =0 $。设系统经过如图所示的循环,图中有个温度为 $T_0$ 的辅助热库。在循环上取一小段过程 $i$。若此过程中的吸热为 $\Delta Q_i$,则设想它是由一个卡诺制冷机供给的,这个制冷机工作于热库 $T_0$ 和系统此时的温度 $T_i$ 之间。工作过程中,制冷机从热库 $T_0$ 吸收的热量为 $\Delta Q_{0i}$,所需的功为 $\Delta W_i$。卡诺制冷机满足
\begin{equation}
\frac{\Delta Q_{0i}}{T_0}=\frac{\Delta Q_i}{T_i}
\end{equation}
将之对闭合回路求和(在有些段,例如图中的 $j$ 段,$\Delta Q_i$ 和 $\Delta Q_{0i}$,可以是负值),得
\begin{equation}
\frac{Q_0}{T_0}=\oint{\frac{\delta Q}{T}}
\end{equation}
式中,$Q_0$ 是从那个热库吸收热量的总值。如果 $Q_0$ 为零,我们就求得了所要的结果
\begin{equation}
\oint{\frac{\delta Q}{T}=\oint{\text{d}S=0}}
\end{equation}

我们将看到,的确如此(表示 $\mathrm dS $ 时,我们用的是 $\delta Q/T$,而不是 $\Delta Q/T$,因为该比值代表微分。)

如果 $Q_0>0$,则意味着热库失去了热量。因为系统和所有辅助的卡诺机都已经复原,根据能量守恒,这部分热量一定由卡诺机和我们的系统全部转化为了等量的功,即 $\eta=1$,而这是不可能的。如果 $Q_0<0$,我们可以使整个过程反向进行(因为对于卡诺机和我们的循环,所有步骤都是可逆的),这样的 $Q_0$ 符号就为正,出现如同前面一样的矛盾。

即使没有跟我们一起进行上面的证明,你也一定注意到了,如果要用 $\delta Q/T=\mathrm d S$ 将 $\delta Q$ 和 $\mathrm dS$ 联系在一起,那么交换的热量 $\delta Q$ 必须是可逆的。

在上面的推导过程中,我们设置了卡诺制冷机和卡诺热机,它们都是可逆过程。如果不可逆,则最终的等号要改为小于等于号,即\textbf{克劳修斯不等式}:
\begin{equation}
\oint \frac{\delta Q}{T}\le 0
\end{equation}
对于可逆过程,等号成立。

在我们了解熵 $S $ 的含义前,先利用下面这个公式,实际计算几个过程的熵变:
\begin{equation}
\left. \text{d}S=\frac{\delta Q}{T} \right|_{\text{可逆}}
\end{equation}

\begin{example}{质量为 $m $ 的冰在 $0^\circ\text{C}$ 时的熔化问题}
先看质量为 $m $ 的冰在 $0^\circ\text{C}$ 时的熔化问题。我们必须将潜热 $L=80\rm cal/g$ 可逆地给系统,也就是用温度为 $\left( 0+\varepsilon \right) ^\circ\text{C}\left( \varepsilon \rightarrow 0 \right)$ 热库为系统提供热量,使冰水混合物的系统充分吸热,达到水增多一点儿的新平衡态。一直这样做,直到所有的冰在此温度下全部熔化,
\begin{equation}
S_2-S_1=\int_1^2{\frac{\delta Q}{T}=\frac{mL}{T_\text{冰}}\left( \text{cal}/\text{K} \right)}
\end{equation}
式中,$T_\text{冰}= 237. 16\rm K $。

\end{example}
在这个简单的例子中,$ T $ 固定,等于 $T_{\text{冰}}$,所以它可以提出到积分号的外面,积分的结果为 $ml$。

接下来我们考虑稍微复杂一些的情况。

\begin{example}{将质量为 $m $ 的水加热($T_1\to T_2$)}
将质量为 $m$ 的水加热,使其温度由 $T_1$ 升高到 $T_2$。熵变为
\begin{equation}
S_2-S_1=\int_1^2{mc_{\text{水}}\frac{\text{d}T}{T}}=mc_{\text{水}}\ln \frac{T_2}{T_1}
\end{equation}
再次强调,记住热量应该可逆地进入系统:你不能将温度为 $T_1$ 的水猛地倒入温度为 $T_2$ 的锅中。相反,你将它与一连串的热库接触,每个热库的温度都比前一个热库的温度高一个无限小量,水有足够的时间与每个热库达到平衡,最终使水的温度从 $T_1$ 升高到 $T_2$。

如果你将水冷却降温,也可以用这个方程,但结果是负值,因为 $T_2<T_1$。

最后我们计算气体的熵变,其结果非常有意义。温度为 $T $ 的气体等温膨胀,体积从 $V_1$ 增大到 $V_2$,如图所示。
\begin{figure}[ht]
\centering
\includegraphics[width=5cm]{./figures/8a5c48f3cf2a9a9a.pdf}
\caption{1、2点温度相同} \label{fig_Entrop_2}
\end{figure}
\end{example}
与冰熔化时类似,温度 $T $ 为常量,可以放到积分号的外面:
\begin{equation}
S_2-S_1=\int_1^2{\frac{\delta Q}{T}}=\frac{1}{T}\int_1^2{\delta Q=\frac{Q}{T}}
\end{equation}
式中,$ Q $ 是气体吸收的总热量。因为等温过程中 $E$ 为定值,$Q$ 等于气体所做的功,因此
\begin{equation}
S_2-S_1=\int_1^2{\frac{\delta Q}{T}}=\int_1^2{\frac{P\text{d}V}{T}}=\int_1^2{\frac{nRT\text{d}V}{VT}=nR\ln \frac{V_2}{V_1}}
\end{equation}
如果每个点都有唯一的熵,那么 $1$、$2$ 两点的熵差与我们怎样从 $1$ 到达 $2 $ 无关。因此,如果不沿等温路径的话,我们可以顺着 $P $ 轴下降到达 $0 $ 点,该点与 $1 $ 点的体积相同,即 $V_0=V_1$ 且 $P_0=P_2$。熵变为
\begin{equation}
S_{0}-S_{1}=n c_{v} \int_{T_{1}}^{T_{0}} \frac{\mathrm{d} T}{T}=n \frac{3 R}{2 |} \ln \frac{T_{0}}{T_{1}}=n \frac{3 R}{2} \ln \frac{T_{0}}{T}
\end{equation}
\begin{equation}
S_{2}-S_{0}=n c_{p} \ln \frac{T_{2}}{T_{0}}=n \frac{5 R}{2} \ln \frac{T}{T_{0}}
\end{equation}
\begin{equation}
\begin{aligned} S_{2}-S_{1} &=n \frac{3 R}{2} \ln \frac{T_{0}}{T}+n \frac{5 R}{2} \ln \frac{T}{T_{0}} \\ &=n \frac{3 R}{2} \ln \frac{T_{0}}{T_{0}}+n \frac{2 R}{2} \ln \frac{T}{T_{0}}=n R \ln \frac{T}{T_{0}} \\ &=n R \ln \frac{V_{2}}{V_{1}} \quad\left(\text { 用到了 } \frac{T}{T_{0}}=\frac{T_{1}}{T_{0}}=\frac{P_{1} V_{1}}{P_{0} V_{0}}=\frac{P_{1}}{P_{2}}=\frac{V_{2}}{V_{1}}\right) \end{aligned}
\end{equation}
这里用到了由\autoref{fig_Entrop_2} 得到的许多等式:$T=T_{1}=T_{2}, \quad V_{1}=V_{0} \text { 和 } P_{0}=P_{2}$。因此,求熵变时,我们可以随意选择路径。通常,我们选择最容易的路径。

\subsection{熵与系统宏观性质的关系}
现在我们还没有完全了解 $S$ 的意义,我们只知道它是个状态参量和它的定义式。下面我们来看一下它到底是啥:先改写一下热力学第一定律,采用 $S$ 进行描述:
\begin{equation}
\mathrm d E = \delta{Q} - P\mathrm d V
\end{equation}
因为 $E$ 是状态参量,因此 $\mathrm d E$ 与热量如何进入系统无关。我们设热最的交换是可逆的,这样有
\begin{equation}
\mathrm d E =T\mathrm dS - P\mathrm d V
\end{equation}
从数学上讲,可以认为 $E$ 是 $S $ 和 $V $ 的函数,且
\begin{equation}
T=\left(\frac{\partial E}{\partial S}\right)_V
\end{equation}
\begin{equation}
P=-\left(\frac{\partial E}{\partial V}\right)_S
\end{equation}
因为我们现在关注的是 $S$,所以将第一定律写为
\begin{equation}
\text{d}S=\frac{1}{T}\text{d}E+\frac{P}{T}\text{d}V
\end{equation}
此方程告诉我们 $S$(对于一定量的气体,如 $197 \rm mol$)是宏观量的函数,如体积 $V $、能量 $E$,且
\begin{equation}
\left. \frac{\partial S}{\partial E} \right|_V=\frac{1}{T}\quad \left. \frac{\partial S}{\partial V} \right|_E=\frac{P}{T}
\end{equation}
要记住这两个方程。

我们还常常用熵来表示系统的热容,这是因为 $T\dd S=\delta Q$,而热容是升高单位温度时系统吸热的量,所以我们有
\begin{equation}
C_V=T\left(\frac{\partial S}{\partial T}\right)_V
\end{equation}
利用这个关系式,我们可以利用系统的热容数据,通过积分计算得到系统的熵:
\begin{equation}
S=\int_{T_0} \frac{C_V(T,V)}{T} \dd T+S_0(T_0,V)=
\int_{T_0} \frac{C_V}{T} \dd T+S_0(V)
\end{equation}

\subsection{微观定义} 
\begin{equation}
S = k_B \ln \Omega
\end{equation}

% 举例: 高温物体热传导给低温物体, 损失多少?
 
% 未完成: 热机中熵如何变化? 等压等温绝热过程中熵如何变化?
