% OpenACC 笔记
% license Usr
% type Note

\begin{issues}
\issueDraft
\end{issues}

\pentry{C++ 基础\upref{Cpp0}}

OpenACC 可以像 OpenMP 那样,通过添加少量的 \verb`#pragma` (预编译指令)给已有的程序进行 GPU 加速。本文以 C++ 为例。

和 OpenMP 一样, OpenACC 需要编译器支持。一般使用英伟达显卡专用的 \verb`NVIDIA HPC SDK` 或者 \verb`GCC`

下面给出一个入门例程。 在 GCC 中,直接加上 \verb`-fopenacc` 选项即可! 例如 \verb`g++ -fopenacc main.cpp -o main.x`。

\begin{lstlisting}[language=none,caption=test1.cpp]
#include <iostream>
#include <vector>
#include <openacc.h>

// Function to initialize the vectors with values
void initialize(std::vector<double>& a, std::vector<double>& b, int n) {
	for(int i = 0; i < n; ++i) {
		a[i] = static_cast<double>(i);
		b[i] = static_cast<double>(2 * i);
	}
}

// Main function
int main() {
	const int n = 1000000; // Size of the vectors
	std::vector<double> a(n), b(n), c(n);
	double *pa = a.data(), *pb = b.data(), *pc = c.data();

	// Initialize vectors a and b
	initialize(a, b, n);

	// Using OpenACC to offload the following computation to an accelerator
	// and explicitly handle data movement
	#pragma acc data copyin(pa[0:n], pb[0:n]) copyout(pc[0:n])
	{
	while (true) {
		#pragma acc parallel loop
			for(int i = 0; i < n; ++i) {
			pc[i] = pa[i] + pb[i];
			if (i == 0 && acc_on_device(acc_device_none)) {
				printf("Not executing on GPU.\n");
			} else {
				printf("Executing on GPU.\n");
			}
		}
	}
	}

	// Display the first 10 results
	for(int i = 0; i < 10; ++i) {
		std::cout << "c[" << i << "] = " << c[i] << std::endl;
	}

	return 0;
}
\end{lstlisting}


