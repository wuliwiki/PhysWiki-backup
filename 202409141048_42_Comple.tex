% 相对论补全和推广
% keys 相对论|补全|四矢量|数量流
% license Usr
% type Tutor

\pentry{协变性和不变性\nref{nod_CoIn}}{nod_029b}
在狭义相对论中,Lorentz变换是给出了两个参考系间的坐标变换关系。特别地,给出了惯性系间的坐标变换关系。狭义相对论的假设之一是,物理定律在所有惯性系中都是相同的。根据协变性和不变性的定义(\autoref{def_CoIn_1}),这就是说,描述狭义相对论的基本方程被假设是在Lorentz变换下是协变的,因而狭义相对论的物理在Lorentz变换下是不变的。因此,所有的物理量在Lorentz变换下必须以一种定义良好的方式变换。因此,在非相对论下的物理量在相对论的框架下必须进行补全,而物理定律应当推广相对论情形。
\subsection{物理量的相对论补全}
Lorentz变换是四维时空中一个基底到另一个基底对应的转换矩阵。因此,四维时空的矢量坐标在不同基底下的转换关系必须由Lorentz变换连接。$\dd x^\mu=(\dd t,\dd{\vec x})$ 是四维时空中的矢量,即在相对论框架下,三维空间中的矢量(简称3-矢量) $\dd{\vec x}$ 必须和三维空间的标量(简称3-标量) $\dd t$ 一起统一成四维空间的矢量(简称4-矢量),才是狭义相对论下具有良好定义的矢量。

\subsubsection{速度补全}
已知3-速度定义为 $\vec v_N:=\dv{\vec x}{t}$($N$ 可理解为Newton定义的),这是一个3-矢量除以一个3-标量,而 $\dd t$ 恰巧是4-矢量的时间分量。容易验证,$\bvec v_N$ 在Lorentz变换下并不像4-矢量任何分量一样变换。

相反,考虑3-矢量 $\dd{\vec x}$ 除以一个Lorentz标量(在Lorentz变换下代表的数值不变的量)$\dd \tau^2:=-\eta_{\mu\nu}\dd x^{\mu}\dd x^\nu=\dd t^2-\dd {\vec x^2}$。

\subsubsection{粒子数密度补全}