% Clifford 代数的基本运算
% license Xiao
% type Tutor




本节利用集合语言,介绍Clifford代数上的二元线性运算\footnote{本文参考Jie Peter《代数学讲义》}。通过线性性可以将这些运算推广到Clifford代数的其他元素上。我们也会证明,更改线性空间的正交基并不会改变运算的形式。
\subsubsection{外积与正交基}
\begin{definition}{外积}
给定Clifford代数$\mathrm {Cl(X,R,s)}$,对于$A,B\in 2^x$,定义
\begin{equation}
A \wedge B=\left\{\begin{aligned}
A B,\quad& A \cap B=\varnothing \\
0,\quad& A \cap B \neq \varnothing~,
\end{aligned}\right.
\end{equation}
并称之为外积(outer product or exterior product)或楔积(wedge product)。我们可以通过线性性将该运算拓展到任意元素之间的外积。
\end{definition}


Clifford积的结合性使得外积也具有\textbf{结合性},读者可以自行验证。由运算的线性可知,外积还具有\textbf{幂零性},即任意$v\in V,v\wedge v=0$。
在上一节,我们说过,集合语言的阐述实际上是指定了线性空间的正交基。正交性体现为CLifford积的反对称性。因而对于\textbf{正交基}$\{\mathrm {e_i}\}$,我们有$\mathrm{e_i\wedge e_j=-e_j\wedge e_i}$。由外积的线性性可得,在给定\textbf{正交基}后,任意两个向量做外积,结果具有\textbf{反对称性}。也就是说,外积的反对称性是依赖于正交基的。


在幂零性和正交性的保证下,正交基的外积表现不随正交变换而改变。例如某线性空间下有两组正交基$\{e_i\},\{\theta_i\}$,且$\mathrm {\theta_1=a^i e_i,\theta_2=b^i e_i}$.那么我们由线性性和$\{e_i\}$的正交关系得到:$\theta_i\wedge\theta_i=0$,而且有:
\begin{equation}
\left\{\begin{array}{l}
\theta_1 \theta_2=\left(a^i e_i\right) \wedge\left(b^j e_j\right)+\sum_k\left(a^k b^k s(e_k)\right) \\
\theta_2 \theta_1=\left(b^j e_j\right) \wedge\left(a^i e_i\right)+\sum_k\left(b^k a^k s(e_k)\right)
\end{array}\right.~,
\end{equation}
$a^i,b^i$是过渡矩阵里的两个列向量。
正交性要求$\theta_1\theta_2=-\theta_2\theta_1$,对应$\theta_i\wedge\theta_j=-\theta_j\wedge\theta_i$。从上式可以看到,基的正交性等价于\textbf{正交变换条件}$\sum_k(a^k b^k s(e_k))=0$,这其实是\textbf{正交矩阵在任意线性空间的推广}。在配备了正定二次型的欧几里得空间中,这意味着正交矩阵定义为列向量组中任意两个列向量内积为$0$。
\subsubsection{左内积}
\begin{definition}{}
给定Clifford代数$\mathrm {Cl(X,R,s)}$,对于$A,B\in 2^x$,定义
\begin{equation}
A\left\llcorner B=\left\{\begin{aligned}
A B,\quad & A \subseteq B ; \\
0,\quad & A \nsubseteq B .
\end{aligned}\right.\right.~,
\end{equation}
并称之为左内积(left inner product or left interior product)。我们依然可以通过线性性将运算推广到Clifford代数的任意元素上。
\end{definition}
右内积的定义是对偶的,即:
\begin{definition}{右内积}
\begin{equation}
A\lrcorner B=\left\{\begin{aligned}
A B,\quad & A \supseteq B ; \\
0,\quad & A \nsupseteq B .
\end{aligned}\right.~,
\end{equation}
\end{definition}
下面我们只研究左内积。

与外积相同,左内积也具有“正交基形式不变性”。
例如,在同一线性空间中选取两组正交基,$\{e_i\}$,$\{\theta _i\}$,且$\mathrm {\theta_1=a^ie_i,\theta_2=b^i e_i},\theta_3=c^i e_i$。那么我们有:
\begin{equation}
\begin{aligned}
\theta_1\left\llcorner\theta_2\right. & =\left(a^i e_i\right)\left\llcorner\left(b^j e_j\right)\right. \\
& =\sum_i s(i) a^i b^i=0
\end{aligned}~,
\end{equation}
\begin{exercise}{}
证明:$\theta_1\left\llcorner(\theta_2\theta_3)\right.$=0。提示:证明每一个分量为0.
\end{exercise}
从上述证明里,我们也可以清晰地看到,左内积的结果可以给出不同的分次结构:
\begin{equation}
A\left\llcorner B=\langle A B\rangle_{s-k} .\right.~,
\end{equation}

\subsubsection{标量积}
标量积是一种特殊情况,即左右内积相等。
\begin{definition}{标量积}
给定Clifford代数$\mathrm {Cl(X,R,s)}$,对于$A,B\in 2^x$,
\begin{equation}
A * B=\left\{\begin{array}{cc}
A B, & A=B \\
0, & A \neq B .
\end{array}\right.~,
\end{equation}
\end{definition}
读者可以自行证明标量积的正交基形式不变性。

在上一节,我们证明了$e_Ae_B\propto e_{A\Delta B}$。实际上这里的系数是A与B交集部分的标量积(默认下标都是从小到大排列),我们用$g_{AB}$表示,即标量积可视作利用V上二次型q诱导出几何代数上的二次型。
\begin{theorem}{}
已知几何代数$\mathcal {G}(V, q)$上的二次型 $q$ 在某基下表示为矩阵 $g_{ij}$,$q $诱导的标量积则表示为 $g_{AB}$,其中 $A,B$ 是复杂指标。则
\begin{equation}
g_{A B}=e_A*e_B=\left\{\begin{aligned}
0,\quad & A \neq B \\
(-1)^{\frac{|A|(|A|-1)}{2}} \prod_{i \in A} g_{i i}, \quad& A=B
\end{aligned}\right.~,
\end{equation}
\end{theorem}

Proof. 按照标量积定义,$A \neq B$时,$g_{AB}$=0。$A = B$时,$\frac{|A|(|A|-1)}{2}$是$e_B$进行对换,使得$e_i$相邻所需要跨过的总次数。比如从$e_1e_2e_3e_1e_2e_3$变换到$e_1e_1e_2e_2e_3e_3$的总步数。根据正交性,每一次变换位置都需要乘以-1。得证。

对于任意$v_1,v_2\in V$,线性运算使得Clifford积可以分解为依赖于二次型的标量积和与二次型无关的外积:$v_1v_2=v_1* v_2+v_1\wedge v_2$

\subsubsection{投影,对合,反转以及共轭}
\begin{definition}{投影}
给定Clifford代数$\mathrm {Cl(X,R,s)}$,对于$A,B\in 2^x$,定义
\begin{equation}
\langle A\rangle_k=\left\{\begin{array}{cc}
A, & |A|=k \\
0, & |A| \neq k
\end{array}\right.~,
\end{equation}
并称之为在 k-次子空间上的投影。
\end{definition}
\begin{definition}{按次对合}
给定Clifford代数$\mathrm {Cl(X,R,s)}$,对于$A,B\in 2^x$,定义
\begin{equation}
A^{\star}=(-1)^{|A|}A~,
\end{equation}
并称之为按次对合(grade involution) 、第一类对合(first main
involution)或者简称为 main involution。
\end{definition}
\begin{definition}{反转}
给定Clifford代数$\mathrm {Cl(X,R,s)}$,对于$A,B\in 2^x$,定义
\begin{equation}
A^{\dagger}=(-1)^{\frac{|A|(|A|-1)}{2}}A~,
\end{equation}
\end{definition}
并称之为反转(reversion)、第二类对合(second main involution)或者主反自同构(principal anti-automorphism)。反转是下指标重排的过程:
$$(e_1e_2...e_n)^{\dagger}=e_ne_{n-1}...e_2e_1~,$$
\begin{definition}{共轭}
定Clifford代数$\mathrm {Cl(X,R,s)}$,对于$A,B\in 2^x$,定义
\begin{equation}
A^{\square}=A^{\star \dagger}~,
\end{equation}
并称之为Clifford共轭(Clifford conjugate)。
\end{definition}
\begin{exercise}{}
已知$\mathcal G(\mathbb R^{0,1})\cong \mathbb C,G(\mathbb R^{0,2})\cong \mathbb H$,证明:按次对合和Clifford共轭在$\mathbb C$上就是复共轭,而Clifford共轭在$\mathbb H$上是四元数共轭
\end{exercise}
\subsubsection{常见结论}
在实际运算的过程中,我们有时候会遇到多个元素通过多个运算结合在一起,下面给出一些结合相关的性质。
\begin{exercise}{}
对于$x,y,z\in\mathrm {CL(X,R,s)}$,证明
\begin{equation}
\left\{\begin{aligned}
x \wedge(y \wedge z) & =(x \wedge y) \wedge z ;\\
x\llcorner(y\lrcorner z) & =(x\llcorner y)\lrcorner z ;\\
x\llcorner(y\llcorner z) & =(x \wedge y)\llcorner z; \\
x *(y\llcorner z) & =(x \wedge y) * z ; \\
1 \wedge x & =x \wedge 1=1\llcorner x=x\lrcorner 1=x .
\end{aligned}\right.~
\end{equation}
\end{exercise}

思路是先证明对基成立,才通过线性运算扩张。
此外,还有以下可以简化运算的结论。
\begin{exercise}{}
对于任意$x,y\in\mathrm {CL(X,R,s)}$和$v\in \mathrm{CL^1(X,R,s)}$,证明
\begin{equation}
\left\{\begin{aligned}
v x&=v\llcorner x+v \wedge x ; \\
v \wedge x&=x^{\star} \wedge v=\frac{1}{2}\left(v x+x^{\star} v\right) ; \\
v\llcorner x&=-x^{\star}\lrcorner v=\frac{1}{2}\left(v x-x^{\star} v\right) ; \\
v\llcorner(x y)&=\left(v\llcorner x) y+x^{\star}(v\llcorner y) .\right.
\end{aligned}\right.~
\end{equation}
\end{exercise}
\subsubsection{几何代数上的对偶}
\begin{definition}{}
给定\textbf{非退化}几何代数\mathcal G(V,q),对$x\in\mathcal G(V,q)$定义$x^c=xI^{-1}$,称为$x$的\textbf{对偶(dual)}。
\end{definition}
定义里可见,由于要求体积形式有逆元,元素的对偶存在性依赖于非退化的几何代数。如果满足非退化条件,元素的外积也有其对应的对偶空间的外积。
\begin{definition}{}
定义\textbf{对偶外积(dual outer product)},用符号$\vee$表示,使得
\begin{equation}
x\vee y=(x^c\wedge y^c)^c~.
\end{equation}
\end{definition}
对偶定义使得左内积和外积也存在对偶关系。
\begin{theorem}{}\label{the_clf02_1}
给定非退化的几何代数$\mathcal G(V,q)$,任取其元素$x,y$,则有
\begin{equation}
x\llcorner y=(x\wedge y^c)^c~.
\end{equation}
\end{theorem}
proof.
只需要证明$x\llcorner y^c=(x\wedge y)^c$。

$x\llcorner y^c=x\llcorner (yI^{-1})=x\llcorner (y\llcorner I^{-1})=(x\wedge y)\llcorner I^{-1}=(x\wedge y)^c$