% 闵可夫斯基空间
% 四维时空|惯性系|洛伦兹不变量|洛伦兹变换|闵可夫斯基度规

\pentry{时空的四维表示\upref{SR4Rep},度量空间\upref{Metric}}

\subsection{时空间隔}
经典力学讨论的时空是一个欧几里得空间,不同惯性系之间的伽利略变换是一个保度量变换,即时空点之间的度量在变换前后不变。但是相对论下的欧几里得度量不再是不变的,两个时空点之间的度量,在洛伦兹变换后一般会变化。不过相对论里有另一个类似欧几里得度量的东西,它在洛伦兹变换下保持不变。

\begin{exercise}{洛伦兹不变量}\label{exe_MinSpa_1}

给定参考系 $K_1$ 和 $K_2$,其中 $K_2$ 相对 $K_1$ 以速度 $\bvec{v}=(v,0,0)^T$ 运动。假设有两个事件,它们在 $K_1$ 中的四位置分别为 $(t_1, x_1, y_1, z_1)$ 和 $(t_2, x_2, y_2, z_2)$,在 $K_2$ 中的四位置分别为 $(t_1', x_1', y'_1, z'_1)$ 和 $(t'_2, x'_2, y_2', z_2')$。请利用洛伦兹变换,通过计算证明
\begin{equation}
\begin{aligned}
&(t_1-t_2)^2-(x_1-x_2)^2-(y_1-y_2)^2-(z_1-z_2)^2\\=&(t_1'-t_2')^2-(x'_1-x'_2)^2-(y_1'-y'_2)^2-(z'_1-z_2')^2~.
\end{aligned}
\end{equation}
\end{exercise}

由\autoref{exe_MinSpa_1} 可知,在任何惯性参考系中,$\sqrt{-(t_1-t_2)^2+(x_1-x_2)^2+(y_1-y_2)^2+(z_1-z_2)^2}$ 都是不变的。我们把这个量记为 $\Delta S$,称为两事件之间的\textbf{时空间隔(interval)}。

\subsection{闵可夫斯基度规}

我们知道,$n$ 维欧几里得空间是 $n$ 个实数集合的笛卡尔积,配上用勾股定理定义的度量所得到的。这样的度量,在物理学上称为欧几里得度规。闵可夫斯基给四维时空定义了一个新的度规,即时空间隔。以时空间隔作为度规的四维空间,被称为闵可夫斯基空间。

时空间隔并不是数学意义上的\textbf{度量}(\autoref{def_Metric_2}~\upref{Metric}),因为它无法满足正定性,三角不等式也普遍不成立或不存在\footnote{时空间隔不是实数的时候,无法比较大小。}。

另一种看待闵可夫斯基度规的方法是,把四维时空看成复数集合的笛卡尔积,依然使用勾股定理所定义的度量,但是把时间看成必须是纯虚数。也就是说,事件的时间为 $t$ 的时候,其在时间轴上的坐标是 $\I t$。这样一来,就可以把事件 $(\I t, x, y, z)$ 的范数写成 $\sqrt{(\I t_1-\I t_2)^2+(x_1-x_2)^2+(y_1-y_2)^2+(z_1-z_2)^2}$,形式上和实数欧几里得空间的欧几里得范数是一样的,只不过用的是复数;进而得到形式上的“复”欧几里得度量。需要特别注意的是,闵可夫斯基度规并不能看成酉空间中的四维复数度量。

