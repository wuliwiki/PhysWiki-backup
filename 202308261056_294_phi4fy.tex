% Φ$^4$ 理论的费曼规则
% keys 标量场|相互作用|费曼图|费曼规则
% license Xiao
% type Tutor

\pentry{Wick 定理(标量场)\upref{wick}}

利用 Wick 定理\footnote{Wick 定理(标量场)\upref{wick}。},可以对相互作用场论的编时格林函数进行微扰展开,并将它微扰展开的每一阶都对应于费曼图表示的贡献。对于 $m$ 阶编时格林函数 $\langle \Omega|T[\phi(x_1)\phi(x_2)\cdots \phi(x_m)]|\Omega\rangle$,利用 \autoref{eq_wick_1}~\upref{wick} 进行微扰展开,先考虑分子,
对于 $n$ 阶相互作用项 $\langle 0| T[\phi_I(x_1)\cdots \phi_I(x_m) H_I(z_1)\cdots H_I(z_n)]|0\rangle$,计算它的 Feynman 规则如下
\begin{theorem}{$\phi^4$ 理论的 Feynman 规则(实空间)}
\begin{enumerate}
\item 画出所有拓扑不等价的 $m$ 个外点 $n$ 个内点(相互作用顶点)的 Feynman 图,满足每个内点都连接 $4$ 条传播子,每个外点都连接 $1$ 条传播子。分别计算每个图的贡献。
\item 对每一个从 $x$ 到 $y$ 的传播子,写下 $D_F(x-y)$。
\item 对每一个相互作用顶点 $z$,写下 $(-i\lambda_0)\int \dd[4]z$。
\item 除以该图对称因子 $S$\footnote{对称因子的具体计算方式见\autoref{the_wick_1}~\upref{wick}。},就是该图的贡献。
\end{enumerate}
\end{theorem}
\addTODO{未完成}
