% 爱德华·泰勒(综述)
% license CCBYSA3
% type Wiki

本文根据 CC-BY-SA 协议转载翻译自维基百科 \href{https://en.wikipedia.org/wiki/Edward_Teller}{相关文章}。

\begin{figure}[ht]
\centering
\includegraphics[width=6cm]{./figures/aa578cdb3e82db6d.png}
\caption{} \label{fig_ADHTL_1}
\end{figure}
爱德华·泰勒(Edward Teller,匈牙利语:Teller Ede,1908年1月15日-2003年9月9日)是一位匈牙利裔美国理论物理学家和化学工程师,因其在氢弹发展中的关键角色而被俗称为“氢弹之父”。他是基于斯坦尼斯瓦夫·乌拉姆设计提出的“泰勒–乌拉姆构型”的共同发明人之一。

泰勒性格激烈,据称“受到百万吨级爆炸梦想的驱使,有救世主情结,展现出专断的行为风格”。\(^\text{[1]}\)他曾设计过一种名为“闹钟模型”的热核炸弹,其爆炸当量高达1000兆吨(即10亿吨TNT),并建议通过船只或潜艇投送。这种武器将具备焚毁一个大陆的能力。\(^\text{[1]}\)

