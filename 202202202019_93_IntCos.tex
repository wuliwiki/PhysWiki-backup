% 干涉,光强的余弦平方分布
% 光的矢量叠加;远场光强

\addTODO{巴俾涅原理,线偏振光,辐照度}
\pentry{电场波动方程\upref{EWEq},平面简谐波\upref{PWave}}
\subsection{干涉}

光是一种电磁波,满足由麦克斯韦方程组导出的电场波动方程\upref{EWEq}.它服从重要的\textbf{叠加原理}.简单地说,干涉就是两束或多束光波的相互作用,这种相互作用产生的总辐照度不等于各束光波的辐照度之和.今后我们将在衍射中了解到巴俾涅原理.

考虑多个光源产生的场$\bvec E_1$ , $\bvec E_2$ , ...空间一点的总电场强度 $\bvec E$ 由下式给出:

$$ \bvec E = \bvec E_1 + \bvec E_2 + \cdots$$

光扰动以大约$\Si{10}^{14}$ Hz 的频率随时间飞快地变化,使得每时每刻的实际光场无法探测. 另一方面, 辐照度 $I$ 则可以使用各种探头直接测量. 因此, 研究干涉最好从辐照度的角度研究.

\subsection{余弦平方分布}

考虑由两个点光源 $S_1$ 和 $S_2$ 在均匀介质(即各向同性介质)中发射同一频率的平面波. 令 $S_1$ 与 $S_2$ 的间隔为 $a$, 且 $ a \gg \lambda $( $\lambda$ 是光的波长 ),考虑简单的线偏振波:

$$ \bvec E_1 ( \bvec r , t) = \bvec E_{01} \cos (\bvec k_1 \vdot \bvec r_1 - \omega t + \varepsilon_1 )$$
$$ \bvec E_2 ( \bvec r , t) = \bvec E_{02} \cos (\bvec k_2 \vdot \bvec r_2 - \omega t + \varepsilon_2 )$$

则 $ P $ 点的辐照度由下式给出:
$$ I = \epsilon v \langle \bvec E^2 \rangle _T $$
$\langle \bvec E^2 \rangle _T$ 的含义为电场强度的平方在一段时间内的平均值. 我们只关注 $ I $ 与 $ \bvec E $ 的关系,因此忽略常系数 $ \epsilon v$, 便于讨论. 

 P 点的合场强为$ \bvec E = \bvec E_1 + \bvec E_2 $ , 则辐照度  
\begin{aligned}
I = \langle( \bvec E_1 + \bvec E_2)\vdot( \bvec E_1 + \bvec E_2 )\rangle _T \\
 = \langle \bvec E_1^2 \rangle _T + \langle \bvec E_2^2 \rangle _T + 2\langle \bvec E_1 \vdot \bvec E_2 \rangle _T = I_1 + I_2 + I_{12}
\end{aligned}

其中 $ I _{12} =2 \langle \bvec E_1 \vdot \bvec E_2 \rangle _T$为\textbf{干涉项}.

将 $\bvec E_1$ 和 $ \bvec E_2$ 的表达式代入 $I_{12}$可得:
$$I_{12} = \bvec E_{01}\bvec E_{02} \cos\delta$$
$$\delta = (\bvec k_1 \vdot \bvec r_1 - \bvec k_2 \vdot \bvec r_2 ) + \varepsilon_1 - \varepsilon_2 $$

常见情况为 $\bvec E_1 \backslash\backslash\bvec E_2$,则 $ I_{12} = E_{01} E_{02}\cos\delta$.

由于$\langle \bvec E^2 \rangle _T = \dfrac{1}{2}E^2$,所以 $ E_1 = \sqrt{2I_1}$ ,  $ E_2 = \sqrt{2I_2}$,则 $ I = I_1 + I_2 + 2\sqrt{}$

由余弦函数的性质知,当 $\delta = 0, 2\pi, 4\pi, \cdots$ 时,$\cos \delta = 1$,得到 $ I_{max} = I_1 + I_2 +2\sqrt{I_1 I_2}$,这种情况称为\textbf{完全相长干涉};
反之,当 $\delta = -\pi, \pi, 3\pi, \cdots$ 时,$\cos \delta = 0$,得到 $ I_{min} = I_1 + I_2 $,这种情况称为\textbf{完全相消干涉}.
