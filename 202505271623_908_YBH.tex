% 伊本·海什木(综述)
% license CCBYSA3
% type Wiki

本文根据 CC-BY-SA 协议转载翻译自维基百科 \href{https://en.wikipedia.org/wiki/Ibn_al-Haytham}{相关文章}。

哈桑·伊本·海赛姆(Ḥasan Ibn al-Haytham,拉丁化名为 Alhazen,/ælˈhæzən/;全名:阿布·阿里·哈桑·伊本·哈桑·伊本·海赛姆,阿拉伯语:أبو علي، الحسن بن الحسن بن الهيثم;约965年—约1040年),是伊斯兰黄金时代的一位中世纪数学家、天文学家和物理学家,来自今天的伊拉克地区。[6][7][8][9]他被誉为“现代光学之父”,[10][11][12] 尤其在光学原理与视觉感知领域做出了重要贡献。他最具影响力的著作是《光学书》(阿拉伯语:كتاب المناظر,*Kitāb al-Manāẓir),写于1011年至1021年之间,现存有其拉丁文译本。[13]在科学革命时期,艾萨克·牛顿、约翰内斯·开普勒、克里斯蒂安·惠更斯和伽利略·伽利莱等人经常引用海赛姆的著作。

伊本·海赛姆是第一个正确解释视觉理论的人,[14] 他提出视觉是在大脑中形成的,并指出视觉具有主观性,会受到个体经验的影响。[15] 他还首次提出了光在折射时走最短时间路径的原理,这一原理后来被称为费马原理。[16]在镜学和透镜学领域,他通过对反射、折射以及光线成像性质的研究做出了重大贡献。[17][18]伊本·海赛姆还是最早倡导假设必须通过可验证程序或数学推理支持的实验来检验的人之一——在文艺复兴科学家出现前五个世纪,他就已是科学方法的早期奠基者,[19][20][21][22] 有时他被称为世界上“第一位真正的科学家”。[12]
此外,他还是一位博学多才的学者,著述涵盖哲学、神学和医学等多个领域。[23]

伊本·海赛姆是第一个正确解释视觉理论的人,[14] 他提出视觉是在大脑中形成的,并指出视觉具有主观性,会受到个体经验的影响。[15] 他还首次提出了光在折射时走最短时间路径的原理,这一原理后来被称为费马原理。[16] 在镜学和透镜学领域,他通过对反射、折射以及光线成像性质的研究做出了重大贡献。[17][18]伊本·海赛姆还是最早倡导假设必须通过可验证程序或数学推理支持的实验来检验的人之一——在文艺复兴科学家出现前五个世纪,他就已是科学方法的早期奠基者,[19][20][21][22] 有时他被称为世界上“第一位真正的科学家”。[12]此外,他还是一位博学多才的学者,著述涵盖哲学、神学和医学等多个领域。[23]
\subsection{生平}
伊本·海赛姆(拉丁化名 Alhazen)出生于约公元965年,其家族具有阿拉伯[9][31][32][33][34]或波斯[35][36][37][38][39]血统,出生地为伊拉克的巴士拉,当时属于布韦王朝的统治范围。他最初受宗教研究和公共服务的影响较深。当时社会中存在许多相互冲突的宗教观点,他最终决定淡出宗教领域,转而投身于数学与科学的研究。[40]他在家乡巴士拉曾担任“宰相”的职位,并因其应用数学方面的才能而闻名,其中一个例证是他曾尝试设计方案以调控尼罗河的泛滥。[41]

他回到开罗后,被任命为一个行政职务。然而,他最终也未能胜任该职,引起了哈里发哈基姆的愤怒,[42] 据说他因此被迫躲藏,直到哈里发于公元1021年去世,他才得以恢复自由,并领回被没收的财产。[43] 传说称,海赛姆假装疯癫,并在这段时间中被软禁在家中。[44] 正是在这段时期,他撰写了其最有影响力的著作《光学书》。此后,海赛姆继续留居开罗,住在著名的爱资哈尔大学附近,靠其著作收入为生,直到约公元1040年去世。[45][41]现存有一部伊本·海赛姆亲笔誊写的阿波罗尼奥斯《圆锥曲线论》手稿,藏于圣索菲亚图书馆,编号为 MS Aya Sofya 2762,第307叶,落款日期为伊斯兰历415年萨法尔月(公元1024年)。[46]:注2

他的学生中包括一位来自塞姆南的波斯人苏尔哈布(Sorkhab,或作 Sohrab),以及一位埃及王子阿布·瓦法·穆巴希尔·伊本·法泰克。[47]
\subsection{《光学书》}
伊本·海赛姆最著名的著作是其七卷本的光学论文集——《光学书》,写于公元1011年至1021年间。[48] 在该书中,他是第一个解释视觉是由于光线从物体反射后进入眼睛而产生的人,[14] 同时他还首次提出视觉是在大脑中形成的,并指出视觉具有主观性,会受到个人经验的影响。[15]

这部著作在12世纪末或13世纪初由一位不知名的学者翻译成拉丁文。[49][a]

该书在中世纪享有极高声誉。其拉丁文版本 De aspectibus 于14世纪末被译成意大利通俗语,题为 《De li aspecti》。[50]

1572年,弗里德里希·里斯纳将其印刷出版,书名为:Opticae thesaurus: Alhazeni Arabis libri septem, nunc primum editi; Eiusdem liber De Crepusculis et nubium ascensionibus(中文译名:《光学宝藏:阿拉伯人阿尔哈曾七卷本著作,首版;另附其关于黄昏与云层高度的著作》)[51]“Alhazen”这一名字变体即由里斯纳所创,在此之前他在西方被称为“Alhacen”。[52]1834年,E. A. 塞迪约在巴黎国家图书馆发现了海赛姆关于几何的若干著作。根据A. Mark Smith 的研究,目前共发现18部完整或近完整的手稿和5部残卷,分布于14个地点,包括牛津大学的博德利图书馆和布鲁日图书馆等地。[53]
\subsubsection{光学理论}
\begin{figure}[ht]
\centering
\includegraphics[width=6cm]{./figures/0669b50632580a1e.png}
\caption{} \label{fig_YBH_1}
\end{figure}
在古典时代,关于视觉的主要理论有两种:第一种是发射理论,由欧几里得和托勒密等思想家支持,他们认为视觉是由于眼睛发出光线与物体接触而产生的。第二种是摄入理论,由亚里士多德及其追随者支持,他们认为物体以某种物理形式将图像传入眼中。早期伊斯兰世界的学者(如金迪 al-Kindi)基本上沿用了欧几里得、盖伦或亚里士多德的理论体系。《光学书》最强烈的影响来源于托勒密的《光学》,而其中关于眼睛的解剖与生理结构的描述,则是基于盖伦的医学论述。[54]海赛姆的成就在于,他创造性地提出了一个理论,成功结合了:欧几里得的数学光线理论;盖伦的医学传统;以及亚里士多德的摄入理论中的部分要素。在他的摄入理论中,海赛姆继承了金迪的观点(并不同于亚里士多德),提出:“在任何被光照亮的彩色物体上,从其每一个点都会沿着所有可以从该点画出的直线,向外发出光与颜色。”[55]这就为他留下了一个重要问题:如何从如此多独立来源的辐射中形成一个连贯的图像?——特别是,当物体的每一个点都向眼睛的每一个点发送光线时,图像如何保持清晰与一致?

海赛姆所需要解决的问题是:物体上的每一个点如何只对应眼睛上的一个点。[55]为了解决这个问题,他提出了一个观点:眼睛只感知来自物体的垂直光线——也就是说,对于眼睛上的任意一点,只有那条直接进入该点、且未被眼睛其他部分折射的光线才会被感知。他使用了一个物理类比来说明垂直光线比斜射光线更“有力”:就像一个球如果垂直击中木板,可能会将其击碎;但如果是斜着打过去,就会被弹开。同样的道理,垂直入射的光线比折射偏转的光线更“强”,因此只有垂直光线才会被眼睛感知。由于从物体某一点发出的光线中,只有一条垂直光线能准确地进入眼睛的某一点,且所有这些光线在眼睛中形成一个朝向中心的光锥结构,这一观点便使他成功解决了“物体每个点发出大量光线会造成视觉混乱”的问题。换句话说,如果只有垂直光线会被感知,那么就可以实现“物体点”和“眼睛点”之间的一对一对应关系,避免了图像混乱。[56]后来,在《光学书》第七卷中,海赛姆又进一步提出:其他(非垂直)光线在进入眼睛后,会被折射,并最终“仿佛”以垂直方式被感知。[57]然而,他关于垂直光线的论证仍存在解释不足之处:[58]为什么只有垂直光线被感知?为什么较“弱”的斜射光线不会被较弱地感知?
他后来提出的“折射光线被看作垂直”的观点,[59] 从逻辑上也不具备足够的说服力。尽管如此,这一理论在当时仍是最为全面的光学体系,并具有极其深远的影响,尤其是在西欧中世纪至近代早期。海赛姆的《De Aspectibus》(即《光学书》的拉丁译本)直接或间接地激发了13至17世纪间大量光学研究活动。开普勒后来关于视网膜成像的理论,正是在海赛姆的概念框架基础上发展而来,最终彻底解决了“物体点与眼睛点一一对应”的问题。[60]

海赛姆通过实验证明了光沿直线传播,并进行了大量关于透镜、镜子、折射和反射的实验研究。[61] 他在分析反射与折射现象时,会将光线的垂直分量与水平分量分开考虑。[62]

海赛姆研究了视觉的形成过程、眼睛的结构、图像在眼内的成像机制以及视觉系统的整体工作原理。在1996年发表于《感知》期刊的一篇文章中,伊恩·P·霍华德指出,许多原本归功于几个世纪后西欧学者的发现与理论,其实应该归功于海赛姆。例如:他描述了后来在19世纪被称为“赫林等神经支配定律”的原理;他比阿奎洛纽斯早600年就对垂直视对应线进行了描述,而且其定义比阿奎洛纽斯的更接近现代观点;他对双眼视差的研究,在1858年被帕努姆重复过一次。[63]尽管如此,学者克雷格·阿恩-斯托克戴尔在肯定海赛姆诸多贡献的同时,也表示需要审慎对待,尤其是在将海赛姆的成就与托勒密脱离来看的时候要特别小心。海赛姆虽然纠正了托勒密在双眼视觉方面的一个重大错误,但除此之外,他的整体理论与托勒密非常相似——托勒密其实也尝试解释过类似于赫林定律的内容。[64]

总体而言,海赛姆是在继承并扩展托勒密光学体系的基础上,进一步推进光学发展的。[65]

在对伊本·海赛姆关于双眼视觉研究贡献的更为详细的分析中,雷诺基于勒琼[66]和萨布拉[67]的研究指出,[68] 在海赛姆的光学理论中,关于视点对应、同名复视和交叉复视的概念已经确立。但与霍华德的观点不同,雷诺解释了为何海赛姆没有提出以圆形形式描述视等线的原因,并指出,通过实验推理的方式,海赛姆实际上更接近于发现“帕努姆融合区”,而非后来的“维特-缪勒圆”。在这一方面,伊本·海赛姆的双眼视觉理论面临两大局限:未能认识到视网膜在视觉中的关键作用;显而易见地,缺乏对眼球通路进行系统的实验性研究。
\begin{figure}[ht]
\centering
\includegraphics[width=6cm]{./figures/062a3ea76c606192.png}
\caption{} \label{fig_YBH_2}
\end{figure}
海赛姆最具原创性的贡献在于:在描述完他对眼睛解剖结构的理解之后,他进一步思考了这种结构作为一个光学系统在功能上的表现。[69]他通过实验掌握了针孔成像的原理,这种理解显然影响了他对眼内图像倒置问题的思考,[70] 并试图设法避免图像倒置的发生。[71]他认为,垂直射入晶状体(他称之为“玻璃状液”,glacial humor)的光线,在穿出玻璃状液时还会被向外进一步折射,从而使图像在进入眼后部的视神经时依然保持正立。[72]他继承了盖伦的观点,认为晶状体是视觉感受器官,但他的一些论述也暗示他可能认为视网膜也在视觉中起到作用。[73]

海赛姆对光与视觉的综合理论严格遵循亚里士多德的体系,以逻辑、完整的方式详尽描述了视觉过程。[74]

他在镜学方面的研究主要集中于球面镜、抛物面镜以及球差问题。他还观察到:入射角与折射角之间的比率并不恒定,并研究了透镜的放大能力。[61]
\subsubsection{反射定律}
海赛姆是第一个完整陈述反射定律的物理学家。[75][76][77]他首次明确指出:入射光线、反射光线与入射点处的法线三者共面,且都位于垂直于反射面的同一平面内。[17][78]
\subsubsection{海赛姆问题}
\begin{figure}[ht]
\centering
\includegraphics[width=8cm]{./figures/ea3add2dbd5ba4cf.png}
\caption{} \label{fig_YBH_3}
\end{figure}
他在《光学书》第五卷中关于镜学的研究,涉及了一个后来被称为“海赛姆问题”的著名几何问题,该问题最早由托勒密于公元150年提出。该问题的基本形式是:从平面内的两个点作线,经过圆周上的某一点,并使这两条线与该点处的法线所成角相等。这个问题的几何意义相当于:
在一张圆形台球桌上,已知两个点,求一个圆周上的反弹点,使得一颗球从第一点击向圆边,在该点反弹后正好击中第二个点上的球。在光学中的应用则是解决这样一个问题:“已知一个光源和一个球面镜,求光线应从镜面上的哪个点反射,才能正好进入观察者的眼睛。”这个问题最终可以归结为一个四次方程。[79]为了解决它,海赛姆推导出了四次幂求和公式,而在此之前,人们只知道平方和与立方和的公式。他的方法实际上可以推广到任意次幂求和的情形,尽管他本人并没有进行这一推广(可能是因为他只需要四次幂来计算其关注的抛物旋转体的体积)。他利用所得的幂和公式,完成了我们今天所说的“积分”运算。具体而言,他用平方和与四次幂和的公式来求抛物体(抛物面旋转体)体积。[80]海赛姆最终用圆锥曲线和几何证明解决了这个问题。不过他的解法极其冗长复杂,可能在拉丁文译本中难以被后来的数学家完全理解。此后,数学家们开始使用笛卡尔的解析几何方法来分析这一问题。[81]一直到1965年,精算师杰克·M·埃尔金才首次给出该问题的代数解法。[82]1989年,哈拉尔德·里德(Harald Riede)也提出了解法,[83]1997年,牛津大学数学家彼得·M·诺伊曼给出了另一种解法。[84][85]近年来,三菱电机研究实验室的研究人员更进一步,解决了海赛姆问题在更广义的情形下的推广版本——即适用于任意旋转对称二次曲面反射镜(包括双曲面、抛物面与椭圆面)的解法。[86]
\subsubsection{暗箱}
暗箱的原理早在中国古代就已为人所知,北宋博学家沈括在其科学著作《梦溪笔谈》(1088年出版)中曾加以描述。亚里士多德也在其著作《问题集》中讨论过其基本原理,但海赛姆的研究提供了关于暗箱的第一个清晰描述[87],并进行了该装置的早期分析[88]。

伊本·海赛姆主要使用暗箱来观察日偏食。[89] 在一篇论文中,他写道,自己观察到日食时太阳呈弯月形。论文的引言写道: “日食之际,若非全食,则太阳之像,当其光穿过一小而圆的孔洞,投射到对面平面上时,便呈现出新月状的形状。”

学界普遍认为,海赛姆的发现巩固了暗箱在科学史中的重要地位,[90] 而这篇论文的重要性远不止于此。

在古代与中世纪,光学被分为两类领域:视觉光学和灼光镜学。前者聚焦于视觉机制的研究,后者则探讨光线与发光射线的物理特性。而这篇关于日食形状的论文,可能是海赛姆首次尝试将这两门学科整合为一体的尝试。

伊本·海赛姆的许多发现都源于数学与实验的交汇,这在《论日食形状》中体现尤为明显。这篇论文不仅让更多人可以研究太阳的日偏食,更重要的是促进了对暗箱成像原理的深入理解。这是一篇关于暗箱内成像的物理-数学研究。海赛姆采用实验方法,通过改变孔径的大小和形状、暗箱的焦距、光源的形状与强度,得出了具体成像结果。[91]

在这项研究中,他解释了暗箱成像的倒置现象,[92] 说明了当孔洞很小时,图像类似于原始光源;但也指出,当孔洞较大时,图像会与原始物体产生差异。这一切结果都是通过图像的点分析法得出的。[93]
\subsubsection{折射仪}
在《光学书》第七卷中,海赛姆描述了一种用于研究折射现象的实验装置,以探究入射角、折射角与偏转角之间的关系。这个装置是他对**托勒密所使用仪器的改进版本,用于相似目的的实验。[94][95][96]
\subsubsection{无意识推理}


海赛姆在讨论颜色的过程中,**基本上提出了“无意识推理”这一概念**。他指出,从感知颜色到辨别颜色之间的推理过程,要比感知并识别其他可见特性(除光之外)所需的时间更短,**“其时间之短,以致观察者无法明显察觉。”** 这显然意味着,**颜色与形状的感知并非发生在眼睛本身,而是另有处理之处。**

海赛姆进一步说明,**信息必须传递到“中央神经腔”进行加工**,并写道:

> “感官器官在被来自可见物体的形状所作用之前,**不会感知这些形状**;因此,它不会将颜色当作颜色、将光当作光加以感知,除非它先受到颜色或光的‘形状’的影响。而感官器官从颜色或光的形状中接收到的影响,是一种特定的变化;而**变化必定在时间中发生**;……
> 正是在从感官表面到‘共同神经腔’的过程中,**以及随后的那段时间内**,分布于整个感知身体中的感知能力,才会**将颜色感知为颜色本身**……
> 所以,**感知者对“颜色作为颜色”以及“光作为光”的真正感知**,其实是在颜色的‘形状’自感官表面到达“共同神经腔”之后才发生的。”\[97]

---

**颜色恒常性**
主条目:**颜色恒常性(Color constancy)**

海赛姆在研究中对**颜色恒常性现象**进行了说明。他观察到,**从物体反射的光线会受到物体颜色的调制影响**。他解释说:光的性质与物体颜色**会混合在一起传入眼中**,而视觉系统**再将光与颜色区分开来**。

在《光学书》第二卷第三章中,他写道:

> “从有色物体发出的光,**并不会脱离颜色而单独到达眼睛**;颜色的形状也不会脱离光而单独进入眼睛。存在于有色物体中的光的形状与颜色的形状,**只能混合在一起传播**,而感知器官**只能以混合状态来感知它们**。
> 然而,感知者仍能意识到:可见物体是发光的,且**物体中所见的光不同于颜色**,这二者是**两个不同的性质**。”\[98]
