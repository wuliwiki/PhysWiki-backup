% 楞次定律
% license CCBYSA3
% type Wiki

(本文根据 CC-BY-SA 协议转载自原搜狗科学百科对英文维基百科的翻译)

\begin{figure}[ht]
\centering
\includegraphics[width=6cm]{./figures/b0ab75a5542d423d.png}
\caption{楞次定律给出了电感线圈被通过线圈的磁流变化间接激发的电流方向。图a、b、c、d、e的情况都是可能发生的。图f不可能发生,因为违背了能量守恒定律。电感中的电荷(电子)不是被磁流的改变直接推动的,而是由一个环状的、围绕包括激发磁场与被激发磁场的总磁场的电场(未画出)推动的。这个总磁场激发了电场。} \label{fig_LCDL_1}
\end{figure}

楞次定律(发音 /ˈlɛnts/),以物理学家埃米尔·楞次(Emil Lenz)命名,他在1834年确立了这个定律。[1] 定律表明由变化的磁场在导体中感应的电流,其方向会使得由感应电流产生的磁场与初始磁场的变化方向相反。 或者,正如格里菲斯所总结的:

自然厌恶流的改变.[2]

楞次定律由法拉第感应定律中的负号表示:

\begin{equation}
\mathcal{E} = -\frac{\partial \Phi_B}{\partial t},~
\end{equation}

这表明感应电动势 $\mathcal{E}$ 和磁通量的变化率 $\Phi_B$ 符号相反。这是一个定理性的定律,它确定了感应电流的方向,但没有说明感应电流的大小。楞次定律解释了电磁学中许多效应发生的方向,例如变化的电流在电感或线圈中感应出的电压方向,或者为什么涡流在磁场中对运动的物体施加拖曳力。

楞次定律可以类似于经典力学中的牛顿第三定律。[4]

\subsection{反向电流}

如果电流 $i_1$ 的磁场变化感应出另一个电流 $i_2$ ,$i_2$ 的方向与 $i_1$ 的变化相反。如果这些电流在两个同轴圆形导体 $i_1$ 和 $i_2$ 中,并且两者最初都是 0,那么电流 $i_1$ 和 $i_2$ 必须反向旋转。结果,反向的电流将相互排斥。

楞次定律指出,由于磁场的变化或运动而在电路中感应出的电流,其方向会对抗磁通量的变化,并施加一个机械力来对抗运动。

\subsubsection{1.1 例子}

强磁体原子内部的电流可以在铜管或铝管中产生反向旋转电流。这可以通过将磁铁穿过管道来显示。磁铁在管道内的下降速度明显慢于在管道外下降的速度。

根据法拉第定律,当磁通量的变化产生电压时,感应电压的极性会使得它产生的电流的磁场与产生它的磁场变化相反。任何线圈内部的感应磁场总是试图保持线圈中的磁通量恒定。在下面的例子中,如果通量增加,感应场与之相反。如果通量减小,感应场与施加场的方向相同以对抗通量变化。

\subsection{这些电流中电荷的详细相互作用}

\begin{figure}[ht]
\centering
\includegraphics[width=6cm]{./figures/171c8f3d2feda8b9.png}
\caption{铝环被感生电磁场移动,从而展示了楞次定律。} \label{fig_LCDL_3}
\end{figure}

在电磁学中,当电荷沿着电场线移动时,不管是储存势能(负功)还是增加动能(正功),电场都会对电荷做功。

当净正功作用于电荷时 $ q_1 $,它获得速度和动量。$ q_1 $ 得到的总功会产生磁场,该磁场的强度(以磁通量密度为单位(1特斯拉 = 1伏秒每平方米))与 $ q_1 $ 的速度增加成正比。这个磁场可以与邻近的电荷 $ q_2 $ 相互作用,给它传递动量,而相应地 $ q_1 $ 会损失动量。


电荷 $ q_2 $  也可以以类似地方式作用于 $ q_1 $,通过这种方式,它将它从 $ q_1 $ 得到的动量部分地返还。动量的这种往复的部分贡献给了磁感应。$ q_1 $ 和 $ q_2 $  距离越近效应越显著。当 $ q_2 $  位于导电介质内时,例如由铜或铝制成的厚板,它更容易对 $q_1$ 施加到其上的力作出响应。$ q_1 $ 的能量不会立即消耗为 $ q_2 $  诱导的电流产生的热,也会存储在这两个反向磁场中。磁场的能量密度往往随着磁场强度的平方而变化;然而,在磁性非线性材料如铁磁体和超导体中,这关系就不成立了。
