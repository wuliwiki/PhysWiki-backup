% 完备域
% complete domain|可分多项式|reducible polynomial|微商|形式微商|导数|多项式|重根|自同构|形式求导|形式微分

\pentry{分裂域\upref{SpltFd}}

由\autoref{SpltFd_the1}~\upref{SpltFd}可见,不可约多项式在其分裂域中有无重根,决定了该分裂域的自同构数量.一个域的全体自同构配合映射的复合,能构成一个群,域的许多性质都蕴含在这个群的结构中.群的元素数量自然是其重要性质之一.

综上所述,研究多项式的重根是非常重要的课题.

\subsection{可分多项式}

为了研究重根,我们借用微积分的知识,引入\textbf{形式微商}的概念

\begin{definition}{形式微商}\label{CmplD_def1}

设$\mathbb{F}$是一个域,$f\in\mathbb{F}[x]$.若$f$表达为
\begin{equation}
f(x) = \sum_{i=0}^n a_ix^i
\end{equation}
其中各$a_i\in\mathbb{F}$,那么定义\textbf{形式微商}算子$\opn{D}:\mathbb{F}[x]\to\mathbb{F}[x]$为:
\begin{equation}
\opn{D}f = \sum_{i=0}^{n-1} a_{i+1}x^i
\end{equation}

形式微商也可称为“形式求导”、“形式微分”等.

\end{definition}

\autoref{CmplD_def1} 形式微商就是直接套用微积分中的求导操作,只不过这里没有求导的概念,而是就进行多项式变换,其形式就是求导或者求导的推广,因此才叫\textbf{形式}微商.

\begin{example}{}
考虑域$\mathbb{Z}_3$上的多项式$f(x)=x^5+2x^2-x-2$,则
\begin{equation}
\opn{D}f(x) = 2x^4+x-1
\end{equation}

显然,这和真正的求导不同.一方面我们没有在域$\mathbb{Z}_3$上定义极限的概念,另一方面实数域上$f$的导函数应该是$5x^4+4x-1$.
\end{example}























