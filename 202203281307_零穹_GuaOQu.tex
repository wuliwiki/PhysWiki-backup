% 二次型的规范型
% 二次型|规范型

\begin{issues}
\issueTODO
\end{issues}

\pentry{二次型\upref{QuaFor}}
在规范基底\upref{QuaFor}下,二次型有比较简单的形式,这在理论和应用上都有重要价值.
\begin{theorem}{}
矢量空间 $V$ 上每个对称的双线性型 $f$ 都有规范基底.
\end{theorem}
\textbf{证明:} 对 $V$ 的维度 $n$ 应用数学归纳法.当 $n=1$ 时命题显然.

如果 $f(\bvec x,\bvec y)=0$ 对所有 $\bvec x,\bvec y\in V$ 都成立(即 $f=0$),定理显然对任意基底都适用.如果 $f\neq 0$ ,那么对应二次型也不为0(\autoref{QuaFor_the1}~\upref{QuaFor}).于是存在这么一个向量 $\bvec e_1$ ,使得 $q(\bvec e_1)=f(\bvec e_1,\bvec e_1)\neq0$ .于是,线性函数
\begin{equation}
f_1:\bvec x\mapsto f(\bvec x,\bvec e_1)
\end{equation}
非零($f_1(\bvec e_1)\neq0$).由\autoref{MatLS2_cor1}~\upref{MatLS2},线性子空间
\begin{equation}
L=\mathrm{Ker} f_1=\{\bvec x_1\in V|f_1(\bvec x)=0\}
\end{equation}
的维数是 $n-1$ .其中,$n$ 是矢量空间 $V$ 的维数.

根据归纳法假设,对 $L$ 必有基底 $(\bvec e_2,\cdots,\bvec e_n)$,在此基底下,$f$ 限制在 $L$ 上的矩阵是对角的,即 
\begin{equation}
f(\bvec e_i,\bvec e_j)=0,\quad i,j=2,\cdots ,n,\quad i\neq j
\end{equation}
按构造方式, $f(\bvec e_i,\bvec e_1)=0,i=2,3,\cdots n$.所以得到 $f(\bvec e_i,\bvec e_j)=0,i\neq j$.现在,只需证明向量组 $\bvec e_1,\cdots,\bvec e_n$ 线性无关, $(\bvec e_k)$ 就具有规范基底的特征了.设,情形相反,那么在任意非平凡关系式
\begin{equation}
\alpha_1\bvec e_1+\cdots+\alpha_n\bvec e_n=\bvec 0
\end{equation}
中,只能有 $\alpha_1\neq0$ ,因为 $(\bvec e_2,\cdots ,\bvec e_n)$ 是 $L$ 的基底,但此时 $\bvec e_1=\sum_\limits{i>1}\beta _i\bvec e_i$ 且
\begin{equation}
0\neq f_1(\bvec e_1)=f_1(\sum\limits_{i>1}\beta_i\bvec e_i)=\sum\limits_{i>1}\beta_if_1(\bvec e_i)=0
\end{equation}
这是一个矛盾,证毕!