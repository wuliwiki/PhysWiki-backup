% 原码、反码、补码
% keys 编码|位运算|ALU|计算机组成原理
% license Xiao
% type Wiki

\subsection{原码(True form)}

原码即 “未经更改” 的码,是指一个二进制数左边加上符号位后所得到的码,且当二进制数大于 $0$ 时, 符号位为0;二进制数小于 $0$ 时,符号位为 1;二进制数等于0时,符号位可以为 $0$ 或 1 (+0/-0)。

使用 $n$ 位原码表示\textbf{有符号数}时,范围是 $-(2^{n-1}-1)$ 到 $+(2^{n-1}-1)$。 当 $n=8$ 时,第一位用于符号位, 这个范围就是$-127\sim +127 $;表示\textbf{无符号数}时,由于不需要考虑数的正负,就不需要用一位来表示符号位,$n$ 位机器数全部用来表示是数值,这时表示数的范围就是 $0\sim 2^{n}-1$。当 $n=8$ 时,这个范围就是 $0\sim 255$。


\textbf{优点:}
简单直观,原码易于人类理解和计算(与真值转换容易)。

\textbf{缺点:}原码不能用无符号的加法器进行运算。 例如,数学上,$1+(-1)=0$,但把原码直接进行无符号的加法运算时(即把两个相加的数都视为 0-255 的整数相加):
\begin{equation}
00000001 + 10000001=10000010~.
\end{equation}
该结果对应数值为 $-2$。显然出错了;
对于减法运算,原码减法需要先将减数取反加 $1$,才能得到正确的数学结果。

也就是说:\textbf{原码的运算,必须将符号位和其他位分开},这就增加了硬件的开销和复杂性。 也有人将该符号问题称作正负 $0$ 现象。

\subsection{反码(1's complement)}
我们发现:原码最大的问题就在于一个数加上它的相反数不等于 0,于是反码的设计思想就是冲着解决这一点,既然一个负数是一个正数的相反数,那干脆用一个正数按位取反来表示负数。

在反码表示法中,正数和0的反码与其原码相同;负数的反码则是将原码(除符号位外)的每一位取反。

\textbf{缺点:}
虽然解决了相反数相加不等于 0 的问题, 但是反码不能直接做减法,并且存在多余的负零 (eg. 1111_1111)

\begin{example}{}
0001+1110=1111,1+(-1)=-0;

1110+1100=1010,(-1)+(-3)=-5。
\end{example}

\subsection{补码(2's complement)}

正数和 $0$ 的补码就是该数字本身再补上最高比特0。负数的补码则是将其绝对值按位取反再加1。

\textbf{优点:}
在实现加法器时,只要一种加法电路就可以处理各种有号数加法,因为减法可以用一个数加上另一个数的补码来表示,因此只要有加法电路及补码电路即可完成各种有号数加法及减法,在电路设计上相当方便。(实际在加法器做减法运算时,只需要让减数通过非门,并让个位的“进位”端口从0变为1即可)

另外,补码系统的0就只有一个表示方式,这和反码系统不同(在反码系统中,0有两种表示方式),因此在判断数字是否为0时,只要比较一次即可。


\subsection{补码的设计}

补码的规则看起来很怪,我们可以\textbf{换一种角度理解:}

在设计一个编码时,我们可以把编码后的数围成一个圆(类似一个时钟),我们希望在这个钟表上:
\begin{enumerate}
\item 
无冲突,每个数值的位置都是独一无二,有唯一的0。
\item 
连续性,每次运算(+1)相当于时钟顺时针移动一个单位,尤其是从最大的正数加1会“溢出”到最小的负数,这种看似异常的行为实际上提供了一个连续且循环的数值范围,使得算术运算能够在这个编码系统中顺畅地进行。
\end{enumerate}

从上述角度看,补码是一种很和谐的编码。
