% 流形上的切空间
\pentry{流形\upref{Manif},切空间(欧几里得空间)\upref{tgSpaE}}

对于流形$N$,如果能将它嵌入到某个$\mathbb{R}^k$中,嵌入映射为$i:N\rightarrow\mathbb{R}^k$,那么根据\textbf{切空间(欧几里得空间)}\upref{tgSpaE}中关于曲面$S$的讨论,我们可以使用道路或者导子来算出特定嵌入$i$下流形$N$的切空间和切丛.但是和测试电荷、测试函数类似,特定嵌入也只是一个测试函数,我们讨论流形本身时不依赖特定的嵌入,这就体现出道路和导子定义的好处了.

和大多数教材不同的是,本书中使用道路的等价类来定义流形上的切向量,这样比起导子要更加容易可视化.

\subsection{流形上切空间的定义}

\begin{definition}{切向量}
给定流形$N$,则在其上一点$p\in N$处的一个\textbf{切向量}就是从$p$出发的一条道路$r$所在的等价类$[r]$.其中,两道路$r_1$和$r_2$等价当且仅当存在$p$处的一个图$(U, \varphi)$,使得道路$\varphi\circ r_1$和$\varphi\circ r_2$都收敛于$\varphi(U)$中的同一个切向量.
\end{definition}

切向量的定义只要求两条道路在某一个图中对应的欧几里得空间里的切向量等价.这种定义方法是合理的,这由以下定理保证:

\begin{theorem}{}
给定流形$N$,其上一点$p\in N$处有两个图$(U, \varphi)$和$(V, \phi)$.如果$p$出发的两条道路$r_1$和$r_2$,使得$\varphi\circ r_1$和$\varphi\circ r_2$收敛于同一个切向量,那么$\phi\circ r_1$和$\phi\circ r_2$也收敛于同一个切向量.
\end{theorem}

\textbf{证明:}
为方便计,将$\phi\circ\varphi^{-1}$记为$f$.

由于$f:\mathbb{R}^n\rightarrow\mathbb{R}^n$是一个双向光滑双射,即$f$和$f^{-1}$都是双射且光滑,于是它的Jacobi矩阵$\partial f/\partial \bvec{v}$是非奇异的;换句话说,如果把向量值函数$f$的第$i$个分量函数记为$f_i:\mathbb{R}^n\rightarrow\mathbb{R}$,那么$f_i$的梯度$\Nabla f_i$处处存在且不为零.类似地,$f^{-1}$的分量函数的梯度也处处存在且不为零.在以下证明中,为了方便,我将直接使用Jacobi矩阵的表示方法.

$\varphi\circ r_1$对应的向量是$\dd(\varphi\circ r_1)/\dd t=(\dd\varphi/\dd\bvec{v})\cdot(\dd r_1/\dd t)=$,其中$\bvec{v}$表示$\mathbb{R}^n$中的向量,$\dd\varphi/\dd\bvec{v}$取在点$\varphi(p)$处的值.

现在直接计算$\phi\circ r_1$和$\phi\circ r_2$对应的向量:$\dd(\phi\circ r_1)=$

\textbf{证毕.}


%休息了,起来继续讲为什么收敛于同一个向量.重点:双向光滑双射,必有分量函数的梯度存在且恒不为零.


%向量的加法定义为$[r_1]+[r_2]=[\varphi^{-1}\circ(\varphi\circ r_1+\varphi\circ r_2)]$,即将代表元素用$\varphi$映射到欧几里得空间后相加,再映射回来的结果作为和的代表元素.向量的数乘类似,定义为$a[r]=[\varphi^{-1}\circ(a\varphi\circ r)]$.全体收敛的道路的等价类按上述向量加法和数乘,构成一个线性空间,称为$p$处的\textbf{切空间(tangent space)}.


