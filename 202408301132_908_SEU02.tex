% 东南大学 2002 年 考研 量子力学
% license Usr
% type Note

\textbf{声明}:“该内容来源于网络公开资料,不保证真实性,如有侵权请联系管理员”

\subsection{(15分)}
设粒子在下列势阱中运动,是否存在束缚态?求存在束缚态的条件。
\[
V(x) =
\begin{cases}
\infty, & x < 0 \\
-\gamma \delta(x - a), & x > 0 \quad (\gamma, a > 0)
\end{cases}~
\]

\subsection{(15分)}
设 $\lambda$ 为常数,$\sigma_z$ 为 Pauli 矩阵,证明:
    \[
    e^{i \lambda \sigma_z} = \cos \lambda + i \sigma_z \sin \lambda~
    \]

\subsection{(15分)}
设 $F(\vec{r},\vec{p})$ 为厄密算符,证明无终止量表象中的求和规则为:
    \[
    \sum_n (E_n - E_k) |F_{nk}|^2 = \frac{1}{2} \langle k | [F, [H, F]] | k \rangle~
    \]

\subsection{(15分)}
设中性原子在含有电场 $\mathbf{E}$ 和均匀磁场 $\mathbf{B}$ 中运动, 计算系统的能级和相关量子数。

\subsection{(20分)}
设碱金属原子中价电子所受原子实(原子核+满壳电子)的作用近似表示为
\[
V(r) = -\frac{e^2}{r} - \lambda \frac{e^2 a}{r^2} \quad (0 < \lambda \ll 1)~
\]
其中 $a$ 为 Bohr 半径,求价电子的能级。
