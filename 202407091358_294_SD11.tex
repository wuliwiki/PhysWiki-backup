% 苏州大学 2011 年硕士物理考试试题
% keys 苏州大学|考研|物理|2011年
% license Copy
% type Tutor
\textbf{科目代码:832}
\begin{enumerate}
\item 两个长方形物体$A$和$B$紧靠放在光滑的水平桌面上,已知$m_A=2kg,m_B=3kg$,有一质量$ m=100g$的子弹以速率 $v_0=800m/s$水平射入长方体 $A$,经$t=0.01s$,又射入长方体$B$,最后停留在长方体$B$内未射出。设子弹射入$A$时所受的摩擦力$F_r=3x10^3N$,求:\\
(1)子弹在射入$A$的过程中,$B $受到$A$的作用力的大小;\\
(2)当子弹留在$ B$ 中时,$A$ 和$B$的速度大小。
\item 一长为$L$,质量为$m$的均质细棒,一端可绕固定的水平光滑轴$O$在竖直平面内转动,在$O$点还系有一长为$b(b<L)$的细绳,绳的另一端悬挂一质量也为$m$的小球。当小球悬线偏离竖直方向某一角度时,由静止释放。已知小球与细棒发生完全弹性碰撞,要使碰撞后小球刚好停止,问绳的长度$b$应为多少?
\item 某火车驶过车站时,站台上的观测者测得火车汽笛频率由 $1200Hz$变到了 $1000 Hz$,设空气中声速为 $330m/s$,求该火车的速率。
\item 一列机械波沿$x$轴正向传播,$t=0$时的波形如图所示,已知波速为 $10m/s$,波长为 $2m$,求:\\
(1)波动方程;\\
(2)$P$点的振动方程及振动曲线;\\
(3)$P $点的坐标。
\item 地面上有一固定的点电荷$ A$,在$A$的正上方有一带电小球$B$,$B$在重力和$A$的库仑斥力的作用下,在$A$上方$ H/2 $到$H$之间作往返的自由振动。试求$B$运动的最大速率$v_{max}$。
\item  如图所示,一根塑料棒带有均匀分布的电荷$-Q$,塑料棒被弯曲成120°半径为$r$的圆弧。建立如图所示的坐标轴,原点在圆弧的曲率中心$P$点。求$P$点的场强$E$(用$Q$和$r$表示)。
\item 一长直圆柱形导线由内外两种导电材料构成,截面如图所示,导线外半径$R_2$,内半径$R_1$,内外导体的电导率和磁导率分别为可$\sigma$和,4。导线中沿轴向通以电流了,求内外导体的磁场强度。
\end{enumerate}
