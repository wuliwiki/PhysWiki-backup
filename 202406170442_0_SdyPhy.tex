% 如何自学物理
% license CCBYSA3
% type Tutor

要谈论这个话题,首当其冲的应该是荷兰诺贝尔物理学奖得主 Gerard 't Hooft 的文章 \href{https://webspace.science.uu.nl/~hooft101/theorist.html}{How to become a GOOD Theoretical Physicist} (\href{https://xialab.pku.edu.cn/kytdyw1/zdylm.m.jsp?wbtreeid=1011&tstreeid=11956&_t_uid=2945&language=en&homepageuuid=BF649325C5584FC683CE0B601D21AC65&templateuuid=4CC182410BA14FF8B55ED726FB2087FB&producttype=0&_tmode_=99&tsitesapptype=zdylm}{一个翻译}), 我们不妨就这篇文章的核心内容展开讨论。

\subsection{物理学的主要研究方向}
首先,虽然文章标题说的是 “理论物理学家”, 但大致来说也\textbf{同样适用于物理学其他研究方向}。 对于公众来说,可能一谈到物理学马上就会想到那些喜闻乐见的理论物理话题例如宇宙、黑洞、粒子物理等。 这些话题的确非常引人入胜,但远非物理学的全部。 事实上只有一小部分从事物理学研究的人会研究这些领域。

\begin{figure}[ht]
\centering
\includegraphics[width=14.25cm]{./figures/b898880f835f999d.png}
\caption{美国一项调查中物理博士的研究领域占比(参考\href{https://ww2.aip.org/statistics/trends-in-physics-phds}{来源})调查中的,从上到下分别是凝聚态物理、粒子和场、天体物理/宇宙学、原子分子光学、生物物理、核物理、材料/纳米/表面、光学/光子学、计算物理、等离子/聚变物理、应用/工程/能源研究、量子基础/信息理论、复杂系统/统计/非线性/热物理、相对论/引力、软物质/聚合物物理、其他} \label{fig_SdyPhy_1}
\end{figure}

\subsection{关于民科}
\href{https://webspace.science.uu.nl/~hooft101/theoristbad.html#:~:text=On\%20your\%20way\%20towards\%20becoming,have\%20your\%20work\%20published\%20anyway.}{How to become a bad theoretical physicist}(\href{https://zhuanlan.zhihu.com/p/38680467}{一个翻译})

\subsection{英语}
正文首先提到的就是英语,可见英语的重要性。 注意这篇文章是面向全世界的读者的,所以对中国人也同样适用。 英语是事实上的国际学术交流语言。 在物理领域(以及其他大部分自然科学领域),绝大多数论文都是用英语在欧美期刊上发表的,包括国内的研究者。 哪怕是为数较少的国产 SCI 期刊,也多数是英语或双语的。 也就是说,即使是国内的物理研究者之间想要互相了解研究进展,也一般需要读英语论文。

即使抛开学术期刊只谈物理科普和教学方面的网络资源(其他自然科学也基本适用),英语内容无论是数量还是质量都要远超中文。 所以如果你用中文在百度搜索感兴趣的话题但没发现什么有价值的资源,你要知道这远非互联网上的全部内容, 正如 Hooft 所说,你成为理论物理学家所需的一切都可以在互联网上找到, 但关键是你需要分辨哪些是真正对你有用的。 具体到语言,他所说的这些资源几乎都是以英语书写的。

所以无论是学习还是科研,一定程度的英语水平是进入物理领域的门槛, 多早开始学英语(尤其是物理方面的)都不为过。
