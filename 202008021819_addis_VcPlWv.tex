% 真空中的平面电磁波
% 麦克斯韦方程组|电磁波|平面波|波动方程

\pentry{麦克斯韦方程组\upref{MWEq}}

\begin{figure}[ht]
\centering
\includegraphics[width=9cm]{./figures/VcPlWv_1.pdf}
\caption{平面电磁波的电磁场分布. 注意于电磁场矢量与 $x, y$ 坐标无关, 并占据整个空间(图片来自维基百科)} \label{VcPlWv_fig1}
\end{figure}

平面电磁波如\autoref{VcPlWv_fig1} 所示. 电场与磁感应强度的关系为
\begin{equation}
B_0 = E_0/c
\end{equation}
平均能流密度(光强)为
\begin{equation}
I = \frac12 c\varepsilon_0 E_0^2
\end{equation}

% 可以考虑从平面波入手介绍波印廷矢量

\subsection{推导}
(参考 \cite{GriffE} 的推导)

\begin{equation}
\curl(\curl \bvec E) = -\curl\pdv{\bvec B}{t} = -\epsilon_0\mu_0 \pdv[2]{\bvec E}{t}
\end{equation}

\begin{equation}
\curl(\curl \bvec E) = \grad(\div \bvec E) - \laplacian \bvec E = - \laplacian \bvec E
\end{equation}

\begin{equation}
\laplacian E = \epsilon_0 \mu_0 \pdv[2]{\bvec E}{t}
\end{equation}

\begin{equation}
E_y = E_0\cos(\omega t - kx)
\end{equation}

\begin{equation}
c = \frac{1}{\sqrt{\epsilon_0\mu_0}}
\end{equation}
