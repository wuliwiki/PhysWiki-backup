% 连分数

\footnote{参考 Wikipedia \href{https://en.wikipedia.org/wiki/Continued_fraction}{相关页面}.}\textbf{连分数(continued fraction)}形如
\begin{equation}
a_0 + \frac{1}{\displaystyle a_1 + \frac{1}{\displaystyle a_2 + \frac{1}{\displaystyle \ddots + \frac{1}{a_n}}}}
\end{equation}
其中要求所有 $a_i$ 都是整数, 且除 $a_0$ 外, 其他 $a_i$ 都大于零. 若 $n\to\infty$, 则将其称为\textbf{无穷连分数}. 如果不要求 $a_i$ 为整数, 也不要求分子都为 1, 那么称其为\textbf{广义连分数}, 见下文. 为了方便书写, 也可以记为
\begin{equation}
a_0 + \frac{1}{a_1 + \dots}\frac{1}{a_2 + \dots}\dots \frac{1}{a_n}
\end{equation}
或者更简洁地, 记为
\begin{equation}
123
\end{equation}




\subsection{广义连分数}
可以记为
\begin{equation}
b_0 + \frac{a_1}{\displaystyle b_1 + \frac{a_2}{\displaystyle b_2 + \frac{a_3}{\displaystyle b_3 + \dots}}}
\end{equation}