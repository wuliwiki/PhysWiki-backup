% Keldysh 参数

\begin{equation}
\gamma = \sqrt{\frac{I_p}{2U_p}} = \frac{\omega}{e\mathcal E_0} \sqrt{2m_e I_p}
\end{equation}
$I_p$ 为电离能量, $U_p$ 为 ponderomotive 能量. $\gamma$ 越小($\gamma < 1$), Tunnelling ionisation 越容易发生.

\subsection{Ponderomotive Energy}
定义为在平面电磁波(忽略磁场)中的粒子一个周期内的平均能量.
\begin{equation}
U_p = \frac{e^2 \mathcal E_0^2}{4m_e\omega^2}
\end{equation}
$\mathcal E_0$ 是电场, $\omega$ 是激光频率.

零电场为
\begin{equation}
\mathcal E(t) = \mathcal E_0 \sin(\omega t + \phi)
\end{equation}
点电荷 $q$ 在电场中的加速度为
\begin{equation}
a = \frac{q\mathcal E_0}{m} \sin(\omega t + \phi)
\end{equation}
速度为
\begin{equation}
v = -\frac{q\mathcal E_0}{m\omega} \cos(\omega t + \phi) + v_0
\end{equation}
动能为
\begin{equation}
\overline{E_k} = \frac{1}{2}m \overline{v^2} = \frac{1}{2}m \qty[-\frac{q\mathcal E_0}{m\omega} \cos(\omega t + \phi) + v_0]^2 = \frac{q^2\mathcal E_0^2}{4m\omega^2} + \frac{1}{2}mv_0^2
\end{equation}
所以当电子做简写震动, 即 $v_0 = 0$ 时的平均动能就是 $U_p$. 如果是简谐振动和平移的叠加, 就要多加上平移的动能.

\subsection{ADK Rate}
M. Ammosov, N. Delone, V. Krainov. 三个人给出了一般原子的瞬时的 tunnelling ionization rate, 叫做 \textbf{ADK rate}.
\begin{equation}
\dv{P}{t} = \omega_p \abs{C_{n^*l^*}}^2 G_{lm}\qty(\frac{4\omega_p}{\omega_T})^{2n^*-m-1}\exp(-\frac{4\omega_p}{3\omega_T})
\end{equation}
其中
\begin{equation}
\omega_p = \frac{I_p}{\hbar} \qquad \omega_T = \frac{e\mathcal E_0}{\sqrt{2mI_p}}
\end{equation}
