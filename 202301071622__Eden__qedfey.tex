% QED的费曼规则
% QED的费曼规则|量子电动力学|费曼图

对旋量场、矢量场的 Wick 定理与 LSZ 约化公式作一个整理和总结,我们得最终到了旋量 QED 的 Feynman 规则:
\textbf{旋量 QED 的 Feynman 规则(动量空间)}
\begin{enumerate}
\item 画出所有连通的、截肢的费曼图。
\item 给每一个传播子一个四动量,并在每个顶点要求动量守恒。
\item 对于动量为 $p$ 的费米子传播子,写下:$\frac{i(\not p+m_0)}{p^2-m_0^2 + i\epsilon}$。
\item 对于动量为 $q$、两端矢量指标为 $\mu,\nu$ 的光子传播子,写下:$\frac{-ig_{\mu\nu}}{q^2 + i\epsilon}$。
\item 对于相互作用顶点,写下:$-ie\gamma^\mu$;
\item 对于外线的费米子和反费米子,写下它们对应的旋量:
\begin{align*}
&\wick{\c1\psi\c1{\ket{\bvec p,s,+}}}=u^s(\bvec p),\quad \wick{\c1{\bra{\bvec p',s',+}} \c1{\bar\psi}}=\bar u^{s'}(\bvec p')\\
&\wick{\c1{\bar\psi}\c1{\ket{\bvec k,r,-}}}=v^{r}(\bvec k),\quad \wick{\c1{\bra{\bvec k',r',-}} \c1\psi}=\bar v^{r'}(\bvec k')
\end{align*}
\item 对于外线的光子,写下它们对应的偏振矢量:
\[
\wick{\c1A^\mu(x) \c1{\ket{\bvec k,\lambda}} }=\epsilon^{(\lambda)\mu}(\bvec k),\quad \wick{\c1{\bra{\bvec k',\lambda'}}\c1A^\nu(x) }=\epsilon^{(\lambda)\nu*}(\bvec k')
\]
\item 
对所有未知动量积分。
\item 
考察每个图中由于费米统计所可能造成的符号,例如一个费米子圈总会贡献一个负号。
\end{enumerate}
\addTODO{规范场传播子的 $R_\xi$ 规范}
不同 $\xi$ 规范下的光子传播子是不同的,不过后面我们将通过 Ward 等式证明 S-矩阵和选取的 $\xi$ 是无关的。

类似于汤川理论的情形,我们可以讨论旋量 QED 中两个非相对论性的费米子的散射。与汤川理论不同的是,这里传递相互作用的粒子是光子。经过一系列计算,最终可以得到两个非相对论性电子的库伦势:
\[
V(r)=\frac{e^2}{4\pi r}=\frac{\alpha}{r}
\]
其中 $\alpha$ 是精细结构常数。对于两个非相对论性反费米子,也可以得到同样的结果,即相互排斥的库仑势;而对于费米子与反费米子,它们之间则是相互吸引的库仑势。
