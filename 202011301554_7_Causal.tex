% 因果结构
% 相对论|时空|广义相对论|因果性|因果|曲线|黎曼度量|伪黎曼度量

\pentry{连通性\upref{Topo3},黎曼度量与伪黎曼度量\upref{RiMetr}}



在研究时空的理论中,因果结构是一个基本问题.根据David B. Malament的论文\textsl{The class of continuous timelike curves determines the topology of spacetime}\footnote{Malament, Journal of Mathematical Physics 18:7, 1399-1404. },在一个给定的时空中,确定了因果性就可以完全确定该时空的微分结构.本词条旨在解释与因果结构相关的部分概念.

\subsection{概念}

一个\textbf{时空(spacetime)}是指一个无界的、连通的四维伪黎曼流形$(M, g)$,其中$g$是一个洛伦兹度规.该度规在任何观察者的参考系中可以表示为对角矩阵$\opn{diag}(1, -1, -1, -1)$.

一条\textbf{曲线(curve)}是指一个从实数轴上的任意开区间$I$到拓扑空间的连续映射,在时空理论中这个拓扑空间自然是指时空.如果曲线是一个光滑映射,那么也被称为一条\textbf{光滑曲线(smooth curve)}.另外,此术语等同于一些拓扑学家口中的\textbf{道路(path)}




