% 库仑函数
% 薛定谔方程|库仑函数|库默尔函数|惠特克函数|库仑相移|朗斯基行列式

\pentry{二阶常系数齐次微分方程\upref{Ode2}, Gamma 函数\upref{Gamma}, 库仑散射(量子)\upref{CulmWf}}

\subsection{定义}
\footnote{参考 \href{https://dlmf.nist.gov/33.2}{NIST 相关页面}。}在球坐标系中解库仑势场中的定态薛定谔方程\upref{SchEq}, 当能量 $E > 0$ 时\footnote{当能量 $E < 0$ 时将得到束缚态, 见 “类氢原子的定态波函数\upref{HWF}”}, 会得到径向方程(\autoref{CulmWf_eq5}~\upref{CulmWf})
\begin{equation}
\dv[2]{u}{\rho} + \qty[1 - \frac{2\eta}{\rho} - \frac{l(l+1)}{\rho^2}]u = 0~.
\end{equation}
这是一个二阶线性齐次常微分方程, 其两个线性无关解为\textbf{第一类库仑函数(Coulomb function of the first kind)} $F_l(\eta, \rho)$ 和\textbf{第二类库仑函数} $G_l(\eta, \rho)$。 当方程的系数为实数时, 第一类和第二类库仑函数也是实函数。
\addTODO{应该把图中的 $Z=-1$ 改成 $\eta = ?$}

\begin{figure}[ht]
\centering
\includegraphics[width=14cm]{./figures/CulmF_1.pdf}
\caption{第一类库仑函数} \label{CulmF_fig1}
\end{figure}

第一类库仑函数(\autoref{CulmF_fig1})的解析式为(式中两套正负号是等价的)
\begin{equation}\label{CulmF_eq1}
F_l(\eta, \rho) = A_l(\eta) \rho^{l+1} \E^{\mp\I\rho} M(l+1\mp\I\eta, 2l+2, \pm 2\I\rho)~,
\end{equation}
其中
\begin{equation}
A_l(\eta) = \frac{2^l \E^{-\pi\eta/2} \abs{\Gamma(l+1+\I\eta)}}{(2l+1)!}
\end{equation}
$M(a, b, z)$ 是\textbf{库默尔(Kummer) M 函数}, 也叫第一类合流超几何函数\upref{HypGeo}, 记为 $_1 F_1(a;b;z)$。 库仑函数也可以用\textbf{惠特克(Whittaker) M 函数}来表示得更紧凑
\begin{equation}\label{CulmF_eq2}
F_l(\eta, \rho) = \qty(\frac{\mp\I}{2})^{l+1} A_l(\eta) M_{\pm\I\eta, l+\frac12}\qty(\pm 2\I\rho)
\end{equation}
其中惠特克 $M$ 函数与库默尔 $M$ 函数的关系为
\begin{equation}
M_{\kappa, \mu}(z) = \E^{-z/2} z^{\mu + 1/2} M(\mu - \kappa + 1/2, 1 + 2\mu, z)
\end{equation}
另见 “库仑函数程序(Matlab 和 Mathematica)\upref{FlCode}”。

\subsubsection{库仑相移}
$F_l(\eta, \rho)$ 是一个实函数, 且渐进形式为
\begin{equation}\label{CulmF_eq7} % 已数值验证
F_l(\eta, \rho) \overset{\rho\to \infty}{\longrightarrow} \sin\qty[\rho - \frac{\pi l}{2} - \eta\ln(2\rho) + \sigma_l(\eta)]~,
\end{equation}
其中 $\sigma_l(\eta)$ 是\textbf{库仑相移(Coulomb phase shift)}
\begin{equation}\label{CulmF_eq8}
\sigma_l(\eta) = \arg[\Gamma(l+1+\I\eta)]
\end{equation}
其中 $\arg$ 函数是复数的幅角(\autoref{CplxNo_eq5}~\upref{CplxNo})。

\subsection{渐进形式}
类似 $\rho$ 乘以第一类球贝塞尔函数\footnote{当 $\eta = 0$ 时 $F_l(\eta, \rho)=\rho j_l(\rho)$。} $j_l(\rho)$, 有
\begin{equation}
F_l(\eta, 0) = 0 \qquad \eval{\dv{F_l(\eta, \rho)}{\rho}}_{\rho=0} = 
\begin{cases}
A_0(\eta) &(l = 0)\\
0     &(l > 0)
\end{cases}
\end{equation}
更一般地, 当 $r\to 0$ 时有
\begin{equation} % mathematica 已验证
F_l(\eta, \rho) \rightarrow A_l(\eta) r^{l+1} + \order{r^{l+1}}
\end{equation}
由此可见当 $r\to 0$ 时只有 $l = 0$ 时 $F_l(\eta, \rho)/r \to A_0(\eta)$ 不为零。

\subsection{其他类型的库仑函数}

第二类库仑函数 $G_l(\eta, \rho)$ 可以由 $H_l^\pm(\eta, \rho)$ 得到, 后者是两类库仑函数的线性组合\footnote{类比 $\exp(\I x) = \cos x + \I\sin x$。}
\begin{equation}
H_l^\pm(\eta, \rho) = G_l(\eta, \rho) \pm \I F_l(\eta, \rho)
\end{equation}	
定义为
\begin{equation}\label{CulmF_eq10}
H_l^\pm(\eta, \rho) = (\mp\I)^l \E^{\pi\eta/2 \pm \I \sigma_l(\eta)} W_{\mp\I\eta, l + 1/2}(\mp 2\I\rho)
\end{equation}
其中 $W_{\kappa, \mu}(z)$ 是\textbf{惠特克(Whittaker) W 函数}
\begin{equation}\label{CulmF_eq11}
W_{\kappa, \mu}(z) = \E^{-z/2} z^{\mu + 1/2} U(\mu - \kappa + 1/2, 1 + 2\mu, z)
\end{equation}
其中 $U(a, b, z)$ 是\textbf{库默尔(Kummer) U 函数}, 定义为
\begin{equation}
U(a, b, z) = \frac{\Gamma(1 - b)}{\Gamma(a + 1 - b)} M(a, b, z) + \frac{\Gamma(b - 1)}{\Gamma(a)} z^{1 - b} M(a + 1 - b, 2 - b, z)
\end{equation}
\autoref{CulmF_eq11} 代入\autoref{CulmF_eq10} 得
\begin{equation}
H_l^\pm(\eta, \rho) = \E^{\pm\I\theta_l(\eta, \rho)}(\mp 2\I\rho)^{l + 1 \pm\I\eta} U(l + 1 \pm \I\eta, 2l + 2, \mp2\I\rho)
\end{equation}
其中
\begin{equation}
\theta_l(\eta, \rho) = \rho - \frac{l\pi}{2} - \eta\ln(2\rho) + \sigma_l(\eta)
\end{equation}
是\autoref{CulmF_eq7} 中的渐进相位。

\subsubsection{性质}
$G_l(\eta,\rho)$ 的渐进形式为
\begin{equation}
\lim_{\rho\to \infty} G_l(\eta, \rho) = \cos\theta_l(\eta, \rho)
\end{equation}

库仑函数的导数可以由惠特克函数的导数得到。
\begin{equation}
\pdv{z} [M_{a, b}(z)] = \qty(\frac12 - \frac{a}{z}) M_{a, b}(z) + \frac{1}{x} \qty(a + b + \frac12) M_{a+1, b}(z)
\end{equation}
\begin{equation}
\pdv{z} [W_{a, b}(z)] = \frac{1}{2z} \qty[(z - 2a) W_{a, b}(z) - 2W_{a+1, b}(z)]
\end{equation}
可见程序中求导数大概要比求函数值多用一倍时间(因为要求两次惠特克函数), 但同时也顺便求出了函数值。 如果同时需要两类库仑函数及它们的导数, 只需求 $H^+$ 函数的导数, 这就顺便求出了 $H^+$, 再分别取实部虚部即可。 另外只需要把 $H^+$ 做复共轭即可得到 $H^-$。

一种验证函数值是否正确的方法是使用以下性质
\begin{equation}
\mathcal{W}\{G, F\} = \mathcal{W}\{H^\pm, F\} = 1
\end{equation}
其中 $\mathcal{W}\{f_1(x), f_2(x)\}$ 是二阶\textbf{朗斯基行列式(Wronskian determinant)}
\begin{equation}
\mathcal{W}\{f_1(x), f_2(x)\} = \begin{vmatrix}
f_1(x)  & f_2(x) \\
f_1'(x) & f_2'(x)
\end{vmatrix} = f_1(x) f_2'(x) - f_2(x) f_1'(x)
\end{equation}


由渐进形式可得径向归一化积分\footnote{积分时可忽略 $\sin$ 中的额外相位, 但笔者不会证。}与球贝塞尔函数乘以 $kr$ 的相同(\autoref{SphBsl_eq4}~\upref{SphBsl})
\begin{equation}
\int_0^\infty F_l(-Z/k', k' r)F_l(-Z/k, kr) \dd{r} = \int_0^\infty \sin(k'r)\sin(kr) \dd{r} = \frac{\pi}{2}\delta(k - k') \qquad (k, k' > 0)~.
\end{equation}
