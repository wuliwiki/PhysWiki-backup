% 平面上的平移、旋转与缩放
% license Usr
% type Tutor

\pentry{平面旋转矩阵\nref{nod_Rot2D}}{nod_0d06}

\subsection{主动理解}
主动理解即坐标系只有一个, $\bvec r = (x, y)$ 变换后得到另一个向量 $\bvec r' = (x',y')$。

\begin{itemize}
\item 旋转矩阵(逆时针为正)为 $\mat R(\theta)$, 即 $\bvec r' = \mat R(\theta)\bvec r$ 把 $\bvec r$ 关于坐标原点旋转 $\theta$ 弧度得到 $\bvec r'$。
\item 缩放常数 $s$ 可以把一点关于原点收缩: $\bvec r' = s\bvec r = (sx, sy)$。
\item 平移向量 $\bvec d = (d_x,d_y)$ 可以描述平移 $\bvec r' = \bvec r + \bvec d = (x+d_x,y+d_y)$。
\end{itemize}

如果三种变换依次作用在 $\bvec r$ 上,但顺序不同,会导致最终结果相差一个平移。例如:

“平移,旋转,缩放” 为
\begin{equation}
\bvec r' = s\mat R (\bvec r + \bvec d) = s\mat R\bvec r + s\mat R\bvec d~.
\end{equation}

“旋转,平移,缩放” 为
\begin{equation}
\bvec r' = s(\mat R\bvec r + \bvec d) = s\mat R\bvec r + s\bvec d~.
\end{equation}

“缩放,平移,旋转” 为
\begin{equation}
\bvec r' = \mat R(s\bvec r + \bvec d) = s\mat R\bvec r + \mat R\bvec d~.
\end{equation}
可见使用\aref{矩阵乘法分配律}{eq_Mat_13}拆括号后,得到 “旋转,缩放,平移” 变换。第一项的旋转缩放都是相同的,只有平移常数不同。

另外注意旋转和缩放都是线性变换,且二者顺序可交换,即 $s(\mat R\bvec r) = \mat R (s \bvec r)$。

\subsection{被动理解}
被动理解的意思是,存在两个坐标系 $S,S'$, 当 $S'$ 相对 $S$ 旋转,缩放或平移后, $S$ 中一个向量 $\bvec r$ 在 $S'$ 中的坐标是什么。 也就是同一个向量在不同坐标系中的坐标如何变换。

\addTODO{用 $\bvec R^{-1}$,$1/s$ 和 $-\bvec d$ 即可!}
