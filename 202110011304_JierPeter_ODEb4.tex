% 高阶线性微分方程的降阶
% 线性微分方程组|常微分方程|ode|ordinary differential equation

\pentry{一阶常系数线性微分方程组(常微分方程)\upref{ODEb3},拉普拉斯变换与常系数线性微分方程\upref{ODELap}}

在\textbf{拉普拉斯变换与常系数线性微分方程}\upref{ODELap}中,我们讨论了高阶常系数线性微分方程的解法,通过拉普拉斯变换,将高阶微分方程化为代数方程进行求解.但是很多时候,单个的高阶微分方程也是可以化为多个一阶方程的,从而可以应用\textbf{一阶常系数线性微分方程组(常微分方程)}\upref{ODEb3}中的方法来进行求解.

一般地,$n$阶微分方程可以写为
\begin{equation}\label{ODEb4_eq1}
F(t, x(t), \frac{\dd}{\dd t}x(t), \cdots, \frac{\mathrm{d}^n}{\dd t^n}x(t))=0
\end{equation}
接下来,我们介绍几种可以降阶的方程.

\subsection{第一种}

如果\autoref{ODEb4_eq1} 不显含$x$的$k$次及以下导函数,即方程形式为
\begin{equation}
F(t, \frac{\mathrm{d}^{k+1}}{\dd t^{k+1}}x(t), \frac{\mathrm{d}^{k+2}}{\dd t^{+2}}x(t), \cdots, \frac{\mathrm{d}^n}{\dd t^n}x(t))=0
\end{equation}

那么我们可以设$y=\frac{\mathrm{d}^k}{\dd t^k}x$,解出$y$以后再求$k$次积分得到$x$.


\begin{example}{}
求解方程
\begin{equation}\label{ODEb4_eq2}
\qty(
    \frac{\mathrm{d}^4}{\dd t^4}+\frac{1}{t}\frac{\mathrm{d}^3}{\dd t^3}
    )
    x(t)=0
\end{equation}

令$y=\frac{\mathrm{d}^3}{\dd t^3}x(t)$,则\autoref{ODEb4_eq2} 化为
\begin{equation}
\frac{\dd y}{\dd t}+\frac{1}{t}y=0
\end{equation}
解得
\begin{equation}
y=\pm\frac{C}{t}
\end{equation}

三次积分后得到
\begin{equation}
x=t^2(C_1\ln\abs{t}-C_2)+C_3t+C_4
\end{equation}



\end{example}


\subsection{第二种}

如果\autoref{ODEb4_eq1} 不显含$t$,即方程形式为
\begin{equation}
F(x(t), \frac{\mathrm{d}}{\dd t}x(t), \frac{\mathrm{d}^2}{\dd t^2}x(t)+\cdots+\frac{\mathrm{d}^n}{\dd t^n}x(t))=0
\end{equation}
那么我们可以设$$
















