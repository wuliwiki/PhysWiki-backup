% 伯恩哈德·黎曼(综述)
% license CCBYSA3
% type Wiki

本文根据 CC-BY-SA 协议转载翻译自维基百科\href{https://en.wikipedia.org/wiki/Bernhard_Riemann}{相关文章}。

\begin{figure}[ht]
\centering
\includegraphics[width=6cm]{./figures/e77e2d063c32a0a3.png}
\caption{黎曼,大约1863年} \label{fig_BEHDLM_1}
\end{figure}
乔治·弗里德里希·伯恩哈德·黎曼(德语:[ˈɡeːɔʁk ˈfʁiːdʁɪç ˈbɛʁnhaʁt ˈʁiːman] ⓘ;1826年9月17日 – 1866年7月20日)是德国数学家,对分析学、数论和微分几何学做出了深远的贡献。在实分析领域,他最著名的是首次严格提出的黎曼积分及其在傅里叶级数方面的工作。他在复分析方面的贡献,尤其是引入了黎曼曲面,为复分析的自然几何处理开辟了新天地。他1859年关于素数计数函数的论文,其中包含了黎曼猜想的原始表述,被认为是分析数论的基础性论文。通过在微分几何学方面的开创性贡献,黎曼为广义相对论的数学奠定了基础。[3] 许多人认为他是历史上最伟大的数学家之一。[4][5]
\subsection{传记}  
\subsubsection{早年}  
黎曼于1826年9月17日出生在布雷泽伦茨(Breselenz),一个位于汉诺威王国丹嫩贝尔格附近的村庄。他的父亲弗里德里希·伯恩哈德·黎曼是布雷泽伦茨的贫穷路德宗牧师,曾参加拿破仑战争。他的母亲夏洛特·艾贝尔于1846年去世。黎曼是六个孩子中的第二个。黎曼从小展现出卓越的数学才能,如出色的计算能力,但他也患有羞怯和害怕公开演讲的症状。
\subsubsection{教育}  
1840年,黎曼前往汉诺威与祖母同住,并就读于文法学校(相当于中学),因为他所在的村庄没有这样的学校。1842年,祖母去世后,黎曼转学至吕讷堡的约翰纽姆中学。在那里,黎曼集中特别学习圣经,但他经常被数学分心。他的老师们惊讶于他的数学运算能力,黎曼经常超越老师的知识水平。1846年,19岁的黎曼开始学习语言学和基督教神学,打算成为一名牧师,并帮助家庭经济。

在1846年春天,他的父亲积攒了一些钱,将黎曼送到哥廷根大学,原计划学习神学并获得神学学位。然而,一到哥廷根,黎曼便开始在卡尔·弗里德里希·高斯的指导下学习数学(尤其是高斯的最小二乘法讲座)。高斯建议黎曼放弃神学事业,转而进入数学领域;得到父亲同意后,黎曼于1847年转学至柏林大学。在柏林学习期间,卡尔·古斯塔夫·雅各布·雅可比、彼得·古斯塔夫·勒让·狄利克雷、雅各布·斯坦纳和戈特霍尔德·艾森斯坦等数学家为他授课。黎曼在柏林待了两年,直到1849年才返回哥廷根。
\subsubsection{学术生涯}  
黎曼于1854年首次讲授课程,创立了黎曼几何学领域,为阿尔伯特·爱因斯坦的广义相对论奠定了基础。[7] 1857年,曾尝试将黎曼晋升为哥廷根大学的特任教授,尽管未能成功,但该尝试促使黎曼最终获得了定期薪水。1859年,随着狄利克雷(当时担任高斯的讲席教授)去世,黎曼被提升为哥廷根大学数学系主任。他也是第一个提出使用三维以上的维度来描述物理现实的人。[8][7]

1862年,他与埃莉丝·科赫结婚;他们的女儿伊达·施林于1862年12月22日出生。[9]
\subsubsection{新教家庭与意大利的死亡}
黎曼在1866年汉诺威和普鲁士军队在哥廷根交战时逃离了哥廷根。[10] 他在第三次前往意大利的途中死于结核病,地点是塞拉斯卡(现为马焦雷湖上的一个小村庄,隶属于维尔巴尼亚),他被埋葬在比甘佐洛的墓地(维尔巴尼亚)。  
黎曼是一个虔诚的基督徒,出生于一位新教牧师的家庭,他将自己作为数学家的生活视为服务上帝的一种方式。在他的一生中,他始终坚持自己的基督教信仰,并认为这是他生活中最重要的部分。在他去世时,他正在与妻子一起诵念《主祷文》,并在他们尚未念完时去世。[11] 与此同时,在哥廷根,他的女管家丢弃了他办公室中的一些文件,包括许多未发表的作品。黎曼拒绝发表不完整的工作,因此一些深刻的见解可能已经丧失。[10]

黎曼位于比甘佐洛(意大利)的墓碑上刻有《罗马书》8:28的经文:[12]

在神的怀抱中安息
乔治·弗里德里希·伯恩哈德·黎曼
哥廷根大学教授
生于1826年9月17日,布雷塞伦茨
死于1866年7月20日,塞拉斯卡
对于那些爱神的人,一切都必定互相效力,为最好的结果
\subsection{黎曼几何学}
黎曼的已发表作品开启了将分析与几何相结合的研究领域。随后,这些领域成为了黎曼几何学、代数几何学和复流形理论的主要组成部分。黎曼曲面的理论由费利克斯·克莱因和特别是阿道夫·赫尔维茨进一步阐述。这一数学领域是拓扑学的基础部分,并且至今仍在数学物理学中以新颖的方式得到应用。

1853年,高斯要求他的学生黎曼准备一篇关于几何基础的Habilitationsschrift。在经过数月的研究后,黎曼发展了他的高维理论,并于1854年6月10日在哥廷根发表了题为《几何学基础假设探讨》(Ueber die Hypothesen, welche der Geometrie zu Grunde liegen)的讲座。直到1868年,在黎曼去世两年后,德德金才将其出版。早期的接受情况似乎较为缓慢,但现在它被公认为几何学中最重要的著作之一。

这篇著作奠定了黎曼几何学的基础。黎曼找到了将高维流形的微分几何扩展到n维的方法,而这一方法是高斯在他的《著名定理》中所证明的。基本对象被称为黎曼度量和黎曼曲率张量。对于二维的曲面情况,曲率可以在每个点上归结为一个数(标量),具有恒定正曲率或负曲率的曲面是非欧几何的模型。

黎曼度量是每个空间点上的一组数字(即张量),它允许在任何轨迹中测量速度,其积分给出了轨迹端点之间的距离。例如,黎曼发现,在四维空间中,每个点需要十个数字来描述流形上的距离和曲率,无论流形如何扭曲。