% 群的矩阵表示及实例
% 群表示|矩阵表示|群线性表示|忠实表示
\pentry{群表示\upref{GrpRep},群乘法表及重排定理\upref{groupt}}
\begin{issues}
\issueDraft
\end{issues}
在群表示一节中曾提到群元可与线性变换建立同态关系,由此可以给出群的线性表示,在选取合适的基后,可以将线性变换写成矩阵的形式。

\begin{example}{$C_n$群的表示}
对于$C_n$群而言,有表示:$D(e)=1$,$D(a^m)=e^{\frac{2m\pi}{n}}$,显然,其乘法关系与$C_n$群乘法关系相同。
\label{gprep_ex1}
\end{example}

\begin{example}{$C_n$群的二维表示}
对于$C_n$群而言,有表示:

$D(e)=\begin{pmatrix}
 1 & 0\\
 0 &1
\end{pmatrix}$,
$D(a^m)=\begin{pmatrix}
 \cos{\frac{2m\pi}{n}} & \sin{\frac{2m\pi}{n}}\\
 -\sin{\frac{2m\pi}{n}} &\cos{\frac{2m\pi}{n}}
\end{pmatrix}$

这实际上是在平面上转动$\frac{2m\pi}{n}$角所对应的旋转矩阵,这与$C_n$群的几何含义相符,通过几何含义给出表示矩阵也是我们常用的方法。
\end{example}

\begin{example}{}
一个特殊的表示是所有群元都对应与1或单位阵,这对于所有群都是成立的。
\label{gprep_ex3}
\end{example}

\begin{definition}{忠实表示}
若群元与线性变换之间的映射是单射则称该表示为忠实表示。
\end{definition}

%这label怎么和我之前用的编译器不太一样,没弄明白,先直接写例一例二了。
显然例1,例2为群的忠实表示,而例3不是。

\begin{definition}{自身表示}
若群$G$本身就是某线性空间的线性变换群,那么其自身的矩阵形式给出表示叫做自身表示。
\end{definition}
\begin{example}{$SO2$群的自身表示}
SO2群是平面转动群其矩阵形式为:$D(\alpha)=\begin{pmatrix}
 \cos{\alpha} & \sin{\alpha}\\
 -\sin{\alpha} & \cos{\alpha}
\end{pmatrix}$。
则这是其自身表示。
\end{example}

\begin{corollary}{}
一个群表示的共轭、取逆后转置、取逆后取复共轭均还是群的一个表示,称为共轭表示、逆步表示和逆步复共轭表示。
\end{corollary}

证明:

若$D_{g_\gamma}=D_{g_\alpha}D_{g_\beta}$,那么$\bar{D_{g_\gamma}}=D_{g_\alpha}D_{g_\beta}$