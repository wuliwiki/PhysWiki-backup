% Julia 开发笔记
% license Xiao
% type Note

\begin{issues}
\issueDraft
\end{issues}

本文是关于 Julia 解释器的原理: 它使用了哪些技术, 可以使得它作为一门动态语言能达到编译语言的性能。

\begin{itemize}
\item \href{https://docs.julialang.org/en/v1/devdocs/init/}{Julia 开发者文档}
\item C 语言这样的叫做 \textbf{Ahead of Time(AOT)} 编译。
\item Julia 完全使用 \textbf{Just-In-Time(JIT)}。 并不是把部分代码使用该优化,而是全部。Julia 默认不直接理解并执行高级语言(一个理解器见\href{https://juliadebug.github.io/JuliaInterpreter.jl/stable/}{这里})。 每个重载的函数第一次被调用的时候, julia 都会先给它生成专门的 llvm 机器码。 这个过程还有一个不那么正规的名字叫做 \textbf{Just Ahead-of-Time(JAOT)}。
\item \textbf{Multiple Dispatch (MD)}: 可以基于参数的类型生成不同的专门代码。
\item \textbf{Type Inference}: 自动推导变量类型。
\item 内建并行, 包括分布计算
\item 高效内存管理: 减少内存分配, 增加内存重复利用。
\item 基于 \textbf{LLVM}: Julia 编译器先把 Julia 语言变为 \textbf{Julia IR} (和 LLVM IR 相似,而且有若干层), 再变为 \enref{LLVM IR}{llvmIR}, 最后交给 LLVM 进行优化。
\item Julia IR 后期使用 \textbf{Single Static Assignment (SSA)} 形式, 把 julia 代码的控制流表示为 \textbf{directed acyclic graph (DAG)}。 Julia IR 中的变量类型都是明确的。
\item \verb`LLVM.jl` 包可以把 julia 代码直接生成 LLVM IR, 或者把 Julia IR 转为 LLVM IR。
\item 一些预备知识: 编译原理, 编译器设计, LLVM, Julia 背后原理。
\end{itemize}

\begin{figure}[ht]
\centering
\includegraphics[width=14.2cm]{./figures/7293eeb0d6cd9c43.png}
\caption{Julia JIT 流程} \label{fig_julia0_1}
\end{figure}

\subsection{把 Julia 代码编译成 LLVM IR 代码}
首先安装 \verb`using Pkg; Pkg.add("LLVM")`
\begin{lstlisting}[language=julia]
using LLVM
function add(x::Int, y::Int)::Int
    return x + y
end
llvm_ir = @code_llvm add(1, 2) # 生成 LLVM IR
\end{lstlisting}
生成的代码如下
\begin{lstlisting}[language=none]
;  @ REPL[3]:1 within `add'
define i64 @julia_add_652(i64 signext %0, i64 signext %1) {
top:
;  @ REPL[3]:2 within `add'
; ┌ @ int.jl:87 within `+'
   %2 = add i64 %1, %0
; └
  ret i64 %2
}
\end{lstlisting}
所以上面参数 \verb`(1,2)` 的作用是提供自变量的类型。
