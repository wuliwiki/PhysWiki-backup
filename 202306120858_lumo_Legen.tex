% 勒让德多项式
% 勒让德方程|正交归一性|罗德里格斯公式|勒让德多项式

%未完成: 预备知识

\textbf{勒让德方程}为
\begin{equation}\label{eq_Legen_1}
\dv{x} \qty[(1-x^2)\dv{x} P_l(x)] + l(l+1)P_l(x) = 0~,
\end{equation}

我们仅在区间 $x \in [-1,1]$ 上考虑 $l$ 为非负整数的情况。 方程的解 $P_l(x)$ 是关于 $x$ 的 $l$ 阶多项式
\begin{equation}\label{eq_Legen_2}
P_l(x) = \frac{1}{2^l}\sum_{s=0}^{[l/2]} \frac{(-1)^s (2l-2s)!}{s!(l-s)!(l-2s)!} x^{l-2s}~,
\end{equation}
其中 $[x]$ 是向下取整函数。 当 $x$ 是整数, $[x] = x$, 当 $x$ 是非整数, $[x]$ 是小于 $x$ 的最大整数。 % 链接未完成

这里列出前几个多项式(\autoref{fig_Legen_1} )
\begin{equation}\ali{
&P_0(x) = 1  && P_3(x) = \frac12 (5x^3 - 3x) \\
&P_1(x) = x  && P_4(x) = \frac18 (35x^4 - 30x^2 + 3) \\
&P_2(x) = \frac12 (3x^2 - 1) \qquad && P_5(x) = \frac18 (63x^5 - 70x^3 + 15x)~.
}\end{equation}

\begin{figure}[ht]
\centering
\includegraphics[width=12cm]{./figures/bb55390b41292b72.pdf}
\caption{勒让德多项式} \label{fig_Legen_1}
\end{figure}

勒让德多项式也可以用\textbf{罗德里格斯}公式表示
\begin{equation}
P_l(x) = \frac{1}{2^l l!} \dv[l]{x} (x^2 - 1)^l~.
\end{equation}
由于求导会改变函数的奇偶性, % 链接未完成
由上式可以证明当 $l$ 为偶(奇)数时 $P_l(x)$ 是偶(奇)函数, 所以只有 $x$ 的偶(奇)次方项。

\subsection{正交归一性质}
勒让德多项式的归一化系数为
\begin{equation}\label{eq_Legen_4}
A_l = \sqrt{\frac{2l + 1}{2}}~,
\end{equation}
满足正交归一化条件
\begin{equation}
\int_{-1}^1  A_{l'} P_{l'}(x) \cdot A_l P_l(x) \dd{x} = \delta_{l,l'}~.
\end{equation}

\subsection{其他性质}
% 未完成: 说好的两个线性无关解呢? 怎么只有一个?

$P_l(x)$ 在 $[-1,1]$ 有 $l$ 个根。 $P'_l(x)$ 有 $l-1$ 个根。

\begin{equation}
P_l(1) = 1, \qquad P_l(-1) = (-1)^l~.
\end{equation}
由\autoref{eq_Legen_2} 得
\begin{equation}\label{eq_Legen_5}
% 已验证 l = 2~18
P_l(0) = \leftgroup{
&0 &\qquad (l = \text{奇数})\\
&\frac{(-1)^{l/2} l!}{2^l [(l/2)!]^2} &\qquad (l = \text{偶数})
}~.\end{equation}

傅里叶变换\upref{FTExp}($j_l$ 是球贝塞尔函数\upref{SphBsl})
\begin{equation}\label{eq_Legen_3}
\int_{-1}^1 P_l(x) \E^{\I k x} \dd{x} = 2 \I^l j_l(k)~.
\end{equation}

求导
\begin{equation}\label{eq_Legen_6} % 已数值验证
P'_l(x) = \leftgroup{
    &\frac{l+1}{1-x^2} [xP_l(x) - P_{l+1}(x)] &\,\, &(\abs{x} < 1) \\
    &l(l+1)/2 & &(x = 1) \\
    &(-1)^{l+1}l(l+1)/2 & &(x = -1)~.
}\end{equation}

\subsection{勒让德方程的级数解}
令
\begin{equation}
P_l(x) = \sum_{n = 0}^\infty c_n x^n
\end{equation}
代入方程, 对比系数得到递推公式
\begin{equation}
c_{n+2} = \frac{n(n+1)-l(l+1)}{(n+2)(n+1)}c_n = \frac{(n-l)(n+1+l)}{(n+2)(n+1)}c_n~.
\end{equation}
可见偶数项系数可用 $c_0$ 表示, 奇数项系数可用 $c_1$ 表示。 所以 $c_0$ 和 $c_1$ 可以看做是二阶微分方程的两个任意常数。

当 $l$ 为整数时, 可以证明 $n > l$ 以上的系数都为 0, 令最高次项系数为
\begin{equation}
c_l = \frac{(2l)!}{2^l (l!)^2}~,
\end{equation}
就得到了\autoref{eq_Legen_2}。

\subsection{生成函数}
勒让德多项式可以表示为以下函数对 $r$ 的泰勒展开的系数
\begin{equation}
\frac{1}{\sqrt{1 + r^2 - 2rx}} = \sum_{l = 0}^\infty P_l(x) r^l~.
\end{equation}
其中 $1/\sqrt {1+ r^2 - 2rx}$ 叫做勒让德多项式的生成函数或母函数