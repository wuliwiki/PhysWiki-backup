% 一维单原子链晶格
% keys 声子|一维晶格|振动

为了研究复杂的三维晶格的性质,我们可以先从较为简单的结构入手,例如研究一维单原子链的情形,并且假定\textbf{仅相邻原子之间存在相互作用}.设一排 N 个相同的原子组成的简单晶体,原子质量为 $m$,相邻两个原子的相互作用能是它们之间距离的函数:$V(r)$.当晶体\textbf{处于平衡位置}时,相邻两原子间距离为 $a$,那么原子略微偏离平衡位置可以产生小振动\upref{Oscil},根据理论力学的相关知识,一维单原子链作为 $N$ 个自由度的力学体系,共有 $N$ 个独立简谐振动模式,对这些简谐振动模式的研究可以帮助我们认识一维晶格振动的情形.

为了衡量相邻原子间弹性恢复力的程度,我们把相互作用能 $V(r)$ 在 $a$ 附近进行傅里叶展开:
\begin{equation}
V(a+\delta)=V(a)+\frac{1}{2}\beta \delta^2+\text{高阶项}
\end{equation}
$\beta$ 被称为\textbf{力常数}.让我们\textbf{暂时地忽略高阶项},那么相邻两原子间的作用力大小为
\begin{equation}
F=-\frac{\dd V}{\dd \delta}=-\beta\delta
\end{equation}

\subsection{格波解与色散关系}
我们先忽略原子链的边界情况,即假设它是无限长的.设第 $n$ 个原子偏离平衡位置的位移为 $\mu_n$,我们可以根据牛顿定理列出方程
\begin{equation}\label{onatom_eq1}
m \ddot \mu_n = \beta(\mu_{n+1}-\mu_n)-\beta(\mu_n - \mu_{n-1})
\end{equation}
为了找到相互独立的简振模,我们假定方程具有“格波”形式的特殊解
\begin{equation}
\mu_{n}(q)=Ae^{i(\omega t-naq)}
\end{equation}
$q$ 为波数.代入\autoref{onatom_eq1} 可以求得
\begin{equation}
\omega^2=\frac{4\beta}{m}\sin^2 \qty(\frac{1}{2}aq)
\end{equation}
这表明对于任意波数 $q$ 可以解出简振模的频率 $\omega$,而 $\omega$ 与 $q$ 的函数关系被称为\textbf{色散关系}.

相邻两个原子间距离为 $a$,这表明 $q$ 和 $q+2\pi/a$ 对应同一种格波.为了方便起见,我们令 $-\pi/a<q<\pi/a$.这个取值范围是这个一维简单晶格的\textbf{布里渊区}.

前面我们讨论的是无穷长的晶格,现在让我们把目光转向 $N$ 个原子的情形.