% 欧拉方程
% keys 欧拉方程

\begin{issues}
\issueDraft
\issueTODO
\end{issues}

具有以下形式的方程称为\textbf{欧拉方程}.
\begin{equation}\label{Eulequ_eq1}
\sum_{i=0}^{n}a_ix^{n-i}y^{(n-i)}=0
\end{equation}
其中 $a_0=1,a_1,\cdots,a_n$为常数.

欧拉方程可化为具有如下形式的常系数线性方程
\begin{equation}\label{Eulequ_eq4}
\varphi(D_t)y=0
\end{equation}
其中
\begin{equation}
\begin{aligned}
&x=e^t\\
&D_t=\dv{}{t}\\
&\varphi(D_t)=\sum_{i=0}^{n-1}a_iD_t(D_t-1)\cdots(D_t-n+i+1)+a_n=0
\end{aligned}
\end{equation}

欧拉方程\autoref{Eulequ_eq1} 具有如下形式的解
\begin{equation}
y=\sum_{i=1}^mx^{r_i}P_{k_i-1}(\ln x)
\end{equation}
其中,$r_s$是下面方程的根
\begin{equation}\label{Eulequ_eq2}
\sum_{i=0}^{n-1}a_ir(r-1)\cdots(r-n+i+1)+a_n=0
\end{equation}
$k_s$ 是根 $r_s$ 的重数,$m$是不同根的个数, $P_{k_s-1}(\ln x)$ 是具有任意系数的 $k_s-1$ 次多项式. 

特别的,当\autoref{Eulequ_eq2} 的所有根均为单根时,欧拉方程\autoref{Eulequ_eq1} 的解为
\begin{equation}
y=\sum_{i=1}^{n}C_ix^{r_i}
\end{equation}

\subsection{证明}
令
\begin{equation}
x=e^t,\quad D_t=\dv{}{t},\quad D_x=\dv{}{x}
\end{equation}
则
\begin{equation}
D_ty=\dv{y}{t}=\dv{y}{x}\dv{x}{t}=e^tD_xy
\end{equation}
即
\begin{equation}
D_xy=e^{-t}D_ty
\end{equation}
上式表明,对自变量 $x$ 的函数 $y$ 施于运算 $D_x$,相当于施于运算 $e^{-t}D_t$.于是
\begin{equation}
D_x^2y=e^{-t}D_t(e^{-t}D_t)y
\end{equation}
由记号因子的性质\autoref{Sign_eq1}~\upref{Sign} 和\autoref{Sign_eq11}~\upref{Sign}
\begin{equation}
D_x^2y=e^{-2t}(D_t-1)D_ty=e^{-2t}D_t(D_t-1)y
\end{equation}
用数学归纳法,容易证明下式成立
\begin{equation}
D_x^sy=e^{-st}D_t(D_t-1)\cdots (D_t-s+1)y
\end{equation}
上式可写成
\begin{equation}\label{Eulequ_eq3}
x^sD_x^sy=D_t(D_t-1)\cdots (D_t-s+1)y
\end{equation}
\autoref{Eulequ_eq3} 代入\autoref{Eulequ_eq1} ,便得\autoref{Eulequ_eq4} .

对于算符\upref{map} $\varphi(D_t)$ ,若存在函数 $f(t)$ ,使得
$\varphi(D_t)f(t)=\lambda f(t)$($\lambda$为常数),则称 $f(t)$为其具有特征值 $\lambda$ 的特征函数.显然,$e^{rt}$($r$为常数)为$\varphi(D_t)$ 的特征函数,对应特征值为 $\varphi(r)$.对方程\autoref{Eulequ_eq4} ,这等价于找 $\varphi(D_t)$的具有特征值 $0$的特征函数,即
\begin{equation}\label{Eulequ_eq5}
\varphi(r)=0
\end{equation}
方程\autoref{Eulequ_eq5} 叫作常系数线性方程\autoref{Eulequ_eq4} 的\textbf{特征方程}.

由线性微分方程
\addTODO{证明结论}