% 百科标点规范

\begin{issues}
\issueDraft
\end{issues}
\begin{itemize}
\item 
\end{itemize}
\begin{equation}
f(x) = \leftgroup{
    &\sqrt{x} &\quad &(x \ge 0)\\
    &\log(-x) & &(x < 0)~.
}
\end{equation}






\begin{itemize}
\item \textbf{行间两行或多行公式要加逗号和句号,例如:}
\end{itemize}
由于 $\mathcal{A}u\in U, \mathcal{A} w\in W,\forall  u\in U, w\in W$,则
\begin{equation}
\begin{aligned}
\mathcal{A}\hat e_j&=\sum_{i=1}^m a_{ij}\hat e_i\quad (j=1,\cdots ,m)~,\\
\mathcal{A}\hat e_j&=\sum_{i=m+1}^n a_{ij}\hat e_i\quad (j=m+1,\cdots ,n)~.
\end{aligned}
\end{equation}
\begin{itemize}
\item \textbf{或者}
\end{itemize}


\begin{equation}
\begin{aligned}
F_1(x_1,\cdots,x_n;y_1\cdots,y_m)&=0~,\\
&\ \vdots\\
F_m(x_1,\cdots,x_n;y_1\cdots,y_m)&=0~.
\end{aligned}
\end{equation}

\begin{itemize}
\item \textbf{带有条件的公式写成如下样式:\\例如:}
\end{itemize}
一般的常微分方程组都可写为下面的形式
\begin{equation}
\dv{y_i}{x}=f_i(x,y_1,\cdots,y_n) \qquad (i=1,\cdots,n)~,
\end{equation}
\begin{aligned}
注意:(i=1,\cdots,n)后面的的标点符号取决于是否在句间和句子的逻辑
\end{aligned}
\begin{itemize}
\item \textbf{注意阶乘(!)不是句子的标点,任然需要在后面加上标点符号,例如:}
\end{itemize}
\begin{equation}
\frac{N!}{n_0! n_1!\dots} \approx N!~.
\end{equation}

\begin{itemize}
\item \textbf{大括号后任然需要句号,例如:}
\end{itemize}

$$
\left(
    \begin{matrix}
    1&2&1&0\\
    3&4&0&1
    \end{matrix}\right)=\left(
    \begin{matrix}
    1&2&1&0\\
    0&-2&-3&1
    \end{matrix}\right)~$$
    $$
    =\left(
    \begin{matrix}
    1&0&-2&1\\
    0&-2&-3&1
    \end{matrix}\right)=\left(
    \begin{matrix}
    1&0&-2&1\\
    0&1&3/2&-1/2
    \end{matrix}
\right)~.
$$
\begin{itemize}
\item \textbf{常见的句间逻辑和标点的使用}
\end{itemize}

\begin{enumerate}
\item xxx和xxx:前一个xxx后不用标点符号,例如:

\begin{equation}
i=1~
\end{equation}
和
\begin{equation}
m=2~.
\end{equation}
\item xxx以及xxx:前一个xxx后不用标点符号,例如:
\begin{equation}
x=1~
\end{equation}
以及
\begin{equation}
y=2~.
\end{equation}

 \item xxx是xxxxx:xxx后不需要标点符号,例如:
\begin{equation}
x=2~
\end{equation}
是方程x+1=3的解。
\item "其中"或者"即"之前一般要加逗号,例如:
\begin{itemize}
\item 则这两个$k$-向量的内积定义为
\begin{equation}
\langle\omega_\alpha, \omega_\beta\rangle = \det \pmat{\langle \alpha_i, \beta_j \rangle_{i, j=1}^k}~,
\end{equation}
即用各$\langle \alpha_i, \beta_j \rangle$构成的方阵的行列式。
\end{itemize}
\begin{itemize}
\item 
我们可以把有$n$个自由度的简谐运动方程写为
\begin{equation}
\ddot{\bvec{x}} = \bvec{K}\bvec{x}~,
\end{equation}
其中$\bvec{x}=\pmat{x_1, x_2, \cdots, x_n}\Tr$为列矩阵,$\bvec{K}$为一个$n$阶方阵。

\end{itemize}
\item 若xxx,那xxxxx:若xxx后要加逗号
\item xxx便xxxxx:xxx后不需要标点符号,例如:
\begin{equation}
x=2~
\end{equation}
便构成了方程x+1=3的解。

\end{enumerate}
\begin{itemize}
\item \textbf{超出屏幕的公式要改成两行或几行}
\end{itemize}




