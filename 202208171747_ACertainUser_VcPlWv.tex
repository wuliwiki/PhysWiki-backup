% 真空中的平面电磁波
% 麦克斯韦方程组|电磁波|平面波|波动方程

\pentry{电场波动方程\upref{EWEq}}
%感觉可以合并掉?

\begin{figure}[ht]
\centering
\includegraphics[width=13cm]{./figures/VcPlWv_1.pdf}
\caption{平面电磁波的电磁场分布. 注意该例子中,电场矢量与 $x, y$ 坐标无关, 并占据整个空间(图片来自维基百科).\href{https://www.geogebra.org/m/xhYwXSsH}{一个可动的模型}(站外链接)} \label{VcPlWv_fig1}
\end{figure}

\footnote{参考 \cite{GriffE} 相关章节与周磊教授的讲义,http://fdjpkc.fudan.edu.cn/d200927/2009/0314/c8569a14801/page.htm}平面电磁波为
\begin{align}
&\bvec E(\bvec r, t) = \bvec E_0 \cos(\bvec k\vdot \bvec r - \omega t)\\
&\bvec B(\bvec r, t) = \bvec B_0 \cos(\bvec k \vdot \bvec r - \omega t)
\end{align}

其中,$\bvec E, \bvec B$均为矢量场\upref{Vfield}.$\bvec E$的各分量即为
$\bvec E = 
\begin{bmatrix}
E_{x0} \cos(\bvec k\vdot \bvec r - \omega t)\\
E_{y0} \cos(\bvec k\vdot \bvec r - \omega t)\\
E_{z0} \cos(\bvec k\vdot \bvec r - \omega t)\\
\end{bmatrix}
$

通解是这些平面波的任意线性组合. 

\subsection{电磁波基本性质}
%需要check一下结论...
\subsubsection{波速}
波速等于真空中的光速 $c$, 且
\begin{equation}\label{VcPlWv_eq1}
c = \frac{1}{\sqrt{\epsilon_0\mu_0}} = 299,792,458 \Si{m/s}
\end{equation}

\subsubsection{波的性质}
\begin{itemize}
\item $\bvec k$被称为波矢,标志电磁波的传播方向.
\item  “色散关系”:$k^2=\frac{\omega^2}{c^2}$.
\item $k=\frac{2\pi}{\lambda}$, $\omega=\frac{2\pi}{T}, v=\frac{\lambda}{T}$
\item 电磁波是横波,即在电磁场传播方向$\bvec k$上没有电场、磁场的分量.$\bvec k \cdot \bvec E = 0$ (如果 $\bvec E_0$ 中存在平行于 $\bvec k$ 的分量, 那么 $\div \bvec E \ne 0$, 所以二者必须垂直)
\end{itemize}

\subsubsection{电场与磁场}
电磁波中电场与磁场二者不是相互独立的.在一点处的电场与磁场满足:
\begin{equation}
\bvec k \times \bvec E = \omega \bvec B
\end{equation}
由此,可推导出以下结论:
\begin{itemize}
\item $\bvec E, \bvec B, \bvec k$ 互相垂直,且成右手螺旋关系.见\autoref{VcPlWv_fig1} 右侧
\item $\bvec E \cdot \bvec B = 0,\bvec E \times \bvec B \parallel \bvec k$,$\bvec v \times \bvec E \parallel \bvec B, ...$.
\item $\abs{E_0}=c\abs{B_0}$
\end{itemize}

部分推导可见 “电场波动方程\upref{EWEq}”

\subsubsection{能量密度}
\pentry{电场的能量\upref{EEng}, 磁场的能量\upref{BEng}}
任意一点的能量密度为
\begin{equation}\label{VcPlWv_eq2}
\rho_E = \frac{1}{2}\qty(\epsilon_0 E^2 + \frac{B^2}{\mu_0}) = \epsilon_0 E^2
\end{equation}
虽然磁场在“数值上”小于电场;但能量上,电场和磁场的能量相同,各贡献总能量一半. 

平均能流密度(光强)为
\begin{equation}
I = \frac12 c\epsilon_0 E_0^2
\end{equation}
推导见\autoref{EBS_ex1}~\upref{EBS}, 也可以认为瞬时能流密度等于能量密度乘以波速 $c$, 对于简谐波,需要除以二得平均值.


