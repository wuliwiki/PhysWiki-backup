% 图灵测试(综述)
% license CCBYSA3
% type Wiki

本文根据 CC-BY-SA 协议转载翻译自维基百科\href{https://en.wikipedia.org/wiki/Turing_test}{相关文章}。

\begin{figure}[ht]
\centering
\includegraphics[width=8cm]{./figures/3f5360e4685a515c.png}
\caption{图灵测试的“标准解释”中,C玩家(询问者)被赋予任务,试图判断哪一位玩家——A还是B——是计算机,哪一位是人类。询问者仅限于通过书面问题的回答来做出判断。[1]} \label{fig_TLCS_1}
\end{figure}
图灵测试,最初由艾伦·图灵于1949年提出,称为“模仿游戏”,是对机器是否能够表现出等同于人类的智能行为的测试,或者说是无法与人类行为区分的测试。图灵提出,测试中一位人类评估者将判断人类与机器之间的自然语言对话,机器被设计成产生类似人类的回应。评估者知道对话中的一方是机器,所有参与者都将被隔开。对话仅限于文字交流,例如使用计算机键盘和屏幕,因此测试结果不依赖于机器将文字转化为语音的能力。如果评估者无法可靠地区分机器与人类,那么机器就被认为通过了测试。测试的结果不依赖于机器是否能给出正确答案,而是看它的答案与人类回答的相似程度。由于图灵测试是对性能能力无法区分性的测试,因此其语言版本自然地推广到了所有人类的表现能力,包括语言和非语言(机器人)的表现能力。

该测试由图灵在1950年发表的论文《计算机与智能》中提出,当时他在曼彻斯特大学工作。论文开头写道:“我提议考虑这个问题,‘机器能思考吗?’”由于“思考”这一概念很难定义,图灵选择用“用另一种更相关且表达相对明确的话语替代这个问题”来描述问题。[6] 图灵以“三人游戏”的形式来描述这一问题,这个游戏称为“模仿游戏”,在这个游戏中,一位询问者通过向一位男士和一位女士提问,试图判断两位参与者的性别。图灵的新问题是:“是否存在可以在模仿游戏中表现得很好的数字计算机?”[2] 图灵认为这个问题是可以回答的。在论文的其余部分,他反驳了关于“机器能思考”这一命题的所有主要反对意见。[7]

自从图灵提出他的测试以来,它既具有深远的影响,也受到了广泛的批评,并成为人工智能哲学中的一个重要概念。[8][9] 哲学家约翰·塞尔在他的“中文房间”论证中评论了图灵测试,这一思想实验认为,无论程序如何使计算机表现得像人类,机器都无法拥有“思维”、“理解”或“意识”。塞尔批评图灵的测试,并声称它不足以检测意识的存在。