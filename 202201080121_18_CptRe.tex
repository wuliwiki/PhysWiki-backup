% 有限覆盖与紧性

\pentry{实数集的拓扑\upref{ReTop} 序列的极限\upref{SeqLim}}

\subsection{紧致性的概念}
紧致性 (compactness) 是一个重要的拓扑概念. 在分析学中, 它首次出现于对定义在实数集子集上函数的研究中. 我们试着举一些例子来说明紧致性是怎样一个概念.

从逻辑上说, 我们还没有引入连续函数这一概念, 但这并不妨碍我们从直观上去理解它. 直观上讲, 对于点集$E\subset\mathbb{R}$上的函数$f:E\to\mathbb{R}$, 如果当$x\in E$越来越接近$x_0$时, 函数值$f(x)$也会越来越接近$f$在$x_0$处的值$f(x_0)$, 那么就可以认为它在点$x_0\in E$处是"连续"的. 说$f$在$E$上连续, 也就是它在$E$的每一点处都连续. 

显然连续性是一个局部性质: 函数在一点处是否连续, 只跟它在这一点的某个邻域里的行为有关. 对于一般的集合$E$, 从其上函数的局部性质是无法推出整体性质的. 例如, 在开区间$(0,1)$上,

\begin{enumerate}
\item 函数$f_1(x)=1/x$是连续的, 但它在$x\to0$时无界; 
\item 函数$f_2(x)=x^2$连续且有界, 但却达不到它的最大和最小值, 例如当$x\to0$时$f_2(x)\to0$, 但它却取不到$0$值; 
\item 函数$f_3(x)=\sin(1/x)$连续且有界, 但在$x\to0$时震荡得越来越厉害, 根本没有极限. 
\end{enumerate}

归根结蒂, 这些"不好"的整体性质, 都来自于定义域$(0,1)$的某种"不好的性质": 由于$0$本身不属于开区间$(0,1)$, 所以没办法用$0$的邻域去覆盖到接近$0$的那些点. 当$x\in(0,1)$越来越接近端点时, 就只好用$x$的越来越小的邻域去作为看待局部性质的标尺了. 如果在开区间$(0,1)$内部来看, 当$x\to0$时, 它实际上是不会接近任何一点的 (它的极限跑出了定义域的范围). 换句话说, 我们没法找到一个一致 (uniform) 的标尺去衡量定义域$(0,1)$上的局部性质. 这时就说它缺乏紧致性. 

在上面这个例子中, 给开区间$(0,1)$补上端点就足够解决很多问题, 例如保证连续函数都有界, 都能达到最大和最小值, 而且不会"过分地震荡", 也就是说, 函数在不同的点处连续的"程度"都一样. 但在更贴近实际应用的复杂场景中, 紧致性的缺失可能会造成一些意料之外的后果. 有一个简单的例子可以说明这一点 (它属于魏尔斯特拉斯): 给定平面上不共线的三点, 则过这三点的曲线长度的最小值是由折线段达到的; 如果一定要求曲线不能有不光滑的角点, 那么寻找长度最小值的问题就无解. 当然, 可以造出光滑的曲线段使之逐渐逼近有角点的折线, 但这样得到的"极限构型"却跑出了光滑曲线的类. 由此所生发出的弱紧性 (weak compactness) 概念在现代分析学和微分方程理论中是非常基本的.

\subsection{定义与例子}
设$K\subset\mathbb{R}$是实数集的子集.
\begin{definition}{开覆盖}
集合$K\subset\mathbb{R}$的一个开覆盖 (open cover) 是指一族开集$\{U_{\alpha}\}_{\alpha\in A}$, 使得这些开集的并集包含$K$.
\end{definition}

\begin{example}{开覆盖的例子}
开区间的族$(k,k+2),\,k\in\mathbb{Z}$组成$\mathbb{R}$的开覆盖. 它是一个可数的覆盖.

以某种方式给有理数集的全体进行编号. 开区间的族$(r_k-2^{-k},r_k+2^{-k}),\,k\in\mathbb{N}$组成有理数集的开覆盖. 它是一个可数的覆盖.

开区间$(-1,0.5),(0.4,2)$组成了闭区间$[0,1]$的开覆盖. 这是一个有限的覆盖.
\end{example}

\begin{definition}{紧集}
集合$K\subset\mathbb{R}$称作是紧致的 (compact), 如果它的任何开覆盖$\{U_{\alpha}\}_{\alpha\in A}$中都存在有限多个开集$U_{\alpha_1},...,U_{\alpha_N}$, 使得这有限个开集仍旧组成$K$的开覆盖. 
\end{definition}

\begin{example}{例子}
实数集的有限子集当然是紧集.

开区间$(0,1)$不是紧集. 例如, 开集族$(1/(n+2),1/n),\,n\in\mathbb{N}$组成了开区间$(0,1)$的开覆盖. 但如果仅限于从中选出$N$个开区间, 那么它们的并集显然不能覆盖到特别接近$0$的点. 所以这个开集族没有有限的子覆盖. 类似地, 实数集中的开集都不是紧集.

但任何闭区间$[a,b]$都是紧集. 设$\{U_\alpha\}_{\alpha\in A}$是闭区间$I=[a,b]$的开覆盖. 如果它没有有限子覆盖, 那么将区间$I$从中二分之后, 至少有一半不能被有限个开集覆盖. 将这一半记为$I_1$, 继续二分, 则至少又有一半不能被有限个开集覆盖. 将这一半记为$I_2$. 如此续行即得到一个长度减半的闭区间套$I\supset I_1\supset I_2\supset...$. 如果写$I_k=[a_k,b_k]$, 那么$a_k$是单调递增的序列, $b_k$是单调递减的序列. 按照实数集的完备公理, 有一个实数$c$介于集合$\{a_k\}$和$\{b_k\}$之间: $a_k\leq c\leq b_k$对于一切$k$都成立. 于是$c\in[a,b]$, 从而属于某个开集$U_\alpha$. 闭区间套$I\supset I_1\supset I_2\supset...$显然缩至$c$, 而这意味着有某个$I_k$能够被$U_\alpha$这单个开集覆盖, 这与当初$I_k$的选取方式相违背.
\end{example}

\begin{exercise}{开区间与闭区间的区别}
如果把论证中的闭区间$[a,b]$改为开区间$(a,b)$, 那么问题会出在哪里? 提示: 按照上述方式构造的区间套可能会缩至端点.
\end{exercise}

\begin{exercise}{可数覆盖性质}
设$E\subset\mathbb{R}$. 如果$\{U_\alpha\}_{\alpha\in A}$是$E$的开覆盖, 那么总可以从中选出至多可数无穷多个开集, 使之仍旧组成$E$的开覆盖. 提示: $E$包含可数wu
\end{exercise}

简单地说, 紧集就是具有"有限覆盖性质"的点集. 乍一看这不好理解, 但实际上, 如果把开集理解为某点的邻域, 那么集合$K$总能被有限多个开集覆盖即说明: 可以找到一个一致的标尺把$K$的所有局部性质都统一起来. 关于这一点的详细解释, 可以参考后续词条连续函数的性质\upref{conff}, 这里进行一点简单的说明: 如果函数$f:K\to\mathbb{R}$是连续的, 那么给定一个误差$\varepsilon>0$之后, 对于任何$x\in K$都可找到它的邻域$U_{x}$, 使得当$y\in K\cap U_x$时函数值$f(y)$与$f(x)$的误差小于$\varepsilon$. 邻域$\{U_x\}_{x\in K}$显然组成$K$的开覆盖, 而如果$K$是紧集, 那么从中便可选出有限个开集, 使得函数$f$在这些开集上的振幅都可用$\varepsilon$控制. 这就显示出集合$K$本身的某种整体性质了.

\subsection{紧集的性质}
紧集有一系列很好的拓扑性质. 这其中的第一个正是上一小节例子中闭区间套的抽象版本: 闭集套性质 (property of nested closed sets):

\begin{theorem}{闭集套}
集合$K\subset\mathbb{R}$是紧致的, 当且仅当下述命题成立: 如果$\{C_\alpha\}_{\alpha\in A}$是一族在$K$中闭的子集, 而且其中任意有限个的交集都非空, 那么交集$\cap_{\alpha\in A}C_\alpha$本身非空. 
\end{theorem}

证明是直接的: 交集$\cap_{\alpha\in A}C_\alpha$为空等价于集族$K\setminus C_\alpha$组成$K$的开覆盖, 所以上述定理其实就是紧致性定义的逆否命题.

\begin{theorem}{收敛子序列}
集合$K\subset\mathbb{R}$是紧致的, 当且仅当其中的每个序列都有收敛的子序列, 其极限仍旧落在$K$中.
\end{theorem}

\textbf{证明.} 设$K$是紧集. 设$\{x_k\}\subset K$是一个序列; 不妨假定$K$是无穷集, 而这个序列本身也是没有重复的无穷集. 如果$x\in K$不是它的子列极限, 那么总有一个邻域$U_x$使得这序列只有有限多元素落在里面. 假定每个$x\in K$都不是它的子列极限, 那么这样构造的$U_x$显然组成$K$的开覆盖. 它的有限子覆盖只能包含$\{x_k\}$中的有限多个元素, 与最初它是无穷集的预设违背.

反过来, 设$K$中的每个序列都有收敛子序列. 设$\{U_\alpha\}_{\alpha\in A}$是$K$的开覆盖, 但却没有有限的子覆盖. 可以选出可数个开集

这个证明的前半部分单独抽取出来, 就是波尔查诺-魏尔斯特拉斯定理 (Bolzano-Weierstrass theorem):
\begin{theorem}{波尔查诺-魏尔斯特拉斯定理}
实数集中有界的无穷序列必然包含收敛的子序列. 实际上, 有界的序列包含于某闭区间中, 而应用闭区间的紧致性即可得到结论.
\end{theorem}

作为重要的推论, 有:
\begin{theorem}{}
集合$K\subset\mathbb{R}$是紧致的, 当且仅当它是有界的闭集.
\end{theorem}