% 单位制和量纲
% keys 单位制|量纲|基本量类|导出量类

\begin{issues}
\issueTODO
\end{issues}

\pentry{现象类\upref{PHEC}}

\subsection{为什么要引入单位制?}
为回答这一问题,先来看一个例子.欧姆定律的数值表达式为
\begin{equation}\label{USD_eq1}
U=IR
\end{equation}
其中 $U,I,R$ 分别是以 \textbf{$\boldsymbol{V}$ }、\textbf{ $\boldsymbol{A}$ }和\textbf{ $\boldsymbol{\Omega}$ }为单位测量问题中的电压 $\boldsymbol{U},\boldsymbol{I},\boldsymbol{R}$ 所得的数.为明确起见,把\autoref{USD_eq1} 写为
\begin{equation}
U_{V}=I_{A}R_{\Omega}
\end{equation}
若以 \textbf{$\boldsymbol{mA}$} 测量电流,并把所得的数记作 $I_{mA}$ ,由\autoref{QCU_eq8}~\upref{QCU} 
\begin{equation}\label{USD_eq2}
I_{A}=\frac{\boldsymbol{mA}}{\boldsymbol{A}}I_{mA}=10^{-3}I_{mA}
\end{equation}
\autoref{USD_eq1} 代入 \autoref{USD_eq2} ,便得
\begin{equation}
U_{V}=10^{-3}I_{mA}R_{\Omega}
\end{equation}
为了简洁起见,通常都去掉下标,于是就有
\begin{equation}\label{USD_eq3}
U=10^{-3}IR
\end{equation}
\autoref{USD_eq1} 和\autoref{USD_eq3} 都称为欧姆定律,两者不同的原因在于采用不同的单位搭配.

通过上面的例子,不难想象,同一规律的各个数值表达式之间的差别仅体现在一个附加因子.因此,只需把式\autoref{USD_eq1} 改写为
\begin{equation}
U=kIR
\end{equation}
便能在任何单位搭配下成立.上式中 $k$ 依赖于式中各量所选的单位.

每一量类中的单位原则上可任选,但这会导致大量的数值表达式中的 $k$ 值复杂得难以记住.为克服这一困难,可用单位制来约束各个量类单位的选法.
\subsection{单位制}
一个\textbf{单位制}由以下3个要素构成:
\begin{enumerate}
\item 选定 $l$ 个量类 $\tilde{\boldsymbol{J}}_1,\cdots,\tilde{\boldsymbol{J}}_l$ 作为\textbf{基本量类}(个数和选法有相当任意性),其它量类一律称为\textbf{导出量类}.  
\item 对每一基本量类 $\tilde{\boldsymbol{J}}_i(i=1,\cdots,l)$ 任选一单位 $\hat{\boldsymbol{J}}_i$ ,称为\textbf{基本单位}.
\item 对每一导出量类 $\tilde{\boldsymbol{C}}$,利用一个适当的、涉及 $\tilde{\boldsymbol{C}}$ 的物理规律来定义它的单位,称为\textbf{导出单位}.
\end{enumerate}
\begin{example}{}
CGS单位制指定长度、质量和时间为基本量类,并选 $\boldsymbol{cm},\bol$基本单位
\end{example}