% 范德瓦尔斯气体
% keys 范式方程|范德瓦尔斯气体

范德瓦尔斯方程是理想气体向真实气体的推广,架起了微观图像与宏观测量之间的桥梁.

范德瓦尔斯对理想气体作了两点修正:1、真实气体占据一定体积;2、真实气体间有分子间作用势(Lennard-Jones 势是一个很好的近似).范德瓦尔斯方程展现出惊人的威力——从它的图像上可以看出气液相变线,可以找到临界点…… 1910年诺贝尔物理学奖授予范德瓦尔斯,以表彰他为气体和液体状态方程所作的工作.

\subsection{范式方程}

\begin{equation}
\left(p+\frac{a}{V_m^2}\right)(V_m-b)=RT
\end{equation}

$b$ 是因为真实气体分子总占据一定体积而做的修正.$a$ 是考虑分子间作用力(主要是吸引力)而做的修正.

设气体分子的有效直径为 $d$,分子原本能达到的空间体积为 $V$,当考虑它与另一分子的碰撞时,它所能达到的空间体积减少了 $\frac{4}{3}\pi d^3$.$1\mol$ 气体含有 $N_A$ 个气体分子,从一个粒子的角度看,它面对 $N_A-1$ 个排斥球,而每个排斥球只有一面可能对它产生排斥,体积只能算一半;

