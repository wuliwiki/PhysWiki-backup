% 李导数
% license Usr
% type Tutor
\pentry{流\nref{nod_flow}}{nod_fbf5}\pentry{前推\nref{nod_pfw}}{nod_a77f}
\subsection{对向量场作用}
区别于仿射联络,\textbf{李导数(Lie derivative)}$\mathcal{L}:\mathfrak{X}(M)\times\mathfrak{X}(M)\to\mathfrak{X}(M)$是对向量场微分的另一种思路。

\begin{definition}{}
设$V$为任意光滑向量场,令$\theta_t:M\rightarrow M$为该向量场的流,定义对$W$关于$V$的李导数为
\begin{equation}(\mathcal{L}_VW)_p=\left.\frac{d}{dt}\right|_{t=0}(\theta_{-t})_*W_{\theta_t(p)}=\lim_{t\to0}\frac{(\theta_{-t})_*W_{\theta_t(p)}-W_p}{t}~.\end{equation}
\end{definition}
由前推定义可知,$\theta_{t*}W_q\in T_{\theta_t(q)}M$,因而$(\theta_{-t})_*W_{\theta_t(p)}\in T_p M$,所以该定义是合理的,我们利用前推得到了同一切空间的两个向量并进行微分。

其次,我们可以证明$(\mathcal{L}_VW)_p=[V,W]_p$,因李括号将两个光滑切场映射为光滑切场,从而可知李导数亦然如此,即$p\mapsto(\mathcal{L}_VW)_p$是光滑切场。结合李括号和前推的性质,我们有:
\begin{theorem}{}
\begin{itemize}
\item $\mathcal{L}_VW=-\mathcal{L}_WV.$
\item 如果$F:M\rightarrow N$是微分同胚,则$F_*(\mathcal{L}_VW)=\mathcal{L}_{F_*V}F_*W.$
\end{itemize}
\end{theorem}
在这里,我们只证明最后一条,即$F_*[V,W]=[F_*V,F_*W]$。

由于$F$是微分同胚,那么对于任意$X\in \mathcal J(M)$,我们都可以在$N$上找到唯一的向量场与之关联。因此我们可以设$Y_i=F_*X_i\in \mathcal J(N),f\in C^{\infty}N$,那么我们有
\begin{equation}
\begin{aligned}
F_*[X_1,X_2]f&=[X_1,X_2]f\circ F\\
&=(X_1X_2-X_2X_1)f\circ F~,
\end{aligned}
\end{equation}
又因为
\begin{equation}
\begin{aligned}
X_1X
\end{aligned}
\end{equation}

\subsection{对张量场作用}
\subsection{对微分形式作用}
\subsection{李导数的基本性质}