% 南京航空航天大学 2002 量子真题
% license Usr
% type Note

\textbf{声明}:“该内容来源于网络公开资料,不保证真实性,如有侵权请联系管理员”

\subsection{一(本题15分)}
\begin{enumerate}
\item 解释:量子力学中的“简并”和“简并度”。
\item 证明:一维无奇性势的薛定谔方程的束缚态无简并。
\end{enumerate}

\subsection{(本题15分)}
氢原子处于状态:
$$\Psi(r, \theta, \phi) = \frac{1}{2} R_{21}(r) Y_{10}(\theta, \phi) - \frac{\sqrt{3}}{2} R_{21}(r) Y_{1-1}(\theta, \phi)~$$
问:氢原子的能量$E$、角动量平方$L^2$、角动量$Z$分量$L$,

这三个量中哪些量具有确定值?哪些量没有确定值?有确定值的求出它的确定值;没有确定值的求出它的可能值及其出现的几率,并求出其平均值。

\subsection{(本题16分)}
已知在角动量 $\hat{L}^2$ 和 $\hat{L}_z$ 的共同表象中,算符 $\hat{L}_x$ 的矩阵
为:$$L_x = \frac{\hbar}{\sqrt{2}} \begin{pmatrix}
0 & 1 & 0 \\
1 & 0 & 1 \\
0 & 1 & 0
\end{pmatrix}~$$

(1)求它的本征值和归一化本征函数。

(2)找出一个么正变换矩阵S,将算符L对角化。

\subsection{(本题18分)}
一维运动粒子处在状态
\psi(x) = 
\begin{cases}
2\lambda^{3/2} x e^{-\lambda x} & x \geq 0 \\
0 & x < 0
\end{cases}
\quad \text{其中} \ \lambda > 0. \ \text{求:}
\begin{enumerate}
    \item 粒子位置坐标的平均值。
    \item 粒子动量的几率分布函数。
    \item 粒子动量的平均值。
\end{enumerate}
已知:
$$\int_{0}^{\infty} x^n e^{-\alpha x} \, dx = \frac{n!}{\alpha^{n+1}}~$$