% 进程和子进程笔记(Linux/C++)
% license Usr
% type Note

如果不改变环境变量也可以用 \verb`execv()`,完整的例子如
\begin{lstlisting}[language=cpp]
#include <unistd.h>    // For execv()
#include <sys/wait.h>  // For wait()
#include <iostream>    // For std::cout
#include <cstdio>      // For perror()

int main() {
    // Define the command and its arguments
    char *args[] = {(char *)"/bin/ls",
        (char *)"-l", (char *)NULL};

    // Fork a new process
    pid_t pid = fork();

    if (pid == -1) {
        // Fork failed
        perror("fork failed");
        return 1;
    } else if (pid == 0) {
        std::cout << "running ls command" << std::endl;
        // Replace the child process with a new program (ls -l)
        execv("/bin/ls", args);
        // If execv() returns, an error occurred
        perror("execv failed");
        return 1;
    } else {
        // Parent process
        int status;
        // Wait for the child to complete. pid is now of the child process
        waitpid(pid, &status, 0);
        std::cout << "Child process finished with status: "
            << status << std::endl;
    }
}
\end{lstlisting}
该程若运行成功会把 \verb`ls` 的输出打到 stdout。

\begin{itemize}
\item C/C++ 中最万能的子进程函数是 \verb`int execve(const char *path, char *const argv[], char *const envp[]);`
\begin{lstlisting}[language=cpp]
char *args[] = {"/bin/ls", "-l", NULL};
char *env[] = {"HOME=/usr/home", "LOGNAME=home", NULL};
execve("/bin/ls", args, env);
\item \verb`getpid()` 可以获取当前进程的 pid
\item 子进程中用 \verb`getppid()` 获取父进程的 pid
\item 对子进程和父进程, \verb`pid_t pid = fork()` 分别返回 0 和子进程的 pid。
\end{lstlisting}
\end{itemize}
