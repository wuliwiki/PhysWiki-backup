% 一元隐函数的存在及可微定理
% 隐函数|存在|可微性|隐函数存在定理

\begin{issues}
$\cdot$ 相关地方需新建词条并引用
\end{issues}

\pentry{隐函数\upref{ImpFun}}
本节将建立保证隐函数单值连续及可微的条件.这由两个定理来保证
\begin{theorem}{存在定理}\label{ImFED_the1}
若:\begin{enumerate}
\item 函数 $F(x,y)$ 在以点 $(x_0,y_0)$ 为中心的长方形邻域\autoref{ImpFun_def2}~\upref{ImpFun}
\begin{equation}
\mathcal{D}=[x_0-\Delta,x_0+\Delta;y_0-\Delta',y_0+\Delta']
\end{equation}
中有定义且连续;
\item $F(x_0,y_0)=0$ ;
\item 当 $x$ 为常数时,函数 $F(x,y)$ 随着 $y$ 的增大而单调增大(或单调减小).
\end{enumerate}
那么,在点 $(x_0,y_0)$ 的某一邻域内:
\begin{enumerate}
\item 方程 $F(x,y)=0$ 确定 $y$ 为 $x$ 的单值函数: $y=f(x)$;\footnote{这个定理的结论1可能不对!可能无法唯一确定这个单值函数!}
\item $f(x_0)=y_0$ ;
\item $f(x)$ 连续.
\end{enumerate}
\end{theorem}

\begin{theorem}{可微定理}
若:
\begin{enumerate}
\item 函数 $F(x,y)$ 在以点 $(x_0,y_0)$ 为中心的长方形领域
\begin{equation}
\mathcal{D}=[x_0-\Delta,x_0+\Delta;y_0-\Delta',y_0+\Delta']
\end{equation}
中有定义且连续;
\item 在 $\mathcal{D}$ 中偏导数 $F_x',F_y'$ 存在且连续;
\item $F(x_0,y_0)=0$;
\item $F_y'(x_0,y_0)\neq0$.
\end{enumerate}
那么:
除\autoref{ImFED_the1} 的结论1,2,3外,还可证明:\\
$\quad$4.函数 $f(x)$ 有连续导数.

\end{theorem}
\subsection{证明}
\subsubsection{存在定理的证明}
\textbf{存在性}
\begin{figure}[ht]
\centering
\includegraphics[width=14cm]{./figures/ImFED_1.pdf}
\caption{隐函数存在定理示意图} \label{ImFED_fig1}
\end{figure}
如\autoref{ImFED_fig1} ,沿着 $x=x_0$ 的竖直线,函数 $F(x,y)$ 变成一个变元 $y$ 的函数 $F(x_0,y)$.根据定理条件2:$F(x_0,y_0)=0$.根据定理条件3
\begin{equation}\label{ImFED_eq1}
F(B_0)=F(x_0,y_0+\Delta')>0,\quad F(A_0)=F(x_0,y_0-\Delta')<0
\end{equation}
沿着通过 $A_0,B_0$ 的两条水平直线,得到两个 $x$ 的函数:$F(x,y+\Delta'),F(x,y-\Delta')$.由\autoref{ImFED_eq1} 已看到,第一个函数有正值,第二个有负值.按条件1:这两函数是连续的.因此必有点 $x_0$ 的某一邻域 $(x_0-\delta,x_0+\delta),\quad(0<\delta\leq\Delta)$ ,使得这两函数保持自己的符号(连续函数的保号性,需编辑词条并引用).于是当 $x\in(x_0-\delta,x_0+\delta)$ 时,
\begin{equation}
F(x,y_0+\Delta')>0,\quad F(x,y_0-\Delta')<0
\end{equation}
换而言之,在原矩形的上下底上,有以点 $A_0$ 及 $B_0$ 为中心而长为 $2\delta$ 的线段  $B_1B_2$ 及 $A_1A_2$ ,沿着这些线段,给定函数 $F(x,y)$ 在 $B_1B_2$ 上有 正值而在 $A_1A_2$ 上有负值.

在区间 $(x_0-\delta,x_0+\delta)$ 上任选一点 $\overline{x}$,其在矩形上下底的垂直对应点为 $\overline{B}$ 和 $\overline{A}$,则
\begin{equation}
F(\overline{A})=F(\overline{x},y_0-\Delta)<0,\quad F(\overline{B})=F(\overline{x},y_0-\Delta)>0
\end{equation}
由连续性(布尔查诺-柯西第一定理(or连续函数介值定理),编辑词条并引用),必有 $\overline{A}\overline{B}$ 间一点 $y=\overline{y}$,\footnote{错误的原因可能在于这里的$\overline{y}$并不是唯一的,条件3只说$x=x_0$时,函数$F(x,y)$关于$y$严格单增, 而在其他位置,不一定严格单增!!事实上可以构造一个反例.}使得
\begin{equation}\label{ImFED_eq2}
F(\overline{x},\overline{y})=0
\end{equation}
由条件3,满足\autoref{ImFED_eq2} 的 $\overline{y}$ 是唯一的.

这样,在点 $(x_0,y_0)$ 的邻域 $(x_0-\delta,x+\delta)$ 内,方程 $F(x,y)=0$ 确实确定 $y$ 为 $x$ 的单值函数\autoref{ImpFun_def1}~\upref{ImpFun}.

\textbf{连续性}

对 $y=f(x)$ 上的任一点 $(\overline{x},\overline{y})$ ,对于任意大于0的 $\epsilon$, 只需作该点的长方形邻域 $(\overline{x}-\delta',\overline{x}+\delta';\overline{y}-\epsilon,\overline{y}+\epsilon)$,使得区间 $(\overline{x}-\delta',\overline{x}+\delta')$ 内任一 $x$,有
\begin{equation}
\abs{f(x)-\overline{y}}=\abs{f(x)-f(\overline{x})}<\epsilon
\end{equation}
就保证了连续性,而上面存在性中的方法保证了这一长方形邻域的存在性.

于是\autoref{ImFED_the1} 得证.
\subsubsection{可微定理的证明}
设 $F'_y(x_0,y_0)>0$,根据定理的条件2, $F'_y(x,y)$ 是连续的,所以可做正方形
\begin{equation}
[x_0-\delta',x_0+\delta';y_0-\delta',y_0+\delta'],\quad(\delta'<\Delta \&\Delta') 
\end{equation}
使得对于一切属于它的点有 $F_y'(x,y)>0$(这意味着对这区域, $x$ 不变,$F(x,y)$ 单调增).于是对这正方形区域而言,\autoref{ImFED_the1} 的一切条件满足.因此,\autoref{ImFED_the1} 的结论1,2,3成立.

现在转而证明函数 $f(x)$ 有连续的导数.在由方程 $F(x,y)=0$ 确定的隐函数 $y=f(x)$ 上,$F(x,y)=0$.那么在这曲线上给 $x$ 于增量 $\Delta x$,成立 $y+\Delta y=f(x+\Delta x)$.它们共同满足 $F(x+\Delta,y+\Delta)=0$.显然,增量
\begin{equation}
\Delta F(x,y)=F(x+\Delta,y+\Delta)-F(x,y)=0
\end{equation}
而由有限增量公式(编辑词条并引用),
\begin{equation}
\Delta F(x,y)=F_x'\Delta+F_y'\Delta y+\alpha\Delta x+\beta\Delta y
\end{equation}
其中,$\alpha,\beta$ 依赖于 $\Delta x,\Delta y$,且当 $\Delta x,\Delta y$ 趋于0时也趋于0.由此
\begin{equation}
\frac{\Delta y}{\Delta x}=-\frac{F_x'(x,y)+\alpha}{F_y'(x,y)+\beta}
\end{equation}
由条件4: $F_y'(x,y)\neq0$,所以
\begin{equation}
f'(x)=y_x'=\lim_{\Delta x\rightarrow 0}\frac{\Delta y}{\Delta x}=-\frac{F_x'(x,y)}{F_y'(x,y)}
\end{equation}
存在.代入 $y=f(x)$,上式写为
\begin{equation}
f'(x)=-\frac{F_x'(x,f(x))}{F_y'(x,f(x))}
\end{equation}
因为等式右边分母分子都是连续函数的连续函数,且分母不为0,故知 $f'(x)$ 连续.定理得证!

