% 可取曲线(变分学)
% 泛函|可取曲线

Hamilton 分析方法是现代理论物理的通用方法,只要给定相应的作用量,就可以通过Hamilton 分析构建一门理论,并判断理论的自洽性及对称性和相应的守恒量等诸多性质.经典力学如此,相对论如此,量子力学如此,杨米尔斯理论亦可如此.要掌握 Hamilton 分析,必要的数学准备是逃不开的.当然,我们不可能去搞懂每一个数学上的细节,因为每一个数学细节背后往往都有一门深厚的学问.同时也不该漏掉一些重要的数学概念,因为往往一个艰巨的物理问题后面只是一个简单的数学原理.我们力求在用到数学的地方,都有一个较合理的解释,以便更清楚的看清问题,把握问题的实质.

变分学是 Hamilton 分析绕不过的一道坎,这也是目前首要的任务之一.当然,我们只挑取以后将用到的一部分内容.现在开始吧!

在变分法中,我们须研究这样的关系,其中因变数的值是由函数所确定的.比如研究连接给定两点 $A(x_A,y_A),B(x_B,y_B)$ 的任意曲线的长度,因变数的值“曲线的长度”是由连接 $A,B$ 两点的曲线的形状决定的.