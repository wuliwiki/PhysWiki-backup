% 牛顿运动定律(综述)
% license CCBYSA3
% type Wiki

(本文根据 CC-BY-SA 协议转载翻译自维基百科\href{https://en.wikipedia.org/wiki/Newton\%27s_laws_of_motion#}{相关文章})

牛顿运动定律是三条描述物体运动与作用于其上的力之间关系的物理定律。这些定律构成了牛顿力学的基础,可概括为以下内容:

\begin{enumerate}
\item 一个物体将保持静止状态或以恒定速度沿直线运动,除非有力作用在其上。
\item 在任一时刻,物体所受的合力等于物体的加速度乘以其质量,或等同于物体动量随时间变化的速率。
\item 如果两个物体相互施加力,这些力的大小相等但方向相反。[1][2]
\end{enumerate}
牛顿的三大运动定律最早由艾萨克·牛顿在他的著作《自然哲学的数学原理》(拉丁文:《Philosophiæ Naturalis Principia Mathematica》)中提出,该书首次出版于1687年。牛顿利用这些定律研究并解释了许多物理物体和系统的运动。自牛顿以来,新的见解,特别是关于能量的概念,基于他的基础构建了经典力学的领域。同时也发现了牛顿定律的局限性:当物体以极高的速度运动时(狭义相对论)、质量极大时(广义相对论)、或在极小尺度上(量子力学),需要新的理论来解释。
\subsection{前提条件 }
牛顿定律通常是以点或质点的形式表述的,即体积可以忽略不计的物体。这在实际物体中是一个合理的近似,当内部各部分的运动可以忽略不计,并且物体之间的距离远大于它们各自的大小时,牛顿定律是适用的。例如,当考虑地球围绕太阳的轨道时,可以将地球和太阳都近似为点状物体,但在考虑地球表面的活动时,地球显然不是点状物体。[注1]

运动的数学描述,即运动学,是基于使用数值坐标来指定位置的概念。物体的运动通过这些数值随时间变化来表示:物体的轨迹是一个函数,它将每个时间变量的值与所有位置坐标的值关联起来。最简单的情况是一维运动,也就是当一个物体只能沿直线运动时。此时,它的位置可以用一个数值表示,说明它相对于某个选定参考点的位置。例如,一个物体可能在一条从左到右的轨道上自由滑动,因此它的位置可以通过距离某个方便的零点(原点)的距离来指定,负数表示位于左侧的位置信息,正数表示位于右侧的位置信息。如果物体的位置是时间的函数 \( s(t) \),那么物体在从 \( t_0 \) 到 \( t_1 \) 的时间间隔内的平均速度为:
\[
\frac{\Delta s}{\Delta t} = \frac{s(t_1) - s(t_0)}{t_1 - t_0}~
\]
在这里,按照传统使用希腊字母 Δ(delta)表示“变化量”。正的平均速度意味着在所讨论的时间间隔内,位置坐标 \( s \) 增加;负的平均速度表示该间隔内的位置净减少;而零的平均速度则意味着物体在时间间隔结束时与开始时位于同一位置。微积分提供了定义瞬时速度的方法,它衡量物体在某一时刻的速度和运动方向,而不是在一个时间区间内。瞬时速度的一种表示法是用符号 \( d \) 替代 \( \Delta \),例如:
\[
v = \frac{ds}{dt}~
\]
这表示瞬时速度是位置随时间变化率的导数。

这表明瞬时速度是位置对时间的导数。大致可以理解为位置的微小变化 \( ds \) 与发生在微小时间间隔 \( dt \) 之间的比率【7】。更准确地说,速度和所有其他导数都可以通过极限的概念来定义【6】。一个函数 \( f(t) \) 在给定的输入值 \( t_0 \) 处有极限 \( L \),如果通过选择足够接近 \( t_0 \) 的输入,可以使 \( f \) 与 \( L \) 之间的差异任意小。我们写作:
\[
\lim _{t\to t_{0}}f(t)=L~
\]
瞬时速度可以定义为当时间间隔趋于零时,平均速度的极限:
\[
\frac{ds}{dt}=\lim _{\Delta t\to 0}\frac{s(t+\Delta t)-s(t)}{\Delta t}~
\]
加速度与速度的关系,就像速度与位置的关系一样:它是速度对时间的导数【注2】。加速度同样可以通过极限定义:
\[
a=\frac{dv}{dt}=\lim _{\Delta t\to 0}\frac{v(t+\Delta t)-v(t)}{\Delta t}~
\]
因此,加速度是位置的二阶导数,通常写作:
\[
\frac{d^{2}s}{dt^{2}}~
\]
位置,当被认为是从原点的位移时,是一个向量:既有大小又有方向的量【9】:1。速度和加速度也是向量量。向量代数的数学工具提供了描述二维、三维或更多维度运动的手段。向量通常用带箭头的符号表示,如 \( \mathbf{s} \),或者用粗体字体表示,如 \( {\bf {s}} \)。向量常常以箭头的形式可视化表示,向量的方向就是箭头的方向,向量的大小则由箭头的长度表示。数字上,一个向量可以用列表表示;例如,一个物体的速度向量可能是:
\[
\mathbf{v} = (3 \ \mathrm{m/s}, 4 \ \mathrm{m/s})~
\]
这表示物体以每秒3米的速度沿水平轴运动,以每秒4米的速度沿垂直轴运动。在不同的坐标系中,描述相同运动时会用不同的数值表示,向量代数可以用于在这些不同的表示法之间转换【9】:4。

力学的研究由于一些日常用词(如“能量”)在物理学中具有特殊的技术含义而变得复杂【10】。此外,某些在日常语言中同义的词在物理学中并非如此:例如,力与功率或压力不同,质量与重量的含义也不同【11】【12】:150。物理学中的“力”概念将日常的“推”或“拉”进行量化。在牛顿力学中,力通常来源于绳索、摩擦、肌肉力量、重力等。与位移、速度和加速度一样,力也是一个向量量。
\subsection{定律}
\subsubsection{第一定律}
从拉丁文翻译过来,牛顿第一定律的表述是:\\
每个物体都保持静止状态或沿直线做匀速运动,除非有外力迫使它改变这种状态【注3】。

牛顿第一定律表达了惯性原理:物体的自然行为是以恒定速度沿直线运动。物体的运动会保持现状,但外部的力可以扰动这种状态。
\begin{figure}[ht]
\centering
\includegraphics[width=6cm]{./figures/97abd6a7044ae3a7.png}
\caption{人造卫星沿着曲线轨道运动,而不是沿直线运动,这是由于地球的重力作用。} \label{fig_NEW01_1}
\end{figure}
对牛顿第一定律的现代理解是:没有一个惯性观察者比另一个惯性观察者更“特权”。惯性观察者的概念量化了日常中不感受到运动的想法。例如,一个站在地面上看火车经过的人是惯性观察者。如果地面的观察者看到火车以恒定速度沿直线平稳移动,那么坐在火车上的乘客也将是惯性观察者:乘客感觉不到运动。

牛顿第一定律表达的原理是,无法判断哪个惯性观察者“真正”在运动,哪个“真正”是静止的。一个观察者的静止状态对于另一个观察者来说可能是匀速直线运动状态,任何实验都无法判定哪种观点是正确或错误的。不存在绝对的静止标准【17】【14】:62–63【18】:7–9。虽然牛顿本人认为绝对的空间和时间是存在的,但他也认为实验所能测量的空间和时间都是相对的【19】。
\subsubsection{第二定律}
物体的运动变化与施加的力成正比,并且沿着施力的直线方向进行。[14]:114 
 
牛顿所说的“运动”,指的是现在称为动量的量,动量取决于物体所包含的物质量、该物体运动的速度以及它运动的方向。[20] 在现代符号中,物体的动量是其质量与速度的乘积:
\[
\mathbf{p} = m \mathbf{v}~
\]
其中,质量 \(m\)、速度 \( \mathbf{v} \) 和动量 \( \mathbf{p} \) 三者都可以随时间变化。牛顿第二定律的现代形式表述为动量的时间导数等于力:
\[
\mathbf{F} = \frac{d\mathbf{p}}{dt}~
\]
如果质量 \(m\) 随时间不变,那么导数只作用在速度上,因此力等于质量与速度的时间导数之积,即加速度:
\[
\mathbf{F} = m \frac{d\mathbf{v}}{dt} = m \mathbf{a}~
\]
由于加速度是位置相对于时间的二阶导数,这条定律也可以写作:
\[
\mathbf{F} = m \frac{d^2 \mathbf{s}}{dt^2}~
\]
作用在物体上的力是以向量形式相加的,因此物体所受的总力不仅取决于各个力的大小,还取决于它们的方向。当物体的合力等于零时,根据牛顿第二定律,物体不会产生加速度,称之为机械平衡状态。如果在物体的位置稍微发生变化时,物体仍然保持在该平衡附近,那么这种机械平衡是稳定的。否则,平衡就是不稳定的。
\begin{figure}[ht]
\centering
\includegraphics[width=6cm]{./figures/5e0843edbdfce42a.png}
\caption{一个表示斜面上物体的受力图,展示了垂直于斜面的法向力 \(N\),向下的重力 \(mg\),以及沿着斜面方向的力 \(f\),例如由摩擦力或绳子施加的力。} \label{fig_NEW01_2}
\end{figure}
常见的力共同作用的视觉表示形式是\textbf{受力图},它以图解的方式描绘了受关注的物体及其受到的外部力量。例如,一个坐在斜面上的方块的受力图可以展示重力、法向力、摩擦力和绳子张力的组合。[注4]

牛顿第二定律有时被作为\textbf{力的定义}来呈现,即当惯性观察者看到一个物体加速时,说明存在一个力。为了避免这种解释成为自我证明的循环推理——加速度意味着存在力,力意味着存在加速度——必须对力做出其他的说明。例如,可以用方程来描述力,比如牛顿的万有引力定律。通过将这样的力表达式 \( \mathbf{F} \) 插入到牛顿第二定律中,就能写出一个具有预测能力的方程。[注5] 牛顿第二定律也被认为为物理学设立了一个研究纲领,明确了该学科的重要目标是识别自然界中存在的各种力,并编制物质成分的目录。[14]: 134 [25]: 12-2
\subsubsection{第三定律}  
每一个作用总有一个与之相反的、大小相等的反作用;或者说,两个物体之间的相互作用总是大小相等,并且方向相反。[14]:116

对牛顿第三定律的过于简化的表述,如“作用等于反作用”,可能让几代学生产生了误解:作用和反作用是作用在不同物体上的。例如,考虑一本静止在桌子上的书。地球的引力向下拉动书本。对此“作用”的“反作用”并不是桌子支撑书本向上的力,而是书本对地球的引力作用。[注6]
\begin{figure}[ht]
\centering
\includegraphics[width=6cm]{./figures/7726788373d9c5b0.png}
\caption{火箭通过火箭发动机向下产生强大的反作用力来工作,这个反作用力将火箭推向上方,而不依赖于地面或大气的存在。} \label{fig_NEW01_3}
\end{figure}
牛顿第三定律与一个更基本的原理——动量守恒定律——有关。即使在某些情况下牛顿的表述不适用(例如,当力场和物体都携带动量时,或者在量子力学中正确定义动量时),动量守恒定律仍然成立。[注7] 在经典力学中,如果两个物体的动量分别是 \( \mathbf{p}_1 \) 和 \( \mathbf{p}_2 \),那么这两个物体的总动量是:$\mathbf{p} = \mathbf{p}_1 + \mathbf{p}_2~$
总动量 \( \mathbf{p} \) 的变化率为:
\[
\frac{d\mathbf{p}}{dt} = \frac{d\mathbf{p}_1}{dt} + \frac{d\mathbf{p}_2}{dt}~
\]
根据牛顿第二定律,动量变化的第一个项是作用在第一个物体上的总力,第二个项是作用在第二个物体上的总力。如果这两个物体是隔离的,不受外部影响,那么第一个物体所受的唯一力只能来自第二个物体,反之亦然。根据牛顿第三定律,这些力的大小相等、方向相反,因此它们相加时会相互抵消,总动量 \( \mathbf{p} \) 保持不变。或者,若已知 \( \mathbf{p} \) 是恒定的,则可以推断出两个物体的力大小相等且方向相反。
\subsubsection{其他可能被视为补充定律的候选法则}
一些来源建议将经典力学中使用的其他概念提升到牛顿定律的地位。例如,在牛顿力学中,由两个较小物体组合而成的物体的总质量是其各自质量的总和。物理学家弗兰克·维尔切克(Frank Wilczek)曾建议通过称其为“牛顿第零定律”来强调这一假设。[33] 另一个“第零定律”的候选者是这样的事实:物体在任一瞬间对该瞬间施加的力作出反应。[34] 同样,力像向量一样相加(或遵守叠加原理)这一概念,以及力改变物体能量的概念,都曾被描述为“第四定律”。[注8]
\subsubsection{例子}
利用牛顿定律研究大质量物体的行为被称为牛顿力学。牛顿力学中的一些问题因其概念性或历史意义而特别值得注意。
\subsubsection{匀加速运动} 
主条目:自由落体和抛体运动
\begin{figure}[ht]
\centering
\includegraphics[width=6cm]{./figures/fc18c394f77b678b.png}
\caption{使用频闪灯以每秒25帧拍摄的弹跳球。在两次弹跳之间,球的高度随时间的变化接近抛物线,但由于空气阻力、旋转以及碰撞时变形为非球形,轨迹会偏离抛物线。} \label{fig_NEW01_4}
\end{figure}
如果一个物体从静止状态开始在地球表面附近下落,那么在没有空气阻力的情况下,它将以恒定的速度加速。这称为自由落体。在自由落体过程中获得的速度与经过的时间成正比,物体所经过的距离与经过时间的平方成正比。[39] 重要的是,加速度对所有物体都是相同的,与它们的质量无关。这是结合牛顿的第二运动定律和万有引力定律得出的。后者表明地球对物体施加的引力大小为
\[
F = \frac{GMm}{r^2}~
\]
其中,\( m \) 表示下落物体的质量,\( M \) 表示地球的质量,\( G \) 是牛顿引力常数,\( r \) 是从地球中心到物体位置的距离,近似为地球的半径。将这个力等同于 \( ma \) (物体的质量乘以加速度),物体的质量 \( m \) 会在方程的两边相互抵消,剩下的加速度只依赖于 \( G \)、\( M \) 和 \( r \),并且 \( r \) 可以视为常数。这个特定的加速度通常用 \( g \) 表示:
\[
g = \frac{GM}{r^2} \approx 9.8~\text{m/s}^2~
\]
如果物体不是从静止状态释放,而是以非零速度向上和/或水平发射,那么自由落体就变成了抛体运动。[40] 当空气阻力可以忽略时,抛体的轨迹呈抛物线形,因为重力只影响物体的垂直运动,而不影响水平运动。在抛体轨迹的最高点,物体的垂直速度为零,但其加速度始终是向下的 \( g \),这一点在任何时刻都不变。将错误的向量设为零是物理学生常见的困惑之一。[41]
\subsubsection{匀速圆周运动} 
主条目:圆周运动
\begin{figure}[ht]
\centering
\includegraphics[width=6cm]{./figures/6fdeb30eb2a61c96.png}
\caption{两个做匀速圆周运动的物体围绕着质心(两个物体的共同质量中心)运行。} \label{fig_NEW01_5}
\end{figure}
当一个物体做匀速圆周运动时,作用在它上的力改变了其运动方向,但不改变其速度。对于一个以恒定速度 \( v \) 在半径为 \( r \) 的圆周上运动的物体,其加速度的大小为
\[
a = \frac{v^2}{r}~
\]
并指向圆心。[注9] 为维持这种加速度所需的力,称为向心力,因此它也指向圆心,大小为\(mv^2/r\)许多轨道运动,例如月球绕地球的运动,可以近似为匀速圆周运动。在这种情况下,向心力是引力,根据牛顿的万有引力定律,其大小为\(GMm/r^2\)其中 \( M \) 是被绕行的较大物体的质量。因此,可以通过观测一个物体绕另一个物体的轨道运动来计算该物体的质量。[43]:130

牛顿的炮弹是一个介于抛体运动和匀速圆周运动之间的思想实验。一颗炮弹如果从高悬崖边上以较小的力量抛出,它落地的时间与静止状态下掉落的时间相同,因为重力只影响炮弹在垂直方向上的动量,水平运动并不会削弱重力的影响。如果炮弹以更大的初始水平速度发射,它在落地前会飞得更远,但仍会在相同的时间内落地。然而,如果炮弹以更大的初始速度发射,地球的曲率将变得显著:地面本身会向远离炮弹的方向弯曲。一颗非常快速的炮弹将以与地球弯曲的速率偏离惯性直线轨迹;换句话说,它将进入轨道(假设它没有受到空气阻力或障碍物的影响)。[44]
\subsubsection{简谐运动} 
主条目:简谐振子
\begin{figure}[ht]
\centering
\includegraphics[width=6cm]{./figures/160ce88531f4bdd1.png}
\caption{无阻尼弹簧-质量系统会进行简谐运动。} \label{fig_NEW01_6}
\end{figure}
考虑一个质量为 \( m \) 的物体,它可以沿着 \( x \) 轴移动,假设在位置 \( x = 0 \) 存在一个平衡点。也就是说,当 \( x = 0 \) 时,作用在物体上的合力为零向量,根据牛顿第二定律,物体将不会加速。如果作用在物体上的力与物体偏离平衡点的位移成正比,且指向平衡点,那么该物体将进行简谐运动。将力写为 \( F = -kx \),牛顿第二定律变为:
\[
m \frac{d^2 x}{dt^2} = -kx~
\]
这个微分方程的解为:
\[
x(t) = A \cos(\omega t) + B \sin(\omega t)~
\]
其中,频率 \( \omega \) 等于 \( \sqrt{k/m} \),常数 \( A \) 和 \( B \) 可以通过已知的初始条件(例如物体在某个时刻的位移和速度)计算出来,比如 \( t = 0\)。

简谐振子之所以是概念上重要的例子之一,原因在于它是许多系统在稳定机械平衡附近的良好近似。[注10] 例如,钟摆在垂直位置有一个稳定的平衡点:如果它静止不动,就会保持在该位置;如果轻微推动,它将来回摆动。在忽略空气阻力和支点摩擦的情况下,作用在钟摆上的力是重力,牛顿第二定律变为
\[
\frac{d^2 \theta}{dt^2} = -\frac{g}{L} \sin \theta~
\]
其中 \( L \) 是钟摆的长度,\( \theta \) 是钟摆偏离垂直方向的角度。当角度 \( \theta \) 很小时,\( \sin \theta \) 近似等于 \( \theta \)(参见泰勒级数),因此该表达式简化为简谐振子的方程,频率为\(\omega = \sqrt{g/L}\)。

简谐振子可能受到阻尼,通常由摩擦或粘性阻力引起,在这种情况下,能量从振子中耗散,振动的振幅会随着时间逐渐减小。简谐振子也可以受到外力驱动,这可能导致共振现象的发生[46]
\subsubsection{质量可变的物体}
主条目:可变质量系统
\begin{figure}[ht]
\centering
\includegraphics[width=6cm]{./figures/2b422205eb32888f.png}
\caption{火箭(如亚特兰蒂斯号航天飞机)通过将物质向一个方向喷射来推动飞船朝另一个方向运动。这意味着被推动的质量——火箭及其剩余的燃料供应——是在不断变化的。} \label{fig_NEW01_7}
\end{figure}
牛顿物理学认为物质既不会被创造也不会被消灭,尽管它可能被重新排列。在某些情况下,物体可能因为物质被添加或移除而增加或减少质量。在这种情况下,可以将牛顿定律应用于物质的各个部分,并随时间跟踪哪些部分属于所研究的物体。例如,如果一个质量为 \( M(t) \) 的火箭以速度 \( \mathbf{v}(t) \) 运动,并以相对于火箭的速度 \( \mathbf{u} \)喷射物质,那么
\[
\mathbf{F} = M\frac{d\mathbf{v}}{dt} - \mathbf{u} \frac{dM}{dt}~
\]
其中 \( \mathbf{F} \) 是净外力(例如行星的引力作用)。[23]:139
\subsection{功和能}