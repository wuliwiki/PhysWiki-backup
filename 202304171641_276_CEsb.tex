% 正则系综法
% 统计力学|亥姆霍兹自由能|正则系综
\pentry{理想气体(微正则系综法)\upref{IdNCE}}
热力学主要研究平衡态的多体系统具有的宏观热力学量之间的联系,即状态方程。而统计力学要做的事是将这些宏观热力学量同系统的微观状态及其统计性质关联起来。于是我们有了另一个研究的“视角”。假使我们知道系统(这个系统是量子系统)中粒子的质量、电荷、自旋等性质,知道粒子之间的相互作用,我们希望通过这些信息来求知多体系统的宏观性质。

在微正则系综理论中,人们固定能量 $E$ 、体积 $V$ 以及粒子数 $N$(确切地说,还包含其他一切“守恒荷”,例如总电荷,总自旋,可以记为$N_1,N_2,\cdots$),并将熵定义为系统的微观状态数的对数即玻尔兹曼关系 $S=k\ln \Omega$。但是具体分析问题的时候,人们发现微正则系综法常常是不方便的:固定能量 $E$ 来求相应的 $\Omega$ 是非常困难的,尤其是当粒子间存在相互作用时,$E$ 还包含了相互作用势能,微观状态数计算将变得相当繁杂。于是人们构建了一个新的\textbf{理论模型},称为正则系综。

正则系综的分布函数可以通过让系统与一个大热源接触来导出。与微正则系综不同的是,正则系综不是去固定能量,而是固定温度 $T$(这个 $T$ 是系统与大热源作为一个大整体所具有的宏观温度,$T$ 的存在性由热力学第零定律\upref{TherEq}保证)。在这个温度下,系统的能量 $E_S$ 是不确定的,其能量的平均值则是我们所关心的系统宏观热力学量,即内能 $E$。由于与微正则系综的理论模型条件不同,由正则系综得到的系统热力学量的\textbf{涨落}与微正则系综是不同的,但它们能够得到\textbf{一致的热力学状态方程},并且在粒子数趋于无穷时可以忽略涨落的影响,两种模型所对应的平衡态多体系统的表现是完全一致的。
\subsection{正则系综}
考虑将待研究的系统和一个温度恒定为 $T$ 的大热源\footnote{即热容为无穷大的系统。}接触。我们想要研究,当两者达到平衡时,系统处于某一量子态 $S$ 的概率 $\rho_S$。由正则系综的理论推导可以解得,这个几率只和系统处于该量子态时的能量有关:即
\begin{equation}\label{eq_CEsb_1}
\rho_S=\frac{1}{Z}\exp\qty(-\beta E_S)~.
\end{equation}
其中 $\beta=1/kT$,$Z$ 为归一化常数,使得 $\sum_S \rho_S=1$。我们称 $Z$ 为正则系综的配分函数,它由下式给出($S$ 遍历所有量子态):
\begin{equation}
Z=\sum_S \exp\qty(-\beta E_S)~.
\end{equation}
一旦得到了系统的配分函数,系统的一切平衡态热力学性质都可以由配分函数导出\upref{TheSta}。注意这里的配分函数是系统的配分函数,而不是玻尔兹曼分布\upref{MBsta}采用的单粒子配分函数;它可以适用于一般的多体系统,如分子间有相互作用势的情况。

除了上述的微观状态数统计分布的性质,我们需要重新考虑平衡态多体系统的状态方程。与微正则系综不同,由于此处系统的 $E$ 不确定,$S=S(E,V,N)$ 也不确定,我们不能够从 $E=E(S,V,N)$ 方程出发建立热力学体系。正则系综中温度是确定的,所以我们需要从 $S$ 表象切换到 $T$ 表象,建立一个系统的热力学势函数同 $T,V,N$ 之间的关系。幸运的是,热力学状态方程 $\dd E=T\dd S-p\dd V+\mu\dd N$ 暗示了我们,自由能 $F=E-TS$ 是关于 $T,V,N$ 的函数,用系统微观状态信息表达状态方程中的热力学量,可以证明 $F=-kT\ln Z$。至此,我们知道了 $F(T,V,N)$,并由此知道了系统的一切热力学性质。

\subsection{从正则系综到玻尔兹曼分布}
对于\textbf{近独立子系}\upref{depsys},系统的总能量 $E$ 可以表示成所有粒子的能量之和。于是
\begin{equation}
\rho_S=\exp(-\beta E_S)=\prod_{i=1}^N\exp(-\beta\epsilon_i)
\end{equation}
如果再假设系统是定域的,即粒子间是可分辨的(不用考虑全同粒子假设),那么每个粒子分别处于某个量子态,对应整个系统的一个量子态 $S$。配分函数可写为:
\begin{equation}
\begin{aligned}
Z&=\sum_S\exp(-\beta E_S)=\qty(\sum_{l}\exp(-\beta\epsilon_{l1}))\cdots\qty(\sum_{l}\exp(-\beta\epsilon_{lN}))\\
&=\qty(\prod_{l}\exp(-\beta\epsilon_{l}))^N=Z_1^N
\end{aligned}
\end{equation}
这说明,对于定域的近独立子系,系统的配分函数就是 $N$ 个单粒子配分函数的乘积。只要我们证明了正则系综的分布 \autoref{eq_CEsb_1},经典理想气体的玻尔兹曼分布就可以作为它的简单推论了\autoref{eq_MBsta_2}~\upref{MBsta}。

在推导近独立的量子气体系统中的粒子分布时,利用正则系综就会出现一些技术上的难点,此时用\textbf{巨正则系综}能更加方便地得到玻色分布和费米分布,可以参考量子气体(巨正则系宗)和\upref{QGsME}量子气体(单能级巨正则系综法)\upref{QGs1ME},或者参考近独立子系\upref{depsys}词条中的讨论。

正则系综法的一个重要意义是,它给出有相互作用势气体的分布(集团展开词条),并可以给出系统偏离最概然分布的概率(涨落的热力学理论词条)。
\addTODO{需要关联集团展开词条,涨落的热力学理论词条}
\subsection{正则系综分布的推导}
我们可以将这个系统与温度为 $T$ 的大热源视作一个整体来分析。设总的能量为 $E_0$,系统能量为 $E_S$,则热源能量为 $E_0-E_S$。由于 $E_0\gg E_S$,这个整体的涨落可以忽略不计,所以 $E_0$ 是恒定的,我们可以对这个整体用微正则系综法进行分析,即对这个整体运用等概率原理。

当系统处于某一个特定的微观状态 $S$,对应能量 $E_S$,这种情形的方案数实际上等于热源的微观状态数,即 $\Omega(E_0-E_S)$。根据等概率原理,状态 $S$ 出现的几率为
\begin{equation}
\rho(E_S)\propto e^{\ln \Omega(E_0-E_S)}
\end{equation}
由于 $E_S\ll E_0$,可以对分子进行泰勒展开。
\begin{equation}\label{eq_CEsb_2}
\begin{aligned}
\ln \Omega(E_0-E_S)&=\ln \Omega(E_0)-E_S\left.\pdv{\ln \Omega}{E}\right|_{E=E_0}\\
&=\ln \Omega(E_0)-\frac{E_S}{k}\pdv{S_{\text{热源}}}{E_{\text{热源}}}\\
&=\ln \Omega(E_0)-\beta E_S
\end{aligned}
\end{equation}
其中 $\beta=1/kT$。因此 $\rho(E_S)\propto \exp(-\beta E_S)$,我们就得到了正则系综的微观状态分布公式 \autoref{eq_CEsb_1}。