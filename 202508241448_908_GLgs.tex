% 格林公式(综述)
% license CCBYSA3
% type Wiki

本文根据 CC-BY-SA 协议转载翻译自维基百科\href{https://en.wikipedia.org/wiki/Green\%27s_theorem}{相关文章}。

在向量分析中,格林公式把围绕一条简单闭合曲线 $C$ 的曲线积分与该曲线所围平面区域 $D$(即 $\mathbb{R}^2$ 中的曲面)的二重积分联系起来。它是斯托克斯定理在二维空间($\mathbb{R}^2$)中的特例。在一维情形下,它等价于微积分基本定理;在三维情形下,它则等价于散度定理。
\subsection{定理}
设 $C$ 是平面上一条按正向(逆时针)取向、分段光滑的简单闭合曲线,$D$ 是 $C$ 所围成的区域。如果 $L$ 和 $M$ 是定义在包含 $D$ 的某个开区域上的函数,且它们在该区域内具有连续偏导数,则有
$$
\oint_{C} (L\,dx + M\,dy) 
= 
\iint_{D} 
\left( 
\frac{\partial M}{\partial x} 
- 
\frac{\partial L}{\partial y} 
\right) dA~
$$
其中,曲线 $C$ 上的积分路径方向为逆时针。
\subsection{应用}
在物理学中,格林公式有许多应用。例如,在处理二维流体积分问题时,可以用它说明:一个区域内流体的总外流量等于该区域边界曲线上的总外流量。在平面几何中,尤其是在面积测量中,格林公式还能用于仅通过对边界积分来求解平面图形的面积和形心位置。
\subsection{当 $D$ 是单连通区域时的证明}
以下是对简化区域 $D$ 的一半定理的证明。这里 $D$ 是I型区域,其边界曲线 $C_1$ 和 $C_3$ 由垂直线段(长度可能为零)连接。对于II型区域(边界曲线 $C_2$ 和 $C_4$ 由水平线段连接)的另一半定理,也存在类似的证明。将这两部分结合起来,就可以证明适用于III 型区域(既是 I 型又是 II 型区域)的格林公式。通过将一般区域 $D$ 分解为一组 III 型区域,还可以将结论推广到更一般的情形。

如果能够证明以下两式:
$$
\oint_{C} L\,dx = \iint_{D} \left( -\frac{\partial L}{\partial y} \right) dA
\tag{1}~
$$
以及
$$
\oint_{C} M\,dy = \iint_{D} \left( \frac{\partial M}{\partial x} \right) dA
\tag{2}~
$$
那么对于区域 $D$ 格林公式就立即成立。对于I 型区域,式 (1) 容易证明;对于 II 型区域,式 (2) 也可以类似地证明。由此,格林公式便适用于III 型区域。
\begin{figure}[ht]
\centering
\includegraphics[width=8cm]{./figures/66e066d22e8a8b87.png}
\caption{如果区域 $D$ 是一种简单类型,其边界由曲线 $C_1, C_2, C_3, C_4$ 构成,那么可以证明格林定理的一半内容。} \label{fig_GLgs_1}
\end{figure}
假设区域 $D$ 是I 型区域,因此它可以表示为(如上图所示):
$$
D = \{(x, y) \mid a \leq x \leq b, \; g_1(x) \leq y \leq g_2(x)\},~
$$
其中 $g_1$ 和 $g_2$ 是区间 $[a, b]$ 上的连续函数。计算式 (1) 中的二重积分:
$$
\begin{aligned}
\iint_D \frac{\partial L}{\partial y} \, dA
&= \int_a^b \int_{g_1(x)}^{g_2(x)} \frac{\partial L}{\partial y}(x, y) \, dy \, dx \\
&= \int_a^b \big[ L(x, g_2(x)) - L(x, g_1(x)) \big] \, dx.
\end{aligned}
\tag{3}~
$$
计算式 (1) 中的曲线积分:曲线 $C$ 可以分解为四段:$C_1, C_2, C_3, C_4$。
沿 $C_1$,参数方程为:$x = x,\; y = g_1(x),\; a \leq x \leq b$.因此:
  $$
  \int_{C_1} L(x, y) \, dx
  = \int_a^b L(x, g_1(x)) \, dx~
  $$
沿$C_3$,参数方程为:$x = x,\; y = g_2(x),\; a \leq x \leq b$.因为方向相反,积分为:
  $$
  \int_{C_3} L(x, y) \, dx
  = \int_b^a L(x, y) \, dx
  = -\int_a^b L(x, g_2(x)) \, dx~
  $$
在 $C_3$ 上积分需要取负号,是因为曲线从 $b$ 到 $a$ 的方向与正向(逆时针方向)相反。
在 $C_2$ 和 $C_4$ 上,$x$ 保持不变,因此:
$$
\int_{C_4} L(x, y)\, dx = \int_{C_2} L(x, y)\, dx = 0~
$$
因此:
$$
\begin{aligned}
\oint_{C} L\, dx
&= \int_{C_1} L(x, y)\, dx
 + \int_{C_2} L(x, y)\, dx
 + \int_{C_3} L(x, y)\, dx
 + \int_{C_4} L(x, y)\, dx \\
&= \int_a^b L(x, g_1(x))\, dx
 - \int_a^b L(x, g_2(x))\, dx
\end{aligned}
\tag{4}~
$$
将式 (3) 与式 (4) 结合,就得到了适用于I型区域的式 (1)。用相同的思路处理相应端点,可以得到适用于II型区域的式 (2)。将这两部分结合,就得到了适用于III型区域的最终结果。
\subsection{可整曲线的证明}
我们要证明如下结论:

\textbf{定理} —— 设 $\Gamma$ 是 $\mathbb{R}^2$ 中一条可整、按正向取向的 Jordan 曲线,并设 $R$ 是其内部区域。假设函数$A, B : \overline{R} \to \mathbb{R}$在闭区域 $\overline{R}$ 上连续,且满足:$A$ 在区域 $R$ 内的每一点都有二阶偏导数;$B$ 在区域 $R$ 内的每一点都有一阶偏导数;函数$
D_1 B, \; D_2 A : R \to \mathbb{R}$在 $R$ 上是黎曼可积的。

则有:
$$
\int_{\Gamma} \big( A\,dx + B\,dy \big)
= 
\int_{R} \big( D_1 B(x, y) - D_2 A(x, y) \big) \, d(x, y)~
$$

我们需要以下引理,其证明可参见文献 [3]:

\textbf{引理 1(分解引理)} —— 假设 $\Gamma$ 是平面上一条可整、按正向取向的Jordan 曲线,并设 $R$ 为其内部区域。对于任意正实数 $\delta$,令 $\mathcal{F}(\delta)$ 表示平面上由直线$x = m\delta,\; y = m\delta$
(其中 $m$ 为整数)所围成的正方形的集合。那么,对于该 $\delta$,可以将闭区域 $\overline{R}$ 分解为有限个互不重叠的子区域,使得:
\begin{enumerate}
\item 每一个完全包含在 $R$ 内的子区域(记为 $R_1, R_2, \ldots, R_k$)都是 $\mathcal{F}(\delta)$ 中的某个正方形。
\item 其余的子区域(记为 $R_{k+1}, \ldots, R_s$)的边界由有限条 $\Gamma$ 的弧段和 $\mathcal{F}(\delta)$ 中某些正方形的边组成,构成一条可整的 Jordan 曲线。
\item 每个边界子区域 $R_{k+1}, \ldots, R_s$ 都可以被一个边长为 $2\delta$ 的正方形所覆盖。
\item 如果 $\Gamma_i$ 是子区域 $R_i$ 的按正向取向的边界曲线,则有:$\Gamma = \Gamma_1 + \Gamma_2 + \cdots + \Gamma_s$。
\item 这些边界子区域的数量 $s-k$ 不超过:$4\left(\frac{\Lambda}{\delta} + 1\right)$,其中 $\Lambda$ 表示曲线 $\Gamma$ 的总长度。
\end{enumerate}
\textbf{引理 2} —— 设 $\Gamma$ 是平面上的一条可整曲线,定义 $\Delta_{\Gamma}(h)$ 为平面上到曲线 $\Gamma$ 的轨迹距离不超过$h$的点集。那么该集合的外 Jordan 测度满足:$\overline{c}\,\Delta_{\Gamma}(h) \leq 2h\Lambda + \pi h^2$,其中 $\Lambda$ 表示曲线 $\Gamma$ 的长度。

\textbf{引理 3} —— 设 $\Gamma$ 是 $\mathbb{R}^2$ 中一条闭合的可整曲线,且 $f : \text{range of } \Gamma \to \mathbb{R}$ 是定义在 $\Gamma$ 上的连续函数。那么:
$$
\left| \int_{\Gamma} f(x, y)\, dy \right| \leq \frac{1}{2} \Lambda \, \Omega_f,~
$$
以及
$$
\left| \int_{\Gamma} f(x, y)\, dx \right| \leq \frac{1}{2} \Lambda \, \Omega_f,~
$$
其中,$\Omega_f$ 表示函数 $f$ 在曲线 $\Gamma$ 上的振幅(最大值与最小值之差)。
现在我们可以开始证明这个定理了:

\textbf{定理的证明。}设 $\varepsilon$ 是任意给定的正实数。由于 $A$、$B$ 在 $\overline{R}$ 上连续,且 $\overline{R}$ 是紧集,所以给定 $\varepsilon > 0$,存在 $0 < \delta < 1$,使得只要 $\overline{R}$ 中任意两点的距离小于 $2\sqrt{2}\,\delta$,它们在 $A,B$ 作用下的函数值之差就小于 $\varepsilon$。

在这个 $\delta$ 下,考虑前一条引理给出的分解。我们有
$$
\int_{\Gamma} A\,dx + B\,dy 
= \sum_{i=1}^{k} \int_{\Gamma_i} A\,dx + B\,dy 
+ \sum_{i=k+1}^{s} \int_{\Gamma_i} A\,dx + B\,dy.~
$$
令$\varphi := D_1 B - D_2 A$。

对于每个 $i \in \{1, \ldots, k\}$,曲线 $\Gamma_i$ 是一条按正向取向的正方形边界,因此格林公式在该区域成立。于是有:
$$
\sum_{i=1}^{k} \int_{\Gamma_i} A\,dx + B\,dy 
= 
\sum_{i=1}^{k} \int_{R_i} \varphi
= 
\int_{\bigcup_{i=1}^{k} R_i} \varphi.~
$$
边界区域中的每个点与曲线 $\Gamma$ 的距离都不超过 $2\sqrt{2}\,\delta$。因此,如果用 $K$ 表示所有边界区域的并集,那么有:$K \subset \Delta_{\Gamma}(2\sqrt{2}\,\delta)$,根据引理 2,可得:$c(K) \leq \overline{c}\,\Delta_{\Gamma}(2\sqrt{2}\,\delta)\leq 4\sqrt{2}\,\delta + 8\pi\delta^2$.
注意:

$$
\int_{R} \varphi - \int_{\bigcup_{i=1}^{k} R_i} \varphi
= 
\int_{K} \varphi.~
$$

由此得到:

$$
\left| 
\sum_{i=1}^{k} \int_{\Gamma_i} A\,dx + B\,dy 
- 
\int_{R} \varphi 
\right|
\leq 
M\delta \bigl(1 + \pi\sqrt{2}\,\delta \bigr),
$$

其中 $M > 0$。

我们可以进一步选择 $\delta$,使得上述不等式右边小于 $\varepsilon$。
