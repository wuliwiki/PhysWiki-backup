% 群代数与正则表示
% 群代数|群空间|正则表示
\pentry{群矩阵表示及实例\upref{gprep}}


\subsection{群代数}

注:本文主要以复表示为例。

\begin{definition}{群空间}
对于有限群$G=\{g_1,g_2...g_m\}$,设$V_G$为群元在复数域$\mathbb{C}$是的所有线性叠加的集合:

\begin{equation}
V_G=\{\displaystyle\sum_\nu x_\nu g_\nu|x_\nu \in \mathbb{C},g_\nu \in G\}
\end{equation}

在这个基础上我们可以定义加法和数乘。

设$x=\displaystyle\sum_\nu x_\nu g_\nu$,$y=\displaystyle\sum_\mu y_\mu g_\mu$,$a\in \mathbb{C}$,则有:
\begin{align}
x+y&=\displaystyle\sum_\nu x_\nu g_\nu+\displaystyle\sum_\mu y_\mu g_\mu=\displaystyle\sum_\nu(x_\nu+y_\nu)g_\nu\\
ax&=\displaystyle\sum_\nu (ax_\nu) g_\nu
\end{align}

这样显然构成了一个$m$维的线性空间,$m$为群$G$的阶数。

线性空间的一组基为$\{g_1,g_2...g_m\}$,称为自然基底。

\end{definition}

在定义完群空间后我们进一步定义群空间中的代数乘法,使其构成一个代数,也就是本节标题——群代数。

\begin{definition}{群代数}
设$V_G$为群$G$的群空间,且有$x=\displaystyle\sum_\nu x_\nu g_\nu$,$y=\displaystyle\sum_\mu y_\mu g_\mu$,我们定义其乘法规则为:

\begin{equation}
x * y = \displaystyle\sum_{\nu} x_\nu g_\nu \sum_{\mu} y_\mu g_\mu =
\displaystyle\sum_{\nu\mu}(x_\nu y_\mu) (g_\nu g_\mu)
\end{equation}

其中$g_\nu g_\mu$一项依照群乘法表的乘法规则给出结果。

在这样的乘法规则下$V_G$构成了复数域$\mathbb{C}$上的一个结合代数,称为$A_G$。

\end{definition}

注:群代数的结合律来自于群元的结合律。

在群代数的视角下,不同的群空间中的矢量相乘时可以将其中的一个视作算符,例如考虑$g_\alpha x=L(g_\alpha)x=\displaystyle\sum_\nu x_\nu (g_\alpha g_\nu)$,在这里值得注意的是我们将算符定义为了左侧相乘的元素,这涉及到对于算符的两种定义,我们将在后文加以区分,在那之前,我们先默认被抽象为算符的是左侧的矢量。

这样给出的算符可以验证其的确符合群乘法规则:

$$L(g_\alpha)L(g_\beta)x=L(g_\alpha)g_\beta x=g_\alpha g_\beta x=L(g_\alpha g_\beta)x$$

显然在给定的基下任意群元所对应的算符可以写成矩阵形式,对于自然基底下算符对应的矩阵表示称为正则表示,由于此处算符被定义为左乘所以此处给出的表示为左正则表示。

在给出左正则表示的具体形式之前,我们可以先定义群空间中的内积为:$(g_\alpha,g_\beta)=\delta_{\alpha,\beta}$。

群的左正则表示是一个$m$维表示,有:

$$D(g_\alpha)_{\mu\nu}=(g_\mu,g_\alpha g_\nu)=\delta_{\mu,(\alpha\nu)}$$

类似的可以定义右正则表示:

$$x g_\alpha=R(g_\alpha)x=\displaystyle\sum_\nu x_\nu (g_\nu g_\alpha)$$

这样定义后有:

$$R(g_\alpha)R(g_\beta)x=R(g_\alpha)x g_\beta= (x g_\alpha) g_\beta=R(g_\alpha g_\beta)x$$

这里有一个值得特别注意的地方,我们在此处定义的算符始终作用于矢量$x$上,所以第二个等号后的$g_\alpha$会紧贴着出现在$x$后边而非$g_\beta$后边。注意到这点后不难发现右正则表示也遵循着群乘法规则。