% 手算开根号

\subsection{试根法}

\begin{example}{}
\begin{figure}[ht]
\centering
\includegraphics[width=6cm]{./figures/Hsqrt_1.pdf}
\caption{手动开根号的例子, 计算 $\sqrt{21}$} \label{Hsqrt_fig1}
\end{figure}
作为一个例子, 我们来计算 $\sqrt{21}$. 首先试一个最大的个位数, 使它的平方恰好略小于 $21$, 已知改数为 $4$.把 $4$ 分别写到根号左边和上方, 相乘得 $16$ 写到 $21$ 下方. 现在计算 $21-16 = 5$, 写到下方并在后面添加两个零得 $500$. 接下来把刚才算出的 $4$ 乘以 2 
\end{example}

若我们要算 $s^2$ 的开根号, 并假设 $s$ 的 $n$ 位有效数字近似为 $s_n$, 例如 $s = \sqrt{2} = 1.414213562\dots$, 则 $s_1 = 1$, $s_2=1.4$, $s_3=1.41$,……
\begin{equation}
s^2 = s_1^2 + (s_2+s_1)(s_2-s_1) + (s_3+s_2)(s_3-s_2) + \dots
\end{equation}
现在, 第 $i$ 位有效数字(小数点位置不变)为 $d_i = s_i-s_{i-1}$, 且 $d_1 = s_1$, 易得
\begin{equation}
s^2 = d_1^2 + (2s_1 + d_2)d_2 + (2s_2 + d_3)d_3 + \dots
\end{equation}
现在, 若已知 $s^2$, 我们就可以先试出最大的 $d_1$, 满足 $d_1^2\leqslant s^2$. 然后再试出最大的 $d_2$, 满足 $(2s_1 + d_2)d_2 \leqslant s^2 - d_1^2$, 以此类推就可以求出任意位的小数.
