% 自旋(综述)
% license CCBYSA3
% type Wiki

本文根据 CC-BY-SA 协议转载翻译自维基百科\href{https://en.wikipedia.org/wiki/Spin_(physics)}{相关文章}。

自旋是基本粒子所具有的一种内禀角动量形式,因此,诸如强子、原子核和原子等复合粒子也具有自旋。[1][2]: 183–184  自旋是量子化的,并且描述自旋相互作用的准确模型需要相对论量子力学或量子场论。

电子自旋角动量的存在是通过实验推断出来的,例如斯特恩-盖拉赫实验,其中银原子被观察到具有两个可能的离散角动量状态,尽管它们没有轨道角动量。[3] 相对论性的自旋-统计定理将电子自旋的量子化与泡利不相容原理联系起来:不相容性的观测结果意味着自旋为半整数,而半整数自旋的观测结果又意味着不相容性。

在数学上,自旋可以用向量来描述(例如光子),也可以用旋量或双旋量(bispinor)来描述(例如电子)。旋量和双旋量在某些方面与向量类似:它们具有确定的大小,并且在旋转下发生变化;然而,它们的“方向”采用了一种非传统的方式。所有同类的基本粒子具有相同大小的自旋角动量,尽管其方向可以变化。这些特性通过赋予粒子一个自旋量子数来表示。[2]: 183–184 

自旋的国际单位制(SI)单位与经典角动量相同(即 \(N\cdot m\cdot s\),\(j\cdot s\) 或\(kg\cdot m^2\cdot s^{-1}\))。在量子力学中,角动量和自旋角动量具有离散的值,并且它们的大小与普朗克常数成比例。在实际应用中,自旋通常通过将自旋角动量除以约化普朗克常数\(\hbar\)来表示为无量纲的自旋量子数。通常,“自旋量子数”也直接被称为“自旋”。
\subsection{模型}
\subsubsection{旋转的带电质量}  
最早的电子自旋模型设想电子是一个旋转的带电质量,但当对该模型进行详细检验时,它无法成立:所需的空间分布不符合对电子半径的限制;此外,所需的旋转速度超过了光速。[4] 在标准模型中,所有基本粒子都被视为“点状”粒子:它们的作用是通过周围的场来体现的。[5] 任何基于质量旋转的自旋模型都需要与这一观点保持一致。
\subsubsection{泡利的“经典上无法描述的二值性” } 
沃尔夫冈·泡利是量子自旋历史上的核心人物,他最初拒绝将他为解释实验观察结果而引入的“自由度”与旋转联系起来。他称之为“经典上无法描述的二值性”。后来,他接受了自旋与角动量有关的观点,但坚持将自旋视为一种抽象属性。[6] 这一方法使泡利能够推导出他的基本泡利不相容原理,该证明后来被称为自旋-统计定理。[7] 从历史的角度来看,泡利对自旋的抽象处理方式及其证明方法开启了现代粒子物理学的时代,在这个时代,由对称性推导出的抽象量子属性占据主导地位,而具体的物理解释则变得次要甚至可有可无。[6]
\subsubsection{经典场的循环}
最早的经典自旋模型假设一个围绕轴旋转的小型刚性粒子,这一设想符合“自旋”一词的日常用法。角动量也可以通过经典场来计算。[8][9]: 63  通过应用弗雷德里克·贝林方特计算场角动量的方法,汉斯·C·奥哈尼安证明了“自旋本质上是一种波动特性……由电子波场中的电荷循环流动所产生”。[10] 这一相同的自旋概念也可以应用于水中的重力波:“自旋由水粒子在亚波长尺度上的圆周运动产生”。[11]

与允许连续角动量值的经典波场循环不同,量子波场仅允许离散角动量值。[10] 因此,向自旋态的能量转移或从自旋态的能量转移总是以固定的量子阶跃方式发生。仅有少数阶跃是允许的:在许多定性讨论中,可以忽略自旋量子波场的复杂性,而只考虑系统属性,以“整数”或“半整数”自旋模型来讨论,如下文所述的量子数部分所示。
\subsubsection{狄拉克的相对论电子}
对电子的自旋性质进行定量计算需要使用狄拉克的相对论波动方程。[7]