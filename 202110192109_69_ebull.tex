% 沸腾
% 沸腾|暴沸|水

\textbf{沸腾}是在液体表面及液体内部同时发生的剧烈的汽化现象.

我们用水壶烧水时将看到几个不同的阶段.\textbf{烧到一定程度},可以在水壶底看到一些小气泡积聚在水壶底部.一些小气泡可能会脱离底部上升,但在上升过程中会越来越小直至消失.\textbf{再过一段时间},一些气泡能够到达液面变成很小的空气气泡而破裂,此时能听到“吱吱”的声音.\textbf{再后来},气泡在上升的过程中不断增大而冒出液面,整个液体呈现上下翻滚的剧烈汽化状态,这就是\textbf{沸腾现象}.

要解释沸腾现象,我们需要借助一定热学知识.设液体内部一个半径为 $r$ 的气泡距表面距离为 $h$.设大气压强为 $p_0$,则在这个深度上液体压强为
\begin{equation}
p_0+\rho gh 
\end{equation}