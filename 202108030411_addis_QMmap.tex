% 量子力学导航

\begin{issues}
\issueDraft
\end{issues}

\subsection{量子力学的数学基础:线性代数}
量子力学需要大量的线性代数作为基础. 矢量空间\upref{LSpace}的概念尤为重要. 我们需要明白为什么矩阵可以表示有限维矢量空间之间的线性映射\upref{LinMap}. 矩阵对易与共同本征矢\upref{OpComu} 也非常重要.

需要明白波函数为什么可以看作矢量空间(希尔伯特空间)中的矢量. 需要明白什么是投影\upref{projOp}, 因为量子力学中的测量就是一个投影操作.

若需要考虑单个粒子的自旋问题, 需要明白张量积空间\upref{DirPro}, 空间波函数和自旋态的相乘就是一个张量积, 总态矢存在于张量积空间中.

多粒子的量子力学更加需要使用张量积空间\upref{DirPro}, 多粒子的波函数就是处于这样一个矢量空间中. 全同粒子\upref{IdPar}假设使波函数只能存在于张量积空间中的对称子空间或者反对称子空间.



在量子力学建立以前, 玻尔原子模型\upref{BohrMd}.

首先我们通过一篇量子力学科普\upref{QM0} 简单介绍, 然后为了避免直接解薛定谔方程, 我们先看 “量子力学与矩阵\upref{QMmat}”. 然后通过 “算符和本征问题\upref{QM1}” 进一步明确量子力学的基本假设.
