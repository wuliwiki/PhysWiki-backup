% 力电类比

从上面的讨论可以知道,电磁振荡和机械振动的规律非常相似,所以运用力电类比就可以把电磁振荡和机械振动对应起来,只要知道一种振动的解,就可以用类比方法得到另一种振动的解.虽然机械振动比较直观,但由于电学的迅速发展,人们对交变电路规律的熟悉程度已经超过机械振动.因此,在工程上,常常把复杂的机械振动问题用力电类比方法化成交变电路问题,然后通过计算或实验测定,找出它们的解机械振动和电磁振荡对应的物理量列在下表中.

\begin{table}[ht]
\centering
\caption{机械振动和电磁振荡对应的物理量}\label{MeElec_tab1}
\begin{tabular}{|c|c|c|c|}
\hline {{ 机械振动 }} & & { 电磁振荡( }{ 串联电路 }) \\ \hline { 位移 } & x & { 电荷 } & q \\ \hline { 速度 } & v & { 电流 } & i \\ \hline { 质量 } & m & { 电感 } & L \\ \hline { 劲度系数 } & k & { 电容的倒数 } & \frac{1}{C} \\ \hline { 阻力系数 } & \gamma & { 电阻 } & R & \\ \hline { 驱动力 } F & { 电动势 } & \mathscr{E} & \\ \hline { 弹性势能 } & \frac{1}{2} k x^{2} & { 电场能量 } & \frac{1}{2} \frac{q^{2}}{C} \\ \hline { 动能 } & \frac{1}{2} m v^{2} & { ?场能量 } & \frac{1}{2} L i^{2} \\ \hline
\end{tabular}
\end{table}