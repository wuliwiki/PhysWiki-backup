% 利昂·库珀(综述)
% license CCBYSA3
% type Wiki

本文根据 CC-BY-SA 协议转载翻译自维基百科\href{https://en.wikipedia.org/wiki/Leon_Cooper}{相关文章}。

莱昂·N·库珀(Leon N. Cooper,原名Kupchik;1930年2月28日-2024年10月23日)是美国的理论物理学家和神经科学家。他因在超导性方面的工作获得诺贝尔物理学奖。库珀提出了库珀对的概念,并与约翰·巴丁和约翰·罗伯特·施里弗合作,发展了常规超导性的BCS理论\(^\text{[1][2]}\)。在神经科学领域,库珀共同开发了BCM理论,用以解释突触可塑性\(^\text{[3]}\)。
\subsection{传记}
\subsubsection{童年与教育}
莱昂·N·库珀于1930年2月28日出生在纽约市的布朗克斯区\(^\text{[4]}\)。他的中间名字“N.”并没有特定含义,尽管一些来源错误地认为他的中间名是尼尔\(^\text{[4]}\)。

他的父亲欧文·库珀来自白俄罗斯,在1917年俄国革命后移民到美国。他的母亲安娜(Anna,原姓Zola)库珀来自波兰;在莱昂七岁时去世\(^\text{[4]}\)。父亲在再婚后将家族姓氏从库珀改为库珀\(^\text{[4]}\)。

莱昂就读于布朗克斯科学高中,并于1947年毕业\(^\text{[5][6]}\)。随后,他在位于曼哈顿上城的哥伦比亚大学学习,1951年获得文学学士学位\(^\text{[7]}\)。他继续留在哥伦比亚大学攻读研究生,分别于1953年获得文学硕士学位\(^\text{[7]}\),并于1954年获得哲学博士学位\(^\text{[7][8]}\)。他的博士论文研究了缪子原子,导师是罗伯特·瑟伯\(^\text{[9][10]}\)。
\subsubsection{科学生涯}
库珀在普林斯顿的高等研究院做了一年的博士后研究。之后,他在伊利诺伊大学厄本那-香槟分校和俄亥俄州立大学任教,直到1958年加入布朗大学\(^\text{[8]}\)。他在布朗大学度过了他余下的职业生涯。

库珀于1973年创立了布朗大学的大脑与神经系统研究所,并成为该所的首任主任\(^\text{[7]}\)。1974年,他被任命为布朗大学的科学教授,这个职位由托马斯·J·沃森资助\(^\text{[7]}\)。库珀曾在多个机构担任访问研究职位,包括普林斯顿高等研究院和瑞士日内瓦的欧洲核子研究中心(CERN)。

他与同事查尔斯·埃尔鲍姆于1975年创立了科技公司Nestor,该公司旨在寻找人工神经网络的商业应用[11][12]。Nestor与英特尔合作,开发了1994年发布的Ni1000神经网络计算机芯片\(^\text{[13]}\)。
\subsubsection{个人生活}
\begin{figure}[ht]
\centering
\includegraphics[width=6cm]{./figures/1d902c058aeb9f86.png}
\caption{1972年,库珀与他的妻子凯·阿拉德一起。} \label{fig_LAkb_1}
\end{figure}
库珀第一次与玛莎·肯尼迪结婚,他们有两个女儿。\(^\text{[4]}\)1969年,他与凯·阿拉德结婚。\(^\text{[14]}\)他于2024年10月23日去世,享年94岁,地点是他位于罗德岛普罗维登斯的家中。\(^\text{[4]}\)
\subsection{研究}
\subsubsection{超导性}
\begin{figure}[ht]
\centering
\includegraphics[width=6cm]{./figures/9c5efef362b3409a.png}
\caption{位于伊利诺伊大学的 plaque,纪念BCS超导理论的发展。} \label{fig_LAkb_2}
\end{figure}
在普林斯顿做博士后期间,库珀受到伊利诺伊大学的约翰·巴丁教授和巴丁的研究生约翰·罗伯特·施里弗的接触。巴丁和施里弗正在研究超导性,这对库珀来说是一个全新的课题,但他同意与他们合作。超导性在1911年被实验发现,但当时没有理论解释这一现象。库珀搬到伊利诺伊大学作为博士后,与巴丁一起工作。

经过一年的理论研究,库珀提出了由两个结合的电子组成的准粒子的概念,这个准粒子现在被称为库珀对。库珀在1956年9月在《物理评论》上发表了他的库珀对概念。\(^\text{[4][15]}\)库珀对通过低温金属的运动几乎不受阻碍,从而产生非常低的电阻。经过进一步发展,巴丁、库珀和施里弗展示了如何通过这一机制产生超导性,并在1957年发表了两篇论文,介绍了他们的理论。\(^\text{[4][16][17]}\)这个理论被称为BCS理论,以作者的首字母命名,并被广泛接受为对常规超导性的解释。巴丁、施里弗和库珀因这一理论获得了1972年诺贝尔物理学奖。\(^\text{[4]}\)
\subsubsection{神经科学}
加入布朗大学后,库珀开始对神经科学产生兴趣,特别是学习过程。1982年,库珀与两位博士生埃利·比恩斯坦和保罗·门罗在《神经科学杂志》上发表了他们关于突触可塑性的理论。\(^\text{[4]}\)他们估计了突触的减弱和增强,这些过程可以在不饱和连接的情况下发生。当突触饱和时,电连接变得不那么有效,从而减少饱和度。因此,连接在饱和和不饱和之间振荡,而不达到极限。他们的理论解释了视觉皮层如何工作以及人们如何学会看东西。该理论被称为BCM理论,以作者的首字母命名。\(^\text{[4]}\)
\subsection{会员与荣誉}
\begin{itemize}
\item 美国物理学会会士 
\item 美国艺术与科学学院会士 
\item 美国国家科学院会员
\item 美国哲学学会会员 
\item 美国科学促进会会员 
\item 神经科学研究计划的副会员 
\item 阿尔弗雷德·P·斯隆基金会研究员(1959–1966)
\item 古根海姆研究院会士(1965–66)
\item 1972年诺贝尔物理学奖获得者 [7]
\item 与约翰·罗伯特·施里弗共同获得国家科学院的康斯托克物理学奖(1968)[18]
\item 获哥伦比亚大学研究生院校友卓越奖 
\item 获巴黎科学院笛卡尔奖,雷内·笛卡尔大学
\item 获哥伦比亚学院约翰·杰伊奖(1985年)[7]
\item 获得七个荣誉博士学位 [7]
\end{itemize}
\subsection{出版物}
《科学与人类体验》,论文集(包括此前未发表的材料),探讨意识与空间结构等议题。(剑桥大学出版社,2014年)。

《物理学意义与结构导论》(Harper and Row,1968年)——一本非传统的文科物理教材,后以稍简化的形式再版,书名为《物理学:结构与意义》(Physics: Structure and Meaning,纽罕布什尔黎巴嫩,新英格兰大学出版社,1992年)。

具体论文与出版作品:
\begin{itemize}
\item Cooper, L. N. 与 J. Rainwater 合著,“扩展核的多重库仑散射理论”,哥伦比亚大学 Nevis 回旋加速器实验室,美国海军研究办公室(ONR),美国能源部(前身原子能委员会),1954年8月。
\item Cooper, Leon N. (1956). “简并费米气体中束缚电子对”,《物理评论》104 (4): 1189–1190. Bibcode:1956PhRv..104.1189C. doi:10.1103/PhysRev.104.1189.
\item Bardeen, J.; Cooper, L. N.; Schrieffer, J. R. (1957). “超导微观理论”,《物理评论》106 (1): 162–164. Bibcode:1957PhRv..106..162B. doi:10.1103/PhysRev.106.162.
\item Bardeen, J.; Cooper, L. N.; Schrieffer, J. R. (1957). “超导理论”,《物理评论》108 (5): 1175–1204. Bibcode:1957PhRv..108.1175B. doi:10.1103/PhysRev.108.1175.
\item Cooper, L. N., Lee, H. J., Schwartz, B. B. & W. Silvert. “超导体中奈特位移与磁通量量子化理论”,布朗大学,美国能源部(前身原子能委员会),1962年5月。
\item Cooper, L. N. & Feldman, D. “BCS:50年”,World Scientific Publishing Co.,2010年11月。
\end{itemize}
\subsection{参见}
\begin{itemize}
\item 犹太裔诺贝尔奖获得者名单
\end{itemize}
\subsection{参考文献}
\begin{enumerate}
\item “超导现象”。CERN 官方网站,CERN,2023年7月21日。
\item Weinberg, Steven(2008年2月)。“从 BCS 到 LHC”。《CERN Courier》48 (1): 17–21。
\item Bienenstock, Elie(1982年)。“神经元选择性发展的理论:视觉皮层中的方向特异性和双眼交互”。《神经科学杂志》2 (1): 32–48. doi:10.1523/JNEUROSCI.02-01-00032.1982. PMC 6564292. PMID 7054394。
\item McClain, Dylan Loeb(2024年10月25日)。“Leon Cooper,94岁去世;诺奖得主揭示超导秘密”。《纽约时报》。2024年10月25日检索。
\item “布朗克斯科学高中被美国物理学会授予历史物理学地标荣誉”。bxscience.edu。2012年7月27日检索。
\item MacDonald, Kerri(2010年10月15日)。“一位诺奖得主重返母校布朗克斯科学高中”。《纽约时报》。2012年7月27日检索。
\item “Leon Cooper”。research.brown.edu。2012年7月27日检索。
\item Vanderkam, Laura(2008年7月15日)。“从生物学到物理学再回归:Leon Cooper”。《科学美国人》。2012年7月27日检索。
\item “Cooper, Leon N. (Leon Neil), 1930-”。history.aip.org。2024年11月1日检索。
\item Leon Cooper 于数学谱系项目。
\item Johnson, Colin(1988年10月17日)。“神经网络初创公司在美国迅速扩张”。《科学家》2 (19)。2018年3月8日检索。
\item Garson, G. David(1998年9月28日)。《神经网络:社会科学家的入门指南》。SAGE 出版社。ISBN 978-0-7619-5730-0。
\item “Nestor 的神经网络芯片命运现已掌握在自己手中”。《Tech Monitor》,1994年4月14日。2022年10月20日检索。
\item Carey, Charles W.(2014年)。《美国科学家》。Infobase 出版社,第66页。ISBN 978-1-4381-0807-0。
\item Cooper, Leon(1956年11月)。“简并费米气体中的束缚电子对”。《物理评论》104 (4): 1189–1190. Bibcode:1956PhRv..104.1189C. doi:10.1103/PhysRev.104.1189. ISSN 0031-899X。
\item Bardeen, J.; Cooper, L. N.; Schrieffer, J. R.(1957年4月)。“超导微观理论”。《物理评论》106 (1): 162–164. Bibcode:1957PhRv..106..162B. doi:10.1103/PhysRev.106.162。
\item Bardeen, J.; Cooper, L. N.; Schrieffer, J. R.(1957年12月)。“超导理论”。《物理评论》108 (5): 1175–1204. Bibcode:1957PhRv..108.1175B. doi:10.1103/PhysRev.108.1175。
\item “康斯托克物理奖”。美国国家科学院。原始存档于2010年12月29日。
\item Cushing, James T.(1978年)。“《物理学意义与结构导论》书评,作者 Leon N. Cooper”。《美国物理学杂志》46 (1): 114–115. Bibcode:1978AmJPh..46..114C. doi:10.1119/1.11116。
\end{enumerate}
