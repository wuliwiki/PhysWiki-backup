% 渐进近似(量子力学)
% license Usr
% type Tutor


\begin{issues}
\issueDraft
\end{issues}

\footnote{参考 \cite{GriffE} 章节: The Adiabatic Approximation}若系统初始时处于某个离散非简并的本征态,那么当哈密顿量 $H$ 随时间缓慢改变时(改变的特征时间远大于本征态的周期), 那改变过程中波函数将仍然处于该本征态,除了一个相位因子的改变。

渐进近似下含时薛定谔方程的通解是
\begin{equation}\label{eq_AdiaQM_2}
\Psi(t) = \sum_n c_n(0) \psi_n(t) \E^{\I\gamma_n(t)}\exp\qty{-\I\frac{1}{\hbar} \int_0^t E_n(t')\dd{t'}}~.
\end{equation}
其中
\begin{equation}
\gamma_m = \I \int_0^t \braket{\psi_m(t')}{\dot\psi_m(t')}\dd{t'}~.
\end{equation}

也就是说,若 $t=0$ 时波函数是若干非简并离散本征态的线性组合, 那么每个系数的模长不会发生改变。

\begin{example}{}
\begin{enumerate}
\item 当无限深势阱\upref{ISW}缓慢变长。
\item 量子简谐振子(升降算符法)\upref{QSHOop}的劲度系数 $k$ 缓慢变化。
\item 给氢原子的任意束缚态 $\psi_{n,l,m}$ 缓慢施加外电场或磁场(未完成:简并情况如何处理?)(参考 “类氢原子斯塔克效应(微扰)\upref{HStark}”)。
\end{enumerate}
\end{example}

\subsection{推导}
若哈密顿量不随时间改变,
\begin{equation}
\Psi_n(t) = \psi_n \E^{-\I E_n t}~.
\end{equation}
若随时间改变, 本征态和本征值都变为时间的函数 $\psi_n(t)$ 和 $E_n(t)$。 但仍然正交归一。 此时的含时波函数仍然可以用它们展开
\begin{equation}
\Psi(t) = \sum_n c_n(t) \psi_n(t) \E^{\I \theta_n(t)}~,
\end{equation}
其中
\begin{equation}
\theta_n(t) = -\frac{1}{\hbar} \int_0^t E_n(t')\dd{t'}~.
\end{equation}
代入含时薛定谔方程
\begin{equation}\label{eq_AdiaQM_1}
H(t)\Psi(t) = \I \dot \Psi(t)~,
\end{equation}
得
\begin{equation}
\dot c_m(t) = -c_m \braket*{\psi_m}{\dot\psi_m} - \sum_{n\ne m} c_n \frac{\mel*{\psi_m}{\dot H}{\psi_n}}{E_n-E_m} \exp\qty{{-\frac{\I}{\hbar}\int_0^t[E_n(t')-E_m(t')]\dd{t'}}}~
\end{equation}
现在还没有使用任何近似。 adiabatic 的近似就在于假设 $\dot H$ 非常小,从而忽略该式中第二项。 那么有
\begin{equation}
c_m(t) = c_m(0)\E^{\I\gamma_m(t)}~,
\end{equation}
其中
\begin{equation}
\gamma_m = \I \int_0^t \braket{\psi_m(t')}{\dot\psi_m(t')}\dd{t'}~.
\end{equation}
这就得到了\autoref{eq_AdiaQM_2}。
