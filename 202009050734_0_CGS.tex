% 厘米—克—秒单位制
% keys 厘米|克|秒|单位|量纲|转换|国际单位|达因

\begin{issues}
\issueDraft
\end{issues}

\pentry{物理量和单位转换\upref{Units}, 国际单位制\upref{SIunit}}

\footnote{参考 Wikipedia \href{https://en.wikipedia.org/wiki/Centimetre-gram-second_system_of_units}{相关页面}.}\textbf{厘米—克—秒单位制(centimetre–gram–second system of units)}也简记为 CGS 单位制. 我们将从 CGS 到 SI 单位制之间的转换常数记为 $\beta_\text{物理量}$, CGS 单位制的物理量符号用角标 $c$ 加以区分. 例如 $x = \beta_x x_c$ 其中 $x$ 是国际单位制的物理量, $x_c$ 是 CGS 单位的物理量. 大部分情况下, $\beta_\text{物理量} = 1$

则 $\beta_x = 1\Si{cm}$, $\beta_m = 1\Si{g}$, $\beta_t = 1\Si{s}$.
为了满足 $a = \ddot x$, 有
\begin{equation}
\beta_a = \beta_x/\beta_s^2 = 1\Si{cm/s^2} = 0.01 \Si{m/s}
\end{equation}
为了满足牛顿定律
\begin{equation}
F = ma
\end{equation}
代入得
\begin{equation}
\beta_F = \beta_m \beta_x/\beta_s^2 = 1\Si{g \cdot cm/s^2} = 1\Si{dyn} = 1\times 10^{-5} \Si{N}
\end{equation}
其单位叫做\textbf{达因(dyne)}, 记作 $\Si{dyn}$.
