% 泊松方程(综述)
% license CCBYSA3
% type Wiki

本文根据 CC-BY-SA 协议转载翻译自维基百科\href{https://en.wikipedia.org/wiki/Poisson\%27s_equation}{相关文章}。
\begin{figure}[ht]
\centering
\includegraphics[width=6cm]{./figures/8fc5796ba24af0b4.png}
\caption{西蒙·丹尼斯·泊松} \label{fig_BSFC_1}
\end{figure}
泊松方程是一个在理论物理中广泛应用的椭圆型偏微分方程。例如,泊松方程的解是由给定的电荷或质量密度分布引起的势场;一旦知道了势场,就可以计算出相应的静电或引力(力)场。它是拉普拉斯方程的推广,后者在物理学中也经常出现。该方程以法国数学家和物理学家西蒙·丹尼斯·泊松的名字命名,泊松于1823年发布了这个方程。\(^\text{[1][2]}\)
\subsection{方程的表述}  
泊松方程是  
\[
\Delta \varphi = f,~
\]
其中,\(\Delta\)是拉普拉斯算子,\(f\)和\(\varphi\)是流形上的实值或复值函数。通常,给定\(f\),而求解\(\varphi\)。当流形是欧几里得空间时,拉普拉斯算子通常表示为\(\nabla^2\),因此泊松方程通常写作  
\[
\nabla^2 \varphi = f.~
\]
在三维笛卡尔坐标系中,它的形式为  
\[
\left( \frac{\partial^2}{\partial x^2} + \frac{\partial^2}{\partial y^2} + \frac{\partial^2}{\partial z^2} \right) \varphi (x, y, z) = f (x, y, z).~
\]
当\(f = 0\)恒成立时,我们得到拉普拉斯方程。
泊松方程可以通过格林函数求解:  
\[
\varphi(\mathbf{r}) = - \iiint \frac{f(\mathbf{r'})}{4 \pi |\mathbf{r} - \mathbf{r'}|} \, \mathrm{d}^3 r',~
\]
其中积分是对整个空间进行的。泊松方程的格林函数的详细说明见于筛选泊松方程的相关文章。还有多种数值求解方法,如松弛法,这是一种迭代算法。
\subsection{物理和工程中的应用}  
\subsubsection{牛顿引力}   
在由于吸引的质量物体密度为\(\rho\)的引力场\(g\)的情况下,可以使用引力的高斯定律的微分形式来得到相应的引力泊松方程。引力的高斯定律为  
\[
\nabla \cdot \mathbf{g} = -4\pi G \rho.~
\]
由于引力场是保守的(且无旋的),它可以用标量势\(\varphi\)来表示:  
\[
\mathbf{g} = - \nabla \varphi.~
\]
将这一表达式代入高斯定律,得到  
\[
\nabla \cdot (-\nabla \varphi) = -4\pi G \rho,~
\]  
从而得到引力的泊松方程:  
\[
\nabla^2 \varphi = 4\pi G \rho.~
\]
如果质量密度为零,泊松方程就化简为拉普拉斯方程。对应的格林函数可以用来计算从中心点质量\(m\)距离\(r\)处的势(即基本解)。在三维中,势为  
\[
\varphi (r) = \frac{-Gm}{r},~
\]  
这等价于牛顿的万有引力定律。
\subsubsection{静电学}
许多静电学中的问题由泊松方程所支配,该方程将电势\(\varphi\)与自由电荷密度\(\rho_f\)联系起来,这在导体等系统中尤为常见。

泊松方程的数学细节通常采用国际单位制(SI单位,区别于高斯单位制)来表达,描述了自由电荷的分布如何在某一区域内产生静电势。

从电学的高斯定律(也是麦克斯韦方程组之一)的微分形式出发,有:\(\nabla \cdot \mathbf{D} = \rho_f\),其中\(\nabla \cdot\)是散度算子,\(\mathbf{D}\)是电位移场,\(\rho_f\)是自由电荷密度(表示从外部引入的电荷)。

假设介质是线性、各向同性且均匀的(见极化密度),我们有本构关系:\(\mathbf{D} = \varepsilon \mathbf{E}\),其中\(\varepsilon\)是介质的介电常数,\(\mathbf{E}\)是电场强度。

将此代入高斯定律,并假设介电常数\(\varepsilon\)在感兴趣的区域内是空间常数,可以得到\(\nabla \cdot \mathbf{E} = \frac{\rho_f}{\varepsilon}\).在静电学中,我们假设不存在磁场(以下推导在存在恒定磁场时也成立)\(^\text{[3]}\)。因此,我们有\(\nabla \times \mathbf{E} = 0\),其中\(\nabla \times\)是旋度算子。这个方程意味着我们可以将电场表示为标量函数\(\varphi\)(称为电势)的梯度,因为任何梯度的旋度为零。因此,我们可以写作\(\mathbf{E} = -\nabla \varphi\), 
引入负号是为了将\(\varphi\)识别为单位电荷的电势能。在这种情况下,泊松方程的推导是直接的。将电场的势梯度代入,得到
\[
\nabla \cdot \mathbf{E} = \nabla \cdot (-\nabla \varphi) = -\nabla^2 \varphi = \frac{\rho_f}{\varepsilon},~
\]
直接得出静电学中的泊松方程:\(\nabla^2 \varphi = -\frac{\rho_f}{\varepsilon}\).

指定泊松方程来求解势函数需要知道电荷密度分布。如果电荷密度为零,则得到拉普拉斯方程。如果电荷密度遵循玻尔兹曼分布,则得到泊松-玻尔兹曼方程。泊松-玻尔兹曼方程在发展德拜-赫克尔稀溶液电解质理论中起着重要作用。

使用格林函数,从一个中心点电荷\(Q\)距离\(r\)处的势(即基本解)为  
\[
\varphi(r) = \frac{Q}{4\pi \varepsilon r},~
\]  
这就是库仑定律的静电学表达式。(出于历史原因,并且与上面引力模型不同,\(4\pi\)因子出现在这里,而不是高斯定律中。)

上述讨论假设磁场不随时间变化。即使磁场随时间变化,只要使用库仑规,仍然会得到相同的泊松方程。在这个更一般的情况下,计算\(\varphi\)不再足以计算电场\(\mathbf{E}\),因为\(\mathbf{E}\)还依赖于磁矢势\(\mathbf{A}\),它必须独立计算。有关\(\varphi\)和\(\mathbf{A}\)在麦克斯韦方程中的应用以及如何在这种情况下得到适当的泊松方程的更多内容,请参见麦克斯韦方程的势函数形式。

\textbf{高斯电荷密度的势}

如果存在一个静态的球对称高斯电荷密度  
\[
\rho_f(r) = \frac{Q}{\sigma^3 (2\pi)^{3}}\, e^{-r^2 / (2\sigma^2)},~
\]  
其中\(Q\)是总电荷,那么泊松方程  
\[
\nabla^2 \varphi = -\frac{\rho_f}{\varepsilon}~
\]  
的解为  
\[
\varphi(r) = \frac{1}{4\pi \varepsilon} \cdot \frac{Q}{r} \cdot \operatorname{erf} \left( \frac{r}{\sqrt{2}\sigma} \right),~
\]  
其中\(\operatorname{erf}(x)\)是误差函数\(^\text{[5]}\)。该解可以通过直接计算\(\nabla^2 \varphi\)来验证。

需要注意的是,当\(r\)远大于\(\sigma\)时,\(\operatorname{erf}(r / \sqrt{2} \sigma)\)趋近于 1,[6] 这时电势\(\varphi(r)\)会趋近于点电荷的电势形式:
\(\varphi \approx \frac{1}{4\pi \varepsilon} \cdot \frac{Q}{r}\),这是符合预期的结果。此外,误差函数随着其自变量的增大而迅速接近 1;在实际应用中,当\(r > 3\sigma\)时,相对误差已小于千分之一。\(^\text{[6]}\)
\subsubsection{表面重建}
表面重建是一个反问题。目标是基于大量的点\(p_i\)(点云),以及每个点携带的局部表面法线估计\(n_i\),数字化重建一个光滑的表面。\(^\text{[7]}\)泊松方程可以通过一种称为泊松表面重建的技术来解决这个问题。\(^\text{[8]}\)

该技术的目标是重建一个隐式函数\(f\),该函数在点\(p_i\)处的值为零,并且在这些点的梯度等于法线向量\(n_i\)。因此,点集\((p_i, n_i)\)被建模为一个连续的向量场\(V\)。隐式函数\(f\)通过对向量场\(V\)进行积分来求得。由于并不是每个向量场都是某个函数的梯度,问题可能有解,也可能没有解:一个光滑的向量场\(V\)是某个函数\(f\)的梯度的充要条件是\(V\)的旋度必须恒为零。如果这个条件很难强加,仍然可以执行最小二乘拟合,以最小化\(V\)和\(f\)梯度之间的差异。

为了有效地将泊松方程应用于表面重建问题,需要找到向量场\(V\)的良好离散化。基本方法是用有限差分网格来约束数据。对于在该网格节点上的函数,其梯度可以表示为在交错网格上赋值,即在原始网格的节点之间的网格上。定义三个交错网格是方便的,每个网格在一个方向上平移,分别对应法线数据的分量。在每个交错网格上,我们对点集执行三线性插值。然后,插值权重用于将相关分量\(n_i\)的大小分配到包含\(p_i\)的特定交错网格单元的节点上。Kazhdan 和合著者提出了一种更精确的离散化方法,使用自适应有限差分网格,即在数据点更多的地方,网格的单元更小(网格被更细地划分)。\(^\text{[8]}\)他们建议使用自适应八叉树来实现这一技术。
\subsubsection{流体动力学}
对于不可压缩的Navier–Stokes方程,给定为  
\[
\frac{\partial \mathbf{v}}{\partial t} + (\mathbf{v} \cdot \nabla) \mathbf{v} = -\frac{1}{\rho} \nabla p + \nu \Delta \mathbf{v} + \mathbf{g},~
\]
\[
\nabla \cdot \mathbf{v} = 0.~
\]
压力场\(p\)的方程是非线性泊松方程的一个例子:  
\[
\Delta p = -\rho \nabla \cdot (\mathbf{v} \cdot \nabla \mathbf{v}) = -\rho \operatorname{Tr} \left( (\nabla \mathbf{v})(\nabla \mathbf{v}) \right).~
\]  
请注意,上述迹不是符号确定的。
\subsection{另见}  
\begin{itemize}
\item 离散泊松方程  
\item 泊松-玻尔兹曼方程  
\item 赫尔姆霍兹方程  
\item 泊松方程的唯一性定理  
\item 弱形式  
\item 调和函数  
\item 热方程  
\item 势理论
\end{itemize}
\subsection{参考文献}  
\begin{enumerate}
\item Jackson, Julia A.; Mehl, James P.; Neuendorf, Klaus K. E., 编著 (2005), Glossary of Geology, 美国地质学会, 斯普林格, 第503页, ISBN 9780922152766  
\item Poisson (1823). "Mémoire sur la théorie du magnétisme en mouvement" [关于运动中的磁学理论的回忆].法国科学院学报(法文),第6卷:441–570。第463页:“因此,根据上述内容,我们最终将得到:  
\[
\frac{\partial^2 V}{\partial x^2} + \frac{\partial^2 V}{\partial y^2} + \frac{\partial^2 V}{\partial z^2} = 0, = -2k\pi, = -4k\pi,~
\]
取决于点 M 是否位于我们所考虑的体积的外部、表面或内部。”(因此,根据前述内容,我们最终将得到:  
\[
\frac{\partial^2 V}{\partial x^2} + \frac{\partial^2 V}{\partial y^2} + \frac{\partial^2 V}{\partial z^2} = 0, = -2k\pi, = -4k\pi,~
\]
取决于点 M 是位于外部、体积表面还是体积内部。)  
\(V\) 定义为(第462页):  
\[
V = \iiint \frac{k'}{\rho} \, dx' \, dy' \, dz',~
\]
其中,在静电学的情况下,积分是在带电物体的体积上进行的,带电物体内部或表面上的点的坐标用 \((x', y', z')\) 表示,\(k'\) 是 \( (x', y', z') \) 的已知函数,在静电学中,\(k'\) 将是电荷密度的度量,\(\rho\) 定义为从点 M 到带电物体内部或表面上某一点的半径长度。点 M 的坐标用 \((x, y, z)\) 表示,\(k\) 表示点 M 处的 \(k'\) 值(电荷密度)。
\item 格里菲斯, D. J. (2017). 《电动力学导论》(第四版). 剑桥大学出版社. 第77–78页.  
\item 格里菲斯, D. J. (2017). 《电动力学导论》(第四版). 剑桥大学出版社. 第83–84页.  
\item 萨勒姆, M.; 阿尔达巴格, O. (2024). "轴对称高斯电荷密度分布下静电特性的泊松方程数值解". 《数学》. 12 (13): 1948. doi:10.3390/math12131948.  
\item 奥尔德姆, K. B.; 迈兰, J. C.; 斯帕尼尔, J. (2008). "误差函数 erf(x) 及其补函数 erfc(x)". 《函数图集》. 纽约: Springer. 第405–415页. doi:10.1007/978-0-387-48807-3_41. ISBN 978-0-387-48806-6.  
\item 卡拉克利, 法提赫; 陶宾, 加布里埃尔 (2011). "平滑符号距离曲面重建" (PDF). 《太平洋图形学》. 30 (7).  
\item 卡日丹, 迈克尔; 博利索, 马修; 霍普, 雨果 (2006). "泊松曲面重建". 《第四届欧洲图形学几何处理研讨会论文集》(SGP '06). 瑞士: Eurographics协会. 第61–70页. ISBN 3-905673-36-3.
\end{enumerate}
\subsection{扩展阅读}
\begin{itemize}
\item 埃文斯, 劳伦斯·C. (1998). 《偏微分方程》. 普罗维登斯: 美国数学学会. ISBN 0-8218-0772-2.  
\item 马修斯, 乔恩; 沃克, 罗伯特·L. (1970). 《物理学中的数学方法》(第二版). 纽约: W. A. 本杰明. ISBN 0-8053-7002-1.  
\item 波利亚宁, 安德烈·D. (2002). 《工程师与科学家线性偏微分方程手册》. 博卡拉顿: CRC出版社. ISBN 1-58488-299-9.
\end{itemize}
\subsection{外部链接}
\begin{itemize}
\item "泊松方程", 《数学百科全书》, EMS出版社, 2001 [1994]  
\item 泊松方程在EqWorld(数学方程世界)的条目: http://eqworld.ipmnet.ru/en/solutions/lpde/lpde302.pdf
\end{itemize}
