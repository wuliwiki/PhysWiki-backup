% 简谐振子(经典力学)
% 弹簧|简谐运动|胡克定律|二阶微分方程|通解

% 未完成:讲述太粗糙
\pentry{胡克定律, 牛顿第二定律\upref{New3}}


\subsection{简谐振子及其运动方程}

\begin{figure}[ht]
\centering
\includegraphics[width=6cm]{./figures/SHO_1.pdf}
\caption{简谐振子模型} \label{SHO_fig1}
\end{figure}

如\autoref{SHO_fig1}, 质量为 $m$ 的质点固定在弹性系数为 $k$ 的弹簧的一端,弹簧另一端固定,忽略弹簧的质量, 任何摩擦以及重力.

在 $t = 0$ 时,若质点不在平衡位置,或者有一个初速度,则接下来会发生振动. 以质点拉伸弹簧的方向为 $x$ 轴正方向,质点的平衡位置为 $x = 0$. 当质点在位置 $x$ 时,根据胡克定律,受力为 $F =  - kx$. 根据牛顿第二定律\upref{New3} $F = ma = m\ddot x$ ( $\ddot x$ 代表对时间的二阶导数).  两式消去 $F$, 得
\begin{equation}\label{SHO_eq1}
m\ddot x =  - kx
\end{equation}
这是一个单自变量\textbf{常微分方程\upref{ODE}}. 我们以弹簧振子为代表,将一切形如\autoref{SHO_eq1} 的方程称为\textbf{简谐振子方程},或者说将一切能量与位置关系满足胡克定律的系统称为简谐振子\footnote{如\textbf{量子简谐振子}\upref{QSHOop}.}.


% 注释原因:没必要专门讲,引用微分方程知识即可.读半天就学了个一种特殊的简单方程,还把思路打断了.
% 由于上式中最高阶导数是二阶,所以叫做\textbf{二阶微分方程}. 要解该方程,就是要寻找一个函数 $x(t)$, 使它的二阶导数与 $- x(t)$ 成正比,比例系数为 $k/m$. 注意到 $\cos'' t =  - \cos t$ 具有类似的性质\footnote{$\sin t$ 也有同样的性质, 所以以下讨论对 $\sin t$ 也成立},不妨继续猜测 $x = \cos(\omega t)$, 则 $\ddot x =  - {\omega ^2}\cos \omega t$. 所以只要令 $\omega = \sqrt{k/m}$ 即可满足方程. 这说明,弹簧的振动可以用余弦函数来描述.但是这只是方程的一个解. 任意情况的振动可以表示为以下函数(令 $A$ 和 $\phi_0$ 为两个任意实数)

微分方程\autoref{SHO_eq1} 的\textbf{通解}表示为:

\begin{equation}\label{SHO_eq2}
x = A\cos(\omega t + \phi_0)  \qquad \qty(\omega  = \sqrt{k/m})
\end{equation}
即无论常数 $A, \phi_0$ 取任意值,形如\autoref{SHO_eq2} 的函数都满足\autoref{SHO_eq1} ,且任意一个满足\autoref{SHO_eq1} 的函数必取\autoref{SHO_eq2} 的形式.

\autoref{SHO_eq1} 的解法可参考\textbf{二阶常系数齐次微分方程的通解}\upref{Ode2},或系统学习\textbf{常微分方程}一章,其中解\autoref{SHO_eq1} 的方法见\textbf{常系数线性齐次微分方程}\upref{ODEb2},是其中一个简单的特例.

像\autoref{SHO_eq2} 这样满足正弦(余弦)规律的运动叫做\textbf{简谐运动(或简谐振动)}.其中 $A$ 为\textbf{振幅(amplitude)}, $\omega t + \phi_0$ 为\textbf{相位(phase)}, $\phi_0$ 为\textbf{初相位(initial phase)}(即 $t = 0$ 时刻的相位). 

但是如何决定 $A$ 和 $\phi_0$ 呢? 根据上面给出的条件还不能判断. 由于有两个待定常数, 我们需要两个额外条件才能解出. 常见的情况是给出初始时刻 $t = 0$ 时质点的位置 $x(0)$ 和速度 $\dot x(0)$ , 这就叫做\textbf{初值条件}.

例如给出 $x(0) = 0$,  $\dot x(0) = v_0$, 把方程的通解代入, 得 $A\cos \phi_0 = 0$,  $ - A\omega \sin \phi_0 = v_0$, 解得 $\phi_0 = \pi /2$,  $A =  -v_0\omega $. 所以
\begin{equation}
x =  - v_0\omega \cos (\omega t + \frac{\pi }{2}) = v_0\omega \sin \omega t \qquad \qty(\omega  = \sqrt{k/m})
\end{equation}


\subsection{其它简谐振子模型}

\subsubsection{竖直的弹簧振子}

将\autoref{SHO_fig1} 中的弹簧振子竖直放置,忽略弹簧质量,那还是简谐振子吗?答案是肯定的,这是因为胡克定律是\textbf{线性}的.下面是具体讨论.

设原点仍然定义为弹簧处于原长时的末端,那么小球的运动方程为:
\begin{equation}\label{SHO_eq3}
m\ddot{x} = -kx+mg
\end{equation}
作变量代换$x=y+\frac{mg}{k}$,代回\autoref{SHO_eq3} 即得
\begin{equation}\label{SHO_eq4}
m\ddot{y} = m\ddot{x} = -kx+mg = -ky
\end{equation}
\autoref{SHO_eq4} 就和\autoref{SHO_eq1} 一模一样,只是把$x$换成了$y$.

由此可见,如果小球所受重力为$mg$,那么我们只需要把原点定义为比弹簧原长的位置还长$\frac{mg}{k}$,那整个系统依然是一个平衡点在原点处的弹簧振子.在这个平衡点,重力和弹簧拉力平衡.


\subsubsection{小角度单摆}

\autoref{SHO_fig3} 所示shi

\begin{figure}[ht]
\centering
\includegraphics[width=10cm]{./figures/SHO_3.pdf}
\caption{请添加图片描述} \label{SHO_fig3}
\end{figure}

摆锤


\subsection{能量}

\begin{figure}[ht]
\centering
\includegraphics[width=6cm]{./figures/SHO_2.pdf}
\caption{简谐振子势能曲线} \label{SHO_fig2}
\end{figure}

简谐振子的总能量等于质点动能加弹簧的弹性势能. 当位移最大时, 动能为 0, 总能量等于势能, 位移为 0 时势能为 0, 总能量等于动能
\begin{equation}
E = \frac{1}{2} mv^2 + \frac12 k x^2 = \frac12 k A^2 = \frac12 m v_0^2
\end{equation}
其中 $v_0$ 是质点经过原点处的速度,即最大速度;$A$是质点速度为零时到原点的距离,即最大距离.


\subsection{频率}

简谐振子作周期性振动,可以计算其振动频率.振动频率的定义是单位时间内完成的周期数量,即完成一个周期所需时间的倒数.

假设一个简谐振子的运动方程为
\begin{equation}
x = A\cos(\omega t+\phi_0)
\end{equation}
其周期为$T$,那么由于余弦函数的周期是$2\pi$,可知
\begin{equation}
\omega T = 2\pi
\end{equation}
于是简谐振子的周期为
\begin{equation}
T = \frac{2\pi}{\omega}
\end{equation}
从而可得其频率
\begin{equation}
\nu = \frac{1}{T} = \frac{\omega}{2\pi}
\end{equation}
由\autoref{SHO_eq2} 知,$\omega=\sqrt{k/m}$,因此给定模型的频率为
\begin{equation}
\nu = \frac{1}{2\pi}\sqrt{\frac{k}{m}}
\end{equation}














