% 厦门大学 2013 年 考研 量子力学
% license Usr
% type Note

\textbf{声明}:“该内容来源于网络公开资料,不保证真实性,如有侵权请联系管理员”

\subsection{一 、}

(1)设$\varphi(x)_1, \varphi(x)_2$是体系的两个可能状态,有下面三种线性
叠加:

① $\phi_A = \varphi(x)_1 + \varphi(x)_2  e^{i\delta}$ 

② $\phi_B= \varphi(x)_1 + \varphi(x)_2 $

③ $\phi_C =e^{i\delta} (\varphi(x)_1 + \varphi(x)_2)$

式中$\delta$为实常数($\delta \ne 2n\pi$),问$\phi_A ,\phi_B , \phi_C$ 是否表示相同的态

(2)什么是厄米算符?它具有什么特征使得可观测量需要有厄米算符表示?试证明动量地 $x$ 分量是厄米算符。

(3)若体系的波函数为$Y_{lm} (\theta, \varphi)(l\ne 0)$,求其轨道角动量矢量与$z$轴的夹角。

(4)若两个力学量算符 $F$ 与 $G$ 的对易关系为$[F,G]=ik$,试写出 $F$与 $G$ 的测不准关系式,有哪些要求。

(5)粒子的总能量 $E=T+V$,若微观粒子处于经典禁区($E$ 小于 $V$),这是否意味着 $T$ 小于 0?为什么?
\subsection{二、}
设力学量算符 $A$ 与体系的哈密顿算符 $H$ 不对易,已知 $A$ 有两个本征态$\Psi_1,\Psi_2 $ (相应的本征值为$A_1 , A_2$  )。$\Psi_1=\frac{\phi_1+\phi_2}{\sqrt{2}},\Psi_1=\frac{\phi_1-\phi_2}{\sqrt{2}}$这里$\phi_1,\phi_2$ 为 $H$ 的归一化本征态(相应的本征值为$E_1 , E_2$ ).设体系的初始态为$\Psi(0)=\Psi_1$  求:

(1)$t$ 时刻($t$ 大于 0)体系的波函数$\Psi(t)$;

(2)将$\Psi(t)$展开为 $A$ 的本征态的叠加;

(3)求出 $t$ 时刻($t$ 大于 0)$A$ 的平均值$\overline A(t)$.
\subsection{三、}
一刚性转子转动惯量为 $I$,它的能量经典表示式为 $H=\frac{L^2}{2I}$,其中
$L$ 为轨道角动量。求与此相应的量子体系在下列情况下的定态能
量及波函数:

(1)转子绕一固定轴($z$ 轴)转动;

(2) 转子绕一固定点转动;
\subsection{四、}
一个质量为$m$ 的粒子处在如下一维势场$V_1(X) = \frac{K}{2}X^2(k \text{大于} 0)$的基态。

(1)若弹性系数 $k$ 突然变为 $2k$,即势场变为$V_2(X) = \frac{K}{2}X^2$ ,立即
测量粒子的能量,求发现粒子处于新的势场$V_2(X)$基态的概率;

(2)势场突然由$V_1(X)$变成$V_2(X)$后不进行测量,经过一段时间τ
后,势场又恢复为$V_1(X)$,问$\tau$取何值时可以恢复到原来势场$V_1(X)$
的基态。
\subsection{五、}
自旋投影算符定义$\vec{S_n}=\frac{\hbar}{2}\vec{\sigma}\cdot \vec{n}$,其中$\vec{\sigma}$为泡利矩阵 ,$\vec{n}(\sin\phi\cos\varphi,\sin\phi\sin\varphi,\cos\phi)$为$(\phi,\varphi)$方向上的单位矢量。
求:

(1)求$\vec{S_n}$的本征值与本征态;

(2)对电子自旋向上的态$X_+(S_z=\frac{\hbar}{2})$,求$\vec{S_n}$的可能测量值以及相应的概率;

(3)在$\sigma_z$表象中求$\sigma_y$的本征值与本征态;

(4)$\vec{\sigma_n}=\vec{\sigma}\cdot\vec{n}$本征值为 1 的本征态,求$\sigma_y$ 的可能测量值以及相应的概率;
\subsection{六、}
一质量为$m$ 的粒子在二维无限深势阱中运动
$$V(x,y)=\begin{cases}
0,&0 < x, y < a\\\\
\infty ,& \text{其余区域}
\end{cases}~
$$
设加上微扰$H^\prime=\Lambda xy(0 < x, y < a)$

求:

(1)不考虑微扰时粒子的能量和波函数;

(2)不考虑微扰时粒子的基态及第一激发态能量是否简并;

(3)基态能量的一级修正;

(4)第一激发态能量的一级修正.