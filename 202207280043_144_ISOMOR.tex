% 匀晶相图

\subsection{匀晶转变}

\begin{figure}[ht]
\centering
\includegraphics[width=14cm]{./figures/ISOMOR_1.png}
\caption{匀晶相图} \label{ISOMOR_fig1}
\end{figure}

匀晶转变(\autoref{ISOMOR_fig1} 中b-d段)指“由单一液相直接生成单一固相 l→s”的过程.以下以Ni-Cu合金为例,介绍匀晶转变
\begin{figure}[ht]
\centering
\includegraphics[width=5cm]{./figures/ISOMOR_2.png}
\caption{请添加图片描述} \label{ISOMOR_fig2}
\end{figure}
匀晶转变不是恒成分转变.当α相刚开始生成时,高熔点组分的浓度高于液相,随后随着α相不断长大,该组分的浓度逐渐降低.例如,Ni熔点高于Cu,因此新生成的相α中Ni的浓度更高

匀晶转变不是恒温转变:f=2-2+1=1,系统有一个自由度

固液共存时,各相的比例:杠杆定律

