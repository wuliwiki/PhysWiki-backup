% 矩阵的本征方程
% 矩阵|本征问题|本征矢|本征值|线性方程组|对角线

\pentry{线性方程组与矢量空间\upref{LinEq}}

若已知矩阵 $\mat A$, 我们把线性方程组
\begin{equation}\label{MatEig_eq1}
\mat A \bvec v = \lambda \bvec v
\end{equation}
称为矩阵 $\mat A$ 的\textbf{本征方程}. 式中 $\mat A$ 是已知的, 而 $\lambda$ 和 $\bvec v$ 是未知的. 显然, 当 $\bvec v = \bvec 0$ 时方程恒成立, 所以我们通常只对非零解感兴趣. 也就是说, 我们希望找到一些非零矢量 $\bvec v$, 使得矩阵 $\mat A$ 乘以该矢量以后方向不变\footnote{“方向” 只是从几何矢量\upref{GVec}中沿用过来的一个习惯说法, 注意\autoref{MatEig_eq1} 中的所有量都可以是复数. 两个矢量方向相同意味着一个矢量乘以标量(包括复数)可以得到另一个.}. 对于每个这样的矢量, 我们用一个标量 $\lambda$ 来描述其模长的改变. 我们把这些矢量叫做\textbf{本征矢(eigen vector)}, 把对应的 $\lambda$ 叫做\textbf{本征值(eigen value)}.

\subsubsection{几何意义}
几何上来讲, 实数矩阵对应的线性变换相当于把坐标网格做旋转、拉伸、翻折操作.% 链接未完成
所以一般而言, 一个矢量在变换后长度和方向都会改变. 但也可能存在一些特殊的矢量, 使得变换后只可能改变长度而不改变方向. 这些矢量就是本征方程的解.

\subsection{求解本征方程}

若令 $\mat I$ 为 $N\times N$ 的单位矩阵\footnote{即对角线上的元为 1, 其他元为 0, 见“矩阵\upref{Mat}”}, 则本征方程本质上是一个齐次方程组
\begin{equation}\label{MatEig_eq2}
(\mat A - \lambda\mat I)\bvec v = \bvec 0
\end{equation}
括号中的矩阵相当于把矩阵 $\mat A$ 的对角线上的元都减去 $\lambda$ 得到的方阵. 要确保方程有非零解, 只需令系数矩阵 $\mat A - \lambda\mat I$ 不是满秩的, 即行列式为零
\begin{equation}
\abs{\mat A - \lambda\mat I} = 0
\end{equation}
这是一个关于 $\lambda$ 的 $N$ 阶多项式, 称为\textbf{特征多项式(characteristic polynomial)}. 特征多项式必存在 $N$ 个复数根(包括重根),% 链接未完成
记为 $\lambda_i$ ($i = 1, 2\dots N$). 将它们依次代入\autoref{MatEig_eq2}, 就可以分别解出对应的本征矢. 考虑到\autoref{MatEig_eq2} 是一个齐次方程, 所以 $\mat A - \lambda_i\mat I$ 的零空间中所有矢量都是本征矢, 且零空间至少是一维的. 我们把这个空间叫做 $\lambda_i$ 的\textbf{本征矢空间}, 是 $\bvec v$ 所在的矢量空间的子空间.

令 $\lambda_i$ 的本征矢空间的维度是 $n_i$, 若 $n_i = 1$, 我们说 $\lambda_i$ 是\textbf{非简并(non-degenerate)}的, 若 $n_i > 1$ 就说 $\lambda_i$ 是 $n_i$ 重\textbf{简并(degenerate)}的, 把 $n_i$ 叫做\textbf{简并数(degeneracy)}.

\begin{example}{二维矩阵的本征方程}
给出任意二维实数矩阵
\begin{equation}
\mat A = \pmat{a & b \\ c & d}
\end{equation}
要求它的本征值和本征矢, 其特正多项式为
\begin{equation}
\vmat{a-\lambda & b \\ c & d-\lambda} = (\lambda-a)(\lambda-d) - bc = 0
\end{equation}
解二次方程得两个本征值为
\begin{equation}
\lambda_\pm = \frac{(a + d) \pm \sqrt{(a-d)^2 + 4bc}}{2}
\end{equation}
复数域中必定存在两个根, 包括重根. 若要求本征值为实数, 则需要另判别式(根号中的式子)大于零, 否则本征方程无解.

若两本征值不相同, 本征矢为
\begin{equation}
\bvec v_\pm = \pmat{b\\ \lambda_\pm - a} = \pmat{\lambda_\pm - d\\ c}
\end{equation}
若两本征值相同, 则任意二维非零矢量都是本征矢, 本征值也都相同.
\end{example}

\subsection{对角化与相似变换}
求解矩阵的本征方程的过程有时候也叫做矩阵的\textbf{对角化(diagonalization)}, 原因如下: 把矩阵 $\mat A$ 的第 $i$ 个本征值和本征列矢量记为 $\lambda_i$ 和 $\bvec v_i$, 如果把本征值按顺序组成对角矩阵 $\mat \Lambda$, 把 $\bvec v_i$ 按顺序从左到右组成方阵 $\mat P$, 那么根据矩阵乘法\upref{Mat}的定义, $\mat A \mat P$ 相当于分别计算 $\mat A\bvec v_i$ 再从左到右排成方阵. 而 $\mat P\mat\Lambda$ 相当于把 $\lambda_i\bvec v_i$ 从左到右排成方阵. 二者应该相等, 所以有
\begin{equation}
\mat A \mat P = \mat P\mat\Lambda
\end{equation}
由于方阵 $\mat P$ 是满秩\upref{MatRnk}的(每列线性无关), 必定存在逆矩阵% 连接未完成
$\mat P^{-1}$. 两边右乘 $\mat P^{-1}$ 或者左乘 $\mat P$ 得
\begin{equation}\label{MatEig_eq3}
\mat A = \mat P\mat\Lambda\mat P^{-1}, \qquad
\mat \Lambda = \mat P^{-1}\mat A \mat P
\end{equation}
这种从 $\mat A$ 和 $\mat\Lambda$ 之间的变换被称为\textbf{相似变换(similarity transform)}\upref{MatSim}. 如果能找到使 $\mat\Lambda$ 为对角矩阵的 $\mat P$ 就相当于解出了本征方程\autoref{MatEig_eq1}, 这就是 “对角化” 名字的由来.
