% 正规扩张
% 分裂域|多项式|共轭|compositum|composite|合成域

\pentry{可分扩张\upref{SprbEx}}

\autoref{SpltFd_the2}~\upref{SpltFd}揭示了分裂域和正规扩张的等价性,所以本词条讨论的对象本质上和\textbf{分裂域}\upref{SpltFd}是一样的,但会更加深入和全面.


\begin{definition}{共轭}
设$\mathbb{F}$是一个域,$\overline{\mathbb{F}}$是其代数闭包.

对于$\alpha\in\overline{\mathbb{F}}$,其关于$\mathbb{K}$的\textbf{共轭元素(conjugate)}定义为其在$\overline{\mathbb{F}}$的某个保$\mathbb{F}$自同构的像.

对于$\mathbb{F}$的代数扩张$\mathbb{K}$,其关于$\mathbb{K}$的\textbf{共轭域(conjugate)}定义为其在$\overline{\mathbb{F}}$的某个保$\mathbb{F}$自同构的像.
\end{definition}

类比\autoref{SpltFd_the3}~\upref{SpltFd}的证明思路,可知两元素共轭的充要条件是,它们是同一个不可约多项式的根.我们也可以用共轭的语言来描述正规扩张:

\begin{theorem}{}
一个代数扩张$\mathbb{K}/\mathbb{F}$是正规的,当且仅当$\mathbb{K}$关于$\mathbb{F}$的共轭只有它自己,当且仅当任意$a\in\mathbb{K}$的共轭元素仍然在$\mathbb{K}$中.
\end{theorem}



\subsection{正规扩张的性质}

现在我们讨论的是,给定域上正规扩张集合的结构.

\begin{theorem}{}
设$\mathbb{K}/\mathbb{F}$是正规扩张,且存在中间域$\mathbb{M}$,则$\mathbb{K}/\mathbb{M}$也是正规扩张.
\end{theorem}

\textbf{证明}:

正规扩张等价于分裂域.如果$\mathbb{K}=\mathbb{F}(a_1, a_2, \cdots)$,那么$\mathbb{K}=\mathbb{M}(a_1, a_2, \cdots)$.由此得证.

\textbf{证毕}.


\begin{definition}{}
设$\mathbb{F}_i$是$\mathbb{K}$的一族子域,则记$\prod_{i}\mathbb{F}_i$为包含全体$\mathbb{F}_i$的最小的子域,称为族$\{\mathbb{F}_i\}$的\textbf{合成(composite或compositum)}.

$\prod_{i}\mathbb{F}_i$也可以记为$\mathbb{F}_1\mathbb{F}_2\cdots$.
\end{definition}


合成也可以看成是一种扩域,$\mathbb{K}\mathbb{F}=\mathbb{K}(\mathbb{F})=\mathbb{F}(\mathbb{K})$.


\begin{theorem}{}
设$\mathbb{K}/\mathbb{F}$是正规扩张,子域$\mathbb{E}\subseteq\overline{\mathbb{F}}$.如果合成域$\mathbb{EK}$存在,那么$\mathbb{EK}/\mathbb{EF}$是正规扩张.
\end{theorem}

\textbf{证明}:



\textbf{证毕}.






\begin{theorem}{}
设$\mathbb{K}_i/\mathbb{F}$是正规扩张,则$\mathbb{K}_1\mathbb{K}_2/\mathbb{F}$是正规扩张.
\end{theorem}


\textbf{证明}:



\textbf{证毕}.


\begin{theorem}{}
设$\mathbb{K}_i/\mathbb{F}$是正规扩张,则$\mathbb{K}_1\cap\mathbb{K}_2/\mathbb{F}$是正规扩张.
\end{theorem}























