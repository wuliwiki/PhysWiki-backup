% 匀速曲线运动

% 未完成: 用微分近似不妥, 还是先推出 a = \omega^2 r, 然后再用曲率半径求 \omega 会比较严谨

\pentry{曲率半径\upref{curvat}, 牛顿运动定律\upref{New3}} % 未完成: 这里引用的曲率半径只是平面曲线的,怎么办?

\subsection{加速度}
质点沿着曲线运动, 我们希望得到它的加速度. 先来看匀速的情况, 这时候加速度\upref{VnA}完全由速度矢量 $\bvec v$ 的方向改变而产生, 就像匀速圆周运动那样. 回顾加速度的定义
\begin{equation}\label{PCuvMo_eq2}
\bvec a = \lim_{\Delta t \to 0} \frac{\Delta \bvec v}{\Delta t}
\end{equation}
类似用几何法推导匀速圆周运动的速度\upref{CMVD}那样, 我们可以近似\upref{LimArc}认为当速度矢量 $\bvec v$ 转过一个小角度 $\Delta \theta$ 时, 它的增量 $\Delta \bvec v$ 垂直于 $\bvec v$, 且大小为
\begin{equation}\label{PCuvMo_eq1}
\abs{\Delta \bvec v} = v\Delta\theta
\end{equation}
代入\autoref{PCuvMo_eq1} 得加速度大小为
\begin{equation}\label{PCuvMo_eq4}
\abs{\bvec{a}} = \lim_{\Delta t \to 0} \frac{\abs{\Delta \bvec v}}{\Delta t}
= v\lim_{\Delta t \to 0}\frac{\Delta \theta}{\Delta t} = v\omega
\end{equation}
其中 $\omega$ 是速度矢量 $\bvec v$ 旋转的角速度. 方向与 $\bvec v$ 垂直, 即曲线切线变化的方向.

如果使用角速度矢量\upref{CMVD} $\bvec \omega$, 那么加速度矢量可以表示为
\begin{equation}
\bvec a = \bvec \omega \cross \bvec v
\end{equation}

那么当质点经过曲线某一点时,如何求 $\omega$ 呢? 我们可以使用曲率\upref{curvat}的概念. 令质点所在位置的曲率半径为 $R$, 根据曲率半径的定义(\autoref{curvat_eq3}~\upref{curvat}), $\Delta t$ 内质点在曲线上走过的长度为 $\Delta l = v \Delta t$, 所以切线的角度变化率为
\begin{equation}
\omega = \lim_{\Delta t\to 0}\frac{\Delta \theta}{\Delta t} = \lim_{\Delta t\to 0}\frac{\Delta \theta}{\Delta l} \frac{\Delta l}{\Delta t} = \frac{v}{R}
\end{equation}
再带入\autoref{PCuvMo_eq4} 得
\begin{equation}\label{PCuvMo_eq3}
\abs{\bvec{a}} = \lim_{\Delta t \to 0} \frac{\abs{\Delta \bvec v}}{\Delta t}
= \frac{v^2}{R}
\end{equation}


我们可以看到, \autoref{PCuvMo_eq3} 和匀速圆周运动的向心加速度(\autoref{CMAD_eq4}~\upref{CMAD})是一样的. 这是意料之中的, 因为圆就是曲率半径恒为 $R$ 的特殊曲线.

另见变速曲线运动\upref{VarCur}.
