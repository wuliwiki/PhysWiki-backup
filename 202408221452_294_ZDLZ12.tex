% 浙江大学 2012 年硕士入学量子力学考试试题
% keys 浙江大学|2012年|量子力学
% license Copy
% type Tutor

\subsection{简答题}
\subsubsection{第一题}
\begin{enumerate}
\item 证明厄米算符的本征值为实数
\item 对于力$\uvec H=\frac{\uvec p^2}{2m}+\alpha \uvec L_x$( $\alpha$为常数),下列力学量中哪些是守恒量?\\
$\uvec H,\uvec p_x,\uvec p_y,\uvec p_z,\uvec p^2,\uvec L_x,\uvec L_y,\uvec L_z,\uvec L^2$。
\item 原子的受激辐射和自发辐射区别在哪里?
\item 你知道哪些纯量子效应?
\item 写出泡利矩阵
$\sigma^x=\pmat{0&1\\1&0}\qquad \sigma^y=\pmat{0&-i\\i&0}\qquad \theta^z=\pmat{1&0\\0&-1}$满足的对易关系。
\end{enumerate}
\subsubsection{第二题}
电子被束缚在简谐振子势场 $v=\frac{1}{2}m\omega^2x^2$  中,若引入 $\displaystyle \uvec a =\frac{1}{\sqrt{2}}(\frac{x}{x_0}+\frac{i\uvec p}{x_0m\omega}),\uvec a^+ =\frac{1}{\sqrt{2}}(\frac{x}{x_0}-\frac{i\uvec p}{x_0m\omega}),x_0=\sqrt{h/m\omega}$ ,则有 $\displaystyle H=h\omega(\uvec a^+ \uvec a+\frac{1}{2})$ ,并有关系 $\uvec a^+ \ket{n}=\sqrt{n+1}\ket{n+1},\uvec a \ket{n}=\sqrt{n}\ket{n-1}$ ,显然基态应满足 $\uvec a\ket{0}=0$ 。\\
(1)试求基态波函数。\\
(2)进一步求第一激发态的波函数。\\
(3)如果势阱中有两个电子(忽略它们间的相互作用,它们整体的基态波函数是什么?(提示:电子为自旋 1/2 的全同粒子)。\\
(4)如果加入均匀磁场B,问当B很强,超过某临界$B_c$时,(3)中所述基态还会是基态吗?试具体求$B_c$
\subsubsection{第三题}
有一个质量为m的粒子处在如下势阱中
\begin{figure}[ht]
\centering
\includegraphics[width=8cm]{./figures/b5d66eedc3f7e962.png}
\caption{} \label{fig_ZDLZ12_1}
\end{figure}
\begin{equation}
V(x)=\leftgroup{&\infty,& x<0 \\ &-V_0,&0<x<a\\&V_0,&a<x<a+b\\&0,& a+b<x}~
\end{equation}
(这里$V_0>0$)\\
(1)试求其能级与波函数。\\
(2)问通过调节势阱宽度a,能否让阱中的粒子有一定的几率穿透出来。
\subsubsection{第四题}
将质子看作是半径为R的带电球壳,$V(r)=\leftgroup{\frac{e}{R}\quad r<R\\\frac{e}{r}\quad r>R}$(其中e为基本电荷值,$a_0$为玻尔半径,R <<$a_0$), 计算由于质子(即氢原子核)的非点性引起氢原子基态能级的一级修正。
\subsubsection{第五题}
求哈密顿量 $H=\sigma_1^x\sigma_2^x+\sigma_1^y\sigma_2^y+\alpha \sigma_1^z\sigma^x_2$的本征值和本征矢量,试分析$a =1$时有何特点。[提示:泡利矩阵的下标 1,2 表示第一个粒子和第二个粒子,因此可用矩阵直乘理解,即$\sigma_1^x\sigma_2^x=\sigma_1^x \sigma_2^x$等等]
第六题(15分):有一个量子系统,假如你已知道基态和激发态的波函数分别是
Vo;Vi;W2;Y;..·,对应于E<E<E<E.·,把两个全同粒子 (不考虑它们
之间的相互作用)放到该系统,
(1)对于自旋为零的粒子,写出基态与第一一激发态的波函数。
(2)对于自旋为1/2 的粒子,写出基态波函数。

