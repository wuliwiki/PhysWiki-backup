% 康托尔集(综述)
% license CCBYSA3
% type Wiki

本文根据 CC-BY-SA 协议转载翻译自维基百科\href{https://en.wikipedia.org/wiki/Cantor_set}{相关文章}。

在数学中,康托尔集是一个位于同一直线线段上的点集,它具有许多违反直觉的性质。该集合最早由亨利·约翰·斯蒂芬·史密斯于1874年发现\(^\text{[1][2][3][4]}\),并在1883年被德国数学家格奥尔格·康托尔提及\(^\text{[5][6]}\)。

通过对该集合的研究,康托尔及其他数学家为现代点集拓扑奠定了基础。最常见的构造是康托尔三分集,它通过不断从一条线段中去除中间三分之一,并对剩余的每一段重复该过程来构造。康托尔在其论文中仅顺带提到了这种三分构造,作为一个“完美但稠密度为零”的集合的例子\(^\text{[5]}\)。

更一般地,在拓扑学中,\textbf{康托尔空间}是指与康托尔三分集同胚的拓扑空间(配备其子空间拓扑)。康托尔集在自然意义上同胚于离散二点空间
\(2^\mathbb{N}\)(即离散二元集合的可数笛卡尔积)。根据L.E.J.布劳威尔的一个定理,这等价于以下五个条件的同时满足:完美(无孤立点)、非空、紧致、可度量且零维\(^\text{[7]}\)。
\subsection{三分集的构造与公式}
康托尔三分集(记作 $\mathcal{C}$)是通过对一组线段不断地删除中间的开放三分之一区间而构造出来的。最开始,从区间 $[0, 1]$ 中删除中间的开放三分之一区间 $\left(\frac{1}{3}, \frac{2}{3}\right)$,留下两个闭合的线段:
$$
\left[0, \frac{1}{3}\right] \cup \left[\frac{2}{3}, 1\right]~
$$
接下来,对这两个剩余的线段分别删除它们的中间三分之一,得到四个线段:
$$
\left[0, \frac{1}{9}\right] \cup \left[\frac{2}{9}, \frac{1}{3}\right] \cup \left[\frac{2}{3}, \frac{7}{9}\right] \cup \left[\frac{8}{9}, 1\right]~
$$
如此重复下去,康托尔三分集就是在这个无限过程中的所有步骤中都未被删除的点所组成的集合,仍然属于区间 $[0, 1]$。

这个构造过程也可以用递归方式来描述。令:
$$
C_0 := [0, 1]~
$$
接着定义:
$$
C_n := \frac{C_{n-1}}{3} \cup \left(\frac{2}{3} +\frac{C_{n-1}}{3}\right) = \frac{1}{3} \left(C_{n-1} \cup (2 + C_{n-1})\right)
\quad \text{对所有 } n \geq 1~
$$
于是,康托尔三分集可以表示为:
$$
\mathcal{C} := \lim_{n \to \infty} C_n = \bigcap_{n=0}^{\infty} C_n = \bigcap_{n=m}^{\infty} C_n \quad \text{对任意 } m \geq 0~
$$
这个构造过程的前六步如下图所示(原文提及有图示,此处略去)。
\begin{figure}[ht]
\centering
\includegraphics[width=14.25cm]{./figures/5a6934f361e2bc93.png}
\caption{} \label{fig_KTRJ_1}
\end{figure}
利用自相似变换的思想设$T_L(x) = x/3$、$T_R(x) = (2 + x)/3$,则递归定义为:$C_n = T_L(C_{n-1}) \cup T_R(C_{n-1})$于是,可以写出康托尔集的显式闭式表达如下 \(^\text{[8]}\):
$$
\mathcal{C} = [0, 1] \setminus \bigcup_{n=0}^{\infty} \bigcup_{k=0}^{3^n - 1} \left( \frac{3k+1}{3^{n+1}}, \frac{3k+2}{3^{n+1}} \right)~
$$
其中每一个被移除的中间三分之一是从闭区间
$$
\left[ \frac{3k+0}{3^{n+1}}, \frac{3k+3}{3^{n+1}} \right] = \left[ \frac{k+0}{3^n}, \frac{k+1}{3^n} \right]~
$$
中去除的开区间
$$
\left( \frac{3k+1}{3^{n+1}}, \frac{3k+2}{3^{n+1}} \right)~
$$。
或者,康托尔集也可以表示为:
$$
\mathcal{C} = \bigcap_{n=1}^{\infty} \bigcup_{k=0}^{3^{n-1}-1} \left( \left[ \frac{3k+0}{3^n}, \frac{3k+1}{3^n} \right] \cup \left[ \frac{3k+2}{3^n}, \frac{3k+3}{3^n} \right] \right)~
$$
这里,每个闭区间$\left[\frac{k+0}{3^{n-1}}, \frac{k+1}{3^{n-1}}\right] = \left[\frac{3k+0}{3^n}, \frac{3k+3}{3^n}\right]$中的中间三分之一区间$\left(\frac{3k+1}{3^n}, \frac{3k+2}{3^n}\right)$
通过与两端子区间$\left[\frac{3k+0}{3^n}, \frac{3k+1}{3^n}\right] \cup \left[\frac{3k+2}{3^n}, \frac{3k+3}{3^n}\right]$的交集被“移除”。

这个移除中间三分之一的过程是一个有限细分规则的简单示例。康托尔三分集的补集是分形弦的一个例子。
\begin{figure}[ht]
\centering
\includegraphics[width=14.25cm]{./figures/c1c347db78f52e5c.png}
\caption{} \label{fig_KTRJ_2}
\end{figure}
从算术角度来看,康托尔集由所有不需要数字1来表示的三进制(以3为底)小数组成,这些数都属于区间 $[0, 1]$。

正如上面的图示所示,康托尔集中的每一个点都可以通过一棵无限深的二叉树中的一条路径唯一确定。这条路径在每一层都会向左或向右分叉,取决于该点位于被删去线段的哪一侧。将每次向左转用0表示,向右转用2表示,便可得到该点对应的三进制小数。

