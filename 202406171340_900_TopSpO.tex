% 拓扑空间之间的运算
% keys 拓扑空间|商|积|不交并
% license Xiao
% type Tutor

\begin{issues}
% \issueOther{用 issues 环境列出文章存在的所有问题。}
\issueDraft
\issueTODO
\issueMissDepend
\issueAbstract
\issueNeedCite
\end{issues}

\subsection{概述}

拓扑空间之间的运算指的是,用给定的一些拓扑空间 $X_1, X_2, \dots$,得到一个唯一确定的新拓扑空间,以及新旧拓扑空间之间的映射。

\addTODO{是否需要加上交换图}

\subsection{拓扑空间之间的运算}

\subsubsection{商}

详情见文章\enref{商拓扑}{Topo7}

\addTODO{将商拓扑文章迁移到此处}

\subsubsection{积}

详情见文章\enref{积拓扑}{Topo6}

\addTODO{将积拓扑文章迁移到此处}

\subsubsection{不交并}

详情见文章\autoref{def_Topo9_3} 

\addTODO{将积不交并文章迁移到此处}

\subsubsection{<待定>}

\href{https://en.m.wikipedia.org/wiki/Join_(topology)}{wiki 链接}

\addTODO{拓扑空间的join运算}

\subsubsection{锥化<待定>}

\begin{definition}{锥化}
拓扑空间 $X$的\textbf{锥化空间}(或者\textbf{锥空间}、\textbf{锥},记作 $C X$)定义为 $X$和一个单点空间的join,即
\[
    C X := X \times [0,1] / \sim~.
\]
$(x, 1) \sim (y, 1)$。

$i: X \hookrightarrow C X, x \mapsto [(x, 0)]$ 被称为\textbf{锥化函数}。
\end{definition}

\addTODO{需要严格定义一下“运算”,比如这里到底 $C X$,$i$,$(C X, i)$ 哪个才是“锥化运算”(答案是都可以)}

\subsubsection{双角锥化<待定>}

\href{https://en.wikipedia.org/wiki/Suspension_(topology)}{wiki 链接}

\begin{definition}{双角锥化}
拓扑空间 $X$的\textbf{双角锥化空间}(记作 $S X$)定义为
\[
    S X := X \times [0,1] / \sim~.
\]
$(x, 1) \sim (y, 1), (x, 0) \sim (y, 0)$。

$i: X \hookrightarrow S X, x \mapsto [(x, 1/2)]$ 被称为\textbf{双角锥化函数}。
\end{definition}

\begin{theorem}{球面的双角锥化}
$S^n$ 是 $n$ 维球面
\[
S S^n \cong S^{n+1}~.
\]
\end{theorem}
这也是为什么使用$S$作为双角锥化的符号的原因了(亦可使用$\mathbb{S}$,如果你喜欢)。

\addTODO{文章:$S^n$,$n$ 维球面}

\subsection{带基点拓扑空间之间的运算}

什么是带基点拓扑空间见\autoref{def_Topo9_2} 

\subsubsection{一点并}

详情见文章\autoref{def_Topo9_1} 

\addTODO{将一点并文章迁移到此处}

\subsubsection{压缩积}

详情见文章\autoref{def_Topo9_1} 

\addTODO{将压缩积文章迁移到此处}

\subsubsection{约化锥化}

\href{https://en.wikipedia.org/wiki/Cone_(topology)\%5C#Reduced_cone}{wiki 链接}

\begin{definition}{约化锥化}
带基点拓扑空间 $(X, x_0)$的\textbf{约化锥化空间}(或者\textbf{约化锥空间}、\textbf{约化锥},记作 $C^* X$)是在 $S$的锥的基础上再商去所有的 $(x_0, t)$得到的,即
\[
    C^* X := X \times [0,1] / \sim~.
\]
$(x, 1) \sim (y, 1), (x_0, t) \sim (x_0, s)$。

$i: X \hookrightarrow C^* X, x \mapsto [(x, 0)]$ 被称为\textbf{约化锥化函数}。
\end{definition}

\subsubsection{约化双角锥化}

\href{https://en.wikipedia.org/wiki/Suspension_(topology)\%5C#Reduced_suspension}{wiki 链接}

\begin{definition}{约化双角锥化}
带基点拓扑空间 $(X, x_0)$的\textbf{约化锥化双角空间}(或者\textbf{约化双角锥空间}、\textbf{约化双角锥},记作 $\Sigma X$)是在 $S$的双角锥的基础上再商去所有的 $(x_0, t)$得到的,即
\[
    \Sigma X := X \times [0,1] / \sim~.
\]
$(x, 1) \sim (y, 1), (x, 0) \sim (y, 0), (x_0, t) \sim (x_0, s)$。

$i: X \hookrightarrow \Sigma X, x \mapsto [(x, 1/2)]$ 被称为\textbf{约化双角锥化函数}。
\end{definition}


