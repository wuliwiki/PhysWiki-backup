% 广义球谐函数
% CG系数|广义球谐函数|宇称算符|交换对称性|正交性

\pentry{CG 系数\upref{SphCup}, 球谐函数\upref{SphHar}}
有了 CG 系数的相位约定和球谐函数的相位约定, 就可以由球谐函数定义\textbf{广义球谐函数(Generalized Spherical Harmonics)}
\begin{equation}\label{eq_GenYlm_1}
\mathcal{Y}_{l_1,l_2}^{L,M}(\uvec r_1, \uvec r_2) = \sum_{m_1, m_2} \bmat{l_1 & l_2 & L\\ m_1 & m_2 & M} Y_{l_1 m_1}(\uvec r_1) Y_{l_2 m_2} (\uvec r_2)~.
\end{equation}
由于只有 $m_1 + m_2 = M$ 时 CG 系数才不为零, 公式中令 $m_2 = M - m_1$, 求和就由双重求和变为了单个求和, 而且只有有限项。 $m_1$ 的上下限满足 $-l_1 \leqslant m_1 \leqslant l_1$, $-l_2 \leqslant M-m_1 \leqslant l_2$ 即
\begin{equation}
\max\{-l_1, M-l_2\} \leqslant m_1 \leqslant \max\{l_1, M+l_1\}~.
\end{equation}
\autoref{eq_GenYlm_1} 可以看作一个酉矩阵\upref{UniMat}乘以列向量。

由于\autoref{eq_GenYlm_1} 是一个正交变换, 其逆变换为
\begin{equation}
Y_{l_1 m_1}(\uvec r_1) Y_{l_2 m_2} (\uvec r_2) = \sum_{L} \bmat{l_1 & l_2 & L\\ m_1 & m_2 & M} \mathcal{Y}_{l_1,l_2}^{L,M}(\uvec r_1, \uvec r_2)~.
\end{equation}
由于球谐函数一般是复函数, 而 CG 系数是实数, 所以 $\mathcal{Y}_{l_1,l_2}^{L,M}$ 一般也是\textbf{复函数}。
\addTODO{什么时候是实数?}

\subsubsection{Mathematica 代码}
在 Mathematica 中可以把广义球谐函数定义为\footnote{使用时出现的 “not triangular” 警告可忽略。}
\begin{lstlisting}[language=mma]
YY[l1_, l2_, L_, M_, θ1_, ϕ1_, θ2_, ϕ2_] := 
 Sum[ClebschGordan[{l1, m1}, {l2, M - m1}, {L, M}] SphericalHarmonicY[
    l1, m1, θ1, ϕ1] SphericalHarmonicY[l2, 
    M - m1, θ2, ϕ2], {m1, Max[-l1, -l2 + M], 
   Min[l2, l2 + M]}]
\end{lstlisting}

\subsection{微分方程}
广义球谐函数是量子力学中双粒子总角动量算符 $\Q L^2, \Q L_z$ 的本征函数。
\begin{equation}\label{eq_GenYlm_2}
\Q L^2 \mathcal Y_{l_1,l_2}^{L,M} = L(L+1) \hbar^2 \mathcal Y_{l_1,l_2}^{L,M}~,
\qquad
\Q L_z \mathcal Y_{l_1,l_2}^{L,M} = M\hbar \mathcal Y_{l_1,l_2}^{L,M}~.
\end{equation}


\subsection{宇称}
宇称(parity)算符 $\Pi$ 作用在一个函数上, 就是把函数所有自变量乘以 $-1$。 其本征函数是所有中心对称或反对称的函数, 本征值为分别为 $\pm 1$。

球谐函数是宇称算符的本征矢, 本征值为 $(-1)^l$ (\autoref{eq_SphHar_6}~\upref{SphHar}), 易得广义球谐函数也是宇称算符的本征矢, 本征值为 $(-1)^{l_1+l_2}$
\begin{equation}\label{eq_GenYlm_6}
\Pi \mathcal{Y}_{l_1,l_2}^{L,M}(\uvec r_1, \uvec r_2) =  \mathcal{Y}_{l_1,l_2}^{L,M}(-\uvec r_1, -\uvec r_2) = (-1)^{l_1+l_2} \mathcal{Y}_{l_1,l_2}^{L,M}(\uvec r_1, \uvec r_2)~.
\end{equation}

\subsection{交换对称性}
广义球谐函数不具有交换对称性\footnote{式中 $l_1+l_2-L$ 前面的负号可以变为加号, 但习惯上仍然写成负号, 因为虽然球谐函数要求 $L$ 为整数, 但在涉及到电子自旋时 $L$ 有可能是半整数, 这时就不能变为加号了。}(除非 $l_1 = l_2$)
\begin{equation}\label{eq_GenYlm_4}
P_{12} \mathcal{Y}_{l_1,l_2}^{L,M}(\uvec r_1, \uvec r_2) = \mathcal{Y}_{l_1,l_2}^{L,M}(\uvec r_2, \uvec r_1) = 
(-1)^{l_1+l_2-L} \mathcal{Y}_{l_2,l_1}^{L,M}(\uvec r_1, \uvec r_2)~,
\end{equation}
也可以记为
\begin{equation}\label{eq_GenYlm_5}
\mathcal{Y}_{l_1,l_2}^{L,M}(\uvec r_1, \uvec r_2) = 
(-1)^{l_1+l_2-L} \mathcal{Y}_{l_2,l_1}^{L,M}(\uvec r_2, \uvec r_1)~.
\end{equation}
但是我们可以轻易构造对称(+)或反对称(-)的函数
\begin{equation}
\mathcal{Y}_{l_1,l_2}^{L,M}(\uvec r_1, \uvec r_2) \pm (-1)^{l_1+l_2-L}\mathcal{Y}_{l_2,l_1}^{L,M}(\uvec r_1, \uvec r_2)~,
\end{equation}
\begin{equation}
\Psi(\bvec r_1, \bvec r_2) = 
R(r_1, r_2)\mathcal{Y}_{l_1,l_2}^{L,M}(\uvec r_1, \uvec r_2) - R(r_1, r_2)\mathcal{Y}_{l_2,l_1}^{L,M}(\uvec r_1, \uvec r_2)~, \qquad (l_1+l_2-L = odd).
\end{equation}

证明可以用 CG 系数的对称性(交换左边两列, \autoref{eq_SphCup_7}~\upref{SphCup})
\begin{equation}
\ali{
\mathcal{Y}_{l_1,l_2}^{L,M}(\uvec r_2, \uvec r_1)
&= \sum_{m_1, m_2} \bmat{l_1 & l_2 & L\\ m_1 & m_2 & M} Y_{l_1, m_1}(\uvec r_2)Y_{l_2, m_2}(\uvec r_1) \\
&= (-1)^{l_1+l_2-L} \sum_{m_1, m_2} \bmat{l_2 & l_1 & L\\ m_2 & m_1 & M} Y_{l_2, m_2}(\uvec r_1) Y_{l_1, m_1}(\uvec r_2)\\
& = (-1)^{l_1+l_2-L} \mathcal{Y}_{l_2,l_1}^{L,M}(\uvec r_1, \uvec r_2)~,
}\end{equation}
从\autoref{eq_GenYlm_4} 也可以得到对易关系
\begin{equation}
\comm{L^2}{P_{12}} = \comm{L_z}{P_{12}} = 0~,
\end{equation}
即交换两个粒子不改变总角动量。

% 未完成: 这个应该放到适当的地方
操作类算符(宇称,平移,交换)如果与某物理量算符对易, 就说明波函数经过该操作, 改物理量守恒。 如果哈密顿中的某一项算符(例如 $z$ 方向的电场的 dipole)如果与某物理量算符(例如 $L_z$)对易, 就说明波函数经过该传播子传播,该物理量守恒。

\subsection{正交性}
将\autoref{eq_SphCup_11}~\upref{SphCup} 用这里的符号表示, 就是
\begin{equation}
\int \mathcal{Y}_{l_1', l_2'}^{L', M'}(\uvec r_1, \uvec r_2) \mathcal{Y}_{l_1, l_2}^{L, M}(\uvec r_1, \uvec r_2) \dd{\Omega_1} \dd{\Omega_2} = \delta_{l_1, l_1'} \delta_{l_2, l_2'} \delta_{L, L'} \delta_{M, M'}~.
\end{equation}

\subsection{其他性质}
\begin{equation}\label{eq_GenYlm_3}
\mathcal{Y}_{l_1,l_2}^{L,-M}(\uvec r_1, \uvec r_2) = (-1)^{l_1 + l_2 + L + M} \mathcal{Y}_{l_1,l_2}^{L,M}(\uvec r_1, \uvec r_2)^*~,
\end{equation}
其中 $*$ 表示复共轭。 推导如下,% 引用未完成 CG 系数的对称性未完成
\begin{equation}
\ali{
\mathcal{Y}_{l_1,l_2}^{L,-M}(\uvec r_1, \uvec r_2) &= \sum_{m_1+m_2 = M} \bmat{l_1 & l_2 & L \\ -m_1 & -m_2 & -M} Y_{l_1,-m_1}(\uvec r_1) Y_{l_2, -m_2} (\uvec r_2)\\
&=  (-1)^{l_1+l_2+L} \sum_{m_1+m_2 = M} \bmat{l_1 & l_2 & L \\ m_1 & m_2 & M} Y_{l_1,-m_1}(\uvec r_1) Y_{l_2, -m_2} (\uvec r_2)\\
&=  (-1)^{l_1+l_2+L+M} \sum_{m_1+m_2 = M} \bmat{l_1 & l_2 & L \\ m_1 & m_2 & M} Y_{l_1,m_1}^*(\uvec r_1) Y_{l_2, m_2}^* (\uvec r_2)\\
&= (-1)^{l_1+l_2+L+M} \mathcal{Y}_{l_1,l_2}^{L,M}(\uvec r_1, \uvec r_2)^*~.
}\end{equation}
