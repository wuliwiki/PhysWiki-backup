% 南京理工大学 2010 年 研究生入学考试试题 普通物理(B)
% license Usr
% type Note

\textbf{声明}:“该内容来源于网络公开资料,不保证真实性,如有侵权请联系管理员”

\subsection{一。填空题(32分,每空2分)}
\begin{enumerate}
    \item 已知一电子的运动方程可表示为 $r = b \cos \omega t\vec{i} + b \sin \omega t\vec{j} + ct\vec{k}$,式中 $a,b$ 为常数,$t$以秒计。$r$以来计,随在$t$时刻,电子的速度为__________,加速度为 ___________。
    \item 一质量为$m$的小球系在长为$L$的细绳的一端,绳的另一端固定于$O$点。先使小球以$v_0$速度做圆周水平匀速运动,然后细绳逐渐缩短,绳始终与运动方向夹角为$\theta$的小球的速度表达式为___________ ,细绳的张力为多大为 ___________。
    \item 设一平面简谐波沿 $x$ 轴正方向传播,已知 $x=0$ 处原点的振动方程为$ y = A \cos(\omega t - \pi/3)$,波速为 $v$。波在 $x=L$ 处终止反射,则$x=x_0$ 处 $(x_0 < L)$ 原点由于反射被引起的振动方程为 ___________ ,$x_0$ 处是波节位置的条件是 $x_0 =$___________。
    \item 如图所示,$x$ 方向上传播简谐波的振动周期为$T = 0.2s$,波长$\lambda = 20m$,当$x = 0$处质点的振动方程为___________ ,求波速为 ___________。
    \item $2 mol$ 氧气在27°C时的内能等于 ___________,其分子的平均动能是 ___________ ,平均平动动能是 ___________。
    \item 设一个气体分子的密度分布函数为$f(v)$,则单位体积中,$v_1$与$v_2$区间内的分子数为 ___________。
    \item 假设气体是平衡态的理想气体,外界压力、外界体积、分子数和温度稳定且和高度无关。假设自由运动的气体分子速度为$B$,则从高度$h$处逃逸的分子速率为 ___________ ,逃逸时间为 ___________ ,逃逸气体的动能为 ___________。
\end{enumerate}
