% 导数近似之差分矩阵算法
% keys 导数近似|差分矩阵
% license Usr
% type Tutor
\pentry{有限差分\nref{nod_Diffen}}{nod_d4df}
本节介绍利用已知函数点近似函数导数的方法,以及相应的计算机处理,它们将给出所谓的差分矩阵。这一内容将使我们能够更快过渡到谱方法。

首先思考这样的问题:给定一组点 $\{x_i\}$ 和对应的函数值 $u(x_i)$,我们如何使用这些数据近似 $u$ 的导数呢?

最直接的想法就是进行\enref{有限差分}{Diffen}。考虑点 $\{x_i\}$ 是均匀分布的,并记 $x_{i+1}-x_i=h$,则该式对每一 $i$ 都成立。

记 $w_i$ 为 $u'(x_i)$ 的近似值。那么标准的二阶有限差分近似为
\begin{equation}\label{eq_DifMa_1}
w_i=\frac{u_{i+1}-u_{i-1}}{2h}.~
\end{equation}
所谓的几阶近似公式是指
\begin{definition}{$n$ 阶近似}
称函数 $u(x)$ 是 $f(x)$ (关于 $h$)的\textbf{$n$ 阶近似},是指对任一点 $x_i$,$\abs{f(x_i)-u(x_i)}$ 是关于 $h$ 的 $n$ 阶无穷小。 
\end{definition}

让我们证明\autoref{eq_DifMa_1} 的差分公式确实是 $u$ 关于 $h$ 的二阶近似。

\textbf{证明:}利用该\enref{泰勒展开}{Taylor}
\begin{equation}
\begin{aligned}
f(x_i+h)&=f(x_i)+f'(x_i)h+\frac{1}{2}f''(x_i)h^2+\frac{1}{6}f'''(x_i)h^3+o(h^3),\\
f(x_i-h)&=f(x_i)-f'(x_i)h+\frac{1}{2}f''(x_i)h^2-\frac{1}{6}f'''(x_i)h^3+o(h^3).
\end{aligned}~
\end{equation}
因此
\begin{equation}
f(x_i+h)-f(x_i-h)=2hf'(x_i)+\frac{1}{3}h^3f'''(x_i)+o(h^3).~
\end{equation}
即
\begin{equation}
\begin{aligned}
f'(x_i)&=\frac{f(x_i+h)-f(x_i-h)}{2h}-\frac{1}{3}h^3f'''(x_i)-o(h^3)\\
&=\frac{f(x_i+h)-f(x_i-h)}{2h}+o'(h^2).
\end{aligned}~
\end{equation}
因此,\autoref{eq_DifMa_1} 表明
\begin{equation}
\abs{u'_i-w_i}=\abs{u'_i-\frac{u_{i+1}-u_{i-1}}{2h}}=\abs{o(h^2)}.~
\end{equation}


\textbf{证毕!}



