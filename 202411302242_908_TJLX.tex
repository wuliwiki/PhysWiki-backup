% 统计力学(综述)
% license CCBYSA3
% type Wiki


在物理学中,**统计力学**是一种数学框架,将统计方法和概率理论应用于大量微观实体的集合。有时也称为**统计物理**或**统计热力学**,其应用包括物理学、生物学、化学、神经科学、计算机科学、信息理论和社会学等多个领域。其主要目的是通过研究原子运动所遵循的物理规律,阐明物质在宏观集合状态下的性质。

统计力学源于经典热力学的发展,成功地解释了宏观物理属性(如温度、压力和热容量),将其与微观参数联系起来。这些微观参数围绕平均值波动,并以概率分布为特征。

虽然经典热力学主要关注**\textbf{热力学平衡}**,但统计力学在**非平衡统计力学**中得到了广泛应用,用于微观建模不可逆过程的速度,这些过程由不平衡驱动。例如,化学反应以及粒子和热的流动。**\textbf{涨落-耗散定理}**是将非平衡统计力学应用于研究多粒子系统中最简单的非平衡状态(即稳态电流流动)时获得的基本理论。