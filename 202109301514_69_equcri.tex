% 热动平衡判据
% keys 熵判据
\pentry{热平衡 热力学第零定律\upref{TherEq}热力学第二定律\upref{Td2Law}}

对于一个单元单相的孤立热力学系统(不考虑外场),它的平衡态意味着系统的\textbf{各个宏观性质在长时间内不发生变化},热平衡的判据为热学平衡、力学平衡、化学平衡\upref{TherEq}.为了能更好地将方法推广到更一般的系统(例如等温等压系统,例如单元复相系),来讨论相变和化学变化问题,我们需要具体地给出判据并在数学上进行分析.

对于单元单相的孤立系统,由熵增加定理\upref{Td2Law},我们可以利用\textbf{熵判据}判定孤立系统的某一状态是否为平衡态.虽然系统的内部可能宏大复杂的,但可以将系统“划分”成许多小部分,每一个小部分的内部 $P,V,T$ 近似处处相等,又仍有大量的微观粒子,这称为\textbf{子系统};对这样的子系统,熵是容易计算的.熵是广延量,将所有子系统的熵相加,可以得到整个孤立系统的熵