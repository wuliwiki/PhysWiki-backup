% 张量的外积
% keys 外积|外代数


\pentry{张量代数\upref{TenAlg},域上的代数\upref{AlgFie}}


张量代数\upref{TenAlg}一节最后提到,通过将张量积作为乘法不能使得 $\mathbb T_p^0(V)$ 的对称子空间和斜对称子空间的外直和 
\begin{equation}
\begin{aligned}
&\mathbb T^+(V^*)=\mathbb F\oplus\mathbb T_1^+(V)\oplus\mathbb T_2^+(V)\oplus\cdots\\
&\Lambda(V^*)=\mathbb F\oplus \Lambda^1(V^*)\oplus\Lambda^2(V^*)\oplus\cdots
\end{aligned}~
\end{equation}
成为 
\begin{equation}
\mathbb T(V^*)=\mathbb F\oplus\mathbb T_1^0(V)\oplus\mathbb T_2^0(V)\oplus\cdots~
\end{equation}
的子代数。这使得可以通过定义其它的乘积,使得这二者各自成为代数。对于斜对称的 $\Lambda(V^*)$,就是本节强调的重点。使 $\Lambda(V^*)$ 成为一个代数的乘法称为\textbf{外积}(\textbf{wedge}),记号为 $\wedge$,它是通过交错化映射 $A$(\autoref{def_SIofTe_1}~\upref{SIofTe} )定义的,具体便是
\begin{equation}
Q\wedge R=A(Q\otimes R)~.
\end{equation}

\subsection{外积}
事实上,基于张量的对称化和交错化\upref{SIofTe}引文处提到的理由,讨论任意张量的对称性和斜对称性,与 $(0,p)$型或 $(p,0)$ 型没有任何区别。为了多样性起见,这里考虑 $(0,p)$ ,即取空间($\Lambda^p(V)$ 的定义可见\autoref{def_SIofTe_2}~\upref{SIofTe}后的段落。) 
\begin{equation}\label{eq_WegofT_9}
\Lambda(V)=\mathbb F\oplus \Lambda^1(V)\oplus\Lambda^2(V)\oplus\cdots~
\end{equation}
使用\autoref{def_SIofTe_2}~\upref{SIofTe} 的术语,$\Lambda^p(V)$ 的元素称为 $p$ 矢量。

交错化映射 $A$ 将 $\mathbb T_0^p(V)$ 的张量映射到其上的斜对称张量子空间 $\Lambda^p(V)$ 上(\autoref{the_SIofTe_1}~\upref{SIofTe} ),而 $\Lambda^p(V)$ 都是 $\Lambda(V)$ 的子空间,因此有可能通过 $A$ 定义一种乘法,使得 $\Lambda(V)$ 在该乘法下封闭。注意 $ \Lambda(V)$ 上的任意元素 $Q$ 可记作(\autoref{eq_TenAlg_5}~\upref{TenAlg} )
\begin{equation}\label{eq_WegofT_2}
Q=\sum_{i=0}^\infty Q_i=(Q_0,Q_1,\cdots) \quad Q_i\in\Lambda^i(V)~.
\end{equation}

此时,为了让张量积不跳出 $\Lambda(V)$,我们试着用 $A$ 作用到 $\Lambda(V)$ 上任意两元素 $Q,R$ 的张量积 $Q\otimes R$ 上:
\begin{equation}
\begin{aligned}
&A(Q\otimes R)=A(\sum_{k=0}^\infty h_k)=\sum_{k=0}^\infty  A(h_k)~,\\
&h_k=\sum_{i=0}^k Q_i\otimes R_{k-i}\in\mathbb T_0^k(V)~.
\end{aligned}
\end{equation}
因此 $A(h_k)\in \Lambda^{k}(V)$,即
\begin{equation}
A(Q\otimes R)=\sum_{k=0}^\infty  A(h_k)\quad A(h_k)\in\Lambda^k(V)~.
\end{equation}
注意\autoref{eq_WegofT_2} ,就有 $A(Q\otimes R)\in\Lambda(V)$。

于是通过 $A(Q\otimes R)\in\Lambda(V)$ 来定义 $Q$ 和 $R$ 的乘法就得到了 $\Lambda(V)$ 上乘法的封闭性。显然,通过张量积运算的性质,$1\in\mathbb F$ 是该乘法的单位元。
\begin{definition}{外积}\label{def_WegofT_1}
对任意 $q$ 矢量 $Q$ 和 任意 $r$ 矢量 $R$,运算 $\wedge:\Lambda(V)\times\Lambda(V)\rightarrow\Lambda(V)$:
\begin{equation}
Q\wedge R=A(Q\otimes R)~
\end{equation}
称为 $\Lambda(V)$ 上的\textbf{外积运算}。
\end{definition}
\begin{theorem}{外积的性质}
外积具有以下的性质:
\begin{enumerate}
\item \begin{equation}\label{eq_WegofT_1}
\wedge:\Lambda^q(V)\times\Lambda^r(V)\rightarrow\Lambda^{q+r}(V)~.
\end{equation}
\item \textbf{双线性:}任意 $\alpha,\beta\in\mathbb F$,有
\begin{equation}
\begin{aligned}
Q\wedge(\alpha R+\beta T)&=\alpha(Q\wedge R)+\beta(Q\wedge T)~,\\
(\alpha Q+\beta S)\wedge R&=\alpha (Q\wedge R)+\beta (S\wedge R)~.
\end{aligned}
\end{equation}
\item \textbf{结合性:}
\begin{equation}
P\wedge (Q\wedge R)=(P\wedge Q)\wedge R~.
\end{equation}
\item $\forall \lambda\in\mathbb F$,都有
\begin{equation}\label{eq_WegofT_4}
\lambda(Q\wedge R)=(\lambda Q)\wedge R=Q\wedge(\lambda R)~.
\end{equation}

\item 任意 $x_{i_1},\cdots,x_{i_p}\in V$,满足
\begin{equation}\label{eq_WegofT_6}
x_{i_{\pi 1}}\wedge\cdots \wedge x_{i\pi p}=\epsilon_\pi x_{i_1}\wedge\cdots\wedge x_{i_p}~.
\end{equation}
\item 
\begin{equation}\label{eq_WegofT_5}
x_1\wedge\cdots\wedge x_p=A(x_1\otimes\cdots\otimes x_p)~.
\end{equation}
\end{enumerate}

\end{theorem}
\textbf{证明:}1。设 $Q$ 为 $q$ 矢量,$R$ 为 $r$ 矢量,那么由\autoref{the_CofTen_2}~\upref{CofTen} , $Q\otimes R\in\mathbb T_0^{q+r}$ 。由\autoref{the_SIofTe_1}~\upref{SIofTe} ,对 $\mathbb T_0^{q+r}(V)$ 上的 $A$,其像 $\Im A=\Lambda ^{q+r}(V)$ 。于是\autoref{eq_WegofT_1} 成立。

2。由张量积和 $A$ 的线性
\begin{equation}
\begin{aligned}
Q\wedge(\alpha R+\beta T)&=A(Q\wedge(\alpha R+\beta T))=A(\alpha Q\otimes R+\beta Q\otimes T)\\
&=\alpha A(Q\otimes R)+\beta A(Q\otimes T)=\alpha Q\wedge R+\beta Q\wedge T~.
\end{aligned}
\end{equation}
第2式同理。

3。由\autoref{the_SIofTe_2}~\upref{SIofTe}
\begin{equation}
\begin{aligned}
(P\wedge Q)\wedge R&=A(A(P\otimes Q)\otimes R)=A((P\otimes Q)\otimes R)\\
&=A(P\otimes (Q\otimes R))=A(P\otimes A(Q\otimes R))\\
&=P\wedge (Q\wedge R)~.
\end{aligned}
\end{equation}

4.利用张量积的运算性质(\autoref{the_TsrPrd_1}~\upref{TsrPrd} )和 $A$ 的线性直接得到。

5。先证 $x\wedge y=-y\wedge x,\;\forall x,y\in V$:
\begin{equation}
x\wedge y=A(x\otimes y)=\frac{1}{2}(x\otimes y-y\otimes x)~,
\end{equation}
因此
\begin{equation}\label{eq_WegofT_3}
x\wedge y=-y\wedge x,\quad x\wedge x=0~.
\end{equation}

对一般情形,可由\autoref{eq_WegofT_3} 和外积的结合性证得,见\autoref{ex_AntMap_1}~\upref{AntMap} 。

6。当 $p=2$ 时显然成立。假设对 $p<k$ 时成立,那么由\autoref{the_SIofTe_2}~\upref{SIofTe}
\begin{equation}
\begin{aligned}
x_1\wedge\cdots\wedge x_p&=(x_1\wedge\cdots\wedge x_{p-1})\wedge x_p=A((x_1\wedge\cdots\wedge x_{p-1})\otimes x_p)\\
&=A(A(x_1\otimes\cdots\otimes x_{p-1})\otimes x_p)=A(x_1\otimes\cdots\otimes x_{p-1}\otimes x_p)~.
\end{aligned}
\end{equation}

\textbf{证毕!}

\subsection{外代数}

从上面知道,由 $Q \wedge R=A(Q\otimes R)$ 定义了 $\Lambda(V)$ 上的乘法运算。其满足封闭性、结合性、纯量和张量外积的补充定律(\autoref{eq_WegofT_4} )并且还有单位元 $1\in\mathbb F$。这就使得 $\Lambda(V)$ 成为了域 $\mathbb F$ 上的一个代数(\autoref{def_AlgFie_1}~\upref{AlgFie})
\begin{definition}{外代数}
称域 $\mathbb F$ 上的代数 $\Lambda(V)$ 是空间 $V$ 的\textbf{外代数}(或\textbf{格拉斯曼代数}或\textbf{G代数}),其上的乘法由外积 “$\wedge$” (\autoref{def_WegofT_1} )所定义。
\end{definition}
\begin{theorem}{}\label{the_WegofT_1}
设 $\{e_1,\cdots,e_n\}$ 是矢量空间 $V$ 的一个基底,那么 $p$ 矢量
\begin{equation}\label{eq_WegofT_8}
e_{i_1}\wedge e_{i_2}\wedge\cdots\wedge e_{i_p}~,\quad 1\leq i_1<i_2<\cdots<i_p\leq n~.
\end{equation}
组成空间 $\Lambda^p(V)$ 的一个基底。
\end{theorem}
\textbf{证明:}任意 $P\in\Lambda^p(V)$ ,正如 $\mathbb T_0^p(V)$ 中的任意元素一样,其可由表示为
\begin{equation}
P=\sum_{j_1\cdots j_p}P^{j_1\cdots j_p}e_{j_1}\otimes\cdots\otimes e_{j_p}~.
\end{equation}
由\autoref{eq_SIofTe_11}~\upref{SIofTe} 和 $A$ 的线性性及\autoref{eq_WegofT_5} 
\begin{equation}
P=A(P)=\sum_{j_1\cdots j_p}P^{j_1\cdots j_p}A(e_{j_1}\otimes\cdots\otimes e_{j_p})=\sum_{j_1\cdots j_p}P^{j_1\cdots j_p}e_{j_1}\wedge\cdots\wedge e_{j_p}~.
\end{equation}
由\autoref{eq_WegofT_6}  ,任意 $e_{j_1}\wedge\cdots\wedge e_{j_p}$ 都可按升序排序来代替, 即任一 $p$ 矢量都可由形如的矢量展开。所以现在只需证明它们的线性无关即可。设


\begin{equation}
\sum_{1\leq i_1<\cdots<i_p\leq n} \lambda^{i_1\cdots i_p} e_{i_1}\wedge\cdots\wedge e_{i_p}=0~.
\end{equation}
由\autoref{eq_WegofT_5}  和\autoref{eq_SIofTe_3}~\upref{SIofTe}
\begin{equation}\label{eq_WegofT_7}
\begin{aligned}
\sum_{1\leq i_1<\cdots<i_p\leq n} \lambda^{i_1\cdots i_p} e_{i_1}\wedge\cdots\wedge e_{i_p}&=\sum_{1\leq i_1<\cdots<i_p\leq n} \lambda^{i_1\cdots i_p} A(e_{i_1}\otimes\cdots\otimes e_{i_p})\\
&=\frac{1}{p!}\sum_{1\leq i_1<\cdots<i_p\leq n}\lambda^{i_1\cdots i_p}\sum_{\pi\in S_p}\epsilon_\pi(e_{i_{\pi^{-1} 1}}\otimes\cdots\otimes e_{i_{\pi^{-1} p}})\\
&=0~.
\end{aligned}
\end{equation}

由\autoref{the_CofTen_2}~\upref{CofTen},$e_{i_1}\otimes\cdots\otimes e_{i_p}$ 构成 $\mathbb T_0^p(V)$ 上的一个基底。所以对不同的 $\pi$ ,$e_{i_{\pi^{-1} 1}}\otimes\cdots\otimes e_{i_{\pi^{-1} p}}$ 是线性无关的。所由\autoref{eq_WegofT_7}  ,只能是
\begin{equation}
\lambda^{i_1\cdots i_p}\epsilon_\pi=0\quad\Rightarrow\quad\lambda^{i_1\cdots i_p}=0~.
\end{equation}

\textbf{证毕!}
\begin{corollary}{ $\Lambda(V)$ 的维数}
空间 $V$ 的外代数 $\Lambda(V)$ 的维数是 $2^n$,同时
\begin{equation}
\dim\Lambda^p(V)=\left(\begin{aligned}
n\\p
\end{aligned}\right)~.
\end{equation}
\end{corollary}
\textbf{证明:}形如\autoref{eq_WegofT_8}  的 $p$ 矢量的个数等于从 $n$ 个中一次取 $p$ 个组合的个数(\autoref{eq_APC_1}~\upref{APC}),所以根据\autoref{the_WegofT_1} 
\begin{equation}
\dim\Lambda^p(V)=\left(\begin{aligned}
n\\p
\end{aligned}\right)~.
\end{equation}
由\autoref{eq_WegofT_9}  ,$\Lambda(V)$ 是各 $\Lambda^p(V)$ 的直和,所以
\begin{equation}
\dim\Lambda(V)=\sum_{p=0}^n\dim\Lambda^p(V)=\sum_{p=0}^n\left(\begin{aligned}
n\\p
\end{aligned}\right)=2^n~.
\end{equation}
 最后一式利用了二项式定理\upref{BiNor}。

\textbf{证毕!}

\begin{example}{}
使证明:若 $Q\in\Lambda^q(V), R\in\Lambda^r(V)$,那么
\begin{equation}\label{eq_WegofT_10}
Q\wedge R=(-1)^{qr}R\wedge Q~.
\end{equation}

\textbf{证明提示:}由\autoref{the_WegofT_1}  ,可用形如
 \begin{equation}
\begin{aligned}
&e_{i_1}\wedge e_{i_2}\wedge\cdots\wedge e_{i_q}~,\quad 1\leq i_1<i_2<\cdots<i_q\leq n~,\\
&e_{i_1}\wedge e_{i_2}\wedge\cdots\wedge e_{i_r}~,\quad 1\leq i_1<i_2<\cdots<i_r\leq n
\end{aligned}
\end{equation}
的式子分别将 $Q,R$ 展开。然后由\autoref{eq_WegofT_6}  ,可将 $R$ 的展开中每一项的基底的矢量一个个和前面矢量对换到 $Q$ 的基底前,最后就得到\autoref{eq_WegofT_10} .
\end{example}
