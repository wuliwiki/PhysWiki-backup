% 地球的公转


众所周知,地球的赤道面($0 ^\circ$纬度确定的平面)与黄道面(地球公转轨道确定的平面)存在倾角,因此导致了丰富的气候现象。本模型演示了地球绕太阳的公转,并体现了:
\begin{itemize}
\item 赤平面与黄道面
\item 自转轴
\item 晨昏线
\item 太阳垂直入射的纬度线
\item 若额外加入地球贴图,本模型也可以体现自转。(纯网格球体的旋转显示效果不好)
\end{itemize}

本模型仅适用于基础性定性演示。
\begin{itemize}
\item 太阳、地球的尺寸不正确。由于宇宙的尺寸如此之大,很难在合理范围内合乎比例地体现相对大小。
\item 没有考虑地球公转轨道的非椭圆性
\item 假定公转周期是365天,自转周期是1天,没有闰年
\item 没有引入初相位对齐自转与公转
\item 没有进动等
\end{itemize}

\subsection{地球的公转(黄道面参考系)}
\begin{figure}[ht]
\centering
\includegraphics[width=8cm]{./figures/georev_1.png}
\caption{地球的公转示意图} \label{georev_fig1}
\end{figure}

以下是相应的octave/matlab代码:
\begin{lstlisting}[language=matlab]
clc
clear

function c=CIRCLE(x,y,z,r)
  t = linspace(0,2*pi,40);
  cx = r*cos(t)+x;
  cy = r*sin(t)+y;
  cz = zeros(size(t))+z;
  c=plot3(cx,cy,cz);
end

function s = SHPERE(x,y,z,r)
[X,Y,Z] = sphere(20);
X = r*X+x;
Y = r*Y+y;
Z = r*Z+z;
s = surf(X,Y,Z);
set(s,'EdgeColor','none')
end

R=5;

for t=linspace(0,365,100)
%t unit in day, 0-1 is one day, 0-365 is one year.
% 大致上,t=0,90,180,270时,分别为北半球冬至,春分,夏至,秋分
%t的设定参考:
%公转:范围为0-365,step为100
%自转:范围为0-1,step为100
%公转+自转:范围为0-365,step为365*6,需要额外的地球贴图以获得更好效果
  clf;

  hold on
  axis equal
  axis([-7 7 -7 7 -2 2])
  xlabel('X')
  ylabel('Y')
  zlabel('Z')
  view(0,30)

  sun = SPHERE(0,0,0,0.5);  %绘制太阳
  set(sun,'facecolor','r')
  c1 = CIRCLE(0,0,0,5); %地球公转轨道
  [mx my] = meshgrid(-R-2:R+2);
  mz = zeros(size(mx));
  s1 = surf(mx,my,mz,'edgecolor','none','facecolor','y','faceAlpha',0.3); %黄道面

  ex = R*cos(2*pi/365*t);
  ey = R*sin(2*pi/365*t);
  earth = SPHERE(ex,ey,0,1); %绘制地球公转
  set(earth, 'FaceColor',[0.8 0.8 1],'EdgeColor',[0.6 0.6 0.6]);

  rotate(earth,[0 0 1],mod(t,1)*360,[ex ey 0]); %自转
  rotate(earth,[0 1 0],23.5,[ex ey 0]); %自转倾角
  l = line([ex ex],[ey ey],[-5 5]);  %绘制自转轴
  rotate(l,[0 1 0],23.5,[ex ey 0]);

  [mx my] = meshgrid(ex-2:ex+2, ey-2:ey+2);
  mz = zeros(size(mx));
  s2 = surf(mx,my,mz,'edgecolor','none','facecolor','r','faceAlpha',0.3);
  %绘制赤道面
  rotate(s2,[0 1 0],23.5,[ex ey 0]);

  rc = 1.01;
  cx = (R-rc)*ex/R;
  cy = (R-rc)*ey/R;
  c2 = CIRCLE(ex,ey,0,rc); %绘制直射纬度
  rotate(c2,[0 1 0],23.5,[cx cy 0]);

  c3 = CIRCLE(ex,ey,0,1.01); %绘制晨昏线
  rotate(c3,[0 1 0],90,[ex ey 0]);
  rotate(c3,[0 0 1],2/365*t*180,[ex ey 0]);
  hold off

  pause(0.05);
end
\end{lstlisting}

\subsection{地球的公转(赤道面参考系)}
将以下代码加入每帧的画图后("hold off"前),即可得到赤道面参考系的地球公转。

\begin{lstlisting}[language=matlab]
  c4 = CIRCLE(0,0,0,5); %太阳的黄道轨迹
  set(c4,'color',[0.8 0.8 0]);
  rotate(c4,[0 1 0],-23.5,[0 0 0]);

  [mx my] = meshgrid(-R-2:R+2); %新的黄道面(可选)
  mz = zeros(size(mx));
  s3 = surf(mx,my,mz,'edgecolor','none','facecolor','y','faceAlpha',0.3); %黄道面
  rotate(s3,[0 1 0],-23.5,[0 0 0]);

  rotate(sun,[0 1 0],-23.5,[ex ey 0]);
  rotate(c1,[0 1 0],-23.5,[ex ey 0]);
  rotate(s1,[0 1 0],-23.5,[ex ey 0]);
  rotate(earth,[0 1 0],-23.5,[ex ey 0]);
  rotate(l,[0 1 0],-23.5,[ex ey 0]);
  rotate(s2,[0 1 0],-23.5,[ex ey 0]);
  rotate(c2,[0 1 0],-23.5,[ex ey 0]);
  rotate(c3,[0 1 0],-23.5,[ex ey 0]);

  TRANSLATE(sun,-ex,-ey,0);
  TRANSLATE(c1,-ex,-ey,0);
  TRANSLATE(s1,-ex,-ey,0);
  TRANSLATE(earth,-ex,-ey,0);
  TRANSLATE(l,-ex,-ey,0);
  TRANSLATE(s2,-ex,-ey,0);
  TRANSLATE(c2,-ex,-ey,0);
  TRANSLATE(c3,-ex,-ey,0);

  delete(s1)%删除旧的黄道面(可选)
\end{lstlisting}
