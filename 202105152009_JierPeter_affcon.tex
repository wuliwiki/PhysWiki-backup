% 仿射联络(流形)
% affine|联络|connection|方向导数|平行移动|directional derivative|parallel transport

\pentry{流形上的切空间\upref{tgSpa}}

\subsection{平行移动}

地球的表面是一个二维实流形.如果我站在赤道上,沿着地球表面,保持向东匀速运动,那么我最终会绕赤道一圈回到原点.由于我被限制在地面,在我看来自己的速度是没有变化的,但如果你站在月球上看我,你会认为我的速度方向一直在变化.我们的答案都没有问题,只不过分别处在不同的视角中.站在月球上的你是从整个三维空间来观察的,也就是$\mathbb{R}^3$,但限制在地球上的我并不知道宇宙和地幔的存在,对我来说整个世界就是地球表面,也就是$S^2$.

在你看来,我处在不同位置上的速度向量是不一样的,但在我看来是一样的.这种情况就被称为一种“平行移动”,即给定流形上的一个切向量,沿着某条道路移动切点,过程中保持切向量不变.例子中的速度向量,其变化率是一直垂直地面的,而对于地面上的我来说是不存在这一方向的,因此在我的计算里我的速度并没有变化.

仔细琢磨以上例子,会发现很多值得注意的点.首先,平行移动是“在沿着某条道路移动过程中”保持切向量不变,也就是说此概念是依赖于道路而定义的,光有两个点可不行.我们可以通过一个例子来理解这一点:假设我手里有一个箭头,我在运动过程中保持这个箭头方向不变.我先绕赤道半圈,再沿着经线抵达北极,记录箭头的指向;回到原点重新开始,这次我直接沿着经线抵达北极,记录箭头的指向.两次运动的起点和终点都是一样的,但是最终记录的箭头指向是相反的.

其次,有变化的地方就有求导.平行移动无非就是要求运动过程中给定切向量对时间(或者对经过的弧长)求导的结果为零.对于地球表面的例子,求导的可行性似乎是显而易见的,但是具体例子的坏处就是自带先入为主的细节.回忆一下前面词条中对流形的讨论,我们似乎根本就没有讨论过求导的可行性.一个光滑向量场虽然是要求“在某个图中是光滑向量场”,但是换用不同的图来表示同一个光滑向量场,其求导结果可能完全不同.这是因为我们只要求图与图之间的变换是光滑的,而正是这个变换的存在导致不同图之间的求导结果不同,毕竟这个变换要参与到链式法则里.

直观来说,就是我们根本没有讨论过“不同切点上的切空间中,哪些向量应该被认为是相等的”.从地球表面的例子还可以看出,不同切空间中两个切向量是否相等,还可能取决于沿着什么道路进行平行移动.这就意味着,我们无法简单地规定一个“切空间之间的同构”,而必须和道路联系起来————好在我们不需要讨论完整的道路,只需要讨论切向量沿着道路的变化率即可.由于张量场沿着道路的变化率可以看作是道路对应的向量作用在该张量场上,我们可以通过“定义什么是流形上的方向导数”来“定义什么是沿着道路的变化率”,而其中“方向导数为零”就意味着“平行移动”了.

\subsection{仿射联络}

流形上不同切空间之间的联系,被称为如下\textbf{仿射联络}.

\begin{definition}{仿射联络}
给定实流形$M$,其中$\mathfrak{X}(M)$为其上光滑向量场的集合.记$\mathbb{F}$为$M$上全体光滑函数构成的环\footnote{此处是为了看起来舒服,实际上这个环通常写为$C^\infty(M)$.},若对于任意的$X, X_i, Y, Y_i\in\mathfrak{M}$,映射$\nabla:\mathfrak{M}\times\mathfrak{M}\to\mathfrak{M}$满足以下要求:
\begin{enumerate}
\item \textbf{$\mathbb{R}$-双线性性}:
\item \textbf{}
\item \textbf{}
\end{enumerate}

\end{definition}









