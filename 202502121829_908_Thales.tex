% 泰勒斯定理(综述)
% license CCBYSA3
% type Wiki

本文根据 CC-BY-SA 协议转载翻译自维基百科\href{https://en.wikipedia.org/wiki/Thales\%27s_theorem}{相关文章}。

\begin{figure}[ht]
\centering
\includegraphics[width=6cm]{./figures/7d4f4b21f0b3ad05.png}
\caption{塔勒斯定理:如果AC是直径,B是直径圆上的一点,则角∠ABC是直角。} \label{fig_Thales_1}
\end{figure}
在几何学中,塔勒斯定理指出,如果A、B和C是圆上的不同点,其中AC是直径,则角∠ABC是直角。塔勒斯定理是内切角定理的一个特例,并作为欧几里得《几何原本》第三卷第31命题的一部分被提及和证明。[1] 通常认为该定理归功于米利都的塔勒斯,但有时也归功于毕达哥拉斯。
\subsection{历史}
\begin{figure}[ht]
\centering
\includegraphics[width=6cm]{./figures/290ee9c3032feba1.png}
\caption{} \label{fig_Thales_2}
\end{figure}
巴比伦数学家在希腊数学家证明之前,就已经知道这个定理的特殊情况。[2]

米利都的泰勒斯(公元前6世纪初)传统上被认为是第一个证明该定理的人;然而,到了公元前5世纪,泰勒斯的著作已不复存在,后来的历史学家根据听说和推测,将许多发明和思想归于泰勒斯和毕达哥拉斯等智慧人物。[3][4] 普罗克鲁斯(公元5世纪)和狄奥根尼·拉厄尔修(公元3世纪)都提到了泰勒斯,并且记录了帕姆菲拉(公元1世纪)的话,称泰勒斯“是第一个在圆内刻画直角三角形的人”。[5]