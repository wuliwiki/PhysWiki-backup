% 欧几里得(综述)
% license CCBYSA3
% type Wiki

本文根据 CC-BY-SA 协议转载翻译自维基百科\href{https://en.wikipedia.org/wiki/Euclid}{相关文章}。

\begin{figure}[ht]
\centering
\includegraphics[width=6cm]{./figures/7848a3bb491b282e.png}
\caption{《欧几里得》,由朱塞佩·德·里贝拉(Jusepe de Ribera)创作,约1630–1635年。} \label{fig_Euclid_1}
\end{figure}
欧几里得(/ˈjuːklɪd/;古希腊语:Εὐκλείδης;约公元前300年活跃)是一位古希腊数学家,主要从事几何学和逻辑学的研究。被称为“几何学之父”,他最著名的成就是《几何原本》这部著作,它奠定了几何学的基础,直到19世纪初期,这些基础一直主导着该领域。他的体系,现被称为欧几里得几何学,结合了之前希腊数学家的理论创新与综合,包括厄多克索斯(Eudoxus of Cnidus)、希波克拉底(Hippocrates of Chios)、泰勒斯(Thales)和提阿托斯(Theaetetus)等人的理论。与阿基米德和佩尔加的阿波罗尼斯一起,欧几里得通常被认为是古代最伟大的数学家之一,也是数学史上最具影响力的人物之一。

关于欧几里得的生平知之甚少,绝大多数资料来源于几百年后学者普罗克洛斯(Proclus)和帕普斯(Pappus)的记载。中世纪的伊斯兰数学家创造了一个富有幻想色彩的生平,而中世纪拜占庭学者和早期文艺复兴学者则误将他与早期哲学家梅加拉的欧几里得混淆。现在普遍认为,欧几里得大部分时间都在亚历山大里亚度过,约生活在公元前300年,介于柏拉图的学生与阿基米德之间。也有一些猜测认为,欧几里得曾在柏拉图学园学习,后来在穆萨尤姆教授数学;他被认为是将早期的柏拉图学派传统与后来的亚历山大里亚学派传统连接起来的桥梁。

在《几何原本》中,欧几里得通过少数公理推导出定理。他还撰写了关于透视学、圆锥曲线、球面几何、数论和数学严谨性的著作。除了《几何原本》,欧几里得还写了光学领域的基础性著作《光学》,以及其他较不为人知的作品,如《数据》和《现象》。关于《几何分割》和《镜面反射学》是否为欧几里得所作,学术界仍有争议。欧几里得还被认为撰写了许多已经失传的作品。
\subsection{生命}  
\subsubsection{传统叙述}
\begin{figure}[ht]
\centering
\includegraphics[width=8cm]{./figures/0d7a20b6f91e0fa9.png}
\caption{拉斐尔在《雅典学派》(1509–1511)中的欧几里得形象细节,展示了他在教授学生。} \label{fig_Euclid_2}
\end{figure}
“Euclid”这个英语名字是古希腊名字Eukleídes(Εὐκλείδης)的英文化版本。[4][a] 它来源于“eu-”(εὖ;意为“好”)和“klês”(-κλῆς;意为“名声”),意思是“著名的,荣耀的”。[6] 在英语中,"Euclid"通过转喻有时指代他最著名的作品《几何原本》,或其副本,[5] 有时也被当作“几何”的同义词。[2]

与许多古希腊数学家一样,关于欧几里得的生平细节大多未知。[7] 他被认为是四部大部分存世的著作的作者——《几何原本》、 《光学》、 《数据》、 《现象》——但除此之外,关于他的确切信息几乎没有。[8][b] 传统的叙述主要依赖于公元5世纪普罗克洛斯在其《欧几里得《几何原本》第一卷注释》中的记载,以及公元4世纪初亚历山大的帕普斯的一些轶事。[4][c]

根据普罗克洛斯的说法,欧几里得生活在柏拉图(公元前347年去世)的几位追随者之后,并且在数学家阿基米德(公元前287年–公元前212年)之前;[d] 具体来说,普罗克洛斯将欧几里得置于托勒密一世统治时期(公元前305/304–282年)。[7][8][e] 欧几里得的出生日期不详;一些学者估计大约在公元前330年[11][12] 或公元前325年,[2][13] 但其他学者则避免做出推测。[14] 假设他是希腊血统,[11] 但他的出生地未知。[15][f] 普罗克洛斯认为欧几里得遵循柏拉图的传统,但没有确凿的证据可以证实这一点。[17] 他不太可能与柏拉图同代,因此通常推测他是柏拉图的学生,在雅典的柏拉图学园接受教育。[18] 历史学家托马斯·希思支持这一理论,指出大多数有能力的几何学家都生活在雅典,包括许多欧几里得依赖的前人的工作;[19] 历史学家米哈利斯·西亚拉罗斯认为这只是一个猜测。[4][20] 无论如何,欧几里得的著作内容表明他熟悉柏拉图几何学的传统。[11]

在《帕普斯集》中,帕普斯提到阿波罗尼乌斯曾与欧几里得的学生一起在亚历山大学习,这表明欧几里得曾在那里工作并创立了一个数学传统。[8][21][19] 该城市由亚历山大大帝于公元前331年建立,[22] 托勒密一世自公元前306年起统治,使其在亚历山大帝国分裂后的混乱战争中拥有相对的稳定性。[23] 托勒密开始了希腊化进程,并委托建造许多建筑,建立了庞大的穆塞翁学术机构,这是当时的一个领先教育中心。[15][g] 推测欧几里得是穆塞翁最早的学者之一。[22] 欧几里得的死亡日期不详;有学者推测他大约在公元前270年去世。[22]