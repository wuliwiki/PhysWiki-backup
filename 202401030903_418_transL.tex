% Transformers 库
% keys Python|机器学习|库
% license Usr
% type Note

\begin{issues}
\issueDraft
\end{issues}

transformers 库是要使用 \href{https://huggingface.co}{HuggingFace 网站}的已经训练好的 torch 网络参数或 tensorflow 网络参数必备的一个包(现在访问 huggingface 网站需要科学上网)。

安装可以使用 pip 命令直接安装:\verb`pip install transformers`,使用 huggingface 网站提供的数据集或神经网络参数需要保证电脑里已经装有 PyTorch。(CPU 版的 torch 可以使用命令 \verb`pip install torch` 安装。安装包较大,建议使用国内 pypi 的镜像站。)

特别的,有的网络参数(.h5 文件)则是 tensorflow 的,那么需要保证电脑已经装有 tensorflow 包。

对于网站上的每个网络参数,介绍一些常见的文件:
\begin{enumerate}
\item \verb`readme.md` 一类文件:介绍文档。
\item \verb`license`: 这文件存储开源协议。
\item \verb`config.json`:这个文件存储了网络的一些配置。
\item 分词配置:\verb`tokenizer.json`、\verb`tokenizer_config.json`、\verb`vocab`,其中 \verb`vocab` 文件可能是 txt 或 json 格式。这些文件给分词器使用。
\item \verb`pytorch_model.bin`:存储了 PyTorch 网络的训练好的参数。
\item \verb`tf_model.h5`:存储了 tensorflow 网络的训练好的参数。
\end{enumerate}

对于其中的 \verb`config.json`,我们可以通过其看出这个网络的一些配置,一个典型的配置文件以及各配置意义如下:
\begin{lstlisting}[language=json]
{
  "architectures": [
    "BertForSequenceClassification"
  ],
  "attention_probs_dropout_prob": 0.1,
  "directionality": "bidi",
  "hidden_act": "gelu",
  "hidden_dropout_prob": 0.1,
  "hidden_size": 768,
  "initializer_range": 0.02,
  "intermediate_size": 3072,
  "layer_norm_eps": 1e-12,
  "max_position_embeddings": 512,
  "model_type": "bert",
  "num_attention_heads": 12,
  "num_hidden_layers": 12,
  "pad_token_id": 0,
  "pooler_fc_size": 768,
  "pooler_num_attention_heads": 12,
  "pooler_num_fc_layers": 3,
  "pooler_size_per_head": 128,// 每个注意力头的 size
  "pooler_type": "first_token_transform",// pooler 层类型,网上介绍很少
  "type_vocab_size": 2,// 词汇表类别,默认为 2
  "vocab_size": 21128
}
\end{lstlisting}

各个配置意义如下:
\begin{enumerate}
\item \verb`architectures`:模型的名称;
\item \verb`attention_probs_dropout_prob`:注意力的 dropout,默认为 0.1;
\item \verb`directionality`:文字编码方向采用的算法,一般为 bidi;
\item \verb`hidden_act`:编码器内激活函数,默认"gelu",还可为"relu"、"swish"或 "gelu_new";
\item \verb`hidden_dropout_prob`:词嵌入层或编码器的 dropout 配置,默认为 0.1;
\item \verb`hidden_size`:编码器内隐藏层神经元数量,默认 768;
\item \verb`initializer_range`:神经元权重的标准差,默认为 0.02;
\item \verb`intermediate_size`:编码器内全连接层的输入维度,默认 3072;
\item \verb`layer_norm_eps`:layer normalization 的 epsilon 值,默认为 1e-12;
\item \verb`max_position_embeddings`:模型使用的最大序列长度,默认为 512;
\item \verb`model_type`:模型的类型,一般是 bert;
\item \verb`num_attention_heads`:编码器内注意力头数,默认 12;
\item \verb`num_hidden_layers`:编码器内隐藏层层数,默认 12;
\item \verb`pooler_fc_size`:Pooler 层(相当于一个全连接层,作为分类器解决序列级 NLP 任务)的大小,默认也为 768;
\item \verb`pooler_num_attention_heads`:Pooler 层注意力头,默认 12;
\item \verb`pooler_num_fc_layers`:Pooler 连接层数,默认 3;
\item \verb`pooler_size_per_head`:每个注意力头的大小;
\item \verb`pooler_type`:Pooler 层类型;
\item \verb`type_vocab_size`:词汇表的类型,默认是 2;
\item \verb`vocab_size`:
\end{enumerate}