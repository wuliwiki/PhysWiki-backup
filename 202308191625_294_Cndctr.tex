% 导体
% keys 电势能|等势体|电场|电荷
% license Xiao
% type Tutor

\begin{issues}
\issueOther{需要与 “导体的静电平衡” 整合}
\end{issues}

\pentry{电场的高斯定律\upref{EGauss}}

\subsection{导体的基本性质}
导体的一个显著性质是导体内部的电荷可以在电场的驱动下自由移动;同时,导体可以在外电场中被极化而产生感应电荷与感应电场;导体也可以携带自由电荷。根据这些性质,可以进一步推导出静电平衡时的导体的如下性质\cite{GriffE}:

\begin{itemize}
\item 导体产生的感应电荷总量为0 
\item 导体内部的电场处处为0(感应电场与外电场相互抵消) $\bvec E_{in} = \bvec 0$
\item 导体内部电荷密度处处为0,所有电荷只分布在导体表面 $\rho=0$
\item 导体表面是等势面;导体表面的电势处处相同 $\varphi=C$
\item 导体表面的电场始终垂直于导体表面
\item 导体表面的切向(平行于表面)电场为0 $\bvec E_\parallel = \bvec 0$
\item 导体表面的法向(垂直于表面)电场 $E_\perp = \frac{\sigma}{\epsilon_0}$,其中$\sigma$是导体的面电荷密度
\end{itemize}

此处的电场E是导体中感应电荷产生的电场与外加电场之和。

\subsection{部分性质的不严格证明}
若导体内部存在电场,那么电场将会驱动导体内部的自由电荷继续定向运动,直到感应电场与外电场相互抵消。

如果导体内部存在任何净电荷, 在净电荷周围做一个高斯面,运用电场的高斯定理计算电场, 则高斯面上必然会有电场, 这就违背了静电平衡的假设。

根据$\bvec E=-\nabla\varphi$,由于E=0,即$\varphi$的导数处处为0,因此导体在静电平衡时处处电势相等, 是一个\textbf{等势体}。
\cite{GriffE}

\begin{example}{匀强电场中的球形导体}
匀强电场 $\bvec E_0$ 中有一个半径为 $R$ 的导体球处于静电平衡, 求导体球表面的电荷分布。

我们取电场的方向为球坐标的极轴, 由问题的对称性, 电荷分布关于极轴对称, 所以电荷面密度可以表示为 $\sigma(\theta)$。 我们现在需要寻找一个 $\sigma(\theta)$ 使得表面电荷在导体球内部可以产生与外电场 $\bvec E_0$ 等大反向的匀强电场。

\begin{figure}[ht]
\centering
\includegraphics[width=5cm]{./figures/77062793195a4bd9.pdf}
\caption{两个重叠的均匀带电导体球} \label{fig_Cndctr_1}
\end{figure}

这里介绍一种巧妙的办法, 我们先来看以下一个模型。 先假设不存在外电场, 令两个电荷密度分别为 $\pm\rho$ 的均匀带电球在极轴方向错开距离 $d$ ($d < R$) 两球重合的部分正负电荷叠加, 净电荷为零。 令 $\pm\rho$ 的球心指向空间中某点的矢量分别为 $\bvec r_\pm$, 根据\autoref{ex_EGauss_30}~\upref{EGauss} 中的结论, 我们知道两球产生的电场分别为
\begin{equation}
\bvec E_\pm = \frac{\rho}{3\epsilon_0}\bvec r_\pm~.
\end{equation}
令 $-\rho$ 的球心指向 $+\rho$ 球心的矢量为 $\bvec d$, 有 $\bvec r_+ - \bvec r_- = -\bvec d$, 所以两球重合区域中任意一点的总电场为
\begin{equation}
\bvec E = \bvec E_+ + \bvec E_- = \frac{\rho}{3\epsilon_0}(\bvec r_+ - \bvec r_-) = -\frac{\rho}{3\epsilon_0}\bvec d~.
\end{equation}
这是一个匀强电场, 恰好符合我们的要求。 现在就可以令 $\bvec d\to \bvec 0$, 然后计算电荷面密度, 面密度与厚度 $h$ 成正比。 由\autoref{fig_Cndctr_2} 可得 $h = d\cos\theta$。

\begin{figure}[ht]
\centering
\includegraphics[width=6.5cm]{./figures/ccf3ca5e3555555b.pdf}
\caption{两个重叠的均匀带电导体球 $\theta$} \label{fig_Cndctr_2}
\end{figure}

所以面密度就是 $\sigma(\theta) = \rho h = \rho d\cos\theta = 3\epsilon_0 E \cos\theta$。 若 $\bvec E$ 要与外电场 $\bvec E_0$ 抵消, 令 $E = E_0$ 可得
\begin{equation}
\sigma(\theta) = 3\epsilon_0 E_0 \cos\theta~.
\end{equation}
\end{example}

