% 伴随映射
% keys 线性代数|伴随映射|伴随算子|伴随变换|adjoint operator|adjoint map|adjoint transformation|dual map|对偶映射
% license Xiao
% type Tutor

\pentry{线性映射\upref{LinMap}, 对偶空间\upref{DualSp}}


\subsection{伴随映射的概念}


\subsubsection{定义}

给定线性空间,则可由此导出其对偶空间。如果给定两个线性空间$V$和$W$之间的线性映射,则我们可以自然导出对应的$W^*$和$V^*$之间的线性映射,即所谓的伴随映射。



\begin{definition}{伴随映射}\label{def_AdjMap_1}

\begin{figure}[ht]
\centering
\includegraphics[width=3.5cm]{./figures/22f579955f587872.pdf}
\caption{伴随映射的概念。} \label{fig_AdjMap_1}
\end{figure}

给定域$\mathbb{F}$上的线性空间$V$和$W$,对于\textbf{线性映射}$f:V\to W$,定义其伴随映射为$f^*:W^*\to V^*$,使得对于任意$\bvec{w}\in W^*$和$\bvec{v}\in V$都有
\begin{equation}
\langle f(\bvec{v}), \bvec{w} \rangle = \langle \bvec{v}, f^*(\bvec{w}) \rangle~. 
\end{equation}
其中$\langle *, * \rangle$表示对偶向量之间的相互作用。
\end{definition}

注意,定义伴随映射时,我们已知的两个元素$\bvec{v}$和$\bvec{w}$分别位于\autoref{fig_AdjMap_1} 的左上角和右下角,它们之间不能相互作用。但是,$f$能够把$\bvec{v}$挪到左下角,使得二者可以相互作用;而其对偶映射$f^*$的作用则是反过来,把$\bvec{w}$挪到右上角,同样使得二者可以相互作用。



\subsubsection{矩阵表示}


当给定$V$和$W$基\footnote{注意,任意基都可以,没有正交性等要求。}$\{\bvec{e}_i\}_{i=1}^n$和$\{\bvec{\varepsilon}_j\}_{j=1}^m$后,$f$可以表示为矩阵
\begin{equation}
\begin{pmatrix}
a^1_1&a^1_2&\cdots&a^1_n\\
a^2_1&a^2_2&\cdots&a^2_n\\
\vdots&\vdots&\ddots&\vdots\\
a^m_1&a^m_2&\cdots&a^m_n
\end{pmatrix}~, 
\end{equation}
其中$f(\bvec{e}_i)=\sum_{j=1}^m a^j_i\bvec{\varepsilon}_j$。

令$\{\bvec{\theta}_i\}_{i=1}^n$和$\{\bvec{\tau}_j\}_{j=1}^m$分别是这两个基的\textbf{对偶基},则$f^*$在这两组基下可以表示为矩阵
\begin{equation}
\begin{pmatrix}
b^1_1&b^1_2&\cdots&b^1_m\\
b^2_1&b^2_2&\cdots&b^2_m\\
\vdots&\vdots&\ddots&\vdots\\
b^n_1&b^n_2&\cdots&b^n_m
\end{pmatrix}~, 
\end{equation}
其中$f^*(\bvec{\tau}_j)=\sum_{i=1}^n b^i_j\bvec{\theta}_i$。

由伴随映射的定义,
\begin{equation}
\langle f(\bvec{e}_i), \bvec{\tau}_j \rangle =\langle \bvec{e}_i, f^*(\bvec{\tau}_j) \rangle ~, 
\end{equation}
因此
\begin{equation}
\langle \sum_{k=1}^m a^k_i\bvec{\varepsilon}_k, \bvec{\tau}_j \rangle =\langle \bvec{e}_i, \sum_{k=1}^n b^k_j\bvec{\theta}_k \rangle ~. 
\end{equation}
由对偶基的\autoref{def_DualSp_4}~\upref{DualSp}可知
\begin{equation}
\sum_{k=1}^m a^k_i\delta_{kj} = \sum_{k=1}^m b^k_j\delta_{ik}~, 
\end{equation}
这等于说$a^j_i=b^i_j$,也就是$f$和$f^*$的矩阵互为\textbf{转置}。


%考虑到自身的伴随变换时,为了保证选取的基一致,要取标准正交基。换句话说,要求互反基等于自身。




\subsection{伴随变换}



$V$和$V^*$之间没有天然的一一对应,如果规定了一个$V\to V^*$的线性同构,其实就等价于规定了$V$上的一个\textbf{非退化双线性形式}\upref{bilinF},但现在你可以局限在一种情形来考虑:\textbf{内积}\upref{InerPd}。

给定$V$上的内积$g$,则任取$\bvec{u}\in V$,我们都可以唯一确定$\bvec{u}$对应的$\bvec{\omega}\in V^*$,使得
\begin{equation}
g(\bvec{u}, \bvec{v}) = \langle \bvec{\omega}, \bvec{v} \rangle~
\end{equation}
对于任意$\bvec{v}\in V$恒成立。这样,我们就用$g$确定了一个\textbf{同构映射}$\sigma_g:V\to V^*$,使得$\sigma_g(\bvec{u})=\bvec{\omega}$。




取域$\mathbb{F}$上的线性空间$V$,设$f:V\to V$是线性变换,那么按照\autoref{def_AdjMap_1} ,存在$f$的伴随映射$f^*:V^*\to V^*$。利用$\sigma_g$可以导出$V\to V$的映射$\sigma_g^{-1}\circ f^*\circ \sigma_g$。按照伴随映射的定义得
\begin{equation}
\begin{aligned}
\langle f(\bvec{v}), \sigma_g(\bvec{u}) \rangle = \langle \bvec{v} , f^*\circ \sigma_g(\bvec{u}) \rangle
\end{aligned}~
\end{equation}
对于任意$\bvec{u}, \bvec{v}\in V$成立。又按照$\sigma_g$的定义得























