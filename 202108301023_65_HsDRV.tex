% 离散型随机变量(高中)
% keys 高中|离散型随机变量

\subsection{定义}
试验结果可以用一个变量 $X$ 来表示,并且 $X$ 是随着试验的结果的不同变化的,我们把这样的变量 $X$ 叫做一个\textbf{随机变量(random variable)}.随机变量常用大写字母 $X,Y,\cdots$ 表示.

如果随机变量  $X$ 的所有可能的取值都能一一列举出来,则称 $X$ 为\textbf{离散型随机变量(discrete random variable)}.

\subsection{离散型随机变量的分布列}
要掌握一个离散型随机变量 $X$ 的取值规律,必须知道:
\begin{enumerate}
\item $X$ 所有可能取其
\end{enumerate}

\begin{table}[h]
\centering
\caption{分布列}\label{HsDRV_tab1}
\begin{tabular}{|c|c|c|c|c|c|c|}
\hline
$X$ & $x_1$ & $x_2$ & $\cdots$ & $x_i$ & $\cdots$ & $x_n$ \\
\hline
$P$ & $p_1$ & $p_2$ & $\cdots$ & $p_i$ & $\cdots$ & $p_n$ \\
\hline
\end{tabular}
\end{table}

%\begin{table}[h]
%	\centering
%	\caption{分布列}
%    \label{HsDRV_tab}
%	\begin{tabular}{|c|c|c|c|}
%    \end{tabular}
%\end{table}
