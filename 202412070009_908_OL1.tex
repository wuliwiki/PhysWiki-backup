% 莱昂哈德·欧拉(综述)
% license CCBYSA3
% type Wiki

本文根据 CC-BY-SA 协议转载翻译自维基百科\href{https://en.wikipedia.org/wiki/Leonhard_Euler}{相关文章}。

莱昂哈德·欧拉(Leonhard Euler,发音:/ˈɔɪlər/ OY-lər;[b] 德语:[ˈleːɔnhaʁt ˈʔɔʏlɐ] ⓘ,瑞士标准德语:[ˈleɔnhard ˈɔʏlər];1707年4月15日 – 1783年9月18日)是瑞士数学家、物理学家、天文学家、地理学家、逻辑学家和工程师。他是图论和拓扑学的创始人,并在其他多个数学分支(如解析数论、复分析和微积分)中做出了开创性和深远的发现。他引入了许多现代数学术语和符号,包括数学函数的概念。他还以在力学、流体动力学、光学、天文学和音乐理论等领域的贡献而闻名。

欧拉被认为是历史上最伟大、最多产的数学家之一,也是18世纪最伟大的数学家。许多在欧拉去世后才产生的伟大数学家都承认他在这一领域的重要性,正如他们的名言所示:皮埃尔-西蒙·拉普拉斯曾通过一句话表达欧拉对数学的影响:“读欧拉,读欧拉,他是我们的导师。”卡尔·弗里德里希·高斯写道:“研究欧拉的作品将是学习数学各个领域的最佳学校,其他任何东西都无法替代它。”欧拉的866篇论文和他的信件正在被收集成《欧拉全集》(Opera Omnia Leonhard Euler),完成后将包含81卷四开本。欧拉大部分成年生活都在俄罗斯圣彼得堡和普鲁士首都柏林度过。

欧拉还被认为是第一个推广使用希腊字母π(小写pi)来表示圆的周长与直径的比率的人,以及第一个使用符号f(x)来表示函数值的人。他还使用字母i表示虚数单位√(-1),使用希腊字母Σ(大写sigma)表示求和,使用希腊字母Δ(大写delta)表示有限差分,使用小写字母表示三角形的边,使用大写字母表示角度。他给出了常数e的定义,它是自然对数的底数,现在被称为欧拉数。

欧拉还被认为是第一个发展图论的人(部分因为他解决了“柯尼斯堡七桥问题”,这也被认为是拓扑学的第一个实际应用)。他还因解决多个未解的数论和分析学问题而声名远播,包括著名的巴塞尔问题。欧拉还被誉为发现了多面体的顶点和面数之和减去边数等于2,这个数字现在被称为欧拉示性数。在物理学领域,欧拉将牛顿的物理定律重新表述为新的定律,并在他的两卷本著作《力学》中更好地解释了刚体的运动。他还为固体物体的弹性变形研究做出了重大贡献。
\subsection{早期生活}
莱昂哈德·欧拉于1707年4月15日出生在巴塞尔,父亲保罗三世·欧拉是改革宗教会的牧师,母亲玛格丽特(娘家姓布鲁克尔),她的祖先包括许多著名的古典学者。[16] 欧拉是家中的长子,下面有两个妹妹,安娜·玛利亚和玛利亚·玛格达莱娜,以及一个弟弟约翰·海因里希。[17][16] 在欧拉出生后不久,欧拉一家从巴塞尔搬到瑞士的里恩镇,父亲成为当地教堂的牧师,而欧拉则在这里度过了大部分童年。[16]

从小,欧拉就接受了父亲的数学教育,父亲曾在巴塞尔大学向雅各布·伯努利学习过一些课程。大约在八岁时,欧拉被送到外祖母家生活,并在巴塞尔的拉丁学校就读。此外,他还接受了年轻神学家约翰内斯·布尔卡特的私人辅导,布尔卡特对数学有浓厚的兴趣。[16]

1720年,欧拉13岁时进入巴塞尔大学。[7] 在当时,年轻就读大学并不罕见。[16] 数学基础课程由已故的雅各布·伯努利的弟弟约翰·伯努利教授(雅各布·伯努利曾是欧拉父亲的老师)讲授。约翰·伯努利和欧拉很快熟识。欧拉在自传中描述了伯努利的教导:[18]

“著名教授约翰·伯努利……特别乐意帮助我在数学科学上取得进展。然而,他由于日程繁忙,拒绝了私人的授课请求。然而,他给了我一个更为有益的建议,就是让我自己去掌握一些更为艰深的数学书籍,并以极大的勤奋把它们研读一遍,如果遇到疑难或困难,他每周六下午都会开放时间免费为我解答,他如此慷慨地评论我的问题,以至于当他解答我一个疑问时,十个疑问随之消失,这无疑是数学科学进步的最佳方法。”

正是在这段时间里,在伯努利的支持下,欧拉得到了父亲的同意,决定成为一名数学家,而不是继续成为牧师。[19][20]

1723年,欧拉获得哲学硕士学位,并撰写了一篇论文,比较了笛卡尔和牛顿的哲学。[16] 此后,他又进入了巴塞尔大学的神学系。[20]

1726年,欧拉完成了一篇关于声音传播的论文《De Sono》[21][22],试图通过这篇论文获得巴塞尔大学的职位,但未成功。[23] 1727年,欧拉第一次参加了巴黎科学院的奖学金竞赛(该竞赛自1720年起每年举办,后来改为每两年一次)[24]。当年题目是寻找在船上最好的桅杆安放方式。皮埃尔·布盖,后被称为“海军建筑学之父”,获得了第一名,欧拉则获得了第二名。[25] 在接下来的几年里,欧拉共参加了15次该竞赛,赢得了其中的12次。[24][25]
\subsection{职业生涯}
\subsubsection{圣彼得堡}
\begin{figure}[ht]
\centering
\includegraphics[width=8cm]{./figures/56057be7fb169013.png}
\caption{1957年苏联邮票,纪念欧拉250周年诞辰。邮票上的文字写道:“伟大数学家、院士莱昂哈德·欧拉诞辰250周年”。} \label{fig_OL1_1}
\end{figure}
约翰·伯努利的两个儿子,丹尼尔和尼古劳斯,于1725年进入圣彼得堡的帝国俄罗斯科学院工作,并向欧拉保证,一旦有职位空缺,他们会推荐他。1726年7月31日,尼古劳斯在俄罗斯逗留不到一年便因阑尾炎去世。当丹尼尔接替弟弟的位置,担任数学/物理学部门的职务时,他建议将他空出的生理学职位提供给他的朋友欧拉。1726年11月,欧拉热切接受了这个提议,但他推迟了前往圣彼得堡的行程,因为他在未成功申请巴塞尔大学的物理学教授职位时,仍想争取机会。

欧拉于1727年5月抵达圣彼得堡。他从科学院医务部门的初级职务晋升为数学部门的职位。他与丹尼尔·伯努利同住,并与其紧密合作。欧拉学会了俄语,适应了圣彼得堡的生活,并在俄罗斯海军担任医务职务。

圣彼得堡的科学院由彼得大帝建立,旨在改善俄罗斯的教育水平,并弥补与西欧的科学差距。因此,它对外国学者,如欧拉,具有特别的吸引力。学院的赞助人凯瑟琳一世继续执行她已故丈夫的进步政策,但在欧拉到达圣彼得堡之前去世。随后,俄罗斯的保守派贵族在12岁彼得二世登基后获得了权力。贵族们对学院的外国科学家心存疑虑,削减了欧拉和他的同事们的资助,并禁止外国和非贵族学生进入中学和大学。

1730年彼得二世去世后,情况略有改善,受到德国影响的安娜·伊凡诺夫娜掌权。欧拉迅速在科学院中晋升,并于1731年被任命为物理学教授。他还离开了俄罗斯海军,拒绝了晋升为中尉的提议。两年后,丹尼尔·伯努利因受不了在圣彼得堡的审查和敌意,离开了巴塞尔。欧拉继任了数学系主任的职务。1734年1月,他与乔治·格塞尔的女儿卡塔里娜·格塞尔结婚。

1740年,弗雷德里希二世曾试图招募欧拉加入他新成立的柏林科学院,但欧拉最初更愿意留在圣彼得堡。然而,在安娜皇后去世后,弗雷德里希二世同意支付他1600枚埃库(与欧拉在俄罗斯的薪水相同),欧拉同意前往柏林。1741年,他请求离开圣彼得堡,前往柏林,理由是他的视力需要更温和的气候。俄罗斯科学院同意了他的请求,并将每年支付他200卢布,作为其积极成员之一。
\subsubsection{柏林}  
由于担心俄罗斯的持续动荡,欧拉于1741年6月离开圣彼得堡,接受了腓特烈大帝提供的柏林科学院职位。他在柏林生活了25年,期间写了数百篇文章。1748年,他的《无穷分析引论》一书出版,1755年,关于微分学的《微积分学原理》也出版了。1755年,他被选为瑞典皇家科学院和法国科学院的外籍成员。欧拉在柏林的著名学生包括后来被认为是第一位俄罗斯天文学家的斯捷潘·鲁莫夫斯基。1748年,欧拉拒绝了巴塞尔大学提供的继承已故约翰·伯努利的职位。1753年,他在查尔滕堡购买了一座房子,与家人和寡母一起居住。

欧拉成为了勃兰登堡-施韦德的弗里德里克·夏洛特的导师,这位公主是安哈尔特-德绍的弗雷德里克的侄女。1760年代初,他向她写了超过200封信,这些信后来被编成《欧拉给德国公主的自然哲学信件》一书。该书内容涉及欧拉对物理学和数学的各种阐述,并为我们提供了对欧拉个性和宗教信仰的宝贵见解。此书被翻译成多种语言,在欧洲和美国出版,并且比他的任何数学著作更广为流传。《信件》的受欢迎程度证明了欧拉能够有效地向普通读者传达科学问题,这对一位专注于研究的科学家来说是一项罕见的能力。

尽管欧拉为学会的声望做出了巨大贡献,并且曾被让·勒·朗·达朗贝尔推荐为学会会长候选人,但腓特烈二世最终自任会长。普鲁士国王在宫廷内有着庞大的知识分子圈子,他认为欧拉在数字和图形之外的事务上知识有限且不够精明。欧拉是一个简单、虔诚的宗教信徒,他从未质疑现有的社会秩序或传统信仰。在许多方面,他与伏尔泰截然相反,伏尔泰在腓特烈的宫廷中享有崇高的声望。欧拉不是一个擅长辩论的人,经常坚持讨论自己并不擅长的主题,这使得他成为伏尔泰讽刺的常客。腓特烈二世也曾对欧拉的工程实践能力感到失望,他曾说:

“我曾想在我的花园里建一个喷水池:欧拉计算了将水泵送到水库所需的轮子力量,从水库通过渠道将水喷出到无忧宫。我的水轮是按几何学原理设计的,结果连水池附近五十步以内都无法扬起一口水。虚荣啊!几何的虚荣!”

然而,从技术角度来看,这种失望几乎是没有根据的。欧拉的计算看起来是正确的,尽管他与腓特烈及建造喷泉的人们之间的互动可能存在问题。

在柏林的这些年里,欧拉始终与圣彼得堡的科学院保持密切联系,并且在俄罗斯发表了109篇论文。他还帮助了圣彼得堡科学院的学生,并在他位于柏林的家中接待过俄罗斯的学生。1760年,七年战争爆发,欧拉在查尔滕堡的农场被俄罗斯军队洗劫。得知此事后,伊凡·彼得罗维奇·萨尔季科夫将赔偿费用支付给欧拉,俄罗斯女皇伊丽莎白还额外支付了4000卢布——这在当时是巨额赔偿。欧拉决定于1766年离开柏林,返回俄罗斯。

在柏林的这些年(1741-1766),欧拉处于生产力的巅峰时期。他写了380篇作品,其中275篇发表。这包括125篇柏林科学院的论文和超过100篇发送给圣彼得堡科学院的论文,后者仍然保留着他作为成员,并支付给他年薪。欧拉的《无穷分析引论》在1748年分两部分出版。除了个人研究,欧拉还管理着科学院的图书馆、天文台、植物园,并参与日历和地图的出版,这些都为科学院提供了收入。他甚至参与了无忧宫喷泉的设计工作。
\subsubsection{返回俄罗斯}  
在叶卡捷琳娜大帝即位后,俄罗斯的政治局势趋于稳定,因此在1766年,欧拉接受了邀请,返回圣彼得堡科学院。他的条件相当苛刻——年薪3000卢布、为妻子提供养老金,以及为他的儿子们提供高职保证。在大学里,他得到了学生安德斯·约翰·莱克塞尔的帮助。1771年,欧拉在圣彼得堡的家被一场火灾摧毁。
\subsection{个人生活}  
1734年1月7日,欧拉与凯瑟琳娜·格塞尔(Katharina Gsell,1707-1773)结婚,她是圣彼得堡科学院画家乔治·格塞尔(Georg Gsell)的女儿。[33] 这对年轻夫妻在涅瓦河边购买了一栋房子。

他们共有十三个孩子,其中只有五个活到了童年,[55] 包括三子两女。[56] 他们的第一个儿子是约翰·阿尔布雷希特·欧拉(Johann Albrecht Euler),他的教父是克里斯蒂安·戈尔巴赫(Christian Goldbach)。[56]

在妻子于1773年去世三年后,[54] 欧拉与她的同母异父妹妹萨洛梅·阿比盖尔·格塞尔(Salome Abigail Gsell,1723-1794)结婚。[57] 这段婚姻持续到欧拉去世(1783年)。

他的兄弟约翰·海因里希于1735年定居圣彼得堡,并在学院担任画家。[34]

欧拉年轻时背诵了维吉尔的《埃涅阿斯纪》全诗,年老时他能够完整地背诵整首诗,并且能够准确说出他所学版本中每一页的第一句和最后一句。[58][59]
\subsubsection{视力恶化}  
欧拉的视力在他的数学生涯中逐渐恶化。1738年,在几乎因高热而去世三年后,[60] 他几乎失去了右眼的视力。欧拉将自己视力受损的原因归咎于他为圣彼得堡科学院绘制的地图,[61] 但导致他失明的具体原因仍然是一个猜测的问题。[62][63] 在德国期间,欧拉右眼的视力持续恶化,以至于弗雷德里克称他为“独眼巨人”(Cyclops)。欧拉对于视力丧失作出评论时表示:“现在我将有更少的干扰。”[61] 1766年,他的左眼被发现有白内障。虽然通过治疗暂时改善了他的视力,但并发症最终使他左眼几乎完全失明。[39] 然而,这一状况似乎对他的工作产出几乎没有影响。在抄写员的帮助下,欧拉在多个学科的工作产出反而有所增加;[64] 到1775年,他平均每周就能完成一篇数学论文。[39]
\subsubsection{去世}  
1783年9月18日,在圣彼得堡,欧拉与家人共进午餐后,正与安德烈·约翰·莱克塞尔讨论新发现的天王星及其轨道时,突然倒下并因脑溢血去世。[62] 雅各布·冯·斯泰林为俄罗斯科学院写了一篇简短的讣告,而欧拉的弟子之一,俄罗斯数学家尼古拉·福斯,则写了一篇更为详细的悼词,并在纪念会中发表了这篇悼词。[55] 法国数学家和哲学家马尔基·德·孔多塞尔在为法国科学院所写的悼文中说道:
\begin{figure}[ht]
\centering
\includegraphics[width=8cm]{./figures/27ee0055fcf31363.png}
\caption{欧拉的墓位于亚历山大·涅夫斯基修道院} \label{fig_OL1_2}
\end{figure}
他停止了计算,也停止了生活—— ... he ceased to calculate and to live.[65]

欧拉被埋葬在斯莫尔南斯克路德教墓地,位于瓦西里岛,与卡塔琳娜同葬。1837年,俄罗斯科学院为他安装了新的纪念碑,替换了那块已被植物覆盖的墓碑。为了纪念欧拉诞辰250周年,1957年,他的墓被迁移至亚历山大·涅夫斯基修道院的拉扎列夫墓地。[66]
\subsection{数学和物理学的贡献}
欧拉几乎涉及了数学的所有领域,包括几何学、微积分、三角学、代数学和数论,以及连续介质物理学、月球理论和其他物理学领域。他是数学史上的奠基人物;如果将他的著作全部印刷出来,这些作品(其中许多是基础性的重要著作)将占据60到80卷四开本的篇幅。[39] 欧拉的名字与大量的课题相关联。从1725年到1783年,欧拉的工作年均达到800页。他还写了超过4500封信件和数百篇手稿。据估计,欧拉是18世纪数学、物理学、力学、天文学和航海学的四分之一著作的作者。[14]
\subsubsection{数学符号}
欧拉通过他大量的教材引入并推广了几种符号惯例。最著名的是,他引入了函数的概念,并首次使用 \( f(x) \) 来表示函数 \( f \) 作用于自变量 \( x \)。他还引入了现代三角函数的符号、自然对数的底数 \( e \)(现在也称为欧拉数)、希腊字母 \( \Sigma \) 来表示求和以及字母 \( i \) 来表示虚数单位。[67] 希腊字母 \( \pi \) 用来表示圆的周长与直径的比值,也是欧拉所推广的,尽管最初是由威尔士数学家威廉·琼斯提出的。[68]
\subsubsection{分析}
无穷小微积分的发展是18世纪数学研究的前沿,伯努利家族——欧拉的家庭朋友——为该领域的早期进展做出了许多贡献。由于他们的影响,学习微积分成为欧拉工作的主要焦点。尽管欧拉的一些证明在现代数学严谨性标准下无法接受[69](特别是他对代数普遍性的依赖),但他的思想促成了许多重大进展。欧拉在分析学中以频繁使用并发展幂级数而著名,幂级数是将函数表示为无限项之和,[70] 如:
\[
e^{x}=\sum_{n=0}^{\infty} \frac{x^n}{n!} = \lim_{n \to \infty} \left( \frac{1}{0!} + \frac{x}{1!} + \frac{x^2}{2!} + \cdots + \frac{x^n}{n!} \right)~
\]
欧拉使用幂级数使他能够解决巴塞尔问题,即求所有自然数平方倒数的和,这一问题他在1735年解决(并在1741年给出了更为详尽的证明)。巴塞尔问题最初由皮耶特罗·门戈利在1644年提出,到了1730年代已成为一个著名的开放问题,被雅各布·伯努利推崇,并被当时许多顶尖数学家尝试解决但未成功。欧拉发现:
\[
\sum_{n=1}^{\infty} \frac{1}{n^2} = \lim_{n \to \infty} \left( \frac{1}{1^2} + \frac{1}{2^2} + \frac{1}{3^2} + \cdots + \frac{1}{n^2} \right) = \frac{\pi^2}{6}~
\]
欧拉还引入了常数:
\[
\gamma = \lim_{n \to \infty} \left( 1 + \frac{1}{2} + \frac{1}{3} + \frac{1}{4} + \cdots + \frac{1}{n} - \ln(n) \right) \approx 0.5772~
\]
这个常数现在被称为欧拉常数或欧拉-马谢罗尼常数,并研究了它与调和级数、伽玛函数和黎曼ζ函数值之间的关系。[73]
\begin{figure}[ht]
\centering
\includegraphics[width=6cm]{./figures/7c9e9a506604d285.png}
\caption{欧拉公式的几何解释} \label{fig_OL1_3}
\end{figure}
欧拉引入了指数函数和对数在解析证明中的使用。他发现了用幂级数表示各种对数函数的方法,并成功地定义了负数和复数的对数,从而极大地扩展了对数在数学应用中的范围。[67] 他还为复数定义了指数函数,并发现它与三角函数之间的关系。对于任何实数 φ(取弧度),欧拉公式表明复指数函数满足:
\[
e^{i\varphi} = \cos \varphi + i \sin \varphi~
\]
理查德·费曼称这公式为“数学中最著名的公式”。[74]

上述公式的一个特例被称为欧拉恒等式:
\[
e^{i\pi} + 1 = 0~
\]
欧拉通过引入伽玛函数[75][76],阐述了高阶超越函数的理论,并为解决四次方程引入了一种新方法。[77] 他找到了一种计算具有复数极限的积分的方法,预示着现代复分析的发展。他发明了变分法,并为在这一领域中将优化问题转化为微分方程的解,提出了欧拉-拉格朗日方程。

欧拉率先使用解析方法来解决数论问题。通过这一工作,他将数学的两个不同分支结合在一起,创造了一个新的研究领域——解析数论。在为这一新领域奠定基础时,欧拉创建了超几何级数、q级数、双曲三角函数和连分数的解析理论。例如,他利用调和级数的发散性证明了素数的无限性,并使用解析方法对素数的分布规律进行了初步探索。欧拉在这一领域的工作促成了素数定理的发展。[78]
\subsubsection{数论}  
欧拉对数论的兴趣可以追溯到他在圣彼得堡科学院的朋友克里斯蒂安·高尔巴赫的影响。[79] 欧拉在数论方面的早期工作主要基于皮埃尔·德·费尔马的研究。欧拉发展了费尔马的一些思想,并驳斥了他的一些猜想,例如费尔马猜想的形式为  \(2^{2^{n}} + 1\)(费尔马数)是素数的猜想。[80]  

欧拉将素数分布的性质与分析中的一些思想联系起来。他证明了素数的倒数之和是发散的。通过这一过程,他发现了黎曼ζ函数与素数之间的关系,这被称为黎曼ζ函数的欧拉积公式。[81]  

欧拉发明了欧拉φ函数 \( \phi(n) \),表示小于或等于整数n且与n互质的正整数的个数。利用这个函数的性质,他将费尔马小定理推广为现在所称的欧拉定理。[82] 他对完美数的理论作出了重要贡献,完美数自欧几里得以来就吸引着数学家的关注。他证明了偶完美数与梅森素数之间的关系是一一对应的,这一结果被称为欧几里得-欧拉定理。[83] 欧拉还猜想了二次互反律。这一概念被视为数论中的基本定理,他的思想为卡尔·弗里德里希·高斯的工作,特别是《算术研究》奠定了基础。[84] 到1772年,欧拉已证明 \( 2^{31} - 1 = 2,147,483,647 \) 是一个梅森素数,这可能是直到1867年为止已知的最大素数。[85]  

欧拉还对整数划分理论作出了重要贡献。[86]
\subsubsection{图论}
\begin{figure}[ht]
\centering
\includegraphics[width=6cm]{./figures/bc8bd9fc80eb3f4b.png}
\caption{} \label{fig_OL1_4}
\end{figure}
在1735年,欧拉提出了解决“柯尼斯堡七桥问题”的方案。[87] 这座普鲁士的柯尼斯堡市位于普雷格尔河上,市区包括两个大岛,这些岛屿通过七座桥梁与彼此及大陆连接。问题是要决定是否有可能找到一条路径,使得每座桥都被正好跨越一次,并最终回到起点。答案是否定的:没有欧拉回路。这个解法被认为是图论的第一个定理。[87]  

欧拉还发现了一个公式  
\( V - E + F = 2 \)  
它描述了一个凸多面体的顶点数、边数和面数之间的关系,[88] 因此也适用于平面图。这个公式中的常数现在被称为图(或其他数学对象)的欧拉示性数,它与该对象的属(genus)有关。[89] 这一公式的研究和推广,特别是由柯西[90] 和吕伊耶[91] 完成,标志着拓扑学的诞生。[88]