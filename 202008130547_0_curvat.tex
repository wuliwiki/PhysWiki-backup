% 曲率 曲率半径

\pentry{切线\upref{TanL}}

我们来看一个平面上的一个光滑曲线(即处处存在切线), 我们如何描述它某点处的弯曲程度呢? 一种常用方法是在这点附近取曲线的一小段, 然后做一个尽量与它吻合的圆, 当这小段的长度趋近于 0 时, 这个圆可以唯一确定. 我们把这个圆叫做\textbf{密切圆(osculating circle)}, 把密切圆的半径叫做曲线在该点的\textbf{曲率半径(radius of curvature)}, 曲率半径的倒数叫做\textbf{曲率(curvature)}.

我们先来看一个半径为 $R$ 的圆的一小段圆弧, 令其长度为 $l$. 现在作这段圆弧两端的切线 它的方向都会增加 $\Delta \theta$, 那么当
