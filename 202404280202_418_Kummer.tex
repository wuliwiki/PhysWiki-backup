% Kummer 函数(1F1)
% license Xiao
% type Tutor

\begin{issues}
\issueDraft
\end{issues}

\footnote{见 NIST \href{https://dlmf.nist.gov/13.2}{相关页面}。}\textbf{Kummer 函数} $M(a, b, z)$, 也叫合流超几何函数 $_1F_1(a, b, z)$, 是以下微分方程的解
\begin{equation}\label{eq_Kummer_1}
z\dv[2]{f}{z} + (b-z)\dv{f}{z} - a f = 0~.
\end{equation}

Kummer 函数是超几何函数\upref{HypGeo}的一个特例, 
\begin{equation}
M(a, b, z) = {_1F_1}(a; b; z) = \sum_{n=0}^\infty \frac{(a)_n}{(b)_n} \frac{z^n}{n!}~.
\end{equation}
其中 $(a)_n = a(a+1)\dots(a+n-1)$, 叫做 \textbf{Pochhammer 符号}。


Kummer 微分方程的形式在类氢原子束缚态的薛定谔方程中很常见,例如\autoref{eq_HWF_11}~\upref{HWF}:
\begin{equation}
\dv[2]{u}{\rho} + \qty[-1 - \frac{2\eta}{\rho} - \frac{l(l+1)}{\rho^2}]u = 0~.
\end{equation}
考察这微分方程在 $\rho \rightarrow 0$ 时的情形。此时变为
\begin{equation}
\dv[2]{u}{\rho} - \frac{l(l+1)}{\rho^2} u = 0 ~.
\end{equation}
从而 $u$ 的通解是 $u(\rho) = C_1 \rho^{l+1} + C_2 \rho^{-l}$ 的形式。er
