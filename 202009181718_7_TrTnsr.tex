% 张量的坐标变换
\begin{issues}
\issueTODO
\end{issues}

\pentry{张量\upref{Tensor}, 过渡矩阵\upref{TransM},爱因斯坦求和约定\upref{EinSum}}



在\textbf{张量}\upref{Tensor}词条中我们看到,张量表示为矩阵依赖于相关的各线性空间中基的选择.选定了线性空间的基以后,线性空间中的各向量都可以用其坐标所构成的列矩阵来表示;同样地,选定了两个线性空间各自的基以后,一个二阶张量也可以表示为一个矩阵.一般地,我们把选定了线性空间各自的基以后用来表示张量的矩阵,称作张量的\textbf{坐标},\textbf{分量}或\textbf{坐标分量}.

正如\textbf{爱因斯坦求和约定}\upref{EinSum}一节中所说,零阶张量的坐标表示为一个$1\times 1$矩阵,一阶张量的坐标表示为一个$n\times 1$或$1\times n$矩阵,一个二阶张量的坐标表示为一个$n\times n$矩阵.三阶及以上的张量就没法用我们已经熟知的矩阵来表示和运算了,因此我们不可避免地要应用爱因斯坦求和约定.从初学者的角度出发,本节将先用传统线性代数的语言描述一阶、二阶张量的坐标变换规则,然后使用爱因斯坦求和约定来描述一般的张量坐标变换规则.如果你还不熟悉爱因斯坦求和约定,建议详细阅读,比对两种表示方法的异同.

\subsection{传统记号表示的一阶、二阶张量坐标变换}
\subsubsection{一阶张量的坐标变换}
一阶张量是将一个向量映射为一个数,因此只涉及一个线性空间,最为简单.

给定$k$维线性空间$V$,及其上一个张量$f:V\rightarrow\mathbb{R}$.如果$V$的基是$\{\bvec{e}_1, \cdots\bvec{e}_k\}$,那么坐标为$\bvec{c}=(x_1\cdots x_k)\Tr$的向量$\bvec{v}$被映射为:
\begin{equation}
f(\bvec{v})=\sum\limits_{i=1}^k x_if(\bvec{e}_i)
\end{equation}

因此,$f$可以表示为$V$中的一个行向量$\bvec{F}$,坐标为$(f(\bvec{e}_1), \cdots, f(\bvec{e}_k))$.对于$V$中任何向量$\bvec{v}$,都有$f(\bvec{v})=\bvec{F}\bvec{c}$(按矩阵乘法).

若取另一个基$\{\bvec{e}_1', \cdots, \\bvec{e}_k'\}$,其中过渡矩阵为$\bvec{Q}$.如果在新的基下$\bvec{v}$的坐标变为$\bvec{c}'$,那么$\bvec{Q}\bvec{c}'=\bvec{c}$\footnote{见过渡矩阵\upref{TransM}.}.

设在新的基下,$f$表示为行向量$\bvec{F}'$,那么应有$f(\bvec{v})=\bvec{F}\bvec{c}=\bvec{F}'\bvec{c}'$.考虑到$\bvec{Q}\bvec{c}'=\bvec{c}$,我们可知对于任何坐标$\bvec{c}, \bvec{c}'$都有$\bvec{F}\bvec{Q}\bvec{c}'=\bvec{F}'\bvec{c}'$,因此

\begin{equation}
\bvec{F}\bvec{Q}=\bvec{F}'
\end{equation}

这就是一阶张量的坐标变换.
\subsubsection{二阶张量的坐标变换}

二阶张量涉及两个同构的线性空间$V$,而且是在物理学中最为常见的张量形式,因此我们将详细讨论该情况.在这里,我们把二阶张量$f$理解为从$V_1$和$V_2$到标量域$\mathbb{K}$的一个线性映射,其中$V_1$和$V_2$同构,$\opn{dim}V_1=\opn{dim}V_2=k$.

给$V_1$指定一组基$\{\bvec{a}_1, \bvec{a}_2, \cdots, \bvec{a}_n\}$, 给$V_2$指定一组基$\{\bvec{b}_1, \bvec{b}_2, \cdots, \bvec{b}_n\}$,在这两组基下,张量$f$被表示为一个矩阵$\bvec{F}$,而向量$\bvec{v}_1\in V_1$和$\bvec{v}_2\in V_2$在这两组基下的坐标列矩阵分别为$\bvec{c}(\bvec{v}_1)$和$\bvec{c}(\bvec{v}_2)$.此时,$f(\bvec{v_1},\bvec{v_2})=\bvec{c}(\bvec{v_2})\Tr\bvec{F}\bvec{c}(\bvec{v_1})$.

如果给$V_1$和$V_2$进行基变换,过渡矩阵分别为$\bvec{P}$和$\bvec{Q}$,则在新的基下,两向量的坐标分别为$\bvec{c}'(\bvec{v}_1)=\bvec{P}^{-1}\bvec{c}(\bvec{v}_1)$和$\bvec{c}'(\bvec{v}_2)=\bvec{Q}^{-1}\bvec{c}(\bvec{v}_2)$,那么$f(\bvec{v}_1, \bvec{v}_2)=\bvec{F}\bvec{v}_1\bvec{v}_2$.此时有$f(\bvec{v_1},\bvec{v_2})=\bvec{c}(\bvec{v_2})\Tr\bvec{F}\bvec{c}(\bvec{v_1})=\bvec{c}'(\bvec{v_2})\Tr\bvec{Q}\Tr\bvec{F}\bvec{P}\bvec{c}'(\bvec{v_1})$.因此在新基下,$f$的矩阵为$\bvec{Q}\Tr\bvec{F}\bvec{P}$.



\subsection{爱因斯坦求和约定表示的一般张量的坐标变换}

高阶张量的坐标变换就无法简单地用矩阵的乘法来表示,因此我们采用爱因斯坦求和约定.

\subsubsection{一阶张量的坐标变换}

给定$k$维线性空间$V$,及其上一个张量$f:V\rightarrow\mathbb{R}$.如果$V$的基是$\{\bvec{e}_1, \cdots\bvec{e}_k\}$,在这组基下,张量$f$的坐标为$T_{i}$,那么我们应有$T_{i}=f(\bvec{e}_i)$.

对$V$进行基的变换,如果过渡矩阵是$a_{ij}$,即新的基中基向量$\bvec{e}'_{i}=a_{ji}\bvec{e}_j$,那么我们易得:

\begin{equation}\label{TrTnsr_eq1}
T'_{i}=f(\bvec{e}'_i)=f(a_{ji}\bvec{e}_j)=a_{ji}f(\bvec{e}_j)=a_{ji}T_{j}
\end{equation}

注意指标的位置,以及哪些指标是用于求和的赝指标.

\subsubsection{二阶张量的坐标变换}

给定线性空间$V_1$和$V_2$,其中$\opn{dim}V_1=\opn{dim}V_2=k$.令$f:V_1\times V_2\rightarrow\mathbb{F}$是一个二阶张量.

给$V_1$指定一组基$\{\bvec{a}_1, \bvec{a}_2, \cdots, \bvec{a}_n\}$, 给$V_2$指定一组基$\{\bvec{b}_1, \bvec{b}_2, \cdots, \bvec{b}_n\}$,在这两组基下,张量$f$的坐标是$T_{ij}=f(\bvec{a}_i, \bvec{b}_j)$.

如果给$V_1$和$V_2$进行基变换,过渡矩阵分别为$m^{ij}$和$n^{ij}$,则在新的基下,$\bvec{a}'_i=m^{ki}\bvec{a}_k$,$\bvec{b}'_j=n^{lj}\bvec{b}_l$,因此张量$f$在新基下的坐标为:

\begin{equation}\label{TrTnsr_eq2}
T'_{ij}=f(\bvec{a}'_i, \bvec{b}'_j)=f(m^{ki}\bvec{a}_k, n^{lj}\bvec{b}_l)=m^{ki}n^{lj}f(\bvec{a}_k, \bvec{b}_l)=m^{ki}n^{lj}T_{kl}
\end{equation}

同样,注意指标的位置,以及哪些指标是用于求和的赝指标.

\subsubsection{一般张量的坐标变换}

相信从\autoref{TrTnsr_eq1} 和\autoref{TrTnsr_eq2} 中你已经比较得出了一般张量的坐标变换规律,因此不再详细展示计算过程,而是直接将坐标变换的方法表述如下:

\begin{theorem}{张量的坐标变换}

给定域$\mathbb{F}$上的一组有限个线性空间$\{V_\alpha\}^n_{\alpha=1}$,设$f:V_1\times\cdots\times V_n\rightarrow\mathbb{F}$.给各$V_\alpha$指定一组基,使得$f$在这些基下的坐标为$T_{i_1i_2\cdots i_n}$.如果对每个$V_\alpha$进行过渡矩阵为$m^\alpha_{ij}$的基变换,那么在变换后的基下,$f$的坐标为
\begin{equation}
T'_{i_1i_2\cdots i_n}=m^{k_1i_1}m^{k_2i_2}\cdots m^{k_ni_n}T_{k_1k_2\cdots k_n}

注意该式对各$k_1, k_2, \cdots, k_n$求和.
\end{equation}

\end{theorem}








