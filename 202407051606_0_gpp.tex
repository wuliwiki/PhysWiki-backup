% gcc/g++ 编译器笔记
% license Xiao
% type Note

\begin{issues}
\issueDraft
\end{issues}

\pentry{在 Linux 上编译 C/C++ 程序\nref{nod_linCpp}}{nod_4419}

\begin{itemize}
\item 参考\href{https://gcc.gnu.org/onlinedocs/gcc-12.2.0/gcc/}{官方文档}。
\item 注意: 不要改变系统默认的 \verb`/usr/bin/gcc` 和 \verb`/usr/bin/g++` symlink 的版本! 因为一些安装过程(例如 cuda)需要 \verb`gcc` 编译 kernel module, 必须使用默认版本。 如果已经改变了, 就卸载 \verb`gcc` 和 \verb`g++`, 再重装一次就行。
\item \verb`-I <dir>` 选项可以声明 .h 所在的目录
\item \verb`-c` 选项只编译不 link
\item 使用其他目录下的 cpp 文件如 \verb`g++ -c <other flags>  <dir1>/1.cpp <dir2>/2.cpp 3.cpp 4.cpp.....`
\item 编译器会自动搜索默认目录 \verb`/usr/local/include/` 下的头文件
\item \verb`-D 宏` 定义宏
\item \verb`-O3` 是最优化, \verb`-g` 是调试, 如果不调试, 需要手动定义 \verb`-D NDEBUG`
\item \verb`-g` 不包含一些信息如宏定义, 需要用 \verb`-g3`, 要打出 Macro 的位置和定义, 用 \verb`info macro 宏名`
\item static link 时候, .o 文件或者库只能调用它后面的 .o 文件或者库。 因为 linker 按顺序搜索, 从 \verb`main()` 函数开始, 如果有未定义的 symbol 就留到以后再找, 而把没有用的 symbol 抛弃。 这样能提高效率。
\item 如果有多个 .o 或者库互相依赖, 就用括号括起来, 例如 \verb`g++ hello.cpp -L. -( -la -lb -)`。 \verb`-(...-)` 是 \verb`--start-group ... --end-group` 的简写。
\item 检查内存泄露可以用 Address Sanitizer: \verb`g++ main.cpp -fsanitize=address -static-libasan -g`。 注意仅适用于动态编译。
\item 自动检查 signed 整数加减乘的溢出可以用 \verb`-ftrapv`, 亲测支持 \verb`int, long, long long` (其他类型貌似不行), 运行时会终止程序, 可以用 gdb 找到位置, 不能在程序中 \verb`catch`。如果想支持所有整数类型, 可以用 \verb`SafeInt` 库。
\item \verb`g++` 的默认输出是 stderr 而不是 stdout, 所以如果 pipe 的时候要指定 stderr
\end{itemize}

\subsection{linker (ld)}
\begin{itemize}
\item \verb`g++` 在链接阶段就是调用 \verb`ld` 命令
\item 例如 \verb`ld -o my_program file1.o file2.o -L/path/to/libs -lmylib -lc -dynamic-linker /lib64/ld-linux-x86-64.so.2`
\item 如果所有的 o 文件中有重复的 symbol (例如函数)链接时就会触发 ODR 错误, \verb`--allow-multiple-definition` 会忽略错误并使用第一个看到的定义。
\item \verb`-L` 指定添加 \verb`-l` 的搜索路径
\item \verb`ld --verbose | grep SEARCH_DIR | tr -s ' ;' \\012` 可以查看 \verb`linker` 的默认搜索路径
\end{itemize}

\subsection{preprocessor}
\begin{itemize}
\item \verb`g++ -E -P xxx.cpp > out.txt` 会显示 preprocessor 的输出, 其中 \verb`-P` 会删掉 linkmarker (用于显示代码在头文件中来源)
\item \verb`g++ -E -dM xxx.cpp > out.txt` 会输出所有 Macro 的定义, 每行一个(但并没有定义的位置)。
\end{itemize}

\subsection{编译警告}
\begin{itemize}
\item 详见\href{https://gcc.gnu.org/onlinedocs/gcc/Warning-Options.html}{文档}。
\item 在编译输出的警告中一般会有 \verb`[-Wxxx]`, 如果不需要警告, 就用 \verb`-Wno-xxx`。
\end{itemize}

\subsection{宏}
\begin{itemize}
\item 参考\href{https://gcc.gnu.org/onlinedocs/cpp/Common-Predefined-Macros.html}{文档}。
\item \verb`#if GCC_VERSION > 30200` 可以要求 GCC 版本大于 3.2.0。 其中 3 是 \verb`__GNUC__`, 2 是 \verb`__GNUC_MINOR__`,  0 是 \verb`__GNUC_PATCHLEVEL__`。
\end{itemize}

\subsection{环境变量}
\begin{itemize}
\item 参考\href{https://gcc.gnu.org/onlinedocs/gcc/Environment-Variables.html}{官方文档}。
\item \verb`CPATH` 是搜索头文件的路径。 相当于 \verb`-I` 选项。
\item \verb`LIBRARY_PATH` 是编译时搜索库文件的路径。 相当于 \verb`-L` 选项。
\item \verb`LD_LIBRARY_PATH` 是运行时搜索动态链接库的路径。 相当于 \verb`rpath`(详见 “\enref{g++ 编译器创建静态和动态链接库}{gppLib}”)。
\end{itemize}

\subsection{标准}
\begin{itemize}
\item 参考\href{https://gcc.gnu.org/onlinedocs/gcc/Standards.html}{文档}。
\item 目前(2022-11-23), \verb`g++` 的默认标准是 \verb`-std=gnu++17` 这是 \verb`-std=c++17` 加上 GNU 的一些拓展。
\item 其他的选项如 \verb`-ansi 或 -std=c++98 或 -std=c++03`, \verb`-std=c++11或14或17或20`
\item 如果要严格符合标准(去除所有 gnu 拓展), 用 \verb`-pedantic` 给出警告,或者 \verb`-pedantic-errors` 给出错误。
\end{itemize}

\subsection{依赖}
\begin{itemize}
\item \verb`g++ -H xxx.cpp` 可以列出所有头文件的路径,尤其是 \verb`#include <...>` 中的有时候很难判断用的是哪里的头文件另见(\verb`CPATH` 环境变量和 \verb`-I` 选项)。
\item \verb`g++` 的 \verb`-MM` 选项可以生成某个 cpp 文件或者 h 文件的所有依赖(包括依赖的依赖)。 \verb`-MM -nostdinc++` 则可以在依赖中去掉标准库中的头文件。
\item \verb`g++` 的 \verb`-M` 选项也一样, 但会生成多条依赖关系, 每个依赖关系只包含直接依赖。
\end{itemize}
