% 欧几里得矢量空间
% 欧几里得矢量空间

\begin{issues}
\issueMissDepend%实二次型
\end{issues}

\begin{definition}{欧几里得矢量空间}
定义在实数域 $\mathbb R$ 上的矢量空间 $V$ ,若其带有一个对称的双线性型 $(\bvec x,\bvec y)\mapsto(\bvec x|\bvec y)$(\autoref{MulMap_def2}~\upref{MulMap}),且对应的二次型 $\bvec x\mapsto(\bvec x|\bvec x)$ 是正定的,则称空间 $V$ 是一个\textbf{欧几里得矢量空间}.
\end{definition}
通常,对称的双线性型 $(*|*)$ 在 $\bvec x,\bvec y$ 处的值称为它们的\textbf{纯量积}.我们用符号 $(*|*)$ 代替通常的 $f(\bvec x,\bvec y)$.这里我们不用 $(\bvec x,\bvec y)$ 和 $\langle\bvec x,\bvec y\rangle$ 代替纯量乘积,出于这样的考虑:已经有简单的矢量对 $(\bvec x,\bvec y)$ ,它是笛卡尔积 $V\times V$ 的元素(\autoref{Set_eq1}~\upref{Set});而 $\langle \bvec x,\bvec y\rangle $ 又是矢量 $\bvec x,\bvec y$ 生成的子空间(\autoref{VecSpn_def1}~\upref{VecSpn}).
再一次把纯量乘积的性质列出来:
\begin{enumerate}
\item \textbf{对称性:}$(\bvec x|\bvec y)=(\bvec y|\bvec x)$;
\item \textbf{线性:}$(\alpha\bvec x+\beta\bvec y|\bvec z)=\alpha(\bvec x|\bvec z)+\beta(\bvec y|\bvec z)$;
\item \textbf{正定性:}$(\bvec x|\bvec x)>0,\;\forall\bvec x\neq0((\bvec 0|\bvec x)=0)$.
\end{enumerate}
\begin{example}{}
次数 $\leq n-1$ 的多项式\upref{OnePol} (其域为实数域 $\mathbb R$)对通常的加法和数乘构成一个矢量空间 $V=P_n$.对任意两个矢量(多项式)$f,g\in P_n$ ,数 
\begin{equation}\label{EuVS_eq1}
(f|g)=\int_a^b f(x)g(x)\dd x
\end{equation}
给出 $P_n$ 上向量间的纯量乘积.此纯量乘积是用\autoref{EuVS_eq1} 在连续函数(区间 $[a,b]$ 上)的无穷维矢量空间 $C(a,b)$ 上给出的.相应的无穷维欧几里得空间则表示为 $C_2(a,b)$.
\end{example}
\begin{example}{}
欧几里得矢量空间 $V$ 的任一子空间 $U$ 本身也是欧几里得矢量空间,因为 $V$ 中纯量乘积在 $U$ 中的限制定义了双线性函数 $U\times U\rightarrow\mathbb R$,且保持纯量乘积的性质.特别的,域 $\mathbb R$ 本身可看成是个1维的实的矢量空间.
\end{example}
\begin{definition}{长度}
在欧几里得矢量空间 $V$ 中,称非负实数
\begin{equation}
\abs{\abs{\bvec v}}=\sqrt{(\bvec v|\bvec v)}
\end{equation}
是任意矢量 $\bvec v\in V$ 的\textbf{长度}或 \textbf{模}.长度为1的矢量称为\textbf{标准的}.
\end{definition}
因为 $(\bvec v|\bvec v)\neq0$ ,所以任意矢量 $\bvec v$ 的长度都是完全确定的.

容易验证下面几个性质:
\begin{enumerate}
\item $\bvec v\neq0\Rightarrow \abs{\abs{\bvec v}}>0$.
\item $\abs{\abs{\lambda\bvec v}}=\abs{\lambda}\cdot\abs{\abs{\bvec v}}$
\end{enumerate}


\begin{example}{矢量的标准化}
任意矢量 $\bvec v$ 乘以它的长度的倒数$\frac{1}{\abs{\abs{\bvec v}}}$便可将其标准化,即
\begin{equation}
\abs{\abs{\frac{1}{\abs{\abs{\bvec v}}}\bvec v}}=1
\end{equation}
\end{example}
\begin{theorem}{柯西-布尼亚科夫斯基不等式}
欧几里得向量空间中,对任意矢量 $\bvec x,\bvec y\in V$,成立不等式
\begin{equation}
\abs{(\bvec x|\bvec y)}\leq\abs{\abs{\bvec x}}\,\abs{\abs{\bvec y}}
\end{equation}
\end{theorem}
\textbf{证明:}由纯量乘积的性质
\begin{equation}\label{EuVS_eq2}
\begin{aligned}
&(\lambda\bvec x-\bvec y|\lambda\bvec x-\bvec y)\\
&\underset{\text{线性}}{=}\lambda^2(\bvec x|\bvec x)-\lambda (\bvec x|\bvec y)-\lambda (\bvec y|\bvec x)+(\bvec y|\bvec y)\\
&\underset{\text{对称性}}{=}\lambda^2(\bvec x|\bvec x)-2\lambda (\bvec x|\bvec y)+(\bvec y|\bvec y)\\
&\underset{\text{正定性}}{\geq}0
\end{aligned}
\end{equation}
最后\autoref{EuVS_eq2} 最后一不等式可看成关于 $\lambda$ 的一元二次方程,其判别式满足
\begin{equation}
(2(\bvec x|\bvec y))^2-4(\bvec x|\bvec x)(\bvec y|\bvec y)\geq0
\end{equation}
即
\begin{equation}

\end{equation}

