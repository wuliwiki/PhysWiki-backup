% 依测度收敛
% keys 实变函数|测度|极限|收敛

\pentry{可测函数\upref{MsbFun}}

我们已经知道,函数有“逐点收敛”和“一致收敛”等不同的收敛方式.逐点收敛的概念最简单,但是性质不太好;一致收敛的性质就非常好.

现在我们有了外测度的概念,就可以构造一种新的收敛方式.

\begin{definition}{依测度收敛}

设$E\subseteq \mathbb{R}^n$是可测集,$f$和$f_i$都是其上的\textbf{几乎处处有限的}可测函数,其中$i$取遍全体正整数.如果对于任意固定的$\epsilon>0$,有
\begin{equation}
\lim\limits_{i\to\infty}\opn{m}\{x\in E|\abs{f_i(x)-f(x)}\geq\epsilon\}=0
\end{equation}
则称函数列$\{f_i\}$\textbf{依测度收敛}到函数$f$,记为$f_i\overset{\opn{m}}\to f$(于$E$).

\end{definition}

简单来说,依测度收敛就是指任取一个精度范围$\epsilon$,$f_i(x)$偏离$f(x)$超过精度范围的$x$称为“不听话”的点,那么随着$i$增大,不听话的点构成的集合的外测度趋于零.

类似地,一致收敛可以简单解释为:任取一个精度范围$\epsilon$,$f_i(x)$偏离$f(x)$超过精度范围的$x$称为“不听话”的点,那么随着$i$增大,不听话的点构成的集合趋于空集.

发现没?简单说法里,一致收敛和依测度收敛的简单解释,只有最后一句有差别.从这个差别可以容易看出,一致收敛能推出依测度收敛.更进一步,几乎处处一致收敛也能推出依测度收敛.

我们把上述讨论总结成以下定理:



\begin{theorem}{}\label{LimMs_the2}

设$E\subseteq \mathbb{R}^n$是测度有限的可测集,$f$和各$f_i$都是其上的\textbf{几乎处处有限的}可测函数,其中$i$取遍全体正整数.如果$f_i$\textbf{一致收敛}到$f$a. e.  ,那么$f_i\overset{\opn{m}}\to f$.

\end{theorem}



在讨论依测度收敛的相关问题的时候,“几乎处处”和“处处”可以看成是等价的,因为这两个限定词在测度意义上没有任何区别.

实际上,几乎处处收敛也是能推出依测度收敛的:

\begin{theorem}{}\label{LimMs_the1}

设$E\subseteq \mathbb{R}^n$是测度有限的可测集,$f$和各$f_i$都是其上的\textbf{几乎处处有限的}可测函数,其中$i$取遍全体正整数.如果$f_i\to f $a. e. ,那么$f_i\overset{\opn{m}}\to f$.



\end{theorem}

\textbf{证明}:

只需讨论$f$在$E$上恒等于$0$的情况.

任意固定一个$\epsilon>0$.

由Egoroff定理,对于任意$\delta>0$,存在可测集$E_\delta$,使得其测度小于$\delta$且$\{f_n\}$在$E-E_\delta$上\textbf{一致收敛}到$f=0$.

因此,存在正整数$N_\delta$,使得只要正整数$n>N_\delta$,就有$\abs{f_n(x)}<\epsilon$在$E-E_\delta$上处处成立.

因此,只需要取$\delta_k=1/2^k$\footnote{或者任意趋于零的正数列也行.},得到各$N_{\delta_k}$(记$N_{\delta_k}=n_k$),那么$\abs{f_{n}(x)}<\epsilon$在$E-E_{\delta_k}$上处处成立,其中$n>n_k$且$\opn{m}E_{\delta_k}<{\delta_k}$.

因此,当$n>n_k$时总有$\opn{m}\{x\in E|\abs{f_n(x)}\geq\epsilon\}\leq\opn{m}E_{\delta_k}<\delta_k$.故
\begin{equation}
\lim\limits_{n\to 0}\opn{m}\{x\in E|\abs{f_n(x)}\geq\epsilon\}=0
\end{equation}

即得证.

\textbf{证毕}.


反过来,依测度收敛是不能推出几乎处处收敛的.









