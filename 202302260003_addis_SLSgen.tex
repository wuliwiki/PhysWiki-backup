% SLISC 库代码生成

\begin{issues}
\issueDraft
\end{issues}

\pentry{SLISC 库概述\upref{SLISC}}



* Read this session if you want to read or modify the code

A template file (\verb`demo.h.in`) looks like
\begin{lstlisting}[language=cpp]
#include <complex>
#include <vector>
using namespace std;

//% types = {
//%  'vector<double>', 'Int', 'vector<float>';
//%  'vector<complex<double>>', 'vector<complex<double>>', 'vector<Int>';
//% };
//%----------------------------------
//% [Tz, Tx, Ty] = varargin{:};
void add(@Tz@ &z, const @Tx@ &x, const @Ty@ &y)
{
//% if is_vec(Tz) && is_vec(Tx) && is_scalar(Ty)
	for (size_t i = 0; i < z.size(); ++i)
		z[i] = x[i] + y;
//% elseif is_vec(Tz) && is_scalar(Tx) && is_vec(Ty)
	for (size_t i = 0; i < z.size(); ++i)
		z[i] = x + y[i];
//% elseif is_vec(Tz) && is_vec(Tx) && is_vec(Ty)
	for (size_t i = 0; i < z.size(); ++i)
		z[i] = x[i] + y[i];
//% else
//%     error('not implemented!');
//% end
}
//%-----------------------------------
\end{lstlisting}

Using \verb`$octave auto_gen.m`, the generated file \verb`demo.h` looks like
\begin{lstlisting}[language=cpp]
#include <complex>
#include <vector>
using namespace std;

void add(vector<double> &z, const Int &x, const vector<float> &y)
{
	for (size_t i = 0; i < z.size(); ++i)
		z[i] = x + y[i];
}

void add(vector<complex<double>> &z, const vector<complex<double>> &x,
        const vector<Int> &y)
{
	for (size_t i = 0; i < z.size(); ++i)
		z[i] = x[i] + y[i];
}
\end{lstlisting}
