% 单体算符
% 多体系统|单体算符|二次量子化

\pentry{二次量子化\upref{SecQua}}

现在我们来考察多体系统量子力学的单体算符。具体而言,我们希望考察多体系统中单粒子相关的物理量,由于是可观测的物理量,我们研究的单体算符是 Fock 空间 $\mathcal{F}$ 上的幺正算符。

回忆一下 Fock 空间的定义\autoref{SecQua_eq6}~\upref{SecQua},我们知道它可以根据粒子数不同划分为多个子 Hilbert 空间的直和:
\begin{equation}\label{oneopr_eq1}
\mathcal{F}=\mathcal{H}_1\oplus \mathcal{H}_2\oplus \cdots
\end{equation}
根据我们对内积的定义,对于两个态矢量 $v_i\in \mathcal{H}_i, v_j\in \mathcal{H}_j$,当 $i\neq j$ 时 $v_i,v_j$ 的内积一定是 $0$。而我们又希望单体算符 $A^{(1)}$ 所对应的物理量由 $\bra{v}A^{(1)} \ket{v}$ 给出,这实际上告诉我们 $A^{(1)}$ 作用于 $v_i\in \mathcal{H}_i$ 后所得到的态矢量 $A^{(1)} \ket{v_i}$ 也一定属于 $\mathcal{H}_i$。因此当我们限制在每个子空间 $\mathcal{H}_i$ 上以后, $A^{(1)}$ 仍然是幺正算符。这不仅对单体算符成立,对于两体算符和 $n$ 体算符这都是成立的。

在我们考察单体算符的形式定义之前,我们先来看一个最简单的例子:\textbf{粒子数算符}。
\subsection{粒子数算符}
设单粒子 Hilbert 空间的一组正交完备基是 $\ket{1},\ket{2},\cdots$,基于这一组基底,可以根据定义\autoref{SecQua_eq3}~\upref{SecQua}与\autoref{SecQua_eq7}~\upref{SecQua}构造多体系统的产生湮灭算符:
\begin{equation}
a_1,a^\dagger_1,a_2,a^\dagger_2,\cdots
\end{equation}
那么粒子数算符被定义为
\begin{equation}\label{oneopr_eq2}
\hat N=\sum_i a_i^\dagger a_i
\end{equation}
根据产生湮灭算符之间的对易关系,我们很容易能证明,对于玻色子系统:
\begin{equation}
\begin{aligned}
&[\hat N,a_i^\dagger]_-=a_i^\dagger,\quad [\hat N,a_i]_-=-a_i\\
&\hat N (a_1^\dagger)^{n_1} (a_2^\dagger)^{n_2}\cdots \ket{0} = (n_1+n_2+\cdots)   (a_1^\dagger)^{n_1} (a_2^\dagger)^{n_2}\cdots \ket{0} 
\end{aligned}
\end{equation}
而对于费米子系统\footnote{也就是说产生湮灭算符之间满足 $[a_i,a_j^\dagger]_+=\delta_{ij}$ 的反对易关系。$[\cdot,\cdot]_+$ 符号常常也写为 $\{\cdot,\cdot\}$。}:
\begin{equation}
\begin{aligned}
&[\hat N,a_i^\dagger]_+=a_i^\dagger,\quad [\hat N,a_i]_+=a_i\\
&\hat N a_{i_1}^\dagger a_{i_2}^\dagger \cdots a_{i_n}^\dagger \ket{0}= n a_{i_1}^\dagger a_{i_2}^\dagger \cdots a_{i_n}^\dagger \ket{0},\quad i_j\neq i_k(j\neq k)
\end{aligned}
\end{equation}
这也就意味着对于 $v_n\in \mathcal{H_n}$,$\hat N v_n=n v_n$。所以 Fock 空间的直和分解 \autoref{oneopr_eq1} 实际上是将整个 Hilbert 空间按照粒子数算符的本征值划分为了不同的子空间。

如果我们将单粒子 Hilbert 空间的基底 $\ket{1},\ket{2},\cdots$ 切换到坐标表象下的基底 $\ket{x},\cdots$,那么产生湮灭算符与原先的产生湮灭算符之间有\autoref{SecQua_eq8}~\upref{SecQua}的等式关系。此时产生湮灭算符之间的对易关系为 $[a_{\bvec x},a_{\bvec y}^\dagger]_{-\xi}=\delta^3(\bvec x-\bvec y)$,而粒子数算符为
\begin{equation}
\begin{aligned}
\hat N=\int \dd[3]{\bvec x} a^\dagger_{\bvec x} a_{\bvec x}
\end{aligned}
\end{equation}
注意相比于\autoref{oneopr_eq2},求和变为了空间积分,这是符合我们的预期的。类似地,如果切换到动量表象下,我们有 $[a_{\bvec k},a_{\bvec q}]_{-\xi}= (2\pi)^3 \delta(\bvec k-\bvec q)$,而粒子数算符为
\begin{equation}
\begin{aligned}
\hat N=\int \frac{\dd[3]{\bvec p}}{(2\pi)^3} a^\dagger_{\bvec p} a_{\bvec p}
\end{aligned}
\end{equation}
注意积分测度的变化。
\subsection{单体算符的形式定义}
当我们构造单粒子 Hilbert 空间上的物理量时,我们只需要用到 $\hat x,\ \hat p=-i\hbar \partial_x$ 这两个算符(显然它们是幺正的),利用它们可以表达哈密顿量、角动量等等,假设作用于单粒子 Hilbert 空间上的单体算符的形式为 $A^{(1)}(\hat x,\hat p)$(如果是三维空间,那括号内就包含三个反向上的坐标和动量算符)。

假设单粒子 Hilbert 空间的一组正交完备基为 $\ket{1},\ket{2},\cdots$。此时单粒子 Hilbert 空间上单体算符可以表达为
\begin{equation}
A^{(1)}(\hat x,\hat p)=\sum_{\alpha\beta}A_{\alpha\beta} \ket{\alpha}\bra{\beta}
\end{equation}
其中 $A_{\alpha\beta}$ 为 $A^{(1)}$ 算符的矩阵表示。现在考察多体系统,多粒子态 $\ket{\psi_1,\cdots,\psi_N}$ 不再是 $N$ 个单粒子态的简单的张量积,因为交换对称性,我们需要仔细考虑。由于 Fock 空间可以表达为一系列多粒子态 $\ket{ijk\cdots}$ 的直和,我们只需要去考量算符 $\hat A^{(1)}$ 作用在 $\ket{\psi_1 \psi_2 \cdots\psi_N}$ 上的结果即可。我们期待单体算符 $\hat A^{(1)}$ 能够被写为
\begin{equation}
\hat A^{(1)}=\sum_{i} A^{(1)}(\hat x_i,\hat p_i)
\end{equation}
其中算符 $A^{(1)}(x_i,p_i)$ 仅仅作用于多粒子态的 $\psi_i$ 成分上。具体而言:
\begin{equation}
\begin{aligned}
\bra{x_1,\cdots,x_N}\hat A^{(1)} \ket{\psi_1,\cdots,\psi_N}&=\sum_{i=1}^N A^{(1)}(\hat x_i,-i\hbar\hat\partial_{x_i})\sum_{P} \xi^P \psi_1(x_{p_1})\psi_2(x_{p_2})\cdots\\
&=\sum_{P}\xi^P \alpha(x_{p_1})A_{\alpha\beta}^{(1)}\braket{\beta}{\psi_1}\psi_2(x_{p_2})\psi_3(x_{p_3})\cdots\\
&\quad +\sum_P\xi^P \psi_1(x_{p_1})\alpha(x_{p_2})A_{\alpha\beta}^{(2)}\braket{\beta}{\psi_2}\psi_3(x_{p_3})\cdots\\
&\quad +\cdots \\
&=\sum_{\alpha\beta} A_{\alpha\beta}^{(1)}\left(\braket{\beta}{\psi_1} \braket{x_1,\cdots,x_N}{\alpha,\psi_2,\cdots,\psi_N}\right.\\
&\quad +\
\end{aligned}
\end{equation}
