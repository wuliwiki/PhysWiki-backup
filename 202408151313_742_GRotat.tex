% 星系旋转曲线
% keys 星系旋转曲线|天文|宇宙学
% license Usr
% type Map

螺旋星系围绕它们的垂直轴旋转。通过测量原子线的多普勒位移,可以确定恒星和其他追踪器(例如,氢云和脉泽)的圆周速度作为它们距离银河中心的距离的函数,从而获得旋转曲线。一些历史和最近的螺旋星系旋转曲线的例子在图1.2中被绘制出来。根据牛顿定律,F = ma,可以预测一个测试粒子(例如,一颗恒星)的圆周速度$vcirc$与其质量m和距离中心r的关系,假设球对称性:

\begin{equation}
mv2 circ(r) / r = GM(r) / r2 ⇒ vcirc(r) = √(GM(r) / r) 
\end{equation}


在螺旋星系中,大部分(可见)质量集中在密集的中心凸起和延伸到约10千帕秒的旋臂中。因此,在足够大的r处,所有可见质量都包含在轨道内,可以在方程(1.2)中用一个常数M替换。

然后速度应该遵循开普勒衰减$vcirc(r) ∝ r^-1/2$。关键是,对这种类型的大量星系的观测表明,旋转曲线在远离银河中心的大距离上保持平坦(即,随r恒定)。因此,根据牛顿引力,必须存在额外的看不见的质量来防止外围恒星飞离,使星系瓦解。假设球对称性,方程(1.2)表明,恒定速度vcirc(r)是通过假设一个质量密度$ρ(r) ∝ 1/r^2$的物质晕,延伸到大r来获得的:这样,$M(r) = 4π ∫ dr' r'^2 ρ(r') = 4πr$,这样$vcirc对r的依赖就消失了。最终,在更大的r处,晕预计会消失,旋转曲线将开始下降。然而,通常在这样大的距离上没有追踪器可用。

Rubin和Ford在1970年对仙女座星系(M31)的旋转曲线进行了首次精确测量,追踪了大约70个氢云。他们确定曲线在大约22千帕秒处相当平坦。这后来通过使用无线电和光学技术对其他数十个星系的观测得到了证实,发现旋转曲线在50千帕秒甚至更远的距离上仍然保持平坦。这些结果很快被解释为“缺失质量”的证据,从而使暗物质问题成为焦点。至今,随着数百个螺旋星系的观测,旋转曲线的平坦性仍然是支持暗物质的最直观和有说服力的证据。

实际研究与上述简化的原理证明不同,仔细模拟了观测到的星系的不同发光组分(凸起、条带、盘、气体等)。在天体物理学术语中,确定了一个星系的不可见质量与可见质量的比例,即给定星系的“质光比”。对于像银河系或仙女座这样的星系,这个比率大约是10。这些测量也可以用来确定暗物质密度分布,包括在银河系中,尽管通常结果并不十分限制性(见第2.1.1节)。