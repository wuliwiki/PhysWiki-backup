% Fortran 入门笔记

\subsection{常识}
\begin{itemize}
\item Fortran 代码不区分大小写
\item 感叹号用于写注释
\item `&` 用于续行, 如果一个 token 需要续行, 那么下一行的行首也需要 `&`. 另外, 必须放在注释前面.
\item 程序首尾  `program <name> ....  end program <name>`, 或者直接用 `end`
\item 没有子程序的时候, `program <name>` 可以省略, 最后一行 `end program <name>` 也可以省略
\item `print*, <item1>, <item2>....` 星号代表默认格式
\item 用 `read*, item1, item2, ...` 来输入变量, 要先定义变量类型. 
\item `read` 读取 character 类型的时候, 只能读取字母, 不能有任何其他符号
\item 字符串可以用单引号也可以用双引号
\item 可以用 `;` 符号在一行写多个语句, 叫 statement separator. 同一行中的每个 statement 不能超过 132 个字符
\item `stop` 可用于终止主程序, `return` 终止子程序
\item 所有的小括号前面可以有空格
\item `include "<file>"` 就是把文件的内容复制到当前位置, 应该相当于 C++ 的 #include
\end{itemize}

\subsection{算符}
\begin{itemize}
\item 指数算符是 `**`
\item 比较算符有: `==   >=   <=  /= `.  注意不等是用斜杠不是感叹号. 感叹号是用于注释的
\item 逻辑算符有 `.and.   .or.   .not.   .eqv.` (同或)  `.neqv.` (异或)
\item `//` 用于连接字符串
\end{itemize}

\subsection{结构}
\begin{itemize}
\item `if (<logical>) then ....  else .... end if` 注意要有括号
\item `if (<logical>) then ... else if (<logical>) then ... else ... end if`
\item `if` 前面可以加上 `<name>:` 如果这样做,  end if 后面也一定要加上 `<name>`  , 也可以选择在 `else if .. then` 或 `else` 后面加上 `<name>`. 注意只有开头有冒号, 后面没有冒号 .
\item select case, 见 http://www.tutorialspoint.com/fortran/select_case_construct.htm
\item do 循环:   `do ii = 1,3; ...; end do;` 也可以设置步长 `ii = int1, int2, step`. 其中 `step` 可以是负值.
\item do while 循环: `do while (<logical>) ....  end do`
\item `exit` 相当于 C 语言的 break, `exit <name>` 用于退出指定循环.
\item `cycle` 相当于 C 语言的 continue
\item do 前面可以加上 `<name>:` 如果这样, `end do` 后面也一定要加上 `<name>`
\item `cycle` 执行下一个当前循环, `cycle <name>` 执行下一个指定循环
\item `do` 后面什么都不加相当于无条件循环, 需要用 `exit` 退出.
\item statement label 是在 statement 前面的 1-5 位数字.
\item `go to <label>` 可以使程序跳转到 `<label>` 的语句.
\end{itemize}

\subsection{数据类型}
\begin{itemize}
\item `integer, real, complex, character, logical` 是五种 intrinsic types, 前三种是 numeric, 剩下的是 non-numeric.
\item 从 fortran90 开始, 所有变量都要定义. 用 implicit none 语句可以使未定义的变量出错. 亲自证明了不声明变量类型会导致很严重的错误且很难调试.
\item 除了 imlicit none, 另一种声明如 implicit double precision (a-h,o-z), 作用是默认所有 a-h, o-z 开头的变量都是 double precision, 而 i-n 开头的所有变量都是普通 integer*4. 这种做法不建议使用.
\item 所有的变量只能在 implicit none 后面紧接着, 不能在程序中定义.
\item 声明变量格式:  `<类型> :: a=1, b=2, ....`  初值可选
\item 变量声明时 `::` 可以省略
\item 5 种基本变量类型可以定义 constant 格式, 如  `real, parameter :: pi = 3.1415927`
\item 声明变量后, 也可以把某个变量变成常数, 直接用 parameter (var1 = 1, var2 = 2) 即可. 括号不可省略
\item `selected_int_kind(8)` 可以返回 integer 类型的 kind, 是的 integer 有 8 位有效数字
\item `selected_real_kind(15, 307)` 可以返回 real 类型的 kind, 使得 real 有 15 位有效数字和最大指数 307
\item `huge()` 可以测试某个变量的最大值
\item `tiny()` 可以测试某个变量的最小值
\item `<类型>(kind=..) :: <name>` 的括号中 kind = 可以省略, 直接写数字. 这个数字叫做 kind type parameter
\item `kind()` 函数可以检查某个变量的字节数.
\item `precision()` 函数可以检查某个变量的精度(小数位数)
\item 整型和整型相除得到整数, real 和 整型相除得到 real. real 赋值给 integer 向下取整.
\item `1` 和 `2` 都是整型! 尤其注意 `1/2 = 0` 而不是 1.   至少要加个小数点 !
\item literal constant 指的是程序中写入的值, 例如 `123, 'good', 1.7e-6` 等.
\item 声明矩阵的时候, 只能直接用 constant 指定矩阵大小和初值.
\end{itemize}

\subsubsection{real}
\begin{itemize}
\item 指定 literal constant 的类型: 1.3e8 只有 real(4) 的精度 (7位有效数字, 默认), 1.3d8 是 real(8) 的精度(16位有效数字, 即 double), 1.3q8 是 real(16) 的精度(33位有效数字). d 代表 double, q 代表 quad. 另外也可以用 1.3e8_16 的格式 , 得到类型  real(16) 与 1.3q8 等效, 这么做的好处是可以用一个常数变量 (parameter) 来指定数据类型,例如 1.3e8_rk 
\item real 的默认是单精度 real(4). 双精度 real(8) 需要声明.
\item real 直接赋值溢出会出错, 算符运算结果溢出产生 Infinity
\item underflow, 例如 1e-40. 会产生 warning.
\item 强制类型转换 real (var, kind)
\end{itemize}

\subsubsection{integer}
\begin{itemize}
\item `integer(kind = <n>)` 设置占用的字节, 可以取 1, 2, 4, 8 或者 16. 即最大值. 默认为 
   kind = 4. (huge 约 2E9)
\item 强制类型转换 int() 注意不是 integer()
\item `nint()` 采用四舍五入进行类型转换, 而 int() 采用向下取整
\item literal constants 设定 kind 用下划线, 例如 `3_4` 是 `kind = 4` 的整数 3.
\end{itemize}

\subsubsection{other}
\begin{itemize}
\item literal constant 是 (<实部>, <虚部>). 赋值时如果需要包含变量, 用强制类型转换 cmplx (Re, Im, 8) 其中 kind = 8 是 double complex
\item literal constant 是 .true. 或 .false.
\item 强制类型转换 logical (var, kind)
\end{itemize}

\subsubsection{char}
\begin{itemize}
\item `character (len = <n>)` 其中 n 是最大长度, 可以等于任意正整数, 默认为 1.  `len =` 可以省略
\item character 不是一个数组, 是一个独立的变量, 所以不能只改变其中一个元素. 声明数组: character(5) :: char(3,3) 生成 3*3 的矩阵, 每个矩阵元是 len=5.
\item `str(n:m)` 获取 character 类型变量 str 的第 n 到第 m 个字符, 注意不能用 str(n) 用而是用 str(n:n)
\item `str(1,2)(n:m)` 获取 character 矩阵的 (1,2) 元的第 n 到第 m 个字符
\item `str = 'some caracters'` 用于赋值(用双引号也行), 字符串左对齐, 剩下的位置补 null (ascii 的 0), 如果长度超出, 多出的部分被截去,且不会报错.
\item str = str1(n:m) 同样是左对齐,但是后面补空格
\item 要获取某个字符的 ascii 代码, 用 iachar(), 如 iachar(str(1,2)(3:3)) . 如果输入的变量有多于一个字符, 返回首字符的 ascii 代码
\item 用 `//` 可以链接两个字符变量,然后把所有的 null 变为空格
\item 用 `trim()` 可以把字符变量最后的空格或者 null 去掉并返回 len 较短的字符常量
\item `len()` 可以用于返回字符变量/常量的长度, 如 len('1234 ') 返回 5, len(trim('1234 ')) 返回 4
\item `write(var, *) var1, var2, ...` 可以像输出到命令行一样将字符串或变量写到 character 变量中
\end{itemize}

\subsection{数组和矩阵}
\subsubsection{基础}
\begin{itemize}
\item 声明矩阵例子:  `integer, dimension (5,5) :: mymatrix`
\item 更简单地: 可以用 `integer :: matrix(5,5)  或者 integer matrix(5,5)`
\item 如果用第二种格式, 可以在同一行声明很多不同尺寸的变量
\item 所有的小括号前面可以有空格.
\item 甚至可以指定矩阵的索引范围!   integer :: matrix(-3:1, 0:4). 这样 matrix(-3,0) 等表达都是合法的!
\item 但是要注意 `a:b` 中 a <= b, 否则返回空矩阵 (某些维度的长度为 0)
\end{itemize}

\subsubsection{Dynamic Matrix}
\begin{itemize}
\item \item 可用于变化尺寸声明: `real, allocatable :: Mat (:,:)`
\item 尺寸声明: `allocate ( Mat1 (size1, size2), Mat2 (size1, size2))`.
\item 用完了以后: `deallocate (Mat)` 养成好习惯, 务必写上.
\item 即使没有 `implicit none`, 也需要 `allocate`
\item allocate 以后, 所有元素都是 0.
\item 子程序中定义的 allocatable array 如果没有 save attribute, 在程序结束时会自动 deallocate.
\end{itemize}

\subsubsection{索引}
\begin{itemize}
\item `rank()` 相当于 Matlab 的 ndims, 返回一个 integer
\item `size()` 相当于Matlab的 numel. 但是 `size(Mat, dim)` 相当于 Matlab 的 size
\item `shape()` 相当于 Matlab 的 size, 返回一个 `integer(2)` 数组.
\item "extent" 指的是一个维度上面的索引数量 (这并不是个函数)
\item `lbound(Mat)` 返回在每个维度 index 的最小值, `lbound (Mat, dim)` 返回指定维度的最小值
\item 矩阵的赋值和 Matlab 一样, `matrix(1:2,3:4) = 1`
\item `a((/1,3,4/))` 也是合法的
\item 如果直接 `print` 矩阵, 会把所有列展开成一行.
\item `n1 : n2 : step` 的格式叫做 subscript triplet
\end{itemize}

\subsubsection{赋值}
\begin{itemize}
\item 数组可以与常数相乘
\item 数组可以用一个常数赋值
\item array constructor 的格式 `(/1,2,3,4/)` 相当于 Matlab 里面的中括号, 但只能表达一维矩阵. 例程 `a(1:4,2) = (/1,2,3,4/)`. 里面可以是变量.
\item 逻辑矩阵可以用 1 和 0 来赋值. `logical :: Barray(4) = (/1,1,0,1/)`, 
   或者 `(/.true.,.true.,.false.,.true./)`
\item `reshape` 可以指定高维度矩阵的赋值: 
   `reshape ( (/5,9,6,10,8,12/), (/3,2/) )`, 生成 3 行 2 列的数组, 排序方式和 Matlab 相同
\item `reshape (  array, shape, pad, order )` 可以按照矢量 order 指定的顺序来填充矩阵, 默认
\item reshape 的 pad 参数是用于 vector 不能把矩阵填满的情况下, 继续填满矩阵.
\item 用函数生成数列  `(/ ( (ii**2+3), ii=0,N ) /)`.  可以在后面加上 *2 之类的. 这种格式中 N 不能超过 1e5. 否则只能用循环.
\end{itemize}

\subsubsection{数学运算}
\begin{itemize}
\item `sum (array, dim, mask)` 所有元素求和
\item `product (array, dim, mask)` 所有元素求积
\item `dot_product(v1, v2)` 矢量点乘 (仅限于一维矢量)
\item `matmul(mat1, mat2)` 矩阵乘法
\item `maxval(array, dim, mask)` 矩阵的最大值 (不知道 mask 是干啥的)
\item `minval` 矩阵的最小值
\item `maxloc(Mat, mask)` 找到矩阵中最大元素的位置
\item `minloc(Mat, mask)` 找到矩阵中最小元素的位置
\item `transpose()` 求转置矩阵
\end{itemize}

\subsubsection{逻辑运算}
\begin{itemize}
\item `all (logical, dim)` 判断逻辑矩阵是否都是 .true.  dim 可选
\item `any (logical, dim)` 判断逻辑矩阵是否有任何 .true.
\item `count (logical, dim) `
\item `>, <, /=` 等算符用于两个矩阵返回逻辑数组. 进而使用 `.and. .or.` 等
\item where statement : 
`where ( a < 0 ); a = 1; elsewhere; a = 5; endwhere`
\end{itemize}

\subsubsection{矩阵处理}
\begin{itemize}
\item `pack(Mat, mask)` 把 Mat 中对应 mask 中为 .true. 的矩阵元按顺序装到 rank1 向量. 如果被赋值的矢量有多余元素, 则随机赋值. 这相当于 Matlab 中的 Mat (mask)
\item `v = pack(Mat, mask, vector)` 多余元素赋值为 vector 中的对应元素. vector 要和 v 的长度一致.
\item `pack(M,.true.)` 相当于 Matlab 中的 M(:)
\item `unpack(vector, mask, array)` 是 pack 的逆过程
\item `spread(Mat, dim, ncopies)` 把 Mat 在 dim 的维度排列 ncopies 次. dim 只能从 1 到 `rank(Mat) +1`
\item `merge(MatTrue, MatFalse, mask)` 在两个矩阵中选取矩阵元组成新的矩阵. mask 是逻辑数组, 用于选择矩阵元.
\item `cshift(Mat, shift, dim)` 是 circle shift, 把矩阵沿着 dim 的维度循环移动, shift > 0 是向 ind 变小的方向移动.
\item `eoshift()` 非循环移动, 规则类似.
\end{itemize}

\subsection{全局变量}
\begin{itemize}
\item `common/group/var1, var2, ...` 其中 var1, var2 的变量名在不同的声明中可以不一致, 但是 group 的名字要一样
\item 全局数组 `common/grou/var1(N1), var2(N2)` 变量要在 common 之前再定义一次
\end{itemize}

\subsection{输入输出}

### 命令行输入输出
* `print (<格式>), item1, item2, item3...`
* 注意格式周围一定要有括号!!!
* free format 格式 (list directed form)
```fortran
   print *, item1, item2...
   read (*,*) item1, item2... 
   write (*,*) item1, item2...
```
   星号代表默认, print 的星号可以替换成 `<format>`, read 和 write 的第一个星号可替换成输入输出的位置代号 (默认是命令行), 第二个星号可替换成 `<format>`
* `write (var, <format>)` 可以将内容写到变量中 (不确定是否只支持字符串变量)
* `<format>` 可以是一个数字, 如 1010, 这样, 必定有另一行指令 1010 format(...)
* 一个 format 的例子如下
```fortran
write(6, 901) a, b, c
901 format(1X,I3/'X=',T7,F5.2/D8.2)
```
这两个命令相当于在命令行中输出 1 个空格, 一个整型(a), 下一行, 'X=', tab 到第 7 列, 一个 real (b), 下一行, 一个 real (c).
* `print <format>` 相当于 `write(*,<format>)`
* `<format>` 是一个指定输出格式的字符串, 例如 '(10ES12.3)' , '(5I6)' 等.
* 在命令行给 read 输入数据的时候, 用空格隔开, 按回车确认.

`<format>` 字符串的定义
* 任意格式中, 如果需要的空间超出列宽, 将显示为 `****`

real type
* 小数格式 <列数>F<列宽>.<小数位数>
* real 类型的科学计数法  <列数>ES<列宽>.<小数位数>, 显示出来的位数 = <小数位数>+6
  所以<列宽> 需大于等于 <小数位数> +7 (正数+6, 负数带负号+7)
* 用 / 表示空白行

integer type
* <列数>I<列宽>.<显示位数>  对于 <显示位数>, 不够的前面用0补齐, 超出的不
   处理)
* 如果 write/read 命令后面是一个 list, 那么可以给 list 的每一项设置格式, 用 '(<fmt1>, <fmt2>...)'  (我猜测<列数>在该情况仍然可以使用)

character type
* <列数>A<列宽>

### 文件输入输出
* 在 visual studio 中, 生成的文件与源文件保存在同一目录下.
* 文件可以用 txt 编辑器打开, 不需要有拓展名或者可以有任何拓展名.
* 创建新数据文件并写入数据如下
```fortran
open (1, file = 'data1.txt', status = 'new')
do ii = 1,100
	write(1,*) x(ii), y(ii)
end do
close (1)
```
* 这里的 1 是文件在程序中的编号, 可以取 1-99, 除了 5 和 6 (分别代表从键盘读写).
* 'new' 是创建新文件, 若打开已存在文件, 替换成 'old'
* 如果不指定 status, 则相当于 status = 'unknown', 如果文件存在就用存在的文件, 不存在就创建新文件
* 在读写文件时, 每个 write 语句写在新的一行, 每个 read 语句读取一行.

#### Unformatted File
* 如果不需要人读取, 或者其他程序读取, 用这种文件效率更高
* ifort 写入的时候不用打开, 但是关闭是好习惯
```fortran
open(unit=7, file='fort.7', form='Unformatted') 这行可以省略!
write(7) array1
write(7) array2
close(unit=7)
```
两个写入的矩阵尺寸可以不一样!

* 读取方法
```fortran
open(unit=7, file='fort.7', form='Unformatted') 这行可以省略!
read(7) array1
read(7) array2
```

