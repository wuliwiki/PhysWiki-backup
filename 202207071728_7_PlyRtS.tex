% 多项式的根式解
% keys Galois理论|求根公式|五次方程|5次方程|Galois群


\pentry{Galois扩张\upref{GExt}}

第一子节讲讲撕逼历史故事,不想看历史的读者请直接跳转到\autoref{PlyRtS_sub1} .

\subsection{古典数学难题}

代数方程的根式解表达,是古典代数学中的经典难题.根式解即用任意多项式的系数进行有限次加、减、乘、除以及开根号运算,得到其根.比如,二次多项式$ax^2+bx+c$的根总可以表达为$\frac{-b\pm\sqrt{b^2-4ac}}{2a}$.

得到二次多项式求根公式的方法很简单,配方即可.

更高次多项式的求根则复杂许多.

中世纪之后,代数学的第一次重大进展是找到了三次和四次方程的根式通解.给定精度后找方程的数值近似解并不是难事,希腊人、阿拉伯人和中国人都早已发展出很多成熟的算术和几何方法来找数值解.但真正的代数解则直到近代才得出.

\subsubsection{三次方程}

1494年,意大利数学家卢卡·帕乔利(Lusa Pacioli)发表了他近三十年心血的结晶,《算术、几何、比及比例概要》(\textsl{Summa de arithmetica, geometria, Proportioni et proportionalita})\footnote{关于这本书的介绍,可参考https://en.wikipedia.org/wiki/Summa_de_arithmetica.},有时也被翻译为《数学大全》或者《算术大全》.在这本书里,他列出了两种无法解出的三次代数方程:
\begin{equation}\label{PlyRtS_eq1}
n=ax+bx^3
\end{equation}
\begin{equation}\label{PlyRtS_eq2}
n=ax^2+bx^3
\end{equation}

但是就在约一个世纪后,一个名叫希皮奥内·德尔·费罗(Scipione del Ferro)的意大利数学家就发现了\autoref{PlyRtS_eq1} 的解法.这个人很“自闭”,他不喜欢公开交流思想,只喜欢和自己的朋友或学生交流——这大概就是为什么没多少人记得他.所幸,费罗在三次方程求根公式上的成果被记录在他的笔记本上,在他1526年去世后由女婿哈尼瓦·纳威(Hannival Nave)继承了,这位女婿也是个数学家.

戏剧的是,在费罗去世之前,还秘密将\autoref{PlyRtS_eq1} 的解法传给了他的学生,安东尼奥·玛丽亚·菲奥利(Antonio Maria del Fiore).目前英文维基上都没有此人的介绍\footnote{参考资料:https://es.wikipedia.org/wiki/Antonio_Maria_del_Fiore(意大利语),https://second.wiki/wiki/antonio_maria_del_fiore.他的名字Fiore也可写作Fior , Fióre ,Flòrido 或者 Floridus.},而现代数学科普书《代数的历史:人类对未知量的不舍追踪》中对菲奥利的介绍只是“威尼斯人”和“\textbf{数学才能平庸之辈”}.与其说他是个数学家,倒不如说是个客观上促进了代数变革的\textbf{商人}.

拿到方程解的秘密后,他开始琢磨怎么捞钱.当时的意大利北部有的是营销的机会,因为学者们很难得到赞助,大学薪水不太理想,还没有终身教职制度,大家巴不得有路子来宣传自己.

于是又一个人物登场了:尼科洛·塔尔塔利亚(Niccolò Tartaglia).塔尔塔利亚13岁遭遇法国军队屠杀,活是活了下来,但是下颌严重创伤,从此变得口吃——所以人们叫他“Tartaglia”——口吃的人.没错,那个时代就这样,外号也能变成姓氏.1530年,塔尔塔利亚开始和一个数学老师交流他关于三次方程的一些成果,比如,\autoref{PlyRtS_eq2} 的通解.

菲奥利不知道怎么听说了这些消息,也不知道谁给他的自信——可能是老师留下的秘籍给了他自信?——总之,他向塔尔塔利亚发起赌局.双方要给对手出30道题,并在1535年2月22日将对手问题的解答递交给公证人,输家要请赢家吃30顿饭.

塔尔塔利亚一开始没把菲奥利的数学能力当回事,没有做充分准备.后来听到传言,说菲奥利十年前就从一位数学大师那里得到了秘传,塔尔塔利亚才开始上心.2月13日,他花了一个上午就解决了\autoref{PlyRtS_eq1} 的通解问题.和他想的一样,菲奥利出的题全都是关于\autoref{PlyRtS_eq1} 根式通解的,这人也就这一板斧了.

毫无悬念,塔尔塔利亚秒杀了菲奥利——菲奥利一题都做不出来.赢得荣誉后,塔尔塔利亚却放弃了赌金.吉罗拉莫·卡尔达诺(Girolamo Cardano)在他的自传里是这么评价塔尔塔利亚的:“与一个可怜的失败者面对面进餐的场景对他毫无吸引力.”不能怪卡尔达诺嘴毒,谁让菲奥利自找不快呢?

\subsubsection{四次方程}

卡尔达诺从自己手下的数学教师达伊科那里得知了塔尔塔利亚获胜的情况.当时,卡尔达诺正准备写一本数学著作,他琢磨着怎么把塔尔塔利亚的解法秘密骗过来,写到自己的书里.

1539年1月至3月,卡尔达诺开始和塔尔塔利亚书信往来.卡尔达诺就像钓鱼一样,变着法地耍塔尔塔利亚,激将法、糖衣炮弹全上阵,打一巴掌给颗糖.其中最大的那颗糖,当属他承诺把塔尔塔利亚引荐给米兰帝国军队的司令官兼伦巴底地方长官——地位仅次于皇帝.当时塔尔塔利亚出版了一本关于火炮技术的著作,卡尔达诺就告诉他自己已经买了两本,送了一本给这位司令官,而这位司令官非常想见见作者本人.

塔尔塔利亚上钩了,急忙奔到米兰.真是不凑巧呢,司令官当时不在米兰,于是塔尔塔利亚只好在卡尔达诺家住了几天.这几天,塔尔塔利亚受到了皇室般的贵宾待遇.3月25日时,上头了的塔尔塔利亚让卡尔达诺发誓不公布三次方程解的秘密,然后用一首25行诗写下了这个秘密.

回家以后冷静下来,塔尔塔利亚开始后悔了.卡尔达诺写信问他这首诗某些地方的解释时,他也很不耐烦地怼了回去.当得知卡尔达诺5月出版的算数书\textsl{Pratica Arithmeticæ et mensurandi singularis}里并没有公布他的秘密时,塔尔塔利亚平静了些.可没多久,他听说卡尔达诺又开始写一本关于代数的书,又开始暴躁和疑心起来.卡尔达诺好生安抚,直到1540年二人断绝书信来往.

1540年到1545年,是代数历史上的重要时期.卡尔达诺在塔尔塔利亚和费罗的基础上,给出了三次方程的一般解.由于费罗的成果早于塔尔塔利亚,并且也正是因为知道菲奥利得到了秘传,塔尔塔利亚才意识到原来三次方程是有解的,因此卡尔达诺不打算遵守誓言了——他觉得塔尔塔利亚只是再次发现而已.1545年,他出版了《大衍术》(\textsl{Ars Magna})\footnote{《大衍术》的电子稿:https://web.archive.org/web/20220201093634/http://www.filosofia.unimi.it/cardano/testi/operaomnia/vol_4_s_4.pdf.},首次发表了\textbf{三次和四次}方程的通解公式.

卡尔达诺有一个年轻有才的秘书,费拉里(Lodovico Ferrari).\textbf{四次}代数方程的根式通解就是这位年轻人在1940时发现的,那时他才18岁.想必是卡尔达诺偷偷把“发誓绝不外传”的三次方程求根公式教给了他的结果.

《大衍术》一出版,可想而知塔尔塔利亚暴怒的样子.他跳出来骂卡尔达诺\footnote{参考https://arxiv.org/ftp/arxiv/papers/1308/1308.2181.pdf.},但后者不接茬,只是让费拉里替父——替师父出征.1548年10月,费拉里和塔尔塔利亚在米兰又比了一场,塔尔塔利亚才消停了.有说法是塔尔塔利亚缺席,但总之比赛结果是费拉里赢了.费拉里从此名声大震,平步青云,而塔尔塔利亚则带着满腔悲愤,于1557年饮恨辞世.

幸运的是,他的身后名还是保住了.今天的数学历史学家承认了塔尔塔利亚的成就,并将卡尔达诺发布的三次方程求根公式称为卡尔达诺-塔尔塔利亚公式.

\subsubsection{小结语}

如果塔尔塔利亚没有受到费罗的启发,可能很久都不会尝试解三次方程.如果卡尔达诺没有钓出塔尔塔利亚的秘密,可能四次方程根式解的发现又要推迟数十年.可如果费罗不自闭,塔尔塔利亚也积极发表自己的成果,这一切发展又可能提前很久.

所以学术交流制度是多么关键\footnote{All hail to Sci-Hub and arxiv.org! }.



\subsection{Galois理论}\label{PlyRtS_sub1}

三次四次方程都有根式解,这不禁让人猜想五次方程是否也有根式解,尽管可能复杂得吓人.事实上,五次及以上的代数方程就没有根式解了.卡尔达诺时代的人们没有合适的数学工具来得出这一结果,但现在我们有了,那就是Galois理论.

\begin{definition}{多项式的Galois群}
给定域$\mathbb{F}$上的多项式$f$,令$\mathbb{K}$是其分裂域.则$f$的Galois群定义为
\begin{equation}
\opn{Gal}(f:\mathbb{F})=\opn{Gal}(\mathbb{K}/\mathbb{F})
\end{equation}
\end{definition}

给定一个多项式,其分裂域总是一个有限扩张,因此Galois群总是有限群.根据\autoref{GExt_the3}~\upref{GExt}和\autoref{GExt_the4}~\upref{GExt},此时Galois群和中间域总是一一对应的.

另外,这个多项式应该是可分的,这样才能保证其根都是可分元素,从而保证$\mathbb{K}/\mathbb{F}$是可分扩张.







































