% 阿基米德(综述)
% license CCBYSA3
% type Wiki

本文根据 CC-BY-SA 协议转载翻译自维基百科\href{https://en.wikipedia.org/wiki/Archimedes}{相关文章}。

阿基米德(约公元前287年 – 约公元前212年)是古希腊的数学家、物理学家、工程师、天文学家和发明家,来自西西里岛古城锡拉库萨。[3] 虽然他的生平细节不多,但他被认为是古典时代的领先科学家之一。阿基米德被誉为古代历史上最伟大的数学家,也是所有时代最伟大的数学家之一。[4] 阿基米德通过应用“无穷小”概念和“穷尽法”,为现代微积分和分析学奠定了基础,推导并严格证明了一系列几何定理。[5][6][7] 这些定理包括圆的面积、球的表面积和体积、椭圆的面积、抛物线下的面积、旋转抛物面段的体积、旋转双曲面段的体积,以及螺旋的面积。[8][9]

阿基米德的其他数学成就包括推导出π(圆周率)的近似值、定义并研究了阿基米德螺旋,并设计了一种使用指数法表示非常大数字的系统。他还是最早将数学应用于物理现象的人之一,致力于静力学和流体静力学的研究。阿基米德在这一领域的成就包括杠杆定律的证明,[10] 重心概念的广泛应用,[11] 以及著名的阿基米德原理,即浮力定律的表述。[12] 他还被认为设计了许多创新的机械装置,如螺旋泵、复合滑轮和用于保护家乡锡拉库萨免受入侵的防御性战争机器。

与他的发明不同,阿基米德的数学著作在古代鲜为人知。亚历山大的数学家曾阅读并引用过他的作品,但直到大约公元530年,米利都的伊西多尔在拜占庭的君士坦丁堡才首次做出了全面的汇编,而尤托修斯在同世纪对阿基米德著作的注释,第一次使这些著作获得了更广泛的读者群体。那些在中世纪幸存下来的阿基米德著作的少数副本,成为了文艺复兴时期和17世纪科学家思想的重要来源,[13][14] 而1906年在阿基米德古抄本中发现的失落作品,提供了新的洞见,揭示了他是如何获得数学成果的。[15][16][17][18]
\subsection{传记}  
\subsubsection{早年生活}
\begin{figure}[ht]
\centering
\includegraphics[width=8cm]{./figures/b835f7aaef6d5bdd.png}
\caption{本杰明·韦斯特(Benjamin West)创作的《西塞罗发现阿基米德的墓》(1805)} \label{fig_Archim_1}
\end{figure}
阿基米德大约于公元前287年出生在锡拉库萨的港口城市,该城市当时是意大利南部大希腊地区的一个自治殖民地。出生日期基于拜占庭希腊学者约翰·特泽茨(John Tzetzes)的一段话,他提到阿基米德在公元前212年去世时活了75年。[9] 普鲁塔克在《希腊罗马英杰传》中写道,阿基米德与锡拉库萨国王赫罗二世有亲戚关系,尽管西塞罗则认为他出身卑微。[19][20] 在《沙数计》中,阿基米德提到父亲的名字是菲迪亚斯(Phidias),他是一位天文学家,其他关于他的事迹没有记载。[20][21] 阿基米德的朋友赫拉克利德斯(Heracleides)曾为他写过传记,但这部作品已遗失,使得他的一生仍然笼罩在谜团中。例如,至今无法确认他是否结过婚、有过子女,或者年轻时是否曾访问过埃及的亚历山大。[22] 从他现存的著作中可以看出,他与当时在亚历山大的学者保持着良好的学术关系,包括他的朋友萨摩斯的科农(Conon of Samos)和塞内的馆长埃拉托斯特尼(Eratosthenes of Cyrene)。[b]