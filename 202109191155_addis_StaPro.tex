% 状态量和过程量

\begin{issues}
\issueDraft
\end{issues}

系统关于时间的变化可以看作状态空间中一点经过的轨迹. 若定义一个量只和该点在状态空间的位置有关, 那么它就是状态量, 若它和状态空间中的运动轨迹有关, 它就是过程量.

一个常见的例子是, 一个系统的状态用矢量 $\bvec x = (x_1, x_2, \dots, x_N)$ 描述, 在一个特定的过程中, $x_i$ 都是时间的函数. 定义一个量为
\begin{equation}
Q = \int_{\mathcal L} \sum_i f_i(x_1, \dots, x_N) \dd{x_i} = \int_{t_1}^{t_2} \sum_i f_i(x_1, \dots, x_N) \dv{x_i}{t} \dd{t}
\end{equation}
$\mathcal L$ 表示给定的 $x_i(t)$ ($i = 1,\dots, N$) 以及.

那么 $Q$ 是过程量还是状态量呢? 这事实上取决于
