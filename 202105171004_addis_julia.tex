% Julia 笔记

\begin{issues}
\issueDraft
\end{issues}

\begin{lstlisting}[language=julia]
println("hello world")
\end{lstlisting}

\begin{lstlisting}[language=julia]
function sphere_vol(r)
    return 4/3*pi*r^3
end
\end{lstlisting}

\subsection{变量类型}

Julia 是动态类型的语言. 具体类型不能互相作为子类, 只能作为抽象类的子类.

Julia 中所有的值都是对象. C++ 中 float, double 等不是对象, 只有 struct, class 定义的才是对象. Julia 中的对象没有成员函数.

可以在 literal、 变量、 表达式、 函数定义后面加上 \verb|::变量类型| 用于确认它具有该类型, 如果类型不符会产生异常. 如果 \verb|变量类型| 是抽象的, 那么表达式只需要是它的一个 sub type.

只有 “值” 有类型, 而不是变量具有类型.

抽象和具体的类都可以使用其他类作为参数

\verb|Int8|, \verb|UInt8|, \verb|Int16|, \verb|UInt16|, \verb|Int32|, \verb|UInt32|, \verb|Int64|, \verb|UInt64|, \verb|Int128|, \verb|UInt128|, \verb|Float16|, \verb|Float32|, \verb|Float64|, \verb|ComplexF64|

\verb|ComplexF16|, \verb|ComplexF32| 和 \verb|ComplexF64| 是 \verb|Complex{Float16}|, \verb|Complex{Float32}| 和 \verb|Complex{Float64}| 的别名.

\verb|BigInt| 是任意大小的整数, \verb|BigFloat| 是任意精度浮点数.

\begin{lstlisting}[language=julia]
julia> BigFloat(2.1) # 2.1 here is a Float64
2.100000000000000088817...

julia> BigFloat("2.1") # the closest BigFloat to 2.1
2.099999999999999999999999999...99999999999999999999986

julia> BigFloat("2.1", RoundUp)
2.100000000000000000000000000...00000000000000000000021

julia> BigFloat("2.1", RoundUp, precision=128)
2.100000000...00000000000007
\end{lstlisting}

\subsubsection{抽象类型}
声明抽象类型
\begin{lstlisting}[language=julia]
abstract type 子类名 <: 母类名 end
\end{lstlisting}
例子
\begin{lstlisting}[language=julia]
abstract type Number end
abstract type Real     <: Number end
abstract type AbstractFloat <: Real end
abstract type Integer  <: Real end
abstract type Signed   <: Integer end
abstract type Unsigned <: Integer end
\end{lstlisting}
最高级的抽象类是 \verb|Any|, 这里的 \verb|Number| 就是 \verb|Any| 的子类.

\subsection{原始类型}
原始类型(primitive type)是有 bit 组成的具体类型. Julia 可以自定义原始类型. 定义原始类型的格式为
\begin{lstlisting}[language=julia]
primitive type 类名 比特数 end
primitive type 类名 <: 父类名 比特数 end
\end{lstlisting}
例如
\begin{lstlisting}[language=julia]
primitive type Float16 <: AbstractFloat 16 end
primitive type Float32 <: AbstractFloat 32 end
primitive type Float64 <: AbstractFloat 64 end

primitive type Bool <: Integer 8 end
primitive type Char <: AbstractChar 32 end

primitive type Int8    <: Signed   8 end
primitive type UInt8   <: Unsigned 8 end
primitive type Int16   <: Signed   16 end
primitive type UInt16  <: Unsigned 16 end
primitive type Int32   <: Signed   32 end
primitive type UInt32  <: Unsigned 32 end
primitive type Int64   <: Signed   64 end
primitive type UInt64  <: Unsigned 64 end
primitive type Int128  <: Signed   128 end
primitive type UInt128 <: Unsigned 128 end
\end{lstlisting}

\subsection{复合类型}




随机数
\begin{lstlisting}[language=julia]
rand(ComplexF64, Nr1, Nr2, Npw)
\end{lstlisting}

hash
\begin{lstlisting}[language=julia]
hash(矩阵)
\end{lstlisting}

当前时间 \verb|time()|, 零向量 \verb|zeros(整数)|

矩阵切割 \verb|Psi[:, j, :]|

\verb|size(Psi, 维度)|

