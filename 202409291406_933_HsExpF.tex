% 指数函数(高中)
% keys 指数|指数函数|自然常数
% license Usr
% type Tutor

\begin{issues}
\issueDraft
\end{issues}

\pentry{函数\nref{nod_functi},函数的性质\nref{nod_HsFunC},幂运算与幂函数\nref{nod_power}}{nod_d767}

在幂运算的基础上。

\subsection{自然常数e}

\subsection{指数函数}

将底数作为参数,指数作为自变量的函数就称为指数函数,指数函数的名称指的就是自变量的在指数位置上,注意不要与幂函数相混淆。

\begin{definition}{指数函数}
形如
\begin{equation}
f(x) = a^x~.
\end{equation}
的函数称作\textbf{指数函数(exponential function)},其中 $a\in\mathbb R$。
\end{definition}
\subsection{指数函数的性质}

zhi shu bao zha

\subsection{柯西函数方程}

事实上,指数函数就是满足柯西函数方程$f(x+y)=f(x)f(y)$的一个解。