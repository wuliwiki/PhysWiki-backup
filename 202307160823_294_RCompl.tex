% 实数集的完备公理
% keys 戴德金分割|区间套|闭区间套|确界原理|单调有界定理|有限覆盖定理|聚点定理|致密性定理|柯西收敛准则|Cauchy收敛准则

\pentry{完备公理(戴德金分割)\upref{Cmplt}}


我们常见到一种争论,$0.\dot{9}$ 到底等不等于 $1$。实际上这个问题无法证明,而是被当作定义实数的公理之一,我们称之为完备性公理。你可以留意一下各种各样所谓的“证明”,认为 $0.\dot{9}\neq 1$ 的论证通常都是否认了完备公理,而认为 $0.\dot{9}=1$ 的都默认了完备公理。

用 $0.\dot{9}=1$ 来当作完备公理很不好用,我们通常使用本节介绍的的完备公理来描述实数的完备性,这些公理彼此是等价的,并且都可以推出 $0.\dot{9}=1$。

我们先列举几条完备公理,再展开证明它们互推的逻辑链条。

\subsection{完备公理的表述}

由于几个完备公理是等价的,可以互推,因此实际建立理论时只挑其中一个作为公理体系的一部分,其它的都当作定理,这也导致我们也常把这几条公理称为“定理”。因此,我们使用“定理”的格式来列举这几条完备公理。

\begin{theorem}{确界原理}\label{the_RCompl_1}
实数集的任何非空有界子集,必有一个实数上确界(见\textbf{上确界与下确界}\upref{SupInf})。
\end{theorem}

\begin{theorem}{单调有界收敛定理}\label{the_RCompl_2}
单调有界数列必有极限。
\end{theorem}

\begin{theorem}{(闭)区间套定理}\label{the_RCompl_3}
设 $a_n$ 是单调递增数列,$b_n$ 是单调递减数列,$b_n-a_n$ 恒为正数且收敛到 $0$,且 $[a_{n+1}, b_{n+1}]$ 都是 $[a_n, b_n]$ 的真子集。称这样的集合 $\{[a_n, b_n]_{n=1}^\infty\}$ 为一个\textbf{(闭)区间套}。

对于任意的区间套 $\{[a_n, b_n]\}$,存在唯一的实数 $x_0$ 使得 $x_0\in [a_n, b_n]$ 对任意正整数 $n$ 成立。
\end{theorem}

\begin{theorem}{Heine-Borel有限覆盖定理}\label{the_RCompl_4}
设 $[a, b]$ 是一个区间,$C$ 是 $[a, b]$ 的一个开覆盖\footnote{即 $C$ 是开集的集合,使得 $C$ 中所有开集的并包含了 $[a, b]$。实数轴上的开集是指开区间的并。},那么 $C$ 中存在有限个开集,使得其并集包含了 $[a, b]$。
\end{theorem}

\begin{theorem}{Bolzano致密性定理}\label{the_RCompl_5}
有界无穷数列必有收敛子列。
\end{theorem}

\begin{theorem}{Weierstrass聚点定理}\label{the_RCompl_6}
有界无穷点集必有聚点(\autoref{def_Topo0_1}~\upref{Topo0})。
\end{theorem}

\begin{theorem}{Cauchy收敛准则}\label{the_RCompl_7}
数列 $\{a_k\}$ 收敛,当且仅当对于任意 $\epsilon>0$,存在 $N_\epsilon$ 使得对于任意 $m, n>N_\epsilon$,都有 $\abs{a_m-a_n}<\epsilon$。
\end{theorem}

以上七条就是最常见的实数完备公理,任取其一都可以用来定义实数的完备性、而把其它的当成定理。

\subsection{完备性定理的互相推出}

\subsubsection{\autoref{the_RCompl_1} $\to$ \autoref{the_RCompl_2} }

首先,确界原理也等价于“有界子集必有下确界”,只需要对子集里各实数取负值构成新的有界子集,取到新子集的上确界再取负值,就得到原子集的下确界了。因此我们这里不妨设单调有界数列是单调不减的,单调不增的证明方式完全一致。

取单调不增有界数列的全体函数值 $\{a_n\}$,构成实数集的一个子集 $S$。由于是有界数列,故 $S$ 是有界点集。于是由\autoref{the_RCompl_1} ,$S$ 有一个上确界 $a$。

由数列单调性,$\abs{a-a_n}$ 随着 $n$ 的增大而减小。同时由上确界的定义,对于任意的 $\epsilon>0$,都存在 $N_\epsilon$,使得只要 $n>N_\epsilon$ 就有 $\abs{a-a_n}<\epsilon$,而这就是 $\{a_n\}$ 收敛于 $a$ 的定义。

\subsubsection{\autoref{the_RCompl_2} $\to$ \autoref{the_RCompl_3} }

区间套中的 $\{a_n\}$ 和 $\{b_n\}$ 都是单调有界数列\footnote{比如说,$\{b_n\}$ 是单调不增的,因此 $b_1$ 是其上界;同时由于 $b_n>a_1$,$a_1$ 也就是其下界。},由\autoref{the_RCompl_2} 可知它们都收敛。

设 $\{a_n\}$ 收敛到 $a$,$\{b_n\}$ 收敛到 $b$。

由单调性,对任意 $n$ 都有 $\abs{b_n-a_n}\geq\abs{b-a}$ 以及 $a, b\in [a_n, b_n]$。但是由区间套的定义,$\abs{b_n-a_n}$ 趋于零,故 $\abs{b-a}=0$,即 $a=b$。

这么一来,$a=b$ 就是区间套中唯一的公共元素。

\subsubsection{\autoref{the_RCompl_3} $\to$ \autoref{the_RCompl_4} }

反设存在 $[a, b]$ 的一个开覆盖 $S$,使得 $S$ 的任何有限子集都不能覆盖 $[a, b]$。不失一般性地,设 $S$ 中的元素都是开区间\footnote{可以这样简化是因为开集都是开区间的并集。}。

将 $[a, b]$ 二等分为 $[a, \frac{a+b}{2}]$ 和 $[\frac{a+b}{2}, b]$,那么由假设,这两个二等分闭区间中至少有一个不能被 $S$ 中的有限子集覆盖。取一个不能被有限覆盖的二等分区间,记它为 $[a_1, b_1]$。

由于 $[a_1, b_1]$ 也是不能被 $S$ 的有限子集覆盖的闭区间,故我们可以同样取出它的一个二等分区间 $[a_2, b_2]$,使得这个新的二等分区间不能被 $S$ 的有限子集覆盖。以此类推,我们可以取出一系列 $[a_k, b_k]$,其中每一个都不能被 $S$ 的有限子集覆盖。由于每一步都是取二等分,故 $\abs{b_k-a_k}=\frac{\abs{b-a}}{2^k}$,因此 $[a_k, b_k]$ 是一个区间套。

由区间套定理,存在唯一的 $x_0\in[a, b]$ 使得 $x_0\in[a_k, b_k]$ 对所有 $k$ 成立。由于 $S$ 是覆盖,故 $S$ 中至少有一个元素是包含 $x_0$ 的,设这个元素是区间 $(a_0, b_0)$。设 $\min(\abs{x_0-a_0}, \abs{x_0-b_0})=r$,那么取 $k$ 使得 $\abs{b_k-a_k}<r$,所对应的 $[a_k, b_k]$ 就是 $(a_0, b_0)\in S$ 的子集,于是 $[a_k, b_k]$ 就被 $S$ 的有限子集覆盖了,矛盾!

因此反设不成立,$S$ 必有有限子覆盖。

\subsubsection{\autoref{the_RCompl_4} $\to$ \autoref{the_RCompl_5} }

设 $\{x_k\}$ 是一个有界数列,即存在实数 $a<b$ 使得 $a<x_k<b$ 对任意 $k$ 都成立。

如果 $s\in(a, b)$ 不是 $\{x_k\}$ 任何子列的收敛点,那么必然存在 $r_s>0$ 使得 $\{x_k\}\cap (s-r_s, s+r_s)=\{s\}$,即数列中没有除了 $s$ 本身外的元素到 $s$ 的距离小于 $r_s$\footnote{这是因为,“$s$ 是某个子列的收敛点”等价于“对于任意的距离 $r$,都存在数列的元素,其靠近 $s$ 的距离小于 $r$”。将这一表述否定,就得到本段的表述了。}。

反设 $\{x_k\}$ 没有收敛子列,即任何 $s\in(a, b)$ 都不是其收敛点。按上段论述,我们就可以对每一个 $s$ 取一个半径 $r_s$,使得 $(s-r_s, s+r_s)$ 中只有数列的一个元素。对所有 $s\in(a, b)$ 都取这样的区间,则我们就得到了 $[a, b]$ 的一个开覆盖。由有限覆盖定理,存在有限多个 $s$ 和 $r_s$ 的组合,使得对应的 $\{(s-r_s, s+r_s)\}$ 是有限子覆盖。这么一来,$\{x_k\}$ 的取值就在这些 $s$ 中,即只有有限多取值。但是我们有可数无穷多个 $x_k$,所以其中必有无穷多个等于同一个 $s$,它们就能构成一个收敛子列。矛盾!

因此反设不成立,$\{x_k\}$ 必有收敛子列。

\subsubsection{\autoref{the_RCompl_5} $\to$ \autoref{the_RCompl_6} }

因为有无穷多个点,所以我们可以每次取一个出来,每次取的都和前面不重复。这样,我们就得到一个有界无穷数列。由致密性定理,该数列必有收敛点。该收敛点就是该点集的聚点。



\subsubsection{\autoref{the_RCompl_6} $\to$ \autoref{the_RCompl_7} }

回顾数列收敛的定义:$\{a_n\}$ 收敛,当且仅当存在实数 $a$ 使得 $\lim\limits_{n\to \infty}\abs{a_n-a}=0$。我们要证明当\autoref{the_RCompl_6} 成立时, \autoref{the_RCompl_7} 的设定(Cauchy收敛准则)等价于数列收敛的定义。

给定数列 $\{a_k\}$,使得 $\epsilon>0$,存在 $N_\epsilon$ 使得对于任意 $m, n>N_\epsilon$,都有 $\abs{a_m-a_n}<\epsilon$。于是这个数列是有界数列,其值域是一个有界无穷点集。由聚点定理,存在一个 $x_0$,使得它是 $\{a_k\}$ 值域的一个聚点。

任取 $\epsilon>0$,由聚点的定义,必然存在无穷多个 $a_k$ 使得 $\abs{a_k-x_0}<\epsilon$;由题设,存在 $N_\epsilon$ 使得对于任意 $m, n>N_\epsilon$,都有 $\abs{a_m-a_n}<\epsilon$。综合起来,存在一个 $M_\epsilon$,使得对于任意 $r>M_\epsilon$,都有 $\abs{a_r-a_{M_\epsilon}}<\epsilon$,且 $\abs{a_{M_\epsilon}-x_0}<\epsilon$。因此 $\abs{a_r-x_0}<2\epsilon$。

重新整理一下以上表述,就是:任取 $\epsilon>0$,存在 $M_\epsilon$,使得对于任意 $r>M_\epsilon$,都有 $\abs{a_r-x_0}<2\epsilon$。这正是 $\{a_k\}$ 收敛于 $x_0$ 的定义。因此 \autoref{the_RCompl_7} 的设定可以推出数列收敛。

反过来,如果数列收敛则Cauchy收敛准则也成立。


\subsubsection{\autoref{the_RCompl_7} $\to$ \autoref{the_RCompl_1} }

取一个非空有界实数集 $S$,已知它有一个上界 $a_0$。由于 $S$ 非空,故存在 $b_0<a_0$,且 $b_0\in S$。

在线段 $b_0a_0$ 上取中点 $x$,看这个中点是不是 $S$ 的上界,如果不是就再取 $x$ 和 $a_0$ 的中点,以此类推,直到取的某一个中点是上界,把这个中点记为 $a_1$。当然,我们不能保证 $a_1$ 一定存在,因此要分类讨论。

如果 $a_1$ 不存在,等价于任何小于 $a_0$ 的实数都不是 $S$ 的上界,那这说明 $a_0$ 就是 $S$ 的上确界了,故上确界存在。以下假设 $a_1$ 是存在的。


取 $a_1=\frac{2^{s}-1}{2^{s}}b_0+\frac{1}{2^{s}}a_0$ 和 $b_1>\frac{2^{{s}-1}-1}{2^{{s}-1}}b_0+\frac{1}{2^{{s}-1}}a_0$,使得 $b_1\in S$。这里的 $s$ 就是我们取中点直到得到 $a_1$ 的次数。接下来,对线段 $b_1a_1$ 进行相同的逐次取中点操作,直到得到第一个是 $S$ 上界的中点 $a_2$,取比 $a_2$ 的前一个中点大的 $b_2\in S$。如果其中有任何一个 $a_k$ 无法取到,那么 $a_{k-1}$ 就是 $S$ 的上确界,故我们以下假设 $a_k$ 总能取到。以此类推,得到两个数列 $\{a_k\}$ 和 $\{b_k\}$。

考虑到 $\sum\limits_{k=1}^\infty \frac{1}{2^k}=1$,而由我们取中点的过程可知 $\abs{a_{k+1}-a_k}<\frac{\abs{a_0-b_0}}{2^k}$,加上 $\{a_k\}$ 是递减数列,因此可知对于任意正整数 $m$ 都有 $\abs{a_{k+m}-a_k}<\frac{\abs{a_0-b_0}}{2^{k-1}}$,即 $\{a_k\}$ 满足柯西收敛准则。

由\autoref{the_RCompl_7} ,可知必存在 $a\in\mathbb{R}$ 是 $\{a_k\}$ 的极限。下证 $a$ 就是 $S$ 的上确界。

反设 $a$ 不是 $S$ 的上确界,即存在 $x<a$ 是 $S$ 的上界。由于总存在正整数 $k$ 使得 $\abs{a_k-b_k}<\abs{a-x}$,这就和 $a_k>a, x\geq b_k$ 矛盾了。故反设不成立,$a$ 就是 $S$ 的上确界。

\subsection{总结}

我们按\autoref{the_RCompl_1} $\to$\autoref{the_RCompl_2} $\to$\autoref{the_RCompl_3} $\to$\autoref{the_RCompl_4} $\to$\autoref{the_RCompl_5} $\to$\autoref{the_RCompl_6} $\to$\autoref{the_RCompl_7} $\to$\autoref{the_RCompl_1} 的顺序完成了证明闭环。这个顺序只是笔者任意选定的,读者也可以尝试其中任意两个定理的互相证明,比如\autoref{the_RCompl_5} 和\autoref{the_RCompl_6} 之间的互相证明。

每个定理,或者说公理,都从各自的角度刻画了“实数不存在缝隙”这一性质。它们都和\textbf{完备公理(戴德金分割)}\upref{Cmplt}中的\autoref{the_Cmplt_1}~\upref{Cmplt}等价。

\begin{exercise}{}
证明确界原理\autoref{the_RCompl_1} 和实数的完备公理\autoref{the_Cmplt_1}~\upref{Cmplt} 等价。
\end{exercise}

实际建立实数理论时,本节介绍的七条公理,或者也可以加上\autoref{the_Cmplt_1}~\upref{Cmplt} ,任选其一来描述实数的完备性即可,又或者利用本节的结论将它们看成同一个公理的不同描述,实践中哪个方便证明就用哪个。

\begin{exercise}{}
使用本节以及\textbf{完备公理(戴德金分割)}\upref{Cmplt}中介绍的八条公理,证明 $0.\dot{9}=1$。注意 $0.\dot{9}$ 被4定义为数列 $\{0.9, 0.99, 0.999, 0.9999\cdots\}$ 的极限。
\end{exercise}













