% 定积分
% 基本区间分划序列|定积分|可积函数|黎曼和

\pentry{极限\upref{Lim}}
\subsection{和 $\sum\limits_{i=0}^{n-1}f(\xi_i)\Delta x_i$ 的极限定义}
\footnote{菲赫金哥尔茨, 微积分教程卷二.}在给出定积分定义之前,我们先给出下面的和
\begin{equation}\label{eq_DInt_1}
\sigma=\sum_{i=0}^{n-1}f(\xi_i)\Delta x_i
\end{equation}
的极限的两种等价定义。
\autoref{eq_DInt_1} 的级数和所表示的意义理解如下:

设函数 $f(x)$ 在区间 $[a,b]$ 上有定义。用任意方法在该区间上插入分点
\begin{equation}
x_0=a<x_1<\cdots<x_i<\cdots<x_n=b.
\end{equation}
用 $\lambda$ 表示差 $\Delta x_i=x_{i+1}-x_i(i=0,1,\cdots,n-1)$ 中最大的一个,即
\begin{equation}
\lambda=\max\qty{\Delta x_i|i=0,\cdots,n-1}
\end{equation}
从每一个区间 $[x_i,x_{i+1}]$ 上任取一点 $x=\xi_i\quad (i=0,\cdots,n-1)$,并做出和
\begin{equation}
\sigma=\sum_{i=0}^{n-1}f(\xi_i)\Delta x_i.
\end{equation}
其几何意义可用下图表出。
\begin{figure}[ht]
\centering
\includegraphics[width=8cm]{./figures/c36fca0b6f2f5aca.pdf}
\caption{\autoref{eq_DInt_1} 的几何意义} \label{fig_DInt1}
\end{figure}
可以证明,以下关于 $\sigma$ 的极限的两种定义等价。
\begin{definition}{第一种定义}\label{def_DInt_1}
称和
\begin{equation}
\sigma=\sum_{i=0}^{n-1}f(\xi_i)\Delta x_i.
\end{equation}
当 $\lambda\rightarrow0$ 时有(有限)\textbf{极限} $I$ ,如果对每个数 $\epsilon>0$ 可以找到这样的数 $\delta>0$,使得,只要 $\lambda<\delta$ ,不等式
\begin{equation}
\abs{\sigma-I}<\epsilon
\end{equation}
在数 $\xi_i$ 的任意选择之下皆成立。并记作
\begin{equation}
I=\lim_{\lambda\rightarrow0}\sigma
\end{equation}
\end{definition}
上面的定义可看作“$\epsilon-\delta$语言”。 

下面用“序列语言”进行定义。在此之前,先引入一些概念。
\begin{definition}{基本区间分划序列}\label{def_DInt_2}
设用不同的方法将区间 $[a,b]$ 进行分划,记第 $i$ 种分划对应的 $\lambda$ 为 $\lambda_i$,如果分划对应的序列 $\lambda_1,\lambda_2,\cdots$ 收敛到0,则称这样的区间分划序列叫做\textbf{基本区间分划序列}。
\end{definition}

\begin{definition}{第二种定义}
不论 $\xi_i$ 如何选取,若对任一基本区间分划序列 $\{\lambda_i\}$ 对应的和 
\begin{equation}
\sigma=\sum_{i=0}^{n-1}f(\xi_i)\Delta x_i.
\end{equation}
的序列 $\{\sigma_i\}$,恒有极限 $I$,则称 $I$ 为和 $\sigma$ 的\textbf{极限}。
\end{definition}
\begin{theorem}{}
数 $I$ 是和 
\begin{equation}
\sigma=\sum_{i=0}^{n-1}f(\xi_i)\Delta x_i.
\end{equation}
的由\autoref{def_DInt_1} 定义的极限,当且仅当 $I$ 是\autoref{def_DInt_2} 定义的极限。
\end{theorem}
\textbf{证明:}
1.\autoref{def_DInt_1} $\Rightarrow$\autoref{def_DInt_2} 

设 $\{\lambda_i\}$ 是任一基本区间分划序列,于是 

\begin{equation}
\lim\limits_{i\rightarrow\infty}\lambda_i=0.
\end{equation}
这相当于对任意 $\delta_0>0$,都存在 $N_0$,使得 $i>N_0$ 时,都有 $\lambda_i<\delta_0$。于是对任意 $\epsilon>0$ 能找到的 $\delta>0$,都有 $N$ 存在,使得 $i>N$ 时,$\lambda_i<\delta$。于是由\autoref{def_DInt_1} 
\begin{equation}
\abs{\sigma_i-I}<\epsilon
\end{equation}
对 $i>N$ 都成立,其中 $\sigma_i$ 表示分划 $\lambda_i$ 对应的和。由序列极限的定义(\autoref{def_Lim_2}~\upref{Lim}),
\begin{equation}
\lim_{i\rightarrow\infty} \sigma_i=I
\end{equation}

2.\autoref{def_DInt_2} $\Rightarrow$\autoref{def_DInt_1} 

假定 $I$ 并非\autoref{def_DInt_1} 定义的极限。那么对某些 $\epsilon>0$,就没有对应的 $\delta$ 存在,即无论取怎么小的 $\delta$,至少能求出一个 $\lambda<\delta$ , 使得
\begin{equation}
\abs{\sigma-I}\geq\epsilon
\end{equation}
取一趋于0的正数序列 $\{\delta_i\}$。由刚刚所说的,对于任一数 $\delta_i$ 恒能找出数值 $\lambda_i$,虽然
\begin{equation}
\lambda_i<\delta
\end{equation}
但是 $\abs{\sigma_i-I}\geq\epsilon$。
于是这些数 $\lambda_i$ 组成某一序列 $\{\lambda_i\}$,对于它们恒有
\begin{equation}
\lambda_i<\delta_i\qquad(i=1,2,\cdots)
\end{equation}
因 $\lim\limits_{i\rightarrow\infty}\delta_i=0$,故
\begin{equation}
\lim\limits_{i\rightarrow\infty}\lambda_i=0.
\end{equation}
这和 $I$ 是\autoref{def_DInt_2} 的极限矛盾。

\textbf{证毕!}

\textbf{第二种定义和函数的序列语言的定义相同,这使得我们能够将极限理论中的概念很多定理应用到这个新定义的极限形状}。
\subsection{定积分}
\begin{definition}{定积分,可积函数}\label{def_DInt_3}
如果 $\lambda\rightarrow0$ 时,和  
\begin{equation}
\sum_{i=0}^{n-1}f(\xi_i)\Delta x_i
\end{equation}
的(有限)极限 $I$ 存在,则称这个极限 $I$ 为函数 $f(x)$ 在从 $a$ 到 $b$ 的区间上的\textbf{定积分},并记作
\begin{equation}
I=\int_a^bf(x)\dd x,
\end{equation}
$a,b$ 分别称为积分的\textbf{下限}和\textbf{上限}。此时,称函数 $f(x)$ 为区间 $[a,b]$ 上的\textbf{可积函数}。
\end{definition}

上面的定义是黎曼给出的,所以有时 $\sigma$ 被称为\textbf{黎曼和}。

应当指出,该定义实际上只能应用于有界函数上。否者,在某个部分区间上函数是无界的,于是 $\xi$ 在这区间上的选取可以使得 $f(\xi)$ 任意大,随之可使和 $\sigma$ 任意大,于是 $\sigma$ 不可能存在有限极限。