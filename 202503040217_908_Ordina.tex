% 序数算术(综述)
% license CCBYSA3
% type Wiki

本文根据 CC-BY-SA 协议转载翻译自维基百科\href{https://en.wikipedia.org/wiki/Ordinal_arithmetic}{相关文章}。

在数学中的集合论领域,序数算术描述了序数上的三种常见运算:加法、乘法和指数运算。每种运算基本上都可以通过两种不同的方式定义:一种是构造一个明确的良序集来表示运算结果;另一种是使用跨无限递归(transfinite recursion)来定义。康托范式(Cantor normal form)提供了一种标准化的序数表示方式。除了这些常见的序数运算之外,还有“自然”序数算术以及nimber运算。
\subsection{加法}  
两个良序集 \(S\) 和 \(T\) 的和是一个序数,它表示在**直积集** \(S \times \{0\}\) 和 \(T \times \{1\}\) 的并集上定义的**变体字典序**(即最不重要的位置优先的字典序)。这种排序方式保证了以下几点:\(S\) 中的每个元素都小于 \(T\) 中的每个元素;\(S\) 内部的比较保持原来的顺序;\(T\) 内部的比较也保持原来的顺序。 

对于序数加法 \(\alpha + \beta\),也可以通过对\(\beta\)进行**跨无限递归**来定义:当右加数 \(\beta = 0\) 时,\(\alpha + 0 = \alpha\) 对任意 \(\alpha\)成立。当 \(\beta > 0\) 时,\(\alpha + \beta\)是所有\(\alpha + \delta\)(其中 \(\delta < \beta\))中最小的严格大于它们的序数。  

具体分成**后继序数**和**极限序数**的情况:
\[\alpha + 0 = \alpha\]
\[\alpha + S(\beta) = S(\alpha + \beta)\] 其中 \(S\) 表示后继函数。\[\alpha + \beta = \bigcup_{\delta < \beta} (\alpha + \delta)\] 当 \(\beta\)是一个极限序数时。

在自然数上,序数加法与通常的加法是相同的。  
第一个超限序数是 \(\omega\),它是所有自然数的集合,接下来的序数是 \(\omega + 1\)、\(\omega + 2\) 等。  

\(\omega + \omega\) 表示两个按通常顺序排列的自然数副本,第二个副本完全排在第一个副本的右侧。  
用 \(0' < 1' < 2' < \dots\) 表示第二个副本,那么 \(\omega + \omega\) 看起来像:

\[
0 < 1 < 2 < 3 < \dots < 0' < 1' < 2' < \dots~
\]

这和 \(\omega\) 不同,因为在 \(\omega\) 中,只有 0 没有直接前驱,而在 \(\omega + \omega\) 中,0 和 0' 都没有直接前驱。