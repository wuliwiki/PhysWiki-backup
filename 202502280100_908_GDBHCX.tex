% 哥德巴赫猜想(综述)
% license CCBYSA3
% type Wiki

本文根据 CC-BY-SA 协议转载翻译自维基百科\href{https://en.wikipedia.org/wiki/Goldbach\%27s_conjecture}{相关文章}。

\begin{figure}[ht]
\centering
\includegraphics[width=6cm]{./figures/0b683c0c5cb7d58d.png}
\caption{哥德巴赫致欧拉的信,日期为1742年6月7日(拉丁语-德语)[1]} \label{fig_GDBHCX_1}
\end{figure}
哥德巴赫猜想是数论和整个数学中最古老且最著名的未解问题之一。它指出,所有大于2的偶数自然数都可以表示为两个质数的和。

尽管经过了大量的努力,这个猜想已经被证明对所有小于 \(4 \times 10^{18}\) 的整数成立,但仍未被完全证明。
\subsection{历史}  
\subsubsection{起源}  
1742年6月7日,普鲁士数学家克里斯蒂安·哥德巴赫写信给莱昂哈德·欧拉(信件编号XLIII),[2] 在信中他提出了以下猜想:

“dass jede Zahl, welche aus zweyen numeris primis zusammengesetzt ist, ein aggregatum so vieler numerorum primorum sey, als man will (die unitatem mit dazu gerechnet), bis auf die congeriem omnium unitatum”  
每个可以表示为两个质数和的整数,也可以表示为任意多个质数之和,直到所有的项都是单位数。  
哥德巴赫遵循了现在已被废弃的将1视为质数的传统[3],因此单位数的和也可以视为质数的和。然后,他在信的边缘提出了第二个猜想,暗示第一个猜想:[4]  

“Es scheinet wenigstens, dass eine jede Zahl, die grösser ist als 2, ein aggregatum trium numerorum primorum sey.”  
至少似乎每个大于2的整数可以表示为三个质数的和。

欧拉在1742年6月30日的信中回复,提醒哥德巴赫他们之前的对话(“… so Ew vormals mit mir communicirt haben …”),当时哥德巴赫提到,第一个猜想将从以下陈述中得到推导:

“每个正偶整数可以表示为两个质数的和。”  
这实际上等同于他的第二个边际猜想。在1742年6月30日的信中,欧拉写道:[6][7]  

“Dass ... ein jeder numerus par eine summa duorum primorum sey, halte ich für ein ganz gewisses theorema, ungeachtet ich dasselbe nicht demonstriren kann.”  
我认为“每个偶数都是两个质数的和”是一个完全确定的定理,尽管我无法证明它。
