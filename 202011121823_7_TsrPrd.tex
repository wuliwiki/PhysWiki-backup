% 张量积
% 张量|张量积|张量积空间|张量变换
\pentry{张量\upref{Tensor},爱因斯坦求和约定\upref{EinSum}}

在张量\upref{Tensor}词条中我们知道,一个张量可以看成是将若干个线性空间$V$映射到域$\mathbb{F}$上的映射.

域是可以进行加减乘除的集合,因此我们可以在域上定义映射的乘积.比如如果有实数域上的实函数$f, g:\mathbb{R}\rightarrow\mathbb{R}$,那么我们可以定义两个函数的乘积$f\cdot g$如下:$\forall x\in\mathbb{R}, f\cdot g(x)=f(x)\cdot g(x)$.就是说,把各点的函数值相乘,得到的结果还是一个函数,它就是两个函数的乘积.

同样的方法可以应用到张量上,这样所得到映射乘积,就是所谓的张量积.

\begin{definition}{张量积}

给定线性空间$V$,在线性空间$V$上有$n-$线性函数$f$和$m-$线性函数$g$,则可以定义$(n+m)-$线性函数$f\cdot g$,方式为:对于任意$x_i, y_j\in V$,$f\cdot g(x_1, \cdots, x_n, y_1, \cdots, y_m)=f(x_1, \cdots, x_n)\cdot g(y_1, \cdots, y_m)$.

\end{definition}

用矩阵可以直观地看出张量积的“升维”性质,但是这个方法很难推广到任意阶的张量.我们举一个一阶张量相乘得到二阶张量的例子:

\begin{example}{一阶张量的张量积}
在一个$3$维线性空间$V$里默认取某个基时,有向量$x^i\in V$和一个对偶向量(不一定是$x^i$的对偶)$y_j\in V^*$,那么$x^i$可以看成是把$V^*$里的向量映射为标量的张量,$y_j$可以看成是把$V$里的向量映射为标量的张量.如果在给定基下,把$V$中元素的坐标都写为列矩阵,$V^*$中的坐标写为行矩阵,比如$x^i=\pmat{x^1, x^2, x^3}\Tr$,$y_j=\pmat{y_1, y_2, y_3}$,那么可以通过乘积得到矩阵$\bvec{M}=\pmat{x^1, x^2, x^3}\Tr y_j=\pmat{y_1, y_2, y_3}$,这个矩阵就表示把一个向量和一个对偶向量映射到一个标量上的二阶张量.

为便于理解,我显式地写一遍这个矩阵乘法过程:

\begin{equation}
\bvec{M}=\pmat{x^1\\x^2\\x^3}\times\pmat{y_1, y_2, y_3}=\pmat{x^1y_1, x^1y_2, x^1y_3\\x^2y_1, x^2y_2, x^2y_3\\x^3y_1, x^3y_2, x^3y_3}
\end{equation}
\end{example}

如果参与进行张量积的张量阶数超过$1$,那么就不再能像上边这样简单地用矩阵表达了,因此我们必须使用爱因斯坦求和约定来简化表达.我把表达的规则写成以下定理:

\begin{theorem}{}
如果在给定基下,两个张量分别表示为$a^{i_1i_2\cdots i_m}_{j_1j_2\cdots j_n}$和$b^{k_1\cdots k_s}_{l_1\cdots l_t}$,那么它们的张量积就是$a^{i_1i_2\cdots i_m}_{j_1j_2\cdots j_n}b^{k_1\cdots k_s}_{l_1\cdots l_t}$.张量积的阶数为各张量阶数之和.
\end{theorem}

比如说,张量$a^{ij}_k$和$b^r_s$的张量积是$a^{ij}_kb^r_s$,其阶数为$3+2=5$.






