% 衍射光栅
% license CCBYSA3
% type Wiki

(本文根据 CC-BY-SA 协议转载自原搜狗科学百科对英文维基百科的翻译)

在光学中,\textbf{衍射光栅}是一种具有周期性结构的光学元件,它将光分裂并衍射成沿不同方向传播的几束光束。它形成的着色是一种具有结构性的着色。[1][2] 这些光束的方向取决于光栅的间距和光的波长,因此光栅充当色散元件。由于这些原因,光栅通常用于制作单色仪和光谱仪。

在实际应用中,光栅通常在其表面有脊或划线,而不是暗线。这种光栅可以是透射型的,也可以是反射型的。 除了振幅调制型光栅以外,也可以制备出相位调制型的光栅。这种光栅通常采用全息法制备[3]

衍射光栅的原理是詹姆斯·格雷戈里在牛顿棱镜实验大约一年后发现的,最初是用鸟类羽毛等物品作为光栅。[4] 第一个人造衍射光栅是费城发明家大卫·黎顿郝斯在1785年左右制造的,他把头发串在两个细螺丝之间。[5][6] 这类似于1821年著名的德国物理学家约瑟夫·冯·夫琅和费的线衍射光栅。[7][8] 19世纪60年代,格里弗斯瓦尔德的弗里德里希·阿道夫·诺贝特(1806-1881)创造了线距d最小的光栅,[9] 随后两位美国人路易斯·莫里斯·拉瑟弗德(1816-1892)和威廉·罗杰斯(1804-1882)占据了领先地位,[10][11] 到19世纪末,亨利·奥古斯谭斯·罗兰(1848-1901)的凹面光栅成为了现有的最佳光栅[12][13]。

当被宽光谱(例如连续谱)光源照射时,衍射可以产生“彩虹”色。 光存储磁盘(如光盘或数字视盘)上紧密间隔的窄轨道产生的闪光效应就是一个例子,而薄层油(或汽油等)产生的类似彩虹效应则不是由光栅引起的,而是由具有紧密间隔的透射层中由于反射导致的干涉效应引起的(参见下面的例子)。光栅有平行线,而光盘有一个螺旋的精细间隔的数据轨道。当透过半透明的细间距伞状织物覆盖物观察亮点光源时,也会出现衍射色。基于反射光栅片的装饰性图案塑料薄膜非常便宜,并且很常见。

\subsection{工作原理}
\begin{figure}[ht]
\centering
\includegraphics[width=6cm]{./figures/4a24c3c178226e65.png}
\caption{一种只反射房间荧光灯光谱中绿色部分的衍射光栅。} \label{fig_YSGS_1}
\end{figure}

光栅间距与入射光束和衍射光束的角度之间的关系称为\textbf{光栅方程}。

根据惠更斯-菲涅尔原理,传播波波前上的每个点都可以被认为是一个点源,通过将这些单独点源的贡献相加,可以找到任何后续点的波前。

光栅可以是“反射”或“透射”型的,分别类似于镜子或透镜。光栅具有“零阶模式”(其中m = 0),在这种模式下没有衍射。光线的行为根据反射和折射定律分别与镜子或透镜相同。

理想光栅由一组间距为d的狭缝组成,光栅间距必须比关注的波长更宽才能引起衍射。假设一束波长为λ的单色平面波正入射(垂直于光栅)到光栅上,光栅中的每个狭缝充当准点光源,光从该准点光源向所有方向传播(尽管这通常限于半球面)。光与光栅相互作用后,衍射光由光栅中每个狭缝发出的干涉波分量之和组成。在衍射光可以通过的任何给定空间点,光栅中每个狭缝的路径长度都是不同的。因为路径长度通常是不同的,所以每个狭缝在该点的波的相位也是不同的。因此,它们相互加减,通过干涉相长或干涉相消产生透射峰和透射谷。当来自相邻狭缝的光之间的路径差等于波长的一半( λ2)时,波是异相的,因此彼此抵消以产生最小强度的点。类似地,当路径差为λ时,相位相加,光强出现最大值。最大值出现在满足关系式 d sinθmλ = | m |的角度θm处,其中θm是衍射光线和光栅法向矢量之间的角度,d是从一个狭缝中心到相邻狭缝中心的距离, m是表示关注的传播模式的阶数。

因此,当光正入射到光栅上时,衍射光强在角度θm处具有最大值,由下式给出:

\begin{figure}[ht]
\centering
\includegraphics[width=6cm]{./figures/b335c498f5cf5235.png}
\caption{通过衍射的衍射光栅(1)和通过反射的棱镜(2)所得光谱的比较。长波段(红色)的衍射较多,但反射比短波段(紫色)少。} \label{fig_YSGS_2}
\end{figure}

