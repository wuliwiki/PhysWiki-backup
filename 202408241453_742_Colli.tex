% 无碰撞暗物质的引力坍缩
% license Usr
% type Tutor

暗物质星系晕的形成是一个概念上定义得很好的理论问题:从一个由无碰撞和无耗散体组成的大致球形轻微过密区域开始,并让它在重力作用下演化。系统开始在自己的重量下坍缩(在某个时刻,不均匀性增长超出了可计算的线性近似),但坍缩自我限制,没有达到无限密度的点。这是因为暗物质是无碰撞的,不能耗散能量,所以在坍缩过程中,它的引力势能必须转化为涉及的暗物质粒子的动能。球体最终放松到一个自重力准静态结构,由暗物质粒子的随机运动支撑。整个过程被称为位力化,产生的系统是位力化的,即其能量在势能$V$和动能$K$之间分布,符合位力定理$\langle K\rangle  = -1/2 \langle V\rangle $。

位力定理提供了足够的信息来估计涉及的典型尺度。为了简化,我们考虑一个具有均匀密度的球形暗物质过密区域,其总质量为$M_{tot}$(也在文献中考虑了具有非恒定密度的椭球形坍缩模型)。下面讨论的主要阶段如\autoref{fig_Colli_1} 所示。
\begin{figure}[ht]
\centering
\includegraphics[width=14cm]{./figures/1d36885fea88a550.png}
\caption{} \label{fig_Colli_1}
\end{figure}

我们定义ρta为在转折点的均匀密度,即过密区域停止跟随由大爆炸引起的膨胀,停滞并开始坍缩成一个受引力束缚的物体的时刻。这标志着结构形成的开始。由于暗物质正在停滞,它的动能Kta消失,因此它的总能量Eta = Kta + Vta完全由它的引力势能决定:

\[ V_{\text{ta}} = -\frac{3}{5} \frac{G M_{\text{tot}}^2}{r_{\text{ta}}} ~.\]

其中rta是球体在转折点的大小,ρin是球体内部的密度(在我们的近似中取为ρta)。如果球对称性绝对精确,即如果所有暗物质粒子都遵循完美的径向轨迹,坍缩将朝着无限密度发展,形成黑洞。实际上,系统平均保持球形,并且由于暗物质粒子的混沌运动(它们只通过重力相互作用),达到了有限的密度。在这个维里化过程之后,能量在动能和势能之间分布,使得Evir = Kvir + Vvir = Vvir/2,最后一个等式遵循维里定理。由于维里化保持总能量守恒,势能Vvir = −3GM2 tot/5rvir(我们仍然近似密度为常数,ρvir = Mtot/(4/3πr3 vir),现在rvir是维里化晕的半径),因此意味着rvir = rta/2。也就是说,坍缩导致了一个松弛的晕,其大小是转折点泡沫的一半,密度大约是其8倍,ρvir = 8ρta。\footnote{这意味着无碰撞暗物质的坍缩不会形成黑洞。正常物质的情况截然不同。由于正常物质具有大的散射截面,对光子是不透明的,因此光子只能由紧凑物体的表面而不能由整个体积发射到周围环境。一个大型正常物质过密区域的坍缩倾向于分裂,形成许多恒星,而不是一个大物体。然后,恒星可以演化成恒星质量黑洞。超大质量黑洞是如何形成的是一个活跃的研究领域,有一种可能性是它们是通过吸积尘埃、气体以及恒星和恒星质量黑洞形成的。}

形成的晕的平均密度 \(\rho_{\text{vir}}\) 大致是周围宇宙密度 \(\bar{\rho}\) 的200倍:

\[ \rho_{\text{vir}} \approx 200 \bar{\rho} = \frac{3H^2}{8\pi G} = \frac{1}{6\pi G t^2_{\text{vir}}}~. \]

其中 \(t_{\text{vir}}\) 是维里化结束的时间标记[32](这里我们使用了附录C中的方程(C.7)和方程(C.14),假设物质主导)。通常,维里半径就是这样定义的:在维里半径内晕的平均密度是宇宙密度的200倍,记作 \(r_{200}\)。在 \(r_{200}\) 内的总质量被视为晕的总质量的度量。实际上,至少在数值模拟中看到的更为复杂,但数量级是正确的。

除了其全局属性外,我们可以通过分析坍缩并特别关注松弛机制来获得对最终暗物质晕(特别是我们在这里感兴趣的其密度和速度分布)稍微更精细的理解。有几个这样的动力学过程在起作用,推动系统达到平衡配置[32]。对于无碰撞暗物质粒子来说,重要的机制是所谓的暴力松弛[33]:由于暗物质粒子本身产生的引力势也在随着持续的坍缩而变化,因此一个暗物质粒子的能量随时间变化。这种机制在坍缩的短时间尺度上有效,因此得名“暴力”。如果引力势是时间依赖的,那么给定暗物质粒子的总能量除以其质量,\(\epsilon = v^2/2 + \phi\),随时间的变化为 \(d\epsilon/dt = \partial\phi/\partial t\)。这个最后的关系很容易验证:

\[ \frac{d\epsilon}{dt} = \frac{\partial\epsilon}{\partial v} \cdot \frac{dv}{dt} + \frac{\partial\epsilon}{\partial\phi} \cdot \frac{d\phi}{dt} = v \cdot (-\nabla\phi) + \frac{1}{2} \frac{\partial\phi}{\partial t} + v \cdot \nabla\phi = \frac{1}{2} \frac{\partial\phi}{\partial t}~. \]

其中等式很容易从链式法则、\(\epsilon\) 的定义以及 \(dv/dt = -\nabla\phi\) 得出,这是牛顿定律 \(a = F/m\) 在仅有引力相互作用的情况下的形式。一个粒子是获得还是失去能量是一个复杂问题。通过考虑一个从晕的外围开始坍缩的暗物质粒子,可以定性地弄清楚发生了什么(它本身将首先坍缩然后扩展,直到粒子达到维里化)。在下落过程中,暗物质粒子将潜入由于坍缩的晕而形成的深势中并增加动能。穿过中心点(由于其无碰撞特性而未受伤害)后,它将爬出一个更浅的势,因为其同伴暗物质粒子正在分散,因此将花费少于之前获得的动能,导致净能量增加。对于从内部位置开始的粒子则相反。例如,一个在中心静止的粒子不会移动,而当原晕坍缩时,其中心的势变得更深。通常,单个粒子的能量增加或减少取决于其初始位置和初始速度,以及所有其他暗物质的分布,以一种复杂的方式[32,33]。

整个过程实际上相当于暗物质粒子经历许多引力相互作用。因此,它们的最终速度是由许多随机贡献的总和给出的。由于中心极限定理,它们的能量分布将趋向于高斯分布:

\[ f(\epsilon) = n_0 \left(\frac{2\pi\sigma^2}{3}\right)^{3/2} e^{-\epsilon/\sigma^2} ~. \]

其中乘以指数的奇异归一化因子的形式很快就会清楚。指数是 \(\epsilon = \frac{1}{2}v^2 + \phi\),由此可以立即得出能量分布 \(f(\epsilon)\) 可以被解释为暗物质速度的分布 \(f(v)\),呈现为麦克斯韦-玻尔兹曼形式:

\[ f(v) \propto e^{-v^2/2\sigma^2}~. \]

对于一个完全维里化的晕,均匀的“温度”(与速度分散σ有关)由其总质量给出。为了证明这一点,我们接下来讨论暗物质密度分布 \(\rho(r)\)。对于球对称系统,\(f(v)\) 和 \(\rho(r)\) 之间存在一个唯一的对应关系(这种一般关系被称为爱丁顿公式,从 \(f(v)\) 导出 \(\rho(r)\) 作为爱丁顿反演的艺术)。它可以通过首先对所有速度积分 \(f(\epsilon)\) 来推导,然后得到暗物质密度分布的隐式表达式:

\[ \rho = M \int_0^\infty dv \, 4\pi v^2 f(\epsilon) = M \int_0^\infty dv \, 4\pi v^2 \frac{n_0}{(2\pi\sigma^2)^{3/2}} e^{-(v^2/2+\phi)/\sigma^2} = \rho_0 e^{-\phi/\sigma^2}~. \]

其中 \(\rho_0 = Mn_0\)。方程(2.6)以 \(\rho\) 的形式给出引力势。将这个表达式代入泊松方程 \(\nabla^2\phi = 4\pi G\rho\),写成球坐标形式,得到:

\[ \frac{d}{dr} \left( r^2 \frac{d}{dr} \ln\rho \right) = -4\pi G \frac{\sigma^2}{r^2} \rho~. \]

解是:

\[ \rho(r) = \frac{\sigma^2}{2\pi G} \frac{1}{r^2}~. \]

5更准确地说,我们实际上处理的是分布函数 \(f(x, v, t)\),定义为 \(f(x, v, t) d^3x d^3v\) 是在时间 \(t\) 时在以 \(x, v\) 为中心的 d^3x d^3v 相空间体积中找到粒子的概率。对于一个球对称稳态系统,可以证明 \(f\) 仅通过总能量(这是一个运动积分量)依赖于 \(x, v\)(\(f\) 被称为是遍历的)。这就是为什么通过对所有速度积分 \(f(\epsilon)\) 可以得到暗物质粒子的空间分布,即它们的数量密度(在公式(2.4)中的 \(n_0\) 使用已经是第一个提示)。将其乘以暗物质质量 \(m\) 就得到了质量密度 \(\rho\)。有关更多细节,参见例如 Binney & Tremaine [32] 的第4章。
 



