% 电磁场的参考系变换
% 参考系变换|洛伦兹变换|电磁场

\begin{issues}
\issueDraft
\end{issues}

\pentry{洛伦兹变换\upref{SRLrtz}, 安培环路定理\upref{AmpLaw}, 洛伦兹力\upref{Lorenz}}

% 让我比较自豪的是,我高中的时候曾经独立推导出了洛伦兹变换,再通过这里给出的例子独立推导出了电磁场的参考系变换. 上大学时我才在新概念电磁学上看到一模一样的变换公式.

我们下面以一道简单的例题来分析

在 $S$ 参考系中有一个电流为 $I$ 的无限长直导线, 令电流正方向为 $\uvec x$. 作为一个简单对称的导线模型, 我们假设导线中的所有正电荷的线密度为 $\lambda$, 以速度 $v_0\uvec x$ 运动, 导线中的所有负电荷的线密度为 $-\lambda$,  以速度 $-v_0\uvec x$ 运动. 距离导线 $r_0$ 处有一个电荷为 $q$ 的粒子沿 $\uvec x$ 方向以速度 $v$ 运动. 另一个参考系 $S'$ 相对 $S$ 沿 $\uvec x$ 方向运动, 速度为 $u$. 在这两个参考系中, 粒子所受的电磁力是否相同?

\addTODO{图}

\subsubsection{错误的分析}
在 $S$ 参考系中, 电流在导线周围产生的磁场使粒子受到垂直于导线洛伦兹力. 而在 $S'$ 系中, 导线同样产生磁场,而粒子却是静止的所以不受力. 这说明在不同参考系中粒子受的电磁力是不同的.

\subsubsection{正确的分析}
上面 $u = 0$ 时的结论是正确的,空间中只有磁场没有电场. 然而当 $u \ne 0$ 时,由于导线中的异号电荷运动快慢不同,相对论尺缩效应使两种电荷的线密度产生区别,从而产生垂直导线的电场,与洛伦兹力共同作用在粒子上, 使粒子受力与 $u$ 无关.

\subsection{具体计算}
导线的电流为
\begin{equation}
I = 2\lambda v_0
\end{equation}
点电荷感受到的磁场大小为
\begin{equation}
B = \frac{\mu_0}{2\pi} \frac{I}{r_0}
\end{equation}
方向右右手螺旋定则决定. 那么它的受力大小为 $F = q v B$, 方向指向导线.



(未完成)
