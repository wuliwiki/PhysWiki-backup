% 数学分析笔记 2

本文参考 \cite{Rudin}.

\subsection{Chap 4. 连续性}

\begin{itemize}
\item 4.1 \textbf{函数的极限}:令 $f:E\subset X\to Y$, $X,Y$ 为度量空间, 且 $p$ 是 $E$ 的极限点. 凡是我们写当 $x\to p$ 时 $f(x)\to q$, 或 $\lim_{x\to p}f(x)=q$, 就是存在 $q\in Y$ 具有以下性质: 对每个 $\epsilon>0$, 存在 $\delta>0$, 使 $d_{Y}(f(x),q)<\epsilon$ 对满足 $0<d_X(x,p)<\delta$ 的一切点 $x\in E$ 成立.

\item 4.2 $\lim_{x\to p}f(x)=q$ 当且仅当 $\lim_{n\to\infty} f(p_n)=q$ 对 $E$ 中满足 $p_n\ne p$, $\lim_{n\to\infty} p_n=p$ 的每个序列 $\{p_n\}$ 成立.

\item 4.6 连续性: $\lim_{x\to p}f(x)=f(p)$.

\item 4.7 $h(x) = g(f(x))$. 如果 $f$ 在点 $p$ 连续, 且 $g$ 在点 $f(p)$ 连续, 那么 $h$ 在点 $p$ 连续.

\item 4.8 定理: 度量空间 $X,Y$ 的函数 $f:X\to Y$ 连续, 当且仅当对 $Y$ 的每个开集 $V$, $f^{-1}(V)$ 是 $X$ 中的开集. 这是连续性的一个极有用的特征.

\item 推论: 度量空间 $X,Y$ 的函数 $f:X\to Y$ 连续, 当且仅当对 $Y$ 的每个闭集 $C$, $f^{-1}(C)$ 是 $X$ 中的闭集.

\item 4.9 设 $f,g$ 是度量空间 $X$ 上的复连续函数, 那么 $f+g$, $fg$ 与 $f/g$ 在 $X$ 上连续. 在最后一个中, 假定对一切 $x\in X$, $g(x)\ne 0$.

\item (a) 度量空间上的$\bvec f:X\to R^k$ 连续当且仅当每个分量函数都连续. (b) 如果 $\bvec f, \bvec g:X\to R^k$ 连续, 那么 $\bvec f+\bvec g$ 与 $\bvec f\vdot \bvec g$ 都在 $X$ 上连续.

\item 4.13 有界: $\abs{\bvec f(x)}\leqslant M$.

\item 4.14 若度量空间 $X,Y$ 的函数 $f:X\to Y$ 是连续的, 那么 $f(X)$ 是紧的.

\item 4.15 如果 $\bvec f$ 是把紧度量空间 $X$ 映入 $R^k$ 内的连续映射, 那么 $\bvec f(X)$ 是闭的和有界的. 因此 $\bvec f$ 是有界的.

\item 4.16 如果 $\bvec f$ 是紧度量空间 $X$ 上的连续实函数, 且 $M = \sup_{p\in X} f(p)$, $m=\inf_{p\in X} f(p)$, 那么一定存在 $r,s\in X$ 使 $f(r)=M$ 以及 $f(x)=m$.

\item 4.17 设 $f$ 是把紧度量空间 $X$ 映满度量空间 $Y$ 的连续 1-1 映射, 那么逆映射 $f^{-1}$ 是 $Y$ 映满 $X$ 的连续映射.

\item 4.18 对度量空间 $X,Y$ 的函数 $f:X\to Y$, 称 $f$ 在 $X$ 上一致连续, 若对每个 $\epsilon>0$ 总存在 $\delta >0$ 对一切满足 $d_X(p,q)<\delta$ 的 $p,q\in X$ 都能使 $d_\gamma(f(p),f(q))<\epsilon$.

\item 4.19 设 $f$ 是把紧度量空间 $X$ 映入度量空间 $Y$ 的连续映射. 那么 $f$ 在 $X$ 上一致连续.

\item 4.20 设 $E$ 是 $R^1$ 中的非紧集, 那么 (a) 有在 $E$ 上连续却无界的函数. (b) 有在 $E$ 上连续且有界, 却没有最大值的函数. (c) 如果 $E$ 是有界的, 有在 $E$ 上连续却不一致连续的函数.

\item 4.22 设 $f$ 是把连通的度量空间 $X$ 映入度量空间 $Y$ 内的连续映射, $E$ 是 $X$ 的连通子集, 那么 $f(E)$ 是连通的.

\item 4.25 设 $f$ 定义在 $(a,b)$ 上, 定义一点的\textbf{左极限}和\textbf{右极限}…… 显然极限存在当且仅当左极限和右极限存在且相等. 如果函数在一点不连续, 就说在这点\textbf{间断(discontinuous)}.

\item 4.26 设 $f$ 定义在 $(a,b)$ 上, 如果 $f$ 在一点 $x$ 间断, 并且如果 $f(x+)$ 和 $f(x-)$ 都存在, 就说 $f$ 在 $x$ 发生了\textbf{第一类间断}. 其他间断称为\textbf{第二类间断}.

\item 4.28 设 $f:(a,b)\to R$, 若 $a<x<y<b$ 时有 $f(x)\leqslant f(y)$, 就说 $f$ 在 $(a,b)$ 上\textbf{单调递增}; 若有 $f(x)\geqslant f(y)$ 就是\textbf{单调递减}. 二者统称为\textbf{单调函数}.

\item 4.30 设 $f$ 在 $(a,b)$ 上单调, 那么 $(a,b)$ 中使 $f$ 间断的点最多是可数的.

\item 4.31 间断点不一定是孤立点. 

\item 4.32 对任意 $c\in R$, 集合 ${x|x>c}$ 叫做 $+\infty$ 的一个邻域, 记为 $(c,+\infty)$. 类似地, $(-\infty, c)$ 是 $-\infty$ 的一个邻域.

\item 4.33 把函数的极限用领域的语言拓展到了广义实数系.
\end{itemize}

\subsection{Chap 5. 微分法}
\begin{itemize}
\item 5.1 \textbf{导数(导函数)}: 定义在 $[a,b]$ 上的实值函数, …… $f'(x) = \lim_{t\to x} [f(t)-f(x)]/(t-x)$.

\item 如果 $f'$ 在点 $x$ 有定义, 就说 $f$ 在 $x$ 点\textbf{可微}或\textbf{可导}. 如果在 $E\subset [a,b]$ 的每一点有定义, 就说 $f$ 在 $E$ 上可微.

\item 5.2 …… 若 $f$ 在 $x\in [a,b]$ 可微(可导), 那么它在 $x$ 点连续.

\item 5.5 …… $h'(x) = g'(f(x))f'(x)$.

\item 设 $f$ 是定义在度量空间 $X$ 上的实函数, 说 $f$ 在点 $p\in X$ 取得\textbf{局部极大值}, 如果存在 $\delta>0$, 对任意 $q\in X$ 且 $d(p,q)<\delta$ 有 $f(q)\leqslant f(p)$. 局部极小值的定义类似.

\item 5.8 设 $f$ 定义在 $[a,b]$ 上; $x\in [a,b]$, 如果 $f$ 在点 $x$ 取得局部极大值而且 $f'(x)$ 存在, 那么 $f'(x) = 0$.

\item 
\end{itemize}


\subsection{Chap 6. Riemann-Stieltjes 积分}
\begin{itemize}
\item 6.2 函数 $f$ 关于单调递增函数 $\alpha$ 在 Riemann 意义上可积, 记为 $f\in \mathcal{R}(\alpha)$: $\int_a^b f(x) \dd\alpha(x)$ 或 $\int_a^b f\dd{\alpha}$.

\item 6.6 在 $[a,b]$ 上 $f\in\mathcal{R}(\alpha)$ 当且仅当对任意的 $\epsilon>0$, 存在一个分法 $P$ 使 $U(P,f,\alpha)-L(P,f,\alpha)<\epsilon$.

\item 6.8 如果 $f$ 在 $[a,b]$ 上连续, 那么在 $[a,b]$ 上 $f\in \mathcal{R}(\alpha)$.

\item 6.9 如果 $f$ 在 $[a,b]$ 上单调, $\alpha$ 在 $[a,b]$ 上连续, 那么 $f\in \mathcal{R}(\alpha)$.

\item 6.10 假设 $f$ 在 $[a,b]$ 上有界, 只有有限个间断点. $\alpha$ 在 $f$ 的每个间断点上连续, 那么 $f\in \mathcal{R}(\alpha)$.

\item 6.11 假设在 $[a,b]$ 上 $f\in \mathcal{R}(\alpha)$, $m\leqslant f\leqslant M$. $\phi$ 在 $[m, M]$ 上连续, 并且在 $[a,b]$ 上 $h(x) = \phi(f(x))$. 那么在 $[a,b]$ 上 $h\in \mathcal{R}(\alpha)$.

\item 6.12 (a) 如果在 $f_1,f_2 \in \mathcal{R}(\alpha)$, 那么 $f_1+f_2 \in \mathcal{R}(\alpha)$. 对任意常数 $c$, $cf\in \mathcal{R}(\alpha)$, 并且 $\int_a^b (f_1+f_2)\dd{\alpha} = \int_a^bf_1\dd{\alpha} + \int_a^bf_2\dd{\alpha}$. $\int_a^b cf\dd{\alpha} = c\int_a^b f\dd{\alpha}$.

\item  6.12 (b) 如果在 $[a,b]$ 上 $f_1\leqslant f_2$, 那么 $\int_a^bf_1\dd{\alpha} \leqslant \int_a^bf_2\dd{\alpha}$.

\item  6.12 (c) 如果在 $[a,b]$ 上 $f\in \mathcal{R}(\alpha)$, 并且 $a<c< b$, 那么在 $[a,c]$ 及 $[c,b]$ 上 $f\in \mathcal{R}(\alpha)$, 并且 $\int_a^cf\dd{\alpha}+\int_c^bf\dd{\alpha} = \int_a^bf\dd{\alpha}$

\item 6.12 (d) 如果在 $[a,b]$ 上 $f\in \mathcal{R}(\alpha)$ 并且 $[a,b]$ 上 $\abs{f(x)}\leqslant M$, 那么 $\abs{\int_a^bf\dd{\alpha}} \leqslant M[\alpha(b)-\alpha(a)]$.

\item 6.12 (e) 如果 $f\in \mathcal{R}(\alpha_1)$ 并且 $f\in \mathcal{R}(\alpha_2)$, 那么 $f\in \mathcal{R}(\alpha_1+\alpha_2)$ 并且 $\int_a^bf\dd{(\alpha_1+\alpha_2)} = \int_a^bf\dd{\alpha_1}+ \int_a^bf\dd{\alpha_2}$. 如果 $f\in \mathcal{R}(\alpha)$ 且 $c$ 是正常数, 那么 $f\in \mathcal{R}(c\alpha)$ 而且 $\int_a^bf\dd{(c\alpha)} = c\int_a^bf\dd{\alpha}$

\item 如果在 $[a,b]$ 上 $f,g\in \mathcal{R}(\alpha)$ 那么 (a) $fg\in \mathcal{R}(\alpha)$; (b) $\abs{f}\in \mathcal{R}(\alpha)$ 而且 $\abs{\int_a^bf\dd{\alpha}} \leqslant \int_a^b\abs{f}\dd{\alpha}$

\item 6.17 $\int_a^b f\dd{\alpha} = \int_a^b f(x)\alpha'(x)\dd{x}$.

\item 6.20 设在 $[a,b]$ 上 $f\in \mathcal{R}$, 对于 $a\leqslant x\leqslant b$, 令 $F(x) = \int_a^x f(t)\dd{t}$. 那么 $F$ 在 $[a,b]$ 上连续; 如果 $f$ 又在 $[a,b]$ 的 $x_0$ 点连续, 那么 $F$ 在 $x_0$ 可微, 并且 $F'(x_0) = f(x_0)$.

\item 6.21 微积分基本定理: 如果在 $[a,b]$ 上 $f\in \mathcal{R}$. 在 $[a,b]$ 上又有可微函数 $F$ 满足 $F' = f$, 那么 $\int_a^b f(x)\dd{x} = F(b)-F(a)$.

\item 6.22 分部积分:……
\end{itemize}

\subsection{Chap 7. 函数序列与函数项级数}

\begin{itemize}
\item 7.1 假设 $n=1,2,...$, $\{f_n\}$ 是一个定义在集 $E$ 上的函数序列, 再假设数列 $\{f_n(x)\}$ 对每个 $x\in E$ 收敛. 我们便可以由 $f(x) = \lim_{n\to\infty} f_n(x)$ ($x\in E$) 确定一个函数 $f$. 这时我们说 $\{f_n\}$ 在 $E$ 上收敛. $f$ 是 $\{f_n\}$ 的极限或\textbf{极限函数}.

\item 类似地, 如果对每个 $x\in E$, $\sum f_n(x)$ 收敛, 如果定义 $f(x) = \sum_{n=1}^\infty f_n(x)$ ($x\in E$), 就说函数 $f$ 是级数 $\sum f_n$ 的和.

\item …… 所以, 积分的极限和极限的积分, 即使两者都是有限的, 也未必相等.

\item 7.7 如果对每一个 $\epsilon >0$, 有一个整数 $N$, 使得 $n\geqslant N$ 时, 对一切 $x\in E$, 有 $\abs{f_n(x)-f(x)} \leqslant \epsilon$, 我们就说函数序列在 $E$ 上\textbf{一致收敛}于函数 $f$. 一致收敛必定\textbf{逐点收敛}.

\end{itemize}


\subsection{Chap 8. 一些特殊函数}
