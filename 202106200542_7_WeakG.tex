% 引力的弱场近似
% 广义相对论|relativity|引力|gravity|弱场|weak field|测地线|geodesic|牛顿力学|闵可夫斯基时空|Minkowski spacetime|时空|spacetime|流形|manifold|闵可夫斯基度规|Minkowski metric

\pentry{测地线\upref{geodes}}

广义相对论的革命性创见在于将引力解释为时空的几何效应.广义相对论将引力视为非力作用,并假设不受力的物质的运动轨迹是测地线,其参数取该物质的本征时间.从测地线一节中的\autoref{geodes_eq1}~\upref{geodes}可知,非平坦的度量在给定图中可能有非零的Christoffel符号,使得测地线方程的解不再是这个图中的一条匀速直线,这就启发了我们,如果把时空看成是流形,那么引力可能通过影响联络来造成表面上的“扭转”物质轨迹.当然,由于曲率是由联络来定义的,也可以说是引力改变了时空的曲率.

以上只是定性说法,那么我们是否真的可以用这种方法来描述引力作用呢?由于牛顿的引力论在低速、弱场且稳定的情况下已经被实验反复证实,我们可以从此下手,在低速弱场稳定近似下尝试解一个质点的测地线方程,看能不能回归到牛顿的引力方程上.

\subsubsection{近似假设}

\textbf{低速}近似意味着,在所讨论的参考系里,质点的四速度非常接近$\pmat{1&0&0&0}$,也就是说,其轨迹的四个坐标$x^i$满足下式:
\begin{equation}
\frac{\dd}{\dd \tau}x^i\ll \frac{\dd}{\dd tau}t\approx 1
\end{equation}

\textbf{弱场}近似意味着,引力的作用非常微弱,也就是说对度量的影响很小.记$\eta_{\mu\nu}$为该参考系中的Minkowski度规,那么引力作用下的度规就是$g_{\mu\nu}=\eta_{\mu\nu}+h_{\mu\nu}$,其中$\abs{h_{\mu\nu}}\ll 1$.考虑到度规的指标升降法则要求$g^{ai}g_{bi}=\delta^a_b$,结合$\abs{h_{\mu\nu}}\ll 1$,可以计算出$g^{\mu\nu}=\eta^{\mu\nu}-h^{\mu\nu}$\footnote{你可以尝试验证:$(\eta_{ai}+h_{ai})(\eta^{bi}-h^{bi})=\eta_{ai}\eta^{bi}+h_{ai}\eta^{bi}-h^{bi}\eta_{ai}-h_{ai}h^{bi}$,其中右边的中间两项由度量的对称性以及$h^{ab}=h_{ij}\eta^{ia}\eta^{jb}$,可以抵消掉,于是只剩下$\delta^b_a-h_{ai}h^{bi}$项,后者是更高阶的小量,弱场近似下就被忽略了.}.

\textbf{稳定}近似意味着,引力场不随时间变化,因此$\frac{\dd}{\dd t}g_{\mu\nu}=0$.






















