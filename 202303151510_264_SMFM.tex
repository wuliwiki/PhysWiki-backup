% 索末菲模型
% 固体物理|自由电子气模型

\begin{issues}
\issueDraft
\end{issues}

虽然德鲁特模型在电子的声、光和热导率方面解释得比较好,但其在解释电子比热时却遇到了困难。

根据经典的能均分定理,金属电子气中每个电子的平均内能为$3k_BT/2$,对热容的贡献为$3k_B/2$,然而实验中的电子热容几乎测不到,只有德鲁特模型给出的1\%左右,而且与温度密切相关。
\begin{figure}[ht]
\centering
\includegraphics[width=6cm]{./figures/SMFM_1.png}
\caption{常见金属热容随温度的变化} \label{SMFM_fig1}
\end{figure}
解释这个现象需要用到索菲亚模型。
\subsection{基本假设}
\begin{enumerate}
\item \textbf{独立电子近似(Independent electron approximation)}:电子之间不会相遇,不存在任何相互作用。
\item \textbf{自由电子近似}:相对电子而言,晶体中的离子的运动忽略不计,同时也忽略电子与离子的库伦力作用。
\item \textbf{不碰撞假设}:电子以物质波的形式存在,并且不会与离子碰撞。
\item \textbf{泡利不相容原理}:电子是费米子,满足泡利不相容原理。
\end{enumerate}
\subsection{k动量空间}
根据上面的假设,我们可以写出电子满足的薛定谔方程:
\begin{equation}
-\frac{\hbar^2}{2m}\nabla^2\psi=E\psi
\end{equation}
方程的解是平面波$\psi=Ae^{i\vec{k}\cdot \vec{r}}$。

为了使平面波函数归一,我们将电子限定在一个$L^3$中的正方形盒子中,考虑到固体中含有大量的电子,含有大量重复的这种小盒子,所以我们认为盒子中的波函数满足周期性边界条件,即:
\begin{equation}
\psi(\vec{r})=\psi(\vec{r}+Le_x)=\psi(\vec{r}+Le_y)=\psi(\vec{r}+Le_z)
\end{equation}
所以最终的波函数为$\psi=L^{-3/2}e^{i\vec{k}\cdot \vec{r}}$,其中$\vec{k}$满足:
\begin{equation}
\left (k_x, k_y, k_z\right )=\frac{2\pi}{L}\left(n_x, n_y, n_z\right )
\end{equation}
能量$E$满足:
\begin{equation}
E=\frac{\hbar ^2 k^2}{2m}
\end{equation}
可以这么理解,每一个电子的波函数中的$\vec{k}$是分立的,对应下列图片中的一个小格子。当格子足够密集时,所有能量相等的电子则可以近似对应一个$\vec{k}$空间中的球面。
\begin{figure}[ht]
\centering
\includegraphics[width=6cm]{./figures/SMFM_2.png}
\caption{k动量空间} \label{SMFM_fig2}
\end{figure}
\subsection{费米分布和费米能}
由于电子是费米子,自由电子不能处在同一个状态内,即上述的k动量空间中一个格子内不能有超过2个电子("2"是来自于电子自旋),热力学统计给出一定温度下电子的分布有:
\begin{equation}
f(E)=\left (\exp(\frac{E-\mu}{k_B\ T})+1\right )^{-1}    \in[0,1]
\end{equation}
表征某个能量值下电子的存在几率。由于能量$E$是$\vec{k}$的函数,那么其自然也表明是否有电子处在某个特定的$\vec{k}$上。那么总的电子数就是$N=\textstyle \sum_{\vec{k}}2\ f(\vec{k})$。总的能量就是$E_{sum}=\textstyle \sum_{\vec{k}}2\ E(\vec{k})f(\vec{k})$。

其中$\mu$是化学势,其与电子总数有关,随温度缓慢变化。$\mu (T=0K)$称为费米能$E_F$.

T=0K时的情况非常特殊。我们先来考虑T=0K时的情况来理解函数$f(E)$。T=0K时的函数是一个阶跃函数:
\begin{cases}
f(E)=1, &\text{if } E<\mu \newline
f(E)=0, &\text{if } E>\mu
\end{cases}
这表明电子会占满能量小于$\mu$的所有格子。实验测得$\mu(T=0K)$,即$E_F$,大致在2~10eV之间,而室温300K对应的能量$k_BT\approx2.59*10^{-2}\ eV$,远远小于$E_F$的尺度,而前面提到$\mu$是随温度缓慢变化的,则可以画出室温300K时的电子分布(假设$E_F=3\ eV$):
\begin{figure}[ht]
\centering
\includegraphics[width=12cm]{./figures/SMFM_3.png}
\caption{电子随能量的分布} \label{SMFM_fig3}
\end{figure}
可以发现,图像中只是$E_F$附近的电子分布出现了变化,则可以推断,电子的热力学性质大致都是$E_F$附近的电子分布变化导致的。

\subsection{求和与积分的转换、态密度}
前面提到,电子总数和总能量有:
\begin{equation}
E_{sum}=\textstyle \sum_{\vec{k}}2\ E(\vec{k})f(\vec{k})
\end{equation}

\begin{equation}
N=\textstyle \sum_{\vec{k}}2\ f(\vec{k})
\end{equation}

为了能计算出电子总数与总能量的表达式,我们需要把求和形式变成积分形式。做一个简单的移动:
\begin{equation}
\iiint g(\vec{k})\,dk_x\,dk_y\,dk_z=dk_x\,dk_y\,dk_z\,\sum g(\vec{k})
\end{equation}
从图2就可以看出,有$dk_x\,dk_y\,dk_z=(\frac{2\pi}{L})^3$。
所以当L的尺度足够大,即$(\frac{2\pi}{L})^3$的尺度足够小时,我们就能有:
\begin{equation}
E_{sum}=\textstyle \sum_{\vec{k}}2\ E(\vec{k})f(\vec{k})=(\frac{L}{2\pi})^3\,2 \int_{\vec{k}}E(\vec{k})f(\vec{k})d^3k
\end{equation}

\begin{equation}
N=\textstyle \sum_{\vec{k}}2\ f(\vec{k})=(\frac{L}{2\pi})^3\,2 \int_{\vec{k}}f(\vec{k})d^3k
\end{equation}
由于$E(\vec{k})$和$f(\vec{k})$与k的方向是无关的,所以上述的积分还可以再次简化成:
\begin{equation}
E_{sum}=(\frac{L}{2\pi})^3\,2\int d\Omega \int E(k)f(k)k^2dk=\frac{L^3}{\pi^2} \int E(k)f(k)k^2dk
\end{equation}
\begin{equation}
N=(\frac{L}{2\pi})^3\,2\int d\Omega \int f(k)k^2dk=\frac{L^3}{\pi^2}\int f(k)k^2dk
\end{equation}
借助$E=\frac{\hbar^2}{2m}k^2$,可以继续化简式子变成:
\begin{equation}
E_{sum}=L^3 \int E\,f(E)\,\frac{\sqrt{2m^3}}{\pi^2\hbar^3}\sqrt{E}dE
\end{equation}
\begin{equation}
N=L^3 \int f(E)\,\frac{\sqrt{2m^3}}{\pi^2\hbar^3}\sqrt{E}dE
\end{equation}
这样就能直接在E空间上积分了。此外,可以定义态密度:
\begin{equation}
g(E)=\frac{\sqrt{2m^3}}{\pi^2\hbar^3}\sqrt{E}
\end{equation}
其表征dE范围内的量子态数目(即k空间中格子的数目X2)。

\subsection{电子热性质}
电子数密度$N/L^3$和单位体积电子气总能量$E_{sum}/L^3$都有一个共同形式:
\begin{equation}
\int_0^{\infty} dE\,f(E)h(E)
\end{equation}
设$h(E)$的原函数为$Q(E)$,即$Q(E)=\int_0^E d\epsilon\,h(\epsilon)$,则$h(E)=Q^\prime(E)$,则有:
\begin{equation}
\int_0^{\infty} dE\,f(E)h(E)=\int_0^{\infty} f(E)dQ(E)= f\,Q\bigg|_0^\infty-\int_0^{\infty} Q(E)f^\prime (E)dE
\end{equation}
其中因为$Q(0)=0$和$f(\infty)=0$($e^{-\infty}$级的无穷小比$1/Q(\infty)$更高阶),所以$f\,Q\bigg|_0^\infty=0-0=0$,所以有:
\begin{equation}
\int_0^{\infty} dE\,f(E)h(E)=-\int_0^{\infty} dE\,Q(E)f^\prime (E)
\end{equation}
我们记得$f(E)$为:
\begin{equation}
f(E)=\left (\exp(\frac{E-\mu}{k_B\ T})+1\right )^{-1}
\end{equation}
令$x=\frac{E-\mu}{k_B\,T}$,则$-f^\prime(E)$为:
\begin{equation}
-f^\prime(E)=\frac{(k_BT)^{-1}}{(e^{x/2}+e^{-x/2})^2}
\end{equation}
画出$-f^\prime(E)$图像,可以发现其只在$E=\mu$附近外非常接近0: