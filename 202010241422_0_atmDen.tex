% 大气密度
% keys 大气|积分方程|理想气体

假设大气是理想气体, 密度随高度变化为 $\rho(z)$. 所以高度 $z$ 处压强为
\begin{equation}\label{atmDen_eq1}
P(z) = \int_{z}^\infty \rho(z') g \dd{z'}
\end{equation}
而根据理想气体状态方程\upref{PVnRT},
\begin{equation}
PV = n R T
\end{equation}
先假设大气只是由一种分子构成, 摩尔质量为 $\mu$, 即 $m = n\mu$, 代入有
\begin{equation}
P = \frac{m}{\mu V} RT = \frac{R}{\mu} \rho T
\end{equation}
其中 $P, T, \rho$ 都是高度的函数. 代入\autoref{atmDen_eq1} 得关于 $\rho(z)$ 的积分方程
\begin{equation}
\frac{R}{\mu} \rho(z) T(z) = \int_{z}^\infty \rho(z') g \dd{z'}
\end{equation}
通常来说海拔越高的地方气温越低, 如果 $T(z)$ 是已知的, 就可以解出 $\rho(z)$. 方程两边对 $z$ 求导得
\begin{equation}
\rho'(z) T(z) + \rho(z) T'(z) - \frac{\mu}{R}\rho(z) g = 0
\end{equation}
但为了简单起见, 先假设温度 $T$ 是个常数.
