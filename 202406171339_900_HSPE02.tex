% 静电场的应用(高中)
% keys 电容|静电|带电粒子的运动
% license Xiao
% type Tutor

\begin{issues}
\issueDraft
\issueTODO
\end{issues}

\pentry{静电场\nref{nod_HSPE01}}{nod_8fdb}

\subsection{电容器}

\textbf{电容器}是用来储存电荷的电学元件。电容器由两个相互靠近、彼此绝缘的导体组成,两导体间可填充绝缘物质——\textbf{电介质}(空气也是一种电介质)。最简单的电容器由两块相距很近的平行金属板(极板)组成,叫做平行板电容器。实际上,任何彼此绝缘又相隔很近的导体,都可以看成一个电容器。

电路中,电容器用字母$C$表示,符号如\autoref{fig_HSPE02_1} .

\begin{figure}[ht]
\centering
\includegraphics[width=5cm]{./figures/909ba6d1f177a961.png}
\caption{固定电容器(左)和可变电容器(右)} \label{fig_HSPE02_1}
\end{figure}

\subsubsection{电容器的充电和放电}

使得两个极板分别带上等量异种电荷的过程,就是电容器的\textbf{充电}过程。充电过程中,流入正极板的电流逐渐减小,电容器所带的电荷量逐渐增加,电容器两极板之间的电场强度逐渐增强,板间电压也逐渐升高,电容器从电源获得的电能转化为电容器中的电场能。充电结束后,电容器两极板间电压与充电电压相等。

使电容器两极板的异种电荷中和的过程,就是电容器的\textbf{放电}过程。放电过程中,电流从正极板流出且逐渐减小,电容器所带的电荷量逐渐减少,电容器两极板间的电场强度逐渐减弱,板间电压也逐渐降低,电容器中的电场能转化为其他形式的能量。

\begin{figure}[ht]
\centering
\includegraphics[width=13cm]{./figures/0051a03af27c4d6a.png}
\caption{电容器的充电(左)和放电(右)} \label{fig_HSPE02_2}
\end{figure}

\subsubsection{电容}

电容器所带电量Q与电容器两极板间电压之比,叫做电容器的\textbf{电容},用$C$表示,即
\begin{equation}
C=\frac{Q}{U}~.
\end{equation}

在国际单位中,电容的单位是\textbf{法拉},简称\textbf{法},符号为$\mathrm{F}$。

常用的电容单位有\textbf{微法}($\mathrm{\mu F}$)和\textbf{皮法}($\mathrm{pF}$)。$1\mathrm{F}=10^6 \mathrm{\mu F}=10^{12} \mathrm{pF}$。

电容是表示电容器容纳电荷本领的物理量。

\subsubsection{电容的串联和并联}
$n$个电容器并联时,有\footnote{详细证明见“\enref{电容的串联和并联}{Ccomb}”}
\begin{equation}
C=\sum_{i=1}^{n}C_i~.
\end{equation}

$n$个电容器串联时,有
\begin{equation}
\frac{1}{C}=\sum_{i=1}^{n}\frac{1}{C_i}~.
\end{equation}

\subsubsection{平行板电容器}

平行板电容器的电容为
\begin{equation}\label{eq_HSPE02_1}
C = \frac {\epsilon_r S}{4\pi kd}~.
\end{equation}

\autoref{eq_HSPE02_1} 中的$S$和$d$分别为电容器两极板的正对面积和板间距离;$k$为静电力常量(\autoref{eq_HSPE01_8});$\epsilon_r$为相对介电常数,真空的$\epsilon_r$值为$1$。

\subsection{带电粒子在电场中的运动}

分析带电粒子在电场中的运动时,如电子、质子、离子、$\alpha$粒子等,一般都会忽略其重力;而对于带电的小球、粉尘、油滴(液滴)等颗粒,一般都不能忽略其重力。此外,若明确带电粒子所受的静电力远大于重力时,忽略其重力对运动的影响。

\subsubsection{带电粒子在电场中的直线运动}

\begin{example}{}
如\autoref{fig_HSPE02_3} 所示,两平行金属板间电压恒为$U$,板间距为$d$。一带电量为$q$、质量为$m$的带正电粒子,在仅受静电力的作用下,由静止开始从正极板向负极板运动,求粒子到达负极板时的速度大小$v$。

\begin{figure}[ht]
\centering
\includegraphics[width=7.5cm]{./figures/26a06347195d586c.png}
\caption{带电粒子在电场中的直线运动} \label{fig_HSPE02_3}
\end{figure}

\begin{enumerate}
\item 从匀加速直线运动来看,粒子运动时的加速度大小为
\begin{equation}
a = \frac{F}{m} = \frac{Eq}{m} = \frac{qU}{md}~.
\end{equation}
根据速度与位移的关系可列式
\begin{equation}
v^2 - 0 = 2ad~.
\end{equation}

\item 从静电力做功来看,根据动能定理$W=E_{k2}-E_{k1}$可列式
\begin{equation}
W=qEd=qU=\frac12 mv^2 - 0~.
\end{equation}
\end{enumerate}

由上述两种方法,可解得粒子到达负极板时的速度大小为
\begin{equation}\label{eq_HSPE02_2}
v = \sqrt{\frac{2qU}{m}}~.
\end{equation}

\end{example}

当带电粒子沿着一条平直电场线的方向进入电场,在仅受静电力的作用时,粒子的运动为直线运动,若电场为匀强电场,粒子做匀变速直线运动;若电场为非匀强电场,粒子做变加速直线运动。

\subsubsection{带电粒子在匀强电场中的偏转}

两平行金属板间电压恒为$U$,板间距为$d$,板长为$l$。一带电量为$q$、质量为$m$的带电粒子(重力忽略不计),以初速度$v_0$沿垂直于电场线方向飞入板间的匀强电场。

\begin{figure}[ht]
\centering
\includegraphics[width=10cm]{./figures/ad4288037f927ae0.png}
\caption{带电粒子在匀强电场中的偏转} \label{fig_HSPE02_4}
\end{figure}


带电粒子进入匀强电场后,仅受到垂直于初速度方向恒定的静电力,带电粒子在匀强电场的运动类似于平抛运动。

\begin{itemize}
\item 由于重力忽略不计,带电粒子在电场中运动的加速度大小为
\begin{equation}
a = \frac{F}{m} = \frac{Eq}{m} = \frac{qU}{md}~.
\end{equation}

\item 若带电粒子能飞出电场,则带电粒子在电场中的运动时间为
\begin{equation}
t = \frac{l}{v_0}~.
\end{equation}

带电粒子在静电力方向的位移大小为
\begin{equation}
y = \frac12 at^2 = \frac{qUl^2}{2mv_0^2 d}~.
\end{equation}

\item 若带电粒子不能飞出电场,则带电粒子将打在极板上,设带电粒子进入电场时与该极板的距离为$d_0$,则
\begin{equation}
y = d_0 = \frac12 at^2 = \frac{qUt^2}{2md}~,
\end{equation}
\begin{equation}
t = \sqrt{\frac{2mdd_0}{qU}}~.
\end{equation}

带电粒子在初速度方向的位移大小为
\begin{equation}
x = v_0\sqrt{\frac{2mdd_0}{qU}}~.
\end{equation}

\item 带电粒子在静电力方向的分速度大小为
\begin{equation}
v_y = at = \frac{qUl}{mv_0d}~.
\end{equation}
对于速度偏转角$\theta$,有
\begin{equation}
\tan \theta = \frac{v_y}{v_x} = \frac{qUl}{mv_0^2 d}~.
\end{equation}

\item 若带电粒子由静止经一个加速电场(电压为$U_0$)加速后,再进入偏转电场,由\autoref{eq_HSPE02_2} 可得带电粒子进入偏转电场的初速度大小为
\begin{equation}
v_0 = \sqrt{\frac{2qU_0}{m}}~.
\end{equation}

相应地,$y$和$\tan\theta$可表示为
\begin{equation}
y = \frac{Ul^2}{4U_0d}~,
\end{equation}

\begin{equation}
\tan\theta = \frac{Ul}{2U_0d}~.
\end{equation}

$y$和$\tan\theta$与带电粒子的比荷无关,由此可见,若干个电性相同的带电粒子由静止经过一个加速电场加速后,再进入同一个偏转电场,它们的运动轨迹是相同的。
\end{itemize}

