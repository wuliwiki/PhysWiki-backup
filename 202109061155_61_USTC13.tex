% 中科大2013年考研普通物理
% 中科大|考研|普通物理|普物

\subsection{ 简答题(每题10分,共30分)}
\begin{enumerate}
\item 将一颗人造卫星发射到半径为$R$的圆形轨道上,将第二颗卫星发射到半径为1.01$R$的轨道上.问第二个卫星的周期比第一个卫星的周期是长些还是短些,或者相同?周期如果不同,那么相差百分之几?
\item 一强度为$p$的电偶极子,在均匀电场$E$中受力多大?
\item 两个点电荷的距离为$d$,分别带电量$Q$和$-Q$,把它们分开到无穷远处,需要多大的能量.
\end{enumerate}
\subsection{计算题(每题20分,共120分)}
\begin{enumerate}
\item 一质点以初速率$v$做直线运动,所受阻力与其速率成正比.试求当质点速率减为$\frac{v_0}{n}(n>1)$时,质点经过的距离.
\item 如图1,细杆绕端点$O$在平面内匀速旋转,角速度为$a$,杆上一小环(可看作质点)相对杆做匀速运动,相对速度为$v$.设$t=0$时刻小环位于杆的端点$O$.试求小环在任意时刻的速度和加速度.
\item 如图2,在相距为$l$的两平行弹性墙壁之间,有质量为$m$的弹性小球以垂直于墙壁的初速度$v$往返弹跳.设墙壁的质量远大于$m$,碰撞时完全弹性的重力和空气阻力可以忽略.试求:(1)每墙壁所受的平均作用力;(2)若用外力使左墙壁以速度$V(V<<v_0)$缓缓右移,证明外力作功等于小球动能的增加.
\item 将两个电容器$(C1,C2)$分别充电达相同的电势差$(V)$,随后将一个电容器的正极与另一电容器的负极相联,然后将其他两极短路.(1)计算每个电容器的最终电荷.(2)计算电场能的损失.
\item 求产生球对称电势$V(r)=\frac{e^{-\lambda r}}{r}$的电荷分布.
\item 理想气体经历$V=hn$的热力学过程,其中$p_0$和$K$是常数.试问:(1)当系统按此体积扩大一倍时,系统对外做了多少功?(2)在这一过程中的热容是多少?(理想气体的物态方程为:$pV=vRT$)
\end{enumerate}
