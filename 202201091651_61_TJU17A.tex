% 天津大学 2017 年考研量子力学答案
% keys 考研|天津大学|量子力学|2017|答案

\begin{issues}
\issueDraft
\end{issues}


\subsection{ }
\begin{enumerate}
\item 
(1)由题可知势能$\overline{U}=-\frac{e^2}{r}$.
\begin{equation}
\begin{aligned}
\overline{U}=&\iiint \psi^{*} \overline{U} \psi \dd{\tau}\\
=&-\frac{e^2}{\pi a^{3}_{0}}\int^{\pi}_{0}\int^{2\pi}_{0}\int^{\infty}_{0} \frac{1}{r}e^{-\frac{2r}{a_0}}r^{2}\sin{\theta} \dd{r}\dd{\theta}\dd{\varphi}\\
=&-\frac{e^{2}}{\pi a^{3}_{0}}\int^{\pi}_{0}\int^{2\pi}_{0}\int^{\infty}_{0} e^{-\frac{2r}{a_0}}r\sin{\theta} \dd{r}\dd{\theta}\dd{\varphi}\\
=&-\frac{4 e^2}{a^{3}_{0}}\int^{\infty}_{0}e^{-\frac{2r}{a_0}}r\dd{r}\\
=&-\frac{4e^{2}}{a^{3}_{0}}(\frac{a_{0}}{2})^2\\
=&-\frac{e^{2}}{a_{0}}
\end{aligned}
\end{equation}
(2)电子$r+dr$在球壳内出现的几率为:\\
\begin{equation}
\begin{aligned}
w(r)\dd{r}=&\int^{\pi}_{0}\int^{2\pi}_{0} \lvert \psi(r,\theta,\varphi) \rvert \sin{\theta}\dd{\theta}\dd{\varphi}\dd{r}\\
=&\frac{4}{a^{3}}e^{-\frac{2r}{a_0}}r^2 \dd{r}\\
\end{aligned}
\end{equation}
% $w(r)=\frac{4}{a^{3}}e^{-\frac{2r}{a_0}}r^2 $
\begin{equation}
\begin{aligned}
\frac{\dd{w(r)}}{\dd{r}}=\frac{4}{a^{3}}e^{-\frac{2r}{a_0}}r^2 
\end{aligned}
\end{equation}
令$\frac{\dd{w(r)}}{\dd{r}}=0,\Longrightarrow r_1 = 0,r_2 = \infty,r_3 = a_0$,因为$\frac{dd^{2}{w(r)}}{\dd{r^{2}}}|_{r = a_{}} < 0$,所以$r = a_0$为最概然半径.

\item
答:康普顿散射是光子与电子做弹性碰撞,在$X$射线通过实物物质发生散射的实验时,除原波长的光外还产生了大于原波长的$X$光,借助光电理论,才可以得到这是由于光子与电子发生碰撞后频率变小的缘故,从而证实了光具有粒子性.

\item 
(1)三维转子的能级为:$E = \frac{l(l+1)}{2I}$,简并度为:$2l+1$.\\
(2)平面转子设沿$z$轴方向,$\hat{H} = \frac{\hat{l}^{2}_{z}}{2I}$,能级为:$E = \frac{m^{2} \hbar^{2}}{2I} ,m = 0 , \pm 1 , \pm 2$,除了$m = 0$外,能级都是二重简并.
\end{enumerate}

\subsection{ }
\begin{enumerate}
\item 
\begin{equation}
(\vec{\gamma} \times \vec{L} + \vec{L} \times \vec{\gamma})_{x} = yL_{z}-zL_{y}
\end{equation}

\end{enumerate}