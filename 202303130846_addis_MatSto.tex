% 数据结构:密矩阵
% 行主序|列主序

\pentry{矩阵\upref{Mat}}

一般来说, 矩阵的每个元在计算机内存中逐个储存, 这种数据结构通常叫做\textbf{密矩阵(dense matrix)}。 由于在计算机内存中所有数据都是按顺序排成一行, 所以在储存矩阵时我们就有两种选择, 一是把矩阵所有行首尾相接, 叫做\textbf{行主序(row-major)}, 二是把矩阵所有列首尾相接, 叫\textbf{列主序(column-major)}。 例如对于 $2 \times 2$ 的矩阵 $\mat A$, 列主序下, 矩阵元在内存中的顺序依次 $A_{11}, A_{21}, A_{12}, A_{22}$, 而在行主序下顺序为 $A_{11},A_{12},A_{21},A_{22}$。

在 Fortran 和 Matlab 中\footnote{Matlab 最初就是由 fortran 编写的。}, 语言自带的矩阵都是列主序, 而在 C/C++ 中如果用指针的指针(或数组的数组)来表示矩阵, 得到的将会是行主序。 当然, 由于 C++ 的灵活性, 我们完全可以创造行主序和列主序两种不同的矩阵类。 由于 C++ 的数组的指标从 0 开始, 那么矩阵的行标和列表也习惯从 0 开始。

在我们用双索引寻找矩阵元时, 我们需要先将其转换为单索引。 假设矩阵尺寸为 $N_1 \times N_2$, 那么
\begin{equation}
\begin{aligned}
&\begin{cases}
n = i + N_1 j  &\text{(列主序)}\\
n = N_2 i + j  &\text{(行主序)}
\end{cases}\\
&(i = 0, \dots, N_1-1,\quad j = 0, \dots, N_2-1)
\end{aligned}
\end{equation}

行主序和列主序也可以延申至高维矩阵, 如果使用列主序, 那么当我们在内存中按顺序读取数据的时候, 第 1 个索引(index)将变化得最快, 第 2 个索引变化得第二快, 最后得索引变化得最慢。 行主序则相反, 最后的索引变化得最快, 而第一个最慢。 例如 4 维数组的多索引变为单索引的公式为
\begin{equation}
\begin{cases}
n = i_1 + N_1 i_2 + N_1 N_2 i_3 + N_1 N_2 N_3 i_4  &\text{(列主序)}\\
n = N_2 N_3 N_4 i_1 + N_3 N_4 i_2 + N_4 i_3 + i_4  &\text{(行主序)}
\end{cases}
\end{equation}
从性能角度来看, 单索引要比多个索引要快。

若要由单索引计算多索引, 我们可以用整数除法(向下取整) \verb|/| 和求余运算 \verb|%|, 例如对列主序的矩阵有
\begin{equation}
\begin{cases}
i = n \% N\\
j = n / N
\end{cases}
\end{equation}

本书的 SLISC 库中提供了 C++ 的密矩阵类, 详见 “SLISC 的密矩阵类\upref{SliMat}”。
