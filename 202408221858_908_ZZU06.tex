% 郑州大学 2006 年 考研 量子力学
% license Usr
% type Note

\textbf{声明}:“该内容来源于网络公开资料,不保证真实性,如有侵权请联系管理员”

\subsection{30分}
两个自旋为 $\hbar/2$ 的非全同粒子构成一个复合体系,设两个粒子间的相互作用为 $c\vec{S}_1 \cdot \vec{S}_2$,其中 $c$ 为实常数。设 $t=0$ 时粒子 1 的自旋沿 $z$ 轴的正方向,粒子 2 的自旋沿 $z$ 轴的负方向(即处在 $\alpha(1)\beta(2)$ 态上),要求:

\begin{enumerate}
    \item (1) 给出 $H$ 的本征值,并给出 $t > 0$ 时体系处的状态 $\psi(t)$;
    \item (2) 给出 $t > 0$ 时,测量粒子 1 的自旋仍处在 $z$ 轴正方向的几率。
\end{enumerate}
\subsection{20分}
设线性谐振子的哈密顿量用升算符 $a^\dagger$ 与降算符 $a$ 表示为 $H_0 = \left(a^\dagger a + \frac{1}{2}\right)\hbar\omega$,此体系受到微扰 $W = \lambda(a^\dagger + a)\hbar\omega$ 的作用。求体系的能级到二级近似(已知 $a^\dagger$ 与 $a$ 对 $H_0$ 的本征态 $|n\rangle$ 的作用为 $a^\dagger|n\rangle = \sqrt{n+1}|n+1\rangle$,$a|n\rangle = \sqrt{n}|n-1\rangle$)。
\subsection{20分}
\textbf{1.}两个全同费米子的弹性散射,在原心系中入射波(空间波函数)表示为 $e^{ikz} + e^{-ikz}$,设相互作用为一个中心势 $V(r)$,给出体系的空间波函数在 $r \to \infty$ 处的渐近行为,并写出散射截面的表达式。

\textbf{2.}用玻恩近似计算粒子对 $\delta$ 势 $V(r) = -V_0 e^{-r/a}$ 的微分散射截面 $\sigma(\theta)$。已知
$$\int_{0}^{\infty} x e^{-ax} \sin(bx) dx = \frac{2ab}{(a^2 + b^2)^2}~$$
\subsection{20分}
已知 $\sigma$ 为 Pauli 算符,在 $ \sigma_z$ 表象中给出 $\sigma_x$, $\sigma_y$, $\sigma_z$ 的矩阵表示式,并求出它们的本征向量及本征值。
