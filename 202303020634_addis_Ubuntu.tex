% Ubuntu 笔记

\subsection{快捷键}
\begin{itemize}
\item 鼠标中键可以粘贴刚才选中的内容
\end{itemize}


\subsection{改主机名}
\begin{itemize}
\item \verb|hostname 新主机名| 可以临时改变 hostname
\item \verb|hostnamectl set-hostname 新主机名| 可以永久改变 hostname
\end{itemize}


\subsection{改键位}
注意 Ubuntu (20.04 亲测, 参考\href{https://manpages.ubuntu.com/manpages/focal/en/man5/keyboard.5.html}{这里})设置 xkb config 文件无效的! 需要在 \verb|/etc/default/keyboard| 里面设置(同样重新 login 以后生效)。 例如添加一行 \verb|XKBOPTIONS=ctrl:swapcaps| 对调 Ctrl 和 Capslock 按键。

\subsection{系统版本升级}
\begin{itemize}
\item 参考\href{https://www.cyberciti.biz/faq/upgrade-ubuntu-18-04-to-20-04-lts-using-command-line/#Make_a_backup}{这里}。
\item \verb|sudo apt update && sudo apt upgrade|
\item \verb|sudo reboot|
\item \verb|sudo apt install update-manager-core|
\item \verb|sudo do-release-upgrade|
\item \verb|sudo reboot|
\end{itemize}

\subsection{取消 Alt+F5 快捷键}
\begin{itemize}
\item \verb|gsettings set org.gnome.desktop.wm.keybindings unmaximize "['<Super>Down']"|
\item 参考\href{https://askubuntu.com/questions/1322105/cant-find-alt-f5-in-settings-keyboard-shortcuts}{这里}。
\end{itemize}

\subsection{自定义屏幕分辨率}
\begin{itemize}
\item 在 22.04 下, 例如要定义 1368x912 的分辨率, 先用 \verb|cvt 1368 912|, 输出第二行为 \verb|Modeline "1368x912_60.00"  103.00  1368 1448 1592 1816  912 915 925 947 -hsync +vsync|
\item 执行 \verb|xrandr --newmode "1368x912"  103.00  1368 1448 1592 1816  912 915 925 947 -hsync +vsync|, 后面的数据是复制过来的。 双引号中的名字可以随意。
\item 执行 \verb|xrandr|, 找到 connected 的显示器名称, 如 \verb|eDP-1 connected primary 1368x912+0+0 ...|
\item 执行 \verb|xrandr --addmode eDP-1 "1368x912"| 即可添加分辨率。
\item 现在打开 setting 中的 display 就可以看到这个分辨率的选项了。
\end{itemize}


\subsection{笔记本合上盖子不作为}
\begin{itemize}
\item \verb|sudo vim /etc/systemd/logind.conf|, 找到所有包含 \verb|LidSwitch| 的选项, 取消注释, 值改成 \verb|ignore|
\item 然后运行 \verb|systemctl restart systemd-logind.service| 生效。 注意系统会重新 login
\end{itemize}

\subsection{自定义桌面、下载等目录}
\begin{itemize}
\item 修改设置文件 \verb|~/.config/user-dirs.dirs| 即可
\end{itemize}


\subsection{recycle}
\begin{itemize}
\item 用 \verb`apt-clone` 是最方便快速的(未测试), 或者释放出 deb 文件单独安装
\item 默认文本编辑器从 gedit 替换为 vscode: 修改 \verb`/usr/bin/gnome-text-editor` 软链指向 \verb`code` 即可
\item CLion 的工具栏 Tools 可以直接创建快捷方式
\item 搜狗拼音还是老老实实按照笔记来弄
\item Matlab 和 Mathematica 和 Intel oneAPI 都可以直接复制目录(亲测有效), 不过前两个需要再次注册。
\item GitHub Desktop 可以在这里安装 https://github.com/shiftkey/desktop/
\item \verb`~/.config` 文件夹(很多常用软件的设置文件都在这,可以选择性拷贝)和 \verb`~/.vim` 文件夹, 以及 \verb`~/.vimrc` 都可以拷贝
\end{itemize}
