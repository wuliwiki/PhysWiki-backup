% 电子轨道与元素周期表
% 原子|电子轨道|元素周期表|量子数|周期

\begin{issues}
\issueTODO
\end{issues}

在这里, 用最简单的方式介绍原子的壳层结构,并解释元素周期表如何根据壳层结构分出每个周期. 在玻尔原子模型\upref{BohrMd} 中, 原子轨道如\autoref{Ptable_fig1}.

\begin{figure}[ht]
\centering
\includegraphics[width=4.5cm]{./figures/Ptable_1.pdf}
\caption{电子轨道}\label{Ptable_fig1}
\end{figure}
%现在要把电子放到这些轨道上面来,使电子的总能量最小. 这种状态叫做原子的基态. 理想的状态是,所有电子都在最小的一圈轨道上,但是由于每条轨道只能容纳一定数目的电子,另一些电子不得不进入其他轨道.

\subsection{电子轨道}
%为了解释每条轨道能容纳多少电子,把每个轨道的 “空位” 用一行格子描述. 当一行格子被电子填满时,该轨道就不能容纳更多电子了.
\begin{figure}[ht]
\centering
\includegraphics[width=4cm]{./figures/Ptable_2.pdf}
\caption{用格子描述电子轨道} 
\end{figure}

\begin{table}[ht]
\centering
\caption{电子轨道}\label{Ptable_tab1}
\begin{tabular}{|c|c|c|}
\hline
名称&符号&取值范围\\
\hline
电子层&主量子数 n& 1(K), 2(L), 3(M), 4(N).. \\
\hline
电子亚层&角量子数 l& 0(S),1(P),2(D),3(F)..., n-1 \\
\hline
亚层的轨道&磁量子数 m & -l, -l+1,...,0,1,2,...,l-1,l \\
\hline
(电子自旋,不是轨道参数)&ms&-1/2, 1/2\\
\hline
\end{tabular}
\end{table}

电子层:如\autoref{Ptable_fig1} 所示,从半径最小的轨道开始,用数字 1,2,3 (或K,L,M,...)等依次命名每条轨道,这些数字也叫主量子数,用 $n$ 来表示. 

电子亚层:一个电子层可以分为若干个电子亚层,以角量子数$l$计.对于一个主量子数为$n$的电子层,角量子数可以取$0,1,2,...,n$(或称$S,P,D,F...$ )等$n$ 个不同的值,即一个主量子数为$n$的电子层可以分为$n$个电子亚层.把行标和列标组合起来, 就能得到任意一个格子的名称,例如第三行第二列的格子叫做 $3P$. 

亚层的轨道:一个电子亚层可以继续分为若干个轨道,以磁量子数$m$计.对于一个角量子数为$l$的电子亚层,$m$ 可以取 $ - l, - l + 1...0,...l - 1,l$ 等 $(2l + 1)$ 个不同的值.因此,一个角量子数为$l$的电子亚层可以分为$(2l+1)$个轨道.

电子自旋:由于不相容原理,一个亚层的轨道中最多只能容纳两个(自旋相反)的电子.

因此,每一个电子亚层可以最多容纳$2(2l+1)$个电子,每一个电子层最多容纳$2n^2$ 个电子(见\autoref{Ptable_fig2}).

\subsection{电子的排序}
到此为止, 每条轨道承载电子的数目已经解释清楚了, 但是应该如何把电子往格子里面放呢? 为了使电子总能量最小,对于氢原子(1 个核外电子), 显然电子应该放在 $1S$ 格子里, 氦原子(2 个核外电子)可以把两个电子都放在 $1S$ 格子里, 从而把 $n=1$ 的轨道填满, 这就是第一周期的两个原子的电子分布. 对于锂原子(3 个核外电子)可以在氦原子的基础上往 $2S$ 格子里放一个电子. 但奇怪的是, 填电子的顺序并不是从下到上从左到右, 而是如下图中的箭头所示, 即按照 $1S, 2S, 2P, 3S, 3P, 4S, 3D, 4P, 5S\dots$ 的顺序来填上图的格子. 格子内的数字表示每格能装下的电子数, 即 $2(2l + 1)$.
\begin{figure}\label{Ptable_fig2}[ht]
\centering
\includegraphics[width=4.5cm]{./figures/Ptable_3.pdf}
\caption{轨道的填充顺序} 
\end{figure}
\addTODO{这张图在第四周期以上不成立,见 \cite{GriffQ}}

\subsection{元素周期表的排序}

要判断某个原子所在的周期, 就先根据原子序号找出上图中所有装有电子的格子, 其中 $n$ 最大的格子就是该元素所在的周期. 例如 30 号元素, 可以按照上图绿色线条的顺序占满 $1S, 2S, 2P, 3S, 3P, 4S, 3D$ (这些格子能容纳的总电子数刚好是 30). 其中 $4S$ 的主量子数最大,$n=4$, 所以 30 号元素在第四周期. 按照这个规律, 把上图按照周期分类如下.
\begin{figure}[ht]
\centering
\includegraphics[width=8cm]{./figures/Ptable_4.pdf}
\caption{划分周期} 
\end{figure}

\addTODO{解释 1s22P2 之类的符号吧}
