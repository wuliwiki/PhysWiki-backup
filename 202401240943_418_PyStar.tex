% Python 快速入门
% license Xiao
% type Tutor

\pentry{Python 简介与安装\upref{Python}}

\subsection{作为计算器}
请在 Jupyter Notebook 中尝试输入以下命令并执行(运行结果略)。 Python 程序使用 \verb|#| 注释一行,  或者用两个 \verb|"""| 或 \verb|'''| 注释多行\footnote{事实上 \lstinline|"""..."""| 或 \lstinline|'''...'''| 是一个字符串\upref{PyStr}而不是注释, 但单独出现的字符串并不会对程序运行造成影响, 所以可以用作注释}。 注释是为了辅助人阅读代码, 不会被执行。

四则运算
\begin{lstlisting}[language=python]
2 + 2
\end{lstlisting}
\begin{lstlisting}[language=python]
123 / 456 # 得到浮点数
123 // 456 # 两数相除后向下取整(注意不是向零)
\end{lstlisting}
幂运算
\begin{lstlisting}[language=python]
3 ** 2
\end{lstlisting}
求余
\begin{lstlisting}[language=python]
4 % 3 # 使得 a == a // b + a % b 恒成立
\end{lstlisting}
使用括号
\begin{lstlisting}[language=python]
(123 - 234*2)**2 / (34 + 54**4)
\end{lstlisting}
如果一行太长, 可以用 \verb|\| 换行
\begin{lstlisting}[language=python]
1 + 2 + 3 + 4 + \
    5 + 6
\end{lstlisting}
各种常见的数学函数都在 math 模块\upref{Module}中, 需要加载。
\begin{lstlisting}[language=python]
import math
\end{lstlisting}
使用模块中的函数, 要在前面加上模块名和一点。 例如开方 (square root)
\begin{lstlisting}[language=python]
math.sqrt(284)
\end{lstlisting}
自然指数函数
\begin{lstlisting}[language=python]
math.exp(5.1)
\end{lstlisting}
这样做虽然略显麻烦, 但可以区分不同模块中同名函数。 在确保没有冲突的情况下我们也可以用以下方法加载模块中的指定函数, 如
\begin{lstlisting}[language=python]
from math import sqrt, exp, sin, cos
\end{lstlisting}
现在使用这些函数就不需要 \verb|math.| 的前缀了, 如 \verb|sin(1)|。

我们甚至可以用这种方式引入一个模块中所有函数(和其他内容)而无需前缀。 这样做引起名称冲突的可能性更大, 不建议使用。
\begin{lstlisting}[language=python]
from math import *
\end{lstlisting}
从模块中不仅可以引入函数, 还有常数, 例如圆周率和自然对数底(注意 \verb|e| 这种单字母名称很可能会产生冲突, 所以不建议取消 \verb|math.| 前缀)
\begin{lstlisting}[language=python]
sin(pi/2)
log(e)
\end{lstlisting}

\verb|math| 模块中的其他常用函数如
\begin{lstlisting}[language=python]
# 绝对值 (absolute value)
abs(-32)
# 自然对数
log(0.5)
# 以 10 为底的对数
log10(1000)
# 弧度转为角度
degrees(pi/2)
\end{lstlisting}

\subsubsection{复数}
\begin{itemize}
\item 复数常数如 \verb|z = 1+2j| 或者直接 \verb|3j|, 相当于 \verb|z = complex(1,2)|, 类型是 \verb|complex| (\verb|builtin| 类型,无需模块)。 \verb|z.real| 和 \verb|z.imag| properties 可以获取实部和虚部(只读)。
\item \verb|math.sin()| 不支持复数, 但是 \verb|numpy.sin()| 可以。 \verb|cmath.sin()| (\href{https://docs.python.org/3/library/cmath.html}{complex math} 模块) 也可以。
\end{itemize}

\subsection{函数}
Python 中的函数与数学中的函数不完全一样, 函数可以有若干个输入变量和输出变量(也可以没有)。 下面我们定义一个简单的函数来计算长方体的体积
\begin{lstlisting}[language=python]
def f(a, b, c):
    volumn = a*b*c
    return volumn
\end{lstlisting}
这段代码用到了两个 Python 的\textbf{关键字(keyword)} \verb|def| 和 \verb|return|。 关键字是指在程序中有特殊含义的单词, 不能作为变量名和函数名的名称。 其中 \verb|def| 用于定义函数, \verb|f| 是函数名, \verb|a|, \verb|b| 和 \verb|c| 分别是函数的\textbf{输入变量(argument)}。 冒号以后是\textbf{函数体}, 可以有若干行命令。 注意这些命令前面必须有\textbf{缩进(indentation)}。  在以上代码中, 函数体的第一行计算面积, 第二行将用关键字 \verb|return| 将计算的结果作为输出并退出函数。

现在我们可以使用这个函数, 使用方法和 \verb|sin|, \verb|sqrt| 等数学函数一样, 只是不同输入变量要用逗号隔开。
\begin{lstlisting}[language=python]
V = f(1.2, 3.4, 6)
print(V)
\end{lstlisting}
