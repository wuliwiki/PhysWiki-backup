% GNU Scientific Library
% 科学计算|GNU|GSL|C 语言

\begin{issues}
\issueDraft
\issueMissDepend
\end{issues}

GNU Scientific Library (GSL) 是一个 GNU 开源项目, 由 C 语言写成.

\subsection{安装}
在 Debian 系统上可执行 \verb|sudo apt install libgsl-dev| 安装.

第二种方法是自己下载源码编译.

\subsubsection{Visual Studio 编译 GSL}
\begin{itemize}
\item 其实与其自己编译,可以看看 MSYS2\upref{Mingw} 里面安装 GSL 后的 dll 能不能拿来直接用
\item 也可以直接到 \href{https://github.com/MacroUniverse/GSL-bin}{GSL-bin repo} 中复制头文件, lib 和 dll 文件
\end{itemize}

以下介绍如何使用用 CMake 和 Visual Studio 2017 编译 lib 和 dll 以及测试文件(貌似测试的时候有一项会 fail)

\begin{itemize}
\item Windows 安装 Cmake
\item 首先下载 GSL \href{https://github.com/ampl/gsl}{源码}(含 CMake)
\item 打开 CMakeLists.txt 文件, 会提示各个编译器选项怎么设置, 如
\begin{lstlisting}[language=none]
    cmake -G"Visual Studio 15 2017 Win64" -DGSL_INSTALL_MULTI_CONFIG=ON \
          -DBUILD_SHARED_LIBS=ON -DMSVC_RUNTIME_DYNAMIC=ON \
          <path to gsl sources>
\end{lstlisting}
\item 在 PowerShell 里面运行这个貌似会出错, 最稳妥的办法还是用 CMake 的 GUI 界面. 输入源码路径和输出路径, 然后 Configure. 选择 \verb`x64`, \verb`Visual Studio 2017`, \verb`native generator`. 成功以后按照上面的选项勾选 \verb`GSL_INSTALL_MULTI_CONFIG=ON` \verb`BUILD_SHARED_LIBS=ON` \verb`MSVC_RUNTIME_DYNAMIC=ON`, 然后点 generate 即可.
\item 在输出路径找到 \verb`sln` 文件, 双击打开 Visual Studio 2017, 选 Release, 确认是 x64, 然后在菜单中 Build -> All 即可.

\item 成功之后, dll 文件在 \verb`bin/Release/gsl.dll`, lib 文件在 \verb`Release/gsl.lib`, 头文件在 \verb`gsl` 中.
\item 要测试, 在 Power Shell 里面 cd 到 CMake 输出目录下, \verb`CTest -VV` 即可 (\verb`-VV` 是 extra verbose)
\end{itemize}


\subsection{使用}
在编译时加上 linker 选项 \verb|-lgsl| 即可.

\subsection{源码结构}
\begin{itemize}
\item \href{https://github.com/MacroUniverse/SLISC}{SLISC} 的源码中实现了自己从部分源码编译.
\end{itemize}
