% 正则化
% 正则化


\pentry{欠拟合\upref{unfit},过度拟合\upref{ovfit},范数、赋范空间\upref{NormV}}

\textbf{正则化}(Regularization)是机器学习中用于减少泛化误差(测试误差),从而缓解过度拟合\upref{ovfit}的设计策略。当使用正则化策略减少泛化误差时,可能会增大训练误差。

在深度学习中,很多正则化策略都是对估计进行正则化。估计的正则化会以偏差的增加换取方差的减少[1]。

\textbf{参数范数惩罚}是最常用的正则化策略之一。传统机器学习方法就有很多使用,而在当今的深度学习中也应用广泛。参数范数惩罚的主要思想是给目标函数
$J$添加一个参数范数惩罚项$\Omega(\bvec \theta)$,限制模型的学习能力,从而减少过度拟合的发生。设$J'$为正则化后的目标函数,则有:
\begin{equation}
J'(\bvec θ; \bvec X,y)=J(\bvec θ;\bvec X,y)+αΩ(\bvec θ)~.
\end{equation}