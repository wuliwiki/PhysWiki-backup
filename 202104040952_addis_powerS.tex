% 幂级数

\begin{issues}
\issueDraft
\end{issues}

\pentry{极限\upref{Lim}}

我们把形如
\begin{equation}\label{powerS_eq1}
s = \sum_{n=0}^\infty c_n (x-x_0)^n
\end{equation}
的表达式叫做\textbf{幂级数(power series)}. 其中 $(x-x_0)^0$ 始终视为 $1$, 即使 $x-x_0 = 0$.

幂级数是无穷级数\upref{infSer}的一种, 是一个极限. 如果我们把前有限项的求和记为
\begin{equation}
s_m = \sum_{n=0}^m c_n (x-x_0)^n
\end{equation}
那么\autoref{powerS_eq1} 幂级数就是
\begin{equation}
s = \lim_{m\to\infty} s_m
\end{equation}
的简写. 对给定的 $x$, 当极限存在时, 我们就说级数\textbf{收敛(converge)}, 反之就说级数\textbf{不收敛}或\textbf{发散(diverge)}. 广义来说, 幂级数中的 $c_n, x, x_0$ 都可以是复数复数\upref{CplxNo}.

\subsection{收敛半径}
一种极端的情况是幂级数\autoref{powerS_eq1} 只在 $x = x_0$ 一点处收敛(例如 $c_n = n^n$). 除此之外, 必定存在一个\textbf{收敛半径(radius of convergence)} $0 < r < 1$, 使得当 $\abs{x-x_0} < r$ 时级数总是收敛, 而 $\abs{x-x_0} > r$ 时总是发散(不收敛). 当 $x$ 是复数时, 复平面上收敛的区域就是以 $x_0$ 为圆心的一个开圆盘. 当 $\abs{x - x_0} = r$ 时, 幂级数可能收敛也可能不收敛.

计算收敛半径可以用比值判别法
\begin{equation}

\end{equation}

