% 矢量的模和度量的关系
% keys 矢量|模|度量|关系
% license Xiao
% type Tutor

\pentry{内积、内积空间\nref{nod_InerPd}}{nod_844e}

在带有内积的矢量空间(实数域 $\mathbb R$ 或复数域 $\mathbb C$ 上的)中,矢量的模的性质可用来定义度量,使得矢量空间可看成一个度量空间。反过来,满足一定性质的度量也可用来定义矢量的模。

先列出矢量\textbf{模} $\norm{v}=\sqrt{\braket{v}{v}}$ 的性质(设矢量空间为 $V$)(\autoref{eq_InerPd_8}):
\begin{enumerate}
\item 
\begin{equation}\label{eq_SNadM_1}
\norm{x}\geq 0\quad (\forall x\in V)~.
\end{equation}
且 $\norm{x}=0\Rightarrow x=0$。
\item \begin{equation}\label{eq_SNadM_2}
\norm{\lambda v}=\abs{\lambda}\norm{v}\qquad (\lambda\in V,\quad\forall\lambda\in\mathbb C,v\in V)~.
\end{equation}
\item 
\begin{equation}
\norm{x+y}\leq\norm{x}+\norm{y}\qquad (\forall x,y\in V)~.
\end{equation}
\end{enumerate}

而\textbf{度量}是满足以下三个条件的函数 $d:E\times E\rightarrow\mathbb R$( $E$ 为任意的集合)(\autoref{def_Metric_2}):
\begin{enumerate}
\item 正定性:$d(u, v) \geq 0$,且 $d(u, v)=0$ 当且仅当 $u=v$.
\item 对称性:$d(u, v) = d(v, u)$.
\item 三角不等式:$d(u, v) \leqslant d(u, w) + d(w, v)$.
\end{enumerate}
\subsection{从矢量模到度量}
通过模 $\norm{x}$ 取
\begin{equation}
d(x,y):=\norm{x-y}~.
\end{equation}
显然,由此定义的 $d$ 满足度量定义中的3个条件,故由此定义的 $d$ 可充当矢量空间的\textbf{度量}。
\begin{example}{}
在 $V=C_2(a,b)$ (\autoref{ex_HVorUV_1})中,
\begin{equation}
d(f,g)=\sqrt{\int_a^b\abs{f(x)-g(x)}^2\dd x}~
\end{equation}
可充当度量。
\end{example}
\subsection{从度量到模}
为从度量定义矢量空间中的模,需要的度量还应具有以下两个性质:

4. (平移不变性)$\forall x,y,z\in V,\quad d(x,y)=(x+z,y+z)$;

5. (与纯量 $\lambda$ 写出相当于距离延长 $\abs{\lambda}$ 倍):$d(\lambda x,\lambda y)=\abs{\lambda}d(x,y)$。

如此,取 
\begin{equation}
\norm{x}:=d(x,0)~,
\end{equation}
则它将具有矢量模的所有性质:矢量模的性质\autoref{eq_SNadM_1} 、\autoref{eq_SNadM_2} 可通过度量的正定性和性质5得到,而第三个性质这样验证:
\begin{equation}
\begin{aligned}
\norm{x+y}&=d(x+y,0)\underset{\text{平移不变性}}{=}d(x,-y)\\
&\underset{\text{三角不等式}}{\leq}d(x,0)+d(0,-y)\\
&\underset{\text{性质5,对称性}}{=}\norm{x}+\norm{y}~.
\end{aligned}
\end{equation}

\subsection{小结}
带有内积的矢量空间通过矢量的模可以直接看成一个度量空间,而度量空间(若 $E=V$)通过给度量函数进行一些限制才能得到矢量空间的模。即带有内积矢量空间包含在度量空间中。
