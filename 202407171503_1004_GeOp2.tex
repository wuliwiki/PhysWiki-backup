% 几何光学基本定律
% license Usr
% type Tutor

\pentry{波面、光线和光束\nref{nod_Opt1}}{nod_ece1}

几何光学是以实验定律为基础发展出来的理论。历史上,人们通过实验观察光的传播路径,总结形成了多个实验定律,如光的直线传播定律、光的独立传播定律、折射定律和反射定律、费马原理与马吕斯定理。

\subsection{光的直线传播定律}

\textbf{在各向同性的均匀介质中,光沿直线传播,这就是光的直线传播定律。}在日常生活中,各种障碍物大小、各种孔径尺寸远远比光的波长大得多,衍射现象极不明显,可以忽略不计,可以简单应用光的直线传播来分析宏观光现象,如影子的形成等。

\subsection{光的独立传播定律}

\textbf{从不同光源发出的光线,以不同方向经过介质某点,各光线互不影响,这就是光的独立传播定律。}利用这条定律,可以让我们对光线传播规律的研究大大简化,即只需要关心某一研究对象光线的传播,而不考虑其他光线。注意,这条定律依然只用于分析宏观光现象,详见光的干涉相关内容。

\subsection{光的折射定律和反射定律}

\subsubsection{内容介绍}

光在传播过程中遇到两种不同介质构成的界面时,在界面上会部分反射,部分折射。光的反射定律和折射定律mn
