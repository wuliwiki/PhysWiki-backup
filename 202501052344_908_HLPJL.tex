% 亨利·庞加莱(综述)
% license CCBYSA3
% type Wiki

本文根据 CC-BY-SA 协议转载翻译自维基百科\href{https://en.wikipedia.org/wiki/Carl_Friedrich_Gauss}{相关文章}。

\begin{figure}[ht]
\centering
\includegraphics[width=6cm]{./figures/8ad8fe061e9149a3.png}
\caption{} \label{fig_HLPJL_1}
\end{figure}
朱尔·亨利·庞加莱(Jules Henri Poincaré,英国发音:/ˈpwæ̃kɑːreɪ/,美国发音:/ˌpwæ̃kɑːˈreɪ/;法语发音:[ɑ̃ʁi pwɛ̃kaʁe] ⓘ;1854年4月29日—1912年7月17日)是法国的数学家、理论物理学家、工程师和科学哲学家。他常被称为博学多才的人,在数学领域被誉为“最后的普遍主义者”,因为他在他的一生中,几乎在所有的数学领域都有卓越的成就。他还被称为“现代数学的高斯”。由于他在科学、哲学方面的成功和影响,他被誉为“现代科学的典范哲学家”。

作为数学家和物理学家,庞加莱对纯数学、应用数学、数学物理学和天体力学做出了许多原创性的基础性贡献。在研究三体问题时,庞加莱成为第一个发现混沌确定性系统的人,这为现代混沌理论奠定了基础。庞加莱被认为是代数拓扑学的创始人,并且他还被认为是引入自同态形式的先驱。他还对代数几何、数论、复分析和李群理论作出了重要贡献。他著名地提出了庞加莱重现定理,该定理表明,经过足够长的时间后,一个状态将最终以任意接近其初始状态的方式重新出现,这一结论具有深远的影响。20世纪初,他提出了庞加莱猜想,后来它成为了数学史上著名的未解问题之一,直到2002–2003年被格里戈里·佩雷尔曼解决。庞加莱还推动了非欧几何在数学中的应用。

庞加莱明确了在不同变换下物理定律不变性的重要性,并且是第一个以现代对称形式呈现洛伦兹变换的人。庞加莱发现了剩余的相对论速度变换,并在1905年通过一封信记录给亨德里克·洛伦兹。这样,他实现了麦克斯韦方程的完全不变性,这是构建狭义相对论的一个重要步骤,他也为此奠定了基础,并在1905年发表了相关的基础性论文。他首次提出了引力波(ondes gravifiques)——由物体发出并以光速传播的波,这一概念要求洛伦兹变换的存在,时间是在1905年提出的。1912年,他写了一篇有影响力的论文,提出了量子力学的数学论证。庞加莱还通过对X射线的研究,为放射性现象的发现播下了种子,这一研究影响了物理学家亨利·贝克勒尔,后者因此发现了放射性现象。物理学和数学中使用的庞加莱群是以他命名的,因为他首次引入了群的概念。

庞加莱在他的时代被认为是数学和理论物理学领域的主导人物,是当时最受尊敬的数学家,被数学家保罗·潘莱维称为“理性科学的活脑”。哲学家卡尔·波普尔认为庞加莱是有史以来最伟大的科学哲学家,并且庞加莱还是科学中的常规主义观点的创立者。庞加莱在他的时代是一个公众知识分子,个人上,他支持所有人享有政治平等,并警惕天主教会当时所持的反智立场的影响。他曾担任法国科学院院长(1906年)、法国天文学会会长(1901–1903年),以及法国数学会两任会长(1886年,1900年)。
\subsection{生活}
庞加莱于1854年4月29日出生在法国摩泽尔省南锡的杜卡尔区(Cité Ducale),出身于一个有影响力的法国家庭。他的父亲莱昂·庞加莱(Léon Poincaré,1828–1892)是南锡大学的医学教授。他的妹妹阿琳嫁给了精神哲学家埃米尔·布特鲁(Émile Boutroux)。庞加莱家族中的另一位显赫人物是他的表亲雷蒙·庞加莱(Raymond Poincaré),他是法国文学学院的成员,曾在1913年至1920年间担任法国总统,并在1913年到1929年期间三度担任法国总理。
\subsubsection{教育}

南锡市大街117号的庞加莱出生地,门前有纪念牌匾。

在他的童年时期,庞加莱曾因白喉病重病一段时间,并由母亲尤金妮·劳努瓦(Eugénie Launois,1830–1897)进行特别的教育指导。

1862年,庞加莱进入了南锡的中学(现在以他的名字命名为“亨利·庞加莱中学”,并且南锡大学也以他命名)。他在这所中学度过了十一年,并且在此期间,他证明自己在所学的各个科目中都是顶尖的学生。他在写作方面表现突出。数学老师曾称他为“数学怪才”,他在法国各大中学举行的“ concours général”(全国顶尖学生比赛)中获得了第一名。他最差的学科是音乐和体育,被形容为“最多只能算及格”。他较差的表现可能与视力不良和健忘倾向有关。他于1871年从中学毕业,获得了文学和科学双学位的高中文凭(baccalauréat)。

在1870年的普法战争期间,庞加莱与父亲一起在救护队服役。

庞加莱于1873年以优异的成绩考入了巴黎高等矿业学校(École Polytechnique),并于1875年毕业。在那里,他作为数学家查尔斯·埃米特(Charles Hermite)的学生继续深造,并在1874年发表了他的第一篇论文《Démonstration nouvelle des propriétés de l'indicatrice d'une surface》(关于曲面指标性质的新证明)。从1875年11月到1878年6月,他在巴黎高等矿业学校进一步学习,并继续学习数学,同时也修习了矿业工程课程,最终于1879年3月获得了普通矿业工程师学位。

作为巴黎高等矿业学校的毕业生,庞加莱加入了法国矿业工程师队(Corps des Mines),成为法国东北部韦苏尔(Vesoul)地区的矿务监察员。在1879年8月,他亲自处理了马尼(Magny)矿区的一起矿难事故,这起事故导致18名矿工死亡。庞加莱对这次事故进行了官方调查。

与此同时,庞加莱正在为他的数学科学博士学位做准备,导师是查尔斯·埃米特。他的博士论文领域是微分方程,题为《Sur les propriétés des fonctions définies par les équations aux différences partielles》(关于由偏微分方程定义的函数的性质)。庞加莱提出了一种全新的方法来研究这些方程的性质。他不仅解决了如何求解这些方程的积分问题,还首次研究了它们的一般几何性质。他意识到,这些方程可以用来模拟太阳系中多个自由运动天体的行为。他于1879年从巴黎大学毕业。
\subsubsection{第一次科学成就}
\begin{figure}[ht]
\centering
\includegraphics[width=6cm]{./figures/838d251c754035a5.png}
\caption{1887年,33岁的年轻亨利·庞加莱} \label{fig_HLPJL_2}
\end{figure}
获得学位后,庞加莱于1879年12月开始在诺曼底的卡昂大学担任数学讲师。同时,他发表了第一篇重要的文章,讨论了一类自守函数的处理方法。

在卡昂期间,他遇到了未来的妻子路易丝·普兰·达昂塞(Louise Poulain d'Andecy,1857–1934),她是伊西多尔·吉奥弗鲁瓦·圣希莱(Isidore Geoffroy Saint-Hilaire)的孙女,埃蒂安·吉奥弗鲁瓦·圣希莱(Étienne Geoffroy Saint-Hilaire)的曾孙女。两人于1881年4月20日结婚。婚后,他们育有四个孩子:让娜(1887年出生)、伊冯娜(1889年出生)、亨丽埃特(1891年出生)和莱昂(1893年出生)。

庞加莱很快在欧洲数学界崭露头角,吸引了许多著名数学家的关注。1881年,庞加莱受邀担任巴黎大学理学院的教职,并接受了邀请。1883年至1897年间,他在高等师范学校教授数学分析。

1881至1882年,庞加莱创立了数学的新分支——微分方程的定性理论。他展示了如何在不解方程的情况下,推导出关于一组解的行为的最重要信息(因为有时解方程是不可行的)。他成功地将这种方法应用于天体力学和数学物理中的问题。
\subsubsection{职业生涯}
庞加莱从未完全放弃在矿业行政方面的职业生涯,他一直从事数学工作之外的其他职责。他曾在1881至1885年间,在公共服务部担任负责北方铁路发展的工程师。最终,他于1893年成为矿业部的总工程师,并于1910年升任监察总工程师。

从1881年开始,直到他职业生涯的结束,庞加莱一直在巴黎大学(索邦大学)任教。最初,他被任命为分析学副教授(maître de conférences d'analyse)。最终,他担任了物理与实验力学、数学物理与概率论、天体力学与天文学等多个学科的教席。

1887年,年仅32岁的庞加莱当选为法国科学院院士。1906年,他成为法国科学院院长,并于1908年3月5日当选为法兰西学院院士。

1887年,庞加莱赢得了瑞典国王奥斯卡二世的数学竞赛,解决了与多个轨道天体自由运动有关的三体问题。(详见三体问题部分)

1893年,庞加莱加入了法国经度局,参与全球时间同步的工作。1897年,庞加莱支持了一项关于圆周量度的十进制化提案,尽管这项提案未获成功,因此时间和经度的十进制化未能实现。正是这项工作促使他开始思考建立国际时区和相对运动体之间的时间同步问题。(详见相对论部分)

1904年,他参与了阿尔弗雷德·德雷福斯案件的审判,驳斥了针对德雷福斯的伪科学证据。

庞加莱曾在1901至1903年间担任法国天文学会(Société Astronomique de France,SAF)会长。

\textbf{学生}

庞加莱在巴黎大学有两位杰出的博士生,路易·巴谢利埃(Louis Bachelier,1900年)和迪米特里·庞佩乌(Dimitrie Pompeiu,1905年)。

\subsubsection{去世}
\begin{figure}[ht]
\centering
\includegraphics[width=6cm]{./figures/187243deafdc671b.png}
\caption{庞加莱家族墓地位于蒙帕纳斯公墓。} \label{fig_HLPJL_3}
\end{figure}
1912年,庞加莱因前列腺问题接受了手术,随后于1912年7月17日在巴黎因栓塞去世,享年58岁。他葬于巴黎蒙帕纳斯公墓庞加莱家族墓地,16区,靠近Émile-Richard路的门口。

2004年,前法国教育部长克劳德·阿列格尔提议将庞加莱重新安葬在巴黎的万神殿,该地点专门用于安葬最受尊敬的法国公民。
\subsection{工作}
\subsubsection{概述}
庞加莱在纯数学和应用数学的多个领域做出了许多贡献,包括:天体力学、流体力学、光学、电学、电报学、毛细作用、弹性学、热力学、势能理论、量子力学、相对论和物理宇宙学。

他具体贡献的领域包括:
\begin{itemize}
\item 代数拓扑(庞加莱几乎是这个领域的创始人)
\item 多复变函数的解析函数理论
\item 阿贝尔函数理论
\item 代数几何
\item 庞加莱猜想(由格里高利·佩雷尔曼于2003年证明)
\item 庞加莱复现定理
\item 双曲几何
\item 数论
\item 三体问题
\item 丢番图方程理论
\item 电磁学
\item 特殊相对论
\item 基本群
\end{itemize}
在微分方程领域,庞加莱提出了许多对微分方程的定性理论至关重要的结果,例如庞加莱球面和庞加莱映射。

庞加莱还提出了“每个人对误差正态法则的信仰”(参见正态分布),并发表了影响深远的论文,提供了一种新的数学论证来支持量子力学。
\subsubsection{三体问题}  
自牛顿时代以来,寻找超过两个天体在太阳系中运动的一般解一直困扰着数学家。最初,这被称为三体问题,后来扩展为n体问题,其中n代表任何数量的相互绕行的天体。n体问题被认为在19世纪末是一个非常重要且具有挑战性的问题。事实上,在1887年,为了纪念他60岁生日,瑞典国王奥斯卡二世(Oscar II)在戈斯塔·米塔格-莱夫勒(Gösta Mittag-Leffler)的建议下设立了一个奖金,奖励能解出该问题的人。公告的内容相当具体:

给定一个由任意多个质量点组成的系统,这些点按照牛顿定律相互吸引,并假设没有两个点会相撞,试图找出每个点的坐标表示为一个关于时间的已知函数的级数,并且对于所有这些时间值,该级数均匀收敛。

如果无法解决该问题,那么任何对经典力学的重要贡献都将被视为值得奖赏。尽管庞加莱并未解出原始问题,奖项最终还是颁给了他。评审之一、著名数学家卡尔·魏尔斯特拉斯(Karl Weierstrass)表示:“这项工作确实不能被视为完全解决所提出的问题,但它的重要性如此之大,以至于它的发表将开启天体力学历史的新纪元。”(他最初的贡献版本甚至包含一个严重错误;详情请参见Diacu的文章和Barrow-Green的书)。最终出版的版本包含了许多重要的思想,最终引导出了混沌理论。最初陈述的三体问题最终由卡尔·F·苏德曼(Karl F. Sundman)于1912年为n = 3情况解决,并由王秋东(Qiudong Wang)在1990年代推广至n > 3体的情况。级数解的收敛速度非常慢,即使是非常短的时间间隔,也需要几百万项才能确定粒子的运动,因此这些解在数值计算中无法使用。
\subsubsection{相对论研究}
\textbf{地方时间}

庞加莱在法国经度局(Bureau des Longitudes)的工作,推动了国际时区的建立,这使他开始思考如何同步静止在地球上的时钟,这些时钟相对于绝对空间(或“光以太”)的速度不同。与此同时,荷兰理论家亨德里克·洛伦兹(Hendrik Lorentz)正在将麦克斯韦理论发展为描述带电粒子(“电子”或“离子”)运动及其与辐射的相互作用的理论。1895年,洛伦兹引入了一个辅助量(没有物理解释),称为“地方时间”:\(t' = t - \frac{vx}{c^2}\)并提出了长度收缩的假设,来解释光学和电学实验未能检测到相对于以太的运动(见迈克耳孙-莫雷实验)。庞加莱是洛伦兹理论的常驻解释者(有时也充当友善的批评者)。作为一名哲学家,庞加莱对“更深层次的意义”感兴趣。因此,他对洛伦兹的理论进行了阐释,并在此过程中提出了许多现在与特殊相对论相关的见解。在《时间的度量》(1898年)一书中,庞加莱写道:“稍微的反思足以理解,这些陈述本身没有意义。只有作为一种约定的结果,它们才有意义。”他还认为,科学家必须将光速的恒定性作为假设,以便为物理理论提供最简单的形式。基于这些假设,庞加莱在1900年讨论了洛伦兹的“奇妙发明”——地方时间,并指出这一概念出现在通过交换光信号同步移动时钟的过程中,假设光信号在移动参考系中两方向上的传播速度是相同的。

\textbf{相对性原理与洛伦兹变换 }

1881年,庞加莱通过超曲面模型描述了双曲几何,提出了保持洛伦兹间隔不变的变换:\(x^2 + y^2 - z^2 = -1\)这使得它们在数学上等同于2+1维中的洛伦兹变换。[41][42] 此外,庞加莱的其他双曲几何模型(庞加莱圆盘模型、庞加莱半平面模型)以及贝尔特拉米–克莱因模型可以与相对论速度空间(见陀螺向量空间)相关联。

1892年,庞加莱发展了包括偏振在内的光的数学理论。他对偏振器和延迟器作用的视角,作用在表示偏振状态的球体上,被称为庞加莱球体。[43] 已证明,庞加莱球体具有底层洛伦兹对称性,可以作为洛伦兹变换和速度加法的几何表示。[44]

他在1900年通过两篇论文讨论了“相对运动原理”,并在1904年将其命名为相对性原理,依据该原理,没有物理实验可以区分匀速运动状态和静止状态。[46] 1905年,庞加莱向洛伦兹写信,讨论洛伦兹1904年的论文,庞加莱称其为“一篇至关重要的论文”。在信中,他指出洛伦兹在应用他的变换时犯了一个错误,特别是在应用到麦克斯韦方程中涉及电荷占据空间时,并且还质疑洛伦兹提出的时间膨胀因子。[47] 在第二封信中,庞加莱给出了他自己的理由,说明洛伦兹的时间膨胀因子实际上是正确的——它是为了使洛伦兹变换形成一个群,并给出了现在所称的相对论速度加法定律。[48] 庞加莱随后在1905年6月5日的巴黎科学院会议上做了演讲,讨论了这些问题。在那篇发表的版本中,他写道:[49]

“由洛伦兹确立的基本点是,电磁场的方程式在某种变换(我称之为洛伦兹变换)的作用下不会改变,其形式如下:
\[
x' = k\ell (x + \varepsilon t), \quad t' = k\ell (t + \varepsilon x), \quad y' = \ell y, \quad z' = \ell z, \quad k = \frac{1}{\sqrt{1-\varepsilon^2}}~
\]
并且展示了任意函数\(\ell(\varepsilon)\)必须在所有\(\varepsilon\)下为常数1(洛伦兹通过不同的理由设定了\(\ell = 1\))才能使变换形成一个群。1906年,庞加莱在论文的扩展版本中指出,组合形式\(x^2 + y^2 + z^2 - c^2 t^2\)是不变的。他注意到,洛伦兹变换仅仅是四维空间中绕原点的旋转,方法是引入\(ct\sqrt{-1}\)作为第四个虚数坐标,他还使用了四维向量的早期形式。[50] 1907年,庞加莱对四维几何形式的新力学表达方式表示缺乏兴趣,因为在他看来,将物理学翻译成四维几何语言会涉及过多的工作,收益却有限。[51] 因此,是赫尔曼·闵可夫斯基在1907年推导出了这一思想的后果。[51][52]

\textbf{质量–能量关系}  

像以前的其他人一样,庞加莱(1900年)发现了质量和电磁能量之间的关系。在研究作用/反作用原理与洛伦兹以太理论之间的冲突时,他试图确定在包含电磁场的情况下,重心是否仍然以均匀速度运动。[40] 他注意到,作用/反作用原理仅对物质有效,而电磁场具有自己的动量。庞加莱得出结论,电磁波的电磁场能量表现得像一种虚拟流体(fluide fictif),其质量密度为 \( E/c^2 \)。如果质心框架由物质的质量和虚拟流体的质量共同定义,并且如果虚拟流体是不可摧毁的——即它既不会被创造也不会被销毁——那么质心框架的运动仍然是均匀的。但电磁能量可以转化为其他形式的能量。因此,庞加莱假设在每个空间点存在一个非电能流体,电磁能量可以转化为该流体的能量,并且该流体也携带与能量成正比的质量。这样,质心的运动保持均匀。庞加莱表示,这些假设并不令人惊讶,因为它们仅仅是数学上的虚构。

然而,庞加莱的解决方案在改变参考框架时导致了一个悖论:如果赫兹振荡器沿某个方向辐射,它将因虚拟流体的惯性而产生反冲。庞加莱对移动源的框架进行了洛伦兹变换(以 \( v/c \) 为顺序)。他指出,能量守恒在两个框架中都成立,但动量守恒定律却被违反。这将允许永动机的存在,而这一概念是他所厌恶的。自然法则必须在参考框架中有所不同,相对性原理将不再成立。因此,他认为在这种情况下,以太中必须存在另一个补偿机制。

庞加莱自己在1904年的圣路易斯讲座中再次回到了这一话题。[46] 他拒绝了[53]能量携带质量的可能性,并批评了他自己为补偿上述问题所提出的解决方案:

“仪器将像炮一样反冲,投射的能量就像一个球,这与牛顿原理相矛盾,因为我们现在的投射物没有质量;它不是物质,它是能量。[...] 我们是否应该说,分隔振荡器与接收器的空间,而干扰必须通过这个空间从一个传递到另一个,不是空的,而是充满了不仅仅是以太,而是空气,甚至在星际空间中充满了某种微妙而又有质量的流体;这个物质在能量到达时会受到冲击,就像接收器一样,并在干扰离开时反冲?这将拯救牛顿的原理,但它并不真实。如果能量在传播过程中始终附着在某种物质基底上,这种物质将与光一起运动,而菲佐已经证明,至少在空气中,没有这种情况。米歇尔森和莫雷也确认了这一点。我们也许可以假设物质的运动恰好被以太的运动所补偿;但这将导致我们重新考虑刚才的那些思考。如果这样解释的话,这一原理可以解释任何事情,因为无论可见的运动是什么,我们都可以想象一些假设的运动来补偿它们。但如果它可以解释一切,它将无法预言任何东西;它不能让我们在不同的假设之间做出选择,因为它预先解释了所有内容。因此,它变得毫无用处。”

在上面的引用中,他提到了赫兹对以太完全牵引的假设,这一假设已被菲佐实验推翻,但该实验确实表明,光被某种物质部分“携带”。最后,在1908年[54],他重新审视了这个问题,并最终放弃了反作用原理,转而支持基于以太本身惯性的解决方案。

“但我们已经看到,菲佐的实验不允许我们保留赫兹的理论,因此必须采用洛伦兹的理论,并因此放弃反作用原理。”

他还讨论了另外两种未解释的效应:(1)洛伦兹的可变质量所暗示的质量不守恒,亚伯拉罕的可变质量理论以及考夫曼关于快速运动电子质量的实验;(2)玛丽·居里放射性实验中的能量不守恒现象。

阿尔伯特·爱因斯坦的质量–能量等价性概念(1905年),即物体失去能量作为辐射或热量时,失去的质量为 \( m = E/c^2 \),解决了[55]庞加莱的悖论,且没有使用任何补偿机制在以太中。[56] 赫兹振荡器在发射过程中失去了质量,动量在任何参考框架中都得到守恒。然而,关于庞加莱对重心问题的解决,爱因斯坦指出,庞加莱的公式和他自己1906年的公式在数学上是等价的。[57]

\textbf{引力波} 

在1905年,庞加莱首次提出了引力波(*ondes gravifiques*),认为它们是从一个物体发射出来并以光速传播的。他写道:

现在变得重要的是更仔细地考察这个假设,特别是要问它在哪些方面要求我们修改引力定律。这就是我试图确定的;最初,我得出的假设是,引力的传播不是瞬时的,而是以光速进行的。[58][49]

\textbf{庞加莱与爱因斯坦}

爱因斯坦关于相对论的第一篇论文在庞加莱的短篇论文发表后三个月发布,但早于庞加莱的长篇版本。爱因斯坦依赖相对性原理推导洛伦兹变换,并使用与庞加莱(1900年)描述的类似的时钟同步程序(爱因斯坦同步),但爱因斯坦的论文有一个显著特点:其中完全没有引用任何文献。庞加莱从未承认爱因斯坦在特殊相对论方面的工作。然而,爱因斯坦在1919年5月3日的一封信中向汉斯·范因格尔(Hans Vaihinger)表达了对庞加莱观点的同情,当时爱因斯坦认为范因格尔的总体观点与自己的相近,而庞加莱的观点与范因格尔的观点相似。在公众场合,爱因斯坦在1921年的一次演讲《几何学与经验》(*Geometrie und Erfahrung*)中追溯性地承认了庞加莱,但这与非欧几何学有关,而非特殊相对论。爱因斯坦在去世前几年评论庞加莱时称其为相对论的先驱之一,并说:“洛伦兹早已认识到他命名的变换对于分析麦克斯韦方程式至关重要,而庞加莱则进一步加深了这一见解……”[60]

\textbf{关于庞加莱和相对论的评价} 

庞加莱在特殊相对论发展中的工作得到了广泛认可,[55] 尽管大多数历史学家强调,尽管庞加莱的工作与爱因斯坦有许多相似之处,但两者的研究议题和对这些工作的解释截然不同。[61] 庞加莱发展了类似的局部时间的物理解释,并注意到与信号速度的关系,但与爱因斯坦不同的是,他在论文中继续使用以太概念,并认为在以太中静止的时钟显示的是“真实”时间,而运动中的时钟显示的是局部时间。因此,庞加莱试图保持相对性原理与经典概念一致,而爱因斯坦则基于空间和时间相对性的全新物理概念,发展出与之数学等价的运动学。[62][63][64][65][66]

虽然这是大多数历史学家的看法,但少数历史学家则更进一步,如E. T. 怀特克(E. T. Whittaker)认为庞加莱和洛伦兹才是相对论的真正发现者。[67]
\subsubsection{代数与数论}
庞加莱将群论引入物理学,并且是第一个研究洛伦兹变换群的人。[68][69] 他还对离散群及其表示理论作出了重要贡献。
\subsubsection{拓扑学}
这一学科在费利克斯·克莱因的《埃尔朗根计划》(1872年)中得到了明确的定义:任意连续变换的几何不变量,一种几何学。根据约翰·本尼迪克特·利斯廷的建议,"拓扑学"这一术语被引入,取代了之前使用的"位势分析"。一些重要的概念由恩里科·贝蒂和伯恩哈德·黎曼提出。然而,这门科学的基础,适用于任何维度的空间,是由庞加莱奠定的。他的第一篇相关论文发表于1894年。[70]

他在几何学方面的研究导致了同伦和同调的抽象拓扑定义。他还首次引入了组合拓扑的基本概念和不变量,如贝蒂数和基本群。庞加莱证明了一个公式,涉及n维多面体的边、顶点和面数(欧拉–庞加莱定理),并首次准确地表述了直观的维度概念。[71]
\subsubsection{天文学与天体力学}
\begin{figure}[ht]
\centering
\includegraphics[width=6cm]{./figures/de4a8c8d448269e9.png}
\caption{《天体力学新方法》第一卷的标题页(1892年)} \label{fig_HLPJL_4}
\end{figure}
庞加莱出版了两本现在被视为经典的专著,《天体力学新方法》(1892–1899)和《天体力学讲义》(1905–1910)。在这些著作中,他成功地将研究成果应用于三体问题,详细研究了解的行为(频率、稳定性、渐近性等)。他引入了小参数法、固定点、积分不变量、变分方程、渐近展开的收敛性等概念。通过对布伦斯(1887年)理论的推广,庞加莱证明了三体问题不可积。换句话说,三体问题的广义解不能通过明确的坐标和天体的速度,使用代数和超越函数来表达。庞加莱在这一领域的工作是自艾萨克·牛顿以来,天体力学的第一个重大成就。[72]

这些专著中包含了庞加莱的一个重要思想,后来成为了数学“混沌理论”(特别是庞加莱回归定理)和动态系统一般理论的基础。庞加莱还撰写了关于天文学的重要著作,研究了引力旋转流体的平衡图形。他引入了分叉点的概念,并证明了像非椭圆体、环形和梨形等平衡图形的存在及其稳定性。因这一发现,庞加莱获得了皇家天文学会的金奖(1900年)。[73]
\subsubsection{微分方程和数学物理}
在完成关于微分方程系统奇点研究的博士论文后,庞加莱撰写了一系列题为《由微分方程定义的曲线》的论文(1881-1882年)。在这些文章中,他创立了数学的新分支——“微分方程的定性理论”。庞加莱展示了,即使微分方程无法用已知函数解出,从方程本身的形式也能提取出关于解的性质和行为的大量信息。特别是,庞加莱研究了积分曲线在平面上的轨迹性质,给出了奇点的分类(鞍点、焦点、中心、节点),引入了极限环和回路指数的概念,并表明,极限环的数量总是有限的,除了一些特殊情况外。庞加莱还发展了积分不变量和变分方程解的一般理论。对于有限差分方程,他开创了一个新的方向——解的渐近分析。他将这些成果应用于研究数学物理和天体力学的实际问题,并且所使用的方法为其拓扑学著作奠定了基础。
\subsection{性格}
\begin{figure}[ht]
\centering
\includegraphics[width=6cm]{./figures/133fd9efc34fc1d9.png}
\caption{亨利·曼纽尔拍摄的H·庞加莱的摄影肖像} \label{fig_HLPJL_5}
\end{figure}
庞加莱的工作习惯常被比作一只蜜蜂从一朵花飞到另一朵花。庞加莱对自己思维的运作方式感兴趣;他研究了自己的习惯,并在1908年于巴黎的综合心理学研究所发表了一场关于他观察的演讲。他将自己的思维方式与自己的一些发现联系起来。

数学家达尔布克斯(Darboux)称他为“非直觉型”人物,认为这一点可以通过庞加莱经常使用视觉化的方式进行工作的事实来证明。雅克·哈达玛(Jacques Hadamard)写道,庞加莱的研究展现了非凡的清晰性[76],而庞加莱本人则写道,他认为逻辑不是发明的方式,而是组织思想的方式,逻辑限制了思想的发挥。

\subsubsection{图卢兹对庞加莱的个性描述}  
庞加莱的思维方式不仅对庞加莱本人有趣,也引起了巴黎高等研究院心理学实验室心理学家爱德华·图卢兹的兴趣。图卢兹写了一本名为《亨利·庞加莱》(1910)的书。[77][78] 书中,他讨论了庞加莱的规律作息:

\begin{itemize}
\item 他每天在固定的时间段进行工作,每次工作时间较短。他每天花四个小时从上午10点到中午12点,再从下午5点到7点进行数学研究。晚上稍晚时,他会阅读期刊上的文章。
\item 他的正常工作习惯是先在脑中完全解决一个问题,然后把解决的问题记录在纸上。
\item 他是左右手都能使用的人,并且视力不好。
\item 他听到的内容能够通过视觉化能力加以理解,这对于他听讲座时尤其有用,因为他的视力差,无法清楚地看到讲师在黑板上写的内容。
\end{itemize}
这些能力在某种程度上被他的缺点所抵消:
\begin{itemize}
\item 他在身体上笨拙,艺术方面不灵活。
\item 他总是匆忙,讨厌回头修改或修正。
\item 他从不在一个问题上花费太长时间,因为他相信潜意识会在他集中精力处理另一个问题时继续处理当前的问题。
\end{itemize}
此外,图卢兹还指出,大多数数学家工作时会依赖于已经建立的原则,而庞加莱每次都从基本原则开始工作(O'Connor等,2002)。

他的思维方式可以总结为:

“习惯忽略细节,只关注顶峰,他在顶峰之间以令人惊讶的迅速转变,他发现的事实围绕着其中心自动聚集,并迅速且自动地被分类存入他的记忆中。”

—— 贝利维尔(1956)
\subsubsection{出版物}
\begin{itemize}
\item 《光的数学理论讲义》(法文)。巴黎:Carrè,1889年。  
\item 《周期解,均匀积分不存在,渐近解》(法文)。第1卷。巴黎:Gauthier-Villars,1892年。  
\item 《Newcomb、Gylden、Lindstedt 和 Bohlin 方法》(法文)。第2卷。巴黎:Gauthier-Villars,1893年。  
\item 《电振荡》(法文)。巴黎:Carrè,1894年。  
\item 《积分不变量,第二类周期解,双重渐近解》(法文)。第3卷。巴黎:Gauthier-Villars,1899年。  
\item 《科学的价值》(法文)。巴黎:Flammarion,1900年。  
\item 《电学与光学》(法文)。巴黎:Carrè & Naud,1901年。  
\item 《科学与假设》(法文)。巴黎:Flammarion,1902年。  
\item 《热力学》(法文)。巴黎:Gauthier-Villars,1908年。  
\item 《最后的思考》(法文)。巴黎:Flammarion,1913年。  
\item 《科学与方法》。伦敦:Nelson and Sons,1914年。
\end{itemize}
\subsubsection{遗产}  
庞加莱被认为为狭义相对论奠定了基础,[10][9] 有人认为他应当被视为狭义相对论的创立者。[79] 据说他“主宰了他那个时代的数学和理论物理”,并且“他无疑是他在世时最受尊敬的数学家,至今仍然是世界上最具象征性的科学人物之一。”[80] 庞加莱被誉为“全才”,因为他精炼了天体力学,推进了他时代几乎所有数学领域,包括创造了新学科,是狭义相对论的奠基人,参与了他那个时代物理学的所有重大辩论,是他那个时代哲学与科学认识论辩论的主要人物之一,此外,他还是1879年马尼矿井瓦斯爆炸事件的调查工程师。[80] 由于其研究的广度,庞加莱是唯一一位被选为法国科学院各个部门成员的人,包括几何学、力学、物理学、天文学和航海学。[81]

物理学家亨利·贝克勒尔于1904年提名庞加莱获得诺贝尔奖,因为贝克勒尔认为“庞加莱的数学与哲学天才审视了所有的物理学,他是那些为人类进步做出最大贡献的人之一,给研究人员提供了进入未知领域的坚实基础。”[82] 庞加莱去世后,许多当时的知识分子对他表示赞扬,作家玛丽·波拿巴写信给他的寡妇路易丝,称:“他是——如你比任何人都更清楚——不仅是最伟大的思想家,是我们时代最强大的天才——还是一个深刻而无与伦比的心灵;而且与他亲近的记忆,是一生中最宝贵的回忆。”[83]

数学家E.T. 贝尔称庞加莱为“最后的全才”,并注意到他在许多领域的才能,表示:[84]

“庞加莱是最后一位几乎涵盖所有纯数学和应用数学的学者……很少有数学家具备庞加莱那样广阔的哲学视野,而没有人能超越他在清晰表述方面的天赋。”

当哲学家兼数学家伯特兰·罗素被问及法国现代史上最伟大的人物是谁时,他立即回答道:“庞加莱。”[84] 贝尔指出,如果庞加莱在实践科学方面的造诣和在理论科学方面一样强大,他可能会“与无与伦比的三位大人物——阿基米德、牛顿和高斯——并列第四。”[85]

贝尔进一步提到庞加莱的强大记忆力,甚至超过了莱昂哈德·欧拉,表示:[85]

“他主要的消遣是阅读,这也是他非凡才华首次展现的地方。一本书一旦读过——以令人难以置信的速度——就成了他的永久财产,他总能准确指出某个特定内容所在的页数和行数。他一生都保留着这种强大的记忆力。这种稀有的能力,庞加莱与欧拉共享,后者虽然程度较轻,但也具备类似的能力,可以称之为视觉记忆或空间记忆。在时间记忆——即回忆长久过去的事件顺序的能力——方面,他也表现得尤为强大。”

贝尔还提到庞加莱的视力非常差,他几乎完全通过耳朵记住公式和定理,且“当他成为高级数学的学生时,由于无法清楚看到黑板,他坐在后排听讲,完全跟随并记住所有内容,且不做笔记——对他来说这很轻松,但对大多数数学家来说则是不可理解的事。”[85]
\subsubsection{荣誉}  
\textbf{奖项}  
\begin{itemize}
\item 瑞典国王奥斯卡二世数学竞赛奖(1887年)  
\item 荷兰皇家艺术与科学院外籍会员(1897年)[86]  
\item 美国哲学会会员(1899年)  
\item 伦敦皇家天文学会金奖(1900年)  
\item 法国荣誉军团指挥官(1903年)[87]  
\item 博尔亚奖(1905年)  
\item 马图奇奖章(1905年)  
\item 法国科学院会员(1906年)  
\item 法国文学院院士(1909年)  
布鲁斯奖章(1911年)
\end{itemize}  
\textbf{以他命名}  
\begin{itemize}
\item 庞加莱研究所(数学与理论物理中心)  
\item 庞加莱奖(数学物理国际奖)  
\item 《庞加莱年刊》(科学期刊)  
\item 庞加莱研讨会(昵称“Bourbaphy”)  
\item 月球上的庞加莱陨石坑  
\item 小行星 2021 庞加莱  
\item 以庞加莱命名的事物列表 
\end{itemize} 

庞加莱没有获得诺贝尔物理学奖,但他有像亨利·贝克勒尔或委员会成员Gösta Mittag-Leffler这样的有影响力的支持者。[88][89] 提名档案显示,庞加莱在1904年至1912年间(即他去世的那一年)共收到51次提名。[90] 在1910年诺贝尔奖的58次提名中,34次提名了庞加莱。[90] 提名人包括诺贝尔奖得主亨德里克·洛伦茨和皮特·泽曼(1902年获奖),玛丽·居里(1903年获奖),阿尔伯特·米歇尔森(1907年获奖),加布里埃尔·利普曼(1908年获奖)和古列尔莫·马可尼(1909年获奖)。[90]

像庞加莱、玻尔兹曼或吉布斯这样的著名理论物理学家未能获得诺贝尔奖,被认为是诺贝尔委员会更看重实验而非理论的证据。[91][92] 在庞加莱的案例中,许多提名他的人指出,最大的困难是要确定他具体的发现、发明或技术。[88]
\subsection{哲学}
\begin{figure}[ht]
\centering
\includegraphics[width=6cm]{./figures/93fa20ad9884533f.png}
\caption{《科学与假设》(1905年)第一页} \label{fig_HLPJL_6}
\end{figure}
庞加莱的哲学观点与伯特兰·罗素和戈特洛布·弗雷格相反,他们认为数学是逻辑的一个分支。庞加莱强烈不同意这一观点,他认为直觉是数学的生命。庞加莱在1902年出版的《科学与假设》一书中给出了一个有趣的观点:

“对于表面观察者来说,科学真理是毫无疑问的;科学的逻辑是无误的,如果科学家们有时犯错,那也仅仅是因为他们误解了其规则。”

庞加莱认为算术是合成的。他认为,皮亚诺公理不能通过归纳原理以非循环的方式证明(Murzi, 1998),因此得出结论,算术是先验合成的,而不是分析的。庞加莱接着说,数学不能从逻辑中推导出来,因为它不是分析的。他的观点类似于伊曼努尔·康德的观点(Kolak, 2001;Folina, 1992)。他强烈反对康托尔的集合论,反对其使用自指定义。[93]

然而,庞加莱并不在所有哲学和数学领域都持康德的观点。例如,在几何学中,庞加莱认为非欧几何空间的结构可以通过分析得知。庞加莱认为惯例在物理学中扮演着重要角色。他的观点(以及一些后来的、更极端的版本)被称为“惯例主义”。[94] 庞加莱认为,牛顿的第一定律不是经验性的,而是力学的一个惯例性框架假设(Gargani, 2012)。[95] 他还认为,物理空间的几何是惯例性的。他举了几个例子,在这些例子中,物理场的几何或温度梯度可以发生变化,既可以描述一个非欧几何的空间,用刚性尺子进行测量,也可以描述一个欧几何空间,其中尺子由于温度分布的变化而扩展或收缩。然而,庞加莱认为,我们已经习惯了欧几何,因此我们更愿意改变物理定律以保留欧几何,而不是转向非欧几何物理学。[96]
\subsubsection{自由意志}
庞加莱在巴黎心理学学会的著名讲座(后出版为《科学与假设》、《科学的价值》和《科学与方法》)被雅克·哈达玛尔引用,作为创意和发明由两个心理阶段组成的观点的来源:首先是随机组合可能的解决方案,然后是对这些解决方案的批判性评估。[97]

尽管庞加莱大多数时候谈论的是一个决定论的宇宙,但他也表示,潜意识生成新可能性的过程涉及到偶然性。

他写道:

“可以肯定的是,那些在经过一段相当长时间的无意识工作后,突然出现在思维中的组合通常是有用和富有成果的组合……所有的组合都是潜意识自我的自动作用的结果,但只有那些有趣的组合才会进入意识领域……只有少数是和谐的,因此既有用又美丽,并且它们将能够影响我所说的几何学家的特殊敏感性;一旦被唤起,这种敏感性将引导我们的注意力集中在它们身上,从而给它们变得意识化的机会……相反,在潜意识自我中,存在着我所称之为自由的东西,如果能用这个名字来形容那种仅仅是没有纪律和由偶然产生的混乱的话。”[98]

庞加莱的这两个阶段——随机组合后跟随选择——成为了丹尼尔·丹尼特自由意志的两阶段模型的基础。[99]
\subsection{参考书目} 
\subsubsection{庞加莱的英文著作}  
关于科学哲学的普及性著作:
\begin{itemize}
\item Poincaré, Henri (1902–1908), The Foundations of Science, New York: Science Press; 1921年再版;本书包含《科学与假设》(1902年)、《科学的价值》(1905年)、《科学与方法》(1908年)的英文翻译。  
\item 1905. “Science and Hypothesis”,The Walter Scott Publishing Co.  
\item 1906. “The End of Matter”,Athenæum  
\item 1913. “The New Mechanics”,The Monist, Vol. XXIII.  
\item 1913. “The Relativity of Space”,The Monist, Vol. XXIII.  
\item 1913. Last Essays,New York: Dover reprint, 1963  
\item 1956. Chance,In James R. Newman, ed., The World of Mathematics (4 Vols).  
\item 1958. The Value of Science, New York: Dover.
\end{itemize}

关于代数拓扑学:
\begin{itemize}
\item 1895. Analysis Situs(PDF),原版存档于2012年3月27日。首部系统研究拓扑学的著作。
\end{itemize}

关于天体力学:
\begin{itemize}
\item 1890. Poincaré, Henri (2017). The Three-Body Problem and the Equations of Dynamics: Poincaré's Foundational Work on Dynamical Systems Theory. Translated by Popp, Bruce D. Cham, Switzerland: Springer International Publishing. ISBN 978-3-319-52898-4.  
\item 1892–99. New Methods of Celestial Mechanics, 3 vols. 英文翻译,1967年。ISBN 1-56396-117-2.  
\item 1905. “The Capture Hypothesis of J. J. See”,The Monist, Vol. XV.  
\item 1905–10. Lessons of Celestial Mechanics
\end{itemize}.

关于数学哲学:
\begin{itemize}
\item Ewald, William B., ed., 1996. From Kant to Hilbert: A Source Book in the Foundations of Mathematics, 2 vols. Oxford Univ. Press. 包含庞加莱的以下作品:  
\item 1894, “On the Nature of Mathematical Reasoning”,972–981.  
\item 1898, “On the Foundations of Geometry”,982–1011.  
\item 1900, “Intuition and Logic in Mathematics”,1012–1020.  
\item 1905–06, “Mathematics and Logic, I–III”,1021–1070.  
\item 1910, “On Transfinite Numbers”,1071–1074.  
\item 1905. “The Principles of Mathematical Physics”,The Monist, Vol. XV.  
\item 1910. “The Future of Mathematics”,The Monist, Vol. XX.  
\item 1910. “Mathematical Creation”,The Monist, Vol. XX.
\end{itemize}

其他:
\begin{itemize}
\item 1904. Maxwell's Theory and Wireless Telegraphy, New York, McGraw Publishing Company.  
\item 1905. “The New Logics”,The Monist, Vol. XV.  
\item 1905. “The Latest Efforts of the Logisticians”,The Monist, Vol. XV.
\end{itemize}

英文翻译的详细书目:
\begin{itemize}
\item 1892–2017. Henri Poincaré Papers, 存档于2020年8月1日
\end{itemize}。
\subsection{参见} 
\subsubsection{概念}
\begin{itemize}
\item 庞加莱–安德罗诺夫–霍普分岔  
\item 庞加莱复形 – 闭合可定向流形的单链复形的抽象  
\item 庞加莱对偶  
\item 庞加莱圆盘模型  
\item 庞加莱展开  
\item 庞加莱规范  
\item 庞加莱群  
\item 庞加莱半平面模型  
\item 庞加莱同调球面  
\item 庞加莱不等式  
\item 庞加莱引理  
\item 庞加莱映射  
\item 庞加莱残量  
\item 庞加莱级数(模形式)  
\item 庞加莱空间  
\item 庞加莱度量  
\item 庞加莱图  
\item 庞加莱多项式  
\item 庞加莱级数  
\item 庞加莱球面  
\item 庞加莱–爱因斯坦同步  
\item 庞加莱–勒隆方程  
\item 庞加莱–林德斯特方法  
\item 庞加莱–林德斯特摄动理论  
\item 庞加莱–斯捷克洛夫算子  
\item 欧拉–庞加莱特征  
\item 诺依曼–庞加莱算子  
\item 反射函数  
\end{itemize}