% 场与粒子的相互作用
% 场论|相互作用|interaction|电磁场

\pentry{自由粒子拉格朗日函数(狭义相对论)\upref{FPLSR}}

本文中$c=1$,闵可夫斯基空间度规为$\opn{diag}(1, -1, -1, -1)$.

特别要强调,我们并不是把粒子和场分开讨论,得到两种作用量或者拉格朗日函数;我们研究的是粒子和场构成的整体,只有对这个整体的\textbf{一个}作用量.

\subsection{场对粒子的作用}

粒子的运动轨迹由拉格朗日函数决定,因此要体现场对粒子的作用,就需要在粒子的拉格朗日函数里有场的出现.

自由粒子的拉格朗日函数\upref{FPLSR}为$L(t, x^i, \dot{x}^i) = -mc^2\sqrt{1-\dot{x}^i\dot{x}^jg_{ij}}$.如何添加一个“相互作用”的项呢?注意到和粒子有关的部分是$\dot{\bvec{x}}$,我们带着这部分,简单地把标量场$\phi$加进去试试:
\begin{equation}\label{IntFP_eq1}
\mathcal{L}(t, x^i, \dot{x}^i) = -(mc^2+g\phi)\sqrt{1-\dot{x}^i\dot{x}^jg_{ij}}
\end{equation}
其中$g$是一个常数,用来表征场对粒子的作用强度.

把\autoref{IntFP_eq1} 代入粒子的欧拉-拉格朗日方程,得到粒子的运动方程
\begin{equation}\label{IntFP_eq2}
\frac{\dd }{\dd t} \frac{-\dot{x}^k}{\sqrt{1-\dot{x}^i\dot{x}^jg_{ij}}} = -g\frac{\partial \phi}{\partial x^k}\sqrt{1-\dot{x}^i\dot{x}^jg_{ij}}
\end{equation}

如果取非相对论极限,即$\dot{x}^k\ll 1$,把$c$添回去,那么\autoref{IntFP_eq1} 化为
\begin{equation}
\mathcal{L}(t, x^i, \dot{x}^i) = 
\end{equation}

\autoref{IntFP_eq2} 简化为
\begin{equation}
\frac{\dd}{\dd t}
\end{equation}
























