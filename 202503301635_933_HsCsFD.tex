% 圆锥曲线的统一定义(高中)
% keys 准线|第二定义|焦点|圆锥曲线|焦点-准线定义
% license Usr
% type Tutor

\begin{issues}
\issueDraft
\end{issues}

\pentry{圆锥曲线与圆锥\nref{nod_ConSec}}{nod_55cd}

古希腊时期,人们通过截取圆锥面来研究圆、椭圆、抛物线和双曲线等曲线。这种方法虽然直观,并赋予了这些曲线共同的几何起源,但在具体研究时,仍将它们视为彼此独立的四类对象,分别分析各自的性质。它们的代数表达不同,图像形状各异,彼此之间缺乏统一的联系。

随着jie xi ji的建立,数学家们逐渐发现:这些看似不同的曲线,在引入一个定点和一条定直线后,竟然可以通过一个简洁而优雅的定义统一起来。这一定义不仅在解析几何中揭示了圆锥曲线的本质,还在射影几何等更深层次的研究中带来了意想不到的收获。

然而,在高中教材中,常常在介绍椭圆的长轴、短轴和焦点等概念时,尚未引入准线或统一定义,便提前给出“离心率”,并简单地解释为“衡量椭圆扁平程度”的参数。但细心的读者可能会注意到,抛物线和双曲线也有各自的“离心率”,这自然会引发疑问:为什么所有圆锥曲线都有离心率?这个量到底为何被这样定义?不同曲线的离心率之间又有怎样的联系?

实际上,在阿波罗尼乌斯的时代,“离心率”这一概念尚未系统建立。它是在圆锥曲线的统一定义提出之后,才逐渐发展成为刻画这些曲线的重要参数。可惜的是,这一部分内容在现行高中阶段的课程中已被删除。

为带给读者更完整的视角与更新的体验,本文将从统一定义出发,系统探讨圆锥曲线的几何构造及其深层联系。

古希腊时期,人们通过截取圆锥面来研究圆、椭圆、抛物线和双曲线等曲线。尽管这种方式直观,而且给予这些曲线同样的来源,但在研究各类曲线时,仍是将视为彼此不同的四类曲线,研究各自的性质。它们的代数表达式不同,图像形状也不相同。随着坐标系的引入,数学家们逐渐发现,这些看似不同的曲线,其实可以在引入一条直线后,通过一个简洁而优雅的定义统一描述。这一定义不仅在解析几何中揭示了圆锥曲线的本质,也在射影几何等更高层次的研究中带来了意想不到的收获。

另外,在高中教材中,通常在讲解椭圆的长轴、短轴和焦点等概念时,尚未引入准线或统一定义,此时“离心率”却被提前引入,并被简单解释为“表示椭圆的扁平程度”。但细心的读者可能会发现,不只是椭圆,抛物线和双曲线也都提到了“离心率”,这不禁引发疑问:为什么要定义这个量?又为什么是这样定义?不同曲线的离心率之间究竟有怎样的联系?

事实上,在阿波罗尼乌斯的时代,“离心率”这个概念还未被系统提出。它是在圆锥曲线的统一定义建立之后,才发展为刻画各类圆锥曲线性质的关键参数。可惜的是,这部分内容在高中阶段已被完全删除。为带给读者更完整的视角与更新的体验,本文将从这一统一定义出发,探索圆锥曲线的几何构造与背后的深层结构。


\subsection{圆锥曲线的焦点-准线定义}

利用准线与焦点得到的。提供了一个统一的视角来看待

\textbf{圆锥曲线的焦点-准线定义(Focus-Directrix Definition of Conic Sections)}。

\begin{definition}{圆锥曲线的焦点-准线定义}
在平面上,所有到一个定点的距离与到一条定直线的距离的比值是一个固定常数的点的轨迹,称为\textbf{圆锥曲线(conic section)}。其中,定点称为圆锥曲线的\textbf{焦点(focus)},定直线称为圆锥曲线的\textbf{准线(directrix)},二者互相对应,对应的焦点与准线的距离称作\textbf{焦准距(focal parameter)},通常记作$p$。比值称作圆锥曲线的\textbf{离心率(eccentricity)},通常记作$e$ 。特别地:
\begin{itemize}
\item 当 $e = 0$ 时,轨迹称为\textbf{圆(circle)}。
\item 当 $0 < e < 1$ 时,轨迹称为\textbf{椭圆(ellipse)}。
\item 当 $e = 1$ 时,轨迹称为\textbf{抛物线(parabola)}。
\item 当 $e > 1$ 时,轨迹称为\textbf{双曲线(hyperbola)}。
\end{itemize}
\end{definition}

显然,定点到定直线的垂线为圆锥曲线的对称轴。

\addTODO{$e = 0$ 为什么是圆?}

\subsection{性质}

离心率表示“扁平程度”:
$$ e = \frac{c}{a} = \sqrt{1 - \frac{b^2}{a^2}} \in [0, 1) ~.$$
椭圆越接近 1 越扁。

\subsection{定义等价性}

\subsubsection{椭圆}

由直角坐标方程可知对称性,可在椭圆的两边做两条准线,令椭圆上任意一点到两焦点的距离分别为 $r_1$ 和 $r_2$,到两准线的距离分别为 $d_1$ 和 $d_2$,则有
\begin{equation}
e = \frac{r_1}{d_1} = \frac{r_2}{d_2} = \frac{r_1 + r_2}{d_1 + d_2}~,
\end{equation}
所以
\begin{equation}
r_1 + r_2 = e(d_1+d_2) = 2e(c + h) = 2\frac{c}{a} \qty( c + \frac{b^2}{c} ) = 2a~,
\end{equation}
证毕。

\subsection{*射影几何视角下的圆锥曲线}

射影几何中的视角使我们能够用一种统一且优雅的方式看待圆锥曲线。但在射影几何中,这些差异被看作是坐标选择与观察角度所导致的表象变化,它们在更本质的层面上是一类对象的不同表现:它们都是圆锥曲线。圆锥曲线不是三类不同的曲线,而是一个统一的几何实体的三种视角。它让我们跳出了直观图形的束缚,从结构上理解几何对象之间的联系,也为代数几何、复几何乃至更高维的几何打下了坚实的基础。

从射影几何的角度看,圆锥曲线定义为一个圆锥面与一个平面相交的轨迹。这个定义在欧几里得空间中也成立,但射影几何更进一步地指出:在射影平面中,所有非退化的圆锥曲线都是射影等价的。这意味着我们可以通过一个合适的射影变换(即坐标的线性变换加上归一化),将任意一个圆锥曲线变为另一个圆锥曲线——比如将一个椭圆变为一个双曲线或抛物线。

换句话说:
\begin{itemize}
\item 椭圆是在射影平面中与无穷远直线没有实交点的圆锥曲线;
\item 双曲线是在射影平面中与无穷远直线有两个实交点的圆锥曲线;
\item 抛物线是恰好与无穷远直线有一个交点的极限情形。
\end{itemize}

这种分类在射影几何中失去了意义,因为无穷远直线被作为与其他直线同等地位来处理,不再是“例外的部分”。因此,抛物线、椭圆和双曲线不再是本质不同的几何对象,而只是一个对象的不同投影或表示。

此外,射影几何还强调了极点与极线的对偶性,并引入了极线极点变换的工具来研究圆锥曲线的性质,使得很多命题具有了对称且优美的形式。例如:对于一个给定的圆锥曲线,任意一点都有与之对应的一条极线,反之亦然。这种对偶关系在欧氏几何中并不自然存在。