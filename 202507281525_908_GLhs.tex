% 格林函数(综述)
% license CCBYSA3
% type Wiki

本文根据 CC-BY-SA 协议转载翻译自维基百科\href{https://en.wikipedia.org/wiki/Green\%27s_function}{相关文章}。

在数学中,格林函数(Green's function,亦称 Green 函数)是定义在某一给定初始条件或边界条件下的非齐次线性微分算子的冲激响应。

这意味着,如果 $L$ 是一个线性微分算子,那么:
\begin{itemize}
\item 格林函数$G$ 是满足方程 $LG = \delta$ 的解,其中 $\delta$ 是狄拉克δ函数;
\item 初值问题 $Ly = f$ 的解为 $G * f$(即格林函数与 $f$ 的卷积)。
\end{itemize}
通过叠加原理,对于一个线性常微分方程(ODE)$Ly = f$,可以先对每个源点 $s$ 求解 $LG = \delta_s$,由于源项可以看作多个δ函数的叠加,根据线性算子 $L$ 的线性性,最终的解也是各个格林函数的线性叠加。

格林函数得名于英国数学家乔治·格林,他在 1820 年代首次提出了这一概念。在现代关于线性偏微分方程的研究中,格林函数更多被视为基本解的一种方式来研究。

在多体理论(many-body theory)中,这一术语也广泛用于物理学,特别是在**量子场论**、**空气动力学**、**空气声学**、**电动力学**、**地震学**和**统计场论**中,用来表示各种类型的**关联函数(correlation function)**,即便它们并不总是符合严格的数学定义。在**量子场论**中,格林函数还扮演着传播子(propagator)的角色。
