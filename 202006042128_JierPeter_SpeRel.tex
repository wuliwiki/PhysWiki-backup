% 光速不变原理
% 只谈光速不变原理的由来
\pentry{公理系统\upref{axioms},麦克斯韦方程组\upref{MWEq}}

“光速在任意参考系下都不变”这一理念,和广为流传的误解不同,并不是实验中得来的.部分书籍会简单粗暴地告诉你,Michelson-Morley(迈克尔逊-莫雷)实验是为了寻找以太存在的证据而进行的,实验结果表明在误差范围内光速在任何参考系都是一样的,因此提出了“光速不变原理”,也就是狭义相对论的两个公理之一.然而,这是对历史的错误描述.一些物理课本使用这样的描述是有其教育意义的,因为用这样的误解很容易引入狭义相对论,而不需要学生有扎实的电动力学基础.

本书秉承准确、翔实的原则,将当代物理史学界对“光速不变原理”的由来阐释如下.

\subsection{关于Michelson-Morley实验的误解}



% 未完成: 应该把 “洛伦兹变换” 中的相关内容放到 “狭义相对论” 中
