% 薛定谔经典场
% keys 经典场论|薛定谔方程|理论力学

\pentry{经典场论基础\upref{classi}}
通常的场论都是相对论性的,满足洛伦兹协变,但是在很多情况下我们也要处理非相对论的场,例如在凝聚态系统中,我们需要处理诸如Gross-Pitaevskii方程等。因此在这里我们讨论一些从场论角度看非相对论量子力学的问题。
\section{薛定谔方程的场论表述}
我们知道对于单粒子,薛定谔方程为\upref{TDSE11}
\begin{equation}
i\hbar \frac{\partial}{\partial t} \Psi = \hat{H} \Psi = -\frac{\hbar^2 \nabla^2}{2m} \Psi + V(x)\Psi ~.
\end{equation}
拉氏量的构造可以有很多种,这里我们选取一种拉式量的构造方法,给出薛定谔方程。我们定义
$$
L =  \Psi^\star [i\hbar \frac{\partial}{\partial t} -\frac{\hbar^2 \nabla^2}{2m} + V(x)]\Psi~,
$$
对应的作用量为
$$
S = \int dt \int d^3 x L~.
$$
在这里场变量为$\Psi$,对应的正则动量为
$$
\pi \equiv \frac{\partial L}{\partial (\partial_t \Psi)} = i\hbar \Psi^\star~.
$$