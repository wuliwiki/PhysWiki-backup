% 初等矩阵与初等变化

\subsection{初等变换 Elementary Operation}
初等变换包括以下三种类型:

\begin{enumerate}
\item 交换两行(列)

\begin{equation}
\left[
    \begin{array}{ccc}
        a_{11} & a_{12} & a_{13}\\
        a_{21} & a_{22} & a_{23}\\
        a_{31} & a_{32} & a_{33}\\
    \end{array}
\right]
\Rightarrow
\left[
    \begin{array}{ccc}
        a_{11} & a_{12} & a_{13}\\
        a_{31} & a_{32} & a_{33}\\
        a_{21} & a_{22} & a_{23}\\
    \end{array}
\right]
\end{equation}

\item 一行(列)扩大非零常数倍数
\begin{equation}
\left[
    \begin{array}{ccc}
        a_{11} & a_{12} & a_{13}\\
        a_{21} & a_{22} & a_{23}\\
        a_{31} & a_{32} & a_{33}\\
    \end{array}
\right]
\Rightarrow
\left[
    \begin{array}{ccc}
        a_{11} & a_{12} & a_{13}\\
        ca_{21} & ca_{22} & ca_{23}\\
        a_{31} & a_{32} & a_{33}\\
    \end{array}
\right]
\end{equation}
\item 一行(列)加上另一行(列)的常数倍数
\begin{equation}
\left[
    \begin{array}{ccc}
        a_{11} & a_{12} & a_{13}\\
        a_{21} & a_{22} & a_{23}\\
        a_{31} & a_{32} & a_{33}\\
    \end{array}
\right]
\Rightarrow
\left[
    \begin{array}{ccc}
        a_{11} & a_{12} & a_{13}\\
        a_{21}+ca_{31} & a_{21}+ca_{32} & a_{21}+ca_{33}\\
        a_{31} & a_{32} & a_{33}\\
    \end{array}
\right]
\end{equation}
\end{enumerate}

\subsection{初等矩阵 Elementary Matrices}
单位矩阵$\mat I$只经一次初等变化得到的矩阵,也包括三种类型,常记作 $\mat E$
\begin{table}[ht]
\centering
\caption{2333}\label{EleOpt_tab1}
\begin{tabular}{|c|c|c|c|}
\hline
种类 & 效果 & 形式 & 逆 & 行列式 \\
\hline
1 & 交换两行(列) & 
$
\mat E_1=
    \begin{bmatrix}
        1 & 0 & 0\\
        0 & 0 & 1\\
        0 & 1 & 0\\
    \end{bmatrix}
$
& 
$
    \mat E_1^{-1}=\mat E_1=
    \begin{bmatrix}
        1 & 0 & 0\\
        0 & 0 & 1\\
        0 & 1 & 0\\
    \end{bmatrix}
$
& $det (\mat E_1) = -1$ \\
\hline
2 & 一行(列)扩大非零常数倍数 & 
$
    \mat E_2= \begin{bmatrix}
        1 & 0 & 0\\
        0 & c & 0\\
        0 & 0 & 1\\
    \end{bmatrix}
$
& 
$
    \mat E_2^{-1}= \begin{bmatrix}
        1 & 0 & 0\\
        0 & \frac{1}{c} & 0\\
        0 & 0 & 1\\
    \end{bmatrix}
$
& 
$det (\mat E_2) = c$
 \\
\hline
3 & 一行(列)加上另一行(列)的常数倍数 & 
$
    \mat E_3 = \begin{bmatrix}
        1 & 0 & 0\\
        0 & 1 & c\\
        0 & 0 & 1\\
    \end{bmatrix}
$
& 
$
    \mat E_3^{-1} = \begin{bmatrix}
        1 & 0 & 0\\
        0 & 1 & -c\\
        0 & 0 & 1\\
    \end{bmatrix}
$
& $det (\mat E_3) = 1$\\
\hline
\end{tabular}
\end{table}

\subsection{初等变化与初等矩阵}
\begin{theorem}{}
对矩阵进行一次初等行操作⇔相应的基本矩阵E*矩阵 (左乘)

对矩阵进行一次初等列操作⇔矩阵*相应的基本矩阵E(右乘)
\end{theorem}

\begin{example}{}
例如
\begin{equation}
\begin{bmatrix}
    a_{11} & a_{12} & a_{13}\\
    a_{31} & a_{32} & a_{33}\\
    a_{21} & a_{22} & a_{23}\\
\end{bmatrix}
=
    \begin{bmatrix}
        1 & 0 & 0\\
        0 & 0 & 1\\
        0 & 1 & 0\\
    \end{bmatrix}
*
\begin{bmatrix}
        a_{11} & a_{12} & a_{13}\\
        a_{21} & a_{22} & a_{23}\\
        a_{31} & a_{32} & a_{33}\\
\end{bmatrix}
\end{equation}
\end{example}

根据初等矩阵的性质,可以很容易得到一些结论.例如行列式的性质\upref{DetPro}中“将行列式的两列交换,结果取相反数.”\autoref{DetPro_the5}~\upref{DetPro}
$$
    \begin{bmatrix}
        1 & 0 & 0\\
        0 & 1 & 0\\
        0 & 0 & 1\\
    \end{bmatrix}
$$