% 平移算符
% keys 算符|微分方程|动量算符|波函数

\pentry{算符的指数函数\upref{OpExp}}

\subsection{一维情况}
在一维的情况下, \textbf{平移算符(translation operator)} $T(a)$ 可以把函数\footnote{这里的讨论是一般性的, 所以这里的函数不一定是量子力学中的波函数} $f(x)$ 整体向右平移 $a$ 得到 $f(x - a)$. 假设 $f(x)$ 是无穷阶可导函数, $a$ 是常数, 那么 $f(x - a)$ 关于某点 $x$ 的泰勒级数\upref{Taylor}可以用表示为\footnote{这里可以考虑设自变量$y=x-a$,将$f(y)$于点$x$处展开,得到$f(y)=f(x)+(y-x)f'(x)+1/2!(y-x)^2f''(x)+\cdots$,再把$f^{(n)}(x)$替换为$\partial^n/\partial x^n f(x)$、把$y-x$替换为$-a$得到.}
\begin{equation}
f(x - a) = \qty[1 + (-a)\pdv{x} + \frac{1}{2!} (-a)^2\pdv[2]{x} + \dots]f(x)
\end{equation}
其中方括号中的部分可以表示为一个算符的指数函数, 即
\begin{equation}
T(a) = \exp(-a\pdv{x})
\end{equation}
我们把这个算符叫做平移算符.

\begin{example}{}
令 $f(x) = x^2$, 现在我们使用平移算符将其向右平移 $a$.
\begin{equation}\ali{
\exp(-a\pdv{x}) x^2 &= \qty(1 - a\pdv{x} + \frac{1}{2!} a^2 \pdv[2]{x} \dots) x^2\\
&= x^2 - 2ax + a^2 = (x - a)^2
}\end{equation}
我们还可以再次使用平移算符,
\begin{equation}\ali{
\exp(-b\pdv{x}) (x - a)^2 &= \qty(1 - b\pdv{x} + \frac{1}{2!} b^2 \pdv[2]{x} \dots) (x - a)^2\\
&= (x - a)^2 - 2b(x - a) + b^2
= (x - a - b)^2
}\end{equation}
这就验证了 $T(b) T(a) = T(a + b)$, 即
\begin{equation}
\exp(-a\pdv{x}) \exp(-b\pdv{x}) = \exp\qty[-(a + b)\pdv{x}]
\end{equation}
\end{example}

在量子力学中, 由于(位置表象下的)动量算符为 $p = -\I\hbar \pdv*{x}$, 平移算符可记为
\begin{equation}
T(a) = \exp(- \frac{\I}{\hbar} a p)
\end{equation}

\subsection{三维情况}
将以上的一维动量算符 $p$ 换成三维的 $\bvec p = -\I \hbar \grad$ 即可
\begin{equation}
T(\bvec a) = \exp(-\bvec a \vdot \grad) = \exp(-\frac{\I}{\hbar} \bvec a \vdot \bvec p)
\end{equation}
推导同理.



\subsection{无穷小平移}

当平移的距离$\mathrm{d} \bvec{x}$趋于零的时候,平移算符的极限$T(\mathrm{d} \bvec{x})$表示为:
\begin{equation}
T(\mathrm{d}\bvec{x}) = \exp\qty({-\frac{\I}{\hbar}}\bvec{p}\cdot \mathrm{d} \bvec{x}) = 1-\frac{\I}{\hbar}\bvec{p}\cdot\mathrm{d}\bvec{x}
\end{equation}

因此动量$\bvec{p}$被认为是一个无穷小平移算符的生成元.





