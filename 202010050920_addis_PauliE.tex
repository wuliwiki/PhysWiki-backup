% 泡利不相容原理

\pentry{全同粒子\upref{IdPar}}
\footnote{参考 Wikipedia \href{https://en.wikipedia.org/wiki/Pauli_exclusion_principle}{相关词条} 和 \cite{GriffQ}.}在量子力学中, \textbf{泡利不相容原理(Pauli exclusion principle)}指的是, 在含有若干个全同费米子(即自旋为 1/2 的粒子)的系统中, 任意两个费米子不能处于同一个态.

事实上费米子对态矢的反对称性要求已经包含了泡利不相容原理, 所以该原理只是费米子性质的一个推论, 常用于解释原子的电子排布.

\begin{example}{原子壳层}
在原子壳层理论中, 如果忽略电子之间的相互作用, 那么单个电子具有一系列正交归一的能量本征态, 其中能量最低的态叫做基态, 具有唯一的空间波函数 $\psi(\bvec r)$. 如果两个电子都处于基态, 就意味着双电子波函数为 $\psi(\bvec r_1)\psi(\bvec r_2)$, 而这必定是交换对称的.

但电子是费米子, 所以好在我们还有自旋态可以补救


 如果考虑自旋, 基态最多可以容纳两个电子, 总态矢为
$$
\psi{\bvec r_1}\psi{\bvec r_2}\ket{\psi}\chi
$$
\end{example}

用矢量空间的语言来严格表述就是: 即双粒子态表示为张量积 $\ket{\psi}\ket{\psi}$. 因为根据定义 $\ket{\psi}\ket{\psi}$ 是一个交换对称的态矢, 而费米子的态矢必须是反对称的.
