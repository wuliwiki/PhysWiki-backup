% 玻色子
% license CCBYSA3
% type Wiki

(本文根据 CC-BY-SA 协议转载自原搜狗科学百科对英文维基百科的翻译)

在量子力学中,玻色子boson(/ˈboʊsɒn/,[1] /ˈboʊzɒn/[2])是遵循玻色-爱因斯坦统计的粒子;玻色子是两类粒子之一,另一类是费米子;[3] 玻色子这个名字是保罗·狄拉克(Paul Dirac)[4][5] 为了纪念与阿尔伯特·爱因斯坦(Albert Einstein)一起发展玻色-爱因斯坦统计理论(描述基本粒子性质的理论)[6]的印度物理学家、加尔各答大学和达卡大学的物理教授萨特延德拉·纳特·玻色(Satyendra Nath Bose)[7][8]而命名的。

玻色子包含基本粒子,如光子、胶子、W玻色子和Z玻色子(标准模型的四个传递力的规范玻色子)、最近发现的希格斯玻色子和量子引力理论中的引力子。一些复合粒子也是玻色子,如介子和稳定的质量数为偶数的原子核,如氘(一个质子和一个中子,原子质量数= 2)、氦-4或铅-208 以及一些准粒子(例如库珀对、等离子体激元和声子)。[9]

玻色子的一个重要特征就是玻色统计不限制占据相同量子态的玻色子的数量。氦-4被冷却成超流体时,就是这种特性的例证。[10] 与玻色子不同,两个相同的费米子不能占据同一个量子态。构成物质的基本粒子(即轻子和夸克)是费米子,而基本玻色子是力的传递者,起到将物质粘合在一起的作用。[11] 该属性适用于所有具有整数自旋的粒子($s=0,1,2$等),这是作为自旋统计定理的结果。当玻色子的气体被冷却到绝对零度附近的时候,粒子的动能减少到可以忽略的程度,并且它们凝结成最低能级状态。这种状态被称为玻色-爱因斯坦凝聚。这一性质被认为是超流现象的解释。

\subsection{类型}
玻色子可以是基本粒子,比如光子,也可以是复合粒子,比如介子。

虽然大多数玻色子是复合粒子,但在粒子物理的标准模型中,还是有五个玻色子是基本粒子:
\begin{itemize}
\item 标准模型需要(至少)一个标量玻色子 (自旋=0)
\end{itemize}
$H^0$希格斯玻色子
\begin{itemize}
\item 四个矢量玻色子 (自旋=1) 是标准模型要求的规范玻色子:
\end{itemize}
$γ$   光子
g$   胶子(八种不同类型)
Z   中性弱玻色子
W ±   带电弱玻色子(两种类型)