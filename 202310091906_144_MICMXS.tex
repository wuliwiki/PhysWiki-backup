% 混合熵的微观解释
% license CCBY4
% type Tutor


\pentry{理想气体混合的熵变\upref{IGME}}

\footnote{参考书目:Gaskel et al. Thermodynamics of Materials}
在理想气体混合的熵变\upref{IGME}我们已经知道了如何从经典的方法计算混合熵:
\begin{equation}
\Delta_{mix} S = - R \sum n_i \ln x_i~,
\end{equation}
现在,我们尝试用统计力学的观点,从微观角度论证混合熵的公式为什么是这样的。

想象我们有$N_A$个$A$球,$N_B$个$B$球,现在将他们混合在一起,那么会有几种可能的混合方式?幸运的是,统计力学假设所有微观粒子都是相同的,因此我们不需要考虑球的摆放顺序。那么,问题简化为\textsl{高中二年级}的组合问题,比如说:
\begin{itemize}
\item 共有$N_A+N_B$个格子,选择其中$N_A$个格子放置$A$球,其余$N_B$个格子放置$B$球;
\item 共有$N_A+N_B$个空白球,选择其中$N_A$个球刷成$A$球,其余$N_B$个球刷成$B$球。
\end{itemize}

问题的答案比高中课本上大多数的习题都简单,就是一个组合数:
\begin{equation}
\Omega = C^{N_A}_{N_A+N_B} C^{N_B}_{N_B} =  C^{N_A}_{N_A+N_B}  = \frac{(N_A+N_B)!}{N_A!N_B!}~,
\end{equation}
对其求对数并运用妇孺皆知的斯特林近似:
\begin{equation}
\begin{aligned}
\ln \Omega &= \ln \frac{(N_A+N_B)!}{N_A!N_B!} \\
\ln \Omega &= (N_A+N_B) (\ln (N_A+N_B) - 1) - N_A (\ln N_A - 1) - N_B (\ln N_B - 1) \\
\end{aligned}

\end{equation}