% Friedmann-Robertson-Walker (FRW) 度规
% keys 度规|尺度因子|FRW|时空弯曲|宇宙学
% license Usr
% type Tutor

\pentry{度规,张量分析}{nod_4424} % \addTODO{链接}


在宇宙学中,我们常用 \textbf{Friedmann-Robertson-Walker (FRW)} 度规,球坐标系下,其一般形式为
\begin{equation}
\mathrm{d}s^2=g_{\mu\nu}\mathrm{d}x^{\mu}\mathrm{d}x^{\nu}=-\mathrm{d}t^2+a(t)^2 \left( \frac{1}{1-kr^2}\mathrm{d}r^2+r^2 \mathrm{d} \Omega^2\right)~.
\end{equation}
其中 $t$ 为时间坐标,$r$ 为空间某一点到原点的共动距离,$\mathrm{d} \Omega^2 =\mathrm{d} \theta^2 + \sin^2\theta\mathrm{d} \phi^2 $,$k$ 对应空间的弯曲性质:当 $k=1$ 时,空间为三维球,因此常称这样的宇宙是\textbf{闭的(closed)};$k=0$时空间为\textbf{平直空间(flat)};当 $k=-1$ 时,空间为双曲面,因此常称这样的宇宙是\textbf{开放的(open)}。$a(t)$ 称为\textbf{尺度因子(scalar factor)}。
