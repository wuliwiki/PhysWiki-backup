% Möbius函数(数论)
% 莫比乌斯函数|初等数论|莫比乌斯反演公式|Möbius反演公式

\pentry{数论函数\upref{NumFun}}

德国数学家August Ferdinand Möbius于1832年提出Möbius函数的概念,这是数论中一个重要的积性函数.Möbius函数在初等数论和解析数论中随处可见,多以Möbius反演的形式出现.


任何正整数都可以唯一地分解为其质因数的幂的乘积.比如说,$24=2^3\times 3$,$300=2^3\times 3\times 5^2$.一般地,我们把整数的质因数分解记为$\prod_{k=1}^r p_k^{f_k}$,其中各$p_k$是互不相等的素数,各$f_k$都是正整数.Möbius的概念正是建立在正整数质因数分解上的:


\begin{definition}{Möbius函数}
对于任意正整数$n=\prod_{k=1}^r p_k^{f_k}$,其Möbius函数$\mu:\mathbb{Z}^+\to\{-1, 0, 1\}$定义为:
\begin{equation}
\mu(n)=\mu(\prod_{k=1}^r p_k^{f_k})=
\leftgroup{
    1, \quad\text{如果}n=1\\
    (-1)^r, \quad\text{如果}f_k=1\text{恒成立}\\
    0, \quad\text{如果有一个}f_k>1
}
\end{equation}

\end{definition}

简单来说,一个正整数的质因子中如果有幂次超过$2$的,则它的Möbius函数为$0$;其余情况,则由不同质因子数量的奇偶性决定,奇则为$-1$,偶则为$+1$.


Möbius函数有以下性质:

\begin{theorem}{Möbius的积性}\label{MbusF_the1}
给定\textbf{互素}的正整数$a, b$,则$\mu(ab)=\mu(a)\mu(b)$.
\end{theorem}

只需要检查$ab, a, b$各自的质因数即可得证\autoref{MbusF_the1} .显然,要求互素是因为,不互素时$a, b$会有公共质因数,此时$\mu(ab)=0$.





\begin{theorem}{求和性质}

\begin{equation}\label{MbusF_eq1}
\sum_{d\mid n}\mu(d)=
\leftgroup{
    &1, \quad n=1;\\
    &0, \quad n>1
}
\end{equation}

\end{theorem}

\textbf{证明}:

设$S$是$n$的全体质因子构成的集合.每个$d$都是$n$的若干质因子求积的结果,而我们只需考虑其中各质因子最多只出现一次的情况.因此,我们所考虑的$d$,和$S$的子集一一对应,即$d$就是这个子集中各素数相乘的结果.\autoref{MbusF_eq1} 极为遍历所有$S$的子集的求和.

\textbf{证毕}.
















