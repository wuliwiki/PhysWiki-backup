% 一般线性群
% keys 一般线性群|矩阵

\begin{issues}
\issueDraft
\end{issues}

\pentry{矢量空间\upref{LSpace},群\upref{Group}}

\begin{definition}{一般线性群}\label{GL_def1}
对于给定域 $\mathbb{F}$ 上的一个向量空间 $V$,$V$上的全体可逆自线性算符构成一个群,记做 $\opn{GL}(V; \mathbb{F})$(或者 $\opn{GL}_{\mathbb{F}}(V)$ ),称为 $V$ (在$\mathbb{F}$上)的\textbf{一般线性群}.

特别的,当 $V = \mathbb{F}^n$ 时, $\opn{GL}(\mathbb{F}^n; \mathbb{F})$ 可以视作可逆矩阵的群,简写为 $\opn{GL}(n; \mathbb{F})$ (或者 $\opn{GL}_n(\mathbb{F})$ ),称为 $\mathbb{F}$ 的 $n$维\textbf{一般线性(矩阵)群}.

当 $\mathbb{F} = \mathbb{R}$ 或者 $\mathbb{C}$ 时,由于 $\mathbb{R}, \mathbb{C}$ 本身是拓扑空间,$\opn{GL}(n; \mathbb{F})$ 可以视作拓扑空间(矩阵空间) $\opn{Mat}(n \times n) \cong \mathbb{F}^{n^2}$的子拓扑空间,因此 $\opn{GL}(n; \mathbb{R}), \opn{GL}(n; \mathbb{C})$ 是拓扑群\autoref{TopGrp_def1}~\upref{TopGrp},更进一步的它们也是李群\autoref{LieGrp_def1}~\upref{LieGrp}.
\end{definition}

对于线性算符 $T \in \opn{GL}(n; \mathbb{F})$(或者更一般的 $\opn{GL}(V; \mathbb{F})$), $T$ 是可逆的当且仅当它的行列式为 $0 \in \mathbb{F}$,即
$$
\opn{GL}(n; \mathbb{F}): = \{T \in \opn{Mat}(n \times n) \mid \det(T) \neq 0\}
$$

\begin{definition}{特殊线性群}
$\mathbb{F}$ 的 $n$维\textbf{特殊线性(矩阵)群},$\opn{SL}(n; \mathbb{F})$,定义为
$$
\{T \in \opn{Mat}(n \times n) \mid \det(T) = 1\}
$$
类似的我们也可以定义\textbf{特殊线性群} $\opn{SL}(V)$.
\end{definition}

因为在域 $\mathbb{F}$ 中, $0 \neq 1$,很容易看出 $\opn{SL}(n; \mathbb{F})$ 是 $\opn{GL}(n; \mathbb{F})$ 的子群.

\subsection{实/复一般线性群}

这个章节我们来重点看看 $\opn{GL}(n; \mathbb{R})$ 和 $\opn{GL}(n; \mathbb{C})$ .

$M_n(\mathbb{C})$ 是全体 $n \times n$ 复矩阵的集合.作为一个向量空间,它同构于 $\mathbb{C}^{n^2}$.作为一个有限维度实向量空间\footnote{复向量空间当然也是实向量空间},它也是一个拓扑空间,而且和 $\mathbb{C}^{n^2} \cong \mathbb{R}^{2 n^2}$ 同胚.$M_n(\mathbb{C})$ 和 $\mathbb{C}^{n^2}$ 不同的地方在于,它本身构成一个 $\mathbb{C}$-代数,即矩阵的乘法.

$\opn{GL}(n, \mathbb{C})$ 构成一个群,同时也是拓扑空间 $M_n(\mathbb{C})$ 的一个开集合(因此是个子流形\autoref{SubMnf_def1}~\upref{SubMnf},同时也是一个李群\autoref{LieGrp_def1}~\upref{LieGrp}).

$\mathbb{C}$ 可以视作一个二维的实向量空间,带上一个实线性函数$J: \mathbb{R}^2 \to \mathbb{R}^2$满足
$$
J \circ J = -\text{id}
$$
$J$ 有很多选择,比如 $J(a, b) = (-b, a)$.

$J$可以推广到 $\mathbb{R}^{2 n}$ 上,此时我们有
$$
\opn{GL}(n; \mathbb{C}): = \{A \in \opn{GL}(2 n; \mathbb{R}) \mid A J = J A \}
$$

% Giacomo: 写的不大好
% \subsection{作为李群的一般线性群}
% $\mathbb{R}$ 和 $\mathbb{C}$ 与一般的域的区别在于,
% \begin{enumerate}
% \item $\mathbb{R}$ 和 $\mathbb{C}$ 都是(Hausdorff\autoref{Topo5_def2}~\upref{Topo5})拓扑空间:因此 $\opn{GL}(n; \mathbb{R})$ 是 $\opn{GL}(n; \mathbb{C})$ 是拓扑群;
% \item 我们有实分析和复分析分别研究 $\mathbb{R}$ 和 $\mathbb{C}$ 上的微积分:因此 $\opn{GL}(n; \mathbb{R})$ 是 $\opn{GL}(n; \mathbb{C})$ 是李群;
% \item 我们在实/复向量空间上定义了很多特别的结构,见“线性代数”部分;因此我们可以定义 $\opn{GL}(n; \mathbb{R})$ 和 $\opn{GL}(n; \mathbb{C})$ 的一些特别的子群:
% \begin{itemize}
% \item 保留非退化对称张量的(广义)正交群 $\opn{O}(n; \mathbb{F})$;
% \item 保留非退化斜对称张量的辛群 $\opn{Sp}(n; \mathbb{F})$.
% \end{itemize}
% % \item $\mathbb{R}$ 和 $\mathbb{C}$ 上可以定义绝对值/模长,
% % $$
% % \begin{aligned}
% % \abs{\cdot}: \mathbb{R} &\to \mathbb{R}_{\geq 0}, x \mapsto \left\{\begin{aligned}
% % x \text{ if } x \geq 0 \\
% % -x \text{ if } x < 0 \\
% % \end{aligned} \right. \\
% % \abs{\cdot}: \mathbb{C} &\to \mathbb{R}_{\geq 0}, z \mapsto \sqrt{z \bar{z}}
% % \end{aligned}
% % $$
% % 因此我们可以定义
% \end{enumerate}

% 对于 $\mathbb{C}$ 来说,共轭(见复数\upref{CplxNo})是一个它独有的运算(没法定义在一般的域上).共轭转置被用来定义保留厄米内积的幺正群(酉群) $\opn{U}(n) \subseteq \opn{GL}(n; \mathbb{C})$.

% \addTODO{添加链接“斜对称”,“厄米内积”}
% \addTODO{在矩阵中定义共轭转置}


% \addTODO{射影群和仿射群 https://en.wikipedia.org/wiki/General_linear_group#Related_groups_and_monoids}