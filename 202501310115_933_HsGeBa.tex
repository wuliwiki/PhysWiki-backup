% 几何与解析几何初步(高中)
% keys 解析几何|初中|角度|三角|几何|初步|坐标系|点斜式
% license Xiao
% type Tutor

\begin{issues}
\issueDraft
\end{issues}

本文主要介绍一些在初中阶段已经学习过,并在高中阶段仍然具有重要作用的几何知识。这些知识不仅构成了高中几何的基础,也在解析几何、三角函数、立体几何等多个数学领域中发挥关键作用。理解和掌握这些概念,有助于更高效地解决高中数学中的几何问题,并为进一步学习提供坚实的基础。

\subsection{几何基础}

\subsubsection{基础的几何元素}

几何中最基本的用于表示位置的元素称为\textbf{点(point)}。尽管在人们的直观印象中,点似乎是一个小的黑色实心圆,但在数学中,点是一种没有大小的抽象概念,并不对应任何具体的实体,仅用于标示位置。通常用大写字母  $A, B, C$  等表示。

由无数个点按照一定顺序排列而成、没有宽度的理想化几何对象称为\textbf{线(line)}。若一条线上的所有点沿固定方向向两侧无限延伸,则称为\textbf{直线(straight line)}。在直线上引入端点截取后,可以得到以下两种特殊情况:

\begin{itemize}
\item \textbf{射线(ray)}:从某个固定点出发,并沿特定方向无限延伸的部分。
\item \textbf{线段(segment)}:由两个端点截取直线中确定的有限长度的部分。
\end{itemize}

线段的\textbf{度量(measurement)}值称为\textbf{长度(length)},它是欧氏几何中最基本的几何量之一,无法通过更基本的概念加以定义。长度可用于计算两个点之间的\textbf{距离(distance)}\footnote{注意距离和长度的区别:长度是线段的性质,而距离是两个点之间的性质。尽管在此处二者的数值相等,但并不总是如此。},并在比例关系等计算中发挥作用。将线段平分为两段等长部分的点称为其\textbf{中点(midpoint)}。

由两条射线以同一点为起点构成的几何元素称为\textbf{角(angle)}。该起点称为\textbf{顶点(vertex)},两条射线分别称为角的\textbf{边(sides)}。

角的度量值称为\textbf{角度(measure of angle)}。角的大小可以用\textbf{度(degree)}、\textbf{分(minute)}、\textbf{秒 (second)}表示,例如 $30^\circ34^{'}56^{"}$。这种表示角大小的单位制称为\textbf{角度制(Degree System)}\footnote{在角度制中,“角度”既可指角的大小,也可指表示角大小的单位,分别类似于线段中的”长度”和”米”。由于语言习惯,这两者通常都称为“角度”。}。角度制规定,\textbf{周角(full angle)}被均分为 $360$ 份,其中每一份称为 $1^\circ$。

根据角度的大小,可以将角分类如下:
\begin{itemize}
\item \textbf{锐角(acute angle)}:角度小于 $90^\circ$的角。
\item \textbf{直角(right angle)}:角度等于 $90^\circ$的角。
\item \textbf{钝角(obtuse angle)}:角度大于 $90^\circ$ 但小于 $180^\circ$的角。
\item \textbf{平角(straight angle)}:角度等于$180^\circ$的角。
\end{itemize}

对于大于 $180^\circ$ 的角,一般会采用其相对于周角另一侧的角度进行描述,以简化表达。

\subsubsection{几何元素间的关系}

在几何中,不同的元素之间存在特定的关系,这些关系决定了它们的相对位置和相互作用方式。以下是几种常见的几何关系。

如果两条直线相交形成 $90^\circ$ 角,则称它们\textbf{垂直(perpendicular)}\footnote{这个概念在向量或更高维的数学里扩展为\textbf{正交(orthogonal)}。},记作 $AB \perp CD$,表示直线 $AB$ 与直线 $CD$ 相互垂直。如果两条直线位于同一平面内,并且无论如何延长都不会相交,则称它们\textbf{平行(parallel)},记作$AB{\kern 0.56em/\kern -0.8em /\kern 0.56em}CD$或$AB \parallel CD$。

一个点到一条直线的距离,指的是该点与直线上最近点之间的长度。具体来说,从该点作一条与该直线相交并成 $90^\circ$ 角的直线,称作\textbf{垂线(perpendicular line)},交点称为\textbf{垂足(perpendicular foot)}。点到直线的距离即为该点与垂足之间的线段长度。

对于一条线段,经过该线段的中点的垂线,称作该线段的\textbf{垂直平分线(perpendicular bisector)}。任何位于垂直平分线上的点,到线段两端点的距离相等。反之,若某个点到线段两端点的距离相等,则该点一定在这条线段的垂直平分线上。

从角的顶点出发,将该角分成两个相等角的射线称作这个角的\textbf{角平分线(angle bisector)}。角平分线上任意一点到角的两边的距离相等,反之,若一个点到角的两边的距离相等,则它必定在该角的平分线上。

若两个角的和等于 $180^\circ$,则称这两个角\textbf{互补(supplementary angles)};若两个角的和等于 $90^\circ$,则称它们\textbf{互余(complementary angles)}。

\subsubsection{平行公设}

\textbf{欧几里得几何(Euclidean geometry)},或简称\textbf{欧氏几何},有一条著名的\textbf{平行公设(Parallel Postulate)},原始表述是:若一条直线与两条直线相交,并使得同侧内角之和小于 $180^\circ$,则这两条直线在该侧必相交。平行公设有多个等价的表达方式,其中包括:
\begin{itemize}
\item 平行线唯一性(Uniqueness of Parallels):在平面上,给定一条直线和直线外的一个点,至多只能作一条不与该直线相交的直线。
\item 普罗克洛斯(Proclus)公理:若一条直线与两条平行直线之一相交,则它也必与另一条直线相交。
\item 克莱因(Playfair)公理:在平面上,给定一条直线和直线外的一点,至多只能作一条平行于该直线的直线。
\item 三角形内角和公理(Triangle Angle Sum Theorem):任意三角形的内角和等于 $180^\circ$。
\end{itemize}

平行公设是欧几里得几何与\textbf{非欧几何(non-Euclidean geometry)}的核心区别。若改变平行公设,将会产生不同的几何体系,例如:

\begin{itemize}
\item \textbf{双曲几何(hyperbolic geometry)}:在该几何体系中,给定一条直线和直线外的一点,可以作无数条不与已知直线相交的直线。这种几何在广义相对论和宇宙学中具有重要应用,例如描述非平坦的时空结构。
\item \textbf{球面几何(spherical geometry)}:在该几何体系中,不存在不相交的直线。例如,在球面上,所有“直线”都是\textbf{大圆(great circles)},它们总会相交。因此,在球面几何中,传统的平行概念不成立。这种几何在导航和天文学中具有重要意义,例如用于描述地球表面的测量和飞行路线的计算。
\end{itemize}

尽管非欧几何在现代数学和物理学中具有深远影响,但高中阶段所接触的几何仍然基于欧几里得几何,此处仅作简要介绍,以帮助理解几何体系的多样性。

\subsection{三角形}

\textbf{三角形(triangle)}是最简单的平面几何形状,它由三条线段首尾相接围成,其基本性质构成了欧几里得几何的重要部分。由于高中阶段研究的都是欧几里得几何,根据前面提及的平行公设的等价的表达方式,任意三角形的内角和等于 $180^\circ$。因此,三角形的内角不会大于$180^\circ$,同时至多有一个内角大于等于$90^\circ$。

\subsubsection{基础概念}

从三角形的一个顶点出发,向其对边所在直线作的垂线称为三角形的\textbf{高(altitude)}。高所对的边称作这个高所对的\textbf{底(base)}。三条高所在的直线交于一点,称作\textbf{垂心(orthocenter)}。垂足可能在边上、与顶点重合或在边的延长线上,所以高可能位于三角形内部、与边重合或在外部。对应地,垂心在锐角三角形内部,直角三角形的直角顶点处,钝角三角形外部。

顶点与对边中点的连线称为\textbf{中线(median)}。三角形的三条中线交于一点,称作\textbf{重心(centroid,或质心)},重心将每条中线分为 $2:1$ 的比例(靠近顶点的一段较长)。

若三角形至少有两条边相等,则称为\textbf{等腰三角形(isosceles triangle)}。相等的两边称为\textbf{腰(legs)},它们所夹的角称为\textbf{顶角(vertex angle)},其余两个角称为\textbf{底角(base angles)}。顶角所对的边称为\textbf{底(base)},底边上的高同时是顶角的平分线。在等腰三角形的基础上,若其中一个内角为 $60^\circ$,或两腰与底的长度相等,则该三角形为\textbf{等边三角形(equilateral triangle)},此时三角形的三条边相等,或三个角均为 $60^\circ$。

\subsubsection{三角形的关系}

两个三角形\textbf{相似(similarity)},意味着它们的对应角相等,对应边的长度成比例。这种关系可以类比于将一个三角形按某个比例放大或缩小——尽管大小发生了变化,但整体的形状保持不变。常见的相似判定方法包括:
\begin{itemize}
\item 两个对应角相等;
\item 两边的长度成比例,且夹角相等;
\item 三边的长度成比例。
\end{itemize}

三角形的\textbf{全等(congruence)}是相似的特殊情况,此时对应边的比例系数为 $1$。这意味着,除了位置可能不同之外,它们的形状和大小完全相同,对应角相等,对应边的长度也完全相等。可以将其理解为,一个三角形可以通过平移、旋转或翻折与另一个完全重合。

\subsubsection{直角三角形与三角函数}

一个包含直角的三角形称为\textbf{直角三角形(right triangle)}。在直角三角形中,每条边的命名与所考察的角 $\theta$ 相关:
\begin{itemize}
\item 直角所对的最长边称为\textbf{斜边(hypotenuse)},它在所有直角三角形中都是最长的边。
\item 角 $\theta$ 所在的边(但不包括斜边)称作\textbf{邻边(adjacent)}。
\item 角 $\theta$ 对面的边称作\textbf{对边(opposite)}。
\end{itemize}

\begin{theorem}{勾股定理}
在直角三角形中,三条边的长度满足\textbf{勾股定理(Pythagorean Theorem)},即两条直角边的平方和等于斜边的平方:
\begin{equation}
a^2 + b^2 = c^2~.
\end{equation}
其中,$a$ 和 $b$ 是直角三角形的两条直角边,$c$ 是斜边。
\end{theorem}

勾股定理既是直角三角形的必要条件,也是充分条件。勾股定理不仅描述了直角三角形的性质,还提供了一种判定三角形是否为直角三角形的方法。若已知三角形的三条边长,检验它们是否满足勾股定理,若成立,则该三角形必为直角三角形;反之,则一定不是直角三角形。

在整数范围内,有一些满足勾股定理的边长组合,被称为\textbf{勾股数(Pythagorean triple)}。最常见的勾股数组合包括$(3,4,5)$,$(5,12,13)$,$(7,24,25)$等。他们都满足一个一般的公式:
\begin{equation}
(m^2 - n^2, 2mn, m^2 + n^2),\qquad(m>n)~.
\end{equation}
它可以构造所有的原始勾股数(即最大公因数为 1 的勾股数)。之前给出的例子分分别对应$(m=2,n=1)$、$(m=3,n=2)$以及$(m=4,n=3)$的情况。

\textbf{三角函数(Trigonometric Functions)}描述了直角三角形中角度与边长之间的关系,是几何和数学分析中的核心概念。常见的三角函数包括\textbf{正弦(sine)}、\textbf{余弦(cosine)}和\textbf{正切(tangent)},分别记作 $\sin\theta$,$\cos\theta$,$\tan\theta$。

设直角三角形 $ABC$,其中角 $C$ 为直角,设角 $A$ 对应的对边、邻边和斜边分别为 $|BC|, |AC|, |AB|$,则角 $A$ 的三角函数定义如下:
\begin{equation}
\sin A = \frac{|BC|}{|AB|}, \quad
\cos A = \frac{|AC|}{|AB|}, \quad
\tan A = \frac{|BC|}{|AC|}~.
\end{equation}

三角函数的定义可以形象地理解为,在一定比例缩放下,直角三角形的形状保持不变,而三角函数值反映了角度和边长比例的稳定关系。这也是三角函数在几何测量、建筑设计、物理学等领域广泛应用的原因。下面给出一些常见的三角函数值。

\begin{table}[ht]
\centering
\caption{常见三角函数值}\label{tab_HsGeBa1}
\begin{tabular}{|c|c|c|c|}
\hline
角$\alpha$ & $30^{\circ}$ & $45^{\circ}$ & $60^{\circ}$ \\
\hline
$\sin\alpha$ & $\displaystyle\frac{1}{2}$ & $\displaystyle\frac{\sqrt{2}}{2}$ & $\displaystyle\frac{\sqrt{3}}{2}$ \\
\hline
$\cos\alpha$ & $\displaystyle\frac{\sqrt{3}}{2}$& $\displaystyle\frac{\sqrt{2}}{2}$ &  $\displaystyle\frac{1}{2}$ \\
\hline
$\tan\alpha$ & $\displaystyle\frac{1}{\sqrt{3}}$ & $1$ & $\sqrt{3}$ \\
\hline
\end{tabular}
\end{table}

\subsection{圆}\label{sub_HsGeBa_1}

\textbf{圆(circle)}是平面内与某个定点距离等于定值的所有点组成的图形。其中,该定点称为\textbf{圆心(center)},通常记作 $O$,并以圆心指代圆,称为圆 $O$。定值称为\textbf{半径(radius)},通常记作 $r$。

对于圆上任意不重合的两点 $A$ 和 $B$,有以下概念:
\begin{itemize}
\item \textbf{弦(chord)}是连接圆上任意两点的线段 $AB$。圆心到弦的垂线是该弦的垂直平分线。
\item \textbf{弧(arc)}是圆上两点之间的弧线。两点 $A$ 和 $B$ 将圆分成两条弧,分别称为\textbf{劣弧(minor arc)}和\textbf{优弧(major arc)},其中劣弧的长度不超过半个圆周,优弧的长度大于半个圆周。
\item 顶点位于圆心,且两边与圆相交的角称为\textbf{圆心角(central angle)}。同样,一般包括一个小于$180^\circ$的角和一个大于$180^\circ$的角,二者分别与他们角内夹的弧对应\footnote{有的人分别称这两种为劣角和优角,以求和弧对应,但使用不广泛。劣角在英语中无对应词,优角对应的词为“Reflex Angle”,直译是反射角,对应平面反射的情况。}。一个圆周对应的圆心角为 $360^\circ$,因此也称其为\textbf{周角(perigon)}。
\end{itemize}

在未特别说明的情况下,通常 $\overset{\frown}{AB}$ 表示两点 $A$ 和 $B$ 之间的劣弧。此时,两点 $A$ 和 $B$ 可以唯一确定一条弦 $AB$,一条$\overset{\frown}{AB}$ 和一个圆心角$\angle AOB$。因此可以称他们三者对应,如:$\overset{\frown}{AB}$ 称为弦 $AB$ 所对应的弧,$\angle AOB$是$AB$所对应的圆心角等\footnote{如果有特殊情况,则不一一对应,一条弦对应两段弧,但每个弧仍对应一个圆心角。}。

对于圆上任意不重合的三点 $A,B,C$,有以下概念:
\begin{itemize}
\item 顶点位于圆上,两边与圆相交的角称作\textbf{圆周角(inscribed angle)}。设顶点为$C$,则圆周角$\angle ACB$所夹圆弧$\overset{\frown}{AB}$称作其对应的弧\footnote{“所夹圆弧”的意思是指,由$A,B$两点截得,且顶点$C$不在的圆弧。},同弧对应的圆心角是圆周角的两倍,即$\angle AOB=2\angle ACB$。
\item 若三角形以 $A$、$B$、$C$ 为顶点,则称为圆的\textbf{内接三角形(inscribed triangle)},圆称为三角形的\textbf{外接圆(circumcircle)}。三角形的三个内角就是外接圆的三个圆周角,三角形的\textbf{外心(circumcenter)}也就是外接圆的圆心。三条边的垂直平分线交于外心,它到三个顶点的距离相等。
\item 若三角形的三边分别与圆相切于 $A$、$B$、$C$,则称为圆的\textbf{外切三角形(circumscribed triangle)},圆称为三角形的\textbf{内切圆(incircle)}。三角形的\textbf{内心(incenter)}也就是内切圆的圆心。三个内角的角平分线交于内心,它到三边的距离相等。
\item 外接三角形某一边及其余两边延长线的圆,称作三角形\textbf{旁切圆(excircle)},旁切圆的圆心称作\textbf{旁心(excenter)}\footnote{这里介绍的外心、内心、旁心与之前提到的垂心和重心称作三角形的“五心”。}。一个三角形有三个旁心,每个旁心对应一个旁切圆。
\end{itemize}

通过圆心的弦称为圆的\textbf{直径(diameter)},通常记作$d$,满足:
\begin{equation}
d = 2r~.
\end{equation}
此时,所对弧称为\textbf{半圆(semicircle)},所对圆心角为平角,即$180^\circ$,所对圆周角为直角,即$90^\circ$。因此,由直径参与构成的圆内接三角形一定是直角三角形。

圆的周长与直径之比是一个定值,其值等于数学常数$\pi$,因此也称作\textbf{圆周率}\footnote{这里并没有采用通常教材中的“圆周长与直径之比定义为圆周率”的表述。现代数学为了避免循环论证,采用其他方法定义$\pi$,之后再得到“圆的周长与直径之比是一个定值”的结论。}。$\pi$是数学中极为重要的常数,它是一个无理数,近似值为 $\pi \approx 3.14159$。

对半径为$r$圆有,其周长:
\begin{equation}
C = \pi d=2 \pi r~. 
\end{equation}

面积:
\begin{equation}
S = \pi r^2~.
\end{equation}


\subsection{解析几何基础}

\subsubsection{坐标系}
将所有的实数和直线上的点一一对应,就形成了\textbf{数轴(number line)}。。数轴的定义基于一个确定的原点、单位长度和正方向,这三个因素唯一地确定了数轴在几何中的位置和方向。法国数学家勒内·笛卡尔(René Descartes)在数学研究中,将两条数轴的原点重叠,并将其正交(即相互垂直)放置,创造了\textbf{坐标系(coordinate system)}。这就是初中阶段学习过的\textbf{笛卡尔坐标系(Cartesian coordinate system)},也称为\textbf{直角坐标系(rectangular coordinate system)}。

引入坐标系后,平面上的任何一点都可以通过一个\enref{有序数对}{CartPr} $(x, y)$ 来表示。借助这种表示法,几何形状可以通过数对来分析和研究,这一方式称为\enref{解析几何}{JXJH}。而当数对中的值对应于函数的变量及其结果时,几何图形就成为了函数的图像。因此,坐标系不仅为函数的图像提供了清晰的视觉表达,还使得人们可以通过几何图形直观地观察函数的性质,例如其变化趋势、最大值和最小值等。

通常,直角坐标系中,两条数轴称为$x$轴和$y$轴,且向右的方向为$x$轴的正方向,向上为$y$轴的正方向。数轴将平面分为四个区域,称为\textbf{象限(quadrant)}。其中,第一象限是两个坐标都为正的区域,之后按逆时针方向依次为第二、第三和第四象限。

\subsubsection{常见表达式}

\begin{definition}{直线的点斜式}\label{def_HsGeBa_1}
经过点$(x_0,y_0)$,且斜率为$k$点直线表达式为:
\begin{equation}
y-y_0=k(x-x_0)~.
\end{equation}
\end{definition}

