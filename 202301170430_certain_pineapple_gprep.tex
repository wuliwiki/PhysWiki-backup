% 群的矩阵表示及实例
% 群表示|矩阵表示|群线性表示|忠实表示|正则表示
\pentry{群表示\upref{GrpRep}}
\begin{issues}
\issueDraft
\end{issues}
在群表示一节中曾提到群元可与线性变换建立同态关系,由此可以给出群的线性表示,在选取合适的基后,可以将线性变换写成矩阵的形式。

\subsection{正则表示}

一个较为简单的线性空间基的取法是直接以群元为基,有$v_i=g_i$

