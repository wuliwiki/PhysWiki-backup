% 浙江大学 2007 年 考研 量子力学
% license Usr
% type Note

\textbf{声明}:“该内容来源于网络公开资料,不保证真实性,如有侵权请联系管理员”

\subsection{第一题(50 分)简答题:}
\begin{enumerate}
  \item 写出泡利矩阵的形式。
  \item 量子力学中的可观察量算符为什么要求是厄米算符?
  \item 放射性指的是某些原子核中的更小粒子有一定的概率逃逸出来,你认为这与什么量子效应有关?
  \item 试求质量为 $m$ 的粒子处在长度为 $L$ 的一维盒子(可看成是无限深势阱)中,试求他对各壁的压力。
  \item 自发辐射和受激辐射的区别是什么?
  \item 写出测测不准关系,并简要说明其物理含义。
  \item 请分别列出下列三种能级对应的量子系统:
  \[    E_n' = \frac{1}{n^2}, \quad E_n'' \propto \frac{1}{n}, \quad E_n''' \propto n^2 ~\]
  \item $\hat{H} = \hat{H_0} + \hat{H'}$,设 $\psi$ 为 $\hat{H_0}$ 的能量本征值为 $E$ 的非简并本征函数,如果 $\hat{H'}$ 可作微扰,试写出能级的微扰修正公式(写到二级修正)。
\end{enumerate}