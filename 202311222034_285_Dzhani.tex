% 网球拍定理(科普)
% keys 网球拍定理|旋转|转动惯量|Tennis racket theorem|贾尼别科夫效应|Dzhanibekov Effect
% license Usr
% type Art

\pentry{角动量(科普)\upref{AngMo}}



\subsection{现象描述}


\textbf{中间轴定理},又称\textbf{网球拍定理}或\textbf{贾尼别科夫效应(Dzhanibekov Effect)},收录于法国数学家、物理学家潘索(Louis Poinsot)于1834年出版的\textsl{Théorie Nouvelle de la Rotation des Corps}(《转动物体的新理论》),并由苏联航天员贾尼别科夫(Vladimir Dzhanibekov)在1985年观测到实验证据。


称之为“网球拍”定理,是因为抛掷网球拍的时候能观察到相应的现象,但我们在家里用一本书或者一部手机就能完成这个实验。以手机为例,将手机正常握持,然后抛向空中(请在安全环境下做这个实验,如在床上,以防手机摔坏),使得手机绕如\autoref{fig_Dzhani_2} \textbf{左边}所示的轴在空中旋转一周后落回手中。绝大多数情况下,落回手中的手机不再是正面朝上,而是背面朝上。再抛掷一次,手机旋转一周后落回手中,又会变成正面朝上。


%\begin{figure}[ht]
%\centering
%\includegraphics[width=7cm]{./figures/edb6af44a3d38d05.pdf}
%\caption{网球拍定理示意图。图中手机是正面朝上。将手机抛向空中,使得手机绕图中虚线轴旋转一周后落回手中,则通常手机变成背面朝上。} \label{fig_Dzhani_1}
%\end{figure}



\begin{figure}[ht]
\centering
\includegraphics[width=12cm]{./figures/28ef0f94d7aeadab.pdf}
\caption{网球拍定理示意图。图中手机是正面朝上。将手机抛向空中,使得手机绕图中\textbf{左边}的虚线轴旋转一周后落回手中,则通常手机变成背面朝上。} \label{fig_Dzhani_2}
\end{figure}

仔细观察抛掷过程会发现,扔手机的时候手给手机施加的力不对称,使得手机在空中也会绕如\autoref{fig_Dzhani_2} 右边的轴旋转,这才造成了最终的翻转。如果你抛掷后手机只转了半圈就落回来,那么也很容易造成一次翻转,使得手机仍然正面朝上,但上下颠倒了。即便是双手握住手机,尽可能对称地用力,仍难以避免手机旋转的同时绕着\autoref{fig_Dzhani_2} 右边的轴翻转。


贾尼别科夫在太空中观察到的现象则更为明显。如\autoref{fig_Dzhani_3} 所示,蝴蝶形螺帽的两个“翅膀”是为了方便手动上紧或松开螺帽设计的。快速松开螺帽的时候,螺帽脱离螺丝后仍然会绕着图中的红色虚线旋转。在地球上,螺帽飞出后会直接落地,因而很难观察到特别的现象;但在太空中,由于螺帽脱离后能在空中保持旋转姿态,从而能观察到奇特的现象:螺帽看似稳定地旋转一会儿后,会突然翻转,再继续稳定地旋转后又翻转回来,如此反复。


具体的翻转效果见\autoref{fig_Dzhani_3} 的描述,或参见\href{https://www.bilibili.com/video/BV1jy4y1Y7CU/?share_source=copy_web&vd_source=5d336c88ab0583d823dccd3a9651b27b}{萌萌战队的视频}和\href{https://www.bilibili.com/video/BV12K411S7Gc/?spm_id_from=333.337.search-card.all.click&vd_source=0ae31f148e26bb548391e861d5f13610}{帆雨动画的视频}。


\begin{figure}[ht]
\centering
\includegraphics[width=6cm]{./figures/074d90d4f05d77fc.pdf}
\caption{贾尼别科夫的蝴蝶形螺帽实验示意图。图中是一个蝴蝶形螺帽,两个“翅膀”朝上,让螺帽绕着红色虚线旋转,则螺帽会在旋转时突然翻转,变成翅膀朝下,继续旋转,过一会儿又突然翻转,变成翅膀朝上,如此循环往复。} \label{fig_Dzhani_3}
\end{figure}



抛掷网球拍、手机时的翻转效果,和贾尼别科夫观察到的螺帽旋转过程中周期性的翻转,是源于相同的机制,这便是本节要介绍的贾尼别科夫效应。为了理解这一现象,我们需要铺垫一些关于旋转的知识。





\subsection{角动量}


\pentry{矢量叉乘\upref{Cross}}

本节介绍角动量的概念,已经熟悉此概念的读者可以跳过。


\subsubsection{提出角动量概念的动机}


我们知道,一个系统不受外力的时候,总动量是守恒的。尽管系统中的若干质点之间可能相互作用,导致各质点的动量改变,但总量依然守恒。

这是因为按照动量的定义,每个质点的动量是$m\bvec{v}$,其中$m$是质点的质量,$\bvec{v}$是它此刻的速度。在经典力学中,质点的质量不变,那么引起动量改变的只能是速度的变化,因此动量的改变率正是质量乘以速度的改变率,即$m\bvec{a}$,其中$\bvec{a}$是速度的改变率,即加速度。根据\textbf{牛顿第二定律},$m\bvec{a}$正是质点所受的外力。由\textbf{牛顿第三定律},两个质点相互作用的时候,彼此所受来自对方的力刚好方向相反,大小相同,因此它们给彼此造成的动量改变速率也相同,但方向相反,正好互相抵消,从而总动量的改变速率为$0$。对更多质点也是如此。


动量守恒为解决问题带来了很多便利。于是,在研究转动的时候,我们也希望找到一种和旋转有关的不变量,这样才有可能方便讨论问题。


和平动不一样,转动依赖于参考点,即“绕某点旋转”或者“绕某轴旋转”。因此,和旋转有关的不变量也应该以参考点来定义。


\begin{figure}[ht]
\centering
\includegraphics[width=14cm]{./figures/8ba4779afcd1aba8.pdf}
\caption{角动量的动机示意图。图中点$X$做匀速直线运动,$P$是参考点。$X$和$P$的连线,在相同的两个时间间隔内扫过的面积相同。} \label{fig_Dzhani_5}
\end{figure}


如\autoref{fig_Dzhani_5} ,点$X$不受外力,做匀速直线运动。不管是什么不变量,我们至少希望它在不受外力的情况下不变。

以点$P$为参考点,$X$和$P$的连线随着$X$的运动,在一定的时间间隔内扫过一个三角形面积。图中面积$S_1$和$S_2$所对应的\textbf{时间间隔相同},则\textbf{两个面积相同}。这是因为$X$做匀速直线运动,相同时间内运动的距离相同,即两个三角形的底边长相同;两个三角形的高都是点$P$到$X$的运动所在直线的垂线段的高,因而也相同。



换个视角看,则是当$X$运动时,连线$PX$扫过的面积逐渐增长,但这个\textbf{面积的增长速率}是不变的。由于扫过的三角形的高是不变的,因此增长速率由底边长的增长速率决定,而这个增长速率恰为点$X$的运动速率。因此,面积的增长速率为$rv$,其中$r$是$P$到$X$的运动直线的垂线段长度,$v$是$X$的运动速率。


这个不变量,就是定义角动量的动机。



\subsubsection{角动量的概念}





表示扫过面积的增长速率还有一个办法:$\bvec{r}\times \bvec{v}$,其中$\bvec{r}$是以点$P$为起点、点$X$为终点的\textbf{几何向量}\upref{GVec},$\bvec{v}$是$X$的速度。由此,我们就可以简洁地定义角动量了。


\begin{definition}{角动量}

给定参考点$P$和一个质点。若质点的质量为$m$,则当点$P$到质点的位移向量是$\bvec{r}$、质点的运动速度是$\bvec{v}$时,质点\textbf{关于参考点}$P$的\textbf{角动量(angular momentum)}是一个\textbf{向量},定义为
\begin{equation}
\bvec{L} = m\bvec{r}\times \bvec{v}~, 
\end{equation}
或等价地,
\begin{equation}
\bvec{L} = \bvec{r}\times\bvec{p}~, 
\end{equation}
其中$\bvec{p}$是质点的\textbf{动量}。

若一个系统由若干质点组成,则所有质点关于$P$的角动量之和,就是系统的角动量。

\end{definition}



角动量相当于“旋转版本的动量”,它刻画了物体关于参考点的旋转运动状态。角动量是一个\textbf{向量},它描绘了物体绕参考点旋转的方向,可以用\textbf{右手定则}来记忆。伸出你的右手,四指握拳,大拇指竖直向上,比出“点赞”的手势,则一个沿着四指弯曲方向旋转的质点的角动量之方向,就是大拇指指向的方向。

自行车向前运动时,其轮子上某处质点关于轮轴中心的角动量方向是垂直轮盘向左的;地球自西向东自转,则赤道上的质点关于\textbf{地心}的角动量是沿着自转轴向北的;但不在赤道上的质点,其关于地心的角动量就不是沿着自转轴向北的,而是和地球表面\textbf{相切},如\autoref{fig_Dzhani_6} 所示。


\begin{figure}[ht]
\centering
\includegraphics[width=10cm]{./figures/d0390d410dd5b5bf.pdf}
\caption{地球上随地球自转的质点关于地心的角动量示意图。点$Q$在赤道上,其角动量方向沿着地球自转轴向北;点$P$不在赤道上,其角动量方向就和自转轴方向不平行。但无论何种情况,质点的角动量都和地面相切,且与地球自转轴在同一平面。} \label{fig_Dzhani_6}
\end{figure}





\subsubsection{参考点与参考轴}



旋转既可以关于参考点定义,也可以关于参考轴定义,定义式都是$\bvec{L}=\bvec{r}\times \bvec{p}$。只不过关于参考点定义时,$\bvec{r}$是参考点到质点的位移向量;而关于参考轴定义时,是过质点位置画一条到参考轴的垂线段,垂点到质点位置的的位移向量就是$\bvec{r}$。


两种定义下的$\bvec{r}$之间的关系也可以描述为“投影”。设有参考点$P$和参考轴$S$,且$P$在$S$上;质点位置为$X$,参考点到质点的位移向量为$\bvec{r}_1$,参考轴到质点的位移向量为$\bvec{r}_2$,则$\bvec{r}_2$是$\bvec{r}_1$在$S$的\textbf{正交补空间}上的投影,如\autoref{fig_Dzhani_8} 所示。



\begin{figure}[ht]
\centering
\includegraphics[width=6cm]{./figures/6cad6c088832ce7e.pdf}
\caption{关于参考点和参考轴的位移向量示意图。$P$是参考点,$S$是参考轴,$X$是质点的位置,$\bvec{r}_1$是参考点到质点的位移向量,$\bvec{r}_2$是参考轴到质点的位移向量。取$S$的正交补空间,即垂直于$S$的一个平面,则$\bvec{r}_1$在该平面上的投影就是$\bvec{r}_2$。} \label{fig_Dzhani_8}
\end{figure}



观察后发现,关于参考轴定义的角动量,其方向总是平行于参考轴。事实上,关于参考轴定义的角动量是关于参考点定义的角动量在参考轴上的投影,它实际上可以看成是一个标量而非向量。


换言之,关于参考轴定义的角动量,是关于参考点定义的角动量的分量。因此,我们默认角动量是关于参考点定义的。






\subsection{角速度与转动惯量}

本节介绍角速度与转动惯量,熟悉它们的读者同样可以跳过,下一节即为贾尼别科夫效应的原理阐释。

\subsubsection{角速度}


角速度即质点绕参考点旋转的“速度”,它是一个向量,其大小为质点与参考点之间连线的角度变化率。

角速度可以看成是“旋转版本的速度”,它只和质点的切向速度有关,而和法向速度\footnote{设$\bvec{r}$是参考点到质点的位移向量,则切向速度是质点速度垂直于$\bvec{r}$的分量,法向速度是质点速度平行于$\bvec{r}$的分量。}无关。\autoref{fig_Dzhani_5} 中的质点$X$也有关于参考点$P$的角速度;尽管其速度是不变的,角速度却一直在变。

\begin{definition}{角速度}\label{def_Dzhani_1}
给定参考点$P$和一个质点$X$,令$P$到$X$的位移向量为$\bvec{r}$,$X$的速度为$\bvec{v}$。

记$X$关于$P$的角速度为$\bvec{\omega}$,它是一个\textbf{向量},其大小和方向定义如下:当$X$运动时,$\bvec{r}$的角度随时间的变化率即为$X$关于$P$的角速度的大小,其方向则和$\bvec{r}\times \bvec{v}$相同。
\end{definition}

由定义可见,角速度的方向同样由右手定则决定:四指弯曲的方向是质点的速度方向,大拇指指向的则是质点的角速度方向。



\begin{figure}[ht]
\centering
\includegraphics[width=10cm]{./figures/5b10b35d4efbb979.pdf}
\caption{角速度示意图。图中$X$是运动的质点,$P$是参考点,黄色的$\bvec{v}$是$X$的速度。以$P$为圆心过$X$画一个圆,则$\bvec{v}$关于$P$的切向投影为$\bvec{v}_t$,它切于这个圆;法向投影为$\bvec{v}_n$,它平行于位移向量$\bvec{r}$。质点运动一小段时间后,位移向量变为$\bvec{v}+\Delta\bvec{v}$,它与原本位移$\bvec{r}$的夹角为$\theta$。} \label{fig_Dzhani_10}
\end{figure}



计算角速度大小的方式如\autoref{fig_Dzhani_10} 所示。在极短时间内可以认为,影响位移向量角度变化的是切向速度$\bvec{v}_t$。在一小段时间$\Delta t$后,质点$X$运动到了$\bvec{r}+\Delta \bvec{r}$的位置,$\bvec{r}$和$\bvec{r}+\Delta \bvec{r}$之间的夹角为
\begin{equation}
\theta = \frac{\abs{\bvec{v}_t \Delta t}}{\abs{\bvec{r}}}~. 
\end{equation}
由此可知,角速度的大小(位移向量的角度随时间的变化率)为$\theta/\Delta t = \frac{\abs{\bvec{v}_t}}{\abs{\bvec{r}}}$。

因此,$\abs{\bvec{\omega}}\cdot\abs{\bvec{r}}=\bvec{v}$。

又根据\autoref{def_Dzhani_1} 中对角速度方向的描述,\autoref{fig_Dzhani_10} 中的角速度方向垂直纸面向外(指向读者),由此可得
\begin{equation}
\bvec{\omega}\times\bvec{r} = \bvec{v}~, 
\end{equation}
以及
\begin{equation}
\bvec{\omega} = \frac{\bvec{r}\times\bvec{v}}{\abs{\bvec{v}}^2}~. 
\end{equation}

由于$\bvec{r}$必定和$\bvec{\omega}$正交,因此从数值上来说,有$\abs{\bvec{\omega}}\abs{\bvec{r}}=\abs{\bvec{v}_t}$,其中$\bvec{v}_t$是质点关于参考点的\textbf{切向速度}。类似地,角动量的大小$\abs{\bvec{L}}=m\abs{\bvec{r}}\abs{\bvec{v}_t}$。这些数值可以协助我们理解和推导旋转相关的性质。



最后要特别注意的是,角速度通常关于参考轴而不是参考点来定义,因为角速度之间不能相加\footnote{正如速度不能相加,如我和你赛跑,把我的速度和你的速度加起来所得结果是毫无意义的。},但讨论刚体\footnote{刚体即“质点之间的距离保持不变”的质点系。}的时候可以找到一个参考轴,使得刚体的所有质点关于该参考轴的角速度相同,我们就说这个角速度是刚体的角速度。



\subsubsection{刚体的转动惯量}



物体的惯性由质量来体现,质量越大的物体越难改变速度。


在研究转动的时候,我们已经分别将物体的\textbf{速度}、\textbf{动量}类比为\textbf{角速度}、\textbf{角动量}了,自然希望继续类比得到描绘转动惯性的量,我们称之为\textbf{转动惯量}。我们只讨论刚体的转动惯量。



给定参考点$P$和一个质点$X$,令$P$到$X$的位移向量为$\bvec{r}$,$X$关于$P$的角速度为$\bvec{\omega}$,质量为$m$。则可以计算得$X$关于$P$的角动量为\footnote{设$\abs{\bvec{v}_t}=v_t$、$\abs{\bvec{r}}=r$和$\abs{\bvec{\omega}}=\omega$,则考虑到这三个向量彼此正交,易得数值上的关系:$\abs{\bvec{L}}=mrv_t=m r^2 \omega$。再分析方向即可得证。}
\begin{equation}
\bvec{L} = m\abs{\bvec{r}}^2\bvec{\omega}~. 
\end{equation}
这个关系式和动量与速度的关系式很像:
\begin{equation}
\bvec{p} = m\bvec{v}~. 
\end{equation}
因此,作为类比,我们将$I=m\abs{\bvec{r}}^2$定义为质点$X$关于$P$的\textbf{转动惯量}。



\begin{definition}{转动惯量}
给定参考点$P$和一个质点$X$,令$P$到$X$的位移向量为$\bvec{r}$,$X$的质量为$m$,则定义$X$关于$P$的转动惯量为
\begin{equation}
I=m\abs{\bvec{r}}^2~. 
\end{equation}

对于质点系,所有质点的转动惯量之和为该质点系的转动惯量。
\end{definition}


和质量不同,转动惯量和单个质点的位置有关。对于质点系,比如刚体,哪怕将参考点选为质点系的质心,转动惯量也会和旋转轴的方向有关。为了方便,我们用最简单的模型\autoref{ex_Dzhani_1} 来说明这一点。



\begin{example}{转动惯量的计算(主轴)}\label{ex_Dzhani_1}


\begin{figure}[ht]
\centering
\includegraphics[width=10cm]{./figures/fc3893ad00044a9c.pdf}
\caption{转动惯量示意图。图中是一个轻质圆盘,其圆心规定为原点,建立直角坐标系,使得圆盘在$x$-$y$平面上。圆盘半径为$r$,在$(\pm r, 0, 0)$处各固定一个质量为$M$的质点,在$(0, \pm r, 0)$处则各固定这一个质量为$m$的质点。} \label{fig_Dzhani_12}
\end{figure}


如\autoref{fig_Dzhani_12} ,如果圆盘绕着$z$轴方向转动,或者说其角速度$\bvec{\omega}_z$的方向为$z$轴正方向或者负方向,那么圆盘的角动量应为
\begin{equation}
\begin{aligned}
\bvec{L} ={}& Mr^2\bvec{\omega}_z+Mr^2\bvec{\omega}_z+mr^2\bvec{\omega}_z+mr^2\bvec{\omega}_z\\
={}& 2r^2(M+m)\bvec{\omega}_z ~.
\end{aligned}
\end{equation}
此时计算出的转动惯量应为$I_z=2r^2(M+m)$。

如果圆盘的角速度$\bvec{\omega}_x$沿着$x$的方向,那么角动量应为
\begin{equation}
\begin{aligned}
\bvec{L} ={}& mr^2\bvec{\omega}_x+mr^2\bvec{\omega}_x\\
={}& 2r^2m\bvec{\omega}_x ~.
\end{aligned}
\end{equation}
此时计算出的转动惯量应为$I_x=2r^2m$。

如果圆盘的角速度$\bvec{\omega}_y$沿着$y$的方向,那么角动量应为
\begin{equation}
\begin{aligned}
\bvec{L} ={}& Mr^2\bvec{\omega}_y+Mr^2\bvec{\omega}_y\\
={}& 2r^2m\bvec{\omega}_y ~.
\end{aligned}
\end{equation}
此时计算出的转动惯量应为$I_y=2r^2M$。




\end{example}



\autoref{ex_Dzhani_1} 中,三个方向计算出的转动惯量大小不同。这说明转动惯量和质量不同,它通常不是一个标量。


例子中计算的角动量和角速度方向相同,我们称这些方向为刚体的\textbf{转动主轴}。而在大多数旋转方向中,角动量的方向和角速度的方向不同,如\autoref{ex_Dzhani_2} 所示。


\begin{example}{转动惯量的计算(非主轴)}\label{ex_Dzhani_2}


\begin{figure}[ht]
\centering
\includegraphics[width=8cm]{./figures/0a9a972815ac3d3b.pdf}
\caption{转动惯量示意图。图中是一个轻质杆,其两端各固定着一个质量为$m$的质点。某一时刻,上面那个质点的速度方向垂直纸面向内,下面质点的速度方向则垂直纸面向外,即此刻轻质杆绕图中虚线轴旋转。参考点是旋转轴和轻质杆的交点,橙色的$\bvec{L}$表示各质点关于参考点的角动量向量,蓝色的$\bvec{F}$则表示各质点在旋转参考系中所受的离心力向量。} \label{fig_Dzhani_13}
\end{figure}

如\autoref{fig_Dzhani_13} ,轻质杆绕轴转动时,两个质点关于参考点的角动量向量方向都指向左上方$45^\circ$,因此总角动量(即这两个向量的和)也指向左上方。但整个系统的旋转轴是竖直的,即刚体本身的角速度是竖直向上的,故此时刚体的角动量和角速度方向不一致。


\end{example}


\autoref{ex_Dzhani_2} 中的情况暂时不需要掌握,不影响理解贾尼别科夫效应。一个同样不需要掌握但你可能会感兴趣的知识点是:转动惯量是一个$(0, 2)$型张量,熟悉线性代数以后你可以很轻松地解决和它相关的问题。





\subsection{原理解释}

我们分两步来解释贾尼别科夫效应:首先说明旋转刚体的旋转方向会受到扰动,然后说明扰动必须沿着特定路线进行,从而解释旋转刚体的奇特性质。

为了能理解这一解释,我们需要先介绍描述问题的方法:刚体坐标系。

\subsubsection{刚体坐标系}



给定任何一个刚体,必能找到三个相互正交的轴,使得刚体绕这三个轴旋转的时候,角动量和角速度的方向一致\footnote{这是因为$(0, 2)$型张量必能正交对角化,线性代数中关于“合同变换\upref{lialg}”和“二次型”的章节会解释。}。我们用这三个轴来建立直角坐标系,称之为刚体坐标系。

刚体坐标系会随着刚体运动,或者说在刚体坐标系看来,刚体永远是静止的,反倒是地面参考系或者实验室参考系在运动。


为了方便接下来的讨论,我们规定刚体坐标系的单位是角动量,即坐标系中的一个向量表示刚体此时的总角动量。以\autoref{fig_Dzhani_13} 为例,可以建立一个二维的刚体坐标系,设$x$轴指向右上方,与垂直方向成$45^\circ$角,而$y$轴指向左上方,与垂直方向成$45^\circ$角。那么\autoref{fig_Dzhani_13} 中角动量就表示为一个沿着$y$轴正方向的向量。


注意,刚体参考系通常不是惯性参考系。




\subsubsection{离心力和科里奥利力的扰动}




如\autoref{fig_Dzhani_13} 所示,由于刚体本身在旋转,所以刚体坐标系不是一个惯性系,因此在刚体坐标系中讨论问题要加入\textbf{惯性力}\upref{Iner},比如\textbf{离心力}。

现在取绕\autoref{fig_Dzhani_13} 中绕虚线轴旋转的参考系,使得这一瞬间在该参考系中轻质杆静止。如果轻质杆能稳定旋转的话,它在这个旋转参考系中应该保持静止。但是轻质杆上两个质点分别受水平向外的离心力(图中蓝色向量),于是产生了一个力矩,使轻质杆有了方向改变的趋势。因此,轻质杆不可能以图中姿态稳定地旋转\footnote{还有一种解释是,如果轻质杆在该旋转参考系中保持静止,那么其角动量方向在该参考系中就会一直指向左上方。但是这样一来在实验室参考系中轻质杆的角动量就不停改变方向了,而轻质杆不受外力,在实验室参考系中应该角动量守恒,故轻质杆在该旋转参考系中不可能保持静止。}。

在实验室参考系(惯性系)中,若轻质杆在空中不受外力,则其角动量守恒,故角动量的方向不变。但是刚体坐标系的方向不停变化,因此如果在刚体坐标系中来看,角动量的方向就会不停变化。


如果在绕虚线轴旋转的参考系中,刚体不是静止的(比如被离心力作用后),那么扰动中还会出现另一种惯性力:\textbf{科里奥利力}\upref{Corio}。科里奥利力计算复杂,很难用它来计算清楚刚体的运动状态。


好在,我们不需要知道我们可以借助守恒量来大大简化讨论。








\subsubsection{能量椭球与角动量球}






现在考虑在空中自由转动的刚体,如你的手机,或者贾尼别科夫的螺帽。由于刚体不受外力,故它的\textbf{能量和角动量都守恒}。


回忆一下,我们定义刚体坐标系的时候,坐标轴的\textbf{长度单位}是\textbf{角动量}。因此尽管在刚体坐标系看来,由于扰动,刚体坐标系本身大大方向会不停改变,因而角动量的方向会不停改变,但是角动量的大小不会变。因此,角动量向量在扰动过程中会一直限制在一个球面上运动,我们称这个球为刚体的\textbf{角动量球}。


同样,我们也在刚体坐标系中画出能量不变的面,这个面是一个椭球面,称为\textbf{能量椭球},其三个主轴所在方向恰好是刚体转动惯量的主轴。

于是,确定刚体的初始角动量方向后,其角动量总是在\textbf{能量椭球}和\textbf{角动量球}的交线上运动。

现在我们要确定能量椭球的三个主轴之间的大小关系。称刚体的三个主轴分别为$x$轴、$y$轴和$z$轴,在这三个方向上的转动惯量可以当作标量,分别是$I_x$、$I_y$和$I_z$,并且有$I_x<I_y<I_z$,和\autoref{fig_Dzhani_12} 中一样。若刚体角动量方向在$x$主轴上,其大小为$L_x$,则此时刚体绕$x$轴转动,其动能为\footnote{计算方式:以\autoref{fig_Dzhani_12} 为例,刚体绕$x$轴以角速度$\omega$转动时,转动惯量为$I_x=2r^2m$,角动量为$2r^2m\omega$,而动能则是$E=m(r\omega)^2$。由此可得,$E=\frac{1}{2}L\omega$。这个和平动情况下的$E=\frac{1}{2}pv$形式上完全相同。}
\begin{equation}\label{eq_Dzhani_1}
E_x = \frac{1}{2}L_x\omega_x~. 
\end{equation}
其中$\omega_x$是刚体转动的角速度,且$\omega_x=L_x/I_x$。

同理,若刚体角动量方向在$y$主轴上,大小为$L_y$,则此时刚体的动能为
\begin{equation}\label{eq_Dzhani_2}
E_y = \frac{1}{2}L_y\omega_y~. 
\end{equation}
其中$\omega_y$是刚体转动的角速度,且$\omega_y=L_y/I_y$。

比较\autoref{eq_Dzhani_1} 和\autoref{eq_Dzhani_2} ,并代入$I_x<I_y$,可知如果$E_x=E_y$,则有$L_x<L_y$。因此,能量椭球的$x$主轴比$y$主轴要短。同理,三个主轴中最长的是$z$主轴。这样,我们就确定了能量椭球三个主轴的大小关系了。



\subsubsection{为什么是中间轴}



现在,还是考虑\autoref{fig_Dzhani_12} 的模型。

在刚体坐标系中,如果某一时刻刚体的角动量在$x$方向附近,那就意味着此时角动量在\autoref{fig_Dzhani_7} 第一行左起第一幅图的能量椭球与角动量球的交线上——因为椭球的$x$主轴是最短主轴。如\autoref{fig_Dzhani_12} 可见,此时二者的交线总是在$x$轴附近,因此角动量的变化幅度不会很大。回到实验室系来看待的话,就是角动量的方向始终不变,而刚体的$x$主轴方向始终在一个小范围内变化。

如果某一时刻刚体的角动量在$z$方向附近,那就意味着此时角动量在\autoref{fig_Dzhani_7} 第二行左起第三幅图的交线上,同样变化范围内不大。

但是,如果某一时刻刚体的角动量在$y$方向附近,则对应角动量在\autoref{fig_Dzhani_7} 第一行左起第三幅图或第二行左起第三幅图的交线上,这个交线的变化范围非常大,从实验室参考系看来,就是刚体的$y$主轴首先在一个小范围内波动,接着会翻转到另一面,即贾尼别科夫效应。

如果某一时刻刚体的角动量恰为$y$方向,那刚体就可以稳定转动,对应角动量在\autoref{fig_Dzhani_7} 中第二行左起第一幅图的尖锐交点处,但实践中这几乎不可能达到。




\begin{figure}[ht]
\centering
\includegraphics[width=12cm]{./figures/db26f8dec2b299be.pdf}
\caption{能量椭球与角动量球的交线示意图。图中深蓝色部分是一个椭球,其三个半主轴的长度分别记为$a, b, c$,关系为$a<b<c$。淡蓝色部分是半径逐渐增大的球,其半径记为$r$。图中第一行,从左到右$r$逐渐增大,但始终有$a<r<b$;第二行第一列的时候恰有$r=b$;第二行右边两个图则的对应$b<r<c$的情况。} \label{fig_Dzhani_7}
\end{figure}









