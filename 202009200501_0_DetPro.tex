% 行列式的性质

\pentry{行列式\upref{Deter}}

根据行列式的几何理解, 我们容易理解它的一些性质.

\begin{theorem}{ } \label{DetPro_the1}
若行列式中某列全为 0, 其结果等于 0.
\end{theorem}
由行列式的定义(\autoref{Deter_eq4}~\upref{Deter})容易证明: 展开后的每一项都含有每一列的一个元素, 所以当某列全为 0, 则结果为 0.

几何理解: 若平行体的某条边长等于 0, 其体积也等于 0.

\begin{theorem}{ }
若行列式中的列矢量线性相关, 其结果等于 0.
\end{theorem}
二维情况下两矢量线性相关意味着他们共线, 平行四边形面积为 0. 三维情况下线性相关意味着三个矢量共面, 平行四面体体积为 0. 高维情况也可类比.

\begin{theorem}{ } \label{DetPro_the2}
行列式的值为 0 当且仅当行列式中存在线性相关的列.
\end{theorem}

\begin{theorem}{ } \label{DetPro_the3}
矩阵的任意一列乘以常数,行列式的值也要乘以该常数.
\end{theorem}
按照几何理解, 将平行体任意一条边长乘以一个常数, 它的体积也需要乘以该常数. 至于定理中的正负号, 可以由各阶行列式的定义证明(留做练习).

\begin{theorem}{ }
把矩阵的第 $i$ 列叠加上 “第 $j$ 列乘任意常数”,行列式的值不变.
\end{theorem}
以平行四边形为例, 由于其体积是底乘以高, 令 $\bvec v_1$ 为底, $\bvec v_2$ 在垂直于 $\bvec v_1$ 方向的投影为高, 则将 $\bvec v_2$ 变为 $\bvec v_2 + \lambda \bvec v_1$ ($\lambda$ 为常数)后高不变, 所以面积不变. 三维情况和高维情况同理.

\begin{theorem}{ }
将行列式的两列交换, 结果取相反数.
\end{theorem}
对二阶和三阶行列式, 无论从代数还是几何上都容易得到这个结论. 对于任意阶的情况, 该操作会给每一项增加一个逆序数(见下文), 导致结果取相反数.

\begin{theorem}{ }
矩阵\textbf{转置}(将所有 $a_{i,j}$ 与 $a_{j,i}$ 交换\upref{Mat})后行其列式的值不变.
\end{theorem}
这个定理没有显然的几何理解, 可以直接用代数定义证明(\autoref{Deter_eq1} 和\autoref{Deter_eq2}, 以及下文的\autoref{Deter_eq4}). 根据这个定理, 以上凡是涉及到 “行” 的定理和说明, 都可以替换为 “列”, 请读者自行回顾一次.

\subsection{拓展}

行列式的代数性质可以抽象为\textbf{外代数}(见\autoref{AlgFie_ex1} 以及例子后的解释),进而用于定义\textbf{外微分}\upref{ExtDif},描述微分形式的代数性质.
