% 堆
% keys 堆|数据结构|C++

堆是一个以一颗完全二叉树的结构存储值的数据结构.

堆分为\textbf{大根堆}和\textbf{小根堆},大根堆的意思是,父节点的值都大于左右两个儿子的值,小根堆则相反.C++ STL 中的 \verb|priority_queue| 优先队列默认就是大根堆.

以一个例题来讲解堆.

维护一个集合,初始时集合为空,支持如下几种操作:
\begin{enumerate}
\item 插入一个数 $x$;
\item 输出当前集合中的最小值;
\item 删除当前集合中的最小值;
\item 删除第 $k$ 个插入的数;
\item 修改第 $k$ 个插入的数,将其变为 $x$.
\end{enumerate}

本题要我们实现一个小根堆,根结点是整课树中最小的结点,父节点永远大于左右两个子节点.
C++ STL 中的堆只能实现前 $3$ 个操作.
\begin{lstlisting}[language=cpp]
priority_queue<int, vector<int>, greater<int>> heap;   // 定义小根堆的方式
int t;
cin >> t;
while (t -- )
{
    string s;
    int x;
    cin >> s;
            
    cin >> x;
    heap.push(x);   // 插入一个数 x

    cout << heap.top() << endl;     // 输出最小值,即栈顶
    heap.pop();     // 删除最小值,即删除栈顶
}
\end{lstlisting}

要想实现随机删除和修改,只能用数组来模拟堆,所以我们讲解一下如何使用数组模拟堆.

首先需要一个数组 \verb|h[M]| 用于存储堆中的元素,由于需要在任意位置进行删除和修改,所以需要多开两个数组 \verb|ph[N]| 和 \verb|hp[N]|.

\verb|ph[i]| 的意思是第 $i$ 个插入的树在堆中的下标是什么,而 \verb|hp[i]| 的意思是在堆中下标是 $i$ 的点是第几个插入的.
比如 \verb|ph[1] = a|,\verb|hp[a] = 1|.

因为是随机插入和删除,所以在堆中删除或者插入某个值的话必定要进行调整.分为 $up$ 操作和 $down$ 操作,$up$ 是如果一个数不符合堆的性质(太小了)就要往上调整,$down$ 也同理,因为某个数不符合堆的性质,就要往下调整.

堆的存储只用开一个一维数组 \verb|h[N]| 就可以了,父节点是 $i$,左子节点就是 $2 \times i$,右子节点就是 $2 \times i + 1$.
$2 \times i$ 可以写成 $\mathtt{u << 1}$ ,$2 \times i + 1$ 可以写成 $\mathtt{u << 1 | 1}$.比如父节点的下标是 $1$,左子节点的下标就是:$2$,右子节点的下标就是:$3$.有一点需要注意:下标必须从 $1$ 开始,如果从 $0$ 开始的话左子节点和右子节点的下标均为 $0$,这样显然是不行的.







