% 弦理论(综述)
% license CCBYSA3
% type Wiki

本文根据 CC-BY-SA 协议转载翻译自维基百科\href{https://en.wikipedia.org/wiki/String_theory}{相关文章}。

在物理学中,弦理论是一种理论框架,其中粒子物理学中的点状粒子被称为弦的一维物体所取代。弦理论描述了这些弦如何在空间中传播并相互作用。在大于弦尺度的距离范围内,弦表现得像一个粒子,其质量、电荷和其他属性由弦的振动状态决定。在弦理论中,弦的众多振动状态之一对应于引力子,一种携带引力的量子力学粒子。因此,弦理论是一种量子引力理论。

弦理论是一个广泛而多样的学科,试图解决许多深刻的基础物理学问题。弦理论为数学物理学做出了诸多贡献,这些贡献已应用于黑洞物理学、早期宇宙宇宙学、核物理学和凝聚态物理学的各种问题,并激发了纯数学领域的一些重大进展。由于弦理论可能提供一个统一的引力和粒子物理学的描述,它成为了万物理论的候选者,即一个自洽的数学模型,能够描述所有基本力和物质形式。尽管在这些问题上进行了大量研究,但目前尚不清楚弦理论在多大程度上描述了真实世界,或者该理论在选择细节时允许多少自由度。

弦理论最早在1960年代末期作为强核力的理论进行研究,但随后由于量子色动力学的兴起而被放弃。随后,人们意识到,正是使弦理论不适合作为核物理学理论的那些特性,使其成为量子引力理论的一个有前景的候选者。弦理论的最早版本是玻色子弦理论,它只包含了被称为玻色子的粒子类别。后来,它发展成了超弦理论,超弦理论假设玻色子和称为费米子的粒子类别之间存在一种叫做超对称的联系。在1990年代中期,人们推测出,超弦理论的五个一致版本其实是一个十一维的单一理论的不同极限情形,这个理论被称为M理论。1997年底,理论物理学家发现了一个重要的关系,叫做反德西特/共形场论对偶性(AdS/CFT对偶性),它将弦理论与另一种物理理论——量子场论——联系了起来。

弦理论的一个挑战是,完整的理论在所有情况下都没有一个令人满意的定义。另一个问题是,该理论被认为描述了一个庞大的可能宇宙的景观,这使得基于弦理论的粒子物理学理论发展变得复杂。这些问题导致物理学界一些人批评这些物理学方法,并质疑继续进行弦理论统一研究的价值。
\subsection{基本原理}
\subsubsection{概述}
\begin{figure}[ht]
\centering
\includegraphics[width=6cm]{./figures/b1d0a7d0dd142f3e.png}
\caption{弦理论的基本物体是开弦和闭弦。} \label{fig_String_1}
\end{figure}  
在20世纪,出现了两种理论框架来阐述物理学的基本法则。第一种是阿尔伯特·爱因斯坦的广义相对论,这一理论解释了引力的作用以及宏观层面上时空的结构。另一种是量子力学,这是一种完全不同的框架,利用已知的概率原理来描述微观层面的物理现象。到1970年代末,这两种框架已被证明足以解释大多数已观察到的宇宙特征,从基本粒子到原子,再到恒星和整个宇宙的演化。[1]

尽管取得了这些成功,但仍有许多问题亟待解决。现代物理学中最深刻的问题之一是量子引力问题。[1]广义相对论是在经典物理框架内 formul化的,而其他基本力则在量子力学框架内描述。为了将广义相对论与量子力学的原则统一起来,需要一个量子引力理论,但当尝试将量子理论的常规法则应用于引力时,便会遇到困难。[2]

弦理论是一种理论框架,试图解决这些问题。

弦理论的起点是这样一个观点:粒子物理学中的点状粒子也可以被建模为一种称为“弦”的一维物体。弦理论描述了弦如何在空间中传播并相互作用。在某个特定版本的弦理论中,只有一种类型的弦,它可能看起来像一个小的环或普通弦的一段,并且可以以不同的方式振动。在大于弦尺度的距离范围内,弦看起来就像是一个普通的粒子,符合非弦模型中的基本粒子,其质量、电荷以及其他特性由弦的振动状态决定。作为量子引力的应用,弦理论提出了一种振动状态,负责产生引力子——一种尚未被证实的量子粒子,理论上它承担引力的作用。[3]

过去几十年中,弦理论的主要发展之一是发现了某些“对偶性”,即将一种物理理论与另一种物理理论联系起来的数学变换。研究弦理论的物理学家发现了不同版本弦理论之间的一些对偶性,这导致了一个猜想:所有一致的弦理论版本都可以被统一在一个单一的框架中,称为\(M\)理论。[4]

弦理论的研究还在黑洞的性质和引力相互作用方面取得了一些成果。当人们试图理解黑洞的量子特性时,出现了一些悖论,弦理论的研究试图澄清这些问题。1997年底,这一领域的研究达到了顶峰,发现了反-德西特/共形场理论对应(AdS/CFT)。[5]这是一个理论结果,将弦理论与其他理论上更为清楚的物理理论联系起来。AdS/CFT对应对于黑洞和量子引力的研究具有重要意义,并且已被应用于其他领域,[6]包括核物理和凝聚态物理。[7][8]

由于弦理论包含了所有基本相互作用,包括引力,许多物理学家希望它最终能够发展到足以完全描述我们的宇宙,从而成为一种“万物理论”。目前弦理论研究的目标之一是找到一个能够再现已观察到的基本粒子谱、具有小的宇宙常数、包含暗物质并提供合理的宇宙膨胀机制的理论解。尽管在这些目标上已有一些进展,但目前尚不清楚弦理论在多大程度上能够描述现实世界,或者该理论在细节选择上允许多少自由度。[9]

弦理论的挑战之一是,完整的理论在所有情况下都没有一个令人满意的定义。弦的散射最直接的定义方法是使用微扰理论的技巧,但通常并不清楚如何非微扰地定义弦理论。[10]此外,是否存在某种原理来选择弦理论的真空态——即决定我们宇宙特性的物理状态——也尚不明确。[11]这些问题使得部分学者批评将物理学统一的这些方法,并质疑继续研究这些问题的价值。[12]
\subsubsection{弦}
\begin{figure}[ht]
\centering
\includegraphics[width=6cm]{./figures/fd6b90c488650f4c.png}
\caption{} \label{fig_String_2}
\end{figure}