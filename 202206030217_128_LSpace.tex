% 矢量空间
% keys 线性空间|向量空间|线性代数|矢量|几何矢量|矢量空间|集合|交换律|结合律|分配率|多项式|线性相关|线性无关|基底|n维空间|行矢量|列矢量|子空间|内积

\pentry{几何矢量\upref{GVec}, 函数\upref{functi}, 抽象\upref{Abstra}}

\textbf{矢量空间(vector space)} 也叫\textbf{向量空间}或\textbf{线性空间(linear space)},是一种满足一定条件的集合\upref{Set}, 有无穷多个元素, 每个元素叫做一个\textbf{矢量}或\textbf{向量(vector)}. 它必须满足, 在其中选择任意两个矢量, 它们的线性组合仍然在这个空间中(以及一些其他条件). 进行归纳后易得, 这个条件等价于 “任意有限个矢量的线性组合仍然在这个空间中”(封闭性). 这里的 “矢量” 是一个广义的概念, 是几何矢量\upref{GVec}的抽象\upref{Abstra}; 反过来,几何矢量是矢量的具象(特例). 一个广义的矢量,不一定具有长度和方向, 例如下面会看到函数也可以看作矢量.

\subsection{定义}
矢量空间的定义必须依赖一个\textbf{域(field)}\upref{field} $\mathbb F$. 简单来说,域就是能进行加减乘除的对象的一个集合, 比如实数域 $\mathbb R$ 和复数域 $\mathbb C$. 这个域本身被称为该矢量空间的\textbf{标量域(scalar field)}或\textbf{标域},它的元素被称为矢量空间的\textbf{标量(scalar)},它们不是矢量空间的元素,但是可以用来和矢量进行数乘. 那么, 我们就说矢量空间是 “域 $\mathbb{F}$ 上的”. 矢量空间选择的域几乎都是 $\mathbb{R}$ 或 $\mathbb{C}$, 即讨论的是\textbf{实数(或复数)域上的矢量空间}. 所以即使我们不了解域的具体定义也没关系, 只要知道实数和复数是两种常见的域就足够了.

\begin{definition}{矢量空间}
标量域 $\mathbb F$ 上的矢量空间定义了矢量的集合 $X$、 两个矢量间的加法运算 $X\times X \to X$ 以及标量和矢量间的数乘运算 $\mathbb F \times X \to \mathbb X$. 令矢量 $u,v,w \in X$, 标量 $a,b \in \mathbb F$, 矢量加法记为 $u + v$, 数乘记为 $a u$. 两种运算必须满足以下性质:

\subsubsection{加法运算}
\begin{enumerate}
\item 满足加法\textbf{交换律} $u + v = v + u$.
\item 满足加法\textbf{结合律} $(u + v) + w = u + (v + w)$.
\item 存在\textbf{零矢量},使得 $v + 0 = v$.
\item 空间中任意矢量 $v$ 存在\textbf{逆矢量} $-v$,使得 $v + (-v) = 0$.
\end{enumerate}

\subsubsection{数乘运算}
\begin{enumerate}
\item 乘法\textbf{分配律} $a(u + v) = au + av$ 
\item 乘法\textbf{分配律} $(a + b)v = av + bv$
\item 乘法\textbf{结合律} $a (b v) = (ab) v$
\end{enumerate}
\end{definition}

\textbf{说明}: 加法运算 $X \times X \to X$ 是一个二元映射(\autoref{map_sub1}~\upref{map}), 注意运算的结果必须仍然落在 $X$ 中. 我们把这样的运算叫做\textbf{封闭(closed)}的\footnote{一些文献中也叫 “闭合”}. 数乘运算同样也是封闭的, 即一个矢量数乘标量后仍然落在 $X$ 中. 我们现在还没有涉及 $X$ 以外的元素, 所以封闭性看似有些多余, 但以后会看到一个矢量空间 $X$ 的子集 $X_1$ 也可以是矢量空间, 称为子空间\upref{SubSpc}, $X_1$ 上的两种运算封闭意味着运算结果只能落在 $X_1$ 中而不能是 $X$ 的其他元素.

\begin{corollary}{}
1. $0_\mathbb{F}\cdot v_X=0_X$

$\quad$ 证明:$0_\mathbb{F}\cdot v_X=(a_\mathbb{F}-a_\mathbb{F})\cdot v_X=a_\mathbb{F}v_X+(-a_\mathbb{F}v_X)=0_X$

2. $a_\mathbb{F} \cdot 0_X=0_X$

$\quad$ 证明:$a_\mathbb{F} \cdot 0_X=a_\mathbb{F}\cdot(v_X+(-v_X))=a_\mathbb{F}v_X+(-a_\mathbb{F}v_X)=0_X$

\end{corollary}

\begin{exercise}{几何矢量}
证明 1,2,3 维空间中的所有几何矢量\upref{GVec}各自构成实数域 $\mathbb R$ 上的矢量空间.
\end{exercise}

作为一个非几何矢量的例子, 我们来看由多项式构成矢量空间.

\begin{example}{多项式}\label{LSpace_ex1}
所有不大于 $n$ 阶的多项式 $c_n x^n + c_{n-1} x^{n-1} + \dots + c_1 x + c_0$ 可以构成一个实数或复数矢量空间.定义矢量加法为两多项式相加, 满足
\begin{itemize}
\item 封闭性:任意两个不大于 $n$ 阶的多项式相加仍然为不大于 $n$ 阶的多项式.
\item 交换律:多项式相加显然满足交换律.
\item 零矢量:常数 0 可以看做一个 0 阶多项式, 任何多项式与之相加都不改变.
\item 逆矢量:把任意多项式乘以 $-1$ 就得到它的逆矢量, 任意多项式与其逆矢量相加等于 0.
\end{itemize}
定义矢量数乘为多项式乘以常数, 显然也满足数乘的各项要求, 不再赘述.
\end{example}

另一个重要的矢量空间,是\textbf{函数空间(function space)}.

\begin{example}{函数空间}\label{LSpace_ex2}
实数到实数的全体函数($f:\mathbb R \to \mathbb R$)的集合 $F$ 构成一个线性空间,称为 $\mathbb{R}$ \textbf{函数空间}. 函数空间中两个向量的加法定义为,对于任何实数 $x$ 和函数(即向量)$f, g\in F$,有 $(f+g)(x)=f(x)+g(x)$;数乘定义为,对于任何实数 $a, x$ 和函数 $f\in F$,有 $(af)(x)=af(x)$.

类似地,复数到复数、实数到复数等的函数都可以构成线性空间;把函数限制在连续函数、可导函数等条件下也依然构成线性空间.特别地,复数域上的归一化可导函数,构成了复数域上的希尔伯特空间,这是一种无穷维的特殊矢量空间,是量子力学的基础概念,我们将会在将来详细讨论.
\end{example}

注意矢量空间的定义并不需要包含内积(点乘) 的概念, 但我们可以在其基础上额外定义内积, 这样的空间叫做\textbf{内积空间}\upref{InerPd}, 留到以后介绍. 除了内积, 我们可以把 “几何矢量的运算\upref{GVecOp}” 和 “线性相关性\upref{linDpe}” 中介绍的概念都拓展到一般的矢量空间中, 这里不再重复.

\begin{exercise}{复数列矢量}
我们把 $N$ 个复数 $c_1, \dots, c_N$ 按顺序排成一列(或一行, 下同), 叫做\textbf{列矢量}(或\textbf{行矢量}, 下同). 列矢量可以看成是 $N \times 1$ 的矩阵\upref{Mat}. 给它们定义通常意义的加法和数乘运算, 这样所有列矢量可以构成一个 $N$ 维矢量空间. 注意由于我们使用了复数, 即使 $N \leqslant 3$ 时我们也无法将这些矢量与几何矢量对应起来.

如果我们将基底取为\footnote{上标 $\mathrm T$ 表示转置, 这里是为了排版方便} $(1, 0, \dots, 0)\Tr$, $(0, 1, 0, \dots, 0)\Tr$, …, $(0, \dots, 0, 1)\Tr$, 那么显然任意列矢量 $(c_1, \dots, c_N)\Tr$ 的坐标就是有序实数 $c_1, \dots, c_N$. 但我们也可以取其他基底, 这时坐标就会改变. 所以再次强调坐标和矢量本身是不同的.我们将会在矢量空间的表示\upref{VecRep}中详细区分矢量本身和矢量的坐标这两个概念.
\end{exercise}

\begin{exercise}{}
证明\autoref{LSpace_ex1} 中多项式空间是 $n+1$ 维空间, $x^k$ ($k = 0, \dots, n$) 是一组基底(提示: 证明它们线性无关, 可以表示空间中的任意矢量).
\end{exercise}





