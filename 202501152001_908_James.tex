% 詹姆士·金斯(综述)
% license CCBYSA3
% type Wiki

本文根据 CC-BY-SA 协议转载翻译自维基百科\href{https://en.wikipedia.org/wiki/James_Jeans}{相关文章}

\begin{figure}[ht]
\centering
\includegraphics[width=6cm]{./figures/2ff79237cff29b57.png}
\caption{} \label{fig_James_1}
\end{figure}
詹姆斯·霍普伍德·吉恩斯爵士 OM FRS(1877年9月11日—1946年9月16日)是英国物理学家、数学家和天文学家。他曾于1919年至1929年担任皇家学会秘书,1925年至1927年担任皇家天文学会会长,并获得了该学会的金奖奖章。
\subsection{早年生活}  
吉恩斯出生在兰开夏郡的奥姆斯基尔克,是威廉·图洛赫·吉恩斯(William Tulloch Jeans)的儿子,威廉是一位议会通讯员和作家。吉恩斯在梅丘特·泰勒学校(Merchant Taylors' School)、威尔逊文法学校(Wilson's Grammar School,位于坎伯威尔)和剑桥大学三一学院接受教育。作为一名天才学生,吉恩斯被建议采取积极进取的态度参加剑桥数学三部曲考试(Cambridge Mathematical Tripos):

1896年迈克尔马斯学期初,沃克(Walker)召集了吉恩斯和哈代(Hardy),建议他们将数学三部曲的第一部分放在两年内完成。他告诉他们,他无法保证他们会排在前十五名的之内,但他理解他们不会后悔这个决定。吉恩斯接受了他的建议,并去找当时最著名的私人教练R.R.韦布(R. R. Webb)。然而,在第一年结束时,吉恩斯告诉沃克,他与教练韦布发生了争执。沃克因此亲自指导吉恩斯,结果取得了胜利:吉恩斯和J. F.卡梅伦(J. F. Cameron)并列为第二名学者,而R.W.H.T.哈德森(R.W.H.T. Hudson)获得了高级学者奖(Senior Wrangler),G. H. 哈代(G. H. Hardy)则是第四名学者。
\subsection{职业生涯}  
吉恩斯于1901年10月被选为剑桥大学三一学院的院士,并在剑桥大学任教。但在1904年,他作为应用数学教授前往普林斯顿大学。1910年,他返回剑桥大学。

吉恩斯在多个物理学领域做出了重要贡献,包括量子理论、辐射理论和恒星演化。他对旋转物体的分析使他得出结论,皮埃尔-西蒙·拉普拉斯(Pierre-Simon Laplace)关于太阳系由单一气体云形成的理论是错误的,而是提出行星是从太阳中抽取的物质凝结而成,假设这是一场与经过的恒星的灾难性近距离碰撞所导致的。这一理论今天已不被接受。

吉恩斯与阿瑟·爱丁顿(Arthur Eddington)一起,是英国宇宙学的奠基人之一。1928年,吉恩斯首次提出了一个稳态宇宙学的猜想,基于假设宇宙中物质的持续创生。他在他的书《天文学与宇宙生成论》(Astronomy and Cosmogony,1928年)中写道:“呈现出来的猜测类型,是中心星云的性质是‘奇点’,在这些奇点上,物质不断从其他完全外部的空间维度注入到我们的宇宙中,因此,对于我们宇宙的居民来说,它们看起来像是物质不断被创生的点。”这一理论在1965年宇宙微波背景辐射(CMB)被发现并广泛解释为大爆炸的标志后,逐渐被抛弃。

吉恩斯的科学声誉建立在他的几部专著上,包括《气体的动力学理论》(1904年)、《理论力学》(1906年)和《电学与磁学的数学理论》(1908年)。1929年退休后,他为普通读者写了几本书,包括《星辰的轨迹》(1931年)、《我们周围的宇宙》(1934年)、《科学的新背景》(1933年)和《神秘的宇宙》。这些书使吉恩斯成为了当时革命性科学发现的阐述者,特别是在相对论和物理宇宙学方面。

1939年,《英国天文学会杂志》报道吉恩斯将作为剑桥大学选区的议会候选人参选。预计于1939年或1940年举行的选举直到1945年才举行,而吉恩斯并未参与其中。

他还写了《物理学与哲学》(1943年),在书中他从科学与哲学两个不同的角度探讨了现实的不同观念。关于他的宗教观,吉恩斯是一个不可知论者和共济会成员。
\subsection{个人生活}  
吉恩斯结过两次婚,第一次是在1907年与美国诗人夏洛特·蒂凡尼·米切尔(Charlotte Tiffany Mitchell)结婚,后来她去世;第二次则是在1935年与奥地利管风琴师和羽管键琴演奏家苏珊娜·霍克(Suzanne Hock)(更为人知的名字是苏西·吉恩斯,Susi Jeans)结婚。苏西和吉恩斯有三个孩子:乔治、克里斯托弗和凯瑟琳。作为送给妻子的生日礼物,他写了《科学与音乐》这本书。
\subsubsection{去世}  
吉恩斯于1947年去世,妻子和乔伊·亚当森(Joy Adamson)在场。乔伊建议吉恩斯的遗孀制作一具死亡面具,这具面具现在由皇家学会保管。
\subsection{主要成就}  
吉恩斯的一个重大发现被命名为“吉恩斯长度”,它是指空间中星际云的临界半径。吉恩斯长度取决于云的温度、密度以及组成云的粒子的质量。一个小于其吉恩斯长度的云将没有足够的引力克服排斥气体压力的作用,因此不能聚集形成恒星;而一个大于其吉恩斯长度的云将会发生坍缩。
\[
\lambda_{\text{J}} = \sqrt{\frac{15 k_{\text{B}} T}{4\pi G m \rho}}~
\]
吉恩斯还提出了这个方程的另一种版本,称为吉恩斯质量或吉恩斯不稳定性,它用于计算云体在能够坍缩之前必须达到的临界质量。

吉恩斯还帮助发现了瑞利–吉恩斯定律,该定律描述了黑体辐射的能量密度与辐射源温度之间的关系。
\[
f(\lambda) = 8\pi c \frac{k_{\text{B}} T}{\lambda^4}~
\]
吉恩斯还被认为是计算行星大气因气体分子动能而逃逸的速率的贡献者,这一过程被称为吉恩斯逃逸。
\subsection{唯心主义}  
吉恩斯在他的演讲和著作中提倡一种根植于唯心主义形而上学学说的科学哲学,并反对物质主义。他的科普作品首次在1929年的《我们周围的宇宙》中提出了这些观点,他在书中将“以时间和空间来讨论宇宙的创造”,比作“试图通过走到画布的边缘来发现艺术家和绘画的动作”。然而,他将这一观点作为他畅销书《神秘的宇宙》(1930年)的主要主题,在这本书中,他宣称当时的科学正在形成一个将宇宙看作“非机械现实”的图景。

“宇宙开始看起来更像是一种伟大的思想,而不是一台伟大的机器。心灵不再显得像是物质领域的偶然入侵者……我们应该视其为物质领域的创造者和主宰。”

——詹姆斯·吉恩斯,《神秘的宇宙》

在1931年《观察家报》的一次采访中,当被问及他是否认为生命是偶然的,或者它是“某个伟大计划的一部分”时,吉恩斯表示他更倾向于“理想主义理论,认为意识是根本的,而物质宇宙是由意识衍生出来的。”他接着提出,“每个个体的意识应该被比作是宇宙心灵中的一个脑细胞。”

在1934年,他作为英国科学促进会会长在阿伯丁的大会上发表演讲时,吉恩斯特别提到了笛卡尔的工作及其对现代科学哲学的相关性。他辩称,“不再有空间容纳笛卡尔时代以来困扰哲学的那种二元论。”

2020年,丹尼尔·海尔辛在《物理学今天》上评审《神秘的宇宙》时,总结了这本书的哲学结论:“吉恩斯认为,我们必须放弃科学长期珍视的物质主义和机械主义世界观,这种世界观认为自然像机器一样运作,且仅由相互作用的物质粒子组成。”他对吉恩斯的评价将这些哲学观点与现代的科学传播者,如尼尔·德格拉斯·泰森和肖恩·卡罗尔作对比,认为他们“可能会反对吉恩斯的唯心主义。”