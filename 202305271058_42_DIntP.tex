% 定积分的性质
% keys 定积分|性质

\pentry{定积分\upref{DInt}}
在定积分的定义(\autoref{def_DInt_3}~\upref{DInt}),说“$a$ 到 $b$ 的区间上的定积分” 时,总是理解成 $a<b$。现在我们先去除这一限制,即对定积分的定义中 $a,b$ 的大小关系不施加任何限制,进而讨论定积分的性质。为此,先介绍定向区间的概念。
\subsection{定向区间上的定积分}
\begin{definition}{定向区间}
满足不等式
\begin{equation}
a\leq x\leq b\quad or\quad a\geq x\geq b
\end{equation}
的顺序从 $a$ 到 $b$ 的 $x$ 的集合称为从 $a$ 到 $b$ 的\textbf{定向区间},记作 $[a,b]$。即 $a<b$ 则是递增顺序,$a>b$ 就是递减顺序。
\end{definition}
对 $a>b$ 时的定向区间 $[a,b]$ 上的积分的定义,可以用完全类似的方法定义,只需从 $a$ 到 $b$ 的方向插入分点:
\begin{equation}
x_0=a>x_1>\cdots>x_n=n
\end{equation}
于是,若积分和
\begin{equation}
\sigma=\sum_{i=0}^{n-1}f(\xi_i)\Delta x_i,\quad\Delta x_i=x_{i+1}-x_i<0
\end{equation}
在 $\lambda=\max\qty{\Delta x_i|i=0,\cdots n-1}$ 趋于 0 时的极限(\autoref{def_DInt_1}~\upref{DInt})存在,则该极限就是 $f(x0$ 在定向区间 $[a,b]$ 上的定积分,并记作
\begin{equation}
\int_a^b f(x)\dd x=\lim_{\lambda\rightarrow0}\sigma
\end{equation}

\begin{theorem}{}
若 $f(x)$ 在区间 $[b,a]$ 上可积,则它们在 $[a,b]$ 也可积,并且
\begin{equation}
\int_a^b f(x)\dd x=-\int_b^a f(x)\dd x
\end{equation}
\end{theorem} 

\textbf{证明:}
只需对 $[a,b],[b,a]$ 取同样的分点和 $\xi_i$ ,则它们的积分和仅仅差一负号,因此其极限也只差一负号。

\textbf{证毕!}
\begin{theorem}{}
$\int_a^a f(x)\dd x=0$
\end{theorem}
\textbf{证明:}
此时 $[a,a]$ 上所有的分点和 $\xi_i$ 都是 $a$,而 $\Delta x_i=0$ 且 $f(a)$ 有限,所以积分和每一项都是0。

\textbf{证毕!} 
