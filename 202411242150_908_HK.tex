% 罗伯特·胡克(综述)
% license CCBYSA3
% type Wiki

本文根据 CC-BY-SA 协议转载翻译自维基百科\href{https://en.wikipedia.org/wiki/Robert_Hooke}{相关文章}。

罗伯特·胡克 (Robert Hooke) FRS (/hʊk/;1635年7月18日-1703年3月3日)[4][a] 是一位英国博学者,活跃于物理学(“自然哲学”)、天文学、地质学、气象学和建筑学领域。[5] 他被认为是最早在1665年利用其设计的复合显微镜研究微观生物的科学家之一。[6][7] 胡克年轻时是一位贫困的科学研究者,后来成为他那个时代最重要的科学家之一。[8] 在1666年的伦敦大火之后,胡克以测量员和建筑师的身份,通过完成超过一半的地产界线测绘工作以及协助城市的快速重建,获得了财富和声誉。[9][8] 在他去世后的几个世纪中,胡克经常受到作家的贬低,但在20世纪末,他的名誉得以恢复,并被誉为“英格兰的达·芬奇”。[10]

胡克是皇家学会的院士,从1662年起担任其首任实验策展人。[9] 从1665年至1703年,他还担任格雷沙姆学院的几何学教授。[11] 胡克的科学生涯始于担任物理科学家罗伯特·波义耳(Robert Boyle)的助手。胡克制作了用于波义耳气体定律实验的真空泵,并亲自进行了实验。[12] 1664年,胡克观测到火星和木星的自转。[11] 胡克在1665年出版的著作《显微图谱》(*Micrographia*)中首次提出了“细胞”(cell)一词,这本书激发了显微研究的热潮。[13][14] 在光学领域的研究中——特别是对光折射的研究——胡克提出了光的波动理论。[15] 他是第一个提出以下假说的人:物质因热膨胀的原因,[16] 空气由不断运动的小颗粒组成,并由此产生压力,[17] 以及热是一种能量的概念。[18]

在物理学中,胡克推测重力遵循反平方定律,并且可以说是第一个提出行星运动中这种关系假设的人。[19][20] 这一原理后来被艾萨克·牛顿(Isaac Newton)进一步发展并形式化为牛顿的万有引力定律。[21] 对这一见解的优先权争议促成了胡克与牛顿之间的竞争。在地质学和古生物学中,胡克创立了“水陆球”理论,[22] 因而质疑了《圣经》中关于地球年龄的观点;他还提出了物种灭绝的假说,并认为山丘和山脉是由地质过程抬升而成的。[23] 通过识别已灭绝物种的化石,胡克预示了生物进化论的诞生。[22][24]