% 力的分解与合成
% 平行四边形法则|几何矢量|三角形法则|合力|分力

\pentry{几何矢量\upref{GVec}}

\subsection{力的合成}
在经典力学中, 力可以用几何矢量\upref{GVec}表示. 力的分解与合成可以看作一个\textbf{基本假设}. 这个假设是牛顿运动定律\upref{New3}的基础, 因为牛顿三定律中的 “力” 都是指质点所受的合力.

当若干个力 $\bvec F_i$ ($i = 1, 2, \dots, N$)作用在同一个质点上时, 等效于一个力 $\bvec F$ 作用在同一个质点上.
\begin{equation}\label{Fdecom_eq1}
\bvec F = \sum_{i=1}^N \bvec F_i = \bvec F_1 + \bvec F_2 + \dots + \bvec F_N
\end{equation}

\begin{figure}[ht]
\centering
\includegraphics[width=10cm]{./figures/Fdecom_2.pdf}
\caption{一组力的作用效果等效于他们的合力} \label{Fdecom_fig2}
\end{figure}

注意这里的加号表示几何矢量\upref{GVec}的加法而不是数的加法. 我们把 $\bvec F$ 叫做 $N$ 个 $\bvec F_i$ 的\textbf{合力}, 每个 $\bvec F_i$ 叫做一个\textbf{分力}. \autoref{Fdecom_eq1} 从左到右的过程叫做\textbf{力的分解}, 从右到左的过程叫做\textbf{力的合成}.

这里所说的 “等效” 可以指这个质点受力后的运动情况,如果我们基于牛顿第二定律\upref{New3} $\bvec F = m \bvec a$计算粒子的运行轨迹,那么 $\bvec F$ 指作用在粒子上的合力$\bvec F_\text{合}$,即有 $\bvec F_\text{合}=m \bvec a$。“等效”也可以指物体发生的形变, 例如该质点固定在弹簧上, 弹簧发生的形变。

回顾两个几何矢量的加法, 我们就得到了所谓的\textbf{平行四边形法则}或者\textbf{三角形法则}. 若 $N > 2$, 用 “首尾相接” 的方法即可. 注意这个过程不需要坐标系的概念. 若建立了直角坐标系, 我们也可以先计算这些矢量的坐标, 然后使用坐标计算矢量加法(\autoref{Gvec2_eq8}~\upref{Gvec2}).

\subsection{力的分解}
我们还可以反向运用\autoref{Fdecom_eq1} ,将一个力分解为多个力:
\begin{equation}
\bvec F = \sum_i \bvec F_{i}
\end{equation}

我们甚至可以进行多次分解, 即继续令某个力等于若干力相加:
\begin{equation}
\bvec F_i = \sum_j \bvec F_{i,j}
\end{equation}

那么 合力 $\bvec F$ 就可以最终分解为:
\begin{equation}
\bvec F = \sum_{i,j} \bvec F_{i,j}
\end{equation}

\begin{figure}[ht]
\centering
\includegraphics[width=10cm]{./figures/Fdecom_1.pdf}
\caption{实际常基于坐标轴、切面、粒子运动方向等,将力分解为一组正交(互相垂直)的力} \label{Fdecom_fig1}
\end{figure}

% 这仍然符合分解的定义, 即一个力矢量表示为多个力矢量相加, 本质上并无不同.
