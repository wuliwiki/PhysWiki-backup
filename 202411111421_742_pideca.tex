% pi 介子衰变
% keys 介子|衰变|弱作用
% license Usr
% type Wiki

\pentry{狄拉克场\nref{nod_Dirac}}{nod_3634}

\subsection*{$\pi$介子衰变} 
这一节我们来讨论$\pi$介子衰变。我们首先讨论带电$\pi$的衰变 
\begin{equation}
\pi^- \rightarrow l^- + \bar \nu_l~.
\end{equation}
其中$l$是$\mu$子或电子。

我们不知道$W$粒子是如何耦合到$\pi$介子的。但是我们知道$W$粒子是如何耦合到轻子的,这个过程的散射振幅可以写为
\begin{equation}
\mathcal M = \frac{g_w^2}{8(M_W c)^2} [\bar u (3) \gamma_\mu (1-\gamma^5)v(2)] F^\mu~.
\end{equation}

其中$F^\mu$是描写$\pi$到$W$的形状因子。它是某个标量乘上$p^\mu$.

\begin{equation}
F^\mu = f_\pi p^\mu~. 
\end{equation}

$f_\pi$被称为$\pi$衰变常数。

对出射自旋求和,我们可以得到
\begin{equation}
\langle |\mathcal M|^2 \rangle = \bigg[ \frac{f_\pi}{8} \bigg( \frac{g_w}{M_W c} \bigg)^2  \bigg]^2 p_\mup_\nu {\rm Tr} [\gamma^\mu (1-\gamma^5)  ]  ~.
\end{equation}