% 费罗贝尼乌斯定理
\pentry{流形\upref{Manif},切空间\upref{tgSpa}}

在微分几何中, \textbf{费罗贝尼乌斯定理 (Frobenius' theorem)} 给出了一阶拟线性偏微分方程组可积的充分必要条件. 它有许多重要的几何推论.

\subsection{对合分布与可积性}
设$M$是$n$维微分流形. 一个\textbf{光滑$k$维分布} (smooth $k$-dimensional distribution, 注意这不是分析学意义下代表广义函数的分布) $D$是指切丛$TM$的$k$维光滑子丛. 等价地说, 这表示在任何一点$p\in M$都给出$T_pM$的$k$维子空间$D_p$, 而且$D_p$光滑地依赖于$p$. 

维数为$k$的光滑分布$D$称为\textbf{对合(involutive)的}, 如果对于$D$的任意两个光滑截面$X,Y$, $[X,Y]$也还是$D$的截面. 

\begin{theorem}{费罗贝尼乌斯定理}
设$D$是$M$上的$k$维对合分布. 则在任意一点$p\in M$, 都有局部坐标系$\{x^i\}$使得$D$在该点处由坐标向量$\partial_1,\cdots ,\partial_k$张成. 等价地说, 在任意一点$p$都有一过点$p$的$k$维子流形$N_p$使得$D$恰为$N_p$的切丛.
\end{theorem}
这样的分布称为\textbf{可积 (integrable)} 的. 由此在小邻域内得到的子流形的族称为$M$局部的一个\textbf{正则叶理 (regular foliation)}. 它将流形$M$在局部上划分成了许多相互不交叠的叶片.

定理的证明大意如下. 不妨设$M$是$\mathbb{R}^n$中坐标原点的小邻域. 取这小邻域里$D$的局部标架$X_1,\cdots ,X_k$, 使得$X_i(0)=\partial_i|_0$, 并定义映射$\Pi: \mathbb{R}^n\to \mathbb{R}^n$为
$$
\Pi(x^1,\cdots ,x^n)=(x^1X_1(x),\cdots ,x^kX_k(x),0,\cdots ,0).
$$
则显然切丛的子丛$\text{ker}d\Pi$在原点的小邻域内与$D$互补, 故$(d\Pi)^{-1}$在丛$\text{span}(\partial_1,\cdots ,\partial_k)$上是良好定义的, 而且其像正是$D$. 命$V_i=(d\Pi)^{-1}\partial_i$, 则诸$V_i$都是$D$的截面, 根据对合性质, $[V_i,V_j]$也是$D$的截面, 而且
$$
d\Pi[V_i,V_j]=[\partial_i,\partial_j]=0.
$$
于是$[V_i,V_j]=0$, 因此它们实际上是坐标向量场. 因此$D$是由坐标向量场张成的. 

\subsection{等价表述}


考虑$n$维自变量$x=(x^1,...,x^n)$的$m$个未知函数$u=(u^1,...,u^m)$的拟线性偏微分方程组:
$$
\frac{\partial u^\alpha}{\partial x^i}=f_i^\alpha(x,u),\,1\leq\alpha\leq m;1\leq i\leq n.
$$
这里$f_i^\alpha(x,z)$是已知的函数. 为了确定这个方程组的解, 需要的是一个"初始条件", 即指定$u$在某$x_0$处的值. 