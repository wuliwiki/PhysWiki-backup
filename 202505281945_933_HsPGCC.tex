% 射影几何视角下的圆锥曲线(高中)
% keys 射影几何|对偶原理|圆锥曲线
% license Usr
% type Tutor

\begin{issues}
\issueDraft
\end{issues}

\pentry{圆锥曲线的统一定义\nref{nod_HsCsFD}}{nod_029c}

在“焦点–准线”定义的探索中,统一表达圆锥曲线的目标基本实现。但还有两个问题值得进一步追问:其一,既然尝试用抛物线的定义统一椭圆与双曲线取得成功,反过来,是否也可以将抛物线纳入椭圆与双曲线的定义框架?其二,在推导过程中曾出现“焦点在无穷远处”或“准线在无穷远处”这样的说法,这些话到底意味着什么?

在高中的数学体系中,这类问题很难得到直接解释。但在射影几何的视角下,它们可以被自然地理解。在高中阶段,射影几何属于完全超纲的内容,大多数教材和教师也并不涉及。但这个分支恰恰提供了圆锥曲线最彻底的统一理解。尤其是对于“为何焦点–准线的定义能够囊括所有圆锥曲线”这一点和前面两个问题,射影几何揭示了其背后更深层的原因和联系。

因此,尽管不要求掌握射影几何的具体技术,这一节仍将简要介绍其中最关键的思想。哪怕只是作为一次短暂的接触,也足以为整个圆锥曲线体系提供一个更完整的背景,也供读者感受数学之美。

\subsection{平行线可以相交吗?}

在熟悉的欧式几何中,有一些“默认”的限制,比如:
	•	平行线不会相交;
	•	直线是无限延伸的;
	•	点只能表示有限的位置。

这些看起来都很自然。可是,在现实中,类似的情形却经常出现:平行的铁路轨道,在远方看起来会相交;沿着很高的楼的一面看,墙的两侧最终好像汇聚到了一个点,在绘画中称为“消失点”

\begin{figure}[ht]
\centering
\includegraphics[width=11cm]{./figures/aa409536797b866f.png}
\caption{平行轨道的消失点} \label{fig_HsCsFD_2}
\end{figure}

\begin{figure}[ht]
\centering
\includegraphics[width=11cm]{./figures/040cd6e71df42276.png}
\caption{仰望高楼时墙体延伸至消失点} \label{fig_HsCsFD_3}
\end{figure}
射影几何的创立,就是为了用数学方法来描述前面提到的这些现象。

观察前面的现象,几个与之前欧氏几何中区别的点是:
平行线也有交点
无穷远处的点在平面上也是不可忽略,或者说存在的。

在射影几何中,我们引入了“无穷远点”和“无穷远线”(例如所有平行线在“无穷远点”相交),这是因为我们不关心距离和角度,只关心交点关系(即拓扑结构/位置结构)。

所有直线在射影几何中都相交——平行线也会在“无穷远点”相交。

于是扩展了平面,引入了一个“无穷远直线”,把所有方向的“无穷远点”放在这条线上。

射影几何给展示了:
	•	同一个对象可以从不同的角度理解;
	•	表面看起来不同的东西,背后可能有统一的结构;
	•	有时,必须打破一些“习惯的规则”,才能看到更完整的图景。

有趣的是,在这样定义之后,在射影几何中,点和直线可以互换、对称对待;也就是说,直线也可以看成是“由点组成”的,点也可以像直线一样进行变换。


\subsection{对偶原理}

在十九世纪,法国数学家腾塞叶(Jean-Victor Poncelet)在俄国战俘营中写下了他的重要著作《论图形的摄影性质》。在这部作品中,他首次系统提出了“对偶原理”与“投影不变性”这两个深刻的几何思想。所谓对偶原理,是指在射影几何中,点与直线可以互换,互换后许多几何命题依然成立。例如,“两点决定一条直线”的命题,对偶后变成了“两条直线决定一个交点”,这两者在射影几何中都同样成立。这种点与线之间的对称关系,揭示了几何结构中隐藏的深层对称性,使人们重新思考“几何事实”背后的逻辑构造。

与此同时,腾塞叶还指出,一些几何性质在投影变换下是保持不变的。换句话说,即使我们改变观察角度或从不同平面进行投影,某些关系仍旧成立,这被称为“投影下的不变性”。比如,共线的点经过中心投影后仍然共线;一个圆在透视下可能变成椭圆、抛物线或双曲线,但这些曲线本质上都是圆锥曲线,因此在射影几何中是等价的。这一思想打破了古典欧几里得几何中对“形状”的执着,把几何研究的焦点从“看上去的样子”转向了“结构中的本质”。

到了二十世纪,随着公理化几何的发展,数学家们进一步发现:在许多几何定理中,把“点共线”换成“线共点”、把“点”换成“直线”后,新的表述仍然成立。这些互换后的命题不仅不是偶然巧合,而是源自整个射影几何体系中点与线的对等地位。对偶原理的提出不仅丰富了几何的思维方式,也为代数几何、拓扑学、以及更现代的数学分支奠定了基础。在这种视角下,几何的研究不再只是对现实图形的模仿,而是一种对空间逻辑结构的深刻把握。


\subsection{射影几何视角下的圆锥曲线}


尤其是当在统一定义中使用了“准线”这个概念时,会发现一个问题:

准线是直线,而焦点是点,它们的地位并不对等。

比如,在抛物线中,焦点和准线之间的距离决定了曲线的开口程度;但在椭圆和双曲线中,焦点之间的关系常常比准线更显眼。这种“不对等”让难以一眼看出统一性。

要解决这个问题,需要让“点”和“直线”变得对等、互换,这正是射影几何擅长的。




射影几何中的视角使能够用一种统一且优雅的方式看待圆锥曲线。但在射影几何中,这些差异被看作是坐标选择与观察角度所导致的表象变化,它们在更本质的层面上是一类对象的不同表现:它们都是圆锥曲线。圆锥曲线不是三类不同的曲线,而是一个统一的几何实体的三种视角。它让跳出了直观图形的束缚,从结构上理解几何对象之间的联系,也为代数几何、复几何乃至更高维的几何打下了坚实的基础。

从射影几何的角度看,圆锥曲线定义为一个圆锥面与一个平面相交的轨迹。这个定义在欧几里得空间中也成立,但射影几何更进一步地指出:在射影平面中,所有非退化的圆锥曲线都是射影等价的。这意味着可以通过一个合适的射影变换(即坐标的线性变换加上归一化),将任意一个圆锥曲线变为另一个圆锥曲线——比如将一个椭圆变为一个双曲线或抛物线。

换句话说:
\begin{itemize}
\item 椭圆是在射影平面中与无穷远直线没有实交点的圆锥曲线;
\item 双曲线是在射影平面中与无穷远直线有两个实交点的圆锥曲线;
\item 抛物线是恰好与无穷远直线有一个交点的极限情形。
\end{itemize}

这种分类在射影几何中失去了意义,因为无穷远直线被作为与其他直线同等地位来处理,不再是“例外的部分”。因此,抛物线、椭圆和双曲线不再是本质不同的几何对象,而只是一个对象的不同投影或表示。


为什么 $e=1$ 是一个“分界线”?

离心率为什么只分成这三类,而不是连续变化出更多种曲线?

四、重新看焦点和准线:变换下的对等性

回到的统一定义:

到焦点距离与到准线距离的比值等于 $e$

焦点是一个点,准线是一个直线,它们是不一样的。但在射影几何中,可以把直线看成是“一个方向上的点的集合”,特别是在加入了“无穷远点”之后,直线也可以被看作是特殊的“点”。

这就让焦点和准线,在某种意义上变得“对等”。

更重要的是:

在射影几何中,通过变换,可以把一个圆锥曲线变换成另一种类型的圆锥曲线,只要它们满足相同的基本结构。

举个例子:
	•	一个椭圆,通过一个适当的“射影变换”,可以变成一个抛物线;
	•	抛物线也可以变成双曲线;
	•	这些变换不会改变圆锥曲线的“本质”,只改变它在眼中的“样子”。

这就说明,射影几何的世界中,圆锥曲线是一个统一的整体,而不是三种各自孤立的图形。


在射影几何中,圆锥曲线的统一定义并不依赖焦点–准线,而是:

所有与一条圆锥面相交的平面交线,在射影平面中都是圆锥曲线。其本质是一个二次齐次方程在 $\mathbb{P}^2$ 中的零点集合。

但——

焦点–准线结构仍然可以嵌入射影几何中,你可以这样理解:
	•	准线可以是一个射影直线;
	•	焦点是一个射影点;
	•	离心率可以通过某种射影不变量(例如交比)来表达。

在射影几何中,“一个点到一条线的比值”不再有意义,但你可以通过共轭二次曲线、交比等射影结构,重新定义出类似“焦点–准线”的行为。

1. “双焦点”与“单焦点”的统一
	•	问题:我们习惯使用两个焦点描述椭圆/双曲线,而这里却只用一个焦点和一条准线,为什么可以?
	•	解读:“另一个焦点”其实可以看作是准线的对偶,它不是必须的,只是在对称性下自然出现。

2. 对偶结构初探
	•	可引入投影几何中“点–直线对偶”的思想,让学生意识到准线和焦点在某种意义下可以互换角色。

此外,射影几何还强调了极点与极线的对偶性,并引入了极线极点变换的工具来研究圆锥曲线的性质,使得很多命题具有了对称且优美的形式。例如:对于一个给定的圆锥曲线,任意一点都有与之对应的一条极线,反之亦然。这种对偶关系在欧氏几何中并不自然存在。对于任意一条圆锥曲线,定点到定直线的垂线方向就是该曲线的对称轴。而对于椭圆和双曲线,由于它们具有两条对称轴,还可以通过另一条对称轴构造出第二组对应的焦点与准线。当然,对于圆,所有的焦点都与圆心重合,由于有无数条对称轴,那些对应的准线都在无穷远处,这一点会在后面说明。