% 数值计算的误差

\pentry{Python 函数\upref{PyFunc}}

\subsection{误差}
\footnote{本文经作者同意转载自知乎专栏 \href{https://www.zhihu.com/column/c_1226443594048942080}{《科学计算》}, 格式有少量修改.}\textbf{误差(errors)}的基本概念在高中物理里面应该有所涉及,这里就不仔细展开了. 对于科学计算而言,我们所关注的主要是\textbf{相对误差(relative error)}和\textbf{绝对误差(absolute error)}.

如果我们把一个数据的实际值记做 $x$ , 把它的近似值记为 $\hat x$. 在这个近似的过程,误差产生的原因主要有:
\begin{itemize}
\item 测量中的误差
\item 运算和计算机表达中的不精确所引入的机器误差或者舍入误差(round-off error)
\item 数值方法和离散化带来的误差(discretization error)
\end{itemize}
尤其是后面两种,是科学计算中要面临的两个主要的误差来源,下面我们会对它们进行逐一分析. 因此,在评估和使用数值方法时,我们需要系统的衡量这两项误差,并且要\textbf{掌握它们的来源},对它们的\textbf{大小有足够的控制}.

\subsection{范数}
在了解误差以前,我们先来回顾一下线性代数中的一个常规概念,范数.

科学计算中经常会涉及到向量(vector),矩阵(matrix)和张量(tensor). 计算与它们有关的误差,需要使用更为一般化的“绝对值”函数,也就是范数 $\norm{\cdot}$.

下面是几个常见的范数,其中向量由小写字母表示,矩阵由大写字母表示.
\begin{itemize}
\item 1-范数:  $\|v\|_1=(|v_1|+|v_2|+\cdots+|v_N|)$
\item 2-范数:  $\|v\|_2=(v_1^2+v_2^2+\cdots+v_N^2)^{\frac{1}{2}}$
\item $p$-范数:  $\|v\|_p=(v_1^p+v_2^p+\cdots+v_N^p)^{\frac{1}{p}},\quad p>0$
\item $\infty$  -范数,  $\|v\|_{\infty}=\max_{1\le i \le N}|v_i|$
\item 矩阵的 $p$-范数:  $\|A\|_p=\max_{\|u\|_p=1}\|Au\|_p$
\item Frobenius-范数:  $\|A\|_F=(\sum_{i=1}^m\sum_{j=1}^n|a_{ij}|^2)^{\frac{1}{2}}$
\end{itemize}

这些范数可以用 Matlab 的 \verb|norm()| 函数或者用 Python 中的 \href{https://docs.scipy.org/doc/numpy/reference/generated/numpy.linalg.norm.html}{numpy.linalg.norm} 函数进行计算. 那么绝对误差的用范数表达式为 $\epsilon=\|x-\hat{x}\|$,相对误差为  $\tau={\|x-\hat{x}\|}/{\|x\|}$.

\subsection{例 1}
\verb|absrelerror(corr, approx)| 函数使用默认的2-范数,计算 \verb|corr| 和 \verb|approx| 之间的绝对和相对误差.

\begin{lstlisting}[language=python]
import numpy as np

def absrelerror(corr=None, approx=None):
    """ 
    Illustrates the relative and absolute error.
    The program calculates the absolute and relative error. 
    For vectors and matrices, numpy.linalg.norm() is used.

    Parameters
    ----------
    corr : float, list, numpy.ndarray, optional
        The exact value(s)
    approx : float, list, numpy.ndarray, optional
        The approximated value(s)

    Returns
    -------
    None
    """

    print('*----------------------------------------------------------*')
    print('This program illustrates the absolute and relative error.')
    print('*----------------------------------------------------------*')

    # Check if the values are given, if not ask to input
    if corr is None:
        corr = float(input('Give the correct, exact number: '))
    if approx is None:
        approx = float(input('Give the approximated, calculated number: '))

    # be default 2-norm/Frobenius-norm is used
    abserror = np.linalg.norm(corr - approx)
    relerror = abserror/np.linalg.norm(corr)

    # Output
    print(f'Absolute error: {abserror}')
    print(f'Relative error: {relerror}')
\end{lstlisting}

\subsection{舍入误差(Round-off error)}

\textbf{舍入误差}, 有时也称为\textbf{机器精度(machine epsilon)},简单理解就是\textbf{由于用计算机有限的内存来表达实数轴上面无限多的数,而引入的近似误差}.

换一个角度来看,这个误差应该不超过计算机中可以表达的相邻两个实数之间的间隔. 通过一些巧妙的设计,例如 \href{https://ieeexplore.ieee.org/document/8766229}{IEEE 标准}.

这个间隔的绝对值可以随计算机所要表达的值的大小变化,并始终保持它的相对值在一个较小的范围. 关于浮点数标准,我们会在后面的一章 “计算机算术” 里面具体解释和分析.

\subsubsection{例子 2: 机器精度}

使用之前定义的 \verb|absrelerror()| 函数,在 Python3 环境下,运行下面的程序.
\begin{lstlisting}[language=pythonC]
a = 4/3
b = a-1
c = b + b +b
print(f'c = {c}')
absrelerror(c, 1)
\end{lstlisting}

输出:
\begin{lstlisting}[language=pythonC]
c = 0.9999999999999998
*----------------------------------------------------------*
This program illustrates the absolute and relative error.
*----------------------------------------------------------*
Absolute error: 2.220446049250313e-16
Relative error: 2.2204460492503136e-16
\end{lstlisting}

\subsubsection{例子3: 机器精度 2}
运行下面的四行代码,观察输出的结果是否存在误差.
\begin{lstlisting}[language=pythonC]
print(1e-20+1e-34)
print(1e-20+1e-35)
print(1e-20+1e-36)
print(1e-20+1e-37)
\end{lstlisting}
输出
\begin{lstlisting}[language=pythonC]
1.0000000000000099e-20
1.000000000000001e-20
1.0000000000000001e-20
1e-20
\end{lstlisting}

由上面两个例子可以观察到,机器精度约为  $10^{-16}$. 值得注意的是,在例子2中, 如果我们直接令 \verb|b=1/3| ,再求和,则不会得到任何误差. 事实上,这个舍入误差出现在 \verb|b=a-1| 时,并且延续到了后面的求和计算. 而对于例子3,求和时当两个数的相对大小差距超过  $10^{-16}$  (或  $10^{16}$)时,较小的数的贡献会消失.这种现象在有些地方也被称作cancellation error. 事实上,它只是舍入误差的一种表现. 至于具体的原理,也会在后面计算机算术这一章和浮点数标准中详细介绍.

\subsubsection{例子4: 矩阵运算中的误差}

下面定义一个测试函数 \verb|testErrA(n)| ,随机生成  $n\times n$  矩阵  $A$  ,然后计算  $A^{-1}A$  与单位矩阵  $I$  之间的误差.

理论上,这两个值应该完全相等,但是由于存在舍入误差, $A^{-1}A$ 并不完全等于单位矩阵.
\begin{lstlisting}[language=python]
# Generate a random nxn matrix and compute
# A^{-1}*A which should be I analytically
def testErrA(n = 10):
    A = np.random.rand(n,n)
    Icomp = np.matmul(np.linalg.inv(A), A)
    Iexact = np.eye(n)
    absrelerror(Iexact, Icomp)
\end{lstlisting}

调用函数可得到类似如下的输出:
\begin{lstlisting}[language=pythonC]
testErrA()
*----------------------------------------------------------*
This program illustrates the absolute and relative error.
*----------------------------------------------------------*
Absolute error: 6.626588434229317e-15
Relative error: 2.095511256869353e-15
\end{lstlisting}

由于矩阵  $A$  的随机性,这里的输出并不会完全一致,但相对误差始终保持在 $10^{-14}$  至  $10^{-16}$  这个范围. 我们也可以尝试不同大小的矩阵,误差也会随着矩阵尺寸变大而增大.

\textbf{注1}:对于矩阵求逆运算的机器精度估算,涉及到矩阵的条件数(condition number),这个会在后面求解线性方程时具体分析.

\textbf{注2}:矩阵求逆的运算复杂度约为  $\mathcal{O}(n^2)$  或以上,因此继续增大  $n$  有可能会让程序的运行时间大大增加.

\subsection{离散化误差}
\pentry{泰勒级数\upref{Taylor}}

顾名思义,\textbf{离散化误差(Discretization error)}是由数值方法的离散化所引入的误差.通常情况下,离散化误差的大小与离散化尺寸直接相关. 以前向差分(forward difference)为例,函数  $f(x)$  的一阶导数可以近似为:
\begin{equation}
f'(x)\approx \frac{f(x+h)-f(x)}{h}
\end{equation}

其中,  $h$  为网格尺寸或步长. 通过泰勒展开\upref{Taylor}可知,前向差分的离散化误差为 $\mathcal{O}(h)$,即每当 $h$ 缩小到它的一半,则误差也相应的缩小一半.

类似的,对于\textbf{中央差分(central difference)}和\textbf{五点差分(five-points difference)}

\begin{equation}
f'(x)\approx \frac{f(x+h)-f(x-h)}{2h},
\end{equation}
\begin{equation}
f'(x) \approx \frac{-f(x+2h)+8f(x+h)-8f(x-h)+f(x-2h)}{12h}.
\end{equation}

它们的离散化误差分别为 $\mathcal{O}(h^2)$ 和  $\mathcal{O}(h^4)$  .对应的,当  $h$  缩小一半,误差分别变为原来的 $1/4$ 和 $1/16$.

\subsubsection{例子 5}

前向差分的离散化误差

取 $f(x)=e^x$,可知  $f'(x)=\E^x$  .我们用前向差分来计算  $f(x)$  的一阶导数,把它们的值和真实值对比,并分别画出  $h=0.2, 0.1$  和  $0.05$  的结果.

\begin{lstlisting}[language=python]
import matplotlib.pyplot as plt
import numpy as np

def ForwardDiff(fx, x, h=0.001):
    # ForwardDiff(@fx, x, h);
    return (fx(x+h) - fx(x))/h

def testDiscretization(h=0.1, l=0, u=2):
    # compute the numerical derivatives
    xh = np.linspace(l, u, int(abs(u-l)/h))
    fprimF = ForwardDiff(np.exp, xh, h)
    return xh, fprimF

# The exact solution
Nx = 400
l, u = 0, 2
x = np.linspace(l, u, Nx)
f_exa = np.exp(x)

# Plot
fig, axs = plt.subplots(3, 1, figsize=(6,6))
fig.tight_layout(pad=0, w_pad=0, h_pad=2)

for i, ax in zip(range(4), axs):
    ax.plot(x, f_exa, color='blue')
    xh, fprimF = testDiscretization(h=0.2*0.5**i)
    ax.plot(xh, fprimF, 'ro', clip_on=False)
    ax.set_xlim([0, 2])
    ax.set_ylim([1,max(fprimF)])
    ax.set_xlabel(r'$x$')
    ax.set_ylabel('Derivatives')
    ax.legend(['Exact Derivatives','Calculated Derivatives'])
\end{lstlisting}

\begin{figure}[ht]
\centering
\includegraphics[width=14cm]{./figures/NumErr_1.png}
\caption{运行结果} \label{NumErr_fig1}
\end{figure}

可以观察到我们用前向差分得到的导数随着 $h$ 的减小,逐渐靠近真实值.

\subsection{小结:舍入误差 vs. 离散化误差}

我们通过下面这个例子来分析和对比这两个误差.

\subsubsection{例子 6: 对比两个误差}

继续例子 4 中的函数 $f(x) = \E^x$ 和它的导数 $f'(x) = \E^x$. 取 $x=1$, 分别用前面提到的三种有限微分方法求出数值导数,并与真实值比较计算出相对误差.

这里我们来观察,当 $h$ 取不同值($10^{-1}$ 至 $10^{-15}$)的时候, 相对误差的变化情况.

\begin{lstlisting}[language=python]
import matplotlib.pyplot as plt
import numpy as np

def ForwardDiff(fx, x, h=0.001):
    # Forward difference
    return (fx(x+h) - fx(x))/h

def CentralDiff(fx, x, h=0.001):
    # Central difference
    return (fx(x+h) - fx(x-h))/h*0.5

def FivePointsDiff(fx, x, h=0.001):
    # Five points difference 
    return (-fx(x+2*h) + 8*fx(x+h) - 8*fx(x-h) + fx(x-2*h)) / (12.0*h)

# choose h from 0.1 to 10^-t, t>=2
t = 15
hx = 10**np.linspace(-1,-t, 30)

# The exact derivative at x=1
x0 = 1
fprimExact = np.exp(1)

# Numerical derivative using the three methods
fprimF = ForwardDiff(np.exp, x0, hx)
fprimC = CentralDiff(np.exp, x0, hx)
fprim5 = FivePointsDiff(np.exp, x0, hx)

# Relative error
felF = abs(fprimExact - fprimF)/abs(fprimExact)
felC = abs(fprimExact - fprimC)/abs(fprimExact)
fel5 = abs(fprimExact - fprim5)/abs(fprimExact)

# Plot
fig, ax = plt.subplots(1)
ax.loglog(hx, felF)
ax.loglog(hx, felC)
ax.loglog(hx, fel5)
ax.autoscale(enable=True, axis='x', tight=True)
ax.set_xlabel(r'Step length $h$')
ax.set_ylabel('Relative error')
ax.legend(['Forward difference',
    'Central difference', 'Five points difference'])
<matplotlib.legend.Legend at 0x12066aad0>
\end{lstlisting}

\begin{figure}[ht]
\centering
\includegraphics[width=14cm]{./figures/NumErr_2.png}
\caption{运行结果} \label{NumErr_fig2}
\end{figure}

这是一个非常有趣的结果,我们发现

\begin{itemize}
\item 这三种方法的相对误差,随着 $h$  的不同,呈现出两种不同的特性.
\item 当  $h$  较大时(右半侧),误差曲线相对规则,是由离散化误差主导的.
\item 当  $h$  较小时(左半侧),误差曲线波动较大,是由舍入误差主导.
\end{itemize}

注:舍入误差主导且随着 $h$ 减小而增大的主要原因是:
\begin{itemize}
\item 当 $h$ 较小时,有限差分法的分子上相近的数相减会造成类似例子2和3中的舍入误差.
\item 由于这个例子中的  $f(x)$ 在  $e$  附近,因此这个误差应为 $10^{-16}$ 左右.它除以较小的 $h$ 时,就会被相应的放大,当  $h$  越小,这个舍入误差越大.
\end{itemize}

因此,\textbf{当我们使用上述方法时,需要注意,尽可能取}  $h$  \textbf{在误差曲线的右半侧,这样我们对于误差才有完全的控制.}

同时,我们也观察到,对于越是高阶的差分(如五点差分),它的离散化误差随  $h$  的下降速率越大,但也越早到达舍入误差的区域. 因此,当我们遇到类似问题上,应\textbf{选择合适阶数的有限差分方法,并根据它的特性选择适合的}  $h$  \textbf{值.并不一定是越高阶的方法越好,}  $h$  \textbf{越小越好.}

本文中所用函数的完整代码见 \href{https://github.com/enigne/ScientificComputingBridging/blob/master/Lab/L2/measureErrors.py}{github.}
