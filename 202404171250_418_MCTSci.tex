% 蒙特卡洛树搜索算法(实现 TicTacToe 机-机对战)
% keys 蒙特卡洛树搜索算法
% license Usr
% type Tutor

\pentry{蒙特卡洛树搜索算法(理论)\nref{nod_MCTS}}{nod_d3cd}

前面已经讨论过蒙特卡洛树搜索算法的理论,下面通过讲解例题进行实战练习,这利于我们更深入地理解这算法。首先回顾例题:
\begin{example}{}
使用蒙特卡洛树搜索算法实现一个\textbf{机器 vs. 机器}的 Tic-Tac-Toe 井字棋对战游戏。Tic-Tac-Toe 有以下规则:
\begin{enumerate}
\item 井字棋棋盘是 $n \times n$ 的正方形网格棋盘,例如下面是一个有一些棋子的 $7 \times 7$ 的棋盘:
\begin{figure}[ht]
\centering
\includegraphics[width=6cm]{./figures/4d8426da153f2ff1.png}
\caption{棋盘例} \label{fig_MCTSci_2}
\end{figure}
\item 游戏的下法是有 $\cross$、$\bigcirc$ 两方,每次都可以在没有棋子的正方形内部落子。输赢定义为:最先有\textbf{连续} $m$ 个我方棋子出现的一方获胜。
\item \textbf{连续}的定义是:横向、纵向,或两个 $45^\circ$ 对角线方向。
\end{enumerate}

你需要使得以下内容是\textbf{可以自定义}的:
\begin{enumerate}
\item $n$ ,即棋盘大小可以自定义。
\item $m$ ,即输赢的(棋子连续数)条件可以自定义。
\item 先下棋的一方可以自定义。也就是谁第一步下棋可以自定义。
\end{enumerate}

这是一个工程问题,不用考虑时间限制。你需要输出机器对战的中间过程、最终的结果棋盘与谁赢得了这场对战。
\end{example}
在实现蒙特卡洛树搜索这个算法前,我们需要先做一些准备,定义好游戏的各种内容类。
\subsection{预备工作——游戏规则相关}
因为要处理连续问题,我们可以使用“求和”的方式检查,故可以取特殊值 $1$ 与 $-1$ 表示两种棋子 $\bigcirc$ 和 $\cross$。定义棋子类表示棋子:
\begin{lstlisting}[language=python]
class Chess:
    def __init__(self, name: str, val: int) -> None:
        self.name: str = name # X/O
        self.val: int = val

    def __repr__(self) -> str:
        return f"Chess Object({str(self.name)}, {str(self.val)})"
\end{lstlisting}

其中 \verb`__init__` 方法相当于是构造函数,\verb`__repr__` 方法提供了一个将类转化为 \verb`str` 的方式。具体原理是利用了 python 的魔法方法。

然后对于每一步操作,我们可以考虑为是落点与棋子类型的组合。故定义一个操作类 \verb`Move`:
\begin{lstlisting}[language=python]
# 操作
class Move:
    def __init__(self, x: int, y: int, chess: Chess) -> None:
        self.x: int = x
        self.y: int = y
        self.chess: Chess = chess

    def __repr__(self) -> str:
        return (
            "Move Object[("
            + str(self.x)
            + ", "
            + str(self.y)
            + ") , "
            + str(self.chess)
            + "]"
        )
\end{lstlisting}
然后就可以定义以下常量便于我们在后面使用:
\begin{lstlisting}[language=python]
STATUS = {0: " ", 1: "O", -1: "X"}
X = Chess(STATUS[-1], -1)
O = Chess(STATUS[1], 1)
\end{lstlisting}
其中 \verb`STATUS` 常量存储了后面棋盘中每个位置的数字代表这个位置的状态的情况,\verb`X`、\verb`O` 分别对应 $\cross$、$\bigcirc$ 两种棋子。

\subsection{搜索的状态——棋盘}
显然对于这个问题来说,搜索的状态应该是当前棋盘的情况。我们定义一个表示状态的类 \verb`State` 并实现一些方法帮助我们在后面进行搜索。
首先考虑其构造函数,需要记录的信息,显然有当前棋盘的情况(使用 \verb`numpy` 提供的 \verb`np.array` 来表示)、下一步应当哪方下棋。我们额外开辟一个属性用来记录需要多少连续棋子可以赢得这场游戏。故可以写出 \verb`__init__` 方法:
\begin{lstlisting}[language=python]
class State:
    def __init__(self, nxtMove,
                checkerboardStat: np.array,
                winNeed: int = -1) -> None:
        """
        Args:
                nxtMove: 接下来该谁下棋了
                checkerboardStat (2 D 网格棋盘):
                    棋盘状态
                winNeed (int, optional):
                    连续多少个棋子可获得胜利. Defaults to -1.
        """
        if len(checkerboardStat.shape) != 2:
            raise Exception(
                "checkerboardStat must be 2D array")

        if (checkerboardStat.shape[0] !=
            checkerboardStat.shape[1]):
            raise Exception(
                "checkerboardStat must be square")

        self.checkerboard: np.array = checkerboardStat
        if winNeed == -1:
            winNeed = self.checkerboard.shape[0]
        self.winNeed = winNeed
        self.nxtMove: Chess = nxtMove
\end{lstlisting}
在声明属性的时候尽量使用“属性名: 类型=值”的方法,这有助于我们后续实现代码。