% 双曲线(高中)
% keys 极坐标系|直角坐标系|圆锥曲线|双曲线|渐近线
% license Xiao
% type Tutor

\begin{issues}
\issueDraft
\end{issues}

\pentry{解析几何\nref{nod_JXJH},圆\nref{nod_HsCirc}}{nod_fd80}

当然可以,下面是扩写后的第一段,保持面向高中生的风格:

在海上搜救任务中,时间就是生命。设想一艘遇险的船只无法精确报告自己的位置,救援队怎么办?他们会派出两艘配备雷达的救援船,分别测出该船到自己的距离。虽然不知道具体在哪,但能知道:这艘船与两个雷达之间的距离差是固定的。而所有满足这个“距离差为定值”的点,连起来就构成一条双曲线。换句话说,只要根据测得的距离差画出一条双曲线,遇险船只的位置一定就在这条曲线上。再结合其他信息,比如海流、风向,就能快速缩小搜救范围。这正是双曲线在实际生活中展现威力的例子,也是它在导航、雷达、通信等领域被广泛应用的原因。


你有没有注意过望远镜的镜片边缘,或者电台信号塔的形状?在这些看似普通的结构背后,藏着一个优雅又神秘的数学图形——双曲线。

双曲线不像圆那么对称可爱,也不像抛物线那么一眼看穿。它看上去像两个永远追不上的兄弟,总是对着一个方向张开,却从不相交。有人说,双曲线是“距离之差为定值”的轨迹——这听起来像是在讲两个焦点之间的“拉锯战”。

举个例子:你站在一个房间的两个扬声器之间,发现无论你怎么移动,总能找到一个位置,那里两个扬声器传来的声音大小之差是固定的。连起这些点,就是一条双曲线。

在数学上,双曲线有许多有趣的性质,比如:
	•	有两个“分离”的分支,却共用一个坐标系的中心;
	•	有两条对称轴,图形看起来非常“规整”;
	•	还有叫做“渐近线”的神奇直线,双曲线永远靠近它们,但永不触碰,就像永恒的“靠近而不达”。

学习双曲线,不仅是为了应对考试中的选择题、解答题,更是为了理解世界中那些“非圆即直”的奇妙构造。走进双曲线,你会发现,数学不仅有形状,还有故事。

\subsection{双曲线的几何定义}

\subsection{“反比例函数是双曲线”}

如前面所说,第一次听到双曲线这个词,最有可能的就是在初中学习反比例函数的时候了。相信大多数人在学习时都听过这句“反比例函数的图像是双曲线。”至于什么是双曲线?在当时朴素的概念里,应该就是说它是“两根”“曲线”,至于再细的,老师不会说,学生自然也不会问。下面就对照前面推导的结果,来看看反比例函数是双曲线这件事。

\begin{example}{求反比例函数$\displaystyle y={1\over x}$的图像逆时针旋转$45^\circ$后的表达式。}
根据\aref{图像旋转的规律}{sub_FunTra_3},将$\displaystyle y={1\over x}$逆时针旋转$45^\circ$,即$\displaystyle\theta=-{\pi\over4}$。
\begin{equation}\label{eq_Hypb3_2}
\begin{cases}
\displaystyle
X_0&=X_1 \cos \left(-{\pi\over4}\right) + Y_1 \sin \left(-{\pi\over4}\right)\\
Y_0&=Y_1 \cos \left(-{\pi\over4}\right) - X_1 \sin \left(-{\pi\over4}\right)\\
\end{cases}\implies
\begin{cases}
X_0&={1\over\sqrt{2}}(X_1 - Y_1)\\
Y_0&={1\over\sqrt{2}}(Y_1+ X_1) \\
\end{cases}~.
\end{equation}
将\autoref{eq_Hypb3_2} 代入$xy=1$有:
\begin{equation}
\displaystyle
{1\over\sqrt{2}}(X_1 - Y_1)\cdot{1\over\sqrt{2}}(Y_1+ X_1)=1\implies {X_1^2\over2}- {Y_1^2\over2}=1~.
\end{equation}
即旋转后的方程是:
\begin{equation}
{x^2\over2}- {y^2\over2}=1~~.
\end{equation}
\end{example}

他的图像是中心位于原点,顶点分别在 $x = \pm \sqrt{2}$ 处,渐近线为$y = \pm x$的离心率为$\sqrt{2}$的双曲线。由于旋转变换不会改变形状,因此原本的反比例函数$\displaystyle y={1\over x}$的图像是双曲线,且渐近线为$x$轴和$y$轴。至此,初中阶段就接触到的“反比例函数图像是双曲线”得到了证明。

\subsection{双曲线的方程}

与椭圆不同的是,双曲线的实轴与虚轴并不因为$a,b$值的大小而影响,而是完全取决于符号。
\begin{theorem}{双曲线的标准方程}
\begin{itemize}
\item 实轴在$x$轴上,虚轴在$y$轴上的双曲线方程为:
\begin{equation}\label{eq_Hypb3_4}
\frac{x^2}{a^2} - \frac{y^2}{b^2} = 1~.
\end{equation}
\item 实轴在$y$轴上,虚轴在$x$轴上的双曲线方程为:
\begin{equation}
\frac{y^2}{b^2} -\frac{x^2}{a^2}  = 1~.
\end{equation}
\end{itemize}
\end{theorem}

\subsection{渐近线}
\begin{figure}[ht]
\centering
\includegraphics[width=4.8cm]{./figures/eb70b650d9fa932b.pdf}
\caption{双曲线的渐近线} \label{fig_Hypb3_1}
\end{figure}

当 $x,y$ 都无穷大时, \autoref{eq_Hypb3_4} 中的 $1$ 可以忽略不计,有 $y/x = \pm b/a$,\enref{渐近线}{Asmpto}与 $x$ 轴夹角为
\begin{equation}\label{eq_Hypb3_1}
\theta_0 = \arctan(b/a)~.
\end{equation}
两条渐近线到两个焦点的距离都为
\begin{equation}\label{eq_Hypb3_11}
c\sin\theta_0 = c\cdot b/c = b~.
\end{equation}


事实上这么推导渐近线并不严谨, 在学习了高数的相关内容(见“\enref{泰勒展开}{Taylor}”)后,由\autoref{eq_Hypb3_4} 得
\begin{equation}
y = \frac{bx}{a} \sqrt{1-\frac{a^2}{x^2}}~.
\end{equation}
把根号部分关于 $a^2/x^2$ 进行泰勒展开, 有
\begin{equation}\label{eq_Hypb3_13}
y = \frac ba x - \frac{ab}{2x} + \order{\frac{1}{x^3}}~.
\end{equation}
所以当 $x\to\infty$ 时, 就有渐进线 $y = bx/a$。 之所以要这样做, 是为了防止\autoref{eq_Hypb3_13} 右边出现常数项。 如果存在常数项 $\lambda$, 那么双曲线的渐近线就是 $y = bx/a + \lambda$ 了。

\subsection{双曲线的性质}
面积









