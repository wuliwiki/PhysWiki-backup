% 约翰·冯诺依曼(综述)
% license CCBYSA3
% type Wiki

本文根据 CC-BY-SA 协议转载翻译自维基百科\href{https://en.wikipedia.org/wiki/John_von_Neumann}{相关文章}。

\begin{figure}[ht]
\centering
\includegraphics[width=6cm]{./figures/c4c9226c223e913e.png}
\caption{冯·诺依曼在1940年代} \label{fig_Neuman_1}
\end{figure}
约翰·冯·诺依曼(John von Neumann,1903年12月28日—1957年2月8日)是一位匈牙利裔美国数学家、物理学家、计算机科学家和工程师。冯·诺依曼可能是他那个时代涵盖面最广泛的数学家之一,他将纯粹科学和应用科学相结合,并对许多领域作出了重要贡献,包括数学、物理学、经济学、计算机学和统计学。他是量子物理学数学框架建设的先驱,在泛函分析和博弈论的发展中也做出了突出贡献,提出或规范了包括细胞自动机、通用构造器和数字计算机等概念。他对自我复制结构的分析,早于DNA结构的发现。

在第二次世界大战期间,冯·诺依曼参与了曼哈顿计划,他开发了用于爆炸透镜的数学模型,这些透镜在内爆型核武器中起到了重要作用。战前和战后,他为许多组织提供咨询服务,包括科学研究与发展办公室、陆军弹道研究实验室、武装部队特殊武器计划和橡树岭国家实验室等。在1950年代的巅峰时期,他主持了多个国防部委员会,包括战略导弹评估委员会和洲际弹道导弹科学顾问委员会。他还是负责全国所有原子能开发的影响力巨大的原子能委员会的成员。在与伯纳德·施里弗和特雷弗·加德纳的合作中,他在美国首个洲际弹道导弹(ICBM)项目的设计和开发中扮演了关键角色。那时,他被认为是美国核武器方面的顶尖专家,也是美国国防部的首席防御科学家。

冯·诺依曼的贡献和智力能力得到了物理学、数学及其他领域同事的高度赞扬。他所获得的荣誉包括自由勋章以及以他名字命名的月球陨石坑。