% 弦论概述
% license Usr
% type Tutor

\begin{issues}
\issueNeedCite
\issueMissDepend
\end{issues}

弦论的基本思想是:组成物质的基本单元不是粒子,而是弦。弦有开弦和闭弦两种,形状如下图所示:
\begin{figure}[ht]
\centering
\includegraphics[width=5cm]{./figures/f62b386191a5ec28.png}
\caption{开弦.} \label{fig_STover_1}
\end{figure}
\begin{figure}[ht]
\centering
\includegraphics[width=5cm]{./figures/60d4108c1648ffd2.png}
\caption{闭弦.} \label{fig_STover_2}
\end{figure}
弦的激发给出了各种不同的粒子。粒子在时空中运动的轨迹叫做\textbf{世界线(world line)}。弦在时空中运动的轨迹叫做\textbf{世界面(world sheet)}。世界面由两个参数 $\sigma$ 和 $\tau$ 来参数化。函数 $x^\mu(\tau,\sigma)$ 把世界面上的坐标 $\sigma$ 和 $\tau$ 映射到时空坐标 $x$ 上。

\begin{figure}[ht]
\centering
\includegraphics[width=5cm]{./figures/c35ed900e670dca5.png}
\caption{世界面 $x^\mu (\tau,\sigma)$.} \label{fig_STover_3}
\end{figure}

\begin{figure}[ht]
\centering
\includegraphics[width=5cm]{./figures/204383cb74c3b2d0.png}
\caption{世界线 $x^\mu(\tau)$.} \label{fig_STover_4}
\end{figure}

在弦论世界里面,组成物质的基本单元是长度大约是普朗克长度($10^{-33}\Si{cm}$)的弦。像所有其他弦一样,这些基本弦能够振动。不同的振动频率形成了不同性质的粒子。对于一个自旋为 $J$ 质量为 $m_J$ 的粒子,质量和自旋通过如下式子联系起来
\begin{align}
J = \alpha' m_J^2~.
\end{align}

弦也能分开和合并。比如弦A可以分开变成弦B和弦C。这个过程对应于粒子衰变
\begin{align}
A \rightarrow B + C~.
\end{align}


