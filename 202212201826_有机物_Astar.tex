% A-star 算法
% 搜索|BFS|C++

A-star 算法的使用场景和双向 BFS 的差不多,都是如果在一个搜索空间非常大的情况下,可能会遍历非常多无需遍历的状态,导致时间效率非常低。由此可以使用 A* 算法。

A* 算法是在优先队列 BFS 的基础上进行优化的,我们新加了一个\textbf{启发函数}的概念,这样就可以优化搜索空间,降低时间复杂度。这个启发函数一般在 A* 算法上是设计了一个\textbf{估价函数},在普通的优先队列 BFS 中,每次只会弹出距离当前点比较小点的临点进行更新,不会考虑未来怎么更新,有可能存在一条距离当前点权值比较大的点,但到未来的点的代价可能很小的点。

所以可以添加一个可以对未来的代价进行预估的\textbf{估价函数},具体地讲,在求最短路的时候,可以存:从起点走到当前点的真实距离,以及从当前点走到终点的估计距离这两个值,在优先队列中使用“当前距离+预估距离”进行扩展,实际意义为当前这条路径到终点的距离。这里的优先队列为小根堆。

A* 算法算法的框架:\begin{lstlisting}[language=cpp]
while (!q.empty())
{
    t <--- 取出优先队列(小根堆)的队头
    当终点第一次出队时,找到了答案,退出循环

    for (枚举 t 的所有出边)
        扩展、将临边入队
}
\end{lstlisting}

当估计距离为 $0$ 时,A* 算法算法变为 Dijkstra。

使用 A* 算法的前提:

设当前状态为 $\texttt{state}$,从起点到当前点的实际距离为 $\texttt{d(state)}$。

\begin{itemize}
\item 从起点到终点的实际距离为 $\texttt{g(state)}$,从起点到终点的估价距离为 $\texttt{f(state)}$。
\item 需要满足 $\texttt{f(state)} \leq \texttt{g(state)}$。
\end{itemize}

简单的证明一下,使用\textsl{反证法}。

假设终点第一次出队时不是最小值。那么此时的 $\texttt{dist}$ 一定严格大于 $\tt d_{\text{最优}}$,此时最优路径中一定存在一个点 $u$,那么存在 $d_u + f_u \leq d_u + g_u = d_{\text{最优}}$。所以 $\texttt{dist} > \tt d_{\text{最优}} \geq d_u + f_u$。此时 $d_u + f_u$ 是一定严格小于终点第一次出队的距离 $\tt dist$ 的,又因为是小根堆,队列中存在一个比已经出队的更新的一个元素,矛盾。

证毕。

性质:A* 算法只能保证\textbf{终点第一次出队的时候是最小值,并不能保证其他点第一次出队是最小值,并且每个点不一定会扩展(入队)一次。}

\begin{figure}[ht]
\centering
\includegraphics[width=14cm]{./figures/Astar_1.png}
\caption{示意图} \label{Astar_fig1}
\end{figure}

证明:

首先假设 $2$ 号点估计距离为 $L$ 为一个很大的数,其他点的估计距离为 $0$,首先从起点开始扩展,因为 $a$ 号点的估计距离加实际距离为 $1$,小于扩展到 $2$ 号点的 $L + 1$,所以会从 $a$ 点开始扩展一直扩展到 $4$ 号点,此时 $4$ 号点的距离为 $6$(从 $a$ 号点开始扩展的,所以会将 $4$ 号点从队列中弹出开始扩展 $5$ 号点,但实际是,从 $2$ 号点开始走到 $4$ 号点的距离这条路径为 $4$,这就证明了上述第一个性质。那 A* 算法是从什么时候发现走错了并且重新走呢?A* 算法有可能会走到 $6$ 号点发现此时距离为 $L + 1$了,走到终点为 $L + 2$,此时就发现走错了,就会重新走,可以发现 $4$ 号点在错误的路径上被扩展了一,在重新走之前的道路上被扩展了一次,所以也证明了每个点不一定会被扩展一次。 

证毕。