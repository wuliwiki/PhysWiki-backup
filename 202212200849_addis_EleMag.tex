% 电磁学笔记(科普)

\begin{issues}
\issueDraft
\end{issues}

\begin{itemize}
\item 两种电荷: 正电荷, 负电荷。 从微观来讲, 负电荷由电子提供, 正电荷由质子提供。 详见 “原子分子笔记(科普)\upref{AtomIn}”
\item 通常来说, 当物体不带电时, 每一个局部都有相同数量的电子和质子, 它们的电荷大小相等符号相反, 所以从宏观来看物体不带电。
\item 同种电荷相吸, 异种电荷相斥。 这样的力叫做\textbf{库仑力}, 或者\textbf{电场力}。
\item 库仑定律: 两个带电粒子之间的库仑力和它们电荷的乘积成正比, 和距离平方成反比。
\item 库仑力的产生: 带电粒子现在其周围形成电场, 处于该电场中的另一个粒子
\end{itemize}
