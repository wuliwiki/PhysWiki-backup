% 波动⼒学 (量子序曲)
% license Usr
% type Art

(本文根据 CC-BY 协议转载自季燕江的《量子序曲》, 进行了重新排版和少量修改)

\subsection{波粒二像性}


我们对粒子和波动的概念来自直接的经验。和粒子有关的经验对象:小到石子大到天上的星星等;和波动有关的经验对象:最常见的例子是水波,还有拨动的琴弦等。但这些还不是物理中所说的模型,物理中所谓粒子和波动是理想化的模型,是我们头脑中抽象的对象。


\subsubsection{粒子的图像}


在经典物理中,粒子的概念可进一步抽象为:大小可忽略不计的具有质量的对象,即所谓质点。质量(mass)在这里是新概念,mass的原初含义就是多少,这里引申为对物体惯性质量多少的量度。一个西瓜,比西瓜籽的质量大,因为西瓜里包含的物质比西瓜籽多,西瓜保持自身运动状态的能力比西瓜籽强。

为叙述的简单,我们现在可把粒子等同于质点。要描述一个质点的运动状态,我们需要知道其位置$x$和速度$v$,速度是对位置的微分。

\begin{equation}
v = \frac{d x}{d t}~
\end{equation}

这是莱布尼茨的记号,我们有时也把它记为$\dot x$,$\dot x$是牛顿的记号。记号会有助于我们思维,比如$\dot x$比较简短,而$frac{d x}{d t}$则会提醒我们微分运算的结构,即微分是如何定义的,粒子在时间$\Delta t$内飞行了$\Delta x$,然后我们对$\Delta t$取极限。

而取极限是构造这样一个可以count的过程,(1)首先,$\Delta t = 1$秒;(2)$\Delta t = 0.1$秒;(3)$\Delta t = 0.01$秒;(4)……。假设运算$\frac{\Delta x }{\Delta t}$在这样一个可以count的过程里趋于一个确定的数,我们就说极限存在。

我们研究数学是为了应用于物理的研究,而物理研究的对象都是确定的现象,这决定了我们特别对这类数学形式感兴趣,比如在这里就是极限存在,如果极限不存在,我们也要努力调整数学结构使其存在,否则我们正在使用的数学对物理研究就是不趁手的。

对时间做微分就必须讨论时间,时间也是一个直观的概念,这里我们可把时间描述为一个时钟,我们会发现当指针指到不同位置时,质点的位置可能不同,于是指针的位置就定义了时刻$t$。有了时刻$t$,我们对质点的描述就变成了$x(t)$,在想象中动起来的质点就是一条线,就是轨迹。由$x(t)$可定义速度$v(t)$,考虑到质点还有质量$m$,我们现在用位置$x$和动量$p$这一对量来表示质点的运动状态。

$x$和$p$并放$(x, p)$就是相空间(phase space)中的一个点。相空间是抽象的数学空间,比如在这里我们要描述一个粒子的运动,位置是三维的($x, y, z$),动量也是三维的($p_x, p_y, p_z$),相空间就是六维的。我们用六维超空间中的一个点来描述一个质点的运动。

在日常经验中我们还有相互作用或所谓力的概念,我们在地球上拎起不同质量物体时肌肉的紧张程度是不同的,或者说在地球上用弹簧秤拎起不同质量物体时弹簧的拉伸程度是不同的。

以上我们对质量、时间、力等的定义都是直观的,是可以操作的。按照以上思路进行研究,研究落体的运动,研究天体的运动,……最终诞生了牛顿的经典力学。这里我们可简单地用两个公式:$F=ma$(牛顿第二定律)
和$F = \frac{GMm}{r^2}$(万有引力公式)
来代表牛顿力学。前者是质点的运动方程,用数学的语言说是一个关于位置$x$的二阶微分方程,根据微分方程的理论,只需要知道初始时刻$t=0$时的位置$x$和速度$v$即可求出以后任意时刻$t$质点所处的位置,即轨迹$x(t)$。

需要强调的是一旦我们知道$t=0$时$x$和$v$的精确值(没任何误差),$x(t)$的取值也是精确的,即我们得到的是对质点未来演化的精确预测,并且这个求解对$t < 0$也精确成立,这意味着我们还可精确地反演质点的历史。这些结论是由牛顿力学的数学结构严格保证的,即轨迹是一根理想的线,它没有宽度。

现在我们就有了一个关于世界的整体图像:宇宙是由很多质点构成的复杂系统,它们两两之间的相互作用由$F=\frac{GMm}{x^2}$决定,对每一个质点我们又可列出$\sum\limits_i F_i = ma$这样的运动方程,$\sum\limits_i
F_i$表示质点所受的合力,与其他质点的位置有关,因此这是一个联立的二阶微分方程组。

还是根据数学的理论,如果我们知道了初始时刻$t=0$时每个质点的位置和速度,我们即可无限精确地知道系统内每个粒子的轨迹。这在哲学上被引申为所谓的“决定论”,我们会倾向于相信:世界只不过是个巨大的机械,人生的命运是确定的,事物的演化也是确定的等等。

当然要想在某一时刻同时测量出全世界所有粒子的位置和速度是不可能的,但这是否意味着——“某一时刻全世界所有粒子具有确定的位置和速度”——本身就不存在呢?有些人可能会持怀疑的态度。另一些人会倾向于相信,当然这种相信并无充分的证据,相信的好处是我们可建立起一个关于世界的整体图像,整个世界变得有秩序了,可以理解了。

不管我们是否相信决定论,牛顿力学本身获得了巨大成功,解释了大到行星运动,小到苹果落地等广泛的现象,因此主流物理学家在100多年前相信牛顿力学提供了描述整个世界的基础。

\subsubsection{波动的图像}

在有了粒子的图像后,我们很容易把波动还原为很多粒子的集体运动。比如最简单的波动,抖动绳子可产生一维波。要解释这样的波动现象,最简单的模型就是假想把绳子分成很多很多份,每部分很小以至我们可将其视为质点,质点间的力用弹性力表示。看起来这就是一根弹簧,上面放了很多等质量的小球,如果你横向摇晃第一个小球,这种运动就会渐次地传递给其他小球,像墨西哥人浪一样,波行进的方向和小球偏离的方向垂直,即所谓横波;如果你纵向压缩拉伸第一个小球,这种纵向的振动也会渐次地传递给其他小球,即所谓纵波,纵波的例子是声波。

由此可见波动是一种整体运动,是由很多粒子参与步调统一的运动。最简单的波动是单色平面波,即体现为正弦或余弦函数:$A\cos(kx - \omega t)$。波动的特点是会传播出去,因此你很难说波在什么地方,或说了也没啥意义,因为它无所不在。对单色平面波来说,波动在每一点的状况都是一样的,但我们可以发现相邻波峰间距离总是相同的,于是可定义波长:$\lambda$,我们还发现每个质点振动一个周期的时间也是相同的,因此可定义频率:$\nu$。

很多质点的整体运动——“波动”——和“粒子”是很不同的,描述粒子的运动,使用位置和动量,描述波动使用振幅$A$、波长$\lambda$和频率$\nu$。粒子的特点是分立的,每个粒子会集中地携带能量$E$和动量$p$,而波动的特点则是弥散的,能量会均匀地分布在介质(中的每个质点)上,波动的能量密度正比于振幅的平方($A^2$)。

粒子是分立的,它们各自在虚空中运行,互相用引力勾连着;而波动是连续的、充盈的整体运动,由波动方程描述。从这个意义上我们说粒子的图像和波动的图像是排斥的,即我们无法想象一个对象既是粒子又是波动。

\subsubsection{电磁波}

尽管粒子的图像和波动的图像是互相排斥的,我们仍会认为粒子的图像更本质,波动的图像可还原为粒子的语言,因此是从属的。但物理学家很快又发现了一种新的波——电磁波。

电磁现象是不同于机械力学(即上面讨论的质点或质点系的运动)的新现象。麦克斯韦是电磁学中的牛顿,他提出的麦克斯韦方程组是解释电磁现象的基础。利用麦克斯韦方程组最重要的预言是“光波就是电磁波”或“电磁波就是光波”。

所谓电磁波就是电场($E$)和磁场($B$)在空间中的传播,和机械波中质点振动会在空间中传播一样,它们都满足类似的波动方程,只是波动传播的速度不同而已。

物理学家自然提出一个任务,即能否把电磁波还原为纯粹的机械运动?追求统一的物理是物理学家永恒的追求,牛顿使天上的物理(行星运动)和地上的物理(苹果落地)统一,麦克斯韦使光学和电磁学统一,那么经典力学和经典电磁学也应该是统一的。但实际上把电磁波还原为机械运动的努力一直没取得啥进展。

如果考虑到光波或电磁波无法还原为机械运动,我们现在可说粒子的图像和波动的图像在概念上是同等重要的,但在量子场论中,粒子被解释为场的激发,这有点把粒子还原为场的意思。

\subsection{双缝实验}

量子力学诞生于原子物理学,即关于原子尺寸物理现象的研究。今天我们知道原子大约是0.1纳米,而人类肉眼可分辨(假设可借助光学显微镜)的尺寸大约是可见光波长的数量级——几百纳米,即我们研究的对象小了至少几万倍。从这个意义上说量子现象是超越于我们日常经验之外的。当我们提到粒子和波动的时候,即便没有系统地学习过物理学,我们也可借助日常经验在“望文生义”的意味下知道粒子大致指的是什么现象,波动指的是什么现象。但当我们讲到原子或电子的运动时,我们就没有这样的直观了。

所以如果不系统地补足物理学史上关于原子物理的研究的话,就必须得有一个机会供我们直观地体验一下量子现象。费曼曾提出著名的双缝实验\footnote{费曼提出单粒子双缝实验时并未真的试图实现它,但随着技术的进步现在已有物理学家完成真正的单电子双缝干涉实验:\url{http://physicsworld.com/cws/article/print/9745}},通过这个实验我们可建立量子力学的基本概念——波粒二像性。

在光学中也有双缝实验,光通过双缝,绕过障碍物,互相叠加,最终在屏上呈现出明暗相间的条纹状分布,这个被称为干涉。干涉现象很容易用波动的图像解释:光是电磁波,当波照射到双缝上时,每个缝相当于是新的波源(惠更斯原理),每个波源都会发出一系列波峰和波谷,当两个波峰相遇时则加强呈现出明亮的条纹,当一个波峰和一个波谷相遇时则抵消呈现出暗条纹。

\begin{figure}[ht]
\centering
\includegraphics[width=6cm]{./figures/6cd12899fd58844e.png}
\caption{双缝⼲涉示意。} \label{fig_QMPre7_1}
\end{figure}

现在我们假设以一束电子入射到双缝上,看看会发生什么现象。电子是量子力学对象,但现在我们先猜测它就是经典的粒子,这种情形下电子穿过双缝——呈上、下两个条状分布。费曼讲的是机枪扫射,看子弹如何穿过双缝,根据日常经验子弹无法绕过障碍,将集中地分布在双缝的方向上。

那么实验的结果是什么呢?是明、暗相间的条纹状分布,就好像光学中的双缝实验结果一样。这是否意味着电子是一种波动呢?就像迄今为止我们都理所当然地认为光就是一种波动,一种电磁波。

我们可以再做实验,让电子一个、一个地通过双缝,看看是否会有干涉现象。实验结果是电子将随机地出现在任意位置,我们根本无法预测电子下一次出现在什么位置。但我们也注意到电子并未弥散开来,每次都只出现在一个位置,这说明电子还是粒子。另外一个特点是当我们进行很多次这样的单电子双缝干涉实验后,电子的总体分布会趋于明暗相间的干涉条纹。

有趣的是,对于我们一直认为是波动的“光”,我们也可完成类似的实验,即当我们降低光的强度,最终我们发现光竟然也是由一个一个的粒子——“光子”组成的。当光强极弱时,我们可完成所谓“单光子干涉实验”,单光子穿过双缝对应一个随机的位置,很多单光子事件累积起来呈现干涉条纹。实际上人眼就是理想的“单光子”探测器,生理实验表明只需要5个光子就可使视杆细胞兴奋。

因此,把电子(或光子)简单地设想为经典的粒子或经典的波动都是不可能的,现在我们说电子(或光子)首先是粒子,但它不是经典的,它的运动状态不能用位置$x(t)$和动量$p(t)$描述,而需要用波函数$\psi(x,t)$来描述。这就是所谓波粒二像性,这与日常经验中的粒子是两回事,但物理学家们一般还称呼它们是粒子。

\subsection{波函数}

根据量子力学,粒子的运动状态是由波函数来描述的,其实经典的波动也是由波函数来描述的。量子力学中的波
函数和经典波动波函数的区别在于:经典波动波函数有确切的物理含义,比如电磁波波函数表示的是变化的电场或磁场;量子力学中波函数不对应确切的物理含义,
它一般是复函数,而物理量(如位置、动量)的取值是实数,但物理系统中所有信息却又都包含在波函数中,即根据波函数我们可求出物理量的取值。从数学形式上
看波函数很类似经典波动的波函数,因为经典波动为计算方便也常常表示为复函数的形式;而量子力学中波函数在某些特定情况下也可表示为实函数的形式。这给思考量子力学问题带来很多直观上的好处,因为想象一个经典波动总是很容易的。

最简单的波函数是单色平面波(plane wave):

\begin{equation}
\psi_k (x, t) = A e^{i(kx -\omega t)}~
\end{equation}

它所描述粒子的动量是:$p = \hbar k$,能量是:$E = \hbar
\omega$(前者是德布罗意的贡献,后者是普朗克的贡献)。动量的表达式很有用,稍作变形:

\begin{equation}
\lambda = \frac{h}{p}~
\end{equation}

这个公式代表了波动语言(左边)和粒子语言(右边)的“翻译”关系。