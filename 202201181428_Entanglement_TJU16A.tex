% 天津大学 2016 年考研量子力学答案
% 考研|天津大学|量子力学|2016|答案

\begin{issues}
\issueDraft
\end{issues}


\subsection{ }
\begin{enumerate}
\item 归一化系数$A=2$,列表如下:\\
\begin{table}[ht]
\centering
\caption{可能的值与相应几率}\label{TJU16A_tab1}
\begin{tabular}{|c|c|c|c|c|}
\hline
  & 几率 & $n$ & $L^{2}=l(l+1)\hbar^{2}$ & $L_{z}=m\hbar$ \\
\hline
$\psi_{210}$ & $\displaystyle \frac{1}{2}$ & 2 & $2\hbar^2$ & 0 \\
\hline
$\psi_{110}$ & $\displaystyle \frac{1}{2}$ & 1 & 0 & 0 \\
\hline
\end{tabular}
\end{table}
\item 不计自旋,守恒量有$\hat H,\hat {L^2},\hat {L_z} $.
\item $\because \quad A^{\dagger} = A,B^{\dagger} = B,(AB)^{\dagger} \neq AB $ \\
$\therefore (1)\quad (AB)^{\dagger} = B^*A^* = BA $,不是厄米算符. \\
$\quad (2)\quad (AB+BA)^{\dagger} = B^*A^*+A^*+B^* = AB+BA $,是厄米算符. \\
$\quad (3)\quad (AB-BA)^{\dagger} = B^*A^*-A^*B^* = -(AB-BA) $,不是厄米算符. \\
$\quad (4)\quad (ABA)^{\dagger} = (ABA)^* = ABA $,是厄米算符. \\
$\quad (5)\quad [i(AB-BA)]^{\dagger} = -i(BA-AB) = i(AB-BA) $,是厄密算符.
\end{enumerate}
\subsection{ }
\begin{enumerate}
\item 由$\displaystyle \psi (x,0) = \frac{1}{\sqrt{2}}[\varphi_{0}(x) + \varphi_{1}(x)] $可得任意时刻的波函数为:\\
\begin{equation}
\psi(x,t) = \frac{1}{\sqrt{2}}\left[\varphi_{0}(x)e^{-\frac{iE_{0}t}{\hbar}} + \varphi_{1}(x)e^{-\frac{iE_{1}t}{\hbar}} \right]
\end{equation}

\item 由谐振子的基本性质,有:\\
\begin{equation}
x\psi_{n} = \frac{1}{\alpha} \left[\sqrt{\frac{n}{2}}\psi_{n-1}(x) + \sqrt{\frac{n+1}{2}}\psi_{n+1}(x) \right]
\end{equation}

所以任意时刻坐标的平均值为:\\
\begin{equation}
\begin{aligned}
\bar{x} =& \int^{\infty}_{-\infty} \psi^{*}(x)\psi \dd{x} \\
=& \frac{1}{2\alpha} \int^{\infty}_{-\infty} \left[\varphi^{*}_{0}(x)+\varphi^{*}_{1}(x)\right](x)\left[\varphi_{0}(x)+\varphi_{1}(x)\right] \dd{x} \\
=& \frac{1}{2\alpha} \left[\sqrt{\frac{1}{2}}\delta_{00}+\sqrt{\frac{1}{2}}\delta_{11} \right] \\
=& \frac{\sqrt{2}}{2} \alpha
\end{aligned}
\end{equation}
其中$\displaystyle \alpha = \sqrt{\frac{m\omega}{\hbar}} $.
\end{enumerate}
\subsection{ }
由一维无限深势阱基本结论得:\\
\begin{equation}
\begin{aligned}
& E^{0}_{n} = \frac{n^{2} \pi^{2} \hbar^{2}}{2ma^{2}},\quad n = 1,2,3\cdots \\
& \psi^{0}_{n} = \sqrt{\frac{2}{a}} \sin{\frac{n \pi x}{a}},\quad n = 1,2,3\cdots
\end{aligned}
\end{equation}

能量的一级修正为:\\
\begin{equation}
\begin{aligned}
E^{(1)}_{n} =& \int^{a}_{\frac{a}{2}} \psi^{(0)*}_{n} (H') \psi^{(0)}_{n} \dd{x} \\
=& \frac{2}{a} \int^{a}_{\frac{a}{2}} V_{0} \sin^{2}{\frac{n \pi x}{a}} \dd{x} \\
=& \frac{2V_{0}}{a} \int^{a}_{\frac{a}{2}} \frac{1}{2} (1-\cos{\frac{2n \pi x}{a}}) \dd{x} \\
=& \frac{V_{0}}{2}
\end{aligned}
\end{equation}
\begin{equation}
\begin{aligned}
H'_{mn} =& \int^{a}_{\frac{a}{2}} \psi^{(0)*}_{m}(H') \psi^{(0)}_{n} \dd{x} \\
=& \frac{2}{a} \int^{a}_{\frac{a}{2}} \sin{\frac{m \pi x}{a}} (V_{0}) \sin{\frac{n \pi x}{a}} \dd{x} \\
=& \frac{2}{a} \int^{a}_{\frac{a}{2}} V_{0} \left[ \frac{1}{2} (\cos{\frac{(m-n) \pi x}{a}} - \cos{\frac{(m+n) \pi x}{a}}) \right] \dd{x} \\
=& \frac{V_{0}}{a} \left[ \frac{a}{(m+n)\pi} \sin{\frac{(m+n)\pi}{a}} - \frac{a}{(m-n)\pi} \sin{\frac{(m-n)}{a}} \right]
\end{aligned}
\end{equation}

能量的二级修正为:\\
\begin{equation}
E^{2}_{n} = \sum_{m}' \frac{\left| H'_{mn} \right|^{2}}{E_{n} - E_{m}}
\end{equation}

波函数的一级修正为:\\
\begin{equation}
\psi^{(1)}_{n} = \sum_{m}' \frac{\left| H'_{mn} \right|}{E_{n} - E_{m}} \psi^{(0)}_{m}
\end{equation}

因此:\\
\begin{equation}
\begin{aligned}
& E = E^{(0)}_{n} + E^{(1)}_{n} + E^{(2)}_{n} \\
& \psi = \psi^{(0)}_{n} + \psi^{(1)}_{n}
\end{aligned}
\end{equation}


\subsection{ }
三维无限深势阱的定态薛定谔方程为:\\
\begin{equation}
\left[ -\frac{\hbar^{2}}{2m} \nabla^{2} +V(x,y,z) \right]\psi(x,y,z) = E\psi(x,y,z)
\end{equation}

分离变量$\psi(x,y,z) = X_(x)Y_(y)Z_(z) $,势阱内$V=0$.\\
\begin{equation}
\frac{-\frac{\hbar^{2}}{2m}\frac{d^{2} {x}^{2}}{d{x^{2}}}}{X_(x)} + \frac{-\frac{\hbar^{2}}{2m}\frac{d^{2} {y}^{2}}{d{y^{2}}}}{Y_(y)} + \frac{-\frac{\hbar^{2}}{2m}\frac{d^{2} {z}^{2}}{d{z^{2}}}}{Z_(z)} = E
\end{equation}

上述第一项只含$x$,第二项只含$y$,第三项只含$z$,$E$与$x,y,z$无关.设第一项为$E_1$利用边界条件$X_{(0)} = 0,X_{(a)} = 0$得:\\
\begin{equation}
X_{n_1}(x) = \sqrt{\frac{2}{a}} \sin{\frac{n_{1} \pi x}{a}},E_{n_1} = \frac{n^2_1 \pi^2 x^2}{2ma^2}
\end{equation}

同理可求得$Y_{n_2},E_{n_2},Z_{n_3},E_{n_3} $,整理得到:\\
\begin{equation}
\begin{aligned}
& E_{n_1,n_2,n_3} =\frac{n^2_1 \pi^2 x^2}{2ma^2}+\frac{n^2_2 \pi^2 y^2}{2ma^2}+\frac{n^2_3 \pi^2 z^2}{2ma^2} \\
& \psi(x,y,z) = (\frac{2}{a})^{\frac{3}{2}} \sin{\frac{n_1 \pi x}{a}} \sin{\frac{n_2 \pi y}{a}} \sin{\frac{n_3 \pi z}{a}} \\
& (0<x,y,z<a \quad , \quad n_1 n_2 n_3 = 1,2,3 \cdots)
\end{aligned}
\end{equation}

当$n_1 = n_2 = n_3$时,能级不简并.$n_1 , n_2 , n_3 $中有二者相等时,一般为三重简并,$n_1 , n_2 , n_3 $三者都不相等时,一般为6度简并,还有偶然简并的情况,如$n_1 , n_2 , n_3 =(5,6,8) $.简并度满足$\displaystyle n^2_1 , n^2_2 , n^2_3 = \frac{2ma^2 E}{\hbar^2 \pi^2}$条件的正整数.

\subsection{ }
\begin{enumerate}
\item 总自旋$S^2$是守恒量,理由为:\\
(1) $S^2$不显含时$t$.\\
(2) $S^2$与各分量的平方对易,进而与各乘积对易,故:\\
\begin{equation}
[S^2 ,H^2] = [S^2 , a \vec{S_1} \vec{S_2} + a \vec{S_2} \vec{S_3} + a \vec{S_3} \vec{S_1}]
\end{equation}

所以$\frac{\dd{S^2}}{\dd{t}}=0$,所以$S^2$是守恒量 .

\item $\displaystyle \vec{S_1} \cdot \vec{S_2} = \frac{1}{2} (\vec{S^2} -\vec{S^2_1} -\vec{S^2_2}) = \frac{1}{2} \vec{S^2} - \frac{3}{4} \hbar^2 $ \\

本征函数为:\\
\begin{equation}
\begin{aligned}
& \chi_{11} = \chi_{\frac{1}{2}}(S_{1z})\chi_{\frac{1}{2}}(S_{2z}) \\
& \chi_{1-1} = \chi_{-\frac{1}{2}}(S_{1z})\chi_{-\frac{1}{2}}(S_{2z}) \\
& \chi_{10} = \frac{1}{\sqrt{2}} \left[ \chi_{\frac{1}{2}}(S_{1z})\chi_{-\frac{1}{2}}(S_{1z}) + \chi_{\frac{1}{2}}(S_{2z})\chi_{-\frac{1}{2}}(S_{1z}) \right] \\
& \chi_{00} = \frac{1}{\sqrt{2}} \left[ \chi_{\frac{1}{2}}(S_{1z})\chi_{-\frac{1}{2}}(S_{1z}) - \chi_{\frac{1}{2}}(S_{2z})\chi_{-\frac{1}{2}}(S_{1z}) \right]  \\
\end{aligned}
\end{equation}

本征值$\hat{H}\psi = E\psi $ \\
\begin{equation}
\begin{aligned}
& \hat{H}\chi_{11} = g(\frac{1}{2}) \cdot 2\hbar - \frac{3}{4} \chi_{11} = \frac{1}{4}g\hbar^{2} \chi_{11}  \\
& \hat{H}\chi_{1-1} = \frac{1}{4}g\hbar \chi_{1-1}  \\
& \hat{H}\chi_{10} = \frac{1}{4}g\hbar \chi_{10}  \\
& \hat{H}\chi_{00} = -\frac{3}{4}g\hbar^{2} \chi_{00}  \\
\end{aligned}
\end{equation}

由上式可得出 ,在自旋对称时,$E$为三重简并.
\end{enumerate}