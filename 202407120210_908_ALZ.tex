% Α粒子
% license CCBYSA3
% type Wiki

(本文根据 CC-BY-SA 协议转载自原搜狗科学百科对英文维基百科的翻译)

阿尔法粒子,也称为阿尔法射线或阿尔法辐射,是由两个质子和两个中子结合成的粒子,与氦-4原子核相同。它们通常是在$\alpha$衰变过程中产生的,但也可能以其他方式产生。$\alpha$粒子是以希腊字母表中的第一个字母$\alpha$命名的。$\alpha$粒子的符号是$\alpha$或$\alpha$²⁺。因为它们与氦核相同,所以有时也被写成$He^{2+}$或4$^4$ $_2He^{2+}$表明氦离子带+2电荷(缺少两个电子)。如果离子从其环境中获得电子,阿尔法粒子就变成正常(电中性)氦原子$^4$ $_2He$。

阿尔法粒子像氦原子核一样,净自旋为零。由于它们在标准$\alpha$放射性衰变中的产生机制,$\alpha$粒子通常具有约5兆电子伏的动能,速度约为光速的5\%。它们是高度电离形式的粒子辐射,并且(当由放射性$\alpha$衰变产生时)具有低穿透深度。它们可以被几厘米的空气阻挡,也可以被皮肤阻挡。

然而,三元裂变产生的所谓长程$\alpha$粒子具有三倍的能量,穿透距离也是三倍。如前所述,构成宇宙射线10-12\%的氦核通常也比核衰变过程产生的氦核能量高得多,因此能够高度穿透并穿越人体,并根据其能量能够穿透数米的致密固体屏蔽。在较小程度上,粒子加速器产生的高能氦核也是如此。

当辐射$\alpha$粒子的同位素被摄入时,它们比半衰期或衰变率所显示的要危险得多,因为$\alpha$辐射具有较高的生物效应,会造成生物损伤。相比比$\beta$辐射或$\gamma$辐射放射性同位素,$\alpha$粒子平均危险约20倍,在吸入$\alpha$辐射源的实验中,危险高达1000倍。[1]

\subsection{命名}
一些科学作者使用双电离氦核($He^{2+}$)和$\alpha$粒子作为可互换的术语。命名法没有很好的定义,因此并不是所有的作者都认为所有的高速氦核都是$\alpha$粒子。与$\beta$和$\gamma$粒子/射线一样,粒子的名称带有一些关于其生产过程和能量的轻微内涵,但这些内涵并不严格适用。[2]因此,当提到恒星氦核反应(例如阿尔法反应)时,甚至当它们作为宇宙射线的组成部分出现时,阿尔法粒子可能被宽泛地用作一个术语。一种称为三元裂变的不寻常核裂变,通常会产生比起阿拉法衰变更高能的阿拉法粒子。然而,由粒子加速器(回旋加速器、同步加速器等)产生的氦核不太可能被称为“$\alpha$粒子”。

\subsection{阿尔法粒子的来源}
\subsubsection{2.1 阿尔法衰变}
\begin{figure}[ht]
\centering
\includegraphics[width=6cm]{./figures/80adf3ba72994d5c.png}
\caption{物理学家在一个云室里从钋源的衰变中观察$\alpha$粒子} \label{fig_ALZ_1}
\end{figure}
$\alpha$粒子最著名的来源是较重(> 106 u原子量)原子的$\alpha$衰变。当一个原子在α衰变中发射α粒子时,由于$\alpha$粒子中四个核子的损失,原子的质量数减少了四。由于失去了两个质子,原子的原子数正好减少了两个——原子变成了一种新的元素。这种核嬗变的例子是铀由于$\alpha$衰变变成钍,或者镭变成氡气体。

阿尔法粒子通常由所有较大的放射性原子核发出,如铀、钍、锕和镭,以及超铀元素。与其他类型的衰变不同,阿尔法衰变只有在原子核大于一定大小时才会发生。迄今为止发现能够发生$\alpha$衰变的最小原子核是铍-8和最轻的碲核素(52号元素),质量数在104和109之间。$\alpha$衰变的过程有时会使原子核处于激发态,通过发射$\gamma$射线移除多余的能量。
\begin{figure}[ht]
\centering
\includegraphics[width=6cm]{./figures/9017bbdaea0ae965.png}
\caption{在异丙醇云室(注入人造源radon-220后)检测到$\alpha$辐射。} \label{fig_ALZ_2}
\end{figure}

\textbf{$\alpha$衰变的产生机制}

与$\beta$衰变相反,导致$\alpha$衰变的基本相互作用是电磁力和核力之间的平衡。阿尔法衰变是由阿尔法粒子和原子核的其余部分之间的库仑排斥造成的,[3]两者都带正电荷,但受到核力的约束。在经典物理学中,阿尔法粒子没有足够的能量逃离原子核中强力产生的势阱(这个势阱包括脱离强力翻过势阱的一侧,随后在另一侧受电磁力排斥而被推开)。

然而,量子隧穿效应允许阿尔法粒子逃脱,即使他们没有足够的能量来克服核力。这是物质的波动性质所允许的,它允许阿尔法粒子在离原子核很远的一个区域停留一段时间,使得排斥电磁力的电势完全补偿了核力的吸引力。从这一点开始,阿尔法粒子可以逃逸,在量子力学中,在一定时间后,它们会逃逸。
\subsubsection{2.2 三元裂变}
核反应中超高能的$\alpha$粒子是在相对罕见的(几百分之一)三元裂变过程中产生的。在这个过程中,裂变产生三个带电粒子,而不是通常的两个,最小的带电粒子很可能(90\%的概率)是α粒子。这种$\alpha$粒子被称为“长程$\alpha$粒子”,因为它们典型能量达到16兆电子伏,远高于$\alpha$衰变产生的能量。三元裂变既发生在中子诱发的裂变(核反应堆中发生的核反应)中,也发生在可裂变的和裂变锕系核素(即能够裂变的重原子)形式经历自发裂变的放射性衰变时。在诱导裂变和自发裂变中,重核中可用的高能量,导致长程α的能量比$\alpha$衰变产生的$\alpha$能量高。
\subsubsection{2.3 加速器}
高能氦核可以由回旋加速器、同步加速器和其他粒子加速器产生,但它们通常不被称为“阿尔法粒子”。
\subsubsection{2.4 太阳核心反应}
如前所述,氦核可以参与恒星中的核反应,偶尔和历史上这些反应被称为$\alpha$反应。
\subsubsection{2.5 宇宙射线}
此外,极高能量氦核有时被称为阿尔法粒子,约占宇宙射线的10\%至12\%。宇宙射线产生的机制仍在争论中。

\subsection{能量和吸收}
在$\alpha$衰变中发射的α的能量略微依赖于发射过程的半衰期,半衰期的许多数量级的差异与小于50\%的能量变化有关。

发射的$\alpha$粒子的能量各不相同,较高能量的$\alpha$粒子从较大的原子核发射,但大多数$\alpha$粒子的能量在3至7兆电子伏之间,分别对应于发射$\alpha$的核素的极长和极短的半衰期。

这种能量对于单个粒子来说是相当大的能量,但是它们的高质量意味着$\alpha$粒子具有较低的速度(典型的动能为5兆电子伏;换算得速度为15,000千米/秒,是光速的5\%),比任何其他常见类型的辐射($\beta$粒子、中子等)都要慢。由于阿尔法粒子的电荷和质量大,它们很容易被物质吸收,在空气中只能传播几厘米。它们可以被薄纸或人类皮肤外层(大约40微米,相当于几个细胞深)吸收。

\subsection{生物效应}
由于吸收范围短且不能穿透皮肤外层,$\alpha$粒子一般不会对生命造成危险,除非来源被摄入或吸入。[4]由于这种高质量和强吸收性,如果发射$\alpha$的放射性核素确实进入人体(在被吸入、摄入或注射时,就像在20世纪50年代之前使用二氧化钍射气获得高质量的$x$光图像那样),$\alpha$辐射是最具破坏性的电离辐射形式。它是最强的电离物质,剂量足够大到会引起任何或所有辐射中毒症状。据估计,$\alpha$粒子造成的染色体损伤比等量的$\gamma$或$\beta$辐射造成的染色体损伤大10至1000倍,平均值设定为20倍。一项对欧洲核工人在内部暴露于钚和铀的$\alpha$辐射的研究发现,当相对生物有效性被认为是20时,$\alpha$辐射的致癌可能性(就肺癌而言)似乎与报告的外部$\gamma$辐射剂量一致,即吸入给定剂量的$\alpha$粒子与高20倍剂量的$\gamma$辐射具有相同的风险。[5]强大的阿尔法发射器钋-210(一毫克的210Po每秒发射的阿尔法粒子与4.215克的226Ra一样多)被怀疑在与吸烟有关的肺癌和膀胱癌中起作用。[6]2006年,钋-210被用来杀害俄罗斯持不同政见者和前联邦安全局官员亚历山大·利特维年科。[7]

\subsection{发现和使用的历史}
\begin{figure}[ht]
\centering
\includegraphics[width=6cm]{./figures/6ad3677255211eee.png}
\caption{$\alpha$辐射由氦-4原子核组成,很容易被一张纸挡住。由电子组成的$\beta$辐射被一块铝板挡住了。$\gamma$射线穿透致密物质时最终被吸收。由于铅的密度,它很容易吸收$\gamma$射线} \label{fig_ALZ_3}
\end{figure}
在1899年和1900年,物理学家欧内斯特·卢瑟福(在加拿大蒙特利尔麦吉尔大学工作)和保罗·维拉德(在巴黎工作)根据物体的穿透和磁场的偏转,将辐射分成三种类型:最终由卢瑟福命名为$\alpha$、$\beta$和$\gamma$。[8]卢瑟福将α射线定义为普通物体穿透力最低的射线。

卢瑟福的工作还包括测量$\alpha$粒子的质量与其电荷的比率,这使他提出了$\alpha$粒子是双电荷氦离子的假设(后来证明是裸氦核)。[9]1907年,欧内斯特·卢瑟福和托马斯·罗伊兹最终证明阿尔法粒子确实是氦离子。为了做到这一点,他们使阿尔法粒子穿透真空管非常薄的玻璃壁,从而在管内捕获大量假设的氦离子。[10]然后,他们在管内产生电火花,提供一簇电子,这些电子被离子吸收,形成气体的中性原子。随后对所得气体光谱的研究表明,它是氦,$\alpha$粒子确实是假设的氦离子。
\begin{figure}[ht]
\centering
\includegraphics[width=8cm]{./figures/922baf27f1f22895.png}
\caption{$\alpha$粒子被磁场偏转} \label{fig_ALZ_4}
\end{figure}
因为$\alpha$粒子是自然产生的,但是能量足够高,可以参与核反应,所以对它们的研究导致了核物理的早期知识。卢瑟福利用溴化镭发射的阿尔法粒子推断出汤姆逊的李子布丁原子模型存在根本缺陷。在卢瑟福的金箔实验中,他的学生汉斯·盖格和内斯特·马斯登(Hans Geiger andErnest Marsden)进行了实验,建立了一股窄束$\alpha$粒子,穿过非常薄(几百个原子厚)的金箔。阿尔法粒子通过硫化锌屏检测,当阿尔法粒子碰撞时,硫化锌屏会发出闪光。卢瑟福设想,如果原子的“李子布丁”模型是正确的,根据预测,正电荷分布分散,带正电荷的α粒子将仅仅稍微偏转,如果有偏转的话。

实验发现,一些阿尔法粒子的偏转角度比预期的大得多(根据卢瑟福的建议进行检查),有些甚至几乎直接反弹回来。尽管大部分阿尔法粒子像预期的那样直接穿过,卢瑟福评论说,如果假设“李子布丁”理论是正确的,少数被偏转的粒子就像向薄纸发射一个15英寸的炮弹,结果只是让它弹开。可以确定,原子的正电荷集中在中心的一小块区域,使得正电荷密度足够大,足以偏转任何接近的$\alpha$粒子,这个带正电的区域后来被称为原子核。

在这一发现之前,人们不知道$\alpha$粒子本身就是原子核,也不知道质子或中子的存在。在这一发现之后,J·J·汤姆逊的“李子布丁”模型被抛弃了,卢瑟福的实验导致了玻尔模型(以尼尔斯·玻尔命名)和后来的现代原子波动力学模型。

1917年,卢瑟福继续使用阿尔法粒子意外地产生了他后来所认识的一种元素到另一种元素的定向核嬗变。自1901年以来,人们已经懂得了自然放射性衰变导致元素从一种元素到另一种元素的嬗变,但是当卢瑟福将$\alpha$衰变产生的$\alpha$粒子投射到空气中时,他发现这产生了一种新型的辐射,这种辐射被证明是氢原子核(卢瑟福将这些质子命名为质子)。进一步的实验表明质子来自空气中的氮成分,并且推断出反应是氮在反应中转化为氧