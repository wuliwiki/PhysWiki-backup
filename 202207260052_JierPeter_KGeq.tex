% Klein-Gordon方程
% 量子场论|高等量子力学|克莱因-戈登方程|克莱茵-戈登方程

\addTODO{预备知识需要薛定谔量子力学的相关内容,但现在该部分还未整理好,不宜引用.}

\pentry{自然单位制、普朗克单位制\upref{NatUni}}

\subsection{问题的引入}

薛定谔方程在量子力学中的地位,就像牛顿三定律在经典力学中的地位一样,是描述理论结构的“公理”.因此,如果要了解量子力学的局限性,可以从研究薛定谔方程本身入手.



回顾单粒子薛定谔方程的表达(注意这里使用了\textbf{自然单位制}\upref{NatUni}):
\begin{equation}\label{KGeq_eq1}
\qty(-\frac{\nabla^2}{2m}+V)\psi = \I \partial_t \psi
\end{equation}

由于$\hat{p}=-\I\nabla$和$\I\partial_t$分别是量子力学中的动量、能量算子,故





















