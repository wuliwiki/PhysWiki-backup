% 薛定谔绘景和海森堡绘景



\pentry{时间演化算符(量子力学)\upref{TOprt}}

\footnote{参考 Wikipedia \href{https://en.wikipedia.org/wiki/Heisenberg_picture}{相关页面}.}薛定谔方程\upref{TDSE}通常使用的是动量表象\upref{moTDSE}和\textbf{薛定谔绘景}, 在海森堡绘景中, 波函数(态矢)不随时间改变, 而测量量的算符随时间改变. 
% 海森堡绘景相当于在薛定谔绘景的基础上做了一个基底变换, 类似于位置和动量表象\upref{moTDSE}的关系.
% 注释上面这句话的原因:不准确.说是基底变换没错,但是不合适.合适的理解方法是,同样是描述展开系数的变化,薛定谔绘景下是态矢量在变,海森堡绘景下是测量算符和对应的本征态在变.

本文中,角标 $H$ 代表海森堡绘景, 角标 $S$ 代表薛定谔绘景. 例如波函数分别记为 $\psi_H(\bvec r, t)$ 和 $\psi_S(\bvec r)$, 后者不是时间的函数, 它的定义是
\begin{equation}
\psi_S(\bvec r) = \psi_H(\bvec r, 0)
\end{equation}

% \addTODO{演化子是什么?链接} %Jier: 就是时间演化算符\upref{TOprt},该词条我已经更新了

% 使用演化子(propagator) $U(t)$, 波函数之间的关系为
% \begin{equation}
% \psi_H(\bvec r, t) = U(t) \psi_H(\bvec r, 0) = U(t) \psi_S(\bvec r)
% \end{equation}
% 薛定谔方程在海森堡绘景中可以记为
% \begin{equation}
% H(U\psi_H) = \I\hbar \pdv{t} (U\psi_H)
% \end{equation}
% 由于 $\psi_H$ 不含时, 两边抵消, 得演化子满足方程
% \begin{equation}
% H U(t) = \I\hbar \pdv{t} U(t)
% \end{equation}

% 定义海森堡绘景中的算符为
% \begin{equation}
% Q_H(t) = U\Her(t) Q_S(t) U(t)
% \end{equation}
% 对时间求导得
% \begin{equation}
% \dv{t}Q_H = \frac{\I}{\hbar} [H_H, Q_H(t)] + \qty(\pdv{Q_S}{t})_H
% \end{equation}
% 平均值公式仍然和薛定谔绘景相同
% \begin{equation}
% \mel{\psi_H}{Q_H}{\psi_H} = \mel{\psi_S}{Q_S}{\psi_S}
% \end{equation}
% 证明:
% \begin{equation}
% \mel{\psi_S}{Q_S}{\psi_S} = \mel{\psi_H}{U\Her(t)Q_SU(t)}{\psi_H} = \mel{\psi_H}{Q_H}{\psi_H}
% \end{equation}
% 证毕.

\subsection{薛定谔绘景}

薛定谔绘景,简而言之,是固定算符不变,研究态矢量的演化.这也是我们在\textbf{量子力学的基本原理(量子力学)}\upref{QMPrcp}中使用的描述.

在薛定谔绘景中,算符是恒定的,从而其对应的本征矢量也是恒定的.实际的量子态$\ket{s}$则随着时间$t$演化为$\mathcal{U}(t)\ket{s}$.这个过程可以理解为,映射$\mathcal{U}(t)$作用在矢量$\ket{s}$上,导致$\ket{s}$变化,于是其关于各可观测量的本征态的基底展开系数变化——这些系数的模方就是测量后得到对应本征态的概率(概率密度),因此我们观测到的概率就会变化.

薛定谔绘景下,初态为$\ket{s}$的量子态,其可观测量$X$的期望值随时间演化:
\begin{equation}
\langle X \rangle(t) = \bra{s}\mathcal{U}^*(t)X\mathcal{U}(t)\ket{s}
\end{equation}



\subsection{海森堡绘景}

海森堡绘景是另一种描述量子力学的框架,量子态本身不变,但可观测量的算符以及对应的本征态则随时间变化,由此造成量子态的基底展开系数变化.海森堡绘景下计算得到的可观测量的演化规律和薛定谔绘景相同.

注意到时间演化算符$\mathcal{U}(t)=\exp(-\I Ht)$是一个\textbf{幺正算符},即$\mathcal{U}(t)^\dagger=\mathcal{U}(t)^{-1}$.


















