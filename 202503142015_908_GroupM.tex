% 群(综述)
% license CCBYSA3
% type Wiki

本文根据 CC-BY-SA 协议转载翻译自维基百科\href{https://en.wikipedia.org/wiki/Group_(mathematics)}{相关文章}。

\begin{figure}[ht]
\centering
\includegraphics[width=6cm]{./figures/3d977276c842b7f1.png}
\caption{魔方的操作构成了魔方群。} \label{fig_GroupM_1}
\end{figure}
在数学中,群是一个具有二元运算的集合,并满足以下约束条件:该运算是结合的,它具有单位元,并且集合中的每个元素都有逆元。  

许多数学结构都是具有其他性质的群。例如,整数在加法运算下构成一个无限群,该群由一个称为\textbf{1}的单一元素生成(这些性质以独特的方式刻画了整数)。

群的概念被提出,以统一方式处理许多数学结构,例如数、几何形状和多项式的根。由于群的概念在数学内外的多个领域中无处不在,一些作者将其视为当代数学的核心组织原则之一。  

在几何学中,群自然地出现在对对称性和几何变换的研究中:一个物体的对称性构成一个群,称为该物体的\textbf{对称群},而某种特定类型的变换构成一个更一般的群。\textbf{李群}在几何中的对称群中出现,也出现在粒子物理学的\textbf{标准模型}中。\textbf{庞加莱群}是一个李群,包含狭义相对论中时空的对称性。\textbf{点群}则用于描述分子化学中的对称性。

群的概念起源于对多项式方程的研究,最早由埃瓦里斯特·伽罗瓦在 1830 年代提出,他使用 群(法语:groupe)这一术语来描述方程根的对称群,这一概念如今被称为\textbf{伽罗瓦群}。随着来自数论、几何等其他领域的贡献,群的概念得到了推广,并在1870 年左右被正式确立。\textbf{现代群论}是一个活跃的数学学科,它研究群本身的性质。为了探索群,数学家引入了各种概念,以便将群分解为更小、更易理解的部分,例如子群、商群和单群。除了研究群的抽象性质之外,群论学者还研究群的具体表现方式,包括表示论(即群的表示)和计算群论的方法。对于有限群,已经发展出一整套理论,并最终在2004年完成了有限单群的分类。自 20世纪80年代中期以来,\textbf{几何群论}这一分支迅速发展,它将\textbf{有限生成群}视为几何对象进行研究,成为群论中的一个活跃领域。
\subsection{定义与示例}  
\subsubsection{第一个例子:整数}
一个常见的群是整数集
\[
\mathbb{Z} = \{\ldots, -4, -3, -2, -1, 0, 1, 2, 3, 4, \ldots\}~
\]
配备\textbf{加法运算} \((+)\) 。对于任意两个整数 \(a\) 和 \(b\),它们的和 \(a + b\) 仍然是整数;这个封闭性表明加法是整数集 \(\mathbb{Z}\) 上的一个二元运算。  

整数加法的以下性质构成了群的基本公理,并在下面的定义中得到了推广:  
\begin{itemize}
\item 结合律(Associativity)对于所有整数 \(a, b, c\),有:\((a + b) + c = a + (b + c)\)这意味着,无论是先将 \(a\) 与 \(b\) 相加再加上 \(c\),还是先将 \(b\) 与 \(c\) 相加再加上 \(a\),最终的结果相同。  
\item 单位元(Identity Element)对于任意整数 \(a\),有:\(0 + a = a \quad \text{且} \quad a + 0 = a\) \textbf{0}被称为\textbf{加法的单位元},因为它与任意整数相加都不改变该整数的值。  
\item 逆元(Inverse Element) 对于任意整数 \(a\),存在一个整数 \(b\),使得: \(a + b = 0 \quad \text{且} \quad b + a = 0\)这个整数 \(b\) 称为 \(a\) 的\textbf{加法逆元},通常记作 \(-a\)。  
\end{itemize}
整数集 \(\mathbb{Z}\) 连同加法运算构成了一个数学结构,该结构属于一类具有相似性质的更广泛的代数对象。为了更系统地理解这类结构,下面给出正式的定义。
\subsubsection{定义} 
一个群是一个\textbf{非空集合 }\( G \),配备一个\textbf{二元运算}(在此记作“\( \cdot \)”),该运算将 \( G \) 中的任意两个元素 \( a \) 和 \( b \) 组合,得到仍属于 \( G \) 的元素 \( a \cdot b \)。此外,该运算必须满足以下三个被称为\textbf{群公理}的条件:[5][6][7][a]  

结合律(Associativity) 

对于所有 \( a, b, c \in G \),有:\((a \cdot b) \cdot c = a \cdot (b \cdot c)\)这意味着运算的计算顺序不会影响最终结果。  

单位元(Identity Element) 

存在一个元素 \( e \in G \),使得对于任意 \( a \in G \),有:\(e \cdot a = a \quad \text{且} \quad a \cdot e = a\)该元素 \( e \) 是唯一的(见下文),称为单位元(或中性元)。  

逆元(Inverse Element)

对于 \( G \) 中的每个元素 \( a \),存在一个元素 \( b \in G \),使得:\(a \cdot b = e \quad \text{且} \quad b \cdot a = e\)其中 \( e \) 是单位元。对于每个 \( a \),这个元素 \( b \) 是唯一的(见下文),称为 \( a \) 的逆元,通常记作 \( a^{-1} \)。
\subsubsection{符号与术语} 
从形式上看,一个群是一个\textbf{有序对},由一个集合和该集合上的二元运算组成,并满足群公理。这个集合称为\textbf{群的底层集合},而该运算称为群运算或群律。  

因此,群和它的底层集合是两个不同的数学对象。为了避免繁琐的符号表示,通常会滥用符号,用同一个符号表示二者。这种做法也反映了一种非正式的思维方式:即认为群只是该集合的一个“扩展版本”,其附加的结构由群运算提供。  

例如,考虑\textbf{实数集} \( \mathbb{R} \),它配备了加法运算 \( a + b \) 和乘法运算 \( ab \):形式上,\( \mathbb{R} \)只是一个集合,\( (\mathbb{R}, +) \)是一个加法群,\( (\mathbb{R}, +, \cdot) \)是一个域(field)。然而,在实际使用中,通常直接用 \( \mathbb{R} \) 来表示这三种对象之一,而不作区分。  

在域 \( \mathbb{R} \) 中,加法群(additive group)是以 \( \mathbb{R} \) 为底层集合,并以加法 \( + \) 作为运算的群,即 \( (\mathbb{R}, +) \)。乘法群记作 \( \mathbb{R}^{\times} \),其底层集合是去掉零的实数集\( \mathbb{R}\setminus \{0\} \),其运算是乘法\( \cdot \),即 \( (\mathbb{R}^\times, \cdot) \)。

当群的运算使用加法表示时,通常称之为加法群。在这种情况下,单位元通常记作\textbf{0},元素 \( x \) 的逆元记作\(-x\)。同样,当群的运算使用乘法表示时,通常称之为乘法群。在这种情况下,单位元通常记作 \textbf{1},元素 \( x \) 的逆元记作 \( x^{-1} \)。在乘法群中,运算符号通常被省略,即直接用并置表示运算,例如 \( ab \) 代替 \( a \cdot b \)。  

群的定义并不要求对所有元素 \( a, b \in G \) 都满足:\(a \cdot b = b \cdot a\) 如果满足这个额外条件,则称该运算是交换的,并称该群为\textbf{阿贝尔群}。通常的惯例是:阿贝尔群可以使用加法记号(\( + \))或乘法记号(\( \cdot \))。非阿贝尔群仅使用乘法记号(\( \cdot \))。

对于元素不是数的群,常见的还有其他记号。例如:当群的元素是函数时,运算通常是函数复合,记作:\(f \circ g\)在这种情况下,单位元通常记作\textbf{id}(即恒等函数)。在更具体的情况中,例如几何变换群、对称群、置换群和自同构群,运算符号 \( \circ \)通常被省略,类似于乘法群的记法。此外,群的记号可能会有许多其他变体,具体取决于应用领域和数学上下文。
\subsubsection{第二个例子:对称群 }
在平面上,如果一个图形可以通过旋转、反射和平移的组合变换成另一个图形,则它们是全等的。任何图形都与自身全等。然而,一些图形不仅与自身全等,而且有多种不同的方式与自身全等,这些额外的全等变换称为对称。  

例如,一个正方形有八种对称变换,它们包括:
\begin{figure}[ht]
\centering
\includegraphics[width=14.25cm]{./figures/f32a15365d3f3335.png}
\caption{} \label{fig_GroupM_2}
\end{figure}
\begin{itemize}
\item 恒等变换:保持所有内容不变,记作\textbf{id}。  
\item 旋转:以正方形的中心为旋转点,顺时针旋转90°,记作\( r_1 \) 180°,记作 \( r_2 \)270°,记作 \( r_3 \) 反射:  
\item 关于水平轴和垂直轴的反射,分别记作 \( f_h \) 和 \( f_v \);关于两条对角线的反射,分别记作 \( f_d \) 和 \( f_c \)。
\end{itemize}
这些对称变换本质上是函数,它们将正方形上的一个点映射到对称变换后的对应点。例如:\( r_1 \):将一个点顺时针旋转90°,围绕正方形的中心。\( f_h \):将一个点关于水平中轴线反射。组合两个这样的对称变换,仍然得到另一个对称变换。因此,这些对称变换构成了一个群,称为四阶二面体群,记作\( D_4 \)。该群的\textbf{底层集合}就是上述八个对称变换,而群运算是函数复合。两个对称变换的组合是按照函数复合的方式进行的,即:先应用第一个变换 \( a \) ,再应用第二个变换 \( b \)。记作:\(b \circ a\)这表示先执行对称变换 \( a \),然后对结果再应用对称变换 \( b \)。这里采用的记号是从右到左的顺序,这是函数复合的标准记法。

凯莱表列出了所有可能的对称变换组合的结果。例如:先顺时针旋转 270°(\( r_3 \)),再进行水平反射(\( f_h \)),其结果与沿对角线反射(\( f_d \))的效果相同。用上述符号表示,在凯莱表中(通常以蓝色高亮):  
\[
f_h \circ r_3 = f_d~
\]
给定这组对称变换及其运算方式,可以如下理解群公理:  

二元运算(Binary Operation):函数复合是一种二元运算,即对于任意两个对称变换 \( a \) 和 \( b \),它们的复合运算 \( a \circ b \) 仍然是一个对称变换。例如:  
 \[
 r_3 \circ f_h = f_c~
\]  
这表示先进行水平反射 \( f_h \),然后顺时针旋转 270°(\( r_3 \)),其结果等同于沿副对角线反射 \( f_c \)。实际上,任意两个对称变换的组合仍然是某种对称变换,这可以通过凯莱表(Cayley table)进行验证。

结合律(Associativity):结合律公理处理的是多个对称变换的组合。从 \( D_4 \) 群中的三个元素 \( a \)、\( b \) 和 \( c \) 开始,存在两种可能的方式按顺序使用这三个对称变换来确定正方形的对称性:先将 \( a \) 和 \( b \) 组合成一个对称变换,然后再将该对称变换与 \( c \) 组合。先将 \( b \) 和 \( c \) 组合成一个对称变换,然后再将该对称变换与 \( a \) 组合。这两种方法必须始终给出相同的结果,即:  
\[
(a \circ b) \circ c = a \circ (b \circ c)~
\]

例如,\((f_d \circ f_v) \circ r_2 = f_d \circ (f_v \circ r_2)\)可以通过凯莱表进行验证:  
\[
(f_d \circ f_v) \circ r_2 = r_3 \circ r_2 = r_1~
\]
\[
f_d \circ (f_v \circ r_2) = f_d \circ f_h = r_1~
\]  
单位元(Identity Element):单位元是\( \text{id} \),因为它与任何对称变换 \( a \) 组合(无论在左侧还是右侧)都不改变该变换的结果。

逆元(Inverse Element): 每个对称变换都有一个逆元:\( \text{id} \)、反射变换\( f_h \)、\( f_v \)、\( f_d \)、\( f_c \)和180° 旋转 \( r_2 \)都是它们自己的逆元,因为执行两次这些变换将正方形恢复到其原始的朝向。旋转变换\( r_3 \)和\( r_1 \)是彼此的逆元,因为先旋转 90° 再旋转 270°(或反之)得到的结果是旋转 360°,这会让正方形保持不变。这可以在凯莱表中轻松验证。

与上面提到的整数群不同,在\( D_4 \)群中,运算的顺序是重要的。例如:\(
f_h \circ r_1 = f_c \quad \text{但} \quad r_1 \circ f_h = f_d\)换句话说,\( D_4 \)不是阿贝尔群(abelian group)。
\subsection{历史}  
现代抽象群的概念源自多个数学领域的发展。[9][10][11] 群论的最初动机是寻求高于四次的多项式方程的解法。19世纪的法国数学家埃瓦里斯特·伽罗瓦在保罗·鲁菲尼和约瑟夫-路易·拉格朗日之前工作的基础上,提出了通过多项式方程根的对称群来判定方程是否可解的标准。这样的\textbf{伽罗瓦群}的元素对应于根的某些置换。最初,伽罗瓦的思想遭到当时学者的拒绝,并且仅在他去世后才得以发表。[12][13] 更一般的置换群被奥古斯丁·路易·柯西等人进一步研究。阿瑟·凯利在其著作《群论的理论》(On the Theory of Groups, as Depending on the Symbolic Equation \( \theta^n = 1 \),1854年)中,首次给出了有限群的抽象定义。[14]

几何学是群论被系统应用的第二个领域,特别是对称群,作为费利克斯·克莱因1872年\textbf{埃尔朗根计划}的一部分。[15] 在双曲几何和射影几何等新几何学出现后,克莱因利用群论将这些几何学以更一致的方式组织起来。进一步发展这些思想,索福斯·李于1884年创立了李群的研究。[16]

群论的第三个贡献领域是数论。在卡尔·弗里德里希·高斯的数论著作《算术研究》(Disquisitiones Arithmeticae,1798年)中,某些阿贝尔群结构已经被隐式使用,而莱奥波德·克罗内克则更加明确地应用了这些结构。[17] 在1847年,恩斯特·库默通过发展描述质数分解的群体,开始尝试证明\textbf{费尔马最后定理}。[18]

这些不同来源的汇聚形成了群论的统一理论,始于卡米尔·乔丹的《置换与代数方程论》(1870年)。[19]沃尔特·冯·迪克(1882年)提出了通过生成元和关系来指定群的概念,并且是第一个给出“抽象群”公理化定义的人(当时的术语)。进入20世纪后,群论通过费迪南德·乔治·弗罗贝纽斯和威廉·伯恩赛德的开创性工作(他们研究了有限群的表示论)、理查德·布劳尔的模表示理论和伊萨伊·舒尔的论文获得了广泛的认可。[21]李群理论,更广泛地说,局部紧群的研究由赫尔曼·维尔、埃利·卡尔坦等人进行。[22]其代数对应物——代数群理论,最初由克劳德·谢瓦雷(从1930年代末期开始)奠定,后来由阿尔芒·博雷尔和雅克·蒂茨的工作进一步发展。[23]

芝加哥大学的1960–61群论年汇聚了群论学者,如丹尼尔·戈伦斯坦、约翰·G·汤普森和沃尔特·费特,为一个合作奠定了基础,这个合作在许多其他数学家的贡献下,最终导致了有限单群的分类,最后一步由阿什巴赫和史密斯在2004年完成。这个项目在数学历史上超越了以往的工作,其规模无论是在证明的长度还是研究人员的数量上都是前所未有的。关于这个分类证明的研究仍在进行中。[24] 群论仍然是一个高度活跃的数学分支,[b] 对许多其他领域产生了影响,以下例子便展示了这一点。
\subsection{群公理的基本推论}  
从群公理直接得到的所有群的基本事实通常归纳于初等群论中。例如,结合性公理的反复应用表明,式子 \( a \cdot b \cdot c = (a \cdot b) \cdot c = a \cdot (b \cdot c) \) 的明确性可以推广到三个以上的因子。因为这意味着括号可以在这样的项序列中随意插入,所以通常省略括号。
\subsubsection{单位元的唯一性 } 
群公理暗示单位元是唯一的;也就是说,只有一个单位元:群中的任何两个单位元 \( e \) 和 \( f \) 是相等的,因为群公理暗示 \( e = e \cdot f = f \)。因此,通常会称之为群的单位元。
\subsubsection{逆元的唯一性}  
群公理还暗示每个元素的逆元是唯一的。设群中元素 \( a \) 具有两个逆元 \( b \) 和 \( c \),则:
\[
\begin{aligned}
b &= b \cdot e && (\text{\( e \) 是单位元}) \\
&= b \cdot (a \cdot c) && (\text{\( c \) 与 \( a \) 互为逆元}) \\
&= (b \cdot a) \cdot c && (\text{结合律}) \\
&= e \cdot c && (\text{\( b \) 是 \( a \) 的逆元}) \\
&= c && (\text{\( e \) 是单位元,且 \( b = c \)}) 
\end{aligned}~
\]
因此,通常称之为元素的逆元。
\subsubsection{除法} 
对于群 \( G \) 中的任意元素 \( a \) 和 \( b \),方程\(a \cdot x = b\)在 \( G \) 中有唯一解 \( x \),即\(x = a^{-1} \cdot b\)因此,对于每个 \( a \in G \),映射\(G \to G, \quad x \mapsto a \cdot x\)是一个双射,这个映射被称为左乘法或左平移由 \( a \) 作用。

类似地,对于任意 \( a, b \in G \),方程 \(x \cdot a = b\)的唯一解是\(x = b \cdot a^{-1}\)因此,对于每个\( a \in G \),映射\(G \to G, \quad x \mapsto x \cdot a\)也是一个双射,被称为右乘法或右平移由 \( a \) 作用。
\subsubsection{等价定义与放宽公理} 
群的单位元与逆元公理可以“弱化”,仅要求左单位元和左逆元的存在。从这些\textbf{单侧公理}出发,可以证明左单位元也是右单位元,而左逆元也是相应元素的右逆元。因此,这些公理实际上定义了与群完全相同的结构,因此整体而言,这些公理并不更弱。  

具体而言,假设群运算满足结合律,并且存在左单位元 \( e \)(即 \( e \cdot f = f \))以及每个元素 \( f \) 具有左逆元 \( f^{-1} \)(即 \( f^{-1} \cdot f = e \)),则可以证明每个左逆元也是相应元素的右逆元。  

证明如下:  
\[
\begin{aligned}
f \cdot f^{-1} &= e \cdot (f \cdot f^{-1}) && (\text{左单位元}) \\
&= ((f^{-1})^{-1} \cdot f^{-1}) \cdot (f \cdot f^{-1}) && (\text{左逆元}) \\
&= (f^{-1})^{-1} \cdot ((f^{-1} \cdot f) \cdot f^{-1}) && (\text{结合律}) \\
&= (f^{-1})^{-1} \cdot (e \cdot f^{-1}) && (\text{左逆元}) \\
&= (f^{-1})^{-1} \cdot f^{-1} && (\text{左单位元}) \\
&= e && (\text{左逆元})
\end{aligned}~
\]
同样,左单位元也是右单位元: 
\[
\begin{aligned}
f \cdot e &= f \cdot (f^{-1} \cdot f) && (\text{左逆元}) \\
&= (f \cdot f^{-1}) \cdot f && (\text{结合律}) \\
&= e \cdot f && (\text{右逆元}) \\
&= f && (\text{左单位元})
\end{aligned}~
\]
这些证明需要满足所有三个公理(结合律、左单位元的存在和左逆元的存在)。对于定义更宽松的结构(如半群),可能会出现左单位元不一定是右单位元的情况。  

同样的结论也可以仅假设\textbf{右单位元的存在和右逆元的存在}来获得。

然而,仅假设左单位元的存在和右逆元的存在(或反之)并不足以定义一个群。例如,考虑集合\(G = \{ e, f \}\)及其运算 \( \cdot \) 满足以下规则:\(e \cdot e = f \cdot e = e\)和\(e \cdot f = f \cdot f = f\)该结构确实具有左单位元(即 \( e \)),并且每个元素都有右逆元(对于两个元素来说都是 \( e \))。此外,该运算是结合的,因为任意多个元素的乘积始终等于最右侧的元素,无论计算顺序如何。然而,该结构 \((G, \cdot)\) 不是一个群,因为它缺少右单位元。
\subsection{基本概念}  
以下章节使用数学符号,例如\(X = \{x, y, z\}\)表示一个包含元素 \( x \)、\( y \) 和 \( z \) 的集合 \( X \),或者\(x \in X\)表示 \( x \) 是 \( X \) 的一个元素。符号\(f: X \to Y\)表示 \( f \) 是一个函数,它将 \( X \) 中的每个元素映射到 \( Y \) 中的某个元素。  

在研究集合时,会使用诸如子集、函数和按等价关系取商等概念。而在研究群时,则使用对应的子群、同态映射和商群等概念。这些概念是对集合论概念的推广,以考虑群的结构。
\subsubsection{群同态}  
群同态是保持群结构的函数,它们可用于建立两个群之间的联系。一个从群 \((G, \cdot)\) 到群 \((H, *)\) 的同态是一个函数\(\varphi: G \to H\)
满足  
\[
\varphi(a \cdot b) = \varphi(a) * \varphi(b), \quad \forall a, b \in G.~
\]
自然地,我们可能希望同态函数也保持单位元和逆元,即\(\varphi(1_G) = 1_H\)\(\varphi(a^{-1}) = \varphi(a)^{-1}, \quad \forall a \in G.\)然而,这些额外的条件不必包含在同态的定义中,因为它们已经被保持群运算的要求所蕴含。

群\(G\)的恒等同态是映射\(\iota_G: G \to G\)它将\( G\)中的每个元素映射到身。一个同态\(\varphi:G\to H\)的逆同态是一个映射\(\psi:H\to G\)满足\(\psi \circ\varphi = \iota_G,\quad \varphi \circ \psi = \iota_H,\)即对于所有\( g \in G \) 和 \( h \in H \)都有:\(\psi(\varphi(g)) = g, \quad \varphi(\psi(h)) = h\).如果一个同态存在逆同态,则称其为同构;等价地,一个同构就是一个双射同态。如果存在一个同构\( \varphi: G \to H \),则称群\(G\)和\( H \)同构。在这种情况下,可以通过函数 \( \varphi \) 将 \( G \) 的元素重命名得到\( H \),并且对\( G \)成立的所有命题,在适当重命名元素后,对于\(H\)也成立。

所有群及其之间的同态的集合构成了一个范畴,即群的范畴。  

一个单射同态\(\phi:G'\to G\)可以规范地分解为一个同构后跟一个包含映射:\(G' \; {\stackrel{\sim}{\to}} \; H \hookrightarrow G\)其中 \( H \) 是 \( G \) 的一个子群。单射同态是群的范畴中的单态射。
\subsubsection{子群}  
非正式地说,子群是一个包含在更大群\( G \)中的群\( H \):它是\( G \)元素的一个子集,并且使用相同的群运算。[32] 具体来说,这意味着 \( G \) 的单位元必须包含在 \( H \)中,并且每当 \( h_1 \) 和 \( h_2 \)都属于\( H \) 时,\(h_1 \cdot h_2 \)和\( h_1^{-1} \) 也必须在 \( H \) 中,因此,\( H \) 中的元素,配合在 \( G \)上限制到\( H \)的群运算,确实构成一个群。在这种情况下,包含映射\( H \to G \)是一个同态。

在正方形的对称性例子中,单位元和旋转构成了一个子群\(R = \{\text{id}, r_1, r_2, r_3\}\)在示例的凯利表格中以红色标出:任意两个旋转组合仍然是旋转,并且旋转可以通过互补的旋转来取消(即,互为逆元素):90°的旋转通过270°的旋转取消,180°的旋转通过另一个180°的旋转取消,270°的旋转通过90°的旋转取消。子群检验提供了一个充分且必要的条件,来判定群 \( G \) 的一个非空子集 \( H \) 是否为子群:只需要检查对于所有 \( g \) 和 \( h \) 属于 \( H \),有 \( g^{-1} \cdot h \in H \)。了解一个群的子群对于理解整个群非常重要。

给定群 \( G \) 的任何子集 \( S \),由 \( S \) 生成的子群由 \( S \) 中元素及其逆元的所有积构成。它是包含 \( S \) 的最小子群。[33] 在正方形对称性例子中,由 \( r_2 \) 和 \( f_v \) 生成的子群包含这两个元素、单位元 \( \text{id} \) 和元素 \( f_h = f_v \cdot r_2 \)。同样,这也是一个子群,因为将这四个元素或它们的逆元(在这个特殊情况下,它们就是这些相同的元素)中的任意两个组合,都会得到该子群中的一个元素。
\subsubsection{陪集}
在许多情况下,如果两个群元素仅由给定子群的一个元素不同,我们希望将它们视为相同。例如,在正方形的对称群中,一旦执行了某个反射操作,旋转单独无法将正方形恢复到其原始位置,因此可以认为正方形的反射位置彼此等价,而与未反射的位置不等价;旋转操作与是否已执行反射操作无关。陪集用于正式化这一见解:一个子群 \( H \) 决定了左右陪集,可以将其视为将 \( H \) 通过任意群元素 \( g \) 进行平移。  

用符号表示,包含元素 \( g \) 的 \( H \) 的左陪集和右陪集分别为  
\[
gH = \{ g \cdot h \mid h \in H \}~
\]
和  
\[
Hg = \{ h \cdot g \mid h \in H \}~
\][34]

任何子群 \( H \) 的左陪集形成群 \( G \) 的一个划分;也就是说,所有左陪集的并集等于 \( G \),并且两个左陪集要么相等,要么交集为空。[35] 第一种情况 \( g_1 H = g_2 H \) 恰好发生在\(g_1^{-1} \cdot g_2 \in H\)即当这两个元素仅相差一个属于 \( H \) 的元素时。类似的考虑也适用于 \( H \) 的右陪集。\( H \) 的左陪集可能与右陪集相同,也可能不同。如果它们相同(即如果群 \( G \) 中的所有元素 \( g \) 满足 \( gH = Hg \)),则称 \( H \) 为\textbf{正规子群}。

在\( D_4 \)中,正方形的对称群,以及其旋转子群 \( R \),左陪集\( gR \)要么等于 \( R \),如果 \( g \) 本身是 \( R \) 的元素,要么等于\(U = f_{\mathrm{c}} R = \{ f_{\mathrm{c}}, f_{\mathrm{d}},f_{\mathrm{v}}, f_{\mathrm{h}}\}\)(在 \( D_4 \) 的凯利表格中以绿色高亮显示)。子群\( R \)是正规子群,因为\(f_{\mathrm{c}} R = U = R f_{\mathrm{c}}\)对群的其他元素也类似成立。(事实上,在 \( D_4 \)的例子中,所有由反射生成的陪集都是相等的:\(f_{\mathrm{h}}R = f_{\mathrm{v}} R = f_{\mathrm{d}}R = f_{\mathrm{c}} R\))。
\subsubsection{商群} 
假设\( N \)是群\( G \)的一个正规子群,且
\[G/N = \{gN \mid g \in G \}~\]
表示其陪集的集合。那么,在\( G/N \)上存在唯一的群运算,使得映射\(G \to G/N\)将每个元素\( g \)映射到\( gN \)是一个同态。具体来说,两个陪集\( gN \)和\( hN \)的乘积是\((gh)N\)单位元\( eN = N \)是\( G/N \)的单位元,而商群中\(gN\)的逆元是\((gN)^{-1} = (g^{-1})N\)群\( G/N \),读作“\( G \) 模 \( N \)”,[36]称为商群或因子群。商群也可以通过一个普适性质来表征。
\begin{table}[ht]
\centering
\caption{商群 \( D_4 / R \) 的凯利表格如下所示:}\label{GroupM}
\begin{tabular}{|c|c|c}
\hline
\(\cdot\) & R & U \\
\hline
R & R & U \\
\hline
U & U & R \\
\hline 
\end{tabular}
\end{table}
商群\( D_4 / R \)的元素是\( R \)和\(U = f_{\mathrm{v}} R\)商群上的群运算如表所示。例如,\(U \cdot U = f_{\mathrm{v}} R \cdot f_{\mathrm{v}} R =(f_{\mathrm{v}}\cdot f_{\mathrm{v}}) R = R\)子群\(R =\{\text{id}, r_1, r_2, r_3\}\)和商群\( D_4 / R \)都是阿贝尔群,但\(D_4\)不是。有时,通过半直积构造,一个群可以由一个子群和商群(加上一些附加数据)重构;\(D_4\)就是一个例子。

第一同构定理意味着任何满射同态\(\phi: G \to H\)都可以规范地分解为一个商同态后跟一个同构:\(G \to G / \ker \phi \; {\stackrel {\sim}{\to}} \; H\)满射同态是群的范畴中的满态射。
\subsubsection{群的表示}  
每个群都与自由群的商群同构,有许多不同的方式。

例如,二面体群\(D_4\)是由右旋转\(r_1\)和垂直线上的反射\(f_{\mathrm{v}}\)生成的(\(D_4\)的每个元素都是这些元素及其逆元的有限积)。因此,从自由群\( \langle r, f \rangle \)到\(D_4\)存在一个满射同态\(\phi\),它将\(r\)映射到\(r_1\),将\(f\)映射到\(f_1\)。\(\ker \phi\)中的元素被称为关系;例如包括\( r^4, f^2, (r\cdot f)^2\)。实际上,结果表明\(\ker \phi\)是包含这三元素的自由群\(\langle r, f \rangle\)中最小的正规子群;换句话说,所有的关系都是这三者的推论。通过这个正规子群的商群表示为\(\langle r, f \mid r^4 = f^2 = (r \cdot f)^2 = 1 \rangle\)这被称为\(D_4\)的生成元与关系的表示,因为对同态\(\varphi\)使用第一同构定理可以得到同构\(\langle r, f \mid r^4 = f^2 = (r \cdot f)^2 = 1 \rangle \to D_4\)。[37]

一个群的表示可以用来构造凯利图,这是离散群的图形表示。[38]
\subsection{例子与应用}
\begin{figure}[ht]
\centering
\includegraphics[width=6cm]{./figures/2e47c5a8af4f98e2.png}
\caption{一个周期性的墙纸图案会产生一个墙纸群。} \label{fig_GroupM_3}
\end{figure}
群的例子和应用比比皆是。一个起点是整数集\( \mathbb{Z} \),其群运算是加法,如上所述。如果将加法替换为乘法,则得到乘法群。这些群是抽象代数中重要构造的前身。

群也广泛应用于许多其他数学领域。数学对象通常通过将群与之关联来进行研究,进而研究相应群的性质。例如,亨利·庞加莱通过引入基本群创立了今天所称的代数拓扑学。[39] 通过这种联系,拓扑性质如邻近性和连续性可以转化为群的性质。[g]
\begin{figure}[ht]
\centering
\includegraphics[width=6cm]{./figures/c72b41c700dd32b0.png}
\caption{去掉一个点的平面(加粗)上的基本群由绕缺失点的闭环组成。这个群同构于整数加法群。} \label{fig_GroupM_4}
\end{figure}
拓扑空间的基本群的元素是闭环的等价类,其中闭环被认为是等价的,如果一个闭环可以平滑地变形为另一个闭环,群运算是“连接”(先描绘一个闭环,再描绘另一个)。例如,如图所示,如果拓扑空间是去掉一个点的平面,那么不绕过缺失点的闭环(蓝色)可以平滑地收缩为一个单一的点,并且是基本群的单位元。一个绕缺失点转圈 \( k \) 次的闭环不能变形为绕转 \( m \) 次的闭环(其中 \( m \neq k \)),因为该闭环无法平滑地穿过空洞,因此每个闭环类由其绕缺失点的旋转次数来表征。由此得到的群同构于整数加法群。

在最近的应用中,影响也发生了反向转变,通过群论背景来激发几何构造的动机。[h] 类似地,几何群论运用几何概念,例如在研究双曲群时。[40] 进一步的分支领域,包括代数几何和数论,极大地依赖于群的应用。[41]

除了上述理论应用外,群的许多实际应用也存在。密码学依赖于抽象群论方法与计算群论中获得的算法知识的结合,特别是在有限群的实现中。[42] 群论的应用不仅限于数学;物理学、化学和计算机科学等科学领域也受益于这一概念。
\subsubsection{数字}  
许多数系,如整数和有理数,天然地具有群结构。在某些情况下,例如有理数,既有加法运算也有乘法运算都能形成群结构。这些数系是更一般的代数结构——环和域的前身。进一步的抽象代数概念,如模、向量空间和代数,也可以形成群。

\textbf{整数}  

整数群 \( \mathbb{Z} \) 在加法下的群运算 \( (\mathbb{Z}, +) \) 如上所述。整数在乘法运算下,\( (\mathbb{Z}, \cdot) \) 并不构成群。虽然满足结合性和单位元公理,但没有逆元:例如,\( a = 2 \) 是一个整数,但方程 \( a \cdot b = 1 \) 的唯一解是 \( b = \frac{1}{2} \),这是一个有理数,而不是整数。因此,\( \mathbb{Z} \) 中并非每个元素都有(乘法)逆元。[i]

\textbf{有理数}  

对乘法逆元存在的需求促使我们考虑分数
\[\frac{a}{b}~\]。  
整数的分数(其中 \( b \) 非零)被称为有理数。[j] 所有不可约分数的集合通常表示为\( \mathbb{Q}\)。对于有理数在乘法下的群运算\( (\mathbb{Q}, \cdot)\),仍然存在一个小障碍:因为零没有乘法逆元(即没有\( x \)满足\( x \cdot 0 = 1 \)),所以\( (\mathbb{Q}, \cdot)\)仍然不是一个群。

然而,所有非零有理数的集合\(\mathbb{Q} \setminus \{0\} = \{q \in \mathbb{Q} \mid q \neq 0\}\)

在乘法下确实构成一个阿贝尔群,通常表示为\( \mathbb{Q}^\times \)。[k] 结合性和单位元公理来源于整数的性质。去掉零后,封闭性要求仍然成立,因为两个非零有理数的乘积永远不会是零。最后,\( a/b \)的逆元是\(b/a\),因此逆元公理得以满足。

有理数(包括零)在加法下也构成一个群。加法和乘法运算的交织产生了更复杂的结构,称为环——如果除零以外的除法是可能的,如在\(\mathbb{Q}\)中——则称为域,它们在抽象代数中占据着核心位置。因此,群论的推理在这些实体的理论中发挥了作用。[l]
\subsubsection{模运算}
\begin{figure}[ht]
\centering
\includegraphics[width=6cm]{./figures/022b369946dfa3e8.png}
\caption{时钟上的小时形成一个以模 12 加法为运算的群。在这里,\( 9 + 4 \equiv 1 \)。} \label{fig_GroupM_5}
\end{figure}
模运算对于模数 \( n \) 定义了任何两个元素 \( a \) 和 \( b \),如果它们相差 \( n \) 的倍数,则认为它们是等价的,记作\(a \equiv b \pmod{n}\).每个整数都与从 0 到 \( n - 1 \) 之间的某个整数等价,模运算的运算方式通过将任何运算的结果替换为其等价代表来修改常规算术运算。对于从 0 到 \( n - 1 \) 的整数,定义的模加法形成一个群,记作 \( Z_n \) 或 \( (\mathbb{Z}/n\mathbb{Z}, +) \),其中 0 是单位元,\( n - a \) 是元素 \( a \) 的逆元。

一个熟悉的例子是钟面上的小时加法,其中选择 12 而不是 0 作为单位元的代表。如果时针指向 9 并且前进 4 小时,它最终会指向 1,如图所示。这可以用“9 + 4 在模 12 意义下同余于 1”来表示,或者用符号表示为  
\[
9 + 4 \equiv 1 \pmod{12}.~
\]
对于任何质数 \( p \),也有整数模 \( p \) 的乘法群。[43] 其元素可以表示为从 1 到 \( p - 1 \) 的整数。群运算是模 \( p \) 的乘法,将通常的乘积替换为其代表,即除以 \( p \) 的余数。例如,对于 \( p = 5 \),四个群元素可以表示为 1, 2, 3, 4。在这个群中,\( 4 \cdot 4 \equiv 1 \pmod{5} \),因为通常的乘积 16 与 1 等价:当除以 5 时,余数为 1。 \( p \) 的质性保证了两个代表的通常乘积不被 \( p \) 整除,因此模乘积非零。[m] 单位元由 1 表示,结合性来自于整数的相应性质。最后,逆元公理要求,对于一个不被 \( p \) 整除的整数 \( a \),存在一个整数 \( b \),使得  
\[
a \cdot b \equiv 1 \pmod{p}~
\]
即 \( p \) 整除 \( a \cdot b - 1 \)。逆元 \( b \) 可以通过使用贝祖恒等式和最大公约数 \( \gcd(a, p) = 1 \) 来求得。[44] 在上面的例子中,当 \( p = 5 \) 时,表示 4 的元素的逆元是表示 4 的元素,表示 3 的元素的逆元是表示 2 的元素,因为 \( 3 \cdot 2 = 6 \equiv 1 \pmod{5} \)。因此,所有群公理都得到了满足。这个例子类似于上面提到的 \( (\mathbb{Q} \setminus \{0\}, \cdot) \):它由环 \( \mathbb{Z}/p\mathbb{Z} \) 中恰好具有乘法逆元的元素构成。[45] 这些群,记作 \( \mathbb{F}_p^\times \),在公钥密码学中至关重要。[n]
\subsubsection{循环群}
\begin{figure}[ht]
\centering
\includegraphics[width=6cm]{./figures/ee1892681a55e745.png}
\caption{6阶复数单位根构成一个循环群。z 是一个原始元素,但 z² 不是,因为 z 的奇数次幂不是 z² 的幂。} \label{fig_GroupM_6}
\end{figure}
一个循环群是一个所有元素都是某个特定元素 \(a\) 的幂的群。[46] 在乘法表示法中,群的元素是  
\[\dots, a^{-3}, a^{-2}, a^{-1}, a^{0}, a, a^2, a^3, \dots,~\]
其中 \(a^2\) 表示 \(a \cdot a\),\(a^{-3}\) 表示 \(a^{-1} \cdot a^{-1} \cdot a^{-1} = (a \cdot a \cdot a)^{-1}\),等等。[o] 这样的元素 \(a\) 被称为群的生成元或原始元素。在加法表示法中,元素要成为原始元素的要求是群的每个元素都可以写成  
\[\dots, (-a) + (-a), -a, 0, a, a + a, \dots,~\]
在上面介绍的群 \((\mathbb{Z}/n\mathbb{Z}, +)\) 中,元素 1 是原始元素,因此这些群是循环群。实际上,每个元素都可以表示为所有项都是 1 的和。任何具有 \(n\) 个元素的循环群都与这个群同构。循环群的第二个例子是 \(n\) 阶复数单位根群,给定满足 \(z^n = 1\) 的复数 \(z\)。这些数字可以被视为正 \(n\) 边形的顶点,如图中蓝色部分所示(对于 \(n = 6\))。群运算是复数的乘法。在图中,与 \(z\) 相乘对应的是逆时针旋转 60°。[47] 从域论的角度看,群 \(\mathbb{F}_p^\times\) 是循环群,其中 \(p\) 为素数;例如,当 \(p = 5\) 时,3 是生成元,因为\(3^1 = 3\),\(3^2 = 9 \equiv 4\),\(3^3 \equiv 2\),且 \(3^4 \equiv 1\)。

一些循环群具有无限个元素。在这些群中,对于每个非零元素 \(a\),所有的 \(a\) 的幂都是不同的;尽管叫做“循环群”,但元素的幂并不会循环。一个无限循环群同构于 \((\mathbb{Z}, +)\),即上面介绍的整数加法群。[48] 由于这两个原型都是阿贝尔群,因此所有循环群也是阿贝尔群。

有限生成阿贝尔群的研究已经相当成熟,包括有限生成阿贝尔群的基本定理;反映这一现状的有许多与群相关的概念,如中心和交换子,它们描述了一个群在多大程度上不是阿贝尔群。[49]
\subsubsection{对称群}
\begin{figure}[ht]
\centering
\includegraphics[width=6cm]{./figures/bd007cc19e807da9.png}
\caption{(2,3,7)三角群,一个双曲反射群,作用于这个双曲平面的镶嵌[50]。} \label{fig_GroupM_7}
\end{figure}
对称群是由给定数学对象的对称性构成的群,主要是几何实体,例如上面作为介绍示例给出的正方形的对称群,尽管它们也出现在代数学中,比如在伽罗瓦理论中处理的多项式方程根之间的对称性(见下文)。[51] 从概念上讲,群论可以看作是对称性的研究。[p] 数学中的对称性极大地简化了几何或分析对象的研究。如果每个群元素可以与对 \(X\) 的某个操作相关联,并且这些操作的组合遵循群法则,则称群作用于另一个数学对象 \(X\)。例如,(2,3,7)三角群的一个元素通过排列三角形作用于双曲平面的三角形镶嵌。[50] 通过群作用,群的模式与被作用对象的结构相联系。

在化学中,点群描述分子对称性,而空间群描述晶体学中的晶体对称性。这些对称性是这些系统化学和物理行为的基础,群论使得量子力学分析这些性质变得更加简化。[52] 例如,群论被用来证明,某些量子能级之间的光学跃迁不能发生,仅仅因为涉及的态的对称性。[53]

群论有助于预测材料在发生相变时物理性质的变化,例如从立方晶体形态到四面体晶体形态的变化。一个例子是铁电材料,在居里温度下从顺电态到铁电态的变化与从高对称性的顺电态到低对称性的铁电态的转变有关,伴随着所谓的软声子模式——一种在相变时频率趋于零的振动晶格模式。[54]

这种自发对称性破缺在粒子物理学中得到了进一步应用,其中它的发生与金斯通玻色子的出现相关。[55]
\begin{figure}[ht]
\centering
\includegraphics[width=6cm]{./figures/e0edc37d6c012b24.png}
\caption{巴克敏斯特富勒烯展示了二十面体对称性[56]。} \label{fig_GroupM_8}
\end{figure}
\begin{figure}[ht]
\centering
\includegraphics[width=6cm]{./figures/5851e7698438cfa4.png}
\caption{氨(NH3)的对称群的阶数为6,由120°旋转和反射生成[57]。} \label{fig_GroupM_9}
\end{figure}
\begin{figure}[ht]
\centering
\includegraphics[width=6cm]{./figures/78fb10d7f011844c.png}
\caption{立方烯(C8H8)具有八面体对称性[58]。} \label{fig_GroupM_10}
\end{figure}
\begin{figure}[ht]
\centering
\includegraphics[width=6cm]{./figures/ea93adcbe78932ce.png}
\caption{四氯铂酸根(II)离子,[PtCl4]²⁻,展示了平面正方形几何结构。} \label{fig_GroupM_11}
\end{figure}
有限对称群,例如马修群,用于编码理论,进而应用于传输数据的错误校正和 CD 播放器中。[59] 另一个应用是微分伽罗瓦理论,它描述了具有特定形式的反导数的函数,给出了群论标准,用于判断某些微分方程的解何时是良好的。[q] 在(几何)不变性理论中,研究在群作用下保持稳定的几何性质。[60]
\subsubsection{一般线性群和表示论}
\begin{figure}[ht]
\centering
\includegraphics[width=6cm]{./figures/3289e41115bf3846.png}
\caption{两个向量(左图)通过矩阵相乘(中图和右图)。中图表示顺时针旋转90°,而最右边的图将 \(x\)-坐标按2的倍数拉伸。} \label{fig_GroupM_12}
\end{figure}
矩阵群由矩阵及其矩阵乘法组成。一般线性群 \(\mathrm{GL}(n, \mathbb{R})\) 包含所有具有实数元素的可逆 \(n \times n\) 矩阵。[61] 它的子群被称为矩阵群或线性群。上面提到的二面体群示例可以被看作是一个(非常小的)矩阵群。另一个重要的矩阵群是特殊正交群 \(\mathrm{SO}(n)\)。它描述了 \(n\) 维空间中的所有可能旋转。这个群中的旋转矩阵在计算机图形学中有应用。[62]

表示论既是群概念的应用,也是深入理解群的重要工具。[63][64] 它通过群对其他空间的群作用来研究群。群表示的一大类是线性表示,其中群作用于向量空间,例如三维欧几里得空间 \(\mathbb{R}^3\)。群 \(G\) 在 \(n\)-维实向量空间上的表示仅仅是从群到一般线性群 \(\mathrm{GL}(n, \mathbb{R})\) 的一个群同态 \(\rho: G \to \mathrm{GL}(n, \mathbb{R})\)。通过这种方式,可能抽象给出的群运算转化为矩阵的乘法,从而使其能够进行显式计算。[r]

群作用为研究被作用的对象提供了进一步的手段。[s] 另一方面,它也提供了关于群的信息。群表示是有限群、李群、代数群和拓扑群理论中的一种组织原理,尤其是在(局部)紧群的研究中。[63][65]
\subsubsection{伽罗瓦群}   
伽罗瓦群是为帮助解多项式方程而发展起来的,通过捕捉它们的对称性特征。[66][67] 例如,二次方程\(ax^2 + bx + c = 0\)的解为  
\[x = \frac{-b \pm \sqrt{b^2 - 4ac}}{2a}~\]
每个解可以通过将 \( \pm \) 符号替换为 \( + \) 或 \( - \) 来得到;类似的公式已知适用于三次方程和四次方程,但对于五次及更高次方程通常不存在。[68] 在二次公式中,改变符号(排列得到的两个解)可以看作是一个(非常简单的)群运算。类似的伽罗瓦群作用于高次多项式方程的解,并与其解的公式的存在密切相关。这些群的抽象性质(特别是它们的可解性)提供了一个标准,用于判断是否可以仅通过加法、乘法和根运算来表达这些多项式的解,类似于上述公式。[69]

现代伽罗瓦理论通过转向域理论并考虑作为多项式分裂域的域扩展,推广了上述类型的伽罗瓦群。通过伽罗瓦理论的基本定理,这一理论建立了域与群之间的精确关系,再次强调了群在数学中的普遍性。[70]
\subsection{有限群}  
如果一个群具有有限个元素,则称该群为有限群。元素的数量称为群的阶数。[71] 一个重要的类别是对称群 \(\mathrm{S}_N\),即 \(N\) 个对象的置换群。例如,3个字母的对称群 \(\mathrm{S}_3\) 是所有可能的对象重排的群。三个字母 ABC 可以重排为 ABC、ACB、BAC、BCA、CAB、CBA,总共形成 6 个元素(3 的阶乘)。群运算是这些重排的合成,单位元素是保持顺序不变的重排操作。这个类别是基础性的,因为根据凯莱定理,任何有限群都可以表示为某个适当整数 \(N\) 的对称群 \(\mathrm{S}_N\) 的子群。与上面正方形的对称群类似,\(\mathrm{S}_3\) 也可以解释为等边三角形的对称群。

群 \(G\) 中元素 \(a\) 的阶是最小的正整数 \(n\),使得 \(a^n = e\),其中 \(a^n\) 表示 \(a\) 重复 \(n\) 次的运算,换句话说,就是 \(a\) 的 \(n\) 个副本应用运算“⋅”。

也就是说,将运算“⋅”应用于 \(a\) 的 \(n\) 个副本。(如果“⋅”表示乘法,那么 \(a^n\) 就是 \(a\) 的 \(n\) 次方。)在无限群中,可能不存在这样的 \(n\),在这种情况下,元素 \(a\) 的阶被称为无限。元素的阶等于由该元素生成的循环子群的阶。

更复杂的计数技术,例如计数陪集,可以提供关于有限群的更精确的陈述:拉格朗日定理指出,对于一个有限群 \(G\),任何有限子群 \(H\) 的阶数都能整除 \(G\) 的阶数。西洛定理给出了部分逆命题。

正方形的对称群 \(D_4\) 是一个阶数为 8 的有限群。在这个群中,元素 \(r_1\) 的阶是 4,生成该元素的子群 \(R\) 的阶也是 4。反射元素 \(f_v\) 等的阶是 2。这两个阶数都能整除 8,正如拉格朗日定理所预测的那样。模素数 \(p\) 的乘法群 \(\mathbb{F}_p^\times\) 的阶为 \(p-1\)。
\subsubsection{有限阿贝尔群}  
任何有限阿贝尔群都同构于有限循环群的积;这一陈述是有限生成阿贝尔群基本定理的一部分。

任何素数阶 \(p\) 的群都同构于循环群 \(Z_p\)(这是拉格朗日定理的一个结果)。任何阶数为 \(p^2\) 的群是阿贝尔群,同构于 \(Z_{p^2}\) 或 \(Z_p \times Z_p\)。但是,存在阶数为 \(p^3\) 的非阿贝尔群;上面提到的阶数为 \(2^3\) 的二面体群 \(D_4\) 就是一个例子。[72]
\subsubsection{简单群}  
当一个群 \(G\) 有一个正常子群 \(N\),且 \(N\) 不是 \(\{1\}\) 或 \(G\) 本身时,有关 \(G\) 的问题有时可以简化为有关 \(N\) 和 \(G/N\) 的问题。如果一个非平凡群没有这样的正常子群,则称该群为简单群。有限简单群对于有限群的关系就像素数对于正整数的关系:它们作为构建块,在乔丹–霍尔德定理中给出了这一点的精确定义。
\subsubsection{有限简单群的分类}    
计算机代数系统已被用来列出所有阶数不超过 2000 的群。[t] 但是,分类所有有限群被认为是一个过于困难的问题,无法解决。

所有有限简单群的分类是当代群论中的一项重大成就。这些群中有几个无限族群,以及 26 个“不规则群”,它们不属于任何一个族群。最大的一个不规则群被称为怪物群。由理查德·博尔赫兹(Richard Borcherds)证明的怪物月亮光猜想将怪物群与某些模函数联系起来。[73]

简单群分类和所有群分类之间的差距在于扩展问题。[74]
\subsection{具有附加结构的群}  
群的一个等价定义是将群公理中的“存在”部分替换为运算,其结果是必须存在的元素。因此,群是一个集合 \(G\),配备有一个二元运算 \(G \times G \to G\)(群运算),一个一元运算 \(G \to G\)(提供逆元素)和一个零元运算,它没有操作数,结果是单位元素。否则,群的公理完全相同。这种定义变体避免了存在量词,广泛应用于群的计算和计算机辅助证明。

这种定义群的方式适用于一般化,例如类别中的群对象的概念。简而言之,这是一个具有模仿群公理的同态的对象。[75]
\subsubsection{拓扑群}
\begin{figure}[ht]
\centering
\includegraphics[width=6cm]{./figures/56764dc97a069391.png}
\caption{复平面中的单位圆在复数乘法下是一个李群,因此也是一个拓扑群。它是拓扑的,因为复数乘法和除法是连续的。它是一个流形,因此是一个李群,因为每一个小部分,例如图中的红色弧线,看起来像实数线的一部分(如底部所示)。} \label{fig_GroupM_13}
\end{figure}
一些拓扑空间可能配备有群运算。为了使群运算和拓扑能够良好地结合,群运算必须是连续函数;非正式地说,当 \(g\) 和 \(h\) 仅略微变化时,\(g \cdot h\) 和 \(g^{-1}\) 不能剧烈变化。这类群被称为拓扑群,它们是拓扑空间范畴中的群对象。[76] 最基本的例子是实数加法群和非零实数乘法群。类似的例子可以从任何其他拓扑域中构造,例如复数域或\(p\)-进数域。这些例子是局部紧的,因此它们具有 Haar 测度,并且可以通过调和分析进行研究。其他局部紧拓扑群包括局部域或阿代尔环上的代数群的点的群;这些在数论中是基础。[77] 无限代数域扩展的伽罗瓦群配备了克鲁尔拓扑,在无限伽罗瓦理论中起到作用。[78] 代数几何中使用的一个推广是 étale 基本群。[79]
\subsubsection{李群}   
李群是一个同时具有可微流形结构的群;非正式地说,这意味着它局部看起来像某个固定维度的欧几里得空间。[80] 再次强调,定义要求附加的结构(此处为流形结构)是兼容的:群运算和逆运算要求是光滑的。

一个标准例子是上面介绍的一般线性群:它是所有 \(n \times n\) 矩阵空间的一个开子集,因为它由不等式 
\[\det(A) \neq 0~\] 
给出,其中 \(A\) 表示一个 \(n \times n\) 矩阵。[81]

李群在现代物理学中具有基础性的重要性:诺特定理将连续对称性与守恒量联系起来。[82] 旋转以及空间和时间中的平移是力学定律的基本对称性。例如,它们可以用来构建简单的模型——例如,在某个情况中施加轴对称性通常会导致所需求解的方程的显著简化,从而提供物理描述。[u] 另一个例子是洛伦兹变换群,它将两个相对运动观察者的时间和速度测量联系起来。这些变换可以通过将其表达为闵可夫斯基空间的旋转对称性来纯粹通过群论推导出来。后者在没有显著引力的情况下作为特殊相对论中时空的模型。[83] 闵可夫斯基空间的完整对称群,即包括平移的群,称为庞加莱群。根据上述内容,它在特殊相对论中起着关键作用,并且在量子场论中也具有重要意义。[84] 随位置变化的对称性对于利用规范理论描述物理相互作用的现代方法至关重要。一个重要的规范理论例子是标准模型,它描述了已知的四种基本力中的三种,并对所有已知的基本粒子进行了分类。[85]
\subsection{泛化}




\subsection{参见}
\begin{itemize}
\item 群论主题列表
\end{itemize}
\subsection{注释}\\  
a. 一些作者包括一个额外的公理,称为“运算下的封闭性”,这意味着对于群 \(G\) 中的每个 \(a\) 和 \(b\),\(a \cdot b\) 是群 \(G\) 的元素。这个条件被要求“⋅”是群 \(G\) 上的二元运算所包含。参见 Lang 2002。\\   
b. 数学出版物的 MathSciNet 数据库列出了仅 2020 年就有 1,779 篇关于群论及其泛化的研究论文。参见 MathSciNet 2021。\\   
c. 通常避免使用分数表示法 \(b/a\),除非 \(G\) 是阿贝尔群,因为这会引起歧义,不清楚它是表示 \(a^{-1} \cdot b\) 还是 \(b \cdot a^{-1}\)。\\   
d. 参见例如 Lang 2002,Lang 2005,Herstein 1996 和 Herstein 1975。\\   
e. 同态一词来源于希腊语 ὁμός(相同)和 μορφή(结构)。参见 Schwartzman 1994,第108页。\\   
f. 然而,一个群并不是由其子群格所决定的。参见 Suzuki 1951。\\   
g. 参见 Seifert–Van Kampen 定理的例子。\\   
h. 一个例子是群的上同调,其等于其分类空间的奇异上同调,参见 Weibel 1994,第8.2节。\\   
i. 具有乘法逆元的元素称为单位,参见 Lang 2002,第84页,§II.1。\\   
j. 从整数到有理数的过渡,通过包括分数,得到了分式域的概念。\\   
k. 对于任何域 \(F\)(代替 \(\mathbb{Q}\))也同样如此。参见 Lang 2005,第86页,§III.1。\\   
l. 例如,一个域的乘法群的有限子群必然是循环群。参见 Lang 2002,定理 IV.1.9。模块的扭转和简单代数的概念是这一原则的其他实例。\\   
m. 所述属性是素数的一个可能定义。参见素元素。\\   
n. 例如,Diffie–Hellman 协议使用离散对数。参见 Gollmann 2011,第15.3.2节。\\   
o. 循环群元素的加法表示法将是 \(t \cdot a\),其中 \(t\) 属于 \(\mathbb{Z}\)。\\   
p. 更严格地说,每个群都是某个图的对称群;参见 Frucht 定理,Frucht 1939。\\   
q. 更精确地说,考虑的是微分方程解的向量空间上的单调作用。参见 Kuga 1993,第105–113页。\\   
r. 这对于有限简单群的分类至关重要,参见 Aschbacher 2004。\\   
s. 参见,例如,Schur 引理,关于群作用对简单模块的影响。一个更复杂的例子是绝对伽罗瓦群对 étale 上同调的作用。\\   
t. 直到同构为止,最多有约 490 亿个阶数不超过 2000 的群。参见 Besche, Eick & O'Brien 2001。\\   
u. 参见 Schwarzschild 度量示例,在该示例中,对称性大大简化了物理系统的复杂性分析。
\subsection{引用文献}  
\begin{enumerate}
\item Herstein 1975,第26页,§2。  
\item Hall 1967,第1页,§1.1:“群的概念贯穿于纯数学和应用数学的各个领域。”  
\item Lang 2005,第360页,附录2。  
\item Cook 2009,第24页。  
\item Artin 2018,第40页,§2.2。  
\item Lang 2002,第3页,I.§1 和第7页,I.§2。  
\item Lang 2005,第16页,II.§1。  
\item Herstein 1975,第54页,§2.6。  
\item Wussing 2007。  
\item Kleiner 1986。  
\item Smith 1906。  
\item Galois 1908。  
\item Kleiner 1986,第202页。  
\item Cayley 1889。  
\item Wussing 2007,§III.2。  
\item Lie 1973。  
\item Kleiner 1986,第204页。  
\item Wussing 2007,§I.3.4。  
\item Jordan 1870。  
\item von Dyck 1882。
\item Curtis 2003。  
\item Mackey 1976。  
\item Borel 2001。  
\item Solomon 2018。  
\item Ledermann 1953,第4–5页,§1.2。  
\item Ledermann 1973,第3页,§I.1。  
\item Lang 2005,第17页,§II.1。  
\item Artin 2018,第40页。  
\item Lang 2002,第7页,§I.2。  
\item Lang 2005,第34页,§II.3。  
\item Mac Lane 1998。  
\item Lang 2005,第19页,§II.1。  
\item Ledermann 1973,第39页,§II.12。  
\item Lang 2005,第41页,§II.4。  
\item Lang 2002,第12页,§I.2。  
\item Lang 2005,第45页,§II.4。  
\item Lang 2002,第9页,§I.2。  
\item Magnus, Karrass & Solitar 2004,第56–67页,§1.6。  
\item Hatcher 2002,第30页,第I章。  
\item Coornaert, Delzant & Papadopoulos 1990。  
\item 例如,类群和 Picard 群;参见 Neukirch 1999,特别是 §§I.12 和 I.13。  
\item Seress 1997。  
\item Lang 2005,第七章。  
\item Rosen 2000,第54页,(定理 2.1)。 
\item Lang 2005,第292页,§VIII.1。  
\item Lang 2005,第22页,§II.1。  
\item Lang 2005,第26页,§II.2。  
\item Lang 2005,第22页,§II.1(示例11)。  
\item Lang 2002,第26、29页,§I.5。  
\item Ellis 2019。  
\item Weyl 1952。  
\item Conway 等,2001。参见 Bishop 1993。  
\item Weyl 1950,第197–202页。  
\item Dove 2003。  
\item Zee 2010,第228页。  
\item Chancey & O'Brien 2021,第15、16页。  
\item Simons 2003,§4.2.1。  
\item Eliel, Wilen & Mander 1994,第82页。  
\item Welsh 1989。  
\item Mumford, Fogarty & Kirwan 1994。  
\item Lay 2003。  
\item Kuipers 1999。  
\item Fulton & Harris 1991。  
\item Serre 1977。  
\item Rudin 1990。  
\item Robinson 1996,第viii页。  
\item Artin 1998。  
\item Lang 2002,第六章(特别参见第273页的具体示例)。  
\item  Lang 2002,第292页,(定理 VI.7.2)。  
\item Stewart 2015,§12.1。
\end{enumerate}