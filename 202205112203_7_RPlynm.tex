% 环上的多项式
% keys 多项式环|长除法

\pentry{一元多项式\upref{OnePol},欧几里得环\upref{EuRing}}

\footnote{本小节节选自《小时百科教材系列》的《代数学》.}
我们知道,函数是一种映射,特指“值域是数字集合”的映射.这里的“数字集合”,通常指任何一个环,换句话说,只要是个环,其元素都可以被视为“数字”.我们熟悉的整数环、有理数域\footnote{域是一类特殊的环.}、实数域、复数域等都是很好的例子.

多项式就是一种极为重要的函数.在微积分中,性质良好的函数(解析函数)都可以被表示为一列多项式函数的极限,或者说总可以用一个多项式函数来逼近它.而多项式的性质较为简单,求导也很容易.我们现在讨论的是代数,所以就不关心可以怎么用求导啊积分啊等手段去处理多项式函数,而是关心这个概念可以怎么在代数上拓展.

先来观察一下我们熟知的多项式吧.作为实变量函数,一个多项式$f(x)$可以表示为:

\begin{equation}
    f(x) = \sum_{i=0}^N a_i x^i
\end{equation}
其中各$a_i$都是实数,而$x^i$是用来抽象表示“任意自变量”的.我们可以给$x$赋值,比如取一个实数$c$,然后令$x=c$,这样就能得到一个\textbf{实数}$f(c)=\sum_{i=0}^N a_i c^i$.但要是不赋值,那$f(x)$就不是一个具体的数,而是一个映射;$x$只是一个抽象符号.

两个多项式之间根本的不同,体现在哪里呢?是抽象符号吗?显然不是.$x^2+1$和$y^2+1$完全可以视为同一个多项式,用的符号\footnote{除非我们已经声明$x$和$y$是不同的符号,比如讨论多元多项式的时候.}不同而已.决定两个多项式差异的,是多项式的\textbf{系数},对吧?$x^2+1$和$x^2+3$就不是同一个多项式.很明显,我们应该拓展的是系数的概念.




\begin{definition}{}

设$R$是一个环,$x$是一个抽象符号.则形如
\begin{equation}
    f(x) = \sum_{i=0}^N a_i x^i
\end{equation}
的表达式称为环$R$上的(一元)\textbf{多项式(polynomial)},其中各$a_i\in R$,称为该多项式的\textbf{系数(coefficient)}.$a_i x^i$称为$f(x)$的$i$次\textbf{项(item)},或者一个单项式,$i$称为其\textbf{次数(degree)}.

全体以$x$为抽象符号的多项式构成的集合,记为$R[x]$\footnote{注意这里用的是中括号.未来我们会讨论到域的扩张,那里用的是小括号,要区分.}.

类似地,设各$x_k$都是抽象符号,那么形如
\begin{equation}
    f(x_1, x_2, \cdots, x_m) = \sum_{k_1+k_2+\cdots k_m=0}^N a_{k_1, k_2, \cdots, k_m} x_1^{k_1}x_2^{k_2}\cdots x_m^{k_m}
\end{equation}
的表达式称为环$R$上的\textbf{$m$元多项式(polynomial)},其中各$a_{k_1, k_2, \cdots, k_m}\in R$,称为其系数.每个$a_{k_1, k_2, \cdots, k_m} x_1^{k_1}x_2^{k_2}\cdots x_m^{k_m}$是该多项式的一个$k_1+k_2+\cdots k_m$次项.

多项式中最高次的单项式的次数,称为该多项式的\textbf{次数(degree)}.多项式$f(x)$的次数记为$\opn{deg} f(x)$.在不至于混淆的情况下,也可以简单地用$f$来表示多项式$f(x)$或$f(x_1, x_2, \cdots, x_m)$.

\end{definition}

举几个例子.$x^2y+xy+y^3+2y$是一个\textbf{整数环}上的\textbf{二元三次}多项式,其中$x^2y$和$y^3$是其$3$次项.

我们常会用到“多项式环”这一术语,这是因为环上的全体多项式构成的集合$R[x]$,还真就是一个环.

\begin{exercise}{}
给定环$R$和一个抽象符号$x$.证明:$R[x]$构成一个环.

$R[x]$上加法和乘法的定义\footnote{说白了,就是实系数多项式的加法和乘法的自然推广.我只是为了严谨,才啰嗦这么多.}:同类单项式(即仅系数不同的单项式)之间的加法定义为以其系数相加的结果为系数的同类单项式,乘法类似地定义为系数的乘法;多项式乘法由单项式加法、乘法以及乘法对加法的分配律定义.
\end{exercise}

上面说了那么多,我一直在强调$x$是“抽象符号”.作为实变量函数的多项式,可以把抽象符号替换为实数来赋值,一般的环上当然也可以这么做.设$f$是环$R$上的一元多项式,那任取一个$r\in R$,依然有$f(r)\in R$.




\subsection{多项式除法}

设$f(x)$是一个多项式,那么使得$f(r)=0$的$r$就被称为这个多项式的\textbf{根(root)}.根的性质决定了多项式的性质.为了理解这一点,我们要先讨论一下多项式之间的除法.

由于环没有要求乘法逆元存在性,故除法并不总是可行的.比如$5/2$的结果就不在整数环中,尽管$5$和$2$都是整数.但是也正因为这样,环上诞生了独特的“带余除法”,对,就是小学学过的$5/2=2\cdots 1$.但是$a/b=c\cdots r$的表述方法挺累赘的,不如写成$a=bc+r$好了,$a$是被除数,$b$是除数,$c$是商,$r$是余数.

尽管上述讨论中的$a, b, c, r$都是整数,我们也可以把这个概念移植到一般的环上.在稍后我们会讨论的“欧几里得环”中,这种带余除法非常重要,但现在我们就着眼于多项式环即可.

考虑环$R$上的多项式$f(x)$和$g(x)$,则总存在$h(x)$和$r(x)$,其中$\opn{deg}r\leq\opn{deg}g$,使得$f(x)=h(x)g(x)+r(x)$.这就是多项式之间的除法.如果遇到$r(x)=0$的情况,那么就说$g$\textbf{整除}$f$,记为$g|f$.

如果环$R$上的多项式$f(x)$可以表示为两个多项式的乘积$h(x)g(x)$,或者说它可以被另一个次数为正的多项式整除,那么我们就说这个多项式是\textbf{可约(reducible)}的,而$h$和$g$就被称为其\textbf{因子};否则,称$f$是\textbf{不可约(irreducible)}的.这样,我们只需要研究好其因子的性质,就能方便地推知$f$本身的性质.

现在,我们引入一个实际计算多项式除法的方法,称为长除法,用表格表示.为方便理解,我们直接用一个具体实例来讲解:在整数环上,用$2x^2+1$去除$7x^5+x^4+3x^3-x^2-2$.

第一步,是比较两个多项式的最高次项,即$2x^2$和$7x^5$,然后凑一个

\begin{table}[ht]
\centering
\caption{}\label{RPlynm_tab1}
\begin{tabular}{}
\hline
 &  & $x^3$ & * \\
\hline
$2x^2+1$ & $7x^5+x^4+3x^3-x^2-2$ & * & * \\
\hline
* & * & * & * \\
\hline
* & * & * & * \\
\hline
\end{tabular}
\end{table}
























