% RDP 远程桌面笔记
% license Xiao
% type Note

% 留在百科

\begin{issues}
\issueDraft
\end{issues}

\subsection{控制 Windows}
\begin{itemize}
\item \textbf{远程桌面协议(Remote Desktop Protocol, RDP)}
\item Windows 自带的\textbf{远程连接(Remote Desktop Connection)}应用
\item 默认端口为 3389,但在 RDC 应用中可以用 \verb|ip:端口号| 来指定别的端口。
\item 并不自带内网穿透, 如果没有代理服务器, 只能在局域网内连接。 设置代理服务器见 “\enref{FRP 内网穿透笔记}{NATthr}”。 至于 WSL1 如何开机启动 FRP 客户端, 参考 “\enref{WSL 笔记}{WSLnt}”
\item Windows 一次只能登录一个用户(包括远程),如果电脑上已经登录一个用户, 远程连接登录另一个, 则当前用户会被 logout,进程全部结束。
\item 如果电脑上登录一个用户,远程也登录这个用户,则当前用户被锁屏,进程继续运行。
\end{itemize}

\subsection{控制 Ubuntu 22.04 (未成功)}
\begin{itemize}
\item \verb|sudo apt install xrdp|
\item \verb|sudo systemctl start xrdp|
\item \verb|sudo systemctl enable xrdp|
\item \verb|sudo ufw allow 3389|
\item 用 windows 远程控制 app 连接, 选择 Xorg, 输入用户名密码,显示蓝屏。
\item 这似乎说明 Gnome 和 Xorg 并不兼容, 建议安装 xfce 轻量级桌面。
\end{itemize}
