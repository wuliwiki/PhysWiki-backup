% 字符串的分片与索引
% 字符串|分片|索引

\pentry{Python 基本变量类型\upref{PyType}\upref{Sample}}

\textbf{字符串}可以通过string[x] 的方式进行索引、分片,也就是加一个[](像不像一把刀).字符串的分片(slice)实际上可以看作是从字符串这个大面包\autoref{Strsi_fig1} 中找出来你要吃的那片美味(复制出来一小段你要的长度),放在你的嘴里(储存在另一个地方),而不会对字符串这个源文件改动.分片获得的每个字符串可以看作是原字符串的一个副本(面包片).

\begin{figure}[ht]
\centering
\includegraphics[width=10cm]{./figures/Strsi_1.png}
\caption{切片} \label{Strsi_fig1}
\end{figure}

我们来看一段程序:
\begin{lstlisting}[language=python]
name = 'My name is Mike'
print(name[0])
'M'
print(name[-4])
'M'
print(name[11:14]) # 从第一个到第十四个,第十四个不包括在内
'Mik'
print(name[11:15]) # 从第一个到第十五个,第十五个不包括在内
'Mike'
print(name[5:])  #代表着从编号为5的字符到结束的字符串分片.
'me is Mike'
print(name[:5]) #从编号为0的字符开始到编号为5但不包含第5个字符
'My na'
\end{lstlisting}
\textbf{\textsl{:}}两边分别代表着字符串的分割从哪里开始,并到哪里结束.
\begin{table}[ht]
\centering
\caption{请输入表格标题}\label{Strsi_tab1}
\begin{tabular}{|c|c|c|c|c|c|c|c|c|c|c|c|c|c|c|c|}
\hline
 字符  & M & y &   & n & a & m & e &   & i & s &   & * & * & * \\
\hline
* 序号 * & * & * & * & * & * & * & * & * & * & * & * & * & * & * \\
\hline
* 反序 * & * & * & * & * & * & * & * & * & * & * & * & * & * & * \\
\hline
\end{tabular}
\end{table}