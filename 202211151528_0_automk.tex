% Automake 笔记

\begin{issues}
\issueDraft
\end{issues}

\pentry{Makefile 笔记\upref{Make}}

\addTODO{libtool 讲一讲}

\subsection{用户}
经典的三部使用
\begin{lstlisting}[language=bash]
./configure [一些选项]
make [-j4]
make check # 跑一些 test (可选)
make install
\end{lstlisting}
其中的选共享可以通过 \verb|.configure --help| 查看. 例如 \verb|--prefix=安装目录|

若要 build 一个 debug 版本, 参考\href{https://stackoverflow.com/questions/4553735/gnu-autotools-debug-release-targets}{这里}的高赞. 例如 arb\upref{ArbLib} 中(注意 \verb|./config --help| 中只有 \verb|CFLAGS| 没有 \verb|CPPFLAGS|), 用
\begin{lstlisting}[language=bash]
./configure --prefix=/home/addis/arb-debug-install \
     CFLAGS=-DDEBUG CFLAGS="-g3 -O0"
make [-j4]
make install
\end{lstlisting}
即可. 现在在编译主程序后用 gdb 调试就可以进入到 arb 的内部函数中调试.

\subsection{作者}
\begin{itemize}
\item 参考\href{https://www.gnu.org/software/automake/manual/html_node/Autotools-Introduction.html}{官方教程}.
\item 一般来说建议学 Cmake\upref{CMakeN} 而不是 autotool. Makefile 的基础还是要学的.
\end{itemize}
