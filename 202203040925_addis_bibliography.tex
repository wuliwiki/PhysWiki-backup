% 每个部分请根据从入门到进阶排序
\begin{thebibliography}{99}
%数理逻辑
%=========================================
\bibitem{BasicSetTheory}
A. Shen, N. K. Vereshchagin, 陈光还译,\textsl{集合论基础}, 高等教育出版社, 2013年
% 微积分
%=========================================
\bibitem{Thomas}
J. Hass, C. Heil, M. Weir, \textsl{Thomas' Cauculus} 14ed
\bibitem{同济高}
同济大学数学系. \textsl{高等数学} 第六版
% 线性代数
%=========================================
\bibitem{Axler}
Sheldon Axler. \textsl{Linear Algebra Done Right} 3ed
\bibitem{同济线}
同济大学数学系. \textsl{线性代数} 第五版
% 抽象代数
%=========================================
% 微分几何
%=========================================
\bibitem{梁书}
梁灿杉, 周杉. \textsl{微分几何与广义相对论} 第二版
\bibitem{GTM275}
Loring W. Tu. \textsl{Differential Geometry: Connections, Curvature, and Characteristic Classes}, GTM 275, Springer press.
\bibitem{CarrollGR}
Sean M. Carroll, \textsl{Lecture Notes on General Relativity}, Institute for Theoretical Physics, UCSB, arXiv:gr-qc/9712019v1 3Dec 1997. 
% 概率与统计
%=========================================
% 偏微分方程和特殊函数
%=========================================
\bibitem{Arfken}
Arfken, Weber, Harris. \textsl{Mathematical Methods for Physicists - A Comprehensive Guide} 7ed
% 数学分析
%=========================================
\bibitem{Rudin}
Walter Rudin. \textsl{Principle of Mathematical Analysis}
% 实变函数、实分析、广义函数
%=========================================
\bibitem{十一五实变函数论}
江泽坚,吴智泉,纪友清.\textsl{实变函数论} 第三版
% 泛函分析
%=========================================
\bibitem{Zeidler}
Eberhard Zeidler. \textsl{Applied Functional Analysis - Applications to Mathematical Physics}
% 力学
%=========================================
\bibitem{量纲}
梁灿彬, 曹周键. \textsl{量纲理论与应用} 第一版
\bibitem{Goldstein}
Herbert Goldstein. \textsl{Classical Mechanics} 3ed
\bibitem{新力}
赵凯华, 罗蔚茵. \textsl{新概念物理教程 力学} 第二版
% 电动力学
%=========================================
\bibitem{GriffE}
David Griffiths, \textsl{Introduction to Electrodynamics}, 4ed
\bibitem{新电}
赵凯华, 陈熙谋. \textsl{新概念物理教程 电磁学} 第二版
% 量子力学
%=========================================
\bibitem{GriffQ}
David Griffiths, \textsl{Introduction to Quantum Mechanics}, 4ed
\bibitem{新量}
赵凯华, 罗蔚茵. \textsl{新概念物理教程 量子物理} 第二版
\bibitem{Shankar}
R. Shankar. \textsl{Principles of Quantum Mechanics} 2ed
\bibitem{Merzbacher}
Eugen Merzbacher. \textsl{Quantum  Mechanics} 3ed
\bibitem{Sakurai}
J.J. Sakurai. \textsl{Modern Quantum Mechanics} Revised Edition
\bibitem{Landau}
朗道. \textsl{理论物理教程 第三卷 量子力学(非相对论理论)} 第六版
\bibitem{Teschl}
Gerald Teschl. \textsl{Mathematical Methods in Quantum Mechanics}
\bibitem{高量}
李蕴才. \textsl{高等量子力学} 河南大学出版社, 2000:337-347
% 原子分子
%=========================================
\bibitem{Brandsen}
Brandsen, \textsl{Physics of Atoms and Molecules}, 2ed
\bibitem{Burke}
Burke, \textsl{R-Matrix Theory of Atomic Collisions - Application to Atomic, Molecular and Optical Processes}
\bibitem{Newton}
Roger G. Newton, \textsl{Scattering Theory of Waves and Particles}, 2ed
% 统计力学
%=========================================
\bibitem{Schroeder}
Daniel V. Schroeder, \textsl{An Introduction to Thermal Physics}
\bibitem{热学}
秦允豪.\textsl{普通物理学教程 热学} 第三版
\bibitem{新热}
赵凯华, 罗蔚茵. \textsl{新概念物理教程 热学} 第二版
%=========================================
\bibitem{热统}
汪志诚.\textsl{热力学·统计物理}  第五版
% 宇宙学
%=========================================
\bibitem{Peebles}
P. Peebles \textsl{Principles of Physical Cosmology}
% 量子场论
%=========================================
\bibitem{Peskin}
Peskin. \textsl{An introduction To Quantum Field Theory}
\bibitem{SchwartzQFT}
Matthew D. Schwartz. \textsl{Quantum Field Theory and the Standard Model}, Cambridge University Press.
% 扭结与万物系列(Knots and Everything)
%=========================================
\bibitem{KnotsVol4}
John Baez, Javier P. Muniain. \textsl{Gauge Fields, Knots and Gravity, Series on Knots and Everthing}-Vol. 4, World Scientific press. ISBN-13: 978-981-02-2034-1. 
% 数值计算
%=========================================
\bibitem{NR3}
W. H. Press, et al. \textsl{Numerical Recipes} 3rd edition. 
\bibitem{唐计}
唐朔飞. 计算机组成原理[M]. 高等教育出版社. 2008
% 其他
%=========================================
\bibitem{PhysWiki}
小时百科志愿者. \textsl{小时百科}. \href{https://wuli.wiki}{https://wuli.wiki}. 
\end{thebibliography}
