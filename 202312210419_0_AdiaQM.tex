% 渐进近似(量子力学)
% license Usr
% type Tutor


\begin{issues}
\issueDraft
\end{issues}

\footnote{参考 \cite{GriffE} 章节: The Adiabatic Approximation}若系统初始时处于某个离散非简并的本征态,那么当哈密顿量 $H$ 随时间缓慢改变时(改变的特征时间远大于本征态的周期), 那改变过程中波函数将仍然处于该本征态,除了一个相位因子的改变。

例1: 当无限深势阱\upref{ISW}缓慢变长。

例2: 量子简谐振子(升降算符法)\upref{QSHOop}的劲度系数 $k$ 缓慢变化。

例2: 给氢原子缓慢施加外电场或磁场。

\subsection{推导}
若哈密顿量不随时间改变,
\begin{equation}
\Psi_n(t) = \psi_n \E^{-\I E_n t}~.
\end{equation}
若随时间改变, 本征态和本征值都变为时间的函数 $\psi_n(t)$ 和 $E_n(t)$。 但仍然正交归一。 此时的含时波函数仍然可以用它们展开
\begin{equation}
\Psi(t) = \sum_n c_n(t) \psi_n(t) \E^{-\I \theta_n(t)}~,
\end{equation}
其中
\begin{equation}
\theta_n(t) = \frac{1}{\hbar} \int_0^t E_n(t')\dd{t'}~.
\end{equation}
代入含时薛定谔方程
\begin{equation}
H(t)\Psi(t) = \I \dot \Psi(t)~.
\end{equation}

