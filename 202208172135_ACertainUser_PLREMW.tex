% 简单的偏振电磁波

\begin{issues}
\issueDraft
\end{issues}

\pentry{真空中的平面电磁波\upref{VcPlWv}}

现在,让我们思考一个沿$z$轴传播的电磁波.

\begin{figure}[ht]
\centering
\includegraphics[width=5cm]{./figures/PLREMW_1.png}
\caption{电磁波只有x,y方向的分量}} \label{PLREMW_fig1}
\end{figure}

以电场$E$为例.由于电磁波是横波,所以$E_z=0$.此时,电场就可以写为
$$\bvec E = 
\begin{bmatrix}
E_{x0} \cos(\bvec k\vdot \bvec r - \omega t)\\
E_{y0} \cos(\bvec k\vdot \bvec r - \omega t)\\
0\\
\end{bmatrix}
$$

事实上,电场的每一个分量还可以具有一个独立的初相因子$\varphi_0$.为简明起见,我们假定电场$x,y$两个分量的振幅相同,且$E_x$分量的相位因子为0.\footnote{重要的是分量间的相位差,而不是具体的初相位}
$$\bvec E = 
\begin{bmatrix}
E_{0} \cos(\bvec k\vdot \bvec r - \omega t)\\
E_{0} \cos(\bvec k\vdot \bvec r - \omega t+\varphi_{0})\\
0\\
\end{bmatrix}
$$
根据$\varphi_0$的取值,电磁波也就呈现不同的偏振类型.

\subsubsection{$\varphi_0=n\pi, n=0,\pm1,\pm2,...$:线偏振}
\begin{figure}[ht]
\centering
\includegraphics[width=10cm]{./figures/PLREMW_2.png}
\caption{线偏振} \label{PLREMW_fig2}
\end{figure}

\subsubsection{$\varphi_0=\frac{n\pi}{2}, n=\pm1,\pm3,\pm5,...$:圆偏振}
\begin{figure}[ht]
\centering
\includegraphics[width=10cm]{./figures/PLREMW_3.png}
\caption{圆偏振,\href{https://www.geogebra.org/m/hj6qsfdu}{一个可动的模型}(站外链接)} \label{PLREMW_fig3}
\end{figure}

\subsubsection{其余情况:椭圆偏振}
\begin{figure}[ht]
\centering
\includegraphics[width=10cm]{./figures/PLREMW_4.png}
\caption{椭圆偏振} \label{PLREMW_fig4}
\end{figure}
