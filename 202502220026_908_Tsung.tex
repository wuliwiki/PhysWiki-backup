% 李政道(综述)
% license CCBYSA3
% type Wiki

本文根据 CC-BY-SA 协议转载翻译自维基百科\href{https://en.wikipedia.org/wiki/Tsung-Dao_Lee}{相关文章}。

\begin{figure}[ht]
\centering
\includegraphics[width=6cm]{./figures/cc118c4172086978.png}
\caption{李氏,1956年} \label{fig_Tsung_1}
\end{figure}
李政道(Tsung-Dao Lee,中文:李政道;拼音:Lǐ Zhèngdào;1926年11月24日-2024年8月4日)是一位中美物理学家,以其在宇称不守恒、李–杨定理、粒子物理学、相对论重离子碰撞(RHIC)物理学、非拓扑孤立子和孤立子星方面的工作而闻名。他是纽约哥伦比亚大学的名誉教授,曾在该校教授直到2012年退休。

1957年,李政道在30岁时与杨振宁共同获得诺贝尔物理学奖,因他们在弱相互作用中宇称不守恒的研究工作,该工作由吴健雄在1956至1957年间通过著名的吴实验实验性地证明。

李政道仍然是第二次世界大战后最年轻的诺贝尔科学奖得主。他是历史上第三年轻的诺贝尔科学奖得主,仅次于威廉·L·布拉格(William L. Bragg,1915年与父亲威廉·H·布拉格一起获奖,年仅25岁)和维尔纳·海森堡(Werner Heisenberg,1932年获奖时同样30岁)。李政道和杨振宁是第一批获奖的中国人。由于他于1962年成为美国公民,李政道也是迄今为止最年轻的美国诺贝尔奖得主。
\subsection{传记}  
\subsubsection{家庭}  
李政道出生于中国上海,祖籍邻近的苏州。他的父亲李駿康(Lǐ Jùn-kāng),是南京大学的首批毕业生之一,曾是一位化学工业家和商人,参与了中国早期现代合成肥料的开发。李政道的祖父李仲覃(Lǐ Zhòng-tán)是苏州圣约翰堂(蘇州聖約翰堂)首位华人卫理公会高级牧师。[4][5]

李政道有四个兄弟和一个妹妹。教育家Robert C. T. Lee是李政道的兄弟之一。[6] 李政道的母亲张和他的兄弟Robert C. T. 在1950年代移居台湾。
\subsubsection{早年生活}  
李政道在上海接受中学教育(东吴大学附属中学)和江西(江西联合中学)。由于第二次中日战争,李政道的高中教育中断,因此他未能获得中学毕业文凭。然而,在1943年,李政道直接申请并被录取到国立浙江大学(当时称国立浙江大学)。最初,李政道注册为化学工程系的学生。很快,李政道的才华被发现,他对物理学的兴趣迅速增长。包括舒星北和王淦昌在内的几位物理学教授给予了李政道很大的指导,他很快转入了国立浙江大学物理系,并在那里学习了1943年至1944年。[5][需要更多引用]

然而,随着日本进一步的侵略,李政道于1945年继续在昆明的国立西南联合大学学习,在那里他跟随吴大猷教授学习物理。[5]
\subsubsection{在美国的生活与研究}  
吴健雄,设计了违反宇称守恒的吴实验  
吴教授提名李政道获得中国政府的奖学金,以便前往美国攻读研究生学位。1946年,李政道前往芝加哥大学,并被恩里科·费米教授选中,成为他的博士研究生。李政道在1950年获得博士学位,研究课题为《白矮星的氢含量》。1950年至1951年,李政道在加利福尼亚大学伯克利分校担任研究员和物理学讲师。[7][5]

1953年,李政道加入哥伦比亚大学,并在此工作直到退休。他在哥伦比亚大学的第一项工作是关于量子场论的一个可解模型,这个模型更为人知的是“李模型”。很快,他的研究重点转向了粒子物理学以及K介子衰变的难题。李政道在1956年初意识到,这个难题的关键在于宇称不守恒。根据李政道的建议,首次实验测试是在斯坦贝格小组的超子衰变研究中进行的。当时,实验结果仅表明可能存在2个标准差的宇称违反效应。受到这一可行性研究的鼓舞,李政道与包括杨振宁在内的合作者对弱相互作用中可能的时间反演(T)、宇称(P)、电荷共轭(C)和CP违反进行了系统研究。在吴健雄及其助手确认宇称不守恒的实验结果后,李政道和杨振宁共同获得了1957年诺贝尔物理学奖。然而,吴健雄未能获奖,这被认为是诺贝尔委员会历史上最大的争议之一。[8]

在1960年代初期,李政道和他的合作者开启了高能中微子物理学这一重要领域。1964年,李政道与M·诺恩伯格共同分析了与零静质量粒子相关的发散问题,并提出了一种通用方法,称为KLN定理,用于处理这些发散问题。该定理在现代量子色动力学(QCD)研究中仍然发挥着重要作用,特别是在处理无质量、自相互作用的胶子时。1974至1975年,李政道发表了几篇关于“高密度下的新型物质”的论文,这些论文为现代RHIC物理学领域奠定了基础,现在该领域主导着整个高能核物理学研究。

除了粒子物理学,李政道还活跃于统计力学、天体物理学、流体动力学、多体系统、固态物理和晶格量子色动力学(lattice QCD)等领域。1983年,李政道撰写了一篇题为《时间能否作为离散的动力学变量?》的论文,这引发了李政道和合作者们关于用差分方程来表述基本物理学的系列研究,这些方程在平移和旋转变换的连续群下具有精确的不变性。从1975年起,李政道和合作者们建立了非拓扑孤立子(non-topological solitons)这一领域,这也为他在1980年代和1990年代对孤立子星(soliton stars)和黑洞的研究奠定了基础。[citation needed]

从1997年到2003年,李政道担任理化学研究所-布鲁克海文国家实验室研究中心(RIKEN-BNL Research Center)的主任(现为名誉主任),该中心与哥伦比亚大学的其他研究人员一起,在1998年完成了一个1 teraflops的超级计算机QCDSP用于晶格量子色动力学计算,并在2001年完成了一个10 teraflops的QCDOC机器。[citation needed] 直到2005年,[9] 李政道与理查德·M·弗里德伯格(Richard M. Friedberg)共同开发了一种新的方法来求解薛定谔方程,从而获得了量子简并双壁势能和其他瞬时子问题的收敛迭代解。他们还研究了中微子映射矩阵。[10]

李政道是2008年5月签署给美国总统乔治·W·布什的信函的20位美国诺贝尔物理学奖得主之一。信中呼吁布什“扭转2008财年综合拨款法案对基础科学研究造成的损害”,并请求为能源部科学办公室、国家科学基金会和国家标准与技术研究院提供额外的紧急资金。[11]
\subsection{教育活动}
\begin{figure}[ht]
\centering
\includegraphics[width=6cm]{./figures/c4bfd772109560f4.png}
\caption{} \label{fig_Tsung_2}
\end{figure}
在中美关系恢复正常化后,李政道和他的妻子秦惠䇹(Jeannette Hui-Chun Chin)得以访问中国,李政道在中国进行了一系列讲座和研讨会,并组织了中美物理考试和申请计划(CUSPEA,China-U.S. Physics Examination and Application)。

1998年,李政道为纪念三年前去世的妻子,成立了“秦惠䇹—李政道中国大学生见习基金”(Chun-Tsung Endowment)。该基金由美国基督教高等教育联合会(New York)监督,旨在为六所大学的本科生提供奖学金,这六所大学分别是上海交通大学、复旦大学、兰州大学、苏州大学、北京大学和清华大学。获得该奖学金的学生被称为“秦惠䇹学者”(Chun-Tsung Scholars)。