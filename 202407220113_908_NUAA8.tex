% 南京航空航天大学 2016 量子真题
% license Usr
% type Note

\textbf{声明}:“该内容来源于网络公开资料,不保证真实性,如有侵权请联系管理员”

\subsection{简答题(20 分,每题 10 分)}
①什么是束缚态?它有何特性?束缚态是否必为定态?定态是否必为束缚
态?举例说明。

②球形对称势场中,角动量本征函数是否可以为非球对称?概率密度呢?

\subsubsection{二}
 给定 $\left( \theta, \varphi \right)$ 方向的单位矢量 $\vec{n} = \left( n_x, n_y, n_z \right) = \left( \sin \theta \cos \varphi, \sin \theta \sin \varphi, \cos \theta \right)$ 以及泡利矩阵 $\vec{\sigma} = \left( \sigma_x, \sigma_y, \sigma_z \right)$,求 $\sigma_n = \vec{\sigma} \cdot \vec{n}$ 的本征值和本征函数。(20 分)

\subsection{三}
一个质量为 $m$ 的粒子在下面的无限深方势阱 $V(x) = 
\begin{cases} 
0, & 0 < x < a \\\\ 
\infty, & \text{others} 
\end{cases}$ 中运动,

开始时$(t=0)$,系统处于状态 $\psi(x) = 4\sin(\pi x/a)\cos^3(\pi x/2a)$,其中 $A$ 为常数。求出 $t$ 时刻系统:(30 分)

1. 处于基态的几率;
2. 能量平均值;
3. 动量平均值;
4. 动量均方差根。

\subsection{四}
求电子氢原子基态时 $\langle r \rangle$ 和 $ \langle r^2 \rangle$。

(氢原子基态为 $\psi_{100}(r, \theta, \phi) = e^{-r/a}/{\sqrt{\pi a^3}} $) (20 分)