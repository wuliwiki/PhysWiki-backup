% 一致收敛
% keys 数学分析|一致收敛|极限

\begin{issues}
\issueDraft
\end{issues}

\addTODO{需添加预备知识,收敛或逐点收敛,以及一致连续.}

\subsection{一致收敛的概念}

函数列$f_n$逐点收敛到$f$,只要求对于任意固定的$x_0$,数列$\{f_n(x_0)\}$收敛到$f(x_0)$即可.接下来介绍的\textbf{一致收敛}则是一个更强的要求.

一个函数列 $f_n(x)$ 一致收敛到 $f(x)$ 的定义是: 当对于任意 $\epsilon > 0$, 存在 $N$, 当 $n \geqslant N$ 时对任意 $x$ 都有
\begin{equation}\label{UniCnv_eq1}
\abs{f_n(x) - f(x)} < \epsilon
\end{equation}
或者记为
\begin{equation}
\lim_{n\to\infty} \qty(\max \abs{f_n(x) - f(x)}) = 0
\end{equation}

一致收敛是比(逐点)收敛更强的条件.

\begin{example}{逐点收敛但不一致收敛的例子}\label{UniCnv_ex1}
在区间$(0, 1)$上考虑函数列$f_n(x)=x^n$和函数$f(x)=0$,则容易验证$f_n$逐点收敛到$f$.

但是,对于任意$\epsilon>0$和任意$f_n$,总可以找到$x\in(\epsilon^{1/n}, 1)$,使得$f_n(x)>\epsilon$.因此按照定义,$f_n$\textbf{不}一致收敛到$f$.

\end{example}

\begin{example}{逐点收敛但不一致收敛的例子}\label{UniCnv_ex2}
在区间$[0, 1]$上考虑函数列$f_n(x)=\sin(\pi\cdot x^n)$和函数$f(x)=0$,则容易验证$f_n$逐点收敛到$f$.

但是,对于任意$\epsilon>0$和任意$f_n$,总可以找到$x=\frac{1}{2^n}$,使得$f_n(x)=1>\epsilon$.因此按照定义,$f_n$\textbf{不}一致收敛到$f$.

\end{example}

逐点收敛只考虑,是不是每个点都收敛.而一致收敛的威力在于,有一个统一的进度,每个点收敛的进度不得比这更慢.\autoref{UniCnv_ex1} 和\autoref{UniCnv_ex2} 里举出的反例就是如此,不管你怎么取$\epsilon$作为限定,总存在跟不上进度的点,导致这种限定不是\textbf{一致}的.

如果$f_n$一致收敛到$f$,那么我们会看到,随着$n$增大,$f_n-f$的上下界越来越小,像是把函数列挤压到$x$轴的过程.但是不一致收敛的函数,由于总存在不听话的、跟不上进度的点,就没法把函数压平.

一致收敛还可以从“函数之间的距离”角度来考虑.

\begin{definition}{函数的距离}\label{UniCnv_def1}

在同一定义域上给定两个函数$f$和$g$,定义它们之间的距离为$\opn{d}(f, g)=\opn{sup}\abs{f-g}$,即距离为函数$\abs{f-g}$的\textbf{上确界}\upref{SupInf}.

有时候也记$\opn{d}(f, g)=\abs{\abs{f, g}}$.

\end{definition}

把每个函数看成“函数的集合”里的一个点,\autoref{UniCnv_eq1} 给出了衡量各点之间距离的方式.一致收敛的函数列,就是这个集合里一致收敛的点列.由此可见,一致收敛是一种更注重\textbf{函数整体}的性质.


\subsection{一致收敛的性质}

\begin{theorem}{线性性}
设给定定义域上,函数列$\{f_n\}$和$\{g_n\}$分别一致收敛到$f$和$g$,且$a, b$是任意常数,那么函数列$\{af_n+bg_n\}$一致收敛到$af+bg$上.
\end{theorem}

\begin{theorem}{有界乘积收敛}\label{UniCnv_the1}
设给定定义域上,函数列$\{f_n\}$一致收敛到$f$上,且$g$是一个有界函数,那么$\{gf_n\}$一致收敛到$gf$上.
\end{theorem}

\autoref{UniCnv_the1} 的一个证明思路提示:由于$g$有界(设上下界绝对值中较大的为$G$),因此$gf_n$和$gf$的偏差也是“一致”的,即不会超过$Gf$.由此,对于给定的$n$,如果$\abs{f_n(x)-f(x)}<\epsilon$恒成立,那么$\abs{gf_n(x)-gf(x)}<G\epsilon$恒成立,再由$\epsilon$的任意性即可得证.

如果取$g$是无界函数,那么$g$的趋于无穷的点就会让$\{gf_n\}$出现一个跟不上进度的点,导致$\{gf_n\}$不是一致收敛的,如\autoref{UniCnv_ex3} 所述.

\begin{example}{}\label{UniCnv_ex3}
在$(0, 1)$上考虑函数列$\{f_n(x)=1/n\}$,显然它一致收敛到$f(x)=0$.

取$g(x)=1/x$,那么$g$没有上下界.对于任意正整数$n$和$\epsilon>0$,总有$x\in(0, \frac{1}{n\epsilon})$,使得$g(x)f_n(x)>\epsilon$,从而$\{gf_n\}$不一致收敛.
\end{example}

\begin{theorem}{双有界乘积收敛}\label{UniCnv_the2}
设给定定义域上,函数列$\{f_n\}$和$\{g_n\}$分别一致收敛到$f$和$g$,且$f, g$都是有界函数,那么$\{f_ng_n\}$一致收敛到$fg$上.
\end{theorem}

\autoref{UniCnv_the2} 的一个证明思路提示:设$g$上下界中绝对值较大的为$G$.$g_n$一致收敛,由一致收敛的定义,任取$\epsilon>0$,存在正整数$N$,使得编号$n>N$时,所有$g_n$都共享上下界$G+\epsilon$和$-G-\epsilon$.这样一来,问题就可以归结为\autoref{UniCnv_the1} 的情况,只不过这里固定的有界函数可以取常数函数$G$.

\begin{theorem}{}\label{UniCnv_the3}
设给定定义域上,函数列$\{f_n\}$和$\{g_n\}$分别一致收敛到$f$和$g$,且$f$是有界函数,$\abs{g}$有一个下界$\delta>0$,那么$\{\frac{f_n}{g_n}\}$一致收敛到$\frac{f}{g}$.
\end{theorem}

\autoref{UniCnv_the3} 的一个证明思路提示:利用$\abs{g}$有一个非零下界,证明$1/g_n$是一个有界函数,且$1/g_n$一致收敛到$1/g$.然后问题就可以归结为\autoref{UniCnv_the2} 的情况.

\begin{theorem}{链式收敛}
设给定定义域$X$上,函数列$\{f_n\}$一致收敛到$f$上,且在区间$I$上,$h(x)$是一个\textbf{一致连续}的函数,另外对于任意$x\in X$,都有$f_n(x)\in I, f(x)\in I$.那么$h(f_n(x))$一致收敛到$h(f(x))$上.
\end{theorem}











