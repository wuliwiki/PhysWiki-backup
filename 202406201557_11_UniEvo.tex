% 宇宙的演化
% keys 宇宙演化|暴胀|尺度|中元素|对称破缺|背景辐射|再融合
% license Usr
% type Tutor

\begin{issues}
\issueNeedCite
\issueMissDepend
\end{issues}
\pentry{宇宙学的基本方程\nref{nod_Frieq}}{nod_2e37}
宇宙演化可以大约分为两个时期,(详见\autoref{tab_UniEvo_1} \footnote{翻译自Daniel Baumann,TASI Lectures on Inflation, \href{https://arxiv.org/abs/0907.5424}{arXiv:0907.5424},1.2 节,Table1}),从创世之初($t=0$) 到创世后3分钟我们可以称为前物质时期。这个时期可以大致分为两个时期:从宇宙诞生后到 $10^{-43}$ 秒,此时宇宙沐浴于 $10^{18}\Si{GeV}$ 的高温中,四大相互作用被统一在一起。随后到 $10^{-34}$ 秒宇宙在高温中经历\textbf{暴胀(Inflation)},宇宙尺度因子迅速扩大(详见\autoref{fig_UniEvo_1} \footnote{引自 Daniel Baumann,TASI Lectures on Inflation, \href{https://arxiv.org/abs/0907.5424}{arXiv:0907.5424},1.2 节,Figure2})。在暴胀快结束时超对称破缺,宇宙中的重元素开始产生。

\begin{table}[ht]
\centering
\caption{宇宙大事记(问号表示理论尚未给出合理解释)}\label{tab_UniEvo_1}
\begin{tabular}{|r|r|r|}
\hline
 & 时间 & 能量 \\
\hline
普朗克时间奇点? & $<10^{-43}\Si{s}$ & $10^{18}\Si{GeV}$ \\
\hline
弦论尺度?       & $>10^{-43}\Si{s}$ & $<10^{18}\Si{GeV}$ \\
\hline
超统一?         & $\sim 10^{-36}\Si{s}$ & $10^{15}\Si{GeV}$ \\
\hline
暴胀开始?       & $>10^{-34}\Si{s}$ & $<10^{15} \Si{GeV}$ \\
\hline
超对称破缺?     & $<10^{-10}\Si{s}$ & $>1 \Si{TeV}$ \\
\hline
重元素产生?     & $<10^{-10}\Si{s}$ & $>1 \Si{TeV}$ \\
\hline
电弱统一时期    & $10^{-10}\Si{s}$ & $1 \Si{TeV}$ \\
\hline
夸克-强子转换时期 & $10^{-4}\Si{s}$ & $10^2 \Si{MeV}$ \\
\hline
核子冷却        & $0.01\Si{s}$ & $10 \Si{MeV}$ \\
\hline
中微子退偶      & $1 \Si{s}$  & $1 \Si{MeV}$ \\
\hline
大爆炸原核初合成  & $3 \Si{min}$ & $0.1 \Si{MeV}$ \\
\hline
物质-辐射密度相等 & $10^4\Si{yrs}$  & $1 \Si{eV}$ \\
\hline
再融合 & $10^5\Si{yrs}$  & $0.1 \Si{eV}$ \\
\hline
宇宙黑暗时期 & $10^5 - 10^8\Si{yrs}$  &  \\
\hline
再电离 & $10^8\Si{yrs}$  &  \\
\hline
星系形成 & $\sim 6\e8\Si{yrs}$  &  \\
\hline
暗能量 & $\sim 10^9\Si{yrs}$  &  \\
\hline
太阳系形成 & $ \sim 8\e9\Si{yrs}$  &  \\
\hline
\end{tabular}
\end{table}

\begin{figure}[ht]
\centering
\includegraphics[width=14.25cm]{./figures/20cf79da627ed12c.png}
\caption{宇宙尺度及成分表} \label{fig_UniEvo_1}
\end{figure}

从 $10^{-10}$ 秒到宇宙诞生后 3 分钟,宇宙暴胀结束,宇宙从 $1\Si{TeV}$ 迅速逐渐冷却到 $0.1 \Si{MeV}$。 在此期间暴胀所释放的能量把宇宙重新加热,宇宙中的电弱相互作用开始分离,夸克结合成强子,中子冷却下来,宇宙中的核子开始在相互作用下结合并形成,称为\textbf{原核初合成}(BBN);中微子在创世后1秒与其他核子退耦并此后不再与其他物质产生相互作用,一直传播到现在。此时,宇宙开始以辐射为主导,我们也称为辐射主导时期的开始,此外,宇宙中的原初扰动和引力波开始形成。

\begin{figure}[ht]
\centering
\includegraphics[width=14.25cm]{./figures/dcc08a07848917ef.png}
\caption{宇宙成分变化} \label{fig_UniEvo_2}
\end{figure}

从创世后3分钟到现在,我们可以称为后物质时期。我们也可以把这个时期大致地分为两个时期:从宇宙诞生后3分钟到380.000年,宇宙一直以辐射为主导(详见\autoref{fig_UniEvo_2} \footnote{引自 Daniel Baumann,Cosmology,Perface,Figure 1})。随着宇宙的膨胀光子和物质从密度对等逐渐分离,且光子、物质和原初引力波进行充分的相互作用称为\textbf{再融合}(recombination),光子逐渐冷却下来形成\textbf{宇宙辐射背景}(CMB),期间暗物质开始形成。

从宇宙诞生后 $380.000$ 年到现在,宇宙进入物质主导时期,另一部分光子形成了宇宙辐射背景并随着时间演化到现在,一部分光子和原初引力波相互作用后成为B-模极化光子; 一部分原初引力波和中微子并不参与到相互作用中,并随着时间演化直到现在。在此期间,各大星系形成于创世后 $10^8$ 年之后,太阳系形成于 $8\e9$ 年。
\subsection{具体过程的解释}
Big bang 理论假设宇宙物质初始是处于极高温的基本粒子海。高能光子与带电粒子散射频繁,以至于原子无法形成\footnote{此时光子的能量远大于氢原子的电离能。},物质密度$\Omega_m=0$,且宇宙整体都是不透明的。加之处于热平衡,所以黑体辐射定律可适用于该阶段的光子:

\begin{equation}\label{eq_UniEvo_1}
\epsilon_r\equiv\rho_rc^2=\alpha T^4~,
\end{equation}
其中$\alpha\equiv\frac{\pi^2k_\mathrm{B}^4}{15\hbar^3c^3}=7.565\times10^{-16} \mathrm{Jm}^{-3} \mathrm{K}^{-4}$。
考虑到此时处于辐射主导阶段,其密度$\rho_r\propto 1/a^4$,代入\autoref{eq_UniEvo_1} 得:
\begin{equation}
T\propto \frac{1}{a^4}~.
\end{equation}
所以\textbf{宇宙温度随其膨胀而下降}。

随着宇宙温度下降,光子能量减小,夸克相变为重子(质子和中子),彼时的粒子海充满了自由电子、质子、中子、光子与中微子。粒子与粒子频繁地发生作用;

温度继续下降,质子和中子结合成原子核,除了中微子,其余粒子依旧碰撞频繁。此时的宇宙虽然依旧是辐射主导的,但物质密度\footnote{在宇宙学里,常简称非相对论粒子为“物质”。由“宇宙学的基本方程”一节可知,辐射阶段是不稳定的,因为辐射密度稀释得比物质密度快,所以会过渡到物质主导阶段}越来越大,最后与辐射密度相等\textbf{(matter-radiation equality)}。设达到密度相等时的时间为$t_{eq}$,这既是辐射阶段的终点,也是物质阶段的起点。因而对于伴有尺度因子$a\propto t^{1/2}$的辐射阶段,我们有:
\begin{equation}
\frac T{T_\mathrm{eq}}=\left(\frac{t_\mathrm{eq}}t\right)^{1/2}~.
\end{equation}
紧接着电子与原子核结合\textbf{(recombination)},在这之后,光子不再与电子,原子核作用,宇宙变得透明,这就是所谓的\textbf{退耦(decoupling)}。光子退耦后,传播到地球,被精确测得其温度为:
\begin{equation}
T_{0}=2.725 \pm 0.001 \,K~.
\end{equation}
通过红移因子的测量,我们知道退耦时期的尺度因子$a\simeq 1/1000$,因此退耦温度约为$T_1=a_0T_0/{a_1}\simeq 3000\,K$。利用尺度因子在物质阶段中与时间的函数关系,我们可以得到退耦时间大约发生在:
\begin{equation}
t_{\mathrm{dec}}\simeq10^{13}\sec=350 000 \mathrm{yrs}~.
\end{equation}


