% 托马斯进动
% 狭义相对论|相对论|洛伦兹变换|洛伦兹群|进动|托马斯进动|复合

\pentry{洛伦兹群\upref{qed1}}

当两个洛伦兹平动复合的时候,其结果通常不是一个洛伦兹平动,而是一个平动加上一个转动。

从物理直觉上来说,考虑三个参考系 $K_1$,$K_2$ 和 $K_3$。每个参考系都架设三把标尺,彼此垂直,各标尺分别测量相应参考系里的 $x$,$y$ 和 $z$ 坐标。令 $K_2$ 以速度 $\bvec{v}$ 相对 $K_1$ 运动,同时保证在 $K_1$ 和 $K_2$ 看来,双方对应的标尺都相互平行;再令 $K_3$ 以速度 $\bvec{w}$ 相对 $K_2$ 运动,同时保证在 $K_2$ 和 $K_3$ 看来,双方对应的标尺都相互平行,那么这个时候 $K_3$ 的标尺还和 $K_1$ 的标尺对应平行吗?

答案是否定的。这是因为尺缩效应只发生在沿着参考系相对速度的方向,而垂直于相对速度的方向不会发生尺缩效应。粗略的解释如\autoref{TmsPrs_fig1} 所示,$K_3$ 的标尺相对于 $K_1$ 的标尺不再平行,而是转动了一个角度。

\begin{figure}[ht]
\centering
\includegraphics[width=8cm]{./figures/TmsPrs_1.pdf}
\caption{托马斯进动的示意图。左图表示 $K_3$ 眼中自己的一根标尺,它是物理存在;垂直的线段表示这个标尺,下面的长箭头表示 $K_3$ 相对 $K_1$ 的运动方向,而上面的短箭头表示垂直于该运动的方向。在 $K_1$ 看来,沿着运动方向发生了尺缩效应,下面的箭头变短,而垂直运动方向的箭头没有变短,结果就是标尺的长度缩短,并且其方向转动了一个角度。} \label{TmsPrs_fig1}
\end{figure}

这种纯粹由于相对运动而产生的标尺旋转,被称为\textbf{托马斯进动(Thomas Precession)},由Llewellyn Thomas于1914年提出。电子的磁矩变化、傅科摆的运动修正,都需要考虑到托马斯进动的影响。

\subsection{推导}

出于方便和实用性的考量,我们研究的是一个系统加速运动中的指向变化\footnote{推导思路完全参考自 Goldstein 的 \textsl{Classical Mechanics}\cite{Goldstein}。}。这里的系统可以是一个电子,也可以是一套傅科摆,指向则可以是任意指定的。以下为了方便表述,我们用一个电子的运动来说明。

设电子在实验室中以速度 $\bvec{v}=(v, 0, 0)\Tr$ 运动,同时具有一个加速度 $\bvec{a}=(a_x, a_y, 0)$。记实验室参考系为 $K_1$,以速度 $\bvec{v}$ 运动的参考系为 $K_2$,而以速度 $\bvec{v}+\bvec{a}\dd t$ 运动的参考系为 $K_3$。这里 $\dd t$ 是一段极短的时间,而 $\bvec{v}+\bvec{a}\dd t$ 是电子在 $\dd t$ 时刻后的速度。记 $a=\abs{\bvec{a}}=\sqrt{a_x^2+a_y^2}$。

以上设定中尽管限制了 $\bvec{v}$ 在 $K_1$ 参考系的 $x$ 轴上、$\bvec{a}$ 在 $x-y$ 平面上,但是并不失一般性,而且方便计算。这样,$K_1$ 到 $K_2$ 的洛伦兹过渡矩阵就是

\begin{equation}
\bvec{L}_{1-2}=\pmat{\frac{1}{\sqrt{1-v^2}}&\frac{-v}{\sqrt{1-v^2}}&0&0\\\frac{-v}{\sqrt{1-v^2}}&\frac{1}{\sqrt{1-v^2}}&0&0\\0&0&1&0\\0&0&0&1}
\end{equation}

而 $K_2$ 到 $K_3$ 的过渡矩阵是

\begin{equation}
\bvec{L}_{2-3}=\pmat{
\frac{1}{\sqrt{1-(a\dd t)^2}}&-\frac{a_x\dd t}{\sqrt{1-(a\dd t)^2}}&-\frac{a_y\dd t}{\sqrt{1-(a\dd t)^2}}&0\\
-\frac{a_x\dd t}{\sqrt{1-(a\dd t)^2}}&1+(\frac{1}{\sqrt{1-(a\dd t)^2}}-1)\frac{(a_x\dd t)^2}{(a\dd t)^2}&(\frac{1}{\sqrt{1-(a\dd t)^2}}-1)\frac{a_x\dd ta_y\dd t}{(a\dd t)^2}&0\\
-\frac{a_y\dd t}{\sqrt{1-(a\dd t)^2}}&(\frac{1}{\sqrt{1-(a\dd t)^2}}-1)\frac{a_x\dd ta_y\dd t}{(a\dd t)^2}&1+(\frac{1}{\sqrt{1-(a\dd t)^2}}-1)\frac{(a_y\dd t)^2}{(a\dd t)^2}\\
0&0&0&1
    }
\end{equation}

那么 $K_1$ 到 $K_3$ 的过渡矩阵就是这两个矩阵的复合(注意它们相乘的方向,并且忽略掉 $\dd t$ 大于一阶的项):

\begin{equation}
\begin{aligned}
\bvec{L}_{1-3}&=\bvec{L}_{2-3}\bvec{L}_{1-2}\\&=\pmat{
\frac{1}{\sqrt{1-v^2}}&-\frac{v}{\sqrt{1-v^2}}&a_y\dd t&0\\
-\frac{v}{\sqrt{1-v^2}}&\frac{1}{\sqrt{1-v^2}}&0&0\\
-a_y\dd t&va_y\dd t&1&0\\
0&0&0&1
    }
\end{aligned}
\end{equation}

这个变换就不是一个平动,而是一个平动和一个转动的复合。

如果记 $\bvec{L}_{3-1}$ 是从 $K_3$ 直接\textbf{平动}回到 $K_1$ 的洛伦兹过渡矩阵\footnote{就是说,直接从 $K_3$ 变换回 $K_1$,保持双方的标尺平行。该矩阵的具体形式此处省略了。},那么我们有

\begin{equation}
\bvec{L}_{1-3}\bvec{L}_{3-1}=\pmat{
1&0&0&0\\
0&1&(\frac{1}{\sqrt{1-v^2}}-1)\frac{a_y\dd t}{v}&0\\
0&(1-\frac{1}{\sqrt{1-v^2}})\frac{a_y\dd t}{v}&1&0\\
0&0&0&1
    }=\bvec{R}
\end{equation}

这是一个 $x-y$ 平面内的转动。也就是说,当粒子有了一个 $y$ 方向的加速度的时候,其参考系指向会变化。

$\bvec{R}$ 的含义是,在 $\dd t$ 内,这个指向在 $x-y$ 平面上\textbf{顺时针}转动了 $\arcsin{(\frac{1}{\sqrt{1-v^2}}-1)\frac{a_y\dd t}{v}}$ 的角度,由于该角度很小,因此由 $\lim_{\theta\rightarrow 0}\frac{\sin\theta}{\theta}=1$ 可知,这个转动角度也可以写为 $\dd\Omega=(\frac{1}{\sqrt{1-v^2}}-1)\frac{a_y\dd t}{v}$。

整理以上结果,我们可以得到,当电子以速度 $\bvec{v}$ 运动且有加速度 $\bvec{a}$ 时,其坐标系指向按角速度 $\bvec{\omega}$ 偏转,其中 $\bvec{\omega}=\frac{\dd\Omega}{\dd t}=(\frac{1}{\sqrt{1-v^2}}-1)\frac{\bvec{a}\times\bvec{v}}{v^2}$。

当 $v$ 也很小,以至于 $\frac{1}{\sqrt{1-v^2}}\approx 1+\frac{1}{2}v^2$ 时,这个角速度还可以近似为 $\frac{\bvec{a}\times\bvec{v}}{2}$。








