% 旋转算符
% 旋转算符|量子力学|角动量

\pentry{平移算符\upref{tranOp}, 梯度\upref{Grad}}

如果要将一个三维函数 $f(\bvec r)$ 绕 $z$ 轴旋转角 $\alpha$, 我们可以使用极坐标 $f(r, \theta, \phi)$, 并用平移算符 $\exp(-\alpha\pdv*{\phi})$ 对坐标 $\phi$ 进行 “平移”。 那么球坐标中的算符 $\pdv*{\phi}$ 在直角坐标中如何表示呢? 令 $s = \sqrt{x^2 + y^2}$, 该算符的意义是求函数在 $\uvec \phi$ 方向的方向导数乘以 $s$。 $\uvec \phi = (-y/s, x/s, 0)$, 所以
\begin{equation}
\pdv{f}{\phi} = s \grad f \vdot \uvec \phi = x\pdv{f}{y} - y\pdv{f}{x} = \qty(x\pdv{y} - y\pdv{x}) f~,
\end{equation}
于是我们就得到直角坐标系中绕 $z$ 轴逆时针旋转的算符为
\begin{equation}
R_z(\alpha) = \exp\qty[-\alpha \qty(x\pdv{y} - y\pdv{x})]~.
\end{equation}
同理, 我们可以将极坐标的极轴指向 $x$ 或 $y$ 轴正方向, 从而得出绕 $x$ 或 $y$ 轴逆时针旋转角 $\alpha$ 的算符分别为
\begin{equation}
\begin{aligned}
R_x(\alpha) &= \exp\qty[-\alpha \qty(y\pdv{z} - z\pdv{y})]~,\\
R_y(\alpha) &= \exp\qty[-\alpha \qty(z\pdv{x} - x\pdv{z})]~.
\end{aligned}
\end{equation}

\begin{example}{}
要将 $f(x, y, z) = x$ 绕 $z$ 轴旋转 $\alpha$ 角, 就计算
\begin{equation}\ali{
&\qquad\exp\qty[-\alpha \qty(x\pdv{y} - y\pdv{x})] x\\
&= x -\alpha \qty(x\pdv{y} - y\pdv{x})x + \frac{1}{2!} \alpha^2 \qty(x\pdv{y} - y\pdv{x})^2 x - \dots
}~\end{equation}
其中
\begin{equation}\ali{
&\qty(x\pdv{y} - y\pdv{x}) x = -y~,\\
&\qty(x\pdv{y} - y\pdv{x})^2 x = \qty(x\pdv{y} - y\pdv{x})(-y) = -x~,\\
&\qty(x\pdv{y} - y\pdv{x})^3 x = \qty(x\pdv{y} - y\pdv{x})(-x) = y~,\\
&\qty(x\pdv{y} - y\pdv{x})^4 x = \qty(x\pdv{y} - y\pdv{x})y = x\\
&\dots
}~\end{equation}
所以
\begin{equation}\ali{
&\qquad\exp\qty[-\alpha \qty(x\pdv{y} - y\pdv{x})] x\\
&= \qty(1 - \frac{1}{2!}\alpha^2 + \frac{1}{4!}\alpha^4 \dots) x + \qty(\alpha - \frac{1}{3!}\alpha^3 + \frac{1}{5!}\alpha^5 \dots) y\\
&= x \cos\alpha + y \sin\alpha~.
}\end{equation}
显然这就是旋转后所得的函数。
\end{example}

在量子力学中, 由直角坐标系中角动量算符 $L_x, L_y, L_z$ 的定义, 三个旋转算符分别可以记为
\begin{equation}
\exp(-\I \frac{\alpha L_x}{\hbar})~,
\qquad
\exp(-\I \frac{\alpha L_y}{\hbar})~,
\qquad
\exp(-\I \frac{\alpha L_z}{\hbar})~.
\end{equation}
