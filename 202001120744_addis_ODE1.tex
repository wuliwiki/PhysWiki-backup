%一阶线性微分方程
% 微积分|微分方程|常微分方程|一阶线性微分方程|齐次|常数易变法

\pentry{常微分方程\upref{ODE}}

具有以下形式的微分方程叫做\textbf{一阶线性微分方程}
\begin{equation}\label{ODE1_eq1}
\dv{y}{x} + p(x)y = f(x)
\end{equation}
一般地, 未知函数及其各阶导数都各占一项时, 方程就是\textbf{线性}的. 另外,如果 $f(x)$ 项不出现, 方程就是\textbf{齐次}的, 否则就是\textbf{非齐次}的. 我们先来看以上方程对应的齐次方程
\begin{equation}\label{ODE1_eq2}
\dv{y}{x} + p(x)y = 0
\end{equation}
这是一个可分离变量的方程, 分离变量得
\begin{equation}
\frac{\dd{y}}{y} = -p(x) \dd{x}
\end{equation}
两边积分得
\begin{equation}
\ln\abs{y} = -\int p(x) dx + C
\end{equation}
两边取自然指数得
\begin{equation}
y = \pm \E^C \E^{-\int p(x) \dd{x}}
\end{equation}
把 $\pm \E^C $ 整体看做一个任意常数 $C$, 上式变为.
\begin{equation}\label{ODE1_eq6}
y = C \E^{-\int p(x) \dd{x}}
\end{equation}
这就是一阶线性齐次微分方程\autoref{ODE1_eq2} 的通解, 也叫\autoref{ODE1_eq1} 的\textbf{齐次解}.

\subsection{常数变易法}

现在我们用\textbf{常数变易法}来解非齐次方程\autoref{ODE1_eq1}. 为书写方便, \autoref{ODE1_eq6} 中令 $y_0(x) = \exp(-\int p(x) \dd{x})$. 假设上式中的 $C$ 是一个函数 $C(x)$ 而不是常数, 代入\autoref{ODE1_eq1} 得
\begin{equation}
C'y_0 + C[y_0' + p(x)y_0] = f(x)
\end{equation}
由于 $y_0$ 是齐次解, 上式方括号中求和为 0, 分离变量得
\begin{equation}
\dd{C}= \frac{f(x)}{y_0} \dd{x}
\end{equation}
两边积分得
\begin{equation}
C(x) = \int \frac{f(x)}{y_0} \dd{x}
\end{equation}
所以一阶线性非齐次微分方程的通解为
\begin{equation}\label{ODE1_eq10}
y = y_0  \int \frac{f(x)}{y_0} \dd{x}
\end{equation}
其中
\begin{equation}\label{ODE1_eq11}
y_0(x) = \E^{-\int p(x) \dd{x}}
\end{equation}
注意待定常数包含在\autoref{ODE1_eq10} 的不定积分中, \autoref{ODE1_eq11} 中的不定积分产生的待定常数在代入\autoref{ODE1_eq10} 后可消去.