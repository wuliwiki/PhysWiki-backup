% 泊松括号
% 哈密顿量|泊松括号|守恒量

\begin{issues}
\issueDraft
\end{issues}

\pentry{哈密顿正则方程\upref{HamCan}}
在理论力学里,泊松括号的引入能更加简洁地表明运动积分需要满足的条件。所谓运动积分,可以简单理解为在某一动力学系统中,不随时间改变的常数。动力学系统总满足二阶微分方程,所以我们总可以找到这样的运动积分。现在设$f(q,p,t)$为粒子关于动量、坐标和时间的函数,且为运动积分,则根据定义我们有:
\begin{equation}
\frac{df}{dt}=\frac{\partial f}{\partial t}+\frac{\partial f}{\partial q}\dot{q}+\frac{\partial f}{\partial p}\dot{ p}
\end{equation}

结合正则方程,我们可以把上式改写为
\begin{equation}
\frac{df}{dt}=\frac{\partial f}{\partial t}+\pb{H}{f}
\end{equation}
引入的泊松记号定义如下:


对于任意两个函数 $u(q, p, t)$ 和 $v(q, p, t)$, 泊松括号的“作用”为
\begin{equation}
\pb{u}{v} = \sum_i \pdv{u}{q_i}\pdv{v}{p_i} - \pdv{v}{q_i}\pdv{u}{p_i}~,
\end{equation}
其中$i$为系统的自由度。显然,如果函数$f$不显含时间,且为运动积分(也就是一守恒量),则该函数与系统哈密顿量的泊松括号为0.

\subsection{泊松括号的性质}
根据定义,我们可以证明泊松括号需要满足如下几条性质。
\begin{enumerate}
\item 反对称性
$[u,v]=-[v,u]$
\item 双线性
$\pb {au+bv}{\phi}=a\pb {u}{\phi}+\pb b{v}{\phi}$
\item 

\end{enumerate}
\subsection{泊松定理}
\subsection{量子力学中的泊松括号}


