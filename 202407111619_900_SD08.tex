% 苏州大学 2008 年硕士考试试题
% keys 苏州大学|考研|物理|2008年
% license Copy
% type Tutor

\textbf{声明}:“该内容来源于网络公开资料,不保证真实性,如有侵权请联系管理员”
\begin{enumerate}
\item 有一同轴电缆,其长$L=1.50x10^3m$,内导体外径$R_1=1.0mm$,外导体内径$R_2=5.0mm$中间填充绝缘介质。由于受潮,测得绝缘介质的电阻率降低到$6.4*10^5\Omega .m$。若信号源是电动势$\varepsilon=24V$,内阻$R_1=3.0\Omega$的直流电源,求:\\
(1)同轴电缆的径向电电阻;\\
(2)在电缆末端的负载电阻$R_0=1.0k \Omega$上的信号电压。
\item 圆形线圈a由50匝细线绕成,横截面积为$4.0cm^2$,放在另一个半径为 20cm,匝数为100匝的另一圆形线圈b的中心,两线圈同轴共面。求:\\
(1)两线圈的互感系数;\\
(2)当线圈b中的电流以 50A/s的变化率减少时,线圈a内磁通量的变化率。\\
(3)线圈a中的感生电动势的大小。
\item 两无限长带异号电荷的同轴圆柱面,单位长度上的电量为$3.0*10^{-8}C/m$,内圆柱面半径为$2*10^{-2}m$,外圆面半径为$4*10^{-2}m$\\
(1)用高斯定理求内圆柱面内、两圆柱面间和外圆柱面外的电场强度;\\
(2)若一电子在两圆柱面之间垂直于轴线的平面内沿半径$3x10^{-2}m$的圆周匀速旋转,问此电子的动能为多少?
\begin{figure}[ht]
\centering
\includegraphics[width=6cm]{./figures/b08a254b1e0e0b20.png}
\caption{} \label{fig_SD08_1}
\end{figure}
\item 某边长为20cm 的方形导体回路放在一圆形均匀磁场的正中,初始时磁场B为 0.5T,垂直圆面向内,现磁场B以0.1T/s减少,求:\\
(1)a、b、c三点感应电场$E_r$的大小,并标明方向;
(2)导体回路中的电流,已知该回路的电阻为2$\Omega$;\\
(3)点a和b之间的电势差。
\begin{figure}[ht]
\centering
\includegraphics[width=6cm]{./figures/1b1f67fbb7055804.png}
\caption{} \label{fig_SD08_2}
\end{figure}
\item 某油轮失事的海域,大量泄的石油(n=1.2)形成了一个很大的油膜。试求:\\
(1)如果从飞机上竖直向下看油膜,只监测到波长为 552nm 的蓝绿光,试求油膜的可能厚度?\\
(2)如果潜水员从水下竖直地向上看这油膜的厚度为460nm 区域,可看见哪些波长的可见光透射最强?(水的折射率为1.33)
\item 迈克耳逊干涉仪中一臂(反射镜),以速度v速推移,用透镜接收干涉条纹,将它会聚到光电元件上,把光强变化为电讯号。\\
(1)若测得电讯号强度变化的时间频率为v,求入射光的波长$\lambda$;\\
(2)若入射光波长为40$\mu$m,要使电讯号频率控制在100Hz,反射镜平移的速度应为多少?
\item (1)在单缝夫琅和费衍射实验中,用400nm 和 700nm 的两种单色光同时垂直入射已知单缝宽度 $a=1.0*10^{-2}cm$,透镜焦距f=50cm,求两种光第一级衍射明纹中心之间的距离。\\
(2)若用光栅常数$d=1.0*10^{-3}cm$的光栅替换单缝,其它条件和上一问相同,求相应两个第一级主极大之间的距离。
\item 已知红宝石的折射率为1.76,欲使线偏振光的激光通过红宝石棒时,在棒的端面上没有反射损失,光在棒内沿棒轴方向传播,试问:\\
(1)光束入射角i应为多少?\\
(2)棒端面对棒轴倾角应取何值?\\
(3)入射光的振动方向应如何?
\item 当用白光照射一平面透射光栅时,能在30°角衍射方向上观察到600nm 的第二级谱线,但在此方向上测不到 400nm 的第三级谱线,求:\\
(1)光栅常数d,光栅的缝宽a和缝间距b。\\
(2)对 400nm 的单色光能看到哪几级谱线。
\item 带电粒子在威尔逊云室(一种径迹探测器)中的轨迹是一串小雾滴,雾滴的线度约为1$\mu$m。观察能量为1000电子伏特的电子径迹时(属于非相对论情形),电子动量与经典力学动量的相对偏差$\Delta p/p$不小于多少?
\item 求氢原子中第一激发态(n=2)电子的德布罗意波长(非相对论情形)。
\item 若一个电子的动能等于它的静能,试求:\\
(1)该电子的速度为多大?\\
(2)其相应的德布罗意波长是多少?(考虑相对论效应)
\end{enumerate}


有关常数:$m_0=9.1*10^{-31}kg$

$h=6.63*10^{-34}J.s$

$\varepsilon_0=8.85*10^{-12}F/m$

$\mu_0=1.26*10^{-6}H/m$

$R_H=1.097*10^7/m$