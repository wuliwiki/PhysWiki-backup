% 世界线和固有时
% keys 世界线|固有时|狭义相对论

\pentry{时空的四维表示\upref{SR4Rep}}

约定使用东海岸度规 $\eta_{\mu\nu}=\rm{diag}(-1,1,1,1)$ 和自然单位制 $c=1$。

我们知道,一个事件被看作是四维时空中的一个点,它有给定的时间和空间坐标。如果我们考虑一个粒子的运动,其在时空中的轨迹就是一条曲线。这条曲线就被称为粒子的\textbf{世界线(world line)}。
\footnote{主要参考教材:\cite{陈斌广相}。}

\subsection{类时曲线和类空曲线}
我们用 $x^\mu(\lambda)$ 来刻画曲线。$\lambda$ 是刻画曲线的一个参数。那么在某点 $\lambda=\lambda_0$ 处的切矢量为
\begin{equation}
\begin{aligned}
V^\mu(\lambda=\lambda_0)=\left.\dv{x^\mu}{\lambda}\right|_{\lambda=\lambda_0}~.
\end{aligned}
\end{equation}
它衡量了该粒子在 $\lambda_0$ 处的四速度\textbf{方向}(注意,由于曲线的参数化是任意的,所以我们只能确定 $V^\mu$ 正比于四速度)。

如果曲线上的任意一点都有 $\eta_{\mu\nu}V^\mu V^\nu<0$,那么曲线是\textbf{类时}的。
如果曲线上的任意一点都有 $\eta_{\mu\nu}V^\mu V^\nu>0$,那么曲线是\textbf{类空}的。如果曲线上的任意一点都有 $\eta_{\mu\nu}V^\mu V^\nu=0$,即粒子总是以光速运动,这样的世界线被称为\textbf{类光}的。

两个没有因果关系\footnote{更具体的介绍见\textbf{因果结构}\upref{Causal}。}的事件只能通过类空曲线相联系。

\subsection{固有时}
对于一条\textbf{类时}世界线上参数间隔($\Delta \lambda$)很小的两个点,如果将它们之间的曲线近似地看作是直线,那么它们之间的时空距离是
\begin{equation}
\Delta s=\Delta \lambda \sqrt{-\eta_{\mu\nu}\dv{x^\mu}{\lambda} \dv{x^\nu}{\lambda}}
\end{equation}
利用牛顿—莱布尼兹公式\footnote{牛顿—莱布尼兹公式(简明微积分)\upref{NLeib}。},可以将上式推广到任意类时曲线的“长度”公式:
\begin{equation}\label{eq_wdline_2}
\Delta s=\int_{\lambda_A}^{\lambda_B} \dd\lambda \sqrt{-\eta_{\mu\nu}\dv{x^\mu}{\lambda} \dv{x^\nu}{\lambda}}
\end{equation}
上式的结果跟 $\lambda$ 的参数化方式无关,也就是说在变换 $\lambda\rightarrow f(\lambda)$ 下,时空间隔是不变的(这个过程类似于换元积分法\upref{IntCV})。这个不变性称为\textbf{世界线的重参数化不变性}。

定义\textbf{固有时}:
\begin{equation}
\tau_{AB}=\int_{A}^{B} \dd\lambda \sqrt{-\eta_{\mu\nu}\dv{x^\mu}{\lambda} \dv{x^\nu}{\lambda}}=\int_A^B \dd \tau = \int_A^B \sqrt{\dd t^2-\dd r^2}
\end{equation}
由此可以看到 $\tau_{AB}\le t_B-t_A$。因此,对于运动的粒子,坐标时总是大于固有时的,这也是钟慢效应\footnote{时间的变换与\textbf{钟慢效应}\upref{SRtime}。}的来源。一个例子是,在实验上观测到的粒子寿命总是比它的固有寿命要长。

为了确定唯一一种参数化方式,我们可以用固有时对曲线参数化:$x^\mu(\tau)$,使得 $x^\mu(\tau_A)$ 到 $x^\mu(\tau_B)$ 的固有时为 $\tau_B-\tau_A$。这要求
\begin{equation}\label{eq_wdline_1}
\sqrt{-\eta_{\mu\nu}\dv{x^\mu}{\tau} \dv{x^\nu}{\tau}}=1
\end{equation}
将\autoref{eq_wdline_1} 代入\autoref{eq_wdline_2} 就可以得到 $\Delta s=\int \dd \tau$,这是符合我们的期待的。我们将
\begin{equation}
u^\mu=\dv{x^\mu}{\tau},\quad \hat u=u^\mu \hat{e}_\mu
\end{equation}
定义为粒子的\textbf{四速度}。它满足归一化条件:
\begin{equation}
\hat u \cdot \hat u = u^\mu u_\mu = -1
\end{equation}
例如,静止粒子的四速度为 $u^\mu=(1,0,0,0)$。而在另一个参考系看来如果粒子的速度为 $\bvec v$,那么其四速度为 $u^\mu=\gamma_v(1,\bvec v)$,其中 $\gamma_v=1/\sqrt{1-|\bvec v|^2}$。
