% SQL 基础笔记

\begin{issues}
\issueDraft
\end{issues}

\begin{itemize}
\item 本文参考 \href{https://www.w3schools.com/sql/default.asp}{w3schools}
\item 以下用 SQLite 做测试。
\item \textbf{SQL(Structured Query Language)} 是操作数据库的语言
\item MySQL 是一个具体的管理数据库的软件, 是 \textbf{RDBMS (relational database management system)}
\item 开发者需要掌握 SQL 语言来使用 MySQL。 不同的 RDBMS 基本都是用 SQL 语言, 虽然有微小差异, 但都兼容 SQL standard。
\item dump 操作常常用于备份数据库(而不是直接拷贝数据库文件)。 数据库文件一般是加密的, 但是 dump 不会加密。
\item 一个 database 里面可以有许多 table, 每个 \textbf{field} 就是 table 的一列, 每个 \textbf{record} 就是一行
\item 以下的命令可以在 \textbf{DB BROWSER for SQLITE} 里面实验(可以从 Ubuntu 商店下载), 先创建数据库, 然后选 Execute SQL, 把命令或者 sql 里面的东西粘贴过去, 执行即可。
\end{itemize}

\subsection{SQL Basics}
\begin{itemize}
\item 显示数据库 \verb`SHOW DATABASES;`
\item 用 \verb|--| 注释直到行末, 用 \verb|/*   */| 注释块
\item 大部分操作都通过 \textbf{SQL statement} 完成, 如 \verb`SELECT * FROM 表名;`
\item 命令可以随意换行,两个命令之间必须用 \verb|;| 隔开
\item SQL 关键字, 表格名, 列名都不区分大小写,一般关键词用大写
\item 数据中的字符串用单引号,大部分数据库双引号也行(但不是标准),字符串区分大小写
\item 一些最重要的关键字如
\item \verb`SELECT` - 提取数据
\item \verb`UPDATE` - 更新数据
\item \verb`DELETE` - 删除数据
\item \verb`INSERT INTO` - 添加数据
\item \verb`CREATE DATABASE` - 创建新数据库
\item \verb`ALTER DATABASE` - 更改数据库
\item \verb`CREATE TABLE` - 创建新表格
\item \verb`ALTER TABLE` - 修改表格
\item \verb`DROP TABLE` - 删除表格
\item \verb`CREATE INDEX` - 新增 index (search key)
\item \verb`DROP INDEX` - 删除 index
\end{itemize}

\subsubsection{SELECT}
\begin{itemize}
\item \verb`SELECT 列1, 列2, ... FROM 表格名;` 获取的数据表叫做 \textbf{result-set}。 如果要选取所有列, 用 \verb`*` 即可 e.g. \verb`SELECT * FROM 表格名;` 显示名为 \verb|表格名| 的表格中全部内容
\item \verb`SELECT DISTINCT 列1, 列2, ... FROM 表格名;` 仅列出(输出中)不完全相同的行, 例如 \verb`SELECT DISTINCT Country FROM 表格名;`
\item 获取表格行数: \verb`SELECT COUNT(*) FROM 表格名`
\item \verb`SELECT COUNT(DISTINCT Country) FROM 表格名;` 显示表格中有多少不同的国家。 (这个命令在 Microsoft Access 里面没用), 除了 \verb|COUNT| 还有 \verb|AVG| 和 \verb|SUM|
\end{itemize}

\subsubsection{WHERE}
\begin{itemize}
\item 如果没有任何满足条件的行, 那么返回一个空表格, 不报错。
\item \verb`SELECT 列1, 列2, ... FROM 表格名 WHERE condition;`
\item \verb`SELECT * FROM 表格名 WHERE Country='Mexico';`
\item \verb`SELECT * FROM 表格名 WHERE CustomerID=1;`
\item \verb`WHERE` 后面可以是 \verb`=`, \verb`>`, \verb`<`, \verb`>=`, \verb`<=`, \verb`<>`(不等于,有时候 \verb`!=` 也行), \verb`BETWEEN` \verb`LIKE` \verb`IN` (在几个可能的值之中)
\item \verb`SELECT * FROM Products WHERE Price BETWEEN 50 AND 60;`
\item \verb`WHERE` 后面的条件可以用 \verb`AND`, \verb`OR`, \verb`NOT`
\item \verb`SELECT 列1, 列2, ... FROM 表格名 WHERE NOT condition;`
\item \verb`SELECT * FROM 表格名 WHERE City='Berlin' OR City='München';`
\item \verb`SELECT OCTET_LENGTH(列名) FROM 表名 WHERE ...;` 可以返回 blob 的大小
\end{itemize}

\subsubsection{ORDER BY}
\begin{itemize}
\item 用 \verb`ORDER BY` 来进行排序 \verb`SELECT 列1, 列2, ... FROM 表格名 ORDER BY 列1, 列2, ... ASC|DESC;`
\item \verb`SELECT * FROM 表格名 ORDER BY Country DESC;`
\end{itemize}

\subsubsection{INSERT INTO}
\begin{itemize}
\item \verb`INSERT INTO 表格名 (列1, 列2, 列3, ...) VALUES (值1, 值2, 值3, ...);` 插入行, 如果每列都插入, 那么 \verb`(列1, 列2,...)` 可以省略
\item 第一列总是数字(无论 field 叫做什么)? 插入时可以指定,  如果不指定会自动更新。 不能指定已经存在的数字。
\item 如果某列是 optional 的, 那么插入一行时可以不输入这列的信息, 这列的值就是 \verb`NULL`(见下文)
\item \verb`SELECT 列 FROM 表格名 WHERE 列 IS NULL;`
\item \verb`SELECT 列 FROM 表格名 WHERE 列 IS NOT NULL;`
\end{itemize}

\subsubsection{UPDATE}
\begin{itemize}
\item 改变已经存在的 record. \verb|WHERE| 可以选择一条或多条 record
\item 如果不用 \verb`WHERE`, 所有行的指定列都会被更新
\item \verb`UPDATE 表格名 SET 列1 = 值1, ..., 列n = 值n WHERE condition;`
\item 例如 \verb`UPDATE 表格名 SET ContactName = 'Alfred Schmidt', City= 'Frankfurt' WHERE CustomerID = 1;`
\item \verb|UPDATE 表格名 SET 列名 = 列名 + 1 WHERE ...| 数值加 1。
\end{itemize}

\subsubsection{DELETE}
\begin{itemize}
\item \verb`DELETE FROM 表格名 WHERE condition;` 删除符合条件的行
\item e.g. \verb`DELETE FROM 表格名 WHERE CustomerName='Alfreds Futterkiste';`
\end{itemize}

\subsubsection{ALTER}
\begin{itemize}
\item 给表格重命名 \verb|ALTER TABLE 旧名字 RENAME TO 新名字;|
\item 从表格中删除一列 \verb|ALTER TABLE 表名 DROP COLUMN 列名;|
\end{itemize}

\subsubsection{CREATE}
\begin{itemize}
\item \verb|CREATE TABLE Person (列1名称 类型 选项1 选项2 ... , 列2名称 类型 选项1 选项2 ... , ...)|
\item 数据类型详见\href{https://www.w3schools.com/sql/sql_datatypes.asp}{这里}, 注意 \verb|INT| 就是 \verb|INTEGER|。
\item sqlite 支持的类型为 \verb|NULL, INT, REAL, TEXT, BLOB|
\item 选项 \verb|PRIMARY KEY| 可以使表格的某一列不出现重复的值且不能为 \verb|NULL|。 一个表格中只能有一个 \verb|PRIMARY KEY|, 但可以是多行。
\item \verb|列名称 类型 UNIQUE| 可以保证该列没有重复数据。
\item \verb|AUTO_INCREMENT| 选项可以在新增一行时(如果不提供)自动比该列当前的最大值加 1, 例如 ID 号。 在 sqlite 中叫做 \verb|AUTOINCREMENT| 且只能出现在 \verb|INT PRIMARY KEY AUTOINCREMENT| 中。
\item 如果要在一个 table 中的某列指定另一个 table 中的某个 \verb|PRIMARY KEY|, 那就使用 \verb|FOREIGN KEY (本表列名) REFERENCES 别的表名 (别的表的列名)|。 例如在 \verb|Orders| 表中链接 \verb|Persons| 表中名为 \verb|PersonID| 的 primary key:
\begin{lstlisting}[language=none]
CREATE TABLE Orders (
    OrderID INT NOT NULL,
    OrderNumber INT NOT NULL,
    PersonID INT,
    PRIMARY KEY (OrderID),
    FOREIGN KEY (PersonID) REFERENCES Persons (PersonID)
);
\end{lstlisting}
\item 一般如果没有指定 \verb|NOT NULL| 或者 \verb|NULL|, 那就默认 \verb|NULL|, 也就是该列是 optional 的, 允许 NULL 值。
\end{itemize}

\subsubsection{DROP}
\begin{itemize}
\item \verb|DROP TABLE 表名| 删除表。
\end{itemize}

