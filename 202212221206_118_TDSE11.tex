% 薛定谔方程(单粒子一维)

\begin{issues}
\issueDraft
\end{issues}

\pentry{定态薛定谔方程(单粒子一维)\upref{SchEq}, 分离变量法解偏微分方程\upref{SepVar}}

单粒子的一维波函数是位置和时间的函数 $\Psi(x, t)$, 薛定谔方程为
\begin{equation}
H\Psi = \I\hbar \pdv{t}\Psi
\end{equation}
其中
\begin{equation}
H = -\frac{\hbar^2}{2m}\pdv[2]{x} + V(x, t)
\end{equation}
也可以直接记为
\begin{equation}\label{TDSE11_eq3}
-\frac{\hbar^2}{2m}\pdv[2]{x}\Psi + V(x, t)\Psi = \I\hbar \pdv{t}\Psi
\end{equation}
$V(x,t)$ 是势能, 满足
\begin{equation}
F_x(t) = -\pdv{V}{x}
\end{equation}
其中 $F_x$ 是经典力学中质点的受力。

\subsection{能量守恒系统的时间演化}
当哈密顿算符 $H$ 不随时间变化时, 我们说这个系统能量守恒\upref{QMcons}。 这时我们可以用分离变量法,令
\begin{equation}
\Psi(x, t) = \psi(x) T(t)
\end{equation}
 
\begin{equation}\label{TDSE11_eq1}
H\psi = E\psi
\end{equation}
以及
\begin{equation}\label{TDSE11_eq2}
\I \hbar\pdv{t}T = ET
\end{equation}
其中\autoref{TDSE11_eq1} 就是定态薛定谔方程\upref{SchEq}, 即哈密顿算符的本征方程。 \autoref{TDSE11_eq2} 有简单的解
\begin{equation}
T(t) = \E^{-\I E t/\hbar}
\end{equation}

根据 $H$ 的不同, 本征值 $E$ 可以取离散或连续的值。 先看只取离散值的简单情况(如无限深势阱\upref{ISW}), 令能级为 $E_n$ ($n = 1, 2, \dots$), 那么含时薛定谔方程的通解为
\begin{equation}\label{TDSE11_eq5}
\Psi(x, t) = \sum_n C_n \psi_n(x) \E^{-\I E_n t/\hbar}
\end{equation}
其中 $C_n$ 为待定系数, 由初始条件决定。 本征波函数满足正交关系
\begin{equation}
\int_{-\infty}^{+\infty}\psi_m(x)\psi_n(x)\dd{x} = \delta_{m,n}
\end{equation}
原因见 “施图姆—刘维尔理论\upref{SLthrm}”, 也可以类比有限维的 “厄米矩阵的本征问题\upref{HerEig}”。 所以系数可以通过投影计算
\begin{equation}
C_n = \int_{-\infty}^{+\infty} \psi_n(x)\Psi(x, 0) \dd{x}
\end{equation}
这个投影过程可以类比傅里叶级数(指数)\upref{FSExp}。 事实上, 傅里叶级数(\autoref{FSTri_sub3}~\upref{FSTri})就是无限深方势阱\upref{ISW}的波函数。

两个具体的例子见 “无限深势阱中的高斯波包\upref{wvISW}” 和 “简谐振子中的高斯波包(Matlab)\upref{SHOgs}”。

如果 $E$ 只在某个区间内取连续值, 我们同样可以使用分离变量法, 只是求和变为积分, 系数变为能量的函数。 我们知道此时每个能量都有二重简并, 所以定态薛定谔方程中一个能量 $E$ 有两个线性无关的解 $\psi_{E,1}(x)$ 和 $\psi_{E,2}(x)$, 那么
\begin{equation}\label{TDSE11_eq4}
\Psi(x, t) = \int [C_1(E) \psi_{E,1}(x) + C_2(E) \psi_{E,2}(x)] \E^{-\I E t/\hbar} \dd{E}
\end{equation}
一个经典的例子见 “一维自由粒子(量子)\upref{FreeP1}” 。 至于系数具体如何求, 见“量子散射(一维)\upref{Sca1D}”。 注意由于简并, 有时候需要先把 $\psi_{E,1}(x), \psi_{E,2}(x)$ 进行正交归一化\upref{ScaNrm}才能投影得到系数。
