% 群论中的证明和习题解答
% keys 群论|正规子群|逆序数
% license Xiao
% type Tutor

\subsection{子群和正规子群}

\begin{example}{对换和逆序数}\label{ex_GroupP_1}
问题来源请见\autoref{ex_NormSG_5}~\upref{NormSG}。

%未完成


\end{example}

\subsection{群作用}

\begin{example}{迷向子群}\label{ex_GroupP_2}
问题来源请见\autoref{exe_Group3_1}~\upref{Group3}。

已知群 $G$ 作用在集合 $X$ 上,且对于 $x\in X$ 有 $G$ 的子集 $F_x=\{g\in G|g\cdot x=x\}$。

那么 $\forall g\in F_x$,$(g^{-1}g)\cdot x=e\cdot x=x$。又因为 $g\cdot x=x$,所以 $g^{-1}\cdot x=g^{-1}\cdot(g\cdot x)=(g^{-1}g)\cdot x=x$。

因此,$\forall g_1, g_2\in F_x$,有 $(g_1^{-1}g_2)\cdot x=g_1^{-1}\cdot x=x$,也就是说,$(g_1^{-1}g_2)\in F_x$。根据判别式\autoref{the_Group1_3}~\upref{Group1},$F_x$ 构成 $G$ 的子群。

\end{example}

\begin{example}{Burnside引理}\label{ex_GroupP_3}
首先转写一个表达:$\sum_{g\in G}|X^g|=$“满足 $g\cdot x=x$ 的 $(g, x)$ 数量”$=\sum_{x\in X}|F_x|$。

由\autoref{cor_Group3_1}~\upref{Group3},$\sum_{x\in X}|F_x|=\sum_{x\in X} |G|/|O_x|=|G| \sum_{x\in X} 1/|O_x|$

现在关键是 $\sum_{x\in X} 1/|O_x|$ 是多少。显然,$\sum_{x\in O_x} 1/|O_x|=1$,也就是说 $1/|O_x|$ 在每一个 $O_x$ 上求和的结果是 $1$,那么它在整个 $X$ 上求和的结果,刚好就是所有轨道的数量 $|\{O_x|x\in X\}|$。

把以上结果整合,我们有:$\sum_{g\in G}|X^g|=|G|\cdot|\{O_x|x\in X\}|$。
\end{example}

\begin{example}{Burnside 引理的应用}
一堆珍珠有$k$种颜色,且每种颜色的珍珠都足够多。其中取$n$个珍珠串成一个手环,试问一共可以组成多少种不同的手环?旋转或翻转可以重合的情况算作一种。

\begin{figure}[ht]
\centering
\includegraphics[width=12cm]{./figures/0a2966096c5d9edb.png}
\caption{$n=7$, $k=3$时,两种不同的手环} \label{fig_GroupP_1}
\end{figure}

我们给每个珍珠从$1$到$n$编号。设$F=\{f_1,...,f_k\}$是不同颜色的集合,$X=F^n$是在固定顺序下,手环的所有颜色排列方案。显然在顺序固定情况下有$k^n$种可能性,对应的是$|X|=k^n$。我们现在要做的,是把所有旋转或翻转可以重合的情况,只保留记作一种。这等价于二面体群$D_{2n}$作用在集合$X$上。对于任意元素$x_1,x_2\in X$,$x_1$在$D_{2n}$作用下得到$x_2$,等价于$x_1,x_2$位于同一个轨道。而我们想求的,也正是轨道的数量。

有了这样的思路,我们应用Burnside 引理:首先对$D_{2n}=\{e,d,...,d^{n-1},s,ds,...,d^{n-1}s\}$中的每个元素,计算其不动点的数量。(其中$d$表示旋转,$s$表示翻转)

\begin{figure}[ht]
\centering
\includegraphics[width=14.25cm]{./figures/07984184160640da.png}
\caption{翻转操作下的不动点} \label{fig_GroupP_2}
\end{figure}

对于翻转,当$n$为奇数时,翻转轴一定经过一颗珍珠。位于这颗珍珠,以及翻转轴任意一侧的珍珠,一共$(n+1)/2$个,我们可以随意为它们选择颜色,从而得到$k^\frac{n+1}{2}$个不动点 偶数时,

\begin{figure}[ht]
\centering
\includegraphics[width=14cm]{./figures/b49c565d5a97d4f1.png}
\caption{旋转操作下的不动点} \label{fig_GroupP_3}
\end{figure}
\end{example}




\begin{example}{自由群的一般性质}\label{ex_GroupP_4}
问题来源请见\autoref{the_FreGrp_1}~\upref{FreGrp}

设集合 $S$ 到群 $G$ 上有一个集合间的映射 $f$。我们来构造一个 $\varphi: F(S)\rightarrow G$ 的同态。

首先,由于 $\varphi$ 是 $f$ 的扩张,因此要定义 $\forall s\in S, \varphi(s)=f(s)$。

其次,由于 $\varphi$ 应为一个群同态,因此要定义 $\forall s_i\in S, \varphi({s_1s_2s_3\cdots})=\varphi(s_1)\varphi(s_2)\varphi(s_3)\cdots$。同时,还需要有 $\varphi(s^{-1})=\varphi(s)^{-1}$。

这样一来,$\varphi$ 的映射规则就确定下来了,并且容易验证它是一个群同态。因此这是唯一一个符合同态条件的扩张。


\end{example}
