% 方势垒散射数值计算
% 方势垒|散射|数值计算

\begin{issues}
\issueDraft
\issueOther{未完成: 需要改进程序, 计算波函数随时间的变化. (给系数加上 $\exp(-\I E t)$ 即可)}
\end{issues}

\pentry{方势垒定态波函数程序\upref{FSBplt}}

\begin{figure}[ht]
\centering
\includegraphics[width=10cm]{./figures/FSBsct_1.png}
\caption{运行结果} \label{FSBsct_fig1}
\end{figure}

需要使用 “方势垒定态波函数程序\upref{FSBplt}” 中的 \verb|FSB.m| 和 \verb|FSB2.m|.
\begin{lstlisting}[language=matlab, caption=FSBsct.m]
% 使用类 sin,cos 基底
clear; close all;

% === params ====
k0 = 4; a = 1/6;
Nk = 300; kmax = k0 + 12*a; kmin = max(kmax/Nk, k0 - 12*a);
x0 = 0;
Nx = 2000; xmin = x0-6/a; xmax = x0+6/a;
psi = @(x)exp(1i*k0*x).*exp(-(a*(x-x0)).^2);
L = 3; V0 = 1; m = 1;
% ==============

% plot FFT
k = linspace(kmin, kmax, Nk); dk = k(2)-k(1);
[g1, k1] = FFT(fftresize(psi(x),Nx*2), dx);
g = interp1(k1, g1, k);
figure; plot(k, abs(g), '.');
axis([kmin, kmax, 0, max(abs(g))*1.1]);
xlabel k; title 'FFT of \psi(x)';

A = zeros(1,Nk); B = A;
E = k.^2/(2*m);
psi1 = zeros(size(x));
for i = 1:Nk
    A(i) = integral(@(x)psi(x).*FSB(x,E(i),L,m,V0,0), ...
            xmin, xmax, 'RelTol', 1e-16);
    B(i) = integral(@(x)psi(x).*FSB(x,E(i),L,m,V0,1), ...
            xmin, xmax, 'RelTol', 1e-16);
    psi1 = psi1 + dk*A(i)*FSB(x,E(i),L,m,V0,0) ...
                + dk*B(i)*FSB(x,E(i),L,m,V0,1);
end
figure; plot(x, real(psi(x)));
hold on; plot(x, real(psi1), '--');
plot([xmin, -L,-L, L, L, xmax], [0, 0, 0.5, 0.5, 0, 0], 'b');
axis([xmin,xmax,-1,1]);
title 'reconstruction of \psi(x)'
\end{lstlisting}

\begin{lstlisting}[language=matlab]
% 使用类 exp(ikx), exp(-ikx) 基底
clear; close all;

% === params ====
k0 = 4; a = 1/6;
Nk = 300; kmax = k0 + 12*a; kmin = max(kmax/Nk, k0 - 12*a);
x0 = 0;
Nx = 2000; xmin = x0-6/a; xmax = x0+6/a;
psi = @(x)exp(1i*k0*x).*exp(-(a*(x-x0)).^2);
L = 3; V0 = 1; m = 1;
% ==============

% plot FFT
k = linspace(kmin, kmax, Nk); dk = k(2)-k(1);
[g1, k1] = FFT(fftresize(psi(x),Nx*2), dx);
g = interp1(k1, g1, k);
figure; plot(k, abs(g), '.');
axis([kmin, kmax, 0, max(abs(g))*1.1]);
xlabel k; title 'FFT of \psi(x)';

C1 = zeros(1,Nk); C2 = C1;
E = k.^2/(2*m);
psi1 = zeros(size(x));
for i = 1:Nk
    C1(i) = integral(@(x)psi(x).*conj(FSB2(x,E(i),L,m,V0,1)), ...
            xmin, xmax, 'RelTol', 1e-16);
    C2(i) = integral(@(x)psi(x).*conj(FSB2(x,E(i),L,m,V0,-1)), ...
            xmin, xmax, 'RelTol', 1e-16);
    psi1 = psi1 + dk*C1(i)*FSB2(x,E(i),L,m,V0,1) ...
                + dk*C2(i)*FSB2(x,E(i),L,m,V0,-1);
end
figure; plot(x, real(psi(x)));
hold on; plot(x, real(psi1), '--');
plot([xmin, -L,-L, L, L, xmax], [0, 0, 0.5, 0.5, 0, 0], 'b');
axis([xmin,xmax,-1,1]);
title 'reconstruction of \psi(x)'
\end{lstlisting}
