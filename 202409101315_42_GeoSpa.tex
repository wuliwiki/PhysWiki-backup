% 时空的几何
% keys 类时|本征时|时空|光锥|类空|类光
% license Usr
% type Tutor

学习狭义相对论的“现代”方式时强调时空的几何,这种方法将引领我们自然的到达广义相对论和Einstein的引力。本部分以一种更为严格的形式展现狭义相对论的几何。

\subsection{基本定义}
相对论认为,任一事件都由它发生的时间和地点确定,因此描述事件的所在“空间”(数学概念,可视为拓扑空间的空间)称为时空。
\begin{definition}{时空,事件,距离}
若 $E$ 是一集合, 其上的元素通过映射 $x:E\rightarrow \mathbb R^4$ 和四维矢量空间 $\mathbb R^4$ 一一对应。若在 $E$ 上定义了如下的距离函数
\begin{equation}
\Delta s^2_{AB}:=\rho^2(A,B):=\langle \Delta x,\Delta x\rangle=\eta_{\alpha\beta}\Delta x^\alpha\Delta x^\beta,~
\end{equation}
其中,$\Delta x:=x(B)-x(A)$,$\eta=\mathrm{diag}[-1,1,1,1]$ 是对角化矩阵。则称 $E$ 的元素为\textbf{事件}(event),$\mathbb R^4$ 为 $E$ 的\textbf{时空},映射 $x$ 称为\textbf{坐标系}, $x_A:=x(A)$ 是称为点 $A\in E$ 的\textbf{时空坐标}。$\Delta s_{AB}=\sqrt{\Delta s^2_{AB}}$ 称为 $A,B$ 的\textbf{间隔}(separation),$\eta$ 称为\textbf{度规}(matric)。规定坐标从0标记,即 $\alpha=0,1,2,3$,并称坐标 $x^0$ 为\textbf{时间(time)坐标},可记为 $t$, $(x^1,x^2,x^3)$ 称为\textbf{空间(space)坐标},可记作 $\bvec r$。
\end{definition}
\textbf{注:}事实上,更严格的狭义相对论时空几何的定义需要借助仿射空间定义在伪欧几里得空间上(指数有限度量空间\upref{EFSp})。即定义在一个点空间配上一个带有不定二次型的矢量空间上。然而,为了减轻物理专业的负重,并考虑到在坐标系确定了的仿射空间中,仿射空间中的点和矢量空间的矢量一一对应(\autoref{sub_AfSp_2}),这里直接放弃仿射空间,而只用到4维矢量空间 $\mathbb R^4$。因此,只需注意,这里讨论的时空实际上是已经选定了某一坐标系的仿射空间,即是某一观测者视角下的时空几何。

注意到度规 $\eta$ 是一个不定型(\autoref{def_DeQua_2}),因此 $s^2(A,B)$ 有三种可能,即 $\Delta s^2_{AB}>0,\Delta s^2_{AB}=0,\Delta s^2_{AB}<0$。

\begin{definition}{类时,类空,类光}
设 $A,B$ 是时空中的两个事件,则称间隔 $\Delta s_{AB})$ 为\textbf{类时间隔}(timelike),若 $\Delta s^2_{AB}<0$;为\textbf{类空间隔}(sapcelike),若 $\Delta s^2_{AB}>0$;为\textbf{类光间隔}(lightlike)或\textbf{零值}(null),若 $\Delta s^2_{AB}=0$。
\end{definition}
“类光” 一词来源于相对论中光的轨迹为时空中的直线,即 $\Delta t^2=\Delta \bvec r^2$。显然,
\begin{equation}
\begin{aligned}
\Delta t^2>\Delta \bvec r^2,\quad \text{类时},\\
\Delta t^2=\Delta \bvec r^2,\quad \text{类光},\\
\Delta t^2<\Delta \bvec r^2,\quad \text{类空}.
\end{aligned}~
\end{equation}

在相对论中,对于(有质量)粒子,其在两邻近点的间隔满足 $\dd s^2<0$。
\begin{definition}{本征时间}
称由
\begin{equation}
\Delta \tau^2:=-\Delta s^2~
\end{equation}
定义的 $\tau$ 为\textbf{本征时间}(proper time)。
\end{definition}

\subsection{洛伦兹变换}
在数学上,运动是使得两点间距离保持不变的线性变换,而在伪欧几里得空间,为了在实数情况下讨论,运动被定义为使得距离平方不变的变换\upref{EFSp}。对应在相对论中,使得间隔平方不变的线性变换则称为Lorentz变换。
\begin{definition}{Lorentz变换}
设 $f:\mathbb R^4\rightarrow\mathbb R^4$ 是线性变换,若对时空中任意两点 $A,B$ ,都有
\begin{equation}
\Delta s^2_{f(x(A))f(B)}=\Delta s^2_{AB},~
\end{equation}
则称 $f$ 为\textbf{Lorentz变换}(Lorentx transformation)。
\end{definition}

\begin{theorem}{}
设 $f$ 是洛伦兹变换,$F$ 是其对应的矩阵,则
\begin{equation}
F^T\eta F=\eta.~
\end{equation} 
\end{theorem}

\textbf{证明:}
事实上,$\inner{Fx,}$
\textbf{证毕!}




















