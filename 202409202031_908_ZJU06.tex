% 浙江大学 2006 年 考研 量子力学
% license Usr
% type Note

\textbf{声明}:“该内容来源于网络公开资料,不保证真实性,如有侵权请联系管理员”

\subsection{第一题(50 分)简答题:}
(1) 从坐标与动量算符的对易关系($[\hat{x}, \hat{p}] = i\hbar$ 等)推出角动量算符与动量算符的对易关系。

(2) 请用泡利矩阵 $\sigma^x = \begin{pmatrix} 0 & 1 \\\\ 1 & 0 \end{pmatrix}$,$\sigma^y = \begin{pmatrix} 0 & -i \\\\ i & 0 \end{pmatrix}$,$\sigma^z = \begin{pmatrix} 1 & 0 \\\\ 0 & -1 \end{pmatrix}$ 定义电子的自旋算符,并验证它们满足角动量对易关系。

(3) 量子力学中的可观测量算符为什么应为厄米算符?

(4) 你知道量子力学中的哪些数定律在经典物理中没有对应?

(5) 设$\Psi_0$ 为$\hat{H}_0$的简并本征函数,相应的能量本征值为 $E_n$,如果 $\hat{H} = \hat{H}_0 + \hat{H}'$,其中 $\hat{H}'$ 可看作微扰。试写出能级的微扰修正公式(写到二级修正)。

(6) 什么叫受激辐射?什么叫自发辐射?

(7) 写出由 $\frac{1}{2}$ 自旋态构成的总自旋为0的态矢和自旋为1的态矢。
\subsection{第二题(20 分):}
已知氢原子的基态波函数为:
\[\psi(r, \theta, \varphi) = \frac{1}{\sqrt{\pi a_0^3}} e^{-r / a_0},~\]
\begin{enumerate}
    \item 求氢原子的最可几半径(即径向几率密度取最大值的 $r$ 值)。
    \item 求氢原子的平均半径(即 $r$ 的平均值)。
\end{enumerate}
\subsection{第三题(20 分):}
有一个质量为 $M$ 的粒子在宽度为 $a$ 的无限深势阱中运动。
\begin{enumerate}
    \item 求出其能级和波函数。
    \item 如果该粒子的自旋为 $\frac{1}{2}$,则能级二重简并。加入磁场后 Zeeman 效应会让能级分裂,简并消除。当磁场为某个特殊值时,又会出现简并能级。试求该磁场的值。
\end{enumerate}
\subsection{第四题(20 分):}
试求
\[\hat{H} = \frac{1}{2M} \left( \hat{p}_x^2 + \hat{p}_y^2 \right) + \frac{M}{2} \omega_c^2 \left( x^2 + y^2 \right) + \omega_c \hat{L}_z~\]
的能级。你觉得能级简并度有什么特点?

[提示:二维各向同性谐振子可用极坐标求解,能级为 
\[E = \left( 2n_\rho + |m| + 1 \right) \hbar \omega_c~\]
 $n_\rho$ 为径向量子数,$m$ 为磁量子数。]
\subsection{第五题(20分):}
一个体系的哈密顿量为 
\[H = \sigma_1^x \sigma_2^x + \sigma_1^x \sigma_2^y + \sigma_1^z \sigma_2^z ~\]
其中 $\lambda$ 为实数,泡利矩阵的下标 $1, 2$ 表示第一个粒子和第二个粒子,用矩阵的直乘理解即为 $\sigma_1^x \sigma_2^x = \sigma_1^x \otimes \sigma_2^x$ 等等。

(1)求出其本征值。

(2)对于不同的 $\lambda$ 取值范围,写出相应的基态矢量。
\subsection{第六题(20分:选做(A)、(B)、(C)其中一题即可):}
\begin{itemize}
    \item (A)用玻恩近似求势场 $V(r) = V_0 e^{-r^2 / x_0^2}$ 的散射截面。
    \item (B)用分波法求势场 
    \[    V(n) = \begin{cases}        0 & r > R \\\\        \infty & r \leq R    \end{cases} ~\]
    散射的 $s$ 波相移。
    \item (C)求一维方势阱 
    \[    V(x) = \begin{cases}        0 & |x| \geq a \\\\        -V_0 & |x| < a    \end{cases} ~\]
    的透射系数,并给出发生共振透射的条件。
\end{itemize}