% 计数原理
% 加法原则|乘法原则

\pentry{集合的基数\upref{CardiN}}
本节介绍组合学里两个一般性的原则——加法原则和乘法原则.

以下假设 $A$ 和 $B$ 是两类不同、互不关联的事件.
\subsection{加法原则}
\textbf{加法原则:}设事件 $A$ 有 $m$ 种选取方式,事件 $B$ 有 $n$ 种选取方式,则选 $A$ 或 $B$ 共有 $m+n$ 种方式.

用集合\upref{Set}的语言可将加法原则描述成如下定理(\autoref{CardiN_the1}~\upref{CardiN}):
\begin{theorem}{}
设 $A,B$ 为有限集,且 $A\cap B=\emptyset$ ,则
\begin{equation}
\abs{A\cup B}=\abs{A}+\abs{B}
\end{equation}
\end{theorem}
\begin{corollary}{}
设 $n$ 个有限集合 $A_1,\cdots,A_n$ 满足
\begin{equation}
A_i\cap A_j=\emptyset,1\leq i\neq j\leq n
\end{equation}
则
\begin{equation}
\abs{\bigcup_{i=1}^n A_i}=\sum_{i=1}^n\abs{A_i}
\end{equation}

\end{corollary}
\subsection{乘法原则}
\textbf{乘法原则:}设事件 $A$ 有 $m$ 种选取方式,事件 $B$ 有 $n$ 种选取方式,那么选取 $A$ 以后再选取 $B$ 共有 $m\cdot n$ 种方式.

同样,用集合的语言可将乘法原则描述成如下的定理:
\begin{theorem}{}
设 $A,B$ 为有限集,$\abs{A}=m,\abs{B}=n$, 则
\begin{equation}\label{ProfPM_eq1}
\abs{A\times B}=\abs{A}\times\abs{B}=m\cdot n
\end{equation}
\end{theorem}
\textbf{证明:}若 $m=0$ 或 $n=0$,则\autoref{ProfPM_eq1} 的两边均为0,故等式成立.

设 $m>0,n>0$,并且记
\begin{equation}
A=\{a_1,\cdots,a_m\},\quad B=\{b_1,\cdots,b_n\}
\end{equation}
定义映射
\begin{equation}
\varphi:(a_i,b_j)\rightarrow (i-1)n+j\quad (1\leq i\leq m,1\leq j\leq n)
\end{equation}
则 $\varphi$ 是 $A\times B$ 到集合 $\{1,2\cdots,mn-1,mn\}$ 上的一一映射(试证明),根据集合基数的定义(\autoref{CardiN_def1}~\upref{CardiN}),\autoref{ProfPM_eq1} 成立.

\textbf{证毕!}
\begin{corollary}{}
设 $A_1,\cdots,A_n$ 为 $n$ 个有限集合,则
\begin{equation}
\abs{A_1\times A_2\cdots A_n}=\abs{A_1}\times\abs{A_2}\times\cdots\times\abs{A_n}
\end{equation}

\end{corollary}