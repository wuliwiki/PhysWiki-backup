% 克里斯蒂安·惠更斯(综述)
% license CCBYSA3
% type Wiki

本文根据 CC-BY-SA 协议转载翻译自维基百科\href{https://en.wikipedia.org/wiki/Christiaan_Huygens}{相关文章}。

\begin{figure}[ht]
\centering
\includegraphics[width=6cm]{./figures/759a661ac1a7d67e.png}
\caption{惠更斯肖像,由卡斯帕·内彻绘于1671年,现藏于莱顿博尔哈夫博物馆[1]} \label{fig_HGS_1}
\end{figure}

克里斯蒂安·惠更斯,泽尔亨领主,英国皇家学会院士(/ˈhaɪɡənz/,音译‘海根斯’,[2] 美国亦发音为 /ˈhɔɪɡənz/,音译‘霍伊根斯’;[3] 荷兰语:[ˈkrɪstijaːn ˈɦœyɣə(n)s] ⓘ;也拼作 Huyghens;拉丁语:Hugenius;1629年4月14日-1695年7月8日),是一位荷兰数学家、物理学家、工程师、天文学家和发明家,被视为科学革命中的关键人物之一。[4][5] 在物理学领域,惠更斯在光学和力学方面做出了开创性的贡献;作为天文学家,他研究了土星的光环并发现了土星最大的卫星——泰坦。作为工程师和发明家,他改进了望远镜的设计,并发明了摆钟,这种时钟在近300年内是最精确的计时工具。他是一位才华横溢的数学家和物理学家,其著作首次通过一组数学参数对物理问题进行了理想化描述,[6] 并首次对一种无法直接观测的物理现象进行了数学和机械论的解释。[7]

惠更斯在其著作《De Motu Corporum ex Percussione》中首次正确地确定了弹性碰撞的定律,该书完成于1656年,但于1703年才在他去世后出版。[8] 1659年,惠更斯在其著作《De vi Centrifuga》中以几何方法推导出了经典力学中描述离心力的公式,这比牛顿早了十年。[9] 在光学领域,他因提出光的波动理论而闻名,这一理论发表于其1690年的《光论》(Traité de la Lumière)。惠更斯的光波理论最初被牛顿的光微粒理论所取代,直到1821年,奥古斯丁-让·菲涅耳(Augustin-Jean Fresnel)改进了惠更斯的原理,完整解释了光的直线传播和衍射现象。今天,这一原理被称为“惠更斯-菲涅耳原理”。

1657年,惠更斯发明了摆钟,并于同年获得专利。他对钟表的研究最终在《摆动时钟》(Horologium Oscillatorium,1673年)中发表,该书被认为是17世纪关于力学的重要著作之一。[6] 虽然书中包含了钟表设计的描述,但大部分内容是对摆动运动的分析和曲线理论。1655年,惠更斯与其兄弟康斯坦丁(Constantijn)开始研磨透镜,制作折射望远镜。他发现了土星最大的卫星——泰坦,并首次解释了土星的奇特外观是由于“一个薄而平坦的环,其不与土星接触,并倾斜于黄道面”。[10] 1662年,惠更斯开发了如今称为“惠更斯目镜”的装置,这是一种采用两个透镜的望远镜,能够减少色散现象。[11]

作为数学家,惠更斯发展了渐伸线的理论,并在《赌博中的计算》(Van Rekeningh in Spelen van Gluck)中研究了几何概率和点数问题。该书由弗朗斯·范·斯库腾(Frans van Schooten)翻译并以《赌博中的推理》(De Ratiociniis in Ludo Aleae,1657年)出版。[12] 惠更斯及其他人对期望值的使用后来启发了雅各布·伯努利(Jacob Bernoulli)对概率论的研究。[13][14]
\subsection{传记}
\begin{figure}[ht]
\centering
\includegraphics[width=6cm]{./figures/805ceabebaf64426.png}
\caption{康斯坦丁与他的五个孩子合影(克里斯蒂安位于右上方)。海牙莫里茨皇家美术馆。} \label{fig_HGS_2}
\end{figure}
克里斯蒂安·惠更斯于1629年4月14日出生在海牙的一个富有且有影响力的荷兰家庭,[15][16] 是康斯坦丁·惠更斯的次子。他以祖父的名字命名。[17][18] 他的母亲苏珊娜·范·巴尔勒(Suzanna van Baerle)在生下惠更斯的妹妹后不久去世。[19] 夫妇俩育有五个孩子:康斯坦丁(1628年)、克里斯蒂安(1629年)、洛德维克(1631年)、菲利普斯(1632年)和苏珊娜(1637年)。[20]

康斯坦丁·惠更斯是奥兰治家族的外交官和顾问,同时也是一位诗人和音乐家。他与欧洲各地的知识分子有广泛的书信往来,他的朋友包括伽利略·伽利莱、马林·梅森和勒内·笛卡尔。[21] 克里斯蒂安在16岁之前接受家庭教育,从小喜欢玩弄磨坊和其他机器的模型。他从父亲那里接受了全面的教育,学习语言、音乐、历史、地理、数学、逻辑和修辞,同时还学习舞蹈、击剑和骑马。[17][20]

1644年,惠更斯的数学导师是扬·扬斯·斯塔姆皮恩(Jan Jansz Stampioen),他为15岁的惠更斯布置了一份有关当代科学的高难度阅读清单。[22] 后来,笛卡尔对他在几何学方面的能力印象深刻,梅森则称他为“新阿基米德”。[23][16][24]
\subsubsection{学生时代}
16岁时,康斯坦丁送克里斯蒂安·惠更斯到莱顿大学学习法律和数学,他从1645年5月学到1647年3月。[17] 从1646年开始,弗朗斯·范·斯库滕(Frans van Schooten)成为莱顿大学的一名学者,并在笛卡尔的建议下接替斯塔姆皮恩(Stampioen),成为惠更斯及其兄长康斯坦丁·小惠更斯的私人导师。[25][26] 范·斯库滕为惠更斯提供了最新的数学教育,向他介绍了韦达(Viète)、笛卡尔和费马(Fermat)的研究成果。[27]

1647年3月起,惠更斯继续在新成立的布雷达奥兰治学院(Orange College)学习两年,该学院的管理者之一是他的父亲康斯坦丁。康斯坦丁深度参与了这所学院的事务,但学院仅持续到1669年,校长是安德烈·里韦(André Rivet)。[28] 惠更斯在学习期间寄住在法学家约翰·亨里克·道伯(Johann Henryk Dauber)家中,并由英国讲师约翰·佩尔(John Pell)教授数学。他在布雷达的学习结束于其兄弟洛德维克因与另一名学生决斗事件而中断学业的时期。[5][29] 惠更斯于1649年8月完成学业后离开布雷达,并短暂以外交官身份随纳骚公爵亨利(Henry, Duke of Nassau)出使。[17] 这次任务带他去了本特海姆(Bentheim)和弗伦斯堡(Flensburg)。随后,他前往丹麦,访问了哥本哈根和赫尔辛格(Helsingør),并希望穿越厄勒海峡(Øresund)前往斯德哥尔摩拜访笛卡尔,但因笛卡尔此时已去世而未能成行。[5][30]

尽管他的父亲康斯坦丁希望克里斯蒂安成为一名外交官,但种种原因阻止了他的从政之路。1650年开始的第一次无护国主时期(First Stadtholderless Period)导致奥兰治家族失去权力,从而削弱了康斯坦丁的影响力。此外,他也意识到儿子对这条职业道路毫无兴趣。[31]
\subsubsection{早期通信}
\begin{figure}[ht]
\centering
\includegraphics[width=6cm]{./figures/4dfc68904d4ad9e9.png}
\caption{惠更斯手稿中悬链线(catenary)图的插图。} \label{fig_HGS_3}
\end{figure}