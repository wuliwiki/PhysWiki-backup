% 全同粒子
% keys 量子力学|交换算符|对易|哈密顿|玻色子|费米子

\pentry{多体量子力学简介% 未完成: 应该引用什么词条呢?  这应该是二级词条, 是否引用一级词条?
, 角动量加法(量子力学)\upref{AMAdd}, 自旋角动量\upref{Spin}}

在量子力学的基本假设规定基本粒子是不可区分的, 例如电子,质子,中子等, 我们把不可区分的粒子称为是\textbf{全同粒子(identical particles)}. 例如我们在两个位置同时测量到两个电子, 我们只能知道两处各有一个电子, 而不知道哪个电子在哪里. 对于宏观物体, 我们可以通过做标记或者追踪轨迹的方式轻易把不同它们区分开, 但量子力学中的粒子既不能做标记也没有轨迹的概念, 所以我们唯一可以区分的粒子的方法就是利用它们的物理属性, 如质量,电荷量,自旋等物理性质.

\subsection{波函数的对称性}
先不考虑自旋\upref{Spin}, 我们如何用它们的波函数体现 “不可区分” 呢?对于位置表象下的双粒子波函数 $\psi(\bvec r_1, \bvec r_2)$, 在任意两点 $\bvec r_1, \bvec r_2$ 同时发现两个电子的概率密度函数为% 链接未完成
\begin{equation}
f(\bvec r_1, \bvec r_2) = \abs{\psi(\bvec r_1, \bvec r_2)}^2
\end{equation}
为了体现两粒子不可区分, 我们要求概率分布满足 $f(\bvec r_2, \bvec r_1) = f(\bvec r_1, \bvec r_2)$. 于是波函数必须满足以下两个条件之一\footnote{乍看之下 $\abs{\psi(\bvec r_2, \bvec r_1)} = \abs{\psi(\bvec r_1, \bvec r_2)}$ 仅要求 $\psi(\bvec r_2, \bvec r_1) = \E^{\I \theta} \psi(\bvec r_1, \bvec r_2)$($\theta \in [0, 2\pi)$), 但如果把右边的 $\bvec r_1, \bvec r_2$ 再次交换就会得到 $\psi(\bvec r_2, \bvec r_1) = \E^{\I 2\theta} \psi(\bvec r_2, \bvec r_1)$, 也就是 $\E^{2\I\theta} = 1$. 所以只能取 $\E^{\I\theta} = \pm 1$.}
\begin{equation}\label{IdPar_eq5}
\psi(\bvec r_2, \bvec r_1) = \pm \psi(\bvec r_1, \bvec r_2)
\end{equation}
如果取正号, 我们说这样的波函数满足\textbf{交换对称(exchange symmetry)}, 或者不混淆的情况下简称\textbf{对称}\footnote{波函数的另外一种对称是宇称对称(链接未完成).}; 取负号时则称波函数满足\textbf{反交换对称(exchange antisymmetry)}. 如果波函数既不是对称也不是反对称, 那么就称为\textbf{不对称的}, 只有两个粒子是不同的基本粒子时, 才可能出现不对称波函数. 注意对于两个全同粒子的波函数, $\bvec r_1, \bvec r_2$ 中的数字不再代表粒子的编号, 而仅仅用于数学上区分两个不同位置\footnote{总不能写成 $\psi(\bvec r, \bvec r)$ 吧?}.

以后会看到,具有交换对称或反对称的波函数在根据薛定谔方程演化的过程中将会一直会保持它的对称性\upref{ExchOp}. % 链接未完成

\begin{example}{}
若某一维势阱中有两个互不作用的全同粒子, 
$\psi(x_1, x_2) = f(x_1) g(x_2)$ 对不同的函数 $f, g$ 既不是对称也不是反对称, 因为 $f(x_2)g(x_1) \ne \pm f(x_1)g(x_2)$.
\end{example}

\subsection{自旋态的对称性}
如果只考虑双粒子的自旋态矢量空间(未完成:链接到自旋相关词条), 同样由于概率原因我们要求全同粒子同样在该空间的态矢满足交换对称.

\begin{example}{}\label{IdPar_ex1}
以常见的双电子自旋为例. 单电子自旋为 $1/2$, 自旋态在 2 维空间, 两个基底通常记为 $\uparrow$ 和 $\downarrow$, 分别代表自旋 $z$ 分量 $m_s=\pm 1/2$. 而双电子的自旋态是该空间和自己的张量积, 是 $2\times2 = 4$ 维的, 四个正交归一基底自然地可以取张量积 $\uparrow\uparrow, \uparrow\downarrow, \downarrow\uparrow, \downarrow\downarrow$ . 但中间两个既不是对称也不是反对称的, 所以我们可以将他们稍作线性变换, 得到 4 个具有交换对称性的正交归一基底
\begin{equation}\label{IdPar_eq1}
\uparrow\uparrow \qquad \frac{1}{\sqrt 2}(\uparrow\downarrow + \downarrow\uparrow) \qquad \downarrow\downarrow
\end{equation}
\begin{equation}\label{IdPar_eq2}
\frac{1}{\sqrt 2}(\uparrow\downarrow - \downarrow\uparrow)
\end{equation}
其中\autoref{IdPar_eq1} 的三个基底是对称的, 即\textbf{三重态(triplet)};\autoref{IdPar_eq2} 的一个基底是反对称的, 即\textbf{单态(singlet)}. % 链接未完成: 角动量加法中的例题
显然,三重态的任意线性组合都是对称的自旋态\footnote{当我们这么说时,默认还没有做归一化.}, 他们构成了 3 维的子空间\upref{SubSpc}, 而反对称的只有单态为基底的 1 维子空间.

但这里的对称性具体如何定义呢? 我们可以用 $\chi_{1,2}$ 表示这个 4 维空间中的任意矢量, 定义 $\chi_{2,1}$ 就是把 $\chi_{1,2}$ 的每项中两个箭头(单粒子态)互换, 例如 $\downarrow\uparrow$ 变为 $\uparrow\downarrow$ 又如 $\downarrow\uparrow - \uparrow\downarrow$ 变为 $\uparrow\downarrow - \downarrow\uparrow$. 于是(反)对称条件就是
\begin{equation}\label{IdPar_eq3}
\chi_{2,1} = \pm\chi_{1,2}
\end{equation}

我们可以把 $\uparrow\uparrow$ 和 $\downarrow\downarrow$ 视为两电子自旋方向相同的态, $(\uparrow\downarrow \pm \downarrow\uparrow)/\sqrt{2}$ 视为两电子自旋方向相反的态. 对于后者, 我们只能说 “一个电子自旋为 $\uparrow$ 另一个为 $\downarrow$” 却不能说哪个是 $\uparrow$ 哪个是 $\downarrow$. 这正体现了全同粒子的不可区分性. 相比之下, $\uparrow\downarrow$ 就表明第一个电子是 $\uparrow$, 第二个是 $\downarrow$, 所以不能对两个电子使用.
\end{example}

一般地, 对于自旋量子数为 $s$ 的两个全同粒子, 单个粒子处于 $N = 2s+1$ 维矢量空间中, 双粒子态空间就是 $N^2$ 维的. 其中对称子空间是 $N(N+1)/2$ 维的, 反对称子空间是 $N(N-1)/2$ 维的, 相加恰好等于 $N^2$. 所以他们是两个互补的正交子空间\upref{OrthSp}. 我们把推导留到 “粒子交换算符\upref{ExchOp}” 中.

\subsection{玻色子和费米子}
当我们考虑两个粒子的总状态时, 我们可以把总空间视为双粒子波函数空间和双粒子自旋态空间的张量积空间.双粒子态矢总可以分解为波函数和自旋的张量积的线性组合 % 链接未完成
\begin{equation}
\ket{\Psi_{1,2}} = \sum_i \psi_i(\bvec r_1, \bvec r_2) \chi^{(i)}_{1,2}
\end{equation}
在我们往往讨论的只是可以记为一项的简单态
\begin{equation}\label{IdPar_eq6}
\ket{\Psi_{1,2}} = \psi(\bvec r_1, \bvec r_2) \chi_{1,2}
\end{equation}
与\autoref{IdPar_eq5} 同理, 全同粒子需要满足
\begin{equation}\label{IdPar_eq4}
\ket{\Psi_{1,2}} = \pm \ket{\Psi_{2,1}}
\end{equation}
对于\autoref{IdPar_eq6} 来说, $\ket{\Psi_{1,2}}$ 是对称的当且仅当波函数和自旋具有相同对称性(都是对称或者都是反对称), $\ket{\Psi_{1,2}}$ 是反对称的当且仅当波函数和自旋具有相反对称性(一个对称一个反对称).

波函数或自旋的两种交换对称性会带来实验上可观测的结果. 实验表明, 我们可以把所有的基本粒子划分为两类, 自旋为整数的称为\textbf{玻色子(boson)},具有对称态矢; 自旋为半整数的称为\textbf{费米子(fermion)}, 具有反对称态矢. 这是量子力学关于全同粒子的基本假设. 至于为什么自旋可以决定波函数对称性, 需要使用量子场论才能解释. % 链接未完成.

例如电子是自旋为 1/2 的费米子, 为了保证总状态的反对称, 如果两电子的自旋处于单态(反对称), 那么波函数就必须是对称的; 反之如果两电子处于三重态(对称), 那么波函数必须是反对称的.
\begin{example}{}\label{IdPar_ex2}
一维情况下,一个粒子处于态$\psi_A$,另一个处于态$\psi$,两个波函数正交且归一,有以下三个情况:
\begin{enumerate}
\item \textbf{可分辨}情况总的波函数为:
\begin{equation}
\psi(x_1,x_2)=\psi_A(x_1)\psi_B(x_2)
\end{equation}
\item \textbf{全同玻色子}情况总的波函数为(对称):
\begin{equation}
\psi(x_1,x_2)_+=\frac{1}{\sqrt{2}}[\psi_A(x_1)\psi_B(x_2)+\psi_B(x_1)\psi_A(x_2)]
\end{equation}
\item \textbf{全同费米子}情况总的波函数为(反对称):
\begin{equation}
\psi(x_1,x_2)_-=\frac{1}{\sqrt{2}}[\psi_A(x_1)\psi_B(x_2)-\psi_B(x_1)\psi_A(x_2)]
\end{equation}
\end{enumerate}
\end{example}
\begin{corollary}{泡利不相容原理}
如果两个全同费米子占据相同的态$\psi_A=\psi_B$,那么就有:
\begin{equation}
\psi(x_1,x_2)_-=\frac{1}{\sqrt{2}}[\psi_A(x_1)\psi_A(x_2)-\psi_A(x_1)\psi_A(x_2)]=0
\end{equation}
使得我们无法得到任何的波函数.因此,两个全同费米子(其中就有广为人知的两个电子),不可能占据相同的状态.
\end{corollary}

\subsubsection{多个粒子的情况}
以上我们只讨论了两个全同粒子, 若有更多, 那么费米子和玻色子分别要求系统的态矢在交换任意两个粒子时都保持对称和反对称.
\begin{exercise}{}
\textbf{a.}假设有三个粒子,一个处于$\psi_A$态,一个处于$\psi_B$态,一个处于$\psi_C$状态.假设$\psi_A,\psi_B,\psi_C$是正交的,构造三粒子为可分辨,全同玻色子,和全同费米子的总的波函数.在构造全同费米子(反对称)我们需要用到\textbf{斯莱特Slater行列式}\upref{symetr}

\textbf{b.}假设有\textbf{两}个粒子,一个处于$\psi_A$态,一个处于$\psi_B$态,一个处于$\psi_C$状态.假设$\psi_A,\psi_B,\psi_C$是正交的,构造三粒子为可分辨,全同玻色子,和全同费米子的总的波函数.(提示:可分辨情况下,有$6$种枚举出来的总的波函数)
\end{exercise}
\subsection{交换“力”}
当波函数出现一定程度的重叠时,整个系统好像受到外力的作用:对全同波色子,这个力是吸引力,把粒子拉近;对全同费米子,这个力是排斥力,使得粒子相互远离.我们把这个“力”称为交换力,它尽管实际上并不存.交换力仅仅是对称性导致的一个几何结果.它也是仅存在于量子力学的现象,在经典力学当中并没有对应.

接下来我们承接着\autoref{IdPar_ex2} 的简化和规定,通过计算两个粒子距离平方的期待值,推导出交换“力”的原理.

\begin{itemize}
\item \textbf{可分辨的两个粒子:}
\begin{equation}
\langle x_1^2\rangle = \int x_1|\psi_A(x_1)|^2dx_1\int |\psi_B(x_2)|^2dx_2=\langle x^2\rangle_A
\end{equation}
类似的可得:
\begin{equation}
\langle x_2^2\rangle = \int |\psi_A(x_1)|^2dx_1\int x_2 |\psi_B(x_2)|^2dx_2=\langle x^2\rangle_B
\end{equation}
还有:
\begin{equation}
\langle x_1x_2\rangle = \int  x_1|\psi_A(x_1)|^2dx_1\int x_2 |\psi_B(x_2)|^2dx_2=\langle x\rangle_A\langle x\rangle_B
\end{equation}
可得两个粒子距离平方的期待值为:
\begin{equation}\label{IdPar_eq7}
\langle (x_1-x_2)^2\rangle_d=\langle x^2\rangle_A+\langle x^2\rangle_B-2\langle x\rangle_A\langle x\rangle_B
\end{equation}
反过来并拓展,粒子$ 1 $处在态$\psi_B(\bvec r)$,粒子$ 2 $处于 $\psi_A(\bvec r)$的情况也是一样的.

\item \textbf{两个全同的粒子:}

\begin{align}
\langle x_1^2\rangle &= \frac{1}{2}\big[\int x_1^2|\psi_A(x_1)|^2dx_1\int |\psi_B(x_2)|^2dx_2\\
&+\int x_1^2|\psi_B(x_1)|^2dx_1\int |\psi_A(x_2)|^2dx_2\\ 
&\pm\int x_1^2\psi_A(x_1)\psi_B(x_2)^*dx_1\int \psi_B(x_1)\psi_A(x_2)^*dx_2\\
&\pm\int x_1^2\psi_B(x_1)\psi_A(x_2)^*dx_1\int \psi_A(x_1)\psi_B(x_2)^*dx_2\big]\\
&=\frac{1}{2}(\langle x^2\rangle_A+\langle x^2\rangle_B)
\end{align}
同理可得:
\begin{equation}
\langle x_2^2\rangle=\frac{1}{2}(\langle x^2\rangle_B+\langle x^2\rangle_A)
\end{equation}
尽管$\langle x_1^2\rangle=\langle x_2^2\rangle$,但是:
\begin{align}
\langle x_1x_2\rangle &= \frac{1}{2}\big[\int x_1|\psi_A(x_1)|^2dx_1\int x_2 |\psi_B(x_2)|^2dx_2\\
&+\int x_1|\psi_B(x_1)|^2dx_1\int x_2|\psi_A(x_2)|^2dx_2\\ 
&\pm\int x_1\psi_A(x_1)\psi_B(x_2)^*dx_1\int x_2\psi_B(x_1)\psi_A(x_2)^*dx_2\\
&\pm\int x_1\psi_B(x_1)\psi_A(x_2)^*dx_1\int x_2\psi_A(x_1)\psi_B(x_2)^*dx_2\big]\\
&=\frac{1}{2}\left(\langle x\rangle_A\langle x\rangle_B+\langle x\rangle_B\langle x\rangle_A\pm\langle x\rangle_{AB}\langle x\rangle_{BA}\pm\langle x\rangle_{BA}\langle x\rangle_{AB}\right)\\
&=\langle x\rangle_A\langle x\rangle_B\pm|\langle x\rangle_{AB}|^2
\end{align}
上面推导中的\begin{equation}
\langle x\rangle_{AB}\equiv \int x\psi_Ax)\psi_B(x)^*dx
\end{equation}
因此我们最后得到了全同粒子的距离平方期待值为:
\begin{equation}
\langle (x_1-x_2)^2\rangle_\pm=\langle x^2\rangle_A+\langle x^2\rangle_B-2\langle x\rangle_A\langle x\rangle_B\mp 2|\langle x\rangle_{AB}|^2
\end{equation}
结合\autoref{IdPar_eq7} 可得:
\begin{equation}
\langle (\Delta x)^2\rangle_\pm=\langle (\Delta x)^2\rangle_d\mp 2|\langle x\rangle_{AB}|^2
\end{equation}


\end{itemize}