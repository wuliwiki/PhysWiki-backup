% 海森堡绘景

\begin{issues}
\issueDraft
\end{issues}

\footnote{参考 Wikipedia \href{https://en.wikipedia.org/wiki/Heisenberg_picture}{相关页面}.}一般薛定谔方程\upref{TDSE}使用的是\textbf{薛定谔绘景}, 在海森堡绘景中, 波函数(态矢)不随时间改变, 而测量量的算符随时间改变. 海森堡绘景相当于在薛定谔绘景的基础上做了一个基底变换, 类似于位置和动量表象\upref{moTDSE}的关系.

本文中,角标 $H$ 代表海森堡绘景, 角标 $S$ 代表薛定谔绘景. 例如波函数分别记为 $\psi_H(\bvec r, t)$ 和 $\psi_S(\bvec r)$, 后者不是时间的函数, 它的定义是
\begin{equation}
\psi_S(\bvec r) = \psi_H(\bvec r, 0)
\end{equation}

\addTODO{演化子是什么?}

使用演化子 $U(t)$, 波函数之间的关系为
\begin{equation}
\psi_H(\bvec r, t) = U(t) \psi_H(\bvec r, 0) = U(t) \psi_S(\bvec r)
\end{equation}
定义海森堡绘景中的算符为
\begin{equation}
Q_H(t) = U\Her(t) Q_S(t) U(t)
\end{equation}
其关于时间的导数为
\begin{equation}
\dv{t}Q_H = \frac{\I}{\hbar} [H_H, Q_H(t)] + \qty(\pdv{Q_S}{t})_H
\end{equation}
