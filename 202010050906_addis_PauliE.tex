% 泡利不相容原理

\pentry{全同粒子\upref{IdPar}}
\textbf{泡利不相容原理(Pauli exclusion principle)}指的是, 任意两个全同费米子(即自旋为 1/2 的粒子)不能处于同一个态矢.

即双粒子态表示为张量积 $\ket{\psi}\ket{\psi}$. 因为根据定义 $\ket{\psi}\ket{\psi}$ 是一个交换对称的态矢, 而费米子的态矢必须是反对称的.

\begin{example}{原子壳层}
在原子壳层理论中, 如果忽略电子之间的相互作用, 那么单个电子具有一系列正交归一的能量本征态, 其中能量最低的态叫做基态, 具有确定的空间波函数(除了一个相位因子) $\psi(\bvec r)$. 
\end{example}
