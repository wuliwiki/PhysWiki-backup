% Chrome 开发工具笔记
% license Usr
% type Note


\begin{issues}
\issueDraft
\end{issues}

\begin{itemize}
\item \verb`F12` 打开开发笔记
\item 主要面板有 \verb`Elements, Console, Sources, Network, ...`
\item \verb`Elements` 就是当前 DOM 树,其中很多元素可能是 js 运行后的结果,所以跟直接看页面的 source 可能是不一样的。
\item 打开开发截面以后,在页面任何地方右键 Inspect 都可以直接跳到 \verb`Elements` 的对应元素中。
\item \verb`Sources` 里面会显示加载的所有源文件,按照 URL 中的层级来分类(云朵图标),如果是方框图标就说明是浏览器插件或者 Chrome 自带的东西。
\item \verb`Network` 里面选中一个文件,在 \verb`header` 子面板里面可以看到 http 的请求头和返回头。 \verb`payload` 里面有请求中的 \verb`query`, 就是 \verb`?` 后面的东西。
\end{itemize}
