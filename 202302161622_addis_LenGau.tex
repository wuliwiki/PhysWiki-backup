% 长度规范
% 偶极子近似|薛定谔方程|库仑规范

\pentry{偶极子近似(量子)\upref{DipApr}, 库仑规范(量子)\upref{CouGau}}

本文使用原子单位制\upref{AU}。 我们只在使用偶极子近似\upref{DipApr}下讨论\textbf{长度规范(length gauge)}, 因为我们接下来需要矢势 $\bvec A(t)$ 与位置无关。 当空间中存在静止的电荷分布时, 我们可以把标量势能分为 $V(\bvec r) + \varphi(t)$ 两部分。 前者由静止电荷根据库仑定律计算, 不参与规范变换, 在这里我们甚至可以不把它看成电磁力而只是某种其他势能。 令不含时哈密顿算符为
\begin{equation}
H_0 = \frac{\bvec p^2}{2m} + qV(\bvec r)
\end{equation}
其中 $\bvec p$ 是所选规范下的广义动量算符(\autoref{QMEM_eq6}~\upref{QMEM})
\begin{equation}
\bvec p = m \bvec v + q\bvec A(t) = -\I \grad
\end{equation}
\addTODO{以上这段论述应该放在偶极子近似里面}

令 $\bvec A_C, \varphi_C, \Psi_C$ 代表库仑规范\upref{CouGau}, $\bvec A_L, \varphi_L, \Psi_L$ 代表长度规范。 将后者代入与规范无关的哈密顿量(\autoref{QMEM_eq2}~\upref{QMEM})得
\begin{equation}\label{LenGau_eq2}
H_L = H_0 - \frac{q}{2m} (\bvec A_L \vdot \bvec p + \bvec p \vdot \bvec A_L)
+ \frac{q^2}{2m} \bvec A_L^2 + q \varphi_L
\end{equation}


长度规范的思路是: 如果使 $\bvec A_L \equiv 0$, 就可以简化该式。 用不带撇的变量表示库仑规范, 我们令\autoref{QMEM_eq5}~\upref{QMEM}和\autoref{QMEM_eq3}~\upref{QMEM}中
\begin{equation}\label{LenGau_eq1}
\Psi_C(\bvec r, t) = \exp(\I q\chi_L)\Psi_L(\bvec r, t)
\end{equation}
\begin{equation}
\chi_L(\bvec r, t) = \bvec A_C(t) \vdot \bvec r
\end{equation}
再利用 $-\pdv*{\bvec A_C}{t} = \bvec {\mathcal E}(t)$, ($\bvec {\mathcal E}$ 是除库仑电场以外的含时电场)(库仑规范+偶极子近似,引用未完成)以及 $\varphi_C = 0$
\begin{equation}\label{LenGau_eq4}
\bvec A_L = \bvec A_C - \grad \chi_L = \bvec 0
\end{equation}
可见\textbf{长度规范下矢势为零}, 广义动量(\autoref{QMEM_eq6}~\upref{QMEM})变为普通动量
\begin{equation}\label{LenGau_eq6}
\bvec p_L = m \bvec v = -\I \grad
\end{equation}
再看标势的变换:
\begin{equation}\label{LenGau_eq5}
\varphi_L(t) = \varphi_C + \pdv{\chi_L}{t} = -\bvec {\mathcal E}(t) \vdot \bvec r
\end{equation}
由于形式不变, 把\autoref{LenGau_eq4} 和\autoref{LenGau_eq5} 代入\autoref{LenGau_eq2} 得长度规范下的哈密顿算符为
\begin{equation}\label{LenGau_eq7}
H_L = H_0 - q\bvec{\mathcal{E}}(t) \vdot \bvec r
\end{equation}
长度规范下的薛定谔方程为
\begin{equation}\label{LenGau_eq3}
H_L \Psi_L = \I \pdv{t} \Psi_L
\end{equation}
