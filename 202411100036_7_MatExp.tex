% 矩阵指数
% keys 矩阵|矩阵指数|本征值|特征子空间|特征分解
% license Xiao
% type Tutor

\begin{issues}
\issueMissDepend
\end{issues}

\subsection{定义}
实数域上的指数函数 $\E^x$ 可以进行Maclaurin展开:\begin{equation}
\E^x=\sum\limits_{n=0}^\infty \frac{x^n}{n!}~.
\end{equation}

展开式使得我们只需要用 $x$ 的幂就可以表示指数 $\E^x$。我们把这一点应用到矩阵中,就可以用方阵的幂来定义出矩阵的指数:

\begin{definition}{矩阵指数}
给定方阵 $\bvec{M}$,定义
\begin{equation}\label{eq_MatExp_2}
\E^{\bvec{M}}=\sum_{n=0}^\infty \frac{\bvec{M}^n}{n!}~.
\end{equation}
并称之为矩阵 $\bvec{M}$ 的\textbf{指数(matrix exponential)}。 其中对于任意方阵 $\bvec{M}$,都有 $\bvec{M}^0=\bvec{I}$,$\bvec{I}$ 是单位矩阵。
\end{definition}

矩阵指数在常微分方程中非常常用,是用来解线性齐次方程组的利器。一个矩阵的指数本身还是一个矩阵。

\subsection{矩阵指数的性质}

\subsubsection{相似变换的统一}

由\enref{过渡矩阵}{TransM}可知,如果矩阵 $\bvec{M}$ 在某基下表示一个线性变换,那么当基按过渡矩阵 $\bvec{Q}$ 改变时,同一个线性变换的矩阵表示就变为 $\bvec{Q}^{-1}\bvec{M}\bvec{Q}$。在原基下,$\E^{\bvec{M}}$ 可以表示另一个线性变换,而它在 $\bvec{Q}$ 下的变换是
\begin{equation}\label{eq_MatExp_1}
\bvec{Q}^{-1}\E^{\bvec{M}}\bvec{Q}=\E^{\bvec{Q}^{-1}\bvec{M}\bvec{Q}}~.
\end{equation}

也就是说,$\E^{\bvec{M}}$ 所表示的变换,在基变换的时候,其矩阵表示的变换相当于给 $\bvec{M}$ 变换后再取矩阵指数。这意味着我们也可以定义线性变换的指数——也可以反过来说,这是因为我们可以定义线性变换的指数,方式也是使用Maclaulin级数。

事实上,如果 $\mathcal{T}_i$ 表示若干线性变换,我们可以用映射的复合来定义线性变换的乘法:那么对于任意向量 $\bvec{v}$,$\mathcal{T}^n_i(\bvec{v})=\mathcal{T}_i(\mathcal{T}^{n-1}_i(\bvec{v}))$,其中 $\mathcal{T}_i^1=\mathcal{T}_i$。类似地,也可以定义线性变换的加法:$(\mathcal{T}_1+\mathcal{T}_2)(\bvec{v})=\mathcal{T}_1(\bvec{v})+\mathcal{T}_2(\bvec{v})$。这样,有了乘法和加法,就可以计算线性变换的级数了,而Maclaulin级数就可以定义为其指数:
\begin{equation}
\E^\mathcal{T}=\sum\limits_{n=0}^\infty \frac{\mathcal{T}^n}{n!}~.
\end{equation}
其中 $\mathcal{T}^0$ 是恒等变换,对应单位矩阵。

\autoref{eq_MatExp_1} 意味着,如果 $\bvec{M}$ 是 $\mathcal{T}$ 在某基下的矩阵表示,那么 $\E^\mathcal{T}$ 在该基下的矩阵表示就是 $\E^{\bvec{M}}$。

\subsubsection{运算性质}

设 $\bvec{M}, \bvec{N}\in \opn{gl}(n, \mathbb{F})$,$a, b\in\mathbb{F}$,则容易得出以下性质:

如果 $\bvec{MN}=\bvec{NM}$,那么我们有 $\E^{\bvec{M}}\E^{\bvec{N}}=\E^{\bvec{M}+\bvec{N}}$。

$\E^{(\bvec{M}\Tr)}=(\E^{\bvec{M}})\Tr$,$\E^{(\bvec{M}^\dagger)}=({\E^{\bvec{M}}})^\dagger$。

\begin{theorem}{矩阵指数的行列式与矩阵的迹}\label{the_MatExp_2}
对于 $\bvec{M}\in\opn{gl}(n, \mathbb{F})$,有 $\abs{\E^{\bvec{M}}}=\E^{\opn{tr}(\bvec{M})}$。即:矩阵指数的行列式,等于矩阵迹的指数。
\end{theorem}

\textbf{证明}:

我们只需要考虑上三角矩阵 $\bvec{M}$ 的情况即可,因为任何矩阵总可以通过相似变换变成上三角矩阵。此时,$\bvec{M}$ 的迹就是主对角元素之和,而 $\bvec{M}^k$ 的第 $i$ 个主对角元素都是 $\bvec{M}$ 的第 $i$ 个主对角元素的 $k$ 次方。

如果只看主对角元素,那么可以记 $\bvec{M}$ 为 $(m_1, m_2,\cdots,m_n)$,其中各 $m_i$ 是 $\bvec{M}$ 的第 $i$ 个元素。类似地,$\bvec{M}^k$ 就可以记为 $(m_1^k, m_2^k,\cdots,m_n^k)$。代入矩阵指数的定义式,可得 $\E^{\bvec{M}}$ 的对角线元素为 $(\E^m_1, \E^m_2,\cdots, \E^m_n)$。

由于上三角矩阵的乘积还是上三角矩阵,可知 $\E^{\bvec{M}}$ 是上三角矩阵,因此 $\abs{\E^{\bvec{M}}}=\E^m_1\times\E^m_2\times\cdots\times\E^m_n=\E^{\E^m_1+\E^m_2+\cdots+\E^m_n}=\E^{\opn{tr}(\bvec{M})}$。

\textbf{证毕}。





\begin{theorem}{矩阵指数求导}\label{the_MatExp_1}
矩阵 $\E^{\mat Mt}$ 是一个关于实变量 $t$ 的函数,则
\begin{equation}\label{eq_MatExp_3}
\frac{\dd}{\dd t}\E^{\mat Mt}=\mat M\E^{\mat Mt}~,
\end{equation}
其中求导定义为对每个矩阵元单独求导的结果。
\end{theorem}

\autoref{the_MatExp_1} 的形式和 $\frac{\dd}{\dd t}\E^{at}=a\E^{at}$ 是一样的,它们也共享同一个证明,我们留作习题:

\begin{exercise}{}
根据\autoref{eq_MatExp_2} 的定义,注意 $\mat M$ 是常数矩阵,证明\autoref{eq_MatExp_3}。
\end{exercise}





\subsection{对角化计算矩阵指数}



最容易计算指数的矩阵,是对角矩阵。由于两个对角矩阵相乘后还是对角矩阵,结果矩阵的第$i$个对角元就是两个矩阵的第$i$个对角元相乘,因此对于任意非负整数$k$有
\begin{equation}
\qty(\opn{diag}(a_1, a_2, \cdots, a_n))^k = \opn{diag}(a_1^k, a_2^k, \cdots, a_n^k)~. 
\end{equation}
于是易得
\begin{equation}
\E^{\opn{diag(a_1, a_2, \cdots, a_n)}} = \opn{diag}(\E^{a_1}, \E^{a_2}, \cdots, \E^{a_n})~. 
\end{equation}


如果矩阵不是对角的,则计算会麻烦很多。但是,如果矩阵能通过相似变换化为对角矩阵,那么根据\autoref{eq_MatExp_1} 即可大大简化计算。对于任意矩阵$\bvec{M}$,如果存在可逆矩阵$\bvec{Q}$使得$\bvec{Q}^{-1}\bvec{MQ}$是对角矩阵,那么$\E^{\bvec{Q}^{-1}\bvec{MQ}}$非常容易计算,从而容易计算出
\begin{equation}
\E^{\bvec{M}} = \bvec{Q}\E^{\bvec{Q}^{-1}\bvec{MQ}}\bvec{Q}^{-1}~. 
\end{equation}
利用可逆矩阵得到对角矩阵$\bvec{Q}^{-1}\bvec{MQ}$的过程,称为矩阵$\bvec{M}$的对角化。

并不是所有矩阵都能对角化,但复数域上的矩阵一定可以。我们可以利用特征方程来做对角化。求解特征方程
\begin{equation}
\det\qty(\bvec{M}-\lambda\bvec{E})=0~, 
\end{equation}
这个方程是关于$\lambda$的$n$次多项式方程,其中$n$是$\bvec{M}$的阶数。根据\enref{代数学基本定理}{BscAlg},此方程一定有$n$个解(计入重数),记为$\lambda_1, \cdots, \lambda_n$,它们就是$\bvec{M}$的特征值。接下来,求解线性方程组(向量的线性方程)
\begin{equation}
\bvec{M}\bvec{X} = \lambda_i\bvec{X}~, 
\end{equation}
每个这样的方程组一定有\textbf{非零解}(为什么?),这些解即为$\bvec{M}$的特征向量。每个非零解作为列向量,排成一排,所得的矩阵便是$\bvec{Q}$。


\begin{example}{复矩阵对角化}

给定复矩阵
\begin{equation}
\pmat{
    1&-1\\
    1&1
}~, 
\end{equation}
其特征方程为
\begin{equation}
(1-\lambda)(1-\lambda)+1=0~, 
\end{equation}
解为
\begin{equation}
\lambda_1 = 1-\I, \lambda_2 = 1+\I~. 
\end{equation}

对特征值$1-\I$求特征向量:
\begin{equation}
\pmat{
    1&-1\\
    1&1
}
\pmat{
    x\\y
}
=
(1-\I)
\pmat{
    x\\y
}~, 
\end{equation}
得到非零解
\begin{equation}
\pmat{

}~. 
\end{equation}
同理,对特征值$1+\I$求特征向量,得到非零解
\begin{equation}
\pmat
\end{equation}

\end{example}


由于实数是复数的子集,故实矩阵也可以在复数域上对角化,虽然对角化后的结果可能不再是实矩阵,但依然可以计算其矩阵指数后再做逆相似变换,还原回实矩阵。














