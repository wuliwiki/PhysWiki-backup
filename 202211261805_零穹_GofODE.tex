% 常微分方程的几何图像
% 向量场|单参数微分同胚群|微分方程|解

\begin{issues}
\issueTODO
\end{issues}

\pentry{相空间和相流\upref{PSaPF},}
这里,将从几何上来理解常微分方程.这需要先引入一些概念.
\subsection{微分同胚}
以下新名词的理解只需掌握函数可微性的知识.
\begin{definition}{可微函数}
设 $U$ 是矢量空间 $\mathbb R^n$ 上的区域,其上坐标为 $x_1,\cdots,x_n$,称函数
\begin{equation}
f:U\rightarrow\mathbb R
\end{equation}
是 $U$ 上的\textbf{可微函数},若 $f(x_1,\cdots,x_n)$ 是 $r$ 次连续可微的,此处,$1\leq r\leq\infty$.通常人们都不关心 $r$ 的具体值,因此并不指明.若有需要将指出“$r$次可微”或函数类 $C^r$.
\end{definition}
\begin{definition}{可微映射}
设 $U$ 是 $\mathbb R^n$ 中区域,$V$ 是 $\mathbb R^m$ 中区域,其中 $x_1,\cdots,x_n$ 是 $U$ 中的坐标,$y_1,\cdots,y_m$ 是 $V$ 中的坐标,称映射
\begin{equation}
f:U\rightarrow V,\quad f(x_1,\cdots,x_n)=(y_1,\cdots,y_m)
\end{equation}
为\textbf{可微映射},若 $y_i=f_i(x_1,\cdots,x_n)$ 是可微函数.其中 $1\leq i\leq m$.
\end{definition}
\begin{definition}{微分同胚}
若映射 $f:U\rightarrow V$ 是个双射,且 $f$ 和其逆 $f^{-1}$ 都是可微映射,则称 $f$ 为\textbf{微分同胚}.
\end{definition}

\begin{definition}{单参数微分同胚群}\label{GofODE_def1}
设 $M$ 是一流形(可认为是欧氏空间中一区域),若映射
\begin{equation}
g:\mathbb R\times M\rightarrow M,\quad g(t,x) =g^t x
\end{equation}
满足:1.$g$ 是可微映射;2.族 $\{g^t|t\in\mathbb R\}$ 是 $M$ 的单参数变换群(\autoref{PSaPF_def1}~\upref{PSaPF}).则称 $\{g^t|t\in\mathbb R\}$ 是 $M$ 的\textbf{单参数微分同胚群}.
\end{definition}
\begin{theorem}{}
若 $\{g^t|t\in\mathbb R\}$ 是单参数微分同胚群,则对每一 $t$,映射 $g^t$ 是微分同胚.
\end{theorem}
\textbf{证明:}因为 $g$ 是可微映射,且
\begin{equation}
\dv{g^tx}{x}=\pdv{g(t,x)}{x}
\end{equation}
所以 $g^t$ 对 $x$ 可微.又 $\{g^t\}$ 是单参数变换群,所以对每一 $t$,$g^t$ 是双射\upref{PSaPF},这意味着 $\dv{g^t x}{x}\neq 0$(因为否则 $g^t(x+\dd x)=g^t(x)$,这违反了 $g^t$ 的单射性).所以
\begin{equation}
\dv{{g^{t}}^{-1}}{x}=\frac{1}{\dv{g^t x}{x}}
\end{equation}
可微.

\textbf{证毕!}
\subsection{向量场}
常微分方程研究的过程是可微过程,即要求相空间 $M$ 是微分流形,状态随时间的变化由可微函数描述.由于状态x经过t时间后变为 $g^tx$,要这个有可微函数描述的映射就是 $g(t,x)=g^tx$,由\autoref{GofODE_def1} ,$\{g^t|t\in\mathbb R\}$ 就是 $M$ 的单参数微分同胚群.因此,以后提到相流 $\{M,\{g^t\}\}$ ,都有将 $\{g^t\}$ 理解成单参数微分同胚群.

掌握了以上概念,下面的概念是容易理解的.
\begin{definition}{相速度}
称
\begin{equation}
v(x):=\left.\dv{}{t}\right|_{t=\tau} g^tx
\end{equation}
为相流 $g^t$ 在点 $x\in M$ 的\textbf{相速度}.
\end{definition}
由于 $M$ 可看成欧几里得空间中的区域,所以其中的点就有坐标,因此 $\dd g^t x$ 有定义,上面的 $v(x)$ 也就有定义.

\subsubsection{两曲线的相切}
同样令 $U$ 是 $\mathbb R^n$ 中一区域,其坐标为 $x_1,\cdots,x_n$,每个 $x_i$ 其实都相当于将 $U$ 映射到 $\mathbb R$ 的映射
\begin{equation}
x_i:U\rightarrow\mathbb R.
\end{equation}
$U$ 上的曲线就是 $t$ 轴上一区间 $I$ 到 $U$ 的连续映射 $\varphi$,通常我们设它连续可微.
\begin{definition}{速度相量}
设 $\varphi$ 是 $U$ 的曲线且 $\varphi(0)=x$,则
\begin{equation}
v=\left.\dv{}{t}\right|_{t=0}(x\circ\varphi)
\end{equation}
称为曲线在点 $x$ 处的\textbf{速度向量}.其分量为 $v_i=\left.\dv{}{t}\right|_{t=0}(x_i\circ\varphi)$
\end{definition}
\begin{definition}{曲线的相切}
若两曲线 $\varphi_1,\varphi_2$ 经过同一点 $x=\varphi_1(0)=\varphi_2(0)$.如果对 $t\rightarrow0$ 时,点 $\varphi_1(t),\varphi_2(t)$ 之间距离是比 $t$ 高阶的无穷小 $o(t)$,则称它们在点 $x$ \textbf{相切}.
\end{definition}

\begin{theorem}{}
两曲线 $\varphi_1,\varphi_2$ 在点 $x$ 处相切,当且仅当它们的速度向量相同.
\end{theorem}
\textbf{证明:}将 $x\circ\varphi_1(t)-x\circ\varphi_2(t)$ 在 $t=0$ 处展开:
\begin{equation}
x\circ\varphi_1(t)-x\circ\varphi_2(t)=\sum_{i=1}\frac{1}{i!}\qty(\left.\dv[i]{}{t}\right|_{t=0}\qty(x\circ\varphi_1-x\circ\varphi_2))t^i
\end{equation}
要使 $\abs{x\circ\varphi_1(t)-x\circ\varphi_2(t)}=o(t)$ 上式就不能含 $t$ 的一次项,于是只能
\begin{equation}
\left.\dv{}{t}\right|_{t=0}x\circ\varphi_1=\left.\dv{}{t}\right|_{t=0}x\circ\varphi_2
\end{equation}

\textbf{证毕!}

这一定理意味着相切的曲线对应同一速度向量,而在一点 $x_0$ 处都有一速度矢量与任一矢量 $v$ 对应,只需这样定义曲线 $\varphi$(注意下面 $x(x_0)$ 代表 $x_0$ 的坐标).
\begin{equation}
x\circ\varphi(t)=\int_0^t v\dd t+x(x_0)
\end{equation}
也就是说在\textbf{点 $x$ 的速度矢量 $v$ 相当于过 $x$ 点的曲线的等价类},该等价类中的任一曲线在点 $x$ 的速度矢量都是 $v$.所以有下面的定义
\begin{definition}{切向量}
称过点 $x\in U$ 的曲线的等价类为点 $x$ 的\textbf{切向量}.
\end{definition}

由于 $n$ 维空间中的区域 $U$ 上的一点处所有的速度矢量构成一 $n$ 维的矢量空间,所以在 $x$ 点的所有切向量构成的几何就是一个 $n$ 维的矢量空间.
\begin{definition}{切空间}
在点 $x\in U$ 处的所有切向量构成的集合称为 $U$ 在 $x$ 的\textbf{切空间},记作 $TU_x$
\end{definition}

我们说过,$n$ 维流形 $M$ 相当于 $\mathbb R^n$ 中一区域,那么在它上面任一点 $x$ 都有一切空间 $TM_x$.
\begin{definition}{切丛}
流形 $M$ 上所有点的切空间构成的集合
\begin{equation}
TM=\bigcup_{x\in U}TM_x
\end{equation}
称为流形 $M$ 的\textbf{切丛}.
\end{definition} 
流形 $M$ 的向量场就是说对 $M$ 上的每一点都有一向量与之对应,每一点处的向量显然是该点的切向量,它是在 该点的切空间 $TM_x$ 中的,而另一点的向量又在另一点的切空间中的,那么向量场就相当于一个映射 $v$,它把 $x\in M$ 映到 $v(x)\in TM_x$,由于向量场 $v$ 定义域在 $M$ ,它的像要 $v$ 取遍 $M$,那么像所在的集合就是
$\bigcup_{x\in M} TM_x$ .这就是说 $M$ 的向量场相当于是流形 $M$ 到其切丛 $TM$ 的映射.一般情况,我们还要求它是可微映射.
\begin{definition}{向量场}
设 $M$ 是一 $n$ 维流形($\mathbb R^n$ 中的一区域) ,映射
\begin{equation}
v:M\rightarrow\mathbb TM
\end{equation}
是可微映射,则称 $v$ 是 $M$ 的一个\textbf{向量场}.它由 $n$ 个可微函数 $v_i:M\rightarrow \mathbb R$ 所指定. 
\end{definition}
\begin{definition}{奇点}
设 $v$ 是 $M$ 的向量场,且 $v(x)=0$,则称点 $x$ 为向量场 $v$ 的\textbf{奇点}.
\end{definition}
\subsection{微分方程}
\begin{definition}{微分方程}
设 $U$ 是 $\mathbb R^n$ 中一区域,$v$ 是 $U$ 上的向量场,则称方程
\begin{equation}
\dot x=v(x),\quad x\in U
\end{equation}
是由向量场 $v$ 确定的\textbf{微分方程}.$U$ 称为 $$ 
\end{definition}


