% 辗转相除法
% 互素|最大公因式|算法|辗转相除

\pentry{多项式的整除\upref{ExDiv},带余除法\upref{DivAlg}}
本节将证明最大公因式的存在性,并介绍最大公因式的算法程序——辗转相除法.简单来说,所谓的\textbf{辗转相除法}是指多次利用带余除法计算两个多项式的最大公因式的算法.

为证明最大公因式的存在,我们先来证明下面一个定理.
\begin{theorem}{}\label{SucDiv_the1}
设 $f(x),g(x),q(x),r(x)\in\mathbb{F}[x]$,并且
\begin{equation}\label{SucDiv_eq1}
f(x)=q(x)g(x)+r(x)
\end{equation}
则 $f(x),g(x)$ 与 $g(x),r(x)$ 有相同的公因式.
\end{theorem}
\textbf{证明:}1.$\phi(x)|g(x),\phi(x)|f(x)\Rightarrow \phi(x)|g(x),\phi(x)|r(x)$ 的证明:

因为$\phi(x)|g(x),\phi(x)|f(x)$ ,由整除的性质3多项式的整除\upref{ExDiv} ,$\phi(x)$ 必整除 $f(x),g(x)$ 的组合
\begin{equation}
\phi(x)|f(x)-q(x)g(x)\Rightarrow \phi(x)|r(x)
\end{equation}

2.$\phi(x)|g(x),\phi(x)|r(x)\Rightarrow\phi(x)|g(x),\phi(x)|f(x)$ 的证明:同样有整除的性质3,立即证得.

下面利用\autoref{SucDiv_eq1} 来证明最大公因式的存在性,该证明方法便是辗转相除法.
\begin{theorem}{最大公因式存在定理}
设 $f(x),g(x)$ 是 $\mathbb{F}[x]$ 中的两个多项式,则在 $\mathbb{F}[x]$ 中, $f(x)$ 与 $g(x)$ 的最大公因式 $d(x)$ 存在,且有 $\mathbb{F}[x]$ 中的两个多项式 $u(x),v(x)$ 使得
\begin{equation}\label{SucDiv_eq2}
d(x)=u(x)f(x)+v(x)g(x)
\end{equation}

\end{theorem}
\textbf{证明:}如果 $f(x)$ 和 $g(x)$ 中有一个为0,比如g(x)=0,那么另一个 $f(x)$ 就是 $f(x)$ 与 $g(x)$ 的最大公因式,且 $f(x)=1\cdot f(x)+1\cdot0$.

以下只需考虑 $g(x)\neq 0$.由带余除法\upref{DivAlg} ,有
\begin{equation}
f(x)=q_1(x)g(x)+r_1(x)
\end{equation}
只考虑 $r_1(x)\neq 0$,否则 $g(x)$ 便是最大公因式.再用 $r_1(x)$ 除 $g(x)$,得
\begin{equation}
g(x)=q_2(x)r_1(x)+r_2(x)
\end{equation}
同样只考虑 $r_2(x)\neq0$,再用 $r_2(x)$ 除 $r_1(x)$,得
\begin{equation}
r_1(x)=q_3(x)r_2(x)+r_3(x)
\end{equation}
如此不断重复该过程,所得余式的次数就不断降低,即
\begin{equation}
\mathrm{deg}\;g(x)>\mathrm{deg}\;r_1(x)>\mathrm{deg}\;r_2(x)>\cdots
\end{equation}
这样经过有限次相除之后,必然有余式为0.于是得到一等式:
\begin{equation}\label{SucDiv_eq3}
\begin{aligned}
f(x)&=q_1(x)g(x)+r_1(x)\\
g(x)&=q_2(x)r_1(x)+r_2(x)\\
r_1(x)&=q_3(x)r_2(x)+r_3(x)\\
&\vdots\\
r_{i-2}(x)&=q_i(x)r_{i-1}(x)+r_i(x)\\
&\vdots\\
r_{s-2}(x)&=q_s(x)r_{s-1}(x)+r_s(x)\\
r_{s-1}(x)&=q_{s+1}(x)r_s(x)
\end{aligned}
\end{equation}
其中 $r_s(x)$ 是 $r_s(x)$ 与 $r_{s-1}(x)$ 的一个最大公因式.由\autoref{SucDiv_the1} ,$r_s(x)$ 也是 $r_{s-1}(x)$ 与 $r_{s-2}(x)$ 的一个最大公因式.照此逐步推上去,便有 $r_s(x)$ 是 $f(x)$ 和 $g(x)$ 的一个最大公因式.这就证明了最大公因式的存在性.

下面来证明\autoref{SucDiv_eq2} .由