% test

例
1. (温州中学期末卷第 10 题) 非负实数列 $\left\{a_{n}\right\}$ 前 $n$ 项和为 $S_{n}\left(S_{n}>0\right) .$ 若分别记 $\left\{n^{2} a_{n}\right\}$
与 $\left\{\frac{a_{n}}{n^{2}}\right\}$ 前 $n$ 项和为 $T_{n}$ 与 $R_{n},$ 求 $\frac{T_{5} R_{5}}{S_{5}^{2}}$ 的最小值.
例
2. (浙江名校仿真卷) 已知 $x, y \geq 0, x+y \leq 1,$ 求 $4 x^{2}+4 y^{2}+(1-x-y)^{2}$ 的最小值.
例
3. (昨日一个同学问我的题目 ) 已知 $a+b+c+d=1,$ 求 $8 a^{2}+3 b^{2}+2 c^{2}-d^{2}$ 的最小值.
例

1. (温州中学期末卷第 10 题)
我们用到五元 Cauchy 不等式:
对于非负实数 $a_{i}, b_{i}(i=1,2, \cdots, n)$,
等号在 $a_{1}>0, a_{2}=a_{3}=\cdots=a_{5}=0$ 时取到.
例 2. (浙江名校仿真卷)
我们用到三元 Cauchy 不等式:
对于非负实数 $a_{i}, b_{i}(i=1,2, \cdots, n)$,
$$
\begin{array}{c}
\left(a_{1}+a_{2}+\cdots+a_{n}\right)\left(b_{1}+b_{2}+\cdots+b_{n}\right) \geq\left(\sqrt{a_{1} b_{1}}+\sqrt{a_{2} b_{2}}+\cdots+\sqrt{a_{n} b_{n}}\right)^{2}, n=3 \\
\text { 由于 } x, y \geq 0,1-x-y \geq 0, \text { 则 }
\end{array}
$$
$\left[4 x^{2}+4 y^{2}+(1-x-y)^{2}\right]\left[\frac{1}{4}+\frac{1}{4}+1\right] \geq\left(\sqrt{4 x^{2} \cdot \frac{1}{4}}+\sqrt{4 y^{2} \cdot \frac{1}{4}}+\sqrt{(1-x-y)^{2}}\right)^{2}=1 \Rightarrow 4 x^{2}+4 y^{2}+(1-x-y)^{2} \geq \frac{2}{3}$
等号在 $4 x=4 y=1-x-y$ 即 $x=y=\frac{1}{6}$ 时取到
[t. (原答案给出的是术造立体平面做的,显然 Cauchy 不等式邦我们减小了计算量

例
3. (詐日一个同学问我的题目) 已知 $a+b+c+d=1,$ 求 $8 a^{2}+3 b^{2}+2 c^{2}-d^{2}$ 的最小值.
我们用到三元 Cauchy 不等式:
对于非负实数 $a_{i}, b_{i}(i=1,2, \cdots, n)$,
$$
\begin{array}{c}
a_{1}+a_{2}+\cdots+a_{n} \geq \frac{\left(\sqrt{a_{1} b_{1}}+\sqrt{a_{2} b_{2}}+\cdots+\sqrt{a_{n} b_{n}}\right)^{2}}{b_{1}+b_{2}+\cdots+b_{n}}, n=3 \\
8 a^{2}+3 b^{2}+2 c^{2}-d^{2} \geq \frac{(a+b+c)^{2}}{\frac{1}{8}+\frac{1}{3}+\frac{1}{2}}-d^{2}=\frac{24}{23}(1-d)^{2}-d^{2} \geq-24(\text { 二次函数 })
\end{array}
$$
等号在 $a=-3, b=-8, c=-12, d=24$ 时取到. 3

二元柯每不等式:
$$
\left(a^{2}+b^{2}\right)\left(x^{2}+y^{2}\right) \geq(a x+b y)^{2}
$$
三元柯每不等式:
$$
\left(a^{2}+b^{2}+c^{2}\right)\left(x^{2}+y^{2}+z^{2}\right) \geq(a x+b y+c z)^{2}
$$
n 元柯西不等式:
$\left(a_{1}^{2}+a_{2}^{2}+\cdots+a_{n}^{2}\right)\left(b_{1}^{2}+b_{2}^{2}+\cdots+b_{n}^{2}\right) \geq\left(a_{1} b_{1}+a_{2} b_{2}+\cdots+a_{n} b_{n}\right)^{2}$
取等条件: $\frac{a_{1}}{b_{1}}=\frac{a_{2}}{b_{2}}=\cdots=\frac{a_{n}}{b_{n}}\left(b_{i} \neq 0\right)$

设 n 维空间中的两个向量 $\vec{\alpha}=\left(a_{1}, a_{2}, \cdots, a_{n}\right), \vec{\beta}=\left(b_{1}, b_{2}, \cdots, b_{n}\right)$,
于是由 $|\vec{\alpha} \cdot \vec{\beta}| \leq|\vec{\alpha}||\vec{\beta}| \Rightarrow\left|a_{1} b_{1}+a_{2} b_{2}+\cdots+a_{n} b_{n}\right| \leq \sqrt{a_{1}^{2}+a_{2}^{2}+\cdots+a_{n}^{2}} \sqrt{b_{1}^{2}+b_{2}^{2}+\cdots+b_{n}^{2}}$
$\Rightarrow\left(a_{1}^{2}+a_{2}^{2}+\cdots+a_{n}^{2}\right)\left(b_{1}^{2}+b_{2}^{2}+\cdots+b_{n}^{2}\right) \geq\left(a_{1} b_{1}+a_{2} b_{2}+\cdots+a_{n} b_{n}\right)^{2}$
其中等号在 $\vec{\alpha}=\left(a_{1}, a_{2}, \cdots, a_{n}\right), \vec{\beta}=\left(b_{1}, b_{2}, \cdots, b_{n}\right)$ 共线, 即 $\frac{a_{1}}{b_{1}}=\frac{a_{2}}{b_{2}}=\cdots=\frac{a_{n}}{b_{n}}\left(b_{i} \neq 0\right)$ 时取到.


