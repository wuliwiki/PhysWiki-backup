% 半双线性形式

\begin{issues}
\issueDraft
\end{issues}

\footnote{参考 Wikipedia \href{https://en.wikipedia.org/wiki/Sesquilinear_form}{相关页面}.}\textbf{半双线性形式(sequilinear form)}是双线性映射的一个变形. 它关于第一个变量是反线性(也叫共轭线性 conjugate linear)的, 关于第二个变量是线性的.
\begin{definition}{}
复数域上的线性空间 $V$ 中, 若映射 $f:V\times V\to \mathbb C$ 对任意 $u, v, w\in V$, $a,b\in \mathbb C$ 满足
\begin{equation}\label{sequil_eq2}
f(au+bv, w) = a^*f(u, w) + b^*f(v, w)
\end{equation}
\begin{equation}\label{sequil_eq1}
f(u, av+bw) = af(u, v) + bf(u, w)
\end{equation}
其中 $a^*$ 是 $a$ 的复共轭\upref{CplxNo}, 那么就说该映射是\textbf{半双线性的(sequilinear)}.
\end{definition}
如果满足 $f(u, v) = f(v, u)^*$, 就说它是\textbf{对称的}.

可以发现, 若把\autoref{sequil_eq2} 中的共轭符号去掉, 那么该定义就是双线性的定义(\autoref{Tensor_eq2}~\upref{Tensor}).

任意半双线性形式 $f(u, v)$ 可以唯一地由 $g(v) = f(v, v)$ 确定:
\begin{equation}
f(u, v) =\frac{1}{2}[g(u+v)-g(u)-g(v)]
-\frac{\I}{2}[g(u+\I v)-g(u)-g(\I v)]
\end{equation}
这叫做\textbf{极化恒等式(polarization identity)}. 这可以由定义直接证明.

如果 $f(u, v)$ 是半双线性形, 那么就把 $g(v) = f(v, v)$ 叫做\textbf{二次型(quadratic form)}.
