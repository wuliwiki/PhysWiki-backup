% 赫尔曼·外尔(综述)
% license CCBYSA3
% type Wiki

本文根据 CC-BY-SA 协议转载翻译自维基百科\href{https://en.wikipedia.org/wiki/Hermann_Weyl#Weyl_equation}{相关文章}。

\begin{figure}[ht]
\centering
\includegraphics[width=6cm]{./figures/b6bfe45a42d757ab.png}
\caption{赫尔曼·克劳斯·雨果·外尔  1885年11月9日  德国帝国,埃尔姆斯霍恩} \label{fig_WR_1}
\end{figure}
赫尔曼·克劳斯·雨果·外尔(Hermann Klaus Hugo Weyl,ForMemRS,德语发音:[vaɪl];1885年11月9日-1955年12月8日)是一位德国数学家、理论物理学家、逻辑学家和哲学家。尽管他的大部分职业生涯是在瑞士苏黎世和美国新泽西州普林斯顿度过的,他依然被认为是哥廷根大学数学传统的一部分,该传统由卡尔·弗里德里希·高斯、大卫·希尔伯特和赫尔曼·闵可夫斯基代表。

外尔的研究在理论物理学和纯数学领域(例如数论)具有重要意义。他是20世纪最具影响力的数学家之一,也是早期普林斯顿高等研究院的重要成员。[4][5]

外尔在空间、时间、物质、哲学、逻辑、对称性以及数学史等领域作出了非凡的贡献。他是最早设想将广义相对论与电磁学定律结合起来的人之一。弗里曼·戴森(Freeman Dyson)曾写道,外尔是唯一一个可以与19世纪“最后的伟大通才数学家”亨利·庞加莱和大卫·希尔伯特相提并论的人。迈克尔·阿蒂亚(Michael Atiyah)特别指出,每当他研究一个数学主题时,总会发现外尔早已涉足其中。[7]

\subsection{传记}
赫尔曼·外尔出生于德国汉堡附近的小镇埃尔姆斯霍恩,曾就读于阿尔托纳的克里斯蒂亚内乌姆文理中学(Gymnasium Christianeum)。[8] 他的父亲路德维希·外尔(Ludwig Weyl)是一名银行家,母亲安娜·外尔(Anna Weyl,娘家姓Dieck)则来自一个富裕家庭。[9]

1904年至1908年间,外尔在哥廷根大学和慕尼黑大学学习数学和物理。他在哥廷根大学获得博士学位,导师是他非常敬仰的大卫·希尔伯特。

1913年9月,外尔在哥廷根与弗里德里克·贝尔塔·海伦·约瑟夫(Friederike Bertha Helene Joseph,1893年3月30日-1948年9月5日)结婚,她的昵称是“海拉”(Hella)。海伦是布鲁诺·约瑟夫博士(Dr. Bruno Joseph,1861年12月13日-1934年6月10日)的女儿,后者是一名医生,在德国里布尼茨-达姆加滕(Ribnitz-Damgarten)担任卫生官职(Sanitätsrat)。海伦是一位哲学家(现象学家埃德蒙·胡塞尔的弟子),也是西班牙文学作品的翻译家,尤其将西班牙哲学家何塞·奥尔特加·伊·加塞特的作品译成德文和英文。[12] 正是通过海伦与胡塞尔的密切联系,赫尔曼得以熟悉并深受胡塞尔思想的影响。

赫尔曼与海伦有两个儿子:弗里茨·约阿希姆·外尔(Fritz Joachim Weyl,1915年2月19日-1977年7月20日)和迈克尔·外尔(Michael Weyl,1917年9月15日-2011年3月19日),两人均出生于瑞士苏黎世。[13] 海伦于1948年9月5日在新泽西州普林斯顿去世。同年9月9日,普林斯顿为她举行了纪念仪式,致辞者包括她的儿子弗里茨·约阿希姆·外尔,以及数学家奥斯瓦尔德·维布伦(Oswald Veblen)和理查德·柯朗特(Richard Courant)。[14]

1950年,赫尔曼与雕塑家埃伦·贝尔(Ellen Bär,娘家姓Lohnstein,1902年4月17日-1988年7月14日)结婚。埃伦是苏黎世教授理查德·约瑟夫·贝尔(Richard Josef Bär,1892年9月11日-1940年12月15日)的遗孀。[15][16]

在哥廷根担任教职数年后,外尔于1913年前往苏黎世,担任苏黎世联邦理工学院(ETH Zürich)的数学教授[17],在那里他与阿尔伯特·爱因斯坦成为同事。当时爱因斯坦正在完善广义相对论的细节。爱因斯坦对外尔产生了深远的影响,使他对数学物理产生了浓厚的兴趣。1921年,外尔结识了理论物理学家埃尔温·薛定谔,后者当时是苏黎世大学的教授。他们后来成为亲密的朋友。外尔曾与薛定谔的妻子安妮玛丽(Annemarie,昵称Anny,娘家姓Bertel)发生过某种形式的无子女恋情,而同时安妮玛丽正在抚养薛定谔与他人所生的非婚生女儿露丝·乔吉·埃里卡·马奇(Ruth Georgie Erica March),露丝于1934年出生在英国牛津。[18][19]

外尔在1928年国际数学家大会(ICM)上作为主旨发言人于意大利博洛尼亚发表演讲,[20] 并在1936年于奥斯陆召开的国际数学家大会上担任特邀发言人。他于1928年被选为美国物理学会会士,[21] 1929年成为美国艺术与科学院院士,[22] 1935年加入美国哲学会,[23] 1940年成为美国国家科学院院士。[24] 在1928-1929学年,他作为访问教授在普林斯顿大学任教,[25] 并与霍华德·P·罗伯逊(Howard P. Robertson)合作撰写了一篇题为《关于无限小几何基础中群论问题》的论文。[26]

1930年,外尔离开苏黎世,接替大卫·希尔伯特在哥廷根的职位。然而,随着1933年纳粹上台,他因妻子是犹太人而离开德国。他曾被新成立的美国新泽西州普林斯顿高等研究院邀请担任教职,但因不愿离开祖国而拒绝。随着德国政治局势恶化,他改变了主意,接受了再次提供的职位,并在高等研究院工作至1951年退休。他与第二任妻子埃伦(Ellen)共同度过了在普林斯顿和苏黎世的时光。1955年12月8日,外尔在苏黎世因心脏病发作去世。

外尔于1955年12月12日在苏黎世火化。[27] 他的骨灰一直保存在私人手中,直到1999年,之后被安葬在普林斯顿公墓的一个户外骨灰存放墙中。[28] 外尔的儿子迈克尔·外尔(1917–2011)的遗体被安葬在同一骨灰存放墙中,与外尔的骨灰相邻。

外尔是一位泛神论者。[29]
\subsection{贡献}
\subsubsection{特征值的分布 } 
\begin{figure}[ht]
\centering
\includegraphics[width=6cm]{./figures/9d4106519803831f.png}
\caption{赫尔曼·魏尔(左)和恩斯特·佩施尔(右)} \label{fig_WR_2}
\end{figure}
1911年,魏尔发表了《关于特征值的渐近分布》(*Über die asymptotische Verteilung der Eigenwerte*),在其中他证明了拉普拉斯算子在紧致域中的特征值遵循所谓的魏尔定律的分布。1912年,他提出了一种基于变分原理的新证明方法。魏尔多次回到这一主题,研究了弹性系统并提出了魏尔猜想。这些研究开启了现代分析的一个重要领域——特征值的渐近分布。
\subsubsection{流形与物理的几何基础}
1913年,魏尔发表了《黎曼曲面的概念》(*Die Idee der Riemannschen Fläche*),对黎曼曲面进行了统一的研究。在这部著作中,魏尔利用点集拓扑的方法,使黎曼曲面理论更加严谨,这种方法后来成为流形研究的模型。他还吸收了L.E.J.布劳威尔早期的拓扑学研究成果,用于这一目的。  

作为哥廷根学派的重要人物,魏尔从一开始就完全掌握了爱因斯坦的研究工作。他在1918年出版的《空间、时间、物质》(*Raum, Zeit, Materie*)中跟踪了相对论物理的发展,该书在1922年出版了第四版。同样在1918年,他引入了规范(*gauge*)的概念,并首次提出了现在称为规范理论的例子。魏尔的规范理论是试图将电磁场和引力场建模为时空几何性质的一次不成功的尝试。在黎曼几何中,魏尔张量对于理解共形几何的本质具有重要意义。  

魏尔在物理学中的总体方法基于埃德蒙·胡塞尔的现象学哲学,特别是胡塞尔1913年的著作《纯粹现象学与现象学哲学的观念:第一卷:纯粹现象学的一般导论》(*Ideen zu einer reinen Phänomenologie und phänomenologischen Philosophie. Erstes Buch: Allgemeine Einführung in die reine Phänomenologie*)。胡塞尔强烈回应了戈特洛布·弗雷格对其第一部关于算术哲学的批评,并探讨了数学和其他结构的意义,这些结构被弗雷格区分为经验参照之外的内容。
\subsubsection{拓扑群、李群与表示理论}
从1923年到1938年,魏尔发展了紧致群的理论,采用矩阵表示的方法。在紧致李群的情况下,他证明了一个基本的特征公式。  

这些成果奠定了理解量子力学对称结构的基础,他将其建立在群论的框架上,其中包括旋量的研究。这与约翰·冯·诺依曼在很大程度上推动的量子力学数学公式化一起,形成了自1930年左右以来广为人知的处理方式。非紧致群及其表示,尤其是海森堡群,也在这一具体背景下得到了简化,他在1927年的魏尔量子化中提出了现存最好的经典物理与量子物理之间的桥梁。从那时起,得益于魏尔的阐述,李群和李代数不仅成为纯数学的主流领域,也成为理论物理的重要组成部分。  

他的著作《经典群》(*The Classical Groups*)重新审视了不变量理论。书中探讨了对称群、一般线性群、正交群和辛群,以及它们的不变量和表示理论的相关成果。
\subsubsection{调和分析与解析数论} 
魏尔还展示了如何在丢番图逼近中使用指数和,并提出了模 1 均匀分布的判据,这是解析数论中的一个基础性步骤。这一研究成果被应用于黎曼 ζ 函数以及加法数论,并被许多后续学者进一步发展。