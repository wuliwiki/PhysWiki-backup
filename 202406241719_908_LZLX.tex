% 量子力学
% license CCBYSA3
% type Wiki

(本文根据 CC-BY-SA 协议转载自原搜狗科学百科对英文维基百科的翻译)

量子力学(英语:quantum mechanics;或称量子论)是描述微观物质(原子、亚原子粒子)行为的物理学理论,量子力学是我们理解除万有引力之外的所有基本力(电磁相互作用、强相互作用、弱相互作用)的基础。

量子力学是许多物理学分支的基础,包括电磁学、粒子物理、凝聚态物理以及宇宙学的部分内容。量子力学也是化学键理论、结构生物学以及电子学等学科的基础。

量子力学主要是用来描述微观下的行为,所描述的粒子现象无法精确地以经典力学诠释。例如:根据哥本哈根诠释,一个粒子在被观测之前,不具有任何物理性质,然而被观测之后,依测量仪器而定,可能观测到其粒子性质,也可能观测到其波动性质,或者观测到一部分粒子性质一部分波动性质,此即波粒二象性。

量子力学始于20世纪初马克斯·普朗克和尼尔斯·玻尔的开创性工作,马克斯·玻恩于1924年创造了“量子力学”一词。因其成功的解释了经典力学无法解释的实验现象,并精确地预言了此后的一些发现,物理学界开始广泛接受这个新理论。量子力学早期的一个主要成就是成功地解释了波粒二象性,此术语源于亚原子粒子同时表现出粒子和波的特性。

\subsection{ 历史}

对光的波动性研究始于17世纪和18世纪,当时像罗伯特·胡克(Robert Hooke)、克里斯蒂安·惠更斯(Christiaan Huygens)和莱昂哈德·欧拉(Leonhard Euler)这样的科学家根据实验观察提出了光的波动说。[1] 1803年,英国博学家托马斯·杨( Thomas Young)在一篇题为 光和颜色的本质的论文中描述了他的著名的双缝实验。这个实验在人们普遍接受光的波动说教方面发挥了重要作用。

1838年,麦可·法拉第(Michael Faraday)发现了阴极射线。在此之后是1859年古斯塔夫·基尔霍夫(Gustav Kirchhoff)关于黑体辐射问题的陈述,1877年路德维希·玻尔兹曼(Ludwig Boltzmann)关于物理系统的能量状态可以是分立的设想,以及1900年马克斯·普朗克的量子假说。[2] 普朗克关于能量以分立的“量子”(或能量包)辐射和吸收的假设与观测到的黑体辐射模式精确匹配。

1896年,威廉·维恩凭(Wilhelm Wien)经验确定了黑体辐射的分配定律,[3] 以他的名字被称为维恩定律,从麦克斯韦方程组的角度考虑,路德维希·玻尔兹曼(Ludwig Boltzmann)也独立地得出了这个结果。然而,它只在高频部分有效,而低估了低频辐射。后来,普朗克利用玻尔兹曼对热力学的统计解释修正了这个模型,并提出了现在所谓的黑体辐射定律,这进一步推动了量子力学的发展。

继马克斯·普朗克于1900年解决了黑体辐射问题(1859年报道)之后,阿尔伯特·爱因斯坦提出了一个基于量子的理论来解释光电效应(1905年,1887年报道)的想法。 大约在1900-1910年,原子理论和光的微粒理论[4] 第一次被广泛接受为科学事实;后一种理论可以分别视为物质的量子理论和电磁辐射的量子理论。

最先研究自然界量子现象的是阿瑟·康普顿(Arthur Compton)、钱德拉塞卡拉·拉曼( C. V. Raman)和皮特·塞曼(Pieter Zeeman),他们每个人都有以他们名字命名的量子效应。罗伯特·安德鲁·密立根(Robert Andrews Millikan)通过实验研究了光电效应,阿尔伯特·爱因斯坦为此发展了一套理论。与此同时,欧内斯特·卢瑟福(Ernest Rutherford)通过实验发现了原子的核模型,为此,尼尔斯·玻尔(Niels Bohr)发展了他的原子结构理论,后来得到了亨利·莫塞莱(Henry Moseley)实验的证实。1913年,彼得·约瑟夫·威廉·德拜(Peter Debye)进一步发展了尼尔斯·玻尔的原子结构理论,引入了椭圆轨道,这一概念也是由阿诺·索末菲(Arnold Sommerfeld)提出的。[5] 这个阶段被称为旧量子论。

\begin{figure}[ht]
\centering
\includegraphics[width=6cm]{./figures/247a720099b3cf2f.png}
\caption{马克斯·普朗克被认为是量子理论之父} \label{fig_LZLX_1}
\end{figure}
根据普朗克的理论,每个能量元($E$)与其频率成比例($\nu$):

$E = h\nu$ ,

其中 h 是普朗克常数。

普朗克谨慎地坚持认为,这只是辐射吸收和发射过程的一个方面,与辐射本身的物理现实无关。[6] 事实上,他认为他的量子假说只是一个获得正确答案的数学把戏,而不是一个相当大的发现。[7] 然而,在1905年,阿尔伯特·爱因斯坦现实地解释了普朗克的量子假说,并用它来解释了光电效应,即光照射在某些材料上从材料中射出电子。他因这项工作获得了1921年诺贝尔物理学奖。

爱因斯坦进一步发展了这一思想,表明电磁波(如光)也可以被描述为粒子(后来称为光子),其能量的分立量子数取决于其频率。[8]

量子力学的基础是在20世纪上半叶由马克斯·普朗克、尼尔斯·玻尔、维尔纳·海森堡、路易·德布罗意、阿瑟·康普顿、阿尔伯特·爱因斯坦、埃尔温·薛定谔、梅克斯·玻恩、约翰·冯·诺依曼、保罗·狄拉克、恩里科·费米、沃尔夫冈·泡利、马克斯·冯·劳厄、弗里曼·戴森、大卫·希尔伯特、威廉·维恩、萨特延德拉·纳特·玻色、阿诺德·索末菲等人建立。尼尔斯·玻尔的哥本哈根诠释被广泛接受。

\begin{figure}[ht]
\centering
\includegraphics[width=10cm]{./figures/e14c29ad473067b4.png}
\caption{1927在布鲁塞尔举行的索尔维会议} \label{fig_LZLX_2}
\end{figure}

20世纪20年代中期,量子力学的发展促使它成为原子物理学的标准表述。1925年夏天,玻尔和海森堡发表的研究成果终结了旧的量子理论。在某些过程和测量中,光的量子表现出类似粒子的特性,因此光的量子被称为光子(1926)。1926年,埃尔温·薛定谔提出了电子等粒子波函数的偏微分方程。当被限制在有限区域时,这个方程只能得到特定的解,对应于分立的量子态 -这与矩阵力学所给出的结果相同。[9] 从爱因斯坦的简单假设中,诞生了一系列的辩论、理论和测试。因此,整个量子物理领域呈现在大众面前,并且在1927年的第五届索尔维会议会议上被广泛接受。

人们发现,亚原子粒子和电磁波既不是简单的粒子也不是波,而是各有特定的性质。这就产生了波粒二象性的概念。

到了1930年,量子力学在戴维·希尔伯特、保罗·狄拉克和约翰·冯·诺依曼的工作下得到进一步的统一和形式化[10] 更加强调测量、我们对现实认识的统计本质以及对“观察者”的哲学推测。此后,它渗透到许多学科,包括量子化学、量子电子学、量子光学和量子信息科学。它的现代理论发展包括弦理论和量子引力理论。它还为现代元素周期表的许多特征提供了一个有用的框架,并描述了原子在化学键形成过程中的行为和计算机半导体中的电子流,因此在许多现代技术中起着至关重要的作用。

虽然量子力学被构建来描述非常小的世界,但它也能解释一些宏观现象,如超导体,[11] 和超流体。[12]

量子 这个词源自拉丁语,意思是“有多大”或“有多少”。[13] 在量子力学中,它指的是分配给特定物理量的分立单位,例如静止原子的能量(见图1)。粒子是具有波状性质的分立能量包的发现导致了物理学中关于原子和亚原子系统的分支,这一分支今天被称为量子力学。它是许多物理和化学领域的数学框架的基础,包括凝聚态物理、固体物理、原子物理、分子物理学、计算物理、计算化学、量子化学、粒子物理、核化学和核物理。[14] 该理论的一些基本方面仍在积极研究中。[15]

量子力学对于理解原子长度的尺度和更小尺度下的系统行为至关重要。如果一个原子的物理性质只能用经典力学来描述,那么电子就不能围绕原子核做轨道圆周运动,因为轨道上的电子发射辐射(由于圆周运动),并且会因为能量的损失而快速的与原子核碰撞。这个理论框架无法解释原子的稳定性。相反,电子可以在绕原子核的轨道上保持不确定的、模糊的、几率波的运动,这挑战了经典力学和电磁学的传统假设。[16]

量子力学最初的发展是为了更好地解释和描述原子,特别是同一化学元素的不同同位素以及亚原子粒子发出的光谱差异。简而言之,量子力学原子模型在动摇经典力学和电磁学的领域取得了惊人的成功。

广义而言,量子力学包含经典物理学无法解释的四类现象:
\begin{itemize}
\item 某些物理性质的量子化
\item 量子纠缠
\item 不确定性原理
\item 波粒二象性
\end{itemize}
然而,后来在2018年10月,有物理学家研究称,对于单粒子,量子行为可以用经典物理学来解释,但对于多个粒子,如量子纠缠和相关的非定域现象则不能。[17][18]

\subsection{ 数学公式}
在由保罗·狄拉克[19] 、戴维·希尔伯特、[20] 约翰·冯·诺依曼、[21] 还有赫尔曼·外尔发展的量子力学的严谨数学表述中,[22] 量子力学系统的可能状态是被符号化的,[23] 作为单位矢量(称为 态矢)中。形式上,它们位于复可分希尔伯特空间中 –该空间被称为系统的态矢空间或关联希尔伯特空间  –可被良好定义复数范数为1(相位因子)。换句话说,可能的状态是希尔伯特空间投影空间中的点,通常称为复投影空间。希尔伯特空间的确切性质取决于系统 –例如,位置和动量状态的态空间是平方可积函数的空间,而单个质子自旋的状态空间只是两个复平面的乘积。每一个可观测值都由作用在态空间上的最大厄米(精确地说:自伴)线性算符表示。可观测值的每个本征态对应于算符的一个本征向量,相关的本征值对应于该本征态的可观测值。如果算符的谱是分立的,则可观测值只能获得这些分立的特征值。

在量子力学的形式中,系统在给定时间的状态由复波函数描述,也称为复向量空间中的态向量。[24] 这个抽象的数学对象允许计算具体实验结果的概率。例如,它允许计算在特定时间在原子核周围的特定区域找到电子的概率。与经典力学相反,人们永远无法以任意精度同时预测共轭物理量,如位置和动量。例如,电子可以被认为(以一定的概率)位于给定空间区域内的某处,但它们的确切位置未知。恒定概率密度的轮廓,通常被称为“云”,可以在原子核周围绘制,用以概念化电子最有可能的位置。海森堡的不确定性原理量化了由于粒子的共轭动量而无法精确定位粒子的能力。[25]

还有一种解释,作为测量的结果,包含系统概率信息的波函数从给定的初态坍缩为特定的本征态。测量的可能结果是代表可观测值的算符的本征值 –这解释了为什么选择厄米算符 ,因为所有特征值都是实数。给定状态下可观测值的概率分布可以通过计算相应算符的谱分析得到。海森堡不确定性原理由对应于某些可观察量的算符的不对易来表示。

因此,量子力学的概率性质源于测量行为。这是量子系统最难理解的方面之一。这是著名的玻尔-爱因斯坦之争的中心话题,两位科学家试图通过思想实验来阐明这些基本原则。在量子力学形成后的几十年里,什么是“测量”的问题已经得到了广泛的研究。量子力学的新解释已经被公式化,不再使用“波函数坍缩”的概念。其基本思想是,当一个量子系统与一个测量仪器相互作用时,它们各自的波函数会纠缠在一起,从而原始的量子系统不再作为一个独立的实体存在。[26]

一般来说,量子力学不会给出确定的值。相反,它使用概率分布进行预测;也就是说,它描述了测量一个可观测值获得可能结果的概率。通常,这些结果会受到许多因素的影响,比如密集的概率云。概率云是近似的(但比玻尔模型更好)模拟,其中电子位置由概率函数即波函数特征值给出,这样概率就是复振幅模的平方,或量子态核吸引。[27][28] 自然,这些概率将取决于测量“瞬间”的量子态。因此,所得到值也有不确定性。然而,某些态与特定可观察值的确定值相关。这些被称为可观测的本征态(“本征”可以从德语翻译为“固有”或“特征”)。[29]

在日常生活中,将一切(所有可观察到的)都视为本征态是自然和直观的。一切似乎都有明确的位置、明确的动量、明确的能量和明确的发生时间。然而,量子力学并不能同时精确地指出一个粒子的位置和动量(因为它们是共轭对)或者能量和时间(因为它们也是共轭对)的精确值。 相反,它只提供了一个概率范围,在这个范围内,粒子可能被赋予动量和动量概率。因此,用不同的词来描述具有不确定值的态和具有确定值的态(本征态)是有用的。

通常,系统不会处于我们感兴趣的可观察(粒子)的本征态。然而,如果测量可观测值,波函数将瞬时成为该可观测值的本征态(或“广义”本征态)。这个过程被称为波函数坍缩,这是一个备受争议的过程[30] 这包括所涉及正在研究的系统,也包括测量设备。如果在测量之前知道相应的波函数,就能计算出波函数坍缩成每个可能的本征态的概率。

例如,上例中的自由粒子通常具有一个波函数,它是一个以某个平均位置x0 (既不是位置的本征态,也不是动量的本征态)为中心的波包 。当我们测量粒子的位置时,不可能确定地预测结果。[26] 很可能,但不确定,它会在x0附近波函数的振幅很大。测量完成后,获得了结果x,波函数在x处坍缩成一个本征态 。[31]

量子态的时间演化由薛定谔方程描述,其中哈密顿量(对应于系统总能量的算符)随时间演化。波函数的时间演化是确定的,在这个意义上,给定一个波函数在初始时刻的样子,就能对以后的任何时刻的波函数做出确定的预测。[32]

另一方面,在测量期间,初始波函数变成另一个,以后的波函数是不确定的,它是不可预测的(即随机的)。[33][34]

波函数随着时间的推移而变化。薛定谔方程描述了波函数是如何随时间变化的,与在经典力学中的牛顿第二定律扮演着相似的角色。将薛定谔方程应用在前面提到的自由粒子例子上进行预测,波包的中心将以恒定的速度在空间中移动(就像一个没有作用力的经典粒子)。然而,波包也会随着时间的推移而向外扩散,这意味着位置会随着时间的推移而变得更加不确定。这还具有将位置本征态(可以被认为是无限尖锐的波包)转变成不再代表(确定的,确定的)位置本征态的加宽波包的效果。[35]

一些波函数产生的概率分布是常数,或者与时间无关 –例如,当处于恒定能量的静止状态时,时间在波函数的绝对值平方中消失。许多在经典力学中被动态处理的系统被这样的“静态”波函数所描述。例如,未激发原子中的单个电子在经典力学中被描述为围绕原子核做圆形轨迹运动的粒子,而在量子力学中,它被描述为围绕原子核的静态球对称波函数(图一)(请注意,只有最低角动量状态,标记为 $s$,球对称)。[36]

薛定谔方程作用于整个几率幅,不仅仅是它的绝对值。尽管几率幅的绝对值传递了概率信息,但它的相位传递了量子态之间的干涉信息。这导致了量子态的“波状”行为。事实证明,只有极少数相对简单的哈密顿算符可以得到薛定谔方程的解析解,其中最重要的代表是量子谐振子、势箱中粒子、二氢阳离子和氢原子。甚至氦原子都不可以 –它只比氢原子多一个电子 –就破坏了所有试图得到解析解的尝试。

然而,有几种得到近似解的技术。在被称为微扰理论的重要方法中,人们使用简单量子力学模型的解析结果来生成与简单模型相关的更复杂模型的结果,例如通过添加弱势能。另一种方法是“半经典运动方程”方法,它适用于量子力学只对经典行为产生微弱(小)偏差的系统。然后可以根据经典运动学计算这些偏差。这种方法在量子混沌领域尤为重要。

\begin{figure}[ht]
\centering
\includegraphics[width=8cm]{./figures/2f68008aeda54408.png}
\caption{请添加图片标题} \label{fig_LZLX_3}
\end{figure}





