% 球坐标中的薛定谔方程
% keys 薛定谔方程|球坐标|本征矢|张量积|球谐函数
% license Xiao
% type Tutor

\begin{issues}
\issueAbstract
\issueOther{这篇大概是用来数值解 TDSE 的。}
\end{issues}

\pentry{球坐标系中的定态薛定谔方程\nref{nod_RadSE}, 张量积空间\nref{nod_DirPro}}{nod_4ca7}

本文使用\enref{原子单位制}{AU}。 无论 TDSE 是否可分离变量, 在球坐标系中用球谐函数都是常用的做法
\begin{equation}\label{eq_RYTDSE_1}
\Psi(\bvec r, t) = \sum_{l,m} R_l (r, t) Y_{l,m} (\uvec r) = \sum_{l,m} \frac{1}{r} \psi_{l,m} (r, t) Y_{l,m} (\uvec r)~,
\end{equation}
我们可以将 $\psi_l (r, t) Y_{l,m} (\uvec r)$ 看做径向空间和角向空间中态矢的张量积。 我们将 $l, m$ 的组合进行排序并给每个组合一个全局下标 $i$ 或 $j$。
\begin{equation}
\Psi(\bvec r, t) = \sum_j R_j (r, t) Y_j (\uvec r) = \sum_j \frac{1}{r} \psi_j (r, t) Y_j (\uvec r)~.
\end{equation}

将波函数代入含时薛定谔方程
\begin{equation}
H \sum_j\ket{R_j}\ket{Y_j} = \I \pdv{t}  \sum_j\ket{R_j}\ket{Y_j}~.
\end{equation}
左乘 $\bra{Y_i}$, 可以将角向坐标积去, 得到一组径向函数的 coupled equation。 这不完全是 TDSE 的矩阵形式, 因为我们没有在径向选取基底\footnote{另一种理解是在径向选取 $\delta(r - r_0)$ 作为基底, 但本征值连续的基底比较复杂, 就不这么想吧。}。
\begin{equation}
\sum_j \mel{Y_i}{H}{Y_j} \ket{R_j} = \I \pdv{t} \ket{R_j}~.
\end{equation}
如果 TDSE 可以分离变量, $\ket{Y_j}$ 是 $H$ 的本征矢, 那各个径向波函数将会是独立的 (uncoupled)。


它们分别都是对角矩阵, $K_r$ 和 $V(r)$ 的每个对角元都是一样的, 我们已经见到过, 第二项与 $l$ 有关。 含时薛定谔方程变为
\begin{equation}
-\frac{1}{2m} \dv[2]{\psi_{l,m}}{r} + \qty[V(r) + \frac{l(l + 1)}{2mr^2}]\psi_{l,m} = \I \pdv{t} \psi_{l,m}~.
\end{equation}
