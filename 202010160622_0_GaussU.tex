% 高斯单位制
% 厘米|克|秒|国际单位|CGS

\pentry{厘米—克—秒单位制\upref{CGS}}

\subsubsection{电荷}
CGS 单位满足库伦定律
\begin{equation}\label{GaussU_eq1}
F = \frac{q_1 q_2}{r^2}
\end{equation}
注意与国际单位相比, 电荷的量纲发生变化, 单位为 $\Si{\sqrt{cm^3 g}/s}$, 为了方便我们不妨记为\footnote{这是笔者发明的记号} $C_g$. 以下为了方便我们认为 $\Si{C}$ 和 $\Si{C_g}$ 是相同的量纲, 则电荷的转换常数为
\begin{equation}
\beta_q = \sqrt{4\pi\epsilon_0\beta_F} \beta_x \approx 3.3356409510736\times 10^{-10} \Si{C/C_g}
\end{equation}

\subsubsection{电场}
电场需要满足
\begin{equation}
\bvec F = \bvec E q
\end{equation}
电场单位为 $\Si{\sqrt{g/cm}/s} = \Si{C_g/cm^2}$. 转换常数为
\begin{equation}
\beta_{\mathcal E} = \beta_F/\beta_q = \sqrt{\frac{\beta_m}{4\pi\epsilon_0 \beta_x}} \approx 2.997924580815998\times 10^4 \Si{\frac{C_g\cdot m\cdot kg}{C\cdot cm\cdot g}}
\end{equation}
若采用 2020 年 5 月以前的国际单位标准, 数值上等于 $10^{-4} c$.

\subsubsection{磁场}
高斯单位制中, 洛伦兹力为
\begin{equation}
\bvec F = \frac{q}{c} \bvec v \cross \bvec B
\end{equation}
可得电磁场具有相同的单位. 转换常数为\footnote{2020 新国际单位标准以前, 这个数值精确等于 $1/100$, 新标准需要乘以 $\sqrt{\mu_0/(4\times 10^{-7}\pi)}$ 的数值.}
\begin{equation}
\beta_B = \frac{\beta_m\beta_x}{c\beta_q} = \frac{\beta_{\mathcal E}}{c\beta_x} \approx 1.000000000272\times 10^{-2} \Si{\frac{C_g\cdot kg \cdot s}{C \cdot g \cdot m}}
\end{equation}

\subsubsection{其他公式}
真空中的平面电磁波\upref{VcPlWv}若用高斯单位表示, 有 $E_0 = B_0$.

场能量密度 % 链接未完成
\begin{equation}
\rho_E = \frac{1}{8\pi} (\bvec E^2 + \bvec B^2)
\end{equation}
坡印廷矢量\upref{EBS}
\begin{equation}
\bvec s = \frac{c}{4\pi} \bvec E \cross \bvec B
\end{equation}

\begin{equation}
\epsilon_0 = \frac{1}{4\pi} \qquad
\mu_0 = 4\pi
\end{equation}
转换常数为 $\beta_\epsilon = 4\pi\epsilon_0$, $\beta_\mu = \mu_0/(4\pi)$.
