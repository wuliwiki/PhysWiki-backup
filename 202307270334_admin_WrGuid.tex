% 小时百科创作指导
% 小时百科|词条|目录|预备知识|目录树|参考文献

\subsection{内容创作}
\begin{itemize}
\item 请遵守 “小时百科符号与规范\upref{Conven}”。 如果需要添加, 请与管理员讨论。
\item 新作者建议先在群里或者和管理员讨论内容和构思再开始写作。
\item 新建词条以前必须确认无重复内容, 若创作了相同内容且质量并不更好, 将会被删除或合并。
\item 我们会尽快修复网站的搜索功能。 可以在 \href{https://github.com/MacroUniverse/PhysWiki-log}{github 页面}进行百科的全文搜索, 输入关键词, 选择 “in this repository”。 但可惜该方法不能匹配连续中文字符中的关键词, 解决方法是把完整的代码下载或者 clone 到本地再用其他软件搜索(例如 vscode, fileseek)。
\item 如果当前内容只是一个大纲或草稿, 需要标记为草稿词条(使用 “标记词条” 按钮)。
\item 小时百科是一个侧重于自学的百科。 所以内容需要尽量对初学者友好, 例如足够的文字讲解, 图片, 例题等。 如果只是简单的堆砌公式定理, 需要标记词条为 “需要更多讲解” (使用 “标记词条” 按钮)。
\item 在正文的任何地方可以用 \verb|\addTODO{}| 来插入需要补充的内容, 如果这么做, 同时需要将词条标记为 “存在未完成内容”。
\item 如果 TODO 的内容不希望被读者看到, 可以直接用注释, 但需要包含 “未完成” 三个字以便搜索。
\item “标记词条” 按钮中也可以下拉选择 “其他”, 添加自定义的改进意见。
\item 注意词条中除了可以用 “内部引用” 按钮(实心冒号图标)一键引用本文的公式图表等, 也可以用 “外部引用” 引用其他词条的内容。
\end{itemize}

\subsection{目录}
\begin{itemize}
\item 请在文件 \verb|main.tex| 中添加或修改目录, 新建的词条并不会自动插入到目录中。
\item 如果词条非常不完善, 可以在插入到目录中以后将其注释, 等完善后再取消注释。
\item 由于词条繁多, 小时百科目录中的 “部分” 和 “章节” 只是给所有内容做一个话题分类, 而不是按照建议阅读顺序排列。 建议的阅读顺序由下面介绍的 “预备知识” 决定。
\end{itemize}

\subsection{预备知识}
\begin{figure}[ht]
\centering
\includegraphics[width=13cm]{./figures/f1e1ba5153f30d39.png}
\caption{使用菜单栏的按钮添加预备知识} \label{fig_WrGuid_1}
\end{figure}

\begin{itemize}
\item 小时百科重视内容的自洽性, 所以几乎所有词条都需要指定其他一些词条作为预备知识(\autoref{fig_WrGuid_1} )。 预备知识相当于 “必备知识”, 如果其中的内容不掌握, 读者阅读词条内容就会遇到较大的困难(例如从来没见过的术语,公式或定理)。
\item 目前我们默认假设读者具有普通高中毕业生的数理水平。 任何超出该水平的内容都需要在 “预备知识” 中有所体现。 如果需要低于该水平的预备知识且存在相关词条, 也需要加入。
\item 注意百科目录并不按照建议阅读的顺序来排序而是按照话题排序, 且不鼓励读者按目录顺序阅读, 所以不能假设读者以已经读过当前词条前面的词条(即使在同一章)。
\item \href{https://wuli.wiki/tree/}{知识树}页面将自动按照 “预备知识” 生成。 读者可以把任意词条作为目标或起点生成该词条相关的知识树(\autoref{fig_WrGuid_3} )。
\end{itemize}

\begin{figure}[ht]
\centering
\includegraphics[width=12cm]{./figures/af3d6f28522de9e4.png}
\caption{由 “牛顿—莱布尼兹公式” 为目标生成的知识树} \label{fig_WrGuid_3}
\end{figure}

\begin{itemize}
\item 注意 “预备知识” 是递归的,意味着你可以默认读者已经掌握 “预备知识” 词条中的 “预备知识”。 如果当前词条的预备知识里面已经列出了词条 A, 而词条 A 的预备知识中含有词条 B, 那么就无需把 B 再次列为当前词条的预备知识。 如果不小心加了也没关系, 管理员会在后台定期检测到这些重复并删除这些多余的预备知识。
\item 即使一些术语在预备知识树中提到过, 也鼓励尽量添加引用以防遗忘。
\item 一些拓展或者选读性质的词条不需要作为预备知识, 例如 “详见……词条”。
\item 在添加预备知识时, 先浏览一下里面的内容, 确保合适与连贯。
\item 如果百科中找不到预备知识所需的词条, 应该把当前词条标记为 “缺少预备知识” (\autoref{fig_WrGuid_2}), 并用注释说明需要什么内容作为预备知识。
\end{itemize}

\begin{figure}[ht]
\centering
\includegraphics[width=13cm]{./figures/63920122b1cb5203.png}
\caption{将词条标记为 “缺少预备知识”} \label{fig_WrGuid_2}
\end{figure}

\subsection{参考文献}
\begin{itemize}
\item 在 \verb|bibliography.tex| 的文献列表中添加你需要的参考文献, 然后在词条中用 \verb|\cite| 引用即可。 如果文献没有在正文中直接被引用, 那么可以在正文开始用 \verb|foornote| 添加, 如\footnote{本文参考 \cite{GriffE}, \cite{GriffQ} 以及 Wikipedia \href{https://www.wikipedia.org/}{相关页面}。}。
\item 原则上每个词条都需要参考文献, 如果没有足够的文献, 需要标记为 “缺少参考文献”(使用 “标记词条” 按钮)。
\item 一般来说比较方便的做法是直接引用 Wikipedia 相关页面(注意尽量选用质量更高的英文版), 读者可以参考由 Wikipedia 列出的更多参考文献。
\end{itemize}
