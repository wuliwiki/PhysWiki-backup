% Python C API 笔记
% license Usr
% type Note

% 参考 baltam_python_api 仓库

\begin{itemize}
\item \verb`PYTHONHOME` 环境变量是 python 解释器和标准库等的路径(子文件夹有 \verb`include, libs, Scripts`)
\item \verb`PYTHONPATH` 环境变量用于指定额外的模块搜索路径(当前路径和 python 安装路径之外的)(Linux 多个路径用 \verb`:` 隔开,Windows cmd 用 \verb`;` 隔开)
\item 其他可能对我有用的环境变量包括 \verb`PYTHONSTARTUP` 指定启动脚本、\verb`PYTHONDEBUG` 使用 debug 模式,相当于 \verb`-d` 选项、\verb`PYTHONUNBUFFERED` 强制标准输出和标准错误不被 buffer、 \verb`PYTHONHASHSEED` 指定随机数种子、 \verb`PYTHONTRACEMALLOC` 用 \verb`tracemalloc` 模块跟踪内存占用、 \verb`PYTHONPYCACHEPREFIX` 可以指定一个专门用于存放 pyc 文件的目录,而不是直接在源码目录中用 \verb`__pycache__`。
\item 这俩环境变量可以在调用 Python C API 的 C/C++ 程序里面设置,在 \verb`Py_Initialize()` 之前即可,而无需在运行前。
\item \verb`PyObject *` 可以容纳任何 python 对象
\item \verb`Py_Initialize()` 和 \verb`Py_Finalize()` 开始和结束 python 解释器。
\item \verb`PyObject *pModule = PyImport_Import(pName);` 可以加载 python 模块。
\item \verb`Py_DECREF(PyObject *obj);` 是一个宏,可以把 \verb`obj` 对象的引用计数减一, 当减少到零该对象就会自动销毁。\verb`Py_INCREF(PyObject *obj);` 把引用加一。
\item \verb`Py_XDECREF()` 更安全,因为它会先检查 \verb`obj` 是否为 \verb`NULL`。
\item \verb`Py_REFCNT()` 可以查看引用计数。
\item \verb`PyObject * PyObject_Str(PyObject *)` 或 \verb`PyObject_Repr` 可以返回字符串对象显示对象信息
\end{itemize}

\subsection{对象管理}
创建变量
\begin{itemize}
\item \verb`PyObject *pName = PyUnicode_FromString("my_python_module");` 可以从 utf-8 编码的 C 字符串创建一个 python 字符串变量。
\item 把字符串变量还原为 utf-8 的 CString,用 \verb`PyUnicode_AsUTF8AndSize`
\item \verb`PyObject *obj = PyLong_FromLong(42);` 创建 python 整数。
\item \verb`PyFloat_FromDouble()` 从 double 创建浮点数, \verb`PyFloat_AsDouble` 提取 double。
\item \verb`PyObject *int_obj = Py_BuildValue("i", 42);` 也可以创建 python 整数,该函数更动态更灵活,例如 \verb`PyObject *tuple_obj = Py_BuildValue("(is)", 42, "Hello");` 创建一个整数和字符串的 touple。
\item 创建 list: \verb`PyObject *arg = PyList_New(2);`, \verb`PyList_SetItem(arg, 0, PyLong_FromLong(1));`, \verb`PyList_SetItem(arg, 1, PyLong_FromLong(2));`
\end{itemize}

获取变量
\begin{itemize}
\item \verb`PyArg_Parse(obj, "i", &result_c);` 从 \verb`PyObject *obj` 获取整数存到 \verb`int result_c` 中。
\item \verb`PyObject* PyObject_GetAttrString(PyObject *obj, const char *attr_name);` 获取对象的属性(attribute)。 模块也是对象,如 \verb`math.pi` 就是模块 \verb`math` 对象的 \verb`pi` 属性。 它也可以用于获取函数如 \verb`math.sin`。 如果失败,返回 \verb`NULL`, 可以用 \verb`PyErr_Print()` 打印错误信息。也可以用于获取对象的成员函数对象(绑定该对象)
\end{itemize}

检查变量
\begin{itemize}
\item \verb`PyLong_Check(obj)` 检查一个对象是否为 \verb`Long`, \verb`PyFloat_Check(obj)` 检查是否为浮点数, \verb`PyUnicode_Check(obj)` 检查是否为字符串, 另外有 \verb`PyList_Check(obj)`, \verb`PyDict_Check(obj)`
\item 查字典
\begin{lstlisting}[language=cpp]
PyObject* get_dict_item(PyObject* namespace, const char* name) {
    if (!PyDict_Check(namespace)) {
        PyErr_SetString(PyExc_TypeError, "namespace must be a dictionary");
        return NULL;
    }
    // Retrieve the variable by name
    PyObject* variable = PyDict_GetItemString(namespace, name);
    if (variable == NULL)
        PyErr_Format(PyExc_NameError, "name '%s' is not defined", name);
    // Increase reference count to return a new reference
    Py_XINCREF(variable);
    return variable;
}
\end{lstlisting}
\item 通过变量名获取指针
\begin{lstlisting}[language=cpp]
PyObject* globals = PyEval_GetGlobals(); // 全局变量的字典(工作区)
PyObject* locals = PyEval_GetLocals(); // 本地变量的字典(工作区)
PyObject* g_variable = get_variable_by_name(globals, "my_variable");
PyObject* g_variable = get_variable_by_name(locals, "my_variable");
\end{lstlisting}
\end{itemize}


\subsection{函数}
\begin{itemize}
\item \verb`PyObject* PyObject_CallFunctionObjArgs(PyObject *callable, ...);` 可以调用函数,后面是函数的参数,以 \verb`NULL` 结尾。
\item \verb`PyObject_CallMethod()`
\item \verb`PyObject *result = PyNumber_Add(obj1, obj2);` 可以对两个对象使用算符 \verb`+`,如果是用户定义的对象则会调用 \verb`__add__()` 方法。
\end{itemize}
