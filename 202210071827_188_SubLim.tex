% 子列极限、上极限与下极限
% 序列|上极限|下极限|子列极限

\begin{issues}
\issueOther{与其它词条内容重复. 需要删减.}
\end{issues}

\pentry{柯西序列\upref{CauSeq}}
\subsection{子列极限与上下极限}

我们知道,有许多序列是没有极限的,例如简单的振荡序列 $\{(-1)^n\}$: 它在正负1之间无穷多次地跳动,因此它不收敛.但是,它的子序列却可以有极限. 

下面我们要介绍的是上下极限的概念,它可以很好地描述序列的“振荡”特征.如下图所示,横坐标为 $n$,纵坐标为序列的第 $n$ 个元素 $x_n$.可以发现序列越来越趋向于在橙色线和蓝色线之间振荡.“在无穷远处”,蓝色线“几乎”是序列的上界;橙色线“几乎”是序列的下界.
\begin{figure}[ht]
\centering
\includegraphics[width=12cm]{./figures/SubLim_1.png}
\caption{上下极限示意图} \label{SubLim_fig1}
\end{figure}
为了给出严格的描述,我们需要借助上确界和下确界的语言进行定义.
\begin{definition}{上下极限}\label{SubLim_def1}
设序列 $\{x_n\}$ ,定义 $l_n=\inf\{x_n,x_{n+1},\cdots\}$,定义 $h_n=\sup\{x_n,x_{n+1},\cdots\}$.也就是说,$l_n$ 是序列从 $n$ 开始的后缀的下确界;$h_n$ 是序列从 $n$ 开始的后缀的上确界.

如果序列有上界,那么 $h_n$ 是良定义的,定义序列的\textbf{上极限}为 $h=\inf\{h_n\}$.

如果序列有下界,那么 $l_n$ 是良定义的,定义序列的\textbf{下极限}为 $l=\sup\{l_n\}$.
记做
\begin{equation}
l=\opn{\underline{\lim}}\limits_{n\rightarrow\infty}x_n,
h=\opn{\overline{\lim}}\limits_{n\rightarrow\infty}x_n
\end{equation}
\end{definition}
\autoref{SubLim_fig1} 中蓝线代表的是序列的上极限,橙色线代表的是序列的下极限.
\begin{exercise}{}
设序列 $\{x_n\}$ 有界,证明序列的下极限 $l$ 可以用 $\{l_n\}$ 的极限表示;序列的上极限 $h$ 可以用 $\{h_n\}$ 的极限表示:
\begin{equation}
\begin{aligned}
l=\opn{\underline{\lim}}\limits_{n\rightarrow\infty}x_n=\lim\limits_{n\rightarrow \infty} l_n\\
h=\opn{\overline{\lim}}\limits_{n\rightarrow\infty}x_n=\lim\limits_{n\rightarrow \infty} h_n
\end{aligned}
\end{equation}
\end{exercise}

从这个习题我们可以看出,序列的上下极限的含义正是序列在“无穷远处”的上界和下界.

\subsection{上下极限的性质}
\begin{theorem}{}
  以下三个命题等价:
\begin{enumerate}
\item $h=\opn{\overline{\lim}}\limits_{n\rightarrow\infty}x_n$
\item 对任意 $\epsilon>0$,总是存在 $N$,当 $n>N$ 时,总有 $x_n<h+\epsilon$;
对任意 $\epsilon>0$,对任意的 $K$,总是存在 $n>K$,使 $x_{n}>h-\epsilon$.
\item 存在子列 $\{x_{n_k}\}$,使得 $\lim\limits_{k\rightarrow\infty}x_{n_k}=h$,并对任何其它收敛子列 $\{x_{n_k}\}$,都有 $\lim\limits_{k\rightarrow\infty}x_{n_k}\leq h$
\end{enumerate}
\end{theorem}

\addTODO{最好再补充证明}

对于一个实数的子集,将其中的数洒在数轴上,可能会存在这样的点 $x_0$,它周围有无穷多个点 $x$ 在集合中,而且无论把数轴放大多少倍,也是如此.直观上看,集合中的“点”聚集在 $x_0$ 周围,所以我们称这样的点为\textbf{聚点}.如果对任意去心邻域 $U_0(x_0,\delta)$,总是存在集合中的某个元素 $x$ 在这个去心邻域中,那么 $x_0$ 就是聚点.下面我们给出形式化的定义:
\begin{definition}{聚点}
  设 $E$ 是 $\mathbb{R}$ 中的一个子集.若 $x_0 \in \mathbb{R}$($x_0$ 不一定属于 $E$)满足:对任意 $\delta>0$,有 $U_0(x_0,\delta)\cap E\not=\emptyset$,则称 $x_0$ 是 $E$ 的一个聚点.
\end{definition}

我们尝试观察序列的上下极限有哪些性质,是否与聚点有关系.从直观上看,如果把序列 $\{x_n\}$ 当作是集合,那么在上极限附近似乎聚集了大量的点,再以上的就都不是聚点了;在下极限附近也似乎聚集了大量的点,再以下的就都不是聚点了.具体地,我们有以下定理:
\begin{theorem}{序列极限与聚点}
\begin{enumerate}
\item 若有界序列 $\{x_n\}$ 由互不相同的数组成,则上极限为其最大聚点,下极限为其最小聚点.
\item 若 $\{x_{n_k}\}$ 为 $\{x_n\}$ 子列,则有 $\opn{\underline{\lim}}\limits_{n\rightarrow\infty}x_n\leq\opn{\underline{\lim}}\limits_{k\rightarrow\infty}x_{n_k}\leq \opn{\overline{\lim}}\limits_{k\rightarrow\infty}x_{n_k}\leq \opn{\overline{\lim}}\limits_{n\rightarrow\infty}x_{n}$.
\item  $\lim\limits_{n\rightarrow\infty}x_n=a\Leftrightarrow\opn{\underline{\lim}}\limits_{n\rightarrow\infty}x_n=\opn{\overline{\lim}}\limits_{n\rightarrow\infty}x_n=a$
\end{enumerate}
\end{theorem}
\begin{exercise}{}
\begin{enumerate}
  \item 设 $E$ 是 $\mathbb{R}$ 中的一个子集.若 $x$ 是 $E$ 的一个聚点,证明:  $\forall \delta>0$, $U(x_0,\delta)\cap E$ 中有无穷多个元素.

  \item 证明性质1和性质2.
  \item 利用上下极限证明 $\{\sin n\}$ 发散.
  \item 序列 $\{x_n\}$ 的上极限为 $h_1$,序列 $\{y_n\}$ 的上极限为 $h_2$,那么序列 $\{x_n+y_n\}$ 的上极限是 $h1+h2$ 吗,$\{x_ny_n\}$ 的上极限是 $h_1\cdot h_2$ 吗?
  \item 若序列 $\{x_n\}$ 是非负收敛序列,上一问的命题成立吗?
\end{enumerate}
\end{exercise}




\subsection{柯西收敛准则}
\begin{definition}{柯西序列}
设{$x_n$}是一个序列,若对任意 $\epsilon>0$,总是存在 $N$,当 $n,m>N$ 时,有 $\vert x_n-x_m\vert<\epsilon$,则称{$x_n$}是一个柯西序列.
\end{definition}
\begin{theorem}{柯西收敛准则}
序列 {$x_n$} 收敛 $\Leftrightarrow$ {$x_n$} 是一个柯西序列.
\end{theorem}
\begin{theorem}{压缩映像定理}
设 $f(x)$ 在 $[a,b]$ 上有定义,$f([a,b])\subset[a,b]$,且存在一个 $(0,1)$ 内的实数 $q$,满足对任意 $x,y\in[a,b]$,都有 $|f(x)-f(y)|\leq q|x-y|$.

那么:存在唯一 $c\in[a,b]$,使 $f(c)=c$.
\end{theorem}
\textbf{证明:}构造序列 $\{x_n=f(x_{n-1})\}$,用柯西准则证明其收敛.

\begin{exercise}{}
\begin{enumerate}
\item 写出柯西收敛准则的否命题形式(用肯定的语气写出:怎样的序列不是柯西序列,怎样的序列不是发散的).
\item 试用闭区间套定理证明压缩映像原理.
\item 设序列 $\{p_n\}, \{q_n\}$ 满足 $p_{n+1}=p_n+2q_n,\ q_{n+1}=p_n+q_n,\ p_1=q_1=1$,求证 $\left\{\large\frac{p_n}{q_n}\right\}$ 的极限存在,并求其极限.
\end{enumerate}
\end{exercise}