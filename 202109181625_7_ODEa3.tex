% 一阶常微分方程解法:恰当方程
% 恰当方程|微分方程|ODE|ordinary differential equation|exact equation|积分因子|integral factor

\pentry{常微分方程简介\upref{ODEint}}

\subsection{恰当方程的概念}

考虑一个二元函数$u(x, y)$,其全导数为
\begin{equation}
\dd u=\frac{\partial u}{\partial x}\dd x+\frac{\partial u}{\partial y}\dd y
\end{equation}

由多元微积分知识可知,$\frac{\partial^2 u}{\partial x\partial y}=\frac{\partial^2 u}{\partial y\partial x}$.因此,如果一个形如
\begin{equation}\label{ODEa3_eq1}
M(x, y)\dd x+N(x, y)\dd y=0
\end{equation}
的常微分方程满足
\begin{equation}\label{ODEa3_eq2}
\frac{\partial M}{\partial y}=\frac{\partial N}{\partial x}
\end{equation}
那么就可以存在一个$u(x, y)$,使得$M=\partial u/\partial x$和$N=\partial u/\partial y$.

这样一来,\autoref{ODEa3_eq1} 就相当于
\begin{equation}
\dd u=0
\end{equation}
其解就是$u=C$,$C$为积分常数.

也就是说,对于这样的方程,我们只需要求出$u$就能求解.

\begin{definition}{恰当方程}
将形如\autoref{ODEa3_eq1} 且满足\autoref{ODEa3_eq2} 的方程,称为\textbf{恰当方程(exact equation)}.
\end{definition}

$u$是很好解的,取$M$或$N$做积分即可:
\begin{equation}
u=\int M\dd x+C_1=\int N\dd y+C_2
\end{equation}

\begin{example}{}
考虑方程$\frac{\dd y}{\dd x}=\frac{y}{3y^2-x}$.

移项后得到$y\dd x+(x-3y^2)\dd y=0$.

记$M=y$,$N=x-3y^2$,则容易验证$\partial M/\partial y=\partial N/\partial x=$

\end{example}

















