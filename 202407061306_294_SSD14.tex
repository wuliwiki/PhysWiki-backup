% 首都师范大学 2014 年硕士考试试题
% keys 首都师范大学|2014|考研|物理
% license Copy
% type Tutor

\begin{enumerate}
\item 湖中有一条小船,岸边有人用绳子通过岸上高于水面$h$的滑轮拉船,设人收绳的速率为$v_0$,求船的速度$u$和加速度$a$。
\item 如图1,一块宽$L=0.60m$、质量$M=lkg$的均匀薄木板,可绕水平固定光滑轴OO'自由转动,当木板静止在平衡位置时,有一质量为$m=10*10^{-3}$的子弹垂直击中木板A点,A离转轴OO'距离为$l=0.36m$,子弹击中木板前速度为$500m/s$,穿出木板后的速度为$200m/s$。求:\\
(1)子弹给予木板的冲量;\\
(2)木板获得的角速度.
\item 用铁锤将铁钉击入木板,设木板对铁钉的阻力与铁钉进入木板的深度成正比,在铁锤击第一次时,能将铁钉击入木板$1cm$,问击第二次时,能击多深?设铁锤两次击钉的速度相同。
\item 有一质量为$m$、长为$l$的均匀细棒,静止平放在滑动摩擦系数为$\mu$的水平桌面上,它可绕通过其端点$O$且与桌面垂直的固定光滑轴转动。另有一水平运动的质量为$m_2$的小滑块,从侧面垂直于棒与棒的另一端$A$发生弹性碰撞,设碰撞时间极短。已知小滑块在碰撞前后的速度分别为$\bar v_1$和$\bar v_1$,如图所示。求碰撞后从细棒开始转动到停止转动的过程需的时间。
\item 一个半径为$R$的金属球,带有电荷$q$,将它浸没在电容率为$\varepsilon$的无限大均匀电介质中,试求:\\
(1)空间的电位移、电场强度、电势的分布;\\
(2)介质表面的极化电荷面密度;\\
(3)空间的电场能量。
\item 图为一横截面图。半径为$R$的无限长导体圆弧柱面沿轴向(垂直纸面)流有均匀面电流,总电流强度为$I$,圆弧的弧度为$2\theta_0(\theta_0<\pi)$。又在离轴线$R$的另一处有一同方向的无限长线电流,电流强度为$I$’,位于半圆柱面的对称面上。$I$和$I$’均垂直纸面向外。试求线电流$I$’上单位长度所受到的磁场力。
\item 匀强磁场局限在半径为$a$的柱形区域内,磁场方向如图所示,磁感应强度的大小正以速率$\lambda$在增加,求:\\
(1)空间感生电场的分布;\\
(2)若在柱形磁场区域外同轴地放置一个柱面形导体,则此导体的感生电动势为多少?
\end{enumerate}