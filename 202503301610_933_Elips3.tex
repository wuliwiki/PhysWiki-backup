% 椭圆(高中)
% keys 极坐标系|直角坐标系|圆锥曲线|椭圆
% license Xiao
% type Tutor

\begin{issues}
\issueDraft
\end{issues}

\pentry{解析几何\nref{nod_JXJH},圆\nref{nod_HsCirc}}{nod_32e0}

% 933:这篇文章的介绍思路是主要考虑高中教材。从圆和解析几何的视角来引入椭圆本身。忽略圆锥曲线这件事。

人们常说地球绕着太阳转。提到“绕着转”,很多人往往会自然联想到完美的圆形轨道。但实际上,真正沿正圆轨道运行的行星几乎不存在。大多数行星的轨道虽然呈环绕状,但并不是正圆,而是略微“压扁”的圆。这种“被压扁的圆”在生活中并不少见。比如,一个圆形水杯的杯口,斜着看时,就会呈现出这种变形的轮廓。这种形状被称为“椭圆”。

如果你在教室里轻声说话,声音会在墙壁间反弹,有时远处的同学反而能听得更清楚。在一些老式博物馆中,有“回声墙”的设计:两个人站在大厅中一段弧形墙的两个特殊位置,即使中间隔着一面墙,也能清晰听到彼此的低语。人们研究发现,有一种特殊的曲线能实现这种效果——从某个点发出的声音或光线,经过反射后总能准确传到另一个固定点。巧的是这条曲线,也是椭圆。

椭圆是圆的“兄弟”,但更灵活、更丰富。接下来,将从椭圆的几何定义、标准方程及其基本参数出发,逐步认识这一优雅而实用的图形。

\subsection{椭圆的几何定义}





从圆的定义开始,如果想要引申圆的定义,可以这样看,假设圆心是两个重合的点,然后圆上每个点到这两个点的距离相等,都是定值,现在如果把这两个点移开,那么显然如果在平面上所有到这两个点距离相等的点就构成了他们连线的垂直平分线。那么,如果加上一些不同的设定,或许会得到不同的效果。比如,圆上点到的这两个点到距离是定值,这启发我们或许可以是点到这两个点的距离之和是定值。

下面给出古希腊时代从几何视角给出的椭圆定义,这也常被称作是椭圆的第一定义。

\begin{definition}{椭圆的几何定义}
平面上到两定点的距离之和为有限定值的几何图形,称为\textbf{椭圆}。两个定点称作椭圆的两个\textbf{焦点}。
\end{definition}

可以这样看,当一个椭圆的两个焦点重合时,椭圆就变成了圆。所以可以这样说,圆是椭圆的一个特例。

椭圆是到两个定点(焦点)距离之和为定值的点的集合。这个定值等于椭圆的长轴长(记作 $2a$),即对任意点 $P$,有
\begin{equation}\label{eq_Elips3_9}
PF_1 + PF_2 = 2a ~.
\end{equation}

\subsection{椭圆的标准方程}

我们已经知道用焦点和准线如何定义\enref{椭圆}{Cone},下面介绍另外三种。 其中 “圆锥截面定义” 揭示了 “圆锥曲线” 一词的由来。

\begin{definition}{椭圆的标准方程}
\begin{equation}\label{eq_Elips3_3}
\frac{x^2}{a^2} + \frac{y^2}{b^2} = 1~.
\end{equation}
\end{definition}

从椭圆的极坐标公式难以看出椭圆的对称性, 另一种定义椭圆的方法是直接在直角坐标系中给出椭圆的方程

\subsubsection{参数介绍}

**长轴(major axis)**长度为 $2a$。
	•	**短轴(minor axis)**长度为 $2b$。
	•	焦点之间的距离为 $2c$,其中
$$ c^2 = a^2 - b^2~. $$
所以 $c < a$,表示焦点在中心两侧。

	•	若 $a > b$,焦点在 $x$ 轴上;
	•	若 $b > a$,焦点在 $y$ 轴上;
	•	若 $a = b$,就是圆。


长轴

短轴
焦距

\subsection{椭圆的参数方程}
表示为参数方程
\begin{equation}\label{eq_Elips3_1}
\leftgroup{
&x(t) = a\cos t\\
&y(t) = b\sin t
},\quad t \in [0, 2\pi) ~.
\end{equation}
\subsection{椭圆的性质}

椭圆上点与焦点连线的斜率之积为定值。

关于 $x$ 轴、$y$ 轴和原点对称。是中心对称图形,中心即椭圆中心。
任何穿过两个焦点的直线段,其在椭圆上的两个端点之间的距离等于 $2a$,且椭圆的长轴为所有通过的弦中最长的。
焦点反射性质(反射定理)
一条从一个焦点 $F_1$ 发出的光线射到椭圆上的某点 $P$,会被反射到另一个焦点 $F_2$,即:
\begin{equation}
\angle F_1PT = \angle F_2PT~.
\end{equation}

($T$ 为切线的切点)
一条从一个焦点出发射到椭圆上的射线,在椭圆上反射后,会朝向另一个焦点。这个性质被用于椭圆形镜面(如椭圆房顶)。椭圆任一点的切线,使得该点到两个焦点的连线之间的夹角相等(即入射角=反射角)。路径 $F_1PF_2$ 是 $F_1 \to F_2$ 所有路径中最短的一条绕射路径。

椭圆面积为:
\begin{equation}
S = \pi a b~.
\end{equation}

可看作是“拉扁”的圆。

焦半径夹角相等
在椭圆上任意一点 $P$,连到两个焦点 $F_1$, $F_2$ 的线段与该点的切线夹角相等。

	10.	切线交焦线定长性质
过椭圆上一点作切线,交焦点连线于两侧,其两个交点到该点的距离之差恒定。

\addTODO{下面移走}


这相当于把一个单位圆(方程 $x^2 + y^2 = 1$)在 $x$ 轴和 $y$ 轴分别拉长了 $a$ 倍和 $b$ 倍。 我们这里用焦点和准线的定义来推导出上式, 以证明它们等价。 我们不妨先以一个焦点为原点定义直角坐标系, 且令 $x$ 轴指向另一个焦点, 则有
\begin{equation}
r = \sqrt{x^2 + y^2}~, \qquad \cos\theta = \frac{x}{\sqrt{x^2 + y^2}}~.
\end{equation}
代入椭圆的极坐标方程\autoref{eq_Cone_5}  得
\begin{equation}
\sqrt{x^2 + y^2} = p + ex~.
\end{equation}
两边平方并整理得
\begin{equation}\label{eq_Elips3_2}
(1 - e^2) \qty( x - \frac{ep}{1 - e^2} )^2 + y^2 = \frac{p^2}{1 - e^2}~.
\end{equation}
由此可见,如果我们把椭圆左移 $ep/(1 - e^2)$,椭圆将具有\autoref{eq_Elips3_3} 的形式。 其中 $a$ 为\textbf{半长轴}, $b$ 为\textbf{半短轴}。这就是椭圆的第二种定义, 即把单位圆沿两个垂直方向分别均匀拉长 $a$ 和 $b$。 所以也可以

下面来看系数的关系。首先定义椭圆的焦距为焦点到椭圆中心的距离(即以上左移的距离)为
\begin{equation}\label{eq_Elips3_5}
c = \frac{ep}{1 - e^2}~.
\end{equation}
\autoref{eq_Elips3_2} 和\autoref{eq_Elips3_3} 对比系数得
\begin{equation}\label{eq_Elips3_6}
a = \frac{p}{1 - e^2}~, \qquad b = \frac{p}{\sqrt {1 - e^2} }~.
\end{equation}
以上两式可以将椭圆的极坐标方程转为直角坐标方程。 另外易证
\begin{equation}\label{eq_Elips3_7}
a^2 = b^2 + c^2~.
\end{equation}
若要从直角坐标方程变回极坐标方程, 将\autoref{eq_Elips3_5} \autoref{eq_Elips3_6} 逆转得
\begin{equation}\label{eq_Elips3_8}
e = \frac{c}{a}~,\qquad
p = \frac{b^2}{a}~.
\end{equation}



