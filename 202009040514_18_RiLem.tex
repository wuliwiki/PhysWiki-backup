% 里斯引理 (赋范空间)
\pentry{范数\upref{NormV}}

里斯引理 (Riesz's lemma) 是赋范空间理论中的一个基本引理. 它的表述如下:

\begin{lemma}{里斯引理 (Riesz's lemma)}
设$X$是赋范空间, $\|\cdot\|$是它的范数, $M$是它的真闭子空间. 则任给$\alpha\in(0,1)$, 都存在$x$使得$\|x\|=1$, 且$\text{dist}(x,M)\geq\alpha$.
\end{lemma}

这个引理有鲜明的直观意义: 任给赋范空间中的"平面", 总有一个单位向量与这平面"几乎垂直".

\subsection{证明}
证明是直接的构造. 任取$y\in X\setminus M$, 命$\delta=\text{dist}(y,M)$. 由于$M$是闭的, 故$\delta>0$. 根据距离的定义, 任给$\varepsilon>0$, 皆存在$z_0\in M$使得
$$
\delta\leq \|y-z_0\|<\delta+\varepsilon.
$$
命$x=(y-z_0)/\|y-z_0\|$, 则$\|x\|=1$, 而且对于任何$z\in M$皆有
$$
\|x-z\|
=\left\|\frac{y-z_0}{\|y-z_0\|}-z\right\|
=\left\|\frac{y-z_0-z\|y-z_0\|}{\|y-z_0\|}\right\|
$$