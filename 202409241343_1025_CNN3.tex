% 深度学习 CNN 入门 3
% license Usr
% type Wiki


\pentry{卷积\nref{nod_Conv},神经网络\nref{nod_NN},全连接网络\nref{nod_FCNN},Python 导航\nref{nod_PyFi}}{nod_567b}

在卷积神经网络中,对于输入的图像,需要多个不同的卷积核对其进行卷积,来提取这张图像不同的特征(多核卷积);同时也需要多个卷积层进行卷积,来提取深层次的特征(深度卷积)。

感受野指的是卷积神经网络每一层输出的特征图(feature map)上每个像素点映射回输入图像上的区域大小。神经元感受野的范围越大表示其能接触到的原始图像范围就越大,也意味着它能学习更为全局,语义层次更高的特征信息;相反,范围越小则表示其所包含的特征越趋向局部和细节。因此感受野的范围可以用来大致判断每一层的抽象层次。并且我们可以很明显地知道网络越深,神经元的感受野越大。由此可知,深度卷积神经网络中靠前的层感受野较小,提取到的是图像的纹理、边缘等局部的、通用的特征;靠后的层由于感受野较大,提取到的是图像更深层次、更具象的特征。因此在迁移学习中常常会将靠前的层的参数冻结(不参与训练,因为他们在迁移到新的场景之前已经具备了提取通用特征的能力),来节省训练的时间和算力消耗。

