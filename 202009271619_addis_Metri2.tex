% 度量空间中的概念

\pentry{度量空间\upref{Metric}}
度量空间与一般集合的最大区别就是元素之间有了距离的概念. 利用距离函数我们可以定义许多度量空间特有的基本概念.

\begin{definition}{邻域}
给定一个半径 $r > 0$, 度量空间 $X$ 中的一点 $x$ 周围所有满足 $d(x, y) < r$ 的点 $y \in X$(包括 $x$ 自己)就是 $x$ 在 $X$ 中的一个\textbf{邻域(neighborhood)} $N_r$. 如果将邻域去掉 $x$ 本身, 就叫做\textbf{去心邻域(deleted/punctured neighbourhood)}.
\end{definition}
注意邻域取决于所讨论的度量空间, 例如即使当 $x$ 和 $r$ 不变, 当 $X$ 分别取有理数集和实数集时, 邻域 $N_r$ 也是不同的. 以下许多概念也与讨论的空间有关, 所以在讨论时我们需要明确使用哪个空间.

\begin{definition}{内点}\label{Metri2_def2}
给定度量空间 $X$ 的一个子集 $A$. 如果某点 $x\in A$ 在 $X$ 中的某邻域是 $A$ 的子集, 那么 $x$ 就是集合 $A$ 的\textbf{内点(interior point)}.

换言之: 给定度量空间 $X$ 的一个子集 $A$ 以及 $x \in A$, 如果存在 $r > 0$ 使所有 $X$ 中所有与 $x$ 距离小于 $r$ 的点都属于 $A$, 那么 $x$ 就是 $A$ 的内点.
\end{definition}

\begin{example}{}\label{Metri2_ex1}
一个点是否是 $A$ 的内点取决于 $X$ 的定义. 例如令 $A$ 为 $[-1, 1]$ 中的有理数, $X$ 为有理数集, 则 $0$ 是 $A$ 的内点. 但如果取 $X$ 为实数集, 那么 $0$ 就不是 $A$ 的内点, 因为可以证明任意两个有理数之间都存在实数.% 链接未完成

另一个例子是:如果令 $A = X$, 那么任何点都是内点.
\end{example}

\begin{definition}{极限点,离散点}
给定度量空间 $X$ 中的一点 $x$, 如果对任意的 $r > 0$, $x$ 的去心邻域都不为空, 那么 $x$ 就是集合 $X$ 的一个\textbf{极限点(limit point)}. 如果一个点不是极限点, 它就是\textbf{离散点(discrete point)}.
\end{definition}
例如有理数(作为 $\mathbb R$ 的一个子集)的极限点却不一定是有理数, 例如 $\sqrt{2}$ 的任意邻域中都有无穷个有理数, 所以是有理数集的一个极限点, 但 $\sqrt{2}$ 却是无理数.

\begin{corollary}{}
度量空间 $X$ 中的点 $x$ 是极限点的充分必要条件是, $x$ 的任意去心邻域都有无穷多个点.
\end{corollary}
证明: 是用反证法. 如果 $x$ 的某个去心领域只有有限个点, 那么必定能找到离 $x$ 最近的一点 $y$, 那么对于任意的 $r < d(x, y)$, $x$ 的去心邻域为空, 与定义矛盾.证毕.

\begin{example}{}\label{Metri2_ex2}
有理数集或实数集($\mathbb R$) 构成的度量空间中任意一点都是极限点(证明显然).
\end{example}

\begin{definition}{序列的极限}\label{Metri2_def1}
给定度量空间 $X$ 中的无穷个点组成的\textbf{序列(sequence)} $x_1, x_2, \dots$ 以及一点 $x$, 若对任意给定的 $\epsilon > 0$, 总存在 $N$ 使得当 $n > N$ 时就有 $d(x_n, x) < \epsilon$, 那么 $x$ 就是该序列的\textbf{极限(limit)}.
\end{definition}
需要注意: 度量空间中序列的极限未必是极限点, 例如整数集 $\mathbb Z$ 中的序列 $1, 2, 3, 3, 3, \dots$ 的极限是 $3$, 但 $\mathbb Z$ 中任意一点都不是($\mathbb Z$ 的)极限点. 2. 极限必须属于 $X$. 例如在 “所有大于零的实数中”, 序列 $1, 1/2, 1/3, \dots$ 不存在极限, 但若改为 “所有实数中”, 那么它就\textbf{存在极限}.

% \begin{definition}{集合的极限点}
%(错!)若 $A$ 是度量空间 $X$ 的一个子集, 且 $X$ 的一个极限点 $x$ 的某个去心邻域是 $A$ 的子集, 那么 $x$ 就是子集 $A$ 的一个极限点, 无论 $x$ 是否属于 $A$.
% \end{definition}

\subsection{开集和闭集}
我们初高中所学的开区间就是实数集 $\mathbb R$ 的开集. 任意开区间的并和有限开区间的交也是 $\mathbb R$ 的开集. 下面我们对任意度量空间通过距离的概念给出开集的定义.

\begin{definition}{度量空间的开集}
若度量空间 $X$ 的子集 $A$ 中的任意一点都是内点(\autoref{Metri2_def2} ), 那么 $A$ 就是一个\textbf{开集(open set)}.

换言之: 给定度量空间 $X$ 的一个子集 $A$, 若对于任意 $x \in A$ 都存在 $r > 0$ 使得 $X$ 中与 $x$ 距离小于 $r$ 的所有点都属于 $A$, 那么 $A$ 就是一个开集.
\end{definition}
由于 $A$ 中的点是否为内点取决于 $X$, 所以 $A$ 是否为开集也是相对于 $X$ 而言的. 在

注意 $A$ 是否为开集取决于 $X$. 例如令 $A$ 为 $[0, 1]$ 中的有理数, $X$ 为有理数集, 则 $A$ 相对于 $X$ 而言是开集. 但如果取 $X$ 为实数集, 那么 $A$ 不是开集, 因为可以证明任意两个有理数之间都存在实数.% 链接未完成

如果取 $X = A$, 则

\begin{definition}{度量空间的闭集}
若度量空间 $X$ 的子集 $A$ 是开集, 那么 $A$ 关于 $X$ 的补集就是\textbf{闭集(closed set)}.
\end{definition}

\begin{theorem}{}
度量空间 $X$ 的子集 $A$ 是闭集的充分必要条件是: $A$ 的所有极限点都属于 $A$.
\end{theorem}

注意即使不是度量空间, 我们也可以通过更广义的方式定义开集, 见 “拓扑空间\upref{Topol}”. 度量空间是拓扑空间的一种, 可以证明上述定义的开集同样满足拓扑空间中对开集的要求, 即 “有限交任意并” 也都是开集.
