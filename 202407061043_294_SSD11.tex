% 首都师范大学 2011 年硕士考试试题
% keys 首都师范大学|考研|2011年|物理
% license Copy
% type Tutor
\begin{enumerate}
\item 如图所示,细直杆一端支在地面上,杆与竖直轴的来角为a:质量为m的小环套在杆上,距离轴为r,环与杆之间的滑动摩擦因数为$\mu$。试求,为保证小环做半径为r的稳定圆周运动,杆旋转角速度ω可以在什么范围内变化。
\begin{figure}[ht]
\centering
\includegraphics[width=6cm]{./figures/36c9206edf4b19a7.png}
\caption{} \label{fig_SSD11_1}
\end{figure}
\item 一质量为m的质点自质量为M的光滑斜面顶端静止下滑,滑落到底部时下落高度为h。斜面放在光滑水平面上。忽略所有的摩擦以及碰撞造成的能量损失。求:\\
(1)质点m离开斜面时,斜面M的速度为多少?\\
(2)质点下滑过程中,斜面对它的作用力做的功(不计空气阻力)。
\item 滑动摩擦因数为$\mu$的水平桌面上放置一个半径为R、质量为m的实心球。一水平拉力F作用在球体的中心轴上。为保证球体不发生滑动,水平拉力的最大值为多少?
\item 一长度为L的均匀杆,其一端固定一个质量为m的小球,另一端可以绕光滑水平轴自由摆动。当杆的摆角很小时,试求:\\
(1)如果忽略杆的质量,系统的振动周期;\\
(2)如果杆的质量为M,则系统振动的周期。

\end{enumerate}
