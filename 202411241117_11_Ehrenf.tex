% Ehrenfest 定理
% keys 量子力学|Ehrenfest
% license Usr
% type Wiki


\pentry{量子力学中的基本算符\nref{nod_ecd8}}{nod_d4e4}
对于给定的量子态,有意义的观测量主要为力学量本征值、本征值概率分布以及平均值。由于态矢随时间演化,平均值自然也有其运动方程。
\begin{theorem}{Ehrenfest 定理}
设$A$为任意算符,$\left\langle\right\rangle$表示平均值,则有:
\begin{equation}
\dv{\left\langle A \right\rangle}{t}=\left\langle\frac{\partial A}{\partial t}\right\rangle+\frac{\I}{\hbar}\langle[H, A]\rangle~.
\end{equation}
\end{theorem}

\textbf{证明:}

设$t$时刻态矢为$\ket a$,$\left\langle A\right\rangle$为$\bra a A(t)\ket a$的简写,则运动方程为
\begin{equation}
\begin{aligned}
\dv{\left\langle A \right\rangle}{t}=\dv{\bra a A(t)\ket a}{t}=~.
\end{aligned}
\end{equation}



如果可观察量$A$不随时间演化。且与哈密顿算符对易:
\begin{equation}
\frac{\partial A}{\partial t}=0,\qquad[H, A]=0~,
\end{equation}
那么$A$的期望值不随时间演化,物理上一个可观测量$A$,如果满足
\begin{equation}
\frac{\partial\langle A\rangle}{\partial t}=0~,
\end{equation}
被称为守恒量。
\begin{example}{范数守恒}
最简单的算符是单位算符:
\begin{equation}
A\equiv \hat 1~.
\end{equation}
其对应的可观测量是全体概率之和(当然是1),也是归一化的波函数$\Psi$在希尔伯特空间的范数的平方:
\begin{equation}
\norm{\Psi(\boldsymbol{x}, t)}^2 = \langle\hat{1}\rangle=\int d^{3}x \Psi^{*}(\boldsymbol{x}, t) 1 \Psi(\boldsymbol{x}, t)=1~.
\end{equation}
因为单位算符不随时间变化,且和所有的算符对易,所以范数一定是守恒量。
\begin{equation}
\frac{\partial\langle \hat 1\rangle}{\partial t}=0~,
\end{equation}
\end{example}
\begin{example}{能量守恒}
我们观察哈密顿算符本身:
\begin{equation}
A\equiv H~.
\end{equation}
哈密顿算符对应的可观测量是系统的能量。由于$H$与$H$自身对易,对于一个系统,如果其哈密顿算符不随时间演化,那么能量就是守恒量。
\begin{equation}
\frac{\partial H}{\partial t}=0 \quad\implies\quad \frac{\partial\langle H\rangle}{\partial t}=0~.
\end{equation}
\end{example}
\begin{example}{动量的期望}
我们令$A$表示动量算符:
\begin{equation}
A \equiv \boldsymbol{p} = \frac{\hbar}{\I} \nabla~.
\end{equation}
计算哈密顿算符和动量算符的对易子:
\begin{equation}
[H, \boldsymbol{p}]=\left[\frac{\boldsymbol{p}^2}{2 m}+V(\boldsymbol{x}), \boldsymbol{p}\right]=[V(\boldsymbol{x}), \boldsymbol{p}]=-\frac{\hbar}{\I} \nabla V(\boldsymbol{x})~.
\end{equation}
再带入Ehrenfest 定理,从而得到动量的期望值:
\begin{equation}\label{eq_Ehrenf_1}
\frac{\partial\langle\boldsymbol{p}\rangle}{\partial t}=-\langle\nabla V(\boldsymbol{x})\rangle \equiv\langle\boldsymbol{F}(\boldsymbol{x})\rangle~.
\end{equation}
如果力的期望$\langle\boldsymbol{F}(\boldsymbol{x})\rangle=-\langle\nabla V(\boldsymbol{x})\rangle$为零,则动量的期望值守恒。
\end{example}
\begin{example}{位置的期望}
我们令$A$表示位置算符:
\begin{equation}
A \equiv \boldsymbol{x}~.
\end{equation}
相关的对易子是:
\begin{equation}
[H, \boldsymbol{x}]=\left[\frac{\boldsymbol{p}^2}{2 m}, \boldsymbol{x}\right]=\frac{1}{2 m}\left[p_k p_k, \boldsymbol{x}\right]=\frac{1}{2 m}\left(p_k\left[p_k, \boldsymbol{x}\right]+\left[p_k, \boldsymbol{x}\right] p_k\right)~,
\end{equation}
为了书写简便,这里使用了爱因斯坦求和约定,即对相同指标求和。继续应用动量与位置的对易关系:
\begin{equation}
\left[p_k, x_l\right]=-\I \hbar \delta_{k l}~,
\end{equation}
最终得到:
\begin{equation}
[H, \boldsymbol{x}]=\frac{1}{m} \frac{\hbar}{\I} \boldsymbol{p}~.
\end{equation}
带入Ehrenfest 定理,从而得到位置的期望值:
\begin{equation}\label{eq_Ehrenf_2}
\frac{\partial\langle\boldsymbol{x}\rangle}{\partial t}=\frac{1}{m}\langle\boldsymbol{p}\rangle ~.
\end{equation}
在 Griffiths 的量子力学概论\cite{GriffQ}中,多次用到 \autoref{eq_Ehrenf_2}。
\end{example}

注意:\autoref{eq_Ehrenf_1} 与 \autoref{eq_Ehrenf_2} 和经典运动方程非常相似。但这并不表示,相关的期望值$\langle\boldsymbol{x}\rangle$与$\langle\boldsymbol{p}\rangle$服从经典力学的运动方程。因为位置的期望的力,通常不等于位置的力的期望:
\begin{equation}
\langle\boldsymbol{F}(\boldsymbol{x})\rangle \ne \boldsymbol{F}(\langle\boldsymbol{x}\rangle)~.
\end{equation}