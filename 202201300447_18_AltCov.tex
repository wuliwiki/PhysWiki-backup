% 交错级数的收敛性判别

\pentry{绝对收敛与条件收敛\upref{Convg}}

在词条 绝对收敛与条件收敛\upref{Convg}中已经解释过, 条件收敛级数的收敛性完全是由相邻项的正负抵消导致的, 而其各项绝对值组成的级数并不收敛. 在这种情况下, 仍然有两个判别法能够判断级数是否收敛.

\begin{theorem}{阿贝尔-迪利克雷判别法 (Abel-Dirichlet Test)}
设$\{a_n\}$是单调的实数序列, $\{b_n\}$是复数序列.

\begin{enumerate}
\item 如果$\{a_n\}$单调下降至零, 而且存在常数$M>0$使得
$$
\left|\sum_{n=1}^N b_n\right|\leq M
$$
对于任意$N$都成立, 那么级数$\sum_{n=1}^\infty a_nb_n$收敛.

\item 如果$\{a_n\}$是单调有界序列, 而级数$\sum_{n=1}^\infty b_n$收敛, 那么级数$\sum_{n=1}^\infty a_nb_n$收敛.
\end{enumerate}
\end{theorem}

\textbf{证明.} 命部分和$S_N=\sum_{n=1}^N a_nb_n$, $B_N=\sum_{n=1}^N b_n$. 利用如下的分部求和公式:
$$
S_N=a_NB_N+\sum_{k=1}^{N-1}B_k(a_{k}-a_{k+1}).
$$

\begin{enumerate}
\item 如果$\{a_n\}$单调下降至零, 而且存在常数$M>0$使得
$$
|B_N|=\left|\sum_{n=1}^N b_n\right|\leq M
$$
对于任意$N$都成立, 那么当$N\to\infty$时$a_NB_N\to0$, 而级数
$$
\sum_{k=1}^{\infty}B_k(a_{k}-a_{k+1})
$$
是绝对收敛的, 因为它的一般项绝对值组成的级数由望远镜级数
$$
\sum_{k=1}^{\infty}M|a_{k}-a_{k+1}|
=M\sum_{k=1}^{\infty}(a_{k}-a_{k+1})
$$
所控制 (这里用到了序列$\{a_n\}$是单调下降的). 由此极限$\lim_{N\to\infty}S_N$存在, 从而级数$\sum_{n=1}^\infty a_nb_n$收敛.

\item 如果$\{a_n\}$是单调有界序列 (从而有极限, 设为$a$), 而级数$\sum_{n=1}^\infty b_n$的和为复数$B$, 那么根据与上一小节类似的理由, 可得到级数
$$
\sum_{k=1}^{\infty}B_k(a_{k}-a_{k+1})
$$
的绝对收敛性, 而当$N\to\infty$时$a_NB_N\to aB$. 由此极限$\lim_{N\to\infty}S_N$存在, 从而级数$\sum_{n=1}^\infty a_nb_n$收敛.
\end{enumerate}
这样, 不论在哪种情形, 级数$\sum_{n=1}^\infty a_nb_n$都收敛. \textbf{证毕.}

\begin{example}{某些三角级数的收敛性}
这是一个令人惊讶的事实: 如果$\{a_n\}$是单调递减至零的正数序列, 那么只要$x\neq 2k\pi$, 级数
$$
\sum_{n=1}^\infty a_n \E^{inx}
$$
就是收敛的, 不论级数$\sum_{n=1}^\infty a_n$本身收敛性如何.

实际上, 利用等比级数求和法, $x\neq 2k\pi$时, 部分和序列
$$
\left|\sum_{n=1}^N\E^{inx}\right|
=\left|\frac{1-\E^{i(N+1)x}}{1-\E^{ix}}\right|
\leq\frac{1}{\left|\sin\frac{x}{2}\right|}.
$$
这样, 应用阿贝尔-迪利克雷判别法的第一条, 就知道级数
$$
\sum_{n=1}^\infty a_n \E^{inx}
$$
在$x\neq 2k\pi$时收敛. 当然, 由此得到的收敛速度估计在$x$非常接近$2k\pi$时会非常粗糙.
\end{example}