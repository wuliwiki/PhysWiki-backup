% 克利福德代数(综述)
% license CCBYSA3
% type Wiki

本文根据 CC-BY-SA 协议转载翻译自维基百科\href{https://en.wikipedia.org/wiki/Clifford_algebra}{相关文章}。

在数学中,克利福德代数[a] 是由具备二次型的向量空间生成的一类代数。它是一个带有单位元的结合代数,同时具有一个特定子空间这一附加结构。作为 $K$-代数,它们推广了实数、复数、四元数以及若干其他超复数体系。[1][2] 克利福德代数理论与二次型理论及正交变换理论紧密相关。克利福德代数在几何学、理论物理以及数字图像处理等诸多领域中具有重要应用。该名称源自英国数学家威廉·金登·克利福德(William Kingdon Clifford, 1845–1879)。

在所有克利福德代数中,最为常见的是正交克利福德代数,它们亦称为(伪)黎曼克利福德代数,以区别于辛克利福德代数。[b]

\subsection{引言与基本性质}
克利福德代数是一个包含并由向量空间 $V$ 生成的有单位元的结合代数,其中 $V$ 定义在一个域 $K$ 上,并配备有一个二次型$Q: V \to K$。克利福德代数 $\mathrm{Cl}(V, Q)$ 是满足如下条件的“最自由”[c] 有单位元结合代数:
$$
v^{2} = Q(v) \cdot 1 \quad \text{对所有 } v \in V,~
$$
其中左侧的乘积是代数内部的乘法,而右侧的 $1$ 表示代数的乘法单位元(需注意,不同于域 $K$ 的乘法单位元)。这里“最自由”或“最一般”的含义,可以通过泛性质的概念形式化地加以刻画,如下所述。

当 $V$ 是有限维实向量空间且 $Q$ 非退化时,$\mathrm{Cl}(V, Q)$ 可以记作 $\mathrm{Cl}_{p,q}(\mathbf{R})$。这表示 $V$ 存在一个正交基,其中 $p$ 个基向量满足 $e_i^2 = +1$,而 $q$ 个基向量满足 $e_i^2 = -1$;$\mathbf{R}$ 表明这是定义在实数上的克利福德代数,即代数元素的系数均为实数。该基底可通过正交对角化方法获得。

由向量空间 $V$ 生成的自由代数可写作张量代数$\bigoplus_{n \geq 0} V^{\otimes n}$,即所有 $n$ 重张量积的直和。于是,克利福德代数可表述为该张量代数关于双边理想的商,其中理想由以下形式的元素生成:$v \otimes v - Q(v) \cdot 1, \quad \forall v \in V$.在商代数中,由张量积诱导的乘法通常用并置表示(例如 $uv$)。其结合性直接源于张量积的结合性。

克利福德代数具有一个**特定子空间** $V$,即嵌入映射的像。然而,一般来说,给定一个与克利福德代数同构的 $K$-代数,并不能唯一确定这样的子空间。

若底域 $K$ 中的 2 可逆,则上述基本恒等式可以改写为
$$
uv + vu = 2 \langle u, v \rangle \cdot 1 \quad \forall u, v \in V,~
$$
其中
$$
\langle u, v \rangle = \tfrac{1}{2}\big(Q(u+v) - Q(u) - Q(v)\big)~
$$
是通过极化恒等式由二次型 $Q$ 所关联的对称双线性型。

在此方面,特征为 2 的情形构成一个特殊情况。特别地,当 $\mathrm{char}(K) = 2$ 时,二次型并不一定能够唯一地决定一个满足$Q(v) = \langle v, v \rangle$的对称双线性型。[3] 因此,本文中的许多叙述均附带条件:底域的特征不为 2;若去除此条件,结论即不再成立。
\subsubsection{作为外代数的量子化}
克利福德代数与外代数密切相关。事实上,当 $Q = 0$ 时,克利福德代数 $\mathrm{Cl}(V, Q)$ 正是外代数 $\bigwedge V$。当底域 $K$ 中的 $2$ 可逆时,存在一个典范线性同构,将 $\bigwedge V$ 与 $\mathrm{Cl}(V, Q)$ 联系起来。也就是说,它们作为向量空间是天然同构的,但其乘法结构不同(在特征为 2 的情况下,它们依然是向量空间同构的,只是该同构不再是天然的)。克利福德代数中的乘法结合其特定子空间,比外代数的外积更加丰富,因为它利用了二次型 $Q$ 所提供的额外信息。

克利福德代数是一个滤过代数;其相关的分次代数正是外代数。

更准确地说,克利福德代数可以被看作是外代数的一种量子化(参见量子群),正如魏尔代数是对称代数的量子化一样。

魏尔代数与克利福德代数还具备进一步的 *-代数结构,它们可以统一地理解为超代数的偶次项与奇次项,这一点在 CCR 与 CAR 代数的研究中已有论述。
\subsection{泛性质与构造}
设 $V$ 是定义在域 $K$ 上的一个向量空间,且 $Q : V \to K$ 是 $V$ 上的一个二次型。在大多数感兴趣的情形中,域 $K$ 是实数域 $\mathbf{R}$、复数域 $\mathbf{C}$,或某个有限域。

一个克利福德代数 $\mathrm{Cl}(V, Q)$ 是一对 $(B, i)$[d][4],其中 $B$ 是一个定义在 $K$ 上的有单位元结合代数,$i : V \to B$ 是一个线性映射,并满足$i(v)^2 = Q(v) \cdot 1_B, \quad \forall v \in V$,这里 $1_B$ 表示代数 $B$ 的乘法单位元。其定义依赖于以下泛性质:若给定任意一个定义在 $K$ 上的有单位元结合代数 $A$,以及任意线性映射 $j : V \to A$,满足
$$
j(v)^2 = Q(v) \cdot 1_A, \quad \forall v \in V,~
$$
其中 $1_A$ 表示代数 $A$ 的乘法单位元,则存在唯一的代数同态$f : B \to A$使得下列图表交换(即 $f \circ i = j$)。
\begin{figure}[ht]
\centering
\includegraphics[width=6cm]{./figures/c396abc3eb93542f.png}
\caption{} \label{fig_KLFds_1}
\end{figure}
二次型 $Q$ 可以被(不必对称的[5])双线性型 $\langle \cdot, \cdot \rangle$ 取代,只要它满足$\langle v, v \rangle = Q(v), \quad \forall v \in V$.此时对 $j$ 的等价要求为
$$
j(v) j(v) = \langle v, v \rangle \cdot 1_A, \quad \forall v \in V.~
$$
当域的特征不为 2 时,该条件又可等价改写为:
$$
j(v) j(w) + j(w) j(v) = \big(\langle v, w \rangle + \langle w, v \rangle\big)\cdot 1_A, \quad \forall v, w \in V,~
$$
其中,双线性型在不失一般性的情况下还可以进一步要求为对称双线性型。

如上所述,克利福德代数总是存在的,其构造方法如下:从包含 $V$ 的最一般代数开始,即张量代数 $T(V)$,然后通过取适当的商来强制满足基本恒等式。在此,我们取 $T(V)$ 中由以下元素生成的双边理想 $I_Q$:
$$
v \otimes v - Q(v) \cdot 1, \quad \forall v \in V,~
$$
并定义克利福德代数为该商代数:
$$
\operatorname{Cl}(V, Q) = T(V) / I_Q.~
$$
该商代数所继承的环乘法有时被称为克利福德乘法[6],以区别于外积和数量积。

随后可直接验证,$\operatorname{Cl}(V, Q)$ 包含 $V$,并满足上述的泛性质,因此 $\operatorname{Cl}(V, Q)$ 在唯一同构意义下是唯一的;这也是通常称为“克利福德代数 $\operatorname{Cl}(V, Q)$”的原因。此外,该构造还蕴含了嵌入映射 $i$ 是单射。通常在表述时会省略 $i$,直接将 $V$ 视为 $\operatorname{Cl}(V, Q)$ 的一个线性子空间。

克利福德代数的泛性质刻画表明,$\operatorname{Cl}(V, Q)$ 的构造本质上是函子性的。换言之,$\operatorname{Cl}$ 可以被视为一个函子:从“带二次型的向量空间”范畴(其态射为保持二次型的线性映射)到“结合代数”范畴。泛性质保证了:在保持二次型的前提下,向量空间之间的线性映射可以唯一扩张为相应克利福德代数之间的代数同态。
\subsection{基与维数}
由于向量空间 $V$ 配备了一个二次型 $Q$,在特征不等于 $2$ 的情况下,$V$ 总是存在正交基。正交基的定义是:对于某个对称双线性型,满足
$$
\langle e_i, e_j \rangle = 0, \quad i \neq j,~
$$
以及
$$
\langle e_i, e_i \rangle = Q(e_i).~
$$
克利福德基本恒等式蕴含了对于正交基有:
$$
e_i e_j = - e_j e_i, \quad i \neq j,~
$$
并且
$$
e_i^2 = Q(e_i).~
$$
这使得对正交基向量的操作大为简化。若取一组互不相同的正交基向量的乘积$e_{i_1} e_{i_2} \cdots e_{i_k}$,则总可以通过有限次两两交换将其排成标准顺序,而总体符号由所需交换次数的奇偶性(即置换的符号)决定。

设 $V$ 在域 $K$ 上的维数为 $n$,并且 $\{e_1, \ldots, e_n\}$ 是 $(V, Q)$ 的一个正交基,则克利福德代数 $\operatorname{Cl}(V, Q)$ 在 $K$ 上是自由的,其一组基可以写作:
$$
\big\{\, e_{i_1} e_{i_2} \cdots e_{i_k} \;\big|\; 1 \leq i_1 < i_2 < \cdots < i_k \leq n, \; 0 \leq k \leq n \,\big\}.~
$$
其中,当 $k = 0$ 时的“空乘积”被定义为代数的乘法单位元。对每一个固定的 $k$,基元素的数量为$\binom{n}{k}$,因此克利福德代数的总维数为:
$$
\dim \operatorname{Cl}(V, Q) = \sum_{k=0}^n \binom{n}{k} = 2^n.~
$$
\subsection{例子:实克利福德代数与复克利福德代数}
最重要的克利福德代数是那些定义在实向量空间与复向量空间上的,并且配备有非退化二次型的情形。

每一个代数 $\mathrm{Cl}_{p,q}(\mathbf{R})$ 与 $\mathrm{Cl}_{n}(\mathbf{C})$ 都同构于 $A$ 或 $A \oplus A$,其中 $A$ 是一个满矩阵环,其元素取自 $\mathbf{R}$、$\mathbf{C}$ 或 $\mathbf{H}$(四元数)。关于这些代数的完整分类,可参见 克利福德代数的分类。
\subsubsection{实数情形}
克利福德代数有时也被称为几何代数,尤其是在实数域上的情形。

任意有限维实向量空间上的非退化二次型,都与如下标准对角形式等价:
$$
Q(v) = v_1^2 + \cdots + v_p^2 - v_{p+1}^2 - \cdots - v_{p+q}^2,~
$$
其中 $n = p + q$ 是向量空间的维数。整数对 $(p, q)$ 被称为该二次型的符号型。带有此二次型的实向量空间通常记作 $\mathbb{R}^{p,q}$。其上的克利福德代数记作$\mathrm{Cl}_{p,q}(\mathbf{R})$。符号 $\mathrm{Cl}_n(\mathbf{R})$ 则指 $\mathrm{Cl}_{n,0}(\mathbf{R})$ 或 $\mathrm{Cl}_{0,n}(\mathbf{R})$,这取决于作者倾向于使用正定空间还是负定空间。

$\mathbf{R}^{p,q}$ 的一个标准基 $\{e_1, \ldots, e_n\}$ 由 $n = p+q$ 个两两正交向量组成,其中有 $p$ 个基向量平方等于 $+1$,另有 $q$ 个基向量平方等于 $-1$。因此,在 $\mathrm{Cl}_{p,q}(\mathbf{R})$ 中,就有 $p$ 个平方为 $+1$ 的向量与 $q$ 个平方为 $-1$ 的向量。

若干低维情形
\begin{itemize}
\item $\mathrm{Cl}_{0,0}(\mathbf{R})$ 自然同构于 $\mathbf{R}$,因为不存在非零向量。
\item $\mathrm{Cl}_{0,1}(\mathbf{R})$ 是一个由平方等于 $-1$ 的 $e_1$ 所生成的二维代数,同构于复数域 $\mathbf{C}$。
\item $\mathrm{Cl}_{1,0}(\mathbf{R})$ 是一个由平方等于 $+1$ 的 $e_1$ 所生成的二维代数,同构于裂复数。
\item $\mathrm{Cl}_{0,2}(\mathbf{R})$ 是一个四维代数,由 $\{1, e_1, e_2, e_1 e_2\}$ 张成。后三个基元素的平方均为 $-1$,并且它们两两反对易,因此该代数同构于四元数代数 $\mathbb{H}$。
\item $\mathrm{Cl}_{2,0}(\mathbf{R}) \cong \mathrm{Cl}_{1,1}(\mathbf{R})$,同构于裂四元数代数。
\item $\mathrm{Cl}_{0,3}(\mathbf{R})$ 是一个八维代数,同构于直和 $\mathbf{H} \oplus \mathbf{H}$,即裂双四元数代数。
\item $\mathrm{Cl}_{3,0}(\mathbf{R}) \cong \mathrm{Cl}_{1,2}(\mathbf{R})$,亦称泡利代数[7][8],同构于双四元数代数。
\end{itemize}
\subsubsection{复数情形}
同样可以研究复向量空间上的克利福德代数。任意一个维数为 $n$ 的复向量空间上的非退化二次型,都与如下标准对角形式等价:
$$
Q(z) = z_1^2 + z_2^2 + \cdots + z_n^2.~
$$
因此,对于每一个维数 $n$,在同构意义下,带有非退化二次型的复向量空间仅对应唯一一个克利福德代数。我们将配备标准二次型的 $\mathbf{C}^n$ 上的克利福德代数记作 $\mathrm{Cl}_n(\mathbf{C})$。

对于若干低维情形,可以得到:
\begin{itemize}
\item $\mathrm{Cl}_0(\mathbf{C}) \cong \mathbf{C}$,即复数域;
\item $\mathrm{Cl}_1(\mathbf{C}) \cong \mathbf{C} \oplus \mathbf{C}$,即双复数;
\item $\mathrm{Cl}_2(\mathbf{C}) \cong M_2(\mathbf{C})$,即双四元数,
\end{itemize}
其中 $M_n(\mathbf{C})$ 表示所有 $n \times n$ 复矩阵所构成的代数。
\subsection{例子:构造四元数与对偶四元数}
\subsubsection{四元数}
在本节中,哈密顿的四元数将被构造为克利福德代数 $\mathrm{Cl}_{3,0}(\mathbf{R})$ 的偶子代数。

设向量空间 $V$ 为实三维空间 $\mathbf{R}^3$,其二次型为通常的欧几里得二次型。于是,对于 $v, w \in \mathbf{R}^3$,我们有双线性型(即数量积):
$$
v \cdot w = v_1 w_1 + v_2 w_2 + v_3 w_3.~
$$
现在引入向量 $v, w$ 的克利福德乘积:
$$
vw + wv = 2 (v \cdot w).~
$$
记 $\mathbf{R}^3$ 的一组正交单位向量为 $\{e_1, e_2, e_3\}$,则克利福德乘法给出以下关系:
$$
e_2 e_3 = - e_3 e_2, \quad 
e_1 e_3 = - e_3 e_1, \quad 
e_1 e_2 = - e_2 e_1,~
$$
以及
$$
e_1^2 = e_2^2 = e_3^2 = 1.~
$$
因此,克利福德代数 $\mathrm{Cl}_{3,0}(\mathbf{R})$ 的一个一般元素可写为:
$$
A = a_0 + a_1 e_1 + a_2 e_2 + a_3 e_3 + a_4 e_2 e_3 + a_5 e_1 e_3 + a_6 e_1 e_2 + a_7 e_1 e_2 e_3,~
$$
其中 $a_0, a_1, \ldots, a_7 \in \mathbf{R}$。

偶次数元素的线性组合构成了 $\mathrm{Cl}_{3,0}(\mathbf{R})$ 的偶子代数$\mathrm{Cl}^{[0]}_{3,0}(\mathbf{R})$,其一般元素为:
$$
q = q_0 + q_1 e_2 e_3 + q_2 e_1 e_3 + q_3 e_1 e_2,~
$$
其中 $q_0, q_1, q_2, q_3 \in \mathbf{R}$。

基元素可与四元数的标准基 $i, j, k$ 对应如下:
$$
i = e_2 e_3, \quad j = e_1 e_3, \quad k = e_1 e_2,~
$$
这表明偶子代数 $\mathrm{Cl}^{[0]}_{3,0}(\mathbf{R})$ 正是哈密顿的实四元数代数。

为了验证这一点,计算如下:
$$
i^2 = (e_2 e_3)^2 = e_2 e_3 e_2 e_3 = - e_2 e_2 e_3 e_3 = -1,~
$$
以及
$$
ij = e_2 e_3 e_1 e_3 = - e_2 e_3 e_3 e_1 = - e_2 e_1 = e_1 e_2 = k.~
$$
最后:
$$
ijk = e_2 e_3 e_1 e_3 e_1 e_2 = -1.~
$$
\subsubsection{对偶四元数}
在本节中,对偶四元数将被构造为实四维空间上某个带有退化二次型的克利福德代数的偶子代数。[9][10]

设向量空间 $V$ 为实四维空间 $\mathbf{R}^4$,其二次型 $Q$ 为由 $\mathbf{R}^3$ 上欧几里得度量导出的一个退化型。对 $v, w \in \mathbf{R}^4$,定义退化双线性型:
$$
d(v, w) = v_1 w_1 + v_2 w_2 + v_3 w_3.~
$$
该退化的数量积实质上是将 $\mathbf{R}^4$ 中的距离测度投影到 $\mathbf{R}^3$ 的超平面上。

向量 $v, w$ 的克利福德乘积定义为:
$$
vw + wv = - 2\, d(v, w).~
$$
注意,这里的负号是特意引入的,以便简化与四元数的对应关系。

记 $\mathbf{R}^4$ 的一组两两正交单位向量为 $\{e_1, e_2, e_3, e_4\}$,则克利福德乘法给出以下关系:
$$
e_m e_n = - e_n e_m, \quad m \neq n,~
$$
以及
$$
e_1^2 = e_2^2 = e_3^2 = -1, \quad e_4^2 = 0.~
$$
$\operatorname{Cl}(\mathbf{R}^4, d)$ 的一般元素共有 16 个分量。其偶次数元素的线性组合定义了偶子代数 $\operatorname{Cl}^{[0]}(\mathbf{R}^4, d)$,其一般元素为:
$$
H = h_0 + h_1 e_2 e_3 + h_2 e_3 e_1 + h_3 e_1 e_2 + h_4 e_4 e_1 + h_5 e_4 e_2 + h_6 e_4 e_3 + h_7 e_1 e_2 e_3 e_4,~
$$
其中 $h_0, \ldots, h_7 \in \mathbf{R}$。

这些基元素可以与四元数的基 $i, j, k$ 以及对偶单位 $\varepsilon$ 对应如下:
$$
i = e_2 e_3, \quad j = e_3 e_1, \quad k = e_1 e_2, \quad \varepsilon = e_1 e_2 e_3 e_4.~
$$
这建立了 $\operatorname{Cl}^{[0]}_{0,3,1}(\mathbf{R})$ 与对偶四元数代数之间的对应关系。

为了验证这一点,计算如下:
$$
\varepsilon^2 = (e_1 e_2 e_3 e_4)^2 = e_1 e_2 e_3 e_4 e_1 e_2 e_3 e_4 
= - e_1 e_2 e_3 (e_4 e_4) e_1 e_2 e_3 = 0,~
$$
以及
$$
\varepsilon i = (e_1 e_2 e_3 e_4) e_2 e_3 = e_1 e_2 e_3 e_4 e_2 e_3 
= e_2 e_3 (e_1 e_2 e_3 e_4) = i \varepsilon.~
$$
交换 $e_1$ 与 $e_4$ 时符号会交替出现,但总次数为偶数,因此对偶单位 $\varepsilon$ 与四元数基 $i, j, k$ 可交换。
\subsection{例子:低维情形}
设 $K$ 是一个特征不为 2 的任意域。
\subsubsection{维数 1}
当 $\dim V = 1$ 时,若二次型 $Q$ 可对角化为 $\operatorname{diag}(a)$,即存在非零向量 $x$ 使得 $Q(x) = a$,则 $\operatorname{Cl}(V, Q)$ 代数同构于一个由元素 $x$ 所生成的 $K$-代数,其中$x^2 = a$,即二次代数$K[X] / (X^2 - a)$.特别地,若 $a = 0$(即 $Q$ 是零二次型),则 $\operatorname{Cl}(V, Q)$ 代数同构于 对偶数代数$K[\varepsilon]/(\varepsilon^2)$。

若 $a$ 是 $K$ 中的非零平方数,则$\operatorname{Cl}(V, Q) \simeq K \oplus K$.

否则,$\operatorname{Cl}(V, Q)$ 同构于 $K$ 的二次域扩张 $K(\sqrt{a})$。
\subsubsection{维数 2}
当 $\dim V = 2$ 时,若二次型 $Q$ 可对角化为 $\operatorname{diag}(a, b)$,其中 $a, b \neq 0$(若 $Q$ 非退化,则这种对角化总是存在),则 $\operatorname{Cl}(V, Q)$ 同构于一个由元素 $x, y$ 生成的 $K$-代数,满足:$x^2 = a, \quad y^2 = b, \quad xy = - yx$.

因此:$\operatorname{Cl}(V, Q) \simeq (a, b)_K$,即(广义)四元数代数。当 $a = b = -1$ 且 $K = \mathbb{R}$ 时,我们正好得到哈密顿的四元数代数:$\mathbb{H} = (-1, -1)_{\mathbb{R}}$.

作为一个特例,若存在某个 $x \in V$ 满足 $Q(x) = 1$,则有$\operatorname{Cl}(V, Q) \simeq M_2(K)$,其中 $M_2(K)$ 表示所有 $2 \times 2$ 的 $K$-矩阵代数。
\subsection{性质}
\subsubsection{与外代数的关系}
给定一个向量空间 $V$,我们可以构造其外代数 $\bigwedge V$,其定义不依赖于 $V$ 上的任何二次型。事实表明:若域 $K$ 的特征不等于 2,则作为向量空间,$\bigwedge V$ 与 $\operatorname{Cl}(V, Q)$ 之间存在一个天然同构(在特征为 2 的情况下,同构仍然存在,但不一定是天然的)。该同构在代数层面上成立的充要条件是 $Q = 0$。因此,可以将克利福德代数 $\operatorname{Cl}(V, Q)$ 视为外代数 $\bigwedge V$ 的一种丰富化(更准确地说,是一种量子化;参见前文的引言),其乘法依赖于二次型 $Q$(尽管外积仍可在不依赖于 $Q$ 的情况下独立定义)。

建立这一同构的最简单方式是:在 $V$ 中选取一个正交基 $\{e_1, \ldots, e_n\}$,并将其扩展为 $\operatorname{Cl}(V, Q)$ 的基(如前所述)。则映射$\operatorname{Cl}(V, Q) \;\longrightarrow\; \bigwedge V$由下式决定:$e_{i_1} e_{i_2} \cdots e_{i_k} \;\mapsto\; e_{i_1} \wedge e_{i_2} \wedge \cdots \wedge e_{i_k}$.

需要注意的是,这一构造仅在基 $\{e_1, \ldots, e_n\}$ 为正交基时有效。但可以证明,该映射与正交基的具体选择无关,因此它给出了一个天然的同构。

若域 $K$ 的特征为 $0$,则还可以通过反对称化来建立该同构。定义函数$f_k : V \times \cdots \times V \;\longrightarrow\; \operatorname{Cl}(V, Q)$为
$$
f_k(v_1, \ldots, v_k) = \frac{1}{k!} \sum_{\sigma \in S_k} \operatorname{sgn}(\sigma)\, v_{\sigma(1)} \cdots v_{\sigma(k)},~
$$
其中求和遍历对称群 $S_k$。由于 $f_k$ 是交替的,它诱导出一个唯一的线性映射$\bigwedge\nolimits^k V \;\longrightarrow\; \operatorname{Cl}(V, Q)$.这些映射的直和给出了一个 $\bigwedge V$ 与 $\operatorname{Cl}(V, Q)$ 之间的线性映射。可以证明,该映射是一个线性同构,并且是天然的。

更高层次的理解方式是:在 $\operatorname{Cl}(V, Q)$ 上构造一个滤过。回忆张量代数 $T(V)$ 具有一个自然的滤过:$F_0 \subset F_1 \subset F_2 \subset \cdots$,其中 $F^k$ 包含所有阶数 $\leq k$ 的张量的和。将其投射到克利福德代数上,可以得到 $\operatorname{Cl}(V, Q)$ 的一个滤过。于是相关的分次代数为
$$
\operatorname{Gr}_F \operatorname{Cl}(V, Q) = \bigoplus_k F^k / F^{k-1},~
$$
它天然同构于外代数 $\bigwedge V$。由于一个滤过代数的相关分次代数总是与该滤过代数在滤过向量空间意义下同构(方法是对每个 $k$ 选择 $F^k$ 在 $F^{k+1}$ 中的补空间),因此这一方法在任意特征下(包括特征为 2 的情形)都给出了一个同构,尽管该同构不一定是天然的。
