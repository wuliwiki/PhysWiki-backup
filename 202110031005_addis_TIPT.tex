% 不含时微扰理论

\begin{issues}
\issueDraft
\end{issues}

\footnote{参考 \cite{GriffQ} 相关章节.}不含时微扰理论.

\begin{equation}\label{TIPT_eq3}
H = H_0 + \lambda H^1
\end{equation}
\begin{equation}
E_n = E_n^0 + \lambda E_n^1 + \lambda^2 E_n^2 + \dots
\end{equation}
令 $\psi_n^0$ 是 $H_0$ 的任意一组完备正交归一基底.
\begin{equation}
\psi_n = \psi_n^0 + \lambda\psi_n^1 + \lambda^2 \psi_n^2 + \dots
\end{equation}
\begin{equation}
H \psi_n = E_n \psi_n
\end{equation}
令 $\lambda \to 0$, 忽略 $\order{\lambda^2}$. 有
\begin{equation}
H^0\psi_n^1 + H^1 \psi_n^0 = E_n^0 \psi_n^1 + E_n^1 \psi_n^0
\end{equation}
要使其恒成立, 就要求投影到任意 $\psi_m^0$ 上都成立:
\begin{equation}\label{TIPT_eq1}
\mel{\psi_m^0}{H^0}{\psi_n^1} + \mel{\psi_m^0}{H^1}{\psi_n^0} = E_n^0 \braket{\psi_m^0}{\psi_n^1} + E_n^1 \braket{\psi_m^0}{\psi_n^0}
\end{equation}
第一项利用厄米算符的性质
\begin{equation}
\mel{\psi_m^0}{H^0}{\psi_n^1} = \braket{H^0\psi_m^0}{\psi_n^1} = E_m^0\braket{\psi_m^0}{\psi_n^1}
\end{equation}
所以\autoref{TIPT_eq1} 化简为
\begin{equation}\label{TIPT_eq2}
\mel{\psi_m^0}{H^1}{\psi_n^0} = (E_n^0 - E_m^0) \braket{\psi_m^0}{\psi_n^1} + E_n^1 \delta_{m,n}
\end{equation}

\subsection{非简并情况}
$H_0$ 非简并时, 当 $m\ne n$ 就有 $E_n^0 - E_m^0$. 所以右边第一项对角线元素全为零. 所以 $m = n$ 时候有
\begin{equation}\label{TIPT_eq6}
E_n^1 = \mel{\psi_n^0}{H^1}{\psi_n^0}
\end{equation}
对 $m \ne n$ 的元素有
\begin{equation}\label{TIPT_eq4}
\braket{\psi_m^0}{\psi_n^1} = \frac{\mel{\psi_m^0}{H^1}{\psi_n^0}}{E_n^0 - E_m^0}
\end{equation}
于是, 要满足\autoref{TIPT_eq2}, 只需要令
\begin{equation}\label{TIPT_eq5}
\psi_n^1 = \sum_m^{m\ne n} \braket{\psi_m^0}{\psi_n^1} \psi_m^0
\end{equation}
即可. 注意 \autoref{TIPT_eq3} 再加上任意 $c \psi_n^0$ 同样能使\autoref{TIPT_eq2} 成立, 说明\autoref{TIPT_eq2} 的解不止一个.

\subsection{简并情况}
$H_0$ 简并时, 当 $m\ne n$ 也未必有 $E_n^0 - E_m^0$, 这使得\autoref{TIPT_eq4} 和\autoref{TIPT_eq5} 失效(分母为零), 所以要重新推导一次.

注意我们只需要通过某种方法找到\autoref{TIPT_eq2} 的某个解即可. 既然简并, 那么每个简并子空间中正交归一基底都是有一定自由度的. 为了让问题更简单, 我们在每个子空间中也要求 $\psi_n^0$ 基底下 $H^1$ 是对角化的, 即 $\mel{\psi_n^0}{H^1}{\psi_n^0}$ 的对角块都是对角化的(每个块是第 $i,j$ 个子空间之间的耦合). 注意这并不要求 $[H^0, H^1] = 0$ 对易, 因为对易要求 $\mel{\psi_n^0}{H^1}{\psi_n^0}$ 彻底对角化. 这样, 通常就能在简并子空间中唯一地确定 $\psi_n^0$.

现在对\autoref{TIPT_eq2} 考虑对角块, 有 $E_n^0 = E_m^0$, 所以同样有\autoref{TIPT_eq6}, 只是现在, 这就是对角块中的对角元, 也就是 $H^1$ 在该简并子空间中的本征值. 在同一个简并子空间中, 为了简洁我们也可以令 $\braket{\psi_m^0}{\psi_n^1} = 0$.

在非对角子空间中, 有 $m\ne n$ 且 $E_n^0 \ne E_m^0$, 所以同样有\autoref{TIPT_eq4}, 和\autoref{TIPT_eq5}, 的性质, 但求和时要限制在非对角块进行, 即不仅要限制 $m \ne n$ 还要求 $E_m^0 \ne E_m^0$. 所以
\begin{equation}
\psi_n^1 = \sum_m^{E_m\ne E_n} \braket{\psi_m^0}{\psi_n^1} \psi_m^0
\end{equation}


============= 回收内容 ===================

若没有微扰时, 波函数处于非简并的束缚态 $\ket{\psi_n^0}$, 且能量为 $E_n^0$. 加入不含时微扰 $H'$ 后, 一阶近似下该束缚态能量变为
\begin{equation}
E_n = E_n^0 + E_n^1
\end{equation}
波函数变为
\begin{equation}
\psi_n = \psi_n^0 + \psi_n^1
\end{equation}
其中
\begin{equation}
E_n^1 = \mel{\psi_n^0}{H'}{\psi_n^0}
\end{equation}
\begin{equation}
\psi_n^1 = \sum_{m \ne n} \frac{\mel{\psi_m^0}{H'}{\psi_n^0}}{E_n^0 - E_m^0} \psi_m^0
\end{equation}

\subsection{简并情况}
\addTODO{这是错的}
假设 $[H, H'] = 0$, 那么在每个简并子空间中用存在一组共同本征矢. 这就叫做 “好” 量子态. 要求好量子态的本征问题, 令 $H$ 在该子空间中的本征态为 $\psi_n^0$, 那么 $H'_{i,j} = \mel*{\psi_i^0}{H'}{\psi_j^0}$. 解出 $N$ 个本征矢和本征值, 就是能量和波函数的一阶修正.

另一个术语叫做\textbf{好量子数(good quantum number)}, 就是好量子态对应的量子数(本征值的编号). 如果好量子态是一个常见物理量的算符 $A$ 的本征态, 也就是说 $H, H', A$ 两两对易, 且 $A$ 在当前简并子空间中并不简并, 那么 $A$ 的量子数就是好量子数.
