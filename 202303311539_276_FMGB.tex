% 高斯光束 2
% keys 基模高斯光束

\begin{issues}
\issueDraft
\end{issues}

\pentry{电场波动方程\upref{EWEq}}

高斯光束是麦克斯韦方程组的一组解,其电矢量可以分离时间和空间部分,即$\boldsymbol E(\boldsymbol r,t)=\boldsymbol E_0(\boldsymbol r)\exp\{i\omega t\}$,空间解$\boldsymbol  E_0$满足亥姆霍兹方程
\begin{equation}
\Delta \boldsymbol E_0 + k^2 \boldsymbol E_0=0~.
\end{equation}
其中波矢$k=2\pi/\lambda$。
在缓变振幅近似下,空间解可以表示成振幅函数$\psi(x,y,z)$与平面波解的乘积形式
\begin{equation}\label{FMGB_eq2}
E_0(x,y,z) = \psi(x,y,z)\exp\{-ikz\}
\end{equation}
这里的振幅函数一般是一个沿着$z$轴缓慢变化的复函数,表示高斯光束与平面波的差异,如:不均匀的强度分布、光束的发散、波阵面的弯曲等。将代入
 


\begin{equation}\label{FMGB_eq1}
\frac{\partial^2 E}{\partial x^2} +\frac{\partial^2 E}{\partial y^2} -ik 
\end{equation}

