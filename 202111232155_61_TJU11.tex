% 天津大学 2011 年考研量子力学
% 考研|天津大学|量子力学|2011

\subsection{ }
\begin{enumerate}
\item 质量为$m$,频率为$\omega$的谐振子,初始时刻处于状态$Ax\varPsi_{n}(x)$,求归一化系数$A$,任意时刻的波函数以及坐标的平均值.
\item 写出历史上确定光的波粒二象性的主要实验,指出至少两个可以验证实物粒子具有波动性的实验.
\end{enumerate}
\subsection{ }
\begin{enumerate}
\item 设$\hat{F}$是任意一个算符,尝试用其构造两个厄米算符.
\item 利用角动量各分量$\hat{L}_{x}$,$\hat{L}_{y}$,$\hat{L}_{z}$之间的对易关系,证明:在任意分量的本征态下,其他两个分量的平均值为0.
\item 将指数$e^{i\alpha \sigma_{y}}$
\end{enumerate}