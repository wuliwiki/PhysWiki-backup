% 温度、温标
% keys 温度|温标|开尔文温标|绝对温度|摄氏度
% license Xiao
% type Tutor

\pentry{理想气体状态方程\upref{PVnRT}}

\subsection{用理想气体定义}
对于气体而言, 温度越高意味着气体分子速度越大(\autoref{eq_PVnRT_3}~\upref{PVnRT}), 而对于固体, 温度越高说分振动越剧烈。 在理想气体模型中, 我们看到温度与气体分子的平均动能成正比。 
\begin{equation}
\bar E_k = \frac{3}{2} k_B T_K~,
\end{equation}
这样我们就在微观上定义了\textbf{热力学温标}(单位是开尔文, $K$, 国际单位的一种)。 当分子动能为 0 时, 热力学温度就是 $0 \Si{K}$, 即\textbf{绝对零度}。 根据这个定义, 最低的可能温度就是绝对零度\footnote{统计力学中的确有负温度这种说法, 但根据定义, 它的温度反而比任何温度要高。}。

在生活中, 我们一般使用\textbf{摄氏温标}或\textbf{华氏温标}来表示温度, 它们的单位分别记 $^\circ\Si{C}$, $^\circ\Si{F}$, 我们以下用 $T_C$ 和 $T_F$ 表示。 三种温标的转换关系如下
\begin{equation}
T_K = T_C + 273.15^\circ\Si{C}~,
\end{equation}
\begin{equation}
T_C = \frac{5}{9}(T_F - 32^\circ\Si{F})~.
\end{equation}
注意一开尔文和一摄氏度的大小一样, 只是相差了一个常数。 零下 $273.15$ 摄氏度就是绝对零度。

\subsection{卡诺定理使得存在普适温标}
根据卡诺定理的推论,任何一个工作于两个一定温度之间的可逆卡诺热机效率都仅与温度有关。由热机效率 $\eta = \frac{W}{Q_1}=1-\frac{Q_2}{Q_1}$,可以定义 $\theta_1$ 与 $\theta_2$ 的普适函数 $F(\theta_1, \theta_2) = \frac{Q_2}{Q_1}$,即这与 $Q_1$、$Q_2$ 等无关,仅与 $\theta_1$、$\theta_2$ 有关。同时,可以有 $\theta$ 的普适函数 $f(\theta)$ 使得 $F(\theta_1, \theta_2) = f(\theta_2)/f(\theta_1)$。

具体的,考虑一个可逆卡诺热机 $A$ 工作于 $\theta_1$、$\theta_2$ 之间,可逆卡诺热机 $B$ 工作于 $\theta_2$、$\theta_3$ 之间。$A$ 机从 $\theta_1$ 吸收 $Q_1$ 热量,做功 $W_1$ 后向 $\theta_2$ 放出 $Q_2$ 热量。$B$ 机从 $\theta_2$ 吸收 $Q_2$ 热量,做功 $W_2$ 后向 $\theta_3$ 放出 $Q_3$ 热量。那么有:

\begin{equation}
\begin{aligned}
\frac{Q_2}{Q_1} &= F(\theta_1, \theta_2)~;\\
\frac{Q_3}{Q_2} &= F(\theta_2, \theta_3)~;\\
\frac{Q_3}{Q_1} &= F(\theta_1, \theta_3)~.
\end{aligned}
\end{equation}
也就可以得到 $F(\theta_1, \theta_2) = \frac{F(\theta_1, \theta_3)}{F(\theta_2, \theta_3)}$,而 $\theta_3$ 可以是任意温度,故可以选取 $F(\theta_1, \theta_2) = f(\theta_2)/f(\theta_1)$,也就完成了证明。这 $f(\theta)$ 即为一普适温标。这最先是由开尔文引进的,称作\textbf{热力学温标}或开尔文温标。

\subsection{一般热力学系统的温度,用熵定义}
根据热平衡、热力学第零定律\upref{TherEq},我们知道不止理想气体有温度这一热力学量,任意处于\textbf{平衡态的热力学系统}都有温度这一热力学量。一种可能的定义方式是将它们与一定温度的理想气体进行接触,若它们能够达到平衡(之间没有传热),那么它们就有同样的温度,我们就能利用已知的理想气体的温度来确定未知的系统的温度。这种方式显然是不够通用的,我们需要寻求\textbf{温度}的更加通用的定义。

在热力学和统计力学中,更广义的温度是从熵的角度定义的:
\begin{equation}\label{eq_tmp_1}
\frac{1}{T} = \left(\pdv{S}{U}\right)_{V,N} ~.
\end{equation}
这可以从平衡态热力学系统的\textbf{内能}全微分表达式(热力学关系式\upref{MWRel})
\begin{equation}
\dd U=T\dd S-p\dd V+\mu \dd N~.
\end{equation}
看出,$(\partial U/\partial S)_{V,N}=T$,从而得到了\autoref{eq_tmp_1} 的结果。

由此可以推出理想气体的温度定义,根据理想气体的熵:纯微观分析\upref{IdeaS} 的结果。其推导基于玻尔兹曼熵公式,由系统的内能、体积、粒子数推出平衡态时粒子的速率分布及微观状态数,从而得到熵的公式,过程中并没有用到温度概念。考察一个\textbf{粒子数 $N$ 确定}的平衡态理想气体系统(对于粒子数不确定的系统,则需要将全同粒子假设考虑进来并相应地改写熵的表达式,才能解决 Gibbs 佯谬,\autoref{eq_tmp_1} 才是有效的,参考玻尔兹曼分布(统计力学)\upref{MBsta}。),那么它的熵可以表示为
\begin{equation}
S=k\ln \Omega =k\left[ N\ln V+\frac{3}{2}N\ln U \right] +k\ln F\left( m,N,a,b \right) ~,
\end{equation}
其中 $F(m,N,a,b)$ 是与 $U,V$ 无关的函数。因此对于这个粒子数 $N$ 确定的系统,我们有
\begin{equation}
U\cdot \left(\frac{\partial S}{\partial U}\right)_{V,N}=\frac{3}{2}Nk~.
\end{equation}
因此由\autoref{eq_tmp_1} 定义的温度满足理想气体状态方程
\begin{equation}
U=\frac{3}{2}NkT~.
\end{equation}
