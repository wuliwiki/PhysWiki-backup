% 算符的矩阵表示
% keys 算符|量子力学|矩阵

\begin{issues}
\issueMissDepend
\issueOther{这是事实上是线性代数的内容,并没有涉及到量子力学}
\end{issues}

%未完成
%(一定要强调一下计算矩阵元的方法 \[{Q_{ij}} = \left\langle {{\psi_i}} \right|\Q Q\left| {{\psi_j}} \right\rangle \])

考虑一个比较基本的问题, 算符的“功能”是什么呢? 算符就是对函数的一种操作方法。给出一个波函数, 经过算符作用, 可以得到一个新的波函数。以下给出量子力学中算符的两个重要性质
\begin{enumerate}
\item 算符都是线性的, 即对任意 $n$ 个波函数 $\psi_1, \psi_2 \dots \psi_n$,  算符 $\Q Q$ 满足
\begin{equation}
\Q Q (c_1 \psi_1 + c_2 \psi_2 \dots c_n \psi_n) = c_1\Q Q \psi_1 + c_2 \Q Q \psi_2 \dots c_n \Q Q \psi_n~.
\end{equation}
\item 算符的本征方程的本征值都是实数。因为根据测量理论, 本征值就是可能出现的测量结果, 所以本征值一定是实数。
\end{enumerate}

我们已经知道, 波函数可以用列矢量表示%链接未完成,词条在哪里?
。既然算符都是线性的, 而矩阵可以表示列矢量的线性变换, 是否可以用矩阵代替算符, 从而作用于列向量呢?根据性质 $1$,  若 $\psi$ 是算符是算符 $\Q Q$ 的本征函数, $\lambda_1 \dots \lambda_n$ 是对应的本征值(实数), 则
\begin{equation}
\begin{cases}
\Q Q  \psi & = \Q Q(c_1 \psi_1 + \dots + c_n \psi_n)\\
& = c_1 \Q Q \psi_1 + \dots + c_n \Q Q \psi_n\\
& = \lambda_1 c_1 \psi_1 + \dots + \lambda_n c_n \psi_n
\end{cases}
\end{equation}
若把上面的波函数表示成列矢量,就相当于在算符 $\Q Q$ 的作用下任意一个列矢量 $\ket{\psi}  = (c_1, \dots, c_n)\Tr$ 总是会变成 $(\lambda_1 c_1, \dots, \lambda_n c_n)\Tr$。 这个变换可以用矩阵
\begin{equation}
\mat Q = \pmat{
\lambda_1 &  & \\
 & \ddots & \\
& & \lambda_n}
\end{equation}
来表示,即
\begin{equation}
\pmat{\lambda_1 & & \\ & \ddots & \\  & & \lambda_n}
\pmat{c_1\\ \vdots \\c_n} 
= \pmat{\lambda_1 c_1\\  \vdots \\ \lambda_n c_n}
\end{equation}
所以矩阵 $\mat Q$ 就是算符 $\Q Q$ 的矩阵形式, 把算符作用在波函数上得到新的波函数, 等效于把算符对应的矩阵作用在波函数对应的列矢量上, 得到新的波函数对应的列矢量。

用矩阵和列向量表示的本征方程如下
\begin{equation}
\mat Q \ket{\psi}  = \lambda \ket{\psi} 
\end{equation}
解得 $\lambda  = \lambda_i$ 时, $\ket{\psi} = \ket{\psi_1}  = (0, \dots, 1, \dots, 0)\Tr$ (只有第 $i$ 个分量等于 1,其余分量等于 0),而 $\ket{\psi_i}$ 正是波函数 $\psi_i$ 对应的列向量。

\subsection{在任意基底中的矩阵}
上面的讨论中用矩阵 $\mat Q$ 表示算符 $\Q Q$, 其局限性在于,只能使用 $\Q Q$ 的本征函数 $\psi_1 \dots \psi_n$ 作为基底。现在若用其他基底(正交归一的) $\phi_1 \dots \phi_n$, 能否求出算符 $\Q Q$ 对应的矩阵 $\mat Q_1$ 呢?

下面讨论中, 为了避免混淆, 用 $\ket{f}_\phi$ 表示波函数 $f$ 以 $\phi_1 \dots \phi_n$ 为基底的列矢量, $\ket{f}_\psi$ 表示波函数 $f$ 以 $\psi_1 \dots \psi_n$ 为基底的列矢量。

现取任意一波函数 $f$,  $\ket{f}_\psi = (c_1\dots c_n)\Tr$, $\ket{f}_\phi = (d_1\dots d_n)\Tr$。 虽然它们表示同一个波函数 $f$,  但是由于选取的基底不同, 列向量也不同。下面讨论它们之间的变换关系。

若把 $f$ 按 $\phi_i$ 展开, 有
\begin{equation}\ali{
\int \phi_i^*f \dd{x} & = \int \phi_i^* ( d_1 \phi_1 + \ldots + d_n \phi_n) \dd{x} = \sum_{j = 1}^n d_j \int \phi_i^*{\phi_j} \dd{x}\\
&= \sum_{j=1}^n d_j \delta_{ij} = d_i
}\end{equation}
若把 $f$ 按 $\psi_i$ 展开, 有
\begin{equation}
d_i = \int \phi_i^*f \dd{x} = \int \phi_i^* (c_1 \psi_1 + \dots + c_n \psi_n)   \dd{x} = \sum_{j = 1}^n c_j \int \phi_i^*{\psi_j} \dd{x}
\end{equation}
上式用矩阵和列矢量表示, 即 $(d_1 \dots d_n)\Tr = \mat P (c_1 \dots c_n)\Tr$,  即 ${\left| f \right\rangle_\phi } = P{\left| f \right\rangle_\psi }$。 其中 $P$ 矩阵的矩阵元 ${P_{ij}} = \int {\phi_i^*{\psi_j}dx} $。  $\mat P$ 叫做基底变换矩阵(或表象变换矩阵)。

若令 $\Q Q f = g$,  根据前面的内容, $Q \ket{f}_\psi = \ket{g}_\psi$ 其中 $Q = \rm{diag}(\lambda_1\dots\lambda_n)$。 % 未完成,确保介绍了 diag 符号

下面应用基底变换矩阵, 有 $\ket{f}_\psi = P^{-1} \ket{f}_\phi$ ; $\ket{g}_\psi = P^{-1} \ket{g}_\phi$。 代入上式得
\begin{equation}
Q P^{-1} \ket{f}_\phi = P^{-1} \ket{g}_\phi
\end{equation}
两边左乘 $P$ 得
\begin{equation}
PQ P^{-1} \ket{f}_\phi = \ket{g}_\phi
\end{equation}
令 $\mat Q_1 = \mat P\mat Q \mat P^{-1}$,  得
\begin{equation}
Q_1 \ket{f}_\phi = \ket{g}_\phi
\end{equation}
所以 $\mat Q_1$ 就是要求的矩阵。

下面证明 $\mat Q_1$ 是厄米矩阵。

我们先学习所谓幺正矩阵。这里给出幺正矩阵的一种定义: 若把矩阵 $\mat P$ 的每一列划分成一个列向量, 从左到右分别为$\ket{p_1} \dots\ket{p_n}$,  若满足 $\braket{p_i}{p_j}=\delta_{ij}$ 则矩阵 $\mat P$ 叫做幺正矩阵。

容易证明式 $*$ 中的 $P$ 就是幺正矩阵(证明略)。

性质 $1$ : 幺正矩阵一个很重要的性质就是其厄米共轭等于其逆矩阵, $\mat P^* = \mat P^{- 1}$

证明: 要证明 $\mat P^* = \mat P^{-1}$,  只需证明 $\mat P^* \mat P$ 是单位矩阵即可。 根据矩阵乘法的定义,
\begin{equation}
(P^* P)_{ij} = \sum_{k=1}^n (P^*)_{ik}P_{kj}
\end{equation}
根据厄米共轭的定义,
\begin{equation}
\sum_{k=1}^n (P^*)_{ik}P_{kj}  = \sum_{k=1}^n (P_{ki})^* P_{kj}  = \braket{p_i}{p_j} \delta_{ij}
\end{equation}
所以 $\mat P^* \mat P$ 是 $n$ 阶的单位矩阵。  证毕。

在上文中, $\mat Q$ 是所谓的实数元的对角矩阵, 所以 $\mat Q^* = \mat Q$

另外容易证明, $(\mat A\mat B)^* = \mat B^* \mat A^*$。  所以
\begin{equation}
\mat Q_1^* = (\mat P\mat Q\mat P^{-1})^* = (\mat P^{-1})^* (\mat P\mat Q)^* = \mat P(\mat Q^* \mat P^*) = \mat P\mat Q\mat P^{-1} = \mat Q_1
\end{equation}
所以 $\mat Q_1$ 是厄米矩阵。

