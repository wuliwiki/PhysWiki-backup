% 函数的旋转和其他变换

令坐标的旋转矩阵为 $\mat R$, 那么把函数 $f(\bvec r)$ 做同样的旋转, 那么将得到 $f(\mat R^{-1}\bvec r) = f(\mat R\Tr \bvec r)$。

括号中为什么要去逆旋转呢? 这可以类比一元函数的平移。 $f(x)$ 向右移动 $x_0$ 得 $f(x-x_0)$, 而 $x-x_0$ 恰好是坐标向左平移。

同理,对于任何一一对应的坐标变换 $\bvec r' = g(\bvec r)$, 若想让函数的图像做同样的变换, 那么就把 $f(\bvec r)$ 变为 $f(g^{-1}(\bvec r)$ 即可。
