% 普朗克黑体辐射定律(综述)
% license CCBYSA3
% type Wiki

本文根据 CC-BY-SA 协议转载翻译自维基百科\href{https://en.wikipedia.org/wiki/Michael_Faraday}{相关文章}。

\begin{figure}[ht]
\centering
\includegraphics[width=8cm]{./figures/70b29e3c83575e69.png}
\caption{普朗克定律准确描述了黑体辐射。这里展示的是不同温度下的一组曲线。经典(黑色)曲线在高频率(短波长)下与观测到的强度偏离。} \label{fig_HTFS_1}
\end{figure}
在物理学中,普朗克定律(也称为普朗克辐射定律)描述了在给定温度T下,黑体在热平衡状态下发射的电磁辐射的谱密度,当黑体与其环境之间没有物质或能量的净流动时。

在19世纪末,物理学家无法解释为什么已经准确测量的黑体辐射谱在高频率处与现有理论预测的谱有显著的偏离。1900年,德国物理学家马克斯·普朗克通过启发式推导得出了一个公式,解释了观测到的谱,假设在含有黑体辐射的腔体中,假设的电荷振荡器只能以最小增量E改变其能量,该增量与其相关电磁波的频率成正比。虽然普朗克最初认为将能量分成增量的假设只是为了得到正确答案的数学技巧,但其他物理学家,包括阿尔伯特·爱因斯坦,在他的基础上进行了进一步发展,普朗克的洞察力现在被认为对量子理论具有根本性的重要性。
\subsection{定律}
每个物体都会自发且持续地发射电磁辐射,物体的谱辐射强度 \( B_\nu \) 描述了特定辐射频率下,每单位面积、每单位立体角和每单位频率的谱发射功率。普朗克辐射定律给出的关系表明,随着温度的升高,物体辐射的总能量增加,且辐射谱的峰值会向短波长方向移动。根据普朗克分布定律,在给定温度下,谱能量密度(单位体积每单位频率的能量)由下式给出:
\[
u_\nu(\nu, T) = \frac{8 \pi h \nu^3}{c^3} \cdot \frac{1}{\exp \left( \frac{h \nu}{k_{\mathrm{B}} T} \right) - 1}~
\]
或者,该定律可以表示为物体在绝对温度 \( T \) 下,频率为 \( \nu \) 的谱辐射强度:
\[
B_\nu(\nu, T) = \frac{2 h \nu^3}{c^2} \cdot \frac{1}{\exp \left( \frac{h \nu}{k_{\mathrm{B}} T} \right) - 1}~
\]
其中 \( k_{\mathrm{B}} \) 是玻尔兹曼常数,\( h \) 是普朗克常数,\( c \) 是介质中的光速,无论是物质还是真空。谱辐射强度 \( B_\nu \) 的 cgs 单位是 \( \text{erg} \cdot \text{s}^{-1} \cdot \text{sr}^{-1} \cdot \text{cm}^{-2} \cdot \text{Hz}^{-1} \)。术语 \( B \) 和 \( u \) 通过因子 \( \frac{4 \pi}{c} \) 相关,因为 \( B \) 与方向无关且辐射以光速 \( c \) 传播。谱辐射强度也可以按单位波长 \( \lambda \) 表示,而不是按单位频率。此外,该定律还可以用其他术语表示,例如在某个波长下发射的光子数,或辐射体积内的能量密度。

在低频极限(即长波长)下,普朗克定律趋近于雷leigh–Jeans定律,而在高频极限(即短波长)下,它趋近于维恩近似。

马克斯·普朗克于1900年提出了这个定律,定律中只有经验确定的常数,后来他证明,将其表示为能量分布时,它是热力学平衡辐射的唯一稳定分布。作为能量分布,它是热平衡分布族中的一个成员,包括玻色–爱因斯坦分布、费米–狄拉克分布和麦克斯韦–玻尔兹曼分布。
\subsection{黑体辐射}
\begin{figure}[ht]
\centering
\includegraphics[width=8cm]{./figures/21e7152f5833fb75.png}
\caption{太阳近似为一个黑体辐射源。它的有效温度约为5777 K。} \label{fig_HTFS_2}
\end{figure}
黑体是一个理想化的物体,能够吸收和发射所有频率的辐射。在接近热力学平衡的状态下,发射的辐射可以通过普朗克定律精确描述。由于其依赖于温度,普朗克辐射被称为热辐射,这意味着物体的温度越高,它在每个波长上发射的辐射越多。

普朗克辐射在一个与物体温度相关的波长处具有最大强度。例如,在室温下(约300 K),物体发出的热辐射主要是红外线,并且是不可见的。在较高的温度下,红外辐射的数量增加并可以感受到热量,同时发射更多的可见辐射,物体呈现可见的红色光。在更高的温度下,物体变为明亮的黄色或蓝白色,并发射大量短波长的辐射,包括紫外线甚至X射线。太阳的表面(约6000 K)发射大量的红外线和紫外线辐射,其发射在可见光谱中达到峰值。由于温度变化引起的这种变化被称为维恩位移定律。

普朗克辐射是任何处于热平衡状态的物体从其表面发射的最大辐射量,无论其化学成分或表面结构如何。辐射穿过介质之间的界面时,可以通过界面的发射率来表征(实际辐射强度与理论普朗克辐射强度的比值),通常用符号 \( \epsilon \) 表示。发射率通常依赖于化学成分、物理结构、温度、波长、传输角度和偏振状态。自然界面上,发射率总是在 \( \epsilon = 0 \) 和 \( \epsilon = 1 \) 之间。

一个与另一个介质相接触的物体,如果该介质的发射率 \( \epsilon = 1 \) 且能够吸收所有射入的辐射,则被称为黑体。黑体的表面可以通过在一个大密封容器的墙壁上开一个小孔来建模,这个容器在均匀温度下保持不透明的墙壁,且在每个波长上都不是完全反射的。在平衡状态下,这个容器内部的辐射由普朗克定律描述,离开小孔的辐射也由此定律描述。

就像麦克斯韦–玻尔兹曼分布是物质粒子气体在热平衡状态下的唯一最大熵能量分布一样,普朗克分布也是光子气体的最大熵分布。与物质气体不同,物质气体的质量和粒子数起着重要作用,而光子气体在热平衡状态下的谱辐射强度、压强和能量密度完全由温度决定。

如果光子气体不是普朗克分布的,热力学第二定律保证了交互作用(光子与其他粒子之间的相互作用,甚至在足够高的温度下,光子之间的相互作用)会导致光子能量分布的变化,并最终趋近于普朗克分布。在热力学平衡的过程中,光子会以正确的数量和能量被创造或湮灭,直到它们填充整个腔体,并且达到普朗克分布,直到达到平衡温度。就像气体是由多个子气体组成的,每个子气体对应一个波长范围,每个子气体最终都会达到共同的温度。

量 \( B_\nu(\nu, T) \) 是谱辐射强度,作为温度和频率的函数。它在国际单位制(SI)中的单位为 \( \text{W} \cdot \text{m}^{-2} \cdot \text{sr}^{-1} \cdot \text{Hz}^{-1} \)。一个无穷小的功率 \( B_\nu(\nu, T) \cos \theta \, dA \, d\Omega \, d\nu \) 被辐射到由角度 \( \theta \) 描述的方向,从表面法线处的无穷小面积 \( dA \) 向无穷小的立体角 \( d\Omega \) 以及无穷小频率带宽 \( d\nu \) 辐射,其中心频率为 \( \nu \)。辐射到任何立体角的总功率是 \( B_\nu(\nu, T) \) 对这三个量的积分,且由斯特凡–玻尔兹曼定律给出。普朗克辐射的谱辐射强度对于每个方向和偏振角度都是相同的,因此黑体被称为兰伯特辐射源。
\subsection{不同形式}
普朗克定律可以根据不同科学领域的惯例和偏好,以多种形式出现。下表总结了谱辐射强度定律的不同形式。左侧的形式通常在实验领域中遇到,而右侧的形式则通常在理论领域中遇到。
\begin{figure}[ht]
\centering
\includegraphics[width=14.25cm]{./figures/a6e4d32ae252241b.png}
\caption{} \label{fig_HTFS_3}
\end{figure}
在分数带宽公式中,\(x = \frac{h \nu}{k_{\mathrm{B}} T} = \frac{hc}{\lambda k_{\mathrm{B}} T}\)积分是相对于\(\mathrm{d} (\ln x) = \mathrm{d} (\ln \nu) = \frac{\mathrm{d} \nu}{\nu} = -\frac{\mathrm{d} \lambda}{\lambda} = -\mathrm{d} (\ln \lambda)\)

普朗克定律也可以通过将 \( B \) 乘以 \( \frac{4 \pi}{c} \) 来用谱能量密度(\( u \))表示:[17]
\[
u_{i}(T) = \frac{4 \pi}{c} B_{i}(T)~
\]
\begin{figure}[ht]
\centering
\includegraphics[width=14.25cm]{./figures/ff7e3670c7064133.png}
\caption{} \label{fig_HTFS_4}
\end{figure}
这些分布表示黑体的谱辐射强度——即从辐射表面发射的功率,单位投影面积、单位立体角、单位谱量(频率、波长、波数或其角度等效物,或分数频率或波长)。由于辐射是各向同性的(即与方向无关),因此以某个角度发射的功率与投影面积成正比,按照兰伯特余弦定律,也与该角度的余弦成正比,并且是非偏振的。
\subsubsection{谱变量形式之间的对应关系}
不同的谱变量需要不同的普朗克定律表达形式。通常,不能通过简单地用一个变量替代另一个变量来转换普朗克定律的各种形式,因为这不会考虑到不同形式的单位不同。波长和频率的单位是倒数关系。

这些对应形式是相关的,因为它们表达的是同一个物理事实:对于特定的物理谱增量,相应的物理能量增量会被辐射出去。

无论是以频率增量 \( d\nu \) 的形式表达,还是以波长增量 \( d\lambda \) 的形式,或者是以分数带宽 \( \frac{d\nu}{\nu} \) 或 \( \frac{d\lambda}{\lambda} \) 的形式表达,结果都是相同的。引入负号可以表示频率增量与波长减小之间的对应关系。

为了转换相应的形式,使它们在相同单位下表达相同的量,我们通过谱增量来相乘。然后,对于特定的谱增量,特定的物理能量增量可以写成:
\[
B_{\lambda}(\lambda, T)\,d\lambda = -B_{\nu}(\nu(\lambda), T)\,d\nu~
\]
从而得到:
\[
B_{\lambda}(\lambda, T) = -\frac{d\nu}{d\lambda} B_{\nu}(\nu~(\lambda), T)~
\]
此外,\( \nu(\lambda) = \frac{c}{\lambda} \),所以 \( \frac{d\nu}{d\lambda} = -\frac{c}{\lambda^2} \)。替换后给出频率和波长形式之间的对应关系,并考虑它们不同的维度和单位。[15][18] 由此得出:
\[
\frac{B_{\lambda}(T)}{B_{\nu}(T)} = \frac{c}{\lambda^2} = \frac{\nu^2}{c}~
\]
显然,普朗克定律的谱分布峰值位置取决于所选择的谱变量。然而,从某种意义上讲,这个公式意味着根据维恩位移定律,谱分布的形状是温度独立的,如下面的 § 性质 §§ 百分位数部分所述。

分数带宽形式与其他形式之间的关系为:[16]
\[
B_{\ln x} = \nu B_{\nu} = \lambda B_{\lambda}~
\]
\subsubsection{第一和第二辐射常数}
在上述普朗克定律的不同形式中,波长和波数形式使用了包含物理常数的术语 \( 2hc^2 \) 和 \( \frac{hc}{k_{\mathrm{B}}} \)。因此,这些术语可以被视为物理常数[19],因此被称为第一辐射常数 \( c_{1L} \) 和第二辐射常数 \( c_2 \),其表达式为:
\[
c_{1L} = 2hc^2~
\]
和
\[
c_2 = \frac{hc}{k_{\mathrm{B}}}~
\]
利用这些辐射常数,普朗克定律的波长形式可以简化为:
\[
L(\lambda, T) = \frac{c_{1L}}{\lambda^5} \frac{1}{\exp\left(\frac{c_2}{\lambda T}\right) - 1}~
\]
波数形式也可以相应简化。

这里使用 \( L \) 而不是 \( B \),因为 \( L \) 是谱辐射强度的国际单位制符号。\( c_{1L} \) 中的 \( L \) 就指代此符号。这个引用是必要的,因为普朗克定律可以重新表述为给出谱辐射离开度 \( M(\lambda, T) \) 而不是谱辐射强度 \( L(\lambda, T) \),在这种情况下,\( c_1 \) 替代 \( c_{1L} \),并且:
\[
c_1 = 2\pi hc^2~
\]
因此,普朗克定律对于谱辐射离开度可以写成:
\[
M(\lambda, T) = \frac{c_1}{\lambda^5} \frac{1}{\exp\left(\frac{c_2}~{\lambda T}\right) - 1}~
\]
随着测量技术的提高,国际计量大会已经修订了对 \( c_2 \) 的估算;有关详情,请参见普朗克位置 § 国际温标部分。
\subsection{普朗克定律}
\begin{figure}[ht]
\centering
\includegraphics[width=10cm]{./figures/1576387588a8fb82.png}
\caption{高能振荡器的冻结} \label{fig_HTFS_5}
\end{figure}
普朗克定律描述了在热力学平衡中电磁辐射的独特和特征性的谱分布,当时物质或能量没有净流动。[2] 其物理学最容易通过考虑一个具有刚性不透明壁的腔体中的辐射来理解。壁的运动可能会影响辐射。如果墙壁不是不透明的,那么热力学平衡就不是孤立的。值得解释的是热力学平衡是如何达到的。主要有两种情况:(a)当热力学平衡的趋近发生在物质存在的情况下,腔体的墙壁对于每个波长都不完全反射,或者墙壁对于所有波长完全反射,而腔体中包含一个小黑体(这是普朗克主要考虑的情况);或(b)当热力学平衡的趋近发生在物质不存在的情况下,墙壁对于所有波长都完全反射,并且腔体中没有物质。对于不封闭在这种腔体中的物质,热辐射可以通过适当使用普朗克定律进行近似解释。

经典物理学通过能量均分定理引出了紫外灾难,预测黑体辐射的总强度是无限的。如果补充一个经典上无法辩解的假设,认为某些原因导致辐射是有限的,那么经典热力学可以解释普朗克分布的某些方面,例如斯特凡–玻尔兹曼定律和维恩位移定律。对于物质存在的情况,量子力学提供了一个良好的解释,正如下面“爱因斯坦系数”部分所述。这是爱因斯坦所考虑的情况,并且如今用于量子光学。[20][21] 对于物质不存在的情况,则需要量子场论,因为具有固定粒子数的非相对论量子力学无法提供足够的解释。
\subsubsection{光子}
普朗克定律的量子理论解释将辐射视为在热力学平衡中的无质量、无电荷的玻色子粒子气体,即光子。光子被视为电荷粒子之间电磁相互作用的载体。光子的数量不是守恒的。光子会在适当的数量和能量下被创造或湮灭,以使腔体充满普朗克分布。在热力学平衡中的光子气体,其内部能量密度完全由温度决定;此外,压力也完全由内部能量密度决定。这与物质气体的热力学平衡不同,在物质气体中,内部能量不仅由温度决定,而且还独立地由不同分子的数量决定,且不同分子具有不同的特性。在给定温度下,对于不同的物质气体,压力和内部能量密度可以独立变化,因为不同的分子可以独立地承载不同的激发能量。

普朗克定律是玻色–爱因斯坦分布的一个极限,后者是描述热力学平衡中非交互作用玻色子的能量分布。在无质量玻色子(如光子和胶子)的情况下,化学势为零,玻色–爱因斯坦分布就简化为普朗克分布。还有另一种基本的平衡能量分布:费米–狄拉克分布,它描述了热力学平衡中的费米子,如电子。两种分布的不同之处在于,多个玻色子可以占据相同的量子态,而多个费米子不能。在低密度的情况下,每个粒子可用的量子态数量较大,这种差异变得无关紧要。在低密度极限下,玻色–爱因斯坦分布和费米–狄拉克分布都会简化为麦克斯韦–玻尔兹曼分布。
\subsubsection{基尔霍夫的热辐射定律}
基尔霍夫的热辐射定律是对一个复杂物理情境的简洁概述。以下是该情境的一个介绍性概述,远不是严格的物理论证。此处的目的是仅总结该情境中的主要物理因素及其主要结论。

\textbf{热辐射的光谱依赖性}

导热和辐射热传递之间是有区别的。辐射热传递可以通过过滤器,只让特定频率范围的辐射通过。

通常知道,物体的温度越高,它在每个频率上辐射的热量越多。

在一个不透明的刚性墙体的腔体中,如果墙体在任何频率上都不是完全反射的,在热力学平衡下,腔体内只有一个温度,而且该温度必须是每种频率辐射的共同温度。

可以想象两个这样的腔体,每个腔体都处于各自独立的辐射和热力学平衡中。可以想象一个光学装置,它允许两腔体之间通过辐射热传递,但只通过一个特定频带的辐射频率。如果两个腔体在该频带的光谱辐射强度值不同,热量可能从较热的腔体传递到较冷的腔体。可以提出使用这种特定频带的热传递来驱动热机。如果两个腔体的温度相同,那么热力学第二定律不允许热机工作。由此可以推断,对于两个腔体共享的温度,所有频带的光谱辐射强度值也必须相同。这对于每个频带都成立。[22][23][24] 这一点首先被巴尔福·斯图尔特(Balfour Stewart)和后来基尔霍夫(Kirchhoff)明确地认识到。巴尔福·斯图尔特通过实验发现,在所有表面中,灯烟黑表面辐射出最多的热辐射,不论辐射的质量如何,经过不同的过滤器测量得出。

从理论上思考,基尔霍夫进一步指出,这意味着任何处于热力学平衡的腔体,其辐射频率作为光谱辐射强度的函数,必须是温度的唯一普遍函数。他假设了一个理想的黑体,它与周围环境的界面正好使其吸收所有射向它的辐射。根据亥姆霍兹互易原理,这样的物体内部的辐射将不受阻碍地直接传递到其周围环境,而不在界面处反射。在热力学平衡下,来自这样的物体的热辐射将具有作为温度函数的那个唯一普遍的光谱辐射强度。这个洞察力正是基尔霍夫热辐射定律的根源。

\textbf{吸收率与发射率的关系}

假设有一个小的均匀球形材料体 X,温度为 TX,位于一个大型腔体内,该腔体的壁材为 Y,温度为 TY。物体 X 会发出自身的热辐射。在某一特定频率 ν下,物体 X 在其中心的某个截面上向该截面法线方向发出的辐射可以表示为 Iν,X(TX),这是物体 X 的特征性辐射强度。对于该频率 ν,腔体壁的辐射功率可以表示为 Iν,Y(TY),即壁材 Y 在温度 TY 下的辐射强度。对于材料 X,定义吸收率 αν,X,Y(TX, TY) 为 X 吸收的该辐射的比例,那么入射能量的吸收速率为 αν,X,Y(TX, TY) Iν,Y(TY)。

在该截面上,物体 X 吸收辐射的能量累积速率 q(ν,TX,TY) 可以表示为:
\[ q(\nu ,T_{X},T_{Y}) = \alpha_{\nu ,X,Y}(T_{X},T_{Y}) I_{\nu ,Y}(T_{Y}) - I_{\nu ,X}(T_{X})~\]
基尔霍夫的开创性洞察,前面提到过,表明在热力学平衡下温度为 T 时,存在一个唯一的普遍辐射分布,现今称为 Bν(T),它与材料 X 和 Y 的化学特性无关,这为我们理解任何物体的辐射交换平衡提供了宝贵的视角。

当温度为 T 的热力学平衡存在时,腔体壁的辐射具有该唯一的普遍值,因此 Iν,Y(TY) = Bν(T)。进一步地,可以定义物体 X 的发射率 εν,X(TX),使得在温度 TX = T 时,Iν,X(TX) = Iν,X(T) = εν,X(T) Bν(T)。

当温度为 T = TX = TY 时,热平衡条件下能量的累积速率为零,即 q(ν,TX,TY) = 0。由此可以得出,在热力学平衡时,当 T = TX = TY:
\[ 0 = \alpha_{\nu ,X,Y}(T,T) B_{\nu}(T) - \epsilon_{\nu ,X}(T) B_{\nu}(T)~\]
基尔霍夫指出,由此可知,在热力学平衡时,当 T = TX = TY:
\[ \alpha_{\nu ,X,Y}(T,T) = \epsilon_{\nu ,X}(T) ~\]
引入特殊符号 αν,X(T),表示材料 X 在热力学平衡下温度为 T 时的吸收率(这一发现由爱因斯坦提出,后文将说明),可以进一步得出以下等式:
\[ \alpha_{\nu ,X}(T) = \epsilon_{\nu ,X}(T) ~\]
这里示范的吸收率和发射率相等的关系特指在热力学平衡下的温度 T,一般情况下如果不满足热力学平衡条件,这种关系不必然成立。发射率和吸收率是材料分子各自的性质,但它们依赖于分子激发态的分布,因为存在一种叫做“受激辐射”的现象,这一现象是由爱因斯坦发现的。当材料处于热力学平衡状态或处于被称为局部热力学平衡的状态时,发射率和吸收率才会相等。非常强的入射辐射或其他因素可能会破坏热力学平衡或局部热力学平衡。气体中的局部热力学平衡意味着分子碰撞对分子激发态分布的影响远大于辐射的吸收和发射。

基尔霍夫指出,他并不知道 Bν(T) 的精确形式,但他认为找出它是非常重要的。在基尔霍夫提出其存在性和特性的一般原理四十年后,普朗克的贡献是确定了这一平衡分布 Bν(T) 的精确数学表达式。
\subsubsection{黑体}
在物理学中,理想的黑体被定义为能够完全吸收所有频率 ν 的电磁辐射的物体(因此称为“黑体”)。根据基尔霍夫的热辐射定律,这意味着对于每一个频率 ν,在热力学平衡下温度为 T 时,黑体的吸收率和发射率都满足:αν,B(T) = εν,B(T) = 1,因此黑体的热辐射总是与普朗克定律所指定的总量相等。没有任何物体能够发出超过黑体的热辐射,因为如果物体与辐射场处于热平衡状态,它所发出的能量将大于它所接收的辐射能量。

尽管完美的黑体材料在自然界中不存在,但在实践中,可以较为准确地近似黑色表面。至于其物质内部,具有明确界面的凝聚态物体(如液体、固体或等离子体),如果完全不透明,它对于辐射而言就是完全黑色的。这意味着它会吸收所有穿透物体与周围环境接触界面的辐射,并进入物体。实际上,这一点并不难实现。另一方面,完美的黑色界面在自然界中并不存在。完美的黑色界面不会反射任何辐射,而是让所有辐射通过它,不论来自哪一侧。为了制造一个有效的黑色界面,最好的方法是通过在一个完全不透明、且在任何频率下都不会完美反射的刚性材料体的墙壁上打一个小孔,来模拟一个“界面”,同时保持其墙壁处于受控的温度状态。除此之外,墙壁的材料成分没有限制。进入小孔的辐射几乎没有可能在不被墙壁多次撞击吸收的情况下逃离腔体。
\subsubsection{兰伯特余弦定律}
如普朗克所解释,辐射体的内部由物质组成,并与其相邻的物质介质接触,通常辐射来自物体表面的辐射是从该介质中观察到的。界面并不是由物理物质组成的,而是一个理论构想,是一个二维的数学面,是两个相邻介质的共同属性,严格来说并不属于任何一个介质。这样的界面既不能吸收也不能发射辐射,因为它不是由物理物质组成的;但它是辐射反射和透射的地方,因为它是光学性质的间断面。界面上的辐射反射和透射遵循斯托克斯-亥姆霍兹互易原理。

在处于热力学平衡状态的腔体内的黑体内部,在温度 T 下,辐射是均匀的、各向同性的且无偏振的。黑体吸收所有照射到它的电磁辐射,并且不反射任何辐射。根据亥姆霍兹互易原理,来自黑体内部的辐射在其表面不会被反射,而是完全透过表面传递到外部。由于辐射在黑体内部是各向同性的,从其内部到外部通过表面传递的辐射的光谱辐射强度与方向无关。

这可以通过以下方式表示,即热力学平衡下,黑体表面发出的辐射遵循兰伯特余弦定律。这意味着,从黑体实际辐射表面上某个微小面积元素 dA 发出的光谱通量 dΦ(dA, θ, dΩ, dν),在由θ与该面积元素的法线方向夹角为θ的给定方向上被检测,进入以该方向为中心的固体角 dΩ 中,在频率带宽 dν 内,可以表示为:
\[
\frac{d\Phi (dA,\theta ,d\Omega ,d\nu )}{d\Omega } = L^{0}(dA,d\nu )\,dA\,d\nu \,\cos \theta~
\]
其中,L0(dA, dν) 表示该面积元素 dA 在法线方向 θ = 0 时测得的单位面积、单位频率、单位固体角的辐射通量。

存在 cos θ 因子是因为光谱辐射强度所指的面积是实际辐射表面面积投影到与θ方向垂直的平面上的面积。这就是余弦定律的名称由来。

考虑到热力学平衡下,黑体表面辐射的光谱辐射强度与方向无关,可以得出 L0(dA, dν) = Bν(T),因此有:
\[
\frac{d\Phi (dA,\theta ,d\Omega ,d\nu )}{d\Omega } = B_{\nu }(T)\,dA\,d\nu \,\cos \theta~
\]
因此,兰伯特余弦定律表达了热力学平衡下,黑体表面辐射强度 Bν(T) 与方向的独立性。
\subsubsection{斯特藩–玻尔兹曼定律}
总功率(P)是黑体表面单位面积辐射的功率,可以通过对兰伯特定律所给出的黑体光谱通量在所有频率和对应于半球(h)上的固体角上进行积分来求得。
\[
P = \int_0^\infty d\nu \int_h d\Omega \, B_{\nu} \cos(\theta)~
\]
无穷小的固体角可以用球坐标表示:
\[
d\Omega = \sin(\theta) \, d\theta \, d\phi~
\]
因此,功率可以表示为:
\[
P = \int_0^\infty d\nu \int_0^{\frac{\pi}{2}} d\theta \int_0^{2\pi} d\phi \, B_{\nu}(T) \cos(\theta) \sin(\theta) = \sigma T^4~
\]
其中,\(\sigma\) 是斯特藩–玻尔兹曼常数,定义为:
\[
\sigma = \frac{2 k_{\mathrm{B}}^4 \pi^5}{15 c^2 h^3} \approx 5.670400 \times 10^{-8} \, \mathrm{J\,s^{-1}m^{-2}K^{-4}}~
\]
斯特藩–玻尔兹曼常数 \(\sigma\) 描述了单位面积黑体在单位时间内辐射的功率与其温度的四次方成正比,即斯特藩–玻尔兹曼定律。
\subsubsection{辐射传输}
辐射传输方程描述了辐射在通过物质介质时如何受到影响。对于介质在某点附近处于热力学平衡的特殊情况,普朗克定律具有特别重要的意义。

为简化起见,我们可以考虑没有散射的线性稳态。在辐射传输方程中,对于一束穿过小距离 \(ds\) 的光束,能量是守恒的:该光束的(光谱)辐射强度 \(I_{\nu}\) 的变化等于被物质介质移除的部分和从物质介质获得的部分。如果辐射场与物质介质处于平衡状态,这两者将相等。物质介质将具有一定的发射系数和吸收系数。

吸收系数 \(\alpha\) 是光束强度在其传播的距离 \(ds\) 上的相对变化,单位为长度的倒数。它由两个部分组成:由于吸收造成的强度降低和由于受激辐射造成的强度增加。受激辐射是物质体由于和入射辐射成比例的作用而产生的辐射,它被包含在吸收项中,因为它与入射辐射的强度成比例。由于吸收量通常会随物质的密度 \(\rho\) 线性变化,我们可以定义“质量吸收系数”\(\kappa_{\nu} = \frac{\alpha}{\rho}\),它是物质本身的特性。光束由于吸收而强度变化的表达式为:
\[
dI_{\nu} = -\kappa_{\nu} \rho I_{\nu} \, ds~
\]
“质量发射系数”\(j_{\nu}\) 是单位体积的小体积元素的辐射强度除以其质量(因为和质量吸收系数一样,发射与发射物质的质量成正比),单位为功率·固体角\(^{-1}\)·频率\(^{-1}\)·密度\(^{-1}\)。像质量吸收系数一样,它也是物质本身的特性。光束由于发射而变化的表达式为:
\[
dI_{\nu} = j_{\nu} \rho \, ds~
\]
辐射传输方程是这两个贡献的总和:
\[
\frac{dI_{\nu}}{ds} = j_{\nu} \rho - \kappa_{\nu} \rho I_{\nu}~
\]
如果辐射场与物质介质处于平衡状态,那么辐射将是均匀的(与位置无关),从而 \(dI_{\nu} = 0\),并且:
\[
\kappa_{\nu} B_{\nu} = j_{\nu}~
\]
这再次是基尔霍夫定律的一个表达,描述了介质的两个物质特性,并且它在介质处于热力学平衡的点上得到了辐射传输方程:
\[
\frac{dI_{\nu}}{ds} = \kappa_{\nu} \rho (B_{\nu} - I_{\nu})~
\]
\subsubsection{爱因斯坦系数}
详细平衡原理指出,在热力学平衡状态下,每一个基本过程都由其逆过程达到平衡。

在1916年,阿尔伯特·爱因斯坦将这一原理应用于原子层面,研究了由于原子在两个特定能级之间跃迁而辐射和吸收辐射的情况,进而对辐射传输方程和基尔霍夫定律在这种辐射中的应用提供了更深的理解。如果能级1是能量为 \(E_1\) 的低能级,而能级2是能量为 \(E_2\) 的高能级,那么辐射或吸收的频率 \(\nu\) 将由玻尔频率条件决定:
\[
E_2 - E_1 = h\nu~
\]
如果 \(n_1\) 和 \(n_2\) 分别是原子处于状态1和状态2的数密度,那么这些数密度随时间变化的速率将由以下三种过程决定:

\textbf{自发辐射}
\[
\left(\frac{dn_1}{dt}\right)_{\mathrm{spon}} = A_{21} n_2~
\]
\textbf{受激辐射}
\[
\left(\frac{dn_1}{dt}\right)_{\mathrm{stim}} = B_{21} n_2 u_\nu~
\]
\textbf{光吸收} 
\[
\left(\frac{dn_2}{dt}\right)_{\mathrm{abs}} = B_{12} n_1 u_\nu~
\]
其中,\( u_\nu \) 是辐射场的光谱能量密度。三个参数 \( A_{21} \)、\( B_{21} \) 和 \( B_{12} \),称为爱因斯坦系数,与两个能级(状态)之间跃迁所产生的光子频率 \( \nu \) 相关。因此,光谱中的每一条线都有一组与之相关的系数。当原子与辐射场处于平衡状态时,辐射强度将遵循普朗克定律,并且根据详细平衡原理,这些速率的总和必须为零:
\[
0 = A_{21} n_2 + B_{21} n_2 \frac{4\pi}{c} B_\nu(T) - B_{12} n_1 \frac{4\pi}{c} B_\nu(T)~
\]
由于原子也处于平衡状态,两个能级的粒子数由玻尔兹曼因子关联:
\[
\frac{n_2}{n_1} = \frac{g_2}{g_1} e^{-h\nu / k_{\mathrm{B}} T}~
\]
其中 \( g_1 \) 和 \( g_2 \) 是相应能级的简并度。将上述两个方程结合,并要求它们在任意温度下有效,可得爱因斯坦系数之间的两个关系:
\[
\frac{A_{21}}{B_{21}} = \frac{8\pi h\nu^3}{c^3}~
\]
\[
\frac{B_{21}}{B_{12}} = \frac{g_1}{g_2}~
\]
因此,知道一个系数即可得出其他两个系数。

对于各向同性的吸收和辐射,上文辐射传输部分定义的辐射系数 (\( j_\nu \)) 和吸收系数 (\( \kappa_\nu \)) 可以通过爱因斯坦系数表示。爱因斯坦系数之间的关系将得出基尔霍夫定律的表达式,如上文辐射传输部分所述,即:
\[
j_\nu = \kappa_\nu B_\nu~
\]
这些系数适用于原子和分子。
\subsection{属性}
\subsubsection{峰值}
分布 \( B_\nu \)、\( B_\omega \)、\( B_{\tilde{\nu}} \) 和 \( B_k \) 在一个光子能量上达到峰值:
\[
E = \left[ 3 + W \left( -3e^{-3} \right) \right] k_{\mathrm{B}} T \approx 2.821 \, k_{\mathrm{B}} T~
\]
其中 \( W \) 是兰伯特 W 函数,\( e \) 是欧拉常数。

然而,分布 \( B_\lambda \) 在一个不同的能量上达到峰值:
\[
E = \left[ 5 + W \left( -5e^{-5} \right) \right] k_{\mathrm{B}} T \approx 4.965 \, k_{\mathrm{B}} T~
\]
这是因为,正如前面所提到的,不能通过简单地用 \( \lambda \) 替代 \( \nu \) 从 \( B_\nu \) 转换到 \( B_\lambda \)。此外,还必须乘以:\(| \d\nu/d\lambda| = \frac{c}{\lambda^2}\)这会将分布的峰值移动到更高的能量。以上的峰值是光子的模态能量,当频率或波长被均匀分成等大小的区间时。通过将 \( hc \)(14387.770 微米·K)除以这些能量表达式,可以得到峰值的波长。

在这些峰值处的光谱辐射度为:
\[
B_{\nu, {\text{max}}}(T) = \frac{2 k_{\mathrm{B}}^3 T^3 x^3}{h^2 c^2} \frac{1}{e^x - 1} \approx 1.896 \times 10^{-19} \, \frac{\text{W}}{\text{m}^2 \cdot \text{Hz} \cdot \text{sr}} \times \left( \frac{T}{\text{K}} \right)^3~
\]
其中
\[
x = 3 + W \left( -3e^{-3} \right)~
\]
而波长方向的辐射度为:
\[
B_{\lambda, {\text{max}}}(T) = \frac{2 k_{\mathrm{B}}^5 T^5 x^5}{h^4 c^3} \frac{1}{e^x - 1} \approx 4.096 \times 10^{-6} \, \frac{\text{W}}{\text{m}^2 \cdot \text{sr}} \times \left( \frac{T}{\text{K}} \right)^5~
\]
其中
\[
x = 5 + W \left( -5e^{-5} \right)~
\]
同时,黑体辐射的光子平均能量为:
\[
E = \left[ \frac{\pi^4}{30 \zeta(3)} \right] k_{\mathrm{B}} T \approx 2.701 \, k_{\mathrm{B}} T~
\]
其中 \( \zeta \) 是黎曼ζ函数。
\subsubsection{近似}
\begin{figure}[ht]
\centering
\includegraphics[width=10cm]{./figures/f2dc52cc65ed57d3.png}
\caption{计划定律(绿色)、瑞利-金斯定律(红色)和维恩近似(蓝色)在 8 毫开尔文温度下的黑体辐射与频率的对数-对数图。} \label{fig_HTFS_6}
\end{figure}
在低频极限(即长波长)下,普朗克定律变为瑞利–金斯定律:
\[
B_\nu (T) \approx \frac{2 \nu^2}{c^2} k_{\mathrm{B}} T~
\]
或者
\[
B_\lambda (T) \approx \frac{2c}{\lambda^4} k_{\mathrm{B}} T~
\]
辐射度随频率的平方增加,展示了紫外灾难。在高频极限(即短波长)下,普朗克定律趋近于维恩近似:
\[
B_\nu (T) \approx \frac{2 h \nu^3}{c^2} e^{-\frac{h\nu}{k_{\mathrm~{B}} T}}~
\]
或者
\[
B_\lambda (T) \approx \frac{2 h c^2}{\lambda^5} e^{-\frac{hc}{\lambda k_{\mathrm{B}} T}}~
\]
\subsubsection{百分位数}
\begin{figure}[ht]
\centering
\includegraphics[width=6cm]{./figures/a049e1ab8b8dd1d7.png}
\caption{} \label{fig_HTFS_7}
\end{figure}
维恩位移定律的较强形式指出,普朗克定律的形状与温度无关。因此,可以列出总辐射的百分位点以及波长和频率的峰值,以一种通过温度 T 除以得到波长 λ 的形式。下表的第二列列出了相应的 λT 值,即那些在第一个列中给定的辐射百分位点对应的 x 值,其中波长 λ 为 ⁠x/T μm。

即,0.01\% 的辐射波长小于 ⁠910/T μm,20\% 小于 ⁠2676/T μm,等等。波长和频率的峰值以粗体显示,分别出现在 25.0\% 和 64.6\% 处。41.8\% 点是波长-频率中性峰值(即波长或频率的对数变化单位功率的峰值)。这些是各自的普朗克定律函数 ⁠1/λ⁵、ν³ 和 ⁠ν²/λ² 分别除以 exp(⁠hν / kBT⁠) − 1 达到最大值的点。波长比率在 0.1\% 和 0.01\% 之间的差距(1110 比 910 多 22\%)比在 99.9\% 和 99.99\% 之间的差距(113374 比 51613 多 120\%)小,反映了在短波长(左端)的能量指数衰减和在长波长的多项式衰减。

选择哪个峰值取决于应用。传统上选择的是由维恩位移定律的较弱形式给出的 25.0\% 处的波长峰值。对于某些用途,将总辐射分为两半的中位数或 50\% 点可能更合适。后者更接近频率峰值而非波长峰值,因为在短波长处辐射呈指数衰减,而在长波长处则呈多项式衰减。中性峰值由于同样的原因出现在比中位数更短的波长处。
\begin{figure}[ht]
\centering
\includegraphics[width=10cm]{./figures/4add84c7283979af.png}
\caption{与5775 K黑体辐射相比的太阳光谱} \label{fig_HTFS_8}
\end{figure}
\textbf{与太阳光谱的比较}  

太阳辐射可以与大约5778 K的黑体辐射进行比较(但请参见图表)。右侧的表格展示了在此温度下黑体辐射的分布情况,并与太阳光的辐射分布进行了对比。为了进行对比,表格还展示了一个被建模为黑体的行星,其辐射温度为名义上的288 K(15°C)。
\begin{figure}[ht]
\centering
\includegraphics[width=10cm]{./figures/657553cb7e2518f9.png}
\caption{} \label{fig_HTFS_9}
\end{figure}
地球的温度变化较大,代表值为288 K(约15°C)。与太阳相比,其波长大约是太阳波长的二十倍,表格的第三列以微米为单位列出了这些数据(千纳米)。

也就是说,太阳辐射中只有1\%处于波长小于296 nm的范围,只有1\%处于波长大于3728 nm的范围。换算为微米,这意味着太阳辐射的98\%处于0.296至3.728 μm的波长范围内。而一个288 K的行星辐射出的98\%能量,则位于5.03至79.5 μm的波长范围,远高于太阳辐射的范围(如果使用频率 ν = c / λ 而非波长 λ 表示,则位于低频区)。

太阳辐射与行星辐射在波长上的这个数量级差异,导致了设计过滤器以允许通过一种辐射而阻挡另一种辐射变得相对容易。例如,普通玻璃或透明塑料制造的窗户可以通过至少80\%的来自5778 K太阳辐射,波长小于1.2 μm,而阻挡超过99\%的来自288 K热辐射,波长大于5 μm,而在这些波长下,大多数建筑级厚度的玻璃和塑料对热辐射是有效不透光的。

太阳辐射是到达大气顶层(TOA)的辐射。如表格所示,波长小于400 nm的紫外线辐射约占8\%,而波长大于700 nm的红外线辐射从约48\%的比例开始,占据总辐射的52\%。因此,只有40\%的TOA入射辐射对人眼是可见的。大气层会大大改变这些百分比,偏向可见光,因为它吸收了大部分紫外线和相当一部分红外线。
\subsection{推导过程}
\subsubsection{光子气体}
考虑一个边长为 \( L \) 的立方体,立方体的墙壁是导电的,内部充满了在温度 \( T \) 下处于热平衡的电磁辐射。如果其中一个墙壁上有一个小孔,那么从孔中发出的辐射将具有完美黑体的特征。我们将首先计算立方体内的光谱能量密度,然后确定发射辐射的光谱辐射强度。

在立方体的墙壁上,电场的平行分量和磁场的正交分量必须为零。类似于箱中粒子的波函数,可以发现这些场是周期函数的叠加。与墙壁正交的三个方向上的波长 \( \lambda_1, \lambda_2, \lambda_3 \) 可以表示为:
\[
\lambda_i = \frac{2L}{n_i},~
\]
其中 \( n_i \) 是正整数。对于每一组整数 \( n_i \),都有两个线性无关的解(称为模式)。每一组 \( n_i \) 对应的两个模式与光子的两个偏振状态相关,光子的自旋为 1。根据量子理论,模式的总能量由下式给出:
\[
E_{n_1, n_2, n_3}(r) = \left( r + \frac{1}{2} \right) \frac{hc}{2L} \sqrt{n_1^2 + n_2^2 + n_3^2}.~
\]
其中 \( r \) 可以解释为该模式中的光子数。对于 \( r = 0 \),模式的能量并不为零。电磁场的这种真空能量就是卡西米尔效应的原因。接下来,我们将计算在绝对温度 \( T \) 下盒子的内能。

根据统计力学,特定模式的能量级的平衡概率分布由以下公式给出:
\[
P_r = \frac{e^{-\beta E(r)}}{Z(\beta)},~
\]
其中我们使用倒温度定义:
\[
\beta \stackrel{\mathrm{def}}{=} \frac{1}{k_B T}.~
\]
分母 \( Z(\beta) \) 是单个模式的配分函数。它使得 \( P_r \) 被正确归一化,并且可以计算为:
\[
Z(\beta) = \sum_{r=0}^{\infty} e^{-\beta E(r)} = \frac{e^{-\beta \varepsilon / 2}}{1 - e^{-\beta \varepsilon}},~
\]
其中,
\[
\varepsilon = \frac{hc}{2L} \sqrt{n_1^2 + n_2^2 + n_3^2},~
\]
是单个光子的能量。可以从配分函数中得到模式的平均能量:
\[
\langle E \rangle = -\frac{d \log (Z)}{d \beta} = \frac{\varepsilon}{2} + \frac{\varepsilon}{e^{\beta \varepsilon} - 1}.~
\]
这个公式,除了第一个真空能量项外,是遵循玻色-爱因斯坦统计的粒子的一个特例。由于光子的总数没有限制,因此化学势为零。

如果我们将能量相对于基态进行测量,盒子中的总能量通过对所有允许的单光子态求和 \( \langle E \rangle - \frac{\varepsilon}{2} \) 来得到。通过在热力学极限下(当 \( L \) 趋近于无穷大时)精确计算这一结果是可行的。在这个极限下,\( \varepsilon \) 变为连续,我们可以对这个参数上的 \( \langle E \rangle - \frac{\varepsilon}{2} \) 进行积分。为了以这种方式计算盒子中的能量,我们需要评估给定能量范围内的光子态的数量。如果我们将单光子状态的总数写为 \( g(\varepsilon) d\varepsilon \),其中 \( g(\varepsilon) \) 是态密度(将在下文计算),那么总能量由以下公式给出:
\[
U = \int_0^\infty \frac{\varepsilon}{e^{\beta \varepsilon} - 1} g(\varepsilon) d\varepsilon.~
\]
为了计算态密度,我们将公式(2)重写为:
\[
\varepsilon = \frac{hc}{2L} n,~
\]
其中 \( n \) 是向量 \( n = (n_1, n_2, n_3) \) 的模。

对于每个具有整数分量且大于或等于零的向量 \( n \),都有两个光子状态。这意味着在 \( n \)-空间的某个区域内,光子状态的数量是该区域体积的两倍。能量范围 \( d\varepsilon \) 对应于 \( n \)-空间中厚度为 \( dn = \frac{2L}{hc} d\varepsilon \) 的壳。由于 \( n \) 的分量必须为正,这个壳在球体的一个八分之一区域内扩展。能量范围 \( d\varepsilon \) 内的光子状态数 \( g(\varepsilon) d\varepsilon \) 因此给出:
\[
g(\varepsilon)\,d\varepsilon = 2 \times \frac{1}{8} \times 4\pi n^2\,dn = \frac{8\pi L^3}{h^3 c^3} \varepsilon^2\,d\varepsilon.~
\]
将此代入公式(3),并除以体积 \( V = L^3 \),得到总能量密度:
\[
\frac{U}{V} = \int_0^\infty u_{\nu}(T)\,d\nu,~
\]
其中,频率依赖的谱能量密度 \( u_{\nu}(T) \) 给出为:
\[
u_{\nu}(T) = \frac{8\pi h \nu^3}{c^3} \frac{1}{e^{h\nu / k_B T} - 1}.~
\]
由于辐射在所有方向上都是相同的,并且以光速传播,因此从小孔射出的辐射的谱辐射度为:
\[
B_{\nu}(T) = \frac{u_{\nu}(T) c}{4\pi},~
\]
这给出了普朗克定律:
\[
B_{\nu}(T) = \frac{2h \nu^3}{c^2} \frac{1}{e^{h\nu / k_B T} - 1}.~
\]
通过在总能量积分中变换变量,可以得到定律的其他形式。上述推导基于 Brehm & Mullin 1989。
\subsubsection{偶极近似和爱因斯坦系数}
对于非简并情况,可以使用量子力学中的时间相关扰动理论通过偶极近似来计算 \( A \) 和 \( B \) 系数。计算 \( A \) 还需要二次量子化,因为半经典理论无法解释自发辐射,它不会在扰动场趋于零时趋于零。因此,计算的跃迁速率(以 SI 单位表示)为:

自发辐射速率:
\[
w_{i\rightarrow f}^{\text{s.emi}} = \frac{\omega_{if}^3 e^2}{3\pi \epsilon_0 \hbar c^3} |\langle f| \vec{r} | i \rangle|^2 = A_{if}~
\]
吸收速率:
\[
w_{i\rightarrow f}^{\text{abs}} = \frac{u(\omega_{fi}) \pi e^2}{3\epsilon_0 \hbar^2} |\langle f| \vec{r} | i \rangle|^2 = B_{if}^{\text{abs}} u(\omega_{fi})~
\]
发射速率:
\[
w_{i\rightarrow f}^{\text{emi}} = \frac{u(\omega_{if}) \pi e^2}{3\epsilon_0 \hbar^2} |\langle f| \vec{r} | i \rangle|^2 = B_{if}^{\text{emi}} u(\omega_{if})~
\]
需要注意的是,跃迁速率公式依赖于偶极矩算符。对于更高阶的近似,它涉及四极矩和其他类似项。\( A \) 和 \( B \) 系数(它们对应于角频率能量分布)因此为:
\[
A_{ab} = \frac{\omega_{ab}^3 e^2}{3\pi \epsilon_0 \hbar c^3} |\langle a| \vec{r} | b \rangle |^2~
\]
\[
B_{ab} = \frac{\pi e^2}{3\epsilon_0 \hbar^2} |\langle a| \vec{r} | b \rangle |^2~
\]
其中,\(\omega_{ab} = \frac{E_a - E_b}{\hbar}\),并且对于非简并情况,\( A \) 和 \( B \) 系数满足给定的比例关系:
\[
\frac{B_{12}}{B_{21}} = 1~
\]
\[
\frac{A_{if}}{B} = \frac{\omega_{if}^3 \hbar}{\pi^2 c^3}~
\]
另一个有用的比例关系来自于麦克斯韦分布,它表示能级 \( E \) 中的粒子数与 \( e^{-\beta E} \) 成正比。数学表达式为:
\[
\frac{N_a}{N_b} = \frac{e^{-E_a \beta}}{e^{-E_b \beta}} = e^{\omega_{ba} \hbar \beta}~
\]
其中,\( N_a \) 和 \( N_b \) 分别是能级 \( E_a \) 和 \( E_b \) 上的粒子数,且 \( E_b > E_a \)。接下来,使用以下关系:
\[
\frac{dN_b}{dt} = -A_{ba} N_b - N_b u(\omega_{ba}) B_{ba} + N_a u(\omega_{ba}) B_{ab} = -N_b w_{b \rightarrow a}^{\text{s.emi}} - N_b w_{b \rightarrow a}^{\text{emi}} + N_a w_{a \rightarrow b}^{\text{abs}}~
\]
通过对 \( \frac{dN_b}{dt} = 0 \) 进行求解,并使用已推导出的比例关系,得到普朗克定律:
\[
u_{\omega}(\omega, T) = \frac{\omega^3 \hbar}{\pi^2 c^3} \cdot \frac{1}{e^{\omega \hbar \beta} - 1}~
\]
\subsection{历史}
\subsubsection{巴尔福·斯图尔特}
在1858年,巴尔福·斯图尔特描述了他关于不同物质的抛光板在相同温度下的热辐射发射能力和吸收能力的实验,这些与灯黑表面的功率进行了比较。[9] 斯图尔特选择灯黑表面作为参考,是因为先前的实验结果,尤其是皮埃尔·普雷沃斯特和约翰·莱斯利的研究。他写道:“灯黑吸收所有落在其上的射线,因此具有最大的吸收能力,同时也将具有最大的辐射能力。”

斯图尔特使用热电堆和灵敏的电流计(通过显微镜读取)来测量辐射功率。他关注的是选择性热辐射,即他研究了辐射和吸收在不同射线特性下有选择性的物质板,而不是对所有射线特性都具有最大辐射和吸收能力。他在实验中讨论了可以被反射和折射的射线,并指出这些射线遵循赫尔姆霍兹互易原理(尽管他没有给这个原理起名)。他在这篇论文中没有提到射线的特性可以通过其波长来描述,也没有使用如棱镜或光栅等光谱分辨装置。他的工作在这些限制下是定量的。他在室温环境中进行测量,并迅速进行,以便捕捉物体在与沸水加热至平衡后接近平衡的状态下的情况。他的测量证实了那些选择性发射和吸收的物质遵循了热平衡下发射与吸收选择性相等的原理。

斯图尔特提供了一个理论证明,表明这一现象应该适用于每种选择的热辐射特性,但他的数学推导并不严格有效。根据历史学家D. M. Siegel的说法:“他并不是19世纪数学物理学中更复杂技巧的使用者;在处理光谱分布时,他甚至没有使用函数符号。”[48] 他在这篇论文中没有提到热力学,尽管他提到过能量守恒(即活力的守恒)。他提出,他的测量结果暗示辐射被物质粒子在其传播的介质深度中既吸收又发射。他应用赫尔姆霍兹互易原理来解释物质界面过程,这与物质内部的过程有所不同。他得出结论,实验表明,在热平衡的封闭体内,辐射热量的反射和发射总和离开任何表面部分时,不论其材质如何,都与灯黑表面相同。他没有提到理想完全反射墙的可能性;特别地,他指出,高度抛光的实际金属仅吸收极少量的辐射。
\subsubsection{古斯塔夫·基尔霍夫}
1859年,在没有了解到斯图尔特工作的情况下,古斯塔夫·罗伯特·基尔霍夫报告了可见光吸收和发射的光谱分辨线的波长的巧合。对于热物理学而言,基尔霍夫还观察到,发射体和吸收体之间的温度差异会导致亮线或暗线的出现。[49]

随后,基尔霍夫继续研究了在不透明封闭体或腔体中发射和吸收热辐射的物体,并讨论了这些物体在温度 T 下的热平衡。

在这里使用了一种与基尔霍夫不同的符号。这里,发射功率 \( E(T, i) \) 表示一个有量纲的量,表示在温度 T 下,标记为 i 的物体所发射的总辐射。该物体的总吸收比 \( a(T, i) \) 是一个无量纲的量,表示在温度 T 下该物体在腔体中吸收的辐射与入射辐射的比值。(与巴尔福·斯图尔特不同,基尔霍夫在定义其吸收比时并没有特别提到灯黑表面作为入射辐射的来源。)因此,发射功率与吸收比的比值 \( \frac{E(T, i)}{a(T, i)} \) 是一个有量纲的量,其量纲与发射功率相同,因为 \( a(T, i) \) 是无量纲的。此外,这里物体在温度 T 下的波长特定发射功率记作 \( E(\lambda, T, i) \),而波长特定的吸收比记作 \( a(\lambda, T, i) \)。同样,发射功率与吸收比的比值 \( \frac{E(\lambda, T, i)}{a(\lambda, T, i)} \) 是一个有量纲的量,其量纲与发射功率相同。

在1859年所做的第二次报告中,基尔霍夫宣布了一个新的普遍原理或定律,并为此提供了理论和数学证明,尽管他没有提供辐射功率的定量测量。[50] 然而,他的理论证明被一些学者认为是无效的。[48][51] 然而,他的原理仍然得以延续:即对于相同波长的热辐射,在给定温度下的平衡状态中,发射功率与吸收比的波长特定比值对于所有发射和吸收该波长的物体具有相同的值。用符号表示,该定律指出,波长特定的比值 \( \frac{E(\lambda, T, i)}{a(\lambda, T, i)} \) 对所有物体(即对于所有索引 i 的值)具有相同的值。在这篇报告中,没有提到黑体的概念。

在1860年,基尔霍夫仍然没有了解到斯图尔特对选定辐射特性的测量,他指出,实验上早已证明,对于总热辐射——即未经选择的质量,物体在热平衡状态下所发射和吸收的辐射——其有量纲的总辐射比 \( \frac{E(T, i)}{a(T, i)} \) 对所有物体具有相同的值,也就是说,对于材料指数 i 的每个值,该比值是统一的。[52] 同样,基尔霍夫没有提供辐射功率的测量或其他新的实验数据,他提出了自己新的理论证明,证明了波长特定的发射功率与吸收比的比值 \( \frac{E(\lambda, T, i)}{a(\lambda, T, i)} \) 在热平衡下是普遍适用的。他的新理论证明仍被一些学者认为是无效的。[48][51]

但更重要的是,他的证明依赖于一个新的理论假设,即“完美的黑体”,这也是为何我们谈到基尔霍夫定律的原因。此类黑体在其无限薄的最外层表面上表现出完全的吸收。它们与巴尔福·斯图尔特所说的参考物体相对应,后者的内部辐射被灯黑涂层覆盖。它们不同于普朗克后期提出的更为现实的完美黑体。普朗克的黑体仅通过其内部材料辐射和吸收;它们与邻接介质的界面仅是数学表面,既不吸收也不发射,而只能反射和透射,并且具备折射能力。[53]

基尔霍夫的证明考虑了一个任意的非理想物体(标记为 i)和几个完美的黑体(标记为 BB)。该证明要求这些物体保持在一个温度为 T 的热平衡腔体中。基尔霍夫的证明意在展示,波长特定的发射功率与吸收比的比值 \( \frac{E(\lambda, T, i)}{a(\lambda, T, i)} \) 与非理想物体 i 的性质无关,无论该物体是部分透明还是部分反射的。

基尔霍夫的证明首先指出,对于波长 λ 和温度 T,在热平衡状态下,所有大小和形状相同的完美黑体的发射功率 \( E(\lambda, T, \text{BB}) \) 都具有相同的通用值,且该值具有功率的量纲。他的证明还指出,完美黑体的无量纲波长特定吸收比 \( a(\lambda, T, \text{BB}) \) 根据定义正好为 1。然后,对于完美黑体,波长特定的发射功率与吸收比的比值 \( \frac{E(\lambda, T, \text{BB})}{a(\lambda, T, \text{BB})} \) 仅仅是 \( E(\lambda, T, \text{BB}) \),其量纲为功率。

基尔霍夫随后依次考虑了与任意非理想物体以及与同样大小和形状的完美黑体在热平衡状态下的情况,假设这些物体处于温度为 T 的热平衡腔体中。他认为热辐射的流量在每种情况下必须相同。因此,他提出,在热平衡下,波长特定的发射功率与吸收比的比值 \( \frac{E(\lambda, T, i)}{a(\lambda, T, i)} \) 等于 \( E(\lambda, T, \text{BB}) \),现在可以表示为 \( B_\lambda (\lambda, T) \),这是一个连续函数,只依赖于固定温度 T 下的波长 λ,在低温下对于可见光波长消失,但对于更长的波长不消失,对于较高温度下的可见光波长具有正值,而这一值与非理想物体 i 的性质无关。(基尔霍夫详细考虑的几何因子在前述讨论中被忽略。)

因此,基尔霍夫的热辐射定律可以表述为:对于任何材料,在任何给定温度 T 下的热平衡状态中,无论是辐射还是吸收,对于每个波长 λ,发射功率与吸收比的比值具有一个普适的值,该值是完美黑体的特征,并且是一个我们用 \( B_\lambda (\lambda, T) \) 来表示的发射功率。(对于我们的符号 \( B_\lambda (\lambda, T) \),基尔霍夫最初的符号仅为 e。)[7][52][54][55][56][57]

基尔霍夫宣布,确定函数 \( B_\lambda (\lambda, T) \) 是一个至关重要的问题,尽管他意识到这将面临实验上的困难。他假设,像其他不依赖于个别物体性质的函数一样,这个函数应该是一个简单的函数。这个函数 \( B_\lambda (\lambda, T) \) 有时被称为“基尔霍夫(发射,普适)函数”[58][59][60][61],尽管其精确的数学形式直到 1900 年由普朗克发现才为人所知。基尔霍夫的普适性原理的理论证明在同一时期及以后被各种物理学家研究和讨论。[51] 基尔霍夫在 1860 年晚些时候表示,他的理论证明优于巴尔福·斯图尔特的,某些方面确实如此。[48] 基尔霍夫的 1860 年论文没有提到热力学第二定律,当然也没有提到在当时尚未建立的熵概念。基尔霍夫在 1862 年的一本书中更为详细地讲述了他定律与“卡诺原理”的联系,后者是热力学第二定律的一种形式。[62]

根据赫尔格·克拉赫(Helge Kragh)的说法,“量子理论的起源归功于热辐射的研究,特别是罗伯特·基尔霍夫在 1859-1860 年首次定义的‘黑体’辐射。”[63]

基尔霍夫的证明首先指出,对于波长 λ 和温度 T,在热平衡状态下,所有大小和形状相同的完美黑体的发射功率 \( E(\lambda, T, \text{BB}) \) 都具有相同的通用值,且该值具有功率的量纲。他的证明还指出,完美黑体的无量纲波长特定吸收比 \( a(\lambda, T, \text{BB}) \) 根据定义正好为 1。然后,对于完美黑体,波长特定的发射功率与吸收比的比值 \( \frac{E(\lambda, T, \text{BB})}{a(\lambda, T, \text{BB})} \) 仅仅是 \( E(\lambda, T, \text{BB}) \),其量纲为功率。

基尔霍夫随后依次考虑了与任意非理想物体以及与同样大小和形状的完美黑体在热平衡状态下的情况,假设这些物体处于温度为 T 的热平衡腔体中。他认为热辐射的流量在每种情况下必须相同。因此,他提出,在热平衡下,波长特定的发射功率与吸收比的比值 \( \frac{E(\lambda, T, i)}{a(\lambda, T, i)} \) 等于 \( E(\lambda, T, \text{BB}) \),现在可以表示为 \( B_\lambda (\lambda, T) \),这是一个连续函数,只依赖于固定温度 T 下的波长 λ,在低温下对于可见光波长消失,但对于更长的波长不消失,对于较高温度下的可见光波长具有正值,而这一值与非理想物体 i 的性质无关。(基尔霍夫详细考虑的几何因子在前述讨论中被忽略。)

因此,基尔霍夫的热辐射定律可以表述为:对于任何材料,在任何给定温度 T 下的热平衡状态中,无论是辐射还是吸收,对于每个波长 λ,发射功率与吸收比的比值具有一个普适的值,该值是完美黑体的特征,并且是一个我们用 \( B_\lambda (\lambda, T) \) 来表示的发射功率。(对于我们的符号 \( B_\lambda (\lambda, T) \),基尔霍夫最初的符号仅为 e。)[7][52][54][55][56][57]

基尔霍夫宣布,确定函数 \( B_\lambda (\lambda, T) \) 是一个至关重要的问题,尽管他意识到这将面临实验上的困难。他假设,像其他不依赖于个别物体性质的函数一样,这个函数应该是一个简单的函数。这个函数 \( B_\lambda (\lambda, T) \) 有时被称为“基尔霍夫(发射,普适)函数”[58][59][60][61],尽管其精确的数学形式直到 1900 年由普朗克发现才为人所知。基尔霍夫的普适性原理的理论证明在同一时期及以后被各种物理学家研究和讨论。[51] 基尔霍夫在 1860 年晚些时候表示,他的理论证明优于巴尔福·斯图尔特的,某些方面确实如此。[48] 基尔霍夫的 1860 年论文没有提到热力学第二定律,当然也没有提到在当时尚未建立的熵概念。基尔霍夫在 1862 年的一本书中更为详细地讲述了他定律与“卡诺原理”的联系,后者是热力学第二定律的一种形式。[62]

根据赫尔格·克拉赫(Helge Kragh)的说法,“量子理论的起源归功于热辐射的研究,特别是罗伯特·基尔霍夫在 1859-1860 年首次定义的‘黑体’辐射。”[63]
\subsubsection{普朗克在经验事实之前的观点,以及他最终发现辐射定律的过程}
普朗克在 1897 年首次将注意力转向黑体辐射问题。[80] 理论和实验的进展使得卢默(Lummer)和普林斯海姆(Pringsheim)在 1899 年写道,现有的实验证据大致与特定强度定律一致,该定律为 \( C\lambda^{-5} e^{-c/\lambda T} \),其中 \( C \) 和 \( c \) 为经验可测的常数,\( \lambda \) 和 \( T \) 分别表示波长和温度。[81][82] 出于理论原因,普朗克当时接受了这种公式,这个公式在短波长上有一个有效的截止值。[83][84][85]

古斯塔夫·基尔霍夫是马克斯·普朗克的导师,并推测存在一个普适的黑体辐射定律,这被称为“基尔霍夫的挑战”。[86] 作为理论物理学家,普朗克认为威廉·温(Wilhelm Wien)已经发现了这个定律,并在 1899 年将温的工作进行扩展,向德国物理学会报告。实验物理学家奥托·卢默(Otto Lummer)、费迪南德·库尔尔鲍姆(Ferdinand Kurlbaum)、恩斯特·普林斯海姆(Ernst Pringsheim Sr.)和海因里希·鲁本斯(Heinrich Rubens)进行了实验,似乎支持温的定律,特别是在较高频率的短波长区域,普朗克完全支持温的定律,并在德国物理学会上如此坚定地支持,以至于它开始被称为温-普朗克定律。[87] 然而,到了 1900 年 9 月,实验物理学家已经无可辩驳地证明温-普朗克定律在长波长区域失败。他们计划在 10 月 19 日发布他们的实验数据。普朗克的朋友鲁本斯将这一消息告知了他,普朗克很快在几天内提出了一个公式。[88] 同年 6 月,雷利(Lord Rayleigh)根据广泛接受的能量均分理论(equipartition theory)提出了一个适用于短波长低频率的公式。[89] 于是,普朗克提出了一个结合了雷利定律(或类似的能量均分理论)和温定律的公式,权衡不同波长下的两种定律,以匹配实验数据。然而,尽管这个公式有效,普朗克自己表示,除非他能够将这个“幸运直觉”推导出来的公式解释为具有“真实物理意义”的公式,否则它没有真正的意义。[90] 普朗克解释说,从那时起,他开始了他一生中最艰难的工作。普朗克不相信原子,也不认为热力学第二定律应该是统计的,因为概率不能提供绝对的答案,而玻尔兹曼的熵定律依赖于原子的假设,是统计的。但是,普朗克无法找到一种方式来将他的黑体公式与像麦克斯韦波动方程这样的连续定律调和。因此,在普朗克称之为“绝望的举动”[91] 中,他转向了玻尔兹曼的原子熵定律,因为这是唯一使他的公式成立的定律。因此,他使用了玻尔兹曼常数 \( k \) 和他的新常数 \( h \) 来解释黑体辐射定律,这个定律通过他的论文广为人知。[92][93]
\subsubsection{发现经验定律}
马克斯·普朗克在 1900 年 10 月 19 日提出了他的定律[94][95],这是对威廉·温(Wilhelm Wien)于 1896 年发表的近似公式的改进。温的公式适用于短波长(高频率)的实验数据,但在长波长(低频率)下与实验数据偏离。[41] 在 1900 年 6 月,基于启发式的理论考虑,雷利(Rayleigh)提出了一个公式[96],他建议可以通过实验来验证。这个建议是,斯图尔特–基尔霍夫的普适函数可能是类似于 \( c_1 T \lambda^{-4} \exp \left( -\frac{c_2}{\lambda T} \right) \) 的形式。这并不是著名的雷利-简斯公式 \( 8\pi k_B T \lambda^{-4} \),后者直到 1905 年才出现[38],尽管它在长波长时会简化为后者,而长波长恰是这里相关的波段。根据克莱因(Klein)的说法,[80] 可以推测普朗克可能见过这个建议,尽管他在 1900 年和 1901 年的论文中并未提及这一点。普朗克应当知道其他一些已提出的公式。[64][97] 1900 年 10 月 7 日,鲁本斯告诉普朗克,在补充领域(长波长、低频率),只有在该范围内,雷利 1900 年的公式才很好地拟合了观察数据。[97]

对于长波长,雷利 1900 年的启发式公式大致意味着能量与温度成正比,即 \( U_{\lambda} = \text{const.} \cdot T \)。[80][97][98] 已知 \( \frac{dS}{dU_{\lambda}} = \frac{1}{T} \),这导致 \( \frac{dS}{dU_{\lambda}} = \frac{\text{const.}}{U_{\lambda}} \),从而得出 \( \frac{d^2S}{dU_{\lambda}^2} = - \frac{\text{const.}}{U_{\lambda}^2} \) 适用于长波长。然而,对于短波长,温公式得出 \( \frac{1}{T} = - \text{const.} \ln U_{\lambda} + \text{const.} \),进而得出 \( \frac{d^2S}{dU_{\lambda}^2} = - \frac{\text{const.}}{U_{\lambda}} \) 适用于短波长。普朗克可能将这两种启发式公式——分别适用于长波长和短波长——结合在一起[97][99],得出了一个公式:
\[
\frac{d^2S}{dU_{\lambda}^2} = \frac{\alpha}{U_{\lambda} (\beta + U_{\lambda})}~
\]
这引导普朗克得出以下公式:
\[
B_{\lambda}(T) = \frac{C \lambda^{-5}}{e^{\frac{c}{\lambda T}} - 1}~
\]
其中普朗克使用 \( C \) 和 \( c \) 来表示经验拟合常数。

普朗克将这个结果发送给鲁本斯,鲁本斯将其与他和库尔尔鲍姆的观测数据进行比较,发现它对所有波长的拟合效果极为出色。1900 年 10 月 19 日,鲁本斯和库尔尔鲍姆简要报告了这个与数据的拟合[100],而普朗克则补充了一段简短的介绍,给出了他的公式的理论框架。[94] 一周之内,鲁本斯和库尔尔鲍姆提供了更详细的测量报告,确认了普朗克定律的正确性。他们用于分辨长波长辐射的技术被称为“残余射线法”。这些射线反复从抛光的晶体表面反射,最终能够完全通过过程的射线被称为“残余射线”,它们的波长是晶体特定材料优先反射的波长。[101][102][103]
\subsubsection{寻找定律的物理解释}
一旦普朗克发现了经验上合适的函数,他便开始构建该定律的物理推导。他的思路主要围绕熵展开,而不是直接关于温度。普朗克考虑了一个具有完美反射壁的腔体;在腔体内部,有有限多个不同但结构相同的共振振荡体,这些振荡体的幅度是确定的,每个共振振荡体对应若干个特定频率的振荡器。对于普朗克而言,这些假设的振荡器仅仅是理论上用于调查的工具,他提到这些振荡器“不需要在自然界中‘真实存在’,只要它们的存在及其性质与热力学和电动力学定律相一致即可。”[104] 普朗克并没有将这些共振振荡器的假设赋予任何明确的物理意义,而是提出它作为一个数学工具,使他能够推导出一个与所有波长的实验数据匹配的黑体辐射谱表达式。[105] 他也略微提到这些振荡器可能与原子有关。从某种意义上讲,这些振荡器对应于普朗克的碳微粒;这些微粒的大小可以非常小,尽管腔体的大小可能很大,只要这些微粒有效地在辐射波长模式之间转换能量。[97]

部分地跟随玻尔兹曼为气体分子开创的启发式计算方法,普朗克考虑了如何将电磁能量分布到他假设的带电材料振荡器的不同模式上。接受这种概率方法,尤其是玻尔兹曼的方法,对普朗克来说是一个根本性的变化,因为在此之前,他一直反对玻尔兹曼提出的这种思路。[106] 普朗克曾说:“我曾将[量子假设]视为一种纯粹的形式假设,除了一个理由,我并未多加思考:那就是我无论如何在任何情况下都得到了一个正面的结果,哪怕代价很大。”[107] 启发式地,玻尔兹曼将能量分布在任意的仅数学意义上的量子 \( \epsilon \) 中,这些量子的幅度趋向于零,因为有限的 \( \epsilon \) 仅用于在数学计算概率时进行计数,并不具备物理意义。普朗克引入了一个新的自然常数 \( h \)[108],并假设,在每个有限频率的振荡器中,总能量以一个确定的物理单位 \( \epsilon \) 来分配,这个单位的能量是各自特定频率的特征。[95][109][110][111] 他的这个新的自然常数 \( h \) 现在被称为普朗克常数。

普朗克进一步解释[95],每个特定单位 \( \epsilon \) 的能量应该与假设的振荡器的特征频率 \( \nu \) 成正比,且在 1901 年他用比例常数 \( h \) 表达了这一点:[112][113]
\[
\epsilon = h\nu~
\]
普朗克并没有提出自由空间中传播的光是量子化的。[114][115][116] 自由电磁场量子化的思想后来才被发展,并最终融入了我们现在所知的量子场论中。[117]

在 1906 年,普朗克承认,他的假想共振器具有线性动力学,无法为频率之间的能量转换提供物理解释。[118][119] 当今的物理学通过原子的量子激发性来解释频率之间的转换,这一思想源于爱因斯坦。普朗克认为,在一个具有完美反射壁且没有物质存在的腔体中,电磁场不能在频率成分之间交换能量。[120] 这是因为麦克斯韦方程的线性特性。[121] 现代的量子场论预测,在没有物质的情况下,电磁场服从非线性方程,并且在这个意义上存在自相互作用。[122][123] 这种在没有物质的情况下的相互作用尚未被直接测量,因为它需要非常高的强度和非常敏感且低噪声的探测器,这些探测器仍在建设中。[122][124] 普朗克认为,只有没有相互作用的场既不遵循也不违反经典的能量均分原理,[125][126] 而是保持在引入时的状态,而不是演化为黑体场。[127] 因此,他的机械假设的线性特性使得普朗克无法对热辐射场在热力学平衡中的熵最大化提供机械解释。这就是他必须诉诸玻尔兹曼的概率论证的原因。[128][129]

普朗克定律可以被视为完成了古斯塔夫·基尔霍夫对其热辐射定律重要性的预言。在他对自己定律的成熟表述中,普朗克提供了一个彻底且详细的理论证明,证明了基尔霍夫定律,[130] 该定律的理论证明直到那时一直存在争议,部分原因是它被认为依赖于不符合物理现实的理论对象,例如基尔霍夫的完美吸收无限薄的黑色表面。[131]
\subsubsection{随后的事件}
直到普朗克在做出能量或作用的抽象元素的启发式假设五年后,阿尔伯特·爱因斯坦才在 1905 年提出了光量子的概念,作为黑体辐射、光致发光、光电效应和紫外线辐射离子化气体的革命性解释。[132] 1905 年,“爱因斯坦认为普朗克的理论无法与光量子的思想相吻合,这一错误他在 1906 年予以纠正。”[133] 与普朗克当时的信念相反,爱因斯坦提出了一个模型和公式,认为光是在自由空间中以能量量子形式发射、吸收并传播,这些量子局限于空间的某些点。[132] 作为他的推理的引言,爱因斯坦回顾了普朗克假设的共振材料电振荡器模型,认为这些振荡器是辐射的源和汇,但随后他提供了一个新的论点,这个论点与普朗克的模型无关,部分基于温的热力学论证,其中普朗克公式 \( \epsilon = h\nu \) 没有起到作用。[134] 爱因斯坦给出了这些量子的能量内容,形式为 \( \frac{R\beta \nu}{N} \)。因此,爱因斯坦在否定普朗克的波动理论,认为光具有量子性方面,与普朗克发生了矛盾。1910 年,在批评普朗克寄给他的手稿时,爱因斯坦写信给普朗克:“对我来说,能量在空间中连续分布而不假设存在以太是荒谬的。”[135]

根据托马斯·库恩的说法,直到 1908 年,普朗克才或多或少地接受了爱因斯坦对于热辐射物理中物理离散性与抽象数学离散性区别的部分论证。即便在 1908 年,当考虑到爱因斯坦提出的量子传播模型时,普朗克仍然认为这种革命性步骤或许是不必要的。[136] 在此之前,普朗克一直认为,作用量子化的离散性既不应存在于他的共振振荡器中,也不应存在于热辐射的传播中。库恩指出,在普朗克早期的论文和他 1906 年的专著中,[137] 并没有提到“间断性”,也没有谈到“对振荡器能量的限制”,或者出现过类似“\( U = nh\nu \)”的公式。库恩进一步指出,他对普朗克 1900 年和 1901 年的论文,以及他 1906 年专著的研究,[137] 使他得出了“异端”的结论,这与其他人普遍的看法相悖,这些人只是从后来的、过时的观点来看待普朗克的著作。[138] 库恩的结论,指出普朗克直到 1908 年一直坚持他的“第一理论”,得到了其他历史学家的认可。[139]

在他的专著第二版中,普朗克在 1912 年继续坚持反对爱因斯坦的光量子理论。他详细地提出,光的吸收可能是连续的,发生在平衡状态下的常定速率,而不同于量子化的吸收。只有光的发射是量子化的。[121][140] 这有时被称为普朗克的“第二理论”。[141]

直到 1919 年,在他的专著第三版中,普朗克或多或少接受了他的“第三理论”,即光的发射和吸收都是量子化的。[142]

“紫外灾难”这一生动的术语是保罗·埃伦费斯特于 1911 年提出的,指的是当将经典统计力学的能量均分定理(错误地)应用于黑体辐射时,总能量趋向于无限大的悖论结果。[143][144] 但这并不是普朗克思考的一部分,因为他并没有尝试应用能量均分定理:当他在 1900 年做出发现时,并没有注意到任何形式的“灾难”。[83][84][85][80][145] 这一点最早是由劳德·雷利于 1900 年注意到的,[96][146][147] 然后在 1901 年由詹姆斯·吉恩斯提出;之后,在 1905 年,爱因斯坦提出支持光以离散光量子传播的观点,雷利[39] 和吉恩斯[38][149][150][151] 也提出了这一观点。

1913 年,玻尔给出了另一种公式,赋予 \( h\nu \) 这一量的不同物理意义。[34][35][36][152][153][154] 与普朗克和爱因斯坦的公式不同,玻尔的公式明确地指向原子的能级。玻尔的公式为 \( W_{\tau_2} - W_{\tau_1} = h\nu \),其中 \( W_{\tau_2} \) 和 \( W_{\tau_1} \) 表示原子量子态的能量,量子数为 \( \tau_2 \) 和 \( \tau_1 \)。符号 \( \nu \) 表示辐射量子的频率,当原子在这两个量子态之间跃迁时,辐射可以被发射或吸收。与普朗克的模型不同,频率 \( \nu \) 并不直接与可能描述这些量子态的频率相关。

后来,1924 年,萨钦德拉·纳特·博斯发展了光子的统计力学理论,允许理论上推导出普朗克定律。[155] 实际上,“光子”这一词是在 1926 年由 G.N. 刘易斯发明的,[156] 他错误地认为光子是守恒的,与玻色–爱因斯坦统计相悖;然而,这个词被采用来表达爱因斯坦关于光传播是以光量子形式传播的假设。在普朗克所考虑的那种完美反射壁的真空隔离电磁场中,确实根据爱因斯坦 1905 年的模型,光子是守恒的,但刘易斯提到的是光子的场,这个场被认为是一个相对于物质封闭的系统,但可以与周围的物质交换电磁能量,他错误地认为光子依然是守恒的,被储存在原子内部。

最终,普朗克的黑体辐射定律为爱因斯坦提出的光量子携带线性动量的概念提供了基础,[34][132] 这一概念成为量子力学发展的基本基础。

上述普朗克机械假设的线性特性,未能考虑频率分量之间的能量相互作用,后来在 1925 年被海森堡的原始量子力学所取代。在他 1925 年 7 月 29 日提交的论文中,海森堡的理论解释了玻尔 1913 年公式中的问题。它接受了非线性振荡器作为原子量子态的模型,允许这些非线性振荡器在发射或吸收辐射量子时,发生内部多个离散傅里叶频率分量之间的能量相互作用。辐射量子的频率是原子内稳定的振荡量子态之间的一个明确耦合的频率。[157][158] 当时,海森堡并不知道矩阵代数,但马克斯·玻恩阅读了海森堡的论文,并意识到海森堡理论中的矩阵特征。随后,玻恩和乔丹发布了量子力学的显式矩阵理论,基于海森堡的原始量子力学,但在形式上有显著不同;今天被称为矩阵力学的理论就是玻恩和乔丹的矩阵理论。[159][160][161] 海森堡对普朗克振荡器的解释,作为傅里叶模式下的非线性效应,表明普朗克的振荡器,若被视为经典物理学中可能构想的持久物理对象,无法充分解释相关现象。

今天,作为光量子能量的表述,常见的公式是 \( E = \hbar \omega \),其中 \( \hbar = \frac{h}{2\pi} \),且 \( \omega = 2\pi\nu \) 表示角频率,[162][163][164][165][166] 而较少见的公式是 \( E = h\nu \)。[165][166][167][168][169] 这一关于真正存在且传播的光量子的表述,基于爱因斯坦的理论,具有不同于普朗克上述关于抽象能
\subsection{另见}  
\begin{itemize}
\item 发射率  
\item 辐射亮度  
\item 坂田–服部方程
\end{itemize}
\subsection{参考文献}  
\begin{enumerate}
\item Young, Hugh D.; Freedman, Roger A.; Ford, A. Lewis (2016). 《大学物理学》(第14版). Pearson. 第1256–1257页. ISBN 9780321973610.  
\item Planck 1914, 第42页  
\item Gaofeng Shao 等, 2019, 第6页  
\item Zangwill, Andrew (2013). 《现代电动力学》. 剑桥大学出版社. 第698页. ISBN 978-0-521-89697-9.  
\item Andrews, David G. (2010). 《大气物理学导论》(第2版). 剑桥大学出版社. 第54页. ISBN 978-0-521-87220-1.  
\item Planck 1914, 第6、168页  
\item Chandrasekhar 1960, 第8页  
\item Rybicki & Lightman 1979, 第22页  
\item Stewart 1858  
\item Hapke 1993, 第362–373页  
\item Planck 1914  
\item Loudon 2000, 第3–45页  
\item Caniou 1999, 第117页  
\item Kramm & Mölders 2009, 第10页  
\item Sharkov 2003, 第210页  
\item Marr, Jonathan M.; Wilkin, Francis P. (2012). "A Better Presentation of Planck's Radiation Law". 《美国物理学杂志》 80 (5): 399. arXiv:1109.3822. Bibcode:2012AmJPh..80..399M. doi:10.1119/1.3696974. S2CID 10556556.  
\item Fischer 2011  
\item Goody & Yung 1989, 第16页  
\item Mohr, Taylor & Newell 2012, 第1591页  
\item Loudon 2000
\item Mandel & Wolf 1995  
\item Wilson 1957, 第182页  
\item Adkins 1983, 第147–148页  
\item Landsberg 1978, 第208页  
\item Siegel & Howell 2002, 第25页  
\item Planck 1914, 第9–11页  
\item Planck 1914, 第35页  
\item Landsberg 1961, 第273–274页  
\item Born & Wolf 1999, 第194–199页  
\item Born & Wolf 1999, 第195页  
\item Rybicki & Lightman 1979, 第19页  
\item Chandrasekhar 1960, 第7页  
\item Chandrasekhar 1960, 第9页  
\item Einstein 1916  
\item Bohr 1913  
\item Jammer 1989, 第113, 115页  
\item Kittel & Kroemer 1980, 第98页  
\item Jeans 1905a, 第98页  
\item Rayleigh 1905  
\item Rybicki & Lightman 1979, 第23页  
\item Wien 1896, 第667页  
\item Planck 1906, 第158页  
\item Lowen & Blanch 1940  
\item 来自Christian Gueymard (2004年4月) 的综合数据. "太阳能总辐射和谱辐射用于太阳能应用和太阳辐射模型". 《太阳能》, 76 (4): 423–453. Bibcode:2004SoEn...76..423G. doi:10.1016/j.solener.2003.08.039.  
\item Zettili, Nouredine (2009). 《量子力学:概念与应用》(第2版). Chichester: Wiley. 第594–596页. ISBN 978-0-470-02679-3.  
\item Segre, Carlo. "爱因斯坦系数 - 量子理论基础 II (PHYS 406)" (PDF). 第32页.  
\item Zwiebach, Barton. "量子物理 III 第四章:时变扰动理论 | 量子物理 III | 物理学". MIT OpenCourseWare. 第108–110页. 检索日期:2023年11月3日.  
\item Siegel 1976  
\item Kirchhoff 1860a  
\item Kirchhoff 1860b  
\item Schirrmacher 2001  
\item Kirchhoff 1860c  
\item Planck 1914, 第11页  
\item Milne 1930, 第80页  
\item Rybicki & Lightman 1979, 第16–17页  
\item Mihalas & Weibel-Mihalas 1984, 第328页  
\item Goody & Yung 1989, 第27–28页  
\item Paschen, F. (1896), 由Hermann 1971引用的私人信件,第6页  
\item Hermann 1971, 第7页  
\item Kuhn 1978, 第8, 29页  
\item Mehra & Rechenberg 1982, 第26, 28, 31, 39页  
\item Kirchhoff 1862, 第573页  
\item Kragh 1999, 第58页  
\item Kangro 1976  
\item Tyndall 1865a  
\item Tyndall 1865b
\item Kangro 1976, 第8–10页  
\item Crova 1880  
\item Crova 1880, 第577页,图版I  
\item Kangro 1976, 第10–15页  
\item Kangro 1976, 第15–26页  
\item Michelson 1888  
\item Kangro 1976, 第30–36页  
\item Kangro 1976, 第122–123页  
\item Lummer & Kurlbaum 1898  
\item Kangro 1976, 第159页  
\item Lummer & Kurlbaum 1901  
\item Kangro 1976, 第75–76页  
\item Paschen 1895, 第297–301页  
\item Klein 1962, 第460页  
\item Lummer & Pringsheim 1899, 第225页  
\item Kangro 1976, 第174页  
\item Planck 1900d  
\item Rayleigh 1900, 第539页  
\item Kangro 1976, 第181–183页  
\item Pasupathy, J. "量子、它的发现与持续的探索." 《当前科学》,第79卷,第11期,临时出版商,2000年,第1609–1614页,http://www.jstor.org/stable/24104871.  
\item Kumar, Manjit, 《量子:爱因斯坦、波尔与关于现实本质的大辩论》,美国第一版,2008年。[缺少ISBN][需要页码]  
\item Stone, A. Douglas, 《爱因斯坦与量子:勇敢的斯瓦比亚人的探索》,2013年,普林斯顿大学出版社。[缺少ISBN][需要页码]  
\item "关于完全辐射定律的评论",发表于《伦敦、爱丁堡与都柏林哲学杂志与科学期刊》,第XLIX卷,1900年1月–6月,第539–541页,Rayleigh,Lord(约翰·威廉·斯特拉特)  
\item Planck,《科学自传与其他论文》(纽约:哲学图书馆,1949年),第41页  
\item Hermann, "量子理论的起源",《美国物理学杂志》40, 1355(1972年),第23页  
\item Max Planck, "关于正常谱能量分布定律的理论",《德国物理学会会报》,第2卷,(1900年)  
\item Physics World, "马克斯·普朗克:不情愿的革命者",2000年12月1日。引用:“根据玻尔兹曼的分子力学解释,系统的熵是分子运动的集体结果。第二定律只有在统计意义上有效。玻尔兹曼的理论假设了原子和分子的存在,但遭到威廉·奥斯特瓦尔德和其他‘能量学派’的挑战,他们希望将物理学从原子概念中解放出来,并将其建立在能量及相关量的基础上。普朗克在这场争论中的立场是什么?我们可能会期望他站在胜者一方,或者那些很快被证明是胜者的一方——即玻尔兹曼和‘原子论者’。但事实并非如此。普朗克对第二定律绝对有效性的信仰使他不仅拒绝了玻尔兹曼的统计热力学版本,还对其所依赖的原子假设表示怀疑。” https://physicsworld.com/a/max-planck-the-reluctant-revolutionary/  \item Planck 1900a  
\item Planck 1900b  
\item Rayleigh 1900  
\item Dougal 1976  
\item Planck 1943, 第156页  
\item Hettner 1922
\item Rubens & Kurlbaum 1900a  
\item Rubens & Kurlbaum 1900b  
\item Kangro 1976, 第165页  
\item Mehra & Rechenberg 1982, 第41页  
\item Planck 1914, 第135页  
\item Kuhn 1978, 第117–118页  
\item Hermann 1971, 第16页  
\item Planck 致罗伯特·威廉·伍兹,1931年10月7日,收录于阿尔敏·赫尔曼,《量子理论的起源》(1899–1913),(马萨诸塞州剑桥:麻省理工学院出版社,1971年),第24页  
\item Planck 1900c  
\item Kangro 1976, 第214页  
\item Kuhn 1978, 第106页  
\item Kragh 2000  
\item Planck 1901  
\item Planck 1915, 第89页  
\item Ehrenfest & Kamerlingh Onnes 1914, 第873页  
\item ter Haar 1967, 第14页  
\item Stehle 1994, 第128页  
\item Scully & Zubairy 1997, 第21页  
\item Planck 1906, 第220页  
\item Kuhn 1978, 第162页  
\item Planck 1914, 第44–45页, 第113–114页  
\item Stehle 1994, 第150页  
\item Jauch & Rohrlich 1980, 第13章  
\item Karplus & Neuman 1951  
\item Tommasini et al. 2008  
\item Jeffreys 1973, 第223页  
\item Planck 1906, 第178页  
\item Planck 1914, 第26页  
\item Boltzmann 1878  
\item Kuhn 1978, 第38–39页  
\item Planck 1914, 第1–45页
\item Cotton 1899  
\item Einstein 1905  
\item Kragh 1999, 第67页  
\item Stehle 1994, 第132–137页  
\item Einstein 1993, 第143页,1910年信件  
\item Planck 1915, 第95页  
\item Planck 1906  
\item Kuhn 1978, 第196–202页  
\item Kragh 1999, 第63–66页  
\item Planck 1914, 第161页  
\item Kuhn 1978, 第235–253页  
\item Kuhn 1978, 第253–254页  
\item Ehrenfest 1911  
\item Kuhn 1978, 第152页  
\item Kuhn 1978, 第151–152页  
\item Kangro 1976, 第190页  
\item Kuhn 1978, 第144–145页  
\item Jeans 1901, 第398页脚注  
\item Jeans 1905b  
\item Jeans 1905c  
\item Jeans 1905d  
\item Sommerfeld 1923, 第43页  
\item Heisenberg 1925, 第108页  
\item Brillouin 1970, 第31页  
\item Bose 1924  
\item Lewis 1926  
\item Heisenberg 1925  
\item Razavy 2011, 第39–41页  
\item Born & Jordan 1925  
\item Stehle 1994, 第286页  
\item Razavy 2011, 第42–43页  
\item Messiah 1958, 第14页  
\item Pauli 1973, 第1页  
\item Feynman, Leighton & Sands 1963, 第38-1页  
\item Schwinger 2001, 第203页  
\item Bohren & Clothiaux 2006, 第2页  
\item Schiff 1949, 第2页  
\item Mihalas & Weibel-Mihalas 1984, 第143页  
\item Rybicki & Lightman 1979, 第20页
\end{enumerate}
\subsubsection{参考书目}
\begin{itemize}
\item Adkins, C. J. (1983). *Equilibrium Thermodynamics* (第三版). 剑桥大学出版社. ISBN 978-0-521-25445-8.  
\item Andrews, David (2000). *An Introduction to Atmospheric Physics*. 剑桥大学出版社. ISBN 0511800770.  
\item Bohr, N. (1913). "On the constitution of atoms and molecules" (PDF). *Philosophical Magazine*. 26 (153): 1–25. Bibcode:1913PMag...26..476B. doi:10.1080/14786441308634993.  
\item Bohren, C. F.; Clothiaux, E. E. (2006). *Fundamentals of Atmospheric Radiation*. Wiley-VCH. ISBN 978-3-527-40503-9.  
\item Boltzmann, L. (1878). "Über die Beziehung zwischen dem zweiten Hauptsatze der mechanischen Wärmetheorie und der Wahrscheinlichkeitsrechnung, respective den Sätzen über das Wärmegleichgewicht". *Sitzungsberichte Mathematisch-Naturwissenschaftlichen Classe der Kaiserlichen Akademie der Wissenschaften in Wien*. 76 (2): 373–435.  
\item Born, M.; Wolf, E. (1999). *Principles of Optics* (第七版). 剑桥大学出版社. ISBN 978-0-521-64222-4.  
\item Born, M.; Jordan, P. (1925). "Zur Quantenmechanik". *Zeitschrift für Physik*. 34 (1): 858–888. Bibcode:1925ZPhy...34..858B. doi:10.1007/BF01328531. S2CID 186114542.  
  翻译部分收录于 van der Waerden, B. L. (1967). *Sources of Quantum Mechanics*. North-Holland Publishing. pp. 277–306.  
\item Bose, Satyendra Nath (1924). "Plancks Gesetz und Lichtquantenhypothese". *Zeitschrift für Physik* (德文). 26 (1): 178–181. Bibcode:1924ZPhy...26..178B. doi:10.1007/BF01327326. S2CID 186235974.  
\item Brehm, J. J.; Mullin, W. J. (1989). *Introduction to the Structure of Matter*. Wiley. ISBN 978-0-471-60531-7.
\item Brillouin, L. (1970). *Relativity Reexamined*. Academic Press. ISBN 978-0-12-134945-5.  
\item Caniou, J. (1999). *Passive Infrared Detection: Theory and Applications*. Springer. ISBN 978-0-7923-8532-5.  
\item Chandrasekhar, S. (1960) [1950]. *Radiative Transfer* (修订再版). Dover Publications. ISBN 978-0-486-60590-6.  
\item Cotton, A. (1899). "The present status of Kirchhoff's law". *The Astrophysical Journal*. 9: 237–268. Bibcode:1899ApJ.....9..237C. doi:10.1086/140585.  
\item Crova, A. P. P. (1880). "Étude des radiations émises par les corps incandescents. Mesure optique des hautes températures". *Annales de chimie et de physique*. Série 5. 19: 472–550.  
\item Dougal, R. C. (1976). "The presentation of the Planck radiation formula (tutorial)". *Physics Education*. 11 (6): 438–443. Bibcode:1976PhyEd..11..438D. doi:10.1088/0031-9120/11/6/008. S2CID 250881729.  
\item Ehrenfest, P. (1911). "Welche Züge der Lichtquantenhypothese spielen in der Theorie der Wärmestrahlung eine wesentliche Rolle?". *Annalen der Physik*. 36 (11): 91–118. Bibcode:1911AnP...341...91E. doi:10.1002/andp.19113411106.  
\item Ehrenfest, P.; Kamerlingh Onnes, H. (1914). "Simplified deduction of the formula from the theory of combinations which Planck uses as the basis of his radiation theory". *Proceedings of the Royal Dutch Academy of Sciences in Amsterdam*. 17 (2): 870–873. Bibcode:1914KNAB...17..870E.  
\item Einstein, A. (1905). "Über einen die Erzeugung und Verwandlung des Lichtes betreffenden heuristischen Gesichtspunkt". *Annalen der Physik*. 17 (6): 132–148. Bibcode:1905AnP...322..132E. doi:10.1002/andp.19053220607.  
  翻译见 Arons, A. B.; Peppard, M. B. (1965). "Einstein's proposal of the photon concept: A translation of the Annalen der Physik paper of 1905" (PDF). *American Journal of Physics*. 33 (5): 367. Bibcode:1965AmJPh..33..367A. doi:10.1119/1.1971542.  
\item Einstein, A. (1916). "Zur Quantentheorie der Strahlung". *Mitteilungen der Physikalischen Gesellschaft Zürich*. 18: 47–62. 以及几乎相同的版本:Einstein, A. (1917). "Zur Quantentheorie der Strahlung". *Physikalische Zeitschrift*. 18: 121–128. Bibcode:1917PhyZ...18..121E.  
  翻译见 ter Haar, D. (1967). "On the Quantum Theory of Radiation". *The Old Quantum Theory*. Pergamon Press. pp. 167–183. LCCN 66029628.  
\item Einstein, A. (1993). *The Collected Papers of Albert Einstein*. 第3卷。由 Beck, A. 翻译。普林斯顿大学出版社。ISBN 978-0-691-10250-4.  
\item Feynman, R. P.; Leighton, R. B.; Sands, M. (1963). *The Feynman Lectures on Physics, Volume 1*. Addison-Wesley. ISBN 978-0-201-02010-6.  
\item Fischer, T. (2011年11月1日). "Topics: Derivation of Planck's Law". *ThermalHUB*. 2015年6月19日检索。  
\item Gaofeng Shao; Yucao Lu; Dorian A.H. Hanaor; Sheng Cui; Jian Jiao; Xiaodong Shen (2019). "Improved oxidation resistance of high emissivity coatings on fibrous ceramic for reusable space systems". *Corrosion Science*. 146. Elsevier: 233–246. arXiv:1902.03943. Bibcode:2019Corro.146..233S. doi:10.1016/j.corsci.2018.11.006. S2CID 118927116. HAL Id: hal-02308467 – via HAL archives ouverts.  
\item Goody, R. M.; Yung, Y. L. (1989). *Atmospheric Radiation: Theoretical Basis* (第二版). 牛津大学出版社. ISBN 978-0-19-510291-8.  
\item Guggenheim, E. A. (1967). *Thermodynamics. An Advanced Treatment for Chemists and Physicists* (第五版修订版). North-Holland Publishing Company.  
\item Haken, H. (1981). *Light* (再版). 阿姆斯特丹: North-Holland Publishing. ISBN 978-0-444-86020-0.
\item Hapke, B. (1993). *Theory of Reflectance and Emittance Spectroscopy*. 剑桥大学出版社,英国剑桥。ISBN 978-0-521-30789-5.  
\item Heisenberg, W. (1925). "Über quantentheoretische Umdeutung kinematischer und mechanischer Beziehungen". *Zeitschrift für Physik*. 33 (1): 879–893. Bibcode:1925ZPhy...33..879H. doi:10.1007/BF01328377. S2CID 186238950.  
  翻译为 "Quantum-theoretical Re-interpretation of kinematic and mechanical relations" 见 van der Waerden, B. L. (1967). *Sources of Quantum Mechanics*. North-Holland Publishing. pp. 261–276.  
\item Heisenberg, W. (1930). *The Physical Principles of the Quantum Theory*. 由 Eckart, C.; Hoyt, F. C. 翻译。芝加哥大学出版社。  
\item Hermann, A. (1971). *The Genesis of Quantum Theory*. 由 Nash, C.W. 翻译。麻省理工学院出版社。ISBN 978-0-262-08047-7.  
  翻译自 *Frühgeschichte der Quantentheorie (1899–1913)*,Physik Verlag,莫斯巴赫/巴登,1969年。  
\item Hettner, G. (1922). "Die Bedeutung von Rubens Arbeiten für die Plancksche Strahlungsformel". *Naturwissenschaften*. 10 (48): 1033–1038. Bibcode:1922NW.....10.1033H. doi:10.1007/BF01565205. S2CID 46268714.  
\item Jammer, M. (1989). *The Conceptual Development of Quantum Mechanics* (第二版)。Tomash Publishers/American Institute of Physics。ISBN 978-0-88318-617-6.  
\item Jauch, J. M.; Rohrlich, F. (1980) [1955]. *The Theory of Photons and Electrons. The Relativistic Quantum Field Theory of Charged Particles with Spin One-half* (第二版再版)。Springer。ISBN 978-0-387-07295-1.  
\item Jeans, J. H. (1901). "The Distribution of Molecular Energy". *Philosophical Transactions of the Royal Society A*. 196 (274–286): 397–430. Bibcode:1901RSPTA.196..397J. doi:10.1098/rsta.1901.0008. JSTOR 90811.  
\item Jeans, J. H. (1905a). "XI. On the partition of energy between matter and æther". *Philosophical Magazine*. 10 (55): 91–98. doi:10.1080/14786440509463348.  
\item Jeans, J. H. (1905b). "On the Application of Statistical Mechanics to the General Dynamics of Matter and Ether". *Proceedings of the Royal Society A*. 76 (510): 296–311. Bibcode:1905RSPSA..76..296J. doi:10.1098/rspa.1905.0029. JSTOR 92714.  
\item Jeans, J. H. (1905c). "A Comparison between Two Theories of Radiation". *Nature*. 72 (1865): 293–294. Bibcode:1905Natur..72..293J. doi:10.1038/072293d0. S2CID 3955227.  
\item Jeans, J. H. (1905d). "On the Laws of Radiation". *Proceedings of the Royal Society A*. 76 (513): 545–552. Bibcode:1905RSPSA..76..545J. doi:10.1098/rspa.1905.0060. JSTOR 92704.  
\item Jeffreys, H. (1973). *Scientific Inference* (第三版)。剑桥大学出版社。ISBN 978-0-521-08446-8.  
\item Kangro, H. (1976). *Early History of Planck's Radiation Law*. Taylor & Francis. ISBN 978-0-85066-063-0.  
\item Karplus, R.; Neuman, M. (1951). "The Scattering of Light by Light". *Physical Review*. 83 (4): 776–784. Bibcode:1951PhRv...83..776K. doi:10.1103/PhysRev.83.776.  
\item Kirchhoff, G. R. (1860a). "Über die Fraunhofer'schen Linien". *Monatsberichte der Königlich Preussischen Akademie der Wissenschaften zu Berlin*: 662–665.  
\item Kirchhoff, G. R. (1860b). "Über den Zusammenhang zwischen Emission und Absorption von Licht und Wärme". *Monatsberichte der Königlich Preussischen Akademie der Wissenschaften zu Berlin*: 783–787
\item Kirchhoff, G. R. (1860c). "Über das Verhältniss zwischen dem Emissionsvermögen und dem Absorptionsvermögen der Körper für Wärme und Licht". *Annalen der Physik und Chemie*. 109 (2): 275–301. Bibcode:1860AnP...185..275K. doi:10.1002/andp.18601850205.  
  由 Guthrie, F. 翻译为 *Kirchhoff, G. R. (1860). "On the relation between the radiating and absorbing powers of different bodies for light and heat"*. *Philosophical Magazine*. Series 4. 20: 1–21.  
\item Kirchhoff, G. R. (1862). "Über das Verhältniss zwischen dem Emissionsvermögen und dem Absorptionsvermögen der Körper für Wärme und Licht", *Gessamelte Abhandlungen*, Johann Ambrosius Barth, pp. 571–598.  
\item Kittel, C.; Kroemer, H. (1980). *Thermal Physics* (第二版)。W. H. Freeman。ISBN 978-0-7167-1088-2.  
\item Klein, M. J. (1962). "Max Planck and the beginnings of the quantum theory". *Archive for History of Exact Sciences*. 1 (5): 459–479. doi:10.1007/BF00327765. S2CID 121189755.  
\item Kragh, H. (1999). *Quantum Generations. A History of Physics in the Twentieth Century*. 普林斯顿大学出版社。ISBN 978-0-691-01206-3.  
\item Kragh, H. (2000年12月). "Max Planck: The reluctant revolutionary". *Physics World*. 13 (12): 31–36. doi:10.1088/2058-7058/13/12/34.  
\item Kramm, Gerhard; Mölders, N. (2009). "Planck's Blackbody Radiation Law: Presentation in Different Domains and Determination of the Related Dimensional Constant". *Journal of the Calcutta Mathematical Society*. 5 (1–2): 27–61. arXiv:0901.1863. Bibcode:2009arXiv0901.1863K.  
\item Kuhn, T. S. (1978). *Black–Body Theory and the Quantum Discontinuity*. 牛津大学出版社。ISBN 978-0-19-502383-1.  
\item Landsberg, P. T. (1961). *Thermodynamics with Quantum Statistical Illustrations*. Interscience Publishers.  
\item Landsberg, P. T. (1978). *Thermodynamics and Statistical Mechanics*. 牛津大学出版社。ISBN 978-0-19-851142-7.  
\item Lewis, G. N. (1926). "The Conservation of Photons". *Nature*. 118 (2981): 874–875. Bibcode:1926Natur.118..874L. doi:10.1038/118874a0. S2CID 4110026.  
\item Loudon, R. (2000). *The Quantum Theory of Light* (第三版)。牛津大学出版社。ISBN 978-0-19-850177-0.  
\item Lowen, A. N.; Blanch, G. (1940). "Tables of Planck's radiation and photon functions". *Journal of the Optical Society of America*. 30 (2): 70. Bibcode:1940JOSA...30...70L. doi:10.1364/JOSA.30.000070.  
\item Lummer, O.; Kurlbaum, F. (1898). "Der electrisch geglühte "absolut schwarze" Körper und seine Temperaturmessung". *Verhandlungen der Deutschen Physikalischen Gesellschaft*. 17: 106–111.  
\item Lummer, O.; Pringsheim, E. (1899). "1. Die Vertheilung der Energie in Spectrum des schwarzen Körpers und des blanken Platins; 2. Temperaturbestimmung fester glühender Körper". *Verhandlungen der Deutschen Physikalischen Gesellschaft*. 1: 215–235.  
\item Lummer, O.; Kurlbaum, F. (1901). "Der elektrisch geglühte "schwarze" Körper". *Annalen der Physik*. 310 (8): 829–836. Bibcode:1901AnP...310..829L. doi:10.1002/andp.19013100809.
\item Mandel, L.; Wolf, E. (1995). *Optical Coherence and Quantum Optics*. 剑桥大学出版社。ISBN 978-0-521-41711-2.  
\item Mehra, J.; Rechenberg, H. (1982). *The Historical Development of Quantum Theory*. Vol. 1. 施普林格出版社。ISBN 978-0-387-90642-3.  
\item Messiah, A. (1958). *Quantum Mechanics*. Temmer, G. G. (翻译). Wiley.  
\item Michelson, V. A. (1888). "Theoretical essay on the distribution of energy in the spectra of solids". *Philosophical Magazine*. Series 5. 25 (156): 425–435. doi:10.1080/14786448808628207.  
\item Mihalas, D.; Weibel-Mihalas, B. (1984). *Foundations of Radiation Hydrodynamics*. 牛津大学出版社。ISBN 978-0-19-503437-0.  
\item Milne, E. A. (1930). "Thermodynamics of the Stars". *Handbuch der Astrophysik*. 3 (1): 63–255.  
\item Mohr, P. J.; Taylor, B. N.; Newell, D. B. (2012). "CODATA Recommended Values of the Fundamental Physical Constants: 2010" (PDF). *Reviews of Modern Physics*. 84 (4): 1527–1605. arXiv:1203.5425. Bibcode:2012RvMP...84.1527M. doi:10.1103/RevModPhys.84.1527.  
\item Paltridge, G. W.; Platt, C. M. R. (1976). *Radiative Processes in Meteorology and Climatology*. Elsevier. ISBN 978-0-444-41444-1.  
\item Paschen, F. (1895). "Über Gesetzmäßigkeiten in den Spectren fester Körper und über ein neue Bestimmung der Sonnentemperatur". *Nachrichten von der Königlichen Gesellschaft der Wissenschaften zu Göttingen* (Mathematisch-Physikalische Klasse): 294–304.  
\item Pauli, W. (1973). Enz, C. P. (编). *Wave Mechanics*. Margulies, S.; Lewis, H. R. (翻译). MIT Press。ISBN 978-0-262-16050-6.  
\item Planck, M. (1900a). "Über eine Verbesserung der Wien'schen Spectralgleichung". *Verhandlungen der Deutschen Physikalischen Gesellschaft*. 2: 202–204.  
  由 ter Haar, D. (1967) 翻译为 "On an Improvement of Wien's Equation for the Spectrum" (PDF). *The Old Quantum Theory*. Pergamon Press. pp. 79–81.  
\item Planck, M. (1900b). "Zur Theorie des Gesetzes der Energieverteilung im Normalspectrum". *Verhandlungen der Deutschen Physikalischen Gesellschaft*. 2: 237–245.  
  由 ter Haar, D. (1967) 翻译为 "On the Theory of the Energy Distribution Law of the Normal Spectrum" (PDF). *The Old Quantum Theory*. Pergamon Press. p. 82.  
\item Planck, M. (1900c). "Entropie und Temperatur strahlender Wärme". *Annalen der Physik*. 306 (4): 719–737. Bibcode:1900AnP...306..719P. doi:10.1002/andp.19003060410.  
\item Planck, M. (1900d). "Über irreversible Strahlungsvorgänge". *Annalen der Physik*. 306 (1): 69–122. Bibcode:1900AnP...306...69P. doi:10.1002/andp.19003060105.  
\item Planck, M. (1901). "Über das Gesetz der Energieverteilung im Normalspektrum". *Annalen der Physik*. 4 (3): 553–563. Bibcode:1901AnP...309..553P. doi:10.1002/andp.19013090310.  
  由 Ando, K. 翻译为 "On the Law of Distribution of Energy in the Normal Spectrum" (PDF). 2011年10月6日档案保存。
\end{itemize}