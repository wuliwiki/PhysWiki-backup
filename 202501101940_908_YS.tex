% 衍射(综述)
% license CCBYSA3
% type Wiki

本文根据 CC-BY-SA 协议转载翻译自维基百科\href{https://en.wikipedia.org/wiki/Diffraction}{相关文章}。

\begin{figure}[ht]
\centering
\includegraphics[width=8cm]{./figures/c528f7b4a151de90.png}
\caption{红色激光束通过另一个板上的小圆孔后投射到板上的衍射图案} \label{fig_YS_1}
\end{figure}
不要与折射混淆,折射是指波从一种介质传递到另一种介质时,方向发生的变化。

衍射是波由于障碍物或通过孔径而偏离直线传播的现象。衍射物体或孔径实际上成为了传播波的二次源。衍射与干涉是相同的物理效应,但干涉通常应用于少数波的叠加,而当许多波叠加时,通常使用“衍射”一词。[1]: 433 

意大利科学家弗朗切斯科·玛丽亚·格里马尔迪(Francesco Maria Grimaldi)创造了“衍射”这个词,并在1660年首次准确记录了这一现象。
\begin{figure}[ht]
\centering
\includegraphics[width=8cm]{./figures/922006afafe01bb9.png}
\caption{沿着长度 \(d\) 的无数个点(展示了三个点)从波前投射出相位贡献,产生在注册板上持续变化的强度 \(I(\theta)\)} \label{fig_YS_2}
\end{figure}
在经典物理学中,衍射现象由**惠更斯–弗涅耳原理**描述,该原理将传播波前的每个点视为一组独立的球面波。衍射的特征性图案在当来自相干光源(如激光)的波遇到与其波长相当大小的狭缝/孔径时最为明显,如插图所示。这是由于波前上不同点(或等效地,每个波面波)的叠加或干涉,它们以不同的路径长度传播到接收表面。如果存在多个间距较近的开口,则可能会产生复杂的强度变化图案。

这些效应也发生在光波通过折射率变化的介质,或声波通过具有变化声阻的介质时——所有波都发生衍射,[包括引力波](#),水波,以及其他电磁波如X射线和无线电波。进一步来说,量子力学也表明物质具有波动性质,因此也会发生衍射(这一现象可以在亚原子到分子层面进行测量)。
\subsection{历史}
\begin{figure}[ht]
\centering
\includegraphics[width=8cm]{./figures/b01610604cdbbed3.png}
\caption{托马斯·杨于1803年向皇家学会展示的水波双缝衍射图示} \label{fig_YS_3}
\end{figure}
光的衍射效应最早由弗朗切斯科·玛丽亚·格里马尔迪仔细观察并表征,他还创造了‘衍射’这个术语,源自拉丁语 diffringere,意思是‘打碎成碎片’,指的是光线分散到不同的方向。[6] 格里马尔迪的观察结果于1665年在他去世后发表。[7][8] 艾萨克·牛顿研究了这些效应,并将其归因于光线的弯曲。詹姆斯·格雷戈里(1638-1675)观察到由鸟羽引起的衍射图样,这实际上是首个被发现的衍射光栅。[9] 托马斯·杨在1803年进行了著名的实验,演示了两个紧密间隔的狭缝产生的干涉现象。[10] 他通过解释从两个不同狭缝发出的波的干涉结果,推导出光必须以波的形式传播。

1818年,支持光的粒子理论的学者提出巴黎科学院奖问题应涉及衍射,期望看到波动理论被击败。然而,奥古斯丁·让·弗涅尔凭借他的新波传播理论赢得了奖项,[11] 该理论将克里斯蒂安·惠更斯的思想与杨的干涉概念相结合。[12] 西门·德尼·泊松通过证明弗涅尔理论预测在圆形障碍物的阴影后有光线,挑战了弗涅尔的理论;而多米尼克-弗朗索瓦-让·阿拉戈随后通过实验演示了这种光是可见的,从而确认了弗涅尔的衍射模型。[13]: xxiii [14]