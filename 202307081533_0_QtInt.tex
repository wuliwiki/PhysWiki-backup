% Qt 笔记

\begin{issues}
\issueDraft
\end{issues}

\begin{itemize}
\item \textbf{Qt Framework} 是最大的概念, Qt 的一切都属于 Framework。 所有的组件\href{https://doc.qt.io/qt-6/qtmodules.html}{见这里}。
\item License 选择: 以 GPL 或者 LGPL 使用 Qt 都是免费的。 LGPL 是说如果你的软件通过动态链接的方式使用 Qt, 那么调用的那部分程序不需要开源,只有你修改 Qt 库, 才需要把修改后的代码开源。 当然也有付费选项。
\item \textbf{Qt Widgets}: 传统, 最受欢迎, 复杂界面, 更底层的控制, 高性能, 适合桌面开发, 基于 widgets (按钮,文字框,菜单)
\item \textbf{Qt Quick}: 现代, 使用 \textbf{QML (Qt Meta-Object Language)}(类似于 HTML) 和 JS, 快速开发迭代, 更适配触屏和移动端(或者桌面移动混合开发)。
\item \textbf{Qt Creator} 是 Qt 的官方 IDE。 除此也可以用 Visual Studio 或者 Eclipse
\item \textbf{PyQt} 是 Qt C++ 库的老牌非官方 Python wrapper, 同样支持 Widget 和 Quick。 在创建 Qt Creator 工程的时候可以选择。
\item \textbf{PySide} 是 Qt 官方的 Python wrapper, 和 PyQt 的 API 基本一样, 在 Qt Creator 里面创建工程的时候可以直接用。 比 PyQt 对商用更友好, 据说功能更多。
\item \textbf{Qt Designer}: 是 Qt Widgets 的一个所见即所得图形编辑器。
\item \textbf{Qt Design Studio}: 是 Qt Quick 的一个图形编辑器, 可以生成代码。
\end{itemize}

\subsection{Qt Essentials}
\begin{itemize}
\item \textbf{Qt Core}: 关键的非 GUI 基础类。
\item \textbf{Qt GUI}: 基础的 GUI 类, 包括 OpenGL。
\item \textbf{Qt Network}: 网络功能库。
\item \textbf{Qt QML}: QML 既是模块名, 也是一个设计 UI 的语言。
\item \textbf{Qt SQL}: 支持 SQL 数据库。
\item \textbf{Qt Multimedia}: 视频,音乐,相机控制等。
\end{itemize}

\subsection{Qt Add-Ons}
\begin{itemize}
\item \textbf{Qt WebEngine}: 把网页内容做成桌面 app(类似于 Electron 吧), 相当于一个浏览器。
\item \textbf{Qt WebView}: 比 Qt WebEngine 更轻量级, 使用 QML。
\item \textbf{Windows Deploy}: 制作 Windows 安装包。
\item \textbf{Qt 3D}: 支持 3D 建模。
\item \textbf{Qt PDF}: 支持显示 pdf (不支持移动端)
\end{itemize}


\subsection{安装}
\subsubsection{官方安装(推荐)}
\begin{itemize}
\item 找到安装包\href{https://www.qt.io/download-qt-installer}{下载地址}
\item 下载对应系统的在线安装包(推荐), 是图形界面。 默认会安装到用户 home 文件夹。
\end{itemize}

\subsubsection{Ubuntu 命令行安装 Qt5}
\begin{itemize}
\item 如果有, 就用 \verb|suao apt install qt5-default|
\item 在 \verb|20.04| 以上的系统暂时没有 \verb|qt5-default|, 可以用 \verb|sudo apt install build-essential qtcreator|
\end{itemize}

\subsection{简单程序}
\begin{itemize}
\item 【试了没用】\href{https://doc.qt.io/qt-6.4/qtwidgets-tutorials-notepad-example.html}{简单的 qt Widget 教程}
\end{itemize}


\subsection{Ubuntu 22.04 尝试 Qt6}
\begin{itemize}
\item \href{https://www.youtube.com/watch?v=sjApF6qnyUI}{Youtube 视频教程}
\end{itemize}
