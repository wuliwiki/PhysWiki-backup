% Liapunov 稳定性(常微分方程)
% keys Liapunov|混沌|稳定性
% license Usr
% type Tutor

\pentry{李普希茨条件\nref{nod_LipCon}}{nod_d374}
\begin{issues}
\issueMissDepend
\end{issues}

微分方程中经常出现混沌系统,而讨论稳定性的问题直到现在都是较有难度的一件事。李雅普诺夫(Liapunov)稳定性是现在较为成熟的一套理论方法。

\subsection{Liapunov 稳定}
考虑微分方程组 
\begin{equation}
\dv{t} \bvec{x}(t) = \bvec{F}(t, \bvec x) ~,
\end{equation}
其中 $t \in \mathbb R$,$\bvec x \in D \subseteq \mathbb R^n$,$\bvec F$ 在 $\mathbb R \times D$ 上连续且关于 $\bvec x$ 满足李普希兹条件。若该方程组有解 $\bvec x = \bvec \varphi(t)$,$t \in [t_0, +\infty)$。

\begin{definition}{Liapunov 稳定}
若 $\forall \varepsilon > 0$, $\exists \delta > 0$,使得对任意满足 
\begin{equation}
\Vert \bvec x_0 - \bvec \varphi(t_0)\Vert < \delta ~~
\end{equation}
的 $\bvec x_0$,方程组以 $(t_0, \bvec x_0)$ 为初值的解 $\bvec x= \bvec x(t, t_0, \bvec x_0)$ 在区间 $[t_0, +\infty)$ 上有定义,且 $\forall t \ge t_0$,
\begin{equation}
\Vert  \bvec x(t, t_0, \bvec x_0) - \bvec \varphi(t) \Vert < \varepsilon ~,
\end{equation}
就称解 $\bvec \varphi(t)$ 是(在 Liapunov 意义下)\textbf{稳定的},否则称为(在 Liapunov 意义下)\textbf{不稳定的}。

\end{definition}

\begin{definition}{渐进稳定}
假设 $\bvec x = \bvec \varphi(t)$ 是稳定的,且存在 $\delta_1$($0 < \delta_1 < \delta$)使得只要
\begin{equation}
\Vert \bvec x_0 - \bvec \varphi(t_0) \Vert < \delta_1 ~~
\end{equation}
就有
\begin{equation}
\lim_{t \to +\infty} \Vert  \bvec x(t, t_0, \bvec x_0) - \bvec \varphi(t) \Vert = 0 ~~
\end{equation}
则称解 $\bvec x = \bvec \varphi(t)$ 是(在 Liapunov 意义下)\textbf{渐近稳定的}。

\end{definition}

\subsection{零解的稳定性}
实际研究过程中,对于任意一个解 $\bvec x = \bvec \varphi(t)$ 的研究,可以通过代换 $\bvec y = \bvec x - \bvec \varphi(t)$ 转化为研究方程 
\begin{equation}
\dv{t} \bvec y(t) = \bvec F(t, \bvec y + \bvec \varphi(t)) ~~
\end{equation}
的零解的稳定性。因为 $\bvec x = \bvec \varphi(t)$ 就对应这方程的 $\bvec y= 0$。

