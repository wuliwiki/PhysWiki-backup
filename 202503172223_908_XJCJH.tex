% 新基础集合论(综述)
% license CCBYSA3
% type Wiki

本文根据 CC-BY-SA 协议转载翻译自维基百科\href{https://en.wikipedia.org/wiki/New_Foundations}{相关文章}。

在数学逻辑中,新基础(New Foundations,简称 NF)是一种非良基、可有限公理化的集合论,由威拉德·范·奥曼·奎因构思,旨在简化《数学原理》中的类型论。
\subsection{定义}
NF 的良构公式是命题演算的标准公式,具有两个基本谓词:相等(\(=\))和成员关系(\(\in\))。NF 可以仅通过两个公理模式来表述:

\begin{itemize}
\item 外延性:具有相同元素的两个对象是相同的对象。形式化地说,给定任意集合 \( A \) 和任意集合 \( B \),如果对于任意集合 \( X \),\( X \) 是 \( A \) 的成员当且仅当 \( X \) 是 \( B \) 的成员,则 \( A \) 等于 \( B \)。
\item 受限的理解公理模式:对于每个分层公式\( \phi \),集合 \( \{x \mid \phi\} \) 存在。
\end{itemize}
一个公式 \( \phi \) 被称为分层的,如果存在一个从 \( \phi \) 的语法结构的各部分到自然数的函数 \( f \),使得:对于 \( \phi \) 中的任意原子子公式 \( x \in y \),满足 \( f(y) = f(x) + 1 \);对于 \( \phi \) 中的任意原子子公式 \( x = y \),满足 \( f(x) = f(y) \)。
\subsubsection{有限公理化}
NF 可以被有限公理化。[1] 这种有限公理化的一个优点是,它消除了分层性的概念。有限公理化中的公理对应于一些自然的基本构造,而分层理解公理虽然强大,但不一定直观。在其入门书籍中,Holmes 选择将有限公理化作为基本框架,并将分层理解公理作为一个定理来证明。[2] 

具体的公理集合可能有所不同,但通常包含以下大部分公理,而其余的可以作为定理证明:[3][1]
\begin{itemize}
\item 外延性:如果 \( A \) 和 \( B \) 是集合,并且对于每个对象 \( x \),\( x \) 是 \( A \) 的元素当且仅当 \( x \) 是 \( B \) 的元素,则 \( A = B \)。[4] 这一公理也可以视为对相等符号的定义。[5][6]
\item 单元素集:对于每个对象 \( x \),集合 \( \iota(x) = \{x\} = \{y \mid y = x\} \) 存在,并称为 \( x \) 的单元素集。[7][8]
\item 笛卡尔积:对于任意集合 \( A \) 和 \( B \),集合\(A \times B = \{(a, b) \mid a \in A \text{ 且 } b \in B\}\)称为 \( A \) 和 \( B \) 的笛卡尔积,并且它的存在性被保证。[9] 该公理可以限制为某个特定的交叉积,例如 \( A \times V \) 或 \( V \times B \) 的存在。[10][11]
\item 逆关系:对于每个关系 \( R \),集合\(R^{-1} = \{(x, y) \mid (y, x) \in R\}\)存在;可以观察到,\( x R^{-1} y \) 当且仅当 \( y R x \)。[12][13][14]
\item 单元素映像:对于任意关系\( R \),集合\(R\iota = \{(\{x\}, \{y\}) \mid (x, y) \in R\}\)存在,并称为\( R \)的单元素映像。[15][16][17]
\item 定义域:如果 \( R \) 是一个关系,则集合\(\text{dom}(R) = \{x \mid \exists y . (x, y) \in R\}\)存在,并称为 \( R \) 的定义域。[12] 这一公理可以通过类型降维操作来定义。[18]
\item 包含关系:集合\([\subseteq] = \{(x, y) \mid x \subseteq y\}\)存在。[19] 等价地,我们可以考虑集合\([\in] = [\subseteq] \cap (1 \times V) = \{(\{x\}, y) \mid x \in y\}\)的存在性。[20][21]
\item 补集:对于每个集合 \( A \),其补集\(A^c = \{x \mid x \notin A\}\)存在。[22]
\item (布尔)并集:如果\(A\)和\(B\)是集合,则它们的并集\(A\cup B=\{x \mid x \in A\text{ 或 }x \in B\text{或两者皆是}\}\)存在。[23]
\item 全集:全集\(V = \{x \mid x = x\}\)存在。显然,对于任何集合\( \),都有\(x \cup x^c = V\)其中\( x^c \)表示\( x \)的补集。[22]
\item 有序对:对于任意对象 \( a \) 和 \( b \),有序对\((a, b)\)存在,并且\((a, b) = (c, d)\iff a = c \text{ 且 } b = d\)这种定义也可以推广到更大的元组。如果使用某种有序对的构造方法,则它可以被定义,而不作为一个独立的公理。[24]
\item 投影:集合\(\pi_1 = \{((x,y), x) \mid x, y \in V\}\)和\(\pi_2 = \{((x,y), y) \mid x, y \in V\}\)存在。这些集合对应于有序对的第一和第二分量的投影关系。[25]
\item 对角线关系:集合\([=] = \{(x, x) \mid x \in V\}\)存在,并被称为相等关系。[25]
\item 集合并:如果\( A \)是一个集合,并且\( A \)的所有元素都是集合,则集合并\(\bigcup [A] = \{x \mid \exists B, x \in B \text{ 且 } B \in A\}\)存在。[26]
\item 相对积:如果\( R \) 和 \( S \)是关系,则相对积\((R | S) = \{(x, y) \mid \exists z, x R z \text{ 且 } z S y\}\)存在。[12]
\item 反交:集合\(x | y = \{z \mid \neg (z \in x \land z \in y)\}\)存在。这一运算等价于补集和并集的组合,其中\(x^c = x | x\)和\(x\cup y =x^c | y^c\)[27]
\item 基数为 1 的集合:所有单元素集的集合\(1 = \{x \mid \exists y : (\forall w, w \in x \leftrightarrow w = y)\}\)存在。[28][29]
\item 元组插入:对于一个关系 \( R \),集合\(I_2(R) = \{(z, w, t) \mid (z, t) \in R\}\)和\(I_3(R) = \{(z, w, t) \mid (z, w) \in R\}\)存在。[30][31]
\item 类型降维:对于任何集合 \( S \),集合\(TL(S) = \{z \mid \forall w, (w, \{z\}) \in S\}\)存在。[32][33]
\end{itemize}
\subsubsection{类型化集合论}
新基础集合论与拉塞尔型的非分层类型化集合论密切相关。TST 是《数学原理》中类型论的简化版本,采用线性类型层次结构。在这种多排序理论(many-sorted theory)中,每个变量和集合都被赋予一个类型(type)。通常,类型指数使用上标表示:\(x^n\)表示类型为\( n \)的变量。类型\textbf{0}由未进一步描述的个体组成。对于每个(元)自然数\( n \),类型\( n+1 \)的对象是类型\( n \)对象的集合。相等关系仅适用于相同类型的对象。类型\( n \)的集合仅包含类型\( n-1 \)的成员。TST 的公理包括:外延性:适用于相同(正)类型的集合。理解公理:如果\( \phi(x^n) \)是一个公式,则集合\(\{x^n \mid \phi(x^n)\}^{n+1}\)存在,即:\(\exists A^{n+1}\forall x^n [x^n \in A^{n+1} \leftrightarrow \phi(x^n)]\)是一个公理,其中\( A^{n+1} \)代表集合\(\{x^n \mid \phi(x^n)\}^{n+1}\),并且\( A^{n+1} \)在\( \phi(x^n) \)中不是自由变量。这种类型论比《数学原理》中最初提出的类型论要简单得多,后者包括了关系的类型,而这些关系的参数不一定具有相同的类型。

NF 和 TST 之间存在一种类型标注的添加或删除的对应关系:在 NF 的理解模式中,公式是“分层的”当且仅当该公式可以按照 TST 的规则赋予类型。这意味着 NF 公式可以映射到 TST 公式的集合,其中每个 TST 公式都带有不同的类型索引。这种映射是一对多的,因为 TST 允许多个相似的公式。例如,在 TST 公式中,将所有类型索引提升 1,仍然会得到一个新的、有效的 TST 公式。
\subsubsection{纠缠类型论}
纠缠类型论(TTT)是TST的扩展,其中每个变量的类型由\textbf{序数}而非\textbf{自然数}标注。其良构的原子公式包括:相等关系:\( x^n = y^n \)成员关系:\( x^m \in y^n \)(其中 \( m < n \))TTT 的公理与TST的公理相同,但其中每个类型为\( i \)的变量都会被映射到一个变量\( s(i) \),其中 \( s \)是一个递增函数。

TTT 被认为是一种“怪异的”理论,因为它的每个类型都以相同的方式与所有较低类型相关。例如:类型 2 的集合既可以包含类型 1 的成员,也可以包含类型 0 的成员。\textbf{外延性公理}声明,类型 2 的集合仅由其类型 1 成员或类型 0 成员唯一确定。与 TST 不同:在 TST 中,自然模型满足每个类型\( i+1 \)都是类型\( i \)的\textbf{幂集}。在 TTT 中,每个类型同时被解释为所有较低类型的幂集。尽管如此:NF 的模型可以很容易地转换为 TTT 的模型,因为在NF 中,所有类型本质上都是相同的。反过来,经过更复杂的论证,可以证明 TTT 的一致性(Consistency of TTT)能够推出 NF 的一致性。[34]
\subsubsection{NFU 及其他变种}
\textbf{带有基数元素}的新基础集合论是 NF 的一个重要变种,由Jensen提出,[35]并由Holmes进一步澄清。[3]基数元素是不是集合的对象:它们不包含任何元素,但可以被包含在集合中。在 NFU 最简单的公理化形式之一中,基数元素被视为多个不同的\textbf{空集},从而削弱了 NF 的外延性公理,改为:
\begin{itemize}
\item 弱外延性(Weak Extensionality):两个非空对象如果具有相同的元素,则它们是相同的对象。形式化地:
\[
\forall x y w . (w \in x) \to (x = y \leftrightarrow (\forall z . z \in x \leftrightarrow z \in y))~
\]
\end{itemize}
这意味着:对于非空集合,外延性仍然成立。对于空集和基数元素,外延性不适用,它们可以是多个不同的对象。在这种公理化下:\textbf{理解模式}保持不变。但是,如果某个集合的定义公式\(\phi(x)\)无解(即不可满足),则集合\(\{x \mid \phi(x)\}\)可能不是唯一的(即它可以对应多个不同的基数元素)。

然而,为了方便使用,通常更希望有一个唯一的、“标准”空集。这可以通过引入集合性谓词\( \text{set}(x)\)来区分集合和基数元素(原子)。  
其公理如下:[36]
\begin{itemize}
\item 集合公理(Sets):只有集合才有成员,即:\(\forall x y . x \in y \to \text{set}(y)\)解释:如果 \( x \) 是 \( y \) 的成员,则 \( y \) 必须是一个集合。这意味着基数元素不能包含任何成员,它们只能\textbf{被包含}在集合中。
\item 外延性公理(Extensionality):两个具有相同元素的集合是相同的集合,即:\(\forall y z . (\text{set}(y) \land \text{set}(z) \land (\forall x . x \in y \leftrightarrow x \in z)) \to y = z.\)解释:只有当\( y \)和\( z \)都是集合时,才适用外延性。基数元素(原子)不受外延性约束,它们可以是多个不同的对象,即使它们没有元素。
\item 理解公理(Comprehension):对于每个分层公式\(\phi(x)\),集合\(\{x \mid \phi(x)\}\)存在,即:\(\exists A . \text{set}(A) \land (\forall x . x \in A \leftrightarrow \phi(x))\)解释:分层公式(Stratified formula):满足 NF 逻辑规则的公式。集合\(A\)的存在性:只要\(\phi(x)\)符合 NF 的分层性规则,该集合就存在。
\end{itemize}
\(NF_3\):NF 的一个子理论,仅允许最多三种类型的分层公式(即\(x, y, z\)这样的最多三层类型)。\(NF_4\):与完整的 NF 等价,即没有基数元素,并且允许所有满足分层性条件的理解公理。

数学逻辑(Mathematical Logic, ML)是 NF 的扩展,它不仅包含集合,还包含\textbf{适当的类}。ML 由Quine提出,并由Hao Wang修订。Wang 证明,NF 与修订后的 ML 具有等一致性,即它们的一致性(无矛盾性)是等价的。
\subsection{构造} 
本节讨论NF中的一些具有问题的构造。关于NFU中数学的进一步发展,以及与ZFC体系中相应发展的比较,请参见集合论中数学的实现。
\subsubsection{有序对}  
在TST(以及NF和NFU)中,关系和函数通常被定义为有序对的集合。从分层性的角度来看,理想的情况是:一个关系或函数的类型应仅比其定义域中的元素的类型高一层。这要求定义有序对的方式,使其类型与其参数相同(即类型级有序对,\textbf{type-level} ordered pair)。然而,常见的有序对定义:\((a, b)_K := \{\{a\},\{a, b\}\}\)会导致\((a, b)\)的类型比参数\(a\)和\(b\)的类型高两层。因此,在分层性分析中,一个函数的类型比其定义域中的元素高三层。在 NF 及其相关理论中,通常采用 Quine 的集合论定义的有序对,该定义能够保持有序对的类型与其参数相同,从而避免类型层次的额外提升。然而,Quine 的定义依赖于对每个元素\(a\)和\(b\)进行集合运算,因此在 NFU 中无法直接适用。

作为一种\textbf{替代方法},Holmes[3] 直接将有序对 \( (a, b) \) 作为原始概念,并同时引入其左投影和右投影:左投影:\(\pi_1((a, b)) = a\)右投影:\(\pi_2((a, b)) = b\)在Holmes 对 NFU 的公理化中:\textbf{理解模式}——即对于任意分层公式\(\phi\),集合\(\{x \mid \phi(x)\}\)的存在性——并非直接作为公理,而是后续证明的定理。

因此,类似于\(\pi_1 = \{((a, b), a) \mid a, b \in V\}\)
这样的表达式不能被视为正式的定义。幸运的是:无论有序对的类型层次(type-level)是通过定义(如 Quine 的方法)还是通过公理假设(即作为原始概念)来保证,通常都不会影响理论的一致性。
\subsubsection{自然数与无穷公理} 
通常形式的\textbf{无穷公理}基于冯·诺伊曼对自然数的构造。然而,该构造不适用于 NF,因为\textbf{后继运算}的定义(以及冯·诺伊曼数的许多其他方面)无法被分层化。 在 NF 中,通常使用弗雷格的自然数定义,即:自然数\( n \)由所有恰好有\(n\)个元素的集合的集合表示。在这种定义下:\textbf{零(0)}可以简单地定义为:\(\{\varnothing \}\)即仅包含空集的集合。\textbf{后继运算}可以以分层化的方式定义:\(S(A) = \{ a \cup \{x\} \mid a \in A \land x \notin a \}\)这里:\(A\)是某个自然数\( n\)对应的集合,\(x\)是一个新元素,不属于 \(a\)。在此定义下,可以写出类似于通常无穷公理的表述。然而,该表述在 NF 中将是平凡成立的,因为全集\(V\)本身就是一个\textbf{归纳集},满足所有自然数构造规则。

由于\textbf{归纳集}总是存在,自然数集合\(\mathbf{N}\)可以定义为所有归纳集的交集。这一定义使得对分层语句\( P(n) \)进行数学归纳成为可能,因为:集合\(\{n \in \mathbf{N} \mid P(n)\}\)可以被构造。当\(P(n)\)满足数学归纳的条件时,该集合本身就是一个归纳集,从而满足数学归纳法的要求。

\textbf{有限集合}可以定义为属于某个自然数的集合。然而,证明全集\(V\)不是“有限集”(即\(|V|\)不是一个自然数)并不显然。假设:\(|V| = n \in \mathbf{N}\)
则:\(n = \{V\}\)(可以通过归纳证明,有限集合不与其任何真子集等势)。由此推导:\(n + 1 = S(n) = \varnothing\)并且所有后续的自然数也都会是空集,从而导致算术体系崩溃。为避免这一问题,可以在 NF 中\textbf{引入无穷公理}:\(\varnothing \notin \mathbf{N}\)这确保\(V\)的基数不会是有限数,从而维持算术的正确性。[37]

从直觉上看,似乎应该可以通过构造某种“外部”无限序列(externally infinite sequence)的集合来在 NF(U) 中证明无穷公理,例如:\(\varnothing, \{\varnothing\}, \{\{\varnothing\}\}, \dots\)然而,这样的序列只能通过非分层化的方法构造(这一点可以通过 TST 本身存在有限模型来佐证),因此无法在 NF(U) 中进行这样的证明。实际上,\textbf{无穷公理}在 NFU 中是逻辑上独立的:存在 NFU 的模型,其中\( |V| \)是非标准自然数。在这些模型中,数学归纳法可以应用于\( |V| \),使得它无法与标准的自然数区分开来。

然而,在某些情况下,可以在 NF 及其变种中证明无穷公理,在这些情况下,它可能被称为\textbf{无穷定理}:
\begin{itemize}
\item 在 NF(无基数元素的版本)中:Specker[38] 证明了选择公理(Axiom of Choice)在 NF 中是错误的。由于可以通过数学归纳法证明每个有限集合都具有一个选择函数(这是一个分层化的条件),因此全集\(V\)必然是无限的。
\item 在 NFU 中,如果引入公理断言类型级有序对(type-level ordered pair)的存在:全集 \(V\)与其真子集\( V \times \{0\} \)等势,这蕴含无穷性。[37]反过来,NFU + Infinity + Choice 可以证明类型级有序对的存在(但此结果尚需引用支持)。NFU + Infinity可以解释(interpret)NFU + “存在类型级有序对”(虽然这两个理论不完全相同,但它们的差异并不重要)。
\end{itemize}
更强的无穷公理:可以假设自然数集合是一个\textbf{强康托尔集合}。NFUM(即NFU + Infinity + Large Ordinals + Small Ordinals)等价于Morse–Kelley 集合论(Morse–Kelley Set Theory, MK),再加上一个对适当类的谓词,该谓词是一个κ-完备的非主超滤,定义在适当类的序数\(\kappa\)上。[39]
\subsubsection{大集合}
NF(以及 NFU + Infinity + Choice,它已知是一致的)允许构造两类在 ZFC 及其扩展中被禁止的集合,因为它们“过大”(在某些集合论中,这些对象被视为适当类)。
\begin{itemize}
\item 全集\( V \)由于\( x = x \)是一个分层化公式,根据理解公理,全集\(V = \{x \mid x = x\}\)存在。直接的一个推论是:所有集合都有补集。在 NF 中,整个集合论宇宙具有\textbf{布尔代数结构}。
\item 基数与序数在NF(以及 TST)中,所有具有 \( n \) 个元素的集合的集合是存在的(这种“循环”只是表面上的)。因此,弗雷格(Frege)的基数定义在 NF 和 NFU 中是有效的:基数定义为\textbf{等势关系}下的集合的等价类。如果存在双射\( f: A \to B\),则集合\( A \)和\( B \)等势,记作:\(A \sim B\)序数的定义类似,它是良序集的等价类。
\end{itemize}
\subsubsection{笛卡尔封闭性} 
在 NF 中,以集合作为对象、以这些集合之间的函数作为箭头构成的范畴不是笛卡尔封闭的。[40]由于 NF 不满足笛卡尔封闭性:并非所有函数都可以柯里化,这与直觉上的期望不同。NF 不是一个拓扑斯。
\subsection{集合论悖论的解决} 
NF 乍看之下似乎会遭遇与\textbf{朴素集合论}类似的问题,但事实并非如此。例如:Russell 集合\(\{x \mid x \notin x\}\)在 NF 中并不是一个公理保证存在的集合,因为公式\( x \notin x \)不能被分层化。NF 通过完全不同于 ZFC 等良基集合论方法,避免了集合论中的三大悖论。此外,NF 及其变种还基于这些悖论的解决方案发展出许多独特且有用的概念。
\subsubsection{罗素悖论}
NF 对罗素悖论的解决方案是直接且显然的:\( x \notin x \)不是一个分层化公式,因此,\textbf{理解公理}并不会断言\(\{x \mid x \notin x\}\)这一集合的存在。Quine 曾表示,他构造 NF 时,首要考虑的就是如何避免罗素悖论。[41]
\subsubsection{康托尔悖论与康托尔集合}
康托尔悖论的核心问题是:是否存在最大的基数,或者等价地,是否存在一个基数最大的集合。在 NF 中:全集\( V \)显然是一个基数最大的集合,因为它包含所有集合。然而,在 ZFC 中,康托尔定理表明:对于任何集合\( A \),其幂集 \( P(A)\)的基数严格大于\( A \):\(|P(A)| > |A|\)这意味着,若取 \( A = V \),则 \( P(V) \) 的基数应该比\( V \)更大,但\( V \)已经是最大集合,这似乎导致矛盾。

当然,由于\( V \)是全集,从\( P(V) \)到\( V \)存在一个单射,因此康托尔定理在其原始形式下在 NF 中不成立。实际上,康托尔定理的证明依赖于对角化论证,即构造集合:\(B = \{ x \in A \mid x \notin f(x) \}\)在 NF 中:\( x \)和\(f(x)\)必须被赋予相同的类型,因此\( B \)的定义无法被分层化,导致该集合无法通过理解公理被构造。更进一步,如果我们选择平凡单射:\(f: P(V) \to V, \quad x \mapsto x\)那么\(B\)将与罗素悖论中的非法集合完全相同,因此无法存在。由此可见,NF 通过分层性规则自然地避免了康托尔悖论。

这种结论的失败并不令人惊讶,因为在 TST 中,\(|A| < |P(A)|\)没有意义:幂集\(P(A)\)的类型比\(A\)高一层,因此它们的基数不具可比性。在 NF 中,由于所有类型被合并,\(|A| < |P(A)|\)成为了一个语法上有效的陈述,但任何涉及理解公理的一般性证明都不太可能成立。

通常修正这种类型问题的方法是:用\( P_1(A) \)代替\( A \),其中:\(P_1(A)\)
表示\( A \)的所有单元素子集的集合(the set of one-element subsets of \( A \))。实际上,康托尔定理的正确类型版本:\(|P_1(A)| < |P(A)|\)在 TST 中是一个定理(依赖于对角化论证),因此它在 NF 中同样成立。特别地,在 NF 中:\(|P_1(V)| < |P(V)|\)即:单元素集合的数量少于一般集合的数量。在 NFU 中,这意味着:
单元素集合的数量少于一般对象的数量。单射\( x \mapsto \{x\} \)在 NFU 中不是集合从全集\( V \)到单元素集合\(P_1(V)\)的“显然的”双射:\(x \mapsto \{x\}\)并不是一个集合,因为:该映射的定义不满足分层化,因此它无法通过 NFU 的理解公理构造。NFU + 选择公理的所有模型中:\(|P_1(V)| < |P(V)| \ll |V|\)即:\( P_1(V) \) 的基数小于 \( P(V) \),且\( P(V) \)的基数远远小于全集\( V \)。选择公理不仅允许证明基数元素的存在,还可以证明\(P(V)\)和\( V \)之间存在多个不同的基数。

然而,不同于 TST,\( |A| = |P_1(A)| \) 在 NF(U) 中是一个句法上的陈述,并且如上所示,可以讨论其对特定值 \( A \) 的真值(例如,当 \( A = V \) 时,它是假的)。满足直观上令人信服的条件 \( |A| = |P_1(A)| \) 的\textbf{集合} \( A \) 被称为康托尔集合:康托尔集合满足通常形式的康托尔定理。满足进一步条件 \((x \mapsto \{x\}) \lceil A\)(即单元素映射在 \( A \) 上的限制仍是一个集合)的集合 \( A \) 不仅是康托尔集合,而且是\textbf{强康托尔集合}。
\subsubsection{Burali-Forti 悖论与 T 运算 } 
关于最大序数的\textbf{Burali-Forti 悖论}在 NF(New Foundations)中以相反的方式得到解决:在 NF 中,即使可以访问序数的集合,也无法构造出“最大序数”。人们可以构造出对应于所有序数的自然良序的序数\( \Omega \),但这并不意味着\(\Omega \)比所有这些序数都大。

要在 NF 中形式化 Burali-Forti 悖论,首先需要对序数的概念进行形式化。在NF中,序数的定义方式与朴素集合论相同,即它们被定义为良序集在同构关系下的等价类。这是一个分层的定义,因此可以无问题地定义序数的集合\( \mathrm{Ord} \)。在 NF 中,超限归纳适用于分层陈述,这使得可以证明序数的自然序关系是\( \mathrm{Ord} \)上的一个良序。具体来说:\(\alpha \leq \beta \iff \text{ 存在良序关系 } R \in \alpha, S \in \beta \text{,使得 } S \text{ 是 } R \text{的延续}\)根据序数的定义,这个良序关系本身也属于某个序数\(\Omega \in \mathrm{Ord}\)。在朴素集合论中,通常可以通过超限归纳证明:每个序数 \( \alpha \) 的序型正好是所有小于\( \alpha \)的序数的自然序关系的序型。但如果在 NF 中应用这一点,就会导致矛盾:因为按照定义,\( \Omega \)是所有序数的序型,而不是它们的某个真初始片段的序型。这就导致了 Burali-Forti 悖论。

然而,陈述“\( \alpha \) 是所有小于 \( \alpha \) 的序数的自然序关系的序型”并不是分层的,因此超限归纳在 NF 中无法使用。实际上,“小于 \( \alpha \) 的序数的自然序关系 \( R_{\alpha} \) 的序型 \( \beta \)”的类型至少比 \( \alpha \)高出两层:关系 \( R_{\alpha} = \{(x, y) \mid x \leq y < \alpha\} \)的类型比\( \alpha \) 高一层(假设有序对 \( (x, y) \) 是类型层次上的有序对)。
- 序型(即等价类)\( \beta = \{S \mid S \sim R_{\alpha}\} \) 的类型比 \( R_{\alpha} \) 再高一层。如果采用的是库拉托夫斯基有序对(即 \( (x, y) = \{\{x\}, \{x, y\}\} \)),那么 \( (x, y) \) 的类型会比 \( x \) 和 \( y \)高两层,因此 \( \beta \) 的类型相对于 \( \alpha \) 总共会高出四层。

为了解决这种\textbf{类型问题},需要使用 \textbf{T 运算},即\( T(\alpha)\),它的作用类似于\( P_1(A) \)“提升”集合\( A \)的类型,而\( T(\alpha)\)“提升”序数 \( \alpha\)的类型。T 运算的定义如下:如果\( W \in \alpha \),那么\( T(\alpha)\)是以下序关系\(W^{\iota}\)的序型:\(W^{\iota} = \{(\{x\}, \{y\}) \mid x W y\}\)现在,可以以分层的方式重新表述关于序型的引理。

所有\(<\alpha \)的序数在自然序关系下的序型是\( T^2(\alpha) \)或\( T^4(\alpha) \),具体取决于所使用的有序对的定义。这两个版本的陈述都可以通过超限归纳来证明;在下文中,我们假设使用类型层次上的有序对。这意味着:\(T^2(\alpha) < \Omega\)其中,\( \Omega \) 是所有序数的序型。特别地,有:\(T^2(\Omega) < \Omega\)

另一个可以通过超限归纳证明的分层陈述是,\( T \) 运算在序数上是\textbf{严格单调}的,即它是\textbf{保持序关系}的运算:\(T(\alpha) < T(\beta) \iff \alpha < \beta\)因此,T 运算不是一个函数:序数的集合\(\{\alpha \mid T(\alpha) < \alpha\}\)不可能有最小元素,因此它不能构成一个集合。更具体地说,T 运算的单调性意味着存在如下的“递降序列”:\(\Omega > T^2(\Omega) > T^4(\Omega) > \dots\)在序数中,这样的序列不能构成一个集合。

有人可能会认为,这一结果表明 NF(U) 的任何模型都不是“标准”,因为在任何 NFU 的模型中,外部看来\textbf{序数并非良序}。然而,这是一个哲学性问题,而不是形式化理论内部可以证明的问题。需要注意的是,即使在 NFU 内部,也可以证明任何 NFU 的集合模型都包含非良序的“序数”。但 NFU 并不得出结论,认为整个宇宙 \( V \)是 NFU 的一个模型,尽管 \( V \) 是一个集合,因为归属关系并不是一个集合关系。
\subsection{一致性} 
一些数学家曾质疑 NF 的一致性,部分原因是它为何能够避免已知的悖论并不明显。其中的一个关键问题是,Specker 证明了 NF 与选择公理结合时是不一致的。该证明涉及 \textbf{T 运算},相当复杂。然而,自 2010 年以来,Holmes 声称已经证明了 NF 的一致性,相对于标准集合论(ZFC)的一致性。2024 年,Sky Wilshaw 使用 Lean 证明助理验证了 Holmes 的证明。[43]  

尽管NFU解决悖论的方式与 NF 类似,但它具有更简单的一致性证明。该证明可以在Peano 算术(PA)内形式化,而 PA 是比 ZF 集合论)更弱的理论,大多数数学家普遍接受它而不质疑。这并不与 Gödel 的第二不完备定理相冲突,因为 NFU 不包含无穷公理,因此 PA 不能在 NFU 中建模,从而避免了矛盾。PA 还证明了:NFU + 无穷公理和 TST + 无穷公理具有相同的一致性。NFU + 无穷公理 + 选择公理和TST + 无穷公理 + 选择公理也具有相同的一致性。因此,像 ZFC 这样的更强理论能够证明 TST 的一致性,也能证NFU + 这些扩展公理的一致性。[35]简单来说,NFU 通常被认为比 NF 更弱,因为在 NFU 中,所有集合的集合(即整个宇宙的幂集)可以比宇宙本身更小,尤其是当 NFU 允许存在无元素个体时,这一点在 NFU + 选择公理 下是必要的。
\subsubsection{NFU 的模型} 
Jensen 的证明提供了一种相对简单的方法,可以大规模地构造 NFU 的模型。利用模型理论中的常见技术,可以构造 Zermelo 集合论的非标准模型(这种方法不需要完整的 ZFC,仅需比 Zermelo 集合论稍强的理论)。在这个模型中,存在一个外部自同构\( j \)(它不是该模型的一个集合),该自同构会移动累积层级中的某个\textbf{秩}\( V_{\alpha} \)[^1^]。我们可以不失一般性地假设:\(j(\alpha) < \alpha\)

NFU 模型的域(domain)将是\textbf{非标准秩}\( V_{\alpha} \)。其基本思想是,自同构\( j \)将“幂集” \( V_{\alpha+1} \) 编码到其在“宇宙” \( V_{\alpha} \)内的外部同构副本\( V_{j(\alpha)+1} \) 中。这样,剩下的那些不对应于宇宙子集的对象会被视为无元素个体。形式化地,NFU 模型的\textbf{归属关系}定义如下:\(x \in_{NFU} y \equiv_{def} j(x) \in y \wedge y \in V_{j(\alpha) + 1}\)

现在可以证明,该构造确实给出了 NFU 的一个模型。设 \( \phi \) 是 NFU 语言中的一个分层公式。选择一个类型分配,为公式中的所有变量赋予类型,使其满足分层性。然后选择一个自然数\( N \),使其大于该分层中分配给所有变量的类型。

步骤 1:将 NFU 公式翻译到 Zermelo 集合论的非标准模型,使用 NFU 模型中归属关系的定义,将公式\( \phi \)展开为 Zermelo 集合论非标准模型(带有自同构 \( j \)中的公式\( \phi_1 \)。由于 \( j \) 是自同构,对等式或归属语句的两边同时应用任何幂次的 \( j \),都不会改变其真值。  

步骤 2:调整公式,使所有变量的类型一致,对公式 \( \phi_1 \) 的每个原子公式应用 \( j \),使得每个分配类型 \( i \) 的变量 \( x \) 被应用 \( j^{N-i} \)。这样,每个变量的类型都会对齐到相同的层次。由于所有 NFU 归属语句转化出的\textbf{原子归属语句}的形式,以及公式本身是分层的,这种变换是可行的。此时,每个全称量化语句\(\forall x \in V_{\alpha}.\psi(j^{N-i}(x))\)可以转换为\(\forall x \in j^{N-i}(V_{\alpha}).\psi(x)\)存在量化的情况也类似。  

步骤 3:消去公式中的 \( j \) 应用执行上述变换,使得在公式\( \phi_2 \)中,\( j \)不再应用于任何\textbf{约束变量}。选择\( \phi \)中一个被分配类型\( i \)的自由变量 \( y \),对整个公式统一应用\( j^{i-N} \),得到新的公式\( \phi_3 \),其中 \( y \)不再带有任何\( j \) 的应用。  

步骤 4:构造 NFU 模型中的集合,集合\(\{ y \in V_{\alpha} \mid \phi_3 \}\)在外部模型中存在(因为 \( j \) 只作用于自由变量和常量),并且属于 \(V_{\alpha+1} \)。它包含了所有在 NFU 模型中满足原始公式 \( \phi \) 的 \( y \)。由于 NFU 模型的归属关系不同,\( j \) 在 NFU 模型中的\textbf{应用方式}会对其进行修正,即:\(j(\{ y \in V_{\alpha} \mid \phi_3 \})\)在 NFU 模型中具有正确的外延。  

结论:分层理解公理成立.通过上述构造,可以得出NFU 模型中满足分层理解公理,从而证明该模型确实是 NFU 的一个有效模型。

要验证\textbf{弱外延性}成立是直接的:每个非空元素\( V_{j(\alpha)+1} \) 都从非标准模型继承了一个唯一的外延。\textbf{空集}也继承了它通常的外延。所有其他对象被视为无元素个体。  

如果\( \alpha \)是一个自然数\( n \),那么所得的 NFU 模型 声称整个宇宙是有限的(当然,从外部来看它仍然是无限的)。如果 \( \alpha \) 是无限的,且非标准 ZFC 模型中满足选择公理,那么构造出的 NFU 模型将满足:\(\text{NFU} + \text{Infinity} + \text{Choice}\).
\subsubsection{NFU 中数学基础的自给自足性}
出于哲学原因,需要注意的是,要完成这一证明,并不需要在 ZFC 或任何相关系统中进行。反对将 NFU 作为数学基础的常见论点之一是,依赖它的理由通常与直觉上认为 ZFC 是正确的有关。然而,接受 TST(实际上是 TSTU)就已经足够了。概述如下:取类型论 TSTU(允许每个正类型中包含无元素集)作为元理论,并在 TSTU 中考虑 TSTU 的集合模型理论(这些模型将是集合序列 \(T_i\)(在元理论中它们全部属于相同类型),其中包含从 \(P(T_i)\) 到 \(P_1(T_{i+1})\) 的嵌入,这些嵌入编码了幂集 \(P(T_i)\) 到 \(T_{i+1}\) 的类型保持映射)。在给定 \(T_0\) 到 \(T_1\) 的嵌入(将基本“类型”的元素识别为该基本类型的子集)的情况下,可以以自然的方式从每个“类型”构造到其后继的嵌入。这可以谨慎地推广到超限序列 \(T_\alpha\)。

请注意,这类集合序列的构造受限于它们所在类型的大小;这一限制使得 TSTU 无法证明自身的一致性(TSTU + 无限公理可以证明 TSTU 的一致性;但要证明 TSTU+无限公理的一致性,则需要一个包含基数 \( \beth_{\omega} \) 的集合的类型,而在不作更强假设的情况下,TSTU+无限公理无法证明这样的类型的存在)。现在,我们可以使用相同的模型理论结果来构造 NFU 的一个模型,并以类似的方式验证它是 NFU 的一个模型,其中 \( T_{\alpha} \) 取代了通常构造中的 \( V_{\alpha} \)。最终的关键步骤是,既然 NFU 是一致的,我们就可以在元理论中放弃使用绝对类型,从而通过自举的方式,将元理论从 TSTU 迁移到 NFU。
\subsubsection{关于自同构 \( j \) 的事实}
这种模型的自同构 \( j \) 与 NFU 中的某些自然运算密切相关。例如,如果 \( W \) 是非标准模型中的一个良序(我们假设使用库拉托夫斯基对,使得两个理论中的函数编码在某种程度上保持一致),并且 \( W \) 在 NFU 中也是一个良序(NFU 中的所有良序在 Zermelo 集合论的非标准模型中也是良序,但反之不成立,因为在模型的构造过程中会形成无元素集),且 \( W \) 在 NFU 中的类型为 \( \alpha \),那么 \( j(W) \) 在 NFU 中将是一个类型为 \( T(\alpha) \) 的良序。  

实际上,\( j \) 由 NFU 模型中的一个函数编码。在非标准模型中,将 \( V_{j(\alpha)} \) 的任何元素的单元素集映射到其唯一元素的函数,在 NFU 中变为一个函数,该函数将每个单元素集 \(\{x\}\)(其中 \( x \) 是宇宙中的任意对象)映射到 \( j(x) \)。称此函数为 Endo,它具有以下性质:Endo 是从单元素集的集合到集合的集合的一个单射,并且满足\(\text{Endo}(\{x\}) = \{\text{Endo}(\{y\}) \mid y \in x\}\)对任意集合 \( x \) 都成立。该函数可以在宇宙上定义一个类型级别的“成员”关系,从而重现原始非标准模型中的成员关系。
\subsection{历史}  
1914 年,诺伯特·维纳展示了如何将有序对编码为集合的集合,使得可以用 TST(类型化集合理论)中的线性集合层次结构来替代《数学原理》中的关系类型。现在通常使用的有序对定义最早由库拉托夫斯基在 1921 年提出。威拉德·范·奥曼·奎因在 1937 年的一篇文章《数学逻辑的新基础》(New 中首次提出 NF(新基础)作为避免 TST 中“令人不悦的后果”的一种方法,因此 NF 得名于此。奎因在其 1940 年出版的著作《数学逻辑》中扩展了这一理论。在书中,奎因引入了“数学逻辑”,简称\textbf{ML})系统,这是 NF 的扩展,包含了适当类和集合。该书第一版中的集合论将 NF 与 NBG(冯·诺伊曼-伯恩赛因-哥德尔)集合论的适当类相结合,并包含了适用于适当类的不受限制的理解公理模式。然而,J. 巴克利·罗斯瑟证明该系统会导致布拉利-福尔蒂悖论。随后,王浩展示了如何修正奎因在 ML 系统中的公理,使其避免该问题。奎因在 1951 年出版的第二版(最终版)中纳入了这一修正后的公理化体系。

1944 年,西奥多·海尔佩林证明了理解公理等价于其有限个实例的合取命题。[1]1953 年,恩斯特·施佩克证明了选择公理在 NF(无无元素集的情况下)中是不成立的。[38]1969 年,詹森(Jensen)证明,向 NF 中添加无元素集会得到一个可证明一致的理论,即 NFU。[35]同年,格里申证明了 NF₃ 的一致性。[46]施佩克还证明了 NF 与 TST 加上“典型歧义”公理模式是等一致的。此外,NF 还等一致于扩展了“类型转换自同构”的 TST,其中该自同构是一个理论外部的操作,它将每个类型映射到更高一级的类型,并保持相等性和成员关系不变。

1983 年,马塞尔·克拉比(Marcel Crabbé)证明了一种他称为 NFI 的系统的一致性。该系统的公理包括\textbf{不受限制的外延性公理},以及那些在理解公理的实例中,没有变量被赋予高于所断言存在的集合的类型。这是一种限定性的限制,尽管 NFI 并不是一个完全的限定性理论:它仍然允许足够的非限定性来定义自然数集(定义为所有\textbf{归纳集}的交集;需要注意的是,被量化的归纳集与所定义的自然数集属于相同的类型)。克拉比还讨论了 NFI 的一个子理论,其中在理解公理的实例中,只有参数(自由变量)可以具有被断言存在的集合的类型。他称该系统为“限定性 NF”(NFP)。当然,如果一个理论允许自成员化的宇宙,那么它是否真正属于限定性理论是值得怀疑的。霍尔姆斯在[日期缺失] 证明,NFP 的一致性强度与《数学原理》(Principia Mathematica)的限定性类型论相同,但不包括可化公理。

Metamath 数据库实现了 Hailperin 对新基础(NF)提出的有限公理化方案。[47]自 2015 年以来,兰德尔·霍尔姆斯(Randall Holmes)提出了多个 NF 一致性相对于 ZF 的候选证明,这些证明可在 arXiv 和他的个人主页上找到。他的证明方法是首先展示一种“奇特”的 TST 变体,即“带 λ-类型的纠缠类型理论”(TTTλ)与 NF 是等一致的,然后再证明 TTTλ 在 ZF 带原子(ZFA)但无选择公理的情况下是一致的,方法是构造 ZFA 的类模型,其中包含在 ZF 带原子与选择公理(ZFA+C)中的“基数的纠缠网”。然而,这些证明“难以阅读、极度复杂,并涉及繁琐的细节记录,使得容易引入错误”。2024 年,Sky Wilshaw 使用 Lean 证明助理形式化了霍尔姆斯的一个证明版本,最终解决了 NF 一致性的问题。[48]蒂莫西·周评价 Wilshaw 的工作,认为这表明同行评审者对难以理解的证明感到迟疑的问题,可以借助证明助理来解决。[49]
\subsection{参见}  
\begin{itemize}
\item 替代集合论
\item 公理化集合论 
\item 数学在集合论中的实现 
\item 正集合论  
\item 自然数的集合论定义
\end{itemize}
\subsection{注释}  
\begin{enumerate}
\item Hailperin 1944.  
\item Holmes 1998,第 8 章。  
\item Holmes 1998。  
\item Holmes 1998,第 16 页。  
\item Hailperin 1944,第 1.02 定义及公理 PId。  
\item 例如,W. V. O. Quine 在 *Mathematical Logic*(1981)中使用了“三种基本符号工具:成员关系、联合否定(joint denial)和量化”,然后按此方式定义了 "="(第 134–136 页)。  
\item Holmes 1998,第 25 页。  
\item Fenton 2015,ax-sn。  
\item Holmes 1998,第 27 页。  
\item Hailperin 1944,第 10 页,公理 P5。  
\item Fenton 2015,ax-xp。  
\item Holmes 1998,第 31 页。  
\item Hailperin 1944,第 10 页,公理 P7。  
\item Fenton 2015,ax-cnv。  
\item Holmes 1998,第 32 页。  
\item Hailperin 1944,第 10 页,公理 P2。  
\item Fenton 2015,ax-si。  
\item Hailperin 1944,第 10 页。  
\item Holmes 1998,第 44 页。  
\item Hailperin 1944,第 10 页,公理 P9。  
\item Fenton 2015,ax-sset。  
\item Holmes 1998,第 19 页。  
\item Holmes 1998,第 20 页。  
\item Holmes 1998,第 26–27 页。  
\item Holmes 1998,第 30 页。  
\item Holmes 1998,第 24 页。  
\item Fenton 2015,ax-nin。  
\item Hailperin 1944,第 10 页,公理 P8。  
\item Fenton 2015,ax-1c。  
\item Hailperin 1944,第 10 页,公理 P3、P4。  
\item Fenton 2015,ax-ins2,ax-ins3。  
\item Hailperin 1944,第 10 页,公理 P6。  
\item Fenton 2015,ax-typlower。  
\item Holmes & Wilshaw 2024。  
\item Jensen 1969。  
\item Holmes 2001。  
\item Holmes 1998,第 12.1 节。  
\item Specker 1953。  
\item Holmes, M. Randall(2001 年 3 月)。"Strong axioms of infinity in NFU" (PDF). Journal of Symbolic Logic. 66 (1): 87–116. doi:10.2307/2694912. JSTOR 2694912。  
\item Forster 2007。  
\item Quine 1987。  
\item Holmes 1998,第 17.5 节。  
\item Smith, Peter(2024 年 4 月 21 日)。"NF really is consistent". Logic Matters. 2024 年 4 月 21 日检索。  
\item Rosser 1942。  
\item Wang 1950。  
\item Grishin 1969。  
\item Fenton 2015。  
\item Smith, Peter(2024 年 4 月 21 日)。"NF really is consistent". Logic Matters. 2024 年 4 月 21 日检索。  
\item Chow, Timothy(2024 年 5 月 3 日)。*"Timothy Chow on the NF consistency proof and Lean".* *Logic Matters.* 2024 年 5 月 3 日检索。
\end{enumerate}
1. 我们讨论的是自同构如何移动秩 \( V_{\alpha} \) ,而不是序数 \( \alpha \),因为我们不想假设模型中的每个序数都是某个秩的索引。
\subsection{参考文献}
\begin{itemize}
\item Crabbé, Marcel (1982). "On the consistency of an impredicative fragment of Quine's NF". *The Journal of Symbolic Logic*. 47 (1): 131–136. doi:10.2307/2273386. JSTOR 2273386. S2CID 42174966.  
\item Fenton, Scott (2015). "New Foundations Explorer Home Page". *Metamath.* Retrieved 25 April 2024.  
\item Forster, Thomas (October 14, 2007). "Why the Sets of NF do not form a Cartesian-closed Category" (PDF). *www.dpmms.cam.ac.uk.*  
\item Forster, T. E. (2008). "The iterative conception of set" (PDF). *The Review of Symbolic Logic*. 1: 97–110. doi:10.1017/S1755020308080064. S2CID 15231169.  
\item Forster, T. E. (1992). *Set theory with a universal set. Exploring an untyped universe.* Oxford Science Publications, Oxford Logic Guides, vol. 20, New York: The Clarendon Press, Oxford University Press, ISBN 0-19-853395-0, MR 1166801.  
\item Forster, T. E. (2018). "Quine's New Foundations". *Stanford Encyclopedia of Philosophy.*  
\item Grishin, V. N. (1969). "Consistency of a fragment of Quine's NF system". *Dokl. Akad. Nauk SSSR.* 189 (2): 41–243.  
\item Hailperin, T. (1944). "A set of axioms for logic". *Journal of Symbolic Logic*. 9 (1): 1–19. doi:10.2307/2267307. JSTOR 2267307. S2CID 39672836.  
\item Holmes, M. Randall (1998). *Elementary set theory with a universal set* (PDF). *Cahiers du Centre de Logique*, vol. 10, Louvain-la-Neuve: Université Catholique de Louvain, Département de Philosophie, ISBN 2-87209-488-1, MR 1759289.  
\item Holmes, M. Randall (2008). "Symmetry as a Criterion for Comprehension Motivating Quine's 'New Foundations'". *Studia Logica*. 88 (2): 195–213. doi:10.1007/s11225-008-9107-8. S2CID 207242273.
\item Holmes, M. Randall; Wilshaw, Sky (2024). *"New Foundations is consistent"* (PDF).  
\item Jensen, R. B. (1969), *"On the Consistency of a Slight(?) Modification of Quine's NF"*, *Synthese*, 19 (1/2): 250–263, doi:10.1007/bf00568059, JSTOR 20114640, S2CID 46960777. *附奎因的讨论*。  
\item Quine, W. V. (1937), *"New Foundations for Mathematical Logic"*, *The American Mathematical Monthly*, 44 (2), *Mathematical Association of America*: 70–80, doi:10.2307/2300564, JSTOR 2300564.  
\item Quine, Willard Van Orman (1940), *Mathematical Logic* (first ed.), New York: W. W. Norton & Co., Inc., MR 0002508.  
\item Quine, Willard Van Orman (1951), *Mathematical Logic* (Revised ed.), Cambridge, Mass.: Harvard University Press, ISBN 0-674-55451-5, MR 0045661.  
\item Quine, W. V. (1980), *"New Foundations for Mathematical Logic"* in *From a Logical Point of View*, 2nd ed., revised. Harvard Univ. Press: 80–101. *该版本是奎因 1937 年发表在《美国数学月刊》的论文的最终修订版*。  
\item Quine, Willard Van Orman (1987). *"The Inception of 'New Foundations'"*. *Selected Logic Papers - Enlarged Edition*. Harvard University Press. ISBN 9780674798373.  
\item Rosser, Barkley (1942), *"The Burali-Forti paradox"*, *Journal of Symbolic Logic*, 7 (1): 1–17, doi:10.2307/2267550, JSTOR 2267550, MR 0006327, S2CID 13389728.  
\item Specker, Ernst P. (1953). *"The axiom of choice in Quine's New Foundations for Mathematical Logic"*. *Proceedings of the National Academy of Sciences*. 39 (9). *National Academy of Sciences*: 972–975. Bibcode:1953PNAS...39..972S. doi:10.1073/pnas.39.9.972. PMC 1063889. PMID 16589362.  
\item Wang, Hao (1950), *"A formal system of logic"*, *Journal of Symbolic Logic*, 15 (1): 25–32, doi:10.2307/2268438, JSTOR 2268438, MR 0034733, S2CID 42852449.
\end{itemize}
\subsection{外部链接}  
\begin{itemize}
\item Solomon Feferman(2011):《丰富的分层系统用于范畴论的基础》("Enriched Stratified Systems for the Foundations of Category Theory")  
\item 斯坦福哲学百科全书(Stanford Encyclopedia of Philosophy):  
\item 奎因的新基础("Quine's New Foundations"*)——作者:Thomas Forster  
\item 替代公理化集合论("Alternative axiomatic set theories")——作者:Randall Holmes    
\item 新基础(NF)主页("New Foundations Home Page")  
\item 带有通用集合的集合论参考书目*("Bibliography of Set Theory with a Universal Set")  
\item NF 一致性证明的新尝试("A new pass at the NF consistency proof")  
\end{itemize}