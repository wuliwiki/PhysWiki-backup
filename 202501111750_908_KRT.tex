% 库尔特·哥德尔(综述)
% license CCBYSA3
% type Wiki

本文根据 CC-BY-SA 协议转载翻译自维基百科\href{https://en.wikipedia.org/wiki/Kurt_G\%C3\%B6del}{相关文章}。

\begin{figure}[ht]
\centering
\includegraphics[width=6cm]{./figures/6b68f02159236857.png}
\caption{哥德尔,大约在1926年} \label{fig_KRT_1}
\end{figure}
库尔特·弗里德里希·哥德尔(Kurt Friedrich Gödel,1906年4月28日-1978年1月14日)是一位逻辑学家、数学家和哲学家。与亚里士多德和戈特洛布·弗雷格一起,被认为是历史上最重要的逻辑学家之一,哥德尔深刻影响了20世纪的科学和哲学思维(当时,伯特兰·罗素、阿尔弗雷德·诺斯·怀特海德和大卫·希尔伯特正在利用逻辑和集合论研究数学基础),并在弗雷格、理查德·德德金德和乔治·康托尔的早期工作基础上进行扩展。

哥德尔在数学基础方面的发现导致了他在1929年通过其维也纳大学博士论文证明的完备性定理,并且两年后在1931年发表了哥德尔不完备性定理。第一个不完备性定理指出,对于任何足够强大、能够描述自然数算术的 ω-一致递归公理系统(例如,佩亚诺算术),存在一些关于自然数的命题,这些命题既无法从公理中证明,也无法被反驳。为了证明这一点,哥德尔发展了一种现在被称为哥德尔编号的技术,该技术将形式表达式编码为自然数。第二个不完备性定理从第一个定理中推导出来,指出该系统不能证明其自身的一致性。

哥德尔还证明了,在接受的策梅洛–弗伦克尔集合论(Zermelo–Fraenkel set theory)中,选择公理和连续统假设无法被反驳,前提是其公理是一致的。前一个结果为数学家们在其证明中假设选择公理打开了大门。他还通过澄清经典逻辑、直觉主义逻辑和模态逻辑之间的联系,对证明论做出了重要贡献。
\subsection{早年生活与教育}
\subsubsection{童年时期}
哥德尔于1906年4月28日出生在奥匈帝国的布伦(现在的捷克共和国布尔诺),出生在一个讲德语的家庭。父亲鲁道夫·哥德尔(1874-1929)是一个主要纺织厂的总经理和部分股东,母亲玛丽安·哥德尔(原姓汉德舒,1879-1966)。在哥德尔出生时,该市有德语为主的居民,其中包括他的父母。父亲是天主教徒,母亲是新教徒,孩子们也被抚养成新教徒。哥德尔的祖先在布伦的文化生活中通常非常活跃。例如,他的祖父约瑟夫·哥德尔是当时著名的歌手,并且曾是布伦男子合唱团(Brünner Männergesangverein)的成员之一。

当奥匈帝国在第一次世界大战后战败并解体时,哥德尔在12岁时自动成为捷克斯洛伐克的公民。据他的同学克莱佩塔尔说,像许多生活在主要由德语人口构成的苏台德地区的人一样,"哥德尔始终认为自己是奥地利人,并且是捷克斯洛伐克的流亡者"。1929年2月,他被允许放弃捷克斯洛伐克国籍,并于4月获得了奥地利国籍。1938年,当德国吞并奥地利时,32岁的哥德尔自动成为德国公民。1948年,二战后,42岁的哥德尔成为美国公民。

在他的家庭中,年幼的哥德尔被昵称为“为什么先生”(Herr Warum),因为他有着无止境的好奇心。根据他的哥哥鲁道夫的说法,哥德尔在六七岁时曾患过风湿热,虽然完全康复,但他一生都坚信自己的心脏遭受了永久性损害。从四岁起,哥德尔就经常“健康状况不佳”,这种情况贯穿了他的一生。

哥德尔于1912年至1916年间就读于布伦的路德学校(Evangelische Volksschule),并于1916年至1924年在德国语国家文理中学(Deutsches Staats-Realgymnasium)就读,在所有科目中都名列前茅,特别是在数学、语言和宗教方面。虽然哥德尔最初在语言学上表现突出,但后来他对历史和数学产生了更大的兴趣。1920年,哥哥鲁道夫(生于1902年)前往维也纳,在维也纳大学医学院学习时,哥德尔的数学兴趣进一步加深。在青少年时期,哥德尔研究了加贝尔斯伯格速记法、艾萨克·牛顿的批评,以及伊曼努尔·康德的著作。
\subsubsection{维也纳的学习经历}
18岁时,哥德尔与哥哥一起进入维也纳大学。他已经掌握了大学水平的数学。 尽管最初打算学习理论物理学,他也参加了数学和哲学的课程。**在此期间,他接受了数学实在论的观点。他阅读了康德的《自然科学的形而上学基础》,并与莫里茨·施里克、汉斯·哈恩和鲁道夫·卡尔纳普一起参加了维也纳学派的活动。哥德尔随后研究了数论,但当他参加由莫里茨·施里克主持的研讨会,该研讨会研究了伯特兰·罗素的《数学哲学导论》时,他对数学逻辑产生了兴趣。根据哥德尔的说法,数学逻辑是“所有学科之前的科学,包含了所有科学背后的思想和原则。”

听完大卫·希尔伯特在博洛尼亚关于数学系统的完备性与一致性的讲座后,哥德尔可能确定了自己的人生方向。 1928年,希尔伯特和威廉·阿克曼出版了《数学逻辑基础》(Grundzüge der theoretischen Logik),这是一本介绍一阶逻辑的书,提出了完备性问题:“一个形式系统的公理是否足够推导出所有在该系统的所有模型中都为真的命题?”

这个问题成为了哥德尔选择作为博士论文主题的课题。 1929年,23岁的哥德尔在汉斯·哈恩的指导下完成了博士论文。论文中,他确立了自己命名的完备性定理,涉及一阶逻辑。他于1930年获得博士学位, 论文(附带额外的工作)由维也纳科学院出版。