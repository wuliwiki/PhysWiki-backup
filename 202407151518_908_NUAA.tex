% 南京航空航天大学 2002 量子真题
% license Usr
% type Note

\textbf{声明}:“该内容来源于网络公开资料,不保证真实性,如有侵权请联系管理员”

\subsection{一(本题15分)}
\begin{enumerate}
\item 解释:量子力学中的“简并”和“简并度”。
\item 证明:一维无奇性势的薛定谔方程的束缚态无简并。
\end{enumerate}

\subsection{(本题15分)}
氢原子处于状态:
$$\Psi(r, \theta, \phi) = \frac{1}{2} R_{21}(r) Y_{10}(\theta, \phi) - \frac{\sqrt{3}}{2} R_{21}(r) Y_{1-1}(\theta, \phi)~$$
问:氢原子的能量$E$、角动量平方$L^2$、角动量$Z$分量$L$,

这三个量中哪些量具有确定值?哪些量没有确定值?有确定值的求出它的确定值;没有确定值的求出它的可能值及其出现的几率,并求出其平均值。

\subsection{(本题16分)}
已知在角动量 $\hat{L}^2$ 和 $\hat{L}_z$ 的共同表象中,算符 $\hat{L}_x$ 的矩阵
为:$$L_x = \frac{\hbar}{\sqrt{2}} \begin{pmatrix}
0 & 1 & 0 \\
1 & 0 & 1 \\
0 & 1 & 0
\end{pmatrix}~$$

(1)求它的本征值和归一化本征函数。

(2)找出一个么正变换矩阵S,将算符L对角化。

\subsection{(本题18分)}
一维运动粒子处在状态
$$\psi(x) = 
\begin{cases} 
2\lambda^{3/2} x e^{-\lambda x} &  x \geq 0 \\
0 &  x < 0
\end{cases}~$$
其中 $\lambda > 0$. 求:
\begin{enumerate}
    \item 粒子位置坐标的平均值。
    \item 粒子动量的几率分布函数。
    \item 粒子动量的平均值。
\end{enumerate}
已知:
$$\left( \int_{0}^{\infty} x^n e^{-ax} \, dx \right) = \frac{n!}{a^{n+1}}~$$

\subsection{(本题18分)}
(1)设$\hat q$为位置坐标算符,$\hat p$为动量算符。
证明测不准关系:
$\Delta q.\Delta p\ge \frac{\hbar}{2}$

(2)卢瑟福$\alpha$粒子散射实验确证原子核半径大小数量级为$10^{-13}$米;原子$\beta$衰变放出电子的能量在10ev以内。根据这些实验事实,试用上面的测不准关系论证:电子不可能是原子核的组成部分。

(1ev=$1.602\times10^{-19}J$\\
$h=6.626\times10^{-34}J.s$\\
电子质量m =$9.109\times {10^{-31}kg}$)

\subsection{(本题18分}
试用微扰理论讨论氢原子$n=2$能级在外电场作用下产生的谱线分裂现象(…级斯塔克效应)。
已知氢原子波函数
\begin{align}
\psi_{200} &= \frac{1}{4\sqrt{2\pi}} \left( \frac{1}{a_0} \right)^{3/2} (2 - \frac{r}{a_0}) e^{-r/2a_0} \\
\psi_{210} &= \frac{1}{4\sqrt{2\pi}} \left( \frac{1}{a_0} \right)^{3/2} \left( \frac{r}{a_0} \right) e^{-r/2a_0} \cos \theta \\
\psi_{211} &= \frac{1}{8\sqrt{\pi}} \left( \frac{1}{a_0} \right)^{3/2} \left( \frac{r}{a_0} \right) e^{-r/2a_0} \sin \theta e^{i\phi} \\
\psi_{21-1} &= \frac{1}{8\sqrt{\pi}} \left( \frac{1}{a_0} \right)^{3/2} \left( \frac{r}{a_0} \right) e^{-r/2a_0} \sin \theta e^{-i\phi}~
\end{align}

$ a_0 = \frac{\hbar^2}{me^2}$ 为第一玻尔轨道半径
$$ \int_{0}^{\infty} x^n e^{-\alpha x} \, dx = \frac{n!}{\alpha^{n+1}}~$$