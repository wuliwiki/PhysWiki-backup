% 圆锥曲线的配极(高中数学)
% keys 高中数学|圆锥曲线|极点|极线
% license Xiao
% type Art






\begin{definition}{关于圆的极点与极线}\label{def_EclPol_1}

给定一个圆$C$,取$C$外一点$P$,如\autoref{fig_EclPol_1} 所示。

过点$P$作圆$C$的切线,切点为$A$和$B$,则称直线$AB$是点$P$\textbf{关于圆}$C$的极线。

反之,取圆$C$的一根弦,与圆的交点为$A$和$B$,过这两个点作圆$C$的切线,称其交点$P$为直线$AB$\textbf{关于圆}$C$的极点。

\begin{figure}[ht]
\centering
\includegraphics[width=8cm]{./figures/542dd0f77d5d5e19.pdf}
\caption{关于圆的极点与极线的示意图。点$P$和直线互为关于给定圆的极点和极线。} \label{fig_EclPol_1}
\end{figure}

\end{definition}




极点和极线总是成对出现,因此一定要强调“关于圆的”。比如说,单独给定一个圆和一个点,不能说这个点就是圆的极点,因为没有“圆的极点”这种说法。

\autoref{def_EclPol_1} 中只局限于关于圆的情况,且$P$点在圆之外。事实上,极点和极线的相对关系可以关于所有圆锥曲线定义,点$P$也可以在平面上任何位置。我们接下来就通过讨论逐步明晰这些概念。




\begin{theorem}{极线的方程}

给定圆$C:x^2+y^2=R^2$和点$P=(x_0, y_0)$,则$P$关于圆$C$的极线为
\begin{equation}
l: x_0x+y_0y = R^2~. 
\end{equation}

\end{theorem}



\textbf{证明}:

如\autoref{fig_EclPol_1} ,极线$l=AB$与直线$PO$正交,而

\textbf{证毕}。


















