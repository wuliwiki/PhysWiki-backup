% 电能(高中)
% keys 电能|电功|电功率|焦耳定律

\subsection{电功}

电流通过导体所做的功(静电力对电荷所做的功)叫做\textbf{电功},反映了电能消耗(或转化为其他形式能)的多少,用$W$表示.

在国际单位中,电功的单位是\textbf{焦耳}(简称\textbf{焦}),符号为$\mathrm{J}$.

若导体两端存在电压$U$,自由电荷在静电力的作用下作定向移动,形成的电流为$I$,在时间$t$内移动的总电荷量为$q=It$,则静电力对电荷做功为:

\begin{equation}
W=qU=UIt
\end{equation}

\subsection{电功率}

电功与对应做功所用时间之比叫做\textbf{电功率},电功率用于表示电流做功的快慢,用$P$表示,则:

\begin{equation}
P=\frac{W}{t}=UI
\end{equation}

在国际单位中,电功率的单位是\textbf{瓦特}(简称\textbf{瓦}),符号为$\mathrm{W}$.

上两式适用于任何电路.

\subsection{纯电阻电路的计算}

按照电路中连接的元件,可将电路分为\textbf{纯电阻电路}和\textbf{非纯电阻电路}.纯电阻电路中只有电阻元件,非纯电阻电路除了电阻元件还包括可将电能转化为其他形式能的元件(如:电动机\footnote{如果电动机通电时被卡住没转起来,则无机械能输出,可看作是纯电阻,这种情况容易烧坏.}).

在纯电阻电路中,电压、电流及电阻之间遵循欧姆定律$I=U/R$,电功和电功率的计算可分别写成:

\begin{equation}
W=UIt=I^2 Rt=\frac{U^2}{R}t
\end{equation}

\begin{equation}
P=UI=I^2 R=\frac{U^2}{R}
\end{equation}

从能量转化的角度看,纯电阻电路工作时电能全部转化为内能,非纯电阻电路工作时电能转化为内能和其他形式能.
