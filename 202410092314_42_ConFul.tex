% 齐次凸泛函
% keys 凸性|泛函
% license Usr
% type Tutor
\pentry{泛函与线性泛函\nref{nod_Funal}}{nod_63ed}

齐次凸泛函是与凸集紧密联系的概念,共有的“凸”也表达了这一印象。以后我们将看到,非负齐次凸泛函与其\enref{核}{ConSet}含0点的\enref{凸体}{ConSet}是一一对应的。当然,凸泛函可以看作是\enref{凸函数}{ConvFu}在一般线性空间中的推广。

\begin{definition}{凸泛函}
设 $L$ 是实线性空间。$L$ 上的\enref{泛函}{Funal} $p$ 称为\textbf{凸的},是指对任意 $x,y\in L,0\leq\alpha\leq1$,成立
\begin{equation}
p(\alpha x+(1-\alpha)y)\leq \alpha p(x)+(1-\alpha)p(y).~
\end{equation}
\end{definition}

\begin{definition}{正齐次}
$L$ 上的泛函 $p$ 称为\textbf{正齐次的},是指对所有的 $x\in L,\alpha >0$,
\begin{equation}
p(\alpha x)=\alpha p(x).~
\end{equation}
\end{definition}
注意,齐次凸泛函就是指有凸性和正齐次性的泛函,而不仅仅是\enref{齐次}{Funal}。这里齐次只是正齐次的简称。
\begin{definition}{齐次凸泛函}
正齐次的凸泛函简称为\textbf{齐次凸泛函}。
\end{definition}

\begin{theorem}{}
设 $p$ 是正齐次泛函,则 $p$ 凸当且仅当 
\begin{equation}\label{eq_ConFul_1}
p(x+y)\leq p(x)+p(y).~
\end{equation}
\end{theorem}

\textbf{证明:}\textbf{当:}设 $p$ 凸。则
\begin{equation}
p(x+y)=2p\qty(\frac{x+y}{2})\leq 2\qty(p\qty(\frac{x}{2})+p\qty(\frac{y}{2}))=p(x)+p(y).~
\end{equation}
\textbf{仅当:}设\autoref{eq_ConFul_1} 成立。则
\begin{equation}
p(\alpha x+(1-\alpha)y)\leq p(\alpha x)+p((1-\alpha)y)\leq \alpha p(x)+(1-\alpha)p(y).~
\end{equation}
即 $p$ 凸。

\textbf{证毕!}

\subsection{凸泛函的性质}

\begin{theorem}{凸泛函的性质}
设 $p$ 是凸泛函,则成立
\begin{enumerate}
\item $p(0)=0$;
\item $0\leq p(x)+p(-x)$;
\item $p(\alpha x)\geq\alpha p(x).\alpha\in\mathbb R$。
\end{enumerate}
\end{theorem}

\textbf{证明:}



\textf{证明!}








