% 阿贝尔微分方程恒等式
% keys 朗斯基行列式|Wronski行列式|Wronskian|线性微分方程|Abel's Identity|Abel's Formula|阿贝尔公式

本文翻译并节选自WikiPedia的\href{https://en.wikipedia.org/wiki/Abel\%27s_identity}{Abel's Identity}.


\subsection{综述}

在数学中,\textbf{阿贝尔恒等式(Abel's identity)}(亦称\textbf{阿贝尔公式(Abel's formula)}\footnote{ Rainville, Earl David; Bedient, Phillip Edward (1969). \href{https://archive.org/details/elementarydiffer00rain}{ Elementary Differential Equations} . Collier-Macmillan International Editions.} 或者 \textbf{阿贝尔微分方程恒等式(Abel's differential equation identity)}),是一个等式,用于表示一个二阶齐次线性常微分方程的两个解的朗斯基行列式,只需要用到原方程的系数.这一关系也可以推广到$n$阶的线性微分方程.该恒等式命名自挪威数学家Niels Henrik Abel.

由于阿贝尔恒等式把微分方程不同的线性独立解联系起来了,因此它也可以用来从一个特解得到另一个特解.它给出了解之间很有用的恒等关系,同时也在参数变易法(variation of parameters)等其它技巧中功不可没.在Bessel方程等无法给出简单解析解的方程中尤其有用,因为这些情况下朗斯基行列式非常难算.

用\href{https://en.wikipedia.org/wiki/Liouville\%27s_formula}{Liouville公式}能将阿贝尔恒等式推广到齐次线性微分方程的一阶系统上.



\subsection{公式描述}

考虑二阶齐次线性微分方程
\begin{equation}\label{AbelID_eq1}
y'' + p(x)y' +q(x)y = 0
\end{equation}
其中$p, q$是实数轴上一区间$I$上的连续函数,函数值为实数或者复数.

阿贝尔恒等式是说,对于\autoref{AbelID_eq1} 的任意两个解$y_1, y_2$以及任意$x_0\in I$,其朗斯基行列式$W[y_1, y_2]$满足以下等式:
\begin{equation}
W[y_1, y_2](x) = W[y_1, y_2](x_0)\cdot \exp\qty(-\int _{x_0}^x p(z)\dd z), \quad x\in I
\end{equation}


\subsubsection{批注}

\begin{itemize}
\item 特别地,朗斯基行列式$W[y_1, y_2]$要么在$I$上恒等于零,
\end{itemize}

In particular, the Wronskian {\displaystyle W(y_{1},y_{2})}{\displaystyle W(y_{1},y_{2})} is either always the zero function or always different from zero with the same sign at every point {\displaystyle x}x in {\displaystyle I}I. In the latter case, the two solutions {\displaystyle y_{1}}y_{1} and {\displaystyle y_{2}}y_{2} are linearly independent (see the article about the Wronskian for a proof).

It is not necessary to assume that the second derivatives of the solutions {\displaystyle y_{1}}y_{1} and {\displaystyle y_{2}}y_{2} are continuous.

Abel's theorem is particularly useful if {\displaystyle p(x)=0}{\displaystyle p(x)=0}, because it implies that {\displaystyle W}W is constant.












