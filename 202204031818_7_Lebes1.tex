% 非负函数的Lebesgue积分
% keys 实变函数|勒贝格积分

\pentry{可测函数\upref{MsbFun}}

Lebesgue积分的思路是对函数的值域进行分划,以相应值域的逆映射作为“柱底”.归根到底,Lebesuge积分还是要对定义域作分划的,但相比Riemann积分的直接对定义域作分划,Lebesgue积分的分划方式更任意.对于可测函数,Lebesgue积分的分划得到的“柱底”都是可测集.

我们就从将可测集划分为两两不交的可测子集入手,先研究这种分划的性质.

\subsection{可测集的分划}



\begin{definition}{可测分划}
设$E\in\mathbb{R}^n$是可测集.如果有限\textbf{族}$\{E_1, E_2, \cdots, E_n\}$中各$E_i$\textbf{两两不交}、都是$E$的子集、\textbf{可测},且$E=\bigcup^n_{i=1}E_i$,那么称集族$\{E_i\}_{i=1}^n$为可测集$E$的一个\textbf{分划},或者\textbf{可测分划}.
\end{definition}

如果$A=\{E_i\}_{i=1}^n$和$B=\{F_i\}_{i=1}^m$都是$E$的分划,那么易证$C=\{E_i\cap F_j|E_i\in A, F_j\in B\}$也是$E$的分划.称$C$是分划$A$和$B$的\textbf{合并}.

容易看到,$C$中存在每一个$E_i$的分划$\{E_i\cap F_j\}_{j=1}^m$,类似地也存在每一个$F_j$的分划,像是更细一层地进行分划.因此,如果分划$C$是$A$和另一个分划的合并,我们就称$C$是比$A$\textbf{更细}的分划,反过来$A$比$C$\textbf{更粗}.

\begin{definition}{上和与下和}

设$f$是$E$上的非负可测函数,$D=\{E_i\}_{i=1}^n$是$E$的一个可测分划.定义$a_i=\inf_{x\in E_i}f(x)$,$A_i=\sup_{x\in E_i}f(x)$,则称
\begin{equation}
s_D=\sum_{i=1}^n a_i \opn{m}E_i
\end{equation}
为$f$在$E$上关于分划$D$的\textbf{下和},而称
\begin{equation}
S_D=\sum_{i=1}^n A_i \opn{m}E_i
\end{equation}
为$f$在$E$上关于分划$D$的\textbf{上和}.

\end{definition}

如果$A\subseteq B\subseteq E$,那么显然$f$在$A$上的上确界要小于等于在$B$上的上确界,在$A$上的下确界要大于等于在$B$上的下确界,因此容易得出以下引理:

\begin{lemma}{}

设$f$是可测集$E$上的可测函数,$A$和$B$是$E$的可测分划,且$A$比$B$更细.那么

\begin{equation}
s_B\leq s_A\leq S_A\leq S_B
\end{equation}


\end{lemma}

由此可得一个有用的推论:

\begin{corollary}{}
设$f$是可测集$E$上的可测函数,$D_1$和$D_2$是$E$的可测分划.那么

\begin{equation}
s_{D_i}\leq S_{D_j}
\end{equation}
对任意$i, j\in\{1, 2\}$成立.

\end{corollary}

就是说,不管怎么求分划,任意两个分划之间,上和一定大于等于下和,不会出现一个分划的下和大于另一个分划的上和这种情况.






\subsubsection{积分}

有了分划和上下和的概念,我们描述起积分就方便多了.

\begin{definition}{}

设$f$是可测集$E$上的可测函数,$\Lambda$

\end{definition}



















