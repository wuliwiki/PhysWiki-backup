% 约翰内斯·开普勒(综述)
% license CCBYSA3
% type Wiki

本文根据 CC-BY-SA 协议转载翻译自维基百科\href{https://en.wikipedia.org/wiki/Johannes_Kepler}{相关文章}。

约翰内斯·开普勒(Johannes Kepler,/ˈkɛplər/;德语:[joˈhanəs ˈkɛplɐ, -nɛs -] ⓘ;1571年12月27日–1630年11月15日)是德国天文学家、数学家、占星家、自然哲学家以及音乐作家。他是17世纪科学革命的关键人物,以行星运动定律最为人知,并且以《新天文学》(Astronomia nova)、《世界和谐论》(Harmonice Mundi)和《哥白尼天文学概要》(Epitome Astronomiae Copernicanae)等著作影响了以艾萨克·牛顿为代表的科学家,为牛顿的万有引力理论提供了基础之一。开普勒的工作具有多样性和深远影响,使他成为现代天文学、科学方法、自然科学和现代科学的奠基人之一。他被誉为“科幻小说之父”,因为他的小说《梦境》(Somnium)。 

开普勒曾是格拉茨一所神学院的数学教师,在那里他成为了汉斯·乌尔里希·冯·埃根贝格(Prince Hans Ulrich von Eggenberg)王子的合作者。后来,他成为天文学家第谷·布拉赫(Tycho Brahe)在布拉格的助手,并最终成为神圣罗马帝国皇帝鲁道夫二世及其继任者马提亚斯和斐迪南二世的皇家数学家。他还曾在林茨教授数学,并且是沃尔斯坦将军的顾问。此外,开普勒在光学领域做出了基础性贡献,被誉为现代光学之父,尤其以《光学天文学》为代表。他还发明了改进版的折射望远镜——开普勒望远镜,成为现代折射望远镜的基础,同时改进了伽利略·伽利莱的望远镜设计,伽利略在他的著作中提到了开普勒的发现。

开普勒生活在一个天文学和占星学没有明确界限的时代,但天文学(作为自由艺术中的一门数学分支)和物理学(作为自然哲学的一门分支)之间却有着明显的区分。开普勒还将宗教论证和推理融入到他的工作中,受宗教信仰的激励,他认为上帝按照可以通过理性之光理解的智能计划创造了这个世界。开普勒将他的新天文学描述为“天体物理学”,作为“对亚里士多德《形而上学》的探索”,并作为“对亚里士多德《天论》的补充”,通过将天文学视为普遍数学物理学的一部分,开普勒彻底改造了古代的物理宇宙学传统。
\subsection{早期生活}  
\subsubsection{童年(1571年–1590年)}

开普勒的出生地,魏尔德施塔特  
开普勒于1571年12月27日出生在魏尔德施塔特的自由帝国城市(现为德国巴登-符腾堡州斯图加特地区的一部分)。他的祖父塞巴尔德·开普勒曾是该市的市长。到约翰内斯出生时,开普勒家族的财富已开始衰退。他的父亲海因里希·开普勒以雇佣兵的身份维持生计,在约翰内斯五岁时离开了家人,据信他在荷兰的八十年战争中去世。他的母亲凯瑟琳娜·古尔登曼是位酒馆老板的女儿,也是一个治疗师和草药师。约翰内斯有六个兄弟姐妹,其中两个兄弟和一个姐妹活到了成年。由于早产,他自称小时候身体虚弱且多病。然而,他常常在祖父的酒馆里给旅客留下深刻的印象,展现出他非凡的数学才能。[22]