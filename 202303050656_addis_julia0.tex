% Julia 解释器笔记

\begin{issues}
\issueDraft
\end{issues}

本文是关于 Julia 解释器的原理: 它使用了哪些技术, 可以使得它作为一门动态语言能达到编译语言的性能。

\begin{itemize}
\item \textbf{JIT} 编译器: 所有的代码都经过 JIT 编译, 官方称为 \textbf{just-ahead-of-time}。
\item \textbf{Multiple Dispatch (MD)}: 可以基于参数的类型生成不同的专门代码。
\item \textbf{Type Inference}: 自动推导变量类型。
\item 内建并行, 包括分布计算
\item 高效内存管理: 减少内存分配, 增加内存重复利用。
\end{itemize}
