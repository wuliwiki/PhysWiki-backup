% 华东师范大学 2013 年硕士入学考试物理试题
% keys 华东师范大学|考研|物理|2013年
% license Copy
% type Tutor

\textbf{声明}:“该内容来源于网络公开资料,不保证真实性,如有侵权请联系管理员”

\textbf{考生注意:}
可以使用计算器
\begin{enumerate}
\item 一质点沿x轴作直线运动,其$v(t)$曲线如图所示,$t=0$时,质点位于坐标原点,则$t=4.5s$时,质点在$x$轴上的位置为:\\
(A)5m\\
(B)2m\\
(C)0\\
(D)-2m\\
(E)-5m
\begin{figure}[ht]
\centering
\includegraphics[width=6cm]{./figures/ba93a1544a8f8e30.png}
\caption{} \label{fig_HDSD13_1}
\end{figure}
\item 质点作曲线运动,F表示位置矢量,$\bar v$表示速度,$\bar a$表示加速度,S表示路程,$\alpha$表示切向加速度,下列表达式中\\
(1)$dv/dt=\alpha \qquad$
(2)$dr/dt=v \qquad$
(3)$dS/dt=v\qquad$
(4)$\abs{dv/dt}=\alpha_t$\\
(A)只有(1)是对的\\
(B)只有(2)是对的\\
(C)只有(1)(2)是对的\\
(D)只有(3)是对的\\
(E)只有(2)(3)是对的\\
(F)只有(4)是对的
\item 下列关于牛顿定律的说法哪些是不正确的?\\
(A)牛顿定律只在惯性系中成立,在非惯性系中不成立。\\
(B)牛顿第一定律是牛顿第二定律在$F=0$的情况下的特例。
(C)在非惯性系中使用牛顿第二定律时,需考虑惯性力。
(D)牛顿第一定律指出惯性是物质本身的一种属性。
\item 下列说法中,正确的是:\\
(A)物体沿竖直面上光滑的圆弧形轨道下滑的过程中,轨道对物体的支持力不断增加。\\
(B)从点作圆周运动时,所受的合力一定指向圆心。\\
(C)用水平压力把一个物体压着靠在粗糙的竖直墙面上保持静止,当$F$逐渐增大时,物体所受的静摩擦力$f$也随$F$成正比地增大。\\
(D)一辆汽车从静止出发,在平直公路上加速前进的过程中,如果发动机的功率一定,力大小不变,则汽车的加速度不变。

\item 体重、身高相同的甲乙两人,分别用双手握住跨过无摩擦轻滑轮的子各一端。他们从同一高度由初速为零向上爬,经过一定时间,甲相对统子的速率是乙相对绳子速率的两倍,则到达顶点的情况是\\
(A)甲先到达\\
(B)乙先到达\\
(C)同时到达\\
(D)谁先到达不能确定
\item 质量为 $M$ 的平板车,以速度 $\bar v$ 在光滑的水平面上滑行,一质量为 $m$ 的物体从 $h$  高处竖直落到车子里,两者一起运动时的速度大小为 $(\qquad)$
\item  一颗速率为700m/s的子弹,打穿一块木板后,速率降到500m/s。如果让它继续穿过厚度和阻力均与第一块完全相同的第二块木板,则子弹的速率将降到$(\qquad)$(空气阻力忽略不计)
\item  下列物理量:质量,动量,冲量,动能,势能,功中与参考系的选取有关的物理量是$(\qquad)$。(不考虑相对论效应)
\item  关于刚体对轴的转动惯量,下列说法中正确的是$(\qquad)$\\
(A)只取决于刚体的质量,与质量的空间分布和轴的位置无关。\\
(B)取决于刚体的质量和质量的空间分布,与轴的位置无关。\\
(C)取决于刚体的质量,质量的空间分布和轴的位置。\\
(D)只取决于转轴的位置,与刚体的质量和质量的空间分布无关。
\item 一圆盘正在饶垂直于盘面的水平光滑固定轴O转动,如图射来两个质量相同,速度大小相同。方向相反并在一条直线上的子弹,子弹射入圆盘并且留在盘内,则子弹射入后的瞬间,圆盘的角速度$(\qquad)$
(A)增大\\
(B)减小\\
(C)不变\\
(D)不能确定
\item 已知冰的熔解热为$3.35$x$10^5J/kg$,则一克$0$°C的冰溶解成$0$°C的水时,它的熵变为$(\qquad)$
\item 两个容积相同的容器,一个盛氢气,另一个盛氦气,开始时它们的压强和温度相等,现将6J热量传给氦气,使之升高一定的温度,若欲使氢气也升高同样的温度,则应向氢气传递的热量为$(\qquad)$
\item 某理想气体体积按$\displaystyle v=\frac{a}{\sqrt{p}}$的规律变化,求气体从体积为$V_1$膨胀到$V_2$以时所做的功为$(\qquad)$
\item 摩尔数相同的氦气和氢气,其压强和分子数密度相同,则它们的$(\qquad)$\\
A.分子平均速率相同\\
B.分子平均动能相何\\
C.内能相同\\
D.分子平均平动动能相等
\item 一热机由温度为$727$℃的高温热源处吸热,向温度为$527$°C的低温热源处放热,若热机在最大效率下工作,且每一循环吸热$2000J$,则此热机每一循环做功为$(\qquad)$\\
(A)1600J\\
(B)400J\\
(C)550J\\
(D)1457J
\item 三个容器A,B,C中装有同种理想气体,其分子数密度相同,而方均根速率之比为$\sqrt{V^2_A }: \sqrt{V^2_B}:\sqrt{V^2_C}=1:2:4$,则其压强之比$P_A:P_B:P_c$为$(\qquad)$\\
(A)1:2:4\\
(B)4:3:1\\
(C)1:4:16\\
(D)1:4:8
\item 在容积$V=4.0*10^-3 m^2$的容器中,装有压強为$p=5*10^2pa$的理想气体,则容器中气体分子的平均平动动能总和为$(\qquad)$\\
(A)2J\\
(B)3J\\
(C)5J\\
(D)9J
\item 若室内温度从15°C升高到27℃,而室内气压不变,则此时室内分子数约减少了$(\qquad)$\\
(A)4\%\\
(B)5\%\\
(C)9\%\\
(D)10\%
\item 对于室温下的双原子分子理想气体,在等压膨胀的情况下,系统对外所做的功与从外界吸收的热盘之比A/Q等于$(\qquad)$\\
(A)$\displaystyle \frac{1}{3}$\\
(B)$\displaystyle \frac{1}{4}$\\
(C)$\displaystyle \frac{2}{5}$\\
(D)$\displaystyle \frac{2}{7}$
\item 一瓶氧气和氦气,它们的质量密度相同,分子的平均平动动能相同,则它们$(\qquad)$
(A)温度和压强都相同\\
(B)温度和压强都不相同\\
(C)温度相同,但氧气的压強大于氦气的压强\\
(D)温度相同,但氧气的压強小于氦气的压強
\item 图中所示曲线表示球对称或轴对称静电场的某一物理量随径向距高r变化的关系,请指出该曲线可描述下列哪方面内容(E为电场强度的大小,U为电势)$(\qquad)$
(A)半径为R的无限长均匀带电圆柱体电场的E-r 关系\\
(B)半径为R的无限长均匀带电圆柱面电场的E-r 关系\\
(C)半径为R的均匀带正电球体电场的U-r关系\\
(D)半径为R的均匀带正电球面电场的U-r关系
\begin{figure}[ht]
\centering
\includegraphics[width=8cm]{./figures/dabd5d896393286f.png}
\caption{} \label{fig_HDSD13_3}
\end{figure}
\item 点电荷Q被曲面S所包围,从无穷远处引入另一点电荷q至曲面外一点,如图所示,则引入前后$(\qquad)$\\
(A)曲面S的电场强度通量不变,曲面上各点场强变化\\
(B)曲面S的电场强度通量变化,曲面上各点场强不变\\
(C)曲面S的电场强度通量变化,曲面上各点场强变化\\
(D)曲面S的电场强度通量不变,曲面上各点场强不变
\begin{figure}[ht]
\centering
\includegraphics[width=8cm]{./figures/d123f0e1abebd67b.png}
\caption{} \label{fig_HDSD13_2}
\end{figure}
\item 两根无限长平行指导线载有大小相等反向相反的电流I,并各以  的变化率增长,一矩形线圈位于导线平面内(如图),则正确的选项是$(\qquad)$\\
(A)线圈中无感应电流\\
(B)线圈中感应电流为顺时针方向\\
(C)线圈中感应电流为逆时针方向\\
(D)线圈中感应电流的方向不确定
\begin{figure}[ht]
\centering
\includegraphics[width=8cm]{./figures/7f065380b165f328.png}
\caption{} \label{fig_HDSD13_4}
\end{figure}
\item 尺寸相同的铁环与铜环所包围的面积中,通以相同变化率的磁通量,当不计环的自感时,环中$(\qquad)$
(A)感应电动势大小不同
(B)感应电动势大小相同,感应电流大小相同
(C)感应电动势大小不同,感应电流大小相同
(D)感应电动势大小相同,感应电流大小不同
\item 电磁波在自由空间传播时,那么电场强度$\bar E$与磁场强度$\bar H$ $(\qquad)$\\
(A)在垂直于传播方向上的同一条直线上\\
(B)朝互相垂直的两个方向传播\\
(C)互相垂直,且都垂直于传播方向\\
(D)有相位差$\pi/2$
\item 在一个不带电的导体球壳内,先放进一电荷为+q的点电荷,点电荷不与球壳内壁接触,然后使该球壳与地接触一下,再将点电荷+q取走。此时,球壳的电荷为$(\qquad)$,电场分布的范围是$(\qquad)$。
\item 一半径为R的均匀带电圆环,电荷线密度为$\lambda$。设无穷远处为电势零点。则圆环中心o点的电势U=$(\qquad)$
\item 如图,无限长直载流导线中通有电流I,其右侧有面积为$S_1$和$S_2$的两个矩形回路,长为l,宽度分别为a和2a。两个回路与长直载流导线在同一平面,且矩形回路的一边与长直载流导线平行,则通过面积为$S_1$的矩形回路的磁通量与通过面积为$S_2$的矩形回路的磁通量分别为$(\qquad)$和$(\qquad)$
\begin{figure}[ht]
\centering
\includegraphics[width=8cm]{./figures/09dfe9aae852b56f.png}
\caption{} \label{fig_HDSD13_5}
\end{figure}
\item 菲涅尔圆孔衍射中,对某一参考点,衍射孔只露出1/3个半波带,现在撤掉衍射屏,则有衍射屏和没有衍射屏两种情况下该点光强之比为$(\qquad)$\\
(A)1:2\\
(B)2:1\\
(C)1:$\sqrt{2}$\\
(D)$\sqrt{2}$:1\\
(E)1:1
\item 在光栅夫琅禾费衍射实验中,单色平行光由垂直射向光栅改变为斜入射光栅,观察到的光谱线$(\qquad)$\\
(A)最高级次变小,条数不变\\
(B)最高级次变大,条数不变\\
(C)最高级次变大,条数变多\\
(D)最高级次不变,条数不变
\item 在单缝夫琅禾费衍射实验中,单色平行光由垂直射向单缝,现将单缝后的成像透镜向上平移一小段距离,则观测屏上的条纹$(\qquad)$\\
(A)位置变化,间距变化\\
(B)位置变化,间距不变\\
(C)位置不变,间距变化\\
(D)位置不变。间距不变
\item 光栅的总缝数为N,光栅常数为d,则对某一级光谱而言,其分辨本领$(\qquad)$\\
(A)由N决定\\
(B)由d决定\\
(C)由Nd 决定\\
(D)由1/(Nd)决定
\item 右旋圆偏振光盘直通过 1/2波片后,其偏振态是$(\qquad)$\\
(A)左旋园偏振光\\
(B)右旋圆偏振光\\
(C)左旋椭圆偏振光\\
(D)右旋椭圆偏振光
\item 理想单色光的波列长度为$(\qquad)$
\item 光场中某一点的复振幅为$-Ae^{i\theta}/ (2t)$,则该点电场振动的相位为$(\qquad)$
\item Michelson干涉仪形成等倾条纹时,两反射镜可等效为厚度为d的空气膜,当入射光波长为$\lambda$时,观测到中心最亮,则第一暗环的角半径为$(\qquad)$
\item 单色光以某一角度$\alpha$(与法线夹角)斜入射一单缝形成Fiaunhofer衍射,光波长为$\lambda$,缝宽为b,则实验测出零级衍射斑的半角宽为$(\qquad)$
\item 圆偏振光入射四分之一波片,其出射光的偏振性质为$(\qquad)$
\item 基于德布罗意假设得出的公式$\lambda=\frac{12.26}{\sqrt{E_R(eV)}}$°A的通用条件是$(\qquad)$\\
(A)带电的任何粒子,非相对论近似\\
(B)一切实物粒子,非相对论近似\\
(C)被电场束缚的电子,相对论结果\\
(D)自由电子,非相对论近似
\item 用能量为12.7eV的电子去激发基态氢原子时,受激氢原子向低能级跃迁时最多可能出现几条光谱线(不考虑自旋)\\
(A)3\\
(B)10\\
(C)1\\
(D)4
\item $\frac{e}{m}$为电子的荷质比,L,S和J分别为电子的总轨道角动量、总自旋角动量和总角动量,则原子的总磁矩为\\
(A)$\displaystyle -\frac{e}{2m}(L+S)$\\
(B)$\displaystyle -\frac{e}{2m}(J+S)$\\
(C)$\displaystyle -\frac{e}{2m}(2L+S)$\\
(D)$\displaystyle -\frac{e}{2m}\frac{(L+S)}{J(J+1)}$
\item 电子填充原子壳层时,下列说法不正确的是\\
(A)一个被填充满的支壳层,所有的角动量为零\\
(B)一个支壳层被填满半数时,总轨道角动量为零\\
(C)必须是填满一个支壳层以后再开始填充另一个新支壳层\\
(D)一个壳层中按泡利原理容纳的电子数为$2n^2$,
\item 由A个核子组成的原子核的结合能为$\delta E=\delta mc^2$,其中$\delta m$指
A.2个质子和 A-Z个中子的静止质量之差B.A个核子的运动所量和核运动质量之差C.A个核子的运动质量和核静止质量之差D.A个核子的静止质量和核静止质盘之差
\item 在电磁波谱中,由原于核产生的射线是
\item 证明电子存在自旋的实验证据有(列举至少3个实验)
\item 发生正常寤曼效应的条件是
\item 波长为1A的X光光子的动量为
\item 我尔氢原子理论的局限性在于
\end{enumerate}
