% 塞曼效应
% keys 磁矩1|能级2|角动量3


在量子力学中,谱线的频率和波长的偏移意味着一种或两种状态的能级的跃迁.由单重态之间的在磁场作用下能级发生分裂,谱线分裂成间隔相等的3条谱线的塞曼效应(Zeeman Effect)被称为正常效应,而在一些实验中所观察到光谱线有时并非分裂成3条,其间隔也都不相同,这种当初始态或最终态或两者的总自旋为非零时所发生的塞曼效应被称为反常效应.尽管这两者并没有什么实质性区别,但在电子磁矩的值非常大时用反常效应解释会使问题复杂化,因此我们先来考虑由单重态的跃迁产生的赛曼正常效应.

当单个原子处于一均匀磁场$\mathbf{B}_{\rm{ext}}$中时,能级会产生跃迁.对于单个电子的扰动(perturbation)为
$$H_Z^{'} = -(\boldsymbol{\mu}_l+\boldsymbol{\mu_s})\cdot \mathbf{B}_{\rm{ext}},$$
其中电子自旋的磁偶极矩为$$\boldsymbol\mu _s =-\frac{e}{m}\mathbf{S}$$经典轨道运动的偶极矩为
$$\boldsymbol\mu _l =-\frac{e}{2m}\mathbf{L}$$因此就有$$H_Z^{'} = \frac{e}{2m}(\mathbf{L+2S})\cdot \mathbf{B}_{\rm{ext}}$$