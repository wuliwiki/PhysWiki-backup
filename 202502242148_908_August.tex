% 奥古斯丁-路易·柯西(综述)
% license CCBYSA3
% type Wiki

本文根据 CC-BY-SA 协议转载翻译自维基百科\href{https://en.wikipedia.org/wiki/Pierre-Simon_Laplace}{相关文章}。

\begin{figure}[ht]
\centering
\includegraphics[width=6cm]{./figures/e9138cc29ee186d2.png}
\caption{大约在1840年的柯西。由Zéphirin Belliard根据Jean Roller的画作所绘制的石版画。} \label{fig_August_1}
\end{figure}
奥古斯丁-路易·柯西男爵(英国发音:/ˈkoʊʃi/ KOH-shee,/ˈkaʊʃi / KOW-shee;美国发音:/koʊˈʃiː / koh-SHEE;法语:[oɡystɛ̃ lwi koʃi];1789年8月21日 – 1857年5月23日)是法国数学家、工程师和物理学家。他是最早严格阐述并证明微积分核心定理的人之一(由此创建了实分析),开创了复分析领域,并研究了抽象代数中的置换群。柯西还在数学物理的多个领域做出了贡献,特别是在连续介质力学方面。

作为一位深刻的数学家,柯西对他的同时代人和后继者产生了深远的影响;汉斯·弗罗伊登塔尔曾说:

“比任何其他数学家都更多的概念和定理是以柯西命名的(仅在弹性学领域,就有十六个概念和定理是以柯西命名的)。”

柯西是一个多产的学者;他撰写了大约八百篇研究文章和五本关于数学及数学物理领域的完整教材。
\subsection{传记}  
\subsubsection{青年时期与教育}  
柯西是路易·弗朗索瓦·柯西(1760–1848)和玛丽-玛德琳·德塞斯特的儿子。柯西有两个兄弟:亚历山大·劳伦特·柯西(1792–1857),他于1847年成为上诉法院某部门的主席,并于1849年成为法国最高法院的法官;尤金·弗朗索瓦·柯西(1802–1877),一位公共知识分子,也撰写了几部数学著作。从小,柯西就擅长数学。

柯西于1818年与阿洛伊丝·德·比尔结婚。她是出版商的亲戚,该出版商出版了柯西的大部分著作。他们有两个女儿:玛丽·弗朗索瓦·阿丽西亚(1819年)和玛丽·马蒂尔德(1823年)。

柯西的父亲是旧 régime 巴黎警察部门的高级官员,但由于法国大革命(1789年7月14日)爆发,他失去了这一职位,而这场革命爆发时,奥古斯丁-路易·柯西正好出生在一个月前。[a] 柯西一家在1793至1794年的恐怖统治期间幸存下来,他们逃到了阿尔居伊尔,在那里柯西从父亲那里接受了初步教育。[6] 1794年罗伯斯比尔被处决后,柯西一家得以安全返回巴黎。在那里,路易-弗朗索瓦·柯西在1800年找到了一份官僚工作,[7] 并迅速晋升。拿破仑在1799年上台后,路易-弗朗索瓦·柯西得到了进一步的晋升,成为参议院的秘书长,直接在拉普拉斯(现因其在数学物理方面的贡献而更为人知)手下工作。数学家拉格朗日也是柯西一家的朋友。[4]

在拉格朗日的建议下,奥古斯丁-路易·柯西于1802年秋季入学了当时巴黎最好的中学——巴黎中央理工学校(École Centrale du Panthéon)。[6] 该校大部分课程是经典语言;雄心勃勃的柯西是一名出色的学生,在拉丁语和人文学科上赢得了许多奖项。尽管取得了这些成功,柯西还是选择了工程师的职业,并为进入巴黎高等工艺学院(École Polytechnique)的入学考试做准备。

1805年,他在293名考生中名列第二,顺利被录取。[6] 这所学校的主要目的是为未来的 civil 和军事工程师提供高水平的科学和数学教育。学校实行军事纪律,这使得柯西在适应上遇到了一些问题。尽管如此,他还是在18岁时于1807年完成了课程,并进入了桥梁与道路学校(École des Ponts et Chaussées)。他以最高荣誉获得土木工程学位。
\subsubsection{工程生涯}  
1810年毕业后,柯西接受了在瑟堡担任初级工程师的工作,那里拿破仑计划建设一个海军基地。柯西在这里待了三年,负责了乌尔克运河工程和圣克劳德桥工程,并在瑟堡港工作。[6] 尽管他的工作非常繁忙,他还是找到了时间准备三篇数学稿件,并将其提交给法国科学院的第一学部。[b] 柯西的前两篇稿件(关于多面体)被接受;第三篇稿件(关于圆锥曲线的导线)被拒绝。

1812年9月,23岁的柯西因过度劳累而生病,返回巴黎。[6] 他回到首都的另一个原因是他对工程工作的兴趣逐渐减退,越来越被数学的抽象美吸引;在巴黎,他有更好的机会找到与数学相关的职位。1813年,当他的健康有所好转时,柯西决定不再返回瑟堡。[6] 尽管他正式保留了工程师的职位,但他被调离了海军部的工资单,转到了内政部。在接下来的三年里,柯西主要处于无薪病假状态;他充分利用这段时间,专心研究数学(涉及对称函数、对称群和高阶代数方程的理论等相关课题)。他曾三次尝试进入法国科学院第一学部,但在1813年至1815年期间三次未能成功。1815年,拿破仑在滑铁卢战败,路易十八国王恢复了王位。法国科学院于1816年3月重新成立;拉扎尔·卡诺和贾斯帕·孟热因政治原因被移除,国王任命柯西取代其中一位的席位。柯西同行的反应非常激烈;他们认为柯西被接纳为科学院成员是一种侮辱,柯西因此在科学界结下了许多敌人。
\subsubsection{巴黎高等工艺学院教授}  
1815年11月,路易·波安索(Louis Poinsot),当时是巴黎高等工艺学院的一名副教授,因健康原因请求免除教学工作。此时,柯西已是冉冉升起的数学新星。他当时的一大成就是证明了费马的多边形数定理。他辞去了工程师的工作,获得了为巴黎高等工艺学院二年级学生教授数学的一年合同。1816年,这所非宗教的拿破仑派学校经过重组,几位自由派教授被解雇;柯西因此被晋升为正教授。

当柯西28岁时,他仍与父母同住。他的父亲认为是时候让儿子结婚了,便为他找到了一个合适的妻子——比他小五岁的阿洛伊丝·德·比尔。德·比尔家族是印刷商和书商,出版了柯西的大部分著作。[8] 阿洛伊丝与奥古斯丁于1818年4月4日在圣叙尔皮斯教堂举行了盛大的天主教婚礼。1819年,这对夫妇的第一个女儿玛丽·弗朗索瓦·阿丽西亚出生,1823年,第二个也是最后一个女儿玛丽·马蒂尔德出生。[9]

直到1830年,保守的政治气候非常适合柯西。在1824年,路易十八去世,由他更加保守的兄弟查理十世继位。在这些年里,柯西高产,接连发表了一部又一部重要的数学著作。他还在法国高等师范学院和巴黎科学院担任交叉任职。[fr]
\subsubsection{流亡}  
1830年7月,七月革命在法国爆发。查理十世逃离了国家,路易-菲利普继位。发生了暴乱,巴黎高等工艺学院的学生们积极参与其中,暴乱的发生地点离柯西的家很近。

这些事件标志着柯西生活的转折点,也使他的数学创作出现了停顿。由于政府的垮台以及对掌权的自由派的深深憎恨,柯西离开了法国,前往国外,留下了他的家人。[10] 他在瑞士弗里堡待了一段时间,在那里他必须决定是否向新政权宣誓效忠。他拒绝了这一要求,因此失去了在巴黎的所有职位,除了他的科学院成员身份,因为这个职位不需要宣誓。1831年,柯西前往意大利的都灵,在那里待了一段时间后,他接受了撒丁王国国王的邀请(撒丁王国统治都灵及周边的皮埃蒙特地区),为他专门设立了一个理论物理学的职位。他在都灵教书,任教于1832年至1833年。1831年,他被选为瑞典皇家科学院外籍成员,第二年被选为美国艺术与科学学院的外籍荣誉会员。[11]

1833年8月,柯西离开都灵前往布拉格,成为十三岁的波尔多公爵亨利·达尔图瓦(1820–1883)的科学导师。亨利·达尔图瓦是被流放的查理十世的孙子和王储。[12] 作为巴黎高等工艺学院的教授,柯西曾是一个臭名昭著的糟糕讲师,他假设学生的理解能力只有他最优秀的几个学生才能达到,并且在有限的时间内塞满了太多内容。亨利·达尔图瓦既没有兴趣也没有才能从事数学或科学。尽管柯西非常认真地对待这项任务,但他做得非常笨拙,并且对亨利·达尔图瓦缺乏令人惊讶的权威。在他担任土木工程师时,柯西曾短暂地负责过修理一些巴黎下水道,他犯了一个错误,提到这段经历给他的学生听;亨利·达尔图瓦怀着极大的恶意,开始说柯西是在巴黎的下水道里开始他的职业生涯。柯西作为导师的角色持续到亨利·达尔图瓦18岁,即1838年9月。[10] 在这五年里,柯西几乎没有做任何研究,而亨利·达尔图瓦则对数学产生了终生的厌恶。柯西被封为男爵,这个头衔对他来说非常重要。

1834年,他的妻子和两个女儿搬到了布拉格,柯西在流亡四年后终于与家人团聚。
\subsubsection{最后的岁月}  
柯西于1838年底返回巴黎,并恢复了他在法国科学院的职位。[10] 由于他仍然拒绝宣誓效忠,他未能重新获得教学职位。
\begin{figure}[ht]
\centering
\includegraphics[width=6cm]{./figures/7c1001f8a8901986.png}
\caption{柯西晚年} \label{fig_August_2}
\end{figure}
1839年8月,长久处于空缺状态的经度局(Bureau des Longitudes)出现了空缺。这个局与法国科学院有一些相似之处;例如,它有权共选成员。此外,人们认为经度局的成员可以“忽略”宣誓效忠,尽管形式上,与科学院成员不同,他们仍然被要求宣誓。经度局是一个成立于1795年的组织,旨在解决海上定位问题——主要是经度坐标的问题,因为纬度可以通过太阳的位置轻易确定。由于认为海上定位最好通过天文观测来确定,经度局已经发展成一个类似于天文科学学院的组织。

1839年8月,长久处于空缺状态的经度局(Bureau des Longitudes)出现了空缺。这个局与法国科学院有一些相似之处;例如,它有权共选成员。此外,人们认为经度局的成员可以“忽略”宣誓效忠,尽管形式上,与科学院成员不同,他们仍然被要求宣誓。经度局是一个成立于1795年的组织,旨在解决海上定位问题——主要是经度坐标的问题,因为纬度可以通过太阳的位置轻易确定。由于认为海上定位最好通过天文观测来确定,经度局已经发展成一个类似于天文科学学院的组织。

在整个十九世纪,法国教育系统一直在努力解决教会与国家的分离问题。在失去对公立教育系统的控制后,天主教会试图建立自己的教育分支,并在柯西身上找到了一个坚定而显赫的盟友。他将自己的声望和知识贡献给了巴黎的圣职教育学院(École Normale Écclésiastique),这是一所由耶稣会士管理的学校,旨在培养他们学院的教师。他还参与了天主教学院(Institut Catholique)的创立。该学院的目的是应对法国缺乏天主教大学教育的影响。这些活动并未让柯西在他的同事中获得欢迎,因为他们大多支持法国革命时期的启蒙思想。1843年,法国高等师范学院数学教席空缺,柯西申请了这个职位,但只得到了45票中的3票。

1848年,路易-菲利普国王逃往英国。宣誓效忠被废除,柯西的学术任命之路变得清晰。1849年3月1日,他重新被任命为法国科学院的数学天文学教授。在1848年整个政治动荡之后,法国选择成为一个共和国,由法国的拿破仑三世担任总统。1852年初,总统自封为法国皇帝,并取名拿破仑三世。

在官僚圈子里出现了一个想法,认为再次要求所有国家公务员,包括大学教授,宣誓效忠是有用的。这一次,一位内阁部长成功说服皇帝免除柯西的宣誓要求。1853年,柯西被选为美国哲学会的国际会员。[13] 柯西一直担任大学教授,直到67岁去世。他在5月23日凌晨4点接受了临终圣礼,并因支气管病症去世。[10]

他的名字是刻在埃菲尔铁塔上的72个名字之一。