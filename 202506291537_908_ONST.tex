% 欧内斯特·劳伦斯(综述)
% license CCBYSA3
% type Wiki

本文根据 CC-BY-SA 协议转载翻译自维基百科\href{https://en.wikipedia.org/wiki/Ernest_Lawrence}{相关文章}。

欧内斯特·奥兰多·劳伦斯(Ernest Orlando Lawrence,1901年8月8日 – 1958年8月27日)是美国的加速器物理学家,因发明回旋加速器而获得1939年诺贝尔物理学奖。他因在曼哈顿计划中进行铀同位素分离的工作而闻名,也因创办了劳伦斯伯克利国家实验室和劳伦斯利物浦国家实验室而著名。

劳伦斯毕业于南达科他大学和明尼苏达大学,1925年在耶鲁大学获得物理学博士学位。1928年,他被聘为加利福尼亚大学伯克利分校的物理学副教授,并在两年后成为该校最年轻的正教授。在某个晚上,他在图书馆里看到了一个产生高能粒子的加速器示意图,这引起了他的兴趣。他思考如何将其做得更加紧凑,最终想出了一个思路,即将加速室设计成一个圆形,置于电磁铁的两极之间。由此诞生了第一台回旋加速器。

劳伦斯随后建造了一系列越来越大、成本越来越高的回旋加速器。他的辐射实验室在1936年成为加利福尼亚大学的一个官方部门,劳伦斯担任其主任。除了回旋加速器在物理学中的应用,劳伦斯还支持其在放射性同位素医学应用研究中的使用。在第二次世界大战期间,劳伦斯在辐射实验室开发了电磁同位素分离技术。该技术使用了被称为“卡鲁特朗”的装置,这是一种结合了标准实验室质谱仪和回旋加速器的混合型设备。在田纳西州橡树岭建造了一个巨大的电磁分离工厂,后来被称为Y-12。这个过程效率低下,但它成功地实现了目标。

战后,劳伦斯广泛倡导政府资助大型科学项目,并且是“大科学”理念的有力支持者,这一理念要求大型机器和大量资金。劳伦斯强烈支持爱德华·泰勒争取建立第二个核武器实验室的运动,这个实验室最终设在加利福尼亚州利物浦。劳伦斯去世后,加利福尼亚大学的理事会将劳伦斯利物浦国家实验室和劳伦斯伯克利国家实验室以他的名字命名。化学元素103号被命名为劳伦素,以表彰他在1961年伯克利发现这一元素。
\subsection{早年生活}
欧内斯特·奥兰多·劳伦斯于1901年8月8日出生在南达科他州的坎顿。他的父母卡尔·古斯塔夫斯(1871–1954)和冈达·雷吉娜(原姓雅各布森)(1874–1959)均为挪威移民的后代,他们在坎顿的一所高中教授课程,劳伦斯的父亲还担任该校的校长。他有一个弟弟约翰·H·劳伦斯,后来成为一名医生,并在核医学领域开创了先河。在成长过程中,他最好的朋友是梅尔·图夫,这位朋友后来也成为了一位杰出的物理学家。

劳伦斯就读于坎顿和皮埃尔的公立学校,然后进入明尼苏达州北菲尔德的圣奥拉夫学院,但在一年后转学到南达科他州的南达科他大学。他于1922年获得化学学士学位,1923年在明尼苏达大学获得物理学硕士(M.A.)学位,导师是威廉·弗朗西斯·格雷·斯旺。作为硕士论文,劳伦斯设计了一种实验装置,通过磁场旋转一个椭球体。

劳伦斯跟随斯旺先后进入芝加哥大学和耶鲁大学,最终在耶鲁大学完成了他的物理学博士(PhD)学位,获得国家研究员奖学金,并于1925年撰写了关于钾蒸气光电效应的博士论文。他被选为Sigma Xi会员,并在斯旺的推荐下,获得了国家研究委员会的奖学金。与当时通常前往欧洲不同,他选择留在耶鲁大学,继续作为研究员与斯旺合作。

在与弗吉尼亚大学的杰西·比姆斯合作后,劳伦斯继续研究光电效应。他们证明了光电子在光子撞击光电表面后的$2 \times10^{-9}$秒内就会出现——这接近当时的测量极限。通过快速开关光源来缩短发射时间,使得发射的能量谱变得更宽,符合维尔纳·海森堡的不确定性原理。
\subsection{早期职业生涯}
1926年和1927年,劳伦斯分别收到来自华盛顿大学和加利福尼亚大学的助理教授职位邀请,年薪为3,500美元(相当于2024年的63,400美元)。耶鲁大学迅速匹配了这个职位,但年薪为3,000美元。劳伦斯选择留在更具声望的耶鲁大学,但由于他从未担任过讲师,部分同事对此职位表示不满,许多人认为这并未能弥补他南达科他移民背景的不足。

1928年,劳伦斯被聘为加利福尼亚大学的物理学副教授,两年后成为正教授,成为该校最年轻的教授。基于弗雷德里克和伊莲娜·居里于1934年发表的关于人工放射性的研究成果,劳伦斯通过在他的实验室中用高能质子轰击碳-13元素,发现了氮-13同位素。他和他的团队,包括马丁·卡门和塞缪尔·鲁本,在用高能质子轰击石墨时意外发现了碳-14同位素。罗伯特·戈登·斯普劳尔在劳伦斯成为教授的第二天成为了加利福尼亚大学的校长,他是博希米亚俱乐部的成员,并在1932年赞助了劳伦斯加入该俱乐部。通过这个俱乐部,劳伦斯结识了威廉·亨利·克罗克、埃德温·保利和约翰·弗朗西斯·内兰。这些有影响力的人物帮助他为他的核粒子研究筹集资金。人们对粒子物理学的医学应用充满了巨大的期望,这也促成了劳伦斯能够获得早期研究资金的大部分来源。

在耶鲁大学时,劳伦斯遇到了玛丽·金伯利(莫莉)·布鲁默,她是耶鲁大学医学院院长乔治·布鲁默的四个女儿中的长女。他们于1926年初次见面,并于1931年订婚,1932年5月14日,在康涅狄格州纽黑文的三一教堂举行了婚礼。他们育有六个孩子:埃里克、玛格丽特、玛丽、罗伯特、芭芭拉和苏珊。劳伦斯将他的儿子罗伯特命名为以纪念他的亲密朋友、理论物理学家罗伯特·奥本海默,后者在伯克利与他关系深厚。1941年,莫莉的妹妹埃尔西与埃德温·麦克米兰结婚,麦克米兰后来与格伦·T·希博格一起获得了1951年诺贝尔化学奖。
\subsection{回旋加速器的开发}
\subsubsection{发明}
使劳伦斯声名鹊起的发明,起初仅是一个草图,画在一张纸餐巾上。1929年某个晚上,劳伦斯在图书馆里翻阅一篇罗尔夫·维德尔厄的期刊文章时,被其中的一幅图示吸引。这幅图描绘了一种通过一系列小的“推动”来产生高能粒子的装置。图中的装置按直线排列,使用越来越长的电极。当时,物理学家们开始探索原子核。1919年,新西兰物理学家欧内斯特·拉塞福德将阿尔法粒子射入氮原子,成功地将一些核中的质子撞击出来。但是,原子核带有正电荷,会排斥其他带正电的原子核,并且它们通过物理学家刚开始理解的力紧密结合在一起。要打破它们、使它们解体,需要更高的能量,达到数百万伏特的级别。
\begin{figure}[ht]
\centering
\includegraphics[width=10cm]{./figures/faed33bdaaa7ac28.png}
\caption{劳伦斯1934年专利中的回旋加速器操作示意图} \label{fig_ONST_1}
\end{figure}
劳伦斯意识到,这种粒子加速器很快就会变得过长且难以操作,不适合他的大学实验室。在思考如何使加速器更紧凑时,劳伦斯决定将一个圆形加速室放置在电磁铁的两极之间。磁场会将带电的质子保持在螺旋路径上,同时它们在两个连接到交变电压的半圆形电极之间加速。经过大约一百圈,质子将以高能粒子束的形式撞击靶标。劳伦斯兴奋地告诉同事们,他发现了一种方法,可以在不使用任何高电压的情况下获得非常高能量的粒子。他最初与尼尔斯·埃德尔夫森合作。他们的第一个回旋加速器是用黄铜、金属丝和密封蜡制成的,直径只有四英寸(10厘米)——可以用一只手拿着,成本大约为25美元(相当于2024年的600美元)。

为了发展这一想法,劳伦斯需要有能力的研究生来进行工作。埃德尔夫森于1930年9月离开,去担任助理教授,劳伦斯用大卫·H·斯洛恩和M·斯坦利·李文斯顿替代了他,分别让他们着手开发维德尔厄的加速器和埃德尔夫森的回旋加速器。两人都有自己的资金支持。这两种设计都证明是可行的,到1931年5月,斯洛恩的线性加速器已能够将离子加速到1 MeV。李文斯顿面临更大的技术挑战,但当他在1931年1月2日对他的11英寸回旋加速器施加1,800伏特电压时,他成功让80,000电子伏的质子旋转起来。一周后,他使用3,000伏特电压达到了1.22 MeV,这远远足够支撑他关于回旋加速器构建的博士论文。
\subsubsection{发展}
\begin{figure}[ht]
\centering
\includegraphics[width=8cm]{./figures/f1028083d22e9a26.png}
\caption{1940年在伯克利举行的会议,讨论计划中的184英寸(4.67米)回旋加速器(见黑板上的示意图):劳伦斯、阿瑟·康普顿、范尼瓦尔·布什、詹姆斯·B·科南特、卡尔·T·康普顿和阿尔弗雷德·李·卢米斯。} \label{fig_ONST_2}
\end{figure}
在接收到成功的初步信号后,劳伦斯开始规划一台更大、更强的机器,这成了一个反复出现的模式。1932年初,劳伦斯和李文斯顿设计了一台27英寸(69厘米)的回旋加速器。800美元的11英寸回旋加速器所用的磁铁重达2吨,但劳伦斯在帕洛阿尔托的一个废品堆场里发现了一个原本用于第一次世界大战时支撑跨大西洋无线电连接的巨大80吨磁铁,用于27英寸回旋加速器。在这个回旋加速器中,劳伦斯拥有了一种强大的科学仪器,但这并未转化为科学发现。1932年4月,约翰·科克罗夫特和欧内斯特·沃尔顿在英国剑桥大学的卡文迪许实验室宣布,他们用质子轰击锂并成功将其转变为氦。所需的能量非常低——完全在11英寸回旋加速器的能力范围内。得知此事后,劳伦斯向伯克利发了一封电报,要求验证科克罗夫特和沃尔顿的实验结果。该团队直到9月才完成验证,主要由于缺乏足够的探测设备。

尽管重要的发现继续与劳伦斯的辐射实验室擦肩而过,主要是因为实验室专注于回旋加速器的发展而非其科学用途,但通过他日益增大的机器,劳伦斯能够为高能物理实验提供至关重要的设备。围绕这一设备,他建立了世界上最顶尖的核物理学研究实验室,这个实验室在1930年代成为了核物理新领域的领先研究中心。他于1934年获得了回旋加速器的专利,并将其转让给了研究公司,这是一个资助劳伦斯早期大部分工作的私人基金会。

1936年2月,哈佛大学校长詹姆斯·B·科南特向劳伦斯和奥本海默提出了有吸引力的邀请。加利福尼亚大学校长罗伯特·戈登·斯普劳尔通过改善条件作出回应。1936年7月1日,辐射实验室成为加利福尼亚大学的一个官方部门,劳伦斯正式被任命为其主任,配有全职副主任,且大学同意每年拨出20,000美元用于实验室的研究活动(相当于2023年的350,000美元)。劳伦斯采用了一个简单的商业模式:“他用物理系的研究生和初级教职员工、愿意为任何薪水工作的刚获得博士学位的人、以及能无偿服务的奖学金获得者和富有的客人来为实验室配备人员。”
\subsubsection{反响}
利用新建的27英寸回旋加速器,伯克利的团队发现,他们用新发现的氘轰击的每种元素都会释放能量,而且释放的能量在相同的范围内。因此,他们推测存在一种新的、迄今未知的粒子,可能是源源不断的能量来源。*纽约时报*的威廉·劳伦斯将劳伦斯描述为“科学界的新奇迹工作者”。应科克罗夫特的邀请,劳伦斯参加了1933年在比利时召开的索尔维大会。这是全球顶级物理学家的定期聚会,几乎所有与会者都来自欧洲,但偶尔像罗伯特·A·米利肯或阿瑟·康普顿这样的杰出美国科学家会被邀请参加。劳伦斯被邀请在会上展示回旋加速器的研究成果。

劳伦斯关于无尽能源的主张在索尔维大会上得到了截然不同的反响。他遭遇了来自卡文迪许实验室的詹姆斯·查德威克的强烈怀疑。查德威克是1932年发现中子并因此获得1935年诺贝尔奖的物理学家。查德威克用一种带有轻蔑语气的英国口音对劳伦斯说,他认为劳伦斯的团队所观察到的现象是他们的仪器遭到污染。
\begin{figure}[ht]
\centering
\includegraphics[width=8cm]{./figures/ac6850fa5a08f73e.png}
\caption{} \label{fig_ONST_3}
\end{figure}