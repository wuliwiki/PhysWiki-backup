% 流
% license Usr
% type Tutor
我们知道,对切向量场积分可以得到一族曲线,比如地球的一系列纬线。如果加上初始条件,便可以确定具体是哪一条曲线。因此,流形$M$上的积分曲线实际上有两个含参变量,设其为$\theta(t,p),\forall p\in M$。我们可以固定$p$点为曲线的初始位置,得到一条曲线:$\theta_p(t)$,亦可以固定$t$,得到$\theta_t(p):M\rightarrow M$。
\subsection{全局流}
令$p\in M$为曲线的初始位置,即$\theta(0,p)=p$。若$t\in\mathbb R$,我们称曲线族$\theta(t,p):\mathbb R\times M\rightarrow M$为\textbf{全局流(global flow)}。

可以证明\footnote{translation lemma proved in \textsl{introduction to smooth maniflod}},对于$\theta_t(p)$,如果$q=\theta_p(s)$,则$\theta^{(q)}(t)=\theta^{(p)}(t+s)$,所以该光滑映射满足:
\begin{equation}\label{eq_flow_1}
\begin{aligned}
\theta(t, \theta(s, p)) & =\theta(t+s, p) \\
\theta(0, p) & =p~.
\end{aligned}
\end{equation}
上式的结合律也可以表示为$\theta_t \circ \theta_s(p)=\theta_{t+s}(p)$。因此$\theta$可看作加法群$\mathbb R$对流形$M$的作用,\autoref{eq_flow_1} 便等价于群作用定律:
\begin{equation}
\begin{aligned}
\theta_t \circ \theta_s & =\theta_{t+s}, \\
\theta_0 & =\operatorname{Id}_M~.
\end{aligned}
\end{equation}
所以全局流又称作\textbf{单参量群作用(one-parameter group action)}。可以证明,$\theta_t:M\rightarrow M$是微分同胚映射\footnote{只要证明该映射是双向光滑双射即可。因为是群作用,由消去律可知是单射,又因为$\theta^{-1}_t=\theta_{-t}$,因此逆映射也是光滑的}。在后续章节里,我们将利用该映射的前推,以得到光滑切场的李导数。


从另一个角度来看,当我们固定$p$时,光滑曲线$\theta_p(t)$便是$\mathbb R$对$p$作用的轨道。由于轨道是等价类划分,因而不同轨道相交为空集,这与我们的直觉是符合的。

下面我们来证明,全局流都是光滑切场的积分曲线。
\begin{theorem}{}
设$\theta$是全局流,对于任意$p\in M$,定义切向量$V_p\in T_p M$为
\begin{equation}
V_p=\theta^{(p) \prime}(0)=\left.\frac{\partial}{\partial t}\right|_{t=0} \theta(t, p) ~,
\end{equation}
则该切向量集合构成$M$上的光滑切场$V$,并且$\theta^p(t)$都是$V$的积分曲线。
\end{theorem}
\textbf{证明:}

首先我们来证明,如上定义的切场是光滑的,即对于任意光滑函数$f\in C^{\infty}M$,$Vf\in C^{\infty}M$。

对于任意$p\in M$,我们有:

\begin{equation}
Vf(p)=V_pf=\theta^{(p)\prime}(0)f=\left.\frac{\dd}{\dd t}\right|_{t=0} f\left(\theta^{(p)}(t)\right)=\left.\frac{\dd}{\dd t}\right|_{t=0} f(\theta(t, p)) ~,
\end{equation}
求导之外是光滑函数的复合,因此结果也是光滑函数。
接下来我们只需证明,$\theta^{(p) \prime}(t)=V_{\theta^{p}(t)}$即可。

设$q=\theta_p(t_0)$,则
\begin{equation}
\begin{aligned}
V_qf&=\theta^{(q)\prime}(0)f\\
&=\left.\frac{\dd}{\dd t}\right|_{t=0} f(\theta(t, q))\\
&=\left.\frac{\dd}{\dd t}\right|_{t=0} f(\theta(t+t_0, p))\\
&=~.
\end{aligned}
\end{equation}
\subsection{局域流}
\subsection{流的基本定理}
