% 李代数的子代数、理想与商代数
% keys 李代数|子代数|理想|正规化子




\pentry{李代数\upref{LieAlg}}
和抽象代数中的子群、子环等类比,李代数也可以有次级结构,即子代数.

设$\mathfrak{m}, \mathfrak{n}$是李代数$\mathfrak{g}$的非空子集,定义子集间的运算为$\mathfrak{m}+\mathfrak{n}=\{M+N|M\in\mathfrak{m}, N\in\mathfrak{n}\}$,以及$[\mathfrak{m}, \mathfrak{n}]=\{[M, N]|M\in\mathfrak{m}, N\in\mathfrak{n}\}$.那么如果$\mathfrak{m}, \mathfrak{n}$和$\mathfrak{p}$都是$\mathfrak{g}$作为线性空间的子空间,我们容易证明以下性质:

\begin{itemize}
\item $[\mathfrak{m}+\mathfrak{n}, \mathfrak{p}]\subseteq[\mathfrak{m}, \mathfrak{p}]+[\mathfrak{n}, \mathfrak{p}]$;
\item $[\mathfrak{m},\mathfrak{n}]=[\mathfrak{n}, \mathfrak{m}]$;
\item $[\mathfrak{m}, [\mathfrak{n}, \mathfrak{p}]]\subseteq[\mathfrak{n}, [\mathfrak{m}, \mathfrak{p}]]+[\mathfrak{p}, [\mathfrak{n}, \mathfrak{m}]]$.
\end{itemize}

\begin{definition}{李代数的子代数}
若$\mathfrak{g}$是李代数,$\mathfrak{h}$是它作为线性空间的子空间,且有$[\mathfrak{h}, \mathfrak{h}]\subseteq\mathfrak{h}$,那么称$\mathfrak{h}$是$\mathfrak{g}$的\textbf{子代数(sub (Lie) algebra)}.
\end{definition}


尽管李代数并不成环,但是我们也可以仿照环的理想的定义,用吸收律来定义李代数的理想:

\begin{definition}{李代数的理想}
若$\mathfrak{g}$是李代数,$\mathfrak{h}$是它作为线性空间的子空间,且有$[\mathfrak{g}, \mathfrak{h}]=\mathfrak{h}$,那么称$\mathfrak{h}$是$\mathfrak{g}$的\textbf{理想(ideal)}.

显然,$\mathfrak{g}$和$\{0\}$都是$\mathfrak{g}$的理想,称为\textbf{平凡理想(trivial ideal)}.
\end{definition}





