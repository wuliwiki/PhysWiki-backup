% 原子单位制
% keys 原子|国际单位|薛定谔方程|量子力学
% license Xiao
% type Tutor

\pentry{国际单位制\upref{SIunit}, 无单位的物理公式\upref{NoUnit}, 玻尔原子模型\upref{BohrMd}}

\footnote{参考 Wikipedia \href{https://en.wikipedia.org/wiki/Hartree_atomic_units}{相关条目}。 以及 \cite{Bransden} 的附录。}在量子力学的许多理论或数值计算中,选用\textbf{原子单位(Hartree atomic unit)}会更方便。 需要特别注意的是, 原子单位下的物理公式在使用时的习惯和其他单位制有所不同, 例如会出现 $E = \omega$ (能量等于角频率, 见\autoref{eq_AU_2})这种看似不符合量纲分析的公式。 要理解这种记号, 这里推荐的方法是把使用原子单位的公式都理解成无单位的物理公式\upref{NoUnit}。


若声明使用原子单位, 物理量的数值后面就不需要标记单位, 例如 “氢原子基态能量为 $E \approx -0.5$”。 有时候为了强调我们使用原子单位, 我们会在数值后面加上 “a.u.”, 如 “$E \approx 0.5 \Si{a.u.}$”。 这里的 “a.u.” 可等效为 “单位 1”, 类似弧度单位 “Rad”\footnote{例如在扇形面积公式 $S = \theta R^2/2$ 中的 $\theta$ 可以看作具有单位 “Rad”, 但面积的单位却只是 “$\Si{m^2}$” 而无需记为 “$\Si{Rad\cdot m^2}$”。}。

\subsection{推导}\label{sub_AU_1}
原子单位下的物理量和公式都可以看作是无量纲的数字。 沿用 “无单位的物理公式\upref{NoUnit}” 中的记号, 长度、质量、时间、能量、一维波函数的转换常数分别记为 $\beta_x$, $\beta_m$, $\beta_t$, $\beta_E$, $\beta_\Psi$。

在量子力学中, 由于普朗克常数 $\hbar$ 大量出现, 我们希望它在原子单位下等于 $1$, 即 $\beta_L = \hbar$。 注意 $\hbar$ 具有角动量量纲。 令角动量公式 $\bvec L = m \bvec r \cross \bvec v$ 在原子单位下依然成立, 也就是
\begin{equation}
\beta_L\bvec L_a = (\beta_m m_a) (\beta_x \bvec r_a) \cross \qty(\frac{\beta_x}{\beta_t} \bvec v_a)~.
\end{equation}
可以化为 $\bvec L_a = m_a \bvec r_a \cross \bvec v_a$, 即
\begin{equation}\label{eq_AU_6}
\beta_t = \frac{\beta_m \beta_x^2}{\hbar}~.
\end{equation}
这样, 量子力学中所有公式中的 $\hbar$ 都可以看作是一个角动量常数并用数字 $1$ 代替了。

再看能量, 令动能公式 $E_k = mv^2/2$ 在原子单位中成立, 那么同理要求
\begin{equation}\label{eq_AU_7}
\beta_E = \frac{\beta_m \beta_x^2}{\beta_t^2} = \frac{\hbar}{\beta_t}~.
\end{equation}


\begin{example}{光子的能量}
我们来看光子的能量公式(链接未完成)
\begin{equation}
E = \hbar \omega \Longrightarrow E_a = \frac{\hbar\beta_\omega}{\beta_E}\omega_a~.
\end{equation}
为满足周期的定义($T_a = 2\pi/\omega_a$), 必须令角频率的转换常数为
\begin{equation}
\beta_\omega = \frac{1}{\beta_t}~,
\end{equation}
再将上文的 $\beta_E$ 和 $\beta_t$ 代入可得原子单位下光子能量等于角频率
\begin{equation}\label{eq_AU_2}
E_a = \omega_a~.
\end{equation}
\end{example}

\subsection{薛定谔方程}
\pentry{薛定谔方程\upref{TDSE},定态薛定谔方程(单粒子一维)\upref{SchEq}}
\begin{example}{无单位的薛定谔方程}\label{ex_AU_1}
国际单位下的一维薛定谔方程为(\autoref{eq_TDSE11_3}~\upref{TDSE11})
\begin{equation}\label{eq_AU_1}
-\frac{\hbar^2}{2m} \pdv[2]{\Psi}{x} + V\Psi= \I\hbar \pdv{\Psi}{t}~,
\end{equation}
令 $x = x_a\beta_x$, $m = m_a\beta_m$, $t = t_a\beta_t$, $V = V_a\beta_E$, $\Psi = \Psi_a \beta_\Psi$ (注意 $x_a, m_a$ 等都是不带单位的)。
代入\autoref{eq_AU_1}, 各项同除 $\beta_E\beta_\Psi$, 得\footnote{根据偏微分的定义, 常数可以移到偏微分算符外, 如 $\pdv*[2]{(\beta_x x_a)} = (1/\beta_x^2) \pdv*[2]{x_a}$}
\begin{equation}\label{eq_AU_3}
-\qty(\frac{\hbar^2}{\beta_m\beta_x^2\beta_E})\frac{1}{2m_a} \pdv[2]{\Psi_a}{x_a} + V_a\Psi_a= \I\qty(\frac{\hbar}{\beta_E\beta_t})\pdv{\Psi_a}{t_a}~.
\end{equation}
使用\autoref{eq_AU_6} 和\autoref{eq_AU_7}, 发现两个括号中的常数都是 1, 于是无单位的薛定谔方程为
\begin{equation}\label{eq_AU_4}
-\frac{1}{2m_a} \pdv[2]{\Psi_a}{x_a} + V_a\Psi_a= \I\pdv{\Psi_a}{t_a}~.
\end{equation}
这个过程也可以简单认为是把\autoref{eq_AU_1} 中的 $\hbar$ 替换为数字 1。

我们再看波函数的归一化公式
\begin{equation}
1 = \int \abs{\Psi}^2 \dd{x} = \beta_\Psi^2 \beta_x \int \abs{\Psi_a}^2 \dd{x_a}~,
\end{equation}
为了使归一化公式的形式不变, 必须令
\begin{equation}\label{eq_AU_5}
\beta_\Psi = \beta_x^{-1/2}~.
\end{equation}
同理, 对 $N$ 维波函数有 $\beta_\Psi = \beta_x^{-N/2}$。
\end{example}

\begin{example}{动量算符}\label{ex_AU_2}
国际单位中 $x$ 方向动量的本征方程为
\begin{equation}\label{eq_AU_8}
-\I\hbar \dv{x} \psi(x) = p \psi(x)~,
\end{equation}
解得本征态(已归一化\upref{EngNor})为平面波
\begin{equation}\label{eq_AU_9}
\psi(x) = \frac{1}{\sqrt{2\pi\hbar}}\exp(\I p x/\hbar)~.
\end{equation}
与\autoref{ex_AU_1} 同理, 在原子单位中\autoref{eq_AU_8} 变为
\begin{equation}
-\I\dv{x_a} \psi_a(x_a) = k_a \psi_a(x_a)~,
\end{equation}
其中 $k_a$ 就是原子单位中的动量, 它和波数 $k_a = 2\pi/\lambda_a$ 数值相同。 从这个意义上, \textbf{原子单位中, 动量等于波数, 能量等于频率。}

动量本征态\autoref{eq_AU_9} 变为
\begin{equation}
\psi_a(x) = \frac{1}{\sqrt{2\pi}}\E^{\I kx}~.
\end{equation}
注意这就是傅里叶变换中的基底。 所以在原子单位中, 把波函数投影的动量本征态上就是做傅里叶变换\upref{FTExp}。
\end{example}

由以上约束条件, 上面出现的所有转换常数中只剩下两个自由的, 例如只要确定 $\beta_x$ 和 $\beta_m$, 剩下的 $\beta$ 也就确定了。

\subsection{Hartree 原子单位}
原子单位最常见的版本是 Hartree 原子单位, 即定义 $\beta_m$ 等于电子的质量, $\beta_x$ 等于玻尔半径(\autoref{eq_BohrMd_3}~\upref{BohrMd}), 再由\autoref{eq_AU_6} 和\autoref{eq_AU_5} 确定 $\beta_E, \beta_t, \beta_\Psi$, 如\autoref{tab_AU_1} 所示\footnote{为了区别能量与电场,以下用 $E$ 表示能量,用 $\mathcal{E}$ 表示电场。} 。 注意许多常数都与氢原子的玻尔模型\upref{BohrMd}(原子核不动)的基态(表中简称基态)有关。 表中还定义了一些其他的物理量的转换常数, 它们的定义可以使以下无单位公式成立(以后我们在不至于混淆的情况下省略角标 $a$)

\begin{table}[ht]
\caption{原子单位转换常数表(参考 “物理学常数\upref{Consts}”)}\label{tab_AU_1}
\begin{tabular}{|c|c|c|c|}
\hline
物理量 & $\beta$ & 描述 & 数值(国际单位)\\
\hline
质量 $m$ & $m_e$ & 电子质量 & $9.10938215\e{-31}\Si{kg}$ \\
\hline
\dfracH 长度 $x$ & $a_0 = \dfrac{4\pi \epsilon_0 \hbar ^2}{m_e e^2}$ & 玻尔半径 & $5.2917721067\e{-11}\Si{m}$ \\
\hline
\dfracH 速度 $v$ & $\dfrac{\hbar}{m_e a_0}$ & 基态电子速度 & $2.1876912633\e6\Si{m/s}$ \\
\hline
时间 $t$ & $m_e a_0^2/\hbar$ & 长度除以速度 & $2.418884326\e{-17}\Si{s}$\\
\hline
\dfracH 角频率 $\omega$ & $\dfrac{\hbar}{m_e a_0^2}$ & 时间的倒数 & $4.134137335\e{16}\Si{s^{-1}}$ \\
\hline
\dfracH 能量 $E$ & $\dfrac{\hbar^2}{m_e a_0^2} = \dfrac{e^2}{4\pi \epsilon_0 a_0}$ & 基态电子势能大小 & $4.3597446499\e{-18}\Si{J}$ \\
\hline
角动量 $L$ & $m_e v_0 a_0 = \hbar$ & 长度乘以动量 & $1.0545718176\e{-34}\Si{kg\cdot m^2/s}$ \\
\hline
电荷 $q$ & $e$ 或 $q_e$ & 电子电荷 & $1.6021766208\e{-19}\Si{C}$\\
\hline
\dfracH 电场强度 $\mathcal{E}$ & $\dfrac{e}{4\pi \epsilon_0 a_0^2}$ & 基态轨道电场强度 & $5.1422067070\e{11}\Si{N/C}$ \\
\hline
\dfracH 磁感应强度 $B$ & $\dfrac{\hbar}{ea_0^2}$ &  & $2.350517567\e5\Si{T}$\\
\hline
\dfracH 电势 $V$ & $\dfrac{e}{4\pi\epsilon_0 a_0}$ & 基态轨道电势 & $27.211386019\Si{J/C}$ \\
\hline
\end{tabular}
\end{table}

常见公式
\begin{align}
&\omega = \frac{2\pi}{T}~,\\
&x = v t~,\\
&\bvec L = m\bvec r \cross \bvec v  \qquad \text{(角动量)}~,\\
&\mathcal{E} = \frac{q}{r^2} \qquad \text{(点电荷电场)}~,\\
&\rho_{E} = \frac{1}{8\pi} (\abs{\bvec{\mathcal{E}}}^2 + \abs{c\bvec B}^2) \qquad \text{(电磁场能量密度)}~,\\
&\bvec F = q \bvec v \cross \bvec B \qquad \text{(洛伦兹力)}~,\\
&U = \frac{q}{r} = \mathcal{E} x \qquad \text{(点电荷电势)}~,\\
&\label{eq_AU_11} V = qU = -q\mathcal{E} x \qquad \text{(匀强电场电势能)}~,\\
&\bvec {\mathcal{E}} = -\grad \varphi - \pdv{t}\bvec A \qquad \text{(标势和矢势)}~,\\
&\bvec B = \curl \bvec A \qquad \text{(标势和矢势)}~,\\
&\bvec s = \bvec{\mathcal{E}} \cross \bvec B/(4\pi\alpha^2) \qquad \text{(坡印廷矢量)}~,\\
\label{eq_AU_12}
&-\frac{1}{2m} \pdv[2]{\Psi}{x} + V\Psi= \I\pdv{\Psi}{t} \qquad \text{(薛定谔方程)}~.
\end{align}
注意当考察对象为电子时, 式中 $m = 1$, 可省略。

\begin{example}{匀强电场中电子的薛定谔方程}
令\autoref{eq_AU_12} 中 $m = 1$, $q = -1$, 再将\autoref{eq_AU_11} 代入, 得
\begin{equation}
-\frac12 \pdv[2]{\Psi}{x} + \mathcal{E} x \Psi= \I\pdv{\Psi}{t}~,
\end{equation}
其中电场 $\mathcal{E}$ 可以是时间的函数。
\end{example}

\begin{exercise}{氢原子的基态能量}
计算原子单位下玻尔模型\upref{BohrMd}中氢原子基态的能量, 假设原子核固定不动(答案:$-1/2$)。
\end{exercise}

\subsubsection{电磁学相关}
把精细结构常数\upref{FinStr}记为 $\alpha$。 国际单位中有 $a_0 = \hbar/(\alpha c m_e)$。 原子单位中有 $c = 1/\alpha$, $\epsilon_0 = 1/(4\pi)$, $\mu_0 = 4\pi\alpha^2$, $\mu_0\epsilon_0 = \alpha^2$。 对平面电磁波 $E = cB$。  

\subsection{其他原子单位}

当问题涉及一主要角频率 $\omega_0$ 的时候(例如研究原子在单频激光中的变化),可选择 $\beta_E = \hbar\omega_0$ 做能量单位。 同样令 $\beta_m$ 等于电子质量, $\beta_q$ 等于元电荷, 由\autoref{eq_AU_6} 得
\begin{equation}\label{eq_AU_15}
\beta_x = \sqrt{\frac{\hbar}{m_e\omega_0}}~,
\qquad
\beta_t = \frac{1}{\omega_0}~.
\end{equation}
为了使\autoref{eq_AU_11} 成立,得电场的转换常数为
\begin{equation}
\beta_\mathcal{E} = \frac{\hbar\omega_0}{e \beta_x}~.
\end{equation}
一种常见的情况是平面电磁波中的电场用国际单位表示为 $\mathcal{E}(t) = \mathcal{E}_0\cos(\omega_0 t)$, 而这种原子单位下为
\begin{equation}
\mathcal{E}(t) = \mathcal{E}_0\cos t~,
\end{equation}
注意右边不含 $\omega$, 形式更简洁。

另一个常见的例子是简谐振子\upref{QSHOxn}, 若使用其振动的固有频率来定义 $\beta_t$, 其薛定谔方程为
\begin{equation}\label{eq_AU_18}
-\frac12 \pdv[2]{\Psi}{x} + \frac12 x^2 \Psi= \I\pdv{\Psi}{t}~,
\end{equation}
能级为
\begin{equation}\label{eq_AU_19}
E_n = \frac12 + n \qquad (n = 0, 1, 2\dots)~,
\end{equation}
归一化的基态波函数为
\begin{equation}
\psi_0(x) = \pi^{-1/4} \E^{-x^2/2}~.
\end{equation}


\subsection{转换为国际单位制}
使用 “无量纲的物理公式\upref{NoUnit}” 中的方法, 容易把原子单位的物理量和公式换回到国际单位, 前提是要明确每个物理量的量纲。

\begin{example}{薛定谔方程}
要将薛定谔方程\autoref{eq_AU_18} 转换为国际单位的公式。 即先把所有无单位的物理量替换成有单位的物理量除以对应的 $\beta$ 常数, 得(两边已同乘 $\beta_\Psi$)
\begin{equation}
-\beta_x^2\frac12 \pdv[2]{\Psi}{x} + \frac{1}{\beta_x^2}\frac12 x^2 \Psi= \beta_t\I\pdv{\Psi}{t}~.
\end{equation}
将\autoref{eq_AU_15} 代入, 两边乘以 $\omega\hbar$ 得国际单位下的简谐振子薛定谔方程
\begin{equation}
-\frac{\hbar^2}{2m} \pdv[2]{\Psi}{x} + \frac12 m\omega^2 x^2 \Psi= \I\hbar\pdv{\Psi}{t}~.
\end{equation}
类似地, 也可以将\autoref{eq_AU_19} 变为
\begin{equation}
E =  \qty(\frac12 + n)\beta_E = \qty(\frac12 + n)\omega\hbar \qquad (n = 0, 1, 2\dots)~.
\end{equation}
\end{example}
