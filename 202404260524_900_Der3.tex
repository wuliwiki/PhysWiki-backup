% 导数的运算法则
% keys 导数|导函数|数学分析|differential|calculus
% license Xiao
% type Tutor

\pentry{导数(简明微积分)\nref{nod_Der},导数(数学分析)\nref{nod_Der2}}{nod_67b0}

若函数 $f(x),g(x)$ 在 $x_0$ 处可导,那么下列各式在 $x_0$ 处成立,

\begin{enumerate}
\item 
\begin{equation}
(f\pm g)'(x)=f'(x)\pm g'(x)~.
\end{equation}
\item 
\begin{equation}
(f\cdot g)'(x)=f'(x)\cdot g(x)+f(x)\cdot g'(x)~.
\end{equation}
\item 
\begin{equation}
\qty(\frac{f}{g})'(x)=\frac{f'(x)g(x)-f(x)g'(x))}{g(x)^2}~.
\end{equation}
\item \textbf{复合函数求导法则}:若 $f(x)$ 在 $u_0$ 处可导,$g(x)$ 在 $x_0$ 处可导,且 $g(x_0)=u_0$。那么 
\begin{equation}
(f\circ g)'(x_0)=f'(g(x_0))g'(x_0)~,
\end{equation}
许多物理书喜欢将它简记为 $\frac{\dd f}{\dd x}=\frac{\dd f}{\dd g} \cdot \frac{\dd g}{\dd x}$。
\end{enumerate}


一些实用的求导公式(下面的一些公式需要在满足一定条件下才成立,例如 $(x^n)'=nx^{n-1}$ 中,若 $n=1/2$,则 $x$ 不能取负数。):
\begin{enumerate}
\item $(x^n)'=nx^{n-1}$。
\item $(a^x)'=a^x\ln a$。
\item $(\ln x)'=1/x$。
\item $(\sin x)'=\cos x,(\cos x)'=-\sin x$。\item $(\sin(\omega x+\phi))''=-\omega^2\sin(\omega x+\phi)$\footnote{这个公式就是简谐振动的运动方程。}。
\item $(\tan x)'=1/\cos^2 x,(\cot x)'=-1/\sin^2 x$。
\item $(\arctan x)'=1/(1+x^2)$。
\end{enumerate}
\begin{exercise}{}
设函数 $f(x)$ 在 $\mathbb{R}$ 上严格单调,其反函数为 $g(y)=f^{-1}(y)$。已知 $f(x)$ 在 $\mathbb{R}$ 上二阶可导,给定它的一阶导函数和二阶导函数,且一阶导数处处不为 $0$,求 $g'(y),g''(y)$ ($x=f^{-1}(y)$ 在其定义域范围内存在)。

提示:观察图像一点切线的斜率容易知道 $g'(y)=1/f'(x)$(其中 $x=f^{-1}(y)$),严格证明可以利用导数定义或函数在一点附近的一阶近似公式。但二阶导不是那么容易求,逐步推导过程如下:
\begin{equation}
\begin{aligned}
\frac{\dd g'(y)}{\dd y}&=\frac{\dd}{\dd y}\left(\frac{1}{f'(x)}\right)=\frac{-1}{f'(x)^2}\frac{\dd f'(x)}{\dd y}=\frac{-1}{f'(x)^2}\frac{\dd x}{\dd y}\frac{\dd f'(x)}{\dd x}\\
&=\frac{-1}{f'(x)^2}\frac{1}{f'(x)}\frac{\dd f'(x)}{\dd x}=-\frac{f''(x)}{f'(x)^3}~.
\end{aligned}
\end{equation}
这个问题有一个更有启发意义的解法。如果我们写出 $f(x)$ 和 $g(y)$ 的二阶近似公式:
\begin{equation}
\begin{aligned}
f(x+h)=f(x)+f'(x)h+f''(x)h^2+o(h^2)~,\\
g(y+\eta)=g(y)+g'(y)\eta+g''(y)\eta^2+o(\eta^2)~.
\end{aligned}
\end{equation}
根据 $f$ 和 $g$ 互为反函数的关系,我们有 $y=f(x),x=g(y)$。不妨令上式中 $y+\eta=f(x+h),x+h=g(y+\eta)$,则 $\eta$ 可以用 $f'(x),f''(x)$ 的表达式代替。代入整理可得:
\begin{equation}
\begin{aligned}
x+h&=x+g'(y)(f'(x)h+f''(x)h^2+o(h^2))\\
&\ +g''(y)(f'(x)h+f''(x)h^2+o(h^2))^2+o(h^2)~,\\
x+h&=x+g'(y)f'(x)h + (g'(y)f''(x)+g''(y)f'(x)^2)h^2+o(h^2)~.
\end{aligned}
\end{equation}
$h^n$ 项系数要相同,可以列出两个等式,解得 $g'(y)=1/f'(x),g''(y)=-f''(x)/f'(x)^3$。这个方法可以推广到求 $g(y)$ 的 $n$ 阶导,更重要的是它提供了一个系统的计算方法,可以编写程序设计相应的算法进行计算。
\end{exercise}
