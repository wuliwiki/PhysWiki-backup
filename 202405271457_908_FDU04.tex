% 复旦大学 2004 硕士研究生入学考试试题
% license Usr
% type Note

\textbf{声明}:“该内容来源于网络公开资料,不保证真实性,如有侵权请联系管理员”


(1) 质量为 $m$ 的粒子处在宽度为 $a$ 的一维无限深势阱中,设在时刻 $t=0$ 粒子的状态为 $\Phi(0) = a_1 \varphi_1 + a_2 \varphi_2 + a_3 \varphi_3 + a_4 \varphi_4$,$\varphi_i (i=1,2,3,4)$ 是能量为 $E_i$ 时一维无限深势阱的归一化本征函数,$a_1, a_2, a_3, a_4$ 是已知的常数,求:

   \begin{enumerate}
      \item 在时刻 $t=0$ 时,测量能量,结果小于 $3 \pi^2 \hbar^2 / ma^2$ 的几率
      \item 在时刻 $t=0$ 时,能量 $E$ 和 $E^2$ 的平均值
      \item 时刻为 $t$ 时的波函数 $\Phi(t)$
      \item 如果在 $\Phi$ 态测量能量,所得结果为 $8 \pi^2 \hbar^2 / ma^2$,问测量后粒子处在何种状态?
    \end{enumerate}

   (2)  设氢原子处在 $R_{21} Y_{1,-1}$ 态,求:

    \begin{enumerate}
        \item 势能 $V = -e^2 / r$ 的平均值
        \item $\mathbf{L}$ 为轨道角动量,求符号 $\langle \mathbf{L}, \mathbf{L}^2, \mathbf{L}_z \rangle$ 的平均值 $\langle L, L^2, L_z^2 \rangle$
    \end{enumerate}


已知 $R_{21} = \frac{1}{2 \sqrt{6 a_0}} r e^{-r / 2a_0}, Y_{1,-1} = \sqrt{\frac{3}{8 \pi}} \sin \theta e^{-i \varphi}, a_0$ 为波尔半径

\end{document}
