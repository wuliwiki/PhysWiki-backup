% 南京航空航天大学 2014 量子真题
% license Usr
% type Note

\textbf{声明}:“该内容来源于网络公开资料,不保证真实性,如有侵权请联系管理员”

\subsection{简答题 (本题 45 分,每小题 15 分)}
①写出氢原子、一维简谐振子、一维无限深势阱的能级,并用示意图表示。

②证明:定态波函数 $\psi(x)$总可以取作实数的。

③能量本征态有可能是角动量 $\hat{L}^2$ 的本征态吗?有可能是 $\hat{L}_z$ 的本征态吗?请回答为什么
并举例说明。

\subsection{二}
在一维无限深势阱中,一个粒子的初始波函数由前两个定态迭加而成:$\psi(x,0)=A[\psi_1 (x)
+\psi_2 (x)]$。为了简化计算可令 $\omega=\pi^2
①归一化 Ψ(x,0),并求 Ψ(x,t)和|Ψ(x,t)| 2,把后者用时间的正弦函数展开。

②计算〈x〉、〈p〉的值。它们是随时间振荡的,角频率是多少?振幅是多少?

③测量粒子的能量,可能得到什么值?得到各个值的几率是多少?求出 Hˆ 的期望值。并
与 E1 和 E2 比较。(本题 20 分)