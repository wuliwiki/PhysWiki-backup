% 天文学笔记(科普)
% license Xiao
% type Note

\begin{issues}
\issueDraft
\end{issues}

\pentry{经典力学笔记(科普)\nref{nod_CMInt}}{nod_ff52}

\begin{itemize}
\item 太阳系\textbf{八大行星}: 水星(Mercury)、金星(Venus)、地球(Earth)、火星(Mars)、木星(Jupiter)、土星(Saturn)、天王星(Uranus)、海王星(Neptune)。
\item 太阳的质量占太阳系总质量的 99.86\%。
\item 八大行星和小行星都近似在\textbf{黄道平面}上运动。
\item \textbf{天文单位}: 约等于地球到太阳的距离(光走八分钟)。
\item \textbf{光年}: 光走一年的长度
\item 夜空中的亮点绝大部分都是恒星。 一些例外包括八大行星、坠入大气的陨石(流星)、 人造卫星以及飞行器。
\item \textbf{主小行星带}: 介于火星和木星之间。
\item \textbf{柯伊伯带}: 海王星之外。
\item \textbf{奥尔特星云}: 太阳系的最外层。
\item 离太阳最近的恒星是(), 距离太阳约 4 光年, 所以我们看到的是它 4 年前的样子。 其他恒星也同理。
\item 天文台一般建在山上, 为了避免大气干扰。
\item \textbf{哈勃望远镜}是一个卫星, 围绕地球旋转, 不受大气干扰。
\end{itemize}
