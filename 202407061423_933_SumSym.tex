% 求和符号(高中)
% license Usr
% type Tutor

\begin{issues}
\issueDraft
\end{issues}

\pentry{数列\nref{nod_HsSeFu}}{nod_571f}

想象一下这样一个场景,老师要求你计算从1到100的所有自然数的和,这时,你旁边的一个名叫高斯的同学脱口而出5050。当然,这个故事里,你不是高斯,而是他旁边那个看起来笨笨的同学。你和其他的同学一样,从$1$开始一个一个往上加。当然,这篇文章的内容不是介绍如何快速的计算这个和的,否则你就应该附身到高斯身上了。

时过境迁,对于计算机随处可见的现在,这样的操作不再像当时那样看起来笨笨的,计算机每天都在进行着类似的操作。不仅如此,通过对这个过程抽象得到的求和符号,不仅能只用一个比较简短的记号就描述之前你的求和过程,在理论上提供了强大的表达能力,还能够推广到远比高斯当时使用的方法广泛的场景中,展现了高效和便捷的特性。

当然,这个符号与之前学习过的其他符号($+,\times,!$等)相比都复杂一些,但了解熟练之后,相信你再看到它,也会像看到老朋友那样自然。

\subsection{初次接触求和符号}

假设有人给了你一项任务,他希望你求出一些数字的和是多少(这个过程的名字叫\textbf{累加}),想一想这个过程,你会需要他给你哪些信息呢?

你需要知道第一个数是多少,并且要他依次把这些数字给你,而且最终需要有一个停止的条件,即这些数字会最终停止在某一个数字上。对于已经接触过数列的同学一定意识到了,这些数字可以构成一个数列。第一个数就是首项,最后一个数就是末项。下面假设这个数列是$\{a_n\}$,它的第一项是$a_1$,末项是$a_n$。于是这个求和结果就是:

\begin{equation}
a_1+a_2+\cdots+a_{n-1}+a_{n}.~
\end{equation}

为了避免每次都要写一堆点来表达这个求和过程,通常会设这个数列的所有项的和是$S$,即:

\begin{equation}
S=a_1+a_2+\cdots+a_{n-1}+a_{n}.~
\end{equation}

但是,尽管使用$S$避免了写一堆点,但却带来了一些其他的问题:
\begin{enumerate}
\item 如果涉及到很多个数列或者很多个求和过程,光是设这些变量,就要花掉很大篇幅,就像上面做的:需要构造一个数列,给这个数列一个符号,然后给这个数列的和一个符号。
\item $S$这个符号太泛用了,在使用的时候也会因为记不清之前都用了哪些符号造成阅读上的困难。
\item $S$这个符号本身并没有包含其他的信息,他只是给这个过程起了一个名字而已,从这个字符没有办法反映出求和过程的性质,任何记号都可以替代它,毕竟只是个名字。
\end{enumerate}

由于累加过于常用,上面的三个问题就显得尤为突出,因此,迫切需要一个新的手段来避免出现这些问题,数学家们使用求和符号来表现累加的过程:

\begin{equation}
\sum_{i=1}^n a_i=a_1+a_2+\cdots+a_{n-1}+a_{n}~.
\end{equation}

对照着刚才的那些问题,观察这个记号,可以发现,用它来代替S就完全解决了那些问题,同时,

\subsection{求和符号的定义}

求和符号,定义为
\begin{equation}
\sum_{i=m}^n a_i = a_m + a_{m+1} + \dots + a_n~.
\end{equation}
其中 $i$ 叫做求和指标。 为了区分不同指标也会经常使用 $j,k,l,m,n,p,q$ 等字母。

许多时候,如果已经在语境中明确了求和中 $i$ 取哪些值, 为了方便就可以直接写作 $\sum\limits_i a_i$。 在印刷排版中,行内的求和符号也经常写成 $\sum_{i=m}^n a_i$,为了保持行高。 小时百科仍然会尽量使用 $\sum\limits_{i=m}^n a_i$。
\subsection{运算技巧}

\subsubsection{乘法}
\begin{equation}
\sum_i C a_i = C\sum_i a_i~.
\end{equation}

\begin{equation}
\qty(\sum_i a_i) \qty(\sum_j b_j) = \sum_{i,j} a_i b_j = \sum_i \qty(a_i \sum_j b_j) = \sum_j \qty(b_j \sum_i a_i)~.
\end{equation}
其中 $\sum\limits_{i,j}$ 表示把所有不同的 $i,j$ 的组合都遍历一次,顺序任意。

\begin{equation}
\qty(\sum_i a_i)^2 = \sum_{i,j} a_i a_j = \sum_i a_i^2 + 2\sum_{i<j} a_i a_j~.
\end{equation}
可见在求和的相乘中,区分求和指标很重要,如果写成 $\sum\limits_{i,i} a_i a_i$ 将产生混乱。


\subsubsection{指标换元}
例如要把指标替换为 $j=i+1$,则
\begin{equation}
\sum_{i=m}^n a_i = \sum_{j=m+1}^{n+1} a_{j-1} ~.
\end{equation}
计算方法是,先用 $j$ 表示 $i$ 得 $i=j-1$, 然后求和号内部的 $i$ 可以全部代入 $j-1$。 对于上下标,也可以直接把 $j-1$ 代入并移项,例如下标代入后得 $j-1=m$,即 $j=m+1$。


\subsection{使用求和符号的好处}

求和符号有下面几点明确的好处:

\begin{enumerate}
\item 提高表达的简洁性:使用求和符号可以大幅减少公式的长度和复杂度,便于书写和理解。
\item 快速化简:在熟练掌握求和符号的性质和技巧后,可以快速地对求和过程进行变形。
\item 注重通项公式的表达:求和符号通过明确的通项公式,着眼于求和这个过程更本质的内容,方便直观地理解和分析求和过程。
\item 适用范围广泛:求和符号不仅可以表达连续自然数的求和,还适用于包括等差数列、等比数列等在内的各种数列的求和情况,它甚至可以推广到无穷级数。
\item 便于计算机实现:求和符号与编程语言中的循环语句的条件一一对应,这使得计算过程可以容易转换为程序并通过计算机完成。
\item 与定积分的紧密联系:求和符号与定积分有着密切的关系,特别是在离散与连续的彼此转换中,它们之间可以快速相互转化。
\end{enumerate}

相信认真阅读完本页内容的你,已经感受到了前四点。其他的内容或许对高中的你而言会觉得不知所谓,但相信随着学习的深入,你会一点一点感受到这些好处,并受用终身。