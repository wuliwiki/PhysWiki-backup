% 双缝实验
% license CCBYSA3
% type Wiki

(本文根据 CC-BY-SA 协议转载自原搜狗科学百科对英文维基百科的翻译)

双缝实验(double-slit experiment,或称双狭缝实验)是一种演示光子或电子等等微观物体的波动性与粒子性的实验。

在现代物理学中,双缝实验证明光和物质可以显示经典的波和粒子的特性;此外,它显示了量子力学现象的基本概率性质。这个实验是托马斯·杨在1801年首次用光进行的。1927年,戴维孙和格默证明了电子也有相同的性质,这种性质后来扩展到原子和分子。早在量子力学和波粒二象性概念出现之前,托马斯·杨的光实验就是经典物理学的一部分。他认为这证明了光波理论是正确的,他的实验有时被称为杨氏实验[1] 或杨氏狭缝。

该实验属于一类普通的“双路径”实验,其中一个波被分成两个独立的波,然后合并成一个波。两种波的路径长度的变化会导致相移,从而产生干涉图样。另一个版本是马赫-曾德尔干涉仪,它用镜子分割光束。

在这个实验的基本版本中,相干光源,例如激光束,照亮有两个平行狭缝的平板,穿过狭缝的光在平板后面的屏幕上被观察到。[2] 光的波动性质导致穿过两个狭缝的光波发生干涉,在屏幕上产生亮带和暗带——如果光是由经典粒子组成的话,这种结果是不可能出现的。[2][3] 然而,人们总是发现光在屏幕上的离散点被吸收,作为单个粒子(而不是波)打在屏幕上,通过粒子密度的变化会显示出干涉图案。[4] 此外,包括狭缝处探测器的实验版本发现,每个探测到的光子穿过一个狭缝(就像经典粒子一样),而不是穿过两个狭缝(就像波一样)。[5][6][7][8][9] 然而,这些实验证明,如果检测到粒子穿过哪个狭缝,它们就不会形成干涉图样。这些结果证明了波粒二象性原理。[10][11]

当向双缝发射时,发现其他原子级实体,如电子,表现出相同的行为。 此外,个别离散撞击的探测被观察到具有内禀的概率性,这用经典力学是无法解释的。

这个实验可以用比电子和光子大得多的实体来完成,尽管随着尺寸的增加会变得更加困难。进行双缝实验的最大实体是每个包含810个原子的分子(其总质量超过10,000个原子质量单位)。

双缝实验(及其变体)已成为经典的思想实验,因为它清晰地表达了量子力学的核心难题。因为它证明了观察者预测实验结果能力的根本局限性,理查德·费曼称之为“一种无法用任何经典方式解释的现象,这种现象蕴含着量子力学的核心。事实上,它包含了量子力学中唯一的神秘。

\subsection{概观}
\begin{figure}[ht]
\centering
\includegraphics[width=8cm]{./figures/423775cb916e2722.png}
\caption{相同的双缝配置(缝间0.7 mm);在上图中,一条狭缝是闭合的。在单缝图像中,由于狭缝的非零宽度,形成了衍射图样(主带两侧的微弱点)。在双缝图像中也可以看到衍射图样,但其强度是单缝图像的两倍,并增加了许多较小的干涉条纹。} \label{fig_SFSY_1}
\end{figure}
如果光严格地由普通粒子或经典粒子组成,这些粒子通过狭缝以直线发射,并被允许照射到另一侧的屏幕上,我们将会看到与狭缝的大小和形状相对应的图案。然而,当实际进行这个“单缝实验”时,屏幕上的图案是光扩散的衍射图案。狭缝越小,扩散角度越大。图像的顶部显示了当红色激光照射狭缝时形成的图案的中心部分,如果仔细观察,还有两个微弱的边带。使用更精细的仪器可以看到更多的波段。这可以用衍射来解释,即图案是光波从狭缝干涉的结果。
\begin{figure}[ht]
\centering
\includegraphics[width=8cm]{./figures/0726aeb0a385322b.png}
\caption{粒子波函数的模拟:双缝实验。白色的模糊表示粒子。像素越白,则在如果测量到那个地方找到一个粒子的概率就越大。} \label{fig_SFSY_2}
\end{figure}
如果一个人照亮两个平行的狭缝,来自两个狭缝的光再次干涉。这里的干涉是一个更明显的模式,有一系列交替的亮带和暗带。波段的宽度是照明光频率的一个特性。[12] (见右边底部的照片。)当托马斯·杨(1773-1829)首次证明这一现象时,它表明光由波组成,因为亮度的分布可以用波前的交替加法和减法干涉来解释。[13] 杨的实验在19世纪初进行,在人们接纳光波理论方面发挥了重要作用,击败了艾萨克·牛顿提出的光微粒学说,后者是17和18世纪公认的光传播模型。然而,后来光电效应的发现表明,在不同的情况下,光可以表现得好像是由离散的粒子组成的。这些看似矛盾的发现使得有必要超越经典物理学,将光的量子性质考虑在内。

费曼喜欢说所有的量子力学都可以通过仔细思考这个单一实验的含义而得到。[13] 他还提出(作为思想实验),如果探测器放置在每个狭缝之前,干涉图样就会消失。[14]

恩格尔特-格林伯格对偶关系提供了量子力学背景下双缝干涉数学的详细处理。

泰勒于1909年首次进行了低强度双缝实验,[15] 方法是降低入射光的水平,直到光子发射/吸收事件基本不重叠。直到1961年,图宾根大学的克劳斯·约翰逊用电子束进行了双缝实验,才使用光以外的东西进行实验。[16][17] 1974年,意大利物理学家皮尔·乔治·梅里(Pier Giorgio Merli)、吉安·弗朗哥·米西罗利(Gian Franco Missiroli)和朱利奥·波齐(Giulio Pozzi)用单电子和双棱镜(而不是狭缝)重复了这个实验,表明每一个电子都会像量子理论预测的那样干涉自己。[18][19] 2002年,单电子版本的实验被《物理世界》的读者评为“最美丽的实验”。[20]

2012年,斯特凡诺·弗拉博尼(Stefano Frabboni)及其同事最终按照费曼提出的原始方案,用电子和真实狭缝进行了双缝实验。他们将单个电子发送到纳米技术制造的狭缝(约100纳米宽)上,通过用单电子探测器收集传输的电子,他们可以显示双缝干涉图样的形成。[21]

\subsection{实验者的变化}
\subsubsection{2.1 单粒子干涉}
\begin{figure}[ht]
\centering
\includegraphics[width=6cm]{./figures/c2906c47e7124052.png}
\caption{电子随时间的积聚} \label{fig_SFSY_3}
\end{figure}
单粒子干涉这个实验的一个重要版本涉及单个粒子(或波——为了一致性,这里称它们为粒子)。如预期的那样,通过双缝设备一次发送一个粒子会导致单个粒子出现在屏幕上。然而,值得注意的是,当这些粒子被允许一个接一个地积累起来时,干涉图样就出现了(见相邻的图像)。这证明了波粒二象性,即所有物质都表现出波粒两种性质:粒子在单一位置被测量为单一脉冲,而波描述了在屏幕上特定位置吸收粒子的概率。[22] 这种现象已经被证明发生在光子、电子、原子甚至包括布基球在内的一些分子身上。[23][24][25][26][27] 因此,电子实验为电子、质子、中子、甚至更大的通常被称为粒子的实体都有自己的波动性质甚至波长(与其动量相关)的观点提供了确证。

被探测到的概率是波振幅的平方,可以用经典波计算(见下文)。粒子不会以可预测的顺序到达屏幕,所以知道所有先前的粒子出现在屏幕上的什么位置以及以什么顺序出现并不能说明未来的粒子将在哪里被检测到。[28] 如果波在某一点被抵消,那并不意味着粒子消失了;它会出现在其他地方。自从量子力学诞生以来,一些理论家就一直在寻找方法来整合额外的决定因素或“隐变量”,如果这些因素或“隐变量”变得为人所知,就能够解释每个个体与目标碰撞的位置。[29]

涉及两个或更多叠加粒子的更复杂的系统不适用上述解释。[30]

\subsubsection{2.2 “双向”实验与互补原则}
一个众所周知的思想实验预测,如果粒子探测器位于狭缝处,显示光子穿过哪个狭缝,干涉图案将消失。[31] 这个单向实验说明了光子既可以作为粒子也可以作为波的互补原理,但是不能同时作为粒子和波来观察。[31][32][33]尽管这个思想实验在量子力学历史上很重要(例如,参见关于爱因斯坦版本实验的讨论),但是直到20世纪70年代才提出这个实验在技术上可行的实现。[34](教科书盖丹肯实验的简单的实现是不可能的,因为没有吸收光子就无法探测到光子。)目前,已经进行了多个实验来说明互补性的各个方面。[35]

1987年 [36][37] 进行的一项实验产生的结果表明,在不完全破坏干涉的情况下,可以知道粒子会选择哪条路径。这表明测量的效果对传输中粒子的干涉程度较小,从而仅在一定程度上影响了干涉图样。换句话说,如果人们不坚持确定光子到底穿过了哪个裂缝,人们仍然可以检测到(退化的)干涉图案。[38]

\subsubsection{2.3 延迟选择和量子擦除变体}
\begin{figure}[ht]
\centering
\includegraphics[width=8cm]{./figures/8e77efff9ca84ce5.png}
\caption{惠勒的延迟选择实验图,展示了光子通过狭缝后确定路径的原理。} \label{fig_SFSY_4}
\end{figure}
惠勒的延迟选择实验表明,在粒子穿过狭缝后提取“哪条路径”信息似乎可以追溯性地改变其先前在狭缝的行为。

量子擦除实验证明,通过擦除或以其他方式使“哪条路径”信息永久不可用,可以恢复波的行为。

《科学美国人》的一篇文章给出了量子擦除现象的一个简单的能够在家做的例子。[39]如果在每个狭缝之前设置偏振器,使其轴相互垂直,干涉图案将被消除。偏振器可以被认为是向每个光束引入了哪条路径的信息。在检测器前面引入第三个偏振器,其轴相对于其他偏振器为45°,从而“擦除”该信息,干涉图案则会重新出现。[39] 这也可以通过将光视为经典波,以及使用圆偏振器和单光子来解决。[40] 使用纠缠光子对的偏振器的实现没有经典的解释。[40]

\subsubsection{2.4 弱测量}
在2012年一次广为人知的实验中,研究人员声称已经确定了每个粒子所走的路径,而对粒子产生的干涉模式没有任何负面影响。[41] 为了做到这一点,他们使用了一种设置,使得到达屏幕的粒子不是来自点状源,而是来自具有两个强度最大值的源。然而,像斯文森这样的评论家指出,[42] 实际上在双缝实验的这个变体中形成的弱测量和海森堡测不准原理之间没有冲突。弱测量后选择不允许同时测量每个粒子的位置和动量,而是允许测量到达不同位置的粒子的平均轨迹。换句话说,实验者正在创建一个完整轨迹景观的统计地图。[42]

\subsubsection{2.5 其他变体}
\begin{figure}[ht]
\centering
\includegraphics[width=6cm]{./figures/e26af4767cb89ff5.png}
\caption{实验室双缝配置; 顶部支柱之间的距离约2.5厘米(1英寸)。} \label{fig_SFSY_5}
\end{figure}
1967年,戈尔和曼德尔用两个独立的激光器作为光源证明了两个光源的干涉。[43][44]

1972年的实验表明,在一个双缝系统中,任何时候只有一个缝是开着的,但只要路径不同,探测到的光子可能来自任何一个缝,就可以观察到干涉。实验条件使得系统中的光子密度远小于1。[45][46]

1999年,双缝实验成功地用布基球分子(每个分子包含60个碳原子)进行。[24][47] 布基球足够大(直径约0.7纳米,比质子大近50万倍),可以在电子显微镜下看到。

在2005年,埃利提出了一个薄金属屏幕的光传输的实验和理论研究,该薄金属屏幕被两个亚波长狭缝穿孔,被许多光波长分开。上述实验显示远场双缝图案的总强度作为入射光束波长的函数而减小或增强。[48]

2012年,内布拉斯加大学林肯分校的研究人员用理查德·费曼描述的电子进行了双缝实验,使用了新的仪器来控制双缝的传输和监控单电子探测事件。电子被电子枪发射,穿过一个或两个62纳米宽×4微米高的狭缝。[49]

2013年,双缝实验成功完成,每个分子包含810个原子(总质量超过10,000个原子质量单位)。[50][50]
\begin{figure}[ht]
\centering
\includegraphics[width=8cm]{./figures/7645d105e9f13e17.png}
\caption{等宽(A)和非等宽(B)等离子体狭缝的近场强度分布规律。} \label{fig_SFSY_6}
\end{figure}

\textbf{流体动力先导波类似物}

流体动力学类似物已经被开发出来,可以重建量子力学系统的各个方面,包括通过双缝的单粒子干涉。[50] 一滴硅油沿着液体表面弹跳,通过与其自身波场的共振相互作用自推进。液滴每次反弹都会轻轻搅动液体。与此同时,过去反弹的涟漪影响了它的进程。液滴与其自身波纹的相互作用形成了所谓的先导波,导致它表现出以前被认为是基本粒子特有的行为——包括通常被认为是基本粒子像波一样在空间中传播而没有任何特定位置的行为,直到它们被测量出。[51][52]

通过这个流体动力学先导波系统模拟的行为包括量子单粒子衍射、[53] 量子轨道、轨道能级分裂、自旋和多模态统计。也可以推断不确定关系和不相容原理。视频显示了该系统的各种功能。

然而,包含两个或更多叠加粒子的更复杂的系统不适合如此简单、经典直观的解释。[30] 因此,还没有开发出纠缠的流体动力学模拟。[50] 然而,光学类似物是可能的。[54]

\subsection{经典波动光学公式}
\begin{figure}[ht]
\centering
\includegraphics[width=8cm]{./figures/f12de47e626543d1.png}
\caption{由平面波产生的双缝衍射图样。} \label{fig_SFSY_7}
\end{figure}
许多光的行为可以用经典波动理论来模拟。惠更斯-菲涅耳原理就是这样一个模型;它指出波前上的每个点产生一个次级小波,并且在任何随后的点上的扰动可以通过对该点上各个小波的贡献求和来找到。这种求和需要考虑各个小波的相位和幅度。只能测量光场的强度——它与振幅的平方成正比。

在双缝实验中,两个狭缝被一束激光照射。如果狭缝的宽度足够小(小于激光的波长),狭缝将光衍射成圆柱形波。这两个圆柱形波阵面是叠加的,并且在组合波阵面的任何点上的振幅以及强度都取决于这两个波阵面的幅度和相位。两个波之间的相位差由两个波传播的距离差决定。

如果观察距离与狭缝的间距(远场)相比很大,则可以使用下图中所示的几何形状找到相位差。以$\theta$角传播的两个波之间的路径差由下式给出:
$$d \sin\theta\approx d\theta~$$
其中$d$是两个狭缝之间的距离。当两个波同相时,即路径差等于波长的整数倍,总振幅,以及总强度最大,当它们反相时,即路径差等于半个波长、一个半波长等,则两个波抵消,并且总强度为零。这种效应被称为干涉。干涉条纹最大值以一定角度出现

~d\theta_{n}=n\lambda,~n=0,1,2,\ldots

其中$\lambda$是光的波长。条纹的角间距$\theta f$由下式给出
$$\theta_f\approx \lambda/d~$$
距狭缝距离z处的条纹间距由下式给出

~w=z\theta_{f}=z\lambda/d

例如,如果两个狭缝相隔0.5毫米($d$),并用0.6μm波长的激光($\lambda$)照射,那么在1米($z$)的距离处,条纹的间距将为1.2毫米。

如果狭缝的宽度$b$大于波长,夫琅和费衍射方程给出衍射光的强度如下:[55]

$I(\theta) \propto \cos^2 \left( \frac{\pi d \sin \theta}{\lambda} \right)$~\mathrm{sinc}^{2}\left[{\frac{\pi b\sin \theta}{\lambda}}\right]

其中sinc函数定义为sinc(x) = sin(x)/x 且x ≠ 0,sinc(0) = 1。

这在上图中示出,其中第一图案是单个狭缝的衍射图案,由该方程中的sinc函数给出,第二图示出从两个狭缝衍射的光的组合强度,其中cos函数表示精细结构,较粗糙的结构表示单个狭缝的衍射,由sinc函数描述。

使用菲涅耳衍射方程可以对近场进行类似的计算。随着观察平面越来越接近狭缝所在的平面,与每个狭缝相关联的衍射图案的尺寸减小,从而发生干涉的区域减小,并且当两个衍射图案没有重叠时,干涉区域可能完全消失。[56]

\subsection{对实验的解释}
像薛定谔的猫思维实验一样,双缝实验经常被用来强调量子力学各种解释之间的异同。
\subsubsection{4.1 哥本哈根诠释}
量子力学领域的一些先驱提出了哥本哈根诠释,它断言,除了数学公式、物理仪器的种类和能够使我们能够获得一些关于原子尺度上发生的事情的反应之外,任何东西都是不可获得的。使实验者能够非常准确地预测某些实验结果的数学结构之一有时被称为概率波。在数学形式上,它类似于物理波的描述,但是它的“波峰”和“波谷”表示在普通人类经验的宏观世界中可以观察到的某些现象(例如,探测器屏幕上某一点上的电火花)发生的概率水平。

概率“波”可以说能“穿过空间”,因为从它的数学表示中可以计算出的概率值取决于时间。人们不能仅仅因为要说某个东西在某个时间位于某个地方,就说它在发射和探测之间的位置,比如光子。对最终出现干涉图案的要求是发射粒子,并且有一个屏幕,该屏幕具有至少两条粒子从发射器到检测屏幕的不同路径。从粒子发射到到达检测屏幕之间,实验没有观察到任何东西。如果接下来进行光线跟踪,就像光波(如经典物理学中所理解的)足够宽以走两条路径一样,那么当许多粒子穿过该设备并逐渐“描绘”预期的干涉图案时,光线跟踪将准确预测探测器屏幕上最大值和最小值的出现。
\subsubsection{4.2 路径积分公式}
\begin{figure}[ht]
\centering
\includegraphics[width=6cm]{./figures/0017d8bfd1f6563c.png}
\caption{在费曼路径积分中使用的无穷多个等可能路径之一(参见:维纳过程)} \label{fig_SFSY_8}
\end{figure}
哥本哈根诠释类似于费曼提供的量子力学的路径积分公式。路径积分公式用所有可能轨迹的总和取代了系统单一唯一轨迹的经典概念。通过使用泛函积分将轨迹相加。

每条路径被认为是同等可能的,因此贡献了相同的数量。然而,该贡献在路径上任何给定点的相位由路径上的作用量决定:
$$\mathbf{A}_{\text{path}}(x, y, z, t) = e^{i S(x, y, z, t)}~$$
然后将所有这些贡献相加,并将最终结果的大小平方,得到粒子位置的概率分布:
p(x, y, z, t) \propto \left| \int_{\text{all paths}} e^{iS(x, y, z, t)} \right|^2

和计算概率时的情况一样,结果必须归一化,方法是:

\iiint_{\text{all space}} p(x, y, z, t) \, dV = 1
总的来说,结果的概率分布是从原点到终点的所有路径上,与沿着每条路径的作用量成比例传播的波的叠加的模的归一化平方。沿着不同路径的累积作用的差异(以及贡献的相对相位)产生了由双缝实验观察到的干涉图案。费曼强调,他的表述仅仅是一种数学描述,而不是试图描述一个我们可以测量的真实过程。
\subsubsection{4.3 相关解释}
\begin{figure}[ht]
\centering
\includegraphics[width=10cm]{./figures/f40d4f18b2bbea6a.png}
\caption{不确定原理与关系解释的一个例子。对粒子位置的了解越多,对速度的了解就越少,反之亦然。} \label{fig_SFSY_9}
\end{figure}
根据量子力学的相关解释,最早由卡尔罗·罗威利提出,[57] 观察(如双缝实验中的观察)是由观察者(测量装置)和被观察物体(物理上相互作用)之间的相互作用产生的,而不是物体所拥有的任何绝对性质。在电子的情况下,如果它最初是在特定的狭缝处被“观察”的,那么观察者-粒子(光子-电子)的相互作用包括关于电子位置的信息。这部分限制了粒子在屏幕上的最终位置。如果它不是在特定的狭缝而是在屏幕上被“观察”(用光子测量),那么就没有“哪条路径”信息作为交互作用的一部分,所以电子在屏幕上的“观察”位置严格由其概率函数决定。这使得屏幕上产生的图案与每个电子穿过两个狭缝时一样。也有人提出,空间和距离本身是相关的,一个电子可以“同时出现在两个地方”——例如,双缝——两条缝的位置到屏幕上特定位置的空间关系是等价的。[58]
\subsubsection{4.4 多世界解释}
物理学家大卫·多伊奇在其著作《现实的结构》中认为,双缝实验是多世界解释的证据。然而,由于量子力学的每一种解释在经验上都无法区分,一些科学家对这一说法持怀疑态度。
\subsubsection{4.5 德布罗意-博姆理论}
德布罗意-博姆理论是对量子力学标准理解的替代,它指出粒子在任何时候都有精确的位置,它们的速度受波函数的影响。因此,虽然在双缝实验中,单个粒子将穿过一个特定的缝隙,但影响它的所谓“先导波”将穿过两个缝隙。德布罗意-博姆轨道的两条狭缝最初是由克里斯·德维尼在比尔贝克学院(伦敦)与克里斯·菲利皮迪斯和巴兹尔·希利合作时计算出来的。[59] 德布罗意-博姆理论产生了与标准量子力学相同的统计结果,但消除了许多概念上的困难。[60]

\subsection{参考文献}
[1]
^While there is no doubt that Young's demonstration of optical interference, using sunlight, pinholes and cards, played a vital part in the acceptance of the wave theory of light, there is some question as to whether he ever actually performed a double-slit interference experiment. Robinson, Andrew (2006). The Last Man Who Knew Everything. New York, NY: Pi Press. pp. 123–124. ISBN 978-0-13-134304-7..

[2]
^Lederman, Leon M.; Christopher T. Hill (2011). Quantum Physics for Poets. US: Prometheus Books. pp. 102–111. ISBN 978-1616142810..

[3]
^Feynman, 1965, p. 1.5.

[4]
^Darling, David (2007). "Wave–Particle Duality". The Internet Encyclopedia of Science. The Worlds of David Darling. Retrieved 2008-10-18..

[5]
^Feynman, 1965, p. 1.7.

[6]
^Leon Lederman; Christopher T. Hill. Quantum Physics for Poets. Prometheus Books, Publishers. p. 109. ISBN 978-1-61614-281-0..

[7]
^"...if in a double-slit experiment, the detectors which register outcoming photons are placed immediately behind the diaphragm with two slits: A photon is registered in one detector, not in both..." Müller-Kirsten, H. J. W. (2006). Introduction to Quantum Mechanics: Schrödinger Equation and Path Integral. US: World Scientific. p. 14. ISBN 978-981-2566911..

[8]
^Plotnitsky, Arkady (2012). Niels Bohr and Complementarity: An Introduction. US: Springer. pp. 75–76. ISBN 978-1461445173..

[9]
^"It seems that light passes through one slit or the other in the form of photons if we set up an experiment to detect which slit the photon passes, but passes through both slits in the form of a wave if we perform an interference experiment." Rae, Alastair I.M. (2004). Quantum Physics: Illusion Or Reality?. UK: Cambridge University Press. pp. 9–10. ISBN 978-1139455275..

[10]
^Feynman, The Feynman Lectures on Physics, 3:Quantum Mechanics p.1-1 "There is one lucky break, however— electrons behave just like light."..

[11]
^See: Davisson–Germer experiment Davisson, C. J (1928). "The diffraction of electrons by a crystal of nickel". Bell System Technical Journal. 7: 90–105. doi:10.1002/j.1538-7305.1928.tb00342.x..

[12]
^Charles Sanders Peirce first proposed the use of this effect as an artifact-independent reference standard for length C.S. Peirce (July 1879). "Note on the Progress of Experiments for Comparing a Wave-length with a Meter". American Journal of Science, as referenced by Crease, Robert P. (2011). World in the Balance: The historic quest for an absolute system of measurement. New York: W.W. Norton. p. 317. ISBN 978-0-393-07298-3. p. 203..

[13]
^Greene, Brian (1999). The Elegant Universe: Super Strings, Hidden Dimensions, and the Quest for the Ultimate Theory. New York: W.W. Norton. pp. 97–109. ISBN 978-0-393-04688-5..

[14]
^Feynman, 1965, chapter 3.

[15]
^Sir Geoffrey, Ingram Taylor (1909). "Interference Fringes with Feeble Light". Prof. Cam. Phil. Soc. 15: 114..

[16]
^Jönsson, Claus (1961-08-01). "Elektroneninterferenzen an mehreren künstlich hergestellten Feinspalten". Zeitschrift für Physik (in 德语). 161 (4): 454–474. Bibcode:1961ZPhy..161..454J. doi:10.1007/BF01342460. ISSN 0044-3328..

[17]
^Jönsson, Claus (1974-01-01). "Electron Diffraction at Multiple Slits". American Journal of Physics. 42 (1): 4–11. Bibcode:1974AmJPh..42....4J. doi:10.1119/1.1987592. ISSN 0002-9505..

[18]
^Merli, P G; Missiroli, G F; Pozzi, G (1976). "On the statistical aspect of electron interference phenomena". American Journal of Physics. 44 (3): 306–307. Bibcode:1976AmJPh..44..306M. doi:10.1119/1.10184..

[19]
^Rosa, R (2012). "The Merli–Missiroli–Pozzi Two-Slit Electron-Interference Experiment". Physics in Perspective. 14 (2): 178–194. Bibcode:2012PhP....14..178R. doi:10.1007/s00016-011-0079-0. PMC 4617474. PMID 26525832..

[20]
^"The most beautiful experiment". Physics World 2002..

[21]
^Frabboni, Stefano; Gabrielli, Alessandro; Carlo Gazzadi, Gian; Giorgi, Filippo; Matteucci, Giorgio; Pozzi, Giulio; Cesari, Nicola Semprini; Villa, Mauro; Zoccoli, Antonio (May 2012). "The Young-Feynman two-slits experiment with single electrons: Build-up of the interference pattern and arrival-time distribution using a fast-readout pixel detector". Ultramicroscopy. 116: 73–76. doi:10.1016/j.ultramic.2012.03.017. ISSN 0304-3991..

[22]
^Greene, Brian (2007). The Fabric of the Cosmos: Space, Time, and the Texture of Reality. Random House LLC. p. 90. ISBN 978-0-307-42853-0..

[23]
^Donati, O; Missiroli, G F; Pozzi, G (1973). "An Experiment on Electron Interference". American Journal of Physics. 41 (5): 639–644. Bibcode:1973AmJPh..41..639D. doi:10.1119/1.1987321..

[24]
^New Scientist: Quantum wonders: Corpuscles and buckyballs, 2010 (Introduction, subscription needed for full text, quoted in full in [1]).

[25]
^Wave Particle Duality of C60 Archived 31 3月 2012 at the Wayback Machine.

[26]
^lNairz, Olaf; Brezger, Björn; Arndt, Markus; Anton Zeilinger, Abstract (2001). "Diffraction of Complex Molecules by Structures Made of Light". Phys. Rev. Lett. 87 (16): 160401. arXiv:quant-ph/0110012. Bibcode:2001PhRvL..87p0401N. doi:10.1103/physrevlett.87.160401. PMID 11690188..

[27]
^Nairz, O; Arndt, M; Zeilinger, A (2003). "Quantum interference experiments with large molecules" (PDF). American Journal of Physics. 71 (4): 319–325. Bibcode:2003AmJPh..71..319N. doi:10.1119/1.1531580..

[28]
^Brian Greene, The Elegant Universe, p. 104, pp. 109–114.

[29]
^Greene, Brian (2004). The Fabric of the Cosmos: Space, Time, and the Texture of Reality. Knopf. pp. 204–213. Bibcode:2004fcst.book.....G. ISBN 978-0-375-41288-2..

[30]
^Baggott, Jim (2011). The Quantum Story: A History in 40 Moments. New York: Oxford University Press. pp. 76. ("The wavefunction of a system containing N particles depends on 3N position coordinates and is a function in a 3N-dimensional configuration space or 'phase space'. It is difficult to visualize a reality comprising imaginary functions in an abstract, multi-dimensional space. No difficulty arises, however, if the imaginary functions are not to be given a real interpretation.").

[31]
^Feynman, Richard P.; Robert B. Leighton; Matthew Sands (1965). The Feynman Lectures on Physics, Vol. 3. Addison-Wesley. pp. 1.1–1.8. ISBN 978-0201021189..

[32]
^Cassidy, David (2008). "Quantum Mechanics 1925–1927: Triumph of the Copenhagen Interpretation". Werner Heisenberg. American Institute of Physics. Retrieved 2008-06-21..

[33]
^Boscá Díaz-Pintado, María C. (29–31 March 2007). "Updating the wave-particle duality". 15th UK and European Meeting on the Foundations of Physics. Leeds, UK. Retrieved 2008-06-21..

[34]
^Bartell, L. (1980). "Complementarity in the double-slit experiment: On simple realizable systems for observing intermediate particle-wave behavior". Physical Review D. 21 (6): 1698–1699. Bibcode:1980PhRvD..21.1698B. doi:10.1103/PhysRevD.21.1698..

[35]
^Zeilinger, A. (1999). "Experiment and the foundations of quantum physics". Reviews of Modern Physics. 71 (2): S288–S297. Bibcode:1999RvMPS..71..288Z. doi:10.1103/RevModPhys.71.S288..

[36]
^P. Mittelstaedt; A. Prieur; R. Schieder (1987). "Unsharp particle-wave duality in a photon split-beam experiment". Foundations of Physics. 17 (9): 891–903. Bibcode:1987FoPh...17..891M. doi:10.1007/BF00734319..

[37]
^D.M. Greenberger and A. Yasin, "Simultaneous wave and particle knowledge in a neutron interferometer", Physics Letters A 128, 391–4 (1988)..

[38]
^Wootters, W. K.; Zurek, W. H. (1979). "Complementarity in the double-slit experiment: Quantum nonseparability and a quantitative statement of Bohr's principle" (PDF). Phys. Rev. D. 19 (2): 473–484. Bibcode:1979PhRvD..19..473W. doi:10.1103/PhysRevD.19.473. Retrieved 5 February 2014..

[39]
^Hillmer, R.; Kwiat, P. (2007). "A do-it-yourself quantum eraser". Scientific American. Vol. 296 no. 5. pp. 90–95. Bibcode:2007SciAm.296e..90H. doi:10.1038/scientificamerican0507-90. Retrieved 2016-01-11..

[40]
^Chiao, R. Y.; P. G. Kwiat; Steinberg, A. M. (1995). "Quantum non-locality in two-photon experiments at Berkeley". Quantum and Semiclassical Optics: Journal of the European Optical Society Part B. 7 (3): 259–278. arXiv:quant-ph/9501016. Bibcode:1995QuSOp...7..259C. doi:10.1088/1355-5111/7/3/006..

[41]
^Francis, Matthew (2012-05-21). "Disentangling the wave-particle duality in the double-slit experiment". Ars Technica..

[42]
^Svensson, Bengt E. Y. (2013). "Pedagogical Review of Quantum Measurement Theory with an Emphasis on Weak Measurements". Quanta. 2 (1): 18–49. arXiv:1202.5148. doi:10.12743/quanta.v2i1.12..

[43]
^Pfleegor, R. L.; Mandel, L. (July 1967). "Interference of Independent Photon Beams". Physical Review. 159 (5): 1084–1088. Bibcode:1967PhRv..159.1084P. doi:10.1103/PhysRev.159.1084..

[44]
^"Interference of Independent Photon Beams: The Pfleegor-Mandel Experiment". Archived from the original on 3 January 2011. Retrieved 2011-06-16.>.

[45]
^Sillitto, R.M.; Wykes, Catherine (1972). "An interference experiment with light beams modulated in anti-phase by an electro-optic shutter". Physics Letters A. 39 (4): 333–334. Bibcode:1972PhLA...39..333S. doi:10.1016/0375-9601(72)91015-8..

[46]
^"To a light particle".

[47]
^Nature: Wave–particle duality of C60 molecules, 14 October 1999. Abstract, subscription needed for full text.

[48]
^Schouten, H.F.; Kuzmin, N.; Dubois, G.; Visser, T.D.; Gbur, G.; Alkemade, P.F.A.; Blok, H.; Hooft, G.W.; Lenstra, D.; Eliel, E.R. (7 February 2005). "Plasmon-Assisted Two-Slit Transmission: Young's Experiment Revisited". Phys. Rev. Lett. 94 (5): 053901. Bibcode:2005PhRvL..94e3901S. doi:10.1103/physrevlett.94.053901. PMID 15783641..

[49]
^Bach, Roger; et al. (March 2013). "Controlled double-slit electron diffraction". New Journal of Physics. 15 (3): 033018. arXiv:1210.6243. Bibcode:2013NJPh...15c3018B. doi:10.1088/1367-2630/15/3/033018..

[50]
^"Physicists Smash Record For Wave-Particle Duality".

[51]
^Bush, John W. M. (2010). "Quantum mechanics writ large". PNAS. 107 (41): 17455–17456. Bibcode:2010PNAS..10717455B. doi:10.1073/pnas.1012399107. PMC 2955131..

[52]
^Natalie Wolchover, Quanta Magazine, Science, 06.30.14 (2014-06-30). "Have We Been Interpreting Quantum Mechanics Wrong This Whole Time?". Wired.CS1 maint: Multiple names: authors list (link).

[53]
^Couder, Y.; Fort, E. (2012). "Probabilities and trajectories in a classical wave-particle duality" (PDF). Journal of Physics: Conference Series. 361 (1): 012001. Bibcode:2012JPhCS.361a2001C. doi:10.1088/1742-6596/361/1/012001. Retrieved 23 June 2015..

[54]
^Li, Pengyun; Sun, Yifan; Yang, Zhenwei; Song, Xinbing; Zhang, Xiangdong (2016). "Classical hypercorrelation and wave-optics analogy of quantum superdense coding". Scientific Reports. 5: 18574. Bibcode:2015NatSR...518574L. doi:10.1038/srep18574. PMC 4686973. PMID 26689679..

[55]
^Jenkins FA and White HE, Fundamentals of Optics, 1967, McGraw Hill, New York.


[56]
^Longhurst RS, Physical and Geometrical Optics, 1967, 2nd Edition, Longmans.

[57]
^Rovelli, Carlo (1996). "Relational Quantum Mechanics". International Journal of Theoretical Physics. 35 (8): 1637–1678. arXiv:quant-ph/9609002. Bibcode:1996IJTP...35.1637R. doi:10.1007/BF02302261..

[58]
^Filk, Thomas (2006). "Relational Interpretation of the Wave Function and a Possible Way Around Bell's Theorem". International Journal of Theoretical Physics. 45 (6): 1205–1219. arXiv:quant-ph/0602060. Bibcode:2006IJTP...45.1166F. doi:10.1007/s10773-006-9125-0..

[59]
^Philippidis, C.; Dewdney, C.; Hiley, B. J. (1979). "Quantum interference and the quantum potential". Il Nuovo Cimento B (in 英语). 52 (1): 15–28. Bibcode:1979NCimB..52...15P. doi:10.1007/bf02743566. ISSN 1826-9877..

[60]
^"Bohmian Mechanics". The Stanford Encyclopedia of Philosophy. Metaphysics Research Lab, Stanford University. 2017..