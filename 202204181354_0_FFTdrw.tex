% 用傅里叶级数画曲线(Matlab)

\begin{issues}
\issueDraft
\end{issues}

\pentry{用 FFT 计算傅里叶变换(Matlab)\upref{FFTft}}

\begin{figure}[ht]
\centering
\includegraphics[width=10cm]{./figures/FFTdrw_1.png}
\caption{运行结果, 动画见\href{https://wuli.wiki/apps/FFTplt.html}{这里}.} \label{FFTdrw_fig1}
\end{figure}


平面上一条任意的连续曲线, 都可以用直角坐标的参数方程 $x(t), y(t)$, 表示. 若把平面看成复平面, 参数方程可以用复值函数 $z(t) = x(t) + \I y(t)$ 代替. 把它展开为傅里叶级数\upref{FSExp}, 那么级数的每一项就代表复平面上的一个做匀速圆周运动的矢量. 把这些矢量首尾相连, 就可以制作漂亮的动画.

用 FFT 代替定积分, 计算傅里叶展开系数. 然后再把系数的模长按照从大到小排序.

其中的曲线数据是笔者手动提取的, 见 “用 Matlab 手动提取图片中的曲线坐标\upref{plt2xy}”. 你也可以把自己的任何曲线的坐标散点 \verb|x, y| 保存到 \verb|logo_curve.mat|. 程序中使用了 “用 FFT 计算傅里叶变换(Matlab)\upref{FFTft}” 中的 \verb|FS.m|

\begin{lstlisting}[language=matlab, caption=FFTplt.m]
% ==== 参数设置 ====
load logo_curve.mat x y;
Nt = 500;
Ncirc = 50;
% ================
close all;
x = x - mean(x);
y = y - mean(y);
N = numel(x);
z = x + 1i*y;
x0 = 1; dx = 1;
[C, w] = FS(z, x0, dx); % 傅里叶级数
C = C(:); w = w(:);

t = linspace(1,N,Nt);
z1 = sum(C.*exp(1i*w.*t), 1);
figure; plot(x, y); hold on;
plot(real(z1),imag(z1));

th = linspace(0, 2*pi, 200);
costh = cos(th); sinth = sin(th);
N1 = round(N/2-1);
[~, order] = sort(abs(C), 'descend');
C1 = C(order);
w1 = w(order);
z1 = sum(C1.*exp(1i*w1.*t), 1);
plot(real(z1),imag(z1));

for it = 1:Nt
    z1 = [0; cumsum(C1.*exp(1i*w1*t(it)), 1)];
    clf;
    plot(x, y); hold on; axis equal;
    for i = 1:Ncirc
        A = abs(z1(i+1)-z1(i)); x0 = real(z1(i)); y0 = imag(z1(i));
        plot(x0+A*costh, y0+A*sinth, 'g');
    end
    plot(real(z1), imag(z1), '.-');
    axis([-0.7, 0.7, -0.4, 0.7]);
    saveas(gcf, [num2str(it), '.png']);
end
\end{lstlisting}
