% 拓扑不变量(综述)
% license CCBYNCSA3
% type Wiki

本文根据 CC-BY-SA 协议转载翻译自维基百科\href{https://en.wikipedia.org/wiki/Topological_property}{相关文章}

在拓扑学及其相关数学领域中,拓扑性质或拓扑不变量是指在同胚下保持不变的拓扑空间的某种性质。换句话说,拓扑性质是一个在同胚映射下封闭的拓扑空间的真类。也就是说,如果某个空间 $X$ 具有某一性质,那么所有与 $X$ 同胚的空间也都具有这一性质。通俗地说,拓扑性质就是能够用开集来描述的空间性质。

在拓扑学中,一个常见的问题是判断两个拓扑空间是否同胚。为了证明两个空间不是同胚,只需找到一个它们不共有的拓扑性质即可。
\subsection{ }
一个性质 $P$ 可以具备以下特征:
\begin{itemize}
\item \textbf{遗传性},若对任意拓扑空间 $(X, \mathcal{T})$ 及其子集 $S \subseteq X$,其子空间 $\bigl(S, \mathcal{T}|_S\bigr)$ 也具有性质 $P$。
\item \textbf{弱遗传性},若对任意拓扑空间 $(X, \mathcal{T})$ 及其闭子集 $S \subseteq X$,其子空间 $\bigl(S, \mathcal{T}|_S\bigr)$ 也具有性质 $P$。
\end{itemize}
\subsection{常见的拓扑性质}
\subsubsection{基数函数}
\begin{itemize}
\item 空间 $X$ 的基数 $|X|$:指空间 $X$ 中点的数量。
\item 空间 $X$ 的拓扑基数$|\tau(X)|$:指该空间拓扑(即所有开集的集合)的基数。
\item 权$w(X)$:指空间 $X$ 的拓扑基的最小基数。
\item 稠密度$d(X)$:指空间 $X$ 的一个稠密子集的最小基数,即闭包等于 $X$ 的子集所含的最少点数。
\end{itemize}
\subsubsection{分离性}
在较早的数学文献中,这些术语的定义有所不同;参见“分离公理的历史”。

\begin{itemize}
\item \textbf{$T_0$ 空间(Kolmogorov 空间)}.若空间中任意两个不同的点 $x$ 和 $y$,至少存在一个开集包含其中一个点但不包含另一个点,则该空间是Kolmogorov 空间。
\item \textbf{$T_1$ 空间(Fréchet 空间)}.若空间中任意两个不同的点 $x$ 和 $y$,总可以找到一个开集包含 $x$ 但不包含 $y$(注意与 $T_0$ 空间的区别,这里可以指定是哪一个点位于开集中)。
  等价地,若空间中每个单点集都是闭集,则该空间是$T_1$ 空间。
  所有 $T_1$ 空间都必然是 $T_0$ 空间。
\item \textbf{清醒空间(Sober 空间)}.若空间中的每个不可约闭集$C$都有一个唯一的*泛点$p$,则该空间是Sober 空间。换句话说,如果闭集 $C$ 不能表示为两个非空闭集的并集(可以相交),那么必定存在某个点 $p$,使得 $\{p\}$ 的闭包正好是 $C$,且只有这个点 $p$ 具有这种性质。
\item \textbf{ $T_2$ 空间(Hausdorff 空间)}.若空间中任意两个不同的点都有**互不相交的邻域**,则该空间是 Hausdorff 空间。所有 $T_2$ 空间都必然是 $T_1$ 空间。
\item \textbf{$T_{2\frac{1}{2}}$ 空间(Urysohn 空间)}.若空间中任意两个不同的点都有**互不相交的闭邻域**,则该空间是 Urysohn 空间。所有 $T_{2\frac{1}{2}}$ 空间都必然是 $T_2$ 空间。
\item \textbf{完全 $T_2$ 空间(完全 Hausdorff 空间)}.若空间中任意两个不同的点可以被某个函数分离,则该空间是完全 Hausdorff 空间。所有完全 Hausdorff 空间都是 Urysohn 空间。
\item \textbf{正规空间(Regular 空间)}.若空间中任意一个闭集 $C$ 和不在 $C$ 中的点 $p$,都存在互不相交的邻域分别包含 $C$ 和 $p$,则该空间是正规空间。
\item \textbf{$T_3$ 空间(正规 Hausdorff 空间,Regular Hausdorff)}.如果一个空间既是正规空间又$T_0$空间,那么它就是正规 Hausdorff 空间。(因为正规空间若且唯若是 $T_0$ 空间时才是 Hausdorff 空间,所以这种命名是一致的。)
\item \textbf{完全正规空间}.
如果对于任意闭集 $C$ 和一个不在 $C$ 中的点 $p$,存在某个函数能够分离 $C$ 和 $\{p\}$,则该空间是完全正规空间。
\item \textbf{$T_{3\frac{1}{2}}$ 空间(Tychonoff 空间,完全正规 Hausdorff 空间,Completely $T_3$ 空间)}.如果一个空间既是完全正规空间又是$T_0$空间,则该空间是Tychonoff 空间。(因为完全正规空间若且唯若是 $T_0$ 空间时才是 Hausdorff 空间,所以这种命名是一致的。)所有 Tychonoff 空间都是正规 Hausdorff 空间。
\item \textbf{正规空间}.如果任意两个互不相交的闭集都有互不相交的邻域,则该空间是正规空间。正规空间允许进行单位分解。
\item \textbf{$T_4$ 空间(正规 Hausdorff 空间,Normal Hausdorff space)}.一个正规空间若且唯若是$T_1$空间时才是 Hausdorff 空间。所有正规 Hausdorff 空间必然是 Tychonoff 空间。
\item \textbf{完全正规空间}如果任意两个分离的集合都有互不相交的邻域,则该空间是**完全正规空间。
\item \textbf{$T_5$ 空间(完全正规 Hausdorff 空间,Completely normal Hausdorff space)}.一个完全正规空间若且唯若是 $T_1$空间时才是 Hausdorff 空间。所有完全正规 Hausdorff 空间都是正规 Hausdorff 空间。
\item \textbf{完全正规化空间}.如果任意两个互不相交的闭集都能被某个函数精确分离**,则该空间是**完全正规化空间。完全正规化空间也必然是完全正规空间。
\item \textbf{$T_6$ 空间(完全正规 Hausdorff 空间,Perfectly normal Hausdorff space 或 完全 $T_4$ 空间)}.果一个空间既是完全正规化空间又是$T_1$空间,则它是完全正规 Hausdorff 空间。所有完全正规 Hausdorff 空间也必然是完全正规 Hausdorff 空间。
\item \textbf{离散空间}.如果空间中的每一个点都是完全孤立的,也就是说空间的任意子集都是开集,则该空间是离散空间。
\item \textbf{孤立点的数量}.指拓扑空间中孤立点的总数量。
\end{itemize}
\subsubsection{可数性条件}
\begin{itemize}
\item \textbf{可分空间}.如果一个空间存在一个可数的稠密子集,则该空间是可分空间。
\item \textbf{第一可数空间}.如果空间中每个点都有一个可数的局部基,则该空间是第一可数空间。
\item \textbf{第二可数空间}.如果空间的拓扑存在一个可数的基,则该空间是第二可数空间。第二可数空间总是可分的、第一可数的,并且是Lindelöf 空间。
\item \textbf{Lindelöf 空间}.如果空间的任意开覆盖都有一个可数的子覆盖,则该空间是Lindelöf 空间。
\item \textbf{σ-紧空间}.如果一个空间可以表示为可数多个紧子空间的并集,则该空间是σ-紧空间。
\end{itemize}
\subsubsection{连通性}
\begin{itemize}
\item \textbf{连通空间}.如果一个空间**不能表示为两个不相交的非空开集的并集**,则它是**连通的**。等价地,若空间中唯一既开又闭的子集(clopen 集)只有空集和空间本身,则该空间是连通的。
\textbf{局部连通空间}.如果空间中每个点都有一个**由连通集组成的局部基**,则该空间是**局部连通的**。

\item \textbf{全不连通空间}.如果一个空间**没有包含多于一个点的连通子集**,则该空间是**全不连通的**。

\item \textbf{极端不连通空间}.如果空间中的**每个开集的闭包仍然是开集**,则该空间是**极端不连通的**。

\item \textbf{路径连通空间}.若空间 $X$ 中任意两点 $x, y$ 之间都存在一条路径 $p$,即一个连续映射$p: [0, 1] \to X,\quad p(0) = x,\; p(1) = y$则该空间是**路径连通的**。路径连通空间总是连通的。

\item \textbf{局部路径连通空间}.如果空间中每个点都有一个**由路径连通集组成的局部基**,则该空间是**局部路径连通的**。在局部路径连通空间中,**连通性与路径连通性是等价的**。

\item \textbf{弧连通空间}.若空间 $X$ 中任意两点 $x, y$ 之间都存在一条弧 $f$,即一个**单射连续映射$f: [0, 1] \to X,\quad f(0) = x,\; f(1) = y$则该空间是**弧连通的**。所有弧连通空间都是路径连通的。

\textbf{单连通空间}.如果空间 $X$ 是路径连通的,并且从单位圆 $S^1$ 到 $X$ 的**任意连续映射$f: S^1 \to X$都与常值映射同伦,则 $X$ 是**单连通的**。

\textbf{局部单连通空间}.如果空间 $X$ 中的每个点 $x$ 都有一个**单连通的邻域基**,则 $X$ 是**局部单连通的**。

\textbf{半局部单连通空间}.如果空间 $X$ 中每个点都有一个邻域基 $U$,使得 $U$ 内的**任意闭合曲线在整个空间 $X$ 中是可缩的**,则 $X$ 是**半局部单连通的**。半局部单连通性比局部单连通性更弱,但它是存在**通用覆盖空间**的必要条件。

\textbf{可缩空间}.如果空间 $X$ 上的恒等映射与某个常值映射是同伦的,则 $X$ 是**可缩的**。所有可缩空间都是单连通的。

\textbf{超连通空间}.如果空间中**不存在两个互不相交的非空开集**,则该空间是**超连通的**。所有超连通空间都是连通的。

\textbf{极连通空间}.如果空间中**不存在两个互不相交的非空闭集**,则该空间是**极连通的**。所有极连通空间都是路径连通的。

\textbf{不分化空间或平凡拓扑空间}.如果空间中唯一的开集只有**空集和空间本身**,则该空间是不分化空间,也称具有**平凡拓扑**的空间。

\end{itemize}