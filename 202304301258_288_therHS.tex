% 热学初步(高中)

\begin{issues}
\issueTODO
\end{issues}
% 分子动理论|气体等x定律|固体液体|热力学定律
% 缩减一部分,把第二章的前两小节合并,第二章整体作为一个新的小节
% 或者直接拆分成分子动力学和热力学初步算了,麻烦
% 第二章和第三章作为热力学初步内容

%\pentry{相互作用\upref{HSPM02}}% 分子动力学\upref{thermo}

\subsection{温度和温标}
\subsubsection{状态参量与平衡态}
以研究容器中气体的热学性质为例,我们的研究对象是一个由大量分子所组成的系统,称之为热力学系统,简称\textbf{系统}。在我们的经验常识中,一小罐热的气体会在室温下逐渐变凉,最终和室温温度保持一致,这个过程中,这一小罐热的气体即是系统,而系统之外与之产生相互作用的其他物体统称为\textbf{外界}。外界影响系统,导致系统的某些物理量发生变化。在这个例子中,热空气温度逐渐下降,最终和外界空气保持一致,热空气(系统)的状态也随之改变。在热学中,为了确定系统的状态,需要用一些物理量来进行描述,这些物理量叫做系统的状态参量,比如体积$V$是描述系统空间范围的几何参量,压强$p$是描述系统之间或内部力的作用的力学参量,温度$T$是确定系统冷热程度的热学参量……

但是,往往在任意时刻确定系统的状态是困难的,只有在系统处于\textbf{平衡态},也即系统经过了足够长的时间的演化,内部的各个部分的状态参量达到稳定状态不再改变时,我们才可以比较准确的描述系统的状态。

\subsubsection{温度和热平衡}
在上述的例子中,可以发现温度逐渐下降,说明系统的热学性质发生了改变,并在足够长的时间之后罐中空气温度和外界保持一致。在这个过程中,系统和外界相互接触,并发生了热传导,最终各自的状态参量不再发生改变,达到了平衡状态,这种平衡叫做\textbf{热平衡}。如果两个系统之间处于热平衡,那它们必然可以被某一相等的状态参量所描述,这里的状态参量则是\textbf{温度}。换言之,温度是决定一个系统是否处于热平衡的物理量。实验表明,如果两个系统分别与第三个系统达到热平衡,则两个系统之间一定也是处于热平衡的,因为它们都具有相同的温度,这个结果被称为\textbf{热平衡定律}。

\subsubsection{温度计}
为了测量温度,人们发明了温度计。首先,温度计需要有一种测温物质,这种测温物质的某些物理性质会随着温度的改变而发生改变,比如说水银的热胀冷缩可以制成水银温度计,气体的压强随温度变化可以制成气体温度计,电阻随温度变化可以制成电阻温度计,由不同金属随温度升高膨胀程度不同制成的双金属温度计……需要注意的是,每一种温度计都有其工作范围,在工作范围之外原来的变化关系未必成立,不能够继续测量温度。为了使用方便,人们尽可能的采用具有线性变化的物质来制作温度计。另一方面,人们需要定义该特性和温度之间的对应关系,每一种不同的定义方式对应于不同的\textbf{温标},同时,还需要定义温度的零点和分度方法。例如,我们常见的摄氏度定义一个标准大气压下冰水混合物的温度为$0^\circ \mathrm{C}$,水沸腾的温度为$100^\circ \mathrm{C}$,并将其间的刻度平均分成$100$份,每份为$1^\circ \mathrm{C}$。

但是当我们进行对热力学的学习时,我们将采用热力学温标,对应的就是\textbf{热力学温度},单位是\textbf{开尔文},简称\textbf{开},符号是$\mathrm{K}$。摄氏温度$t$和热力学温度$T$之间的换算关系由国际计量大会所确定:$$T=t+273.15\mathrm{K}$$可以发现,摄氏温度和热力学温度的温度差都是相同的,也即$1^\circ \mathrm{C}=1\mathrm{K}$。

\subsection{气体的等温变化}
为了探索气体的热学性质,首先先考察一种特殊的情况。在温度不变的条件下,一定质量的气体,它的压强和体积将满足什么样的变化关系呢?这种变化被称之为气体的\textbf{等温变化}。英国科学家玻意耳和法国科学家马洛特各自通过实验发现,一定质量的气体,在温度不变的情况下压强$p$与体积$V$成反比,即$$p\propto\dfrac{1}{V}$$稍作整理可以得到$$pV=C$$其中$C$是常数,该定律也称\textbf{玻意耳定律}。但是实际应用中我们并不是特别关系常数$C$的具体取值,而更关系在初末状态时系统状态参量的关系,因此上式可以继续整理得到$$p_1V_1=p_2V_2$$其中$p_1$、$V_1$和$p_2$、$V_2$可以分别看成初末状态下系统的压强和体积。
\subsection{气体的等压变化}
同理,我们也将研究气体在压强不变的情况下,气体体积和温度之间的变化关系,这种变化叫气体的\textbf{等压变化}。实验表明,在压强不变时,一定质量的某种气体的体积随温度$T$线性变化,当温度采用热力学温度$T$时,所绘制的$V-T$图像中,等压线是一条过原点的直线。

法国科学家盖-吕萨克首先发现了这一线性关系,并将其表述为:一定质量的某种气体,在压强不变的情况下,其体积$V$与热力学温度$T$成正比,即$$V=CT$$其中$C$是常量,称之为\textbf{盖-吕萨克定律}。同上所述,我们更关心体系初末状态之间的关系,即有$$\dfrac{V_1}{T_1}=\dfrac{V_2}{T_2}$$,其中$V_1$、$T_1$和$V_2$、$T_2$可以分别看成初末状态下系统的体积和热力学温度。
\subsection{气体的等容变化} 
现在,我们只剩下体积不变这一个选项了。一定质量的气体,在体积不变的情况下,压强随着温度变化的过程叫做气体的\textbf{等容变化}。实验表明,在等容变化中,压强$p$和温度成正比,如果选用热力学温度$T$,则呈简单的正比例关系。当气体压强不太大、温度不太低时,可以发现$p-T$图像连线的延长线过原点,此时对应的就是热力学温度$T$等于$0$的状态,也称\textbf{绝对零度}。

法国科学家查理通过实验第一次发现这个定律,即一定质量的某种气体,在体积不变的情况下,压强$p$和热力学温度$T$成正比,即$p\propto T$,写成等式有$$p=CT$$其中$C$为常数,称之为\textbf{查理定律}。我们依旧更关心体系初末状态之间的关系,由此可以整理得到$$\dfrac{p_1}{T_1}=\dfrac{p_2}{T_2}$$其中$p_1$、$T_1$和$p_2$、$T_2$可以分别看成初末状态下系统的压强和热力学温度。

\subsection{理想气体}
\subsubsection{理想气体假设}
事实上,我们上述得到的结论都需要满足理想气体的条件。\textbf{理想气体}指的是气体的分子大小和相互作用力可以忽略不计,也不计气体分子和容器壁碰撞时的动能损失。换句话说,理想气体中的分子可以看成质点,且彼此之间是完全独立的,不需要考虑我们先前介绍的分子间相互作用力\upref{thermo},且只发生弹性碰撞。现实中的气体,其分子具有大小,且具有相互作用力。但是在压强不太大(相比于一个大气压)、温度不太低(相比于室温)的前提下,分子大小远远小于分子间距,且分子间作用力很小,因此可以近似成理想气体处理。

当温度很低、压强很大时,由于气体分子大小和气体分子间相互作用的影响,我们不再可以使用玻意耳定律、盖-吕萨克定律以及查理定律进行处理,而是采用更为精确的近似方程\footnote{以后会在大学热学的学习中有更深入的说明。}。
\subsubsection{气体实验定律的微观解释}
 % 图2.3-3等,可选
 回顾分子动理论的\upref{thermo}内容,可以从微观的角度定性的解释气体实验定律。一定质量的某种气体,在温度不变时,气体的平均动能是一定的。这时候,如果减小体积,则单位时间单位面积内,碰撞到容器壁上的气体分子数目增多,而气体分子动量的改变量和之前相等,由此,压强便会增大,这对应的是玻意耳定律。

 同理,如果升高温度,则气体分子的平均动能更大,且由于分子速度变快,单位时间单位面积内和容器壁碰撞的分子数目也会更多,因此如果维持体积不变,压强就会增大,这就是查理定律的微观解释。如果想保持压强不变,只能通过降低气体分子数密度的方式来使得单位时间单位面积内和容器壁碰撞的分子数目减小,因此需要增加气体的体积,这对应的就是盖-吕萨克定律。
\subsection{固体和液体简介}
\subsubsection{固体}
生活中常见的物质状态除了气态以外,还有固态和液态。对于固体,我们可以将其分为\textbf{晶体}和\textbf{非晶体}。常见的晶体有石英、云母、食盐、硫酸铜等等,它们都具有规则的几何形状。
非晶体有玻璃、松香、沥青、橡胶等,它们通常没有规则的几何外形。

我们知道,晶体具有确定的熔点,而非晶体的熔化温度是不确定的,事实上,晶体和非晶体在物理性质上也有诸多不同。晶体往往具有\textbf{各向异性},比如说玻璃片在各个方向上导热性能相同,而云母片某一个轴的导热性能要大于另一个轴;有些晶体具有不同的导电特性;有些晶体各个方向的光学性质不同,比如说方解石是会形成双折射现象,可以将光分解为两束沿着不同方向折射的光。这些物理性质还包括弹性、硬度、磁性等等。而对于非晶体,它的物理性质往往是\textbf{各向同性}的。

晶体和非晶体物理性质的区别可以在微观上找到解释:通过X射线在晶体上的衍射实验,以及后续的电子显微镜和扫描隧道显微镜,人们逐渐观察到了物质表面上原子的排列方式。经由观察,人们发现,在各种晶体中,原子(或者分子、离子)都是规则排布的,具有空间上的周期性,这也就解释了为什么晶体具有规则的几何外形。以及由于排列规则,因此晶体不同方向上的原子(也可能是分子、离子)的数目和排布方式往往是不同的,因此会产生各向异性。至于非晶体,组成它们的原子(或者分子、离子)排布是随机的,由此,非晶体不具有规则的几何外形,而且各向同性。

需要注意的是,同一种原子有可能形成不同的晶体,比如说碳原子既可以构成石墨、也可以构成金刚石。同种原子乃至同种物质也未必一直都是晶体或者非晶体,比如说石英是一种晶体,但如果高温熔化之后再重新凝聚,则会变成一种非晶体——石英玻璃。

另外,我们也会发现某些物质具有确定的熔点,但是自然状态下不具有规则的几何外形,比如说金属,这类称之为\textbf{多晶体}。多晶体顾名思义,可以理解为由多个细小的晶粒随机排布所组成的物质,宏观上的随机排布使得它不具备规则的几何外形和各向异性,但是每一个晶粒小单元却是一个小的单晶体,具有规则的几何形状和各向异性。

% 将晶体、非晶体、多晶体的性质稍加总结,形成下表:
% 插不了合并行列吗
\subsubsection{液体}
液体具有表面张力,对于生活中的观察,我们可以发现,昆虫可以停在水面上,肥皂泡可以在空气中形成 球形,这些现象告诉我们液体是具有表面张力的。

由于液体表面的分子数密度较液体内部较小,在液体内部,分子间平均间距$r$略小于平衡位置$r_0$,分子表现为斥力,但是表面层的分子平均距离$r$略大于$r_0$,所以液体表面的分子之间呈现相互吸引力,这种力的方向总是和液面相切的,且与分界面垂直,叫做液体的\textbf{表面张力},表面张力让液体表面具有收缩趋势,这种收缩的趋势让液体表面绷紧。

当然,对于不同的界面,对于液体表面分子间距的影响是不同的。水滴可以在荷叶上形成近乎于完美的球形,但是滴在玻璃上却会散开并逐渐附着在玻璃表面上,这种区别可以定义浸润和不浸润。液体如果可以润湿某种固体并附着在固体的表面上,这种现象就叫做\textbf{浸润},与之相反,如果不会附着,那就是\textbf{不浸润}。浸润现象由于液体和固体表面分子之间的引力大于液体内部分子引力,所以分界面上的分子数密度将会大于液体内部的,分界面上的液体分子之间相互表现为排斥力。反之对于不浸润现象,固体分子对液体分子的引力小于液体分子内部的引力,就和我们讨论的表面张力现象一样,分界面上的分子数密度小于液体内部的分子数密度,因此界面上分子表现为相互吸引,液面会有收缩趋势。

生活中也可以在毛细现象中观察到浸润和不浸润的区别,水可以浸润玻璃,因此插入到水中的玻璃细管可以观察到水面上升,并且水面呈现出向下凹的形态;相反,由于水银不浸润玻璃,因此如果将玻璃细管插入到水银当中,玻璃管中的水银液面会下降并且向上鼓起。% 补充示意图
实验证明,如果管直径越小,则这种上升和下降的现象就越明显,这种现象称之为\textbf{毛细现象}。这在理论上也是很好理解的,当管直径越小,固液界面处的分子数目比之于细管内的液体数目就越少,界面处分子间作用力的影响表现得就更明显。

\subsubsection{其他物质形态}
除了我们最熟悉的固态、液态和气态以外,自然界中存在更多的物态。


\subsection{功、热、内能}
\subsection{热力学第一定律}
\subsection{能量守恒定律}
\subsection{热力学第二定律}

 

%% 画图时间

%% 错别字纠正
