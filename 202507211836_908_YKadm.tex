% 雅克·阿达马(综述)
% license CCBYSA3
% type Wiki

本文根据 CC-BY-SA 协议转载翻译自维基百科\href{https://en.wikipedia.org/wiki/Jacques_Hadamard}{相关文章}。

\begin{figure}[ht]
\centering
\includegraphics[width=6cm]{./figures/a10c8928aff7fe27.png}
\caption{} \label{fig_YKadm_1}
\end{figure}
雅克·所罗门·哈达玛(Jacques Salomon Hadamard,法语发音:[adamaʁ];1865年12月8日-1963年10月17日)是一位法国数学家,在数论、复分析、微分几何和偏微分方程等领域作出了重要贡献。\(^\text{[3][4][5]}\)
\subsection{生平}
雅克·哈达玛出生于法国凡尔赛,是一位教师阿梅代·哈达玛和克莱尔·玛丽·让娜·皮卡尔之子,具有犹太血统。他曾就读于沙勒曼中学和路易大帝中学,其父亲也在这些学校任教。1884年,哈达玛进入高等师范学校,在该校和巴黎综合理工学院的入学考试中均名列第一。他的老师包括坦纳里、埃尔米特、达布、阿佩尔、古尔萨和皮卡尔。1892年,他获得博士学位,并凭借关于黎曼ζ函数的论文获得了数学科学院大奖。

1892年,哈达玛与同样具有犹太血统的露易丝-安娜·特雷内尔结婚,育有三子二女。次年,他在波尔多大学担任讲师,期间证明了著名的行列式不等式,这一结果在等号成立时引出了“哈达玛矩阵”的概念。1896年,他作出两项重要贡献:一是使用复函数理论证明了素数定理(同时期查尔斯·让·德·拉·瓦莱-普桑亦独立证明了该定理);二是因其在曲面微分几何与动力系统中测地线研究方面的成果获得法国科学院的博尔丹奖。同年,他被任命为波尔多大学天文学与理性力学教授。1898年,他继续在几何学与符号动力学方面开展奠基性研究,探讨负曲率曲面上的测地线问题。凭借其累积性的工作,哈达玛于1898年获得庞赛列奖。

德雷福斯事件之后,哈达玛开始积极参与政治,这一事件与他有直接关系,因为德雷福斯的妻子露西是他的表姐妹。自此,哈达玛成为犹太事业的坚定支持者[6],尽管他在宗教上自称是无神论者。[7][8]

1897年,哈达玛迁回巴黎,先后在索邦大学和法兰西公学院任教,1909年被任命为力学教授。此外,他还于1912年出任巴黎综合理工学院的数学分析讲座教授,并于1920年接替若尔当和阿佩尔,在中央理工学院担任同类职位。在巴黎期间,哈达玛将研究重点转向数学物理问题,尤其关注偏微分方程、变分法及泛函分析的基础。他在偏微分方程理论中引入了“适定问题”的概念以及“降维法”,并最终在1922年于耶鲁大学讲授的课程基础上出版了这方面的奠基性著作。晚年,哈达玛还涉猎概率论及数学教育领域。

哈达玛于1916年当选为法国科学院院士,接替庞加莱的席位,并参与编辑庞加莱的全集。他于1920年成为荷兰皇家艺术与科学学院的外籍院士[9]。1929年,他当选为苏联科学院外籍院士。他曾于1930年和1934年访问苏联,1936年应苏联和中国数学家的邀请访问中国。

第二次世界大战初期,哈达玛留在法国,1940年逃至法国南部。1941年,维希政府允许他前往美国,他在纽约哥伦比亚大学获得一个访问教授职位。1944年他迁往伦敦,战后于1945年返回法国。

1901年10月,耶鲁大学在其建校200周年庆典上授予哈达玛荣誉法学博士学位[10]。1956年,他因毕生成就获得法国国家科学研究中心金质奖章。他于1963年在巴黎去世,享年97岁。

哈达玛的学生包括莫里斯·弗雷歇、保罗·勒维、索勒姆·曼德尔布罗伊特和安德烈·韦伊。
\subsection{关于创造力}
在他的著作《数学领域中的发明心理学》一书中,[11] 哈达玛通过内省法研究数学思维过程,[11]: 2  试图报告并解释自己的观察结果,这些观察既包括他个人的体验,也包括他从其他从事创造性工作的学者那里收集的信息。[11]: 133  与那些将语言与认知等同起来的作者截然不同,哈达玛描述自己进行数学思考时大多无言,常常伴随着能够代表问题整体解答的心理图像。他曾对当时(约1900年)的一百位顶尖物理学家进行调查,询问他们是如何进行科研工作的。

哈达玛描述了数学家/理论物理学家高斯、亥姆霍兹、庞加莱等人的经验,他们常常以“一种突然而自发的方式”看到完整的解决方案。[11]: 13–16 

哈达玛将创造过程描述为五步创意模型中的四个步骤,这一模型由格雷厄姆·华莱士提出,其中前三步也由亥姆霍兹提出过:[11]: 56  准备、潜伏、顿悟(和验证。华莱士的五步模型还在“顿悟”之前加了一步“预感”,即一种即将找到问题答案的突然感觉。[12]
