% 每个部分请根据从入门到进阶排序
% 每个部分请根据先中文后英文排序
\begin{thebibliography}{99}
%数理逻辑
%=========================================
\bibitem{BasicSetTheory}
A. Shen, N. K. Vereshchagin, 陈光还译,\textsl{集合论基础}, 高等教育出版社, 2013年

% 微积分
%=========================================
\bibitem{同济高}
同济大学数学系. \textsl{高等数学} (上下册) 高等教育出版社 (2014) 第七版
\bibitem{北大高}
李忠,周建莹. \textsl{高等数学} (上下册) 北京大学出版社 (2009) 第二版
\bibitem{Thomas}
J. Hass, C. Heil, M. Weir.\textsl{Thomas' Cauculus} 14ed

% 线性代数
%=========================================
\bibitem{同济线}
同济大学数学系. \textsl{线性代数} 第五版
\bibitem{Axler}
Sheldon Axler. \textsl{Linear Algebra Done Right} 3ed

% 抽象代数
%=========================================
\bibitem{GTM242}
Pierre Antoine Grillet. \textsl{Abstract Algebra}, 2nd edition, GTM 242, Springer press. 

% 微分几何
%=========================================
\bibitem{梁书}
梁灿彬, 周彬. \textsl{微分几何与广义相对论} 第二版
\bibitem{陈斌}
陈斌. \textsl{广义相对论} 北京大学出版社(2018)第一版
\bibitem{GTM275}
Loring W. Tu. \textsl{Differential Geometry: Connections, Curvature, and Characteristic Classes}, GTM 275, Springer press. 
\bibitem{CarrollGR}
Sean M. Carroll, \textsl{Lecture Notes on General Relativity}, Institute for Theoretical Physics, UCSB, arXiv:gr-qc/9712019v1 3Dec 1997. 

% 表示论
\bibitem{GTM222}
Brian C. Hall. \textsl{Lie Groups, Lie Algebras, and Representations An Elementary Introduction}, GTM 222, Springer press. 
\bibitem{维声表示}
丘维声. \textsl{群表示论}. 高等教育出版社, 2011.12.
\bibitem{马中骐表示}
马中骐. \textsl{物理学中的群论}. 科学出版社, 2006.2.
\bibitem{表示新方法}
陈金全. \textsl{群表示论的新途径}. 上海科学出版社, 1984.8.
\bibitem{约什群论}
A.W.约什. \textsl{物理学中的群论基础}. 科学出版社, 1982.12.
\bibitem{韩孙群论}
韩其智,孙洪洲. \textsl{群论}. 北京大学出版社, 1987.2.
% 概率与统计
%=========================================
% 偏微分方程和特殊函数
%=========================================
\bibitem{Arfken}
Arfken, Weber, Harris. \textsl{Mathematical Methods for Physicists - A Comprehensive Guide} 7ed

% 数学分析
%=========================================
\bibitem{Cui1}
崔尚斌. \textsl{数学分析教程} (上中下三册). 科学出版社 (2013) 第一版
\bibitem{Rudin}
Walter Rudin. \textsl{Principle of Mathematical Analysis}

% 实变函数、实分析、广义函数
%=========================================
\bibitem{十一五实变函数论}
江泽坚,吴智泉,纪友清.\textsl{实变函数论} 第三版

% 泛函分析
%=========================================
\bibitem{Zeidler}
Eberhard Zeidler. \textsl{Applied Functional Analysis - Applications to Mathematical Physics}

% 力学
%=========================================
\bibitem{量纲}
梁灿彬, 曹周键. \textsl{量纲理论与应用} 第一版
\bibitem{Goldstein}
Herbert Goldstein. \textsl{Classical Mechanics} 3ed
\bibitem{新力}
赵凯华, 罗蔚茵. \textsl{新概念物理教程 力学} 第二版
% 电动力学
%=========================================
\bibitem{GriffE}
David Griffiths, \textsl{Introduction to Electrodynamics}, 4ed
\bibitem{新电}
赵凯华, 陈熙谋. \textsl{新概念物理教程 电磁学} 第二版
% 量子力学
%=========================================
\bibitem{GriffQ}
David Griffiths, \textsl{Introduction to Quantum Mechanics}, 4ed
\bibitem{曾谨言}
曾谨言.\textsl{量子力学} 第五版
\bibitem{新量}
赵凯华, 罗蔚茵. \textsl{新概念物理教程 量子物理},科学出版社,第二版
\bibitem{刘觉平}
刘绝平. \textsl{量子力学} 北京高等教育出版社, 2012.8
\bibitem{Shankar}
R. Shankar. \textsl{Principles of Quantum Mechanics} 2ed
\bibitem{Merzbacher}
Eugen Merzbacher. \textsl{Quantum  Mechanics} 3ed
\bibitem{Sakurai}
J.J. Sakurai. \textsl{Modern Quantum Mechanics} Revised Edition
\bibitem{Landau}
朗道. \textsl{理论物理教程 第三卷 量子力学(非相对论理论)} 第六版
\bibitem{Teschl}
Gerald Teschl. \textsl{Mathematical Methods in Quantum Mechanics}
\bibitem{高量}
李蕴才. \textsl{高等量子力学} 河南大学出版社, 2000:337-347
% 原子分子
%=========================================
\bibitem{黄昆}
黄昆. \textsl{固体物理学} 高等教育出版社
\bibitem{阎守胜}
阎守胜. \textsl{固体物理基础} 北京大学出版社 第三版
\bibitem{Bransden}
Bransden, \textsl{Physics of Atoms and Molecules}, 2ed
\bibitem{Burke}
Burke, \textsl{R-Matrix Theory of Atomic Collisions - Application to Atomic, Molecular and Optical Processes}
\bibitem{Newton}
Roger G. Newton, \textsl{Scattering Theory of Waves and Particles}, 2ed
% 统计力学
%=========================================
\bibitem{Schroeder}
Daniel V. Schroeder, \textsl{An Introduction to Thermal Physics}
\bibitem{热学}
秦允豪.\textsl{普通物理学教程 热学} 第三版
\bibitem{新热}
赵凯华, 罗蔚茵. \textsl{新概念物理教程 热学} 第二版
%=========================================
\bibitem{热统}
汪志诚.\textsl{热力学·统计物理}  第五版
% 宇宙学
%=========================================
\bibitem{Peebles}
P. Peebles \textsl{Principles of Physical Cosmology}
% 量子场论
%=========================================
\bibitem{Peskin}
Peskin. \textsl{An introduction To Quantum Field Theory}
\bibitem{SchwartzQFT}
Matthew D. Schwartz. \textsl{Quantum Field Theory and the Standard Model}, Cambridge University Press.
\bibitem{SusskindClassicalFields}
Leonard Susskind, Art Friedman. \textsl{Special Relativity and Classical Field Theory}, Basic Books; 1st Edition (September 26, 2017). ISBN-13: 978-0465093342; ISBN-10: 0465093345. 
\bibitem{WeinbergQFT1}
Steven L. Weinberg. \textsl{The Quantum Theory of Fields, Volume 1: Foundations}, Cambridge University Press; 1st Edition (May 9, 2005). ISBN-10: 0521670535; ISBN-13: 978-0521670531. 
\bibitem{WeinbergQFT2}
Steven L. Weinberg. \textsl{The Quantum Theory of Fields, Volume 1: Modern Applications}, Cambridge University Press; Illustrated edition (May 9, 2005). ISBN-10: 0521670543; ISBN-13: 978-0521670548. 
% 扭结与万物系列(Knots and Everything)
%=========================================
\bibitem{KnotsVol4}
John Baez, Javier P. Muniain. \textsl{Gauge Fields, Knots and Gravity, Series on Knots and Everthing}-Vol. 4, World Scientific press. ISBN-13: 978-981-02-2034-1. 
% 数值计算
%=========================================
\bibitem{NR3}
W. H. Press, et al. \textsl{Numerical Recipes} 3rd edition. 
\bibitem{唐计}
唐朔飞. 计算机组成原理[M]. 高等教育出版社. 2008
\bibitem{GDL}
I. Goodfellow, Y. Bengio, and A. Courville, Deep learning. MIT press, 2016.
\bibitem{DCGAN}
A. Radford, L. Metz, and S. Chintala, “Unsupervised representation learning with deep convolutional generative adversarial networks,” arXiv preprint arXiv:1511.06434, 2015.
% 其他
%=========================================
\bibitem{PhysWiki}
小时百科志愿者. \textsl{小时百科}. \href{https://wuli.wiki}{https://wuli.wiki}. 
\end{thebibliography}
