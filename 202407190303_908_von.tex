% 约翰·冯·诺依曼
% license Usr
% type Wiki

(本文根据 CC-BY-SA 协议转载自原搜狗科学百科对英文维基百科的翻译)

约翰·冯·诺伊曼(/vɒn ˈnɔɪmən/;匈牙利语:纽曼·雅诺斯·拉霍斯,发音;1903年12月28日-1957年2月8日)是匈牙利裔美国数学家、物理学家、计算机科学家和博学多才者。冯·诺伊曼被普遍认为是他那个时代最重要的数学家,[1] 并被称为“伟大数学家的最后代表”; 他是一个擅长融合纯科学和应用科学的天才。

他在许多领域做出了重大贡献,包括数学(数学基础、函数分析、遍历理论、表象理论、算子代数、几何、拓扑和数值分析)、物理(量子力学、流体力学和量子统计力学)、经济学(博弈论)、计算(冯·诺依曼架构、线性规划、自我复制机器、随机计算)和统计学。

他是将算子理论应用于量子力学发展函数分析的先驱,也是博弈论和细胞自动机概念、通用构造器和数字计算机发展的关键人物。

他一生发表了150多篇论文:大约60篇是纯数学论文,60篇是应用数学论文,20篇是物理论文,其余的是特殊数学或非数学学科的论文。[2]他的最后一部作品是他住院期间写的一份未完成的手稿,后来以书的形式出版了《计算机和大脑》。

他对自我复制结构的分析先于脱氧核糖核酸(DNA)结构的发现。在他提交给国家科学院(National Academy of Sciences)的一份罗列了他一生研究工作简短的清单表中,他说,“我认为最重要的研究工作是量子力学的研究,它是1926年在哥廷根发展起来的,随后又于1927-1929年在柏林发展。此外,我1930年在柏林大学和1935-1939年在普林斯顿大学期间,研究了各种形式的算子理论;1931-1932年在普林斯顿大学,我研究了遍历定理。”

第二次世界大战期间,冯·诺伊曼与理论物理学家爱德华·泰勒(Edward Taller)、数学家斯塔尼斯瓦夫·乌兰(Stanislaw Ulam)等人一起参与曼哈顿计划,解决热核反应和氢弹中核物理的关键步骤问题。他创建了内爆型核武器中使用的爆炸透镜背后的数学模型,并创造了“千吨”(梯恩梯TNT)一词,作为所产生爆炸力的量度。

战后,他在美国原子能委员会总顾问委员会任职,并为许多组织提供咨询,包括美国空军、陆军弹道研究实验室、武装部队特种武器项目和劳伦斯·利弗莫尔国家实验室。作为一名匈牙利移民,他担心苏联会取得核优势,于是设计并推行了相互保证毁灭的政策,以限制军备竞赛。

\subsection{早期生活和教育}
\subsubsection{1.1 家庭背景}
\begin{figure}[ht]
\centering
\includegraphics[width=6cm]{./figures/0570a56209846ea4.png}
\caption{图为冯·诺伊曼的出生地,布达佩斯巴索利街16号。自1968年以来,它一直是约翰·冯·诺依曼计算机协会的所在地。} \label{fig_von_1}
\end{figure}
冯·诺伊曼原名为诺伊曼·雅诺斯·拉霍斯,他生于一个富裕的、适应新文化的、不循规蹈距的犹太家庭(在匈牙利,姓氏排在第一位)。他的名字在英语中等同于约翰·路易斯。

冯·诺伊曼出生在匈牙利王国布达佩斯,当时匈牙利是奥匈帝国的一部分。[3][4] 他是三个兄弟中的老大;他的两个弟弟妹妹是米哈伊尔(英文:迈克尔·冯·诺依曼;1907-1989)和米克尔斯(尼古拉斯·冯·诺伊曼,1911-2011)。[5]他的父亲诺依曼·米卡(马克斯·冯·诺依曼,1873-1928)是一名银行家,拥有法学博士学位。19世纪80年代末,他从佩奇搬到布达佩斯。[6]米卡的父亲和祖父都出生在匈牙利北部泽姆普伦(Zemplen)县的翁德(Ond)(现在是泽伦茨(Szerencs)镇的一部分)。约翰的母亲是坎恩·玛姬特(英文:玛格丽特·坎恩)[7];她的父母是Jakab Kann和Katalin Meisels。[8]坎恩家族的三代人都住在布达佩斯坎恩-赫勒办公室上方的宽敞公寓里;冯·诺依曼一家住在顶层的一套有18个房间的公寓里。[9]

1913年2月20日,弗朗兹·约瑟夫皇帝将约翰的父亲提升为匈牙利贵族,以表彰他对奥匈帝国的贡献。诺伊曼家族因此获得了“玛吉塔(Margitta)”的世袭称谓,意思是“玛吉塔”(今天罗马尼亚,玛吉塔)。这个家庭与这个城镇没有任何关系;这个称呼是根据玛格丽特选择的,正如他们选择的描绘三个玛格丽特的纹章一样。诺伊曼·诺伊曼·雅诺斯后来改名为诺玛姬塔·诺依曼·雅诺斯(约翰·诺伊曼·德·玛姬塔),后来改为德国约翰·冯·诺伊曼。
\subsubsection{1.2 神童}
冯·诺依曼是个神童。当他6岁的时候,他能在脑子里进行两个8位数的除法[10][11] ,还能用古希腊语交谈。当6岁的冯·诺依曼发现母亲漫无目的地盯着他时,他问她,“你在盘算什么?”[12]

匈牙利的儿童直到10岁才开始正式上学;家庭女教师教导冯·诺依曼、他的兄弟和堂兄弟姐妹。马克斯认为除了匈牙利语之外,语言知识也很重要,所以孩子们接受了英语、法语、德语和意大利语的辅导[13]。到8岁时,冯·诺伊曼对微积分就很熟悉了,但他对历史特别感兴趣。他阅读了威廉·昂肯(Wimhelm Oncken)的46卷本《Allgemeine Geschichte in Einzeldarslellungen》。马克斯购买的一个私人图书馆里有一份副本。公寓里的一个房间被改造成了图书馆和阅览室,书架从天花板延伸到了地板。[14]

冯·诺伊曼于1911年加入路德教( Lutheran Fasori Evangélikus Gimnázium)。尤金·维格纳比冯·诺依曼早一年进入路德教会学校,并很快成为了他的朋友[15]。这是布达佩斯最好的学校之一,也是为精英设计的优秀教育体系的一部分。在匈牙利体制下,孩子们在一所体育馆接受所有教育。匈牙利的学校体系造就了以智力成就著称的一代人,其中包括西奥多·冯·卡尔曼(1881年)、乔治·德·赫维希(1885年)、迈克尔·波兰尼(1891年)、莱昂斯·西拉德(1898年)、丹尼斯·加博尔(1900年)、维格纳(1902年)、爱德华·泰勒(1908年)和保罗·erdős(1913年)。[16]这批人有时被称为“火星人”。[17]

虽然马克斯坚持冯·诺依曼在适合他年龄的年级上学,但他同意聘请私人教师,在他表现出天赋的领域给他高级指导。15岁时,他开始在著名分析师Gábor·szegő.的指导下学习高级微积分[15],他们第一次见面时,Szegő被这个男孩的数学天赋惊呆了,他激动地留下了眼泪。冯·诺依曼对Szegő在微积分中提出的问题的一些即时解决方案被描绘在他父亲的信纸上,现在仍在布达佩斯的冯·诺依曼档案馆展出[15]。到19岁时,冯·诺伊曼已经发表了两篇重要的数学论文,其中第二篇给出了序数的现代定义[18],取代了乔治·康托的定义。在体育馆结束学业后,冯·诺伊曼参加并获得了国家数学奖——厄茨奖。[19]
\begin{figure}[ht]
\centering
\includegraphics[width=10cm]{./figures/532c35616045e04b.png}
\caption{} \label{fig_von_2}
\end{figure}
\subsubsection{1.3 大学学习}
根据他的朋友西奥多·冯·卡门的说法,冯·诺依曼的父亲希望约翰跟随他进入工业界,从而把他的时间投入到比数学更有经济价值的工作中。事实上,他的父亲要求西奥多·冯·卡门说服他的儿子不要把数学作为他的专业。[20]冯·诺依曼和他的父亲决定最好的职业道路是成为一名化学工程师。冯·诺伊曼对此知之甚少,所以他被安排在柏林大学学习两年的非学位化学课程,之后他参加了著名苏黎世联邦理工学院(ETH Zurich)的入学考试[21],并于1923年9月通过了[22]。与此同时,冯·诺伊曼(von Neumann)也进入布达佩斯的帕兹曼·彼得大学,作为数学博士候选人。在论文中,他选择的主题为康托集合论的公理化。[23][24] 他于1926年从苏黎世联邦理工学院毕业,成为一名化学工程师(尽管魏格纳说冯·诺依曼从来就不太喜欢化学这门学科)[25],并在获得化学工程学位的同时通过了数学博士学位的最后的考试,其中魏格纳写道,“显然,博士论文和考试并没有给他带来太大的压力。”[25]然后,他在洛克菲勒基金会的资助下进入了哥廷根大学,在戴维·希尔伯特的指导下学习数学。[26]

\subsection{早期职业和私人生活}
\begin{figure}[ht]
\centering
\includegraphics[width=10cm]{./figures/66256a5ea6f7fc09.png}
\caption{摘自柏林弗里德里希·威廉姆斯大学1928年和1928-29年的大学日历,记载了诺伊曼关于公理集理论和数学逻辑,量子力学的新工作和数学物理学的特殊功能相关的演讲。} \label{fig_von_3}
\end{figure}
冯·诺伊曼在1927年12月13日完成了他的教授论文,并于1928年作为一名编外讲师在柏林大学开始了他的授课,[27]成为该大学历史上任何学科中最年轻的一位授课教师。[28]到1927年底,冯·诺伊曼已经发表了12篇数学专业学术论文,到1929年底,以每月近一篇专业学术论文的速度发表了32篇论文。[29]他具有很强的记忆和回忆的能力,这使他能够快速记忆电话簿的页面,并背诵其中的姓名、地址和号码。[30]1929年,他短暂地成为汉堡大学的私人教师,那里成为终身教授的前景更好[30],但同年10月,当他被邀请到新泽西州普林斯顿的普林斯顿大学时,一个更好的机会出现了。[31]

1930年的元旦,冯·诺伊曼娶了玛丽埃塔·柯维斯(Marietta Kovesi),她曾在布达佩斯大学学习经济学。[31]冯·诺依曼和玛丽埃塔有一个孩子,女儿玛丽娜,出生于1935年,截至2017年,她是密歇根大学工商管理和公共政策的杰出教授。[32]这对夫妇于1937年离婚。1938年10月,冯·诺伊曼与克拉拉·丹结婚,克拉拉·丹是他在二战爆发前最后一次回布达佩斯时遇到的。[33]

冯·诺依曼在与玛丽埃塔结婚之前,于1930年受洗成为天主教徒。[34]冯·诺依曼的父亲马克斯于1929年去世。马克斯在世时,家里没有一个人皈依基督教,但后来都皈依了。[35]

1933年,当新泽西高等研究院任命赫尔曼·韦勒为终身教授的计划落空时,他被授予终身教授职位。[36]他在那里一直担任数学教授,直到去世,尽管他已经宣布打算辞职,成为加州大学的一名讲座教授。[37]1939年,他的母亲、兄弟和姻亲跟随冯·诺依曼来到美国。[38]冯·诺依曼把他的名字英语化为约翰,保留了冯·诺依曼的德国贵族姓氏。他的兄弟们把他们的名字换成了“诺依曼”和“冯尼曼”。[39]冯·诺伊曼(Von Neumann)于1937年加入美国国籍,并立即试图成为美国陆军军官后备队的中尉。他轻而易举地通过了考试,但最终由于年龄原因被拒绝了。[39]他战前对法国如何对抗德国的分析经常被引用:“哦,法国不重要。”[40]

克拉拉和约翰·冯·诺依曼在当地学术界非常活跃。[41]他在韦斯特科特路26号的白色隔板房子是普林斯顿最大的私人住宅之一。[42]他非常注重自己的衣服,总是穿正式的西装。有一次,他骑着骡子穿过大峡谷时,身上穿了一套三件套的细条纹套装。[43]据说希尔伯特在冯·诺伊曼1926年的博士考试中曾问:“请问,这位候选人的裁缝是谁?”,因为他从未见过如此漂亮的晚礼服。[44]

冯·诺依曼一生热爱古代史,以其惊人的历史知识而闻名。普林斯顿的拜占庭历史教授曾说冯·诺伊曼在拜占庭历史方面比他更专业。[45]

冯·诺伊曼喜欢吃喝;他的妻子克拉拉说,除了卡路里,他什么都能计数。他喜欢意第绪语和“低俗的”幽默(尤其是打油诗)。[46]他不吸烟。[46]在普林斯顿,他因经常在留声机上播放极其响亮的德国进行曲而受到投诉,这些音乐让包括阿尔伯特·爱因斯坦在内的在邻近办公室工作的邻居们都无法专心工作。[47]冯·诺伊曼在嘈杂、混乱的环境中完成了一些最出色的工作,他曾经告诫他的妻子为他准备一间安静的书房,但他从来没有用过它,更喜欢在他们夫妇的客厅里大声放着电视。[48]尽管他是一个出了名的坏司机,但他仍然喜欢开车——经常是一边看书一边开车——这导致了多次被逮捕和多起交通事故。当库斯伯特·赫德雇用他为IBM公司的顾问时,赫德经常悄悄地为他的交通罚单支付罚款。[49]

冯·诺伊曼在美国最亲密的朋友是数学家斯塔尼斯瓦乌拉姆(Stanistlaw Ulam)。乌兰后来的一个朋友吉安-卡尔洛·罗塔写道,“他们会花几个小时没完没了地闲聊和咯咯笑,交换犹太笑话,并时不时地聊到数学话题。”当冯·诺依曼在医院里奄奄一息时,每次乌拉姆来访,他都会带着一套新的笑话来逗他开心。[50]他相信他的许多数学思想都是凭直觉产生的,他经常会带着一个未解决的问题睡觉,一醒来就能知道答案。[48]乌兰指出,冯·诺依曼的思维方式可能不是视觉的,而是听觉的。[51]

\subsection{数学}
\subsubsection{3.1 集合论}
\begin{figure}[ht]
\centering
\includegraphics[width=14.25cm]{./figures/8254322e64a7a820.png}
\caption{NBG集理论形成的方法的历史} \label{fig_von_4}
\end{figure}
在欧几里德(Euclid)的元素模型基础上,数学的公理化在19世纪末达到了新的严谨和广度水平,特别是在算术上,要归功于理查德·戴德金和查尔斯·桑德斯·皮尔士的公理化模式,在几何学上,这要归功于希尔伯特的公理化。[52]但是在20世纪初,将数学建立在朴素集合论基础上的努力由于罗素悖论(建立在不属于自己的所有集合的集合上)而受挫。[53]大约二十年后,恩斯特·策梅洛(Ernst Zermelo)和亚伯拉罕·弗兰肯(Abraham Fraenkel)暗中解决了集合论充分公理化的问题。zermelo–Fraenkel集合论提供了一系列原则,允许构建日常数学实践中使用的集合,但它们没有明确排除存在属于自身的集合的可能性。冯·诺依曼在1925年的博士论文中展示了排除这种集合的两种技术——基础公理和阶级概念。[52]

根据基础公理,每一个集合都可以通过策梅洛和弗兰肯的原则,从下到上,按照一系列有序的步骤来构造。如果一个集合属于另一个集合,那么第一个集合必须在连续的第二个集合之前。这排除了一个集合属于它自己的可能性。为了证明把这个新公理加到其他公理上并没有产生矛盾,冯·诺依曼引入了一种证明方法,称为内模法,后来它成为集合论中的一种基本工具。[52]

第二种解决属于自己的集合问题的方法以类的概念为基础,将集合定义为属于其他类的类,而适当的类定义为不属于其他类的类。在策梅洛-弗兰科尔(Zermelo-Fraenkel)方法下,公理阻碍了不属于自己的所有集合的构造。相反,在冯·诺依曼方法下,所有不属于自己的集合的类都可以被构造,但它是一个适当的类而不是集合。[52]

由于冯·诺依曼的这一贡献,集合理论的公理系统避免了早期系统的矛盾,并成为可用的数学基础,尽管缺乏其一致性的证明。下一个问题是,它是否能为所有可能在其中提出的数学问题提供明确的答案,或者是否可以通过添加更强有力的公理来改进它,这些公理可以用来证明更广泛的一类定理。后来,这个问题得到了强有力的否定回答,1930年9月,在历史悠久的柯尼希斯堡数学大会上,库尔特·哥德尔宣布了他的第一个不完全性定理:通常的公理系统是不完整的,就这个意义而言,它们不能证明用他们的语言表达的每一个真理。此外,这些系统的每一次持续扩展都必然是不完整的。[54]

过了不到一个月,参加大会的冯·诺依曼向哥德尔传达了他的定理的一个有趣的结果:通常的公理系统无法证明它们自己的一致性。[54]然而,哥德尔已经发现了这个结果,现在被称为他的第二个不完全性定理,他给冯·诺依曼寄去了一份他的文章的预印本,其中包含两个不完全性定理。[55]冯·诺伊曼在下一封信中承认哥德尔的优先地位。[56]他从来没有过多考虑“美国人事事以个人利益为重的制度”。[57]

\textbf{冯·诺依曼悖论}

在费利克斯·豪斯多夫工作的基础上,斯特凡·巴拿赫和阿尔弗雷德·塔尔斯基在1924年证明了在三维空间中给定一个实心球,球会分解成有限数量的不相交子集,这些子集可以用不同的方式重新组合在一起,产生原始球的两个完全相同的副本。巴纳赫和塔尔斯基证明,使用等距变换,将一个二维图形的拆开并重新组合,其面积必然与原始图形相同。这样就不可能把一个正方形变成两个正方形。然而,在1929年的一篇论文中,[58]冯·诺依曼证明了反常分解可以使用一组转换,其中包括一个带有两个生成器的自由群作为子群。面积保持变换的组包含这些子组,这提供了使用这些子组进行反常分解的可能性。冯·诺依曼在他关于巴拿赫-塔尔斯基分解的工作中分离出的一类群随后对数学的许多领域都非常重要,包括冯·诺依曼自己后来在测量理论方面的工作(见下文)。
\subsubsection{3.2 遍历理论}
在1932年发表的一系列著名论文中,冯·诺伊曼对遍历理论作出了基础性贡献。遍历理论是数学的一个分支,涉及具有不变测度的动力系统的状态。[59]在1932年关于遍历理论的论文中,保罗·哈尔莫斯写道,即使“冯·诺依曼从未做过任何其他事情,这也足以保证他在数学上不朽的声名”。[60]那时冯·诺依曼已经写了他著名的算子理论文章,这项工作的应用在冯·诺依曼平均遍历定理中起了重要作用。[60]
\subsubsection{3.3 算子理论}
冯·诺依曼通过冯·诺依曼代数引入了算子环的研究。冯诺依曼代数是希尔伯特空间上有界算子的*-代数,它在弱算子拓扑中是封闭的,并且包含恒等式算子。[61]冯·诺伊曼双突变定理表明,解析定义等同于纯代数定义,等于双突变体。[62]冯·诺依曼在1936年开始着手于冯·诺依曼代数的因子分类的一般性研究,并得到了默里的部分合作。1936年至1940年间,他在六篇重要论文中提出了这一理论,这六篇论文“跻身于二十世纪的分析杰作之列”。[63]约翰·冯·诺依曼后来在1949年引入了直接积分。[63]
\subsubsection{3.4 测度论}
在测度理论中,n维欧氏空间Rn的“测度问题”可以表述为:“Rn的所有子集的类上是否存在一个正的、归一化的、不变的和可加的集合函数?”[60]费利克斯·豪斯多夫和斯特凡·巴拿赫的工作己经暗示,在所有其他情况下,如果$n = 1$或$n = 2$,测度问题有一个正解,和一个负解(因为巴拿赫-塔尔斯基悖论)。冯·诺依曼的工作认为“问题本质上是群论性质的”:[60]一个测度的存在可以通过观察给定空间的变换群的性质来确定。至多两维空间的正解和更高维空间的负解来自于欧几里德群是至多两维空间的可解群这一事实,并且对于更高维空间是不可解的。“因此,根据冯·诺伊曼的观点,造成差异的是群体的变化,而不是空间的变化。”[60]

在冯·诺依曼的许多论文中,他使用的论证方法被认为比结果更有意义。冯·诺依曼在他以后对算子代数中维数理论的研究中,利用有限分解的等价结果,用函数重新表述了测度问题。[64]在他1936年关于分析测度理论的论文中,他在紧群情况下用哈尔定理求解希尔伯特第五问题。[60][65] 1938年,他因在分析方面的杰出研究而被授予伯歇纪念奖。[66]
\subsubsection{3.5 几何学}
冯·诺依曼创立了连续几何领域。[67]他继承了他对算子环的开创性工作。在数学中,连续几何是复杂射影几何的替代,其中子空间的维数不是离散集0,1,...,n,它可以是单位区间[0,1]的一个元素。在此之前,门格尔和伯克霍夫已经根据线性子空间格的性质将复射影几何公理化。冯·诺依曼在继续算子环方面的研究时,弱化了这些公理来描述更广泛的格类——连续几何。虽然射影几何子空间的维数是一个离散集(非负整数),但连续几何元素的维数可以在单位区间内连续变化。冯·诺依曼的研究动机是他发现了冯·诺依曼代数,其维数函数取连续的维数范围,而第一个不同于射影空间的连续几何的例子是超有限第二类因子的投影。[68][69]
\subsubsection{3.6 晶格理论}
在1937年到1939年间,冯·诺依曼致力于晶格理论的研究,这是一种半序集理论,其中每两个元素有一个最大的下界和一个最小的上界。加勒特·伯克霍夫写道:“约翰·冯·诺依曼的聪明才智像流星一样闪耀在晶格理论之上”。[70]

冯·诺伊曼(Von Neumann)对完备的补模拓扑格(内积空间子空间格中出现的性质)中的维数进行了抽象探索:“维数由以下两个性质决定,直到正线性变换。它由透视映射(“透视”)保存,并由包含物排序。证据的最深层部分涉及透视性与“分解投射性”的等价性——其推论是透视性的传递性。[70]

另外,“在一般情况下,冯·诺依曼证明了以下基本表示定理。任何具有n ≥ 4个成对透视元素“基”的补模格L,都与一个合适正则环r的所有主右理想的格ℛ(R)同构。这个结论是140页包含全新公理的精彩精辟代数的顶峰。任何想对冯·诺依曼头脑中的惊心动魄的才华留下难忘印象的人,只需努力为自己追寻这一系列精确的推理埸景——想象那个天才常常穿着浴袍坐在客厅里的写字台旁,随手写下五页充满智慧的想法。”[70]
\subsubsection{3.7 量子力学的数学公式}
冯·诺依曼在1932年的著作《量子力学的数学基础》中首次为量子力学建立了严格的数学框架,即狄拉克-冯·诺依曼公理。[64]在完成集合论的公理化之后,他开始面对量子力学的公理化。他在1926年意识到,量子系统的状态可以用(复杂的)希尔伯特空间中的一个点来表示,一般来说,即使是单个粒子,这个点也可以是无限维的。在量子力学的这种形式中,位置或动量等可观测的量被表示为作用于与量子系统相关的希尔伯特空间的线性算子。[71]

量子力学的物理学因此被简化为希尔伯特空间和作用于其上的线性算子的数学。例如,根据不确定性原理,粒子位置的确定阻止了动量的确定(反之亦然),被转化为两个相应算子的不可交换性。这个新的数学公式作为特例包括海森堡和薛定谔的公式。[71]当海森堡被告知冯·诺依曼已经阐明了作为自伴算子的无界算子和仅仅是对称算子之间的区别时,海森堡回答道:“嗯?有什么区别?”[72]

冯·诺依曼的抽象处理也使他能够面对决定论和非决定论的基本问题,他在书中提出了一个证据,证明量子力学的统计结果不可能像经典统计力学那样是一组确定的“隐藏变量”的平均值。1935年,格雷特·赫尔曼(Grete Hermann)发表了一篇论文,认为该证明包含一个概念性错误,因此无效。[73]赫尔曼的工作在基本上被忽视了,直到约翰·贝尔在1966年提出了本质上相同的论点。[74]然而,在2010年,杰弗里·布博(Jeffrey Bub)认为贝尔误解了冯·诺依曼的证明,并指出该证明虽然并不适用于所有的隐藏变量理论,但确实排除了一个定义明确且重要的子集。布博还指出冯·诺依曼意识到了这一局限性,冯·诺依曼并没有声称他的证明完全排除了隐藏变量理论。[75]反过来,Bub的论点的有效性也有争议。[76]无论如何,格里森1957年的理论填补了冯·诺依曼方法的空白。

冯·诺依曼的证明开创了一条研究路线,最终通过贝尔在1964年研究的贝尔定理,以及阿兰Aspect在1982年的实验,证明量子物理要么需要一个与经典物理学完全不同的现实概念,要么必须包含明显违反狭义相对论的非局域性。[77]

在《量子力学的数学基础》一章中,冯·诺依曼深入分析了所谓的测量问题。他得出结论,整个物理宇宙可以服从宇宙波函数。由于需要“计算之外”的东西来瓦解波函数,冯·诺依曼得出结论,瓦解是由实验者的意识引起的。冯·诺依曼认为量子力学的数学允许波函数的崩溃被放置在从测量装置到人类观察者的“主观意识”的因果链中的任何位置。尽管这一观点被尤金·维格纳(Eugene Wigner)所接受[78],但冯·诺伊曼-魏格纳的解释从未得到大多数物理学家的认可。[79]冯·诺伊曼-魏格纳的解释概述如下:

量子力学的规则是正确的,但是只有一个系统可以用量子力学来处理,那就是整个物质世界。存在着量子力学无法处理的外部观察者,即人类(或许还有动物)的大脑,他们对大脑进行测量,导致波函数崩溃。

尽管量子力学的理论一直发展到今天,但量子力学存在一个数学形式主义难题的基本框架,它是大多数方法的基础,可以追溯到冯·诺伊曼最先使用的数学形式主义和技术。换句话说,关于理论解释及其扩展的讨论,现在大多是在对数学基础的共同假设的基础上进行的。[64]

\textbf{冯·诺依曼熵}

冯诺依曼熵在量子信息论框架中以不同的形式(条件熵、相对熵等)得到了广泛的应用。其纠缠度测量是基于与冯·诺依曼熵直接相关的一些量。给定具有密度矩阵的量子力学系统的统计系综$\rho$ ,它来源于$S(p)=-Tr(\rho ln\rho经典信息论中许多相同的熵测度也可以推广到量子情况,如霍尔沃熵和条件量子熵。