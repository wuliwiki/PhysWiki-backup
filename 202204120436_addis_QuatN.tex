% 四元数与旋转矩阵
% 线性代数|矩阵|绕轴旋转矩阵|旋转矩阵|四元数|基底|基底变换矩阵

\pentry{旋转矩阵的导数\upref{RotDer}, 四元数\upref{Quat}, 罗德里格旋转公式、绕轴旋转矩阵\upref{RotA}}

四元数可以用来简洁地表示三维空间中的旋转,极大地减少了计算量. 所有模长为 $1$ 的四元数 $q$ 与所有可能的(绕原点的)三维旋转\upref{RotA}存在一一对应关系.

三维空间中的一个向量表示为标部为 $0$ 的四元数:$\pmat{0, \bvec{v}}$, 简记为 $\bvec v$. 如果我们绕着一个单位向量 ${\uvec{n}}$ 把 $\bvec{v}$ 旋转一个角度 $\theta$,所得的结果应该是哪个向量呢? 任意模长为 $1$ 的四元数都可以表示为
\begin{equation}
q = \qty(\cos{\frac{\theta}{2}},\ {\uvec{n}}\sin{\frac{\theta}{2}})
\end{equation}
那么旋转后的向量 $\bvec v'$ 就可以表示为
\begin{equation}\label{QuatN_eq4}
\bvec v' = q\bvec vq^{-1} = q\bvec v \tilde q
\end{equation}
其中 $q^{-1}$ 表示 $q$ 的逆, $\tilde q$ 表示 $q$ 的共轭(\autoref{Quat_def2}~\upref{Quat}), 满足 $qq^{-1} = q^{-1}q = (1,0,0,0)$, 易证
\begin{equation}
q^{-1} = \tilde q = (\cos{\frac{-\theta}{2}},\ {\uvec{n}}\sin{\frac{-\theta}{2}})
= (\cos{\frac{\theta}{2}},\ -{\uvec{n}}\sin{\frac{\theta}{2}})
\end{equation}
表示绕 $\uvec n$ 旋转 $-\theta$.

例如当 $\theta = 0$ 时, $q = (1,0,0,0)$ 是单位元, $q\bvec vq^{-1} = \bvec v$. 又例如取 $\bvec v = (0, 1, 0, 0)$,它代表一个 $x$ 轴上的单位向量.如果想要把它绕着 $z$ 轴上的单位向量转 $\pi/2$,那么结果向量的四元数表示应该是 $(0, 0, 1, 0)$.按照我们定义的规则,旋转表示为四元数 $q=(\sqrt{2}/2, 0, 0, \sqrt{2}/2)$, $q^{-1} = \tilde q = (\sqrt{2}/2,0,0,-\sqrt{2}/2)$, 按照四元数的乘法规则易得 $q\bvec v q^{-1}=(0,0,1,0)$.

\subsubsection{证明}
直接计算\autoref{QuatN_eq4} 得(使用连续叉乘的化简\upref{TriCro} 以及三角降幂公式\upref{TriEqv}):
\begin{equation}
\begin{aligned}
q\bvec vq^{-1}&= (\cos{\frac{\theta}{2}}, {\uvec{n}}\sin{\frac{\theta}{2}})\cdot (0, \bvec{v})\cdot q^{-1}\\&
=(-\bvec{v}\cdot {\uvec{n}}\sin{\frac{\theta}{2}}, \bvec{v}\cos{\frac{\theta}{2}}+ {\uvec{n}}\times\bvec{v}\sin{\frac{\theta}{2}})\cdot (\cos{\frac{\theta}{2}}, -{\uvec{n}}\sin{\frac{\theta}{2}})\\&
= (0,\ \cos\theta\ \bvec v + (1-\cos\theta)\uvec n(\uvec n\vdot \bvec v) + \sin\theta\ \uvec n\cross\bvec v)
\end{aligned}
\end{equation}
这正是直接使用罗德里格旋转公式(\autoref{RotA_eq1}~\upref{RotA})使 $\bvec{v}$ 绕 ${\uvec{n}}$ 旋转 $\theta$ 的结果.

一点补充:考虑到四元数限定向量方向时退化为复数(见\textbf{四元数}\upref{Quat}的讨论),故 $\bvec{v}\parallel{\uvec{n}}$ 的情况其实可以直接套用复数的交换性来得到,即 $qvq^{-1}=qq^{-1}v=v$.

再来证明 $q$ 和三维旋转的一一对应关系: 若 $q_1,q_2$ 表示同一个旋转, 那么对任意 $\bvec v$ 都有
\begin{equation}
q_1\bvec v \tilde q_1 = q_2\bvec v \tilde q_2
\end{equation}
也就是对任意 $\bvec v$ 都有
\begin{equation}
\bvec v = (\tilde q_1q_2)\bvec v \widetilde{(\tilde q_1q_2)}
\end{equation}
所以必定有 $\tilde q_1q_2 = 1$, 即 $q_2 = \tilde q_1^{-1} = q_1$. 证毕.

\subsection{旋转的复合}
四元数的乘法运算(\autoref{Quat_eq1}~\upref{Quat})可以把两次旋转合并为一次旋转. 注意 $q_2q_1$ 表示先做 $q_1$ 的旋转, 再做 $q_2$ 的旋转. 另外易证 $(q_2q_1)^{-1} = q_1^{-1} q_2^{-1} = \tilde q_1\tilde q_2$.

由于一般来说四元数乘法不满足交换律, 所以两次旋转也一般不能交换, 除非两次转动的转轴(两四元数的矢部)共线.

\textbf{证明}: 令 $\bvec v$ 经过 $q_1$ 旋转变为 $\bvec v'$, 再经过 $q_2$ 旋转变为 $\bvec v''$, 则
\begin{equation}
\bvec v'' = q_2\bvec v'q_2^{-1} = q_2(q_1\bvec vq_1^{-1}) q_2^{-1} = (q_2q_1)\bvec v(q_1^{-1} q_2^{-1}) = (q_2q_1)\bvec v(q_2q_1)^{-1}
\end{equation}
证毕.

\subsection{旋转矩阵}
令四元数 $q = (s, \bvec v)$ 表示绕 $\uvec v$ 方向以右手定则\upref{RHRul}旋转 $\theta$ 的矩阵, 其中
\begin{equation}
s = \cos\frac{\theta}{2} \qquad
\abs{\bvec v} = \sin\frac{\theta}{2}
\end{equation}
则对应的三维旋转矩阵(\autoref{RotA_eq3}~\upref{RotA})可以表示为
\begin{equation}\label{QuatN_eq2}
\mat R =
\begin{pmatrix}
1 - 2v_y^2 - 2v_z^2 & 2v_xv_y - 2sv_z  & 2v_x v_z + 2s v_y\\
2v_x v_y + 2sv_z & 1 - 2v_x^2 - 2v_z^2 & 2v_y v_z - 2s v_x\\
2v_x v_z - 2s v_y & 2v_y v_z + 2s v_x & 1 - 2v_x^2 - 2v_y^2
\end{pmatrix}
\end{equation}
特殊地, $\theta = 0$ 时, 四元数为 $(1,0,0,0)$, 对应单位矩阵, 无任何旋转. 另外由上一节的分析可知, 若 $q_i$ 的旋转矩阵为 $\mat R_i$, 那么 $q_1q_2$ 对应的旋转矩阵就是 $\mat R_1\mat R_2$.

\subsection{时间导数}
若从坐标系 $B$ 到坐标系 $A$ 的基底变换矩阵为 $\mat R$, 当 $B$ 相对于 $A$ 绕原点以角速度 $\bvec \omega$ 旋转时有(\autoref{RotDer_eq4}~\upref{RotDer})
\begin{equation}
\dot{\mat R} = \mat \Omega \mat R
\end{equation}
其中 $\mat\Omega$ 乘以任意位置矢量 $\bvec r$ 等于 $\bvec \omega \cross \bvec r$
\begin{equation}
\mat \Omega = \pmat{
0 & -\omega_z & \omega_y\\
\omega_z & 0 & -\omega_x\\
-\omega_y & \omega_x & 0
}\end{equation}
若旋转矩阵 $\mat R$ 对应的四元数为 $q(t)$, 则
\begin{equation}\label{QuatN_eq1}
\dot {q} = \frac12 (0, \bvec \omega) q
\end{equation}
这里的 $\dot{q}$ 表示对四个标量中的每个分别求导.

\textbf{推导}: 把 ${q}(t+\Delta t)$ 分解为 ${q}(t)$ 和 ${q}_1(\Delta t)$ 两次旋转, 
\begin{equation}\label{QuatN_eq3}
\dot {q}
= \lim_{\Delta t\to 0}\frac{{q}(t+\Delta t) - {q}(t)}{\Delta t}
= \lim_{\Delta t\to 0}\frac{{q}_1(\Delta t) - (1,0,0,0)}{\Delta t}q(t)
\end{equation}
其中 ${q}_1(t)$ 相当于以瞬时角速度 $\bvec \omega$ 旋转时间 $t$
\begin{equation}
{q_1}(t) = (\cos\frac{\omega t}{2},\ \ \uvec \omega \sin\frac{\omega t}{2})
\end{equation}
注意 $q_1(0) = (1,0,0,0)$, 所以\autoref{QuatN_eq3} 中的极限就是
\begin{equation}
\lim_{t\to 0}\frac{{q}_1(t) - {q}_1(0)}{\Delta t} = \dot{q}_1(0) = \frac{1}{2}(0, \bvec\omega)
\end{equation}
代入\autoref{QuatN_eq3} 得\autoref{QuatN_eq1}.
