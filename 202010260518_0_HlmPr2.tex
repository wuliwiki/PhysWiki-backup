% 亥姆霍兹定理证明 2
% keys 散度|旋度

令 $\bvec F$ 为无散场, 即
\begin{equation}
\div \bvec F = 0
\end{equation}
则 $\bvec F(\bvec r)$ 总能表示为另一个矢量场 $\bvec G(\bvec r)$ (不唯一)的旋度, 即
\begin{equation}
\bvec F = \curl \bvec G
\end{equation}
且任意 $\bvec G(\bvec r)$ 都可以表示为
\begin{equation}
\bvec G = \int \bvec F \cross \frac{\bvec R}{R^3} \dd[3]{r'} + \bvec H(\bvec r)
\end{equation}
其中 $\bvec H(\bvec r)$ 是一个任意的无旋场($\curl \bvec H = \bvec 0$), $\bvec R = \bvec r' - \bvec r$, $R = \abs{\bvec R}$. 则有
\begin{equation}
\bvec F = \curl \bvec G
\end{equation}
