% Python RoboMaster EP 教程—初始化机器人
% keys Robomaster|机器人
% license Usr
% type Tutor

在进行与机器人相关的操作之前,需要先初始化机器人对象

首先从 robomaster 包中导入 robot 模块:

\begin{lstlisting}[language=python]
from robomaster import robot
\end{lstlisting}

当机器人设置为组网模式时需要手动指定 RoboMaster SDK 的本地ip地址,在本示例中,查询得到本地的ip地址为 192.168.2.20 (比如在Windows操作系统下,通过 快捷键 Win + R 调出的窗口中输入 cmd, 然后在 CMD窗口中输入 ipconfig , 即可以查看设备ip的信息), 指定ip使用以下语句:

\begin{lstlisting}[language=python]
robomaster.config.LOCAL_IP_STR = "192.168.2.20"
\end{lstlisting}

大部分情况下SDK能够自动获取正确的本地ip,无需手动指定这一步骤,但是当SDK运行在多网卡同时使用的设备时, 自动获取的ip可能不是与机器人进行连接的ip,此时需要手动指定ip
