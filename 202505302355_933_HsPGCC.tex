% 射影几何视角下的圆锥曲线(高中)
% keys 射影几何|对偶原理|圆锥曲线
% license Usr
% type Tutor

\begin{issues}
\issueDraft
\end{issues}

\pentry{圆锥曲线的统一定义\nref{nod_HsCsFD}}{nod_029c}

之前用“焦点–准线”来定义圆锥曲线的\enref{探索}{HsCsFD}中,已经把椭圆、双曲线和抛物线统一在一个框架下了,不过还有两个问题尚待解决:
\begin{enumerate}
\item 既然用抛物线的定义统一了椭圆和双曲线,那么反过来,能不能也把抛物线纳入椭圆和双曲线的定义方式中?
\item 在推导的过程中,曾经出现过“焦点在无穷远处”或者“准线在无穷远处”这样的说法。到底是什么意思?无穷远并不是一个具体的点,那怎么理解这种说法?
\end{enumerate}

在高中的数学体系中,这类问题很难得到直接解释。但在一个叫“射影几何”的视角下,它们都可以变得清晰起来。虽然射影几何属于超纲内容,大多数教材和教师也并不涉及,但它恰恰能提供一种更完整的方式,把所有圆锥曲线背后的联系统一起来。特别是“焦点–准线”的定义为什么能够覆盖所有圆锥曲线,以及前面提到的两个问题,在这个视角下能够看得更透彻。

这一篇会简单介绍射影几何中最关键的想法,不涉及太多技术细节。就算只是作为一次短暂接触的初步了解,也能为整个圆锥曲线的结构提供一个更完整的背景,同时感受到数学统一性的魅力。

\subsection{平行线可以相交吗?}

从初中就开始接触的“几何”,其实指的是\textbf{欧几里得几何(Euclidean Geometry,欧氏几何)}的体系。这个体系的规则听上去非常自然,比如:
\begin{itemize}
\item 两条平行线永远不会相交;
\item 一条直线可以无限延伸;
\item 平面上的每一个点都表示一个具体的位置。
\end{itemize}
这些看似理所当然的规则,其实是从人们的日常经验中抽象总结出来的。整个中学阶段的几何学习——无论是平面几何、解析几何,还是立体几何和函数图像,都是在这套逻辑体系之下进行的。

可是真实世界中一些肉眼可见的现象,却似乎在挑战这种“常识”。比如,在铁轨之间远眺时,两条原本平行的铁轨,看起来竟然在远方逐渐靠近,最后仿佛汇聚到一个点上;又比如仰头站在高楼下,两面垂直上升的墙体,也好像在天际会合于某个点。

\begin{figure}[ht]
\centering
\includegraphics[width=11cm]{./figures/aa409536797b866f.png}
\caption{平行轨道逐渐靠拢} \label{fig_HsCsFD_2}
\end{figure}

\begin{figure}[ht]
\centering
\includegraphics[width=11cm]{./figures/040cd6e71df42276.png}
\caption{仰望高楼时墙体汇合} \label{fig_HsCsFD_3}
\end{figure}

有人可能会说,这只是视觉上的错觉而已。但问题在于,这种现象几乎每个人都能看见。如果希望一个几何系统能够更贴近人眼所看到的真实景象,就必须接受一个看起来有些“违反常识”的结论——平行线是可以相交的。而这就已经超出了欧氏几何所能描述的范围,毕竟,“平行线永不相交”被写入第五条公理,是整个欧几里得几何体系的重要基础。一旦这条公理不再成立,人们自然会感到不安:如果连几何的根基都被动摇了,这个体系还能继续成立吗?

事实上,早在文艺复兴时期,这类现象在艺术领域就已经引起了注意。15世纪初的艺术家 \textbf{布鲁内莱斯基(Filippo Brunelleschi)}从绘画的角度出发,首次提出了线性透视法。这种方法通过把视觉上平行线的“交点”设定为“消失点”来还原远近关系,让观者仿佛能够“走进”画中。自此,绘画不再是平面的拼图,而成为可以表现深度和距离的“窗口”。

线性透视法的广泛应用,不仅改变了艺术的表达方式,也促使数学家重新思考几何的基础。事实上,在欧氏几何中,许多定理都需要分开讨论“相交”与“平行”的情况,这让整个体系在表达上显得不够统一,也缺乏美感。这种不够协调的逻辑结构,也促使数学家开始尝试突破原有框架。18世纪,数学家 \textbf{鲍耶伊(Giovanni Girolamo Saccheri)}试图证明第五公设可以从其他公设中推导出来,但最终没有成功。然而,正是这次失败,意外地打开了通往一个全新几何世界的大门:\textbf{非欧几何(Non-Euclidean Geometry)}由此诞生。

后面要介绍的\textbf{射影几何(projective geometry)}就描绘了“画家眼中的几何”,作为非欧几何的一种,它不再着眼于长度或角度等度量信息,而是更关注图形之间的位置、交点、\textbf{对齐性(collinearity)}\footnote{“对齐性”指的是若干点共线(在同一条直线上),或者若干直线共点(交于一点)的性质。}等关系。

\subsection{射影几何的基本结构}

为了将前面的提到的消失点现象完整地纳入新的几何体系,使这些看似“违反规则”的现象变得合理,甚至可以加以推理和计算,这个新的体系就必须要囊括下前面得到的这些特点:

\begin{itemize}
\item 无穷远点是不可忽略的,或者说存在的;
\item 当站在一个点向无穷远看去时,所有的无穷远点构成一条直线(也就是\autoref{fig_HsCsFD_2} 中的地平线。);
\item 平行线可以相交,交点在无穷远处。
\end{itemize}

在这个新的视角下,很多在欧式几何中被当作例外的事情,就变得自然了。比如,平行线不再是“永远无法相交”的例外,而是会在无穷远处相遇;“无穷远的点”也不再是虚幻的想象,而是被认为确实存在,只不过位置在我们看不到的远方。从射影几何的角度看,“看不见”并不等于“不存在”,那些在欧式几何中被忽略的部分,反而是统一一切几何形状的关键。

就像欧氏几何在平面上研究一样,射影几何也基于前面的特点设计了一个研究空间,称作\textbf{射影平面(Projective Plane)}。而既然有了前面的要点就可以知道,它必然要引入的一个关键的构想就是\textbf{无穷远点(points at infinity)}。于是直线不再是无限延伸的,而是停止在了无穷远点上。也就是沿着直线走向无穷远,最终到达的那个点。

原本的欧氏平面$\mathbb{R}^2$ 上,是不包含所谓无穷远点的。就像实数或数轴上没有无穷远点,也就是$\mathbb{R}=(-\infty,+\infty)$,可是有时为了研究方便希望一个集合里面有无穷大,因此,数学家们设计了一个新的集合称作\textbf{超实数(Hyperreal number)}$^*\mathbb{R}$,在超实数里就有无穷大的位置。射影几何的处理方法也很像:为了更好地理解图像中的“消失点”、平行线的汇聚和空间的透视效果,它在原有平面之外,增加了那些“看不见却有意义”的无穷远点,从而构成了一个新的几何空间。

(这里需要介绍一下,可以用球极投影(球面去掉一个极点投影到平面)来直观理解,球面上的对径点(如南极和北极)对应于射影平面上的同一个点(无穷远点)。)

单纯加入无穷远点,只是加入了一个点而已,看上去好像并没有什么特别的。重要的是如何确定点和其他元素的关系,射影几何引入的关系就是:所有方向一致的直线都在一个“无穷远点”相交,也就是同一个方向上的所有平行线都交于同一个无穷远点。而一个平面上的方向是连续变化的,所以不同方向的平行线对应了连续的一些点,于是,这就对应了之前所说的地平线那样的,所有无穷远点构成一条特殊的直线,称作\textbf{无穷远直线(Line at infinity)},记作$l^\infty$。



\addTODO{无穷远点,无穷远直线,如何想象?是圆吗?}

在射影几何中,引入了“无穷远点”和“无穷远线”(例如所有平行线在“无穷远点”相交),这是因为我们不关心距离和角度,只关心交点关系(即拓扑结构/位置结构)。




射影几何给展示了:
	•	同一个对象可以从不同的角度理解;
	•	表面看起来不同的东西,背后可能有统一的结构;
	•	有时,必须打破一些“习惯的规则”,才能看到更完整的图景。

有趣的是,在这样定义之后,在射影几何中,点和直线可以互换、对称对待;也就是说,直线也可以看成是“由点组成”的,点也可以像直线一样进行变换。

3. 射影变换(Projective Transformations)
	•	在射影几何中,图形可以通过射影变换相互变换而不改变其本质特征。
	•	不保长不保角,但保对点、对直线的交点关系。
	•	交点不变性是圆锥曲线统一的重要基础。
射影变换	矩阵表示、不保长角、对交点不变

\subsection{对偶原理}

在十九世纪,法国数学家腾塞叶(Jean-Victor Poncelet)在俄国战俘营中写下了他的重要著作《论图形的摄影性质》。在这部作品中,他首次系统提出了“对偶原理”与“投影不变性”这两个深刻的几何思想。所谓对偶原理,是指在射影几何中,点与直线可以互换,互换后许多几何命题依然成立。例如,“两点决定一条直线”的命题,对偶后变成了“两条直线决定一个交点”,这两者在射影几何中都同样成立。这种点与线之间的对称关系,揭示了几何结构中隐藏的深层对称性,使人们重新思考“几何事实”背后的逻辑构造。

与此同时,腾塞叶还指出,一些几何性质在投影变换下是保持不变的。换句话说,即使我们改变观察角度或从不同平面进行投影,某些关系仍旧成立,这被称为“投影下的不变性”。比如,共线的点经过中心投影后仍然共线;一个圆在透视下可能变成椭圆、抛物线或双曲线,但这些曲线本质上都是圆锥曲线,因此在射影几何中是等价的。这一思想打破了古典欧几里得几何中对“形状”的执着,把几何研究的焦点从“看上去的样子”转向了“结构中的本质”。

到了二十世纪,随着公理化几何的发展,数学家们进一步发现:在许多几何定理中,把“点共线”换成“线共点”、把“点”换成“直线”后,新的表述仍然成立。这些互换后的命题不仅不是偶然巧合,而是源自整个射影几何体系中点与线的对等地位。对偶原理的提出不仅丰富了几何的思维方式,也为代数几何、拓扑学、以及更现代的数学分支奠定了基础。在这种视角下,几何的研究不再只是对现实图形的模仿,而是一种对空间逻辑结构的深刻把握。


\subsection{齐次坐标}
例如,所有竖直线在无穷远直线ℓ∞上相交于同一个点 [0:1:0] (在齐次坐标下)。
\subsection{射影几何视角下的圆锥曲线}

之前提到过,圆锥曲线的定义依赖于度量(焦点、准线)或与平面的截取角度。

尤其是当在统一定义中使用了“准线”这个概念时,会发现一个问题:

准线是直线,而焦点是点,它们的地位并不对等。

比如,在抛物线中,焦点和准线之间的距离决定了曲线的开口程度;但在椭圆和双曲线中,焦点之间的关系常常比准线更显眼。这种“不对等”让难以一眼看出统一性。

要解决这个问题,需要让“点”和“直线”变得对等、互换,这正是射影几何擅长的。
	5.	展示“锥面截平面”的统一来源
	6.	引出:在射影几何中它们其实是一类图形
	7.	提出五点定圆锥曲线,引发兴趣

\subsubsection{圆锥曲线的射影统一定义}

圆锥曲线的统一定义	为什么五点能唯一确定一个圆锥曲线


下面这几种定义是等价的,你不必都用,但至少理解两种:
	•	锥面截平面定义(传统几何背景)
	•	二次方程组定义(代数角度)
	•	五点确定一圆锥曲线(几何公理性)
	•	交比保持性(更高级,辅助理解)

这里就能回答之前的问题了,其实两个焦点也好,焦点与准线也好,本身是等同的。所以自然,可以通过添加另一个焦点来把抛物线也纳入椭圆和双曲线的定义体系里。

抛物线的焦点在无穷远处。
具体的统一定义不给出了。

	•	抛物线、椭圆、双曲线在欧氏几何中有不同定义。
	•	射影几何中,三者都可以看作是锥面截平面所成的截线,变换后可以互相转化。
	•	在射影几何中,这三类曲线合称为圆锥曲线(conics),不再区分为“开口”、“闭合”。

有的读者可能会疑惑,那椭圆和双曲线怎么一样?明明一个是开放的一个是封闭的。它们是从射影几何的角度看,是同一种几何对象,只是我们以前习惯从度量角度区分它们。”

“原本认为是圆锥在平面上的截线,可是如果把圆锥本身看作是投影面,那么圆锥曲线其实就是(另一个)平面的(二次)曲线在该圆锥面上的投影。”

1. 经典视角(欧氏几何):

圆锥为源,平面为接收屏: 固定一个圆锥(二次锥面),然后用一个平面(“截平面”)去切割它。相交得到的截线就是圆锥曲线(椭圆/抛物线/双曲线)。
过程: 圆锥 (3D) → 截面 → 平面曲线 (2D)。
2. 你的洞察(射影几何视角):

平面为源,圆锥为“投影幕布”: 固定一个平面(称为 π),然后将其“投影”到一个圆锥面上。投影后落在圆锥面上的“像”就是圆锥曲线。
过程: 平面 π (2D) → 投影 → 圆锥上的曲线 (3D曲面上的曲线)。
关键: 你敏锐地指出,如果选择合适的投影中心(通常是圆锥的顶点 O)和投影方式(中心投影),那么平面 π 上的曲线投影到圆锥面上形成的曲线,恰好等于用另一个特定的平面(称为 σ)去截该圆锥所得的截线!也就是说:
π上的曲线 --(以O为中心投影到圆锥面)--> 圆锥面上的曲线 = 平面σ截圆锥所得的曲线
为什么这个视角如此重要和正确?

揭示了内在统一性:
经典视角认为圆锥曲线是圆锥(固定)被不同平面(变动)切割产生的不同“碎片”。
你的视角则表明,圆锥曲线也可以看作是平面(变动)在固定圆锥(作为“屏幕”)上的投影结果。
核心洞见: 无论是“切割”还是“投影”,最终得到的曲线在射影几何意义下是等价的。它们都是由二次方程定义的曲线,并且可以通过射影变换相互转化。
射影变换的本质:
射影几何的核心研究对象是在中心投影下保持不变的性质。
将平面 π 投影到圆锥面上(以 O 为中心),再将圆锥面展开或投影到另一个平面 σ 上,这个过程本身就是一个射影变换(更精确地说,是由两个中心投影复合而成的射影变换)。
因此,平面 π 上的任何图形(特别是圆锥曲线)与其在平面 σ 上的“像”(即圆锥截线),在射影几何中被视为同一个对象的不同表现形式! 它们拥有相同的射影性质(如交比、Pascal定理成立性等)。
圆锥作为“中介”的二次性:
一个平面(一次几何对象)通过一个固定点 O(顶点)的中心投影映射到另一个平面 σ 上,这个映射(射影变换)会把平面 π 上的直线映射成平面 σ 上的直线。
然而,圆锥面是一个二次曲面。将平面 π 投影到圆锥面上:
平面 π 上的一条直线,在中心投影下,会映射为圆锥面上的一条直母线(过 O 的直线在圆锥面上)。
但是,平面 π 上的一个圆(或其他圆锥曲线),在中心投影下,会映射为圆锥面上的一条空间曲线。这条空间曲线正是圆锥面与某个平面的交线(即圆锥曲线本身)。
关键点: 你的视角“圆锥曲线其实就是平面的投影”中,“平面”指的是 π 平面上的某条曲线(最终也是圆锥曲线)在圆锥面上的像。而“投影”过程利用了圆锥面这个二次曲面作为“幕布”,最终呈现的像(圆锥面上的曲线)天然满足二次约束。


射影几何中的视角使能够用一种统一且优雅的方式看待圆锥曲线。但在射影几何中,这些差异被看作是坐标选择与观察角度所导致的表象变化,它们在更本质的层面上是一类对象的不同表现:它们都是圆锥曲线。圆锥曲线不是三类不同的曲线,而是一个统一的几何实体的三种视角。它让跳出了直观图形的束缚,从结构上理解几何对象之间的联系,也为代数几何、复几何乃至更高维的几何打下了坚实的基础。

从射影几何的角度看,圆锥曲线定义为一个圆锥面与一个平面相交的轨迹。这个定义在欧几里得空间中也成立,但射影几何更进一步地指出:在射影平面中,所有非退化的圆锥曲线都是射影等价的。这意味着可以通过一个合适的射影变换(即坐标的线性变换加上归一化),将任意一个圆锥曲线变为另一个圆锥曲线——比如将一个椭圆变为一个双曲线或抛物线。

换句话说:
\begin{itemize}
\item 椭圆是在射影平面中与无穷远直线没有实交点的圆锥曲线;
\item 双曲线是在射影平面中与无穷远直线有两个实交点的圆锥曲线;
\item 抛物线是恰好与无穷远直线有一个交点的极限情形。
\end{itemize}

这种分类在射影几何中失去了意义,因为无穷远直线被作为与其他直线同等地位来处理,不再是“例外的部分”。因此,抛物线、椭圆和双曲线不再是本质不同的几何对象,而只是一个对象的不同投影或表示。


为什么 $e=1$ 是一个“分界线”?

离心率为什么只分成这三类,而不是连续变化出更多种曲线?

四、重新看焦点和准线:变换下的对等性

回到的统一定义:

到焦点距离与到准线距离的比值等于 $e$

焦点是一个点,准线是一个直线,它们是不一样的。但在射影几何中,可以把直线看成是“一个方向上的点的集合”,特别是在加入了“无穷远点”之后,直线也可以被看作是特殊的“点”。

这就让焦点和准线,在某种意义上变得“对等”。

更重要的是:

在射影几何中,通过变换,可以把一个圆锥曲线变换成另一种类型的圆锥曲线,只要它们满足相同的基本结构。

举个例子:


一个椭圆,通过一个适当的“射影变换”,可以变成一个抛物线;
抛物线也可以变成双曲线;
这些变换不会改变圆锥曲线的“本质”,只改变它在眼中的“样子”。

这就说明,射影几何的世界中,圆锥曲线是一个统一的整体,而不是几种各自孤立的图形。


在射影几何中,圆锥曲线的统一定义并不依赖焦点–准线,而是:

所有与一条圆锥面相交的平面交线,在射影平面中都是圆锥曲线。其本质是一个二次齐次方程在 $\mathbb{P}^2$ 中的零点集合。

但——

焦点–准线结构仍然可以嵌入射影几何中,你可以这样理解:
	•	准线可以是一个射影直线;
	•	焦点是一个射影点;
	•	离心率可以通过某种射影不变量(例如交比)来表达。

在射影几何中,“一个点到一条线的比值”不再有意义,但你可以通过共轭二次曲线、交比等射影结构,重新定义出类似“焦点–准线”的行为。

1. “双焦点”与“单焦点”的统一
	•	问题:我们习惯使用两个焦点描述椭圆/双曲线,而这里却只用一个焦点和一条准线,为什么可以?
	•	解读:“另一个焦点”其实可以看作是准线的对偶,它不是必须的,只是在对称性下自然出现。

2. 对偶结构初探
	•	可引入投影几何中“点–直线对偶”的思想,让学生意识到准线和焦点在某种意义下可以互换角色。

\subsubsection{圆锥曲线中的对偶性}


	8.	介绍“极与极线”的奇妙对偶结构
极与极线	如何从几何角度构造对偶

此外,射影几何还强调了极点与极线的对偶性,并引入了极线极点变换的工具来研究圆锥曲线的性质,使得很多命题具有了对称且优美的形式。例如:对于一个给定的圆锥曲线,任意一点都有与之对应的一条极线,反之亦然。这种对偶关系在欧氏几何中并不自然存在。

圆锥曲线的\nref{极线与极点}{EclPol}是从射影几何的角度研究圆锥曲线的精髓,具体说来每个圆锥曲线可以定义一种点与线之间的对偶关系(极点 ↔ 极线)。至于共轭点、对称结构、切线群等现象,就是另外的故事了。


