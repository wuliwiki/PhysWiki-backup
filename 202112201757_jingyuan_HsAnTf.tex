% 三角恒等变换(高中)
% 高中|三角恒等变换

\begin{issues}
\issueDraft
\end{issues}

\subsection{两个基本公式}
\begin{equation}
\sin^2\alpha + \cos^2\alpha = 1
\end{equation}
\begin{equation}
\frac{\sin\alpha}{\cos\alpha} = \tan\alpha
\end{equation}

\subsection{两角和与两角差}
\begin{equation}
\sin(\alpha + \beta) = \sin\alpha \cos\beta + \cos\alpha \sin\beta
\end{equation}
\begin{equation}
\sin(\alpha - \beta) = \sin\alpha \cos\beta - \cos\alpha \sin\beta
\end{equation}
\begin{equation}
\cos(\alpha + \beta) = \cos\alpha \cos\beta - \cos\alpha \cos\beta
\end{equation}
\begin{equation}
\cos(\alpha - \beta) = \cos\alpha \cos\beta + \cos\alpha \cos\beta
\end{equation}
\begin{equation}
\tan(\alpha + \beta) = \frac{\tan\alpha+\tan\beta}{1-\tan\alpha \tan\beta}
\end{equation}
\begin{equation}
\tan(\alpha - \beta) = \frac{\tan\alpha - \tan\beta}{1+\tan\alpha \tan\beta}
\end{equation}

\subsection{二倍角公式}
\begin{equation}
\sin2\alpha = 2\sin\alpha \cos\beta
\end{equation}
\begin{equation}
\cos2\alpha = \cos^2\alpha - \sin^2\alpha = 1 - 2\sin^2\alpha = 2\cos^2\alpha -1
\end{equation}
\begin{equation}
\tan2\alpha = \frac{2\tan\alpha}{1-\tan^2\alpha}
\end{equation}

\subsection{半角公式}
\begin{equation}
\sin\frac{\alpha}{2} = \pm\sqrt{\frac{1-\cos\alpha}{2}}
\end{equation}
\begin{equation}
\cos\frac{\alpha}{2}= \pm\sqrt{\frac{1+\cos\alpha}{2}}
\end{equation}
\begin{equation}
\tan\frac{\alpha}{2} = \pm\sqrt{\frac{1-\cos\alpha}{1+\cos\alpha}} = \frac{\sin\alpha}{1+\cos\alpha} = \frac{1-\cos\alpha}{\sin\alpha}
\end{equation}

\subsection{升幂公式}
\begin{equation}
\cos2\alpha + 1 = 2\cos^2\alpha
\end{equation}
\begin{equation}
1-\cos2\alpha = 2\sin^2\alpha
\end{equation}

\subsection{降幂公式}
\begin{equation}
\cos\alpha = \pm\sqrt{\frac{1+2\cos2\alpha}{2}}
\end{equation}
\begin{equation}
\sin\alpha = \pm\sqrt{\frac{1-2\cos2\alpha}{2}}
\end{equation}

\subsection{万能公式}
\begin{equation}
\sin2\alpha = \frac{2\sin\alpha \cos\alpha}{\sin^2\alpha + \cos^2\beta} = \frac{2\tan\alpha}{1+\tan^2\alpha}
\end{equation}
\begin{equation}
\cos2\alpha = \frac{\cos^2\alpha-\sin^2\alpha}{\sin^2\alpha+\cos^2\alpha} = \frac{1-\tan^2\alpha}{1+\tan^2\alpha}
\end{equation}

\subsection{辅助角公式}
\begin{equation}
a\sin\alpha + b\cos\alpha = \sqrt{a^2+b^2}\sin(\alpha + \phi)
\end{equation}
注: $\tan\phi = \frac{b}{a}$

\subsection{证明}
\subsubsection{两角和与两角差}
\begin{figure}[ht]
\centering
\includegraphics[width=14.25cm]{./figures/HsAnTf_1.png}
\caption{图示} \label{HsAnTf_fig1}
\end{figure}
设 $\alpha$、$\beta$ 对应的单位向量分别为 $a(\cos\alpha,\sin\alpha)$、$b(\cos(\alpha+\beta),\sin(\alpha+\beta))$\\
设 $a$ 与其垂直的单位向量 $m(-\sin\alpha,\cos\alpha)$ 为基向量,则
\begin{equation}
\begin{aligned}
b &= \cos\beta \cdot a + \sin\beta \cdot m \\
&= (\cos\beta \cos\alpha,\cos\beta \sin\alpha) + (-\sin\beta \sin\alpha,\sin\beta \cos\alpha) \\
&= (\cos\alpha \cos\beta-\sin\alpha \sin\beta,\sin\alpha \cos\beta + \cos\alpha \sin\beta)
\end{aligned}
\end{equation}
由此可得
\begin{equation}
\sin(\alpha+\beta) = \cos\alpha \cos\beta-\sin\alpha \sin\beta
\end{equation}
\begin{equation}
\cos(\alpha+\beta) = \sin\alpha \cos\beta + \cos\alpha \sin\beta
\end{equation}
\subsubsection{二倍角公式}
在两角和公式中,令 $\alpha = \beta$
\begin{equation}
\sin2\alpha = \sin\alpha\cos\alpha+\cos\alpha\sin\alpha = 2\sin\alpha\cos\alpha
\end{equation}