% 类氢原子的跃迁偶极子矩阵

\pentry{氢原子的选择定则\upref{SelRul}}

\autoref{SelRul_eq3}~\upref{SelRul}
\begin{equation}
\begin{aligned}
&\quad\mel{\psi_{n,l,m}}{\bvec{\mathcal E}\vdot\bvec r}{\psi_{n',l',m'}} = \bvec{\mathcal E}\vdot\mel{\psi_{n,l,m}}{\bvec r}{\psi_{n',l',m'}}\\
&= \sqrt{\frac{4\pi}{3}}\int R_{n,l}(r) r R_{n',l'}(r) r^2 \dd{r}\times\\
&\Big[\frac{E_x}{\sqrt 2} \qty(\mel{Y_{l,m}}{Y_{1,-1}}{Y_{l',m'}} - \mel{Y_{l,m}}{Y_{1,1}}{Y_{l',m'}})\\
&+ \frac{\I E_y}{\sqrt 2} \qty(\mel{Y_{l,m}}{Y_{1,-1}}{Y_{l',m'}} + \mel{Y_{l,m}}{Y_{1,1}}{Y_{l',m'}})\\
&+  E_z \mel{Y_{l,m}}{Y_{1,0}}{Y_{l',m'}}\Big]
\end{aligned}
\end{equation}
其中三个球谐函数之积的积分为\autoref{SphHar_eq5}~\upref{SphHar}
\begin{equation}\label{HDipol_eq1}
\mel{Y_{l,m}}{Y_{1,m_1}}{Y_{l',m'}} = (-1)^m\sqrt{\frac{3(2l+1)(2l'+1)}{4\pi}} \pmat{l & 1 & l'\\ 0 & 0 & 0}\pmat{l & 1 & l'\\ -m & m_1 & m'}
\end{equation}
得
\begin{equation}
\begin{aligned}
&\quad\mel{\psi_{n,l,m}}{\bvec{\mathcal E}\vdot\bvec r}{\psi_{n',l',m'}}\\
&= (-1)^m\sqrt{(2l+1)(2l'+1)} \pmat{l & 1 & l'\\ 0 & 0 & 0} \int R_{n,l}(r) r R_{n',l'}(r) r^2 \dd{r}\times\\
&\Big\{\frac{E_x}{\sqrt 2} \qty[\pmat{l & 1 & l'\\ -m & -1 & m'} - \pmat{l & 1 & l'\\ -m & 1 & m'}]\\
&+ \frac{\I E_y}{\sqrt 2} \qty[\pmat{l & 1 & l'\\ -m & -1 & m'} + \pmat{l & 1 & l'\\ -m & 1 & m'}]\\
&+  E_z \pmat{l & 1 & l'\\ -m & 0 & m'}\Big\}
\end{aligned}
\end{equation}
