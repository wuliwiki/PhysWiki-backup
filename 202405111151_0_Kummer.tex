% Kummer 函数(1F1)
% keys 合流超几何函数
% license Xiao
% type Tutor
\begin{issues}
\issueDraft
\end{issues}

\footnote{见 NIST \href{https://dlmf.nist.gov/13.2}{相关页面}。}\textbf{Kummer 函数} $M(a, b, z)$, 也叫合流超几何函数 $_1F_1(a, b; z)$(有时省略两个“$1$” 的下标,直接写作 $F(a, b; z)$), 是以下微分方程的解
\begin{equation}\label{eq_Kummer_1}
z\dv[2]{f}{z} + (b-z)\dv{f}{z} - a f = 0~.
\end{equation}
特别的,当 $1 - b \neq \mathbb Z$ 时方程有另一解
$$z^{1-b} {_1F_1}(a-b+1, 2-b; z)~.$$

Kummer 函数是\enref{超几何函数}{HypGeo}的一个特例, 
\begin{equation}
M(a, b, z) = {_1F_1}(a, b; z) = \sum_{n=0}^\infty \frac{(a)_n}{(b)_n} \frac{z^n}{n!}~.
\end{equation}
其中 $(a)_n = a(a+1)\dots(a+n-1)$, 叫做 \textbf{Pochhammer 符号}(等价于上升幂)。

${_1F_1}$ 还存在一个积分表示
\begin{equation}
{_1F_1}(a, b; z) = \frac{\Gamma(b)}{\Gamma(a) \Gamma(b-a)} \int_0^1 {e^{zt} t^{a-1} (1-t)^{b-a-1}\dd t} ~,
\end{equation}
其中 $\Re b > \Re a > 0$。

\subsection{合流超几何函数的应用}
合流超几何函数在类氢原子束缚态的薛定谔方程的\textbf{径向波函数}的解中很常见,例如\autoref{eq_HWF_11}~\upref{HWF}:
\begin{equation}
\dv[2]{u}{\rho} + \qty[-1 - \frac{2\eta}{\rho} - \frac{l(l+1)}{\rho^2}]u = 0~.
\end{equation}
考察这微分方程在 $\rho \rightarrow 0$ 时的情形。此时变为
\begin{equation}
\dv[2]{u}{\rho} - \frac{l(l+1)}{\rho^2} u = 0 ~.
\end{equation}
从而 $u$ 的通解是 $u(\rho) = C_1 \rho^{l+1} + C_2 \rho^{-l}$ 的形式。而在 $\rho \rightarrow 0$ 时 $\rho^{-l} \rightarrow \infty$,故必然有 $C_2 = 0$。类似地,考察其在 $\rho \rightarrow \infty$ 时的情形有渐进解 $\exp(-\rho/2)$。故可以将 $u$ 的解写作
\begin{equation}
u(\rho) = \rho^{l+1} \exp(-\rho/2) y(\rho) ~,
\end{equation}
的形式。

将 $y$ 代入会发现新大陆:
\begin{equation}
\rho \dv[2]{y}{\rho} + (c-\rho)\dv{y}{\rho} -ay = 0 ~.
\end{equation}
$a$ 和 $c$ 是带入后计算出的常数。这微分方程恰好符合\autoref{eq_Kummer_1} 的形式。这方程有通解
\begin{equation}
y(\rho) = A_{} {}_{1}F_1(a, c; \rho) + B \rho^{1-c}{}  _{1}F_{1}(a-c+1, 2-c; \rho) ~.
\end{equation}
类似的考虑 $\rho\rightarrow \infty$ 时的情况,应有 $B=0$。从而有 $u$ 的解可以表示为
\begin{equation}
u(\rho) = A \rho^{l+1} \exp(-\rho/2) {_1F_1}(a, c; \rho) ~~
\end{equation}
的形式。
