% 哈密顿力学(综述)
% license CCBYSA3
% type Wiki

本文根据 CC-BY-SA 协议转载翻译自维基百科\href{https://en.wikipedia.org/wiki/Hamiltonian_mechanics}{相关文章})

\begin{figure}[ht]
\centering
\includegraphics[width=6cm]{./figures/7110c2a74929e25b.png}
\caption{威廉·罗恩·哈密顿爵士} \label{fig_HMD_1}
\end{figure}
在物理学中,哈密顿力学是拉格朗日力学的重新表述,起源于1833年。由威廉·罗恩·哈密顿爵士提出【1】,哈密顿力学用(广义)动量替代了拉格朗日力学中使用的(广义)速度 \( \dot{q}^i \)。这两种理论都提供了对经典力学的解释,并描述了相同的物理现象。

哈密顿力学与几何学(特别是辛几何和泊松结构)有密切关系,并且作为经典力学与量子力学之间的纽带。
\subsection{概述}  
\subsubsection{相空间坐标 \( (p, q) \) 和哈密顿量 \( H \)}  
设 \( (M, \mathcal{L}) \) 为一个具有构型空间 \( M \) 和光滑拉格朗日量 \( \mathcal{L} \) 的力学系统。选择 \( M \) 上的标准坐标系 \( (\boldsymbol{q}, \boldsymbol{\dot{q}}) \)。量 \( p_i(\boldsymbol{q}, \boldsymbol{\dot{q}}, t) \stackrel{\text{def}}{=} \partial \mathcal{L} / \partial \dot{q}^i \) 称为动量(也称为广义动量、共轭动量或正则动量)。对于时间瞬间 \( t \),拉格朗日量 \( \mathcal{L} \) 的勒让德变换定义为映射 \( (\boldsymbol{q}, \boldsymbol{\dot{q}}) \to (\boldsymbol{p}, \boldsymbol{q}) \),假设其具有光滑逆映射 \( (\boldsymbol{p}, \boldsymbol{q}) \to (\boldsymbol{q}, \boldsymbol{\dot{q}}) \)。对于具有 \( n \) 个自由度的系统,拉格朗日力学定义了能量函数
\[
E_{\mathcal{L}}(\boldsymbol{q}, \boldsymbol{\dot{q}}, t) \stackrel{\text{def}}{=} \sum_{i=1}^{n} \dot{q}^i \frac{\partial \mathcal{L}}{\partial \dot{q}^i} - \mathcal{L} ~.
\]
拉格朗日量 \( \mathcal{L} \) 的勒让德变换将 \( E_{\mathcal{L}} \) 转化为称为哈密顿量的函数 \( \mathcal{H}(\boldsymbol{p}, \boldsymbol{q}, t) \)。哈密顿量满足:
\[
\mathcal{H}\left(\frac{\partial \mathcal{L}}{\partial \boldsymbol{\dot{q}}}, \boldsymbol{q}, t\right) = E_{\mathcal{L}}(\boldsymbol{q}, \boldsymbol{\dot{q}}, t)~
\]
这意味着:
\[
\mathcal{H}(\boldsymbol{p}, \boldsymbol{q}, t) = \sum_{i=1}^{n} p_i \dot{q}^i - \mathcal{L}(\boldsymbol{q}, \boldsymbol{\dot{q}}, t),~
\]
其中速度 \( \boldsymbol{\dot{q}} = (\dot{q}^1, \dots, \dot{q}^n) \) 从 \( \boldsymbol{p} = \partial \mathcal{L} / \partial \boldsymbol{\dot{q}} \) (一个 \( n \) 维方程组)中得出,假设它对于 \( \boldsymbol{\dot{q}} \) 的解是唯一的。\( ( \boldsymbol{p}, \boldsymbol{q} ) \)(一个 \( 2n \) 维对)称为相空间坐标(也称为正则坐标)。
\subsubsection{从欧拉-拉格朗日方程到哈密顿方程 } 
在相空间坐标 \( (\boldsymbol{p}, \boldsymbol{q}) \) 中,\( n \) 维的欧拉-拉格朗日方程
\[
\frac{\partial \mathcal{L}}{\partial \boldsymbol{q}} - \frac{d}{dt} \frac{\partial \mathcal{L}}{\partial \boldsymbol{\dot{q}}} = 0~
\]
变为 \( 2n \) 维的哈密顿方程:
\[
\frac{d \boldsymbol{q}}{dt} = \frac{\partial \mathcal{H}}{\partial \boldsymbol{p}}, \quad \frac{d \boldsymbol{p}}{dt} = -\frac{\partial \mathcal{H}}{\partial \boldsymbol{q}}.~
\]
\textbf{证明}  

哈密顿量 \( \mathcal{H}(\boldsymbol{p}, \boldsymbol{q}) \) 是拉格朗日量 \( \mathcal{L}(\boldsymbol{q}, \boldsymbol{\dot{q}}) \) 的勒让德变换,因此有:
\[
\mathcal{L}(\boldsymbol{q}, \boldsymbol{\dot{q}}) + \mathcal{H}(\boldsymbol{p}, \boldsymbol{q}) = \boldsymbol{p} \cdot \boldsymbol{\dot{q}}~
\]
因此
\[
\partial \mathcal{H}/\partial \boldsymbol{p} = \boldsymbol{\dot{q}}, \quad \partial \mathcal{L}/\partial \boldsymbol{q} = -\partial \mathcal{H}/\partial \boldsymbol{q}.~
\]
此外,由于 \( \boldsymbol{p} = \frac{\partial \mathcal{L}}{\partial \boldsymbol{\dot{q}}} \),欧拉-拉格朗日方程给出
\[
d \boldsymbol{p}/dt = \partial \mathcal{L}/\partial \boldsymbol{q} = -\partial \mathcal{H}/\partial \boldsymbol{q}.~
\]
\subsubsection{从驻定作用量原理到哈密顿方程 } 
设 \( \mathcal{P}(a, b, \boldsymbol{x}_a, \boldsymbol{x}_b) \) 为满足 \( \boldsymbol{q}(a) = \boldsymbol{x}_a \) 和 \( \boldsymbol{q}(b) = \boldsymbol{x}_b \) 的光滑路径集合 \( \boldsymbol{q}: [a, b] \to M \)。作用泛函 \( \mathcal{S}: \mathcal{P}(a, b, \boldsymbol{x}_a, \boldsymbol{x}_b) \to \mathbb{R} \) 定义为
\[
\mathcal{S}[\boldsymbol{q}] = \int_a^b \mathcal{L}(t, \boldsymbol{q}(t), \dot{\boldsymbol{q}}(t)) \, dt = \int_a^b \left(\sum_{i=1}^n p_i \dot{q}^i - \mathcal{H}(\boldsymbol{p}, \boldsymbol{q}, t)\right) \, dt,~
\]
其中 \( \boldsymbol{q} = \boldsymbol{q}(t) \),且 \( \boldsymbol{p} = \partial \mathcal{L} / \partial \boldsymbol{\dot{q}} \)(见上文)。若路径 \( \boldsymbol{q} \in \mathcal{P}(a, b, \boldsymbol{x}_a, \boldsymbol{x}_b) \) 为 \( \mathcal{S} \) 的驻点(因此满足运动方程),则相空间坐标 \( (\boldsymbol{p}(t), \boldsymbol{q}(t)) \) 必须满足哈密顿方程。
\subsubsection{基本物理解释}  
对哈密顿力学的简单解释来自其在由质量为 \( m \) 的单个非相对论粒子组成的一维系统中的应用。此情况下,哈密顿量 \( H(p, q) \) 的值是系统的总能量,即动能和势能之和,分别传统地表示为 \( T \) 和 \( V \)。其中 \( p \) 是动量 \( mv \),而 \( q \) 是空间坐标。则有:
\[
\mathcal{H} = T + V, \quad T = \frac{p^2}{2m}, \quad V = V(q)~
\]
动能 \( T \) 仅为 \( p \) 的函数,而势能 \( V \) 仅为 \( q \) 的函数(即 \( T \) 和 \( V \) 是非时间依赖的)。

在此示例中,\( q \) 的时间导数是速度,因此第一哈密顿方程意味着粒子的速度等于其动能对动量的导数。动量 \( p \) 的时间导数等于牛顿力,因此第二哈密顿方程意味着力等于势能的负梯度。
\subsection{示例}   
球面摆由一个质量为 \( m \) 的物体组成,其在球面上无摩擦地运动。作用在该质量上的唯一力是球面的反作用力和重力。使用球坐标 \( (r, \theta, \phi) \) 来描述该质量的位置,其中 \( r \) 是固定的,\( r = \ell \)。
\begin{figure}[ht]
\centering
\includegraphics[width=7cm]{./figures/d809868bb20c7be5.png}
\caption{球面摆:角度和速度。} \label{fig_HMD_2}
\end{figure}
该系统的拉格朗日量为【2】:
\[
L = \frac{1}{2}m\ell^2\left(\dot{\theta}^2 + \sin^2 \theta \, \dot{\varphi}^2\right) + mg\ell \cos \theta.~
\]
因此,哈密顿量为
\[
H = P_{\theta} \dot{\theta} + P_{\varphi} \dot{\varphi} - L~
\]
其中
\[
P_{\theta} = \frac{\partial L}{\partial \dot{\theta}} = m\ell^2 \dot{\theta}~
\]
和
\[
P_{\varphi} = \frac{\partial L}{\partial \dot{\varphi}} = m\ell^2 \sin^2 \theta \, \dot{\varphi}.~
\]
用坐标和动量表示时,哈密顿量为
\[
H = \underbrace{\left[\frac{1}{2}m\ell^2 \dot{\theta}^2 + \frac{1}{2}m\ell^2 \sin^2 \theta \, \dot{\varphi}^2 \right]}_{T} + \underbrace{\left[-mg\ell \cos \theta \right]}_{V} = \frac{P_{\theta}^2}{2m\ell^2} + \frac{P_{\varphi}^2}{2m\ell^2 \sin^2 \theta} - mg\ell \cos \theta.~
\]
哈密顿方程给出了坐标和共轭动量的时间演化,这些是一组四个一阶微分方程:
\[
\begin{aligned}
\dot{\theta} &= \frac{P_{\theta}}{m\ell^2}, \\
\dot{\varphi} &= \frac{P_{\varphi}}{m\ell^2 \sin^2 \theta}, \\
\dot{P_{\theta}} &= \frac{P_{\varphi}^2}{m\ell^2 \sin^3 \theta} \cos \theta - mg\ell \sin \theta, \\
\dot{P_{\varphi}} &= 0.
\end{aligned}~
\]
动量 \( P_{\varphi} \) 对应于角动量的垂直分量 \( L_z = \ell \sin \theta \times m\ell \sin \theta \, \dot{\varphi} \),是一个守恒量。这是系统围绕垂直轴旋转对称性的结果。由于方位角 \( \varphi \) 不出现在哈密顿量中,因此它是一个循环坐标,这意味着其共轭动量守恒。
\subsection{推导哈密顿方程}  
哈密顿方程可以通过拉格朗日量 \( \mathcal{L} \)、广义位置 \( q^i \) 和广义速度 \( \dot{q}^i \) 的计算推导出来,其中 \( i = 1, \ldots, n \)【3】。这里我们在“非壳”情况下工作,意味着 \( q^i \)、\( \dot{q}^i \)、\( t \) 是相空间中的独立坐标,不受任何运动方程的约束(特别是 \( \dot{q}^i \) 不是 \( q^i \) 的导数)。拉格朗日量的全微分为:
\[
\mathrm{d} \mathcal{L} = \sum_{i} \left( \frac{\partial \mathcal{L}}{\partial q^i} \, \mathrm{d} q^i + \frac{\partial \mathcal{L}}{\partial \dot{q}^i} \, \mathrm{d} \dot{q}^i \right) + \frac{\partial \mathcal{L}}{\partial t} \, \mathrm{d} t.~
\]
广义动量坐标定义为 \( p_i = \frac{\partial \mathcal{L}}{\partial \dot{q}^i} \),因此我们可以将方程改写为:
\[
\mathrm{d} \mathcal{L} = \sum_{i} \left( \frac{\partial \mathcal{L}}{\partial q^i} \, \mathrm{d} q^i + p_i \, \mathrm{d} \dot{q}^i \right) + \frac{\partial \mathcal{L}}{\partial t} \, \mathrm{d} t = \sum_{i} \left( \frac{\partial \mathcal{L}}{\partial q^i} \, \mathrm{d} q^i + \mathrm{d} (p_i \dot{q}^i) - \dot{q}^i \, \mathrm{d} p_i \right) + \frac{\partial \mathcal{L}}{\partial t} \, \mathrm{d} t.~
\]
重新排列后得到:
\[
\mathrm{d} \left(\sum_{i} p_i \dot{q}^i - \mathcal{L}\right) = \sum_{i} \left( -\frac{\partial \mathcal{L}}{\partial q^i} \, \mathrm{d} q^i + \dot{q}^i \, \mathrm{d} p_i \right) - \frac{\partial \mathcal{L}}{\partial t} \, \mathrm{d} t.~
\]
左侧括号中的项即为之前定义的哈密顿量 \( \mathcal{H} = \sum p_i \dot{q}^i - \mathcal{L} \),因此:
\[
\mathrm{d} \mathcal{H} = \sum_{i} \left( -\frac{\partial \mathcal{L}}{\partial q^i} \, \mathrm{d} q^i + \dot{q}^i \, \mathrm{d} p_i \right) - \frac{\partial \mathcal{L}}{\partial t} \, \mathrm{d} t.~
\]