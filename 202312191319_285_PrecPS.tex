% 进动:旋转的陀螺为什么不会倒(科普)
% keys 陀螺|章动|进动|旋转|经典力学
% license Usr
% type Art


不旋转的陀螺,尖端朝下放在桌上无法立住,因为几乎不可能让陀螺完美对称且完美竖直,于是在重力矩作用下就会倾倒。一旦陀螺旋转起来,尽管重力矩依然要拉着陀螺倾倒,但陀螺似乎只是在不停改变旋转轴的方向,一直不倾倒。

考虑一个空塑料瓶,称从它的瓶盖到瓶底的轴为其“长轴”。将塑料瓶扔到空中,使得其长轴有翻转的角速度,你会发现塑料瓶的长轴大致在一个平面内不停翻转,瓶盖的指向变化范围也差不多有$360^\circ$。但如果扔的同时让塑料瓶本身还绕着长轴旋转,那么塑料瓶的长轴就会在一个双圆锥面内反转,瓶盖变化范围显著小于$360^\circ$。

上述两个现象有一个共同的名字:进动。简单来说,进动就是指,旋转的刚体在旋转轴方向变化的时候会受到一个回转力矩,从而改变其旋转轴变化的方向。我们通过受力分析就能理解进动的成因。


\subsection{术语准备}


为了方便接下来的讨论,我需要定义一些概念,这样能大大简化描述和理解的难度。


首先确定研究对象:一个固定在轻质杆上的均质圆环,



































