% 常微分方程解的存在、唯一及对参数的连续依赖定理
% keys 存在|唯一|连续依赖|常微分方程
% license Xiao
% type Tutor

\pentry{皮卡映射\upref{PicMap},基本知识(常微分方程)\upref{ODEPr}}
本节证明常微分方程的解的存在、唯一、及对参数的依赖定理。其是对形如
\begin{equation}\label{eq_ODEUC_1}
\dot x=v(x,\mu,t)~
\end{equation}
的常微分方程组进行证明的。其中写成含有参数 $\mu$ 的形式是为了证明解对参数是连续依赖的,只需假定在某个固定参数 $\mu$ 下就能把 $v(x,\mu,t)$ 写成 $v(x,t)$,这就回到了以前常见的形式。一般的常微分方程都可以写成\autoref{eq_ODEUC_1} 的形式(\upref{GO2SOD}),所以对形如\autoref{eq_ODEUC_1} 证明的定理具有普适性。
\subsection{存在、唯一及对参数的连续依赖定理}
\begin{theorem}{存在、唯一及对参数的连续依赖定理}
设微分方程
\begin{equation}
\dot x=v(x,\mu,t)~
\end{equation}
中向量场 $v$ 是定义在其自变量所在欧氏空间中的区域 $\tilde\Gamma$ 上的连续可微映射,对任一点 $(x_0,\mu_0,t_0)\in\tilde\Gamma$
\end{theorem}
