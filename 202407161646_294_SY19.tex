% 中山大学 2019 年研究生入学物理考试试题
% keys 中山大学|考研|物理|2019年
% license Copy
% type Tutor


\textbf{声明}:“该内容来源于网络公开资料,不保证真实性,如有侵权请联系管理员”
\subsection{简答题}
\begin{enumerate}
\item “牛顿第二定律”的含义是什么?它的基础是什么?
\item  Mayer 公式$C_P-C_v=R>0$,请用热力学第一定律解释它的物理意义。
\item 长条形的电介质在外电场中极化后,两端出现等量异号电荷。若把它截成两半后并撤去外电场问,这两个半截的电介质是否带电?为什么?
\item 为什么窗玻璃在日光照射下我们观察不到于涉条纹?
\end{enumerate}
\subsection{计算题}
\begin{enumerate}
\item 有一半圆形的光滑槽,质量为$ M$,半径为$R$,放在光滑的水平面上。一个小物体质量为 $m$,可以在槽内自由滑动。开始时半圆槽静止,小物体静止于$A$处,如图所示。试求:\\
(1)当小物体滑到C点处($\theta$角)时,小物体$m$相对于槽和槽相对于地的速度的大小。\\
(2)当小物体滑到最低点$B$时,槽移动的距离$S_1$。
\end{enumerate}