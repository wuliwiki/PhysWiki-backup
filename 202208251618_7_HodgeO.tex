% 霍奇星算子
% Hodge算子|Hodge 算子|Hodge star operator|星算子|Hodge 星算子|外代数|Grassmann 代数|Exterior algebra|麦克斯韦方程组|对偶

\pentry{外导数\upref{ExtDer},体积形式\upref{VolFom}}

\addTODO{未完成:需添加更多讨论以及麦克斯韦方程组的表示.}
% \addTODO{定义有问题,修正前暂移出目录.}

\subsection{星算子的定义}

考虑 $n$ 维线性空间 $V$ 上的外代数 $\bigwedge V$,我们注意到各阶的外积空间具有明显的对称性:$\opn{dim}\bigwedge^k V=C^k_n=C^{n-k}_n=\opn{dim}\bigwedge^{n-k} V$.这意味着这样的一对空间之间存在线性同构,我们使用星算子 $\star$ 来描述这一同构.


\subsubsection{定义前的准备}


星算子是一个映射,把一个 $\bigwedge^k V$ 中的元素 $\omega$ 映射为一个 $\bigwedge^{n-k} V$ 中的元素 $\star\omega$.为了方便定义星算子,我们要先扩张一下内积的定义:


\begin{definition}{$k$-向量的内积}\label{HodgeO_def1}

设线性空间上有内积,即对于向量$\alpha$和$\beta$,有内积$\langle \alpha, \beta \rangle$.

给定向量$\alpha_i$和$\beta_j$,构成两个$k$-向量$\omega_\alpha=\alpha_1\wedge\alpha_2\wedge\cdots\wedge\alpha_k$和$\omega_\beta=\beta_1\wedge\beta_2\wedge\cdots\wedge\beta_k$.则这两个$k$-向量的内积定义为
\begin{equation}\label{HodgeO_eq1}
\langle\omega_\alpha, \omega_\beta\rangle = \det \pmat{\langle \alpha_i, \beta_j \rangle_{i, j=1}^k}
\end{equation}
即用各$\langle \alpha_i, \beta_j \rangle$构成的方阵的行列式.

\end{definition}

显然,如果有某个$\beta_j$正交于所有$\alpha_i$,那么\autoref{HodgeO_eq1} 就是$0$.这一点对于霍奇星算子的性质很重要.







\begin{example}{$2$-向量的内积}

设二维欧几里得空间有极坐标的坐标函数$r$和$\theta$,诱导出余切向量的基$\{\dd r, \dd \theta\}$.于是,基向量的内积为
\begin{equation}
\leftgroup{
    \langle\dd r, \dd r\rangle &= 1\\
    \langle\dd r, \dd \theta\rangle &= 0 = \langle\dd \theta, \dd r\rangle\\
    \langle\dd \theta, \dd \theta\rangle &= \frac{1}{r^2}
}
\end{equation}

构造$2$-向量$\omega_1=a_1\dd r\wedge\dd \theta$和$\omega_2=a_2\dd r\wedge\dd \theta$,则它们的内积为
\begin{equation}
\ali{
    \langle\omega_1, \omega_2\rangle &= a_1a_2\det\pmat{
        \langle\dd r, \dd r\rangle&\langle\dd r, \dd \theta\rangle\\
        \langle\dd \theta, \dd r\rangle&\langle\dd \theta, \dd \theta\rangle
    }\\
    &= \frac{a_1a_2}{r^2}
}
\end{equation}

\end{example}



另一等价所需要的预备知识则是如下定义的复杂指标.

\begin{definition}{复杂指标(正整数指标)}
给定正整数指标集合$\Gamma=\{1, 2, \cdots, n\}$.设$I$是$\Gamma$的$k$次\textbf{有序}子集,即对于一个正整数$k\leq n$,存在一个置换$\sigma\in S_n$,将$I$表示为$(\sigma(1), \sigma(2), \cdots, \sigma(k))$.

再定义$I$的补集$\bar{I}=(\sigma(k+1), \cdots, \sigma(n))$,如果$k=n$则$\bar{I}=\varnothing$.

则光滑函数$\omega_{\sigma(1), \sigma(2), \cdots, \sigma(k)}$表示为$\omega_I$,向量$v^{\sigma(1)}\wedge v^{\sigma(2)}\wedge\cdots\wedge v^{\sigma(k)}$表示为$v^I$.

复杂指标也应用爱因斯坦求和约定:$\omega_I v^I$是$I$遍历所有$k$次有序子集、或者特别声明的范围后,所得结果的乘积.
\end{definition}


可以看到,之所以要求$I$是有序的,是因为外积有次序要求.定义中使用正整数是为了方便,实际上复杂指标的概念也可以推广到任意指标集合上.



\begin{example}{复杂指标的一个例子}
设$I$的取值范围为$\{(1, 2), (2, 3), (3, 1)\}$,那么$\omega_{1,2}\dd x^1\wedge \dd x^2+\omega_{2,3}\dd x^2\wedge \dd x^3+\omega_{3, 1}\dd x^3\wedge \dd x^1$表示为$\omega_I\dd x^I$.

特别地,外导数可以表示为
\begin{equation}
\dd (\omega_I\dd x^I)=\dd \omega_I\wedge \dd x^I
\end{equation}
\end{example}


% 为了方便描述星算子的定义,我们先引入一些新的表示方法.

% 选定 $V$ 的基 $\{\bvec{e}_i\}$,那么任意 $\omega\in\bigwedge^k V$ 都可以表示为各 $\bvec{e}_{i_1}\wedge\bvec{e}_{i_2}\wedge\cdots\wedge\bvec{e}_{i_k}$ 的线性组合,因此我们只需要描述 $\star\bvec{e}_{i_1}\wedge\bvec{e}_{i_2}\wedge\cdots\wedge\bvec{e}_{i_k}$ 即可定义星算子.

% 为了方便,我们只考虑 $\bvec{e}_{i_1}\wedge\bvec{e}_{i_2}\wedge\cdots\wedge\bvec{e}_{i_k}$ 中各 $i_{r+1}>i_r$ 的情况,也就是下标顺序排列的情况\footnote{乱序排列的情况无非两种,奇排列和偶排列,根据外代数的定义,前者加上负号即可,后者和顺序排列是相等的.}.规定了只考虑下标顺序排列的情况后,就可以暂时不管顺序的问题,把 $\bvec{e}_{i_1}\wedge\bvec{e}_{i_2}\wedge\cdots\wedge\bvec{e}_{i_k}$ 表示为集合 $\{i_1, i_2, \cdots, i_k\}$.利用这个表达,我们就可以定义 $\star\{i_1, i_2, \cdots i_k\}=\{1, 2, \cdots, n\}-\{i_1, i_2, \cdots, i_k\}$.这样,我们就定义出了 $\bigwedge V$ 中各基向量的星算子了.再加上一条“星算子是线性的”,即 $\star(\sum a_i\omega_i)=\sum a_i\star\omega_i$,就得到星算子的完整定义了:

% \begin{definition}{霍奇星算子}%注释掉了
% 定义域 $\mathbb{F}$ 上 $n$ 维线性空间 $V$ 上的外代数 $\bigwedge V$ 的自同构映射 $\star:\bigwedge V\to \bigwedge V$,满足:
% \begin{enumerate}
% \item 线性性:对于任意 $a_i\in\mathbb{F}$,$\omega_i\in\bigwedge V$,有 $\star(\sum a_i\omega_i)=\sum a_i\star\omega_i$;
% \item 如果把下标顺序排列的 $\bvec{e}_{i_1}\wedge\bvec{e}_{i_2}\wedge\cdots\wedge\bvec{e}_{i_k}$ 表示为集合 $\{i_1, i_2, \cdots, i_k\}$,那么 $\star\{i_1, i_2, \cdots i_k\}=\{1, 2, \cdots, n\}-\{i_1, i_2, \cdots, i_k\}$.
% \end{enumerate}
% 称这一同构为\textbf{霍奇星算子(Hodge star operator)}.
% \end{definition}



\subsubsection{星算子的定义}


\begin{definition}{霍奇星算子}

在$n$维线性空间$V$的外代数$\bigwedge V$上\textbf{任取}$k$-向量$\alpha$和$\beta$,且有标准正交基$\{\bvec{e}_1, \bvec{e}_2, \cdots, \bvec{e}_n\}$.为方便,记$\omega=\bvec{e}_1\wedge\bvec{e}_2\wedge\cdots\wedge\bvec{e}_n$\footnote{如果这里的$V$是流形上的余切空间,则$\omega$就是给定点处的体积形式.}.定义映射$\star:\bigwedge^k V \to \bigwedge^{n-k} V$如下:
\begin{equation}
\alpha\wedge\star\beta = \langle\alpha, \beta\rangle\omega
\end{equation}
称$\star$是$\bigwedge V$上的\textbf{霍奇星算子(Hodge star operator或Hodge star)},或\textbf{霍奇对偶(Hodge dual)}.

\end{definition}


使用复杂指标可以得到另一种定义方式:


\begin{theorem}{霍奇星算子(等价定义)}
设$I$是\autoref{HodgeO_def1} 中描述的一个复杂指标,$\bar{I}$是其补,$\{e_1, e_2, \cdots, e_n\}$是线性空间的一组\textbf{标准正交基},则
\begin{equation}
\star e_I = \opn{sign}\sigma e_{\overline{I}}
\end{equation}
\end{theorem}



欧几里得空间中,如果用通常的标准正交坐标系,坐标函数为$x^1, x^2, \cdots, x^n$,那么体积形式为$\dd x^1\wedge \dd x^2\wedge \cdots \wedge \dd x^n$,于是
\begin{equation}\label{HodgeO_eq2}
\leftgroup{
    \star\dd x^1 \wedge \dd x^2 &= \dd x^3\wedge \dd x^4\wedge\cdots\wedge\dd x^n\\
    \star\dd x^1 \wedge \dd x^3 &= -\dd x^2\wedge \dd x^4\wedge\cdots\wedge\dd x^n
}
\end{equation}

如果流形上坐标函数仍记为$x^1, x^2, \cdots, x^n$,而某处的体积形式为$\sqrt{\abs{g}}\dd x^1\wedge \dd x^2\wedge \cdots \wedge \dd x^n$,则\autoref{HodgeO_eq2} 的例子应变为
\begin{equation}\label{HodgeO_eq3}
\leftgroup{
    \frac{\star\dd x^1 \wedge \dd x^2}{\langle\dd x^1\wedge\dd x^2, \dd x^1\wedge \dd x^2\rangle} &= \sqrt{\abs{g}}\dd x^3\wedge \dd x^4\wedge\cdots\wedge\dd x^n\\
    \frac{\star\dd x^1 \wedge \dd x^3}{\langle\dd x^1\wedge\dd x^3, \dd x^1\wedge \dd x^3\rangle} &= -\sqrt{\abs{g}}\dd x^2\wedge \dd x^4\wedge\cdots\wedge\dd x^n
}
\end{equation}



\subsection{星算子的性质}






% \addTODO{应该并入\textbf{麦克斯韦方程组(外微分形式)}\upref{MWEq2},以本词条为预备.}
% \subsection{麦克斯韦方程组的外微分形式:后两个方程\cite{KnotsVol4}}

% 星算子描述的是外积空间中一个非常显眼的对偶,即 $\bigwedge^k V$ 和 $\bigwedge^{n-k} V$ 之间的同构.




