% 圆周运动的向心力
% keys 向心力|圆周运动

\pentry{圆周运动的加速度\upref{CMAD}, 牛顿运动定律\upref{New3}}

在惯性参考系中, 质点的运动符合牛顿第二定律 $\bvec F = m\bvec a$。 所以要使一个质点做半径为 $R$ 的圆周运动, 那么我们就可以把任意时刻圆周运动的加速度矢量(\autoref{CMAD_eq4}~\upref{CMAD})乘以质量 $m$ 得到它该时刻受到的合力。 我们把这个力叫做\textbf{向心力(centripetal force)}。
\begin{equation}
\bvec F_c = -m \omega^2 \bvec r = -\frac{mv^2}{r}\uvec r = m\bvec \omega \cross \bvec v
\end{equation}
我们需要注意区分向心力和离心力\upref{Centri}, 向心力不是圆周运动给物体施加的力, 而是维持圆周运动所需要的力。 一旦这个力消失, 物体将保持当前速度,按照牛顿第一定律做匀速直线运动。 如果我们使用惯性参考系\upref{New3}, 就不存在离心力的概念。 只有当我们使用旋转的参考系(非惯性系), 才能讨论离心力, 因为离心力是一种\textbf{惯性力}\upref{Iner}, 是在非惯性系中讨论物体运动的一个数学工具。

那么为什么例如在车转弯的时候我们会感觉有离心力把我们向外推呢? 事实上人的惯性使得人有继续保持直线运动的倾向, 而车突然改变方向偏离原来的直线, 所以在地面惯性系看来是车在向内推人, 但人因身处车中, 本能地觉得车是静止的, 以为牛顿第二定律仍然成立, 所以感觉自己受到了一种无形的离心力。 这是一种错觉, 就像我们本能地认为光是沿直线传播, 所以会觉得伸进水中的筷子被折弯了, 或者觉得镜子里面还有一个空间。

\subsection{有趣的推导}
为了更好地理解为何向心力会产生圆周运动, 我们可以假设一个小球在光滑水平面运动的过程中被反弹若干次, 使得它的轨道称为闭合的正 $N$ 边形, 每个顶点距离中心为 $R$。 每次撞击后, 速度不变, 方向改变 $2\theta  = 2\pi/N$。
\addTODO{图}
所以在每次碰撞时, 我们可以把速度垂直分解为径向(指向圆心)和角向速度。 碰撞后角向速度不变, 径向速度增量为 $2v\sin\theta$, 即动量改变为
\begin{equation}
\Delta p = 2mv\sin\theta
\end{equation}
正 $N$ 边形每条边的长度为 $L = 2R\sin(\theta)$, 所以相邻两次碰撞的时间为
\begin{equation}
\Delta t = \frac{2R\sin\theta}{v}
\end{equation}
我们可以令 $N \to \infty$, 这样, 多边形轨迹就趋近于圆周运动, 每次撞击时, 冲量趋近于 0, 时间间隔也趋近于 0, 我们就可以把这些离散的撞击看成时一个连续的力。 平均来说, 力的大小等于动量改变除以时间
\begin{equation}
F_c = \frac{\Delta p}{\Delta t} = \frac{mv^2}{R} = m\omega^2 R
\end{equation}
注意该式甚至不需要取极限 $\Delta t \to 0$ 就可以成立。

\addTODO{例题和习题}
