% Klein-Gordon方程
% 量子场论|高等量子力学|克莱因-戈登方程|克莱茵-戈登方程

\addTODO{预备知识需要薛定谔量子力学的相关内容,但现在该部分还未整理好,不宜引用.}

\pentry{自然单位制、普朗克单位制\upref{NatUni}}

\subsection{问题的引入}

薛定谔方程在量子力学中的地位,就像牛顿三定律在经典力学中的地位一样,是描述理论结构的“公理”.因此,如果要了解量子力学的局限性,可以从研究薛定谔方程本身入手.

\subsubsection{质能关系问题}

回顾单粒子薛定谔方程的表达(注意这里使用了\textbf{自然单位制}\upref{NatUni}):
\begin{equation}\label{KGeq_eq1}
\qty(-\frac{\nabla^2}{2m}+V)\psi = \I \partial_t \psi
\end{equation}

由于$\hat{p}=-\I\nabla$和$\hat{E}=\I\partial_t$分别是量子力学中的动量、能量算子,故\autoref{KGeq_eq1} 左边体现的是经典力学中的\textbf{哈密顿量}:

\begin{equation}\label{KGeq_eq2}
\begin{aligned}
\hat{H}=-\frac{\nabla^2}{2m}+V &= \frac{\hat{p}^2}{2m}+V \\
&\updownarrow\\
H &= \frac{p^2}{2m}+V
\end{aligned}
\end{equation}

\autoref{KGeq_eq2} 上下两部分含义完全不同\footnote{上面一行各项是算符,它们作为量子态之间线性变换的\textbf{特征值}才是能量、动量等可观测量;下面一行各项就是实数,本身即为能量、动量等可观测量.},但其描述的能量-动量-质量关系是一致的.因此薛定谔方程本质上是经典力学的推广,与经典时空观契合,但与相对论时空观矛盾.

\subsubsection{粒子数守恒问题}

回顾量子力学的概率守恒.取薛定谔方程的复共轭,得

\begin{equation}\label{KGeq_eq3}
\qty(-\frac{\nabla^2}{2m}+V)\psi^* = -\I \partial_t \psi^*
\end{equation}

在\autoref{KGeq_eq1} 上乘以$\psi^*$,再减去\autoref{KGeq_eq3} 乘以$\psi$ ,得

\begin{equation}\label{KGeq_eq4}
\begin{aligned}
\psi^*\qty(-\frac{\nabla^2}{2m}+V)\psi - \psi\qty(-\frac{\nabla^2}{2m}+V)\psi^* &= \I\qty(\psi^*\partial_t\psi + \psi\partial_t\psi^*)\\
-\psi^*\frac{\nabla^2}{2m}\psi + \psi\frac{\nabla^2}{2m}\psi^* &= \I \partial_t\qty(\psi\psi^*)\\
\partial_t\qty(\abs{\psi}^2) + \frac{-\I\nabla}{2m}\qty(\psi^*\nabla\psi-\psi\nabla\psi^*) &= 0\\
\partial_t \rho + \frac{\nabla}{2m} \qty(\psi^*\hat{p}\psi + (\psi^*\hat{p}\psi)^*) &= 0
\end{aligned}
\end{equation}
其中$\rho=\abs{\psi}^2$可以理解为粒子的空间位置分布.

对于动量本征态容易验证,












