% 李代数(综述)
% license CCBYSA3
% type Wiki

本文根据 CC-BY-SA 协议转载翻译自维基百科\href{https://en.wikipedia.org/wiki/Lie_algebra}{相关文章}。

在数学中,李代数(发音为 /liː/,LEE)是一个向量空间 {\displaystyle {\mathfrak {g}}},配有一个名为李括号的运算,它是一个交替双线性映射 {\displaystyle {\mathfrak {g}}\times {\mathfrak {g}}\rightarrow {\mathfrak {g}}},满足雅可比恒等式。换句话说,李代数是一个定义在域上的代数,其中的乘法运算(称为李括号)是交替的,并且满足雅可比恒等式。两个向量 {\displaystyle x} 和 {\displaystyle y} 的李括号记作 {\displaystyle [x,y]}。李代数通常是一个非结合代数。然而,每个结合代数都可以生成一个李代数,该李代数由相同的向量空间构成,且使用交换子李括号,即 \([x,y]=xy-yx\)。