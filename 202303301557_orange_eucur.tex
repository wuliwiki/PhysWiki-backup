% 欧几里得空间中的曲线

\pentry{光滑映射(简明微积分)\upref{SmthM}}

\addTODO{增加线性代数的预备知识}

\subsection{一般欧几里得空间中曲线的概念}
% 原作者:JierPeter

把 $\mathbb{R}^n$ 视作一个向量空间,任意取定一个坐标系(等价于向量空间的一组基),我们可以把向量值函数 $f$ 分为 $n$ 个标量值函数,简称为 $f$ 的\textbf{分量}。

\begin{definition}{参数曲线}
令 $I = (a, b)$ 是实数轴 $\mathbb{R}$ 上的一个开区间,则称\textbf{连续函数}$f:I\to \mathbb{R}^n$ 为 $\mathbb{R}^n$ 中的一条(连通的)\textbf{参数曲线(curve)}(若$I = [a, b]$是一个闭区间则称 $f$ 为从 $f(a)$ 到 $f(b)$ 的\textbf{道路(path)}),$f$的值域 $f(I) \subseteq \mathbb{R}^n$ 被称为参数曲线的\textbf{轨迹}。此处连续是指函数的 $n$ 个分量都是 $\mathbb{R}\to\mathbb{R}$ 的连续函数。如果 $f$ 是一个单射,则被称为\textbf{简单参数曲线}。
\end{definition}

你可能自然会想确认,由 $f$ 的分量定义的连续性,和取定坐标系的方式是否有关?答案是无关的。我们可以用另一种方式来理解此处的“连续”:取集合 $I$ 与 $\mathbb{R}^n$,配上通常的拓扑——即 $I$ 取 $\mathbb{R}$ 的子拓扑,$\mathbb{R}^n$ 取 $\mathbb{R}$ 的乘积拓扑——所得到的拓扑空间,那么 $f$ 就是拓扑空间之间的映射,其连续性取决于拓扑意义上的连续性,和具体坐标系的选择就无关了。

要强调的一点是,即便两条参数曲线的轨迹一样,它们也不一定是同一条参数曲线。比如说,取 $f, g:\mathbb{R}\to\mathbb{R}^2$ 这两个函数,定义为 $f(t)=\pmat{\cos t\\\sin t}$ 和 $g(t)=\pmat{\cos 2t\\\sin 2t}$,那么尽管它们的轨迹都是平面上的单位圆,但由于两条参数曲线的“速度”不一样,我们依然把它们认为是不同的参数曲线。

想要严格的定义“速度”我们需要更强大的结构:

\begin{definition}{连续可微参数曲线}
如果参数曲线 $f: I \to \mathbb{R}^n$ 对于任意分量都是连续可微的(即其导函数连续),那么称 $f$ 是一个\textbf{连续可微参数曲线(differentiable curve)}($I$为闭曲线时称为\textbf{连续可微道路(differentiable path)}),也记为 $C^1$ 参数曲线。类似的我们可以定义 $p$ 阶连续可微参数曲线,也记为 $C^p$ 参数曲线。
\end{definition}

同样地,$f$ 本身的连续可微性,和具体坐标系的选择也无关\footnote{由于“微分”这一概念是实数空间特有的,我们没法像前面一样直接用拓扑的概念绕过坐标系的选择;证明这一点的思路也可以应用到之前对一般参数曲线的讨论上去:考虑变换坐标系后各点的变换,会发现变换后的分量就是过渡矩阵乘以原先的函数列矩阵,也就是说,变换坐标后新的函数分量,是旧分量的某种线性组合。这样一来,变换前连续可微当且仅当变换后也连续可微,从而可知和变换无关。}。

\begin{definition}{速度、速率、弧长}
对于连续可微参数曲线 $f: I \to \mathbb{R}^n$,我们把 $t = t_0$ 时的\textbf{速度}(velocity)定义为向量 $f'(t_0)$,或者记做 $\frac{\dd}{\dd t} f |_{t = t_0}$,把速度的模长 $\|f'(t_0)\|$ 称为 $t = t_0$ 时的\textbf{速率}(speed);取闭区间 $[a, b] \subseteq I$,我们定义从 $f(a)$ 到 $f(b)$ 的\textbf{弧长}为 $\int_a^b \|f'(t)\| \dd t$。
\end{definition}

\begin{definition}{常速率参数曲线}\label{eucur_def1}
如果连续可微参数曲线 $f: I \to \mathbb{R}^n$ 每个点的速率是一个常数,那么称 $f$ 是一个\textbf{常速率参数曲线}。特别的我们把速率为 $1$ 的常速率参数曲线称为\textbf{单位速率参数曲线},也称\textbf{(以)弧长(为)参数(的)曲线},此时我们常用 $s$ 来代替参数 $t$。
\end{definition}

$p$ 可以推广到无限,即:

\begin{definition}{光滑参数曲线}
如果参数曲线 $f: I \to \mathbb{R}^n$ 对于任意分量都是光滑的,那么称 $f$ 是一个\textbf{光滑参数曲线(smooth curve)}($I$为闭曲线时称为\textbf{光滑道路(smooth path)}),有时也记为 $C^\infty$ 的参数曲线。
\end{definition}

和上述坐标变换的讨论类似,可证明光滑参数曲线的光滑性不依赖于坐标系的选择。等价的,光滑参数曲线是任意阶连续可微参数曲线。

除了参数曲线之外,我们还可以定义曲线为 $\mathbb{R}^n$ 的子集。我们可以把曲线理解成参数曲线的轨迹,但需要一些额外的要求。

\begin{definition}{曲线}
如果 $\mathbb{R}^n$ 的子集 $C$ 满足对 $C$ 上任意一点 $p$, 存在点 $p$ 的一个(足够小的)开邻域 $U \subseteq \mathbb{R}^n$,使得 $C \cap U$ 是一条简单参数曲线的轨迹,那么 $C$ 就被称为 $\mathbb{R}^n$ 上的一条\textbf{曲线}。如果参数曲线是连续可微的/光滑的而且速度处处不为零,那么我们就称之为\textbf{连续可微的/光滑曲线},有时也简称为曲线。在微分几何中,我们一般要求曲线是连通的。
\end{definition}

在不会引起误解的情况下,我们也会把参数曲线简称为曲线,读者需要依凭上下文进行判断它到底是一个映射还是一个子集。

