% 马克斯·玻恩(综述)
% license CCBYSA3
% type Wiki

本文根据 CC-BY-SA 协议转载翻译自维基百科\href{https://en.wikipedia.org/wiki/Max_Born}{相关文章}。

\textbf{马克斯·玻恩}(Max Born,FRS FRSE)(德语发音:[ˈmaks ˈbɔʁn];1882年12月11日-1970年1月5日)是一位德裔英国物理学家和数学家,对量子力学的发展起到了关键作用。他还在固态物理学和光学领域做出了重要贡献,并在20世纪20年代和30年代指导了许多著名物理学家的研究工作。玻恩因其“对量子力学的基础研究,特别是对波函数统计解释的贡献”于1954年获得了诺贝尔物理学奖。[1]

玻恩于1904年进入哥廷根大学,在那里他结识了三位著名的数学家:费利克斯·克莱因(Felix Klein)、大卫·希尔伯特(David Hilbert)和赫尔曼·闵可夫斯基(Hermann Minkowski)。他以弹性丝和带的稳定性为主题撰写了博士论文,并因此获得了该校哲学系奖项。1905年,他开始与闵可夫斯基研究狭义相对论,随后完成了以汤姆森原子模型为主题的资格论文(habilitationsschrift)。1918年,他在柏林偶然遇到弗里茨·哈伯(Fritz Haber),讨论了一种金属与卤素反应生成离子化合物的过程,这一过程今天被称为\textbf{玻恩–哈伯循环}(Born–Haber cycle)。

在第一次世界大战期间,玻恩最初被安置为无线电操作员,但由于他的专业知识,他被调往从事声测位研究工作。1921年,玻恩返回哥廷根大学,为他长期的朋友和同事詹姆斯·弗兰克(James Franck)安排了另一张教授职位。在玻恩的领导下,哥廷根大学成为当时世界上物理学的主要中心之一。

1925年,玻恩与维尔纳·海森堡(Werner Heisenberg)共同提出了量子力学的矩阵力学表述。次年,他提出了薛定谔方程中 \textbf{ψ\ψ} 概率密度函数的标准解释,这一贡献为他赢得了1954年的诺贝尔物理学奖。

玻恩的影响远远超出了他的个人研究领域。包括马克斯·德尔布吕克(Max Delbrück)、齐格弗里德·弗吕格(Siegfried Flügge)、弗里德里希·洪德(Friedrich Hund)、帕斯夸尔·约尔丹(Pascual Jordan)、玛丽亚·哥佩特-梅耶(Maria Goeppert-Mayer)、洛塔尔·沃尔夫冈·诺德海姆(Lothar Wolfgang Nordheim)、罗伯特·奥本海默(Robert Oppenheimer)和维克多·魏斯科普夫(Victor Weisskopf)等人都在哥廷根大学师从玻恩取得了博士学位。

此外,玻恩的助手中也包括许多著名的物理学家,例如恩里科·费米(Enrico Fermi)、维尔纳·海森堡(Werner Heisenberg)、格哈德·赫兹伯格(Gerhard Herzberg)、弗里德里希·洪德(Friedrich Hund)、沃尔夫冈·泡利(Wolfgang Pauli)、莱昂·罗森菲尔德(Léon Rosenfeld)、爱德华·泰勒(Edward Teller)以及尤金·维格纳(Eugene Wigner)。

1933年1月,纳粹党在德国上台,作为犹太人的玻恩被暂停了他在哥廷根大学的教授职位。他移居到英国,并在剑桥大学圣约翰学院担任职务,同时撰写了一本科普书《不安定的宇宙》(*The Restless Universe*)以及《原子物理学》(*Atomic Physics*),后者很快成为一本标准教材。

1936年10月,他成为爱丁堡大学自然哲学的泰特教授。在那里,他与两位德国出生的助手E.沃尔特·凯勒曼(E. Walter Kellermann)和克劳斯·富克斯(Klaus Fuchs)合作,继续他的物理学研究。

1939年8月31日,也就是第二次世界大战在欧洲爆发的前一天,玻恩成为了英国公民。他一直在爱丁堡工作到1952年,随后退休回到西德的巴特皮尔蒙特(Bad Pyrmont)。1970年1月5日,玻恩在哥廷根的一家医院去世。[2]
\subsection{早年生活}  
马克斯·玻恩于1882年12月11日出生在布雷斯劳(现波兰弗罗茨瓦夫)。在玻恩出生时,布雷斯劳属于德意志帝国普鲁士的西里西亚省,他的家族具有犹太血统。[3] 玻恩是解剖学家和胚胎学家古斯塔夫·玻恩(Gustav Born)与其妻子玛格丽特(Margarethe,昵称格雷琴)·考夫曼(Kauffmann)所生的两个孩子之一。古斯塔夫是布雷斯劳大学的胚胎学教授,[4] 玛格丽特则出身于一个西里西亚工业家家族。玛格丽特于1886年8月29日去世,当时马克斯只有四岁。[5]  

马克斯有一个妹妹,名叫凯特(Käthe),她于1884年出生。马克斯还有一个同父异母的弟弟沃尔夫冈(Wolfgang),是父亲与第二任妻子贝莎·利普斯坦(Bertha Lipstein)所生。沃尔夫冈后来成为纽约城市学院的艺术史教授。[6]
 
玻恩最初在布雷斯劳的国王威廉文理中学(König-Wilhelm-Gymnasium)接受教育,并于1901年进入布雷斯劳大学学习。德国的大学体系允许学生自由地在各大学之间转学,因此他分别于1902年夏季学期在海德堡大学、1903年夏季学期在苏黎世大学学习。在布雷斯劳的同学奥托·托普利茨(Otto Toeplitz)和恩斯特·赫林格(Ernst Hellinger)告诉玻恩有关哥廷根大学的信息,[7] 玻恩于1904年4月进入哥廷根大学。在那里,他遇到了三位著名的数学家:费利克斯·克莱因(Felix Klein)、大卫·希尔伯特(David Hilbert)和赫尔曼·闵可夫斯基(Hermann Minkowski)。  

玻恩到达哥廷根后不久,就与后两位数学家建立了密切关系。从他上希尔伯特的第一堂课起,希尔伯特就发现玻恩具有非凡的能力,并挑选他担任课程记录员,负责为哥廷根大学数学阅览室整理课程笔记。这一职位使玻恩与希尔伯特保持了定期且极具价值的联系。希尔伯特后来选择玻恩担任首位无薪的半官方助理,使他成为玻恩的导师。  

玻恩与闵可夫斯基的初次接触来自他的继母贝尔塔(Bertha),她曾在柯尼斯堡与闵可夫斯基一起参加舞蹈课。通过这层关系,玻恩获得了每周日受邀到闵可夫斯基家中共进晚餐的机会。此外,在担任课程记录员和助理期间,玻恩经常在希尔伯特的家中见到闵可夫斯基。[8][9]

玻恩与克莱因的关系较为复杂。他参加了一场由克莱因以及应用数学教授卡尔·龙格(Carl Runge)和路德维希·普朗特(Ludwig Prandtl)共同主持的关于弹性理论的研讨会。虽然玻恩对这个主题并不特别感兴趣,但他不得不在会上提交一篇报告。他选择了一个简单的案例:研究两端固定的弯曲线条,并利用希尔伯特的变分法计算出最小化势能、从而达到最稳定状态的配置。克莱因对此印象深刻,邀请玻恩以“平面与空间中的弹性稳定性”为题提交论文。这个题目正是克莱因非常感兴趣的研究领域,而且克莱因已经将其设定为哥廷根大学年度哲学系大奖(Philosophy Faculty Prize)的主题。该奖项不仅享有盛誉,而且提交的作品还可以作为博士论文。

然而,玻恩拒绝了这个邀请,因为他并不喜欢应用数学领域的研究。克莱因对此极为愤怒。[10][11]

克莱因在学术界拥有很大的影响力,可以决定学术职业的成败,因此玻恩感到必须弥补过失,最终同意为该奖项提交作品。但由于克莱因拒绝为其指导,玻恩转而请卡尔·龙格担任其导师。此外,沃尔德玛·福伊特(Woldemar Voigt)和卡尔·史瓦西(Karl Schwarzschild)成为了他的其他考官。玻恩从自己的报告入手,进一步发展了稳定性条件的方程。随着研究的深入,他对这一主题越来越感兴趣,并设计了一个实验装置以验证自己的理论预测。

1906年6月13日,校长宣布玻恩赢得了哲学系大奖。一个月后,他通过了口试,以优异成绩(magna cum laude)获得数学博士学位。[12]

毕业后,玻恩不得不履行学生时期延期的兵役义务。他被征召入德国军队,分配到驻扎在柏林的“俄国皇后亚历山德拉第二卫队龙骑兵团”。他的服役时间很短,因为1907年1月的一次哮喘发作使他提前退役。随后,他前往英国,被剑桥大学冈维尔与凯斯学院录取,在卡文迪许实验室师从J. J. 汤姆森、乔治·塞尔(George Searle)和约瑟夫·拉默(Joseph Larmor)学习物理六个月。

返回德国后,玻恩再次被军队征召,并服役于精英部队“第一(西里西亚)生命胸甲骑兵团‘大选帝侯’”。然而,他仅服役六周后再次因健康原因退役。退役后,他回到布雷斯劳,在奥托·卢默(Otto Lummer)和恩斯特·普林斯海姆(Ernst Pringsheim)的指导下工作,试图完成物理学的授课资格考试。然而,玻恩的黑体实验发生了一次小事故——冷却水管破裂导致实验室被水淹,卢默因此告诫他永远不可能成为一名物理学家。[13]

1905年,阿尔伯特·爱因斯坦发表了关于狭义相对论的论文《论动体的电动力学》。玻恩对此非常感兴趣,开始研究这一课题。然而,当他得知赫尔曼·闵可夫斯基(Hermann Minkowski)也在进行类似研究时,他感到非常沮丧。玻恩将自己的研究结果写信告诉闵可夫斯基后,闵可夫斯基邀请他返回哥廷根大学并在那里完成他的授课资格考试。玻恩接受了邀请。

在哥廷根,托普利茨(Toeplitz)帮助玻恩复习矩阵代数,以便他能够使用闵可夫斯基的四维时空矩阵完成将相对论与电动力学相结合的项目。玻恩和闵可夫斯基相处融洽,他们的工作进展顺利。然而,1909年1月12日,闵可夫斯基因阑尾炎突然去世。数学系的学生推举玻恩在闵可夫斯基的葬礼上代表他们发言。[14]