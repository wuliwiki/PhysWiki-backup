% 环
% 集合|运算|结合律|分配律

\pentry{群\upref{Group}}

群是代数学中研究的最简单的结构,只由一个集合配上一个运算构成,这个运算只有$4$条公理进行限制.

通常,在群之后会介绍的一个更复杂一些的概念,是\textbf{环}.一个环就是由一个集合配上两个运算构成的集合,通常把这两个运算叫做加法和乘法;环上的加法和乘法分别构成群,不过乘法群不包括加法的单位元,而加法群是阿贝尔群.

\begin{definition}{环}
一个\textbf{环(ring)}是一个集合 $R$ 与两种运算 “加” 和 “乘”, 分别记为 $+$ 和 $\cdot$. 其中加法配合$R$中所有元素构成一个\textbf{阿贝尔群},加法群的单位元通常称为零元,记为$0$;乘法配合$R^*$中所有元素构成一个群,其中$R^*=R-\{0\}$,乘法群的单位元通常称为幺元,记为$1$.除了加法和乘法分别构成两个群之外,还要求加法和乘法满足分配律
\begin{itemize}
\item $a \cdot (b + c) = (a \cdot b) + (a \cdot c)$
\item $(b + c) \cdot a = (b \cdot a) + (c \cdot a)$
\end{itemize}
\end{definition}

通常,为了方便表示,我们也会省略环中乘法的符号,将$a\cdot b$写为$ab$.

由定义,环的加法必须是可交换的,但乘法却不一定.如果$R$的乘法也交换的话,我们就称$R$为一个\textbf{交换环(commutative ring)}.

\begin{example}{整数环}
整数集合配上通常的加法和减法,构成一个
\end{example}

环的定义在一个细节上有争议,那就是乘法需不需要有幺元.有些书中的定义不要求有幺元,也就是说乘法只构成
