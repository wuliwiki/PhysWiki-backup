% 罗伯特·胡克(综述)
% license CCBYSA3
% type Wiki

本文根据 CC-BY-SA 协议转载翻译自维基百科\href{https://en.wikipedia.org/wiki/Robert_Hooke}{相关文章}。

罗伯特·胡克 (Robert Hooke) FRS (/hʊk/;1635年7月18日-1703年3月3日)[4][a] 是一位英国博学者,活跃于物理学(“自然哲学”)、天文学、地质学、气象学和建筑学领域。[5] 他被认为是最早在1665年利用其设计的复合显微镜研究微观生物的科学家之一。[6][7] 胡克年轻时是一位贫困的科学研究者,后来成为他那个时代最重要的科学家之一。[8] 在1666年的伦敦大火之后,胡克以测量员和建筑师的身份,通过完成超过一半的地产界线测绘工作以及协助城市的快速重建,获得了财富和声誉。[9][8] 在他去世后的几个世纪中,胡克经常受到作家的贬低,但在20世纪末,他的名誉得以恢复,并被誉为“英格兰的达·芬奇”。[10]

胡克是皇家学会的院士,从1662年起担任其首任实验策展人。[9] 从1665年至1703年,他还担任格雷沙姆学院的几何学教授。[11] 胡克的科学生涯始于担任物理科学家罗伯特·波义耳(Robert Boyle)的助手。胡克制作了用于波义耳气体定律实验的真空泵,并亲自进行了实验。[12] 1664年,胡克观测到火星和木星的自转。[11] 胡克在1665年出版的著作《显微图谱》(*Micrographia*)中首次提出了“细胞”(cell)一词,这本书激发了显微研究的热潮。[13][14] 在光学领域的研究中——特别是对光折射的研究——胡克提出了光的波动理论。[15] 他是第一个提出以下假说的人:物质因热膨胀的原因,[16] 空气由不断运动的小颗粒组成,并由此产生压力,[17] 以及热是一种能量的概念。[18]

在物理学中,胡克推测重力遵循反平方定律,并且可以说是第一个提出行星运动中这种关系假设的人。[19][20] 这一原理后来被艾萨克·牛顿(Isaac Newton)进一步发展并形式化为牛顿的万有引力定律。[21] 对这一见解的优先权争议促成了胡克与牛顿之间的竞争。在地质学和古生物学中,胡克创立了“水陆球”理论,[22] 因而质疑了《圣经》中关于地球年龄的观点;他还提出了物种灭绝的假说,并认为山丘和山脉是由地质过程抬升而成的。[23] 通过识别已灭绝物种的化石,胡克预示了生物进化论的诞生。[22][24]
\subsection{生平与作品}
\subsubsection{早年生活}  
有关胡克早年生活的大部分信息来自他于1696年开始撰写但未完成的自传;理查德·沃勒 (Richard Waller) 在1705年出版的《罗伯特·胡克博士遗著》序言中提到了这部自传。[25][b] 沃勒的作品,以及约翰·沃德 (John Ward) 的《格雷沙姆教授的生平》[27] 和约翰·奥布里的《简短生平记》[28],构成了胡克生平最重要的同时期传记资料。

胡克于1635年出生在怀特岛的弗雷什沃特 (Freshwater),父母是塞西莉·贾尔斯 (Cecily Gyles) 和英国国教牧师约翰·胡克 (John Hooke),后者是弗雷什沃特全圣教堂的助理牧师。[29] 罗伯特是四个兄弟姐妹中最小的,比其他人小七岁(两个男孩和两个女孩);他身体虚弱,起初并不被认为能活下来。[30][31] 虽然他的父亲教授了他一些英语、(拉丁)语法和神学,但罗伯特的教育大体上被忽视了。[32] 自己摸索着成长的胡克制作了许多机械玩具;看到一个拆开的黄铜时钟后,他用木头制作了一个复制品,并“能正常运转”。[32]

胡克的父亲于1648年10月去世,遗嘱中留给罗伯特40英镑(外加祖母留给他的10英镑)。[33][c] 13岁时,他带着这笔钱去了伦敦,成为著名画家彼得·莱利 (Peter Lely) 的学徒。[35] 胡克还接受了肖像画家塞缪尔·库珀 (Samuel Cowper) 的“绘画指导”,[34] 但“油画颜料的气味不适合他的体质,导致他经常头痛”,于是他成为威斯敏斯特学校校长理查德·布斯比 (Richard Busby) 的学生。[37] 胡克很快掌握了拉丁语、希腊语和欧几里得《几何原本》;[11] 他还学会了弹奏管风琴[38],并开始了对力学的终生研究。[11] 胡克后来在为罗伯特·波义耳的作品和自己《显微图谱》配图时,展示了他精湛的绘画技艺。[39]
\subsubsection{牛津}
\begin{figure}[ht]
\centering
\includegraphics[width=6cm]{./figures/00ca14e549d75fcd.png}
\caption{罗伯特·波义耳的画像,由约翰·克尔斯布姆创作,收藏于兰开夏郡的高瑟普庄园。} \label{fig_HK_1}
\end{figure}
1653年,胡克进入牛津大学基督教会学院,他以风琴师和唱诗班成员的身份获得了免费学费和住宿,并通过担任勤务生获得了基本收入,[40][d] 尽管他直到1658年才正式注册。[40] 1662年,胡克获得了文学硕士学位。[38]

在牛津学习期间,胡克还受雇为托马斯·威利斯博士的助理——威利斯是一位医生、化学家和牛津哲学俱乐部的成员。[42][e] 哲学俱乐部由沃达姆学院院长约翰·威尔金斯创立,他领导了这个重要的科学家团体,这个团体后来成为皇家学会的核心。[44] 1659年,胡克向俱乐部描述了一些比空气重的飞行方法的要素,但他得出结论认为人类的肌肉力量不足以实现这一目标。[45] 通过俱乐部,胡克认识了赛思·沃德(大学的萨维利安天文学教授),并为沃德开发了一种机制,以改进用于天文时间测量的摆钟的规律性。[46] 胡克将他在牛津的日子描述为他终生科学热情的基础。[47] 他在这里结识的朋友,尤其是克里斯托弗·雷恩,在他整个职业生涯中都对他至关重要。威利斯还将胡克介绍给罗伯特·波义耳,俱乐部试图吸引波义耳来牛津工作。[48]

1655年,波义耳搬到牛津,胡克名义上成为他的助理,但实际上是他的共同实验者。[48] 波义耳一直在研究气体压力;尽管亚里士多德的格言“自然厌恶真空”广为流传,但真空可能存在的想法刚刚开始被讨论。胡克为波义耳的实验开发了一种空气泵,而不是使用拉尔夫·格雷特雷克斯的泵,胡克认为后者“过于粗糙,无法完成任何重要任务”。[49] 胡克的设备促成了随后归因于波义耳的著名定律的发展;[50][f] 胡克具有特别敏锐的观察力,并且是一个熟练的数学家,这些特质波义耳并不具备。胡克教授波义耳学习了欧几里得的《几何原本》和笛卡尔的《哲学原理》;[9] 他们还通过实验认识到火是一种化学反应,而不是亚里士多德所说的自然界的基本元素。[52]
\subsubsection{皇家学会}
霍克在皇家学会任职期间的科学工作概述见下文“科学”部分。

根据1935年皇家学会图书管理员亨利·罗宾逊的说法:

如果没有他每周的实验和丰富的工作,皇家学会几乎不可能存续,或者至少会以完全不同的方式发展。毫不夸张地说,他在历史上是皇家学会的创造者。[53]

皇家学会(全称“通过实验改进自然知识的皇家学会”[g])成立于1660年,并于1662年7月获得皇家特许状。[54] 1661年11月5日,罗伯特·莫雷提议任命一位实验策展人为学会提供实验支持,这一提议获得一致通过,并在波义耳的推荐下任命霍克为策展人。[9] 学会没有稳定的收入来完全支付“实验策展人”职位的费用,但在1664年,约翰·卡特勒设立了一笔每年50英镑的津贴,用于在格雷沙姆学院开设“机械”讲座,[55] 条件是学会必须任命霍克担任此任务。[56] 1664年6月27日,霍克被正式任命为此职务,并于1665年1月11日被授予“终身策展人”头衔,年薪为80英镑,[h] 其中包括学会支付的30英镑和卡特勒的50英镑年金。[56][i]

1663年6月,霍克被选为皇家学会会士(FRS)。[57] 1665年3月20日,他还被任命为格雷沙姆学院几何学教授。[58][59] 1667年9月13日,霍克成为学会的代理秘书,[60] 并于1677年12月19日被任命为联合秘书。[61]
\subsubsection{性格、人际关系、健康与去世}
\begin{figure}[ht]
\centering
\includegraphics[width=8cm]{./figures/c967f4e5b4d04f35.png}
\caption{插图选自《罗伯特·胡克的遗作……》,发表于《学者学报》(Acta Eruditorum),1707年。} \label{fig_HK_3}
\end{figure}
尽管约翰·奥布里形容霍克是一个“具有伟大美德和善良”的人,[62] 但关于霍克性格中令人不快的一面也有许多记录。据他的第一位传记作者理查德·沃勒描述,霍克“其人卑微”,“性情忧郁、多疑且嫉妒”。[63] 沃勒的评论在随后的200多年里影响了其他作者,导致许多书籍和文章——特别是艾萨克·牛顿的传记——将霍克描绘成一个不满、自私、反社会的牢骚满腹者。例如,亚瑟·贝里称霍克“声称几乎所有当时的科学发现都是他的功劳”。[64] 萨利文写道,他是“彻底不择手段的”,在与牛顿的交往中表现出“不安的自负”。[65] 曼纽尔形容霍克“爱争吵、嫉妒且报复心强”。[66] 根据莫尔的描述,霍克有着“愤世嫉俗的性情”和“刻薄的言辞”。[67] 安德拉德对霍克更具同情心,但仍将其描述为“难以相处”、“多疑”和“易怒”。[68] 1675年10月,皇家学会理事会曾考虑过一项将霍克开除的动议,原因是他就钟表设计的科学优先权问题攻击克里斯蒂安·惠更斯,但该动议未获通过。[69] 霍克的传记作者艾伦·德雷克指出:

> 如果研究当时的知识氛围,他所涉及的争议和竞争几乎是普遍现象,而非例外。相比于他的一些同时代人,霍克在面对涉及自己发现和发明的争议时的反应显得相对温和。[70]

1935年霍克日记的出版[71] 揭示了关于他社会和家庭关系的许多此前未知的细节。他的传记作者玛格丽特·埃斯皮纳斯指出:“通常将霍克描绘成一个忧郁...隐士的形象完全是错误的”。[72] 他与一些著名的工匠有交往,比如钟表匠托马斯·汤皮恩[73] 和仪器制造师克里斯托弗·考克斯。[74] 霍克经常与克里斯托弗·雷恩会面,两人有许多共同兴趣,并与约翰·奥布里保持了持久的友谊。他的日记还经常提到在咖啡馆和酒馆的会面,以及与罗伯特·波义耳共进晚餐的记录。在许多场合,他会与实验室助理哈里·亨特一起喝茶。尽管他大部分时间独自生活,除了负责打理他家务的仆人外,他的侄女格雷斯·霍克和他的堂弟汤姆·贾尔斯曾在他们年幼时与他同住过几年。[75]

胡克从未结婚。根据他的日记,胡克在侄女格蕾丝年满16岁后与她发生了性关系。自10岁起,格蕾丝一直由胡克监护。[76][77]他还与几名女仆和管家有过性关系。胡克的传记作者斯蒂芬·英伍德认为格蕾丝是他生命中的挚爱,当她在1687年去世时,他深受打击。英伍德还提到“他和格蕾丝之间的年龄差距很常见,不会像我们这样让同时代的人感到不安”。尽管如此,这种乱伦关系如果被发现,仍然会受到教会法庭的谴责和审判,但在1660年后,这并不是一项死罪。[78][]

自幼,霍克患有偏头痛、耳鸣、眩晕和失眠发作。[80] 他还患有脊柱畸形,被诊断为舍尔曼氏驼背,这使得他在中年和晚年时期身体“瘦弱且弯曲,头部过大,眼睛凸出”。[81] 霍克以科学的态度对待这些问题,通过自我治疗进行实验,在日记中认真记录症状、药物和效果。他经常使用氨水、催吐剂、泻药和鸦片,这些药物似乎随着时间的推移对他的身心健康产生了越来越大的影响。[82]

霍克于1703年3月3日逝世于伦敦,在生命的最后一年中,他失明且卧病在床。在格雷沙姆学院的房间里发现了一个装有8000英镑现金和黄金的箱子。[83][k] 他的藏书包括超过3000本拉丁文、法文、意大利文和英文书籍。[83] 尽管他曾提到希望为皇家学会留下丰厚的遗赠,以他的名字命名图书馆、实验室和讲座,但没有遗嘱被找到,资金最终传给了一位名为伊丽莎白·斯蒂芬斯的堂亲。[84] 霍克被安葬在伦敦城毕晓普门的圣海伦教堂,[85] 但他的墓地确切位置至今未知。
\subsection{科学研究}  
胡克在皇家学会的职责是根据自己的方法或会员的建议进行实验演示。他最早的演示之一包括关于空气本质的讨论,以及密封热空气的玻璃泡的内爆实验。[57] 他还展示了一个狗在胸腔打开的情况下可以存活,只要将空气泵入和排出其肺部。[86][l] 胡克观察到了静脉血和动脉血之间的差异,从而证明了“生命之食”(Pabulum vitae)和“火焰”(flammae)是同一种东西。[89][90]  

此外,他还进行了关于重力、物体下落、物体称重、不同时高度下气压的测量,以及长达200英尺(约61米)摆锤运动的实验。[89] 他的传记作者玛格丽特·埃斯皮纳斯(Margaret 'Espinasse)将他描述为英国第一位气象学家,这是基于他在《制作天气历史的方法》(Method for making a history of the weather)一文中的描述。[91] (胡克在文中指出,温度计、湿度计、风速计和记录表是进行规范天气记录所必需的。[92][n])
\subsubsection{天文学}
\begin{figure}[ht]
\centering
\includegraphics[width=8cm]{./figures/773342ab19eb7193.png}
\caption{胡克在这幅土星的绘图中指出了由土星的球体和环彼此投射的阴影(标记为a和b)。} \label{fig_HK_4}
\end{figure}
1664年5月,胡克使用一台12英尺(3.7米)的折射望远镜观测木星的大红斑,并持续观察了两个小时,记录下它在木星表面移动的过程。1665年3月,他发表了这一发现,并由此意大利天文学家乔瓦尼·卡西尼(Giovanni Cassini)计算出木星的自转周期为9小时55分钟。[93]

胡克研究的最具挑战性的问题之一是测量地球到太阳以外的恒星的距离。他选择了龙座伽马星(Gamma Draconis)并采用视差测量法进行观测。1669年,经过数月的观测,胡克认为他已获得所需的结果。然而,现在已知他的仪器精度远不足以获得准确测量结果。[94]  

胡克的《显微图志》(*Micrographia*)包含对昴宿星团和月球陨石坑的插图。他还通过实验研究这些陨石坑的形成,并得出结论:它们的存在意味着月球必须具有自身的重力,这一观点大大偏离了当时的亚里士多德天体模型。[95] 此外,他还是土星环的早期观察者之一,[96] 并于1664年发现了首批双星系统之一的白羊座伽马星(Gamma Arietis)。[97]  

为了实现这些发现,胡克需要比当时已有仪器更先进的设备。因此,他发明了三种新机制:胡克关节,一种复杂的万向节,用于使仪器平滑跟踪被观测天体的视运动;首个钟表驱动装置,用于自动化观测过程;以及微米螺杆,使其观测精度达到十角秒。[98][99] 胡克对折射望远镜感到不满,因此他制造了第一台实用的格里高利反射望远镜,使用镀银的玻璃镜面。[100][101][o]
\begin{figure}[ht]
\centering
\includegraphics[width=6cm]{./figures/87071c0e0c5e8539.png}
\caption{胡克《显微图志》中的月球和昴宿星团绘图} \label{fig_HK_5}
\end{figure}
\subsubsection{力学}
1660年,胡克发现了以他名字命名的弹性定律,该定律描述了弹簧拉力与伸长量之间的线性关系。胡克最初以一个字谜的形式表达了这一发现:“ceiiinosssttuv”,并在1678年解开这个字谜,公开其含义为“Ut tensio, sic vis”(“拉力随伸长量变化”)。[103] 他在弹性研究上的成果最终促成了平衡弹簧(或称发条游丝)的发明,这使得便携式计时器——手表能够以相对准确的方式记录时间。关于这一发明的优先权问题,胡克与克里斯蒂安·惠更斯展开了一场长期而激烈的争论,甚至在两人去世后仍持续了几个世纪。然而,皇家学会于1670年6月23日的记录中提到的一次实验展示了由平衡弹簧控制的手表,这可能支持胡克对这一创意优先权的主张。[104] 尽管如此,惠更斯通常被认为是第一个实际制造出使用平衡弹簧手表的人。[105][106]

胡克通过字谜宣布弹性定律的方式,是当时科学家(如胡克、惠更斯和伽利略)用来确立优先权的一种方法,而无需立即公开细节。[107] 胡克还利用机械类比来理解基本的物理过程,例如球形摆的运动、球在空心锥中的运动,以证明重力的中心力作用,以及带有点荷载的悬挂链网,用以寻找顶部承重十字架的穹顶的最佳形状。[108][109]

尽管仍有报道声称相反,[110] 胡克并未对托马斯·纽科门发明蒸汽机产生影响。这一错误的说法起源于《大英百科全书》第三版的一篇文章,现已被证实是误传。[111]

\subsubsection{万有引力}
胡克的许多同时代人(如艾萨克·牛顿)相信以太是一种介质,用于在分离的天体之间传递引力和斥力。[112][113] 然而,胡克在其著作《显微图谱》(*Micrographia*, 1665)中提出了引力的吸引原理。在1666年提交给皇家学会的一份通信中,[114] 他写道:

我将解释一个与现有任何理论完全不同的宇宙体系。它基于以下几个原则:1. 所有天体不仅其各部分对自身的中心具有引力,它们在各自的作用范围内还会相互吸引。2. 所有具有简单运动的物体,将会继续沿直线运动,除非受到某种外力的持续作用,使其偏离直线并描述圆、椭圆或其他曲线。3. 这种引力随着物体距离的接近而增强。至于这些力量随距离增加而减弱的比例,我承认尚未发现……

胡克在1674年发表的格雷沙姆演讲《通过观察证明地球运动的尝试》(*An Attempt to Prove the Motion of the Earth by Observations*,于1679年出版)中指出,引力适用于“所有天体”,[115] 并重申了上述三个原则。[116]

然而,直到1674年,胡克的陈述中都未提及这些引力可能符合或适用反平方定律。他的引力模型尚未实现普适性,尽管比之前的假说更接近普适性。[117] 胡克未能提供相应的证据或数学证明;他在1674年表示:“至于这些(引力吸引)不同程度,我尚未通过实验验证。”这表明他尚不清楚引力可能遵循的具体规律。关于他的整体提议,他补充说:“我目前仅仅是提出一个想法……由于我自己手头还有许多其他事情要完成,因此无法很好地专注于此研究。”[116]

1679年11月,胡克发起了与牛顿的一次著名书信交流,这些信件于1960年首次出版。[118] 胡克表面上的目的是告知牛顿,他(胡克)已被任命为皇家学会的通信管理人;[119] 因此,胡克希望从学会成员处了解他们的研究或他们对其他人研究的看法。胡克向牛顿询问了关于各种问题的意见。其中,胡克提到了“通过切线的直线运动与朝向中心天体的吸引运动的合成形成的行星天体运动”;他的关于弹性的“定律或原因”的假说;一项来自巴黎的新行星运动假说(胡克详细描述了该假说);改善或执行国家测绘工作的努力;以及伦敦和剑桥之间的纬度差异。[120]  

牛顿的回信提出了“我自己的一个想法”,涉及一种地面实验,而不是针对行星运动的提议。这一实验设想使用悬挂在空中的物体并随后释放它。胡克希望探讨牛顿如何认为该实验能够通过物体偏离垂直线的方向来揭示地球的运动。但胡克还进一步假设,如果没有地球的阻碍,物体的运动会沿着一条螺旋路径继续到达地球中心。胡克不同意牛顿对物体继续运动的想法。随后,两人展开了进一步的简短通信;在通信的末尾,胡克于1680年1月6日写信给牛顿,表述了他的“假设……吸引力总是与中心距离的倒数平方成比例,因此速度将与吸引力的平方根成反比,并因此如开普勒所假设的,与距离的倒数成正比”。[121] (胡克关于速度的推论是不正确的。[122])

1686年,当牛顿的《自然哲学的数学原理》(\textbf{Principia})第一卷提交皇家学会时,胡克表示他曾向牛顿提出过“重力随中心距离的倒数平方递减的规律”的“概念”。与此同时,根据埃德蒙·哈雷当时的记录,胡克也承认“由此生成的曲线的论证”完全是牛顿的功劳。[123]  

根据2002年对倒数平方定律早期历史的评估:“到17世纪60年代末,‘重力与距离平方成反比’这一假设相当普遍,并且因不同理由由不同的人提出”。[124] 在17世纪60年代,牛顿在假设行星运动为圆形的情况下证明,径向力与到中心的距离呈反平方关系。[125] 牛顿在1686年5月面对胡克关于倒数平方定律优先权的主张时否认胡克应被认作这一观点的创作者,理由包括引用其他人的先前工作。[126] 牛顿还表示,即使他确实首次从胡克那里听说了倒数平方比例(牛顿称并非如此),他仍然应有部分权利,因为他进行了数学推导和论证。这些推导和论证使得观测可以作为这一定律准确性的证据,而根据牛顿的说法,胡克在没有数学论证和支持假设的证据的情况下,只能猜测这一规律在“离中心很远的距离”上大致成立。[127]

牛顿在《自然哲学的数学原理》(\textbf{Principia})的所有版本中确实承认并认可了胡克及其他人分别认识到太阳系中的倒数平方定律。在第一卷中关于命题四的“学术注释”(\textbf{Scholium to Proposition 4})中,牛顿提到了伦、胡克和哈雷的贡献。[128] 在给哈雷的一封信中,牛顿还承认他在1679年至1680年间与胡克的通信重新激发了他对天文学问题的兴趣,但牛顿表示,这并不意味着胡克告诉了他任何新的或原创性的东西。牛顿写道:

“然而,我并未从他那里获得任何关于此事的启发……只是由于他给我带来的分心,使我从其他研究转向思考这些问题;以及由于他在写作中以一种独断的方式声称自己已经找到了椭圆运动,这促使我尝试验证它。”[129]

尽管牛顿主要以数学分析及其应用和光学实验的先驱而著称,但胡克则是一位极具创造力的实验者,其实验范围非常广泛。然而,胡克将一些他的想法(如关于重力的理论)留在了未完善的状态。1759年,在牛顿和胡克去世数十年后,著名的数学天文学家亚历克西斯·克莱罗特(Alexis Clairaut)回顾了胡克关于重力的已发表研究。根据斯蒂芬·彼得·里高(Stephen Peter Rigaud)的说法,克莱罗特写道:

“胡克和开普勒的例子表明,从模糊地察觉到真理到真正证明真理之间有多么遥远的距离。”[q][130]

I. 伯纳德·科恩(I. Bernard Cohen)评价道:“胡克对倒数平方定律的主张掩盖了牛顿更为根本的一个贡献:对曲线轨道运动的分析。在要求过多的荣誉时,胡克实际上剥夺了自己应得的荣誉,因为他确实提出了一个开创性的想法。”[131]
\subsubsection{钟表学}
\begin{figure}[ht]
\centering
\includegraphics[width=8cm]{./figures/95e4a9fa037bc419.png}
\caption{克里斯蒂安·惠更斯绘制的图示,展示了他最早的游丝之一,该游丝与一个摆轮相连。} \label{fig_HK_6}
\end{figure}
胡克在计时科学领域做出了重要贡献,并深度参与了当时的技术进步,包括:改进钟摆作为钟表的更佳调节器,提高钟表机制的精确度,以及利用游丝改进怀表的计时性能。

伽利略曾观察到钟摆的规律性,惠更斯首次将钟摆应用于钟表中;[132] 1668年,胡克展示了一种新装置,可以在不稳定的环境中保持钟摆规律摆动。[133] 他发明的齿轮切割机大幅提高了钟表的精确度和准确性。[133] 据沃勒(Waller)报道,到胡克去世时,这一发明已在钟表制造商中被广泛使用。[89]

胡克宣称,他想出了建造海洋航时器以确定经度的方法。[134][q] 在博伊尔及其他人的帮助下,他试图为其申请专利。在此过程中,胡克展示了一种他自己设计的怀表,该怀表配备了一个与摆轮轴相连的螺旋弹簧。然而,胡克拒绝接受拟议合同中关于独家使用这一想法的逃避条款,最终导致该专利的放弃。[134][r]

胡克独立于惠更斯开发了游丝的原理,并且至少早于惠更斯五年。[135] 惠更斯于1675年2月在《学者杂志》(*Journal de Scavans*)上发表了自己的研究成果,并制造了第一块使用游丝的功能性怀表。[136]
\subsubsection{显微镜学} 
1663年至1664年间,胡克进行了显微镜观察和部分天文观察,并将其整理于1665年出版的《显微图志》中。这本书描述了显微镜和望远镜的观察结果,同时包含了他在生物学领域的原创研究。书中记录了对微生物(如微型真菌**毛霉菌**)的最早观察。[13][14] 胡克创造了“细胞”(cell)这一术语,形容植物结构与蜂巢中的蜂窝相似。[137]  

胡克用于观察并创作《显微图志》的显微镜由克里斯托弗·科克在伦敦为其手工制作,皮革和金饰相辅,现陈列于马里兰州国家健康与医学博物馆。[7] 胡克的研究基于亨利·鲍尔的工作,后者在《实验哲学》(1663年)中发表了显微镜研究成果。[6] 此后,荷兰科学家安东尼·范·列文虎克通过提高显微镜的放大倍数,进一步揭示了原生动物、血细胞和精子细胞的存在。[138][139]  

《显微图志》还包含胡克(或胡克与博伊尔合作)关于燃烧的想法。胡克通过实验得出结论:燃烧需要空气中的某种成分。现代科学家对此结论表示认可,但在17世纪,这一概念鲜为人知。他还推断呼吸和燃烧需要空气中的特定且有限的成分。[140] 根据化学史学家帕廷顿的说法,如果“胡克继续他的燃烧实验,很可能他会发现氧气”。[141]  

塞缪尔·佩皮斯在1665年1月21日的日记中写道:“睡觉前,我在房间里坐着一直读到两点,读的是胡克先生的《显微观察》,这是我一生中读过的最有创意的书。”[142]
\begin{figure}[ht]
\centering
\includegraphics[width=14.25cm]{./figures/3cd04278a65cfabc.png}
\caption{} \label{fig_HK_7}
\end{figure}
\subsubsection{古生物学与地质学} 
在《显微图志》中,胡克对化石木材进行了观察,并将其显微结构与普通木材进行了比较。这使他得出结论:如硅化木和菊石等化石化的物体实际上是曾经的生物遗骸,它们被浸泡在富含矿物质的水中而石化。[143] 胡克认为,这些化石可以提供有关地球生命历史的可靠线索,尽管当时像约翰·雷这样的博物学家出于神学上的理由反对灭绝的概念,但胡克提出,在某些情况下,这些化石可能代表由于某些地质灾难而灭绝的物种。[144]  

1668年的一系列讲座中,胡克提出了一个在当时被视为异端的观点,即地球表面是由火山和地震形成的,地震导致贝壳化石出现在远高于海平面的地方。[145]  

1835年,苏格兰地质学家、达尔文的同事查尔斯·莱尔在其《地质学原理》中评价胡克时写道:“他的论述……是那个时代最具有哲学性的作品,关于自然界有机和无机领域中过去变化的成因。”[146]
\subsubsection{记忆}  
胡克关于人类记忆的科学模型是最早的此类模型之一。1682年,他在皇家学会的一次讲座中提出了一个机械类比的人类记忆模型,与之前作者主要基于哲学的模型有很大不同。[147] 这一模型涵盖了记忆的编码、记忆容量、重复、提取和遗忘等要素——其中一些与现代认知极为接近。[148] 据心理学教授道格拉斯·亨茨曼(Douglas Hintzman)所言,胡克模型的最有趣之处在于:它允许注意力和其他自上而下的因素影响记忆编码;它利用共振实现了并行的、依赖提示的记忆提取;它解释了记忆的新近效应;它为重复和启动效应提供了单一系统的解释;遗忘的幂律可以直接从该模型的假设中推导出来。[148]  
\subsubsection{其他} 
1680年7月8日,胡克观察到了与玻璃板振动模式相关的节点图案。他用弓擦过撒有面粉的玻璃板边缘,观察到节点图案显现。[149][150] 在声学方面,1681年,胡克向皇家学会展示了如何利用旋转的黄铜齿轮产生特定比例切割齿所产生的音乐音调。[151]
\subsection{建筑学}
\textbf{罗伯特·胡克是伦敦市的测量师,也是克里斯托弗·雷恩的首席助理,他在1666年伦敦大火后协助雷恩重建伦敦。[153]} 胡克设计了\textbf{伦敦大火纪念碑}(1672年)、[154][155][s] \textbf{布卢姆斯伯里的蒙塔古宅邸}(1674年)、[156] 和后来被称为“贝德兰”的\textbf{贝特勒姆皇家医院}(1674年)。[157] 其他由胡克设计的建筑包括\textbf{皇家内科医学院}(1679年);[158] \textbf{阿斯克医院}(1679年)、[159] \textbf{华威郡的拉格利大厅}(1680年);[160] \textbf{白金汉郡的威伦圣玛丽抹大拉教堂}(1680年);[161] 以及\textbf{威尔特郡的拉姆斯伯里庄园}(1681年)。[162] 他还参与了许多伦敦教堂的重建工作,这些教堂在大火中被毁;胡克通常被雷恩分包。在1671年至1696年间,雷恩的办公室向胡克支付了2,820英镑的费用,[t] 这一数额超过了他在皇家学会和卡特勒讲席职位上的所有收入。[163]  

\textbf{雷恩和胡克都是热衷于天文学的人}。\textbf{伦敦大火纪念碑}被设计为可用于天文观测的天顶望远镜,但由于交通震动未能实现这一功能。[164][165] 这一设计遗产可以在螺旋楼梯的构造中看到,该楼梯没有中央支柱,并且地下观察室仍然保留。胡克还与雷恩合作设计了\textbf{圣保罗大教堂};胡克计算出理想的拱形形状是倒置的悬链线,因此得出一系列圆形拱形结构构成了大教堂圆顶的理想形状。[109]  

\textbf{在大火后的重建中,胡克提出以网格模式设计伦敦街道,规划宽阔的林荫大道和交通主干道,}[166] 这一模式后来被用于\textbf{奥斯曼改造巴黎}以及许多美国城市的设计中,雷恩和其他人也提出了类似的提案。国王的决定是,建造的预期成本、赔偿以及尽快恢复贸易和人口的需要意味着城市将按照原有的地产边界重建。[167] 胡克负责测绘废墟,以确定地基、街道边界和地产界限。他还积极参与了\textbf{市议会法案}(1667年4月)的起草,该法案规定了正式确认和认证原始地基的程序。[168] 根据丽莎·贾丁(Lisa Jardine)的说法:“从10月4日开始的四周内,胡克帮助绘制了火灾受损区域的地图,开始为伦敦编制土地信息系统,并起草了重建的议会法案建筑法规”。[169] 史蒂芬·因伍德(Stephen Inwood)表示:“测量报告——通常由胡克撰写——展现出他能巧妙地处理复杂邻里纠纷的能力,并从纷繁的诉求和反诉中得出简明而明智的建议”。[170]  

胡克还必须测量并确认将被强制购买用于规划道路拓宽的土地,以便支付补偿金。[171] 1670年,他被任命为\textbf{皇家工程测量师}。[172] 胡克与苏格兰制图师兼印刷家\textbf{约翰·奥格尔比}合作,精准详尽的测量工作促成了1677年一幅大比例尺伦敦地图的绘制,[152] 这是已知的首幅具有特定比例(1:1200)的地图。[173]


**胡克的画像**

没有经过确认证实的罗伯特·胡克画像现存,这种情况有时被归因于胡克与艾萨克·牛顿之间激烈的冲突。然而,胡克的传记作者艾伦·查普曼驳斥了关于牛顿或其追随者故意毁掉胡克画像的说法,称其为一种神话。[176] 1710年,德国古董学家兼学者扎卡里亚斯·康拉德·冯·乌芬巴赫访问皇家学会时提到,他被展示了“波义耳和胡克”的画像,并称画像非常逼真。然而,尽管波义耳的画像保存了下来,胡克的画像却遗失了。[10][177] 在胡克的时代,皇家学会在格雷沙姆学院举行会议。但在胡克去世后不久,牛顿成为学会会长,并计划迁入新的会址。1710年,当皇家学会搬迁到新址时,胡克的画像是唯一遗失的画像,并且至今仍下落不明。[178] 根据胡克的日记,他曾为知名艺术家玛丽·比尔坐过画像,因此这样的画像可能曾经存在。[179] 相反,查普曼注意到,胡克去世后不久出版的理查德·沃勒的《罗伯特·胡克的遗作》中,尽管插图丰富,但却没有胡克的画像。[176]

两篇同时代的文字描述记录了胡克的外貌。他的密友约翰·奥布里在胡克中年时创作巅峰期这样描述他:

他中等身材,有点驼背,面色苍白,脸不长,但头颅很大,眼睛圆鼓,不够敏捷,是灰色眼睛。他有一头漂亮的棕色头发,微微卷曲且湿润。他在饮食等方面一向节制。

——《简短的生平》[9]

理查德·沃勒在1705年的《罗伯特·胡克的遗作》中这样描述年老的胡克:

至于他的外貌,他很不起眼,身体非常驼背。不过我听他说过,并从其他人那里得知,他在16岁之前一直很挺直,后来因经常练习用车床而变得弯曲……他一直很苍白和瘦削,晚年简直成了皮包骨,面容憔悴。他的眼睛是灰色的,年轻时目光敏锐,鼻子细长适中,嘴巴宽度一般,上唇薄,下巴尖,额头宽阔,头颅中等大小。他留着自己棕色的头发,很长并且凌乱地垂在脸上,未修剪。[63]

1939年7月3日,《时代》杂志发表了一幅据称是胡克的画像,但当阿什利·蒙塔古追溯其来源时,发现画像与胡克没有可验证的关联。蒙塔古指出,这两篇同时代的文字描述彼此一致,但都与《时代》杂志上的画像不符。[180]

2003年,历史学家丽莎·贾丁推测一幅新发现的画像可能是胡克,但辛辛那提大学的威廉·B·詹森驳斥了这一假设,并确认画像的主人是佛兰德学者扬·巴普蒂斯特·范·赫尔蒙特。[174][175]