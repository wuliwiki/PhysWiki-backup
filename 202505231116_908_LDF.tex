% 鲁道夫·佩尔斯(综述)
% license CCBYSA3
% type Wiki

本文根据 CC-BY-SA 协议转载翻译自维基百科\href{https://en.wikipedia.org/wiki/Rudolf_Peierls}{相关文章}。

\begin{figure}[ht]
\centering
\includegraphics[width=6cm]{./figures/de58c2e988983035.png}
\caption{佩耶尔斯,摄于1966年} \label{fig_LDF_1}
\end{figure}
鲁道夫·恩斯特·佩耶尔斯爵士,CBE勋衔,英国皇家学会会士(FRS)(/ˈpaɪ.ərlz/;德语:[ˈpaɪɐls];1907年6月5日-1995年9月19日),是一位出生于德国的英国物理学家,在英国核武器计划“管合金”以及其后的“曼哈顿计划”(即盟军联合核弹计划)中发挥了重要作用。《今日物理》于1996年发表的讣告称他是“在核物理爆发进入世界事务的戏剧中扮演关键角色的人物之一”\(^\text{[1]}\)。

佩耶尔斯在柏林大学学习物理学,随后在慕尼黑大学跟随阿诺德·索末菲尔德、在莱比锡大学跟随沃尔夫冈·海森堡、以及在苏黎世联邦理工大学跟随沃尔夫冈·泡利深造。1929年,他获得莱比锡大学的博士学位后,成为泡利在苏黎世的助理。1932年,他获得了洛克菲勒奖学金,并利用这一奖学金在罗马跟随恩里科·费米学习,随后又在剑桥大学的卡文迪许实验室跟随拉尔夫·霍普金森·福勒进行研究。由于其犹太背景,他决定在1933年阿道夫·希特勒上台后不返回德国,而是留在英国,在那里他与汉斯·贝特一起工作,先是在曼彻斯特维多利亚大学,然后在剑桥大学的蒙德实验室。1937年,马克·奥利芬特,新任的伯明翰大学澳大利亚籍物理学教授,邀请他担任应用数学的新职位。

1940年3月,佩耶尔斯与奥托·罗伯特·弗里施共同撰写了《弗里施–佩耶尔斯备忘录》。这篇简短的论文首次指出,仅需少量可裂变的铀-235就可以制造出一枚原子弹。在此之前,人们普遍认为制造这种炸弹需要数吨铀,因此被认为在实际操作上不可行。这篇论文在最初引起英国、随后引起美国政府对核武器的兴趣方面起到了关键作用。佩耶尔斯还促成了他的同胞克劳斯·富克斯加入“管合金”(Tube Alloys,即英国核武器项目)的工作,但由于富克斯于1950年被揭露为苏联间谍,佩耶尔斯也因此一度受到怀疑。

战后,佩耶尔斯回到伯明翰大学工作,直至1963年。随后,他在牛津大学担任物理学怀克姆讲座教授并成为新学院(New College)的院士,直至1974年退休。\(^\text{[2]}\)在伯明翰期间,他的研究涵盖核力、散射、量子场论、原子核的集体运动、输运理论和统计力学,并担任哈威尔原子能研究机构的顾问。他获得了许多荣誉奖项,包括1968年获封爵士。他还著有多部著作,包括《固体的量子理论》(Quantum Theory of Solids)、《自然法则》(The Laws of Nature, 1955)、《理论物理的惊奇》(Surprises in Theoretical Physics, 1979)、《理论物理的更多惊奇》(More Surprises in Theoretical Physics, 1991)以及自传《过客鸟》(Bird of Passage, 1985)。由于对自己曾参与推动的核武器研发心存忧虑,他积极参与《原子科学家公报》的工作,曾担任英国原子科学家协会主席,并积极参与“普格沃什运动”。
\subsection{早年生活}
鲁道夫·恩斯特·佩耶尔斯出生于柏林郊区奥伯舍讷韦德,是海因里希·佩耶尔斯和第一任妻子伊丽莎白(娘家姓魏格特,Weigert)所生的三个孩子中最小的一个。海因里希是一位电气工程师,出身于一个犹太商人家庭,担任通用电气公司(Allgemeine Elektricitäts-Gesellschaft,简称AEG)一家电缆厂的总经理。鲁道夫有一个哥哥阿尔弗雷德和一个姐姐安妮。\(^\text{[3]}\)他母亲于1921年死于霍奇金淋巴瘤,\(^\text{[4]}\) 父亲之后再婚,迎娶了赫尔曼,即剧作家路德维希·富尔达的嫂子。\(^\text{[5]}\)这个家庭是犹太人,但已完成同化,佩耶尔斯和他的兄姐在孩提时代受洗成为路德宗教徒。\(^\text{[3]}\)成年后,佩耶尔斯脱离了教会。\(^\text{[6]}\)

由于需要配戴眼镜,而父母担心他会弄丢或摔坏眼镜,佩耶尔斯比同龄人晚一年才开始上学。在当地的预科学校学习了两年后,他进入了本地的洪堡文理中学(Humboldt Gymnasium \(^\text{[de]}\)),在那里度过了接下来的九年,并于1925年通过中学毕业考试。\(^\text{[7]}\)他原本希望学习工程学,但父母怀疑他缺乏实践能力,建议他改学物理。\(^\text{[5]}\)他随后进入柏林大学,听了马克斯·普朗克、瓦尔特·玻特和瓦尔特·能斯特等人的讲座。同期的学生还包括库尔特·赫希和凯特·斯佩尔林。由于物理实验课程人数过多,新生被鼓励先修理论物理课。佩耶尔斯发现自己非常喜欢这一学科。\(^\text{[8]}\)

1926年,佩耶尔斯决定转学到慕尼黑大学,师从被誉为理论物理最伟大教师的阿诺德·索末菲尔德7。在那里,他的同学包括汉斯·贝特7、赫尔曼·布吕克和威廉·V·休斯顿7。\(^\text{[9]}\)当时,玻尔–索末菲尔德理论正逐渐被维尔纳·海森堡与保罗·狄拉克提出的新量子力学所取代。\(^\text{[10]}\) 1928年,索末菲尔德开始进行一次世界旅行。根据他的建议,佩耶尔斯转学至莱比锡大学,当时海森堡已于1927年被任命为该校的教授。\(^\text{[5][11]}\)

海森堡为佩耶尔斯布置了一个关于铁磁性的研究课题。人们已经知道铁磁性是由于金属中电子自旋的排列一致所引起的,但其背后的原因尚不清楚。海森堡怀疑这是由泡利不相容原理引发的一种量子力学效应所致。\(^\text{[12]}\)尽管佩耶尔斯未能成功发展出关于铁磁性的理论,但他在霍尔效应方面的研究却更为富有成果。异常霍尔效应无法用经典金属理论来解释,而海森堡敏锐地意识到,这是一个可以用量子力学加以解释的机会。佩耶尔斯成功做到了这一点,并由此发表了他的第一篇学术论文。\(^\text{[13]}\)

1929年,海森堡前往美国、中国、日本和印度讲学,\(^\text{[13]}\)并推荐佩耶尔斯转到苏黎世联邦理工学院,在那里他师从沃尔夫冈·泡利。泡利给他布置了一个课题,研究晶体晶格中原子的振动。佩耶尔斯在研究中发现并命名了一种现象——“Umklapp 散射”(回折散射)。他将这项研究作为博士论文提交,题为《晶体中热传导的动力学理论》,\(^\text{[14]}\)并于1929年被莱比锡大学接受授予博士学位。\(^\text{[15]}\)他的理论对金属在极低温下的行为做出了具体预测,但又过了二十年,相关实验技术的发展才得以验证这些预测。\(^\text{[3]}\)
\subsection{早期职业生涯}
佩耶尔斯接受了泡利的邀请,接替费利克斯·布洛赫担任其助手。当时,列夫·朗道(Lev Landau)正在苏联政府的资助下在苏黎世进行访问研究,佩耶尔斯与朗道成为朋友。他们合作推导出一系列类似薛定谔方程的光子波动方程。不幸的是,这些方程虽然形式复杂,但从物理角度来看却毫无意义。\(^\text{[16]}\)
1930年,佩耶尔斯前往荷兰拜访汉斯·克拉默斯,又前往哥本哈根拜访尼尔斯·玻尔(Niels Bohr)。\(^\text{[17]}\)

1930年8月,泡利与佩耶尔斯前往敖德萨参加一场物理学大会,在那里他们结识了一位年轻的物理学毕业生——尤金妮娅(Eugenia, 小名 Genia)·尼古拉耶芙娜·坎尼吉瑟尔,她与朗道一样来自列宁格勒。由于他不会俄语,她也不会德语,两人便以英语交流。\(^\text{[16]}\)后来佩耶尔斯去列宁格勒讲学,两人在1931年3月15日结婚。\(^\text{[18]}\)不过,由于护照和出境签证的问题,她不得不等待一段时间。直到当年夏天他们才一同前往苏黎世。他们育有四个孩子:加比·艾伦(Gaby Ellen,生于1933年)、罗纳德·弗兰克(Ronald Frank,生于1935年)、凯瑟琳(Catherine,昵称Kitty,生于1948年)和乔安娜(Joanna,生于1949年)。\(^\text{[3]}\)

佩耶尔斯协助埃贡·奥罗万理解使位错移动所需的力,这项研究后来由弗兰克·纳巴罗进一步发展,被称为“佩耶尔斯–纳巴罗力”。1929年,他在海森堡和泡利的指导下在苏黎世研究固体物理。他早期关于量子物理的研究促成了“正电载流子”理论,以解释半导体的热导和电导行为。他是“空穴”概念在半导体理论中的先驱之一。\(^\text{[19]}\)他在莱昂·布里渊之前便建立了“能带区”的概念,尽管今天这一概念更多地与布里渊的名字联系在一起;他还将这一理论应用于声子,由此他推导出了声子的玻尔兹曼方程和 Umklapp 散射过程。。\(^\text{[1]}\)他将相关论文作为讲师资格论文提交,获得在德国大学任教的资格。\(^\text{[20]}\)《今日物理》曾评论道:“他关于金属中电子的诸多论文已经如此深刻地融入文献之中,以至于很难明确指出他在磁场中电导现象和固体电子空穴理论方面的具体贡献。”\(^\text{[1]}\)
\subsection{流亡中的学术生涯}
1932年,佩耶尔斯获得了洛克菲勒奖学金,用于出国深造。他利用这笔奖学金,先是在罗马跟随恩里科·费米学习,随后前往英国剑桥大学的卡文迪许实验室,在拉尔夫·H·福勒指导下继续研究。\(^\text{[21]}\)在罗马期间,佩耶尔斯完成了两篇关于电子能带结构的论文,其中他引入了“佩耶尔斯替代”,并推导出一个描述金属在低温下抗磁性的通用表达式。这一成果解释了此前困扰物理学界的铋(bismuth)在低温下具有远强于其他金属的抗磁性这一奇异现象。\(^\text{[22][23][24]}\)
\begin{figure}[ht]
\centering
\includegraphics[width=6cm]{./figures/6040822fe73927cf.png}
\caption{伯明翰大学的波音廷物理楼。其建筑风格促成了“红砖大学”这一说法的产生。} \label{fig_LDF_2}
\end{figure}
由于阿道夫·希特勒于1933年在德国上台,佩耶尔斯决定不再返回祖国,而是留在英国。他拒绝了奥托·斯特恩提供的汉堡大学职位邀请。获得英国的居留许可后,他在曼彻斯特维多利亚大学工作,所需资金由“学术援助委员会”提供,该组织旨在帮助来自德国及其他法西斯国家的学术难民。\(^\text{[25]}\)他的直系亲属大多也离开了德国:他的哥哥及其家人定居英国,而他的姐姐一家以及父亲和继母则移居美国,他们的叔叔齐格弗里德也住在那里。\(^\text{[26]}\)

佩耶尔斯在詹姆斯·查德威克的启发下,与汉斯·贝特合作研究了光致裂变以及合金的统计力学。他们的研究成果至今仍是完整合金中结构相变平均场理论的基础。\(^\text{[22]}\)尽管他的主要研究仍集中在金属的电子理论方面,他也涉猎了狄拉克的空穴理论,\(^\text{[27]}\)并与贝特合写了一篇关于中微子的论文。\(^\text{[28]}\)曼彻斯特大学授予他科学博士学位。\(^\text{[29]}\)后来他回到剑桥大学,在蒙德实验室与戴维·肖恩伯格合作研究超导性与液氦。\(^\text{[27]}\)为了使他符合讲授课程的资格,剑桥圣约翰学院(依据校规授予他名誉性质的硕士学位。\(^\text{[30]}\)
\begin{figure}[ht]
\centering
\includegraphics[width=6cm]{./figures/313b1641afd675b4.png}
\caption{佩耶尔斯,摄于1937年} \label{fig_LDF_3}
\end{figure}
1936年,马克·奥利芬特被任命为伯明翰大学物理学教授,他邀请佩耶尔斯出任他正在筹建的一个应用数学讲座教授职位(“应用数学”在当时相当于今天所说的理论物理)。尽管有哈里·马西和哈里·琼斯(Harry Jones \(^\text{[de]}\))的竞争,佩耶尔斯最终获得了这个职位,这也使他终于拥有了一份稳定的终身教职。\(^\text{[31]}\)他的学生包括弗雷德·霍伊尔和来自印度的学生P. L. 卡普尔。\(^\text{[32]}\)他与卡普尔共同推导了核反应的色散公式,这一公式最初由格雷戈里·布赖特和尤金·维格纳在微扰理论中提出,但他们的工作将其推广到更一般的条件。该公式如今被称为“卡普尔–佩耶尔斯推导”。这一推导至今仍被使用,不过在1947年,维格纳与伦纳德·艾森巴德发展出了一种更为广泛使用的替代方法。\(^\text{[32][33]}\)1938年,佩耶尔斯访问哥本哈根,与玻尔和乔治·普拉兹克合作撰写了一篇论文,内容涉及后来被称为“玻尔–佩耶尔斯–普拉兹克关系”的理论。由于第二次世界大战爆发,论文未能正式发表;但草稿被广泛流传供同行评议,最终成为史上被引用最多的未发表论文之一。\(^\text{[34]}\)
\subsection{第二次世界大战}
\subsubsection{弗里施–佩耶尔斯备忘录}
第二次世界大战于1939年9月爆发后,佩耶尔斯开始与同为德国难民的奥托·罗伯特·弗里施共同从事核武器研究。具有讽刺意味的是,由于他们被视为“敌国侨民”,两人被排除在伯明翰大学的雷达研究之外,理由是该项目保密等级过高。\(^\text{[35]}\)佩耶尔斯于1940年3月27日正式归化为英国公民。\(^\text{[36]}\)他渴望参与反对法西斯主义和军国主义的斗争,但唯一愿意接纳他的组织是辅助消防队。\(^\text{[37]}\)他还接受了多伦多大学的提议,将两个孩子送往加拿大,与一个寄养家庭共同生活。\(^\text{[38]}\)
\begin{figure}[ht]
\centering
\includegraphics[width=6cm]{./figures/98e411fd31c8cd49.png}
\caption{纪念弗里施–佩尔斯备忘录的牌匾,位于伯明翰大学波因廷物理楼。} \label{fig_LDF_4}
\end{figure}
1940年2月至3月间,佩耶尔斯与弗里施共同撰写了《弗里施–佩耶尔斯备忘录》,文稿由佩耶尔斯亲自打字。这篇简短的论文首次确立了:仅凭少量可裂变的铀-235便可制造出一枚原子弹。根据当时掌握的信息,他们计算出所需铀-235不到1千克。虽然实际上临界质量约为这个数值的四倍,但在此之前,人们普遍认为制造原子弹需要数吨铀,因此一直被认为在实际操作上不可行。他们还进一步估算了爆炸的威力及其在物理、军事和政治上的影响。\(^\text{[39][40]}\)

《弗里施–佩耶尔斯备忘录》在最初引发英国、随后引发美国政府对原子武器兴趣方面起到了关键作用。1941年,该备忘录的研究成果通过“莫德委员会”的报告传到了美国,这一报告成为促成“曼哈顿计划”启动以及原子弹研制进展的重要契机。正是借助《弗里施–佩耶尔斯备忘录》和莫德委员会的报告,英美科学家才得以从“是否可能造出原子弹”的问题,转向具体思考“如何造出一枚原子弹”。\(^\text{[41]}\)

作为“敌国侨民”,弗里施和佩耶尔斯最初被排除在莫德委员会之外,但这一决定的荒谬性很快被认识到,两人随后被任命为该委员会技术小组的成员。\(^\text{[42]}\)然而,这并不意味着他们获得了从事雷达研究的许可。1940年9月,当奥利芬特允许他的秘书帮助打字弗里施和佩耶尔斯为莫德委员会撰写的论文时,他们仍然不被允许进入秘书所在的纳菲尔德楼,于是佩耶尔斯只能通过蜡筒留声机口述稿件,由秘书转录。\(^\text{[43]}\)

弗里施和佩耶尔斯最初认为热扩散法是实现铀浓缩的最佳途径,但随着这种方法所面临的问题日益显现,他们转而支持气体扩散法,并邀请同为德国难民的专家弗朗茨·西蒙加入,担任该领域的顾问。\(^\text{[43]}\)佩耶尔斯还于1941年5月招募了另一位来自德国的难民克劳斯·富克斯担任自己的助手。\(^\text{[44]}\)
\subsubsection{曼哈顿计划}
根据莫德委员会的研究结果,英国成立了一个名为“管合金”的新机构,用以统筹核武器的研发工作。枢密院主席约翰·安德森爵士成为负责该项目的部长,帝国化学工业公司的华莱士·阿克斯被任命为“管合金”项目的主管。佩耶尔斯、詹姆斯·查德威克和弗朗茨·西蒙被任命为该项目技术委员会的成员,该委员会由阿克斯担任主席。

技术委员会于1941年11月召开了第一次会议,\(^\text{[45]}\)会议有两位美国来访者参加——哈罗德·尤里和乔治·B·皮格拉姆。\(^\text{[46]}\)同年晚些时候,佩耶尔斯飞往美国,先后拜访了纽约的尤里和费米、芝加哥的阿瑟·H·康普顿、伯克利的罗伯特·奥本海默以及弗吉尼亚夏洛茨维尔的杰西·比姆斯。\(^\text{[47]}\)

当乔治·基斯佳科夫斯基提出核武器不会造成太大破坏,因为大部分能量会被消耗在加热空气上时,佩耶尔斯、富克斯、杰弗里·泰勒和J·G·金奇共同推导出相关的流体力学计算,驳斥了这一观点。\(^\text{[48]}\)

1943年8月19日,《魁北克协定》的签署将“Tube Alloys”项目并入了“曼哈顿计划”。\(^\text{[49]}\)阿克斯早已通过电报通知伦敦,指示查德威克、佩尔斯、奥利芬特和西蒙立即前往北美,加入英国派往曼哈顿计划的任务小组。他们于协议签署当天抵达。\(^\text{[50]}\)

西蒙和佩尔斯被派驻至凯莱克斯公司,该公司负责“K-25计划”,即设计和建造美国的气体扩散铀浓缩工厂。\(^\text{[51]}\)虽然凯莱克斯的总部设在伍尔沃斯大楼,但佩尔斯、西蒙和尼古拉斯·库尔蒂的办公室则设在华尔街的英国物资供应使团。\(^\text{[52]}\)托尼·斯凯姆和弗兰克·基尔顿于1944年3月抵达,也加入了他们的行列。库尔蒂于1944年4月返回英国,基尔顿则于9月回国。\(^\text{[51]}\)佩尔斯于1944年2月调往洛斯阿拉莫斯实验室,斯凯姆于7月随后前往,弗克斯则在8月抵达。\(^\text{[53]}\)

在洛斯阿拉莫斯,英国代表团被完全整合进了实验室,英国科学家参与了几乎所有的研究部门,唯独钚的化学和冶金工作除外。\(^\text{[54]}\)当奥本海默任命贝特为实验室重要的理论部(T部)负责人时,得罪了爱德华·泰勒,后者被分配领导一个独立小组,负责研究他提出的“超级炸弹”。随后,奥本海默写信给曼哈顿计划主管莱斯利·R·格罗夫斯准将,请求派遣佩尔斯前来接替泰勒在 T 部门的位置。\(^\text{[55]}\)佩尔斯于1944年2月8日从纽约抵达洛斯阿拉莫斯,\(^\text{[53]}\)后来接替查德威克,成为洛斯阿拉莫斯英国代表团的负责人。\(^\text{[56]}\)

佩尔斯还成为了 T-1(内爆)小组的负责人,\(^\text{[57][58]}\)因此他负责设计用于内爆式核武器的爆炸透镜,以便将爆炸能量聚焦成球形冲击波。\(^\text{[59]}\)他定期向位于华盛顿特区、担任曼哈顿计划英国代表团团长的查德威克汇报工作。当格罗夫斯得知此事后,也要求佩尔斯向他本人提交报告。\(^\text{[60]}\)1945年7月16日,佩尔斯出席了三位一体核试验。\(^\text{[61]}\) 他于1946年1月返回英国。\(^\text{[62]}\)鉴于他在核武器项目中的贡献,佩尔斯在1946年新年授勋中被授予大英帝国司令勋章,\(^\text{[63]}\)同年还获颁美国自由勋章(银棕叶等级)。\(^\text{[64]}\)
\subsubsection{间谍指控}
佩尔斯是将克劳斯·福克斯招募进英国核项目的人,这一举动导致在福克斯于1950年被揭露为苏联间谍后,佩尔斯本人也受到怀疑。1999年,《旁观者》杂志发表了一篇由记者尼古拉斯·法瑞尔撰写的文章,指控佩尔斯是苏联的间谍,引发了佩尔斯家人的强烈愤慨。\(^\text{[65][66]}\) 该文章所依据的信息来自情报史学者奈杰尔·韦斯特,他声称在维诺纳拦截电报中,佩尔斯是代号为“Fogel”,后改为“Pers”的间谍,而他的妻子热尼娅则是代号为“Tina”的间谍。\(^\text{[67]}\)然而,Tina 与 Genia 的特征并不吻合,后经确证,“Tina” 实际上是梅利塔·诺伍德,这一点在1999年被明确揭示。此外,“Pers”在克林顿工程工地工作,而佩尔斯从未在那里工作,因此也不可能是 Pers。\(^\text{[68]}\)最终在2009年,有确凿证据表明“Vogel/Pers”实际上是拉塞尔·A·麦纳特,他是一位土木工程师,受雇于凯莱克斯公司,参与橡树岭设施的建设,曾被尤利乌斯·罗森堡发展为苏联间谍。\(^\text{[69]}\)

战后情报机构怀疑佩尔斯,是有充分理由的。他不仅是招募福克斯的人,还在福克斯的录用和安全审查事务中担任了他的“担保人”,\(^\text{[67]}\)并曾敦促当局给予福克斯完整的安全许可,否则福克斯就无法协助他开展工作。福克斯曾在一段时间内住在佩尔斯家中。\(^\text{[70]}\)此外,佩尔斯的妻子是俄罗斯人,他的兄弟也是如此;在二战前后,他始终与苏联的学术同行保持密切联系。\(^\text{[71]}\)

尽管不像福克斯那样是共产党员,佩尔斯却以持有左翼政治观点而广为人知,\(^\text{[72]}\)其学术圈中也有许多持相似政治立场的同事。\(^\text{[73]}\)1951年,他被拒发签证,无法前往芝加哥参加一次核物理会议。次年提出的类似签证请求虽被批准,\(^\text{[74]}\)但到了1957年,美国方面对他表达了担忧,表示只要他还担任哈威尔原子能研究机构的顾问,美国就不愿与该机构共享情报。\(^\text{[3]}\)
\subsection{战后}
\begin{figure}[ht]
\centering
\includegraphics[width=8cm]{./figures/4d6986525a16fe49.png}
\caption{} \label{fig_LDF_5}
\end{figure}
战后,物理学家非常抢手,佩尔斯收到了多所大学的聘请邀请。\(^\text{[75]}\)他曾认真考虑威廉·劳伦斯·布拉格提供的剑桥职位,但最终决定返回伯明翰大学。\(^\text{[76]}\)他在核力、散射、量子场论、原子核中的集体运动、输运理论以及统计力学等领域展开了研究。\(^\text{[1]}\)佩尔斯基本上已离开固体物理领域,但在1953年,他开始整理自己在该领域的讲义,打算出版成书。在重新审视金属晶体中原子的排列方式时,他注意到一种不稳定性现象,这一发现后来被称为佩尔斯转变。\(^\text{[77]}\)

佩尔斯通过吸引高水平的研究人员,逐步壮大了伯明翰大学的物理系。这些人包括杰拉尔德·E·布朗、马克斯·克鲁克、托尼·斯凯姆、迪克·达利茨、弗里曼·戴森、路易吉·阿里亚尔多·拉迪卡蒂·迪·布罗佐洛、斯图尔特·巴特勒、沃尔特·马歇尔、斯坦利·曼德尔斯塔姆和埃利奥特·H·利布。\(^\text{[78]}\)他还创建了一个数学物理的本科教学体系。佩尔斯亲自讲授量子力学课程,这门课在战前的伯明翰从未开设过。\(^\text{[79]}\)

1946年,佩尔斯成为哈威尔原子能研究机构的顾问。1950年,福克斯被解职后,其理论物理部主任职位由莫里斯·普赖斯兼职代理。当普赖斯因公休假前往美国一年时,佩尔斯临时接替了他的职务。该职位最终由布赖恩·弗劳尔斯正式接任。\(^\text{[80]}\)1957年,佩尔斯从哈威尔辞职,理由是出于美国方面要求的安全审查缺乏透明度,他认为这反映出高级管理人员对他缺乏信任;不过,他于1960年接到重新加入的邀请,并在1963年正式复职,之后又担任顾问长达30年。\(^\text{[81]}\)

1963年,佩尔斯被任命为牛津大学怀克姆物理学教授,并一直任职至1974年退休。\(^\text{[3]}\)他著有多部重要著作,包括:《固体的量子理论》(Quantum Theory of Solids, 1955)、《自然法则》(The Laws of Nature, 1955)、《理论物理中的意外发现》(Surprises in Theoretical Physics, 1979)、《理论物理中的更多意外》(More Surprises in Theoretical Physics, 1991)以及自传《过客一生》(Bird of Passage, 1985)。出于对自己曾参与开发的核武器的忧虑,他参与了《原子科学家公报》的工作,曾任英国原子科学家协会主席,并积极投身于普格沃什运动以及后来称为 的组织“FREEZE”。\(^\text{[1][82]}\)

他的妻子热尼娅于1986年10月26日去世。尽管视力不断恶化,佩尔斯依然活跃于学术界。他喜欢通过计算机将科学论文放大后阅读。1994年,他的健康出现多方面问题,包括心脏、肾脏和肺部疾病,因此搬至牛津郡法默尔附近的一家疗养院——奥肯霍尔特养老院居住。1995年间,他的健康持续恶化,\(^\text{[83]}\)并需定期前往丘吉尔医院进行肾透析治疗,最终于1995年9月19日在那里逝世。\(^\text{[3]}\)
\subsection{荣誉}
佩尔斯于1968年英王寿辰授勋中被封为爵士。\(^\text{[84]}\)他曾获得多个重要奖项,包括:1952年:卢瑟福纪念奖章\(^\text{[85]}\),1959年:皇家奖章\(^\text{[86]}\),1962年:洛伦兹奖章\(^\text{[87]}\),1963年:马克斯·普朗克奖章\(^\text{[88]}\),1968年:古思里奖章与奖金\(^\text{[3]}\),1982年:马泰乌奇奖章\(^\text{[89]}\),1980年:恩里科·费米奖,由美国政府授予,以表彰他在原子能科学方面的杰出贡献。\(^\text{[90]}\)

1986年,他获得科普利奖章,并发表了卢瑟福纪念讲座。\(^\text{[91]}\)1991年,他获得狄拉克奖章与奖金。\(^\text{[3]}\)2004年10月2日,牛津大学理论物理子系所在的大楼被正式命名为鲁道夫·佩尔斯爵士理论物理中心。\(^\text{[92]}\)
\subsection{注释}
\begin{enumerate}
\item Edwards, S.(1996)。“Rudolph E. Peierls”,《今日物理》,第49卷第2期,第74–75页。Bibcode:1996PhT....49b..74E。doi:10.1063/1.2807521。
\item Cathcart, Brian(1995年9月21日)。“讣告:鲁道夫·佩尔斯爵士”,《独立报》。访问日期:2023年4月24日。
\item Dalitz, Richard(2008年 [2004年原版])。“Peierls, Rudolf Ernst (1907–1995)”,载《牛津国家人物传记辞典》(在线版),牛津大学出版社。doi:10.1093/ref\:odnb/60076。(需订阅或英国公共图书馆账号)
\item Peierls 1985,第5、11页。
\item Lee 2007,第268页。
\item Peierls 1985,第6页。
\item Peierls 1985,第6–13页。
\item Peierls 1985,第16–20页。
\item Peierls 1985,第23–24页。
\item Peierls 1985,第25–27页。
\item Peierls 1985,第32–33页。
\item Peierls 1985,第33–34页。
\item Lee 2007,第269页。
\item Peierls, R.(1929)。“Zur kinetischen Theorie der Wärmeleitung in Kristallen”(晶体中热传导的动力学理论),载《物理年鉴》,第395卷第8期,第1055–1101页。Bibcode:1929AnP...395.1055P。doi:10.1002/andp.19293950803。ISSN 1521-3889。
\item Peierls 1985,第40–45页。
\item Lee 2007,第269–270页。
\item Peierls 1985,第54–55页。
\item Peierls 1985,第68页。
\item 1. R.E. Peierls,《论霍尔效应的理论》,1929年。2. R.E. Peierls,《论霍尔效应的理论》,1929年。这两篇论文的英文译文收录于 (《鲁道夫·佩尔斯爵士科学论文选》),由 R.H. Dalitz 与鲁道夫·佩尔斯编辑,World Scientific 出版,1997年。
\item Peierls 1985,第80页。
\item Peierls 1985,第82–93页。
\item Lee 2007,第271页。
\item Peierls, R.(1933年11月)。“Zur Theorie des Diamagnetismus von Leitungselektronen”(导电电子的抗磁性理论),载《物理杂志》,第80卷第11–12期,第763–791页。Bibcode:1933ZPhy...80..763P。doi:10.1007/BF01342591。ISSN 0044-3328。S2CID 119930820。
\item Peierls, R.(1933年3月)。“Zur Theorie des Diamagnetismus von Leitungselektronen. II Starke Magnetfelder”(导电电子的抗磁性理论之二:强磁场),载《物理杂志》(Zeitschrift für Physik,德文),第81卷第3–4期,第186–194页。Bibcode:1933ZPhy...81..186P。doi:10.1007/BF01338364。ISSN 0044-3328。S2CID 122881533。
\item Peierls 1985,第89–96页。
\item Peierls 1985,第141页。
\item Lee 2007,第271–272页。
\item Bethe, H.; Peierls, R.(1934年5月5日)。“The Neutrino”,《自然》,第133卷第532期,第689–690页。Bibcode:1934Natur.133..689B。doi:10.1038/133689b0。ISSN 0028-0836。S2CID 4098234。
\item Peierls 1985,第235页。
\item Peierls 1985,第120–121页。
\item Peierls 1985,第127–128页。
\item Peierls 1985,第134–135页。
\item Kapur, P. L.; Peierls, R.(1938年5月9日)。“The Dispersion Formula for Nuclear Reactions”(核反应的色散公式),《皇家学会会刊A辑》,第166卷第925期,第277–295页。Bibcode:1938RSPSA.166..277K。doi:10.1098/rspa.1938.0093。ISSN 1364-5021。
\item Lee 2007,第273页。
\item Lee 2007,第273–274页。
\item “No. 34844”,《伦敦宪报》,1940年5月7日,第2717页。
\item Peierls 1985,第147–148页。
\item Peierls 1985,第151页、第173页。
\item Gowing 1964,第40–45页。
\item Bernstein, Jeremy(2011年5月1日)。“A Memorandum that Changed the World”(改变世界的一份备忘录)(PDF),《美国物理杂志》,第79卷第5期,第441–446页。Bibcode:2011AmJPh..79..440B。doi:10.1119/1.3533426。ISSN 0002-9505。S2CID 7928950。
\item Gowing 1964,第77–80页。
\item Peierls 1985,第155–156页。
\item Peierls 1985,第158–159页。
\item Peierls 1985,第163页。
\item Gowing 1964,第108–111页。
\item Peierls 1985,第166页。
\item Peierls 1985,第170–174页。
\item Peierls 1985,第176–177页。
\item Gowing 1964,第164–174页。
* Jones 1985,第242–243页。
* Gowing 1964,第250–256页。
* Peierls 1985,第184–185页。
* Szasz 1992,第148–151页。

\end{enumerate}
