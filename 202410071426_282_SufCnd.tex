% 充分必要条件
% keys 充分条件|必要条件|命题
% license Usr
% type Tutor

% 未完成: 引用韦恩图文章, 

\pentry{集合(高中)\nref{nod_HsSet}}{nod_af30}

若由命题 $A$ 能推导出命题 $B$, 则 $A$ 是 $B$ 的\textbf{充分条件}, $B$ 是 $A$ 的\textbf{必要条件}。如何理解这个定义呢?下面举两个例子。

\begin{example}{}
命题 $A$:四边形 $ABCD$ 是一个正方形。

命题 $B$:四边形 $ABCD$ 的四条边相等。

首先我们考虑 $A$ 对 $B$ 的关系。显然,由 $A$ 可以推出 $B$, 说明 $A$ 中有充分的信息能得到 $B$, 所以叫做 $B$ 的\textbf{充分条件}。 $A$ 中包括得到 $B$ 所必要的信息,还\textbf{可能}包括一些其他信息,例如由命题 $A$ 可以得出四边形任意两条临边垂直。 这些多出来的信息并不一定是得到 $B$ 所必须的,因为还有许多其他的四边形四条边相等但并不是正方形。

那如何判断 $A$ 中有没有多余的信息呢?我们可以反过来试图用 $B$ 推导命题 $A$, 若原则上得不出 $A$ (而不是因为我们逻辑水平不够),则证明 $A$ 中有多余的条件。这时我们说 $A$ 不是 $B$ 的\textbf{必要条件},因为 $A$ 中的一些信息是多余的,也就是没有必要的。综上, $A$ 是 $B$ 的\textbf{充分非必要条件}。

现在我们从 $B$ 的角度考虑。虽然由条件 $B$ 不能推导出条件 $A$, 但是 $B$ 是 $A$ 中信息的一部分, $B$ 必须要成立才有可能使 $A$ 成立,也就是说如果 $B$ 不成立 $A$ 就不可能成立(四条边不全相等的四边形一定不是正方形)。所以说 $B$ 是 $A$ 的必要条件。另外,由 $B$ 中的少量信息不能得到 $A$, 所以 $B$ 不是 $A$ 的充分条件。 综上, $B$ 是 $A$ 的\textbf{必要非充分条件}。
\end{example}

\begin{example}{}
命题 $A$ :三角形 $X$ 的其中两内个角分别为 $90^\circ$ 和 $45^\circ$。

命题 $B$ :三角形 $X$ 有两个 $45^\circ$ 的内角。

利用三角形三个内角和为 $180^\circ$ 的事实,可以从 $A$ 推出 $B$, 说明 $A$ 是 $B$ 的充分条件, $B$ 是 $A$ 的必要条件。但也可以从 $B$ 推出 $A$, 说明 $B$ 是 $A$ 的充分条件, $A$ 是 $B$ 的必要条件。所以 $A$ 和 $B$ 既是彼此的充分条件也是彼此的必要条件。所以我们说 $A$ 和 $B$ \textbf{互为充分必要条件}。若 $A$ 是 $B$ 的充分必要条件, $B$ 一定也是 $A$ 的充分必要条件。因为两种表述都意味着 $A$,  $B$ 命题\textbf{等效},所提供的信息都是一样的,两者都没有任何多余的或者缺失的信息。
\end{example}

需要注意的是:
\begin{enumerate}
\item 充分/必要条件是两个命题之间的关系,说一个孤立命题是充分/必要条件没有意义。
\item 讨论充分/必要条件需要在一定的前提下进行。以上两个例子中的前提如: 我们讨论的是欧几里得几何中的平面四边形和三角形。 当然,我们也可以把这个前提直接写在每个命题中。
\item 在证明 $A$ 是 $B$ 的充分必要条件时,需要分别证明 $A$ (相对于 $B$)的充分性和必要性。充分性需要由 $A$ 证明 $B$, 必要性需要由 $B$ 证明 $A$。 
\item 在证明 $A$ 是 $B$ 的充分非必要条件时,除了需要证明 $A$ 的充分性,还需非必要性,即 $B$ 不能推出 $A$。 只要我们可以举出一个 $B$ 成立 $A$ 不成立的反例,就立刻证明了不可能由 $B$ 推出 $A$。 
\end{enumerate}

\subsection{逻辑学中的内涵和外延}
在\textbf{逻辑学}中,我们这样解释:
“马”是一类事物的总称。限制条件诸如“高大的”,“白色的” 等等,具体了其\textbf{内涵},缩小了其\textbf{外延}。

概念的内涵就是指反映在概念中的对象的本质属性或特有属性。

概念的外延是指具有概念所反映的本质属性或特有属性的对象,即概念的适用范围。

两者的区别:

1. 本质不同:
内涵是“属性”,外延是“范围”。属性越多,则符合这个属性的范围越小。属性越少,符合这个属性的范围就越大。

2. 侧重不同:
概念的内涵是指概念的质的方面。
概念的外延是指概念的量的方面。



\subsection{用韦恩图理解}
\pentry{韦恩图\nref{nod_VennD}}{nod_3721}
简要说明:

有命题 $A$ 和 $B$ 
\begin{enumerate}
\item $A$ 推出 $B$ , $B$ 推不出 $A$ ,则为充分不必要,如图:
\begin{figure}[ht]
\centering
\includegraphics[width=8cm]{./figures/813479566ad8bd27.png}
\caption{充分不必要} \label{fig_SufCnd_2}
\end{figure}
口诀:有之必然,无之未必不然
\item $A$ 推出 $B$ , $B$ 推出 $A$ ,则为充要,如图:\begin{figure}[ht]
\centering
\includegraphics[width=10cm]{./figures/977aec7c48daaeea.png}
\caption{充要} \label{fig_SufCnd_3}
\end{figure}
\item $A$ 推不出 $B$ , $B$ 推不出 $A$ ,则既不充分也不必要,如图:\begin{figure}[ht]
\centering
\includegraphics[width=12cm]{./figures/72797bcad5cfcfdc.png}
\caption{既不充分也不必要} \label{fig_SufCnd_4}
\end{figure}
\item $A$ 推不出 $B$ , $B$ 推出 $A$ ,则必要不充分,如图:\begin{figure}[ht]
\centering
\includegraphics[width=9cm]{./figures/b8c97ea69e5f63a9.png}
\caption{必要不充分} \label{fig_SufCnd_5}
\end{figure}
口诀:有之未必然,无之必不然
\end {enumerate}
