% KD-Tree
% keys 数据结构|计算机
% license Usr
% type Wiki

KD-Tree (K-Dimension Tree)是一种空间划分数据结构,顾名思义,可以对 $k$ 维空间进行划分。它常被用于高维空间中的搜索,比如范围搜索和最近邻搜索,时间复杂度由 $k$ 和总节点数 $n$ 共同保障。一般当 $n$ 远大于 $2^k$ 时,应用 KD-Tree 的效果是最好的。

\begin{definition}{二叉搜索树}
又称 BST(Binary Search Tree),是一种特殊的二叉树,对于二叉树上每个节点以及其上权值,满足:
\begin{enumerate}
\item 若二叉搜索树的左子树不为空,则其左子树上所有点的权值均小于其根节点的值。
\item 若二叉搜索树的右子树不为空,则其右子树上所有点的权值均大于其根节点的值。
\item 二叉搜索树的左右子树均为二叉搜索树。
\end{enumerate}
特别的,定义空树也为二叉搜索树。
\end{definition}

KD-Tree 具有\textbf{二叉搜索树}的形态,二叉搜索树上的每个结点都对应 $k$ 维空间内的一个点。其每个子树中的点都在一个 $k$ 维的超长方体内,这个超长方体内的所有点也都在这个子树中。

接下来我们考虑如何建树,也就是已经知道了这 $n$ 个 $k$ 维空间中的点的坐标,如何建立一颗对应的 KD-Tree:

\begin{itemize}
\item 考虑当前要分割的点集,这些点都在一个 $k$ 维超长方体内。若这超长方体内有且仅有一个点,返回这个点。
\item 若这个超长方体内有多于 $1$ 个点,考虑对这个超长方体进行分割:

\begin{enumerate}
\item 选择 $k$ 个维度中的一个。
    \item 选择一个分割点,这选择的这一维度上的值小于这个点的归入一个超长方体(左子树),其余的归入另一个超长方体(右子树)。
    \item 将选择的点作为这棵子树的根节点,递归建立左右子树并在过程中维护出需要的信息。

    
\end{itemize}