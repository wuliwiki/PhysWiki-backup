% 康托尔集(综述)
% license CCBYSA3
% type Wiki

本文根据 CC-BY-SA 协议转载翻译自维基百科\href{https://en.wikipedia.org/wiki/Cantor_set}{相关文章}。

在数学中,康托尔集是一个位于同一直线线段上的点集,它具有许多违反直觉的性质。该集合最早由亨利·约翰·斯蒂芬·史密斯于1874年发现\(^\text[1][2][3][4]\),并在1883年被德国数学家格奥尔格·康托尔提及\(^\text[5][6]\)。

通过对该集合的研究,康托尔及其他数学家为现代点集拓扑奠定了基础。最常见的构造是康托尔三分集,它通过不断从一条线段中去除中间三分之一,并对剩余的每一段重复该过程来构造。康托尔在其论文中仅顺带提到了这种三分构造,作为一个“完美但稠密度为零”的集合的例子\(^\text{[5]}\)。

更一般地,在拓扑学中,**康托尔空间**是指与康托尔三分集同胚的拓扑空间(配备其子空间拓扑)。康托尔集在自然意义上同胚于离散二点空间
**𝟐̲ᴺ**(即离散二元集合的可数笛卡尔积)。根据L.E.J.布劳威尔(L. E. J. Brouwer)的一个定理,这等价于以下五个条件的同时满足:完美(无孤立点)、非空、紧致、可度量且零维\(^\text[7]\)。


