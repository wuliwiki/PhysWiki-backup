% 伽利略变换
% keys 参考系|坐标系|变换|时间|时空|牛顿时空

\pentry{牛顿定律\upref{New3}, 线性变换\upref{LTrans}}

% 未完成, 还需解释与牛顿定律的关系
\begin{figure}[ht]
\centering
\includegraphics[width=7cm]{./figures/GaliTr_1.pdf}
\caption{伽利略变换示意图} \label{GaliTr_fig1}
\end{figure}

伽利略变换是描述一个事件所发生的时间和地点,随着惯性参考系的不同而变换的规律。
\begin{equation}
\begin{cases}
\bvec r_{K1} &= \bvec r_{r}+\bvec r_{K2}\\
t_{K2}&=t_{K1}
\end{cases}
\end{equation}
这样的两组坐标之间的变换,称为\textbf{伽利略变换}。它符合我们天生的直觉:两个参考系中同一个物体的长度仍然是一样的,时间独立于空间自由流动,所以事件发生的时间和参考系的选取无关。

正如Griffiths

更简单的,假设有两个一维的惯性参考系 $K_1$ 和 $K_2$,其中 $K_2$ 沿着 $K_1$ 的正方向以速率 $v$ 移动。如果一个事件在 $K_1$ 的视角下,是在 $t$ 时刻发生于 $x$ 位置的,那么它在 $K_2$ 视角下,是在 $t'$ 时刻发生于 $x'$ 位置的;如果我们知道了 $x$,$t$,就可以相应地计算出 $x'$ 和 $t'$ 来:
\begin{equation}
\begin{cases}
x' = x - vt\\
t' = t
\end{cases}
\end{equation}
如果把 $(x, t)$ 当作一个二维的向量空间,那么伽利略变换就是向量空间之间的一个线性变换。
