% 麦克斯韦方程组(介质)
% keys 麦克斯韦方程组|介质磁导率

\begin{issues}
\issueDraft
\end{issues}

\pentry{麦克斯韦方程组\upref{MWEq}}

\begin{equation}\ali{
&\div\bvec D = \rho_f\\
&\curl\bvec E = -\pdv{\bvec B}{t}\\
&\div\bvec B = 0\\
&\curl\bvec H = J_f + \pdv{\bvec D}{t}
}\end{equation}


在各向同性、非铁磁性介质中,有本构关系
\begin{equation}
\bvec D = \epsilon \bvec E = \epsilon_0 \epsilon_r \bvec E = \epsilon_0(1 + \chi_E)\bvec E
\end{equation}

\begin{equation}
\bvec H = \frac{\bvec B}{\mu} = \frac{\bvec B}{\mu_0\mu_r} = \frac{\bvec B}{\mu_0(1 + \chi_B)}
\end{equation}


其中电位移矢量 $\bvec D = \epsilon_0 \bvec E + \bvec P$,  磁场强度 $\bvec H = \frac{\bvec B}{\mu_0} - \bvec M$.

% 在各向同性线性介质中,有 $\bvec P = \chi_E \epsilon_0 \bvec E$,  $\bvec M = \chi_B \bvec H$.  代入上式得 $\bvec D = (1 + \chi_E)\epsilon_0\bvec E$ 和  $\bvec H = \frac{\bvec B}{(1 + \chi_B)\mu_0}$. 

% 定义相对介电常数为 $\epsilon_r = 1 + \chi_E$, 相对磁导率为 $\mu_r = 1 + \chi_B$, 则 $\bvec D = \epsilon_r\epsilon_0\bvec E$, $\bvec H = \frac{\bvec B}{\mu_r\mu_0}$,  
