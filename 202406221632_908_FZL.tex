% 反重力
% license CCBYSA3
% type Wiki

(本文根据 CC-BY-SA 协议转载自原搜狗科学百科对英文维基百科的翻译)

\begin{figure}[ht]
\centering
\includegraphics[width=10cm]{./figures/470f495ed810fee0.png}
\caption{反重力示意图} \label{fig_FZL_1}
\end{figure}
\textbf{反重力}(也称为非重力场)是一种关于创造不受重力影响的空间或物体的理论。它并不是指在自由落体或绕轨运动中重力作用下重量的不足,也不是指重力与其他力的平衡,例如电磁或气动升力。反重力是科幻小说中反复出现的概念,尤其是在航天器推进的情节中。例如H. G .威尔斯的《月球上的第一个人》中的重力阻挡物质“卡弗利特”,以及詹姆斯·布里斯的《飞行中的城市》中的自旋晕机器。

在牛顿万有引力定律中,重力是一种通过未知方式传递的外力。在20世纪,牛顿的模型被广义相对论所取代,在广义相对论中,重力不是一种力,而是时空几何的产物。在广义相对论下,除非在人为的情况下,反重力是不可能的。[1][2][3] 量子物理学家假设重力子的存在,重力子是传递重力的无质量基本粒子,但创造或摧毁重力子的可能性尚不清楚。

“反重力”经常被用来指看起来像是逆转重力的装置,即使它们是通过其他方式操作的,例如通过利用电磁场移动空气而在空气中飞行的升降机。[4][5]
\subsection{假设的解决方案}
\subsubsection{1.1 重力防护罩}
\begin{figure}[ht]
\centering
\includegraphics[width=8cm]{./figures/00bfd2e6a9f0d316.png}
\caption{巴布森学院纪念罗杰·巴布森的纪念碑,巴布森主要研究反重力和部分重力绝缘体。} \label{fig_FZL_2}
\end{figure}
1948年,商人罗杰·巴布森(巴布森学院的创始人)成立了重力研究基金会,致力于研究减轻重力影响的方法。[6]他们的努力最初有些“古怪”,但是他们偶尔召开的会议吸引了像以冷冻食品闻名的克拉伦斯·伯德赛和直升机发明者伊戈尔·西科尔斯基这样的人。随着时间的推移,基金会将注意力从试图控制重力转移到仅仅去更好地理解重力上。巴布森于1967年去世后,该基金会几乎消失了。然而,该基金会继续举办论文评比奖金高达4000美元。截至到2017年,该基金仍由原董事之子小乔治·赖德奥特在马萨诸塞州韦尔斯利管理。[7]获奖者包括加州天体物理学家乔治·F·斯穆特,他后来获得了2006年诺贝尔物理学奖。
\subsubsection{1.2 20世纪50年代的广义相对论研究}
广义相对论是在20世纪10年代年代引入的,但由于缺乏合适的数学工具,这一理论的发展大大放缓。根据广义相对论,反重力似乎是超出规律范围的。

据称,美国空军在整个20世纪50年代和60年代也进行了一项研究。[8]前中校安塞尔·塔尔伯特在报纸上写了两篇系列文章,声称大多数大型航空公司在20世纪50年代就开始了重力控制推进研究。然而,这些故事几乎没有被外界确认,而且由于它们实是在新闻发布时代的政策中发生的,因此不确定这些报道需要被给予多大的权重。

众所周知,洛德马丁司正在认真地努力着,该公司成立了高级研究所。[9][10] 主流纸公布了理论物理学家布尔夏德·海姆和洛德马丁公司之间的合同。私营部门为了了解重力而做出的另一项努力是由重力研究基金会理事格纽·H·巴赫森于1956年在北卡罗来纳大学建立的场物理研究所

1973年《曼斯菲尔德修正案》终止了对反重力项目的军事支持,该修正案将国防部的开支限制在具有明确军事用途的科学研究领域。曼斯菲尔德修正案的通过是专门为了结束长期运行的项目,这些项目几乎没有显示出它们的努力。

在广义相对论下,重力是由局部质量能量引起的遵循空间几何规律(空间正常形状的变化)的结果。这一理论认为,改变的空间形状是由大的物体变形引起的,重力实际上是变形空间的一种属性,而不是真正的力。尽管这些公式通常不能产生“负几何”,但可以通过使用“负质量”来实现。同样的公式本身并不排除负质量的存在。

广义相对论和牛顿引力似乎都预测到负质量会产生排斥的引力场。特别是赫尔曼·邦迪爵士(Sir Hermann Bondi)在1957年提出,负引力质量与负惯性质量相结合,将符合广义相对论的强等价原理以及线性动量和能量守恒的牛顿定律。邦迪的证明给出了相对论方程的无奇点解。[11]1988年7月,罗伯特· L·福特在AIAA/美国机械工程师协会/美国汽车工程师协会/ASEE第24届联合推进会议上提交了一篇论文,提出了一个邦迪负重力质量推进系统。[12]

邦迪指出,一个负质量将落向(而不是远离)“正常”物质,因为尽管引力是排斥性的,负质量(根据牛顿定律,F=ma)的反应是以与力相反的方向加速。另一方面,正常质量将会远离负物质。他指出,两个相同的质量,一个正的,一个负的,彼此靠近放置,因此会在它们之间的直线方向上自我加速,负的质量在追正的质量。[11]注意,因为负质量获得负动能,加速质量的总能量保持为零。进一步出,自加速效应是由负惯性质量引起的,并且可以在没有粒子间重力的情况下被观察到。[12]

所有描述目前已知物质形式的粒子物理标准模型不包括负质量。虽然宇宙暗物质可能由标准模型之外的粒子组成,其性质未知,但它们的质量表面上是已知的——因为它们是根据对周围物体的引力效应假设的,这意味着它们的质量是正的。另一方面,提出的宇宙学暗能量更为复杂,因为根据广义相对论,它的能量密度和负压的影响都有助于它的引力效应。
\subsubsection{1.3 第五势力}
在广义相对论下,任何形式的能量都与时空耦合,以产生引起重力的几何形状。一个长期存在的问题是这些相同的方程是否适用于反物质。随着CPT对称性的发展,这个问题在1960年得到了解决,这表明反物质遵循与“正常”物质相同的物理定律,因此具有正能量含量,并且像正常物质一样引起(并对其作出反应)重力。

在20世纪最后25年的大部分时间里,物理学界参与了产生统一场论尝试,这是一种解释四种基本力的单一物理理论: 重力、电磁力以及强核力和弱核力。科学家们在统一三种量子力方面取得了进展,但是重力仍然是每次尝试中的“问题”。然而,这并未阻止此类尝试的数量的减少。

一般来说,这些尝试试图通过定位一个粒子(引力子)来“量化重力”而实现的,引力子以光子(光)携带电磁的方式携带重力。然而,沿着这个方向的简单尝试都失败了,导致产生了更复杂的试图解释这些问题的例子。其中的两个,超对称性和相对论相关的超重力,都需要重力光子携带的极其微弱的“第五力”的存在,它以有组织的方式将量子场论中的几个“松散端”耦合在一起。其中的一个副作用就是这两种理论几乎都要求反物质以类似于反重力的方式受到第五种力的影响,这意味着排斥远离质量。20世纪90年代进行了几次实验来测量这种效应,但没有一次产生积极的结果。[13]

2013年,欧洲核子研究中心在一项旨在研究反氢能量水平的实验中寻找了反重力效应。反重力测量只是一个“有趣的串演节目”,并没有结论。[14]
\subsubsection{1.4 广义相对论“翘曲驱动”}
广义相对论的场方程的解描述了“翘曲驱动”(如阿尔库比尔度量)和稳定的可穿越的虫洞这本身并不重要,因为任何时空几何都是应力-能量张量场的某些构型的场方程的解(见广义相对论中的精确解)。除非对应力-能量张量施加外部约束,否则广义相对论不仅会限制时空的几何形状。翘曲驱动和可穿越虫洞几何形状在大多数区域表现良好,但需要外来物质区域的辅助;因此,如果应力-能量张量仅限于已知的物质形式,它们就被排除在解之外。目前对暗物质和暗能量的理解还不足以对它们在翘曲驱动中的适用性做出一般性的陈述。
\subsubsection{1.5 突破性推进物理计划}
二十世纪末,美国航天局从1996年到2002年为突破性推进物理计划(BPP)提供资金支持。该项目研究了许多“遥远”的空间推进设计,这些设计都没有通过师范大学或商业渠道获得资助。反重力类概念是以“直径驱动”的名义进行研究的。BPP计划的工作在独立的、非美国宇航局附属的陶零基金会支持下继续进行。[15]
\subsection{经验性主张和商业努力}
至今,已经有许多人尝试建造反重力装置,科学文献中也有少量的关于反重力效应的报道。以下所有示例都不被认为是反重力领域的可重复示例。