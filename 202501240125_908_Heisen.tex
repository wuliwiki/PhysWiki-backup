% 维尔纳·海森堡(综述)
% license CCBYSA3
% type Wiki

本文根据 CC-BY-SA 协议转载翻译自维基百科\href{https://en.wikipedia.org/wiki/Werner_Heisenberg}{相关文章}。

\textbf{维尔纳·卡尔·海森堡}(Werner Karl Heisenberg,/ˈhaɪzənbɜːrɡ/;[2] 德语:[ˈvɛʁnɐ ˈhaɪzn̩bɛʁk] ⓘ;1901年12月5日-1976年2月1日)[3] 是德国理论物理学家,量子力学理论的主要奠基人之一,也是二战期间纳粹核武器计划中的核心科学家。他于1925年发表了他的《重新诠释》论文,对旧量子理论进行了重要的重新解释。同年,他与马克斯·玻恩和帕斯库尔·约尔丹合写了一系列论文,详细阐述了他的矩阵量子力学表述。他以1927年发表的不确定性原理而闻名。海森堡因“创立量子力学”而获得1932年诺贝尔物理学奖。[4][a]  

海森堡还对湍流流体动力学、原子核、铁磁性、宇宙射线和亚原子粒子的理论作出了贡献。他在1957年协助规划了卡尔斯鲁厄的第一个西德核反应堆,以及慕尼黑的一个研究反应堆。  

二战后,他被任命为\textbf{威廉皇帝物理研究所}(Kaiser Wilhelm Institute for Physics)所长,不久后该研究所更名为马\textbf{克斯·普朗克物理研究所}。他一直担任所长,直到研究所于1958年迁至慕尼黑。1960年至1970年间,他担任\textbf{马克斯·普朗克物理与天体物理研究所}所长。  

此外,海森堡还担任\textbf{德国研究委员会}主席、\textbf{原子物理委员会}\textbf{主席}、\textbf{核物理工作组}主席以及\textbf{洪堡基金会}主席。[1]  

