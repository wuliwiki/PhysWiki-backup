% 格奥尔格·康托尔(综述)
% license CCBYSA3
% type Wiki

本文根据 CC-BY-SA 协议转载翻译自维基百科\href{https://en.wikipedia.org/wiki/Georg_\%E5\%BA\%B7\%E6\%89\%98\%E5\%B0\%94}{相关文章}。

\begin{figure}[ht]
\centering
\includegraphics[width=6cm]{./figures/409c043d8a08ce28.png}
\caption{} \label{fig_Canto_1}
\end{figure}
乔治·费迪南德·路德维希·菲利普·康托尔(Georg Ferdinand Ludwig Philipp 康托尔,/ˈkæntɔːr/ KAN-tor;德语发音:[ˈɡeːɔʁk ˈfɛʁdinant ˈluːtvɪç ˈfiːlɪp ˈkantoːɐ̯];1845年3月3日(旧历2月19日)-1918年1月6日)是一位数学家,他在集合论的创建中发挥了关键作用,集合论已成为数学中的一项基本理论。康托尔确立了两个集合成员之间一对一对应的重要性,定义了无限集合和良序集合,并证明了实数比自然数更多。康托尔证明该定理的方法意味着存在无数个不同大小的无限集合。他定义了基数和序数以及它们的算术运算。康托尔的工作在哲学上具有重大意义,他对此非常清楚。

最初,康托尔的超无限数理论被认为是反直觉的——甚至是震惊的。这使得它遭遇了数学界 contemporaries 的抵制,如利奥波德·克罗内克和亨利·庞加莱[3],以及后来的赫尔曼·外尔和L·E·J·布劳威尔,而路德维希·维特根斯坦则提出了哲学上的反对意见;参见康托尔理论的争议。康托尔是一个虔诚的路德宗基督徒[4],他认为这一理论是上帝传达给他的[5]。一些基督教神学家(尤其是新经院哲学家)认为康托尔的工作挑战了上帝本质中绝对无限的独特性[6]——曾有一次将超无限数理论与泛神论等同起来[7]——这一命题被康托尔坚决拒绝。然而,并非所有神学家都反对康托尔的理论;著名的新经院哲学家康斯坦丁·古特贝尔特支持这一理论,而枢机主教约翰·巴普蒂斯特·弗朗泽林在康托尔做出一些重要澄清后也接受了这一理论作为有效理论[8]。

对康托尔工作的反对有时非常激烈:利奥波德·克罗内克的公开反对和个人攻击包括将康托尔描述为“科学江湖医生”、“叛徒”和“青年堕落者”[9]。克罗内克反对康托尔证明代数数是可数的,以及超越数是不可数的,这些结果如今已被纳入标准数学课程中。维特根斯坦在康托尔去世几十年后写道,他感叹数学“完全被集合论的有害习语所支配”,并将其斥为“完全的胡说八道”,“可笑”且“错误”[10]。从1884年到他生命的尽头,康托尔反复遭遇抑郁症,这被归咎于许多同时代人对他的敌对态度[11],尽管也有人将这些症状解释为双相情感障碍的可能表现[12]。

激烈的批评与后来的赞誉相匹配。1904年,皇家学会授予康托尔西尔维斯特奖,这是它能授予数学工作者的最高荣誉[13]。大卫·希尔伯特为其辩护,宣称:“没有人能够将我们从康托尔创造的乐园中驱逐出去”[14][15]。
\subsection{传记}  
\subsubsection{青年时期与学业}
\begin{figure}[ht]
\centering
\includegraphics[width=6cm]{./figures/b8db46daef6392b0.png}
\caption{康托尔,约1870年} \label{fig_Canto_2}
\end{figure}
乔治·康托尔,1845年出生于俄罗斯帝国圣彼得堡,直到十一岁时一直在该市成长。他是六个孩子中的长子,被认为是一位出色的小提琴手。他的祖父弗朗茨·伯姆(Franz Böhm,1788–1846)(小提琴家约瑟夫·伯姆的兄弟)是俄罗斯帝国乐团著名的音乐家和独奏家[16]。康托尔的父亲曾是圣彼得堡证券交易所的一名成员;当他生病时,家人于1856年搬到德国,首先定居在威斯巴登,随后又迁往法兰克福,寻求比圣彼得堡温和的冬季。1860年,康托尔以优异的成绩从达姆施塔特的实科学校毕业;他的数学天赋,尤其是在三角学方面,得到了认可。1862年8月,他从达姆施塔特的“高等职业学校”毕业,现为达姆施塔特工业大学[17][18]。同年,他进入瑞士联邦理工学院苏黎世校区学习。1863年6月,父亲去世后,康托尔获得了一笔可观的遗产,他转学到柏林大学,听取了利奥波德·克罗内克、卡尔·魏尔斯特拉斯和恩斯特·库默的讲座。他于1866年夏季在哥廷根大学度过,那时和后来哥廷根是数学研究的中心。康托尔是一位优秀的学生,并于1867年获得博士学位[19][20]。
\subsubsection{教师与研究员}  
康托尔于1867年在柏林大学提交了他的关于数论的博士论文。在柏林的一所女子学校短暂任教后,他在哈雷大学获得了职位,并在那里度过了他的一生职业生涯。他凭借关于数论的论文获得了必要的资格认证,并于1869年在哈雷大学任职时提交了该论文[20][21]。

1874年,康托尔与瓦莉·古特曼结婚,他们共有六个孩子,最小的一个(鲁道夫)出生于1886年。尽管康托尔的学术工资 modest,他依靠父亲的遗产得以养活家庭。在蜜月期间,康托尔与理查德·德德金德进行了许多数学讨论,他们在两年前在瑞士因度假而相识,当时康托尔在哈茨山脉度假[22]。

康托尔于1872年晋升为特聘教授,并于1879年成为正教授[20][19]。在34岁时获得这一职位是一个值得注意的成就,但康托尔渴望能够在更有声望的大学担任教职,尤其是在当时德国领先的大学——柏林大学。然而,他的工作遭遇了过多的反对,使得这一目标难以实现[23]。克罗内克是柏林大学数学系的负责人,直到1891年去世,他对康托尔作为同事的前景感到越来越不安[24],认为他是“堕落青年”的人,因为康托尔将自己的观点教授给年轻一代数学家[25]。更糟糕的是,克罗内克,这位在数学界已经确立地位并曾是康托尔的教授,自从故意推迟康托尔的第一篇重要论文于1874年发表以来,就从根本上不同意康托尔工作的核心内容[20]。克罗内克,现被视为数学中构造性观点的奠基人之一,不喜欢康托尔的大部分集合论,因为它宣称存在满足特定性质的集合,却没有提供具体的集合实例,证明其成员确实满足这些性质。每当康托尔申请柏林的职位时,他总是被拒绝,而这一过程通常涉及克罗内克[20],因此康托尔开始相信,克罗内克的立场将使他永远无法离开哈雷[26]。

1881年,康托尔在哈雷的同事爱德华·海内去世。哈雷大学接受了康托尔的建议,提议将海内留下的职位依次提供给德德金德、海因里希·M·韦伯和弗朗茨·梅尔滕斯,但每位教授在被邀请后都拒绝了这个职位。最终,弗里德里希·万格林被任命,但他与康托尔并未建立亲近关系[27]。

1882年,康托尔与德德金德的数学通信结束,显然是因为德德金德拒绝了哈雷大学的职位[28]。康托尔还开始了另一重要的通信,与瑞典的Gösta Mittag-Leffler,并很快开始在Mittag-Leffler的期刊《数学年刊》(Acta Mathematica)上发表文章。但在1885年,Mittag-Leffler对康托尔提交给《数学年刊》的论文中的哲学性质和新术语感到担忧[29]。他要求康托尔在论文校样阶段撤回该论文,并写道:“……大约晚了一百年。”康托尔同意了,但随后缩短了与Mittag-Leffler的关系和通信,他写信给第三方说:“如果Mittag-Leffler有他的意愿,我就得等到1984年,而对我来说那简直是个过大的要求!……不过当然,我再也不想知道关于《数学年刊》的任何事情了”[30]。

康托尔在1884年5月首次出现已知的抑郁症发作[19][31]。对他工作的批评让他感到沉重:他在1884年写给Mittag-Leffler的五十二封信中,每一封都提到了克罗内克。其中一封信中的一段话揭示了康托尔自信心受损的情况:

“…我不知道何时才能继续我的科学工作。目前我根本做不下去,只能做最基本的讲座职责;如果我有足够的心力,我多么希望能够继续科学上的工作,哪怕只是活跃于此。[32]”

这次危机促使他申请讲授哲学,而不是数学。他还开始深入研究伊丽莎白时代的文学,认为可能有证据表明弗朗西斯·培根写了那些被归于威廉·莎士比亚的戏剧(见莎士比亚作者问题);最终,这导致了两本小册子的出版,分别是在1896年和1897年[33]。

康托尔很快恢复了,随后做出了更多重要的贡献,包括他的对角线论证和定理。然而,尽管克罗内克于1891年12月29日去世[20],他再也未能达到1874年至1884年期间那些杰出论文的高度。他最终寻求并达成了与克罗内克的和解。然而,他们之间的哲学分歧和困难依旧存在。

1889年,康托尔在成立德国数学学会方面发挥了重要作用[20],并在1891年主持了该学会的第一次会议,会议上他首次介绍了他的对角线论证;尽管克罗内克反对他的工作,康托尔的声誉足够强大,以确保他当选为该学会的首任会长。抛开克罗内克对他的敌意,康托尔邀请他在会议上发言,但克罗内克由于当时妻子在滑雪事故中受伤致死,未能如愿。乔治·康托尔还在1897年推动了第一届国际数学家大会的成立,该大会在瑞士苏黎世举行[20]。
\subsubsection{晚年与去世}  
在康托尔1884年住院后,直到1899年没有记录显示他再次住进任何疗养院。[31] 在那次第二次住院后,康托尔的小儿子鲁道夫于12月16日突然去世(康托尔当时正在讲授关于培根理论与莎士比亚的演讲),这一悲剧使康托尔失去了大部分对数学的热情。[34] 康托尔在1903年再次住院。一年后,他对尤利乌斯·科尼希在第三届国际数学家大会上所发表的论文感到愤怒和激动。该论文试图证明超限集合理论的基本原则是错误的。由于论文是在他的女儿和同事面前宣读的,康托尔感到自己遭受了公开的羞辱。[35] 尽管恩斯特·泽梅洛在不到一天后证明科尼希的证明失败了,康托尔依然感到震惊,甚至短暂地对上帝产生了疑问。[13] 康托尔在余生中一直患有慢性抑郁症,因此他在几次教学中被豁免,并且多次被送往各个疗养院。1904年的事件导致了他之后每隔两三年就住院一次。[36] 然而,他并未完全放弃数学,在1903年,他向德国数学家协会会议讲授集合论悖论(布拉利-福尔蒂悖论、康托尔悖论和拉塞尔悖论),并于1904年参加了海德堡的国际数学家大会。

1911年,康托尔成为受邀参加苏格兰圣安德鲁斯大学成立500周年庆典的杰出外国学者之一。康托尔参加了此次庆典, hoping to meet Bertrand Russell, whose newly published *Principia Mathematica* repeatedly cited 康托尔's work, but the encounter did not come about. 第二年,圣安德鲁斯大学授予康托尔荣誉博士学位,但由于康托尔的身体状况,他未能亲自接受该学位。

康托尔于1913年退休,并在第一次世界大战期间生活贫困,营养不良。[37] 由于战争,他的70岁生日庆祝活动被取消。1917年6月,康托尔最后一次进入疗养院,并不断给妻子写信,要求允许他回家。乔治·康托尔于1918年1月6日在他度过生命最后一年的疗养院中因心脏病发作去世。[19]
\subsection{数学工作}  
康托尔在1874年到1884年之间的工作是集合论的起源。[38] 在这之前,集合的概念是一个相当初步的概念,从数学的起源开始就被隐含地使用,追溯到亚里士多德的思想。没有人意识到集合论有任何非平凡的内容。在康托尔之前,只有有限集合(这些集合容易理解)和“无限”概念(被视为哲学讨论的主题,而非数学讨论的主题)。通过证明无限集合有(无限多种)不同的大小,康托尔确立了集合论并非平凡,且需要进行研究。集合论已经成为现代数学的基础理论之一,意味着它通过一个统一的理论来解释关于数学对象(例如数和函数)的命题,涵盖所有传统的数学领域(如代数、分析学和拓扑学),并提供了一套标准的公理来证明或反驳这些命题。集合论的基本概念现在已广泛应用于整个数学领域。[39]

在他的一篇早期论文中,[40] 康托尔证明了实数集合比自然数集合“更多”——这是首次证明存在不同大小的无限集合。他也是第一个意识到一一对应(以下简称“1对1对应”)在集合论中重要性的人。他利用这个概念定义了有限集和无限集,并将后者细分为可列(或可数无限)集合和不可列集合(不可数无限集合)。[41]

康托尔在拓扑学中发展了重要的概念,并探讨了这些概念与基数的关系。例如,他证明了康托尔集(由亨利·约翰·斯蒂芬·史密斯于1875年发现)在任何地方都不稠密,但其基数与实数集合相同,而有理数则处处稠密,但却是可数的。他还证明了所有没有端点的可数稠密线性序列都与有理数是顺序同构的。

康托尔在集合论中引入了基础的构造方法,例如集合A的幂集,它是A的所有可能子集的集合。他后来证明了幂集A的大小严格大于A的大小,即使A是一个无限集合;这个结果很快被称为康托尔定理。康托尔发展了一个完整的无限集合的理论和算术,称为基数和序数,它扩展了自然数的算术。他为基数所用的符号是希伯来字母ℵ(ℵ,aleph),并附有自然数下标;而序数则使用希腊字母ω(ω,omega)。这种符号至今仍在使用。

由康托尔提出的连续统假设在1900年巴黎国际数学大会上被大卫·希尔伯特作为他二十三个未解问题中的第一个问题提出。康托尔的工作也受到了希尔伯特著名赞美之外的广泛关注。[15] 美国哲学家查尔斯·桑德斯·皮尔士赞扬了康托尔的集合论,并且在康托尔于1897年在苏黎世举行的第一次国际数学大会上发表公开讲座后,阿道夫·赫尔维茨和雅克·阿达马也都表达了他们的钦佩。在那次大会上,康托尔与德德金德重拾了友谊和通信。从1905年起,康托尔开始与他的英国崇拜者和翻译菲利普·乔丹就集合论的历史以及康托尔的宗教思想进行通信。这些信件后来被出版,还有一些他的阐述性作品也相继出版。
\subsubsection{数论、三角级数与序数}  
康托尔的前十篇论文涉及数论,这是他的博士论文主题。在哈雷大学教授爱德华·海涅的建议下,康托尔转向了分析学。海涅建议康托尔解决一个未解的问题,这个问题曾让彼得·古斯塔夫·勒让·狄利克雷、鲁道夫·利普希茨、伯恩哈德·黎曼和海涅本人都无法解决:通过三角级数表示函数的唯一性。康托尔于1869年解决了这个问题。正是在研究这个问题时,他发现了超有限序数,这些序数作为指数\(n\)出现在三角级数零点集\(S\)的第\(n\)阶导集\(S_n\)中。给定一个三角级数\(f(x)\),其零点集为\(S\),康托尔发现了一种方法可以得到另一个三角级数,其零点集为\(S_1\),其中\(S_1\)是\(S\)的极限点集。如果\(S_{k+1}\)是Sk的极限点集,那么他可以构造一个三角级数,其零点为\(S_{k+1}\)。因为这些集合\(S_k\)是封闭的,所以它们包含了自己的极限点,而无限递减序列集合\(S\)、\(S_1\)、\(S_2\)、\(S_3\),…的交集形成了一个极限集,我们现在称之为\(S_\omega\),随后他注意到\(S_\omega\)也必然有一个极限点集\(S_{\omega+1}\),依此类推。他得到了一些可以无限延续的例子,因此出现了一个自然发生的无限数列:\(\omega\), \(\omega+1\),\(\omega+2\), …[43]

在1870年至1872年间,康托尔发表了更多关于三角级数的论文,还发表了一篇定义无理数为收敛有理数序列的论文。康托尔在1872年与德德金德建立了友谊,德德金德后来在同年引用了康托尔的这篇论文,并在他的论文中首次提出了著名的实数定义——德德金分割。尽管通过其革命性的无限基数概念扩展了数的概念,康托尔却出人意料地反对他同时代的奥托·斯托尔茨和保罗·杜·布瓦-雷蒙的无穷小理论,称它们既是“可憎的”,又是“数学中的霍乱菌”。[44]康托尔还发表了一个错误的“证明”,声称无穷小的内在矛盾。[45]
\subsubsection{集合论}
\begin{figure}[ht]
\centering
\includegraphics[width=6cm]{./figures/65e841d34d6944de.png}
\caption{这是康托尔对不可数集合存在性进行对角线论证的示意图。[46] 底部的序列无法出现在上面无限序列的任何位置。} \label{fig_Canto_3}
\end{figure}
集合论作为数学分支的开端通常被标志为康托尔于1874年发表的论文《Ueber eine Eigenschaft des Inbegriffes aller reellen algebraischen Zahlen》(《关于所有实代数数集合的一个性质》)的发表。[38][47] 这篇论文首次提供了一个严格的证明,证明了存在不止一种类型的无穷大。在此之前,所有的无穷集合都被隐含地认为是等势的(即“大小相同”或包含相同数量的元素)。[48] 康托尔证明了实数集合和正整数集合不是等势的。换句话说,实数集合是不可数的。他的证明不同于他在1891年给出的对角线论证。[49] 康托尔的文章还包含了一种构造超越数的新方法。超越数最早是由Joseph Liouville于1844年构造的。[50]

康托尔通过两种构造法建立了这些结果。他的第一种构造法展示了如何将实代数数[51]写成一个序列\(a_1\), \(a_2\), \(a_3\), … 换句话说,实代数数是可数的。康托尔的第二种构造法从任何实数序列开始。利用这个序列,他构造了嵌套区间,这些区间的交集包含一个不在序列中的实数。由于每个实数序列都可以用来构造一个不在该序列中的实数,因此实数不能被写成一个序列——即,实数是不可数的。通过将他的构造法应用于实代数数序列,康托尔得到了一个超越数。康托尔指出,他的构造法证明了更多内容——即,它们为Liouville定理提供了一个新证明:每个区间都包含无穷多个超越数。[52] 康托尔的下一篇文章包含了一种构造法,证明了超越数集合的“势”(见下文)与实数集合相同。[53]

在1879年到1884年之间,康托尔在《Mathematische Annalen》期刊上发表了一系列六篇文章,这些文章共同构成了他集合论的介绍。同时,针对康托尔的观点,反对声音逐渐增多,主要由Leopold Kronecker领导。Kronecker只接受那些可以通过有限步骤从自然数构造出来的数学概念,而他认为自然数是直观上给定的。对于Kronecker来说,康托尔的无限阶层是不可接受的,因为接受实际无穷大的概念将打开通向悖论的大门,挑战数学整体的有效性。[54] 康托尔在这一时期还引入了康托尔集。

这一系列的第五篇论文\textbf{《Grundlagen einer allgemeinen Mannigfaltigkeitslehre》}(《一般集合理论的基础》),发表于1883年,[55] 是六篇论文中最重要的一篇,并且也作为独立的专著出版。该论文包含了康托尔对批评者的回应,并展示了超有限数是如何系统地扩展自然数的。论文首先定义了良序集合。然后引入了序数作为良序集合的顺序类型。接下来,康托尔定义了基数和序数的加法和乘法。1885年,康托尔扩展了他的顺序类型理论,使得序数成为顺序类型的一个特例。

1891年,他发表了一篇论文,提出了优雅的“对角线论证”,证明了不可数集合的存在。他应用相同的思想证明了康托尔定理:集合\(A\)的幂集的基数严格大于\(A\)的基数。这一结果确立了无限集合的层次结构,以及康托尔所定义的基数和序数的算术。这个论证在解决停机问题和证明Gödel第一不完全性定理中具有基础性作用。康托尔在1894年还曾写过关于哥德巴赫猜想的论文。
\begin{figure}[ht]
\centering
\includegraphics[width=6cm]{./figures/a0b50c3cf3b9f84d.png}
\caption{乔治·康托尔文章中的定义集合的段落} \label{fig_Canto_4}
\end{figure}
在1895年和1897年,康托尔在Felix Klein编辑的《Mathematische Annalen》上发表了两篇论文,这是他关于集合理论的最后几篇重要论文。[56] 第一篇论文从定义集合、子集等开始,这些定义在今天基本上仍然是可以接受的。论文回顾了基数和序数的算术。康托尔原本希望第二篇论文包括连续统假设的证明,但最终只能阐述他对良序集合和序数的理论。康托尔试图证明,如果集合\(A\)和\(B\)满足\(A\)与\(B\)的一个子集等价,\(B\)与\(A\)的一个子集等价,那么\(A\)和\(B\)是等价的。恩斯特·施罗德早些时候就提出了这个定理,但他的证明以及康托尔的证明都存在缺陷。费利克斯·伯恩斯坦在1898年的博士论文中提供了正确的证明,因此这个定理被称为康托尔-伯恩斯坦-施罗德定理。

\textbf{一一对应}

\begin{figure}[ht]
\centering
\includegraphics[width=6cm]{./figures/83a671629215d8a8.png}
\caption{双射函数} \label{fig_Canto_5}
\end{figure}
康托尔的1874年《Crelle》论文首次提出了“一一对应”这一概念,尽管他并没有使用这个词语。随后,他开始寻找单位正方形的点与单位线段的点之间的一一对应关系。在1877年写给理查德·德德金的一封信中,康托尔证明了一个更强的结果:对于任何正整数\(n\),存在单位线段与n维空间中所有点之间的一一对应关系。关于这个发现,康托尔写信给德德金说:“Je le vois, mais je ne le crois pas!”(“我看到了,但我不相信!”)这个结果令他感到非常惊讶,它对几何学和维度的概念产生了深远的影响。

在1878年,康托尔向《Crelle's Journal》提交了另一篇论文,他在其中精确定义了一一对应的概念,并引入了“幂”(他从Jakob Steiner那里借用的术语)或“等价”集合的概念:如果存在一一对应关系,两个集合是等价的(有相同的幂)。康托尔将可数集合(或可列集合)定义为能够与自然数建立一一对应关系的集合,并证明有理数是可列的。他还证明了n维欧几里得空间\(R^n\)与实数\textbf{R}具有相同的幂,任何可列无限的\textbf{R}的副本的乘积也具有相同的幂。尽管他在概念上频繁使用了可数性,但直到1883年才正式写出“可数”一词。康托尔还讨论了自己对维度的思考,强调他在单位区间和单位正方形之间的映射并不是连续的。

这篇论文让克罗内克尔不满,康托尔曾想撤回它;然而,德德金劝说他不要这样做,卡尔·魏尔斯特拉斯也支持它的发表。尽管如此,康托尔从此再也没有向《Crelle》提交任何论文。