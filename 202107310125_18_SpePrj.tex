% 谱投影

\pentry{有界算子的预解式\upref{BddRsv}}

\begin{definition}{谱投影}
设$X$是复巴拿赫空间, $T:X\to X$是有界算子. 设有一条简单闭道路$\gamma$将谱集$\sigma(T)$分成了不相交的两部分, 将包含在$\gamma$内部的部分记为$\Lambda$. 则算子
$$
P_\Lambda:=\frac{1}{2\pi i}\int_\gamma(z-T)^{-1}dz
$$
称为$T$在$\Lambda$上的谱投影.
\end{definition}

为何要像这样定义谱投影? 原来, 这其实是在推广线性代数中将空间分解为矩阵的不变子空间的操作. 对于矩阵的情形, 参见词条例: 有限维方阵\upref{SpeMat}, 在那里谱投影的意义可以通过直接计算看出. 在一般的巴拿赫空间的情形, 我们首先有如下命题:

\begin{lemma}{}
如上定义的算子$P_\Lambda$的确是有界的投影算子, 即满足
$$
P_\Lambda^2=P_\Lambda.
$$
另外, $P_\Lambda$与$T$可交换.
\end{lemma}