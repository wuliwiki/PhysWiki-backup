% 静电场(高中)
% 电荷|电场|库仑定律

\subsection{电荷}

经摩擦的物体能吸引轻小的物体,我们就说它带有\textbf{电荷}.电荷分为\textbf{正电荷}和\textbf{负电荷},用毛皮摩擦过的橡胶棒带负电荷,用丝绸摩擦过的玻璃棒带正电荷.

电荷间的相互作用规律:同种电荷相互排斥,异种电荷相互吸引.

\subsubsection{电荷量}

电荷量表示电荷的多少.电荷量的单位是\textbf{库仑}(简称\textbf{库}),符号为$\mathrm{C}$.

正电荷的电荷量为正值($+$号通常会省略),负电荷的电荷量为负值,要注意这里的正、负表示电荷种类,比较电荷量的多少要看电荷量的绝对值.

\subsubsection{元电荷}

又称\textbf{基本电荷},是一个电子或一个质子所带电荷量的绝对值,用字母$e$表示,$e = 1.602176634 \times 10^{-19} \mathrm{C}$\footnote{详见“物理学常数\upref{Consts}”},近似计算时取$e = 1.60 \times 10^{-19} \mathrm{C}$.任何带电体所带的电荷量都是元电荷的整数倍.

\subsubsection{点电荷}

带电体之间的距离比带电体自身大小大得多时,带电体自身的大小、形状、电荷分布对带电体之间作用力的影响可忽略不计,这时可把带电体看成带电的点,称为\textbf{点电荷}.点电荷是一种理想化模型.

\subsubsection{比荷}

带电体所带电量与其质量之比,叫做\textbf{比荷},单位为库仑每千克($\mathrm{C/kg}$).

电子的质量$m_e \approx 9.11 \times 10^{-31} \mathrm{kg}$,则电子的比荷
\begin{equation}
\frac{e}{m_e} \approx 1.76 \times 10^{-11} \mathrm{C/kg}
\end{equation}

\subsection{起电方式}

\subsubsection{摩擦起电}

用摩擦的方法使两个由不同物质组成的物体带电的现象,叫做\textbf{摩擦起电}.摩擦起电的实质是电子的转移,在两物体相互摩擦时,电子从束缚电子能力弱的物体转移到束缚电子能力强的物体,摩擦后两物体带等量异种电荷.

\subsubsection{感应起电}

一个带电的物体和一个不带电的导体相互靠近时,由于电荷之间的相互作用,导体靠近带电体的一端带异种电荷,远离带电体的一端则带同种电荷,这种现象叫做\textbf{静电感应}.静电感应的实质是导体内部电子的转移.利用静电感应时物体带电的方法叫做\textbf{感应起电}.

\subsubsection{接触起电}

让不带电的物体与带电体相互接触,使不带电的物体带上与带电体同种的电荷,这种起电方式叫做\textbf{接触起电}.接触起电的实质是电子在两接触物体之间的转移.
