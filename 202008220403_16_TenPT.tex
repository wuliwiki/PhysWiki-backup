% 张量扰动
对于在视界内部的模式来说,张量扰动对应了在FRW背景下传播的引力波.

\subsection{宇宙演化}
在共形牛顿规范下,我们只保留$h_{ij}^{TT}$,可得
\begin{equation}
ds^2 = a^2[-d  \eta^2+(\delta_{ij}+h_{ij}^{TT})dx^i dx^j ]~.
\end{equation}
对于爱因斯坦张量,我们有$\delta G^0_0 = 0, \delta G^i_0 = 0$以及
\begin{equation}
\delta G^i_j = \frac{1}{2 a^2} [ (h_{ij}^{TT}  )'' + 2\mathcal H (h_{ij}^{TT})' - \nabla^2 h_{ij}^{TT}  ]~.
\end{equation}
于是,扰动的爱因斯坦方程可以写成如下形式
\begin{equation}\label{TenPT_eq1}
(h_{ij}^{TT})'' + 2 \mathcal H (h_{ij}^{TT})' - \nabla^2 h_{ij}^{TT} = 16 \pi G a^2 \sigma_{ij}^{TT} ~.
\end{equation}
我们可以换到动量空间然后以极化张量为基进行如下展开
\begin{equation}
\tilde h_{ij}^{TT} (\eta,\mathbf k) = \sum_{A = +,\times} e^A_{ij} (\hat{\mathbf k}) \tilde h_A (\eta,\mathbf k)~, 
\end{equation}
类似地,我们有
\begin{equation}
\tilde \sigma_{ij}^{TT} (\eta,\mathbf k) = \sum_{A = +,\times} e^A_{ij} (\hat{\mathbf k}) \tilde \sigma_A (\eta,\mathbf k) ~.
\end{equation}
极化张量的定义如下
\begin{equation}
\begin{aligned}
e^+_{ij}(\hat{\mathbf k}) & = \hat{\mathbf u}_i \hat{\mathbf u}_j - \hat{\mathbf v}_i \hat{\mathbf v}_j ~, \\
e^\times_{ij} (\hat{\mathbf k}) & = \hat{\mathbf u}_i 
\hat{\mathbf v}_j + \hat{\mathbf v}_i \hat{\mathbf u}_j~,
\end{aligned}
\end{equation}
极化张量按照下式归一化
\begin{equation}
e^A_{ij} (\hat{\mathbf k}) e^{A'}_{ij} (\hat{\mathbf k}) = 2 \delta^{AA'} ~.
\end{equation}
如果$\hat{\mathbf k}$是沿着$\hat z$方向的,我们可以选取$\hat{\mathbf u} = \hat{\mathbf x}$以及$\hat{\mathbf v} = \hat{\mathbf y}$于是我们有
\begin{equation}
e^+_{ab} = \begin{pmatrix}
1 & 0 \\
0 & -1 
\end{pmatrix} \quad 
e^\times_{ab} = \begin{pmatrix}
0 & 1 \\
1 & 0
\end{pmatrix}
~.
\end{equation}
于是\autoref{TenPT_eq1} 变成了两个关于$h_A(\eta,k)$的独立方程
\begin{equation}\label{TenPT_eq2}
\tilde h''_A + 2 \mathcal H \tilde h'_A + k^2\tilde h_A = 16 \pi G a^2 \tilde \sigma_A ~.
\end{equation}
因此
\begin{equation}
\tilde h_A(\eta,\mathbf k) \propto \frac{1}{a(\eta)} \sin(k\eta+\alpha)  \quad (k\eta \ll 1) ~.
\end{equation}
假如
\begin{equation}
h_A(\eta,\mathbf k) \propto \frac{1}{a(\eta)} \sin(k\eta+\alpha)   ~.
\end{equation}
我们就有
\begin{equation}
h'_A(\eta,\mathbf k) \propto \frac{k \cos(k\eta+\alpha)}{a(\eta)} + O \bigg( \frac{1}{a^2} \bigg) ~,
\end{equation}
于是
\begin{equation}
\dot h_A (\eta,\mathbf k) \propto \frac{k\cos(k\eta+\alpha)}{a^2 (\eta)} + O \bigg( \frac{1}{a^3} \bigg)~.
\end{equation}
因为引力波的强度正比于$\sum_A\langle \dot h_A^2 \rangle$. 所以我们得出了$\rho_{\rm gw} \propto a^{-4}$. 需要注意的是,只有张量在视界内的时候,它们才能够被引力子描述,也就是一些能量和动量都是能被很好定义的,随着宇宙的膨胀,能量密度按照$1/a^4$变化.

对于超出视界的模式,我们有
\begin{equation}
\tilde h'' + \frac{2}{\eta} \tilde h' \simeq 0 ~,
\end{equation}
我们把初始条件$\eta_{\rm in}$设在辐射为主导的时期,初始条件设成$\tilde h(\eta_{\rm in},k) = \tilde h_{\rm in} (k)$以及$\tilde h'(\eta_{\rm in},k) = 0$. 由于\autoref{TenPT_eq2} 是一个线性方程,不同动量的模式不会混合.所以研究一个特定的模式如何演化我们可以简单地设置$\tilde h_{\rm in}(k) = 1$,也就是说,我们研究初始值为1下$\tilde h({\eta,k})$的演化.





















