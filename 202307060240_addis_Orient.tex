% 定向
% 取向

本文是\href{https://encyclopediaofmath.org/wiki/Orientation}{Orientation}词条\footnote{Orientation. \textsl{Encyclopedia of Mathematics}. URL: \lstinline|http://encyclopediaofmath.org/index.php?title=Orientation&oldid=49719|}的\textbf{翻译},来自European Mathematical Society(欧洲数学学会)和Spinger出版社公开的 Encyclopedia of Mathematics(数学百科)。\footnote{本文由爱好者翻译,永久免费阅读,如有侵权请联系。}


\subsection{一般概念}

在传统数学中,一个\textbf{定向(orientation)}(或译作\textbf{取向})是指一种坐标系的等价划分,如果两个坐标系\textbf{正相关(positively related)}则是等价的。

% In classical mathematics, an orientation is the choice of an equivalence class of coordinate systems, where two coordinate systems belong to the same class if they are positively related (in a specific sense).

对于有限实线性空间$\mathbb{R}^n$,一个坐标系由一组基确定,而两组基等价的条件是\textbf{转移矩阵}\upref{TransM}的行列式为正数。这个等价关系划分出两个等价类。对于复数的情况,即$\mathbb{C}^n$,任取其复基$\{e_1, \cdots, e_n\}$,则能导出实基$\{e_1, \cdots, e_n, \I e_1, \cdots, \I e_n\}$,从而可以将其视为$\mathbb{R}^{2n}$。任意两个复基分别导出的实基就是正相关的(也就是说,复结构定义了$\mathbb{R}^{2n}$上的定向)。

% In the case of a finite-dimensional vector space Rn, a coordinate system is given by a basis, and two bases are positively related if the determinant of the transition matrix from one to the other is positive. There are two classes here. In a complex space Cn with complex basis e1,…,en, a real basis is given by e1,…,en,ie1,…,ien, considering the space as R2n. Any two real bases arising in this way from complex bases are positively related (i.e. a complex structure defines an orientation on R2n).

在一条线、一个面或者更一般的实\textbf{仿射空间}\upref{AfSp}$E^n$上,一个坐标系由一个点(原点)和一组基给定,坐标系的变换由一个平移(改变原点)和一个基变换给定。坐标系的变换是正的,当且仅当基变换的转移矩阵行列式为正数。(举个例子:基向量的偶置换。)两个坐标系定义的定向相同,当且仅当其中一个可以连续地变为另一个,即存在由参数$t\in[0, 1]$给定的一族坐标系$O_t, e_t$关于$t$是连续的,则$O_0, e_0$到$O_1, e_1$的变换就是连续的。在$n-1$维超平面上的\textbf{反射(reflection)}映射能反转定向,即将一个定向中的坐标系映入另一个定向。


% In a line, plane or, generally, a real affine space En, a coordinate system is given by the choice of a point (origin) and a basis. The change of coordinates is defined by a translation (changing the origin) and a change of basis. This change is positive if the matrix of the base change has positive determinant. (For example, an even permutation of the vectors in the basis.) Two coordinate systems define the same orientation if one of them can be continuously transformed into the other, i.e. if a family of coordinate systems Ot,et connecting the given systems O0,e0 and O1,e1 and depending continuously on t∈[0,1] exists. Reflection in an (n−1)-dimensional plane gives the opposite orientation, i.e. the other class.


坐标系的等价类也能用不同的\textbf{几何体(geometric figures)}\footnote{译注:geometric figures指任何点、线、面等构成的集合,是几何空间的子集。}来定义。如果一个几何体$X$按照某种规则与一个坐标系关联,那么它的镜像在同一个规则下就与取向不同的另一个坐标系关联,于是$X$(以及给定的那个规则)就定义了一个定向。比如说,在仿射平面$E^2$上,一个给定了方向的圆就定义了一个定向,其中正定向里的代表坐标系就是原点在圆心处、中点在圆上的两个向量,而第一个向量到第二个向量沿着给定方向走的角度最小\footnote{译注:原文比这还绕口。总之,给定的方向就是规定逆时针或者顺时针之类的方向。}。在$E^3$中,可以用一根螺丝来作参考系\footnote{译注:这里原文改成参考系(frame)了,译者也很疑惑。},令第一个基向量沿着螺丝旋进的方向,而第二个和第三个基向量之间的旋转则沿着螺丝旋进时旋转的方向。一个基(或称参考系)也可以用著名的\textbf{右手定则}来定义,即用右手大拇指、食指和中指来确定\footnote{译注:即向量叉乘的记忆法则,食指指向前方,中指向掌心弯折,大拇指翘起,则食指方向叉乘中指方向,所得方向就是大拇指所指方向。}。



% Classes of coordinate systems can be defined by different geometric figures. If a figure X is related by a specific rule to a coordinate system, then its mirror image should be related by the same rule to a coordinate system with the opposite orientation. In this way, X (along with the given rule) defines an orientation. For example, in the plane E2, a circle with a given direction of traversal defines a coordinate system from one class by the rule that the origin is at the centre of the circle, with the first basis vector taken arbitrarily while the second is taken so that the rotation from the first to the second through the smaller angle is the direction of traversal on the circle. In E3, a frame can be related to a screw. The first vector goes in the direction the screw moves when being screwed in, and the rotation from the second vector to the third coincides with the rotation of the screw as it is screwed in (it is supposed that all screws are threaded in the same way). A basis (frame) can also be defined in a well-known way by using the thumb and first two fingers on one's hand, as in the right-hand rule.


如果给定了$E^n$的一个定向,那么每一个半空间$E^n_+$就定义了边界面$E^{n-1}$上的一个定向。比如说,如果$E^n$的定向中后$n-1$个基向量都落在$E^{n-1}$中,而第一个基向量指入$E^n_+$,那么后$n-1$个基向量就定义了$E^{n-1}$上的一个定向。在$E^n$中,也可以用一个$n$维\textbf{单形}\footnote{译注:见\textbf{单纯形与复形}\upref{SimCom}。}($E^2$中的三角形,$E^3$中的三角锥)的顶点顺序来定义,即将原点选为第一个顶点,基向量则是从顶点顺次指向其它顶点的向量。同一组顶点的两个顺序属于统一定向,当且仅当它们之间是偶置换关系。一个给定了顶点顺序(至多差一个偶置换)的单形称为\textbf{定向的(oriented)}。一个$n$维定向单形的每一个$(n-1)$-面$\sigma^{n-1}$都有一个诱导定向:如果第一个顶点不在$\sigma^{n-1}$中,那么剩下的顶点顺序就被定义为$\sigma^{n-1}$的正向。



% If an orientation of En is given, then every half-space En+ defines an orientation on the boundary plane En−1. For example, it may be agreed that if the last n−1 vectors in an oriented basis lie in En−1, while the first vector points outwards from En+, then the last n−1 vectors define the orientation of En−1. In En an orientation can be defined by the order of the vertices of an n-dimensional simplex (a triangle in E2, a tetrahedron in E3). A basis is defined by choosing the origin at the first vertex, while the vectors of the basis point to the other vertices. Two orders define the same orientation if and only if they differ by an even permutation. A simplex with a fixed order of vertices up to an even permutation is said to be oriented. Every (n−1)-face σn−1 of an oriented simplex has an induced orientation: If the first vertex does not belong to σn−1, then the order of the others is taken to be positive for σn−1.

在一个\textbf{连通}\upref{Topo3}的\textbf{流形}\upref{Manif}$M$上,坐标系以一个\textbf{图册(atlas)}的形式出现:一组覆盖了$M$的图。如果各图之间的变换都是正的,那么称这些图构成的图册是定向的。对于一个微分流形来说,这意味着任何两个图之间的Jacobi矩阵处处为正。如果存在一个定向的图册,那么称$M$是可定向的。此时,全体定向图册的集合被分成两个等价类,两个图册等价当且仅当在它们俩中各任取一个图,这两图之间的变换都是正的。如此选择的等价类就被称为该流形的一个定向,选择方式可以是先选择一个图或者一个点$x_0$上的局部定向(包含$x_0$的连通图自动分为两个等价类)。对于微分流形,可以通过选择$x_0$处切平面的基来定义局部定向(比如说,圆上的旋转方向可以通过给定一个切向量来给定)。如果$M$有边界且已定向,那么边界也是可定向的,比如说按照下列规则定向:在边界的一个点上,选择一组用于给$M$定向的基向量,第一个基向量从边界$\partial M$指向内部,其它的基向量则在边界的切空间中,那么后面这些切向量则定义了边界上的一组定向基。


% In a connected manifold M, the coordinate system takes the form of an atlas: A set of charts (cf. Chart) which cover M. The atlas is said to be orienting if the coordinate transformations between charts are all positive. In the case of a differentiable manifold this means that the Jacobians of the coordinate transformations between any two charts are positive at every point. If an orienting atlas exists, then M is orientable. In this case, all orienting atlases divide into two classes such that the transition from the charts of one atlas to the charts of another is positive if and only if both atlases belong to the same class. A choice of this class is called an orientation of the manifold. This choice can be made by choosing one chart or local orientation at a point x0 (connected charts which contain x0 naturally divide into two classes). In the case of a differentiable manifold, a local orientation can be defined by choosing a basis in the tangent plane at the point x0 (for example, a rotation direction on the circle can be defined by choosing one tangent vector). If M has a boundary and is oriented, then the boundary is also orientable, for example according to the rule: At a point of the boundary, a basis is taken which orients M, the first vector of which is directed from ∂M, while the other vectors lie in the tangent plane to the boundary; these latter vectors are taken to be an orienting basis of the boundary.




在任何道路$q:[0, 1\to M]$上,我们可以选择一串图(覆盖该道路),使得两个相邻的图都是正连通的。这样一来,点$q(0)$处的定向就决定了$q(1)$处的定向,而且只需要道路是连续的且起点、终点确定,即可有此定义。如果$q$是个回路,即$q(0)=q(1)=x_0$,那么当按上述方式决定的$q(1)$定向与$q(0)$的相反,则称$q$是一个\textbf{反转定向回路}\footnote{译注:原文为If $q$ is a loop, i.e. $q(0)=q(1)=x_0$, then $q$ is called an orientation-reserving loop if these orientations are opposite. 根据句意和原文下一句判断,这里是原作者笔误,reserving应为reversing。}。于是,我们得到了一个从基本群$\pi_1(M, x_0)$到一个二元群的同态:只要让反转定向回路映射到$-1$即可。通过这一同态,我们能定义一个覆盖,对于不可定向流形来说此覆盖是2-层(two-sheeted)覆盖。我们说它是orienting的(因为覆盖空间会是可定向的)\footnote{译注:It是啥?那个manifold?}。这一同态还能定义$M$上的一个线丛(line bundle),当且仅当$M$可定向时它是平凡的。对于微分流形$M$,这可以定义为$n$次外微分形式的丛$\bigwedge^n(M)$,当且仅当流形可定向时,它有一个非零的截面,且这样的截面同时定义了$M$上的一个体积形式和定向。这个丛有一个特征映射$k:M\to \mathbb{R}P^n$。流形$M$可定向当且仅当其特征$\mu\in H^{n-1}(M; \mathbb{Z})$不为零,此特征是对偶于$\mathbb{R}P^{n-1}\subseteq \mathbb{R}P^{n}$的特征的像。它对偶于一个cycle,即$\mathbb{R}P^{n-1}$在映射$k$下在一般位置所取的(taken in general position)的预像,其支撑集(support)是整个流形。这个cycle就叫orienting的,因为它的补是可定向的:如果用此cycle切开$M$,就可以得到一个可定向流形。$M$本身也是可定向的(不可定向的),当且仅当这么切了以后能得到一个不连通流形(连通补)。比如,在$\mathbb{R}P^2$中,投影线$\mathbb{R}P^1$就可以当作一个orienting cycle。

% Along any path q:[0,1]→M, a chain of charts can be chosen such that two neighbouring charts are positively connected. Thus, an orientation at the point q(0) defines an orientation at the point q(1), and this relation depends on the path q only up to its continuous deformation when its ends are fixed. If q is a loop, i.e. q(0)=q(1)=x0, then q is called an orientation-reserving loop if these orientations are opposite. A homomorphism of the fundamental group π1(M,x0) into a group of order 2 arises: The orientation-reversing loops are sent to −1, while the others are sent to +1. Through this homomorphism a covering is created, which is a two-sheeted covering in the case of a non-orientable manifold. It is said to be orienting (since the covering space will be orientable). This same homomorphism defines a line bundle over M which is trivial if and only if M is orientable. For a differentiable M it can be defined as the bundle Λn(M) of differential forms of order n. It has a non-zero section only in the orientable case and then such a section simultaneously defines a volume form on M and an orientation. This bundle has a classifying mapping k:M→RPn. The manifold M is orientable if and only if the class μ∈Hn−1(M;Z) which is the image of the class dual to RPn−1⊂RPn, is not equal to zero. It is dual to a cycle whose support is the manifold which is the pre-image of RPn−1 under the mapping k, taken in general position. This cycle is said to be orienting, since its complement is orientable: If M is cut by means of the cycle, then an orientable manifold is obtained. M is itself orientable (non-orientable) if and only if a disconnected manifold (a connected complement) is obtained after the cut. For example, in RP2, a projective line RP1 serves as orienting cycle.



一个作了单纯剖分的流形$M$(或者一个伪流形)是可定向的,当且仅当可以把所有$n$维单纯形都定向,使得任意两个有公共$n-1$维面的单纯性在此公共面上诱导的定向相反。给定一条$n$维单纯性的闭链,其相邻单纯形的公共面是$n-1$维度的,那么称此链为\textbf{反转定向}的,如果作为起点和终点的两个单纯形在公共面上导出的定向相同,而其它相邻单纯形则导出相反定向。


我们也可以用同调论的语言来定义定向:对于一个连通、无界的可定向流形,其同调群$H_N(M; \mathbb{Z})$(其支撑集是闭的)同构于$\mathbb{Z}$,于是可以从两个生成元中选一个来定义定向\footnote{译注:就是$\pm 1\in\mathbb{Z}$分别代表两个不同的定向。}。对于有边界的连通流形这依然成立,只需要考虑$H_n(M, \partial M; \mathbb{Z]})$。在第一种情况中,可定向的性质是$M$的同伦不变量,而在第二种情况中是$(M, \partial M)$的同伦不变量。所以,莫比乌斯带和圆环有一个共同的同调型,但考虑边界则不同了\footnote{原文:So, the Möbius strip and the annulus have one and the same homotopy type but a different one if one considers the boundary. }。该流形的一个局部定向也可以通过选择同调群$H_n(M, M\setminus x_0)$的生成元来定义,该群同构于$\mathbb{Z}$。定向的同调论阐释让我们能把这一概念应用到一般的同调流形上(参见\href{https://encyclopediaofmath.org/wiki/Homology_manifold}{Homology manifold})。

% An orientation can be defined in the language of homology theory thus: For a connected orientable manifold without boundary, the homology group Hn(M;Z) (with closed supports) is isomorphic to Z, and the choice of one of the two generators defines an orientation. This is also true for a connected manifold with boundary, using Hn(M,∂M;Z). In the first instance, orientability is a homotopy invariant of M, while in the second, of the pair (M,∂M). So, the Möbius strip and the annulus have one and the same homotopy type but a different one if one considers the boundary. A local orientation of the manifold can also be defined by the choice of generators in the group Hn(M,M∖x0;Z), isomorphic to Z. The homological interpretation of orientation enables this concept to be applied to generalized homology manifolds (cf. Homology manifold).




令$p: E\to X$是一个唯一地定义在空间$X$上的纤维丛\upref{Fibre},纤维为$F^n$。任取道路$\gamma: (0, 1)\to X$\footnote{译注:同样据该百科的\href{https://encyclopediaofmath.org/wiki/Path}{道路},道路的定义应该是闭区间$[0, 1]$上的映射。},如果所有纤维的定向使得任意(非奇异)映射\footnote{参见\href{https://solitaryroad.com/c218.html}{这个页面},奇异映射(singular mapping)指将非零元素映入零元素的映射,反之即为非奇异映射。}$p^{-1}(\gamma(0))\to p^{-1}(\gamma(1))$都保持定向,那么该纤维丛就是定向(oriented)的,而各纤维的这些定向的选择即为该纤维丛的定向。举个例子,将一条莫比乌斯带视为一个圆上的向量丛,就没有定向,与之相比,一个圆柱的侧面就有定向。


% Let a fibration p:E→X with standard fibre Fn be defined uniquely over a space X. If the orientation of all fibres can be chosen such that any (non-singular) mapping p−1(γ(0))→p−1(γ(1)), defined by the path γ:(0,1)→X up to a non-singular homotopy, preserves the orientation, then the fibration is oriented, while the choice of the orientation of the fibres is the orientation of the fibration. For example, a Möbius strip, looked at as a vector bundle over a circle, does not possess an orientation, whereas the lateral surface of a cylinder does.




定向的概念,也使得建立在一个无穷维\textbf{巴拿赫空间}\upref{banach}或拓扑向量空间上的无穷维流形的情况得以自然推广。为了做到这一点,我们需要对一类线性算子作出限制,这里说的线性算子是从一个图到另一个图的转移映射的微分:这些线性算子应不仅是所讨论空间的全体同构映射的一般线性群,而是在这个一般线性群的一个不连通子群里;全体同构映射的一般线性群对于大多数经典向量空间都是同调意义上平凡的(一致拓扑)\footnote{本句原话引用如下:They must not simply belong to the general linear group of all isomorphisms of the structure space, which is homotopically trivial (in the uniform topology) for the majority of classical vector spaces, but must also be contained in a disconnected subgroup of the general linear group. }。这里说的子群,它的连通分支给定了定向的“正负号”。这个子群通常被选为Fredholm群,其元素是与恒等映射之差为一个\href{https://encyclopediaofmath.org/wiki/Completely-continuous_operator}{完全连续算子}的同构映射\footnote{本句原话引用如下:The subgroup usually chosen is the Fredholm group, consisting of those isomorphisms of the structure space for which the difference from the identity isomorphism is a completely-continuous operator.}。


% The concept of orientation also allows a natural generalization for the case of an infinite-dimensional manifold modelled on an infinite-dimensional Banach or topological vector space. This requires restrictions on the linear operators which are differentials of transitions from one chart to another: They must not simply belong to the general linear group of all isomorphisms of the structure space, which is homotopically trivial (in the uniform topology) for the majority of classical vector spaces, but must also be contained in a disconnected subgroup of the general linear group. The connected component of the given subgroup will then also provide the "sign" of the orientation. The subgroup usually chosen is the Fredholm group, consisting of those isomorphisms of the structure space for which the difference from the identity isomorphism is a completely-continuous operator.






\subsection{一般上同调理论中的定向}

令$E^*$是一个乘性的一般性上同调理论(下称“理论”)。给定一个悬垂同构$\widetilde{E}^0(S^0)\approx\widetilde{E}^n(S^n)$,这里$S^n$是$n$维球面,则存在一个单位$i\in \widetilde{E}^n(S^n)$,它有对应的元素$\gamma\in \widetilde{E}^n(S^n)$。

% Let E∗ be a multiplicative generalized cohomology theory (hereafter, simply a theory). There is a unit i∈E˜n(Sn) for which, given the suspension isomorphism E˜0(S0)≈E˜n(Sn), there is a corresponding element γ∈E˜n(Sn), where Sn is the n-dimensional sphere.

令$\xi$是一个\href{https://encyclopediaofmath.org/wiki/Path-connected_space}{道路连通(arcise connected)}空间$X$上的一个$n$维向量丛,令$T\xi$是$\xi$的\href{https://encyclopediaofmath.org/wiki/Thom_space}{Thom 空间}。令$i:S^n\to T\xi$为一个标准的嵌入,也就是点$x_0\in X$上的“纤维”的一个同胚映射。元素$u\in \widetilde{E}^n(T\xi)$称为丛$\xi$上的一个\textbf{定向}或者一个Thom类,如果$i^*(u)=\epsilon \gamma_n$,这里$i\in\widetilde{E}^0(S^0)$是一个可逆元素(比如$\epsilon=1$。)一个具有定向的向量丛,称为在理论$E^*$中可定向的,或者简称$E$-可定向的,选定了一个$E$-定向后的向量丛称为$E$-定向了的。此处的Thom同构$\widetilde{E}^*(T\xi)\approx E^*(X)$是合理的(见\href{https://encyclopediaofmath.org/index.php?title=Orientation&oldid=49719#Do}{[Do]})。对于$X$上一个$E$-定向的向量丛$\xi$,其全体定向的集合和群$\widetilde{E}^0(X) = \qty(\widetilde{E}^0(S^0))^*$

Let ξ be an n-dimensional vector bundle over an arcwise connected space X and let Tξ be the Thom space of ξ. Let i:Sn→Tξ be a standard imbedding, i.e. a homeomorphism on the "fibre" over a point x0∈X. The element u∈E˜n(Tξ) is called an orientation or a Thom class of the bundle ξ if i∗(u)=ϵγn, where ϵ∈E˜0(S0) is an invertible element (for example, ϵ=1). A bundle possessing an orientation is orientable in the theory E∗ or simply E-orientable, while a bundle with a chosen E-orientation is E-oriented. The Thom isomorphism E˜∗(Tξ)≈E∗(X) is valid (see [Do]). The set of orientations of a given E-oriented bundle ξ over X is in one-to-one correspondence with the elements of the group E˜0(X)⊕(E˜0(S0))∗, where (⋅)∗ is the group of invertible elements of the ring (⋅).

The trivial n-dimensional bundle θn possesses an orientation in any theory En, and if two out of the three bundles ξ,η,ξ⊕η are E-orientable, then the third is also E-orientable (see [Ma]). Moreover, the E-orientability of ξ entails the E-orientability of ξ⊕θn.

The concept of E-orientability is also introduced for any bundle in the sense of Hurewicz p:M→B, a fibre of which is homotopically equivalent to a sphere. The cone of the mapping p is called the Thom space of this bundle; further definitions are analogous. The definition of orientation of a vector bundle ξ reduces to this if a bundle of unit spheres (in some Riemannian metric on ξ) associated with ξ is taken as M. E-orientability is an invariant of the stable fibre-wise homotopy type of a vector (sphere) bundle. A bundle which is orientable in one theory is not necessarily orientable in another, but given a ring homomorphism of theories E∗→F∗, the property of E-orientability follows from F-orientability.

%未完待续。这玩意儿是真的长啊。















