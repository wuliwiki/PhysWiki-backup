% 北京大学 2004 年 考研 量子力学
% license Usr
% type Note

\textbf{声明}:“该内容来源于网络公开资料,不保证真实性,如有侵权请联系管理员”

\subsection{(15分)}

(a) 解释态叠加原理,给出性原理和态的统计解释。

(b) 写出由薛定谔方程一阶近似得到的二级能量修正公式。

(c) 在中心立场中,径向波函数 $R_{10}(r)$, $R_{20}(r)$, $R_{12}(r)$ 各有几个零点。

(d) 什么叫定态?有哪些量的特征态的线性组合加的态是否是定态?

(e) 简述并解释费曼规则。

(f) 解释(正常)塞曼效应及其多普勒效应。

\subsection{10分}

(a) 氢原子和谐振子的某套态波函数集是否是完备的?

(b) 在外电磁场的下,求电子在其中的哈密顿量。

(c) 两个自旋为$\frac{1}{2}$的全同粒子处于一维无限深势阱中,试求两粒子处于基态的总自旋波函数。

(d)$\hat{\sigma}_{\pm} = \hat{\sigma}_x \pm i \hat{\sigma}_y
$, 求 $\hat{\sigma}_{\pm}^2,\quad (\hat{\sigma}_+ \hat{\sigma}_-)^2$。

(e)$\hat{L}_{\pm} = \hat{L}_x \pm i \hat{L}_y$,  求 $[\hat{L}_z, \hat{L}_\pm],\quad [\hat{L}_+,\hat{L}_-], \quad [\hat{L}^2, \hat{L}_\pm]$。

(f) 在中心立场中,基态的轨道角动量为何值?并做简要解释。


\subsection{16分}
在 $(\hat{\vec S}^2, \hat{S}_z)$ 表象中,

(a) 求 $\hat{\vec S}^2, \hat{S}_z$ 的共同本征态及其对应的本征函数,

(b) $\hat{S}_y, \hat{S}_y^2$ 在 (a) 中所求各态中的平均值.

\subsection{11分}
已知薛定谔定常方程:$i\hbar \frac{\partial \psi}{\partial t} = \left[ -\frac{\hbar^2}{2m} \nabla^2 + V(\mathbf{\vec r}) \right] \Psi$,试求动量表象中的薛定谔方程。

\subsection{16分}
已知两个电子均处于单量子态 $\vec {a}, \vec{b}$ 的方空间任意态$(\vec{a_1} \cdot \vec{a})$, $(\vec{a_1} \cdot \vec{b})$,求在上述单态中的平均值。

\subsection{11分}
已知 $\Psi(x) = \frac{1}{\sqrt{a}} e^{-\frac{|x|}{a}}$, 且有 $|x| \rightarrow \infty$, 有$V(x)\to 0$,试求势能 $V(x)$ 的具体表示。

\subsection{11分}
已知5个自旋为1,质量为 $m$ 的全同粒子处于一个平面上的半径为 $R$ 的一个圆周,并且这5个粒子组成正5边形,5个粒子绕通过圆心的轴线转动而构成转动体系。

(a)写出上述体系的哈密顿量,并讨论守恒量有哪些。

(b)求出上述体系的本征值和本征函数。