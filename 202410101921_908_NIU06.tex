% 南京理工大学 2006 年 研究生入学考试试题 普通物理(B)
% license Usr
% type Note

\textbf{声明}:“该内容来源于网络公开资料,不保证真实性,如有侵权请联系管理员”

\subsection{填空题(26分,每空2分)}
\begin{enumerate}
        \item 质点作平面运动,运动方程为 $\vec{r} = (3t + 1) \hat{i} + (t^2 - 1) \hat{j}$,则t时刻质点的速度为$\underline{\hspace{2cm}}$,加速度为$\underline{\hspace{2cm}}$。
        
        \item 质量 $m$ 均匀分布的链条,总长为 $L$,有长度b伸在桌外。若自静止释放,则链条全部脱离光滑桌面的速度为$\underline{\hspace{3cm}}$,加速度为$\underline{\hspace{3cm}}$,若以桌面为势能零点,此时的势能为$\underline{\hspace{3cm}}$。
        \begin{figure}[ht]
        \centering
        \includegraphics[width=6cm]{./figures/4a1e4323cfe6e162.png}
        \caption{} \label{fig_NIU06_1}
        \end{figure}

        \item 如图,曲线I表示 $27^\circ C$ 的氧气分子的Maxwell速度分布,则图示中 $v_1 = \underline{\hspace{3cm}}$(6)。设曲线II也表示氧气分子某一温度下的Maxwell速度分布,且 $v_2 = 600 , m/s$,则曲线II对应的理论体温标 $T_2 = \underline{\hspace{3cm}}$。
        \begin{figure}[ht]
        \centering
        \includegraphics[width=6cm]{./figures/4263a7e1e4f0d198.png}
        \caption{} \label{fig_NIU06_2}
        \end{figure}

        \item 一理想卡诺循环一次卡诺循环对外做功1000J,卡诺循环高温热源温度 $T_1 = 500K$,低温热源温度 $T_2 = 300K$,则在一次循环中,在高温热源处吸热 $Q_1 = \underline{\hspace{3cm}}$,在低温热源处放热 $Q_2 = \underline{\hspace{3cm}}$。
        
        \item 弹簧振子的振动周期为T,现将弹簧截去一半,则新弹簧质子的振动周期为 $\underline{\hspace{3cm}}$,再将两半弹簧并联使用,则其振动周期变为 $\underline{\hspace{3cm}}$。
        
        \item 一平面简谐波方程(波函数)为 $y_1 = A \cos 2 \pi \left( \frac{t}{T} - \frac{x}{\lambda} \right)$,在 $x = \frac{\lambda}{2}$ 处有一反射壁,若平面波从空气传到反射壁而反射,反射时振幅不变,已知空气为波介质,则反射波波动方程为 $\underline{\hspace{3cm}}$,波节点的位置为 $\underline{\hspace{3cm}}$。
    \end{enumerate}
    \subsection{填空题(30分,每空2分)}
    \begin{enumerate}
    \item 半径为 $R$,带电量为 $Q$ 的金属导体球,其外表面处的电场强度大小为 $\underline{\hspace{2cm}}$,导体球内部或距球心 $r (< R)$ 处的电势为 $\underline{\hspace{2cm}}$。导体球外距球心 $r$ 处的电势为$ \underline{\hspace{3cm}}$。
    \item 一半径为 $R$ 的无限长薄圆管,平行轴向有一宽度为 $a$ ( $a \\ll R$ ) 的无限长细线,如图所示。管壁上均匀地通有稳恒电流,设管壁圆周上单位长度电流为 $I$,方向垂直纸面向上,则轴 $O$ 处的磁感应强度大小 $B_0 =$ $\underline{\hspace{2cm}}$,方向为 $\underline{\hspace{2cm}}$;
    \begin{figure}[ht]
    \centering
    \includegraphics[width=6cm]{./figures/2bd1e63cfabe151f.png}
    \caption{} \label{fig_NIU06_3}
    \end{figure}
    \item 在两个偏振化方向正交的偏振片之间插入第三个偏振片,当最后透过的光强为入射的自然光强的 $\frac{1}{8}$ 时,第三个偏振片偏振化方向与第一个偏振片偏振化方向的夹角为 $\underline{\hspace{2cm}}$,当最后透射光强为零时,第三个偏振片偏振化方向与第一个偏振片偏振化方向的夹角为 $\underline{\hspace{2cm}}$。
    \item 用氢-氖激光器发出的波长为 $632.8\text{nm}$ 的单色光做牛顿环实验,测得第 $k$ 级暗环的半径为 $5.625\text{mm}$,第 $k+5$ 级暗环的半径为 $7.956\text{mm}$,则所用平凸透镜的曲率半径 $R =$ $\underline{\hspace{1cm}}$,$k$ 的级数为 $\underline{\hspace{1cm}}$。
\end{enumerate}

\subsection{(12分)}