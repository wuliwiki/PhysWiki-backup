% 电动力学
% keys 电动力学|电磁波|静电学|静磁学|麦克斯韦方程组

\pentry{经典力学\upref{CM0}}

本科阶段通常会有几门不同深度关于电磁场场的课程, 第一门往往叫做\textbf{电磁学(electromagnetism)}, 进阶的课程叫做\textbf{电动力学(electrodynamics)}\footnote{电动力学字面上就是 “电的动力学” 而不是 “电动的力学”, 动力学英文就是 dynamics, 又例如热力学的英文是 thermodynamics.}. 从课程内容上来说, 电磁学就是简单的电动力学. 严格来说, 电动力学强调场和电荷的运动规律, 如果研究静止的电荷和电磁场, 可以将其称为\textbf{静电学(electrostatics)}或\textbf{静磁学(magnetostatics)}. 我们以下统一使用电动力学.

经典力学研究若干粒子(质点)受若干力后的运动情况,这里的粒子可以是任何有质量的粒子,力也可以是任何力.电动力学研究的是一堆带电荷的质点(既有质量又有电荷)受一堆电磁力后的运动情况.所以唯一需要做的是,弄清如何计算这些电磁力,再用经典力学就可以得出粒子的运动.

\subsection{电磁力和电磁场}

\begin{figure}[ht]
\centering
\includegraphics[width=14cm]{./figures/EM0_1.pdf}
\caption{静止的电荷突然振动几下又停下来,扰动就会以电磁波(橙色部分)的形式向外传播,到达另一电荷的位置时,才会有力的变化.图中只画出了一个电荷产生的场.} \label{EM0_fig1}
\end{figure}

经典力学中万有引力是超距的,就是说只要两个质点存在,不管他们相隔多远都会立即有作用力(与质量和距离有关),如果它们之间的距离发生改变,那么这个作用力也会立即改变.这是不合理的(现在我们知道任何信息的传递都不可能超过光速,如果一个粒子动一下,另一个粒子马上就能感到力的变化,那就可以超过光速传递信息).电动力学不同,我们需要先计算电场和磁场(场就像波一样,传播需要时间),再根据粒子所在位置的场计算粒子的受力(其他地方的场如何与粒子受力无关).例如库仑定理的形式虽然和万有引力一样,但是在电动力学中我们先计算一个粒子在另一个粒子处产生的电场,再计算处于该电场中的粒子所受电场力(库仑力).当两个粒子都静止时,这么做似乎和直接由距离计算力有没有区别,但如果某时刻其中一个粒子抖动了一下,电场(先不提磁场)就会像扰动的水波一样将这个扰动沿各个方向以一定的速度传播,直到传播到另一个粒子所在的地方,另一个粒子才能从电场中感觉到受力的变化.

磁场虽然产生的方式和对粒子的作用与电场不同,但也会像波一样传播.事实上,电场和磁场并不是独立传播的,上面说的电场扰动的传播时必须同时借助磁场.

\subsubsection{静电学问题}
若空间中的电场和磁场不随时间变化, 那么我们就说这是一个静电学问题. 静电学问题的充分必要条件是空间中电荷密度和电流密度不随时间变化. 静电学问题相比于一般问题更容易求解. 在电磁场变化缓慢的情况下, 我们也可以把一些问题近似为静电学问题, 例如在高中物理中我们从来不考虑点电荷运动时产生的磁场以及.

\subsection{电荷}
电荷在电动力学中扮演了两个角色.一是电磁场是由电荷产生的.二是只有带电荷的粒子在电磁场中才会受力,其他粒子不会.

\subsection{麦克斯韦方程组}
麦克斯韦方程组是一组写描述场变化规律的数学公式.可以想象成一个懂电动力学的计算机,只要告诉它所有的带电粒子在哪里,以及如何运动(位置关于时间的函数),它就能计算出空间中任何一点的电磁场.和机械波(如水波)不同,电场和磁场在每个位置都既有方向也有大小(有方向和大小的量叫做矢量,这种场叫做矢量场).

麦克斯韦方程组是电动力学的公设之一.

\subsection{电磁力}
任何一个带电粒子某时刻受到的电磁力(也叫广义洛伦兹力),都可以由它所在的位置处的电磁场(两个矢量)以及它当时运动速度(也是矢量)通过公式计算出来.这是电动力学的另一个公设.

可见,在经典力学的基础上加上麦克斯韦方程和广义洛伦兹力的公式后,如果我们知道一堆带电粒子在某个时刻的运动状态(位置和速度),我们就可以知道接下来每个粒子如何运动.
