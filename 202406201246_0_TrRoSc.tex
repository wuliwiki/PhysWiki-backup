% 平面上的平移、旋转与缩放
% license Usr
% type Tutor

\pentry{平面旋转矩阵\nref{nod_Rot2D}}{nod_0d06}

先看主动理解,即坐标系不变,若一个二维向量为 $\bvec r = (x, y)$, 经过三种变换后, 变为另一个向量 $\bvec r'$。

\begin{itemize}
\item 旋转矩阵(逆时针为正)为 $\mat R(\theta)$, 即 $\bvec r' = \mat R(\theta)\bvec r$ 把 $\bvec r$ 关于坐标原点旋转 $\theta$ 弧度得到 $\bvec r'$。
\item 缩放常数 $s$ 可以把一点关于原点收缩: $\bvec r' = s\bvec r = (sx, sy)$。
\item 平移向量 $\bvec d = (d_x,d_y)$ 可以描述平移 $\bvec r' = \bvec r + \bvec d = (x+d_x,y+d_y)$。
\end{itemize}

如果三种变换依次作用在 $\bvec r$ 上,但顺序不同,只会导致最终结果相差一个平移。例如:

“平移,旋转,缩放” 为
\begin{equation}
s\mat R (\bvec r + \bvec d) = s\mat R\bvec r + s\mat R\bvec d~.
\end{equation}

“旋转,平移,缩放” 为
\begin{equation}
s(\mat R\bvec r + \bvec d) = s\mat R\bvec r + s\bvec d~.
\end{equation}

“缩放,平移,旋转” 为
\begin{equation}
\mat R(s\bvec r + \bvec d) = s\mat R\bvec r + \mat R\bvec d~.
\end{equation}
可见第一项都是相同的,只有第二个常数项不同。

