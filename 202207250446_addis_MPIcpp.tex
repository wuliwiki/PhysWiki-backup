% MPI 笔记(C++)

\begin{issues}
\issueDraft
\end{issues}

\begin{itemize}
\item intel MPI 和 MKL 一样是免费的, google 一下, 下载安装包即可, 运行 \verb`sudo ./install.sh` 安装, 过程和 MKL 差不多
\item 如果已经安装了 MKL, 可能会提示目录已经存在(不确定 MKL 是否已经包含 MPI)
\item 安装好以后同样需要在 \verb`~/.bashrc` 中添加路径, 即在文件最后加入命令 \verb`source /opt/intel/compilers_and_libraries_2020.1.217/linux/mpi/intel64/bin/mpivars.sh`
\item 重启一下 shell, 运行 \verb`mpicxx --help`, 如果进入帮助页面就说明成功了

\item \href{https://people.sc.fsu.edu/~jburkardt/cpp_src/hello_mpi/hello_mpi.html}{入门例程}, \href{https://www.codingame.com/playgrounds/349/introduction-to-mpi/introduction-to-distributed-computing}{一个教程}
\item 尝试直接使用 Intel 的 MPI
\item \verb`mpicxx` 或者 \verb`mpigxx` 都是 g++ 的一个 wraper, 和 g++ 一样使用即可
\item 编译完以后, 如果直接运行程序只会有一个 process, 需要用例如 \verb|mpiexec -np 4 ./main.x| 指定进程数量. \verb|mpirun| 也一样.
\item \verb|MPI_Init(&argc, &argv);| 开始 MPI, \verb|MPI_Finalize();| 结束.
\item \verb|MPI_Comm_size(MPI_Comm comm, &size);| 获取指定 communicator 的进程数
\item \verb|MPI_Comm_rank(MPI_Comm comm, &rank);|获取当前进程的 id, 称为 rank.
\item \verb|int MPI_Send(const void *buf, int count, MPI_Datatype datatype, int dest, int tag, MPI_Comm comm)| 其中 \verb|dest| 是进程 \verb|id|, \verb|tag| 是一个编号用于区分不同的消息, \verb|comm| 一般是 \verb|MPI_COMM_WORLD| (所有进程).
\item 常用的 \verb|MPI_Datatype| 有 \verb|MPI_CHAR|, \verb|MPI_UNSIGNED_CHAR|, \verb|MPI_INT|, \verb|MPI_UNSIGNED|, \verb|MPI_LONG|, \verb|MPI_LONG_LONG|, \verb|MPI_FLOAT|, \verb|MPI_DOUBLE|, \verb|MPI_LONG_DOUBLE|, \verb|MPI_COMPLEX|, \verb|MPI_REAL16|, \verb|MPI_COMPLEX32|.
\item \verb|int MPI_Recv(void *buf, int count, MPI_Datatype datatype, int source, int tag, MPI_Comm comm, MPI_Status *status)|, 接收信息. \verb|status| 可以用于获取更多信息, 其他参数同理.
\item \verb|int MPI_Bcast(void *buffer, int count, MPI_Datatype datatype, int root,  MPI_Comm comm)| 只从 id 为 root 的进程调用, 对所有其他进程中的 buffer 赋值.
\item \verb|double MPI_Wtime()| 返回当前进程的时间(从任意起点开始, 单位秒).
\item \verb|int MPI_Reduce(const void* send_buffer, void* receive_buffer, int count, MPI_Datatype datatype, MPI_Op operation, int root, MPI_Comm communicator);| 把每个进程中的某个变量相加或相乘等, 赋值给 id 为 \verb|root| 的某个变量. 其中 \verb|operation| 可以是 \verb|MPI_MAX, MPI_MIN, MPI_SUM, MPI_PROD|
\end{itemize}
