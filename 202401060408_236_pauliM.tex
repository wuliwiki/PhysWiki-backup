% 泡利矩阵
% keys 泡利矩阵|自旋算符
% license Xiao
% type Tutor
\pentry{幺正变换\upref{Unitar},爱因斯坦求和约定\upref{EinSum}}
\begin{definition}{泡利矩阵}\label{def_pauliM_1}
泡利矩阵为 $2\times 2$ 幺正厄米矩阵:
\begin{equation}
\sigma^1 = \begin{pmatrix}
0 & 1\\
1 & 0
\end{pmatrix},\quad 
\sigma^2 = \begin{pmatrix}
0 & -i\\
i & 0
\end{pmatrix},\quad 
\sigma^3 = 
\begin{pmatrix}
1 & 0\\
0 & -1
\end{pmatrix}
~.
\end{equation}
有时也定义零号泡利矩阵为单位矩阵:
\begin{equation}
\sigma^0 = \pmat{1&0\\0&1}~.
\end{equation}

\end{definition}

\subsection{泡利矩阵的性质}
泡利矩阵是 $2\times 2$ 幺正厄米矩阵,它们有以下的性质
\begin{equation}
\sigma^i\cdot \sigma^i = \begin{pmatrix}1&0\\0&1\end{pmatrix},\quad \sigma^i = {\sigma^i}^\dagger~.
\end{equation}
即每个泡利矩阵的平方都是单位矩阵,由由于它是厄米矩阵,所以泡利矩阵也是幺正矩阵。

泡利矩阵有以下性质:
\begin{theorem}{}
\begin{equation}\label{eq_pauliM_1}
\sigma^a\sigma^b = \delta^{ab} + \epsilon^{abc}\sigma^c~.
\end{equation}
\end{theorem}
由这条性质可以推出
\begin{lemma}{}
\begin{equation}
[\sigma^i,\sigma^j] = 2\epsilon^{ijk}\sigma^k,\quad \{\sigma^i,\sigma^j\}=2\delta^{ij}~.
\end{equation}

\end{lemma}
