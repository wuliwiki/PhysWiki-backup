% 2019 年计算机学科专业基础综合全国联考卷
% 2019 年计算机学科专业基础综合全国联考卷

\subsection{一、单项选择题}
1~40小题, 每小腿2分, 共80分.下列每题输出的四个选项中,只有一个选项符合试题要求.

1.设$n$是描述问题规模的非负整数,下列程序段的时间复杂度是
\begin{lstlisting}[language=cpp]
x=0;
while(n>=(x+1)*(x+1))
    x=x+1;
\end{lstlisting}
A. $O(logn)$  $\quad$  B.$O(n)$  $\quad$   C.$O(n)$  $\quad$  D.$O(n^2)$

2. 若将一棵树$T$转化为对应的二叉树$BT$,则下列对$BT$的遍历中,其遍历序列与$T$的后根遍历序列相同的是 \\
A.先序遍历  $\quad$  B.中序遍历  $\quad$  C.后序遍历  $\quad$ D.按层遍历

3. 对$n$个互对不相同的符号进行哈夫曼编码.若生成的哈夫曼树共有115个结点,则$n$的值 \\
A.56  $\quad$  B.57  $\quad$  C. 58  $\quad$  D.60

4. 在任意一棵非平衡二叉树(AVL树)$T_1$中,删除某结点$v$之后形成平衡二叉树$T_2$,再将$v$插入$T_2$形成平衡二叉树$T_3$. 下列关于$T_1$与$T_3$的叙述中,正确的是 \\
I .若$v$是$T_1$的叶结点,则$T_1$与$T_3$可能不相同 \\
II .若$v$不是$T_1$的叶结点,则$T_1$与$T_3$定不相同  \\
III.若$v$不是$T_1$的叶结点,则$T_1$与$T_3$一定相同 \\
A.仅I  $\quad$  B.仅H  $\quad$ C.仅I , II  $\quad$  D.仅I 、III

5.下图所示的AOE网表示一项包含8个活动的工程.活动d的最早开始时间和最迟开始时间分别是
\begin{figure}[ht]
\centering
\includegraphics[width=12.5cm]{./figures/CSN19_1.png}
\caption{第5题图} \label{CSN19_fig1}
\end{figure}
A.1  $\quad$  B.2  $\quad$  C.3  $\quad$  D.4

6.用有向无环图描述表达式$(x+y)*(x+y)/x$,需要的顶点个数至少是 \\
A.5  $\quad$  B. 6  $\quad$  C.8  $\quad$  D.9

7.选择一个排序算法时,除算法的时空效率外.下列因素中,还需要考虑的是 \\
I.数据的规模 $\quad$ II.数据的存储方式 \\
III.算法的稳定性 $\quad$ IV.数据的初始状态 \\
A.仅II $\quad$ B.仅I、II \\
C.仅II、III.IV $\quad$ D. I、II、III、IV

8.现有长度为$11$且初始为空的散列表$HT$,散列函数是$H(hey)= key\%7$,采用线性探查(线性探测再散列)法解决冲突将关键字序列$87$,$40$,$30$,$6$,$11$,$22$,$98$,$20$依次插人到$HT$后,$HT$查找失败的平均查找长度是 \\
A.4 $\quad$ B.5.25 \\
C.6 $\quad$ D.6.29

9.设主串T=“abaabaabecabaabe" ,模式申S=“abaabe" ,采用KMP算法进行模式匹配,到匹配成功时为止,在匹配过程中进行的单个字符间的比较次数是 \\
A.9 $\quad$ B.10 $\quad$ C.12 $\quad$ D.15

10.排序过程中,对尚未确定最终位置的所有元素进行一遍处理称为一“趟”.下列序列中,不可能是快速排序第二趟结果的是 \\
A. 5,2,16,12,28,60,32,72 \\
B. 2,16,5,28,12 ,60,32,72 \\
C. 2,12,16,5,28,32,72,60 \\
D. 5,2,12,28,16,32,72,60

11.设外存上有120个初始归并段,进行12路归并时,为实现最佳归④g-620并,需要补充的虚段个数是 \\
A.1 $\quad$ B.2 $\quad$ C.3 $\quad$ D.4

12.下列关于冯.诺依曼结构计算机基本思想的叙述中,错误的 \\
A.程序的功能都通过中央处理器执行指令实现 \\
B.指令和数据都用二进制表示,形式上无差别 \\
C.指令按地址访问,数据都在指令中直接给出 \\
D.程序执行前,指令和数据需预先存放在存储器中

13.考虑以下C语言代码:
\begin{lstlisting}[language=cpp]
unsigned short usi = 65535;
short si = usi;
\end{lstlisting}

14.下列关于缺页处理的叙述中,\textbf{错误}的是 \\
A.缺页是在地址转换时CPU检测到的-种异常 \\
B.缺页处理由操作系统提供的缺页处理程序来完成 \\
c.缺页处理程序根据页故障地址从外存读入所缺失的页 \\
D.缺页处理完成后回到发生缺页的指令的下一条指令执行

15. 某计算机采用大端方式,按字节编址.某指令中操作数的机器数为1234 FFOOH,该操作数采用基址寻址方式,形式地址(用补码表示)为FFI2H,基址寄存器内容为F000 0000H,则该操作数的LSB(最低有效字节)所在的地址是 \\
A. F000 FFI2H \\
B. F000 FF15H \\
C. EFFF FF12H \\
D. EFFF FF15H

16.下列有关处理器时钟脉冲信号的叙述中,\textbf{错误}的是 \\
A.时钟脉冲信号由机器脉冲源发出的脉冲信号经整形和分频后形成 \\
B.时钟脉冲信号的寬度称为时钟周期,时钟周期的倒数为机器主频 \\
C.时钟周期以相邻状态单元间组合逻辑电路的最大延迟为基准确定 \\
D.处理器总是在每来一个时钟脉冲信号时就是开始执行一条新的指令

17.某指令功能为$R[r2]$←$R[r1]+M[R[r0]]$,其两个源操作数分别采用寄存器、寄存器间接寻址方式.对于下列给定部件,该指令在取数及执行过程中需要用到的是 \\
I.通用寄存器组(GPRs) $\quad$ II.算术逻辑单元( ALU) \\
III.存储器( Memory) $\quad$ IV.指令译码器(ID) \\
A.仅I、I $\quad$ B.仅1、1、I \\
C.仅I、M、IV $\quad$ D.仅I、I、IV

18.在采用“取指、译码/取数, 执行、访存, 写回”5段流水线的处理器中,执行如下指令序列,其中s0、s1.s2.s3和12表示寄存器编号. \\
\begin{lstlisting}
I1: add s2, sl, s0     // R[s2]←R[s1] + R[s0]
I2: load s3, 0(t2)     // R[s3]←M[R[[2] + 0]
I3: add s2, s2 s3      // R[s2]←-R[s2] + R[s3]
I4: store s2, 0(t2)    // M[R[t2] + 0]←R[s2]
\end{lstlisting}
下列指令对中,不存在数据冒险的是 \\
A. I1和13 $\quad$ B.12和I3 $\quad$ C.12和14 $\quad$ D.I3和I4

19.假定-台计算机采用3通道存储器总线,配套的内存条型号为
DDR3-1333,即内存条所接插的存储器总线的工作频率为1333
MHz、总线宽度为64位,则存储器总线的总带宽大约是 \\
A. 10.66 GB/s $\quad$ B. 32 GB/s $\quad$ C.64 GB/s $\quad$ D.96 GB/s

20.下列关于磁盘存储器的叙述中,错误的是 \\
A.磁盘的格式化容量比非格式化容量小 \\
B.扇区中包含数据、地址和校验等信息 \\
C.磁盘存储器的最小读写单位为一个字节 \\
D.磁盘存储器由磁盘控制器磁盘驱动器和盘片组成

21.某设备以中断方式与CPU进行数据交换,CPU主频为1 GHz,设备.接口中的数据缓冲寄存器为32位,设备的数据传输率为50 kB/s.若每次中断开销(包括中断响应和中断处理)为1000个时钟周期,则CPU用于该设备输入/输出的时间占整个CPU时间的百分比最多是 \\
A.1.25\% $\quad$ B.2.5\% $\quad$ C.5\% $\quad$ D. 12.5\%

22.下列关于DMA方式的叙述中,正确的是 \\
I. DMA传送前由设备驱动程序设置传送参数 \\
II. 数据传送前由DMA控制器请求总线使用权 \\
III. 数据传送由DMA控制器直接控制总线完成 \\
IV. DMA传送结束后的处理由中断服务程序完成 \\
A.仅I、I $\quad$ B.仅1、M、IV $\quad$ C.仅I、M、IV $\quad$ D. I、I、I、IV

23.下列关于线程的描述中,\textbf{错误}的是 \\
A.内核级线程的调度由操作系统完成 \\
B.操作系统为每个用户级线程建立一个线程控制块 \\
C.用户级线程间的切换比内核级线程间的切换效率高 \\
D.用户级线程可以在不支持内核级线程的操作系统上实现

24.下列选项中,可能将进程唤醒的事件是 \\
I. I/O结束 $\quad$ II.某进程退出临界区 \\
III.当前进程的时间片用完 \\
A.仅I $\quad$ B.仅III $\quad$ C.仅I、II $\quad$ D.I、II 、III

25.下列关于系统调用的叙述中,正确的是 \\
I.在执行系统调用服务程序的过程中,CPU处于内核态 \\
II.操作系统通过提供系统调用避免用户程序直接访向外设 \\
III.不同的操作系统为应用程序提供了统一的系统调用接口 \\
IV.系统调用是操作系统内核为应用程序提供服务的接口
A.仅I、IV $\quad$ B.仅II、III $\quad$ C.仅I、II、IV $\quad$ D.仅I、III、IV

26.下列选项中,可用于文件系统管理空闲磁盘块的数据结构是 \\
I.位图  $\quad$ II.索引节点 \\
III.空闲磁盘块链 $\quad$ IV.文件分配表(FAT) \\
A.仅I、II $\quad$ B.仅I、III、IV \\
C.仅I、III $\quad$ D.仅II、III、IV

27.系统采用二级反馈队列调度算法进行进程调度.就绪队列Q1采
用时间片轮转调度算法,时间片为10 ms;就绪队列Q2采用短进程
优先调度算法;系统优先调度Q1队列中的进程,当Q1为空时系统
才会调度Q2中的进程;新创建的进程首先进人QI;Q1中的进程
执行一个时间片后,若未结束,则转入Q2.若当前Q1 .Q2为空,系
统依次创建进程PI、P2后即开始进程调度P1、P2 需要的CPU时
间分别为30 ms和20 ms,则进程PI、P2在系统中的平均等待时
间为 \\
A.25 ms $\quad$ B.20 ms $\quad$ C.15 ms $\quad$ D. 10 ms

28.在分段存储管理系统中,用共享段表描述所有被共享的段.若进
程PI和P2共享段S,下列叙述中,错误的是 \\
A. 在物理内存中仅保存一份段S的内容 \\
B. 段S在P1和P2中应该具有相同的段号 \\
C. P1和P2共享段S在共享段表中的段表项 \\
D. P1和P2都不再使用段S时才回收段S所占的内存空间

29.某系统采用LRU页置换算法和局部置换策略,若系统为进程P预
分配了4个页框,进程P访问页号的序列为0,1,2,7,0,5,3,5,0,
2,7 ,6,则进程访问上述页的过程中,产生页置换的总次数是 \\
A.3 $\quad$ B.4 $\quad$ C.5 $\quad$ D.6

30.下列关于死锁的叙述中,正确的是 \\
I.可以通过剥夺进程资源解除死锁 \\
II.死锁的预防方法能确保系统不发生死锁 \\
III.银行家算法可以判断系统是否处于死锁状态 \\
IV.当系统出现死锁时,必然有两个或两个以上的进程处于阻塞态 \\
A.仅II、III $\quad$ B.仅I、II、IV $\quad$ C.仅I、II、III $\quad$ D.仅I、III、IV

31. 某计算机主存按字节编址, 采用二级分页存储管理,地址结构如下所示 \\
页目录号(10位) $\qquad$ 页号(10位) $\qquad$ 页内偏移(12位) \\
虚拟地址2050 1225H对应的页目录号、页号分别是 \\
A.081H ,101H $\qquad$ B.081H. ,401H \\
C.201H、101H $\quad$ D.201H、401H

32.在下列动态分区分配算法中,最容易产生内存碎片的是 \\
A.首次适应算法 $\quad$ B.最坏适应算法 \\
C.最佳适应算法 $\quad$ D.循环首次适应算法

33.OSI参考模型的第5层(自下而上)完成的主要功能是 \\
A.差错控制 $\quad$ B.路由选择 \\
C.会话管理 $\quad$ D.数据表示转换

34.100BaseT快速以太网使用的导向传输介质是 \\
A.双绞线 $\quad$ B.单模光纤 $\quad$ C.多模光纤 $\quad$ D.同轴电缆

35.对于滑动窗口协议,如果分组序号采用3比特编号,发送窗口大小为5,则接收窗口最大是 \\
A.2 $\quad$ B.3 $\quad$ C.4 $\quad$ D.5

36.假设一个采用CSMA/CD协议的100Mbps局域网,最小帧长是128B,则在一个冲突域内两个站点之间的单向传播延时最多是 \\
A. 2.56 μs $\quad$ B.5.12 μs $\quad$ C.10.24 μs
$\quad$ D.20.48 μs

37.若将101.200.16.0/20 划分为5个子网,则可能的最小子网的可分配IP地址数是 \\
A.126 $\quad$ B.254 $\quad$ C.510 $\quad$ D.1022

38.某客户通过一个TCP连接向服务器发送数据的部分过程如题38
图所示客户在t.时刻第一次收到确认序列号ack. _seq=100 的段,并发送序列号seq=100的段,但发生丢失.若TCP支持快速重传,则客户重新发送seq= 100段的时刻是 \\
\begin{figure}[ht]
\centering
\includegraphics[width=12cm]{./figures/CSN19_2.png}
\caption{第38题图} \label{CSN19_fig2}
\end{figure}
A. $t_1$ $\quad$ B. $t_2$ $\quad$ C. $t_3$ $\quad$ D. $t_4$

39.若主机甲主动发起一个与主机乙的TCP连接,甲,乙选择的初始序列号分别为2018和2046,则第三次握手TCP段的确认序列号是 \\
A.2018 $\quad$ B.2019 $\quad$ C.2046 $\quad$ D.2047

40.下列关于网络应用模型的叙述中,\textbf{错误}的是 \\
A.在P2P模型中,结点之间具有对等关系 \\
B.在客户/服务器(C/S)模型中,客户与客户之间可以直接通信 \\
C.在C/S模型中,主动发起通信的是客户,被动通信的是服务器 \\
D.在向多用户分发一个文件时,P2P模型通常比C/S模型所需时间短.

\subsection{二、综合应用题}
41~47小题,共70分.

41. (13分)设线性表L =(a+,a,..",a_-r,a-,a.)采用带头结点的
单链表保存,链表中结点定义如下:
\begin{lstlisting}[language=cpp]
typedef struct node
{
    int data;
    struct node *next;
} NODE;
\end{lstlisting}
请设计一个空间复杂度为$O(1)$且时间上尽可能高效的算法,重新
排列$L$中的各结点,得到线性表$L'=(a_1,a_n,a_2,a_{n-1},a_3,a_{n-2},...)$.要求: \\
(1)给出算法的基本设计思想 \\
(2)根据设计思想,采用C或C++语言描述算法,关键之处给出注释. \\
(3)说明你所设计的算法的时间复杂度.

42. (10分)请设计一个队列,要求满足:①初始时队列为空;②人队时,允许增加队列占用空间;③出队后,出队元素所占用的空间可重复使用,即整个队列所占用的空间只增不减;④人队操作和出队操作
的时间复杂度始终保持为$O(1)$.请回答下列问题: \\
(1)该队列应该选择链式存储结构,还是顺序存储结构? \\
(2)画出队列的初始状态.并给出判断队空和队满的条件 \\
(3)画出第一个元素人队后的队列状态. \\
(4)给出人队操作和出队操作的基本过程.

43. (8分)有$n(n\geqslant3)$位哲学家围坐在-张圆桌边,每位哲学家交替地就餐和思考.在圆桌中心有$m(m\geqslant1)$个碗,每两位哲学家之间有1根筷子.每位哲学家必须取到一个碗和两侧的筷之后,才能就餐,进餐完毕,将碗和筷子放回原位,并继续思考.为使尽可能多的哲学家同时就餐,且防止出现死锁现象,请使用信号量的P、V操作(wait()、signal()操作)描述上述过程中的互斥与同步,并说明所用信号量及初值的含义.

44. (7分)某计算机系统中的磁盘有300个柱面,每个柱面有10个磁道,每个磁道有200个扇区.扇区大小为512 B.文件系统的每个簇包含2个扇区.请回答下列问题: \\
(1)磁盘的容量是多少? \\
(2)假设磁头在85号柱面上,此时有4个磁盘访问请求,簇号分别为:100260、60005、101660和110560.若采用最短寻道时间优先(SSTF)调度算法,则系统访问簇的先后次序是什么? \\
(3)第100530簇在磁盘上的物理地址是什么?将簇号转换成磁盘物理地址的过程是由I/O系统的什么程序完成的?

45. (16分)已知$f(n)=n!=n\times(n-1)\times(n-2)\times...\times2\times1$.计算$f(n)$的$C$语言函数$f1$的源程序(阴影部分)及其在$32$位计算机$M$上的部分机器级代码如下:
\begin{lstlisting}[language=cpp]
int f1(int n){
1   00401000    55    push    ebp

    if(n>1)
11  00401018    83 7D 08 01    cmp    dword ptr[ebp+8],1
12  0040101C    7E17    jle  fl+35h (00401035)
    return n*f1(n-1);
13  0040101E    8B  45  08   mov  eax, dword ptr [ebp+8]
14  00401021    83  E8  01   sub  eax, 1
15  00401024    50           push  eax
16  00401025    E8  D6  FF  FF  FF    call f1 (00401000)
19  00401030    0F  AF  C1  imul eax, ecx
20  00401033    EB  05      jmp f1+3Ah(0040103a)
    else return 1;
21  00401035    B8  01  00  00  00    mov eax,1


26  00401040    3B  EC    cmp ebp ,esp
30  0040104A    C3        ret
}
\end{lstlisting}
其中,机器级代码行包括行号、虚拟地址.机器指令和汇编指令,计算机M按字节编址,int型数据占32位.请回答下列问题: \\
(1)计算f(10)需要调用函数f1多少次?执行哪条指令会递归调
用f1? \\
(2)上述代码中,哪条指令是条件转移指令?哪几条指令一定会
使程序跳转执行?  \\
(3)根据第16行call指令,第17行指令的虚拟地址应是多少? \\
(4) f(13)=6227 020800,但f1(13)的返回值为1 932 053 504,为什么两者不相等?要使f1(13)能区回正确的结果,应如何修改fl源程序? \\
(5)第19行in
R[ecx],当乘法器输出的高、低32位乘积之间满足什么条件时,溢出标志OF=1?要使CPU在发生溢出时转异常处理,编泽器应在imul指令后加一条什么指令?

46. (7分)对于题45,若计算机M的主存地址为32位,采用分页存储管理方式,页大小为4 KB,则第1行push指令和第30行ret指令是否在同一页中(说明理由)?若指令Cache有64行,采用4路组相联映射方式,主存块大小为64B,则32位主存地址中,哪几位表示块内地址?哪儿位表示Cache组号?哪几位表示标记(tag)信息?读取第16行call指令时,只可能在指令Cache的哪一组中命中(说明理由)?

47. (9分)某网络拓扑如题47图所示,其中R为路由器,主机H1~H4.的IP地址配置以及R的各接口IP地址配置如图中所示.现有若f台以太网交换机(无VLAN功能)和路由器两类网络互连设备可
供选择. \\
\begin{figure}[ht]
\centering
\includegraphics[width=14cm]{./figures/CSN19_3.png}
\caption{第47题图} \label{CSN19_fig3}
\end{figure}
请回答下列问题: \\
(1)设备1、设备2和设备3分别应选择什么类型网络设备?  \\
(2)设备1、设备2和设备3中,哪几个设备的接口需要配置IP地.
址?并为对应的接口配置正确的IP地址. \\
(3)为确保主机H1~H4能够访问Internet,R需要提供什么服务? \\
(4)若主机H3发送一个目的地址为192.168.1.127的IP数据报,网络中哪几个主机会接收该数据报?

\subsection{参考答案}

\subsubsection{一、单项选择题}
1. B $\quad$ 2. B $\quad$ 3. C $\quad$ 4. A $\quad$ 5. C \\
6. A $\quad$ 7. D $\quad$ 8. C $\quad$ 9. B $\quad$ 10. D \\
11. B $\quad$ 12.C. $\quad$ 13. A $\quad$ 14. D $\quad$ 15. D \\
16. D $\quad$ 17. B $\quad$ 18.C. $\quad$ 19. B $\quad$ 20.C. \\
21. A $\quad$ 22.D. $\quad$ 23. B $\quad$ 24.C. $\quad$ 25. C \\
26. B $\quad$ 27. C $\quad$ 28. B $\quad$ 29. C $\quad$ 30. B \\
31. A $\quad$ 32. C $\quad$ 33.C. $\quad$ 34. A $\quad$ 35. B \\
36. B $\quad$ 37. B $\quad$ 38. C $\quad$ 39.D. $\quad$ 40. B \\

\subsubsection{二、综合应用题}
41. [答案要点] \\
(1)算法的基本设计思想: \\
算法分3步完成.第1步,采用两个指针交替前行,找到单链表的中间结点;第2步,将单链表的后半段结点原地逆置;第3步,从单链表前后两段中依次各取一个结点,按要求重排. \\
(2)算法实现:
\begin{lstlisting}[language=cpp]
void change_list(NODE *h)
{
    NODE *p,*q,*r,*s;
    p=q=h;
    while (q->next != NULL)       // 寻找中间结点
    {
        p = p->next;                // p走一步.
        q = q->next;
        if (q->next != NULL) q = q->next; // q走两步
    }
    q = p->next;         // p所指结点为中间结点,q为后半段链表的首结点
    p->next = NULL;
    while (q != NULL)    // 将链表后半段逆置
    {
        r = q->nexl;
        q->next = p->nexl;
        p->next = q:
        q = r;
    }
    s = h->next;       // s指向前半段的第一个数据结点,即插入点
    q = p->nexl;       // q指向后半段的第一个数据结点
    p->next = NULL;
    while( q! = NULL )  //将链表后半段的结点插入
到指定位置
r = q->next;
// r指向后半段的下一个结点
q->next= s->next; /1 将q所指结点插入到s所指
结点之后
8->next = q:
s = q->next;
// s指向前半段的下一个插入点
q=r:
}
\end{lstlisting}
(3)算法的时间复杂度:$O(n)$.