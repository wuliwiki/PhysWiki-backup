% 延迟选择量子擦除实验
% 量子|干涉|擦除

\pentry{狄拉克符号\upref{braket},量子测量\upref{qmeas}}

如图\ref{fig:Experimental device}所示,左上角的光源发射光子,光子随后经过双缝。把红色路径标识的缝记为左缝,另一条则称为右缝。光子经过双缝后会进入$BBO$晶体。$BBO$将会吸收每一个入射的光子并发射出两个纠缠的光子。这两个光子经过棱镜后会分别射向不同的方向。我们把射向屏幕$D_0$的光子称为$A$光子,另一个称为$B$光子。$D_0$、$D_1$、$D_2$、$D_3$、$D_4$都是感光屏。$BS_a$、$BS_b$、$BS_c$是半透半反射镜,入射的光子有一半概率反射、一半概率透射。$M_a$、$M_b$是全反射镜。在经过$BS_c$后,从两条缝中射出的光子所产生的$B$光子的路径完全重合。从双缝到$D_1$、$D_2$、$D_3$、$D_4$光程远大于到$D_0$的光程,因此可以在观察到$D_0$上的现象后再对虚线框中的$B$光子观测系统进行调整。

在进行实验时,按如下的步骤操作:

(i)令光源每次只发射一个光子。

(ii)观察$D_0$屏幕上的光子落点,并记下另外四个屏幕中感光的屏幕编号。若$D_i(i=1,2,3,4)$屏幕感光,则将对应的光子落点记为$i$类点。重复进行多次。

(iii)只开放左狭缝,观察$D_0$屏幕上的光子落点,这类落点记为$L$类落点。重复进行多次。然后,在只开放右狭缝的情况下进行同样的工作,对应的落点记为$R$类落点。

实验完成后,将光子的落点画在图上,构成光强分布图。将$i(i=1,2,3,4,L,R)$类点的分布称为$i$类分布,所有光子落点的分布称为总分布。实验现象如下:

(i)总分布图中不存在干涉条纹。

(ii)$1$类分布与$2$类分布中存在干涉条纹。

(iii)$3$类分布与$4$类分布中不存在干涉条纹。

(iv)$3$类分布等于$R$类分布,$4$类分布等于$L$类分布。

(v)总分布等于$|\alpha|^2$倍的$L$类分布与$|\beta|^2$倍的$R$类分布之和。


