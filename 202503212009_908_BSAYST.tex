% 玻色–爱因斯坦统计(综述)
% license CCBYSA3
% type Wiki

本文根据 CC-BY-SA 协议转载翻译自维基百科\href{https://en.wikipedia.org/wiki/Bose\%E2\%80\%93Einstein_statistics}{相关文章}。

在量子统计学中,玻色–爱因斯坦统计(B–E统计)描述了在热力学平衡下,一组非相互作用的相同粒子占据一组可用离散能级的两种可能方式之一。粒子聚集在同一状态中的现象是遵循\textbf{玻色–爱因斯坦统计}的粒子的特征,它解释了激光光束的凝聚流动和超流氦的无摩擦爬升。这一行为的理论由萨廷德拉·纳特·玻色于1924-25年提出,他认识到一组相同且不可区分的粒子可以以这种方式分布。这个想法后来被阿尔伯特·爱因斯坦与玻色合作进行了采纳和扩展。

玻色–爱因斯坦统计仅适用于不遵循泡利不相容原理限制的粒子。遵循玻色–爱因斯坦统计的粒子称为玻色子,它们具有整数自旋。与此相对,遵循费米–狄拉克统计的粒子称为费米子,具有半整数自旋。
\subsection{玻色–爱因斯坦分布}  

在低温下,玻色子与费米子(遵循费米–狄拉克统计)表现不同,玻色子可以“凝聚”到相同的能级中,数量没有限制。这个看似不寻常的性质也导致了物质的特殊状态——玻色–爱因斯坦凝聚态。费米–狄拉克统计和玻色–爱因斯坦统计在量子效应重要且粒子“不可区分”时适用。如果粒子浓度满足
\[
\frac{N}{V} \geq n_{\text{q}},~
\]
其中\( N \)是粒子数,\( V \)是体积,\( n_q \)是量子浓度,对于量子浓度,粒子间的距离等于热德布罗意波长,因此粒子的波函数几乎不重叠。