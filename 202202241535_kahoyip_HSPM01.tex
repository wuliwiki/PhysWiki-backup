% 匀变速直线运动
% 高中物理|位移|速度|加速度

\addTODO{图像分析示例}

\subsection{基本概念}
\subsubsection{质点}
某些情况下,物体的大小和形状对研究的问题没有影响或影响可以忽略,而只需突出“物体具有质量”这个要素,我们可以把这个物体简化为一个具有质量的物质点,这样的点称为\textbf{质点}.

\subsubsection{参考系}
判断一个物体是运动还是静止,总要选取一个物体作为标准,被选作标准的物体叫\textbf{参照物}.
为了描述一个物体在空间中位置随时间的变化,我们要在参照物上建立一套\textbf{坐标系},并在同一坐标系的各处都配置同步的时钟,这便组成了一个\textbf{参考系}.

一般地,我们不严格区分“参照物”和“参考系”,而是强调\textbf{坐标系是参考系的数学抽象}.

\subsubsection{时间和时刻}
时刻:某一瞬时,是时间轴上的一点.常见描述为“第$n$秒初(末)”.

时间间隔:两个时刻之间的间隔,是时间轴上的一段.

设$t_1$和$t_2$分别为先后的两个时刻,$\Delta t$表示这两个时刻之间的时间间隔,则$\Delta t = t_2 - t_1$.

\subsubsection{路程,位置,位移}
路程:质点运动轨迹的长度.存在局限性,不能反应运动的某些本质,描述不够精确.常用符号为$s$,是一个\textbf{标量}.

位置:质点相对于参考点(常为坐标系原点)的距离和方向,在坐标系中是一个点.

位移:初位置指向末位置的有向线段,是描述质点位置变化的物理量,是一个\textbf{矢量}.常用符号为$x$,单位为$m$.

\subsubsection{速度}
定义:位移与发生这段位移所用时间之比,表示物体运动的快慢.表达式为
\begin{equation}
v=\frac{\Delta x}{\Delta t}
\end{equation}

其中,$v$是速度,$\Delta x$是$\Delta t$时间内的位移.速度采用比值定义法,不能说$v$与$\Delta x$成正比.如果$t$时间内物体发生的位移为$x$,则公式可表示为$v=x/t$.

单位:米每秒,符号是$m/s$,常用的单位还有$km/h$,$1m/s=3.6km/h$.

方向:速度是矢量,其方向与物体的运动方向相同.对于直线运动来说,如果我们选定某一个方向为正方向,则速度方向就可以用正、负号来表示.

\textbf{平均速度}:物体的位移与发生位移所用时间之比,描述物体位置变化的快慢,其方向与$\Delta x$一致.

\textbf{瞬时速度}:物体在某一时刻或某一位置的速度,描述物体在某一时刻或经过某一位置时运动的快慢,其方向与该物体在这个时刻或经过这个位置时的运动方向一致.瞬时速度可以用极短时间$\Delta t$内的平均速度来计算.另外,在匀速直线运动中,瞬时速度始终和平均速度相同.

\textbf{平均速率}:路程与时间之比,是一个标量.一般情况下,平均速度只表示位置变化的平均快慢,而平均速率才能表示通常意义的物体运动的平均快慢.一般情况下,平均速度的大小不等于平均速率的大小,只有在单方向直线运动中才相等,但也不能描述为平均速率就是平均速度.

\subsubsection{加速度}
定义:速度的变化量$\Delta v$与发生这一变化所用时间$\Delta t$之比.
公式:
\begin{equation}
a=\frac{\Delta v}{\Delta t}
\end{equation}

物理意义:描述物体运动速度变化的快慢.

单位:在国际单位制中,加速度的单位是$m/s^2$,读作米每平方秒.

方向:加速度的方向总是与速度变化量的方向相同,与速度方向无关.

\subsection{匀变速直线运动}
速度均匀变化的直线运动,加速度不变的直线运动.
\subsubsection{速度—时间公式}
\begin{equation}\label{HSPM01_eq1}
v=v_0+at
\end{equation}

$v_0$为初速度,$a$为加速度,$t$为时间,$v$为$t$时刻的瞬时速度.当初速度$v_0=0$时,有$v=at$.

物理意义:末速度等于初速度和速度变化量的矢量和.

若$a$与$v_0$同向,则物体做匀加速直线运动,$v$逐渐增大.若$a$与$v_0$同向,则物体做匀减速直线运动,$v$逐渐减小.

\subsubsection{位移—时间公式}
\begin{equation}\label{HSPM01_eq2}
x=v_0 t+\frac12at^2
\end{equation}

$v_0$为初速度,$a$为加速度,$t$为时间,$x$为$t$时刻的位移.

\subsubsection{速度—位移公式}
联立\autoref{HSPM01_eq1} 和\autoref{HSPM01_eq2} 消去时间$t$可得
\begin{equation}
v^2-v_0^2=2ax
\end{equation}

该式由匀变速直线运动的两个基本公式推导出来,便于解决不含时间的问题.