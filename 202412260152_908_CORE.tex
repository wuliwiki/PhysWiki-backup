% 计算机系统结构(综述)
% license CCBYSA3
% type Wiki

本文根据 CC-BY-SA 协议转载翻译自维基百科\href{https://en.wikipedia.org/wiki/Computer_architecture}{相关文章}。
  
在计算机科学和计算机工程中,计算机架构是对由组件部分组成的计算机系统结构的描述。它有时可以是忽略实现细节的高级描述。在更详细的层次上,描述可能包括指令集架构设计、微架构设计、逻辑设计和实现。
\begin{figure}[ht]
\centering
\includegraphics[width=10cm]{./figures/bd6cd8f496125c45.png}
\caption{基本计算机的框图,具有单处理器CPU。黑线表示控制流,红线表示数据流。箭头表示流动方向。} \label{fig_CORE_1}
\end{figure}
\subsection{历史}  
第一个有文献记载的计算机架构出现在查尔斯·巴贝奇与阿达·洛夫莱斯的通信中,描述了分析机。在1936年建造Z1计算机时,科纳德·祖泽在两项专利申请中描述了机器指令可以存储在与数据相同的存储器中,即存储程序概念。另有两个早期且重要的例子:

\begin{itemize}
\item 约翰·冯·诺依曼1945年的论文《EDVAC报告初稿》,描述了逻辑元素的组织;  
\item 艾伦·图灵的更详细的《自动计算引擎的提议电子计算器》,同样是1945年,并引用了约翰·冯·诺依曼的论文。
\end{itemize}

计算机文献中“架构”一词可以追溯到1959年Lyle R. Johnson和Frederick P. Brooks Jr.的工作,他们是IBM主要研究中心机器组织部门的成员。Johnson有机会撰写关于Stretch的专有研究通信,Stretch是IBM为洛斯阿拉莫斯国家实验室(当时称为洛斯阿拉莫斯科学实验室)开发的超级计算机。为了描述讨论这台奢华计算机的细节,他指出自己对格式、指令类型、硬件参数和速度增强的描述属于“系统架构”层次,这一术语比“机器组织”更有用。

随后,Brooks作为Stretch设计师,在一本名为《规划计算机系统:Stretch项目》的书中开篇提到:“计算机架构,像其他建筑架构一样,是确定结构用户需求的艺术,然后在经济和技术约束内尽可能有效地设计以满足这些需求。”

Brooks后来帮助开发了IBM System/360系列计算机,在这个系列中,“架构”成为了定义“用户需要知道的内容”的名词。System/360系列之后又推出了几款兼容的计算机系列,包括当前的IBM Z系列。后来,计算机用户开始以许多不那么明确的方式使用这一术语。

最早的计算机架构是通过纸面设计,然后直接构建成最终硬件形式的。后来,计算机架构原型被物理构建成晶体管–晶体管逻辑(TTL)计算机形式——例如6800和PA-RISC的原型——进行测试和调整,然后才投入最终硬件形式。从1990年代开始,新型计算机架构通常在某些其他计算机架构内部的计算机架构模拟器中“构建”、测试和调整;或者作为软微处理器在FPGA中构建;或者两者结合——然后才投入最终硬件形式。
\subsection{子类别}  
计算机架构学科有三个主要的子类别:[14]
\begin{itemize}
\item 指令集架构(ISA):定义了处理器读取和执行的机器代码,以及字长、内存地址模式、处理器寄存器和数据类型。
\item 微架构:也称为“计算机组织”,描述了特定处理器如何实现ISA。[15] 例如,计算机CPU缓存的大小是一个通常与ISA无关的问题。
\item 系统设计:包括计算系统中所有其他硬件组件,如除CPU外的数据处理(例如直接内存访问)、虚拟化和多处理。
\end{itemize}
计算机架构中还有其他技术。以下技术被像Intel这样的公司使用,估计在2002年[14]占计算机架构的1%:
\begin{itemize}
\item 宏架构:比微架构更抽象的架构层次。
\item 汇编指令集架构:一个智能汇编器可以将一组机器的通用抽象汇编语言转换为略有不同的机器语言,以适应不同的实现。
\item 程序员可见宏架构:高级语言工具如编译器可以定义一致的接口或契约,供程序员使用,抽象底层ISA和微架构之间的差异。例如,C、C++或Java标准定义了不同的程序员可见宏架构。
\item 微代码:微代码是将指令转换为芯片上运行的软件。它像硬件的封装器,呈现硬件指令集接口的首选版本。这个指令翻译设施给芯片设计师提供了灵活的选项:例如,1. 新改进版的芯片可以使用微代码呈现与旧芯片版本完全相同的指令集,这样所有面向该指令集的软件都可以在新芯片上运行,无需更改。 例如,2. 微代码可以为相同的底层芯片呈现多种指令集,使其能够运行更多种类的软件。
\item 引脚架构:微处理器应为硬件平台提供的硬件功能,例如x86引脚A20M、FERR/IGNNE或FLUSH等。还包括处理器应发出的消息,以便外部缓存可以被使无效(清空)。引脚架构功能比ISA功能更灵活,因为外部硬件可以适应新的编码,或将引脚功能转换为消息。这个术语“架构”适用,因为这些功能必须为兼容系统提供,即使详细的方法发生变化。”
\end{itemize}