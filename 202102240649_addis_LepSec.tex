% 时间的计算

\subsection{秒的定义}
\footnote{参考 Wikipedia 页面: \href{https://en.wikipedia.org/wiki/Second}{second},\href{https://en.wikipedia.org/wiki/Leap_second}{leap second}.}秒在历史上有过三种定义, 16 世纪末开始出现以秒为最小单位的钟表.当时秒的定义是将一个太阳日划分为 $24\times60\times60 = 86400$ 等分, 并定义每等分一秒. 而一个\textbf{太阳日(solar day)}可以定义为地球上某处观察到太阳从一天的最高点到第二天最高点的时间间隔. 太阳日的定义今天仍然有效.

第一个定义的问题在于潮汐力的作用, 地球的自转速度并不恒定, 1940 年左右石英钟的精确度已经超过了地球自转所定义的秒. 科学家发现用地球的回归年定义秒更精确, 于是从 1956 年, 科学家使用回归年来定义秒.

从 1967, 国际单位制\upref{SIunit}将秒重新定义为: 一秒等于铯(Cs)原子 133 基态的超精细能级之间的跃迁辐射的电磁波周期的 $9,192,631,770$ 倍. 该定义一直沿用至今. 本书中如无特别说明, “秒” 一律指国际单位的定义.

\subsection{闰秒}
由于地球日在下逐渐变长, 使目前一个地球日略大于 86400 秒(国际单位的定义), 而我们使用的钟表一天只有 86400 秒. 所以如果不做任何修正, 那么地方时(如格林尼治标准时间)的中午 12 点太阳将不会出现在(格林尼治的)最高点. 所以 UTC 时间规定在每个月的最后

在一些特殊的日子中加入一个闰秒. 闰秒只能在每年的 6 月 30 日 UTC 和 UT1
