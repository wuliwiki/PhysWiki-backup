% 进制、二进制
% license Usr
% type Tutor

\begin{issues}
\issueDraft
\end{issues}

\subsection{$N$ 进制需要几个符号?}
我们日常熟悉使用十进制。 但我们可能由于对其习以为常,而忘记它一般的规则。 所以我们不妨总结一下。 用阿拉伯数字表示十进制数时,每一位共有十个符号: $0,1,\dots, 9$。要特别注意其中并没有 “十” 这个符号,必须要用两位数 $10$ 才能表示出十, 而它并不是一个整体的符号。

许多国家的自然语言中存在比 9 大的数字的整体单词, 如英语的 eleven, twelve 可以用一个单词表示 $11$ 和 $12$, 但是当我们讨论十进制的阿拉伯数字写法时, 一位数符号中只有 $0,1,\dots, 9$ 而没有其他符号。 

下面我们会看到这个规律对任何进制都是一样的,也就是 $N$ 进制中的一位只需要 $N$ 个不同的符号(包括 0),而表示 $N$ 本身则需要进位成两位数才可以。

\subsection{什么是进位?}
如果从 $0$ 开始, 如何数数呢? 从十进制中我们可以总结出来, 当一位数的符号从第一个数到最后一个后,如果还要往下数,就在左边一位使用下一个符号,并把当前位归零。 例如 $9$ 可以看成 $09$, 下一个数在第二位使用下一个符号 $10$。 又例如 $199$ 的下一个数是 $200$, 这是因为最右边两位同时达到了最后一个符号,所以要在第三位使用下一个符号。 注意我们假设可以在一个数字左边写上任意多位的 $0$ 而不影响表示的值。

所以十进制的(右边)第二位的每个数代表第一位的 10 倍,而第三个数代表第一位的 100 倍。 例如十进制的 $48576$ 表示
\begin{equation}
48576 = 4\e{4} + 8\e{3} + 5\e{2} + 7\times 10 + 6~.
\end{equation}
这里的每一项使用了科学计数法。% 连接未完成

\subsection{三进制}
那么相似地, 对于任意的 $N$ 进制,




负数参考\enref{原码、反码、补码}{InvCom}。

