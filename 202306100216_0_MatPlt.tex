% Matlab 画图
% keys Matlab|编程|画图

\pentry{Matlab 的函数\upref{MatFun}}

Matlab 具有强大的画图功能, 这里仅介绍一些基础知识。 最常用的画图函数是 \verb|plot|, 例如
\begin{lstlisting}[language=matlabC]
>> x = linspace(0, 2*pi, 100); y = sin(x);
>> plot(x, y);
\end{lstlisting}
结果如\autoref{fig_MatPlt_1} (左) 所示。 如果要在该坐标系继续画图, 要用 \verb|hold on| 命令(\verb|on| 是 \verb|hold| 的输入变量), 否则每用一次 \verb|plot|, 之前画过的图都会被清除。 用 \verb|hold off| 可以重新恢复自动清除。
\begin{lstlisting}[language=matlabC]
>> y1 = cos(x);
>> hold on; plot(x, y1);
\end{lstlisting}
结果如\autoref{fig_MatPlt_1} (右) 所示, 注意新增曲线的颜色变化。
\begin{figure}[ht]
\centering
\includegraphics[width=13cm]{./figures/bf4a0fb4af39929b.pdf}
\caption{\lstinline|plot| 函数} \label{fig_MatPlt_1}
\end{figure}

如果我们要新建一个画图窗口, 用 \verb|figure| 函数。 若要指定画图的颜色, 可以添加 \verb|figure| 的第三个变量, 用一个字符表示颜色(red:\verb|'r'|, green:\verb|'g'|, blue:\verb|'b'|, yellow:\verb|'y'|, magenta: \verb|'m'|, cyan: \verb|'c'|, black: \verb|'k'|, white: \verb|'w'|)。 例如
\begin{lstlisting}[language=matlabC]
>> x2 = cos(x); y2 = sin(x);
>> figure; plot(x2, y2, 'r');
\end{lstlisting}
在新增的窗口中, 结果如\autoref{fig_MatPlt_2} (左)所示。 注意根据窗口尺寸的不同, $x$ 轴和 $y$ 轴的单位长度一般不同, 若要使其相同, 可以在 \verb|plot| 后面用 \verb|axis equal| 命令(其中字符串 \verb|equal| 是 \verb|axis| 函数的输入变量), 得到\autoref{fig_MatPlt_2} (右)。
\begin{figure}[ht]
\centering
\includegraphics[width=13cm]{./figures/3eed7b14ff837ee7.pdf}
\caption{红色的单位圆} \label{fig_MatPlt_2}
\end{figure}
若要调整坐标轴的范围, 也可用 \verb|axis| 函数。 另外可以在 \verb|plot| 的第三个变量的字符串中设定曲线的形状, 用 \verb|xlabel| 和 \verb|ylabel| 函数分别设置 $x$ 轴和 $y$ 轴的文字, 用 \verb|title| 函数设置图片标题
\begin{lstlisting}[language=matlabC]
>> plot(x2, y2, '.-r');
>> axis([-1.2, 1.2, -1.2, 1.2]);
>> xlabel('x'); ylabel('y'); title('unit circle');
\end{lstlisting}
其中 \verb|'.-'| 表示带点的连线,点的坐标由 \verb|x2| 和 \verb|y2| 决定(另外 \verb|'+-'| 表示带加号的连线, \verb|'o-'| 表示带圆圈的连线)。 \verb|axis| 中行矢量中的四个数分别是 $x$ 轴的最小最大值和 $y$ 轴的最小最大值。 结果如\autoref{fig_MatPlt_3} (左)所示。
\begin{figure}[ht]
\centering
\includegraphics[width=11cm]{./figures/1895681d1cfa00af.pdf}
\caption{红色的单位圆} \label{fig_MatPlt_3}
\end{figure}

要改变当前窗口中的字号, 例如 \verb|set(gca, 'FontSize', 14);|。 其中 \verb|gca| 获取当前坐标系的对象(get current axis), \verb|set| 函数设置该对象的 \verb|FontSize| 属性为 \verb|14|。 在画图窗口菜单中的 \verb|View -> Property Inspector| 可以查看和修改一张图中任何对象的属性, 包括画图窗口的大小和位置。

除了 \verb|plot| 以外, 常用的还有 \verb|scatter| 函数, 用于画散点图。 格式与 \verb|plot| 相似。 默认的散点形状是圆圈, 但也可以在第三个变量中设置颜色和 \verb|'+'|, \verb|'x'|, \verb|'.'| 等形状。 例如
\begin{lstlisting}[language=matlabC]
>> hold on; scatter(0, 0, 'b');
\end{lstlisting}
结果如\autoref{fig_MatPlt_3} (右)所示。

如果直接通过菜单保存图片, 会默认使用显示器的分辨率, 要按指定的分辨率保存图片例如 \verb|exportgraphics(gcf, '文件名.png', 'Resolution', 300)|。 另外也可以用菜单或者 \verb|saveas()| 保存为矢量图等格式。

最后, 如果要关闭当前画图窗口, 用 \verb|close| 函数(无输入变量), 如果要关闭所有窗口, 用 \verb|close all| 即可。

\subsubsection{坐标显示}
可以用 \verb|axis| 对象的 \verb|XTick| 和 \verb|XTickLabel| 属性来设置坐标点和显示的文字, 例如 
\begin{lstlisting}[language=matlab]
figure; plot([1,2,3,2*pi],[1,2,3,5]);
set(gca,'XTick',0:pi/2:2*pi); % gca 用户获取当前的坐标轴对象
set(gca,'XTickLabel',{'0','\pi/2','\pi','3\pi/2','2\pi'});
\end{lstlisting}
\begin{figure}[ht]
\centering
\includegraphics[width=9cm]{./figures/5283caea47d60abd.pdf}
\caption{设置坐标点和显示文字} \label{fig_MatPlt_4}
\end{figure}
