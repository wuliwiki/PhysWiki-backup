% 利用复数方法证明三角恒等式
% 复数|三角恒等式

\begin{issues}
\issueTODO
\end{issues}

\pentry{复数\upref{CplxNo}}

\subsection{理论基础}
要借助复数证明三角恒等式,我们一般需要构造具有以下形式的复数:
$$w=\cos \alpha+i\sin\alpha$$

这类复数具有许多好的性质,我们熟知的有:

\begin{equation}% \tag{1}
(\cos\alpha+i\sin\alpha)(\cos\beta+i\sin\beta)=\cos(\alpha+\beta)+i\sin(\alpha+\beta)
\end{equation}

\begin{equation}% \tag{2}
(\cos\alpha+i\sin\alpha)^n=\cos n\alpha+i\sin n\alpha
\end{equation}

此外,我们再引入另外两条常用的性质:

$$\begin{aligned}
 1-(\cos\alpha+i\sin\alpha)^n&=1-\left(\cos{\frac{n\alpha}2+i\sin{\frac{n\alpha}2}}\right)^{2}\\ &=1-\left(\cos^2{\frac{n\alpha}2}-\sin^2{\frac{n\alpha}2}+2i\sin{\frac{n\alpha}2}\cos{\frac{n\alpha}2}\right)\\ &=2\sin^2{\frac{n\alpha}2}-2i\sin{\frac{n\alpha}2}\cos{\frac{n\alpha}2}\\ &=-2i\sin{\frac{n\alpha}2}\left(\cos{\frac{n\alpha}2}+i\sin{\frac{n\alpha}2}\right) 
\end{aligned}$$

故:
% \tag{3}
\begin{equation}\label{TriCom_eq4}
1-(\cos\alpha+i\sin\alpha)^n=-2i\sin{\frac{n\alpha}2}\left(\cos{\frac{n\alpha}2}+i\sin{\frac{n\alpha}2}\right)
\end{equation}

同理可得:
\begin{equation}\label{TriCom_eq5}% \tag{4}
1+(\cos\alpha+i\sin\alpha)^n=2\cos{\frac{n\alpha}2}\left(\cos{\frac{n\alpha}2}+i\sin{\frac{n\alpha}2}\right)
\end{equation}

\subsection{累加}
\begin{theorem}{}

\begin{equation}% \tag{1}
\begin{aligned}
\sum_{k=0}^n\sin(\alpha+k\beta)&=\frac{\sin\left(\alpha+\frac{n\beta}2\right)\sin\left(\frac{n+1}2\beta\right)}{\sin\frac{\beta}2}\\\sum_{k=0}^n\cos(\alpha+k\beta)&=\frac{\cos\left(\alpha+\frac{n\beta}2\right)\sin\left(\frac{n+1}2\beta\right)}{\sin\frac{\beta}2}
\end{aligned}
\end{equation}

\textbf{证明:}设 $w_1=\cos \alpha+i\sin\alpha\,,w_2=\cos \beta+i\sin\beta$ ,则:

$$\begin{aligned}
\sum_{k=0}^n\cos(\alpha+k\beta)+i\sum_{k=0}^n\sin(\alpha+k\beta)&=\sum_{k=0}^n\left[\cos(\alpha+k\beta)+i\sin(\alpha+k\beta)\right]\\&=w_1\sum_{k=0}^nw_2^k \\&=w_1\left(\frac{1-w_2^{n+1}}{1-w_2}\right)\\ &=(\cos \alpha+i\sin\alpha)\frac{-2i\sin{\frac{(n+1)\beta}2}\left[\cos{\frac{(n+1)\beta}2}+i\sin{\frac{(n+1)\beta}2}\right]}{-2i\sin{\frac{\beta}2}\left(\cos{\frac{\beta}2}+i\sin{\frac{\beta}2}\right)}\\ &=\frac{\cos\left(\alpha+\frac{n\beta}2\right)\sin\left(\frac{n+1}2\beta\right)}{\sin\frac{\beta}2}+i\frac{\sin\left(\alpha+\frac{n\beta}2\right)\sin\left(\frac{n+1}2\beta\right)}{\sin\frac{\beta}2}
\end{aligned}$$

对比虚实部,即证
\end{theorem}

令 $\alpha=\beta$ ,得:
$$\sum_{k=1}^n\sin k\alpha=\frac{\sin{\frac{(n+1)\alpha}2}\sin{\frac{n\alpha}{2}}}{\sin{\frac{\alpha}2}}$$ $$\sum_{k=1}^n\cos k\alpha=\frac{\cos{\frac{(n+1)\alpha}2}\sin{\frac{n\alpha}{2}}}{\sin{\frac{\alpha}2}}$$ 

可以推知:

$\displaystyle{\cos\frac{\pi}{7}+\cos\frac{3\pi}{7}+\cos\frac{5\pi}{7}=\frac{1}{2}}$

$\displaystyle{\cos\frac{\pi}{9}+\cos\frac{5\pi}{9}+\cos\frac{7\pi}{9}=0}$

\subsection{连乘}
要证明与三角函数有关的连乘式,我们需要考虑多项式的分解,例如:
$$z^5-1=(z-1)(z^4+z^3+z^2+z+1)$$

记 $\displaystyle{w=\cos\frac{\pi}5+i\sin\frac{\pi}5}$,$z^5-1=0$ 有 $w^2,w^4,w^6,w^8,w^{10}$ 五个根,而 $w^{10}=1$ ,于是有:
$$z^4+z^3+z^2+z+1=(z-w^2)(z-w^4)(z-w^6)(z-w^8)$$

我们发现,如果我们代入 $z=1$ ,就能利用上述的\autoref{TriCom_eq4}  ,得到关于 $\sin$ 的连乘式;如果我们代入 $z=-1$ ,就能利用上述的\autoref{TriCom_eq5}, 得到关于 $\cos$ 的连乘式

下面我们具体讨论以下三类不同的方程:
$$\begin{aligned}
z^{2k-1}+1&=0\\z^{2k-1}-1&=0 \\z^{2k}-1&=0
\end{aligned}$$

记 $\displaystyle{w=\cos\frac{\pi}{2m-1}+i\sin\frac{\pi}{2m-1}}$ ,则 $w,w^3\cdots,w^{4m-3}$ 是 $z^{2m-1}+1=0$ 的 $2m-1$ 个根

又因为 $w^{2m-1}=-1$ ,于是 $w\cdots ,w^{2m-3},w^{2m+1}\cdots ,w^{4m-3}$ 是 $z^{2m-1}+1=0$ 的 $2m-2$ 个虚根

由因式分解 $z^{2m-1}+1=(z+1)(z^{2m-2}-z^{2m-3}+\cdots+z^{2}-z+1)$ ,知:
$$z^{2m-2}-z^{2m-3}+\cdots-z+1=(z-w)\cdots(z-w^{2m-3})(z-w^{2m+1})\cdots(z-w^{4m-3})$$

代入 $z=1$,并置 $\mu=w^{\frac12}$ 得:

$$\begin{aligned}
1&=(1-w)\cdots(1-w^{2m-3})(1-w^{2m+1})\cdots(1-w^{4m-3})\\&=\prod_{\substack{k=1\\k\neq m}}^{2m-1}(-2i)\mu^{2k-1}\sin\frac{(2k-1)\pi}{4m-2}\\&=(-2i)^{2m-2}{\mu}^{(2m-1)(2m-2)}\prod_{k=1}^{2m-1}\sin{\frac{(2k-1)\pi}{4m-2}}\\ &=4^{m-1}(-1)^{m-1}\cos(m-1)\pi\prod_{k=1}^{2m-1}\sin{\frac{(2k-1)\pi}{4m-2}}\\ &=4^{m-1}\prod_{k=1}^{2m-1}\sin{\frac{(2k-1)\pi}{4m-2}}  
\end{aligned}$$

所以: 
\begin{equation}% \tag{6}
\prod_{k=1}^{2m-1}\sin{\frac{(2k-1)\pi}{4m-2}}=\frac{1}{4^{m-1}}
\end{equation}

又因为 $\displaystyle{\sin\frac{(2k-1)\pi}{4m-2}=\sin\frac{(4m-2k-1)\pi}{4m-2}}$ ,故:
\begin{equation}% \tag{7}
\prod_{k=1}^{m-1}\sin{\frac{(2k-1)\pi}{4m-2}}=\prod_{k=m+1}^{2m-1}\sin{\frac{(2k-1)\pi}{4m-2}}=\frac{1}{2^{m-1}}
\end{equation}

代入 $z=-1$ 得:

\begin{equation}
\begin{aligned}
2m-1&=(-1-w)\cdots(-1-w^{2m-3})(-1-w^{2m+1})\cdots(-1-w^{4m-3})\\ &=(-1)^{2m-2}(1+w)\cdots(1+w^{2m-3})(1+w^{2m+1})\cdots(1+w^{2m-3})\\ &=\prod_{\substack{k=1\\k\neq m}}^{2m-1}2\mu^{2k-1}\cos\frac{(2k-1)\pi}{4m-2}\\ &=2^{2m-2}\mu^{(2m-1)(2m-2)}\prod_{k=1}^{m-1}\cos{\frac{(2k-1)\pi}{4m-2}}\prod_{k=m+1}^{2m-1}\cos{\frac{(2k-1)\pi}{4m-2}}\\ &=4^{m-1}\cos(m-1)\pi\prod_{k=1}^{m-1}\cos{\frac{(2k-1)\pi}{4m-2}}\prod_{k=m+1}^{2m-1}\cos{\frac{(2k-1)\pi}{4m-2}}
\end{aligned}
\end{equation}

又因为 $\displaystyle{\cos\frac{(2k-1)\pi}{4m-2}=-\cos\frac{(4m-2k-1)}{4m-2}}$ ,故:

\begin{equation}
\begin{aligned}
2m-1&=4^{m-1}\cos(m-1)\pi\prod_{k=1}^{m-1}\cos{\frac{(2k-1)\pi}{4m-2}}\prod_{k=m+1}^{2m-1}\cos{\frac{(2k-1)\pi}{4m-2}}\\ &=4^{m-1}\cos(m-1)\pi(-1)^{m-1}\left[\prod_{k=1}^{m-1}\cos{\frac{(2k-1)\pi}{4m-2}}\right]^2
\end{aligned}
\end{equation}

于是:

\begin{equation}% \tag{1}
\prod_{k=1}^{m-1}\cos{\frac{(2k-1)\pi}{4m-2}}=\prod_{k=m+1}^{2m-1}\cos{\frac{(2k-1)\pi}{4m-2}}=\frac{\sqrt{2m-1}}{2^{m-1}}
\end{equation}

记 $\displaystyle{w=\cos\frac{\pi}{2m-1}+i\sin\frac{\pi}{2m-1}}$ ,则 $w^2,w^4\cdots,w^{4m-2}$ 是 $z^{2m-1}-1=0$ 的 $2m-1$ 个根

又因为 $w^{4m-2}=1$ ,于是 $w^2,w^4\cdots ,w^{4m-4}$ 是 $z^{2m-1}-1=0$ 的 $2m-2$ 个虚根

由因式分解 $\displaystyle{z^{2m-1}-1=(z-1)(z^{2m-2}+z^{2m-3}+\cdots+z^{2}+z+1)}$ ,知:
$$z^{2m-2}+z^{2m-3}+\cdots+z+1=(z-w^2)\cdots(z-w^{4m-4})$$

代入 $z=1$ ,得:
$$\begin{aligned} 2m-1&=(1-w^2)(1-w^4)\cdots(1-w^{4m-4})\\ &=\left(-2i\mu^2\sin\frac{2\pi}{4m-2}\right)\left(-2i\mu^4\sin\frac{4\pi}{4m-2}\right)\cdots\left[-2i\mu^{4m-4}\sin\frac{(4m-4)\pi}{4m-2}\right]\\ &=(-2i)^{2m-2}\mu^{(2m-1)(2m-2)}\prod_{k=1}^{2m-2}\sin\frac{k\pi}{2m-1}\\ &=4^{m-1}(-1)^{m-1}\cos(m-1)\pi\prod_{k=1}^{2m-2}\sin\frac{k\pi}{2m-1} \end{aligned} $$

于是: 
\begin{equation}\label{TriCom_eq2}% \tag{9} 
\prod_{k=1}^{2m-2}\sin\frac{k\pi}{2m-1}=\frac{2m-1}{4^{m-1}}
\end{equation}

代入 $z=-1$ ,得:

$\begin{aligned} 1&=(-1-w^2)(-1-w^4)\cdots(1-w^{4m-4})\\ &=(-1)^{2m-2}(1+w^2)(1+w^4)\cdots(1+w^{4m-4})\\ &=\left(2\mu^2\cos\frac{2\pi}{4m-2}\right)\left(2\mu^4\cos\frac{4\pi}{4m-2}\right)\cdots\left[2\mu^{4m-4}\cos\frac{(4m-4)\pi}{4m-2}\right]\\ &=2^{2m-2}\mu^{(2m-1)(2m-2)}\prod_{k=1}^{2m-2}\cos\frac{k\pi}{2m-1}\\ &=4^{m-1}\cos(m-1)\pi\prod_{k=1}^{2m-2}\cos\frac{k\pi}{2m-1} \end{aligned}$ 

故:
\begin{equation}% \tag{10}
\prod_{k=1}^{2m-2}\cos\frac{k\pi}{2m-1}=\frac{(-1)^{m-1}}{4^{m-1}} 
\end{equation}

记 $\displaystyle{w=\cos\frac{\pi}{2m}+i\sin\frac{\pi}{2m}}$,则 $w^2,w^4,\dots ,w^{4m}$ 是 $z^{2m}-1=0$ 的 $2m$ 个根

又因为 $w^{2m}=-1\,,w^{4m}=1$ ,于是 $w^2\cdots,w^{2m-2},w^{2m+2}\cdots,w^{4m-2}$ 是 $z^{2m}-1=0$ 的 $2m-2$ 个虚根

由因式分解 $z^{2m}-1=(z^2-1)(z^{2m-2}+\cdots+z^4+z^2+1)$ ,知:
$$z^{2m-2}+\cdots+z^4+z^2+1=(z-w^2)\cdots(z-w^{2m-2})(z-w^{2m+2})\cdots(z-w^{4m-2})$$

代入 $z=1$ ,得:

$$\begin{aligned} m&=(1-w^2)\cdots(1-w^{2m-2})(1-w^{2m+2})\cdots(1-w^{4m-2})\\ &=\prod_{\substack{k=1\\k\neq m}}^{2m-1}(-2i)\mu^{2k}\frac{2k\pi}{4m}\\ &=(-2i)^{2m-2}\mu^{2m(2m-2)}\prod_{k=1}^{2m-1}\sin\frac{k\pi}{2m}\\ &=2^{2m-2}(-1)^{m-1}\cos(m-1)\pi\prod_{k=1}^{2m-1}\sin\frac{k\pi}{2m}\\ &=4^{m-1}\prod_{k=1}^{2m-1}\sin\frac{k\pi}{2m} \end{aligned}$$

于是:
\begin{equation}\label{TriCom_eq3}% \tag{11}
\prod_{k=1}^{2m-1}\sin\frac{k\pi}{2m}=\frac{m}{4^{m-1}}
\end{equation}

代入 $z=-1$ ,得:

$$\begin{aligned} m&=(-1-w^2)\cdots(-1-w^{2m-2})(-1-w^{2m+2})\cdots(-1-w^{4m-2})\\ &=(-1)^{2m-2}(1+w^2)\cdots(1+w^{2m-2})(1+w^{2m+2})\cdots(1+w^{4m-2})\\ &=\prod_{\substack{k=1\\k\neq m}}^{2m-1}2\mu^{2k}\cos\frac{2k\pi}{4m}\\ &=2^{2m-2}\mu^{2m(m-2)}\prod_{k=1}^{m-1}\cos\frac{k\pi}{2m}\prod_{k=m+1}^{2m-1}\cos\frac{k\pi}{2m}\\ &=4^{m-1}\cos(m-1)\pi\prod_{k=1}^{m-1}\cos\frac{k\pi}{2m}\prod_{k=m+1}^{2m-1}\cos\frac{k\pi}{2m}  \end{aligned}$$

又因为 $\displaystyle{\cos\frac{k\pi}{2m}=-\sin\frac{(m+k)\pi}{2m}}$ ,所以:

$$\begin{aligned} m&=4^{m-1}\cos(m-1)\pi\prod_{k=1}^{m-1}\cos\frac{k\pi}{2m}\prod_{k=m+1}^{2m-1}\cos\frac{k\pi}{2m}\\ &=(-1)^{m-1}4^{m-1}\cos(m-1)\pi\prod_{k=1}^{m-1}\sin\frac{k\pi}{2m}\cos\frac{k\pi}{2m}\\ &=2^{m-1}\prod_{k=1}^{m-1}\sin\frac{k\pi}{m} \end{aligned}$$

故:
\begin{equation}\label{TriCom_eq1}% \tag{12}
\prod_{k=1}^{m-1}\sin\frac{k\pi}{m}=\frac{m}{2^{m-1}}
\end{equation}

实际上, \autoref{TriCom_eq1}  不过是\autoref{TriCom_eq2}  和\autoref{TriCom_eq3} 的更一般的形式
