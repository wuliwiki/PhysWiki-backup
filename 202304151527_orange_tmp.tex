% 温度、温标
% 温度|温标|开尔文温标|绝对温度|摄氏度

\pentry{理想气体状态方程\upref{PVnRT}}

\subsection{用理想气体定义}
对于气体而言, 温度越高意味着气体分子速度越大(\autoref{eq_PVnRT_3}~\upref{PVnRT}), 而对于固体, 温度越高说分振动越剧烈。 在理想气体模型中, 我们看到温度与气体分子的平均动能成正比。 
\begin{equation}
\bar E_k = \frac{3}{2} k_B T_K~,
\end{equation}
这样我们就在微观上定义了\textbf{热力学温标}(单位是开尔文, $K$, 国际单位的一种)。 当分子动能为 0 时, 热力学温度就是 $0 \Si{K}$, 即\textbf{绝对零度}。 根据这个定义, 最低的可能温度就是绝对零度\footnote{统计力学中的确有负温度这种说法, 但根据定义, 它的温度反而比任何温度要高。}。

在生活中, 我们一般使用\textbf{摄氏温标}或\textbf{华氏温标}来表示温度, 它们的单位分别记 $^\circ\Si{C}$, $^\circ\Si{F}$, 我们以下用 $T_C$ 和 $T_F$ 表示。 三种温标的转换关系如下
\begin{equation}
T_K = T_C + 273.15^\circ\Si{C}
\end{equation}
\begin{equation}
T_C = \frac{5}{9}(T_F - 32^\circ\Si{F})
\end{equation}
注意一开尔文和一摄氏度的大小一样, 只是相差了一个常数。 零下 $273.15$ 摄氏度就是绝对零度。

\subsection{一般热力学系统的温度,用熵定义}
根据热平衡、热力学第零定律\upref{TherEq},我们知道不止理想气体有温度这一热力学量,任意处于\textbf{平衡态的热力学系统}都有温度这一热力学量。一种可能的定义方式是将它们与一定温度的理想气体进行接触,若它们能够达到平衡(之间没有传热),那么它们就有同样的温度,我们就能利用已知的理想气体的温度来确定未知的系统的温度。这种方式显然是不够通用的,我们需要寻求\textbf{温度}的更加通用的定义。

在热力学和统计力学中,更广义的温度是从熵的角度定义的:
\begin{equation}\label{eq_tmp_1}
\frac{1}{T} = \left(\pdv{S}{U}\right)_{V,N} 
\end{equation}
这可以从平衡态热力学系统的\textbf{内能}全微分表达式(热力学关系式\upref{MWRel})
\begin{equation}
\dd U=T\dd S-p\dd V+\mu \dd N
\end{equation}
看出,$(\partial U/\partial S)_{V,N}=T$,从而得到了\autoref{eq_tmp_1} 的结果。

由此可以推出理想气体的温度定义,根据理想气体的熵:纯微观分析\upref{IdeaS} 的结果。其推导基于玻尔兹曼熵公式,由系统的内能、体积、粒子数推出平衡态时粒子的速率分布及微观状态数,从而得到熵的公式,过程中并没有用到温度概念。考察一个\textbf{粒子数 $N$ 确定}的平衡态理想气体系统(对于粒子数不确定的系统,则需要将全同粒子假设考虑进来并相应地改写熵的表达式,才能解决 Gibbs 佯谬,\autoref{eq_tmp_1} 才是有效的,参考玻尔兹曼分布(统计力学)\upref{MBsta}。),那么它的熵可以表示为
\begin{equation}
S=k\ln \Omega =k\left[ N\ln V+\frac{3}{2}N\ln U \right] +k\ln F\left( m,N,a,b \right) 
\end{equation}
其中 $F(m,N,a,b)$ 是与 $U,V$ 无关的函数。因此对于这个粒子数 $N$ 确定的系统,我们有
\begin{equation}
U\cdot \left(\frac{\partial S}{\partial U}\right)_{V,N}=\frac{3}{2}Nk
\end{equation}
因此由\autoref{eq_tmp_1} 定义的温度满足理想气体状态方程
\begin{equation}
U=\frac{3}{2}NkT
\end{equation}
