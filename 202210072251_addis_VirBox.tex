% VirtualBox 笔记

\begin{issues}
\issueDraft
\end{issues}

\subsection{安装和基本设置}

\begin{itemize}
\item 用 virtualbox 运行 linux, 首先下载 virtualbox 和 ubuntu 安装包.
\item 安装任何操作系统需要先下载安装系统的 iso 文件(windows 可以用 media creator), Ubuntu 直接从官网下载
\item 如果选择系统的时候只出现 32bit 就检查三件事情 1. 搜索 turn windows features on and off, 检查 Hyper-V 是否开启,需要关闭. 2. 在 EUFI 里面检查 Intel Vitualization Technology 和 Intel VT-d Feature 有没有打开,需要打开. 参考 http://www.fixedbyvonnie.com/2014/11/virtualbox-showing-32-bit-guest-versions-64-bit-host-os/#.XNOTTI5KiUk
\item cpu, 内存等是可以在每次启动虚拟机之前设置的
\item 硬盘选 dynamic 的话,无论虚拟硬盘有多大,实际上占的空间只是实际使用的空间.
\item 如果安装完重启的时候卡住, 按 ctrl + C 即可.
\item 至于分辨率只有 1024 的问题, 在命令行安装如下软件, 就可以在 view 菜单中调整分辨率了.
\begin{lstlisting}[language=bash]
sudo apt-get install virtualbox-guest-utils
virtualbox-guest-x11 virtualbox-guest-dkms
\end{lstlisting}
\item 这些包好像在 ubuntu 18.04 上安装不了, 试图从 Virtualbox 的 Device 菜单中选择 Insert Guest Addition CD image, 看是不是同样的功能. 安装完需要重启. 果然, 屏幕尺寸可以调了, 剪切板可以复制了.
\item snapshot 是一个非常重要好用的功能, 在虚拟机列表中右边的菜单中
\item 备份虚拟机, 只需要在虚拟机列表中通过右键菜单找到所在文件夹, 复制即可. 需要恢复, 则在菜单中 Machne->Add 即可.
\item 共享文件夹: 打开虚拟机, 菜单中选 device -> shared folders, 创建即可, Folder Path 是 host 中的 path, 可以下拉选择 Folder Name 就是 path 中最后那个文件夹的名字. shared folder 在 windows guest 中以网络磁盘的形式存在. Mount point 可以指定一个盘符, 如果想要自动分配盘符, 可以选择 automount. 选择 Make permanent, 重启以后也不会消失.
\item 要开启共享剪贴板, 在菜单栏的 Devices 设置. Drag and Drop 的功能传文件最方便. 从虚拟机里面
\item 如果要断网, 在 setting 里面的 Network 将 Cable Connected 取消勾选即可.
\end{itemize}

\subsection{网络}

* 使用 Virtualbox 运行若干个 linux (ubuntu 18 lts) 系统, 并组成 LAN 网络, 以及连接 internet
* 为了节约空间, 只需要一台 desktop 系统, 其他都是 server 系统(没有 GUI 的)即可
* 设置好 ssh 以后, server 开机后无需登录就可以接收 ssh 连接
* 以下 host 表示运行虚拟机的 OS, guest 表示虚拟机中运行的 OS
* clone 虚拟机的时候选择改 mac 地址 (也可以 clone 完了以后在网络设置里面刷新 mac 地址)
* 克隆完后进入每个 guest 在 `/etc/hostname` 修改 hostname, 重启即可
* 参考 https://www.thomas-krenn.com/en/wiki/Network_Configuration_in_VirtualBox
* 看网页上不同 network type 的表, 其中 `NAT network` 满足要求, 但不能从 host 连接虚拟机, 而 `bridged networking` 可以
* `Not attached` 模式, guest 会有一张网卡, 但是没有连接 cable
* `NAT (Network Address Translation)` 模式, 新建虚拟机默认使用这个, guest 可以连接 internet, 但是外部不能访问 guest (例如 guest 上的 web server), 包括 host
* `NAT network` 模式: **我需要的就是这个, 目前用得是这个** 类似于用 router 将虚拟机连起来,并连互联网, host 连不连暂时无所谓. 在选择这个之前需要在 File->Preference 菜单中的 Network 中新建一个网卡(直接按+按钮即可,无需设置), 然后在对每个虚拟机的 `NAT network` 设置里面选这个网卡.
* 完了以后好像还要在 linux 里面设置 `netplan` (ubuntu 18) 或者 `ifupdown` (old ubuntu), 可以参考这里 https://www.linux.com/tutorials/how-use-netplan-network-configuration-tool-linux/
* netplan 配置文件目录为 `cd /etc/netplan/`, 里面可以手动指定 ip 地址
* 首先备份 netplan 的配置文件 `sudo cp 50-cloud-init.yaml 50-cloud-init.yaml.backup`
* 然后配置文件这样设置 (host 和每个 guest 给一个不同的 ip) (注意其中 `enp0s3` 是 ifconfig 的连接名)
```
network:
    ethernets:
        enp0s3:
            dhcp4: false
            addresses: [10.0.2.4/24]
    version: 2
```
* 注意 ubuntu desktop 这么做以后好像连不上 internet, 其实也没关系, 因为 virtualbox 分配的 ip 似乎完全不会变
* 然后用 `sudo netplan try`, 然后 `ifconfig` 看一下 ip 是否变为指定 ip
* 使用 `sudo netplan apply` 使配置生效
* 现在可以试试从一个 ssh, 如 `ssh 10.0.2.5`, 如果可以就大功告成了
* 另外试一下 `apt update` 可不可以连接外网
* 如果上不了外网, `sudo ifconfig enp0s3 down; sudo ifconfig enp0s3 up` 重启网卡即可(如果通过 ssh 链接, 一定不能拆开使用, 否则会断开链接).

## 当前的虚拟机设置 (2022/9/1)
* 用户名一律 `addis`, 密码一律四位生日
* 所有机器之间两两可以用 ssh 用 public key 免密码链接, `ssh s0` (desktop), `ssh s1`, `ssh s2`, `ssh s3` (server)
* 所有的 server 虚拟机的角色相当于云服务器, 如无必要只能从 desktop 用 ssh 远程操作.

### ubuntu18-desktop
* hostname `ubuntu18-desktop`, 角色相当于个人电脑.
* netplan 的设置使用默认, 是否固定 ip 无所谓, 如果修改 netplan 设置貌似就连不上网了, 这是只要改回来, 然后重启就好.

### ubuntu_server1, 2, 3
* hostname 是 `ubuntu18-1`,`ubuntu18-2`,`ubuntu18-3`
* netplan 使用上文的设置, ip 分别为 `10.0.2.101` 到 `10.0.2.103`
* `ubuntu18-1` 上 `~/common/` 文件夹用于在所有服务器之间 `sshfs` 共享(包括 desktop), 已经设置开机自动 mount 到 `/mnt/common/`, 然后 softlink 到 `~/common`.
* ssh 连接的时候外网貌似非常不稳定, 可能是 virtualbox 的问题, 需要不停地 `ifconfit 网卡名 down; ifconfig 网卡名 up` 来重启网卡.
