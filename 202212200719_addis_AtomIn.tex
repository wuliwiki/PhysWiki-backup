% 原子(科普)

\begin{issues}
\issueDraft
\end{issues}

\pentry{电荷(科普)}

以下列出原子和分子的一些基本常识。

\subsection{原子}
\begin{itemize}
\item 日常中的绝大部分物质(固体,液体,气体等)都是由原子组成的。
\item 原子的大小一般在几个 “埃”, 一埃是 $1\e{-10}\Si{m}$ \footnote{$1\e{-10}$ 是科学计数法, 也就是 $0.0\dots01$ 中共有 10 个零, 或者说把 $1.0$ 的小数点向左移动 10 位。 $\Si{m}$ 是长度单位 “米”}。 % 未完成:链接到科学计数法。
\item 原子由一个原子核以及一个或多个电子组成, 电子围绕原子旋转\footnote{这只是一个模糊的说法, 在原子的尺度下, 电子的运动由量子力学描述, 不存在运动轨迹的概念。}。
\item 原子核的尺寸一般在 $1\e{-15}\Si{m}$ 的数量级, 电子还要更小, 所以原子内部是很空旷的。
\item 原子核和电子之间通过库仑力(正负电荷之间的相互吸引力)束缚在一起。 原子核带正电荷, 电子带负电荷。 % 未完成: 库仑力
\item 电子所带的负电荷的大小(去掉负号)叫做\textbf{元电荷(elementary charge)}, 是电荷的一种计量单位。 \item 任何带电物体所带的电荷(无论正负)都是元电荷的整数倍。
\item 原子核由若干质子和中子构成, 它们之间通过\textbf{核力}\footnote{核力又可以进一步分为\textbf{强核力}和\textbf{弱核力}。}束缚在一起。
\item 一个质子携带一个正的元电荷\footnote{即质子和电子所带电荷的大小相同, 但是符号相反。}, 中子不带电。 所以原子核带正电荷, 电荷大小等于元电荷乘以质子的个数。
\item 质子和中子的质量很接近,约为电子质量的一千八百多倍。
\item \textbf{原子质量单位(atomic mass unit)}(定义略)是一种质量单位。 质子和中子的质量约等于一个原子质量单位。 由于电子的质量可以忽略不计, 每个原子的质量都约等于\textbf{原子质量单位}的整数倍, 也就是质子和中子的总个数。
\item 所以质子和中子的总个数叫做原子的\textbf{质量数(mass number)}。
\item \textbf{相对原子质量(relative atomic mass)}是某元素中原子平均质量除以原子质量单位得到的倍数(非整数)。 相对原子质量四舍五入得到的整数就是质量数。
\item 原子中质子和电子的数量相等, 所以原子整体不带电。
\item \textbf{原子序数(atomic number)}是原子中质子的个数, 即电子的个数。 元素周期表按照原子序数排序。
\item 例: 氢原子是 1 号元素, 原子核有一个质子(带一个整的元电荷)和一个电子(带一个负的元电荷)。 氦原子是 2 号元素, 原子核有两个质子和两个电子。
\item \textbf{化学元素(chemical element)}简称\textbf{元素(element)}, 是指具有某个原子序数的一类原子。 元素和原子是近义词, 元素是宏观概念, 只区分种类不谈论个数, 而原子是微观个体, 既区分种类也可以谈论个数。
\item 同一元素按照中子数可以进一步划分为不同的\textbf{核素(nuclide)}, 它们之间互为\textbf{同位素(isotope)}。
\item 例: 氢原子在自然界中有三种同位素: 氕氘氚。 它们都具有 1 个质子和 1 个电子, 但是中子数分别为 0, 1, 2。 相对原子质量分别为 1, 2, 3。
\end{itemize}

\subsection{分子}
\begin{itemize}
\item 分子由若干个原子构成。
\item 原子之间通过电子
\end{itemize}
