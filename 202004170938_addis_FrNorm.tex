% 径向函数的归一化
% 球面波|归一化|球贝赛尔函数

\pentry{球贝赛尔函数\upref{SphBsl}, 平面波的归一化}% 未完成

令一组 scaled 径向函数基底为 $u_{l,m}(k, r) = rR_{l,m}(k, r)$, 若满足
\begin{equation}
\int_0^\infty u^*_{l,m}(k',r) u_{l,m}(k, r) \dd{r} = \delta(k - k')
\end{equation}
则一组完备正交归一的球面波基底为
\begin{equation}
s_{l,m}(k, \bvec r) = \frac{1}{r} u_{l,m}(k, r) Y_{l,m}(\uvec r)
\end{equation}
正交归一条件为
\begin{equation}\ali{
&\quad \braket{s_{l,m}(k, \bvec r)}{s_{l',m'}(k', \bvec r)}\\
&= \int_0^\infty \frac{1}{r} u_{l,m}^*(k, r)  \frac{1}{r} u_{l',m'}(k', r)  r^2 \dd{r} \int Y_{l,m}^*(\uvec r) Y_{l',m'}(\uvec r) \dd{\Omega}\\
&= \delta_{l,l'}\delta_{m,m'}\delta(k-k')
}\end{equation}

\subsection{函数的分波展开}

空间中任意复函数可以用球面波展开为
\begin{equation}\ali{
f(\bvec r) &= \sum_{l,m} \int_0^\infty c_{l,m}(k) \ket{s_{l,m}(k, \bvec r)} \dd{k}\\
&= \frac{1}{r}\sum_{l,m} \qty(\int_0^\infty c_{l,m}(k) u_{l,m}(k, r) \dd{k}) Y_{l,m}(\uvec r)
}\end{equation}
函数在任意基底上的投影为
\begin{equation}\ali{
\braket{s_{l,m}(k, \bvec r)}{f(\bvec r)} &= \sum_{l',m'} \int_0^\infty \dd{k'} c_{l',m'}(k') \braket{s_{l,m}(k, \bvec r)}{s_{l',m'}(k', \bvec r)}\\
& = \sum_{l',m'} \int_0^\infty \dd{k'} c_{l',m'}(k') \delta_{l,l'}\delta_{m,m'}\delta(k-k')\\
& = c_{l,m}(k)
}\end{equation}

令 $\ket{f}$ 的分波展开形式为
\begin{equation}
f(\bvec r) = 
\frac{1}{r}\sum_{l,m} g_{l,m}(r) Y_{l,m}(\uvec r)
\end{equation}
则 $g_{l,m}(r)$ 和 $c_{l,m}(k)$ 有类似傅里叶变换的关系
\begin{equation}
g_{l,m}(r) = \int_0^\infty c_{l,m}(k) u_{l,m}(k, r) \dd{k}
\end{equation}
\begin{equation}
c_{l,m}(k) = \int_0^\infty u_{l,m}^*(k, r) g_{l,m}(r) \dd{r}
\end{equation}

\subsection{计算归一化积分}
要计算归一化积分, 我们来看从负无穷到正无穷的归一化积分. 从零到正无穷的积分只需要除以 2 即可. 由傅里叶变换已知
\begin{equation}
\braket{k'}{k} = \delta(k - k')
\end{equation}
所以
\begin{equation}
\int_{-\infty}^{\infty} \E^{-\I k' x} \E^{\I k x} \dd{x} = 2\pi\braket{k'}{k} = 2\pi\delta(k - k')
\end{equation}
\begin{equation}
\int_{-\infty}^{\infty} \sin(\I k' x) \sin(\I k x) \dd{x}
= \abs{-\I\sqrt{\frac{\pi}{2}}\ket{k} + \I \sqrt{\frac{\pi}{2}}\ket{-k}}^2 = \pi\delta(k - k')
\end{equation}
\begin{equation}
\int_{-\infty}^{\infty} \cos(\I k' x) \cos(\I k x) \dd{x}
= \abs{\sqrt{\frac{\pi}{2}}\ket{k} + \sqrt{\frac{\pi}{2}}\ket{-k}}^2 = \pi\delta(k - k')
\end{equation}
要证明 $\sin(kx)$ 和 $\cos(kx)$ 正交,
\begin{equation}\ali{
&\quad \int_{-\infty}^{\infty} \sin(k'x)^*\cos(kx) \dd{x}\\
& = \int_{-\infty}^{\infty}\qty[\I\sqrt{\frac{\pi}{2}}\bra{k} - \I \sqrt{\frac{\pi}{2}}\bra{-k}] \qty[\sqrt{\frac{\pi}{2}}\ket{k} + \sqrt{\frac{\pi}{2}}\ket{-k}] \dd{x} = 0
}\end{equation}
所以零到正无穷区间中完备正交归一的实波函数分别是
\begin{equation}
\sqrt{\frac{2}{\pi}} \cos(kx) \qquad
\sqrt{\frac{2}{\pi}} \sin(kx) \qquad (k \geqslant 0)
\end{equation}

现在再来看波函数中有相位 $\phi(k)$ 的情况, 例如
\begin{equation}
\int_{0}^{\infty} \sqrt{\frac{\pi}{2}}\sin[k' x + \phi(k')] \sqrt{\frac{\pi}{2}}\sin[k x + \phi(k)] \dd{x}
\end{equation}
结果会不会仍然等于 $\delta(k - k')$ 呢? 我们利用 $\sin(a + b) = \sin a\cos b + \cos a\sin b$ 和已知的归一化积分得到结果为 $\cos[\phi(k) - \phi(k')] \delta(k - k')$. 如果我们假设 $\phi(k)$ 是连续的, 那么结果就是 $\delta(k - k')$.

同理可得, 以上所有基底中给波函数添加相移 $\phi(k)$ (连续函数), 归一化积分不变.

\subsection{归一化的球贝赛尔函数}
由于 $\E^{\I kx}$ 从负无穷到正无穷的归一化积分为 $2\pi\delta(k-k')$, 易得 $\sin(kx)$ 的积分为 $\pi\delta(k-k')$ (先表示成指数形式), 从 0 到正无穷的积分为 $\pi\delta(k-k')/2$.

注意归一化只需要渐进表达式即可(因为局部的不同相对于无穷积分来说可以忽略). 球贝赛尔函数的渐进形式为 $j_l(kr) \to \sin(kr - l\pi/2)/(kr)$, 所以 $kr j_l(kr)$ 从 0 到正无穷的归一化积分同样是 $\pi\delta(k-k')/2$. 所以归一化的球贝赛尔函数为
\begin{equation}
u_{l,m}(k, r) = \sqrt{\frac{2}{\pi}} kr j_l(kr)
\end{equation}
渐进形式为
\begin{equation}
u_{l,m}(k, r) \to  \sqrt{\frac{2}{\pi}} \sin(kr - l\pi/2)
\end{equation}
