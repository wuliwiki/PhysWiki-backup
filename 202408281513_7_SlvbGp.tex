% 可解群
% keys 可解群|solvable group|导出列|换位子|换位子群|derived series|commutator|commutator group|正规序列|次正规序列|normal series|subnormal series|合成序列|composite series
% license Usr
% type Tutor


\pentry{换位子群\nref{nod_CmtGrp}}{nod_3aa9}


可解群在Galois理论中起到关键作用,用于判断代数方程的根式可解性,由此而得名。由于我们现在尚未深入Galois理论,就不讨论何谓“可解”,而仅仅从群结构的角度研究这种群的性质。


\begin{definition}{可解群}\label{def_SlvbGp_1}
给定群$G$,定义$G^{(0)}=G$且对于任意正整数$k$,$G^{(k)}=[G^{(k-1)}, G^{(k-1)}]$。

称
\begin{equation}\label{eq_SlvbGp_1}
G^{(0)}\rhd G^{(1)}\rhd G^{(2)}\rhd \cdots~
\end{equation}
为$G$的\textbf{导出列(derived series)}。

若导出列在有限步内终结于$\{e\}$,或者等价地说,存在非负整数$n$使得$G^{(n)}=\{e\}$,则称$G$\textbf{可解(solvable)}。

\end{definition}


\autoref{def_SlvbGp_1} 的\autoref{eq_SlvbGp_1} 直接使用了正规子群符号$\rhd$,这是由\autoref{the_CmtGrp_2} 保证的。



\autoref{def_SlvbGp_1} 的基础是换位子群的概念,但我们也可以仅用正规子群的概念来定义可解群:




































