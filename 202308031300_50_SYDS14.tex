% 中山大学 2014 年913专业基础(数据结构)考研真题
% keys 中山大学|2014|数据结构|考研
% license Xiao
% type Tutor


\subsection{一、单项选择题(每题2分,共40分)}
1.算法复杂度通常是表达算法在最坏情况下所需要的计算量。一般不用来表达算法复杂度的表达式为( )。 \\
(A). $O(n^2)$  $\qquad$ (B). $O(100)$ \\
(C). $O(nlogn)$ $\qquad$ (D). $O(1.5^2)$

2.数据结构有四类基本结构,不是其四类结构之一的是( )。 \\
(A).集合 $\qquad$ (B).线性结构 \\
(C).存储结构 $\qquad$ (D).树形结构

3.在存储信息过程中,通过对关键字的计算来确定其存储位置的数据结构是( )。 \\
(A). Hash表  $\qquad$  (B).二叉搜索树  \\
(C).链式结构  $\qquad$  (D).顺序结构

4.有关单向链表的正确描述是( )。 \\
(A).在0(1)时间内找到指定的关键字 \\
(B).在插入和删除操作时无需移动链表结点 \\
(C).在0(1)时间内删除指定的关键字 \\
(D).单向链表的存储效率高于数组的存储效率

5.假设Head是不带头结点的双向循环链的头结点指针。判断链表为空的条件是( )。 \\
(A). Head = NULL  \\
(B). Head->next == Head  \\
(C). Head.next = NULL  \\
(D). Head->next = NULL

6.在下列关于“字符串”的陈述中,正确的描述是( )。 \\
(A).字符串- -定有一个结束符 \\
(B).字符串只能用连续存储空间来存储 \\
(C).“空串”与“空白串"是同一个含义 \\
(D).字符串是一种特殊的线性表

7.关于队列的不正确描述是( )。 \\
(A).FIFO \\
(B).可用链表实现动态队列 \\
(C).可访问队列中任何元素 \\
(D).可用动态连续存储空间实现动态队列

8.假设循环队列的长度为QSize,其头、尾下标分别为Front和Rear。在队列不满的情况下,“入队”后相应下标变化的语句为( )。 \\
(A). Rear=Rear+1
(B). Rear=(Rear+1)\%Qsize \\
(C). Front=Front+1 \\
(D). Front=(Front+1)\%QSize

9.用链表来实现堆栈,next是链表结点中的指针字段,Top为栈顶指针。在确定堆栈非空的情况下,出栈的语句是( ), 其中:所有变量都合法定义了,fee(Poin)是释 放指针Point所指向的存储空间。  \\
(A). Top = Top->next; \\
(B). fe(Top); Top = Top->next; \\
(C). Top = Top->next; fee(Top); \\
(D). Pt=Top; Top= Top->next; fee(P);

10.设A[10][10]为一个对称矩阵, 数组下标从[0][0]开始。为了节省存储,将其下三角部分按行存放在一维数组B[0..54]。B[40]所对应的数组元素( )。 \\
(A). A[3][8] \\
(B). A[2][8] \\
(C). A[3][7] \\
(D). A[2][7]

11.若一棵二叉树的后序和中序序列分别是dfebca 和dabfeac,则其先序序列是( )。 \\
(A). abdfec \\
(B). abdefe \\
(C). acbdef \\
(D). acbefd

12.用一维数组来存储满二叉树,若数组下标从0开始,则元素下标为K(k20)的左子结点下标是()C不考虑数组 下标越界问题)。 \\
(A). $2k-1$ \\
(B). $2k$ \\
(C). $2k+1$ \\
(D). $2k+2$

13.假设$T_L$和$T_R$是二叉搜索树T的左右子树,$H(D)$表示树$T$的高度。若树$T$是$AVL$树,则( )。 \\
(A). $H(T_L)-H(T_R)=0$ \\
(B). $H(T_L)-H(T_R)=1$ \\
(C). $H(T_L)-H(T_R)\leqslant1$ \\
(D). $|H(T_L)-H(T_R)|\leqslant1$

14.用邻接矩阵存储有$n$个顶点$(0,..-1)$和$e$条边的无向图$(0\leqslant e\leqslant(n-1)/2)$。在图中没有无向边$(i,j)(0\leqslant i,j\leqslant n-1)$的情况下,增加此边后,修改邻接矩阵的时间复杂度是( )。 \\
(A). $O(1)$ \\
(B). $O(m)$ \\
(C). $O(e)$ \\
(D). $O(n+e)$

15.用邻接矩阵存储有$n$个顶点$(0,1,...,-1)$和e条边的有向图$(0\leqslant e\leqslant n(n-1))$。计算结点$i(0\leqslant i\leqslant n-1)$入度的时间复杂度是( )。 \\
(A). $O(1)$ \\
(B). $O(m)$ \\
(C). $O(e)$ \\
(D). $O(n+e)$

16.下列排序算法中,时间复杂度最差的是( )。 \\
(A).选择排序 \\
(B).归并排序 \\
(C).快速排序 \\
(D).堆排序

17.对$n$个数进行排序时,对基于比较的排序算法,其时间复杂度下界为( )。 \\
(A). $O(n^2)$ \\
(B). $O(logn)$ \\
(C). $O(nlogn)$ \\
(D). $O(n)$

18.用基数(桶)排序算法对仅由字母和数字组成字符串进行排序(不区分字母大小写)时,需要桶的个数是( )。 \\
(A). 10 \\
(B). 26 \\
(C). 36 \\
(D). 62

19.假设有$n$个无序关键字,有关其查找算法的不正确描述是( )。 \\
(A).关键字可存储在数组中 \\
(C).关键字可存储在单向链表中 \\
(C).最坏搜索效率为$O(n)$ \\
(D).平均搜索效率为$O(logn)$

20. 在下列算法中,求连通图的最小生成树算法是( )。 \\
(A). $DFS$算法 \\
(B). $KMP$算法 \\
(C). $Dijkstra$算法 \\
(D). $Kruskal$算法

\subsection{二、解答题(每题10分,共50分)}
1.已知一个无向图的顶点集为${1,2,3,4,5,6,7}$,其邻接矩阵如下所示(0-无边,1-有边)。
\begin{figure}[ht]
\centering
\includegraphics[width=10cm]{./figures/882aa086b1515651.png}
\caption{第二1题图:邻接矩阵} \label{fig_SYDS14_1}
\end{figure}
(1).画出该图的图形; \\
(2).根据邻接矩阵从顶点4出发进行宽度优先遍历(同一个结点的邻接结点按结点编号的大小为序),画出相应的宽度优先遍历树。

2.简单描述求图最小生成树的Prim算法(普里姆算法)的基本思想,并按算法步骤从结点$D$开始,列出图$2$的最小生成树的求解过程。
\begin{figure}[ht]
\centering
\includegraphics[width=8cm]{./figures/6b77a1a9292f4782.png}
\caption{第二2题图} \label{fig_SYDS14_2}
\end{figure}

3.简单叙述合并排序算法(Merge Sort)的基本 思想。按递增顺序对下面所给数值进行排序,并按步骤列出每步排序后的数值序列。假设在排序过程需要划分时,用函数"$\lfloor x \rfloor$"来处理。 \\
待排序的数值序列: 87 56 10 23 44 83 72

4.己知有下列13个元素的散列表:
\begin{figure}[ht]
\centering
\includegraphics[width=12cm]{./figures/08a77a15a75ebb96.png}
\caption{第二4题图} \label{fig_SYDS14_3}
\end{figure}
其散列函数为h(key) = (3key + 5) \% m(m=13),处理冲突的方法为线性探测再散列法,探查序
列为: h=(h(key)+d)\%m,d= 1,2,3, ... m-1。 \\
问:在表中对关键字50和56进行查找时,所需进行的比较次数为多少?依次写出每次计算公式和值。

5. 假设在通信中,字符a, b,c,d, e,f,g出现的频率如下: \\
a: 10\% $\qquad$ b: 12\% $\qquad$ c:7\% $\qquad$ d: 21\% $\qquad$ e:9\% $\qquad$ f 28\% $\qquad$ g: 13\% \\
(1)根据Huffrman算法(赫夫曼算法)画出其赫夫曼树: \\
(2)给出每个字母所对应的赫夫曼编码,规定:结点左分支边上标0,右分支边上标1; \\
(3)计算其加权路径的长度WPL.

\subsection{三、阅读理解题,按空白编号填写相应的C/C++语言语句,以实现函数功能。(每空2分,每题10分,共30分)}
1.假设定义了下面的链式堆栈类,试编写相关成员函数。
\begin{lstlisting}[language=cpp]
struct Node {
    int key;
    Node *next;
    
public:
    Node() { next = NULL; };
};

class Stack {
public:
    Stack0 { Top= NULL;};
    ~Stack0,
    bool Push(const int &key);
    bool Empty() { returm (Top=-NULD); };
    ...
    
private:
    Node *Top;
};
\end{lstlisting}
(1)成员函数 \verb|Push(const int &key)|: 若申请不到存储空间,返回false, 否则,把参数key压进堆栈,并返回true.
\begin{lstlisting}[language=cpp]
bool Stack:Push(const int &key)
{
    Node *Pt = new Node();
    if(__ (1)____ ) return false;
    Pt->key= key;
    ____(2)____;
    ____(3)____;
    return true;
}
\end{lstlisting}
(2)析构函数-Stack():释放堆栈所用的所有存储空间。
\begin{lstlisting}[language=cpp]
Stack:~Stack()
{
    Node *Pt;
    while(___(4)___ ){
        Pt=Top;
        ____(5)____ ;
        delete Pt;
    }
}
\end{lstlisting}
2.假设用不带头结点的单向链表存储一-元多项式,并按指数有序存储(从大到小)。其链表结点的结构定义如下:
\begin{lstlisting}[language=cpp]
typedef struct_PNode {
    int Coef;  // 系数
    int Expn;  // 指数(规定:指数>=0)
    struct_PNode *next;
} PNode;
\end{lstlisting}
(1)函数PNode *Copy(PNode *P1):把P1指向的多项式复制一份,并返回新多项式的首地址(不考虑申请结点内存失败的情况)。
\begin{lstlisting}[language=cpp]
PNode *Copy(PNode *P1)
{
    PNode *Head, *Pt1, *Pt2;
    Head = NULL;
    while (P1 != NULL) {
        Ptl = (PNode *) clloc(1, sizeof(PNode);
        Pt1-> Coef= P1->Coef;
        Ptl> Expn= PI->Expn;
        if(___ (1)____ ) Head = Pt1;
        else Pt2->next= Ptl;
        ____(2)____;
        P1 = P1->next;
    }
    ____(3)____;
}
\end{lstlisting}
(2)函数Time(PNode *P1, PNode P2):计算多项式P1←P1$\times$P2,其中: P2是一个多项式结点。运算过程中,不考虑数值的溢出问题。
\begin{lstlisting}[language=cpp]
void Time(PNode *P1, PNode P2)
{
    PNode *Pt;
    for (Pt= P1; Pt!= NULL; Pt= Pt->next) {
        ____(4)____;
        ____(5)____;
    }
}
\end{lstlisting}
假设: PI 是多项式"$3x^5-2x^3+x^2-10$”的首地址,P2=(-4,2), 即: P2为$-4x^2$.执行Time(P1, P2)后,P1 指向的多项式为: $-12x^2+8x^5-4x^4+40x^2$.

3.假设二叉树$T=<T_L, root_s, T_R>$中叶子数的定义如下: \\
$Leaves(T)=\leftgroup{&0, & T\text{是空树} \\ &1, & T\text{的根结点是叶结点} \\ & Leaves(T_L), Leaves(T_R)) & \text{其他}}$
\\
\begin{lstlisting}[language=cpp]
typedef struct_ BinNode {
int key;
struct_ BinNode *LChild, *RChild;
} BinNode;
\end{lstlisting}
(1)函数Leaves(root)是求以结点root为根的二叉树中的叶子数。
\begin{lstlisting}[language=cpp]
int Leaves(BinNode *root)
{
    if( root == NULL ) return 0;
    if(____(1)____) return 1;
    return(____(2)____);
}
\end{lstlisting}
(2)函数PostOrder(root)是后序遍历以结点root为根的二叉树。
\begin{lstlisting}[language=cpp]
void PostOrder(BinNode *root)
{
    if(__ (3)___ ){
        ____(4)____;
        ____(5)____;
        printf("Key: %d\n", root->key);
    }
}
\end{lstlisting}

\subsection{四、算法设计题(每题15分,共30分)}
用C/C++语言实现下面函数的功能。 \\
1.假设用链表存储集合,空链表示空集。存储集合的链表结点定义如下:
\begin{lstlisting}[language=cpp]
typedef struct_Element {
    int element; // 集合元素
    struct_Element *next;
} Element;    // 集合的结点定义
\end{lstlisting}
例如: A={11,22}, B={}。这些集合的存储链表如下图所示。
\begin{figure}[ht]
\centering
\includegraphics[width=12cm]{./figures/b1180929e193165a.png}
\caption{第四1题图} \label{fig_SYDS14_4}
\end{figure}
(1) Element *Union(Elerment *A, Element *B),其功能是生成集合A和B的并集链表,返回并集链表的头指针(不考虑申请结点失败的情况)。(10 分) \\
(2) void Display(char *Name, Element *A),其功能是显示集合A中的元素列表,其中: Name是集合A的符号名或任何字符串。(5 分)
例如有下列语句:
\begin{lstlisting}[language=cpp]
Element *A, *B, *C;
C = Union(A, B);    // C是集合A和B并集的首地址,集合A和B已按要求存储好
Display("A=",A);    // 输出结果: A={11,22}
Display("Set B:", B);  // 1输出结果: Set B:{}
\end{lstlisting}

2.已知二叉搜索树(Binary Search Tree)或二叉排序树(Binary Sorting Tree)的结点定义如下:
\begin{lstlisting}[language=cpp]
typedef struct_ BSNode {
    int key;
    struct_BSNode *LChild, *RChild; // 左子树的关键字比根的小,右子树的关键字比根的大
} BSNode; 
\end{lstlisting}
编写函数int Insert(BSNode **root, int key),其功能是在以结点*root为根的二叉搜索树中插入关键字key。若插入成功,则返回0。若关键字已存在,则返回1。若申请结点失败,则返回2.
