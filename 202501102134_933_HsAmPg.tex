% 等差数列(高中)
% keys 高中|等差数
% license Usr
% type Tutor

\begin{issues}
\issueDraft
\end{issues}

\pentry{数列(高中)\nref{nod_HsSeFu}}{nod_53a2}



在日常生活中,常会遇到一些有规律的现象。例如,每月存固定金额,或者每天增加相同的步行距离。这些“有规律的增加”可以通过数学语言描述,体现为等差数列的应用。

\subsection*{定义与基本概念}

如果一个数列中,每一项与它的下一项之间的差是固定的,则称该数列为\textbf{等差数列(arithmetic sequence)}。这个固定的差值称为\textbf{公差(common difference)},通常用字母 $d$ 表示。而数列的第一项称为\textbf{首项(first term)},用字母 $a_1$ 表示。

例如,数列 $1, 4, 7, 10, \dots$ 是一个等差数列。每两项之间的差为 $3$,因此公差 $d=3$,首项 $a_1=1$。

\subsection*{等差数列的基本性质}

\begin{enumerate}
    \item \textbf{递推公式}  
    等差数列的每一项(从第二项开始)可以通过上一项加上公差得到:
    \[
    a_{n+1} = a_n + d
    \]

    \item \textbf{通项公式}  
    任意一项可以用首项和公差表示为:
    \[
    a_n = a_1 + (n-1)d
    \]
    其中,$a_n$ 表示数列的第 $n$ 项,$a_1$ 是首项,$d$ 是公差,$n$ 是项数。
\end{enumerate}

\subsection*{例题解析}

\textbf{例题 1}:已知数列的前三项是 $2, 5, 8$,求数列的通项公式。

解:从数列中可得首项 $a_1 = 2$,公差 $d = 3$。将其代入通项公式:
\[
a_n = 2 + (n-1) \times 3 = 3n - 1
\]

\subsection*{等差数列的前 $n$ 项和}

等差数列前 $n$ 项的和可以通过如下方法计算:  
将数列两次排列后相加:
\[
S_n = a_1 + a_2 + \dots + a_n
\]
倒序排列后:
\[
S_n = a_n + a_{n-1} + \dots + a_1
\]
每一列的和都是 $a_1 + a_n$,共有 $n$ 项,因此:
\[
S_n = \frac{n}{2} \times (a_1 + a_n)
\]
将通项公式 $a_n = a_1 + (n-1)d$ 代入,有:
\[
S_n = \frac{n}{2} \times [2a_1 + (n-1)d]
\]

\textbf{例题 2}:求数列 $1, 4, 7, \dots$ 的前 $10$ 项和。

解:已知 $a_1 = 1$, $d = 3$, $n = 10$。代入公式:
\[
S_{10} = \frac{10}{2} \times [2 \times 1 + (10-1) \times 3] = 5 \times (2 + 27) = 145
\]

\subsection*{实际应用}

\begin{enumerate}
    \item \textbf{存钱计划}  
    每月存 $200$ 元,持续 $12$ 个月,最终存款是多少?
    \[
    S_{12} = \frac{12}{2} \times [2 \times 200 + (12-1) \times 0] = 12 \times 200 = 2400
    \]

    \item \textbf{训练计划}  
    每天增加 $2$ 个俯卧撑,从第一天的 $5$ 个开始,累计 $30$ 天总共完成多少个俯卧撑?
    \[
    S_{30} = \frac{30}{2} \times [2 \times 5 + (30-1) \times 2] = 15 \times (10 + 58) = 1020
    \]
\end{enumerate}

\subsection*{拓展与思考}

\begin{itemize}
    \item 如果公差 $d$ 为负,数列的值会逐渐减少,例如 $10, 8, 6, \dots$。
    \item 等差数列与等比数列的区别:等差数列具有加法规律,而等比数列具有乘法规律。
    \item 在直角坐标系中,将等差数列的项数作为横坐标,数列的值作为纵坐标,点的分布呈直线。
\end{itemize}

\subsection*{小结}

等差数列是一种简单且常见的数学模式。掌握等差数列的定义、通项公式和前 $n$ 项和公式,可以帮助解决许多实际问题,进一步感受数学的规律与美妙。

\subsection*{附加题目}

\begin{enumerate}
    \item 已知等差数列 $3, 7, 11, \dots$,求通项公式和前 $20$ 项的和。
    \item 某人每天比前一天多存 $5$ 元,从第一天开始存 $10$ 元,计算第 $30$ 天的存款和总存款金额。
\end{enumerate}


\subsection{定义}

\begin{definition}{等差数列}
如果数列 $\{a_n\}$ 满足对于 $n > 1$ 的所有项,每一项与前一项的差为同一个常数 $d$,则称 $\{a_n\}$ 为\textbf{等差数列(arithmetic sequence)},$d$ 称为$\{a_n\}$的\textbf{公差(common difference)},即等差数列满足递推公式
\begin{equation}
a_{n}=a_{n-1}+d\qquad(n>1)~.
\end{equation}
\end{definition}

特别地,之前提到的常数列是 $d = 0$ 的等差数列。

\subsection{通项}
如果等差数列 $\begin{Bmatrix} a_n \end{Bmatrix}$ 的首项是 $a_1$,公差是 $d$,那么根据等差数列的定义可得
\begin{equation}
\begin{aligned}
&a_1 = a_1,\\
&a_2 = a_1 + d,\\
&a_3 = a_2 + d = a_1 + 2d,\\
&\cdots \\
&a_n = a_{n-1} + d = a_1 + (n - 1)d~.
\end{aligned}
\end{equation}

当 $n = 1$ 时
\begin{equation}
a_1 = a_1 + (1 - 1)d = a_1~,
\end{equation}
也就是说这个公式对 $n = 1$ 同样适用。

综上,等差数列通项公式为
\begin{equation}
a_n = a_1 + (n - 1)d~.
\end{equation}

\textsl{注:这里需要说明一下,带入 $n = 1$ 验算的原因是,我们推算的是 $n > 1$ 时的通项公式,不能说明对首项成立。正如上一节所说,不是所有数列都能写出通项公式,在题目中,经常会出现首项不符合其余项通项公式的情况。}

\subsection{等差中项}
如果在 $a$ 和 $b$ 中间插入一个数 $A$,使 $a,A,b$ 成等差数列,那么 $A$ 叫作 $a$ 与 $b$ 的\textbf{等差中项}.

如果 $A$ 是 $a$ 与 $b$ 的等差中项,那么
\begin{equation}
A - a = b - A~,
\end{equation}
\begin{equation}
A = \frac{a+b}{2}~.
\end{equation}

\textsl{注:等差中项是等差数列的重要考点,大部分考察等差数列的题目都会考察等差中项。}

\subsection{等差数列的数列和}

等差数列求和的思路非常简单,设一个等差数列
\begin{equation}
a_1,a_2,a_3\cdots,a_n~.
\end{equation}
我们将这个数列倒序排列
\begin{equation}
a_n,a_{n-1},\cdots,a_1~.
\end{equation}
则两个数列的和为
\begin{equation}\label{eq_HsAmPg_1}
\begin{aligned}
S &= a_1 + a_2 + \cdots + a_{n-1} + a_n ~,\\
S &= a_n + a_{n - 1} + \cdots + a_2 + a_1~.
\end{aligned}
\end{equation}
由\autoref{eq_HsAmPg_1} 得
\begin{equation}
2S = (a_1+a_n) + (a_2+a_{n-1}) +\cdots (a_n + a_1)~.
\end{equation}
此时我们会意识到,所有括号中的两项都具有相同的等差中项,也就是是说
\begin{equation}
2S = n \cdot (a_1+a_n)~,
\end{equation}
\begin{equation}
S = \frac{n\cdot(a_1+a_n)}{2}~.
\end{equation}

\textbf{等差数列与函数}

求数列和的最小值

若$a_0<0,d>0$,则一定有$a_1<a_2<\dots<a_{k}<0$。

验证$a_{k+1} =0$且$a_{k+m}>0(m\in\mathbb{N}^+)$,或$a_{k+m}>0(m\in\mathbb{N})$。

因此$S_n$的最小値$S_k=S_{k+1}=S$或$S_k=S$