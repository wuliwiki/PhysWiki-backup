% 常用的数学运算
% 常用的数学运算

本文授权转载自郝林的 《Julia 编程基础》. 原文链接:\href{https://github.com/hyper0x/JuliaBasics/blob/master/book/ch05.md}{第 5 章 数值与运算}.


\subsection{5.5 常用的数学运算}

Julia 中的一些操作符可以用于数学运算或位运算(也就是比特运算).这样的操作符也可以被称为运算符.因此,我们就有了数学运算符和位运算符这两种说法.

\subsubsection{5.5.1 数学运算符}

可用于数学运算的运算符请见下表.

\textsl{\textbf{(此处有一个待输入的表格!目前因为编辑器保存时报告错误暂时未能输入)}}
% \begin{table}[ht]
% \centering
% \caption{数学运算符}\label{JuC5S5_tab1}
% \begin{tabular}{|c|c|c|c|}
% \hline
% 运算名称&运算符&示意表达式&用途 \\
% \hline
% 一元加&+&+x&求 x 的原值 \\
% \hline
% 一元减&-&-x&求 x 的相反数,相当于 0 - x \\
% \hline
% 平方根&√&√x&求 x 的平方根 \\
% \hline
% 二元加&+&x + y&求 x 和 y 的和 \\
% \hline
% 二元减&-&x - y&求 x 与 y 的差 \\
% \hline
% 乘&*&x * y&求 x 和 y 的积 \\
% \hline
% 除&/&x / y&求 x 与 y 的商 \\
% \hline
% 逆向除&\&x \ y&相当于 y / x \\
% \hline
% 整除&÷&x ÷ y&求 x 与 y 的商且只保留整数 \\
% \hline
% 求余运算&%&x % y&求 x 除以 y 后得到的余数 \\
% \hline
% 幂运算&^&x ^ y&求 x 的 y 次方 \\
% \hline
% \end{tabular}
% \end{table}

可以看到,Julia 中通用的数学运算符共有 9 个.其中,与\verb|+|和\verb|-|一样,\verb|√|也是一个一元运算符.它的含义是求平方根.在REPL环境中,我们可以通过输入\verb|\sqrt[Tab]|写出这个符号.我们还可以用函数调用\verb|sqrt(x)|来替代表达式\verb|√x|.

所谓的一元运算是指,只有一个数值参与的运算,比如\verb|√x|.更宽泛地讲,根据参与操作的对象的数量,操作符可被划分为一元操作符(unary operator)、二元操作符(binary operator)或三元操作符(ternary operator).其中,参与操作的对象又被称为操作数(operand).

除上述的运算符之外,Julia还有一个专用于\verb|Bool|类型值的一元运算符\verb|!|,称为求反运算符.它会将\verb|true|变为\verb|false|,反之亦然.

这些数学运算符都是完全符合数学逻辑的.所以我在这里就不再展示它们的示例了.

\subsubsection{5.5.2 位运算符}

我们都知道,任何值在底层都是根据某种规则以二进制的形式存储的.数值也不例外.我们把以二进制形式表示的数值简称为二进制数.所谓的位运算,就是针对二进制数中的比特(或者说位)进行的运算.这种运算可以逐个地控制数中每个比特的具体状态(\verb|0|或\verb|1|).

Julia 中的位运算符共有 7 个.如下表所示.

_表 5-5 位运算符_

| 运算名称 | 运算符 | 示意表达式 | 简要说明                                                             |
| :------- | :----- | :--------- | :------------------------------------------------------------------- |
| 按位求反 | ~      | ~x         | 求 x 的反码,相当于每一个二进制位都变反                              |
| 按位求与 | &      | x & y      | 逐个对比 x 和 y 的每一个二进制位,只要有`0`就为`0`,否则为`1`        |
| 按位求或 | `|`    | `x | y`    | 逐个对比 x 和 y 的每一个二进制位,只要有`1`就为`1`,否则为`0`        |
| 按位异或 | ⊻      | x ⊻ y      | 逐个对比 x 和 y 的每一个二进制位,只要不同就为`1`,否则为`0`         |
| 逻辑右移 | >>>    | x >>> y    | 把 x 中的所有二进制位统一向右移动 y 次,并在空出的位上补`0`          |
| 算术右移 | >>     | x >> y     | 把 x 中的所有二进制位统一向右移动 y 次,并在空出的位上补原值的最高位 |
| 逻辑左移 | <<     | x << y     | 把 x 中的所有二进制位统一向左移动 y 次,并在空出的位上补`0`          |

利用`bitstring`函数,我们可以很直观地见到这些位运算符的作用.例如:

```julia
julia> x = Int8(-10)
-10

julia> bitstring(x)
"11110110"

julia> bitstring(~x)
"00001001"

julia> 
```

可以看到,按位求反的运算符`~`会把`x`中的每一个比特的状态都变反(由`0`变成`1`或由`1`变成`0`).这也是 Julia 中唯一的一个只需一个操作数的位运算符.因此,它与前面的`+`和`-`一样,都可以被称为一元运算符.

我们再来看按位求与和按位求或:

```julia
julia> y = Int8(17)
17

julia> bitstring(x)
"11110110"

julia> bitstring(y)
"00010001"

julia> bitstring(x & y)
"00010000"

julia> bitstring(x | y)
"11110111"

julia>
```

我们定义变量`y`,并由它来代表`Int8`类型的整数`17`.`y`的二进制表示是`00010001`.对比变量`x`的二进制表示`11110110`,它们只在左边数的第 4 位上都为`1`.因此,`x & y`的结果就是`00010000`.另一方面,它们只在右数第 4 位上都为`0`,所以`x | y`的结果就是`11110111`.

按位异或的运算符`⊻`看起来很特别.因为在别的编程语言中没有这个操作符.在 REPL 环境中,我们可以通过输入`\xor[Tab]`或`\veebar[Tab]`写出这个符号.我们还可以用函数调用`xor(x, y)`来替代表达式`x ⊻ y`.

我们在前表中也说明了,`x ⊻ y`的含义就是逐个对比`x`和`y`的每一个二进制位,只要不同就为`1`,否则为`0`.示例如下:

```julia
julia> bitstring(x), bitstring(y), bitstring(x ⊻ y)
("11110110", "00010001", "11100111")

julia> 
```

Julia 提供了 3 种位移运算,分别是逻辑右移、算术右移和逻辑左移.下面是演示代码:

```julia
julia> bitstring(x)
"11110110"

julia> bitstring(x >>> 3)
"00011110"

julia> bitstring(x >> 3)
"11111110"

julia> bitstring(x << 3)
"10110000"

julia>
```

在位移运算的过程中,数值的宽度(或者说占用的比特数)是不变的.我们可以把承载一个数值的存储空间看成一条板凳,而数值的宽度就是这条板凳的宽度.现在,有一条板凳承载了`x`变量代表的那个整数,并且宽度是`8`.也就是说,这条板凳上有 8 个位置,可以坐 8 个比特(假设比特是某种生物).

每一次位移,板凳上的 8 个比特都会作为整体向左或向右移动一个位置.在移动完成后,总会有 1 个比特被挤出板凳而没有位置可坐,并且也总会有 1 个位置空出来.比如,如果向右位移一次,那么最右边的那个比特就会被挤出板凳,同时最左边会空出一个位置.没有位置可坐的比特会被淘汰,而空出来的位置还必须引进 1 个新的比特.

好了,我们现在来看从`11110110`到`00011110`的运算过程.后者是前者逻辑右移三次之后的结果.按照前面的描述,在向右移动三次之后,最右边的 3 个比特被淘汰了.因此,这时的二进制数就变为了`11110`.又由于,逻辑右移运算会为所有的空位都填补`0`(状态为`0`的比特),所以最终的二进制数就是`00011110`.

![图 5-2 逻辑右移的示意](images/5-2_逻辑右移的示意.png)
_图 5-2 逻辑右移的示意_

与逻辑右移相比,算术右移只有一点不同,那就是:它在空位上填补的不是`0`,而是原值的最高位.什么叫最高位?其实它指代的就是位置最高的那个比特.对于一个二进制数,最左边的那个位置就是最高位,而最右边的那个位置就是最低位.`x`的值`11110110`的最高位是`1`.因此,在算术右移三次之后,我们得到的新值就是`11111110`.

与右移运算不同,左移运算只有一种.我们把它称为逻辑左移.这主要是因为该运算也会为空位填补`0`.所以,`11110110`经过逻辑左移三次之后就得到了`10110000`.

### 5.5.3 运算同时赋值

Julia 中的每一个二元的数学运算符和位运算符都可以与赋值符号`=`联用,可称之为更新运算符.联用的含义是把运算的结果再赋给参与运算的变量.例如:

```julia
julia> x = 10; x %= 3
1

julia>
```

REPL 环境回显的`1`就是变量`x`的新值.但要注意,这种更新运算相当于把新的值与原有的变量进行绑定,所以原有变量的类型可能会因此发生改变.示例如下:

```julia
julia> x = 10; x /= 3
3.3333333333333335

julia> typeof(x)
Float64

julia> 
```

显然,`x`变量原有的类型肯定是某个整数类型(`Int64`或`Int32`).但更新运算使它的值变成了一个`Float64`类型的浮点数.因此,该变量的类型也随之变为了`Float64`.

所有的更新运算符罗列如下:

```julia
+= -= *= /= \= ÷= %= ^= &= |= ⊻= >>>= >>= <<=
```

前 8 个属于数学运算,后 6 个属于位运算.

### 5.5.4 数值的比较

理所应当,数值与数值之间是可以比较的.在 Julia 中,这种比较不但可以发生在同类型的值之间,还可以发生在不同类型的值之间,比如整数和浮点数.通常,比较的结果会是一个`Bool`类型的值.

对于整数之间的比较,我们就不多说了.它与数学中的标准定义没有什么两样.至于浮点数,相关操作仍然会遵循 IEEE 754 技术标准.这里存在 4 种互斥的比较关系,即:小于(less than)、等于(equal)、大于(greater than)和无序的(unordered).

具体的浮点数比较规则如下:

+ 只要参与比较的两个数值中有一个是 NaN,比较的结果就必然是`false`.因为 NaN 不与任何东西相等,包括它自己.或者说,这种情况下的所有比较关系都是无序的.
+ Inf 等于它自己,并且一定大于除了 NaN 之外的任何数.
+ -Inf 等于它自己,并且一定小于除了 NaN 之外的任何数.
+ 正零(0.0)和负零(-0.0)是相等的.尽管它们在底层存储上是不同的.
+ 其他情况下的有限浮点数比较将会按照数学中的标准定义进行.

Julia 中标准的比较操作符如下表.

_表 5-6 比较操作符_

| 操作符 | 含义       |
| :----- | :--------- |
| ==     | 等于       |
| != ≠   | 不等于     |
| <      | 小于       |
| <= ≤   | 小于或等于 |
| >      | 大于       |
| >= ≥   | 大于或等于 |

注意,对于不等于、小于或等于以及大于或等于,它们都有两个等价的操作符可用.表中已用空格将它们分隔开了.

这些比较操作符都可以用于链式比较,例如:

```julia
julia> 1 < 3 < 5 > 2
true

julia> 
```

只有当链式比较中的各个二元比较的结果都为`true`时,链式比较的结果才会是`true`.注意,我们不要揣测链中的比较顺序,因为 Julia 未对此做出任何定义.

在这些比较操作符当中,我们需要重点关注一下`==`.我们之前使用过一个用于判断相等的操作符`===`.另外,还有一个名叫`isequal`的函数也可以用于判等.我们需要明确这三者之间的联系和区别.

首先,操作符`===`代表最深入的判等操作.我们在前面说过,对于可变的值,这个操作符会比较它们在内存中的存储地址.而对于不可变的值,该操作符会逐个比特地比较它们.

其次是操作符`==`.它完全符合数学中的判等定义.它只会比较数值本身,而不会在意数值的类型和底层存储方式.对于浮点数,这种判等操作会严格遵循 IEEE 754 技术标准.顺便说一句,在判断两个字符串是否相等时,它会逐个字符地进行比较,而忽略其底层编码.

函数`isequal`用于更加浅表的判等.在大多数情况下,它的行为都会依从于操作符`==`.在不涉及浮点数的时候,它会直接返回`==`的判断结果.那为什么说它更加浅表呢?这是因为,对于那些特殊的浮点数值,它只会去比较它们的字面含义.它同样会判断两个 Inf(或者两个 -Inf)是相等的,但也会判断两个 NaN 是相等的,还会判断`0.0`和`-0.0`是不相等的.这些显然并未完全遵从 IEEE 754 技术标准中的规定.下面是相应的示例:

```julia
julia> isequal(NaN, NaN)
true

julia> isequal(NaN, NaN16)
true

julia> isequal(Inf32, Inf16)
true

julia> isequal(-Inf, -Inf32)
true

julia> isequal(0.0, -0.0)
false

julia> 
```

另外,`===`和`isequal`无论如何都会返回一个`Bool`类型的值作为结果.操作符`==`在绝大多数情况下也会如此.但当至少有一方的值是`missing`时,它就会返回`missing`.`missing`是一个常量,也是类型`Missing`的唯一实例.它用于表示当前值是缺失的.

下面的代码展示了上述 3 种判等操作在涉及`missing`时的判断结果:

```julia
julia> missing === missing
true

julia> missing === 0.0
false

julia> missing == missing
missing

julia> missing == 0.0
missing

julia> isequal(missing, missing)
true

julia> isequal(missing, 0.0)
false

julia> 
```

最后,对于不同类型数值之间的比较,Julia 一般会贴合数学上的定义.比如:

```julia
julia> 0 == 0.0
true

julia> 1/3 == 1//3
false

julia> 1 == 1+0im
true

julia> 
```

### 5.5.5 操作符的优先级

Julia 对各种操作符都设定了特定的优先级.另外,Julia 还规定了它们的结合性.操作符的优先级越高,它涉及的操作就会越提前进行.比如:对于运算表达式`10 + 3^2`来说,由于运算符`^`的优先级比作为二元运算符的`+`更高,所以幂运算`3^2`会先进行,然后才是求和运算.

操作符的结合性主要用于解决这样的问题:当一个表达式中存在且仅存在多个优先级相同的操作符时,操作的顺序应该是怎样的.一个操作符的结合性可能是,从左到右的、从右到左的或者未定义的.像我们在前面说的比较操作符的结合性就是未定义的.

下表展示了本章所述运算符的优先级和结合性.上方运算符的优先级会高于下方的运算符.

_表 5-7 运算符的优先级和结合性_

| 操作符                                            | 用途                               | 结合性     |
| :------------------------------------------------ | :--------------------------------- | :--------- |
| + - √ ~ ^                                         | 一元的数学运算和位运算,以及幂运算 | 从右到左的 |
| << >> >>>                                         | 位移运算                           | 从左到右的 |
| `* / \ ÷ % &`                                     | 乘法、除法和按位与                 | 从左到右的 |
| `+ - | ⊻`                                         | 加法、减法、按位或和按位异或       | 从左到右的 |
| == != < <= > >= === !==                           | 比较操作                           | 未定义的   |
| `= += -= *= /= \= ÷= %= ^= &= |= ⊻= >>>= >>= <<=` | 赋值操作和更新运算                 | 从右到左的 |

此外,数值字面量系数(如`-3x+1`中的`x`)的优先级略低于那几个一元运算符.因此,表达式`-3x`会被解析为`(-3) * x`,而表达式`√4x`则会被解析为`(√4) * x`.可是,它与幂运算符的优先级却是相当的.所以,表达式`3^2x`和`2x^3`会被分别解析为`3^(2x)`和`2 * (x^3)`.也就是说,它们之间会依照从右到左的顺序来结合.

对于运算表达式,我们理应更加注重正确性和(人类)可读性.因此,我们总是应该在复杂的表达式中使用圆括号来明确运算的顺序.比如,表达式`(2x)^3`的运算顺序就一定是先做乘法运算再做幂运算.不过,过多的括号有时也会降低可读性.所以我们往往需要对此做出权衡.如有必要,我们可以分别定义表达式的各个部分,然后再把它们组合在一起.