% 卡尔·弗里德里希·高斯(综述)
% license CCBYSA3
% type Wiki

本文根据 CC-BY-SA 协议转载翻译自维基百科\href{https://en.wikipedia.org/wiki/Carl_Friedrich_Gauss}{相关文章}。

\begin{figure}[ht]
\centering
\includegraphics[width=6cm]{./figures/6c9aaaacb17b4d2e.png}
\caption{由克里斯蒂安·阿尔布雷希特·延森绘制的肖像,1840年(由戈特利布·比尔曼复制,1887年)} \label{fig_KRGS_3}
\end{figure}
约翰·卡尔·弗里德里希·高斯(德语:Gauß [kaʁl ˈfʁiːdʁɪç ˈɡaʊs];拉丁语:Carolus Fridericus Gauss;1777年4月30日–1855年2月23日)是德国数学家、天文学家、测量学家和物理学家,对数学和科学的多个领域做出了重要贡献。他自1807年起成为哥廷根天文台台长和天文学教授,直至1855年去世。高斯被广泛认为是历史上最伟大的数学家之一。

在哥廷根大学学习期间,他提出了多个数学定理。高斯以私人学者的身份完成了他的代表作《算术研究》和《天体运动理论》。他给出了代数基本定理的第二个和第三个完整证明,对数论作出了贡献,并发展了二次和三次二次型的理论。

高斯在发现冥王星作为矮行星的工作中发挥了重要作用。他关于受到大行星影响的行星状物体的运动的研究,导致了高斯引力常数和最小二乘法的引入,而高斯在Adrien-Marie Legendre发表之前就已发现了这一方法。高斯与其他学者一起负责了1820年至1844年期间对汉诺威王国的大规模地理测量和弧长测量项目;他是地球物理学的创始人之一,并提出了磁学的基本原理。他实际工作的成果包括1821年发明了太阳能标,1833年发明了磁力计,以及与威廉·爱德华·韦伯一起于1833年发明了第一台电磁电报机。

高斯是第一个发现并研究非欧几里得几何的人,并且他自己创造了这个术语。他还在约160年前就发展了快速傅里叶变换,比约翰·图基和詹姆斯·库利提前了数十年。

高斯拒绝发表未完成的工作,留下了多部未完成的作品,交由后人编辑。他认为学习的过程,而非拥有知识本身,才是最令人享受的。高斯曾坦言自己不喜欢教学,但他的一些学生后来成为了有影响力的数学家,如理查德·德德金德和伯恩哈德·黎曼。
\subsection{传记}  
\subsubsection{青年时期与教育}
\begin{figure}[ht]
\centering
\includegraphics[width=6cm]{./figures/b21e0aea7920ceb9.png}
\caption{布伦瑞克出生之家(在第二次世界大战中被摧毁)} \label{fig_KRGS_1}
\end{figure}
高斯于1777年4月30日出生在布伦瑞克公国(今德国下萨克森州的布伦瑞克市)。他的家庭社会地位相对较低。父亲盖布哈德·迪特里希·高斯(1744-1808)从事过屠夫、砖瓦匠、园丁和丧葬基金的财务工作。高斯曾形容他的父亲是一个正直而受人尊敬的人,但在家中则是粗暴且专横的。他的父亲擅长写作和计算,而高斯的继母多萝西娅几乎是文盲。高斯有一个从父亲第一次婚姻中生的哥哥。

高斯在数学方面是一个天才儿童。当他的小学老师注意到他的智力时,便将他推荐给布伦瑞克公爵。公爵将他送到当地的卡罗林学院学习,并在那里从1792年到1795年学习,埃伯哈德·奥古斯特·威廉·冯·齐默曼是他的老师之一。之后,公爵为他提供了在哥廷根大学学习数学、科学和古典语言的资源,直到1798年为止。高斯的数学教授是亚伯拉罕·戈特尔夫·凯斯特纳,高斯称他为“诗人中的数学大师,数学家中的诗人”,因为凯斯特纳有许多讽刺性诗句。天文学是由卡尔·费利克斯·赛弗教授的,毕业后高斯与赛弗保持通信;奥尔伯斯和高斯在他们的信件中取笑了他。另一方面,高斯对他的物理学老师乔治·克里斯托夫·李希滕贝格和基督教·戈特洛布·海恩的古典学课程给予高度评价,他愉快地参加了海恩的讲座。这段时间的同学包括约翰·弗里德里希·本岑贝格、法尔卡斯·博尔亚伊和海因里希·威廉·布兰德斯。

他可能是一个自学成才的数学学生,因为他独立重新发现了几条定理。1796年,他解决了一个自古希腊以来困扰数学家的几何问题,确定了哪些规则多边形可以通过圆规和直尺作图。这一发现最终使高斯选择了数学而不是语言学作为职业。高斯的数学日记,记录了他从1796年到1814年间的许多数学成果的简短备注,显示出他许多思想的萌芽,这些思想最终成为他数学巨著《算术研究》(1801)的基础。
\begin{figure}[ht]
\centering
\includegraphics[width=6cm]{./figures/ca9e9a3b21b12fba.png}
\caption{高斯在哥廷根作为学生时的住所} \label{fig_KRGS_2}
\end{figure}
\subsubsection{私人学者}  
高斯于1799年获得哲学博士学位,这一事实有时被误称为在哥廷根大学毕业,实际上是应布伦瑞克公爵的特别请求,从赫尔姆施塔特大学(公国唯一的州立大学)毕业的。约翰·弗里德里希·法夫评审了他的博士论文,高斯在没有进一步口试的情况下以缺席方式获得了学位。随后,公爵为他提供了作为私人学者在布伦瑞克的生活费用。高斯因此拒绝了圣彼得堡的俄罗斯科学院和兰茨胡特大学的邀请。后来,公爵在1804年承诺为他在布伦瑞克建立一个天文台。建筑师彼得·约瑟夫·克拉赫设计了初步的方案,但由于拿破仑战争,这些计划被取消:公爵在1806年的耶拿战役中阵亡。次年,公国被废除,高斯的经济支持也随之停止。

当高斯在世纪初的几年里计算小行星轨道时,他与不来梅和利连塔尔的天文界建立了联系,特别是与威廉·奥尔伯斯、卡尔·路德维希·哈丁和弗里德里希·威廉·贝塞尔等人,他们是“天体警察”这一非正式天文学家小组的一部分。该小组的目标之一是发现更多的行星。他们收集了小行星和彗星的数据,为高斯后来在其天文学巨著《天体运动论》(1809年)中发表的轨道研究提供了基础。
\subsubsection{哥廷根大学教授}
\begin{figure}[ht]
\centering
\includegraphics[width=8cm]{./figures/27ae38e6ad98b0b0.png}
\caption{约1800年的旧哥廷根天文台} \label{fig_KRGS_4}
\end{figure}
1807年11月,高斯应召到哥廷根大学,时该校隶属于新成立的西法利亚王国,由 Jérôme Bonaparte 统治,担任天文学教授兼天文台主任,并一直担任此职务直到1855年去世。很快,他就面临了西法利亚政府要求支付两千法郎作为战争贡献的要求,这笔费用他无力支付。奥尔伯斯和拉普拉斯都希望帮助他支付这笔费用,但高斯拒绝了他们的援助。最终,一位来自法兰克福的匿名人士——后来被发现是大公子达尔贝格——支付了这笔款项。

高斯接管了这座已有60年历史的天文台,该天文台由选帝侯乔治二世于1748年建立,建在一座改建过的防御塔上,仪器可用,但部分已过时。选帝侯乔治三世自1802年起原则上批准了新天文台的建设,西法利亚政府也继续进行规划,但高斯直到1816年9月才得以搬到新的工作地点。他获得了新的现代化仪器,包括从Repsold和Reichenbach公司购买的两台经纬仪和从弗劳恩霍夫购买的日心仪。

高斯的科学活动,除了纯数学外,大致可以分为三个阶段:19世纪前二十年以天文学为主,第三十年为测地学,第四十年则主要以物理学,特别是磁学为主。

高斯毫不掩饰自己对讲授学术课程的反感。但自从在哥廷根大学开始学术生涯以来,他一直持续讲授直到1854年。他经常抱怨教学的负担,觉得这是浪费时间。另一方面,他偶尔也会称某些学生才华横溢。他的大多数讲座涉及天文学、测地学和应用数学,仅有三次讲授纯数学的课程。高斯的学生中,有些人成为著名的数学家、物理学家和天文学家,如莫里茨·坎托尔、德德金德、迪尔克森、恩克、古尔德、海涅、克林克费乌斯、库普费尔、利斯廷、莫比乌斯、尼古莱、黎曼、里特、谢林、谢尔克、舒马赫、冯·斯陶特、斯特恩、乌尔辛;在地球科学领域,如萨托留斯·冯·瓦尔特豪森和瓦普厄斯。

高斯没有写过任何教科书,也不喜欢科学内容的普及。他唯一的普及尝试是他关于复活节日期的著作(1800/1802年)和1836年的《地磁学与磁力计》论文。高斯的论文和著作完全用拉丁文或德语发表。他的拉丁文写作风格古典,但使用了一些当代数学家所设定的常见修饰。
\begin{figure}[ht]
\centering
\includegraphics[width=8cm]{./figures/573bd7843f14cb7f.png}
\caption{1816年新的哥廷根天文台;高斯的起居室位于西翼(右侧)} \label{fig_KRGS_6}
\end{figure}
在1808年哥廷根大学的开学讲座中,高斯宣称,天文学的唯一任务是通过强大的微积分工具获得可靠的观察和结果。 在大学期间,他有其他讲师随行,负责他学科的教学工作,完成教育计划,其中包括数学家蒂博(Thibaut)及其讲座、物理学家迈耶(Mayer),以其教科书而闻名,以及自1831年起接替他的魏伯(Weber),在天文台则有哈丁(Harding),他主讲实践天文学。当天文台建成后,高斯住进了新天文台的西翼,哈丁住进东翼。他们曾是朋友,但随着时间的推移,关系疏远,可能是因为——正如一些传记作者推测——高斯希望和哈丁平起平坐,只能是他的助手或观察员。高斯几乎独自使用新的经纬仪,并将它们与哈丁隔离,除了极少数的联合观测。
\begin{figure}[ht]
\centering
\includegraphics[width=6cm]{./figures/d490526cca326556.png}
\caption{威廉·韦伯和海因里希·埃瓦尔德(前排),作为哥廷根七贤的成员} \label{fig_KRGS_7}
\end{figure}
布伦德尔(Brendel)将高斯的天文活动按时间顺序划分为七个阶段,其中自1820年以来被视为“天文活动较低的时期”。新建的、设备齐全的天文台并未像其他天文台那样高效运作;高斯的天文研究具有单人企业的特征,缺乏长期的观察计划,直到哈丁于1834年去世后,大学才为他设立了助手职位。

尽管如此,高斯两次拒绝了通过接受来自1810年和1825年柏林的邀请,成为普鲁士科学院的正式成员,而不承担讲课责任,以及1810年莱比锡大学和1842年维也纳大学的邀请,可能是因为家庭的经济困境。高斯的薪水从1810年的1000里希塔尔升至1824年的2400里希塔尔,晚年他成为大学薪酬最高的教授之一。
\begin{figure}[ht]
\centering
\includegraphics[width=8cm]{./figures/8e5041a341a9bab4.png}
\caption{高斯临终时(1855年)(菲利普·佩特里拍摄的银版照)} \label{fig_KRGS_8}
\end{figure}
在1810年,高斯的同事和朋友弗里德里希·威廉·贝塞尔(Friedrich Wilhelm Bessel)因缺乏学术头衔而在柯尼斯堡大学遇到困境时,高斯为他提供了荣誉博士学位,来自哥廷根大学哲学系。高斯也曾为索菲·热尔曼(Sophie Germain)提供过荣誉学位的推荐,但是在她去世前不久,所以她未能获得该学位。他还成功支持了数学家戈特霍尔德·艾森斯坦(Gotthold Eisenstein)在柏林的工作。

高斯忠诚于汉诺威王室。在威廉四世国王于1837年去世后,新国王厄尔内斯特·奥古斯都(Ernest Augustus)废除了1833年的宪法。七名教授,后来被称为“哥廷根七人”,对此进行了抗议,其中包括他的朋友和合作者威廉·魏伯(Wilhelm Weber)和高斯的女婿海因里希·埃瓦尔德(Heinrich Ewald)。所有人都被解职,其中三人被驱逐,但埃瓦尔德和魏伯得以留在哥廷根。高斯对此争执深感痛心,但认为无法帮助他们。

高斯参与了学术管理:三次被选为哲学系的院长。作为大学遗孀抚恤金基金的负责人,他处理了精算学,并写了一份关于稳定福利的策略报告。他还被任命为哥廷根皇家科学院的院长,担任了九年。

高斯即使在年老体衰、患有痛风和普遍不满的情况下,依然保持着思想上的活跃。1855年2月23日,他因心脏病发作在哥廷根去世;并葬于那里的阿尔巴尼公墓。高斯的女婿海因里希·埃瓦尔德和高斯的亲密朋友兼传记作家沃尔夫冈·萨托里乌斯·冯·瓦尔特斯豪森为他在葬礼上作了悼词。

高斯是一位成功的投资者,通过股票和证券积累了可观的财富,最终财富价值超过15万塔尔。高斯去世后,他的房间里发现了约18,000塔尔的藏款。
\subsubsection{高斯的大脑}  
高斯去世后的第二天,他的大脑被取出、保存并由鲁道夫·瓦格纳进行研究,发现其质量略高于平均值,为1492克(3.29磅)。瓦格纳的儿子赫尔曼,一位地理学家,在他的博士论文中估算了大脑的面积为219,588平方毫米(340.362平方英寸)。2013年,一位来自哥廷根马克斯·普朗克生物物理化学研究所的神经生物学家发现,由于标签错误,高斯的大脑在第一次研究后很快就与几个月后在哥廷根去世的医生康拉德·海因里希·福克斯的大脑混淆。进一步的研究显示,两者的大脑并无显著异常。因此,直到1998年,所有关于高斯大脑的研究(除了鲁道夫和赫尔曼·瓦格纳的首次研究)实际上都指的是福克斯的大脑。
\subsubsection{家庭}
\begin{figure}[ht]
\centering
\includegraphics[width=6cm]{./figures/d1ed993b231bc6d7.png}
\caption{哥萨斯的第二任妻子,威尔赫尔米娜·瓦尔德克} \label{fig_KRGS_5}
\end{figure}
高斯于1805年10月9日在布伦瑞克的圣凯瑟琳教堂与约翰娜·奥斯托夫结婚。他们有两个儿子和一个女儿:约瑟夫(1806–1873)、威尔赫尔米娜(1808–1840)和路易斯(1809–1810)。约翰娜于1809年10月11日去世,路易斯出生一个月后也去世。高斯为孩子们起名时,分别以第一颗小行星的发现者朱塞佩·皮亚齐、威廉·奥尔伯斯和卡尔·路德维希·哈丁的名字命名。

1810年8月4日,高斯与约翰娜的朋友威尔赫尔米娜(米娜)·瓦尔德克结婚,他们有了三个孩子:尤金(后来的尤金)(1811–1896)、威廉(后来的威廉)(1813–1879)和特雷莎(1816–1864)。米娜·高斯在1831年9月12日去世,之前她病重了十多年。此后,特雷莎接管了家庭并照顾高斯直到他去世;她父亲去世后,她嫁给了演员康斯坦丁·斯陶芬诺。她的妹妹威尔赫尔米娜嫁给了东方学者海因里希·埃瓦尔德。高斯的母亲多萝西娅从1817年起住在他家中,直到1839年去世。

长子约瑟夫在还是中学生时,曾在1821年夏天的测量工作中作为助手帮助父亲。短暂在大学学习后,约瑟夫于1824年加入了汉诺威军队,并在1829年再次参与测量工作。1830年代,他负责扩展测量网络到王国西部地区。凭借他的测量资格,他离开了军队,并作为皇家汉诺威国家铁路公司总监从事铁路建设。1836年,他曾在美国研究了几个月的铁路系统。

尤金于1830年9月离开哥廷根,移民到美国,加入军队服役五年。之后,他在美国中西部为美国毛皮公司工作。后来,他移居密苏里州,成为一名成功的商人。威廉娶了天文学家贝塞尔的侄女;随后他搬到密苏里州,开始做农场主,后来在圣路易斯的制鞋业中变得富有。尤金和威廉在美国有许多后代,而留在德国的高斯后代全部来自约瑟夫,因为他的女儿们没有子嗣。
\subsubsection{个性}  
\textbf{学者}
\begin{figure}[ht]
\centering
\includegraphics[width=8cm]{./figures/9d218617e444768f.png}
\caption{学生画他的数学教授:高斯(1795年)画的亚伯拉罕·戈特赫尔夫·凯斯特纳的讽刺画} \label{fig_KRGS_9}
\end{figure}
在19世纪的前二十年,哥萨斯是德国唯一重要的数学家,可以与当时法国的数学领袖相媲美;他的《算术研究》是第一本被翻译成法语的德国数学著作。

哥萨斯“走在新发展前沿”,自1799年起开始有文献记载的研究,凭借丰富的新思想和严谨的证明方法。[71]与之前的数学家如莱昂哈德·欧拉不同,后者让读者参与思考过程,展现一些错误的偏差,哥萨斯则引入了一种全新的风格,直接且完整的解释,避免了让读者理解作者思路的尝试。[73]
\begin{figure}[ht]
\centering
\includegraphics[width=6cm]{./figures/a58c371101fa76f1.png}
\caption{学生画他的数学教授:高斯由他的学生约翰·本尼迪克特·利斯廷(1830年)所画} \label{fig_KRGS_10}
\end{figure}
哥萨斯是第一个恢复了我们在古代所钦佩的严谨证明的方法,而这种方法在前一时期由于对新发展的过度关注而被不当忽视。

——克莱因 1894年,第101页  
然而,他对自己推广了一种完全不同的理想,在一封写给法卡什·博尔亚的信中,他写道:

“不是知识,而是学习的过程;不是拥有,而是到达的过程,才是带来最大享受的。当我已弄清并穷尽一个课题时,我便会离开它,进入黑暗中再度探索。”

——邓宁顿 2004年,第416页  
他死后的论文、科学日记以及他自己教材中的简短注释显示,哥萨斯在很大程度上是通过经验的方式进行工作。[75][76][77]他一生忙碌且充满热情,擅长计算,通常能够以惊人的速度进行计算,大多数情况下不进行精确控制,但通过巧妙的估算来检验结果。[78]尽管如此,他的计算并不总是无误的。[79]他通过使用高效工具来应对庞大的工作量。[80]哥萨斯使用了大量的数学表格,检验它们的精确性,并为个人使用构建了新的表格。[81]他还开发了有效计算的新工具,例如高斯消元法。[82]有趣的是,他通常会进行比实际所需精度更高的计算,并为实际应用准备比实际需要更多小数位的表格。[83]很可能,这种方法为他在数论中发现定理提供了大量的材料。[79][83]
\begin{figure}[ht]
\centering
\includegraphics[width=6cm]{./figures/10a55cdd9806a8ec.png}
\caption{高斯的印章及其座右铭“Pauca sed Matura”(少而精)} \label{fig_KRGS_11}
\end{figure}
高斯拒绝发表他认为不完整或无法经受批评的作品。这种完美主义与他个人印章上的座右铭“Pauca sed Matura”(“少而精”)相一致。许多同事鼓励他公开新的想法,有时如果他拖得太久,认为他应该发表时,他们会对他提出批评。高斯为自己辩护,声称想法的初步发现很容易,但将这些想法整理成一份可公开的成果对他而言是一个艰巨的任务,要么是因为缺乏时间,要么是因为“心境不宁”。尽管如此,他还是在各种期刊上发表了许多紧急内容的简短通讯,同时也留下了相当可观的文献遗产。高斯称数学为“科学之王”,算术为“数学之王”,并且据说曾经认为,要成为一名一流的数学家,必须立即理解欧拉公式作为一个标杆。

在某些情况下,高斯声称某些学者的想法早已在他自己的脑海中。因此,他关于“发现第一,出版第二”的优先权概念与他的科学同时代人有所不同。与他在呈现数学思想时的完美主义相比,他因引用文献时的粗心大意而受到批评。他为自己辩解,认为引用文献必须以非常完整的方式进行,涉及到所有重要的前人作者,这些人不应该被忽视;但这种引用方式需要对科学史有深入了解,并且比他愿意花的时间要多。

\textbf{私人生活}  

高斯去世后不久,他的朋友萨托里乌斯出版了第一本传记(1856年),以相当热情的风格写成。萨托里乌斯将他视为一个平静且具有进取心的人,具有孩子般的谦逊,但也拥有“铁的性格”和坚定不移的精神力量。除了亲近的人圈子,其他人则认为他是一个内敛且难以接近的人,“像一位奥林匹斯的神坐在科学的顶峰上”。他的同时代人一致认为高斯是一个性格复杂的人。他常常拒绝接受赞美。有时他的访客会因他的脾气暴躁而感到不悦,但过了一会儿,他的情绪会发生变化,成为一个迷人且开明的主人。高斯厌恶争论性格的人;他和同事豪斯曼一起反对让尤斯图斯·李比希在哥廷根大学担任教授,“因为他总是参与一些争论”。
\begin{figure}[ht]
\centering
\includegraphics[width=8cm]{./figures/082120e5db2658d6.png}
\caption{高斯1808至1816年的住所在二楼} \label{fig_KRGS_12}
\end{figure}
高斯的一生受到了家庭重大问题的影响。当他的第一任妻子约翰娜在第三个孩子出生后不久突然去世时,他在给已故妻子的最后一封信中表达了自己的悲痛,这封信以古代悲歌的风格写成,是高斯最为个人化的遗存文件。情况在第二任妻子米娜因结核病折磨身体长达13年后愈发严重;他的两个女儿也都患上了同样的疾病。高斯自己很少暗示自己内心的痛苦:在1831年12月的一封信中,他向贝塞尔提到自己是“最惨痛的家庭苦难的受害者”。

由于妻子的病情,两个较小的儿子被送往远离哥廷根的切尔,接受了几年的教育。他的大儿子约瑟夫在服役超过二十年后,最终仅以一名薪水微薄的中尉军官身份结束了军旅生涯,尽管他在测量学方面有相当的造诣。即使结婚后,他依然需要父亲的经济支持。二儿子尤金也具有和父亲相当的计算和语言天赋,但性格活跃且有时反叛。他本想学习语言学,而高斯希望他成为一名律师。尤金在公共场合陷入了债务危机并引发丑闻后,于1830年9月在戏剧性情况下突然离开哥廷根,通过不来梅移民美国。他很快挥霍了所带的钱,之后父亲拒绝再给予任何经济支持。最小的儿子威廉本想从事农业管理工作,但由于难以获得合适的教育,最终也选择了移民。只有高斯最小的女儿特雷莎陪伴在他晚年左右。

在晚年,高斯养成了收集各种数字数据的习惯,是否有用并不重要,比如他家到哥廷根某些地方的路径数,或某人活过的天数;他在1851年12月祝贺洪堡,以此来庆祝洪堡和艾萨克·牛顿死时的年龄相同,计算单位是天数。

与他出色的拉丁语知识相似,他也精通现代语言。在62岁时,他开始自学俄语,很可能是为了理解来自俄罗斯的科学著作,其中包括罗巴切夫斯基关于非欧几何的著作。高斯阅读了古典和现代文学,并能以原文阅读英语和法语作品。他最喜欢的英语作家是沃尔特·斯科特,最喜欢的德国作家是让·保尔。高斯喜欢唱歌,常常去听音乐会。他是一个热衷的报纸读者,在晚年,他每天中午都会去大学的学术沙龙。

高斯对哲学兴趣不大,曾讽刺过“自称形而上学家”的人们,指的是当时的自然哲学派的支持者。

高斯具有“贵族般的,彻底保守的天性”,对人们的智慧和道德缺乏敬重,秉持着“世界愿意被欺骗”的格言。他不喜欢拿破仑及其制度,对一切暴力和革命都感到恐惧。因此,他谴责了1848年革命中的方法,尽管他同意其中的一些目标,如统一德国的理念。至于政治制度,他对宪政制度的评价较低,批评当时的议员们缺乏知识和逻辑错误。

一些高斯的传记作家推测他的宗教信仰。他曾说过“上帝进行算术”,并表示“我成功了——不是因为我努力工作,而是上帝的恩典”。高斯是路德教会的成员,像北德大部分人一样。似乎他并不完全相信所有教义,也不完全字面理解圣经。萨托里乌斯提到高斯的宗教宽容,并认为他“对真理的渴求”和他的正义感受到了宗教信仰的激励。
\subsection{科学工作}  
\subsubsection{代数与数论}  
\textbf{代数基本定理}
\begin{figure}[ht]
\centering
\includegraphics[width=6cm]{./figures/a6872b087560a2dd.png}
\caption{纪念高斯200周年的德国邮票:复平面或高斯平面} \label{fig_KRGS_13}
\end{figure}
在他1799年的博士论文中,高斯证明了代数基本定理,该定理表明每个具有复系数的非恒定一元多项式至少有一个复数根。此前,包括让·勒朗·达朗贝尔在内的数学家曾给出过错误的证明,高斯的论文中批判了达朗贝尔的工作。此后,他又给出了三种其他的证明,最后一种证明于1849年完成,通常被认为是严格的。他的这些尝试在此过程中大大澄清了复数的概念。[110]

\textbf{《算术研究》}

在《算术研究》的序言中,高斯将他在数论上的工作开始时间定为1795年。通过研究费马、欧拉、拉格朗日和勒让德等前辈数学家的工作,他意识到这些学者已经发现了许多他自己才刚刚得到的结论。[111]《算术研究》从1798年开始写作,于1801年出版,巩固了数论作为一门学科,并涵盖了初等数论和代数数论的内容。在书中,他引入了三重横线符号(≡)表示同余,并利用该符号清晰地呈现了模运算。[112]该书讨论了唯一因数分解定理和模n的原根问题。在主要章节中,高斯给出了二次互反律的前两种证明[113],并发展了二元[114]和三元二次型的理论[115]。

《算术研究》包括了高斯二次型合成定理,以及整数作为三个平方和的表示个数的枚举。作为他关于三个平方定理的几乎直接推论,他证明了费马多边形数定理的三角形情形,即当n = 3时的情况。[116]在第五章末尾,高斯给出了一些关于类数的分析结果,这些结果没有证明,显示出高斯在1801年时已经知道了类数公式。[117][118]

在最后一章中,高斯通过将一个几何问题归结为代数问题,给出了用直尺和圆规构造正17边形(17边形)的证明。[119]他展示了,如果正多边形的边数是2的幂次,或是2的幂次与任意数量的不同费马素数的积,则该正多边形是可构造的。在同一章节中,他还给出了某些三次多项式在有限域上的解的个数的结果,这实际上是对椭圆曲线上整数点的计数。[120]一章未完成的第八章在他去世后才在遗留文件中被发现,这些内容是在1797至1799年间完成的。[121][122]

\textbf{进一步的研究}  

高斯的第一个结果之一是1792年通过经验发现的猜想——后来的素数定理——该定理通过使用积分对数估计素数的数量。[123][o]  

当奥尔伯斯在1816年鼓励高斯竞选法国科学院的奖项,证明费马大定理时,高斯拒绝了,因为他对这一问题的评价较低。然而,在他留下的作品中发现了一篇未注明日期的短文,其中包含了对费马大定理在n = 3和n = 5情况下的证明。[125] n = 3的特殊情况早在莱昂哈德·欧拉时期就已被证明,但高斯发展出了一种更加简洁的证明方法,利用了艾森斯坦整数;尽管证明更为一般,但比实整数情况下的证明要简单。[126]  

高斯在1831年通过证明三维空间中球体的最大堆积密度是在球心形成一个立方体面心排列时给出的,从而为解决开普勒猜想做出了贡献。[127]这是他在回顾路德维希·奥古斯特·泽伯关于正三元二次型约化理论的书籍时发现的。[128] 发现泽伯的证明存在一些不足后,他简化了其中的许多论证,证明了中心猜想,并指出该定理等价于规则排列下的开普勒猜想。[129]  

在关于四次剩余的两篇论文(1828年,1832年)中,高斯引入了高斯整数环\(\mathbb{Z}[i]\)
并证明它是一个唯一因子分解域。[130] 他还推广了一些关键的算术概念,如费马小定理和高斯引理。引入这个环的主要目的是提出四次互反律[130]——正如高斯所发现的,复整数环是这类更高互反律的自然背景。[131]  

在第二篇论文中,他陈述了四次互反律的一般法则,并证明了其中几个特例。在1818年他发布的一篇文章中,包含了他对二次互反律的第五和第六次证明,他宣称这些证明的技巧(高斯和)可以应用于证明更高的互反律。[132]
\subsubsection{分析}  
高斯的第一个发现之一是两个正实数的算术-几何平均数(AGM)的概念。[133] 他在1798至1799年间通过兰登变换发现了它与椭圆积分的关系,一篇日记记录了高斯常数与二次椭圆函数的联系,这一结果高斯表示“将无疑打开一个全新的分析领域”。[134] 他还早期开始研究复分析基础的更正式问题,并且从1811年给贝塞尔的信件中可以看出,他已经知道了“复分析基本定理”——柯西积分定理——并理解了在围绕极点积分时复残差的概念。[120][135]

欧拉的五边形数定理,以及他对AGM和二次椭圆函数的其他研究,促使他获得了许多关于雅可比θ函数的结果,[120] 最终在1808年发现了后来的雅可比三重积恒等式,这包括了欧拉定理作为一个特例。[136] 他的工作表明,从1808年起他已经知道椭圆函数的3阶、5阶和7阶模变换。[137][p][q]  

他在遗稿中的几篇数学片段表明,他知道现代模形式理论的部分内容。[120] 在他关于两个复数的多值AGM的研究中,他发现了AGM的无数个值与其两个“最简单值”之间的深刻联系。[134] 在他未发表的著作中,他认识并画出了模群基本域的关键概念草图。[139][140] 高斯的其中一幅此类草图是对单位圆盘的“等边”双曲三角形镶嵌的描绘,这些三角形的所有角度都等于 
\(\pi /4\)。[141]

高斯在分析领域的洞察力的一个例子是他关于圆分割的原则可以应用于分割二次椭圆曲线的神秘评论,这启发了阿贝尔关于二次椭圆分割的定理。[r] 另一个例子是他在《Summatio quarundam serierum singularium》(1811)中关于二次高斯和符号判定的研究,他通过引入二项式系数的q-类比并使用几种原始恒等式对其进行操作,成功解决了这个主要问题,这些恒等式似乎源自他在椭圆函数理论方面的工作;然而,高斯以一种正式的方式表达他的论证,未揭示出其在椭圆函数理论中的来源,直到后来如雅可比和埃尔米特等数学家的工作才揭示了他的论证的核心。[142]

在《Disquisitiones generales circa series infinitam...》(1813)中,他首次系统地处理了广义超几何函数\(F(\alpha, \beta, \gamma, x)\)并证明了当时已知的许多函数都是超几何函数的特例。[143] 这项工作是数学史上首次对无限级数的收敛性进行精确探讨。[144] 此外,它还涉及了作为超几何函数比值而产生的无限连分数,现在被称为高斯连分数。[145]

1823年,高斯因其关于共形映射的论文获得了丹麦学会奖,该论文包含了与复分析领域相关的若干发展。[146] 高斯指出,复平面中的保持角度的映射必须是复分析函数,并使用后来的贝尔特拉米方程证明了在分析曲面上存在等温坐标。该论文以共形映射到球面和旋转椭球体的例子作为结尾。[147]

\textbf{数值分析}  

高斯经常通过归纳法从他通过经验收集的数值数据中推导定理。[77] 因此,使用高效的算法来促进计算对于他的研究至关重要,他在数值分析方面做出了许多贡献,其中包括在1816年发布的高斯求积法。[148]

在1823年写给格尔林的私人信件中,[149] 高斯描述了使用高斯-赛德尔法求解4x4线性方程组的过程——这是一种求解线性方程组的“间接”迭代方法,并推荐它优于通常的“直接消元法”,特别是对于方程数量超过两个的系统。[150]

高斯在1805年计算帕拉斯星和朱诺星的轨道时发明了一种算法,用于计算现在称为离散傅里叶变换(Discrete Fourier Transform, DFT),这比库利和图基在160年后发现的库利-图基快速傅里叶变换(Cooley–Tukey FFT)算法要早。[151] 他将其作为一种三角插值方法进行开发,但相关论文《Theoria Interpolationis Methodo Nova Tractata》直到1876年才出版,这是高斯去世后发布的;这篇论文早于约瑟夫·傅里叶于1807年首次介绍的相关研究。[153]
\subsubsection{时间表}  
继博士论文之后的第一篇出版物是关于复活节日期的确定(1800年),这是一个基础的数学问题。高斯旨在为没有任何教会或天文学年表知识的人提供最便捷的算法,因此避免使用通常需要的术语,如黄金数、复历、太阳周期、星期字母以及任何宗教含义。[154] 传记作者曾推测高斯为何会处理这个问题,但从历史背景来看,或许是可以理解的。自16世纪以来,儒略历被改为格里历在神圣罗马帝国内引起了混乱,并且直到1700年德国才完成这一更替,届时删除了11天的差距,但在复活节日期的计算上,天主教和新教地区之间依然存在差异。1776年的进一步协议统一了不同宗派的计算方式;因此,在像布伦瑞克公国这样的新教国家,1777年的复活节——距高斯出生五周——是第一个按新方法计算的复活节。[155] 更替过程中的公众困扰可能构成了高斯家族对此问题混淆的历史背景(参见章节:轶事)。由于与复活节规定相关,关于逾越节日期的论文随后于1802年发表。[156]
\subsubsection{天文学}
\begin{figure}[ht]
\centering
\includegraphics[width=6cm]{./figures/2e87bd19be233e67.png}
\caption{卡尔·弗里德里希·高斯,1803年,约翰·克里斯蒂安·奥古斯特·施瓦茨画作} \label{fig_KRGS_14}
\end{figure}
1801年1月1日,意大利天文学家朱塞佩·皮亚兹发现了一个新的天体,依据所谓的提图斯–博德定律,他推测这是长期寻找的位于火星和木星之间的行星,并命名为谷神星(Ceres)。[157] 但他只能追踪这个天体短暂的时间,直到它消失在太阳的光辉后。那时的数学工具不足以从有限的数据推算出其重新出现的位置。高斯解决了这个问题,并预测了该天体可能的重新发现位置在1801年12月。最终,弗朗茨·泽维尔·冯·扎赫(Franz Xaver von Zach)于12月7日和31日在哥达,亨利·奥尔伯斯(Heinrich Olbers)于1月1日和2日在不来梅,独立地在预定位置附近发现了这个天体,偏差仅为半度。[158][s]

高斯的方法得出了一一个8次方的方程,其中一个解为地球的轨道已知。接下来,依据物理条件将所寻找的解与其余六个解分开。在这项工作中,高斯使用了他为此目的创造的综合近似方法。[159]

谷神星的发现促使高斯提出了行星小天体在大行星引力干扰下的运动理论,最终于1809年以《天体运动理论——绕太阳运行的圆锥曲线截面》为名发表。[160] 这篇论文引入了高斯引力常数。[33]

自从新的小行星被发现后,高斯便开始研究它们轨道元素的摄动。首先,他用类似拉普拉斯的方法分析了谷神星,但他最喜欢的天体是帕拉斯,因为它具有很大的偏心率和轨道倾角,而拉普拉斯的方法无法应用。高斯使用了自己的工具:算术–几何平均数、超几何函数和插值法。[161] 1812年,他发现帕拉斯与木星有18:7的轨道共振;高斯以密码形式给出了这个结果,并在给奥尔伯斯和贝塞尔的信中才明确其含义。[162][163][t] 在经过多年的工作后,他在1816年完成了这项研究,但没有得出他认为足够的结果。这标志着他在理论天文学领域活动的结束。[165]
\begin{figure}[ht]
\centering
\includegraphics[width=8cm]{./figures/0b1ce75abe7ea7cd.png}
\caption{ Göttingen天文台从西北方向看(由弗里德里希·贝泽曼绘制,约1835年)} \label{fig_KRGS_15}
\end{figure}
高斯对帕拉斯扰动的研究成果之一是《Determinatitio Attractionis...》(1818年),该方法是理论天文学中的一种方法,后来被称为“椭圆环法”。它引入了一个平均概念,其中轨道上的行星被一个虚拟环替代,这个环的质量密度与行星沿相应轨道弧段所需的时间成比例。[166] 高斯展示了如何计算这样一个椭圆环的引力吸引力的方法,这个方法包括多个步骤;其中之一涉及直接应用算术-几何平均(AGM)算法来计算椭圆积分。[167]

虽然高斯的理论天文学贡献到此为止,但他在观察天文学方面的实际活动持续并贯穿了他的整个职业生涯。早在1799年初,高斯就开始研究利用月球视差来确定经度的问题,他为此开发了比当时常用的公式更为便捷的计算方法。[168] 在被任命为天文台台长后,他重视与贝塞尔的天文常数通信。高斯本人提供了关于岁差和天体偏差、太阳坐标以及折射的表格。[169] 他还对球面几何学做出了许多贡献,并在此背景下解决了一些关于星象导航的实际问题。[170] 他发表了大量的天文观测,主要涉及小行星和彗星;他的最后一次观测是1851年7月28日的日全食。[171]