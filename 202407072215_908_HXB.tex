% 哈希表(综述)
% license Usr
% type Wiki

(本文根据 CC-BY-SA 协议转载自原搜狗科学百科对英文维基百科的翻译)

在计算领域中,哈希表(hash map)是一种实现关联数组抽象数据类型的数据结构,这种结构可以将关键码映射到给定值。哈希表使用哈希函数计算桶单元或槽位数组中的索引,从中可以找到所需的给定值。

理想情况下,哈希函数会将每个关键码分配给一个唯一的存储桶单元,但是大多数哈希表设计都使用不完美的哈希函数,这可能会导致哈希冲突,也就是哈希函数会为多个关键码生成相同的索引。这种冲突必须以某种方式解决。

在维度良好的哈希表中,每次查找的平均成本(指令数)与表中存储的记录个数无关。许多哈希表设计还允许任意插入和删除键值对,每次操作的平均成本(摊销[1])为常数。[2][3]

在许多情况下,哈希表比搜索树或任何其他表查找结构平均效率更高。因此,它们被广泛应用于多种计算机软件中,特别是用于关联数组、数据库索引、缓存和集合。

\subsection{ 哈希(散列)法}
散列的概念是将条目(键/值对) 存储分布在桶单元数组中。设定一个关键码,用哈希算法计算一个索引,该索引建议在哪里可以找到条目:

