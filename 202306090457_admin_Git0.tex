% Git 命令行教程

\begin{issues}
\issueDraft
\end{issues}

\pentry{GitHub Desktop 的简单使用\upref{GitHub}}

\footnote{可以参考\href{https://git-scm.com/book/en/v2}{官方教程}。}Git 最初是一个 Linux\upref{Linux0} 上的命令行程序, 也就是说使用者需要在 Linux 系统的命令行中手动输入 \verb|git| 命令和相应的参数进行所有操作。 GitHub Desktop\upref{GitHub} 这样的程序仅仅提供了图形界面(GUI) 但底层仍然通过调用 \verb|git| 命令实现具体操作。

虽然严格来说使用 git 的命令行程序并不需要配合图形界面程序或者 GitHub 等代码托管平台一起使用, 但关于 Git 的一些基本概念和操作我们都在 GitHub Desktop 教程\upref{GitHub} 中介绍了, 这里不再重复。

\subsubsection{安装}
由于 Git 是非常常用的工具, 一般各种 Linux 发行版的包管理程序都可以用一个简单的命令安装。 例如在 Ubuntu 中,使用 \verb|sudo apt install git| 即可。

\subsection{git status}
\begin{itemize}
\item \verb|git status|
\end{itemize}
