% Linux 创建网络文件夹(命令行)

\subsection{SFTP 网盘}
\begin{itemize}
* 参考 https://www.digitalocean.com/community/tutorials/how-to-use-sshfs-to-mount-remote-file-systems-over-ssh
* 首先确保 `ssh` 可以连接, 但这个程序其实和 `ssh` 是独立的, 不适用 `ssh` 的配置文件
* 安装 `sshfs`
* 注意 `sshfs` 在 WSL 下无效, 好失望… 希望 wsl 2 (据说要 2020 年)可以
* 创建需要网盘目录, 一般在 `/mnt/` 目录下, 例如 `/mnt/101`
* `sudo sshfs -o allow_other addis@10.0.2.101:/home/addis /mnt/101` 注意 `addis@` 一定要有, 否则无法进入网盘
* 要弹出网盘, 用 `umount [dir]`, 然后可以手动删掉路径
\end{itemize}

\subsection{永久 mount}
\begin{itemize}
* 参考 https://www.linode.com/docs/networking/ssh/using-sshfs-on-linux/
* 要重启以后自动 mount 网盘(permanent mount, 两边重启都不怕), **一定** 要先用上面 `sshfs` 命令先临时 mount 一次 (否则会出错!), 成功以后用 `sudo vim /etc/fstab` 在配置文件中最后一行加入 `addis@10.0.2.4:/some/path /mnt/folder  fuse.sshfs noauto,x-systemd.automount,_netdev,follow_symlinks,identityfile=/home/addis/.ssh/id_rsa,allow_other,default_permissions,reconnect 0 0`. 可能需要修改的: `addis@10.0.2.4:/some/path /mnt/folder/`, `/home/addis/.ssh/id_rsa`, 其他的什么意思不清楚也不用管
* 现在可以试一试重启, 如果启动失败, 会进入一个安全命令行, 将 `/etc/fstab` 加入的一行删掉即可. 如果成功的话, 重启后进入 `/mnt/folder` 就会自动 mount
\end{itemize}