% 素数定理(综述)
% license CCBYSA3
% type Wiki

本文根据 CC-BY-SA 协议转载翻译自维基百科\href{https://en.wikipedia.org/wiki/Prime_number_theorem}{相关文章}。

在数学中,素数定理(Prime Number Theorem, PNT)描述了素数在正整数中的渐近分布情况。它形式化地表达了一个直观的观点:随着数值的增大,素数变得越来越稀疏,并且精确地量化了这一稀疏现象发生的速度。

该定理由雅克·阿达马(Jacques Hadamard)和夏尔·让·德拉瓦莱-普桑(Charles Jean de la Vallée Poussin)于1896年各自独立证明,所用的方法基于伯恩哈德·黎曼(Bernhard Riemann)引入的一些思想,尤其是黎曼ζ函数。最早发现的分布形式为:$\pi(N) \sim \frac{N}{\log N}$其中,$\pi(N)$ 表示素数计数函数,即不超过 $N$ 的素数个数,$\log N$ 是 $N$ 的自然对数。这意味着对于足够大的 $N$,从不超过 $N$ 的整数中随机选一个数是素数的概率大约为 $1 / \log N$。换句话说,在前 $N$ 个整数中,相邻两个素数之间的平均间隔大约为 $\log N$。

因此,一个最多有 $2n$ 位数的随机整数(当 $n$ 足够大时)成为素数的可能性,大约是一个最多有 $n$ 位数的随机整数的一半。例如,在所有最多有 1000 位的正整数中,大约每 2300 个数中有一个是素数(因为 $\log(10^{1000}) \approx 2302.6$);而在最多有 2000 位的正整数中,大约每 4600 个数中才有一个是素数(因为 $\log(10^{2000}) \approx 4605.2$)。
\subsection{定理的表述}
\begin{figure}[ht]
\centering
\includegraphics[width=8cm]{./figures/17ef6d25124d34a0.png}
\caption{} \label{fig_SDL_1}
\end{figure}