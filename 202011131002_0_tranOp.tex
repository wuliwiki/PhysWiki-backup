% 平移算符
% keys 算符|微分方程|动量算符|波函数

\pentry{算符的指数函数\upref{OpExp}}

\subsection{一维情况}
在一维的情况下, \textbf{平移算符(translation operator)} $T(a)$ 可以把函数\footnote{这里的讨论是一般性的, 所以这里的函数不一定是量子力学中的波函数} $f(x)$ 整体向右平移 $a$ 得到 $f(x - a)$.

假设 $f(x)$ 是无穷阶可导函数, 那么 $f(x - a)$ 关于某点 $x$ 的泰勒级数\upref{Taylor}可以用表示为
\begin{equation}
f(x - a) = \qty[1 + (-a)\pdv{x} + \frac{1}{2!} (-a)^2\pdv[2]{x} + \dots]f(x)
\end{equation}
其中方括号中的部分可以表示为一个指数函数, 即
\begin{equation}
f(x - a) = \exp(-a\pdv{x}) f(x)
\end{equation}

\addTODO{这方法有点奇怪,还是直接从泰勒级数来解释不就好了?}
要获得这个算符, 我们首先可以构造一个一阶微分方程: 对函数关于 $a$ 求偏导得
\begin{equation}
\pdv{a} f(x - a) = -\pdv{x} f(x - a)
\end{equation}
其中 $-\pdv*{x}$ 可以看作是一个算符. 于是有
\begin{equation}
f(x - a) = \exp(-a\pdv{x}) f(x)
\end{equation}

\begin{example}{}
令 $f(x) = x^2$, 现在我们使用平移算符将其向右平移 $a$.
\begin{equation}\ali{
\exp(-a\pdv{x}) x^2 &= \qty(1 - a\pdv{x} + \frac{1}{2!} a^2 \pdv[2]{x} \dots) x^2\\
&= x^2 - 2ax + a^2 = (x - a)^2
}\end{equation}
我们还可以再次使用平移算符,
\begin{equation}\ali{
\exp(-b\pdv{x}) (x - a)^2 &= \qty(1 - b\pdv{x} + \frac{1}{2!} b^2 \pdv[2]{x} \dots) (x - a)^2\\
&= (x - a)^2 - 2b(x - a) + b^2
= (x - a - b)^2
}\end{equation}
这就验证了 $T(b) T(a) = T(a + b)$, 即
\begin{equation}
\exp(-a\pdv{x}) \exp(-b\pdv{x}) = \exp\qty[-(a + b)\pdv{x}]
\end{equation}
\end{example}

在量子力学中, 由于(位置表象下的)动量算符为 $p = -\I\hbar \pdv*{x}$, 平移算符可记为
\begin{equation}
T(x) = \exp(- \I\frac{x p}{\hbar})
\end{equation}

\subsection{三维情况}
将标量算符换成矢量算符即可
\begin{equation}
T(\bvec x) = \exp(-\I \frac{\bvec x \vdot \bvec p}{\hbar})
\end{equation}