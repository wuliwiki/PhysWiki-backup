% 投影和表示 (转载)
% license Usr
% type Art


\subsection{太阳比喻}

在晴朗的日子里出去走一走,我们看不见自己,但却能看见自己的影子。这是很有意思的事情。

如果太阳不是很晒的话,我们可以站在一个空旷平坦的地上观察我们的影子,它和阳光射来的方向相对,在地上留下一个阴影,如果时间早的话,太阳升的不是很“高”,光线会斜斜地在地上投下一个较长的阴影,随着时间的流逝,太阳会沿着自己的轨道在天空中划出一个圆弧,随着太阳的升“高”,阴影会越来越短,当太阳升到最“高”的时候,阴影也最短。

但说高并不精确,我们可以把眼睛眯起,朝太阳的方向看,所谓“高”就是我们要仰起脖子才能“追踪”到太阳,我们仰起脖子的角度越大、太阳越高,我们可以把这个仰角定义为“太阳-观察者”连接线与地面的夹角$\theta$。当这个角度为$90^o$的时候,太阳在天顶,光线垂直地射下来,此时我们在地上的影子会“消失”\footnote{阴影之内没有光线是暗的,而阴影之外会被阳光照亮,光在这里更多地体现出“粒子性”,它以直线传播,绝对不会绕过障碍物。光从$\theta$方向照射到物体上,在地面上留下一个影子,假设物体的高度是$H$,影子的长度将是$H \cdot \frac{\cos \theta}{\sin \theta } = H \cdot \cot \theta$。}。

\begin{figure}[ht]
\centering
\includegraphics[width=6cm]{./figures/2f63172810481ba4.png}
\caption{太阳光⼊射,与竖直⽅向成 α 角。} \label{fig_QMPre3_4}
\end{figure}

有时我们也以竖直的方向为基准,定义太阳光与竖直方向的夹角为$\alpha$($\alpha = \frac{\pi}{2} - \theta $),当$\alpha = 0$时,阳光笔直地照射在地面上,这时照射到单位面积上太阳光的能量最大,当角度$\alpha$逐渐增大时,照射到单位面积上太阳光的能量会变小,变小的比例正比于$\cos \alpha$。
人类走出非洲后,一路向北,先来到中近东,然后扩散到欧洲、亚洲等其它地方。中近东、欧洲、亚洲比非洲的纬度高,太阳会以一个更大的角度$\alpha$照射下来,随着$\alpha$的增大,单位表面积上地球吸收到的能量会减少,气温会随之降低,尤其是夜晚温度会更低。

我们现在都是住在屋子里的,但在远古人类甚至连制造房屋的技术都没有发明,冷了只能去山洞。但山洞里已经有其他动物占领了,比如曾广泛分布于欧洲和中近东各地的洞熊(cave bear)。洞熊的体型庞大,雄性洞熊的体重可高达1吨,可以想象与洞熊争夺山洞的战役是人类走出非洲后碰到的一大挑战。在这个过程中,火的使用是决定性的,因为在各种动物中只有人类不怕火,甚至还学会了使用火,发明了保存火种的方法,甚至制造火种的技术\footnote{维特鲁威在《建筑十书》中说:“远古时候,人类生来就像出没森林、洞穴和丛林中的野兽一样,茹毛饮血,辛苦度日。那时有一个地方,生长着密集繁茂的森林,狂风袭来,树木剧烈摇晃,树枝相互摩擦而起火。住在附近的人们被火焰吓坏了,逃之夭夭。但后来他们凑近时发现,火的热量对人体有极大的好处,他们将原木投入火中,将火种保存下来。”}。

可以想象人类曾长期生活在生有篝火的洞穴里,而这样的一个原始记忆也被用于比喻说理中,比如柏拉图在《理想国》中借用“洞穴”比喻了城邦和知识。

那么我们的洞穴经验是什么样的呢?

首先需要一个封闭的空间,比如在伸手不见五指的夜晚,任何一个房屋都可以是个洞穴,山洞无非也是个封闭的空间。

漆黑的夜晚,我们呆在山洞或封闭的房子里。我们什么都看不见,我们看不见自己,也看不见他人和物体。我们点燃一个火把或蜡烛。人是喜欢光亮的,于是都凑过去,此时我们在墙壁上看到影子,因为火把的光比较弱,反射一次后就基本没亮光了,洞穴中的影子会比阳光下的更显著和夸张,光和影在一起给我们的视觉极大的刺激。

阳光下我们不能清晰地看到物体的轮廓,但在洞穴经验中,阴影和光亮是截然分开的,我们甚至可以想象一个人去描摹阴影的轮廓。

用简单的线条去对象化一个物体是认识活动的开始,比如自我是不可见的,俗话说我们是在别人的眼睛(其实就是镜子)里认识自己的,但在洞穴经验里,我们在墙壁上能直接看见自己的阴影,比如我们可以面对着墙壁,背对着火把,伸出一只手,举过头顶……然后,我看见我面对的那个阴影会同步地作出这种种动作。

这就从视觉经验上把自我对象化了,同时我还能看见别人的阴影和其他物体的阴影……

火把的好处是可以随意移动,要想看清楚什么东西我们只需要把火把拿过来照一照就可以了。这意味着我们可以控制光线行进的方向,我可以让光向上方射,只需要我们把火把放在物体的下方,我们也可以让光向左射,只需要把火把放在物体的右边……

在洞穴中,我们举着火把从各个方向照物体,为的是要看清某物,光从某个方向射过来,我们看到的是光照亮的那个“面”,物体其他面的形象对我们是隐藏的,我们必须移动火把,使光从另外的方向射向物体,这个动作其实就是选择,我们选择从另一个角度“照亮”物体,刚才对我们显现的将隐藏在黑暗里,但新的面,新的形象会对我们显现。

\begin{figure}[ht]
\centering
\includegraphics[width=6cm]{./figures/d272e38bdd758e77.png}
\caption{三视图就是往三个⽅向做投影。} \label{fig_QMPre3_5}
\end{figure}

同时照亮所有的面则需要很多火把,比如我们可以从两个、三个,甚至六个方向上照亮物体。假设物体是三维的,并且假设物体是“透明”的,我们需要至少从三个互相垂直的方向上照亮物体,才能获得对物体的整体认识。这个其实就是工程里的三视图,上视、侧视和前视\footnote{假如物体不是透明的那就很复杂,因为还涉及物体内部构造的问题,即便不考虑内部构造,我们也得假设物体必须是“凸起”的,才能通过六视图获得物体的整体概念。}。

光源(太阳)、物体、阴影也构成一个常见的“认识论比喻”,这就是柏拉图的“太阳喻”。我们能“看”,是因为有光,而光是源自太阳的;光照射在物体上,我们像洞穴中背对着光源的原始人一样只能看到物体的阴影,即物体本身是不对我们显现的,对我们显现的只是物体的阴影。

这里我们的兴趣并不是介绍哲学上的“太阳喻”,我们只是借助这一图像建立量子力学中的“表示概念”。

在量子力学中没有物体,只有量子态,使用狄拉克记号,记作$\left| \alpha \right\rangle$,量子态本身是无法直接被“看”到的。我们需要对量子态建立一个表示,所谓表示就是选择一个观看的方式。

以观看物体为例,就是我们拿着火把以什么样的方式把物体仔细打量一番?比如我们可以选择从$x$,$y$,$z$三个方向上照亮物体,从三个方向照亮物体其实就是把物体对这三个方向做投影。

我们把矢量$V$看做是最简单的物体,往三个方向做投影就是:

\begin{equation}
\text{$x$方向,方向是$e_x$,投影是$V_x$ = $e_x$ $\cdot V$}~
\text{$y$方向,方向是$e_y$,投影是$V_y = e_y \cdot V$}~
\text{$z$方向,方向是$e_z$,投影是$V_z = e_z \cdot V$}~
\end{equation}

我们把量子态想象成一个矢量(态矢量,state vector),它可能有很多“方向”,每个方向都有一个单位矢量,称作基矢,记为$\left| n \right\rangle$。

在量子力学中,态矢量$\left| \alpha \right\rangle $并不直接对应观测值,在这个意义下我们也说我们是“看不见”量子态的。但我们能“看见”态矢量的投影$\left\langle n | \alpha \right\rangle $,根据玻恩的统计解释,一个量子态处在$\left| n \right\rangle$态的几率正比于$\left|  \left\langle n | \alpha \right\rangle  \right|^2 $,我们管$\left\langle n | \alpha \right\rangle$叫几率幅\footnote{我们一般只讨论已经归一化了的量子态,所以就是等于了,即量子态$\left| \alpha \right\rangle$处在$\left| n \right\rangle$态的几率等于$\left\langle n | \alpha \right\rangle^2 $。}。

我们把投影算符$P_n$定义为:$\left| n \right\rangle \left\langle n \right|$,对量子态$\left| \alpha \right\rangle$投影的效果就是获得投影$\left| n \right\rangle \left\langle n | \alpha \right\rangle$。

\begin{figure}[ht]
\centering
\includegraphics[width=6cm]{./figures/07c4087d68d4bb07.png}
\caption{ ⼀个量⼦⼒学版的“太阳比喻”} \label{fig_QMPre3_6}
\end{figure}

这里基矢$\{ \left| n \right\rangle \}$的选取是关键,它决定了观看方式,我们一般是通过构造一组和哈密顿$H$两两相互都对易的算符集$\{  H, A, B, ...  \}$来构造$\{ \left| n \right\rangle \}$的。

这样几率$\left| \left\langle n | \alpha \right\rangle \right|^2 $就有了明确的物理意义:

\begin{equation}
\text{现在我们把$\{ \left| n \right\rangle \}$改写成$\{ \left| E_n , a, b, ... \right\rangle  \}$,几率$\left|  \left\langle E_n, a, b | \alpha \right\rangle  \right|^2 $就是量子态$\left| \alpha \right\rangle $坍缩到态$\left| E_n, a, b \right\rangle$上的几率。}~
\end{equation}

\begin{figure}[ht]
\centering
\includegraphics[width=6cm]{./figures/9900ea0c38088112.png}
\caption{请添加图片标题} \label{fig_QMPre3_7}
\end{figure}

\subsection{狄拉克记号}

下面我们将结合自旋1/2这个例子讨论量子力学中的狄拉克记号(Dirac Notation)。在理论物理学中记号法很重要,合适的记号法会使数学推导清晰简洁并且不容易出错。

狄拉克记号就是括号,括号在英文里是:

\begin{equation}
bracket~
\end{equation}

我们把它表示为:

\begin{equation}
\left\langle  {bra}  \right|  {c} \left|   {ket}  \right\rangle~
\end{equation}

左边括号的“尖头”是向左的,我们称$\left\langle  {  }  \right|$是左矢空间(bra space)中的一个矢量(简称“左矢”),右边的括号的“尖头”是向右的,我们称$\left|  { }  \right\rangle $是右矢空间(ket space)中的一个矢量(简称“右矢”)。左矢和右矢中间夹着一个“内容”(content),这个内容是算符(operator),我们一般用大写字母$A$表示算符。

\begin{figure}[ht]
\centering
\includegraphics[width=6cm]{./figures/0209b0d5b21f90af.png}
\caption{⼀个卡通化的“bra”和“cat”,即“胸罩”和“咪咪”。这个图对应的是“内积”。} \label{fig_QMPre3_8}
\end{figure}

\subsubsection{向量、左矢和右矢}

针对自旋1/2的量子态,我们建立如下映射/表示关系:
\begin{equation}
|+\rangle \equiv \begin{pmatrix}1 \\ 0 
\end{pmatrix}\qquad \\\\ 
|-\rangle \equiv \begin{pmatrix}0\\1
\end{pmatrix}~
\end{equation}
\begin{equation}
\end{equation}

这里我们把$\left| z \pm \right\rangle$简记为$\left| \pm \right\rangle$,$\left| \pm \right\rangle$是互相排斥的同时完备的两个分类“标准”,任意的一个二维列向量$\left| \alpha \right\rangle$可以表示为它们的叠加:

\begin{equation}
|\alpha\rangle = a|+\rangle + b|-\rangle = \begin{pmatrix}
a \\ b \end{pmatrix}~
\end{equation}

这里的叠加系数 $a, b$ 是复数 (complex number)。在量子力学中我们用态矢量 $|\alpha\rangle$ 表示系统的一个态。复数 $c$ 是所调对易数,它可以随便出现在列向量 (态) 的左边或者右边,或者我们把它和 $\alpha$ 写在一起放在括号里:

\begin{equation}
c|\alpha\rangle = |\alpha\rangle c = |c\alpha\rangle~
\end{equation}

假设有两个向量$\left|  \alpha \right\rangle$、$\left| \beta \right\rangle$,我们想知道这两个向量的相似程度。如果在笛卡尔空间,我们让$\left|  \alpha \right\rangle$向$\left|  \beta \right\rangle$投影,记做:

\begin{equation}
\left|  \beta \right\rangle \left\langle \beta | \alpha  \right\rangle~
\end{equation}

投影之后,向量在$\left| \beta \right\rangle$方向上,同时大小变成$\left\langle \beta | \alpha  \right\rangle$,即$\left| \alpha \right\rangle$在$\left| \beta \right\rangle$方向上的投影是$\left\langle \beta | \alpha  \right\rangle$倍的$\left| \beta \right\rangle$。

$\left\langle \beta | \alpha  \right\rangle$是个数,假如:

\begin{equation}
\left| \alpha \right\rangle = \left( \begin{array}{ccc} a \\ b \end{array} \right)~ \\
\left| \beta \right\rangle = \left( \begin{array}{ccc} c \\ d \end{array} \right)~
\end{equation}

$\left\langle \beta | \alpha  \right\rangle$被定义为:

\begin{equation}
\left\langle \beta | \alpha  \right\rangle = \left( c^* , d^* \right) \left( \begin{array}{ccc}  a \\ b  \end{array}  \right) = c^* a + d^* b~
\end{equation}