% 泊松分布

\pentry{自然对数底(简明微积分)\upref{E}, 随机变量、概率密度函数\upref{RandF}}

在一段时间内, 若一个事件在接下来的任意时刻都有均等的几率发生, 且发生次数不限, 那么在一段时间内发生 $k$ 次的概率为
\begin{equation}
f(\lambda, k) = \frac{\E^{-\lambda}\lambda^k}{k!} \qquad (k = 0,1,2,\dots)
\end{equation}
其中 $\lambda$ 是一个常参数, 该分布的平均值和方差都是 $\lambda$.

具体而言, 若在一段极小时间 $\Delta t$ 内发生的概率为 $\Delta P$, 那么令 $\alpha = \dv*{P}{t}$, 若考察的时长为 $T$, 那么
\begin{equation}
\lambda = \alpha T
\end{equation}


证明方差需要使用
\begin{equation}
\begin{aligned}
\sum_{k=0}^\infty \frac{k^2\lambda^{k}}{k!}
&= \sum_{k=0}^\infty \frac{(k+1)\lambda^{k+1}}{k!}
= \lambda \sum_{k=0}^\infty \frac{(k+1)\lambda^k}{k!}\\
&= \lambda\dv{\lambda} \sum_{k=0}^\infty \frac{\lambda^{k+1}}{k!}
= \lambda \dv{\lambda} (\lambda \E^{\lambda})
= \lambda(\lambda+1)\E^{\lambda}
\end{aligned}
\end{equation}

\subsection{推导}
若 $\dv*{P}{t} = \alpha$. 令 $\lambda = \alpha T$, 把 $[0,T]$ 划分成许多小时间段 $\Delta t$, 那么在时刻 $t$ 也就是第 $t/\Delta t+1$ 时间段发生第一次的概率为
\begin{equation}
\Delta P = \lim_{\Delta t\to 0}(1 - \alpha\Delta t)^{t/\Delta t} \alpha\Delta t
= \qty[\lim_{\Delta t\to 0}(1 - \alpha\Delta t)^{1/(-\alpha\Delta t)}]^{-\alpha t} \alpha\Delta t
= \E^{-\alpha t} \alpha \Delta t
\end{equation}
所以第一次出现的时间分布为
\begin{equation}
f(t) = \alpha\E^{-\alpha t}
\end{equation}
在时间 $[0,T]$ 内不发生的概率为
\begin{equation}
P_0(T) = \int_{T}^\infty f(t) \dd{t} = \E^{-\lambda}
\end{equation}
例如一半可能性不发生的时间为 $T_{1/2} = \ln 2/\alpha$.

在时间 $[0,T]$ 内发生一次的概率为
\begin{equation}
P_1(T) = \int_{0}^{T} f(t') P_0(T-t')\dd{t'} = \lambda \E^{-\lambda}
\end{equation}
即假设事件在 $t'$ 时刻发生一次, 在接下来的长度为 $T-t'$ 的时间段内不发生.

在时间 $[0,T]$ 内发生两次的概率为
\begin{equation}
P_2(T) = \int_{0}^{T} f(t') P_1(T-t')\dd{t'} = \frac{\lambda^2}{2}\E^{-\lambda}
\end{equation}
同理递归可得
\begin{equation}
P_{k}(T) = \int_{0}^{T} f(t') P_{k-1}(T-t')\dd{t'}
\end{equation}
\begin{equation}
P_k(T) = \frac{\lambda^k}{k!}\E^{-\lambda}
\end{equation}
证毕.
