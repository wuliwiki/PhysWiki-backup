% 日晷的计算

\begin{issues}
\issueDraft
\end{issues}

\subsection{表盘水平}
一种常见的日晷如\autoref{SunDia_fig1} 所示. 把图片打印后, 把直角三角形的底边垂直固定在半圆的 12 点方向, 然后把半圆水平放于地面, 12 点方向指向正北(地轴北)即可. 此时三角形的斜边, 也就是日晷的指针与地轴平行. 相对于日晷, 太阳绕三角形斜边以每小时 15° (一天 360°)的角速度匀速转动.

该来行日晷的计算网站见\href{https://www.blocklayer.com/sundial.aspx}{这里}.
\begin{figure}[ht]
\centering
\includegraphics[width=13cm]{./figures/SunDia_1.png}
\caption{北纬 30° 的日晷} \label{SunDia_fig1}
\end{figure}

日晷圆盘上的刻度与所在纬度有关, 若将其放在北极或南极, 那么直角三角形的就变为一个和地面垂直的线段, 且表盘上的刻度是均匀的. 相反, 若把这种日晷放在赤道上, 那么三角形的斜边将会与 12 点的刻度共线, 此时这种日晷将失效.

刻度的具体计算并不复杂, 公式为
\begin{equation}
\beta = \tan^{-1}(\sin\alpha \tan \theta)
\end{equation}
其中 $\beta$ 是表盘上某个刻度到 12 点刻度的夹角, $\theta$ 是太阳当前位置和 12 点位置关于地轴(指针)的夹角, $\alpha$ 是当地维度. 可以验证在北极点处 $\alpha = \pi/2$ 时 $\beta = \phi$.

证明: 令\autoref{PrmSol_eq3}~\upref{PrmSol} 中 $\theta_2 = \pi/2$, $\theta_1 = \theta$ 即可.

\subsection{表盘与赤道平行}
另一种在任何纬度都适用的方案是让刻度盘始终与指针保持垂直, 而指针始终指向地轴. 这种日晷的结构相对更复杂, 但表盘上的刻度始终是均匀的, 秩序调整表盘与底座之间的夹角就能适用于不同纬度.
