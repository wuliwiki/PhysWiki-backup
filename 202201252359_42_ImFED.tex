% 一元隐函数的存在及可微定理
% keys 隐函数|存在|可微性|隐函数存在定理

\pentry{隐函数\upref{ImpFun}}
本节将建立保证隐函数单值连续及可微的条件.这由两个定理来保证
\begin{theorem}{存在定理}\label{ImFED_the1}
若:\begin{enumerate}
\item 函数 $F(x,y)$ 在以点 $(x_0,y_0)$ 为中心的长方形邻域
\begin{equation}
\mathcal{D}=[x_0-\delta,x_0+\delta;y_0-\delta',y_0+\delta']
\end{equation}
中有定义且连续;
\item $F(x_0,y_0)=0$ ;
\item 当 $x=x_0$ 时,函数 $F(x,y)$ 随着 $y$ 的增大而单调增大(或单调减小).
\end{enumerate}
那么:
\begin{enumerate}
\item 在点 $(x_0,y_0)$ 的某一邻域内,方程 $F(x,y)=0$ 确定 $y$ 为 $x$ 的单值函数: $y=f(x)$;
\item $f(x_0)=y_0$ ;
\item $f(x)$ 连续.
\end{enumerate}
\end{theorem}

\begin{theorem}{可微定理}
若:
\begin{enumerate}
\item 函数 $F(x,y)$ 在以点 $(x_0,y_0)$ 为中心的长方形领域
\begin{equation}
\mathcal{D}=[x_0-\delta,x_0+\delta;y_0-\delta',y_0+\delta']
\end{equation}
中有定义且连续;
\item 在 $\mathcal{D}$ 中偏导数 $F_x',F_y'$ 存在且连续;
\item $F(x_0,y_0)=0$;
\item $F_y'(x_0,y_0)\neq0$.
\end{enumerate}
那么:
除\autoref{ImFED_the1} 的结论

\end{theorem}