% Crank-Nicolson 算法解一维含时薛定谔方程(Matlab)
% keys 算法|薛定谔方程|数值解|偏微分方程
% license Xiao
% type Tutor

\pentry{薛定谔方程(单粒子一维)\nref{nod_TDSE11}}{nod_4e84}

\footnote{参考 \cite{NR3}。}本文使用\enref{原子单位制}{AU}。 薛定谔方程为
\begin{equation}
-\frac12 \pdv[2]{\psi}{x} + V\psi = \I \pdv{\psi}{t}~.
\end{equation}
传播子作用于波函数为
\begin{equation}
\psi(x,t+\Delta t) = \exp(-\I H \Delta t)\psi(x,t)~.
\end{equation}
用 Crank-Nicolson 或 Caley scheme\footnote{二者是一回事, 见 \cite{NR3} 19.2 节。} 得到的结果是
\begin{equation}\label{eq_CraNic_2}
\qty(1+\frac{\I}{2}\mat H_{n+1}\Delta t)\bvec\psi_{n+1} = \qty(1-\frac{\I}{2}\mat H_n\Delta t)\bvec\psi_n~.
\end{equation}
其中 $\bvec\psi_n$ 是时刻 $t_n$ 的波函数列矢量(已知), $\bvec\psi_{n+1}$ 为时刻 $t_{n+1}$ 的波函数列矢量(未知), $\mat H_n$ 是 $t_n$ 时刻的哈密顿矩阵。

但事实上, 还可以继续减少计算量。 若近似认为\footnote{或者更公平地, 把\autoref{eq_CraNic_5} 中 $\mat H_n$ 改为 $\mat H_{n+1/2}$, 即 $(t_n+t_{n+1})/2$ 时的哈密顿矩阵。} $\mat H_{n+1}\approx \mat H_{n}$, 将\autoref{eq_CraNic_2} 整理后得
\begin{equation}\label{eq_CraNic_5}
\qty(\frac12 + \frac{\I}{4}\mat H_n\Delta t)\qty(\bvec\psi_{n+1}+\bvec\psi_n) = \bvec\psi_n~.
\end{equation}
解这个方程, 再减去 $\bvec \psi_n$ 即可。

\subsection{等间距网格}
对于等间距坐标网格 $x_1,x_2,\dots$, 可以用差分法(\autoref{eq_DerDif_5})计算二阶导数, 表示为矩阵有
\begin{equation}
\mat D_2 = \frac{1}{\Delta x^2}\pmat{-2 & 1 & 0 & 0 & \dots\\
1 & -2 & 1 & 0 & \dots\\
0 & 1 & -2 & 1 & \dots\\
0 & 0 & 1 & -2 & \dots\\
\vdots & \vdots & \vdots & \vdots & \ddots}~.
\end{equation}
那么 $\mat H_n = -\mat D_2/(2m) + \mat V_n$, 其中 $\mat V_n$ 是对角矩阵, 第 $i$ 个对角元为 $V(x_i, t_n)$。 现在, 就可以代入\autoref{eq_CraNic_2} 或\autoref{eq_CraNic_5} 求解了。

\subsection{虚时间法}
\pentry{虚时间法求基态波函数\nref{nod_ImgT}}{nod_f168}
虚时间法用于求解势阱中的基态, 使用虚时间后, \autoref{eq_CraNic_2} 和\autoref{eq_CraNic_5} 分别变为
\begin{equation}
\qty(1+\frac12\mat H\Delta t)\bvec\psi_{n+1} = \qty(1-\frac12\mat H\Delta t)\bvec\psi_n~,
\end{equation}
\begin{equation}
\qty(\frac12 + \frac14\mat H\Delta t)\qty(\bvec\psi_{n+1}+\bvec\psi_n) = \bvec\psi_n~.
\end{equation}

\subsection{代码}
\pentry{Matlab 的稀疏矩阵\nref{nod_MatSpa}}{nod_6a5b}
Matlab 代码如下, 使用\autoref{eq_CraNic_5}, 以及 \enref{Matlab 的稀疏矩阵}{MatSpa}。 势能函数可以在 \verb|V_fun| 中设置, 我们以方势垒为例, 所有参数和 “\enref{高斯波包的方势垒散射数值计算(Matlab)}{FSBsct}” 相同。 不同的是, 由于我们使用迪利克雷边界条件, 波函数到达边界后会发生全反射。 要避免反射, 可以用开放边界条件%\addTODO{}
以及吸收势能。%\addTODO{}

\begin{figure}[ht]
\centering
\includegraphics[width=14cm]{./figures/f80876421ad5b5d8.png}
\caption{运行结果} \label{fig_CraNic_1}
\end{figure}

\begin{lstlisting}[language=matlab, caption=TDSE\_cn1d.m]
% Crank-Nicolson 法解一维薛定谔方程
% 等间距网格,稀疏矩阵

function TDSE_cn1d
% ==== 参数设置 ======
xmin = -80; xmax = 80; Nx = 1000; % x 网格
tmin = 0; tmax = 20; Nt = 400; % 时间网格
Nplot = 10; % 画图步数
ax = [xmin, xmax, -0.5, 0.5];
% 高斯波包
x0 = -17; t0 = 0; m = 1; % 高斯波包的初始时间
p0 = 4; % 初始动量
sig_x = 2; % 初始位置, 波包宽度
% 方势垒 [-L,L], 高 V0
L = 4; V0 = 4;
V = @(x,t) V_fun(x,t,L,V0); % 势能函数
% ===================
close all;
psi_gs = @(x) 1/(2*pi*sig_x^2)^0.25/sqrt(1 + 1i*t0/(2*m*sig_x^2))...
      *exp(-(x-x0-p0*t0/m).^2/(2*sig_x)^2/(1 + 1i*t0/(2*m*sig_x^2)))...
      .*exp(1i*p0*(x-p0*t0/(2*m)));
x = linspace(xmin, xmax, Nx); dx = (xmax-xmin)/(Nx-1);
t = linspace(tmin, tmax, Nt); dt = (tmax-tmin)/(Nt-1);
psi = psi_gs(x).';
% 准备稀疏矩阵 [对角线, 上对角线, 下对角线]
ind1 = [1:Nx, 1:Nx-1, 2:Nx]; % 行标
ind2 = [1:Nx, 2:Nx, 1:Nx-1]; % 列标
% 动能矩阵非零元
T = -1/(2*m)*[-2*ones(1,Nx), ones(1,2*Nx-2)]/dx^2;
A = (1i*dt/4)*T; % 对角元稍后更新
figure; plot(x, real(psi)); hold on;
plot(x, imag(psi));
for it = 1:Nt
    % 更新对角元
    A(1:Nx) = 0.5 + (1i*dt/4)*(T(1:Nx) + V(x, t(it)+dt/2));
    Asp = sparse(ind1, ind2, A, Nx, Nx);
    tmp = Asp \ psi;
    % err = norm(Asp*tmp - psi);
    psi = tmp - psi;
    if mod(it, Nplot) == 0
        clf;
        plot(x, real(psi)); hold on;
        plot(x, imag(psi));
        plot([xmin, -L, -L, L, L, xmax], [0, 0, 0.4, 0.4, 0, 0], 'k');
        axis(ax);
        pause(0.2);
    end
end
end

% 势能函数:方势垒
function ret = V_fun(x,~,L,V0)
ret = zeros(size(x));
ret(x > -L & x < L) = V0;
end
\end{lstlisting}
