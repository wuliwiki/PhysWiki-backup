% 爱德华·泰勒(综述)
% license CCBYSA3
% type Wiki

本文根据 CC-BY-SA 协议转载翻译自维基百科 \href{https://en.wikipedia.org/wiki/Edward_Teller}{相关文章}。

\begin{figure}[ht]
\centering
\includegraphics[width=6cm]{./figures/aa578cdb3e82db6d.png}
\caption{} \label{fig_ADHTL_1}
\end{figure}
爱德华·泰勒(Edward Teller,匈牙利语:Teller Ede,1908年1月15日-2003年9月9日)是一位匈牙利裔美国理论物理学家和化学工程师,因其在氢弹发展中的关键角色而被俗称为“氢弹之父”。他是基于斯坦尼斯瓦夫·乌拉姆设计提出的“泰勒–乌拉姆构型”的共同发明人之一。

泰勒性格激烈,据称“受到百万吨级爆炸梦想的驱使,有救世主情结,展现出专断的行为风格”。\(^\text{[1]}\)他曾设计过一种名为“闹钟模型”的热核炸弹,其爆炸当量高达1000兆吨(即10亿吨TNT),并建议通过船只或潜艇投送。这种武器将具备焚毁一个大陆的能力。\(^\text{[1]}\)

泰勒于1908年出生于奥匈帝国,20世纪30年代移民至美国,是一批被称为“火星人”的匈牙利科学家移民中的一员。他在核物理、分子物理、光谱学以及表面物理等领域作出了诸多贡献。他对恩里科·费米的β衰变理论的拓展,以“伽莫夫–泰勒跃迁”的形式,为该理论的实际应用奠定了重要基础;而“杨–泰勒效应”和“布鲁瑙尔–埃米特–泰勒理论”(Brunauer–Emmett–Teller theory,简称 BET 理论)至今仍以原始形式被广泛应用,是物理与化学领域的核心理论之一。\(^\text{[2]}\)泰勒惯于以基础物理原理思考问题,常常与同行讨论,以突破难题。这一特点在他与斯坦尼斯瓦夫·乌拉姆共同设计可行的热核聚变炸弹方案时表现得尤为明显。然而,后来他在性格上却否定了乌拉姆所起的关键作用。赫伯特·约克指出,泰勒实际上是利用了乌拉姆提出的“压缩与加热启动热核聚变”的基本思想,绘制出他自己的“超级炸弹”方案草图。\(^\text{[1]}\)在乌拉姆提出其方案之前,泰勒原始设想的“经典超级炸弹”本质上是一个通过加热未压缩液态氘以期引发持续热核燃烧的系统。\(^\text{[1]}\)这个设想虽然简单,源自基本物理原理,但泰勒对其执着追求的程度极为强烈,即使被证明是错误的,或已有人指出无法实现,他仍不放弃。为了获得华盛顿对其“超级武器”计划的支持,泰勒提出在“绿色屋行动”中进行一次热核辐射内爆实验,即所谓的“乔治”试验。\(^\text{[1]}\)

泰勒对托马斯–费米理论也作出了重要贡献,该理论是密度泛函理论的前身——密度泛函理论是现代量子力学处理中复杂分子的标准工具之一。1953年,泰勒与尼古拉斯·梅特罗波利斯、阿里安娜·罗森布鲁斯、马歇尔·罗森布鲁斯以及奥古斯塔·泰勒共同发表了一篇论文,该论文成为蒙特卡洛方法在统计力学中的应用、以及贝叶斯统计中马尔可夫链蒙特卡洛(MCMC)研究文献的重要起点。\(^\text{[3]}\)泰勒早期即参与了“曼哈顿计划”,该计划研发了世界上第一枚原子弹。他积极推动热核武器的研发,但融合(聚变)型武器最终是在二战后才出现。他共同创立了劳伦斯利弗莫尔国家实验室,并曾担任该实验室的主任或副主任。然而,由于他在其前上司、洛斯阿拉莫斯实验室负责人J·罗伯特·奥本海默的安全审查听证会上发表了有争议的反面对证言,泰勒遭到了科学界的排斥。

泰勒继续获得美国政府和军方科研体系的支持,尤其是在他倡导发展核能、保持强大核武库以及推进积极的核试验计划方面。在晚年,他提出了许多颇具争议的技术方案,以解决军事和民用问题,例如“战车计划”——利用热核爆炸在阿拉斯加开凿一个人工港口的设想,以及支持罗纳德·里根提出的“战略防御倡议”。泰勒曾获得恩里科·费米奖和阿尔伯特·爱因斯坦奖。他于2003年去世,享年95岁。
\subsection{早年生活与学术起步}
爱德·泰勒于1908年1月15日出生在布达佩斯,当时属于奥匈帝国的领土,他出身于一个犹太家庭。父亲米克沙·泰勒是一位律师,母亲伊洛娜(Ilona,娘家姓Deutsch)是一位钢琴家。\(^\text{[4][5][6]}\)他在布达佩斯就读于明塔文理中学。\(^\text{[7]}\)泰勒是不可知论者。他后来写道:“宗教在我家从来不是个问题,事实上,几乎从未被提起过。我唯一的宗教教育,是因为明塔中学规定所有学生必须修习自己所属宗教的课程。我们家只过一个节日——赎罪日,那天我们全家会一起禁食。但我父亲仍会在每个安息日和所有犹太节日为他的父母祈祷。我所理解的上帝,是:如果他真的存在,那太好了;我们非常需要他,但几千年来我们从未见过他。”\(^\text{[8]}\)泰勒学说话较晚,但从小就对数字产生了浓厚兴趣,经常以心算大数作为娱乐。\(^\text{[9]}\)
\begin{figure}[ht]
\centering
\includegraphics[width=6cm]{./figures/953820095e9928c9.png}
\caption{泰勒的青年时期照片} \label{fig_ADHTL_2}
\end{figure}
泰勒于1926年离开匈牙利前往德国,部分原因是米克洛什·霍尔蒂政权实施的歧视性“入学限额”政策。青少年时期匈牙利的政治动荡与革命,使他心中长期对共产主义和法西斯主义怀有敌意。\(^\text{[10]}\)

1926年至1928年间,泰勒在卡尔斯鲁厄大学学习数学与化学,并获得化学工程理学学士学位。\(^\text{[11][12]}\)他曾表示,使他转向物理学的关键人物是来访教授赫尔曼·马克(Herman Mark)。\(^\text{[13]}\)在聆听马克关于分子光谱学的讲座后,泰勒意识到,是物理学中的全新思想正在根本性地改变化学的前沿。\(^\text{[14]}\)马克是高分子化学领域的专家,而该领域对于理解生物化学至关重要;他还向泰勒介绍了路易·德布罗意等人提出的量子物理领域的最新突破。正是这些讲座激发了泰勒改学物理的强烈动机。\(^\text{[15]}\)当他将这一决定告诉父亲时,父亲感到十分担忧,特地前往学校拜访他并与教授们面谈。化学工程学位可为毕业生在化工企业提供一条稳定且高薪的职业路径,而物理学则没有如此清晰的职业前景。虽然泰勒并未得知父亲与教授们具体谈了什么,但最终的结果是他获得了父亲的许可,得以追随自己成为物理学家的梦想。\(^\text{[16]}\)

随后,泰勒进入慕尼黑大学,师从著名物理学家阿诺德·索末菲学习物理。1928年,他仍是慕尼黑大学的学生时,不幸被一辆有轨电车撞倒,右脚几乎被截断。此后终生,他走路都一瘸一拐,有时还需佩戴义肢。\(^\text{[17][18]}\)他所服用的止痛药干扰了思考能力,于是他决定停药,转而凭借意志力来对抗疼痛。他甚至利用安慰剂效应来自我暗示——让自己相信喝下的是止痛药而不是水。\(^\text{[19]}\)物理学家维尔纳·海森堡曾评价说,泰勒之所以能很好地应对这场事故,靠的并不是单纯的坚忍,而是他精神上的强韧与顽强。\(^\text{[20]}\)
\begin{figure}[ht]
\centering
\includegraphics[width=6cm]{./figures/cab8a66db84879d1.png}
\caption{泰勒1935年入境美国时携带的匈牙利护照。} \label{fig_ADHTL_3}
\end{figure}
1929年,泰勒转学至莱比锡大学,并于1930年在海森堡的指导下获得物理学博士学位。他的博士论文是关于氢分子离子的量子力学处理,属于该领域最早的精确研究之一。同年,他结识了俄国物理学家乔治·伽莫夫和列夫·朗道。此外,捷克物理学家乔治·普拉切克与泰勒的终生友谊,也对他在科学与哲学方面的发展产生了重要影响。正是普拉切克促成了泰勒于1932年前往罗马与恩里科·费米共度的一个暑期,从而使泰勒的科研方向转向核物理。\(^\text{[21]}\)同样在1930年,泰勒前往哥廷根大学进修。由于马克斯·玻恩和詹姆斯·弗兰克的存在,该校当时是世界物理研究的重镇。\(^\text{[22]}\)但随着1933年1月阿道夫·希特勒出任德国总理,德国对犹太人而言变得不再安全。泰勒在国际救援委员会的协助下离开德国。\(^\text{[23]}\)他短暂前往英国,随后移居哥本哈根一年,期间在尼尔斯·玻尔的指导下工作。\(^\text{[24]}\)1934年2月,他与多年的恋人奥古斯塔·玛丽亚·“米齐”(发音为“米茨”)·哈尔卡尼结婚,米齐是他朋友的妹妹。由于米齐信奉加尔文教,婚礼也在一座加尔文教堂中举行。\(^\text{[20][25]}\)1934年9月,泰勒返回英国。\(^\text{[26][27]}\)

米齐曾在匹兹堡留学,她一直希望能重返美国。这个机会在1935年来临——在乔治·伽莫夫的推荐下,泰勒受邀前往美国,成为乔治·华盛顿大学的物理学教授,自此他与伽莫夫共事至1941年。\(^\text{[28]}\)1937年,泰勒在乔治·华盛顿大学提出了“扬-泰勒效应”,这一效应描述了在某些情况下分子的几何结构会发生扭曲;该效应影响金属的化学反应,尤其是在某些金属染料的颜色变化方面具有重要意义。\(^\text{[29]}\)泰勒与赫尔曼·亚瑟·扬(Hermann Arthur Jahn)对该效应进行了纯粹数学物理层面的分析。与此同时,泰勒还与斯蒂芬·布鲁瑙尔(Stephen Brunauer)和保罗·休·埃米特(Paul Hugh Emmett)合作,在表面物理和化学领域作出了重要贡献——他们共同提出了著名的布鲁瑙尔–埃米特–泰勒(BET)等温吸附模型。\(^\text{[30]}\)1941年3月6日,泰勒与米齐正式归化为美国公民。\(^\text{[31]}\)

在乔治·华盛顿大学任职期间,泰勒与伽莫夫每年共同组织“华盛顿理论物理会议”(1935–1947),吸引了众多顶级物理学家参会。\(^\text{[32]}\)

二战爆发后,泰勒希望为战争贡献力量。在知名航空动力学家、同为匈牙利移民的西奥多·冯·卡门(Theodore von Kármán)建议下,泰勒与好友汉斯·贝特(Hans Bethe)合作,共同研究冲击波传播理论。多年后,他们对冲击波后方气体行为的解释,对导弹再入研究的科学家而言具有重要价值。\(^\text{[33]}\)
\subsection{曼哈顿计划}
\subsubsection{洛斯阿拉莫斯实验室}
1942年,泰勒受邀参加由罗伯特·奥本海默在加利福尼亚大学伯克利分校主持的夏季规划研讨会,讨论曼哈顿计划的起始方向——即美国研发第一批核武器的努力。就在几周前,泰勒曾与他的朋友兼同事恩里科·费米会面,探讨核战争的前景。费米漫不经心地提出,也许可以利用核裂变武器引发更大规模的核聚变反应。虽然泰勒最初向费米解释了他认为这种设想行不通的理由,但他很快就被这个可能性所吸引,并对“仅仅制造”一枚原子弹感到无聊——尽管这项工作当时还远未完成。在伯克利的会议上,泰勒将讨论的焦点从裂变武器引向了聚变武器的可能性——他称之为“超级炸弹”,这是氢弹的一种早期设想。\(^\text{[34][35]}\)

芝加哥大学物理系主任阿瑟·康普顿负责协调哥伦比亚大学、普林斯顿大学、芝加哥大学和加州大学伯克利分校的铀研究工作。为了消除分歧和重复,康普顿将科学家们集中调往芝加哥的冶金实验室。\(^\text{[36]}\)尽管泰勒和米茜此时已是美国公民,但由于他们在敌对国家仍有亲属,泰勒最初并未前往芝加哥。\(^\text{[37]}\)1943年初,新墨西哥州洛斯阿拉莫斯实验室开始建设。奥本海默担任该实验室的主任,其任务是设计原子弹。泰勒于1943年3月搬到了那里。\(^\text{[38]}\)在洛斯阿拉莫斯,泰勒因深夜弹钢琴而惹恼了邻居。\(^\text{[39]}\)

泰勒加入了理论(T)部。\(^\text{[40][41]}\)他获得了一个秘密身份,名为“埃德·蒂尔登”(Ed Tilden)。\(^\text{[42]}\)他对自己未被任命为该部门负责人感到不满,该职位最终由汉斯·贝特担任。奥本海默让他研究构建裂变武器的一些非常规方法,比如“自催化”机制,即在核链式反应进行过程中,炸弹的效率会提升,但这一方法被证明并不现实。\(^\text{[41]}\)他还研究了使用氢化铀代替金属铀,但其效率被发现“微不足道甚至更差”。\(^\text{[43]}\)尽管在战争期间氢弹的开发被列为低优先级(因为裂变武器的研制已足够困难),泰勒仍坚持推进自己的聚变武器设想。\(^\text{[40][41]}\)在一次前往纽约的途中,他请玛丽亚·哥珀特-梅耶为他计算“超级炸弹”的相关数据。她证实了泰勒自己的结论:Super 不会奏效。\(^\text{[44]}\)

1944年3月,在泰勒的领导下成立了一个特别小组,负责研究内爆式核武器的数学模型。\(^\text{[45]}\)该小组同样遇到了不少困难。由于泰勒对“超级炸弹”的兴趣,他在内爆计算上的投入不如贝特所期望的那么多。这项任务起初也是低优先级,但随着埃米利奥·塞格雷小组发现钚存在自发裂变现象,内爆式核弹的重要性大大提升。1944年6月,应贝特的请求,奥本海默将泰勒调出T部门,让他直接向自己汇报,领导一个专责于“超级炸弹”的特别小组。他的职位由英国代表团的鲁道夫·佩耶尔斯接替,后者又引入了后来被揭露为苏联间谍的克劳斯·福克斯。[46][44] 当恩里科·费米于1944年9月加入洛斯阿拉莫斯实验室时,泰勒的“超级炸弹”小组并入了费米所领导的F部门。\(^\text{[46]}\)该小组成员包括斯坦尼斯拉夫·乌拉姆、简·罗伯格、杰弗里·丘、哈罗德和玛丽·阿戈夫夫妇,\(^\text{[47]}\)以及玛丽亚·哥珀特-梅耶。\(^\text{[48]}\)

泰勒在核弹研究中作出了重要贡献,尤其是在揭示内爆机制方面。他是第一个提出使用实心核心设计(的科学家,该设计最终被证明是成功的。这种设计后来被称为“克里斯蒂核心”,以实现它的物理学家罗伯特·F·克里斯蒂命名。[49][50][51][52]1945年7月的“三位一体”核试验中,泰勒是少数几个戴着护目装备而非背对地面观看核爆的人之一。他后来回忆称,那道核爆闪光“就像我拉开了黑屋的窗帘,阳光顿时洒满房间”。[53]
