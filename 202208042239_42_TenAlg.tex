% 张量代数
% 张量代数|对称代数

\pentry{张量的对称化和交错化\upref{SIofTe}}
这一节将用 $\mathbb T_1^0(V)$ 来构造一个代数,其上的乘法为张量积.
\subsection{共变张量代数}
由矢量空间的张量积\upref{TPofSp}知道, $\mathbb T_p^0(V)$ 上矢量和 $\mathbb T_q^0(V)$ 上矢量的张量积所在的空间为  $\mathbb T_{p+q}^0(V)$.所以要 $\mathbb T_1^0(V)$ 构造的代数其乘法为张量积,那么该代数需包含矢量空间 $\mathbb T_2^0(V)$ .于是该代数需包含子空间 $\mathbb T_1^0(V)\oplus\mathbb T_2^0(V) $

 考虑到张量积的单位元为 $1\in\mathbb F$,所以 1在该代数上,而代数的矢量空间性质要求 $\mathbb F$ 也在该代数上.所以该代数需包含子空间
\begin{equation}
\mathbb F\oplus\mathbb T_1^0(V)\oplus\mathbb T_2^0(V) 
\end{equation} 
同理,继续将该矢量空间上进行张量积,可得该代数包含子空间
\begin{equation}
\mathbb F\oplus\mathbb T_1^0(V)\oplus\mathbb T_2^0(V)\oplus\mathbb T_3^0(V)\oplus\mathbb T_4^0(V)  
\end{equation}
重复这一过程,便得所需的代数为
\begin{equation}
\mathbb F\oplus\mathbb T_1^0(V)\oplus\mathbb T_2^0(V)\oplus\cdots
\end{equation}
由直和\upref{DirSum}的性质知道,该代数上的矢量 $f$ 可记作
\begin{equation}
f=\sum_{i=0}^\infty f_i=(f_0,f_1,f_2,\cdots),\quad f_i\in\mathbb T_i^0(V)
\end{equation}
其中 $\mathbb T_0^0(V)=\mathbb F$.

很显然,其上的加法为
\begin{equation}
f+g=\sum_{i}f_i+g_i
\end{equation}
