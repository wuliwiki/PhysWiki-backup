% 指数函数(高中)
% keys 指数|指数函数|自然常数
% license Usr
% type Tutor

\begin{issues}
\issueDraft
\end{issues}

\pentry{函数\nref{nod_functi},函数的性质\nref{nod_HsFunC},幂运算与幂函数\nref{nod_power}}{nod_d767}

在幂运算的基础上。

\subsection{自然常数$e$}

这里要先介绍一个特殊的常数$e$。它和早已在小学时就接触过的$\pi$非常像。他们都是


\subsection{指数函数}

将底数作为参数,指数作为自变量的函数就称为指数函数,指数函数的名称指的就是自变量的在指数位置上,注意不要与幂函数相混淆。

\begin{definition}{指数函数}
形如
\begin{equation}
f(x) = a^x~.
\end{equation}
的函数称作\textbf{指数函数(exponential function)},其中 $a\in\mathbb R$。
\end{definition}
\subsection{指数函数的性质}

\subsection{指数爆炸}

指数爆炸意味着一个函数值随自变量呈指数级别的快速增长,它的显著特征是初期增速缓慢,但随后会急剧加速。

指数函数的增长速度非常快,对于初等函数而言,当参数 x 足够大时,指数函数的增长速度是最快的。具体来说,若参数 $a > 1$,在第一象限内($x > 0$)的典型函数增长速度从慢到快通常满足以下顺序:

\begin{equation}
\log_a{x} < a < x^a < a^x~.
\end{equation}

解释:

	1.	对数函数 $\log_a{x}$ 增长得最慢,它的值在 $x$越大时,增速也会越来越慢。
	2.	常数 a 是一个固定值,不随 $x$ 改变,因此在比较函数时是增长速度较慢的参考点。
	3.	幂函数 x^a 的增长速度比常数 a 快,但比指数函数 a^x 慢。随着 x 增加,x^a 增速逐渐加快,但仍低于 a^x。
	4.	指数函数 a^x 增长得最快。特别是当 x 较大时,指数增长会呈现“爆炸式”的加速,远超其他初等函数。


\subsection{柯西函数方程}

事实上,指数函数就是满足柯西函数方程$f(x+y)=f(x)f(y)$的一个解。