% 施瓦西度规下时空的测地线
% keys 广义相对论|施瓦西度规
% license Usr
% type Tutor

\pentry{测地线\nref{nod_geodes},世界线与固有时\nref{nod_wdline},Christoffel 符号\nref{nod_CrstfS},自然单位制、普朗克单位制\nref{nod_NatUni}}{nod_6191}
\subsection{施瓦西度规下时空的测地线}
回顾施瓦西度规
\begin{equation}
\dd  s^2 = -\left(c^2 - \frac{2GM}{r}\right) \dd t^2+ \left(1-\frac{2GM}{r c^2}\right)^{-1}\dd r^2 + r^2(\dd \theta^2 + \sin^2 \theta \dd \varphi^2) ~,
\end{equation}
与其 Christoffel 符号(用 $x^0$ 代表时间坐标 $t$,$x^1$ 代表 $r$,$x^2$ 代表 $\theta$,$x^3$ 代表 $\varphi$,使用自然单位制 $G=1, c=1$)
\begin{equation}
\left\{
\begin{aligned}
\Gamma^0_{01} &= \Gamma^0_{10} = (M/r)(1-2M/r)^{-1},\\
\Gamma^1_{00} &= (M/r^2)(1-2M/r),\\ 
\Gamma^1_{11} &= -(M/r^2)(1-2M/r)^{-1},\\
\Gamma^1_{22} &= -r(1-2M/r),\\
\Gamma^1_{33} &= -r(1-2M/r)\sin^2 \theta,\\
\Gamma^2_{12} &= \Gamma^2_{21} = 1/r, \\
\Gamma^2_{33} &= -\sin \theta \cos \theta, \\
\Gamma^3_{13} &= \Gamma^3_{31} = 1/r, \\
\Gamma^3_{23} &= \Gamma^3_{32} = \cot \theta ~.
\end{aligned}\right. ~~
\end{equation}
以及度规降指标后的黎曼曲率张量 $R_{ijkm}$
\begin{equation}
\left\{
\begin{aligned}
R_{0101} &= -2M/r^3,\\
R_{0202} &= (M/r)(1-2M/r),\\
R_{0303} &= (M/r)(1-2M/r)\sin^2 \theta,\\
R_{1212} &= -(M/r)(1-2M/r)^{-1},\\
R_{1313} &= -(M/r)(1-2M/r)^{-1} \sin^2 \theta,\\
R_{2323} &= 2M r \sin^2 \theta .
\end{aligned}\right. ~~
\end{equation}

这指出在大天体 $M$ 附近的时空的情况。

而考虑对于一条类时的测地线 $\gamma(\tau)$,其中 $\tau$ 是固有时,若要求分量的参数表达式 $x^\mu(\tau)$ 则是要解测地线方程
\begin{equation}
\dv[2]{x^\mu}{\tau} + \Gamma^{\mu}_{\nu \lambda} \dv{x^\nu}{\tau} \dv{x^\lambda}{\tau} = 0~.
\end{equation}
其中 $\Gamma^\mu_{\nu \lambda}$ 是 Christoffel 符号。

而对于施瓦西度规下的时空,由于对称性,总可以选取某个坐标系使得 $\gamma(\tau)$ 的 $\theta$ 值恒为 $\pi/2$,即这类时测地线总在赤道面内。利用这性质,考虑 $\theta$ 应满足的测地线方程是
\begin{equation}
\dv[2]{\theta}{\tau} + \frac{2}{r} \dv{r}{\tau} \dv{\theta}{\tau} - \sin \theta \cos \theta \left(\dv{\theta}{\tau}\right)^{-1} = 0 ~.
\end{equation}
而考虑参数方程 $t = t(\tau)$,$r = r(\tau)$,$\theta = \pi/2$,$\varphi = \varphi(\tau)$ 与 $\gamma(\tau)$ 的切矢量 $T^\mu = (\partial/\partial \tau)^\mu$。取 $\kappa = -g_{\mu\nu} T^\mu T^\nu$,则对于类时测地线 $\kappa = 1$,且利用施瓦西度规可以直接得到
\begin{equation}
-\kappa = -\left(1 - \frac{2M}{r}\right)\left(\dv{t}{\tau}\right)^2 + \left(1-\frac{2M}{r}\right)^{-1} \left(\dv{r}{\tau}\right)^2 + r^2 \left(\dv{\varphi}{\tau}\right)^2 ~.
\end{equation}

而若取 $E = (1-2M/r) \dv*{t}{\tau}$ 与 $L = r^2 \dv*{\varphi}{\tau}$,则可改写为
\begin{equation}\label{eq_grScGe_1}
-\kappa = -\left(1-\frac{2M}{r}\right)^{-1} E^2  + \left(1-\frac{2M}{r}\right)^{-1} \left(\dv{r}{\tau}\right)^2 + L^2/r^2 ~,
\end{equation}
可以证明,$E$ 是在那点处单位质量的引力势能,而 $L$ 则是单位质量的角动量。

\subsection{应用:水星近日进动}\label{sub_grScGe_1}
我们曾在狭义相对论效应造成的近日进动\upref{relPre}中讨论了狭义相对论的影响,下面用广义相对论进行理论预测。对于类时测地线 $\kappa = 1$,从而可以通过两边同时除以 $(\dv*{\varphi}{r})^2$ 将\autoref{eq_grScGe_1} 改写为
\begin{equation}
\left( \dv{r}{\varphi} \right)^{-1} - E^2r^4/L^2 + r^2 \left( 1+r^2/L^2 \right)\left(1 - 2M/r\right) = 0 ~.
\end{equation}
其中利用了 $L = r^2 \dv*{\varphi}{r}$。

仍考虑中心天体问题中的常见代换 $u = 1/r$ 会得到
\begin{equation}
\left( \dv{u}{\varphi} \right)^2 + u^2 = (E^2-1)/L^2 + 2Mu/L^2 + 2Mu^3 ~.
\end{equation}
等式两侧可以同时对 $\varphi$ 求导得到
\begin{equation}\label{eq_grScGe_2}
\dv[2]{u}{\varphi} + u = \frac{M}{L^2} + 3M u^2 ~.
\end{equation}
这与常规的预测多了 $3Mu^2$ 项,便是广义相对论的修正。而由于水星与太阳的体系中 $M/r \ll 1$,从而 $3Mu^2 \ll u$。可以取近似解,先用常规预测出的轨道
\begin{equation}
u_0(\varphi) = \frac{M}{L^2}(1 + e \cos \varphi) ~~
\end{equation}
作为零阶近似并代入\autoref{eq_grScGe_2} 后得到一阶近似 $u_1(\varphi)$ 的解:
\begin{equation}
\dv[2]{u_1}{\varphi} + u_1 = M/L^2 + 3M u_0^2 ~,
\end{equation}
会得到解是
\begin{equation}
u_1(\varphi) = u_0(\varphi) + \frac{3M^3}{L^4} \left( 1 + e\varphi \sin \varphi + e^2\left( \frac12 - \frac16 \cos(2\varphi) \right) \right) ~.
\end{equation}

估计近日点进动时候可以忽略 $e\varphi \sin \varphi$ 以外的项而
\begin{equation}
u_1(\varphi) = \frac{M}{L^2} \left( 1 + e\left(\cos \varphi + \frac{3M^2}{L^2} \varphi \sin \varphi \right) \right) ~.
\end{equation}

而由于 $u_0(\varphi) = (M/L^2) (1+e\cos \varphi)$,这指出近日点处可以估计 $M/L^2 \sim u$,从而 $M^2/L^2 \ll 1$。取小量 $a = 3M^2/L^2$ 则 $\cos(a\varphi) \approx 1$ 而 $\sin(a \varphi) \approx a\varphi$,从而可以改写轨道方程为
\begin{equation}
\frac{1}{\varphi} \approx u_1(\varphi) \approx \frac{M}{L^2} (1 + e \cos (\varphi - a \varphi)) ~.
\end{equation}

为此我们考虑近日点处的进动,近日点是 $r$ 最小处,即 $\cos(\varphi - a\varphi) = 1$,而 $\varphi=0$ 是但 $\varphi=2\pi$ 时就不再一定是了,此时
\begin{equation}
\cos(\varphi - a\varphi) = \cos(2\pi - 2a\pi) \neq 1 ~,
\end{equation}
考虑接近这值的有解 $\varphi'$ 是近日点,则 $\varphi' - a \varphi' = 2\pi$,即 $(1 - a) \varphi' = 2\pi$,从而可以近似解为 $\varphi' \approx 2\pi (1 + a)$。也就是说进动
\begin{equation}
\varphi' - \varphi \approx 2a\pi = 6\pi M^2/L^2 ~.
\end{equation}

代入水星数据可以得到是 $43''$ 每世纪,是非常接近的结果。


