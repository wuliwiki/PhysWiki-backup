% 方程与不等式(高中)
% keys 方程|不等式|代数基本定理
% license Xiao
% type Tutor

\begin{issues}
\issueDraft
\end{issues}

\subsection{方程}

广泛使用的方程的定义是“含有未知数的等式”,但这样的定义会使人困惑如$x=3$是否也是一个方程。

方程是包含未知数的表达两个代数表达式相等的数学关系等式。

若方程的解可以由有限次常见运算的组合,这种解称为\textbf{解析解},较复杂的方程式不一定可以找出解析解,或解析解根本不存在,但仍可以利用数值分析的方式解方程,此时得到的解称为\textbf{数值解}。

方程的解可以分为两大类:解析解和数值解。

	•	解析解(Analytical Solution):如果方程的解可以通过有限次的常见运算(如加、减、乘、除、幂、对数、三角函数等)得到,这种解称为解析解。这意味着解的表达式可以用代数形式清晰地表示出来。例如,求解简单的一次方程 $ax + b = 0$ 可以得到解析解 $x = -\frac{b}{a}$。
	•	数值解(Numerical Solution):对于较复杂的方程,解析解可能非常难找到,甚至根本不存在。在这种情况下,可以使用数值分析方法,如二分法、牛顿法等,通过迭代和近似计算来求解方程。此时得到的解称为数值解,它不一定是一个精确的表达式,而是接近方程真实解的数值结果。数值解通常通过计算机来计算,能够为复杂问题提供高精度的近似解。

这两种解法各有优缺点,解析解提供的是精确、明确的形式,而数值解则在无法求得解析解时提供了实用的近似结果。

解析解是指方程的解可以通过常见的数学运算(如加、减、乘、除等)明确地表示出来,像你之前学过的一元一次方程解的方式。你可以写出明确的公式解。

然而,有些复杂的方程很难找到这样的解析解,甚至解析解根本不存在。这时候,我们可以通过计算的方法,利用计算机或近似方法来求解方程,这种得到的解叫做数值解。数值解虽然不精确,但它能给我们一个非常接近实际解的结果。

简单来说,解析解是精确的,数值解是近似的。

\begin{definition}{代数学基本定理}
任何一个 $n$ 次多项式函数在复数域上都有 $n$ 个零点(重数计入)。
\end{definition}
这意味着在复数范围内,可以找到所有多项式方程的解。

\subsection{基本不等式}

\begin{equation}
{a^2+b^2\over2}\geq ab~.
\end{equation}

\subsection{*柯西不等式}

柯西不等式(Cauchy-Schwarz Inequality)为两个向量或数列的内积与它们的模长之间建立了不等关系。

高中常用的二维模式如下:

\begin{equation}
\left( a^2 + b^2\right) \left(c^2 + d^2 \right) \geq \left( ac+bd \right)^2~.
\end{equation}

多维模式如下:

\begin{equation}
\left( \sum_{i=1}^{n} a_i^2 \right) \left( \sum_{i=1}^{n} b_i^2 \right) \geq \left( \sum_{i=1}^{n} a_i b_i \right)^2~.
\end{equation}

证明:在几何上,柯西不等式可以通过向量内积和向量模的关系得到解释。设 $\mathbf{a}$ 和 $\mathbf{b}$ 是两个向量,则它们的内积可以表示为:

$$\mathbf{a} \cdot \mathbf{b} = \|\mathbf{a}\| \|\mathbf{b}\| \cos \theta~.$$

其中 $\theta$ 是两个向量之间的夹角。根据 $\cos \theta$ 的取值范围,显然有:

$$|\mathbf{a} \cdot \mathbf{b}| \leq \|\mathbf{a}\| \|\mathbf{b}\|~.$$

这就是柯西不等式的向量形式。等号成立的条件是当 $\theta = 0$ 或 $\theta = \pi$ 时,即两个向量平行。
