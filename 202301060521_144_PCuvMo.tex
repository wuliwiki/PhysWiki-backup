% 曲线运动的加速度
% keys 曲率半径|加速度|圆周运动|切线|角速度

\pentry{曲率半径\upref{curvat}, 圆周运动的加速度\upref{CMAD}} % 未完成: 这里引用的曲率半径只是平面曲线的,怎么办?

要讨论质点的任意曲线运动, 如果我们已经知道曲线的形状和每个位置的运动的速度, 那么重要的是算出它的加速度, 以便用牛顿第二定律进行受力分析。 质点沿曲线运动时, 即使它的速度不变, 也存在加速度, 与速度的方向垂直。 最典型的例子就是匀速圆周运动。 我们先把对圆周运动加速度的分析拓展到沿任意曲线的匀速运动, 然后拓展到变速曲线运动。

虽然我们可以直接由定义根据直角坐标写出加速度分量如 $a_x = \dv*[2]{x}{t}$, $a_y = \dv*[2]{y}{t}$, 但有时候我们采用另一种正交分解会更方便计算和理解: 把加速度分解到与速度垂直和平行的两个方向。

\subsection{匀速曲线运动}
在以下的推导中我们会发现匀速曲线运动的加速度总是垂直于速度的方向。 加速度的大小可以直接使用匀速圆周运动的向心加速度\upref{CMAD}公式来计算: 只要把半径替换成曲线在质点处的曲率半径\upref{curvat}即可; 加速度的方向指向密切圆的圆心。
\begin{figure}[ht]
\centering
\includegraphics[width=10cm]{./figures/PCuvMo_1.pdf}
\caption{$\Delta v$与$\Delta \theta$关系示意图} \label{PCuvMo_fig1}
\end{figure}

质点做匀速曲线运动时, 由于速度矢量模长 $\abs{\bvec v}$ 恒定, 加速度\upref{VnA}矢量 $\bvec a$ 完全由速度矢量 $\bvec v$ 的方向改变而产生, 就像匀速圆周运动那样。 回顾加速度的定义(\autoref{VnA_eq4}~\upref{VnA})
\begin{equation}\label{PCuvMo_eq2}
\bvec a = \lim_{\Delta t \to 0} \frac{\Delta \bvec v}{\Delta t}
\end{equation}
类似用几何法推导匀速圆周运动的速度\upref{CMVD}那样, 我们可以近似认为\upref{LimArc}当速度矢量 $\bvec v$ 转过一个小角度 $\Delta \theta$ 时, 它的增量 $\Delta \bvec v$ 垂直于 $\bvec v$, 且大小为
\begin{equation}\label{PCuvMo_eq1}
\abs{\Delta \bvec v} = v\Delta\theta
\end{equation}
代入\autoref{PCuvMo_eq1} 得加速度大小为
\begin{equation}\label{PCuvMo_eq4}
\abs{\bvec{a}} = \lim_{\Delta t \to 0} \frac{\abs{\Delta \bvec v}}{\Delta t}
= v\lim_{\Delta t \to 0}\frac{\Delta \theta}{\Delta t} = v\omega
\end{equation}
其中 $\omega$ 是速度矢量 $\bvec v$ 在某个时刻旋转的角速度。 方向与 $\bvec v$ 垂直, 即曲线切线变化的方向。

如果使用角速度矢量(\autoref{CMVD_fig2}~\upref{CMVD}) $\bvec \omega$, 那么加速度矢量可以用矢量叉乘\upref{Cross}表示为
\begin{equation}
\bvec a = \bvec \omega \cross \bvec v
\end{equation}

那么当质点经过曲线某一点时,如何求 $\omega$ 呢? 我们可以使用曲率\upref{curvat}的概念。 令质点所在位置的曲率半径为 $R$, 根据曲率半径的定义(\autoref{curvat_eq3}~\upref{curvat}), $\Delta t$ 内质点在曲线上走过的长度为 $\Delta l = v \Delta t$, 所以切线的角度变化率为
\begin{equation}
\omega = \lim_{\Delta t\to 0}\frac{\Delta \theta}{\Delta t} = \lim_{\Delta t\to 0}\frac{\Delta \theta}{\Delta l} \frac{\Delta l}{\Delta t} = \frac{v}{R}
\end{equation}
再带入\autoref{PCuvMo_eq4} 得
\begin{equation}\label{PCuvMo_eq3}
\abs{\bvec{a}} = v\omega = \frac{v^2}{R} = \omega^2 R
\end{equation}
这和匀速圆周运动的向心加速度(\autoref{CMAD_eq4}~\upref{CMAD})的形式一样, 只是把半径换为曲率半径。 这是意料之中的, 因为圆就是曲率半径恒为 $R$ 的特殊曲线。

\subsection{一般曲线运动}
\pentry{矢量的导数、求导法则\upref{DerV}}
\begin{figure}[ht]
\centering
\includegraphics[width=8cm]{./figures/PCuvMo_2.pdf}
\caption{切向加速度$\bvec a_\tau$与法向加速度$\bvec a_n$} \label{PCuvMo_fig2}
\end{figure}
把速度矢量表示为大小和方向的标量积, 有 $\bvec v = v\uvec v$。 那么根据求导法则(\autoref{DerV_eq3}~\upref{DerV}),加速度$\bvec a$可以被分解为相互垂直的两项:
\begin{equation}\label{PCuvMo_eq5}
\bvec a = \dv{\bvec v}{t} = \dv{v}{t}\uvec v + v\dv{\uvec v}{t} = \bvec a_{\tau}+\bvec a_n
\end{equation}
其中第一项$\bvec a_{\tau}$为切向加速度,平行于速度方向,反映质点速度大小的改变;第二项$\bvec a_n$为法向加速度,垂直于速度方向,反映质点运动方向的改变。此外可以证明,$a_n=\frac{v^2}{R}$, R为此处的曲率半径\upref{curvat} \footnote{赵凯华,《新概念物理教程-力学》}

特别注意曲线运动中速度大小的变化率 $\dv*{v}{t}$ 和速度矢量的变化率 $\dv*{\bvec v}{t}$ 是不同的, 后者才是(总)加速度矢量。 只有在直线运动中, 二者才可以认为是等效的。

\begin{example}{变速圆周运动}\label{PCuvMo_ex1}
由\autoref{PCuvMo_eq5} 我们可以得到变速圆周运动的加速度为(和\autoref{CMAD_eq6}~\upref{CMAD}相同)
\begin{equation}
\bvec a = \dot v \uvec v - \omega ^2 \bvec r
\end{equation}
其中第二项为匀速圆周运动的向心加速度, 第一项为速度大小变化产生的切向加速度。
\end{example}

有时候也可以在已知加速度的情况下求出曲线某点的曲率半径
\begin{exercise}{抛物线的曲率半径}
已知抛物运动的的加速度恒为重力加速度 $\bvec g$, 求轨迹上任意一点的曲率半径。
\end{exercise}
