% 夸克
% license CCBYSA3
% type Wiki

(本文根据 CC-BY-SA 协议转载自原搜狗科学百科对英文维基百科的翻译)

\textbf{夸克}是一种基本粒子,同时也是构成物质的基本成分。夸克相互结合,形成一种复合粒子,叫强子,其中最稳定的是质子和中子,它们是原子核的组成部分。[1]基于一种叫”夸克禁闭“的现象,夸克永远不会被直接观察到或孤立地发现。它们只能在强子中找到,包括重子(如质子和中子)和介子。[2][3]因此,人们对夸克的大部分认识来自对强子的观察。

夸克有各种各样的内禀性质,包括电荷、质量、色荷和自旋。在粒子物理学标准模型中,夸克是唯一一种经历所有四个基本互相作用的基本粒子,基本相互作用有时也会被称为“基本力”(电磁力,万有引力,强相互作用力,和弱相互作用)。夸克同时也是唯一已知的基本电荷非整数的粒子。

夸克有六种"味",分别是上,下,粲, 奇,底和顶。在所有夸克中,上夸克和下夸克具有最低的质量。较重的夸克通过粒子衰变过程迅速转变成上或下夸克。粒子衰变是一个从较高质量状态转变成较低质量状态的过程。因此,上夸克和下夸克通常是稳定的,所以在宇宙中最常见,而奇夸克、粲夸克、底夸克和顶夸克只能在高能碰撞中产生(例如宇宙线的和粒子加速器)。夸克每一种味都有一种对应的反粒子,称为\textbf{反夸克},它与夸克的不同之处,在于它的一些特性跟夸克大小一样但正负不同。

夸克模型分别由物理学家默里·盖尔·曼和乔治·茨威格在1964年独立提出的。引入夸克这一概念,是为了能更好地整理各种强子,而当时并没有什么能证实夸克存在的物理证据,直到1968年SLAC开发出深度非弹性散射实验为止。夸克的六种味已经全部被加速器实验所观测到;而于1995年在费米实验室被观测到的顶夸克,是最后发现的一种。

\subsection{分类}
\begin{figure}[ht]
\centering
\includegraphics[width=14.25cm]{./figures/9a82ca6b63e8f20f.png}
\caption{标准模型中的六个夸克粒子(以紫色显示)。前三列中的每一列都形成了一个重要的世代。} \label{fig_Quark_1}
\end{figure}

标准模型是描述目前已知所有基本粒子的理论框架。这个模型包含了夸克的六个“味”, 分别味 上 (u),下(d),粲(s),奇(c),底(b),以及顶(t)中。[4] 夸克的反粒子由对应夸克符号上的条表示,例如u表示上反夸克。与一般的反物质一样,反夸克和其相应的夸克有相同的质量,平均寿命,自旋,但是电荷相反。[4]

夸克自旋为-1⁄2,根据自旋统计定理,它们是费米子,并服从泡利不相容原理,即相同的量子态不可以被多个费米子占据。这与玻色子(具有整数自旋的粒子)不同,即相同的量子态可以被多个玻色子占据。[5]与轻子不同,夸克拥有色荷,因此它们之间有强相互作用。这种吸引力使得夸克结合在一起,形成了复合粒子,称为强子。决定强子量子数的夸克被称为价夸克;除此之外,在强子内不定数量的虚“”夸克、反夸克和胶子并不影响其量子数。[6]强子有两个家族,它们分别是重子和介子,其中重子有三个价夸克,介子有一个价夸克和一个反夸克。[7]最常见的重子是质子和中子,它们是原子核的组成部分。目前发现很多强子(参见重子列表和介子列表),它们通常通过其夸克含量以及其性质来区分。拥有较多价夸克的“奇异”强子,例如从夸克模型预测的四夸克(qqqq)和五夸克(qqqqq),直到21世纪早期才被发现。[8][9][10][11]

基本费米子分为三代,每个都包含两个轻子和两个夸克。第一代包括上夸克和下夸克,第二代包括奇夸克和粲夸克,以及第三代包括底夸克和顶夸克。所有对第四代夸克和其他费米子的研究都失败了,[12][13]有强有力的间接证据表明不超过三代。[14][15][16]较高代的粒子通常质量较大,稳定性较低。它们通过弱相互作用衰变转化为下一代粒子。自然界中最常见的只有第一代(上下)夸克。较重的夸克只能在高能碰撞中产生(例如宇宙射线),然后迅速衰变。然而,人们认为夸克的出现开始于大爆炸时期,当时宇宙处于非常热和稠密的阶段。目前对重夸克的研究多是在人工创造的条件下进行的,例如粒子加速器。[17]

夸克具有电荷、质量、色荷和味这四个属性,是唯一已知的参与当代物理学所有四个基本相互作用理论的基本粒子。这四个基本相互作用力分别是,电磁相互作用力,万有引力,强相互作用力,和弱相互作用。除了在能量的极限(普朗克能量)和距离标度(普朗克距离)的条件下,涉及到粒子相互作用的粒子太弱了。目前,标准模型重不考虑万用引力作用,主要是因为没有成功的涉及万有引力的量子力学理论存在。

引力太弱,除了在能量的极限(普朗克能量)和距离标度(普朗克距离)中。然而,既然没有成功重力量子理论引力不是标准模型所描述的。

请参见下面的来查看夸克六种“味”的性质。

\subsection{历史}
\begin{figure}[ht]
\centering
\includegraphics[width=6cm]{./figures/62ccfb331751f703.png}
\caption{默里·盖尔·曼} \label{fig_Quark_2}
\end{figure}

\begin{figure}[ht]
\centering
\includegraphics[width=6cm]{./figures/5a8a3037822de78c.png}
\caption{乔治·茨威格} \label{fig_Quark_3}
\end{figure}

\begin{figure}[ht]
\centering
\includegraphics[width=6cm]{./figures/c4a65218fb231c65.png}
\caption{导致发现的事件的照片Σ++c重子,1974年在布鲁克黑文国家实验室} \label{fig_Quark_4}
\end{figure}
夸克模型是由物理学家默里·盖尔曼和和乔治·茨威格[18][19]在[20]1964年独立提出的。[21]盖尔曼于1961年提出粒子分类系统-八重法,用专业的术语来说,又称为U(3) 味对称。在此之后,他提出了夸克模型,用来精简早期的结构。[21]物理学家尤瓦尔·内曼在同一年独立开发了一个类似于八重道的理论,[22][23]在坂田模型中可以看到对组成结构的早期尝试。

在夸克理论诞生的时候,当时的“粒子园”除了其他各种粒子,还包括了许多强子。盖尔曼和茨威格假设它们不是基本粒子,而是由夸克和反夸克组成的。在他们的模型中,夸克有三种味,分别是上,下和奇,他们将自旋和电荷等属性都归因于这些味。[20][18][19]物理界对这项提议的最初反应不一。人们对于夸克的性质有所争议,主要集中在夸克是一种物理实体,还是仅仅用来解释当时未明物理的抽象概念。[24]

一年之后,有人提出了盖尔曼-茨威格模型的延展方案。谢尔登·格拉肖和詹姆斯·比约肯预测了第四种夸克味的存在,他们称之粲夸克。之所以加上第四种夸克是因为,一、能更好地描述弱相互作用(导致夸克衰变的机制);二、夸克的数量会变得与当时已知的轻子数量一样;三、能产生一条质量公式,可以计算出已知介子的质量。[25]

1968年,在斯坦福线性加速器中心 (SLAC)的深度非弹性散射实验表明,质子包含比自己小得多的点状物,因此不是基本粒子。[26][26][26]物理学家当时不愿意将这些物体视为夸克,而是称之为“成子”(parton) ——一个由理查德·费曼创造的术语。[27][28][29]随着在SLAC所观测到的粒子后来被鉴定为上夸克和下夸克。[30]不过,成子仍然被用作强子(夸克、反夸克和胶子)成分的总称。

SLAC散射实验间接验证了奇夸克的存在:它不仅是盖尔曼和茨威格三夸克模型的必要组成部分,而且还解释了于1947年在宇宙射线中发现的K和π强子。[31]

在1970年的一篇论文中,格拉肖、约翰·李尔普罗斯和卢奇亚诺·马伊阿尼一起对当时尚未发现的粲夸克,提出了一个新的理论模型,即GIM机制。[32][33]到1973年,小林诚和益川敏英在实验中注意到宇称不守恒定律。他们指出只有再加一对夸克,就可以解释该实验观察[34],因此夸克的味增加至现在的六种。

粲夸克在1974年11月由两个研究小组几乎同时产生(见十一月革命)—一组在SLAC,由伯顿·里克特领导;而另一组在布鲁克黑文国家实验室,由丁肇中领导。观察到粲夸克在介子里面与一个反粲夸克束缚在一起。两组给发现的这个介子分别起了不同的名字:$J$和$\psi$;因此,它也被正式命名为 $J/\psi$介子。这一发现最终使物理界相信了夸克模型是正确的。[29]

在之后的几年里,物理界出现了一些把夸克数量扩展到六个的建议。其中,以色列物理学家海姆·哈拉里在1975年的论文中,第一次将加上的夸克命名为顶夸克和底夸克。

1977年,利昂·莱德曼领导的费米国立加速器实验室团队观察到了底夸克。[35][36]这是一个代表顶夸克存在的有力证据:没有顶夸克,底夸克就没有伙伴。然而,直到1995年,顶夸克才最终被费米实验室的 CDF[37]和DØ [38]小组观察到。[21]它的质量比以前预期的要大得多,[39]几乎和金原子一样重。[40]

\subsection{语源}
有一段时间,盖尔曼打算用鸭的叫声来命名夸克。刚开始他并不确定这个词的真正拼写,直到他在詹姆斯·乔伊斯的小说芬尼根守灵夜找到了这个词夸克':[41]

– 给马斯特 马克来三个夸克!

这个词夸克本身就是一个斯拉夫语借入德国的并且表示乳制品,[42]但也是“垃圾”的口语术语。[43][44]盖尔曼在其著作《夸克与美洲豹》中,详细说明了夸克这个词的:[45]

在1963年,我把核子的基本构成部分命名为“夸克”(quark),我先有的是声音,而没有拼法,所以当时也可以写成“郭克”(kwork)。不久之后,在我偶尔翻阅詹姆斯·乔伊斯所著的《芬尼根的守灵夜》时,我在“向麦克老大三呼夸克”这句中看到夸克这个词。由于“夸克”(字面上意为海鸥的叫声)很明显是要跟“麦克”及其他这样的词押韵,所以我要找个借口让它读起来像“郭克”。但是书中代表的是酒馆老板伊厄威克的梦,词源多是同时有好几种。书中的词很多时候是酒馆点酒用的词。所以我认为或许“向麦克老大三呼夸克”源头可能是“敬麦克老大三个夸脱”,那么我要它读“郭克”也不是完全没根据。再怎么样,字句里的三跟自然中夸克的性质完全不谋而合。

茨威格更喜欢这个叫”埃斯“(Ace)来称呼他所理论化的粒子。但是,随着夸克模型被广泛接受,盖尔曼的术语就变得很有名。[46]

夸克味的命名都是有原因的。上及下夸克被这样叫,主要是源于同位旋的上和下分量,而它们确实各自带有这样的一个量。[47]奇夸克之所以得名,是因为它们在在宇宙射线的奇异粒子中被发现的,发现奇夸克的时候还没有夸克理论;这些粒子被认为是“奇异”,因为它们的寿命异常长。[48]格拉肖和布约肯共同提出了粲夸克,据说格拉肖说,“我们称它为‘粲夸克’,是因为在构建它的过程中,见到它为亚原子世界所带来的对称。我们被这种美迷住了,对成果感到很满意”[49]哈拉里创造的“底”和“顶”这两个名字,是因为它们是“上及下夸克的逻辑伙伴”。[50][50][48]过去,底部夸克和顶部夸克有时分别被称为“美”和“真”夸克,但是这些名字已经不再使用了。[50]虽然“真”从未流行,但致力于大规模生产底部夸克的加速器复合体有时被称为“美容院”。[51]

\subsection{性能}
\subsubsection{4.1 电荷}
夸克的电荷是分数——基本电荷的−1⁄3或+2⁄3倍,取决于其“味”。上,粲和顶夸克(统称为“上型夸克”)的电荷为+2⁄3,而下,奇及底夸克(统称为“下型夸克”)的电荷为-1⁄3。反夸克与其对应的夸克电荷相反;上型反夸克的电荷为-2⁄3和下型反夸克的电荷为+1⁄3。由于强子的电荷是组成它的夸克的电荷总和,所以所有强子的电荷均为整数:三个夸克的组合(重子)、三个反夸克的组合(反重子)、或一个夸克配一个反夸克(介子),加起来的电荷均为整数。[52]例如,原子核的强子成分,中子和质子,其电荷分别为0 e和+1 e;中子由两个下夸克和一个上夸克组成,质子由两个上夸克和一个下夸克组成。[53]

\subsubsection{4.2 自旋}
自旋是基本粒子的内禀属性,它的方向是一个重要的自由度。在视像化时,有时它被视为一沿着自己中轴旋转的物体(因此得名“ 自旋 ”)。因为基本粒子经常被认为是点状的,所以这种视像化概念在亚原子尺度上有点被误导了。[53]

自旋可以用矢量来表示,其长度用约化普朗克常量来量度。对于夸克,在任何轴上量度自旋的矢量分量均为+\hbar/2或$-\hbar/2$;因此夸克是自旋$\frac{1}{2}$粒子。

沿着给定轴旋转的分量(惯例为z轴),一般用上箭头$\ket{↑}$代表$+\frac{1}{2}$,下箭头$\ket{↓}$代表$-\frac{1}{2}$,然后在后加上味的符号。例如,自旋为$+\frac{1}{2}$的上夸克可被写成$\ket{u↑}$。


\subsubsection{4.3 弱相互作用}
\begin{figure}[ht]
\centering
\includegraphics[width=6cm]{./figures/9eb9dd672716dfd5.png}
\caption{随着时间的推移,β衰变的费曼图。CKM矩阵(下面讨论)展示了这种和其他夸克衰变的概率。} \label{fig_Quark_5}
\end{figure}
夸克只能通过弱相互作用由一种味转变为另一种味,弱相互作用是粒子物理学的四种基本互相作用力之一。通过吸收或释放W玻色子,任何上型夸克(上,粲和顶夸克)都可以转变成任何下型夸克(下,奇和底夸克),反之亦然。这种味变机制正是导致β衰变的原因,在β衰变过程中,一个中子($n$)“分裂”成一个质子($p$),一个电子($e^{-}$)和一个反电中微子($v_e$)(见右图)。在$\beta$衰变过程中,中子(udd)的一个下夸克释放一个虚 $W^{-}$玻色子后,随机衰变成一上夸克,于是中子就变成了质子(uud)。随后 $W^{-}$玻色子衰变为一个电子和一个反电中微子。[56]






