% https 笔记
% license Usr
% type Note

\begin{itemize}
\item 不对称和对称加密结合。 用不对称加密来传输对称加密的密钥。
\item 服务器持有私钥,并给客户端发公钥。用公钥加密的内容只有私钥可以解密。
\item 事实上也可以用私钥加密内容,同样只有公钥才能解密。
\item 但一般公钥是公开的,用私钥加密通常是为了验证私钥持有者的身份,所以这个过程一般不叫作加密而是叫做\textbf{签名}。
\item 如果私钥(公钥)对一个约定好的明文进行加密,而对应的公钥(私钥)可以成功解密出该铭文,就可以证明加密者一定持有私钥(公钥)。
\item 浏览器或操作系统已经事先储存了 CA(权威机构)的公钥。
\item CA 会对网站的公钥签名以证明该公钥的确是该网站的。签名后的公钥就是数字证书。
\item 拿到网站公钥后,浏览器就可以生成一个对称加密的密码, 用公钥加密,服务器用私钥解密。 但现在服务器还无法发送任何需要加密的信息给浏览器,因为被私钥加密的信息相当于是公开的。
\item 需要双方协商生产「会话秘钥」(对称加密)。
\item 以上都是在握手阶段产生的。 之后双方会采用「会话秘钥」进行加密通信。
\item SSL/TLS 的「握手阶段」涉及四次通信(不是四次握手,每次通信都有几个来回)
\end{itemize}

