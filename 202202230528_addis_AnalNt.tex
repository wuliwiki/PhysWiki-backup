% 数学分析笔记 1
% rudin|数学分析|实数|复数|集合论|集合

\begin{issues}
\issueMissDepend
\issueOther{这是一个总结, 应该放到所有相应内容之后, 以及给出详细词条的链接}
\end{issues}

本文参考 \cite{Rudin}.

\subsection{Chap 1. 实数系和复数系}

\begin{itemize}
\item \textbf{有理数(rational number)}记为 $Q$, 实数记为 $R$

\item 虽然任意两个不同的有理数间还有一个有理数, 但是有理数集中还是会有 “间隙”, 而实数集填补了这些间隙.

\item \textbf{集合(set)}:\textbf{属于(in)} $x \in A$, \textbf{不属于(not in)} $x \notin A$

\item \textbf{空集(empty set)}, \textbf{非空(none empty)},\textbf{子集(subset)} $A \subseteq B$,\textbf{超集(superset)} $B \supseteq A$, \textbf{真子集(proper subset)}

\item \textbf{有序集(ordered set)}, 任意不相等的两个元可以比较大小

\item \textbf{有上界(bounded above)}: 任意元小于等于超集中的某个元. \textbf{上界(upper bound)}

\item \textbf{最小上界(least upper bound, supremem)} : $\alpha = \sup E$; \textbf{最大下界(greatest lower bound, infimum)}: $\alpha = \inf E$

\item 有上界未必有最小上界. 例如有理数集合中,小于 $\sqrt{2}$ 的子集不存在最小上界.

\item 如果对于任意非空有上界的 $E \subset S$, 都有 $\sup E \in S$, 那 $S$ 就具有 \textbf{upper bound property}

\item \textbf{域(field)} 集合 $F$ 定义了\textbf{加法}和\textbf{乘法}. 加法满足: 闭合性, 交换律, 结合律, 存在 0 元, 存在逆元. 乘法满足: 闭合性, 交换律, 结合律, 存在单位元, 存在倒数. 加法和乘法满足分配律.

\item 有理数集是一个域

\item \textbf{有序域(ordered field)}

\item 存在一个有序域 $R$ 具有 upper bound property, 且有理数集 $Q$ 是其子集. $R$ 就是实数.

\item 实数的\textbf{阿基米德性质}: 存在整数 $n$ 使 $nx > y$ ($x > 0$)

\item $x \in R$, $x > 0$, $n$ 为整数, 存在实数 $y$ 使 $y^n = x$

\item \textbf{稠密(dense)}: 两个不同的实数间必有一个有理数

\item \textbf{extended real number system} 是在实数集基础上加入 $\pm\infty$ 两个符号. 对任何实数有 $-\infty < x < +\infty$. 所有非空子集都有最小上界和最大下界. 相比于无穷, 实数集中的元被称为 \textbf{finite}.

\item 复数是一对有序实数 $(a, b)$, 定义了加法和乘法后, 就变成了一个域. 定义 $\I = (0, 1)$.

\item 对正整数 $k$, $R^k$ 定义为所有 $k$ 个有序实数的集合 $\bvec x = (x_1, \dots, x_k)$, 其中 $x_i$ 叫做坐标.

\item 定义 $R^k$ 中的内积为 $\bvec x \vdot \bvec y = \sum_{i = 1}^k x_i y_i$

\item 定义模长为 $\abs{x} = (\bvec x \vdot \bvec x)^{1/2}$

\item 定义了内积和模长的 $R^k$ 被称为\textbf{欧几里得 $k$ 空间(euclidean k-space)}. 这也是一个\textbf{度量空间(metric space)}(见下文). 通常 $R^1$ 叫做线, $R^2$ 叫做面
\end{itemize}

\subsection{Chap 2. 基本拓扑}

\begin{itemize}
\item 2.1 \textbf{函数}是两个集合 $A$, $B$ 之间的\textbf{映射}; 定义域, 值域, 值. $f(A)$ 就是集合 $A$ 的 \textbf{image}; $f(A) \subset B$. 如果 $f(A) = B$, 那么 $f$ 把 $A$ 映射到(onto) $B$. $f^{-1}(E)$ \textbf{inverse image}. \textbf{1-1 映射}

\item 象, 逆象, 一一对应(1-1 映射)

\item 2.3 如果 $A$ 到 $B$ 存在 1-1 映射, 记为 $A \sim B$: \textbf{reflexive} $A \sim A$, \textbf{symmetric} $A \sim B \to B \sim A$, \textbf{transitive} $A \sim B, B \sim C \to A \sim C$. 此时称 $A$ 和 $B$ 等效, 他们具有相同的\textbf{基数(cardinal number)}即元素个数.

\item 2.4 定义 $J_n$ 为集合 $1,2,\dots, n$, 定义 $J$ 为 $1, 2, \dots$

\item $A \sim J_n$ 为\textbf{有限(finite)}, 不是有限就是\textbf{无限(infinite)}, $A \sim J$ 为\textbf{可数(countable)}\footnote{也叫 enumerable 或者 denumerable}. \textbf{至多可数(at most countable)} 就是有限或者可数

\item 对无限集来说, “含有同样多个元素” 变得很模糊, 但是 1-1 映射的定义仍然有效(只要写出一个表达式)

\item 有限集不可能与其真子集等效, 而无限集可以

\item 2.7 \textbf{数列(sequence)}是 $J$ 的映射 $f(n) = x_n$, 记为 $\{x_n\}$. $x_n$ 叫做一项. 如果 $x_n \in A$, 那该序列就叫 $A$ 中(元素)的序列.

\item 2.8 可数集的无穷子集仍然是可数的

\item 2.9 设 $A,\Omega$ 为集合, 对每个 $\alpha\in A$ 关联一个 $\Omega$ 的子集 $E_\alpha$. 元素为 $E_\alpha$ 的集合记为 $\{E_\alpha\}$. 这不叫集合的集合, 而是叫做 \textbf{collection of sets} 或 \textbf{类(family of sets)}.

\item \textbf{并集(Union)}: $\bigcup\limits_{m = 1}^n E_m$, \textbf{交集(intersection)}: $\bigcap\limits_{m = 1}^n E_m$

\item 并集和交集的混合运算法则与加法和乘法差不多

\item 2.12 如果 $E_n\ (n = 1, 2\dots)$ (无穷多个)是可数的, 那么它们的并集仍然是可数的

\item 如果 $A$ 是至多可数, 且对 $\alpha \in A$, $B_\alpha$ 也是至多可数, 那么 $T = \bigcup_{\alpha \in A} B_\alpha$ 也是至多可数

\item 2.13 如果 $A$ 可数, $a_i \in A$,  而 $B_n$ ($n$ 为固定的正整数)是所有 $(a_1, \dots, a_n)$ 的集合, 那么 $B_n$ 也是可数的

\item 自然数和有理数(可以看作两个有序整数)都可数, 无理数不可数

\item 2.15 如果集合 $X$ 中的元素可以叫做\textbf{点(point)}, 如果一个值为实数的函数 $d(p, q), \ p \in X,\ q \in X$ 满足: 当 $p = q$, $d(p, q) = 0$, 当 $p \ne q$, $d(p, q) > 0$, $d(p, q) = d(q, p)$, $d(p, q) \leqslant d(p, r) + d(r, q)$, $r \in X$, 我们就说这是一个\textbf{度量空间(metrix space)}, 函数 $d$ 叫做\textbf{距离函数(distance function)}, 或者\textbf{度规(metric)}. 度读 du 第四声.

\item 2.17 \textbf{开区间(segment)} $(a, b)$ 是所有 $a < x < b$ 的实数, \textbf{闭区间(interval)} $[a, b]$ 是所有 $a \leqslant x \leqslant b$ 的实数.

\item 闭区间也叫 1-方格, 类似地, $R^k$ 中可以定义 \textbf{k-方格(k-cell)}, 2-方格是长方形

\item 类似地, $R^k$ 空间也可以定义 \textbf{开/闭球(open/closed ball)}

\item \textbf{凸(convex)}: $E \subset R^k$ 对任意 $0 < \lambda < 1$ 和 $\bvec x, \bvec y \in E$ 满足 $\lambda \bvec x + (1 - \lambda) \bvec y \in E$. 例如, ball 和 k-方格都是 convex 的.

\item 2.18 度量空间中, \textbf{邻域(neighborhood)} $N_r$: 到某点距离小于 $r$ 的集合($r > 0$)

\item 度量空间中, \textbf{极限点(limit point)} $p$: 所有邻域存在一个与 $p$ 不同的点(无论半径有多小)

\item 如果不是极限点, 那就是 \textbf{孤立点(isolated point)}

\item 对度量空间的子集 $E$, 如果所有极限点都属于 $E$, $E$ 就是\textbf{闭(closed)}的.

\item 对度量空间的子集 $E$, 如果点 $p \in E$ 的在度量空间中某个邻域是 $E$ 的子集, $p$ 就是 $E$ 的\textbf{内点(interior point)}

\item 对度量空间的子集 $E$, 如果 $E$ 中的任意一点都是内点, $E$ 就是 \textbf{开(open)} 的.

\item 特例: 在孤立点构成的度量空间中, 任何子集都既开又闭.

\item 注意度量空间 $X$ 的子集 $E$ 的开或闭取决于 $X$ 的选取. 如果 $E$ 就是 $X$ 本身, 那么 $E$ 既开又闭.

\item \textbf{补集(complement)}

\item 如果一个闭集合中每一点都是它的极限点, 那么该集合就是\textbf{完全(perfect)} 的

\item 如果集合中任意一点都在某个 $r$ 为实数的邻域内, 这个集合就是\textbf{有界的(bounded)}

\item (私货)对一个度量空间 $X$, 若存在 $r\in R$ 使任意两点 $p,q\in X$ 都满足 $d(p,q) \leqslant r$, 那么 $X$ 就是\textbf{有界的(bounded)}

\item 集合 $E$ 在集合 $X$ 上\textbf{稠密(dense)}: $X$ 中任意一点都是 $E$ 的一个极限点或者 $E$ 中的一点. (例如有理数在实数上稠密)

\item 2.19 任何邻域都是开的

\item 2.20 如果 $p$ 是一个极限点, 那么它的任何邻域都有无限多个点

\item 有限个点的集合中没有极限点

\item 2.22 $(\bigcup_\alpha E_\alpha)^c = \bigcap_\alpha (E_\alpha^c)$ 其中 $c$ 代表补集

\item 2.23 集合 $E$ 是开的当且仅当它的补集是闭的. $E$ 是闭的当且仅当它的补集是开的.

\item 空集和全集既开又闭

\item 2.24 任意多开集合的并集仍然是开的, 任意多闭集合的交集仍然是闭的; 有限个开集合的交集仍然是开的, 有限个闭集合的并集仍然是闭的

\item 2.26 设 $X$ 是度量空间, 如果 $E \subset X$, $E'$ 表示 $E$ 在 $X$ 中所有极限点组成的集. 那么, 把 $\bar E = E \cup E'$ 叫做 $E$ 的\textbf{闭包(closure)}

\item 2.27 设 $X$ 是度量空间,而 $E \subset X$, 那么 (a) $\bar E$ 是闭的, (b) $E = \bar E$ 当且仅当 $E$ 闭, (c) 如果闭集 $F \subset X$ 且 $E \subset F$, 那么 $\bar E \subset F$. 由 (a) 和 (c), $\bar E$ 是 $X$ 中包含 $E$ 的最小闭子集

\item 2.28 设 $E$ 是一个非空实数集, 上有界, 令 $y = \sup E$ ,那么 $y \in \bar E$.
因此, 如果 $E$ 闭, 那么 $y \in E$.

\item 2.30 令 $Y \subset X$, $E \subset Y$, $E$ 相对 $Y$ 是开的当且仅当 $E = Y \bigcap G$, 对某个开的 $G \subset X$

\item 2.31 若 $X$ 的一组开子集 $\{G_\alpha\}$ 使 $E \subset \bigcup_\alpha G_\alpha$, 那么 $\{G_\alpha\}$ 就是 $E$ 的\textbf{开覆盖(open cover)}

\item 2.32 \textbf{紧集(compact set)}: 如果 $\{G_\alpha\}$ 是 $K$ 的开覆盖, 那么存在有限个 $\alpha_1,\dots, \alpha_n$ 使得 $K \subset G_{\alpha_1} \bigcup \dots \bigcup G_{\alpha_n}$. 即任何开覆盖都存在有限的子覆盖. 紧集是分析中的非常重要概念.

\item 有限集都是紧集

\item 如果 $E \subset Y \subset X$, 那么 $E$ 可能是 $Y$ 中的开集而不是 $X$ 的开集. 闭集也同理.

\item 2.33 假设 $K \subset Y \subset X$. 那么 $K$ 在 $X$ 中是紧的当且仅当它在 $Y$ 中也是紧的.

\item 2.34 度量空间的紧子集是闭的

\item (私货)一个例子: 证明开区间 $(0,1)$ 不是一个紧集: 易知它的一组无穷开覆盖为 $\bigcup_{n=0}^{\infty}((2/3)^n/3, (2/3)^n)$, 且不存在有限子覆盖. 反之 $[0,1]$ 则不存在类似的问题. 另外注意这性质与所考虑区间的父集无关.

\item (私货)度量空间的紧集都是有界的.

\item 2.35 紧集的闭子集也是紧的

\item 闭集和紧集的交集是紧的

\item 2.36 如果 $\{K_\alpha\}$ 是度量空间 $X$ 的一组紧子集且任意有限个 $\{K_\alpha\}$ 的交集为非空, 那么 $\bigcap K_\alpha$ 也是非空的

\item 2.37 如果 $E$ 是紧集 $K$ 的无穷子集, 那么 $E$ 在 $K$ 中存在极限点

\item 2.38 如果 $\{I_n\}$ 是 $R^1$ 中的一组闭区间序列, 且 $I_n \supset I_{n+1} (n = 1, 2, 3,\dots)$, 那么 $\bigcap_1^\infty I_n$ 非空

\item 2.40 $k$-方格是紧的

\item 2.41 对 $R^k$ 中的集合 $E$, 这三个条件等价: (a) $E$ 闭且有界. (b) $E$ 是紧的. (c) $E$ 中的任意无限集在 $E$ 中存在极限点

\item 2.42 Weierstrass 定理: $R^k$ 中任何有界的无限集在 $R^k$ 中有(至少)一个极限点

\item 2.43 令 $P$ 为 $R^k$ 内的非空完全集. 那么 $P$ 是不可数的

\item 2.44 \textbf{Cantor 集}说明 $R^1$ 中存在没有区间的完全集.

\item 2.45 度量空间 $X$ 的两个子集 $A$ 和 $B$ 被称为\textbf{分离的(separated)} 如果 $A \cap \bar B$ 和 $\bar A \cap B$ 都是空集.

\item 如果 $E \subset X$ 不是两个非空分离集的并, 就说 $E$ 是连通(connected)集.

\item 2.46 分离的两个集是不相交的, 但不相交的集合不一定是分离的. 例如 $[0,1]$ 和 $(1,2)$ 不是分离的.

\item 2.47 实数集 $R^1$ 的子集 $E$ 是连通的, 当且仅当: 如果 $x\in E$, $y\in E$, 且 $x < z < y$, 那么 $z \in E$.
\end{itemize}

\subsection{Chap 3. 数列与级数}

\begin{itemize}
\item 3.1 度量空间 $X$ 中的序列 $\{p_n\}$ 叫做\textbf{收敛的(converged)}, 如果有一个有下述性质的点 $p\in X$: 对每个 $\epsilon > 0$, 有一个正整数 $N$, 使得 $n \geqslant N$ 时, $d(p_n,p) < \epsilon$. 这时候也说 $\{p_n\}$ 收敛于 $p$, 或者说 $p$ 是 $\{p_n\}$ 的极限, 写作 $p_n\to p$ 或 $\lim_{n\to \infty} p_n = p$. 如果不收敛, 就说它发散.

\item 收敛的定义不仅依赖于数列还依赖于 $X$, 例如 $\{1/n\}$ 在 $R^1$ 中收敛于 $0$, 但在正实数集合中不收敛. 所以要强调 “在 $X$ 中” 收敛. (这与小时百科的\autoref{cauchy_def1}~\upref{cauchy} 不同)

\item 一切点 $p_n$ 的集合是 $\{p_n\}$ 的\textbf{值域(range)}, 序列的值域可以是有限的, 也可以是无限的. 如果值域是有界的, 就说序列是有界的.

\item 3.2 度量空间 $X$ 中的序列 $\{p_n\}$: (a) $\{p_n\}$ 收敛于 $p\in X$, 当且仅当 $p$ 的每个邻域, 能包含除了有限项以外的一切项. (b) 如果数列同时收敛于 $p, p'$, 那么 $p' = p$. (c) 数列收敛则必有界. (d) 如果 $E \subset X$, 而 $p$ 是 $E$ 的极限点, 那么在 $E$ 中有一个序列收敛到 $p$.

\item 3.3 假定 $\{s_n\}, \{t_n\}$ 是复序列,且极限为 $s, t$ 那么 (a) $\lim_{n\to\infty} (s_n+t_n) = s+t$, (b) 对任何数 $c$, $\lim_{n\to\infty}cs_n = cs$; (c) $\lim_{n\to\infty}(s_n t_n) = st$; (d) $\lim_{n\to\infty} (c+s_n) = c+s$

\item 3.4 (a) 假定 $\bvec x_n \in R^k$ ($n=1,2,...$) 而 $\bvec x_n = (a_{1,n}, ...,a_{k,n})$ 那么序列收敛于 $(a_1,...,a_k)$ 当且仅当 $\lim_{n\to\infty} a_{j,n} = a_j$; (b) 假定 $\{\bvec x_n\}, \{\bvec y_n\}$ 是 $R^k$ 的序列, $\{\beta_n\}$ 是实数序列, 并且 $\bvec x_n\to \bvec x, \bvec y_n\to \bvec y, \beta_n\to\bvec \beta$. 那么 $\lim_{n\to\infty} (\bvec x_n+\bvec y_n) = \bvec x+\bvec y$, $\lim_{n\to\infty} \bvec x_n \vdot \bvec y_n = \bvec x \vdot \bvec y$, $\lim_{n\to\infty} \beta_n \bvec x_n = \beta \bvec x$.

\item 3.5 有序列 $\{p_n\}$, 取正整数序列 $\{n_k\}$, 使 $n_1<n_2<...$ 那么序列 $\{p_{n_i}\}$ 便叫做 $\{p_n\}$ 的\textbf{子序列(subsequence)}, 如果 $\{p_{n_i}\}$ 收敛, 就把它的极限叫做 $\{p_n\}$ 的\textbf{部分极限(subsequential limit)}. 序列收敛于 $p$ 当且仅当它的任何子序列收敛于 $p$.

\item 3.6 如果 $\{p_n\}$ 是紧度量空间 $X$ 中的序列, 那么 $\{p_n\}$ 有某个子序列收敛到 $X$ 中的某个点. (b) $R^k$ 中的每个有界序列含有收敛的子序列.

\item 3.7 度量空间 $X$ 里的序列 $\{p_n\}$ 的部分极限组成 $X$ 的闭子集.

\item 3.8 度量空间 $X$ 中的序列 $\{p_n\}$ 叫做\textbf{柯西序列(Cauchy)}, 如果对于任何 $\epsilon>0$ 存在着正整数 $N$, 只要 $m,n\geqslant N$ 就有 $d(p_n, p_m)<\epsilon$.

\item 3.9 设 $E$ 是度量空间 $X$ 的非空子集, 又设 $S$ 是一切形式为 $d(p,q)$ 的实数集, $p,q\in E$. $\sup S$ 叫做 $E$ 的直径, 记为 $\opn{diam} E$.

\item 3.10 (a) 如果 $E$ 是度量空间 $X$ 中的集, 那么闭包满足 $\opn{diam}\bar E = \opn{diam} E$. (b) 如果 $\{K_n\}$ 是 $X$ 中的紧集的序列, 且 $K_n \supseteq K_{n+1}$ 又若 $\lim_{n\to\infty} \opn{diam} K_n = 0$, 那么 $\bigcap_1^\infty K_n$ 由一个点组成.

\item 3.11 (a) 在度量空间中, 收敛序列是柯西序列. (b) 如果 $X$ 是紧度量空间, 并且如果 $\{p_n\}$ 是 $X$ 中的柯西序列, 那么序列收敛于 $X$ 的某个点\footnote{闭空间中的柯西序列都收敛, 度量空间的紧子集是闭的 (2.34).}. (c) 在 $R^k$ 中, 每个柯西序列收敛.

\item 3.12 如果度量空间 $X$ 的每个柯西序列都在 $X$ 中收敛, 就说该空间是完备的.

\item 因此所有紧度量空间以及所有欧氏空间都是完备的. 还说明度量空间 $X$ 的闭子集是完备的.

\item 3.13 实数序列的\textbf{单调递增}($s_n\leqslant s_{n+1}$) 和\textbf{单调递减}.

\item 3.14 单调序列收敛, 当且仅当它是有界的.

\item 3.15 $s_n\to \pm \infty$ 的定义

\item 3.16 设 $\{s_n\}$ 是实数序列. $E$ 是所有可能的子序列的极限组成的集(可能含有 $\pm\infty$). $s^* = \sup E$, $s_* = \inf E$ 这两个数叫做序列的\textbf{上极限(upper limit)}和\textbf{下极限(lower limit)}. 记为 $\lim_{n\to\infty} \sup s_n = s^*$, $\lim_{n\to\infty} \inf s_n = s_*$.

\item 3.17 

\item 3.18 例: (a) 给出一个包含一切有理数的序列, 那么每个实数是它的部分极限, 且 $\lim_{n\to\infty}\sup s_n = +\infty$, $\lim_{n\to\infty}\inf s_n = -\infty$. (b) 设 $s_n = (-1)^n/[1+(1/n)]$, 则上下极限为 $1,-1$. (c) 实数序列的极限为 $s$ 当且仅当上下极限都等于 $s$.

\item 3.19 如果 $N$ 是固定的正整数, 当 $n\geqslant N$ 时 $s_n \leqslant t_n$, 那么 $\lim_{n\to\infty}\inf s_n \leqslant \lim_{n\to\infty}\inf t_n$, $\lim_{n\to\infty}\sup s_n \leqslant \lim_{n\to\infty}\sup t_n$.

\item 3.20 (a) $p > 0$ 时 $\lim_{n\to\infty}1/n^p = 0$. (b) $p>0$ 时 $\lim_{n\to\infty}{}\sqrt[n]{p} = 1$. (c) $\lim_{n\to\infty} \sqrt[n]{n} = 1$. (d) $p>0$, 且 $\alpha$ 是实数时 $\lim_{n\to\infty} n^\alpha/(1+p)^n = 0$. (e) $\abs{x}<1$ 时 $\lim_{n\to\infty} x^n = 0$.

\item 3.21 对序列 ${a_n}$, 令 $s_n = \sum_{k=1}^n a_k$ 为\textbf{部分和}, $\sum_{n=1}^\infty a_n$ 叫做无穷级数, 简称级数. 如果 $s_n$ 收敛, 就说级数收敛, 并记为 $\sum_{n=1}^\infty a_n = s$. $s$ 叫做级数的和, 是 $s_n$ 的极限. 如果 $s_n$ 发散, 就说级数发散.

\item 柯西准则(3.11)可以重新表述为, $\sum a_n$ 收敛, 当且仅当, 对于任意 $\epsilon>0$, 存在整数 $N$, 使得 $m \geqslant n \geqslant N$ 时 $\abs{\sum_{k=n}^m a_k} \leqslant \epsilon$. 特别地,当 $m=n$ 时 $\abs{a_n}\leqslant\epsilon$($n\geqslant N$)

\item 3.23 如果 $\sum a_n$ 收敛, 则 $\lim_{n\to\infty} a_n = 0$.

\item 3.24 各项为非负的级数收敛, 当且仅当其部分和构成有界数列.

\item 3.25 (a) 如果 $N_0$ 是某个固定的正整数. $n \geqslant N_0$ 时 $\abs{a_n}\leqslant c_n$ 而且 $\sum c_n$ 收敛, 那么 $\sum a_n$ 也收敛. (b) 如果当 $n\geqslant N_0$ 时 $a_n\geqslant d_n\geqslant 0$ 而且 $\sum d_n$ 发散, 那么 $\sum a_n$ 也发散.

\item 3.26 若 $0\leqslant 0< 1$, 那么 $\sum_{n=0}^\infty x^n = 1/(1-x)$. 若 $x\geqslant 1$, 它就发散.

\item 3.27 令 $a_1\geqslant a_2\geqslant \dots \geqslant 0$, 那么 $\sum_{n=1}^\infty a_n$ 收敛, 当且仅当级数 $\sum_{n=0}^\infty 2^k a_{2^k}$ 收敛.

\item 3.28 若 $p>1$, $\sum 1/n^p$ 就收敛, 若 $p\leqslant 1$, 它就发散.

\item 3.29 若 $p>1$, $\sum_{n=2}^\infty 1/[n(\log n)^p]$ 就收敛; 若 $p\leqslant 1$, 它就发散.

\item 3.30 定义 $\E=\sum_{n=0}^\infty 1/n!$.

\item 3.31 $\lim_{n\to\infty} (1+1/n)^n = \E$.

\item 3.32 $\E$ 是无理数.

\item 3.33 \textbf{根值审敛法} 对 $\sum a_n$, 令 $\alpha = \limsup_{n\to\infty} \sqrt[n]{\abs{a_n}}$. 那么 (a) $\alpha<1$ 时级数收敛 (b) $\alpha>1$ 时级数发散 (c) $\alpha=1$ 时不确定.

\item 3.34 \textbf{比值审敛法} 对级数 $\sum a_n$ (a) 如果 $\limsup_{n\to\infty} \abs{a_{n+1}/a_n}<1$, 它就收敛. (b) 如果有某个固定的正整数 $n_0$, $n\geqslant n_0$ 时 $\abs{a_{n+1}/a_n}\geqslant 1$, 它就发散.

\item 3.37 对于任意正数序列 $\{c_n\}$, 有 $\liminf_{n\to\infty} c_{n+1}/c_n \leqslant \liminf_{n\to\infty} \sqrt[n]{c_n}$, $\limsup_{n\to\infty} \sqrt[n]{c_n}\leqslant \limsup_{n\to\infty} c_{n+1}/c_n$.

\item 3.38 对复数序列 $\{c_n\}$, 级数 $\sum_{n=0}^\infty c_n z^n$ 叫做\textbf{幂级数}. $c_n$ 叫做这个级数的\textbf{系数}; $z$ 是复数. 每个幂级数有一个圆, 叫做\textbf{收敛圆}, 如果 $z$ 在圆内, 幂级数就收敛, 如果在圆外就发散. (这里把平面看作半径无限大的圆的内部,把一点看作是半径为零的圆)级数在收敛圆上的性质不能简单地叙述.

\item 3.39 对幂级数 $\sum c_n z^n$, 令 $\alpha=\limsup_{n\to\infty} \sqrt[n]{\abs{c_n}}$, $R=1/\alpha$ (若 $\alpha=0$, 令 $R=+\infty$; 若 $\alpha=+\infty$, 令 $R=0$). 那么 $\sum c_n z^n$ 在 $\abs{z}<R$ 时收敛, $\abs{z}>R$ 时发散.

\item 3.41 令 $A_n=\sum_{k=0}^n a_k$, $A_{-1}=0$. 当 $0\leqslant p\leqslant q$ 时, 有 $\sum_{n=p}^q a_n b_n = \sum_{n=p}^{q-1} A_n(b_n-b_{n+1}) + A_qb_q - A_{p-1} b_p$.

\item 3.42 假设 (a) $\sum a_n$ 的部分和 $A_n$ 构成有界序列 (b) $b_0\geqslant b_1\geqslant \dots$ (c) $\lim_{n\to\infty} b_n=0$. 那么 $\sum a_n b_n$ 收敛.

\item 3.43 假定 (a) $\abs{c_1}\geqslant \abs{c_2}\geqslant \dots$, (b) $c_{2m-1}\geqslant 0$, $c_{2m}\leqslant 0$ (c) $\lim_{n\to\infty} c_n=0$. 那么 $\sum c_n$ 收敛. 该级数叫做 “交错级数”.

\item 如果 $\sum \abs{a_n}$ 收敛, 就说 $\sum a_n$ \textbf{绝对收敛}.

\item 3.44 假定 $\sum c_n z^n$ 的收敛半径是 $1$, 再假定 $c_0\geqslant c_1\geqslant \dots$, $\lim_{n\to \infty}c_n=0$, 那么 $\sum c_nz^n$ 在 $\abs{z}=1$ 的每个点收敛, 只有 $z=1$ 这一点可能是例外.

\item 3.45 绝对收敛必收敛.

\item 3.47 如果 $\sum a_n=A$, $\sum b_n=B$, 那么 $\sum (a_n+b_n)=A+B$, 而且对于任意常数 $c$, $\sum ca_n=cA$.

\item 3.48 设有 $\sum a_n$, $\sum b_n$. 令 $c_n=a_0b_n+\dots+a_nb_0$ 那么就称级数 $\sum c_n$ 为两个级数的\textbf{积}. (私货: 考虑 $\sum a_n x^n \sum b_n x^n = \sum c_n x^n$, 合并同类项, 再令 $x=1$).

\item 3.50 如果 $\sum a_n$ 绝对收敛, $\sum a_n$, $\sum b_n$ 收敛于 $A,B$, 且 $c_n=a_0b_n+\dots+a_nb_0$, 那么 $\sum c_n = AB$.

\item 3.51 如果级数 $\sum a_n$, $\sum b_n$, $\sum c_n$ 分别收敛于 $A,B,C$, 且 $c_n=a_0b_n+\dots+a_nb_0$, 那么 $C=AB$.

\item 3.52 设 $\{k_n\}$, $n=1,2,3,\dots$ 是由正整数作成的序列, 每个正整数出现且仅出现一次. 令 $a'_n=a_{k_n}$($n=1,2,3,\dots$) 就说 $\sum a'_n$ 是 $\sum a_n$ 的\textbf{重排}.

\item 3.54 设实数级数 $\sum a_n$ 收敛但不绝对收敛. 假定 $-\infty\leqslant \alpha\leqslant\beta\leqslant\infty$. 那么一定存在重排 $\sum a'_n$, 它的部分和 $s'_n$ 满足下极限和上极限等于 $\alpha, \beta$.

\item 3.55 设 $\sum a_n$ 是绝对收敛的复数项级数, 那么 $\sum a_n$ 的每个重排收敛, 且都收敛于同一个和.
\end{itemize}
