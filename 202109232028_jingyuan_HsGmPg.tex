% 等比数列(高中)
% 高中|等比数列

\begin{issues}
\issueDraft
\end{issues}

\subsection{定义}
一般地,如果一个数列从第2项起,每一项与它前一项的比都等于同一个常数,那么这个数列叫作\textbf{等比数列},这个常数叫作等比数列的\textbf{公比},公比通常用字母 $q$ 表示 $(q\ne 0)$

\subsection{通项}
由定义可得,等比数列的通项公式
\begin{equation}
a_n = a_1 q^{n-1}(a_1 \ne 0,q\ne 0)
\end{equation}

\subsection{等比中项}
与等差数列类似,如果在 $a$ 和 $b$ 中插入一个数 $G$,使得 $a,G,b$ 成等比数列,那么根据等比数列的定义, $\frac{G}{a} = \frac{b}{G},G^2 = ab,G \pm \sqrt{ab}$.我们称 $G$ 为 $a,b$ 的\textbf{等比中项}.

易得,在等比数列中,首末两项除外,每一项都是它前后两项的等比中项.

\subsection{前 $n$ 项的和}
等比数列求和与等差数列求和有相似之处,
\begin{equation}\label{HsGmPg_eq1}
S = a_1 + a_2 + \cdots + a_n
\end{equation}
\begin{equation}
qS = qa_1 + qa_2 + \cdots + qa_n
\end{equation}
\begin{equation}\label{HsGmPg_eq2}
qS= a_2 + a_3 + \cdots + a_{n + 1}
\end{equation}
用