%Vorlage
\documentclass[12pt,a4paper]{article}
\usepackage[german]{babel} %Für die indirekte Angabe von Umlauten. Es müssen dann Umlaute wie folgt im Code angegeben werden: "a "o "u "s.

\usepackage[utf8]{inputenc}
%dieses Paket ermöglicht uns, Umlaute im Text als solche eingeben zu können (Windows/Linux)

%\usepackage[applemac]{inputenc}
%Mac-Nutzer, die Ihre Latex-Dateien lokal kompilieren wollen, müssen dieses Paket aktivieren, um Umlaute im Code direkt angeben zu können.
\usepackage{color}
\usepackage{amsmath, amsthm, amssymb}
\usepackage{enumerate}
\usepackage{graphicx}
\usepackage{lscape}
\usepackage{setspace}
\onehalfspacing
\usepackage{wrapfig}
\usepackage{hyperref}% für die Einbettung von Hyperlinks
\usepackage{multirow}
\usepackage[round]{natbib}
\usepackage{titlesec}
\usepackage{enumerate}
\titleformat{\subsection}[block]{\large\itshape\up}{}{0em}{}[]

\newtheorem{definition}{Definition}[section]
\newtheorem{interview}{Interview}[section]
\newtheorem{satz}{Satz}[section]
\newtheorem{beispiel}{Beispiel}[section]
\newtheorem{bemerkung}{Bemerkung}[section]
\newtheorem{literaturverzeichnis}{Literaturverzeichnis}[section]

% Margins
\usepackage[left=25mm,right=25mm,bottom=20mm,top=25mm]{geometry} % Document Margins
\setlength{\topmargin}{0cm}
\setlength{\parindent}{5mm}
\setlength{\parskip}{2mm}
\setlength{\evensidemargin}{0mm}
\setlength{\oddsidemargin}{0cm}
%\pagestyle{headings}
\renewcommand {\thefigure} {\arabic{figure}}


\begin{document}
\thispagestyle{empty}
\vspace*{-3cm}
\begin{center}
\large \textsc{University of Stuttgart}
\vspace{0.5cm}
\hrule
\vspace{5.5cm}
{\Large \textsc{Advanced Mathematics\\
[Laboratory Report 2]}}\\
{\large WS 2020/21}\\
\vspace{1cm}
{\Large 
Yang Li}\\
 [Student ID:3519970]\\
\vspace*{1cm}
{\large Instructor: [Prof.] Dr.Foster James}
\end{center}
\vspace*{7.5cm}
{\large

\hspace*{7cm}
\parbox{8.2cm}
{
\begin{tabular}{ll}
\bf\rightline{20.11.2020}

\end{tabular}}}


\newpage
\section{Exercise – A little bit of div and curl}\label{intro}
\subsection{Determine the divergence and the curl of the vector field (spherical coordinates).}
\[\Bar{G} = \frac{1}{r_2}\hat{h_r}-\cos{\lambda}\sin{\theta}\hat{h_\theta}+\sin{2\theta\sin{\lambda\hat{h_\lambda}}}\]
\begin{itemize}
    \item According to the Definition about Divergence in Spherical Coordinates, the vector field $v = \sum_{i=1}^3 v_i\hat{h_i}$ has the divergence:\\
    \[div \textbf{v} = \frac{1}{r^2}\frac{\partial}{\partial r}(r^2v_1)+\frac{1}{r\sin{\theta}}\frac{\partial}{\partial\theta}(\sin{\theta}v_2)+\frac{1}{\lambda\sin\theta}\frac{\partial}{\partial\lambda}v_3\]
    \item In Exercise 1, we could get:\\
    \[\Bar{G_r} = v_1 = \frac{1}{r^2},\Bar{G_\theta} = v_2 = -\cos{\theta}\sin{\theta},\Bar{G_\lambda} = v_3 = \sin{2\theta}\sin{\lambda}\]
    \begin{equation}
    \begin{aligned}
        \Rightarrow div(\Bar{G}) &=  \frac{1}{r^2}\frac{\partial}{\partial r}(r^2\cdot\frac{1}{r^2})+\frac{1}{r\sin{\theta}}\frac{\partial}{\partial\theta}(\sin{\theta}\cdot(-\cos{\theta}\sin{\theta}))+\frac{1}{\lambda\sin\theta}\frac{\partial}{\partial\lambda}\cdot(\sin{2\theta}\sin{\lambda})\notag\\
        &= 0 + \frac{1}{r\sin{\theta}}(-2\cos{\theta}\sin{\theta}\cos{\lambda})) + \frac{1}{r\sin{\theta}}(2\sin{\theta}\cos{\theta}\cos{\lambda})\\
        &= 0 + 0 + 0\\
        &= 0
    \end{aligned}
    \end{equation}
    \irem According to the Definition about Curl in Spherical Coordinates, the vector field $v = \sum_{i=1}^3 v_i\hat{h_i}$ has the curl:\\
    \[curl\ \textbf{v} = \frac{1}{r\sin{\theta}}(\frac{\partial}{\partial\theta}(\sin{\theta}v_3) - \frac{\partial v_2}{\partial \lambda})\hat{h_r} + (\frac{1}{r\sin{\theta}}\frac{\partial v_1}{\partial\lambda}-\frac{1}{r}\frac{\partial}{\partial r}(rv_3))\hat{h_\theta}+(\frac{1}{r}\frac{\partial}{\partial r}(rv_2)-\frac{1}{r}\frac{\partial}{\partial\theta}v_1)\hat{h_\lambda}\]
    \item In Exercise 1, we could get:\\
    \[\Bar{G_r} = v_1 = \frac{1}{r^2},\Bar{G_\theta} = v_2 = -\cos{\theta}\sin{\theta},\Bar{G_\lambda} = v_3 = \sin{2\theta}\sin{\lambda}\]
    \begin{equation}
    \begin{aligned}
        \Rightarrow curl(\Bar{G}) &=  \frac{1}{r\sin{\theta}}(\frac{\partial}{\partial\theta}(\sin{\theta}(\sin{2\theta}\sin{\lambda})) - \frac{\partial (-\cos{\lambda}\sin{\theta})}{\partial \lambda})\hat{h_r}\\ &+ (\frac{1}{r\sin{\theta}}\frac{\partial (\frac{1}{r^2})}{\partial\lambda}-\frac{1}{r}\frac{\partial}{\partial r}(r(\sin{2\theta}\sin{\lambda})))\hat{h_\theta}\\&+(\frac{1}{r}\frac{\partial}{\partial r}(r(-\cos{\theta}\sin{\theta}))-\frac{1}{r}\frac{\partial}{\partial\theta}(\frac{1}{r^2}))\hat{h_\lambda}\\
        &= \frac{1}{r}(4\cos^2\theta\sin{\lambda}-2\sin^2\theta\sin{\lambda}-\sin{\lambda})\hat{h_r}+(0 - \frac{\sin{2\theta}\sin{\lambda}}{r})\hat{h_\theta}\\&+(\frac{-\cos{\lambda}\sin{\theta}}{r})\hat{h_\lambda}\\
        &= \frac{\sin{\lambda}}{r}(3\cos{2\theta})\hat{h_r}-(\frac{\sin{2\theta}\sin{\lambda}}{r})\hat{h_\theta}-(\frac{\cos{\lambda}\sin{\theta}}{r})\hat{h_\lambda}\notag
    \end{aligned}
    \end{equation}
\end{itemize}
\section{Exercise – Gradient search}\label{intro}
\subsection{This relationship defines a new set of coordinates. Determine the gradient in this system for an arbitrary function $\Phi$.}
\[x = \frac{\alpha}{\alpha^2+\beta^2};y = \frac{\beta}{\alpha^2+\beta^2};z = \zeta\]
    \begin{itemize}
        \item According to the relationships, we could define:\\
        \[\vec{P} &= x\vec{i}+y\vec{j}+z\vec{k}\\
        &= \frac{\alpha}{\alpha^2+\beta^2}\Vec{i}+\frac{\beta}{\alpha^2+\beta^2}\Vec{j}+\zeta\Vec{k}\]
        \item Let $\vec{h_1},\  \vec{h_2},\  \vec{h_3}$ be the tangent vectors to the coordinate lines:
        \begin{equation}
        \begin{aligned}
        \vec{h_1} &= \frac{\partial P}{\partial\alpha}
        = \frac{\beta^2 - \alpha^2}{(\alpha^2+\beta^2)^2}\hat{i}+(-\frac{2\alpha\beta}{(\alpha^2+\beta^2)^2})\hat{j}+0\\
        &= \frac{\beta^2 - \alpha^2}{(\alpha^2+\beta^2)^2}\hat{i}+(-\frac{2\alpha\beta}{(\alpha^2+\beta^2)^2})\hat{j}\notag
        \end{aligned}
        \end{equation}
        \begin{equation}
    \begin{aligned}
        \vec{h_2} &= \frac{\partial P}{\partial\beta}
        = (-\frac{2\alpha\beta}{(\alpha^2+\beta^2)^2})\hat{i}+\frac{\alpha^2 - \beta^2}{(\alpha^2+\beta^2)^2}\hat{j}+0\\
        &= -\frac{2\alpha\beta}{(\alpha^2+\beta^2)^2}\hat{i}+\frac{\alpha^2 - \beta^2}{(\alpha^2+\beta^2)^2}\hat{j}\notag
        \end{aligned}
        \end{equation}
        \begin{equation}
    \begin{aligned}
        \vec{h_3} &= \frac{\partial P}{\partial\zeta} 
        = 0 + 0 +1\cdot\Vec{k}\\
        &= \vec{k}\notag
        \end{aligned}
        \end{equation}
       \item In general, these vectors are not normalised. We could get their length:\\
       
        \begin{equation}
    \begin{aligned}
       ||\vec{h_1}|| &= \sqrt{(\frac{\beta^2 - \alpha^2}{(\alpha^2+\beta^2)^2})^2 + (-\frac{2\alpha\beta}{(\alpha^2+\beta^2)^2})^2}\\
       &= (\frac{1}{\alpha^2+\beta^2})^2\notag
       \end{aligned}
    \end{equation}
    \begin{equation}
    \begin{aligned}
       ||\vec{h_2}|| &= \sqrt{(-\frac{-2\alpha\beta^2}{(\alpha^2+\beta^2)^2})^2 + (\frac{\alpha^2-\beta^2}{(\alpha^2+\beta^2)^2})^2}\\
       &= (\frac{1}{\alpha^2+\beta^2})^2\notag
       \end{aligned}
    \end{equation}
    \begin{equation}
    \begin{aligned}
       ||\vec{h_3}|| &= \sqrt{1^2}\\
       &=1\notag
       \end{aligned}
    \end{equation}
    \item The normalised tangential vectors are denoted by a hat:\\
    \begin{equation}
    \begin{aligned}
       \hat{h_1} &= \frac{\vec{h_1}}{||h_1||}
       = \frac{\frac{\beta^2 -\alpha^2}{(\alpha^2+\beta^2)^2}\hat{i}+(-\frac{2\alpha\beta}{(\alpha^2+\beta^2)^2})\hat{j}}{(\frac{1}{\alpha^2+\beta^2})^2}\\
       &= \frac{\beta^2 -\alpha^2}{(\alpha^2+\beta^2)}\hat{i}+(-\frac{2\alpha\beta}{(\alpha^2+\beta^2)})\hat{j}\notag
       \end{aligned}
    \end{equation}
    \begin{equation}
    \begin{aligned}
       \hat{h_2} &= \frac{\vec{h_2}}{||h_2||}
       = \frac{(-\frac{2\alpha\beta}{(\alpha^2+\beta^2)^2})\hat{i}+\frac{\alpha^2 - \beta^2}{(\alpha^2+\beta^2)^2}\hat{j}}{(\frac{1}{\alpha^2+\beta^2})^2}\\
       &= -\frac{2\alpha\beta}{(\alpha^2+\beta^2)})\hat{i}+\frac{\alpha^2 - \beta^2}{(\alpha^2+\beta^2)}\hat{j}\notag
       \end{aligned}
    \end{equation}
    \begin{equation}
    \begin{aligned}
       \hat{h_3} &= \frac{\vec{h_3}}{||h_3||}
       = \frac{\vec{k}}{1}\\
       &= \vec{k}
       \end{aligned}
    \end{equation}
    \item According to the Definition of Gradient:\\
    \begin{equation}
    \begin{aligned}
    \nabla\Phi &= \frac{1}{||\vec{h_1}||}\frac{\partial \Phi}{\partial\alpha}\hat{h_1}+\frac{1}{||\vec{h_2}||}\frac{\partial \Phi}{\partial\beta}\hat{h_2}+\frac{1}{||\vec{h_3}||}\frac{\partial \Phi}{\partial\zeta}\hat{h_3}\\
    &= (\alpha^2 + \beta^2)\frac{\partial \Phi}{\partial\alpha}\hat{h_1} + (\alpha^2 + \beta^2)\frac{\partial \Phi}{\partial\beta}\hat{h_2}+\frac{\partial \Phi}{\partial\zeta}\hat{h_3}\notag
      \end{aligned}
    \end{equation}
    \end{itemize}
\section{Exercise Cylinder coordinates}\label{intro}
\subsection{Express the vector field in standard cylinder coordinates and determine the curl and the divergence.}
\[V = 
\begin{pmatrix}
    -\omega y\\
    \omega x\\
    1 - x^2 - y^2
\end{pmatrix}
with\ \omega>0\]
\begin{itemize}
    \item According to the relationship between cylindrical and Cartesian coordinates system, we could get:\\
    \[\begin{bmatrix}
        \vec{i}\\
        \vec{j}\\
        \vec{k}
    \end{bmatrix} = 
    \begin{bmatrix}
        \cos{q_2} & -\sin{q_2} & 0\\
        \sin{q_2} & \cos{q_2} & 0\\
        0 & 0 & 1
    \end{bmatrix}
    \begin{bmatrix}
        \hat{h_1}\\
        \hat{h_2}\\
        \hat{h_3}
    \end{bmatrix}\]
    \[\vec{i} = \cos{q_2}\hat{h_1} - \sin{q_2}\hat{h_2},\ \vec{j} = \sin{q_2}\hat{h_1} + \cos{q_2}\hat{h_2},\ \vec{k} = \hat{h_3}\]
    \[x = q_1\sin{q_2}\cos{q_3},\ y = q_1\sin{q_2}\sin{q_3},\ z = q_1\cos{q_2}\]
    \item So, we could get new \vec{V} in cylindrical coordinates system:
    \begin{equation}
    \begin{aligned}\vec{V} &= -\omega y\vec{i} + \omega x\vec{j}+(1 - x^2 -y^2)\vec{k}\\
    &= -\omega(q_1\sin{q_2}\sin{q_3})(\cos{q_2}\hat{h_1} - \sin{q_2}\hat{h_2}) + \omega(q_1\sin{q_2}\cos{q_3})(\sin{q_2}\hat{h_1} + \cos{q_2}\hat{h_2}) \\
    &+(1 - (q_1\sin{q_2}\cos{q_3})^2 - (q_1\sin{q_2}\sin{q_3})^2)\hat{h_3}\\
    &= \omega q_1(\sin^2q_2\hat{h2} - \cos{q_2}\sin{q_2}\hat{h_1}) + \omega q_1(\cos{q_2}\sin2q_2\hat{h1} + \cos^2q_2\hat{h_2}) +(1 - q_1^2)\hat{h_3}\\
    &= \omega q_1\hat{h_2} + (1-q_1^2)\hat{h_3}(\omega > 0)\notag
    \end{aligned}
    \end{equation}
    \item According to the Definition about Curl in Cylinder Coordinates, the vector has the curl:
    \begin{equation}
    \begin{aligned}
    curl\ \vec{V} &= (\frac{1}{q_1}\frac{\partial v_3}{\partial q_2} - \frac{\partial v_2}{\partial q_3})\hat{h_1} +(\frac{\partial v_1}{\partial q_3} - \frac{\partial v_3}{\partial q_1})\hat{h_2}+(\frac{1}{q_1}\frac{\partial}{\partial q_1}(q_1v_2) - \frac{1}{q_1}\frac{\partial v_1}{\partial q_2})\hat{h_3}\\
    &= (0 - 0)\hat{h_1} + (2q_1)\hat{h_2}+(\frac{1}{q_1}(2\omega q_1) - 0)\hat{h_3}\\
    &= 2q_1\hat{h_2}+2\omega\hat{h_3}(\omega > 0)\notag
    \end{aligned}
    \end{equation}
    \item According to the Definition about Divergence in Cylinder Coordinates, the vector has the divergence:
    \begin{equation}
    \begin{aligned}
    div\ \vec{v} &= \frac{1}{q_1}\frac{\partial}{\partial q_1}(q_1v_1) + \frac{1}{q_1}\frac{\partial v_2}{\partial q_2}+ \frac{\partial v_3}{\partial q_3}\\
    &= 0 + 0 + 0\\
    &= 0
    \end{aligned}
    \end{equation}
    
\end{itemize}

\section{Exercise Matlab}\label{intro}
\subsection{Use the 3 data sheets (AISIceSheetice, AIUBCHAMP01Sgeoid and weathermodelwinds) on Matlab to plot and visualize the data. Load the data directly and use the metadatas to understand what is inside each file.\\
\begin{itemize}
    \item Use the X, Y and CHANGES we could build:\\
    \begin{figure}[!htb]
        \centering
        \includegraphics[height=5.5cm,width=8cm]{ICE.png}
        \caption{ICE Change}
        \label{fig:my_label}
    \end{figure}
    \item Use the LON, LAT and HEIGHT we could build:\\
    \begin{figure}[!htb]
        \centering
        \includegraphics[height=5.5cm,width=8cm]{Geoid.png}
        \caption{Geoid Height}
        \label{fig:my_label}
    \end{figure}
    \item Use the X, Y, Z, U, V and W we could build:\\
    \begin{figure}[!htb]
        \centering
        \includegraphics[height=5.5cm,width=8cm]{Winds.png}
        \caption{Weather Model- Winds}
        \label{fig:my_label}
    \end{figure}
\end{itemize}
Then, compute grad, div and curl operations. Combine them is possible to check that div(curl) and curl(grad) are equal to 0. Give for each calculation a physical explication. }



\end{document}


