% 等比数列(高中)
% keys 高中|等比数列
% license Usr
% type Tutor

\begin{issues}
\issueDraft
\end{issues}

\pentry{数列 \nref{nod_HsSeFu},等差数列 \nref{nod_HsAmPg}}{nod_d6a2}

在介绍\aref{数列和}{sub_HsSeFu_1}时,曾提到国际象棋棋盘和麦粒的故事。故事中提到:“依此类推,每一格的麦粒数量是上一格的两倍。”尽管这一描述看似简单,但敏锐的读者可能已经注意到,如果将每一格的麦粒数量视为一个数列中的一项,那么这个“二倍”实际上就是数列生成的递推规则。换句话说,数列的每一项都是前一项的若干倍,这种生成关系构成了等比数列的基本特征。与最简单的等差数列相比,等比数列的规则从“差”转变为“比”,这一变化看似微小,却使得数列的增长形式从线性变为指数。

这一规律在自然界和生活中非常常见。例如,细胞分裂的数量倍增规律、银行存款的复利计算、声音传播中强度的逐步衰减等现象都符合相似的模式:每个阶段的变化都是在上一次的基础上按比例增加或减少。虽然这些场景的具体数值不同,但它们背后的数学本质却是一致的,都是通过固定的比值连接各阶段的变化过程。人们将具有这种倍数特征的数列称为等比数列。等比数列不仅在基础数学中具有重要地位,其延伸——等比级数,在大学数学中更是研究无限序列和收敛性的核心工具。等比级数广泛应用于物理学、经济学等领域,用于描述振荡、增长、衰减等复杂现象。

在学习等差数列的过程中,读者已经熟悉了递推公式、通项公式以及数列和的推导方法,同时也了解了数列问题研究的核心内容和常见思路。在等比数列的探索中,这些经验同样可以迁移应用。接下来的研究思路,沿用了学习等差数列时的逻辑框架,相信读者一定会感到非常熟悉。但需要特别注意“比”的固定以及由此引发的性质和增长形式的差异,这也是深入理解等比数列的关键所在。

\subsection{等比数列}

前面提到的种种例子,无论是细胞分裂、银行存款计息,还是棋盘上的麦粒故事,其核心规律都是相同的,即“按比例增长或减少”的过程。本质上,这些现象都是通过相邻项的固定比值将整个变化过程紧密连接起来。基于这一规律,可以明确等比数列的递推定义:

\begin{definition}{等比数列}
如果数列 $\{a_n\}$ 满足对于 $n > 1$ 的所有项\footnote{通常对于数列而言,$n>2$才有意义,此处不讨论$n\leq2$的情况。},每一项与前一项的比为同一个常数 $q(q\neq0)$,则称 $\{a_n\}$ 为\textbf{等比数列(geometric sequence)},$q$ 称为 $\{a_n\}$ 的\textbf{公比(common ratio)},即等比数列满足递推公式:
\begin{equation}
a_{n}=a_{n-1}\cdot q\qquad( a_1\neq0,q\neq0,n>1)~.
\end{equation}
\end{definition}
在等比数列的递推公式中,需要注意首项 $a_1 \neq 0$。如果允许 $a_1 = 0$,根据递推公式可以推导出 $a_2 = 0$,这将会产生两个问题:
\begin{itemize}
\item 分母的问题:由于 $0$ 不能作为分母,这会使得数列无法满足“每一项与前一项的比为同一个常数”的定义要求。
\item 结果的模糊性:$a_1=a_2=0$,进而推出数列中任意项 $a_n$ 都为 $0$。但在这种情况下,即使将 $a_n$ 写成倍数的形式,也无法保证比值 $q$ 是一个固定的常数。事实上,这样的数列可以得到任意的比值 $q$,从而失去了定义的严谨性。
\end{itemize}
此外,由于等比数列具有“任意连续部分截取出来仍满足定义”的性质,这意味着等比数列中的任何一项都不能为 $0$,否则将无法保证其满足定义。特别地,所有非零的常数列都可以看作 $q = 1$ 的等比数列。

根据定义,迭代递推公式可以推导等比数列的通项公式。当 $n > 1$ 时,利用递推公式 $a_n = a_{n-1} \cdot q$,逐步展开数列的前一项,可以得到:
\begin{equation}\label{eq_HsGmPg_3}
\begin{aligned}
a_n &= a_{n-1} \cdot q\\
&= a_{n-2}\cdot q\cdot q\\
&\cdots \\
&= a_2\cdot q^{n-2}\\
&=  a_1 \cdot q^{n-1}~.
\end{aligned}
\end{equation}
可以看出,乘积的递推关系最终转化为公比 $q$ 的指数形式。这也解释了在麦粒故事中,为什么会提到最终的结果背后蕴含着“指数的力量”。接下来,检查边界条件,当 $n = 1$ 时,将其代入\autoref{eq_HsGmPg_3},由于$q\neq0$显然有:
\begin{equation}
a_1=a_1\cdot q^{1-1}=a_1~.
\end{equation}
显然,边界条件与公式完全吻合。综上,可以得出等比数列的通项公式:
\begin{corollary}{等比数列通项公式}
对等比数列$\{a_n\}$,其通项公式为:
\begin{equation}
a_n = a_1 q^{n-1} \quad (a_1\neq0, q\ne 0,\ n=1,2,3\dots)~
\end{equation}
其中,$a_1$ 是首项,$q$ 是公比,$n$ 是项数。
\end{corollary}

与等差数列的判断方法相同,既可以利用通项公式,也检测该数列是否满足定义的条件。再次提醒,后一种方法虽然在实际应用中常被忽略,但同样有效。首项与公比可以是唯一确定一个等差数列。另外,根据前面的描述如果已知一个数列有$0$项,那么它一定不是等比数列。

\subsection{等比数列的性质}

观察\autoref{ex_HsAmPg_3} 可以发现,其形式与一次函数完全一致。其实,通过变形等差数列的通项公式,可以得到以下形式:
\begin{equation}
a_n=dn+(a_1-d)~.
\end{equation}
在\enref{数列}{HsSeFu}中曾介绍过数列与函数的关系,\autoref{eq_HsAmPg_9} 与\autoref{ex_HsAmPg_3} 都可以说明,等差数列实际上是一次函数的离散形式。换句话说,在直角坐标系中,如果将数列的项数 $n$ 作为横坐标,数列的值 $a_n$ 作为纵坐标,则点 $(n, a_n)$ 将分布在一条直线上。

由于直线上任意一点和斜率即可确定整条直线,对于等差数列,同样可以不难证明,只需知道任意一项及公差,就能够确定整个数列,满足以下关系式:
\begin{equation}\label{eq_HsGmPg_5}
a_n=a_k\cdot q^{n-k}~.
\end{equation}
进而,可以推导出另一个重要关系:
\begin{equation}
q=\left({a_n\over a_k}\right)^{1\over n-k}~.
\end{equation}



\begin{corollary}{等比数列的增减性}
对于公比为 $q$ 的等比数列 $\{a_n\}$:
\begin{itemize}
\item 如果 $q > 0$,则 $\{a_n\}$ 是递增数列;
\item 如果 $q < 0$,则 $\{a_n\}$ 是递减数列。
\end{itemize}
\end{corollary}

这也符合指数函数的增减性的判断。

这一点也可以从等差数列的递推公式变形得到的$a_{n+1}-a_n=d$中得到印证。

另外,根据\autoref{eq_HsGmPg_5} ,还可以得到等比数列的一个重要性质:取四个整数,满足 $m+n=p+q$,即 $p-n=m-q$,则有:
\begin{equation}
{a_p-a_n\over p-n}=d={a_m-a_q\over m-q}~.
\end{equation}
带入两侧分母相等的条件可以得到:
\begin{corollary}{}
对于等差数列$\{a_n\}$,若满足$m+n=p+q$,则有:
\begin{equation}
a_m+a_n=a_p+a_q~.
\end{equation}
\end{corollary}

与等差数列类似,如果在 $a$ 和 $b$ 中插入一个数 $G$,使得 $a,G,b$ 成等比数列,即:
\begin{equation}
{G\over a} = {b\over G}~.
\end{equation}
则称 $G$ 为 $a$ 与 $b$ 的\textbf{等比中项(median of an geometric sequence)}。显然有
\begin{equation}
G^2 = ab~.
\end{equation}
由于等比数列不含$0$,可以得到$ab>0$,即二者同号。因此,拥有等比中项的两个数一定是同号的,且有:
\begin{equation}
G = \pm \sqrt{ab}~.
\end{equation}
这说明,$a$ 与 $b$ 为正的等比中项正好是他们的\textbf{几何平均值(geometric mean)}\footnote{关于几何平均值参见\aref{基本不等式}{sub_HsIden_2}。}。反过来,从\autoref{eq_HsAmPg_10} 也可以得到:
\begin{equation}
2A = a+b~.
\end{equation}


如果在 $a$ 和 $b$ 之间插入一个数 $A$,使 $a, A, b$ 成等差数列,即:
\begin{equation}
A - a = b - A~.
\end{equation}
则称 $A$ 为 $a$ 与 $b$ 的\textbf{等差中项(median of an arithmetic sequence)}。显然,可以得到:
\begin{equation}
A = \frac{a+b}{2}~.
\end{equation}
这说明,$a$ 与 $b$ 的等差中项正好是他们的\textbf{算术平均值(arithmetic mean)}\footnote{关于算术平均值参见\aref{基本不等式}{sub_HsIden_2}。}。反过来,从\autoref{eq_HsAmPg_10} 也可以得到:
\begin{equation}
2A = a+b~.
\end{equation}
结合\autoref{cor_HsAmPg_1} 可以知道,对于一个等差数列,如果某两项的项数和恰好是另一项的两倍,那么这第三项就是那两项的等差中项。不过需要注意,这要求那两项的项数和必须是偶数,否则无法在等差数列中找到符合条件的项\footnote{与此不同的是,对于直线,无论什么情况下都可以找到对应的点。}。

从上述内容可以看出,“等差中项”这个名字可以从两个角度理解:一方面,它可以指在两个数之间插入的一个数,与它们构成等差关系;另一方面,它也可以指等差数列中,处于某两项中间位置的那项。
\subsubsection{等比中项}
与等差数列类似,如果在 $a$ 和 $b$ 中插入一个数 $G$,使得 $a,G,b$ 成等比数列,那么根据等比数列的定义, $\frac{G}{a} = \frac{b}{G},G^2 = ab,G = \pm \sqrt{ab}$。我们称 $G$ 为 $a,b$ 的\textbf{等比中项}。

易得,在等比数列中,首末两项除外,每一项都是它前后两项的等比中项。

\subsection{等比数列的数列和}


讨论完等比数列的性质后,接下来研究其数列和的计算方法。等比数列求和与等差数列求和有相似之处,都是将S展开后求和或差,然后利用数列的性质来处理展开部分中相同的内容。在学习数列时,曾提到过数列和的另一个\aref{性质}{eq_HsSeFu_3}。基于这一性质,取$k=q$可以得到:

\begin{equation}\label{eq_HsGmPg_1}
\begin{aligned}
qS_n &= qa_1 + qa_2 + \cdots qa_{n-1}+ qa_n\\
&=a_2 + a_3 + \cdots + a_n+qa_n~.
\end{aligned}
\end{equation}
而$S_n$展开后有许多相似的部分:
\begin{equation}\label{eq_HsGmPg_2}
S_n = a_1 + a_2 + \cdots + a_n~,
\end{equation}
将\autoref{eq_HsGmPg_1} 与 \autoref{eq_HsGmPg_2} 做差,可以消掉这些相同的部分,得到:
\begin{equation}\label{eq_HsGmPg_4}
\begin{aligned}
(1 - q)S_n &= a_1 - qa_n\\
&= a_1(1 - q^n)~,
\end{aligned}
\end{equation}
下面针对$q$的取值进行讨论:如果$q\neq1$,那么,将\autoref{eq_HsGmPg_4} 中的$(1-q)$除到左边就可以得到:
\begin{equation}
S_n = \frac{a_1(1-q^n)}{1-q} \quad (q\neq 1)~.
\end{equation}
检验边界条件,在\autoref{eq_HsGmPg_4} 中带入$n=1$,左右的$(1-q)$可以约掉,显然成立。
如果$q=1$,则由于\autoref{eq_HsGmPg_4} 两侧均为0,无法得出$S_n$的表达式。但此时,等比数列事实上成为了常数列,即$a_n=a_1$,因此,求和结果为:
\begin{equation}
S_n=na_n~.
\end{equation}
边界条件也显然成立。因此综上有:

\begin{corollary}{等比数列和的通项公式}
对等比数列$\{a_n\}$,其数列和$\{S_n\}$的通项公式为:
\begin{equation}
S_n = \begin{cases}
\displaystyle\frac{a_1(1-q^n)}{1-q} ,\quad &(q \neq 1)\\
na_1, \quad &(q = 1)~.
\end{cases}
\end{equation}
其中,$a_1,q$ 是等差数列的首项和公比,$n$ 是项数。
\end{corollary}



\subsubsection{*等比级数}

如果取等比数列和的项数$n\rightarrow \infty$,就称为\textbf{等比级数},此时如果$0<|q|<1$,则能得到收敛的结果:
\begin{equation}
S = \frac{a_1}{1 - q}~.
\end{equation}
否则,这个结果会发散。这里的极限详见 “\enref{数列的极限(简明微积分)}{Lim0}”