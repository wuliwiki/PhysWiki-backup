% 竖直面内的圆周运动(高中)
% keys 高中物理|圆周运动
% license Usr
% type Tutor

\pentry{圆周运动\nref{nod_HSPM05}}{nod_HSPM04}

\subsection{引言}
本文主要介绍竖直平面内变速圆周运动的临界问题。这类问题常有如下设问:“如图所示........若物块不脱离轨道,求轨道半径R的取值范围”。这个问题也是高中物理的一个重要模型,常作为综合性大题的一个组成部分,与动能定理、动量定理、带电粒子运动、电磁感应等知识共同用作命题。

\subsection{能量观点的引入}
常见的竖直面内圆周运动模型有绳模型、杆模型、轨道模型等,无论什么模型,竖直面内受重力作用的物体总是做变速圆周运动,想要解决与之有关的临界问题,至少在高中范围内,不用能量,无从下手。
现在考察一个具体物理情境与实例。
\begin{figure}[ht]
\centering
\includegraphics[width=6cm]{./figures/1cdd7a99f5c6bf3a.png}
\caption{轻绳模型} \label{fig_CirVer_1}
\end{figure}
如图1所示,一质量为 m 的小球从半径为R 的固定光滑圆轨道的最低点A点,以速度v0向
右运动,若它运动过程中不脱离轨道,试分析(大致想象)小球的运动情景。
