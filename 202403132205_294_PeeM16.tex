% 2016 年考研数学试题(数学一)
% keys 考研|数学
% license Copy
% type Tutor

\subsection{选择题}

\begin{enumerate}
\item 若反常积分 $\displaystyle \int_{0}^{+\infty} \frac{1}{x^a(1+x)^b}\dd{x}$ 收敛,则$(\quad)$\\
(A) $a<1$ 且 $b>1 \qquad$
(B) $a>1$ 且 $b<1$\\
(C) $a<1$ 且 $a+b>1 \quad$
(D) $a>1$ 且 $a+b>1$
\item 已知函数 $f(x)=\leftgroup{&2(x-1), &x<1\\ &\ln x, &x \ge 1}$ ,则 $f(x)$ 的一个原函数是  $(\quad)$\\
(A)$F(x)=\leftgroup{&(x-1)^2, &x<1\\&x(\ln x-1),&x \ge 1}$\\
(B)$F(x)=\leftgroup{&(x-1)^2, &x<1\\&x(\ln x+1)-1,&x \ge 1}$\\
(C)$F(x)=\leftgroup{&(x-1)^2, &x<1\\&x(\ln x+1)+1,&x \ge 1}$\\
(D)$F(x)=\leftgroup{&(x-1)^2, &x<1\\&x(\ln x-1)+1,&x \ge 1}$\\
\item 若  $y=(1+x^2)^2-\sqrt{1+x^2},y=(1+x^2)^2+\sqrt{1+x^2}$  是微分方程 $y'+p(x)y=q(x)$  的两个解,则 $q(x)= (\quad)$\\
(A) $3x(1+x^2) \quad$
(B) $-3x(1+x^2) \quad$
(C) $\displaystyle \frac{x}{1+x^2} \quad$
(D) $\displaystyle -\frac{x}{1+x^2} \quad$
\item 已知函数 $f(x)=\leftgroup{&x, &&x \le 0,\\&\frac{1}{n},&&\frac{1}{n+1}<x \le \frac{1}{n},n=1,2,\dots}$  则 $(\quad)$\\
(A) $x=0$ 是 $f(x)$  的第一类间断点\\
(B) $x=0$ 是 $f(x)$  的第二类间断点\\
(C) $f(x)$ 在 $x=0$ 处连续但不可导\\
(D) $f(x)$ 在 $x=0$ 处可导
\item  设 $\mat A,\mat B$  是可逆矩阵,且 $\mat A$ 与 $\mat B$相似,则下列结论错误的是 $(\quad)$ \\
(A)$\mat A \Tr$ 与 $\mat B \Tr$ 相似 \\
(B)$\mat A^{-1}$ 与 $\mat B^{-1}$ 相似\\
(C)$\mat A+\mat A \Tr$ 与 $\mat B+\mat B \Tr$ 相似\\
(D)$\mat A +\mat A^{-1}$ 与 $\mat B +\mat B^{-1}$ 相似
\item 设二次型  $f(x_1,x_2,x_3)=x_1^2+x_2^2+x_3^2+4x_1x_2+4x_1x_3+4x_2x_3$  ,则  $f(x_1,x_2,x_3)=2$  在空间直角坐标下表示的二次曲面为$(\quad)$\\
(A)单叶双曲面 $\qquad$
(B)双叶双曲面$\qquad$
(C)椭球面$\qquad$
(D) 柱面

\item 设随机变量 $X$~$N(\mu,\sigma^2)(\sigma>0)$  ,记  $p=P\{X \le \mu +\sigma ^2\}$ ,则 $(\quad)$\\
(A)$p$ 随着 $\mu$ 的增加而增加  $\quad$
(B)$p$ 随着 $\sigma$ 的增加而增加\\
(C)$p$ 随着 $\mu$ 的增加而减少 $\quad$
(D)$p$ 随着 $\sigma$ 的增加而减少
\item 随机试验 $E$ 有三种两两不相容的结果 $A_1,A_2,A_3$,且三种结果发生的概率均为$\frac1 3$,将试验 $E$ 独立重复做2次,$X$表示2次试验中结果 $A_1$ 发生的次数,$Y$表示2次试验中结果 $A_2$ 发生的次数,则$X$ 与 $Y$的相关系数为  $(\quad)$\\
(A)-$\frac1 2$ $\qquad$
(B)-$\frac1 3$ $\qquad$
(C)$\frac1 3$ $\qquad$
(D)$\frac1 2$
\end{enumerate}

\subsection{填空题}
\begin{enumerate}
\item $\displaystyle \lim_{{x\to 0 }} \frac{\int_0^x t\ln(1+t\sin t)\dd{t}}{1-\cos x^2} $=$(\quad)$
\item 向量场 $\bvec A(x,y,z)=(x+y+z)\bvec i+xy \bvec j+z \bvec k$ 的旋度 $\bvec {rot} \bvec A$ =$(\quad)$
\item 设函数 $f(u,v)$ 可微,$z=z(x,y)$   由方程 $(x+1)z-y^2=x^2f(x-z,y)$  确定,则 $\dd{z}|_{(0,1)}$=$(\quad)$
\item 设函数 $\displaystyle f(x)=\arctan x-\frac{x}{1+ax^2}$  且 $f''(0)=1$ ,则 $a=(\quad)$
\item 行列式 $\vmat{\lambda&-1&0&0\\0&\lambda&-1&0\\0&0&\lambda&-1\\4&3&2&\lambda+1}$=$(\quad)$
\item 设 $X_1,x_2,\dots,X_n$ 为来自总体 $N(\mu,\sigma^2)$  的简单随机样本,样本均值 $\bar X=0.95$  ,参数 $\mu$ 的置信度为$0.95$的双侧置信区间的置信上限为$10.8$,则 $\mu$ 的置信度为$0.95$的双侧置信区间为 $(\quad)$
\end{enumerate}


\subsection{简答题}
\begin{enumerate}
\item 已知平面区域 $\displaystyle D=\{(r,\theta)|2 \le r \le 2(1+\cos \theta),-\frac{\pi}{2} \le \theta \frac{\pi}{2}\}$  ,计算二重积分 $\displaystyle \int\int_D x\dd{x}\dd{y}$ 。
\item 设函数 $y(x)$ 满足方程 $y''+2y'+ky=0$ ,其中 $0<k<1$。\\
(1) 证明:反常积分  $\displaystyle \int_0^{+\infty} y(x)\dd{x}$ 收敛;\\
(2)若 $y(0)=1,y'(0)=1$  ,求 $\displaystyle \int_0^{+\infty} y(x)\dd{x}$ 的值。
\item 设函数 $f(x,y)$ 满足 $\displaystyle \pdv{f(x,y)}{x}=(2x+1)e^{2x-y}$ ,且 $f(0,y)=y+1$  ,$L_t$ 是从点 $(0,0)$ 到点 $(1,t)$ 的光滑曲线。计算曲线积分 $\displaystyle I(t)=\int_{L_t}\pdv{f(x,y)}{x}\dd{x}+\pdv{f(x,y)}{y}\dd{y}$ ,并求 $I(t)$ 的最小值。
\item 设有界区域 $\Omega$ 由平面 $2x+y+2z=2$ 与三个坐标平面围成,$\Sigma$  为 $\Omega$ 整个表面的外侧,计算曲面积分 $\displaystyle I=\int\int_\Sigma (x^2+1)\dd{y}\dd{z}-2y\dd{z}\dd{x}+3z\dd{x}\dd{y}$。
\item 已知函数 $f(x)$ 可导,且 $f(0)=1,0<f'(x)<\frac{1}{2}$ 。设数列$\{{x_n}\}$满足 $x_{n+1}$ 。证明:\\
(1)级数 $\displaystyle \sum_{n=1}^\infty(x_{n+1}-x_n)$ 绝对收敛;\\
(2)$\displaystyle \lim_{n \to \infty}$存在,且$\displaystyle 0<\lim_{n \to \infty}x_n<2$
\item 
\end{enumerate}
