% 万有引力定律(高中)
% 开普勒定律|万有引力|天体|宇宙速度|同步卫星

\begin{issues}
\issueDraft
\issueTODO
\end{issues}

\pentry{圆周运动\upref{HSPM05}}

\subsection{开普勒定律}

开普勒定律是开普勒根据对太阳系中行星运动的观测数据总结出来的,是一个普适定律,也适用于其他天体绕中心天体运动的情况,如卫星围绕地球的运动等.

\textbf{第一定律}:所有行星绕太阳运动的轨道都是椭圆,太阳在椭圆的一个焦点上.

\textbf{第二定律}:对任意一个行星来说,它与太阳的连线在相等时间内扫过相等的面积.

\textbf{第三定律}:所有行星的轨道半长轴($a$)的三次方跟它的公转周期($T$)的二次方之比都相等.表达式为
\begin{equation}\label{HSPM06_eq1}
\frac{a^3}{T^2}=k
\end{equation}
要注意的是,\autoref{HSPM06_eq1} 中代表“都相等“的比值$k$,是针对围绕同一中心天体运动的所有天体而言的,因为$k$的大小与中心天体的质量有关.

\subsection{万有引力定律}

\textbf{内容}:任意两个质点都相互吸引,这个引力的大小与两质点的质量的乘积成正比,与两质点的距离的平方成反比.

万有引力的大小为:
\begin{equation}\label{HSPM06_eq5}
F=G\frac{m_1m_2}{r^2}
\end{equation}
$m_1$、$m_2$分别为两质点的质量,$r$为两质点间的距离,$G$为引力常量,且$G=6.67\times 10^{-11}\mathrm{N\cdot m^2/kg^2}$

对于两个不能被视为质点的物体,它们之间的引力等于组成其中一个物体的所有质点与组成另一个物体的所有质点之间的所有引力的合力(矢量和).这种情况已经属于大学阶段研究的问题,在高中阶段,以处理可视作质点的问题为主.

这里还要注意的是,当两个物体间的距离非常小($r$趋于$0$)时,两个物体不能被看成质点, 质点的万有引力公式(\autoref{HSPM06_eq5})不再适用,进而也无法得出万有引力趋于无穷大的结论\footnote{在大学课程中会证明\upref{SphF}, 质量为 $M$, 质量呈中心对称分部的球体, 对球外任意一点的万有引力等效于球心处质量为 $M$ 的质点产生的引力. 所以在考虑例如卫星绕地球运动时, 虽然卫星离地表很近, 仍然可以把地球近似看成是球心处的一个质点.}.

\subsection{地球对物体的万有引力与物体所受重力}

首先,我们把地球近似看成半径为$R$、质量为$M$且均匀分布的球体.由于地球自转,地球上的物体会随之做圆周运动,则地球对物体的万有引力可分解为提供物体圆周运动的向心力和物体所受的重力.可见由于地球自转,万有引力和重力是存在差异的.

\subsubsection{地球表面的重力}

在地球两极处,向心力为零,此时物体所受重力等于万有引力,达到最大值,方向指向地心.

在赤道上,由$F=m\omega^2R$可知,向心力达到最大值,且与万有引力、重力同向,此时物体所受的重力最小,也指向地心.

在地球其他地面位置处,物体受到的万有引力、重力及其做圆周运动的向心力遵循平行四边形定则,重力小于万有引力且不指向地心,随纬度增大而增大.

\subsubsection{距地表高$h$处的重力(重力加速度)}

忽略地球自转时,在地球表面:
\begin{equation}\label{HSPM06_eq2}
mg=G\frac{Mm}{R^2}
\end{equation}

距离地表$h$高度处:
\begin{equation}
mg'=G\frac{Mm}{(R+h)^2}
\end{equation}

\begin{equation}
g'=(\frac{R}{R+h})^2g
\end{equation}

可见高度$h$越大,重力(重力加速度)$g'$越小

\subsubsection{距地表深$d$处的重力(重力加速度)}

距离地表$d$深度处,已知质量分布均匀的球壳对壳内物体的引力为零\footnote{可参考壳层定理\autoref{SphF_sub1}~\upref{SphF}的几何证明法,利用相似三角形的性质来理解计算.},只有地球内部半径为$R-d$的球形区域对物体有引力的作用,则:
\begin{equation}\label{HSPM06_eq3}
mg'=G\frac{M'm}{(R-d)^2}
\end{equation}
通过密度的计算可知:
\begin{equation}\label{HSPM06_eq4}
\frac{M}{R^3}=\frac{M'}{(R-d)^3}
\end{equation}
结合\autoref{HSPM06_eq2} \autoref{HSPM06_eq3} \autoref{HSPM06_eq4} 可得
\begin{equation}
g'=\frac{R-d}{R}g
\end{equation}
可见深度$d$越大,重力(重力加速度)$g'$越小

\subsection{绕天体的圆周运动}

对于绕中心天体的匀速圆周运动, 设半径为 $R$, 由万有引力提供所需的向心力, 联立\autoref{HSPM06_eq5} 和\autoref{HSPM05_eq4}~\upref{HSPM05}可得:
\begin{equation}
G\frac{Mm}{R^2}=m\frac{v^2}{R}=mR\omega^2=m\frac{4\pi^2}{T^2}R
\end{equation}
注意所有的 $m$ 都可以消去, 所以所有运动参数之间的关系($R,v,\omega,T$)和 $m$ 无关, 但和中心天体 $M$ 有关, 这和抛体运动类似. 由上式解得圆周运动的速度为
\begin{equation}\label{HSPM06_eq6}
v = \sqrt{\frac{GM}{R}}
\end{equation}
可见半径越大, 速度越小.

\subsection{天体质量、密度的计算}

对于处于天体表面的物体,在忽略地球自转的情况下,可认为物体所受的重力等于万有引力,设该天体表面的重力加速度为$g'$,则有: $GMm/R^2=mg'$, 消去 $m$ 得
\begin{equation}
g' = \frac{GM}{R^2}
\end{equation}
根据实际按上述情况列式,可计算出天体的质量 $M$, 其中第一种情况只能计算中心天体的质量.联立密度公式,可求得天体的(平均)密度.

\subsection{宇宙速度}

\textbf{第一宇宙速度}:指物体在地面附近绕地球做匀速圆周运动的速度,也叫\textbf{环绕速度},大小为$7.9\mathrm{km/s}$, 这可以由\autoref{HSPM06_eq6} 求出.在地球上发射人造卫星时,若发射速度小于该值,人造卫星将落回地面,不能绕地球做圆周运动.因此,第一宇宙速度也是人造卫星的最小发射速度. 当地表发射的航天器的初速度大于 $7.9\mathrm{km/s}$ 且小于 $11.2\mathrm{km/s}$ 时, 航天器绕地球的运动轨迹是椭圆.

\textbf{第二宇宙速度}:指物垂直地表发射的物体能摆脱地球引力的最小速度.如果该速度大于等于 $11.2\mathrm{km/s}$, 它就会彻底摆脱地球的引力, 不再绕地球运动, 因此第二宇宙速度也叫做\textbf{逃逸速度}, 大小为 $11.2\mathrm{km/s}$. 要计算第二宇宙速度, 可以令地表的引力势能等于初始动能即 $GMm/R = mv^2/2$, 得 $v = \sqrt{2GM/R}$, 这样, 当动能耗尽后, 航天器恰好摆脱地球引力势能. 然而对比\autoref{HSPM06_eq6} 可以发现, 同样的速度也可以

\textbf{第三宇宙速度}:指物体摆脱太阳引力的速度.当航天器要摆脱太阳引力飞出太阳系时,它的速度必须达到$16.7\mathrm{km/s}$.

\subsection{人造卫星}

人造卫星绕地球的运动轨道有三类:

\begin{enumerate}
\item 赤道轨道:轨道平面与赤道平面重合,卫星始终位于赤道的上空.
\item 极地轨道:轨道平面与赤道平面垂直,卫星会经过两极的上空.
\item 一般轨道:轨道平面与赤道平面成一定角度(既不为$0^\circ$也不为$90^\circ$).
\end{enumerate}

\subsubsection{地球同步卫星}

在赤道轨道中,处于地球赤道上空约$3.6\times10^4\mathrm{km}$处,有一条叫做地球静止轨道的特殊轨道.在地球静止轨道上运行的卫星叫做地球同步卫星,这种卫星相对地球是静止的,其运动周期为$24\mathrm{h}$.
