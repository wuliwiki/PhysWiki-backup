% 集合的极限
% keys 集合|极限|特征函数|并集|交集|上极限|下极限

\pentry{集合\upref{Set},极限\upref{Lim}}

\subsection{特征函数}
如果我们有可数个集合构成的族$\{A_n\}$,每个$A_n$都是编了号的集合,那么对于给定的元素,我们可以观察它是否存在于各$A_n$中.为了表达方便,我们用一个\textbf{特征函数},$\mathcal{X}$,来表达属于关系:对于元素$x$和集合$A_n$,当$x\in A_n$时$\mathcal{X}_{A_n}(x)=1$;当$x\not\in A_n$时$\mathcal{X}_{A_n}(x)=0$.

有了特征函数,我们就可以套用数列的极限来讨论集合的极限.

\subsection{上极限和下极限}
给定一列编好了号的集合:$\{A_n\}$.为了方便之后的讨论,先定义两列集合:$U_k=\underset{i\ge k}{\bigcup}A_i$,$J_k=\underset{i\ge k}{\bigcap} A_i$.

这么取的两列集合有很棒的性质:对于任何正整数$k$,$U_{k+1}\subseteq U_k$,而$J_{k+1}\supseteq J_k$.如果说真子集时比原来的集合要小的话,我们可以认为$U_k$是一个\textbf{单调不增}的集列,而$J_k$\textbf{单调不减}.这么一来,对于任何元素$x$,不管$x$是一个数字,一种动物,还是一片凋落在安田讲堂\footnote{安田讲堂是东京大学的大厅.}前的银杏叶,它的特征函数$\mathcal{X}_{U_n}(x)$\textbf{单调不增},也就是说一旦某个$U_n$不包含$x$,那么任何$m>n$的$U_m$也不包含$x$;同样地,$\mathcal{X}_{J_n}(x)$\textbf{单调不减},也就是说也就是说一旦某个$J_n$包含$x$,那么任何$m>n$的$J_m$也包含$x$.

单调函数都是必然有极限的,这么一来,任给元素$x$,$\mathcal{X}_{U_n}(x)$和$\mathcal{X}_{J_n}(x)$都有极限.那么我们可以定义$U_n$和$J_n$的极限如下:

\begin{definition}{上极限}
给定一列编好了号的集合:$\{A_n\}$,令$U_k=\underset{i\ge k}{\bigcup} A_i$.定义 $\lim\limits_{k\to\infty} U_k=U$,其中$U=\{x|\text{存在无穷多个正整数}n,\text{使}\mathcal{X}_{U_n}(x)=1\}$.称这个$U$是集列$\{A_n\}$的\textbf{上极限集合(upper limit set)}.
\end{definition}

类似地有:

\begin{definition}{下极限}
给定一列编好了号的集合:$\{A_n\}$,令$J_k=\underset{i\ge k}{\bigcap} A_i$.定义$\lim\limits_{k\to\infty} J_k=J$,其中$J$包含了所有$\mathcal{X}_{J_n}(x)$的极限是$1$的$x$.称这个$J$是集列$\{A_n\}$的\textbf{下极限集合(lower limit set)}.
\end{definition}

如果从$A_n$的视角来看的话,上极限$U$中的元素是包含于无穷多个$A_n$的.我们也可以记上极限为$U=\underset{n\to \infty}{\overline{\lim}} A_n$.

下极限$J$中的元素,从某个编号$m$开始,包含于\textbf{每一个} $n>m$ 的 $A_n$.我们也可以记下极限为$J=\underset{n\to \infty}{\underline{\lim}}A_n$.

\subsubsection{上下极限的计算}
给定一列编好了号的集合:$\{A_n\}$,则
\begin{equation}
\underset{n\to \infty}{\overline{\lim}} A_n = \overset {\infty}{\underset{k=1}{\bigcap}} \overset{\infty}{\underset{i=k}{\bigcup}} A_i = \overset{\infty}{\underset{k=1}{\bigcap}} U_k
\end{equation}
类似地
\begin{equation}
\underset{n\to \infty}{\underline{\lim}}A_n=\overset{\infty}{\underset{k=1}{\bigcup}}\overset{\infty}{\underset{i=k}{\bigcap}} A_i=\overset{\infty} {\underset{k=1}{\bigcup}} J_k
\end{equation}
这两个计算公式的依据是单调性.由于$U_k$单调不增,$J_k$单调不减,于是 $\displaystyle \overset{N}{\underset{k=1}{\bigcap}}U_k=U_N$,因此
\begin{equation}
\lim_{k\to\infty}U_k=\lim\limits_{N\to\infty}\overset{N}{\underset{k=1}{\bigcap}} U_k=\overset{\infty}{\underset{k=1}{\bigcap}}U_k
\qquad
\lim\limits_{k\to\infty}J_k=\lim\limits_{N\to\infty}\overset{N}{\underset{k=1}{\bigcup}} J_k=\overset{\infty}{\underset{k=1}{\bigcup}} J_k
\end{equation}

\subsection{极限}
给定一列编好了号的集合:$\{A_n\}$,那么它的上极限和下极限总是存在的.

但是,如果有一个元素$x$很不听话,随着$n$向无穷增大,$\mathcal{X}_{A_n}(x)$一会儿是$1$,一会儿是$0$,那就没法定义$\{A_n\}$的极限,因为你没法说清楚这个不听话的$x$究竟是不是极限集合中的元素.所以,为了良好地定义$\{A_n\}$的极限,就不允许不听话的$x$出现.

单调集列自然不存在不听话的$x$,但是单调性是一个过分强的要求.为了让集列$\{A_n\}$有极限,我们只需要它的上极限等于下极限:$\underset{n\to \infty}{\overline{\lim}} A_n=\underset{n\to \infty}{\underline{\lim}} A_n$.由于每一个$J_k\subseteq U_k$,故已经必有$\underset{n\to \infty}{\underline{\lim}} A_n\subseteq \underset{n\to \infty}{\overline{\lim}} A_n$,所以我们只要求$\underset{n\to \infty}{\overline{\lim}}A_n\subseteq \underset{n\to \infty}{\underline{\lim}} A_n$.
当$\underset{n\to \infty}{\overline{\lim}} A_n\subseteq \underset{n\to \infty}{\underline{\lim}} A_n$时,上下极限相等,我们就统一把它们叫做$\{A_n\}$的\textbf{极限(limit set)}.
