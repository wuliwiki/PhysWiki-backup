% 北京航空航天大学 2012 年数据结构与 C 语言程序设计
% 北京航空航天大学 2012 年数据结构与 C 语言程序设计


考生注意:所有答题务必书写在考场提供的答题纸上,写在本试题单上 的答题一律无效(本题单不参与阅卷).

\subsection{一、填空题}
(本题共20分,每小题各2分)

1. 从总体上说,“数据结构”课程主要研究(    )三个方面的内容.

2. 若对某线性表及常用的操作是在表中插入元素或者删除表中元素,则对于顺序存储结构和链式存储结构这两种存储结构而言,线性表应该采用(    ).

3. 在长度为$n$的非空队列中进行插入或者删除操作的时间复杂度用大$O$符号表示为(    ).

4. 若一棵度为4的树中度为1、2、3和4的结点个数分别为4、2、1和1,则该树中叶结点的个数为(    ).

5. 若某一叉树的中序遍历序列为B,A,F,D,G,C,E,按层次遍历序列为A,B,C,D,E,F,G,则该二义树的后序遍历序列为(    ).

6. 将一棵结点总数为$n$、且具有$m$个叶结点的树转换为一棵二叉树以后,该二叉树中右子树为空的结点有(    )个.

7. 对于图$G=(V,E)$与$G'=(V'E)$,若有$V'\subseteq V$,$E'\subseteq E$,则称$G'$是$G$的(    ).

8. 在顺序表(6,15,30,37,65,68,70,72,89,99)中采用折半查找法看找元素37,与表中进行过比较的元素依次是(    ).

9. 若已知$n$个关键字值具有相同的散列函数值,并且采用线性探测再散列法处理冲突,那么,将这$n$个关键字值全部散列到初始为空的地址空间中,发生散列冲突的次数是(    ).

10. 若长度为$n$的序列$K=(k_1,k_2,...,k_n)$当且仅当满足$k_i\leqslant k_{2i}$,并且$k_i\leqslant k_{2i+1} (1\leqslant i\leqslant \lfloor n/2 \rfloor)$时,则称该序列为一个小顶堆积(Heap).根据该定义,序列(26,5,77,1,61,11,59,48,15,19) 对应的小顶堆积是(     ).

\subsection{二、简答题}
(本题共20分,每小题各5分)

1. 如果一个具有100个顶点、200条边的有向图采用邻接矩阵存储,该邻接矩阵是 否是稀疏矩阵?为什么?(这里我们假设:当矩阵中非零元索的数门小「整个知阵总元 素的数目的5\%时认为该如阵为稀疏矩阵)

2. 一般情况F,建立散列表时难以避免出现散列冲突,常用处理散列冲突的方法之 是开放定址法,该方法的基木思想是什么?

3. 若对序列(2,12,列,88510)按值从小到大进行排序,前三趟排序的结果分别为: \\
第一趟排序的结果:(2J 2.16.5.10.88) \\
第二趟排序的结果:(2.12,5,10,16,88) \\
笫三趟排序的结果:(2,5,10」2,16,88) \\
请问:该结果是采用了选择排序法还是采用了(起)泡排序法得到的?为付么?

4. 快速排序法的排序过程是递归的.若待排序序列的长度为n,则快速排序的最小 递归深度与最大递归深度分别是多少?

\subsection{三、综合题}
(本题共20分,每小题各5分)

1.若非空双向循环链表中链结点结构为$llink$|$data$|$rlink$ ,则依次执行下列$4$条语句的目的是在该链表中由$q$指的结点后面插入一个由$p$指的结点,其中$1$条语句有错误,请找出该语句,并写出正确的语句.
\begin{lstlisting}[language=cpp]
p->llink=q;         /*笫1条语句*/
p->rlink=q->rlink;  /*第2条语句*/
q->rlink=p;         /*第3条语句*/
q->rlink->llink=p;  /*第4条语句*/
\end{lstlisting}

2.已知某完全二叉树的第$7$层有$10$个叶结点,请求出该完全二叉树的结点总数的最大值.(要求写出结论的求解过程)

3.证明:具有$n$个顶点的无向图最多有$n\times(n-1)/2$条边.

4.请分别写出对数据元素序列($80$,$30$,$50$,$10$,$90$,$20$)按值从大到小进行选择排序时每一趟的排序结果.

\subsection{四、算法设计题}
(本题15分)

已知某具有$n$个顶点的有向图采用邻接表方法存储,其中,用以存储有向边信息的边结点类型为
\begin{lstlisting}[language=cpp]
typedef struct edge{
    int adjvex;           /*某有向边的终止顶点在顶点结点中的位置*/
    struct edge *next;    /*指向下,个边结点*/
} ELink;
\end{lstlisting}
用以存储顶点信息的顶点结点类型为
\begin{lstlisting}[language=cpp]
typedef struct ver{
    int indegree;    /*某顶点的入度*/
    vertype vertex;  /*某顶点的数据信息*/
    ELink *link;     /*指向以该顶点为出发点的第一个边结点*/
}VLink;
\end{lstlisting}
并且$n$个顶点结点构成一个数组结构$G[0..n-1]$.请写一个算法,该算法判断给定的顶点序列$V[0..n-1]=(v_1,v_2,v_3,...,v_n)$是否是该有向图的一个拓扑序列,若是该有向图的一个拓扑序列,算法返回1,否则,算法返回0.

\subsection{五、单项选择题}
(本题共20分,每小题各2分)

1.在$C$语言中,标识符只能由字母、数字和卜划线三种字符组成,并且第一个字符(    ). \\
A.必须是字母 $\qquad$ B.必须是下划线 \\
C.必须是字母或者下划线 $\qquad$ D.可以是字母、数字和下划线之一

2.若整型变量$x$的初值为$6$,则计算表达式$"x+=x-=x*x"$之后,$x$的值是. \\
A. 50 $\qquad$ B. 60 $\qquad$ C. -50 $\qquad$ D. -60

3.下列4个程序段中,不是无限循环的是(    ). \\
A. for(b=0,a=1;a>-++b;a=k++) k=a; \\
B. for(;;a++=k); \\
C. while(1) {a++;} \\
D. for(k=10;;k--) total+=k;

4.说明 “double (*ptr)[N];“ 中的标识符ptr是(    ) \\
A.$N$个指向double类型变量的指针 \\
B.指向$N$个double类型变量的函数指针 \\
C.一个指向由$N$个double类型元素组成的一维数组的指针 \\
D.具有$N$个指针元素的一维指针数组,其每一个元素都只能指向double类型变量

5.下列4个叙述中,正确的是(    ) \\
A.char *r=“china";等价于char *r; *r="china"; \\
B.char *ptr=“china”;等价于char *ptr; ptr="china"; \\
C.char string[10]={"china"};等价于char string[lO];string[ ]={"china”}; \\
D.char str[4]="abc”,temp[4]="abc”;等价于char str[4]=temp[4]="abc";

6.在$C$程序中,语句“char *func(int x,int y);”表示(    ) \\
A.对函数func的定义 \\
B.对函数func的调用 \\
C.对函数func返回值类型的说明 \\
D.对函数func的原型说明

7.对于下列程序,若从键盘上输入:abcdef<回车>,则输出结果是(    )  \\
\begin{lstlisting}[language=cpp]
#include <stdio.h>
#include <malloc.h>
main()
{
    char *p,*q;
    p=(char *)malloc(sizeof(char)*20);
    q=p;
    scanf("%s%s",p,q);
    printf("%s%s\n",p,q);
}
\end{lstlisting}
A. defdef $\qquad$ B. abcdef $\qquad$ C. abc d	$\qquad$ D. d d

8.当说明一个结构体变量时系统分配给它的内存是()  \\
A.结构中最后一个成员所需的内存量 \\
B.结构中第一个成员所需的内存量 \\
C.成员中占内存是最大者所需的容量 \\
D.各成员所需内存量的总和