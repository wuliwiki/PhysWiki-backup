% 多元系热力学导引
% keys 多元系|平衡态
\pentry{热力学与统计力学导航\upref{StatMe},吉布斯自由能\upref{GibbsG},热动平衡判据\upref{equcri}}

让我们从几个例子出发来审视多元系问题的复杂之处,我们主要考察处于平衡态的系统\footnote{了解一个热力学系统的最好的方式是先从平衡态入手.}.设一个开口玻璃管底端有半透膜将管中糖的水溶液与容器内的水隔开,半透膜只让水透过,不让糖透过.实验发现,糖水溶液的液面比容器内水的液面上升一个高度 $h$.
\addTODO{画图!}

这个实验事实表明半透膜上方的水压要比下方的水压大 $p-p_0=\rho g h$.这一压强差也被称为渗透压.在压强差存在的情况下,两侧水的化学势“似乎”不再相同了\footnote{事实上我们应该重新审视多元系化学势、压强、温度的定义和关系.}.之前我们得到的单元体系平衡条件很难用来分析这种情况,很难用来解释为什么半透膜的上下表面会产生一个压强差,更别提定量计算了.

另一个例子是饱和蒸气压\upref{VaporP}.实验表明,水的饱和蒸气压随水温度变化而变化(处于气液相变平衡线上,它的方程可以由克拉伯龙方程\upref{Clapey}计算),在这个压强和温度下,蒸气和液体可以平衡共存.但让我们来考虑这样一个问题,如果蒸气中混入了空气,会变成什么情况呢?考虑水蒸气中混入了大量空气,水蒸气的量变的微不足道,而同时维持体系的压强不变.这时系统显然不处于相平衡.那么问题来了,混入了空气后水蒸气的化学势是否改变了\footnote{这里已经在启发我们:我们的确需要审视化学势以及吉布斯函数的概念,这些热力学量在多元体系中将变得很不一样.}?

下面我们将对多元系重新建立热力学函数和热力学方程.

\subsection{热力学函数}

设有 $k$ 种组元(例如冰、水、糖是互不相同的组元,混合的氮气和氧气是不同的组员),选 $T,P,n_1,\cdots,n_k$ 为状态参量,$n_i$ 为第 $i$ 个组员的物质的量.三个基本热力学函数为 $V,U,S$——体积、内能、熵:

\begin{equation}\label{mulTh_eq1}
\begin{aligned}
V=V(T,P,n_1,\cdots,n_k)\\
U=U(T,P,n_1,\cdots,n_k)\\
S=S(T,P,n_1,\cdots,n_k)
\end{aligned}
\end{equation}

根据前面提到的糖水、水蒸气与空气等混合物的例子,我们不能再同以前一样简单地定义化学势(即摩尔吉布斯自由能\upref{GibbsG}),体积、内能、熵也是如此.但我们可以对\autoref{mulTh_eq1}  这里我们定义\textbf{偏摩尔体积、偏摩尔内能和偏摩尔熵}.

\begin{equation}
\begin{aligned}
v_i=\qty(\frac{\partial V}{\partial n_i})_{T,P,n_j}\\
u_i=\qty(\frac{\partial U}{\partial n_i})_{T,P,u_j}\\
s_i=\qty(\frac{\partial S}{\partial n_i})_{T,P,s_j}
\end{aligned}
\end{equation}

下角标的 $T,P,n_j$ 表示在 $T,p,n_j(j\neq i)$ 不变的情况下热力学函数对 $n_i$ 的偏微商.

$V,U,S$ 是关于 $n_1,\cdots,n_k$ 的齐次函数\footnote{当所有的 $n_i$ 扩大 $\lambda$ 倍,系统各组分的比例不变,容易理解其广延量也扩大 $\lambda$ 倍}.所以根据齐次函数欧拉定理,我们有
\begin{equation}
V=\sum_i n_iv_i,U=\sum_i n_iu_i,S=\sum_i n_is_i
\end{equation}

对吉布斯函数\upref{GibbsG} ,可以定义偏摩尔吉布斯函数(称它为 $i$ 组元的化学势),则有:
\begin{equation}
\mu_i=\qty(\frac{\partial G}{\partial n_i})_{T,P,n_j}
\end{equation}

\begin{equation}
G=\sum n_i \mu_i
\end{equation}
\addTODO{增加混合理想气体的词条和例子}

\subsection{多元系热力学方程}
热力学第一定律告诉我们 $\dd U=T\dd S-p\dd V$,

在所有组元的量不发生变化的条件下,我们有