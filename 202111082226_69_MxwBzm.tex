% 麦克斯韦—玻尔兹曼分布
% keys 麦克斯韦|玻尔兹曼|速度分布|理想气体|动能|能量分布

\begin{issues}
\issueDraft
\end{issues}

\pentry{随机变量的变换\upref{RandCV}, 气体分子的速度分布\upref{VelPdf}}
\footnote{参考 \cite{新热} 以及维基百科\href{https://en.wikipedia.org/wiki/Maxwell-Boltzmann_distribution}{相关页面}.}理想气体分子的速率分布由\textbf{麦克斯韦—玻尔兹曼}分布来描述
\begin{equation}
f(v) = 4\pi \qty(\frac{m}{2\pi kT})^{3/2} v^2 \exp(-\frac{mv^2}{2kT})
\end{equation}
这是一个\textbf{概率分布函数}\upref{RandF}, 即速度模长在某个区间 $v \in [v_a, v_b]$ 的概率为
\begin{equation}
P_{ab} = \int_{v_a}^{v_b} f(v) \dd{v}
\end{equation}
假设系统中总分子数为 $N$,则速率在 $v_a$ 到 $v_b$ 范围内的分子个数为 $P_{ab}N$.如果我们对系统中所有分子的速率求平均,则平均速率为
\begin{equation}
\bar v = \int_{0}^\infty v f(v)\dd v= \sqrt{\frac{8kT}{\pi m}}
\end{equation}
速度平方平均值为
\begin{equation}
\overline {v^2} = \frac{3kT}{m}
\end{equation}
概率最大的位置为
\begin{equation}
v_p = \sqrt{\frac{2kT}{m}}
\end{equation}
动能分布为
\begin{equation}
f(E) = \frac{2}{kT}\sqrt{\frac{E}{\pi kT}} \exp(-\frac{E}{kT})
\end{equation}

\subsection{推导}
对于一个理想气体系统,我们基于以下几个基本假设来给出麦克斯韦速度分布.

\begin{enumerate}
\item 各向同性:如果我们任意地旋转系统,单个分子的速度方向改变了,但作为一个整体来说,系统中分子的\textbf{速度分布}不改变.
\end{enumerate}


\addTODO{推导, 简单方法参考新概念热学}
