% 集合的测度(实变函数)
% keys 测度|实变函数|measure|集合|广义函数|泛函

\pentry{集合\upref{Set},微积分或数学分析}

%放在实变函数还是广义函数里?或者广义函数本身作为实变函数的一个部分?

本节为实变函数的一部分,因此讨论中涉及的集合均为实数集合$\mathbb{R}^n$或者其子集.


\subsection{从Riemann积分到Lebesgue积分}

\pentry{黎曼积分与勒贝格积分\upref{Rieman}}

在微积分或数学分析中,介绍积分的时候通常都是指Riemann积分.Riemann积分的思路是\textbf{将定义域作分划},在每个分划区间里取一个代表的函数值作为“高度”,分划区间作为“底部”,构成许多“柱子”,计算这些柱子的“体积”并求和.这种思路的好处是逻辑上容易处理,只需要有极限的概念就能讨论清楚怎么积分.但是它有很多局限性,比如严重依赖函数的连续性,导致理论不完备,无法处理很多例外情况.比较典型的例子有Riemman函数和Dirichlet函数.

Lebesgue积分就是换了一种积分的思路,反过来\textbf{对值域作分划}




\subsection{点集的测度}

\subsubsection{开集的测度}



\subsubsection{任意点集的外测度}




















