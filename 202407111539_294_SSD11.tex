% 首都师范大学 2011 年硕士考试试题
% keys 首都师范大学|考研|2011年|物理
% license Copy
% type Tutor

\textbf{声明}:“改内容来源于网络公开资料,不保证真实性,如有侵权请联系管理员”

\begin{enumerate}
\item 如图所示,细直杆一端支在地面上,杆与竖直轴的来角为a:质量为m的小环套在杆上,距离轴为r,环与杆之间的滑动摩擦因数为$\mu$。试求,为保证小环做半径为r的稳定圆周运动,杆旋转角速度ω可以在什么范围内变化。
\begin{figure}[ht]
\centering
\includegraphics[width=6cm]{./figures/36c9206edf4b19a7.png}
\caption{} \label{fig_SSD11_1}
\end{figure}
\item 一质量为m的质点自质量为M的光滑斜面顶端静止下滑,滑落到底部时下落高度为h。斜面放在光滑水平面上。忽略所有的摩擦以及碰撞造成的能量损失。求:\\
(1)质点m离开斜面时,斜面M的速度为多少?\\
(2)质点下滑过程中,斜面对它的作用力做的功(不计空气阻力)。
\item 滑动摩擦因数为$\mu$的水平桌面上放置一个半径为R、质量为m的实心球。一水平拉力F作用在球体的中心轴上。为保证球体不发生滑动,水平拉力的最大值为多少?
\item 一长度为L的均匀杆,其一端固定一个质量为m的小球,另一端可以绕光滑水平轴自由摆动。当杆的摆角很小时,试求:\\
(1)如果忽略杆的质量,系统的振动周期;\\
(2)如果杆的质量为M,则系统振动的周期。
\item 半径为$r_1$上的导体球带有电荷+q,球外有一个内外半径分别为$r_1$和$r_2$的同心导体球壳,壳上带有电荷+Q。设导体球和导体球壳电势分别为$U_1$和$U_2$,求:\\
(1)空间电场分布(如图四个区间),导体球和导体球壳的电势$U_1$和$U_2$。\\
(2)若导体球接地,电荷如何分布?给出导体球壳电势$U_2$数值。
\begin{figure}[ht]
\centering
\includegraphics[width=8cm]{./figures/e2fd8f6a9c734f82.png}
\caption{} \label{fig_SSD11_2}
\end{figure}
\item 半径为R的导体球带电荷电量为+Q,外层有均匀电介质,相对介电常数为$\varepsilon_r$,外半径为 $R_2$。最外层为空气,设介质区为I区,最外层空气区为II区,求:\\
(1)I和 II区的电位移矢量$\vec D$,电场强度矢量$\vec E$和电极化强度矢量$\vec P$。\\
(2)和导体球接触的介质表面极化电荷面密度。
\item 长直单芯电缆的芯是一根半径为$ R_1$的金属实心导线,它与外壁之间充满均匀磁介质$\mu$,电流I从芯流过再沿薄金属外壁流回,外壁半径为$R_2$,厚度忽略。求:\\
(1)介质中的磁场强度矢量$\vec H$和磁感应强度矢量$\vec B$。\\
(2)与金属实心导线相邻的介质表面上的总束缚电流。
\begin{figure}[ht]
\centering
\includegraphics[width=8cm]{./figures/be95eb3b53aacd4e.png}
\caption{} \label{fig_SSD11_3}
\end{figure}
\item 一长直密绕螺线管,已知其长度$l$,半径$R_1$,匝数$N_1$。\\
(1)求其自感系数。(忽略边缘效应)。\\
(2)若在其中再放入一个长度$l$,半径$R_2(R_1>R_2)$匝数$N_2$的长直密绕螺线管,且二者共轴。
求这对共轴的两个长直螺线管之间的互感系数。(忽略边缘效应)。
\end{enumerate}
