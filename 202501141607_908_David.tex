% 大卫·希尔伯特(综述)
% license CCBYSA3
% type Wiki

本文根据 CC-BY-SA 协议转载翻译自维基百科\href{https://en.wikipedia.org/wiki/David_Hilbert}{相关文章}。

\begin{figure}[ht]
\centering
\includegraphics[width=6cm]{./figures/9019106ae7482c98.png}
\caption{1912年的希尔伯特} \label{fig_David_1}
\end{figure}
大卫·希尔伯特(David Hilbert,发音:/ˈhɪlbərt/;德语:[ˈdaːvɪt ˈhɪlbɐt];1862年1月23日 – 1943年2月14日)是德国数学家和数学哲学家,是他那个时代最具影响力的数学家之一。

希尔伯特发现并发展了广泛的基础性思想,包括不变理论、变分法、交换代数、代数数论、几何学基础、算子谱理论及其在积分方程中的应用、数学物理学,以及数学基础(特别是证明理论)。他采纳并捍卫了乔治·康托尔的集合论和超限数理论。1900年,他提出了一系列问题,为20世纪的数学研究指明了方向。

希尔伯特及其学生为建立严格的数学理论做出了贡献,并发展了现代数学物理中重要的工具。他是证明理论和数学逻辑的共同创始人。
\subsection{生活}  
\subsubsection{早期生活与教育}  
希尔伯特是奥托(Otto),一位县法官,和玛丽亚·特蕾莎·希尔伯特(Maria Therese Hilbert,原姓Erdtmann,一位商人的女儿)的长子和唯一的儿子。他出生在普鲁士省(当时属于普鲁士王国),具体地点是哥尼斯堡(根据希尔伯特本人所说)或哥尼斯堡附近的维劳(自1946年起称为兹南门斯克),当时他的父亲在该地工作。希尔伯特的祖父是大卫·希尔伯特,一名法官和秘密顾问(Geheimrat)。母亲玛丽亚对哲学、天文学和质数有兴趣,而父亲奥托则教他普鲁士的美德。父亲成为市法官后,家庭迁至哥尼斯堡。大卫的妹妹伊丽丝(Elise)在他六岁时出生。他在八岁时开始上学,比通常的入学年龄晚了两年。

1872年底,希尔伯特进入了弗里德里希中学(Friedrichskolleg Gymnasium,亦称哥尼斯堡皇家学院,是伊曼努尔·康德140年前曾就读的学校);然而,在一段不愉快的时期后,他于1879年底转学并于1880年初从更注重科学的威廉中学(Wilhelm Gymnasium)毕业。毕业后,希尔伯特于1880年秋季入读哥尼斯堡大学(“阿尔贝尔蒂纳”大学)。1882年初,赫尔曼·闵可夫斯基(Hermann Minkowski,希尔伯特比他年长两岁,同为哥尼斯堡人,但曾到柏林学习了三个学期)回到哥尼斯堡并进入了这所大学。希尔伯特与这位害羞但才华横溢的闵可夫斯基建立了终生的友谊。
\subsubsection{职业生涯}
\begin{figure}[ht]
\centering
\includegraphics[width=6cm]{./figures/2e091f1db63eaea3.png}
\caption{1886年的希尔伯特} \label{fig_David_2}
\end{figure}
\begin{figure}[ht]
\centering
\includegraphics[width=6cm]{./figures/f6ef6825f2a61abb.png}
\caption{1907年的希尔伯特} \label{fig_David_3}
\end{figure}
1884年,阿道夫·赫尔维茨(Adolf Hurwitz)从哥廷根大学来到哥尼斯堡大学,担任外籍教授(即副教授)。三人之间开始了密切且富有成果的学术交流,尤其是闵可夫斯基和希尔伯特,他们在各自的科学事业中多次互相影响。希尔伯特于1885年获得博士学位,博士论文题为《Über invariante Eigenschaften spezieller binärer Formen, insbesondere der Kugelfunktionen》(《关于特殊二元形式的不变性质,特别是球面谐波函数》),该论文是在费尔迪南·冯·林德曼(Ferdinand von Lindemann)的指导下写的。

希尔伯特于1886年到1895年期间,担任哥尼斯堡大学的私人讲师(Privatdozent)。1895年,在费利克斯·克莱因(Felix Klein)的帮助下,他获得了哥廷根大学数学教授的职位。在克莱因和希尔伯特的领导下,哥廷根大学成为了数学界的顶尖学府。他在那里度过了余生。
\subsubsection{哥廷根学派}
\begin{figure}[ht]
\centering
\includegraphics[width=8cm]{./figures/e40cafaed8f9a186.png}
\caption{哥廷根数学研究所。其新建筑由洛克菲勒基金会资助,希尔伯特和库朗于1930年共同揭幕。} \label{fig_David_4}
\end{figure}
希尔伯特的学生包括赫尔曼·外尔(Hermann Weyl)、国际象棋冠军埃马努埃尔·拉斯克(Emanuel Lasker)、恩斯特·策梅洛(Ernst Zermelo)和卡尔·古斯塔夫·亨佩尔(Carl Gustav Hempel)。约翰·冯·诺伊曼(John von Neumann)曾是他的助手。在哥廷根大学,希尔伯特与20世纪一些最重要的数学家共同工作,他的社交圈中包括了艾米·诺瑟(Emmy Noether)和阿隆佐·丘奇(Alonzo Church)等人。

希尔伯特在哥廷根的69名博士生中,有许多人后来成为了著名的数学家,包括(及其论文答辩年份):奥托·布卢门塔尔(Otto Blumenthal,1898年)、费利克斯·伯恩斯坦(Felix Bernstein,1901年)、赫尔曼·外尔(Hermann Weyl,1908年)、理查德·库朗(Richard Courant,1910年)、埃里希·黑克(Erich Hecke,1910年)、雨果·施泰因豪斯(Hugo Steinhaus,1911年)和威廉·阿克曼(Wilhelm Ackermann,1925年)。  

1902年至1939年间,希尔伯特担任《数学年刊》(Mathematische Annalen)的编辑,这是当时最重要的数学期刊之一。1907年,他被选为美国国家科学院的国际会员。
\subsubsection{个人生活}
\begin{figure}[ht]
\centering
\includegraphics[width=6cm]{./figures/734202e44a335bb0.png}
\caption{希尔伯特与他的妻子凯瑟·耶罗施(1892年)} \label{fig_David_5}
\end{figure}
\begin{figure}[ht]
\centering
\includegraphics[width=6cm]{./figures/70a233823a56b9b1.png}
\caption{弗朗茨·希尔伯特} \label{fig_David_6}
\end{figure}
1892年,希尔伯特与凯瑟·耶罗施(Käthe Jerosch,1864–1945)结婚,她是哥尼斯堡一位商人的女儿,“是一位直言不讳、思想独立的年轻女士,与希尔伯特的独立思想不谋而合。”在哥尼斯堡期间,他们有了唯一的孩子,弗朗茨·希尔伯特(Franz Hilbert,1893–1969)。弗朗茨一生饱受精神疾病困扰,在他被送入精神病诊所后,希尔伯特曾说:“从今以后,我必须认为自己没有儿子。”他对弗朗茨的态度给凯瑟带来了相当大的痛苦。

希尔伯特认为数学家赫尔曼·闵可夫斯基是他“最好的和最忠实的朋友”。

希尔伯特在普鲁士福音教会接受洗礼并成长为一名加尔文主义者。[a] 后来他离开了教会,成为了一名不可知论者。[b] 他还认为,数学真理独立于上帝的存在或其他先验假设。[c][d] 当伽利略·伽利莱因未能坚持他的日心说理论时,希尔伯特对此提出异议:“但[伽利略]并不是傻瓜。只有傻瓜才会认为科学真理需要殉道;这在宗教中或许是必要的,但科学结果终会自己证明。”[e]
\begin{figure}[ht]
\centering
\includegraphics[width=6cm]{./figures/b030fb58ef9d3b4d.png}
\caption{凯瑟·希尔伯特与康斯坦丁·卡拉塞奥多里(1932年之前)} \label{fig_David_7}
\end{figure}
\subsubsection{晚年} 
像阿尔伯特·爱因斯坦一样,希尔伯特与柏林学派保持着密切联系,该学派的主要创始人曾在哥廷根大学向希尔伯特学习(包括库尔特·格雷林(Kurt Grelling)、汉斯·赖兴巴赫(Hans Reichenbach)和瓦尔特·杜比斯拉夫(Walter Dubislav))。[18]

大约在1925年,希尔伯特患上了恶性贫血,这是一种当时无法治疗的维生素缺乏症,其主要症状是极度疲劳;他的助手尤金·维格纳(Eugene Wigner)描述他经历了“极度的疲劳”,并表示他“看起来非常衰老”。即使在最终被诊断并接受治疗后,他“几乎不再是一个科学家”,而且“毫无疑问,1925年以后他不再是希尔伯特”。[19]

希尔伯特于1932年当选为美国哲学学会会员。[20]

希尔伯特活到了纳粹在1933年清洗哥廷根大学许多著名教员的时期。[21] 被迫离开的包括赫尔曼·外尔(Hermann Weyl)(他在1930年希尔伯特退休后接替了希尔伯特的职位)、艾米·诺瑟(Emmy Noether)和埃德蒙·兰道(Edmund Landau)。其中一位不得不离开德国的保罗·伯奈耶(Paul Bernays),他曾与希尔伯特合作过数学逻辑,并共同撰写了重要著作《数学基础》(Grundlagen der Mathematik)[22](该书最终分两卷出版,分别是1934年和1939年)。这本书是希尔伯特与阿克曼(Ackermann)1928年出版的《数学逻辑原理》一书的续集。赫尔曼·外尔的继任者是赫尔穆特·哈塞(Helmut Hasse)。

大约一年后,希尔伯特参加了一场宴会,并与新任教育部长伯恩哈德·鲁斯特(Bernhard Rust)同座。鲁斯特问道:“数学研究所真的因为犹太人的离去而遭受如此重大的打击吗?”希尔伯特回答道:“打击?它不再存在了,不是吗?”[23][24]
\subsubsection{死亡}
\begin{figure}[ht]
\centering
\includegraphics[width=6cm]{./figures/46768d3dffe299bf.png}
\caption{希尔伯特的墓碑: 我们必须知道  我们将知道} \label{fig_David_8}
\end{figure}
希尔伯特于1943年去世时,纳粹几乎完全更换了哥廷根大学的教职员工,因为许多前任教员要么是犹太人,要么是犹太人的配偶。希尔伯特的葬礼只有不到十人参加,其中只有两位是同行学者,其中包括阿诺德·索末菲尔德(Arnold Sommerfeld),一位理论物理学家,也是哥尼斯堡人。[25] 他的死讯直到他去世几个月后才为外界所知。[26]

他在哥廷根墓碑上的碑文摘自他在1930年9月8日向德国科学家与医生协会发表退休演讲时所说的著名话语。这些话是对拉丁格言“Ignoramus et ignorabimus”或“我们不知道,也不会知道”的回应:[27]

Wir müssen wissen.  
Wir werden wissen.

我们必须知道。  
我们将知道。

在希尔伯特在1930年德国科学家与医生协会年会上发表这些话的前一天,库尔特·哥德尔(Kurt Gödel)——在与该协会会议共同举行的认识论会议的圆桌讨论中——初步宣布了他不完备定理的第一个表述。[f] 哥德尔的不完备定理表明,即使是像佩亚诺算术这样简单的公理系统,也要么是自相矛盾的,要么包含无法在该系统内证明或反驳的逻辑命题。
\subsection{对数学和物理的贡献}
\subsubsection{解决戈尔丹问题}  
希尔伯特的第一次关于不变函数的研究导致他在1888年证明了著名的有限性定理。二十年前,保罗·戈尔丹(Paul Gordan)曾使用复杂的计算方法证明了二次型生成元的有限性定理。试图将他的计算方法推广到具有多于两个变量的函数时,由于涉及的计算难度极大,这些尝试都失败了。为了应对这一被一些学者称为“戈尔丹问题”的难题,希尔伯特意识到必须采取完全不同的方法。因此,他证明了希尔伯特基定理,展示了任意多个变量的量子不变量的有限生成元的存在,但这种证明是抽象的。也就是说,虽然证明了这种集合的存在,但它并不是一个构造性证明——它没有展示“一个具体的对象”——而是一种存在性证明[28],并依赖于在无限扩展中使用排中律。

希尔伯特将他的结果提交给了《数学年刊》(Mathematische Annalen)。戈尔丹,这本期刊上负责不变量理论的专家,未能理解希尔伯特定理的革命性,并拒绝了这篇文章,批评其内容因为不够全面。戈尔丹的评论是:

Das ist nicht Mathematik. Das ist Theologie.  
这不是数学。这是神学。[29]

另一方面,费利克斯·克莱因(Felix Klein)认识到这项工作的重大意义,并保证这篇文章会未经修改地发表。在克莱因的鼓励下,希尔伯特在第二篇文章中扩展了他的方法,提供了最小生成元集的最大度数估计,并再次将其提交给《数学年刊》。在阅读了这篇手稿后,克莱因写信给他,表示:

Without doubt this is the most important work on general algebra that the Annalen has ever published.  
毫无疑问,这是《数学年刊》上发布的最重要的代数工作。[30]

后来,在希尔伯特方法的实用性被普遍认可之后,戈尔丹自己也表示:

I have convinced myself that even theology has its merits. 
我已经说服自己,甚至神学也有它的优点。[31]

尽管希尔伯特取得了如此多的成功,但他证明的性质却带来了比他想象的更多麻烦。虽然克罗内克(Kronecker)已经承认了这一点,希尔伯特后来回应类似的批评时曾说:“许多不同的构造都可以归结为一个基本的思想”——换句话说(引用Reid的话):“通过存在性证明,希尔伯特能够获得一个构造”;“这个证明”(即页面上的符号)就是“对象”[31]。并非所有人都信服。虽然克罗内克很快去世,但他的构造主义哲学仍然由年轻的布劳威尔(Brouwer)和他正在发展的直觉主义“学派”继承下去,这给希尔伯特的晚年带来了极大的困扰[32]。事实上,希尔伯特最终会失去他的“天才学生”外尔(Weyl),他转向了直觉主义——“希尔伯特对他曾经学生对布劳威尔思想的迷恋感到不安,这唤起了他对克罗内克的记忆”[33]。直觉主义者布劳威尔尤其反对对无限集使用排中律(正如希尔伯特所做的那样)。希尔伯特回应道:

Taking the Principle of the Excluded Middle from the mathematician ... is the same as ... prohibiting the boxer the use of his fists.
从数学家那里剥夺排中律的原则……就像是……禁止拳击手使用拳头。[34]
\subsubsection{零点定理(Nullstellensatz)}
在代数学科中,一个域被称为\textbf{代数闭合的},当且仅当它上面的每个多项式都有一个根在这个域中。在这个条件下,希尔伯特给出了一个标准,用于判断一组多项式 \((p_{\lambda})_{\lambda \in \Lambda}\) (有 \(n\) 个变量)是否有公共根:当且仅当不存在多项式 \(q_1, \dots, q_k\) 和指数 \(\lambda_1, \dots, \lambda_k\),使得:

\[
1 = \sum_{j=1}^{k} p_{\lambda_j}(\vec{x}) q_j(\vec{x})~
\]

这个结果被称为\textbf{希尔伯特根定理}(Hilbert's Root Theorem),或在德语中称为\textbf{希尔伯特零点定理}(Hilberts Nullstellensatz)。他还证明了,消失理想与其消失集之间的对应关系在仿射簇与 \( \mathbb{C}[x_1, \dots, x_n] \) 中的极小理想之间是一一对应的。
\subsubsection{曲线}
\begin{figure}[ht]
\centering
\includegraphics[width=6cm]{./figures/928fa7ce33d45ec2.png}
\caption{替代规则} \label{fig_David_9}
\end{figure}
1890年,朱塞佩·皮亚诺(Giuseppe Peano)在《数学年鉴》(Mathematische Annalen)上发表了一篇文章,描述了历史上第一个填充空间的曲线。作为回应,希尔伯特设计了他自己的这种曲线构造,现称为希尔伯特曲线。该曲线的逼近是通过迭代构造的,依据本节第一张图片中的替代规则进行。曲线本身则是逐点极限。
\begin{figure}[ht]
\centering
\includegraphics[width=6cm]{./figures/91bbeda6a9cddcff.png}
\caption{希尔伯特曲线的前六个逼近} \label{fig_David_10}
\end{figure}
\subsubsection{几何的公理化}
希尔伯特于1899年出版的《几何基础》(Grundlagen der Geometrie,英文译名《Foundations of Geometry》)提出了一组正式的公理集,称为**希尔伯特公理**,用来替代传统的欧几里得公理。这些公理避免了欧几里得公理中被发现的弱点,欧几里得的作品在当时仍作为教科书使用。要具体说明希尔伯特使用的公理,必须参考《几何基础》的出版历史,因为希尔伯特曾多次修改和调整这些公理。原版专著很快就有了法文翻译,其中希尔伯特增加了V.2,即完备性公理(Completeness Axiom)。由E.J. Townsend翻译的英文版在1902年获得了希尔伯特的授权,并注册版权。[35][36] 该翻译版本包含了法文版的修改,因此被认为是第二版的翻译。希尔伯特继续修改文本,并且该书在德文版中出现了多个版本,第七版是希尔伯特去世时的最后一版。第七版之后又有新版本出版,但主要文本基本没有修改。[g]

希尔伯特的这一方法标志着现代公理化方法的转变。在这方面,希尔伯特的工作受到了莫里茨·帕施(Moritz Pasch)1882年工作的预示。公理不被视为自明的真理。几何学可以处理一些我们直观上非常理解的事物,但不需要为这些未定义的概念指定任何明确的意义。元素,如点、线、平面等,可以被替换成,如希尔伯特对舍恩弗里斯和科特尔所说的,桌子、椅子、啤酒杯以及其他类似的物体。[37] 讨论的重点是它们之间的定义关系。

希尔伯特首先列举了未定义的概念:点、线、平面、在(点与线、点与平面、线与平面之间的关系)、相对位置、点对(线段)的全等性以及角度的全等性。这些公理将欧几里得的平面几何和立体几何统一为一个系统。
\subsubsection{23个问题}
希尔伯特在1900年巴黎国际数学家大会上提出了一份高度影响力的清单,包含23个未解问题。这通常被认为是由单个数学家提出的最成功、最深刻的未解问题集合。[来源需要]  

在重新研究经典几何的基础之后,希尔伯特有可能将其推演到整个数学领域。他的方法与后来的“基础主义者”拉塞尔–怀特黑德(Russell-Whitehead)或“百科全书派”布尔巴基(Nicolas Bourbaki)以及同时代的朱塞佩·皮亚诺(Giuseppe Peano)有所不同。整体数学界可以参与他认为是数学重要领域的关键问题的解决。

这些问题集合作为一次演讲“数学问题”在巴黎举行的第二届国际数学家大会上首次提出。希尔伯特在演讲的引言中说:

“我们之中有谁不愿意揭开隐藏未来的面纱,目睹我们科学的未来发展,探索其在未来几个世纪的发展奥秘?未来一代数学家的精神将会朝着什么目标前进?新世纪将揭示数学思想这一广阔而丰富领域中的哪些方法和新事实?”[38]

在大会上,希尔伯特提出了不到一半的问题,这些问题被收录在大会记录中。随后,他扩展了这一视野,并提出了现在被视为经典的23个希尔伯特问题。另见希尔伯特的第二十四个问题。完整的文本非常重要,因为对这些问题的解释仍然是一个不可避免的辩论话题,每当讨论有多少个问题已被解决时,都会涉及这些问题。

其中一些问题在短时间内得以解决。另一些问题则在整个20世纪被讨论过,少数问题至今仍被认为过于开放,难以得出结论。一些问题至今仍然是挑战。

以下是希尔伯特在《美国数学会公报》1902年翻译中所列的23个问题的标题:

\begin{enumerate}
\item 康托尔的连续体基数问题。
\item 算术公理的一致性。
\item 两个底面和高度相等的四面体体积相等问题。
\item 直线是两点之间最短距离的问题。
\item 不假设定义群的函数可微的李群的连续群概念。
6. 物理学公理的数学处理。
7. 某些数的无理性与超越性。
8. 素数问题(“黎曼假设”)。
9. 任何数域中最一般的互反律的证明。
10. 迪奥方程解的可解性判定。
11. 具有任意代数数值系数的二次型。
12. 克罗内克定理关于阿贝尔域的扩展到任何代数有理性领域。
13. 一般七次方程的解不能仅通过两个变量的函数求解。
14. 某些完备函数系统的有限性证明。
15. 施伯特的枚举微积分的严格基础。
16. 代数曲线和曲面拓扑问题。
17. 通过平方表示定向形式。
18. 通过全等多面体构建空间。
19. 变分法中的常规问题的解是否总是解析的?
20. 边界值问题的一般问题(偏微分方程的边界值问题)。
21. 存在具有规定单群的线性微分方程的证明。
22. 通过自同构函数对解析关系进行统一化。
23. 变分法方法的进一步发展。
\end{enumerate}