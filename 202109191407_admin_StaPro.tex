% 状态量和过程量

\pentry{力场 保守场 势能\upref{V}}

系统关于时间的变化可以看作状态空间中一点经过的轨迹. 姑且把这点叫做\textbf{状态点}. 若一个物理量 $Q$ 只和状态空间的位置有关, 即可以表示为已知状态量的函数, 那么它就也是状态量, 若它和状态空间中的运动过程有关, 它就是过程量. 典型的状态量例如系统的能量, 动量, 温度, 体积, 压强等; 典型的过程量如做功, 冲量, 传热等.

把一个系统的状态用矢量 $\bvec x = (x_1, x_2, \dots, x_N)$ 描述, 在一个特定的过程中, $x_i$ 都是时间的函数. 一些常见的过程量可以定义为
\begin{equation}\label{StaPro_eq1}
Q_{12} = \int_{\mathcal L} \sum_i f_i(x_1, \dots, x_N) \dd{x_i} = \int_{t_1}^{t_2} \sum_i f_i(x_1, \dots, x_N) \dv{x_i}{t} \dd{t}
\end{equation}
$\mathcal L$ 表示状态点的 “运动方程” $x_i(t)$ ($i = 1,\dots, N$) 以及起点终点 $\bvec x(t_1), \bvec x(t_2)$. 易得,积分结果只和路径的形状有关而与状态点在轨迹上运动的快慢无关. 所以这里的 $t$ 可以看作轨迹的参数随时间变化而未必是时间本身.

一个具体的例子是力场对单个质点的做功, 下面会在\autoref{StaPro_ex1} 详细讨论.
\begin{equation}
W_{12} = \int_{\mathcal L} \bvec F(\bvec x) \vdot \dd{\bvec x} = \int_{\mathcal L} \bvec F(\bvec x(t)) \vdot \bvec v(t) \dd{t}
\end{equation}


从定义上来说, $Q$ 是一个过程量, 但如果在某个系统中它可以表示为某个状态量的增量, 那么\textbf{对这个系统}区分 $Q$ 是状态量和过程量将没有太大实用价值(trivial): 任何状态量在不同时间的差都能看作一个这样的过程量, 反之这个过程量只要固定了起点也能变为一个状态量. 此时\autoref{StaPro_eq1} 积分的结果不取决于路径, 只取决于初末状态. 把该状态量记为 $V(\bvec x)$, 那么总有
\begin{equation}
Q_{12} = V(\bvec x(t_2)) - V(\bvec x(t_1))
\end{equation}
例如在二维或三维状态空间, 若令矢量函数为 $\bvec f(\bvec x) = \sum_i f_i(\bvec x) \uvec x_i$, 那么当旋度 $\curl \bvec f = \bvec 0$ 时, $\bvec f(\bvec x)$ 就是一个保守场, 必存在势函数 $V(\bvec x)$. 对于高维情况, 需要使用外导数\upref{ExtDer} 来判断保守场.

但事实上远非所有情况下\autoref{StaPro_eq1} 的积分都可以表示为两个状态量之差. 此时积分的结果必须取决于路径的形状, 那么区分状态量和过程量就至关重要. 例如, 虽然我们往往写出微分关系
\begin{equation}
dQ = \sum_i f_i(x_1, \dots, x_N) \dd{x_i}
\end{equation}
但是却不可能把 $Q$ 表示为 $x_i$ 的函数, $f_i$ 也不能看作偏导 $\pdv*{Q}{x_i}$.

\begin{example}{力场}\label{StaPro_ex1}
一个具体的例子是力场对单个质点的做功. 在分析力学中, 此时状态空间是 $(\bvec x, \bvec p)$ 即位置和动量, 做功的微分为
\begin{equation}
\dd{W} = \sum_i F_i(\bvec x) \dd{x_i} = \bvec F(\bvec x) \vdot \dd{\bvec x}
\end{equation}
如果力场 $\bvec F(\bvec x)$ 是保守场, 那么做功就是势能之差; 如果是非保守场, 做功只能由具体路径决定, 此时 “功”(过程量) 和 “能”(状态量) 的区分就很重要了. 例如动能总可以表示为状态 $\bvec p$ 的函数, 但做功却不行, 因为它不是状态量.
\end{example}

\begin{example}{热力学第一定律}
另一个例子是热力学第一定律\upref{Th1Law}往往记为
\begin{equation}
\dd{Q} = P\dd{V} + \dd{E}
\end{equation}
或者
\begin{equation}
Q_{12} = \int_1^2 P\dd{V} + \Delta E
\end{equation}
但状态空间中的环积分并不总是为零, 例如著名的卡诺热机\upref{Carnot}, 即积分取决于路径. 所以 $Q$ 不能看作 $V, E$ 的函数, 也不能记
\begin{equation}
\qty(\pdv{Q}{V})_E = P \qquad \qty(\pdv{Q}{E})_V = 1 \qquad \text{(错)}
\end{equation}
\end{example}
