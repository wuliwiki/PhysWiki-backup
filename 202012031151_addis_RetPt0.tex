% 电磁场推迟势
% 标势|矢势|推迟|光速

\pentry{洛伦兹规范\upref{LoGaug}}

\footnote{参考 \cite{GriffE}.}在静电学情况下($\rho(\bvec r)$ 和 $\bvec j(\bvec r)$ 不随时间变化), 标势和矢势也与时间无关, 这时它们满足(\autoref{LoGaug_eq1}~\upref{LoGaug} \autoref{LoGaug_eq2}~\upref{LoGaug} 中时间导数项为零)
\begin{equation}
\laplacian \varphi = -\frac{\rho}{\epsilon_0}
\end{equation}
\begin{equation}
\laplacian \bvec A = -\mu_0 \bvec j
\end{equation}
解的
\begin{equation}
V(\bvec r) = \frac{1}{4\pi\epsilon_0} \int \frac{\rho(\bvec r')}{\abs{\bvec r - \bvec r'}} \dd{V'}
\end{equation}
\begin{equation}
\bvec A(\bvec r) = \frac{\mu_0}{4\pi} \int \frac{\bvec j(\bvec r')}{\abs{\bvec r - \bvec r'}} \dd{V'}
\end{equation}

在非静电学情况下, 电荷密度 $\rho(\bvec r, t)$ 和电流 $\bvec j(\bvec r, t)$ 既是空间的函数也是时间的函数. 定义\textbf{推迟时间(retarded time)}为
\begin{equation}
t_r \equiv t - \abs{\bvec r - \bvec r'}/c
\end{equation}
那么可以证明(\autoref{LoGaug_eq1}~\upref{LoGaug} \autoref{LoGaug_eq2}~\upref{LoGaug}的解)
\begin{equation}
V(\bvec r) = \frac{1}{4\pi\epsilon_0} \int \frac{\rho(\bvec r', t_r)}{\abs{\bvec r - \bvec r'}} \dd{V'}
\end{equation}
\begin{equation}
\bvec A(\bvec r) = \frac{\mu_0}{4\pi} \int \frac{\bvec j(\bvec r', t_r)}{\abs{\bvec r - \bvec r'}} \dd{V'}
\end{equation}
