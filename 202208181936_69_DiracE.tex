% 电磁场中的狄拉克方程
% keys 电磁场|狄拉克方程|狄拉克场

\pentry{狄拉克方程\upref{qed4},狄拉克场\upref{Dirac}}

\subsection{量子电动力学的拉格朗日量}
让我们从狄拉克场\upref{Dirac}的拉氏量出发:
\begin{equation}
\mathcal{L}=\bar\psi (i\gamma^\mu \partial_\mu - m)\psi 
\end{equation}
我们已经知道了该方程在整体 $U(1)$ 规范变换下是不变的,即 $\psi\rightarrow e^{iq\alpha}\psi,\bar\psi \rightarrow \bar\psi e^{-iq\alpha}$ 变换下拉氏量保持不变.根据诺特定理\footnote{可以参考经典场论基础\upref{classi}.},整体规范变换对称性导致狄拉克场的电荷守恒.现在我们希望进一步地引入\textbf{定域规范不变性}.即 $\psi\rightarrow e^{iq\alpha(x)}\psi $,其中 $\alpha(x)$ 为时空坐标的函数.然而,引入这样的规范变换以后,拉氏量不再保持不变,会多出一项 $-\bar\psi \gamma^\mu \qty[q\partial_\mu \alpha(x)]\psi$.所以为了抵消掉这一项,我们约定 $\partial_\mu$ 在定域规范变换下变为 $\partial_\mu+i[q\partial_\mu \alpha(x)]$.

因此,我们引入一个协变导数算子 $D_\mu = \partial_\mu +iqA_\mu(x)$,其中 $A_\mu(x)$ 是一个矢量场,也称它为规范场.在规范变换下,$A_\mu\rightarrow A_\mu - \partial_\mu \alpha(x)$.于是定域规范不变的拉氏量可以重新写为
\begin{equation}
\mathcal{L}=\bar\psi(i\gamma^\mu D_\mu-m)\psi 
\end{equation}

因此,我们的理论中出现了两组广义坐标,一组是物质场 $\psi$ 和规范场 $A_\mu$,我们当然还可以引入规范场的拉氏量.注意到 $F_{\mu\nu} = \partial_\mu A_\nu - \partial_\nu A_\mu$ 也是定域规范不变的,而它又是一个二阶张量.因此可以利用 $F_{\mu\nu}$ 组成一个新的拉氏量:
\begin{equation}
\mathcal{L}=-\frac{1}{4}F_{\mu\nu}F^{\mu\nu} + \bar\psi (i\gamma^\mu D_\mu - m)\psi
\end{equation}
这个拉氏量生成的理论被称为\textbf{旋量量子电动力学}.注意到它比自由旋量场和自由电磁场的拉氏量还多出一项
\begin{equation}
\mathcal{L}_{\rm int}=-q A_\mu \bar\psi \gamma^\mu \psi=-A_\mu J^\mu
\end{equation}
其中 $J^\mu = q\bar\psi \gamma^\mu \psi$ 正是整体 $U(1)$ 规范对称性的守恒流,它对应的是电荷电流密度.而这一项对应了两个场耦合的相互作用拉氏量.根据上述的拉氏量所导出的两个经典方程为
\begin{equation}
\begin{aligned}
&(i\gamma^\mu D_\mu - m)\psi(x)=0\\
&\partial^\mu F_{\mu\nu}=J_\nu
\end{aligned}
\end{equation}
\subsection{非相对论极限下电磁场中的狄拉克方程}
我们常常需要讨论一个缓变的电磁场中电子的行为.将 $A_\mu$ 视为一个缓变的外场,那么电子的波函数满足方程
\begin{equation}
(i\gamma^\mu \partial_\mu -\gamma^\mu A_\mu - m)\psi(x)=0
\end{equation}
利用 $\gamma$ 矩阵的 weyl 手性表示\autoref{Dirac_eq9}~\upref{Dirac},并且设 $\psi = \pmat{\psi_L\\\psi_R}$,可以将上述方程改写为
\begin{equation}
\begin{aligned}
&(i\partial_0 - qA_0-\bvec \sigma\cdot (\bvec p+q\bvec A))\psi_R=m\psi_L\\
&(i\partial_0-qA_0+\bvec \sigma\cdot (\bvec p+q\bvec A))\psi_L = m\psi_R
\end{aligned}
\end{equation}
或者也可以改用标准表示(从 \autoref{qed4_eq6}~\upref{qed4} 出发得到 $\gamma$ 矩阵的标准表示.与 weyl 表示不同的是,这里的 $\gamma^0$ 为 $\pmat{&I&0\\&0&-I}$.设标准表示下 $\psi=\pmat{\varphi\\ \chi}$,则
\begin{equation}
\begin{aligned}
(i\partial_0-qA_0-m)\varphi=\bvec \sigma\cdot (\bvec p+q\bvec A)\chi\\
(i\partial_0-qA_0+m)\chi=\bvec \sigma\cdot (\bvec p+q\bvec A)\varphi
\end{aligned}
\end{equation}
两个表示下的电子波函数可以通过以下变换联系:
\begin{equation}
\varphi = \frac{1}{\sqrt{2}}(\psi_L+\psi_R)
\end{equation}
