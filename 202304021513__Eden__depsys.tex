% 近独立子系
% 热力学系统|近独立子系|无相互作用

\pentry{热力学笔记(科普)\upref{HeatIn}}

\textbf{近独立子系}被定义为:大量粒子组成的热力学系统,且忽略粒子间的力学相互作用,不同粒子间可以看作是近独立的。

近独立子系是热力学的重要研究对象,由于不同粒子间可看作是近独立的,可以通过分析系统的能级来计算系统的配分函数\footnote{例如正则系综法\upref{CEsb}。},可以分析粒子的速率分布函数\footnote{例如麦克斯韦—玻尔兹曼分布\upref{MxwBzm}。},因此系统的各个热力学量就可以方便地用统计力学公式进行计算\footnote{例如玻尔兹曼分布(统计力学)\upref{MBsta}。},可以得到与实验符合得很好的结果。

近独立子系一般有三种分布:玻尔兹曼分布,费米狄拉克分布,玻色爱因斯坦分布,其中玻尔兹曼分布是经典极限情形下的分布。同时,粒子间可以有交换相互作用,考虑到量子系统中玻色子和费米子的性质,系统会呈现出同经典玻尔兹曼分布不同的结果\footnote{例如玻色爱因斯坦凝聚\upref{BEC}、金属中的自由电子气体\upref{mfcgas}。}。