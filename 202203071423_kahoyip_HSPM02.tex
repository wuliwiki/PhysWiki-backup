% 相互作用(高中)
% 高中物理|相互作用|力|牛顿

\addTODO{示意图需重新美化}

\subsection{力}

\subsubsection{基本概念}
力是物体对物体的\textbf{相互作用}.力不能离开物体而独立存在,力的产生一定同时涉及施力物体和受力物体.

符号:$\bvec F$.

单位:牛顿(简称 牛),$\mathrm N$.

\textbf{相互性}:物体间力的作用是相互的.一个物体对另一个物体施力时,另一个物体也同时对它施加力的作用.力总是成对出现的,具有同时性,不存在先后关系.

\textbf{矢量性}:力既有大小,也有方向.

\textbf{作用效果}:改变物体的形状(\textbf{形变});改变物体的运动状态(包括\textbf{速度大小}和\textbf{方向}).

\textbf{三要素}:\textbf{大小}、\textbf{方向}和\textbf{作用点}.三要素决定一个力,当其中一个要素改变时,力也发生改变,同时力的作用效果也随之改变.两个完全相同的力必须是三要素完全相同.

\textbf{力的图示}:用有向线段把力的三要素表示出来,箭头表示力的方向,端点表示力的作用点,按选定标度的线段长表示力的大小.

\textbf{力的示意图}:与力的图示基本一样,用适当长度的有向线段表示力,但无需按选定标度来表示大小.

\subsubsection{力的分类}
(1)根据性质(产生的原因):重力、弹力、摩擦力、电磁力、库仑力、分子力、核力等.

(2)根据作用效果:拉力、压力、动力、阻力、支持力等.

(3)根据研究对象:内力、外力等.

(4)根据作用方式:接触力、非接触力等.

\subsubsection{四种基本相互作用}
\textbf{万有引力}:一切具有质量的物体之间存在着相互吸引的力,相互作用的强度随距离的增大而减小,是长程力\footnote{长程力:作用强度随距离的增加而减少,从理论上说,可以作用到无限远.}.

\textbf{电磁相互作用}:带电粒子与电磁场的相互作用,以及带电粒子之间通过电磁场传递的相互作用,是长程力.

\textbf{强相互作用}:克服原子核内核子之间的斥力并维持原子核稳定的作用力,是短程力\footnote{短程力:作用范围很小,在原子核尺度内.}.

\textbf{弱相互作用}:在某些放射现象中起作用的一种基本相互作用,是短程力.

\subsection{常见力}

\subsubsection{弹力}
形变分为\textbf{弹性形变}(去掉外力能恢复原状)和\textbf{塑性形变}(去掉外力不能恢复原状).对于弹性形变来说,存在一个形变的极限,若形变超过这个极限,物体就不能完全恢复原状,这个极限称为\textbf{弹性限度}.

发生弹性形变的物体,由于要恢复原状而对与其接触的物体会产生力的作用,这种力叫做\textbf{弹力}.

\textbf{弹力的产生条件}:两个物体直接接触;两个物体发生形变并有恢复原状的趋势.

\textbf{弹力的三要素}:大小与物体的弹性强弱和形变程度有关;方向与物体形变的方向相反,指向恢复原状的方向;作用点在两物体的接触面上,作用在使物体发生弹性形变的施力物体上.

\textbf{胡克定律}:在弹性限度内,弹簧发生弹性形变时,弹力的大小跟弹簧伸长或缩短的长度成正比, 方向与位移相反.弹力大小的计算公式式为
\begin{equation}
F=kx
\end{equation}
严格来说, 应该写成 $F=-kx$, 但如果只讨论大小则可以省略负号. $k$为劲度系数,由弹簧本身决定,单位是$\mathrm{N/m}$;$x$是形变量,即弹簧相对于原长的改变量.

\subsubsection{重力}
由于地球的吸引而使物体受到的力,叫做重力,通常用字母$\bvec G$表示,施力物体是地球.需要留意的是,重力是地球吸引而产生的,但其大小不一定等于地球的吸引力,不能说“重力就是地球对物体的吸引力”.

\textbf{重力的三要素}:大小和物体的质量成正比;方向竖直向下;作用点是重心.

\textbf{重力大小的计算公式}:
\begin{equation}
G=mg
\end{equation}

$m$为物体的质量;$g$为重力系数,常取$g=9.8\mathrm{N/kg}$或$g=\mathrm{10N/kg}$.

\textbf{重心}:重心是物体各部分所受重力的\textbf{等效作用点}.重心的位置与物体的质量分布和形状有关,一旦物体的质量分布发生变化,其重心的位置也发生变化.对于形状为中心对称的物体,其重心位于对称中心.重心的位置不一定在物体上,如质量均匀分布的圆环,其重心在圆心处,不在圆环上.对于一般的物体,可以用悬挂法测量其重心.


\subsubsection{摩擦力}
当两个相互接触挤压的物体做\textbf{相对运动}或\textbf{有相对运动趋势}时,在接触面上会产生一种\textbf{阻碍物体发生相对运动}的力,叫做\textbf{摩擦力}.两个相互接触挤压的物体相对静止(相对于参考系静止或都做速度相同的运动)且存在相对运动趋势而没有发生相对运动时,在它们的接触面上产生的摩擦力叫做\textbf{静摩擦力}.两个相互接触挤压的物体因相对滑动而产生的摩擦力叫做\textbf{滑动摩擦力}.此外,初中物理中还介绍过\textbf{滚动摩擦力}.

\textbf{摩擦力的方向}:与物体相对运动或相对运动趋势的方向相反.

\textbf{摩擦力的作用点}:摩擦力作用在整个接触面上,但为了研究方便,可以把摩擦力的作用等效到一个点上,可以取在接触面上,也可以取在物体的重心上.

\textbf{摩擦力的产生条件}:(1)两个物体间有弹力;(2)接触面不光滑;(3)两个物体发生了相对滑动或有相对运动趋势.只有同时满足这三个条件,才能确定摩擦力的存在.

\textbf{滑动摩擦力的大小}:影响因素有二,一是两个物体接触面间的压力大小,二是接触面的粗糙程度.计算公式为
\begin{equation}
F_f=\mu F_N
\end{equation}

$\mu$为动摩擦因素,没有单位,与接触面的粗糙程度有关;$F_N$是两个物体接触面之间的压力大小,在性质力上是弹力,等于物体受到的支持力.

\textbf{静摩擦力的大小}:随外力的变化而变化,必须结合物体的运动状态及其受力情况确定.物体刚要发生相对运动时所受的静摩擦力称为最大静摩擦力$F_{max}$,其大小略大于滑动摩擦力.

\subsection{力的合成与分解}
\subsubsection{力的运算法则}

在两个力合成时,以表示这两个力的有向线段为邻边作平行四边形,这两个邻边之间的对角线就代表合力的大小和方向.这个规律叫做\textbf{平行四边形定则}.
\begin{figure}[ht]
\centering
\includegraphics[width=8cm]{./figures/HSPM02_1.png}
\caption{平行四边形定则} \label{HSPM02_fig1}
\end{figure}

力的平行四边形定则也可以转化为力的\textbf{三角形定则}.将\autoref{HSPM02_fig1} 中的力$\bvec F_{\bvec 2}$向上平移,使它的始端与$\bvec F_{\bvec 1}$的末端相接,则从$\bvec F_{\bvec 1}$始端指向$\bvec F_{\bvec 2}$末端的有向线段$\bvec F$便是它们的合力.平行四边形定则和三角形定则的实质是一样的.
\begin{figure}[ht]
\centering
\includegraphics[width=6cm]{./figures/HSPM02_2.png}
\caption{三角形定则} \label{HSPM02_fig2}
\end{figure}

\subsubsection{力的合成}
几个力如果都作用在物体的同一个点,或者它们的作用线相交于一点,那么这几个力叫做\textbf{共点力}.另外,可看作质点的物体所受的力也是共点力.

假设一个力单独作用的效果和某几个力共同作用的效果一致,那么这个力就叫做那几个力的\textbf{合力}.假设几个力共同作用的效果和某个力单独作用的效果一致,那么这几个力就叫做那个力的\textbf{分力}.

合力和分力的关系:\textbf{等效性}(相互替代);\textbf{同体性}(受力物体相同);\textbf{瞬时性}(分力变化,合力同时发生变化).

求几个力的合成的过程叫\textbf{力的合成}.合力是假设的,不是真实的力.

合力和分力的大小关系:

(1)两个共点力的合力(设两分力大小分别为$F_1$、$F_2$,它们的夹角为$\theta$,合力大小为$F$,$\bvec F$与$\bvec F_{\bvec 1}$的夹角为$\varphi $)

\begin{table}[ht]
\centering
\caption{共点力的二力合成情况}\label{HSPM02_tab1}
\begin{tabular}{|c|c|c|c|}
\hline
两分力关系 &  合力大小 & 合力方向 \\
\hline
同向($\theta=0^{\circ}$) &  $F=F_1+F_2$(最大) & 与两分力同向 \\
\hline
反向($\theta=180^{\circ}$) &  $F=\abs{F_1-F_2}$(最小) & 与较大分力同向 \\
\hline
垂直($\theta=90^{\circ}$) &  $F=\sqrt{F_1^2+F_2^2}$ & $\varphi=\arctan{(F_2/F_1)}$ \\
\hline
二力等大 &  $F=2F_1\cos{(\theta/2)}$ & $\varphi=\theta/2$ \\
\hline
二力等大,且($\theta=120^{\circ}$) &  $F=F_1=F_2$ & $\varphi=60^{\circ}$ \\
\hline
\end{tabular}
\end{table}

\begin{aligned}
\theta越大,F越小,且\abs{F_1-F_2}\leq F\leq F_1+F_2
\end{aligned}

(2)三个共点力的合力范围(设三个力的大小分别为$F_1$、$F_2$、$F_3$,且$F_1\leq F_2 \leq F_3$)

易知,当三个力同向时,合力有最大值$F_{max}=F_1+F_2+F_3$.

对于最小值,如果三个力的任意一个力的大小在另外两个力的合力范围内,即满足$\abs{F_1-F_2}\leq F_3\leq F_1+F_2$,由两个共点力的合成情况可知,$F_{min}=0$,此时$\bvec F_{\bvec 1}$与$\bvec F_{\bvec 2}$的合力方向与$\bvec F_{\bvec 3}$相反.否则,合力的最小值应为最大的力与另外两个力之和的差,此时$\bvec F_{\bvec 1}$与$\bvec F_{\bvec 2}$同向,与$\bvec F_{\bvec 3}$反向,$F_{min}=F_3-(F_1+F_2)$.

\subsubsection{力的分解}
求一个力的分力的过程叫做\textbf{力的分解}.分力是假设的,不是真实的力.

以待分解的力 $F$ 为对角线作平行四边形,平行四边形的两邻边就是$\bvec F$的分力.如\autoref{HSPM02_fig3} ,$\bvec F_{\bvec 1}$、$\bvec F_{\bvec 2}$或$\bvec F_{\bvec 1}'$、$\bvec F_{\bvec 2}'$都是$\bvec F$的两个分力.这也说明了,只要没有限制条件,对给定的同一个力$\bvec F$可以分解成无数对大小、方向不同的分力.分解某个力时,一般会根据实际研究的情况来确定分力的方向,再沿着分力方向根据平行四边形定则作出平行四边形,然后利用边角关系计算出分力的大小.
\begin{figure}[ht]
\centering
\includegraphics[width=8cm]{./figures/HSPM02_3.png}
\caption{力的分解} \label{HSPM02_fig3}
\end{figure}

\subsubsection{正交分解法}
把力沿着两个选定的相互垂直的方向分解,这种分解方法叫做\textbf{正交分解法}.如\autoref{HSPM02_fig4},$F_{x}=F\cos\theta$,$F_{y}=F\sin\theta$.
\begin{figure}[ht]
\centering
\includegraphics[width=10cm]{./figures/HSPM02_4.png}
\caption{力的正交分解} \label{HSPM02_fig4}
\end{figure}

使用正交分解法求多个力的合力:

首先,以共点力的作用点作为坐标原点,建立直角坐标系,原则有二:\textbf{简单}(让尽量多的力在坐标轴上,减少待分解的量)和\textbf{方便}(尽量让坐标轴落在运动方向或待求的力上).

正交分解不在坐标轴上的各个力,将它们都分解到$x$轴和$y$轴上.

求出 $x$ 轴和 $y$ 轴上各分力的合力,$F_{x}=F_{{1x}}+F_{{2x}}+\dots+F_{{nx}}$,$F_{y}=F_{{1y}}+F_{{2y}}+\dots+F_{{ny}}$

求出共点力的合力大小$F=\sqrt{F_x^2+F_y^2}$,合力的方向与$x$轴的夹角$\theta=\arctan(F_y/F_x)$.

\subsection{受力分析}
研究物体运动状态的变化与否、运动过程、平衡条件等问题时,正确的受力分析是前提.

\textbf{基本步骤}

明确研究对象:确定待分析受力情况的物体.

隔离物体分析:将所确定的研究对象从周围物体中隔离出来,进而分析周围有哪些物体对其施加了力的作用(通常按重力、弹力、摩擦力和其他外力的顺序).

画受力示意图:边分析边将这些力一一画在受力图上(必须准确地标出各力的方向).

\textbf{常用方法}

定义法:分析研究对象是否受力,就看研究对象周围有哪些物体对其施加了力的作用.

隔离法:研究系统内物体之间的相互作用力.

整体法:研究系统外的物体对系统整体的作用力(遇到多物体平衡的问题时,一般要将整体法和隔离法结合运用,通常先整体后隔离).

假设法:在不确定某个力的存在时,先假设其存在或不存在,再根据该力的存在与否对物体运动状态的影响来判断该力是否存在.

\subsection{共点力的平衡}
\subsubsection{平衡状态}
物体处于静止或者保持匀速直线运动的状态叫做\textbf{平衡状态}.

物体如果受到共点力的作用且处于平衡状态,就叫做\textbf{共点力的平衡}.

\subsubsection{共点力的平衡条件}
二力平衡:二力作用在同一物体(\textbf{同体})、大小相等(\textbf{等大})、方向相反(\textbf{反向})、且作用在同一直线上(\textbf{共线}).

三力作用:任意两个力的合力与第三个力等大、反向.

多力作用:任意一个力与其他所有力的合力等大、反向.三个或以上的共点力的平衡,总可以转化为二力平衡.

共点力平衡时,从合力大小来看,总有合力$F=0$,从正交分解来看,则有$F_x=0$且$F_y=0$.