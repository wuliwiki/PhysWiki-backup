% 线性连续泛函
% keys 线性|连续|泛函
% license Usr
% type Tutor

\pentry{拓扑向量空间\nref{nod_tvs},泛函与线性泛函\nref{nod_Funal}}{nod_5b23}
拓扑线性(或拓扑向量)空间的拓扑空间性质表明其上映射的连续性具有基本的重要性,而其线性空间性表明线性映射具有基本的重要性。因此,拓扑线性空间上的线性连续映射具有基本的重要性。特别,对拓扑线性空间上的泛函,线性连续泛函具有基本的重要性。
\begin{definition}{连续}
设 $f$ 是拓扑线性空间 $E$ 上的\enref{线性泛函}{Funal}。若对任意 $x_0\in E,0<\epsilon\in\mathbb R$,存在 $x_0$ 的邻域 $U$,使得 $x\in U$ 就有
\begin{equation}\label{eq_LinCon_1}
\abs{f(x)-f(x_0)}<\epsilon,~
\end{equation}
则称 $f$ 是\textbf{线性连续泛函}。
\end{definition}

一般定义映射的连续,往往是先定义映射在一点的连续。然而上面的定义是对每一点都连续,而没有事先定义在一点的连续。事实上,在拓扑线性空间中,线性映射在一点连续必定在全空间连续。这由下面的定理指定。
\begin{theorem}{在一点处连续的线性泛函处处连续}\label{the_LinCon_1}
设线性泛函 $f$ 在某一点 $x\in E$ 处连续,则 $f$ 必定在 $E$ 上处处连续。
\end{theorem}
\textbf{证明:}由\autoref{the_tvs_1},设 $U$ 是 $x$ 的满足\autoref{eq_LinCon_1} 的邻域,于是 $U-x$ 是零邻域,从而 $V=U-x+y=U+(y-x)$ 是(任一点) $y$ 的邻域。因此,若 $z\in V$,则 $z+x-y\in U$,进而
\begin{equation}
\abs{f(z)-f(y)}=\abs{f(z-y+x)-f(x)}<\epsilon.~
\end{equation}
即验证线性泛函在点 $y$ 的连续性,只需要验证在某一点 $x$ 的连续性即可。

\textbf{证毕!}

\begin{theorem}{有限维的线性泛函必连续}
设 $E$ 是有限维的拓扑向量空间,则 $E$ 中的任何线性泛函必定连续。
\end{theorem}
\textbf{证明:}设 $e_1,\cdots,e_n$ 是 $E$ 的基矢,于是对任意 $\epsilon>0$,要
\begin{equation}\label{eq_LinCon_2}
f(x)-f(x_0)=\sum_{i=1}^n\Delta x_0^i f(e_i)<\epsilon,\quad\Delta x^i=x^i-x^i_0~
\end{equation}
 只需取 $\sum_{i=1}^n\Delta x_0^i<\frac{\epsilon}{M}$ 即可,其中 $M=\max\{\abs{f(e_i)}\}_{i=1,\cdots,n}$ 。从而 $x_0$ 的邻域可以这样选择:由于 $\norm{x-x_0}\leq\sum_{i=1}^n\abs{\Delta x_0^i}\norm{e_i}\leq m\sum_{i=1}^n\abs{\Delta x_0^i}$,其中 $m=\max\{\norm{e_i}\}_{i=1,\cdots,n}$。因此只需选择 $x_0$ 的 $\frac{\epsilon}{Mm}$ 开球邻域,则\autoref{eq_LinCon_2} 成立。

\textbf{证毕!}

\begin{theorem}{}
 拓扑线性空间 $E$ 上线性泛函 $f$ 连续的充要条件是:存在 $E$ 中零邻域,使得 $f$ 在该邻域上有界。
\end{theorem}

\textbf{证明:}\textbf{必要性}:设 $f$ 在点0连续,则由线性泛函连续的定义,那么对任意 $\epsilon>0$,存在零邻域,在该邻域上 $\abs{f(x)}<\epsilon$(注意 $f(0)=0$)。

\textbf{充分性:}设 $U$ 是使得 $f$ 有界的零邻域,即 $\abs{f(U)}<C$,$C$ 是某一正数。并设 $\epsilon>0$,则 $\frac{\epsilon}{C}U$ 是这样的零邻域,在其上 $\abs{f(x)}<\epsilon$。即 $f$ 在点0处连续,于是由\autoref{the_LinCon_1} ,$f$ 在 $E$ 上处处连续。



\textbf{证毕!}




