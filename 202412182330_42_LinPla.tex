% 线性规划
% keys 线性规划|单纯形法
% license Usr
% type Tutor
\pentry{线性方程组(高中)\nref{nod_LinEqu}}{nod_9aef}

1939年,苏联数学家Kantorovich出版了《生产组织与计划中的线性规划模型》一书,为列宁格勒胶合板厂的计划任务建立了一个线性规划的数学模型,为用数学方法解决管理并使二者结合做出了开创性的工作。后来,由于战争的需要,美国经济学家Koopmans重新独立地研究运输问题,并很快看到了线性规划在经济学中应用的意义。此后线性规划也被广泛应用于军事、经济等各方面。鉴于他们在线性规划方面的突出贡献,1975年的诺贝尔经济学奖授予了他们。1947年美国数学家Dantzig提出了求解一般线性规划问题的方法——单纯性法,之后线性规划问题在理论上日益成熟,并在实际中日益广泛应用。

线性规划研究的是在线性不等式或等式限制下,使得某一线性目标取得最大(或最小)的问题。

\subsection{线性规划}

\begin{definition}{线性规划}
若研究的问题可以归结到求解一个关于某些变量的线性函数,使得该函数在变量的某些线性限制条件下,取最大或最小的问题,则称该问题为\textbf{线性规划}模型。设变量有 $n$ 个,并记变量为 $x_1,\cdots,x_n$,所求最大或最小的线性函数为 $z=z(x_1,\cdots,x_n)$,线性限制条件为 $f_i(x_1,\cdots,x_n)\leq(\geq) 0,i=1,\dots,m$,则线性规划模型可写为(s.t. 是“使得”的意思):
\begin{equation}\label{eq_LinPla_1}
\begin{aligned}
&\min(\text{或}\max) \quad z=z(x_1,\cdots ,x_n),\\
&s.t.\quad f_i(x_1,\cdots,x_n)\leq(\text{或}\geq) c_i,\quad i=1,\dots,m.
\end{aligned}~
\end{equation}
其中,z称为\textbf{目标函数},$x_1,\cdots,x_n$ 称为\textbf{决策变量},关于 $f_i$ 的不等式或等式称为\textbf{约束条件}, $z,f_i$ 都要求是关于变量 $x_1,\cdots,x_n$ 的线性函数,$c_i$ 是常数。
\end{definition}

由线性规划的定义,线性规划模型\autoref{eq_LinPla_1} 具有很多形式,但是我们总可以将其转化为某一种特定的形式,并只需要对这种形式进行研究。

\subsection{线性规划的标准形式}
通常称下面定义的形式为线性规划模型的标准形式。
\begin{definition}{标准形式}
具有如下形式的目标函数和约束条件称为线性规划模型的\textbf{标准形式}:
\begin{equation}\label{eq_LinPla_2}
\begin{aligned}
&\min  &&z=c_1x_1+\cdots +c_nx_n,\\
&s.t. &&a_{11}x_1+\cdots+a_{1n}x_n=b_1,\\
&&&a_{21}x_1+\cdots+a_{2n}x_n=b_2,\\
&&&\cdots\\
&&&a_{m1}x_1+\cdots+a_{mn}x_n=b_m,\\
&&&x_1\geq0,\cdots,x_n\geq0.
\end{aligned}~
\end{equation}

\end{definition}
任何线性规划模型\autoref{eq_LinPla_1} 都可以化为标准型\autoref{eq_LinPla_2} :若约束条件为 $\sum_{j}^na_{ij} x_i\leq b_i$,则引入 $y_{i}$,从而约束条件等价于
\begin{equation}
\sum_{j}^na_{ij} x_i+y_i= b_i,y_i\geq 0.~
\end{equation}
此时 $y_i$ 称为\textbf{松弛变量}。

若约束条件为 $\sum_{j}^na_{ij} x_i\geq 0$,则引入 $y_{i}$,从而约束条件等价于
\begin{equation}
\sum_{j}^na_{ij} x_i-y_i= b_i,y_i\geq 0.~
\end{equation}
此时 $y_i$ 称为\textbf{剩余变量}。

若目标函数为 $\max z$,则定义 $z'=-z$,于是目标函数等价于 $\min z'$。

若某个变量 $x_i$ 无限制,则引入两个变量 $x'_i,x''_i$,从而 $x_i$ 无限制等价于 
\begin{equation}
x_i=x'_i-x''_i,x'_i\geq0,x''_i\geq0.~
\end{equation}
在标准形式下,使得目标函数最小的变量 $x_1,\cdots,x_n$ 在以变量为的坐标的坐标系下将只能在“第一象限”。


\subsection{单纯形法}

首先介绍求解线性规划常用的术语。

\begin{definition}{可行解,最优解}

\end{definition}








