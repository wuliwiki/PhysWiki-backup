% 格奥尔格·康托尔(综述)
% license CCBYSA3
% type Wiki

本文根据 CC-BY-SA 协议转载翻译自维基百科\href{https://en.wikipedia.org/wiki/Georg_Cantor}{相关文章}。

\begin{figure}[ht]
\centering
\includegraphics[width=6cm]{./figures/409c043d8a08ce28.png}
\caption{} \label{fig_Canto_1}
\end{figure}
乔治·费迪南德·路德维希·菲利普·康托尔(Georg Ferdinand Ludwig Philipp Cantor,/ˈkæntɔːr/ KAN-tor;德语发音:[ˈɡeːɔʁk ˈfɛʁdinant ˈluːtvɪç ˈfiːlɪp ˈkantoːɐ̯];1845年3月3日(旧历2月19日)-1918年1月6日)是一位数学家,他在集合论的创建中发挥了关键作用,集合论已成为数学中的一项基本理论。康托尔确立了两个集合成员之间一对一对应的重要性,定义了无限集合和良序集合,并证明了实数比自然数更多。康托尔证明该定理的方法意味着存在无数个不同大小的无限集合。他定义了基数和序数以及它们的算术运算。康托尔的工作在哲学上具有重大意义,他对此非常清楚。

