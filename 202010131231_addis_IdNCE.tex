% 理想气体(微正则系综法)
% 理想气体|微正则系综|相空间|熵

\begin{issues}
\issueDraft
\end{issues}

\pentry{理想气体的状态密度(相空间)\upref{IdSDp}}
\subsection{$N$ 粒子相空间}

当理想气体处于某个能量 $E$ 时, 相空间只是位置相空间($3N$ 维)以及动量相空间中的一个球表面($3N-1$ 维). 这个 $6N - 1$ 维相空间中的体积除以 $h^{3N}$ (\autoref{IdSDp_eq4}~\upref{IdSDp})就可以作为熵公式%未完成链接
中的 $\Omega$, 得
\begin{equation}\label{IdNCE_eq1}
S(E, V, N) = k\ln \Omega  = Nk \qty(\ln \frac{V}{N\lambda^3} + \frac52)
\end{equation}
该式被称为 \textbf{Sackur-Tetrode 公式}. 其中
\begin{equation}\label{IdNCE_eq2}
\lambda = \frac{h}{\sqrt{4\pi mE/(3N)}}
\end{equation}
\autoref{IdNCE_eq1} 也用到了 Stirling 近似 %未完成链接
$\ln N! = N\ln N - N$.

根据熵的微分关系 % 链接未完成
\begin{equation}
\dd{S} = \frac{1}{T} \dd{E} + \frac{P}{T} \dd{V} - \frac{\mu}{T} \dd{N}
\end{equation}
求 $T$ 得
\begin{equation}
E = \frac{3}{2}NkT
\end{equation}
代入\autoref{IdNCE_eq2} 得
\begin{equation}
\lambda = \frac{h}{\sqrt{2\pi mkT}}
\end{equation}
求 $P$ 得
\begin{equation}
PV = NkT
\end{equation}
求 $\mu$ 得
\begin{equation}
123
\end{equation}
