% 戈特弗里德·莱布尼茨(综述)
% license CCBYSA3
% type Wiki

本文根据 CC-BY-SA 协议转载翻译自维基百科\href{https://en.wikipedia.org/wiki/Gottfried_Wilhelm_Leibniz}{相关文章}。

“戈特弗里德·威廉·莱布尼茨或莱布尼茨[a](1646年7月1日 [旧历6月21日] – 1716年11月14日)是一位德国博学家,活跃于数学家、哲学家、科学家和外交官等多个身份,与艾萨克·牛顿爵士一起被认为是微积分的发明者,此外还在二进制算术和统计学等数学分支做出了许多贡献。由于其在不同领域的知识和技能,以及随着工业革命的到来和专业化劳动的普及,像他这样的人变得越来越少,莱布尼茨被称为“最后的通才”。[15]他在哲学史和数学史上都是一个重要人物。他撰写了关于哲学、神学、伦理学、政治学、法律、历史、语文学、游戏、音乐和其他学科的作品。莱布尼茨还对物理学和技术做出了重大贡献,并预见了后来在概率论、生物学、医学、地质学、心理学、语言学和计算机科学中出现的概念。

此外,他在沃尔芬比特尔的赫尔佐格·奥古斯特图书馆工作时,设计了一种目录系统,为欧洲许多大型图书馆提供了指导。[16][17] 莱布尼茨在众多领域的贡献散见于各种学术期刊、成千上万封信件以及未发表的手稿中。他用多种语言写作,主要是拉丁语、法语和德语。[18][b]

作为一位哲学家,他是17世纪理性主义和唯心主义的主要代表之一。作为数学家,他的主要成就是独立于艾萨克·牛顿的同时代发展而提出了微积分的核心思想。[20] 数学家们一直偏爱莱布尼茨的符号法,因为它被认为是微积分的惯用且更为精确的表达方式。[21][22][23]

在20世纪,莱布尼茨关于连续性法则和超越同质性法则的概念通过非标准分析得到了一致的数学表述。他也是机械计算器领域的先驱。在为帕斯卡计算器添加自动乘法和除法功能的过程中,他于1685年首次描述了齿轮计算器,并发明了莱布尼茨轮,后来被用于首台大规模生产的机械计算器——算筹器中。

在哲学和神学方面,莱布尼茨以其乐观主义最为著名,即他得出的结论是,我们的世界在某种限定意义上是上帝所能创造的最好的世界,这一观点有时被其他思想家讽刺,例如伏尔泰在其讽刺小说《老实人》中就调侃了这一观点。莱布尼茨与勒内·笛卡尔和巴鲁赫·斯宾诺莎一起,是三位具有影响力的早期现代理性主义者。他的哲学还吸收了经院哲学传统的元素,尤其是假设通过从第一原理或先验定义推理,可以获得对现实的实质性知识。莱布尼茨的工作预示了现代逻辑的发展,并且仍然影响着当代分析哲学,例如采用“可能世界”这一术语来定义模态概念。