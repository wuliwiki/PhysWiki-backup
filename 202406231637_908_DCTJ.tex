% 电磁推进
% license CCBYSA3
% type Wiki

(本文根据 CC-BY-SA 协议转载自原搜狗科学百科对英文维基百科的翻译)

电磁推进是利用流动的电流和磁场加速物体的一种方法。电流被用来产生一个与运动方向相反的磁场,或者给一个磁场提供电能,然后该磁场就会受到排斥力作用。当电流在磁场中流经导体时,产生的电磁力,也就是洛伦兹力,会将导体推向垂直于导体和磁场的方向。这种排斥力是电磁推进系统中产生推进力的成因。电磁推进这个术语可以用它的两个组成部分来描述:电磁——用电产生磁场,推进——推进物体的过程。当流体(液体或气体)用作移动导体时,推进系统可以称为磁流体动力驱动。尽管电磁推进和电动机推进都使用磁场和流动电流,但两者之间的一个关键区别是:电磁推进所用的电能不用于产生运动的旋转动能。

电磁推进的科学理论并不是起源于特定一个人,却应用于许多不同的领域。自从1897年约翰·芒罗发表他的虚构故事《金星之旅》开始,人们就一直梦想着使用磁体推进物体,这一想法一直持续到今天。[1]目前电磁推进的应用可以在磁悬浮列车和军用轨道炮上看到。其他应用尚未广泛使用或仍在开发中,其中包括低轨卫星的离子推进器以及船舶和潜艇的磁流体动力驱动系统。
\subsection{历史}
关于电磁推进的第一个记载是在1889年,埃利胡·汤姆森教授公开了他在电磁波和交流电方面的工作成果。[2][3] 几年后,埃米尔·巴切莱特提出了现代铁路中金属车厢悬浮在铁轨上方的想法,并且他在19世纪90年代早期展示了这一想法。[2][3]20世纪60年代,埃里克·罗伯茨·莱思韦特研制了直线感应电动机,该电动机建立在电磁原理的基础上,并首次实现了电磁推进的实际应用。[4]1966年,詹姆斯·R·鲍威尔和戈登·丹比为超导磁悬浮交通系统申请了专利。此后,世界各地的工程师竞相建造第一条高速铁路。[4][5] 1984年到1995年间,第一个商业自动磁悬浮系统在伯明翰运行。这是一辆从伯明翰国际机场到伯明翰国际火车站的低速磁悬浮列车。
\subsection{应用}
\subsubsection{2.1 列车}
电磁推进在运输系统中被用来在长距离轨道上实现摩擦最小化和速度最大化。这主要是在高速铁路系统中应用的,该系统使用线性感应电机,通过磁电流为列车提供动力。电磁推进也被用于主题公园,建造高速过山车和水上游乐设施。

\textbf{磁悬浮列车}

\begin{figure}[ht]
\centering
\includegraphics[width=6cm]{./figures/894123c76e80d580.png}
\caption{2005年11月在日本山梨试验跑道上的磁悬浮列车} \label{fig_DCTJ_1}
\end{figure}
在磁悬浮列车中,初级线圈组件位于反力板下方。[6]两者之间有1-10厘米(0.39-3.93英寸)的空气间隙,可以消除摩擦,列车速度可达500公里/小时(310英里/小时)。[6]交流电通过线圈,产生了磁场极性的变化。[7]因此该系统可以从前面拉动列车,从后面向前推动列车。[8]

典型的磁悬浮列车每人每英里收费3美分,或者运输货物每吨每英里收费7美分(不包括建设成本)。[9]相比之下,乘飞机旅行每人每英里15美分,城际物流卡车运输每吨每英里30美分。[9]由于磁悬浮轨道摩擦力极小且重量分布均匀,因此寿命较长。[7]大多数磁悬浮轨道可持续运行至少50年,在此期间几乎不需要维护。[10]磁悬浮列车由于能源效率较高而得到推广,这是因为列车依靠电力运行,而电力可以由煤、核能、水能、核聚变、风能或太阳能产生,不只是需要石油。[4]大多数列车平均时速为483公里(300英里),每位乘客每英里消耗0.4兆焦耳。[9]相比之下,载客1.8人、耗油量20英里/加仑的汽车,行车速度通常为97公里/小时(60英里/小时),每英里每人消耗4兆焦耳。[9]二氧化碳排放量由发电和燃料使用的方式来决定。许多可再生发电方式在生产过程中很少产生或不产生二氧化碳(尽管在部件制造过程中可能释放二氧化碳,例如风力涡轮机中使用的钢材)。磁悬浮列车的运行比其他火车、卡车和飞机安静得多。[5]\textbf{ 组件:直线感应电机} 

直线感应电机由两部分组成:初级线圈组件和反力板。[7][10] 初级线圈组件由钢片包围的多相导线绕组而成,并包括热环氧树脂在内的热传感器。[9]反力板由3.2毫米(0.125英寸)厚的铝或铜板以及6.4毫米(0.25英寸)厚的冷轧钢板组成。[10]初级线圈组件和反力板之间有一个气隙,提供了电磁推进系统所具有的无摩擦特性。[6][10] 线性感应电机的功能来源于初级线圈组件内的线圈绕组所提供的交流电。[4]交流电场产生移动磁场,进而在反力板中产生感应电流,然后反力板中就有了磁场。[8]初级线圈组件和反力板中的磁场交替,产生作用力和直线运动。[10]

\textbf{2.2 航天器}

电磁推进技术在航空航天领域有多种应用。虽然到目前为止,这些应用中的大部分还只是概念性的,但是,有一些应用方式将从近期延续到下个世纪。[11]其中一个应用途径是使用电磁推进控制轨道卫星的微调。这个特殊系统的其中一种是基于航天器自身电磁场和地球磁场的直接相互作用。推力可以被认为是其导体中内部电流与地球施加的自然磁场相互作用的电动力。[12]为了获得更大的相互作用力,磁场必须扩大到离飞行器更远的地方。这种系统的优点是控制非常精确和实时控制推力。此外,预期的电能效率远远高于目前通过利用中间热量获得推进的化学燃料火箭;而且,化学燃料推进效率较低并且产生大量气体污染物。[13]电磁推进系统中线圈的电能通过直接能量转换转化为势能和动能。这使得该系统具有与其他电机相同的高效率特点,同时没有向环境中排放任何物质。[13]

目前这些系统的推重比相对较低。然而,由于它们不需要燃料质量,所以航天器质量是恒定的。此外,电磁推进可以在耗电量相对较低的情况下持续产生推力。[12]电磁推进最大的限制主要是材料的电导性,以产生推进系统中所必需的电流的大小。