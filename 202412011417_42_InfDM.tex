% 无穷阶差分矩阵
% keys 差分矩阵|无穷阶|傅里叶变换
% license Usr
% type Tutor

\pentry{导数近似之差分矩阵算法\nref{nod_DifMa}}{nod_3681}

\subsection{傅里叶变换回顾}
\subsubsection{连续区间上的傅里叶变换}
由\enref{正交函数系}{OFS}中的\autoref{ex_OFS_3} 可知,函数 $u(x),x\in\mathbb R$ 的傅里叶变换定义为
\begin{equation}\label{eq_InfDM_1}
\hat u(k):=\int_{-\infty}^{\infty} e^{-\I kx}u(x)\dd x,\quad k\in\mathbb R.~
\end{equation}
而从 $\hat u$ 同构逆傅里叶变换可以重构 $u$:
\begin{equation}\label{eq_InfDM_2}
u(x)=\frac{1}{2\pi}\int_{-\infty}^{\infty} e^{\I kx}\hat u(k)\dd k.~
\end{equation}
这被称为\textbf{傅里叶合成}(Fourier synthesis),变量 $x$ 称为\textbf{物理变量}(physical variable),$k$ 称为\textbf{傅里叶变量}(Fourier variable)或\textbf{波数}(wavenumber)。

\subsubsection{离散点上的傅里叶变换}
当限定 $x\in hZ$时,即此时 $x$ 只能取离散点,那么只要 $k_1-k_2=\frac{2\pi}{h}$,就有 $e^{\I k_1x}=e^{\I k_2x}$ 对所有的 $x$ 成立。这使得对任意复指数 $e^{\I kx}$,有无穷多个复指数和它在 $h\mathbb Z$ 上具有相同的值。这一现象被称为\textbf{混叠}。为了保持\autoref{eq_InfDM_1} 中傅里叶变换 $\hat u(k)$ 的单值性,需要将 $k$ 限制在长为 $\frac{2\pi}{h}$ 的区间上。因此,我们可以选择 $k$ 的取值区间为 $[-\pi/h,\pi/h]$。注意到 $x$ 是物理变量,$k$ 是傅里叶变量,因此我们可以将这个结果总结为下面的图示:
\begin{equation}
\begin{aligned}
&\text{物理空间:}& \text{离散},&\qquad\text{无界:} &x\in h\mathbb Z\\
&&\updownarrow&\qquad\updownarrow&\\
&\text{傅里叶空间:}& \text{有界},&\qquad\text{有界:} &k\in [-pi/h,\pi/h]
\end{aligned}~
\end{equation}
因此,在离散的情形,傅里叶变换及其逆变换\autoref{eq_InfDM_1} 和\autoref{eq_InfDM_2} 应当写为下面的级数和:
\begin{equation}\label{eq_InfDM_3}
\begin{aligned}
\hat u(k)&=h\sum_{i=-\infty}^{\infty}e^{-\I kx_i}u(x_i),\quad k\in[-\pi/h,\pi/h],\\
u(x_j)&=\frac{1}{2\pi}\int_{-\pi/h}^{\pi/h} e^{\I kx_j}\hat u(k)\dd k,\quad x=jh,j\in \mathbb Z.
\end{aligned}~
\end{equation}



\subsection{插值函数}
假若我们有定义在 $x_i\in h\mathbb Z$ 上的数据 $u_i=u(x_i)$,那么\autoref{eq_InfDM_3} 将会给我们插值函数:

1.通过\autoref{eq_InfDM_3} 的第一式我们可以获得 $\hat u(k)$,它关于 $k$ 是可微的;

2.而\autoref{eq_InfDM_3} 的第二式可以确定一个函数 
\begin{equation}\label{eq_InfDM_4}
p(x)=\frac{1}{2\pi}\int_{-\pi/h}^{\pi/h} e^{\I kx}\hat u(k)\dd k,\quad x\in \mathbb R.~
\end{equation}
显然 $p(x)$ 满足 $p(x_i)=u_i$,并且 $p(x)$ 是可微的。因此可以用 $p(x)$ 的导数来近似 $u(x)$ 的导数。$p(x)$ 给了我们一个插值公式。

由\autoref{eq_InfDM_4} ,只需令 $\hat u(k)$ 仅在 $[-\pi/h,\pi/h]$ 上不为零。那么\autoref{eq_InfDM_4} 的积分下限和上限就可以变到 $-\infty,\infty$。因此由\autoref{eq_InfDM_1} 和\autoref{eq_InfDM_2} 可知 $p(x)$ 的傅里叶变换为
\begin{equation}
\hat p(k)=\left\{\begin{aligned}
&\hat u(k),k\in[-\pi/h,\pi/h],\\
&0,k\not\in[-\pi/h,\pi/h].
\end{aligned}\right.~
\end{equation}












