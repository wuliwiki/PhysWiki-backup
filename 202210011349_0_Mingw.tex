% Mingw 笔记

\begin{itemize}
\item Mingw 仅支持 32-bit 程序, 现在一般用 \href{https://en.wikipedia.org/wiki/Mingw-w64}{Mingw-64}, 既支持 32 也支持 64-bit
\item 一个\href{https://www3.ntu.edu.sg/home/ehchua/programming/howto/Cygwin_HowTo.html}{教程}.
\item MYSYS2 可以让在 linux 上开发的软件在 windows 上运行.
\item 双击 \verb|mingw64.exe| 即可打开 mingw 命令行. \verb|mingw64.exe| 所在的目录就是命令行的根目录.
\item 可以检查 \verb|g++| 的版本.
\item 用 \verb|g++| 编译程序以后, 会出现 \verb|a.exe| 而不是 \verb|a.out|. 这个执行文件是可以双击执行的, 但是运行完会马上退出. 可以在程序最后用 \verb|getchar()|. 也可以打开一个 cmd 命令行然后运行 \verb|a.exe|. 当然在 \verb|mingw64| 的命令行也可以执行. 另外执行时 \verb|.exe| 拓展名可以省略.
\item 不能使用 apt, 一切库都要自己手动编译.
\item \verb|c| 盘的目录为 \verb|/c|
\end{itemize}
