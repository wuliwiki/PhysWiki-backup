% 集合
% 朴素集合论|元素|子集|并集|交集|补集|可数|不可数

\pentry{公理系统\upref{axioms}}
\subsection{集合}

对于物理学习而言,集合论没必要从公理角度来严格理解,所以在此不会给出用于划定集合论讨论范围的公理系统,而是\textbf{朴素集合论(naive set theory)}的解释,即比较接近自然语言的表达方式.

\textbf{集合(set)}是由\textbf{元素(element)}组成的.任何事物和概念都可以成为元素,任何不同的元素都可以放在一起,构成一个集合.可以说,如果我们划定一个讨论的范围,那么这个范围就是一个集合,范围涉及到的事物和概念就是这个集合当中的元素.公理系统的作用,也就是在所有可能讨论的话题所构成的集合中,限定一个子集作为讨论范围. 对于集合$A$,定义$|A|$为$A$中元素的数量, 称为集合$A$的\textbf{基数(cardinal number)}或\textbf{势(cardinality)}.

表达一个集合的方式有多种.最简单的方式是列出所有集合中的元素,在数学中规定的语法规范是用大括号“$\{\}$”来列举集合中的一切元素,以逗号隔开彼此.比如,$\{\text{猪}, \text{牛}, \text{狗}, \text{羊}, \text{猫}\}$是了一个具有五个元素的集合,$\{1,2,3,4,\dots\}$则是全体正整数的集合.第二个例子并没有显然地列举出所有正整数,只是用省略号表达了这个意思;也就是说,表达一个集合的方式并没有死板的规定,只要能让读者理解就可以了.

另一种常见的表达集合的方式是确定一个规则,语法规范是“$\{x|x \text{需要满足的条件}\}$”.比如全体正整数的集合,也可以写为$\{x|\text{$x$ 是一个正整数}\}$.如果有多个条件,也可以列在一起,比如全体正整数的集合:$\{x|x \text{是一个正数,且 $x$ 是一个整数}\}$.特别地,如果某条规则是“$x$属于某集合”,我们通常会将这个条件写到单竖线的前面,如全体正整数的集合:$\{x\in\mathbb{Z}|\text{$x$ 是一个正数}\}$. 这里,$\in$是一个简写的符号,$A\in B$等于说“$A$是$B$的元素”.

有一个特殊的集合,它不含有任何元素,被称为\textbf{空集(empty set)},记作 $\varnothing$,有时也写作$\phi$. $\varnothing$是一切集合的子集.

\subsubsection{元素和子集}
若 $a$ 是集合 $A$ 的元素, 我们就说 $a$ \textbf{属于(in)} $A$, 记为 $a \in A$ 或者 $A \ni a$.

如果集合$A$的元素都是集合$B$的元素,那么称$A$是$B$的\textbf{子集(subset)},本书中记为$A\subset B$或者$A\subseteq B$,也可以反过来写为$B\supset A$和$B\supseteq A$.一切集合都是自身的子集.如果$A$是$B$的子集但又和$B$不同,也就是说$A$没有包含$B$的所有元素,那么称$A$是$B$的\textbf{真子集(proper subset)}, 本书中记为 $A\subsetneq B$,$A\subsetneqq B$,$B\supsetneqq A$或$B\supsetneq A$\footnote{有的地方会用$\subset$来表示“真子集”,和我们这里的定义矛盾;本书中$\subset$就表示子集,但是尽量避免使用这个符号,以尽力避免读者的混淆.}.$B$的真子集一定是$B$的子集,但是$B$本身是$B$的子集而非真子集.

注意区分这两个情况,前一个情况中$A$是$B$的元素,后一个情况中$A$是$B$的子集.另外,集合本身也可以是别的集合的元素,元素的概念没有限定,任何事物和概念都可以成为元素,包括集合.不过,集合的集合我们一般用\textbf{集族(family of sets)}或者简单地用\textbf{族(family)}来称呼.


\subsection{集合运算}

集合间可以互相操作,生成新的集合,这种操作被称为集合间的\textbf{运算(operation)}.

$\cap$表示两个集合的交,意思是将两个集合中共有的元素提取出来,组成一个新的集合.比如说,$\mathbb{N^+}$表示全体自然数的集合,$\mathbb{R^+}$表示全体正实数的集合,$\mathbb{Z}$表示全体整数的集合,那么显然我们可以有$\mathbb{N^+}=\mathbb{R^+}\cap\mathbb{Z}$. 多个集合$A_i$的交集,可以写为$A_0\cap A_1\cap A_2\cap A_3\dots$,也可以用一个大号的交集符号简记为$\bigcap A_i$,表示“所有形式为$A_i$的集合的交集”.

类似地,将两个集合中都有的元素提取出来,组成一个新的集合的操作,被称为集合的并,用符号$\cup$ 和$\bigcup$表示.注意,如果两个元素中有相同元素,那么在并集中这个元素只出现一次.这是因为我们关心的是每个元素是否出现在集合中,计算集合元素数量时也不会重复计算同一个元素.这是一个并集的例子:
\begin{equation}
\{\text{猪}, \text{牛}, \text{狗}, \text{羊}, \text{猫}\}\cup\mathbb{N^+}=\{\text{猪}, \text{狗}, \text{猫}, \text{牛}, \text{羊}, 1, 2,3,4,\dots\}
\end{equation}
注意,列举时元素的顺序也不影响集合的本质.

对于集合 $A$ 和 $B$, $A\backslash B$ 或者 $A-B$ 表示他们的\textbf{差集}. 差集所包含的元素是 $A$ 中全体元素中减去 $B$ 中元素, 如果 $B$ 还含有 $A$ 中所没有的元素, 那么这部分元素可以忽略掉.例如,如果令$A=\{0,1,2,3\}$, $B=\{2,3,4\}$, 那么 $A-B=\{0,1\}$.

如果我们划定了一个讨论范围,被讨论的元素构成的集合称为$U$,那么$U$中的任意一个子集$A$都可以进行\textbf{取补集}运算,得到$A^C=U-A$,称为$A$的\textbf{补集(complement)}.

\subsubsection{笛卡尔积}
对于集合$A$和$B$,$A\times B$ 表示集合间的\textbf{笛卡尔积(Cartesian product)}, 得到一个新的集合. $A\times B$ 中的元素表示为 $(a,b)$,其中$a\in A$, $b\in B$.用集合论的术语表达就是
\begin{equation}\label{Set_eq1}
A\times B=\{(a,b)|a\in A, b\in B\}
\end{equation}
例如,如果令$A=\{0,1,2,3\}, B=\{a, b, c\}$,那么
\begin{equation}
A\times B=\{ (0,a),(0,b),(0,c),(1,a),(1,b),(1,c),(2,a),(2,b),(2,c),(3,a),(3,b),(3,c) \}
\end{equation}
可以看到,集合$A$有4个元素,集合$B$有3个元素,而集合$A\times B$有3$\times$4=12个元素.

\subsection{集合运算的规律}

\textbf{de Morgan公式(de Morgan's laws):}$(\bigcup A_i)^C=\bigcap A_i^C$,$(\bigcap A_i)^C=\bigcup A_i^C$.用自然语言表达,就是:交集的补等于补集的并,并集的补等于补集的交.

\textbf{分配律(distribution laws):}$A\cup(\bigcap B_i)=\bigcap (A\cup B_i)$,$A\cap(\bigcup B_i)=\bigcup (A\cap B_i)$.用自然语言表达,就是:并集运算对交集运算满足分配律,交集运算也对并集运算满足分配律,就像小学所学的乘法对加法满足分配律一样.

\subsection{拓展}
目前为止我们只介绍了集合之间的运算以及集合中元素的数目,但没有对集合本身建立一个结构.从基数的角度来看,只要两个集合之间存在一一对应,那么就可以把它们看作同一个集合,因为在我们的讨论范围里它们都是完全一致的(只讨论了基数).但是这样很无聊,没有太多研究的意义,因此数学家们开始给集合赋予各种各样的结构,进一步细分集合的分类,由此诞生了拓扑学、代数学等分支.现代数学的绝大部分分支都是通过给集合赋予结构来描述的,可以说集合论是现代数学的基石,几乎每一条数学定理都是集合论的定理.
