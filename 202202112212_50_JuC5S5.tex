% 常用的数学运算
% keys 常用的数学运算

本文授权转载自郝林的 《Julia 编程基础》. 原文链接:\href{https://github.com/hyper0x/JuliaBasics/blob/master/book/ch05.md}{第 5 章 数值与运算}.


\subsection{5.5 常用的数学运算}

Julia 中的一些操作符可以用于数学运算或位运算(也就是比特运算).这样的操作符也可以被称为运算符.因此,我们就有了数学运算符和位运算符这两种说法.

\subsubsection{5.5.1 数学运算符}

可用于数学运算的运算符请见下表.
\begin{table}[ht]
\centering
\caption{数学运算符}\label{JuC5S5_tab1}
\begin{tabular}{|c|c|c|c|}
\hline
运算名称 & 运算符 & 示意表达式 & 用途 \\
\hline
一元加 & $+$ & $+x$ & 求$x$的原值 \\
\hline
一元减 & $-$ & $-x$ & 求$x$的相反数,相当于$0-x$ \\
\hline
平方根 & √ & √$x$ & 求 $x$ 的平方根 \\
\hline
二元加 & $+$ & $x + y$ & 求 $x$ 和 $y$ 的和 \\
\hline
二元减 & $-$ & $x - y$ & 求 $x$ 与 $y$ 的差 \\
\hline
乘 & $*$ & $x * y$ & 求 $x$ 和 $y$ 的积 \\
\hline
除 & $/$ & $x / y$ & 求 $x$ 与 $y$ 的商 \\
\hline
逆向除 & \ & $x$ \ $y$ & 相当于 $y / x$ \\
\hline
整除 & ÷ & $x$ ÷ $y$ & 求 $x$ 与 $y$ 的商且只保留整数 \\
\hline
求余运算 & $\%$ & x $\%$ y & 求 $x$ 除以 $y$ 后得到的余数 \\
\hline
幂运算 & ^ & $x$ ^ $y$ & 求 $x$ 的 $y$ 次方 \\
\hline
\end{tabular}
\end{table}

可以看到,Julia 中通用的数学运算符共有 9 个.其中,与\verb|+|和\verb|-|一样,\verb|√|也是一个一元运算符.它的含义是求平方根.在REPL环境中,我们可以通过输入\verb|\sqrt[Tab]|写出这个符号.我们还可以用函数调用\verb|sqrt(x)|来替代表达式\verb|√x|.

所谓的一元运算是指,只有一个数值参与的运算,比如\verb|√x|.更宽泛地讲,根据参与操作的对象的数量,操作符可被划分为一元操作符(unary operator)、二元操作符(binary operator)或三元操作符(ternary operator).其中,参与操作的对象又被称为操作数(operand).

除上述的运算符之外,Julia还有一个专用于\verb|Bool|类型值的一元运算符\verb|!|,称为求反运算符.它会将\verb|true|变为\verb|false|,反之亦然.

这些数学运算符都是完全符合数学逻辑的.所以我在这里就不再展示它们的示例了.

\subsubsection{5.5.2 位运算符}

我们都知道,任何值在底层都是根据某种规则以二进制的形式存储的.数值也不例外.我们把以二进制形式表示的数值简称为二进制数.所谓的位运算,就是针对二进制数中的比特(或者说位)进行的运算.这种运算可以逐个地控制数中每个比特的具体状态(\verb|0|或\verb|1|).

Julia 中的位运算符共有7个.如下表所示.
\begin{table}[ht]
\centering
\caption{位运算符}\label{JuC5S5_tab4}
\begin{tabular}{|c|c|c|c|}
\hline
运算名称 & 运算符 & 示意表达式 & 简要说明 \\
\hline
按位求反 & ~ & ~x & 求 x 的反码,相当于每一个二进制位都变反 \\
\hline
按位求与 & \ & $x$ $&$ $y$ & 逐个对比 x 和 y 的每一个二进制位,只要有0就为0,否则为1 \\
\hline
按位求或& ` & ` & `x \\
\hline
按位异或 & ⊻ & x ⊻ y & 逐个对比 x 和 y 的每一个二进制位,只要不同就为1,否则为0 \\
\hline
逻辑右移 & >>> & x >>> y & 把 x 中的所有二进制位统一向右移动 y 次,并在空出的位上补0 \\
\hline
算术右移 & >> & x >> y & 把 x 中的所有二进制位统一向右移动 y 次,并在空出的位上补原值的最高位 \\
% \hline
% 逻辑左移 & << & x << y & 把 x 中的所有二进制位统一向左移动 y 次,并在空出的位上补0
\hline
\end{tabular}
\end{table}

利用\verb|bitstring|函数,我们可以很直观地见到这些位运算符的作用.例如:
\begin{lstlisting}[language=julia]
julia> x = Int8(-10)
-10

julia> bitstring(x)
"11110110"

julia> bitstring(~x)
"00001001"

julia> 
\end{lstlisting}

可以看到,按位求反的运算符\verb|~|会把\verb|x|中的每一个比特的状态都变反(由\verb|0|变成\verb|1|或由\verb|1|变成\verb|0|).这也是Julia中唯一的一个只需一个操作数的位运算符.因此,它与前面的\verb|+|和\verb|-|一样,都可以被称为一元运算符.

我们再来看按位求与和按位求或:
\begin{lstlisting}[language=julia]
julia> y = Int8(17)
17

julia> bitstring(x)
"11110110"

julia> bitstring(y)
"00010001"

julia> bitstring(x & y)
"00010000"

julia> bitstring(x | y)
"11110111"

julia>
\end{lstlisting}

我们定义变量\verb|y|,并由它来代表\verb|Int8|类型的整数\verb|17|.\verb|y|的二进制表示是\verb|00010001|.对比变量\verb|x|的二进制表示\verb|11110110|,它们只在左边数的第 4 位上都为\verb|1|.因此,\verb|x & y|的结果就是\verb|00010000|.另一方面,它们只在右数第 4 位上都为\verb|0|,所以x | y的结果就是\verb|11110111|.

按位异或的运算符\verb|⊻|看起来很特别.因为在别的编程语言中没有这个操作符.在 REPL 环境中,我们可以通过输入\verb|\xor[Tab]|或\verb|\veebar[Tab]|写出这个符号.我们还可以用函数调用\verb|xor(x, y)|来替代表达式\verb|x ⊻ y|.

我们在前表中也说明了,\verb|x ⊻ y|的含义就是逐个对比\verb|x|和\verb|y|的每一个二进制位,只要不同就为\verb|1|,否则为\verb|0|.示例如下:
\begin{lstlisting}[language=julia]
julia> bitstring(x), bitstring(y), bitstring(x ⊻ y)
("11110110", "00010001", "11100111")

julia> 
\end{lstlisting}

Julia 提供了 3 种位移运算,分别是逻辑右移、算术右移和逻辑左移.下面是演示代码:
\begin{lstlisting}[language=julia]
julia> bitstring(x)
"11110110"

julia> bitstring(x >>> 3)
"00011110"

julia> bitstring(x >> 3)
"11111110"

julia> bitstring(x << 3)
"10110000"

julia>
\end{lstlisting}

在位移运算的过程中,数值的宽度(或者说占用的比特数)是不变的.我们可以把承载一个数值的存储空间看成一条板凳,而数值的宽度就是这条板凳的宽度.现在,有一条板凳承载了\verb|x|变量代表的那个整数,并且宽度是\verb|8|.也就是说,这条板凳上有 8 个位置,可以坐 8 个比特(假设比特是某种生物).

每一次位移,板凳上的 8 个比特都会作为整体向左或向右移动一个位置.在移动完成后,总会有 1 个比特被挤出板凳而没有位置可坐,并且也总会有 1 个位置空出来.比如,如果向右位移一次,那么最右边的那个比特就会被挤出板凳,同时最左边会空出一个位置.没有位置可坐的比特会被淘汰,而空出来的位置还必须引进 1 个新的比特.

好了,我们现在来看从\verb|11110110|到\verb|00011110|的运算过程.后者是前者逻辑右移三次之后的结果.按照前面的描述,在向右移动三次之后,最右边的 3 个比特被淘汰了.因此,这时的二进制数就变为了\verb|11110|.又由于,逻辑右移运算会为所有的空位都填补\verb|0|(状态为\verb|0|的比特),所以最终的二进制数就是\verb|00011110|.
\begin{figure}[ht]
\centering
\includegraphics[width=12.5cm]{./figures/JuC5S5_1.png}
\caption{逻辑右移的示意} \label{JuC5S5_fig1}
\end{figure}

与逻辑右移相比,算术右移只有一点不同,那就是:它在空位上填补的不是\verb|0|,而是原值的最高位.什么叫最高位?其实它指代的就是位置最高的那个比特.对于一个二进制数,最左边的那个位置就是最高位,而最右边的那个位置就是最低位.\verb|x|的值\verb|11110110|的最高位是\verb|1|.因此,在算术右移三次之后,我们得到的新值就是\verb|11111110|.

与右移运算不同,左移运算只有一种.我们把它称为逻辑左移.这主要是因为该运算也会为空位填补\verb|0|.所以,\verb|11110110|经过逻辑左移三次之后就得到了\verb|10110000|.

\subsubsection{5.5.3 运算同时赋值}

Julia 中的每一个二元的数学运算符和位运算符都可以与赋值符号\verb|=|联用,可称之为更新运算符.联用的含义是把运算的结果再赋给参与运算的变量.例如:
\begin{lstlisting}[language=julia]
julia> x = 10; x %= 3
1

julia>
\end{lstlisting}

REPL 环境回显的\verb|1|就是变量\verb|x|的新值.但要注意,这种更新运算相当于把新的值与原有的变量进行绑定,所以原有变量的类型可能会因此发生改变.示例如下:
\begin{lstlisting}[language=julia]
julia> x = 10; x /= 3
3.3333333333333335

julia> typeof(x)
Float64

julia> 
\end{lstlisting}

显然,\verb|x|变量原有的类型肯定是某个整数类型(\verb|Int64|或\verb|Int32|).但更新运算使它的值变成了一个\verb|Float64|类型的浮点数.因此,该变量的类型也随之变为了\verb|Float64|.

所有的更新运算符罗列如下:
\begin{lstlisting}[language=julia]
+= -= *= /= \= ÷= %= ^= &= |= ⊻= >>>= >>= <<=
\end{lstlisting}

前 8 个属于数学运算,后 6 个属于位运算.

\subsubsection{5.5.4 数值的比较}

理所应当,数值与数值之间是可以比较的.在 Julia 中,这种比较不但可以发生在同类型的值之间,还可以发生在不同类型的值之间,比如整数和浮点数.通常,比较的结果会是一个\verb|Bool|类型的值.

对于整数之间的比较,我们就不多说了.它与数学中的标准定义没有什么两样.至于浮点数,相关操作仍然会遵循 IEEE 754 技术标准.这里存在 4 种互斥的比较关系,即:小于(less than)、等于(equal)、大于(greater than)和无序的(unordered).

具体的浮点数比较规则如下:
\begin{itemize}
\item 只要参与比较的两个数值中有一个是 NaN,比较的结果就必然是\verb|false|.因为 NaN 不与任何东西相等,包括它自己.或者说,这种情况下的所有比较关系都是无序的.
\item Inf 等于它自己,并且一定大于除了 NaN 之外的任何数.
\item -Inf 等于它自己,并且一定小于除了 NaN 之外的任何数.
\item 正零(0.0)和负零(-0.0)是相等的.尽管它们在底层存储上是不同的.
\item 其他情况下的有限浮点数比较将会按照数学中的标准定义进行.
\end{itemize}

Julia 中标准的比较操作符如下表.
\begin{table}[ht]
\centering
\caption{比较操作符}\label{JuC5S5_tab2}
\begin{tabular}{|c|c|}
\hline
操作符 & 含义 \\
\hline
== & 等于 \\
\hline
!= ≠ & 不等于 \\
\hline
< & 小于 \\
\hline
<= ≤ & 小于或等于 \\
\hline
> & 大于 \\
\hline
>= ≥ & 大于或等于 \\
\hline
\end{tabular}
\end{table}

注意,对于不等于、小于或等于以及大于或等于,它们都有两个等价的操作符可用.表中已用空格将它们分隔开了.

这些比较操作符都可以用于链式比较,例如:
\begin{lstlisting}[language=julia]
julia> 1 < 3 < 5 > 2
true

julia> 
\end{lstlisting}

只有当链式比较中的各个二元比较的结果都为\verb|true|时,链式比较的结果才会是\verb|true|.注意,我们不要揣测链中的比较顺序,因为 Julia 未对此做出任何定义.

在这些比较操作符当中,我们需要重点关注一下\verb|==|我们之前使用过一个用于判断相等的操作符\verb|===|.另外,还有一个名叫\verb|isequal|的函数也可以用于判等.我们需要明确这三者之间的联系和区别.

首先,操作符\verb|===|代表最深入的判等操作.我们在前面说过,对于可变的值,这个操作符会比较它们在内存中的存储地址.而对于不可变的值,该操作符会逐个比特地比较它们.

其次是操作符\verb|==|.它完全符合数学中的判等定义.它只会比较数值本身,而不会在意数值的类型和底层存储方式.对于浮点数,这种判等操作会严格遵循 IEEE 754 技术标准.顺便说一句,在判断两个字符串是否相等时,它会逐个字符地进行比较,而忽略其底层编码.

函数\verb|isequal|用于更加浅表的判等.在大多数情况下,它的行为都会依从于操作符\verb|==|.在不涉及浮点数的时候,它会直接返回\verb|==|的判断结果.那为什么说它更加浅表呢?这是因为,对于那些特殊的浮点数值,它只会去比较它们的字面含义.它同样会判断两个 Inf(或者两个 -Inf)是相等的,但也会判断两个 NaN 是相等的,还会判断\verb|0.0|和\verb|-0.0|是不相等的.这些显然并未完全遵从 IEEE 754 技术标准中的规定.下面是相应的示例:
\begin{lstlisting}[language=julia]
julia> isequal(NaN, NaN)
true

julia> isequal(NaN, NaN16)
true

julia> isequal(Inf32, Inf16)
true

julia> isequal(-Inf, -Inf32)
true

julia> isequal(0.0, -0.0)
false

julia> 
\end{lstlisting}

另外,\verb|===|和\verb|isequal|无论如何都会返回一个\verb|Bool|类型的值作为结果.操作符\verb|==|在绝大多数情况下也会如此.但当至少有一方的值是\verb|missing|时,它就会返回\verb|missing|.\verb|missing|是一个常量,也是类型\verb|Missing|的唯一实例.它用于表示当前值是缺失的.

下面的代码展示了上述 3 种判等操作在涉及\verb|missing|时的判断结果:
\begin{lstlisting}[language=julia]
julia> missing === missing
true

julia> missing === 0.0
false

julia> missing == missing
missing

julia> missing == 0.0
missing

julia> isequal(missing, missing)
true

julia> isequal(missing, 0.0)
false

julia> 
\end{lstlisting}

最后,对于不同类型数值之间的比较,Julia 一般会贴合数学上的定义.比如:
\begin{lstlisting}[language=julia]
julia> 0 == 0.0
true

julia> 1/3 == 1//3
false

julia> 1 == 1+0im
true

julia> 
\end{lstlisting}

\subsubsection{5.5.5 操作符的优先级}

Julia 对各种操作符都设定了特定的优先级.另外,Julia 还规定了它们的结合性.操作符的优先级越高,它涉及的操作就会越提前进行.比如:对于运算表达式\verb|10+3^2|来说,由于运算符\verb|^|的优先级比作为二元运算符的\verb|+|更高,所以幂运算\verb|3^2|会先进行,然后才是求和运算.

操作符的结合性主要用于解决这样的问题:当一个表达式中存在且仅存在多个优先级相同的操作符时,操作的顺序应该是怎样的.一个操作符的结合性可能是,从左到右的、从右到左的或者未定义的.像我们在前面说的比较操作符的结合性就是未定义的.

下表展示了本章所述运算符的优先级和结合性.上方运算符的优先级会高于下方的运算符.
\begin{table}[ht]
\centering
\caption{运算符的优先级和结合性}\label{JuC5S5_tab3}
\begin{tabular}{|c|c|c|}
\hline
操作符&用途&结合性 \\
\hline
+ - √ ~ ^ & 一元的数学运算和位运算,以及幂运算 & 从右到左的 \\
\hline
<< >> >>> & 位移运算 & 从左到右的 \\
\hline
* / \ ÷ \% $&$ & 乘法、除法和按位与 & 从左到右的 \\
\hline
`+ - ⊻` & 加法、减法、按位或和按位异或 & \\
\hline
== != < <= > >= === !== & 比较操作 & 未定义的 \\
\hline
`= += -= *= /= = ÷= \%= ^=  == ⊻= >>>= >>= <<=` & 赋值操作和更新运算 & \\
\hline
\end{tabular}
\end{table}

此外,数值字面量系数(如\verb|-3x+1|中的\verb|x|)的优先级略低于那几个一元运算符.因此,表达式\verb|-3x|会被解析为\verb|(-3) * x|,而表达式\verb|√4x|则会被解析为\verb|(√4) * x|.可是,它与幂运算符的优先级却是相当的.所以,表达式\verb|3^2x|和\verb|2x^3|会被分别解析为\verb|3^(2x)|和\verb|2 * (x^3)|.也就是说,它们之间会依照从右到左的顺序来结合.

对于运算表达式,我们理应更加注重正确性和(人类)可读性.因此,我们总是应该在复杂的表达式中使用圆括号来明确运算的顺序.比如,表达式\verb|(2x)^3|的运算顺序就一定是先做乘法运算再做幂运算.不过,过多的括号有时也会降低可读性.所以我们往往需要对此做出权衡.如有必要,我们可以分别定义表达式的各个部分,然后再把它们组合在一起.