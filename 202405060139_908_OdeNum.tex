% 常微分方程(组)的数值解
% keys 常微分方程组|微分方程|数值计算|Matlab
% license Xiao
% type Tutor

\pentry{天体运动的简单数值计算\nref{nod_KPNum0}}{nod_ee05}

数值解微分方程或微分方程组时, 一般需要先把微分方程(组)化成一阶微分方程组。 例如在“\enref{弹簧振子受迫运动的简单数值计算}{SHOFN}” 中列出的微分方程为
\begin{equation}
m y'' = \alpha y' - ky + f(t)~.
\end{equation}
新增变量, 令 $v = y'$, 则可变为一阶微分方程组
\begin{equation}
\begin{cases}
v' = [-\alpha u - ky + f(t)]/m\\
y' = v
\end{cases}~.
\end{equation}
方程组中, $t$ 是自变量, $y$ 和 $u$ 是 $t$ 的函数。 给出某时刻的 $y(t), u(t), t$ 就可以通过方程组求出 $u'(t)$ 和 $y'(t)$ 的数值。

又例如在“\enref{天体运动的简单数值计算}{KPNum0}”中, 列出的二阶微分方程组为
\begin{equation}
\leftgroup{
x'' &= -\frac{GM}{(x^2+y^2)^{3/2}}x\\
y'' &= -\frac{GM}{(x^2+y^2)^{3/2}}y
}~.\end{equation}

新增变量, 令 $v_x = x'$, $v_y = y'$, 上式也可变为一阶微分方程组
\begin{equation}\label{eq_OdeNum_4}
\begin{cases}
x' = v_x\\
v'_x = -GMx/(x^2 + y^2)^{3/2}\\
y' = v_y\\
v'_y = -GMy/(x^2 + y^2)^{3/2}
\end{cases}~.
\end{equation}
该式中同样 $t$ 是自变量, 其他都是 $t$ 的函数, 给出某时刻的 $x, y, v_x, v_y, t$ 就可以由该方程组求出各函数的一阶导数。

在以上两个例子中, 我们使用了一种较为原始的方法(微分近似)。 这种方法相当于把某时刻 $t$ 的各函数值代入一阶微分方程组, 得到 $t$ 时刻各函数的一阶导数, 再通过微分近似由这些一阶导数来计算其 $[t, t + \Delta t]$ 时间内的增量, 得到 $t +\Delta t$ 时刻的各函数值, 再代入一阶方程组得到 $t +\Delta t$ 时刻各函数的一阶导数, 如此一直循环, 得到各函数每隔 $\Delta t$ 时间的值。 这种方法叫做\textbf{欧拉法}。 对一阶常微分方程 $y'(t) = f(y, t)$, 令 $h = \Delta t$, $t_n = t_0 + nh$, $y_n = y(t_n)$ 欧拉法可以表示为
\begin{equation}\label{eq_OdeNum_5}
y_{n+1} = y_n + h f(y_n, t_n)~.
\end{equation}
%未完成: 应该在这里就介绍矢量表达式!

以下介绍三种更精确的算法, 在步长($\Delta t$) 相同时它们的精确度递增。 注意虽然大多数问题中自变量是时间, 但这些方法适用于大多数单个自变量的微分方程(组)。

%\addTODO{链接}
\begin{itemize}
\item \enref{中点法}{OdeMid}
\item \enref{四阶龙格库塔法}{OdeRK4}
\item Matlab 中的 Ode45 算法
\end{itemize}












