% 张量的分类
% 张量|型张量|张量类型|张量指标|指标升降

\begin{issues}
\issueDraft
\end{issues}

\pentry{协变和逆变\upref{CoCon}}

协变和逆变\upref{CoCon}词条中提到,尽管张量是多个任意的线性空间$V$之间的多重线性映射,我们通常只考虑用$V$和其对偶空间$V^*$定义的张量,这样只需要定义一个空间的基就可以得到所有空间的基了.

回忆张量\upref{Tensor}词条中所教的判断张量阶数的方法:看一共有多少线性空间参与映射,不过是作为自变量的一部分还是像的一部分.在协变和逆变\upref{CoCon}词条中我们介绍了定义一个张量时给线性空间分类的方法,即分为$V$和$V^*$,这样就可以进一步把张量分出类型来,比阶数更细致一些.

\begin{definition}{}
一个$(m, n)$型张量是涉及了$n$个逆变向量和$m$个协变向量的$(m+n)$阶张量.
\end{definition}
