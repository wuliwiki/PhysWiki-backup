% 电路和水路的类比

\begin{issues}
\issueDraft
\end{issues}

\subsection{导线}
形象来说, 我们可以把电路中的导线类比为水平面上的水管, 管道中充满某种不可压缩的非粘稠流体, 即该流体不会因自身的摩擦损耗能量. 导线中自由电子的电荷可以类比流体的质量, 电流则类比液体单位时间流经管道横截面的质量, 称为\textbf{水流}, 注意这区别于水流的速度即\textbf{流速}. 没有电阻的理想导线类比为内壁无摩擦的管道.

我们可以认为一般情况下水流的\textbf{流速}极小, 正如一般情况电子在电路中移动的速度同样很小. 这使得在出现压强差的一瞬间水流就可以从零加速到某个正常值. 如果管道中没有任何阻力, 那么可以认为在其两端加上压强差的一瞬间水流就会变得远大于正常值, 可以视为无穷大. 注意这并不是数学意义上的无穷大, 物理中时常会把远大于一般情况的数值视为无穷大.

在该模型中, 电势可以理解为管道内某点处的绝对水压. 事实上把整个管道系统各处的绝对水压整体增加一个常数并不会影响水流动的方式和快慢, 所以我们主要关心的是管道两点之间的\textbf{压强差}, 正如在电路中我们关心的不是绝对电势而是电势差, 即电压.

\subsection{电阻}
在以上类比中, 电路中的电阻\upref{Resist}可以看成在管道中塞入一块海绵用于阻碍水流. 海绵越密对水的阻力就越大, 只有通过增加海绵两侧的压强差才能让水流变快. 这样, 欧姆定律\upref{Resist} 类比过来就可以理解为水流和海绵两侧的压强差成正比. 由于上文中提到水流的一般流速很慢, 我们可以认为当压强瞬突然改变时, 流速也会突然改变, 即加速或减速的时间很短可以忽略不记.

电阻所消耗的功率等于两端电压乘以电流. 类比过来就是: 水流对海绵之间摩擦力正比于力乘以流速, 即正比于海绵两端压强差乘以水流大小.

\subsection{电源}
电源可以看成一个水泵, 当水泵两端的水管都是封闭的, 那么水泵运行时输出水管中的绝对水压将会大于输入水管中的绝对水压, 于是在水泵两端就形成了一个压强差. 但由于此时水没有流动, 水泵虽然施加了压力, 但做功为零. 电源的开路电动势(电压)可以类比为此时的水压压强差.

\begin{exercise}{}
考虑以下由电源和电阻组成的回路, 你是否能想象出一个等效的水管回路以及水管中压强的分布?
\addTODO{一个电源驱动一个电阻}
\end{exercise}

\subsection{电容}
一个圆柱容器中间有弹簧隔板, 当隔板两边存在压强差时, 隔板 当弹簧弹性系数不变时, 容器横截面越大在相同的压强差下蓄水越多, 对应的电容就越大.
\begin{figure}[ht]
\centering
\includegraphics[width=8cm]{./figures/EleWat_1.pdf}
\caption{用于类比电容的水容器} \label{EleWat_fig1}
\end{figure}

电容所储存的能量类比到该模型就是弹簧的弹性势能.
