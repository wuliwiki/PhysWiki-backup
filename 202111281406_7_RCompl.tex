% 实数的完备公理
% keys 戴德金分割|区间套|闭区间套|确界原理|单调有界定理|有限覆盖定理|聚点定理|致密性定理|柯西收敛准则|Cauchy收敛准则

\pentry{实数\upref{ReNum}}


我们常见到一种争论,$0.\dot{9}$到底等不等于$1$.实际上这个问题无法证明,而是被当作定义实数的公理之一,我们称之为完备性公理.你可以留意一下各种各样所谓的“证明”,认为$0.\dot{9}\neq 1$的论证通常都是否认了完备公理,而认为$0.\dot{9}=1$的都默认了完备公理.

用$0.\dot{9}=1$来当作完备公理很不好用,我们通常使用本节介绍的的完备公理来描述实数的完备性,这些公理彼此是等价的,并且都可以推出$0.\dot{9}=1$.

我们先列举几条完备公理,再展开证明它们互推的逻辑链条.

\subsection{完备公理的表述}

由于几个完备公理是等价的,可以互推,因此实际建立理论时只挑其中一个作为公理体系的一部分,其它的都当作定理,这也导致我们也常把这几条公理称为“定理”.因此,我们使用“定理”的格式来列举这几条完备公理.

\begin{theorem}{确界原理}\label{RCompl_the1}
实数集的任何非空有界子集,必有一个实数上确界(见\textbf{上确界与下确界}\upref{SupInf}).
\end{theorem}

\begin{theorem}{单调有界收敛定理}\label{RCompl_the2}
单调有界数列必有极限.
\end{theorem}

\begin{theorem}{(闭)区间套定理}\label{RCompl_the3}
设$a_n$是单调递增数列,$b_n$是单调递减数列,$b_n-a_n$恒为正数且收敛到$0$,且$[a_{n+1}, b_{n+1}]$都是$[a_n, b_n]$的真子集.称这样的集合$\{[a_n, b_n]_{n=1}^\infty\}$为一个\textbf{(闭)区间套}.

对于任意的区间套$\{[a_n, b_n]\}$,存在唯一的实数$x_0$使得$x_0\in [a_n, b_n]$对任意正整数$n$成立.
\end{theorem}

\begin{theorem}{Heine-Borel有限覆盖定理}\label{RCompl_the4}
设$[a, b]$是一个区间,$C$是$[a, b]$的一个开覆盖\footnote{即$C$是开集的集合,使得$C$中所有开集的并包含了$[a, b]$.实数轴上的开集是指开区间的并.},那么$C$中存在有限个开集,使得其并集包含了$[a, b]$.
\end{theorem}

\begin{theorem}{Bolzano致密性定理}\label{RCompl_the5}
有界无穷数列必有收敛子列.
\end{theorem}

\begin{theorem}{Weierstrass聚点定理}\label{RCompl_the6}
有界无穷点集必有聚点(\autoref{Topo0_def1}~\upref{Topo0}).
\end{theorem}

\begin{theorem}{Cauchy收敛准则}\label{RCompl_the7}
数列$\{a_k\}$收敛,当且仅当对于任意$\epsilon>0$,存在$N_\epsilon$使得对于任意$m, n>N_\epsilon$,都有$\abs{a_m-a_n}<\epsilon$.
\end{theorem}

以上七条就是最常见的实数完备公理,任取其一都可以用来定义实数的完备性、而把其它的当成定理.

\subsection{完备性定理的互相推出}

\subsubsection{\autoref{RCompl_the1} $\to$ \autoref{RCompl_the2} }

首先,确界原理也等价于“有界子集必有下确界”,只需要对子集里各实数取负值构成新的有界子集,取到新子集的上确界再取负值,就得到原子集的下确界了.因此我们这里不妨设单调有界数列是单调不减的,单调不增的证明方式完全一致.

取单调不增有界数列的全体函数值$\{a_n\}$,构成实数集的一个子集$S$.由于是有界数列,故$S$是有界点集.于是由\autoref{RCompl_the1} ,$S$有一个上确界$a$.

由数列单调性,$\abs{a-a_n}$随着$n$的增大而减小.同时由上确界的定义,对于任意的$\epsilon>0$,都存在$N_\epsilon$,使得只要$n>N_\epsilon$就有$\abs{a-a_n}<\epsilon$,而这就是$\{a_n\}$收敛于$a$的定义.

\subsubsection{\autoref{RCompl_the2} $\to$ \autoref{RCompl_the3} }

区间套中的$\{a_n\}$和$\{b_n\}$都是单调有界数列\footnote{比如说,$\{b_n\}$是单调不增的,因此$b_1$是其上界;同时由于$b_n>a_1$,$a_1$也就是其下界.},由\autoref{RCompl_the2} 可知它们都收敛.

设$\{a_n\}$收敛到$a$,$\{b_n\}$收敛到$b$.

由单调性,对任意$n$都有$\abs{b_n-a_n}\geq\abs{b-a}$以及$a, b\in [a_n, b_n]$.但是由区间套的定义,$\abs{b_n-a_n}$趋于零,故$\abs{b-a}=0$,即$a=b$.

这么一来,$a=b$就是区间套中唯一的公共元素.

\subsubsection{\autoref{RCompl_the3} $\to$ \autoref{RCompl_the4} }

反设存在$[a, b]$的一个开覆盖$S$,使得$S$的任何有限子集都不能覆盖$[a, b]$.不失一般性地,设$S$中的元素都是开区间\footnote{可以这样简化是因为开集都是开区间的并集.}.

将$[a, b]$二等分为$[a, \frac{a+b}{2}]$和$[\frac{a+b}{2}, b]$,那么由假设,这两个二等分闭区间中至少有一个不能被$S$中的有限子集覆盖.取一个不能被有限覆盖的二等分区间,记它为$[a_1, b_1]$.

由于$[a_1, b_1]$也是不能被$S$的有限子集覆盖的闭区间,故我们可以同样取出它的一个二等分区间$[a_2, b_2]$,使得这个新的二等分区间不能被$S$的有限子集覆盖.以此类推,我们可以取出一系列$[a_k, b_k]$,其中每一个都不能被$S$的有限子集覆盖.由于每一步都是取二等分,故$\abs{b_k-a_k}=\frac{\abs{b-a}}{2^k}$,因此$[a_k, b_k]$是一个区间套.

由区间套定理,存在唯一的$x_0\in[a, b]$使得$x_0\in[a_k, b_k]$对所有$k$成立.由于$S$是覆盖,故$S$中至少有一个元素是包含$x_0$的,设这个元素是区间$(a_0, b_0)$.设$\min(\abs{x_0-a_0}, \abs{x_0-b_0})=r$,那么取$k$使得$\abs{b_k-a_k}<r$,所对应的$[a_k, b_k]$就是$(a_0, b_0)\in S$的子集,于是$[a_k, b_k]$就被$S$的有限子集覆盖了,矛盾!

因此反设不成立,$S$必有有限子覆盖.

\subsubsection{\autoref{RCompl_the4} $\to$ \autoref{RCompl_the5} }

设$\{x_k\}$是一个有界数列,即存在实数$a<b$使得$a<x_k<b$对任意$k$都成立.

如果$s\in(a, b)$不是$\{x_k\}$任何子列的收敛点,那么必然存在$r_s>0$使得$\{x_k\}\cap (s-r_s, s+r_s)=\{s\}$,即数列中没有除了$s$本身外的元素到$s$的距离小于$r_s$\footnote{这是因为,“$s$是某个子列的收敛点”等价于“对于任意的距离$r$,都存在数列的元素,其靠近$s$的距离小于$r$”.将这一表述否定,就得到本段的表述了.}.

反设$\{x_k\}$没有收敛子列,即任何$s\in(a, b)$都不是其收敛点.按上段论述,我们就可以对每一个$s$取一个半径$r_s$,使得$(s-r_s, s+r_s)$中只有数列的一个元素.对所有$s\in(a, b)$都取这样的区间,则我们就得到了$[a, b]$的一个开覆盖.由有限覆盖定理,存在有限多个$s$和$r_s$的组合,使得对应的$\{(s-r_s, s+r_s)\}$是有限子覆盖.这么一来,$\{x_k\}$的取值就在这些$s$中,即只有有限多取值.但是我们有可数无穷多个$x_k$,所以其中必有无穷多个等于同一个$s$,它们就能构成一个收敛子列.矛盾!

因此反设不成立,$\{x_k\}$必有收敛子列.

\subsubsection{\autoref{RCompl_the5} $\to$ \autoref{RCompl_the6} }

因为有无穷多个点,所以我们可以每次取一个出来,每次取的都和前面不重复.这样,我们就得到一个有界无穷数列.由致密性定理,该数列必有收敛点.该收敛点就是该点集的聚点.



\subsubsection{\autoref{RCompl_the6} $\to$ \autoref{RCompl_the7} }

回顾数列收敛的定义:$\{a_n\}$收敛,当且仅当存在实数$a$使得$\lim\limits_{n\to \infty}\abs{a_n-a}=0$.我们要证明当\autoref{RCompl_the6} 成立时, \autoref{RCompl_the7} 的设定(Cauchy收敛准则)等价于数列收敛的定义.

给定数列$\{a_k\}$,使得$\epsilon>0$,存在$N_\epsilon$使得对于任意$m, n>N_\epsilon$,都有$\abs{a_m-a_n}<\epsilon$.于是这个数列是有界数列,其值域是一个有界无穷点集.由聚点定理,存在一个$x_0$,使得它是$\{a_k\}$值域的一个聚点.

任取$\epsilon>0$,由聚点的定义,必然存在无穷多个$a_k$使得$\abs{a_k-x_0}<\epsilon$;由题设,存在$N_\epsilon$使得对于任意$m, n>N_\epsilon$,都有$\abs{a_m-a_n}<\epsilon$.综合起来,存在一个$M_\epsilon$,使得对于任意$r>M_\epsilon$,都有$\abs{a_r-a_{M_\epsilon}}<\epsilon$,且$\abs{a_{M_\epsilon}-x_0}<\epsilon$.因此$\abs{a_r-x_0}<2\epsilon$.

重新整理一下以上表述,就是:任取$\epsilon>0$,存在$M_\epsilon$,使得对于任意$r>M_\epsilon$,都有$\abs{a_r-x_0}<2\epsilon$.这正是$\{a_k\}$收敛于$x_0$的定义.因此 \autoref{RCompl_the7} 的设定可以推出数列收敛.

反过来,如果数列收敛则Cauchy收敛准则也成立.


\subsubsection{\autoref{RCompl_the7} $\to$ \autoref{RCompl_the1} }

取一个非空有界实数集$S$,已知它有一个上界$a_0$.由于$S$非空,故存在$b_0<a_0$,且$b_0\in S$.

取$a_1=\frac{2^k-1}{2^k}b_0+\frac{1}{2^k}a_0$和$b_1>\frac{2^{k-1}-1}{2^{k-1}}b_0+\frac{1}{2^{k-1}}a_0$,使得$b_1\in S$.



















