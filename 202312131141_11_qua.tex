% 二次型(线性代数)
% license Xiao
% type Tutor


\begin{issues}
\issueTODO 
\begin{itemize}
\item 本篇放在线性代数第五章 线性方程组,面向初学者。用矩阵语言介绍二次型,区分高代中张量形式的二次型(虽然它们当然是一回事)。
\item  本篇最好加上一些应用实例,譬如几何上通过二次型简化方程,以及物理上的惯性张量(?)等等。
\end{itemize}
\end{issues}
\begin{definition}{}
二次型是关于变量的二次齐次多项式。即满足$q(k\textbf{v})=k^2 q(\textbf{v})$,对于任意$\textbf{v}=(x_1,x_2...x_n),k\in \mathbb F$。容易验证,任意二次型都可以写为如下形式:
\begin{equation}
q(\textbf{v}=\textbf{v}^TQ\textbf v)~.
\end{equation}
一般称矩阵$Q$为二次型$q$的矩阵形式。
\end{definition}
从定义可知,二次型可以对应多个矩阵。但是由于对任意变量有$v_iv_j=v_jv_i$,二次型总能对应唯一一个对称矩阵\footnote{前提为:域的特征不为2。因为特征为2的域有:-1=1}。
\subsection{二次型的坐标变换}
当我们用上述定义表示二次型时,如果基向量组变换,二次型的形式亦有所不同。具体而言,如果利用过渡矩阵$B$改变基向量组,由相似变换的知识可知,在新基下向量$\textbf {v'}=B\textbf{v}$,那么新的二次型形式为:
\begin{equation}
\textbf{v}^T Qv=(B^{-1}\textbf{v})^T Q(B^{-1}v)=\textbf{v}^T (B^{-1})^{T}QB^{-1}v~.
\end{equation}
因此,新的二次型形式对应矩阵$Q'=(B^{-1})^{T}QB^{-1}$,这就是常说的合同变换。合同变换的结果是同一二次型在不同基下的表示。

\subsection{二次型的正定性}