% 块对角矩阵

\begin{issues}
\issueDraft
\end{issues}

\pentry{分块矩阵\upref{BlkMat}, 矩阵与线性映射\upref{MatLS}, 子空间的直和、补空间\upref{DirSum}}
若 $N$ 维线性空间 $V$ 中有若干 $N_i$ 维子空间 $V_i$($i=1,\dots,n$), 满足
\begin{equation}
V = V_1 \oplus V_2 \oplus \dots \oplus V_n
\end{equation}
每个 $V_i$ 的一组基底为 $v_{i,j}$ ($j=1,\dots,N_i$). 那么所有的 $v_{i,j}$ 是 $V$ 的一组基底, 注意基底不需要是正交归一的. 定义基底的顺序为
\begin{equation}
v_{1,1},\dots, v_{1,N_1}, v_{2,1}, \dots, v_{2,N_2}, \dots
\end{equation}
此时若线性映射 $A: V\to V$ 在每个 $V_i$ 中都闭合. 那么 $A$ 关于这组基底的矩阵就是块对角矩阵, 第 $i$ 块的大小为 $N_i$. 第 $i$ 块对较快就是算符 $A:V_i\to V_i$ 关于基底 $v_{i,j}$ ($j=1,\dots,N_i$) 的矩阵.

证明留作习题.

\subsection{块对角厄米矩阵的本征问题}
\pentry{厄米矩阵的本征问题\upref{HerEig}}
若一个厄米矩阵 $\mat H$ 是块对角的, 那么每个对角块 $\mat H_i$ 也显然是厄米矩阵. 只需要分别解每个 $\mat H_i$ 的本征方程, 得到相同大小的本征矢列矩阵 $\mat P_i$, 然后按相同顺序拼成块对角矩阵 $\mat P$, 就是 $\mat H$ 的本征矢列矩阵.
