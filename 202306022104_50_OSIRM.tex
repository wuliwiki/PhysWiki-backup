% 开放系统互联基本参考模型
% keys 网络 模型 五层 OSI

\textbf{开放系统互联基本参考模型}(Open System Interconnection Reference Model, OSI/RM)是由国际标准化组织提出的一种试图使得各种计算机在世界范围内互联成网的标准框架。该模型的目标是使得全球计算机能够遵循同一个协议,互相连接并交换数据。

OSI模型具有七个逻辑层次。从下到上依次为:物理层、数据链路层、网络层、传输层、会话层、表示层、应用层。逻辑层次的意思是对于网络上所传输的数据流的封装的层次,而并非是真实物理存在的实体。
\begin{figure}[ht]
\centering
\includegraphics[width=5cm]{./figures/fe3db6a185508986.png}
\caption{OSI七层模型} \label{fig_OSIRM_1}
\end{figure}

(1)物理层(physical layer)

物理层其实就是最底层的通讯层。物理层上传输的是数据的二进制位,即比特(bit)。物理层要做的是用适当的高、低电压来表示0和1。物理层还要确定连接电缆的插头应当有多少根引脚以及各引脚应如何连接。当然,解释比特代表的意思,就不是物理层的任务。请注意,传递信息所利用的一些物理媒体,如双绞线、同轴电缆、光缆、无线信道等,并不在物理层协议之内而是在物.理层协议的下面。因此也有人把物理层下面的物理媒体当作第0层。