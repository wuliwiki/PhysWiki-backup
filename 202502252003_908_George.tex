% 乔治·布尔(综述)
% license CCBYSA3
% type Wiki

本文根据 CC-BY-SA 协议转载翻译自维基百科\href{https://en.wikipedia.org/wiki/George_Boole#}{相关文章}。

\begin{figure}[ht]
\centering
\includegraphics[width=6cm]{./figures/228c41af0dd53eca.png}
\caption{} \label{fig_George_1}
\end{figure}
乔治·布尔(George Boole,发音:/buːl/ 布尔,1815年11月2日-1864年12月8日)是一位主要自学成才的英国数学家、哲学家和逻辑学家,他的大部分短暂职业生涯都在爱尔兰科克的女王学院担任数学教授。他在微分方程和代数逻辑领域有所研究,最著名的作品是《思维的法则》(The Laws of Thought,1854),该书中包含了布尔代数。布尔逻辑对计算机编程至关重要,并被认为为信息时代奠定了基础。

布尔是一个鞋匠的儿子。他接受了初等教育,并通过各种方式学习了拉丁语和现代语言。16岁时,他开始教授工作以养家糊口。19岁时,他创办了自己的学校,后来在林肯经营了一所寄宿学校。布尔积极参与当地社团活动,并与其他数学家合作。1849年,他被任命为爱尔兰科克女王学院(现为科克大学)首任数学教授,在那里他遇见了未来的妻子玛丽·埃弗雷斯特。他继续参与社会事业,并保持与林肯的联系。1864年,布尔因患肺炎引发的胸膜积液而去世。

