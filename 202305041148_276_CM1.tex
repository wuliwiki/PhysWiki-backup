% 动量和能量、一维势能曲线
% keys 动量|动能|能量|势能|势能曲线

\begin{issues}
\issueTODO
\end{issues}

\pentry{经典力学\upref{CM0}, 高中物理相关内容}

如果我们把自己局限在经典力学框架中, 以牛顿三定律为基本假设, 那么力就是最基本的概念, 而动量和能量就只是衍生出来的概念。 本文将以这种思路来讲解动量和能量。 然而需要注意的是在更近代的物理中(例如相对论和量子力学), 我们会把动量和能量当作基本的概念, 力只是定义为动量关于时间的变化率或者势能关于位置的(负)变化率。

\subsection{动量}\label{sub_CM1_1}
为了方便起见, 我们暂时假设物体匀速直线运动。 在高中物理中我们都熟悉动量的定义是 $p = mv$, 可这么定义有什么好处? 为什么值得研究? 为什么不研究 $m^2 v^2$ 或者 $mv^3$?

为了能更好地理解这个量,我们不妨来看动量的另一种等效定义: 动量的大小描述了让一个运动的物体停下来有多困难, 或者让一个物体从静止达到某个状态有多困难。 这个困难程度又可以定义为力在时间上的累加。 例如一个在光滑水平面有动量为 $p$ 的箱子,要使它停下来,人就要沿运动反方向推箱子, 若使用了恒力 $F$ 一段时间 $t$ 后箱子停了下来, 那么箱子的动量大小就定义为 $p = Ft$。 同理,如果要使箱子从静止达到动量 $p$, 就需要以恒力 $F$ 作用在箱子上同样的时间 $t$。 所以\textbf{动量也可以等效地定义为力在时间上的累加}。

由牛顿第二定律 $F = ma$ 容易证明两种定义是等效的。 高中课本通常使用恒力和恒定加速度来进行推导这一关系:
\begin{equation}
p = mv = m(at) = (ma)t = Ft~.
\end{equation}
然而若力随时间变化, 如何定义力在时间上的累加呢? 当力随时间变化时, 我们可以把时间划分成许多非常小的 $\Delta t$, 在每一个 $\Delta t$ 中, 把力近似为恒力, 所以在第 $i$ 段时间中动量的变化 $\Delta p_i = m\Delta v_i = F_i\Delta t$, 然后把这些时间段的增量相加即可证明, 总动量就是对所有 $F_i\Delta t$ 求和, 记为 $\sum_i F_i \Delta t$。 当 $\Delta t$ 越小, 这种近似就越精确, 最后逼近 $p = mv$ 的数值。 事实上,这种细分的方法就是微积分课程中\textbf{定积分}\upref{DefInt}的基本思想, 表示为
\begin{equation}
p = \int_a^b F(t)\dd{t}
\end{equation}

\subsection{动量守恒}
那么,为什么动量作为力在时间上的累加, 就非常具有研究价值呢? 原因是这样定义直接导致它成为一个守恒量——\textbf{封闭系统的总动量守恒}。 在这里, 封闭系统具体指的是系统内的物体不受系统外的力。 我们考虑两个做直线运动的木块组成的系统, 规定速度 $v$ 和动量 $p = mv$ 向右为正, 向左为负。 若两个木块的动量分别为 $p_1 = m_1 v_1$, $p_2 = m_2 v_2$, 那么系统的总动量就是 $p = p_1 + p_2$。

如果这两个木块既不受系统外的力, 互相之间也没有作用力, 那么根据牛顿第一定律, 它们都做匀速直线运动, 总动量 $p$ 显然不随时间变化, 也就是说动量守恒。 但如果在运动过程中, 两个木块之间产生了一对相互作用力。 根据牛顿第三定律, 无论这两个力随时间如何变化, 它们在同一时刻一定是等大反向的。 所以一个木块受到的力在一段时间上产生了多少累加, 另一个木块在相同一段时间内必定会得到相反的累加。 换言之, 同一段时间内一个木块的动量增加多少, 另一个木块的动量就会减少多少。 这样, 系统的总动量就守恒了——只要不受来自系统外的力。 如果有多个物体, 该分析也是一样的, 它们两两之间都满足牛顿第三定律, 同样有总动量守恒。

另外注意相互作用力可能使封闭系统的总机械能发生改变。 例如直线运动的两个木块发生完全非弹性碰撞后, 总机械能减小。 此时机械能不守恒但动量却依然守恒。 可见动量守恒的适用情况比机械能守恒要多。

\subsection{动能}
相比动量, 动能可以看作是力在空间上的累加。 同样是箱子的例子, 如果想让与地面无摩擦的静止的箱子达到动能 $E_k$ (角标 k 表示 kinetic energy), 就需要用恒力 $F$ 推箱子, 在力的方向移动距离 $s$,使 $E_k = Fs$。 同样可以根据牛顿第二定律证明 $E_k = mv^2/2$。 在恒力的作用下:
\begin{equation}
E_k = Fs = mas
\end{equation}
使用匀加速公式 $2as = v^2 - v_0^2$ ($v_0, v$ 分别表示在移动位移 $s$ 前后的速度), 假设初速度 $v_0 = 0$, 代入上式, 就得到了 $E_k = mv^2/2$。 使用定积分的思想, 也可以证明在力随时间变化的情况下, 同样有 $E_k = mv^2/2$。

\subsubsection{为什么需要动量和动能?}
理论上, 有了牛顿定律我们就可以求出所有物体的运动情况, 那为什么还需要动量和动能? 为什么要研究力在时间上和空间上的累加? 因为动量和能量在封闭系统中是守恒的, 可以只根据初末状态就得到一些结论而不需要知道具体过程。 例如两个箱子完全非弹性碰撞, 我不需要知道碰撞用了多久, 缓冲距离是多少, 材料的性质等等就可以根据碰撞前的状态求出碰撞后的状态。 又例如物体从静止开始延光滑轨道滚落一定高度, 不需要知道过程就可以知道末速度。 由此可见守恒量在物理中具有十分重要的意义。

\subsection{势能曲线}
在更高级的物理理论中我们往往不讨论力,而是势能。 一维直线运动中, 如果受力只是关于位置的函数(如简谐振子),那么这个力(也可以称为\textbf{力场})就叫\textbf{保守力}, 保守力在高维的情况下有更复杂的定义\upref{V}先不讨论。 之所以叫做保守力, 是因为其对质点做功只与质点的初末位置有关而与运动过程无关。 对于保守力,我们可以计算出每个位置的\textbf{势能}, 也叫\textbf{势能函数}或\textbf{势能曲线}。 一维直线运动情况下,坐标为 $x$,势能函数可以记为 $V(x)$。 某个位置力的大小 $F(x)$ 就是 $V(x)$ 曲线的斜率(曲线与水平方向夹角的 $\tan$ 值),方向就是曲线下降的方向。 粒子沿受力方向运动,动能增加,势能减小,\textbf{机械能}(动能加势能)不变。

\begin{figure}[ht]
\centering
\includegraphics[width=8cm]{./figures/c8aa23eff7c6e60d.pdf}
\caption{轨道小车模型, 小车只能在其所在的 $V(x) \leqslant E$ 区间内往返运动, 每个位置的速度大小和动能都是确定的。} \label{fig_CM1_1}
\end{figure}

作为一个形象但\textbf{不准确}的比喻,想象高低不平的光滑轨道上的小车, 我们可以让轨道的高度为 $h(x)$, 小车在 $x$ 处时重力势能为 $V(x) = mgh(x)$,这样具有一定总机械能(动能加势能) $E$ 的小车就会在轨道上运动。 机械能是守恒的, 所以小车高度越高, 势能越大, 动能就越小。 动能为零时势能最大, 小车达到运动的最高点, 满足 $V(x) = E$, 然后开始返回。 如果小车开始时在某 $E \leq V(x)$ 的区间运动, 那么它将一直在该区间往返运动, 某点 $x$ 的速度大小为\footnote{推导: $\frac{1}{2}mv(x)^2 = E - V(x)$, 两边乘以 $2/m$ 再开根号。}
\begin{equation}\label{eq_CM1_1}
v(x) = \sqrt{\frac{2}{m}[E - V(x)]}
\end{equation}
注意这个速度是沿轨道方向的速度而不是 $x$ 方向的速度, 这就是为什么说这个模型 “不准确”。 在真正的直线运动中该式同样成立, 请看\autoref{ex_CM1_1}。

根据势能曲线, 我们可以给一些特殊的点分类。 当势能曲线的切线为水平, 即质点受力为零时, 这样的点叫做\textbf{平衡点}。 但平衡点也分为不同情况: 如果平衡点处的势能曲线是凸的, 即处于 “山峰”, 那么它是\textbf{非稳定}的平衡点; 当平衡点处势能曲线是凹的, 即处于 “山谷”, 那么它是\textbf{稳定}的平衡点。 请读者找出\autoref{fig_CM1_1} 中的稳定和非稳定平衡点。 事实上还有一种平衡点就是一个方向势能变大而另一个方向势能变小, 它同样是不稳定的。

% 如何从图中看出粒子在各个可能位置的运动情况?先画代表总能量的横线,横线与势能曲线的交点就是拐弯的地方,横线的高度减掉某点势能曲线的高度就是动能 $mv^2/2$。

\begin{example}{简谐振子}\label{ex_CM1_1}
\begin{figure}[ht]
\centering
\includegraphics[width=5.5cm]{./figures/8139cc6b621ddcd6.pdf}
\caption{简谐振子的势能曲线} \label{fig_CM1_2}
\end{figure}

作为势能曲线的一个经典例子, 我们来看简谐振子\upref{SHO}。 若一个质点通过质量不计的理想弹簧固定在 $x$ 轴的某点, 把质点的平衡位置作为原点, 那么势能曲线为 $V(x) = kx^2$(\autoref{fig_CM1_1} )。 若质点的总能量为 $E$, 那么我们根据曲线可以马上确定质点的运动范围, 即简谐运动的振幅 $A$。 当质点经过原点时, 势能为零, 动能取最大值 $E$; 当质点达到 $x = \pm A$ 时, 动能为零, 势能取最大值 $E$。
\end{example}
相比于上面小车轨道的例子, 这个例子是准确严谨的。 \autoref{fig_CM1_2} 中的曲线并不是质点的运动轨迹, 而是势能关于 $x$ 的函数。 质点始终做直线运动。

\subsubsection{势阱和势垒}
\addTODO{未完成}
在量子力学中, 有几个特殊的势能曲线 $V(x)$ 较为常见, 为了给量子力学的科普做铺垫, 我们先看看他们的经典力学版本。

\addTODO{有限深/无限深势阱/势垒, 散射}

% 未完成: 我们可以用穿孔的无限大平行板电容器以及带电粒子来解释势垒, 如果电势差为定值, 而平行板无限靠近, 我们就得到了不连续的势能曲线。 如果保持一定距离, 而电场变得无穷大, 我们就得到了无限深势垒的一侧!

仍然考虑一维运动, 假设光滑水平面上有两面墙, 粒子来两面墙之间来回反弹, 我们该用什么样的势能函数呢? 如果墙是软的, 例如把两面墙比作弹簧, 得到的势能将如(未完成)

中的公式不会出现力, 只会出现势能。 常见的势能(三角势垒见下文,方势垒,方势阱,边界处斜率无穷大怎么办? 类比碰撞时的冲力。 初始动能(总能量), 大于, 小于最大势能的时候分别会沿原方向运动, 反弹。 更理想的情况:无限深势阱(小球在两面墙之间无限反弹),delta 势垒/势阱(受到一个微小扰动,相当于无限窄的方势垒/势阱),对经典粒子运动没有任何影响(物理上不存在,但是作为模型有计算简单的优点)。
