% 线性相关 线性无关
% 矢量空间|线性相关|线性无关|线性代数

\pentry{矢量空间\upref{LSpace}, 高斯消元法\upref{GAUSS}}

如果这些矢量是\textbf{线性相关的}, 就代表存在一组常数 $c_i$ 使它们的线性组合为 0. 或者说它们中至少有一个矢量可以表示为其他矢量的线性组合. 但如果我们依次将所有可以表示为其他矢量的线性组合的矢量排除, 就仍然可以得到  $R$ 个线性无关的矢量($1 \leqslant R < N$).

\begin{exercise}{}
证明 $(1,0)$, $(1,1)$ 和 $(0,1)$ 这三个矢量中最多有两个矢量线性无关($R = 2$).
\end{exercise}

\subsection{高斯消元法求线性相关性}
要求\autoref{LinDep_eq1} 中线性无关矢量的个数 $R$, 一种简单的方法就是把这些矢量都转置成行矢量组并成一个矩阵 $\mat A$, 使 $A_{i,j} = v_{i,j}$. 然后对矩阵用高斯消元法, 若得到的梯形矩阵有 $R$ 个不全为零的行, 那么这 $M$ 个矢量中就有且仅有 $R$ 个是线性无关的.

% 未完成

% 我们来证明高斯消元法中的三种行变换不会改变 $R$.

% 首先显然有 $R \le N$, 因为 $N$ 维空间中不可能找到多于 $N$ 个线性无关的矢量.

% 如果 $R = N$,

% 高斯消元法不光可以用来解线性方程组, 还可以用于判断几个矢量是否线性无关.
 