% 华东师范大学 2015 年硕士研究生物理考试试题
% keys 华东师范大学|物理|考研|2015年
% license Copy
% type Tutor

\textbf{声明}:“该内容来源于网络公开资料,不保证真实性,如有侵权请联系管理员”


普适气体常量  $R=8.31 J/(mol.k)$ ,玻尔兹曼常量 $k=1.38*10^{-23}$  , 电子质量  $m_4=9.11*10^{-31}$, 真空介电常量 $\varepsilon_0=8.85*10^{-12}C^2.N^{-1}$ , 普朗克常量 $h=6.63*10^{-34}$
\begin{enumerate}
\item 如图所示,一圆盘绕通过其中心且垂直于盘面的转轴,以角速度 $\omega$ 作定轴转动,$A,B,C$  三点与中心的距离均为$r$。则图示 $A,B$  点速度差 $\bar v_{AB}=\bar v_A-\bar v_B$ 与 $A,C$ 点的速度差 $\bar v_{AC}=\bar v_A-\bar v_C$  的关系为$(\qquad)$\\
(A)$\bar v_{AB}>\bar v_{AC}$\\
(B)$\bar v_{AB}<\bar v_{AC}$\\
(C)$\bar v_{AB}=\bar v_{AC}$\\
(D)$\abs{\bar v_{AB}}=\abs{\bar v_{AC}}$
\begin{figure}[ht]
\centering
\includegraphics[width=6cm]{./figures/22a12d6c47f71e14.png}
\caption{} \label{fig_HSD15_1}
\end{figure}
\item 题1中,如果圆盘的质量为$M$,在$A$点处有一个质量为$m$的人。刚开始人和圆盘相对于地面均处于静止状态。当人沿着圆盘走一圈时,圆盘相对于地面转过的角度为$(\qquad)$\\
(A)$\displaystyle \frac{4\pi m}{2m+M}$\\
(B)$\displaystyle \frac{2\pi m}{2m+M}$\\
(C)$\displaystyle \frac{2\pi m}{m+M}$\\
(D)$0$
\item 令电子的速率为$v$,则电子的动能 $E_x$ 对于比值v/c的图线可用下列图中哪一个图表示?(c表示真空中光速)\\
\begin{figure}[ht]
\centering
\includegraphics[width=12cm]{./figures/327338a7924728b9.png}
\caption{} \label{fig_HSD15_2}
\end{figure}
\item 一质点沿螺旋线状的曲线自外向内运动,如图所示、已知其走过的弧长与时间的一次方成正比。则该运动为$(\qquad)$\\
(A)加速度值越来越小的匀速率曲线运动\\
(B)加速度值越来越大的匀速率曲线运动\\
(C)加速度值越来越小的变速率曲线运动\\
(D)加速度值越来越大的变速率曲线运动
\item 质量分别为 $m_A$ 和$m_B (m_A>m_B)$、速度分别为和的两质点$A$和$B$,受到相同的冲量作用,则$(\qquad)$\\
(A)$A$的动量增量的绝对值比$B$ 的小\\
(B)$A$的动量增量的绝对值比 $B$的大\\
(C)$A,B$的动盘增量相等\\
(D)$A,B$的速度增量相等
\item 老师和学生各带一个钟。他们在长沙站对好钟(显示是同一时刻),学生就坐上火车往武汉去。如果考虑到相对论效应,当学生到达武汉后,他的钟与老师的钟相比,哪一个会变慢?设同一参考系的钟都是同步的。
\item 如图所示,一圆盘形工件K套装在一根可绕自身轴线转动的固定轴$A$上,圆盘K的中心线与固定轴A的中轴线互相重合,圆盘的内外直径分别为$D$和$D_1$,该工件在外力矩作用下获得角速度$\omega_0$这时撤掉外力矩,工件在轴所受的阻力矩作用下最后停止转动,其间经过了时间t,则轴所受的ping'jun
这里圆盘其问经过了时间,则抽所受的平均阻力为工件绕其中心轴转动的转动损量为。(0+D)/8,m为圆盘的质最,轴的转动惯盘忽略不计。
(A)
(B)
(C)
(D)
一质点在二恒力共同作用下,位移为AF=37+87(ST):在此过程中,动能增量为24J,已知其中一恒力月=127-37(SI),则另一恒力所作的功为
\end{enumerate}
