% SymPy 符号计算笔记

\begin{issues}
\issueDraft
\end{issues}

\pentry{Python 符号计算简介\upref{SymPy}}

为了书写方便, 本文使用 \verb|from sympy import *|. 所有的变量都假设用 \verb|sympy.symbols()| 声明过.

\subsection{基础}
\begin{itemize}
\item \verb|import sympy as sm|
\item \verb|x0, x1 = symbols('x0, x1')| 声明变量, 类型为 \verb|sympy.core.symbol.Symbol|
\item \verb|x, y, z = symbols('x:z')|, \verb|x4, x5, x6, x7 = symbols('x4:8')|
\item 常数如 \verb|pi, E, I, oo| (无穷)
\item 整数为 \verb|numer(n)|, 类型为 \verb|sympy.core.numbers.Integer|, 也可能是 \verb|sympy.core.numbers.One|
\item 有理数 \verb|Rational(n,m)|. 如果 \verb|n, m| 已经是 \verb|numbers.Integer| 类型, 也可以直接 \verb|n/m|
\item 虚数单位 \verb|I| 的类型是 \verb|sympy.core.numbers.ImaginaryUnit|, 其他虚数和复数都没有专门的类型而是 \verb|I| 和其他实数相乘相加.
\item 函数如 \verb|sin(), asin(), sinh(), exp(), log(), sqrt()|
\end{itemize}

\subsection{基本运算}
\begin{itemize}
\item \verb|summation(含i的表达式, (i, 1, 5))|, 例如 \verb|summation(1/2**n, (n, 1, oo))| 得 \verb|1|.
\end{itemize}

\subsection{线性代数}
\begin{itemize}
\item 矩阵 \verb|mat = Matrix([[1, 2], [2, 2]])|, 类型为 \verb|sympy.matrices.dense.MutableDenseMatrix|
\item \verb|mat.eigenvals()| 求本征值
\end{itemize}


\subsection{微积分}
\begin{itemize}
\item 极限如 \verb|limit(sin(x)/x, x, 0)|
\item 积分如 \verb|integrate(exp(x)*(sin(x) + cos(x)), x)|
\item 格式为 \verb|integrate(表达式, (x, 下限, 上限))|
\item 微分方程 \verb|dsolve(Eq(y(t).diff(t, t) - y(t), exp(t)), y(t))|
\end{itemize}


\subsection{检查表达式结构}
\begin{itemize}
\item 虽然表达式的类型有很多, 但都有 \verb|args| 参数.
\item 例如令 \verb|expr = sin(x)**2 + cos(x)**2|, 那么 \verb|expr| 的类型是 \verb|sympy.core.add.Add|, \verb|expr.args| 是 \verb|(cos(x)**2, sin(x)**2)|. \verb|expr.args[0]| 的类型是 \verb|sympy.core.power.Pow|, \verb|expr.args[0].args| 是 \verb|(cos(x), 2)|, 以此类推. 这样就可以生成一个树状结构.
\end{itemize}


\subsection{任意精度求值}
\begin{itemize}
\item 通过 \href{https://mpmath.org/}{mpmath} 库完成任意精度计算(和 arb\upref{ArbLib} 是同一个作者, 但是完全用 python 编写)
\item \verb|N(表达式, 有效数字)| 对表达式求值, 例如 \verb|N(pi, 50)|, \verb|N(sin(numer(1)), 50)|
\end{itemize}

