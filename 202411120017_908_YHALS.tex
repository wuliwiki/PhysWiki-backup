% 约翰·沃利斯(综述)
% license CCBYSA3
% type Wiki

本文根据 CC-BY-SA 协议转载翻译自维基百科\href{https://en.wikipedia.org/wiki/John_Wallis}{相关文章}。

\begin{figure}[ht]
\centering
\includegraphics[width=6cm]{./figures/7dd5979fcc742ec6.png}
\caption{约翰·沃利斯} \label{fig_YHALS_1}
\end{figure}
约翰·沃利斯(/ˈwɒlɪs/;[2]拉丁语:Wallisius;1616年12月3日 [旧历11月23日] – 1703年11月8日 [旧历10月28日])是英国的一位牧师和数学家,部分功劳归于他对微积分发展的贡献。

1643年至1689年间,沃利斯担任议会及后来王室的首席密码学家。[3] 他被认为引入了符号 ∞ 来表示无穷的概念。[4] 他同样使用 1/∞ 表示无穷小。他是牛顿的同时代人,也是数学早期复兴时期最伟大的知识分子之一。[5]
\subsection{传记}
\subsubsection{教育背景}  
\begin{itemize}
\item 剑桥大学,硕士学位;牛津大学,神学博士学位。  
\item 1625–1631年,就读于肯特郡滕特登的文法学校。  
\item 1631–1632年,就读于埃塞克斯郡费尔斯特的马丁·霍尔比奇学校。  
\item 1632–1640年,剑桥大学伊曼纽尔学院;1637年获学士学位,1640年获硕士学位。  
\item 1654年获牛津大学神学博士学位。
\end{itemize}
\subsubsection{家庭}
1645年3月14日,他与苏珊娜·格莱德(约1600年 – 1687年3月16日)结婚。他们有三个孩子:
\begin{itemize}
\item 安妮,布伦科夫夫人(1656年6月4日 – 1718年4月5日),1675年嫁给约翰·布伦科爵士(1642年11月30日 – 1726年5月6日),并有后代。[6]
\item 约翰·沃利斯(1650年12月26日 – 1717年3月14日),1690年至1695年担任沃灵福德的国会议员,1682年2月1日娶伊丽莎白·哈里斯(卒于1693年),育有一子两女。[7]
\item 伊丽莎白·沃利斯(1658年 – 1703年),嫁给陶斯特的威廉·本森(1649年 – 1691年),无子嗣。[8]
\end{itemize}
\subsection{生活}
约翰·沃利斯出生于肯特郡的阿什福德。他是约翰·沃利斯牧师和乔安娜·查普曼的五个孩子中的第三个。起初他在阿什福德的一所学校接受教育,但由于瘟疫爆发,他于1625年转入滕特登的詹姆斯·莫瓦特学校。1631年,沃利斯在费尔斯特学校(当时称为费尔斯特的马丁·霍尔比奇学校)首次接触数学;他对数学很感兴趣,但学习却不太系统,因为“在当时,数学在我们这儿几乎不被视为学术研究,而是机械学科”(Scriba 1970)。在费尔斯特学校,沃利斯学会了说和写拉丁语。到那时,他还熟练掌握了法语、希腊语和希伯来语。[9] 由于家人希望他成为一名医生,他于1632年被送到剑桥大学伊曼纽尔学院。[10] 在那里,他进行了一次关于血液循环理论的辩论,据说这是欧洲首次在辩论中公开维护这一理论。然而,他的兴趣集中在数学上。1637年,他获得了文学学士学位,1640年获得硕士学位,随后进入牧师行列。1643年至1649年间,他担任威斯敏斯特大会的无表决权书记。1644年,他当选为剑桥大学皇后学院的研究员,但在结婚后不得不辞去这一职位。

在此期间,沃利斯一直与议会党关系密切,这可能是由于他在费尔斯特学校受到了霍尔比奇的影响。他在破解保皇党密电方面为他们提供了极大的实际帮助。那时的密码学质量参差不齐;尽管像弗朗索瓦·维埃特这样的数学家取得了个别成功,但加密设计和分析的基本原理仍然很少被理解。大多数密码是基于秘密算法的临时方法,而非基于可变密钥的系统。沃利斯意识到后者要安全得多,甚至将其描述为“不可破解”,尽管他对这一论断的信心不足,以至于不敢鼓励公开密码算法。此外,他还担心外国势力使用密码,例如,1697年,他拒绝了戈特弗里德·莱布尼茨关于教授汉诺威学生密码学的请求。[11]

回到伦敦后——1643年他成为了圣加布里埃尔芬丘奇的牧师——沃利斯加入了一个科学家小组,这个小组后来演变为皇家学会。他终于能够满足自己对数学的兴趣,1647年在几周内掌握了威廉·奥特雷德的《数学钥匙》。他很快开始撰写自己的论文,涉及广泛的主题,并持续了他的一生。沃利斯撰写了关于英国数学概念的第一份综述,其中讨论了印地-阿拉伯数字系统。[12]

沃利斯加入了温和派的长老会,签署了反对处决查理一世的抗议书,因此遭到了独立派的长期敌视。尽管他们反对,1649年他仍被任命为牛津大学萨维尔几何学教授,并一直居住在牛津,直至1703年11月8日(旧历10月28日)去世。1650年,沃利斯被授予牧师身份。之后,他作为私人牧师与理查德·达利爵士和维尔夫人共度了两年。1661年,他作为十二位长老会代表之一参加了萨沃伊会议。[citation needed]

除了他的数学著作外,他还撰写了神学、逻辑、英语语法和哲学方面的作品,并参与设计了一种在利特尔科特庄园教授聋哑男孩说话的方法。[13] 威廉·霍尔德早些时候曾教授一名聋人亚历山大·波普汉姆说话,使其“发音清晰、优雅且语调得当”。[14] 后来,沃利斯声称对此功劳归于自己,导致霍尔德指责他“掠夺邻人,以他人之成果装饰自己”。[15]
\subsubsection{沃利斯被任命为牛津大学萨维尔几何学教授}
议会对牛津的访问始于1647年,这次访问免去了许多高级学者的职务,包括1648年11月被免职的萨维尔几何学和天文学教授。1649年,沃利斯被任命为萨维尔几何学教授。沃利斯的选择似乎主要基于政治原因(他的保皇派前任彼得·特纳也可能是如此,尽管他被任命为两个教授职位,却从未发表任何数学作品);虽然沃利斯或许是当时全国领先的密码学家,且是后来成为皇家学会的非正式科学家小组的成员,但他并没有作为数学家的特别名声。尽管如此,沃利斯的任命在他随后担任萨维尔教授的54年里,通过他的工作得到了充分的证明。[16]
\subsection{对数学的贡献}
\begin{figure}[ht]
\centering
\includegraphics[width=6cm]{./figures/c57fa52257530796.png}
\caption{《数学作品集》,1699} \label{fig_YHALS_2}
\end{figure}
沃利斯在三角学、微积分、几何学和无穷级数分析方面作出了重要贡献。在他的《数学作品集I》(1695年)中,他引入了“连分数”一词。
\subsubsection{解析几何}
1655年,沃利斯发表了一篇关于圆锥曲线的论文,在其中对这些曲线进行了解析定义。这是第一本将这些曲线视为并定义为二次曲线的书籍,有助于消除人们对笛卡尔解析几何工作中某些难以理解的内容的困惑。

在《圆锥曲线论》中,沃利斯普及了用于表示无穷的符号 \(\infty\)。他写道:“我假设任何平面(遵循卡瓦列里的不可分量几何学)是由无数条平行线组成的,或者更确切地说,是由无数个相同高度的平行四边形组成的;(令这些平行四边形的每一个高度为整个高度的一个无穷小部分 \(1/\infty\),并让符号 \(\infty\) 表示无穷),而所有这些高度的和构成图形的总高度。”[17]