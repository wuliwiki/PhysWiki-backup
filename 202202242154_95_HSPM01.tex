% 匀变速直线运动
% 高中物理|位移|速度|加速度

\addTODO{常见问题介绍,图像分析示例}

\subsection{基本概念}
\subsubsection{质点}
某些情况下,物体的大小和形状对研究的问题没有影响或影响可以忽略,而只需突出“物体具有质量”这个要素,我们可以把这个物体简化为一个具有质量的物质点,这样的点称为\textbf{质点}.

\subsubsection{参考系}
判断一个物体是运动还是静止,总要选取一个物体作为标准,被选作标准的物体叫\textbf{参照物}.
为了描述一个物体在空间中位置随时间的变化,我们要在参照物上建立一套\textbf{坐标系},并在同一坐标系的各处都配置同步的时钟,这便组成了一个\textbf{参考系}.

一般地,我们不严格区分“参照物”和“参考系”,而是强调\textbf{坐标系是参考系的数学抽象}.

\subsubsection{时间和时刻}
\textbf{时刻}:某一瞬时,是时间轴上的一点.常见描述为“第$n$秒初(末)”.

\textbf{时间间隔}:两个时刻之间的间隔,是时间轴上的一段.

设$t_1$和$t_2$分别为先后的两个时刻,$\Delta t$表示这两个时刻之间的时间间隔,则$\Delta t = t_2 - t_1$.

\subsubsection{路程,位置,位移}
\textbf{路程}:质点运动轨迹的长度.存在局限性,不能反应运动的某些本质,描述不够精确.常用符号为$s$,是一个\textbf{标量}.

\textbf{位置}:质点相对于参考点(常为坐标系原点)的距离和方向,在坐标系中是一个点.

\textbf{位移}:初位置指向末位置的有向线段,是描述质点位置变化的物理量,是一个\textbf{矢量}.常用符号为$x$,单位为$m$.

\subsubsection{速度}
定义:位移与发生这段位移所用时间之比,表示物体运动的快慢.表达式为
\begin{equation}\label{HSPM01_eq3}
v=\frac{\Delta x}{\Delta t}
\end{equation}

其中,$v$是速度,$\Delta x$是$\Delta t$时间内的位移.速度采用比值定义法,不能说$v$与$\Delta x$成正比.如果$t$时间内物体发生的位移为$x$,则公式可表示为$v=x/t$.

单位:米每秒,符号是$m/s$,常用的单位还有$km/h$,$1m/s=3.6km/h$.

方向:速度是矢量,其方向与物体的运动方向相同.对于直线运动来说,如果我们选定某一个方向为正方向,则速度方向就可以用正、负号来表示.

\textbf{平均速度}:物体的位移与发生位移所用时间之比,描述物体位置变化的快慢,其方向与$\Delta x$一致.

\textbf{瞬时速度}:物体在某一时刻或某一位置的速度,描述物体在某一时刻或经过某一位置时运动的快慢,其方向与该物体在这个时刻或经过这个位置时的运动方向一致.瞬时速度可以用极短时间$\Delta t$内的平均速度来计算.另外,在匀速直线运动中,瞬时速度始终和平均速度相同.

\textbf{平均速率}:路程与时间之比,是一个标量.一般情况下,平均速度只表示位置变化的平均快慢,而平均速率才能表示通常意义的物体运动的平均快慢.一般情况下,平均速度的大小不等于平均速率的大小,只有在单方向直线运动中才相等,但也不能描述为平均速率就是平均速度.

\subsubsection{加速度}
定义:速度的变化量$\Delta v$与发生这一变化所用时间$\Delta t$之比.
公式:
\begin{equation}
a=\frac{\Delta v}{\Delta t}
\end{equation}

物理意义:描述物体运动速度变化的快慢.

单位:在国际单位制中,加速度的单位是$m/s^2$,读作米每平方秒.

方向:加速度的方向总是与速度变化量的方向相同,与速度方向无关.

\subsection{匀变速直线运动}
速度均匀变化的直线运动,即加速度不变的直线运动.
\subsubsection{速度—时间公式}
\begin{equation}\label{HSPM01_eq1}
v=v_0+at
\end{equation}

$v_0$为初速度,$a$为加速度,$t$为时间,$v$为$t$时刻的瞬时速度.当初速度$v_0=0$时,有$v=at$.

物理意义:末速度等于初速度和速度变化量的矢量和.

若$a$与$v_0$同向,则物体做匀加速直线运动,$v$逐渐增大.若$a$与$v_0$同向,则物体做匀减速直线运动,$v$逐渐减小.

\subsubsection{位移—时间公式}
\begin{equation}\label{HSPM01_eq2}
x=v_0 t+\frac12at^2
\end{equation}

$v_0$为初速度,$a$为加速度,$t$为时间,$x$为$t$时刻的位移.

\subsubsection{速度—位移公式}
联立\autoref{HSPM01_eq1} 和\autoref{HSPM01_eq2} 消去时间$t$可得
\begin{equation}\label{HSPM01_eq4}
v^2-v_0^2=2ax
\end{equation}

该式由匀变速直线运动的两个基本公式推导出来,便于解决不含时间的问题.

\subsubsection{二级结论}
(1)做匀变速直线运动的物体某段时间里的平均速度等于这段时间内中间时刻的瞬时速度,且等于初、末速度矢量和的一半,由\autoref{HSPM01_eq3} \autoref{HSPM01_eq2} 可证:
\begin{equation}
\bar v=\frac xt=\frac{v_0t+\frac 12at^2}{t}=v_0+a\frac t2=v_{\frac t2}
\end{equation}

联立\autoref{HSPM01_eq1} 消去$a$可得:
\begin{equation}
\bar v=\frac{v_0+v}{2}
\end{equation}

(2)某段位移内中间位置的瞬时速度与这段位移的初速度$v_0$和末速度$v$之间满足关系式:
\begin{equation}
v_\frac x2=\sqrt \frac{v_0^2+v^2}2
\end{equation}

由\autoref{HSPM01_eq4} 可知:
\begin{equation}
v_{\frac x2}^2-v_0^2=2a\frac x2
\end{equation}

联立\autoref{HSPM01_eq4} 消去$ax$即可得证.

(3)在连续相等的时间间隔$T$里的位移之差$\Delta x$为恒定值.

在第一个时间$T$内位移为
\begin{equation}
x_1=v_0T+\frac12aT^2
\end{equation}

在时间$2T$内位移为
\begin{equation}
x_{2T}=v_0\cdot2T+\frac12a(2T)^2
\end{equation}

在第二个时间$T$内位移为
\begin{equation}
x_2=x_{2T}-x_1=v_0T+\frac32aT^2
\end{equation}

在连续相等的时间间隔$T$里的位移之差为
\begin{equation}
\Delta x=x_2-x_1=aT^2
\end{equation}

\subsection{常见问题}
\subsubsection{相遇和追及问题}
当两个物体在同一直线上运动时,由于各自的运动情况不一,两物体之间的距离会不断变化,在研究这两个物体的运动时,就涉及到相遇和追及的问题.相遇和追及问题实质上就是研究两个物体是否会在同一时刻到达同一位置的问题.研究方法有临界条件法、判断法和图像法等.

\subsubsection{自由落体运动}
条件:物体在只受重力的情况下(理想化模型),从静止即$v_0=0$开始下落.忽略下落高度变化导致的重力变化.

实质:初速度$v_0=0$,加速度$a=g$的匀加速直线运动.

自由落体的加速度:也就是当地的重力加速度.重力加速度的大小随纬度的增大而增大.由于忽略了下落高度变化导致的重力变化,我们可以认为在同一地区的重力加速度是恒定的,通常默认取$g=9.8m/s^2$,有时为了计算方便会注明取$g=10m/s^2$.

自由落体运动是匀变速直线运动的特例,根据 
\subsubsection{竖直上抛运动}