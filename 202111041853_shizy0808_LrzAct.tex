% LorenzAttractor
\subsection{洛伦兹方程}

20世纪60年代,蓬勃发展的计算机技术开始得到广泛应用,其中包括长期天气预报. 大气与液体同属流体,太阳照射使地面升温, 靠近地面的气体受到加热, 而高层大气还是冷的, 于是上、下层气体之间将会出现对流, 产生类似于贝纳德实验中的对流现象. 在美国气象局工作的数学家洛伦兹(E.N. Lorenz)将大气对流与贝纳德液体对流联系起来,企图用数值方法进行长期天气预报. 从贝纳德对流出发, 利用流体力学中的纳维叶-斯托克斯(Navier-Stokes)方程、热传导方程和连续性方程,洛伦兹推导出了描述大气对流的微分方程
\begin{align}
&\frac{dx}{dt}=-\sigma (x-y),\\
&\frac{dy}{dt}=rx-y-xz,\\
&\frac{dz}{dt}=-bz+xy,
\end{align}
式子中,$r$是对流的翻动速率; $y$ 正比于上流与下流液体之间的温差;$z$是垂直方向的温度梯度; $\sigma$为无量纲因子,称为Prandtl数; $b$ 为反映速度阻尼的常数. 其中,$xz$与$xy$是非线性项.

洛伦兹方程是一个能量耗散系统,这可以从它的相空间随时间变化的特性来证明.设在$(x,y, z)$三维相空间取一个闭合曲面,该曲面所包围的体积$V$随时间的变化为:
\begin{align}
\frac{dV}{dt}=\int_V \Big(\frac{d\dot{x}}{dt}+\frac{d\dot{y}}{dt}+\frac{d\dot{z}}{dt}\Big)
\end{align}
式中,$d\dot{x}$, $d\dot{y}$, $d\dot{z}$ 为代表点在相空间的相应方向上的运动速度.由洛伦兹方程得
\begin{align}
&\frac{d\dot{x}}{dt}=-\sigma,\\
&\frac{d\dot{y}}{dt}=-1,\\
&\frac{d\dot{z}}{dt}=-b,
\end{align}
于是,就有
\begin{align}
\frac{dV}{dt}=-(\sigma+1+b)V.
\end{align}
解此方程得到
\begin{align}
V(t)=V_0\exp[-(\sigma+1+b)t],
\end{align}
式中,$V_0$为初始相空间的体积. 由于参数$b>0,\sigma>0$,可见洛伦兹方程的相空间体积$V(t)$是随时间收缩的,由初始时的有限相体积$V_0$随时间收缩到一点,这一点应是坐标系的原点$x=y=z=0$.由此可见,洛伦兹方程描述的是一个耗散系统.正如阻尼单摆那样,耗散意味着系统存在吸引子.

\subsection{洛伦兹方程解的分岔}

