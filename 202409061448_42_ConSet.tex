% 凸集和凸体
% keys 凸集|凸体
% license Usr
% type Tutor
\pentry{向量空间\nref{nod_LSpace}}{nod_573c}
凸性是向量空间理论中许多重要部分的基础概念。它不仅是直观的几何概念,也允许纯粹解析的叙述。

\subsection{凸集的引入}
设 $L$ 是一实向量空间,$x_1,x_2$ 是它的两点。那么过 $x_1,x_2$ 的直线方向与矢量 $x_2-x_1$ 平行,因此该直线可表示为
\begin{equation}\label{eq_ConSet_1}
x_1+k(x_2-x_1).~
\end{equation}
或写为
\begin{equation}\label{eq_ConSet_2}
(1-k)x_1+kx_2.~
\end{equation}
明显的,当 $k\geq0$ 时,矢量 $k(x_2-x_1)$ 的模(长度)随 $k$ 的增大而增大。即 $\{x_1+k(x_2-x_1)|k\geq0\}$ 是以 $x_1$ 为原点的正方向(由 $x_1$ 指向 $x_2$ 的方向)的直线部分;反之,$\{x_1+k(x_2-x_1)|k<0\}$ 是以 $x_1$ 为原点的负方向部分。明显的,$0\leq k\leq 1$ 时是直线 $x_1+k(x_2-x_1)$ 的 $x_1$ 到 $x_2$ 之间的部分。

注意\autoref{eq_ConSet_1} 等价于\autoref{eq_ConSet_2} ,因此可得下面的定义。
\begin{definition}{线段}
设 $L$ 是实向量空间,$x_1,x_2\in L$,则称
\begin{equation}
\alpha x_1+\beta x_2, \quad\alpha,\beta\geq0,\alpha+\beta=1~
\end{equation}
的所有元素的全体为连接点 $x_1$ 与 $x_2$ 的\textbf{闭线段}(close segment)。而闭线段去掉端点 $x_1,x_2$ 后叫做\textbf{开线段}(open segment)。
\end{definition}

\begin{definition}{凸集}
设 $M\subset L$,若对 $M$ 上的任意两点 $x_1,x_2\in M$,$M$ 都包含连接它们的线段,则称 $M$ 是\textbf{凸的}(convex)。 
\end{definition}

\begin{definition}{核}
设 $E\subset L$ 是任意集,则称
\begin{equation}
\{x|x\in L,\text{且}\forall y\in L,\exists \epsilon(y)>0, \text{使得只要} \abs{t}<\epsilon,\text{就有} x+ty\in E\}~
\end{equation}
为 $E$ 的\textbf{核}(kernel),记作 $J(E)$。
\end{definition}
也就是说,集 $E$ 的核是那些某个邻域都在包含在 $E$ 中的点的全体。
























