% 狄拉克场
% keys 狄拉克|场论|费米子

\begin{definition}{相对论性不变}
如果$\phi$是一个场或者是多个场,$\mathcal D$是微分算符.那么,我们说\textbf{$\mathcal D \phi = 0$是相对论不变的},就是说如果$\phi(x)$满足这个方程,我们再\textbf{对参考系}进行转动或者boost这样的操作,换到别的参考系,则变换后的场,在新的参考系下,满足同样的方程.我们也可以考虑\textbf{物理上}对所有的粒子或者场进行转动或者boost这样的操作,这时候方程$\mathcal\phi = 0$仍然保持不变.这种物理上对场进行操作的办法叫做\textbf{主动}的办法.
\end{definition}
用拉式量写出的场论让洛仑兹不变的讨论变得非常容易.如果一个理论的运动方程是从洛仑兹标量的拉式量推导出来的,那么这个理论的运动方程一定是自动洛仑兹不变的.






