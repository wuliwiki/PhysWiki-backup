% 有限深方势阱束缚态程序(Matlab)

\pentry{有限深方势阱\upref{FSW}}

以下给出求解束缚态的 Matlab 代码. 算法用二分法解\autoref{FSW_fig2}~\upref{FSW} 和\autoref{FSW_eq4}~\upref{FSW} 即可, 这可以确保找到所有解.

\begin{figure}[ht]
\centering
\includegraphics[width=12cm]{./figures/FSWmat_1.png}
\caption{运行结果:第四激发态} \label{FSWmat_fig1}
\end{figure}

\begin{lstlisting}[language=matlab, caption=FSW\_bound.m]
% 有限深势阱的束缚态
% 区间 [-L/2,L/2], 内部势能 -V0
% ka 按小到大排序
% 波函数分别是偶函数、奇函数、偶函数…
function [k, ka] = FSW_bound(L,V0,m)
if V0 <= 0
    error('V0 > 0');
end
% 奇波函数
z0 = L*sqrt(2*m*V0)/2;
N = floor(z0/pi + 0.5); % 解的个数
k1 = zeros(1,N);
fun = @(z)-cot(z) - sqrt((z0/z)^2 - 1);
for i = 1:N-1
    k1(i) = (2/L)*fzero(fun, [(i-0.5)*pi, i*pi-eps]);
end
k1(N) = (2/L)*...
    fzero(fun, [(N-0.5)*pi, min(N*pi-eps, z0)]);
% 偶波函数
N = floor(z0/pi) + 1; % 解的个数
k2 = zeros(1,N);
fun = @(z)tan(z) - sqrt((z0/z)^2 - 1);
for i = 0:N-2
    k2(i+1) = (2/L)*fzero(fun, [i*pi+eps, (i+0.5)*pi-eps]);
end
k2(N) = (2/L)*...
    fzero(fun, [(N-1)*pi, min((N-0.5)*pi-eps, z0)]);
k = sort([k1, k2]);
ka = sqrt(2*m*V0 - k.^2);
end
\end{lstlisting}

画图
\begin{lstlisting}[language=matlab, caption=FSW\_bound\_demo.m]
% === 参数设置 ===
L = 6; V0 = 20; m = 1;
xmin = -1.5*L; xmax = 1.5*L; Nx = 1000;
% ================
[k, ka] = FSW_bound(L,V0,m);
N = numel(ka);
x = linspace(xmin, xmax, Nx);
mark = abs(x) <= L/2;
markL = x < -L/2; markR = x > L/2;
psi = zeros(1,Nx);
for n = 1:N
    if mod(n,2) == 0 % 奇函数
        D = sin(k(n)*L/2)/exp(-ka(n)*L/2);
        psi(mark) = sin(k(n)*x(mark));
        psi(markL) = -D*exp(ka(n)*x(markL));
    else % 偶函数
        D = cos(k(n)*L/2)/exp(-ka(n)*L/2);
        psi(mark) = cos(k(n)*x(mark));
        psi(markL) = D*exp(ka(n)*x(markL));
    end
    psi(markR) = D*exp(-ka(n)*x(markR));
    figure; plot(x, psi); hold on;
    plot([xmin,-L/2,-L/2,L/2,L/2,xmax],...
        [1,1,0,0,1,1]*0.5);
end
\end{lstlisting}
