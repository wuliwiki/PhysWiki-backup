% Julia 调用 C 语言

\begin{issues}
\issueDraft
\end{issues}

\begin{itemize}
\item 参考\href{https://docs.julialang.org/en/v1/manual/calling-c-and-fortran-code/}{这里}.
\item 基本没有 overhead (如果不考虑 inline function)
\item 下面是 C 语言的例子, 如果是 C++ 需要把要导出的函数定义在 \verb|extern "C" {}| 中.
\end{itemize}

\begin{lstlisting}[language=cpp]
int get2 () {
 return 2;
}
double sumMyArgs (float i, float j) {
 return i+j;
}
\end{lstlisting}

编译动态链接库:
\begin{lstlisting}[language=bash]
gcc -o myclib.o -c myclib.c
gcc -shared -o libmyclib.so myclib.o -fPIC
\end{lstlisting}

在 Julia 中调用
\begin{lstlisting}[language=julia]
const myclib = joinpath(@__DIR__, "libmyclib.so")
a = ccall((:get2,myclib), Int32, ())
b = ccall((:sumMyArgs,myclib), Float64, (Float32,Float32), 2.5, 1.5)
\end{lstlisting}
其中 \verb|@__DIR__| 是当前路径. \verb|:函数名| 的类型是 \verb|Symbol|. 第二个参数是返回类型, 第三个参数是函数的参数类型, 后面是具体的参数. 如果只有一个参数, 用 \verb|(类型,)| 即 1-Tuple.
