% 二次型(线性代数)
% license Xiao
% type Tutor

\begin{issues}
\issueTODO 
\begin{itemize}
\item 本篇放在线性代数第五章 线性方程组,面向初学者。用矩阵语言介绍二次型,区分高代中张量形式的二次型(虽然它们当然是一回事)。
\item 不知道初等变换是哪个词条,需要加入到预备知识里
\item  本篇最好加上一些应用实例,譬如几何上通过二次型简化方程,以及物理上的惯性张量(?)等等。
\end{itemize}
\end{issues}
\begin{definition}{}
二次型是关于变量的二次齐次多项式。即满足$q(k\textbf{v})=k^2 q(\textbf{v})$,对于任意$\textbf{v}=(x_1,x_2...x_n),k\in \mathbb F$。容易验证,任意二次型都可以写为如下形式:
\begin{equation}
q(\textbf{v}=\textbf{v}^TQ\textbf v)~.
\end{equation}
一般称矩阵$Q$为二次型$q$的方阵形式。
\end{definition}
从定义可知,二次型可以对应多个矩阵。但是由于对任意变量有$v_iv_j=v_jv_i$,二次型总能对应唯一一个对称矩阵\footnote{前提为:域的特征不为2。因为特征为2的域有:-1=1}。
\subsection{二次型的坐标变换}
当我们用上述定义表示二次型时,如果基向量组变换,二次型的形式亦有所不同。具体而言,如果利用过渡矩阵$B$改变基向量组,由相似变换的知识可知,在新基下向量$\textbf {v'}=B\textbf{v}$,那么新的二次型形式为:
\begin{equation}
\textbf{v}^T Qv=(B^{-1}\textbf{v})^T Q(B^{-1}v)=\textbf{v}^T (B^{-1})^{T}QB^{-1}v~.
\end{equation}
因此,新的二次型形式对应矩阵$Q'=(B^{-1})^{T}QB^{-1}$,这就是常说的合同变换。合同变换的结果是同一二次型在不同基下的表示\footnote{$f(\textbf{v}:V\rightarrow \mathbb F$没有发生改变,虽然坐标不同,但还是同一个向量嘛}。
\begin{definition}{合同}
如果存在可逆方阵$C$使得
\begin{equation}
B=C^T AC~
\end{equation}
则称矩阵$A,B$合同。
\end{definition}
可以证明合同关系是一种\textbf{等价划分}。即满足反身性、传递性与对称性,等价划分实际上是在划分不同的二次型,等价类内二次型有不同的矩阵形式而已。

\begin{definition}{二次型的等价性}
给定线性空间的二次型$q_i$,如果$q_1,q_2$在某\textbf{两个}基下矩阵形式相同,则称这两个二次型等价。
\end{definition}

通过坐标变换,二次型可以简化为最简单的一种形式:对角矩阵。在对角矩阵下,二次型形式只有平方项,这就是所说的\textbf{标准二次型}。如果二次型 $q$
等价于标准二次型 $p$,那么称 $p$ 是 $q$ 的\textbf{标准形}。

由于实对称矩阵总能通过合同变换化为对角矩阵。因此实数域上的二次型总有标准形。
\begin{theorem}{}
给定实数域上的二次型 $V^T QV$,那么它总有标准形。
\end{theorem}
由于合同变换的结果总为对称矩阵,因此证明过程相对简洁,只需要利用对角元,通过初等变换把上三角的非对角元部分化为0即可\footnote{回顾初等变换,左乘可逆矩阵是行变换,右乘是列变换}。
\subsection{二次型的正定性}
二次型正定意味着$f(\textbf{v})>0$,负定则意味着$f(\textbf{v})< 0$。将二次型化为对角元为$\pm 1$,非对角元为$0$的矩阵形式,这种形式意味着基向量组是“标准正交”的。比如\textbf{“正定”}即对角矩阵为$E$,则二次型$v^{T}Qv=v^{T}v$。

二次型实际上是内积的推广。

可以证明,二次型的“正负号”数量不会随基的改变而改变。
\begin{theorem}{惯性定理}

\end{theorem}