% Landau-Ginzburg 理论
% keys 相变|超导|平均场
% license Usr
% type Wiki

\begin{issues}
\issueTODO
\end{issues}

\subsection{背景介绍}
Ginzburg 谈到,最初他想要解决在超导体中出现的温差电现象。当超导体出现温度梯度时,导体中会出现特殊的超导电流,并且涌现出电磁场。当时的伦敦方程仅仅能解释其中的部分现象,而且伦敦方程预测普通金属与超导金属的界面能是负的,因此当时急需一个非电磁场起源的界面能来解决温差电现象。到1950年,Landau-Ginzburg 理论解决了这一问题。\footnote{Ginzburg, Vitaly L. "Nobel Lecture: On superconductivity and superfluidity (what I have and have not managed to do) as well as on the “physical minimum” at the beginning of the XXI century." Reviews of Modern Physics 76.3 (2004): 981.}
\subsection{Landau-Ginzburg 理论}
基于二级相变的理论,朗道提出在超导转变温度附近,超导体的自由能可以用一个复的序参量$\psi = |\psi(r)|e^{i\phi(r)}$来描述,序参量的模平方$|\psi(r)|^2$ 可以解释为超导电子密度,自由能密度可以写为\footnote{wikipedia: Ginzburg-Landau theory}
\begin{aligned}
f_s = f_n + \alpha(T) |\psi|^2 + \beta(T) |\psi|^4 + \frac{1}{2m^*} \left| \left(i\hbar \nabla - e^* A \right)\psi \right|^2 + \frac{B^2}{8\pi}
\end{aligned}
这一方程