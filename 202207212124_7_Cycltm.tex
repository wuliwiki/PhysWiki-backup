% 分圆域
% keys 分圆域|cyclotomy|域扩张|单位根|本原根|原根|primitive element

\addTODO{预备知识待确定}



$k$次\textbf{单位根}即形如$x^k-1$的多项式在$\mathbb{C}$中的根,可以理解为$1$的$k$次根.$\pm 1$都是$1$的$2^k$次根,$\pm \I$都是$1$的$4$次单位根,而$\omega=\frac{1}{2}\qty(-1+\I\sqrt{3})$是$1$的$3$次根.

本节将使用Galois理论来处理单位根及其最小多项式对有理数域的扩张,作为Galois理论的应用.


\subsection{分圆多项式}

任意$n$次单位根都形如$\exp{2\pi k/n}$,其中$k$取从$1$到$n$的全体整数,即包括了所有$n$次单位根.如$4$次单位根构成的集合是$\{1, -1, \I, -\I\}$,而这四个元素又可以两两分组:$\pm 1$是$2$次单位根\footnote{更细节些,$1$是$1$次单位根.},$\pm \I$是$4$次单位根.显然两组性质是不同的:$\mathbb{Q}(\pm 1)=\mathbb{Q}\subsetneq \mathbb{Q}(\pm\I)$.我们将这分类表述为以下定义:

\begin{definition}{本原单位根}

当$k$与$n$互素时,称单位根$\exp{2\pi k/n}$是\textbf{本原(primitive)}的.

\end{definition}


















