% 北京大学 2005 年 考研 量子力学
% license Usr
% type Note

\textbf{声明}:“该内容来源于网络公开资料,不保证真实性,如有侵权请联系管理员”

1. (60分)简答题,可直接写出结果。

(a) 约化普朗克常量$\hbar = ?$

(b) 氢原子、二维谐振子、三维谐振子的简并度分别为?

(c) 一维谐振子、二维谐振子的第一激发态的节拍数分别为?

(d) 已知 $l_\pm = l_x \pm il_y $,求 $[l_+, l_-]$, $[l^2, l_+]$。

(e) 求 $[\hat p, \frac{1}{r}]$, $[\hat p, r^2]$。

(f) 在$x$表象中自旋本征值为 $x_0$ 的坐标波函数和动量波函数。

(g) 在$p$表象中自旋本征值为 $p_0$ 的坐标波函数和动量波函数。

(h) 求 $\hat l_x,\hat l_y$ 的共同本征态。

(i) 在相似表象中求 $e^i\frac{\pi}{4}\sigma_x\alpha$,其中 $\alpha$是$S_z =\frac{\hbar}{2} $ 的自旋态。

(j) 写出二维谐振子的两个自量子量完全集。

(k) 粒子处于势 $V(x) = \frac{1}{2}m\omega^2x^2$中,试在动量表象中写出其薛定谔方程。

2.(40分)简答题,可直接写出结果。

(a)在自然单位制下,已知相位差为$V(x) = \frac{1}{2} (x-a)^2$, 能量本征值为 $\frac{13}{2}$, 在此能量本征态下$x,\hat p,x^2,\hat p^2$的平均值。

(b)证明 $F-H$ 定理,即 
$\left( \frac{\partial E_{n}}{\partial \lambda} \right) = \overline{(\frac{\partial H}{\partial \lambda})} _n$。

(c)$\alpha,\beta$是自旋向上、向下态,有归一化本征函数$\Psi=c_1\alpha+c_2\beta$,求算符 $\frac{1}{6}\hat s_x^2 + \frac{5}{6}\hat s_y^2$ 在态 $\Psi$ 中的平均值。

(d) 已知波函数 $\psi(x) = 
\begin{cases} 
\sqrt{\frac{2\pi}{b}} \sin{bx} & |x| \leq \frac{2\pi}{b} \\
0 & |x| > \frac{2\pi}{b} 
\end{cases}$ ,试求动量的本征值及其几率振幅。

(e) 试在自然单位下求氢原子的 $\frac{1}{r},\frac{1}{r^2}$ 的平均值和径向动能。

3. (10 分) 在薛定谔表象中,坐标、动量算符用 $\hat x_8,\hat p_8$ 表示,试在海森堡表象中求解坐标、动量算符 $\hat x_H, \hat p_{H}$ 的表达式,要求用 $\hat x_8,\hat p_8$ 表示。

4. (22 分)

(a) 体系处于 $V(x) = \frac{1}{2}m\omega^2x^2$ 的第 $n$ 个本征态 $\psi_n(x)$ 中,有两个自旋 $s = 0$ 的全同粒子处于上述势中,试求最低的四个能量本征值,本征函数及其简并度。

(b) 在势场 $V(x) = 
\begin{cases} 
\frac{1}{2}m\omega^2x^2 & x \geq 0 \\
\infty & x > 0 
\end{cases}$ 中,有两个自旋为 $\frac{1}{2}$ 的全同粒子,试求最低的四个能量本征值,本征函数及其简并度。

4. (18 分)