% 字符编码

\subsection{文本文件}
我们这里讨论的是\textbf{文本文件(text file)}, 也就是 Windows 中为人熟知的 txt 拓展名文件. 文本文件只能储存字符, 也就是说文件中的信息只能是 “有几个字符” 以及 “每个字符是什么”. 所以文本文件本质上不包括字体,字号,下划线等等的其他信息. 一个文件是否是文本文件是由它的储存格式决定的, 而不是由拓展名决定的. 一些其他的文件拓展名如 html, xml, md, 以及大部分变成语言的代码文件都是文本文件.

我们把所有不是文本文件的文件统称为二进制文件(binary file), 因为从原理上来说任何文件都是以二进制的形式储存的. 例如 Word 文档保存的 doc 或 docx 拓展名文件就是二进制文件, 因为里面用其特定的格式储存了许多其他信息.  又例如图片文件 jpg, png, 视频文件 mp4 等也都是二进制文件.

文本文件的一个基本的问题是, 字符以什么形式保存? 从概念上来说, 每个字符对应一个整数, 我们把对应的规则叫做\textbf{编码(encoding)}, 一个文本文件一般只使用一种编码. 例如在著名的 ascii (读作 asky)编码中, 128 个字符被一一对应到 0-127 的整数, 这些字符包括大写和小写字母, 数字, 常见标点, 以及一些格式上的符号如空格, 换行符, 制表符等. ascii 编码可以满足通常的英语写作需求, 但显然不支持其他语言如中文. 下文会看到中文的文本文档通常以 UTF-8 (Unicode 的一种)或者 GBK 编码储存.

遗憾的是, 一般情况下文本文档中不会声明使用的编码, 所以如果储存和打开文件的默认编码方式不一致, 就会导致显示乱码. 尽管现在大部分系统和软件都用 Unicode 编码保存和打开文本文件, 但仍然有一些软件为了兼容性使用以前的编码, 例如 Windows 写字板(Notepad, 不建议使用)默认将中文保存为 GBK(即中文环境下所谓的 ANSI).

如果经常编辑文本文档(尤其是编程), 我们推荐一款功能较完善的编辑器, 例如 Visual Studio Code (VScode), Notepad++ 等. 这里我们使用 VScode 编辑器为例进行介绍.

当 VScode 打开一个文本文档时, 会尝试自动检测编码, 并显示在右下角的状态栏\autoref{encode_fig1}.
\begin{figure}[ht]
\centering
\includegraphics[width=9cm]{./figures/encode_1.png}
\caption{VScode 的右下角状态栏显示Tab 尺寸为 4, 编码为 UTF-8, 换行符为 LF, 文件为普通文本文档} \label{encode_fig1}
\end{figure}
如果这时编辑器中仍然显示乱码, 说明自动检测失败, 可以尝试手动使用不同编码查看文件: 点击状态栏上显示的编码, 选择 “Reopen with Encoding”, 选择需要的编码即可. 以正确的编码打开后, 如果想改为别的编码, 就选 “Save with Encoding”.

\subsection{换行符}
除了编码外, 换行符也是一个重要的设置. 历史上, 在使用机械打字机的时候, 当一行打完, 打字头停留在一行的最右端. 回车键(carrage return, 简写为 CR)用于将纸张上移一行, 而 LF (line feed)键用于将打字头移动到页面的最左端. 在 ascii 编码中, CR 和 LF 分别对应 13 和 10. 个 Windows 系统沿用了这个传统, 所以当我们在写字板程序中按下回车时, 分别会在文件中插入 CR 和 LF 两个字符. Linux 系统默认使用单个 LF 换行, 而 MacOS 则默认用单个 CR 换行. 所以同一个文件在不同系统的默认编辑器中可能会出现全部文字排成一行的情况. 一般来说, 一个文本文件的换行符

VScode 在打开文本文件时会自动检测

\subsection{Unicode}
Unicode 支持许多语言.
