% Bohr-Sommerfeld 氢原子模型
% keys 玻尔原子模型|半经典|量子|角动量|量子化

\pentry{玻尔原子模型\upref{BohrMd}}

\footnote{参考 Wikipedia \href{https://en.wikipedia.org/wiki/Old_quantum_theory}{相关页面}
.}本文使用原子单位制\upref{AU}. Sommerfeld 模型和玻尔原子模型一样属于量子力学发展早期的半经典原子模型, 它对玻尔模型进行了改进, 能更好地符合一些实验结果(如塞曼效应). 玻尔模型假设电子以圆形轨道绕原子核旋转, 而 Sommerfeld 模型允许椭圆轨道. 为了简单我们同样先假设原子核固定不动, 要考虑原子核运动使用相对坐标和约化质量\upref{TwoBD}即可.

Sommerfeld 模型中, 轨道量子化的条件有两个
\begin{equation}\label{BohrEc_eq4}
L = l
\end{equation}
\begin{equation}\label{BohrEc_eq3}
\oint m\dot r \dd{r} = kh
\end{equation}
其中 $L$ 是轨道角动量的模长, $l, k$ 是正整数, $h$ 是普朗克常数(原子单位制下等于 $2\pi$), $\oint$ 表示对轨道的一个周期积分. 

根据这两个量子化条件, 可以由量子数 $k, l$ 唯一地确定电子的轨道, 可以证明轨道总能量为
\begin{equation}\label{BohrEc_eq5}
E = -\frac{Z^2}{2n^2}
\end{equation}
其中
\begin{equation}\label{BohrEc_eq2}
n = l + k
\end{equation}

\subsection{数值验证}

把\autoref{CenFrc_eq10}~\upref{CenFrc} 和\autoref{BohrEc_eq4}  代入\autoref{BohrEc_eq3} 得
\begin{equation}\label{BohrEc_eq1}
2\int_{a-c}^{a+c} \sqrt{2m\qty(E - \frac{k}{r} - \frac{l^2}{2mr^2})} \dd{r} = kh
\end{equation}
其中环积分被替换为两个相等的积分, 即从椭圆轨道的近日点 $a-c$ 积分到远日点 $a+c$(详见 “椭圆的三种定义\upref{Elips3}”). 在开普勒问题(\autoref{CelBd_eq7}~\upref{CelBd},\autoref{CelBd_eq8}~\upref{CelBd})中,
\begin{equation}
e = \frac{c}{a} = \sqrt{1 + \frac{2EL^2}{mZ^2}}
\qquad
a = \frac{k}{2\abs{E}}
\end{equation}
把\autoref{BohrEc_eq5}, \autoref{BohrEc_eq2} 代入\autoref{BohrEc_eq1}, 若等式两边对所有的 $l,k > 0$ 成立, 就可以验证\autoref{BohrEc_eq5}, \autoref{BohrEc_eq2} 的是正确的. 由于该积分较为复杂, 我们姑且采用 Matlab 数值积分的办法. 注意原子单位中 $h = 2\pi$.
\begin{lstlisting}[language=matlab]
n = 2; l = 1; n = l + k;
Z = 1;
E = -Z^2/(2*n^2);
me = 1;
a = abs(k)/(2*abs(E)); b = l/sqrt(2*me*abs(E));
c = sqrt(a^2 - b^2);
f = @(r)real(sqrt(2*me*(E + Z./r - l^2./(2*me*r.^2))));
r = linspace(a-c, a+c, 500); figure; plot(r, f(r));
I = 2*integral(f, a - c, a + c);
rel_err = (I - 2*pi*k)/(2*pi*k);
disp('relative error =');
disp(rel_err);
\end{lstlisting}
运行结果: 
\begin{lstlisting}[language=matlabC]
relative error =
  -1.4136e-16
\end{lstlisting}
读者也可以把 $k, l$ 替换成其他正整数进行验证.
