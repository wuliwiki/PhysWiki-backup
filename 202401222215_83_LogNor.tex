% 对数正态分布
% keys 统计学|正态分布|概率论
% license Usr
% type Wiki

\textbf{对数正态分布} (
Log-normal distribution) 是一种概率分布,它的特点是其对数服从正态分布。对数正态分布常用于描述那些取值范围为正数的随机变量,如金融领域的股票价格、收益率等。它具有右偏(正偏)的特性,即分布的尾部向右延伸。对数正态分布在统计学、金融学和生物学等领域有广泛应用,特别是当研究对象的增长是指数型的时候,对数正态分布常常能够提供较好的拟合。

对数正态分布的概率密度函数(PDF)可以表示为:

\begin{equation}
f(x|\mu, \sigma) = \frac{1}{x \sigma \sqrt{2\pi}} e^{-\frac{(\ln(x) - \mu)^2}{2\sigma^2}}~.
\end{equation}

其中,\( x > 0 \) 是随机变量,\( \mu \) 是对数正态分布的均值,\( \sigma \) 是标准差,\( \ln \) 表示自然对数。

