% 堆放排列组合
% 排列组合|堆放排列组合

\begin{issues}
\issueDraft
\end{issues}

比热力学课本上简单得多的方法推导出这种排列组合来.

题目是这样的, 

如果有 $n$ 个不加区分的小球, 有 $N$ 个有编号的盒子( $N \geqslant n$ ), 那么把所有小球都放到盒子里有几种方法(每个盒子能装的个数没有限制)?

现在把所有的情况根据非空盒子的个数分类. 非空盒子个数可能为1个( $n$ 个小球都在里面), 2个, 一直到 $n$ 个(每个盒子只装1个). 如果用 $i$ 个盒子装小球, 那么首先从 $N$ 个盒子里面选择 $i$ 个会有 $C_n^i$ 种情况. 然后要考虑的是, 如果用已选的 $i$ 个有编号盒子装 $n$ 个小球, 又有几种情况. 用所谓的插空法%(链接未完成)
得到共有 $C_{n-1}^{i-1}$ 种情况. 所以一个 $i$ 对应 $C_N^i C_{n-1}^{i-1}$ 种情况.

最后把所有不同 $i$ 的情况数加在一起, 得出所有情况的总数为
\begin{equation}
\sum_{i = 1}^n C_N^i C_{n-1}^{i-1}
\end{equation}
又由于 $C_a^b = a!/[(a-b)!b!] = C_a^{a-b}$, 上式可变为
\begin{equation}
\sum_{i=1}^n  C_N^i C_{n-1}^{n-i}
\end{equation}
又由于 $\sum_i C_a^i C_b^{n-i} = C_{a+b}^n$ ( $i$ 取所有可能的整数使得 $i \leqslant a$,  $n - i \leqslant b$  )(见%(链接未完成)
), 上式变为
\begin{equation}
\sum_{i=1}^n C_N^i C_{n-1}^{n-i} = C_{N+n-1}^n = \frac{(N+n-1)!}{n!(N - 1)!}
\end{equation}
这就是最后的答案.

