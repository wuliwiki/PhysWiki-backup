% 麦克斯韦方程组(介质)
% 麦克斯韦方程组|介质磁导率

% 麦克斯韦方程组(介质)
% 麦克斯韦方程组|介质磁导率
\footnote{本文参考了\cite{GriffE}}
\begin{issues}
\issueDraft
\end{issues}

\pentry{麦克斯韦方程组\upref{MWEq}}

\begin{equation}\ali{
&\div\bvec D = \rho_f\\
&\curl\bvec E = -\pdv{\bvec B}{t}\\
&\div\bvec B = 0\\
&\curl\bvec H = J_f + \pdv{\bvec D}{t}
}\end{equation}

在各向同性、非铁磁性的均匀介质中,有构成关系:
\begin{align}
\bvec D &= \epsilon \bvec E = \epsilon_0 \epsilon_r \bvec E = \epsilon_0(1 + \chi_E)\bvec E\\

\bvec H &= \frac{\bvec B}{\mu} = \frac{\bvec B}{\mu_0\mu_r} = \frac{\bvec B}{\mu_0(1 + \chi_B)}\\
\end{align}
其中$\bvec D$为“电位移矢量”,$\bvec H$为“磁场强度”,$\epsilon_r$为相对介电常数,$\mu_r$为相对磁导率.

根据"电介质 \upref{dieleS} \upref{Dielec}"与"磁介质\upref{MagMat}"的物理模型,他们还可被定义为\footnote{如此定义有一定的历史因素.有一些作者认为$\bvec D, \bvec H$是数学工具而没有实际的物理含义,因为电荷只能感受到$\bvec E, \bvec B$}:
\begin{align}
\bvec D &= \epsilon_0 \bvec E + \bvec P\\
\bvec H &= \frac{\bvec B}{\mu_0} - \bvec M\\
\end{align}

% 在各向同性线性介质中,有 $\bvec P = \chi_E \epsilon_0 \bvec E$,  $\bvec M = \chi_B \bvec H$.  代入上式得 $\bvec D = (1 + \chi_E)\epsilon_0\bvec E$ 和  $\bvec H = \frac{\bvec B}{(1 + \chi_B)\mu_0}$. 

% 定义相对介电常数为 $\epsilon_r = 1 + \chi_E$, 相对磁导率为 $\mu_r = 1 + \chi_B$, 则 $\bvec D = \epsilon_r\epsilon_0\bvec E$, $\bvec H = \frac{\bvec B}{\mu_r\mu_0}$,  
