% 北京航空航天大学 2012 年数据结构与 C 语言程序设计
% keys 北京航空航天大学 2012 年数据结构与 C 语言程序设计


考生注意:所有答题务必书写在考场提供的答题纸上,写在本试题单上 的答题一律无效(本题单不参与阅卷)。

\subsection{一、填空题}
(本题共20分,每小题各2分)

1. 从总体上说,“数据结构”课程主要研究(    )三个方面的内容。

2. 若对某线性表及常用的操作是在表中插入元素或者删除表中元素,则对于顺序存储结构和链式存储结构这两种存储结构而言,线性表应该采用(    )。

3. 在长度为$n$的非空队列中进行插入或者删除操作的时间复杂度用大$O$符号表示为(    ).

4. 若一棵度为4的树中度为1、2、3和4的结点个数分别为4、2、1和1,则该树中叶结点的个数为(    )。

5. 若某一叉树的中序遍历序列为B,A,F,D,G,C,E,按层次遍历序列为A,B,C,D,E,F,G,则该二义树的后序遍历序列为(    ).

6. 将一棵结点总数为$n$、且具有$m$个叶结点的树转换为一棵二叉树以后,该二叉树中右子树为空的结点有(    )个。

7. 对于图$G=(V,E)$与$G'=(V'E)$,若有$V'\subseteq V$,$E'\subseteq E$,则称$G'$是$G$的(    ).

8. 在顺序表(6,15,30,37,65,68,70,72,89,99)中采用折半查找法看找元素37,与表中进行过比较的元素依次是(    ).

9. 若已知$n$个关键字值具有相同的散列函数值,并且采用线性探测再散列法处理冲突,那么,将这$n$个关键字值全部散列到初始为空的地址空间中,发生散列冲突的次数是(    )。

10. 若长度为$n$的序列$K=(k_1,k_2,...,k_n)$当且仅当满足$k_i\leqslant k_{2i}$,并且$k_i\leqslant k_{2i+1} (1\leqslant i\leqslant \lfloor n/2 \rfloor)$时,则称该序列为一个小顶堆积(Heap)。根据该定义,序列(26,5,77,1,61,11,59,48,15,19) 对应的小顶堆积是(     )。

\subsection{二、简答题}
(本题共20分,每小题各5分)

1. 如果一个具有100个顶点、200条边的有向图采用邻接矩阵存储,该邻接矩阵是 否是稀疏矩阵?为什么?(这里我们假设:当矩阵中非零元索的数门小「整个知阵总元 素的数目的5\%时认为该如阵为稀疏矩阵)

2. 一般情况F,建立散列表时难以避免出现散列冲突,常用处理散列冲突的方法之 是开放定址法,该方法的基木思想是什么?

3. 若对序列(2,12,列,88510)按值从小到大进行排序,前三趟排序的结果分别为: \\
第一趟排序的结果:(2J 2.16.5.10.88) \\
第二趟排序的结果:(2.12,5,10,16,88) \\
笫三趟排序的结果:(2,5,10」2,16,88) \\
请问:该结果是采用了选择排序法还是采用了(起)泡排序法得到的?为付么?

4. 快速排序法的排序过程是递归的。若待排序序列的长度为n,则快速排序的最小 递归深度与最大递归深度分别是多少?

\subsection{三、综合题}
(本题共20分,每小题各5分)

1.若非空双向循环链表中链结点结构为$llink$|$data$|$rlink$ ,则依次执行下列$4$条语句的目的是在该链表中由$q$指的结点后面插入一个由$p$指的结点,其中$1$条语句有错误,请找出该语句,并写出正确的语句。
\begin{lstlisting}[language=cpp]
p->llink=q;         /*笫1条语句*/
p->rlink=q->rlink;  /*第2条语句*/
q->rlink=p;         /*第3条语句*/
q->rlink->llink=p;  /*第4条语句*/
\end{lstlisting}

2.已知某完全二叉树的第$7$层有$10$个叶结点,请求出该完全二叉树的结点总数的最大值。(要求写出结论的求解过程)

3.证明:具有$n$个顶点的无向图最多有$n\times(n-1)/2$条边。

4.请分别写出对数据元素序列($80$,$30$,$50$,$10$,$90$,$20$)按值从大到小进行选择排序时每一趟的排序结果。

\subsection{四、算法设计题}
(本题15分)

已知某具有$n$个顶点的有向图采用邻接表方法存储,其中,用以存储有向边信息的边结点类型为
\begin{lstlisting}[language=cpp]
typedef struct edge{
    int adjvex;           /*某有向边的终止顶点在顶点结点中的位置*/
    struct edge *next;    /*指向下,个边结点*/
} ELink;
\end{lstlisting}
用以存储顶点信息的顶点结点类型为
\begin{lstlisting}[language=cpp]
typedef struct ver{
    int indegree;    /*某顶点的入度*/
    vertype vertex;  /*某顶点的数据信息*/
    ELink *link;     /*指向以该顶点为出发点的第一个边结点*/
}VLink;
\end{lstlisting}
并且$n$个顶点结点构成一个数组结构$G[0..n-1]$.请写一个算法,该算法判断给定的顶点序列$V[0..n-1]=(v_1,v_2,v_3,...,v_n)$是否是该有向图的一个拓扑序列,若是该有向图的一个拓扑序列,算法返回1,否则,算法返回0.

\subsection{五、单项选择题}
(本题共20分,每小题各2分)

1.在$C$语言中,标识符只能由字母、数字和卜划线三种字符组成,并且第一个字符(    )。 \\
A.必须是字母 $\qquad$ B.必须是下划线 \\
C.必须是字母或者下划线 $\qquad$ D.可以是字母、数字和下划线之一

2.若整型变量$x$的初值为$6$,则计算表达式$"x+=x-=x*x"$之后,$x$的值是。 \\
A. 50 $\qquad$ B. 60 $\qquad$ C. -50 $\qquad$ D. -60

3.下列4个程序段中,不是无限循环的是(    )。 \\
A. for(b=0,a=1;a>-++b;a=k++) k=a; \\
B. for(;;a++=k); \\
C. while(1) {a++;} \\
D. for(k=10;;k--) total+=k;

4.说明 “double (*ptr)[N];“ 中的标识符ptr是(    ) \\
A.$N$个指向double类型变量的指针 \\
B.指向$N$个double类型变量的函数指针 \\
C.一个指向由$N$个double类型元素组成的一维数组的指针 \\
D.具有$N$个指针元素的一维指针数组,其每一个元素都只能指向double类型变量

5.下列4个叙述中,正确的是(    ) \\
A.char *r=“china";等价于char *r; *r="china"; \\
B.char *ptr=“china”;等价于char *ptr; ptr="china"; \\
C.char string[10]={"china"};等价于char string[lO];string[ ]={"china”}; \\
D.char str[4]="abc”,temp[4]="abc”;等价于char str[4]=temp[4]="abc";

6.在$C$程序中,语句“char *func(int x,int y);”表示(    ) \\
A.对函数func的定义 \\
B.对函数func的调用 \\
C.对函数func返回值类型的说明 \\
D.对函数func的原型说明

7.对于下列程序,若从键盘上输入:abcdef<回车>,则输出结果是(    )  \\
\begin{lstlisting}[language=cpp]
#include <stdio.h>
#include <malloc.h>
main()
{
    char *p,*q;
    p=(char *)malloc(sizeof(char)*20);
    q=p;
    scanf("%s%s",p,q);
    printf("%s%s\n",p,q);
}
\end{lstlisting}
A. defdef $\qquad$ B. abcdef $\qquad$ C. abc d	$\qquad$ D. d d

8.当说明一个结构体变量时系统分配给它的内存是(    )  \\
A.结构中最后一个成员所需的内存量 \\
B.结构中第一个成员所需的内存量 \\
C.成员中占内存是最大者所需的容量 \\
D.各成员所需内存量的总和

9.下列程序的输出结果为(    ). \\
\begin{lstlisting}[language=cpp]
#define ABC(x) x*x
main( )
{
    int a,k=3;
    a=++ABC(k+1);
    print("%d",a);
}
\end{lstlisting}
A.8 $\qquad$ B.9 $\qquad$ C.14 $\qquad$ D.17

10.若要以a+方式打开一个已经存在的文件,则下列叙述中,正确的是(    ) \\
A.文件被打开时,原有的文件内容不被删除,位置指针移动到文件的末尾,可进行添加和读操作 \\
B.文件被打开时,原有的文件内容不被删除,位置指针移动到文件的开头,可进行重写和读操作 \\
C.文件被打开时,原有的文件内容被删除,只能进行写操作 \\
D.以上三种说法都不正确

\subsection{六、简答题}
(本题共20分,每小题各5分)

1.在C语言中,头文件的作用是什么? \\
2.在C语言中,\verb|#include "flename.h"和#include <filename.h>| 的区别是什么?

3.在C语言中,全局变量和局部变量的主要区别是什么?

4.字符指针、浮点数指针、以及函数指针这三种类型的变量哪个占用的内存最大?为什么?

\subsection{七、填空题}
(本题共20分,每小题各2分)

1.下列代码的功能包括:定义一个x数组,说明一个结构体,同时对变量$t$进行初始化,使得$t$的$a$成员的值为$50$,$b$成员的值为$x$数组的首地址。 \\
请在空白处(方框内)填入合适的内容,以完成上述功能。
\begin{lstlisting}[language=cpp]
int x[5]={1,2,3,4,5};
struct{
    int a;
    int *b;
)t{(1),(2)};
\end{lstlisting}

2.下列函数的功能是根据公式
\begin{equation}
s=1-\frac{1}{3}+\frac{1}{5}-\frac{1}{7}+...+\frac{1}{2n+1}~
\end{equation}
计算$s$的值,其中,$n$通过形参传入($n\geqslant0$),计算结果通过形参指针传回。 \\
请在函数的空白处(方框内)填入合适的内容,使函数完整。
\begin{lstlisting}[language=cpp]
void fun(float *sn,int n)
{
    float s=0,w,f=-l;
    int i;
    for(i=0;i<=n;i++){
        f=( (1) );
        w=f/( (2) );
        s+=w;
    }
    *sn=s;
}
\end{lstlisting}

3.下列程序实现将输入的一个小写字母循环后移5个位置后输出。例如,若输入字母'a',则输出字母'f',若输入字母'w',则输出字母'b'。 \\
请在程序的空白处(方框内)填入合适的内容,使程序完整。
\begin{lstlisting}[language=cpp]
#include <stdio.h>
main( )
{
    char c;
    c=getchar();
    if(c>='a'&&c<='u')
      ( (1) );
    else if(c>='v'&&c<='z')
      ( (2) );
    putchar(c);
}
\end{lstlisting}

4.下列自定义函数的功能是实现两个字符串的比较。 \\
请在函数的空白处(方框内)填入合适的内容,使函数完整。
\begin{lstlisting}[language=cpp]
int sstrmp(char *s,char *t)
{
    while(*s&&*t&&*s=( (1) )){
        s++;
        t++;
    }
    return( (2) };
}
\end{lstlisting}

5.下列程序的功能是将已经按升序排好序的两个字符串strl和str2中的字符再按升序归并到字符串str3中。 \\
请在程序的空白处(方框内)填入合适的内容,使程序完整。
\begin{lstlisting}[language=cpp]
#include <stdio.h>
main( )
{
    char str1[]="acegikm";
    char str2[]="bdfhjlnpq";
    char str3[],*p;
    int i=0,j=0,k=0;
    while(str1[i]!='\0'&&st2[j]!='\0'){
        if(str1[i]<str2[j])
            str3[k]=str1[i++];
        else
            ( (1) );
        k++;
    }
    str3[k]='\0';
    if( (2) )
        p=str2+j;
    else
        p=str1+i;
    strcat(str3,p);
    puts(str3);
}
\end{lstlisting}

6.对于下列main函数,经过编译、连接后得到的可执行文件名为file.exe, 并且已知在系统的命令状态下输入命令行"file Beijing Shanghai<回车>”后得到的输出结果是
\begin{lstlisting}[language=none]
Beijing
Shanghai
\end{lstlisting}
请在函数的空白处(方框内)填入合适的内容,使函数完整。
\begin{lstlisting}[language=cpp]
main(int argc,char *argv[])
{
    while( (1) ){
        ++argv;
        print("%s\n",( (2) ));
        --argc;
    }
}
\end{lstlisting}

7.下列程序的功能是打开两个已存在的文件filel和file2,并将file2拼接到filel的后面。 \\
请在程序的空白处(方框内)填入合适的内容,使程序完整。
\begin{lstlisting}[language=cpp]
#include <stdio.h>
int main()
{
    FILE *fp1,*fp2;
    if(fp1=fopen("file1!"," (1) "])==NULL){
        printf("Cannot open file!\n");
        return 0;
    }
    if(fp2=fopen("file2",( (2) ))==NULL){
        printf(*Cannot open file2!\n");
        return 0;
    while(!feof( (3) ))
        fputc( (4),fp1);
    fclose(fp1);
    fclose(fp2);
}
\end{lstlisting}

8.设$n>0$,下列函数的功能是  \\
\begin{lstlisting}[language=cpp]
int fun(int n)
{
    int count=0;
    while(n){
        count++;
        n=n/10;
    }
    return count;
}
\end{lstlisting}

9.下列程序的功能是 \\
\begin{lstlisting}[language=cpp]
#include <stdio.h>
#include <string.h>
main()
{
    char str[81],*ptrl,*ptr2;
    int n;
    gets(str);
    n=strlen(str);
    ptr1=str;
    ptr2=str+n-1;
    while(ptr1<ptr2){
        if(*ptr1!=* ptr2)
            break;
        else{
            ptr1++;
            ptr2--;
        }
    }
    if(ptr1<ptr2)
        printf("No!\!n");
    else
        printf("Yes!\n");
}
\end{lstlisting}

10.下列程序的功能是 \\
(注: fill(*FlLE)返回long类型的文件指针位置)
\begin{lstlisting}[language=cpp]
#include <stdio.h>
void main()
{
    FILE *fp;
    long position;
    fp=fopen("file.tex","a");
    fprintf(fp,"data");
    position=fell(fp);
    print("position=%ld\n",position);
    fclose(fp);
}
\end{lstlisting}