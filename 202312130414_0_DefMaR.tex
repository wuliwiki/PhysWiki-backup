% 正定矩阵(实数)
% license Usr
% type Tutor

\pentry{对称矩阵的本征问题\upref{SymEig},充分必要条件\upref{SufCnd},矩阵与线性映射\upref{MatLS}}

\footnote{参考 Wikipedia \href{https://en.wikipedia.org/wiki/Definite_matrix}{相关页面}。}\textbf{实数正定矩阵(positive definite matrix)}定义如下。 另见 “正定矩阵(复数)\upref{DefMat}”, 由于复数包含实数的情况,你也可以直接学习这篇而跳过本文。
\begin{definition}{}
若一个对称矩阵 $\mat A$, 对任意非零实数列向量 $\bvec v$ 都满足
\begin{equation}\label{eq_DefMaR_1}
\bvec v\Tr \mat A \bvec v > 0~,
\end{equation}
那么它就是\textbf{正定矩阵}。

类似地, 也可以定义\textbf{半正定矩阵}(把\autoref{eq_DefMaR_1} 中 $>$ 替换为 $\geqslant$), \textbf{负定矩阵}($<$), \textbf{半负定矩阵}($\leqslant$)。
\end{definition}
其中 $\bvec v\Tr$ 表示 $\bvec v$ 的转置\upref{Mat}。

它对应一个对称 2-线性函数, $q(v) = \bvec v\Tr \mat A \bvec v$ 是对应的二次型\upref{QuaFor}。

\begin{theorem}{}
一个矩阵 $\mat A$ 是正定矩阵当且仅当其本征值都大于零。 半正定矩阵和(半)负定矩阵的定义也类似。
\end{theorem}

证明: 令 $\mat A$ 的本征矢为 $\{\uvec u_i\}$ (一组正交归一基底), 对应本征值为 $\lambda_i$(实数), 令非零矢量为 $\bvec v = \sum_i c_i \uvec u_i$ ($c_i$ 不全为零)。 那么
\begin{equation}
\bvec v\Tr \mat A \bvec v = \sum_i \lambda_i \abs{c_i}^2~,
\end{equation}
可见若所有 $\lambda_i > 0$, 结果必然是正的。 若要求对任意不全为零的 $c_1,c_2,\dots$ 等式都大于零, 那么也能反推出所有 $\lambda_i > 0$。

\begin{theorem}{}
正定(负定)矩阵都是满秩\upref{MatRnk}的。
\end{theorem}
证明: 对于非零 $\bvec x$, $\bvec x\Tr \mat A \bvec x \ne \mat 0$, 说明 $\mat A\bvec x \ne \bvec 0$, 所以齐次方程组 $\mat A\bvec x = \bvec 0$ 唯一的解就是 $\bvec x = \bvec 0$, 所以\upref{LinEq}矩阵是满秩的, 证毕。

\begin{example}{}
求二维对称矩阵
\begin{equation}
H = \pmat{a & b\\ b & d}~
\end{equation}
正定的充分必要条件。

用特征多项式直接求本征值
\begin{equation}
(\lambda - a)(\lambda - d) - \abs{b}^2 = 0~.
\end{equation}
$\lambda$ 必定有解, 利用求根公式, 两个解大于零的充要条件是
\begin{equation}
ad > \abs{b}^2, \qquad
a > 0~.
\end{equation}
\end{example}

