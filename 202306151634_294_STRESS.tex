% 应力

\begin{issues}
\issueTODO
\end{issues}
%我的理解是这样的,不知有无dalao斧正

\footnote{本文参考自P. Beer的Mechanics of Materials与陆万明的《弹性力学》}

对于一块材料,我们很容易用截面法\upref{INTFRC}分析出材料某一截面上的受力情况;但是,截面法只能告诉我们整个截面的“总内力”,却不能告诉我们截面上“具体一点处”的受力。事实上,在不少情况下,材料各处的受力是不一样的。

\begin{example}{弯曲的棍}
\begin{figure}[ht]
\centering
\includegraphics[width=10cm]{./figures/b0cd7ca6341ec620.pdf}
\caption{弯曲的棍} \label{fig_STRESS_6}
\end{figure}
如\autoref{fig_STRESS_6} 所示,可以直观地看出,棍上部与下部的变形程度不同:上方区域被压缩,而下方区域被拉伸。在这种情况(以及其他许多情况)下,材料各处的应力不同,不能再笼统地说“材料所受的应力为...”,至少得说“材料...处的应力为...”。

\end{example}

\subsection{微元体与应力}
为了更好地处理材料某处的受力,类似于微积分\upref{IntN}中“划分小块体积”的思想,我们假定材料是由无数小正方形块组成的\footnote{这要求材料是“无限可分”的,并且每一小块还能维持物理性质不变。这当然是不“现实”\upref{MetInt}的,不过在初步的学习中,这是一个好的简化近似。},每一小块被称作“微元体 Element”。这样,材料每一点处的受力就转换为相应处一个微元体的受力。

\begin{figure}[ht]
\centering
\includegraphics[width=10cm]{./figures/53e3d7c6bde5d1a4.pdf}
\caption{三维微元体。仿自P. Beer的Mechanics of Materials} \label{fig_STRESS_1}
\end{figure}

\subsubsection{应力}

为了更好地刻画微元体受力的“局域性”,类似于密度$\rho=\dv{m}{V}, \dd m = \rho \dd V$等概念,我们引入应力的概念。

\begin{definition}{正应力、切应力、体力}
%需要补充“体力”?
微元体所受的力可以分为两类,一类作用在微元体的表面上,另一类作用在微元体的体积内。作用在表面上的力源于作用在宏观物体表面的力(\textsl{感觉有点像废话!}),例如杆件之间互相支持而产生的拉、压力等;而作用在体积上的力源于作用在宏观物体内部的力,例如杆件的重力等。

在微元体的一个面上,定义垂直于表面的“力”为正应力$\sigma$、平行于表面的“力”为切应力$\tau$。\footnote{有时不在符号上区分$\sigma$与$\tau$,并统称为应力}

正应力:$$\sigma_{ii} =\dv{F_{ii}}{A}, \dd F_{ii} = \sigma_{ii} \dd A~.$$

切应力:$$\tau_{ij}=\dv{F_{ij}}{A}, \dd F_{ij} = \tau_{ij} \dd A~.$$

在微元体的体内,定义体力:$f_i = \dv{F_i}{V}$

$\dd F_{ij}$表示微元体这个面上“分担”的内力,$\dd A$表示这个微元体这个面的表面积,$\dd V$表示微元体的体积。类似于微积分的思想,当微元体足够小时,微元体表面上不同处的受力大小也趋于一致。

i表示这个力的作用面的法方向,j表示这个力的方向\footnote{不同作者可能选取不同的约定}。
\end{definition}

\subsubsection{三维情况}
如\autoref{fig_STRESS_1} ,微元体的每一个面上可以受$3$个力,包括一个垂直于表面的力与两个平行于表面的力。看起来,一个微元体的表面上共有$3\times6=18$个力;但考虑到微元体处于静力平衡(\upref{RBSt},\autoref{sub_RGDFA_1}~\upref{RGDFA}),事实上\textbf{一个微元体上只有6个相互独立的力}。具体的论证过程比较繁琐,\textsl{按惯例留给读者作为练习}。
\addTODO{补充证明}

一个微元体的受力情况可以记为一个三阶矩阵(也称应力张量)。应力张量完整地描述了一点处物体的应力。

\begin{equation}
\mat \sigma=
\begin{bmatrix}
\sigma_{xx} & \tau_{xy} & \tau_{xz} \\
\tau_{yx} & \sigma_{yy} & \tau_{yz} \\
\tau_{zx} & \tau_{zy} & \sigma_{zz} \\
\end{bmatrix}~.
\end{equation}

这个矩阵是\textbf{对称}的,即$\tau_{xy} = \tau_{yx}, \tau_{xz}=\tau_{zx}, \tau_{yz}=\tau_{zy}$。六个独立的力可以分别选取 $\sigma_{xx}, \sigma_{yy},\sigma_{zz}, \tau_{xy}, \tau_{xz},  \tau_{yz}$。

\subsubsection{二维情况}
\begin{figure}[ht]
\centering
\includegraphics[width=5cm]{./figures/a0fcb7164fb451d8.pdf}
\caption{二维微元体.仿自P. Beer的Mechanics of Materials} \label{fig_STRESS_2}
\end{figure}

二维情况下的微元体更为简单。每一个面上只受一个正应力与一个切应力,共受$2\times4=8$个力,但只有$3$个相互独立的力。此时受力情况可以记为一个二阶矩阵,这个矩阵也是\textbf{对称}的:
\begin{equation}
\mat \sigma=
\begin{bmatrix}
\sigma_{xx} & \tau_{xy}\\
\tau_{yx} & \sigma_{yy}\\
\end{bmatrix}~.
\end{equation}
这三个力可以分别选取 $\sigma_{xx}, \sigma_{yy}, \tau_{xy}$。

\subsection{应力平衡方程}
\begin{figure}[ht]
\centering
\includegraphics[width=12cm]{./figures/4a949bdf3e7446f0.pdf}
\caption{$x$方向上的力} \label{fig_STRESS_7}
\end{figure}
对一个微元体运用力平衡定律,可以得到一些有趣的结论。在$x$方向上,我们列出力的平衡方程:
\begin{equation}
\begin{aligned}
&(\sigma_{xx} + \pdv{\sigma_{xx}}{x} \dd x - \sigma_{xx})\dd y \dd z
+ (\tau_{yx} + \pdv{\sigma_{yx}}{y} \dd y - \tau_{yx}) \dd x \dd z\\
&+ (\tau_{zx} + \pdv{\sigma_{zx}}{z} \dd z - \tau_{zx}) \dd x \dd y 
+ f_x \dd x \dd y \dd = 0~.
\end{aligned}
\end{equation}
别看形式很复杂,其实就是把各个应力都加起来。化简后得
$$
\pdv{\sigma_{xx}}{x} + \pdv{\sigma_{yx}}{y} + \pdv{\sigma_{zx}}{z} + f_x = 0~.
$$
该结论可以推广至$y$, $z$方向,因此我们一共可以列出$3$个应力平衡方程。

如果约定以$1$代表$x$轴,$2$代表$y$轴,$3$代表$z$轴,那该结论可以写为更\textsl{优雅}、紧凑的形式:
\begin{theorem}{应力平衡方程}
$$
\sum_{i=1}^3 \pdv{\sigma_{ij}}{x_i} + f_j=0~ \qquad (j=1,2,3)~.
$$
\end{theorem}

\begin{figure}[ht]
\centering
\includegraphics[width=5cm]{./figures/f27f0c3339c77588.pdf}
\caption{总表面力(应力)等于总体力} \label{fig_STRESS_8}
\end{figure}
如果你对数学与物理具有足够的热枕与敏锐,那么还有另一种简洁的方式说明应力平衡定律:如\autoref{fig_STRESS_8} 所示,我们在物体中任选取一个体积。由于这个体积受力平衡,所以这个区域受的总表面力(应力)等于总体力。
$$\oint \bvec \sigma \cdot \dd \bvec S + \int \bvec f \dd V = 0~.$$
运用散度定理
$$\int \div \bvec \sigma \dd V + \int \bvec f \dd V = 0~.$$
由于体积是任意选取的,因此
$$\div \bvec \sigma + \bvec f = 0~.$$
该结论形式上与我们之前得到的平衡方程是一致的,尽管应力实际上是张量而不是简单的矢量,因此具体的数学过程更为复杂。

\subsection{应力的宏观效果}
以上讨论了一个微元体的受力。那微元体上的应力是如何和截面上的总内力联系起来呢?答案是总内力等于各微元体应力的累和。

例如,宏观拉力 $F = \iint \sigma_x dA~.$
\begin{figure}[ht]
\centering
\includegraphics[width=10cm]{./figures/3cbe16cfdd901a9d.pdf}
\caption{拉力} \label{fig_STRESS_3}
\end{figure}

力偶 $F = \iint y\sigma_x dA~.$
\begin{figure}[ht]
\centering
\includegraphics[width=10cm]{./figures/390da6e8cf9e4c29.pdf}
\caption{力偶} \label{fig_STRESS_4}
\end{figure}

此外,叠加原理依旧适用。假如某截面处的内力既包括拉力、又包括力偶,那某点处的内应力是拉力、力偶单独存在时的内应力之和:
\begin{figure}[ht]
\centering
\includegraphics[width=10cm]{./figures/0206d59980e819a4.pdf}
\caption{叠加原理.仿自P. Beer的Mechanics of Materials} \label{fig_STRESS_5}
\end{figure}

那么反过来,我们怎么从截面上的总内力得到各微元体上的应力、即截面上“具体一点处”的受力呢?很遗憾,这没有普适的简单方法;但材料力学已经分析了材料的几类常见受力情况,并建立了相应模型,运用这些模型(\textsl{俗称套公式})就可以计算相应的应力。
