% 威廉·哈密顿(综述)
% license CCBYSA3
% type Wiki

本文根据 CC-BY-SA 协议转载翻译自维基百科\href{https://en.wikipedia.org/wiki/William_Rowan_Hamilton}{相关文章}。

威廉·罗温·汉密尔顿爵士(1805年8月4日-1865年9月2日)是爱尔兰的数学家、物理学家和天文学家。他曾担任都柏林三一学院的天文学安德鲁斯教授。

汉密尔顿是1827年至1865年间邓辛克天文台的第三任台长。他的职业生涯包括对几何光学、傅里叶分析和四元数的研究,其中四元数使他成为现代线性代数的奠基人之一。他在光学、经典力学和抽象代数方面做出了重要贡献。他的工作是现代理论物理的基础,特别是他对牛顿力学的重新表述。汉密尔顿力学,包括其汉密尔顿函数,现在在电磁学和量子力学中都占据着核心地位。

\subsection{早年生活}  
汉密尔顿是莎拉·哈顿(1780年–1817年)和阿奇博尔德·汉密尔顿(1778年–1819年)的第九个孩子,他们住在都柏林的多米尼克街29号,后改为36号。汉密尔顿的父亲来自都柏林,曾担任律师。三岁时,汉密尔顿被送去与他的叔叔詹姆斯·汉密尔顿一起生活,詹姆斯·汉密尔顿是三一学院的毕业生,在梅斯郡的特里姆经营一所学校。

汉密尔顿从小就展现出天赋。叔叔观察到,汉密尔顿从小就表现出惊人的语言学习能力——这一说法曾被一些历史学家质疑,他们认为汉密尔顿对语言的理解仅仅是基础性的。在七岁时,他已开始学习希伯来语,13岁之前,在叔叔的教导下,他已掌握了12种语言:古典和现代欧洲语言、波斯语、阿拉伯语、印度斯坦语、梵语、马拉地语和马来语。汉密尔顿早期教育中对语言的重视,归因于他父亲希望他能为英国东印度公司工作。

作为一名熟练的心算高手,年轻的汉密尔顿能够将一些计算得出许多小数位数。1813年9月,美国心算神童泽拉·科尔本在都柏林展示。科尔本9岁,比汉密尔顿大一岁。两人参加了一场心算比赛,科尔本显然获胜。

面对失败,汉密尔顿减少了学习语言的时间,转而专注于数学。10岁时,他偶然发现了一本拉丁语版的《欧几里得几何原本》;12岁时,他学习了牛顿的《算术通论》。到16岁时,他已经读完了《自然哲学的数学原理》的大部分内容,并学习了一些关于解析几何和微积分的最新著作。
\subsection{学生时期}
1822年中期,哈密尔顿开始系统地学习拉普拉斯的《天体力学》。在此期间,他发现了《天体力学》中的一个逻辑错误,这一观察使哈密尔顿认识到约翰·布林克利(当时是爱尔兰皇家天文学家)的重要性。1822年11月和12月,他完成了自己的前三篇原创数学论文。在第一次访问邓辛克天文台时,他将其中两篇展示给了布林克利,布林克利要求他进一步发展这些论文。哈密尔顿遵从了要求,并于1823年初将修改后的版本提交,布林克利认可了这个修改版。

1823年7月,哈密尔顿通过考试进入都柏林三一学院,17岁时开始了他的大学生涯。他的导师是查尔斯·博伊顿,一位家族朋友,博伊顿向他推荐了巴黎高等师范学院数学小组的现代数学著作。约翰·布林克利评论这位早熟的哈密尔顿:“这个年轻人,我不是说将来会是,而是现在就是他这个时代的第一位数学家。”

学院为哈密尔顿颁发了希腊语和物理学两项顶级奖项(最高成绩)。在所有科目和考试中,他都是第一。他的目标是通过竞争考试赢得三一学院的奖学金,但未能如愿。1826年,布林克利被任命为克洛因的主教后,哈密尔顿在1827年被任命为布林克利离职后留下的两个空缺职位:安德鲁斯天文学教授和爱尔兰皇家天文学家。尽管大学生涯被这样缩短,但他分别获得了经典学科和数学学位(1827年获得学士学位,1837年获得硕士学位)。
\subsubsection{个人生活与诗歌} 
1824年,哈密尔顿在爱德沃斯镇通过理查德·巴特勒牧师(他叔叔詹姆斯·哈密尔顿的助理)认识了小说家玛丽亚·爱德沃斯。与此同时,他的叔叔还将他介绍给了位于梅斯郡的迪士尼家族。迪士尼家的儿子们就读于三一学院,哈密尔顿与他们成为了朋友。在夏丘,他遇到了迪士尼家的妹妹凯瑟琳·迪士尼。

哈密尔顿对凯瑟琳·迪士尼产生了感情,但她的家庭不赞成这段关系,凯瑟琳被迫嫁给了威廉·巴洛牧师(她姐姐丈夫的弟弟)。婚礼于1825年举行。哈密尔顿在1826年写了一首长诗《热情者》来表达他对她的感情。二十多年后,在1847年,他向约翰·赫歇尔透露,在这段时期他可能成为了一位诗人。

1825年,哈密尔顿遇到了阿拉贝拉·劳伦斯,莎拉·劳伦斯的妹妹。莎拉是他诗歌的主要通信对象和直言不讳的批评者。他通过玛丽亚·爱德沃斯的社交圈与阿拉贝拉认识。
\subsection{在邓辛克天文台}
哈密尔顿现在是爱尔兰皇家天文学家,他定居在邓辛克天文台,并在那里度过了余生。[8] 从1827年到1865年他一直住在那里。[18] 在邓辛克天文台的早期,哈密尔顿相当规律地观察天体;[19] 后来他将常规的观察工作交给了助手查尔斯·汤普森。[20][21] 哈密尔顿的姐妹们也支持天文台的工作。[3]

哈密尔顿的天文学入门讲座广受赞誉;除了学生,讲座还吸引了学者、诗人和女性。[22] 费利西亚·赫曼斯在听过他的讲座后写下了她的诗《孤独学生的祈祷》。[23]
\subsubsection{个人生活、旅行和诗歌访问 } 
亨密尔顿在1827年邀请四个姐妹来到天文台并与他一起生活,他们一直照料家庭,直到他在1833年结婚。四个姐妹包括伊丽莎·玛丽·亨密尔顿(1807–1851),她是位诗人。[3] 在1827年,亨密尔顿写信给他的妹妹格雷丝,提到“部分”劳伦斯姐妹在都柏林与他的妹妹伊丽莎见过面。[24][25]

亨密尔顿刚被任命为天文台的皇家天文学家,他便与亚历山大·尼莫(Alexander Nimmo)一起开始了在爱尔兰和英格兰的旅行,尼莫指导他学习经纬度。[26] 其中一次旅行是到位于利物浦附近Gateacre的莎拉·劳伦斯的学校,在那里,亨密尔顿有机会评估计算员诺克斯先生。[27] 他们在同年9月访问了威廉·华兹华斯位于瑞达尔山的住所,当时作家凯撒·奥特威也在场。[28][29]: 410  访问结束后,亨密尔顿向华兹华斯送去了许多诗歌,成为了他的“诗歌弟子”[30]

当华兹华斯在1829年夏天访问都柏林时,和约翰·马歇尔及其家人一起,他住在了亨密尔顿的天文台Dunsink。[29]: 411  在1831年与尼莫的第二次英格兰旅行中,亨密尔顿在伯明翰与尼莫分道扬镳,前往利物浦地区拜访母亲那边的劳伦斯姐妹及其家族。之后,他们在湖区重新会合,攀登了赫尔维林山并与华兹华斯共进茶。亨密尔顿随后通过爱丁堡和格拉斯哥返回都柏林。[15][31]

1832年,亨密尔顿访问了位于海格特的塞缪尔·泰勒·柯尔律治,在此期间,莎拉·劳伦斯给他的一封意外介绍信帮助他顺利拜访了柯尔律治。此外,他还与阿拉贝拉一起拜访了威廉·罗斯科的家族,罗斯科于1831年去世。[32][33]

亨密尔顿是位虔诚的基督徒,被描述为“圣经的爱好者,正统且忠实的国教成员”,并且有着“对启示宗教真理的深刻信仰”。[34][35][36]