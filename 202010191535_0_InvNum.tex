% 逆序数
% keys 逆序数|排列

\pentry{集合\upref{Set}}

\subsection{逆序对}
我们讨论含有 $N$ 个元素的任意集合, 由于集合中元素的名称不重要, 我们以下将它记为 $\qty{1,2,\dots, N}$. 注意集合的是没有顺序的, 例如 $\qty{1,2,3}$ 和 $\qty{1,3,2}$ 是同一个集合. 当我们把集合 $S$ 中的的元素按照某种顺序排列成一个序列时, 就称为它是集合 $S$ 的一种\textbf{排列(permutation)}.

那么 $N$ 个元素的集合一共有几种不同的排列呢? 第 1 个位置有 $N$ 种不同的可能, 确定之后第 2 个位置有 $N-1$ 种不同的可能, 第 3 个位置有 $N-2$ 种…… 最后一个位置只有 1 种. 所以可能性的种数为
\begin{equation}
N! = N(N-1)(N-2)\dots 1
\end{equation}
符号 $N!$ 称为 $N$ 的\textbf{阶乘(factorial)}.

我们可以把第 $i$ 种排列记为 $p_i$, 该排列的元素按照顺序分别记为 $p_{i,1}, p_{i,2}, \dots, p_{i,N}$.

\subsection{逆序数}
\textbf{逆序数(Inversion number)}的定义为
\begin{equation}\label{InvNum_eq5}
N_p = \sum_{i=2}^N \text{满足}\, j<i \,\text{且}\, P_n(j) > P_{n,i} \, \text{的}\, j\, \text{的个数} 
\end{equation}
$N$ 阶行列式可以用逆序数定义为
\begin{equation}\label{InvNum_eq4}
\opn{det}{A} = \sum_{n=1}^{N!} (-1)^{N_p} \prod_{i=1}^N a_{i,P_{n,i}}
\end{equation}
