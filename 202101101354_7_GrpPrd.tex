% 群的直积和直和
% 群|直积|笛卡尔积|直和|半直和
\pentry{集合\upref{Set},群\upref{Group}}

群的直积,是在群作为集合的笛卡尔积上,由群运算自然导出的一个群.

\begin{definition}{两个群的直积}
给定群$G$和$H$,群运算的符号省略.在集合$G\times H$上定义运算:对于任意$(g_i, h_i)\in G\times H$,有$(g_1, h_1)(g_2, h_2)=(g_1g_2, h_1h_2)$.集合$G\times H$配合以上定义的运算,构成一个群,称为群$G$和$H$的\textbf{直积(direct product)}.
\end{definition}

容易看出,两个群直积的单位元是$(e, e)$——注意这里的两个$e$分属不同的群,通常是不同的元素.

这个定义分割开了参与直积的不同群的运算,因此可以很方便地直接推广到任意多个群的直积:

\begin{definition}{任意多个群的直积}
给定任意多个群,在这些群作为集合的笛卡尔积上,各分量运算分别进行运算,且遵循各自所属群的运算规则.该笛卡尔积配合该运算规则构成一个群,称为这些群的\textbf{直积(direct product)}.
\end{definition}

在群论中还有一个和直积很类似的概念,常使人混淆,这就是群的直和.直和实际上是直积的一个特例:任意给定群,都可以使用这些群来构造直积,但是直和指的是已经给定了一个群,使用它的特定子群来生成它.

在介绍群的直和之前,首先要介绍半直和的概念:









