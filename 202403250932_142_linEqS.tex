% 线性方程的解
% license Usr
% type Wiki

\begin{issues}
\issueDraft
\end{issues}
% TODO移动到线性映射

% 移动自 LinEq,原作者 ACertainUser; addis

\subsection{线性方程的解}
\begin{definition}{线性方程}
对给定的线性映射 $A:X\to Y$ 和 $b \in Y$, \textbf{线性方程}为 $A (x) = b$;所有满足该式的 $x \in X$ 的集合 $X_s$ 叫做方程的\textbf{解集}。
\end{definition}

首先注意 $A$ 未必把 $Y$ 中的每个元素都射中, 即值空间\upref{LinMap} $A(X) \subseteq Y$ 只是 $Y$ 的一个子空间。 所以只有 $b \in A(X)$ 时方程有解, 否则无解(解集为空集)。 用映射的语言, 解集 $X_s$ 就是集合 $\qty{b}$ 的逆像\upref{map} $A^{-1}(\qty{b})$。

当 $b = 0$ 时方程叫做算符 $A$ 的\textbf{齐次方程}。齐次方程的解就是映射的零空间(\autoref{def_LinMap_2}~\upref{LinMap})。\footnote{参考 Wikipedia \href{https://en.wikipedia.org/wiki/Kernel_(linear_algebra)}{相关页面}。}

\begin{theorem}{}
线性方程 $A (x) = b$ 的解集可以表示为
\begin{equation}
X_s = X_0 + x_1~.
\end{equation}
其中 $x_1$ 为 $X_s$ 中的任意元素,  $X_0$ 为映射的零空间。
\end{theorem}
说明: $X_0 + x_1 = \{x \in x_1 \mid x \in X_0\}$ 表示把 $X_0$ 中的每一个向量与 $x_1$ 相加得到的集合。 易证当 $x_1 \ne 0$ 时解集 $X_s$ 不是一个向量空间(例如不存在零向量)。

首先证明集合 $X_0 + x_1$ 中的元素满足 $Ax = b$。 令任意 $x_0 \in X_0$
\begin{equation}
A(x_0 + x_1) = Ax_0 + Ax_1 = 0 + b = b~,
\end{equation}
证毕。 再来证明解集中不存在 $X_0 + x_1$ 之外的向量。 令 $x_2 \in X_S$ 且 $x_2 \ne x_1$, 那么
\begin{equation}
A(x_2 - x_1) = Ax_2 - Ax_1 = b - b = 0~,
\end{equation}
即 $x_2 - x_1 \in X_0$, 即 $x_2 \in x_1 + X_0$。 证毕。

\subsection{非齐次方程的解集}

到此为止我们就可以非常清晰地勾画出多对一线性映射的结构了 $A:X\to Y$。 我们先找到\autoref{the_MatLS2_1}~\upref{MatLS2} 中的零空间 $X_0$ 和它的补空间 $X_1$, 其中 $X_1$ 的元素和值域空间 $A(X)$ 一一对应。 那么对每个 $x_1 \in X_1$, 线性映射就会把集合 $x_1 + X_0$ 所有元素映射到 $A(X)$ 中的同一个元素 $y_1 = Ax_1$ 上。

\addTODO{需要一个 3 维几何向量空间的真实例子, 零空间 1 维, 补空间 2 维}

