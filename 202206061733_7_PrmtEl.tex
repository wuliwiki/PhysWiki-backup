% 本原元素
% primitive element|可分扩张|域|多项式|本原元定理|单扩张|代数扩张|域扩张

\pentry{可分扩张\upref{SprbEx}}

%primitive element

有限扩张都是代数扩张,其中最简单的是单代数扩张,因为只需要添加一个元素就可以生成.而我们加下来讨论的“本原元定理”,就准确地描述了哪些有限扩张是单扩张.

\begin{definition}{本原元}

设$\mathbb{K}/\mathbb{F}$是有限次的域扩张,如果存在$a\in\mathbb{K}$使得$\mathbb{K}=\mathbb{F}(a)$,则称$a$是域扩张$\mathbb{K}/\mathbb{F}$的\textbf{本原元素(primitive element)},或者说是$\mathbb{K}$对$\mathbb{F}$的本原元素,简称本原元.

\end{definition}

下面给出一个引理,有助于理解和证明本原元定理.事实上,这个引理正是本原元定理的其中一个方向的核心思路.


\begin{lemma}{}\label{PrmtEl_lem1}
设$\mathbb{K}=\mathbb{F}(a, b)$,$\mathbb{F}$是\textbf{无限域},且$\mathbb{K}/\mathbb{F}$只有\textbf{有限多个中间域},那么必存在$x\in\mathbb{K}$使得$\mathbb{K}=\mathbb{F}(x)$.
\end{lemma}

\textbf{证明}:

任取$c\in\mathbb{F}$,构造单扩域$\mathbb{K}_c=\mathbb{F}(a+cb)$.

由于$\mathbb{F}\subseteq\mathbb{F}(a+cb)\subseteq\mathbb{K}$,因此$\mathbb{K}_c$的数量有限.但$\mathbb{F}$是无限域,因此必存在\textbf{不相等}的$c_1, c_2\in\mathbb{F}$,使得$\mathbb{K}_{c_1}=\mathbb{K}_{c_2}$,把它们都记为$\mathbb{E}$.

考虑域的封闭性.$a+c_1b$和$a+c_2b$都在$\mathbb{E}$中,故其差$(c_1-c_2)b$也在其中.因为$c_1, c_2$都在$\mathbb{F}$中,故$b$应在$\mathbb{E}$中,进而$a$也在$\mathbb{E}$中.于是$\mathbb{K}=\mathbb{F}(a, b)\subseteq\mathbb{E}$.

又因为$\mathbb{E}=\mathbb{F}(a+c_1b)\subseteq\mathbb{F}(a, b)=\mathbb{K}$,故$\mathbb{E}=\mathbb{K}$.

即,取$x=a+c_1b$即得证.



\textbf{证毕}.


\begin{corollary}{}\label{PrmtEl_cor1}
设$\mathbb{K}=\mathbb{F}(a_1, \cdots, a_n)$,$\mathbb{F}$是\textbf{无限域},且$\mathbb{K}/\mathbb{F}$只有\textbf{有限多个中间域},那么必存在$x\in\mathbb{K}$使得$\mathbb{K}=\mathbb{F}(x)$.
\end{corollary}

\textbf{证明}:

由\autoref{PrmtEl_lem1} ,$\mathbb{F}(a_1, a_2)$是单扩张,可以写为$\mathbb{F}(x_2)$.于是,$\mathbb{F}(a_1, a_2, a_3)=\mathbb{F}(x_2, a_3)$也是单扩张,可写为$\mathbb{F}(x_3)$.以此类推即得证.

\textbf{证毕}.


理解了\autoref{PrmtEl_lem1} 和\autoref{PrmtEl_cor1} ,我们就可以更方便地讨论本原元定理了.




\begin{theorem}{本原元定理}

设$\mathbb{K}/\mathbb{F}$是有限次的域扩张,则

$\mathbb{K}/\mathbb{F}$是单扩张 $\iff$ $\mathbb{K}/\mathbb{F}$的中间域\footnote{见\autoref{FldExp_the3}~\upref{FldExp}证明后的一句讨论.}只有有限多个.

\end{theorem}

\textbf{证明}:

$\mathbb{F}$是有限域时,由\autoref{FntFld_cor1}~\upref{FntFld},两个条件恒成立.因此下面只考虑$\mathbb{F}$是无限域的情况.

$\Leftarrow$:

有限扩张一定是有限生成的,即存在正整数$n$和$n$个$a_i\in\mathbb{K}$,使得$\mathbb{K}=\mathbb{F}(a_1, \cdots, a_n)$.于是,由\autoref{PrmtEl_cor1} 立刻得证.

$\Rightarrow$:

现在已知$\mathbb{K}=\mathbb{F}(a)$.设$a$在$\mathbb{F}$上的最小多项式为$f$.对于$\mathbb{K}/\mathbb{F}$任意中间域$\mathbb{M}$,设$a$在$\mathbb{M}$上的最小多项式为$f_\mathbb{M}$,则必有$f_{\mathbb{M}}\mid f$.

$f_\mathbb{M}(x)$在$\mathbb{M}\subseteq\mathbb{K}$上展开为
\begin{equation}
f_\mathbb{M} = \sum_{i=0}^n m_ix^i
\end{equation}
若设$\mathbb{M}'=\mathbb{F}(m_0, m_1, \cdots, m_n)$,那么显然$\mathbb{M}'\subseteq\mathbb{M}$,进而$f_\mathbb{M}$也是$\mathbb{M}'$上的不可约多项式,或者说$a$的最小多项式.

由\autoref{FldExp_the1}~\upref{FldExp},$[\mathbb{K}:\mathbb{M}]=\opn{deg}f_\mathbb{M}=n$,$[\mathbb{K}:\mathbb{M}']=\opn{deg}f_\mathbb{M}=n$,因此由\autoref{FldExp_the3}~\upref{FldExp},
\begin{equation}
[\mathbb{M}:\mathbb{M}']=\frac{[\mathbb{K}:\mathbb{M}']}{[\mathbb{K}:\mathbb{M}]}=\frac{n}{n}=1
\end{equation}
故$\mathbb{M}=\mathbb{M}'$.

也就是说,任何一个中间域$\mathbb{M}$,都由$f_\mathbb{M}$唯一决定.

\textbf{证毕}.
































