% 有限覆盖与紧性

\pentry{实数集的拓扑\upref{ReTop} 序列的极限\upref{SeqLim}}

\subsection{紧致性的概念}
紧致性 (compactness) 是一个重要的拓扑概念. 在分析学中, 它首次出现于对定义在实数集子集上函数的研究中. 我们试着举一些例子来说明紧致性是怎样一个概念.

从逻辑上说, 我们还没有引入连续函数这一概念, 但这并不妨碍我们从直观上去理解它. 直观上讲, 对于点集$E\subset\mathbb{R}$上的函数$f:E\to\mathbb{R}$, 如果当$x\in E$越来越接近$x_0$时, 函数值$f(x)$也会越来越接近$f$在$x_0$处的值$f(x_0)$, 那么就可以认为它在点$x_0\in E$处是"连续"的. 说$f$在$E$上连续, 也就是它在$E$的每一点处都连续. 

显然连续性是一个局部性质: 函数在一点处是否连续, 只跟它在这一点的某个邻域里的行为有关. 对于一般的集合$E$, 从其上函数的局部性质是无法推出整体性质的. 例如, 在开区间$(0,1)$上,

\begin{enumerate}
\item 函数$f_1(x)=1/x$是连续的, 但它在$x\to0$时无界; 
\item 函数$f_2(x)=x^2$连续且有界, 但却达不到它的最大和最小值, 例如当$x\to0$时$f_2(x)\to0$, 但它却取不到$0$值; 
\item 函数$f_3(x)=\sin(1/x)$连续且有界, 但在$x\to0$时震荡得越来越厉害, 根本没有极限. 
\end{enumerate}

归根结蒂, 这些"不好"的整体性质, 都来自于定义域$(0,1)$的某种"不好的性质": 由于$0$本身不属于开区间$(0,1)$, 所以没办法用$0$的邻域去覆盖到接近$0$的那些点. 当$x\in(0,1)$越来越接近端点时, 就只好用$x$的越来越小的邻域去作为看待局部性质的标尺了. 如果在开区间$(0,1)$内部来看, 当$x\to0$时, 它实际上是不会接近任何一点的 (它的极限跑出了定义域的范围). 换句话说, 我们没法找到一个一致 (uniform) 的标尺去衡量定义域$(0,1)$上的局部性质. 这时就说它缺乏紧致性. 

在上面这个例子中, 给开区间$(0,1)$补上端点就足够解决很多问题, 例如保证连续函数都有界, 都能达到最大和最小值, 而且不会"过分地震荡", 也就是说, 函数在不同的点处连续的"程度"都一样. 但在更贴近实际应用的复杂场景中, 紧致性的缺失可能会造成一些意料之外的后果. 有一个简单的例子可以说明这一点 (它属于魏尔斯特拉斯): 给定平面上不共线的三点, 则过这三点的曲线长度的最小值是由折线段达到的; 如果一定要求曲线不能有不光滑的角点, 那么寻找长度最小值的问题就无解. 当然, 可以造出光滑的曲线段使之逐渐逼近有角点的折线, 但这样得到的"极限构型"却跑出了光滑曲线的类. 由此所生发出的弱紧性 (weak compactness) 概念在现代分析学和微分方程理论中是非常基本的.

\subsection{定义}
设$K\subset\mathbb{R}$是实数集的子集.
\begin{definition}{开覆盖}
集合$K\subset\mathbb{R}$的一个开覆盖 (open cover) 是指一族开集$\{U_{\alpha}\}_{\alpha\in A}$, 使得这些开集的并集包含$K$.
\end{definition}

\begin{example}{开覆盖的例子}
开区间的族$(k,k+2),\,k\in\mathbb{Z}$组成$\mathbb{R}$的开覆盖. 它是一个可数的覆盖.

开区间$(-1,0.5),(0.4,2)$组成了闭区间$[0,1]$的开覆盖. 这是一个有限的覆盖.
\end{example}

\begin{definition}{紧集}
集合$K\subset\mathbb{R}$称作是紧致的 (compact), 如果它的任何开覆盖$\{U_{\alpha}\}_{\alpha\in A}$中都存在有限多个开集$U_{\alpha_1},...,U_{\alpha_N}$, 使得这有限个开集仍旧组成$K$的开覆盖. 
\end{definition}

简单地说, 紧集就是具有"有限覆盖性质"的点集. 乍一看这不好理解, 但实际上, 如果把某个开集都理解为