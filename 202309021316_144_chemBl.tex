% 化学反应平衡
% license Usr
% type Tutor

\begin{issues}
\issueDraft
\end{issues}

\pentry{化学势, 态函数\upref{statef}, 化学反应进度, 吉布斯自由能\upref{GibbsG}}

\subsection{化学反应平衡判据}
什么是化学反应?唯象地说,化学反应前后系统中各物质的含量发生改变,因此系统的状态量也发生改变。

我们假定一个始终保持等温等压的系统。系统反应前的状态量是 
$$G_0 = \sum_i n_i \mu_i~,$$
现在,我们想象系统中发生了一点化学反应、系统物质的组成轻微变化:
$$G_1 = \sum_i (n_i + \dd n_i) \mu_i~,$$
那么这个微小过程前后,Gibbs能变化自然是:
$$\dd G = \sum_i \mu_i \dd n_i~,$$
根据我们对Gibbs判据 \upref{GibbsG} 的理解,
$$
\begin{aligned}
\text{反应能继续发生} &\Longleftrightarrow \dd G < 0~,\\
\text{反应达到平衡} &\Longleftrightarrow \dd G = 0~.\\
\end{aligned}
$$
原则上,至此我们已经解决了化学平衡问题。

\subsection{化学反应的物质守恒:化学反应进度}
\pentry{物质的量与摩尔(高中)\upref{MOLE}}
我们上文中直接描述了每个物质的变化量$\dd n_i$。然而,根据物质守恒,每个物质的物质的量变动不是独立的。比如说,对于一个最简单的分解反应$A\to2B$,我们都知道消耗$1 \Si{mol} A$后,总会生成$2 \Si{mol} B$,而不能生成$3 \Si{mol} B$。

因此,一个化学反应其实只有一个自由度,我们称之为化学反应进度 $\dd \xi$。系统中各个物质的物质的量变化$\dd n_i$都可由$\dd \xi$确定:$\dd n_i = \nu_i \dd \xi$。其中$\nu_i$是化学计量数,即化学方程式中物质前的数字,生成物记为正、反应物记为负。

例如,在$A\to2B$中,若$\dd \xi=1 \Si{mol}$(读作“发生$1 \Si{mol}$反应后”),意味着$\dd n_A = \nu_A \dd \xi = - 1 \Si{mol}$以及$\dd n_B = \nu_i \dd \xi = 2 \Si{mol}$。

使用化学反应进度的语言,我们惊奇地发现Gibbs量变化可以写得如此简洁:
$$\dd G = \sum_i \mu_i \dd n_i = \sum_i \mu_i \nu_i \dd \xi =  \dd \xi \sum_i \mu_i \nu_i~.$$
在化学学科,一般将$\dd \xi$ "除"到左边、变为偏微分形式,从而定义反应的摩尔Gibbs变$\Delta_r G_M$:
$$\Delta_r G_M = \left(\pdv{G}{\xi}\right)_{p,T} = \sum_i \mu_i \nu_i~.$$
$\Delta_r G_M \dd \xi$的含义可以不严谨地理解为,发生少量反应后,系统Gibbs能的变化。

由于$\Delta_r G_M$只是用化学反应进度的话术复读了$\Delta G$,因此Gibbs判据对$\Delta_r G_M$依旧成立:
$$
\begin{aligned}
\text{反应能继续发生} &\Longleftrightarrow \Delta_r G_M < 0~,\\
\text{反应达到平衡} &\Longleftrightarrow \Delta_r G_M = 0~.\\
\end{aligned}
$$

\subsection{化学平衡常数}
\pentry{理想混合物的热力学量\upref{IMCPTV}}
如果你对理想混合物稍有理解,你就知道$\mu_i = \mu_i^* + RT \ln x_i$。其中$\mu_i^*$是纯物质的化学势,$x_i$是物质的物质的量浓度。

将其带回上述表达式,
$$
\begin{aligned}
\Delta_r G_M &= \sum_i \mu_i \nu_i = \sum_i (\mu_i^* + RT \ln x_i) \nu_i ~.\\
&= \sum_i \mu_i^* \nu_i + RT \sum_i \nu_i \ln x_i
\end{aligned}
$$

$\Delta_r G_M$神奇地分为了两项:
\begin{itemize}
\item $\sum_i \mu_i^* \nu_i$项与物质的浓度无关,只与物质种类(以及温度、压力、化学方程式的书写)有关,称为标准摩尔Gibbs变:$\Delta_r G_M^* = \sum_i \mu_i^* \nu_i$
\item $RT \sum_i \nu_i \ln x_i$项与物质的浓度有关。历史上,一般将$\ln$提到最前,使累加变累乘:$RT \ln \Pi_i x_i^{\nu_i}$。$\Pi_i x_i^{\nu_i}$称为活度积 $J=\Pi_i x_i^{\nu_i}$
\end{itemize}
因此,$\Delta_r G_M = \Delta_r G_M^* + RT \ln J~.$


对其应用化学平衡判据:
$$
\begin{aligned}
\text{反应达到平衡} &\Rightarrow \Delta_r G_M = 0\\
&\Rightarrow \Delta_r G_M^* + RT \ln J_{eq} = 0\\
&\Rightarrow \Delta_r G_M^* = - RT \ln J_{eq}\\
&\Rightarrow J_{eq} = e^{-\frac{\Delta_r G_M^*}{RT}}\\
\end{aligned}~.
$$

$J_{eq}$也称标准平衡常数 $K^*$,是不少化学书的标准术语。

对于一个多元系统,可以写出
\begin{equation}
G=G(p,T,n_1,n_2,...)~.
\end{equation}
那么,
\begin{equation}
\dd G=-S \dd T +V \dd P + \sum \mu_B \dd n_B~.
\end{equation}
在等温等压下,p,T均为定值
\begin{equation}
\dd G=\sum \mu_B \dd n_B~,
\end{equation}

代入化学反应进度 $\dd \xi=\dd n_B/\nu_b$,得
\begin{equation}
\dd G=\sum \nu_B \mu_B \dd \xi~.
\end{equation}
$\nu$: 化学反应系数

或写为导数形式
\begin{equation}
\dv{G}{\xi}=\sum \nu_B \mu_B~.
\end{equation}

定义摩尔Gibbs变
\begin{equation}
\Delta _r G_M = \dv{G}{\xi}=\sum \nu_B \mu_B~.
\end{equation}
含义为“化学反应进度再多进行一点后,系统的Gibbs能变”。

\subsection{化学势判据}
根据吉布斯自由能的含义\upref{GibbsG},
若 $\Delta _r G_M<0$,则反应会继续进行;若$\Delta _r G_M = 0$,则反应达到平衡。

\subsection{活度积、标准平衡常数、活度积判据}
注意到 
$\mu_B=\mu_B^*+RT \ln x_B$,那么
\begin{equation} \label{eq_chemBl_1}
\Delta _r G_M =\sum \nu_B \mu_B = \sum \nu_B (\mu_B^*+RT\ln _B)=\sum \nu_B \mu_B^* + RT \sum \nu_B \ln x_B~.
\end{equation}

定义标准摩尔Gibbs焓变 $\Delta _r G_M^*=\sum \nu_B \mu_B^*$与活度积 $J=\prod x_B^{\nu_B}~,$

那么
\begin{equation}
\Delta _r G_M = \Delta _r G_M^* + RT \ln J~.
\end{equation}

再定义标准平衡常数 $K = e^{\frac{-\Delta _r G_M^*}{RT}}$,即$\Delta _r G_M^* = -RT \ln K~.$

那么 
\begin{equation}
\Delta _r G_M = -RT \ln K + RT \ln J = RT \ln \frac{J}{K}~.
\end{equation}

根据化学势判据,当 
$RT \ln \frac{J}{K} < 0$,即$J<K$时,反应正向进行;$J=K$时,反应达到平衡。此为活度积判据。
