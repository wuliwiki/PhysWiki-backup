% 吉布斯相律
% keys 相律 相平衡 Gibbs

\footnote{本文参考自朱文涛《简明物理化学》}

\begin{theorem}{吉布斯相律}
在平衡的系统中
\begin{equation}
f=C-P+n
\end{equation}
其中:

f: 自由度
C: 独立组分数
P: 相数
n: 约束条件
\end{theorem}
\subsubsection{独立组分数}
\begin{definition}{独立 组 分数}
系统的独立组分数由以下公式定义:
\begin{equation}
C=S-R-R'
\end{equation}
其中:

C: 独立组分数
S: 物质种类数
R: 化学反应数
R': 浓度关系数
\end{definition}

可以认为,独立组分数是被更严格定义的物质种类数.

例如,在一杯纯净水中,S=1 (H2O), R=0, R'=0,因此C=1.

也有人认为,H2O会自发电离,生成H+与OH-,所以物质种类S=3 (H2O, H+, OH-);但是在这种情况下,系统中还有化学反应平衡 $H_2O\rightarrow H^++OH^-$, R=1;且有物质守恒 c(H+)=c(OH-), R'=1.最终仍有 C=S-R-R'=1.

\subsubsection{自由度}
自由度f的含义为 可以在一定范围内独立变动而不引起系统相变系统的变量的个数.自由度必须大等于0,否则这样的系统是热力学不稳定的.例如常温常压下,略微独立地改变一杯纯净水的温度、压力,都不会导致水发生相变.

\subsubsection{约束条件}
%原书没有指明该变量的名称%
n与系统所受约束条件有关,在无外场的情况下,一般取2(温度、压力均可变)或1(仅温度可变).例如,一个恒压系统的n=1.

\begin{example}{水沸腾时温度不变}
众所周知,1大气压下水沸腾时水温总为100℃.试用相律说明.

此时系统中C=1(水),P=2(气相与液相),n=1(恒压系统),因此f=0,系统不再有可变的变量,系统的温度必须是一个定值,即此时水的沸点100℃.

若不假定压力条件,则f=1,意味着系统有一个自由度.或许你已经猜到,这意味着水的沸点随压力变化.
\end{example}

\begin{example}{铁的三相共存}
\begin{figure}[ht]
\centering
\includegraphics[width=14cm]{./figures/GBPL_1.png}
\caption{铁碳相图}} \label{GBPL_fig1}
\end{figure}
\footnote{该图片来自网络}

如图,为什么727℃时,铁碳合金的三相共存区是一条直线?

此时系统中C=2(铁与碳),P=3($\alpha, \gamma, Fe_3C$),n=1(恒压系统),因此f=0,系统不再有可变的变量,系统的温度必须是一个定值,即三相平衡温度727℃.系统温度略高或略低于此都会导致三相不再能稳定共存,而发生相变.

此结论可以推广至所有的恒压二元平衡相图.二元相图中,所有三相共存区均为水平直线段.
\end{example}
