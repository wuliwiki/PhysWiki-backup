% 群的直积和直和
% 群|直积|笛卡尔积|直和|半直和
\pentry{正规子群\upref{NormSG}}

\subsection{直积}

群的直积,是在群作为集合的笛卡尔积上,由群运算自然导出的一个群.

\begin{definition}{两个群的直积}
给定群$G$和$H$,群运算的符号省略.在集合$G\times H$上定义运算:对于任意$(g_i, h_i)\in G\times H$,有$(g_1, h_1)(g_2, h_2)=(g_1g_2, h_1h_2)$.集合$G\times H$配合以上定义的运算,构成一个群,称为群$G$和$H$的\textbf{直积(direct product)}.
\end{definition}

容易看出,两个群直积的单位元是$(e, e)$——注意这里的两个$e$分属不同的群,通常是不同的元素.

这个定义分割开了参与直积的不同群的运算,因此可以很方便地直接推广到任意多个群的直积:

\begin{definition}{任意多个群的直积}
给定任意多个群$\{H_i\}$,在这些群作为集合的笛卡尔积上,各分量运算分别进行运算,且遵循各自所属群的运算规则.该笛卡尔积配合该运算规则构成一个群,称为这些群的\textbf{直积(direct product)},记为$\bigotimes_iH_i$或$\prod_iH_i$.
\end{definition}

在群论中还有一个和直积很类似的概念,常使人混淆,这就是群的\textbf{直和}.直和实际上是直积的一个特例:任意给定群,都可以使用这些群来构造直积,但是直和指的是已经给定了一个群,使用它的特定子群来生成它.

\begin{definition}{群的直和}
当$G$和$H$都是交换群时,我们也称$G\times H$为这两个群的\textbf{直和(direct sum)},此时也可以把它表示为$G+H$.另外,任意个交换群的直积$\bigotimes_iH_i$也可以表示为$\bigoplus_iH_i$或$\sum_iH_i$.
\end{definition}

直和这一术语的来源不难理解:交换群的群运算通常被称为“加法”.

\subsection{半直积}

群的直积可以推广为以下概念:

\begin{definition}{半直积}
给定群$G$,如果有$G$的一个\textbf{正规子群}$N$和一个\textbf{子群}$H$,使得$G=NH$\footnote{就是说,集合$G=\{nh|n\in N, h\in H\}$.},并且$N\cap H=\{e\}$,那么我们称$G$是$N$和$H$的\textbf{半直积(semi-direct product)},记为$G=N\rtimes H$.
\end{definition}

可以注意到,直积是半直积的一种,只要把$\{(g, e)\}$和$\{g\}$等同、把$\{(e, h)\}$和$\{h\}$等同即可.这样,尽管本节中直积是用“运算的笛卡尔积”来定义的,而半直积是用“已有的群运算”来定义的,这两个在特定情况下是等价的.

半直积不一定是直积,这是因为定义中我们只要求参与运算的两个群中的一个为正规子群,而如果两个群$G$和$H$进行直积,那么容易证明它们俩都是群$G\times H$的正规子群.从这也可以看出来为什么此处半直积的定义要先给出$G$,而不是像直积的定义一样直接用两个群的乘积得到$G\times H$.

事实上,我们也可以用以上定义半直积的语言来描述直积:给定群$G$,如果有$G$的两个正规子群$H$和$N$,满足$H\cap N=\{e\}$,并且$G=NH$,那么称$G$是$N$和$H$的直积,记为$G=N\times H$.

半直积的定义也可以不依赖于预先给定的$G$,只是这样会稍显复杂一些:

\begin{definition}{半直积}
给定群$N$和$H$,并且有同态:$f:H\rightarrow \opn{Aut}N$.在笛卡尔积集合$N\times H$上定义运算:对于$n_i\in N, h_i\in H$,有$(n_1, h_1)\cdot(n_2, h_2)=(n_1\cdot f_{h_1}(n_2), h_1\cdot h_2)$.集合$N\times H$配合这个运算可以得到一个群,称为群$N$和群$H$关于同态$f$的半直积,记为$N\rtimes_fH$.
\end{definition}














