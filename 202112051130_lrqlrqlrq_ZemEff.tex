% 塞曼效应
% 磁矩1|能级2|角动量3


在量子力学中,谱线的频率和波长的偏移意味着电子能级的跃迁.由单重态之间的电子在磁场作用下能级发生分裂,谱线分裂成间隔相等的3条谱线的塞曼效应(Zeeman Effect)被称为正常效应,而在一些实验中所观察到光谱线有时并非分裂成3条,其间隔也都不相同,这种当初始态或最终态或两者的总自旋为非零时所发生的塞曼效应被称为反常效应.尽管这两者并没有什么实质性区别,但在电子磁矩的值非常大时用反常效应解释会使问题复杂化,因此我们先来考虑由单重态的跃迁产生的赛曼正常效应.

\subsection{正常赛曼效应}
单重态的自旋为零,因此总的角动量$\mathbf{J}$等于轨道角动量$\mathbf{L}$.当原子被置于外磁场中时,原子的能量因其磁场中磁矩的能量而改变,因此在外磁场$B$中磁矩为$\mu$的原子的势能如下,
$$\Delta E=U = -\boldsymbol\mu\cdot \mathbf{B}=-\mu_z B$$方向$z$是磁场$B$的方向,我们先来处理轨道角动量$\mathbf{L}$的量子化(详情请见量子化角动量在氢原子中的范例)

首先,我们先来定义之前在谐振子问题就用到过的升降阶算符的方法
\begin{equation}
L_{\pm}\equiv L_x \pm iL_y;\ \ L_{\mp}\equiv L_x \mp iL_y
\end{equation}
接下来我们可以得到
\begin{equation}
L_{\mp}L_{\pm}=(L_x \pm i L_y)(L_x \mp i L_y)=L^2-L_z^2\mp i(i\hbar L_z)
\end{equation}
因此对于阶梯$\sigma_t$有
\begin{equation}
L^2\sigma_t=(L_{-}L_{+}+L^2_z+\hbar_z)\sigma_t=(0+\hbar^2l^2+\hbar^2l)=\hbar^2l(l+1)\sigma_t
\end{equation}
这样我们就得到了在氢原子中角动量的磁矩为
$$\mu=\frac{e}{2m_e}L=\frac{e\hbar}{2m_e}\sqrt{l(l+1)}=\sqrt{l(l+1)}\mu_B$$
其中$l$被称为角动量量子数(angular momentum quantum number)或轨道量子数(orbital quantum number),$\mu_B$为波尔磁子(Bohr magneton)
$$\mu_B\equiv\frac{e\hbar}{2m_e}=5.788\times 10^{-5}\rm{eV/T}$$
因此,我们有$$\mu_z=-m_l\mu_B=-m_l\frac{e\hbar}{2m_e}$$
\begin{equation}
\boxed{\Delta E = m_l\mu_B B}
\end{equation}
由于$m_l$有$2l+1$个取值,因此每一个能级都被分裂成$2l+1$个能级.如\autoref{ZemEff_fig1} 所示,当施加外部磁场时,像氢原子由$l=2$到$l=1$这样的跃迁使得光谱线分裂成多条紧密间隔的线.Pieter Zeeman首先观察到,这种分裂归因于磁场与轨道角动量相关的磁偶极矩之间的相互作用.
\begin{figure}[ht]
\centering
\includegraphics[width=10cm]{./figures/ZemEff_1.pdf}
\caption{氢原子在$l=1$和$l=2$之间能级的跃迁在正常赛曼效应下分裂} \label{ZemEff_fig1}
\end{figure}
根据电子偶极的选择规则(selection rule)有
$$
\Delta m_l = m_2-m_1=0,\pm 1
$$
使得能级变化的数值只有三个,
$$
\Delta E = 0,\pm \mu_B B
$$
因此,正常赛曼效应的结果就是使得一条谱线在外磁场的环境下分裂成为三条,它们彼此之间的间隔是相等且等于$\mu_B B$.这也解释了\autoref{ZemEff_fig1} 中氢原子受到外磁场的光谱现象.
\subsection{反常赛曼效应}
反常赛曼效应的解释是乌仑贝克-古兹米特的电子自选假设的有一个佐证和运用.如上所述,反常塞曼效应发生在初始态或最状态,或两者的自旋都是非零的时候.由于自旋产生的磁矩是$1$而不是$\frac{1}{2}$玻尔磁子,所以能级分裂的计算很是复杂,因此总磁矩等同于总角动量.假设一个原子的轨道角动量为$\mathbf{L}$,自旋为$\mathbf{S}$,它的总角动量为
$$\mathbf{J=L+S}$$
当单个原子处于一均匀磁场$\mathbf{B}_{\rm{ext}}$中时,能级会产生跃迁.对于单个电子的微扰(perturbation)为
$$H_z^{'} = -(\boldsymbol{\mu}_l+\boldsymbol{\mu_s})\cdot \mathbf{B}_{\rm{ext}},$$
其中电子自旋的磁偶极矩为$$\boldsymbol\mu _s =-\frac{e}{m}\mathbf{S}$$经典轨道运动的偶极矩为
$$\boldsymbol\mu _l =-\frac{e}{2m}\mathbf{L}$$
因此就有赛曼哈密顿量
\begin{equation}\label{ZemEff_eq5}
\boxed{H_z^{'} = \frac{e}{2m}(\mathbf{L+2S})\cdot \mathbf{B}_{\rm{ext}}}
\end{equation}

由于自选-轨道耦合,赛曼效应在不同的外磁场和内磁场的相对强度决定下有不同的特性.当$\mathbf{B}_{\rm{ext}}\ll \mathbf{B}_\rm{ext}$时,精细结构为决定性作用,因此$H_z^{'}$应当被认为是微扰.当$\mathbf{B}_{\rm{ext}}\gg \mathbf{B}_\rm{ext}$时,精细结构被看作微扰,因此赛曼效应起主导作用.当内磁场和外磁场没有很大的差异时,那么就需要运用简并微扰理论,此外还需要对角化哈密顿量$H_z^{'}$.

接下来,我们仍然将以氢原子为例,介绍和讨论上述三种情况分别为:弱场赛曼效应,强场赛曼效应,和中间情况下的赛曼效应.
\begin{exercise}{}
根据
\begin{equation}
\mathbf{B}=\frac{1}{4\pi\epsilon_0}\frac{e}{mc^2r^3}\mathbf{L}
\end{equation}
估算氢原子内部磁场的大小,然后定量的给出强和弱的赛曼场大小.
\end{exercise}
\subsection{弱场赛曼效应}
弱场赛曼效应由精细结构所主导,其中氢原子能级的精细结构为\begin{equation}\label{ZemEff_eq3}
\boxed{E_{nj}-\frac{13.6\rm{eV}}{n^2}\left[1+\frac{\alpha}{n^2}\left(\frac{n}{j+1/2}-\frac{3}{4}\right)\right]}
\end{equation}
轨道角动量和自旋角动量的定态是“好的”量子数$n,l,j,m_j$取不同值时所对应态的线性组合,不过由于自选-轨道耦合的影响$\mathbf{L,S}$都不再是守恒量,因此$m_s$不在其中.那么,在一级微扰理论的情况下,由赛曼效应引起的对能量的修正为
$$
E_z^{1}=\langle nljm_j|H_z^{'}|nljm_j\rangle=\frac{e}{2m}\mathbf{B}_{\rm{ext}}\cdot\langle\mathbf{L+2S}\rangle
$$
在这里$\mathbf{L+2S=J+S}$,然而我们无法直接知道$\mathbf{S}$的期待值.好在将其类比做进动的方法是可行的.我们知道总的角动量$\mathbf{J=L+S}$是一个定值,不过由于自旋-轨道耦合,$\mathbf{L}$和$\mathbf{S}$都不再是守恒量.因此,我们不妨试想$\mathbf{L}$和$\mathbf{S}$围绕着固定的总角动量矢量$\mathbf{J}$做进动.那么,这样一来$\mathbf{S}$在时间上的平均值就正好是其沿着$\mathbf{J}$上的投影
\begin{equation}\label{ZemEff_eq1}
\mathbf{S}_{\rm{ave}}=\frac{(\mathbf{S\cdot J})}{J^2}\mathbf{J}
\end{equation}
由于$\mathbf{L=J-S}$,所以$L^2=J^2+S^2-2\mathbf{J\cdot S}$,使得有
\begin{equation}
\mathbf{S\cdot J}=\frac{1}{2}(J^2+S^2-L^2)=\frac{\hbar^2}{2}[j(j+1)+s(s+1)-l(l+1)]
\end{equation}
那么结合 \autoref{ZemEff_eq1} 可得到
\begin{equation}\label{ZemEff_eq2}
\langle \mathbf{L+2S}\rangle =\left\langle \mathbf{J}+\frac{\mathbf{S\cdot J}}{J^2}\mathbf{J}\right\rangle=\left[1+\frac{j(j+1)-l(l+1)+s(s+1)}{2j(j+1)}\right]\langle\mathbf{J}\rangle
\end{equation}
在 \autoref{ZemEff_eq2} 的方括号中的内容被称作朗道$\mathbf{g}$因子,这里$s=1/2$也就有
\begin{equation}
g_j=1+\frac{j(j+1)-l(l+1)+3/4}{2j(j+1)}
\end{equation}
这样一来,当我们选择$\mathbf{B}_{\rm{ext}}$沿着$z$轴的方向,就有
\begin{equation}\label{ZemEff_eq4}
\boxed{E^1_z =\mu_Bg_J\mathbf{B}_{\rm{ext}}m_j}
\end{equation}
最后我们所要找到的总能量就是之前的精细结构部分\autoref{ZemEff_eq3} 和赛曼效应部分\autoref{ZemEff_eq4} 的总和
\begin{equation}
\boxed{E=E_Z^1+E_{nj}}
\end{equation}
\begin{example}{}
基态$s=1/2,n=0,l=0,j=1/2$分裂出的两个能级为:
\begin{equation}
-13.6\rm{eV}(1+\alpha^2/4)\pm\mu_B\mathbf{B}_{\rm{ext}}\notag
\end{equation}
正负号的$\pm\mu_B\mathbf{B}_{\rm{ext}}$分别对应$m_j=\pm1/2$.
\end{example}
\subsection{强场赛曼效应}
当$\mathbf{B}_{\rm{ext}}\gg \mathbf{B}_\rm{ext}$时,精细结构被看作微扰,因此我们主要考虑赛曼效应,设$\mathbf{B}_{\rm{ext}}$沿着$z$轴的方向.由于外力矩的影响,使得总角动量不是守恒
量,不过$S_z,L_z$是守恒量,那么好的量子数就是$n,l,m_l$和$m_z$.由\autoref{ZemEff_eq5} 可得赛曼哈密顿量为
\begin{equation}\label{ZemEff_eq12}
H_z^{'} = \frac{e}{2m}B_{\rm{ext}}(L_z+2S_z)
\end{equation}
如果完全忽略精细结构,无微扰下的能量为
\begin{equation}\label{ZemEff_eq7}
\boxed{E_{n,m_l,m_s}=-\frac{13.6\rm{eV}}{n^2}+\mu_B B_{\rm{ext}}(m_l+2m_s)}
\end{equation}
我们自然不能满足于此,在一级微扰理论的修正下,这些能级的精细结构为
\begin{equation}\label{ZemEff_eq8}
E_{fs}^l=\langle nlm_lm_s|(H^{'}_r+H^{'}_{so})|nlm_lm_s\rangle
\end{equation}
相对论效应的贡献为
\begin{equation}\label{ZemEff_eq6}
E_r^1=-\frac{(E_n)^2}{2mc^2}\left[\frac{4n}{l+1/2}-3\right]
\end{equation}
自选-轨道的相互作用项为
\begin{equation}
H^{'}_{so}=\frac{e^2}{8\pi\epsilon_0}\frac{1}{m^2c^2r^3}\mathbf{S\cdot L}
\end{equation}
\autoref{ZemEff_eq6} 中$\mathbf{S\cdot L}$的期待值为
\begin{equation}
\langle \mathbf{S\cdot L} \rangle=\langle S_x\rangle\langle L_x\rangle+\langle S_y\rangle\langle L_y\rangle+\langle S_z\rangle\langle L_y\rangle=\hbar^2
m_lm_s\end{equation}
注意对于本征态$S_z$和$L_z$,我们有$\langle S_x\rangle=\langle S_y\rangle=\langle L_y\rangle=\langle L_z\rangle=0$,最后我们看到\autoref{ZemEff_eq6} 中$\frac{1}{r^3}$的期待值为
\begin{equation}\label{ZemEff_eq11}
\left\langle\frac{1}{r^3}\right\rangle=\frac{1}{l(l+1/2)(l+1)n^3a^3}
\end{equation}
结合\autoref{ZemEff_eq7} \autoref{ZemEff_eq8} \autoref{ZemEff_eq6} 可得
\begin{equation}\label{ZemEff_eq9}
E_{fs}^l= -\frac{(E_n)^2}{2mc^2}\left[\frac{4n}{l+1/2}-3\right]+\frac{e^2}{8\pi\epsilon_0m^2c^2}\frac{\hbar^2
m_lm_s}{l(l+1/2)(l+1)n^3a^3}
\end{equation}
\begin{equation}\label{ZemEff_eq10}
\frac{2E_n^2}{mc^2}=\frac{2E_1}{mc^2}\frac{E_1}{n^4}=\frac{\alpha^2}{n^4}(13.6\rm{eV})
\end{equation}
\begin{equation}
\frac{e^2\hbar^2}{8\pi\epsilon_0m^2c^2a^3}=\frac{e^2\hbar^2}{8\pi\epsilon_0m^2c^2}\left(\frac{me^2}{4\pi\epsilon_0\hbar^2}\right)=\left(\frac{e^2}{4\pi\epsilon_0\hbar c}\right)\frac{m}{2\hbar^2}\left(\frac{e^2}{4\pi\epsilon_0}\right)^2=\alpha^2(13.6\rm{eV})
\end{equation}
我们将\autoref{ZemEff_eq9} \autoref{ZemEff_eq10} 带入到\autoref{ZemEff_eq11} 可得
\begin{equation}\label{ZemEff_eq13}
E_{fs}^1=\frac{\alpha^2}{n^3}\left[-\frac{1}{l+1/2}+\frac{3}{4n}+\frac{m_lm_s}{l(l+1/2)(l+1)}\right](13.6 \rm eV) 
\end{equation}
最后化简为我们的最终结果
\begin{equation}
\boxed{E_{fs}^1=\frac{\alpha^2}{n^3}\left[\frac{3}{4n}+\frac{l(l+1)-m_lm_s}{l(l+1/2)(l+1)}\right](13.6 \rm eV)}
\end{equation}
这样我们就终于得到了总能量为赛曼效应\autoref{ZemEff_eq12} 和精细结构 \autoref{ZemEff_eq13} 之和.

注意,方括号项在$l=0$时分母为零无法确定,这种情况下我们需取值为$1$.
\begin{example}{}
如果$l=0$那么$j=s,m_j=m_s$.使得对于q
\end{example}

\begin{example}{}

\end{example}
\subsection{中间情况下的赛曼效应}




