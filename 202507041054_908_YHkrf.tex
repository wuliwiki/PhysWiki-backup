% 约翰·考克饶夫(综述)
% license CCBYSA3
% type Wiki

本文根据 CC-BY-SA 协议转载翻译自维基百科\href{https://en.wikipedia.org/wiki/John_Cockcroft}{相关文章}。

\begin{figure}[ht]
\centering
\includegraphics[width=6cm]{./figures/ed431b9628bc8c4d.png}
\caption{1961年的考克饶夫} \label{fig_YHkrf_1}
\end{figure}
约翰·道格拉斯·考克饶夫爵士(Sir John Douglas Cockcroft,1897年5月27日-1967年9月18日)是英国核物理学家,因与欧内斯特·沃尔顿共同实现原子核裂变而获1951年诺贝尔物理学奖,这一成就对核能的发展起到了关键作用。

在第一次世界大战期间,考克饶夫曾在西线担任皇家野战炮兵服役。战后,他在曼彻斯特市立工艺学院学习电气工程,同时在大都会维克斯特拉福德园区担任学徒,并成为该公司研究部门的一员。随后,他获得奖学金进入剑桥大学圣约翰学院,并于1924年6月参加三一试,成为Wrangler(剑桥数学优等生)。欧内斯特·卢瑟福接纳考克饶夫在卡文迪许实验室攻读研究生,考克饶夫于1928年在卢瑟福的指导下完成博士学位。在沃尔顿和马克·奥利芬特的合作下,他建造了后来被称为考克饶夫–沃尔顿发生器的装置。考克饶夫和沃尔顿利用这一装置首次实现了对原子核的人造裂变,这一壮举被大众称为“劈开原子”。

在第二次世界大战期间,考克饶夫担任英国供应部科研助理主任,负责雷达相关工作。他还是处理弗里施–佩尔斯备忘录(该备忘录计算出原子弹在技术上可行)相关问题的委员会成员,并参与了随后成立的MAUD委员会。1940年,作为提泽德代表团的一员,他将英国技术与美国同行共享。战争后期,提泽德代表团成果以SCR-584雷达和近炸引信的形式返回英国,并被用于协助击落V-1飞弹。

1944年5月,他出任蒙特利尔实验室主任,负责监督ZEEP和NRX反应堆的开发,以及乔克河实验室的创建。

战后,考克饶夫出任哈韦尔原子能研究机构(AERE)主任,1947年8月15日,低功率、石墨慢化的GLEEP反应堆在哈韦尔启动,成为西欧首座投入运行的核反应堆。随后在1948年又建成了英国实验堆0号(BEPO)。哈韦尔参与了温斯凯尔反应堆和化学分离工厂的设计。在他的领导下,哈韦尔还参与了前沿聚变研究,包括ZETA计划。他坚持要求在温斯凯尔反应堆的排气烟囱上安装过滤器,这一做法曾被讥讽为“考克饶夫的愚行”,但在1957年温斯凯尔火灾导致其中一座反应堆堆芯燃烧并释放放射性物质后,这一措施证明了其重要性。

1959年至1967年,他出任剑桥大学丘吉尔学院首任院长。1961年至1965年,他还担任堪培拉澳大利亚国立大学校监。
\subsection{早年经历}
约翰·道格拉斯·考克饶夫,也被称作“Johnny W.”,于1897年5月27日出生在英格兰约克郡西区托德莫登,是纺织厂主约翰·阿瑟·考克饶夫和妻子安妮·莫德(娘家姓菲尔登,Annie Maude née Fielden)的长子。他有四个弟弟:埃里克、菲利普、基思和莱昂内尔。1901年至1908年,他在沃尔斯登的英格兰教会学校接受早期教育,1908年至1909年就读于托德莫登小学,1909年至1914年就读于托德莫登中学,在校期间,他参加了足球和板球运动。在这所学校就读的女生中,有他未来的妻子尤尼斯·伊丽莎白·克拉布特里。1914年,他获得了约克郡西区的郡优秀奖学金,进入曼彻斯特维多利亚大学学习数学。

1914年8月,第一次世界大战爆发。考克饶夫于1915年6月完成在曼彻斯特的第一学年。他加入了校内的军官训练团,但并不希望成为军官。在暑假期间,他在威尔士金梅尔军营的基督教青年会食堂工作。1915年11月24日,他参军入伍。1916年3月29日,他加入了皇家野战炮兵第59训练旅,在此接受通信兵训练。随后,他被分配到西线战场第20(轻型)师所属的第92野战炮兵旅B炮兵连服役。

考克饶夫曾参与了向兴登堡防线的推进战役和第三次伊普尔战役。他申请转任军官并获批准。1918年2月,他被派往布莱顿学习炮兵知识,1918年4月前往北安普敦郡威登贝克的候补军官学校,接受野战炮兵军官培训。1918年10月17日,他被任命为皇家野战炮兵中尉。

战争结束后,考克饶夫于1919年1月从军队退役。他选择不返回曼彻斯特维多利亚大学,而是在曼彻斯特市立工艺学院学习电气工程。由于他已在曼彻斯特维多利亚大学完成一年学业,因此获准跳过课程的第一年。他于1920年6月获得理学士学位。该校电气工程教授迈尔斯·沃克(Miles Walker)说服他在大都会维克斯公司进行学徒训练。他获得了英国1851年博览会皇家委员会颁发的“1851年博览会奖学金”,并于1922年6月提交了硕士论文《交流电的谐波分析》。

随后,沃克建议考克饶夫申请剑桥大学圣约翰学院(沃克的母校)的奖学金。考克饶夫申请成功,获得了30英镑的奖学金和20英镑的助学金(发放给经济条件有限的本科生)。大都会维克斯同意向他提供50英镑,条件是他完成学业后返回公司任职。沃克和考克饶夫的一位姑妈帮助他凑齐了总计316英镑的学费。作为其他大学的毕业生,他获准跳过三一试第一年的课程。他于1924年6月参加了三一试考试,取得了B*等级并成为Wrangler(剑桥数学优等生),并获得了学士学位。

1925年8月26日,考克饶夫在托德莫登桥街联合卫理公会教堂与伊丽莎白·克拉布特里结婚。他们共育有六个孩子,第一个孩子是男孩,名为蒂莫西,不幸在婴儿期夭折。此后,他们育有四个女儿:琼·多萝西娅(Joan Dorothea,昵称Thea)、乔斯林、伊丽莎白·菲尔登、凯瑟琳·海伦娜,以及另一位儿子克里斯托弗·休·约翰。

\begin{figure}[ht]
\centering
\includegraphics[width=6cm]{./figures/966e76c528980952.png}
\caption{} \label{fig_YHkrf_2}
\end{figure}
在大都会维克斯公司研究主管和迈尔斯·沃克的推荐下,欧内斯特·卢瑟福同意接收考克饶夫到卡文迪许实验室担任研究生。1924年,考克饶夫以剑桥圣约翰学院基金奖学金和国家奖学金的资助身份,正式注册为博士研究生。在卢瑟福的指导下,他撰写了博士论文《分子流在表面凝结时出现的现象》,并发表在《皇家学会会刊》上。1928年9月6日,他获得了博士学位。在此期间,他曾担任俄罗斯物理学家彼得·卡皮察的助手,协助其进行极低温下磁场物理的研究工作,并帮助设计和建造了液化氦设备。

1919年,卢瑟福利用衰变镭原子释放的α粒子成功实现了氮原子的裂变。这项实验及后续实验为探索原子核结构提供了线索。为了进一步研究这一领域,卢瑟福需要一种能够以足够高的速度克服原子核电荷排斥力的人造粒子加速手段,这为卡文迪许实验室开辟了一条新的研究方向。他将这一课题分配给考克饶夫、托马斯·阿利博恩和欧内斯特·沃尔顿研究。他们随后建造了后来被称为“考克饶夫–沃尔顿加速器”的装置。马克·奥利芬特为他们设计了质子源。

一个关键时刻是考克饶夫阅读了乔治·伽莫夫关于量子隧穿效应的论文后,意识到由于这一现象,所需的加速电压比最初设想的要低得多。实际上,他计算出只需能量为30万电子伏特的质子即可穿透硼原子核。随后,考克饶夫和沃尔顿花了两年时间继续改进他们的加速器。卢瑟福从剑桥大学为他们申请到一笔1000英镑的经费,用于购买变压器和其他所需设备。