% 尼古拉·特斯拉(综述)
% license CCBYSA3
% type Wiki

本文根据 CC-BY-SA 协议转载翻译自维基百科\href{https://en.wikipedia.org/wiki/Nikola_Tesla}{相关文章}。

\begin{figure}[ht]
\centering
\includegraphics[width=6cm]{./figures/4da9af8e4f34bc2b.png}
\caption{} \label{fig_Tesla_1}
\end{figure}
尼古拉·特斯拉(/ˈnɪkələ ˈtɛslə/;塞尔维亚西里尔字母:Никола Тесла,[nǐkola têsla];1856年7月10日 – 1943年1月7日)是塞尔维亚裔美国工程师、未来学家和发明家。他以对现代交流电(AC)电力供应系统设计的贡献而闻名。

特斯拉出生并成长于奥斯曼帝国,在1870年代,他首先学习了工程学和物理学,但并未获得学位。随后,他在1880年代初期,在电话通信和大陆爱迪生公司(Continental Edison)新兴的电力行业中积累了实践经验。1884年,他移民到美国,并成为美国公民。他在纽约市的爱迪生机械厂工作了短暂时间后,便开始独立创业。在合作伙伴的帮助下,为了融资和推广自己的创意,特斯拉在纽约设立了实验室和公司,开发各种电气和机械设备。他的交流电感应电动机和相关的多相交流电专利,于1888年获得了西屋电气公司的许可,这使他赚得了可观的财富,并成为该公司最终推广的多相电系统的基石。

为了开发可以申请专利并商业化的发明,特斯拉进行了多种实验,包括机械振荡器/发电机、电气放电管和早期的X射线成像。他还制造了一艘无线控制的船,是最早展出的一批之一。特斯拉作为发明家广为人知,并在他的实验室向名人和富有的赞助人展示自己的成就,他的公共讲座也因其表演性质而备受关注。整个1890年代,特斯拉在纽约和科罗拉多斯普林斯进行高电压、高频率的电力实验,追求无线照明和全球无线电力传输的构想。1893年,他宣布了使用自己设备进行无线通信的可能性。特斯拉试图将这些想法付诸实践,通过未完成的沃登克利夫塔项目,这是一个跨洲的无线通信和电力传输塔,但在资金耗尽之前他未能完成该项目。

在沃登克利夫塔之后,特斯拉在1910年代和1920年代进行了一系列发明实验,取得了不同程度的成功。由于花费了大部分的钱,特斯拉在一系列纽约酒店中居住,并留下了未付的账单。他于1943年1月在纽约市去世。特斯拉的工作在他去世后逐渐被遗忘,直到1960年,国际计量大会将国际单位制(SI)中磁通密度的单位命名为“特斯拉”,以此向他致敬。自1990年代以来,特斯拉的公众兴趣重新兴起。
\subsection{早年时期}
\begin{figure}[ht]
\centering
\includegraphics[width=6cm]{./figures/4e958945d0c8af76.png}
\caption{特斯拉重建的出生地(教区大厅)和他父亲曾服务的教堂,位于克罗地亚的斯米扬。该地点已被改建为博物馆,以纪念他。} \label{fig_Tesla_2}
\end{figure}
尼古拉·特斯拉于1856年7月10日出生在奥斯曼帝国(今克罗地亚)军事边境的斯米扬村,来自一个塞尔维亚裔家庭。他的父亲米卢廷·特斯拉(1819–1879)是东正教的牧师。他父亲的兄弟约瑟夫是军事学院的讲师,撰写了几本数学教材。

特斯拉的母亲,乔治娜“杜卡”曼迪奇(1822–1892),她的父亲也是一位东正教牧师,具有制作家用工具和机械设备的天赋,并能背诵塞尔维亚史诗诗篇。杜卡从未接受过正式教育。特斯拉将自己的过目不忘的记忆力和创造力归功于母亲的遗传和影响。

特斯拉是五个孩子中的第四个。他有三个姐妹,分别是米尔卡、安杰丽娜和马里察,还有一个名叫丹尼的哥哥,他在特斯拉六七岁时因马术事故去世。1861年,特斯拉在斯米扬的初级学校上学,学习德语、算术和宗教。1862年,特斯拉一家搬到附近的戈斯皮奇镇,特斯拉的父亲在那儿担任教区牧师。尼古拉完成了小学学业后,继续上了中学。1870年,特斯拉搬到卡尔洛瓦茨[19],在那里他进入高等实科中学(Higher Real Gymnasium)上学,课堂上讲授的是德语,这是奥匈帝国军事边境地区学校的常规语言。[20][21] 后来,在申请专利时,在获得美国国籍之前,特斯拉会将自己标识为“来自奥匈帝国边界地区的斯米扬,利卡”。[22]
\begin{figure}[ht]
\centering
\includegraphics[width=6cm]{./figures/f94da1546174f1e0.png}
\caption{特斯拉的父亲米卢廷是斯米扬村的东正教牧师。} \label{fig_Tesla_3}
\end{figure}
特斯拉后来写道,他对物理教授展示的电学实验产生了兴趣。[a] 特斯拉指出,这些“神秘现象”的展示让他想“了解更多关于这种奇妙力量的知识”。[25] 特斯拉能够在脑海中进行积分计算,这使得他的老师们认为他在作弊。[26] 他在三年内完成了四年的学业,并于1873年毕业。[27]

毕业后,特斯拉返回了斯米扬,但很快感染了霍乱,卧床不起九个月,并多次濒临死亡。在绝望时刻,特斯拉的父亲(原本希望他进入神职)[28]承诺,如果他从病中恢复,将送他去最好的工程学院。[29] 特斯拉后来表示,在恢复期间,他读了马克·吐温的早期作品。[30][31]

次年,特斯拉通过逃避征召,避开了奥匈帝国军队的征兵,他从斯米扬逃到利卡东南部的托米纳吉,靠近格拉查茨。在那里,他穿着猎人的衣服探索山脉。特斯拉表示,这种与大自然的接触让他在身体和心理上都变得更强壮。1875年,特斯拉凭借军事边境奖学金入读格拉茨的帝国皇家技术学院。在那里,特斯拉通过了九门考试(几乎是所需考试数量的两倍[33]),并收到了技术系院长写给他父亲的表扬信,信中写道:“您的儿子是一级明星。”[33] 在格拉茨,特斯拉提到他对雅各布·佩什尔教授关于电学的详细讲座产生了浓厚的兴趣,并描述了他如何提出改进教授展示的电动机设计的建议。[30][需要更好的来源][34] 但到第三年时,他在学校成绩不佳,最终未能毕业,并于1878年12月离开了格拉茨。某位传记作家认为,特斯拉并没有认真学习,可能因为赌博和花心被开除。[35]
\begin{figure}[ht]
\centering
\includegraphics[width=6cm]{./figures/6c6ed982c821e12f.png}
\caption{大约在1879年,23岁的特斯拉} \label{fig_Tesla_4}
\end{figure}
特斯拉离开学校后,他的家人再也没有收到他的消息。[36] 他同学中有一个谣言,说他在附近的穆尔河里溺水身亡,但到了1月,其中一位同学在马里博尔镇遇到了特斯拉,并将这一遭遇报告给了特斯拉的家人。[37] 结果发现,特斯拉当时在那里做草图员,每月赚取60弗罗林。[35] 1879年3月,米卢廷终于找到了他的儿子,并试图说服他回家继续在布拉格接受教育。[37] 特斯拉在同月晚些时候回到了戈斯皮奇,但因没有居住许可被驱逐出境。[37] 特斯拉的父亲在次月去世,享年60岁,死于一种不明的疾病。[37] 在那一年剩余的时间里,特斯拉在戈斯皮奇的老学校教授了一个大班的学生。

1880年1月,特斯拉的两位叔叔凑够了足够的钱,帮助他离开戈斯皮奇前往布拉格,那里他计划继续学习。然而,他到达时已经错过了查理-费迪南大学的报名时间,而且他从未学习过希腊语,这是一门必修科目;同时,他也不懂捷克语,另一门必修科目。然而,特斯拉确实以听课生的身份参加了大学的哲学讲座,但他并未为这些课程获得成绩。[38][39]
\subsubsection{在布达佩斯电话交换公司工作}
特斯拉于1881年移居匈牙利布达佩斯,在蒂瓦达尔·普什卡什的领导下,加入了一家电报公司——布达佩斯电话交换公司。到达后,特斯拉发现该公司当时正在建设中,并未投入使用,因此他改为在中央电报局担任草图员。几个月后,布达佩斯电话交换公司开始投入使用,特斯拉被分配为首席电工。在任职期间,特斯拉对中央站的设备进行了许多改进,并声称自己完善了一种电话中继器或放大器,但这一发明从未获得专利,也未公开描述。[30]\subsection{在爱迪生公司工作}  
1882年,蒂瓦达尔·普什卡什为特斯拉在巴黎的大陆爱迪生公司(Continental Edison Company)找到了另一份工作。[40] 特斯拉开始从事当时全新的行业——在大规模电力公用事业中,安装室内白炽灯,覆盖整个城市。公司有多个子公司,特斯拉在位于巴黎伊夫里-苏尔-塞纳郊区的爱迪生电气公司(Société Electrique Edison)工作,负责安装照明系统。在那里,他获得了大量电气工程的实践经验。管理层注意到他在工程学和物理学方面的深厚知识,很快就让他设计和建造改进版的发电机和电动机。[41] 他们还派他去法国和德国其他爱迪生公用事业公司解决工程问题。
\subsubsection{移居美国}
\begin{figure}[ht]
\centering
\includegraphics[width=6cm]{./figures/e2a69211c5447039.png}
\caption{位于纽约哥尔克街的爱迪生机器厂。特斯拉发现,从充满国际化气息的欧洲转到这家位于曼哈顿下东区贫民区的工厂,令他感到“痛苦的惊讶”。[42]} \label{fig_Tesla_5}
\end{figure}
1884年,负责巴黎安装项目的爱迪生经理查尔斯·巴切洛尔被召回美国,接管位于纽约市的爱迪生机器厂,并要求将特斯拉也带到美国。[43] 1884年6月,特斯拉移民美国[44],并几乎立刻开始在位于曼哈顿下东区的机器厂工作,这是一家拥挤的工厂,拥有数百名机械师、工人、管理人员和20名“现场工程师”,他们在努力建造该市的大型电力公用事业。[45] 和在巴黎一样,特斯拉主要负责排查安装问题并改进发电机。[46] 历史学家W·伯纳德·卡尔森指出,特斯拉可能只见过公司创始人托马斯·爱迪生几次。[45] 其中一次在特斯拉的自传中提到,特斯拉在整夜修理海轮“俄勒冈号”上的损坏发电机后,遇到了巴切洛尔和爱迪生,爱迪生打趣道他们的“巴黎人”整夜未归。特斯拉告诉他们自己整夜都在修理“俄勒冈号”,爱迪生对巴切洛尔评论道:“这是个了不起的人。”[42] 特斯拉负责的一个项目是开发基于弧光灯的街道照明系统。[47][48] 弧光照明是最流行的街道照明方式,但它需要高电压,与爱迪生的低电压白炽灯系统不兼容,这导致公司在一些城市失去了合同。特斯拉的设计未能投入生产,可能是由于白炽灯街道照明的技术改进,或者是因为爱迪生与一家弧光照明公司达成了安装协议。[49]

特斯拉在爱迪生机器厂工作了六个月后辞职。[45] 促使他离开的事件并不明确。可能是因为他未能获得某个奖金,奖金可能是为重新设计发电机或为被搁置的弧光照明系统设计所支付的。[47] 特斯拉此前曾因未能获得他认为自己应得的奖金而与爱迪生公司发生过冲突。[50][51] 在自传中,特斯拉提到,爱迪生机器厂的经理曾提供5万美元奖金,让他设计“24种不同类型的标准机器”,但“结果证明这是一个恶作剧”。[52][需要更好的来源] 后来的版本中,这个故事变成了托马斯·爱迪生本人提出这个奖金,并且在后来反悔,开玩笑说:“特斯拉,你不懂我们的美国幽默”。[53][54] 无论是哪种版本,奖金的数额都被认为很奇怪,因为机器厂的经理巴切洛尔对工资非常吝啬,且公司手头并没有这么多现金(今天相当于1,695,556美元)。[56][57] 特斯拉的日记中只包含了一条关于他离开工作时的评论,那是他在1884年12月7日至1885年1月4日的两页上写下的一句话:“告别爱迪生机器厂”。[48][58]
\subsection{特斯拉电光与制造公司}  
离开爱迪生公司后不久,特斯拉开始致力于为一个弧光照明系统申请专利,[59] 可能就是他在爱迪生公司开发的那个系统。[45] 1885年3月,他与专利律师莱穆埃尔·W·塞雷尔(Lemuel W. Serrell)会面,塞雷尔是爱迪生公司使用的同一律师,以寻求帮助提交专利申请。[59] 塞雷尔将特斯拉介绍给了两位商人——罗伯特·莱恩和本杰明·韦尔,他们同意为特斯拉名下的弧光照明制造和公用事业公司提供资金,成立了特斯拉电光与制造公司(Tesla Electric Light and Manufacturing Company)。[60] 特斯拉在接下来的时间里,致力于获得专利,其中包括一种改进的直流发电机,这是特斯拉在美国获得的首批专利,并在新泽西州的拉威安装和建造了该系统。[61] 特斯拉的新系统引起了技术媒体的关注,评论称其具有先进的特点。

投资者对特斯拉的新型交流电机和电力传输设备的想法兴趣不大。1886年,当公用事业公司运营起来后,他们决定制造业务竞争过于激烈,选择仅仅经营电力公用事业。[62] 他们成立了一个新的公用事业公司,抛弃了特斯拉的公司,使得这位发明家陷入贫困。[62] 特斯拉甚至失去了他所创造的专利的控制权,因为他将这些专利转让给了公司,换取了股票。[62] 他不得不从事各种电力修理工作,并且作为一名挖沟工,每天挣2美元。特斯拉在晚年回忆起1886年那段日子时,称之为艰难的时光,他写道:“我在各个科学、机械和文学领域的高等教育,仿佛对我来说是一种讽刺。”[62][c]
\subsection{交流电与感应电动机}
\begin{figure}[ht]
\centering
\includegraphics[width=6cm]{./figures/6cdded301a8bf03d.png}
\caption{来自美国专利381,968的图示,展示了特斯拉交流电感应电动机的原理} \label{fig_Tesla_6}
\end{figure}
1886年底,特斯拉遇到了西联电报公司的主管阿尔弗雷德·S·布朗和纽约律师查尔斯·弗莱彻·佩克。[64] 这两人有丰富的经验,在为发明和专利提供资金支持并推广方面颇有建树。[65] 基于特斯拉的新电气设备构想,包括热磁电动机的想法,[66] 他们同意为这位发明家提供资金支持,并处理他的专利事务。1887年4月,他们共同成立了特斯拉电气公司,并达成协议:专利产生的利润将按三分之一分配给特斯拉,三分之一分配给佩克和布朗,三分之一用于资金开发。[65] 他们在曼哈顿自由街89号为特斯拉建立了一个实验室,特斯拉在那里改进并开发新型电动机、发电机及其他设备。

1887年,特斯拉开发了一种使用交流电(AC)运行的感应电动机,这种电力系统在欧洲和美国迅速扩展,因为它在长距离高电压传输方面具有明显优势。该电动机使用多相电流,产生旋转磁场来驱动电动机(特斯拉声称这一原理是在1882年构思的)。[67][68][69] 这种创新的电动机于1888年5月获得专利,采用了简单的自启动设计,不需要换向器,从而避免了火花和频繁更换机械刷子的高维护成本。[70][71]

除了获得电动机专利,佩克和布朗还安排了让电动机广为宣传的工作,首先进行独立测试以验证其功能性改进,随后向技术出版物发布新闻稿,以便在专利发布时同时刊登相关文章。[72] 物理学家威廉·阿诺德·安东尼(测试电动机的人)和《电气世界》杂志主编托马斯·科默福德·马丁安排了特斯拉在1888年5月16日于美国电气工程师学会展示他的交流电动机。[72][73] 西屋电气制造公司的一些工程师向乔治·西屋报告,特斯拉的交流电动机及其相关电力系统是可行的——这是西屋在推广交流电系统时所需要的。西屋考虑过为意大利物理学家伽利略·费拉里斯在1885年开发的、基于旋转磁场的无换向器感应电动机申请专利,这一成果已在1888年3月的论文中展示,但他最终决定,特斯拉的专利可能会主导市场。[74][75]
\begin{figure}[ht]
\centering
\includegraphics[width=6cm]{./figures/1d473daee5879fca.png}
\caption{特斯拉的交流电发电机(AC电动发电机)在1888年美国专利390,721中的描述} \label{fig_Tesla_7}
\end{figure}
1888年7月,布朗和佩克与乔治·威斯汀豪斯(George Westinghouse)就特斯拉的多相感应电动机和变压器设计达成了许可协议,金额为6万美元现金和股票,并且每台电动机生产的每千瓦交流功率将支付2.50美元的版税。威斯汀豪斯还以每月2,000美元(按今天的通货膨胀计算相当于67,800美元)的高薪聘请特斯拉作为西屋电气制造公司匹兹堡实验室的顾问,为期一年。

在这一年里,特斯拉在匹兹堡工作,帮助创建一个交流电系统,用于为该市的电车供电。他认为这段时间很沮丧,因为与西屋的其他工程师在如何最佳实施交流电供电系统的问题上发生了冲突。他们最终采纳了特斯拉提出的60赫兹交流电系统(与特斯拉电动机的工作频率匹配),但他们很快发现,这个系统无法用于电车,因为特斯拉的感应电动机只能以恒定的速度运转。最后,他们改用了直流牵引电动机。
\subsubsection{市场动荡}  
特斯拉展示他的感应电动机以及威斯汀豪斯随后的专利许可,都是发生在1888年,这一时期恰逢电力公司之间的极端竞争。[80][81] 三大公司——威斯汀豪斯、爱迪生和汤姆森-休斯顿电力公司——都在资本密集型行业中争夺市场份额,同时通过相互压低价格来削弱对方的财务状况。甚至还爆发了“电流战争”的宣传战,爱迪生电气宣称他们的直流电系统比威斯汀豪斯的交流电系统更好、更安全,而汤姆森-休斯顿有时也支持爱迪生。[82][83] 在这样的市场竞争中,威斯汀豪斯没有足够的现金或工程资源来立即开发特斯拉的电动机和相关的多相系统。[84]

两年后,西屋电气因财务问题陷入困境。1890年,伦敦贝林银行(Barings Bank)濒临破产,引发了1890年的金融恐慌,导致投资者纷纷收回对西屋电气的贷款。[85] 资金突然短缺迫使公司不得不重新融资。新的贷款方要求西屋削减他们认为过度的支出,包括收购其他公司、研究投资以及特斯拉合同中每台电动机的专利使用费。[86][87] 此时,特斯拉的感应电动机仍未取得成功,开发进度受阻。[84][85] 尽管这种电动机的实际应用案例极少,运行它所需的多相电力系统更是罕见,西屋仍需支付每年15,000美元的保证专利使用费。[70][85]  

1891年初,乔治·西屋直言不讳地向特斯拉解释了公司的财务困难,表示如果无法满足贷款方的要求,他将失去对西屋电气的控制权,特斯拉将不得不“与银行家们交涉”以获取未来的专利使用费。[89] 考虑到让西屋继续推广感应电动机的优势,特斯拉可能认为这一点显而易见,因此同意免除合同中的专利使用费条款。[89][90]  

六年后,作为与通用电气(由1892年爱迪生公司与汤姆森-休斯顿公司合并而成)签署专利共享协议的一部分,西屋公司以一次性支付216,000美元的方式购买了特斯拉的专利。[91][92][93]
\subsection{纽约实验室}
\begin{figure}[ht]
\centering
\includegraphics[width=6cm]{./figures/74e75cf8225146ef.png}
\caption{1894年,马克·吐温在特斯拉位于南第五大道的实验室中} \label{fig_Tesla_8}
\end{figure}
特斯拉通过授权他的交流电专利赚取的财富使他变得独立富有,并为他提供了时间和资金来追求自己的兴趣。[94] 1889年,特斯拉搬出了佩克和布朗租用的自由街商店,接下来的十多年里,他在曼哈顿的多个工作室/实验室中工作。这些实验室包括位于175号大街的实验室(1889–1892年),位于南第五大道33–35号的四楼(1892–1895年),以及位于东休斯顿街46号和48号的六楼和七楼(1895–1902年)。[95][96] 特斯拉和他的雇员在这些工作室中进行了他一些最重要的工作。
\subsubsection{特斯拉线圈} 
1889年夏,特斯拉前往巴黎参加1889年世界博览会,并了解了海因里希·赫兹(Heinrich Hertz)1886年至1888年间的实验,这些实验证明了电磁辐射的存在,包括无线电波。[97] 在重复并扩展这些实验时,特斯拉尝试使用他为改进弧光照明系统而开发的高速交流发电机为鲁姆科夫线圈供电,但发现高频电流使铁芯过热,并且融化了线圈中初级和次级绕组之间的绝缘材料。为了解决这个问题,特斯拉设计了他的“振荡变压器”,在初级和次级绕组之间使用空气间隙代替绝缘材料,并设计了一个可以在变压器内部或外部移动的铁芯。[98] 后来被称为特斯拉线圈,它能够产生高电压、低电流、高频率的交流电。[99] 他在后来的无线电力传输工作中使用了这种共振变压器电路。[100][101]
\subsubsection{国籍} 
1891年7月30日,35岁的特斯拉成为美国的自然化公民。[102][103] 同年,他获得了特斯拉线圈的专利。[104]
\subsubsection{无线照明}
\begin{figure}[ht]
\centering
\includegraphics[width=6cm]{./figures/7724cea168491c9f.png}
\caption{特斯拉在1891年哥伦比亚学院的讲座中演示通过“静电感应”进行无线照明,他手中持有两根长型盖斯勒管(类似霓虹管)。} \label{fig_Tesla_9}
\end{figure}
1890年后,特斯拉开始尝试通过电感耦合和电容耦合传输电力,使用他设计的特斯拉线圈产生高电压交流电。他试图开发一种基于近场电感和电容耦合的无线照明系统,并进行了多次公开演示,在这些演示中,他成功地使盖斯勒管甚至白炽灯泡在舞台上远程点亮。特斯拉花了大部分时间与各种投资者合作,探索这一新型照明方式的不同变种,但这些尝试都未能将他的发现商业化。

1893年,在美国密苏里州圣路易斯市的演示中,特斯拉在费城的富兰克林学会和美国电气光照协会的观众面前表示,他确信像他所设计的系统,最终可以通过地球传导“可理解的信号,甚至可能传输电力到任何距离,而无需使用电线”。

从1892年到1894年,特斯拉担任美国电气工程师学会(即今天的IEEE前身之一,另一个前身是无线电工程师学会)的副主席。
\subsubsection{多相系统与哥伦比亚博览会}
\begin{figure}[ht]
\centering
\includegraphics[width=6cm]{./figures/acd22d5d3f319d24.png}
\caption{西屋公司在1893年芝加哥哥伦比亚博览会上的“特斯拉多相系统”展示} \label{fig_Tesla_10}
\end{figure}
到1893年初,西屋公司工程师查尔斯·F·斯科特和本杰明·G·拉梅在特斯拉的感应电动机的高效版本上取得了进展。拉梅找到了使其所需的多相系统与旧的单相交流电和直流电系统兼容的方法,通过开发旋转变流器。[111] 西屋电气公司现在有了一种为所有潜在客户提供电力的方法,并开始将其多相交流电系统品牌化为“特斯拉多相系统”。他们认为特斯拉的专利使他们在多相交流电系统方面拥有优先权。[112]

西屋电气公司邀请特斯拉参与1893年芝加哥哥伦比亚博览会,公司在“电力大楼”中有一个专门展示电气设备的大展区。西屋电气公司赢得了为博览会提供交流电照明的合同,这是交流电历史上的一个重要事件,展示了该公司向美国公众证明其多相交流电系统的安全性、可靠性和效率,并能够为博览会上的其他交流电和直流电展品提供电力。[113][114][115]

特设的展区展示了特斯拉感应电动机的各种形式和模型。驱动这些电动机的旋转磁场通过一系列演示来解释,其中包括使用感应电动机中的双相线圈来旋转铜蛋,使其竖立起来的“哥伦布的蛋”演示。[116]

特斯拉在博览会为期六个月的展期中,曾访问过一周,参加了国际电气大会,并在西屋展区进行了一系列演示。[117][118] 一个特别的昏暗房间被布置出来,特斯拉展示了他的无线照明系统,使用他此前在美国和欧洲演示过的实验;[119] 这些演示包括使用高电压、高频率交流电来点亮无线气体放电灯。[120]

有一位观察者注意到:

在房间内,悬挂着两块硬橡胶板,表面覆盖着锡箔。它们大约相距十五英尺,作为来自变压器的电线端子。当电流开启时,那些没有接线的灯泡或管子,放置在悬挂板之间的桌子上,或几乎可以被手持在房间的任何地方,都被点亮了。这些实验和设备与特斯拉大约两年前在伦敦展示的相同,“当时它们引起了极大的惊奇和震撼”。[121]
\subsubsection{蒸汽驱动振荡发电机}  
在哥伦比亚博览会农业厅的国际电气大会上,特斯拉介绍了他当年获得专利的蒸汽驱动往复电力发电机,他认为这是一种更好的交流电发电方式。[122] 蒸汽被压入振荡器,通过一系列的开口迅速排出,推动一个与电枢连接的活塞上下运动。磁性电枢以高速上下振动,产生交流磁场。这在相邻的线圈中感应出交流电流。它去掉了蒸汽机/发电机的复杂部件,但作为发电的可行工程解决方案始终未能普及。[123][124]
\subsubsection{尼亚加拉顾问工作}  
1893年,负责尼亚加拉大瀑布建设公司的爱德华·迪安·亚当斯(Edward Dean Adams)寻求特斯拉的意见,关于如何选择最佳的系统来传输瀑布发电的电力。几年来,关于如何最佳地实现这一目标,已经有一系列提案和公开竞赛。美国和欧洲几家公司提出的方案包括两相和三相交流电、高压直流电和压缩空气等。亚当斯向特斯拉询问了所有竞争系统的当前状态。特斯拉建议亚当斯,二相系统将是最可靠的,并指出西屋电气公司曾在哥伦比亚博览会上展示过使用二相交流电点亮白炽灯的系统。根据特斯拉的建议以及西屋电气在哥伦比亚博览会上的展示,公司向西屋电气授予了在尼亚加拉大瀑布建设二相交流电发电系统的合同。同时,通用电气公司也获得了建设交流电配电系统的合同。[125]
\subsubsection{尼古拉·特斯拉公司} 
1895年,爱德华·迪安·亚当斯(Edward Dean Adams)在参观特斯拉实验室时对所见印象深刻,同意帮助成立尼古拉·特斯拉公司,该公司旨在资助、开发和营销特斯拉的多项专利和发明,包括一些新发明。阿尔弗雷德·布朗(Alfred Brown)也签约,带来了在佩克和布朗公司开发的专利。董事会成员还包括威廉·伯奇·兰金(William Birch Rankine)和查尔斯·F·科尼(Charles F. Coaney)。然而,由于90年代中期的经济困境,投资者寥寥无几,而公司计划推广的无线照明和振荡器专利未能取得成功。该公司继续处理特斯拉的专利,直到多年以后。
\subsubsection{实验室火灾}  
1895年3月13日凌晨,特斯拉实验室所在的南第五大道大楼发生火灾。火灾从大楼的地下室起火,火势非常猛烈,以至于特斯拉位于四楼的实验室被烧毁并坍塌到二楼。火灾不仅使特斯拉的正在进行的项目受到严重影响,还摧毁了他的一大批早期笔记和研究资料、模型以及演示作品,其中许多曾在1893年世界哥伦比亚博览会上展出。特斯拉在接受《纽约时报》采访时表示:“我太伤心了,无法言语。我还能说什么呢?”[127] 火灾后,特斯拉搬到东休斯顿街46号和48号,并在6楼和7楼重建了他的实验室。
\subsubsection{X射线实验}
\begin{figure}[ht]
\centering
\includegraphics[width=6cm]{./figures/546876667dccb745.png}
\caption{特斯拉拍摄的他手部的X光照片} \label{fig_Tesla_18}
\end{figure}
从1894年开始,特斯拉开始研究他所称之为“不可见”种类的辐射能量,在之前的实验中,他注意到实验室中的胶片受损(后来被确认是“伦琴射线”或“X射线”)。他早期的实验使用了克鲁克斯管,这是一种冷阴极电气放电管。特斯拉可能无意中拍摄到了X射线图像——这比威廉·伦琴在1895年12月宣布X射线发现的时间早了几周——当时他试图用盖斯勒管(一种早期的气体放电管)拍摄马克·吐温的照片,然而在图像中,唯一拍摄到的东西是相机镜头上的金属锁螺丝。

1896年3月,在得知伦琴发现X射线及X射线成像(放射摄影)后,[130] 特斯拉开始进行自己的X射线成像实验。他设计了一种高能量单端真空管,该管没有靶电极,并且通过特斯拉线圈的输出工作(现代对此设备所产生现象的术语是“制动辐射”或“刹车辐射”)。在研究过程中,特斯拉设计了几种实验装置来产生X射线。他认为,利用他的电路,“这种仪器将……能够产生比普通设备更强大的伦琴射线”。[131]

特斯拉注意到使用他的电路和单节点X射线产生设备时的危险。在他关于这一现象的早期研究笔记中,他将皮肤损伤归因于多种原因。他早期认为皮肤损伤不是由伦琴射线引起的,而是由与皮肤接触时产生的臭氧所致,较小程度上是由硝酸引起的。特斯拉错误地认为X射线是纵波,就像等离子体中的波动那样。这些等离子体波可以出现在无力磁场中。[132][133]

1934年7月11日,《纽约先驱论坛报》刊登了一篇关于特斯拉的文章,其中回顾了他在实验单电极真空管时偶尔发生的一个事件。一个微小的粒子从阴极上脱落,穿过管子并物理撞击到他身上:[134]

特斯拉说他能感觉到粒子进入身体的地方传来一阵尖锐的刺痛感,然后在它穿出身体的地方再次感到疼痛。在将这些粒子与他“电枪”发射出的金属片进行比较时,特斯拉说:“这些力束中的粒子……将比这些粒子旅行得更快……而且它们会以集中状态旅行。”
\subsubsection{无线电遥控}
\begin{figure}[ht]
\centering
\includegraphics[width=6cm]{./figures/abee86456e2de99e.png}
\caption{1898年,特斯拉展示了一艘无线电控制的船只,他希望将其作为一种引导型鱼雷销售给世界各国的海军。[135]} \label{fig_Tesla_11}
\end{figure}
1898年,特斯拉展示了一艘使用基于同调器的无线电控制(他称之为“远程自动机”)的船只,公开演示是在麦迪逊广场花园的一场电气展览上进行的。[136] 特斯拉试图将这一创意卖给美国军方,作为一种无线电控制的鱼雷,但他们表现得兴趣不大。[137] 无线电遥控在第一次世界大战之前仍然是一个新奇的概念,直到战后,多个国家才开始将其用于军事项目。[138] 1899年5月13日,在前往科罗拉多斯普林斯的途中,特斯拉在芝加哥商业俱乐部的会议上进一步展示了“远程自动机”。
\subsection{无线电能}
\begin{figure}[ht]
\centering
\includegraphics[width=6cm]{./figures/7fe7dd2f4a7d9a3f.png}
\caption{特斯拉坐在他位于东休斯顿街实验室中,用于无线电力实验的螺旋线圈前} \label{fig_Tesla_17}
\end{figure}
从1890年代到1906年,特斯拉将大量时间和财富投入到一系列项目中,试图开发无线电能传输。这是他利用线圈进行无线照明实验的延伸。他认为,这不仅是将大量电力传输到全球的一种方式,而且,正如他在早期的演讲中所指出的,这也是实现全球通信的一种方式。

当时,特斯拉在构思他的想法时,尚没有可行的方式在长距离上传输通信信号,更不用说传输大量电力了。特斯拉早期就研究过无线电波,并得出结论,赫兹对其的部分研究是错误的。[139][140][d] 此外,这种新形式的辐射在当时被广泛认为是短距离现象,似乎在不到一英里的距离内就会衰减消失。[142] 特斯拉指出,即便无线电波的理论是正确的,它们对他预期的目的完全没有价值,因为这种“看不见的光”会像任何其他辐射一样,随着距离的增加而衰减,并且会沿直线传播,最终进入太空,变得“彻底迷失”。[143]

到1890年代中期,特斯拉开始着手研究一个想法,即他可能通过地球或大气层长距离传导电力,并开始进行实验来验证这个想法,包括在他位于东休斯顿街的实验室中设置一个大型共振变压器放大发射器。[144][145][146] 他似乎借鉴了当时流行的一个观点,即地球的大气层具有导电性,[147][148] 他提出了一个由气球悬浮的电极组成的系统,用来传输和接收信号,这些电极位于海拔30,000英尺(约9,100米)以上的高空,特斯拉认为在低气压环境下,他能够将高电压(数百万伏特)传输到远距离。
\subsubsection{科罗拉多斯普林斯}
\begin{figure}[ht]
\centering
\includegraphics[width=6cm]{./figures/f3fa24d11468d705.png}
\caption{特斯拉的科罗拉多斯普林斯实验室} \label{fig_Tesla_12}
\end{figure}
为了进一步研究低压空气的导电性质,特斯拉于1899年在科罗拉多斯普林斯建立了一个高海拔实验站。[149][150][151][152] 在那里,他可以安全地操作比纽约实验室更大号的线圈,并且一位合作伙伴安排了埃尔帕索电力公司免费提供交流电。[152] 为了资助他的实验,他说服约翰·雅各布·阿斯特四世投资10万美元(相当于今天的366.24万美元[76]),成为尼古拉·特斯拉公司的大股东。阿斯特以为自己主要是在投资新的无线照明系统,但特斯拉却将这笔钱用于资助他在科罗拉多斯普林斯的实验。[153] 到达后,特斯拉告诉记者,他计划进行无线电报实验,从派克峰向巴黎传输信号。[154]
\begin{figure}[ht]
\centering
\includegraphics[width=6cm]{./figures/e89f9a2ffb9ef345.png}
\caption{一张多重曝光的照片,特斯拉坐在他的“放大发射器”旁,该装置产生数百万伏特。7米(23英尺)长的电弧并非正常操作的一部分,而是通过快速切换电源开关制造的效果。} \label{fig_Tesla_16}
\end{figure}
在那里,他进行了一系列实验,使用一个大电感器工作在兆伏特范围内,产生了人工闪电(和雷鸣),电压高达数百万伏特,放电长度最长可达135英尺(41米)。有一次,他不小心烧坏了埃尔帕索的发电机,导致停电。[156] 他对闪电击中的电子噪声进行的观察使他(错误地)得出结论,认为他可以利用整个地球来传导电能。[157][158][159]

在他在实验室的期间,特斯拉观察到接收器传来的异常信号,他推测这些信号可能来自另一个星球。他在1899年12月写信给一名记者时提到了这些信号[160],并在1900年12月向红十字会报告了这些信号[161][162]。记者们将此事当作轰动新闻,并急于得出结论,认为特斯拉听到了来自火星的信号[161]。他在1901年2月9日《科利尔周刊》的一篇文章《与行星对话》中进一步阐述了他所听到的信号,文章中他说,他当时没有立即意识到自己听到了“智能控制的信号”,这些信号可能来自火星、金星或其他行星[162]。有人推测,他可能在1899年7月拦截了古列尔莫·马可尼在欧洲进行的实验——马可尼可能在一次海军演示中传输了字母“S”(点/点/点),这正是特斯拉在科罗拉多提到的那三个脉冲[162]——或者是来自其他无线传输实验者的信号[163]

特斯拉与《世纪杂志》的编辑达成了协议,准备撰写一篇关于他发现的文章。杂志派了一名摄影师前往科罗拉多拍摄那里的工作情况。文章题为《增加人类能量的问题》,发表于1900年6月的杂志上。他在文章中解释了他设想的无线系统的优越性,但这篇文章更多的是一篇冗长的哲学论文,而非一篇易于理解的科学描述[164],并附有一些成为特斯拉和他在科罗拉多春季实验标志性图像的照片。
\subsubsection{沃登克利夫}
\begin{figure}[ht]
\centering
\includegraphics[width=6cm]{./figures/cd58cab06d6912cc.png}
\caption{特斯拉位于长岛的沃登克利夫工厂,拍摄于1904年。通过这个设施,特斯拉希望展示无线电能跨越大西洋的传输。} \label{fig_Tesla_13}
\end{figure}
特斯拉在纽约四处游说,试图为他认为可行的无线传输系统寻找投资者,他在当时自己住的华尔道夫-阿斯托里亚酒店的棕榈花园、玩家俱乐部和德尔莫尼科餐厅招待他们。1901年3月,他从J.P.摩根那里获得了15万美元(相当于今天的5,493,600美元),作为交换,他将无线专利的51\%股权交给了摩根,并开始规划在纽约肖赫姆建设沃登克利夫塔项目,这个设施位于离纽约市100英里(161公里)远的长岛北岸。

到1901年7月,特斯拉扩展了他的计划,打算建造一个更强大的发射器,超越马可尼的基于无线电的系统,而他认为马可尼的系统是抄袭了自己的。他找到了摩根,要求追加资金以建造更大的系统,但摩根拒绝提供进一步的资金支持。1901年12月,马可尼成功地从英格兰向纽芬兰传输了字母"S",在完成这种传输的竞赛中击败了特斯拉。在马可尼成功一个月后,特斯拉再次尝试让摩根支持一个更大的计划,通过控制“全世界的振动”来传输信息和电力。在接下来的五年里,特斯拉写了超过50封信给摩根,恳求并要求追加资金以完成沃登克利夫塔的建设。特斯拉继续进行这个项目直到1902年,还在九个月后塔楼达到了全高187英尺(57米)。1902年6月,特斯拉将他的实验室从休斯顿街搬到了沃登克利夫。

华尔街的投资者将资金投入马可尼的系统,一些媒体开始转向批评特斯拉的项目,声称这是一个骗局。该项目在1905年停滞不前,1906年,财务问题和其他事件可能导致特斯拉传记作者Marc J. Seifer怀疑特斯拉出现了精神崩溃。特斯拉将沃登克利夫的财产抵押以支付他在华尔道夫-阿斯托里亚酒店的债务,这笔债务最终达到了20,000美元(今天相当于608,400美元)。他在1915年失去了这块财产,1917年,新主人将塔楼拆除,将土地改为更具潜力的房地产资产。
\subsection{晚年}  
沃登克利夫项目停工后,特斯拉继续写信给摩根;在“伟人”去世后,特斯拉又写信给摩根的儿子杰克,试图为该项目争取进一步的资金支持。1906年,特斯拉在曼哈顿的百老汇165号开设了办公室,试图通过开发和营销自己的专利来筹集资金。他随后在1910年至1914年间在大都会人寿大厦设有办公室;在伍尔沃思大厦租了几个月的办公室,但因支付不起租金而搬走;之后又在1915年至1925年间租用了40街西8号的办公空间。搬到40街西8号后,他实际上已经破产。大多数专利已过期,他也在开发新发明时遇到了困难。
\subsubsection{无叶涡轮}
\begin{figure}[ht]
\centering
\includegraphics[width=6cm]{./figures/d33799599ec627b5.png}
\caption{特斯拉的无叶涡轮设计} \label{fig_Tesla_14}
\end{figure}
在他50岁生日那年,即1906年,特斯拉展示了一台200马力(150千瓦)、16,000转/分钟的无叶涡轮。在1910年至1911年期间,他的几台无叶涡轮发动机在纽约的Waterside发电站进行了100到5,000马力的测试。特斯拉曾与几家公司合作,其中包括在1919至1922年间与密尔沃基的Allis-Chalmers公司合作。他的大部分时间都用来与该公司首席工程师汉斯·达尔斯特兰德一起完善特斯拉涡轮,但由于工程上的困难,最终该涡轮未能成为一款实用设备。特斯拉确实将这一想法授权给了一家精密仪器公司,并且该技术被用于豪华汽车的车速表和其他仪器中。
\subsubsection{无线诉讼}  
第一次世界大战爆发时,英国切断了连接美国与德国的跨大西洋电报电缆,以控制两国之间的信息流通。他们还试图通过让美国马可尼公司起诉德国电台公司Telefunken侵犯专利,从而切断德国与美国之间的无线通信。[176]Telefunken公司聘请了物理学家乔纳森·泽内克和卡尔·费迪南德·布劳恩为其辩护,并在两年内以每月1,000美元的费用聘请特斯拉作为证人。案件陷入停滞,并在美国于1917年参战后变得无关紧要。[176][177]

1915年,特斯拉试图起诉马可尼公司侵犯他的无线调谐专利。马可尼最初的无线电专利在1897年获得美国授权,但他在1900年提交的关于无线电传输改进的专利申请曾多次被拒绝,直到1904年才最终获批,理由是该专利侵犯了其他现有专利,包括两项特斯拉于1897年申请的无线电力调谐专利。[140][178][179] 特斯拉的1915年诉讼没有进展,[180]但在一项相关案件中,马可尼公司试图起诉美国政府侵犯第一次世界大战期间的专利,1943年美国最高法院作出裁决,恢复了奥利弗·洛奇、约翰·斯通和特斯拉的先前专利。[181]法院声明,这一裁决不影响马可尼作为无线电传输首创者的声明,只是因为马可尼对某些专利改进的主张存在疑问,因此该公司不能声称侵犯这些专利。[140][182]
\subsubsection{诺贝尔奖谣言}  
1915年11月6日,路透社伦敦报道宣布,1915年诺贝尔物理学奖授予托马斯·爱迪生和尼古拉·特斯拉;然而,11月15日,路透社斯德哥尔摩的报道指出,那个年份的诺贝尔奖授予威廉·亨利·布拉格和劳伦斯·布拉格,“以表彰他们在利用X射线分析晶体结构方面的贡献”。[183][184][185] 当时有未经证实的谣言称,特斯拉或爱迪生曾拒绝接受奖项。[183]诺贝尔基金会表示:“任何关于某人因为表明拒绝奖项的意图而未获得诺贝尔奖的谣言都是荒谬的”;获奖者只能在宣布获奖后拒绝奖项。[183]

此后,特斯拉的传记作者提出了这样的说法:爱迪生和特斯拉原本是诺贝尔奖的获奖者,但由于彼此之间的敌意,最终两人都没有获得奖项;他们都试图削弱对方的成就及获得奖项的资格;两人都拒绝接受奖项,如果对方先获得的话;他们都拒绝分享奖项的可能性;甚至有说法认为,富有的爱迪生拒绝了奖项,以阻止特斯拉获得2万美元的奖金。[17][183]

在这些谣言传播后的几年里,特斯拉和爱迪生都未能获得诺贝尔奖(尽管爱迪生在1915年获得了38个可能的提名之一,特斯拉在1937年也获得了38个可能的提名之一)。[186]
\subsubsection{其他奖项、专利与想法}  
特斯拉在这段时间内获得了众多奖章和奖项,包括:
\begin{itemize}
\item 圣萨瓦勋章大十字勋章(塞尔维亚,1892年)
\item 埃利奥特·克雷森奖章(富兰克林学会,美国,1894年)[187]
\item 丹尼洛一世亲王勋章大十字勋章(黑山,1895年)[188]
\item 美国哲学学会会员(美国,1896年)[189]
\item 爱迪生奖章(美国电气与电子工程师协会,美国,1916年)[190]
\item 圣萨瓦勋章大十字勋章(南斯拉夫,1926年)[191]
\item 南斯拉夫王冠勋章(南斯拉夫,1931年)
\item 约翰·斯科特奖章(富兰克林学会与费城市议会,美国,1934年)[187]
\item 白鹰勋章(南斯拉夫,1936年)
\item 白狮勋章大十字勋章(捷克斯洛伐克,1937年)[192]
\item 巴黎大学奖章(法国巴黎,1937年)
\item 奥赫里德圣克莱门特大学奖章(保加利亚索非亚,1939年)
\end{itemize}
特斯拉曾尝试推销几种基于臭氧生成的设备。这些设备包括他于1900年创办的特斯拉臭氧公司,该公司销售一种基于特斯拉线圈的1896年专利设备,用于将臭氧泡过不同类型的油,以制作治疗凝胶。[193] 他还在几年后尝试开发该设备的变种,作为医院的空气净化器。[194]
\begin{figure}[ht]
\centering
\includegraphics[width=6cm]{./figures/2e4ee7bea10daefc.png}
\caption{1915年4月23日,电气工程师学会第二次宴会会议。特斯拉站在中央。} \label{fig_Tesla_15}
\end{figure}
特斯拉曾理论化,电流作用于大脑可以增强智力。1912年,他制定了“通过无意识地用电饱和学生,使其变得聪明”的计划,计划通过给教室的墙壁布线,“用高频震动的微小电波将整个教室饱和。特斯拉先生声称,整个房间将因此转变为一个具有健康促进和刺激作用的电磁场或‘浴室’。”[195] 这个计划至少在某种程度上得到了当时纽约市学校主管威廉·H·麦克斯韦尔的批准。[195]

第一次世界大战之前,特斯拉曾寻求海外投资者。战争爆发后,特斯拉失去了来自欧洲国家专利的资金支持。

在1917年8月的《电气实验者》杂志中,特斯拉假设可以利用“电光线”反射来定位潜艇,这种“电光线”具有“巨大的频率”,并且信号可以在荧光屏上显示出来(这一系统与现代雷达有一些表面上的相似之处)。[196] 特斯拉错误地认为高频无线电波能够穿透水面。[197] 1953年,帮助开发法国首个雷达系统的埃米尔·吉拉尔多指出,特斯拉关于需要非常强的高频信号的基本猜测是正确的。吉拉尔多说,“(特斯拉)是在预言或做梦,因为他当时并没有任何手段来实现这些,但我们必须补充说,如果他是在做梦,那么至少他做对了梦”。[198]

在1928年,特斯拉获得了美国专利1,655,114,专利内容为一种能够垂直起降(VTOL)的双翼飞机设计,该飞机通过操纵升降装置,在飞行过程中逐渐倾斜,直到像传统飞机一样飞行。[199] 这个不切实际的设计是特斯拉认为售价不到1000美元的产品。[200][201]

特斯拉在350 Madison Ave还设有办公室[202],但到1928年,他已不再拥有实验室或资金。[201]
\subsubsection{生活状况}  
特斯拉自1900年起住在纽约的华尔道夫-阿斯托里亚酒店,并积累了大量账单。[203] 他于1922年搬到圣雷吉斯酒店,并从那时起开始了每隔几年换一个酒店的生活模式,且每次都留下未支付的账单。[204][205]  

特斯拉每天都会步行去公园喂鸽子。他最初在酒店房间的窗户上喂鸽子,还照顾受伤的鸟儿直到它们恢复健康。[205][206][207] 他曾说,每天都有一只受伤的白鸽来看他。他花费了超过2000美元(相当于2023年的36,410美元)照料这只鸟,其中包括他自己制作的一个装置,以便让这只鸽子在骨折的翅膀和腿恢复期间能够舒适地休息。[208] 特斯拉曾表示:

“我喂了无数的鸽子,已经很多年了。但是有一只,漂亮的白鸽,翅膀上有浅灰色的尖端;那只与众不同。它是一只母鸽。我只需一呼唤,它就会飞到我身边。我爱那只鸽子,就像一个男人爱女人一样,它也爱我。只要我有它,我的生活就有了意义。”[209]

特斯拉未付的账单,以及有关鸽子造成的混乱的投诉,导致他于1923年被圣雷吉斯酒店驱逐。他还被迫离开了1930年的宾夕法尼亚酒店和1934年的州长克林顿酒店。[205] 一度,他还住在了玛格丽酒店。[210]

特斯拉于1934年搬进了纽约客酒店。此时,西屋电气制造公司开始每月支付他125美元(相当于2023年的2,850美元),并承担他的房租。关于这一安排的具体经过有不同的说法。几种来源称,西屋公司担心或可能被警告,关于他们曾经的明星发明家如今生活困窘的情况可能会带来不好的公众舆论。[211][212][213][214] 这种付款被形容为一种“咨询费”,旨在规避特斯拉不愿接受慈善援助的心理障碍。特斯拉传记作家马克·塞弗(Marc Seifer)将西屋的支付描述为一种“未说明的和解”形式。[213]
\subsubsection{生日新闻发布会}
\begin{figure}[ht]
\centering
\includegraphics[width=6cm]{./figures/6b3548632873225d.png}
\caption{特斯拉出现在《时代》杂志上,纪念他的75岁生日。} \label{fig_Tesla_19}
\end{figure}
1931年,一位年轻记者肯尼斯·M·斯韦齐(Kenneth M. Swezey),特斯拉的朋友,组织了一个庆祝特斯拉75岁生日的活动。[215] 特斯拉收到了来自科学和工程界人物的祝贺,例如阿尔伯特·爱因斯坦,[216] 他还出现在了《时代》杂志的封面上。[217] 封面标题写道‘全世界是他的发电厂’,表彰了他在电力生产方面的贡献。派对举办得非常成功,以至于特斯拉将其变成了每年的活动,届时他会提供大量食物和饮品—包括他自己创作的菜肴。他邀请媒体前来参观他的发明,并听取他关于过去经历、当前事件的看法以及有时令人困惑的言论。

在1932年的派对上,特斯拉声称他发明了一种能够利用宇宙射线驱动的电动机。[219] 1933年,77岁的特斯拉在一次活动中告诉记者,经过35年的努力,他即将提供一种新型能源的证据。他声称,这是一种与爱因斯坦物理学‘强烈对立’的能源理论,可以通过一种廉价且能使用500年的装置来获取。他还告诉记者,他正在研究一种方法,用于传输个性化的私人无线电波长,并且在金属学方面有突破性进展,同时还在开发一种拍摄视网膜以记录思想的方法。[220]
\begin{figure}[ht]
\centering
\includegraphics[width=6cm]{./figures/8ec71714665f48ce.png}
\caption{报纸上对特斯拉在1933年生日派对上描述的‘思想相机’的报道。} \label{fig_Tesla_20}
\end{figure}
在1934年的一次聚会上,特斯拉告诉记者,他设计了一种超级武器,声称可以结束所有战争。[221][222] 他称之为“电力武器”,但通常被称为他的死亡射线。[223] 1940年,《纽约时报》报道该射线的射程为250英里(400公里),预计开发成本为200万美元(相当于2023年的4350万美元)。[224] 特斯拉将其描述为一种防御武器,应该沿着国家边界部署,用于对抗攻击的地面步兵或飞机。特斯拉在生前从未透露过该武器的详细工作原理,但在1984年,这些资料出现在贝尔格莱德的尼古拉·特斯拉博物馆档案中。[225] 该论文《通过自然介质投射集中的非扩散能量的新艺术》描述了一种开放式真空管,带有气体喷嘴封口,允许粒子排出,充电方式是将钨或汞的弹头充电到数百万伏特,并通过静电排斥将其定向成流。[219][226] 特斯拉曾试图吸引美国战争部,[227] 英国、苏联和南斯拉夫对该设备的兴趣。[228]

在1935年,特斯拉在他的79岁生日聚会上讨论了许多话题。他声称自己在1896年发现了宇宙射线,并发明了一种通过感应产生直流电的方法,还发表了许多关于他机械振荡器的言论。[229] 他描述了这一设备(他预期在两年内赚取1亿美元),并告诉记者,他的振荡器版本曾在1898年导致他位于46号东休斯顿街的实验室以及下曼哈顿邻近街道发生地震。[229] 他接着告诉记者,他的振荡器可以用5磅(2.3公斤)的空气压力摧毁帝国大厦。[230] 他还提议利用他的振荡器将振动传入地下。他声称,这种方法可以用于任何距离,并可用于通讯或寻找地下矿藏,这一技术他称为“遥感动力学”(telegeodynamics)。[134]

1937年,在纽约新大楼酒店的盛大舞厅活动上,特斯拉获得了捷克斯洛伐克大使授予的白狮勋章,并从南斯拉夫大使处获得一枚奖章。关于死亡射线的问题,特斯拉表示:“但这不是一个实验……我已经建造、演示并使用了它。只需一点时间,我就能把它交给全世界。”[219]
\subsection{死亡}
\begin{figure}[ht]
\centering
\includegraphics[width=6cm]{./figures/839dd7a25f25648c.png}
\caption{特斯拉去世的地方,纽约新大酒店的3327号房间} \label{fig_Tesla_22}
\end{figure}
1937年秋天,81岁的特斯拉在一个午夜后离开纽约新大楼酒店,像往常一样前往大教堂和图书馆喂鸽子。在酒店几条街外过马路时,特斯拉被一辆行驶中的出租车撞到,摔倒在地。事故中,他的背部严重扭伤,三根肋骨骨折。事故的具体伤势从未完全得知;特斯拉拒绝看医生,这是他几乎一生的习惯,他也没有完全恢复过来。[231][232]

1943年1月7日,86岁的特斯拉在纽约新大楼酒店的3327号房间里孤独去世。两天前,特斯拉在门上挂了“不打扰”的标志,但女仆爱丽丝·莫纳汉仍进入了特斯拉的房间,发现了他的尸体。助理验尸官H.W.温布利检查了尸体,判定死亡原因是冠状动脉血栓(心脏病的一种)。
\begin{figure}[ht]
\centering
\includegraphics[width=6cm]{./figures/1e35e779d6b13e87.png}
\caption{纪念 plaque,纽约客酒店} \label{fig_Tesla_21}
\end{figure}
两天后,联邦调查局命令外侨财产保管局没收特斯拉的财物。麻省理工学院的教授、著名电气工程师约翰·G·特朗普被召来分析特斯拉的物品。在为期三天的调查后,特朗普的报告得出结论,认为没有任何东西会在敌对势力手中构成危险,报告中写道:

“他(特斯拉)在过去至少15年的思想和努力,主要是以投机性、哲学性和某种程度上具有宣传性质的方式,常常关注电力的生产和无线传输;但并未包含实现这些结果的新颖、可靠、可行的原理或方法。” [233]

在一个据说包含特斯拉“死亡射线”部件的盒子里,特朗普发现了一只45年的多档电阻箱。[234]

1943年1月10日,纽约市市长费奥雷洛·拉瓜迪亚在WNYC广播电台现场直播朗读了由斯洛文尼亚裔美国作家路易斯·亚达米奇所写的悼词,背景音乐中播放了小提琴曲《圣母颂》和《那遥远的地方》。1月12日,约有2000人参加了在曼哈顿圣约翰神殿举行的特斯拉国葬。葬礼结束后,特斯拉的遗体被送往纽约州阿兹利的费恩克里夫公墓,之后进行了火化。次日,纽约市的三一教堂(现今的圣萨瓦塞尔维亚东正教大教堂)举行了由著名神父主持的第二场追悼仪式。
\subsection{个人生活与性格}
\begin{figure}[ht]
\centering
\includegraphics[width=6cm]{./figures/b19c6acf5d5cac85.png}
\caption{特斯拉,约1896年} \label{fig_Tesla_23}
\end{figure}
特斯拉终生未婚,他曾解释过,保持贞洁对他的科学能力非常有帮助。[235] 在1924年8月10日接受《加尔维斯顿日报》采访时,他表示:“如今,我崇敬的温柔女士几乎已经消失。取而代之的是那种认为自己成功的关键在于尽可能让自己像男人一样的女人——在穿着、声音和行为上……”[210] 尽管他在后来曾告诉记者,有时他觉得没有结婚,可能对自己的工作做出了过大的牺牲,[208] 特斯拉选择了永远不去追求或参与任何已知的关系,而是将所有的刺激都寄托在他的工作中。

特斯拉性格孤僻,倾向于把自己与工作隔离开来。[136][236][237] 然而,当他参与社交活动时,许多人对特斯拉评价极高,赞赏之词不绝于耳。罗伯特·安德伍德·约翰逊形容他拥有“卓越的温柔、真诚、谦逊、优雅、慷慨和力量”。[238] 他的秘书多萝西·斯凯里特写道:“他的和蔼笑容和高贵气质总是体现出他那种深深植根于灵魂中的绅士风度。”[239] 特斯拉的朋友朱利安·霍桑写道:“很少有人能遇到既是科学家或工程师,又是诗人、哲学家、音乐鉴赏家、语言学家,以及美食美酒的鉴赏家的。”[240]

特斯拉是弗朗西斯·马里昂·克劳福德、罗伯特·安德伍德·约翰逊[241]、斯坦福·怀特[242]、弗里茨·洛文斯坦、乔治·舍夫、和肯尼斯·斯威泽的好友。[243][244][245] 在中年时,特斯拉与马克·吐温成为了亲密的朋友;他们经常一起在他的实验室和其他地方度过时光。[241] 吐温特别形容特斯拉的感应电动机发明是“自电话以来最有价值的专利”。[246] 在1896年由女演员萨拉·伯恩哈特举办的一个聚会上,特斯拉遇到了印度教僧侣斯瓦米·维韦卡南达。维韦卡南达后来写道,特斯拉曾表示,他能够通过数学证明物质与能量之间的关系,这也是维韦卡南达希望为印度教宇宙学提供科学基础的内容。[247][248] 与斯瓦米·维韦卡南达的会面激发了特斯拉对东方科学的兴趣,这促使他研究印度教和吠陀哲学多年。[249] 特斯拉后来写了一篇名为《人类最大的成就》的文章,使用梵语词汇“akasha”(虚空)和“prana”(生命能量)来描述物质与能量之间的关系。[250][251] 在1920年代末,特斯拉与乔治·西尔维斯特·维雷克建立了友谊,维雷克是诗人、作家、神秘学家,后来成为纳粹宣传员。特斯拉偶尔会参加维雷克和他妻子举办的晚宴。[252][253]

特斯拉有时脾气较为严厉,并且公开表达了对肥胖人士的厌恶,比如他曾因一位秘书的体重而解雇她。[254] 他对衣着非常挑剔;曾多次指示下属回家换衣服。[235] 当托马斯·爱迪生于1931年去世时,特斯拉在《纽约时报》对爱迪生生平的广泛报道中贡献了唯一的负面评价:

“他没有任何爱好,不关心任何形式的娱乐,完全无视最基本的卫生规则……他的方法极其低效,因为为了取得任何成果,必须覆盖巨大的范围,除非盲目机遇介入,起初,我几乎是一个悲伤的见证者,知道只需一点理论和计算,就能节省90\%的劳动。但他对书本知识和数学知识有着真正的蔑视,完全依赖于自己的发明直觉和实用的美国常识。”[255][256]

特斯拉在晚年成为了一名素食主义者,仅以牛奶、面包、蜂蜜和蔬菜汁为食。[222][257]
\subsection{观点与信仰}
\subsubsection{关于实验物理与理论物理}  
\begin{figure}[ht]
\centering
\includegraphics[width=6cm]{./figures/fc474e202bfeb760.png}
\caption{特斯拉,约1885年} \label{fig_Tesla_24}
\end{figure}
特斯拉不同意原子由更小的亚原子粒子组成的理论,认为不存在电子产生电荷的现象。他认为,如果电子确实存在,它们是某种第四种物质状态或“亚原子”,只能在实验真空中存在,并且与电力无关。[258][259] 特斯拉相信原子是不可改变的——它们不能改变状态或以任何方式被分裂。他是19世纪全能以太理论的支持者,认为以太传递电能。[260]

特斯拉通常对物质转化为能量的理论持反对态度。[261] 他也批评爱因斯坦的相对论,表示:

“我认为空间不能弯曲,原因很简单,因为它没有任何性质。也可以说,上帝是有性质的。但他没有,只有属性,而这些是我们自己创造的。只有在处理填充空间的物质时,我们才能谈论性质。说在大物体存在的情况下,空间变得弯曲,相当于说某物可以作用于无物。我个人拒绝认同这种观点。”[262]

1935年,他将相对论形容为“一个披着紫色袍子的乞丐,愚昧的人们把他当作国王”,并表示他的实验测量到从参宿四(Arcturus)发出的宇宙射线速度是光速的五十倍。[263]

特斯拉声称他已经发展出自己关于物质和能量的物理原理,他从1892年开始着手研究。[261] 1937年,81岁的特斯拉在一封信中声称,他已经完成了一种“动态引力理论”,并表示该理论“将结束无意义的猜测和错误的观念,比如弯曲空间的理论”。他指出,该理论“已经在所有细节上得到了完善”,并希望很快将其公之于众。[264] 但他对该理论的进一步阐述从未出现在他的著作中。[265]
\subsubsection{关于社会}
特斯拉被传记作者广泛认为具有人文主义的哲学观点。[266][267] 然而,这并不妨碍特斯拉像他那个时代的许多人一样,成为强制性选择性繁殖版本优生学的支持者。

特斯拉表达了这样一种信念:人类的“怜悯”已经开始干扰自然界“无情的运作”。尽管他的论证并未依赖于“优越种族”或某个人天生优于另一个人的概念,他还是支持优生学。在1937年的一次采访中,他说道:

“……人类的新怜悯之情开始干扰自然界无情的运作。与我们文明和种族观念相符的唯一方法就是通过绝育和故意引导交配本能来防止不适者繁殖……优生学家们的意见趋势是,我们必须使婚姻变得更加困难。显然,任何不是理想父母的人都不应该被允许生育后代。一百年后,正常人不会再想到与优生学上不适合的人结婚,就像不会与惯犯结婚一样。”[268]

1926年,特斯拉评论了女性社会从属地位的弊端以及女性争取性别平等的斗争,并指出人类的未来将由“蜂王”来主导。他相信,女性将在未来成为主导性别。[269]

特斯拉在一篇名为《科学与发现是引领战争圆满结束的伟大力量》(1914年12月20日)中,对第一次世界大战后环境相关问题作出了预测。[270] 特斯拉认为,国际联盟并不是解决当时问题的良方。[30]
\subsubsection{关于宗教}  
特斯拉在正教基督教家庭中长大。后来,他不再认为自己是“按传统意义上的信徒”,他表示反对宗教狂热,并说:“佛教和基督教是最伟大的宗教,无论是在信徒人数上还是在重要性上。”[271] 他还说:“对我来说,宇宙只是一个伟大的机器,它从未产生过,也永远不会结束。”以及“我们所说的‘灵魂’或‘精神’,不过是身体功能的总和。当这种功能停止时,‘灵魂’或‘精神’也随之停止。”[271]
\subsection{文学作品}  
特斯拉撰写了许多书籍和文章,发表在杂志和期刊上。[272] 他的书籍包括《我的发明:尼古拉·特斯拉自传》,由本·约翰斯顿于1983年编纂和编辑,内容来自特斯拉1919年发表的一系列杂志文章,后于1977年再版;《尼古拉·特斯拉的奇妙发明》(1993年),由大卫·哈切尔·奇尔德雷斯编纂和编辑;以及《特斯拉文献》。

许多特斯拉的著作可以在线免费阅读,[273] 包括1900年发表在《世纪杂志》上的文章《人类能量增长问题》[274],以及他在《尼古拉·特斯拉的发明、研究和著作》一书中发表的文章《高电压高频交流电的实验》[275][276]。
\subsection{遗产与荣誉}
\begin{figure}[ht]
\centering
\includegraphics[width=6cm]{./figures/e5a08f7d2f5241e6.png}
\caption{镀金骨灰坛,内含特斯拉的骨灰,呈他最喜欢的几何形状——球体(尼古拉·特斯拉博物馆,贝尔格莱德)} \label{fig_Tesla_25}
\end{figure}
1952年,在特斯拉侄子、影响力巨大的南斯拉夫政治家萨瓦·科萨诺维奇(Sava Kosanović)施加压力后,特斯拉的全部遗产被运送到贝尔格莱德,装在80个标有“N.T.”的箱子里。1957年,科萨诺维奇的秘书夏洛特·穆扎尔将特斯拉的骨灰从美国运送到贝尔格莱德。这些骨灰被展示在一座金镀的球体内,置于尼古拉·特斯拉博物馆的一个大理石基座上。[277] 尼古拉·特斯拉档案馆包含超过160,000份原始文件,并被纳入联合国教科文组织世界记忆项目。[278][279]

特斯拉获得了约300项全球专利。[280] 一些特斯拉的专利未被记录在案,且有多个来源发现了一些曾隐藏在专利档案中的专利。已知至少有278项专利[280] 在26个国家颁发给特斯拉。特斯拉的许多专利在美国、英国和加拿大获得,但还有许多其他专利在全球各国批准。[281] 特斯拉开发的许多发明并未申请专利保护。
\subsection{另见}  
\begin{itemize}
\item 大气电学 – 行星大气中的电力  
\item 迈克尔·法拉第 – 英国化学家和物理学家(1791–1867)  
\item 查尔斯·普罗修斯·斯坦梅茨 – 美国数学家和电气工程师(1865–1923)  
\item 陆电流 – 地壳中的自然电流
\end{itemize}
\subsection{注释}  
\subsubsection{脚注}\\  
a.特斯拉没有提到是哪位教授,但一些来源推测是马丁·塞库利奇(Martin Sekulić)。[23][24]\\  
b.特斯拉的同时代人回忆说,之前一次,机器厂经理巴切洛尔曾不愿意给特斯拉每周7美元的加薪。[55]\\  
c.此记载来自特斯拉在1938年发给国家移民福利研究所的一封信,以此庆祝他获得奖项。[63]\\  
d.特斯拉自己的实验让他错误地认为赫兹误将一种导电形式识别为新的电磁辐射形式,这一错误假设特斯拉维持了几十年。[140][141]
