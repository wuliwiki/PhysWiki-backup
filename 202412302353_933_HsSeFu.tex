% 数列(高中)
% keys 高中|数列的概念|数列的函数特性
% license Usr
% type Tutor

\begin{issues}
\issueDraft
\end{issues}
\pentry{函数\nref{nod_functi}}{nod_7191}

在之前的学习中,已经介绍过函数的概念。函数是一种描述变量之间关系的数学工具,高中阶段接触到的函数,其定义域通常是连续的,比如实数集 $\mathbb{R}$ 或区间,因此函数的图像往往是一条连续的曲线。然而,数学世界中并非所有的关系都具有连续性。一些仅定义在自然数集或其子集上的特殊函数,被称为\textbf{数列}。

数列是非常古老的数学内容,在某些方面,古代数学家们已经做了很深入的研究。最初,数列是人们将观测到的对象按顺序排列而形成的一种表示方式,那时甚至还没有函数的概念。尽管数列作为独立的数学领域有着丰富的研究内容和独特的性质,但从函数的视角观察数列,会发现它实际上继承了函数的许多特点。

\subsection{数列}

在小学阶段,常见的题目之一是类似这样的填空题:$1,3,7,(),31$。题目的目的是引导学生通过观察这些数字的排列规律,推断出空缺处的数字\footnote{然而,尽管这类题目通常有“标准答案”,但实际上,填写任何数字,都能够给出合理的规律。}。显然,在这里出题者希望隐含的规律是 $2^n - 1$,其中 $n$ 表示第几个出现的数。

另一个简单的例子是将一个月中的每一天按照星期的数字标记。假设一个月的第一天是星期日,那么接下来的数字标记会形成一个循环的数列:$7, 1, 2, 3, 4, 5, 6, 7, 1, 2, 3, \dots$。它的规律可以表示为 $n-1 \mod 7$\footnote{$\mod$为取余函数,即$a \mod b$ 为 $a$ 除以 $b$ 得到的余数。},其中结果为 $0$ 时代表星期日,非零时则为对应的星期数。

最初研究数列的人,或许只是简单地将一些有关系的数字排列成一列,关注数字之间的直接关系,例如相邻数字的差值、比值或其他变化模式。然而,随着数学的发展,研究数列的视角逐渐发生了转变,人们引入函数的视角,将数字的位置视为自变量。例如,上述表达式若被看作函数,其中的 $n$ 就是表示位置的自变量。这种视角的改变突出了位置的重要性。通过这种方式,数列的研究不再局限于数字之间的关系,而是扩展到数字与其位置的对应关系,逐渐将重点转向提炼和揭示数字排列的内在规律,并为数列的表达和分析提供了更加系统化的工具。

\begin{definition}{数列}
将一些数按照一定的次序排列成一列,称为\textbf{数列(sequence)}或\textbf{序列},通常记作:
\begin{equation}
a_1, a_2, a_3, \cdots, a_n, \cdots
\end{equation}
简记为数列 ${a_n}$,字母$a$可以替换为其他字母。

其中,$n$ 是自然数\footnote{在数学中,自然数通常从 $0$ 开始,但在高中阶段,一般要求数列从 $1$ 开始。当然也有其他教材会从 $0$ 开始。},数列中的每一个数被称为该数列的\textbf{项}。第一项 $a_1$ 称为\textbf{首项},而 $a_n$ 表示数列的第 $n$ 项。
\end{definition}



数列通常用符号 
	•	示例:$a_1, a_2, a_3, \dots$ 表示数列的前几项。
;$a_n$ 是数列的第 $n$ 项,也叫数列的\textbf{通项}
数列的分类
	1.	有限数列:只包含有限项的数列,例如 $1, 2, 3, 4$。
	2.	无限数列:项数无限多,例如 $1, 2, 3, 4, \dots$。
项数有限的数列,称为\textbf{有限数列};项数无限的数列,称为\textbf{无穷数列}。

如果数列 $\begin{Bmatrix} a_n \end{Bmatrix}$ 的第 $n$ 项 $a_n$ 与 $n$ 之间的函数关系可以用一个式子表示成 $a_n = f(n)$,那么这个式子就叫作这个数列的\textbf{通项公式},数列的通项公式就是相应函数的解析式。

\textsl{注意:不是所有数列都能写出通项公式。}

\subsection{数列和}

无穷数列的数列和也称为级数,事实上


数列的基本性质和特点

数列的规律性
每个数列都有独特的规律,例如:
等差数列的公差:相邻两项之差为 $d$。
等比数列的公比:相邻两项之比为 $r$。

常见问题
如何根据数列规律求通项公式?
如何计算数列的前 $n$ 项和?
递推公式是数列的另一种表示方法,通过已知项推导出下一项。例如:
斐波那契数列的递推公式:$a_n = a_{n-1} + a_{n-2}$。

数列是按照某种规律排列的一列数字,具有明显的规律性。通项公式和递推公式是数列问题的重要工具。
\subsection{函数特性}

\subsection{增减性}
一般地,一个数列 $\begin{Bmatrix} a_n \end{Bmatrix}$,如果从第2项起,每一项都大于前一项,即 $a_{n+1}>a_n$,那么这个数列叫作\textbf{递增数列}。

如果从第2项起,每一项都小于前一项,即 $a_{n+1}<a_n$,那么这个数列叫作\textbf{递减数列}。

如果数列 $\begin{Bmatrix} a_n \end{Bmatrix}$ 的各项都相等,那么这个数列叫作\textbf{常数列}。

如果数列 $\begin{Bmatrix} a_n \end{Bmatrix}$ 从第2项起,有些项大于它的前一项,有些项小于它的前一项,这样的数列叫\textbf{摆动数列}。
循环性
这个数列具有周期性,每隔 7 天就会重复一次,是一种典型的循环数列(cyclic sequence)。

在高中阶段,数列的研究相对狭隘,主要集中在古代早期就被发现的等差数列和等比数列。然而,建议读者不要因此局限视野,数列的概念在更高层次的数学学习中,将与许多重要概念密切相关,例如实数的构建、级数以及分析学中的广泛应用。因此,在学习数列时,需要多关注其规律和研究方法,而不仅仅停留于记忆一些常见公式。理解这些公式背后的原理尤为重要,这不仅有助于掌握数列的变化规律,也能培养更深层次的数学思维,扎实当前的知识体系,为未来更复杂的数学学习奠定坚实的基础。