% 约翰内斯·开普勒(综述)
% license CCBYSA3
% type Wiki

本文根据 CC-BY-SA 协议转载翻译自维基百科\href{https://en.wikipedia.org/wiki/Johannes_Kepler}{相关文章}。

\begin{figure}[ht]
\centering
\includegraphics[width=6cm]{./figures/cc1aaaf12433e376.png}
\caption{由奥古斯特·科勒(August Köhler)于约1910年绘制的肖像,基于1627年的原作。} \label{fig_KPL1_2}
\end{figure}
约翰内斯·开普勒(Johannes Kepler,/ˈkɛplər/;德语:[joˈhanəs ˈkɛplɐ, -nɛs -] ⓘ;1571年12月27日–1630年11月15日)是德国天文学家、数学家、占星家、自然哲学家以及音乐作家。他是17世纪科学革命的关键人物,以行星运动定律最为人知,并且以《新天文学》(Astronomia nova)、《世界和谐论》(Harmonice Mundi)和《哥白尼天文学概要》(Epitome Astronomiae Copernicanae)等著作影响了以艾萨克·牛顿为代表的科学家,为牛顿的万有引力理论提供了基础之一。开普勒的工作具有多样性和深远影响,使他成为现代天文学、科学方法、自然科学和现代科学的奠基人之一。他被誉为“科幻小说之父”,因为他的小说《梦境》(Somnium)。 

开普勒曾是格拉茨一所神学院的数学教师,在那里他成为了汉斯·乌尔里希·冯·埃根贝格(Prince Hans Ulrich von Eggenberg)王子的合作者。后来,他成为天文学家第谷·布拉赫(Tycho Brahe)在布拉格的助手,并最终成为神圣罗马帝国皇帝鲁道夫二世及其继任者马提亚斯和斐迪南二世的皇家数学家。他还曾在林茨教授数学,并且是沃尔斯坦将军的顾问。此外,开普勒在光学领域做出了基础性贡献,被誉为现代光学之父,尤其以《光学天文学》为代表。他还发明了改进版的折射望远镜——开普勒望远镜,成为现代折射望远镜的基础,同时改进了伽利略·伽利莱的望远镜设计,伽利略在他的著作中提到了开普勒的发现。

开普勒生活在一个天文学和占星学没有明确界限的时代,但天文学(作为自由艺术中的一门数学分支)和物理学(作为自然哲学的一门分支)之间却有着明显的区分。开普勒还将宗教论证和推理融入到他的工作中,受宗教信仰的激励,他认为上帝按照可以通过理性之光理解的智能计划创造了这个世界。开普勒将他的新天文学描述为“天体物理学”,作为“对亚里士多德《形而上学》的探索”,并作为“对亚里士多德《天论》的补充”,通过将天文学视为普遍数学物理学的一部分,开普勒彻底改造了古代的物理宇宙学传统。
\subsection{早期生活}  
\subsubsection{童年(1571年–1590年)}
开普勒的出生地,魏尔德施塔特  
开普勒于1571年12月27日出生在魏尔德施塔特的自由帝国城市(现为德国巴登-符腾堡州斯图加特地区的一部分)。他的祖父塞巴尔德·开普勒曾是该市的市长。到约翰内斯出生时,开普勒家族的财富已开始衰退。他的父亲海因里希·开普勒以雇佣兵的身份维持生计,在约翰内斯五岁时离开了家人,据信他在荷兰的八十年战争中去世。他的母亲凯瑟琳娜·古尔登曼是位酒馆老板的女儿,也是一个治疗师和草药师。约翰内斯有六个兄弟姐妹,其中两个兄弟和一个姐妹活到了成年。由于早产,他自称小时候身体虚弱且多病。然而,他常常在祖父的酒馆里给旅客留下深刻的印象,展现出他非凡的数学才能。[22]

他在很小的时候就接触了天文学,并发展出了对它的浓厚兴趣,这份热情贯穿了他的一生。六岁时,他观察到了1577年的大彗星,并写道他“被母亲带到一个高处去观看它。”[23] 1580年,九岁时,他又观察到了一个天文现象——月全食,他记得“被叫到户外”去观看,月亮“显得非常红。”[23] 然而,童年的天花让他留下了视力虚弱和双手残疾的问题,这限制了他在天文观测方面的能力。[24]
\begin{figure}[ht]
\centering
\includegraphics[width=6cm]{./figures/73a80a0cf2cfd6a7.png}
\caption{开普勒的出生地,位于韦尔德施塔特} \label{fig_KPL1_1}
\end{figure}
1589年,在经历了文法学校、拉丁学校和毛尔布龙神学院的学习后,开普勒进入了图宾根大学的图宾根神学院。在那里,他在维图斯·穆勒的指导下学习哲学,[25] 并在雅各布·赫尔布兰德(菲利普·梅兰希顿在维滕贝格的学生)的指导下学习神学。赫尔布兰德也曾教过迈克尔·梅斯特林,直到1590年他成为图宾根大学的校长。[26] 他证明自己是一个出色的数学家,并因精湛的占星术而赢得了声誉,为同学们制作星座图。在图宾根大学担任数学教授的迈克尔·梅斯特林(1583年至1631年)指导下,他学习了托勒密的行星运动体系和哥白尼的行星运动体系。那时,他成为了哥白尼主义者。在一次学生辩论中,他从理论和神学角度捍卫了日心说,主张太阳是宇宙中运动力的主要来源。[27] 尽管他渴望成为路德宗教会的牧师,但由于其信仰与《和协议》相悖,他未能获得圣职。[28] 在学业接近尾声时,开普勒被推荐到格拉茨的基督教学校担任数学和天文学教师。他在1594年4月接受了这个职位,当时他22岁。[29]
\subsubsection{格拉茨(1594–1600年)}
\begin{figure}[ht]
\centering
\includegraphics[width=8cm]{./figures/0f766dbc7312ae9f.png}
\caption{小时候,开普勒目睹了1577年的大彗星,这一事件引起了全欧洲天文学家的关注。} \label{fig_KPL1_3}
\end{figure}
在结束他在图宾根的学业之前,开普勒接受了在格拉茨(现位于奥地利斯蒂里亚州)的新教学校教数学的工作,接替乔治·斯塔迪乌斯的职位(1594–1600年期间)。在这段时间里,他发布了许多官方的日历和预言,这些增强了他作为占星家的声誉。尽管开普勒对占星术有着复杂的感情,并且贬低了许多占星家的传统做法,他仍然深信宇宙与个体之间存在某种联系。他最终将自己在学生时代的一些想法写成了《宇宙神秘学》(1596年),这本书在他到达格拉茨一年多后出版。
\begin{figure}[ht]
\centering
\includegraphics[width=8cm]{./figures/1c6219a05abae40a.png}
\caption{约1600年,开普勒和妻子的油画铜板肖像。} \label{fig_KPL1_4}
\end{figure}
1595年12月,开普勒遇到了巴巴拉·穆勒(Barbara Müller),一位23岁的寡妇(已经结过两次婚)并且有一个年轻的女儿,名叫瑞吉娜·洛伦茨(Regina Lorenz),他开始追求她。穆勒是她已故丈夫们遗产的继承人,也是一个成功的磨坊主的女儿。穆勒的父亲约布斯最初反对这桩婚事。尽管开普勒继承了祖父的贵族身份,但开普勒的贫困使得他成为一个不被接受的婚配对象。在开普勒完成《宇宙神秘学》的工作后,约布斯最终同意了婚事,但在开普勒去处理出版细节期间,这段婚约几乎破裂。然而,帮助撮合这桩婚姻的 protestant 官员们施加了压力,迫使穆勒家族履行他们的承诺。巴巴拉和约翰内斯于1597年4月27日结婚。

在婚后的头几年,开普勒夫妇有了两个孩子(海因里希和苏珊娜),但两人都在婴儿时期去世。1602年,他们有了一个女儿(苏珊娜);1604年,生了一个儿子(弗里德里希);1607年,又有了另一个儿子(路德维希)。
\begin{figure}[ht]
\centering
\includegraphics[width=8cm]{./figures/367e4ebb9d714791.png}
\caption{开普勒与芭芭拉·穆勒在格拉茨附近的戈森多夫的住所(1597–1599)。} \label{fig_KPL1_5}
\end{figure}
\subsubsection{其他研究}  
在《神秘宇宙》出版后,并在格拉茨学校检查员的支持下,开普勒开始了一项雄心勃勃的计划,旨在扩展和详细阐述他的工作。他计划撰写四本附加的书籍:一本关于宇宙的静态方面(太阳和固定星星);一本关于行星及其运动;一本关于行星的物理性质和地理特征的形成(尤其关注地球);以及一本关于天体对地球的影响,涉及大气光学、气象学和占星学。

他还征求了许多他曾向其发送《神秘宇宙》的天文学家的意见,其中包括雷马鲁斯·乌尔苏斯(尼古劳斯·雷默斯·贝尔)——鲁道夫二世的皇帝数学家,并且是第谷·布拉赫的劲敌。乌尔苏斯并没有直接回复,但重新出版了开普勒的恭维信,以追求他与第谷关于(现在称为)第谷学说的优先权争议。尽管有这一污点,第谷也开始与开普勒通信,首先是对开普勒学说的严厉但合理的批评;在众多反对意见中,第谷对开普勒从哥白尼那里获取的不准确数据提出了异议。通过信件,第谷和开普勒讨论了许多天文学问题,重点讨论了月球现象和哥白尼理论(特别是其神学可行性)。但由于缺乏第谷天文台的准确数据,开普勒无法解决许多这些问题。

于是,他将注意力转向了年代学和“和谐”,即音乐、数学和物理世界之间的数字关系及其占星学后果。通过假设地球拥有灵魂(这一特性他后来用来解释太阳如何引发行星运动),他建立了一个将占星学方面和天文距离与天气和其他地球现象联系起来的推测性系统。然而,到1599年,他再次感到由于数据不准确,自己的工作受限——与此同时,日益严重的宗教紧张局势也威胁着他在格拉茨的继续工作。那年12月,第谷邀请开普勒前往布拉格;1600年1月1日(在他收到邀请之前),开普勒便启程,希望第谷的赞助能解决他在哲学上的问题,以及社会和经济上的困境。
\subsection{科学生涯}  
\subsubsection{布拉格(1600–1612)}
\begin{figure}[ht]
\centering
\includegraphics[width=6cm]{./figures/e593a059939abf69.png}
\caption{第谷·布拉赫} \label{fig_KPL1_6}
\end{figure}
1600年2月4日,开普勒在本纳特基·纳德·吉泽罗(距布拉格35公里)与第谷·布拉赫以及他的助手弗朗茨·滕格纳格尔和隆戈蒙坦努斯会面,这里是第谷新天文台的建设地点。在接下来的两个月里,开普勒作为客人住在第谷家中,分析了第谷关于火星的一些观测数据;尽管第谷对他的数据保密,但他对开普勒的理论思想印象深刻,很快允许他获得更多的数据。开普勒计划基于火星数据来验证他在《宇宙神秘学》中的理论,但他估计这项工作可能需要两年时间(因为他不能直接复制数据用于自己的研究)。在约翰内斯·耶塞纽斯的帮助下,开普勒试图与第谷谈判达成更正式的工作安排,但谈判最终因激烈争论破裂,开普勒于4月6日离开前往布拉格。开普勒与第谷很快和解,并最终达成了薪水和生活安排的协议,随后在6月,开普勒返回格拉茨收集家人。

由于在格拉茨的政治和宗教困难,开普勒未能立即返回布拉赫。为了继续他的天文研究,开普勒寻求成为费尔迪南大公的数学家。为此,开普勒写了一篇文章,献给费尔迪南,在其中提出了一种基于力的月球运动理论:“In Terra inest virtus, quae Lunam ciet”(“地球中存在一种力量,使月亮运动”)。尽管这篇文章未能使他获得费尔迪南宫廷的职位,但它详细描述了一种新的月食测量方法,开普勒在7月10日的格拉茨月食中应用了这一方法。这些观测成为了他对光学定律研究的基础,最终在《天文学光学部分》中得到发展。

1600年8月2日,开普勒因拒绝改信天主教而被逐出格拉茨。几个月后,开普勒与家人一同返回布拉格。在1601年大部分时间里,他由第谷直接资助,负责分析行星观测数据并写作一篇针对第谷已故对手乌尔苏斯的文章。9月,第谷为他争取到了一项新任务,成为他提议的“鲁道夫表”的合作人员,该表应当取代伊拉斯谟·赖因霍尔德的《普鲁滕表》。1601年10月24日第谷突然去世后,开普勒被任命为他的继任者,担任皇帝数学家,负责完成第谷未完成的工作。在随后的11年里,作为帝国数学家的岁月将是开普勒一生中最富有成果的时期。
\subsubsection{帝国顾问}
作为帝国数学家,开普勒的主要职责是为皇帝提供占星建议。尽管开普勒对当时占星家试图精确预测未来或预言特定事件的做法持怀疑态度,但自从他在图宾根大学时起,便开始为朋友、家人和赞助人提供受欢迎的详细星座图。除了为盟友和外国领袖绘制星座图外,皇帝在政治困境时也寻求开普勒的建议。鲁道夫皇帝对许多宫廷学者的工作(包括多位炼金术士)表现出积极兴趣,并且也关注开普勒在物理天文学方面的研究成果。[41]

在官方层面上,布拉格仅接受天主教和乌特拉教(捷克宗教改革派)的宗教教义,但由于开普勒在帝国宫廷中的位置,他得以不受阻碍地实践路德宗信仰。虽然皇帝名义上为他的家庭提供了可观的收入,但由于帝国财政过度扩张,实际上获得足够的钱来满足财务义务始终是一个持续的斗争。部分由于财政问题,开普勒与巴巴拉的家庭生活并不愉快,常常发生争吵和疾病的折磨。然而,宫廷生活使开普勒与其他著名学者(如约翰内斯·马修斯·瓦克赫尔·冯·瓦赫肯费尔斯、约斯特·布尔吉、戴维·法布里修斯、马丁·巴哈泽克和约翰内斯·布伦格尔等)接触,天文学的研究进展迅速。[42]
\subsubsection{1604年超新星}
\begin{figure}[ht]
\centering
\includegraphics[width=8cm]{./figures/9af7403388dbf778.png}
\caption{开普勒超新星SN 1604的残余} \label{fig_KPL1_7}
\end{figure}
\textbf{1604年10月,明亮的新晚星(SN 1604)出现,但开普勒直到亲眼看到它时才相信有关它的传言。}[43] 开普勒开始系统地观测这颗超新星。在占星学上,1603年底标志着一个火象三分相的开始,这是约800年周期的大合相的起始;占星家们将之前的两个这样的周期与查理曼大帝的崛起(大约800年前)和基督的诞生(大约1600年前)联系在一起,因此预期会发生一些具有重大预兆的事件,尤其是关于皇帝的。[44]

正是在这种背景下,作为皇帝的帝国数学家和占星家,开普勒在两年后在《新星论》(De Stella Nova)中描述了这颗新星。在这篇作品中,开普勒探讨了这颗星星的天文特性,同时对当时流传的许多占星学解释持怀疑态度。他注意到这颗星的亮度逐渐衰退,推测它的起源,并且利用没有观测到视差的现象来论证它位于恒星的天球上,进一步削弱了天体不变的学说(自亚里士多德以来人们一直接受的观点,认为天体球是完美且不变的)。一颗新星的诞生意味着天体的可变性。开普勒还附加了一个附录,讨论了波兰历史学家劳伦提乌斯·苏斯里加的最新年代学工作;他计算出,如果苏斯里加的观点是正确的,即现有时间表滞后了四年,那么伯利恒之星——与目前的新星相似——将与早期800年周期的第一次大合相相吻合。[45]

在随后的几年里,开普勒尝试(未成功)与意大利天文学家乔凡尼·安东尼奥·马吉尼合作,并处理年代学问题,特别是耶稣生平事件的日期问题。大约在1611年,开普勒传播了一份手稿,这份手稿最终(死后)出版为《梦境》(Somnium)。《梦境》部分目的在于描述从另一个星球的角度,实践天文学是什么样的,展示一个非地心系统的可行性。手稿在多次转手后消失,内容描述了一次奇妙的月球之旅;它既是寓言,又是自传,同时也是关于星际旅行的论述(有时被称为第一部科幻作品)。多年后,这个故事的扭曲版本可能促使了针对其母亲的巫术审判,因为故事中的叙述者的母亲向一个魔鬼求教,了解太空旅行的方法。在母亲最终被宣判无罪后,开普勒为这个故事写了223条脚注——比实际文本长了几倍——这些脚注解释了寓言的意义以及隐藏在文本中的大量科学内容(特别是关于月球地理的部分)。[46]
\subsection{晚年生活}
\subsubsection{困境}
\begin{figure}[ht]
\centering
\includegraphics[width=6cm]{./figures/9e8fff76da82be05.png}
\caption{布拉格旧城区的卡尔洛瓦街——开普勒曾居住的房子,现在是博物馆} \label{fig_KPL1_8}
\end{figure}
1611年,布拉格日益加剧的政治宗教紧张局势达到了高潮。健康逐渐衰退的鲁道夫皇帝被他的弟弟马蒂亚斯迫使退位,放弃了波希米亚国王的职位。双方都寻求开普勒的星象建议,开普勒利用这一机会提供了调解性的政治建议(不过他仅在一般性声明中提到星象,旨在阻止采取极端行动)。然而,很明显,开普勒在马蒂亚斯宫廷中的未来前景黯淡。[47]

同年,巴巴拉·开普勒感染了匈牙利斑疹热,随后开始发生癫痫发作。当巴巴拉在恢复过程中,开普勒的三个孩子都感染了天花;其中六岁的弗里德里希不幸去世。儿子去世后,开普勒写信给可能的资助者,寻求支持,特别是在维尔茨堡和帕多瓦。在维尔茨堡的图宾根大学,由于开普勒被认为有加尔文主义异端思想,这与《奥斯堡信条》和《和解公式》相违背,因此他未能回到该校。帕多瓦大学在即将离职的伽利略推荐下寻求开普勒担任数学教授,但开普勒更愿意将家人留在德国语境下,因此决定前往奥地利,在林茨安排一份教师及地区数学家的职位。然而,巴巴拉病情复发,不久后去世。[48]

开普勒推迟了前往林茨的计划,直到1612年初鲁道夫去世后,他才离开布拉格。但在这一期间,由于政治动荡、宗教紧张以及家庭悲剧(加上与妻子遗产有关的法律纠纷),开普勒无法进行任何科研工作。相反,他整理了一本名为《编年史摘录》的手稿,内容主要基于他的信件和早期工作。随着马蒂亚斯继位为神圣罗马帝国皇帝,他重新确认了开普勒作为帝国数学家的职位(及其薪水),并允许他迁移到林茨。[49]
\subsubsection{林茨 (1612–1630)}
\begin{figure}[ht]
\centering
\includegraphics[width=8cm]{./figures/e72538aa0ad81311.png}
\caption{林茨的开普勒雕像} \label{fig_KPL1_9}
\end{figure}
在林茨,开普勒的主要职责(除了完成《鲁道夫天文表》)是教授区内学校的课程,并提供占星和天文服务。在那里度过的最初几年,他相比在布拉格的生活享有相对的财政安全和宗教自由——尽管由于神学上的分歧,他被路德宗教会排除在圣餐之外。正是在林茨,开普勒不得不处理对他母亲凯瑟琳娜在莱昂贝格(一个新教城市)被指控并最终判定为巫术的事件。这一打击发生在开普勒被逐出教会后的几年,且被视为新教徒对开普勒进行全面攻击的一个标志,而非巧合。[50]

他在林茨的首部出版物是《基督诞生之年论》(1613),这是对基督诞生年份的扩展论述。他还参与了是否向德国的新教地区引入格里高利历改革的讨论。1613年10月30日,开普勒与苏珊娜·罗伊廷格结婚。在第一任妻子巴巴拉去世后,开普勒在两年的时间里考虑了11个婚配对象(这一决策过程后来被正式化为婚姻问题)。[51]最终,他选择了第五个配偶——罗伊廷格,他写道:“她以爱、谦逊的忠诚、家庭经济、勤勉以及她对继子女的爱打动了我。”[52] 这段婚姻比他的第一段更加幸福。这个婚姻的前三个孩子(玛格丽塔·雷吉娜、凯瑟琳娜和塞巴尔德)在童年时去世。其后三个孩子长大成人:科尔杜拉(1621年生)、弗里德玛尔(1623年生)和希尔德贝特(1625年生)。根据开普勒的传记作家所说,这段婚姻比他的第一段更加幸福。[53]

1630年10月8日,开普勒前往雷根斯堡,计划收取他以前所做工作的报酬。到达雷根斯堡几天后,开普勒生病了,病情逐渐加重。1630年11月15日,在到达雷根斯堡一个多月后,他去世了。他被埋葬在雷根斯堡的一座新教教堂墓地中,这座墓地在三十年战争期间被完全摧毁。[54]
\begin{figure}[ht]
\centering
\includegraphics[width=6cm]{./figures/48be9dba8865d434.png}
\caption{雷根斯堡,教堂 [Peterskirchlein] 错误: {{Lang}}: 无效的参数: |label=(帮助),约翰内斯·开普勒墓碑的纪念牌。} \label{fig_KPL1_10}
\end{figure}
\subsubsection{基督教信仰}  
开普勒认为上帝以有序的方式创造了宇宙,这促使他试图确定并理解支配自然界的法则,尤其是在天文学方面。[55][56]他曾说过“我只是在按照上帝的思想思考”,这一句话虽有流传,但更可能是他一段文字的简化版本:

这些自然法则是人类心智所能理解的;上帝希望我们通过创造我们为他的形象来认识这些法则,让我们能够分享他自己的思想。[57]

开普勒提倡基督教各教派之间的宽容,例如他主张天主教徒和路德宗教徒应该能够共同领圣餐。他曾写道:“基督主既不是路德宗的,也不是加尔文宗的,也不是天主教的。[58]
