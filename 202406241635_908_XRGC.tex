% 吸热过程
% license CCBYSA3
% type Wiki

(本文根据 CC-BY-SA 协议转载自原搜狗科学百科对英文维基百科的翻译)

“\textbf{吸热过程}” 是指任何需要或吸收周围环境能量的过程,通常是以热的形式存在。它可能是一个化学过程,如将硝酸铵溶解在水中,也可能是一个物理过程,如冰块的融化。这个术语是由马尔塞林·贝特洛从希腊词根“'endo-” 中创造出来的,其中词“endon”(ἔνδον)意为“内在”,而词“therm”(θεμ-),意为“热的”或“温暖的”,意思是一个反应的进行取决于是否吸收热量。与吸热过程相反的是放热过程,放热过程以热的形式释放“释放”能量。 因此,在每一个术语(吸热和放热)中,前缀是指反应发生时热量的去向,尽管实际上它只指能量的去向,而不是指热量的形式。

\subsection{细节}
所有的化学反应都涉及现有化学键的断裂和新化学键的形成。破坏键的反应总是需要能量的输入,所以这个过程总是吸热的。当原子聚集在一起形成新的化学键时,将它们聚集在一起的静电力会留下大量过剩的能量(通常以振动和旋转的形式)。如果这种能量没有消散,新的化学键会很快再次分裂。相反,新的键可以通过辐射,转移到分子中的其他运动或者通过碰撞转移到其他分子中来释放多余的能量,然后成为一个稳定的新键。释放多余的能量即离开分子系统的放热性。给定的总反应是放热还是吸热取决于这些键断裂吸收的热量和新的键稳定释放的热量的相对贡献。

这个概念经常应用于物理科学,例如,化学反应即热能(热量)转化为化学键能。

吸热(和放热)分析仅用来解释反应的焓($\Delta H$)的变化。反应的总能量分析是吉布斯自由能($\Delta G$),它包括熵($\Delta S$)、温度项以及焓。一个反应如果产物具有较低的吉布斯自由能(一种放能反应),在一定温度下它将是一个自发过程,即使产品的焓很更高。熵和焓是不同的术语,所以熵能的变化可以克服焓能的相反变化,使吸热反应变得有利。

\subsection{例子}
\begin{itemize}
\item 光合作用
\item 冰的融化
\item 液态水的蒸发
\item 二氧化碳(干冰)升华
\item 烷烃裂解
\item 热解反应
\item 电解分解:氯化钠变成氢氧化钠和氯化氢
\item 氯化铵在水中溶解
\item 恒星核中重镍元素的核合成
\item 高能中子可以在吸热反应中从锂-7中产生氚,消耗2.466兆电子伏。这是在1954年布拉沃\item 城堡核试验产生了出乎意料的高产量时发现的。[1]
\item 超新星中重于铁元素的核聚变 [2]
\item 氢氧化钡和氯化铵溶解在一起
\end{itemize}

\subsection{参考文献}
[1]
^Austin, Patrick (January 1996). "Tritium: The environmental, health, budgetary, and strategic effects of the Department of Energy's decision to produce tritium". Institute for Energy and Environmental Research. Retrieved 2010-09-15..

[2]
^Qian, Y.-Z.; Vogel, P.; Wasserburg, G. J. (1998). "Diverse Supernova Sources for the r-Process". Astrophysical Journal 494 (1): 285–296. arXiv:astro-ph/9706120. Bibcode: 1998ApJ...494..285Q. doi:10.1086/305198..