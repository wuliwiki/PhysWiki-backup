% 时空中的标量场
% 标量|场论|时空

\pentry{标量场的谱\upref{spectr}}
在海森堡绘景下,算符$\phi$和$\pi$对时间的依赖关系为
\begin{equation}
\phi(x) = \phi(\mathbf x, t) = e^{i H t}\phi(\mathbf x)e^{-iHt}~.
\end{equation}
运动方程为 
\begin{equation}
i \frac{\partial}{\partial t}\mathcal O = [\mathcal O, H]~.
\end{equation}
由此我们可以计算出$\phi$和$\pi$的时间依赖
\begin{equation}
\begin{aligned}
i \frac{\partial}{\partial t} \phi(\mathbf x,t) & = \bigg[\phi(\mathbf x,t),\int d^3 x' \bigg\{ \frac{1}{2} \pi^2 (\mathbf x',t) + \frac{1}{2} (\nabla \phi(\mathbf x',t))^2 + \frac{1}{2} m^2 \phi^2(\mathbf x',t) \bigg\}\bigg] \\
& = \int d^3 x' \bigg(i \delta^{(3)} (\mathbf x - \mathbf x') \pi(\mathbf x',t) \bigg) \\
& = i \pi (\mathbf x, t)~.
\end{aligned}
\end{equation}
\begin{equation}
\begin{aligned}
i\frac{\partial}{\partial t} \pi(\mathbf x, t) & = \bigg[ \pi(\mathbf x,t), \int d^3 x' \bigg\{  \frac{1}{2} \pi^2 (\mathbf x', t) + \frac{1}{2} \phi(\mathbf x, t) (-\nabla^2 + m^2 ) \phi(\mathbf x', t) \bigg\} \bigg] \\
& = \int d^3 x' \bigg( - i \delta^{(3)} (\mathbf x - \mathbf x') (-\nabla^2+m^2) \phi(\mathbf x',t) \bigg)~.
\end{aligned}
\end{equation}
综上我们可以得知
\begin{equation}
\frac{\partial^2}{\partial t^2} \phi = (\nabla^2 - m^2)\phi~.
\end{equation}
这就是著名的克莱因-戈登方程.

我们可以把$\phi$场和$\pi$场用产生湮灭算符表示出来.首先我们可以写出
\begin{equation}
H a_{\mathbf p} = a_{\mathbf p} (H - E_{\mathbf p})
\end{equation}
因此
\begin{equation}
H^n a_{\mathbf p} = a_{\mathbf p} (H - E_{\mathbf p})^n
\end{equation}
于是我们可以推出
\begin{equation}
e^{iHt} a_{\mathbf p} e^{-iHt} = a_{\mathbf p} e^{-i E_{\mathbf p}t}~, \quad e^{iHt} a^\dagger_{\mathbf p} e^{-iHt} = a^\dagger_{\mathbf p} e^{iE_{\mathbf p}t} ~.
\end{equation}
之后我们可以用产生湮灭算符来表示$\pi$场和$\phi$场,结果是
\begin{equation}
\phi(\mathbf x,t)= \int \frac{d^3 p }{(2\pi)^3} \frac{1}{\sqrt{2E_{\mathbf p}}} (a_{\mathbf p} e^{-i p \cdot x} + a^\dagger_{\mathbf p} e^{i p \cdot x}) \bigg|_{p^0 = E_{\mathbf p}}
\end{equation}




