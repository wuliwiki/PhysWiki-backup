% 正规子群和商群
% keys 子群|正规子群|商群|交错群|置换群|左陪集|中心|中心化子|正规化子
% license Xiao
% type Tutor

\pentry{陪集和同余\nref{nod_coset}}{nod_3e34}

\subsection{正规子群}

在上文中,我们在群上定义了元素和子集间、子集和子集间的运算,还知道了左陪集是一种等价类划分。假设我们有群 $G$ 和它的子群 $H$,能不能用 $H$ 的全部左陪集当元素,构成一个集合,并在上面定义一个和 $G$ 的运算相关的群运算呢?

最直接的想法,就是直接进行左陪集之间的运算:对于左陪集 $A$ 和 $B$,有 $AB=\{ab|a\in A, b\in B\}$。 但是一般来说,如果 $G$ 的运算不是可交换的,我们这样算出来的 $AB$ 甚至不一定是一个左陪集,那么这个左陪集之间的运算甚至不满足封闭性,自然不能是个群运算了。

如何让左陪集之间的运算满足封闭性呢?

假设我们有 $x, y\in G$,它们对应的左陪集分别是 $xH$ 和 $yH$。由于 $xH=\{xh_1|h_1\in H\}$,$yH=\{yh_2|h_2\in H\}$, 那么根据定义有:$xHyH=\{xh_1yh_2|h_1, h_2\in H\}$。每一个 $xh_1yh_2$ 都在左陪集 $xh_1yH$ 中,我们希望每一个 $h_1$ 对应的 $xh_1yH$ 都是同一个左陪集。

在\autoref{the_coset_1}~\upref{coset} 的证明中我们提到了一个判断两元素是否在同一个左陪集中的方法。现在假设 $xh_1y$ 和 $xh_2y$ 在同一个左陪集里,应用判断方法得到:$(xh_1y)^{-1}(xh_2y)\in H$。由于 $(xh_1y)^{-1}=y^{-1}h_1^{-1}x^{-1}$,我们有:
\begin{equation}
y^{-1}h_1^{-1}x^{-1}xh_2y\in H~,
\end{equation}
也就是说
\begin{equation}
y^{-1}h^{-1}_1h_2y\in H~.
\end{equation}

由于这里的 $y$ 是 $G$ 中任意元素,$h_1$ 和 $h_2$ 是 $H$ 中任意元素,故我们要求 $y^{-1}Hy\subseteq H$ 对于任何 $y\in G$ 都成立, 其中 $y^{-1}Hy = \qty{y^{-1}h_i y|h_i \in H}$。巧的是,让 $H$ 满足了这个要求之后,封闭性就满足了。

\begin{exercise}{}\label{exe_NormSG_1}
证明:对于群 $G$ 和它的子群 $H$,如果 $H$ 满足对于任意的 $g\in G$,都有 $g^{-1}Hg\subseteq H$, 那么对于任意的 $x, y\in G$,$xHyH$ 中的元素都在同一个左陪集中。
\end{exercise}

\begin{corollary}{}\label{cor_NormSG_1}
由于群运算的唯一性,从 $g^{-1}Hg\subseteq H$ 我们还可以推知 $|g^{-1}Hg|=|H|$,因此所需要的条件等价于 $g^{-1}Hg=H$, 即 $Hg = gH$, 也就是左陪集等于右陪集。
\end{corollary}

\begin{definition}{正规子群}\label{def_NormSG_1}
满足\autoref{cor_NormSG_1} 中条件的子群,被称作\textbf{正规子群(normal subgroup)}。我们把“$H$ 是 $G$ 的正规子群”简记为 $H\vartriangleleft G$ 或 $G\vartriangleright H$\footnote{三角形总指向子群。}。
\end{definition}

现在我们考察一下,如果 $H$ 满足了条件,那么在左陪集的集合中通过 $G$ 的运算而导出的集合间的运算,是不是一个群运算。
\begin{itemize}
\item 封闭性:由\autoref{exe_NormSG_1} 知,该运算有封闭性。
\item 结合性:由于 $G$ 的运算具有结合性可知,设 $x, y, z\in G$,那么 $(xh_1yh_2)zh_3=xh_1(yh_2zh_3)$ 对任意的 $h_n\in H$ 都成立,故而 $(xHyH)zH=xH(yHzH)$ 成立。
\item 单位元存在性:$H$ 就是单位元。
\item 逆元存在性:$xHx^{-1}H=HH=H$。
\end{itemize}

这样一来,在左陪集的集合上也诱导了一个群运算,构成了一个群。我们把这个群叫做群 $G$ 模去正规子群 $H$ 的\textbf{商群(quotient group)},记作 $G/H$。由拉格朗日定理(\autoref{the_coset_2})的证明过程可知,$|G/H|={|G|}/{|H|}$。

\begin{example}{$\mathbb{Z}$ 的正规子群和商群}\label{ex_NormSG_4}
$n\mathbb{Z}$ 是 $\mathbb{Z}$ 的正规子群,这对任何 $n$ 都成立。证明很简单,因为 $\mathbb{Z}$ 是交换群,所以元素位置可以自由互换,因此 $g^{-1}Hg=g^{-1}gH=H$ 对任何\textbf{子集}都成立,因此交换群的子群都是正规子群。

考虑 $\mathbb{Z}$ 模去 $3\mathbb{Z}$ 得到的商群,我们记为 $\mathbb{Z}/3\mathbb{Z}=\mathbb{Z}_3$。 这个商群一共有 $3$ 个元素,每个元素都是一个左陪集:$0+3\mathbb{Z}=3\mathbb{Z}=\{\cdots, -6, -3, 0, 3, 6, \cdots\}$,$1+3\mathbb{Z}=\{\cdots, -5, -2, 1, 4, 7, \cdots\}$ 和 $2+3\mathbb{Z}=\{\cdots, -4, -1, 2, 5, 8, \cdots\}$。 换句话说,集合 $\mathbb{Z}_3=\{3\mathbb{Z}, 1+3\mathbb{Z}, 2+3\mathbb{Z}\}$。

注意,虽然 $\mathbb{Z}_3$ 的元素是集合,我们仍然要把它们看成一个个的元素。事实上,$\mathbb{Z}_n$ 的群结构和一个有着 $n$ 个数字的钟表(\autoref{ex_Group_2}~\upref{Group})是一模一样的,因此我们也用 $\mathbb{Z}_n$ 来表示循环群。 
\end{example}

\begin{example}{置换群和交错群}\label{ex_NormSG_5}
我们知道,置换群 $S_n$(\autoref{ex_Group_3})是 $n$ 个元素的变换所构成的群。为了简洁地表达变换,我们可以用以下记号:

$(1,2,3)$ 表示一个变换,它是将 $1$ 号桶中的球放入 $2$ 号桶,$2$ 号桶中的放入 $3$ 号桶,$3$ 号桶中的又放入 $1$ 号桶。显然,$(2,3,1)$ 表示的是同一个变换。$(1,2,3)(4,5)$ 表示的是先进行 $(1,2,3)$ 变换,再进行 $(4,5)$ 变换。显然,这两个变换没有涉及相同数字,所以它们是彼此分离的,可以交换:$(1,2,3)(4,5)=(4,5)(1,2,3)$。

如果两个变换有相同数字,那么我们可以把它们合并为一个括号:$(1,2)(2,3)$ 代表先“把 $1$ 号桶中的球放入 $2$ 号桶,$2$ 号桶中的放入 $1$ 号桶”,再“把 $2$ 号桶中的球放入 $3$ 号桶,$3$ 号桶中的放入 $2$ 号桶”, 最终结果是“$1$ 号桶中的球放入 $3$ 号桶,$2$ 号桶中的放入 $1$ 号桶,$3$ 号桶中的球放入 $2$ 号桶”,也就是说,$(1,2)(2,3)=(1,3,2)$。注意 $(1,2,3)\not=(1,3,2)$。

只涉及到两个元素之间的变换显然是最简单的,我们称之为\textbf{对换}。一切变换都是由若干个对换相乘(复合运算)得到的。可以证明(\autoref{ex_GroupP_1}~\upref{GroupP}),每进行一次对换,桶中球的“逆序数”的奇偶性会改变。初始状态下的逆序数是 $0$,一个偶数;因此如果一个变换可以被奇数个对换组合得到,那么这个变换会使得桶中球的逆序数变成奇数,这个变换就叫做奇变换;同样地,偶数个对换组合得到的变换称为偶变换,它会使得桶中球的逆序数仍为偶数。

偶变换之间的复合仍然能得到偶变换,因此全体偶变换构成置换群 $S_n$ 的一个子群,称为\textbf{交错群(alternating group)},记为 $A_n$。交错群是置换群的一个正规子群,元素数量是置换群的一半——也就是说,$|S_n/A_n|=2$ 对任何 $n$ 都成立。
\end{example}

\begin{example}{中心}\label{ex_NormSG_1}
给定群 $G$ 和它的一个子集 $S$,记 $C(S)$ 为 $G$ 中所有能和 $S$ 中元素交换的元素之集合,即 $C(S)=\{g\in G|\forall s\in S, gs=sg\}$,称之为 $S$ 在 $G$ 中的\textbf{中心化子(centralizer)},有时为了强调也记为 $C_G(S)$。特别地,称 $C(G)$ 是 $G$ 的\textbf{中心(center)}。

考虑到对于任意 $c\in C(G)$ 和任意 $g\in G$ 都有 $cg=gc$,那么任取 $h\in G$,也必有 $hch^{-1}g=hh^{-1}cg=cg=gc=gchh^{-1}=ghch^{-1}$,即 $hch^{-1}\in C(G)$。因此,$C(G)\lhd G$。
\end{example}


任意给定群 $G$ 的子群 $H$,它不一定是正规子群,但我们可以适当“修剪”群 $G$ 来获得一个子群,使得 $H$ 在这个子群里是正规子群。这样的子群就是以下定义的正规化子:


\begin{definition}{正规化子}\label{def_NormSG_2}
给定群 $G$ 和它的一个子集 $S$。如果 $H$ 是包含 $S$ 的最小的子群\footnote{我们通常称 $H$ 是由 $S$ 生成的子群,用 $<S>$ 表示。},记 $N(S)=\{n\in G|nH=Hn\}$,那么 $N(S)$ 构成一个子群,而 $H\vartriangleright N(S)$。称 $N(S)$ 是 $S$ 的\textbf{正规化子(normalizer)}。
\end{definition}

注意区分中心化子和正规化子的定义,对于同样的 $S$,其正规化子比中心化子的定义要弱一些,并且正规化子本身不一定是 $G$ 的正规子群。
\subsection{商群继承yuanqun}
\subsection{小结}

对于任何群 $G\vartriangleright H$,所有左陪集的元素数量都是 $|H|$,包括 $H$ 本身。这些左陪集彼此没有交集,但群 $G$ 是所有左陪集的并集,我们由此得到了拉格朗日定理,揭示了一个任何群都具有的简洁的结构特征。商群 $G/H$ 的元素数量,正是 $G$ 和 $H$ 的元素数量的商。

当子群 $H$ 满足,对于任意的 $g\in G$ 都有 $g^{-1}Hg=H$ 时,$H$ 就成了一个正规子群。只有正规子群才能生成商群;非正规的子群划分出来的等价类,按照二元关系\upref{Relat}中提到的方法构成的商集,却并不能用 $G$ 的运算来导出一个群运算。当然,你也可以把这个商集看成和 $G$ 完全无关的集合,给它独立定义一个群运算,但那样这个群就和 $G$ 没有一点关系了,不能叫做 $G$ 的商群。

$G/H$ 的群运算是 $G$ 的运算导出来的,所以继承了 $G$ 中很多相似的性质;但是 $G/H$ 把同一个左陪集中各个元素的差别都抹去了,相当于一个简化版的 $G$。所以说,商群继承了原群的部分特征,但是没有完全继承。

两个定义相似的概念:中心化子和正规化子。

接下来当我们讲到群之间的映射的时候,商群与原群“似而不同”的结构非常重要。
