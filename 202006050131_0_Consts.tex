% 物理学常数定义
% keys 物理常数|国际单位|测量

\subsection{国际单位的定义(SI Units)}

物理单位的一个重要性质就是可测量, 至少理论上可测量. 以下的数值除了有特殊说明, 都是精确值(无限位小数用省略号表示), 不存在误差.

\subsubsection{秒(s)的定义}
铯原子133基态的超精细能级之间的跃迁辐射的电磁波周期的 $9, 192, 631, 770$ 倍. 
说明: 我们知道原子中的电子具有不同的能级, 当电子从一个能级跃迁到一个更低的能级时, 会放出一个光子. 光子的频率为 $\nu  = \varepsilon /h$,   其中 $\varepsilon $ 是光子的能量, $h$ 为普朗克常数.

\subsubsection{米(m)的定义}
真空中, 光在 $1/299792458$ 秒内传播的距离
说明: 由于真空中的光速是物质和信息能传播的最快速度, 且在任何参考系中都相同, 所以可以作为一个精确的标准. 结合秒的定义, 就可以通过实验得到一米的长度.

\subsubsection{光速(c)的定义}
 \begin{equation}
c = 299792458 \Si{m/s}
\end{equation} 
说明: 根据米的定义, 一秒中光可以在真空中传播 $299792459 \Si{m}$.  

\subsubsection{千克(kg)的定义}
等于国际公斤原器的质量
说明: 千克是现有的唯一一个由特定的物品所定义单位. 国际公斤原器是国际计量大会制造的, 并复制若干份分别存放, 但经过长时间后被发现和复制品存在细微误差. 国际计量大会最终在 2014 年决定原则上千克应该由普朗克常数所决定, 但是最终的定义再次被推迟.

\subsubsection{牛顿(N)的定义}
等于使 $1\Si{kg}$ 物体获得 $1\Si{m/s^2}$ 加速度的力.

\subsubsection{真空磁导率( $\mu_0$ )的定义}
\begin{equation}
\mu_0 = 4\pi \times 10^{-7} \Si{N/A^2}
\end{equation}

\subsubsection{真空介电常数( $\epsilon_0$ )的定义}
\begin{equation}
\epsilon_0 = 1/(c^2 \mu_0) = 8.8541878176\dots \times 10^{-12} \Si{F/m}
\end{equation}
说明: 其中 $c$ 是光速, $\mu_0$ 是真空磁导率. 根据该定义,$c = 1/\sqrt{\epsilon_0 \mu_0} $.  

\subsubsection{库仑(C)的定义}
若真空中两个相同的点电荷相距一米, 产生的相互作用力为 $1/(4\pi\epsilon_0)$,   则该点电荷为1 库仑.

\subsubsection{安培(A)的定义}
以下两种定义等效:
\begin{enumerate}
\item 每秒钟经过横截面的电荷量为 1 库仑的电流就是 1 安培.
\item 两根相距一米的无限长平行细导线流入 1 安培电流后, 相互作用力是 $2 \times 10^{-7}\Si{N/m}$. 
\end{enumerate}



