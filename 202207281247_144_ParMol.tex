% 偏摩尔量

\subsection{偏摩尔量}
一个相的态函数$Z=f(p,T,n_1,n_2,...)$,其中Z为广度量.

对其求全微分\upref{statef}  
\begin{equation}\label{ParMol_eq1}
dZ = \left(\pdv{Z}{p}\right)_{T,n_1,n_2,...}dp + \left(\pdv{Z}{T}\right)_{p,n_1,n_2,...}dT+\left(\pdv{Z}{n_1}\right)_{p,T,n_2,...}dn_1+\left(\pdv{Z}{n_2}\right)_{p,T,n_1,n_3,...}dn_2+...
\end{equation}

定义偏摩尔量
\begin{equation}
\overline {Z_B} = \left(\pdv{Z}{n_1}\right)_{p,T,n_i, i \neq B} 
\end{equation}
那么 \autoref{ParMol_eq1} 化为 
\begin{equation}
dZ = \left(\pdv{Z}{p}\right)_{T,n_1,n_2,...}dp + \left(\pdv{Z}{T}\right)_{p,n_1,n_2,...}dT+\sum \overline {Z_B} dn_B
\end{equation}
特别地,如果温度、压力一定,
\begin{equation}\label{ParMol_eq3}
dZ = \sum \overline {Z_B} dn_B
\end{equation}

$\overline {Z_B} d n_B$ 的物理含义可以理解为,再往系统中加入少量B,系统状态X的变化.

注意,混合物中的$\overline {Z_B}=\overline {Z_B}(p,T,n_1,n_2,...)$不是常数,与各物质的含量等均有关

\subsection{偏摩尔量常用公式}
\subsubsection{集合公式}
\begin{equation}\label{ParMol_eq2}
Z=\sum \overline {Z_B}  n_B
\end{equation}
\subsubsection{Gibbs-Duhem 公式}
\begin{equation}
\sum n_B \dd {\overline {Z_B}} = 0
\end{equation}
对于二元混合物,
\begin{equation}
1
\end{equation}



推导:

对集合公式(\autoref{ParMol_eq2} )两端求导,$dZ=\sum n_B \dd {\overline {Z_B}} + \sum \overline {Z_B}  \dd n_B$

代入 \autoref{ParMol_eq3} ,对比,即得证.


	∑▒〖n_B d¯(Z_B )〗=0 (等温等压)
	对于二元混合物,则 n_1 d¯(v_1 )+n_2 d¯(v_2 )=0⇒(d¯(v_1 ))/(d¯(v_2 ))=-¯(v_2 )/¯(v_1 )
	推导:
	状态量全微分:X(p,T,n_1,n_2 )⇒dX=(∂X/∂p)_(T,n_B )+(∂X/∂T)_(p,n_B )+∑▒〖(X_B ) ̅dn_B 〗⇒dX=∑▒〖(X_B ) ̅dn_B 〗  (p,T固定)
	加和公式: X=∑▒〖(X_B ) ̅n_B 〗⇒dX=∑▒〖(X_B ) ̅dn_B 〗+∑▒〖n_B d(X_B ) ̅ 〗
	∴dX=∑▒〖(X_B ) ̅dn_B 〗=∑▒〖(X_B ) ̅dn_B 〗+∑▒〖n_B d(X_B ) ̅ 〗⇒∑▒〖n_B d(X_B ) ̅ 〗=0
	将状态函数关系式中的广度量换成相应的偏摩尔量,可得相应的偏摩尔量关系式
	(∂G/∂p)_T=V,prove ((∂(G_B ) ̅)/∂p)_T=(V_B ) ̅
	((∂G ̅)/∂p)_T=(∂/∂p (∂G/(∂n_B ))_(T,p,n_c ) )_T=(∂/(∂n_B ) (∂G/∂p)_T )_(T,p,n_c )=(∂V/(∂n_B ))_(T,p,n_c )=(V_B ) ̅
