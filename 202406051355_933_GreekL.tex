% 希腊字母表
% keys 希腊字母|字母
% license Usr
% type Tutor
\begin{issues}
\issueDraft
\end{issues}

在数学、物理学等领域,通常借用拉丁字母或希腊字母作为变量符号,不少符号作为约定俗成更具备特殊的意义。鉴于初学者往往对约定俗成的内容一头雾水,尤其是无人指引和介绍的情况下。本文会先针对各领域一些约定俗成的内容进行介绍,并在最后给出完整的希腊字母表,表中包含了中英文读音、记法和写法。

请注意,针对自己要学习的领域先去了解领域内的使用方法,就足够平时使用了。完全不必将希腊字母表完全背下来,待用待查即可。

\subsection{数学领域}

下面是常见的数学领域中,约定俗成的希腊字母使用方法。

\subsubsection{初等数学}

几何:

\begin{itemize}
\item $\Delta ABC$表示由A、B、C作为顶点的三角形。
\item $\alpha,\beta,\gamma,\theta$用于表示角,前三个一般指三角形中以A、B、C作为顶点的角。
\item $\pi$特指圆周率。
\end{itemize}

代数:

\begin{itemize}
\item $\Delta$用于表示一元二次方程中的判别式。
\item $\Sigma,\Pi$用于表示连加/连乘运算。
\end{itemize}

\subsubsection{高等数学}

\begin{itemize}
\item $\delta$一般用于表示临域的范围,例如$U_\delta(x):=[x-\delta,x+\delta]$。
\item $\varepsilon$一般用于表示一个非常小的正数,例如$\forall\varepsilon>0$。
\item $\omicron (x)$特指$x$的高阶无穷小。
\item $\xi$一般用于表示一个小区间上存在但未知的量,例如$\xi\in[x,x+\Delta x]$。
\item $\Delta x$特指$x$的变化量、差分。
\item $\kappa,\rho$表示曲率和曲率半径。
\end{itemize}

\subsubsection{线性代数}

\begin{itemize}
\item $\Lambda$特指对角矩阵。
\item $\lambda$特指矩阵的特征值。
\end{itemize}

\subsubsection{概率与统计}

概率:
\begin{itemize}
\item $\Omega,\omega$常用于表示样本空间和样本空间中的元素。
\item $\xi$一般用于表示一个随机变量,例如$\xi\sim{\rm N}(0,1)$。
\item $\varepsilon$一般用于表示指随机偏差,例如$y=x+\varepsilon$。
\item $\Phi(x),\phi(x)$分别表示正态分布的累积分布函数和概率密度函数。
\item $\chi^2(k)$表示$\chi^2$分布即k个独立的标准正态分布变量的平方和服从的分布。
\item ${\rm B(\alpha,\beta)}$和$\Gamma (\alpha,\beta)$分别表示由欧拉积分导出的Beta分布和Gamma分布。
\item $\lambda$一般作为指数分布${\rm Exp}(\lambda)$的参数,表示单位时间内事件发生的次数。
\end{itemize}

统计:
\begin{itemize}
\item $\mu,\sigma^2,\sigma$表示总体的均值、方差、标准差。
\item $\Sigma$表示协方差矩阵。
\item $\rho$表示相关系数。
\end{itemize}


\subsubsection{其他}

一般习惯:

\begin{itemize}
\item $\eta$用于表示求得的系数
\item $\lambda$用于表示未知的参数。
\item $\alpha,\beta,\gamma,\theta$用于表示已知的参数、常数或几何结构。
\item $\tau$用于表示时间区间。
\end{itemize}

函数名:

\begin{itemize}
\item ${\rm B(x,y)}$和${\Gamma (z)}$特指第一类欧拉积分(也称Beta函数)和第二类欧拉积分(也称Gamma函数)。
\item $\delta(x)$特指冲击函数。
\item $\zeta(s)$指黎曼$\zeta$函数。
\end{itemize}

\subsection{物理学领域}

下面是物理学领域中,约定俗成的希腊字母使用方法。

\subsection{字母表}

\begin{table}[ht]
\centering
\caption{希腊字母表}\label{tab_GreekL1}
\begin{tabular}{|c|c|c|c|c|}
\hline
英文拼写 & 英语音标 & 中文近似读音 & 大写字母 & 小写字母 \\
\hline
alpha & /'ælfə/&阿尔法 & ${\rm A}$ & $\alpha$ \\
\hline
beta & /'beɪtə/&贝塔 & ${\rm B}$ & $\beta$ \\
\hline
gamma & /'gæmə/&伽马 & $\Gamma$ & $\gamma$ \\
\hline
delta & /'deltə/&德尔塔 &$\Delta$ & $\delta$ \\
\hline
epsilon & /'epsɪlɒn/&艾普西隆 & ${\rm E}$ & $\epsilon$,$\varepsilon$ \\
\hline
zeta & /'zi:tə/&泽塔 & ${\rm Z}$ & $\zeta$ \\
\hline
eta & /'i:tə/&伊塔 & ${\rm H}$ & $\eta$ \\
\hline
theta & /'θi:tə/&西塔 & $\Theta$ & $\theta$,$\vartheta$ \\
\hline
iota & /aɪ'əʊtə/&尤塔 & ${\rm I}$ & $\iota$ \\
\hline
kappa & /'kæpə/&卡帕 & ${\rm K}$ & $\kappa$ \\
\hline
lambda & /'læmdə/&拉姆达 &$\Lambda$ & $\lambda$ \\
\hline
mu & /mju:/&缪 & ${\rm M}$ & $\mu$ \\
\hline
nu & /nju:/&纽 & ${\rm N}$ & $\nu$ \\
\hline
xi & /kˈsaɪ/&克西 &$\Xi$ & $\xi$ \\
\hline
omicron &/əuˈmaikrən/&奥密克戎 & ${\rm O}$ & $\omicron$ \\
\hline
pi & /paɪ/&派 &$\Pi$ & $\pi$,$\varpi$ \\
\hline
rho & /rəʊ/&肉 & ${\rm P}$ & $\rho$,$\varrho$ \\
\hline
sigma & /'sɪɡmə/&西格马 & $\Sigma$ & $\sigma$,$\varsigma$ \\
\hline
tau & /taʊ/&涛 & ${\rm T}$ & $\tau$ \\
\hline
upsilon & /ˈʌpsɪlɒn/&宇普西隆 & $\Upsilon$,${Y}$ & $\upsilon$ \\
\hline
phi & /faɪ/&斐 &$\Phi$ & $\phi$,$\varphi$ \\
\hline
chi & /kaɪ/&卡 & ${\rm X}$ & $\chi$ \\
\hline
psi & /psaɪ/&普塞 & $\Psi$ & $\psi$ \\
\hline
omega & /'əʊmɪɡə/&欧米伽 & $\Omega$ & $\omega$ \\
\hline
\end{tabular}
\end{table}

注:
\begin{enumerate}
\item 部分字母存在两种写法,两种写法往往在不同领域代表不同的意思,一般不混用。
\item 部分字母的大写(A、B、E、Z、H、I、K、M、N、O、P、T、Y、X)和小写($\iota$、$\omicron$、$\upsilon$)因与拉丁字母的写法相同或相似,为免引起歧义一般较少使用。
\item 部分字母的发音中文中不存在,因此,表中读音只作近似参考。目前世界上,普遍采用英语读音,同时在$\LaTeX$中输入时,也采用英文拼写。
\end{enumerate}