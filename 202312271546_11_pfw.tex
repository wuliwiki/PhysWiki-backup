% 前推
% license Usr
% type Tutor
赋予流形以联络,便可以比较同一流形上不同点的切向量“大小”。那对于不同流形,我们能否找到一个定义去联系同一点上的不同切空间呢?

\begin{issues}
\issueDraft 缺图
\end{issues}

\subsection{前推}
\begin{definition}{前推}
设$M,N$都是光滑流形。$f,g\in C^{\infty}(N)$。对任意$p\in M,\,F:M\rightarrow N$为光滑映射。定义$F_*:T_p M\rightarrow T_{F(p)N}$为
\begin{equation}
(F_*X)(f)=X(f\circ F)~,
\end{equation}
称之为与$F$关联的\textbf{前推(push-forward)}。
\end{definition}
可记忆前推是顺着光滑映射的方向对切向量进行转移,在$M$上$p$点的切空间与$N$上$F(p)$点的切空间建立关联。这样的定义是合理的,$F_*X$确实是一个切向量,满足导子的性质:
\begin{equation}
\begin{aligned}
(F_*X)(fg)&=X((fg)\circ F)\\
&=X(f\circ F)(g\circ F)\\
&=(g\circ F)X(f\circ F(p))+(f\circ F)X(g\circ F(p))\\
&=g(F)F_*X(f)+f(F)F_*X(g)~.
\end{aligned}
\end{equation}
易证前推具有如下性质:\begin{theorem}{}
令$F:M\rightarrow N$与$G:N\rightarrow P$都是光滑映射。且有$p\in M$。那么我们有:
\begin{enumerate}
\item $F_*:T_P M\rightarrow T_{F(p)}N$是线性的。
\item $(G\circ F)_*=G_*\circ F_*:T_p M\rightarrow T_{G\circ F(p)}P$
\item $(Id_M)_*=Id_{T_p M}:T_p M\rightarrow T_p M$
\item 若$F$是微分同胚,那么前推:$F_*: T_p M\rightarrow T_{F(p)N}$是同构映射。
\end{enumerate}
\end{theorem}

\subsection{前推的应用}
\subsubsection{流形上的切空间}
令$(U,\phi)$是$M$上的光滑坐标卡,也就是说$\phi$是双向光滑的微分同胚映射。因此,$\phi_*:T_p M\rightarrow T_{\phi(p)}R^n$是同构映射。那么我们可以利用前推来定义该坐标卡上每一点切空间的basis:
\begin{equation}
\frac{\partial}{\partial x^i}\bigg|_p=\phi^{-1}_* (\frac{\partial}{\partial x^i})\bigg|_{\phi(p)}~.
\end{equation}

设$\{\widetilde{e_i}\}$为$U$上$p$点的切空间基矢,对应$\phi (U)$上切空间的基矢$\{{e_i}\}$,并设$f\in C^{\infty}(U)$,那么我们有:
\begin{equation}
\begin{aligned}
\widetilde e_i f=\frac{\partial}{\partial x^i}\bigg|_p f&= (\frac{\partial}{\partial x^i})\bigg|_{\phi(p)}(f\circ \phi^{-1})\\
&=e_i\hat f(\hat p)~.
\end{aligned}
\end{equation}
可见,固定一个图,则流形上该点切空间的基矢定义与光滑函数的定义是自洽的。称$\hat f=f\circ \phi^{-1}$为光滑函数$f$的坐标表示,$\hat p=\phi (p)$为$p$点的坐标表示。
切空间是线性空间,因此也可以通过过渡矩阵进行基矢的变换。设$M$上相容的图。