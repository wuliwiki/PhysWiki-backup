% 分离性公理成立的充要条件
% keys 分离性公理|充要条件
% license Usr
% type Tutor

\pentry{分离性\nref{nod_Topo5}}{nod_cf78}
拓扑空间是较度量空间更一般的对象,很多度量空间的概念可以拓广到拓扑空间。然而正因为这一一般性,拓扑空间会出现本质上不同于度量空间的情况。例如,有限点集可能不是闭的,收敛序列的极限点可能不唯一等等。为了获得和度量空间更相近的空间来,需要添加一些补充条件。分离性公理\upref{Topo5}就是人们提出的一类重要条件,它衡量了拓扑空间中点的分离程度。本节将给出其中几个分离性公理成立的充要条件。

\subsection{分离性公理成立的充要条件}
第一分离性公理 $T_1$ 是指对拓扑空间中的任意两个不同点,每一点都存在不包含另一点的邻域。这样的拓扑空间被称为\textbf{ $T_1$ 空间}。
\begin{theorem}{$T_1$ 分离性公理成立的充要条件}\label{the_DisAx_1}
第一分离性公理 $T_1$ 成立的充要条件是:空间中任意单点集都是闭的。
\end{theorem}
\textbf{证明:}
\textbf{必要性:}假设第一分离性公理成立,设 $x\neq y$,则存在 $y$ 的邻域 $O_y$,使得
\begin{equation}
x\notin O_y~.
\end{equation}
由闭包的定义,$y\notin [x]$\footnote{为了方便,本文约定单点集 $\{x\}$ 直接写为 $x$。},即任意不同于 $x$ 的点都不在 $[x]$ 中,因此 $[x]=x$。因此任意单点集都是闭的。

\textbf{充分性:}假设任意单点集都是闭的。设 $x\neq y$,则那么 $x$ 的补集 $C(x)$ 是开的,并且 $y\in C(X),x\notin C(X)$,于是 $C(x)$ 是 $y$ 的满足 $x\notin C(x)$ 的邻域。同理 $C(y)$ 是 $x$ 的满足 $y\notin C(y)$ 的邻域。因此第一分离性公理成立。

\textbf{证毕!}



第三分离性公理 $T_3$ 是指任一点和不包含它的闭集都各自有一邻域存在,使得两邻域不相交。包含一个集合的邻域是指这个邻域包含一个包含该集合的开集。


\begin{theorem}{$T_3$ 分离性公理成立的充要条件}
第三分离性公理 $T_3$ 成立的充要条件是:任意单点集 $x$ 和其邻域 $O_x$,都存在 $x$ 的邻域 $ O'_x$,使得 $[O'_x]\subset O_x$。
\end{theorem}

\textbf{证明:}
\textbf{必要性:}假设 $T_3$ 分离性公理成立。对任意 $x$ 和其邻域 $O_x$,存在开集 $U\subset O_x$ 包含 $x$。于是 $C(U)$ 是闭的,因此由 $T_3$ 公理,存在开集 $U_x, O$,使得
\begin{equation}
x\in U_x,\quad C(U)\subset O,\quad U_x\cap O=\emptyset~.
\end{equation}
因此
\begin{equation}
U_x\subset C(O)=[C(O)]\subset U\subset O_x.~
\end{equation}
$U_x$ 刚好是所需要的 $O_x'$。

\textbf{充分性:} 假设任意单点集 $x$ 和其邻域 $O_x$,都存在 $x$ 的邻域 $ O'_x$,使得 $[O'_x]\subset O_x$。设 $x$ 是任意点,$A$ 是不包含 $x$ 的任一闭集。那么 $C(A)$ 是包含 $x$ 的邻域。于是由假设,存在 $x$ 的邻域 $O$,使得 
\begin{equation}
[O]\subset C(A)~.
\end{equation}
于是 $C([O])$ 是包含 $A$ 的邻域。且 $C([O])\cap O=\emptyset $。显然 $O,C([O])$ 正是所需要的。

\textbf{证毕!}


正规分离性公理是指任意两个不相交的闭集都各自有一邻域存在,使得两邻域不相交。

\begin{theorem}{}
正规分离性公理成立的充要条件是:任意闭集 $A$ 和其邻域 $O_A$,都存在 $A$ 的邻域 $ O'_A$,使得 $[O'_A]\subset O_A$。
\end{theorem}

\textbf{证明:}
\textbf{必要性:}假设正规分离性公理成立。对任意闭集 $A$ 和其邻域 $O_A$,存在开集 $U\subset O_A$ 包含 $A$。于是 $C(U)$ 是闭的,因此由正规性公理,存在开集 $U_A, O$,使得
\begin{equation}
A\in U_A,\quad C(U)\subset O,\quad U_A\cap O=\emptyset~.
\end{equation}
因此
\begin{equation}
U_A\subset C(O)=[C(O)]\subset U\subset O_A.~
\end{equation}
$U_A$ 刚好是所需要的 $O_A'$。

\textbf{充分性:} 假设任意闭集 $A$ 和其邻域 $O_A$,都存在 $A$ 的邻域 $ O'_A$,使得 $[O'_A]\subset O_A$。设 $A,B$ 是任意两不相交的闭集,那么 $C(A)$ 是包含 $B$ 的邻域。于是由假设,存在 $B$ 的邻域 $O$,使得 
\begin{equation}
[O]\subset C(A)~.
\end{equation}
于是 $C([O])$ 是包含 $A$ 的邻域。且 $C([O])\cap O=\emptyset $。显然 $O,C([O])$ 正是所需要的。

\textbf{证毕!}


\subsection{一些与分离性有关的性质}

上面未提及第二分离性公理 $T_2$ (即任意两个点都各自存在不包含另一点的邻域)成立的充要条件,事实上 $T_2$ 只是比 $T_1$ 更强的条件,即凡 $T_2$ 空间都是 $T_1$ 的,因为 $T_2$ 公理中的两个邻域本身就是不包含另一个点的。

\begin{definition}{正则空间}
满足 $T_1,T_3$ 公理的空间称为\textbf{正则空间}。
\end{definition}

满足 $T_2$ 公理的空间称为\textbf{Hausdorff空间},下面定理给出了它和正则空间的关系
\begin{theorem}{}
正则空间是Hausdorff 空间。
\end{theorem}
\textbf{证明:}
设 $x\neq y$ 是正则空间中的任意两点。于是由\autoref{the_DisAx_1} ,单点集 $y$ 是闭的。显然 $x\notin y$,于是由公理 $T_3$,存在开集 $O_x,O_y$,使得
\begin{equation}
x\in O_x, \quad y\in y\subset O_y, \quad O_x\cap O_y=\emptyset.~
\end{equation}
因此,正则空间满足 $T_2$,从而是Hausdorff 空间。

\textbf{证毕!}

然而,上述定理的逆命题是不成立的。


\begin{definition}{正规空间}
满足正规分离性公理的 $T_1$ 空间称为\textbf{正规的}(Normal)。
\end{definition}

\begin{theorem}{}\label{the_DisAx_2}
正规空间满足 $T_3$ 分离性公理。
\end{theorem}

\textbf{证明:}设 $x$ 是任一正规空间中的一点,$A$ 是不包含 $x$ 的正规空间中的闭集。由\autoref{the_DisAx_1} ,单点集 $x$ 是闭集,因此由正规分离性定理,存在开集 $O_x,O_A$ 满足
\begin{equation}
x\in x\subset O_x, \quad A\subset O_A,\quad O_x\cap O_A=\emptyset.~ 
\end{equation}
这表明了 $T_3$ 分离性公理对正规空间成立。

\textbf{证毕!}

\begin{theorem}{}
度量空间\upref{Metric}是正规空间。
\end{theorem}

\textbf{证明:}设 $X,Y$ 是度量空间中的两个不相交的闭集。于是由\autoref{the_DisAx_2} ,对每一 $x\in X$,都存在 $x$ 的邻域 $O_x$,使得 $O_x\cap Y=\emptyset$。于是存在开球 $B(x,r)\subset O_x$,其中 $r>0$。因此,每一 $x$ 和 $Y$ 的距离为某一正数 $d_x$。类似的,每一 $y\in Y$ 和 $X$ 的距离是某一正数 $d_y$。显然,下面两个开集分别包含 $X,Y$
\begin{equation}
U=\bigcup_{x\in X} B(x,d_x/2),\quad V=\bigcup_{y\in Y} B(y,d_y/2)~.
\end{equation}
下面证明它们的交是空的。假设 $z\in U\cap V$,于是存在 $x_0\in X,y_0\in Y$,使得 $d(z,x_0)<d_{x_0}/2,d(z,y_0)<d_{y_0}/2$。不是一般性,设 $d_{x_0}\leq d_{y_0}$,因此
\begin{equation}
d(x_0,y_0)\leq d(z,x_0)+d(z,y_0)<d_{x_0}/2+d_{y_0}/2\leq d_{y_0}.~
\end{equation}
即, $x_0$ 属于 $B(y_0,d_{y_0})$,但这与 $d_{y_0}$ 的定义矛盾,于是 $U\cap V=\emptyset$。于是度量空间满足正规性公理。

此外,度量空间中任意两不同点都被某两个不相交的开球分开,因而满足 $T_1$ 分离性公理。于是命题得证。

\textbf{证毕!}

\begin{corollary}{}
任一度量空间的子空间仍是正规空间。
\end{corollary}
\textbf{证明:}任一度量空间的zi'k'jia

\textbf{证毕!}

