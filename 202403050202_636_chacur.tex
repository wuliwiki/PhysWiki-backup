% 偏微分方程的分类与特征线
% keys 特征线|PDE|偏微分方程
% license Usr
% type Tutor

\begin{issues}
\issueMissDepend
\issueTODO
\issueDraft
\end{issues}

特征线法又被称为达朗贝尔法和行波法,一般适用于解初值问题的一阶 PDE,对二阶 PDE 的分类也颇有帮助。

特征线有如下性质,
\begin{enumerate}
\item 对于 $n$ 维(这里包含时间维)的 PDE,其特征线总是 $n-1$ 维的。
\item 解的“间断”性质通过、且仅能通过特征线传播。
\item 特征线上 PDE 的解是一样的,特征线的斜率是解“传播”的速度。
\end{enumerate}
其中性质 $2$ 对应着如果不存在特征线(特征线不是实的,类似一元二次方程没有实数根称为根不存在),那么 PDE 的解连续(这就是 elliptic PDE)。

\subsection{一维一阶 PDE}
利用一个一维一阶的 PDE 举例来说明如何求解 PDE 的特征线。

\begin{theorem}{}
对于关于 $u(x, t)$ 的 PDE:
\begin{equation}
\frac{\partial u}{\partial t} + A(x, t) \frac{\partial u}{\partial x} + B(x, t) u = f(x, t) ~,
\end{equation}
其中,$-\infty < x < +\infty, t > 0 ~$,设特征曲线族为 $x = x(t, \tau)$,是下面 ODE 的解,
\begin{equation}
~.
\end{equation}

\end{theorem}
% todo

很多人都对偏微分方程为什么用圆锥曲线来分类有疑惑,一个偏微分方程为什么会跟平面上曲线的分类有关呢?对 PDE(偏微分方程)进行分类又对解其有何帮助?下面来讨论这些问题。

\subsection{二阶线性 PDE 的分类}
在一般的数理方程或介绍 PDE 的书籍中,会把二阶线性 PDE 作为着重点来讲并一般将之分为三类,分别是
\begin{enumerate}
\item 椭圆类(elliptic PDE),例如泊松方程 $\laplacian u = f(x,y,z)$;
\item 抛物线类(parabolic PDE),例如热传导方程 $k \laplacian T = \frac{\partial T}{\partial t}$;
\item 双曲类(hyperbolic PDE),例如波动方程 $\laplacian u = \frac{1}{c^2} \frac{\partial^2 u}{\partial t^2}$。
\end{enumerate}
