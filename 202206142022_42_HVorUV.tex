% 埃尔米特矢量空间(酉空间)
% 埃尔米特矢量空间|酉空间|施瓦茨不等式

\begin{issues}
\issueTODO
\end{issues}

\pentry{欧几里得矢量空间\upref{EuVS}}
$\mathbb{R}$ 上的欧几里得矢量空间的度量关系完全可由纯量积来刻画,这成为在复数域 $\mathbb{C}$ 上的矢量空间中引入纯量乘积的刺激因素.然而,在复矢量空间的情形,使用标准双线性型 $s(\bvec x,\bvec y)=\sum_{i}x_iy_i\;x_i,y_i\in\mathbb{C}$ 作为纯量积并不能胜任此任务,因为对 $\norm{x}\geq0$,模
\begin{equation}
\norm{\I\bvec x}=\sqrt{s(\I\bvec x,\I\bvec x)}=-\norm{\bvec x}\leq0
\end{equation}
要是想利用直观的矢量长度的概念,上面的纯量乘积定义显然是不能接受的.但是,使用正定的埃米尔特型(\autoref{HeFor_def1}~\upref{HeFor})作为纯量乘积的定义却是适合的.
\subsection{埃尔米特矢量空间(酉空间)}
\begin{definition}{}
域 $\mathbb{C}$ 上一个有限维矢量空间 $V$ 配备一个正定埃尔米特型$(\bvec x|\bvec y):=f(\bvec x|\bvec y)$,则称为\textbf{埃尔米特矢量空间}(或\textbf{酉空间}).复数 $(\bvec x|\bvec y)$ 称为是矢量 $\bvec x,\bvec y\in V$ 的\textbf{纯量积}(或\textbf{内积}).
\end{definition}
列出纯量积的性质:
\begin{enumerate}
\item $(\bvec x|\bvec y)=\overline{(\bvec x|\bvec y)}$;
\item $(\alpha\bvec x+\beta\bvec y|\bvec z)=\alpha^*(\bvec x|\bvec z)+\beta(\bvec y|\bvec z)$;
\item $(\bvec x|\bvec x)\geq0$,仅当 $\bvec x=\bvec 0$ 时等式成立.
\end{enumerate}

这里,仍用 $(*|*)$ 表示纯量乘积,是因为将矢量空间限制在实数域上以上性质仍成立,并不会发生任何矛盾.

有了纯量乘积的定义以后,和欧几里得矢量空间情形完全类似的,在埃尔米特空间,有
\begin{definition}{模}
称 $\norm{\bvec v}=\sqrt{(\bvec v|\bvec v)}$ 为矢量 $\bvec v$ 的\textbf{模}.
\end{definition} 
\begin{theorem}{施瓦茨不等式}
\begin{equation}\label{HVorUV_eq1}
\abs{(\bvec x|\bvec y)}\leq \norm{\bvec x}\cdot \norm{\bvec y}
\end{equation}
仅当 $\bvec y=\lambda\bvec x,\lambda\in\mathbb{C}$ 时,等式成立.
\end{theorem}
\textbf{证明:}将复数记为指数形式,即 $(\bvec x|\bvec y)=\abs{(\bvec x|\bvec y)}\E^{\I \varphi}$.那么由纯量乘积的正定性,对 $\forall t\in\mathbb{R}$
\begin{equation}
\begin{aligned}
&(\bvec x t+\bvec y\E^{-\I\varphi}|\bvec x t+\bvec y\E^{-\I\varphi})\\
&=\norm{\bvec x}^2t^2+\qty((\bvec x|\bvec y)\E^{-\I\varphi}+\overline{(\bvec x|\bvec y)}\E^{\I\varphi})t+\norm{\bvec y}^2\\
&=\norm{\bvec x}^2t^2+2\abs{(\bvec x|\bvec y)}t+\norm{\bvec y}^2\geq0
\end{aligned}
\end{equation}
由判别式即得\autoref{HVorUV_eq1} .

\textbf{证毕!}
\begin{corollary}{三角不等式}
\begin{equation}\label{HVorUV_eq2}
\norm{\bvec x\pm\bvec y}\leq\norm{\bvec x}+\norm{\bvec y}
\end{equation}
\end{corollary}
证明完全和欧氏矢量空间一样(\autoref{EuVS_cor1}~\upref{EuVS}).
\begin{example}{}
在 $\mathbb{C}$ 上的矢量空间 $C_2(a,b)$ (\autoref{EuVS_ex1}~\upref{EuVS}) 和 $P_n$ 上附加纯量乘积
\begin{equation}
(f|g)=\int_a^{b}f(x)\overline{g(x)}\dd x
\end{equation}
显然是个埃尔米特空间.利用\autoref{HVorUV_eq2} ,即得
\begin{equation}
\sqrt{\int_a^b\abs{f(x)\pm g(x)}^2\dd x}\leq\sqrt{\int_a^b\abs{f(x)}^2\dd x}+\sqrt{\int_a^b\abs{g(x)}^2\dd x}
\end{equation}
\end{example}
由\autoref{HVorUV_eq1} ,存在唯一角 $\varphi,0\leq\varphi\leq\frac{\pi}{2}$ 使得
\begin{equation}
\cos\varphi=\frac{\abs{(\bvec x|\bvec y)}}{\norm{\bvec x}\cdot \norm{\bvec y}}
\end{equation}
\subsection{正交性}
\begin{definition}{正交}
若 $(\bvec x|\bvec y)=0$,则称 $\bvec x,\bvec y$是\textbf{正交的},记作 $\bvec x\perp\bvec y$.
此外若还成立 $(\bvec x|\bvec x)=(\bvec y|\bvec y)=1$,则称 $\bvec x$ 和 $\bvec y$ \textbf{标准正交}.
\end{definition}
在正交性方面,也完全类似于欧几里得矢量空间情形,\autoref{EVOIOG_the1}~\upref{EVOIOG}、\autoref{EVOIOG_the2}~\upref{EVOIOG}仍成立,只需将域 $\mathbb{R}$ 上的欧氏矢量空间替换为域 $\mathbb{C}$ 上的埃尔米特矢量空间.

注意:由于 $n$ 维的埃尔米特矢量空间必有 $n$ 个线性无关的基底,而\autoref{EuVS_the1}~\upref{EuVS}保证了其上必存在标准正交基底.
\begin{example}{}
试证明在埃尔米特空间,也有类似于“毕达哥拉斯定理”成立,即:对两两正交的矢量 $\bvec v_1,\cdots,\bvec v_n$,成立
\begin{equation}
\abs{\abs{\bvec v_1+\cdots+\bvec v_n}}^2=\abs{\abs{\bvec v_1}}^2+\cdots+\abs{\abs{\bvec v_n}}^2
\end{equation}
\end{example}

\begin{theorem}{}
设 $(\bvec e_1,\cdots,\bvec e_n)$ 是埃尔米特空间 $V$ 上的一个标准正交基底,那么
\begin{enumerate}
\item \begin{equation}\label{HVorUV_eq3}
\forall\bvec x\in V,\bvec x=\sum_{i}(\bvec e_i|\bvec x)\bvec e_i
\end{equation}
\item 帕塞瓦尔等式
\begin{equation}
\forall\bvec x,\bvec y\in V,(\bvec x|\bvec y)=\sum_{i}(\bvec x|\bvec e_i)(\bvec e_i|\bvec y)
\end{equation}
\item \begin{equation}
\\bvec x\in V,\Rightarrow\norm{\bvec x}^2=\sum_{i}\abs{(\bvec e_i|\bvec x)}^2.
\end{equation}
\end{enumerate}
\end{theorem}
\textbf{证明:}\begin{enumerate}
\item 设 $\bvec x=\sum_{i}x_i\bvec e_i$,则
\begin{equation}
(\bvec e_i|\bvec x)=\sum_{j}x_j(\bvec e_i|\bvec e_j)=x_i
\end{equation}
\item 由\autoref{HVorUV_eq3} 
\begin{equation}
(\bvec x|\bvec y)=(\bvec x|\sum_{i}(\bvec e_i|\bvec y)\bvec e_i)=\sum_{i}(\bvec e_i|\bvec y)(\bvec x|\bvec e_i)=\sum_{i}(\bvec x|\bvec e_i)(\bvec e_i|\bvec y)
\end{equation}
\item 由\autoref{HVorUV_eq3} 直接证得!
\end{enumerate}
\textbf{证毕!}
\subsection{埃尔米特空间的同构}
\begin{theorem}{}
任意两个维数相同的埃尔米特矢量空间 $V,V'$ 都是同构的.即存在矢量空间的同构映射 $f:V\rightarrow V'$,它还保持纯量乘积:
\begin{equation}
(\bvec x|\bvec y)=(f(\bvec x)|f(\bvec y))'
\end{equation}
其中,$(*|*)'$ 是 $V'$ 上的纯量乘积.
\end{theorem}
\textbf{证明:}和欧几里得矢量空间一样,这样的同构映射是
\begin{equation}
f:\bvec x=\sum_{i}x_i\bvec e_i\mapsto\bvec x'=\sum_i x_i\bvec e'_i
\end{equation}
其中,$\{\bvec e_i\}$ 和 $\{\bvec e'_i\}$ 分别是空间 $V,V'$ 的标准正交基底.这显然保持纯量乘积.

从上面可以看到,欧几里得矢量空间的很多性质都可以直接推广到埃尔米特矢量空间.然而,在欧几里得矢量空间的情形,可以把欧几里得矢量空间和其对偶空间等同起来;而在埃米尔特空间,并不存在这样的等同关系.这是因为,欧几里得矢量空间和其对偶空间存在着自然的同构映射(\autoref{EVOIOG_the3}~\upref{EVOIOG});而对埃米尔特矢量空间,并没有这样的自然同构存在,究其原因,是因为欧氏情形纯量积是双线性的,而埃氏情形则是半双线性的,而矢量空间的同构必须是线性的.
\subsection{酉群}
 $n$ 维欧氏情形里,从一个标准正交基底到另一个标准正交基底的转换矩阵称为\textbf{正交矩阵},它们构成一个群——\textbf{正交群},记作 $O(n)$ (\autoref{EVOIOG_sub1}~\upref{EVOIOG}).同样的,在 $n$ 维埃氏情形,从一个标准正交基底到另一个标准正交基底的转换矩阵称为\textbf{酉矩阵},它们构成的群称为\textbf{酉群},记作 $U(n)$.关于酉群的具体介绍查看词条酉群\upref{UQ}.
