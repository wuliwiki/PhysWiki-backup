% 平方反比定律
% license CCBYSA3
% type Wiki

(本文根据 CC-BY-SA 协议转载自原搜狗科学百科对英文维基百科的翻译)
\begin{itemize}
\item 在物理学中,如果一个物理定律中,某个物理量在空间中某点处的数值或强度与该点与该物理量源点的距离的平方成反比,则这个物理定律称为一个平方反比定律。其根本原因在于点源辐射形式的物理量因在三维空间中的扩散而按与点源距离的平方而衰减。
\item 雷达信号能量在传输和反射过程中都会在空间中扩散,因此在往返路径中均按照平方反比律衰减,这意味着雷达接收的反射信号能量将与距离的四次方成反比。
\item 为了防止在信号传播过程中的能量衰减,可以使用某些方法限制其在空间中的扩散,例如波导管可以让电磁波在其中传播,类似于排水管道之于水的作用,再如枪管将高温气体的膨胀限制在单一方向,以减少气体推进子弹加速过程中的能量损失。
\end{itemize}
\subsection{公式}
数学表示:
\begin{equation}
intensity \propto \frac{1}{distance^2}~
\end{equation}

也可以写为:\begin{equation}
\frac{intensity_1}{intensity_2}=\frac{distance_2^2}{distance_1^2}~ 
\end{equation}

或者写作守恒量的形式:\begin{equation}
intensity_1 * distance_1^2 =intensity_2*distance_2^2~
\end{equation}

符合平方反比律的物理量在空间中的向量场分布在某一点的散度与该点处的源强成正比,因此若一点处无源,则其散度为零。牛顿万有引力定律及电、磁、光、声音和辐射等多种物理现象都遵循平方反比定律。
\subsection{ 平方反比定律的论证}
平方反比定律适用于力、能量和其他从三维空间中的点源均匀地向外辐射的保守物理量。因为球体的表面积$(4\pi r^2)$与半径的平方成正比,所以当发射的辐射远离源时,它会分布在一个面积与到源距离的平方成正比的区域。因此单位面积(正对点源方向)的辐射强度与该处与点源距离的平方成反比。高斯定律也适用于任何满足平方反比定律的物理量。

\subsection{示例}
\subsubsection{3.1 引力}
引力是有质量的物体之间的吸引力。牛顿定律表明:

“两个质点之间的引力与它们质量的乘积成正比,与它们距离的平方成反比。引力使两个质点沿着它们连线方向相互吸引。”

壳层定理(shell theorem)表明,如果一个物体中的质量分布是球对称的,那么该物体可以被视为质点而不需要作近似。否则,如果我们想计算大质量物体之间的引力,就需要计算所有的点-点引力矢量并进行求和,而净引力可能并不精确得满足平方反比定律。然而,如果质量体之间的距离比它们的尺寸大得多,那么在计算重力时,将物体等价为位于物体质心的质点是合理有效的近似。

引力符合平方反比律的观点最早在1645年由伊斯梅尔·布列多斯(Ismael Bullialdus)在1645年提出,然而布列多斯不接受开普勒的第二和第三定律,也不赞同克里斯蒂安·惠更斯(Christiaan Huygens)对圆周运动(一种存在中心力牵引的直线运动)的计算结果。事实上,布列多斯认为太阳对行星的作用力在远日点是吸引的,而在近日点是排斥的。1666年,罗伯特·胡克(Robert Hooke)和乔瓦尼·阿方索·博雷利(Giovanni Alfonso Borelli)在论述中都认为是引力一种吸引力[1] (根据胡克3月21日在伦敦皇家学会做的题为“关于万有引力”的演讲;[2] 博雷利发表于1666年晚些时候的论著《行星理论》[3])。1670年,胡克在格雷欣的讲座中阐明引力定律适用于所有天体,并论述了引力随距离减小以及在没有引力的情况下天体将沿直线运动的原理。到1679年,胡克认为引力遵循平方反比关系,并在给艾萨克·牛顿的信中传达了这一点:[4]我的假设是物体之间的引力总是与其中心距离的平方成反比。[5]

胡克一直对牛顿声称发明了这一原理而耿耿于怀,尽管牛顿在1686年的《原理》一书中承认胡克、雷恩和哈雷分别独立的意识到了太阳系中引力的平方反比定律,[6]并对布列多斯给予了一定的肯定。[7]
\subsubsection{3.2 静电学}
两个带电粒子之间的引力或斥力,除了与电荷的乘积成正比外,还与它们之间距离的平方成反比;这就是众所周知的库仑定律,其指数与2的偏差小于$10^{15}$。[8]
\subsubsection{3.3 光和其他电磁辐射}
点源发出的光或其他线性波的强度(也称为照度或辐照度)是指垂直于电源方向单位面积上的能量,该值与同光源距离的平方成反比;所以一个同样大小的物体在距离光源两倍远的位置单位时间内接收到的能量是原来的四分之一。

更一般地,球面波波前处的辐照度(强度)与同光源距离的平方成反比(假设没有吸收或散射损失)。

例如,太阳的辐射强度在水星处($0.387$个天文)为每平方米$9126$瓦,而在地球处($1$ 个天文单位)仅为每平方米只有$1367$瓦——距离增加大约三倍,辐射强度减少约九倍。

对于非各向同性辐射源,如抛物面天线、前照灯和激光等,其等效辐射源位置位于发光孔后面很远的地方。当你离辐射源很近时,远离光源不大的距离就能使半径翻倍,因此信号下降很快对于一些方向性很好的辐射,如激光,你距离等效离辐射源非常远,因此必须经过很远的距离才能使半径加倍从而减小信号。这意味着相对于向空间各方向均匀发出辐射的各向同性天线,仅向一个方向发射窄波束的天线可以产生更强的信号,这种现象称为天线增益。

在摄影和舞台照明中,平方反比定律用于确定物体靠近或远离光源时光照的变化。为进行快速计算光照变化,只需记住距离加倍时照度降低到四分之一;[9] 类似地,距离增加$1.4$倍(约等于$2$的平方根)时照度减半,距离减小到$0.7$倍(约为$\frac{1}{2}$的平方根)时照度加倍。当光源不是点光源时,平方反比定律通常仍然是一个有效的近似值;当光源的尺寸小于到物体距离的五分之一时,计算误差小于$1 \%$。[10]

间接电离辐射的电磁能通量密度 随离点源距离的增加的减少量可以用平方反比定律来计算。由于点源的辐射沿径向方向,垂直射线方向的截面构成一个球面,其面积是$4\pi r^2$,其中$r$是与源的径向距离。这一规律在放射性诊断和放射治疗方案制定中特别重要,尽管这种比例关系在实际情况下一般不严格成立,除非放射源尺寸远远小于与辐照物的距离。

\textbf{例子}

假定某个全方向且各向同性的一个点辐射源的总辐射功率为$P$。当与源距离远大于源的尺寸时,随着离源距离的增加,辐射功率分布在越来越大的球面上。因为半径为r的球体的表面积是$A=4\pi r^2$,所以距离源$r$处的辐射强度(单位面积上的功率)是
\begin{equation}
I=\frac{P}{A}=\frac{P}{4\pi r^2}~
\end{equation}

当距离加倍时,辐射强度减小为四分之一;以分贝为单位测量,每增加一倍距离,将减少$6.02$分贝。

\subsubsection{3.4 气体中的声波}
