% 用算法求解任意三角形面积
% 三角形|算法|行列式|线性代数|数形结合|向量|矩阵|面积

\pentry{矢量叉乘\upref{Cross}}

有些时候我们拿到一个三角形,不知道它的面积如何计算,有些人就会暴力地补成一个矩形,剪掉两个三角形,结果就特别复杂.如果我们学会了线性代数,解决这个问题就异常简单.
首先需要定义矢量的叉乘.

本网站的矢量叉乘\upref{Cross}中\autoref{Cross_def1} (略有改动):
\begin{definition}{矢量叉乘} \label{TriAre_def1}
若$\vec a \cross \vec b = \vec c $
\begin{enumerate}s
\item $\vec c$ 的模长等于 $\vec a, \vec b$ 的模长之积与夹角 $\theta$ ($0 \leqslant \theta \leqslant \pi$)的正弦值相乘.
\begin{equation}\label{TriAre_eq1}
\abs{\vec c}  = \abs{\vec a} \abs{\vec b} \sin\left<a, b \right>
\end{equation}
\item $\vec c$ 的方向垂直于 $\vec a, \vec b$ 所在的平面,且由右手定则\upref{RHRul}决定.
\end{enumerate}
\end{definition}
\begin{figure}[ht]
\centering
\includegraphics[width=14.25cm]{./figures/TriAre_1.png}
\caption{矢量叉乘与三角形面积} \label{TriAre_fig1}
\end{figure}
如\autoref{TriAre_fig1} 所示,
\begin{equation}
\begin{aligned}
\lvert\vec a \cross \vec b\rvert &= \lvert\vec b\rvert \lvert\vec a\rvert\sin\theta \\
&= \lvert\vec b\rvert BH \\
&= AC \cdot BH \\
&= S_{ABCD}
\end{aligned}
\end{equation}

所以两个矢量的叉乘大小等于两者夹成的平行四边形的面积.
\begin{lemma}{行列式计算叉乘}
用行列式可以计算叉乘:
\begin{equation}
\lvert\lvert \begin{bmatrix}
\uvec i & \uvec j & \uvec k \\
x_a & y_a & z_a\\
x_b & y_b & z_b
\end{bmatrix}\rvert\rvert = \lvert\lvert \vec a \cdot \vec b \rvert\rvert
\end{equation}
\end{lemma}

设$A(x_0, y_0), B(x_1, y_1), C(x_2, y_2)$,可得
\begin{equation}
\begin{aligned}
\vec{AB}=\begin{pmatrix}x_1-x_0 \\ y_1-y_0\\0\end{pmatrix}\\
\vec{AC}=\begin{pmatrix}x_2-x_0 \\ y_2-y_0\\0\end{pmatrix}\\
\end{aligned}
\end{equation}

因此:
\begin{equation}
\begin{aligned}
S_{ABCD}&=\lvert\lvert \begin{bmatrix}
\uvec i & \uvec j & \uvec k \\
x_1-x_0 & y_1-y_0 & 0\\
x_2-x_0 & y_2-y_0 & 0
\end{bmatrix}\rvert\rvert \\
&= \lvert \begin{bmatrix}
1 & 1 & 1 \\
x_1-x_0 & y_1-y_0 & 0\\
x_2-x_0 & y_2-y_0 & 0
\end{bmatrix}\rvert\\
&= \lvert \begin{bmatrix}
x_1-x_0 & y_1-y_0 \\
x_2-x_0 & y_2-y_0 
\end{bmatrix}\rvert \\
&=(x_1-x_0)(y_2-y_0)-(x_2-x_0)(y_1-y_0)
\end{aligned}
\end{equation}
$$\therefore S_{\Delta ABC}=\frac{1}{2} S_{ABCD}$$

因此,可以用编程来实现(此处使用的是Python,可用于开发一个类似GeoGebra的几何计算器):
\begin{lstlisting}[language=python]
# 未全部完成
class Point:
    def __init__(self, x, y):
        self.x = x
        self.y = y
    ...

class Triangle:
    def __init__(self, p1: Point, p2: Point, p3: Point):
        self.p = (p1, p2, p3)
    
    def area(self): -> float
        return abs((self.p2.x-self.p1.x)*(self.p3.y-self.p1.y)-\
                   (self.p3.x-self.p1.x)*(self.p2.x-self.p1.y))/2
    ...
\end{lstlisting}