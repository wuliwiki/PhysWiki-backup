% 高斯波包
% keys 高斯分布|波包|光学|量子力学
% license Xiao
% type Tutor

\pentry{波包\upref{WvPck}, 高斯分布\upref{GausPD}}

\begin{figure}[ht]
\centering
\includegraphics[width=14.25cm]{./figures/98b376107396a32e.pdf}
\caption{高斯波包(\autoref{eq_GausPk_1} ),蓝色为实部,红色为虚部, $x_0 = 0$, $A_0 = 1$, $a = 1/20$, $k_0 = 5$。} \label{fig_GausPk_1}
\end{figure}

\footnote{参考 Wikipedia \href{https://en.wikipedia.org/wiki/Wave_packet}{相关页面}。}\textbf{高斯波包(Gaussian wave packet)}是指轮廓为高斯分布的波包, 在光学和量子力学中有重要应用。 高斯波包用复函数表示为($A_0$ 为复数)
\begin{equation}\label{eq_GausPk_1}
f(x) = A_0 \E^{-a(x-x_0)^2}\E^{\I k_0 x}~,
\end{equation}

FWHMI (光强半高宽)是多少? 即 $f(x)^2$ 等于其峰值一半时的宽度。令
\begin{equation}
\E^{-2a\Delta x^2} = 1/2~,
\end{equation}
得
\begin{equation}
\mathrm{FWHMI} = 2\Delta x = \sqrt{\frac{2\ln 2}{a}}~.
\end{equation}

它的shi'lian

\subsection{频谱}
\pentry{傅里叶变换(指数)\upref{FTExp}}
(未完成:哪里有介绍频谱的概念?)

要求\autoref{eq_GausPk_1} 的傅里叶变换 $g(k)$, 由\autoref{ex_FTExp_1}~\upref{FTExp}以及傅里叶变换性质\autoref{eq_FTExp_4}~\upref{FTExp}和\autoref{eq_FTExp_7}~\upref{FTExp}得
\begin{equation}
g(k) = A_0\sqrt{\frac{\pi}{a}} \exp[-\frac{(k-k_0)^2}{4a}]~.
\end{equation}

\subsection{高斯和 cos2 波包比较}
$\cos^2$ 波包也叫 $\sin^2$ 波包,比起高斯波包,它的优点是存在明确的范围。 它的函数形式为
\begin{equation}
f(x) = \leftgroup{
&A_0\cos^2[\pi(x-x_0)/L] \E^{\I k_0x} && (\abs{x-x_0} < L/2)\\
&0 && (\text{otherwise})
}~.
\end{equation}
其中 $L$ 是波包的总长度。 它的 FWHMI 为
\begin{equation}
\text{FWHMI} = \frac{2}{\pi}\opn{acos}(2^{-1/4}) L~.
\end{equation}
画图对比如下(代码见文末):
\begin{figure}[ht]
\centering
\includegraphics[width=13cm]{./figures/925904dd88cd4821.pdf}
\caption{高斯波包和 cos2 波包的对比} \label{fig_GausPk_2}
\end{figure}

\subsection{附:Matlab 画图代码}
\begin{lstlisting}[language=matlab, caption=sin2\_gaussian\_compare.m]
% plot Gaussian vs cos2 profile

% gaussian
FWHMI = 1;
a = iFWHMIexp(FWHMI);
x = linspace(-2*FWHMI, 2*FWHMI, 1000);
field_gauss = exp(-a.*x.^2);

% cos2
field_cos2  = zeros(size(x));
dur_cos2 = FWHMI / FWHMIsin2;
mark = abs(x) < dur_cos2/2;
field_cos2(mark) = cos((pi/2)*x(mark)/(dur_cos2/2)).^2;

% plot field profile
figure;
subplot(2, 1, 1); hold on;
axis([min(x), max(x), 0, 1.1]);
plot_vert(-FWHMI/2, 'c--');
plot_vert(FWHMI/2, 'c--');
plot_hori(sqrt(1/2), 'c--');
plot(x, field_gauss, 'r');
plot(x, field_cos2, 'b--');
legend({'', '', '', 'Gaussian', 'cos2'});
% xlabel('t [FWHM]');
ylabel('field');
title('Gaussian vs cos2 profile (lines show FWHMI)');

% plot intensity profile
subplot(2, 1, 2); hold on;
axis([min(x), max(x), 0, 1.1]);
plot_vert(-FWHMI/2, 'c--');
plot_vert(FWHMI/2, 'c--');
plot_hori(1/2, 'c--');
plot(x, field_gauss.^2, 'r');
plot(x, field_cos2.^2, 'b--');
xlabel('t [FWHM]');
ylabel('intensity');
\end{lstlisting}
