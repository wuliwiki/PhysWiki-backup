% 类氢原子斯塔克效应(微扰)

\begin{issues}
\issueDraft
\end{issues}

\pentry{不含时微扰理论\upref{TIPT}}

微扰理论($\mathcal{E_z}$ 是 $z$ 方向电场):
\begin{equation}
H' = \mathcal{E_z} z
\end{equation}
矩阵元为
\begin{equation}\label{HStark_eq1}
H'_{l',l} = \mathcal{E_z}\mel{\psi_{n,l',m}}{z}{\psi_{n,l,m}}
\end{equation}


\begin{example}{氢原子 $n=2$ 的斯塔克效应}
根据\autoref{HDipM_tab1}~\upref{HDipM}, \autoref{HStark_eq1} 为
\begin{equation}
\mat H' = -3\mathcal{E_z}\pmat{0 & 1\\ 1 & 0}
\end{equation}
本征值为 $E_{\pm}^1 = \mp 3\mathcal{E_z}$, 好本征态为 $\ket{2\pm} = (\ket{20} \pm \ket{21})/{\sqrt 2}$.

\begin{figure}[ht]
\centering
\includegraphics[width=8cm]{./figures/HStark_1.png}
\caption{$\ket{2+}$ 的概率密度函数的 $x$-$z$ 切面, 可见电子向下偏移, 电场向上为正, 所以本征能量变小. $\ket{2-}$ 态是此图上下翻转, 本征能量变大.} \label{HStark_fig1}
\end{figure}

不要以为\autoref{HStark_fig1} 是外电场扭曲波函数的结果, $\ket{2\pm}$ 本身就是无电场的氢原子 $n=2$ 本征态. 施加了电场后波函数反而需要进一步修正.

如果初始时, 波函数处于 $n=2$ 子空间的任意状态, 例如 $\ket{20} = (\ket{2+} + \ket{2-})/\sqrt 2$, 那么当逐渐施加电场后, 波函数会如何变化? 根据渐进近似
\end{example}
