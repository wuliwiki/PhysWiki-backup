% 拉回映射
% license Usr
% type Tutor
\pentry{伴随映射\nref{nod_AdjMap}}{nod_712c}

\begin{issues}
\issueDraft
\end{issues}

设$f:M\rightarrow N$是光滑映射,则可以定义切空间之间前推映射:
\begin{equation}
f_*: T_p M\rightarrow T_{f(p)}N~,
\end{equation}
切空间是线性空间,因此可以诱导其对偶空间上的伴随映射(或称对偶映射)。简写该伴随映射为$f^*$:
\begin{equation}
f^*\equiv(f_*)^*:\,T^*_f(p)N\rightarrow T_p^*M~,
\end{equation}
使得其满足伴随映射的基本定义,即对于任意$X\in T_p M,\eta\in T^*_{f(p)}N$有:
\begin{equation}
f^*(\eta)X=f_*(X)\eta~.
\end{equation}
一般称如上定义的伴随映射为伴随$f$的\textbf{拉回(pull-back)},以示其把$N$上的余切向量“拉回”到$M$上这一特点。

既然拉回映射是把余切向量映射为余切向量,自然也可以拉回$N$上的余切向量场。
\begin{lemma}{}
设$g\in C^{\infty }N,\sigma$是$N$上的光滑余切场,则有
\begin{enumerate}
\item $G^*df=d(f\circ G).$
\item $G^*(g\sigma)=(g\circ G)G^*\sigma$
\end{enumerate}
\end{lemma}
\begin{theorem}{}
伴随\textbf{光滑映射}的拉回映射把光滑余切场映射为光滑余切场
\end{theorem}
\subsubsection{微分形式的拉回}
\begin{definition}{}
设$f:M\rightarrow N$是光滑映射,$\omega$是$N$上的$k$次微分形式,则可以定义微分形式的拉回,使之也是微分形式,满足对于任意$M$上的一组切向量$(X_1,X_2...X_k)$有:
\begin{equation}
f^*(\omega)(X_1,X_2...X_k)=\omega(f_*(X_1),f_*(X_2)...f_*(X_k))~.
\end{equation}
\end{definition}
由于微分形式的外积依然是微分形式,因此也可以被“拉回”。
\begin{theorem}{}
若$f:M\rightarrow N$是光滑的,$\omega,\eta$是$N$上的微分形式。那么有:
\begin{enumerate}
\item $f^*(\omega\wedge \eta)=f^*(\omega)\wedge f^*(\eta)$,即保微分形式的外代数同态。
\item 给定一个$N$上的一个局部坐标系$\{dy^{i_1}\}$,定义$k$形式$\omega$为$\omega=\Sigma_I$
\end{enumerate}
\end{theorem}