% 理想气体化学平衡条件
% 化学反应|化学平衡条件|理想气体

\pentry{多元系热力学导引\upref{mulTh}}

设在一组化学反应中,$i$ 组元的物质的量变动为 $\delta n_i$,它们之间满足一定的比例关系,简写为以下方程:
\begin{equation}
\delta n_i=\nu_i \delta n~.
\end{equation}

$\nu_i>0$ 表示生成物,$\nu_i<0$ 表示反应物。这一般用来区分化学方程式的左侧和右侧。

在等温等压条件下,化学平衡要求平衡态的吉布斯函数最小,即\textbf{吉布斯判据}(\autoref{eq_GibbsG_2}~\upref{GibbsG})。
\begin{equation}
\delta G=\sum_i \mu_i\delta n_i=\delta n\sum_i \nu_i\mu_i
\end{equation}

于是我们得到了化学平衡条件。
\begin{equation}\label{eq_ICheEq_2}
\sum_i\nu_i\mu_i=0
\end{equation}

要注意的是,因为化学反应的系统是多元系\upref{mulTh},这里组元 $i$ 的化学势 $\mu_i$ 是偏摩尔吉布斯函数,它是温度、压强以及各组员摩尔分数的函数。

\subsection{理想气体化学平衡}

我们考虑的体系为温度 $T$ 压强 $P$ 的混合理想气体,第 $i$ 个组元的摩尔分数为 $x_i$,则该组元分压为 $x_iP$,化学势为 $\mu_i=g(T,p_i)$,$g$ 为纯 $i$ 组元的化学势,因此根据理想气体吉布斯函数的公式(\autoref{eq_GibbsG_3}~\upref{GibbsG}),有下列等式
\begin{equation}\label{eq_ICheEq_1}
\mu_i=RT(\phi_i(T)+\ln(x_iP))
\end{equation}

如果具体地把 $\phi_i$ 写出来,那么有(这里小写字母 $h_0,c_p,s_0$ 表示每摩尔的焓常量、定压热容、熵常量)
\begin{equation}
\phi_i=\frac{h_{0,i}}{RT}-\int\frac{\dd T}{RT^2}\int c_{p,i}\dd T-\frac{s_{0,i}}{R}
\end{equation}

混合理想气体的摩尔吉布斯函数为
\begin{equation}
G(T,P,n_1,\cdots,n_k)=\sum_i n_iRT[\phi_i(T)+\ln(x_iP)]
\end{equation}

根据吉布斯判据推出的化学平衡条件\autoref{eq_ICheEq_2} ,有
\begin{equation}\label{eq_ICheEq_3}
\sum_i\nu_i[\phi_i(T)+\ln (x_iP)]=0
\end{equation}
定义\textbf{定压平衡常量} $K_p(T)$:
\begin{equation}\label{eq_ICheEq_5}
\ln K_p=-\sum_i\nu_i\phi_i(T)
\end{equation}

那么\autoref{eq_ICheEq_3} 可以化简为
\begin{equation}\label{eq_ICheEq_4}
\begin{aligned}
&\prod_i (x_iP)^{\nu_i}=K_p(T)\\
&\prod_i x_i^{\nu_i}=P^{-\nu} K_p(T),(\nu=\sum_i\nu_i)
\end{aligned}
\end{equation}

\autoref{eq_ICheEq_4} 就是化学反应的平衡条件。当等式不成立,那么化学反应就要进行。根据吉布斯判据,化学反应正向进行的条件是:
\begin{equation}\label{eq_ICheEq_6}
\begin{aligned}
&\sum_i \nu_i\mu_i<0\\
&\Rightarrow \prod_i x_i^{\nu_i}<P^{-\nu}K_p(T)
\end{aligned}
\end{equation}

\begin{example}{四氧化二氮的分解}\label{ex_ICheEq_1}
四氧化二氮的分解式为
\begin{equation}
N_2O_4\rightleftharpoons 2NO_2
\end{equation}
设组元 $1$ 为 $NO_2$,组元 $2$ 为 $N_2O_4$,那么
\begin{equation}
\begin{aligned}
&\nu_1=2,\nu_2=-1,\\
&2\mu_1-\mu_2=0
\end{aligned}
\end{equation}
设初态有 $n_0 \rm{mol}$ 的 $N_2O_4$,反应平衡时 $N_2O_4$ 的物质的量为 $n_0(1-\epsilon) \rm{mol}$,也就是说有 $n_0 \epsilon \rm{mol}$ 的反应物分解成了生成物(\textbf{分解度}为 $\epsilon$)。生成物为 $2n_0\epsilon \rm{mol}$,因此总物质的量为 $n_0(1+\epsilon)$。各组分摩尔分数为
\begin{equation}
x_1=\frac{2\epsilon}{1+\epsilon},x_2=\frac{1-\epsilon}{1+\epsilon}
\end{equation}
代入平衡条件 \autoref{eq_ICheEq_6} 可得
\begin{equation}
K_p(T)=\frac{4\epsilon^2}{1-\epsilon^2}p
\end{equation}
如果平衡常量 $K_p$ 已知,就可以求出分解度与温度压强的关系。
\end{example}


\subsection{化学反应分解度与温度压强的关系}
由\autoref{eq_ICheEq_6} 可知,$P^{-\nu}K_p(T)$ 的增大可以促进化学反应的进行。所以当 $\nu>0$ 时,压强的减小促进反应的进行,$\nu<0$ 时,压强的增大不利于反应的进行。下面我们要探究的是温度的变化对化学反应分解度的影响(分解度是对反应程度的定量描述。分解度越大,反应平衡时的化学反应正向进行程度更大)。

将\autoref{eq_GibbsG_4}~\upref{GibbsG}代入\autoref{eq_ICheEq_5} 可以得到平衡常量 $K_p$ 的热力学公式(式中小写字母 $h_0,c_p,s_0$ 代表摩尔焓常量、摩尔定压热容和摩尔熵常量,下标 $i$ 表示第 $i$ 个组元):
\begin{equation}
\ln K_p=-\frac{\sum_i \nu_i h_{0,i}}{RT}+\frac{1}{R}\sum_i\nu_i\int \frac{\dd T}{T^2}\int c_{p,i}\dd T + \frac{\sum_i \nu_i s_{0,i}}{R}
\end{equation}

如果温度的变化范围不大,那么气体热容可以看成常量,该式可以用\autoref{eq_GibbsG_5}~\upref{GibbsG}简化为
\begin{equation}
\begin{aligned}
&\ln K_p=-\frac{A}{T}+C\ln T+B\\
&A=\frac{1}{R} \sum_i \nu_i h_{0,i},B=\frac{1}{R}\sum_i \nu_i(s_{0,i}-c_{p,i}),C=\frac{1}{R}\sum_i \nu_ic_{p,i}
\end{aligned}
\end{equation}

如果维持压强不变,改变温度会改变 $K_p$ 的大小:
\begin{equation}
\frac{\dd \ln K_p}{\dd T}=\frac{A}{T^2}+\frac{C}{T}=\frac{\sum_i\nu_ih_{0,i}}{RT^2}+\frac{\sum_i\nu_ic_{p,i}}{RT}
\end{equation}

第一项中 $\sum_i \nu_i h_{0,i}$ 代表的是反应后的焓减去反应前的焓,实际上就是\textbf{反应热} $\Delta H$,即经过化学反应产生的热量。对于许多化学反应过程,上式的第一项影响比第二项大。假如我们忽略第二项,则上式可以写成
\begin{equation}\label{eq_ICheEq_7}
\frac{\dd \ln K_p}{\dd T}=\frac{\Delta H}{RT^2}
\end{equation}
由此可知,当反应吸热时,$\Delta H>0$,$K_p$ 随温度的升高而增大,于是温度的升高将促进反应正向进行。如果 $\Delta H<0$,则温度的降低将促进反应正向进行。
\begin{example}{四氧化二氮的分解}
紧接着\autoref{ex_ICheEq_1} ,四氧化二氮的分解式 $\nu=\sum_i\nu_i>0$,说明总压强减小时分解度增大。实验指出,该反应是放热反应,即$\Delta H<0$。从\autoref{eq_ICheEq_7} 可以知道,升高温度时平衡将向生成 $N_2O_4$ 的方向移动。$NO_2$ 是红棕色气体,而 $N_2O_4$ 无色,在改变压强温度时两者浓度将发生变化,实验上容易验证上述结论。 
\end{example}
