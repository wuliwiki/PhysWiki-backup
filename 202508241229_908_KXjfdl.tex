% 柯西积分定理(综述)
% license CCBYSA3
% type Wiki

本文根据 CC-BY-SA 协议转载翻译自维基百科\href{https://en.wikipedia.org/wiki/Cauchy\%27s_integral_theorem}{相关文章}。

在数学中,柯西积分定理(也称为柯西–古尔萨定理)是复分析中的一个重要结论,以 奥古斯丁–路易·柯西(和爱德华·古尔萨的名字命名。该定理描述了复平面上全纯函数的路径积分性质。其核心内容是:如果函数$f(z)$在一个单连通域$\Omega$ 内是全纯的,那么对于 $\Omega$ 内的任何闭合路径$C$,沿着该路径的积分都为零:
$$
\int_{C} f(z)\, dz = 0.~
$$
\subsection{命题}
\subsubsection{复线积分的基本定理}
如果函数 $f(z)$ 在某个开区域 $U$ 上是全纯函数,且曲线 $\gamma$ 位于该区域内,从点 $z_0$ 延伸到点 $z_1$,则有:
$$
\int_{\gamma} f'(z)\,dz = f(z_1) - f(z_0).~
$$
此外,如果 $f(z)$ 在开区域 $U$ 内存在一个单值原函数,那么在该区域内,路径积分$\int_{\gamma} f(z)\,dz$对于所有路径来说都是路径无关的。

\textbf{在单连通区域上的表述}

设 $U \subseteq \mathbb{C}$ 是一个单连通开集,并且 $f: U \to \mathbb{C}$ 是一个全纯函数。如果 $\gamma: [a, b] \to U$ 是一条光滑的闭曲线,则有:
$$
\int_{\gamma} f(z)\,dz = 0.~
$$
其中,$U$ 是单连通集意味着它没有“洞”,换句话说,$U$ 的基本群是平凡的。

\textbf{一般形式}

设 $U \subseteq \mathbb{C}$ 是一个开集,且 $f: U \to \mathbb{C}$ 是一个全纯函数。如果 $\gamma: [a, b] \to U$ 是一条光滑的闭曲线,并且 $\gamma$同伦于一条常值曲线,则有:
$$
\int_{\gamma} f(z)\,dz = 0,~
$$
其中 $z \in U$。

一条曲线如果能通过在 $U$ 内的平滑同伦逐渐收缩到某一点(即常值曲线),则称这条曲线同伦于常值曲线。直观地说,这意味着可以在不离开区域 $U$ 的情况下,把闭合曲线“缩成一个点”。第一种表述是这一一般情况的特殊情形,因为在单连通区域中,任意闭曲线都可以同伦收缩为一点。

\textbf{主要示例}

在两种情形下,都需要注意:曲线$\gamma$不能环绕定义域中的“洞”,否则该定理不再适用。一个著名的例子如下:
$$
\gamma(t) = e^{it}, \quad t \in [0, 2\pi],~
$$
这条曲线描绘的是单位圆。

在这种情况下,积分:
$$
\int_{\gamma} \frac{1}{z} \, dz = 2\pi i \neq 0~
$$
结果不为零。这是因为在此情形下,函数:$f(z) = 1/z$在 $z = 0$没有定义。直观地看,曲线 $\gamma$ 围绕了定义域中的一个“洞”,因此无法在不离开区域的情况下把曲线缩成一个点。因此,柯西积分定理在该情形下不适用。
\subsection{讨论}
正如爱德华·古尔萨所示,柯西积分定理只需要假设函数 $f'(z)$ 在区域 $U$ 内处处存在即可成立。这一点非常重要,因为这意味着可以进一步证明这些函数满足柯西积分公式,并由此推导出这些函数是无限可微的。

条件 $U$ 是单连通区域意味着该区域内没有“洞”;从同伦的角度来看,这意味着 $U$ 的基本群是平凡的。例如,对于任意 $z_0 \in \mathbb{C}$,开圆盘$U_{z_0} = \{z : |z - z_0| < r \}$都满足单连通的条件。

这个条件非常关键。考虑以下情形:
$$
\gamma(t) = e^{it}, \quad t \in [0, 2\pi],~
$$
这条曲线描绘的是单位圆。对应的路径积分为:
$$
\oint_{\gamma} \frac{1}{z}\,dz
= \int_0^{2\pi} \frac{1}{e^{it}} \big(i e^{it} dt\big)
= \int_0^{2\pi} i\, dt
= 2\pi i.~
$$
结果不为零。这是因为:$f(z) = 1/z$在 $z = 0$没有定义(并且当然也不是全纯的)。因此,柯西积分定理在该情形下不适用。

该定理的一个重要推论是:在单连通区域内,全纯函数的路径积分可以用类似于微积分基本定理的方式来计算。设 $U$ 是复平面 $\mathbb{C}$ 中的一个单连通开子集,函数$f: U \to \mathbb{C}$在 $U$ 上是全纯的,$\gamma$ 是 $U$ 中一条分段连续可微的路径,起点为 $a$,终点为 $b$。如果 $F$ 是 $f$ 的一个复原函数(复反导函数),那么有:
$$
\int_{\gamma} f(z)\,dz = F(b) - F(a).~
$$
柯西积分定理在比上述更弱的假设下同样成立。例如,如果 $U$ 是复平面中的一个单连通开子集,则只需假设:$f$ 在 $U$ 内全纯,且在闭包 $\overline{U}$ 上连续,并且 $\gamma$ 是 $\overline{U}$ 内一条可求长的简单闭曲线,该定理依然成立$1$。

---

**进一步推论**

柯西积分定理是推导**柯西积分公式**以及**留数定理**的重要基础。
