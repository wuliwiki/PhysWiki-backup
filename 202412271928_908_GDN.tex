% 高德纳(综述)
% license CCBYSA3
% type Wiki

本文根据 CC-BY-SA 协议转载翻译自维基百科\href{https://en.wikipedia.org/wiki/Donald_Knuth}{相关文章}。

\begin{figure}[ht]
\centering
\includegraphics[width=6cm]{./figures/56ac8349febf8e6f.png}
\caption{2011年时的克努斯} \label{fig_GDN_1}
\end{figure}
唐纳德·厄尔文·克努斯(Donald Ervin Knuth,1938年1月10日出生)是美国计算机科学家和数学家,斯坦福大学荣誉教授。他是1974年图灵奖获得者,通常被视为计算机科学领域的“诺贝尔奖”[4]。克努斯被誉为“算法分析之父”[5]。

克努斯是《计算机程序设计的艺术》这一多卷本著作的作者。他为计算算法的复杂度分析做出了贡献,并系统化了相关的数学技巧。在此过程中,他还普及了渐近符号法。除了在多个计算机科学理论领域的基础性贡献外,克努斯还是TeX计算机排版系统的创造者,相关的METAFONT字体定义语言和渲染系统,以及Computer Modern字体系列的发明者。

作为作家和学者,克努斯创造了WEB和CWEB计算机编程系统,旨在鼓励和促进文献化编程,并设计了MIX/MMIX指令集架构。他坚决反对软件专利的授予,并已向美国专利商标局和欧洲专利组织表达过自己的看法。
\subsection{传记} 
\subsubsection{早年生活}  
Donald Knuth 生于威斯康星州密尔沃基市,父亲是 Ervin Henry Knuth,母亲是 Louise Marie Bohning。[6] 他将自己的血统描述为“中西部路德教德国人”。[7]: 66  他的父亲经营着一家小型印刷公司,并教授簿记。[8] 在密尔沃基路德高中就读时,Knuth 想出了许多巧妙的解决问题的方法。例如,在八年级时,他参加了一个竞赛,挑战计算出可以用“Ziegler's Giant Bar”中的字母重新排列形成的单词数;评委已经找到了2,500个这样的单词。在学校因假装肚子疼而获得的空闲时间里,Knuth 使用一本未删减版的字典,逐一检查字典条目,看看是否能用短语中的字母拼出该词。他通过这种算法找出了超过4,500个单词,赢得了比赛。[7]: 3  作为奖励,学校获得了一台新电视机,并为他的同学们提供了足够的巧克力棒。[9][10]  
\subsubsection{教育背景}  
Knuth 获得了物理学奖学金,进入克利夫兰的凯斯技术学院(现为凯斯西储大学)就读,时间是1956年。[11] 他还加入了Theta Chi兄弟会的Beta Nu分会。在凯斯学习物理时,Knuth 接触了IBM 650,一款早期的商业计算机。阅读了计算机手册后,Knuth 决定重新编写该机器的汇编语言和编译器代码,因为他认为自己能做得更好。[12]  

1958年,Knuth 创建了一个程序,帮助学校的篮球队提高获胜的机会。[13] 他为球员分配了“值”,用来评估他们得分的概率,这一新颖的做法后来被《新闻周刊》和 CBS 晚间新闻报道。[12]  

Knuth 是凯斯技术学院《工程与科学评论》的创始编辑之一,该杂志于1959年获得了全国最佳技术杂志奖。[14][15] 他随后从物理学转向数学,并于1960年获得凯斯技术学院的两项学位:[11] 理学学士学位,并通过特别奖励同时获得理学硕士学位,因为他的工作被认为非常出色。[4][12]  

在凯斯的毕业年(1960年)结束时,Knuth 向 Burroughs 公司提出了编写 B205 的 ALGOL 编译器的提案,价格为5,500美元。该提案被接受,他在从凯斯毕业后、前往加州理工学院之间的时间里,开始了 ALGOL 编译器的工作。[7]: 66 [16]: 7  

1963年,在数学家 Marshall Hall 的指导下,[2] 他获得了加州理工学院的数学博士学位,论文题目为《有限半场与射影平面》。[17]
\subsubsection{早期工作}  
1963年,获得博士学位后,Knuth加入了加利福尼亚理工学院(Caltech)的教职,成为助理教授。

在加利福尼亚理工学院工作期间,并且在巴罗赫公司(Burroughs Corporation)B205 ALGOL编译器成功的推动下,他成为巴罗赫公司的顾问,加入了产品规划部门。在加利福尼亚理工学院,他作为数学家工作,而在巴罗赫公司,他则作为程序员与他认为在当时编写了最佳软件的团队合作,负责B220计算机(B205的继任者)的ALGOL编译器。

Knuth曾获得绿树公司(Green Tree Corporation)提供的100,000美元合同,要求他编写编译器,但他拒绝了这一提议,做出了不追求收入最大化的决定,继续留在加利福尼亚理工学院和巴罗赫公司。他获得了国家科学基金会奖学金和伍德罗·威尔逊基金会奖学金,但这些奖学金有条件,要求他只能做研究生学习,而不能继续作为巴罗赫公司的顾问。他选择拒绝这些奖学金,继续与巴罗赫合作。

1962年夏天,Knuth为Univac编写了一个FORTRAN编译器,但他认为,“我把灵魂卖给了魔鬼”来编写FORTRAN编译器。

毕业后,Knuth于1961年6月返回巴罗赫公司,但没有告诉他们自己是以硕士学位而非预期的学士学位毕业的。他对ALGOL语法图、符号表、递归下降方法以及编译器中扫描、解析和发射功能的分离印象深刻,提出了一种扩展符号表的方法,其中一个符号可以代表一串符号。这成为了巴罗赫ALGOL中的DEFINE的基础,后来被其他语言采用。然而,也有人强烈反对这个想法,认为应该删除DEFINE。最后一个认为这是个糟糕主意的人是Edsger Dijkstra,他曾访问巴罗赫公司。

Knuth在巴罗赫公司从事模拟语言的工作,开发了SOL(Simulation Oriented Language)——一种比当时最先进的模拟语言更先进的语言,这一语言与J. McNeeley共同设计。他参加了1967年5月在挪威举行的一个会议,这个会议由Simula语言的发明者主办。Knuth影响了巴罗赫公司采用Simula语言。Knuth与巴罗赫公司有着长时间的合作关系,从1960年到1968年,直到他于1969年转向更多的学术工作,加入斯坦福大学。

1962年,Knuth接受了Addison-Wesley出版社的委托,编写一本关于计算机编程语言编译器的书。在这个项目进行过程中,他决定,在没有先制定出计算机编程的基本理论之前,他无法充分讨论这一主题,而这最终形成了《计算机程序设计艺术》。他原本打算将其作为一本书出版,但随着他为这本书制定大纲,他得出结论,需要六卷,甚至七卷,才能彻底涵盖这一主题。他于1968年出版了第一卷。

在出版《计算机程序设计艺术》的第一卷之前,Knuth离开了加利福尼亚理工学院,接受了国防分析研究所(Institute for Defense Analyses)通信研究部门的工作,时该部门位于普林斯顿校园,进行数学研究,支持国家安全局的密码学工作。

1967年,Knuth参加了工业与应用数学学会的会议,当时有人问他做什么。那时,计算机科学被分为数值分析、人工智能和编程语言。根据他的研究以及《计算机程序设计艺术》一书,Knuth决定下次有人问他时,他会说:“算法分析”。

1969年,Knuth离开普林斯顿的职位,加入斯坦福大学教职,1977年成为Fletcher Jones计算机科学教授。1990年,他成为《计算机程序设计艺术》的教授,并自1993年起成为名誉教授。
\subsection{著作}
Knuth不仅是一位计算机科学家,还是一位作家。
\subsubsection{《计算机程序设计艺术》(TAOCP)}

“人与人之间最好的沟通方式是通过故事。”

—— 唐纳德·克努斯

在1970年代,克努斯曾称计算机科学为“一个完全新兴的领域,没有真正的身份。而当时的出版物标准也不高,很多发表的论文根本就是错误的……所以,我的一个动机就是将这个故事讲清楚,它讲得非常糟糕。”

从1972年到1973年,克努斯在奥斯陆大学度过了一年,与如Ole-Johan Dahl等人共事。克努斯原本打算在那里完成他书籍系列中的第七卷,内容是关于编程语言的。但当他到达奥斯陆时,只完成了前两卷,因此他将这一年用在了第三卷的编写工作上,除此之外还教授课程。第三卷在克努斯于1973年回到斯坦福后发布。

《Concrete Mathematics: A Foundation for Computer Science》起初是TAOCP第一卷数学预备部分的扩展。克努斯发现,在第一卷中有一些数学工具是必要的,但自己并未掌握这些工具,于是他决定为计算机科学的学生开设一门介绍这些工具的课程。克努斯于1970年在斯坦福大学开设了这门课程。Oren Patashnik编写的课程笔记最终演变成了1988年出版的《Concrete Mathematics》一书,作者包括罗纳德·格雷厄姆、克努斯和帕塔什尼克。该书的第二版于1994年出版。

到2011年,TAOCP的第4A卷已经出版。2020年4月,克努斯表示,他预计TAOCP的第4卷至少会包括A到F部分。第4B卷于2022年10月发布。

如果你对他的其他作品感兴趣,我可以继续提供详细信息。
\subsubsection{其他著作}
克努斯还是《超现实数》(*Surreal Numbers*)的作者,这本数学小说探讨了约翰·霍顿·康威(John Horton Conway)集合理论构建的一个替代数系。与简单地解释这一主题不同,克努斯通过这本书展示了数学发展的过程。他希望这本书能为学生做原创性、创造性研究做好准备。

1995年,克努斯为马克·佩特科夫谢克(Marko Petkovšek)、赫伯特·威尔夫(Herbert Wilf)和多龙·泽尔伯格(Doron Zeilberger)的《A=B》一书写了前言。他还偶尔为《Word Ways: The Journal of Recreational Linguistics》提供语言谜题。

克努斯深入研究了娱乐数学。从1960年代起,他开始为《Journal of Recreational Mathematics》撰写文章,并在约瑟夫·马达奇(Joseph Madachy)的《Mathematics on Vacation》一书中被认定为主要贡献者。

克努斯还出现在多个YouTube上的《Numberphile》和《Computerphile》视频中,讨论从写作《超现实数》到他为何不使用电子邮件等话题。

克努斯曾提议使用“算法学”(*algorithmics*)作为计算机科学这一学科的更好名称。
\subsubsection{关于他的宗教信仰的著作}
除了计算机科学方面的著作,克努斯还是《3:16 Bible Texts Illuminated》一书的作者,书中他通过系统采样的方法分析圣经,特别是每本书的第三章第十六节。每节经文都配有一幅书法艺术作品,作品由赫尔曼·扎普夫(Hermann Zapf)带领的一组书法家创作。克努斯曾受邀在麻省理工学院(MIT)讲授关于《3:16》项目背后宗教与计算机科学观点的系列讲座,并由此出版了《Things a Computer Scientist Rarely Talks About》一书,其中包括了《上帝与计算机科学》讲座的内容。
\subsubsection{对软件专利的看法}
克努斯强烈反对授予软件专利,特别是对于那些显而易见、过于简单的解决方案。然而,对于一些非平凡的解决方案(如线性规划中的内点法),他则表达了更加细致的看法。他曾直接向美国专利和商标局以及欧洲专利组织表达了自己的不同意见。
\subsection{编程}
\subsubsection{数字排版}
在1970年代,TAOCP的出版商放弃了单式排版(Monotype),转而采用光学排版系统。克努斯对后者系统无法达到前几卷使用传统排版系统的质量感到非常沮丧,于是他花时间研究数字排版,并创造了TeX和Metafont。[44]
\subsubsection{文学化编程}
在开发TeX的过程中,克努斯创造了一种新的编程方法论,称为文学化编程(Literate Programming),因为他认为程序员应该将程序视为文学作品:

“与其认为我们的主要任务是指示计算机该做什么,不如集中精力向人类解释我们希望计算机做什么。”[45]

克努斯在WEB系统中体现了文学化编程的理念。相同的WEB源代码既可以生成TeX文件,也可以生成Pascal源代码文件。它们分别产生可读的程序描述和可执行的二进制文件。该系统的后续版本CWEB将Pascal替换为C、C++和Java。[46]

克努斯用WEB编写了TeX和METAFONT,并将这两个程序作为书籍出版,这两本书都是1986年出版的:*TeX: The Program*(1986年)和*METAFONT: The Program*(1986年)。[47] 同时,LaTeX(一个基于TeX的宏包,现已广泛应用)由Leslie Lamport首次开发,并于1986年发布了其第一版用户手册。[48]
\subsection{个人生活}
唐纳德·克努斯于1961年6月24日与南希·吉尔·卡特(Nancy Jill Carter)结婚,当时他是加利福尼亚理工学院的研究生。他们有两个孩子:约翰·马丁·克努斯(John Martin Knuth)和詹妮弗·西亚·克努斯(Jennifer Sierra Knuth)。[49]

克努斯每年在斯坦福大学进行几次非正式的讲座,他称之为“计算机沉思”(Computer Musings)。直到2017年,他曾是英国牛津大学计算机科学系的访问教授,并且是玛格达伦学院的名誉院士。[50][51]

克努斯是位管风琴师和作曲家。他和父亲曾是路德教会的管风琴师。克努斯和妻子家里有一台16音阶的管风琴。[52] 2016年,他完成了一部管风琴作品《启示录幻想曲》(Fantasia Apocalyptica),他称之为“将《圣约翰启示录》的希腊文本翻译成音乐”。该作品于2018年1月10日在瑞典首演。[53]
\subsubsection{中文名字}
克努斯的中文名字是高德纳(简体中文:高德纳;繁体中文:高德納;拼音:Gāo Dénà)。[54][3] 这个名字是他在1977年由姚菲(Frances Yao)所取,在前往中国进行为期三周的旅行之前。[3][55] 在1980年《计算机程序设计艺术》第一卷的中文翻译(简体中文:计算机程序设计艺术;繁体中文:計算機程式設計藝術;拼音:Jìsuànjī chéngxù shèjì yìshù)中,克努斯解释了他接受这个中文名字的原因,因为当时中国的计算机程序员人数逐渐增加,他希望能够被他们认识。1989年,他的中文名字被置于《计算机科学与技术期刊》的首页,克努斯表示:“这让我感到与所有中国人亲近,尽管我不会说你们的语言。”[55]
\subsubsection{幽默}
\begin{figure}[ht]
\centering
\includegraphics[width=8cm]{./figures/d931689579aef3bc.png}
\caption{克努斯的奖励支票之一} \label{fig_GDN_2}
\end{figure}
克努斯曾经为他书中的任何排版错误或失误支付\$2.56的赏金,因为“256个便士是一个十六进制的美元”,而对于“有价值的建议”则支付\$0.32。根据麻省理工学院《技术评论》中的一篇文章,这些克努斯的奖励支票被认为是“计算机界最受珍视的奖品之一”。由于银行诈骗问题,克努斯在2008年不得不停止发放真实支票,现在他会向每位发现错误的人发放来自他虚构的“圣塞里费银行”的“存款证明”,该银行有一个公开上市的余额。[56]

他曾警告一位通信者:“当心上面代码中的错误;我只证明了它是正确的,并没有尝试过。”[3]

克努斯在1957年首次发表了他的第一篇“科学”文章,标题为《Potrzebie单位制》。在这篇文章中,他将基本的长度单位定义为《疯狂杂志》26号的厚度,并将基本的力单位命名为“whatmeworry”。《疯狂杂志》在第33期(1957年6月)上发表了这篇文章。[57][58]

为了演示递归的概念,克努斯故意在《计算机程序设计艺术》第一卷的索引中将“循环定义”和“定义,循环”相互引用。

《Concrete Mathematics》的前言有这样一段话:
当克努斯第一次在斯坦福教授《Concrete Mathematics》时,他解释了这个有些奇怪的标题,表示这是他尝试教授一门困难而非容易的数学课程。他宣布,与同事们的期望相反,他不会教授聚集体理论、斯通嵌入定理,甚至也不会讲斯通–切赫紧化。(几位来自土木工程系的学生起身悄悄离开了教室。)

在2010年TUG大会上,克努斯宣布了一个带有讽刺意味的基于XML的TeX继任者,名为“iTeX”(发音为[iː˨˩˦tɛks˧˥],伴随铃声),它将支持如任意缩放的无理单位、3D打印、地震仪和心脏监测仪输入、动画和立体声等功能。[59][60][61]
\subsection{奖项与荣誉}  
1971年,克努斯获得了首届ACM Grace Murray Hopper奖。[4] 他还获得了其他多个奖项,包括图灵奖、国家科学奖章、约翰·冯·诺依曼奖章和京都奖。[4]

1980年,克努斯被选为英国计算机学会的杰出会员(DFBCS),以表彰他对计算机科学领域的贡献。[62]

1990年,他被授予了唯一的学术头衔“计算机程序设计艺术教授”;该头衔后来被修订为“计算机程序设计艺术名誉教授”。

克努斯于1975年当选为美国国家科学院院士。他还于1981年当选为美国工程院院士,以表彰他将计算机科学的广泛学科领域进行了组织,使其对计算机界的所有层面都能访问。1992年,他成为法国科学院的院士。同年,他从斯坦福大学的常规研究和教学工作中退休,以完成《计算机程序设计艺术》的写作。2003年,他当选为皇家学会外籍院士(ForMemRS)。[1]

克努斯于2009年当选为工业与应用数学学会的会士(第一届会士),以表彰他对数学的杰出贡献。[63] 他是挪威科学院和文学学会的成员。[64] 2012年,他成为美国数学学会的会士[65],并加入了美国哲学学会。[66] 其他奖项和荣誉包括:
\begin{itemize}
\item 1971年首届ACM Grace Murray Hopper奖[4]  
\item 1974年图灵奖[4]  
\item 1975年和1993年莱斯特·R·福特奖[67][68]  
\item 1978年乔赛亚·威拉德·吉布斯讲座[69][70]  
\item 1979年国家科学奖章[71]  
\item 1985年美国成就学院金盘奖[72]  
\item 1988年富兰克林奖章[4]  
\item 1995年约翰·冯·诺依曼奖章[4]  
\item 1995年Technion哈维奖[73]  
\item 1996年京都奖[4]  
\item 1998年计算机历史博物馆会士奖“表彰他在计算机算法历史、TeX排版语言的发展以及在数学和计算机科学方面的重大贡献。”  
\item 2001年,发现的小行星21656 Knuth,以他的名字命名[75][76]  
\item 2010年片山奖[77]  
\item 2010年BBVA基金会知识前沿奖,信息与通信技术类别[78]  
\item 2011年图灵讲座  
\item 2011年斯坦福大学工程学院英雄奖[79]  
\item 2014年弗拉约雷特讲座奖[80]
\end{itemize}
\subsection{出版物} 
克努斯的部分出版物包括:[81]

《计算机程序设计艺术》:
\begin{enumerate}
\item ——— (1997)。 计算机程序设计艺术。第1卷:基本算法(第三版)。Addison-Wesley Professional。ISBN 978-0-201-89683-1。  
\item ——— (1997)。 计算机程序设计艺术。第2卷:半数值算法(第三版)。Addison-Wesley Professional。ISBN 978-0-201-89684-8。  
\item ——— (1998)。 计算机程序设计艺术。第3卷:排序与查找(第二版)。Addison-Wesley Professional。ISBN 978-0-201-89685-5。  
\item ——— (2011)。 计算机程序设计艺术。第4A卷:组合算法,第1部分。Addison-Wesley Professional。ISBN 978-0-201-03804-0。  
\item ——— (2022)。 计算机程序设计艺术。第4B卷:组合算法,第2部分。Addison-Wesley Professional。ISBN 978-0-201-03806-4。  
\item ——— (2005)。 MMIX—新千年RISC计算机。第1卷,第1部分。ISBN 978-0-201-85392-6。  
\item ——— (2008)。 计算机程序设计艺术。第4卷,第0部分:组合算法与布尔函数介绍。Addison-Wesley。ISBN 978-0-321-53496-5。  
\item ——— (2009)。 计算机程序设计艺术。第4卷,第1部分:位操作技巧与技术,二进制决策图。Addison-Wesley。ISBN 978-0-321-58050-4。  
\item ——— (2005)。 计算机程序设计艺术。第4卷,第2部分:生成所有元组和排列。Addison-Wesley。ISBN 978-0-201-85393-3。  
\item ——— (2005)。 计算机程序设计艺术。第4卷,第3部分:生成所有组合与划分。ISBN 978-0-201-85394-0。  
\item ——— (2006)。 计算机程序设计艺术。第4卷,第4部分:生成所有树——组合生成的历史。Addison-Wesley。ISBN 978-0-321-33570-8。  
\item ——— (2018)。 计算机程序设计艺术。第4卷,第5部分:数学预备知识再版,回溯法,跳舞链。ISBN 978-0-134-67179-6。  
\item ——— (2015)。 计算机程序设计艺术。第4卷,第6部分:可满足性。ISBN 978-0-134-39760-3。 
\end{enumerate} 
《计算机与排版》(除非另有说明,否则所有书籍均为精装):
\begin{itemize}
\item ——— (1984)。 计算机与排版。第A卷,《TeX手册》。马萨诸塞州雷丁:Addison-Wesley。ISBN 978-0-201-13447-6,x+483页。  
\item ——— (1984)。 计算机与排版。第A卷,《TeX手册》。马萨诸塞州雷丁:Addison-Wesley。ISBN 978-0-201-13448-3。(平装)。  
\item ——— (1986)。 计算机与排版。第B卷,《TeX:程序》。马萨诸塞州雷丁:Addison-Wesley。ISBN 978-0-201-13437-7,xviii+600页。  
\item ——— (1986)。 计算机与排版。第C卷,《METAFONT手册》。马萨诸塞州雷丁:Addison-Wesley。ISBN 978-0-201-13445-2,xii+361页。  
\item ——— (1986)。 计算机与排版。第C卷,《METAFONT手册》。马萨诸塞州雷丁:Addison-Wesley。ISBN 978-0-201-13444-5。(平装)。  
\item ——— (1986)。 计算机与排版。第D卷,《METAFONT:程序》。马萨诸塞州雷丁:Addison-Wesley。ISBN 978-0-201-13438-4,xviii+566页。  
\item ——— (1986)。 计算机与排版。第E卷,《计算机现代字体》。马萨诸塞州雷丁:Addison-Wesley。ISBN 978-0-201-13446-9,xvi+588页。  
\item ——— (2000)。 计算机与排版。第A-E卷盒装套装。马萨诸塞州雷丁:Addison-Wesley。ISBN 978-0-201-73416-4。  
\end{itemize}
收录论文集:
\begin{itemize}
\item ——— (1992)。文学化编程。讲座笔记。加利福尼亚州斯坦福:语言与信息研究中心—CSLI。ISBN 978-0-937073-80-3。[82]  
\item ——— (1996)。 计算机科学选集论文。讲座笔记。加利福尼亚州斯坦福:语言与信息研究中心—CSLI。ISBN 978-1-881526-91-9。[83]  
\item ——— (1999)。 *数字排版*。讲座笔记。加利福尼亚州斯坦福:语言与信息研究中心—CSLI。ISBN 978-1-57586-010-7。[84]  
\item ——— (2000)。 *算法分析选集论文*。讲座笔记。加利福尼亚州斯坦福:语言与信息研究中心—CSLI。ISBN 978-1-57586-212-5。[85]  
\item ——— (2003)。 *计算机语言选集论文*。讲座笔记。加利福尼亚州斯坦福:语言与信息研究中心—CSLI。ISBN 978-1-57586-381-8,ISBN 1-57586-382-0(平装)[86]  
\item ——— (2003)。 *离散数学选集论文*。讲座笔记。加利福尼亚州斯坦福:语言与信息研究中心—CSLI。ISBN 978-1-57586-249-1,ISBN 1-57586-248-4(平装)[87]  
\item Donald E. Knuth,《算法设计选集论文》(加利福尼亚州斯坦福:语言与信息研究中心—CSLI讲座笔记,第191号),2010年。ISBN 1-57586-583-1(精装),ISBN 1-57586-582-3(平装)[88]  
\item Donald E. Knuth,《趣味与游戏选集论文》(加利福尼亚州斯坦福:语言与信息研究中心—CSLI讲座笔记,第192号),2011年。ISBN 978-1-57586-585-0(精装),ISBN 978-1-57586-584-3(平装)[89]  
\item Donald E. Knuth,《Donald Knuth论文集伴侣》(加利福尼亚州斯坦福:语言与信息研究中心—CSLI讲座笔记,第202号),2011年。ISBN 978-1-57586-635-2(精装),ISBN 978-1-57586-634-5(平装)[90]
\end{itemize}
其他书籍:
\begin{itemize}
\item Graham, Ronald L; Knuth, Donald E.; Patashnik, Oren (1994)。 *Concrete Mathematics: A Foundation for Computer Science*(第二版)。马萨诸塞州雷丁:Addison-Wesley。ISBN 978-0-201-55802-9。MR 1397498。xiv+657页。  
\item Knuth, Donald Ervin (1974)。 *Surreal Numbers: How Two Ex-Students Turned on to Pure Mathematics and Found Total Happiness: A Mathematical Novelette*。Addison-Wesley。ISBN 978-0-201-03812-5。[31]  
\item Donald E. Knuth,《The Stanford GraphBase: A Platform for Combinatorial Computing》(纽约,ACM Press),1993年。第二版平装本,2009年。ISBN 0-321-60632-9。  
\item Donald E. Knuth,《3:16 Bible Texts Illuminated》(威斯康星州麦迪逊:A-R Editions),1990年。ISBN 0-89579-252-4。  
\item Donald E. Knuth,《Things a Computer Scientist Rarely Talks About》(语言与信息研究中心—CSLI讲座笔记第136号),2001年。ISBN 1-57586-326-X。  
\item Donald E. Knuth,《MMIXware: A RISC Computer for the Third Millennium》(海德堡:Springer-Verlag—计算机科学讲座笔记,第1750号),1999年。viii+550页。ISBN 978-3-540-66938-8。  
\item Donald E. Knuth 和 Silvio Levy,《The CWEB System of Structured Documentation》(马萨诸塞州雷丁:Addison-Wesley),1993年。iv+227页。ISBN 0-201-57569-8。2001年第三版,增加超文本支持,ii+237页。  
\item Donald E. Knuth, Tracy L. Larrabee 和 Paul M. Roberts,《Mathematical Writing》(华盛顿特区:美国数学学会),1989年。ii+115页。ISBN 978-0883850633。  
\item Daniel H. Greene 和 Donald E. Knuth,《Mathematics for the Analysis of Algorithms》(波士顿:Birkhäuser),1990年。viii+132页。ISBN 978-0817647285。  
\item Donald E. Knuth,《Mariages Stables: et leurs relations avec d'autres problèmes combinatoires》(蒙特利尔:蒙特利尔大学出版社),1976年。106页。ISBN 978-0840503428。  
\item Donald E. Knuth,《Stable Marriage and Its Relation to Other Combinatorial Problems: An Introduction to the Mathematical Analysis of Algorithms》。ISBN 978-0821806036。  
\item Donald E. Knuth,《Axioms and Hulls》(海德堡:Springer-Verlag—计算机科学讲座笔记,第606号),1992年。ix+109页。ISBN 3-540-55611-7。
\end{itemize}
\subsection{参见}
\begin{itemize}
\item 渐进符号
\item 属性文法
\item CC 系统
\item 跳跃链接(Dancing Links)
\item Knuth -yllion
\item Knuth–Bendix 完备算法
\item Knuth 奖
\item Knuth 洗牌算法
\item Knuth 的算法 X
\item Knuth 的 Simpath 算法
\item Knuth 的上箭头表示法
\item Knuth–Morris–Pratt 算法
\item Davis–Knuth 龙(Davis-Knuth Dragon)
\item Bender–Knuth 反演
\item Trabb Pardo–Knuth 算法
\item Fisher–Yates 洗牌
\item Robinson–Schensted–Knuth 对应
\item 男人或男孩测试(Man or boy test)
\item Plactic 单群
\item 四象虚基数(Quater-imaginary base)
\item TeX
\item Termial
\item 歌曲的复杂性(The Complexity of Songs)
\item 均匀二分查找(Uniform binary search)
\item 计算机科学先驱列表
\item 科学与宗教学者列表
\end{itemize}
\subsection{参考文献}
\begin{enumerate}
\item "Professor Donald Knuth ForMemRS". 伦敦:皇家学会。原文存档于2015年11月17日。
\item Donald Knuth 在数学家谱项目中的资料。
\item Knuth, Donald Ervin. "常见问题解答". 主页。斯坦福大学。原文存档于2019年8月3日。检索日期:2010年11月2日。
\item Walden, David. "Donald E. Knuth - A.M. 图灵奖得主"。原文存档于2019年10月17日。检索日期:2022年12月14日。
\item Karp, Richard M. (1986年2月). "组合学、复杂性与随机性". 《ACM通信》, 29 (2): 98–109. doi:10.1145/5657.5658.
\item O'Connor, John J.; Robertson, Edmund F. (2015年10月), "Donald Knuth", 《数学历史档案》,圣安德鲁斯大学,检索日期:2021年7月2日。
\item Feigenbaum, Edward (2007). "Donald Knuth的口述历史"(PDF)。计算机历史博物馆。原文存档于2008年12月9日。检索日期:2020年9月17日。
\item Molly Knight Raskin (2013). 《没有更好的时机:丹尼·路文的短暂而卓越的人生——改变互联网的天才》. Da Capo出版社,第61-62页。ISBN 978-0-306-82166-0。
\item Shasha, Dennis Elliott; Lazere, Cathy A (1998). 《出自他们的脑海:15位伟大的计算机科学家的生活与发现》. 施普林格出版社,第90页。ISBN 978-0-387-98269-4。
\item Knuth, Donald (2011). 《关于娱乐和游戏的精选论文》. 语言与信息研究中心——CSLI讲义,第192号,第400页。ISBN 978-1-57586-584-3。
\item "Donald E. Knuth". 《Encyclopedia.com》. 检索日期:2020年9月17日。
\item Koshy, Thomas (2004). 《离散数学与应用》. 学术出版社,第244页。ISBN 978-0-12-421180-3。原文存档于2012年11月12日。检索日期:2011年7月30日。
\item Lyons, Keith (2018年9月25日). "Donald Knuth, 篮球与运动中的计算机"。Clyde Street Archive。原文存档于2019年8月16日。检索日期:2019年8月16日。
\item "Beta Nu of Theta Chi, Beta Nu 章节历史". CWRU。原文存档于2016年9月4日。检索日期:2019年4月15日。
\item "Beta Nu, Theta Chi". Theta Chi。原文存档于2019年12月21日。检索日期:2019年12月21日。
\item Waychoff, Richard. "关于B5000的故事与那些参与者"(PDF)。计算机历史博物馆。
\item Knuth, Donald Ervin (1963). 《有限半域与射影平面》 (PDF)(博士论文)。加利福尼亚理工学院。
\item Knuth, Donald Ervin. "简历". 斯坦福大学。原文存档于2019年8月3日。检索日期:2020年3月26日。
\item Dahl, Ole-Johan. "面向对象的诞生:Simula语言"(PDF)。
\item "传记"。
\item "与Richard Nance的访谈 2013"。
\item Dahl, Ole-Johan. "面向对象的诞生:Simula语言"。
\item Knuth, Donald Ervin (2019年8月3日). "计算机编程的艺术(TAOCP)"。原文存档于2019年8月3日。检索日期:2018年2月6日。
\item "国防分析研究所"。INFORMS。2021年8月27日。检索日期:2024年1月8日。
\item D'Agostino, Susan (2020年4月16日). "这位计算机科学家无法停止讲故事"。Quanta Magazine。检索日期:2020年4月19日。
\item "时间表"。斯坦福大学计算机科学@斯坦福专题。2019年6月21日。检索日期:2024年1月8日。
\item Knuth, Donald Ervin. "主页". 斯坦福大学。原文存档于2019年11月27日。检索日期:2005年3月16日。
\item "Donald Knuth". 个人档案。斯坦福大学。原文存档于2016年6月12日。检索日期:2020年8月24日。
\item "BBVA基金会知识前沿奖"。原文存档于2016年8月19日。检索日期:2016年10月15日。
\item "出版物《计算机历史:奥斯陆大学计算机科学系 1977-1997》发布". 奥斯陆大学(挪威语)。1997年。原文存档于2021年4月29日。检索日期:2021年4月29日。
\item Knuth, Donald Ervin. "超现实数". 主页。原文存档于2019年8月3日。检索日期:2020年3月26日。
\item Zeilberg. "DEK". 罗格斯大学。原文存档于2017年8月28日。检索日期:2020年3月26日。
\item "语言学家列表——期刊页面"。语言学家列表。原文存档于2021年6月11日。检索日期:2022年12月14日。
\item Madachy, Joseph S., 《假期中的数学》,Thomas Nelson & Sons Ltd. 1966年。
\item "关于数字与其他事物的视频"。Numberphile。原文存档于2018年11月4日。检索日期:2019年8月16日。
- Numberphile (2016年6月27日),《超现实数(写作第一本书)》 - Numberphile,原文存档于2021年12月11日,检索日期:2019年7月19日。
- Computerphile (2015年8月21日),《为什么Don Knuth不使用电子邮件》 - Computerphile,原文存档于2018年7月11日,检索日期:2019年7月19日。
- Burkholder, Leslie (1992). 《哲学与计算机》。Taylor & Francis。ISBN 9780429301629。
- Platoni 2006。
- Knuth, Donald Ervin (1991). 《3:16 : 圣经文本的启示》。麦迪逊,WI:A-R Eds。ISBN 978-0-89579-252-5。
- Knuth, Donald Ervin (2001). 《计算机科学家很少谈论的事情》。斯坦福,加利福尼亚:语言与信息研究中心出版。ISBN 978-1-57586-326-9。
- "所有问题的答案"(PDF)。《通知》(文章)。2002年3月。原文存档(PDF)于2019年4月30日。检索日期:2020年3月26日。
- Knuth, Donald Ervin. "反对软件专利"(PDF)。原文存档(PDF)于2015年9月24日。检索日期:2020年2月1日。致美国和欧洲专利局的信件。
- Knuth, Donald Erwin (1997). "数字排版(京都奖讲座,1996)"(PDF)。原文存档(PDF)于2018年1月27日。
- Knuth, Donald Erwin (1984). "文学编程"(PDF)。原文存档(PDF)于2019年8月19日。检索日期:2020年3月26日。
- "Knuth与Levy:CWEB"。
- Knuth, Donald (2019年4月11日). "Knuth: 计算机与排版"。www-cs-faculty.stanford.edu。原文存档于2019年4月11日。检索日期:2019年7月19日。
- Lamport, Leslie (1986). 《LATEX: 一种文档准备系统》。Addison-Wesley Pub. Co. ISBN 020115790X。OCLC 12550262。
\end{enumerate}