% 关于小时百科

\subsection{教材还是百科?}
看到《小时百科》这几个字, 可能许多人会以为这是一个类似 Wikipedia 的网站. 但我们并不是在做一个国产的 Wikipedia, 虽然我们的网站做成了\textbf{百科}的形式, 但我们目前大部分内容都更接近\textbf{教材}. 这二者有什么区别呢? 非常大.

Wikipedia 上每个条目基本可以看做是围绕其标题的一个\textbf{综述}, 追求大而全. 例如在 Wikipedia 的\href{https://en.wikipedia.org/wiki/Newton's_laws_of_motion}{牛顿运动定律}页面中囊括了牛顿力学的发展历史, 三定律的具体表述, 功和能, 刚体力学, 混沌, 分析力学, 与热力学、电磁学、相对论以及量子力学的关系. 全文约 9000 单词, 引用文献约 120 个. 这样的结构作为一个百科条目是符合标准的, 但对初学者却并不那么友好——它包含的信息量太大, 而每个具体的知识点讲得又不够详细.

作为对比, 如果我们翻开大学低年级的力学课本, 里面则可能会先介绍一些简单的矢量微积分, 然后讲解如何画受力分析图, 力的合成与分解, 然后再讲解牛顿三定律, 刚体力学, 最后才会简单介绍分析力学和相对论等. 也就是说, Wikipedia 上的牛顿运动定律条目几乎把一本几百页的教材用 9000 词进行了概括.

当然, 很多时候我们并不想把整本教材从头读到尾(即使在学校的课程中也很少把整本书都覆盖到), 而是想快速学习某一个知识点. 此时最好的办法是找一个老师告诉他你想学什么, 学到什么深度, 然后让他根据这个目标给你专门定制一个课程. 但显然大部分人并不具有这样的条件. 我们的目标就是

 这时你通过目录找到某个章节, 如无意外会发现由于前面的内容没读, 想要学的章节看不懂. 在没有人指引的情况下, 最后不得不从第一页开始看. 更糟糕的是, 如果这本书假设你已经学过一些其他的课程(例如微积分). 于是你又找来一本微积分课本(如无意外是同济版的高数), 开始从第一页看……

小时百科想要做的大概就是, 把上面的力学课本中的每一小节作为一个相对独立的页面, 并且注明每个小节需要哪些其他的小节

\subsection{简介}

小时百科从结构上尝试将教材和百科这两种不同形式的文本融合到一起, 使其既适合初学者自学, 又可以按照灵活的顺序阅读. 我们计划涵盖理工科专业本科课程中的主要内容, 适用于具有普通高中及以上数学物理基础的读者. 小时百科是一个庞大的工程, 将长期处于更新状态.

在介绍小时百科的特点以前,我们先来看一般数理教材的不足:
\begin{enumerate}
\item 需要按顺序学习,不适合初学者快速了解或查找某个话题或知识点. 例如某高中生需要了解角动量的概念, 直接翻开大学力学教材的相关章节发现看不懂, 却又不知道需要先学什么, 也没时间从头先看完微积分和线性代数的教材再开始学习.
\item 读者不能自己选择所学内容的深度和严谨性. 例如一些常用的高等数学教材在读者对微积分还没有一个大概的了解时就介绍最严谨的定义和证明. 我们认为这样做对初学者并不友好.
\item 不够自洽(self-contained). 一本教材的自洽性指目标读者在学习前是否还需要学习其他教材. 大部分本科物理教材对高中生都是不自洽的, 因为它们往往假设读者具有一定的微积分和线性代数基础.
\end{enumerate}

再来看一般网络百科(如百度百科或维基百科)的不足:
\begin{enumerate}
\item 每个词条都相当于一个综述, 追求大而全, 概括所有相关内容.
\item 每个词条都以最广义最严谨的方式呈现, 读者不能选择适合自己的深度和严谨程度.
\item 容易出现公式定理的堆积, 缺乏知识点导入和讲述, 缺乏例题, 习题等.
\end{enumerate}

\begin{figure}[ht]
\centering
\includegraphics[width=10cm]{./figures/about_1.pdf}
\caption{由 “预备知识” 画出的知识树(目标词条为“力场、势能\upref{V}”)}\label{about_fig1}
\end{figure}

为了克服上述困难,百科采用以下形式:
\begin{enumerate}
\item 将知识点划分为词条, 且在每个词条中列出学习该词条前需要先学习哪些词条. 这样相当于建立了一个知识树(如\autoref{about_fig1}). 完整的知识树见 \href{https://wuli.wiki/tree}{wuli.wiki/tree}, 可以选中任意词条为目标,生成知识树.
\item 采用词条分级,把同一个话题以不同深度, 严谨度和适用范围等划分成若干个等级的同名词条. 这样读者可以选择螺旋式学习(例如初中,高中,大学物理中所学的话题几乎相同,但程度不同). 暂定初级词条从科普开始,尽量少使用数学.随着词条级别升高,会使用适用范围更广的定义,更严谨的表述和更抽象的数学等.
\end{enumerate}

\subsection{词条}
百科内容繁多,不同词条的重要性相差甚远,不建议初学者按照词条的排列顺序依次学习,而是应该以每章给出的导航来学习,再根据兴趣和需要阅读其余词条.

理论上来说,读者可以直接跳到最感兴的词条,如果 “预备知识” 中列出的词条都已经掌握,就可以开始学习该词条,否则就先掌握 “预备知识” 中的词条. 如果 “预备知识” 出现在词条开始,则必须先掌握,如果出现在正文中,则只有阅读该部分时需要掌握.如果正文中引用了没有出现在“预备知识”中的内容,则读者可自行决定是否阅读.

为了便于书内的跳查,词条之间进行了大量的交叉引用,例如 “导数简介\upref{Der}” 右上角中括号中的数字代表被引用词条的页码. 由于每个词条的公式编号都从 1 开始, 引用其他词条中的公式有时会用类似 “\autoref{Der_eq2}~\upref{Der}” 的格式, 右上角的方括号中是公式所在词条的起始页码. 在本书的 pdf 电子版中,点击该页码即可自动跳转到对应的页面. 在网页版中, 使用快捷键组合 “Alt +左箭头” 即可返回跳查前的位置, 在 app 中也有相应的返回按钮.
