% 菲利普·莱纳德(综述)
% license CCBYSA3
% type Wiki

本文根据 CC-BY-SA 协议转载翻译自维基百科\href{https://en.wikipedia.org/wiki/Philipp_Lenard}{相关文章}。

菲利普·爱德华·安东·冯·莱纳德(德语发音:[ˈfɪlɪp ˈleːnaʁt],匈牙利语:Lénárd Fülöp Eduárd Antal,1862年6月7日-1947年5月20日)是一位匈牙利裔德国物理学家,因“在阴极射线研究中取得的成果”以及发现其多种性质,于1905年获得诺贝尔物理学奖。

他最重要的贡献之一是对光电效应的实验验证:他发现,从阴极中逸出的电子的能量(速度)只依赖于入射光的频率,而与其强度无关。

莱纳德是民族主义者和反犹主义者;他是纳粹意识形态的积极支持者,早在1920年代便支持阿道夫·希特勒,并在纳粹时期成为“德国物理学”运动的重要榜样。值得注意的是,他将阿尔伯特·爱因斯坦的科学贡献贬称为“犹太物理学”。

\subsection{早年生活与工作}
菲利普·莱纳德于1862年6月7日出生在匈牙利王国的普雷斯堡(当时称为 Pozsony,即今日斯洛伐克的布拉迪斯拉发)。莱纳德家族最早在17世纪来自蒂罗尔,而他母亲的家族则源自巴登;他的父母都是讲德语的。[5] 父亲名为菲利普·冯·莱纳德,是普雷斯堡的一位葡萄酒商人;母亲名为安东妮·鲍曼。[6] 莱纳德在大多为日耳曼血统的祖先中也有一些马扎尔人血统。年幼的莱纳德曾就读于波若尼皇家天主教高级文理中学(Pozsonyi Királyi Katolikus Főgymnasium,今称 Gamča),据他在自传中记述,这段经历给他留下了深刻印象,尤其是他老师维吉尔·克拉特的个人魅力。[7]
1880年,他先后在维也纳和布达佩斯学习物理和化学。[7]

1882年,莱纳德离开布达佩斯,返回普雷斯堡。但在1883年,他前往海德堡,因为他申请布达佩斯大学助教职位遭到拒绝。在海德堡,他师从著名的罗伯特·本生,其间曾在柏林跟随赫尔曼·冯·亥姆霍兹学习了一个学期。他还曾师从格奥尔格·赫尔曼·昆克[1],并于1886年获得博士学位。[1][8]1887年,他回到布达佩斯,在洛兰·厄特沃什手下担任演示员。[7]此后他曾在亚琛、波恩、布雷斯劳、海德堡(1896–1898年)和基尔(1898–1907年)任教,并最终于1907年回到海德堡大学,出任菲利普·莱纳德研究所所长。莱纳德于1905年当选为瑞典皇家科学院院士,1907年成为匈牙利科学院院士。[7] 他早期的研究涉及磷光和荧光现象以及火焰的导电性等课题。
\subsection{对物理学的贡献}
\subsubsection{光电效应研究}
\begin{figure}[ht]
\centering
\includegraphics[width=6cm]{./figures/6f04b289d4a626c3.png}
\caption{} \label{fig_FLPL_1}
\end{figure}
作为物理学家,莱纳德的主要贡献在于对阴极射线的研究,他从 1888 年开始涉足这一领域。在他的研究之前,阴极射线是通过一种原始的、部分抽真空的玻璃管产生的,这种玻璃管中装有金属电极,可施加高电压。然而,这种装置使得阴极射线的研究极为困难,因为射线被封闭在玻璃管中,不易接近,同时射线也会受到玻璃管内空气分子的影响。为了解决这些问题,莱纳德设计出一种方法:在玻璃管上制作金属小窗,这些小窗既足够厚以承受内外压力差,又足够薄以允许射线穿过。借助这样的“窗户”,他可以将阴极射线引出到实验室环境中,或者引入另一个完全抽真空的腔体中。这种窗后来被称为莱纳德窗。借助这些窗,他可以更方便地检测阴极射线,并通过涂有磷光材料的纸张来测量其强度。[9]特别地,他采用了一种名为十五烷基对甲苯酮的材料作为阴极射线探测剂,这种材料在探测阴极射线方面非常有效,但不幸的是,它对X射线不具荧光反应。这对莱纳德来说是个遗憾。
当伦琴试图复现莱纳德的实验时,发现莱纳德已经买下了市面上所有的十五烷基对甲苯酮,他不得不用氰酸铂钡(替代。恰巧的是,这种替代材料对紫外线和X射线都很敏感,从而让伦琴发现了X射线。[10]

莱纳德观察到,阴极射线在物质中的吸收程度在一阶近似下与材料的密度成正比。这一发现似乎与当时普遍认为阴极射线是某种电磁辐射的观点相矛盾。他还展示出,阴极射线可以穿过数英寸厚的常态空气,并且在穿越过程中会被散射,这意味着这些射线很可能是比空气分子还要小的粒子。他验证了一些由J. J. 汤姆孙提出的实验结果,最终得出阴极射线实际上是带负电的高能粒子流的结论。他将这些粒子称为电量子,简称量子,这一术语借用了亥姆霍兹的说法。而汤姆孙则提出将其称为微粒,但最终“电子”这一名称成为了通用术语。[11]结合他本人及其他人早期关于阴极射线在金属中吸收行为的研究,再加上人们逐渐认识到电子是原子的组成部分,莱纳德得出一个正确的结论:原子在很大程度上是由空空间构成的。他提出:每个原子由空空间和称为“动力子”的电中性微粒组成,而每个“动力子”本身由一个电子和一个等量的正电荷组成。
\begin{figure}[ht]
\centering
\includegraphics[width=8cm]{./figures/9d72548e780d6e23.png}
\caption{莱纳德窗管} \label{fig_FLPL_2}
\end{figure}
通过对克鲁克斯管的研究,莱纳德发现:在真空中用紫外光照射金属所产生的射线,在许多方面都与阴极射线类似。他最重要的观察结果是:光电效应中电子的能量与光的强度无关。然而,他对这一现象的解释是——光只是释放了原子内部已经在运动的电子,他并未将光的能量与电子的能量联系起来。[12]

这些后来被阿尔伯特·爱因斯坦解释为量子效应。爱因斯坦指出:每个具有足够能量的光量子(光子)会释放出一个光电子,因此光的强度影响的是电子的数量(通量强度),而不是电子的能量。该理论预测,阴极射线能量与入射光频率的关系图是一条斜率为普朗克常数 $h$ 的直线。这一预测在若干年后被实验证实。正是由于对光电量子理论的贡献,爱因斯坦获得了1921 年诺贝尔物理学奖。莱纳德对公众对爱因斯坦的推崇始终持怀疑态度,他逐渐成为相对论和爱因斯坦理论的坚定怀疑者;不过,他并不否认爱因斯坦对光电效应的解释。莱纳德对威廉·伦琴所获得的荣誉也表现出极大不满。伦琴因发现X射线获得了1901年首届诺贝尔物理学奖,[13][14] 尽管伦琴是德国人、并非犹太人,莱纳德依然极为怨恨他获得的认可。莱纳德曾写道:X射线的“母亲”是他自己,而不是伦琴,因为是他发明了用于产生X射线的实验装置。他晚年时甚至将伦琴的角色比作一个“助产士”,只是协助“分娩”而已。[15]

莱纳德本人则因在阴极射线方面的研究而获得1905年诺贝尔物理学奖。
\subsubsection{气象学上的贡献}
莱纳德是最早研究后来被称为“莱纳德效应”的人,时间是1892年。该效应指的是水滴在气动力作用下破碎时所伴随的电荷分离现象,也被称为喷雾带电效应或瀑布效应。[16]

他还研究了雨滴的大小与形状分布,并设计出一种新型风洞装置,可以在其中让不同大小的水滴悬浮几秒钟,以便观察。他是第一个认识到大型雨滴并非呈泪滴状,而是更类似于汉堡面包状的人。[17]
\subsection{德意志物理学}
莱纳德今日被人们铭记,主要是因为他是一个极端的德国民族主义者,他蔑视所谓的“英式物理学”,认为那是从德国“偷来”的思想。[18][19][20]由于他拒绝相对论和量子力学,莱纳德与另一位实验物理学家约翰内斯·施塔克在20世纪20年代逐渐被边缘化并遭到学界忽视。[21]在纳粹政权时期,莱纳德则成为公开鼓吹“德意志物理学”理念的代表人物。他主张德国应摒弃他所称的“犹太物理学”——即他认为是错误且故意误导的理论,主要指的是阿尔伯特·爱因斯坦的学说,特别是他称之为“犹太骗局”的相对论(参见:对相对论的批评)。[22]在纳粹统治下,莱纳德被任命为所谓的“雅利安物理学”总负责人。[23] 在他为四卷本教科书《德意志物理学》所写的前言中,他声称:正如人类创造的其他一切,科学也受到种族的决定。[24]

莱纳德的著作《科学中的伟人:科学进步史》最早于1933年出版英文版,[25] 他在书中声称这些科学家都是“雅利安科学家”。[26] 他所选入书中的科学人物,并不包括爱因斯坦、玛丽·居里,也没有任何20世纪的科学家。该书1954年英文版的第十九页上,出版社作出了一段如今看来颇为轻描淡写的说明:“尽管莱纳德教授对前辈科学家的研究展现出深厚的学识和令人钦佩的客观态度,但一旦涉及到他同时代的人物,他往往会因强烈的现实观点而影响判断。他在世时也不愿接受对该系列最后一篇研究所提出的某些修改建议。”

德国的纳粹主义意识形态鼓吹者声称,相对论与唯物主义和马克思主义密不可分。[27] 其中最著名的代表人物便是莱纳德。[27]莱纳德后来成为“德意志物理学”(Deutsche Physik,即“雅利安物理学”)的领袖人物,该科学哲学体系将阿尔伯特·爱因斯坦及其他犹太物理学家的贡献贬斥为“犹太物理学”。[28] 爱因斯坦揭示了光电效应的机制,这是莱纳德虽在实验上观测到但在理论上未能解释清楚的现象。[28]

莱纳德称爱因斯坦的理论是“犹太人的骗局”。[29]他尤其痛恨基于高等数学的物理学。[30] 莱纳德本人,以及当时许多实验物理学家,普遍不理解高级数学。[31]他们通常用文字(而非数学语言)来描述实验室中的观察结果。[31]
\subsection{晚年}
1931年,莱纳德从海德堡大学理论物理教授职位上退休,并获得名誉教授(荣休)的头衔。然而,1945年,盟军占领当局将他从该职位驱逐,当时他已年满83岁。位于海德堡的赫尔姆霍兹中学曾在1927年至1945年间以他命名为“菲利普·莱纳德学校”。但随着战后清除纳粹街道名称和纪念碑的政策推行,该校于945年9月根据盟军军事政府的命令改名。[32]莱纳德于1947年在德国梅塞尔豪森去世。
\subsection{荣誉与奖项}
\begin{itemize}
\item 英国皇家学会:伦福德奖章,1896年
\item 意大利科学学会:马泰乌奇奖章,1896年
\item 法国科学院:拉卡兹奖,1897年[33]
\item 富兰克林研究所:富兰克林奖章,1932年
\item 诺贝尔物理学奖,1905年
\end{itemize}
2005年,月球北极附近有一座环形山以莱纳德命名(2008年正式批准)。但在2020年,国际天文学联合会(IAU)得知莱纳德与纳粹的联系后,决定取消该名称。[34][35]
\subsection{文化描写}
\begin{itemize}
\item 莱纳德对相对论的批评及其反对爱因斯坦的行动,被收录在纪录片系列 《黑暗物质:扭曲但真实》 中的一集,片段标题为“爱因斯坦的复仇”。
\item 莱纳德的生平及其与阿尔伯特·爱因斯坦之间在科学上的交错关系,是图书 《追踪爱因斯坦的人:纳粹科学家菲利普·莱纳德如何改变历史进程》的核心内容,作者为布鲁斯·J·希尔曼、比吉特·埃特尔-瓦格纳和伯恩德·C·瓦格纳。
\item 在2017年国家地理频道历史剧《天才》中,莱纳德由演员迈克尔·麦克埃尔哈顿饰演。[36]
\item 莱纳德还作为反派角色出现在动画系列《超级科学朋友》第七集中。
\end{itemize}
\subsection{参考书目}
\begin{itemize}
\item 莱纳德,菲利普(1906):《论阴极射线》(Über Kathodenstrahlen,德文)
\item 莱纳德,菲利普:《论以太与物质》(Über Aether und Materie,德文)
\item 莱纳德,菲利普(1914):《复杂分子的问题》(Probleme komplexer Moleküle,德文)
\item 莱纳德,菲利普(1918):《关于阴极射线的定量研究》(Quantitatives über Kathodenstrahlen,德文)
\item 莱纳德,菲利普(1918):《论相对性原理》(Über das Relativitätsprinzip,德文)
\item 莱纳德,菲利普(1921):《以太与原始以太》(Aether und Uraether,德文)
\item 莱纳德,菲利普(1930):《伟大的自然科学家》(Grosse Naturforscher,德文)
\item 莱纳德,菲利普(1931):《一位自然科学家的回忆》(Erinnerungen eines Naturforschers*,德文)新版:《一位自然科学家的回忆——1931/1843年原始打字稿的注释版》,Arne Schirrmacher 编,Springer Verlag 出版,海德堡,2010年,344页,ISBN 978-3-540-89047-8,电子版 ISBN 978-3-540-89048-5
\item 莱纳德,菲利普(1933):《科学中的伟人》,伦敦:G. Bell and Sons,OCLC 1156317
\item 莱纳德,菲利普(1936):《德意志物理学》(Deutsche Physik,德文),共四卷,J.F. Lehmann 出版社,OCLC 13814543
第1卷:导论与力学
第2卷:声学与热学
第3卷:光学与静电学、电动力学初探(或:光学与电学 第一部分)
第4卷:磁学、电动力学及其后续初探(或:电学 第二部分)
后续版本出版于1943年
\end{itemize}
\subsection{另见}
\begin{itemize}
\item 空腔磁控管
\item 液体细丝断裂现象
\end{itemize}
