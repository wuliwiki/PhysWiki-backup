% 莫尔斯引理

\begin{issues}
\issueTODO
\issueDraft
\end{issues}

\pentry{隐函数定理的不动点证明 \upref{IFTFix}}

莫尔斯引理在微分几何学中占有基本地位。 粗略地说, 它表示: 一个光滑函数在其非退化临界点附近的表现, 完全由其在这一点处的二阶导数决定。

\subsection{表述与直观}
\begin{lemma}{莫尔斯引理}
设$n$元光滑函数$f$定义在坐标原点附近, 满足$f'(0)=0$. 如果 Hessian 矩阵
$$
f''(0)=\left(\frac{\partial^2f}{\partial x^i\partial x^j}(0)\right)
$$
是可逆的, 则在坐标原点的某个小邻域上, 存在微分同胚$\varphi(x)$, 使得
$$
(f\circ\varphi)(x)=f(0)+\frac{1}{2}\langle f''(0)x,x\rangle.
$$
\end{lemma}

我们注意到, $f(0)+\frac{1}{2}\langle f''(0)x,x\rangle$正是$f$的泰勒展开截断到二阶的近似表达式. 因此, 莫尔斯引理说明: 在非退化临界点附近, 可以找到一个坐标变换, 使得$f$的表达式在新坐标之下与$f$的二阶近似重合.

\subsection{证明}
设$f''(0)=(c_{ij})$. 将新坐标$\varphi^{-1}(x)$记作$y$, 从而将方程重写为
$$
f(x)=f(0)+\sum_{i,j=1}^nc_{ij}y^iy^j.
$$
我们希望新坐标$y$是二阶接近旧坐标$x$的, 所以可以设
$$
y^k
=x^k+\sum_{i,j=1}^nA^k_{ij}(x)x^ix^j,
$$
其中$A_{ij}(x)$是待求解的光滑函数. 将$f(x)$进行泰勒展开到三阶, 得到
$$
f(x)=f(0)+\frac{1}{2}\langle f''(0)x,x\rangle+H(x)(x,x,x),
$$
其中最后一项是积分形式的余项给出的, $H(x)$是光滑的对称三重线性形式. 代入到待求解的方程, 得到
$$
\langle f''(0)A(x)(x,x),x\rangle
=H(x)(x,x,x)-\frac{1}{2}\langle f''(0)A(x)(x,x),A(x)(x,x)\rangle.
$$