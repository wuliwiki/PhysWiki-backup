% test
% keys test|测试|编辑器

\subsubsection{一些符号}
参考 \href{https://docs.julialang.org/en/v1/manual/unicode-input/}{Julia 符号表}和 \href{https://oeis.org/wiki/List_of_LaTeX_mathematical_symbols}{LaTeX 符号表}, 以及 \href{http://www.onemathematicalcat.org/MathJaxDocumentation/TeXSyntax.htm}{MathJax 符号表}.

\begin{equation}
\cap, \bigcap, \cup, \bigcup, \vee, \wedge, \int, \iint, \iiint, \oint
\end{equation}
\begin{equation}
\diamond, \ominus, \triangleleft, \triangleright, \Longleftarrow, \Longrightarrow, \iff, \leftrightarrow, \updownarrow, \cdots
\end{equation}
\begin{equation}
\ddots, \top, \bot, \measuredangle
\end{equation}

\begin{figure}[ht]
\centering
\includegraphics[width=5cm]{./figures/test_1.png}
\caption{Zero Two} \label{test_fig1}
\end{figure}

\subsubsection{化学式}
编辑器预览的 MathJax 3 开始支持化学式了. 但是网站用的 2 还不支持. 也不确定 LaTeX 是否支持.
\begin{equation}
\ce{SO4^2- + Ba^2+ -> BaSO4 v}
\end{equation}

\subsubsection{公式中的链接}
公式里面居然可以用 \verb|\href|, 这个功能很强大, 但不知道 texlive 是否支持.
\begin{equation}
\href{https://wuli.wiki/online}{a}^2 + b^2 = c^2
\end{equation}

\subsubsection{付费内容}
我们要学习的公式为
\begin{equation}\label{test_eq1}
a = 1
\end{equation}

\begin{example}{}
请问 $a$ 为多少?
\pay

实际答案就是
\begin{equation}\label{test_eq2}
a + 1 = 2
\end{equation}
\paid
\end{example}

\begin{equation}\label{test_eq3}
b = 3
\end{equation}


引用\autoref{test_eq1}, 以及隐藏的\autoref{test_eq2}. 注意\autoref{test_eq3} 的序号在隐藏后不会改变.
