\documentclass{article}  %类型
\usepackage[UTF8]{ctex}  %支持中文
\usepackage{amsmath} %数学公式
\usepackage{geometry} %页边距
\usepackage{graphicx} %插图



\geometry{left=2.0cm,right=2.0cm,top=0.5cm,bottom=2cm} %调整页边距
\title{周报——Deep Learning for Person Re-identification:
A Survey and Outlook}
\author{马磊 }
\date{July 31, 2020}

\begin{document}
\maketitle %显示标题
\section{简介}
这是一篇综述性文章,本文较细致地对现有的深度Re-ID方法的优缺点进行了讨论.\par
首先,作者通过将Re-ID分为封闭世界(closed-world)的ReID和开放世界(open-world)的Re-ID来陈述.其次,作者设计了一个强大的baseline——AGW,无论是单模态还是跨模态行人重识别,该方法都达到了SOTA,本文也提出了新的性能评价指标——mINP,来对CMC/mAP进行补充.最后,作者讨论了行人重识别任务中重要但尚在发展路上的问题.
\section{创新点}
\begin{itemize}
    \item 分Re-ID为closed-world和open-world两种来详细探讨;
    \item 设计了强大的baseline——AGW,在单模态和跨模态Re-ID任务均达到SOTA;
    \item 提出了新的性能评价指标——mINP,来指明找到所有正确匹配项的成本.
\end{itemize}
\section{详细内容}
\section{心得体会}
\end{document}