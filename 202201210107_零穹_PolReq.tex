% 极值的必要条件(变分学)
% 极值

\pentry{绝对极值与相对极值(变分学)\upref{AbPol},变分\upref{Varia}}
正如\autoref{AbPol_exe1}~\upref{AbPol}所示,绝对极值必定是相对极值,相对强的极值必定是相对弱的极值.所以,弱的相对极值的必要条件也是绝对极值和强的相对极值的必要条件.因为这个原因,弱的极值的必要条件就称为\textbf{极值的必要条件}.

曲线 $y=y(x)$ 实现了泛函 $J(y)$ 的极值的必要条件是:在曲线 $y=y(x)$ 的某个一级 $\epsilon-$ 邻区中,对任一曲线 $y=\overline{y}(x)$, $\Delta J=J(\overline{y})-J(y)$ 符号一定.

对于泛函 $J(y)=\int_a^bF(x,y,y')\dd x$ ,有下面关于极值的必要条件定理
\begin{theorem}{}\label{PolReq_the1}
设 $y(x)$ 是 $C_1$ 类的\autoref{Varia_sub1}~\upref{Varia},
\end{theorem},且 $y(a)=y_0,y(b)=y_1$.则 $y(x)$ 给出泛函 $J(y)=\int_a^bF(x,y,y')\dd x$ 的极值的必要条件是:变分
\begin{equation}
\delta J=\int_a^b[F_y(x,y,y')\eta(x)+F_{y'}(x,y,y')\eta'(x)]\dd x
\end{equation}
对于 $C_1$ 类的任意满足 $\eta(a)=\eta(b)=0$ 的函数 $\eta(x)$,$\delta J=0$.
\subsection{证明}
由弱的相对极值的定义\upref{AbPol} ,若 $y(x)$ 实现 $J(y)$ 的弱的相对极值,则对 $y(x)$ 的某个一级 $\epsilon-$ 邻区中,任意 $\overline{y}(x)$ ,在弱极大时, $\Delta J=J(\overline{y})-J(y)\leq 0$;在若极小时,$\Delta J=J(\overline{y})-J(y)\geq 0$.即 $\Delta J$ 符号一定.

为证明