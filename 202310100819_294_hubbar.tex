% Hubbard 模型
% keys Hubbard model|hubbard model|Hubbard模型|hubbard模型
% license Xiao
% type Tutor

\pentry{二次量子化\upref{SecQua}}

\subsection{Hubbard 模型的哈密顿量}

在本节中,通过对电子多体系统之间的相互作用进行近似计算,仅考虑其中的库伦相互作用来得到Hubbard模型的哈密顿量。

选择原子轨道基$|j\rangle =c_j^\dagger |0\rangle$,写出系统的哈密顿量:

\begin{equation}
H=\sum\limits_{i,j,\sigma}\langle i |H_0| j \rangle c_{i,\sigma}^\dagger c_{j,\sigma}^+\sum\limits_{l,m,k,n,\sigma,\sigma'}\langle l,m|V|k,n\rangle c_{l,\sigma}^\dagger  c_{m,\sigma'}^\dagger c_{n,\sigma'}^~c_{k,\sigma}^~~.
\end{equation}

上式中的$H_0$代表动能部分的哈密顿量,其对应的$\sum\limits_{i,j,\sigma}\langle i |H_0| j \rangle c_{i,\sigma}^\dagger c_{j,\sigma}^~$部分为跃迁项。

上式中后一部分代表相互作用项,由于库仑相互作用并不影响自旋,所以电子被相互作用散射后自旋不变,这一点在哈密顿量中的体现则为发生散射后原本自旋为$\sigma$的粒子自旋还为$\sigma$,自旋为$
\sigma'$的粒子散射后自旋还为$\sigma'$。也就是说上式中后一项中$l$和$k$对应的是同一粒子的两个态,而$m$和$n$对应的是同一个粒子的两个态。

下面分别对哈密顿量的两项进行近似计算。

\subsubsection{跃迁项}
作为近似,我们仅考虑最近邻跃迁:

\begin{equation}
\langle i|H_0|j \rangle=\left\{
\begin{array}{lc}
~~\varepsilon~,~~~~~ i=j, \\
~-t~,~i,j\text{最近邻,} \\
~~0~,~~~~\text{其他情况。}
\end{array}\right.~
\end{equation}

这样便有:

\begin{equation}
\sum\limits_{i,j,\sigma}\langle i|H_0|j \rangle c_{i,\sigma}^\dagger c_{j,\sigma}^~=\sum\limits_{i,\sigma}\varepsilon c_{i,\sigma}^\dagger c_{i,\sigma}^~-\sum\limits_{i,\Delta,\sigma}t\left(c_{i+\Delta,\sigma}^\dagger c_{i,\sigma}^~+h.c.\right)~.
\end{equation}

\subsubsection{相互作用项}

考虑库仑相互作用是按照$r^{-1}$衰减的,且:

$$\langle l,m|V|k,n\rangle=\iint \phi_l^*(r)\phi_m^*(r')\frac{e^2}{\abs{r-r'}}\phi_n^~(r')\phi_k^~(r)drdr'~.$$

那么显而易见的是当$l=m=n=k$时积分值最大。那么仅考虑此最大值,忽略其他情况。计其积分值为$\frac{U}{2}$。那么:

\begin{equation}
\sum\limits_{l,m,n,k,\sigma,\sigma'}\langle l,m|V|k,n\rangle c_{l,\sigma}^\dagger c_{m,\sigma'}^\dagger c_{n,\sigma'}^~c_{k,\sigma'}^~=\frac{U}{2}\sum\limits_{l,\sigma,\sigma'}c_{l,\sigma}^\dagger c_{l,\sigma'}^\dagger c_{l,\sigma'}^~c_{l,\sigma'}^~~.
\end{equation}
展开自旋的求和可得:
\begin{equation}
\frac{U}{2}\sum\limits_l \left(c_{l,\uparrow}^\dagger c_{l,\uparrow}^\dagger c_{l,\uparrow}^~c_{l,\uparrow}^~+c_{l,\uparrow}^\dagger c_{l,\downarrow}^\dagger c_{l,\downarrow}^~c_{l,\uparrow}^~+c_{l,\downarrow}^\dagger c_{l,\uparrow}^\dagger c_{l,\uparrow}^~c_{l,\downarrow}^~+c_{l,\downarrow}^\dagger c_{l,\downarrow}^\dagger c_{l,\downarrow}^~c_{l,\downarrow}^~\right)~.
\end{equation}
由于电子是费米子,而第一项与第四项中都出现了两个湮灭算符,所以其都为0,那么我们只需要考虑第二项和第三项,考虑反对易关系$\left\{c_i^\dagger,c_j^~\right\}=\delta_{ij}$,$\left\{c_i^~,c_j^~\right\}=0$(以$\bar{\sigma}$表示与$\sigma$相反的自旋):
\begin{equation}
\begin{aligned}
\sum\limits_{l,\sigma,\sigma'}c_{l,\sigma}^\dagger c_{l,\sigma'}^\dagger c_{l,\sigma'}^~c_{l,\sigma'}^~&=\frac{U}{2}\sum\limits_l \left(c_{l,\uparrow}^\dagger c_{l,\downarrow}^\dagger c_{l,\downarrow}^~c_{l,\uparrow}^~+c_{l,\downarrow}^\dagger c_{l,\uparrow}^\dagger c_{l,\uparrow}^~c_{l,\downarrow}^~\right) \\
&=\frac{U}{2}\sum\limits_{l,\sigma}c_{l,\sigma}^\dagger c_{l,\bar\sigma}^\dagger c_{l,\bar\sigma}^~ c_{l,\sigma}^~ \\
&=\frac{U}{2}\sum\limits_{l,\sigma}c_{l,\sigma}^\dagger c_{l,\sigma}^~ c_{l,\bar\sigma}^\dagger c_{l,\bar\sigma}^~ \\
&=\frac{U}{2}\sum\limits_{l,\sigma}\hat{n}_{l,\sigma}^~ \hat{n}_{l,\bar{\sigma}}^~ \\
&=U\sum\limits_{l}\hat{n}_{l,\uparrow}^~\hat{n}_{l,\downarrow}^~
~.
\end{aligned}
\end{equation}

经过两次近似,得到Hubbard模型的哈密顿量为:

\begin{equation}
H=\sum\limits_{i,\sigma}\varepsilon c_{i,\sigma}^\dagger c_{i,\sigma}^~-\sum\limits_{i,\Delta,\sigma}t\left(c_{i+\Delta,\sigma}^\dagger c_{i,\sigma}^~+h.c.\right)+U\sum\limits_{l}\hat{n}_{l,\uparrow}^~\hat{n}_{l,\downarrow}^~~.
\end{equation}
