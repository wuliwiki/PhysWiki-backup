% 正定二次型
% 正定型|雅可比方法

\begin{issues}
\issueOther{雅可比方法补充例子}
\end{issues}

\pentry{实二次型\upref{RQuaF}}
\subsection{正定二次型}
\begin{definition}{实二次型的分类}
非退化的实二次型 $q:V\rightarrow\mathbb R$ 称为\textbf{正定的}(\textbf{负定的}),如果 $q(\bvec x)>0(q(\bvec x)<0)$ 对任意矢量 $\bvec x\neq0$ 都成立。$q$ 称为\textbf{半正定的}(或\textbf{非负定的}),如果 $q(\bvec x)\geq0$ 对所有 $\bvec x\in V$ 成立。最后,$q$ 称为\textbf{不定的},如果它有时取正有时取负。
\end{definition}
由于实二次型均可化为标准型\autoref{RQuaF_the1}~\upref{RQuaF},故实二次型的各种类型对应的标准型如下($n=\mathrm{dim}_\mathbb R \,V$):
\begin{enumerate}
\item 正定型:\begin{equation}\label{DeQua_eq1}
q(\bvec x)=\sum_{i=1}^n x_i^2;
\end{equation}
\item 负定型:\begin{equation}
q(\bvec x)=-\sum_{i=1}^n x_i^2;
\end{equation}
\item 半正定型:\begin{equation}
q(\bvec x)=\sum_{i=1}^r x_i^2,\quad r\leq n;
\end{equation}
\item 不定型:\begin{equation}
q(\bvec x)=\sum_{i=1}^s x_i^2-\sum_{i=s+1}^r x_i^2,\quad0<s< r.
\end{equation}
\end{enumerate}
\begin{definition}{正定双线性型}\label{DeQua_def1}
与正定二次型相配极的双线性型\upref{QuaFor}称为\textbf{正定的}。
\end{definition}
类似的术语同样可照搬到矩阵上,因为二次型对应一个与之配极的双线性型,双线性型又对应一个矩阵,它们之间这样一一对应的关系使得术语可照搬。
\begin{theorem}{}
矩阵 $F$ 是正定矩阵的充要条件为
\begin{equation}
F=A^TA~.
\end{equation}
其中,$A$ 是实的非退化矩阵。
\end{theorem}
\textbf{证明:}
\begin{enumerate}
\item \textbf{必要性:}
因为正定矩阵的标准型为单位矩阵 $E$,即在某基底下,正定矩阵 $F$ 化为 $E$,设这两基底对应的过渡矩阵为 $B$ (这显然是个非退化矩阵,因为两基底可相互表示),于是
\begin{equation}
B^TFB=E\quad\Rightarrow\quad F={(B^T)}^{-1}EB^{-1}={(B^{-1})}^TB^{-1}
\end{equation}
令 $A=B^{-1}$,便得 $F=A^TA$。
\item \textbf{充分性:}因为 $F=A^TA=A^TEA$,而 $A$ 非退化,所以
\begin{equation}
{(A^{-1})}^TFA^{-1}=E
\end{equation}
即在过渡矩阵 $A^{-1}$ 下,矩阵 $F$ 化为 $E$,由\autoref{DeQua_eq1} ,可知 $F$ 正定。
\end{enumerate}
\textbf{证毕!}
\subsection{雅可比方法}
\begin{definition}{顺序主子式}
\begin{equation}
\Delta_1=f_{11},\;\cdots,\;\Delta_i=\begin{vmatrix}
f_{11}&\cdots&f_{1k}\\
\vdots&\vdots&\vdots\\
f_{i1}&\cdots&f{ii}
\end{vmatrix},\quad
\cdots
\end{equation}
称为矩阵 $F=(f_{ij})$的\textbf{顺序主子式}。$\Delta_i$ 称为\textbf{ $F$ 的 $i$ 阶顺序主子式}。且约定 $\Delta_0=1$
\end{definition}
\begin{theorem}{雅可比方法}
设 $q$ 以 $F$ 为矩阵的实二次型,$F$ 的所有顺序主子式都不为9。那么,必有空间 $V$ 的基底 $(\bvec e'_1,\cdots,\bvec e'_n)$ ,使得 $q(\bvec x)$ 具有规范形式
\begin{equation}\label{DeQua_eq4}
q(\bvec x)=\sum_{i=1}^n\frac{\Delta_{i-1}}{\Delta_i}(x'_i)^2
\end{equation}
\end{theorem}
\textbf{证明:}
\begin{enumerate}
\item 对 $n=1$ 的矢量空间 $V_1$,命题显然成立。
\item 假设对 $n=k-1$ 的矢量空间 $V_{k-1}$,命题成立。设 $(\bvec e_1,\cdots,\bvec e_k)$ 是矢量空间 $V_k$ 的初始基底,$q$ 是其上以 $F$ 为矩阵的二次型,考查 $k-1$ 维子空间
\begin{equation}
L=\langle\bvec e_1,\cdots,\bvec e_{k-1}\rangle
\end{equation}
设 $\overline q=q|_L$ 是 $q$ 在 $L$ 上的限制,则型 $\overline q$ 的矩阵 $\overline F$ 是由 $F$ 去掉最后一行与最后一列得到的,故据条件,它的顺序主子式 $\overline\Delta_i=\Delta_i,\;i=1,\cdots,k-1$ 都不为0。由归纳假设,在 $L$ 中必有一基底 $(\bvec e'_1,\cdots,\bvec e'_{k-1})$,使得对 $\overline{\bvec x}\in L$
\begin{equation}
\overline q(\overline{\bvec x})=q(\overline{\bvec x})=\sum_{i=1}^{k-1}\frac{\Delta_{i-1}}{\Delta_i}{x'_{i}}^2
\end{equation}
上式相当于,对与 $q$ 配极的双线性型 $f$:
\begin{equation}
f(\bvec e'_i,\bvec e'_i)=\frac{\Delta_{i-1}}{\Delta_{i}},\quad f(\bvec e'_i,\bvec e'_j)=0,\quad 1\leq i\neq j\leq k-1.
\end{equation}
 $k$ 个未知量 $x'_1,\cdots,x'_k$ 的 $k-1$ 个齐次方程组
 \begin{equation}\label{DeQua_eq2}
 f(\bvec x,\bvec e'_i)=0,\quad i=1,\cdots,k-1,\;\bvec x\in V_k
 \end{equation}
 必 在 $V_k$ 中有非零解,设为 $\bvec x=\bvec e'_k$。向量组 $(\bvec e'_1,\cdots,\bvec e'_k)$ 必构成 $V_k$ 的一基底,因为否则
 \begin{equation}
 \sum_{i=1}^k\alpha_i\bvec e'_i=\bvec 0
 \end{equation}
 意味着只能 $\alpha_k\neq0$,那么 
 \begin{equation}
 \bvec e'_k=\sum_i^{k-1}\beta_i\bvec e'_i\quad
 \Rightarrow\quad 0\neq f(\bvec e'_k,\bvec e'_k)=f(\bvec e'_k,\sum_i^{k-1}\beta_i\bvec e'_i)=0
 \end{equation}
 矛盾。\\

 确定矢量 $\bvec e'_k$ 可以精确到其坐标,由\autoref{DeQua_eq2}, $\bvec e'_k$ 所在的解空间至少是1维的(对应 $k-1$ 个线性函数 $f_i=f(*,\bvec e'_i),\;(i=1,\cdots,k-1)$ 线性无关的情形),即在确定 $\bvec e'_k\in V_k$ 坐标的时候,至少有一个坐标分量可任选,而 $\bvec e'_k$ 一旦选定,其在初始基底下的坐标分量就确定了,也即过渡矩阵得到确定。设基底 $\bvec e_i$ 到 $\bvec e'_i$ 的过渡矩阵为 $A$,于是我们可以令
 \begin{equation}\label{DeQua_eq3}
 \det A=(\Delta_k)^{-1}=(\det\,F)^{-1}
 \end{equation}
 来确定这一任一性 (亦可从\autoref{DeQua_eq2} 和\autoref{DeQua_eq3} 关于 $k$个未知数 $k$ 个方程看出)。

设 $F'$ 是线性型 $f$ 在 $\bvec e'_i$ 下的矩阵,于是
\begin{equation}
\begin{aligned}
\frac{f(\bvec e'_k,\bvec e'_k)}{\Delta_{k-1}}&=\frac{\Delta_0}{\Delta_1}\frac{\Delta_1}{\Delta_2}\cdots\frac{\Delta_{k-2}}{\Delta_{k-1}}f(\bvec e'_k,\bvec e'_k)\\
&=\prod_{i=1}^k f(\bvec e'_i,\bvec e'_i)=\det\, F'=\det\, (A^TFA)\\
&=(\det{\,A})^2\det\, F=\frac{1}{\Delta_k}
\end{aligned}
\end{equation}
于是
\begin{equation}
f(\bvec e'_k,\bvec e'_k)=\frac{\Delta_{k-1}}{\Delta_k}
\end{equation}
于是,二次型在基底 $\bvec e'_i$ 下就为\autoref{DeQua_eq4} 的形式。\\
根据数学归纳法,命题\textbf{得证!}
\end{enumerate}
