% 波面、光线和光束
% license Usr
% type Tutor

\addTODO{请先申请成为作者,方便交流}

研究光,就必须要有发出光的物体,像太阳、灯泡等可以\textbf{辐射光能的物体称为光源}。现实生活中的光源都有着一定大小,光源大小相应会影响到光场光的特性。但如果这一影响微乎其微,我们则可以将这样的光源抽象成一个几何点,称为\textbf{点光源}。

光源发出的光是一种电磁波,按照波动光学的研究方法,我们可以使用描述电磁波的基本方法来描述光波,比如使用频率、波长和相位等物理量。波动光学中,对于由同一光源发出的单色波(即只有一种频率的光波),把\textbf{同一时刻相位相同的各点连接起来形成的曲面,即为该光波的波面}。对于单色点光源来说,它的波面为球面,光波沿着垂直于球面的方向向外传播(即法线方向),波动光学中把这个方向定义为光波的方向,用波矢量来描述。

光波的传播过程实际上是光能量的传播过程。光能量在空间中的传播用能流密度矢量描述。在各向同性介质中,能流密度矢量与波矢量方向相同,即光波方向代表了能量流动方向。我们可以把这两个矢量对应的方向抽象为一条几何线,这就是几何光学中\textbf{光线}的由来。

按照上述抽象,一个波面可以对应无限条光线,它们构成了一个光线束,即\textbf{光束}。如果光束中光线直接相交于一点,或者它们的反向延长线相交于一点,这样的光束称为同心光束;如果光束中光线均平行,即平行光束。

\begin{issues}
\issueTODO 同心光束与平行光束的示例图
\end{issues}

几何光学中对光传播规律和各常规光学元件成像原理的研究,就是从光线的概念开始的。
