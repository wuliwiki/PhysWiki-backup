% 仿射集
% keys 仿射|直线
% license Xiao
% type Tutor

\begin{issues}
\issueDraft
\end{issues}

\pentry{向量子空间\nref{nod_SubSpc}}{nod_6f36}

\addTODO{增加一个简介}

\begin{definition}{仿射组合与仿射集}
取向量空间 $V$ (记其域为 $\mathbb{F}$) 中的两点 $x_1, x_2 \in V$ 的线性组合 $a_1 x_1 + a_2 x_2$ 被称为 $x_1, x_2$ 的\textbf{仿射组合},如果满足 $a_1 + a_2 = 1$;更一般的,系数和为一(即$\sum_i a_i = 1$)的线性组合 $\sum_i a_i x_i$ 被称为 $x_1, \dots, x_n$ 的\textbf{仿射组合}。

向量空间的子集 $C \subseteq V$ 被称为\textbf{仿射集}(affine set),意味着 $C$ 中的任意仿射组合都在 $C$ 中;等价的,我们只需要考虑任意两个向量的仿射组合即可(和向量子空间的情况一样)。
\end{definition}

从几何上看,仿射集仍然是平直的,或者说“线性的”:

\begin{theorem}{}
对任意的仿射集 $C \subseteq V$,存在唯一的向量子空间 $U \subseteq V$,使得对任意的 $x \in C$,我们有
\begin{equation}
C = x + U = \{ x + v \mid v \in U\}~.
\end{equation}
\end{theorem}

\textbf{证明:}第一步:取一点 $x_0 \in C$,我们定义
\begin{equation}
U_0: = \{ x - x_0 \mid x \in C \}~,
\end{equation}
要证明它是一个向量子空间:
\begin{enumerate}
\item $0_V = x_0 - x_0 \in U_0$,
\item 
\end{enumerate}



\textbf{证毕!}



\begin{theorem}{}
对于向量空间的子集 $C \subseteq V$,当过$C$中任意不同的两点的直线仍然在$C$中时,$C$为\textbf{仿射集}(Affine set),此时,对于任意$x_1,x_2 \in C$,$\theta \in \mathbb{F}$,有$\theta x_1+(1-\theta)x_2 \in C$.换句话说,$C$中任意两点的线性组合仍然在$C$内,线性组合的系数之和为$1$.
\end{theorem}


\begin{figure}[ht]
\centering
\includegraphics[width=12cm]{./figures/a3c7ce94fa5d9e0a.png}
\caption{仿射集定义示意图} \label{fig_AffSet_1}
\end{figure}

图$1$表示的是一条直线穿过$x_1,x_2$两点,也即$x_1,x_2$两点的仿射组合$\theta x_1+(1-\theta)x_2$的全体形成了该直线。根据仿射集的定义,该条直线就是一个仿射集。当$0<\theta<1$,形成图中直线加粗的部分;反之,形成直线上细线表示的部分。

仿射集上述定义可以推广到多点的情形。我们把$\theta_1x_1+\theta_2x_2+...+\theta_kx_k$称为点$x_1,x_2,...,x_k$的\textbf{仿射组合}(Affine combination),其中$\theta_1+\theta_2+...+\theta_k=1$。

基于仿射组合的概念,仿射集有性质:若集合$C$是仿射集,有点$x_1,x_2,...,x_k \in C$,且$\theta_1+\theta_2+...+\theta_k=1$,则$\theta_1x_1+\theta_2x_2+...+\theta_kx_k$也在集合$C$内。
