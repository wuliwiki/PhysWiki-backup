% 纳维-斯托克斯方程(综述)
% license CCBYSA3
% type Wiki

本文根据 CC-BY-SA 协议转载翻译自维基百科\href{https://en.wikipedia.org/wiki/Navier\%E2\%80\%93Stokes_equations}{相关文章}。

纳维–斯托克斯方程(/nævˈjeɪ stoʊks/,音译为“纳维-耶 斯托克斯”)是一组描述黏性流体运动的偏微分方程。这些方程以法国工程师兼物理学家克洛德–路易·纳维和爱尔兰物理学家、数学家乔治·加布里埃尔·斯托克斯的名字命名。它们的理论是在数十年的发展过程中逐步建立起来的,从1822年(纳维的工作)到1842–1850年(斯托克斯的研究)。

纳维–斯托克斯方程在数学上表达了牛顿流体的动量平衡,并结合了质量守恒原理。有时,这些方程会配合状态方程一起使用,用来联系压力、温度和密度$1$。它们的推导源于将牛顿第二定律应用于流体运动,并假设流体中的应力可以分解为扩散型的黏性项(与速度梯度成正比)和压力项,从而描述黏性流动。与密切相关的欧拉方程相比,纳维–斯托克斯方程考虑了黏性,而欧拉方程仅适用于无黏性流动。正因为如此,纳维–斯托克斯方程属于椭圆型方程,因此具备更好的解析性质,但代价是数学结构较少(例如,它们从不完全可积)。

纳维–斯托克斯方程非常有用,因为它们描述了许多科学和工程领域中感兴趣现象的物理规律。它们可以用来模拟天气、洋流、管道中的水流以及机翼周围的空气流动。在完整形式或简化形式下,纳维–斯托克斯方程都能帮助进行飞机和汽车的设计、研究血液流动、设计发电站、分析污染问题,以及解决许多其他复杂问题。如果将其与麦克斯韦方程组耦合,还可以用于模拟和研究磁流体力学现象。

从纯数学角度来看,纳维–斯托克斯方程同样具有极高的研究价值。尽管它们有着广泛的实际应用,但至今尚未被证明三维情况下的光滑解是否总是存在——也就是说,解在定义域的所有点上是否无限可微(甚至只是有界)仍是一个未解的问题。这被称为“纳维–斯托克斯方程的存在性与光滑性问题”。克雷数学研究所将其列为数学界七大未解难题之一,并悬赏100万美元奖励对该问题的证明或反例\(^\text{[2][3]}\)。
