% 格林恒等式(综述)
% license CCBYSA3
% type Wiki

本文根据 CC-BY-SA 协议转载翻译自维基百科\href{https://en.wikipedia.org/wiki/Green\%27s_identities}{相关文章}。

在数学中,格林恒等式是一组包含三条公式的向量分析恒等式,用于关联区域内部(体积部分)与其边界在微分算子作用下的关系。它们以发现格林定理的数学家乔治·格林的名字命名。
\subsection{格林第一恒等式}
这个恒等式可以通过将散度定理应用于向量场 $\mathbf{F} = \psi \nabla \varphi$ 并利用乘积法则的推广形式
$$
\nabla \cdot (\psi \mathbf{X}) = \nabla \psi \cdot \mathbf{X} + \psi \nabla \cdot \mathbf{X}~
$$
推导出来。
设 $\varphi$ 和 $\psi$ 是定义在某个区域 $U \subset \mathbb{R}^d$ 上的标量函数,其中 $\varphi$ 是二阶连续可微函数,$\psi$ 是一阶连续可微函数。令 $\mathbf{X} = \nabla \varphi$,并将 $\nabla \cdot (\psi \nabla \varphi)$ 在 $U$ 上积分,则有[1]:
$$
\int_U \left(\psi \,\Delta \varphi + \nabla \psi \cdot \nabla \varphi \right) \, dV
= 
\oint_{\partial U} \psi \, (\nabla \varphi \cdot \mathbf{n}) \, dS
=
\oint_{\partial U} \psi \, \nabla \varphi \cdot d\mathbf{S},~
$$
该定理是**散度定理**的一个特殊情形,本质上是**分部积分**在高维情况下的对应形式,其中 $\psi$ 和 $\varphi$ 的梯度分别代替了 $u$ 和 $v$。

需要注意的是,上述 **格林第一恒等式** 其实是一个更一般恒等式的特例,这个更一般的恒等式是通过在散度定理中代入 $\mathbf{F} = \psi \mathbf{\Gamma}$ 得到的:
$$
\int_U \left(\psi \, \nabla \cdot \mathbf{\Gamma} + \mathbf{\Gamma} \cdot \nabla \psi \right)\, dV
=
\oint_{\partial U} \psi \, (\mathbf{\Gamma} \cdot \mathbf{n}) \, dS
=
\oint_{\partial U} \psi \, \mathbf{\Gamma} \cdot d\mathbf{S}~
$$
\subsection{格林第二恒等式}
如果 $\varphi$ 和 $\psi$ 都是在区域 $U \subset \mathbb{R}^3$ 上二阶连续可微的函数,并且 $\varepsilon$ 在 $U$ 上是一阶连续可微的函数,则取$\mathbf{F} = \psi \varepsilon \nabla \varphi - \varphi \varepsilon \nabla \psi$可以得到:
$$
\int_U \left[\psi \,\nabla \cdot (\varepsilon \nabla \varphi) - \varphi \,\nabla \cdot (\varepsilon \nabla \psi)\right]\, dV
=
\oint_{\partial U} \varepsilon \left(\psi \frac{\partial \varphi}{\partial \mathbf{n}} - \varphi \frac{\partial \psi}{\partial \mathbf{n}}\right)\, dS~
$$
在 $\varepsilon = 1$ 的特殊情况下,公式化简为:
$$
\int_U \left(\psi \nabla^2 \varphi - \varphi \nabla^2 \psi\right) \, dV
=
\oint_{\partial U} \left(\psi \frac{\partial \varphi}{\partial \mathbf{n}} - \varphi \frac{\partial \psi}{\partial \mathbf{n}}\right) \, dS~
$$
在上式中,$\displaystyle \partial \varphi/\partial \mathbf{n}$ 表示 $\varphi$ 沿着指向外侧的单位法向量 $\mathbf{n}$ 的方向导数:
$$
\frac{\partial \varphi}{\partial \mathbf{n}}
= \nabla \varphi \cdot \mathbf{n}
= \nabla_{\mathbf{n}} \varphi~
$$
将这个定义显式代入 $\varepsilon = 1$ 的第二格林恒等式,得到:
$$
\int_U \left(\psi \nabla^2 \varphi - \varphi \nabla^2 \psi\right) \, dV
=
\oint_{\partial U} \left(\psi \nabla \varphi - \varphi \nabla \psi\right) \cdot d\mathbf{S}v~
$$
特别地,这个结果表明:对于在边界上消失的函数,拉普拉斯算子在 $L^2$ 内积下是自伴算子,因此上式右边的边界积分项为零。
\subsection{格林第三恒等式}
格林第三恒等式是从第二恒等式推导出来的,方法是取 $\varphi = G$,其中 $G$ 是拉普拉斯算子 $\Delta$ 的基本解(即格林函数)。这意味着:

$$
\Delta G(\mathbf{x}, \boldsymbol{\eta}) = \delta(\mathbf{x} - \boldsymbol{\eta}).~
$$

---

例如,在 $\mathbb{R}^3$ 中,格林函数的一个解为:

$$
G(\mathbf{x}, \boldsymbol{\eta}) = \frac{-1}{4\pi \|\mathbf{x} - \boldsymbol{\eta}\|} ~.~
$$

---

格林第三恒等式表明,如果 $\psi$ 是在区域 $U$ 上**二阶连续可微**的函数,则有:

$$
\int_{U} \left[ G(\mathbf{y}, \boldsymbol{\eta}) \, \Delta \psi(\mathbf{y}) \right] \, dV_{\mathbf{y}}
- \psi(\boldsymbol{\eta})
=
\oint_{\partial U} 
\left[
G(\mathbf{y}, \boldsymbol{\eta}) 
\frac{\partial \psi}{\partial \mathbf{n}} (\mathbf{y})
-
\psi(\mathbf{y})
\frac{\partial G(\mathbf{y}, \boldsymbol{\eta})}{\partial \mathbf{n}}
\right] 
\, dS_{\mathbf{y}} ~.
$$

---

其中:

* $\Delta$ 表示拉普拉斯算子;
* $\partial U$ 是区域 $U$ 的边界;
* $\mathbf{n}$ 表示边界外法向量;
* $\displaystyle \frac{\partial \psi}{\partial \mathbf{n}}$ 和 $\displaystyle \frac{\partial G}{\partial \mathbf{n}}$ 是沿外法向量方向的导数;
* $\delta$ 是狄拉克δ函数,表示在 $\mathbf{x} = \boldsymbol{\eta}$ 的奇异点。
