% 南京理工大学 普通物理 B(845)模拟五套卷 第三套
% license Usr
% type Note

\textbf{声明}:“该内容来源于网络公开资料,不保证真实性,如有侵权请联系管理员”

\subsection{一、 填空题 I(26 分,每空 2 分)}
1. 一质点的运动方程(SI)为:$x=-10t+30t^2,y=15t-20t^2$ ,则质点的起始速度为__________,质点加速度为____________。

2. 质量为 $m$,长为 $l$ 的匀质细杆,可绕其端点的水平轴在竖直平面内自由转动。如果将细杆置于水平位置,然后让其由静止开始自由下摆,则开始转动的瞬间,细杆的角加速度为_____________,细杆转动到竖直位置时的角速度为_________________。

3. 如图所示,一长为 $l$ 的均匀直棒可绕过其一端且与棒垂直的水平光滑固定轴转动。抬起另一端使棒向上与水平面成 60°,然后无初速地将棒释放。已知棒对轴的转动惯量为$\frac{1}{3}ml^2$ ,其中 $m$ 和 $l$ 分别为棒的质量和长度,则放手时棒的角加速度为________,棒转到水平位置时的角加速度为_________________。
\begin{figure}[ht]
\centering
\includegraphics[width=6cm]{./figures/cab3d81ca7772da7.png}
\caption{} \label{fig_NJUD3_1}
\end{figure}
4. 已知一平面简谐波频率为$1000Hz$,波速为 $300m/s$,则波上相差 $\pi/4$的两点之间的距离为_______________,在某点处时间间隔为 $0.001s$的两个振动状态间的相位差为_________。

5. 一质点作简谐振动,速度最大值 $Vm=5cm/s$,振幅 $A=2cm$,若令速度具有正最大值的那一时刻为 $t=0$,则振动表达式为_____________。

6. 互感系数的物理意义是______________。

7. 在容积为 $10^{-2}m^3$ 的容易中,装有质量为 $100g$ 的气体,若气体分子的方均根速率为 $300mS^{-1}$,则气体的压强为____________。
\subsection{、 填空题 II(20 分,每空 2 分)}