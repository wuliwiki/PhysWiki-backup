% 开普勒问题的数值计算(Matlab)

\begin{issues}
\issueOther{应该数值计算含时运动并制作动画}
\issueOther{验证是否支持所有圆锥曲线, 包括引力和斥力}
\end{issues}

\pentry{Matlab 画图\upref{MatPlt}, 开普勒问题\upref{CelBd}}

本文讨论开普勒问题中, 若已知初始位置和初速度矢量, 求轨道方程和动画的算法并给出 Matlab 代码. 以下所有符号和定义沿用 “开普勒问题\upref{CelBd}”.
\begin{figure}[ht]
\centering
\includegraphics[width=11cm]{./figures/KepNum_1.pdf}
\caption{在同一位置以同一初速度不同仰角发射的椭圆轨道, 虚线为参考圆(例如地球表面).} \label{KepNum_fig1}
\end{figure}

\subsection{轨道的计算}
若 $F(r)$ 是平方反比的力(斥力为正引力为负), 即
\begin{equation}
F(r) = \frac{k}{r^2}
\end{equation}
其中 $k$ 为非零实常数. 由于质点的运动只和初始位置、初始速度和 $k$ 有关, 与质点的质量无关, 所以我们令 $m=1$ 并不影响一般性. 令发射位置的极坐标\upref{Polar}为 $(R,\alpha)$, 初速度为 $v_0$, 发射仰角为 $\beta$ (顺时针增量为正, $\beta=0$ 时质点延逆时针方向水平发射).

所以质点机械能为动能加势能 $E = v_0^2/2 + k/R$, 角动量(逆时针为正)为 $L = v_0 R\cos\beta$. 于是离心率 $e$ 可由\autoref{CelBd_eq2}~\upref{CelBd} 计算, 半通径 $p$ 由\autoref{CelBd_eq3}~\upref{CelBd}计算. 然后代入圆锥曲线的极坐标方程(\autoref{Cone_eq5}~\upref{Cone}) 即可得到轨道方程. 然而由于发射点已经选好, 实际的轨道还需要旋转一个角度 $\theta_0$ (逆时针为正), 即轨道方程变为
\begin{equation}\label{KepNum_eq1}
r(\theta)  = \frac{p}{1 - e\cos (\theta-\theta_0)}
\end{equation}
要求 $\theta_0$, 可以代入初始条件, 即 $r(\alpha) = R$. 得
\begin{equation}
\theta_0 = \alpha \pm \arccos(\frac{1-p/R}{e})
\end{equation}
这里有两个解是因为圆锥曲线和半径为 $R$ 的圆必定有两个交点. 我们需要根据发射角度来判断哪个是发射点, 稍作分析易得当 $\cos\beta \geqslant 0$ 时上式取正号, 否则取负号.

\subsection{运动的计算}
由\autoref{EqMoKp_eq3}~\upref{EqMoKp} 到\autoref{EqMoKp_eq4}~\upref{EqMoKp} 可以解出三种圆锥曲线的含时运动.

\textbf{椭圆}: 时间 $t$ 在 $[-T/2,T/2]$ 取值, $t=0$ 时质点位于近日点, $t=T/2$ (周期一半)时位于远日点, 周期由\autoref{Keple2_eq1}~\upref{Keple2} 计算. 先用二分法对每个 $t$ 数值解
\begin{equation}\label{KepNum_eq2}
t = \sqrt{\frac{ma^3}{-k}} (\psi - e \sin\psi)
\end{equation}
得到 $\psi(t)$, 再由\autoref{EqMoKp_eq1}~\upref{EqMoKp} 和\autoref{KepNum_eq1} 得
\begin{equation}
\theta = \theta_0 \pm \arccos\qty[\frac{1}{e} - \frac{p}{ae(1-e\cos\psi)}]
\end{equation}
正负号分别代表顺时针和逆时针运动(待验证), $\cos\beta > 0$ 时逆时针, $\cos\beta < 0$ 时顺时针. 根据对称性, 可以继续数值计算出 $t\in[-T/2,0]$ 时间段的角位置
\begin{equation}\label{KepNum_eq3}
\theta(t) = -\theta(-t) \quad (-T/2\leqslant t < 0)
\end{equation}
当然, 要生成 $t$ 的格点, 最好一开始就求出发射质点的初始时刻 $t_0$, 满足 $\theta(t_0) = \alpha$. 可以先求 $r(\alpha)$, 再代入\autoref{EqMoKp_eq1}~\upref{EqMoKp} 求 $\psi$, 最后代入\autoref{KepNum_eq2} 得 $t_0$.

\textbf{抛物线}: 时间 $t$ 在 $(-\infty,\infty)$ 取值, $t=0$ 时质点位于近日点, \autoref{EqMoKp_eq3}~\upref{EqMoKp}已经直接给出顺时针运动方程
\begin{equation}
t = \frac{L^3}{2mk^2} \qty(\tan\frac{\theta}{2} +  \frac{1}{3}\tan^3 \frac{\theta}{2}) \quad (0\leqslant \theta<\pi)
\end{equation}
若要考虑逆时针运动, 添加负号即可. 同样求出 $t_0$, 生成时间格点, 解方程得到 $t\in[0,\infty)$ 的 $\theta(t)$, 再使用\autoref{KepNum_eq3} 即可.

\textbf{双曲线}: 时间 $t$ 在 $(-\infty,\infty)$ 取值, $t=0$ 时质点位于近日点,
\addTODO{需要分引力和斥力两种情况讨论.}

以下 Matlab 程序支持对同一发射位置画出多组不同初速度和仰角的轨道.

\begin{lstlisting}[language=matlab]
% 已知初始位置、发射速度、发射方向, 求轨道以及运动方程
function kepler
% === 参数设置 ===
k = -1; % -GMm, (m=1)
R = 1; alpha = 0; % 初始位置(极坐标)
v0 = 1.1; % 初速度(支持行矢量或标量)
beta = linspace(0, 4*pi/11, 6); % 发射仰角(支持行矢量或标量)
% ==============

if isscalar(v0)
    v0 = repmat(v0, size(beta));
elseif isscalar(beta)
    beta = repmat(beta, size(v0));
elseif ~all(size(v0)==size(beta))
    error('v0 和 beta 的尺寸必须一样,或者其中之一为标量');
end

% 画参考圆
close all; figure;
tmp = linspace(0, 2*pi, 500);
plot(R*cos(tmp), R*sin(tmp), 'k--'); hold on;
scatter(0, 0); axis equal;
xlabel x; ylabel y;

for i = 1:numel(v0)
    kepler1(k, R, alpha, v0(i), beta(i)); hold on;
end
end

% 计算并画出一条轨道
function kepler1(k, R, alpha, v0, beta)
E = v0^2/2 + k/R; % 轨道总机械能
L = v0*cos(beta)*R; % 角动量
e = sqrt(1 + 2*E*L^2/k^2); % 离心率
p = L^2/abs(k); % 半通径

% 画图
dth = pi/20; Nth = 500;
if e < 1 % 椭圆
    th = linspace(0, 2*pi, Nth);
elseif e == 1 % 抛物线
    th = linspace(dth, 2*pi-dth, Nth);
else % 双曲线
    th1 = atan(sqrt(e^2-1)); % 渐进张角的一半
    th = linspace(th1+dth, 2*pi-th1-dth, Nth);
end

% 计算圆锥曲线需要旋转的角度 th0
if e == 0
    th0 = 0;
elseif cos(beta) >= 0
    th0 = alpha + acos((1-p/R)/e);
else
    th0 = alpha - acos((1-p/R)/e);
end
th0 = real(th0);
scatter(R*cos(alpha), R*sin(alpha));

th = th + th0;
r = p./(1 - e*cos(th-th0));
x = r .* cos(th); y = r .* sin(th);
plot(x, y); axis equal;
end
\end{lstlisting}
