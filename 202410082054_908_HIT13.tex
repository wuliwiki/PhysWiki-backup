% 哈尔滨工业大学 2013 年 考研 量子力学
% license Usr
% type Note

\subsection{(共 60 分,每小题10 分)}
\begin{enumerate}
\item 简单描述玻尔子模型理论的基本假设。由于验证了玻尔理论而获得诺贝尔物理学奖的著名物理学家是谁?实验证明了什么?

\item 系统的哈密顿算符为 
$$\hat{H} = \frac{p^2}{2\mu} + V(x),~$$
计算
$$\left[e^{ikx}[\hat{H}e^{-ikx}]\right],~$$
式中 $a$ 为常数。

\item 线谐振子在 $t = 0$ 时处于
$$ \psi(x, 0) = \frac{1}{2}\varphi_0(x) + \frac{\sqrt{3}}{2}\varphi_1(x) + \frac{1}{\sqrt{2}}\varphi_3(x)~ $$
态中,其中 $\varphi_n(x)$ 为第 $n$ 能量本征态对应的本征函数。
\begin{enumerate}
\item 求在 $\psi(x, 0)$ 态上能量的可测值、取值概率与平均值;
\item 写出 $t > 0$ 时刻的波函数及各能量取值概率与平均值。
\end{enumerate}

\item 平移算符 $\hat{D}(a)$ 的定义为 $\hat{D}(a)\varphi(x) = \varphi(x + a)$,证明 $\hat{D}(a)$ 可以用动量算符表示。

\item 证明:若一个 $2 \times 2$ 矩阵与 $2 \times 2$ 泡利矩阵均匀,则此 $2 \times 2$ 矩阵必可写为单位矩阵与一个常数之积:
$$\left( \begin{matrix} a & 0 \\\\ 0 & a \end{matrix} \right) = a \left( \begin{matrix} 1 & 0 \\\\ 0 & 1 \end{matrix} \right)~$$

\item 求厄米(Hermite)算符 $\hat{F} = \alpha\hat{p} + \beta\hat{x}$ 的本征值与本征态。
\end{enumerate}
