% 浙江大学 2012 年 考研 量子力学
% license Usr
% type Note

\textbf{声明}:“该内容来源于网络公开资料,不保证真实性,如有侵权请联系管理员”

\subsection{第一题(35分)简答题}
\begin{enumerate}
    \item  证明厄米算符的本征值为实数。
    
    \item  对于 $\hat{H} = \frac{\hat{p}^2}{2m} + \alpha \hat{l}_z$ ($\alpha$ 为常数),下列力学量中哪些是守恒量?
    
    \[
    \hat{H}, \hat{p}_x, \hat{p}_y, \hat{p}_z, \hat{p}^2, \hat{L}_x, \hat{L}_y, \hat{L}_z, \hat{L}^2~
    \]
    
    \item  原子的受激辐射和自发辐射区别在哪里?
    
    \item  你知道哪些纯粹量子效应?
    
    \item  写出泡利矩阵:
    
    \[
    \sigma^x = \begin{pmatrix}
    0 & 1 \\
    1 & 0
    \end{pmatrix}, \quad
    \sigma^y = \begin{pmatrix}
    0 & -i \\
    i & 0
    \end{pmatrix}, \quad
    \sigma^z = \begin{pmatrix}
    1 & 0 \\
    0 & -1
    \end{pmatrix}~
    \]
    
    满足的对易关系。
\end{enumerate}
\subsection{第二题(30分)}
电子被束缚在简谐振子势场 $V = \frac{1}{2} m \omega^2 x^2$ 中,若引入

\[
\hat{a} = \frac{1}{\sqrt{2}}\left( \frac{x}{x_0} + \frac{i \hat{p}}{x_0 m \omega} \right), \quad
\hat{a}^+ = \frac{1}{\sqrt{2}}\left( \frac{x}{x_0} - \frac{i \hat{p}}{x_0 m \omega} \right), \quad
x_0 = \sqrt{\hbar/m \omega}~
\]

则有

\[
H = \hbar \omega \left( \hat{a}^+ \hat{a} + \frac{1}{2} \right)~
\]

并有关系

\[
\hat{a}^+ |n\rangle = \sqrt{n+1} |n+1\rangle, \quad
\hat{a} |n\rangle = \sqrt{n} |n-1\rangle~
\]

显然基态满足$\hat{a} |0\rangle = 0$

\begin{enumerate}
    \item 试求基态波函数。
    \item 进一步求第一激发态的波函数。
    \item 如果势阱中有两个电子(忽略它们间的相互作用),它们整体的基态波函数是什么?(提示:电子为自旋1/2的全同粒子)。
    \item 如果加入均匀磁场$B$,问当$B$很强,超过某临界$B_c$时,(3)中所述基态还会是基态吗?试具体求$B_c$.
\end{enumerate}
\subsection{第三题(30 分)}
有一个质量为 $m$ 的粒子处在如下势阱中:

\[
V(x) =
\begin{cases}
\infty, & x < 0 \\
-V_0, & 0 < x < a \\
V_0, & a < x < a + b \\
0, & a + b < x 
\end{cases}
\quad \text{(这里 } V_0 > 0\text{)}~
\]

\begin{enumerate}
    \item 试求其能级与波函数。
    \item 问通过调节势阱宽度 $a$,能否让阱中的粒子有一定的几率穿透出来。
\end{enumerate}
\subsection{第四题(20 分)}
将质子看作是半径为 $R$ 的带电球壳,

\[
V(r) =
\begin{cases}
\frac{e}{R}, & r < R \\
\frac{e}{r}, & r > R 
\end{cases}
\quad \text{(其中 } e \text{ 为基本电荷值,} a_0 \text{ 为玻尔半径,} R \ll a_0\text{)}~
\]

计算由于质子(即氢原子核)的非点性引起氢原子基态能级的一阶修正。
\subsection{第五题(20 分)}
求哈密顿算符

\[
H = \sigma_1^x \sigma_2^x + \sigma_1^y \sigma_2^y + \alpha \sigma_1^z \sigma_2^z~
\]

的本征值和本征矢量,试分析 $\alpha = 1$ 时有何特点。提示:[泡利矩阵的下标 1, 2 表示第一个粒子和第二个粒子。因此可以用张量积来理解,即 $ \sigma_1^x \sigma_2^x = \sigma_1^x \otimes \sigma_2^x$ 等等]。
\subsection{第六题(15 分)}
有一个量子系统,假如你已知基态和激发态的波函数分别是 $\psi_0, \psi_1, \psi_2, \cdots$,其对应于 $E_0<E_1<E_2<E_3, \cdots$,把两个全同粒子(不考虑它们之间的相互作用)放到该系统:

\begin{enumerate}
    \item 对于自旋为零的粒子,写出基态与第一激发态的波函数。
    \item 对于自旋为 $1/2$ 的粒子,写出基态波函数。
\end{enumerate}
