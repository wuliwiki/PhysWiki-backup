% R-矩阵法(量子力学)

\pentry{球坐标系中的定态薛定谔方程\upref{RadSE}}

本文使用原子单位制\upref{AU}. $R$-矩阵法的中心思想是, 若要解
\begin{equation}
H = -\frac{1}{2m}\dv[2]{x} + V(r)
\end{equation}
在 $[0, \infty)$ 的散射态波函数, 把 $[0,a]$ 内的波函数用一组正交归一基底展开, 而在 $[a,\infty)$ 根据 $V(r)$ 的渐进形式写出近似的波函数, $a$ 越大, 该近似越精确. 最后在 $x=a$ 处, 匹配波函数.

\subsection{一维薛定谔方程}
一个算符是否为厄米算符与边界条件有关. 例如
\begin{equation}
H = -\frac{1}{2m}\dv[2]{x} + V(r)
\end{equation}
要证明厄米性, 用分部积分法得
\begin{equation}\label{Rmat_eq1}
\int_{-\infty}^{+\infty} uHv\dd{x} - \int_{-\infty}^{+\infty} vHu\dd{x}
= \eval{-\frac{1}{2m}[uv' - u'v]}_{-\infty}^{+\infty}
\end{equation}
由于我们假设波函数在无穷远处消失, 则该式为零, 说明 $H$ 是厄米的. 但如果在有限区间 $[0, a]$ 中, 则波函数的边界条件必须满足 $\eval{[uv' - u'v]}_{0}^{a} = 0$ 才能保证厄米性.

但若边界条件不符合该要求, 为了在 $[0,a]$ 内构造一组离散的正交归一基底, 我们可以拼凑一个厄米算符. 为了方便起见, 我们要求 $u(0) = v(0) = 0$. 把\autoref{Rmat_eq1} 移项并修改积分区间得
\begin{equation}
\qty[\int_{0}^{a} uHv\dd{x} + \frac{1}{2m}u(a)v'(a)] - \qty[\int_{0}^{a} vHu\dd{x} + \frac{1}{2m}u'(a)v(a)]
= 0
\end{equation}
而又可以通过狄拉克 $\delta$ 函数\upref{Delta} 表示为
\begin{equation}
\int_{0}^{a} u\qty[H + \frac{\delta(x-a)}{2m}\dv{x}] v\dd{x} -
\int_{0}^{a} v\qty[H + \frac{\delta(x-a)}{2m}\dv{x}] u\dd{x} = 0
\end{equation}
所以无论波函数在 $x=a$ 端的边界条件如何, 方括号中的算符都是厄米的. 令\textbf{布洛赫算符(Bloch operator)}为
\begin{equation}\label{Rmat_eq6}
\mathscr L = \delta(x-a)\dv{x}
\end{equation}
那么修正后的哈密顿算符(厄米算符)就是 $H + \mathscr L/(2m)$. 于是本征方程为
\begin{equation}\label{Rmat_eq2}
\qty(H + \frac{\mathscr L}{2m})\chi_j = \frac{k_j^2}{2m} \chi_j
\end{equation}
本征值为 ${k_i^2}/{2m}$. 于是 $\chi_i$ 就构成一组 $[0, a]$ 内的正交归一基底, 满足 $\int_0^a \chi_i \chi_j \dd{x} = \delta_{ij}$. 为了明确\autoref{Rmat_eq2} 的意义, 把\autoref{Rmat_eq2} 左乘 $\chi_i$ 并在 $[0,a]$ 积分得\footnote{该积分实际上是在区间 $[0,a+\epsilon]$ 积分然后令 $\epsilon\to 0^+$, 下同.}
\begin{equation}\label{Rmat_eq3}
\mel{\chi_i}{2mH + \mathscr L}{\chi_j} = k_i^2 \delta_{ij}
\end{equation}
这组基底如何具体计算呢? 首先还是要明确 $x=a$ 处的边界条件. 一个简单的例子是令 $u(a) = 0$, 解 $H$ 的本征基底. 这相当于解 $[0,a]$ 中的无限深势阱加上势能 $V(x)$. 另一个简单的例子是把边界条件 $u(a) = 0$ 改成 $u'(a) = 0$, 也能得到一组基底. 更妙地, 也可以把这两组基底合并. 可以验证这三组基底都满足\autoref{Rmat_eq3} 也就是\autoref{Rmat_eq2}. 当然还有其他不同的条件.

\subsection{展开散射态}
求出区间 $[0,a]$ 的基底以后, 就可以在该区间展开任意的散射态, 散射态满足
\begin{equation}\label{Rmat_eq4}
H\psi_k = \frac{k^2}{2m}\psi_k
\end{equation}
令
\begin{equation}\label{Rmat_eq5}
\psi_k = \sum_j c_j(k)\chi_j
\end{equation}
代入并左乘 $2m\chi_i$ 并在 $[0,a]$ 积分得
\begin{equation}
\sum_j \qty(2mH_{ij} - \delta_{ij}k^2) c_j = 0
\end{equation}
这是一个其次线性方程组, 可以解出坐标 $c_j$. 其中
\begin{equation}
2mH_{ij} = \mel{\chi_i}{2mH}{\chi_j} = \mel{\chi_i}{2mH + \mathscr L}{\chi_j} - \mel{\chi_i}{\mathscr L}{\chi_j} 
= k_i^2\delta_{ij} - \chi_i(a)\chi'_j(a)
\end{equation}
代入得
\begin{equation}\label{Rmat_eq7}
\sum_j \qty[(k_i^2 - k^2)\delta_{ij} - \chi_i(a)\chi'_j(a)] c_j = 0
\end{equation}

由此可以证明一个有用的关系(留做习题)
\begin{equation}\label{Rmat_eq8}
\frac{\psi_k'(a)}{\psi_k(a)} = \frac{1}{aR(k)}
\end{equation}
其中 $R(k^2)$ 就是 $R$-矩阵
\begin{equation}
R(k) = \frac{1}{a} \sum_{i=1}^\infty \frac{\chi_i^2(a)}{k_i^2 - k^2}
\end{equation}
事实上目前这只是一个数, 即 $1\times 1$ 的矩阵. 在多通道问题中才会成为真正的矩阵.

\subsection{势能修正项}
容易证明, 若给 $\mathscr L$ 的定义(\autoref{Rmat_eq6}) 加上一个任意实函数 $U(x)$, 也可以使 $H+\mathscr L/(2m)$ 为厄米矩阵. 有时候选取适当的 $U(x)$ 可以使基底 $\chi_i$ 的求解变得更简单. 按照同样的推导, \autoref{Rmat_eq7} 和\autoref{Rmat_eq8} 变为
\begin{equation}
\sum_j \qty[(k_i^2 - k^2)\delta_{ij} - \chi_i(a)\chi'_j(a) - U(a)\chi_i(a)\chi_j(a)] c_j = 0
\end{equation}
\begin{equation}
\frac{\psi_k'(a)}{\psi_k(a)} = \frac{1}{aR(k)} - U(a)
\end{equation}
