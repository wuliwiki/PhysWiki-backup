% 正规算子
% keys 酉变换|厄米算子|酉空间|厄米变换|厄密变换|厄密算子|hermitian|normal operator|hermitian operator
% license Xiao
% type Tutor



\pentry{伴随映射\nref{nod_AdjMap}}{nod_?}


“算子”一词是“线性算子”的简称,在本文中指代“线性变换”或者“线性映射”。


\begin{definition}{正规算子}

给定酉空间$U$上的线性变换$A$,令$A^\dagger$是其\textbf{伴随变换}。若$AA^\dagger = A^\dagger A$,则称$A$是$U$上的\textbf{正规算子(normal operator)}。

如果$A=A^\dagger$,则称$A$是一个\textbf{厄米算子(hermitian operator)}。如果$AA^\dagger = E$,其中$E$是单位方阵,则称$A$是一个\textbf{酉算子(unitary operator)}。

\end{definition}


显然,厄米算子和酉算子都是正规算子。厄米算子的性质非常好,而一般的算子总可以分解为厄米算子的组合,这就为研究算子带来了便利:



\begin{lemma}{}
任取酉空间$U$上的线性变换$A$,则存在$U$上\textbf{唯一}的厄米算子$H_1$和$H_2$,使得
\begin{equation}
A = H_1 + \I H_2~. 
\end{equation}
\end{lemma}


\textbf{证明}:

对于任意的$A$,定义
\begin{equation}
\begin{cases}
H_1 =\frac{1}{2}\qty(A+A^\dagger), \\
H_2 =\frac{1}{2\I}\qty(A-A^\dagger)~. 
\end{cases}
\end{equation}

显然$H_1+\I H_2=A$,因此\textbf{存在性}得证。

反过来,假设$A = H_1 + \I H_2$,则$A^\dagger = H_1 - \I H_2$,从而$H_1$和$H_2$能由$A$\textbf{唯一}决定。

\textbf{证毕}。




正规算子的定义依赖于酉空间的内积,因为伴随变换的定义依赖于内积或者一个非退化双线性形式。因此,讨论正规算子的同时可以讨论向量的正交性,而正规算子实际上就是“可以被正交对角化的线性变换”。我们接下来就一步步推导出这一点。





\subsection{正规算子的对角化}



\begin{lemma}{算子交换则有公共特征向量}

给定酉空间$U$上的线性变换$A$和$B$,若$AB=BA$,则存在$\bvec{u}\in U$,它是$A$和$B$的公共特征向量。

\end{lemma}


\textbf{证明}:

酉空间上任何非零算子都至少有一个特征向量,这是

\textbf{证毕}。
























