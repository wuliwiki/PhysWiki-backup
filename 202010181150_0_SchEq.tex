% 定态薛定谔方程
% keys 量子力学|波函数|薛定谔方程|哈密顿|束缚态

\pentry{量子力学基本假设\upref{QMPos}, 矢量算符\upref{VecOp}}

\subsection{定态薛定谔方程}
用于求能量的本征态
\begin{equation}
H \ket{\Psi} = E \ket{\Psi}
\end{equation}

% 引用一些词条作为例子

单个粒子问题中, 哈密顿算符对应粒子的总能量, 总能量算符可以表示为动能算符和势能算符之和
\begin{equation}
H = T + V
\end{equation}

一维运动的单个质点, 波函数是坐标 $x$ 的函数 $\Psi(x)$
\begin{equation}
T = -\frac{\hbar^2}{2m} \pdv[2]{x} \qquad V = V(x)
\end{equation}
所以定态薛定谔方程为
\begin{equation}\label{SchEq_eq1}
-\frac{\hbar^2}{2m} \pdv[2]{\Psi}{x} + V(x)\Psi = E \Psi
\end{equation}

二维或三维的情况下, 波函数是位置矢量\upref{Disp}的函数 $\Psi(\bvec r)$
\begin{equation}
T = -\frac{\hbar^2}{2m} \laplacian \qquad V = V(\bvec r)
\end{equation}
定态薛定谔方程为
\begin{equation}
-\frac{\hbar^2}{2m} \laplacian {\Psi} + V(\bvec r)\Psi = E \Psi
\end{equation}
