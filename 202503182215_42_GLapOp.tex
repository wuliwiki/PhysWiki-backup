% 高维弯曲空间中的拉普拉斯算符
% keys 高维空间|弯曲空间|拉普拉斯算符
% license Usr
% type Map

\pentry{拉普拉斯算符\nref{nod_Laplac}}{nod_057a}
\enref{三维空间中的拉普拉斯算符}{Laplac} $\Delta$ 由下式所定义:
\begin{equation}\label{eq_GLapOp_1}
\Delta u:=\Nabla\cdot (\Nabla u)=\frac{\partial^{2}{u}}{\partial{x}^{2}} + \frac{\partial^{2}{u}}{\partial{y}^{2}} + \frac{\partial^{2}{u}}{\partial{z}^{2}}.~
\end{equation}
其中 $u$ 是空间坐标的函数。拉普拉斯算符在求解许多物理问题时都会出现,包括真空中电势满足的\enref{泊松方程}{EPoiEQ},\enref{定态薛定谔方程}{SchEq}和线性化的爱因斯坦场方程(作为d'Alember算子的一部分)。

将拉普拉斯算符推广到一般的高维弯曲空间中有着现实的物理意义。一方面,无论我们所处的时空是广义相对论说的4维,还是超弦论所谓的10维,或者某个物理理论说的多少维,它们都在告诉我们高维时空具有重要的物理意义。而各种场的物理理论似乎告诉我们,在很多情形下(比如真空情形),物理理论的基本场方程必然会出现拉普拉斯算符。另一方面,平坦的时空仅仅是弯曲时空的一个(曲率为0的)特例。因此,一般的高维弯曲空间中拉普拉斯算符的形式有着其独特的价值。

记号约定:为方便起见,假设空间维度为 $N$,坐标变量为 $x^i,i=1,\cdots,N$,而描述线元的(代表二次型的度量矩阵的)度规记作 $g_{ij}$。

\subsection{一般时空中拉普拉斯算符的导出}
观察\autoref{eq_GLapOp_1} ,容易想到,拉普拉斯算符 $\Delta$ 应当保持算符 $\Nabla$ 的点积形式。即应当继续由
\begin{equation}\label{eq_GLapOp_2}
\Delta u:=\Nabla\cdot (\Nabla u)~
\end{equation}
定义。

在一般空间中,矢量的\textbf{点积}是由度规定义的,即
\begin{equation}
v\cdot w:=(v|w)=(e_i,e_j)v^iv^j=g_{ij}v^iv^j.~
\end{equation}
由于度规矩阵的可逆性,因此常常用其和其逆矩阵 $g^{ij}$ 来升降指标,比如 $v_i=g_{ij}v^j$。

因此\autoref{eq_GLapOp_2} 现在成为
\begin{equation}
\Delta u=g^{ij}\Nabla_i(\Nabla_j u).~
\end{equation}

然而,虽然在一般 $N$ 维空间中,算符 $\Nabla$ 作用在函数上仍有
\begin{equation}
\Nabla u=\pdv{u}{x^i}.~
\end{equation}
但是算符 $\Nabla$ 作用在带有分量指标的矢量上时,时间的弯曲将带来额外的影响。此时算符 $\Nabla$ 称为\enref{协变导数}{CoDer},其作用在矢量 $v_j$ 上效果为
\begin{equation}
\Nabla_i v_j= \pdv{v_j}{x^i}+{\Gamma^{k}}_{ji}v_k.~
\end{equation}
右边额外的部分来自时空弯曲的影响,(在黎曼几何下)其可以由每一点的度规进行通过 $\Nabla_i g_{ik}=0$ 进行定义。因此

\begin{equation}
\begin{aligned}
g^{ij}\Nabla_i(\Nabla_j u)&=g^{ij}\Nabla_i\qty(\pdv{u}{x^j})\\
&=g^{ij}\pdv{}{x^i}\\
&=\Nabla_i.
\end{aligned}~
\end{equation}


定义。其中 $u$ 是 $N$ 个坐标变量的函数。





