% 概率密度函数
% latex
\pentry{此文件保存为学习人工智能的监督学习和非监督学习算法的笔记页,概率密度函数是学习人工智能的基础函数.\upref{Sample}}
\begin{definition}{}
由于中心极限定理,假设所有假设样本事件都为独立事件,并且
\end{definition}
\begin{equation}
y^{\left(i\right)}=\theta^T x^{\left(i\right)} + \varepsilon^{\left(i\right)} 
\end{equation}

\begin{equation}
p\left(\epsilon^{\left(i\right)}\right)=\frac{1}{\sqrt{2\pi}\sigma}exp\left(-\frac{\left(\epsilon^{\left(i\right)}\right)^2}{2\sigma^2}\right)
\end{equation}

\begin{equation}
p\left(y^{\left(i\right)}|x^{\left(i\right)};\theta\right)=\frac{1}{\sqrt{2\pi}\sigma}exp\left(-\frac{\left(y^{\left(i\right)}-\theta^Tx^{\left(i\right)}\right)^2}{2\sigma^2}\right)
\end{equation}

\begin{equation}
\ali{
L\left(\theta\right)&=\prod_{i=1}^m p\left(y^{\left(i\right)} | x^{\left(i\right)};\theta\right)\\
&=\prod_{i=1}^m \frac{1}{\sqrt{2\pi}\sigma}exp\left(-\frac{\left(\epsilon^{\left(i\right)}\right)^2}{2\sigma^2}\right)
}
\end{equation}
