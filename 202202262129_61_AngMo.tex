% 角动量
% keys 角动量

\begin{issues}
\issueTODO
\end{issues}

在讨论物体的运动时,我们用动量来描述机械运动的状态,并讨论了在机械运动的转移过程中所遵循的动量守恒定律.同样,在讨论物体围绕某一点的运动时,我们也可以用角动量来描述物体的运动状态.
\subsection{角动量}
我们在研究物体的运动中经常会遇到物体围绕一定中心的转动的情况.例如,地球围绕太阳的公转、卫星绕地球的运转、原子中的电子围绕着原子核运转等等.为了方便起见,我们以质量为$m$作圆周运动的质点为例,来引入角动量的概念.

设圆的半径是$r$,则质点对圆心的位失$\bvec r$的量值便是$r$,质点的速度是$v$,方向沿着圆的切线方向.从图可以看出,质点的动量$\bvec p = m\bvec v$处处和它的位失$\bvec r$相垂直.我们把质点动量$\bvec p$的量值$p$和位失$\bvec r$的量值$r$的乘积定义为作圆周运动的质点对圆心$O$的角动量的量值,用$L$表示.
\begin{equation}
L=pr=mvr
\end{equation}
\begin{figure}[ht]
\centering
\includegraphics[width=5cm]{./figures/AngMo_1.pdf}
\caption{质点对圆心的角动量} \label{AngMo_fig1}
\end{figure}
\subsection{角动量守恒}


\addTODO{角动量性质、转动惯量、添加例子}

