% 阿尔-花拉子米(综述)
% license CCBYSA3
% type Wiki

本文根据 CC-BY-SA 协议转载翻译自维基百科\href{https://en.wikipedia.org/wiki/Al-Khwarizmi}{相关文章}。
\begin{figure}[ht]
\centering
\includegraphics[width=6cm]{./figures/6cab70e0324244c5.png}
\caption{20世纪木刻版画,描绘阿尔-花拉子米} \label{fig_HLZM_1}
\end{figure}
穆罕默德·伊本·穆萨·阿尔-花拉子米(约780年–约850年),或简称阿尔-花拉子米,是一位波斯学者,致力于数学、天文学和地理学的研究,并在阿拉伯语学术领域产生了深远的影响。约在公元820年,他在巴格达的智慧之家工作,该地是阿拔斯王朝的首都。

他在代数领域的代表性著作《阿尔-贾布尔》(《计算的简明书》)编写于813年至833年之间,提出了线性方程和二次方程的系统解法。他在代数方面的成就之一是通过完成平方法来解二次方程,并为此提供了几何证明。由于阿尔-花拉子米是第一个将代数作为独立学科来研究的人,并且引入了“简化”和“平衡”方法(将减去的项转移到方程的另一边,即在方程两边相同项的消去),因此他被誉为代数的奠基人或创始人。英语中的“algebra”一词就来源于他那本著作的简写标题(阿尔-贾布尔,意为“完成”或“重新联合”)。他的名字还衍生出了英语中的“algorism”和“algorithm”以及西班牙语、意大利语和葡萄牙语中的“algoritmo”;以及西班牙语中的“guarismo”和葡萄牙语中的“algarismo”,均表示“数字”。

12世纪时,阿尔-花拉子米关于印度算术的教科书《印度人数字算法》(Algorithmo de Numero Indorum)的拉丁文翻译,规范了印度数字系统,并将基于十进制的位值计数系统引入西方世界。同样,阿尔-贾布尔由英学者罗伯特·查斯特在1145年翻译成拉丁文,并一直作为欧洲大学的主要数学教材,直到16世纪。

阿尔-花拉子米修订了公元2世纪由罗马学者克劳狄乌斯·托勒密撰写的《地理学》,列出了城市和地方的经纬度。他还制作了一套天文表格,并撰写了关于历法的作品,以及关于天文仪器和日晷的研究。阿尔-花拉子米在三角学方面也作出了重要贡献,编制了准确的正弦和余弦表,并制作了第一个正切表。
\subsection{生活}
\begin{figure}[ht]
\centering
\includegraphics[width=6cm]{./figures/27f5f3869a4239e7.png}
\caption{马德里的大学城(Ciudad Universitaria)中穆罕默德·伊本·穆萨·阿尔·胡瓦里兹米的纪念碑} \label{fig_HLZM_2}
\end{figure}
很少有关于穆罕默德·伊本·穆萨·阿尔·胡瓦里兹米(al-Khwarizmi)生活的确切细节。伊本·纳迪姆(Ibn al-Nadim)将他的出生地定为胡瓦尔兹姆(Khwarazm),他通常被认为来自这个地区。作为波斯人,他的名字意味着“来自胡瓦尔兹姆”,这个地区曾是大伊朗的一部分,现在属于土库曼斯坦和乌兹别克斯坦。

塔巴里(al-Tabari)给出的名字是穆罕默德·伊本·穆萨·阿尔·胡瓦里兹米·阿尔·马久西·阿尔·库特鲁布布利(Muḥammad ibn Musá al-Khwārizmī al-Majūsī al-Qūṭrubbullī)。其中的“阿尔·库特鲁布布利”(al-Qutrubbulli)可能表示他来自巴格达附近的库特鲁布尔(Qutrubbul)。然而,罗什迪·拉谢德(Roshdi Rashed)对此表示否认,认为这只是早期手稿的错误。在另一种看法中,大卫·A·金(David A. King)认为他是来自库特鲁布尔地区,因为他被称为“阿尔·胡瓦里兹米·阿尔·库特鲁布布利”,可能是因为他出生在巴格达附近。

关于他的宗教信仰,托默(Toomer)写道,塔巴里给他的另一个称号“阿尔·马久西”(al-Majūsī)似乎表明他是古老的琐罗亚斯德教信徒。这种信仰在那个时代对于伊朗裔人士来说依然有可能,但胡瓦里兹米的《代数》序言表明他是正统的穆斯林,因此“阿尔·马久西”这一称号可能仅仅意味着他的祖先,甚至可能是他年轻时曾是琐罗亚斯德教徒。

伊本·纳迪姆的《历史大典》(Al-Fihrist)中有简短的胡瓦里兹米传记及其著作清单。胡瓦里兹米的工作主要集中在813至833年之间。穆斯林征服波斯后,巴格达成为了科学研究和贸易的中心。大约820年,他被任命为天文学家,并成为智慧之宫(House of Wisdom)图书馆的馆长。智慧之宫是由阿拔斯哈里发阿尔·马蒙(al-Ma'mūn)创立的。胡瓦里兹米研究了包括翻译希腊文和梵文科学手稿在内的各种科学和数学内容。他还是一位历史学家,曾被塔巴里等人引用。

在阿尔·瓦西克(al-Wathiq)统治时期,他据说参与了两次使节任务,其中之一是前往可萨(Khazars)。道格拉斯·莫顿·邓洛普(Douglas Morton Dunlop)建议,穆罕默德·伊本·穆萨·阿尔·胡瓦里兹米可能与穆罕默德·伊本·穆萨·伊本·沙基尔(Muḥammad ibn Mūsā ibn Shākir),即三兄弟班努·穆萨(Banū Mūsā)中的长子是同一个人。
\subsection{贡献}
\begin{figure}[ht]
\centering
\includegraphics[width=6cm]{./figures/dfca5e36d9defc42.png}
\caption{《阿尔·胡瓦里兹米的代数》中的一页} \label{fig_HLZM_3}
\end{figure}
阿尔·胡瓦里兹米在数学、地理、天文学和制图学方面的贡献为代数和三角学的创新奠定了基础。他系统化的线性和二次方程求解方法促成了代数的诞生,而“代数”这一词汇源自他关于这一主题的著作《Al-Jabr》(《完备与平衡计算书》)。

大约在820年写成的《使用印度数字的计算方法》在中东和欧洲传播了印度-阿拉伯数字系统。当该书在12世纪被翻译成拉丁文,名为《Algoritmi de numero Indorum》(阿尔·胡瓦里兹米关于印度算术的著作)时,“算法”这一术语被引入西方世界。

他的一些工作基于波斯和巴比伦天文学、印度数字和希腊数学。

阿尔·胡瓦里兹米对托勒密关于非洲和中东的资料进行了系统化和修正。另一部重要著作是《Kitab surat al-ard》(《地球的图像》;翻译为《地理学》),其中给出了基于托勒密《地理学》中的坐标值,改进了地中海、亚洲和非洲的值。

他还撰写了有关天文仪器如天体仪和日晷的书籍。他参与了一项测定地球周长的项目,并为阿尔·马蒙哈里发绘制了一张世界地图,监督了70位地理学家的工作。当他的著作通过拉丁文翻译在12世纪传播到欧洲时,对欧洲数学的发展产生了深远的影响。
\subsubsection{代数}
\begin{figure}[ht]
\centering
\includegraphics[width=6cm]{./figures/ff3c66dbb23f817d.png}
\caption{《阿尔·卡瓦里兹米的代数书》原始阿拉伯文印刷手稿} \label{fig_HLZM_6}
\end{figure}
\begin{figure}[ht]
\centering
\includegraphics[width=6cm]{./figures/773a045cdfee199e.png}
\caption{《阿尔·卡瓦里兹米的代数》由弗雷德里克·罗森翻译的英文版中的一页} \label{fig_HLZM_7}
\end{figure}
《代数学》(Al-Jabr,阿拉伯文:الكتاب المختصر في حساب الجبر والمقابلة,al-Kitāb al-mukhtaṣar fī ḥisāb al-jabr wal-muqābala)是一本大约在公元820年左右写成的数学书籍。该书在哈里发阿尔·马蒙的鼓励下编写,作为一本关于计算的流行著作,书中充满了许多示例和应用,涵盖了贸易、测量和法律继承等问题。[49] “代数”一词来源于该书中描述的基本方程操作之一(*al-jabr*,意为“恢复”,指的是在方程两边加上一个数以合并或取消项)。该书由罗伯特·查尔斯特(Robert of Chester)于1145年将其翻译为拉丁文《代数与相互平衡之书》(*Liber algebrae et almucabala*),因此形成了“代数”这一术语。杰拉德·克雷莫纳(Gerard of Cremona)也进行过翻译。一本独特的阿拉伯文手稿保存在牛津大学,并由F. 罗斯恩(F. Rosen)于1831年翻译。一份拉丁文翻译本保存在剑桥大学。[50]

该书提供了关于解决二次方程的详尽说明,并讨论了“还原”和“平衡”的基本方法,指的是将方程的项移到方程的另一边,即在方程两边取消相同的项。[51][52]

阿尔-花拉兹米(Al-Khwārizmī)解线性和二次方程的方法首先将方程化简为六种标准形式之一(其中 \(b\) 和 \(c\) 是正整数):
\begin{itemize}
\item 平方等于根(\(ax^2 = bx\))
\item 平方等于数(\(ax^2 = c\))
\item 根等于数(\(bx = c\))
\item 平方与根等于数(\(ax^2 + bx = c\))
\item 平方与数等于根(\(ax^2 + c = bx\))
\item 根与数等于平方(\(bx + c = ax^2\))
\end{itemize}
通过除去平方项的系数并使用两个操作,\textbf{al-jabr}(阿拉伯语:الجبر,意为“恢复”或“完成”)和 al-muqābala(意为“平衡”)。\textbf{al-jabr}是通过向方程两边加上相同的数值,从而消去负数项、根和平方项的过程。例如,\(x^2 = 40x - 4x^2\) 被化简为 \(5x^2 = 40x\)。al-muqābala 是将同类型的量移到方程同一边的过程。例如,\(x^2 + 14 = x + 5\) 被化简为 \(x^2 + 9 = x\)。

以上讨论使用了现代数学符号来表示书中讨论的问题类型。然而,在阿尔-花拉兹米的时代,大部分符号尚未被发明,因此他必须用普通文本来呈现问题和其解法。例如,对于一个问题,他写道(摘自1831年的翻译):

如果有人说:“你把十分成两部分:将其中一部分乘以自身;它将等于另一部分的八十一倍。” 计算:你说,十减去某物,乘以自身,等于一百加上一个平方减去二十个某物,这等于八十一个某物。从一百和一个平方中分离出二十个某物,并加到八十一上。然后它就会变成一百加上一个平方,等于一百零一个根。将根的二分之一求出,结果是五十又半。将其平方,得到二千五百五十又四分之一。从中减去一百,剩余二千四百五十又四分之一。提取平方根,结果是四十九又半。从根的二分之一,即五十又半中减去它,剩下的是一,这是两个部分之一。

在现代符号中,这个过程,其中 \(x\) 是“东西”(阿拉伯语:شيء,shayʾ)或“根”,可以通过以下步骤表示:
\[
(10 - x)^2 = 81x~
\]
\[
100 + x^2 - 20x = 81x~
\]
\[
x^2 + 100 = 101x~
\]
设方程的根为 \(x = p\) 和 \(x = q\)。那么:\(\frac{p + q}{2} = 50 \frac{1}{2}\)\(pq = 100\)
并且:
\[
\frac{p - q}{2} = \sqrt{\left(\frac{p + q}{2}\right)^2 - pq} = \sqrt{2550 \frac{1}{4} - 100} = 49 \frac{1}{2}~
\]
因此,根是:
\[
x = 50 \frac{1}{2} - 49 \frac{1}{2} = 1~
\]
有几位作者曾以《Kitāb al-jabr wal-muqābala》为题出版过著作,包括阿布·哈尼法·迪纳瓦里(Abū Ḥanīfa Dīnawarī)、阿布·卡米尔(Abū Kāmil)、阿布·穆罕默德·阿德利(Abū Muḥammad al-'Adlī)、阿布·尤素福·米斯西(Abū Yūsuf al-Miṣṣīṣī)、阿卜杜·哈米德·伊本·图尔克('Abd al-Hamīd ibn Turk)、辛德·伊本·阿里(Sind ibn 'Alī)、萨赫尔·伊本·比什尔(Sahl ibn Bišr)和沙拉夫·阿尔·丁·图西(Sharaf al-Dīn al-Ṭūsī)等。

所罗门·甘兹(Solomon Gandz)曾将阿尔-花拉兹米誉为代数学的奠基人:

阿尔-花拉兹米的代数学被认为是科学的基础和基石。从某种意义上讲,阿尔-花拉兹米比狄奥凡图斯更有资格被称为“代数学之父”,因为阿尔-花拉兹米是第一个以基础形式并专门讲授代数学的人,而狄奥凡图斯主要关注的是数论。[53]

维克托·J·凯茨(Victor J. Katz)补充道:

至今仍然存在的第一部真正的代数教材是穆罕默德·伊本·穆萨·阿尔-花拉兹米于公元825年左右在巴格达编写的《代数与平衡》一书。[54]

约翰·J·奥康纳和埃德蒙·F·罗伯逊在《MacTutor数学史档案》中写道:

阿拉伯数学的一个最重要的进步之一是阿尔-花拉兹米的工作,也就是代数学的起源。理解这一新思想的重要性是至关重要的。这是一次革命性的突破,摆脱了希腊数学的概念,后者本质上是几何学。代数学是一种统一的理论,它允许有理数、无理数、几何量等都可以作为“代数对象”来处理。它为数学提供了一条全新的发展路径,概念上远比之前存在的数学更加广阔,并为未来的学科发展提供了一个载体。代数思想的引入的另一个重要方面是,它使得数学能够以一种以前从未发生过的方式应用于自身。[55]

罗什迪·拉谢德和安吉拉·阿姆斯特朗写道:

阿尔-花拉兹米的文本不仅与巴比伦的泥板不同,而且与狄奥凡图斯的《算术》也有所不同。它不再是一个个待解的问题系列,而是一个从原始术语开始的阐述,这些组合必须给出所有可能的方程原型,方程也明确构成了研究的真正对象。另一方面,方程作为一个独立对象的思想从一开始就出现,可以说是以一种通用的方式出现,因为它不仅仅是在解决问题的过程中产生的,而是专门被用来定义一个无限的类问题。[56]

根据瑞士裔美国数学史学家弗洛里安·卡乔里的说法,阿尔-花拉兹米的代数学与印度数学家的工作不同,因为印度人没有像“恢复”和“减少”这样的规则。[57] 关于阿尔-花拉兹米的代数工作与印度数学家布拉马古普塔的工作的不同及其重要性,卡尔·B·博耶写道:

确实,在两个方面,阿尔-花拉兹米的工作比狄奥凡图斯的工作有所退步。首先,它远远比狄奥凡图斯问题中的代数要基础得多;其次,阿尔-花拉兹米的代数是完全修辞性的,没有希腊《算术》或布拉马古普塔工作中的省略符号!甚至数字也写成了文字,而不是符号!阿尔-花拉兹米很可能不知道狄奥凡图斯的工作,但他必须至少熟悉布拉马古普塔的天文和计算部分;然而,阿尔-花拉兹米和其他阿拉伯学者并未使用省略符号或负数。尽管如此,《代数》更接近于今天的初等代数,而不是狄奥凡图斯或布拉马古普塔的作品,因为这本书并不涉及不确定分析中的难题,而是对方程(尤其是二次方程)求解的直接且基础的阐述。阿拉伯人通常喜欢从前提到结论的清晰论证以及系统的组织结构——在这些方面,狄奥凡图斯和印度人都不如阿尔-花拉兹米。[58]
\subsubsection{算术}
\begin{figure}[ht]
\centering
\includegraphics[width=6cm]{./figures/c86ea77c4ab8bde1.png}
\caption{算法学派与算盘学派的争论,描绘于公元1508年的一幅素描中} \label{fig_HLZM_4}
\end{figure}
阿尔-花拉兹米的第二部最具影响力的作品是关于算术的,虽然原始阿拉伯文已失传,但该作品通过拉丁文翻译流传下来。他的著作包括《印度计算书》 (kitāb al-ḥisāb al-hindī) 和可能更加基础的《印度算术的加法与减法书》 (kitāb al-jam' wa'l-tafriq al-ḥisāb al-hindī) [60][61]。这些文本描述了可以在尘板上执行的十进制数字(印度-阿拉伯数字)算法。阿拉伯语中称此为“takht”(拉丁语:tabula),即用薄薄的尘土或沙子覆盖的板子,供进行计算,数字可以用尖笔写在上面,并在必要时轻松擦除或更换。阿尔-花拉兹米的算法被使用了近三百年,直到被阿尔-乌克利迪西的算法所取代,这些算法可以用笔和纸进行计算。[62]

作为12世纪阿拉伯科学通过翻译传入欧洲的浪潮的一部分,这些文本在欧洲引发了革命性的影响。[63] 阿尔-花拉兹米的拉丁化名字“Algorismus”演变成了用于计算的方法名称,并且至今存在于“算法”(algorithm)这一术语中。它逐渐取代了欧洲之前使用的算盘计算方法。[64]
\begin{figure}[ht]
\centering
\includegraphics[width=6cm]{./figures/6cd64d08f25039ba.png}
\caption{拉丁文翻译中的一页,开头为“Dixit algorizmi”} \label{fig_HLZM_5}
\end{figure}
虽然没有哪一部被认为是阿尔-花拉兹米原著的字面翻译,但仍然保留下了四部采用阿尔-花拉兹米方法的拉丁文著作:[60]

\begin{itemize}
\item 《Dixit Algorizmi》(1857年出版,标题为《Algoritmi de Numero Indorum》[65])[66]
\item 《Liber Alchoarismi de Practica Arismetice》
\item 《Liber Ysagogarum Alchorismi》
\item 《Liber Pulveris》
\end{itemize}

《Dixit Algorizmi》("阿尔-花拉兹米如是说")是剑桥大学图书馆一份手稿的开头短语,通常被称为1857年的《Algoritmi de Numero Indorum》。它归功于1126年翻译过天文表的巴斯的阿德拉尔德。这部作品也许最接近阿尔-花拉兹米的原著。[66]

阿尔-花拉兹米关于算术的工作为将基于印度数学中发展出的印度-阿拉伯数字系统的阿拉伯数字引入西方世界做出了贡献。术语“算法”源自于“algorism”,即阿尔-花拉兹米开发的使用印度-阿拉伯数字进行算术运算的技术。“算法”和“algorism”这两个词都源自阿尔-花拉兹米名字的拉丁化形式,分别为“Algoritmi”和“Algorismi”。[67]
\subsubsection{天文学}
\begin{figure}[ht]
\centering
\includegraphics[width=6cm]{./figures/5f588450d50b8326.png}
\caption{《科珀斯克里斯蒂学院手稿283页》,为阿尔·花拉子米的《天文表》(Zīj)的拉丁文翻译。} \label{fig_HLZM_8}
\end{figure}
《阿尔·花拉子米的《天文表》》(Zīj as-Sindhind,阿拉伯文:زيج السند هند,意为“《悉檀达天文表》”)是一部包含约37章的天文和历法计算的著作,内有116个表格,涉及历法、天文和占星数据,还包括正弦值表。这是第一部基于印度天文方法(称为悉檀达)的阿拉伯天文表(Zij)。"Sindhind"一词是梵文"Siddhānta"的变形,Siddhānta通常是指天文教科书。实际上,阿尔·花拉子米表格中的天体运动数据来源于“修正版布拉马悉檀达”(Brahmasphutasiddhanta),即布拉马吉普塔的天文学著作。

这部作品包含了当时已知的太阳、月亮和五大行星的运动表格。它标志着伊斯兰天文学的一个转折点。在此之前,穆斯林天文学家主要采用研究的方法,翻译他人的作品,并学习已发现的知识。

这部作品的原始阿拉伯文版本(约公元820年)已经遗失,但西班牙天文学家马斯拉马·马吉里提(约1000年)所写的版本,经过拉丁文翻译后得以流传,推测由巴斯的阿德拉尔(Adelard of Bath)于1126年1月26日完成翻译。现存的四部拉丁文手稿保存在以下图书馆:沙特尔公共图书馆、马萨林图书馆(巴黎)、西班牙国家图书馆(马德里)和博德利图书馆(牛津)。
\subsubsection{三角学}  
阿尔·花拉子米的《天文表》包含了正弦和余弦的三角函数表。[69] 一部关于球面三角学的相关论文也归于他。[55]

阿尔·花拉子米制作了精确的正弦和余弦表,并且是首个编制正切表的人。[72][73]
\subsubsection{《地理学》}
\begin{figure}[ht]
\centering
\includegraphics[width=14.25cm]{./figures/a2df7dfdfb5508b5.png}
\caption{Gianluca Gorni重建的《阿尔-花拉子密世界地图》中有关印度洋的部分。阿尔-花拉子密使用的大多数地名与托勒密、马特尔卢斯和贝海姆的地名相符。沿海的总体形状在塔普罗班那和卡蒂加拉之间相同。龙尾,即印度洋的东部入口,在托勒密的描述中并不存在,但在阿尔-花拉子密的地图上有很少的细节描绘,尽管在马特尔卢斯的地图和后来的贝海姆版本中则非常清晰和精确。} \label{fig_HLZM_9}
\end{figure}
阿尔·花拉子米的第三部重要著作是《地球图像书》(阿拉伯语:كتاب صورة الأرض,"Book of the Description of the Earth"),也称为他的《地理学》,完成于833年。这是对托勒密二世纪《地理学》的重大修订,内容包括在一个总引言之后列出了2402个城市和其他地理特征的坐标。[75]

《地球图像书》仅存一份副本,现保存在斯特拉斯堡大学图书馆。[76][77] 一份拉丁文翻译保存在西班牙马德里的国家图书馆。[78] 这本书以天气区的纬度和经度顺序开篇,即按纬度的区块进行排列,在每个天气区内则按经度顺序排列。正如保罗·加莱兹(Paul Gallez)所指出的,这种系统可以推导出许多纬度和经度,即使原文档的状况非常糟糕,几乎无法辨认。无论是阿拉伯文原本还是拉丁文翻译,都没有包含世界地图;然而,休伯特·道尼赫特(Hubert Daunicht)通过坐标列表成功重建了缺失的地图。他读取了手稿中沿海点的纬度和经度,或者从无法辨认的上下文中推导出它们。然后,他将这些点转移到绘图纸上,并用直线将它们连接起来,从而获得了接近原始地图的海岸线图。他还对河流和城镇进行了相同的处理。[79]
\begin{figure}[ht]
\centering
\includegraphics[width=6cm]{./figures/9ae26dc993dcb7c3.png}
\caption{一份15世纪的托勒密《地理学》版本,供对比使用。} \label{fig_HLZM_10}
\end{figure}
阿尔·花拉子米修正了托勒密对地中海长度的过高估计[80],从加那利群岛到地中海东岸,托勒密估计为63度经度,而阿尔·花拉子米则几乎准确地估计为接近50度经度。他“将大西洋和印度洋描绘为开放水域,而非像托勒密那样画成封闭的海洋。”[81] 因此,阿尔·花拉子米的本初子午线位于幸运岛附近,约比马里努斯和托勒密使用的本初子午线东偏10°。大多数中世纪的穆斯林地理学家继续使用阿尔·花拉子米的本初子午线。[80]
\subsubsection{《犹太历法》}  
\begin{figure}[ht]
\centering
\includegraphics[width=6cm]{./figures/c3e5b74c16a1ea0a.png}
\caption{《阿尔·花拉兹米的《地球描述书》中的尼罗河最早现存地图》} \label{fig_HLZM_11}
\end{figure}
阿尔·花拉兹米还写了几部其他作品,其中包括一部关于希伯来历法的论著,题为《Risāla fi istikhrāj ta'rīkh al-yahūd》(阿拉伯语:رسالة في إستخراج تأريخ اليهود,“提取犹太纪元”)。该书描述了梅托尼周期,即19年的插入周期;确定犹太历法新年第一天(提什雷月的第一天)所在星期几的规则;计算犹太纪年(Anno Mundi)与塞琉古纪年之间的间隔;并给出使用希伯来历法确定太阳和月亮的平均经度的规则。类似的内容也出现在阿尔·比鲁尼和迈蒙尼德的著作中。[37]
\subsubsection{其他著作}  
伊本·纳迪姆的《费赫里斯特》是一本阿拉伯书籍索引,提到阿尔·花拉兹米的《历史书》(阿拉伯文:كتاب التأريخ),一本年鉴类著作。该书没有现存的直接手稿;然而,到11世纪时,它的一本副本已经到达努赛宾,并被该地的主教马尔·以利亚斯·巴尔·什纳亚发现。以利亚斯的编年史引用了该书,从“先知去世”一直到公元169年,此时以利亚斯的编年史本身中断。[82]

柏林、伊斯坦布尔、塔什干、开罗和巴黎的几部阿拉伯手稿包含了更多的材料,这些材料可以肯定或有很大可能来自阿尔·花拉兹米。伊斯坦布尔的手稿包含了一篇关于日晷的论文;《费赫里斯特》将《阿尔·花拉兹米的书籍》列为《石碑书》(阿拉伯文:كتاب الرخامة)。其他论文包括关于确定麦加方向的论文,以及关于球面天文学的研究。

两部关于早晨宽度(《认识每个城市的东方时间》)和从高度测定方位角(《认识从高度测定方位角》)的著作值得特别关注。他还写了两本关于使用和制作星盘的书。
\subsection{荣誉}
\begin{itemize}
\item 阿尔·花拉兹米(陨石坑)——位于月球远端的一个陨石坑。[83]  
\item 13498 阿尔·花拉兹米——主带小行星,1986年8月6日由E·W·埃尔斯特和V·G·伊万诺娃在斯莫良发现。[84]  
\item 11156 阿尔·花拉兹米——主带小行星,1997年12月31日由P·G·科姆巴在普雷斯科特发现。[85]
\end{itemize}
\subsection{注释}  
\begin{figure}[ht]
\centering
\includegraphics[width=6cm]{./figures/d57b45f785f16108.png}
\caption{} \label{fig_HLZM_12}
\end{figure}
\begin{enumerate}
\item 文献中对 al-Khwārizmī 的全名存在一些混淆,有的写作 **Abū ʿAbdallāh Muḥammad ibn Mūsā al-Khwārizmī**(阿布·阿卜杜拉·穆罕默德·本·穆萨·赫瓦尔兹米),也有的写作 **Abū Ja'far Muḥammad ibn Mūsā al-Khwārizmī**(阿布·贾法尔·穆罕默德·本·穆萨·赫瓦尔兹米)。伊本·赫尔敦在其《先知的历史》导言中提到:“第一个写这门学科(代数学)的,是阿布·阿卜杜拉·赫瓦尔兹米。此后是阿布·卡米尔·舒贾'·本·阿斯拉姆,之后的人都跟随他的步伐。”  
在他对罗伯特·切斯特拉丁语翻译本的批评性注释中,L. C. Karpinski指出 **Abū Ja'far Muḥammad ibn Mūsā** 是指 Banū Mūsā 兄弟中的长子。Karpinski 在他对(Ruska 1917)的评论中指出,在(Ruska 1918)中,“Ruska 无意中把作者写作 Abū Ga'far M. b. M.,而不是 Abū Abdallah M. b. M.”  
唐纳德·克努斯(Donald Knuth)将其写作 **Abū ʿAbdallāh Muḥammad ibn Mūsā al-Khwārizmī** 并引用其含义为“字面意思是‘阿卜杜拉的父亲,穆罕默德,穆萨的儿子,赫瓦尔兹米人’”,并引用了 Heinz Zemanek 之前的工作。  
\item 一些学者将 **al-ḥisāb al-hindī** 翻译为“用印度数字进行的计算”,但阿拉伯语中的 **Hindī** 意指‘印度的’,而非‘印度教的’。A. S. Saidan 认为应理解为“以印度方式进行的算术运算”,即使用印度-阿拉伯数字,而非单纯地理解为“印度算术”。阿拉伯数学家在其文本中融入了他们自己的创新。
\end{enumerate}
\subsection{参考文献}  
\begin{enumerate}
\item O'Connor, John J.; Robertson, Edmund F., "Abū Kāmil Shujā' ibn Aslam" 归档于2013年12月11日,MacTutor数学历史档案,圣安德鲁斯大学。  
\item Toomer, Gerald J. (1970–1980). "al-Khuwārizmī, Abu Ja'far Muḥammad ibn Mūsā"。收录于 Gillispie, Charles Coulston(主编)。《科学传记词典》。第七卷。Scribner。第358-365页。ISBN 978-0-684-16966-8。  
\item Vernet, Juan (1960–2005). "Al-Khwārizmī"。收录于 Gibb, H. A. R.; Kramers, J. H.; Lévi-Provençal, E.; Schacht, J.(主编)。《伊斯兰百科全书》第四卷(第二版)。莱顿:布里尔。第1070-1071页。OCLC 399624。  
\item Ibn Khaldūn, 《穆卡迪玛:历史导论》 归档于2016年9月17日,由Franz Rosenthal翻译,普林斯顿:普林斯顿大学出版社(1958年),第六章:19。  
\item Knuth, Donald (1997). "Basic Concepts". 《计算机编程艺术》第一卷(第三版)。Addison-Wesley。第1页。ISBN 978-0-201-89683-1。  
\item "Was al-Khwarizmi an applied algebraist? - UIndy"。uindy.edu。于2024年12月10日检索。  
\item Oaks, J. (2009), "Polynomials and Equations in Arabic Algebra", 《精确科学历史档案》,63(2), 169–203。  
\item Maher, P. (1998), "From Al-Jabr to Algebra", 《数学在学校中的应用》,27(4), 14–15。  
\item (Boyer 1991, "阿拉伯霸权" 第229页) "al-jabr和muqabalah这两个术语的确切含义尚不确定,但通常的解释与上述翻译相似。al-jabr一词可能意味着类似于'恢复'或'完成'的意思,似乎指的是将减法项转移到方程的另一侧;muqabalah一词被认为是指'简化'或'平衡'——即,消去方程两边的相同项。"  
\item Corbin, Henry (1998). 《航程与使者:伊朗与哲学》。北大西洋出版社。第44页。ISBN 978-1-55643-269-9。原文归档于2023年3月28日。于2020年10月19日检索。
\item Boyer, Carl B., 1985. 《数学史》,第252页,普林斯顿大学出版社。“狄奥芬托斯有时被称为代数学之父,但这个称号更适合于阿尔-赫瓦里兹米...”,“...《代数术》比狄奥芬托斯或布拉马吉普塔的作品更接近今天的基础代数...”  
\item Gandz, Solomon, 《阿尔-赫瓦里兹米代数学的来源》,《奥西里斯》,第一卷(1936),263-277页,“阿尔-赫瓦里兹米的代数学被认为是科学的基础和基石。从某种意义上说,阿尔-赫瓦里兹米比狄奥芬托斯更有资格被称为“代数学之父”,因为阿尔-赫瓦里兹米是第一个以基础形式和为了代数学本身而教授代数的人,而狄奥芬托斯主要关注的是数论。”  
\item Katz, Victor J. 《代数学历史阶段与教学的影响》(PDF)。维克托·J·卡茨,哥伦比亚特区大学,美国:190页。归档于2019年3月27日(PDF)。于2017年10月7日检索—来自哥伦比亚特区大学,美国。现存的第一本真正的代数学教材是由穆罕默德·伊本·穆萨·阿尔-赫瓦里兹米撰写的关于《代数术》和《平衡法》的作品,约写于825年的巴格达。  
\item Esposito, John L.(2000年4月6日)。《牛津伊斯兰史》。牛津大学出版社。第188页。ISBN 978-0-19-988041-6。归档于2023年3月28日。于2020年9月29日检索。阿尔-赫瓦里兹米常被认为是代数学的创始人,他的名字也成为了“算法”一词的来源。  
\item Brentjes, Sonja(2007年6月1日)。“代数学”。《伊斯兰百科全书》(第三版)。归档于2019年12月22日。于2019年6月5日检索。  
\item Knuth, Donald(1979)。《现代数学与计算机科学中的算法》(PDF)。Springer-Verlag。ISBN 978-0-387-11157-5。归档于2006年11月7日(PDF)。  
\item Gandz, Solomon(1926)。“《代数》一词的起源”。《美国数学月刊》,33(9):437–440页。doi:10.2307/2299605。ISSN 0002-9890。JSTOR 2299605。  
\item Struik 1987,第93页  
\item Hitti, Philip Khuri(2002)。《阿拉伯史》。帕尔格雷夫·麦克米兰。第379页。ISBN 978-1-137-03982-8。归档于2019年12月20日。  
\item Hill, Fred James; Awde, Nicholas(2003)。《伊斯兰世界史》。希波克林。第55页。ISBN 978-0-7818-1015-9。“《完美与平衡计算大全》”(Hisab al-Jabr wa H-Muqabala)对该学科的发展不可低估。十二世纪时该书被翻译成拉丁语,并且直到十六世纪它仍是欧洲大学的主要数学教材。
\item Overbay, Shawn; Schorer, Jimmy; Conger, Heather. “Al-Khwarizmi”。肯塔基大学。原文已于2013年12月12日归档。  
\item “伊斯兰西班牙与技术史”。原文已于2018年10月11日归档。于2018年1月24日检索。  
van der Waerden, Bartel Leendert(1985)。《代数学史:从阿尔-赫瓦里兹米到埃米·诺特》。柏林:Springer-Verlag。  
\item Arndt 1983,第669页  
\item Saliba, George(1998年9月)。“科学与医学”。《伊朗研究》。31(3–4):681–690页。doi:10.1080/00210869808701940。例如,像穆罕默德·伊本·穆萨·阿尔-赫瓦里兹米(大约850年活跃)这样的人物,可能对《伊朗百科全书》造成问题,因为尽管他显然是波斯裔,但他生活并工作于巴格达,且没有已知的作品是用波斯语创作的。  
\item Oaks, Jeffrey A.(2014)。“Khwārizmī”。见Kalin, Ibrahim(编)。《牛津伊斯兰哲学、科学与技术百科全书》。第1卷。牛津:牛津大学出版社。第451–459页。ISBN 978-0-19-981257-8。原文已于2022年1月30日归档。于2021年9月6日检索。  
\item “Ibn al-Nadīm和Ibn al-Qifṭī均记载阿尔-赫瓦里兹米的家族来自赫瓦里兹姆,这是位于阿尔尔海以南的地区。”  
\item 另见 → al-Nadīm, Abu'l-Faraj(1871–1872)。《Kitāb al-Fihrist》,由Gustav Flügel编辑,莱比锡:Vogel,第274页。al-Qifṭī, Jamāl al-Dīn(1903)。《Taʾrīkh al-Hukamā》,由August Müller和Julius Lippert编辑,莱比锡:Theodor Weicher,第286页。  
\item Dodge, Bayard(编)(1970)。《阿尔-纳迪姆的《费赫里斯》:十世纪伊斯兰文化调查》,第2卷,由Dodge翻译,纽约:哥伦比亚大学出版社。  
\item Clifford A. Pickover(2009)。《数学书:从毕达哥拉斯到第57维,数学史的250个里程碑》。斯特林出版社,第84页。ISBN 978-1-4027-5796-9。原文已于2023年3月28日归档。于2020年10月19日检索。  
\item 《世界文化中的科学史:知识的声音》。劳特利奇,第228页。“穆罕默德·伊本·穆萨·阿尔-赫瓦里兹米(780–850)是来自赫瓦里兹姆地区(今天乌兹别克斯坦中亚地区)的波斯天文学家和数学家。”  
Ben-Menahem, Ari(2009)。《自然与数学科学的历史百科全书》(第1版)。柏林:Springer。第942–943页。ISBN 978-3-540-68831-0。波斯数学家阿尔-赫瓦里兹米。
\item Wiesner-Hanks, Merry E.; Ebrey, Patricia Buckley; Beck, Roger B.; Davila, Jerry; Crowston, Clare Haru; McKay, John P.(2017)。《世界社会史》(第11版)。Bedford/St. Martin's,第419页。在这一时期的初期,波斯学者阿尔-赫瓦里兹米(大约850年去世)将希腊和印度的发现结合起来,制作了天文表,这些天文表成为后期东西方研究的基础。  
\item Encyclopaedia Iranica-online,“赫瓦里兹姆,伊斯兰时期”,由Clifford E. Bosworth编写,原文已于2021年9月2日归档。  
\item Bosworth, Clifford Edmund(1960–2005)。“赫瓦里兹姆”。见Gibb, H. A. R.; Kramers, J. H.; Lévi-Provençal, E.; Schacht, J.(编)。《伊斯兰百科全书》第四卷(第2版)。莱顿:Brill,第1060–1065页。OCLC 399624。  
\item “穆斯林征服后的伊拉克”,作者:Michael G. Morony,ISBN 1-59333-315-3(2005年重印自1984年原书),第145页,原文已于2014年6月27日归档。  
\item Rashed, Roshdi(1988)。“阿尔-赫瓦里兹米的代数观念”。见Zurayq, Qusṭanṭīn;Atiyeh, George Nicholas;Oweiss, Ibrahim M.(编)。《阿拉伯文明:挑战与回应:献给Constantine K. Zurayk的研究》。SUNY出版社,第108页。ISBN 978-0-88706-698-6。原文已于2023年3月28日归档。于2015年10月19日检索。  
\item King, David A.(2018年3月7日)。《伊斯兰天文学》。Al-Furqān伊斯兰遗产基金会—伊斯兰手稿研究中心。事件发生于20:51。原文已于2021年12月1日归档。于2021年11月26日检索。“我提到赫瓦里兹米的另一个名字,以表明他并非来自中亚。他来自巴格达外的Qutrubul。他出生在那里,否则他不会被称为al-Qutrubulli。许多人说他来自赫瓦里兹姆,真是荒谬。”
\item Toomer 1990  
\item Bosworth, C. E., 编(1987)。《塔巴里历史,第32卷:阿拔斯王朝的重新统一:阿尔-马蒙的哈里发统治,公元813–833年/公元198–213年》。SUNY近东研究系列。纽约州奥尔巴尼:纽约州立大学出版社,第158页。ISBN 978-0-88706-058-8。  
\item Golden, Peter;Ben-Shammai, Haggai;Roná-Tas, András(2007年8月13日)。《可萨人的世界:新视角》。选自1999年耶路撒冷国际可萨学讨论会的论文。BRILL,第376页。ISBN 978-90-474-2145-0。  
\item Dunlop 1943  
\item Yahya Tabesh;Shima Salehi。“伊朗的数学教育:从古代到现代”(PDF)。沙里夫大学技术学院。原文已于2018年4月16日归档。2018年4月16日检索。  
\item Presner, Todd(2024年9月24日)。《算法伦理学:数字人文学科与大屠杀记忆》。普林斯顿大学出版社,第20页。ISBN 978-0-691-25896-6。  
\item Daffa 1977  
\item Clegg, Brian(2019年10月1日)。《科学历史:世界伟大科学书籍如何描绘知识的历史》。Ivy出版社,第61页。ISBN 978-1-78240-879-6。原文已于2023年3月28日归档。2021年12月30日检索。  
\item Edu, World History(2022年9月28日)。《阿尔-赫瓦里兹米 - 传记,杰出成就与事实》。
\item Joseph Frank, 《阿尔-赫瓦里兹米关于星盘的著作》,1922年。  
\item "阿尔-赫瓦里兹米"。《大英百科全书》。原文已于2008年1月5日归档。2008年5月30日检索。  
\item "阿尔-赫瓦里兹米 | 传记与事实 | 大英百科全书"。www.britannica.com。2023年12月1日。  
\item Rosen, Frederic。《计算完成与平衡的综合书籍,阿尔-赫瓦里兹米》。1831年英文翻译。原文已于2011年7月16日归档。2009年9月14日检索。  
\item Karpinski, L.C.(1912)。"《大英百科全书》中数学历史的最新版"。科学,35(888):29-31。Bibcode:1912Sci....35...29K。doi:10.1126/science.35.888.29。PMID 17752897。原文已于2020年10月30日归档。2020年9月29日检索。  
\item Boyer 1991,第228页:“阿拉伯人通常喜欢清晰的从前提出发的论证和系统化的组织——这些方面是狄奥范托斯和印度人都不擅长的。”  
\item (Boyer 1991,《阿拉伯霸权》第229页)“关于‘al-jabr’和‘muqabalah’这两个术语的确切含义尚不确定,但通常的解释与上文翻译所暗示的相似。‘al-jabr’这个词大概意味着‘恢复’或‘完成’,似乎指的是将被减去的项移到方程的另一边;而‘muqabalah’则被认为指的是‘简化’或‘平衡’——即取消方程两边相同的项。”  
\item Gandz, Solomon,《阿尔-赫瓦里兹米代数的来源》,《奥西里斯》,第1期(1936),263-277页。  
\item Katz, Victor J.《代数历史的阶段与教学的启示》(PDF)。维克托·J·卡茨,哥伦比亚特区大学,美国:190页。原文已于2019年3月27日归档。2017年10月7日检索——通过哥伦比亚特区大学,华盛顿D.C., USA。
\item O'Connor, John J.; Robertson, Edmund F., "Abu Ja'far Muhammad ibn Musa Al-Khwarizmi", MacTutor History of Mathematics Archive, University of St Andrews  
\item Rashed, R.; Armstrong, Angela (1994). 《阿拉伯数学的发展》. Springer. 第11-12页. ISBN 978-0-7923-2565-9. OCLC 29181926.  
\item Florian Cajori (1919). 《数学史》. Macmillan. 第103页. “它不可能来自印度来源,因为印度人没有像‘恢复’和‘简化’这样的规则。他们从未有过将方程中所有项都变为正数的习惯,就像在‘恢复’过程中所做的那样。”  
\item Boyer, Carl Benjamin (1968). 《数学史》. 第252页.  
\item Saidan, A. S. (1966年冬季),"最早的现存阿拉伯算术:《阿布·哈桑·艾哈迈德·伊本·易卜拉欣·乌克立迪斯的《印度算术》》",《伊西斯》, 57(4), 芝加哥大学出版社: 475-490,doi:10.1086/350163,JSTOR 228518,S2CID 143979243  
\item Burnett 2017,第39页。  
\item Avari, Burjor (2013),《南亚的伊斯兰文明:印度次大陆的穆斯林势力与存在史》,Routledge,第31-32页,ISBN 978-0-415-58061-8,原文已于2023年3月28日归档,2020年9月29日检索。  
\item Van Brummelen, Glen (2017),"算术",在Thomas F. Glick(主编),《Routledge复刊:中世纪科学、技术与医学(2006年):百科全书》, Taylor & Francis,第46页,ISBN 978-1-351-67617-5,原文已于2023年3月28日归档,2019年5月5日检索。  
\item Thomas F. Glick, 主编 (2017),"阿尔-赫瓦里兹米",《Routledge复刊:中世纪科学、技术与医学(2006年):百科全书》, Taylor & Francis,ISBN 978-1-351-67617-5,原文已于2023年3月28日归档,2019年5月6日检索。
\item Van Brummelen, Glen (2017),"算术",收录于Thomas F. Glick(主编),《Routledge复刊:中世纪科学、技术与医学(2006年):百科全书》,Taylor & Francis,第46-47页,ISBN 978-1-351-67617-5,原文已于2023年3月28日归档,2019年5月5日检索  
\item "Algoritmi de numero Indorum",《算术论文集》,罗马:物理与数学科学印刷厂,1857年,第1页起,原文已于2023年3月28日归档,2019年5月6日检索  
\item Crossley, John N.; Henry, Alan S. (1990),"阿尔-赫瓦里兹米的语录:剑桥大学图书馆手稿Ms. Ii.vi.5的翻译",《数学史》,17(2):103–131,doi:10.1016/0315-0860(90)90048-I  
\item "算法如何得名"。earthobservatory.nasa.gov,2018年1月8日  
\item Thurston, Hugh (1996),《早期天文学》,Springer科学与商业媒体,第204页起,ISBN 978-0-387-94822-5  
\item Kennedy 1956,第26-29页  
\item van der Waerden, Bartel Leendert (1985),《代数史:从阿尔-赫瓦里兹米到艾米·诺泽》,柏林:Springer-Verlag,第10页,ISBN 978-3-642-51601-6,原文已于2021年6月24日归档,2021年6月22日检索  
\item Kennedy 1956,第128页  
\item Jacques Sesiano,"伊斯兰数学",第157页,收录于Selin, Helaine;D'Ambrosio, Ubiratan(主编),《跨文化数学:非西方数学的历史》(2000年),Springer科学+商业媒体,ISBN 978-1-4020-0260-1  
\item "三角学"。《大英百科全书》。原文已于2008年7月6日归档,2008年7月21日检索
\item 完整标题为《地球描述书,包含其城市、山脉、海洋、所有岛屿和河流,由阿布·贾法尔·穆罕默德·伊本·穆萨·阿尔-赫瓦里兹米编写,根据托勒密·克劳狄安的地理学论文》,尽管由于“surah”一词的歧义,也可以理解为“地球图像书”或甚至是“世界地图书”。
\item "制图史"。《GAP计算机代数系统》。原文已于2008年5月24日归档,2008年5月30日检索。
\item "咨询"。archivesetmanuscrits.bnf.fr。2024年8月27日检索。
\item 阿尔-赫瓦里兹米,穆罕默德·伊本·穆萨(1926年)。《阿布·贾法尔·穆罕默德·伊本·穆萨·阿尔-赫瓦里兹米的Kitāb ṣūrat al-arḍ》(阿拉伯语)。
\item Keith J. Devlin(2012年)。《数字之人:斐波那契的算术革命》(平装)。布鲁姆斯伯里,第55页,ISBN 9781408822487。
\item Daunicht
\item Edward S. Kennedy,《数学地理学》,第188页,收录于(Rashed & Morelon 1996,第185-201页)
\item Covington, Richard(2007年)。"第三维度",《沙特阿美世界》,2007年5月-6月,第17-21页。原文已于2008年5月12日归档,2008年7月6日检索。
\item LJ Delaporte(1910年)。《马尔·埃利耶·巴尔·西奈的编年史》,第十三页。
\item El-Baz, Farouk(1973年)。"阿尔-赫瓦里兹米:月球背面新发现的盆地",《科学》,180(4091):1173-1176,Bibcode:1973Sci...180.1173E,doi:10.1126/science.180.4091.1173,JSTOR 1736378,PMID 17743602,S2CID 10623582。NASA门户:阿波罗11号,摄影索引。
\item "小天体数据库查询"。ssd.jpl.nasa.gov。
\item "小天体数据库查询"。ssd.jpl.nasa.gov。
\end{enumerate}
\subsubsection{来源}  
\begin{itemize}
\item Arndt, A. B. (1983年12月)。“阿尔-赫瓦里兹米”。《数学教师》, 76(9):668–670。doi:10.5951/MT.76.9.0668。JSTOR 27963784。  
\item Boyer, Carl B. (1991年)。“阿拉伯霸权”。《数学史》(第二版)。约翰·威利父子公司。ISBN 978-0-471-54397-8。  
\item Burnett, Charles (2017年),“阿拉伯数字”,收录于Thomas F. Glick(编),《路透复刻:中世纪科学、技术与医学(2006):百科全书》,泰勒与弗朗西斯,ISBN 978-1-351-67617-5,原文已于2023年3月28日归档,2019年5月5日检索。  
\item Daffa, Ali Abdullah al-(1977年)。《穆斯林对数学的贡献》。伦敦:Croom Helm。ISBN 978-0-85664-464-1。  
\item Dunlop, Douglas Morton (1943年)。“穆罕默德·伊本·穆萨·阿尔-赫瓦里兹米”。《英国与爱尔兰皇家亚洲学会会刊》,2(3–4):248–250。doi:10.1017/S0035869X00098464。JSTOR 25221920。S2CID 161841351。原文已于2021年6月25日归档,2021年6月24日检索。  
\item Kennedy, E. S. (1956年)。“伊斯兰天文学表的概述”。《美国哲学学会会刊》,46(2):123–177。doi:10.2307/1005726。hdl:2027/mdp.39076006359272。JSTOR 1005726。原文已于2021年6月4日归档,2021年6月24日检索。  
\item Rashed, Roshdi; Morelon, Régis (1996年),《阿拉伯科学史百科全书》,第一卷,路透出版社,ISBN 0-415-12410-7。  
\item Struik, Dirk Jan (1987年)。《数学简史》(第4版)。多佛出版公司。ISBN 978-0-486-60255-4。  
\item Toomer, Gerald (1990年)。“阿尔-赫瓦里兹米,阿布·贾法尔·穆罕默德·伊本·穆萨”。收录于Charles Coulston Gillispie(编),《科学传记词典》,第7卷。纽约:查尔斯·斯克里布纳之子。ISBN 978-0-684-16962-0。原文已于2016年7月2日归档,2010年12月31日检索。
\end{itemize}
\subsection{进一步阅读}  
\subsubsection{传记}  
\begin{itemize}
\item Brentjes, Sonja (2007年)。“赫瓦里兹米:穆罕默德·伊本·穆萨·阿尔-赫瓦里兹米”,收录于Thomas Hockey等(编),《天文学家的传记百科全书》,斯普林格参考书目。纽约:斯普林格,2007年,第631–633页。(PDF版本已于2012年1月14日归档)  
\item Hogendijk, Jan P.,《穆罕默德·伊本·穆萨(阿尔)赫瓦里兹米(约780–850年)》,已于2018年2月3日归档 – 他的著作、手稿、版本和翻译的书目。  
\item O'Connor, John J.; Robertson, Edmund F.,“阿布·贾法尔·穆罕默德·伊本·穆萨·阿尔-赫瓦里兹米”,《麦克图尔数学历史档案》,圣安德鲁斯大学  
\item Sezgin, F.(编),《伊斯兰数学与天文学》,法兰克福:阿拉伯-伊斯兰科学历史研究所,1997–99年。
\end{itemize}
\subsubsection{代数}  
\begin{itemize}
\item Gandz, Solomon (1926年11月)。“‘代数’一词的起源”,《美国数学月刊》,33(9):437–440。doi:10.2307/2299605。JSTOR 2299605。已于2021年6月25日归档,2021年6月24日访问。  
\item Gandz, Solomon (1936年)。“阿尔-赫瓦里兹米代数的来源”,《奥西里斯》,1(1):263–277。doi:10.1086/368426。JSTOR 301610。S2CID 60770737。已于2021年6月25日归档,2021年6月24日访问。  
\item Gandz, Solomon (1938年)。“继承的代数:阿尔-赫瓦里兹米的重建”,《奥西里斯》,5(5):319–391。doi:10.1086/368492。JSTOR 301569。S2CID 143683763。已于2021年6月25日归档,2021年6月24日访问。  
\item Hughes, Barnabas (1986年)。“杰拉德·克雷莫纳对阿尔-赫瓦里兹米《代数》的翻译,批判版”,《中世纪研究》,48:211–263。doi:10.1484/J.MS.2.306339。  
\item Hughes, Barnabas。罗伯特·切斯特的拉丁文翻译《代数》:新批判版。拉丁文。Wiesbaden: F. Steiner Verlag (1989)。ISBN 3-515-04589-9。
\item Karpinski, L.C. (1915年)。《罗伯特·切斯特的拉丁文翻译阿尔-赫瓦里兹米的代数:附介绍、批注和英文版》。麦克米兰公司。已于2020年9月24日归档,2020年5月21日访问。  
\item Rosen, Fredrick (1831年)。《穆罕默德·本·穆萨的代数》。伦敦。
\end{itemize}
\subsubsection{天文学}  
\begin{itemize}
\item Goldstein, B.R. (1968年)。《阿尔-赫瓦里兹米的天文表的注释:由伊本·穆萨纳撰写》。耶鲁大学出版社。ISBN 978-0-300-00498-4。  
\item Hogendijk, Jan P. (1991年)。《阿尔-赫瓦里兹米的“小时正弦表”及其基础正弦表》。*Historia Scientiarum*,42:1–12。已于2021年5月7日归档,2021年6月24日访问。  
\item King, David A. (1983年)。《阿尔-赫瓦里兹米与9世纪数学天文学的新趋势》。纽约大学:哈戈普·凯沃基安近东研究中心:近东论文系列2。已于2021年6月25日归档,2021年6月24日访问。  
\item Neugebauer, Otto (1962年)。《阿尔-赫瓦里兹米的天文表》。  
\item Rosenfeld, Boris A. (1993年)。“阿尔-赫瓦里兹米、阿尔-马哈尼与伊本·海赛姆的‘几何三角学’”。收录于Folkerts, Menso; Hogendijk, Jan P. (编)。《Vestigia \item Mathematica: 向H.L.L. Busard教授致敬的中世纪与早期现代数学研究》。莱顿:布里尔出版社,第305-308页。ISBN 978-90-5183-536-6。  
\item Van Dalen, Benno (1996年)。《重新审视阿尔-赫瓦里兹米的天文表:时间方程的分析》。收录于Casulleras, Josep; Samsó, Julio (编)。《从巴格达到巴塞罗那:献给胡安·\item 韦尔内特教授的伊斯兰精密科学研究》。巴塞罗那:米拉斯·瓦利克罗萨阿拉伯科学史研究所,第195-252页。已于2021年6月24日归档,2021年6月24日访问。
\end{itemize}
\subsubsection{犹太历}  
\begin{itemize}
\item Kennedy, E. S. (1964年)。《阿尔-赫瓦里兹米论犹太历》。*Scripta Mathematica*,27:55–59。
\end{itemize}
\subsection{外部链接}
\begin{itemize}
\item 与穆罕默德·伊本·穆萨·阿尔-赫瓦里兹米相关的媒体,见于维基共享资源  
\item 《Kitab Surat al-Ard》最早手稿,收藏于斯特拉斯堡国家图书馆
\end{itemize}