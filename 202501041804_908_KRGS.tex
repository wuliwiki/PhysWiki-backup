% 卡尔·弗里德里希·高斯(综述)
% license CCBYSA3
% type Wiki

本文根据 CC-BY-SA 协议转载翻译自维基百科\href{https://en.wikipedia.org/wiki/Carl_Friedrich_Gauss}{相关文章}。

\begin{figure}[ht]
\centering
\includegraphics[width=6cm]{./figures/6c9aaaacb17b4d2e.png}
\caption{由克里斯蒂安·阿尔布雷希特·延森绘制的肖像,1840年(由戈特利布·比尔曼复制,1887年)} \label{fig_KRGS_3}
\end{figure}
约翰·卡尔·弗里德里希·高斯(德语:Gauß [kaʁl ˈfʁiːdʁɪç ˈɡaʊs];拉丁语:Carolus Fridericus Gauss;1777年4月30日–1855年2月23日)是德国数学家、天文学家、测量学家和物理学家,对数学和科学的多个领域做出了重要贡献。他自1807年起成为哥廷根天文台台长和天文学教授,直至1855年去世。高斯被广泛认为是历史上最伟大的数学家之一。

在哥廷根大学学习期间,他提出了多个数学定理。高斯以私人学者的身份完成了他的代表作《算术研究》和《天体运动理论》。他给出了代数基本定理的第二个和第三个完整证明,对数论作出了贡献,并发展了二次和三次二次型的理论。

高斯在发现冥王星作为矮行星的工作中发挥了重要作用。他关于受到大行星影响的行星状物体的运动的研究,导致了高斯引力常数和最小二乘法的引入,而高斯在Adrien-Marie Legendre发表之前就已发现了这一方法。高斯与其他学者一起负责了1820年至1844年期间对汉诺威王国的大规模地理测量和弧长测量项目;他是地球物理学的创始人之一,并提出了磁学的基本原理。他实际工作的成果包括1821年发明了太阳能标,1833年发明了磁力计,以及与威廉·爱德华·韦伯一起于1833年发明了第一台电磁电报机。

高斯是第一个发现并研究非欧几里得几何的人,并且他自己创造了这个术语。他还在约160年前就发展了快速傅里叶变换,比约翰·图基和詹姆斯·库利提前了数十年。

高斯拒绝发表未完成的工作,留下了多部未完成的作品,交由后人编辑。他认为学习的过程,而非拥有知识本身,才是最令人享受的。高斯曾坦言自己不喜欢教学,但他的一些学生后来成为了有影响力的数学家,如理查德·德德金德和伯恩哈德·黎曼。
\subsection{传记}  
\subsubsection{青年时期与教育}
\begin{figure}[ht]
\centering
\includegraphics[width=6cm]{./figures/b21e0aea7920ceb9.png}
\caption{布伦瑞克出生之家(在第二次世界大战中被摧毁)} \label{fig_KRGS_1}
\end{figure}
高斯于1777年4月30日出生在布伦瑞克公国(今德国下萨克森州的布伦瑞克市)。他的家庭社会地位相对较低。父亲盖布哈德·迪特里希·高斯(1744-1808)从事过屠夫、砖瓦匠、园丁和丧葬基金的财务工作。高斯曾形容他的父亲是一个正直而受人尊敬的人,但在家中则是粗暴且专横的。他的父亲擅长写作和计算,而高斯的继母多萝西娅几乎是文盲。高斯有一个从父亲第一次婚姻中生的哥哥。

高斯在数学方面是一个天才儿童。当他的小学老师注意到他的智力时,便将他推荐给布伦瑞克公爵。公爵将他送到当地的卡罗林学院学习,并在那里从1792年到1795年学习,埃伯哈德·奥古斯特·威廉·冯·齐默曼是他的老师之一。之后,公爵为他提供了在哥廷根大学学习数学、科学和古典语言的资源,直到1798年为止。高斯的数学教授是亚伯拉罕·戈特尔夫·凯斯特纳,高斯称他为“诗人中的数学大师,数学家中的诗人”,因为凯斯特纳有许多讽刺性诗句。天文学是由卡尔·费利克斯·赛弗教授的,毕业后高斯与赛弗保持通信;奥尔伯斯和高斯在他们的信件中取笑了他。另一方面,高斯对他的物理学老师乔治·克里斯托夫·李希滕贝格和基督教·戈特洛布·海恩的古典学课程给予高度评价,他愉快地参加了海恩的讲座。这段时间的同学包括约翰·弗里德里希·本岑贝格、法尔卡斯·博尔亚伊和海因里希·威廉·布兰德斯。

他可能是一个自学成才的数学学生,因为他独立重新发现了几条定理。1796年,他解决了一个自古希腊以来困扰数学家的几何问题,确定了哪些规则多边形可以通过圆规和直尺作图。这一发现最终使高斯选择了数学而不是语言学作为职业。高斯的数学日记,记录了他从1796年到1814年间的许多数学成果的简短备注,显示出他许多思想的萌芽,这些思想最终成为他数学巨著《算术研究》(1801)的基础。
\begin{figure}[ht]
\centering
\includegraphics[width=6cm]{./figures/ca9e9a3b21b12fba.png}
\caption{高斯在哥廷根作为学生时的住所} \label{fig_KRGS_2}
\end{figure}
\subsubsection{私人学者}  
高斯于1799年获得哲学博士学位,这一事实有时被误称为在哥廷根大学毕业,实际上是应布伦瑞克公爵的特别请求,从赫尔姆施塔特大学(公国唯一的州立大学)毕业的。约翰·弗里德里希·法夫评审了他的博士论文,高斯在没有进一步口试的情况下以缺席方式获得了学位。随后,公爵为他提供了作为私人学者在布伦瑞克的生活费用。高斯因此拒绝了圣彼得堡的俄罗斯科学院和兰茨胡特大学的邀请。后来,公爵在1804年承诺为他在布伦瑞克建立一个天文台。建筑师彼得·约瑟夫·克拉赫设计了初步的方案,但由于拿破仑战争,这些计划被取消:公爵在1806年的耶拿战役中阵亡。次年,公国被废除,高斯的经济支持也随之停止。

当高斯在世纪初的几年里计算小行星轨道时,他与不来梅和利连塔尔的天文界建立了联系,特别是与威廉·奥尔伯斯、卡尔·路德维希·哈丁和弗里德里希·威廉·贝塞尔等人,他们是“天体警察”这一非正式天文学家小组的一部分。该小组的目标之一是发现更多的行星。他们收集了小行星和彗星的数据,为高斯后来在其天文学巨著《天体运动论》(1809年)中发表的轨道研究提供了基础。
\subsubsection{哥廷根大学教授}
\begin{figure}[ht]
\centering
\includegraphics[width=8cm]{./figures/27ae38e6ad98b0b0.png}
\caption{约1800年的旧哥廷根天文台} \label{fig_KRGS_4}
\end{figure}
1807年11月,高斯应召到哥廷根大学,时该校隶属于新成立的西法利亚王国,由 Jérôme Bonaparte 统治,担任天文学教授兼天文台主任,并一直担任此职务直到1855年去世。很快,他就面临了西法利亚政府要求支付两千法郎作为战争贡献的要求,这笔费用他无力支付。奥尔伯斯和拉普拉斯都希望帮助他支付这笔费用,但高斯拒绝了他们的援助。最终,一位来自法兰克福的匿名人士——后来被发现是大公子达尔贝格——支付了这笔款项。

高斯接管了这座已有60年历史的天文台,该天文台由选帝侯乔治二世于1748年建立,建在一座改建过的防御塔上,仪器可用,但部分已过时。选帝侯乔治三世自1802年起原则上批准了新天文台的建设,西法利亚政府也继续进行规划,但高斯直到1816年9月才得以搬到新的工作地点。他获得了新的现代化仪器,包括从Repsold和Reichenbach公司购买的两台经纬仪和从弗劳恩霍夫购买的日心仪。

高斯的科学活动,除了纯数学外,大致可以分为三个阶段:19世纪前二十年以天文学为主,第三十年为测地学,第四十年则主要以物理学,特别是磁学为主。

高斯毫不掩饰自己对讲授学术课程的反感。但自从在哥廷根大学开始学术生涯以来,他一直持续讲授直到1854年。他经常抱怨教学的负担,觉得这是浪费时间。另一方面,他偶尔也会称某些学生才华横溢。他的大多数讲座涉及天文学、测地学和应用数学,仅有三次讲授纯数学的课程。高斯的学生中,有些人成为著名的数学家、物理学家和天文学家,如莫里茨·坎托尔、德德金德、迪尔克森、恩克、古尔德、海涅、克林克费乌斯、库普费尔、利斯廷、莫比乌斯、尼古莱、黎曼、里特、谢林、谢尔克、舒马赫、冯·斯陶特、斯特恩、乌尔辛;在地球科学领域,如萨托留斯·冯·瓦尔特豪森和瓦普厄斯。

高斯没有写过任何教科书,也不喜欢科学内容的普及。他唯一的普及尝试是他关于复活节日期的著作(1800/1802年)和1836年的《地磁学与磁力计》论文。高斯的论文和著作完全用拉丁文或德语发表。他的拉丁文写作风格古典,但使用了一些当代数学家所设定的常见修饰。