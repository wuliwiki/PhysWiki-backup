% Python 模块
% python|module|模块

\pentry{Python 入门\upref{Python}}

在 “Python 入门\upref{Python}” 中, 我们示范了如何从 \verb|math| 模块中引入一些变量和函数(如 \verb|pi| 和 \verb|sin|). 现在我们来学习 Python 模块的原理.

事实上, Python 的模块可以是任意一个普通的 \verb|.py| 拓展名的文件. 这个文件中可以定义全局变量, 函数, 类, 甚至是可执行的命令. 当我们要在其他 \verb|.py| 文件中使用这个文件中的函数或全局变量, 只需要声明把前者作为模块使用即可.

例如创建模块文件 \verb|mymodule.py| 如下
\begin{lstlisting}[language=python]
# 函数
def plus1(num):
    return num + 1
# 全局变量
num = 3.5
num2 = plus1(num)
\end{lstlisting}
在该模块中定义了一个函数和两个全局变量. 注意在定义第二个变量时执行了函数 \verb|plus1|, 这个命令会在模块被调用时执行.

\subsection{使用模块}
现在,我们我们在同一文件夹中新建如下 \verb|test.py|, 就可以用 \verb|import| 语句来使用我们刚刚创建的模块.
\begin{lstlisting}[language=python]
import mymodule
print('num = ', mymodule.num)
print('num2 = ', mymodule.num2)
print('num2 + 1 = ', mymodule.plus1(mymodule.num2))
\end{lstlisting}
在 “Python 入门\upref{Python}” 中我们还介绍了模块的不同导入方法, 这里不再赘述.

除了内建模块外, Python 默认只会在当前文件夹(即路径)下搜索模块文件. 若要添加其他搜索路径, 可以在 \verb|import| 命令之前使用
\begin{lstlisting}[language=python]
import sys
sys.path.insert(1, 路径)
\end{lstlisting}
其中 \verb|路径| 既可以是绝对路径也可以是绝对路径. % 未完成:介绍当前路径
例如要指定当前路径的子文件夹 \verb|modules|, 则使用 \verb|'modules/'| 即可; 如果是上层文件夹中的子文件夹, 可用 \verb|'../modules/'|.

\subsection{模块嵌套}
在上面的例子中, 如果在 \verb|mymodule.py| 文件中使用了 \verb|import| 语句导入其他模块中的函数和变量等, 那么在其 \verb|test.py| 中也可以使用这些函数和变量. 例如在 \verb|mymodule.py| 第一行插入
\begin{lstlisting}[language=python]
from math import pi
\end{lstlisting}
那么在 \verb|test.py| 中就可以使用 \verb|mymodule.pi| 获取这个变量. 但如果把这行改为
\begin{lstlisting}[language=python]
import math
\end{lstlisting}
那么在 \verb|test.py| 中要使用 \verb|mymodule.math.pi| 才能获取这个变量.

\subsection{Python 内建模块}
Python 中有几个\textbf{内建模块(built-in module)},可以随时导入.

\subsubsection{math 模块}
(未完成)

\subsubsection{time 模块}
在Python中,通常有这几种方式来表示时间:

时间戳(timestamp) :通常来说,时间戳表示的是从1970年1月1日 00:00:00 开始按秒计算的偏移量.我们运行 \verb|type(time.time())|, 返回的是 \verb|float| 类型.

格式化的时间字符串

元组(\verb|struct_time|) :\verb|struct_time| 元组共有9个元素:(年,月,日,时,分,秒,一年中第几周,一年中第几天,夏令时)

\subsubsection{random 模块}
随机生成内容.
\begin{lstlisting}[language=python]
import random
# 0,1之间时间生成的浮点数  float
print(random.random())
# 随机生成传入参数范围内的数字 即 1,2,3
print(random.randint(1, 3))
# 随机生成传入参数范围内的数字,range 顾头不顾尾
print(random.randrange(1, 3))
# 随机选择任意一个数字
print(random.choice([1, '23', [4, 5]]))
\end{lstlisting}

\subsubsection{sys 模块}
\begin{lstlisting}[language=python]
import sys
# 命令行参数List,第一个元素是程序本身路径
print(sys.argv)
# 结果['D:/Pycharm Community/python内置函数/sysTest.py']
# 获取Python解释程序的版本信息
print(sys.version)
\end{lstlisting}

\subsubsection{os 模块}
\verb|os| 模块是与操作系统交互的一个接口.
\begin{lstlisting}[language=python]
os.getcwd() # 获取当前工作目录,即当前python脚本工作的目录路径
os.chdir("dirname")  # 改变当前脚本工作目录;相当于shell下cd
os.curdir  # 返回当前目录: ('.')
os.makedirs('dirname1/dirname2')    # 可生成多层递归目录
# 若目录为空,则删除,并递归到上一级目录,如若也为空,则删除,依此类推
os.removedirs('dirname1')
os.mkdir('dirname')    # 生成单级目录
os.rmdir('dirname')    # 删除单级空目录,若目录不为空则无法删除,报错;
# 列出指定目录下的所有文件和子目录,包括隐藏文件,并以列表方式打印
os.listdir('dirname')
os.remove()  # 删除一个文件
os.rename("oldname","newname")  # 重命名文件/目录
os.stat('path/filename')  # 获取文件/目录信息
os.path.abspath(path)  # 返回path规范化的绝对路径
os.path.split(path)  # 将path分割成目录和文件名二元组返回
os.path.exists(path) #  如果path存在,返回True;如果path不存在,返回False
os.path.isabs(path)  # 如果path是绝对路径,返回True
os.path.isfile(path)  # 如果path是一个存在的文件,返回True.否则返回False
os.path.isdir(path) # 如果path是一个存在的目录,则返回True.否则返回False
os.path.join(path1[, path2[, ...]])  将多个路径组合后返回
\end{lstlisting}
