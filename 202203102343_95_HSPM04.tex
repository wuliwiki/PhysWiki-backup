% 曲线运动(高中)
% keys 曲线运动|合运动|分运动|小船渡河|抛体运动|平抛|斜抛

\begin{issues}
\issueDraft
\issueTODO
\end{issues}

\subsection{曲线运动}

定义:轨迹是曲线的运动.

条件:当物体所受的合外力(加速度)与其速度方向不在同一直线上时,物体做曲线运动.

物体所受合外力方向(加速度方向):总是指向曲线的凹侧.

速度方向:质点在某一位置的速度方向与曲线在这一点的切线方向一致.

\subsection{运动的合成与分解}

如果物体同时参与了几个运动,物体实际发生的运动叫做那几个运动的\textbf{合运动},那几个运动就叫做这个实际运动的\textbf{分运动}.事实上,任何物体在确定的参考系中只能实际参与一个运动,分运动是为了便于分析而对实际运动所进行的等效替代,并非真实存在.合运动和分运动必须是对应同一个物体在相同的时间内发生的.

与力的合成与分解类似,由分运动求跟合运动的过程叫做\textbf{运动的合成},由合运动求分运动的过程叫做\textbf{运动的分解}.在描述运动时,我们用到了位移、速度和加速度这些量,因此在进行运动的合成与分解时,实际上就是对位移或速度或加速度的合成与分解.由于位移、速度和加速度都是矢量,因此运动的合成与分解都遵循平行四边形定则.

运动的合成与分解,从目的来说,是为了将复杂的曲线运动转化为独立而互不影响的直线分运动.利用运动的分解来分析运动问题时,通常会对运动进行正交分解(其中一个方向由题目而定),这样的分解方法能让问题的求解过程变得更为简单.

\subsection{小船渡河问题}

小船渡河是学习运动的合成与分解时较为典型的问题.在具体分析前,我们先假设河宽为$d$,小船在静水中的速度为$\bvec v_1$,小船静止随水漂流的速度为$\bvec v_2$.小船渡河实际运动就是小船在静水时的运动和随水漂流的运动的合运动,实际速度$\bvec v$就是小船在静水中的速度和随水漂流的速度的合速度.这里要注意的是不能说成“实际速度是小船在静水中的速度和水流速度的合速度”,因为这个描述并不符合合运动与分运动的同体性.

在小船渡河的过程中,如果任意时刻$\bvec v_1$和$\bvec v_2$都保持不变,那么船的实际速度$\bvec v$也对应地保持不变,此时船的实际运动就是匀速直线运动.由于$\bvec v_2$的方向是与河岸平行的,因此小船渡河的实际方向,就由船头的朝向即$\bvec v_1$的方向决定.如果$\bvec v_1$和$\bvec v_2$会随时间发生改变,那么船的实际运动就可能变成曲线运动.接下来我们只对第一种情况展开讨论.

设$\bvec v_1$与$\bvec v_2$的夹角为$\theta$,显然要渡河就必须有$0^\circ < \theta < 180^\circ $,否则这个问题就变成了顺水、逆水行船问题.

\subsubsection{最短渡河时间}

由于小船随水漂流的速度方向与河岸平行,因此这个速度并不会影响小船垂直于河岸方向的运动,因此小船在垂直于河岸方向的速度只与小船的静水速度有关.以沿河岸方向为$x$轴以及垂直于河岸的方向为$y$轴建立直角坐标系,设船的静水速度与垂直于河岸的方向夹角为$\alpha$,则$v_y = v_1 \cos \alpha$,渡河时间$t=d/v_y$.由此可知,当$\alpha = 0^\circ$,即船头方向垂直于河岸时,小船的实际速度在垂直河岸方向的分速度达到最大值$v_1$,即船的静水速度全部用于渡河,此时渡河时间最短,$t_{min}=d/v_1$,实际速度大小$v=\sqrt{v_1^2+v_2^2}$,对应渡河位移大小$s=vt_{min}=\sqrt{v_1^2+v_2^2}\cdot d/v_1$.

\subsubsection{最短渡河位移}

从平行线的关系来看,最短的渡河位移大小就是河宽,那是否任意的船速都能实现呢?以河宽为最短位移大小时,船的实际速度方向应垂直于河岸,则船头方向应偏向上游.根据三角形定则,此时$v_1$、$v_2$和$v$构成了以$v_1$为斜边的直角三角形,在这个三角形中有$v_1>v_2$.因此,在$v_1>v_2$的情况下,小船渡河的最短位移大小等于河宽$d$,小船的实际速度大小$v=\sqrt{v_1^2-v_2^2}$,对应的渡河时间$t=d/\sqrt{v_1^2-v_2^2}$.

当$v_1<v_2$时,船无法垂直于河岸运动,此时无论船头方向如何,小船的实际运动方向总是偏向下游,设小船的实际速度与水流方向的夹角为$\varphi$,则渡河位移$s = d/\sin \varphi$,要使位移最小,则$\varphi$要达最大值.通过三角形定则分析几何关系可知,当$v_1$、$v_2$和$v$构成了以$v_2$为斜边的直角三角形时,$\varphi$角有最大值,$\varphi=\arcsin(v_1/v_2)$,此时小船的实际速度大小$v=\sqrt{v_2^2-v_1^2}$,最短渡河位移大小$s_{min} = d/\sin \varphi=dv_2/v_1$.

通常,关于小船渡河问题的最短位移的讨论不会提及$v_1=v_2$的情况,因为这种情况下只能结合极限的概念得出一个趋近的数值.此处简单地分析一下:当$v_1=v_2$,小船的实际速度与水流方向的夹角为$\theta /2$,实际速度大小$v=2v_2\cos(\theta /2)$,渡河位移大小$s= d/\sin(\theta/2)$,由于$0^\circ < \theta < 180^\circ $,可见$\theta$越大渡河位移越小,因此$\theta$趋近于$180^\circ $时,渡河位移大小趋近于$d$,而实际速度大小趋于$0$,行船时间则趋于无穷大.

\subsection{抛体运动}

\subsubsection{平抛运动}

\subsubsection{斜抛运动}
