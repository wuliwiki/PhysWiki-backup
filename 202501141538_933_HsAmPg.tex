% 等差数列(高中)
% keys 高中|等差数
% license Usr
% type Tutor

\begin{issues}
\issueDraft
\end{issues}

\pentry{数列(高中)\nref{nod_HsSeFu}}{nod_53a2}

不知道你是否注意过,人们在上台阶时,一般会一阶一阶地走;有的人喜欢两阶两阶地跨,总是隔着一阶;还有的人更大胆,三阶三阶地跨。这些看似随意的方式,其实都有一个固定的模式——每次跨的台阶数是恒定的。这种固定的规律,就像一条隐形的线,把每次的动作串联在一起,形成了数学中所说的等差数列。

等差数列是最简单的数列之一,早在小学阶段就已经接触过了——如果从 $1$ 开始选取到  $n$  的  $n$  个自然数,它们按顺序排列就构成了一个等差数列,而这个数列的求和方法更是耳熟能详。相传,高斯小时候被老师要求计算从 1 加到 100 的总和,他敏锐地发现了其中的规律:可以将首项和末项配对,这样所有的项都变成了相同的和,再乘以项数的一半就能快速得到结果。这种求和方式总结为一句顺口溜:“\textbf{首项加末项,乘以项数除以 2}”,至今仍为许多人津津乐道。

在接下来的内容中,将探索等差数列及其相关的数学性质,有以下几点需要关注:首先是等差数列本身的结构与特性,其次是它的求和公式及相关推导过程。此外,还将探讨等差数列与函数的关系,理解它在更广泛数学背景中的地位。作为学习数列的第一步,等差数列也为研究其他数列提供了一个范例,通过它可以学习如何分析数列的构成、关注哪些信息,以及用哪些方法来进行研究。这不仅是对数列知识的学习,也是一种数学思维方式的培养。

\subsection{等差数列}\label{sub_HsAmPg_1}

从台阶的故事中可以看出,不论是每次跨一阶、两阶还是三阶,这些方式都有一个共同的特点:每次跨的台阶数是固定的,即两次动作之间的差始终相等。这种规律不仅出现在上台阶这样的情景中,还广泛存在于其他现象中,例如每天增加固定的储蓄金额,或者不断调整音量时每次调节相同的数值。无论这些场景多么不同,它们的本质都是一种“均匀增加或减少”的过程,由固定的差值串联起来。

\begin{definition}{等差数列}\label{def_HsAmPg_1}
如果数列 $\{a_n\}$ 满足对于 $n > 1$ 的所有项,每一项与前一项的差为同一个常数 $d$,则称 $\{a_n\}$ 为\textbf{等差数列(arithmetic sequence)},$d$ 称为$\{a_n\}$的\textbf{公差(common difference)},即等差数列满足递推公式
\begin{equation}
a_{n}=a_{n-1}+d\qquad(n>1)~.
\end{equation}
\end{definition}

特别地,之前提到的常数列是 $d = 0$ 的等差数列。如果公差 $d$ 为负,数列的值会逐渐减少,例如 $10, 8, 6, \dots$。有了递推公式,自然要研究一下是否可以得到通项公式,下面介绍两种推导等差数列通项公式的方法:通常,可以通过检测一个数列是否满足定义的条件来判断其是否是\textbf{等差数列(arithmetic sequence)}。

方法一:迭代递推公式

当$n>1$ 时,利用递推公式  $a_n = a_{n-1} + d$ 逐步将数列的前一项用递推公式展开,可以得到:
\begin{equation}\label{eq_HsAmPg_6}
\begin{aligned}
a_n &= a_{n-1} + d \\
&= a_{n-2}+d + d\\
&\cdots \\
&= a_2 + (n-2)d\\
&=  a_1 + (n-1)d~.
\end{aligned}
\end{equation}

方法二:错位相减法

它是累加法的一个特例,核心思想是利用数列的结构性(如相邻项有规律的差或比),通过累加将多项式或等式中的部分消去,从而简化计算。实际操作时,首先根据递推公式写出$n-1$个等式:
\begin{equation}\label{eq_HsAmPg_7}
\begin{split}
        &      &  &    & & a_2& -a_1 & = d, \\
        &      & & a_3 & -&a_2&      & = d, \\
        & \ \ \ \ a_4  &- &a_3& &    &      & = d, \\
        &        &    && &    &      &\cdots \\
a_n -   & a_{n-1}&   & & &    &      & = d~.
\end{split}
\end{equation}
从上面的格式就可以看出,相邻两行相同的项正好可以抵消。实际操作时,可以把将这些等式的左侧和右侧分别相加,得到:
\begin{equation}
(a_2 - a_1) + (a_3 - a_2) + (a_4 - a_3) + \cdots + (a_n - a_{n-1}) = d + d +d+ \cdots + d~.
\end{equation}
将左侧的括号打开,这时由于相邻项的抵消,剩余的部分只有$a_n-a_1$,左侧则是$n-1$项和,即:
\begin{equation}\label{eq_HsAmPg_8}
a_n -a_1=(n-1)d~.
\end{equation}

\autoref{eq_HsAmPg_6} 的迭代推导与 \autoref{eq_HsAmPg_7} 通过差的累加推导,分别代表了研究数列的两种重要构造方法:前者通过递推公式逐步展开,体现了数列构造中的递推思想;后者利用$a_n=(a_n-a_{n-1})+ \cdots+(a_3 - a_2) +(a_2 - a_1)  + a_1$的技巧,将通项的构造转化为求和问题,从而利用数列的内在规律化简计算。尽管两种方法路径不同,最终均将第 $n$ 项表示为首项 $a_1$ 和公差 $d$ 的线性关系。需要注意的是,在推导出公式后,必须验证其是否符合边界条件。当  $n = 1$  时,代入\autoref{eq_HsAmPg_6} 得:
\begin{equation}
a_1 = a_1 + (1 - 1)d = a_1~,
\end{equation}
这个结果表明,这个关系对 $n = 1$ 的边界条件依然成立,因此最终得到等差数列的通项公式。
\begin{corollary}{等差数列通项公式}
对等差数列$\{a_n\}$,其通项公式为:
\begin{equation}\label{eq_HsAmPg_5}
a_n = a_1 + (n - 1)d~.
\end{equation}
其中,$a_1$ 是首项,$d$ 是公差,$n$ 是项数。
\end{corollary}

在判断一个数列是否为等差数列时,可以通过两种方法:一种是利用通项公式,由于通项公式常见,因此经常成为主要的判断依据;另一种是检测该数列是否满足\aref{定义}{def_HsAmPg_1}的条件,即任意相邻两项的差为定值。这后一种方法虽然在实际应用中常被忽略,但同样有效。由于通项公式与定义是等效的,因此这两种方法在判断上的效力相同。不仅如此,通过它们判断一个数列为等差数列后,还可以直接得到公差 $d$。需要注意的是,公差在等差数列中起着核心作用,它决定了数列的整体变化规律和趋势。进一步地,对于任何等差数列,一旦确认其为等差数列,首项和公差便能够唯一确定。因此,如果两个数列的首项和公差对应相等,那么可以认定它们是同一个等差数列。

\begin{example}{证明:若$\{a_n\}$通项公式形式为$a_n=kn+b$,则其为等差数列。}\label{ex_HsAmPg_3}
对于$n>1$,有:
\begin{equation}
a_n-a_{n-1}=kn+b-k(n-1)-b=k~.
\end{equation}
说明$\{a_n\}$是以$k$为公差的等差数列,通项公式带入$n=1$有,$a_1=b+k$。
\end{example}

\autoref{ex_HsAmPg_3} 提供了另一个常见的判断等差数列的方法。

\subsection{等差数列的性质}

观察\autoref{ex_HsAmPg_3} 可以发现,其形式与一次函数完全一致。其实,通过变形等差数列的通项公式,可以得到以下形式:
\begin{equation}\label{eq_HsAmPg_9}
a_n=dn+(a_1-d)~.
\end{equation}
在\enref{数列}{HsSeFu}中曾介绍过数列与函数的关系,\autoref{eq_HsAmPg_9} 与\autoref{ex_HsAmPg_3} 都可以说明,等差数列实际上是一次函数的离散形式。换句话说,在直角坐标系中,如果将数列的项数 $n$ 作为横坐标,数列的值 $a_n$ 作为纵坐标,则点 $(n, a_n)$ 将分布在一条直线上。

由于直线上任意一点和斜率即可确定整条直线,对于等差数列,同样可以不难证明,只需知道任意一项及公差,就能够确定整个数列,满足以下关系式:
\begin{equation}
a_n=a_k+(n-k)d~.
\end{equation}
进而,可以推导出另一个重要关系:
\begin{equation}\label{eq_HsAmPg_1}
d={a_n-a_k\over n-k}~.
\end{equation}
这一公式与\aref{平均变化率}{def_HsFunC_3}或\aref{斜率}{def_HsFunC_5}的定义非常相似,也进一步说明了等差数列与一次函数之间的密切关系。根据直线的性质,可以类比得到下面的判断:

\begin{corollary}{等差数列的增减性}\label{cor_HsAmPg_2}
对于公差为 $d$ 的等差数列 $\{a_n\}$:
\begin{itemize}
\item 如果 $d > 0$,则 $\{a_n\}$ 是递增数列;
\item 如果 $d < 0$,则 $\{a_n\}$ 是递减数列。
\end{itemize}
\end{corollary}

这一点也可以从等差数列的递推公式变形得到的$a_{n+1}-a_n=d$中得到印证。另外,根据\autoref{eq_HsAmPg_1},还可以得到等差数列的一个重要性质:取四个整数,满足 $m+n=p+q$,即 $p-n=m-q$,则有:
\begin{equation}
{a_p-a_n\over p-n}=d={a_m-a_q\over m-q}~.
\end{equation}
带入两侧分母相等的条件可以得到:
\begin{corollary}{}\label{cor_HsAmPg_1}
对于等差数列$\{a_n\}$,若满足$m+n=p+q$,则有:
\begin{equation}
a_m+a_n=a_p+a_q~.
\end{equation}
\end{corollary}

如果在 $a$ 和 $b$ 之间插入一个数 $A$,使 $a, A, b$ 成等差数列,即:
\begin{equation}\label{eq_HsAmPg_10}
A - a = b - A~.
\end{equation}
则称 $A$ 为 $a$ 与 $b$ 的\textbf{等差中项(median of an arithmetic sequence)}。显然,可以得到:
\begin{equation}
A = \frac{a+b}{2}~.
\end{equation}
这说明,$a$ 与 $b$ 的等差中项正好是他们的\textbf{算术平均值(arithmetic mean)}\footnote{关于算术平均值参见\aref{基本不等式}{sub_HsIden_2}}。反过来,从\autoref{eq_HsAmPg_10} 也可以得到:
\begin{equation}
2A = a+b~.
\end{equation}
结合\autoref{cor_HsAmPg_1} 可以知道,对于一个等差数列,如果某两项的项数和恰好是另一项的两倍,那么这第三项就是那两项的等差中项。不过需要注意,这要求那两项的项数和必须是偶数,否则无法在等差数列中找到符合条件的项\footnote{与此不同的是,对于直线,无论什么情况下都可以找到对应的点。}。

从上述内容可以看出,“等差中项”这个名字可以从两个角度理解:一方面,它可以指在两个数之间插入的一个数,与它们构成等差关系;另一方面,它也可以指等差数列中,处于某两项中间位置的那项。

\subsection{等差数列的数列和}

讨论完等差数列的性质后,接下来研究其数列和的计算方法。在学习数列时,曾提到过数列和的一个\aref{性质}{eq_HsSeFu_1}。基于这一性质,可以通过将数列分别正序和倒序排列后相加来计算其和:
\begin{equation}\label{eq_HsAmPg_4}
\begin{split}
2S_n &= (a_1 + a_2 + \cdots + a_n)+(a_n + a_{n-1} + \cdots + a_1)\\
&=(a_1+a_{n})+(a_2+a_{n-1}) +\cdots +(a_n+a_1)~.
\end{split}
\end{equation}
观察\autoref{eq_HsAmPg_4} ,每对括号中的两项下标之和均为 $n+1$,根据\autoref{cor_HsAmPg_1} ,这意味着括号内的和均相等,设数列共有 $n$ 项,取和为$a_1 + a_n$,则共有 $n$ 对这样的和,因此:
\begin{equation}\label{eq_HsAmPg_2}
2S = n \cdot (a_1+a_n)\implies S = \frac{n\cdot(a_1+a_n)}{2}~.
\end{equation}

这一公式表明,等差数列的和可以通过一种简洁的方法计算。这与本文开头提到的“首项加末项,乘以项数除以2”方法完全一致。将等差数列的\aref{通项公式}{eq_HsAmPg_5}代入\autoref{eq_HsAmPg_2} 可以得到等差数列和的通项公式:
\begin{corollary}{等差数列和的通项公式}
对等差数列$\{a_n\}$,其数列和$\{S_n\}$的通项公式为:
\begin{equation}\label{eq_HsAmPg_3}
S_n = na_1+\frac{n(n-1)}{2}d~.
\end{equation}
其中,$a_1,d$ 是等差数列的首项和公差,$n$ 是项数。
\end{corollary}

等差数列的数列和通项公式与等差数列自身密切相关,两者都由首项和公差唯一决定。通过对\autoref{eq_HsAmPg_3} 变形,可以将其表达为:
\begin{equation}
S_n = \frac{d}{2}n^2+\left(a_1-\frac{d}{2}\right)n~.
\end{equation}
这一表达式表明,数列和分布在关于 $n$ 的经过原点的二次函数上,可以验证,若 $\{a_n\}$ 是等差数列,则 $S_n$ 必须具有 $S_n = An^2 + Bn$ 的形式。

\begin{example}{若$\{a_n\}$的数列和为$S_n=An^2+Bn+C,(ABC\neq0)$,判断$\{a_n\}$是否为等差数列。}\label{ex_HsAmPg_1}
答:

$\{a_n\}$不是等差数列。

解析:

首先,由数列的和 $S_n$ 的定义可得首项为$a_1=S_1=A+B+C$。

对于$n>1$,根据 $a_n = S_n - S_{n-1}$,有
\begin{equation}
\begin{split}
a_n &= S_n-S_{n-1}\\
&=An^2+Bn+C-A(n-1)^2-B(n-1)-C\\
&=(2n+1)A+B~.
\end{split}
\end{equation}

分别计算相邻两项的差:

\begin{equation}
\begin{split}
a_2 - a_1 &= (2A \cdot 2 + A + B) - (A + B + C) \\
&= 4A + A + B - A - B - C \\
&= 2A - C~.
\end{split}
\end{equation}

$n>2$时,两项之差为:
\begin{equation}
a_n-a_{n-1} = (2n+1)A+B-(2n-1)A-B=2A~.
\end{equation}

由此可以看出,从 $a_2$ 开始,数列 $\{a_n\}$ 的相邻两项之间的差是固定的,即 $2A$,因此后续项构成等差数列。然而,由于 $C \neq 0$,$a_2 - a_1 = 2A - C$ 与 $a_n - a_{n-1} = 2A$ 不相等,说明首项与后续项之间的差异导致 $\{a_n\}$ 整体不构成等差数列。

\end{example}

在\autoref{ex_HsAmPg_1} 中,尽管数列 ${a_n}$ 的后续项具有等差性,但由于边界条件的特殊性(首项与后续项之间存在差异)影响了整体的判断。这里再次提醒,边界条件在问题中起到关键作用,在处理计算题时,必须特别注意边界条件是否成立,以避免因忽视边界条件而导致的错误结论。

数列和分布的具体形态由公差 $d$ 决定:
\begin{itemize}
\item 当 $d > 0$ 时,数列和分布在一条开口向上的抛物线上;
\item 当 $d < 0$ 时,数列和分布在一条开口向下的抛物线上。
\end{itemize}

由于数列和的分布呈抛物线形状,自然会引发一个常见问题:它的最值出现在何处?以下以 $d > 0$ 为例进行分析,即数列和分布在一条开口向上的抛物线上。

这样的抛物线导数为正正好对应\autoref{cor_HsAmPg_2} 中“${a_n}$ 是递增数列”的结论。由于抛物线是开口向上的,其最小值是必然存在的。根据抛物线性质,已知一个零点为原点,如果抛物线的对称轴位于原点左侧,则右侧部分单调递增,这意味着最小值出现在自变量最小处,即原点。如果对称轴位于原点右侧,则最小值出现在对称轴对应的位置。因此,对称轴的位置成为分析抛物线最值的关键。事实上,对称轴对应的正是导数为零的点。

这时说明抛物线的导数为正,这也正对应\autoref{cor_HsAmPg_2} 中“$\{a_n\}$ 是递增数列”的判断。既然分布开口向上的抛物线上,那么必然有最小值。根据平时对于抛物线的分析,已知一个零点是原点,那么如果抛物线的对称轴在原点左侧,则右侧部分单调递增,也就是说自变量最小对应的就是最小值。如果抛物线的对称轴在原点右侧,那么最小值应该出现在对称轴处。这说明对称轴是一个很重要的分析目标,事实上,抛物线的对称轴对应的正是导数为零的点。根据数列与数列和的关系,求解的其实就是数列值为零的点,通过令\autoref{eq_HsAmPg_5} 中的$a_n=0$可以解得:
\begin{equation}\label{eq_HsAmPg_11}
n_0=1-\frac{a_1}{d}~.
\end{equation}
由于$\frac{a_1}{d}$并不一定是一个整数,这里解出的$n_0$一般会出现两种情况,一种是$n_0$是整数,此时$a_{n_0}=0$,另一种情况$n_0$介于某个数$n$和$n+1$之间,这时$a_n<0$,$a_{n+1}>0$。这也正体现了处理数列时,出现的与函数直接得到一个具体实数对称轴的区别。

根据上面的分析,如果$n_0\leq1$,最小的$S_n$都是$S_1$,则可解得$a_1\geq0$,即$a_1$非负时,最小的$S_n$在$n=1$处。反之,若$a_0<0,d>0$,则一定有$a_1<a_2<\dots<a_{k}\leq0$。此时可以验证$a_{k+1} =0$且$a_{k+m}>0(m\in\mathbb{N}^+)$,或$a_{k+m}>0(m\in\mathbb{N})$。从而,可以直接声称$S_n$的最小値$S_k=S_{k+1}=S$或$S_k=S$

当然,也会有人好奇抛物线的对称轴为:
\begin{equation}\label{eq_HsAmPg_12}
x=-\frac{a_1-\frac{d}{2}}{2\cdot\frac{d}{2}}=\frac{1}{2}-\frac{a_1}{d}~.
\end{equation}
为何\autoref{eq_HsAmPg_11} 与\autoref{eq_HsAmPg_12} 在参数上有些许区别,这其实是由差分运算和求导运算的区别引起的,此处只作为提醒不予赘述。
