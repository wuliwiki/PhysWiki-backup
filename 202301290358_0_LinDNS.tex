% Linux DNS 笔记

\begin{itemize}
\item 基本原理参考《30 张图解网络》系列教程
\item \textbf{domain name system (DNS)}
\item \verb|nslookup 域名| 可以查看用于解析域名的服务器, 在 ubuntu22.04 上会返回 \verb|127.0.0.53|。 这实际上是本机的 \verb|systemd-resolved|
\item 操作系统会使用 \verb|/etc/resolv.conf| 文件中可以配置 linux 的 dns 服务器, 但如果用了 \verb|systemd-resolved|, 该文件实际上是一个软链, 自动生成, 不建议修改。
\item 如果没用 \verb|resolved| 的话, 可以在该文件里面添加若干行类似 \verb|nameserver 8.8.8.8| 的设置, 其中 \verb|8.8.8.8| 是 google 提供的 DNS 服务。
\item \verb|dig +trace +nodnssec 域名| 可以看到每一个域名服务器的 ip 和详细信息。 \verb|*.root-servers.net| 是跟域名服务器, \verb|tld| 是\textbf{顶级域名(top level domain)}服务器。 顶级域名就是 \verb|.com|, \verb|.net| 这些
\end{itemize}

\subsection{DNS record 的类型}
\begin{itemize}
\item \textbf{Zone} :DNS zones need an SOA。 a 'zone' is an area of control over namespace. \textsl{A zone can include a single domain name, one domain and many subdomains, or many domain names.} In some cases, 'zone' is essentially equivalent with 'domain,' but this is not always true.
\item \textbf{start of authority (SOA) record}: stores important information about a domain or \textbf{zone}
\item A \textbf{zone file} is a plain text file stored in a DNS server that contains an actual representation of the zone and \textsl{contains all the records for every domain within the zone}.
\item A \textbf{zone serial number} is a version number for the SOA record
\item A DNS zone transfer is the process of sending DNS record data from a primary nameserver to a secondary nameserver.
\item \textbf{A records} map domain names to IPv4 addresses.
\item \textbf{AAAA record} is the IPv6 equivalent of an A record
\item \textbf{CNAME}	Redirects a domain to a different domain
\item \textbf{MX} Provides the domain names of mail servers that receive emails on behalf of a domain
\item \textbf{SOA} Provides important details about a DNS zone; required for every DNS zone
\end{itemize}
