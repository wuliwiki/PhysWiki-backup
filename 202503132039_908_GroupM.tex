% 群(综述)
% license CCBYSA3
% type Wiki

本文根据 CC-BY-SA 协议转载翻译自维基百科\href{https://en.wikipedia.org/wiki/Group_(mathematics)}{相关文章}。

\begin{figure}[ht]
\centering
\includegraphics[width=6cm]{./figures/3d977276c842b7f1.png}
\caption{魔方的操作构成了魔方群。} \label{fig_GroupM_1}
\end{figure}
在数学中,群是一个具有二元运算的集合,并满足以下约束条件:该运算是结合的,它具有单位元,并且集合中的每个元素都有逆元。  

许多数学结构都是具有其他性质的群。例如,整数在加法运算下构成一个无限群,该群由一个称为\textbf{1}的单一元素生成(这些性质以独特的方式刻画了整数)。

群的概念被提出,以统一方式处理许多数学结构,例如数、几何形状和多项式的根。由于群的概念在数学内外的多个领域中无处不在,一些作者将其视为当代数学的核心组织原则之一。  

在几何学中,群自然地出现在对对称性和几何变换的研究中:一个物体的对称性构成一个群,称为该物体的\textbf{对称群},而某种特定类型的变换构成一个更一般的群。\textbf{李群}在几何中的对称群中出现,也出现在粒子物理学的\textbf{标准模型}中。\textbf{庞加莱群}是一个李群,包含狭义相对论中时空的对称性。\textbf{点群}则用于描述分子化学中的对称性。

群的概念起源于对多项式方程的研究,最早由埃瓦里斯特·伽罗瓦在 1830 年代提出,他使用 群(法语:groupe)这一术语来描述方程根的对称群,这一概念如今被称为\textbf{伽罗瓦群}。随着来自数论、几何等其他领域的贡献,群的概念得到了推广,并在1870 年左右被正式确立。\textbf{现代群论}是一个活跃的数学学科,它研究群本身的性质。为了探索群,数学家引入了各种概念,以便将群分解为更小、更易理解的部分,例如子群、商群和单群。除了研究群的抽象性质之外,群论学者还研究群的具体表现方式,包括表示论(即群的表示)和计算群论的方法。对于有限群,已经发展出一整套理论,并最终在2004年完成了有限单群的分类。自 20世纪80年代中期以来,\textbf{几何群论}这一分支迅速发展,它将\textbf{有限生成群}视为几何对象进行研究,成为群论中的一个活跃领域。
\subsection{定义与示例}  
\subsubsection{第一个例子:整数}
一个常见的群是整数集
\[
\mathbb{Z} = \{\ldots, -4, -3, -2, -1, 0, 1, 2, 3, 4, \ldots\}~
\]
配备\textbf{加法运算} \((+)\) 。对于任意两个整数 \(a\) 和 \(b\),它们的和 \(a + b\) 仍然是整数;这个封闭性表明加法是整数集 \(\mathbb{Z}\) 上的一个二元运算。  

整数加法的以下性质构成了群的基本公理,并在下面的定义中得到了推广:  
\begin{itemize}
\item 结合律(Associativity)对于所有整数 \(a, b, c\),有:\((a + b) + c = a + (b + c)\)这意味着,无论是先将 \(a\) 与 \(b\) 相加再加上 \(c\),还是先将 \(b\) 与 \(c\) 相加再加上 \(a\),最终的结果相同。  
\item 单位元(Identity Element)对于任意整数 \(a\),有:\(0 + a = a \quad \text{且} \quad a + 0 = a\) \textbf{0}被称为\textbf{加法的单位元},因为它与任意整数相加都不改变该整数的值。  
\item 逆元(Inverse Element) 对于任意整数 \(a\),存在一个整数 \(b\),使得: \(a + b = 0 \quad \text{且} \quad b + a = 0\)这个整数 \(b\) 称为 \(a\) 的\textbf{加法逆元},通常记作 \(-a\)。  
\end{itemize}
整数集 \(\mathbb{Z}\) 连同加法运算构成了一个数学结构,该结构属于一类具有相似性质的更广泛的代数对象。为了更系统地理解这类结构,下面给出正式的定义。
\subsubsection{定义} 
一个群是一个\textbf{非空集合 }\( G \),配备一个\textbf{二元运算}(在此记作“\( \cdot \)”),该运算将 \( G \) 中的任意两个元素 \( a \) 和 \( b \) 组合,得到仍属于 \( G \) 的元素 \( a \cdot b \)。此外,该运算必须满足以下三个被称为\textbf{群公理}的条件:[5][6][7][a]  

结合律(Associativity) 

对于所有 \( a, b, c \in G \),有:\((a \cdot b) \cdot c = a \cdot (b \cdot c)\)这意味着运算的计算顺序不会影响最终结果。  

单位元(Identity Element) 

存在一个元素 \( e \in G \),使得对于任意 \( a \in G \),有:\(e \cdot a = a \quad \text{且} \quad a \cdot e = a\)该元素 \( e \) 是唯一的(见下文),称为单位元(或中性元)。  

逆元(Inverse Element)

对于 \( G \) 中的每个元素 \( a \),存在一个元素 \( b \in G \),使得:\(a \cdot b = e \quad \text{且} \quad b \cdot a = e\)其中 \( e \) 是单位元。对于每个 \( a \),这个元素 \( b \) 是唯一的(见下文),称为 \( a \) 的逆元,通常记作 \( a^{-1} \)。