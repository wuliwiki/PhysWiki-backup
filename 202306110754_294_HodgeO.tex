% 霍奇星算子
% keys Hodge算子|Hodge 算子|Hodge star operator|星算子|Hodge 星算子|外代数|Grassmann 代数|Exterior algebra|麦克斯韦方程组|对偶

\pentry{外导数\upref{ExtDer},体积形式\upref{VolFom}}

\addTODO{加入目录。}

\subsection{星算子的定义}

考虑 $n$ 维线性空间 $V$ 上的外代数 $\bigwedge V$,我们注意到各阶的外积空间具有明显的对称性:$\opn{dim}\bigwedge^k V=C^k_n=C^{n-k}_n=\opn{dim}\bigwedge^{n-k} V$。这意味着这样的一对空间之间存在线性同构,我们使用星算子 $\star$ 来描述这一同构。


\subsubsection{定义前的准备}


星算子是一个映射,把一个 $\bigwedge^k V$ 中的元素 $\omega$ 映射为一个 $\bigwedge^{n-k} V$ 中的元素 $\star\omega$。为了方便定义星算子,我们要先扩张一下内积的定义:


\begin{definition}{$k$-向量的内积}\label{def_HodgeO_1}

设线性空间上有内积,即对于向量$\alpha$和$\beta$,有内积$\langle \alpha, \beta \rangle$。

给定向量$\alpha_i$和$\beta_j$,构成两个$k$-向量$\omega_\alpha=\alpha_1\wedge \alpha_2\wedge \cdots\wedge \alpha_k$和$\omega_\beta=\beta_1\wedge \beta_2\wedge \cdots\wedge \beta_k$。则这两个$k$-向量的内积定义为
\begin{equation}\label{eq_HodgeO_1}
\langle\omega_\alpha, \omega_\beta\rangle = \det \pmat{\langle \alpha_i, \beta_j \rangle_{i, j=1}^k}~
\end{equation}
即用各$\langle \alpha_i, \beta_j \rangle$构成的方阵的行列式。

\end{definition}

显然,如果有某个$\beta_j$正交于所有$\alpha_i$,那么\autoref{eq_HodgeO_1} 就是$0$。这一点对于霍奇星算子的性质很重要。







\begin{example}{$2$-向量的内积}

设二维欧几里得空间有极坐标的坐标函数$r$和$\theta$,诱导出余切向量的基$\{\dd r, \dd \theta\}$。于是,基向量的内积为
\begin{equation}
\leftgroup{
    \langle\dd r, \dd r\rangle &= 1\\
    \langle\dd r, \dd \theta\rangle &= 0 = \langle\dd \theta, \dd r\rangle\\
    \langle\dd \theta, \dd \theta\rangle &= \frac{1}{r^2}
}~.
\end{equation}

构造$2$-向量$\omega_1=a_1\dd r\wedge \dd \theta$和$\omega_2=a_2\dd r\wedge \dd \theta$,则它们的内积为
\begin{equation}
\ali{
    \langle\omega_1, \omega_2\rangle &= a_1a_2\det\pmat{
        \langle\dd r, \dd r\rangle&\langle\dd r, \dd \theta\rangle\\
        \langle\dd \theta, \dd r\rangle&\langle\dd \theta, \dd \theta\rangle
    }\\
    &= \frac{a_1a_2}{r^2}~,
}
\end{equation}

\end{example}



另一等价所需要的预备知识则是如下定义的复杂指标。

\begin{definition}{复杂指标(正整数指标)}
给定正整数指标集合$\Gamma=\{1, 2, \cdots, n\}$。设$I$是$\Gamma$的$k$次\textbf{有序}子集,即对于一个正整数$k\leq n$,存在一个置换$\sigma\in S_n$,将$I$表示为$(\sigma(1), \sigma(2), \cdots, \sigma(k))$。

再定义$I$的补集$\bar{I}=(\sigma(k+1), \cdots, \sigma(n))$,如果$k=n$则$\bar{I}=\varnothing$。

则光滑函数$\omega_{\sigma(1), \sigma(2), \cdots, \sigma(k)}$表示为$\omega_I$,向量$v^{\sigma(1)}\wedge  v^{\sigma(2)}\wedge \cdots\wedge  v^{\sigma(k)}$表示为$v^I$。

复杂指标也应用爱因斯坦求和约定:$\omega_I v^I$是$I$遍历所有$k$次有序子集、或者特别声明的范围后,所得结果的乘积。
\end{definition}


可以看到,之所以要求$I$是有序的,是因为外积有次序要求。定义中使用正整数是为了方便,实际上复杂指标的概念也可以推广到任意指标集合上。



\begin{example}{复杂指标的一个例子}
设$I$的取值范围为$\{(1, 2), (2, 3), (3, 1)\}$,那么$\omega_{1,2}\dd x^1\wedge  \dd x^2+\omega_{2,3}\dd x^2\wedge  \dd x^3+\omega_{3, 1}\dd x^3\wedge  \dd x^1$表示为$\omega_I\dd x^I$。

特别地,外导数可以表示为
\begin{equation}
\dd (\omega_I\dd x^I)=\dd \omega_I\wedge  \dd x^I~.
\end{equation}
\end{example}


% 为了方便描述星算子的定义,我们先引入一些新的表示方法。

% 选定 $V$ 的基 $\{e_i\}$,那么任意 $\omega\in\bigwedge^k V$ 都可以表示为各 $e_{i_1}\wedge e_{i_2}\wedge \cdots\wedge e_{i_k}$ 的线性组合,因此我们只需要描述 $\stare_{i_1}\wedge e_{i_2}\wedge \cdots\wedge e_{i_k}$ 即可定义星算子。

% 为了方便,我们只考虑 $e_{i_1}\wedge e_{i_2}\wedge \cdots\wedge e_{i_k}$ 中各 $i_{r+1}>i_r$ 的情况,也就是下标顺序排列的情况\footnote{乱序排列的情况无非两种,奇排列和偶排列,根据外代数的定义,前者加上负号即可,后者和顺序排列是相等的。}。规定了只考虑下标顺序排列的情况后,就可以暂时不管顺序的问题,把 $e_{i_1}\wedge e_{i_2}\wedge \cdots\wedge e_{i_k}$ 表示为集合 $\{i_1, i_2, \cdots, i_k\}$。利用这个表达,我们就可以定义 $\star\{i_1, i_2, \cdots i_k\}=\{1, 2, \cdots, n\}-\{i_1, i_2, \cdots, i_k\}$。这样,我们就定义出了 $\bigwedge V$ 中各基向量的星算子了。再加上一条“星算子是线性的”,即 $\star(\sum a_i\omega_i)=\sum a_i\star\omega_i$,就得到星算子的完整定义了:





\subsubsection{星算子的定义}


\begin{definition}{霍奇星算子}\label{def_HodgeO_2}

在$n$维线性空间$V$的外代数$\bigwedge V$上\textbf{任取}$k$-向量$\alpha$和$\beta$,且有标准正交基$\{e_1, e_2, \cdots, e_n\}$。为方便,记$\omega=e_1\wedge e_2\wedge \cdots\wedge e_n$\footnote{如果这里的$V$是流形上的余切空间,则$\omega$就是给定点处的体积形式。}。定义映射$\star:\bigwedge^k V \to \bigwedge^{n-k} V$如下:
\begin{equation}
\alpha\wedge \star\beta = \langle\alpha, \beta\rangle\omega~.
\end{equation}
称$\star$是$\bigwedge V$上的\textbf{霍奇星算子(Hodge star operator或Hodge star)},或\textbf{霍奇对偶(Hodge dual)}。

\end{definition}


使用复杂指标可以得到另一种定义方式:


\begin{theorem}{霍奇星算子(等价定义)}\label{the_HodgeO_1}
设$I$是\autoref{def_HodgeO_1} 中描述的一个复杂指标,$\bar{I}$是其补,$\{e_1, e_2, \cdots, e_n\}$是线性空间的一组\textbf{标准正交基},则
\begin{equation}
\star e_I = (\opn{sign}\sigma) e_{\overline{I}}~.
\end{equation}
\end{theorem}



欧几里得空间中,如果用通常的标准正交坐标系,坐标函数为$x^1, x^2, \cdots, x^n$,那么体积形式为$\dd x^1\wedge  \dd x^2\wedge  \cdots \wedge  \dd x^n$,于是
\begin{equation}\label{eq_HodgeO_2}
\leftgroup{
    \star\dd x^1 \wedge  \dd x^2 &= \dd x^3\wedge  \dd x^4\wedge \cdots\wedge \dd x^n\\
    \star\dd x^1 \wedge  \dd x^3 &= -\dd x^2\wedge  \dd x^4\wedge \cdots\wedge \dd x^n
}~
\end{equation}

如果流形上坐标函数仍记为$x^1, x^2, \cdots, x^n$,而某处的体积形式为$\sqrt{\abs{g}}\dd x^1\wedge  \dd x^2\wedge  \cdots \wedge  \dd x^n$,则\autoref{eq_HodgeO_2} 的例子应变为
\begin{equation}\label{eq_HodgeO_3}
\leftgroup{
    \frac{\star\dd x^1 \wedge  \dd x^2}{\langle\dd x^1\wedge \dd x^2, \dd x^1\wedge  \dd x^2\rangle} &= \sqrt{\abs{g}}\dd x^3\wedge  \dd x^4\wedge \cdots\wedge \dd x^n\\
    \frac{\star\dd x^1 \wedge  \dd x^3}{\langle\dd x^1\wedge \dd x^3, \dd x^1\wedge  \dd x^3\rangle} &= -\sqrt{\abs{g}}\dd x^2\wedge  \dd x^4\wedge \cdots\wedge \dd x^n
}~.
\end{equation}



\subsection{星算子的性质}


\begin{theorem}{线性性}\label{the_HodgeO_2}
设$V$是$n$维线性空间,取$v_i\in\bigwedge^k V$,$a_i$为$V$的基域中的元素,则$\star(a_1v_1+a_2v_2)=a_1\star v_1+a_2\star v_2$。
\end{theorem}

\textbf{证明}:

取$V$的标准正交基$\{e_i\}$,记其体积形式为$\omega$。任取$k$次复杂指标$I$、$I_1$和$I_2$,分别对应置换$\sigma$、$\sigma_1$和$\sigma_2$,则
\begin{equation}
\ali{
    e_{\overline{I}}\wedge \star a_1 e_I &= a_1\langle  e_{\overline{I}}, e_I \rangle \omega\\
    &= e_{\overline{I}}\wedge a_1\star  e_I~.
}
\end{equation}
故
\begin{equation}\label{eq_HodgeO_4}
\star a_1e_I = a_1\star e_I~,
\end{equation}

又因为
\begin{equation}
\ali{
    \langle e_I, (e_{I_1}+e_{I_2}) \rangle \omega &= \langle e_I, e_{I_1} \rangle\omega + \langle e_I, e_{I_2} \rangle\omega\\
}~,
\end{equation}
故得
\begin{equation}\label{eq_HodgeO_5}
e_I\wedge \star(e_{I_1}+e_{I_2}) = e_I\wedge e_{I_1} + e_I\wedge e_{I_2}~.
\end{equation}

综合\autoref{eq_HodgeO_4} 和\autoref{eq_HodgeO_5} ,并考虑到复杂指标选取的任意性,即可得证。


\textbf{证毕}。



\begin{theorem}{对偶}\label{the_HodgeO_3}

设$V$是$n$维线性空间,取$v\in\bigwedge^k V$,则$\star \star v = (-1)^{(n-k)k} v$。

\end{theorem}

\textbf{证明}:

取$V$的标准正交基$\{e_i\}$,任取一个$k$次复杂指标$I$,其对应的置换为$\sigma$,补为$\bar{I}$,则由\autoref{the_HodgeO_1} 和\autoref{the_HodgeO_2} ,
\begin{equation}\label{eq_HodgeO_6}
\ali{
    \star \star e_I &= \star(\opn{sign}\sigma)e_{\overline{I}}\\
    &= (\opn{sign}\sigma)\star e_{\overline{I}}\\~.
}
\end{equation}

复杂指标$\bar{I}$对应的置换$\bar{\sigma}$,相当于进行$\sigma$置换后再把后$n-k$个元素整体挪到前面,也就是进行$\sigma$后再进行$(n-k)k$次\textbf{对换}。因此,$\opn{sign}\bar{\sigma}=(-1)^{(n-k)k}\opn{sign}\sigma$,故将$\star e_{\overline{I}}=\opn{sign}\bar{\sigma}e_I$代入\autoref{eq_HodgeO_6} 后有
\begin{equation}
\ali{
    \star \star e_I &= (\opn{sign}\sigma)\star e_{\overline{I}}\\
    &= (-1)^{(n-k)k}(\opn{sign}\sigma)^2 e_I~.
}
\end{equation}

\textbf{证毕}。

\autoref{the_HodgeO_3} 的例子请参见\autoref{sub_HodgeO_1} 。



\subsection{常见例子}\label{sub_HodgeO_1}


\subsubsection{二维欧几里得空间}


\begin{equation}
\ali{
    \star e_x &= e_y~,\\
    \star e_y &= -e_x~.
}
\end{equation}


\subsubsection{三维欧几里得空间}

\begin{equation}
\leftgroup{
    \star 1 &=\dd x\wedge  \dd y\wedge  \dd z\\
    \star \dd x &= \dd y\wedge  \dd z\\
    \star \dd y &= \dd z\wedge  \dd x\\
    \star \dd z &= \dd x\wedge  \dd y\\
    \star \dd y\wedge  \dd z &= \dd x\\
    \star \dd z\wedge  \dd x &= \dd y\\
    \star \dd x\wedge  \dd y &= \dd z\\
    \star \dd x\wedge  \dd y\wedge  \dd z & = 1
}~
\end{equation}




\subsubsection{四维闵可夫斯基时空}

取通常的参考系$(t, x, y, z)$,注意由于闵可夫斯基度量是$\eta_{\mu\nu}=\opn{diag} (1, -1, -1, -1)$,故这不是一个标准正交基,不适用\autoref{the_HodgeO_1} 。但是可以取体积形式$\omega=\dd t\wedge \dd x\wedge \dd y\wedge \dd z$,适用\autoref{def_HodgeO_2}  。


\begin{equation}\label{eq_HodgeO_7}
\ali{
    \star \dd t &= \dd x\wedge \dd y\wedge \dd z\\
    \star \dd x &= \dd t\wedge \dd y\wedge \dd z\\
    \star \dd y &= -\dd t\wedge \dd x\wedge \dd z\\
    \star \dd z &= \dd t\wedge \dd x\wedge \dd y
}~
\end{equation}

如果取不同的号差,即$\eta_{\mu\nu}=\opn{diag} (-1, 1, 1, 1)$,则\autoref{eq_HodgeO_7} 应变为
\begin{equation}
\ali{
    \star \dd t &= -\dd x\wedge \dd y\wedge \dd z\\
    \star \dd x &= -\dd t\wedge \dd y\wedge \dd z\\
    \star \dd y &= \dd t\wedge \dd x\wedge \dd z\\
    \star \dd z &= -\dd t\wedge \dd x\wedge \dd y
}~
\end{equation}





% \addTODO{应该并入\textbf{麦克斯韦方程组(外微分形式)}\upref{MWEq2},以本词条为预备。}
% \subsection{麦克斯韦方程组的外微分形式:后两个方程\cite{KnotsVol4}}

% 星算子描述的是外积空间中一个非常显眼的对偶,即 $\bigwedge^k V$ 和 $\bigwedge^{n-k} V$ 之间的同构。




