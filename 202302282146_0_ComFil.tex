% 计算机文件

\subsection{计算机文件的构成}
计算机的硬盘中会有不同的\textbf{文件系统(file system)}, 例如 Windows 的 NTFS, MacOS 的 APFS, Linix 的 Ext4 等。 他们在一些特性上有所不同, 本文中不具体讨论, 但要注意若想要移动硬盘支持所有操作系统, 一般推荐在格式化时选择 exFAT 文件系统。

一些几乎所有文件系统都支持的信息:
\begin{itemize}
\item \textbf{文件名(含路径)}: 注意不同的文件系统或操作系统对名字的长度和包含的字符有不同的要求。
\item \textbf{文件内容}: 由若干\textbf{字节(Byte, 缩写 B)}构成, 每个字节有 \textbf{8 个比特(bit, 缩写 b)}, 每个比特就是一个二进制的 0 或 1。 文本文件(如 txt)和二进制文件(如 exe)的不同仅仅在于如何用 0 和 1 编码信息, 并无本质区别。 注意文件名不包含在文件内容中。
\item \textbf{文件大小}: 即上一条中文件内容的大小, 注意文件名的长度不会影响文件大小。 一些常见的单位有 \textbf{kB}(1000 byte), \textbf{MB}(1000 kB), \textbf{GB}(1000 MB), \textbf{TB}(1000 MB)等。 注意严格来说这里都是 1000 而不是 1024(即 $2^{10}$), 为了区分, 后者经常表示为 \textbf{kiB}(1024 byte), \textbf{MiB}(1024 kB), \textbf{GiB}(1024 MiB)等(注意这个定义并不是所有系统都严格遵守,例如 Windows 的文件管理器会把 MiB 显示为 MB 等)。 顺便提一下, 网速的快慢通常用 \textbf{bit/s} 来表示, 即每秒钟传输的比特数。 例如 8Mbit/s 就是 1MB/s。
\item \textbf{修改时间}: 文件被修改的时间。 注意该时间并不是非常可靠, 软件和硬件\footnote{例如硬盘老化出现坏点, 也就是 bit rot, 但出现的概率较小。}上的错误都有可能导致文件内容被改变而修改时间没有更新。 更常见地, 也有可能修改时间更新了但文件内容实际上没有更新(例如打开文件不做任何修改就保存)。 修改时间属于文件的\textbf{元数据(meta data)}, 不储存于\textbf{文件内容}中。 另一个类似的元数据是\textbf{创建时间}, 但并不是所有文件系统都支持, 本文不讨论。 可见如果用户想稳定地记录关于文件的一些时间, 最好记录在\textbf{文件内容}中(例如在 word 文档中手动写下创建和修改时间)而不是依赖元数据。
\end{itemize}
