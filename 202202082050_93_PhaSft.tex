% 相移
% 内反射|外反射|相移

\begin{issues}
\
\end{issues}

\pentry{菲涅尔公式、布儒斯特角\upref{Fresnl}}

当光线以不同入射角入射时,振幅反射系数 $r$ 与振幅透射系数 $t$  可能出现负数情形,其负号的物理意义为电场 $\boldsymbol{E}$ 的相应分量反向,相当于发生 $\pi$ 弧度的相移.相移在光的干涉与衍射中有重要影响.
下面我们分情况讨论相移:
\begin{enumerate}
\item 若 $\boldsymbol{E}$ 平行于入射面,令
$$ r_p =  \dfrac{n_2\cos{\theta_i} - n_1\cos\theta_t}{n_1\cos\theta_t + n_2\cos\theta_i} < 0 $$
化简得
$$ \sin(\theta_i - \theta_t)\cos(\theta_i + \theta_t) > 0$$

对于外反射情形,$n_t > n_i \Rightarrow \theta_i > \theta_t$,则
$$\theta_i + \theta_t > \frac{\pi}{2}$$
以 $\theta_i + \theta_t = \pi/2$为界,由于$\theta_i \propto \theta_t$, 则当$\theta_i > \theta_B$ 时, $\theta_i + \theta_t > \pi/2$. 也即,当入射角大于布儒斯特角时,电场的平行分量发生 $\pi$ 弧度的相移.

对于内反射情形,$ n_t < n_i \Rightarrow \theta_i < \theta_t$,则
$$\theta_i + \theta_t < \frac{\pi}{2}$$
同理可得,当入射角小于布儒斯特角时,电场的平行分量发生 $\pi$ 弧度的相移.

\item 若 $\boldsymbol{E}$ 垂直于入射面,令
$$r_s  = \frac{n_1\cos{\theta_i} - n_2\cos\theta_t}{n_1\cos\theta_i + n_2\cos\theta_t} < 0$$

\end{enumerate}




\end{enumerate}
