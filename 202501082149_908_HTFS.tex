% 普朗克黑体辐射定律(综述)
% license CCBYSA3
% type Wiki

本文根据 CC-BY-SA 协议转载翻译自维基百科\href{https://en.wikipedia.org/wiki/Michael_Faraday}{相关文章}。

\begin{figure}[ht]
\centering
\includegraphics[width=8cm]{./figures/70b29e3c83575e69.png}
\caption{普朗克定律准确描述了黑体辐射。这里展示的是不同温度下的一组曲线。经典(黑色)曲线在高频率(短波长)下与观测到的强度偏离。} \label{fig_HTFS_1}
\end{figure}
在物理学中,普朗克定律(也称为普朗克辐射定律)描述了在给定温度T下,黑体在热平衡状态下发射的电磁辐射的谱密度,当黑体与其环境之间没有物质或能量的净流动时。

在19世纪末,物理学家无法解释为什么已经准确测量的黑体辐射谱在高频率处与现有理论预测的谱有显著的偏离。1900年,德国物理学家马克斯·普朗克通过启发式推导得出了一个公式,解释了观测到的谱,假设在含有黑体辐射的腔体中,假设的电荷振荡器只能以最小增量E改变其能量,该增量与其相关电磁波的频率成正比。虽然普朗克最初认为将能量分成增量的假设只是为了得到正确答案的数学技巧,但其他物理学家,包括阿尔伯特·爱因斯坦,在他的基础上进行了进一步发展,普朗克的洞察力现在被认为对量子理论具有根本性的重要性。