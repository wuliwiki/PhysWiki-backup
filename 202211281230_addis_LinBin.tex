% Linux 的二进制兼容问题

\begin{issues}
\issueDraft
\end{issues}

\pentry{g++ 编译器创建静态和动态链接库\upref{gppLib}}

参考这里的\href{https://stackoverflow.com/questions/20183883/determining-binary-compatibility-under-linux}{回答}.

静态链接的程序一般没什么问题. 但对于动态链接的程序, 可以把所有非基础的 so 依赖库打包到一个文件夹里面并设置 rpath/runpath 或者 LD_LIBRARY_PATH. 但是这里面不能包括一些基础的系统库例如
\begin{itemize}
\item linux-vdso.so.1 直接由内核提供,不存在于文件系统, 见\href{https://unix.stackexchange.com/questions/476971/ldd-shows-no-location-after-arrow-library-does-not-exist-on-system}{这里}.
\item libc.so.6 是 GNU C library, 提供 C 语言支持, 包括 system call 的 wrapper.
\item libm.so.6 是 GNU C library 的一部分, 提供数学函数.
\item libgcc_s.so.1 是 GCC runtime library 例如在 32bit 系统中提供 64bit 整数运算.
\item ld-linux-x86-64.so.2 是 dynamic linker
\item libgomp.so.1
\item libstdc++.so.6 是 C++ 库
\item libgfortran.so.5 是 fortran 库
\item libquadmath.so.0 是和 G++ 一起安装的.
\end{itemize}
