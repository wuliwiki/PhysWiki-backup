% 柯西-施瓦茨不等式(综述)
% license CCBYNCSA3
% type Wiki

本文根据 CC-BY-SA 协议转载翻译自维基百科 \href{https://en.wikipedia.org/wiki/Cauchy\%E2\%80\%93Schwarz_inequality}{相关文章}。

柯西–施瓦茨不等式(也称为柯西–布尼亚科夫斯基–施瓦茨不等式)是对内积空间中两个向量的内积绝对值的一个上界,其上界由这两个向量范数的乘积给出。它被认为是数学中最重要且应用最广泛的不等式之一。

向量的内积可以用于描述有限和(通过有限维向量空间)、无穷级数(通过序列空间中的向量)以及积分(通过希尔伯特空间中的向量)。柯西于1821年首次发表了关于求和形式的不等式。相应的积分形式的不等式由布尼亚科夫斯基于1859年发表,赫尔曼·施瓦茨于1888年发表了积分形式的现代证明。
\subsection{不等式的表述}
柯西–施瓦茨不等式指出,对于内积空间中的任意两个向量$\mathbf{u}$和$\mathbf{v}$,都有:
$$
|\langle \mathbf{u}, \mathbf{v} \rangle|^2 \leq \langle \mathbf{u}, \mathbf{u} \rangle \cdot \langle \mathbf{v}, \mathbf{v} \rangle~
$$
其中,$\langle \cdot , \cdot \rangle$ 表示内积运算。例如,实数或复数的点积就是常见的内积形式。每一个内积都对应一个欧几里得 $\ell_2$ 范数,也叫做“标准范数”或“诱导范数”,记作 $|\mathbf{u}|$,其定义为:
$$
|\mathbf{u}\| := \sqrt{\langle \mathbf{u}, \mathbf{u} \rangle},~
$$
其中 $\langle \mathbf{u}, \mathbf{u} \rangle$ 总是一个非负实数(即使内积是复值的)。对上述不等式两边取平方根,就可以得到柯西–施瓦茨不等式更常见的形式,用范数表示为:
$$
|\langle \mathbf{u}, \mathbf{v} \rangle| \leq |\mathbf{u}| \cdot |\mathbf{v}|~
$$
当且仅当 $\mathbf{u}$ 和 $\mathbf{v}$ 线性相关时,上述不等式取等号。\(^\text{[8][9][10]}\)
\subsection{特殊情形}
\subsubsection{Sedrakyan 引理 —— 正实数情形}
Sedrakyan 不等式,又称为 Bergström 不等式、Engel 形式、Titu 引理(或 T2 引理),表述如下:对于实数 $u_1, u_2, \dots, u_n$ 和正实数 $v_1, v_2, \dots, v_n$,有:
$$
\frac{(u_1 + u_2 + \cdots + u_n)^2}{v_1 + v_2 + \cdots + v_n} \leq \frac{u_1^2}{v_1} + \frac{u_2^2}{v_2} + \cdots + \frac{u_n^2}{v_n},~
$$
或者用求和符号表示为:
$$
\left( \sum_{i=1}^{n} u_i \right)^2 \bigg/ \sum_{i=1}^{n} v_i \leq \sum_{i=1}^{n} \frac{u_i^2}{v_i}.~
$$
这个不等式是柯西–施瓦茨不等式的直接推论,具体地,可以将其看作是在欧几里得空间 $\mathbb{R}^n$ 中对向量点积应用柯西–施瓦茨不等式得到的。

方法是令:
$$
u_i' = \frac{u_i}{\sqrt{v_i}}, \quad v_i' = \sqrt{v_i},~
$$
将其代入向量内积后即可得出上述不等式。这种形式在处理分式型不等式(尤其是分子为完全平方形式)时尤其有用。
\subsubsection{$R^2$ —— 平面}
\begin{figure}[ht]
\centering
\includegraphics[width=8cm]{./figures/099935da4974289a.png}
\caption{欧几里得平面单位圆中的柯西–施瓦茨不等式} \label{fig_KXSW_1}
\end{figure}
实向量空间 $\mathbb{R}^2$ 表示二维平面。它也是二维欧几里得空间,其中的内积就是点积。若$\mathbf{u} = (u_1, u_2),\quad \mathbf{v} = (v_1, v_2)$
则柯西–施瓦茨不等式变为:
$$
\langle \mathbf{u}, \mathbf{v} \rangle^2 = (\|\mathbf{u}\| \|\mathbf{v}\| \cos \theta)^2 \leq \|\mathbf{u}\|^2 \|\mathbf{v}\|^2,~
$$
其中 $\theta$ 是向量 $\mathbf{u}$ 和 $\mathbf{v}$ 之间的夹角。

上述形式也许是最容易理解此不等式的方式,因为余弦的平方最大为 1,当且仅当两个向量方向相同或相反时达到最大值。该不等式也可用向量坐标 $u_1$、$u_2$、$v_1$ 和 $v_2$ 表示为:
$$
(u_1 v_1 + u_2 v_2)^2 \leq (u_1^2 + u_2^2)(v_1^2 + v_2^2),~
$$
其中等号成立的充要条件是向量 $(u_1, u_2)$ 与向量 $(v_1, v_2)$ 共线(同向或反向),或其中一个为零向量。
\subsubsection{$R^n$:n 维欧几里得空间}
在具有标准内积(即点积)的欧几里得空间 $\mathbb{R}^n$ 中,柯西–施瓦茨不等式变为:
$$
\left( \sum_{i=1}^{n} u_i v_i \right)^2 \leq \left( \sum_{i=1}^{n} u_i^2 \right) \left( \sum_{i=1}^{n} v_i^2 \right).~
$$
在此情形下,柯西–施瓦茨不等式可以仅使用初等代数证明,其关键是注意到右侧与左侧之差为:
$$
\frac{1}{2} \sum_{i=1}^{n} \sum_{j=1}^{n} (u_i v_j - u_j v_i)^2 \geq 0,~
$$
或者通过考虑如下关于 $x$ 的二次多项式:
$$
(u_1 x + v_1)^2 + \cdots + (u_n x + v_n)^2 = \left( \sum_i u_i^2 \right) x^2 + 2 \left( \sum_i u_i v_i \right) x + \sum_i v_i^2.~
$$
由于该多项式恒为非负,因此它至多有一个实根,因而其判别式小于等于零,即:
$$
\left( \sum_i u_i v_i \right)^2 - \left( \sum_i u_i^2 \right) \left( \sum_i v_i^2 \right) \leq 0.~
$$
这就得出了柯西–施瓦茨不等式。
\subsubsection{$C^n$:n维复空间}
若 $\mathbf{u}, \mathbf{v} \in \mathbb{C}^n$,其中$\mathbf{u} = (u_1, \ldots, u_n), \quad \mathbf{v} = (v_1, \ldots, v_n)$(其中 $u_1, \ldots, u_n \in \mathbb{C}$,$v_1, \ldots, v_n \in \mathbb{C}$),并且若向量空间 $\mathbb{C}^n$ 上的内积定义为标准复数内积:
$$
\langle \mathbf{u}, \mathbf{v} \rangle := u_1 \overline{v_1} + \cdots + u_n \overline{v_n},~
$$
其中上划线表示复共轭,那么柯西–施瓦茨不等式可以更明确地写为:
$$
\left| \langle \mathbf{u}, \mathbf{v} \rangle \right|^2 = \left| \sum_{k=1}^n u_k \overline{v_k} \right|^2 \leq \langle \mathbf{u}, \mathbf{u} \rangle \cdot \langle \mathbf{v}, \mathbf{v} \rangle = \left( \sum_{k=1}^n u_k \overline{u_k} \right) \left( \sum_{k=1}^n v_k \overline{v_k} \right) = \left( \sum_{j=1}^n |u_j|^2 \right) \left( \sum_{k=1}^n |v_k|^2 \right).~
$$
即:
$$
\left| u_1 \overline{v_1} + \cdots + u_n \overline{v_n} \right|^2 \leq \left( |u_1|^2 + \cdots + |u_n|^2 \right) \left( |v_1|^2 + \cdots + |v_n|^2 \right).~
$$
\subsubsection{$L^2$ 空间}
对于平方可积复值函数所构成的内积空间,有如下不等式成立:
$$
\left| \int_{\mathbb{R}^n} f(x)\, \overline{g(x)}\, dx \right|^2 \leq \left( \int_{\mathbb{R}^n} |f(x)|^2\, dx \right) \left( \int_{\mathbb{R}^n} |g(x)|^2\, dx \right).~
$$
这个不等式是 Hölder 不等式的一个特例。
翻译如下:
\subsection{应用}
\subsubsection{分析学中的应用}
在任意内积空间中,**三角不等式**可以由柯西–施瓦茨不等式推导出来,推导如下:
$$
\|\mathbf{u} + \mathbf{v} \|^2 = \langle \mathbf{u} + \mathbf{v}, \mathbf{u} + \mathbf{v} \rangle
= \|\mathbf{u}\|^2 + \langle \mathbf{u}, \mathbf{v} \rangle + \langle \mathbf{v}, \mathbf{u} \rangle + \|\mathbf{v}\|^2~
$$
由于
$$
\langle \mathbf{v}, \mathbf{u} \rangle = \overline{\langle \mathbf{u}, \mathbf{v} \rangle}~
$$
所以:
$$
\|\mathbf{u} + \mathbf{v} \|^2 = \|\mathbf{u}\|^2 + 2\operatorname{Re} \langle \mathbf{u}, \mathbf{v} \rangle + \|\mathbf{v}\|^2~
$$
应用柯西–施瓦茨不等式(CS)得:
$$
\leq \|\mathbf{u}\|^2 + 2|\langle \mathbf{u}, \mathbf{v} \rangle| + \|\mathbf{v}\|^2 
\leq \|\mathbf{u}\|^2 + 2\|\mathbf{u}\|\|\mathbf{v}\| + \|\mathbf{v}\|^2 
= (\|\mathbf{u}\| + \|\mathbf{v}\|)^2~
$$
对两边取平方根,得到三角不等式:
$$
\|\mathbf{u} + \mathbf{v} \| \leq \|\mathbf{u} \| + \|\mathbf{v} \|~
$$
此外,柯西–施瓦茨不等式还可用于证明:内积是关于其自身诱导的拓扑结构下的连续函数。\(^\text{[11][12]}\)
\subsubsection{几何学}
柯西–施瓦茨不等式使我们能够将“两向量之间的角度”这一概念扩展到任意实内积空间中,其定义为:\(^\text{[13][14]}\)
$$
\cos \theta_{\mathbf{u}\mathbf{v}} = \frac{\langle \mathbf{u}, \mathbf{v} \rangle}{\|\mathbf{u}\| \|\mathbf{v}\|}.~
$$
柯西–施瓦茨不等式证明了该定义是合理的,因为右侧表达式的值始终位于区间 $[ -1, 1 ]$ 之间,这就为我们将(实)Hilbert 空间视为欧几里得空间的推广提供了正当性。
在复内积空间中,也可以使用这个不等式来定义角度,不过需要取右侧表达式的绝对值或实部,\(^\text{[15][16]}\)—— 这种方法在从量子保真度中提取度量时尤其常见。
\subsubsection{概率论}
设 $X$ 和 $Y$ 为随机变量,则协方差不等式为:\(^\text{[17][18]}\)
$$
\operatorname{Var}(X) \geq \frac{\operatorname{Cov}(X, Y)^2}{\operatorname{Var}(Y)}.~
$$
我们可以将随机变量的乘积的期望定义为内积:
$$
\langle X, Y \rangle := \operatorname{E}(XY),~
$$
在这种定义下,柯西–施瓦茨不等式变为:
$$
|\operatorname{E}(XY)|^2 \leq \operatorname{E}(X^2)\operatorname{E}(Y^2).~
$$
为了利用柯西–施瓦茨不等式证明协方差不等式,设:$\mu = \operatorname{E}(X), \quad \nu = \operatorname{E}(Y)$
则有:
$$
\begin{aligned}
|\operatorname{Cov}(X, Y)|^2 
&= |\operatorname{E}[(X - \mu)(Y - \nu)]|^2 \\
&= |\langle X - \mu, Y - \nu \rangle|^2 \\
&\leq \langle X - \mu, X - \mu \rangle \cdot \langle Y - \nu, Y - \nu \rangle \\
&= \operatorname{E}[(X - \mu)^2] \cdot \operatorname{E}[(Y - \nu)^2] \\
&= \operatorname{Var}(X) \cdot \operatorname{Var}(Y),
\end{aligned}~
$$
其中 $\operatorname{Var}$ 表示方差,$\operatorname{Cov}$ 表示协方差。
\subsection{证明}
除了下述方法外,柯西–施瓦茨不等式还有许多不同的证明方式。\(^\text{[19][5][7]}\)在查阅其他资料时,读者常会遇到两个混淆点:第一,部分作者将内积符号 $\langle \cdot, \cdot \rangle$ 定义为对第二个变量是线性的(而非第一个变量);第二,有些证明只在数域为实数 $\mathbb{R}$ 时成立,不适用于复数域 $\mathbb{C}$。\(^\text{[20]}\)本节将给出以下定理的两种证明方式:

\textbf{柯西–施瓦茨不等式设}$\mathbf{u}$ 和 $\mathbf{v}$是任意内积空间中的向量,其标量域为$\mathbb{F}$,其中$\mathbb{F}$ 为实数域 $\mathbb{R}$ 或复数域 $\mathbb{C}$。

则有:
$$
\left| \langle \mathbf{u}, \mathbf{v} \rangle \right| \leq \|\mathbf{u}\| \, \|\mathbf{v}\|~
$$
\textbf{(柯西–施瓦茨不等式)}

当且仅当 $\mathbf{u}$ 和 $\mathbf{v}$ 线性相关时,上述不等式取等号。

进一步地,如果:
$$
\left| \langle \mathbf{u}, \mathbf{v} \rangle \right| = \|\mathbf{u}\| \, \|\mathbf{v}\|~
$$
且 $\mathbf{v} \neq \mathbf{0}$,则:
$$
\mathbf{u} = \frac{ \langle \mathbf{u}, \mathbf{v} \rangle }{ \|\mathbf{v}\|^2 } \, \mathbf{v}.~
$$
在下面所给出的两个证明中,针对至少有一个向量为零(或者等价地,当$\|\mathbf{u}\|\|\mathbf{v}\| = 0$时)的平凡情形,其证明过程是相同的。为了避免重复,下面只给出一次该情形的证明。这一部分也包含了上文中“等号成立的判别条件”中较容易的一部分证明:即证明了如果$\mathbf{u}$ 和 $\mathbf{v}$ 线性相关,则有
$$
\left| \langle \mathbf{u}, \mathbf{v} \rangle \right| = \|\mathbf{u}\| \, \|\mathbf{v}\|.~
$$

平凡部分的证明:向量为$\mathbf{0}$的情形,以及等号成立判别条件中的一个方向的证明:

按定义,若且唯若其中一个向量是另一个的数倍,则$\mathbf{u}$ 和 $\mathbf{v}$ 是线性相关的。若$\mathbf{u} = c\mathbf{v}$,其中$c$ 是某个标量,则有:
$$
|\langle \mathbf{u}, \mathbf{v} \rangle| = |\langle c\mathbf{v}, \mathbf{v} \rangle| = |c\langle \mathbf{v}, \mathbf{v} \rangle| = |c| \|\mathbf{v}\|\|\mathbf{v}\| = \|c\mathbf{v}\|\|\mathbf{v}\| = \|\mathbf{u}\|\|\mathbf{v}\|~
$$
这就表明在柯西–施瓦茨不等式中等号成立。若$\mathbf{v} = c\mathbf{u}$(即$\mathbf{v}$ 是 $\mathbf{u}$ 的倍数),则有:
$$
|\langle \mathbf{u}, \mathbf{v} \rangle| = |\langle \mathbf{v}, \mathbf{u} \rangle| = \|\mathbf{v}\|\|\mathbf{u}\|~
$$
特别地,如果$\mathbf{u}$ 和 $\mathbf{v}$ 中至少有一个是零向量,则二者必然线性相关(例如若
$\mathbf{u} = \mathbf{0}$,则有$\mathbf{u} = c\mathbf{v}$,其中$c = 0$)。因此,上述计算表明,在这种情况下柯西–施瓦茨不等式同样成立。

因此,柯西–施瓦茨不等式只需要对非零向量进行证明,同时仅需证明等号成立条件中的非平凡方向。

\subsubsection{通过毕达哥拉斯定理证明}
特殊情形 $\mathbf{v} = \mathbf{0}$ 已在前文证明,因此下文假设 $\mathbf{v} \neq \mathbf{0}$。设
$$
\mathbf{z} := \mathbf{u} - \frac{\langle \mathbf{u}, \mathbf{v} \rangle}{\langle \mathbf{v}, \mathbf{v} \rangle} \mathbf{v}.~
$$
由内积在第一个参数上的线性性可知:
$$
\langle \mathbf{z}, \mathbf{v} \rangle = \left\langle \mathbf{u} - \frac{\langle \mathbf{u}, \mathbf{v} \rangle}{\langle \mathbf{v}, \mathbf{v} \rangle} \mathbf{v}, \mathbf{v} \right\rangle = \langle \mathbf{u}, \mathbf{v} \rangle - \frac{\langle \mathbf{u}, \mathbf{v} \rangle}{\langle \mathbf{v}, \mathbf{v} \rangle} \langle \mathbf{v}, \mathbf{v} \rangle = 0.~
$$
因此,$\mathbf{z}$ 是一个与 $\mathbf{v}$ 正交的向量(实际上,$\mathbf{z}$ 就是 $\mathbf{u}$ 在与 $\mathbf{v}$ 正交的平面上的投影)。因此,我们可以对下式应用毕达哥拉斯定理:
$$
\mathbf{u} = \frac{\langle \mathbf{u}, \mathbf{v} \rangle}{\langle \mathbf{v}, \mathbf{v} \rangle} \mathbf{v} + \mathbf{z}.~
$$
由此得到:
$$
\|\mathbf{u}\|^{2} = \left| \frac{\langle \mathbf{u}, \mathbf{v} \rangle}{\langle \mathbf{v}, \mathbf{v} \rangle} \right|^{2} \|\mathbf{v}\|^{2} + \|\mathbf{z}\|^{2} = \frac{|\langle \mathbf{u}, \mathbf{v} \rangle|^{2}}{(\|\mathbf{v}\|^{2})^{2}} \cdot \|\mathbf{v}\|^{2} + \|\mathbf{z}\|^{2} = \frac{|\langle \mathbf{u}, \mathbf{v} \rangle|^{2}}{\|\mathbf{v}\|^{2}} + \|\mathbf{z}\|^{2} \geq \frac{|\langle \mathbf{u}, \mathbf{v} \rangle|^{2}}{\|\mathbf{v}\|^{2}}.~
$$
将两边乘以 $\|\mathbf{v}\|^{2}$ 并取平方根,即得出柯西–施瓦茨不等式。
此外,若上式中的不等号实际为等号,则 $\|\mathbf{z}\|^{2} = 0$,因此 $\mathbf{z} = \mathbf{0}$;由 $\mathbf{z}$ 的定义可得出 $\mathbf{u}$ 与 $\mathbf{v}$ 线性相关。
反过来的情况已在本节开头证明,因此证明完成。
\subsubsection{通过分析二次多项式的证明}
考虑任意一对向量 $\mathbf{u}$ 和 $\mathbf{v}$。定义函数 $p : \mathbb{R} \to \mathbb{R}$,其定义为:
$$
p(t) = \langle t \alpha \mathbf{u} + \mathbf{v}, t \alpha \mathbf{u} + \mathbf{v} \rangle~
$$
其中 $\alpha$ 是满足 $|\alpha| = 1$ 且 $\alpha \langle \mathbf{u}, \mathbf{v} \rangle = |\langle \mathbf{u}, \mathbf{v} \rangle|$ 的复数。这样的 $\alpha$ 总是存在,因为当 $\langle \mathbf{u}, \mathbf{v} \rangle = 0$ 时,可以简单地取 $\alpha = 1$。

由于内积是正定的,$p(t)$ 只能取非负实数值。另一方面,利用内积的双线性性质,可以将 $p(t)$ 展开为:
$$
\begin{aligned}
p(t) &= \langle t \alpha \mathbf{u}, t \alpha \mathbf{u} \rangle + \langle t \alpha \mathbf{u}, \mathbf{v} \rangle + \langle \mathbf{v}, t \alpha \mathbf{u} \rangle + \langle \mathbf{v}, \mathbf{v} \rangle \\
&= t \alpha \cdot t \overline{\alpha} \langle \mathbf{u}, \mathbf{u} \rangle + t \alpha \langle \mathbf{u}, \mathbf{v} \rangle + t \overline{\alpha} \langle \mathbf{v}, \mathbf{u} \rangle + \langle \mathbf{v}, \mathbf{v} \rangle \\
&= \|\mathbf{u}\|^2 t^2 + 2 |\langle \mathbf{u}, \mathbf{v} \rangle| t + \|\mathbf{v}\|^2.
\end{aligned}~
$$
因此,$p(t)$ 是一个二次多项式(除非 $\mathbf{u} = 0$,但这种情形已在前面讨论过)。由于 $p(t)$ 恒不小于零,其判别式必须小于等于零:
$$
\Delta = 4 \left( |\langle \mathbf{u}, \mathbf{v} \rangle|^2 - \|\mathbf{u}\|^2 \|\mathbf{v}\|^2 \right) \leq 0.~
$$
由此推出了柯西–施瓦茨不等式。\(^\text{[21]}\)

关于等号成立的情形,当且仅当 $\Delta = 0$ 时成立,此时:$p(t) = \left( t \|\mathbf{u}\| + \|\mathbf{v}\| \right)^2$取 $t_0 = -\|\mathbf{v}\|/\|\mathbf{u}\|$,则有:$p(t_0) = \langle t_0 \alpha \mathbf{u} + \mathbf{v}, t_0 \alpha \mathbf{u} + \mathbf{v} \rangle = 0$从而得到:$\mathbf{v} = -t_0 \alpha \mathbf{u}$也就是说,$\mathbf{u}$ 和 $\mathbf{v}$ 线性相关。
\subsection{推广}
柯西–施瓦茨不等式存在多种推广形式。赫尔德不等式(Hölder's inequality)将其推广到了 $L^p$ 范数的情形。更一般地,它可以被看作是巴拿赫空间上线性算子范数定义的一个特殊情形(即当空间是希尔伯特空间时)。在算子理论的背景下,还有更广泛的推广,比如针对算子凸函数和算子代数的推广,其中定义域和/或值域被替换为 C\*-代数或 W\*-代数。

内积可用来定义正线性泛函。例如,在希尔伯特空间 $L^2(m)$ 上(其中 $m$ 是有限测度),标准内积可导出正泛函 $\varphi$,定义为:$\varphi(g) = \langle g, 1 \rangle$反过来,任意正线性泛函 $\varphi$ 在 $L^2(m)$ 上都可以用来定义内积:$
\langle f, g \rangle_{\varphi} := \varphi(g^* f)$其中 $g^*$ 是 $g$ 的逐点复共轭。在这种语言下,柯西–施瓦茨不等式可以写成\(^\text{[22]}\):
$$
|\varphi(g^* f)|^2 \leq \varphi(f^* f) \, \varphi(g^* g)~
$$
它可以逐字推广到 C\*-代数上的正泛函:

\textbf{C*-代数上正泛函的柯西–施瓦茨不等式}\(^\text{[23][24]}\)——若 $\varphi$ 是 C\*-代数 $A$ 上的正线性泛函,则对所有 $a, b \in A$,有:
$$
\left|\varphi \left(b^{*}a\right)\right|^{2}\leq \varphi \left(b^{*}b\right)\varphi \left(a^{*}a\right).~
$$

下面两个定理是算子代数中的进一步例子:

\textbf{Kadison–Schwarz 不等式}\(^\text{[25][26]}\)(以理查德·卡迪森命名)——若$\varphi$ 是一个单位正映射,则对于其定义域内的任意正规元 $a$,都有:$\varphi (a^{*}a)\geq \varphi \left(a^{*}\right)\varphi (a)$以及
$$
\varphi \left(a^{*}a\right)\geq \varphi (a)\varphi \left(a^{*}\right)~
$$
这推广了如下事实:
$$
\varphi \left(a^{*}a\right)\cdot 1 \geq \varphi (a)^{*}\varphi (a) = |\varphi (a)|^{2}~
$$
当 $\varphi$ 是线性泛函时成立。当 $a$ 是自伴元素(即 $a = a^{*}$)时,这一结果有时被称为\textbf{Kadison不等式}。

\textbf{柯西–施瓦茨不等式(针对2-正映射的修正施瓦茨不等式)}\(^\text{[27]}\) ——对于 C\*-代数间的 2-正映射 $\varphi$,对其定义域中的任意 $a, b$,有:
$$
\varphi (a)^{*} \varphi (a) \leq \Vert \varphi (1) \Vert \cdot \varphi \left(a^{*}a\right)~
$$
以及
$$
\Vert \varphi \left(a^{*}b\right)\Vert ^{2} \leq \Vert \varphi \left(a^{*}a\right)\Vert \cdot \Vert \varphi \left(b^{*}b\right)\Vert.~
$$
另一个推广形式是一种细化形式,通过在柯西–施瓦茨不等式两端之间插值得到:

\textbf{Callebaut 不等式}\(^\text{[28]}\) —— 对于实数 $0 \leq s \leq t \leq 1$,有

$$
\left( \sum_{i=1}^{n} a_i b_i \right)^2 
\leq 
\left( \sum_{i=1}^{n} a_i^{1+s} b_i^{1-s} \right) \left( \sum_{i=1}^{n} a_i^{1-s} b_i^{1+s} \right)
\leq 
\left( \sum_{i=1}^{n} a_i^{1+t} b_i^{1-t} \right) \left( \sum_{i=1}^{n} a_i^{1-t} b_i^{1+t} \right)
\leq 
\left( \sum_{i=1}^{n} a_i^{2} \right) \left( \sum_{i=1}^{n} b_i^{2} \right).~
$$
这个定理可以由 Hölder 不等式 推导而来\(^\text{[29]}\)。同时,也存在适用于算子和矩阵张量积的非交换版本\(^\text{[30]}\)。

柯西–施瓦茨不等式以及 Kantorovich 不等式的若干矩阵形式也被应用于线性回归模型中\(^\text{[31][32]}\)。
\subsection{参见}
\begin{itemize}
\item 贝塞尔不等式—— 关于正交归一序列的定理
\item 赫尔德不等式 —— $L^p$ 空间中积分之间的不等式
\item 詹森不等式 —— 凸函数定理
\item 康托洛维奇不等式
\item 国田–渡边不等式
\item 闵可夫斯基不等式 —— $L^p$ 空间中的三角不等式
\item 帕利–齐格蒙德不等式 —— 数学中的概率不等式
\end{itemize}
\subsection{注释}
\subsection{引文}
\begin{enumerate}
\item O'Connor, J.J.; Robertson, E.F. “赫尔曼·阿曼杜斯·施瓦茨”。苏格兰圣安德鲁斯大学。
\item Bityutskov, V. I. (2001) [1994],《布尼亚科夫斯基不等式》,《数学百科全书》,EMS出版社。
\item Ćurgus, Branko. “柯西-布尼亚科夫斯基-施瓦茨不等式”。数学系,西华盛顿大学。
\item Joyce, David E. “柯西不等式(Cauchy's inequality)”(PDF)。数学与计算机科学系,克拉克大学。PDF档案存档于2022-10-09。
\item Steele, J. Michael (2004)。《柯西–施瓦茨大师课程:数学不等式艺术导论》。美国数学学会,第1页。ISBN 978-0521546775。……毫无疑问,这是整个数学中应用最广泛、最重要的不等式之一。
\item Strang, Gilbert (2005年7月19日)。“3.2”《线性代数及其应用》第4版。康涅狄格州斯坦福:Cengage Learning,第154–155页。ISBN 978-0030105678。
\item Hunter, John K.; Nachtergaele, Bruno (2001)。《应用分析》。世界科学出版社。ISBN 981-02-4191-7。
\item Bachmann, George; Narici, Lawrence; Beckenstein, Edward (2012年12月6日)。《傅里叶与小波分析》。施普林格科学与商业传媒,第14页。ISBN 9781461205050。
\item Hassani, Sadri (1999)。《数学物理学:其基础的现代导论》施普林格出版社,第29页。ISBN 0-387-98579-4。等号成立当且仅当 \<c|c> = 0 或 |c> = 0。根据 |c> 的定义,可得 |a> 与 |b> 必须成比例。
\item Axler, Sheldon (2015)。《线性代数的正确打开方式》第3版。施普林格国际出版,第172页。ISBN 978-3-319-11079-0。该不等式当且仅当 u 与 v 之一为另一个的数量倍时取等号。
\item Bachman, George;Narici, Lawrence(2012-09-26)。《泛函分析(Functional Analysis)》。科瑞尔出版社,第141页。ISBN 9780486136554。
\item Swartz, Charles(1994-02-21)。《测度、积分与函数空间》。世界科学出版社,第236页。ISBN 9789814502511。
\item Ricardo, Henry(2009-10-21)。《线性代数现代导论(A Modern 》。CRC出版社(CRC Press),第18页。ISBN 9781439894613。
\item Banerjee, Sudipto;Roy, Anindya(2014-06-06)。《统计学的线性代数与矩阵分析》。CRC出版社(CRC Press),第181页。ISBN 9781482248241。
\item Valenza, Robert J.(2012-12-06)。《线性代数:抽象数学导论》。施普林格科学与商业传媒,第146页。ISBN 9781461209010。
\item Constantin, Adrian(2016-05-21)。《傅里叶分析及其应用》。剑桥大学出版社,第74页。ISBN 9781107044104。
\item Mukhopadhyay, Nitis(2000-03-22)。《概率与统计推断》。CRC出版社(CRC Press),第150页。ISBN 9780824703790。
\item Keener, Robert W.(2010年9月8日)。《理论统计学:核心课程主题》。施普林格科学与商业媒体,第71页。ISBN 9780387938394。
\item Wu, Hui-Hua;Wu, Shanhe(2009年4月)。《柯西-施瓦茨不等式的多种证明》(PDF)。*八边形数学杂志*,第17卷第1期,第221–229页。ISBN 978-973-88255-5-0。ISSN 1222-5657。存档于2022-10-09(PDF)。检索于2016年5月18日。
\item Aliprantis, Charalambos D.;Border, Kim C.(2007年5月2日)。《无限维分析:搭便车者指南》。施普林格科学与商业媒体。ISBN 9783540326960。
\item Rudin, Walter(1987年)[1966年]。《实分析与复分析》(第3版)。纽约:麦格劳-希尔。ISBN 0070542341。
\item Faria, Edson de;Melo, Welington de(2010年8月12日)。《量子场论的数学方面》。剑桥大学出版社,第273页。ISBN 9781139489805。
\item Lin, Huaxin(2001年1月1日)。《可和C\*-代数分类导论》。世界科学出版社,第27页。ISBN 9789812799883。
\item Arveson, W.(2012年12月6日)。《C\*-代数导论》。施普林格科学与商业媒体,第28页。ISBN 9781461263715。
\item Størmer, Erling(2012年12月13日)。《算子代数的正线性映射》。施普林格数学专著。施普林格科学与商业媒体。ISBN 9783642343698。
\item Kadison, Richard V.(1952年1月1日)。《广义施瓦茨不等式与算子代数的代数不变量》(*A Generalized Schwarz Inequality and Algebraic Invariants for Operator Algebras*)。*数学年刊*(Annals of Mathematics),第56卷第3期,第494–503页。doi:10.2307/1969657。JSTOR 1969657。
\item Paulsen, Vern(2002年)。《完全有界映射与算子代数》。剑桥高级数学研究丛书,第78卷。剑桥大学出版社,第40页。ISBN 9780521816694。
\item Callebaut, D.K.(1965年)。《柯西-施瓦茨不等式的推广》。*数学分析与应用杂志*(J. Math. Anal. Appl.),第12卷第3期,第491–494页。doi:10.1016/0022-247X(65)90016-8。
\item Callebaut不等式。AoPS Wiki词条。
\item Moslehian, M.S.; Matharu, J.S.; Aujla, J.S.(2011年)。《非交换Callebaut不等式》。线性代数及其应用,第436卷第9期,第3347–3353页。arXiv:1112.3003。doi:10.1016/j.laa.2011.11.024。S2CID 119592971。
\item Liu, Shuangzhe;Neudecker, Heinz(1999年)。《柯西-施瓦茨与Kantorovich型矩阵不等式综述》。统计论文,第40卷,第55–73页。doi:10.1007/BF02927110。S2CID 122719088。
\item Liu, Shuangzhe;Trenkler, Götz;Kollo, Tõnu;von Rosen, Dietrich;Baksalary, Oskar Maria(2023年)。《Heinz Neudecker教授与矩阵微分演算》。统计论文,第65卷第4期,第2605–2639页。doi:10.1007/s00362-023-01499-w。S2CID 263661094。
\end{enumerate}
