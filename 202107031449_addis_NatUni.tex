% 自然单位制 普朗克单位制
% 单位制|原子单位制|转换常数

\begin{issues}
\issueDraft
\end{issues}

\pentry{原子单位制\upref{AU}}

\footnote{参考 Wikipedia \href{https://en.wikipedia.org/wiki/Natural_units}{相关页面}.}在高能物理和场论中, 我们往往使用一套单位制, 使得普朗克常数 $\hbar = 1$ 以及真空光速 $c = 1$, 下面我们对这种单位制进行说明.

在 “原子单位制\upref{AU}” 的第一节中, 为了使 $\beta_L = \hbar$ (也就是所谓的 “令 $\hbar = 1$”), 我们得到(\autoref{AU_eq6}~\upref{AU})
\begin{equation}\label{NatUni_eq1}
\beta_t = \frac{\beta_m \beta_x^2}{\hbar}
\end{equation}

现在为了让 $c = 1$, 即规定速度的转换常数为光速 $\beta_v = c$. 如果我们希望满足 $x = vt$, 那么必须有
\begin{equation}
\beta_x = \beta_v \beta _t = c\beta_t
\end{equation}
此时 $\beta_x, \beta_m, \beta_t$ 中只剩一个自由度.

\subsection{普朗克单位制}
\footnote{参考 Wikipedia \href{https://en.wikipedia.org/wiki/Planck_units}{相关页面}.}普朗克单位制中, 所有转换常数都使用 $c, G, \hbar$ 来定义. 例如定义\textbf{普朗克长度(Plank length)}为\footnote{这是 $\hbar, G, c$ 拼凑出长度量纲的唯一组合.}
\begin{equation}
\beta_x = l_p = \sqrt{\frac{\hbar G}{c^3}}
\end{equation}


那么以上所有常数就确定了(\autoref{NatUni_tab1}).

为确定力的量纲, 令万有引力公式为(“令引力常数 $G = 1$”)
\begin{equation}
F = \frac{m_1 m_2}{r^2}
\end{equation}
得
\begin{equation}
\beta_F = G\frac{\beta_m^2}{\beta_x^2}
\end{equation}
令做功公式 $E = Fs$ 成立得
\begin{equation}
\beta_E = \frac{\beta_m^2}{\beta_x}
\end{equation}

\addTODO{以下未完成, 表格都是错的}

\begin{table}[ht]
\caption{普朗克单位制换常数}\label{NatUni_tab1}
\begin{tabular}{|c|c|c|c|}
\hline
物理量 & $\beta$ & 描述 & 数值(国际单位)\\
\hline
\dfracH 长度 $x$ & $\sqrt{\dfrac{\hbar G}{c^3}}$ &  - & $1.616255(18)\e{-35}$ \\
\hline
质量 $m$ & $\sqrt{\dfrac{\hbar c}{G}}$ & - & $2.176434(24)\e{-8}$ \\
\hline
时间 $t$ & $\sqrt{\dfrac{\hbar G}{c^5}}$ & - & $5.391247(60)\e{-44}$ \\
\hline
\dfracH 速度 $v$ & $c$ & 光速 & $299792458$ \\
\hline
力 $F$ & $\dfrac{c^4}{G}$ & - & $1.2103\e{44}$ \\
\hline
\dfracH 能量 $E$ & $\sqrt{\dfrac{c^5\hbar}{G}}$ & - & $1.9561\e9$ \\
\hline
角动量 $L$ & $\hbar$ & 普朗克常数 & $1.054571817646156\e{-34}$ \\
\hline
\end{tabular}
\end{table}

\subsection{电磁常数}

普朗克单位制并不规定电磁学常数, 但我们可以创造一套.

(以下未完成)
\begin{table}[ht]
\caption{电磁学}\label{NatUni_tab2}
\begin{tabular}{|c|c|c|c|}
\hline
电荷 $q$ & $e$ 或 $q_e$ & 电子电荷 & $1.6021766208\e{-19}$\\
\hline
\dfracH 电场强度 $\mathcal{E}$ & $\dfrac{e}{4\pi \epsilon_0 a_0^2}$ & 基态轨道电场强度 & $5.1422067070\e{11}$ \\
\hline
\dfracH 磁感应强度 $B$ & $\dfrac{\hbar}{ea_0^2}$ &  & $2.350517567\e5$\\
\hline
\dfracH 电势 $V$ & $\dfrac{e}{4\pi\epsilon_0 a_0}$ & 基态轨道电势 & $27.211386019$ \\
\hline
\end{tabular}
\end{table}
