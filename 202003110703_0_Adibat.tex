% Adiabatic 笔记
% keys 绝热|量子力学|散射|基底

Multi-channel 散射可以取 diabatic 基底(相当于 $R\to \infty$ 时的状态), 在这种情况下, $R\to\infty$ 时势能矩阵就没有任何 coupling, 而当 $R$ 较小时就会有 coupling.

但如果我们将势能矩阵对角化, 得到的基底就叫做 adiabatic 基底.
\begin{equation}
H \Psi = E \Psi
\end{equation}
\begin{equation}
H = T_s + H_{ad}(\bvec R_s)
\end{equation}
$s$ 角标代表 slow, $ad$ 下标代表 adiabatic.

\begin{equation}
H_{ad} \Phi_\nu = u_\nu (\bvec R_s) \Phi_\nu (\bvec R_s, \Omega_f)
\end{equation}
$f$ 角标代表 fast. 解的时候 $\bvec R_s$ 是常数. $u_\nu$ 包括离散和连续.

波函数表示为
\begin{equation}
\Psi(\bvec R_s, \Omega_f) = \sumint_\nu F_\nu (\bvec R_s) \Phi_\nu(\bvec R_s, \Omega_f)
\end{equation}
带入薛定谔方程得
\begin{equation}\label{Adibat_eq5}
-\frac{1}{2\mu} \qty{[\grad_{\bvec R_s} + \bvec P(\bvec R_s)]^2 + \bvec u} \bvec F = E \bvec F
\end{equation}
其中
\begin{equation}
\bvec P_{i,j} = \mel{\Phi_i}{\grad_{\bvec R_s}}{\Phi_j}
\end{equation}
这里的积分是对 $\Omega_f$ 进行. 得到的是一个矩阵,矩阵元为矢量.

为了明确起见, 我们把\autoref{Adibat_eq5} 记为分量的形式为
\begin{equation}
-\frac{1}{2\mu} \qty[\laplacian_{\bvec R_s} F_i + 2\sum_j ( P_{ij} \grad_{\bvec R_s} F_j) + P_{i,j}^{(2)} F_j] = E F_j
\end{equation}
其中 $\mat P^{(2)}$ 矩阵定义为 $\mel{\Phi_i}{\laplacian_{\bvec R_s}}{\Phi_j}$, 其实就是 $\mat P$ 矩阵的平方.

到此为止所有公式都是精确的.
