% 解三棱锥顶角
% keys 三棱锥|立体几何|球坐标系|矢量|线面角


\pentry{高中立体几何}

\begin{figure}[ht]
\centering
\includegraphics[width=13cm]{./figures/cb303244cc1a1a5c.pdf}
\caption{(左图)已知 $\theta_1$, $\theta_2$ 和 $\phi$ 求 $\alpha$; (右图) 已知 $\theta$ 和 $\phi$ 求 $\gamma$} \label{fig_PrmSol_1}
\end{figure}

我们考虑如\autoref{fig_PrmSol_1} 中三棱锥顶点 $O$ 处的 3 条棱和 3 个面之间的角度关系。 这里涉及了三种角: 两条棱之间的夹角(线线角) $\theta$,棱和面的夹角(线面角) $\gamma$, 以及面和面的夹角(面面角) $\phi$。 对给定的顶点, 每种角都有各有 3 个,共 9 个。 我们先不讨论底面 $ABC$ 的位置如何,如果顶点的三条棱的方向都确定, 我们就说顶点的形状确定。

显然, 我们无需知道所有 9 个角的大小才能确定顶点的形状。 这里给出一个类似于三角形余弦定理的公式, 将线线角和面面角联系起来。 线线角可以类比余弦定理中的边长, 面面角类比余弦定理中的夹角。
\begin{equation}\label{eq_PrmSol_1}
\cos\alpha = \cos\theta_1 \cos\theta_2 + \sin\theta_1 \sin\theta_2 \cos\phi~.
\end{equation}
如果已知该式中的 4 个角中的 3 个, 就可以求出另一个角。 与余弦定理类似, 在求解 $\theta_1$ 或 $\theta_2$ 时可能存在两个解, 也可能无解。 注意如果我们交换 $\theta_1$ 和 $\theta_2$ 的值, 上式仍然满足(这相当于创造一个镜像三棱锥)。

一个关于线面角的常用公式是
\begin{equation}\label{eq_PrmSol_2}
\sin\gamma = \sin\phi\sin\theta_1~,
\end{equation}
其中 $\gamma$ 是线段 $OB$ 和三角形 $OAC$ 的线面角。 虽然乍看之下从该式无法判断出 $\gamma$ 是钝角还是锐角, 但仔细分析可以发现若 $\phi$ 为钝角 $\gamma$ 也一定是钝角。

\begin{example}{已知三个线线角求线面角}
若已知 $\theta_1$, $\theta_2$ 和 $\alpha$, 则
\begin{equation}
\cos\phi = \frac{\cos\alpha - \cos\theta_1 \cos\theta_2}{\sin\theta_1 \sin\theta_2}~.
\end{equation}
\end{example}

\begin{example}{}
\addTODO{图}
若一个正方形, 一个正五边形, 一个正六边形拼在一起, 求正方形和正五边形的共同边与正六边形所成的线面角。

利用\autoref{eq_PrmSol_1} , 令正方形的内角为 $\theta_1 = \pi/2$, 六边形的内角为 $\theta_2 = 2\pi/3$, 五边形的内角为 $\alpha = 3\pi/5$, 代入\autoref{eq_PrmSol_1} 解得正方形与六边形的二面角为 $\phi$ 为
\begin{equation}
\phi = \arccos\frac{\cos\alpha - \cos\theta_1 \cos\theta_2}{\sin\theta_1\sin\theta_2} \approx 1.936\Si{rad} \approx 110.9^\circ~,
\end{equation}

再代入\autoref{eq_PrmSol_2}, $\phi$ 是钝角, 所以 $\gamma$ 也是钝角, 所以
\begin{equation}
\gamma = \pi - \arcsin(\sin\phi \sin\theta_1) \approx 1.9357 \Si{rad} \approx 110.9^\circ~.
\end{equation}
这与 $\phi$ 相同, 这是因为我们有一个正方形。
\end{example}

\subsubsection{角边角}
事实上解三棱锥顶角和解三角形类似, 面面角可以看作三角形的 “角”, 而两条棱之间的角可以看作三角形的 “边”。 例如\autoref{eq_PrmSol_1} 就可以看作 “边角边” 问题 (已知 $\theta_1, \phi, \theta_2$) 求第三边 $\alpha$。


那么我们再来看若已知 “角边角” ($\theta_1, \alpha, \theta_2$) 求下一边 $\beta$(\autoref{fig_PrmSol_2})
\begin{figure}[ht]
\centering
\includegraphics[width=5cm]{./figures/74e12884a3752ea9.png}
\caption{角边角问题} \label{fig_PrmSol_2}
\end{figure}
$\beta$ 满足
\begin{equation}\label{eq_PrmSol_3}
% 已数值验证
(C_1^2 - S_\alpha^2 C_1^2 C_2^2 - C_\alpha^2 C_2^2)T_\beta^2 + 2S_\alpha C_\alpha S_1^2 C_2 T_\beta - S_1^2 S_\alpha^2 = 0~,
\end{equation}
其中 $C,S,T$ 分别代表 $\cos, \sin, \tan$, $C_1 = \cos\theta_1, C_2 = \cos\theta_2$。 该式子可以直接使用两次 “边角边” 公式解出来。 注意上式若存在两个解,必须剔除一个无意义的解。

\subsection{证明}
我们可以用球坐标系来证明\autoref{eq_PrmSol_1}。 令极坐标 $(\theta_1, \phi_1)$ 和 $(\theta_2, \phi_2)$ 代表的两个单位矢量的坐标分别为
\begin{equation}
(\sin\theta_1\cos\phi_1, \sin\theta_1\sin\phi_1, \cos\theta_1)~,
\qquad
(\sin\theta_2\cos\phi_2, \sin\theta_2\sin\phi_2, \cos\theta_2)~,
\end{equation}
\begin{equation}
\cos\alpha = \cos\theta_1 \cos\theta_2 + \sin\theta_1 \sin\theta_2 \cos(\phi_2 - \phi_1)~.
\end{equation}
证毕。

再来证明\autoref{eq_PrmSol_2}, 我们仍然使用球坐标, 将射线 OA 作为极轴, 面 OAC 为 $xz$ 平面, 线面角中“线” 的极坐标为 $(1, \theta, \phi)$, 该点的 $y$ 坐标为 $\sin\theta \sin\phi$ 而 $y = \sin\gamma$, 所以 $\sin\gamma = \sin\theta \sin\phi$。 证毕。