% 2019 年计算机学科专业基础综合全国联考卷
% keys 2019 年计算机学科专业基础综合全国联考卷

\subsection{一、单项选择题}
1~40小题, 每小腿2分, 共80分.下列每题输出的四个选项中,只有一个选项符合试题要求.

1.设$n$是描述问题规模的非负整数,下列程序段的时间复杂度是
\begin{lstlisting}[language=cpp]
x=0;
while(n>=(x+1)*(x+1))
    x=x+1;
\end{lstlisting}
A. $O(logn)$  $\quad$  B.$O(n)$  $\quad$   C.$O(n)$  $\quad$  D.$O(n^2)$

2. 若将一棵树$T$转化为对应的二叉树$BT$,则下列对$BT$的遍历中,其遍历序列与$T$的后根遍历序列相同的是 \\
A.先序遍历  $\quad$  B.中序遍历  $\quad$  C.后序遍历  $\quad$ D.按层遍历

3. 对$n$个互对不相同的符号进行哈夫曼编码.若生成的哈夫曼树共有115个结点,则$n$的值 \\
A.56  $\quad$  B.57  $\quad$  C. 58  $\quad$  D.60

4. 在任意一棵非平衡二叉树(AVL树)$T_1$中,删除某结点$v$之后形成平衡二叉树$T_2$,再将$v$插入$T_2$形成平衡二叉树$T_3$. 下列关于$T_1$与$T_3$的叙述中,正确的是 \\
I .若$v$是$T_1$的叶结点,则$T_1$与$T_3$可能不相同 \\
II .若$v$不是$T_1$的叶结点,则$T_1$与$T_3$定不相同  \\
III.若$v$不是$T_1$的叶结点,则$T_1$与$T_3$一定相同 \\
A.仅I  $\quad$  B.仅H  $\quad$ C.仅I , II  $\quad$  D.仅I 、III

5.下图所示的AOE网表示一项包含8个活动的工程.活动d的最早开始时间和最迟开始时间分别是
\begin{figure}[ht]
\centering
\includegraphics[width=12.5cm]{./figures/CSN19_1.png}
\caption{第5题图} \label{CSN19_fig1}
\end{figure}
A.1  $\quad$  B.2  $\quad$  C.3  $\quad$  D.4

6.用有向无环图描述表达式$(x+y)*(x+y)/x$,需要的顶点个数至少是 \\
A.5  $\quad$  B. 6  $\quad$  C.8  $\quad$  D.9

7.选择一个排序算法时,除算法的时空效率外.下列因素中,还需要考
虑的是
1.数据的规模
II.数据的存储方式
M.算法的稳定性
IV.数据的初始状态
A.仅II
B.仅1、1
C.仅I、M.IV
D. I、I、W、IV
8.现有长度为11且初始为空的散列表HT,散列函数是H(hey)= key
% 7,采用线性探查(线性探测再散列)法解决冲突将 关键字序列
87 ,40,30,6,11,22,98,20依次插人到HT后,HT查找失败的平均查
找长度是
A.4
B.5.25
C.6
D.6.29
9.设主串T=“abaabaabecabaabe" ,模式申S=“abaabe" ,采用KMP算法
进行模式匹配,到匹配成功时为止,在匹配过程中进行的单个字符
间的比较次数是
A.9
B.10
C.12
D.15
10.排序过程中,对尚未确定最终位置的所有元素进行一遍处理称为
一“趟”.下列序列中,不可能是快速排序第二趟结果的是
A. 5,2,16,12,28,60,32,72
B.2,16,5,28,12 ,60,32,72
C. 2,12,16,5,28,32,72, 60
D. 5,2,12,28,16,32,72 ,60 .


11.设外存上有120个初始归并段,进行12路归并时,为实现最佳归
④
g-6
20
并,需要补充的虚段个数是
b-4
c=6
A.1
B.2
C.3
D.4
h=9
12.下列关于冯.诺依曼结构计算机基本思想的叙述中,错误的是
c=8
f=10
A.程序的功能都通过中央处理器执行指令实现
A.3和7
B.12和12
C.12和14.
D.15和15.
B.指令和数据都用二进制表示,形式上无差别
6. 用有向无环图描述表达式(x+y) * ((x+y)/x) ,需要的顶点个数至
C.指令按地址访问,数据都在指令中直接给出
少是
D.程序执行前,指令和数据需预先存放在存储器中
A. 5
B. 6
C.8
D.9
13.考虑以下C语言代码:
7.选择一个排序算法时.除算法的时空效率外,下列因素中,还需要考
unsigned short usi = 65535;
虑的是
short si = usi;
1.数据的规模
II.数据的存储方式
执行上述程序段后,si的值是
m.算法的稳定性
IV.数据的初始状态
A. -1
B. -32767
C. -32768
D. -65535
A.仅M
B.仅1、I
14.下列关于缺页处理的叙述中,错误的是
C.仅1I、山、IV
D. I、I、W、IV
A.缺页是在地址转换时CPU检测到的-种异常
8.现有长度为11且初始为空的散列表HT,散列的数是H(hey)= hey
B.缺页处理由操作系统提供的缺页处理程序来完成
% 7,采用线性探查(线性探测再散列)法解决冲突将关键字序列
c.缺页处理程序根据页故障地址从外存读人所缺失的页
87 ,40 ,30,6,11,22,98 ,20依次插人到HT后,HT查找失败的平均查
D).缺页处理完成后回到发生缺页的指令的下一条指令执行
找长度是
15. 某计算机采用大端方式,按字节编址.某指令中操作数的机器数
A.4
B.5.25
C.6
D.6.29
为1234 FFOOH,该操作数采用基址寻址方式,形式地址(用补码表
9.设主串T=“abaabaabeabaabe",模式申S=" abaabe" ,采用KMP算法
示)为FFI2H,基址寄存器内容为F000000,则该操作数的LSB
进行模式匹配,到匹配成功时为止,在匹配过程中进行的单个字符
(最低有效字节)所在的地址是
间的比较次数是
A. F000FFI2H
B. F000 FF15H
A.9
B.10
C.12
D. 15
C. EFFF FF12H
D. EFFF FFI5H
10.排序过程中,对尚未确定最终位置的所有元索进行一遍处理称为
16.下列有关处理器时钟脉冲信号的叙述中,错误的是
一“趟".下列序列中,不可能是快速排序第二趟结果的是
A.时钟脉冲信号由机器脉冲源发出的脉冲信号经整形和分频后
A.5,2,16,12,28 ,60,32,72
形成
B.2,16,5,28,12 ,60,32,72
B.时钟脉冲信号的宽度称为时钟周期,时钟周期的倒数为机器
C.2,12,16,5,28,32,72,60
主频
D.5,2,12,28,16,32,72 ,60
C.时钟周期以相邻状态单元间组合逻辑电路的最大延迟为基准

确定
D.处理器总是在每来一个时钟脉冲信号时就开始执行一条新的
指令
17.某指令功能为R[r2]←R[r1] +M[ R[ r0]] ,其两个源操作数分别采
用寄存器、寄存器间接寻址方式.对于下列给定部件,该指令在取
数及执行过程中需要用到的是
I.通用寄存器组(GPRs)
JI.算术逻辑单元( ALU)
m.存储器( Memory)
IV.指令译码器(ID)
A.仅I、I
B.仅1、1、I
C.仅I、M、IV
D.仅I、I、IV
18.在采用“取指、译码/取数, 执行、访存, 写回”5段流水线的处理器
中,执行如下指令序列,其中s0、s1.s2.s3和12表示寄存器编号.
1: add s2, sl, s0
. // R[s2]←-R[s1] + R[s0]
I2: load s3, 0(2)
// R[s3]←-M[R[[2] + 0]
I3: add s2, s2 s3
// R[s2]←-R[s2] + R[s3]
I4: store s2, 0(t2)
// M[R[12] + 0]←-R[s2]
下列指令对中,不存在数据冒险的是
A. I1和13
B.12和I3
C.12和14
D.I3和I4
19.假定-台计算机采用3通道存储器总线,配套的内存条型号为
DDR3-1333,即内存条所接插的存储器总线的工作频率为1333
MHz、总线宽度为64位,则存储器总线的总带宽大约是
A. 10.66 GB/s
B. 32 GB/s
C.64 GB/s
D.96 GB/s .
20.下列关于磁盘存储器的叙述中,错误的是
A.磁盘的格式化容量比非格式化容量小
B.扇区中包含数据、地址和校验等信息
C.磁盘存储器的最小读写单位为一个字节
D.磁盘存储器由磁盘控制器磁盘驱动器和盘片组成
21.某设备以中断方式与CPU进行数据交换,CPU主频为1 GHz,设备.
接口中的数据缓冲寄存器为32位,设备的数据传输率为50 kB/s.
若每次中断开销(包括中断响应和中断处理)为1000个时钟周期,则CPU用于该设备输入/输出的时间占整个CPU时间的百分比最
多是
A.1.25%
B.2.5% 
C.5%
D. 12.5%
22.下列关于DMA方式的叙述中,正确的是
IDMA传送前由设备驱动程序设置传送参数
I.数据传送前由DMA控制器请求总线使用权
川.数据传送由DMA控制器直接控制总线完成
IV.DMA传送结束后的处理由中断服务程序完成
A.仅I、I
B.仅1、M、IV
C.仅I、M、IV
D. I、I、I、IV
23.下列关于线程的描述中,错误的是
A.内核级线程的调度由操作系统完成
B.操作系统为每个用户级线程建立一个线程控制块
C.用户级线程间的切换比内核级线程间的切换效率高
D.用户级线程可以在不支持内核级线程的操作系统上实现
24.下列选项中,可能将进程唤醒的事件是
I. I0结束
I.某进程退出临界区
M.当前进程的时间片用完
A.仅I
B.仅M
C.仅I、I
D.I、I .I
25.下列关于系统调用的叙述中,正确的是
I.在执行系统调用服务程序的过程中,CPU处于内核态
I.操作系统通过提供系统调用避免用户程序直接访向外设
M.不同的操作系统为应用程序提供了统一的系统调用接口
V.系统调用是操作系统内核为应用程序提供服务的接口
A.仅I、IV
B.仅I、I
C.仅I、I、IV
D.仅I .M、IV
26.下列选项中,可用于文件系统管理空闲磁盘块的数据结构是
I.位图
1.索引节点
.空闲磁盘块链
IV.文件分配表( FAT)
A.仅I、I
B.仅1、M、IV

C.仅1.M
D.仅1、M、IV
27.系统采用二级反馈队列调度算法进行进程调度.就绪队列Q1采
用时间片轮转调度算法,时间片为10 ms;就绪队列Q2采用短进程
优先调度算法;系统优先调度Q1队列中的进程,当Q1为空时系统
才会调度Q2中的进程;新创建的进程首先进人QI;Q1中的进程
执行一个时间片后,若未结束,则转入Q2.若当前Q1 .Q2为空,系
统依次创建进程PI、P2后即开始进程调度P1、P2 需要的CPU时
间分别为30 ms和20 ms,则进程PI、P2在系统中的平均等待时
间为
A.25 ms
B.20 ms
C.15 ms
D). 10 ms
28.在分段存储管理系统中,用共享段表描述所有被共享的段.若进
程PI和P2共享段S,下列叙述中,错误的是
A.在物理内存中仅保存一份段S的内容
B.段S在PI和P2中应该具有相同的段号
C. PI和P2共享段S在共享段表中的段表项
D. PI和P2都不再使用段S时才回收段S所占的内存空间
29.某系统采用LRU页置换算法和局部置换策略,若系统为进程P预
分配了4个页框,进程P访问页号的序列为0,1,2,7,0,5,3,5,0,
2,7 ,6,则进程访问上述页的过程中,产生页置换的总次数是
A.3
B.4
C.5
D.6
30.下列关于死锁的叙述中,正确的是
1.可以通过剥夺进程资源解除死锁
I.死锁的预防方法能确保系统不发生死锁
m.银行家算法可以判断系统是否处于死锁状态
IV.当系统出现死锁时,必然有两个或两个以上的进程处于阻塞态
A.仅11、I
B.仅I.I、IV
C.仅1 II1.W
D.仅1.M、IV
31某计算机主存按字节编址, 采用二级分页存储管理,地址结构如下
所示
页日来号(10位)
页号(10位)
页内偏移(12位)

虚拟地址2050 1225H对应的页目录号、页号分别是
A.081H ,101H
B.081H. ,401H
C.201H、101H
D.201H、401 H
32.在下列动态分区分配算法中,最容易产生内存碎片的是
A.首次适应算法
B.最坏适应算法
C.最佳适应算法
D.循环首次适应算法
33.0SI参考模型的第5层(自下而上)完成的主要功能是
A.差错控制
B.路由选择
C.会话管理
D.数据表示转换
34.100BaseT快速以太网使用的导向传输介质是
A.双绞线.
B.单模光纤
C.多模光纤 D.同轴电缆
35.对于滑动窗口协议,如果分组序号采用3比特编号,发送窗口大小
为5,则接收窗口最大是
A.2
B.3
C.4
D.5
36.假设--个采用CSMA/CD协议的100Mbps局域网,最小帧长是128
B,则在一个冲突域内两个站点之间的单向传播延时最多是
A. 2.56 μs
B.5.12 μs
C.10.24 μs
D.20.48 μs .
37.若将101.200.16.0/20 划分为5个子网,则可能的最小子网的可分
配IP地址数是
A.126
B.254
C.510
D.1022
38.某客户通过一个TCP连接向服务器发送数据的部分过程如题38
图所示客户在 t.时刻第一次收到确认序列号ack. _seq=100 的段,
并发送序列号seq=100的段,但发生丢失.若TCP支持快速重传,
则客户重新发送seq= 100段的时刻是
A.1,
B2
C. t,
D. t,
39.若主机甲主动发起一个与主机乙的TCP连接,甲,乙选择的初始序
列号分别为2018和2046,则第三次握手TCP段的确认序列号是
A.2018
B.2019
C.2046
D.2047
40.下列关于网络应用模型的叙述中,错误的是
A.在P2P模型中,结点之间具有对等关系
