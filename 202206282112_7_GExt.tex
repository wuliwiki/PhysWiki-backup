% Galois扩域
% keys 伽罗华扩域|伽罗瓦扩域|代数方程|根式解|古典数学难题

\pentry{正规扩张\upref{NomEx}}

完成了对正规扩张和可分扩张的讨论,我们引入极为核心的Galois扩域.在上述讨论中,我们时常涉及域自同构,也看到了域自同构和多项式的根之间对应的关系.对域自同构的结构的研究,将把我们引向著名的古典数学难题“多项式方程的根式解”.



\subsection{Galois扩张和域自同构群}

\begin{definition}{Galois扩张}

若域扩张$\mathbb{K}/\mathbb{F}$是\textbf{正规}且\textbf{可分}的,那么称之为一个\textbf{伽罗华扩张(Galois extension)},或译作\textbf{伽罗瓦扩张}.

此时称$\mathbb{K}$\textbf{在}$\mathbb{F}$\textbf{上是Galois的}($\mathbb{K}$ \textbf{is Galois over} $\mathbb{F}$).

\end{definition}

由于特征为零的域都是完美域,因此对这类域,正规扩张都是Galois扩张.

\begin{example}{}
$\mathbb{Q}(2^{1/3})/\mathbb{Q}$不是Galois扩域,因为它不正规,不包含$\opn{irr}(2^{1/3}, \mathbb{Q})(x)=x^3-2$的两个复数根.
\end{example}

\begin{theorem}{}
有限域都是其素域的Galois扩域.
\end{theorem}

\textbf{证明}:

有限域是其素域的有限扩张,而有限扩张都是代数扩张(\autoref{FldExp_cor1}~\upref{FldExp}).

由于有限域都是完美域(\autoref{SprbEx_cor4}~\upref{SprbEx}),故$\mathbb{Z}_p$的代数扩张都是可分扩张.

又结合\autoref{SpltFd_cor2}~\upref{SpltFd}和\autoref{FntFld_cor1}~\upref{FntFld},可知有限域的都是其素域的\textbf{单}正规扩张.

\textbf{证毕}.


据正规扩张和可分扩张的知识,我们容易得到Galois扩张的几条性质:


\begin{theorem}{}
设$\mathbb{K}/\mathbb{M}/\mathbb{F}$是域扩张链,且$\mathbb{K}/\mathbb{F}$是Galois扩张,那么$\mathbb{K}/\mathbb{M}$也是Galois扩张.
\end{theorem}

证明用可分扩张的继承性\autoref{SprbEx_lem3}~\upref{SprbEx}和正规扩张的继承性\autoref{NomEx_the6}~\upref{NomEx}得到.

\begin{theorem}{}
设$\mathbb{K}/\mathbb{M}/\mathbb{F}$是域扩张链,其中$\mathbb{K}/\mathbb{F}$是Galois扩张,$\mathbb{M}/\mathbb{F}$是正规扩张,那么$\mathbb{M}/\mathbb{F}$是Galois扩张.
\end{theorem}

\textbf{证明}:

由\autoref{SprbEx_def4}~\upref{SprbEx}易得,$\mathbb{K}/\mathbb{F}$是可分扩张且$\mathbb{M}\subseteq\mathbb{K}$,可推得$\mathbb{M}/\mathbb{F}$也是可分扩张.

\textbf{证明}.













