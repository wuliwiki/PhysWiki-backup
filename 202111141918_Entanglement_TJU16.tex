% 天津大学 2016 年考研量子力学
% 考研|天津大学|量子力学|2017

\subsection{30分}
\begin{enumerate}
\item 氢原子波函数为$\varPsi (r,\theta,\varphi)=\sqrt{\frac{1}{A}}[\varphi_{210}(r,\theta,\varphi)+\varphi_{100}(r,\theta,\varphi)]$,写出主量子数n,以及角动量的平方和第三分量的可能值与相对应的几率.
\item 不计自旋,粒子在中心势场中运动的守恒量有哪些?
\item 有两个不对易的厄米算符,判断下列是否为厄米算符?\\
(1)$BA$\\(2)$AB+BA$\\(3)$AB-BA$\\(4)$ABC$\\(5)$i(AB-BA)$
\end{enumerate}
\subsection{30分}
已知谐振子初态的波函数为$\varPsi(x,0)=\frac{1}{\sqrt{2}}[\varPsi_{0}(x)+\varPsi_{1}(x)]$,求任意时刻体系的波函数及$x$的平均值.
\subsection{30分}
已知粒子在$U=\leftgroup{
    & 0, \quad 0<x< \frac{a}{2}\\
    & U_{0}, \quad \frac{a}{2}<x<a\\
    & \infty, \quad x<0,x>a
}$中运动,用微扰法求波函数至一级修正,能量至二级修正.
\subsection{30分}
质量为$M$的粒子在边长为$a$的三维势阱中运动,求体系的能级和波函数,并讨论能级的简并度.
\subsection{30分}
\begin{enumerate}
\item 由三个粒子组成的自旋体系中,哈密顿量为$\hat {H}=a\vec{S}_1\cdot\vec{S}_2+a\vec{S}_2\cdot\vec{S}_3+a\vec{S}_3\cdot\vec{S}_1$
\end{enumerate}