% 含连续态的不含时微扰理论(量子力学)
% license Usr
% type Tutor

\pentry{二阶不含时微扰理论(量子力学)\nref{nod_TIPT2},库仑散射(量子)\nref{nod_CulmWf}}{nod_05a7}

若不含时微扰需要包含连续态(例如氢原子 stark 效应), 那么一阶微扰的\autoref{eq_TIPT_1}~\upref{TIPT} 到\autoref{eq_TIPT_2}~\upref{TIPT} 以及二阶微扰的\autoref{eq_TIPT2_1}~\upref{TIPT2} 中, $\ket{\psi_n^0}$ 仍然是束缚态,$\ket{\psi_n^1}$ 是束缚态和连续态叠加s。 $\bra{\psi_m^0}$ 可以是束缚态或者连续态。

这样的改变对束缚态的 $E_n^1$ 修正没有任何影响,但\autoref{eq_TIPT_5}~\upref{TIPT}和\autoref{eq_TIPT2_2}~\upref{TIPT2} 的求和中需要包含连续态。

对于氢原子来说,包含连续态是很重要的。 若不含, 则 $E_n^2$ 会比实际值偏小!
\begin{equation}
\psi_n^1 = \sum_m^{E_m^0 \ne E_n^0} \frac{\mel{\psi_m^0}{H^1}{\psi_n^0}}{E_n^0 - E_m^0} \psi_m^0
+ \int \frac{\mel{\bvec k}{H^1}{\psi_n^0}}{E_n^0 - E^0(\bvec k)} \ket{\bvec k}\dd{\bvec k}~,
\end{equation}
%
\begin{equation}\label{eq_TIPTc_1}
E_n^2 = \sum_{m}^{E_m\ne E_n} \frac{\abs{\mel{\psi_m^0}{H^1}{\psi_n^0}}^2}{E_n^0-E_m^0}
+ \int \frac{\abs{\mel{\bvec k}{H^1}{\psi_n^0}}^2}{E_n^0-E^0(\bvec k)}\dd{\bvec k}~.
\end{equation}
其中 $\bvec k$ 只是描述连续态的一个参数,且连续态需要满足正交归一。
\begin{equation}
\braket{\bvec k'}{\bvec k} = \delta(\bvec k'-\bvec k)~.
\end{equation}
例如在氢原子的 stark 效应\upref{HStark}中, 这里的 $\ket{\bvec k}$ 可以表示库仑平面波,但实时上用球面波基底更方便。

具体例子见 stark 效应\upref{HStark}。
