% 随机存储器
% keys 随机存储器|随机存取存储器|内存
% license Xiao
% type Tutor

\begin{issues}
\issueDraft
\end{issues}

\subsection{随机存储器(Random Access Memory, RAM)}

随机存储器(Random Access Memory, RAM)是一种可读可写的存储器,其任何一个存储单元都是可以随机存取的,而且存取时间与单元的物理位置无关。

计算机系统的主存储器通常都是采用随机存储器。根据信息存储的物理原理,又可以分为静态随机存储器(SRAM)和动态存储器(DRAM)。

一个SRAM基本单元有0 和 1两个电平稳定状态。SRAM基本单元由两个CMOS反相器组成。两个反相器的输入、输出交叉连接,即第一个反相器的输出连接第二个反相器的输入,第二个反相器的输出连接第一个反相器的输入。这就能实现两个反相器的输出状态的锁定、保存,即储存了1个比特的状态。

\subsection{SRAM}

\begin{figure}[ht]
\centering
\includegraphics[width=6cm]{./figures/87070ace2adbda7b.png}
\caption{六管SRAM存储单元示意图} \label{fig_RAM_3}
\end{figure}


静态随机访问存储器(Static random-access memory,SRAM)


SRAM的Static:只要保持通电,存储的数据就可以恒常保持

但是SRAM用了太多MOS管,而且总有两个MOS管饱和导通,占空间,功耗大。而且地址线多,导致不能做太大。

作为x86等微处理器的缓存(如L1、L2、L3)
作为寄存器


\subsection{DRAM}

通常我们俗称为内存,而我们使用的内存条全称叫做,


\subsection{NVRAM}

我们上述所说的内存介质都需要通电来维持数据,这些存储器都属于易失性存储器。然而NVRAM()



参考文献:
\begin{enumerate}
\item 唐朔飞。 计算机组成原理[M]. 高等教育出版社。 2008
\end{enumerate}
