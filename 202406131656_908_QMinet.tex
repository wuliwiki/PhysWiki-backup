% 量子力学诠释
% license CCBYSA3
% type Wiki

(本文根据 CC-BY-SA 协议转载自原搜狗科学百科对英文维基百科的翻译)

所谓\textbf{量子力学的解释}是试图用数学的方式将量子力学与现实“对应”起来。尽管量子力学在广泛的实验中经受住了严格和极其精确的考验(所有量子力学的预测都与实验相符),但是对于量子力学的解释存在着许多不同的思想流派。这些流派在诸如量子力学是确定的还是随机的、量子力学的哪些量可以被认为是“真实的”、测量的本质是什么等基本问题的观点上有所不同。

尽管进行了近一个世纪的辩论和实验,物理学家和哲学家们还没有就哪种解释最能“代表”现实达成共识。

\subsection{历史}

\begin{figure}[ht]
\centering
\includegraphics[width=6cm]{./figures/3ef8c507a5bfcd47.png}
\caption{薛定谔} \label{fig_QMinet_1}
\end{figure}

\begin{figure}[ht]
\centering
\includegraphics[width=6cm]{./figures/33a6c91157a45fe2.png}
\caption{玻姆} \label{fig_QMinet_2}
\end{figure}

量子力学解释中具有影响力的人物

薛定谔
玻姆
量子物理学家所使用的术语的定义,如\textbf{波函数}和\textbf{矩阵力学},经历了许多变化阶段。例如,埃尔温·薛定谔最初认为电子的波函数是它在空间中的电荷密度,而马克斯·玻恩将波函数的绝对值的平方重新解释为电子在空间中的概率密度。

如尼尔斯·玻尔和维尔纳 ·海森堡这样的早期量子力学先驱的观点,经常被归为“哥本哈根解释”,尽管物理学家和物理史学家认为这个术语掩盖了观点之间的差异。虽然哥本哈根式的想法从未被普遍接受,但对哥本哈根正统观念的挑战在20世纪50年代随着大卫·玻姆的先导波解释和休·艾弗雷特三世的多世界解释的兴起而日益受到关注。[2][3] 此外,试图回避区分解释学派的严苛主义立场受到了可证伪实验的挑战,这些实验可能有朝一日可以利用诸如人工智能测量或量子计算来区分不同的解释学派。[4] [5]

物理学家大卫·梅尔明曾经打趣道:“新的解释学派每年都会出现。但是没有哪个解释学派会消失。”作为20世纪90年代至21世纪初主流观点发展的粗略指南,可以参考一下施洛斯豪尔等人在2011年7月“量子物理和现实本质”会议上的民意调查中收集的“快照”。[6]作者引用了马克斯·泰格马克在1997年8月“量子中的基本问题”会议上进行的一项类似的非正式投票调查。作者的主要结论是,“除了哥本哈根的解释仍然是占据很大优势,在他们的投票中获得最多的选票(42\verb|%| ),多世界解释也得到了显著上升:

“哥本哈根解释在这里仍然是有着巨大的优势,特别是如果我们把它和以此产生相关解释(如基于信息的解释和量子贝叶斯解释)算在一起。在泰格马克的民意测验中,埃弗雷特的解释获得了17\verb|%|的选票,这与我们民意测验中的票数(18\verb|%|)相似。"
值得注意的是,只有克莱默在1986年发表的事务性解释,才为梅克斯·玻恩的断言赋予物理基础,即波函数的绝对平方是概率密度。[7]

\subsection{本质}

或多或少,量子力学的所有解释都有两个特质:

\begin{enumerate}
\item  解释的方式都具有——有一系列的方程和规则,以及通过对其输入初始条件来进行预测——这种\textbf{形式}。
\item 解释都是——有一系列的观察结果,包括通过实证研究获得的观察结果和非正式获得的观察结果,例如人类在这个客观世界中获得的经验——这样的\textbf{唯象学}。
\end{enumerate}

不同的解释有两种类型:

\begin{enumerate}
\item 本体论——主张世界上\textbf{存在}什么东西,例如类别和实体
\item 认知论——主张对世界相关知识的可能性、范围和手段的\textbf{认识}
\end{enumerate}

在科学哲学中,认识与现实的区别被称为\textbf{认知论}与\textbf{本体论}。普遍的规律是结果(认知)的\textbf{规律性},而因果机制可以\textbf{调节}结果(本体)。可以从本体角度解释一个现象,也可以从认知角度解释一个现象。例如,不确定性可能归因于人类观察和感知的局限性(认知),或者也许是一个宇宙中的真实存在\textbf{可能}编码的结果(本体)。混淆认知论和本体论,就像假设一个一般性法则实际上“支配”了结果——但是对于规律的表述却具有因果机制的效应——是一个范畴性错误。

从广义上讲,科学理论可以看作是科学实在论——近乎真实地描述或解释自然世界——或者可以看作为反实在论。实在论的立场寻求认知论和本体论,而反实在论的立场则寻求认知论而非本体论。在20世纪上半叶,反实在论基本上是逻辑实证主义,它试图将科学理论中现实中不可观察的方面排除出去。

自20世纪50年代以来,反实在论变得更加温和,某种程度上可以算是工具论,允许谈论不可观察的方面,但最终抛弃了实在论中的这个关键问题,并把科学理论看作一种帮助人类做出预测工具,而不是获得对世界的形而上学理解。工具论的观点是由大卫·梅明的名言“闭嘴,计算”所承载的,这句名言经常被误认为是理查德·费曼所言。[8]

为了解决一些概念问题,其他解释方法引入了新的数学形式,因此也发展出了其他的理论及其解释。一个例子就是波希米亚力学,它与三种标准形式——薛定谔的波动力学、海森堡的矩阵力学和费曼的路径积分——都是等价的,并且已被验证。

\subsection{挑战}

\begin{enumerate}
\item 抽象,量子场论的数学本质:量子力学的数学结构在数学上是抽象的,没有对其变量的明确定义。
\item 明显不确定和不可逆过程的存在:在经典场论中,在场中给定位置的物理性质很容易推导出来。在量子力学的大多数数学公式中,测量在理论中扮演了不同寻常的角色,因为它是唯一能导致状态不均匀、不可逆演化的过程。
\item 观察者在结果中的决定作用:哥本哈根解释中波函数是一个计算工具,并且仅代表在测量之后那一刻的现实,可以由观察者得到;埃弗里特的解释承认所有的可能性都是真实的,并且可测量相互作用会导致有效的分支过程。[9]
\item 远距离物体之间意想不到的关联:纠缠量子系统,如EPR悖论所示,服从统计却破坏局域因果关系。[10]
\item 描述的互补性:互补原理认为没有哪一组经典物理概念可以同时适用于量子系统的所有性质。例如,波形式的描述A和粒子形式的描述B可以各自描述量子系统S,但不能同时描述。这意味着当使用命题连接词时,S的物理性质的组成不遵守经典命题逻辑的规则。与量子互文性类似,“互补原理的起源在于所描述的量子对象的算符的非对易性”(Omnès 1999) 。
\item 随着系统规模的增加,复杂性迅速上升,远远超过了人类目前的计算能力:因为量子系统的态空间发展到子系统是数量上是指数增长的,所以很难得到经典近似。
\item 系统的定域互文行为:量子互文性证明了系统的物理量有确定的值,这与它们的测量方式无关的这种经典经验直觉即使对定域系统也是失败的。此外,物理原理,如莱布尼茨的不可分辨的同一性原理,也不再适用于量子领域,这说明大多数经典直觉对量子世界可能是不正确的。
\end{enumerate}

\subsection{总结}

\subsubsection{4.1 爱因斯坦采用的分类}
一种解释(即量子力学的数学形式的语义解释)可以按照于它对爱因斯坦提出的某些问题的处理来分类,比如:

\begin{itemize}
\item 实在论
\item 完整性
\item 局域实在论
\item 决定论
\end{itemize}

为了说明这些属性,我们需要更明确地知道一个解释可以给我们展现什么样的图景。为此,我们将把解释视为数学符号\textbf{M}的元素和解释结构\textbf{I}的元素之间的对应关系,其中:

\begin{itemize}
\item 数学符号\textbf{M}包括基向量的希尔伯特空间机器、作用于向量空间的自伴算子、依赖向量的幺正时间和测量算符。在这种情况下,测量算符是将基矢量转换成概率分布的转换)。
\item 解释结构\textbf{I}包括态、态之间的转换、测量算符,以及这些元素的空间扩展的相关信息。测量算符是指可能导致系统态改变的操作并返回一个数值。空间信息将由空间中以函数形式表示的状态来表示。转换可以是非确定性或概率性的,或者可以存在有无限多的态。
\end{itemize}

一个解释的关键方面是确定I中的哪个或哪些元素被认为是物理真实的。

实在论和完备性的当前用法起源于1935年的论文,[11] 爱因斯坦与合作者在该论文中提出了EPR悖论。在那篇论文中,作者提出了\textbf{现实的概念要素}和\textbf{物理理论的完备性}。他们把现实中的要素描述为一个量,在测量或干扰前可以预测出一个确定值,并把一个完备的物理理论定义为理论内每一个物理量都可以被该理论所阐明意义的理论。从解释的语义角度来看,如果解释结构的每一个物理量都可以用数学的形式表示出来,则该解释就是完备的。实在论也是每个量的数学形式的体现;如果一个量对应于解释结构中的一个元素,那么我们称它为实的。例如,在量子力学的一些解释中(如多世界解释),与系统态相关的右矢被认为对应于物理现实的量,而在其他解释中则不是。

决定论是表征态随时间演化的一种性质,即未来某时刻的态是当前态的函数。我们并不总是清楚某一特定解释是否具有确定性,因为它可能不具有显现的时间参数。此外,任一给定的理论都可能有两种解释,一种是确定性的,另一种不是。

定域实在论包含有两个方面:

\begin{itemize}
\item 测量得到的值对应于态空间中某个函数的值。换句话说,这个值对应于现实的量;
\item 测量结果的传播速度不超过某个通用极限(例如光速)。为了使这一点有意义,解释结构中的测量算符必须是定域的。
\end{itemize}

约翰·贝尔根据定域隐变量理论给出了定域实在论的精确表述。

贝尔定理与实验测试相结合,限制了量子理论可以具有的性质,也就是量子力学不能同时满足局域性和反事实确定性原则。

不管爱因斯坦对解释问题的关注多少,狄拉克和其他量子名宿接受了新理论中的技术进步,却很少或根本没有关注解释方面。

\subsubsection{4.2 哥本哈根诠释}

哥本哈根解释是尼尔斯·玻尔和维尔纳·海森堡1927年前后在哥本哈根合作时确定的量子力学“标准”解释。玻尔和海森堡扩展了马克思·玻恩最初提出的波函数的概率的含义。哥本哈根解释拒绝回答“在测量粒子位置之前,粒子在哪里?”毫无意义。测量过程是随机地选取一个态的波函数允许的过程进行测量,得到的结果与理论上得到的概率是一致的。根据哥本哈根解释,量子系统外部的观测者或仪器与系统的相互作用是导致波函数坍塌的原因,因此保罗·戴维斯(Paul Davies)认为,“现实是在观测中,而不是在电子中”。[12] 一般来说,在测量(盖革计数器、火花或气泡室中的轨迹点)后,两者不再相关,除非可以进行后续的实验观测。

\subsubsection{4.3 多世界}

多世界解释是量子力学的另一种解释,在这种解释中,普适的波函数始终遵循相同的确定性、可逆的定律;特别地, (不确定的和不可逆的)波函数坍塌与测量没有关系。与测量相关的现象被认为是由退相干现象解释的,退相干现象是态与环境相互作用产生纠缠,反复将宇宙“分裂”成相互不可观察的另一种历史——在一个更大的多元宇宙中有效的区分不同的宇宙。

\subsubsection{4.4 一致性历史}

一致性历史解释不仅包括了传统的哥本哈根解释, 还试图给出一个关于量子宇宙学更本质的解释。该理论基于一个一致性标准,该标准要求下一个系统的每个历史的概率都服从经典概率的加法规则,并且符合薛定谔方程。

根据这种解释,量子力学理论的目的是预测各种可能的历史(例如,粒子)的相对概率。

\subsubsection{4.5 系综解释}

系综解释,也称为统计解释,可以被视为最低限度的解释。也就是说,它所使用与标准数学相关的假设最少。它最大限度地利用了玻恩的统计解释。解释指出波函数并不适用于单个系统——例如,单个粒子——而是一个抽象的统计量,只适用于大量(数目巨大)相似系统或多粒子体系。这种解释最熟悉的支持者就是爱因斯坦了:

试图将量子理论描述设想为对独立系统的完整描述会导致不自然的理论解释,如果描述是指系综系统而不是独立系统,就不会出现这个问题。

——— 爱因斯坦在《阿尔伯特•爱因斯坦:哲学家-科学家》, P.A. Schilpp编辑 ( 哈珀与罗出版公司, 纽约)
目前最著名的系综解释倡导者是,撰写了研究生教材《\textbf{量子力学——现代发展}》的西蒙·弗雷泽大学教授莱斯利·巴林丁。Akira Tonomura的视频剪辑1中给出了一个演示系综解释的实验。[13] 从单电子双缝实验中可以明显看出,量子力学波函数(绝对值平方)给出了\textbf{完整} 的干涉图样,所以它一定可以用来描述一个系综。Raed Shaiia也引入了一个依赖于概率论的重新表述的系综解释。[14][15]

\subsubsection{4.6 德布罗意-玻姆理论}

德布罗意-玻姆量子力学理论(也称为导频波理论)是由路易·德布罗意开创,后来被大卫·玻姆扩展到包括测量的理论。一个粒子有它自己的位置,但是可以用波函数来描述它。根据薛定谔波动方程得到的波函数,不会坍塌。该理论发生在单一的时空中,是非局域的,具有确定性的。粒子位置和速度的同时确定受测不准原理制约。德布罗意-玻姆被认为是一个隐变量理论,包含非局部性,所以满足贝尔不等式。因为粒子总是有确定的位置,所以测量的问题解决了。[16] 波函数坍塌可以用唯象学来理解 phenomenological。[17]

\subsubsection{4.7 关系量子力学}

遵循狭义相对论的先例,关联量子力学背后的基本思想是,不同的观测者可能对同一系列事件给出不同的解释:例如,对于给定时间点的一个观测者,一个系统可能处于单一的“坍塌”的本征态,而对于同一时间点的另一个观测者,系统可能处于两个或更多的叠加态。因此,如果量子力学要成为一个完整的理论,关联量子力学主张“态”的概念不是描述被观察的系统本身,而是描述系统与其观察者之间的关系或相关性。传统量子力学的状态向量对应于观测器中某些\textbf{自由度}与被观测系统的相关描述。然而,关联量子力学认为这适用于所有物理对象,无论它们是有意识的还是宏观的。任何“测量事件”都被简单地看作是一种普通的物理相互作用,一种上述关联性关系的建立。因此,理论的物理内容与对象本身无关,而与它们之间的关系有关。[18][19]

关联量子力学是与大卫·博姆对狭义相对论的阐释相类比而发展起来的,[20] 在狭义相对论中,探测事件就是建立了量子化场和探测器之间的联系。这也避免了应用海森堡测不准原理产生的固有歧义。[21]

\subsubsection{4.8 交易解释}

量子力学的交易解释(TIQM)是约翰·克莱姆受惠勒-费曼吸收理论启发而发展的量子力学解释。[22] 它描述了波函数的坍塌,这是由于从源到接收器(波函数)的概率波和从接收器到源的概率波(波函数的复共轭)之间的时间对称的交易造成的。这种量子力学解释是独特的,因为它不仅把波函数看作是一个实变量,而且把波函数的复共轭看作是实变量,它出现在玻恩法则中,用来计算一个观测量的期望值。

\subsubsection{4.9 随机力学}

普林斯顿大学教授爱德华·纳尔逊(Edward Nelson)在1966年提出了一种通过类比布朗运动对薛定谔波动方程进行了完全经典的推导和阐释。[23] 类似的想法以前也曾发表过,例如弗罗斯特(1933年)、伊·费奈斯(1952年)和沃尔特·韦泽尔(1953年),纳尔逊的论文中也引用过。帕冯在随机解释方面做了进一步的工作。[24] 鲁门·茨科夫提出了另一种随机解释[25] Roumen Tsekov。

\subsubsection{4.10 客观塌缩理论}

客观崩溃理论不同于哥本哈根解释,在于它将波函数和它所过程都视为本体论上的客观(意味着它们独立于观察者而存在和发生)。在客观理论中,塌缩或者随机发生(“自发定位”),或者当达到某个物理阈值时发生,观察者没有特殊作用。因此,客观塌缩理论是现实的、不确定的、无隐藏变量的理论。标准量子力学没有规定任何坍塌机制;如果客观塌缩是正确的,量子力学则需要进一步扩展。量子力学扩展的要求意味着客观塌缩更多的是一种理论而不是解释。客观塌缩理论包括:

\begin{itemize}
\item 吉拉尔迪-里米尼-韦伯理论[26]
\item 彭罗斯解释;[27]
\item 客观塌缩理论的确定性变形[28]
\end{itemize}

\subsubsection{4.11 意识导致坍缩 (冯·诺依曼 –维格纳解释)}

约翰·冯·诺依曼在他的论文《\textbf{量子力学的数学基础}》中,深入分析了所谓的测量问题。他得出结论,整个物理宇宙可以服从薛定谔方程(广义波函数)。他还描述了测量是如何导致波函数塌缩的。[29] 尤金·维格纳对这一观点进行了进一步的阐述,他认为人类实验者的意识(或者甚至是狗的意识)对塌缩至关重要,但他后来放弃了这一主张。[30][31]

意识导致塌缩的解释包括:

\textbf{主观还原原理}
这个原理,即意识导致塌缩,是量子力学和身心问题的交叉点;研究人员正在努力探索与物理事件相关的意识事件,根据量子理论,这些事件应该包括波函数塌缩;但是,到目前为止,还没有明确结果。 [32] [33]
\textbf{参与式人择原理(PAP)}
约翰·阿奇博尔德·惠勒的参与式人择原理认为意识在宇宙的形成中起着一定的作用。 [34]
其他物理学家也阐述了他们自己对意识引起塌缩的变异的解释;包括:

·        Henry P. Stapp (\textbf{警觉的宇宙:量子力学和它的观察者})

·        Bruce Rosenblum 和 Fred Kuttner (\textbf{量子之谜:物理遭遇意识})

·        Amit Goswami (\textbf{自我觉醒的宇宙})

\subsubsection{4.12 多心灵}

量子力学的多心灵解释扩展了多世界解释,提出世界之间的区别应该在独立观察者的思想层面上进行。

\subsubsection{4.13 量子逻辑}

量子逻辑可以被看作是一种命题逻辑,适合于理解源于量子测量的反常,尤其是那些对于互补变量算符的测量而造成的反常。这个研究领域及其命名起源于加勒特·伯克霍夫和约翰·冯·诺依曼1936年的论文,他试图调和经典布尔逻辑与量子力学中测量和观察相关事实的一些明显不一致之处。

\subsubsection{4.14 量子信息理论}

量子信息方法[35] 已经吸引了越来越多的关注和支持。[36][6] 可以细分为两种[37]

\begin{itemize}
\item 信息本体论,如惠勒的“从比特开始”。这些方法被认为是唯心论的复兴[38]
\item 量子力学解释为观察者对世界的认识,而不是世界本身。这种方法与玻尔的思想有些相似。[39] 塌缩(也称为还原)通常被解释为观察者从测量中获取信息,而不是客观事件。这些方法类似于工具论。
\end{itemize}

态不是一个独立系统的客观属性,而是从系统如何组成中获得的信息,可用于预测未来的测量结果。量子力学的态是观察者关于独立的物理系统的信息的总结,它既通过动力学规律变化获得,也通过测量过程获得关于系统的新信息。态向量的演化有两个规律,只有当认为状态向量是系统的客观属性时,才会有问题。“波包的衰减”确实发生在观察者的意识中,不是因为在那里发生了的任何独特的物理过程,而是仅仅因为态是观察者的构造,而不是物理系统的客观属性。

\subsubsection{4.15 量子理论的模态解释}

量子力学的模态解释最初是由范·弗拉森在1972年的论文《科学哲学的形式方法》中提出的然而,这个术语现在被用来描述由这种方法产生的更大的模态集。斯坦福哲学百科全书描述了几个版本:[40]

\begin{itemize}
\item 哥本哈根变体
\item 科钦-迪克斯-希利解释
\item 早期模态解释的灵感来源于克利夫顿、迪克森和布博的工作。
\end{itemize}

\subsubsection{4.16 时间对称理论}

一些理论提出修改量子力学的方程来符合时间反演对称性。[41][42][43][44][45][46] 这就产生了追溯因果关系:未来的事件会影响过去的事件,就像过去的事件会影响未来的事件一样。在这些理论中,一次测量不能完全确定一个系统的态(使它们成为一种隐藏变量理论),但是给定在不同时间执行的两次测量,就有可能计算出两个时间点之间的任意时间系统的精确状态。因此波函数的塌缩不是系统的物理变化,而是由于第二次测量,我们对它的认识发生了变化。同样,他们解释纠缠也不是一种真实的物理状态,而是一种通过忽略追溯因果关系而产生的幻觉。两个粒子看起来“纠缠”的点仅仅是每个粒子受到另一个粒子未来发生的事件影响的点。

并非所有时间对称因果关系的倡导者都赞成修改标准量子力学的幺正动力学。因此,两态向量模式的主要倡导者列夫·瓦伊德曼强调了两态向量模式与休·埃弗雷特的多世界解释是高度吻合的。[47]

\subsubsection{4.17 分支时空理论}

分支时理论类似于多世界的解释;然而,“主要的区别在于,分支时空解释将历史的分支作为一系列事件及其因果关系的拓扑体现,而不是态向量的不同分量单独进化的结果。”[48] 在分支时空理论中的时空拓扑分支就是多世界解释中波函数。分支时空理论在贝尔定理、量子计算和量子引力中都有应用。它也与隐变量理论和系综解释有一些相似之处:虽然分支时空理论中的粒子在微观层面上可以有多个明确的定义。但是在粗粒级别的水平上只能随机处理,这与系综解释相符。[48]

\subsubsection{4.18 其他解释}

除了上面讨论的主流解释学派之外,还提出了许多其他解释学派,这些学派无论出于何种原因都没有产生重大的科学影响。这其中包括从主流物理学家的提议到更神秘的量子神秘主义思想的各种流派都有。

\subsection{对照}

下表总结了最常见的解释。表格单元格中显示的结果并非没有争议,因为所涉及的一些概念的确切含义并不清楚,事实上,它们本身就是围绕给定解释的争议的中心。关于另一个比较量子理论解释的表格,见参考文献。[49]

没有实验证据可以区分这些解释。在这种程度上,物理理论是站得住脚的,并且与自身和现实相一致;只有当一个人试图“解释”这个理论时,困难才会出现。然而,设计能分辨出不同解释的实验是研究热点。.

这些解释大多有变体。例如,哥本哈根解释很难有精确的定义,因为它是由许多有不同理解的人发展起来的。

\begin{table}[ht]
\centering
\caption\label{tab_QMinte}
\begin{tabular}{|c|c|c|c|c|c|c|c|c|c|c|c|}
\hline
\textbf{年份}& \textbf{作者}& \textbf{确定性?} & \textbf{实波函数?} & \textbf{唯一的历史?} & \textbf{隐变量?} & \textbf{波函数坍塌?}&\textbf{观察者的角色?}&\textbf{局域动力学?}&\textbf{反事实确定性原则?}&\textsl{显现的广义波函数?} \\
\hline
$(1 + x)^{1/x}$ & 2.59374 & 2.70481 & 2.71692 & 2.71815 & 2.71827 & 2.71828 \\
\hline
\end{tabular}
\end{table}