% 积性函数
% keys 数论函数|积性函数
% license Xiao
% type Tutor

\pentry{数论函数\upref{NumFun}}

\begin{issues}
\issueAbstract
\end{issues}

每当我们提出一个概念,我们都需要考察它的性质,对可能的例子进行分类。数论函数中有着很重要的两大类——积性函数和加性函数。最基本的数论函数基本都可以归为这两类。由于积性函数和加性函数之间存在着简单的转化关系,而积性函数更方便我们处理,因此积性函数相比下要重要得多。

什么是积性函数?先从一个比较自然的例子开始。

\begin{example}{除数函数的积性}

我们很容易算出除数函数$d(n)$当$n$从$1$到$20$的值,分别是1,2,2,3,1,4,2,4,3,4,2,6,2,4,4,5,2,6,2,6.

可以发现这些恒等式:
\being{equation}
d(6)=d(2)d(3),d(12)=d(3)d(4),d(17)=d(1)d(17),\cdots
\end{equation}

你可能会猜测
\being{equation}\label{chuji}
d(xy)=d(x)d(y)
\end{equation}

但是:
\being{equation}
d(4)\neq d(2)d(2),d(12)\neq d(2)d(6), \cdots
\end{equation}

实际上,\autoref{eq_MulFun_2}中$x,y$互素时才成立。这可以通过$d(n)$的下列计算式说明:

\being{equation}
d(n)=\prod_{p^{k_p}\| n}(1+\k_p)
\end{equation}

其中$p^{k_p}\| n$表示$n$的质因数分解中$p$的指标为$k_p$。该式可由唯一分解和乘法原理很快得到。

\end{example}

于是,我们抽象出下列定义:

\begin{definition}{积性函数}
如果数论函数$f(n)$使
\begin{equation}
f(ab)=f(a)f(b),(a,b)=1~.
\end{equation}
恒成立,则称$f(n)$是\textbf{积性的}。
\end{definition}

\begin{example}{}
$I(n),u(n),e(n),d(n),\sigma(n),\mu(n),\varphi(n),\lambda(n)$都是积性的。
\end{example}

\begin{definition}{完全积性函数}
如果数论函数$f(n)$使
\begin{equation}
f(ab)=f(a)f(b)~.
\end{equation}
恒成立,则称$f(n)$是完全积性的。
\end{definition}

\begin{example}{}
$I(n),u(n),e(n),\lambda(n)$都是完全积性的。
\end{example}

\begin{theorem}{积性函数的性质}
$f(n)$是积性的,则:
\begin{enumerate}
\item $f(1)=1$.
\item $f((a,b))f([a,b])=f(a)f(b)$.
\item 函数$F(n)=\sum\limits_{d|n}f(d)$也是积性的。
\end{enumerate}
\end{theorem}

证明留给读者。
