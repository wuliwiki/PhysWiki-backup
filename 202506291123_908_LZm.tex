% 莉泽·迈特纳(综述)
% license CCBYSA3
% type Wiki

本文根据 CC-BY-SA 协议转载翻译自维基百科\href{https://en.wikipedia.org/wiki/Lise_Meitner}{相关文章}。

\begin{figure}[ht]
\centering
\includegraphics[width=6cm]{./figures/03b50d02f58c4e7f.png}
\caption{} \label{fig_LZm_1}
\end{figure}
伊莉莎·“莉泽”·迈特纳(Elise "Lise" Meitner,/ˈmaɪtnər/,德语:[ˈliːzə ˈmaɪtnɐ] ⓘ,1878年11月7日-1968年10月27日)是一位奥地利-瑞典核物理学家,在核裂变的发现中发挥了关键作用。

1906年完成博士研究后,迈特纳成为维也纳大学第二位获得物理学博士学位的女性。她的科研生涯大部分时间在柏林度过,在那里她曾任教于凯撒·威廉化学研究所,担任物理学教授及系主任。她是德国第一位晋升为物理学正教授的女性。由于纳粹德国反犹太的纽伦堡法案,她在1935年失去了职位,而1938年的奥地利并入德意志第三帝国导致她失去奥地利国籍。1938年7月13日至14日,在德克·科斯特的帮助下,她逃往荷兰。之后她在斯德哥尔摩生活多年,并于1949年成为瑞典公民,但在1950年代迁往英国,与家人团聚。

1938年年中,凯撒·威廉化学研究所的化学家奥托·哈恩和弗里茨·施特拉斯曼证明,通过中子轰击铀可以形成钡的同位素。哈恩将他们的发现告知了迈特纳,1938年12月底,迈特纳与她的侄子、同为物理学家的奥托·罗伯特·弗里施一起,通过对哈恩和施特拉斯曼实验数据的正确解释,阐明了这一过程的物理机制。1939年1月13日,弗里施重复了哈恩和施特拉斯曼观察到的过程。在1939年2月的《自然》杂志上,迈特纳和弗里施发表报告,将这一过程命名为“裂变”。核裂变的发现推动了二战期间核反应堆和原子弹的发展。

迈特纳并未与长期合作伙伴奥托·哈恩共享1944年因核裂变而授予的诺贝尔化学奖,许多科学家和记者称她未获奖“不公”。根据诺贝尔奖档案,1924年至1948年间,她曾19次被提名诺贝尔化学奖,1937年至1967年间曾30次被提名诺贝尔物理学奖。尽管未获得诺贝尔奖,迈特纳在1962年仍受邀出席林岛诺贝尔奖获得者大会。她还获得了许多其他荣誉,包括1997年以她命名的元素109号“镆”。爱因斯坦曾称赞迈特纳为“德国的居里夫人”。
\subsection{早年}
1878年11月,伊莉莎·迈特纳出生于维也纳利奥波德城区凯撒约瑟夫大街27号的一个犹太中上层家庭,是国际象棋大师菲利普·迈特纳和其妻子赫德维希的八个孩子中的第三个。维也纳犹太社区的出生登记记录显示她的出生日期为1878年11月17日,但所有其他文件均记载为11月7日,她本人也使用这个日期。\(^\text{[4]}\)

她的父亲是奥地利首批获得执业资格的犹太律师之一。\(^\text{[3]}\)她有两个姐姐吉塞拉和奥古斯特(Auguste,昵称Gusti),以及四个弟妹莫里茨(Moriz,昵称Fritz)、卡萝拉(Carola,昵称Lola)、弗里达和瓦尔特;他们最终都接受了高等教育。\(^\text{[5]}\)她的父亲是自由思想者,她也是在这种环境中成长起来的。\(^\text{[3]}\)

成年后,她皈依基督教信仰路德宗,并于1908年受洗;\(^\text{[6][7]}\)同年,她的姐姐吉塞拉和卡萝拉也皈依了天主教。\(^\text{[7]}\)此后她也采用了缩短后的名字“莉泽(Lise)”。\(^\text{[8]}\)

莉泽·迈特纳终生未婚,她专注于自己的科研事业,并且没有任何长期的恋爱关系。
\subsection{教育}
迈特纳对科学的兴趣始于八岁时,当时她在枕头下放着一本记录自己科学研究的笔记本。她对数学和科学十分着迷,曾研究过油膜的颜色、薄膜以及反射光。那时女性唯一可从事的职业是教师,因此她就读于一所女子中学,接受法语教师培训。除了法语外,她的课程还包括簿记、算术、历史、地理、科学和体操。她于1892年完成中学学业。

当时维也纳的公立高等教育机构不允许女性入学,直到1897年这一限制才被解除。解除后,女性不再需要完成文理中学教育,只需通过中学毕业考试即可获得大学入学资格。她的姐姐吉塞拉于1900年通过了matura考试并进入医学院学习。迈特纳于1899年开始与另外两名年轻女性一起接受私人辅导,用两年时间补上了中学缺失的课程,物理课程由阿尔图尔·萨尔瓦西教授授课。

1901年7月,她们在维也纳学术中学参加了外部matura考试。在14名女性考生中,有4人通过了考试,其中包括迈特纳和物理学家路德维希·玻尔兹曼的女儿亨丽埃特·玻尔兹曼。\(^\text{[9][10]}\)
\subsection{研究、工作与学术生涯}
\subsubsection{维也纳大学}
\begin{figure}[ht]
\centering
\includegraphics[width=6cm]{./figures/cc6280eed9d898b6.png}
\caption{1906年的迈特纳} \label{fig_LZm_2}
\end{figure}
梅特纳于 1901 年 10 月进入维也纳大学。\(^\text{[11]}\)她深受路德维希·玻尔兹曼的启发,经常热情地谈论他的讲座。\(^\text{[12]}\)她的博士论文由弗朗茨·埃克斯纳及其助手汉斯·本多夫指导。\(^\text{[13]}\)她于 1905 年 11 月 20 日提交论文,并于 11 月 28 日获得批准。她于 12 月 19 日通过了埃克斯纳和玻尔兹曼的口试,\(^\text{[14]}\) 并于 1906 年 2 月 1 日获得博士学位。\(^\text{[15]}\)她成为维也纳大学第二位获得物理学博士学位的女性,仅次于 1903 年获得博士学位的奥尔加·施泰因德勒;\(^\text{[16]}\)第三位是与梅特纳在同一实验室工作的塞尔玛·弗洛伊德,她在 1906 年获得学位。\(^\text{[14]}\)梅特纳的论文于 1906 年 2 月 22 日以《不均匀体内的热传导》发表。\(^\text{[14][17]}\)

保罗·埃伦费斯特曾请她研究一篇由雷利勋爵撰写的光学文章,文章中描述的一项实验产生了雷利无法解释的结果。她成功解释了这些结果,并基于她的解释提出了预测,随后通过实验证实了这些预测,展现了她独立且无需监督开展研究的能力。\(^\text{[18]}\)她在《由菲涅尔反射公式得出的一些结论》报告中发表了这些研究结果。\(^\text{[19]}\)1906 年,在进行这项研究的同时,斯特凡·迈耶向她介绍了当时仍是新兴领域的放射性研究。她从研究$\alpha$粒子开始。在利用准直器和金属箔进行的实验中,她发现$\alpha$粒子束的散射会随金属原子的质量增加而增强。她于 1907 年 6 月 29 日将研究结果提交给《物理学杂志》。这一实验是促使欧内斯特·卢瑟福提出原子核模型的实验之一。\(^\text{[18][20]}\)
\subsubsection{弗里德里希·威廉大学}
\begin{figure}[ht]
\centering
\includegraphics[width=10cm]{./figures/30c842c5fe9feb08.png}
\caption{丽泽·迈特纳与奥托·哈恩,摄于1912年} \label{fig_LZm_3}
\end{figure}
在父亲的经济支持和鼓励下,迈特纳进入了柏林的弗里德里希·威廉大学(即后来的柏林大学),在那里著名物理学家马克斯·普朗克授课。普朗克邀请她到家中做客,并允许她旁听他的课程。这在当时是不寻常的举动,因为普朗克曾公开反对女性进入大学,但显然他认为迈特纳是个例外。\(^\text{[21]}\)她与普朗克的双胞胎女儿艾玛和格蕾特成为朋友,这对姐妹出生于1889年,与迈特纳一样热爱音乐。\(^\text{[22][23]}\)

旁听普朗克的课程并未占据迈特纳所有时间,她便去找实验物理研究所的负责人海因里希·鲁本斯,询问是否能做一些研究。鲁本斯表示非常欢迎她在实验室工作,并补充说化学研究所的奥托·哈恩正在寻找一位物理学家进行合作。几分钟后,她便被介绍给了哈恩。哈恩曾在威廉·拉姆齐和欧内斯特·卢瑟福门下研究放射性物质,且已经因发现当时被认为是几种新的放射性元素而获得认可。\(^\text{[24][25][a] }\)哈恩与迈特纳同龄,她注意到他态度随和、平易近人。\(^\text{[24][25]}\)在蒙特利尔期间,哈恩已经习惯与物理学家合作——其中包括至少一位女性物理学家哈丽雅特·布鲁克斯。\(^\text{[27]}\)
\begin{figure}[ht]
\centering
\includegraphics[width=6cm]{./figures/8b9dd47196e9ab14.png}
\caption{梅特纳和哈恩在他们的实验室(1913年)。当一位她不认识的同事说他们以前见过面时,梅特纳回答:“您大概是把我误认成哈恩教授了。”\(^\text{[28]}\)} \label{fig_LZm_4}
\end{figure}
化学研究所所长埃米尔·费舍尔将位于地下室的一间原木工坊提供给哈恩使用,作为实验室。哈恩用电离计装备了实验室,以测量$\alpha$粒子、$\beta$粒子和$\gamma$射线。然而在这间木工坊内无法进行真正的研究,后来无机化学系主任阿尔弗雷德·施托克允许哈恩使用他私人实验室的一个空间。\(^\text{[29]}\)与迈特纳一样,哈恩当时也没有薪水,靠父亲提供的津贴生活,虽然他的津贴比迈特纳的稍高一些。1907年初,哈恩完成了他的讲师资格论文,成为了一名私人讲师。\(^\text{[30]}\)化学研究所的大多数有机化学家并不认为哈恩的工作——通过放射性检测肉眼看不见、无法称量或闻到的微量同位素——是真正的化学。\(^\text{[25]}\)一位系主任曾评论道:“如今能成为私人讲师的人,真是令人难以置信!”\(^\text{[25]}\)丽泽·迈特纳曾协助发现放射性元素镤。\(^\text{[31]}\)

最初,这种安排对迈特纳来说非常艰难。当时,德国普鲁士王国(包括柏林)尚未允许女性进入大学。迈特纳只能在带有独立外门的木工坊工作,无法进入研究所其他区域,包括哈恩在楼上的实验室。如果她需要上厕所,只能到街道尽头的餐馆使用卫生间。第二年,普鲁士大学开始允许女性入学,费舍尔取消了限制,并在大楼内安装了女性厕所。然而,并非所有化学家都对此表示欢迎。\(^\text{[27]}\)相较之下,物理研究所更具接纳性,她在那里结识了许多物理学家朋友,包括奥托·冯·拜尔 [de]、詹姆斯·弗兰克、古斯塔夫·赫兹、罗伯特·波尔、马克斯·普朗克、彼得·普林斯海姆 [de] 和威廉·韦斯特法尔。\(^\text{[32]}\)

在最初与哈恩合作的几年中,迈特纳与他共同发表了九篇论文:1908 年发表了三篇,1909 年发表了六篇。她与哈恩一起发现并发展了一种被称为放射性反冲的物理分离方法,即在衰变瞬间,子核会因反冲而被强力排出。哈恩更关注于发现新的元素(如今已知是同位素),而迈特纳则更感兴趣于理解它们的辐射特性。她注意到,由哈丽特·布鲁克斯在 1904 年发现的放射性反冲,可能成为检测放射性物质的一种新方法。不久,他们便发现了两个新的同位素:铋-211 和铊-207。\(^\text{[33][34][35]}\)

迈特纳尤其对$\beta$粒子(当时已知是电子)感兴趣。$\alpha$粒子的发射具有特征能量,她预期$\beta$粒子也应如此。哈恩和迈特纳仔细测量了$\beta$粒子在铝中的吸收情况,但结果令人困惑。1914 年,詹姆斯·查德威克发现从原子核发射出的电子形成了连续谱,而迈特纳难以相信这一点,因为这似乎与量子物理学相矛盾,后者认为原子内的电子只能占据离散的能量状态(量子)。\(^\text{[36]}\)
\subsubsection{凯撒·威廉姆化学研究所}
\begin{figure}[ht]
\centering
\includegraphics[width=10cm]{./figures/d5c5f07f595ea2c0.png}
\caption{1920年柏林的物理学家和化学家合影。前排(从左到右):赫尔塔·斯波纳、阿尔伯特·爱因斯坦、英格丽德·弗兰克、詹姆斯·弗兰克、丽泽·梅特纳、弗里茨·哈伯和奥托·哈恩。后排(从左到右):瓦尔特·格罗特里安、威廉·韦斯特法尔、奥托·冯·拜尔、彼得·普林斯海姆和古斯塔夫·赫兹。} \label{fig_LZm_5}
\end{figure}
1912年,哈恩和梅特纳搬到了新成立的柏林凯撒·威廉化学研究所(KWI)。哈恩接受了费舍尔的邀请,担任该所放射化学部门的初级助理,负责德国第一个此类实验室的管理。这份工作附带“教授”头衔,年薪5,000马克(相当于2021年的29,000欧元)。与大学不同,私人资助的凯撒·威廉研究所没有排斥女性的政策,但梅特纳以“访客”身份在哈恩的部门无偿工作。\(^\text{[37][38]}\)

1910年她父亲去世后,梅特纳可能曾面临经济困难。普朗克担心她可能会返回维也纳,因此任命她为弗里德里希·威廉大学理论物理研究所的助理,作为助理,她需要批改学生的试卷。这是她的第一份有薪职位。助理是学术阶梯的最低一级,而梅特纳是普鲁士第一位女性科研助理。\(^\text{[37][28]}\)

1912年10月23日凯撒·威廉化学研究所(KWI)正式开幕时,官员们自豪地向德皇威廉二世介绍了梅特纳。\(^\text{[39]}\)次年,她和哈恩一样成为“成员”(尽管她的薪水仍然较低),\(^\text{[38]}\)放射性部门被命名为“哈恩-梅特纳实验室”。梅特纳在阿德隆大酒店举办了庆祝晚宴。
不久,哈恩和梅特纳的薪水便被医用中钍(镭-228,又称“德国镭”)生产带来的专利使用费远远超过:1914年,哈恩从中获得66,000马克(相当于2021年的369,000欧元),他将其中的10\%分给了梅特纳。\(^\text{[40]}\)1914年,梅特纳获得奥匈帝国当时属地布拉格的一份教职邀请。普朗克明确告知费舍尔,他不希望梅特纳离开,于是费舍尔安排将她的薪水加倍至3,000马克(相当于2021年的17,000欧元)。\(^\text{[41]}\)

搬迁至新实验室非常幸运,因为原先的木工车间已被溢出的放射性液体和释放后沉积为放射性尘埃的放射性气体严重污染,导致无法进行精密测量。为确保新实验室保持洁净,哈恩和梅特纳制定了严格的操作规程:化学测量与物理测量分室进行,处理放射性物质的人员必须遵守不握手等安全规范,每部电话和门把手旁都挂上了厕纸以擦拭残留物。高放射性物质则存放在旧木工车间,后期又迁至研究所场地内专门建造的镭库中。\(^\text{[41]}\)
