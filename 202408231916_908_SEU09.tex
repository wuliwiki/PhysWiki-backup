% 东南大学 2009 年 考研 量子力学
% license Usr
% type Note

\textbf{声明}:“该内容来源于网络公开资料,不保证真实性,如有侵权请联系管理员”

\textbf{1.(15 分)}以下叙述是否正确:(1)在定态下,任意不是含的力学量的平均值均不随时间变化:(2)若厄密算符与对易,则它们必有共同本征态:(3)一维谐振子的所有能级均是非简并的:(4)厄密算符的本征值必为正数(5)时间反演对称性导致能量守恒

\textbf{2.(15 分)}质量为 $m$ 的粒子作一维运动,几率守恒定理为
\[
\partial \rho/\partial t + \partial j/\partial x = 0,~
\]
其中,$\rho(x,t) = |\psi|^2$, $j(x,t) = -\frac{\hbar}{2m}\left(\psi^* \frac{\partial \psi}{\partial x} - \psi \frac{\partial \psi^*}{\partial x}\right)$。

\begin{enumerate}
    \item 若粒子处于定态 $\psi = \phi(x) \exp(-iEt/\hbar)$,试证 $j = c$(与 $x,t$ 无关的常数);
    \item 若自由粒子处于动量本征态 $\psi(x,t) = \exp(ipx/\hbar - iEt/\hbar)$,试证 $j = p/m$。
\end{enumerate}
