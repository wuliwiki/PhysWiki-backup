% 混合熵的微观解释
% license CCBY4
% type Tutor


\pentry{理想气体混合的熵变\upref{IGME},熵的微观定义与玻尔兹曼公式\upref{entro2}}

\footnote{参考书目:Gaskel et al. Thermodynamics of Materials}

\subsection{从微观角度论证混合熵的公式}
\subsubsection{异种粒子}
在理想气体混合的熵变\upref{IGME}我们已经知道了如何从经典的方法计算混合熵:
\begin{equation}
\Delta_{mix} S = - R \sum n_i \ln x_i~,
\end{equation}
现在,我们尝试用统计力学的观点,从微观角度论证混合熵的公式为什么是这样的。

想象我们有两种不同的球:$N_A$个$A$球,$N_B$个$B$球,现在将他们混合在一起,那么会有几种可能的混合方式?幸运的是,统计力学假设所有微观粒子都是相同的,因此我们不需要考虑球的摆放顺序。那么,问题简化为\textsl{高中二年级}的组合问题,比如说:
\begin{itemize}
\item 共有$(N_A+N_B)$个格子,选择其中$N_A$个格子放置$A$球,其余$N_B$个格子放置$B$球,那么一共有几种选法?
\item 共有$(N_A+N_B)$个空白球,选择其中$N_A$个球刷成$A$球,其余$N_B$个球刷成$B$球,那么一共有几种选法?
\end{itemize}
问题的答案比高中课本上的大多数习题都简单,就是一个组合数\footnote{如果你忘了组合数公式:$C^n_m = \frac{m!}{n!(m-n)!}$}:
\begin{equation}
\Omega = C^{N_A}_{N_A+N_B} C^{N_B}_{N_B} =  C^{N_A}_{N_A+N_B}  = \frac{(N_A+N_B)!}{N_A!N_B!}~,
\end{equation}
取对数并运用妇孺皆知的斯特林近似\footnote{如果你忘了斯特林近似:$\lim_{N \to +\infty} \ln N! = N (\ln N -1)$}
\begin{equation}
\begin{aligned}
\ln \Omega &= \ln \frac{(N_A+N_B)!}{N_A!N_B!} \\
 &= (N_A+N_B) (\ln (N_A+N_B) - 1) - N_A (\ln N_A - 1) - N_B (\ln N_B - 1) \\
 &= (N_A+N_B) \ln (N_A+N_B) - N_A \ln N_A - N_B \ln N_B \\
 &= N_A \ln \frac{N_A+N_B}{N_A} + N_B \ln \frac{N_A+N_B}{N_B} \\
 &= - N_A \ln x_A - N_B \ln x_B \qquad \left [x_A = \frac{N_A}{N_A+N_B}, x_B = \frac{N_B}{N_A+N_B} \right ]\\
\end{aligned}~,
\end{equation}
因此根据Boltzmann公式,
\begin{equation}
\Delta_{mix} S = k \ln \Omega = - k(N_A \ln x_A + N_B \ln x_B) = - R(n_A \ln x_A + n_B \ln x_B) ~.
\end{equation}
有点眼熟?那就对了:这就是混合熵。至此,我们从熵的微观含义论证了混合熵公式。($n_A$是以摩尔数计的$N_A$。)

\subsubsection{同种粒子}
现在,我们想一想如果$A$球和$B$球其实是相同的一种球时,会发生什么。你可能还会说,$\Omega = C^{N_A}_{N_A+N_B}$。但是,从问题的本质来看,不管你如何选取空格放球,最终我们只能把所有的格子放满同一种球。因此,我们实则只有一种方法:
\begin{equation}
\Omega = 1~,
\end{equation}
与
\begin{equation}
\Delta_{mix} S = 0~.
\end{equation}
这还是符合我们以前的论断。

\subsection{为什么混合是自发过程?}
要回答这个问题,我们先理解什么叫“混合”。比如说,我们认为所有$A$球都处于左侧格子、所有$B$球都处于右侧格子的情况叫做“不混合”,其余所有的情况都叫做(某种程度上的)“混合”。

按照这个描述,要使$A$、$B$球不混合,我们只能选取左边的所有格子放置$A$球,因此只有一种选法:$\Omega_{unmixed} = 1$;而剩下的所有放置方式都将使$A$、$B$球混合,选法种数是:$\Omega_{mixed} =  C^{N_A}_{N_A+N_B} - 1$;

当球很多时,$\Omega_{mixed} = C^{N_A}_{N_A+N_B} - 1$将\textsl{远远远远远远远远大于}$\Omega_{unmixed} = 1$,从而根据等概率假设,如果允许自由放置$A$、$B$球,那么系统大概一定会处于混合的状态。
