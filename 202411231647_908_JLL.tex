% 伽利略·伽利莱(综述)
% license CCBYSA3
% type Wiki

本文根据 CC-BY-SA 协议转载翻译自维基百科\href{https://en.wikipedia.org/wiki/Galileo_Galilei}{相关文章}。

伽利略·迪·文琴佐·博纳尤蒂·德·伽利略(\textbf{Galileo di Vincenzo Bonaiuti de' Galilei},1564年2月15日-1642年1月8日),通常简称为伽利略·伽利莱(\textbf{Galileo Galilei},/ˌɡælɪˈleɪoʊ ˌɡælɪˈleɪ/,美国英语中也读作 /ˌɡælɪˈliːoʊ -/;意大利语:[ɡaliˈlɛːo ɡaliˈlɛːi]),或单名称为伽利略(\textbf{Galileo}),是一位佛罗伦萨的天文学家、物理学家和工程师,有时被描述为“博学多才之人”。他出生于当时属于佛罗伦萨公国的比萨市。伽利略被誉为观测天文学之父、现代经典物理学之父、科学方法之父以及现代科学之父。

伽利略研究了速度与加速度、重力与自由落体、相对性原理、惯性、抛体运动等,并在应用科学和技术领域开展了工作,描述了摆的特性和“静水天平”。他是文艺复兴早期温度计(即热测量仪)的开发者之一,还发明了多种军用罗盘。通过他改进的望远镜,伽利略观测到了银河的恒星、金星的位相、木星的四大卫星、土星的光环、月球的陨石坑以及太阳黑子。他还制作了一种早期显微镜。

伽利略对哥白尼日心说的支持遭到了天主教会内部和一些天文学家的反对。1615年,罗马宗教裁判所对这一问题进行了调查,得出结论认为他的观点与当时普遍接受的《圣经》解释相矛盾。[9][10][11]

后来,伽利略在《两大世界体系的对话》(1632年)中为自己的观点辩护,但书中似乎对教皇乌尔班八世进行了攻击和嘲讽,这使伽利略疏远了教皇及之前一直大力支持他的耶稣会士。[9] 他因此受到宗教裁判所的审判,被认定为“严重可疑的异端”,并被迫公开认错。伽利略此后被软禁在家中度过余生。[12][13] 在此期间,他撰写了《两种新科学》(1638年),主要探讨了运动学和材料强度问题。[14]