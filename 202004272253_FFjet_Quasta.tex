% 准静态过程

\subsection{平衡态}

在不受外界影响的条件下,系统的宏观性质不随时间变化的状态,称为\textbf{平衡态}(equilibrium state), 否则就是非平衡念态(nonequilibrium state).

具体来说,一定质量的气体在一容器内,如果它与外界没有交换能量,没有外场作用,内部也没有任何形式的能量转化,经过一段时间后,气体各部分终将达到相同的密度、相同的温度、压强等,所有的宏观性质都不随时间而变化,这种状态就是平衡态.

在实际情况中,并不存在完全不受外界影响、而且宏观性质绝对保持不变的系统,所以平衡态只是一个理想的模型,它是在一定条件下对实际情况的概括和抽象.

应当指出,平衡态是指系统的宏观性质不随时间变化;从微观方面看,组成系统的分子的热运动是永不停息的,通过分子的热运动的相互碰撞,其总效果在宏观上表现为不随时间变化的,所以平衡态实际上是\textbf{热动平衡状态}(thermodynamical equilibrium state).

来思考这样一个例子:如果将一根金属棒的两端分别放在沸水和冰水混合物中,经过一段时间后,虽然棒上各处的温度不随时间变化,但这种状态仍不是平衡态,而是\textbf{定常态}(steady state),因为金属棒与外界有能量交换.

\subsection{准静态过程}

当气体的外界条件发生改变时,它的状态就会发生变化.气体从一个状态不断地变化到另一状态,所经历的是一个状态变化的过程.过程进展的速度可以很快,也可以很慢,实际过程通常比较复杂.如果过程进展得十分缓慢,使所经历的一系列中间状态都无限接近平衡状态,这个过程就叫做\textbf{准静态过程}(quasi-static process)或\textbf{平衡过程}(equilibrium process).

显然,准静态过程是个理想的过程,它和实际过程毕竟是有差别的,但在许多情况下,可近似地把实际过程当作准静态过程处理,所以准静态过程是个很有用的理想模型.