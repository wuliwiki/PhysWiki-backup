% 仿射空间中的曲线坐标系
% keys 曲线坐标系|仿射空间

\begin{issues}
\issueDraft
\end{issues}

\pentry{仿射空间\upref{AfSp}}
\subsection{曲线坐标系}
设 $\{\dot o;e_1,\cdots,e_n\}$ 是仿射空间 $\mathbb A$ 中的坐标系.则对其上任一点 $M$ ,其坐标为矢量 $\overrightarrow{OM}=x^i e_i$ 的坐标(采取坐标和基矢指标的对立约定\upref{CofTen}及爱因斯坦求和约定\upref{EinSum}).若选择一新坐标系 $\{\dot o';e'_1,\cdots,e'_n\}$,则点的坐标变换规则为(\autoref{AfSp_the3}~\upref{AfSp})
\begin{equation}
x'^i={B^i}_jx^j-{B^i}_j b^j
\end{equation}
其中,${B^i}_j$ 是基 $\{e_1,\cdots,e_n\}$ 到基 $\{e'_1,\cdots,e'_n\}$ 的转换矩阵 ${A^i}_j$ 的逆矩阵.而 $b^i$ 是 $\dot o'$ 在旧坐标系 $\{\dot o;e_1,\cdots,e_n\}$ 中的坐标.

到目前为止,点的描述都是在仿射坐标(即对应基底 $\{e_i\}$ 的坐标)下进行的,下面将建立曲线坐标的概念来描述点的坐标.为描述方便,先引进仿射空间中领域和 $n$ 维区域的概念.
\begin{definition}{领域,$n$ 维区域}
设 $\mathbb A$ 是 $n$ 维仿射空间,点 $M=x^ie_i\in\mathbb A$ 的 $\delta$ \textbf{领域}是指坐标满足
\begin{equation}
\sum_{i}^n(x'^i-x^i)^2<\delta^2
\end{equation}
的所有点 $M'=x'^i e_i$ 构成的集合.

若 $\Omega$ 是 $\mathbb A$ 中这样一个集合:对 $\Omega$ 中任一点 $M$,必有 $M$ 的某个领域也属于 $\Omega$.则称 $\Omega$ 是 $\mathbb A$ 中的\textbf{ $n$ 维区域}.若区域 $\Omega$ 中任一点都可连续的变动到 $\Omega$ 中的另一点,即一点的坐标连续变化到另一点的坐标,则区域 $\Omega$ 称为\textbf{连通的}.
\end{definition}

凡提到区域总是指连通区域.

\begin{definition}{曲线坐标}
设 $\Omega\in\mathbb A$ 是一 $n$ 维连通区域,在其上给定仿射坐标的 $n$ 个连续可微的单值函数 $f_k(x^1,\cdots,x^n)$ ($k=1,\cdots,n$),且函数组 $\{f_i\}$ 是可逆的.则新定义的变量
\begin{equation}
x'^i=f_i(x^1,\cdots,x^n),\quad i=1,\cdots,n
\end{equation}
称为 $\Omega$ 上的\textbf{曲线坐标}.
\end{definition}

\textbf{注:}函数组 $\{f_i\}$ 的可逆性意味着
\begin{equation}\label{CFinAf_eq2}
x^i=g_i(x'^1,\cdots,x'^n),\quad i=1,\cdots,n
\end{equation}
且 $g_i$ 也是连续可微的单值函数,且函数组 $\{g_i\}$ 也是可逆的.所谓的\textbf{连续可微},是指具有直到某一阶数 $N$ 的连续偏导数.通常并不预先说明具体的 $N$ 值,而是简单地写出已知阶的导数就表明假定这些导数的存在和连续.
\begin{theorem}{}\label{CFinAf_the1}
设
\begin{equation}
x'^i=f_i(x^1,\cdots,x^n),\quad i=1,\cdots,n
\end{equation}
是区域 $\Omega$ 的曲线坐标,则正逆变换的雅可比行列式均不为0:
\begin{equation}\label{CFinAf_eq1}
\det\abs{\pdv{x'^i}{x^j}}\neq0;\quad\det\abs{\pdv{x^i}{x'^j}}\neq0
\end{equation}
且 $\pdv{x'^i}{x^j}$ 和 $\pdv{x^i}{x'^j}$ 对应的矩阵是互逆的.
\end{theorem}
\textbf{证明:}由于定义曲线坐标的函数组 $\{f_i\}$ 是可逆的,即变量 $\{x'^i\}$ 和 $\{x^i\}$ 之间可相互单值的表示,那么可把 $x^j$ 看成 $\{x^i\}$ 的复合函数.于是:
\begin{equation}
\pdv{x^i}{x^j}=\pdv{x^i}{x'^k}\pdv{x'^k}{x^j}
\end{equation}
由于自变量的导数 $\pdv{x^i}{x^j}=\delta^i_j$,那么
\begin{equation}
\pdv{x^i}{x^j}=\pdv{x^i}{x'^k}\pdv{x'^k}{x^j}=\delta^i_j
\end{equation}
即 $\pdv{x^i}{x'^k}$ 和 $\pdv{x'^k}{x^j}$ 对应矩阵的积是一个单位矩阵,所以它们对应的矩阵是互逆的,因而对应行列式不为0,即\autoref{CFinAf_eq1} 是成立的.

 
\textbf{证毕!}

利用 \autoref{CFinAf_eq2} 用曲线坐标 $x'^i$ 来代替仿射坐标 $x^i$,点 $M$ 的向径 $\overrightarrow {OM}=x^i e_i$ 可表达成
\begin{equation}\label{CFinAf_eq3}
\overrightarrow{OM}=g_i(x'^1,\cdots,x'^n)e_i
\end{equation}
将 $\overrightarrow{OM}$ 简记为 $x$,则可写为
\begin{equation}
x=x(x'^1,\cdots,x'^n)
\end{equation}

下面的定理在以后都起着重要的作用.
\begin{theorem}{}\label{CFinAf_the2}
所有偏导数 $\pdv{x}{x'^i},i=1,\cdots,n$ 在每一点都是线性无关的矢量.
\end{theorem}
\textbf{证明:}\autoref{CFinAf_eq3} 关于 $x'^i$ 微分,得
\begin{equation}
\pdv{x}{x'^i}=\pdv{x^j}{x'^i}e_j
\end{equation}
由\autoref{CFinAf_the1} ,系数 $\pdv{x^j}{x'^i}$ 对应的矩阵行列式 $\det\abs{\pdv{x^j}{x'^i}}\neq0$,所以矢量 $\pdv{x}{x'^i}$ 是线性无关的.


\textbf{证毕!}
\subsection{坐标曲线}
在引入了曲线坐标后,直接记 $x^i$ 为曲线坐标而不带撇号,那么区域 $\Omega$ 上任一点 $M$ 的向径 $x$,可表示为
\begin{equation}
x=x(x^1,\cdots,x^n)
\end{equation}

要很好的理解坐标系的结构,坐标曲线是很有用的.
\begin{definition}{坐标曲线}
在区域 $\Omega$ 中,曲线坐标中只有某一个不同的所有点构成的集合称为\textbf{坐标曲线},若该不同的坐标为 $x^i$,则该曲线称为坐标曲线 $x^i$.
\end{definition}
也就是说,在坐标曲线上仅有一个坐标在变化,而其余坐标保持常数.

由坐标曲线的定义,$\pdv{x}{x^i}$ 是坐标曲线 $x^i$ 的\textbf{切矢量},所以通过 $\Omega$ 上的每一点 $M$,有 $n$ 条切矢量各为 $\pdv{x}{x^i}$ 的坐标曲线,这些矢量记作 $\partial_i x=\pdv{x}{x^i}$ ,由 \autoref{CFinAf_the2} ,它们是线性无关的.
\subsection{局部标架}
由于在点 $M$ 处,$n$ 个矢量 $\partial_i x$ 是线性无关的,因为在每一点 $M$ 都可取这些矢量为仿射标架矢量 $\{M;\partial_1 x,\cdots,\partial_n x\}$ . $x=x(x^1,\cdots,x^n)$ 中的 $x^i$ 是曲线坐标,这就是说,给定了曲线坐标,就在每一点 $M$ 引出了一个完全确定的仿射标架.
\begin{definition}{局部标架}
在区域 $\Omega$ 中,曲线坐标在每一点 $M$ 引出的仿射标架 $\{M;\partial_1 x,\cdots,\partial_n x\}$ ,称着点 $M$ 的\textbf{局部标架}(或\textbf{局部坐标系}).
\end{definition} 
\subsubsection{曲线坐标和仿射坐标的关系}
在仿射坐标下,从点 $M(x^i)$