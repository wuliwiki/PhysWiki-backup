% Tychonoff定理
% 紧致性|紧性|compact|积拓扑|乘积拓扑

\pentry{紧致性\upref{Topo2},积拓扑\upref{Topo6}}

\subsection{定理的描述}

紧空间具有仅次于有限空间的优良性质.然而,一个有限空间的无限次积空间通常是无限集(只要这个有限空间的元素数大于$1$),我们可能会猜想,一个紧空间的无限次积空间可能会失去紧性.然而Tychonoff定理表明,对于积拓扑,紧空间的任意次笛卡尔积仍然是紧空间,紧性被保留了.

我们先列出该定理,然后讨论并给出其证明.

\begin{theorem}{Tychonoff定理}
给定一族拓扑空间$\{(X_\alpha, \mathcal{T}_\alpha)\}_{\alpha\in \Lambda}$,如果对每个$\alpha\in\Lambda$,$(X_\alpha, \mathcal{T}_\alpha)$都是紧空间,那么乘积空间$(\prod_{\alpha\in\Lambda}X_\alpha, \mathcal{T})$也是紧空间.这里$\mathcal{T}$是各$\mathcal{T}_\alpha$的积拓扑.
\end{theorem}



\subsection{有限积空间继承紧致性}

考虑两个紧空间$(X_1, \mathcal{T}_1)$和$(X_2, \mathcal{T}_2)$的积拓扑.为了方便视觉化理解,我们可以把$X_1$和$X_2$分别画成一根轴,于是它们的积空间就是一个平面.把各$X_i$的开集都想象成一条线段,于是$X_i$的紧致性可以描述为“任意覆盖了整条轴的线段集合,总存在有限子覆盖”.那么$X_i$的紧致性能继承到$X_1\times X_2$上吗?

乍一看似乎没关联,因为$X_1\times X_2$的开集和$X_i$开集的联系似乎只有\textbf{投影}.如果记$\{U_\alpha\}$是$X_1\times X_2$的一族开覆盖,那么投影后的$\{\pi_i(U_\alpha)\}$就是$X_i$的一族开覆盖.由$X_i$的紧致性虽然能得出$\{\pi_i(U_\alpha)\}$存在有限子覆盖$\{\pi_i(U_i)\}_{i\in I}$($I$是某个有限指标集),但没法保证$\{U_i\}_{i\in I}$是$X_1\times X_2$的覆盖.

不过,稍加优化我们会发现,由于开集都是基本开机的并、基本开集都是开集,可以得到“任意开覆盖都有有限子覆盖”等价于“任意拓扑基覆盖都有有限子覆盖”,这里拓扑基覆盖是指取拓扑基中的基本开集来覆盖.那是不是用拓扑基覆盖可以证明呢?不行,没有解决上述问题.

尽管如此,我们还是把这个结论记下来:

\begin{exercise}{}
证明:对于拓扑空间$X$的任意子空间$A$,下列命题等价:
\begin{enumerate}
\item $A$的任意开覆盖都有有限子覆盖;
\item $A$的任意\textbf{拓扑基}覆盖都有有限子覆盖.
\end{enumerate}
\end{exercise}

再优化思路:“任意拓扑基覆盖都有有限子覆盖”可以等价于“任意子基覆盖都有有限子覆盖”吗?这里的子基覆盖就是指用子基中的元素紧性覆盖.

答案是肯定的.

\begin{theorem}{}
对于拓扑空间$X$和它的子空间$A$,下列命题等价:
\begin{enumerate}
\item $A$的任意开覆盖都有有限子覆盖;
\item $A$的任意\textbf{子基}覆盖都有有限子覆盖.
\end{enumerate}
\end{theorem}

\textbf{证明}:

1. $\implies$ 2. 是显然的,因为子基的元素都是开集,从而子基覆盖都是开覆盖.下证 2. $\implies$1. .

设$\{U_\alpha\}_{\alpha\in\Lambda}$是$A$的开覆盖,则各$U_\alpha$是由若干基本开集的\textbf{并}:
\begin{equation}\label{TychT_eq1}
U_\alpha = \bigcup_{\beta\in \Gamma} B_{\alpha, \beta}
\end{equation}

每个基本开集$B_{\alpha, \beta}$又是\textbf{有限}个子基中元素的\textbf{交}:
\begin{equation}\label{TychT_eq2}
B_{\alpha, \beta} = \bigcap_{i\in I} S_{\alpha, \beta, i}
\end{equation}
即,希腊字母指标$\alpha$和$\beta$的指标集$\Lambda$和$\Gamma$基数都是任意的,而拉丁字母指标$i$的指标集$I$是\textbf{有限集}.

$\{U_\alpha\}$是$A$的开覆盖$\iff$ 对于任意$a\in A$,都存在$\alpha$使得$a\in U_\alpha$ $\iff$ 对于任意$a\in A$,都存在$\alpha, \beta$使得$a\in B_{\alpha, \beta}$ $\iff$ 对于任意$a\in A$,都存在$\alpha, \beta$使得$a\in S_{\alpha, \beta, i}$对任意$i$都成立 $\iff$ $\{S_{\alpha, \beta, i}\}$是$A$的开覆盖.

上一段的第一个等价是开覆盖的定义,第二个等价是并集的定义结合\autoref{TychT_eq1} ,第三个等价是交集的定义结合\autoref{TychT_eq2} ,第四个等价又是开覆盖的定义.

由题设,存在有限指标集$\Lambda'$,$\Gamma'$,和$I'$,使得$\{S_{\alpha, \beta, i}\}_{\alpha\in \Lambda', \beta\in \Gamma', i\in I'}$覆盖$A$\footnote{有限多个$S_{\alpha, \beta, i}$,三个指标的指标集都必须有限.}.我们可以扩大范围,取$\{S_{\alpha, \beta, i}\}_{\alpha\in \Lambda', \beta\in \Gamma', i\in I}$,即让$i$取遍整个$I$,则$I$的有限性保证了这依然是一个$A$的有限开覆盖.

又由\autoref{TychT_eq1} 和\autoref{TychT_eq2} 

\textbf{证毕}.























