% 泰勒展开(简明微积分)
% 微积分|导数|高阶导数|泰勒展开|泰勒级数

\pentry{阶乘\upref{factor}, 高阶导数\upref{HigDer}, 分部积分法\upref{IntBP},幂级数(简明微积分)\upref{powerS}}

\footnote{参考 Wikipedia \href{https://en.wikipedia.org/wiki/Taylor's_theorem}{相关页面}。}若函数 $f$ 在开区间 $I$ 内可以求任意阶的导数(例如多项式, 幂函数,三角函数,指数函数,对数函数等),那么这个函数可以用多项式近似, 且在某种意义下, 总项数 $N$ 越多,近似得越精确。 确切地说, 对于任何 $x_0\in I$, 存在唯一一个数列 $\{c_n\}$, 使得对于任何正整数 $N$, 皆有(其中 $(x-x_0)^0$ 始终视为 $1$, 即使 $x = x_0$)
\begin{equation}\label{Taylor_eq1}
f(x) = \sum_{n = 0}^N  c_n (x - x_0)^n + \order{|x-x_0|^{N+1}}.
\end{equation}
每一个系数 $c_n$ 由函数在 $x_0$ 处的第 $n$ 阶导数求得
\begin{equation}\label{Taylor_eq2}
c_n = \frac{1}{n!} f^{(n)}(x_0);
\end{equation}
注意其中 0 的阶乘为 $0! = 1$。 另外由\autoref{Taylor_eq1} 得,当 $x=x_0$ 时,函数值等于多项式值。 当项数 $N$ 有限时,通常 $\abs{x-x_0}$ 越小多项式就越接近函数 。 以上这种把函数展开成多项式的方法就叫\textbf{泰勒展开(Taylor expansion)}, 得到的多项式叫做\textbf{泰勒级数(Taylor series)}。 我们先来看一个例子:

\begin{example}{正弦函数}
我们在 $x_0=0$ 处展开 $\sin x$, 由\autoref{Taylor_eq1} 和\autoref{Taylor_eq2} 得
\begin{equation}\label{Taylor_eq3}
\sin x = x - \frac{1}{3!}{x^3} + \frac{1}{5!}{x^5} - \frac{1}{7!} x^7 + \ldots 
\end{equation}
取不同的项数 $N$ 求和,画图如\autoref{Taylor_fig1}。 可见随着项数增加,多项式慢慢趋近正弦函数。

\begin{figure}[ht]
\centering
\includegraphics[width=14cm]{./figures/Taylor_1.pdf}
\caption{$\sin x$ 在原点处的泰勒展开的前 $N$ 项求和。容易看出,求和的项数越多,多项式(橙)与 $\sin x$ (蓝)吻合得越好。}\label{Taylor_fig1}
\end{figure}
\end{example}

\begin{exercise}{一些常见函数关于原点的泰勒展开}
关于原点的泰勒展开又叫麦克劳林展开, 请验证以下的麦克劳林展开:
\begin{equation}\label{Taylor_eq10}
\sin x = x - \frac{1}{3!} x^3 + \frac{1}{5!} x^5 - \frac{1}{7!} x^7 \ldots
\qquad (x \in \mathbb R)
\end{equation}
\begin{equation}
\cos x = 1 - \frac{1}{2!} x^2 + \frac{1}{4!} x^4 -\frac{1}{6!} x^6 \ldots
\qquad (x \in \mathbb R)
\end{equation}
\begin{equation}\label{Taylor_eq11}
\E^x =1 + x + \frac{1}{2!} x^2 + \frac{1}{3!} x^3  \ldots
\qquad (x \in \mathbb R)
\end{equation}
\begin{equation}\label{Taylor_eq4}
\ln (1+x) = x - \frac12 x^2 + \frac13 x^3 - \frac14 x^4 \ldots
\qquad (-1 < x < 1)
\end{equation}
\begin{equation}\label{Taylor_eq6}
\frac{1}{1 \pm x} = 1 \mp x + x^2 \mp x^3 + x^4 \ldots
\qquad (-1 < x < 1)
\end{equation}
\begin{equation}\label{Taylor_eq8}
\sqrt{1\pm x} = 1 \pm \frac12 x - \frac18 x^2 \pm \frac{1}{16} x^3 - \frac{5}{128}x^4 \ldots
\qquad (-1 < x < 1)
\end{equation}
\end{exercise}

\subsection{收敛半径}
我们把形如 $\sum_{n=0}^\infty c_n (x-x_0)^n$ 的表达式叫做\textbf{幂级数(power series)}\upref{powerS}。 泰勒展开就是在用幂级数表示函数。 如果某幂级数收敛的点不止 $x_0$ 一点, 那么必定存在一个\textbf{收敛半径(radius of convergence)} $0 < r < 1$, 使得当 $\abs{x-x_0} < r$ 时级数必定收敛, 而 $\abs{x-x_0} > r$ 是必定发散(不收敛)。

所以在进行泰勒展开时, 如果函数在除了若干发散点\footnote{如\autoref{Taylor_eq4} 中的 $x = -1$, \autoref{Taylor_eq6} 和\autoref{Taylor_eq8} 中的 $x = \mp 1$}外都无穷阶可导, 令离 $x_0$ 最近的发散点为 $a$, 那么 $f(x)$ 关于 $x_0$ 泰勒展开后, 幂级数的收敛半径就是 $r = \abs{a - x_0}$。

例如在\autoref{Taylor_eq4} 到\autoref{Taylor_eq8} 中, 函数的发散点距离原点都是 1, 所以幂级数收敛半径为 1, 所以即使在 $x_0$ 的另一侧 $f(x)$ 处处无穷可导, 仍然不能取 $\abs{x} > 1$。

\subsection{幼稚的推导}
这里给出一个比较直观的对比系数推导方法, 可能对初学者有一定启发或者帮助记忆。 但以后会看到这是不严谨的。

我们假设当项数 $N \to \infty$ 时, 存在唯一的多项式在某区间内处处趋于无穷可导函数 $f(x)$,即
\begin{equation}\label{Taylor_eq5}
f(x) = \sum_{n = 0}^\infty  c_n (x - x_0)^n
\end{equation}
首先代入 $x = x_0$,可得第一个系数 $c_0 = f(x_0)$。 现在我们对上式两边在 $x_0$ 处求导,得
\begin{equation}
f'(x_0) = c_1 + \eval{ \sum_{n = 2}^\infty n c_n (x - x_0)^{n - 1} }_{x = x_0}  = c_1
\end{equation}
如果对\autoref{Taylor_eq5} 两边在 $x_0$ 处求二阶导数,得
\begin{equation}
f''(x_0) = 2 c_2 + \eval{ \sum_{n = 3}^\infty  n(n - 1) c_n (x - x_0)^{n - 2} }_{x = x_0}  = 2 c_2
\end{equation}
即 $c_2 = f''(x_0)/2!$。  以此类推,如果对\autoref{Taylor_eq5} 两边在 $x_0$ 处求 $m$ 阶导数得
\begin{equation}
f^{(m)}(x_0) = m! c_m + \eval{ \sum_{n = m + 1}^\infty  \frac{n!}{(n - m)!} c_n (x - x_0)^{n - m} }_{x = x_0}  = m! c_m
\end{equation}
所以系数公式为
\begin{equation}
{c_m} = \frac{1}{m!} f^{(m)}(x_0)
\end{equation}

泰勒展开的存在说明了一些函数(称为\textbf{解析函数})具有这样的性质:任何一点的性质都能决定完整的函数曲线,这可以类比生物中用一个细胞克隆出一个完整生物体。

\subsection{无法被泰勒多项式逼近的函数}

考虑函数
\begin{equation}
f(x) = \E^{-\frac{1}{x^2}}
\end{equation}
