% 2014 年计算机学科专业基础综合全国联考卷
% keys 2014 计算机 考研 真题 全国卷

\subsection{一、单项选择题}
第1~40 小题,每小题2 分,共80 分.下列每题给出的四个选项中,只有一个选项最符合试题要求.

1.下列程序段的时间复杂度是:
\begin{lstlisting}[language=cpp]
count=0;
for(k=1;k<=n;k*=2)
  for(j=1;j<=n;j++)
    count++;
\end{lstlisting}
A.$O(log2n)$ $\quad$ B.$O(n)$ $\quad$ C.$O(nlog2n)$ $\quad$ D.$O(n2)$

2.假设栈初始为空,将中缀表达式$a/b+(c*d-e*f)/g$转换为等价的后缀表达式的过程中,当扫描到$f$时,栈中的元素依次是. \\
A.$+ ( * -$  $\quad$ B.$+ ( - *$  $\quad$ C.$/ + ( * - *$  $\quad$ D.$/ + - *$

3.循环队列放在一维数组$A[0...M-1]$中,$end1$指向队头元素,$end2$指向队尾元素的后一个位置.假设队列两端均可进行入队和出队操作,队列中最多能容纳$M-1$ 个元素.初始时为空.下列判断队空和队满的条件中,\textbf{正确}的是. \\
A.队空:end1 == end2; 队满:end1 == (end2+1)mod M \\
B.队空:end1 == end2; 队满:end2 == (end1+1)mod (M-1) \\
C.队空:end2 == (end1+1)mod M; 队满:end1 == (end2+1)mod M \\
D.队空:end1 == (end2+1)mod M;队满:end2 == (end1+1)mod (M-1)

4.若对如下的二叉树进行中序线索化,则结点$x$的左、右线索指向的结点分别是:
\begin{figure}[ht]
\centering
\includegraphics[width=5cm]{./figures/CSN14_1.png}
\caption{第3题图} \label{CSN14_fig1}
\end{figure}
A.e、c $\quad$ B.e、a $\quad$ C.d、c $\quad$ D.b、a

5.将森林F转换为对应的二叉树T,F中叶结点的个数等于. \\
A.T 中叶结点的个数 $\quad$ B.T 中度为1 的结点个数 \\
C.T 中左孩子指针为空的结点个数 $\quad$ D.T 中右孩子指针为空的结点个数

