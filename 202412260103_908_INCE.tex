% 信息科学(综述)
% license CCBYSA3
% type Wiki

本文根据 CC-BY-SA 协议转载翻译自维基百科\href{https://en.wikipedia.org/wiki/Information_science}{相关文章}。

\begin{figure}[ht]
\centering
\includegraphics[width=10cm]{./figures/15c7d956dfa4ab1f.png}
\caption{从元数据领域获取洞察的各种方法论方法的可视化} \label{fig_INCE_1}
\end{figure}
信息科学[1][2][3] 是一门主要关注信息的分析、收集、分类、处理、存储、检索、传输、传播和保护的学术领域。[4] 该领域内外的从业者不仅研究知识在组织中的应用与使用,还研究人与组织以及任何现有信息系统之间的互动,旨在创建、替代、改进或理解信息系统。

从历史上看,信息科学与信息学、计算机科学、数据科学、心理学、技术、文献学、图书馆学、医疗保健和情报机构等领域有关联。[5] 然而,信息科学也包括了诸如档案学、认知科学、商业、法律、语言学、博物馆学、管理学、数学、哲学、公共政策和社会科学等多种领域的内容。
\subsection{基础} 
\subsubsection{范围与方法} 
信息科学专注于从利益相关者的角度理解问题,然后根据需要应用信息和其他技术。换句话说,它首先解决的是系统性问题,而不是系统内部的单个技术组件。在这一点上,可以将信息科学视为对技术决定论的回应,技术决定论认为技术“按照自身的规律发展,发挥自身的潜力,仅受到可用物质资源和开发者创造力的限制。因此,它必须被视为一个自治系统,控制并最终渗透社会的所有其他子系统。”[6]

许多大学拥有专门研究信息科学的学院、部门或学校,而众多信息科学学者则在传播学、医疗保健、计算机科学、法学和社会学等学科中工作。一些机构已成立信息学院联盟(参见信息学院列表),除此之外,许多其他机构也具有全面的信息研究方向。

在信息科学领域,2013年时的当前问题包括:
\begin{itemize}
\item 科学中的人机交互
\item 协同软件
\item 语义网
\item 价值敏感设计
\item 迭代设计过程
\item 人们生成、使用和寻找信息的方式
\end{itemize}
\subsubsection{定义}  
“信息科学”这一术语的首次已知使用是在1955年。[7] 信息科学的早期定义(追溯到1968年,美国文献学会将其名称更改为美国信息科学与技术学会的那一年)指出:

信息科学是研究信息的属性和行为、控制信息流动的力量,以及处理信息以实现最佳可访问性和可用性的方法的学科。它关注与信息的生成、收集、组织、存储、检索、解释、传输、转化和利用相关的知识体系。这包括自然和人工系统中信息表现的真实性,使用编码进行高效信息传递,以及研究信息处理设备和技术,如计算机及其编程系统。它是一门跨学科的科学,源自并与数学、逻辑学、语言学、心理学、计算机技术、运筹学、图形艺术、通讯、管理以及其他类似领域相关。它既有纯科学的成分,探索这一主题而不考虑其应用,也有应用科学的成分,开发服务和产品。(Borko 1968,第3页)[8]

\textbf{相关术语}

一些作者将信息学(Informatics)视为信息科学的同义词。这种说法尤其在与A. I. 米哈伊洛夫(A. I. Mikhailov)及其他苏联作者在1960年代中期提出的概念相关时尤为常见。米哈伊洛夫学派将信息学视为与科学信息研究相关的学科。[9] 由于信息学领域快速发展且跨学科的特点,使得信息学的定义很难精确定义。依赖于用于从数据中提取有意义信息的工具特性的定义,正在信息学的学术项目中出现。[10]

区域差异和国际术语使得这个问题更加复杂。有些人[哪一些?]指出,今天所称为“信息学”的许多内容曾经被称为“信息科学”——至少在医学信息学等领域中是这样。例如,当图书馆学家开始使用“信息科学”一词来指代他们的工作时,“信息学”这一术语便应运而生:
\begin{itemize}
\item 在美国,信息学作为计算机科学家为区分自己与图书馆学的工作所做的回应;
\item 在英国,信息学作为研究自然信息处理系统以及人工或工程化信息处理系统的科学术语。
\end{itemize}
另一个被讨论为“信息研究”的同义词的术语是“信息系统”。Brian Campbell Vickery的《信息系统》(1973)将信息系统纳入信息科学(IS)范畴。[11] 另一方面,Ellis、Allen和Wilson(1999)提供了一项文献计量学调查,描述了“信息科学”和“信息系统”这两个不同领域之间的关系。[12]
\subsubsection{信息哲学}   
信息哲学研究心理学、计算机科学、信息技术和哲学交汇处出现的概念性问题。它包括对信息的概念性质和基本原理的研究,包括信息的动态、利用及其科学,以及将信息理论和计算方法应用于其哲学问题的阐述与应用。[13]
\subsubsection{本体论}   
在科学和信息科学中,本体论正式地表示一个领域内的知识作为一组概念及这些概念之间的关系。它可以用来推理该领域中的实体,并且可以用来描述该领域。

更具体地说,本体论是描述世界的模型,由一组类型、属性和关系类型构成。围绕这些内容提供的具体内容有所不同,但它们是本体论的基本要素。通常,本体论的模型与现实世界之间应该有着密切的相似性。[14]

理论上,本体论是“对共享概念化的正式、明确规范”。[15] 本体论提供了共享的词汇和分类法,用来通过定义对象和/或概念及其属性和关系来建模某一领域。[16]

本体论是组织信息的结构框架,并在人工智能、语义网、系统工程、软件工程、生物医学信息学、图书馆学、企业书签和信息架构等领域中作为一种知识表示形式,用于表示世界或某一部分世界的知识。领域本体论的创建对于定义和使用企业架构框架也至关重要。
\subsubsection{科学还是学科?}  
像Ingwersen[17]这样的作者认为,信息学面临与其他学科界定自身边界的问题。根据波普尔(Popper)所说,“信息科学忙于处理一个包含大量常识性实际应用的海洋,这些应用越来越涉及计算机……而计算机科学在语言、沟通、知识和信息等常识性看法上几乎没有更好的状态。”[18] 其他作者,如Furner,否认信息科学是一门真正的科学。[19]
\subsection{职业}  
\subsubsection{信息科学家}  
信息科学家是指具有相关学科学位或高水平学科知识的个人,他们为工业界的科学和技术研究人员提供专门的信息,或者为学术界的学科教师和学生提供信息。工业界的信息专家/科学家和学术界的学科信息专家/图书馆员通常有相似的学科背景培训,但学术职位持有者通常需要拥有第二个高级学位(如信息与图书馆学的MLS/MI/MA学位),除了学科硕士学位外。这个职位也适用于从事信息科学研究的个人。
\subsubsection{系统分析师}  
系统分析师负责创建、设计和改进为特定需求提供的信息系统。系统分析师通常与一个或多个企业合作,评估和实施组织流程和技术,以改善组织内的信息访问,从而提高效率和生产力。
\subsubsection{信息专业人士}    
信息专业人士是指那些负责保存、组织和传播信息的个人。信息专业人士擅长组织和检索记录的知识。传统上,他们的工作涉及印刷材料,但这些技能现在越来越多地用于电子、视觉、音频和数字材料。信息专业人士在各种公共、私营、非营利和学术机构中工作,也可以在组织和工业环境中找到。他们的角色包括系统设计和开发、系统分析等。
\subsection{历史}  
\subsubsection{早期起源}
\begin{figure}[ht]
\centering
\includegraphics[width=6cm]{./figures/d3b8a79f2fbc4764.png}
\caption{戈特弗里德·威廉·莱布尼茨,一位德国博学家,主要用拉丁语和法语写作。他的研究领域包括形而上学、数学和神义学。} \label{fig_INCE_2}
\end{figure}
信息科学在研究信息的收集、分类、处理、存储、检索和传播时,其根源可以追溯到人类知识的共同积累。信息分析至少可以追溯到亚述帝国时期,当时文化存储库的出现,也就是今天所称的图书馆和档案馆。[20] 从机构上看,信息科学在19世纪与许多其他社会科学学科一同兴起。然而,作为一门科学,它的制度化根源可以追溯到科学史,始于1665年皇家学会(伦敦)出版的《哲学会刊》的第一期,这通常被认为是世界上第一本科学期刊。

科学的制度化过程贯穿了18世纪。1731年,本杰明·富兰克林建立了费城图书公司,这是由一群公民共同拥有的第一座图书馆,它很快超越了图书的范畴,成为了科学实验的中心,并举办了公开的科学实验展示。[21] 本杰明·富兰克林还将一批书籍捐赠给马萨诸塞州的一个小镇,镇上投票决定免费向所有人开放这些书籍,从而成立了美国第一座公共图书馆。[22] 1736年,巴黎的外科学院(Academie de Chirurgia)出版了《外科医师纪要》,被普遍认为是第一本医学期刊。模仿伦敦皇家学会的模式,美国哲学学会于1743年在费城成立。随着许多其他科学期刊和学会的成立,阿洛伊斯·塞内费尔德于1796年在德国开发了石版印刷的概念,用于大规模印刷工作。


\begin{figure}[ht]
\centering
\includegraphics[width=6cm]{./figures/4bd649af023629f6.png}
\caption{约瑟夫·玛丽·雅卡尔} \label{fig_INCE_3}
\end{figure}
到19世纪,信息科学的雏形开始显现,成为与其他科学和社会科学分开且独立的学科,但同时与通信和计算相结合。1801年,约瑟夫·玛丽·雅卡尔发明了一种穿孔卡片系统,用于控制法国织布机的操作。这是首次使用“图案的记忆存储”系统。[23] 随着化学期刊在1820年代和1830年代的出现,[24] 查尔斯·巴贝奇于1822年开发了“差分机”,这是现代计算机的第一步,并于1834年完成了“分析机”。到1843年,理查德·霍伊发明了旋转印刷机,1844年,塞缪尔·莫尔斯发出了第一条公共电报消息。1848年,威廉·F·普尔开始编写《期刊文献索引》,这是美国第一本通用的期刊文献索引。

1854年,乔治·布尔出版了《思维法则研究...》,奠定了布尔代数的基础,后者在信息检索中得到了应用。[25] 1860年,在卡尔斯鲁厄技术高等学院举行了一次大会,讨论建立化学学科的系统和理性命名法的可行性。虽然大会没有得出结论,但几位关键参与者带回了斯坦尼斯劳·卡尼查罗(1858年)的提纲,最终说服他们接受他关于计算原子质量的方案。[26]

到1865年,史密森学会开始编制当前的科学论文目录,1902年该目录成为《国际科学论文目录》。[27] 次年,皇家学会开始在伦敦出版《论文目录》。1868年,克里斯托弗·肖尔斯、卡洛斯·格里登和S·W·苏尔共同发明了第一台实用的打字机。到1872年,开尔文勋爵设计了一种模拟计算机,用于预测潮汐,1875年,弗兰克·斯蒂芬·鲍德温获得了美国第一个实用计算机器的专利,该机器能执行四种算术运算。[24] 1876年和1877年,亚历山大·格雷厄姆·贝尔和托马斯·爱迪生分别发明了电话和留声机,美国图书馆协会在费城成立。1879年,《医学索引》首次由美国陆军外科总长图书馆出版,约翰·肖·比林斯担任馆长,后来该图书馆发行《索引目录》,并以其作为最完整的医学文献目录而享有国际声誉。[28]