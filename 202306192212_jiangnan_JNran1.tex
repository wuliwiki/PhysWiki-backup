% 从集合论角度看随机事件
% 集合论|概率论|随机样本
\pentry{离散型随机变量(高中)\upref{HsDRV},条件概率\upref{HsCpMi}}
在概率论问题中,我们通常要处理各种各样的时间。比如,我们要问,当掷出一个骰子时,点数大于四的概率是多少?或者当掷出两个骰子,一个点数大于二,另一个点数小于三的概率是多少?在这里我们将处理各种事件的想法与集合论中集合的运算相对应,来看待如何从集合运算视角处理概率论中的问题。
\subsection{样本空间}
我们定义在随机试验过程中,每次获得的一个数据称为一个样本,或称一个样本点。所有可能的样本点所构成的集合称为样本空间
\begin{equation}
S = \{e_1,e_2,...,e_n\}~.
\end{equation}
我们称抽样过程中的某一个事件是由一组样本点所构成的集合。定义事件$A$
\begin{equation}
A = \{e_1,e_2,..e_i\}~.
\end{equation}
其中当$A= \emptyset $时,我们称A为不可能事件。而当$A=S$时,我们称A为必然事件。如果$A$中只包含一个样本,则称$A$为基本事件。\\
在这里我们多说几句这个定义的一些想法,我们在进行随机试验时,得到的总是一个个样本,但是我们通常需要根据样本的性质对各种采样结果分类。比如,掷出两个骰子点数如果总和大于6点的时候我们说这个结果是大。这样所谓"大"的结果作为一个事件,就将包含多个可能的样本点,所以我们将事件定义为样本点的集合。
\subsection{事件的关系与运算}
当我们有了集合论的视角后,我们来看不同事件之间的关系。
\begin{enumerate}
\item 如果$A\subseteq B$,则事件A的发生一定导致事件B的发生。
\item 如果$A \cap B = \emptyset$,则$A$事件发生时$B$事件必定不会发生,反之亦然。即$A$事件与$B$事件是互斥事件。
\end{enumerate}

进而我们可以利用集合之间的运算,利用一些事件构造新的事件。例如
\begin{enumerate}
\item 设A,B是事件,则$C = A \cap B$,$C$事件可以被解释为$A$事件与$B$事件同时发生。
\item $C = A \cup B$,$C$事件可以被解释为$A$事件与$B$事件中至少一个发生了。
\item 设$A$的补集为$\bar{A}$,则$\bar{A}$就被解释为$A$事件未发生,我们记为$\bar{A}=S-A$。
\item $C = A \cap \bar{B}$, $C$事件可以被解释为$A$事件发生而$B$事件不发生。
\item 以及多个事件的运算,$\cup_i A_i$,$A_i$中至少一个事件发生,$\cap_i A_i$,$A_i$中所有事件都发生。
\end{enumerate}
更多类似的事件之间的运算与解释读者可自行进行构造。在这里我们再回顾一下集合间运算的一些规则,这里也直接对应事件构造的规则
\begin{enumerate}
\item 交换律:$A \cup B = B \cup A$, $A \cap B = B \cap A$.
\item 结合律:$(A \cup B)\cup C = A \cup (B \cup C)$, $(A \cap B)\cap C = A \cap (B\cap C)$
\item 分配律:$(A\cup B)\cap C = (A \cap C) \cup (B \cap C)$, $(A\cap B) \cup C = (A \cup C)\cap (B \cup C)$
\item De Morgan定律: $\bar{\cup_i A_i} = \cap_i \bar{A}_i$, $\bar{\cap_i A_i} = \cup_i \bar{A}_i$
\end{enumerate}
\subsection{概率}
现在我们从集合论的视角澄清了事件之间的关系,接下来我们引入概率的概念。概率是指每个事件发生的几率,按照我们对概率一般的理解,我们将概率定义为一个从集合$A$映射到实数的一个函数$P$,这个函数应该满足如下几个性质
\begin{enumerate}
\item 非负性:对于任意一个事件$A$,都有$P(A)\leq 0$
\item 归一条件:对于样本空间$S$,有$P(S) = 1$,对于空集$P(\emptyset) = 0$
\item 可加性: 对于两个事件$A_1,A_2$,如果有$A_1 \cap A_2 = \emptyset$,则有 $P(A_1 \cup A_2) = P(A_1)+P(A_2)$
\end{enumerate}
这时,我们回到条件概率的定义$P(A|B) = P(A\cap B)/P(B)$,


