% 量子化
% license CCBYSA3
% type Wiki

(本文根据 CC-BY-SA 协议转载自原搜狗科学百科对英文维基百科的翻译)

在物理学中,\textbf{量子化}是从对物理现象的经典理解过渡到被称为量子力学的新理解的过程,这是一个从经典场论开始构建量子场论的过程,从经典力学建立量子力学的过程的概括。同样相关的是\textbf{场量子化},如在“电磁场的量子化”中,将光子称为场“量子”(例如光量子)。这个过程是粒子物理、核物理、凝聚态物理和量子光学理论的基础。

\subsection{量化方法}
量子化将经典场转换成作用于场论量子态的算符。最低能量状态称为真空状态。发展量子化理论是为了通过计算量子振幅来推断材料、物体或粒子的各种复杂性质。在做这种计算时必须要解决好“重整化”的问题,如果忽视这些细微的差异,往往会导致无意义的结果,例如出现各种振幅的不定式。量化过程的完整规范需要执行重正化的方法。

场理论量子化的第一种方法是规范量子化。虽然这在足够简单的理论上非常容易实现,但在许多情况下,其他量化方法会产生更有效的计算量子振幅的过程。然而,规范量子化的使用在量子场论的语言和解释上留下了印记。

\subsubsection{1.1 规范量子化}
场论的规范量子化类似于从经典力学中构造量子力学。经典场被视为一个叫做规范坐标的动力学变量,它的时间导数就是规范动量。人们引入了它们之间的换向关系,这与量子力学中粒子位置和动量之间的换向关系完全相同。从技术上讲,人们通过创造和湮灭算子的组合,将场转化为算子。场算符作用于理论的量子态。最低能量状态称为真空状态。该过程也称为二次量化。

这个过程可以应用于任何场论的量子化:无论是费米子还是玻色子,以及任何内部对称性。然而,它导致真空状态的相当简单的图像,并且不容易适用于某些量子场论,例如量子色动力学,已知其具有以许多不同冷凝物为特征的复杂真空。










