% 选择公理(综述)
% license CCBYSA3
% type Wiki

本文根据 CC-BY-SA 协议转载翻译自维基百科\href{https://en.wikipedia.org/wiki/Axiom_of_choice}{相关文章}。

\begin{figure}[ht]
\centering
\includegraphics[width=8cm]{./figures/ce05c80ac3fab240.png}
\caption{} \label{fig_AC_1}
\end{figure}
在数学中,选择公理(简称AC或AoC)是集合论的一个公理,它等价于“非空集合的笛卡尔积是非空的”这一命题。非正式地说,选择公理表明,给定任何一个集合的集合,每个集合至少包含一个元素,便可以通过从每个集合中选择一个元素来构造一个新集合,即使这个集合是无限的。形式上,它声明,对于每一个索引族\( (S_i)_{i \in I} \)的非空集合,存在一个索引集合\( (x_i)_{i \in I} \),使得对于每个\( i \in I \),都有\( x_i \in S_i\)\(^\text{[1]}\)。选择公理是由恩斯特·泽梅洛在1904年提出的,旨在形式化他的良序定理证明。选择公理等价于每个划分都有一个横切集的命题\(^\text{[2]}\)。