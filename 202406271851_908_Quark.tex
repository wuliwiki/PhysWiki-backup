% 夸克
% license CCBYSA3
% type Wiki

(本文根据 CC-BY-SA 协议转载自原搜狗科学百科对英文维基百科的翻译)

\begin{figure}[ht]
\centering
\includegraphics[width=6cm]{./figures/855a38f26f1887d7.png}
\caption{由二个上夸克及一个下夸克所构成的质子} \label{fig_Quark_13}
\end{figure}

\textbf{夸克}是一种基本粒子,同时也是构成物质的基本成分。夸克相互结合,形成一种复合粒子,叫强子,其中最稳定的是质子和中子,它们是原子核的组成部分。[1]基于一种叫”夸克禁闭“的现象,夸克永远不会被直接观察到或孤立地发现。它们只能在强子中找到,包括重子(如质子和中子)和介子。[2][3]因此,人们对夸克的大部分认识来自对强子的观察。

夸克有各种各样的内禀性质,包括电荷、质量、色荷和自旋。在粒子物理学标准模型中,夸克是唯一一种经历所有四个基本互相作用的基本粒子,基本相互作用有时也会被称为“基本力”(电磁力,万有引力,强相互作用力,和弱相互作用)。夸克同时也是唯一已知的基本电荷非整数的粒子。

夸克有六种"味",分别是上,下,粲, 奇,底和顶。在所有夸克中,上夸克和下夸克具有最低的质量。较重的夸克通过粒子衰变过程迅速转变成上或下夸克。粒子衰变是一个从较高质量状态转变成较低质量状态的过程。因此,上夸克和下夸克通常是稳定的,所以在宇宙中最常见,而奇夸克、粲夸克、底夸克和顶夸克只能在高能碰撞中产生(例如宇宙线的和粒子加速器)。夸克每一种味都有一种对应的反粒子,称为\textbf{反夸克},它与夸克的不同之处,在于它的一些特性跟夸克大小一样但正负不同。

夸克模型分别由物理学家默里·盖尔·曼和乔治·茨威格在1964年独立提出的。引入夸克这一概念,是为了能更好地整理各种强子,而当时并没有什么能证实夸克存在的物理证据,直到1968年SLAC开发出深度非弹性散射实验为止。夸克的六种味已经全部被加速器实验所观测到;而于1995年在费米实验室被观测到的顶夸克,是最后发现的一种。

\subsection{分类}
\begin{figure}[ht]
\centering
\includegraphics[width=14.25cm]{./figures/9a82ca6b63e8f20f.png}
\caption{标准模型中的六个夸克粒子(以紫色显示)。前三列中的每一列都形成了一个重要的世代。} \label{fig_Quark_1}
\end{figure}

标准模型是描述目前已知所有基本粒子的理论框架。这个模型包含了夸克的六个“味”, 分别味 上 (u),下(d),粲(s),奇(c),底(b),以及顶(t)中。[4] 夸克的反粒子由对应夸克符号上的条表示,例如u表示上反夸克。与一般的反物质一样,反夸克和其相应的夸克有相同的质量,平均寿命,自旋,但是电荷相反。[4]

夸克自旋为-1⁄2,根据自旋统计定理,它们是费米子,并服从泡利不相容原理,即相同的量子态不可以被多个费米子占据。这与玻色子(具有整数自旋的粒子)不同,即相同的量子态可以被多个玻色子占据。[5]与轻子不同,夸克拥有色荷,因此它们之间有强相互作用。这种吸引力使得夸克结合在一起,形成了复合粒子,称为强子。决定强子量子数的夸克被称为价夸克;除此之外,在强子内不定数量的虚“”夸克、反夸克和胶子并不影响其量子数。[6]强子有两个家族,它们分别是重子和介子,其中重子有三个价夸克,介子有一个价夸克和一个反夸克。[7]最常见的重子是质子和中子,它们是原子核的组成部分。目前发现很多强子(参见重子列表和介子列表),它们通常通过其夸克含量以及其性质来区分。拥有较多价夸克的“奇异”强子,例如从夸克模型预测的四夸克(qqqq)和五夸克(qqqqq),直到21世纪早期才被发现。[8][9][10][11]

基本费米子分为三代,每个都包含两个轻子和两个夸克。第一代包括上夸克和下夸克,第二代包括奇夸克和粲夸克,以及第三代包括底夸克和顶夸克。所有对第四代夸克和其他费米子的研究都失败了,[12][13]有强有力的间接证据表明不超过三代。[14][15][16]较高代的粒子通常质量较大,稳定性较低。它们通过弱相互作用衰变转化为下一代粒子。自然界中最常见的只有第一代(上下)夸克。较重的夸克只能在高能碰撞中产生(例如宇宙射线),然后迅速衰变。然而,人们认为夸克的出现开始于大爆炸时期,当时宇宙处于非常热和稠密的阶段。目前对重夸克的研究多是在人工创造的条件下进行的,例如粒子加速器。[17]

夸克具有电荷、质量、色荷和味这四个属性,是唯一已知的参与当代物理学所有四个基本相互作用理论的基本粒子。这四个基本相互作用力分别是,电磁相互作用力,万有引力,强相互作用力,和弱相互作用。除了在能量的极限(普朗克能量)和距离标度(普朗克距离)的条件下,涉及到粒子相互作用的粒子太弱了。目前,标准模型重不考虑万用引力作用,主要是因为没有成功的涉及万有引力的量子力学理论存在。

引力太弱,除了在能量的极限(普朗克能量)和距离标度(普朗克距离)中。然而,既然没有成功重力量子理论引力不是标准模型所描述的。

请参见下面的来查看夸克六种“味”的性质。

\subsection{历史}
\begin{figure}[ht]
\centering
\includegraphics[width=6cm]{./figures/62ccfb331751f703.png}
\caption{默里·盖尔·曼} \label{fig_Quark_2}
\end{figure}

\begin{figure}[ht]
\centering
\includegraphics[width=6cm]{./figures/5a8a3037822de78c.png}
\caption{乔治·茨威格} \label{fig_Quark_3}
\end{figure}

\begin{figure}[ht]
\centering
\includegraphics[width=6cm]{./figures/c4a65218fb231c65.png}
\caption{导致发现的事件的照片Σ++c重子,1974年在布鲁克黑文国家实验室} \label{fig_Quark_4}
\end{figure}
夸克模型是由物理学家默里·盖尔曼和和乔治·茨威格[18][19]在[20]1964年独立提出的。[21]盖尔曼于1961年提出粒子分类系统-八重法,用专业的术语来说,又称为U(3) 味对称。在此之后,他提出了夸克模型,用来精简早期的结构。[21]物理学家尤瓦尔·内曼在同一年独立开发了一个类似于八重道的理论,[22][23]在坂田模型中可以看到对组成结构的早期尝试。

在夸克理论诞生的时候,当时的“粒子园”除了其他各种粒子,还包括了许多强子。盖尔曼和茨威格假设它们不是基本粒子,而是由夸克和反夸克组成的。在他们的模型中,夸克有三种味,分别是上,下和奇,他们将自旋和电荷等属性都归因于这些味。[20][18][19]物理界对这项提议的最初反应不一。人们对于夸克的性质有所争议,主要集中在夸克是一种物理实体,还是仅仅用来解释当时未明物理的抽象概念。[24]

一年之后,有人提出了盖尔曼-茨威格模型的延展方案。谢尔登·格拉肖和詹姆斯·比约肯预测了第四种夸克味的存在,他们称之粲夸克。之所以加上第四种夸克是因为,一、能更好地描述弱相互作用(导致夸克衰变的机制);二、夸克的数量会变得与当时已知的轻子数量一样;三、能产生一条质量公式,可以计算出已知介子的质量。[25]

1968年,在斯坦福线性加速器中心 (SLAC)的深度非弹性散射实验表明,质子包含比自己小得多的点状物,因此不是基本粒子。[26][26][26]物理学家当时不愿意将这些物体视为夸克,而是称之为“成子”(parton) ——一个由理查德·费曼创造的术语。[27][28][29]随着在SLAC所观测到的粒子后来被鉴定为上夸克和下夸克。[30]不过,成子仍然被用作强子(夸克、反夸克和胶子)成分的总称。

SLAC散射实验间接验证了奇夸克的存在:它不仅是盖尔曼和茨威格三夸克模型的必要组成部分,而且还解释了于1947年在宇宙射线中发现的K和π强子。[31]

在1970年的一篇论文中,格拉肖、约翰·李尔普罗斯和卢奇亚诺·马伊阿尼一起对当时尚未发现的粲夸克,提出了一个新的理论模型,即GIM机制。[32][33]到1973年,小林诚和益川敏英在实验中注意到宇称不守恒定律。他们指出只有再加一对夸克,就可以解释该实验观察[34],因此夸克的味增加至现在的六种。

粲夸克在1974年11月由两个研究小组几乎同时产生(见十一月革命)—一组在SLAC,由伯顿·里克特领导;而另一组在布鲁克黑文国家实验室,由丁肇中领导。观察到粲夸克在介子里面与一个反粲夸克束缚在一起。两组给发现的这个介子分别起了不同的名字:$J$和$\psi$;因此,它也被正式命名为 $J/\psi$介子。这一发现最终使物理界相信了夸克模型是正确的。[29]

在之后的几年里,物理界出现了一些把夸克数量扩展到六个的建议。其中,以色列物理学家海姆·哈拉里在1975年的论文中,第一次将加上的夸克命名为顶夸克和底夸克。

1977年,利昂·莱德曼领导的费米国立加速器实验室团队观察到了底夸克。[35][36]这是一个代表顶夸克存在的有力证据:没有顶夸克,底夸克就没有伙伴。然而,直到1995年,顶夸克才最终被费米实验室的 CDF[37]和DØ [38]小组观察到。[21]它的质量比以前预期的要大得多,[39]几乎和金原子一样重。[40]

\subsection{语源}
有一段时间,盖尔曼打算用鸭的叫声来命名夸克。刚开始他并不确定这个词的真正拼写,直到他在詹姆斯·乔伊斯的小说芬尼根守灵夜找到了这个词夸克':[41]

– 给马斯特 马克来三个夸克!

这个词夸克本身就是一个斯拉夫语借入德国的并且表示乳制品,[42]但也是“垃圾”的口语术语。[43][44]盖尔曼在其著作《夸克与美洲豹》中,详细说明了夸克这个词的:[45]

在1963年,我把核子的基本构成部分命名为“夸克”(quark),我先有的是声音,而没有拼法,所以当时也可以写成“郭克”(kwork)。不久之后,在我偶尔翻阅詹姆斯·乔伊斯所著的《芬尼根的守灵夜》时,我在“向麦克老大三呼夸克”这句中看到夸克这个词。由于“夸克”(字面上意为海鸥的叫声)很明显是要跟“麦克”及其他这样的词押韵,所以我要找个借口让它读起来像“郭克”。但是书中代表的是酒馆老板伊厄威克的梦,词源多是同时有好几种。书中的词很多时候是酒馆点酒用的词。所以我认为或许“向麦克老大三呼夸克”源头可能是“敬麦克老大三个夸脱”,那么我要它读“郭克”也不是完全没根据。再怎么样,字句里的三跟自然中夸克的性质完全不谋而合。

茨威格更喜欢这个叫”埃斯“(Ace)来称呼他所理论化的粒子。但是,随着夸克模型被广泛接受,盖尔曼的术语就变得很有名。[46]

夸克味的命名都是有原因的。上及下夸克被这样叫,主要是源于同位旋的上和下分量,而它们确实各自带有这样的一个量。[47]奇夸克之所以得名,是因为它们在在宇宙射线的奇异粒子中被发现的,发现奇夸克的时候还没有夸克理论;这些粒子被认为是“奇异”,因为它们的寿命异常长。[48]格拉肖和布约肯共同提出了粲夸克,据说格拉肖说,“我们称它为‘粲夸克’,是因为在构建它的过程中,见到它为亚原子世界所带来的对称。我们被这种美迷住了,对成果感到很满意”[49]哈拉里创造的“底”和“顶”这两个名字,是因为它们是“上及下夸克的逻辑伙伴”。[50][50][48]过去,底部夸克和顶部夸克有时分别被称为“美”和“真”夸克,但是这些名字已经不再使用了。[50]虽然“真”从未流行,但致力于大规模生产底部夸克的加速器复合体有时被称为“美容院”。[51]

\subsection{性能}
\subsubsection{4.1 电荷}
夸克的电荷是分数——基本电荷的−1⁄3或+2⁄3倍,取决于其“味”。上,粲和顶夸克(统称为“上型夸克”)的电荷为+2⁄3,而下,奇及底夸克(统称为“下型夸克”)的电荷为-1⁄3。反夸克与其对应的夸克电荷相反;上型反夸克的电荷为-2⁄3和下型反夸克的电荷为+1⁄3。由于强子的电荷是组成它的夸克的电荷总和,所以所有强子的电荷均为整数:三个夸克的组合(重子)、三个反夸克的组合(反重子)、或一个夸克配一个反夸克(介子),加起来的电荷均为整数。[52]例如,原子核的强子成分,中子和质子,其电荷分别为0 e和+1 e;中子由两个下夸克和一个上夸克组成,质子由两个上夸克和一个下夸克组成。[53]

\subsubsection{4.2 自旋}
自旋是基本粒子的内禀属性,它的方向是一个重要的自由度。在视像化时,有时它被视为一沿着自己中轴旋转的物体(因此得名“ 自旋 ”)。因为基本粒子经常被认为是点状的,所以这种视像化概念在亚原子尺度上有点被误导了。[53]

自旋可以用矢量来表示,其长度用约化普朗克常量$\hbar$来量度。对于夸克,在任何轴上量度自旋的矢量分量均为$+\hbar/2$或$\hbar/2$;因此夸克是自旋$\frac{1}{2}$粒子。沿着给定轴旋转的分量(惯例为z轴),一般用上箭头$\uparrow$代表$+\frac{1}{2}$,下箭头$\downarrow$代表$-\frac{1}{2}$,然后在后加上味的符号。例如,自旋为$+\frac{1}{2}$的上夸克可被写成$u\uparrow$。


\subsubsection{4.3 弱相互作用}
\begin{figure}[ht]
\centering
\includegraphics[width=6cm]{./figures/9eb9dd672716dfd5.png}
\caption{随着时间的推移,β衰变的费曼图。CKM矩阵(下面讨论)展示了这种和其他夸克衰变的概率。} \label{fig_Quark_5}
\end{figure}
夸克只能通过弱相互作用由一种味转变为另一种味,弱相互作用是粒子物理学的四种基本互相作用力之一。通过吸收或释放W玻色子,任何上型夸克(上,粲和顶夸克)都可以转变成任何下型夸克(下,奇和底夸克),反之亦然。这种味变机制正是导致β衰变的原因,在β衰变过程中,一个中子($n$)“分裂”成一个质子($p$),一个电子($e^{-}$)和一个反电中微子($v_e$)(见右图)。在$\beta$衰变过程中,中子(udd)的一个下夸克释放一个虚 $W^{-}$玻色子后,随机衰变成一上夸克,于是中子就变成了质子(uud)。随后 $W^{-}$玻色子衰变为一个电子和一个反电中微子。[56]
\begin{table}[ht]
\centering
\caption\label{Quark}
\begin{tabular}{|c|c|c|c}
\hline
n & $\rightarrow$ & p & + & $e^{-}$ & + &  $\nu_{e}$ & $\beta$衰变,重子符号) \\
\hline
udd & $\rightarrow$ & uud & + & $e^{-}$ & + & $\nu_{e}$ &$\beta$衰变,夸克符号) \\
\hline
\end{tabular}
\end{table}
$\beta$衰变和其逆过程逆β衰变通常用于医疗,例如正电子发射计算机断层扫描。这两个过程在高能实验中也会涉及,比如中微子检测。

六个夸克之间弱相互作用的强度。线的“强度”由 CKM矩阵元素决定。
虽然所有夸克的味变过程都是相同的,但每个夸克都偏好转换成跟自己同一代的另一夸克。所有味变的这种相对趋势,可以由一个叫做卡比博-小林-益川矩阵 (CKM矩阵)的数学表描述。CKM矩阵内所有数值的大约量值如下:[57]
\begin{equation}
\begin{Bmatrix}
|V_{ud}| & |V_{us}|& |V_{ub}| \\
|V_{cd}| & |V_{cs}| & |V_{cb}| \\
|V_{td}| & |V_{ts}| & |V_{tb}| \\
\end{Bmatrix}
\approx
\begin{Bmatrix}
0.974 & 0.225 & 0.003 \\
0.225 & 0.973 & 0.041 \\
0.009 & 0.040 & 0.999
\end{Bmatrix}~
\end{equation}
其中$V_{ij}$代表夸克味i 变成夸克味j 的可能性(反之亦然)。

轻子(上图$\beta$衰变中在$W$玻色子右侧边的粒子) 也有一个等效的弱相互作用矩阵,称为庞蒂科夫-牧-中川-坂田矩阵 (PMNS矩阵)。[58]CKM矩阵和PMNS矩阵合起来描述了所有的味变,但是两者之间的联系还不清楚。[59]

\subsubsection{4.4 强相互作用与色荷}
\begin{figure}[ht]
\centering
\includegraphics[width=6cm]{./figures/f0e5c74209b4217f.png}
\caption{所有类型的强子的总色荷为零(即白色)。} \label{fig_Quark_8}
\end{figure}

\begin{figure}[ht]
\centering
\includegraphics[width=6cm]{./figures/5f13756c40cadc6f.png}
\caption{夸克、反夸克和八胶子的三种颜色的强电荷模式(两种零电荷重叠)。} \label{fig_Quark_9}
\end{figure}

根据量子色动力学,夸克拥有一种叫做色荷的性质。色荷有三种类型,分别标记为蓝,绿,和红。每种颜色都有一种反色荷——反蓝,反绿,和反红。每个夸克都带有一种色,而每个反夸克都带有一种反色。[60]

夸克之间互相吸引和排斥的系统,是由三种色的各种不同组合引起的,叫强相互作用力,它是由一种叫胶子的规范玻色子来传递的。描述强相互作用的理论叫做量子色动力学。具有单一色荷的夸克可以与携带相反色荷的反夸克形成一个束缚系统。两个相互吸引的夸克会达到色中性:一个夸克带有色荷$\xi$,加上一个带有$-\xi$的夸克,将形成色荷为零(或“白”色)的介子。这种类似于基本光学中的颜色叠加模型,把三个夸克(带有不同色荷)或三个反夸克(带有不同反色荷)的组合结合在一起,将形成相同的“白“色的

带三种颜色不同组合电荷的夸克之间的吸引和排斥系统被称为强相互作用力,这是由携带粒子的力被称为胶子;这将在下面详细讨论。描述强相互作用的理论叫做量子色动力学 (QCD)。具有单一颜色值的夸克可以与携带相应反色的反夸克形成一个束缚系统。两个吸引夸克的结果将是颜色中性:一个带有色荷的夸克$\xi$加上色荷的古董ξ将导致0(或“白色”的色荷)和形成介子。这类似于基本光学中的加色模型。类似地,三个夸克(每个夸克带有不同的色荷)或三个反夸克(每个反夸克带有反色荷)的组合将会生成带有“白色”色荷的重子或反重子。[61]

在现代粒子物理学中,粒子之间的相互作用可以用一种规范对称性的空间对称群(参见规范场论)来描述。色荷SU(3) (通常缩写为$SU(3)_c$是夸克中色荷的规范对称,也是量子色动力学的定义对称。[62]正如物理定律独立于空间方向一样(如x,y和z方向),即使坐标轴旋转到一个新的方向,物理定律保持不变,量子色动力学也一样,其不受三维色空间(如蓝,红和绿)的方向影响。

中的哪些方向被识别为蓝色、红色和绿色无关。$SU(3)_c$的色变换对应于色空间中的“旋转”(从数学上讲,色空间是一个复数空间)。每种夸克味$f$,下面都有三个子类型$f_{B},f_{G},f_{r}$,分别对应于三种夸克色荷,蓝,绿和红,[63]形成一个三重态:一股有三个分量的量子场,并且在变换时遵从$SU(3)_c$的基本表示。[64]这个时候$SU(3)_c$应是局部的,换句话说,就是容许变换随空间及时间而定,所以说,这个局部表示决定了强相互作用的性质,尤其是有八种载力用胶子这一点。[62][65]

\subsubsection{4.5 质量}
夸克的质量可以用两个术语来代指:一个是净夸克质量,也就是夸克本身的质量;另一个是组夸克质量,也就是净夸克质量加上其周围胶子场的质量。[66]这些质量的数值通常不相同。强子的大部分质量来自将夸克结合在一起的胶子,而不是夸克本身。虽然胶子的内在质量为零,但它们拥有能量——更具体地说,量子色动力学束缚能(QCBE)——为强子提供了大部分的质量(见狭义相对论中的质量)。例如,一个质子的质量约为938 MeV/c2,其中三个价夸克的静止质量只有$9MeV/c^2$;其余大部分质量都可以归因于胶子的束缚场能量。[67][68]

标准模型中假设基本粒子的质量都是来自于与希格斯玻色子相关的希格斯机制。顶夸克的质量大约有$173GeV/c^2$,几乎与一个金原子核一样重(~$171GeV/c^2$)。

[67][69]透过研究为什么顶夸克的质量这么大,物理学家希望能找到更多有关夸克以及其他粒子的质量来源。[70]

\begin{figure}[ht]
\centering
\includegraphics[width=6cm]{./figures/6fb0ad7d24b5c349.png}
\caption{六种味夸克的质量比例图。各夸克的质量与其对应的球体积成正比。左下图为对照的质子(灰色)和电子 (红色)的大小。} \label{fig_Quark_10}
\end{figure}

\subsubsection{4.6 大小}
在量子色动力学中,夸克被认为是零尺寸的点状实体。截至2014年,实验证据表明它们不超过$10^{-4}$的质子大小,即小于$10^{-19}$米。[71]

\subsubsection{4.7 味性质列表}

下表总结了六个夸克的主要性质。每种夸克味都有自己的一组味量子数 ( 同位旋($i_3$),粲数($C$),奇异数($S$),顶数($T$)和底部($B$)),它们代表着夸克系统及强子的一些特性。因为重子由三个夸克组成,所以所有夸克的重子数(B)均为+1⁄3。对于反夸克,电荷($Q$)和其他味量子数($B,i_3,C,S,T\text{和 B}$)和夸克的符号相反。质量和总角动量($J$;相当于点粒子的自旋)跟夸克符号一致。

\begin{figure}[ht]
\centering
\includegraphics[width=14.25cm]{./figures/a145346192075f0c.png}
\caption \label{fig_Quark_11}
\end{figure}
$J = \text{总角动量},B = \text{重子数}, Q = \text{电荷},I_3 = \text{同位旋}, C = \text{魅力},S = \text{陌生}, T = \text{顶数}, B = \text{底部}$。

$\ast \text{符号,例如} 173210 \pm 510 , 710$ 表示两种类型的测量不确定度。在顶夸克的情况下,第一个不确定性本质上是统计的,第二个是系统的。

\subsection{相互作用的夸克}
正如量子色动力学所描述的,夸克之间的强相互作用是由胶子传递,胶子是一种无质量的矢量规范玻色子。每个胶子携带一个色荷和一个反色电荷。在粒子相互作用的标准框架(称为微扰理论的更通用公式的一部分)中,胶子通过发射和吸收虚粒子过程在夸克之间不断交换。当胶子在夸克之间转移时,两者都会发生色荷变化;例如,如果一个红色夸克发射出一个红-反绿色胶子,它就变成绿夸克;如果一个绿色夸克吸收了一个红-反绿色胶子,它就变成红夸克。因此,当每个夸克的颜色不断变化时,它们之间的强相互作用得以维持。[72][73][74]

由于胶子携带色荷,所以它们本身能够发射和吸收其他胶子。这导致“渐近自由“:随着夸克越来越靠近,它们之间的色动力学束缚力减弱。[75]相反,随着夸克越来越远离,束缚力增强。色场在受到”应力“影响时变得不稳定,就像橡皮筋拉长时受应力影响而快断开一样,于是色场会自发生成许多合适色荷的胶子,来增强色场。当能量超过某个数值时,色场中就会产生成对的夸克和反夸克。这些对与被分离的夸克结合在一起,形成新的强子。这种现象被称为夸克禁闭:夸克不能孤立出现。[76][77]夸克通常在高等碰撞中产生,在与其他夸克做出相互作用之前,种强子化过程就会发生。唯一的例外是顶夸克,因为在强子化之前就衰变了。[78]

\subsubsection{5.1 海夸克}
除了贡献量子数的价夸克($q_v$)之外,强子内部还有虚夸克-反夸克(qq)对,这些粒子称为海夸克($q_s$)。当强子色场的胶子分裂时,就会产生海夸克;这个过程是可逆的,当两个海夸克湮灭时就会产生一个胶子。于是胶子会不断地分裂和形成,形成所谓的“海”。[79]海夸克比它们的价夸克稳定得多,它们通常在强子内部相互湮灭。尽管如此,在某些情况下,海夸克可以强子化,形成重子或介子类的粒子。[80]

\subsubsection{5.2 夸克物质的其他相}
\begin{figure}[ht]
\centering
\includegraphics[width=10cm]{./figures/6416910e0e4b3bac.png}
\caption{夸克物质相图的定性呈现。当前研究的主题是明确该图表的细节。[10][11]} \label{fig_Quark_12}
\end{figure}
在足够极端的条件下,夸克可能会被解禁闭,并以自由粒子的形式存在。在渐近自由过程中,强相互作用在较高温度下变弱。最终,色禁闭将消失,并形成一股超热的等离子体,由自由移动的夸克和胶子组成。这种物质的理论相叫夸克-胶子浆。[81]需要达到这个相的确切条件,现在仍是未知,但是这方面一直都有不少的猜测和实验。温度需求的估计为$(1.90 \pm 0.02) \times 10^{12}$开尔文。[82]虽然完全自由的夸克和胶子状态从未实现过(尽管欧洲核子研究组织 在20世纪80年代和90年代进行了多次尝试),但是在相对论性重离子对撞机的实验中,有证据证实像液体的夸克物质能够表现出“近乎完美的”流体运动。[83]

夸克-胶子浆的特点是,相对于上、下夸克对的数量,重夸克对的数量大幅增加。科学家们相信在大爆炸(夸克时期)后$10^{-6}$秒之前,宇宙充满了夸克-胶子浆,因为温度太高,重子无法稳定存在。[84]

当重子密度足够高以及温度相对低时——可能与中子星中发现的情况相似——根据理论推测,夸克物质会退化为弱作用夸克的费米液体。这种液体的特点是,它时由色夸克的库柏对凝聚而成的 ,因此会对局部SU(3)c对称性造成破缺。由于库珀对含有色荷,所以夸克物质的这一相叫做色荷超导;也就是说,色荷能够在无色阻的情况下通过它。[85]

\subsection{笔记}
\begin{enumerate}
\item 主要证据基于共振宽度的Z0玻色子这限制了第四代中微子的质量大于~45 $GeV/c^2$。这将与其他三代中微子形成强烈对比,它们的质量不能超过2 $MeV/c^2$。
\item 宇称不守恒定律现象是当左和右交换( P对称)并且粒子被它们相应的反粒子(电荷共轭)取代时,导致弱相互作用表现不同的现象。
\item "Beauty" and "truth" are contrasted in the last lines of Keats' 1819 poem "Ode on a Grecian Urn", and may have been the origin of those names.[86][87][88]
\item 一个夸克衰变到另一个夸克的实际概率是衰变夸克质量、衰变产物质量和CKM矩阵对应元素(以及其他变量)的复杂函数。这个概率与星等的平方成正比(但不相等)($|V\text{颈内}|2$)的相应CKM条目。
\item 尽管有它的名字,色荷与可见光的色谱无关。
\end{enumerate}

\subsection{参考文献}
[1]
^"Quark (subatomic particle)". Encyclopædia Britannica. Retrieved 2008-06-29..

[2]
^R. Nave. "Confinement of Quarks". HyperPhysics. Georgia State University, Department of Physics and Astronomy. Retrieved 2008-06-29..

[3]
^R. Nave. "Bag Model of Quark Confinement". HyperPhysics. Georgia State University, Department of Physics and Astronomy. Retrieved 2008-06-29..

[4]
^R. Nave. "Quarks". HyperPhysics. Georgia State University, Department of Physics and Astronomy. Retrieved 2008-06-29..

[5]
^K. A. Peacock (2008). The Quantum Revolution. Greenwood Publishing Group. p. 125. ISBN 978-0-313-33448-1..

[6]
^B. Povh; C. Scholz; K. Rith; F. Zetsche (2008). Particles and Nuclei. Springer. p. 98. ISBN 978-3-540-79367-0..

[7]
^第6.1节。在P. C. W. Davies (1979). The Forces of Nature. Cambridge University Press. ISBN 978-0-521-22523-6..

[8]
^S.-K. Choi; et al. (Belle Collaboration) (2008). "Observation of a Resonance-like Structure in the π±Ψ′ Mass Distribution in Exclusive B→Kπ±Ψ′ decays". Physical Review Letters. 100 (14): 142001. arXiv:0708.1790. Bibcode:2008PhRvL.100n2001C. doi:10.1103/PhysRevLett.100.142001. PMID 18518023..

[9]
^"Belle Discovers a New Type of Meson" (Press release). KEK. 2007. Archived from the original on 2009-01-22. Retrieved 2009-06-20..

[10]
^R. Aaij; et al. (LHCb collaboration) (2014). "Observation of the Resonant Character of the Z(4430)− State". Physical Review Letters. 112 (22): 222002. arXiv:1404.1903. Bibcode:2014PhRvL.112v2002A. doi:10.1103/PhysRevLett.112.222002. PMID 24949760..

[11]
^R. Aaij; et al. (LHCb collaboration) (2015). "Observation of J/ψp Resonances Consistent with Pentaquark States in Λ0 b→J/ψK−p Decays". Physical Review Letters. 115 (7): 072001. arXiv:1507.03414. Bibcode:2015PhRvL.115g2001A. doi:10.1103/PhysRevLett.115.072001. PMID 26317714..

[12]
^C. Amsler; et al. (Particle Data Group) (2008). "Review of Particle Physics: b′ (4th Generation) Quarks, Searches for" (PDF). Physics Letters B. 667 (1): 1–1340. Bibcode:2008PhLB..667....1A. doi:10.1016/j.physletb.2008.07.018..

[13]
^C. Amsler; et al. (Particle Data Group) (2008). "Review of Particle Physics: t′ (4th Generation) Quarks, Searches for" (PDF). Physics Letters B. 667 (1): 1–1340. Bibcode:2008PhLB..667....1A. doi:10.1016/j.physletb.2008.07.018..

[14]
^D. Decamp; et al. (ALEPH Collaboration) (1989). "Determination of the Number of Light Neutrino Species" (PDF). Physics Letters B. 231 (4): 519. Bibcode:1989PhLB..231..519D. doi:10.1016/0370-2693(89)90704-1..

[15]
^A. Fisher (1991). "Searching for the Beginning of Time: Cosmic Connection". Popular Science. 238 (4): 70..

[16]
^J. D. Barrow (1997) [1994]. "The Singularity and Other Problems". The Origin of the Universe (Reprint ed.). Basic Books. ISBN 978-0-465-05314-8..

[17]
^D. H. Perkins (2003). Particle Astrophysics. Oxford University Press. p. 4. ISBN 978-0-19-850952-3..

[18]
^G. Zweig (1964). "An SU(3) Model for Strong Interaction Symmetry and its Breaking" (PDF). CERN Report No.8182/TH.401..

[19]
^G. Zweig (1964). "An SU(3) Model for Strong Interaction Symmetry and its Breaking: II". CERN Report No.8419/TH.412..

[20]
^M. Gell-Mann (1964). "A Schematic Model of Baryons and Mesons". Physics Letters. 8 (3): 214–215. Bibcode:1964PhL.....8..214G. doi:10.1016/S0031-9163(64)92001-3..

[21]
^B. Carithers; P. Grannis (1995). "Discovery of the Top Quark" (PDF). Beam Line. 25 (3): 4–16. Retrieved 2008-09-23..

[22]
^Y. Ne'eman (2000) [1964]. "Derivation of Strong Interactions from Gauge Invariance". In M. Gell-Mann, Y. Ne'eman. The Eightfold Way. Westview Press. ISBN 978-0-7382-0299-0. 原创Y. Ne'eman (1961). "Derivation of Strong Interactions from Gauge Invariance". Nuclear Physics. 26 (2): 222. Bibcode:1961NucPh..26..222N. doi:10.1016/0029-5582(61)90134-1..

[23]
^R. C. Olby; G. N. Cantor (1996). Companion to the History of Modern Science. Taylor & Francis. p. 673. ISBN 978-0-415-14578-7..

[24]
^A. Pickering (1984). Constructing Quarks. University of Chicago Press. pp. 114–125. ISBN 978-0-226-66799-7..

[25]
^B. J. Bjorken; S. L. Glashow (1964). "Elementary Particles and SU(4)". Physics Letters. 11 (3): 255–257. Bibcode:1964PhL....11..255B. doi:10.1016/0031-9163(64)90433-0..

[26]
^E. D. Bloom; et al. (1969). "High-Energy Inelastic e–p Scattering at 6° and 10°". Physical Review Letters. 23 (16): 930–934. Bibcode:1969PhRvL..23..930B. doi:10.1103/PhysRevLett.23.930..

[27]
^R. P. Feynman (1969). "Very High-Energy Collisions of Hadrons" (PDF). Physical Review Letters. 23 (24): 1415–1417. Bibcode:1969PhRvL..23.1415F. doi:10.1103/PhysRevLett.23.1415..

[28]
^S. Kretzer; H. L. Lai; F. I. Olness; W. K. Tung (2004). "CTEQ6 Parton Distributions with Heavy Quark Mass Effects". Physical Review D. 69 (11): 114005. arXiv:hep-ph/0307022. Bibcode:2004PhRvD..69k4005K. doi:10.1103/PhysRevD.69.114005..

[29]
^D. J. Griffiths (1987). Introduction to Elementary Particles. John Wiley & Sons. p. 42. ISBN 978-0-471-60386-3..

[30]
^M. E. Peskin; D. V. Schroeder (1995). An Introduction to Quantum Field Theory. Addison–Wesley. p. 556. ISBN 978-0-201-50397-5..

[31]
^V. V. Ezhela (1996). Particle Physics. Springer. p. 2. ISBN 978-1-56396-642-2..

[32]
^S. L. Glashow; J. Iliopoulos; L. Maiani (1970). "Weak Interactions with Lepton–Hadron Symmetry". Physical Review D. 2 (7): 1285–1292. Bibcode:1970PhRvD...2.1285G. doi:10.1103/PhysRevD.2.1285..

[33]
^D. J. Griffiths (1987). Introduction to Elementary Particles. John Wiley & Sons. p. 44. ISBN 978-0-471-60386-3..

[34]
^M. Kobayashi; T. Maskawa (1973). "CP-Violation in the Renormalizable Theory of Weak Interaction". Progress of Theoretical Physics. 49 (2): 652–657. Bibcode:1973PThPh..49..652K. doi:10.1143/PTP.49.652. hdl:2433/66179. Archived from the original on 2008-12-24..

[35]
^S. W. Herb; et al. (1977). "Observation of a Dimuon Resonance at 9.5 GeV in 400-GeV Proton-Nucleus Collisions". Physical Review Letters. 39 (5): 252. Bibcode:1977PhRvL..39..252H. doi:10.1103/PhysRevLett.39.252..

[36]
^M. Bartusiak (1994). A Positron named Priscilla. National Academies Press. p. 245. ISBN 978-0-309-04893-4..

[37]
^F. Abe; et al. (CDF Collaboration) (1995). "Observation of Top Quark Production in pp Collisions with the Collider Detector at Fermilab". Physical Review Letters. 74 (14): 2626–2631. arXiv:hep-ex/9503002. Bibcode:1995PhRvL..74.2626A. doi:10.1103/PhysRevLett.74.2626. PMID 10057978..

[38]
^S. Abachi; et al. (DØ Collaboration) (1995). "Search for High Mass Top Quark Production in pp Collisions at √s = 1.8 TeV". Physical Review Letters. 74 (13): 2422–2426. arXiv:hep-ex/9411001. Bibcode:1995PhRvL..74.2422A. doi:10.1103/PhysRevLett.74.2422. PMID 10057924..

[39]
^K. W. Staley (2004). The Evidence for the Top Quark. Cambridge University Press. p. 144. ISBN 978-0-521-82710-2..

[40]
^"New Precision Measurement of Top Quark Mass". Brookhaven National Laboratory News. 2004. Retrieved 2013-11-03..

[41]
^J. Joyce (1982) [1939]. Finnegans Wake. Penguin Books. p. 383. ISBN 978-0-14-006286-1..

[42]
^S. Pronk-Tiethoff (2013). The Germanic loanwords in Proto-Slavic. Rodopi. p. 71. ISBN 978-9401209847..

[43]
^"What Does 'Quark' Have to Do with Finnegans Wake?". Merriam-Webster. Retrieved 2018-01-17..

[44]
^G. E. P. Gillespie. "Why Joyce Is and Is Not Responsible for the Quark in Contemporary Physics" (PDF). Papers on Joyce 16. Retrieved 2018-01-17..

[45]
^M. Gell-Mann (1995). The Quark and the Jaguar: Adventures in the Simple and the Complex. Henry Holt and Co. p. 180. ISBN 978-0-8050-7253-2..

[46]
^J. Gleick (1992). Genius: Richard Feynman and Modern Physics. Little Brown and Company. p. 390. ISBN 978-0-316-90316-5..

[47]
^J. J. Sakurai (1994). S. F. Tuan, ed. Modern Quantum Mechanics (Revised ed.). Addison–Wesley. p. 376. ISBN 978-0-201-53929-5..

[48]
^D. H. Perkins (2000). Introduction to High Energy Physics. Cambridge University Press. p. 8. ISBN 978-0-521-62196-0..

[49]
^M. Riordan (1987). The Hunting of the Quark: A True Story of Modern Physics. Simon & Schuster. p. 210. ISBN 978-0-671-50466-3..

[50]
^H. Harari (1975). "A New Quark Model for hadrons". Physics Letters B. 57 (3): 265. Bibcode:1975PhLB...57..265H. doi:10.1016/0370-2693(75)90072-6..

[51]
^J. T. Volk; et al. (1987). "Letter of Intent for a Tevatron Beauty Factory" (PDF). Fermilab Proposal #783..
[52]
^C. Quigg (2006). "Particles and the Standard Model". In G. Fraser. The New Physics for the Twenty-First Century. Cambridge University Press. p. 91. ISBN 978-0-521-81600-7..

[53]
^M. Munowitz (2005). Knowing. Oxford University Press. p. 35. ISBN 978-0-19-516737-5..

[54]
^F. Close (2006). The New Cosmic Onion. CRC Press. pp. 80–90. ISBN 978-1-58488-798-0..

[55]
^D. Lincoln (2004). Understanding the Universe. World Scientific. p. 116. ISBN 978-981-238-705-9..

[56]
^"Weak Interactions". Virtual Visitor Center. Stanford Linear Accelerator Center. 2008. Retrieved 2008-09-28..

[57]
^K. Nakamura; et al. (Particle Data Group) (2010). "Review of Particles Physics: The CKM Quark-Mixing Matrix" (PDF). Journal of Physics G. 37 (7A): 075021. Bibcode:2010JPhG...37g5021N. doi:10.1088/0954-3899/37/7A/075021..

[58]
^Z. Maki; M. Nakagawa; S. Sakata (1962). "Remarks on the Unified Model of Elementary Particles". Progress of Theoretical Physics. 28 (5): 870. Bibcode:1962PThPh..28..870M. doi:10.1143/PTP.28.870..

[59]
^B. C. Chauhan; M. Picariello; J. Pulido; E. Torrente-Lujan (2007). "Quark–Lepton Complementarity, Neutrino and Standard Model Data Predict θPMNS13 = 9°+1°−2°". European Physical Journal. C50 (3): 573–578. arXiv:hep-ph/0605032. Bibcode:2007EPJC...50..573C. doi:10.1140/epjc/s10052-007-0212-z..

[60]
^R. Nave. "The Color Force". HyperPhysics. Georgia State University, Department of Physics and Astronomy. Retrieved 2009-04-26..

[61]
^B. A. Schumm (2004). Deep Down Things. Johns Hopkins University Press. pp. 131–132. ISBN 978-0-8018-7971-5..

[62]
^第三部分M. E. Peskin; D. V. Schroeder (1995). An Introduction to Quantum Field Theory. Addison–Wesley. ISBN 978-0-201-50397-5..

[63]
^V. Icke (1995). The Force of Symmetry. Cambridge University Press. p. 216. ISBN 978-0-521-45591-6..

[64]
^M. Y. Han (2004). A Story of Light. World Scientific. p. 78. ISBN 978-981-256-034-6..

[65]
^C. Sutton. "Quantum Chromodynamics (physics)". Encyclopædia Britannica Online. Retrieved 2009-05-12..

[66]
^A. Watson (2004). The Quantum Quark. Cambridge University Press. pp. 285–286. ISBN 978-0-521-82907-6..

[67]
^K. A. Olive; et al. (Particle Data Group) (2014). "Review of Particle Physics". Chinese Physics C. 38 (9): 090001. Bibcode:2014ChPhC..38i0001O. doi:10.1088/1674-1137/38/9/090001..

[68]
^W. Weise; A. M. Green (1984). Quarks and Nuclei. World Scientific. pp. 65–66. ISBN 978-9971-966-61-4..

[69]
^D. McMahon (2008). Quantum Field Theory Demystified. McGraw–Hill. p. 17. ISBN 978-0-07-154382-8..

[70]
^S. G. Roth (2007). Precision Electroweak Physics at Electron–Positron Colliders. Springer. p. VI. ISBN 978-3-540-35164-1..

[71]
^比小:寻找LHC的新东西PBS Nova博客2014年10月28日.

[72]
^R. P. Feynman (1985). QED: The Strange Theory of Light and Matter (1st ed.). Princeton University Press. pp. 136–137. ISBN 978-0-691-08388-9..

[73]
^M. Veltman (2003). Facts and Mysteries in Elementary Particle Physics. World Scientific. pp. 45–47. ISBN 978-981-238-149-1..

[74]
^F. Wilczek; B. Devine (2006). Fantastic Realities. World Scientific. p. 85. ISBN 978-981-256-649-2..

[75]
^F. Wilczek; B. Devine (2006). Fantastic Realities. World Scientific. pp. 400ff. ISBN 978-981-256-649-2..

[76]
^M. Veltman (2003). Facts and Mysteries in Elementary Particle Physics. World Scientific. pp. 295–297. ISBN 978-981-238-149-1..

[77]
^T. Yulsman (2002). Origin. CRC Press. p. 55. ISBN 978-0-7503-0765-9..

[78]
^F. Garberson (2008). "Top Quark Mass and Cross Section Results from the Tevatron". arXiv:0808.0273 [hep-ex]..

[79]
^J. Steinberger (2005). Learning about Particles. Springer. p. 130. ISBN 978-3-540-21329-1..

[80]
^C.-Y. Wong (1994). Introduction to High-energy Heavy-ion Collisions. World Scientific. p. 149. ISBN 978-981-02-0263-7..

[81]
^S. Mrowczynski (1998). "Quark–Gluon Plasma". Acta Physica Polonica B. 29 (12): 3711. arXiv:nucl-th/9905005. Bibcode:1998AcPPB..29.3711M..

[82]
^Z. Fodor; S. D. Katz (2004). "Critical Point of QCD at Finite T and μ, Lattice Results for Physical Quark Masses". Journal of High Energy Physics. 2004 (4): 50. arXiv:hep-lat/0402006. Bibcode:2004JHEP...04..050F. doi:10.1088/1126-6708/2004/04/050..

[83]
^"RHIC Scientists Serve Up "Perfect" Liquid". Brookhaven National Laboratory. 2005. Archived from the original on 2013-04-15. Retrieved 2009-05-22..

[84]
^T. Yulsman (2002). Origins: The Quest for Our Cosmic Roots. CRC Press. p. 75. ISBN 978-0-7503-0765-9..

[85]
^A. Sedrakian; J. W. Clark; M. G. Alford (2007). Pairing in Fermionic Systems. World Scientific. pp. 2–3. ISBN 978-981-256-907-3..

[86]
^Rolnick, William B. (31 May 2003). Remnants Of The Fall: Revelations Of Particle Secrets. World Scientific Pub Co Inc. p. 136. ISBN 978-9812380609. Retrieved 14 October 2018..

[87]
^Mee, Nicholas (24 August 2012). Higgs Force: Cosmic Symmetry Shattered. Quantum Wave Publishing. ISBN 978-0957274617. Retrieved 14 October 2018..

[88]
^Gooden, Philip (3 November 2016). May We Borrow Your Language?: How English Steals Words From All Over the World. Head of Zeus. ISBN 978-1784977986. Retrieved 14 October 2018..
