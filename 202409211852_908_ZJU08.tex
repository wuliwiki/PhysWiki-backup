% 浙江大学 2008 年 考研 量子力学
% license Usr
% type Note

\textbf{声明}:“该内容来源于网络公开资料,不保证真实性,如有侵权请联系管理员”

\subsection{第一题:简答题(30分)}
(1) 从正则对易关系 $[x_i, \hat{p}_j] = i\hbar \delta_{ij}$ 推出角动量算符的对易关系;

(2) 用测不准关系估算氢原子的基态能量;

(3) 什么是量子跃迁?什么是选择定则?线偏振光和圆偏振光照射下的选择定则有什么区别?

(4) 什么是塞曼效应?什么是斯塔克效应?

(5) 什么是受激辐射?什么是光电效应?

\subsection{第二题:(25分)}
设电子以给定的能量 $E = \frac{\hbar^2 k^2}{2m}$ 自左入射,遇到一个方势阱
\[V(x) = \begin{cases} 0 & x < 0, x > a \\\\- V_0 & 0 \leq x \leq a\end{cases}~\]