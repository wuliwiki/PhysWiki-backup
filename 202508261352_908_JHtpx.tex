% 几何拓扑学(综述)
% license CCBYSA3
% type Wiki

本文根据 CC-BY-SA 协议转载翻译自维基百科\href{https://en.wikipedia.org/wiki/Geometric_topology}{相关文章}

\begin{figure}[ht]
\centering
\includegraphics[width=6cm]{./figures/d945e3a478caa898.png}
\caption{} \label{fig_JHtpx_1}
\end{figure}
在数学中,几何拓扑学研究的是流形及其之间的映射,尤其是一个流形嵌入到另一个流形中的情形。
\subsection{历史}
作为一个独立于代数拓扑的研究领域,几何拓扑学可以追溯到1935 年,当时通过Reidemeister 扭结对透镜空间进行了分类。这项工作首次要求区分那些同伦等价但不同胚的空间,也由此催生了简单同伦理论。“几何拓扑学”这一术语用来描述这一研究方向则是相对较近才出现的用法。\(^\text{[1]}\)
\subsection{低维拓扑与高维拓扑的区别}
流形在高维与低维中的行为存在显著差异。高维拓扑通常指维度 5 及以上的流形,或者从相对角度来看,指余维数3 及以上的嵌入问题。低维拓扑主要研究4 维及以下**的流形,或者余维数不超过 2 的嵌入问题。


四维流形具有特殊性:从某些角度(如拓扑结构)看,四维表现得像高维;而从其他角度(如可微结构)看,四维又表现得像低维。这种双重特性导致了许多四维独有的现象,例如 $\mathbb{R}^4$ 上的奇异可微结构。因此,四维流形的拓扑分类在理论上是可处理的,其核心问题包括:一个拓扑流形是否承认可微结构?如果承认,可能存在多少种不同的可微结构?值得注意的是,光滑四维情形仍然是广义庞加莱猜想的最后一个未解案例,可参见Gluck 扭转。

这种差异主要源于手术理论的适用范围:在五维及更高维度,手术理论适用;事实上,在某些情况下,它在四维的拓扑层面上也适用,但证明过程非常复杂。在四维及以下维度(拓扑意义上,三维及以下),手术理论则无法使用。

因此,研究低维流形时,一种常见的思路是:假设手术理论在低维情形下也成立,它会预测哪些结果?然后通过比较实际情况与这种预测的差异,来理解低维流形中那些偏离高维规律的特殊现象。
\begin{figure}[ht]
\centering
\includegraphics[width=6cm]{./figures/d454ad50d06c800e.png}
\caption{} \label{fig_JHtpx_2}
\end{figure}\begin{figure}[ht]
\centering
\includegraphics[width=6cm]{./figures/d454ad50d06c800e.png}
\caption{} \label{fig_JHtpx_3}
\end{figure}