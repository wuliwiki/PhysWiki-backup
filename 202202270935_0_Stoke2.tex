% 广义斯托克斯定理(简明微积分)

\pentry{微分形式(简明微积分)\upref{DForm}}

若一个 $\omega$ 是一个 $k$ 微分形式, $\mathcal V$ 是 $N> k$ 维空间中的曲面, 其表面为 $\partial \mathcal V$
\begin{equation}\label{Stoke2_eq1}
\int_{\mathcal V} \dd{\omega} = \int_{\partial \mathcal V} \omega
\end{equation}

\addTODO{举例子: 二维,三维的散度定理, 三维斯托克斯定理}

\begin{example}{二维散度定理}
二维平面上的一个区域, 其边界取逆时针为正, 线积分的 $1$-微分形式为
\begin{equation}
\omega = f_x \dd{x} + f_y \dd{y}
\end{equation}
它的微分为
\begin{equation}
\dd{\omega} = \pdv{f_x}{x}\dd{x}\dd{x} + \pdv{f_x}{y}\dd{y}\dd{x}
+ \pdv{f_y}{x}\dd{x}\dd{y} + \pdv{f_y}{y}\dd{y}\dd{y}
\end{equation}
其中第一项和第四项出现了重复, 故为零. 第二项中 $\dd{y}\dd{x} = -\dd{x}\dd{y}$, 所以
\begin{equation}
\dd{\omega} = \qty(\pdv{f_y}{x}-\pdv{f_x}{y})\dd{x}\dd{y}
\end{equation}
代入\autoref{Stoke2_eq1} 得
\begin{equation}
\oint f_x \dd{x} + f_y \dd{y} = \iint \qty(\pdv{f_y}{x}-\pdv{f_x}{y})\dd{x}\dd{y}
\end{equation}
这样就得到了二维的散度定理(链接未完成). 这也可以看作是二维的旋度定理.
\end{example}

\begin{example}{三维旋度定理}
微分形式为
\begin{equation}
\omega = 
\end{equation}

\end{example}
