% 南京理工大学 普通物理 B(845)模拟五套卷 第三套
% license Usr
% type Note

\textbf{声明}:“该内容来源于网络公开资料,不保证真实性,如有侵权请联系管理员”

\subsection{一、 填空题 I(26 分,每空 2 分)}
1. 一质点的运动方程(SI)为: ,则质点的起始速度为__________,质点加速度为____________。

2. 质量为 m,长为 l 的匀质细杆,可绕其端点的水平轴在竖直平面内自由转动。如果将细杆置于水平位置,然后让其由静止开始自由下摆,则开始转动的瞬间,细杆的角加速度为_____________,细杆转动到竖直位置时的角速度为_________________。

3. 如图所示,一长为 l 的均匀直棒可绕过其一端且与棒垂直的水平光滑固定轴转动。抬起另一端使棒向上与水平面成 60°,然后无初速地将棒释放。已知棒对轴的转动惯量为 ,其中 m 和 l 分别为棒的质量和长度,则放手时棒的角加速度为_____________,棒转到水平位置时的角加速度为_________________。

4. 已知一平面简谐波频率为 1000Hz,波速为 300m/s,则波上相差 π/4的两点之间的距离为_______________,在某点处时间间隔为 0.001s的两个振动状态间的相位差为________________。
5. 一质点作简谐振动,速度最大值 Vm=5cm/s,振幅 A=2cm,若令速度
具有正最大值的那一时刻为 t=0,则振动表达式为_____________。
6. 互感系数的物理意义是______________。
7. 在容积为 10-2m3 的容易中,装有质量为 100g 的气体,若气体分子的
方均根速率为 300mS-1,则气体的压强为____________。