% 玻尔兹曼分布(统计力学)
% 玻尔兹曼分布|等概率原理|经典统计

\pentry{麦克斯韦—玻尔兹曼分布\upref{MxwBzm},相空间\upref{PhSpace},理想气体单粒子能级密度\upref{IdED1},拉格朗日乘数法\upref{LagMul}}

根据等概率原理,对于平衡状态的孤立系统,每个可能的微观状态出现的概率是相等的.

根据统计力学中的量子力学假设,系统中单粒子态能级为分立的:$\epsilon_1,\epsilon_2,\cdots$.其中设第 $l$ 个能级的简并数为 $\omega_l$(意味着这个能级一共有 $\omega_l$ 种线性无关的态).设第 $l$ 个能级上共有 $a_l$ 个粒子,序列 $\{a_l\}$ 构成粒子系统的一种分布.微观状态数最多的分布出现的概率最大,称为\textbf{最概然分布}.

注意等概率原理中只涉及“可能出现”的微观状态,也就是说要满足孤立系统的粒子数守恒、能量守恒条件:
\begin{equation}\label{MBsta_eq1}
\begin{aligned}
\sum_l n_l=N\\
\sum_l \epsilon_l a_l=E
\end{aligned}
\end{equation}

我们下面要谈的是玻尔兹曼分布,它\textbf{不涉及全同粒子假设},即粒子之间是可区分的.我们将得到同经典统计中一样的结果\upref{MxwBzm}.

\subsection{初步推导}
\addTODO{文字说明}
\begin{equation}
\Omega=\frac{N!}{\Pi_l a_l!}\omega_l^{a_l}
\end{equation}

\begin{equation}
\ln \Omega=\ln N!-\sum_{l}a_l!+\sum_l a_l\ln \omega_l
\end{equation}

假设所有的 $a_l$ 都很大,根据近似等式 $\ln n! = n(\ln n-1)$,可以化简得到

\begin{equation}
\ln \Omega=N\ln N-\sum_l a_l\ln a_l+\sum_l a_l\ln \omega_l
\end{equation}

在约束条件\autoref{MBsta_eq1} 下,要求 $\ln \Omega$ 极大的分布 $\{a_l\}$,即 $\delta \ln \Omega =0$.由于在约束条件下 $\delta a_l$ 并非独立,需要利用拉格朗日乘数法,引入参量 $\alpha,\beta$,
\begin{equation}
\delta \ln \Omega -\alpha \delta N-\beta \delta E=-\sum_l [\ln a_l-\ln \omega_l+\alpha +\beta\epsilon_l]\delta a_l=0
\end{equation}

所以
\begin{equation}
\ln \frac{a_l}{\omega_l}+\beta\epsilon_l = 0
\end{equation}
求得最概然分布:
\begin{equation}
a_l=\omega_l e^{-\alpha -\beta \epsilon_l}
\end{equation}
这种分布出现的概率最大.下面我们来确定 $\alpha,\beta$ 的值.现在,我们令这种分布就是玻尔兹曼分布,并大胆地假设其他分布出现的概率为 $0$\footnote{这实际上是不正确的假设,根据等概率原理,其他可能的分布的概率总 $>0$,粒子状态分布总会有一定涨落.但当我们考虑 $N$ 很大的系统,其他分布出现的概率将远小于最概然分布出现的概率.所以这个假设是合理的.}.$\omega_l$ 是简并度,那么对于单个量子态,粒子数为 $e^{-\alpha-\beta \epsilon_l}$.现在我们列举一切能级中的一切量子态,它们的能级依次是 $\{\epsilon_s\}$.我们有
\begin{equation}\label{MBsta_eq2}
\begin{aligned}
N=\sum_s e^{-\alpha-\beta \epsilon_s}\\
E=\sum_s \epsilon_s e^{-\alpha-\beta \epsilon_s}
\end{aligned}
\end{equation}
\subsection{玻尔兹曼分布}
我们定义配分函数 $Z(\beta)=\sum_s e^{-\beta \epsilon_s}$,求和的时候 $s$ 覆盖了每一个能级的一切量子态.由 \autoref{MBsta_eq2},可以得到 $E,N$ 和配分函数 $Z$ 的关系:
\begin{equation}
\begin{aligned}
N=e^{-\alpha}Z\\
E=-e^{-\alpha}\frac{\partial Z}{\partial \beta}
\end{aligned}
\end{equation}

化简得

\begin{equation}
\begin{aligned}
E=-N\frac{\partial \ln Z}{\partial \beta}
\end{aligned}
\end{equation}

根据量子力学,体积为 $V$ 的容器中,单粒子的能级为

\begin{equation}
\varepsilon = \frac{\hbar ^2}{2m} \qty[\qty(\frac{\pi n_x}{L_x})^2 + \qty(\frac{\pi n_y}{L_y})^2 + \qty(\frac{\pi n_z}{L_z})^2] = \frac{\hbar ^2}{2m} (k_x^2 + k_y^2 + k_z^2)
\end{equation}

$n_x,n_y,n_z$ 可以取一切正整数.计算配分函数:
\begin{equation}
Z\approx \int_{-\infty}^\infty\int_{-\infty}^\infty\int_{-\infty}^\infty\dd n_x\dd n_y\dd n_z  e^{-\beta \hbar^2[(\pi n_x/L_x)^2]/2m-\beta \hbar^2/2m-\beta \hbar^2(\pi n_z/L_z)^2/2m}
\end{equation}


\begin{aligned}
\cdots
\beta=\frac{1}{kT}
\end{aligned}

