% Lapack 笔记

\begin{issues}
\issueDraft
\end{issues}

\pentry{BLAS 简介\upref{BLAS}}

Intel MKL 提供的 \href{https://www.intel.com/content/www/us/en/developer/tools/oneapi/onemkl-function-finding-advisor.html}{Lapack 函数搜索}.

如果有 Driver 就用尽量用, 其次再选 Computational

问题类型
\begin{itemize}
\item linear equations/system of linear equations 线性方程组
\item nonsymmetric eigenvalue problems 非对称或厄米矩阵本征值问题
\item symmetric eigenvalue problems 对称或厄米矩阵本征值问题
\item generalized symmetric-definite eigenvalue problems 广义对称或厄米正定矩阵本征值问题
\item generalized nonsymmetric eigenvalue problems 广义非对称或厄米正定矩阵本征值问题
\item linear least square (LLS) problems 线性最小二乘法问题
\item generalized LLS problems 广义线性最小二乘法问题
\item singular value decomposition 奇异值分解
\item cosine-sine decomposition 余弦-正弦分解
\end{itemize}

矩阵类型
\begin{itemize}
\item 对称矩阵
\item 厄米矩阵
\item 正交矩阵
\item 酉矩阵
\item 带对角矩阵
\item 三对角矩阵
\end{itemize}

\begin{figure}[ht]
\centering
\includegraphics[width=9cm]{./figures/Lapack_1.png}
\caption{packed storage 把上半或下半三角矩阵存成一行} \label{Lapack_fig1}
\end{figure}

以下是一些分类搜索结果

\subsection{线性方程组}
\subsubsection{Driver - 线性方程组 - 一般矩阵}
\begin{itemize}
\item \verb|?gesv| 解方矩阵线性方程组, 多个 RHS
\item \verb|?gesvx| gesv 并提供误差
\item \verb|?gesvxx| 用额外循环减小 gesv 的误差
\item \verb|?gbsv| gesv 的带对角矩阵版本
\item \verb|?gbsvx| gbsv 并提供误差
\item \verb|?gbsvxx| 用额外循环减小 gbsv 的误差
\item \verb|?gtsv| gesv 的三对角矩阵版本
\item \verb|?gtsvx| gtsv 并提供误差
\end{itemize}


\subsection{对称本征方程}
\subsubsection{Driver - 实对称矩阵本征方程 - 所有本征值和本征矢(可选)}
\begin{itemize}
\item \verb|?syev| 对称矩阵的本征值和本征矢(可选)
\item \verb|?syevd| syev 使用 divide and conquer 算法
\item \verb|?spev| syev 用 packed storage
\item \verb|?spevd| spev 用 packed storage
\item \verb|?sbev| syev 用带对角矩阵
\item \verb|?sbevd| sbev 用 divide and conquer 算法
\item \verb|?stev| syev 使用三对角矩阵
\end{itemize}

\subsection{线性最小二乘法}
\begin{itemize}
\item \verb|?gels| Uses QR or LQ factorization to solve a overdetermined or underdetermined linear system with full rank matrix.
\item \verb|?gelsy| Computes the minimum-norm solution to a linear least squares problem using a complete orthogonal factorization of A.
\item \verb|?gelss| Computes the minimum-norm solution to a linear least squares problem using the singular value decomposition of A.
\item \verb|?gelsd| Computes the minimum-norm solution to a linear least squares problem using the singular value decomposition of A and a divide and conquer method.
\end{itemize}

\subsection{编译 Netlib 的 reference LAPACK}
\begin{itemize}
\item 一般用 (C)BLAS 也需要用 LAPACK(E), 所以可以直接用 LAPACK 的源码编译, 见 \href{https://github.com/Reference-LAPACK/lapack/}{github}.
\item 编译时一些有用的选项如
\begin{lstlisting}[language=none]
cmake -DBUILD_INDEX64=OFF -DBUILD_SHARED_LIBS=ON \
    -DLAPACKE=ON -DCBLAS=ON  ../lapack-3.*/
\end{lstlisting}
\item \verb|BUILD_INDEX64| 选择是使用 32bit 的整数还是 64bit 的.
\item \verb|BUILD_SHARED_LIBS| 选择是编译 \verb|.so| 库还是 \verb|.a| 库.
\item \verb|LAPACKE| 和 \verb|CBLAS| 选择是否编译 C 接口.
\end{itemize}
