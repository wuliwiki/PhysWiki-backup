% 点集的内部、外部和边界
\pentry{拓扑空间\upref{Topol}}
拓扑空间定义了什么样的集合算开集,而开集的补集被称为闭集.除此之外,还有一些集合是既不开也不闭的.比如说,在$\mathcal{R}$上定义的度量拓扑中,开区间都是开集,闭区间和孤立点都是闭集,但是半开半闭区间两者都不属于.

\subsection{基本概念}
\begin{definition}{}
给定拓扑空间$(X, \mathcal{T})$以及它的一个子集$A$.
\begin{itemize}
\item 如果对于$x\in X$,存在一个开集$V\in \mathcal{T}$使得$x\in V\subset A$,那么称这个$x$是$A$的一个\textbf{内点(interior point)}.
\item 如果存在一个开集$V\in \mathcal{T}$使得$x\in V\subset A^C$,那么称这个$x$是$A$的一个\textbf{外点(exterior point)}.
\item 如果一个$x$既不是$A$的外点也不是内点,那么称$x$是$A$的一个\textbf{边界点(boarder point)}.

\item 全体内点构成的集合,称为$A$的\textbf{内部(interior)},记为$A^\circ$.
\item 全体外点构成的集合,称为$A$的\textbf{外部(exterior)}.
\item 全体边界点构成的集合,称为$A$的\textbf{边界(boarder)}.
\item 如果包含点$x\in X$的任何开集$U_x$,都含有和$x$不相同的、$A$的元素:$(U_x-\{x\})\cap A\not=\varnothing$,那么我们说$x$是$A$的一个\textbf{聚点}或者\textbf{极限点(limit point)}.
\end{itemize}
\end{definition}

外点就是“补集的内点”.

聚点的概念在数学分析和高等微积分里就会出现.在微积分中,一个集合$A\subset\mathbb{R}$的聚点$x$,就是不管取多小的$r>0$作为半径,总有某个$A$中的点$x_r$落在$(x-r, x+r)$里面.$A$自己的点$x\in A$当然是$A$的聚点,但是$x\not\in A$也可以是聚点.比如说,区间$(0,1)$的全体聚点构成的集合,就是$[0,1]$.

有了这些定义,我们就可以更直观地理解开集和闭集了:

\begin{exercise}{}\label{Topo0_exe1}
给定$(X, \mathcal{T})$以及它的一个子集$A$.那么,$A$是开集当且仅当$A$的点都是内点;$A$是闭集当且仅当$A$的聚点都在$A$里.证明留做习题.
\end{exercise}

同样地,使用区间为例子来理解开集、闭集和内点、聚点的关系会非常直观.

由定义,对于任何拓扑空间的子集$A$,外点都不是聚点,但边界点和内点都是聚点;内点都是$A$的点,但边界点不一定是.这么一来,计算边界点的方法,就是求出所有聚点,再减去所有内点即可.

对于$A$,它的聚点构成的集合被称为$A$的\textbf{导出集(induced set)}或者\textbf{导集},在本书中记为$A'$\footnote{有的教材中,$A'$代表的是补集,但在本书中我们使用$A^C$表示补集了,于是$A'$可以腾出来表示导集.}.

\begin{example}{}
\begin{itemize}
\item 令$A=\{\frac{1}{n}|n\in\mathbb{Z}^+\}$,那么$A'=\{0\}$.
\item 令$A=(0,1)\cup(1,2)$,那么$A'=[0,2]$
\end{itemize}
\end{example}

\begin{exercise}{}
证明:任何集合的导集都是闭集.
\end{exercise}

\subsection{闭包}

给定任意拓扑空间$X$的任意子集$A$,那么$A^\circ$是$A$所包含的最大的开集.对称地,包含$A$的最小的闭集被称为$A$的\textbf{闭包(closure)},记为$\bar{A}$.闭包很容易得到:

\begin{theorem}{}
给定任意拓扑空间$X$的任意子集$A$,则有$\bar{A}=A\cup A'$.
\end{theorem}
\begin{corollary}{}
考虑到$A'-A^\circ$是$A$的边界,故$\bar{A}$等于$A$和$A$的边界的并.
\end{corollary}

以区间作为例子:如果$A=(0,1)$,那么$A'=[0,1]$,$\bar{A}=[0,1]$.

以圆盘作为例子:如果$D=\{(x,y)\in\mathbb{R}^2|x^2+y^2<1\}$,那么$D'=\bar{D}=\{(x,y)\in\mathbb{R}^2|x^2+y^2\leq1\}$.尝试在$x^2+y^2=1$的圆盘上添加部分点进$D$,得到的$D$还是具有相同的闭包.这是因为这些添加的点都是在最初的$D$的边界上的.

\subsection{稀疏和稠密}