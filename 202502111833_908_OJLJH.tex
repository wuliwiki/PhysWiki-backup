% 欧几里得几何(综述)
% license CCBYSA3
% type Wiki

本文根据 CC-BY-SA 协议转载翻译自维基百科\href{https://en.wikipedia.org/wiki/Euclidean_geometry#}{相关文章}。

\begin{figure}[ht]
\centering
\includegraphics[width=6cm]{./figures/29202c1c81a65fe8.png}
\caption{拉斐尔的《雅典学派》中的细节,展示了一位希腊数学家——可能代表欧几里得或阿基米德——正在使用圆规绘制几何构图。} \label{fig_OJLJH_1}
\end{figure}
欧几里得几何是归功于古希腊数学家欧几里得的数学体系,他在其几何学教材《几何原本》中对其进行了描述。欧几里得的方法是假设一小组直观上令人信服的公设(公理),并从这些公理中推导出许多其他命题(定理)。尽管欧几里得的许多结果早已被提出,[1] 但他是第一个将这些命题组织成一个逻辑系统的人,其中每个结果都是从公理和先前证明的定理推导出来的。[2]

《几何原本》以平面几何开始,至今仍在中学(高中)教授,作为第一个公理化系统和数学证明的初步例子。它接着讲解了三维的立体几何。《几何原本》中的许多内容阐述了现在被称为代数和数论的结果,用几何语言来表达。[1]

在超过两千年的时间里,“欧几里得”这个形容词是不必要的,因为欧几里得的公理似乎是如此直观明显(平行公设可能是唯一例外),以至于从这些公理中推导出的定理被认为是绝对正确的,因此没有其他类型的几何被认为是可能的。然而,今天许多自洽的非欧几里得几何已被发现,最早的几何形式是在19世纪初发现的。爱因斯坦的广义相对论理论的一个含义是,物理空间本身并非欧几里得空间,欧几里得空间仅在短距离内(相对于引力场的强度)对其进行良好近似。[3]

超过两千年来,“欧几里得”这个形容词并不必要,因为欧几里得的公理看起来非常直观明显(平行公设可能是个例外),从这些公理推导出来的定理被认为是绝对正确的,因此没有其他形式的几何被认为是可能的。然而,今天我们知道许多其他自洽的非欧几里得几何,最早的发现是在19世纪初。阿尔伯特·爱因斯坦的广义相对论理论的一个含义是,物理空间本身并非欧几里得的,欧几里得空间只有在短距离(相对于引力场的强度)内才是一个很好的近似。

欧几里得几何是合成几何的一个例子,因为它从描述几何对象(如点和线)的基本属性的公理出发,逻辑地推导出关于这些对象的命题。这与近两千年后由勒内·笛卡尔引入的解析几何形成对比,后者通过坐标使用代数公式来表达几何属性。
\subsection{《几何原本》}
《几何原本》主要是对早期几何知识的系统化。它在比早期的几何处理方法中取得了显著的进步,这一点很快被认可,因此几乎没有人再对保留早期的几何作品感兴趣,而这些作品如今几乎都已遗失。

《几何原本》共包含13卷:

第I至IV卷和第VI卷讨论平面几何。许多关于平面图形的定理被证明,例如“任意三角形中,任意两个角之和小于两个直角。”(第I卷命题17)以及“在直角三角形中,斜边的平方等于两直角边的平方和。”(第I卷命题47)

第V卷和第VII至X卷涉及数论,数被几何地处理,作为线段的长度或平面区域的面积。介绍了素数、有理数和无理数等概念。并且证明了素数的个数是无限的。

第XI至XIII卷涉及立体几何。一个典型的定理是圆锥体和底面和高度相同的圆柱体的体积比为1:3。并且构造了柏拉图立体。
\subsubsection{公理}
\begin{figure}[ht]
\centering
\includegraphics[width=6cm]{./figures/8f816c81e5e3a080.png}
\caption{平行公设(公设 5):如果两条直线与第三条直线相交,并且在一侧的内角之和小于两个直角,那么这两条直线如果延伸足够远,必定会在该侧相交。} \label{fig_OJLJH_2}
\end{figure}
欧几里得几何是一种公理化系统,其中所有定理(“真命题”)都从少数简单的公理推导出来。在非欧几里得几何出现之前,这些公理被认为在物理世界中显而易见,因此所有的定理也都被认为是同样真实的。然而,欧几里得从假设到结论的推理依然独立于物理现实有效。[4]

在《几何原本》第一卷的开头,欧几里得给出了五个平面几何公设(公理),用构造的方式来表述(根据托马斯·希斯的翻译):[5]

假设如下:
\begin{enumerate}
\item 从任意一点到任意一点画一条直线。
\item 将一条有限的直线继续延伸成一条直线。
\item 以任意中心和任意距离(半径)画一个圆。
\item 所有的直角都相等。
\item [平行公设]:如果一条直线与两条直线相交,并且在同一侧形成的内角小于两个直角,那么这两条直线如果无限延伸,必定会在角小于两个直角的那一侧相交。
\end{enumerate}

尽管欧几里得明确地只主张构造物体的存在,但在他的推理中,他也隐含地假设这些物体是唯一的。

《几何原本》还包括以下五个“共通命题”:
\begin{enumerate}
\item 与同一事物相等的事物也彼此相等(欧几里得关系的传递性)。
\item 如果相等的东西相加,那么整体也相等(相等加法公理)。
\item 如果相等的东西相减,那么差也相等(相等减法公理)。
\item 与彼此重合的事物相等(反身性公理)。
\item 整体大于部分。
\end{enumerate}
现代学者一致认为,欧几里得的公设并未提供欧几里得为其论述所要求的完整逻辑基础。[6] 现代的研究使用了更广泛和完整的公理集合。
\subsubsection{平行公设}  
对古代人来说,平行公设似乎不像其他公设那样显而易见。古人力图建立一个绝对确定的命题体系,而平行线公设似乎需要从更简单的命题中证明出来。现在我们知道,这种证明是不可能的,因为可以构造出一致的几何系统(遵循其他公理),其中平行公设为真,也可以构造出平行公设为假的系统。[7] 欧几里得本人似乎认为平行公设与其他公设在性质上有所不同,这一点可以从《几何原本》的组织方式中看出:他前28个命题是那些可以在没有平行公设的情况下证明的。

许多可以与平行公设逻辑等价的替代公理可以被提出(在其他公理的背景下)。例如,普莱费尔公理指出:

在平面上,经过一个不在给定直线上的点,最多只能画一条与给定直线永不相交的直线。  
“最多”这一条件是必要的,因为可以从其余公理中证明,至少存在一条平行线。
\subsubsection{证明方法} 
\begin{figure}[ht]
\centering
\includegraphics[width=6cm]{./figures/a7ea5743fea08888.png}
\caption{给定一条线段,可以构造一个包含该线段作为其中一边的等边三角形:通过在点 Α 和 Β 上分别以 Δ 和 Ε 为圆心画圆,并取两个圆的交点之一作为三角形的第三个顶点,从而构造出一个等边三角形 ΑΒΓ。} \label{fig_OJLJH_3}
\end{figure}
欧几里得几何是构造性的。公设 1、2、3 和 5 断言了某些几何图形的存在性和唯一性,这些断言具有构造性:也就是说,我们不仅被告知某些事物存在,而且还给出了用圆规和无标尺直尺来构造它们的方法。在这一意义上,欧几里得几何比许多现代公理化系统(如集合论)更加具体,因为后者通常仅仅断言对象的存在,而没有说明如何构造它们,甚至有时断言某些无法在该理论内构造的对象的存在。严格来说,纸上的直线是形式系统中定义的对象的模型,而不是这些对象的实例。例如,欧几里得的直线没有宽度,但任何实际绘制的直线都有宽度。尽管几乎所有现代数学家认为非构造性证明与构造性证明一样有效,但它们通常被认为不如构造性证明优雅、直观或在实践中有用。欧几里得的构造性证明常常取代了错误的非构造性证明,例如一些假设所有数都是有理数的毕达哥拉斯证明,通常需要像“找出…的最大公约数”这样的语句。

欧几里得常常使用反证法。
\subsection{符号和术语}  
\subsubsection{点和图形的命名}  
点通常使用字母表的大写字母来命名。其他图形,如直线、三角形或圆,通常通过列出足够数量的点来命名,以便从相关图形中明确区分它们。例如,三角形 ABC 通常指的是一个顶点分别位于点 A、B 和 C 的三角形。
\subsubsection{互补角和补角}  
角度和为直角的角叫做互补角。互补角形成于当一条射线与原有两条射线共享同一个顶点,并且指向位于两条原射线之间的方向时。两条原射线之间的射线数量是无限的。

角度和为平角的角叫做补角。补角形成于当一条射线与原有两条射线共享同一个顶点,并且指向位于两条原射线之间的方向时,这两条原射线形成一个平角(180度角)。两条原射线之间的射线数量也是无限的。
\subsection{欧几里得符号的现代版本}  
在现代术语中,角度通常以度数或弧度来衡量。

现代的教科书通常定义了不同的图形,称为直线(无限长)、射线(半无限长)和线段(有限长度)。欧几里得并不像现代那样将射线视为一个在一个方向上延伸至无限的对象,他通常会使用类似“如果直线延长到足够的长度”这样的表达方式,尽管他偶尔也提到“无限直线”。对于欧几里得来说,“直线”可以是直的,也可以是曲的,必要时他会使用更具体的术语“直线”。
\subsection{一些重要或著名的结果}
\begin{figure}[ht]
\centering
\includegraphics[width=6cm]{./figures/31b84b2faa941374.png}
\caption{驴桥定理(Pons Asinorum)指出,在一个等腰三角形中,角 α = 角 β 且角 γ = 角 δ。} \label{fig_OJLJH_4}
\end{figure}
\begin{figure}[ht]
\centering
\includegraphics[width=6cm]{./figures/230550cbb7372788.png}
\caption{“三角形内角和定理指出,任何三角形的三个角的和,无论是角α、β还是γ,总是等于180度。”} \label{fig_OJLJH_5}
\end{figure}
\begin{figure}[ht]
\centering
\includegraphics[width=6cm]{./figures/be3cecb54b8c3ef1.png}
\caption{“勾股定理指出,直角三角形两条直角边(a和b)上方的两个正方形的面积之和等于斜边(c)上方正方形的面积。”} \label{fig_OJLJH_6}
\end{figure}
\begin{figure}[ht]
\centering
\includegraphics[width=6cm]{./figures/ff2445419aa66ec9.png}
\caption{泰勒斯定理指出,如果AC是直径,那么角B是直角。} \label{fig_OJLJH_7}
\end{figure}
\subsubsection{驴桥定理}  
驴桥定理(Pons asinorum)指出,在等腰三角形中,底角相等,并且如果相等的直线延长,底下的角也相等。[12] 这个名字可能来源于它在《几何原本》中经常作为检验读者智慧的第一个真正考验,以及作为通向后面更难命题的桥梁。也有可能因为这个几何图形像一座陡峭的桥,只有脚步稳健的驴子才能通过,故得此名。[13]
\subsubsection{三角形的全等}
\begin{figure}[ht]
\centering
\includegraphics[width=6cm]{./figures/c172989b69fc4696.png}
\caption{三角形的全等可以通过指定两边及其夹角(SAS)、两角及其夹边(ASA)或两角及其对应的邻边(AAS)来确定。然而,通过指定两边及一个邻角(SSA)可能会产生两个不同的三角形,除非所指定的角是直角。} \label{fig_OJLJH_8}
\end{figure}
如果三角形的三边相等(SSS)、两边及其夹角相等(SAS),或两角及其一边相等(ASA),则这两个三角形全等(《几何原本》第一卷,第4、8和26命题)。三个角相等的三角形(AAA)是相似的,但不一定全等。此外,两边相等且夹角相等的三角形不一定相等或全等。
\subsubsection{三角形内角和}  
三角形的三个内角的和等于一个平角(180度)。[14] 这使得等边三角形的三个内角均为60度。此外,这也使得每个三角形至少有两个锐角,并且最多有一个钝角或直角。
\subsubsection{勾股定理}  
著名的勾股定理(《几何原本》第一卷,第47命题)指出,在任何直角三角形中,以斜边为边的正方形的面积等于以两条直角边为边的两个正方形面积之和。
\subsubsection{泰勒斯定理}
\begin{figure}[ht]
\centering
\includegraphics[width=6cm]{./figures/6b315361ec5f8a84.png}
\caption{全等的一个例子。左边的两个图形是全等的,而第三个图形与它们相似。最后一个图形既不全等也不相似。全等变换会改变一些属性,如位置和方向,但保持其他属性不变,比如距离和角度。这些不变的属性被称为不变量,研究这些不变量是几何学的本质。} \label{fig_OJLJH_9}
\end{figure}
泰勒斯定理,得名于米利都的泰勒斯,指出,如果A、B、C是圆上的三点,且直线AC是圆的直径,则角ABC是直角。坎托尔假设泰勒斯是通过欧几里得《几何原本》第一卷第32命题,以欧几里得《几何原本》第三卷第31命题的方式来证明他的定理的。[15][16]
\subsubsection{面积和体积的缩放 } 
在现代术语中,平面图形的面积与其任意线性尺寸的平方成正比,\(A \propto L^2\)固体的体积与其线性尺寸的立方成正比,\(V \propto L^3\)欧几里得在各种特殊情况下证明了这些结果,如圆的面积[17]和平行六面体的体积[18]。欧几里得确定了一些比例常数,但并非全部。例如,证明球体体积为外接圆柱体体积的2/3的是他的继任者阿基米德[19]。
\subsection{度量和算术体系}  
欧几里得几何学有两种基本的度量类型:角度和距离。角度尺度是绝对的,欧几里得使用直角作为基本单位,因此,例如,45度角被称为直角的一半。距离尺度是相对的;我们可以任意选择一个具有一定非零长度的线段作为单位,其他的距离则相对于它来表示。距离的加法通过一个构造表示,其中一个线段被复制到另一个线段的末端,以延长其长度,减法也是类似的方式进行。

面积和体积的度量来源于距离。例如,一个宽度为3,长度为4的矩形,其面积是它们的乘积,12。由于这种几何学上的乘法解释仅限于三维,因此没有直接的方式来解释四个或更多数字的乘积,欧几里得避免了这样的乘积,尽管它们在《几何原本》第九卷第20命题的证明中有暗示。

欧几里得称一对线段或一对平面或立体图形为“相等”(ἴσος),如果它们的长度、面积或体积分别相等,角度也是类似的定义。更强的术语“全等”指的是一个图形与另一个图形大小和形状完全相同的概念。或者说,如果一个图形可以移动到另一个图形上并完全匹配,那么这两个图形是全等的(允许翻转)。因此,例如,一个2x6的矩形和一个3x4的矩形是相等的,但不是全等的,而字母R与其镜像是全等的。除大小不同外,其他相似的图形称为相似图形。相似形状中对应的角度相等,且对应的边按比例相等。
\subsection{在工程学中}  
\subsubsection{设计与分析}
\begin{itemize}
\item 应力分析:应力分析——欧几里得几何在确定机械组件中的应力分布中起着至关重要的作用,这对于确保结构的完整性和耐久性至关重要。
\begin{figure}[ht]
\centering
\includegraphics[width=6cm]{./figures/d3b419fd88b8bbfe.png}
\caption{机械应力} \label{fig_OJLJH_10}
\end{figure}
\item 齿轮设计:齿轮——齿轮是许多机械系统中的关键元件,其设计在很大程度上依赖于欧几里得几何,以确保齿形和啮合的正确性,从而实现高效的动力传输。
\begin{figure}[ht]
\centering
\includegraphics[width=6cm]{./figures/d2d6a88f751ecabe.png}
\caption{齿轮} \label{fig_OJLJH_11}
\end{figure}
\item 热交换器设计:热交换器——在热工学中,欧几里得几何用于设计热交换器,其中几何配置对热效率有着重要影响。有关更多细节,请参见壳管式热交换器和板式热交换器。
\begin{figure}[ht]
\centering
\includegraphics[width=6cm]{./figures/e3df9bedf547627f.png}
\caption{} \label{fig_OJLJH_12}
\end{figure}
\item 镜头设计:镜头——在光学工程中,欧几里得几何在镜头设计中至关重要,精确的几何形状决定了聚焦特性。几何光学分析镜头和镜面对光的聚焦。
\begin{figure}[ht]
\centering
\includegraphics[width=8cm]{./figures/9d4765fbb97e8417.png}
\caption{镜头类型} \label{fig_OJLJH_13}
\end{figure}
\end{itemize}
\subsubsection{动力学}
\begin{itemize}
\item 振动分析:振动——欧几里得几何在分析和理解机械系统中的振动中至关重要,帮助设计能够有效承受或利用这些振动的系统。
\item 机翼设计:飞机机翼设计——欧几里得几何在空气动力学中的应用体现在飞机机翼设计、翼型和水翼中,其中几何形状直接影响升力和阻力特性。
\item 卫星轨道:卫星轨道——欧几里得几何有助于计算和预测卫星的轨道,这对于成功的太空任务和卫星操作至关重要。另请参见天体动力学、天体力学和椭圆轨道。
\begin{figure}[ht]
\centering
\includegraphics[width=8cm]{./figures/208926cf31b4d385.png}
\caption{翼型命名法} \label{fig_OJLJH_14}
\end{figure}
\end{itemize}
\subsubsection{计算机辅助设计系统(CAD系统)}
\begin{itemize}
\item 3D建模:在CAD(计算机辅助设计)系统中,欧几里得几何是创建机械零件准确三维模型的基础。这些模型对于在制造之前可视化和测试设计至关重要。  
\item 设计与制造:CAM(计算机辅助制造)在很大程度上依赖于欧几里得几何。CAD/CAM中的设计几何通常由平面、圆柱体、圆锥体、环面和其他类似的欧几里得形状构成。如今,CAD/CAM在从汽车和飞机到船舶和智能手机等广泛产品的设计中都至关重要。  
\item 制图实践的发展:历史上,先进的欧几里得几何,包括像帕斯卡定理和布里昂松定理这样的定理,是制图实践的重要组成部分。然而,随着现代CAD系统的出现,这些定理的深入知识在当代设计和制造过程中变得不那么必要。  
另见:CAD软件的历史。
\end{itemize}
\subsubsection{电路设计}
\begin{figure}[ht]
\centering
\includegraphics[width=6cm]{./figures/b739ff2f7fb75ac4.png}
\caption{3D CAD模型} \label{fig_OJLJH_15}
\end{figure}
PCB布局:印刷电路板(PCB)设计利用欧几里得几何来有效地布置和连接组件,确保功能性同时优化空间。电子组件在PCB上的高效布局对于最小化信号干扰和优化电路性能至关重要。
\subsubsection{电磁场和流体流动场 } 
\begin{itemize}
\item 天线设计:天线设计——天线的欧几里得几何在天线设计中起着重要作用,天线的空间排列和尺寸直接影响天线阵列在传输和接收电磁波时的性能。
\begin{figure}[ht]
\centering
\includegraphics[width=6cm]{./figures/676be464fc0ad1e2.png}
\caption{DVD播放器的PCB(印刷电路板)} \label{fig_OJLJH_16}
\end{figure}
\item 场论:复势流——在无粘流场和电磁场的研究中,欧几里得几何有助于可视化和解决势流问题。这对于理解流体速度场和电磁场在三维空间中的相互作用至关重要。其关系由无旋涡的螺旋场或保守向量场来表征。
\begin{figure}[ht]
\centering
\includegraphics[width=6cm]{./figures/76cd8dd8970e2b7d.png}
\caption{NASA卡塞格雷天线,极高增益 ~70 dBi。} \label{fig_OJLJH_17}
\end{figure}
\end{itemize}
\subsubsection{控制}
\begin{itemize}
\item 控制系统分析:控制系统——欧几里得几何在控制理论中的应用有助于控制系统的分析与设计,特别是在理解和优化系统的稳定性与响应方面。
\begin{figure}[ht]
\centering
\includegraphics[width=6cm]{./figures/b9c5c1815ab52b94.png}
\caption{无环量源周围的势流} \label{fig_OJLJH_18}
\end{figure}
\item 计算工具:雅可比矩阵——欧几里得几何在使用雅可比矩阵进行变换和控制系统分析中至关重要,涵盖了机械和电气工程领域,提供了系统行为和特性的深入了解。雅可比矩阵作为统计回归和曲线拟合中的线性化设计矩阵;参见非线性最小二乘法。雅可比矩阵还广泛应用于随机矩阵、矩、统计和诊断。
\begin{figure}[ht]
\centering
\includegraphics[width=6cm]{./figures/4184dc5863b04a1b.png}
\caption{基本反馈回路。} \label{fig_OJLJH_19}
\end{figure}
\end{itemize}
\subsection{其他一般应用}  
由于欧几里得几何在数学中的基础地位,在这里列举超过一部分代表性应用是不切实际的。
\begin{figure}[ht]
\centering
\includegraphics[width=14.25cm]{./figures/a65f62be21a26a23.png}
\caption{测量员使用水平仪/球体堆积适用于一堆橙子。/抛物面镜将平行的光线聚焦到一点。} \label{fig_OJLJH_20}
\end{figure}
正如词源所示,几何学最早的兴趣之一,也是目前最常见的应用之一,就是测量。[20] 此外,它还被应用于经典力学以及认知和计算视觉感知对象的方法。某些来自欧几里得几何的实际结果(例如3-4-5三角形的直角性质)在被正式证明之前就已经被使用。[21] 欧几里得几何中的基本度量类型是距离和角度,这两者都可以由测量员直接测量。从历史上看,距离通常是通过链条来测量的,例如甘特链,而角度则是使用刻度圆和后来的经纬仪来测量的。

欧几里得立体几何的一个应用是确定堆积排列,例如寻找n维空间中球体最有效堆积的问题。这个问题在错误检测和修正中有应用。
\begin{figure}[ht]
\centering
\includegraphics[width=6cm]{./figures/415a7288795071f2.png}
\caption{几何学在艺术和建筑中得到了应用。} \label{fig_OJLJH_21}
\end{figure}
\begin{figure}[ht]
\centering
\includegraphics[width=6cm]{./figures/6d9f3632796af46d.png}
\caption{水塔由一个圆锥、一个圆柱和一个半球组成。其体积可以通过立体几何来计算。} \label{fig_OJLJH_22}
\end{figure}
\begin{figure}[ht]
\centering
\includegraphics[width=6cm]{./figures/544c1336aadd63be.png}
\caption{几何学可以用于设计折纸。} \label{fig_OJLJH_23}
\end{figure}
几何学在建筑中得到了广泛应用。

几何学可以用于设计折纸。一些经典的几何构造问题使用圆规和直尺无法解决,但可以通过折纸来解决。[22]
\subsection{后期历史}  
\subsubsection{阿基米德与阿波罗尼乌斯}
\begin{figure}[ht]
\centering
\includegraphics[width=6cm]{./figures/ca7a8b16fb215d2e.png}
\caption{一个球体的体积和表面积是其外接圆柱的 2/3。根据阿基米德的要求,一个球体和圆柱被放置在阿基米德的墓碑上。} \label{fig_OJLJH_24}
\end{figure}
阿基米德(约公元前287年-约公元前212年),一位有着丰富历史轶事的人物,与欧几里得一起被铭记为古代最伟大的数学家之一。虽然他的工作基础是由欧几里得奠定的,但与欧几里得不同,阿基米德的工作被认为完全是原创的。[23] 他证明了各种二维和三维图形的体积和面积公式,并阐述了阿基米德有限数的性质。

伯加的阿波罗尼乌斯(约公元前240年-约公元前190年)主要以对圆锥曲线的研究而闻名。