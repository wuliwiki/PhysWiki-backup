% 导数的性质与构造(高中)
% keys 导数|性质|构造|恒等
% license Usr
% type Tutor

\begin{issues}
\issueDraft
\end{issues}

\pentry{导数\nref{nod_HsDerv},函数的性质\nref{nod_HsFunC}}{nod_139b}

导数是分析函数变化规律的关键工具,它揭示了函数随变量变化时的趋势和行为。可以把导函数看作原函数的一张“素描画”,虽然简化了某些信息,但保留了足够的信息来描述函数的几乎所有性质。例如,导数可以用来判断函数在某一区间内是上升还是下降,还可以分析图像的弯曲方向是“开口向上”还是“开口向下”,甚至帮助找到最高点和最低点的位置。这种功能类似于在地图上标注道路的坡度,通过这些标记可以快速了解路段的起伏情况。同样,导函数为函数的变化贴上了清晰的“标签”,使得人们能够一目了然地把握它在不同区间的行为。

在解题过程中,导数相关的典型问题主要包括以下三类:

\begin{itemize}
\item \textbf{最值问题}:通过确定导数为零的位置,分析函数的变化趋势,找到函数的最高点或最低点。
\item \textbf{零点问题}:利用导数得到单调区间,结合区间两端函数值的符号情况,利用\aref{零点存在定理}{the_HsFunC_1},判断函数在某区间内的零点情况。
\item \textbf{恒成立问题}:通过导数符号的变化,分析函数在某区间内的单调性和极值的符号,判断函数是否满足某些恒成立条件。
\end{itemize}

解题的核心是找出导数为零的点,并分析导数在不同区间内的正负变化,从而勾勒出函数的变化趋势和形态。在此过程中,构造适当的函数并进行求导,以及合理选取区间和相异的函数值,是解决这类问题的两个主要难点。本文会对此提供一些常见的方法。

\subsection{近似代替}

在导数的\aref{几何含义}{sub_HsDerv_1}中就提到过“以直代曲”。

\begin{equation}
f(x_0+\Delta x)\approx f(x_0)+f'(x_0)\Delta x~.
\end{equation}

导函数的奇偶性通常与原函数相反,而它们的周期性却保持一致。


\subsection{单调性}
在介绍\aref{导函数}{sub_HsDerv_2}时,提及区间的中函数的增减与导函数的符号相关。
\begin{theorem}{单调性与导数的关系}
导函数  $f'(x)$  代表了原函数  f(x)  图像在每一点的切线斜率。
\begin{itemize}
\item 在$f'(x)>0$的区间上,原函数的图像单调递增。
\item 在$f'(x)<0$的区间上,原函数的图像单调递减。
\item 在$f'(x)=0$的区间上,原函数的图像是水平的。
\end{itemize}
\end{theorem}


\subsection{驻点、极值点与最值点}

前面提到过,在讨论导数相关的问题时,最值问题是一个核心内容。极值指的是函数在某个局部范围内达到的最大值或最小值,而最值则是函数在整个定义域内的最大值或最小值。极值强调局部性,例如某点附近的函数值高于或低于周围的值;而最值则具有全局性,表示函数在整个定义域内的最高或最低点。

尽管高中教科书主要关注极值和最值这两个概念,但还有一个重要的概念被忽略了,那就是驻点。驻点是函数导数为零的点,是导数图像的水平切点。驻点与极值点有着密切的联系,因为极值点通常是驻点的一种。然而,驻点并不一定是极值点。例如,对于函数 $y = x^3$,在 $x = 0$ 处导数为零,但函数值既未达到极大值也未达到极小值,因此该驻点不是极值点。

在高中阶段,由于对驻点和极值点的区分较少,通常直接以极值点替代驻点的介绍。然而,随着理解的深入,明确两者的区别变得愈发重要。驻点的存在提供了一个分析极值的起点,而极值点需要进一步通过导数的变化情况进行判定。因此,从驻点的定义入手,系统地理解极值和最值的本质,不仅能够提升对函数行为的掌握,还为更深层次的数学学习奠定了基础。

\begin{definition}{驻点}
对于函数$y=f(x)$,如果某点$x_0$满足$f'(x_0)=0$,即$x_0$是$f'(x)$的零点,则称$x_0$为$f(x)$的\textbf{驻点(stationary point)}。
\end{definition}

驻点表示的是函数值暂时停止变化的点,或者说在这里函数是水平的。



极值点是针对某个局部范围而言的,而这个局部在数学上被称为\textbf{邻域(neighborhood)}。具体来说,对于一个点 $x_0$,邻域可以表示为 $\left( x_0 - \delta, x_0 + \delta \right)$,其中 $\delta > 0$。这意味着在点 $x_0$ 的左右各延伸 $\delta$ 的范围内的所有点都属于 $x_0$ 的邻域。这就像一个人在自己家附近的范围内活动,这个范围可以由一定的距离(类似于 $\delta$)决定。邻域概念可以保证在研究函数时聚焦于某一点的周围情况,而不必考虑整个区域的性质。

\begin{definition}{极值点}
于函数 $y=f(x)$及其定义域内一点 $x_0$,若存在$x_0$的某个邻域$U=\left( x_0-\delta,x_0+\delta \right)$,使得$U$中的任意点$x$满足:
\begin{itemize}
\item $f(x) \leq f\left(x_0\right)$,则称 $x_0$ 是$f(x)$的\textbf{极大值点(maximum point)}。
\item $f(x) \geq f\left(x_0\right)$,则称 $x_0$ 是$f(x)$的\textbf{极小值点(minimum point)}。
\end{itemize}
极大值点和极小值点合称为\textbf{极值点(extremum point)}。
\end{definition}

极值点的定义与驻点密切相关,但两者并不等价。根据定义,非所有驻点都是极值点。例如,对于函数 $y = x^3$,在 $x = 0$ 处满足 $f'(x) = 0$,但在该点两侧,函数值既有增大也有减小,因此 $x = 0$ 并不是极值点。

值得注意的是,极值点的判定并不依赖于函数在这一点的导数是否存在。只要在某个邻域内满足极值的定义,该点就可以被视为极值点。例如,对于函数 $y = |x|$,在 $x = 0$ 处,函数的一阶导数不存在,因此$x = 0$并非驻点,但由于在任意的区间 $(-\delta, \delta)$ 上都有 $|0|$ 最小,因此 $x = 0$ 是一个极值点。

实际上,极值点的定义可以等价为某点两侧导数符号的变化,而这一点的可导性并非必要条件。如果导数从负变正,则该点是极小值点;如果从正变负,则该点是极大值点。由于驻点的导数为零,因此驻点经常与导数符号变化对应,从而容易引起概念上的混淆。

在高中阶段的考察中,通常不涉及不可导点的极值问题,因此在实际求解时可以简化为以下步骤:
\begin{enumerate}
\item 求驻点:通过解 $f'(x) = 0$ 找到函数的驻点。
\item 判定导数符号变化:通过观察每个驻点左右的导数符号,判定该点是否为极值点。
\item 确定极值类型:根据导数符号的变化方向,判定驻点是极大值点还是极小值点。
\end{enumerate}

这种方法避免了处理不可导点的复杂性,更适合高中阶段的学习。类比来说,驻点就像一个检测站,只有确认车辆(函数值)的行驶方向在该点发生变化,才能确定它是山顶(极大值点)或山谷(极小值点)。

前面提到极值点是对与某一个局部而言的,那么如果想要求整个区间(通常是定义域)上的就要计算最值。

在了解了极值的概念后,可以进一步讨论最值。最值是函数在其定义域内的整体最大值或最小值,而极值仅在某个局部区域内达到极大或极小。
	1.	定义
	•	最大值:若函数 $f(x)$ 在某点 $x_0$ 上的函数值 $f(x_0)$ 满足对定义域内任意 $x$,都有 $f(x_0) \geq f(x)$,则称 $f(x_0)$ 为函数的最大值。
	•	最小值:若函数 $f(x)$ 在某点 $x_0$ 上的函数值 $f(x_0)$ 满足对定义域内任意 $x$,都有 $f(x_0) \leq f(x)$,则称 $f(x_0)$ 为函数的最小值。
	2.	极值与最值的关系
	•	极值是局部的,而最值是全局的。最值可以是极值的一种,但并非所有极值都是最值。例如,在函数的某一局部区域中达到的极大值可能并不是定义域内的最大值。
	•	判断最值时,需要综合考虑函数在所有极值点和定义域的边界点的函数值。
	3.	求最值的方法
	•	先通过求导找到驻点,判定驻点是否为极值点。
	•	检查函数在定义域的边界点或端点的函数值。
	•	比较所有极值点和边界点的函数值,从中确定最大值和最小值。
	4.	例子
考虑函数 $f(x) = -x^2 + 4x + 1$,定义域为 $[0, 5]$:
	•	求导得到 $f’(x) = -2x + 4$,解 $f’(x) = 0$ 得驻点 $x = 2$。
	•	在 $x = 2$ 附近,导数从正变负,说明 $x = 2$ 是极大值点。
	•	计算 $f(x)$ 在 $x = 2$ 处的函数值,以及在定义域边界点 $x = 0$ 和 $x = 5$ 处的函数值:

	•	比较得 $f(x)$ 的最大值为 $5$,最小值为 $-4$。
	5.	类比
极值像一个局部的“高峰”或“低谷”,而最值则是整个地图中最高的山峰或最低的洼地。判断最值时,需要放眼整个定义域,而不仅仅关注局部区域。


\subsection{高阶导数}

导函数作为原函数,则又可以求得它的导函数,这也被称为高阶导数。

凹凸性

连续曲线的凹弧与凸弧的交界点。(在该点的二阶导并不一定有定义)
f"(x)=0,且该点 两侧 二阶导数变号,那么该点就是极值点。
当然,可能在一点x0处,二阶导数并不存在,在x0左侧的二阶导数趋于正无穷,右侧的二阶导数趋于负无穷,该点也是拐点。

\subsection{常用构造}

\pentry{恒等式与不等式恒成立\nref{nod_HsIden}}{nod_b069}

\subsubsection{逆向使用求导法则}

逆用积法则:

$x^n f'(x) + f(x) \geq 0$,构造 $F(x) = x^n f(x)$,$[x^n f(x)] = x^n f'(x) + nx^{n-1} f(x) = x^{n-1} [x f'(x) + nf(x)]$。特别地,当$n=1$时有$x f'(x) + f(x) \geq 0$,构造 $F(x) = x f(x)$,$[x f(x)]' = x f'(x) + f(x)$

$f'(x) + k f(x) \geq 0$,构造 $F(x) = \E^{kx} f(x)$,$[e^{kx} f(x)]' = e^{kx} [f'(x) + kf(x)]$。特别地,当$k=1$时有$f'(x) + f(x) \geq 0$,构造 $F(x) = \E^x f(x)$,$[e^x f(x)]' = e^x [f'(x) + f(x)]$

逆用商法则:

$xf'(x) - f(x) \geq 0$,构造 $F(x) = \frac{f(x)}{x}$,  
    $\therefore \left[\frac{f(x)}{x}\right]' = \frac{f'(x) \cdot x - f(x)}{x^2}$

$f'(x) - f(x) \geq 0$,构造 $F(x) = \frac{f(x)}{e^x}$,  
    $\therefore \left[\frac{f(x)}{e^x}\right]' = \frac{e^x \cdot f'(x) - e^x \cdot f(x)}{e^{2x}} = \frac{f'(x) - f(x)}{e^x}$

$x^n f'(x) - n f(x) \geq 0$,构造 $F(x) = \frac{f(x)}{x^n}$,  
    $\therefore \left[\frac{f(x)}{x^n}\right]' = \frac{x^n \cdot f'(x) - n x^{n-1} \cdot f(x)}{x^{2n}} = \frac{f'(x) - n f(x)}{x^{n+1}}$

$f'(x) - k f(x) \geq 0$,构造 $F(x) = \frac{f(x)}{e^{kx}}$,  
    $\therefore \left[\frac{f(x)}{e^{kx}}\right]' = \frac{e^{kx} \cdot f'(x) - k e^{kx} \cdot f(x)}{e^{2kx}} = \frac{f'(x) - k f(x)}{e^{kx}}$
