% 相对论动力学
\pentry{相对论加速度变换\upref{SRAcc},时空的四维表示\upref{SR4Rep}}


\subsection{什么是动力学 常见误解的辟谣}
初学牛顿力学时,你也许注意到了,课本将它分为两大部分:运动学和动力学.

运动学的研究范畴,基本上是纯数学的,只讨论了什么是位移、速度、加速度等概念,以及这些概念之间的数学联系.在运动学中有一个看起来很自然的假设,在了解相对论之前常常被忽略,那就是在不同参考系中这些概念和它们的联系是如何变化的.

动力学的研究范畴,则加上了“力”的概念,讨论力是如何影响物体的运动状态、力有什么性质等的.在牛顿力学中,力的性质由牛顿三定律决定,由此可以引申出更深刻的动量守恒和能量守恒等定律.

在狭义相对论中,参考系的变换导致的运动学变换复杂了很多,力的作用也不再像牛顿力学中那样简单.

许多科普读物和视频会使用光子钟等模型来“推导”出爱因斯坦的明星方程,$E=mc^2$,或者用我们约定的单位制,$E=m$.这些推导的思路是用牛顿第二定律$F=m\dd x/\dd t$来定义“力”,然后利用洛伦兹变换计算在不同惯性系中,物体加速度的变化.当然,在使用伽利略变换的牛顿力学中,任何惯性系下物体的加速度都一样,因此质量是不会变化的;但是在狭义相对论中,由于不同惯性系下加速度一般是不同的,加之相对性原理要求力是不随参考系而变化的\footnote{比如说,一个正在施加张力的弹簧,应该看成是在任何参考系下都产生一样的张力,而不受弹簧尺缩效应的影响.},因此套入牛顿第二定律以后,必须认为质量是可以变化的.这种思路所计算出的质点质量变化,就是$m/\sqrt{1-v^2}$,其中$v$是质点在观察者眼中的运动速度.这个质量,就是所谓的“动质量”.以$v^2$为自变量,对$m/\sqrt{1-v^2}$进行Maclaulin展开(即$v_0^2=0$处的泰勒展开),我们可以得到

\begin{equation}
\frac{m}{\sqrt{1-v^2}}=m+\frac{1}{2}mv^2+\frac{3}{4}mv^4+\cdots
\end{equation}
其中第二项就是牛顿力学中的动能.在


从相对论加速度变换\upref{SRAcc}词条中可以看到,物体