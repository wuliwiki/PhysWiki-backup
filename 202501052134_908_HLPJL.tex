% 亨利·庞加莱(综述)
% license CCBYSA3
% type Wiki

本文根据 CC-BY-SA 协议转载翻译自维基百科\href{https://en.wikipedia.org/wiki/Carl_Friedrich_Gauss}{相关文章}。

\begin{figure}[ht]
\centering
\includegraphics[width=6cm]{./figures/8ad8fe061e9149a3.png}
\caption{} \label{fig_HLPJL_1}
\end{figure}
朱尔·亨利·庞加莱(Jules Henri Poincaré,英国发音:/ˈpwæ̃kɑːreɪ/,美国发音:/ˌpwæ̃kɑːˈreɪ/;法语发音:[ɑ̃ʁi pwɛ̃kaʁe] ⓘ;1854年4月29日—1912年7月17日)是法国的数学家、理论物理学家、工程师和科学哲学家。他常被称为博学多才的人,在数学领域被誉为“最后的普遍主义者”,因为他在他的一生中,几乎在所有的数学领域都有卓越的成就。他还被称为“现代数学的高斯”。由于他在科学、哲学方面的成功和影响,他被誉为“现代科学的典范哲学家”。

作为数学家和物理学家,庞加莱对纯数学、应用数学、数学物理学和天体力学做出了许多原创性的基础性贡献。在研究三体问题时,庞加莱成为第一个发现混沌确定性系统的人,这为现代混沌理论奠定了基础。庞加莱被认为是代数拓扑学的创始人,并且他还被认为是引入自同态形式的先驱。他还对代数几何、数论、复分析和李群理论作出了重要贡献。他著名地提出了庞加莱重现定理,该定理表明,经过足够长的时间后,一个状态将最终以任意接近其初始状态的方式重新出现,这一结论具有深远的影响。20世纪初,他提出了庞加莱猜想,后来它成为了数学史上著名的未解问题之一,直到2002–2003年被格里戈里·佩雷尔曼解决。庞加莱还推动了非欧几何在数学中的应用。

庞加莱明确了在不同变换下物理定律不变性的重要性,并且是第一个以现代对称形式呈现洛伦兹变换的人。庞加莱发现了剩余的相对论速度变换,并在1905年通过一封信记录给亨德里克·洛伦兹。这样,他实现了麦克斯韦方程的完全不变性,这是构建狭义相对论的一个重要步骤,他也为此奠定了基础,并在1905年发表了相关的基础性论文。他首次提出了引力波(ondes gravifiques)——由物体发出并以光速传播的波,这一概念要求洛伦兹变换的存在,时间是在1905年提出的。1912年,他写了一篇有影响力的论文,提出了量子力学的数学论证。庞加莱还通过对X射线的研究,为放射性现象的发现播下了种子,这一研究影响了物理学家亨利·贝克勒尔,后者因此发现了放射性现象。物理学和数学中使用的庞加莱群是以他命名的,因为他首次引入了群的概念。

庞加莱在他的时代被认为是数学和理论物理学领域的主导人物,是当时最受尊敬的数学家,被数学家保罗·潘莱维称为“理性科学的活脑”。哲学家卡尔·波普尔认为庞加莱是有史以来最伟大的科学哲学家,并且庞加莱还是科学中的常规主义观点的创立者。庞加莱在他的时代是一个公众知识分子,个人上,他支持所有人享有政治平等,并警惕天主教会当时所持的反智立场的影响。他曾担任法国科学院院长(1906年)、法国天文学会会长(1901–1903年),以及法国数学会两任会长(1886年,1900年)。
\subsection{生活}
庞加莱于1854年4月29日出生在法国摩泽尔省南锡的杜卡尔区(Cité Ducale),出身于一个有影响力的法国家庭。他的父亲莱昂·庞加莱(Léon Poincaré,1828–1892)是南锡大学的医学教授。他的妹妹阿琳嫁给了精神哲学家埃米尔·布特鲁(Émile Boutroux)。庞加莱家族中的另一位显赫人物是他的表亲雷蒙·庞加莱(Raymond Poincaré),他是法国文学学院的成员,曾在1913年至1920年间担任法国总统,并在1913年到1929年期间三度担任法国总理。
\subsubsection{教育}

南锡市大街117号的庞加莱出生地,门前有纪念牌匾。

在他的童年时期,庞加莱曾因白喉病重病一段时间,并由母亲尤金妮·劳努瓦(Eugénie Launois,1830–1897)进行特别的教育指导。

1862年,庞加莱进入了南锡的中学(现在以他的名字命名为“亨利·庞加莱中学”,并且南锡大学也以他命名)。他在这所中学度过了十一年,并且在此期间,他证明自己在所学的各个科目中都是顶尖的学生。他在写作方面表现突出。数学老师曾称他为“数学怪才”,他在法国各大中学举行的“ concours général”(全国顶尖学生比赛)中获得了第一名。他最差的学科是音乐和体育,被形容为“最多只能算及格”。他较差的表现可能与视力不良和健忘倾向有关。他于1871年从中学毕业,获得了文学和科学双学位的高中文凭(baccalauréat)。

在1870年的普法战争期间,庞加莱与父亲一起在救护队服役。

庞加莱于1873年以优异的成绩考入了巴黎高等矿业学校(École Polytechnique),并于1875年毕业。在那里,他作为数学家查尔斯·埃米特(Charles Hermite)的学生继续深造,并在1874年发表了他的第一篇论文《Démonstration nouvelle des propriétés de l'indicatrice d'une surface》(关于曲面指标性质的新证明)。从1875年11月到1878年6月,他在巴黎高等矿业学校进一步学习,并继续学习数学,同时也修习了矿业工程课程,最终于1879年3月获得了普通矿业工程师学位。

作为巴黎高等矿业学校的毕业生,庞加莱加入了法国矿业工程师队(Corps des Mines),成为法国东北部韦苏尔(Vesoul)地区的矿务监察员。在1879年8月,他亲自处理了马尼(Magny)矿区的一起矿难事故,这起事故导致18名矿工死亡。庞加莱对这次事故进行了官方调查。

与此同时,庞加莱正在为他的数学科学博士学位做准备,导师是查尔斯·埃米特。他的博士论文领域是微分方程,题为《Sur les propriétés des fonctions définies par les équations aux différences partielles》(关于由偏微分方程定义的函数的性质)。庞加莱提出了一种全新的方法来研究这些方程的性质。他不仅解决了如何求解这些方程的积分问题,还首次研究了它们的一般几何性质。他意识到,这些方程可以用来模拟太阳系中多个自由运动天体的行为。他于1879年从巴黎大学毕业。
\subsubsection{第一次科学成就}
\begin{figure}[ht]
\centering
\includegraphics[width=6cm]{./figures/838d251c754035a5.png}
\caption{1887年,33岁的年轻亨利·庞加莱} \label{fig_HLPJL_2}
\end{figure}
获得学位后,庞加莱于1879年12月开始在诺曼底的卡昂大学担任数学讲师。同时,他发表了第一篇重要的文章,讨论了一类自守函数的处理方法。

在卡昂期间,他遇到了未来的妻子路易丝·普兰·达昂塞(Louise Poulain d'Andecy,1857–1934),她是伊西多尔·吉奥弗鲁瓦·圣希莱(Isidore Geoffroy Saint-Hilaire)的孙女,埃蒂安·吉奥弗鲁瓦·圣希莱(Étienne Geoffroy Saint-Hilaire)的曾孙女。两人于1881年4月20日结婚。婚后,他们育有四个孩子:让娜(1887年出生)、伊冯娜(1889年出生)、亨丽埃特(1891年出生)和莱昂(1893年出生)。

庞加莱很快在欧洲数学界崭露头角,吸引了许多著名数学家的关注。1881年,庞加莱受邀担任巴黎大学理学院的教职,并接受了邀请。1883年至1897年间,他在高等师范学校教授数学分析。

1881至1882年,庞加莱创立了数学的新分支——微分方程的定性理论。他展示了如何在不解方程的情况下,推导出关于一组解的行为的最重要信息(因为有时解方程是不可行的)。他成功地将这种方法应用于天体力学和数学物理中的问题。
\subsubsection{职业生涯}
庞加莱从未完全放弃在矿业行政方面的职业生涯,他一直从事数学工作之外的其他职责。他曾在1881至1885年间,在公共服务部担任负责北方铁路发展的工程师。最终,他于1893年成为矿业部的总工程师,并于1910年升任监察总工程师。

从1881年开始,直到他职业生涯的结束,庞加莱一直在巴黎大学(索邦大学)任教。最初,他被任命为分析学副教授(maître de conférences d'analyse)。最终,他担任了物理与实验力学、数学物理与概率论、天体力学与天文学等多个学科的教席。

1887年,年仅32岁的庞加莱当选为法国科学院院士。1906年,他成为法国科学院院长,并于1908年3月5日当选为法兰西学院院士。

1887年,庞加莱赢得了瑞典国王奥斯卡二世的数学竞赛,解决了与多个轨道天体自由运动有关的三体问题。(详见三体问题部分)

1893年,庞加莱加入了法国经度局,参与全球时间同步的工作。1897年,庞加莱支持了一项关于圆周量度的十进制化提案,尽管这项提案未获成功,因此时间和经度的十进制化未能实现。正是这项工作促使他开始思考建立国际时区和相对运动体之间的时间同步问题。(详见相对论部分)

1904年,他参与了阿尔弗雷德·德雷福斯案件的审判,驳斥了针对德雷福斯的伪科学证据。

庞加莱曾在1901至1903年间担任法国天文学会(Société Astronomique de France,SAF)会长。

\textbf{学生}

庞加莱在巴黎大学有两位杰出的博士生,路易·巴谢利埃(Louis Bachelier,1900年)和迪米特里·庞佩乌(Dimitrie Pompeiu,1905年)。

\subsubsection{去世}
\begin{figure}[ht]
\centering
\includegraphics[width=6cm]{./figures/187243deafdc671b.png}
\caption{庞加莱家族墓地位于蒙帕纳斯公墓。} \label{fig_HLPJL_3}
\end{figure}
1912年,庞加莱因前列腺问题接受了手术,随后于1912年7月17日在巴黎因栓塞去世,享年58岁。他葬于巴黎蒙帕纳斯公墓庞加莱家族墓地,16区,靠近Émile-Richard路的门口。

2004年,前法国教育部长克劳德·阿列格尔提议将庞加莱重新安葬在巴黎的万神殿,该地点专门用于安葬最受尊敬的法国公民。
\subsection{工作}
\subsubsection{概述}
庞加莱在纯数学和应用数学的多个领域做出了许多贡献,包括:天体力学、流体力学、光学、电学、电报学、毛细作用、弹性学、热力学、势能理论、量子力学、相对论和物理宇宙学。

他具体贡献的领域包括:
\begin{itemize}
\item 代数拓扑(庞加莱几乎是这个领域的创始人)
\item 多复变函数的解析函数理论
\item 阿贝尔函数理论
\item 代数几何
\item 庞加莱猜想(由格里高利·佩雷尔曼于2003年证明)
\item 庞加莱复现定理
\item 双曲几何
\item 数论
\item 三体问题
\item 丢番图方程理论
\item 电磁学
\item 特殊相对论
\item 基本群
\end{itemize}
在微分方程领域,庞加莱提出了许多对微分方程的定性理论至关重要的结果,例如庞加莱球面和庞加莱映射。

庞加莱还提出了“每个人对误差正态法则的信仰”(参见正态分布),并发表了影响深远的论文,提供了一种新的数学论证来支持量子力学。
\subsubsection{三体问题}  
自牛顿时代以来,寻找超过两个天体在太阳系中运动的一般解一直困扰着数学家。最初,这被称为三体问题,后来扩展为n体问题,其中n代表任何数量的相互绕行的天体。n体问题被认为在19世纪末是一个非常重要且具有挑战性的问题。事实上,在1887年,为了纪念他60岁生日,瑞典国王奥斯卡二世(Oscar II)在戈斯塔·米塔格-莱夫勒(Gösta Mittag-Leffler)的建议下设立了一个奖金,奖励能解出该问题的人。公告的内容相当具体:

给定一个由任意多个质量点组成的系统,这些点按照牛顿定律相互吸引,并假设没有两个点会相撞,试图找出每个点的坐标表示为一个关于时间的已知函数的级数,并且对于所有这些时间值,该级数均匀收敛。

如果无法解决该问题,那么任何对经典力学的重要贡献都将被视为值得奖赏。尽管庞加莱并未解出原始问题,奖项最终还是颁给了他。评审之一、著名数学家卡尔·魏尔斯特拉斯(Karl Weierstrass)表示:“这项工作确实不能被视为完全解决所提出的问题,但它的重要性如此之大,以至于它的发表将开启天体力学历史的新纪元。”(他最初的贡献版本甚至包含一个严重错误;详情请参见Diacu的文章和Barrow-Green的书)。最终出版的版本包含了许多重要的思想,最终引导出了混沌理论。最初陈述的三体问题最终由卡尔·F·苏德曼(Karl F. Sundman)于1912年为n = 3情况解决,并由王秋东(Qiudong Wang)在1990年代推广至n > 3体的情况。级数解的收敛速度非常慢,即使是非常短的时间间隔,也需要几百万项才能确定粒子的运动,因此这些解在数值计算中无法使用。
\subsubsection{相对论研究}
\textbf{地方时间}

庞加莱在法国经度局(Bureau des Longitudes)的工作,推动了国际时区的建立,这使他开始思考如何同步静止在地球上的时钟,这些时钟相对于绝对空间(或“光以太”)的速度不同。与此同时,荷兰理论家亨德里克·洛伦兹(Hendrik Lorentz)正在将麦克斯韦理论发展为描述带电粒子(“电子”或“离子”)运动及其与辐射的相互作用的理论。1895年,洛伦兹引入了一个辅助量(没有物理解释),称为“地方时间”:\(t' = t - \frac{vx}{c^2}\)并提出了长度收缩的假设,来解释光学和电学实验未能检测到相对于以太的运动(见迈克耳孙-莫雷实验)。庞加莱是洛伦兹理论的常驻解释者(有时也充当友善的批评者)。作为一名哲学家,庞加莱对“更深层次的意义”感兴趣。因此,他对洛伦兹的理论进行了阐释,并在此过程中提出了许多现在与特殊相对论相关的见解。在《时间的度量》(1898年)一书中,庞加莱写道:“稍微的反思足以理解,这些陈述本身没有意义。只有作为一种约定的结果,它们才有意义。”他还认为,科学家必须将光速的恒定性作为假设,以便为物理理论提供最简单的形式。基于这些假设,庞加莱在1900年讨论了洛伦兹的“奇妙发明”——地方时间,并指出这一概念出现在通过交换光信号同步移动时钟的过程中,假设光信号在移动参考系中两方向上的传播速度是相同的。