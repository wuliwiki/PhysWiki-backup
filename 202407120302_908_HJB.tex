% 核聚变
% license CCBYSA3
% type Wiki

(本文根据 CC-BY-SA 协议转载自原搜狗科学百科对英文维基百科的翻译)

在核化学中,核聚变是两个或两个以上的原子核结合形成一个或多个不同的原子核和亚原子粒子(中子或质子)的反应。反应物和产物之间的质量差异表现为能量的释放或吸收。这种质量差异是由于反应前后原子核之间原子“结合能”的差异引起的。聚变是为活跃的或“主序”恒星或其他高星等恒星提供能量的过程。

产生比铁-56或镍-62更轻的原子核的聚变过程通常会产生净能量释放。这些元素每个核子的质量最小,每个核子的结合能最大。轻元素向这些方向的聚变释放能量(放热过程),而产生比这些元素更重的原子核的聚变将导致产生的核子保留能量,而产生的反应是吸热的。相反的情况适用于相反的过程,核子裂变。这意味着较轻的元素,如氢和氦,通常更易聚变;而较重的元素,例如铀,钍和钚,更容易裂变。极端天体物理学的事件例如超新星能产生足够的能量将原子核融合成比铁还重的元素。

1920年,亚瑟·爱丁顿提出氢氦聚变可能是恒星能量的主要来源。1929年弗里德里希·洪德(Friedrich Hund)发现了量子隧穿现象,不久之后,罗伯特·阿特金森(Robert Atkinson)和弗里茨·霍特曼斯(Fritz Houtermans)利用测量到的轻元素质量,证明了融合小原子核可以释放出大量能量。在欧内斯特•卢瑟福(Ernest Rutherford)早期核嬗变实验的基础上,1932年,马克•奥列芬特(Mark Oliphant)在实验室完成了氢同位素的融合。在那十年剩下的时间里,汉斯·贝特(Hans Bethe)提出了恒星核聚变主要周期的理论。为军事目的而进行的核聚变研究始于20世纪40年代初,是曼哈顿计划(Manhattan Project)的一部分。核聚变是在1951年的温室项目核试验中完成的。1952年11月1日,在一次爆炸中进行了大规模的核聚变实验。。

自20世纪40年代以来,在聚变反应堆内开发受控聚变的研究一直在进行,但该技术仍处于发展阶段。。

\subsection{过程}
\begin{figure}[ht]
\centering
\includegraphics[width=6cm]{./figures/57966ab97be822f7.png}
\caption{氘与氚的融合产生了氦-4 ,释放出中子,并释放出17.59 MeV能量作为产物的动能,同时相应量的质量消失,符合动力学 $E = \Delta mc2$,其中$\Delta m$是减少总粒子的静止质量。[1]} \label{fig_HJB_1}
\end{figure}

轻元素聚变释放能量是由于两种相反力的相互作用:一种是将质子和中子结合在一起的核力,另一种是导致质子相互排斥的库仑力。质子带正电荷,在库仑力的作用下相互排斥,但它们仍然可以粘在一起,这证明了另一种短程力的存在,这种力被称为原子核引力。[1]轻核(或比铁和镍小的核)足够小,质子较少,允许核力克服库仑力排斥作用。这是因为原子核足够小,所有的核子都感受到短程吸引力,它们所受的核力与受到的无限程库仑排斥力一样强烈。通过核聚变从较轻的原子核构建原子核,会从粒子的净吸引力中释放额外的能量。然而,对于较大的原子核,没有能量释放,因为核力是短程的,并且不能在较长的核长度尺度上继续作用。因此,能量不会随着这种原子核的融合而释放;相反,这些过程需要能量作为输入。

核聚变为恒星提供能量,并在一个叫做核合成的过程中产生几乎所有的元素。太阳是一颗主序星,因此,它的能量来自于氢核与氦的核聚变。在其核心,太阳每秒融合6.2亿吨氢,每秒产生6.06亿吨氦。恒星中较轻元素的融合释放出能量质量。例如,在两个氢核融合形成氦的过程中,0.7\%的质量以$\alpha$粒子动能或其他形式的能量(如电磁辐射)的形式被带走。[2]

迫使原子核融合需要相当大的能量,即使是最轻的元素——氢。当加速到足够高的速度时,原子核可以克服这种静电斥力,并靠得足够近,使得吸引力核力大于排斥库仑力。一旦原子核足够接近,强力会迅速增长,聚变的核子实质上就会“落入”彼此之中,其结果就是聚变和产生的净能量。较轻原子核的融合产生了较重的原子核,通常还会产生一个自由的中子或质子,而且释放的能量比迫使原子核聚合所需的能量要多;这是一个可以产生自我维持反应的放热过程。

大多数核反应释放的能量比化学反应大得多,因为把原子核结合在一起的结合能比把电子结合在原子核上的能量大。例如,在氢原子核中加入一个电子所获得的电离能是13.6 ev,比图中所示的氘-氚(D-T)反应所释放的17.6 MeV的百万分之一还少;聚变反应的能量密度比核裂变大很多倍;这些反应每单位质量产生的能量要大得多,尽管个别的裂变反应通常比个别的聚变反应能量大得多,而聚变反应本身的能量是化学反应的几百万倍。只有直接将质量转化为能量,例如由物质和反物质的湮灭碰撞所引起的能量,在单位质量上才比核聚变具有更高的能量。(完全转换一克物质会释放出9×1013焦耳的能量。)

利用核聚变发电的研究已经进行了60多年。受控聚变的成功实现一直受到科技难题的阻碍;尽管如此,已经取得了重要进展。目前,受控聚变反应还不能产生“保本的”(自维持)受控聚变。最先进的两种方法是磁约束(环形设计)和惯性约束(激光设计)。

理论上,环形反应堆的聚变能量是将等离子体加热到所需温度所需能量的10倍,这种反应堆的可行设计正在开发中。ITER设施预计将在2025年完成建设阶段。它将在同年开始运行该反应堆,并在2025年启动等离子体实验,但预计要到2035年才能开始完全氘氚融合。[3]

美国国家点火装置使用激光驱动惯性约束核聚变,旨在实现核聚变的收支平衡;第一次大规模的激光目标实验于2009年6月进行,点火实验于2011年初开始。[4][5]

\subsection{恒星中的核聚变}
\begin{figure}[ht]
\centering
\includegraphics[width=6cm]{./figures/c891451119cbd119.png}
\caption{质子-质子连锁反应I分支在太阳大小或更小的恒星中占主导地位。} \label{fig_HJB_2}
\end{figure}
一个重要的聚变过程是为恒星(如太阳)提供能量的恒星核合成。二十世纪,人们认识到核聚变反应释放的能量造成恒星释放热和光的原因。恒星中的原子核融合从最初的氢和氦丰度开始,提供这种能量,并作为融合过程的副产品合成新的原子核。不同的反应链涉及,取决于恒星的质量(因此也取决于其核心的压力和温度)。

1920年左右,亚瑟·爱丁顿在他的论文中《恒星的内部结构》预测了恒星的核聚变过程的发现和机制。[6][7]当时,恒星能量的来源完全是个谜;爱丁顿正确地推测了氢融合成氦的来源,根据爱因斯坦方程,释放出巨大的能量 $E = mc^2$。这是一个特别显著的发展,因为当时聚变和热核能,甚至恒星主要由氢组成,还没有被发现。爱丁顿的论文基于当时的知识,推论道:
\begin{figure}[ht]
\centering
\includegraphics[width=6cm]{./figures/ddea438d8c660fa7.png}
\caption{碳氮氧循环循环在比太阳重的恒星中占主导地位。} \label{fig_HJB_3}
\end{figure}
\begin{enumerate}
\item 恒星能量的主要理论,即收缩假说,由于角动量守恒,应该会使恒星的自转明显加快。但是对造父变星的观察表明,这并没有发生。
\item 唯一已知的其他似乎可信的能源是物质向能量的转化;爱因斯坦几年前就已经证明,少量的物质等于大量的能量。
\item 弗朗西斯·阿斯顿(Francis Aston)最近还发现,一个氦原子的质量大约比四个氢原子结合形成一个氦原子的质量低0.8\%,这表明,如果这种结合能够发生,它将作为一种副产品释放出相当大的能量。
\item 如果一颗恒星只含有5\%的可熔氢,这就足以解释恒星是如何获得能量的。(我们现在知道,大多数‘普通’恒星的氢含量远远超过5\%)
\item 其他的元素也可能被融合,其他科学家推测,恒星是轻元素结合产生重元素的“坩埚”,但如果没有对它们的原子质量进行更精确的测量,当时就什么也说不出来了。
\end{enumerate}
在接下来的几十年里,所有这些推测都被证明是正确的。

太阳能量和类似大小的恒星的主要来源是氢聚变形成氦(质子-质子链反应),它发生在1400万K的太阳核心温度下。最终的结果是四个质子融合成一个粒子,释放出两个正电子和两个中微子(其中两个质子变成中子)和能量。在较重的恒星中,碳氮氧循环和其他过程更重要。当恒星耗尽大部分氢时,它开始合成更重的元素。这种聚变发生在质量更大的恒星在生命结束时经历一颗猛烈的超新星,这个过程被称为超新星核合成。

\subsection{聚变发生的条件}
\begin{figure}[ht]
\centering
\includegraphics[width=8cm]{./figures/297c340dea4ce7ef.png}
\caption{正电荷原子核之间的静电力是排斥的,但是当间距足够小时,量子效应将隧道穿过壁。因此,核聚变的先决条件是两个原子核足够靠近,在足够长的时间内量子隧穿发生作用。} \label{fig_HJB_4}
\end{figure}

在发生聚变之前,必须克服静电力的巨大能量障碍。在很远的距离,两个裸露的原子核互相排斥,因为它们带正电荷的质子之间有排斥力。然而,如果两个原子核能靠得足够近,静电斥力就能被量子效应克服,在量子效应中,原子核可以通过库仑力穿隧。

当一个核子(如质子或中子)被加到原子核上时,核力会把它吸引到原子核的所有其他核子上(如果原子足够小的话),但主要是吸引到它的近邻上,因为核力的作用范围很短。原子核内部的核子比表面的核子有更多的相邻核子。由于较小的原子核具有较大的表面积体积比,因此核力作用下的每核子结合能一般随原子核的大小而增大,但趋近于与直径约为4个核子的原子核的结合能的极限值。重要的是要记住核子是量子物体。例如,由于原子核中的两个中子彼此相同,区分两个中子的目标,例如哪个在内部,哪个在表面,实际上是没有意义的,因此包含量子力学对于正确的计算是必要的。

另一方面,静电力是一个平方反比力,所以加入原子核的质子会感受到原子核中所有其他质子的静电斥力。由于静电力,每个核子的静电能随原子核原子数的增加而无限制地增加。

作用效果相反的静电力和强核力最终效果是使每个核子的结合能通常随着尺寸的增加而增加,直到元素铁和镍,然后随原子核的数量增加而减少。最终,结合能变成负的,非常重的原子核(都有超过208个核子,相当于直径约为6个核子)是不稳定的。四个结合最紧密的原子核,按每个核子的结合能递减顺序是$^{62}Ni,^{58}Fe,^{56}Fe,$和$^{60}Ni$。[8]镍同位素,$^{62}Ni$较稳定,铁同位素$^{56}Fe$更常见。这是因为恒星没有简单的方法来产生$^{62}Ni$,只有通过通过$\alpha$反应过程。

这一普遍趋势的一个例外是氦-4原子核,它的结合能高于仅次于它的最重元素锂。这是因为质子和中子是费米子,根据泡利不相容原理,它们不可能以完全相同的状态存在于同一个原子核中。原子核中每个质子或中子的能态都能同时容纳一个自旋向上的粒子和一个自旋向下的粒子。氦-4具有异常大的结合能,因为它的原子核由两个质子和两个中子组成,所以它的四个核子都处于基态。任何额外的核子都会进入高能态。事实上,氦-4原子核是如此紧密地结合在一起,以至于在核物理中它通常被视为一个单一的量子力学粒子,即$\alpha$粒子。

如果把两个原子核结合在一起,情况也是类似的。当它们相互靠近时,一个原子核中的所有质子排斥另一个原子核中的所有质子。直到两个原子核足够长时间地靠近,强核力才会(通过隧穿)取代排斥性静电力。因此,即使最后的能量状态较低,也有一个必须首先克服的巨大能量障碍。它叫做库仑势垒。

对于氢的同位素来说,库仑势垒是最小的,因为它们的原子核只含有一个正电荷。双质子是不稳定的,所以中子也必须参与其中,理想情况下,氦原子核与它的结合非常紧密,是产物之一。

使用氘氚燃料,产生的能量势垒约为0.1兆电子伏。相比之下,从氢中移除一个电子所需的能量是13.6 eV,大约是所需能量的7500倍。聚变的(中间)结果是一个不稳定的$^{5}He$原子核,它立即射出一个14.1 MeV的中子。剩余$^{4}He$核的反冲能为3.5 MeV,释放的总能量为17.6 MeV。这比克服能量障碍所需的能量要多很多倍。
\begin{figure}[ht]
\centering
\includegraphics[width=10cm]{./figures/a0ef5d289d9df2f4.png}
\caption{聚变反应速率随着温度的升高而迅速增加,直到达到最大值,然后逐渐下降。DT速率在较低的温度(约70kev,或8亿开尔文)达到峰值,并且比通常认为的其他聚变能反应的速率值要高。} \label{fig_HJB_5}
\end{figure}
反应\textbf{截面$\sigma$}是聚变反应概率的一种量度,是两个反应物原子核相对速度的函数。如果反应物具有速度分布,例如热分布,那么对横截面积和速度的分布进行平均是有效的。这个平均值被称为“反应性”,表示为$<\sigma v >$。反应速率(每次每体积的聚变)为$<\sigma v >$乘以反应物数量密度的乘积:
$$f=n_1n_2<\sigma v>~$$
如果一种原子核与自身一样的原子核反应,例如DD反应,那么产物$n_1n_2$必须替换为$(1/2)n^2$。

$<\sigma v >$从室温下的几乎为零增加到温度为 10 – 100 keV时的有意义幅度数值。在这些温度下,聚变反应物以等离子态存在,远远高于典型的电离能(氢的电离能为13.6 eV)。

$<\sigma v >$通过考虑劳森准则,可以得到具有特定能量的器件中温度的函数限制时间。这是一个极具挑战性的障碍,需要在地球上克服,这就解释了为什么核聚变研究花了许多年才达到目前的先进技术水平。[9]

\subsection{实现聚变的方法}
\subsubsection{4.1 热核聚变}
如果物质被充分加热(因此成为等离子体)并受到限制,则由于粒子的极端热动能的碰撞,可能会发生聚变反应。热核武器产生的核聚变能量是无法控制的。受控热核融合的概念是利用磁场来限制等离子体。
\subsubsection{4.2 惯性约束核聚变}
惯性约束聚变(ICF)是一种旨在通过加热和压缩燃料靶(通常是含有氘和氚的小球)来释放聚变能的方法。
\subsubsection{4.3 惯性静电约束}
惯性静电约束是一套利用电场将离子加热到熔融状态的设备。最著名的是《fusor 》。从1999年开始,许多业余爱好者已经能够使用这些自制的设备进行业余的核聚变。[10][11][12][13]其他IEC设备包括:Polywell ,MIX POPS[14]和“大理石”概念。[15]
\subsubsection{4.4 波束-波束或波束-目标融合}
如果引发反应的能量来自于加速原子核中的一个,这个过程称为波束-目标融合;如果两个原子核都被加速,那就是波束-波束融合。

基于加速器的光离子融合是一种利用粒子加速器获得足够的粒子动能来引发光离子融合反应的技术。加速光离子是相对容易的,并且可以在一个有效的方式中完成——只需要一个真空管,一对电极,和一个高压变压器;在两个电极之间用10kv的电压就可以观察到融合。基于加速器的聚变(通常是冷目标)的关键问题是聚变截面比库仑相互作用截面低许多个数量级。因此,绝大多数离子消耗它们的能量来发射轫致辐射和靶原子的电离。被称为密封管中子发生器的装置与这一讨论特别相关。这些小型设备是填充有氘和氚气体的微型粒子加速器,其排列方式允许这些原子核的离子相对于氢化物靶加速,氢化物靶中也包含氘和氚,在这里发生聚变,释放出中子通量。每年全球生产数百台中子发生器,供石油工业使用,并用于测量设备,以确定和绘制石油储量。

为了克服束-靶融合中的轫致辐射问题,$3 \alpha$太阳能源公司提出了一种组合方法,该方法基于两个方向相反的等离子体团的相互渗透。[16]理论工作表明,通过创建和加热两个加速的迎面碰撞的等离子体团,使其能达到几千电子伏的热能,这与热核聚变所需的热能相比是低的,即使使用像P-11B这样的电子燃料,也可以获得净聚变增益。为了通过这种方法达到盈亏平衡的必要条件,加速等离子体必须有足够的碰撞速度,根据聚变燃料的种类,其速度大约为每秒几千公里(10^6米/秒)。此外,等离子体密度必须介于惯性和磁融合标准之间。