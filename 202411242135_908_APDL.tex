% 安培力定律(综述)
% license CCBYSA3
% type Wiki

本文根据 CC-BY-SA 协议转载翻译自维基百科\href{https://en.wikipedia.org/wiki/Amp\%C3\%A8re\%27s_force_law}{相关文章}。

在静磁学中,两根载流导线之间的吸引或排斥力(见下方第一幅图)通常被称为安培力定律。这种力的物理来源是每根导线根据毕奥-萨伐尔定律产生磁场,而另一根导线则根据洛伦兹力定律受到磁力作用。

\subsection{公式}
\subsubsection{特殊情况:两条直平行导线}
安培力定律中最为人熟知且最简单的例子(在2019年5月20日之前[1])用来定义电流的国际单位制(SI)单位\textbf{安培}。该定律表述,两条直平行导线之间的每单位长度的磁力为:
\[
\frac{F_m}{L} = 2k_{\text{A}} \frac{I_1 I_2}{r},~
\]
其中: \( k_{\text{A}} \) 是由毕奥-萨伐尔定律定义的磁力常数;\( \frac{F_m}{L} \) 是单位长度上的总磁力(在较短导线上,较长导线被近似为相对于较短导线无限长); \( r \) 是两导线之间的距离;\( I_1 \) 和 \( I_2 \) 是两导线中传输的直流电流。

这种公式在以下情况下是良好的近似:如果一根导线的长度远大于另一根,可以将较长的导线近似为无限长;
如果两根导线之间的距离相对于导线的长度较小(使得无限长导线近似成立),但同时相对于导线的直径又较大(使得导线可以被近似为无限细的线)。\( k_{\text{A}} \) 的值取决于所选的单位系统,而 \( k_{\text{A}} \) 的值决定了电流单位的大小。

在国际单位制(SI)中[2][3]:
\[
k_{\text{A}} = \frac{\mu_0}{4\pi},~
\]
其中:\( \mu_0 \) 是磁常数(在国际单位中,称为真空磁导率)。

SI单位中,磁常数的值为:
\[
\mu_0 = 1.25663706212(19) \times 10^{-6} \, \text{H/m}.~
\]
\subsubsection{一般情况}

针对任意几何形状的磁力的一般公式基于迭代线积分,并将毕奥-萨伐尔定律和洛伦兹力结合在一个公式中,具体如下:[4][5][6]
\[
\mathbf{F}_{12} = \frac{\mu_0}{4\pi} \int_{L_1} \int_{L_2} \frac{I_1 d\boldsymbol{\ell}_1 \times \left( I_2 d\boldsymbol{\ell}_2 \times \hat{\mathbf{r}}_{21} \right)}{|r|^2},~
\]
其中:
\begin{itemize}
\item \( \mathbf{F}_{12} \) 是由导线 2 对导线 1 施加的总磁力(通常以牛顿为单位测量);
\item \( I_1 \) 和 \( I_2 \) 分别是通过导线 1 和导线 2 的电流(通常以安培为单位测量);
\item 双重线积分表示导线 2 上的每个微元对导线 1 上的每个微元产生的磁力的总和;
\item \( d\boldsymbol{\ell}_1 \) 和 \( d\boldsymbol{\ell}_2 \) 是与导线 1 和导线 2 对应的微小向量(通常以米为单位),关于线积分的详细定义可参考相关资料;
\item \( \hat{\mathbf{r}}_{21} \) 是从导线 2 上的微元指向导线 1 上微元的单位向量,|r|** 是这两个微元之间的距离;
\item 符号 × 表示矢量叉乘;
\item 电流 \( I_n \) 的正负号取决于其与 \( d\boldsymbol{\ell}_n \) 的方向关系(例如,如果 \( d\boldsymbol{\ell}_1 \) 指向传统电流方向,则 \( I_1 > 0 \))。
\end{itemize}

若需确定材料介质中导线之间的磁力,则需要将磁常数 \( \mu_0 \) 替换为介质的实际磁导率。

通过展开向量三重积并应用斯托克斯定理,该定律可以以以下等效形式重写:[7]
\[
\mathbf{F}_{12} = -\frac{\mu_0}{4\pi} \int_{L_1} \int_{L_2} \frac{\left( I_1 d\boldsymbol{\ell}_1 \cdot I_2 d\boldsymbol{\ell}_2 \right) \hat{\mathbf{r}}_{21}}{|r|^2}.~
\]
在这种形式下,可以立即看出,由导线 2 对导线 1 施加的力与由导线 1 对导线 2 施加的力大小相等方向相反,这与牛顿第三定律一致。
\subsection{历史背景}
\begin{figure}[ht]
\centering
\includegraphics[width=10cm]{./figures/4b3c60656ba01f5b.png}
\caption{安培原始实验的示意图} \label{fig_APDL_1}
\end{figure}

1873年,詹姆斯·克拉克·麦克斯韦推导出了通常表述的安培力定律,这是与安培和高斯的原始实验一致的多种表达式之一。关于两个直线电流 \(I\) 和 \(I'\) 之间的力的 \(x\) 分量(见相邻图示),安培在1825年和高斯在1833年分别给出了如下公式:[8]
\[
dF_x = kII' ds' \int ds \frac{\cos(x ds) \cos(r ds') - \cos(rx) \cos(ds ds')}{r^2}.~
\]
安培之后,众多科学家(包括威廉·韦伯、鲁道夫·克劳修斯、麦克斯韦、伯恩哈德·黎曼、赫尔曼·格拉斯曼和瓦尔特·里茨)对这一表达式进行了发展,试图寻找磁力的基本表达式。通过微分计算可以得到:
\[
\frac{\cos(x\,ds) \cos(r\,ds')}{r^2} = -\cos(rx) \frac{\cos \varepsilon - 3 \cos \phi \cos \phi'}{r^2},~
\]
以及:
\[
\frac{\cos(rx) \cos(ds\,ds')}{r^2} = \frac{\cos(rx) \cos \varepsilon}{r^2}.~
\]
基于这些表达式,安培力定律可以进一步表示为:
\[
dF_x = kII' ds' \int ds' \cos(rx) \frac{2 \cos \varepsilon - 3 \cos \phi \cos \phi'}{r^2}.~
\]
此外,使用以下关系:
\[
\frac{\partial r}{\partial s} = \cos \phi,\ \frac{\partial r}{\partial s'} = -\cos \phi',~
\]
以及:
\[
\frac{\partial^2 r}{\partial s \partial s'} = \frac{-\cos \varepsilon + \cos \phi \cos \phi'}{r}.~
\]
可以将安培的结果表示为:
\[
d^2F = \frac{kII' ds ds'}{r^2} \left( \frac{\partial r}{\partial s} \frac{\partial r}{\partial s'} - 2r \frac{\partial^2 r}{\partial s \partial s'} \right).~
\]
麦克斯韦指出,该表达式中可以添加关于函数 \(Q(r)\) 的导数项,这些项在积分时会相互抵消。麦克斯韦给出了与实验事实一致的“最通用形式”:
\[
d^2F_x = kII' ds ds' \frac{1}{r^2} \left[ \left( \frac{\partial r}{\partial s} \frac{\partial r}{\partial s'} - 2r \frac{\partial^2 r}{\partial s \partial s'} + r \frac{\partial^2 Q}{\partial s \partial s'} \right) \cos(rx) + \frac{\partial Q}{\partial s'} \cos(x\,ds) - \frac{\partial Q}{\partial s} \cos(x\,ds') \right].~
\]

麦克斯韦指出,对于一个封闭电路,函数 \(Q(r)\) 的形式无法通过实验直接确定。假设 \(Q(r)\) 的形式为:
\[
Q = -\frac{(1+k)}{2r}.~
\]
我们得到 ds 上由 ds' 施加的力的一般表达式:
\[
d^{2}\mathbf{F} = -\frac{kII'}{2r^{2}} \left[ \left(3-k\right)\hat{\mathbf{r}}_{1}\left(d\mathbf{s} \, d\mathbf{s}'\right) - 3\left(1-k\right)\hat{\mathbf{r}}_{1}\left(\hat{\mathbf{r}}_{1} d\mathbf{s}\right)\left(\hat{\mathbf{r}}_{1} d\mathbf{s}'\right) - \left(1+k\right)d\mathbf{s}\left(\hat{\mathbf{r}}_{1} d\mathbf{s}'\right) - \left(1+k\right)d\mathbf{s}'\left(\hat{\mathbf{r}}_{1} d\mathbf{s}\right) \right].~
\]
通过对 \(s'\) 的积分可以消去 \(k\),从而恢复安培和高斯给出的原始表达式。因此,就安培的原始实验而言,\(k\) 的取值并没有实际意义。安培取 \(k = -1\);高斯取 \(k = +1\),格拉斯曼和克劳修斯也取 \(k = +1\),不过克劳修斯忽略了 \(S\) 分量。在非以太电子理论中,韦伯取 \(k = -1\),而黎曼取 \(k = +1\)。里兹在其理论中未明确指定 \(k\) 的值。如果取 \(k = -1\),我们得到安培的表达式:
\[
d^{2}\mathbf{F} = -\frac{kII'}{r^{3}}\left[2\mathbf{r}(d\mathbf{s} \, d\mathbf{s'}) - 3\mathbf{r}(\mathbf{r} d\mathbf{s})(\mathbf{r} d\mathbf{s'})\right]~
\]
如果取 \(k=+1\),我们得到:
\[
d^{2}\mathbf{F} = -\frac{kII'}{r^{3}}\left[\mathbf{r}(d\mathbf{s} \, d\mathbf{s'}) - d\mathbf{s}(\mathbf{r} \, d\mathbf{s'}) - d\mathbf{s'}(\mathbf{r} \, d\mathbf{s})\right]~
\]

利用三重叉积的矢量恒等式,可以将结果表示为:
\[
d^{2}\mathbf{F} = \frac{kII'}{r^{3}}\left[\left(d\mathbf{s} \times d\mathbf{s'} \times \mathbf{r}\right) + d\mathbf{s'}(\mathbf{r} \, d\mathbf{s})\right]~
\]
对 \(d\mathbf{s'}\) 积分时,第二项为零,因此可以得到麦克斯韦形式的安培力定律:
\[
\mathbf{F} = kII'\iint \frac{d\mathbf{s} \times (d\mathbf{s'} \times \mathbf{r})}{|r|^{3}}~
\]
\subsection{从一般公式推导平行直导线的情形}
从一般公式开始:
\[
\mathbf{F}_{12} = \frac{\mu_{0}}{4\pi} \int_{L_{1}} \int_{L_{2}} \frac{I_{1} d\boldsymbol{\ell}_{1} \times (I_{2} d\boldsymbol{\ell}_{2} \times \hat{\mathbf{r}}_{21})}{|r|^{2}},~
\]
假设导线 2 位于 \(x\)-轴上,导线 1 位于 \(y=D, z=0\),与 \(x\)-轴平行。令 \(x_1, x_2\) 分别表示导线 1 和导线 2 的微小线元的 \(x\)-坐标。换句话说,导线 1 的微小线元位于 \((x_1, D, 0)\),而导线 2 的微小线元位于 \((x_2, 0, 0)\)。根据线积分的性质:\(d\boldsymbol{\ell}_1 = (dx_1, 0, 0)\) 且\(d\boldsymbol{\ell}_2 = (dx_2, 0, 0).\)

此外,
\[\hat{\mathbf{r}}_{21} = \frac{1}{\sqrt{(x_1-x_2)^2 + D^2}}(x_1-x_2, D, 0),~\]
且
\[|r| = \sqrt{(x_1-x_2)^2 + D^2}.~\]
因此积分为:
\[
\mathbf{F}_{12} = \frac{\mu_0 I_1 I_2}{4\pi} \int_{L_1} \int_{L_2} \frac{(dx_1, 0, 0) \times \left[(dx_2, 0, 0) \times (x_1-x_2, D, 0)\right]}{|(x_1-x_2)^2 + D^2|^{3/2}}.~
\]
计算叉积:
\[
\mathbf{F}_{12} = \frac{\mu_0 I_1 I_2}{4\pi} \int_{L_1} \int_{L_2} dx_1 dx_2 \frac{(0, -D, 0)}{|(x_1-x_2)^2 + D^2|^{3/2}}~
\]
接下来对 \(x_2\) 从 \(-\infty\) 到 \(+\infty\) 积分:
\[
\mathbf{F}_{12} = \frac{\mu_0 I_1 I_2}{4\pi} \frac{2}{D}(0, -1, 0) \int_{L_1} dx_1.~
\]
如果导线 1 也是无限长的,则积分会发散,因为两条无限长平行导线之间的总吸引力是无限大的。实际上,我们真正关心的是导线 1 单位长度上的吸引力。因此,假设导线 1 有一个很大但有限的长度 \(L_1\)。那么导线 1 受到的力矢量为:
\[
\mathbf{F}_{12} = \frac{\mu_0 I_1 I_2}{4\pi} \frac{2}{D}(0, -1, 0)L_1.~
\]
正如预期的那样,导线受到的力与其长度成正比。单位长度上的力为:
\[
\frac{\mathbf{F}_{12}}{L_1} = \frac{\mu_0 I_1 I_2}{2\pi D}(0, -1, 0)~
\]
力的方向沿 \(y\)-轴,这表示如果电流是平行的,导线 1 会被拉向导线 2。单位长度的力大小与上文给出的 \(\frac{F_m}{L}\) 表达式一致。
\subsection{著名推导}

按时间顺序排列:

\begin{itemize}
\item 安培最初的1823年推导:
\item Assis, André Koch Torres; Chaib, J. P. M. C.; Ampère, André-Marie (2015)。Ampère's electrodynamics: analysis of the meaning and evolution of Ampère's force between current elements, together with a complete translation of his masterpiece: Theory of electrodynamic phenomena, uniquely deduced from experience (PDF)。蒙特利尔:Apeiron。ISBN 978-1-987980-03-5。
\item 麦克斯韦1873年的推导:
\item Treatise on Electricity and Magnetism*,第2卷,第4部分,第2章(§§502–527)。
\item 皮埃尔·杜亨1892年的推导:
\item Duhem, Pierre Maurice Marie (2018年9月9日)。Ampère's Force Law: A Modern Introduction。Alan Aversa(翻译)。[doi:10.13140/RG.2.2.31100.03206/1](https://doi.org/10.13140/RG.2.2.31100.03206/1)。2019年7月3日检索。(EPUB)。
\item 翻译自:Leçons sur l’électricité et le magnétisme,第3卷,第14册附录,第309–332页(法文)。
\item 阿尔弗雷德·奥拉希利1938年的推导:
\item Electromagnetic Theory: A Critical Examination of Fundamentals,第1卷,第102–104页。
\end{itemize}
\subsection{另见}
\begin{itemize}
\item 安培  
\item 磁常数  
\item 洛伦兹力  
\item 安培环路定律  
\item 真空(自由空间
\end{itemize}
\subsection{参考文献和注释}\\
1. "第26届CGPM决议" (PDF). BIPM. 检索于2020年8月1日。\\
2. Raymond A Serway & Jewett JW (2006). Serway的物理学原理:基于微积分的文本 (第四版). 贝尔蒙特,加利福尼亚:汤普森Brooks/Cole. 第746页. ISBN 0-534-49143-X。\\
3. Paul M. S. Monk (2004). 物理化学:理解我们的化学世界. 纽约:奇切斯特:Wiley. 第16页. ISBN 0-471-49181-0。\\
4. 此表达式的被积函数出现在关于定义安培的官方文档中。BIPM SI Units brochure, 8th Edition, 第105页。\\
5. Tai L. Chow (2006). 现代视角下的电磁理论导论. 波士顿:Jones and Bartlett. 第153页. ISBN 0-7637-3827-1。\\
6. 安培力定律。参见“积分方程”部分获取公式。\\
7. Christodoulides, C. (1988). “安培和毕奥-萨伐尔静磁力定律在线电流元形式中的比较”. 美国物理学杂志. 56 (4): 357–362. Bibcode:1988AmJPh..56..357C. doi:10.1119/1.15613。\\
8. O'Rahilly, Alfred (1965). 电磁理论. Dover. 第104页. (参见 Duhem, P. (1886). "关于安培定律". 理论物理学杂志. 5 (1): 26–29. doi:10.1051/jphystap:018860050020601. 检索于2015年1月7日,该文出现在 Duhem, Pierre Maurice Marie (1891). *电与磁的教程. 第3卷. 巴黎: Gauthier-Villars.)\\
9. Petsche, Hans-Joachim (2009). 赫尔曼·格拉斯曼传记. 巴塞尔波士顿:Birkhäuser. 第39页. ISBN 9783764388591。\\
10. Maxwell, James Clerk (1904). 电磁学教程. 牛津. 第173页。\\
\subsection{外部链接}
\begin{itemize}
\item 安培力定律 - 包含力矢量的动画图示。
\end{itemize}
