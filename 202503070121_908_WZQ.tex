% 完整群(综述)
% license CCBYSA3
% type Wiki

本文根据 CC-BY-SA 协议转载翻译自维基百科\href{https://en.wikipedia.org/wiki/Holonomy}{相关文章}。

\begin{figure}[ht]
\centering
\includegraphics[width=6cm]{./figures/8ee31b3dfe98af26.png}
\caption{球面上沿分段光滑路径的平行移动。初始向量标记为 \(V\),它沿着曲线被平行移动,最终得到的向量标记为 \( \mathcal{P}_\gamma (V) \)。如果路径发生变化,平行移动的结果也会不同。} \label{fig_WZQ_1}
\end{figure}
在微分几何中,光滑流形上一个联络的\textbf{平行迁移群}(holonomy)描述的是:沿着闭合回路进行平行移动时,几何数据未被保持的程度。平行迁移群是联络曲率所导致的一种普遍几何效应。对于平坦联络,相关的平行迁移群是一种单值延拓(monodromy),并且本质上是一个全局概念。而对于曲率非零的联络,平行迁移群同时具有非平凡的局部和全局特征。

任何流形上的联络都会通过其平行移动映射引出某种\textbf{平行迁移群}(holonomy)的概念。最常见的平行迁移群形式是具有某种对称性的联络。重要的例子包括:黎曼几何中Levi-Civita联络的平行迁移群(称为黎曼平行迁移群),向量丛上联络的平行迁移群,Cartan联络的平行迁移群,以及主丛上联络的平行迁移群。在这些情形下,联络的平行迁移群都可以与某个李群(即\textbf{平行迁移群})对应起来。根据Ambrose-Singer定理,联络的平行迁移群与该联络的曲率密切相关。

对\textbf{黎曼平行迁移群}(Riemannian holonomy)的研究推动了许多重要的发展。\textbf{平行迁移群}最早由\textbf{埃利·嘉当}(Élie Cartan)于1926年引入,用于研究和分类对称空间。然而,直到很久之后,平行迁移群才被用来在更一般的背景下研究黎曼几何。

1952年,\textbf{乔治·德拉姆}(Georges de Rham)证明了\textbf{德拉姆分解定理},该定理通过将切丛分解为在局部平行迁移群作用下的不可约子空间,从而将黎曼流形分解为多个黎曼流形的笛卡尔积。随后在1953年,\textbf{马塞尔·贝尔热}(Marcel Berger)对可能出现的不可约平行迁移群进行了分类。

黎曼平行迁移群的分解和分类在物理学和弦理论中都有应用。
\subsection{定义} 
\subsubsection{向量丛上联络的平行迁移群}  
设\(E\)是光滑流形\(M\)上的一个秩为\(k\)的向量丛,\(\nabla\)是\(E\)上的一个联络。对于一个以\(M\)中点 \(x\) 为基点的分段光滑闭合路径\(\gamma : [0,1] \to M\),联络\(\nabla\)定义了一个\textbf{平行移动映射}\(P_\gamma : E_x \to E_x\)它作用在 \(E\)在点\(x\)处的纤维上。这个映射是线性的且可逆的,因此它定义了一个属于一般线性群\(GL(E_x)\)的元素。联络\(\nabla\)在基点\(x\)处的\textbf{平行迁移群}(holonomy group)定义为:
\[
\operatorname{Hol}_x(\nabla) = \{P_\gamma \in \mathrm{GL}(E_x) \mid \gamma \text{ 是以 } x \text{ 为基点的闭合路径}\}.~
\]
其中\(P_\gamma\) 是路径 \(\gamma\)对应的平行移动映射。基点\(x\)处的\textbf{限制平行迁移群}(restricted holonomy group)是子群:\(\operatorname{Hol}_x^0(\nabla)\)它由\textbf{可缩至点的闭合路径} \(\gamma\)所对应的平行移动映射组成。

如果 \(M\) 是连通的,那么平行迁移群(holonomy group)对于基点 \(x\) 的依赖仅体现在一般线性群 \(GL(k, \operatorname{R})\) 中的共轭关系上。具体来说,如果 \(\gamma\) 是从 \(x\) 到 \(y\) 的一条路径,那么:
\[
\operatorname{Hol}_y(\nabla) = P_\gamma \operatorname{Hol}_x(\nabla) P_\gamma^{-1}.~
\]
此外,选择不同的方式将纤维 \(E_x\) 与 \(\operatorname{R}^k\) 进行同构,也会得到共轭的子群。  

因此,有时,特别是在一般性的或非正式的讨论中(比如下面的内容),人们会省略对基点的引用,此时默认该定义仅确定到共轭的意义下。

关于平行迁移群(holonomy group),以下是一些重要性质:
\begin{itemize}
\item \(\operatorname{Hol}^0(\nabla)\)是\(GL(k,\operatorname{R})\) 的一个连通的李子群。
\item \(\operatorname{Hol}^0(\nabla)\)是\(\operatorname{Hol}(\nabla)\) 的单位连通分量(identity component)。
\item 存在一个从基本群\(\pi_1(M)\)到商群\(\operatorname{Hol}(\nabla)/\operatorname{Hol}^0(\nabla)\) 的自然满群同态:\(\pi_1(M) \to \operatorname{Hol}(\nabla)/\operatorname{Hol}^0(\nabla)\)
其中\(\pi_1(M)\)是\(M\)的基本群,该同态将同伦类\([\gamma]\)映射为:\(P_\gamma \cdot \operatorname{Hol}^0(\nabla)\).
\item 如果\(M\)是单连通的,则:\(\operatorname{Hol}(\nabla) = \operatorname{Hol}^0(\nabla)\).
\item 当且仅当\(\nabla\)是平坦的(即曲率为零),\(\operatorname{Hol}^0(\nabla)\)是平凡群(只包含单位元)。
\end{itemize}
\subsection{主丛上联络的平行迁移群}  
主丛上联络的平行迁移群的定义与向量丛上的定义类似。  

设 \(G\) 是一个李群,\(P\) 是一个在光滑流形 \(M\) 上的\textbf{主 \(G\)-丛},并且 \(M\) 是可度量紧化的(paracompact)。设 \(\omega\) 是 \(P\) 上的一个联络。  

给定一条以 \(M\) 中点 \(x\) 为基点的分段光滑闭合路径 \(\gamma : [0,1] \to M\),以及纤维上点 \(p \in P_x\)(\(p\) 属于 \(x\) 处的纤维),该联络定义了一条唯一的\textbf{水平提升}路径:\(\tilde{\gamma} : [0,1] \to P\)
满足:\(\tilde{\gamma}(0) = p\).这条提升路径的终点 \(\tilde{\gamma}(1)\) 一般不会等于 \(p\),而是纤维中某个形如 \(p \cdot g\) 的点,其中 \(g \in G\)。

在 \(P\) 上定义一个等价关系“\(\sim\)”,规定 \(p \sim q\) 当且仅当 \(p\) 和 \(q\) 可以通过一条分段光滑的\textbf{水平路径}连接。

\(\omega\) 在基点 \(p\) 处的\textbf{平行迁移群}(holonomy group)定义为:
\[
\operatorname{Hol}_p(\omega) = \{g \in G \mid p \sim p \cdot g\}.~
\]
基点 \(p\) 处的\textbf{限制平行迁移群}(restricted holonomy group)是子群 \(\operatorname{Hol}_p^0(\omega)\),它由可缩闭合路径 \(\gamma\) 的水平提升所产生。

如果 \(M\) 和 \(P\) 都是连通的,那么平行迁移群对于基点 \(p\) 的依赖仅表现为在 \(G\) 中的共轭。具体来说,如果 \(q\) 是平行迁移群选取的另一个基点,则存在唯一的 \(g \in G\),使得:\(q \sim p \cdot g\)在这个 \(g\) 下,有:
\[
\operatorname{Hol}_q(\omega) = g^{-1}\operatorname{Hol}_p(\omega)g~
\]
特别地:
\[
\operatorname{Hol}_{p \cdot g}(\omega) = g^{-1}\operatorname{Hol}_p(\omega)g~
\]
此外,如果\(p \sim q\),则有:\(\operatorname{Hol}_p(\omega)=\operatorname{Hol}_q(\omega)\)与前面类似,有时在讨论平行迁移群时会省略基点的标注,此时默认该定义只在共轭的意义下成立。

关于平行迁移群和限制平行迁移群的一些重要性质包括:
\begin{itemize}
\item \(\operatorname{Hol}_p^0(\omega)\)是李群\(G\) 的一个\textbf{连通李子群}。
\item \(\operatorname{Hol}_p^0(\omega)\)是\(\operatorname{Hol}_p(\omega)\)的\textbf{单位连通分量}。
\item 存在一个从基本群 \(\pi_1\) 到商群 \(\operatorname{Hol}_p(\omega)/\operatorname{Hol}_p^0(\omega)\) 的\textbf{自然的满群同态}:\(\pi_1 \to \operatorname{Hol}_p(\omega)/\operatorname{Hol}_p^0(\omega)\)
\item 如果 \(M\) 是\textbf{单连通}的,那么:\(\operatorname{Hol}_p(\omega) = \operatorname{Hol}_p^0(\omega)\)
\item \(\omega\) 是\textbf{平坦的}(即曲率为零)当且仅当:\(\operatorname{Hol}_p^0(\omega)\)是\textbf{平凡子群}(只包含单位元)。
\end{itemize}