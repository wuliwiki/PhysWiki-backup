% 椭圆(高中)
% keys 极坐标系|直角坐标系|圆锥曲线|椭圆
% license Xiao
% type Tutor

\begin{issues}
\issueDraft
\end{issues}

\pentry{解析几何\nref{nod_JXJH},圆\nref{nod_HsCirc}}{nod_32e0}

% 933:这篇文章的介绍思路是主要考虑高中教材。从圆和解析几何的视角来引入椭圆本身。忽略圆锥曲线这件事。

人们常说地球绕着太阳转。提到“绕着转”,很多人往往会自然联想到完美的圆形轨道。但实际上,真正沿正圆轨道运行的行星几乎不存在。大多数行星的轨道虽然呈环绕状,但并不是正圆,而是略微“压扁”的圆。这种“被压扁的圆”在生活中并不少见。比如,一个圆形水杯的杯口,斜着看时,就会呈现出这种变形的轮廓。这种形状被称为“椭圆”。

如果你在教室里轻声说话,声音会在墙壁间反弹,有时远处的同学反而能听得更清楚。在一些老式博物馆中,有“回声墙”的设计:两个人站在大厅中一段弧形墙的两个特殊位置,即使中间隔着一面墙,也能清晰听到彼此的低语。人们研究发现,有一种特殊的曲线能实现这种效果——从某个点发出的声音或光线,经过反射后总能准确传到另一个固定点。巧的是这条曲线,也是椭圆。

椭圆是圆的“兄弟”,但更灵活、更丰富。接下来,将分别从上面两个视角来研究椭圆的定义,并在研究标准方程及其基本参数后,发现二者对应的是同一个曲线,逐步认识这一优雅而实用的图形。

\subsection{从“压扁的圆”得到椭圆}

这里我们先用之前学过的放缩的方法来研究将单位圆在 $x$ 轴和 $y$ 轴分别拉长了 $p$ 倍和 $q$ 倍,$p,q$为任意给出的定值。参考\enref{函数的变换}{FunTra}中提到的方法。假设圆的方程为:
\begin{equation}
x^2 + y^2 = 1~.
\end{equation}
此时,圆与$x$轴的交点为$(1,0),(-1,0)$,与$y$轴的交点为$(0,1),(0,-1)$。由于$x$轴和$y$轴分别只有一个操作,拉长$k$倍相当于对应变量变为$\displaystyle{1\over k}$,因此得到曲线方程为:
\begin{equation}
\left(\frac{x}{p}\right)^2 + \left(\frac{y}{q}\right)^2 = 1~.
\end{equation}
即:
\begin{equation}\label{eq_Elips3_2}
\frac{x^2}{p^2} + \frac{y^2}{q^2} = 1~.
\end{equation}
根据图像变换的结果,此时,曲线与$x$轴的交点为$(p,0),(-p,0)$,与$y$轴的交点为$(0,q),(0,-q)$。

$p,q$中更大的那个所在的对称轴称作长轴,更小的那个所在的对称轴称作短轴。

$a$ 为\textbf{半长轴}, $b$ 为\textbf{半短轴}。

\subsection{椭圆的几何定义}

从圆的定义开始,如果想要引申圆的定义,可以这样看,假设圆心是两个重合的点,然后圆上每个点到这两个点的距离相等,都是定值,现在如果把这两个点移开,那么显然如果在平面上所有到这两个点距离相等的点就构成了他们连线的垂直平分线。那么,如果加上一些不同的设定,或许会得到不同的效果。比如,圆上点到的这两个点到距离是定值,这启发我们或许可以是点到这两个点的距离之和是定值。

下面给出古希腊时代从几何视角给出的椭圆定义,这也常被称作是椭圆的第一定义。

\begin{definition}{椭圆的几何定义}
平面上到两定点的距离之和为有限定值的几何图形,称为\textbf{椭圆}。两个定点称作椭圆的两个\textbf{焦点}。
\end{definition}

可以这样看,当一个椭圆的两个焦点重合时,椭圆就变成了圆。所以可以这样说,圆是椭圆的一个特例。


\begin{example}{对两定点 $F_1(-c, 0)$ 和 $F_2(c, 0),(c>0)$,若点$P$满足$|PF_1| + |PF_2| = M,(M > 2c)$,求$P$方程。}\label{ex_Elips3_1}
解:

设椭圆上的任意点为 $P(x, y)$,根据题意有:
\begin{equation}
\sqrt{(x + c)^2 + y^2} + \sqrt{(x - c)^2 + y^2} = M~.
\end{equation}
移项后,两边平方有:
\begin{equation}
(x + c)^2 + y^2 = M^2 - 2M\sqrt{(x - c)^2 + y^2} + (x - c)^2 + y^2~.
\end{equation}
打开整理有:
\begin{equation}
2M\sqrt{(x - c)^2 + y^2}= M^2 - 4cx~.
\end{equation}
两边平方,打开有:
\begin{equation}
4M^2(x^2 - 2cx+c^2) + 4M^2y^2= M^4-4M^2\cdot2cx+16c^2x^2~.
\end{equation}
整理后得到:
\begin{equation}
4(M^2 -4c^2)x^2 + 4M^2y^2= M^2(M^2-4c^2)~.
\end{equation}
两侧同时除以$(M^2-4c^2)M^2$后得到:
\begin{equation}
\frac{x^2}{\left(\displaystyle\frac{M}{2}\right)^2} + \frac{y^2}{\displaystyle\left(\frac{M}{2}\right)^2-c^2}=1~.
\end{equation}
\end{example}

由于$M>2c>0$,也就是$\displaystyle\left(\frac{M}{2}\right)^2-c^2>0$,可以看出,\autoref{ex_Elips3_1} 的结果的形式与\autoref{eq_Elips3_2} 相同,也就是说二者的结果对应着同一个曲线。同理易知,如果两个定点分别为 $F_1(0,-c)$ 和 $F_2(0,c),(c>0)$,则相当于更换$x,y$的结果位置,得到的表达式是:
\begin{equation}
\frac{x^2}{\displaystyle\left(\frac{M}{2}\right)^2-c^2}+\frac{y^2}{\left(\displaystyle\frac{M}{2}\right)^2} =1~.
\end{equation}

令对应的项相等时,总有$\displaystyle\frac{M}{2}$对应着a,b中更大的那个值。

椭圆是到两个定点(焦点)距离之和为定值的点的集合。这个定值等于椭圆的长轴长(记作 $2a$),即对任意点 $P$,有
\begin{equation}\label{eq_Elips3_9}
PF_1 + PF_2 = 2a ~.
\end{equation}

\begin{definition}{椭圆的标准方程}
\begin{equation}\label{eq_Elips3_3}
\frac{x^2}{a^2} + \frac{y^2}{b^2} = 1~.
\end{equation}
\end{definition}

从椭圆的极坐标公式难以看出椭圆的对称性, 另一种定义椭圆的方法是直接在直角坐标系中给出椭圆的方程

\subsubsection{参数介绍}

\textbf{长轴(major axis)}长度为 $2a$,其中$a$称为\textbf{半长轴(semi-major axes)}

and 
\textbf{短轴(minor axis)}长度为 $2b$,其中$b$称为\textbf{半短轴(semi-minor axes)}
焦点之间的距离为 $2c$,其中半焦距 c 也叫做椭圆的线性离心率。
 c^2 = a^2 - b^2~. 
所以 $c < a$,表示焦点在中心两侧。

	•	若 $a > b$,焦点在 $x$ 轴上;
	•	若 $b > a$,焦点在 $y$ 轴上;
	•	若 $a = b$,就是圆。


长轴

短轴
焦距

\subsection{椭圆的参数方程}
表示为参数方程
\begin{equation}\label{eq_Elips3_1}
\leftgroup{
&x(t) = a\cos t\\
&y(t) = b\sin t
},\quad t \in [0, 2\pi) ~.
\end{equation}
\subsection{椭圆的性质}

椭圆上点与焦点连线的斜率之积为定值。

关于 $x$ 轴、$y$ 轴和原点对称。是中心对称图形,中心即椭圆中心。
任何穿过两个焦点的直线段,其在椭圆上的两个端点之间的距离等于 $2a$,且椭圆的长轴为所有通过的弦中最长的。
焦点反射性质(反射定理)
一条从一个焦点 $F_1$ 发出的光线射到椭圆上的某点 $P$,会被反射到另一个焦点 $F_2$,即:
\begin{equation}
\angle F_1PT = \angle F_2PT~.
\end{equation}

($T$ 为切线的切点)
一条从一个焦点出发射到椭圆上的射线,在椭圆上反射后,会朝向另一个焦点。这个性质被用于椭圆形镜面(如椭圆房顶)。椭圆任一点的切线,使得该点到两个焦点的连线之间的夹角相等(即入射角=反射角)。路径 $F_1PF_2$ 是 $F_1 \to F_2$ 所有路径中最短的一条绕射路径。

椭圆面积为:
\begin{equation}
S = \pi a b~.
\end{equation}

可看作是“拉扁”的圆。

焦半径夹角相等
在椭圆上任意一点 $P$,连到两个焦点 $F_1$, $F_2$ 的线段与该点的切线夹角相等。

	10.	切线交焦线定长性质
过椭圆上一点作切线,交焦点连线于两侧,其两个交点到该点的距离之差恒定。







