% 傅里叶变换与矢量空间
% 傅里叶变换|矢量空间|基底|表象|酉变换|正交归一

\pentry{傅里叶变换\upref{FTExp}}
% 未完成: 需要狄拉克 delta 符号, 去哪里找? 最好在讲几何矢量的时候就讲一下
% 未完成: 预备知识中是否有傅里叶级数的矢量空间理解?

我们知道傅里叶级数\upref{FSExp} 可以从矢量空间的角度理解, 而傅里叶变换可以看作傅里叶级数区间取无穷大时的极限, 所以我们以下使用一种 “幼稚” 的观点, 从矢量空间的角度来理解傅里叶变换.

类比傅里叶级数, 我们仍然可以将傅里叶变换看作是矢量空间中两组\textbf{正交归一基底}之间的变换, 我们分别把他们叫做 $x$ 基底和 $k$ 基底. 若使用 $x$ 基底, 就说在 $x$ \textbf{表象}下, 若用 $k$ 基底就说在 $k$ 表象下. 每个实数 $x_0$ 对应一个基底 $\ket {x_0}$, 所有的 $\ket{x_0}$ 构成 $x$ 基底. 每个实数 $k_0$ 对应一个基底 $\ket {k_0}$, 所有的 $\ket {k_0}$ 构成 $k$ 基底. 就像有限维空间中用 $(0, \dots , 1, \dots , 0)$ 表示一组基底的第 $i$ 个关于这组基底的坐标, 可以用 $\delta (x - x_0)$ 表示 $\ket {x_0}$ 关于所有 $x$ 基底的坐标, 用 $\delta (k - k_0)$ 表示 $\ket {k_0}$ 关于所有 $k$ 基底的坐标.

现在, 函数 $f(x)$ 可以看作是某个矢量 $\ket{v}$ 关于 $x$ 基底的坐标(有限维空间中的求和在无穷维空间中变为积分), 而 $g(k)$ 可以看作是 $k$ 基底的坐标\footnote{从该式可以看出\textbf{表象}这个词的由来, 同一个矢量, 使用不同基底后, 其坐标(即函数) $f(x)$ 和 $g(x)$ 表面上看起来不同.}.
\begin{equation}
\ket{v} = \int f(x_0) \ket {x_0} \dd{x_0} = \int g(k_0) \ket {k_0} \dd{k_0}
\end{equation}

矢量空间中两个矢量的内积, 在 $x$ 表象下为
\begin{equation}
\begin{aligned}
&\braket{v_1}{v_2} = \qty(\int f_1(x_1) \ket{x_1} \dd{x_1})\Her  \int f_2(x_2) \ket{x_2} \dd{x_2}\\
&=\iint f_1(x_1)^* f_2(x_2) \delta (x_2 - x_1) \dd{x_1}\dd{x_2}\\
&=\int f_1(x)^* f_2(x) \dd{x}
\end{aligned}
\end{equation}
$k$ 表象同理.

将某矢量 $\ket{v}$ 投影到基底 $\ket{x_0}$ 上, 可以验证其系数为
\begin{equation}
\braket{x_0}{v} = \int f(x) \delta (x - x_0) \dd{x} = f(x_0)
\end{equation}
$x$ 表象下, 可以验证 $x$ 基底的正交归一化% 连接未完成, 另开词条说明该式
\begin{equation}
\braket{x_1}{x_2} = \int \delta (x - x_1) \delta (x - x_2) \dd{x} = \delta (x_2 - x_1)
\end{equation}
$k$ 表象同理.

\subsection{基底变换}

将傅里叶变换对应的算符记为 $\mathcal F$, 反变换记为 $\mathcal F^{-1}$. 傅里叶变换可以看成一个无穷维且连续的\textbf{酉矩阵}, 将同一个矢量从 $x$ 表象坐标变换到 $k$ 表象坐标. 反变换从 $k$ 表象变回 $x$ 表象. 傅里叶变换和反变换并不改变矢量 $\ket{v}$ 本身, $\mathcal F$ 是一个单位算符, 即 $\mathcal F \ket{v} = \ket{v}$.

令 $\ket{k_0}$ 基底在 $x$ 表象下的函数(即坐标)为 $\E^{\I k_0 x}/\sqrt{2\pi }$, 即
\begin{equation}
\ket{k_0} = \int \frac{\E^{\I k_0 x_0}}{\sqrt{2\pi }} \ket{x_0} \dd{x_0}
\end{equation}
则基底变换矩阵的 “矩阵元” 为
\begin{equation}
\begin{aligned}
\braket{x_0}{k_0} &= \bra{x_0}\int \frac{\E^{\I k_0 x_1}}{\sqrt{2\pi }} \ket{x_1} \dd{x_1}\\
&= \int \frac{\E^{\I k_0 x_1}}{\sqrt{2\pi }} \delta (x_1 - x_0) \dd{x_1}
= \frac{\E^{\I k_0 x_0}}{\sqrt{2\pi }}
\end{aligned}
\end{equation}
我们将矩阵元记为 $F^*(x_0, k_0)$, 也叫做\textbf{核(kernel)}. 两个变量就相当于矩阵的两个角标.

该矩阵乘以 $k$ 表象的坐标, 就是 $x$ 表象的坐标, 矩阵的求和同样需要写成积分的形式
\begin{equation}
f(x_0) = \int F^*(x_0, k_0) g(k_0) \dd{k_0} = \int \frac{\E^{\I k_0 x_0}}{\sqrt{2\pi }} g(k_0) \dd{k_0}
\end{equation}
这就是反傅里叶变换.

由于酉矩阵的厄米共轭就是它的逆矩阵, 所以逆矩阵的矩阵元为
\begin{equation}
\braket{k_0}{x_0} = \braket{x_0}{k_0}^* = \frac{\E^{-\I k_0 x_0}}{\sqrt{2\pi }}
\end{equation}
核为 $F(k_0, x_0)$. 该矩阵乘以 $x$ 表象的坐标, 就是 $k$ 表象的坐标
\begin{equation}
g(k_0) = \int F(k_0, x_0) f(x_0) \dd{x_0} = \int \frac{\E^{-\I k_0 x_0}}{\sqrt{2\pi }} f(x_0) \dd{x_0}
\end{equation}
这就是傅里叶变换.

$x$ 表象下, $k$ 基底的正交归一化可以记为
\begin{equation}\label{FTvec_eq1}
\braket{k_1}{k_2} = \int \frac{\E^{-\I k_1 x}}{\sqrt{2\pi }} \frac{\E^{-\I k_2 x}}{\sqrt{2\pi }} \dd{x} = \delta (k_2 - k_1)
\end{equation}
$x$ 基底在 $k$ 表象下的正交归一也同理. 之所以说这时一种 “幼稚” 的做法, 是因为\autoref{FTvec_eq1} 的积分(从 $-\infty $ 到 $\infty$)并不收敛, $\delta $ 函数也不是一个严格意义上的函数.


我们就可以用以下过程在 $x$ 表象下简洁地 “证明” $f(x)$ 依次经过傅里叶变换和反变换后, 仍然可以得到 $f(x)$. 证明的矢量形式为
\begin{equation}
\ket{v} = \int \ket{k}\braket{k}{v} \dd{k}
\end{equation}
将各个矢量替换为 $x$ 表象的坐标, 得

\begin{equation}
\begin{aligned}
\mathcal F^{-1} [\mathcal F f(x)] &= \int \frac{\E^{\I k x}}{\sqrt{2\pi }} \qty(\int \frac{\E^{-\I k x'}}{\sqrt{2\pi }} f(x') \dd{x'}) \dd{k}\\
&= \int f(x') \qty(\int \frac{\E^{-\I k x'}}{\sqrt{2\pi }}\frac{\E^{\I k x}}{\sqrt{2\pi }}\dd{k}) \dd{x'}\\
&= \int f(x') \delta (x - x') \dd{x'}\\
&= f(x)
\end{aligned}
\end{equation}
