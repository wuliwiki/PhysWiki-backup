% 例: 有限维方阵

\pentry{
矩阵的本征方程\upref{MatEig}
有界算子的谱\upref{BddSpe}
有界算子的预解式\upref{BddRsv}
谱投影\upref{SpePrj}
}

如果将之前提到的泛函分析概念应用于有限维方阵, 则更容易看出它们的意义, 加深直观理解. 

这里一直设$A$是$n\times n$复矩阵. 我们把它视为$\mathbb{C}^n$到自己的一个线性算子. 如词条 有界算子的谱\upref{BddSpe} 所说, 谱集$\sigma(A)$恰为$A$的特征值的集合, 而谱半径$r(A)$当然就是特征值的最大模. 

现在设$\sigma(A)=\{\lambda_1,...,\lambda_m\}$(重数大于1的特征值算作一个谱点). 我们可以把预解式$R(z;A)=(z-A)^{-1}$视为矩阵值亚纯函数, 而显然$\sigma(A)$就是它的极点集. 我们更可以借助矩阵的若尔当分解而写出$R(z;A)$的明显表达式. 实际上, $(z-A)^{-1}$的谱集显然是$\{z-\lambda_1,...,z-\lambda_m\}$, 