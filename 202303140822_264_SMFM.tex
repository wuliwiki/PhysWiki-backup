% 索末菲模型
% 固体物理|自由电子气模型

\begin{issues}
\issueDraft
\end{issues}

虽然德鲁特模型在电子的声、光和热导率方面解释得比较好,但其在解释电子比热时却遇到了困难。

根据经典的能均分定理,金属电子气中每个电子的平均内能为$3k_BT/2$,对热容的贡献为$3k_B/2$,然而实验中的电子热容几乎测不到,只有德鲁特模型给出的1\%左右,而且与温度密切相关。
\begin{figure}[ht]
\centering
\includegraphics[width=6cm]{./figures/SMFM_1.png}
\caption{常见金属热容随温度的变化} \label{SMFM_fig1}
\end{figure}
解释这个现象需要用到索菲亚模型。
\subsection{基本假设}
\begin{enumerate}
\item \textbf{独立电子近似(Independent electron approximation)}:电子之间不会相遇,不存在任何相互作用。
\item \textbf{自由电子近似}:相对电子而言,晶体中的离子的运动忽略不计,同时也忽略电子与离子的库伦力作用。
\item \textbf{不碰撞假设}:电子以物质波的形式存在,并且不会与离子碰撞。
\item \textbf{泡利不相容原理}:电子是费米子,满足泡利不相容原理。
\end{enumerate}
\subsection{k动量空间}
根据上面的假设,我们可以写出电子满足的薛定谔方程:
\begin{equation}
-\frac{\hbar^2}{2m}\nabla^2\psi=E\psi
\end{equation}
方程的解是平面波$\psi=Ae^{i\vec{k}\cdot \vec{r}}$。

为了使平面波函数归一,我们将电子限定在一个$L^3$中的正方形盒子中,考虑到固体中含有大量的电子,含有大量重复的这种小盒子,所以我们认为盒子中的波函数满足周期性边界条件,即:
\begin{equation}
\psi(\vec{r})=\psi(\vec{r}+Le_x)=\psi(\vec{r}+Le_y)=\psi(\vec{r}+Le_z)
\end{equation}
所以最终的波函数为$\psi=L^{-3/2}e^{i\vec{k}\cdot \vec{r}}$,其中$\vec{k}$满足:
\begin{equation}
(k_x, k_y, k_z)=\frac{2\pi}{L}(n_x, n_y, n_z)
\end{equation}
能量$E$满足:
\begin{equation}
E=\frac{\hbar ^2 k^2}{2m}
\end{equation}
可以这么理解,每一个电子的波函数中的$\vec{k}$是分立的,对应下列图片中的一个小格子。当格子足够密集时,所有能量相等的电子则可以近似对应一个$\vec{k}$空间中的球面。
\begin{figure}[ht]
\centering
\includegraphics[width=6cm]{./figures/SMFM_2.png}
\caption{k动量空间} \label{SMFM_fig2}
\end{figure}
\subsection{费米分布和费米能}
由于电子是费米子,自由电子不能处在同一个状态内,即上述的k动量空间中一个格子内不能有超过2个电子("2"是来自于电子自旋),热力学统计给出一定温度下电子的分布有:
\begin{equation}
f(E)=(\exp(\frac{E-\mu}{k_BT})+1)^{-1}    \in[0,1]
\end{equation}
表征某个能量值下电子的存在几率。由于能量$E$是$\vec{k}$的函数,那么其自然也表明是否有电子处在某个特定的$\vec{k}$上。那么总的电子数就是$N=\textstyle \sum_{\vec{k}}2f(\vec{k})$。总的能量就是$E_{sum}=\textstyle \sum_{\vec{k}}2E(\vec{k})f(\vec{k})$。

其中$\mu$是化学势,其随温度缓慢变化。$\mu (T=0K)$称为费米能$E_F$.
T=0K时的情况非常特殊。我们先来考虑T=0K时的情况来理解函数

