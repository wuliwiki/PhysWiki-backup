% 卡尔·萨根
% license CCBYSA3
% type Wiki

(本文根据 CC-BY-SA 协议转载自原搜狗科学百科对英文维基百科的翻译)

\textbf{卡尔·爱德华·萨根} (1934年11月9日 – 1996年12月20日)是美国天文学家, 宇宙学家, 天体物理学家, 天体生物学家,作家,科普工作者,以及天文学和其他自然科学的科学传播者。他最出名的是他作为科学普及者和传播者的工作。他最著名的科学贡献是在外星生命领域的研究,包括利用辐射从基本化学品中实验生产氨基酸。萨根组装了第一批发送到太空的物理信息 :先驱者号镀金铝板 还有旅行者号黄金唱片,包含有任何找到它们的外星智慧都可能理解的通用信息。萨根认为,现在普遍接受的假设是,金星表面的高温可以归因于温室效应,并且可以利用温室效应来进行计算。

萨根发表了600多篇科学论文和文章,撰写、编辑出版了20多本著作。 他撰写了很多大众科学书籍,例如伊甸园的飞龙、布罗卡的脑 和暗淡蓝点 ,并为1980年获奖的《宇宙:个人之旅》系列担任旁白和编剧。这是美国历史上最受广泛关注的大众电视系列, 宇宙 已经被60个不同国家的至少5亿人观看。 为了配合这个系列出版了同名书籍 宇宙 。他还写了科幻小说 接触,1997年在此基础上拍摄了 同名电影。他的论文中包含涉及595,000份资料,[1] 存档于 国会图书馆。[2]

萨根倡导科学怀疑调查和科学方法,开创了外太空生物学的先河,并推动了对外星智能的探索(SETI)。他的大部分职业生涯是在康奈尔大学(Cornell University)担任天文学教授,并在那里领导行星研究实验室(Laboratory for Planetary Studies)。萨根和他的作品获得了无数奖项和荣誉,包括 美国宇航局杰出公共服务奖章, 国家科学院 公益奖章,《伊甸园之龙 》获得了 普利策普通非小说奖 ,以及 《宇宙:个人航行》,获得了两次 艾美奖, 皮博迪奖以及 雨果奖。他结了三次婚,有五个孩子。萨根患有脊髓发育不良 ,1996年12月20日因 肺炎逝世,享年62岁时。

\subsection{}