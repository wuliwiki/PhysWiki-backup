% Python 路径笔记
% license Usr
% type Note

\begin{issues}
\issueDraft
\end{issues}

\begin{itemize}
\item \verb`os.getcwd()` 获取当前路径, 也就是运行 python 时命令行的 \verb`pwd`
\item \verb`os.chdir('路径')` 改变当前路径 
\item \verb`for root, dirs, files in os.walk('文件夹'): ...` 可以在 \verb`文件夹` 中遍历整个文件树。 默认从顶到底一层层遍历, 如果添加参数 \verb`topdown=False` 则从底到顶。
\item 要知道当前遍历的层数,可以在 \verb`os.walk('文件夹')` 的循环中用 \verb`depth = root[len('文件夹'):].count(os.sep)`
\item \verb`root,dirs,files` 分别是当前处理的路径、 该路径下的所有文件夹和所有文件。 其中 \verb`root` 开始的部分和 \verb`'文件夹'` 相同。 如果后者是相对路径前者也会是。
\item \verb`os.walk` 一开始就决定所有路径不会动态更新。如果从顶到低,重命名或新建文件夹不会被自动 walk, 所以要在循环内部重命名文件夹要从底到顶(例如切割限制文件夹名称长度)。
\item \verb`dir,path = os.path.split('路径')` 可以把路径划分成前面部分和最后的文件名或文件夹名, 这相当于分别使用 \verb`os.path.dirname('路径')` 和 \verb`os.path.basename('路径')`。
\item \verb`os.path.abspath('路径')` 获取绝对路径(可能包含软链)
\item \verb`os.path.realpath('路径')` 获取正则化路径,也就是不含任何\textbf{软链}以及 \verb`.` 和 \verb`..` 的\textbf{绝对}路径。
\item \verb`os.path.relpath(路径2, 路径1)` 可以显示 \verb`路径2` 相对 \verb`路径1` 的相对路径
\item \verb`os.path.join(路径1, 路径2)` 可以把两个路径拼接,后者必须是相对路径。
\item \verb`root, ext = os.path.splitext(字符串)` 可以把 \verb`字符串` 根据最后一个英文句号拆分成 \verb`root` 和 \verb`ext` (含前面的句号)两部分,如果 \textbf{basename} 中没有英文句号,则 \verb`ext` 为空(前面路径名中的句号不算)。
\item \verb`os.path.exists(路径)` 可以判断文件或文件夹是否存在
\item \verb`os.path.isfile(路径)` 和 \verb`os.path.isdir(路径)` 可以判断某个路径是文件还是文件夹,如果不存在,同样返回 \verb`False`。
\end{itemize}
