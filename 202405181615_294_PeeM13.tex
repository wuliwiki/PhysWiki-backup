% 2013 年考研数学试题(数学一)
% keys 考研|数学
% license Copy
% type Tutor

\subsection{选择题}
\begin{enumerate}
\item 已知极限 $\lim_{n\to 0}\frac{x-\arctan x}{x^k}=c$ ,其中 $k,c$ 为常数,且 $c \neq 0$ ,则 ($\quad$)\\
(A) $\displaystyle k=2,c=-\frac{1}{2}$\\
(B) $\displaystyle k=2,c=\frac{1}{2}$\\
(C) $\displaystyle k=3,c=-\frac{1}{3}$\\
(D) $\displaystyle k=3,c=-\frac{1}{3}$
\item  曲面 $x^2+\cos (xy)+xyz+x=0$ 在点 $(0,1,-1)$ 处的切平面方程为 ($\quad$)\\
(A) $x-y+z=-2$\\
(B) $x+y+z=0$\\
(C) $x-2y+z=-3$\\
(D)  $x-y-z=0$
\item  设 $\displaystyle f(x)=\abs{x-\frac{1}{2}},b_n=2\int_{0}^{1}f(x)\sin n\pi x\dd{x} (n=1,2,\dots)$ ,令 $\displaystyle S(x)=\sum_{n=1}^\infty b_n \sin n\pi x$ ,则 $S(-\frac{9}{4})$ =($\quad$)\\
(A) $\displaystyle \frac{3}{4}$\\
(B) $\displaystyle \frac{1}{4}$\\
(C) $\displaystyle -\frac{1}{4}$\\
(D)  $\displaystyle -\frac{3}{4}$
\item  设 $L_1:x^2+y^2=1,L_2:x^2+y^2=2,  L_3:x^2+zy^2=2, L_4:2x^2+y^2=2$  为四条逆时针方向的平面曲线,记 $\displaystyle I_i=\oint (y+\frac{y_3}{6})\dd{x}+(2x-\frac{x^3}{3})\dd{y}\quad (i=1,2,3,4)$  ,则 $max\{I_1,I_2,I_3,I_4\}$ =($\quad$)\\
(A) $I_1$\\
(B)  $I_2$\\
(C)  $I_3$\\
(D)  $I_4$

\item 设 $\mat {A,B,C}$ 均为 $n$ 阶矩阵,若 $\mat {AB=C}$ ,且 $\mat B$ 可逆,则($\quad$)\\
(A) 矩阵 $C$ 的行向量组与矩阵 $A$ 的行向量组等价\\
(B) 矩阵 $C$ 的列向量组与矩阵 $A$ 的列向量组等价\\
(C) 矩阵 $C$ 的行向量组与矩阵 $B$ 的行向量组等价\\
(D) 矩阵 $C$ 的列向量组与矩阵 $B$ 的列向量组等价

\item  矩阵 $\pmat{1&a&1\\a&b&a\\1&a&1}$ 与 $\pmat{2&0&0\\0&b&0\\0&0&0}$ 相似的充分必要条件为 ($\quad$)\\
(A) $a=0,b=2$\\
(B) $a=0,b$为任意常数\\
(C) $a=2,b=0$\\
(D) $a=2,b$为任意常数
\item  设 $X_1,X_2,X_3$ 是随机变量,且 $X_1\~N(0,2^2)$ ,则
(A) 
(B)
(C)
(D)
\end{enumerate}