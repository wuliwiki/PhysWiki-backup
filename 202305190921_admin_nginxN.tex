% Nginx 笔记

\begin{issues}
\issueDraft
\end{issues}

\begin{itemize}
\item 一个\href{https://zhuanlan.zhihu.com/p/80600540}{知乎教程}
\item \href{https://nginx.org/en/docs/}{官方文档}
\item \verb|sudo apt install nginx|
\item 要重启 nginx 服务用 \verb|sudo systemctl restart nginx|
\item 要查看是否连接成功, 用 \verb|curl localhost| 或者 \verb|curl http://localhost|(默认访问 80 接口)。 如果打印出一个 html 文本, 包含 \verb|Welcome to nginx!|, 就是成功了(当然也可以用浏览器访问, 只是有时候只有命令行)。
\item 如果要限制 nginx 只监听某个网卡, 编辑配置文件 \verb|sudo vim /etc/nginx/sites-enabled/default|, 然后在 \verb|listen 80 default_server;| 的 \verb|80| 改成 \verb|网卡ip:80|, 然后重启 \verb|nginx| 服务即可生效。
\item 事实上不光是本机, 监听的网卡所在的所有机器访问该网卡的 ip 的 80 端口都会收到
\item 要重置所有 \verb|/etc/nginx| 配置文件(为了确保可以先把这个文件夹删掉), 用 \verb|apt purge nginx nginx-common nginx-full|, 然后 \verb|apt install nginx|。
\end{itemize}

\subsection{http 静态网站}
\begin{itemize}
\item 配置文件: \verb|/etc/nginx/nginx.conf|。 在 \verb|http| section 里面加入
\begin{lstlisting}[language=none]
server {
    listen 80;
    server_name 公网ip或域名;
    
    location / {
        root /静态网页根目录;
    }
}
\end{lstlisting}
\item 要特别注意 \verb|静态网页根目录| 以及它的所有上层目录需要可以被 nginx 的用户 \verb|www-data| 读取和执行, 里面的文件也一样。如果权限不对访问网页会出现错误 \verb|403 forbidden|。
\item 在非 ubuntu 系统中 nginx 可能会有别的用户名。  要查看具体的用户名, 用 \verb`ps aux | grep nginx`, 看第一列中除了 \verb|root| 都有哪些用户。
\item 要访问静态, 在本机或者其他机器的命令行用 \verb|curl 公网ip或域名|, 如果 \verb|ip| 不是公网 ip 就只能在局域网的机器上访问。 如果机器上有 GUI 浏览器, 也可以直接在网址栏输入 \verb|公网ip或域名|。
\item 如果只是从本机访问, 那么 \verb|公网ip或域名| 中可以使用任何域名不需要注册。
\end{itemize}

\subsection{使用 https}
\begin{itemize}
\item 首先要申请一个 \textbf{SSL/TLS 证书}: 比较著名的证书颁发机构如 \href{https://letsencrypt.org/}{Let's Encrypt}。 我们以它为例。
\item 一般最好有一个域名(子域名也可以), 因为 SSL/TLS 证书是颁发给域名而不是 ip 的。 确保 \verb|80| 端口可以访问: \verb|http://域名|。
\item 安装 \href{https://certbot.eff.org/instructions?ws=nginx&os=ubuntufocal}{Certbot}, 用于自动获取以及更新证书:
\item 首先更新 snapd: \verb|sudo snap install core; sudo snap refresh core|
\item 卸载老版本: \verb|sudo apt remove certbot|
\item 安装 cerbot: \verb|sudo snap install --classic certbot|
\item \verb|sudo ln -s /snap/bin/certbot /usr/bin/certbot|
\item 获取证书,指定 nginx: \verb|sudo certbot certonly --nginx|, 这时会互动提示输入域名等信息。
\item 测试自动更新: \verb|sudo certbot renew --dry-run|
\item 生成的证书存在 \verb|/etc/letsencrypt/live/域名| 中: \verb|cert.pem  chain.pem  fullchain.pem  privkey.pem|
\item 更改 nginx 设置, 在 \verb|http { ... }| 的最后添加一个:
\begin{lstlisting}[language=none]
server {
    listen 443 ssl;
    server_name 域名;
    ssl_certificate /etc/letsencrypt/live/域名/fullchain.pem;
    ssl_certificate_key /etc/letsencrypt/live/域名/privkey.pem;

    location / {
        root /var/www/html;
        index index.html;
    }
}
\end{lstlisting}
\item 重启 nginx: \verb|systemctl restart nginx.service|
\item 现在就可以访问域名: \verb|https://域名|
\end{itemize}

\subsubsection{创建多个网站}
\begin{itemize}
\item 即使 ip 地址和端口号(80, 443) 相同, nginx 也可以同时创建多个网站, 通过 http 中的域名来判断用户需要的是哪个网站。
\item 可以继续在 \verb|/etc/nginx/nginx.conf| 中插入新的 section (包括下面的反向代理 section), 同样先添加 80 端口, 输入不同的域名和目录, 重启 nginx 即可。
\end{itemize}


\subsection{反向代理其他网站}
比如你在国内访问 github 很慢, 但在美国有一个私人服务器, 那么你可以用 Nginx 作为反向代理, 这样你就可以访问你代理服务器的域名了。

同样只需要在设置中添加一个 server section 即可

如果只需要 http 代理, 用
\begin{lstlisting}[language=none]
server {
   listen 80;
   server_name 公网ip或域名;

   location / {
       proxy_pass https://github.com/;
       proxy_set_header Host github.com;
       proxy_set_header X-Real-IP $remote_addr;
       proxy_set_header X-Forwarded-For $proxy_add_x_forwarded_for;
   }
}
\end{lstlisting}

如果需要 https 代理, 用
\begin{lstlisting}[language=none]
server {
  listen 443 ssl;
  server_name 域名;

  ssl_certificate /etc/letsencrypt/live/域名/fullchain.pem;
  ssl_certificate_key /etc/letsencrypt/live/域名/privkey.pem;

  location / {
    proxy_pass https://github.com/;
    proxy_set_header Host github.com;
    proxy_set_header X-Real-IP $remote_addr;
    proxy_set_header X-Forwarded-For $proxy_add_x_forwarded_for;
  }
}
\end{lstlisting}

尝试了对 wikipedia 也进行反向代理, 但根据 wikipedia 的设置浏览器 ulr 总是会跳转回 wikipedia.org, GPT-4 建议在 location 中添加以下设置, 但仍然无效。 这可能超出了 nginx 的能力范围。
\begin{lstlisting}[language=none]
proxy_set_header Referer "";
proxy_redirect https://wikipedia.org/ $scheme://$host/;

# Enable URL rewriting
sub_filter_once off;
sub_filter_types *;
sub_filter 'https://wikipedia.org' '$scheme://$host';
sub_filter 'https://www.wikipedia.org' '$scheme://$host';
\end{lstlisting}
