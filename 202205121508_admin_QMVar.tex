% 量子力学中的变分法
% 变分法|量子力学|波函数|鞍点

当平均能量是波函数的鞍点时, 波函数就是能量的本征态. 对一维单粒子
\begin{equation}
E = \ev{H}{\psi}
\end{equation}
但注意这里的波函数必须已经归一化. 由于变分法需要假设任意的增量函数 $\delta \psi $,  我们只好用一个不要求归一化的能量平均值公式
\begin{equation}\label{QMVar_eq1}
E = \frac{\ev{H}{\psi}}{\braket{\psi}}
\end{equation}
该式可以给出基态波函数的能量上限.

现在假设波函数增加 $\delta \psi$ 
\begin{equation}
E \braket{\delta\psi}{\psi} + E\braket{\psi}{\delta\psi}
= \mel{\delta \psi}{H}{\psi} + \mel{\psi}{H}{\delta\psi}
\end{equation}
由于 $\delta\psi$ 是任意的, 我们也可以使用 $\I\delta\psi$ 
\begin{equation}
-E \braket{\delta\psi}{\psi} + E \braket{\psi}{\delta\psi}
= -\mel{\delta\psi}{H}{\psi} + \mel{\psi}{H}{\delta\psi}
\end{equation}
以上两式等效, 两式相减, 得(相当于 $\psi^*$ 与 $\psi$ 是两个独立的变量函数)
\begin{equation}
E\braket{\delta\psi}{\psi} = \mel{\delta\psi}{H}{\psi}
\end{equation}
该式对任意微小函数增量 $\delta\psi $ 都要求成立. 现在如果令 $\delta \psi  = \delta (x)$,  我们得到薛定谔方程
\begin{equation}
H \ket{\psi} = E\ket{\psi}
\end{equation}
归一化条件下的变分法也可以由拉格朗日乘数法完成, 令
\begin{equation}
L = \ev{H}{\psi} - \lambda [\braket{\psi} - 1]
\end{equation}
类似以上过程, 同样有
\begin{equation}
\mel{\delta\psi}{H}{\psi} - \lambda\braket{\delta\psi}{\psi} = 0
\end{equation}
即
\begin{equation}
H \ket{\psi} = \lambda \ket{\psi}
\end{equation}
显然, 乘数 $\lambda $ 就是本征态能量.

\subsection{Rayleigh-Ritz 变分法}
用一些变分参数拟合波函数 $\ket{\psi}$, 然后找到这些参数使\autoref{QMVar_eq1} 最小化的值.
