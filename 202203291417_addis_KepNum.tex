% 开普勒问题的数值计算(Matlab)

\begin{issues}
\issueOther{搞一个卢瑟福散射的模拟, 支持不同的入射距离, 初速度相同}
\end{issues}

\pentry{Matlab 画图\upref{MatPlt}, 开普勒问题\upref{CelBd}}

本文讨论开普勒问题中, 若已知初始位置和初速度矢量, 求轨道方程和动画的算法并给出 Matlab 代码. 以下所有符号和定义沿用 “开普勒问题\upref{CelBd}”.
\begin{figure}[ht]
\centering
\includegraphics[width=11cm]{./figures/KepNum_1.pdf}
\caption{在同一位置以同一初速度不同仰角发射的质点的椭圆轨道, 具有相同的半长轴和不同的离心率, 动画见\href{https://wuli.wiki/apps/Kepler.html}{这里}. 虚线为参考圆, 例如地球表面, 可见除了水平发射的质点都会落回地面. 若初速度很小, 那么参考圆外的运动就趋近于抛体运动(足够扁的椭圆的一端趋近于抛物线).} \label{KepNum_fig1}
\end{figure}

\begin{figure}[ht]
\centering
\includegraphics[width=9cm]{./figures/KepNum_2.pdf}
\caption{在同一位置以同 $45^\circ$ 仰角和不同初速度发射的质点的轨道, 动画见\href{https://wuli.wiki/apps/Kepler2.html}{这里}. 速度最大的轨道为双曲线, 其他轨道为椭圆.} \label{KepNum_fig2}
\end{figure}

\subsection{轨道的计算}
若 $F(r)$ 是平方反比的力(斥力为正引力为负), 即
\begin{equation}
F(r) = \frac{k}{r^2}
\end{equation}
其中 $k$ 为非零实常数. 由于质点的运动只和初始位置、初始速度和 $k$ 有关, 与质点的质量无关, 所以我们令 $m=1$ 并不影响一般性. 令发射位置的极坐标\upref{Polar}为 $(R,\alpha)$, 初速度为 $v_0$, 发射仰角为 $\beta$ (顺时针增量为正, $\beta=0$ 时质点延逆时针方向水平发射).

所以质点机械能为动能加势能 $E = v_0^2/2 + k/R$, 角动量(逆时针为正)为 $L = v_0 R\cos\beta$. 于是离心率 $e$ 可由\autoref{CelBd_eq2}~\upref{CelBd} 计算, 半通径 $p$ 由\autoref{CelBd_eq3}~\upref{CelBd}计算. 然后代入圆锥曲线的极坐标方程(\autoref{Cone_eq5}~\upref{Cone}) 即可得到轨道方程. 然而由于发射点已经选好, 实际的轨道还需要旋转一个角度 $\theta_0$ (逆时针为正), 即轨道方程变为
\begin{equation}\label{KepNum_eq1}
r(\theta)  = \frac{-s p}{1 - e\cos (\theta-\theta_0)} \qquad (r > 0)
\end{equation}
其中当 $k>0$ 时 $s=1$, $k<0$ 时 $s=-1$. 要求 $\theta_0$, 可以代入初始条件, 即 $r(\alpha) = R$. 得
\begin{equation}
\theta_0 = \alpha \pm \arccos(\frac{1+ sp/R}{e}) \qquad (e\ne 0)
\end{equation}
这里有两个解是因为圆锥曲线和半径为 $R$ 的圆必定有两个交点. 我们需要根据发射角度来判断哪个是发射点, 稍作分析易得当 $s\cdot\sin2\beta \geqslant 0$ 时上式取负号, 否则取正号.

\subsection{运动的计算}

\pentry{开普勒问题的运动方程\upref{EqMoKp}}

我们使用 “开普勒问题的运动方程\upref{EqMoKp}” 给出的方法求解三种圆锥曲线的含时运动. 沿用其中的符号定义, 令 $t=0$ 时质点位于近日点, 对应极角为 $\theta_0 + \pi$. 令相对近日点的极角增量为 $\Delta\theta = \theta - \theta_0 - \pi$. 接下来, 先计算函数 $t(\Delta\theta)$, 默认逆时针旋转, 然后数值求解反函数 $\Delta\theta(t)$ 即可. 若要考虑顺时针旋转, 乘以 $-1$ 即可. 易得 $\cos\beta > 0$ 时质点绕中心天体逆时针运动, $\cos\beta < 0$ 时顺时针运动.

\textbf{抛物线}: 时间 $t$ 在 $(-\infty,\infty)$ 取值, \autoref{EqMoKp_eq3}~\upref{EqMoKp}就是 $t(\Delta\theta)$.

\textbf{椭圆}: 时间 $t$ 在 $[-T/2,T/2]$ 取值, 周期 $T$ 由\autoref{Keple2_eq1}~\upref{Keple2} 计算. 由于这是周期运动, 其他任意时间都可以加减整数个周期后落到 $[-T/2,T/2]$ 中. 由\autoref{EqMoKp_eq1}~\upref{EqMoKp} 和\autoref{EqMoKp_eq5}~\upref{EqMoKp} 计算 $t(\Delta\theta)$.

\textbf{双曲线}: 时间 $t$ 在 $(-\infty,\infty)$ 取值, 使用\autoref{EqMoKp_eq2}~\upref{EqMoKp} 到\autoref{EqMoKp_eq6}~\upref{EqMoKp} 可分别得 $k<0$ 和 $k>0$ 时的 $t(\Delta\theta)$.

发射质点的初始时刻为 $t_0 = t(\alpha - \theta_0 - \pi)$. 同样求出 $t_0$, 生成时间格点 $t_i$ ($i=1,2,\dots$), 解方程得到对应的 $\theta_i$ 即可.

以下 Matlab 程序支持对同一发射位置画出多组不同初速度和仰角的轨道. 相比于使用解微分方程的方法(\autoref{OdeRK4_sub1}~\upref{OdeRK4})计算轨道和运动, 该程序计算的轨道和运动都可以达到 13 位有效数字以上. 然而缺点是只能计算理想的开普勒问题, 无法拓展到非平方反比力, 也无法拓展到三体乃至多体运动.

\begin{lstlisting}[language=matlab,caption=kepler.m]
% 已知初始位置、发射速度、发射方向, 求轨道以及运动方程
function kepler
% === 参数设置 ===
k = -1; % -GMm, (m=1)
R = 1; alpha = 0; % 初始位置(极坐标)
v0 = 1.1; % 初速度(支持行矢量或标量)
beta = linspace(0, 4*pi/11, 6); % 发射仰角(支持行矢量或标量)
t_max = 8.94827; Nt = 100; % 模拟时间和步数
axis_param = 'auto'; % 坐标范围 [xmin,xmax,ymin,ymax] 或 'auto'
% ==============

% 匹配 v0 和 beta 的尺寸
if isscalar(v0)
    v0 = v0*ones(size(beta));
elseif isscalar(beta)
    beta = beta*ones(size(v0));
elseif ~all(size(v0)==size(beta))
    error('v0 和 beta 的尺寸必须一样,或者其中之一为标量');
end

if any(cos(beta)==0)
    error('不支持纯径向运动!');
end

% 画参考圆
close all; figure;
tmp = linspace(0, 2*pi, 500);
plot(R*cos(tmp), R*sin(tmp), 'k--'); hold on;
scatter(0, 0); axis equal;
xlabel x; ylabel y;
N = numel(v0); t_cell = cell(1,N); th_cell = t_cell;
p_list = zeros(1,N); e_list = p_list; th0_list = p_list;
for i = 1:N
    [p_list(i), e_list(i), th0_list(i), t_cell{i}, th_cell{i}] = ...
        kepler1(k, R, alpha, v0(i), beta(i), t_max, Nt); hold on;
    axis(axis_param);
end

% 生成动画
h = nan(1,N); % point handles
for it = 1:10000000
    exit = true;
    for i = 1:N
        if it > size(th_cell{i})
            h(i) = nan; continue;
        end
        exit = false;
        th = th_cell{i}(it);
        r = -sign(k) * p_list(i) / (1 - e_list(i)*cos(th - th0_list(i)));
        if (~isnan(h(i))), delete(h(i)); end
        h(i) = scatter(r*cos(th), r*sin(th), 'r^');
    end
    if exit
        break;
    end
    title(['t = ' num2str(t_cell{1}(it) - t_cell{1}(1), '%.2f')]);
    saveas(gcf, [num2str(it) '.png']);
end
end

% 计算并画出一条轨道
% 轨道方程为 r = -sign(k) * p / (1 - e*cos(th - th0));
function [p, e, th0, t, th_t] = kepler1(k, R, alpha, v0, beta, t_max, Nt)
E = v0^2/2 + k/R; % 轨道总机械能
L = v0*cos(beta)*R; % 角动量
e = sqrt(1 + 2*E*L^2/k^2); % 离心率
p = L^2/abs(k); % 半通径

% 画图
dth = pi/1000; Nth = 500;
if e < 1 % 椭圆
    th = linspace(0, 2*pi, Nth);
elseif e == 1 % 抛物线
    th = linspace(dth, 2*pi-dth, Nth);
else % 双曲线
    gamma = atan(sqrt(e^2-1)); % 渐进张角 atan(b/a)
    if k < 0
        th = linspace(gamma+dth, 2*pi-gamma-dth, Nth);
    else
        th = linspace(-gamma+dth, gamma-dth, Nth);
    end
end

% 计算圆锥曲线需要旋转的角度 th0
if e == 0 % 圆
    th0 = 0;
else
    if sin(2*beta)*sign(k) >= 0
        th0 = alpha - acos((1 + sign(k)*p/R)/e);
    else
        th0 = alpha + acos((1 + sign(k)*p/R)/e);
    end
end
th0 = real(mod(real(th0)+pi, 2*pi) - pi);
scatter(R*cos(alpha), R*sin(alpha));
th = th + th0;

% 画出轨道
r = -sign(k) * p./(1 - e*cos(th-th0));
x = r .* cos(th); y = r .* sin(th);
plot(x, y); axis equal;

% 含时运动
% 时间格点
rot_dir = sign(cos(beta)); % 1: 逆时针 -1: 顺时针
if k < 0
    Dth0 = alpha-th0-pi;
else
    Dth0 = alpha-th0;
end
t0 = rot_dir*t_Dth(p, e, k, Dth0); % 初始时间
t = linspace(t0, t0+t_max, Nt);
if k < 0
    th_t = rot_dir*Dth_t(p, e, k, dth, t) + th0 + pi;
else
    th_t = rot_dir*Dth_t(p, e, k, dth, t) + th0;
end
end

% 求 t(\Delta\theta)
% Dth 支持矢量
% 对椭圆 Dth 支持任意实数,会自动 wrap 到 (-pi, pi]
function t = t_Dth(p, e, k, Dth)
if Dth ~= pi
    Dth = mod(Dth+pi, 2*pi) - pi;
end
if e == 1 && k < 0 % 抛物线
    tmp = tan(Dth/2);
    t = L^3/(2*k^2) * (tmp + tmp.^3/3);
elseif e < 1 && k < 0 % 椭圆
    a = p/(1-e^2);
    r = p./(1 + e*cos(Dth));
    psi = sign(Dth).*real(acos((1 - r/a)/e));
    t = sqrt(a^3/(abs(k)))*(psi - e*sin(psi));
elseif e > 1 && k < 0 % 双曲线 (k<0)
    a = p/(e^2-1);
    r = p./(1 + e*cos(Dth));
    xi = sign(Dth).*real(acosh((1 + r/a)/e));
    t = sqrt(a^3/(abs(k)))*(e*sinh(xi) - xi);
elseif e > 1 && k > 0 % 双曲线 (k>0)
    % 此时 Dth = th
    a = p/(e^2-1);
    r = p./(1 - e*cos(Dth));
    xi = sign(Dth).*acosh((-r/a-1)/e);
    t = sqrt(a^3/(abs(k)))*(e*sinh(xi) + xi);
else
    error('参数错误或未实现!');
end
end

% 已知时间求解 \Delta\theta
% t 支持矢量
% 对于椭圆, 支持 t 为任意实数
function Dth = Dth_t(p, e, k, dth, t)
N = numel(t); Dth = zeros(1,N);
if e < 1 % 椭圆
    a = p/(1-e^2);
    T = 2*pi*a^1.5*sqrt(1/abs(k));
    t = mod(t + T/2, T) - T/2;
    Dth_range = [-pi, pi];
elseif e == 1 % 抛物线
    Dth_range = [-pi+dth, pi-dth];
else % 双曲线
    th1 = atan(sqrt(e^2-1)); % 渐进张角的一半
    if k < 0
        Dth_range = [-pi+th1+dth, pi-th1-dth];
    else
        Dth_range = [-th1+dth, th1-dth];
    end
end
% 二分法解方程
for i = 1:N
    fun = @(Dth) t_Dth(p, e, k, Dth) - t(i);
    Dth(i) = fzero(fun, Dth_range);
end
end
\end{lstlisting}




这里给出另一组参数,计算以 $45^\circ$ 仰角和不同初速度发射的情况,替换到以上代码中即可.
\begin{lstlisting}[language=matlab]
% === 参数设置 ===
k = -1; % -GMm, (m=1)
R = 1; alpha = 0; % 初始位置(极坐标)
v0 = linspace(1, 1.5, 6); % 初速度(支持行矢量或标量)
beta = pi/4; % 发射仰角(支持行矢量或标量)
t_max = 4*pi; Nt = 100; % 模拟时间和步数
axis_param = [-1.1,3.5,-1.1,4]; % 坐标范围 [xmin,xmax,ymin,ymax] 或 'auto'
% ==============
\end{lstlisting}
