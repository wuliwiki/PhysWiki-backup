% 标量场论的单圈修正
% 重整化|单圈修正
\pentry{LSZ约化公式(标量场)\upref{LSZ},Wick 定理(标量场)\upref{wick}}
为了消除 $\phi^4$ 理论中圈图的紫外发散,我们采取一定的正规化方案(常见的有截断正规化、维数正规化、格点正规化方案,在这里我们采用的是维数正规化),并引入重整化的拉氏量:
\begin{equation}
\begin{aligned}
\mathcal{L}
&=\frac{1}{2}Z_\phi \partial_\mu \phi \partial^\mu \phi - \frac{1}{2} Z_m m^2\phi^2 - \frac{Z_\lambda \lambda}{4!}\phi^4\\
&=\frac{1}{2}(\partial_\mu\phi)^2 - \frac{1}{2}m^2\phi^2 - \frac{\lambda}{4!}\phi^4+\frac{1}{2}\delta_Z (\partial_\mu \phi)^2 - \frac{1}{2}\delta_m \phi^2 - \frac{\delta_\lambda}{4!}\phi^4
\end{aligned}
\end{equation}
其中 $\delta_Z=Z_\phi-1,\delta_m = (Z_m-1)m^2,\delta_\lambda = (Z_\lambda-1)\lambda$,它们为微扰论的 Feynman 图表示贡献了抵消项顶点。我们需要在维数正规化方案下,调整这三个参数来保证可重整化条件的成立:
\begin{equation}
\begin{aligned}
&\text{two-point function:}\quad G^{(2)}(p^2)=\frac{i}{p^2-m^2+\epsilon}+(\text{terms regular at}\ p^2=m^2)\\
&\text{four-point function:}\quad G^{(4)}(s,t,u)_\text{amputated}=-i\lambda\quad (\text{at}\ s=4m^2,t=u=0)
\end{aligned}
\end{equation}
其中 $G^{(4)}_\text{amputated}$ 代表截肢的四点函数,即通过 LSZ 公式消去了外腿在壳时的极点行为,因此这个重整化条件的 $-i\lambda$ 正对应于实验中可观测的两个标量粒子的散射截面。通过第二个重整化条件可以确定 $\delta_\lambda$。$\delta_Z,\delta_m$ 的确定需要第一个重整化条件,通过计算自能 $M^2(p^2)$(所有单粒子不可约图的贡献为 $-iM^2(p^2)$),那么两点函数可以如下计算:
\begin{equation}
\begin{aligned}
G^{(2)}(p^2)&=\frac{i}{p^2-m^2+\epsilon}+\frac{i}{p^2-m^2+\epsilon} (-iM^2(p^2))\frac{i}{p^2-m^2+\epsilon}+\cdots\\
&=\frac{i}{p^2-m^2-M^2(p^2)+\epsilon}
\end{aligned}
\end{equation}
为了保证该函数在 $p^2=m^2$ 处有一阶极点的行为,且留数为 $1$,我们要求
\begin{equation}
\begin{aligned}
M^2(p^2)|_{p^2=m^2}=0,\quad \frac{\dd }{\dd p^2} M^2(p^2) |_{p^2=m^2}=0
\end{aligned}
\end{equation}
有了这两个方程以后,我们就可以确定 $\delta_Z$ 和 $\delta_m$。

\subsection{单圈修正}
计算自能到单圈:
\begin{equation}
\begin{aligned}
-iM^2(p^2)=-i\lambda \frac{1}{2}\int \frac{\dd[4]{k}}{(2\pi)^4} \frac{i}{k^2-m^2+i\epsilon} + i(p^2\delta_Z-\delta_m)
\end{aligned}
\end{equation}
这里有因子 $1/2$ ,是因为单圈自能图的对称因子为 $2$。$i(p^2\delta_Z-\delta_m)$ 为抵消项顶点的贡献。由于动量积分发散,我们采用维数正规化:
\begin{equation}
\begin{aligned}
-iM^2(p^2)=-i\lambda \frac{1}{2} \mu^{4-d}\frac{1}{(4\pi)^{d/2}}\frac{\Gamma(1-d/2)}{(m^2)^{1-d/2}} + i(p^2\delta_Z-\delta_m)
\end{aligned}
\end{equation}
其中第一项与 $p^2$ 无关,因此我们可以在 $O(\lambda)$ 的修正下确定 $\delta_Z$ 和 $\delta_m$:
\begin{equation}
\delta_Z = 0,\quad \delta_m = -\frac{\lambda}{2(4\pi)^{d/2}}\mu^{4-d} \frac{\Gamma(1-d/2)}{(m^2)^{1-d/2}}
\end{equation}
事实上,$\delta_Z$ 是不为零的,如果我们引入二圈图的修正以后,$\delta_Z=O(\lambda^2)$。

下面再来计算 $\delta_\lambda$,四点截肢的单圈图共有三种,分别对应 $u,s,t$ 通道的双粒子散射图(注意它们的对称因子为 $2$)。除了计算三个单圈图的贡献,还有一个抵消项顶点。所以
\begin{equation}
\begin{aligned}
i\mathcal{M}(p_1p_2\rightarrow p_3p_4) &= -i\lambda + \frac{(-i\lambda)^2}{2} \int \frac{\dd[4]{k}}{(2\pi)^4} \frac{i}{k^2-m^2+i\epsilon} \frac{i}{(k+p_1+p_2)^2-m^2+i\epsilon}\\
&\quad\quad + (p_2\leftrightarrow -p_3)+(p_2\leftrightarrow -p_4)-i\delta_\lambda\\
&=-i\lambda + (-i\lambda)^2 [iV((p_1+p_2)^2)+iV((p_1-p_3)^2)+iV((p_1-p_4)^2)]-i\delta_\lambda\\
&=-i\lambda + (-i\lambda)^2 [iV(s)+iV(t)+iV(u)]-i\delta_\lambda
\end{aligned}
\end{equation}
因此为了满足 $s=4m^2,t=u=0$ 时的重整化条件,$\delta_\lambda$ 应取为
\begin{equation}\label{phi4c1_eq1}
\delta_\lambda = -\lambda^2[V(4m^2)+2V(0)]
\end{equation}
可以用维数正规化和费曼参数化技巧对 $V(p^2)$ 作具体的计算:
\begin{equation}
\begin{aligned}
V(p^2)&=\frac{i}{2}\mu^{4-d}\int_0^1 \dd x \int \frac{\dd[d]{k}}{(2\pi)^d} \frac{1}{[k^2+2xk\cdot p+xp^2-m^2+i\epsilon]^2}\\
&=\frac{i}{2}\mu^{4-d}\int_0^1 \dd x \int \frac{\dd[d]{k}}{(2\pi)^d} \frac{1}{[l^2+x(1-x)p^2-m^2+i\epsilon]^2}\\
&=\frac{i}{2} \cdot \mu^{4-d}\frac{i\Gamma(2-d/2)}{(4\pi)^{d/2}}\int_0^1 \dd x\frac{1}{(m^2-x(1-x)p^2)^{2-d/2}}\\
(d\rightarrow 4)&=-\frac{1}{2(4\pi)^2}\int_0^1 \dd x 
\left(
    \frac{1}{\epsilon}-\gamma+\log(4\pi\mu^2)-\log[m^2-x(1-x)p^2]
    \right) ,\quad \epsilon =(4-d)/2
\end{aligned}
\end{equation}
将这个结果代入\autoref{phi4c1_eq1} 可以得到
\begin{equation}
\begin{aligned}
\delta_\lambda (d\rightarrow 4)=\frac{\lambda^2}{32\pi^2}\int_0^1\dd x \left(\frac{3}{\epsilon}
-3\gamma +3\log(4\pi\mu^2) - \log[m^2-x(1-x)4m^2]
-2\log[m^2]
\right)
\end{aligned}
\end{equation}

对于其他给定的 $s,t,u$,利用以上结果可以求出两个标量粒子的散射振幅,由于所有的耦合常数都被设置得满足重整化条件,圈图中发散的部分被抵消项相消,我们最终得到的 Feynman 矩阵元的结果是有限的:
\begin{equation}
i\mathcal{M}(s,t,u)=-i\lambda - \frac{i\lambda^2}{32\pi^2}\int_0^1 \dd x\left(\log\frac{m^2-x(1-x)s}{m^2-x(1-x)4m^2}\right)
\end{equation}
