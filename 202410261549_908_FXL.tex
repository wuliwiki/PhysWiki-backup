% 分析力学(综述)
% license CCBYSA3
% type Wiki

(本文根据 CC-BY-SA 协议转载翻译自维基百科\href{https://en.wikipedia.org/wiki/Analytical_mechanics}{相关文章})

在理论物理和数学物理中,分析力学或理论力学是一系列紧密相关的经典力学表述。分析力学利用表示整个系统的标量运动性质——通常是其动能和势能。运动方程通过某种标量变化的基本原理从标量量推导出来。

分析力学是在牛顿力学之后,于18世纪及以后由许多科学家和数学家发展起来的。牛顿力学考虑运动的矢量量,特别是系统组成部分的加速度、动量、力等,因此也可以称为矢量力学。标量是一个数量,而矢量则由数量和方向表示。这两种方法的结果是等价的,但分析力学在处理复杂问题时具有许多优势。

分析力学利用系统的约束条件来解决问题。这些约束限制了系统的自由度,并可以用于减少求解运动所需的坐标数。这种形式适合任意选择的坐标,在此语境下称为广义坐标。系统的动能和势能用这些广义坐标或广义动量表示,运动方程可以轻松建立,因此,分析力学比完全矢量化的方法能够更高效地解决许多力学问题。然而,对于非保守力或如摩擦力等耗散力,分析力学并不总是有效,此时可以回归到牛顿力学。

分析力学的两个主要分支是拉格朗日力学(在构型空间中使用广义坐标及其对应的广义速度)和哈密顿力学(在相空间中使用坐标及对应的动量)。两种表述通过广义坐标、速度和动量上的勒让德变换互相等价,因此它们包含相同的信息来描述系统的动力学。还有其他表述方法,如哈密顿-雅可比理论、劳斯力学和阿佩尔运动方程。任意形式的粒子和场的运动方程都可以从广泛适用的最小作用量原理推导而出。其中一个结果是诺特定理,它将守恒定律与其相关对称性联系起来。

分析力学并没有引入新的物理概念,也不比牛顿力学更为普遍。它是一系列等效的形式,具有广泛的应用。事实上,相同的原理和形式可以用于相对论力学和广义相对论,并在经过一些修正后用于量子力学和量子场论。

分析力学广泛应用于基础物理学和应用数学,尤其是在混沌理论中。

分析力学的方法适用于离散粒子系统,每个粒子具有有限的自由度。它们可以被修改以描述具有无限自由度的连续场或流体。这些定义和方程与力学中的定义和方程有着密切的类比。
\subsection{分析力学的动机}
力学理论的目标是解决物理学和工程学中出现的力学问题。从一个物理系统出发(如一个机械装置或一个恒星系统),建立一个微分方程形式的数学模型。该模型可以通过数值或解析方法求解,以确定系统的运动。

牛顿的矢量方法通过使用力、速度、加速度等矢量量来描述运动。这些量表征了被理想化为“质点”或“粒子”的物体的运动,即一个附有质量的单一质点。牛顿的方法已成功应用于广泛的物理问题中,包括粒子在地球引力场中的运动以及行星绕太阳的运动。在这种方法中,牛顿定律通过微分方程描述运动,问题随之简化为解该方程。

然而,当一个力学系统包含许多粒子时(如复杂的机械装置或流体),牛顿的方法难以应用。在适当的预防措施下(如将每个粒子与其他粒子隔离并确定作用在其上的所有力),可以使用牛顿的方法。然而,即使在相对简单的系统中,这样的分析也是繁琐的。牛顿认为他的第三定律“作用等于反作用”可以解决所有的复杂情况。【需要引用】但即使是旋转刚体等简单系统,这一说法也不完全正确。【需要澄清】在更复杂的系统中,矢量方法无法提供充分的描述。

分析方法通过将机械系统视为相互作用的粒子集合来简化问题,而不是将每个粒子视为孤立单元。在矢量方法中,必须分别确定每个粒子的力,而在分析方法中,只需知道一个单一的函数,它隐含地包含了系统中作用的所有力。这种简化通常通过先验规定的某些运动学条件来实现。然而,分析处理不需要知道这些力,而是将这些运动学条件视为已知。【需要引用】

然而,要推导出复杂机械系统的运动方程,仍需要一个统一的基础。【需要澄清】这种基础由各种变分原理提供:在每组方程背后,都有一个表达整个方程组含义的原理。给定一个被称为作用量的基本和普遍量,当某些其他力学量发生微小变化时,该作用量保持不变的原理生成所需的微分方程组。该原理的表述不依赖于任何特定的坐标系,所有结果都以广义坐标表示。这意味着分析运动方程在坐标变换下不会改变,这是矢量运动方程所不具备的不变性属性。【2】

对于“解”一组微分方程的确切含义并不完全清楚。当粒子的坐标以时间 \( t \) 和定义初始位置及速度的参数的简单函数表达时,问题被视为已解决。然而,“简单函数”并不是一个明确的概念:如今,一个函数 \( f(t) \) 不再像牛顿时代那样仅被视为 \( t \) 的形式表达(基本函数),而通常被视为由 \( t \) 确定的量,无法清晰地界定“简单”与“非简单”函数的界限。如果仅谈论“函数”,那么每个力学问题一旦在微分方程中被良好表述便已得到解决,因为给定初始条件,\( t \) 的值就能确定该时刻的坐标。特别是在现代计算机建模方法下,通过差分方程替代微分方程,可以以任何所需精度获得力学问题的数值解。