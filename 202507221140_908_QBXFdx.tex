% 切比雪夫多项式(综述)
% license CCBYSA3
% type Wiki

本文根据 CC-BY-SA 协议转载翻译自维基百科\href{https://en.wikipedia.org/wiki/Hermann_von_Helmholtz}{相关文章}。

\begin{figure}[ht]
\centering
\includegraphics[width=6cm]{./figures/aa69d9e30009e97a.png}
\caption{第一类切比雪夫多项式 $T_n$ 的前五项绘图} \label{fig_QBXFdx_1}
\end{figure}
切比雪夫多项式是两组与余弦函数和正弦函数相关的正交多项式,分别记作 $T_n(x)$ 和 $U_n(x)$。它们有多种等价定义方式,其中一种起始于三角函数的表示:

第一类切比雪夫多项式 $T_n$ 定义为:
$$
T_n(\cos \theta) = \cos(n\theta)~
$$
类似地,第二类切比雪夫多项式 $U_n$ 定义为:
$$
U_n(\cos \theta)\sin \theta = \sin((n+1)\theta)~
$$
乍一看,这些表达式是否真的定义了关于 $\cos \theta$ 的多项式并不明显,但可以通过莫阿弗公式(de Moivre’s formula)来证明这一点(见下文)。
\begin{figure}[ht]
\centering
\includegraphics[width=6cm]{./figures/57edd55432eeb41d.png}
\caption{第二类切比雪夫多项式 $U_n$ 的前五个多项式图像} \label{fig_QBXFdx_2}
\end{figure}
切比雪夫多项式 $T_n$ 是在区间 $[-1, 1]$ 上绝对值被限制在 1 以内、且首项系数最大的多项式。它们也是满足许多其他性质的“极值”多项式之一\(^\text{[1]}\)。

1952 年,科尔内利乌斯·兰齐奥斯指出,切比雪夫多项式在线性系统求解的逼近理论中具有重要作用\(^\text{[2]}\);$T_n(x)$ 的根,也称为切比雪夫节点,被用作多项式插值中的匹配点,从而优化插值过程。由此得到的插值多项式能够减小龙格现象的问题,并在最大范数意义下提供接近最佳的函数逼近,这也称为“极小极大”准则。这种逼近直接引出了克伦肖–柯蒂斯求积法的方法。

这些多项式以帕夫努季·切比雪夫的名字命名\(^\text{[3]}\)。使用字母 $T$ 是因为该名字的其他音译方式,如法语的 Tchebycheff、Tchebyshev,或德语的 Tschebyschow。
\subsection{定义}
\subsubsection{递推定义}
第一类切比雪夫多项式 $T_n(x)$ 可由以下递推关系定义:
$$
\begin{aligned}
T_0(x) &= 1, \\
T_1(x) &= x, \\
T_{n+1}(x) &= 2x \, T_n(x) - T_{n-1}(x).
\end{aligned}~
$$
第二类切比雪夫多项式 $U_n(x)$ 可由以下递推关系定义:
$$
\begin{aligned}
U_0(x) &= 1, \\
U_1(x) &= 2x, \\
U_{n+1}(x) &= 2x \, U_n(x) - U_{n-1}(x),
\end{aligned}~
$$
这个递推关系与第一类切比雪夫多项式的定义几乎相同,仅在 $n = 1$ 时的初始值规则上有所不同。
