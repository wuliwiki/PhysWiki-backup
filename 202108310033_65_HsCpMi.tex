% 条件概率与事件的独立性(高中)
% keys 高中|条件概率|相互独立事件

\begin{issues}
\issueDraft
\end{issues}

\subsection{条件概率}
对于任何两个事件 $A$ 和 $B$,在已知事件 $A$发生的条件下,事件 $B$ 发生的概率叫做\textbf{条件概率},用符号 $P(B|A)$ 来表示.

我们把事件 $A$ 和 $B$ 同时发生所构成的事件 $D$,称为事件 $A$ 与 $B$ 的 \textbf{交} (或\textbf{积}),记作 $D = A \cap B$ (或 $D = AB$).

一般地,我们有条件概率公式
\begin{equation}
P(B|A) = \frac{P(A \cap B)}{P(A)},P(A)>0
\end{equation}

\subsection{事件的独立性}
事件 $A$ 是否发生对事件 $B$ 发生的概率没有影响,即
\begin{equation}
P(B|A) = P(B)
\end{equation}
这时,我们称两个事件 $A,B$ \textbf{相互独立},并把这两个事件叫做\textbf{相互独立事件(mutually independent events)}.

在实际问题中,常常通过事件本质进行分析就可知道它们是否相互独立,而不需要进行类似上面的计算去验证.

一般地,当事件 $A$ 和 $B$ 相互独立时,$A$ 与 $\overline{B}$,$\overline{A}$ 与 $B$,$\overline{A}$ 与 $\overline{B}$ 也相互独立.

由条件概率公式和相互独立事件 $A$,$B$ 的定义,可以得到
\begin{equation}
\begin{aligned}
&P(B) = P(B|A) = \frac{P(A \cap B)}{P(A)} \\
&P(A\cap B) = P(A) \cdot P(B)
\end{aligned}
\end{equation}

我们可以进一步推得,
\begin{equation}
P(A_1\cap A_2 \cdots A_n) = P(A_1) \cdot P(A_2) \cdots P(A_n)
\end{equation}

\subsection{独立重复试验}
在相同的条件下,重复地做 $n$ 次试验,各次试验从的结果相互独立,那么一般就称它为 \textbf{ $n$ 次独立重复试验(independent repeated trials)}.

一般地,事件 $A$ 在 $n$次试验中发生 $k$ 次,共有 $C_n^k$ 种情形,由试验的独立性知 $A$ 在 $k$ 次试验中发生,而在其余 $n-k$ 次试验中不发生的概率都是 $p^k(1-p)^{n-k}$,所以由概率加法公式知,如果在一次试验中事件 $A$ 发生的概率是 $p$ 那么在 $n$ 次独立重复试验中,事件 $A$ 恰好发生 $k$ 次的概率为
\begin{equation}\label{HsCpMi_eq1} 
P_n(k) = C_n^kp^k(1-p)^{n-k}(k=0,1,2,\cdots,n)
\end{equation}
这里解释一下 $p^k(1-p)^{n-k}$ 的含义,我们在一次试验全部发生事件 $A$ 的概率为 $p$,不发生的概率为 $1-p$,发生的次数为 $k$ 次,不发生的次数为 $n-k$ 次,根据分步乘法计数原理,可得 $p^k(1-p)^{n-k}$.

在\autoref{HsCpMi_eq1} 中若将事件 $A$ 发生的次数设为 $X$,事件 $A$ 不发生的概率为 $q = 1 - k$,那么在 $n$ 次独立重复试验中,事件 $A$ 恰好发生 $k$ 的概率是
\begin{equation}
P(X=k) = C_n^kp^kq^{n-k}(k = 1,2,\cdots,n)
\end{equation}
于是得到 $X$ 的分布列

\begin{table}[ht]
\centering
\caption{分布列}
\begin{tabular}{|c|c|c|c|c|c|c|}
\hline
$X$ & $0$ & $1$ & $\cdots$ & $k$ & $\cdots$ & $n$ \\
\hline
$P$ & $C_n^0p^0q^n$ & $C_n^1p^1q^{n-1}$ & $\cdots$ & $C_n^kp^kq^{n-k}$ & $\cdots$ & $C_n^np^nq^0$ \\
\hline
\end{tabular}
\end{table}