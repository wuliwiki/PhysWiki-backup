% 韦达定理
% 代数方程|根与系数|数论|域|伽罗瓦|伽罗华|维达定理|Vieta's formula

\addTODO{未确定预备知识.}

韦达定理描述了多项式根与系数的关系.在中学数学中常见韦达定理最简单的形式,即实系数一元二次多项式的根与系数的关系.本词条要讨论的是最一般形式下的韦达定理.

\subsection{定理描述}

\begin{theorem}{一元二次多项书的韦达定理}\label{VietaF_the1}

设有实系数多项式$ax^2+bx+c$,其有两个根$x_1$和$x_2$,则有:
\begin{equation}
x_1+x_2 = -\frac{b}{a}
\end{equation}
\begin{equation}
x_1x_2 = \frac{c}{a}
\end{equation}

\end{theorem}

\autoref{VietaF_the1} 的推导在这里不赘述,用一元二次方程的求根公式就能解决.

事实上,韦达定理还适用于一般的实系数多项式的根,记忆起来也非常方便:

\begin{theorem}{韦达定理}\label{VietaF_the2}
设有实系数多项式$\sum_{i=0}^n a_ix^i$,据\textbf{代数学基本定理}\upref{BscAlg}知其应有$n$个根(重根按重数记),分别记为$x_1, x_2, \cdots, x_n$.则有:
\begin{equation}\label{VietaF_eq1}
\leftgroup{
    x_1+x_2+\cdots+x_n &= -\frac{a_{n-1}}{a_n}\\
    x_1x_2+x_1x_3+\cdots+x_{n-1}x_n &= \frac{a_{n-2}}{a_n}\\
    &\vdots\\
    x_1x_2\cdots x_n &= (-1)^n\frac{a_0}{a_n}
}
\end{equation}
\end{theorem}

显然,\autoref{VietaF_the1} 只是\autoref{VietaF_the2} 的一个特例.记忆\autoref{VietaF_eq1} 也不难:第$i$个式子就是每$i$个根为一组相乘、所有组相加,结果是$(-1)^i\frac{a_{n-i}}{a_n}$,分母永远是最高次项的系数.





\subsection{定理证明}

按\autoref{VietaF_the2} 题设,知
\begin{equation}
\sum_{i=0}^n a_ix^i = (x-x_1)(x-x_2)\cdots(x-x_n)
\end{equation}












