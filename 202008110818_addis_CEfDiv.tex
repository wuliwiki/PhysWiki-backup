% 点电荷电场的散度

\pentry{球坐标系中的梯度散度旋度及拉普拉斯算符\upref{SphNab}}

我们先看这么一个方程

\begin{equation}\label{CEfDiv_eq1}
\bvec E(\bvec r) = \frac{q}{4\pi\epsilon_0}\frac{\uvec r}{r^2}
\end{equation}
要求散度, 由\autoref{SphNab_eq2}~\upref{SphNab} 得
\begin{equation}
\div \frac{\uvec r}{r^2} = \frac{1}{r^2} \pdv{r} \qty(r^2 \frac{1}{r^2}) = 0
\end{equation}
注意由于\autoref{CEfDiv_eq1} 在原点处无定义, 该结论不适用于 $\bvec r = \bvec 0$.

根据电场的叠加原理, 即使空间中有许多点电荷, 在没有电荷的空间中, 电场的散度恒为零.
\begin{equation}
\div \bvec E = 0
\end{equation}
