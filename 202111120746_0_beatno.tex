% 拍频
% 简谐振动|频率|拍频

\begin{issues}
\issueDraft
\end{issues}

\pentry{简谐振子\upref{SHO}}

若两个简谐振动\upref{SHO}的频率较接近, 即它们的频率之差远小于各自的频率, 那么把它们叠加后会出现时间上的拍频, 如\autoref{beatno_fig1}.
\begin{figure}[ht]
\centering
\includegraphics[width=14cm]{./figures/beatno_1.pdf}
\caption{拍. 两个简谐振动相加后得到一个振幅随时间缓慢变化的振动(蓝线), 振幅随时间的变化用红线表示.} \label{beatno_fig1}
\end{figure}
图中两个独立的简谐振动分别为 $x_1(t) = \cos(t)$ 和 $x_2(t) = \sin(1.2t)$, 把它们相加后可以发现得到合成后的振动并不是一个简谐振动. 该振动频率与 $x_1(t)$ 和 $x_2(t)$ 相近, 但振幅却以另一个较大的周期变化. 这种两个频率差值较小的谐振动合成时, 合振幅出现时强时弱周期性缓慢变化的现象, 叫做\textbf{拍(beat)}. 拍出现的频率叫做\textbf{拍频}.

拍现象在技术上有重要应用.例如,管乐器中的双簧管就是利用两个簧片振动频率的微小差别产生颤动的拍音;调整乐器时,使它和标准音叉出现的拍音消失来校准乐器;拍现象常用于汽车速度监视器、地面卫星跟踪等.此外,在各种电子学测量仪器中,也常常用到拍现象.

从\autoref{beatno_fig1} 中不难发现拍是如何产生的, 在 $t = 0$ 附近的几个简谐振动中, $x_1, x_2$ 的相位大致相同, 即它们总是几乎同时达到最大值或最小值, 所以把他们相加后, 在这个时间内合成的振动的振幅就约等于各自的两倍.而当 $t = 16$ 左右时, 两个简谐振动的相位大约相差 $\pi$, 这就意味着两个振动的方向相反, $x_1$ 的最大值对应 $x_2$ 的最小值, 合成后几乎互相抵消. 以此类推, 合成后振动的振幅就会出现周期性变化.
容易猜到, 虽然本文讨论的是两个简谐振动产生的拍, 但其他常见的周期性振动若频率相近, 同样会产生拍的现象.

\subsection{简单的推导}
\addTODO{图:两个刻度线, 每个区间代表一个波长. 第一个区间大, 第二个区间小.}

令 $f_1(t), f_2(x)$ 的频率分别为 $\omega_1, \omega_2$, 周期为 $T_1, T_2$, 它们都相差很小, 即 $\abs{\omega_2 - \omega_1} \ll \min{\omega_1, \omega_2}$ 令 $\Delta \lambda = \lambda_2 - \lambda_1$, $\Delta k = k_2 - k_1$. 即 $\Delta \lambda/\lambda \ll 1$, $\Delta k/k \ll 1$.

为讨论方便我们讨论 $k_2 > k_1$ ($\lambda_2 < \lambda_1$) 的情况, 若某个点处两标记重合, 那么再经过大约 $\lambda/\Delta\lambda$ 个波长后, 会再次重合. 所以两个波包的距离为 $\lambda^2/\Delta\lambda$, 波包的空间频率为 $2\pi \Delta \lambda/\lambda^2$.

\subsection{使用三角函数推导}
设一质点在一直线上同时参与两个不同频率的谐振动,其振动表达式为
\begin{equation}
\begin{array}{l}x_{1}=A_{1} \cos \left(\omega_{1} t+\phi_{01}\right) \\ x_{2}=A_{2} \cos \left(\omega_{2} t+\phi_{02}\right)\end{array}
\end{equation}
根据叠加原理,合运动的位移为
\begin{equation}
x=x_{1}+x_{2}=A_{1} \cos \left(\omega_{1} t+\phi_{01}\right)+A_{2} \cos \left(\omega_{2} t+\phi_{02}\right)
\end{equation}

为方便计算,设 $A_1=A_2=A,\phi_{01}=\phi_{02}=\phi_{0}$,则上式可化成(\autoref{TriEqv_eq9}~\upref{TriEqv})
\begin{equation} \label{beatno_eq1}
x=2 A \cos \left(\frac{\omega_{2}-\omega_{1}}{2} t\right) \cos \left(\frac{\omega_{2}+\omega_{1}}{2} t+\phi_{0}\right)
\end{equation}
对于通常实际遇到的情况而言,两个频率比较接近,且 $\left|\omega_{2}-\omega_{1}\right|\ll \omega_1$,\autoref{beatno_eq1} 中第一项因子随时间作缓慢地变化,第二项因子是角频率近于 $\omega$(即接近于 $\omega_1,\omega_2$)的简谐函数,因此合成的振动可近似看成是角频率为 $(\omega_{1}+\omega_{2})/2 \approx \omega_{1} \approx \omega_{2}$,振幅为 $\left | 2 A \cos (\omega_{2}-\omega_{1})t/{2} \right |$ 的谐振动.

