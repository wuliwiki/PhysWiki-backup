% 磁性纳米粒子
% license CCBYSA3
% type Wiki

(本文根据 CC-BY-SA 协议转载自原搜狗科学百科对英文维基百科的翻译)

磁性纳米粒子是一类可以利用磁场操纵的纳米粒子。这种颗粒通常由两种成分组成,一种是磁性材料,通常是铁、镍和钴,另一种是具有功能性的化学成分。纳米粒子的直径小于1微米(通常为1-100纳米),而较大的微珠的直径为0.5-500微米。由许多单个磁性纳米粒子组成的磁性纳米粒子簇被称为直径为50-200纳米的磁性纳米珠。[1]磁性纳米粒子簇是它们进一步磁性组装成磁性纳米链的基础。磁性纳米粒子最近成为许多研究的焦点,因为它们具有吸引人的性质,可以在催化领域看到潜在的用途,包括纳米材料基催化剂、[2] 生物医学 [3]和组织特异性靶向,[4]磁性可调胶体光子晶体,[5] 微流体、[6] 磁共振成像,[7] 磁性粒子成像、[8] 数据存储,[9] 环境修复、[10] 纳米流体,[11][12]光学过滤器、[13]缺陷传感器[14],磁冷却[15][16]和阳离子。[17]

\subsection{性能}
磁性纳米粒子的物理化学性质很大程度上取决于合成方法和化学结构。在大多数情况下,颗粒的尺寸在1至100纳米的范围内,并且可能显示出超顺磁性。

\subsection{磁性纳米粒子的类型}
\subsubsection{2.1 氧化物:铁氧体}
\begin{figure}[ht]
\centering
\includegraphics[width=8cm]{./figures/5b7d98a62fae6427.png}
\caption{具有二氧化硅外壳的磁赤铁矿磁性纳米粒子团簇的透射电镜图像。[4][5]} \label{fig_CXNMLZ_1}
\end{figure}

\subsubsection{2.2 带壳铁氧体}
磁赤铁矿或磁铁矿磁性纳米粒子的表面相对惰性,通常不允许与功能化分子形成强共价键。然而,磁性纳米粒子的反应活性可以通过在其表面涂覆一层二氧化硅来提高。[20]二氧化硅壳可以通过有机硅烷分子和二氧化硅壳之间的共价键容易地用各种表面官能团改性。[21]此外,一些荧光染料分子可以共价键合到功能化的二氧化硅壳上。[22]

由涂覆有二氧化硅壳的超顺磁性氧化物纳米粒子(每珠约80个磁赤铁矿超顺磁性纳米粒子)组成的具有窄尺寸分布的铁氧体纳米粒子簇比金属纳米粒子有几个优点:[19]
\begin{itemize}
\item 更高的化学稳定性(对生物医学应用至关重要)
\item 窄尺寸分布(对生物医学应用至关重要)
\item 更高的胶体稳定性,因为它们不会磁性聚集
\item 磁矩可以根据纳米粒子簇的大小来调节
\item 保留的超顺磁性(与纳米粒子簇的大小无关)
\item 二氧化硅表面能够直接共价功能化
\end{itemize}

\subsubsection{2.3 金属的}
\begin{figure}[ht]
\centering
\includegraphics[width=8cm]{./figures/5ccf99e5e11601a6.png}
\caption{具有石墨烯壳的钴纳米粒子(注意:单个石墨烯层是可见的)[9]} \label{fig_CXNMLZ_2}
\end{figure}
金属纳米颗粒由于其较高的磁矩而可能有利于某些技术应用,而氧化物(磁赤铁矿、磁铁矿)将有利于生物医学应用。这也意味着在同一时刻,金属纳米粒子可以做得比它们相对应的氧化物更小。另一方面,金属纳米颗粒的最大缺点是容易自燃,和对氧化剂有不同程度的反应。该缺点使对它们的处理变得困难,并导致不必要的副反应,这使得它们不太适合生物医学应用。金属颗粒的胶体形成也更具挑战性。

\subsubsection{2.4 带壳的金属}
磁性纳米粒子的金属核可以通过温和氧化、表面活性剂、聚合物和贵金属进行钝化。[23]在氧环境中,钴纳米粒子在钴纳米粒子的表面形成反铁磁CoO层。最近,人们研究了具有金外壳的钴核氧化钴壳纳米粒子的合成和交换偏置效应。[23]最近合成了具有由单质铁或钴组成的磁芯和由石墨烯制成的无活性壳的纳米粒子。与铁氧体或单质纳米粒子相比,优势在于:
\begin{itemize}
\item 更高的磁化强度
\item 在酸性和碱性溶液以及有机溶剂中具有更高的稳定性
\item 通过已知的碳纳米管方法在石墨烯表面进行化学反应[24]
\end{itemize}

\subsection{综合}
合成存在的几种制备磁性纳米粒子的方法。

\subsubsection{3.1 共沉淀}
共沉淀是在室温或高温惰性气氛下通过加入碱从$Fe^{2+}/Fe^{3+}$盐水溶液合成氧化铁($Fe_3O_4\text{或}\gamma-Fe_2O_3$)的简便方法。磁性纳米粒子的大小、形状和组成在很大程度上取决于所用盐的类型(例如氯化物、硫酸盐、硝酸盐)、$Fe^{2+}/Fe^{3+}$比率、反应温度、介质的酸碱度和离子强度,[25]以及与用于引发沉淀的基础溶液的混合速率。[25]共沉淀方法被广泛用于生产尺寸和磁性可控的铁氧体纳米粒子。[26][27][28][29]已经报道了多种实验安排,以通过快速混合促进磁性粒子的连续和大规模共沉淀。[30][31]最近,在反应物混合区内,通过集成交流磁化率计在磁铁矿纳米颗粒沉淀过程中实时测量磁性纳米颗粒的生长率。[32]

\subsubsection{3.2 热解}
较小尺寸的磁性纳米晶体基本上可以通过碱性有机金属化合物在含有稳定表面活性剂的高沸点有机溶剂中的热分解来合成。[25][33][34]

\subsubsection{3.3 微乳液}
采用微乳液技术,以1-丁醇为助表面活性剂,辛烷为油相,在十六烷基三甲基溴化铵反胶束中合成了金属钴、钴/铂合金和镀金钴/铂纳米粒子。[25][35]

\subsubsection{3.4 火焰喷雾合成}
使用火焰喷雾热解[36][36]并改变反应条件,以大于30 g/h的速率生产氧化物、金属或碳包覆的纳米颗粒。
\begin{figure}[ht]
\centering
\includegraphics[width=6cm]{./figures/a0e7c9e9fdd218db.png}
\caption{请添加图片标题} \label{fig_CXNMLZ_3}
\end{figure}
\begin{figure}[ht]
\centering
\includegraphics[width=6cm]{./figures/3ccbc8aae63db745.png}
\caption{请添加图片标题} \label{fig_CXNMLZ_4}
\end{figure}
