% 矢量函数

\begin{issues}
\issueOther{需要画图}
\end{issues}

\pentry{函数(高中)\upref{functi}, 几何矢量的基底和坐标\upref{Gvec2}}

\footnote{本文参考了\cite{Thomas}。}我们在一元函数函数\upref{functi}与多元函数中已经学会了函数的映射含义:一元函数就是把定义域中的一个数通过一定的方式转换为另一个数;而多元函数相当于把定义域中的\textsl{一组数}通过一定的方式转换为另一个数。
$$
r=f(a) \qquad r=f(a_1, a_2, a_3,...)
$$

如今,我们要面对矢量函数。他的含义也完全类似:矢量函数把定义域中的\textsl{一个矢量\upref{GVec}}通过一定的方式转换为\textsl{另一个矢量}。
$$
\bvec r = \bvec f (\bvec a)
$$
当然,我们也有\textsl{多元矢量函数}(这是笔者使用的名称,似乎没有人这么称呼!),他可以把定义域中的\textsl{一组矢量与/或标量}通过一定的方式转换为\textsl{另一个矢量}。
$$
\bvec r = \bvec f (\bvec a, t, ...)
$$

如果你还是暂时理解不了矢量函数的概念,更简单粗暴的方法是拆分所有矢量为分量形式,这样你就把一个矢量函数化为了一组多元函数。当然,为了能够写出矢量的分量形式,你必须指明坐标系。

例如,我们上文所言的矢量函数可以拆分为这$3$个多元函数:
$$
\bvec r = \bvec f (\bvec a)
\Leftrightarrow
\left \{
\begin{aligned}
r_x &= f_x(\bvec a, t)\\
r_y &= f_y(\bvec a, t)\\
r_z &= f_z(\bvec a, t)\\
\end{aligned}
\right.
\Leftrightarrow
\left \{
\begin{aligned}
r_x &= f_x(a_x,a_y,a_z,t)\\
r_y &= f_y(a_x,a_y,a_z,t)\\
r_z &= f_z(a_x,a_y,a_z,t)\\
\end{aligned}
\right.
$$
其中 $r_x, r_y, r_z$是$\bvec r$分别在三个轴上的分量,$f_x,f_y,f_z$是三个多元函数(这里的下标不是求偏导的意思!)。

有时,矢量形式的矢量函数更能直观地反映函数背后的物理含义,而分量形式反而让你陷入具体数值的池沼。因此,在你熟悉矢量函数的概念后,我不建议你继续做这种拆分。
