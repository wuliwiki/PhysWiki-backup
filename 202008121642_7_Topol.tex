% 拓扑空间
% keys 拓扑|映射|集合|开子集|非开|连续性

\pentry{集合\upref{Set}}

\textbf{拓扑空间(topological space)}是能够定义连续性, 连通性, 收敛等性质的最一般化的数学空间. 度量空间和流形等都是拓扑空间的例子.

\subsection{拓扑}

从实函数的连续性\upref{contin}中我们知道开集是讨论连续性的基础,所以拓扑学首先要定义什么是开集.我们提取了开集最重要的特征,然后用这些特征来定义开集.

\begin{definition}{拓扑}\label{Topol_def1}
对于任意给定的集合 $X$, 如果我们按照一定规则将它的子集划分为\textbf{开集(open set)}和其它子集, 那么所有开集的集合就叫做集合 $X$ 的一个\textbf{拓扑} $\mathcal{T}$. 这个规则是:
\begin{enumerate}
\item 空集 $\varnothing$ 和 $X$ 本身必须是开子集
\item \textbf{有限交封闭}:有限个开子集的交集为开子集
\item \textbf{任意并封闭}:任意个开子集的并集为开子集
\end{enumerate}
\end{definition}

$X$的全体子集构成一个族,即$X$的幂集$2^X$,显然有$\mathcal{T}\subset2^X$.同一个集合上的拓扑一般不止一种.

每一个开集$U$的补集$U^C=X-U$,被称为一个\textbf{闭集(closed set)}.$2^X$的元素中,有些是开集,有些是闭集,有的既开又闭,也有的既不开也不闭.

\begin{example}{凝聚拓扑和离散拓扑}
对给定的集合 $X$, 若只定义 $\varnothing$ 和 $X$ 本身为开集, $X$ 的其他子集为非开, 则这个拓扑称为\textbf{凝聚拓扑}或者\textbf{平凡拓扑}, 这是符合\autoref{Topol_def1} 的元素最少的拓扑.

相对地, 若令 $X$ 的任意子集都为开集, 则得到\textbf{离散拓扑}, 这是元素最多的拓扑.
\end{example}

\begin{example}{Sierpinski拓扑}
令集合为$X=\{0, 1\}$,赋予拓扑$\mathcal{T}=\{\varnothing, \{0\}, X\}$,则我们得到了一个Sierpinski空间.
\end{example}

光有正例也并不容易建立对“拓扑”这一概念的直观,我们再看一个简单的反例:

\begin{example}{反例}
令集合为$X=\{0, 1\}$,则集族$\{\varnothing, \{0\},\{1\}, X\}$并不是一个拓扑,因为$\{0\}\cup\{1\}=\{0,1\}$并不在这个集族里,从而不满足\textbf{任意并封闭}条件.
\end{example}

\begin{definition}{拓扑基}\label{Topol_def2}
对于给定的拓扑空间$(X, \mathcal{T})$,如果$\mathcal{B}$是$\mathcal{T}$的一个子族,而且使得拓扑$\mathcal{T}$中的任何开集都可以表示为$\mathcal{B}$中若干开集(个数是任意的,可以是非负整数也可以是任意无限数)的并,那么称$\mathcal{B}$是$\mathcal{T}$的一个\textbf{拓扑基(topological basis)}.
\end{definition}

拓扑基的定义中,要求$\mathcal{B}$中的元素都是$X$的开集,而$X$中一切开集都可以由这些元素取并集而得到.事实上,也可以不用强调
$\mathcal{B}$中的元素都是开集,而是简单地说一个集合$U$是开集当且仅当$U$是$\mathcal{B}$中若干元素的并,这其中当然也包括单个元素,所以这种说法蕴含了“$\mathcal{B}$中的元素都是开集”.

任意拿一些$X$的子集来构成一个族$\mathcal{A}$,这个族不一定是某个拓扑的拓扑基;也就是说,$\mathcal{A}$中任意多个元素的并集所构成的集合,并不一定是一个拓扑.所以不是所有的$\mathcal{A}$都可以成为某个拓扑的拓扑基.如果$\mathcal{A}$是某个拓扑的拓扑基,则简单称其为一个拓扑基\footnote{$2^X$的子族$\mathcal{A}$是一个拓扑基,当且仅当任意$A_1, A_2\in\mathcal{A}$以及任意的$x\in A_1\cap A_2$,则存在$A_x\subseteq A_1\cap A_2$,使得$x\in A_x$且$A_x\in \mathcal{A}$.这是拓扑基的判别法,不要求掌握,感兴趣的读者可以自行证明.}.


\begin{example}{度量拓扑}

参见度量空间\upref{Metric}词条.

通常的欧几里得空间$\mathbb{R}^n=\{(x_1, \cdots x_n)|x_i\in \mathbb{R}\}$,记$\{(y_1, \cdots y_n)|y_i\in \mathbb{R}, \sum_i(y_i-x_i)^2\leq r^2\}$为以$(x_1, \cdots x_n)$为球心、$r\geqslant 0$为半径的\textbf{开球(open sphere)}. 那么全体开球的集合$\mathcal{B}$是某个拓扑$\mathcal{T}$的拓扑基,$\mathcal{T}$此时就是一个\textbf{度量拓扑(metric topology)}.这个拓扑就是最常见的实空间上的拓扑,有很直观的几何意义.

由开球生成的这个拓扑,还有一种定义方法:$A\subset \mathbb{R}^n$是一个开集,当且仅当对于任意的$x\in A$,存在一个半径$r$,使得到$x$距离小于$r$的所有点都在$A$内.也就是说,开集的点都是内点.
\end{example}


\begin{definition}{子拓扑}\label{Topol_def3}
如果已经给定了一个拓扑$(X, \mathcal{T})$,那么在$X$的一个子集$A$上可以继承一个拓扑$\mathcal{T}|_A$,定义为:
$\mathcal{T}|_A=\{U\cap A|U\in \mathcal{T}\}$.那么$\mathcal{T}|_A$称为$\mathcal{T}$的\textbf{子拓扑(subtopology)}或者\textbf{限制拓扑},$(A, \mathcal{T}|_A)$构成了$(X, \mathcal{T})$的一个\textbf{子拓扑空间(subspace)}.
\end{definition}

子拓扑的开集,就是原拓扑空间$X$的开集和子空间集合$A$的交.

\begin{example}{度量空间的子空间}
设$(\mathbb{R}^2, \mathcal{T}_2)$是二维实度量空间,即$x-y$平面;$(\mathbb{R}, \mathcal{T}_1)$是一维实度量空间,即$x$轴.那么$\mathcal{T}_1$刚好是$\mathcal{T}_2$的子拓扑.
\end{example}

\subsection{小结和一点拓展}

给定集合$X$上的一个拓扑$\mathcal{T}$,就是$X$上所有开集的集合.也就是说,一个拓扑是$X$的幂集\footnote{幂集的概念请见集合\upref{Set}中脚注.}的子集:$\mathcal{T}\subset 2^X$.给定拓扑,相当于规定了哪些子集是开集.一个子集可以是开集,也可以是闭集;可以既开又闭,还可以既不开也不闭.

我们如上定义了拓扑基的概念,一个拓扑基$\mathcal{B}$也是$2^X$的一个子集,但是它不一定成为拓扑,因为它不要求满足任意并封闭和有限交封闭.

事实上,任意给定$2^X$的一个子集$\mathcal{S}$,即使它不是一个拓扑,我们也可以找到一个包含它的最小拓扑$T(\mathcal{S})$,这个时候称$\mathcal{S}$是$T(\mathcal{S})$的一个\textbf{子基(sub-basis)}.可以证明,先把$\mathcal{S}$的有限多个元素\footnote{注意,$\mathcal{S}$的元素本身是集合.}拿出来计算出交集,把一切这样的有限交放在一起,构成一个新的集合$U(\mathcal{S})$;然后再把$U(\mathcal{S})$中的任意多元素拿出来,取并集,再构成一个新的集合$I(U(\mathcal{S}))$.这样先取全体有限交、再取全体任意并所得到的集合$I(U(\mathcal{S}))$,就是$T(\mathcal{S})$.

拓扑基也是子基,但是它特殊的地方在于,一个拓扑基$\mathcal{B}$只需要进行一次任意并运算就可以得到拓扑了:$T(\mathcal{B})=I(\mathcal{B})$,省去了考虑交集的麻烦.

对于任意的$\mathcal{S}\subset2^X$,判定它是否是(某个拓扑的)拓扑基的一个常用等价条件是:对于任意的$S_1, S_2\in \mathcal{S}$以及任意的$x\in S_1\cap S_2$,总能找到一个$S_3$,使得$x\in S_3\subset S_1\cap S_2$.
