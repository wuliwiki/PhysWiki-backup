% 角动量的叠加原理

假设氢原子处于基态(这样就不用考虑轨道角动量),其电子和质子都是自旋为$1/2$的粒子.这两个粒子都可以自旋向上或者自旋向下,也就是说由四种自旋的可能性:
\begin{equation}
\chi^{(1)}_+\chi^{(2)}_+\equiv\uparrow\uparrow;\ \chi^{(1)}_+\chi^{(2)}_-\equiv\uparrow\downarrow;\ \chi^{(1)}_-\chi^{(2)}_+\equiv\downarrow\uparrow;\ \chi^{(1)}_-\chi^{(2)}_-\equiv\downarrow\downarrow
\end{equation}
更严格地讲,每一个粒子是处在上自旋和下自旋线性组合的状态,两个粒子构成的体系是上面
四个态的线性组合态.其中$\chi^{(1)}$也就是第一个箭头代表电子自旋,第二个$\chi^{(2)}$代表质子自旋.

我们将这个氢原子的总角动量定义为:
\begin{equation}
\bvec S \equiv \bvec S^{(1)}+\bvec S^{(2)}
\end{equation}

