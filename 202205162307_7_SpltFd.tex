% 分裂域
% keys splitting field|正规扩张|regular extension


\pentry{域的扩张\upref{FldExp}}


本节我们要介绍一个在代数中非常基础且重要的概念:分裂域.简单来说,分裂域就是在一个域中添加某个多项式的全体根所得到的扩域.从分裂域出发,我们可以讨论代数扩域的自同构问题.




\subsection{分裂域的存在性}


\begin{definition}{分裂域}
给定域$\mathbb{F}$及其上一个多项式$f(x)$.若存在扩域$\mathbb{K}/\mathbb{F}$,使得$f(x)$在$\mathbb{K}$上可以分解为$f(x)=\prod_{i=1}^n (x-a_i)$,且$\mathbb{K}=\mathbb{F}(a_1, a_2, \cdots, a_n)$,则称$\mathbb{K}$是$f(x)\in \mathbb{F}[x]$上的\textbf{分裂域(splitting field)}.
\end{definition}

定义看起来有些绕口,先说$f$在$\mathbb{K}$中可以分解,也就是说每一个根都存在,再说$\mathbb{K}$可以看成用这些根对$\mathbb{F}$进行扩域的结果.这么定义是因为我们要先确定元素$a_i$都存在,而为此就需要先确定$\mathbb{K}$存在.但是定义中只说了“若$\mathbb{K}$”存在,这个假设到底成立与否呢?答案是肯定的.

\begin{theorem}{分裂域的存在性}
给定域$\mathbb{F}$及其上一个多项式$f(x)$,则$f(x)\in \mathbb{F}[x]$上的分裂域存在.
\end{theorem}

\textbf{证明}:

当$\opn{deg}f=1$时,定理自然成立,此时$f\in\mathbb{F}[x]$的分裂域就是其本身.

首先在环$\mathbb{F}[x]$上对元素$f(x)$进行因式分解\footnote{也就是画出它的一棵\textbf{真因子树}\upref{FctTre}.},得到其不可约因子.任选其中一个不可约因子$h(x)$,如果$\opn{deg} h = 1$,则跳过本段接下来的步骤.构造环$R_1=\mathbb{F}(x)/<h(x)>=\mathbb{F}(a_1)$,再取其分式域$\mathbb{F}_1$,则$\mathbb{F}_1$就是$\mathbb{F}$的单扩张$\mathbb{F}(a_1)$.

由\textbf{多项式环}\upref{RPlynm}的\autoref{RPlynm_the1}~\upref{RPlynm},$(x-a_1)|h(x)$,因此在$\mathbb{F}_1$上可以分解出$h_1(x)=h(x)/(x-a_1)$.如果$\opn{deg}h_1 = 1$,则跳过本段接下来的步骤.对$h_1(x)$进行相同的操作:构造环$R_2=\mathbb{F}(x)/<h_1(x)>=\mathbb{F}(a_2)$,再取其分式域$\mathbb{F}_2=\mathbb{F}_1(a_2)=\mathbb{F}(a_1, a_2)$.

以此类推,直到$h(x)$在$\mathbb{F}_{k_1}$上分解为一阶多项式之积.

接下来,取$f$在$\mathbb{F}_{k_1}$上的不可约因子$g(x)$,如果$\opn{deg} g = 1$,则跳过本段接下来的步骤.执行相同的扩域操作,直到得到$\mathbb{F}_{k_1+k_2}$,使得$g$在$\mathbb{F}_{k_1+k_2}$上分解为一阶多项式之积.

以此类推,最终可以得到$\mathbb{F}_k$,使得$f$在$\mathbb{F}_k$上可以分解为一阶多项式之积.则$\mathbb{F}_k$就是$f\in\mathbb{F}[x]$的分裂域.

\textbf{证毕}.

该证明过程的大体思路,就是看$f$的根是否在已知的域中.根$a$的最小多项式$h(x)$必是$f(x)$的一个不可约因子.如果$a$在已知的域中,那么$f$就可以因式分解出一阶多项式因子$(x-a)$;否则,就添加$a$进行一次单扩域,这次扩域至少能把$a$纳入,但也有可能把其它根一起纳入.因此我们容易得到以下推论:

\begin{corollary}{}
设$\mathbb{K}$是$f(x)\in \mathbb{F}[x]$上的分裂域,则$[\mathbb{K}:\mathbb{F}]\leq \opn{deg}f$.
\end{corollary}

\begin{corollary}{}
设$\mathbb{K}$是$f(x)\in \mathbb{F}[x]$上的分裂域,$\mathbb{M}$是$\mathbb{K}$和$\mathbb{F}$之间的\textbf{中间域},则$\mathbb{K}$也是$f(x)\in \mathbb{E}[x]$上的分裂域.
\end{corollary}

为了加深理解,我们讨论一个分裂域的例子.添加元素的过程中会遇到的主要情况在这里都出现了.

\begin{example}{分裂域的一个例子}


在有理数域$\mathbb{Q}$上有多项式$f(x)=(x^2-2)^2(x^2-3)(x^2-6)(x^2+1)$,其在$\mathbb{Q}$上有五个不可约因子:$(x^2-2), (x^2-2), (x^2-3), (x^2-6), (x^2+1)$.

考虑因子$(x^2-2)$,得到扩域$\mathbb{Q}(\sqrt{2})$.在$\mathbb{Q}(\sqrt{2})$有分解:
\begin{equation}
f(x)=(x+\sqrt{2})^2(x-\sqrt{2})^2(x^2-3)(x^2-6)(x^2+1)
\end{equation}

取其不可约因子$x^2-3$,得到扩域$\mathbb{Q}(\sqrt{2}, \sqrt{3})$.

在$\mathbb{Q}(\sqrt{2}, \sqrt{3})$上,$f$有分解:

\begin{equation}
\begin{aligned}
f(x)=&(x+\sqrt{2})^2(x-\sqrt{2})^2(x+\sqrt{3})(x-\sqrt{3})\times\\
&(x+\sqrt{2}\sqrt{3})(x-\sqrt{2}\sqrt{3})(x^2+1)
\end{aligned}
\end{equation}

取其不可约因子$x^2+1$,得到扩域$\mathbb{Q}(\sqrt{2}, \sqrt{3}, \I)$.

则$\mathbb{Q}(\sqrt{2}, \sqrt{3}, \I)=\{a+A\I+(b+B\I)\sqrt{2}+(c+C\I)\sqrt{3}+(d+D\I)\sqrt{6}|a, A, b, B, c, C, d, D\in\mahtbb{Q}\}$就是$f\in\mathbb{Q}[x]$的分裂域.

\end{example}

\begin{corollary}{}
设$\mathbb{K}$是$f(x)\in \mathbb{F}[x]$上的分裂域,则$[\mathbb{K}:\mathbb{F}]\leq \opn{deg}f$.
\end{corollary}

\begin{corollary}{}
设$\mathbb{K}$是$f(x)\in \mathbb{F}[x]$上的分裂域,$\mathbb{M}$是$\mathbb{K}$和$\mathbb{F}$之间的\textbf{中间域},则$\mathbb{K}$也是$f(x)\in \mathbb{E}[x]$上的分裂域.

\end{corollary}


分裂域的构造也可以这样理解:给定一个域$\mathbb{F}$和其上一个\textbf{不可约}多项式$f$,则$f\in\mathbb{F}[x]$的分裂域就是$\mathbb{F}[x]/<f(x)>$的\textbf{分式域}\upref{FrcFld}.对于任意的$f\in\mathbb{F}[x]$,如果它能写为若干不可约多项式之积,如$f=h_1h_2\cdots h_n$,那么$f$的分裂域是:求$h_1\in\mathbb{F}[x]$的分裂域$\mathbb{F}_1$,再求$h_2\in\mathbb{F}_1[x]$的分裂域$\mathbb{F}_2$,以此类推,$\mathbb{F}_n$即为$f\in\mathbb{F}[x]$的分裂域.










\subsection{分裂域的唯一性}

从\textbf{开拓}(\autoref{FldExp_def4}~\upref{FldExp})的角度来说,如果存在域同构$\sigma:\mathbb{F}_1\to\mathbb{F}_2$,将其开拓为环同构$\sigma:\mathbb{F}_1[x]\to\mathbb{F}_2[x]$,任取$f\in\mathbb{F}_1[x]$,设$f\in\mathbb{F}_1[x]$的分裂域为$\mathbb{K}_1$,$\sigma(f)\in\mathbb{F}_2[x]$的分裂域为$\mathbb{K}_2$,则$\sigma$可以开拓为$\mathbb{K}_1\to\mathbb{K}_2$的同构.

上述开拓的角度或许有些绕,但考虑到“同构的域就是同一个域”,我们完全可以大大简化上述表达:

\begin{theorem}{}
给定域$\mathbb{F}$和其上一个多项式$f$以后,所构造出来的分裂域是唯一的,或者说构造出来的两个分裂域都是同构的.
\end{theorem}

证明非常简单,只需要利用多项式环和分式域的唯一性即可.多项式环的唯一性依赖于\autoref{RPlynm_the2}~\upref{RPlynm},分式域的唯一性由\autoref{FrcFld_the1}~\upref{FrcFld}得到.

再回到开拓的角度来思考:上面只说了$\sigma$可以开拓为$\mathbb{K}_1\to\mathbb{K}_2$的同构,那这种同构是不是唯一的呢?一般情况下不是的.继续用“同构就是同一个”的思维,我们有如下定理:

\begin{theorem}{}\label{SpltFd_the1}
给定域$\mathbb{F}$和其上一个多项式$f$,设$f\in\mathbb{F}[x]$的分裂域是$\mathbb{K}$,$\mathbb{K}$到自身的保$\mathbb{F}$自同构数量为$N$,那么$N\leq[\mathbb{K}:\mathbb{F}]$.

当且仅当$f$的每一个不可约因子$h$的不同根数目恰为$\opn{deg}h$时,等号成立.
\end{theorem}

\textbf{证明}:

定理中第二段的描述已经暗示了证明思路.

设$\sigma:\mathbb{K}\to\mathbb{K}$是自同构,且对于任意$a\in\mathbb{F}$有$\sigma(a)=a$.

对于任意$g\in\mathbb{F}[x]$和任意$k\in\mathbb{K}$,必有$\sigma(g(k))=g(\sigma(k))$.因此$k$和$\sigma(k)$的最小多项式相同.因此,$f$的任意不可约因子$h$的根,必须被$\sigma$映射为$h$的根.

由于分式域的唯一性,要保证$\sigma$是同态,$h$的任何根都可以映射到任何根上.

综上,$\sigma$必须将$h$的根映射为$h$的根,但可以映射到任何根上.结合\autoref{FldExp_the1}~\upref{FldExp}即得证.

\textbf{证毕}.

\begin{example}{}

给定有理数域$\mathbb{Q}$及其上的多项式$f(x)=x^2-2$.则$f\in\mathbb{Q}$的分裂域为$\mathbb{Q}(\sqrt{2})=\{a+b\sqrt{2}|a, b\in\mathbb{Q}\}$.

$\mathbb{Q}(\sqrt{2})$一共有两个保$\mathbb{Q}$自同构:第一个就是恒等映射,第二个$\sigma$则定义如下:
\begin{equation}
\sigma(a+b\sqrt{2})=a-b\sqrt{2}
\end{equation}
也就是说,$\sigma$把根$\pm\sqrt{2}$映射到根$\mp\sqrt{2}$.


而$[\mathbb{Q}(\sqrt{2}):\mathbb{Q}]=2$,因此这是一个\autoref{SpltFd_the1} 取等号的例子.

\end{example}

\begin{exercise}{}
求$x^3-2\in\sqrt{Q}[x]$的分裂域及其所有保$\mathbb{Q}$自同构.
\end{exercise}



\begin{exercise}{}
求$x^p-1\in\mathbb{Q}[x]$的分裂域及其所有保$\mathbb{Q}$自同构.这里$p$为素数.注意判断$x^p-1$是否为不可约多项式.
\end{exercise}


\begin{exercise}{}
求$x^6-1\in\mathbb{Q}[x]$的分裂域及其所有保$\mathbb{Q}$自同构.
\end{exercise}





\subsection{正规扩张}

分裂域的性质,其实对应的是一种非常重要的域扩张,它与代数方程的根式解问题息息相关.

\begin{definition}{正规扩张}
设$\mathbb{K}/\mathbb{F}$是一个\textbf{有限}扩域.如果对于$\mathbb{F}$上的任意不可约多项式$f$,要么$f$在$\mathbb{K}$中无根,要么就所有根都在$\mathbb{K}$中,则称$\mathbb{K}/\mathbb{F}$是一个\textbf{正规扩张}.
\end{definition}

\begin{theorem}{}
设$\mathbb{K}/\mathbb{F}$是一个\textbf{有限}扩域,那么有:

$\mathbb{K}/\mathbb{F}$为正规扩张$\iff$ $\mathbb{K}$是某个多项式$f\in\mathbb{F}$的分裂域.
\end{theorem}

\textbf{证明}:

$\Leftarrow$易证,在此不赘述.下证$\Rightarrow$.



\textbf{证毕}.
















