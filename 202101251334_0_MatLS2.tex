% 线性映射的结构

\pentry{补空间\upref{DirSum}, 矩阵的秩\upref{MatRnk}}

\begin{figure}[ht]
\centering
\includegraphics[width=9cm]{./figures/MatLS2_1.pdf}
\caption{两个子空间的线性映射: $X_0$ 是零空间, $X_1$ 是 $X_0$ 在 $X$ 中的补空间} \label{MatLS2_fig1}
\end{figure}
我们来看一个揭示线性映射的结构的定理(\autoref{MatLS2_fig1} ). 零空间(\autoref{LinMap_def2}~\upref{LinMap}) $X_0 \subset X$ 中的每个矢量都被 $A$ 映射到 $Y$ 空间中的零矢量, 即 $A(X_0) = \qty{0}$.
\begin{theorem}{}
令 $A:X \to Y$ 的零空间为 $X_0$, $X_1$ 为 $X_0$ 在 $X$ 中的一个补空间(\autoref{DirSum_def1}~\upref{DirSum}), 令 $Y_1 = A(X)$, 那么
\begin{equation}
A(X_1) = Y_1
\end{equation}
且映射 $A:X_1\to Y_1$ 是一一对应的.
\end{theorem}
由这个定理可得线性映射的一个重要性质
\begin{corollary}{}
令线性映射为 $A:X\to Y$ 对应的矩阵为 $\mat A$. $X$ 是 $N_X$ 维矢量空间, $A$ 的零空间为 $X_0$, 是 $X$ 的子空间. 令 $\mat A$ 的秩为 $R$, 那么
\begin{equation}
N_X = N_0 + R
\end{equation}
\end{corollary}

\addTODO{线性方程组词条\upref{LinEq}中说明: 线性方程组的解空间就是 $X_1$ 中的特解加上齐次解 $X_0$.}

\subsubsection{证明}
\addTODO{未完成}
