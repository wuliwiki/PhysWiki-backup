% Galois扩域
% keys 伽罗华扩域|伽罗瓦扩域|代数方程|根式解|古典数学难题

\pentry{正规扩张\upref{NomEx},群作用\upref{Group3}}

完成了对正规扩张和可分扩张的讨论,我们引入极为核心的Galois扩域.在上述讨论中,我们时常涉及域自同构,也看到了域自同构和多项式的根之间对应的关系.对域自同构的结构的研究,将把我们引向著名的古典数学难题“多项式方程的根式解”.



\subsection{Galois扩张的基本性质}

\begin{definition}{Galois扩张}

若域扩张$\mathbb{K}/\mathbb{F}$是\textbf{正规}且\textbf{可分}的,那么称之为一个\textbf{伽罗华扩张(Galois extension)},或译作\textbf{伽罗瓦扩张}.

此时称$\mathbb{K}$\textbf{在}$\mathbb{F}$\textbf{上是Galois的}($\mathbb{K}$ \textbf{is Galois over} $\mathbb{F}$).

\end{definition}

由于特征为零的域都是完美域,因此对这类域,正规扩张都是Galois扩张.

\begin{example}{}
$\mathbb{Q}(2^{1/3})/\mathbb{Q}$不是Galois扩域,因为它不正规,不包含$\opn{irr}(2^{1/3}, \mathbb{Q})(x)=x^3-2$的两个复数根.
\end{example}

\begin{theorem}{}
有限域都是其素域的Galois扩域.
\end{theorem}

\textbf{证明}:

有限域是其素域的有限扩张,而有限扩张都是代数扩张(\autoref{FldExp_cor1}~\upref{FldExp}).

由于有限域都是完美域(\autoref{SprbEx_cor4}~\upref{SprbEx}),故$\mathbb{Z}_p$的代数扩张都是可分扩张.

参照\textbf{有限域}\upref{FntFld}的讨论可知,有限域都是其素域的分裂域,从而是正规扩张.

\textbf{证毕}.


据正规扩张和可分扩张的知识,我们容易得到Galois扩张的几条性质:


\begin{theorem}{}
设$\mathbb{K}/\mathbb{M}/\mathbb{F}$是域扩张链,且$\mathbb{K}/\mathbb{F}$是Galois扩张,那么$\mathbb{K}/\mathbb{M}$也是Galois扩张.
\end{theorem}

证明用可分扩张的继承性\autoref{SprbEx_lem3}~\upref{SprbEx}和正规扩张的继承性\autoref{NomEx_the6}~\upref{NomEx}得到.

\begin{theorem}{}
设$\mathbb{K}/\mathbb{M}/\mathbb{F}$是域扩张链,其中$\mathbb{K}/\mathbb{F}$是Galois扩张,$\mathbb{M}/\mathbb{F}$是正规扩张,那么$\mathbb{M}/\mathbb{F}$是Galois扩张.
\end{theorem}

\textbf{证明}:

由\autoref{SprbEx_def4}~\upref{SprbEx}易得,$\mathbb{K}/\mathbb{F}$是可分扩张且$\mathbb{M}\subseteq\mathbb{K}$,可推得$\mathbb{M}/\mathbb{F}$也是可分扩张.

\textbf{证明}.

\begin{theorem}{}\label{GExt_the1}
如果$\mathbb{K}/\mathbb{F}$是Galois扩域,且域合成$\mathbb{EK}$存在,那么$\mathbb{EK}/\mathbb{EF}$是Galois扩域.
\end{theorem}

\textbf{证明}:

据\autoref{NomEx_the7}~\upref{NomEx},$\mathbb{EK}/\mathbb{EF}$是正规扩域.

考虑到$\mathbb{K}$的元素全都是$\mathbb{F}$的可分元素,从而是$\mathbb{EF}$的可分元素,而$\mathbb{EK}=\mathbb{EF}(\mathbb{K})$,可知$\mathbb{EK}/\mathbb{EF}$是可分扩张.

\textbf{证毕}.



\begin{theorem}{}
设$\mathbb{K}/\mathbb{F}$和$\mathbb{E}/\mathbb{F}$都是Galois扩域,且域合成$\mathbb{EK}$存在,则$\mathbb{EK}/\mathbb{F}$是Galois扩域.
\end{theorem}

\textbf{证明}:

已知\autoref{GExt_the1} 成立.

由$\mathbb{EK}/\mathbb{EF}$是正规扩张,及$\mathbb{F}[x]\subseteq\mathbb{EF}[x]$,知$\mathbb{EK}/\mathbb{F}$是正规扩张.

由\textbf{可分元素的封闭性}\autoref{SprbE2_cor4}~\upref{SprbE2}知$\mathbb{EK}/\mathbb{F}$是可分扩张.

\textbf{证毕}.




\begin{theorem}{}
域$\mathbb{K}$的Galois扩域之交,还是它的Galois扩域
\end{theorem}

\textbf{证明}:

由正规扩张和可分扩张相交还是正规扩张和可分扩张,得证.

\textbf{证毕}.



\subsection{Galois群}

回顾\autoref{Group_ex6}~\upref{Group},集合间的全体双射配合复合运算能构成群.既然域的自同构也是双射,我们也可以研究这些自同构构成的群.不过,相比于一般的域扩张,我们重点关注性质最良好的Galois扩张的情况.

\begin{definition}{Galois群}

给定Galois域扩张$\mathbb{K}/\mathbb{F}$,称$\mathbb{K}$上全体保$\mathbb{F}$自同构构成的群为该扩张的\textbf{Galois 群},记为$\opn{Gal}(\mathbb{K}/\mathbb{F})$,或$\opn{Gal}(\mathbb{K}:\mathbb{F})$.

\end{definition}


随便举一个具体的例子:复数域$\mathbb{C}$之于实数域$\mathbb{R}$是一个Galois扩域:正规性来自$\mathbb{C}=\overline{\mathbb{R}}$的事实,可分性是由于$\mathbb{R}$是完美域.那么$\opn{Gal}(\mathbb{C}/\mathbb{R})$是哪个群?或者最基础的问题,这个群有几个元素?

单回答这个问题也许不难,不过我们可以直接得出一般的Galois群元素数量规则:

\begin{theorem}{}\label{GExt_the2}
如果$\mathbb{K}/\mathbb{F}$是Galois扩域,那么
\begin{equation}
\abs{\opn{Gal}(\mathbb{K}/\mathbb{F})} = [\mathbb{K}:\mathbb{F}]
\end{equation}
\end{theorem}

\textbf{证明}:

当$[\mathbb{K}:\mathbb{F}]$有限时,正规扩张等价于分裂域,且由于可分扩张,因此适用\autoref{FldExp_the4}~\upref{FldExp}的等号情况,得证.

当$[\mathbb{K}:\mathbb{F}]$无限时,任取$n$个根在$\mathbb{K}-\mathbb{F}$中的多项式$f_i\in\mathbb{F}[x]$,得到$\prod f_i\in\mathbb{F}[x]$的分裂域$\mathbb{F}_1$,则$[\mathbb{K}:\mathbb{F}]>[\mathbb{F}_1:\mathbb{F}]>n$.$\mathbb{F}_1\mathbb{F}$适用有限情况,其Galois群的元素数量等于其扩张次数,而由于$[\mathbb{K}:\mathbb{F}]$无限,还能再取根在$\mathbb{K}-\mathbb{F}_1$中的多项式$g_i\in\mathbb{F}[x]\subseteq \mathbb{F}_1[x]$,构成更大的分裂域,得到更多自同构,因此必有
\begin{equation}
\opn{Gal}(\mathbb{K}/\mathbb{F})>\opn{Gal}(\mathbb{F}_1/\mathbb{F})=[\mathbb{F}_1:\mathbb{F}]>n
\end{equation}
由$n$的任意性,则可知$\opn{Gal}(\mathbb{K}/\mathbb{F})=\infty$.

\textbf{证毕}.



由于$\mathbb{C}$是$x^2+1\in\mathbb{R}[x]$的分裂域,故易证扩张次数为$2$,结合\autoref{GExt_the2} 就能确定$\opn{Gal}(\mathbb{C}/\mathbb{R})$只有两个元素.显然,除了恒等映射以外,求共轭映射也是一个保$\mathbb{R}$自同构,那这就已经找全了.




\begin{definition}{不变子域}
给定域$\mathbb{F}$,设$G$是$\mathbb{F}$的全体自同构群.

取$G$的子群$H$,则集合$\{a\in\mathbb{F}\mid \sigma(a)=a, \forall \sigma\in H\}$构成一个域,称为$\mathbb{F}$的$H$ \textbf{不变子域(fixed field of }$H$\textbf{)},或译作$H$ \textbf{固定子域},记为$\opn{Inv}_\mathbb{F}(G)$或$\opn{Fix}_\mathbb{F}(G)$.
\end{definition}

如果取$H=\opn{Gal}(\mathbb{C}/\mathbb{R})$,那么$\opn{Fix}_{\mathbb{C}}(H)=\mathbb{R}$.同样地,我们也可以直接给出更一般的情况:



\begin{theorem}{(Artin)}
给定域$\mathbb{F}$,$G$是它的全体自同构群的\textbf{有限}子群.

则$\mathbb{F}/\opn{Fix}_\mathbb{F}(G)$是Galois扩张,且$\opn{Gal}(\mathbb{F}/\opn{Fix}_\mathbb{F}(G))=G$.
\end{theorem}

\textbf{证明}:

任取$\alpha\in\mathbb{F}$.由于$G$是有限群,故其轨道是有限的,不妨记为$G\alpha=\{\alpha_i\}_{i=1}^n$,其中$\alpha_1=\alpha$,各$\alpha_i$彼此不等.

构造多项式$f_\alpha(x)=\prod_{i=1}^n(x-\alpha_i)$.对于任意$\sigma\in G$,都有$\sigma G=G$,因此$\sigma(\{\alpha_i\})=\{\alpha_i\}$,即$\sigma$是$\{\alpha_i\}$的一个置换.于是,$f_\alpha$的各系数都在$\sigma$下不变,即都是$\opn{Fix}_\mathbb{F}(G)$的元素.因此,$f_\alpha$是$\alpha$在$\opn{Fix}_\mathbb{F}(G)$上的零化多项式.

\textbf{证毕}.





