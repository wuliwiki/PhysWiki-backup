% 条件去噪扩散概率模型
% keys 扩散 条件 生成模型

\textbf{条件去噪扩散概率模型}(Conditional Denoising Diffusion Model)是去噪扩散概率模型的一个改进版本,通过将源图像信息纳入到模型当中,使得模型可以学习一个从源图像域到目标图像域的映射。

假设有一个图像转换过程,源图像域图像为$x$,目标图像域图像为$y$.

正向扩散过程的概率转移公式如下:
\begin{equation}
q({y_{1:T}}|{y_0}) = \prod\limits_{t = 1}^T {q({y_t}|{y_{t - 1},x})}
\end{equation}
\begin{equation}
q({y_t}|{y_{t - 1}}) = {\rm{N}}({y_t};\sqrt {1 - {\beta _t}} {y_{t - 1}},{\beta _t}I)
\end{equation}

反向去噪过程的概率转移公式如下:
\begin{equation}
{p_\theta }({y_{0:T}|x}) = p({y_T})\prod\limits_{t = 1}^T {{p_\theta }({y_{t - 1}}|{y_t,x})}
\end{equation}
\begin{equation}
% {p_\theta }({y_{t - 1}}|{y_t,x}) = N({y_{t - 1}};{\mu _\theta }({y_t},t),\sum\nolimits_\theta  {({y_t},t)} )
{p_\theta }({y_{t - 1}}|{y_t,x}) = N({y_{t - 1}};{\mu _\theta }({y_t},t),\sum\nolimits_\theta  {({y_t},t)} )
\end{equation}
其中,$y_0$为随机采样的高斯噪音,$y_0$~$q(y_0)$;$x$为源图像;$p_\theta$为参数化的概率转移规则,即为深度神经网络模型所表示。


参考文献:
\begin{enumerate}
\item J. Ho, A. Jain, and P. Abbeel, “Denoising Diffusion Probabilistic Models,” in Advances in Neural Information Processing Systems, 2020, vol. 33, pp. 6840–6851.
\item C. Saharia, J. Ho, W. Chan, T. Salimans, D. J. Fleet, and M. Norouzi, “Image Super-Resolution via Iterative Refinement,” IEEE Transactions on Pattern Analysis and Machine Intelligence, vol. 45, no. 4, pp. 4713–4726, 2023, doi: 10.1109/TPAMI.2022.3204461.
\end{enumerate}