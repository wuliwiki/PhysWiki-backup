% 一般线性变换的 Jordan(若尔当)标准形
% license Usr
% type Tutor


\begin{issues}
\issueDraft
\issueTODO Jordan块阶数和个数与线性变换的联系
\end{issues}
利用幂零变换的结论(\autoref{the_Jordan_1}~\upref{Jordan}),本节进一步证明,任意线性变换都能化为类似的Jordan形,并且Jordan形是唯一的。
\subsection{线性变换的Jordan形}
由\autoref{the_nullpl_1}~\upref{nullpl}可知,任意线性空间都可以分解为任意线性变换的不变子空间之直和,例如对于线性变换$A$,我们有:
\begin{equation}
V=\opn{ker}f(A)=\opn{ker}f_1(A)\oplus\opn{ker}f_2(A)...\oplus\opn{ker}f_m(A)~,
\end{equation}
其中$f(A)$是$A$的特征多项式,$f_i(A)$是特征值相关的互素项:$f_i(A)=(A-\lambda_i)^{k_i}$。为方便计,设$\opn{ker}f_i(A)=W_i,f_i(A)=B_i$,则$A=A|_{W_1}\oplus A|_{W_2}\oplus...\oplus A|_{W_n}=\bigoplus^n_{i=1}(B_i+\lambda_iI)$

由\autoref{the_Jordan_1}~\upref{Jordan}可知,幂零变换意味着在$W_i$的某组基上,$B_i$都是Jordan形矩阵,称对应的$A|_{W_i)$为一般线性变换的Jordan形。归纳此段讨论为下述定理:
\begin{theorem}{}
设$A$为域$\mathbb F$上$n$维线性空间$V$内的一个线性变换,其特征多项式的根$
\end{theorem}

\subsection{Jordan形的唯一性}
\subsection{Jordan矩阵的计算方法}