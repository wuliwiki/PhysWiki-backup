% 广义函数
% 泛函分析|广义函数|Fourier变换

\verb|写在前面:|除了函数空间理论,广义函数论和Fourier变换是研究现代偏微分方程的基础.这里从泛函分析的角度,介绍广义函数的概念.

\pentry{紧集、支集、$n$重指标、赋范空间、收敛性、连续线性泛函、对偶空间、局部可积函数}
在给出广义函数的定义之前,需要先引入$C_0^\infty(\Omega)$空间的收敛性概念.

\begin{definition}{($C_0^\infty(\Omega)$ 空间中的收敛性)}
设 $\varphi_i,\varphi\in C_0^\infty(\Omega),\,i=1,2,...$,  若满足以下两个条件:

$(1)$ 存在紧集 $K\subset\Omega$, 使得 supp $\varphi\subset K$, supp $\varphi_{j}\subset K,$
$i=1,2,...;$ 

$(2)$ 对任意的 $n$ 重指标 $\alpha$, 都有 ${\displaystyle \lim_{j\rightarrow\infty}{\displaystyle \sup_{x\in\Omega}\left|\partial^{\alpha}\varphi_{j}(x)-\partial^{\alpha}\varphi(x)\right|=0}}$,


则称\textbf{ \textbf{$\varphi_{j}$ 在 $C_{0}^{\infty}(\Omega)$ 中收敛于 $\varphi$}}, 记作 
$\varphi_{j}\rightarrow\varphi$ $(C_{0}^{\infty}(\Omega))$.

\end{definition}

有了上面收敛性的概念,广义函数即是$C_0^\infty(\Omega)$上的连续线性泛函.

\begin{definition}{(广义函数)}
$C_{0}^{\infty}(\Omega)$ 上的一个连续线性泛函称为一个广义函数, 换句话说, 若 $f:C_{0}^{\infty}(\Omega)\rightarrow\mathbb{R}$
满足

$(1)$ 连续性: 当 $\varphi_{j}\rightarrow\varphi(C_{0}^{\infty}(\Omega))$
时, 有 ${\displaystyle \lim_{j\rightarrow\infty}f(\varphi_{j})=f(\varphi)}$; 

$(2)$ 线性性: 当 $\psi,\varphi\in C_{0}^{\infty}(\Omega)$, $k\in\mathbb{R}$
时, 有 $f(k\psi+\varphi)=kf(\psi)+f(\varphi),$

则称 $f$ 是一个 \textbf{广义函数} (\textbf{generalized function,} 简称 \textbf{广函}) 或 \textbf{分布} (\textbf{distribution})

所有广义函数构成的全体构成一个线性空间, 称为 \textbf{广义函数空间} (简称 \textbf{广函空间}), 记作 $D'(\Omega)$. 

当 $f\in D'(\Omega)$ 而 $\varphi\in C^\infty_0(\Omega)$ 时, 有时也用符号 $\left \langle f,\varphi \right \rangle $ 来表示 $f(\varphi)$ , 这一点是与普通的对偶空间类似的.

\end{definition}

之所以将广函空间记作 $D'(\Omega)$, 是因为广义函数论的创始人 L.Schwartz 最先引入此概念时将具有上述收敛拓扑的
$C_{0}^{\infty}(\Omega)$ 空间记作 $D(\Omega)$, 而广函空间正好作为 $D(\Omega)$
的对偶空间出现, 因而记作 $D'(\Omega)$. 
\begin{example}{(Dirac $\delta$ 函数)
}

设 $\Omega\subset\mathbb{R}^{n}$ 开集, $x_{0}\in\Omega,$ 定义 $\delta_{x_{0}}:C_{0}^{\infty}(\Omega)\rightarrow\mathbb{R}$
如下: 
\[
\delta_{x_{0}}(\varphi)=\varphi(x_{0}),\quad\forall\varphi\in C_{0}^{\infty}(\Omega).
\]
易知 $\delta_{x_{0}}$ 是一个广义函数, 即 $\delta_{x_{0}}\in D'(\Omega)$, 称为
\textbf{Dirac $\delta$ 函数}. 
\end{example}

\begin{example}{(局部可积函数都是广义函数)}
$\Omega$ 上的每一个局部可积函数 $f$ 都可以看成是一个广义函数, 按以下意义: 
\[
\tilde{f}:C_{0}^{\infty}(\Omega)\rightarrow\mathbb{R},\quad\quad\varphi\mapsto{\displaystyle \int_{\Omega}f(x)\varphi(x)dx}.
\]
明显 $f\in D'(\Omega)$. 所以当 $f$ 是局部可积函数, 我们总是可以将 $f$ 与 $\tilde{f}$
视作等同. 

$\Omega$ 上的所有局部可积函数组成的集合记为 $L_{\mathrm{loc}}^1(\Omega)$, 所以
$L_{\mathrm{loc}}^1(\Omega)\subset D'(\Omega).$
\end{example}

未完待续