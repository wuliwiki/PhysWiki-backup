% 北京科技大学 2006 年考研普通物理 A 卷
% keys 北京科技大学|考研|物理
% license Copy
% type Tutor

\textbf{声明}:“改内容来源于网络公开资料,不保证真实性,如有侵权请联系管理员”



适用专业:凝聚态物理、物理电子学
\subsection{选择题}
\begin{enumerate}
\item 质点作半径为R的变速圆周运动的加速度大小为(v表示任一时刻质点的速率)\\
(A) $\displaystyle \dv{v}{t}$\\
(B) $\displaystyle \frac{v^2}{R}$\\
(C) $\displaystyle \dv{v}{t}+\frac{v^2}{R}$\\
(D) $\displaystyle [(\dv{v}{t})+\frac{v^4}{R^2}]^\frac{1}{2}$
\item 一质点在如图1所示的坐标平面内作圆周运动,有一力$\vec F=F_0(x\vec i+y \vec j)$作用在质点上,在该质点从坐标原点运动到$(0,2R)$位置过程中,此力$\vec F$下对它作的功为:\\
\begin{figure}[ht]
\centering
\includegraphics[width=8cm]{./figures/da3765657468b7ac.png}
\caption{} \label{fig_BKDP06_13}
\end{figure}
(A) $F_0R^2S$\\
(B) $2F_0R^2$\\
(C) $3F_0R^2$\\
(D) $4F_0R^2$
\item 若一倾角为$\theta$ 的斜面上放一质量为m的物体,m与斜面间的摩擦系数为 $\mu$,斜面向左加速运动,欲使m沿斜面向上滑动,则斜面的加速度值至少应为:\\
\begin{figure}[ht]
\centering
\includegraphics[width=8cm]{./figures/f42b7eca01b6ce64.png}
\caption{} \label{fig_BKDP06_12}
\end{figure}
(A) $\displaystyle \frac{(\mu \cos \theta+\sin \theta)g}{\cos 2\theta -\mu \sin \theta}$\\
(B) $\displaystyle\frac{(\mu \sin \theta+\cos \theta)g}{\cos 2\theta -\mu \sin \theta}$\\
(C) $\displaystyle \frac{(\mu \cos \theta+\sin \theta)g}{\sin \theta -\mu \cos\theta}$\\
(D)$\displaystyle \frac{(\mu \cos \theta+\sin \theta)g}{\cos \theta -\mu \sin\theta}$
\item 如图所示,两个同心球壳。内球半径为 $R_1$ ,均带有电荷 $Q$ ;外求壳半径为 $R_2$ ,壳的厚度忽略,原先不带电,但与地相连接。设地为电势零点,则在两球之间,距离球心为 $r$ 的 $P$ 点处电场强度的大小与电势分别为:\\
\begin{figure}[ht]
\centering
\includegraphics[width=6cm]{./figures/be70e5d8ab907a9b.png}
\caption{} \label{fig_BKDP06_11}
\end{figure}
(A) $\displaystyle E=\frac{Q}{4\pi\varepsilon_0 r^2},U=\frac{Q}{4\pi\varepsilon_0 r}$\\
(B) $\displaystyle E=\frac{Q}{4\pi\varepsilon_0 r^2},U=\frac{Q}{4\pi\varepsilon_0} (\frac{1}{R_1}-\frac{1}{r})$\\
(C) $\displaystyle E=\frac{Q}{4\pi\varepsilon_0 r^2},U=\frac{Q}{4\pi\varepsilon_0} (\frac{1}{r}-\frac{1}{R_2}) $\\
(D)$\displaystyle E=0,\qquad U=\frac{Q}{4 \pi \varepsilon_0 R_2}$
\item 无限长直导线在 $P$ 处弯成半径为 $R$ 的圆,当通以电流 $I$ 时,则在圆心  $O$ 点的磁场感应强度大小等于\\
\begin{figure}[ht]
\centering
\includegraphics[width=8cm]{./figures/0f5390c058664399.png}
\caption{} \label{fig_BKDP06_14}
\end{figure}
(A) $\displaystyle \frac{\mu_0 I}{2 \pi R}$\\
(B) $\displaystyle \frac{\mu_0 I}{4R}$\\
(C) $0$\\
(D) $\displaystyle \frac{\mu_0 I}{2R}(1-\frac{1}{\pi})$
\item 已知一定量的某种理想气体,在温度为 $T_1$ 和 $T_2$ 时的分子最概然速率分别为 $V_{p1}$ 和 $V_{p2}$ ,分子速率分布的最大值分别为 $f(V_{p1})$ 和 $f(V_{p2})$ 。若 $T_1>T_2$ ,则\\
(A) $V_{p1}>V_{p2},f(V_{p1}>f(V_{p2})$\\
(B)$V_{p1}>V_{p2},f(V_{p1}<f(V_{p2})$\\
(C)$V_{p1}<V_{p2},f(V_{p1}>f(V_{p2})$\\
(D)$V_{p1}<V_{p2},f(V_{p1}<f(V_{p2})$\\
\subsection{填空题}
\item 一质量为$M$的物体沿$x$轴止向运动,假设该质点在通过坐标为$x$的位置时速度的大小为$kx$($k$为正值常量),则此时作用于该质点上力$F=()$,该质点从$x=x_0$点出发运动到$x=x_1$处所经历的时间$\Delta t=()$\\
\item 把一个均匀带有电荷$+Q$的球形肥皂泡由半径$r_1$吹胀到$r_2$,则半径为$R(r_1<R<r_2)$的任一点的场强大小$E$由$(\qquad)$变为$(\qquad)$;电势$U$由$(\qquad)$。(选无穷远处为电势零点)。
\item 如图所示,一质量为$m$的滑块,两边分别与劲度系数为$kl$和$k2$的轻弹簧相连,两弹簧的另外两端分别规定在墙上。滑块$m$可在光滑的水平面上滑动,$0$点为系统的平衡位置,将滑块向右移动到$x0$,自静止释放,并从释放时开始计时。取坐标如图所示,则其振动方程为:$(\qquad)$
\begin{figure}[ht]
\centering
\includegraphics[width=8cm]{./figures/7afb488b83c2f8f6.png}
\caption{} \label{fig_BKDP06_10}
\end{figure}
\item 在弦线上有一简谐波,其表达式为:$y_1=2.0*10^{-2} \cos[100\pi(t+\frac{x}{20})-\frac{4}{3}\pi]$,为了在此弦线上形成驻波,并且$x=0$处为一波腹,此弦线上还应有一简谐波,其表达式为:$(\qquad)$
\item 右图为一理想气体几种状态变化过程的$p-V$图,其中$MT$为等温线,$MQ$为绝热线,在$AM,BM,CM$三种准静态过程中:\\
\begin{figure}[ht]
\centering
\includegraphics[width=8cm]{./figures/c21aceaaa9c34d61.png}
\caption{} \label{fig_BKDP06_9}
\end{figure}
温度升高的是$(\qquad)$过程,\\
气体吸热的是$(\qquad)$过程。
\item $1,2$是两个完全相同的空气电容器,将其充电后与电源断开, 再将一块各向同性均匀电介质板插入电容器$1$的两极板间,如图所示。则电容器$2$的电压$U_2$,电场能量$W_2$如何变化?(填增大,减小或不变)\\
\begin{figure}[ht]
\centering
\includegraphics[width=8cm]{./figures/4fc8b260dc5d0b9d.png}
\caption{} \label{fig_BKDP06_8}
\end{figure}
$U_2(\qquad)$\\
$W_2(\qquad)$
\item 光波的干涉是指 $(\qquad)$,相干条件为$(\qquad)$。
\item  用平行的白光垂入射在平面透射光栅上时,波长为$\lambda_1=440nm $的第三级光谱线将与波长为$\lambda_2=(\quad)nm$的第二级光谱线重叠$(1nm=10^{-9}m)$。
\item 一束自然光垂直穿过两个偏振片,两个偏振片的偏振化方向成 $45$°角,已知通过此两偏振片后的光强为 $1$ ,则入射至第二个偏振片的线偏振光强度为$(\qquad)$。
\subsection{计算题}
\item (此题单考生不做!)如图所示,$A$和$B$两飞轮的轴杆在同一中心线上,设两轮的转动惯量分别为$J=10kg.m^2$和$J=20kg.m^2$。开始时,$A$ 轮的转速为 $600rev/min$,$B$轮静止,$C$为摩擦齿和器,其转动惯量可忽略不计。$A,B$分别与$C$的左右两个两个组件相连,当$C$的左右组件齿合时,$B$轮得到加速而$A$轮减速,直到两轮的转速相等为止。设轴光滑,求:\\
\begin{figure}[ht]
\centering
\includegraphics[width=8cm]{./figures/093d558081f403e6.png}
\caption{} \label{fig_BKDP06_7}
\end{figure}
(1)两轮齿合后的转速$n$;\\
(2)两轮各自所受的冲量矩。
\item (此题统考生不做!)一轴承光滑的定滑轮,质量为$M=2.00kg$,半径为$ R=0.10m$。一根不能伸长的轻绳。一端固定在定滑轮上,另一端系有一质量为$m=5.00kg $的物体,如图所示。已知定滑轮的转动的初角速度 $\omega_0=1.00 rad/s$,方向垂直纸面向里。求:\\
\begin{figure}[ht]
\centering
\includegraphics[width=6cm]{./figures/557634a374bc67d7.png}
\caption{} \label{fig_BKDP06_6}
\end{figure}
(1)定滑轮角加速度的大小和方向;\\
(2)定滑轮的角速度变化到$\omega=0$时,物体上升的高度;\\
(3)当物体回到原来位置时,定滑轮的角速度的大小和方向。
\item $1 mol $的氦气作如图所示的可逆循环,其中$ ab $和 $cd$ 是绝热过程,$bc$ 和 $da$ 为等容过程,已知$V_1=16.4L,V_2=32.8L,p_a=l atm,p_b=3.18 atm,p_c=4 atm. p_d=1.26 atm $试求:\\
\begin{figure}[ht]
\centering
\includegraphics[width=8cm]{./figures/c202594a85756b51.png}
\caption{} \label{fig_BKDP06_5}
\end{figure}
(1)在各态氨气的温度;\\
(2)在各态氨气的内能;\\
(3)在一循环过程中氨气所做的净功。
\item (此题单考生不做!)如图所示,半径为$R$的均匀带电球面,带有电荷$q$。沿某一半径方向上有均匀带电细线,电荷线密度为$\lambda$,长度为$1$.细线左端离球心距离为$r_0$。设球和线上的电荷分布不受相互作用影响。试求细线所受球面由荷的电场力和合细线在该电场中的电势能(设无穷远处的电势为零)。
\begin{figure}[ht]
\centering
\includegraphics[width=8cm]{./figures/090bb3cf78b45e82.png}
\caption{} \label{fig_BKDP06_4}
\end{figure}
\item (此题统考生不做!)图示一球形电容器,在外球壳的半径$b$和内外导体间的电势差$U$维持条件下,求:\\
\begin{figure}[ht]
\centering
\includegraphics[width=6cm]{./figures/250e60260f177312.png}
\caption{} \label{fig_BKDP06_3}
\end{figure}
(1)两球壳之间的电场强度分布;\\
(2)内球半径$a$为多大时才能使内球表面附近的电场强度为最小?求这个最小电场强度的大小。
\item 载流长直导线与矩形回路$ABCD$共面,导线平行于$ AB$,如图所示。求下列情况下$ABCD$ 中的感应电动势:\\
\begin{figure}[ht]
\centering
\includegraphics[width=8cm]{./figures/19ad4f3910057941.png}
\caption{} \label{fig_BKDP06_2}
\end{figure}
(1)长直导线中电流$I=I_0$不变,$ABCD $以垂直于导线的速度$v$远离导线匀速平移,当移动到图示位置时;\\
(2)长直导线中电流$I=I_0\sin \omega t$,$ABCD$ 不动;\\
(3)长直导线中电流$I=I_0\sin \omega t$,$ABCD$以垂直于导线的速度$v$远离导线匀速运动,位置也如图。
\item 用波长为$ 500nm(lnm=10^{-9}m)$的单色光垂直照射到由两块光学平玻璃构成的空气劈形膜上,在观察度射光的干涉现象中,距劈形模棱边$l=1.56cm$的$A$处是从棱边算起的第四条暗条纹中心。\\
(1)求此劈形膜的劈尖角$\theta$;\\
(2)改用$600nm$的单色光垂直照射到此劈尖上仍观察反射光的干涉条纹,$A$处是明条纹还是暗条纹?\\(3)在第(2)问的情形从棱边到A处的范围共有几条明纹?几条暗纹?
\end{enumerate}