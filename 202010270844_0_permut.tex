% 排列
% keys 排列|集合|序列|阶乘

\begin{issues}
\issueDraft
\end{issues}

\pentry{集合\upref{Set}}

我们讨论含有 $N$ 个元素的任意集合, 由于集合中元素的名称不重要, 我们以下将它记为 $\qty{1,2,\dots, N}$. 注意集合的是没有顺序的, 例如 $\qty{1,2,3}$ 和 $\qty{1,3,2}$ 是同一个集合. 当我们把集合 $S$ 中的的元素按照某种顺序排列成一个序列时, 就称为它是集合 $S$ 的一种\textbf{排列(permutation)}.

那么 $N$ 个元素的集合一共有几种不同的排列呢? 第 1 个位置有 $N$ 种不同的可能, 确定之后第 2 个位置有 $N-1$ 种不同的可能, 第 3 个位置有 $N-2$ 种…… 最后一个位置只有 1 种. 所以可能性的种数可以用阶乘\upref{factor} 表示为
\begin{equation}
N! = N(N-1)(N-2)\dots 1
\end{equation}

我们可以把第 $i$ 种排列记为 $p_i$, 该排列的元素按照顺序分别记为 $p_{i,1}, p_{i,2}, \dots, p_{i,N}$.
