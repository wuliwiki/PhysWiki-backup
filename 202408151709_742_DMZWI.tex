% 兹威基关于暗物质的工作
% license Usr
% type Tutor


一般认为暗物质是由兹威基(Fritz Zwicky)在1933年提出的。然而,那时候,暗物质的概念和术语在天文学界已经存在一段时间了,例如在H. Poincaré、W. Thomson(即Lord Kelvin)、E. Öpik、J. Kapteyn、K. Lundmark和J. Oort以及S. Smith的工作中。然而,兹威基的工作在历史上很重要,而且特别简单,所以值得我们hao'hao。通过观察Coma星系团的速度分散,Zwicky发现需要额外的物质来保持星系团的凝聚力。星系团是宇宙中最大的引力束缚系统。它们包含数百到数千个星系,延伸到数Mpc的大小(见图3.1)。由于它们的规模,星系团是探测“平均”宇宙的好探针。虽然目前最精确的暗物质密度测量不是来自星系团,但它们确实导致了ΩDM ≈ 0.2。Zwicky的确定是基于维里尔定理。维里尔定理将平均动能与平均势能联系起来,⟨K⟩ = −1/2⟨V⟩对于引力。在一个玩具系统中,有N ≫ 1个质量为m的物体在相等的距离r上通过引力相互作用,这允许从速度v和大小R来确定它们的总质量mN:

N mv2 / 2 = 1/2 N2 Gm2 / R ⇒ mN = 2Rv2 / G  

将这种考虑应用于Coma星系团,Zwicky认为星系团的总质量大于可见质量,因此需要额外的暗物质。暗物质的假设并没有被广泛接受,但也没有被忽视。一个常见的解释是需要更多的信息才能理解这些系统。

自20世纪80年代以来,X射线观测已成为评估星系团中普通和暗物质数量的更有效方法(见[14]的综述)。星系团包含大量的电离氢和氦。当这些气体坍缩到星系团的势阱中时,它会发生冲击和绝热压缩,加热,并达到温度Tgas ∼ mpv2 esc ∼ 10 keV ∼ 10^8 K,这将是核反应的典型温度。然而,由于环境密度低,核聚变可以忽略不计。然后,气体主要通过热轫致辐射发出X射线。假设球对称性和流体静力平衡,可以写出将气体压力梯度℘gas与引力势梯度ϕ联系起来的平衡方程:

ρgas(r) d℘gas/dr = −dϕ/dr = −GM(r) / r^2~,  

其中ρgas(r)是气体密度,M(r)是半径r内总的引力质量(即,氢/氦气体和暗物质)。假设理想气体,压力和密度通过℘gas = ρgaskTgas/µ

mp相关联,其中µ ≈ 0.6是由约75\%的氢和约25\%的氦混合而成的平均分子量。气体的密度ρgas和温度Tgas可以从X射线发射的强度和光谱中测量出来,从而允许使用方程(1.4)重建星系团的总质量和其分布。结果证实,气体只占星系团总质量的一部分。在某种程度上,使用X射线发射是一种应用Zwicky方法到微观尺度:气体分子的动能(即它们的温度)取代了星系的动能,然后从中推断出保持星系团引力束缚的总引力质量。X射线测量在分析独特的空间配置方面也非常有用,例如星系团的碰撞,我们将在接下来讨论。

6Zwicky使用了一个与方程(1.3)不同的玩具模型来近似星系团:一个恒定密度ρ和半径R的球体。结果的差异是一个数量级因子,平均势能现在由−⟨V⟩ = ∫ R 0 Gρ 4πr^2drM(r)/r = 3GM^2/5R给出,其中M(r)是r内包含的质量,M是总质量。7如今我们测量的是,在典型的星系团中,星系中的恒星质量占总质量的1-2\%,星系间的气体占5-15\%。其余的都是缺失的,并被解释为暗物质。