% 经典力学(综述)
% license CCBYSA3
% type Wiki

本文根据 CC-BY-SA 协议转载翻译自维基百科\href{https://en.wikipedia.org/wiki/Classical_mechanics}{相关文章}。

\begin{figure}[ht]
\centering
\includegraphics[width=8cm]{./figures/c0ac5b53deabd6dc.png}
\caption{显示卫星绕地球轨道运动的示意图,其中展示了垂直的速度和加速度(力)矢量,通过经典力学的解释进行表示。} \label{fig_JDLX_1}
\end{figure}
经典力学是一种物理理论,用于描述物体的运动,例如抛射物、机械部件、航天器、行星、恒星和星系等。经典力学的发展涉及物理学方法和哲学的重大变革。[1] “经典”这一限定词将这种力学类型与20世纪初物理学革命之后发展起来的物理学区分开来,这些现代物理理论揭示了经典力学的局限性。[2]

经典力学最早的形式通常被称为牛顿力学。它基于17世纪以艾萨克·牛顿爵士为代表的奠基性工作的物理概念,以及牛顿、戈特弗里德·威廉·莱布尼茨、莱昂哈德·欧拉等人发明的数学方法,用来描述物体在力的作用下的运动。后来,基于能量的方法由欧拉、约瑟夫-路易·拉格朗日、威廉·罗恩·哈密顿等人发展,最终形成了分析力学(包括拉格朗日力学和哈密顿力学)。这些进步主要发生在18世纪和19世纪,超越了早期的工作;经过一定的修正,它们被应用于现代物理学的各个领域。

如果一个遵循经典力学定律的物体的当前状态已知,就可以确定它将来的运动方式以及过去的运动轨迹。然而,混沌理论表明,经典力学对长期预测并不可靠。在研究非极端质量且速度未接近光速的物体时,经典力学能够提供精确的结果。当研究的对象接近原子直径大小时,就需要使用量子力学;而描述接近光速的速度时,则需要使用狭义相对论。当物体质量极大时,广义相对论则变得适用。一些现代文献将相对论力学归入经典物理学范畴,认为它是该领域最成熟和精确的形式。