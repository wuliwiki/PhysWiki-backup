% A-star 算法
% 搜索|BFS|C++

A-star 算法的使用场景和双向 BFS 的差不多,都是如果在一个搜索空间非常大的情况下,可能会遍历非常多无需遍历的状态,导致时间效率非常低。由此可以使用 A* 算法。

A* 算法是在优先队列 BFS 的基础上进行优化的,我们新加了一个\textbf{启发函数}的概念,这样就可以优化搜索空间,降低时间复杂度。这个启发函数一般在 A* 算法上是设计了一个\textbf{估价函数},在普通的优先队列 BFS 中,每次只会弹出距离当前点比较小点的临点进行更新,不会考虑未来怎么更新,有可能存在一条距离当前点权值比较大的点,但到未来的点的代价可能很小的点。

所以可以添加一个可以对未来的代价进行预估的\textbf{估价函数},具体地讲,在求最短路的时候,可以存:从起点走到当前点的真实距离,以及从当前点走到终点的估计距离这两个值,在优先队列中使用“当前距离+预估距离”进行扩展,实际意义为当前这条路径到终点的距离。这里的优先队列为小根堆。

A* 算法算法的框架:\begin{lstlisting}[language=cpp]
while (!q.empty())
{
    t <--- 取出优先队列(小根堆)的队头
    当终点第一次出队时,找到了答案,退出循环

    for (枚举 t 的所有出边)
        扩展、将临边入队
}
\end{lstlisting}

当估计距离为 $0$ 时,A* 算法算法变为 Dijkstra。

使用 A* 算法的前提:

设当前状态为 $\texttt{state}$,从起点到当前点的实际距离为 $\texttt{d(state)}$。

\begin{itemize}
\item 从起点到终点的实际距离为 $\texttt{g(state)}$,从起点到终点的估价距离为 $\texttt{f(state)}$。
\item 需要满足 $\texttt{f(state)} \leq \texttt{g(state)}$。
\end{itemize}

