% 曲线运动(高中)
% 曲线运动|抛体运动|平抛|斜抛

\begin{issues}
\issueDraft
\issueTODO
\end{issues}

\subsection{曲线运动}

定义:轨迹是曲线的运动.

条件:当物体所受的合外力(加速度)与其速度方向不在同一直线上时,物体做曲线运动.

物体所受合外力方向(加速度方向):总是指向曲线的凹侧.

速度方向:质点在某一位置的速度方向与曲线在这一点的切线方向一致.

\subsection{运动的合成与分解}

如果物体同时参与了几个运动,物体实际发生的运动叫做那几个运动的\textbf{合运动},那几个运动就叫做这个实际运动的\textbf{分运动}.事实上,任何物体在确定的参考系中只能实际参与一个运动,分运动是为了便于分析而对实际运动进行的等效替代,并非真实存在.

与力的合成与分解类似,由分运动求跟合运动的过程叫做\textbf{运动的合成},由合运动求分运动的过程叫做\textbf{运动的分解}.在描述运动时,我们用到了位移、速度和加速度这些量,因此在进行运动的合成与分解时,实际上就是对位移或速度或加速度的合成与分解.由于位移、速度和加速度都是矢量,因此运动的合成与分解都遵循平行四边形定则.

\subsection{抛体运动}

\subsubsection{平抛运动}

\subsubsection{斜抛运动}
