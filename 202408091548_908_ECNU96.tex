% 华东师范大学 1996 年 考研 量子力学
% license Usr
% type Note

\textbf{声明}:“该内容来源于网络公开资料,不保证真实性,如有侵权请联系管理员”

(1)量子力学和经典力学对于粒子系统的描述有何不同?为什么?

(2)量子力学中的一个重要特点是,需要通过力学量算符和波函数的适当组合运算才能求出实验上的可测量。试举出三个例子。(应写出必要的公式)

(3)证明:在有心力场 $V(\vec r)$ 中微观粒子的角动量平方 $ L^2 $ 和角动量的任一分量 $L_i (i=1,2,3)$都是守恒量。

(4)一维无限深方势井
$$V(x) = 
\begin{cases} 
0 & 0 < x \leq a \\
\infty & x < 0, x > a 
\end{cases}~$$
中,微观粒子能量的本征值为
$$E_n = \frac{h^2 \pi^2 n^2}{2ma^2} \quad (n = 1, 2, 3, \ldots)~$$
相应的本征函数为
$$\psi_n(x) = \sqrt{\frac{2}{a}} \sin \left( \frac{n \pi }{a} x \right)~$$
现设某一粒子在该势井中的运动状态由
$$\psi(x) = \frac{4}{\sqrt{a}} \sin \left( \frac{\pi x}{a} \right) \cos^2 \left( \frac{\pi x}{a} \right)~$$
描述,求该粒子的能量可能值和相应的几率。

(5)描述电子自旋的角动量沿直角坐标系各轴方向的分量为 $\sigma_i (i=x,y,z)$ :

$$\vec{n} = (\sin \theta \cos \phi, \sin \theta \sin \phi, \cos \theta)~$$

的分量为 $\vec{\sigma} \cdot \vec{n}$ :

1. 写出这些分量算符 $\sigma_i$和$\vec{\sigma} \cdot \vec{n}$在 $\sigma_z$表象的短形式式。

2. 分别求出它们的本征值和本征函数。

3. 在 $\vec{\sigma} \cdot \vec{n}$ 的属于本征值 $+1$ 的本征函数所描述的状态中,出现 $\sigma_z = +1$ 或$-1$状态的几率多少?出现 $\sigma_x = -1$ 或$-1$状态的几率多少?

(6)利用Schrodinger 议程证明,动量算符在任意状态 $\psi(x,t)$ 中的平均值
$\langle \vec{p} \rangle = \int \psi^* p\vec{\psi}\, dx $
和势能的平均值
$\langle V \rangle = \int \psi^* V(x) \psi \, dx $
之间存在如下关系(称为Ehrenfest 定理):
$ \frac{d}{dt} \langle \vec{p} \rangle = \langle -\nabla V \rangle $
注意到势能的负梯度 $ - \nabla V $ 即为势场中该粒子所受到的力,上式与经典力学中的牛顿第二定律相对应。