% 克里斯蒂安·哥德巴赫(综述)
% license CCBYSA3
% type Wiki

本文根据 CC-BY-SA 协议转载翻译自维基百科\href{https://en.wikipedia.org/wiki/Christian_Goldbach}{相关文章}。

克里斯蒂安·哥德巴赫(/ˈɡoʊldbɑːk/ GOHLD-bahk,德语:[ˈkʁɪsti̯a(ː)n ˈɡɔltbax];1690年3月18日 – 1764年11月20日)是一位普鲁士数学家,参与了若干重要的研究,主要集中在数论领域;他还学习过法律,并对俄罗斯宫廷产生了兴趣并参与其中。[1][2] 在他早期的欧洲旅行之后,他于1725年以圣彼得堡科学院的教授身份来到俄罗斯。[3] 哥德巴赫于1737年共同领导了科学院。[4] 然而,他在1742年放弃了科学院的职务,并在俄罗斯外交部工作,直至1764年去世。[4] 他今天因哥德巴赫猜想和哥德巴赫–欧拉定理而被铭记。[1] 他与著名数学家莱昂哈德·欧拉有着深厚的友谊,并成为欧拉数学追求的灵感来源。[2]
