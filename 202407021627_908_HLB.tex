% 核裂变
% license CCBYSA3
% type Wiki

(本文根据 CC-BY-SA 协议转载自原搜狗科学百科对英文维基百科的翻译)

在核物理和核化学中,\textbf{核裂变}是指核反应或放射性衰变过程中原子核分裂成更小、更轻的核的现象。裂变过程通常产生自由中子和γ 光子,同时释放出大量的能量。

重元素核裂变于1938年12月17日由德国人奥托·哈恩和他的助手弗里茨·施特拉斯曼发现,1939年1月莉泽·迈特纳和她的侄子奥托·弗里施给出了理论解释。弗里希将这一过程比喻为生物细胞的分裂。重核素的裂变是一个放热反应,会以电磁辐射和裂变碎片的动能形式释放大量能量。为了使裂变过程释放能量,裂变产物元素的总结合能应该大于起始元素的结合能。

裂变是一种核嬗变,因为裂变产物与初始原子属于不同的元素。裂变产生的两个核的质量通常比较接近,对于一般的可裂变同位素,其裂变产物的质量比约为3比2。[1][2]大多数裂变是二元裂变(产生两个带电碎片),但偶尔(每1000次事件发生2到4次)会发生三元裂变,产生三个带正电荷的碎片。三元裂变过程中最小的裂变碎片大小可位于质子到氩核之间。

除了已经被人类开发利用的中子诱导裂变之外,还存在另一种形式的裂变,称为自发放射性衰变(不需要中子诱导),该现象易发生在具有较高质量数的同位素中。自发裂变于1940年由弗廖罗夫、彼得扎克和库尔恰托夫[3]在莫斯科发现,当时他们决定通过实验验证尼尔斯·玻尔作出的一个预测,即没有中子轰击时,铀几乎不发生裂变,然而实验结论正相反。[3]

产物组成的不可预测性(产物可能的种类很多且无规律性)将裂变与量子隧穿过程区分开来,如质子发射、α衰变和团簇衰变等量子隧穿过程每次的产物都是相同的。核裂变是核电站及核武器的能量来源。作为核燃料的物质在被裂变中子撞击时会发生裂变,而在它们裂变过程中又会发射中子。这使得自持的核链式反应成为可能,这种核链式反应可以在核反应堆中以受控的速率释放能量,或者在核武器中以非常快速、不受控制的速率释放能量。

核燃料中包含的自由能是同等质量的化学燃料(如汽油)的数百万倍,这使得核裂变成为一种非常高效的能源。然而,核裂变的产物的放射性通常比作为裂变燃料的重元素高得多,并且其半衰期相当长,导致了核废料的问题。对核废料积聚和核武器的潜在破坏力的担忧影响了人们和平利用裂变作为能源的愿望。