% 纯不可分扩张
% 可分扩张|域|扩域|代数扩张|可分次数|separable degree

\pentry{可分扩张\upref{SprbEx}}
%GTM 242 V.6


\subsection{纯不可分的概念}

我们容易想到并且熟悉的扩域,通常是可分扩张.相对地,凡不是可分扩张的情况,都叫做不可分扩张——也就是说,不可分扩张是可分元、不可分元混杂的扩域.这样一来,代数扩张就分为两类了.似乎不可分扩张只是可分扩张的例外,不像可分扩张有那么好的性质;可相对的,下面定义的纯不可分扩张就很有意思了.

\begin{definition}{纯不可分扩张}\label{PInsEx_def2}

设$\mathbb{K}/\mathbb{F}$是代数扩域.如果$\mathbb{K}-\mathbb{F}$中的元素在$\mathbb{F}$上\textbf{都是不可分的},那么称$\mathbb{K}/\mathbb{F}$是\textbf{纯不可分扩张(purely inseparable extension)},且称$\mathbb{K}$在$\mathbb{F}$上(over $\mathbb{F}$)是\textbf{纯不可分}的.

\end{definition}

任何域$\mathbb{F}$都是自己的纯不可分扩张,因为不存在$\mathbb{F}-\mathbb{F}$的元素,命题前提为假则命题恒真.

留意\autoref{SprbEx_the4}~\upref{SprbEx},可以猜测可分性质与$p$次根是否在给定域中息息相关,其中$p$是这个域的特征.由此提示,我们可以定义“纯不可分元素”:


\begin{definition}{纯不可分元素}\label{PInsEx_def1}
设域$\mathbb{F}$的特征为素数$p$.称$\alpha\in\overline{\mathbb{F}}$是$\mathbb{F}$上的\textbf{纯不可分元素(purely inseparable element)},如果存在非负整数$k$使得$\alpha^{p^k}\in\mathbb{F}$.

或者说,$\alpha$是$\mathbb{F}$的代数元,且其最小多项式为$x^{p^n}-a$,其中$n$是上述$k$中最小的,$a\in\mathbb{F}$.
\end{definition}

同样,域$\mathbb{F}$的元素都是纯不可分元素.



一切代数扩张,必是进行一次可分扩张后再进行一次纯不可分扩张的结果.要注意,反过来的“代数扩张必是先进行纯不可分扩张再进行可分扩张的结果”并不总是成立.
\addTODO{是否需要举一个反例,说明为什么并不总是成立?}




\begin{theorem}{}\label{PInsEx_the3}
给定代数扩域$\mathbb{K}/\mathbb{F}$,则集合$\mathbb{S}=\{\alpha\in\mathbb{K}|\alpha\text{在}\mathbb{F}\text{上可分}\}$是一个域.

且$\mathbb{K}/\mathbb{S}$是一个纯不可分扩张.
\end{theorem}

\textbf{证明}:

由\textbf{可分元素的封闭性}(\autoref{SprbE2_cor4}~\upref{SprbE2}),可知$\mathbb{S}$是$\mathbb{F}$的可分闭包与$\mathbb{K}$的交集,显然是个域.

反设$\mathbb{K}-\mathbb{S}$中存在$\mathbb{S}$的可分元素$a$,那么$\mathbb{S}(a)/\mathbb{S}$是可分扩张.由可分扩张的传递性(\autoref{SprbE2_cor1}~\upref{SprbE2}),可知$\mathbb{S}(a)/\mathbb{F}$是可分扩张,这与可分闭包的定义矛盾.因此反设不成立,即$\mathbb{K}-\mathbb{S}$中全是$\mathbb{S}$的不可分元素.

\textbf{证毕}.

该定理还引出了一个重要的概念:

\begin{definition}{可分次数}
给定代数扩域$\mathbb{K}/\mathbb{F}$,构造中间域$\mathbb{K}_S=\{\alpha\in\mathbb{K}|\alpha\text{在}\mathbb{F}\text{上可分}\}$.

则称$[\mathbb{K}_S:\mathbb{F}]$是$\mathbb{K}/\mathbb{F}$的\textbf{可分次数(separable degree)},记为$[\mathbb{K}:\mathbb{F}]_S$.

称$[\mathbb{K}:\mathbb{F}] / [\mathbb{K}:\mathbb{F}]_S$为$\mathbb{K}/\mathbb{F}$的\textbf{不可分次数(inseparable degree)}.
\end{definition}






由于特征为$0$的域都是\textbf{完美域}(定义见\autoref{SprbEx_the5}~\upref{SprbEx}),不存在不可分元素,因此纯不可分扩张的讨论集中在特征为素数$p$的域上.

考虑域$\mathbb{F}$,$\opn{ch}\mathbb{F}=p$.对于任意$\alpha, \beta\in\overline{\mathbb{F}}$和$k\in\mathbb{Z}^+$,都有$\alpha^{p^k}-\beta^{p^k}=(\alpha-\beta)^{p^k}$,其中$\overline{\mathbb{F}}$是$\mathbb{F}$的代数闭包.于是,任意$a\in\mathbb{F}$在其代数闭包里有唯一的$p^k$次根$\alpha$,因为
\begin{equation}
x^{p^k}-a=x^{p^k}-\alpha^{p^k}=(x-\alpha)^{p^k}
\end{equation}
显然这是$\alpha$在$\mathbb{F}$上的最小多项式,以及该最小多项式在$\overline{\mathbb{F}}$上的分解.

% 由于\autoref{SprbEx_the4}~\upref{SprbEx},$\mathbb{F}$是完美域当且仅当任意元素$a\in\mathbb{F}$在$\mathbb{F}$中都有$p$次方根.

受此启发,我们可以得到一个纯不可分扩张的例子:

\begin{example}{}\label{PInsEx_ex1}
设域$\mathbb{F}$的特征为素数$p$.定义集合
\begin{equation}
\mathbb{F}^{1/p^\infty} = \{\alpha\in\overline{\mathbb{F}}\mid \alpha^{p^k}\in\mathbb{F}, k\in\mathbb{Z}^+\}
\end{equation}
即全体$\mathbb{F}$元素的$p^k$次根构成的集合,显然$\mathbb{F}$是它的子集.

这个集合满足加法和乘法的封闭性:设$\alpha^m\in\mathbb{F}$和$\beta^n\in\mathbb{F}$,另$k$为$m$和$n$的任意公倍数,则$\alpha^k$和$\beta^k$都是$\mathbb{F}$的元素,那么有
\begin{equation}
\begin{aligned}
(\alpha+\beta)^k&=\alpha^k+\beta^k\\
(\alpha\beta)^k&=\alpha^k\beta^k
\end{aligned}
\end{equation}
故$\mathbb{F}^{1/p^\infty}$构成一个域.

任取$\alpha\in\mathbb{F}^{1/p^\infty}-\mathbb{F}$,令$k$是使得$\alpha^k\in\mathbb{F}$的最小非负整数,那么$\alpha$在$\mathbb{F}$上的最小多项式为
\begin{equation}
\opn{Irr}(\alpha, \mathbb{F})(x) = x^{p^k}-a
\end{equation}

而在$\mathbb{F}^{1/p^\infty}$上有
\begin{equation}
x^{p^k}-a = (x-\alpha)^{p^k}
\end{equation}

因此$\alpha$在$\mathbb{F}$上不可分.

\end{example}



\begin{exercise}{}
定义集合$\mathbb{F}^{1/p^k}=\{\alpha\in\overline{\mathbb{F}}\mid \alpha^{p^k}\in\mathbb{F}\}$,其中$k$是给定的任意正整数.证明这个集合是$\mathbb{F}$的扩域.

$\mathbb{F}^{1/p^\infty}/\mathbb{F}^{1/p^k}$是不是纯不可分扩张?
\end{exercise}


\begin{exercise}{}
设域$\mathbb{F}$的特征为素数$p$.证明$\mathbb{F}$是完美域(\autoref{SprbEx_def3}~\upref{SprbEx}),当且仅当$\mathbb{F}^{1/p^\infty}=\mathbb{F}$.
\end{exercise}



\autoref{PInsEx_ex1} 启发我们得到限制条件下纯不可分扩张的另一定义.为了描述这一定义,我们先看一个简单的引理:

\begin{lemma}{}
设域$\mathbb{F}$的特征是素数$p$,且$\alpha$是其代数元.

则“存在非负整数$k$使得$\alpha^{p^k}\in\mathbb{F}$”$\iff$“存在$a\in\mathbb{F}$和非负整数$n$使得$\opn{Irr}(\alpha, \mathbb{F})(x) = x^{p^n}-a$”.

\end{lemma}

\textbf{证明}:

$\implies$:

取$n$是所有符合条件的$k$中最小的.则$\opn{Irr}(\alpha, \mathbb{F})(x) = x^{p^n}-a$.

$\Leftarrow$:

取$k=n$即可.

\textbf{证毕}.

\begin{corollary}{}\label{PInsEx_cor1}
设域$\mathbb{F}$的特征是素数$p$,$\overline{\mathbb{F}}$是其代数闭包,则对于任意$a\in\mathbb{F}$和$k\in\mathbb{Z}^+$,存在唯一的$\alpha\in\overline{\mathbb{F}}$,使得$\alpha^{p^k}=a$.
\end{corollary}

也就是说,特征为$p$的域上,任意元素都有唯一的$p^k$次根.这同时意味着,纯不可分元素在$\mathbb{F}$上的最小多项式只有一个根——可以说,这种元素的最小多项式“不可分”的程度、或者说“根的重数”已经达到极致了,“纯”不可分确实名不虚传.


现在,我们可以给出纯不可分扩张在特征不为零时的等价定义了:




\begin{theorem}{特征为$p$时纯不可分扩张的等价定义}\label{PInsEx_the1}


给定代数域扩张$\mathbb{K}/\mathbb{F}$,其特征为素数$p$.则下列命题等价:

1. $\mathbb{K}$在$\mathbb{F}$上纯不可分;

2. $\forall \alpha\in\mathbb{K}$,$\alpha$在$\mathbb{F}$上纯不可分;

3. 存在一个$\mathbb{K}\to\mathbb{F}^{1/p^\infty}$\footnote{$\mathbb{F}^{1/p^\infty}$的定义见\autoref{PInsEx_ex1} .}的保$\mathbb{F}$\textbf{单}同态;

4. $[\mathbb{K}:\mathbb{F}]_S=1$.


\end{theorem}

\textbf{证明}:

1. $\implies$2. 

只需要考虑$\alpha\in\mathbb{K}-\mathbb{F}$的情况.记$\opn{Irr}(\alpha, \mathbb{F})=f$,则$f$不可分、不可约,故由\autoref{SprbEx_def2}~\upref{SprbEx}和\autoref{SprbEx_cor2}~\upref{SprbEx} 知,可以把$f$中次数不是$p$整数倍的单项式挖掉.

换言之,存在可分的不可约多项式$g(x)\in\mathbb{F}[x]$和非负整数$k$,使得$f(x)=g(x^{p^k})$.于是$g(\alpha^{p^k})=0$,因此$g=\opn{Irr}(\alpha^{p^k}, \mathbb{F})$,从而$\alpha^{p^k}$是$\mathbb{F}$上的可分元素.又因为$\mathbb{K}/\mathbb{F}$纯不可分,故$\alpha^{p^k}\in\mathbb{F}$.


2. $\implies$3. 

由纯不可分元素的\autoref{PInsEx_def1} ,显然$\mathbb{K}\subseteq\mathbb{F}^{1/p^\infty}$.故恒等映射即为所求.

3. $\implies$4. 

令$\varphi:\mathbb{K}\to\mathbb{F}^{1/p^\infty}$和$\sigma:\mathbb{K}\to\overline{\mathbb{F}}$都是保$\mathbb{F}$单同态.下证它们是同一个映射.

由于$\opn{ch}\mathbb{F}=p$,于是由\autoref{PInsEx_cor1} ,任取正整数$k$,每个$a\in\mathbb{F}$在其代数闭包上都有唯一的$p^k$次根.

\textbf{任取}$\alpha\in\mathbb{K}$,那么$\varphi(\alpha^{p^k})=(\varphi(\alpha))^{p^k}\in\mathbb{F}$.由于$\varphi$是单射且保$\mathbb{F}$的元素不变,故$\alpha^{p^k}\in\mathbb{F}$.于是同理,$\sigma(\alpha^{p^k})=\alpha^{p^k}$.

由于都是同态,故$\varphi(\alpha^{p^k})=\sigma(\alpha^{p^k})\iff \varphi(\alpha)=\sigma(\alpha)$.

4. $\implies$1. 

由可分次数的定义,题设意味着$\mathbb{K}-\mathbb{F}$的元素都不是$\mathbb{F}$上的可分元素.


\textbf{证毕}.

由\autoref{PInsEx_the1} 中第1和第3条的等价性,可知域$\mathbb{F}$最大的纯不可分扩张就是$\mathbb{F}^{1/p^\infty}$.





\subsection{纯不可分扩张的一些性质}


\begin{theorem}{}
给定特征为素数$p$的域$\mathbb{F}$和$\overline{\mathbb{F}}$的子集$S$,如果$S$的元素都是$\mathbb{F}$的纯不可分元素,那么$\mathbb{F}(S)/\mathbb{F}$是纯不可分扩张.
\end{theorem}

\textbf{证明}:

任取$\alpha_i\in S\cup\mathbb{F}$,则$\alpha_i$纯不可分,于是存在正整数$k_i$使得$\alpha_i^{p^{k_i}}\in\mathbb{F}$.于是,$\alpha_1^{p^{k_1}p^{k_2}}$和$\alpha_2^{p^{k_2}p^{k_1}}$都在$\mathbb{F}$中.

则有
\begin{equation}\label{PInsEx_eq1}
(\alpha_1\alpha_2)^{p^{k_1+k_2}}=\alpha_1^{p^{k_1}p^{k_2}}\alpha_2^{p^{k_2}p^{k_1}}\in\mathbb{F}
\end{equation}
\begin{equation}\label{PInsEx_eq2}
(\alpha_1+\alpha_2)^{p^{k_1+k_2}}=\alpha_1^{p^{k_1}p^{k_2}}+\alpha_2^{p^{k_2}p^{k_1}}\in\mathbb{F}
\end{equation}
和
\begin{equation}\label{PInsEx_eq3}
(\alpha_1^{-1})^{p^{k_1}} = (\alpha_1^{p^{k_1}})^{-1} \in\mathbb{F}
\end{equation}

换言之,纯不可分元素之积、和、逆都是纯不可分元素,因此$\mathbb{F}(S)$中的元素都是$\mathbb{F}$的纯不可分元素.由\autoref{PInsEx_the1} 第2条的定义得证.

\textbf{证毕}.




\begin{theorem}{}
设$\mathbb{L}/\mathbb{K}/\mathbb{F}$是代数扩域链.如果$\mathbb{L}/\mathbb{F}$纯不可分,那么$\mathbb{K}/\mathbb{F}$和$\mathbb{L}/\mathbb{K}$也纯不可分.
\end{theorem}

\textbf{证明}:

$\mathbb{K}\subseteq\mathbb{L}$,因此据\autoref{PInsEx_the1} 第2条的定义,$\mathbb{K}/\mathbb{F}$纯不可分.

$\mathbb{F}\subseteq\mathbb{K}$,因此据可分元素的定义和\autoref{PInsEx_the1} 第2条的定义,$\mathbb{F}/\mathbb{K}$纯不可分.


\textbf{证毕}.


\begin{theorem}{塔性质}
设$\mathbb{L}/\mathbb{K}/\mathbb{F}$是代数扩域链.如果$\mathbb{K}/\mathbb{F}$和$\mathbb{L}/\mathbb{K}$纯不可分,那么$\mathbb{L}/\mathbb{F}$纯不可分.
\end{theorem}

\textbf{证明}:

任取$\alpha\in\mathbb{L}$,则由于$\mathbb{L}/\mathbb{K}$纯不可分,存在正整数$m$使得$\alpha^{p^m}\in\mathbb{K}$.由于$\mathbb{K}/\mathbb{F}$纯不可分,存在正整数$n$使得$(\alpha^{p^m})^{p^n}\in\mathbb{F}$.于是,$\alpha$是$\mathbb{F}$上的纯不可分元素.

\textbf{证毕}.


\begin{theorem}{}
如果$\mathbb{K}/\mathbb{F}$是一个纯不可分的代数扩张,且域合成$\mathbb{EK}$存在,那么$\mathbb{EK}/\mathbb{EF}$是纯不可分扩张.
\end{theorem}

\textbf{证明}:

任取$\epsilon\in\mathbb{E}$和$\kappa\in\mathbb{K}$,则存在正整数$k$使得$(\epsilon\kappa)^k\in\mathbb{EF}$.注意到$\mathbb{EK}$是由集合$\{\epsilon\kappa\mid\epsilon\in\mathbb{E}, \kappa\in\mathbb{K}\}$生成的,再类比\autoref{PInsEx_eq1} 、\autoref{PInsEx_eq2} 和\autoref{PInsEx_eq3} 可知,$\mathbb{EK}$中的元素都是$\mathbb{EF}$上的纯不可分元素.

\textbf{证毕}.


\begin{exercise}{}
给定域$\mathbb{F}$,设$\mathbb{K}_i$都是$\mathbb{F}$的代数纯不可分扩域.证明$\mathbb{K}_1\mathbb{K}_2/\mathbb{F}$是纯不可分扩张.
\end{exercise}

\begin{theorem}{}\label{PInsEx_the2}
给定域$\mathbb{F}$及其代数闭包$\overline{\mathbb{F}}$.则$\alpha\in\overline{\mathbb{F}}$是$\mathbb{F}$上的纯不可分元素,当且仅当对于$\overline{\mathbb{F}}$的保$\mathbb{K}$自同构$\sigma$,都有$\sigma\alpha=\alpha$.
\end{theorem}

\textbf{证明}:

$\opn{ch}\mathbb{F}=0$以及其它$\mathbb{F}$是完美域的情况是平凡情况.下面讨论$\mathbb{F}$存在不可分元素的情况,设其特征为$p$.

必要性:

设$\alpha\in\overline{\mathbb{F}}$是$\mathbb{F}$上的纯不可分元素,则存在正整数$k$和$a\in\mathbb{F}$使得$\alpha$是$\mathbb{F}$上不可约多项式$x^{p^k}-a$唯一的根.$\sigma$必须把$\mathbb{F}$上多项式的一个根映射为另一个根,因此只能有$\sigma\alpha=\alpha$.







充分性:

设对于$\overline{\mathbb{F}}$的保$\mathbb{K}$自同构$\sigma$,都有$\sigma\alpha=\alpha$.设$\alpha$在$\mathbb{F}$上的最小多项式为$f(x)$,那么由于任意的$\sigma$可以将$\alpha$变为$f(x)$的任意其它根,可知$f(x)$在代数闭包$\overline{\mathbb{F}}$上只有一个根,因此$f$在$\overline{\mathbb{F}}$上应分解为$(x-\alpha)^n$,其中$n$是一个正整数.

展开$(x-\alpha)^n$,其常数项$\alpha^n=a\in\mathbb{F}$.因此$x^n-a$也是$\alpha$的零化多项式.由于次数相等以及最小多项式的定义,可知$(x-\alpha)^n=x^n-a$.

类比\autoref{SprbEx_the3}~\upref{SprbEx}一开始的证明过程,可知存在可分的不可约多项式$h(x)\in\mathbb{F}$和非负整数$k$,使得$h(x^{p^k})=f(x)=x^n-a$,因此$p^k\mid n$.设$n=p^km$,那么$h(x)=x^m-a$.由于$f$只有唯一的根$\alpha$,故$h$也只能有唯一的根\footnote{$h$的根都是$a$的$m$次根,而$\alpha$是$a$唯一的$p^km$次根,因此也是$h$的根唯一的$p^k$次根,故$h$的根也唯一}.加上$h$是可分的,不存在重根,故$m=1$.因此,$\alpha$是$a$唯一的$p^k$次根,符合\autoref{PInsEx_def1} .

\textbf{证毕}.

\autoref{PInsEx_the2} 充分性的证明很能说明纯不可分扩张的特点:就一个根,因此对应定理中$\sigma \alpha$的可能性就一个.








