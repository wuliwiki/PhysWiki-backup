% 欧多克索斯(综述)
% license CCBYSA3
% type Wiki

本文根据 CC-BY-SA 协议转载翻译自维基百科 \href{https://en.wikipedia.org/wiki/Eudoxus_of_Cnidus}{相关文章}。

欧多克斯(Eudoxus of Cnidus,/ˈjuːdəksəs/;古希腊语:Εὔδοξος ὁ Κνίδιος,Eúdoxos ho Knídios;约公元前390年—约公元前340年)是一位古希腊的天文学家、数学家、医生和立法者。\(^\text{[1]}\)他是阿尔喀塔斯和柏拉图的学生。他的原始著作均已佚失,但有些片段保存在希帕克斯的《阿拉托斯与欧多克斯〈天象〉注释》中。\(^\text{[2]}\)比提尼亚的西奥多修斯所著的《球面论》可能是基于欧多克斯的某部著作。
\subsection{生平}
欧多克斯是埃斯喀涅斯之子,出生并去世于克尼多斯(Cnidus,又作 Knidos),这是位于安纳托利亚西南海岸的一座城市。\(^\text{[3]}\)欧多克斯的出生与去世年份并不确切,但第欧根尼·拉尔修记载了一些传记细节,并提到阿波罗多洛斯说他在第103届奥林匹亚会期(公元前368年至前365年)达到了事业巅峰,并声称他在53岁时去世。19世纪的数学史学家据此推断其生卒年为公元前408年至前355年,\(^\text{[4]}\)但20世纪的学者认为这些日期相互矛盾,更倾向于其出生于约公元前390年。\(^\text{[5]}\)“欧多克斯”这个名字意为“受尊敬的”或“声誉良好的”(希腊语:εὔδοξος,源自 eu “好”与 doxa “意见、信念、名声”,相当于拉丁语中的 Benedictus)。

据第欧根尼·拉尔修援引卡利马科斯的《书目》所述,欧多克斯曾在大希腊地区塔兰图姆的阿尔喀塔斯门下学习数学,并向西西里人腓利斯顿学习医学。23岁时,他与医生忒俄墨冬一同前往雅典学习哲学,忒俄墨冬是他的资助者,也可能是他的情人。\(^\text{[6]}\)在雅典他只停留了两个月,住在比雷埃夫斯,每天往返七英里(约11公里)步行去听智者们的讲座。之后他返回克尼多斯。朋友们后来集资送他前往埃及赫利奥波利斯,在那里他学习了16个月的天文学与数学。自埃及北上,他又前往马尔马拉海南岸的基齐库斯,即古称的普罗庞提斯地区。随后他南下前往摩索拉斯之宫廷。在旅途中,他也收了一批自己的学生。

大约在公元前368年,欧多克斯带着他的学生返回雅典。有些资料称,在约公元前367年柏拉图前往叙拉古期间,他曾担任柏拉图学园的学监,并曾教授过亚里士多德。他最终回到故乡克尼多斯,在城邦议会任职。在克尼多斯期间,他建造了一个天文观测所,并继续撰写著作,讲授神学、天文学和气象学。他育有一子,名为阿里斯塔戈拉斯,另有三位女儿,分别名为阿克提斯、菲尔提斯和德尔菲斯。

在数学天文学方面,欧多克斯因提出同心球体理论而闻名,这一理论是对行星运动的早期解释之一。诗人阿拉托斯也称赞他曾制造过一个天球仪。\(^\text{[7]}\)

他在比例理论上的工作展现出对无理数和线性连续统的洞察:该理论能严谨地处理连续量,而不仅仅局限于整数或有理数。当16世纪塔塔利亚等人重新发现这一理论时,它成为自然科学中定量研究的基础,并启发了理查德·戴德金对实数理论的研究。\(^\text{[8]}\)

火星和月球上都有以他命名的陨石坑。此外,一条代数曲线——欧多克斯双曲线也以他命名。
\subsection{数学}
欧多克斯被一些人视为古典希腊最伟大的数学家,整个古代时期仅次于阿基米德。[9] 他很可能是欧几里得《几何原本》第五卷的大部分内容的主要来源。[10] 欧多克斯严谨地发展了安提丰的“穷竭法”——这是积分学的前身,在后一个世纪中也被阿基米德巧妙地加以运用。在应用这一方法时,欧多克斯证明了一些重要的数学命题,例如:圆的面积之比等于其半径的平方之比,球的体积之比等于其半径的立方之比,锥体的体积等于与其底面积和高相同的柱体体积的三分之一,而金字塔的体积也是相应棱柱体体积的三分之一。[11]

欧多克斯引入了“未量化的数学量”这一概念,用以描述并操作诸如线段、角度、面积和体积等连续的几何实体,从而避免了对无理数的直接使用。在此过程中,他逆转了毕达哥拉斯学派对“数”与“算术”的强调,转而以几何概念作为严谨数学的基础。毕达哥拉斯学派的一些成员,例如欧多克斯的老师阿尔喀塔斯,曾认为只有算术才能为数学证明提供基础。出于对不可通约量(即无法用整数比例表示的量)进行理解与运算的需要,欧多克斯建立了可能是历史上第一个基于显式公理体系的数学演绎结构。欧多克斯的这一思想转向,引发了数学领域长达两千年的分裂。结合当时希腊知识界普遍对实际应用问题缺乏兴趣的态度,也导致了对算术和代数技术发展的显著退却。[11]
