% 北京大学 2000 年 考研 固体物理
% license Usr
% type Note

\textbf{声明}:“该内容来源于网络公开资料,不保证真实性,如有侵权请联系管理员”


\subsection{(20分)}
说明半导体硅单品的晶体结构,布拉伐格子,所属晶系、点群和空间群,每个单胞(Conventional unit cell)中的硅原子数;如果晶格常数为 a,求正格子空间原胞(Primitive cell)的体积和第一布里渊区的体积。
\subsection{(20分)}
双原子链。考虑一个线型的简正模式,链上最近邻原子间的力常数交错地等于$C$和 10C。令两种原子的质量$m$相等,近邻原子间距为 $a$,

(1)作用,求色散关系$\omega(k)$要求写出推导过程。粗略地画出简约区的色散关系图。

(2)讨论在布里渊区的边界处光学波和声学波的特点。

(3)说明声子的物理意义,求低温下声学支声子对此双原子链比热的贡献与温度关系。
\subsection{(16分)}
由泡利不相容原理,金属中费米面附近的自由电子容易被激发,费米能级以下很低能级上的自由电子很难激发,通常称为费米冻结。用此物理图象,

(1)估算在室温下金属中自由电子的比热。

(2)估算$T\to 0K$金属中自由电子的泡,利自旋顺磁性磁化率。