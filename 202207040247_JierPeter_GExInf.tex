% 无穷Galois扩张
% 域论|域扩张|Galois群|伽罗华|伽罗瓦|Krull拓扑|Krull定理|无限Galois扩张|伽罗华扩张

\pentry{Galois扩张\upref{GExt},拓扑空间\upref{Topol}}

%GTM242的Proposition 4.1 已经整合在\autoref{GExt_the8}~\upref{GExt}和\autoref{GExt_the9}~\upref{GExt}中了.

\textbf{Galois扩张}\upref{GExt}中除了Galois扩张和Galois群的基本性质,剩下的重点内容全是\textbf{有限}Galois扩张的情况,见\autoref{GExt_sub1}~\upref{GExt}.作为提醒,再总结一次:有限Galois扩张都是单代数扩张,且为分裂域.

本节介绍的是无限Galois扩张中的性质,将有限扩张的\textbf{Galois理论基本定理}(\autoref{GExt_the10}~\upref{GExt})进行拓展,得到\textbf{Krull}定理.Krull的工作亮点,在于给Galois群赋予了一个拓扑结构.


为了得到Krull拓扑,我们要先观察Galois扩域的一些性质:







\begin{theorem}{}
设$\mathbb{K}/\mathbb{F}$是Galois扩域,$\mathcal{M}=\{\opn{Gal}(\mathbb{K}/\mathbb{M})\mid \mathbb{M}/\mathbb{F}\text{是有限扩张}\}$.则有:

1.$\forall H\in\mathcal{M}$,$[\opn{Gal}(\mathbb{K}/\mathbb{F}): H]$\footnote{即群指数,子群$H$在$\opn{Gal}(\mathbb{E}/\mathbb{K})$中陪集的数量.}是有限的;

2.$\bigcap_{H\in\mathcal{M}}=e$.这里$e$是群的单位元,即$\mathbb{K}$上的恒等映射$\opn{id}_{\mathbb{K}}$.

3.$\forall H_1, H_2\in\mathbb{M}$,有$H_1\cap H_2\in\mathbb{M}$;

4.$\forall H\in\mathcal{M}$,$\exists N\triangleleft\opn{Gal}(\mathbb{K}/\mathbb{F})$,且$N\subseteq H$;

\end{theorem}

\textbf{证明}:

1.

\textbf{证毕}.





































