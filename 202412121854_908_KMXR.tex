% 卡西米尔效应(综述)
% license CCBYSA3
% type Wiki

本文根据 CC-BY-SA 协议转载翻译自维基百科\href{https://en.wikipedia.org/wiki/Casimir_effect}{相关文章}。

\begin{figure}[ht]
\centering
\includegraphics[width=6cm]{./figures/0b8861a1a0020cbd.png}
\caption{平行板上的卡西米尔力} \label{fig_KMXR_1}
\end{figure}
在量子场论中,\textbf{卡西米尔效应}(或称\textbf{卡西米尔力})是一种作用在受限空间宏观边界上的物理力,它源于场的量子涨落。当该效应以单位面积上的力来描述时,有时会使用“卡西米尔压力”这一术语。[2][3] 这一效应以荷兰物理学家\textbf{亨德里克·卡西米尔}的名字命名,他于1948年预测了电磁系统中的这一效应。

同年,卡西米尔与迪尔克·波尔德一起描述了一个类似的效应,这一效应发生在中性原子靠近宏观界面时,称为卡西米尔–波尔德力。[4] 他们的结果是对伦敦–范德瓦尔斯力的推广,且包括了由于光速有限所导致的时滞。伦敦–范德瓦尔斯力、卡西米尔力和卡西米尔–波尔德力的基本原理可以在同一框架下进行表述。[5][6]

在1997年,史蒂文·K·拉莫罗进行了直接实验,定量测量了卡西米尔力,结果与理论预测值相差不超过5\%。[7]

卡西米尔效应可以通过以下观点理解:宏观物质界面的存在,如电导体和介电体,改变了第二量子化电磁场能量的真空期望值。[8][9] 由于这一能量的值依赖于材料的形状和位置,卡西米尔效应表现为这些物体之间的力。

任何支持振荡的介质都有类似的卡西米尔效应。例如,绳上的珠子[10][11],以及浸入湍流水或气体中的板[12][13]都能说明卡西米尔力。

在现代理论物理中,卡西米尔效应在质子的手征袋模型中起着重要作用;在应用物理中,它在一些新兴的微技术和纳米技术中具有重要意义。[14]
\subsection{物理性质}
典型的例子是两个未带电的导电板,在真空中相距几纳米。在经典描述中,缺乏外部场意味着板间不存在场,也没有力将它们连接在一起。[15] 但是,当使用量子电动力学真空来研究这个场时,可以看到板对构成场的虚拟光子产生影响,并产生一个净力[16]——这种力要么是吸引力,要么是排斥力,取决于板的具体排列。尽管卡西米尔效应可以通过虚拟粒子与物体的相互作用来表达,但最好通过量子化场在物体之间的零点能量来描述,并且更容易计算。这个力已经被测量,并且是一个由第二量子化正式捕捉到的效应的典型例子。[17][18]

在这些计算中,边界条件的处理存在争议。事实上,“卡西米尔的原始目标是计算导电板上可极化分子之间的范德瓦尔斯力。”因此,它可以在没有任何提及量子场的零点能量(真空能量)的情况下进行解释。[19]

由于该力的强度随距离迅速减小,因此只有在物体之间的距离很小时才能测量到。这种力变得非常强大,以至于在亚微米尺度下,它成为未带电导体之间的主导力。实际上,在10纳米的间距下——约为原子典型尺寸的100倍——卡西米尔效应产生相当于约1大气压的压力(精确值依赖于表面几何形状和其他因素)。[17]
\subsection{历史}
荷兰物理学家亨德里克·卡西米尔和迪尔克·波尔德在1947年于飞利浦研究实验室提出了两种可极化原子之间以及这种原子与导电板之间存在一种力;这种特殊形式的力被称为卡西米尔–波尔德力。[4] 在与尼尔斯·玻尔的对话后,玻尔建议这与零点能量有关,卡西米尔单独提出了理论,预测了中性导电板之间的力,这一现象被称为卡西米尔效应。[20]

该力的预测后来扩展到了有限导电性的金属和介电体,而后来的计算则考虑了更一般的几何形状。在1997年前的实验中,卡西米尔力被定性地观察到,并通过测量液态氦薄膜的厚度间接验证了预测的卡西米尔能量。最终,1997年拉莫罗的直接实验定量测量了卡西米尔力,结果与理论预测值相差不超过5\%。[7] 随后的实验则达到了几百分点的精度。