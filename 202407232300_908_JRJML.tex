% 加尔加梅勒
% license CCBYSA3
% type Wiki

(本文根据 CC-BY-SA 协议转载自原搜狗科学百科对英文维基百科的翻译)

\textbf{Gargamelle}是1970年到1979年在欧洲核子研究中心(CERN)运行的一种重液体气泡室探测器。它被设计用来探测中微子和反中微子,这些中微子和反中微子是在1970年至1976年由质子同步加速器()发出的光束产生的,因此探测器被移到超级质子同步加速器(SPS)上。[1] 1979年,由于在气泡室发现了一个不可修复的裂缝,因此探测器停止使用了。它目前是在欧洲核子研究中心微观展览的那一部分,对公众开放。

加尔加梅勒以发现中性电流的实验闻名于世。1973年7月提出的中性线电流是Z0 玻色子存在的第一个实验表明,因此这意味着向验证弱电理论迈出的重要一步。

加尔加梅勒既可以指气泡室探测器本身,也可以指同名的高能物理学实验。这个名字来源于16世纪弗朗索瓦·拉伯雷的一部小说《巨人和潘塔格鲁的生活》,其中女巨人加尔加梅勒是巨人的母亲。[1]