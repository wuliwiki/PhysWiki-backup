% 南京航空航天大学 2004 量子真题答案
% license Usr
% type Note

\textbf{声明}:“该内容来源于网络公开资料,不保证真实性,如有侵权请联系管理员”


\subsection{一}
\subsubsection{1.} 
解:\\
(1) $\because \hat{p} \psi = p\left[e^{\frac{i}{\hbar}(PX - Et)} - e^{-\frac{i}{\hbar}(PX + Et)}\right] \neq p\psi$\\
 $\therefore \psi$不是动量算符$\hat{p}$的本征函数。\\\\
(2) $\because \hat{p}^2 \psi = p^2\left[e^{\frac{i}{\hbar}(PX - Et)} + e^{-\frac{i}{\hbar}(PX + Et)}\right] = p^2\psi$ \\
$\therefore \psi$是动量平方算符$\hat{p}^2$ 的本征值函数。 
\subsubsection{2.}解:\\
$(\hat{p} + \hat{x})\psi = \lambda \psi$\\
即:$(-i\hbar\frac{\partial}{\partial x} + x)\psi = \lambda \psi \quad \frac{\hbar}{i} \frac{d\psi}{dx} = (\lambda - x)\psi$ \\
$\therefore \frac{\hbar}{i} \frac{d\psi}{\psi} = (\lambda - x)dx$\\
附注: 本征函数 $\psi = Ae^{\frac{1}{\hbar}(\lambda x - \frac{x^2}{2})}$ \\
      本征值 $\lambda$ 为所有实数。
\subsubsection{3.}
解:
\begin{align}
\overline{V}(r) &= -\frac{e^2}{4 \pi \epsilon_0} \int_{-\infty}^{+\infty} \psi^* \frac{1}{r} \psi \, d\tau 
= -\frac{e^2}{4 \pi \epsilon_0} \int_0^\infty \frac{1}{r} |\psi|^2 \, d\tau\\
&= \left( -\frac{e^2}{4 \pi \epsilon_0} \right) \left( \frac{1}{\pi a_0^3} \right) \int_0^\pi \sin \theta \, d\theta \int_0^{2\pi} \, d\phi \int_0^\infty \frac{1}{r} r^2 e^{-\frac{2}{a_0}^2} \, dr\\
&= -\frac{1}{4 \pi \epsilon_0} \frac{e^2}{a_0}~
\end{align}
\subsubsection{4.}
解:\\
(1) $\because B = A^\dagger A$\\
$\therefore B^2 = A^\dagger A A^\dagger A = A^\dagger A \left(1 - AA^\dagger \right) = A^\dagger A - A^\dagger A^2 A^\dagger=A^\dagger A=B$\\\\
(2) 
$\because \quad B^2 = B$\\
$\therefore B \mathbf{v} = \lambda \mathbf{v}, \quad B^2 \mathbf{v} = \lambda^2 \mathbf{v} \quad \lambda^2 = \lambda = 0, 1$\\
$\text{而 } B \text{ 的本征值为 } 0, 1 \text{,在 } B \text{ 表示中 } B \text{ 为对角矩阵。}
B = \begin{pmatrix} 0 & 0 \\\\ 0 & 1 \end{pmatrix}$\\
设
$$A = \begin{pmatrix} a & b \\\\ c & d \end{pmatrix} \text{则}A^\dagger = \begin{pmatrix} a^* & c^* \\\\ b^* & d^* \end{pmatrix}~$$
则有  $$\quad  A^ \dagger A^* =  \begin{pmatrix} a^*a + c^*c & a^*b + c^*d \\\\ b^*a + d^*c & b^*b + d^*d  \end{pmatrix} = B =  \begin{pmatrix} 0 & 0 \\\\ 0 & 1  \end{pmatrix}  \quad  \text{(1)}~$$

$$ A^* A^ \dagger = \begin{pmatrix} a^* a + b b^* & c^* a + d^* b \\\\ a^* c + b^*d & c^* c + d^* d  \end{pmatrix} = 1 - B =  \begin{pmatrix} 1 & 0 \\\\ 0 & 0  \end{pmatrix} \quad \text{(2)}~$$

$$A^2 =  \begin{pmatrix} a^2 + b c & a b + b d \\\\ a c + d c & d^2 + b c \end{pmatrix} = 0 = \begin{pmatrix} 0 & 0 \\\\ 0 & 0 \end{pmatrix} \quad \text{(3)}~$$
由 (1), (2), (3) 得解: $\quad A = \begin{pmatrix} 0 & e^{i \varphi} \\\\ 0 & 0 \end{pmatrix} \quad \text{或} \quad A = \begin{pmatrix} 0 & 0 \\\\ 0 & 0 \end{pmatrix}$
\subsubsection{5、}
解:一维无限深势阱中粒子的一维定态波函数:
$$\psi_n(x) = \sqrt{\frac{2}{a}} \sin \frac{n\pi x}{a}~,$$能量
$$E_n = n^2 \frac{\pi^2 \hbar^2}{2m a^2}, \quad n = 1, 2, 3, \dots~$$
将$\psi(x)$按$\psi_n$展开:
\begin{align}
\psi(x) &= \frac{4}{\sqrt{a}} \sin \frac{\pi x}{a} \cos^2 \frac{\pi x}{a} = \frac{4}{\sqrt{a}} \sin \frac{\pi x}{a} \cdot \frac{1}{2} (1 + \cos \frac{2\pi x}{a})\\
&= \frac{4}{\sqrt{a}} \left[ \frac{1}{2} \sin \frac{\pi x}{a} + \frac{1}{2} \sin \frac{\pi x}{a} \cdot \cos \frac{2\pi x}{a} \right]\\
&= \frac{2}{\sqrt{a}} \left[ \sin \frac{\pi x}{a} + \frac{1}{2} [\sin \frac{3\pi x}{a} + \sin(-\frac{\pi x}{a})] \right]\\
&= \sqrt{\frac{2}{a}} \left( \frac{1}{\sqrt{2}} \sin \frac{3\pi x}{a} + \frac{1}{\sqrt{2}} \sin \frac{\pi x}{a} \right)~
\end{align}
$$\therefore \text{能量可能值:}
E_1 = \frac{\pi^2 \hbar^2}{2m a^2} \text{概率} \\, W_1 = \frac{1}{2}~$$
$$E_3 = \frac{9\pi^2 \hbar^2}{2m a^2}\text{概率} \\,W_3 = \frac{1}{2}~$$
\subsection{二}
\subsubsection{1.}
解:
电子自旋变 $X(t)$满足方程 , $i\hbar \frac{d}{dt} X(t) = \hat{H} X(t) \quad $(1)
$$\hat{H} = -\hat{\mu} \cdot \mathbf{B} = \mu_0 B \hat{\sigma}_y = \mu_0 B \begin{pmatrix} 0 & -i \\\\ i & 0 \end{pmatrix}~$$
$$X(t) = \begin{pmatrix} a(t) \\\\ b(t) \end{pmatrix}~$$
则在 $S_g$表中方程(1)可表示为: $$\quad i\hbar \frac{d}{dt} \begin{pmatrix} a(t) \\\\ b(t) \end{pmatrix} = \mu_0 B \begin{pmatrix} 0 & -i \\\\ i & 0 \end{pmatrix} \begin{pmatrix} a(t) \\\\ b(t) \end{pmatrix} \quad (2)~$$

$$\therefore \begin{cases} \frac{da(t)}{dt} = -\frac{\mu_0 B}{\hbar} b(t) = -\omega b(t) \\\\ \frac{db(t)}{dt} = \frac{\mu_0 B}{\hbar} a(t) = \omega a(t) \end{cases} \quad~ \text{其中},\omega = \frac{\mu_0 B}{\hbar}~$$

$\text{反归纳条件} X(0) = \begin{pmatrix} 1 \\\\ 0 \end{pmatrix} \\, \text{有解:}
a(t)=\cos \omega t,b(t)=\sin  \omega t$
$\therefore \text{t时刻自旋变为:} X(t) = \begin{pmatrix} \cos \omega t \\\\ \sin \omega t \end{pmatrix}$
\subsubsection{2.}
\begin{align}
\quad & \bar{S}_X(t) = X^{\dagger}(t) \hat{S}_X X(t) = \frac{\hbar}{2} 
\begin{pmatrix} \cos \omega t \\\\ \sin \omega t \end{pmatrix}^{\dagger} 
\begin{pmatrix} 0 & 1 \\\\ 1 & 0 \end{pmatrix}
\begin{pmatrix} \cos \omega t \\\\ \sin \omega t \end{pmatrix} = \frac{\hbar}{2} \sin 2\omega t \\\\
\bar{S}_Y(t) = & X^{\dagger}(t) \hat{S}_Y X(t) = \frac{\hbar}{2} 
\begin{pmatrix} \cos \omega t \\\\ \sin \omega t \end{pmatrix}^{\dagger} 
\begin{pmatrix} 0 & -i \\\\ i & 0 \end{pmatrix}
\begin{pmatrix} \cos \omega t \\\\ \sin \omega t \end{pmatrix} = 0 \\\\
\bar{S}_Z(t) = & X^{\dagger}(t) \hat{S}_Z X(t) = \frac{\hbar}{2} 
\begin{pmatrix} \cos \omega t \\\\ \sin \omega t \end{pmatrix}^{\dagger} 
\begin{pmatrix} 1 & 0 \\\\ 0 & -1 \end{pmatrix}
\begin{pmatrix} \cos \omega t \\\\  \sin \omega t \end{pmatrix} = \frac{\hbar}{2} \cos 2\omega t \\\\~
\end{align}
\subsubsection{3.}
$$\text{将} X(t) \text{按} \hat S_0 \text{的本征函数展开:}
X(t) =
\begin{pmatrix}
\cos \omega t \\\\
\sin \omega t 
\end{pmatrix}
= \cos \omega t \begin{pmatrix}
1 \\\\
0 
\end{pmatrix} 
+ \sin \omega t \begin{pmatrix}
0 \\\\
1 
\end{pmatrix}~$$
\begin{align}
\therefore \text{t时刻:} \quad & \text{测量电子自旋向上的概率} \quad W_{\uparrow} = \cos^2\omega t \\\\
\quad & \text{测量电子自旋向上的概率} \quad W_{\downarrow} = \sin^2\omega t~
\end{align}
\subsubsection{4.}
$$\quad t = 0 \quad \text{时} \quad X(0) =
\begin{bmatrix}
\cos \omega_0 t \\\\
\sin \omega_0 t
\end{bmatrix}
=
\begin{bmatrix}
1 \\
0
\end{bmatrix}~$$

$$\text{设 } t = \tau \text{时}\\, \text{相位反转向下}. \quad X(\tau) = 
\begin{bmatrix}
\cos \omega \tau \\
\sin \omega \tau
\end{bmatrix}
=
\begin{bmatrix}
0 \\\\
1
\end{bmatrix}~$$

$$\therefore 
\begin{cases}
\cos \omega \tau = 0 \\\\
\sin \omega \tau = 1
\end{cases}
\quad \text{解得:} \quad \omega \tau = \left(2k+1\right)-\frac{\pi}{2} \quad \left( k=0,1,2,\ldots \right)~$$
$$\therefore \quad \tau = \frac{\left(2k+1\right)\pi}{2\omega} = \frac{\left(2k+1\right)\pi \hbar}{2 \mu_0 B} \quad (k=0,1,2,\ldots)~$$
\subsection{三.}
解:
\subsubsection{ (1)}转盘转动惯量 $I = \frac{1}{2}MR^2$,取 $\xi$ 轴为转轴。\\
 哈密顿算符:$\hat{H}_0 = \frac{1}{2I}\hat{L}_\xi^2 = -\frac{\hbar^2}{2I}\frac{\partial^2}{\partial\varphi^2}$\\
 本征方程:$-\frac{\hbar^2}{2I}\frac{\partial^2}{\partial\varphi^2}\psi = E\psi$\\
对应本征函数:$\psi_m^{(0)} = \frac{1}{\sqrt{2\pi}}e^{im\varphi} \quad  m = 0, \pm1, \pm2, \ldots $\\
能量本征值:$E_m^{(0)} = m^2\frac{\hbar^2}{2I} = m^2\frac{\hbar^2}{MR^2} $\\
除基态($m = 0$)外,第$\left|m\right|$ 激发态能级为二重简并:$\psi_{1m}^{(0)} = \frac{1}{\sqrt{2\pi}}e^{im\varphi}, \quad \psi_{2m}^{(0)} = \frac{1}{\sqrt{2\pi}}e^{-im\varphi}$\\
或$\psi_{1m}^{(0)} = \frac{1}{\sqrt{\pi}}\cos m\varphi, \quad \psi_{2m}^{(0)} = \frac{1}{\sqrt{\pi}}\sin m\varphi $
\subsubsection{(2)}
受微扰后,$H' = F_0 \delta(\varphi-\varphi_0)$。微扰矩阵元:
$$ H'_{11} = \int_0^{2\pi} \frac{1}{\sqrt{\pi}}\cos m_0 \varphi F_0 \delta(\varphi-\varphi_0)\frac{1}{\sqrt{\pi}}\cos m_0 \varphi d\varphi = \frac{F_0}{\pi} \cos^2(m\varphi_0)~$$
$$H'_{22} = \int_0^{2\pi} \frac{1}{\sqrt{\pi}}\sin m_0 \varphi F_0 \delta(\varphi-\varphi_0)\frac{1}{\sqrt{\pi}}\sin m_0 \varphi d\varphi = \frac{F_0}{\pi} \sin^2(m\varphi_0)~$$
$$H'_{12} = H'_{21} = \int_0^{2\pi} \frac{1}{\sqrt{\pi}}\cos m_0 \varphi F_0 \delta(\varphi-\varphi_0)\frac{1}{\sqrt{\pi}}\sin m_0 \varphi d\varphi = \frac{F_0}{2\pi} \sin(2m \varphi_0)~$$
由久期方程:
$$\left| 
\begin{array}{cc}
\frac{F_0}{\pi} \cos^2(m\varphi_0) - E_m^{(1)} & \frac{F_0}{2\pi} \sin(2m\varphi_0) \\\\
\frac{F_0}{2\pi} \sin(2m\varphi_0) & \frac{F_0}{\pi} \sin^2(m\varphi_0) - E_m^{(1)}
\end{array}
\right| = 0~$$
解得:
$E_m^{(1)} = 
\begin{cases} 
\frac{E_0}{x} \\\\0
\end{cases}$\\
$$\therefore \\ \text{能量二阶近似为:} E_m = E_m^{(0)} + E_m^{(1)} = 
\begin{cases}
m^2\frac{\hbar^2}{M R^2} + \frac{F_0}{\pi}, \\\\
m^2\frac{\hbar^2}{M R^2}
\end{cases}~$$
零级波函数:$\Psi_m^{(0)} = C_1 \frac{1}{\sqrt{\pi}} \cos m \varphi + C_2 \frac{1}{\sqrt{\pi}} \sin m \varphi$
$$\text{由方程} \quad 
\begin{pmatrix}
\frac{F_0}{\pi} \cos^2 m \varphi_0 - E_m^{(1)} & \frac{F_0}{\pi} \sin m \varphi_0 \cdot \cos m \varphi_0 \\\\
\frac{F_0}{\pi} \sin m \varphi_0 \cdot \cos m \varphi_0 & \frac{F_0}{\pi} \sin^2 m \varphi_0 - E_m^{(1)}
\end{pmatrix}
\begin{pmatrix}
C_1 \\\\
C_2
\end{pmatrix} 
= 0~$$
$\text{当 } E_m^{(1)} = 0 \text{时,解得:} \quad C_1 = -\sin m \varphi_0, \quad C_2 = \cos m \varphi_0$\\
$\therefore \quad \overline{\Psi}_m^{(1)} = \frac{1}{\sqrt{\pi}} \sin \left[m (\varphi - \varphi_0)\right]$
$\because\text{当} E^{(1)}_m = \frac{F_0}{\pi} \text{ 时:} C_1 = \cos m\varphi_0, \quad C_2 = \sin m\varphi_0$\\
$\therefore \Phi^{(0)}_{2m} = \frac{1}{\sqrt{\pi}} \cos[m(\varphi-\varphi_0)] $
\subsection{四.}
解:
$t > 0$ 时,总能量算符 $H = H_0 + H'$ 波函数$\psi(x,t)$满足方程。$$i\hbar \frac{\partial}{\partial t}\psi(x,t) = H \psi(x,t)(1)~$$\\
将$\psi(x,t)$按$H_0$定态波函数:$\psi_n(x) e^{-\frac{i}{\hbar} E_n t}$层开:\\
$$\psi(x,t) = \sum_{n} c_n(t) \psi_n(x) e^{-\frac{i}{\hbar} E_n t}(2)~$$\\
初始条件为 $t=0$,$\psi(x,0) = \psi_1(x)$,即 $c_n(0) = \delta_{n1}$。\\
将(2)式化入(1)式得:$$i\hbar \sum_n \frac{d c_n}{dt} \psi_n e^{-\frac{i}{\hbar} E_n t} = \sum_n H' \psi_n c_n e^{-\frac{i}{\hbar} E_n t}~$$\\
以 $\psi_n^*(x)$ 左乘上式,对全空问积分,并应用正文归一条件:$\int \psi_m^*(x) \psi_n(x) , dx = \delta_{mn}$\\
得: $$\frac{dC_m}{dt} e^{-\frac{i}{\hbar} E_m t} = \sum_n H'_{mn} C_n(t) e^{-\frac{i}{\hbar} E_n t}(3)~$$\\
其中绝对矩阵元:$$H'_{mn} = \psi_m | H' | \psi_n \rangle = \int \psi_m^*(x) H' \psi_n(x) , dx~$$\\
对一级近似(3)式中取$C_n(t) \approx C_n(0) = \delta_{n1},$
$$\text{得:}i\hbar\frac{dC_n}{dt} e^{-\frac{i}{\hbar} E_n t} = H'_{n1} e^{-\frac{i}{\hbar} E_1 t}~$$
$$\text{即:}i\hbar\frac{dC_n}{dt} = H'_{n1} e^{i\omega_{n1}t}, \quad \omega_{n1} = \frac{E_n - E_1}{\hbar}~$$
$$\text{解得:}C_n(t) = \frac{1}{i\hbar} \int_0^t H'_{n1}(t') e^{i\omega_{n1} t'} dt~$$
其中
\begin{align}
    H_{n1}^{'(t)} &= \int \psi_n^*(x) F(x) e^{-\frac{t}{\tau}} \psi_1(x) dx = e^{-\frac{t}{\tau}} \int \psi_n^*(x) F(x) \psi_1(x) dx \\\\
    &= e^{-\frac{t}{\tau}} F_{n1} \quad, F_{n1} = \int \psi_n^*(x) F(x) \psi_1(x) dx~
\end{align}
$$\therefore C_n(t) = \frac{F_{n1}}{i\hbar} \int_0^t.e^{-\frac{t}{\tau}} e^{i\omega_n t} dt \\\\
= \frac{F_{n1}}{i\hbar} \frac{e^{i\omega_{n1}} t{-\frac{t}{\tau}} - 1}{i\omega_{n1} - \frac{1}{\tau}}~$$
当 $t \gg \tau \text{ (} t \rightarrow \infty \text{) 时,} \quad e^{i \omega_{nl} t - \frac{t}{\tau}} \rightarrow 1$\\
$$\therefore \quad C_n(\infty) = \frac{F_{nl}}{E_n - E_l + i \hbar \frac{1}{\tau}}.~$$\\
厚子处于$\psi_n$的概率为:
$$W = \left| C_n(\infty) \right|^2 = \frac{|F_{n1}|^2}{(E_n - E_1)^2 + \left(\frac{\hbar^2}{\tau^2}\right)} = \frac{\left|\int \psi_n^*(x) F(x) \psi_1(x) dx\right|^2}{(E_n - E_1)^2 + \frac{\hbar^2}{\tau^2}}.~$$
