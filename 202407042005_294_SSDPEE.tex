% 首都师范大学 2001 年硕士研究生入学考试实体
% keys 首都师范大学|考研|物理
% license Copy
% type Tutor

\begin{enumerate}
\item 水平地面上有质量为$M$、倾角为$\theta$的斜面,其上有质量为$m$的物体。初始二者静止,然后物体$m$由斜面$M$的顶端$h$处滑下至地面。不考虑所有摩擦和阻力。(如图1)以地面为参考系,试求:\\
(1)当物体$m$刚落地时,斜面$M$的速度。\\
(2)在上述运动过程中,斜面$M$对于物体$m$的支持力所作的功。
\begin{figure}[ht]
\centering
\includegraphics[width=8cm]{./figures/e4120f7227680533.png}
\caption{} \label{fig_SSDPEE_1}
\end{figure}
\item 在倾角为0的固定斜面上有一弹性系数为$k$的轻质弹簧,其一端固定,另一端与质量为$m$的实心圆柱体的轴连接,圆柱体在固定斜面上做纯滚动。忽略圆柱体轴承的摩擦。(如图2)试求圆柱体振动的周期。
\begin{figure}[ht]
\centering
\includegraphics[width=8cm]{./figures/07be73ade6b798de.png}
\caption{} \label{fig_SSDPEE_2}
\end{figure}
\item 一根铜导线电阻率为$p_1$,一根铁导线电阻率为$p_2$,长度均为$L$,直径均为$d$,把两者首尾连接起来并在复合导线两端之间施加电位差 $V$。试求:\\
(1)每根导线中的电流密度。\\
(2)每根导线中的电场强度。\\
(3)每根导线两端之间的电位差。
\item 一个水平放置的档板中间有一个小孔$S$,一个质量为 $m$、电量为$q$的粒子从小孔进入档板以下空间。设粒子初速度为0,档板以下为足够大的均匀磁场区域,其磁感应强度为$\vec{B}$。(如图3)。求在重力和磁场力的共同作用下,粒子的运动速度和位置坐标。
\begin{figure}[ht]
\centering
\includegraphics[width=10cm]{./figures/28d8342304a10ccd.png}
\caption{} \label{fig_SSDPEE_3}
\end{figure}
\item (1)根据氢原子精细结构理论,绘出巴耳末系第一条谱线的精细结构能级跃迁图,并标出各能级相应的原子态符号。\\
(2)若将上述光谱置于磁感应强度为$B$的弱外磁场中,绘图说明上述光谱中波长最长的一条,其能级和谱线将如何分裂。
\item 什么是原子核的结合能?各种原子核中核子的平均结合能随核质量数A的变化有何规律?该规律对原子(核)能的利用有什么启示?
\end{enumerate}