% 中科大2014年考研普通物理
% 中科大|考研|普通物理

\subsection{简答题(每题10分,共10分)}
\begin{enumerate}
\item 随着气候的变暖,地球上的冰雪会融化流入海洋,试说明这是否对地球的自转速度有影响.若有影响,是什么样的影响.
\end{enumerate}
\subsection{计算题(每题20分,共140分)}
\begin{enumerate}
\item 将真空长直管沿竖直方向放置,自其中0点向上抛小球又落至原处所用的时间为$T_2$.在小球运动过程中经过比0点高$H$的$\mathrm A$处,小球离开$\mathrm A$处到达最高点又落回到$\mathrm A$处所用时间为T,现测得$T_1$、$T_2$和$H$,试确定重力加速度$g$.
\item 质点在平面上的运动轨道为对数螺旋线r=be,=c,其中b、k、c均为正常数.t=0时,质点位于r=b、φ=0处,请给出极坐标下质点速度和加速度大小以及轨道曲率半径随时间的变化.
\item 已知某气体的膨胀系数∝和等温系数x分别为:1+7其中a为常数,求该气体的状态方程.
\item 1mol范德瓦尔斯气体向真空自由膨胀,其体积从V增加到2V,气体的初始温度为T,求气体的熵变(设气体的摩尔定容比热容C为常数).
\item 一个平板电容器(理想导体平行板)板间间隔为d,中间填充了两层物质(1)和(2).第一层具有介电常数61,电导率为σ1,第二层为2,a2.他们的厚度分别为d和d2,d1+d2=d.加在电容器两端的电位是V.\\
(a)忽略边缘效应,物质(1)和(2)两层中的电场是多大?\\
(b)流过电容器的电流是多少?(c)在(1)和(2)层界面上的总面电荷密度是多少?
\item 6.一个长为L的圆筒形电容器由一半径为a的内层导线和一半径为b的外部薄导体壳构成.内外层之间的空间填以介电常数为E的绝缘材料.\\
求:(a)当电容器带电荷Q时,求电场作为径向位置的函数.略去边缘效应(b)求电容大小

\end{enumerate}