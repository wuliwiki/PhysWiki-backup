% 幂函数(高中)
% keys 幂函数
% license Usr
% type Tutor

\begin{issues}
\issueDraft
\end{issues}

\pentry{函数\nref{nod_functi},函数的性质\nref{nod_HsFunC},}{nod_9bbd}

在初中阶段,已经学过的几类典型的函数形式:正比例函数$f(x) = ax$、反比例函数$f(x) = \frac{k}{x}$和二次函数$f(x) = ax^2+bx+c$,其实都是幂函数的特例。可以这么说,幂函数是初中唯一接触过的函数。在高中阶段,将扩展这些知识,进一步研究幂函数的更一般形式。

\begin{definition}{幂函数}
形如
\begin{equation}
f(x) = x^a~.
\end{equation}
的函数称作\textbf{幂函数},其中 $a\in\mathbb Q$。
\end{definition}
幂函数的参数在指数位置,自变量在底数位置,幂函数的名称指的就是自变量的幂次是函数值,注意不要与指数函数相混淆。$a$会影响函数的性质的各个方面,在学习时需要时刻注意。本文除最后一章,全都默认$a\in\mathbb Q$。

由于幂函数的样式很多,直接研究每个具体的形式会很混乱。但对$y=x^a$,在$x>0$时,总是有定义的,而且此时$y$一定为正。因此,函数一定会在第一象限有图象。本文会先讨论定义域以及奇偶性,奠定好研究基础,然后再专注研究其在第一象限的规律。这也反映了一般分析函数的过程。

\subsection{定义域与奇偶性}

取$\displaystyle a=\frac{n}{m}$($n,m$互质,$m>0$)的形式进行讨论。

\subsubsection{定义域}

通常来讲,幂函数的定义域是$x\in\mathbb{R}$,但是在一些特殊情况时,定义域需要调整。

首先,$0$当分母的运算是非法的,为了逻辑一致性,幂运算要求$0^{-k},k\in \mathbb{N}$时不存在。因此,如果$n<0$,则幂函数的定义域为$({-\infty},0)\cup(0,{+\infty})$。

这里暂停讨论一下$0^0$,尽管在$x\neq0$时,$x^0=1$和$0^x=0$都成立,严格来说,$0^0$在高中阶段是未定义的,但在大部分场合(例如幂级数展开)会默认其值为$1$。事实上,如果从幂运算的最基础定义来讲,一般会先定义$0^0=1$来保证归纳得到的其他幂次不会出现多余的参数。这里也可以理解为,幂运算是比指数运算更底层的运算,因此要优先保证幂运算$x^0=1$成立。

其次,在实数域上负数的偶数次方根均无意义,为了逻辑一致性,幂运算要求$\displaystyle(-x)^\frac{1}{2k}$,在$k\in \mathbb{N},x\in\mathbb{R}^+$时不存在。因此如果${m}$为偶数,则幂函数的定义域为$[0,{+\infty})$。

\subsubsection{奇偶性讨论}

根据定义域的讨论,只有m为奇数时,定义域才是关于$0$对称的,幂运算的特性,一般而言,幂函数分为三类:
\begin{itemize}
\item $x\in\mathbb{R}$,且图象在一、二象限的偶函数:$n>0$,$n$是偶数,$m$是奇数。
\item $x\in\mathbb{R}$,且图象在一、三象限的奇函数:$n>0$,$n$是奇数,$m$是奇数。
\item $x\neq 0$,且图象在一、二象限的定义在一二象限的一是$n<0$,二是

\item 只在第一象限有定义:$n>0$且$m$是偶数。
\end{itemize}


1. 当 $a$ 为正整数时:

	•	如果 $a$ 为正奇数,则幂函数是奇函数,满足 $f(-x) = -f(x)$,其图像关于原点对称。例如 $f(x) = x^3$。在这种情况下,函数在整个实数域上都有定义,并且随着 $x$ 的增大,函数值快速增大  。
	•	如果 $a$ 为正偶数,则幂函数是偶函数,满足 $f(-x) = f(x)$,其图像关于 $y$ 轴对称。例如 $f(x) = x^2$。此类函数在整个实数域上定义,但其图像在 $x = 0$ 两侧对称,并且所有负数输入的结果都与正数相同  。

2. 当 $a$ 为负整数时:

	•	如果 $a$ 是负整数,例如 $f(x) = x^{-1}$(即反比例函数),此类函数的图像只存在于第一象限和第四象限。它在 $x=0$ 处没有定义,并且随着 $x$ 的增大,函数值迅速减小,趋近于 $0$。负整数次幂通常用于描述反比例关系 。

3. 当 $a$ 为分数时:

	•	当 $a = \frac{p}{q}$,$p$ 和 $q$ 是互质的整数时,幂函数的定义域取决于 $q$ 的奇偶性。如果 $q$ 为奇数,则 $f(x)$ 在整个实数域上有定义;如果 $q$ 为偶数,则 $f(x)$ 仅在 $x \geq 0$ 时有定义,因为负数的偶次根在实数域中没有定义。例如 $f(x) = x^{\frac{1}{2}}$ 仅在非负数上有定义,而 $f(x) = x^{\frac{1}{3}}$ 则在整个实数域上定义  。

4. 当 $a$ 为负分数时:

	•	如果 $a$ 是负分数,如 $f(x) = x^{-\frac{1}{2}}$,则函数值仅在 $x > 0$ 时有定义。此类函数的行为类似于反比例函数,其图像仅在第一象限存在,并且随着 $x$ 增加,函数值逐渐减小 。

5. 特殊情况:$a = 0$

	•	当 $a = 0$ 时,幂函数退化为常数函数 $f(x) = 1$,该函数在所有 $x$ 上的值恒为 1,是一个水平直线图像 。



\subsection{幂函数在第一象限的行为研究}

我们先来看实参数的幂函数 $f(x) = x^a$ 在 $a\in\mathbb Q$ 和 $x > 0$ 时函数曲线如\autoref{fig_power_1} 所示。 注意在该区间 $x^{1/a}$ 是 $x^a$ 的反函数。

\begin{figure}[ht]
\centering
\includegraphics[width=8cm]{./figures/86604297d1436480.pdf}
\caption{实参数的幂函数(相同颜色的函数互为反函数)}\label{fig_power_1}
\end{figure}

由图可知, 对正数次幂($a > 0$), 其定义域可以包含 $0$, 且 $0^a = 0$。

显然, 负数的整数次幂是良好定义的, 因为这只涉及实数的乘法运算。 当 $a$ 为偶数时, $x^a = (-x)^a$ 是偶函数, $a$ 为奇数时, $x^a = -(-x)^a$ 是奇函数。 这样我们就可以把(TODO:插图) 中整数次幂的曲线根据对称性延申到负半轴。

\subsection{*复数域扩展}

\pentry{复数\nref{nod_CplxNo},三角函数\nref{nod_HsTrFu}}{nod_b678}

在实数域上,无理数次幂和偶次方根只定义在正数上。因此,函数只在第一象限有图象,但是根据欧拉公式,引入复数后将函数拓展到复数域中可以有一些新知,下面略窥这些新想法。请注意,下面的内容在高中阶段完全不需要理解,甚至在本科基础阶段都不会涉及。此处给出只是为了扩展视野,如果想要具体了解需要学习\enref{复变函数}{Cplx}。

首先,不加证明地给出\textbf{欧拉公式}\footnote{如果代入$x=\pi$,则会得到恒等式$e^{i\pi}+1=0$,这被很多人称为数学中最优美的等式之一。}:

\begin{theorem}{欧拉公式}
\begin{equation}
\forall x\in\mathbb{R},e^{ix} = \cos(x) + i\sin(x)~.
\end{equation}
\end{theorem}

注意到,等号右侧是一个模为$1$的复数。因此,任意复数可以表示为 $z = r (\cos \theta + i \sin \theta)$,称$r$ 是复数$z$的\textbf{模},$\theta$ 是复数与实轴的夹角,称为\textbf{辐角},由于正弦与余弦都是周期函数,所以$\theta+2k\pi,k\in\mathbb{Z}$都是辐角,称呼$\theta\in[0,2\pi)$为\textbf{辐角主值}。于是,通过欧拉公式有\textbf{复数的指数形式}:

\begin{equation}
z = r e^{i(\theta+2k\pi)},k\in\mathbb{Z}~.
\end{equation}

\subsubsection{负数偶次方根的情况}

假设要计算负负实数的偶次方根 $(-x)^{\frac{1}{4}},x>0$,由于 $-x$的辐角为 $\pi$,模为$|x|$,将 $-x$ 写成指数形式为:
\begin{equation}
-x = |x| e^{i\pi(1+2k)},k\in\mathbb{Z}~.
\end{equation}

接下来,使用幂运算公式:

\begin{equation}
\displaystyle
(-x)^{\frac{1}{4}} = \left( |x| e^{i\pi(1+2k)} \right)^{\frac{1}{4}} =|x|^{\frac{1}{4}} \cdot e^{i\pi\frac{1+2k}{4}}=|x|^{\frac{\pi}{4}}\cos(\frac{1+2k}{4}\pi) + i|x|^{\frac{\pi}{4}}\sin(\frac{1+2k}{4}\pi),k\in\mathbb{Z}~.
\end{equation}

由于正、余弦函数的周期性,与无理数部分不同,这只会产生四个不同的解,分别为:

\begin{itemize}
\item $\displaystyle k = 0,\quad |x|^{\frac{1}{4}} \left( \cos\frac{\pi}{4} + i\sin\frac{\pi}{4} \right) = |x|^{\frac{1}{4}} \cdot \frac{\sqrt{2}}{2} \left( 1 + i \right)$
\item $\displaystyle k = 1,\quad |x|^{\frac{1}{4}} \left( \cos\frac{3\pi}{4} + i\sin\frac{3\pi}{4} \right) = |x|^{\frac{1}{4}} \cdot \frac{\sqrt{2}}{2} \left( -1 +i \right)$
\item $\displaystyle k = 2,\quad |x|^{\frac{1}{4}} \left( \cos\frac{5\pi}{4} + i\sin\frac{5\pi}{4} \right) = |x|^{\frac{1}{4}} \cdot \frac{\sqrt{2}}{2} \left( -1 - i \right)$
\item $\displaystyle k = 3,\quad |x|^{\frac{1}{4}} \left( \cos\frac{7\pi}{4} + i\sin\frac{7\pi}{4} \right) = |x|^{\frac{1}{4}} \cdot \frac{\sqrt{2}}{2} \left( 1 - i \right)$
\end{itemize}

其实,有理数次幂函数 $x^{n/m}$ ($x\in \mathbb R$, $n$ 为整数, $m$ 为正整数) 总是有 $m$ 个可能的值
\begin{equation}
x^{n/m} = \leftgroup{
&\abs{x^n}^{1/m}\E^{\I 2\pi k/m} & (x^n > 0)\\
&\abs{x^n}^{1/m}\E^{\I 2\pi (k+1/2)/m} & (x^n < 0)
}\qquad (k = 0,1,\dots, m-1)~.
\end{equation}

\subsubsection{无理数次幂}

假设要计算负实数的无理数($\pi$)次幂 $(-x)^\pi,(x>0)$。由于 $-x$的辐角为 $\pi$,模为$|x|$,将 $-x$ 写成指数形式为:
\begin{equation}
-x = |x| e^{i\pi(1+2k)},k\in\mathbb{Z}~.
\end{equation}

接下来,使用幂运算公式:

\begin{equation}
(-x)^\pi = \left( |x| e^{i\pi(1+2k)} \right)^\pi =|x|^\pi \cdot e^{i\pi^2(1+2k)}=|x|^\pi\cos(\pi^2(1+2k)) + i|x|^\pi\sin(\pi^2(1+2k)),k\in\mathbb{Z}~.
\end{equation}

可以看出,负实数的$\pi$次幂是复数,且是由于$\pi^2(1+2k)$与正、余弦函数的周期$2\pi$没有最小公倍数,使得得到的复数有无穷多个。其他无理数次幂也依此计算。事实上,在这样的观点下,正实数的无理数次幂,除了幅角主值对应的是一个实数之外,其余的都是复数。

上面的内容带你对复变函数初窥门径,希望能够让你感受到数学的魅力。