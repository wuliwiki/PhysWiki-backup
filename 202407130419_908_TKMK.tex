% 托卡马克
% license CCBYSA3
% type Wiki

(本文根据 CC-BY-SA 协议转载自原搜狗科学百科对英文维基百科的翻译)

\textbf{托卡马克}(俄语:Токамáк)是一种使用强磁场将热的等离子体限制为圆环形状的装置,是为了产生受控核聚变而开发的几种磁约束装置之一。[1]

托卡马克最初是在20世纪50年代由苏联物理学家伊戈尔·塔姆(Igor Tamm)和安德烈·萨哈罗夫(Andrei Sakharov)根据奥列格·拉夫伦蒂耶夫(Oleg Lavrentiev)的一封信提出的概念。同时,托卡马克的第一个工作被认为是Natan Yavlinskii在T-1上的工作。已经证明,稳定的等离子体平衡需要绕在环面上呈螺旋状的磁力线。像z-pinch和stellarator这样的设备已经尝试过这样做,但显示出严重的不稳定性。现在被称为安全系数(数学标记为q)的概念的发展引导了托卡马克的发展;通过将反应堆的关键因素q设置为大于1,托卡马克装置有效地抑制了早期设计的不稳定性。

第一台托卡马克T-1于1958年开始运行。到20世纪60年代中期,托卡马克的设计开始显示出性能的极大提高。最初的结果于1965年公布,但被忽略了;莱曼·斯皮策在注意到他们测量温度的系统存在潜在问题后,立即将其驳回。第二组结果发表于1968年,这一次声称性能远远领先于任何其他机器,同样被认为是不可靠的。这导致了联合王国的一个代表团被邀请来进行他们自己的测量。这些结果证实了苏联的结论,并在1969年发表,导致各国疯狂建设托卡马克装置。

到20世纪70年代中期,世界各地使用了数十台托卡马克。到20世纪70年代末,这些机器已经达到了实际聚变所需的所有条件,尽管不是同时进行,也不是在一个反应堆中进行。随着盈亏平衡的目标在望,一系列新的机器被设计出来,它们将使用氘和氚的聚变燃料运行。这些机器,特别是欧洲联合环流器(喷气)、托卡马克聚变试验堆 (TFTR)和JT-60 ,有明确的目标达到盈亏平衡。

相反,这些机器显现出了限制其性能的新问题。解决这些问题需要更大,更昂贵的机器,超出了任何一个国家的能力。在1985年11月罗纳德·里根和米哈伊尔·戈尔巴乔夫达成初步协议后,国际热核聚变实验堆计划 (ITER)项目应运而生,并且仍然是发展实际聚变发电的主要国际合作。许多较小的设计和分支,如球形托卡马克,继续用于研究性能参数和其他问题。

\subsection{语源}
托卡马克这个词是俄语单词токамак的音译,是以下两者的首字母缩写:
\begin{itemize}
\item "тороидальнаясасесассагнитнымикатушками"(带磁性线圈的环形室;
\end{itemize}
或者
\begin{itemize}
\item "тороидальнаясамесасаксиальным магнитным полем"(具有轴向磁场的环形腔体。[2]
\end{itemize}
这个术语是由伊戈尔·戈洛温在1957年提出的,他是科学院测量仪器实验室的副主任,现在的库尔恰托夫研究所。类似的术语“tokamag”也曾被提出过一段时间。

\subsection{历史}
\begin{figure}[ht]
\centering
\includegraphics[width=6cm]{./figures/a022bdad1bb07f4d.png}
\caption{苏联邮票,1987:托卡马克热核系统} \label{fig_TKMK_1}
\end{figure}
\subsubsection{2.1 第一步}
1934年,马克·奥列芬特、保罗·哈特克和欧内斯特·卢瑟福是第一个在地球上实现聚变的人,他们使用粒子加速器将氘原子核射入含有氘或其他原子的金属箔中。[3]这使他们能够测量各种聚变反应的核截面,并确定氘-氘反应发生的能量低于其他反应,峰值约为10万电子伏特(100 keV)。[4]

基于加速器的聚变是不现实的,因为反应截面很小;加速器里的大部分粒子会从燃料中散开,而不是与燃料融合。这些散射使粒子失去能量,达到不再发生聚变的程度。因此,注入这些粒子的能量就消失了,很容易证明这比由此产生的聚变反应释放的能量要多得多。[5]

为了维持聚变和产生净能量输出,燃料的大部分必须提高到高温,这样它的原子就会不断高速碰撞;这就产生了“热核”这个名字,因为产生它需要很高的温度。1944年,恩里科·费米(Enrico Fermi)计算出,在大约5000万K的情况下,这个反应是可以自我维持的;在那个温度下,反应释放能量的速率足够高,使周围的燃料迅速升温,以保持温度不受环境损失,继续进行反应。[5]

在曼哈顿计划期间,第一个达到这些温度的实际方法是用原子弹创造的。1944年,费米在当时假设的氢弹背景下发表了一场关于核聚变物理学的演讲。然而,人们已经想到了一种可控核聚变装置,吉姆·塔克和斯坦尼斯拉夫·乌兰姆曾尝试使用聚能炸药驱动注入氘的金属箔,尽管没有成功。[6]

第一次尝试制造实用的核聚变机器是在英国,乔治·佩吉特·汤姆森在1945年选择了箍缩效应作为一种有前途的技术。在几次试图获得资助失败后,他放弃了,并要求两名研究生斯坦·卡曾斯(Stan Cousins)和艾伦·韦尔(Alan Ware)用剩余的雷达设备制造一种设备。该系统于1948年成功运行,但没有明确的证据表明核聚变,未能获得原子能研究机构的兴趣。[7]
\subsubsection{2.2 奥列格·拉夫罗夫(Oleg Lavrentiev)的信}
1950年,时驻扎在库页岛的一名无事可做的红军中士奥列格·拉夫罗夫给苏联共产党中央委员会写了一封信。信中概述了用原子弹点燃聚变燃料的想法,然后描述了一个利用静电场来控制稳定状态的热等离子体以产生能量的系统。[8][9]

这封信被送往安德烈·德米特里耶维奇·萨哈罗夫那里征求意见。萨哈罗夫指出“作者提出了一个非常重要但不一定是无望的问题”,并发现他在构想中主要关心的是等离子会撞击电极线,“宽网格和薄载流部分将几乎所有入射的原子核反射回到反应堆中。这一要求很可能与设备的机械强度不相容。”[8]

从处理的速度可以看出拉夫罗夫的信的重要性;这封信在7月29日被中央委员会收到,萨哈罗夫在8月18日发出了他的评论,到10月,萨哈罗夫和伊戈尔·塔姆已经完成了第一个关于核聚变反应堆的详细研究,他们在1951年1月申请了建造资金。[10]
\subsubsection{2.3 磁约束}
当加热到熔化温度时,原子中的电子电离,形成一种原子核和电子的流体,称为等离子体。与电中性原子不同,等离子体是导电的,因此可以被电场或磁场控制。[11]

萨哈罗夫对电极的担心导致他考虑使用磁约束代替静电。在磁场的情况下,粒子将围绕磁力线旋转。[11]当粒子高速运动时,它们产生的路径看起来像螺旋。如果排列的磁场使得磁力线平行且靠近,围绕相邻磁力线的粒子可能会碰撞并融合。[12]

这种磁场可以在螺线管中产生,螺线管是外部缠绕有磁体的圆柱体。磁体的组合磁场产生了一组沿着圆柱体长度延伸的平行磁力线。这种布置防止粒子侧向移动到圆柱体的壁上,但并不防止它们从末端跑出。这种排列方式可以防止颗粒向圆筒壁面的侧面移动,但却不能防止颗粒跑出圆筒的末端。解决这个问题的一个明显的办法是把圆柱体弯曲成一个甜甜圈的形状,这样这些线条就形成了一系列连续的圆环。在这种排列中,粒子无休止地循环。[12]

萨哈罗夫与伊戈尔·塔姆(Igor Tamm)讨论了这个概念,到1950年10月底,两人写了一份提案,并把它寄给苏联原子弹项目主管伊戈尔·库尔恰托夫(Igor Kurchatov)及其副手伊戈尔·戈洛文(Igor Golovin)。[12]但是,这一初步建议忽略了一个基本问题;当沿直线螺线管排列时,外部磁铁的间隔是均匀的,但当它们弯曲成环形时,它们在环内比在环外靠得更近。这导致不均匀的力量,导致粒子漂移远离他们的磁力线。[13][13]

萨哈罗夫在访问苏联核研究中心苏联科学院测量仪器实验室期间,提出了解决这一问题的两个可能办法。一种方法是在环面中心悬挂载流环。环中的电流会产生一个磁场,这个磁场会与外部磁铁产生的磁场混合。由此产生的磁场将被扭曲成螺旋状,因此任何给定的粒子都会发现自己反复出现在环面内外。由不均匀场引起的漂移在内外方向相反,所以在环绕环面长轴的多次轨道运行过程中,相反的漂移会相互抵消。另外,他建议使用外部磁铁在等离子体中产生电流,而不是使用单独的金属环,这样也会产生同样的效果。[13]

1951年1月,库尔恰托夫在利潘安排了一次会议,讨论萨哈罗夫的概念。他们获得了广泛的兴趣和支持,2月,一份关于这一主题的报告被转交给了监督苏联原子能工作的拉夫连季·帕夫洛维奇·贝利亚。有一段时间,没有任何回音。[13]
\subsubsection{2.4 里克特和融合研究的诞生}
1951年3月25日,阿根廷总统胡安·裴隆宣布,前德国科学家罗纳德·里克特已经成功地在实验室规模上生产聚变,这是现在称为休穆尔项目的一部分。世界各地的科学家都对这一消息感到兴奋,但很快得出结论:这不是真的;简单的计算表明,他的实验装置不能产生足够的能量将聚变燃料加热到所需的温度。[14]

尽管遭到核研究人员的驳斥,但广泛的新闻报道意味着政治家们突然意识到并接受了聚变研究。在英国,屡次遭到拒绝的汤姆森突然获得了大量的研究资金。接下来的几个月里,有两个基于pinch系统的项目开始运行。[15]在美国,莱曼·斯必泽读了休穆尔的故事,意识到它是错误的,并着手设计一台可以工作的机器。[16]5月,他获得了5万美元,开始研究他的仿星器概念。[17]吉姆•塔克(Jim Tuck)曾短暂返回英国,参观了汤姆森的缩放机。当他回到洛斯阿拉莫斯的时候,他也和斯皮策同时申请了资金,但是被拒绝了。但直接从洛斯阿拉莫斯的预算中得到了5万美元。[18]

苏联也发生了类似的事件。4月中旬,电物理仪器科学研究所的德米特里·埃夫雷莫夫带着一本杂志闯入库尔恰托夫的研究,杂志上有一篇关于里克特工作的报道,要求知道他们为什么被阿根廷人打败。库尔恰托夫立即联系了贝利亚,提议建立一个独立的聚变研究实验室,由列夫·阿特莫维奇(Lev Artsimovich)担任主任。仅仅几天之后,即5月5日,约瑟夫·斯大林签署了这项建议。[13]
\subsubsection{2.5 新想法}
到10月份,萨哈罗夫和塔姆已经完成了对他们最初提议的更详细的考虑,要求一个整个圆环的外半径为12米,内半径为2米的设备。该系统每天可以生产100克(3.5盎司)氚,或每天生产10公斤(22磅)U233[13]

随着这一想法的进一步发展,人们认识到等离子体中的电流可以产生一个足够强的磁场来限制等离子体,从而消除了对外部磁体的需求。在这一点上,苏联研究人员重新发明了英国正在开发的箍缩系统,[6]尽管他们是从一个截然不同的起点开始设计的。

一旦提出使用箍缩效应,一个简单得多的解决方案就显而易见了。人们可以简单地将电流导入一个线性管中,而不是一个大的环形管,这将导致管内的等离子体坍缩成一条细丝。这有一个巨大的优势;等离子体中的电流会通过正常的电阻加热使其升温,但不会将等离子体加热到熔化温度。然而,当等离子体崩溃时,绝热过程将导致温度急剧上升,远远超过聚变所需的温度。随着这一发展,只有戈洛温和纳坦·亚夫林斯基继续考虑更静态的环形布置。
\subsubsection{2.6 不稳定性}
1952年7月4日,尼古拉·菲利波夫(Nikolai Filippov)的团队测量了一台线性箍缩机释放的中子。列夫·阿齐莫维奇要求他们在得出核聚变发生的结论之前检查所有的东西,在这些检查中,他们发现中子根本不是聚变产生的。英国和美国的研究人员也经历了同样的线性排列,他们的机器表现出了同样的行为。但是这项研究的高度保密意味着没有一个小组意识到其他小组正在研究,更不用说有同样的问题了。[19]

经过大量研究,发现中子是由等离子体的不稳定性引起的。有两种常见的不稳定性“香肠”主要出现在线性机器中,“扭结”在环面机器中最常见。[19]这三个国家的团队都开始研究这些不稳定性的形成以及解决这些不稳定性的潜在方法。[20]美国的马丁·大卫·克鲁斯卡尔、马丁·史瓦西和苏联的沙夫拉诺夫对该领域做出了重要贡献。[21]

来自这些研究的一个想法被称为“稳定夹点”。这一概念在腔室的外部增加了额外的磁体,将出现在等离子体之前的缩放放电。在大多数概念中,外场相对较弱,因为等离子体是抗磁性的,所以它只穿透等离子体的外部区域。[19]当收缩放电发生时,等离子体迅速收缩,这个场被“冻结”在产生的细丝上,在其外层产生一个强场。在美国,这被称为“给等离子体一个主干”[22]

萨哈罗夫重新审视了他最初的环形概念,并对如何稳定等离子体得出了略有不同的结论。布局将与稳定收缩概念相同,但两个场的作用将相反。在新的布局中,外部磁体将更强大,以提供大部分限制,而不是提供稳定的弱外部场和负责限制的强收缩电流,而电流将更小,并负责稳定效果。
\subsubsection{2.7 步骤解密}
1955年,随着线性方法仍然受到不稳定性的影响,苏联制造了第一个环形装置。TMP是一种经典的箍缩机,与英美同时期的机型相似。真空室由陶瓷制成,放电光谱显示是二氧化硅,这意味着等离子体没有完全被磁场限制,并撞击了室壁。随后出现了两台使用铜质外壳的小型机器。[23]这种导电外壳原本是用来稳定等离子体的,但在任何一台尝试过的机器上都没有完全成功。[23]

随着进展明显停滞,库尔恰托夫于1955年召开了苏联研究人员全联盟会议,其最终目标是在苏联内部开展核聚变研究。[24]1956年4月,作为尼古拉·赫鲁晓夫和尼古拉·布尔加宁广泛宣传访问的一部分,库尔恰托夫前往英国。他主动提出要在前英国皇家空军(RAF)哈维尔(Harwell)的原子能研究机构(Atomic Energy Research institution)做一次演讲。在那里,他详细介绍了苏联核聚变努力的历史概况,震惊了主办方。[30]他花了一些时间,特别注意到早期机器中出现的中子,并警告说,中子并不意味着聚变。[25]

库尔恰托夫不知道,英国的 ZETA 稳定箍缩机正在原跑道的远端建造。当时,ZETA是最大、最强的聚变机器。ZETA得到了早期设计实验的支持,这些实验已经被修改为包括稳定性,旨在产生低水平的聚变反应。这显然是一个巨大的成功,1958年1月,他们宣布基于中子释放和等离子体温度的测量,在ZETA中实现了聚变。[26]

维塔利·沙夫拉诺夫和斯坦尼斯拉夫·布拉金斯基研究了新闻报道,并试图弄清楚其中的原理。他们考虑的一种可能性是使用弱“冻结”磁场,但最终拒绝了这种可能性,因为他们认为磁场不会持续足够长的时间。然后,他们得出结论,ZETA与他们研究的设备基本相同,具有强大的外磁场。[27]
\subsubsection{2.8 第一台托卡马克装置}