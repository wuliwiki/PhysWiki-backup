% 多元隐函数的存在定理
% 隐函数|存在定理
\pentry{一元隐函数的存在及可微定理\upref{ImFED}}
\begin{theorem}{n元隐函数存在定理}\label{Mulmp_the1}
若:\begin{enumerate}
\item 函数 $F(x_1,\cdots,x_n,y)$ 在以点 $(x_1^0,\cdots,x_n^0,y_0)$ 为中心的$n+1$ 维长方体
\begin{equation}
\mathcal{D}=[x_1^0-\Delta_1,x_1^0+\Delta_1;\cdots;x_n^0+\Delta_n;y_0-\Delta',y_0+\Delta']
\end{equation}
中有定义且连续;
\item 在 $\mathcal{D}$ 中偏导数 $F_{x_1}',\cdots,F_{x_n}',F_y'$ 存在且连续;
\item $F(x_1^0,\cdots,x_n^0,y_0)=0$ ;
\item $F'_y(x_1^0,\cdots,x_n^0,y_0)\neq 0$
\end{enumerate}
那么,在点 $(x_1^0,\cdots,x_n^0,y_0)$ 的某一邻域内:
\begin{enumerate}
\item 方程 $F(x_1,\cdots,x_n,y)=0$ 确定 $y$ 为 $x_1,\cdots,x_n$ 的单值函数: $y=f(x_1,\cdots,x_n)$;
\item $f(x_1^0,\cdots,x_n^0)=y_0$ ;
\item $f(x_1,\cdots,x_n)$ 关于其所有的变元连续,且
\item $f(x_1,\cdots,x_n)$ 有连续偏导数 $f_{x_1}',\cdots,f_{x_n}'$
\end{enumerate}
\end{theorem}

\begin{theorem}{最一般的隐函数存在定理}\label{Mulmp_the2}
若:\begin{enumerate}
\item 函数 $F_1,\cdots,F_m$ 在以点 $(x_1^0,\cdots,x_n^0,y_1^0,\cdots,y_m^0)$ 为中心的$n+m$ 维长方体
\begin{equation}
\mathcal{D}=[x_1^0-\Delta_1,x_1^0+\Delta_1;\cdots;x_n^0+\Delta_n;y_1^0-\Delta_1',y_1^0+\Delta_1';\cdots;y_m^0-\Delta_m',y_m^0+\Delta_m']
\end{equation}
中有定义且连续;
\item 在 $\mathcal{D}$ 中这些函数关于一切变元的偏导数存在且连续;
\item 在点 $(x_1^0,\cdots,x_n^0,y_1^0,\cdots,y_m^0)$ 处这些函数值都为0:
\begin{equation}\label{Mulmp_eq1}
F_1(x_1^0,\cdots,x_n^0,y_1^0,\cdots,y_m^0)=0,\cdots,F_m(x_1^0,\cdots,x_n^0,y_1^0,\cdots,y_m^0)=0
\end{equation}

\item 雅可比行列式 $\abs{\bvec J}$\upref{JcbDet}在点 $(x_1^0,\cdots,y_m^0)$ 不为0:
\begin{equation}\label{Mulmp_eq7}
\abs{\bvec J(x_1^0,\cdots,x_n^0,y_1^0,\cdots,y_m^0)}=\vmat{\pdv{F_1}{y_1}&\cdots &\pdv{F_1}{y_m}\\\vdots&\vdots&\vdots\\\pdv{F_m}{y_1}&\cdots&\pdv{F_m}{y_m}}_{(x_1^0,\cdots,x_n^0,y_1^0,\cdots,y_m^0)}\neq0
\end{equation}

\end{enumerate}
那么,在点 $(x_1^0,\cdots,x_n^0,y_1^0,\cdots,y_m^0)$ 的某一邻域内:
\begin{enumerate}
\item 方程组
\begin{equation}\label{Mulmp_eq4}
F_1(x_1,\cdots,x_n,y_1,\cdots,y_m)=0,\cdots,F_m(x_1,\cdots,x_n,y_1,\cdots,y_m)=0
\end{equation}
 确定 $y_1,\cdots,y_m$ 为 $x_1,\cdots,x_n$ 的单值函数: 
\begin{equation}
y_1=f_1(x_1,\cdots,x_n),\cdots,y_m=f_m(x_1,\cdots,x_n)
\end{equation}

\item $f_1(x_1^0,\cdots,x_n^0)=y_1^0,\cdots,f_m(x_1^0,\cdots,x_n^0)=y_m^0$ ;
\item 函数 $f_1,\cdots,f_m$ 连续,且
\item 函数 $f_1,\cdots,f_m$ 有关于一切变元的连续偏导数.
\end{enumerate}
\end{theorem}
\subsection{证明}
\autoref{Mulmp_the1} 的证明和一元隐函数\upref{ImFED}的完全类似.我们只证明\autoref{Mulmp_the2} .证明采用数学归纳法.

当 $m=1$ 时,即是\autoref{Mulmp_the1} 的情形,此时定理成立.

假设 $m-1$ 时定理成立,现在证明对 $m$ 时定理也成立.

由于雅可比行列式在点 $(x_1^0,\cdots,y_m^0)$ 不为0,那么,最后一行内至少有一个元素在这点不为0;例如设
\begin{equation}
\pdv{F_m(x_1^0,\cdots,x_n^0,y_1^0,\cdots,y_m^0)}{y_m}\neq0
\end{equation}
此时,按\autoref{Mulmp_the1} ,方程
\begin{equation}\label{Mulmp_eq2}
F_m(x_1,\cdots,x_n,y_1,\cdots,y_m)=0
\end{equation}
在点 $(x_1^0,\cdots,x_n^0,y_1^0,\cdots,y_m^0)=0$ 的某一邻域 $\mathcal{D}^*$ 内,确定 $y_m$ 为其余变元的单值函数:
\begin{equation}\label{Mulmp_eq3}
y_m=\phi(x_1,\cdots,x_n,y_1,\cdots,y_{m-1})
\end{equation}
这函数 $\phi$ 是连续的,且有连续偏导数;此外
\begin{equation}
\phi(x_1^0,\cdots,x_n^0,y_1^0,\cdots,y_{m-1}^0)=y_m^0
\end{equation}

由于下面我们讨论的邻域以 $\mathcal{D}^*$ 为限,\autoref{Mulmp_eq2} 和\autoref{Mulmp_eq3} 是等价的.

用\autoref{Mulmp_eq3} 代替\autoref{Mulmp_eq4} 中最后一方程,并把函数 $\phi$ 代入\autoref{Mulmp_eq4} 中其余方程,得到具有 $n+m-1$ 个变元的 $m-1$ 个方程的新方程组
\begin{equation}\label{Mulmp_eq5}
\varphi_1(x_1,\cdots,x_n,y_1,\cdots,y_{m-1})=0,\cdots,\varphi_{m-1}(x_1,\cdots,x_n,y_1,\cdots,y_{m-1})=0
\end{equation}
其中
\begin{equation}\label{Mulmp_eq6}\varphi_j(x_1,\cdots,x_n,y_1,\cdots,y_{m-1})=F_j(x_1,\cdots,x_n,y_1,\cdots,y_{m-1},\phi(x_1,\cdots,x_n,y_1,\cdots,y_{m-1}))
\end{equation}

在不越出 $\mathcal{D}^*$ 的邻域,方程组\autoref{Mulmp_eq4} 显然与方程组\autoref{Mulmp_eq5} 连同附加的方程\autoref{Mulmp_eq3} \textbf{等价}.因此,若在邻域 $\mathcal{D}^*$ 内, $m-1$ 个变元 $y_1,\cdots,y_{m-1}$ 为 $n$ 个变元 $x_1,\cdots,x_n$ 的单值函数,则由\autoref{Mulmp_eq3} ,变元 $y_m$ 也是单值函数,从而结论1成立.

根据\autoref{Mulmp_eq6} 和 $F_j,\phi$ 的性质, 函数组 $\varphi_1,\cdots,\varphi_{m-1}$ 满足于定理的条件1,2,3.

接下来证明
\begin{equation}\label{Mulmp_eq8}
\abs{\bvec J^*}=\vmat{\pdv{\varphi_1}{y_1}&\cdots &\pdv{\varphi_1}{y_{m-1}}\\\vdots&\vdots&\vdots\\\pdv{\varphi_{m-1}}{y_1}&\cdots&\pdv{\varphi_{m-1}}{y_{m-1}}}_{(x_1^0,\cdots,x_n^0,y_1^0,\cdots,y_{m-1}^0)}\neq0
\end{equation}
为此,考察雅可比行列式\autoref{Mulmp_eq7} ,依次用 $\pdv{\phi}{y_1},\pdv{\phi}{y_{m-1}}$ 乘第 $m$ 列,然后加到前 $m-1$ 列去,即
\begin{equation}
\abs{\bvec J}=\vmat{\pdv{F_1}{y_1}+\pdv{F_1}{y_m}\pdv{\phi}{y_1}&\cdots &\pdv{F_1}{y_{m-1}}+\pdv{F_1}{y_m}\pdv{\phi}{y_{m-1}}&\pdv{F_1}{y_m}\\\vdots&\vdots&\vdots&\vdots\\\pdv{F_m}{y_1}+\pdv{F_m}{y_m}\pdv{\phi}{y_1}&\cdots&\pdv{F_m}{y_1}+\pdv{F_m}{y_{m}}\pdv{\phi}{y_{m-1}}&\pdv{F_m}{y_m}}
\end{equation}
若视 $y_m=\phi(x_1,\cdots,y_{m-1})$,则除最后一行及最后一列以外的一切元素都是函数 $\varphi_j$ 的偏导数.另一方面,若分别对 $y_1,\cdots,y_{m-1}$ 而微分恒等式\autoref{Mulmp_eq2} ,于是最后一行除了最后一个外其它全为0
\begin{equation}
\abs{\bvec J}=\vmat{\pdv{\varphi_1}{y_1}&\cdots &\pdv{\varphi_1}{y_{m-1}}&\frac{F_1}{y_m}\\\vdots&\vdots&\vdots&\vdots\\0&\cdots&0&\pdv{F_m}{y_m}}
\end{equation}
于是
\begin{equation}
\abs{\bvec J}=\abs{\bvec J^*}\cdot \pdv{F_m}{y_m}
\end{equation}
令 $x_1=x_1^0,\cdots,y_{m-1}=y_{m-1}^0$,则 $y_m=\phi(x_1,\cdots,y_{m-1})$变为 $y_m^0$.但由定理条件4,$\abs{\bvec J}$ 不为0,故\autoref{Mulmp_eq8} 成立.

对于 $m-1$ 个方程组\autoref{Mulmp_eq5} ,定理已假定正确.因此,这方程组在点 $(x_1^0,\cdots,y_{m-1}^0)$ 邻域内确定 $y_1,\cdots,y_{m-1}$ 为 $(x_1,\cdots,x_n)$ 的单值函数
\begin{equation}
y_1=f_1(x_1,\cdots,x_n),\cdots,y_{m-1}=f_{m-1}(x_1,\cdots,x_n)
\end{equation}
它们是连续且有连续导数的.此外
\begin{equation}
f_1(x_1^0,\cdots,x_n^0)=y_1^0,\cdots,f_{m-1}(x_1^0,\cdots,x_n^0)=y_{m-1}^0
\end{equation}
由此推得,第 $m$ 个函数