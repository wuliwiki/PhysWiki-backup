% 序列的极限

\pentry{数列的极限(简明微积分)\upref{Lim0}, 极限\upref{Lim}, 实数集的拓扑\upref{ReTop}}

\subsection{基本定义与性质}

序列的极限是分析数学中最基本的定义. 词条 数列的极限(简明微积分)\upref{Lim0} 和 极限\upref{Lim} 已经给出了一些序列极限的例子, 它的形式定义以及背后的直观解释. 为完整起见, 这里再重复一次序列极限的定义:

\begin{definition}{数列的极限}
考虑数列$\{a_n\}$.若存在一个实数$A$,使得对于\textbf{任意}给定的\textbf{正实数} $\varepsilon > 0$(无论它有多么小),总存在正整数 $N_\epsilon$, 使得对于所有编号 $n>N_\epsilon$ ,都有 $\abs{a_n - A} < \varepsilon$ ($A$ 为常数) 成立,那么数列 $a_n$ 的极限就是 $A$.

将“数列$\{a_n\}$的极限是$A$”表示为$\lim\limits_{n\to\infty}a_n=A$.
\end{definition}

正如之前两个词条所解释的, 等式$\lim\limits_{n\to\infty}a_n=A$所表达的含义是"序列$a_n$随着$n$的增大将可以任意地接近$A$"". 或者说, 对于序列$\{a_n\}$进行极限运算, 就是要找到"序列$a_n$越来越接近的那个数". 这种运算显然跟实数的四则运算不一样.

\subsection{柯西序列; 柯西收敛准则}

\subsection{序列的上极限与下极限}