% Volkov 波函数
% 波函数|本征波函数|偶极子近似|速度规范|库仑势

\pentry{加速度规范\upref{AccGau}}

本文使用原子单位\upref{AU}. 速度规范\upref{LVgaug}下, Volkov 波函数为
\begin{equation}\label{Volkov_eq1}
\Psi_{\bvec k}^V(\bvec r, t) = (2\pi)^{-3/2} \exp\qty[\I \bvec k \vdot (\bvec r - \bvec \alpha(t)) - \I Et/\hbar]
\end{equation}
其中 $\bvec \alpha(t)$ 对应的是一个经典点电荷在电场中的位移.
\begin{equation}
\bvec \alpha(t) = -\frac{q}{m} \int_{-\infty}^t \bvec A(t') \dd{t'}
\end{equation}
这是以下薛定谔方程的解(速度规范,偶极子近似)
\begin{equation}
\I\hbar \pdv{t} \Psi^V = \qty[\frac{\bvec p^2}{2m} - \frac{q}{m}\bvec A(t) \vdot \bvec p] \Psi^V
\end{equation}

要求长度规范只需要做一个规范变换即可(\autoref{LVgaug_eq3}~\upref{LVgaug} 和\autoref{LVgaug_eq4}~\upref{LVgaug}).

\subsubsection{长度规范}
长度规范下, 有 $E = \bvec k^2/2$, 结合\autoref{LVgaug_eq6}~\upref{LVgaug}得
\begin{equation}
\Psi_{\bvec k}^L(\bvec r, t) = (2\pi)^{-3/2} \exp\qty[\bvec p(t)\vdot \bvec r - \frac{1}{2}\int_{-\infty}^t \bvec p(t')^2 \dd{t'}]
\end{equation}
其中 $\bvec p(t)$ 是广义动量(\autoref{QMEM_eq6}~\upref{QMEM})
\begin{equation}
\bvec p(t) = \bvec k + q\bvec A(t)
\end{equation}

\subsubsection{加速度规范}
使用\autoref{Volkov_eq1} 和\autoref{AccGau_eq1}~\upref{AccGau}直接可得
\begin{equation}
\Psi_{\bvec k}^A(\bvec r, t) = (2\pi)^{-3/2} \exp\qty(\I\bvec k \vdot \bvec r - \I Et)
\end{equation}
可见在 K-H 参考系中, 波函数始终保持平面波的形式.

\begin{example}{高斯波包与电磁场}
在 K-H 参考系中, 若使用偶极子近似且令势能函数 $V(\bvec r) = 0$, 我们会发现高斯始终是高斯波包. 电磁波消失以后, K-H 系变为原来的惯性系, 这是因为电磁波包不能含有直流分量. (未完成:讲详细点?)
\end{example}
