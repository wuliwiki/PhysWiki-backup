% 高阶线性微分方程的降阶
% 线性微分方程组|常微分方程|ode|ordinary differential equation

\pentry{一阶常系数线性微分方程组(常微分方程)\upref{ODEb3},拉普拉斯变换与常系数线性微分方程\upref{ODELap}}

在\textbf{拉普拉斯变换与常系数线性微分方程}\upref{ODELap}中,我们讨论了高阶常系数线性微分方程的解法,通过拉普拉斯变换,将高阶微分方程化为代数方程进行求解。但是很多时候,单个的高阶微分方程也是可以化为多个一阶方程的,从而可以应用\textbf{一阶常系数线性微分方程组(常微分方程)}\upref{ODEb3}中的方法来进行求解。

一般地,$n$ 阶微分方程可以写为
\begin{equation}\label{eq_ODEb4_1}
F(t, x(t), \frac{\dd}{\dd t}x(t), \cdots, \frac{\mathrm{d}^n}{\dd t^n}x(t))=0~,
\end{equation}
接下来,我们介绍几种可以降阶的方程。

\subsection{第一种}

如果\autoref{eq_ODEb4_1} 不显含 $x$ 的 $k$ 次及以下导函数,即方程形式为
\begin{equation}
F(t, \frac{\mathrm{d}^{k+1}}{\dd t^{k+1}}x(t), \frac{\mathrm{d}^{k+2}}{\dd t^{+2}}x(t), \cdots, \frac{\mathrm{d}^n}{\dd t^n}x(t))=0~,
\end{equation}

那么我们可以设 $y=\frac{\mathrm{d}^k}{\dd t^k}x$,解出 $y$ 以后再求 $k$ 次积分得到 $x$。


\begin{example}{}
求解方程
\begin{equation}\label{eq_ODEb4_2}
\qty(
    \frac{\mathrm{d}^4}{\dd t^4}+\frac{1}{t}\frac{\mathrm{d}^3}{\dd t^3}
    )
    x(t)=0~,
\end{equation}

令 $y=\frac{\mathrm{d}^3}{\dd t^3}x(t)$,则\autoref{eq_ODEb4_2} 化为
\begin{equation}
\frac{\dd y}{\dd t}+\frac{1}{t}y=0~,
\end{equation}
解得
\begin{equation}
y=\pm\frac{C}{t}~.
\end{equation}

三次积分后得到
\begin{equation}
x=t^2(C_1\ln\abs{t}-C_2)+C_3t+C_4~.
\end{equation}



\end{example}


\subsection{第二种}

如果\autoref{eq_ODEb4_1} 不显含 $t$,即方程形式为
\begin{equation}\label{eq_ODEb4_3}
F(x(t), \frac{\mathrm{d}}{\dd t}x(t), \frac{\mathrm{d}^2}{\dd t^2}x(t)+\cdots+\frac{\mathrm{d}^n}{\dd t^n}x(t))=0~,
\end{equation}
那么我们可以设 $y=\frac{\dd x}{\dd t}$,以 $y$ 为未知函数。此时 $\frac{\mathrm{d}^2}{\dd t^2}x=\frac{\dd}{\dd t}y=\frac{\dd y}{\dd x}\frac{\dd x}{\dd t}=y\frac{\dd y}{\dd x}$。类似地,$\frac{\mathrm{d}^3}{\dd t^3}x=y\qty(\frac{\dd y}{\dd x})^2+y^2\frac{\mathrm{d}^2}{\dd x^2}$。一般地,我们可以证明任何 $\frac{\mathrm{d}^k}{\dd t^k}x$ 都可以用 $y$ 和 $y$ 关于 $x$ 的各阶导函数表示出来。这样改造以后的\autoref{eq_ODEb4_3} 就成为形如
\begin{equation}
G(y, \frac{\dd}{\dd x}y+\cdots+\frac{\mathrm{d}^{n-1}}{\dd t^{n-1}})=0~
\end{equation}
的方程,降了一阶。

\begin{example}{}
考虑方程
\begin{equation}\label{eq_ODEb4_4}
x\frac{\mathrm{d}^2 x}{\dd t^2}+\frac{\dd x}{\dd t}=0~,
\end{equation}

令 $y=\frac{\dd x}{\dd t}$,则\autoref{eq_ODEb4_4} 化为
\begin{equation}\label{eq_ODEb4_5}
xy\frac{\dd y}{\dd x}+y=0~,
\end{equation}

整理得
\begin{equation}
y\dd x+xy\dd y=0~.
\end{equation}

容易验证,$y=0$ 是\autoref{eq_ODEb4_5} 的一个特解。

$y=0$ 时有
\begin{equation}
\frac{\dd x}{\dd t}=0~,
\end{equation}
从而有特解
\begin{equation}
x=C~.
\end{equation}

现在考虑 $y\neq 0$ 的情况,此时可以把 $y$ 约掉,得到
\begin{equation}\label{eq_ODEb4_6}
\dd x+x\dd y=0~,
\end{equation}
解得
\begin{equation}
y=C-\ln\abs{x}~,
\end{equation}

从而有
\begin{equation}
\frac{\dd x}{\dd t}=C-\ln\abs{x}~.
\end{equation}

求积分即可得到 $x\neq 0$ 时的通解。

注意,$\int\frac{1}{C-\ln\abs{x}}\dd x$ 无法用初等函数表示出来,这里写出这个积分形式就算得到通解了。事实上,很多微分方程都没法写成初等函数的表达式,比如最常见的单摆运动方程(未进行线性近似的原始形式),因此实践中常注重数值计算,以及相应的解的稳定性问题。

\end{example}


\subsection{例题}

\begin{example}{第二宇宙速度}

在地球表面,引力的变化很小,所以我们可以近似把它看成匀强引力场,方便计算。但是如果考虑范围远不止是地球表面,就不得不考虑引力随距离的变化了。

我们现在考虑的问题是,如果一个质点的速度始终平行于地球到该质点的位移(即一维情况),那么质点的运动轨迹是什么样的?

令 $R$ 为地球半径,$M$ 为地球质量,$G$ 为万有引力常数,$r$ 为质点到地球质心的距离。我们只考虑 $r\geq R$ 的情况,忽略空气阻力,那么此时质点仅受地球引力作用,故可以列出运动方程:
\begin{equation}
\frac{\mathrm{d}^2}{\dd t^2}r(t)=-\frac{GM}{r^2}~.
\end{equation}
注意 $G$ 是正数。

令 $\frac{\dd r}{\dd t}=v$,那么方程可以改写为(注意,被抛弃的变量是时间 $t$)
\begin{equation}\label{eq_ODEb4_7}
v\frac{\dd v}{\dd r}=-\frac{GM}{r^2}~.
\end{equation}

解\autoref{eq_ODEb4_7} 得
\begin{equation}\label{eq_ODEb4_8}
v=\pm\sqrt{\frac{2GM}{r}+C}~.
\end{equation}

考虑在地面上垂直向上发射的火箭,初速度为 $v_0$,即初值条件为 $v|_{r=R}=v_0$,那么代入\autoref{eq_ODEb4_8} 可以计算出常数 $C$:
\begin{equation}
C=v_0^2-\frac{2GM}{R}~,
\end{equation}

此时
\begin{equation}\label{eq_ODEb4_9}
v=\pm\sqrt{v_0^2+2GM(\frac{1}{r}-\frac{1}{R})}~.
\end{equation}

此时 $v$ 是一个关于 $r$ 单调递减或单调递增的函数。在离开地球的过程中,$v>0$,\autoref{eq_ODEb4_9} 取正号,因此这个过程中 $v$ 是递减函数。

如果我们希望火箭摆脱地球引力,再也不回来,那么在离开地球的过程中 $v$ 应该恒大于 $0$。由于 $1/r$ 可以任意小,我们要求的条件就变为
\begin{equation}
v_0^2-\frac{2GM}{R}\geq 0~,
\end{equation}
即
\begin{equation}
v_0\geq\sqrt{\frac{2GM}{R}}~.
\end{equation}

这里的 $\sqrt{\frac{2GM}{R}}$ 就是火箭彻底摆脱地球引力的最小初速度,被称为第二宇宙速度,又叫逃逸速度。

查阅资料后代入 $G=6.67\times10^{-11} \opn{N}\cdot\opn{m}^2/\opn{kg}^2$,$M=5.965\times10^{24}\opn{kg}$ 和 $R=6.371\times10^{6}m$,计算出第二宇宙速度的数值为
\begin{equation}
\sqrt{\frac{2GM}{R}}=11.176\opn{km}/\opn{s}~.
\end{equation}

通常只取到 $11.2\opn{km}/\opn{s}$。

\end{example}
















