% 集合(高中)
% keys 高中|集合

\begin{issues}
\issueDraft
\end{issues}

\subsection{概述}
集合语言是现代数学的基本语言,这种语言可以简洁、准确的表达数学内容.

\subsection{定义}
一般地,指定的某些对象的全体称为\textbf{集合}.集合常用大写字母 $A,B,C,D,\cdots$ 标记.集合中的每个对象叫作这个集合的\textbf{元素}.常用小写字母 $a,b,c,d,\cdots$ 表示集合中的元素.

若 $a$ 在集合 $A$ 中,就说 $a$ \textbf{属于}集合 $A$ ,记作 $a \in A$.若 $a$ 不在集合 $A$ 中,就说 $a$ \textbf{不属于}集合 $A$,记作 $a\notin A$.

数的集合简称\textbf{数集}.\\
自然数组成的集合简称\textbf{自然数集},记作 $N$;\\
正整数组成的集合简称\textbf{正整数集},记作 $N^{*}$ 或 $N^{+}$;\\
整数组成的集合简称\textbf{整数集},记作 $Z$;\\
有理数组成的集合简称\textbf{有理数集},记作 $Q$;\\
实数组成的集合简称\textbf{实数集},记作 $R$.

一般地,我们把含有有限个元素的集合叫\textbf{有限集},含有无限个元素的集合叫\textbf{无限集}.

我们将不含有任何元素的集合叫作\textbf{空集},记作 $\phi$.

\subsection{表示}
\textbf{列举法}是把集合中的元素一一列举出来写在大括号内的方法.
符号表示为 $\begin{Bmatrix} ,\cdots, \end{Bmatrix}$,如 $\begin{Bmatrix} x_1,x_2, \cdots ,x_n \end{Bmatrix}$.

用确定的条件表示某些对象属于一个集合并写在大括号内的方法叫\textbf{描述法},符号表示为 $\begin{Bmatrix} | \end{Bmatrix}$,如 $\begin{Bmatrix} x\in A|p(x) \end{Bmatrix}$.

\subsection{性质}
\begin{enumerate}
\item 集合中的元素是\textbf{互异}的.
\item 集合中的元素是\textbf{无序}的.
\end{enumerate}

\subsection{集合的基本运算}
一般地,由既属于集合 $A$ 又属于集合 $B$ 的所有元素组成的集合叫做 $A$ 与 $B$ 的交集,记作 $A \cap B$(读作“A交B”),即
\begin{equation}
A\cap B = \begin{Bmatrix} x|x\in A \wedge x\in B \end{Bmatrix}
\end{equation}

由属于集合 $A$ 或属于集合 $B$ 的所有元素组成的集合,叫作 $A$ 与 $B$ 的并集,记作 $A\cup B$(读作“A并B”),即
\begin{equation}
A\cup B = \begin{Bmatrix}x|x\in A \vee x\in \end{Bmatrix}
\end{equation}

根据交集定义