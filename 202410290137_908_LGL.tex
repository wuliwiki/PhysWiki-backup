% 拉格朗日力学 (综述)
% license CCBYSA3
% type Wiki

本文根据 CC-BY-SA 协议转载翻译自维基百科\href{https://en.wikipedia.org/wiki/Lagrangian_mechanics}{相关文章})

在物理学中,拉格朗日力学是一种基于平稳作用原理(也称为最小作用原理)的经典力学表述。它由意大利-法国数学家和天文学家约瑟夫-路易·拉格朗日在1760年向都灵科学学院的演讲中提出,并在1788年他的巨著《解析力学》中达到巅峰。

拉格朗日力学将一个力学系统描述为由构型空间 \( M \) 和其中的一个光滑函数 \( L \)(称为拉格朗日量)组成的对 \( (M, L) \)。对于许多系统,\( L = T - V \),其中 \( T \) 和 \( V \) 分别为系统的动能和势能。

平稳作用原理要求由 \( L \) 推导出的系统的作用泛函在系统的时间演化过程中保持在一个平稳点(最大值、最小值或鞍点)。这一约束条件使得可以利用拉格朗日方程计算系统的运动方程。