% 子列极限、上极限与下极限
% 序列|上极限|下极限|子列极限
\subsection{定义:上极限,下极限}

\begin{figure}[ht]
\centering
\includegraphics[width=12cm]{./figures/SubLim_1.png}
\caption{上下极限示意图} \label{SubLim_fig1}
\end{figure}
\begin{definition}{上下极限}
  若序列 $\{x_n\}$ 有界.
  \begin{equation}
  \begin{matrix}
  \text{记}\ &l_n=\inf\{x_n,x_{n+1},\cdot\cdot\cdot\}, &h_n=\sup\{x_n,x_{n+1},\cdot\cdot\cdot\},\\
  &\text{下极限}\ l=\sup\{l_n\}, &\text{上极限}\ h=\inf\{h_n\},\\
  \text{记为}&l=\underline{\lim}_\limits{n\rightarrow\infty}x_n,
  &h=\overline{\lim}_\limits{n\rightarrow\infty}x_n.
  \end{matrix}
  \end{equation}
  对于无界序列,若有上界,则可以定义上极限;若有下界,则可以定义下极限.
\end{definition}
\autoref{SubLim_fig1} 中蓝线代表的是序列的上极限,橙色线代表的是序列的下极限.
\begin{exercise}{}
设序列 $\{x_n\}$ 有界,证明:$l=\underline{\lim}_\limits{n\rightarrow\infty}x_n=\lim\limits_{n\rightarrow \infty} l_n,h=\overline{\lim}_\limits{n\rightarrow\infty}x_n=\lim\limits_{n\rightarrow \infty} h_n$.
\end{exercise}

\subsection{上下极限的性质}
\begin{theorem}{}
  以下三个命题等价:

  ​		$(1)h=\overline{\lim}_\limits{n\rightarrow\infty}x_n$

  ​		$(2)\forall\epsilon>0$,$\exists N$,当 $n>N$,有 $x_n<h+\epsilon$

  ​			  $\forall \epsilon>0$,$\forall K$,$\exists n>K$,使 $x_{n}>h-\epsilon$

  ​		$(3)$存在子列 $\{x_{n_k}\}$,使得 $\lim_\limits{k\rightarrow\infty}x_{n_k}=h$,并对任何其它收敛子列 $\{x_{n_k}\}$,有$\lim_\limits{k\rightarrow\infty}x_{n_k}\leq h$
  
\end{theorem}

聚点(定义)

  设 $E$ 是 $\mathbb{R}$ 中的一个子集.若$x_0 \in \mathbb{R}$($x_0$不一定属于$E$)满足:对 $\forall \delta>0$,有 $U_0(x_0,\delta)\cap E\not=\emptyset$,则称 $x_0$ 是 $E$ 的一个聚点.

性质 2

  ​		$(1)$若有界序列 $\{x_n\}$ 由互不相同的数组成,则上极限为其最大聚点,下极限为其最小聚点.

  ​		$(2)$若 $\{x_{n_k}\}$ 为 $\{x_n\}$ 子列,则有$\underline{\lim}_\limits{n\rightarrow\infty}x_n\leq\underline{\lim}_\limits{k\rightarrow\infty}x_{n_k}\leq\overline{\lim}_\limits{k\rightarrow\infty}x_{n_k}\leq\overline{\lim}_\limits{n\rightarrow\infty}x_{n}$.

  ​		$(3)$ $\lim_\limits{n\rightarrow\infty}x_n=a\Leftrightarrow\underline{\lim}_\limits{n\rightarrow\infty}x_n=\overline{\lim}_\limits{n\rightarrow\infty}x_n=a$

问题

  1. 设 $E$ 是 $\mathbb{R}$ 中的一个子集.若 $x$ 是 $E$ 的一个聚点,证明:  $\forall \delta>0$, $U(x_0,\delta)\cap E$ 中有无穷多个元素. 
  2. 证明性质1和性质2.
  3. 利用上下极限证明 $\{\sin n\}$ 发散.
  4. 序列 $\{x_n\}$ 的上极限为 $h_1$,序列 $\{y_n\}$ 的上极限为 $h_2$,那么序列 $\{x_n+y_n\}$ 的上极限是 $h1+h2$ 吗,$\{x_ny_n\}$ 的上极限是 $h_1\cdot h_2$ 吗?
  5. 若序列 $\{x_n\}$ 是非负收敛序列,$\rm{Question}\ 4$ 的命题成立吗?




压缩映像原理

- 设 $f(x)$ 在 $[a,b]$ 上有定义,$f([a,b])\subset[a,b]$,且满足$|f(x)-f(y)|\leq q|x-y|,\forall x,y\in[a,b]$,

  其中$0<q<1$.

  那么:存在唯一 $c\in[a,b]$,使 $f(c)=c$.

  \textbf{证明:}构造序列 $\{x_n=f(x_{n-1})\}$,用柯西准则证明其收敛.

柯西收敛准则

定义(柯西序列)

    定义:设{$x_n$}是一个序列,若$\forall\epsilon>0$,$\exists N$,当$n,m>N$时,有$\vert x_n-x_m\vert<\epsilon$,则称{$x_n$}是一个柯西序列.

柯西收敛准则

    序列 {$x_n$} 收敛 $\Leftrightarrow$ {$x_n$} 是一个柯西序列.

Question:

  1. 写出柯西收敛准则的否命题形式(用肯定的语气写出:怎样的序列不是柯西序列,怎样的序列不是发散的).
  2. 试用闭区间套定理证明压缩映像原理.
  3. 设序列 $\{p_n\}, \{q_n\}$ 满足 $p_{n+1}=p_n+2q_n,\ q_{n+1}=p_n+q_n,\ p_1=q_1=1$,求证 $\left\{\large\frac{p_n}{q_n}\right\}$ 的极限存在,并求其极限.