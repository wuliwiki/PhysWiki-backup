% 数论三角和
% keys 三角和
% license Usr
% type Tutor

\subsection{数论三角和}
由于 $e^{\pi \I} = -1$,利用 $e^{2\pi \I} = 1$ 可以构造出诸多三角和。在数论中我们记
\begin{equation}
e(t) := e^{2\pi \I t} ~,
\end{equation}
则对于 $t$ 的有理值,当 $x \equiv y \pmod m$ 时,将有
\begin{equation}
e\left(\frac{x}{m}\right) = e\left(\frac{y}{m}\right) ~.
\end{equation}
这是数论三角和的重要性质。

\subsection{Gauss 和}
\begin{definition}{Gauss 数论三角和}
定义 $S(m, n)$ 为\textbf{高斯(数论三角)和},\footnote{在数论中,求和取遍剩余系时,一般都用字母 $h$ 而非其他字母。}
\begin{equation}
S(m, n) := \sum_{h = 0}^{n-1} e\left(\frac{h^2m}{n}\right) ~.
\end{equation}

\end{definition}
Gauss 和在二次剩余中应用较多。

考虑对于任意 $r$,
\begin{equation}
e\left(\frac{(h + rn)^2m}{n}\right) = e\left(\frac{h^2m}{n}\right) ~,
\end{equation}
故我们可以不将 $h$ 限定在从 $0$ 到 $(n-1)$ 取遍,而只要 $h$ 取遍一个完全剩余系即可。此时用记号 $h(n)$ 表示取遍一个完全剩余系。
\begin{corollary}{}
\equa
\end{corollary}