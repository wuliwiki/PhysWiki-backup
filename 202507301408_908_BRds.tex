% 布尔代数(综述)
% license CCBYNCSA3
% type Wiki

本文根据 CC-BY-SA 协议转载翻译自维基百科\href{https://en.wikipedia.org/wiki/Boolean_algebra_(structure)}{相关文章}。

在抽象代数中,布尔代数或布尔格是一个带补的分配格。这种代数结构刻画了集合运算和逻辑运算的基本性质。布尔代数可以被看作是幂集代数或集合域的推广,或者其元素可以被看作是广义真值。它也是德摩根代数和克莱尼代数的一个特殊情形。

每一个布尔代数都可以产生一个布尔环,反之亦然,其中环的乘法对应于合取或交运算∧,环的加法对应于异或或对称差(不是析取 ∨)。然而,布尔环理论在两个运算之间本质上是不对称的,而布尔代数的公理和定理则通过对偶性原理体现出理论的对称性。[1]
\subsection{历史}
\begin{figure}[ht]
\centering
\includegraphics[width=6cm]{./figures/679baac505e439e9.png}
\caption{} \label{fig_BRds_1}
\end{figure}
“布尔代数”一词是为了纪念自学成才的英国数学家乔治·布尔而命名的。他最初在 1847 年出版的一本小册子《逻辑的数学分析》中引入了这种代数系统,该书是为回应奥古斯都·德·摩根与威廉·汉密尔顿之间一场持续的公共争论而写的。随后,他在 1854 年出版了更为重要的著作 《思维定律》。布尔的表述在一些重要方面与上文所描述的有所不同。例如,在布尔的体系中,合取和析取并不是一对对偶运算。布尔代数在 1860 年代逐渐成形,出现在威廉·杰文斯和查尔斯·桑德斯·皮尔斯撰写的论文中。布尔代数和分配格的首次系统化呈现归功于恩斯特·施罗德1890 年的 《讲义》。布尔代数在英语中的第一次大规模研究则是 A. N. 怀特海1898 年的 《普遍代数》。布尔代数作为一种现代公理化意义上的公理化代数结构始于爱德华·V·亨廷顿1904 年的一篇论文。布尔代数真正成为严肃数学的一部分是在 1930 年代马歇尔·斯通的工作中,以及 1940 年加勒特·伯科夫的 《格论》 中。在 1960 年代,保罗·科恩、达纳·斯科特等人使用布尔代数的衍生理论——强迫法和布尔值模型——在数理逻辑和公理化集合论中发现了深刻的新成果。
