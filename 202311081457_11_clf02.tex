% Clifford 代数的基本运算
% license Xiao
% type Tutor


本节利用集合语言,介绍Clifford代数上的二元线性运算\footnote{本文参考Jie Peter《代数学讲义》}。我们也会证明,更改线性空间的正交基并不会改变运算的形式。
\subsubsection{外积与正交基}
\begin{definition}{外积}
给定Clifford代数$\mathrm {Cl(X,R,s)}$,对于$A,B\in 2^x$,定义
\begin{equation}
A \wedge B=\left\{\begin{aligned}
A B,\quad& A \cap B=\varnothing \\
0,\quad& A \cap B \neq \varnothing~,
\end{aligned}\right.
\end{equation}
并称之为外积(outer product or exterior product)或楔积(wedge product)。我们可以通过线性性将该运算拓展到任意元素之间的外积。
\end{definition}

在上一节,我们说过,集合语言的阐述实际上是指定了线性空间的正交基。正交性体现为CLifford积的反对称性。因而对于正交基$\{\mathrm {e_i}\}$,我们有$\mathrm{e_i\wedge e_j=-e_j\wedge e_i}$。所以对线性空间的任意两个向量作外积,反对称性亦能满足。

不仅如此,基底之间的正交关系不随线性变换而改变。例如,$\mathrm {e_1=a_1}$
