% 爱德华·泰勒(综述)
% license CCBYSA3
% type Wiki

本文根据 CC-BY-SA 协议转载翻译自维基百科 \href{https://en.wikipedia.org/wiki/Edward_Teller}{相关文章}。

\begin{figure}[ht]
\centering
\includegraphics[width=6cm]{./figures/aa578cdb3e82db6d.png}
\caption{} \label{fig_ADHTL_1}
\end{figure}
爱德华·泰勒(Edward Teller,匈牙利语:Teller Ede,1908年1月15日-2003年9月9日)是一位匈牙利裔美国理论物理学家和化学工程师,因其在氢弹发展中的关键角色而被俗称为“氢弹之父”。他是基于斯坦尼斯瓦夫·乌拉姆设计提出的“泰勒–乌拉姆构型”的共同发明人之一。

泰勒性格激烈,据称“受到百万吨级爆炸梦想的驱使,有救世主情结,展现出专断的行为风格”。\(^\text{[1]}\)他曾设计过一种名为“闹钟模型”的热核炸弹,其爆炸当量高达1000兆吨(即10亿吨TNT),并建议通过船只或潜艇投送。这种武器将具备焚毁一个大陆的能力。\(^\text{[1]}\)

泰勒于1908年出生于奥匈帝国,20世纪30年代移民至美国,是一批被称为“火星人”的匈牙利科学家移民中的一员。他在核物理、分子物理、光谱学以及表面物理等领域作出了诸多贡献。他对恩里科·费米的β衰变理论的拓展,以“伽莫夫–泰勒跃迁”的形式,为该理论的实际应用奠定了重要基础;而“杨–泰勒效应”和“布鲁瑙尔–埃米特–泰勒理论”(Brunauer–Emmett–Teller theory,简称 BET 理论)至今仍以原始形式被广泛应用,是物理与化学领域的核心理论之一。\(^\text{[2]}\)泰勒惯于以基础物理原理思考问题,常常与同行讨论,以突破难题。这一特点在他与斯坦尼斯瓦夫·乌拉姆共同设计可行的热核聚变炸弹方案时表现得尤为明显。然而,后来他在性格上却否定了乌拉姆所起的关键作用。赫伯特·约克指出,泰勒实际上是利用了乌拉姆提出的“压缩与加热启动热核聚变”的基本思想,绘制出他自己的“超级炸弹”方案草图。\(^\text{[1]}\)在乌拉姆提出其方案之前,泰勒原始设想的“经典超级炸弹”本质上是一个通过加热未压缩液态氘以期引发持续热核燃烧的系统。\(^\text{[1]}\)这个设想虽然简单,源自基本物理原理,但泰勒对其执着追求的程度极为强烈,即使被证明是错误的,或已有人指出无法实现,他仍不放弃。为了获得华盛顿对其“超级武器”计划的支持,泰勒提出在“绿色屋行动”中进行一次热核辐射内爆实验,即所谓的“乔治”试验。\(^\text{[1]}\)

泰勒对托马斯–费米理论也作出了重要贡献,该理论是密度泛函理论的前身——密度泛函理论是现代量子力学处理中复杂分子的标准工具之一。1953年,泰勒与尼古拉斯·梅特罗波利斯、阿里安娜·罗森布鲁斯、马歇尔·罗森布鲁斯以及奥古斯塔·泰勒共同发表了一篇论文,该论文成为蒙特卡洛方法在统计力学中的应用、以及贝叶斯统计中马尔可夫链蒙特卡洛(MCMC)研究文献的重要起点。\(^\text{[3]}\)泰勒早期即参与了“曼哈顿计划”,该计划研发了世界上第一枚原子弹。他积极推动热核武器的研发,但融合(聚变)型武器最终是在二战后才出现。他共同创立了劳伦斯利弗莫尔国家实验室,并曾担任该实验室的主任或副主任。然而,由于他在其前上司、洛斯阿拉莫斯实验室负责人J·罗伯特·奥本海默的安全审查听证会上发表了有争议的反面对证言,泰勒遭到了科学界的排斥。

泰勒继续获得美国政府和军方科研体系的支持,尤其是在他倡导发展核能、保持强大核武库以及推进积极的核试验计划方面。在晚年,他提出了许多颇具争议的技术方案,以解决军事和民用问题,例如“战车计划”——利用热核爆炸在阿拉斯加开凿一个人工港口的设想,以及支持罗纳德·里根提出的“战略防御倡议”。泰勒曾获得恩里科·费米奖和阿尔伯特·爱因斯坦奖。他于2003年去世,享年95岁。
\subsection{早年生活与学术起步}
爱德·泰勒于1908年1月15日出生在布达佩斯,当时属于奥匈帝国的领土,他出身于一个犹太家庭。父亲米克沙·泰勒是一位律师,母亲伊洛娜(Ilona,娘家姓Deutsch)是一位钢琴家。[4][5][6] 他在布达佩斯就读于明塔文理中学。[7]泰勒是不可知论者。他后来写道:“宗教在我家从来不是个问题,事实上,几乎从未被提起过。我唯一的宗教教育,是因为明塔中学规定所有学生必须修习自己所属宗教的课程。我们家只过一个节日——赎罪日,那天我们全家会一起禁食。但我父亲仍会在每个安息日和所有犹太节日为他的父母祈祷。我所理解的上帝,是:如果他真的存在,那太好了;我们非常需要他,但几千年来我们从未见过他。”[8]泰勒学说话较晚,但从小就对数字产生了浓厚兴趣,经常以心算大数作为娱乐。[9]
