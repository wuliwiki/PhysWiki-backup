% 狭义相对论(科普)

\pentry{经典力学和相对论(科普)\upref{CM0}}

狭义相对论有两个基本假设:
\begin{enumerate}
\item 惯性系平权
\item 光速不变
\end{enumerate}
下文我们会详细解释它们的含义, 注意这仅仅是两个假设, 就像牛顿三定律\upref{New3}一样。 不幸的是由于技术限制我们并没有能力创造一个速度足够快的(例如光速的 5\%)飞船在上面直接测量光速以及其他相对论实验, 但我们有大量间接的方法可以实验验证(链接未完成)。 光速不变常被民间科学家用于攻击相对论, 关于光速不变假设的历史背景和一些误解见 “光速不变假设的一些误解和历史\upref{SpeRel}”。

我们知道经典力学中速度是可以相加的, 如果一个人在火车上向前射击, 那么地面上看来子弹的速度等于火车的速度加上子弹相对枪口或者火车的速度。 所以如果有人告诉你说子弹的速度无论相对于火车还是相对于地面都一样快, 那你肯定会觉得他在胡说。 而爱因斯坦提出的光速不变假设恰恰就是说无论使用哪个\textbf{参考系}(本文的参考系都是指\textbf{惯性参考系}\upref{New3}, 即满足牛顿第一定律的参考系), 任何光相对于该参考系的速度都是一样的, 无论这个光是从哪里发出如何发出的。 可见狭义相对论推翻了经典力学的根基——\textbf{时空观}。

\subsection{绝对时空观} \label{sub_Relat0_1}
为了方便讨论时间的概念, 我们不妨认为每个参考系中, 空间中的每一点都挂满了和这个参考系相对静止的时钟, 且这些时钟制作精良没有误差, 不受外界因素干扰, 也不取决于特定的原理(水滴、摆锤、电路等等)。 经典力学的绝对时空观认为, 我们可以一劳永逸地一次让这些钟全部彼此校准, 那么它们就会永远保持绝对的同步,使得无论从什么角度来看, 只要两个事件发生时, 它们所在位置的时钟读数相同, 那么这两个事件就是同时发生的, 毫无歧义。

绝对的空间也可以类似理解。 每个参考系中的长度都是一样的, 火车上的一截棍子无论朝向如何, 如果你在某个时刻把它贴近地面, \textbf{同时}把它的两端的位置在地面做一个标记, 那么地面上这两个标记之间的长度测量出来也等于在火车上测量的棍子的长度。 注意这个概念需要建立在上面对\textbf{同时}的定义的基础上。 做这两个标记就可以看成两个事件, 如果我们无法在所有参考系都对两个事件是否同时达成一致, 那么我们也无法得出火车上测量的长度和地面上测量的一致。 可见时间和空间的概念往往是纠缠在一起的, 这也是为什么经常把它们统称为\textbf{时空}。

由于绝对时空观在相对论提出以前根深蒂固, 所以在当时的人看来光速不变是荒谬的。 因为速度的定义是绝对的距离除以绝对的时间间隔, 那么在一段绝对的时间 $\Delta t = t_2 - t_1$ 内(上文的钟表读数相减), 光在火车上走过的距离 $s_1$ 地上的人不会有歧义, 而火车在这段时间内相对地面还走过了一段距离 $s_0$, 那么由于地面上的距离可以相加(即使在狭义相对论中, 同一个参考系中的距离也是可以相加的), 所以光相对于地面走过的距离为 $s_2 = s_0 + s_1$, 所以把 $s_1, s_2$ 分别除以时长 $\Delta t$, 得到光相对于两个参考系的速度必定是不同的, 任何物体的运动也都一样, 光没有理由例外。

\subsection{同一参考系的相对论时空观}
所以如果一定要假设光速不变, 那么就必须要改变时空观。 光并不特殊, 它只是用于揭示时空性质的一个工具, 如果有一个别的粒子以光速运动, 那么它的速度同样不变。 我们尽量保守地改变上面关于时空的假设, 不多不少, 直到新的时空观能容纳光速不变这个现象为止。 首先是关于同时性的问题, 在一个参考系中, 我们如何确定两个不同位置的时钟同步呢? 既然我们假设光速不变, 那最直接的方法就是从这两个时钟的中点同时向它们各发射一道光, 如果当光到达两个时钟时, 它们的读数相同, 那我们就说它们\textbf{在当前参考系中同步}。 注意我们在使用文字时必须非常谨慎, 因为我们在推翻根植于常识中的认知。 我们还没开始讨论别的参考系中观察到的东西, 所以要强调只是在当前参考系中同时。 另外这个过程中我们还假设了同一个参考系中长度是绝对的, 可以用一把尺子测量任何地方的长度, 否则我们无法确定两个钟的中点在什么位置。

所以在同一个参考系中, 相对论时空观和绝对时空观的区别并不大, 处于不同位置的观察者仍然对两个事件的同时性和空间的长度么有任何争议。 所以要建立相对时空观, 我们需要从具有相对运动的不同参考系的观察者之间如何看待对方入手, 所以还是回到火车的问题。 注意即使在绝对时空观中, 也不存在这两个参考系哪个更优越的问题——地面上的人可以认为自己静止火车在动, 火车上的人也完全可以认为自己静止而火车外的所有物体都在运动(从太阳的参考系看地球也的确如此)。 在相对论时空观中, 我们已经知道两个参考系分别已经把属于自己的时钟都分别校准了, 但还没有对不同参考系之间的时钟做出任何比较。

\subsection{事件、时空坐标}
在进一步讨论之前,我们还要把\textbf{事件}这个概念也做一个抽象。 在给定的参考系中, 一个事件就可以抽象为一个空间位置和该位置的时钟的一个读数, 例如使用三维直角坐标就有 $(x, y, z, t)$, 也就是时间在\textbf{四维时空}中的坐标。 本文只讨论一个直线方向上的运动, 只需要一个空间坐标 $x$, 所以给定参考系中一个事件可以简化为两个坐标 $(x, t)$。 为了区分两个参考系, 我们把火车参考系的时空坐标后面都加一撇, 如 $(x', t')$ 而地面参考系的坐标则不加。

在绝对时空观中, 同一个事件在不同的参考系中一般也会具有不同的空间坐标, 但时间坐标是与参考系无关的, 两个空间坐标之间的距离也与参考系无关。 为了避免回到绝对时空观, 我们需要假设同一事件在不同参考系中的每个坐标都可能是不同的, 两个事件的各个坐标之间的差值也可能不同。

\subsection{测量长度}\label{sub_Relat0_3}
来看如何测量长度。 如果要在地面参考系测量\textbf{运动}棍子的长度,那么可以把棍子贴近地面, 然后地面上的观察者在\textbf{同一时刻}(使用地面的钟)在地面上标记棍子两端的位置。 两个标记事件的时空坐标分别记为 $(x_1, t_1)$ 以及 $(x_2, t_2)$ 并有 $t_1 = t_2$, $x_2 > x_1$。 那么测量的结果就是 $L = x_2 - x_1$。 在火车参考系中, 测量相对地面静止的棍子也可以用相同的方法。 如果要在火车上测量静止的棍子的长度, 那么可以直接把棍子两端在任意时刻的坐标相减即可, $L' = x_2' - x_1'$, 因为棍子是静止的, 两端的坐标不会随时间变化。

如果要测量一根静止在地面上的棍子的长度也同理, 火车上的人必须用火车系的时钟同时标记棍子两端的位置, 再把坐标相减, 而地面上的人不需要同时测量直接相减即可。

\subsection{测量质点的速度}
现在回到测量速度的问题中。 在两个参考系中测量任何物体(视为质点)匀速运动的速度都可以简化成测量两个事件, 事件 1 是某时刻物体从某一点出发, 用火车参考系的坐标记为 $(x_1', t_1')$, 事件 2 是该物体做匀速运动一段时间后到达了空间中某个位置, 记为 $(x_2', t_2')$。 那么在火车中测量该物体的速度为
\begin{equation} \label{eq_Relat0_2}
v' = \frac{x_2' - x_1'}{t_2' - t_1'}~.
\end{equation}
同理, 在地面上观测这两个事件, 可以分别用 $(x_1, t_1)$ 和 $(x_2, t_2)$ 来描述两个事件, 那么测到的速度为
\begin{equation} \label{eq_Relat0_1}
v = \frac{x_2 - x_1}{t_2 - t_1}~.
\end{equation}
现在对如何测量速度已经有了严谨的定义, 那么\textbf{光速不变}的要求就是说, 当任何两个事件使得以上 $v' = \pm c$ 时($c$ 表示真空中光速), 必然也有 $v = v'$, 反之亦然。 注意任何两个事件都可以, 无论是否真的有一束光从起点到达终点(没有的话你可以想象嘛)。

\subsection{一般观测}\label{sub_Relat0_2}
有一种低级的误解是, 相对论的各种效应(例如尺缩短和钟变慢,见下文)是因为光从物体发出到传到观测者眼中需要一段时间, 导致运动的物体\textbf{看起来}发生了某种扭曲。 这是错误的。 在讨论相对论时, 为了避免歧义, 规定每个观测者或者测量装置只能观测它所在的位置发生的事件而不能直接观测别处的事件。 例如上面在测量运动棍子的长度时, 我们不能在远方用一个相机直接拍摄棍子, 而只能在地面装一排局部传感器,并把棍子无限贴近它们, 当棍子的两个端点经过每个传感器, 传感器就会记录下该事件的时空坐标 $(x_i, t_i)$。 另外在\textbf{比较不同参考系的两个时钟的读数时, 也只能比较恰好经过同一位置的两个时钟}, 这点非常重要。

事实上我们可以想象每个参考系的空间中的每一点除了有一个固定的钟, 还有一个固定的局部传感器, 它只能记录它所在的位置发生的事件和时钟读数, 即事件的时空坐标 $(x_i, t_i)$, 而对别处发生的事件无能为力。 除此之外它也可以在事件发生时与恰好经过该位置的, 固定在其他参考系中的传感器交换坐标信息, 即得到同一事件在其他参考系的时空坐标 $(x_i', t_i')$。 这些条件对理解下文的尺缩短和钟变慢至关重要。

\subsection{洛伦兹变换}
要从光速不变的假设建立相对论时空观, 就是在不同参考系中的时空坐标之间建立一种转换关系, 使得无论两个事件的坐标怎么取都能满足光速不变条件。 也就是无论光从什么位置发射, 向什么方向发射, 传播时间多长, 都有 $v = v' = \pm c$。 可以用数学证明对于任意给定的两个不同的惯性系, 满足这个要求的变换是存在且唯一的, 它有一个大名鼎鼎的名字叫做\textbf{洛伦兹变换(Lorenz transform)}。 虽然我们可以给出详细的推导\upref{LornzT}, 但这有点超出了这篇科普对数学的要求。 那么我们不妨在余下的篇幅对洛伦兹变换做一些半定量的描述。

洛伦兹变换其实就是中学数学的\textbf{映射}\upref{map},把任何 $(x_i, t_i)$ 通过一组公式映射到 $(x_i', t_i')$。 可以安全地假设这个映射是\textbf{一对一映射}的或者称为\textbf{双射}, 不可能说一个参考系中两个不同的事件的坐标变换到另一个参考系中后具有相同的时空坐标(反之也不可能)。 这是因为我们通常对两个点的接触事件不会发生歧义, 不可能说在一个参考系中两个物体发生了接触另一个参考系中没有接触。 例如一个坐在火车上的乘客用把笔尖伸出窗外不动, 在火车匀速经过固定在地面的电线杆时, 笔尖和电线杆发生了一瞬间的接触, 那么任何参考系的人都会同意笔尖被接触和柱子被接触发生在同一位置以及同一时刻, 可以认为是同一事件。

关于该映射的另一个假设是, \textbf{两个惯性参考系之间的相对速度没有歧义: 它们等大, 方向相反且恒定不变}。 也就是火车上任意标记一点, 在地面参考系用\autoref{eq_Relat0_1} 测量出该点的速度都是一样的。 同理在地面任意取一点, 在火车参考系中用\autoref{eq_Relat0_2} 测速也会得到相同的速度大小, 但方向相反。 这其实使用了第一个基本假设, 即两个惯性系是等价的, 不可能我看你快你看我慢。

我们再来像\autoref{sub_Relat0_1} 一样, 把光相对于地面行走的距离划分为两部分, 第一部分是火车相对地面的移动, 第二部分是光相对火车的移动。 所以我们假设我们事先在火车上标记了光的出发点, 然后在地面参考系的 $t_2$ 时刻除了记录光的终点位置 $x_2$ 还记录下该标记在地面系的坐标 $x_a$。 那么我们在地面参考系就可以定义第三个事件 $(x_a, t_a)$(其中 $t_a = t_2$), 在火车参考系中的坐标照例记为 $(x_a', t_a')$(其中 $x_a' = x_1'$)。 于是光在地面参考系走过的路程就可以拆分成火车的标记点在这段时间内走的路程 $x_a - x_1$ 以及从地面参考系看来光相对于火车的标记点多走的路程 $x_2 - x_a$。 再次强调移动物体的长度必须同时测量, 这就是规定 $t_a = t_2$ 的原因\footnote{但计算 $x_a-x_1$ 时并不需要令 $t_a = t_1$, 这是因为这个减法不是在测量移动物体的长度, 而是测量移动物体上同一点移动的距离, 就像\autoref{eq_Relat0_1} 的分子那样。}。 而火车上的时间也可以分为两个部分, 即 $t_2' - t_1' = (t_2' - t_a') + (t_a' - t_1')$。 注意虽然事件 $a$ 和 2 在地面参考系中同时, 但在火车参考系却未必同时, 所以不能假设 $t_2' - t_a' = 0$ 甚至暂时还不能确定它的正负(留到下文讨论)。

所以要让光速不变, 就是要求下面关系成立\footnote{当然我们也可以把式中的 $x_2' - x_1'$ 写成 $(x_2'-x_a') + (x_a'-x_1')$, 但我们已经知道 $x_a'=x_1'$, 所以就没必要了。 $t_2-t_1$ 的拆分也同理。}:
\begin{equation} \label{eq_Relat0_3}
\frac{x_2' - x_1'}{(t_2' - t_a') + (t_a' - t_1')} = c = \frac{(x_a - x_1) + (x_2 - x_a)}{t_2 - t_1}~.
\end{equation}
其中 $x_2' - x_1'$ 是火车参考系中测到的光经过的车身长度, 而在地面看来, 同样的一段车身长度是 $x_2 - x_a$。 同样, 火车坐标 $x_1'$ 处的同一个钟的读数从 $t_1'$ 走到 $t_a'$ 的过程中, 如果它不断和地面参考系中它经过的钟的读数做对比, 就会发现那些钟的读数从 $t_1$ 变化到了 $t_2$。

洛伦兹变换之所以可以满足\autoref{eq_Relat0_3}, 是因为它有两个著名的效应: \textbf{尺缩短}和\textbf{钟变慢}。 尺缩短效应使得火车上的一段车厢在地面上测量起来变短了(测量长度按照\autoref{sub_Relat0_3} 的方法),或者说火车上测量的静止木棍的长度比地面上测量该木棍的长度要长, 所以有
\begin{equation}
x_2 - x_a < x_2'-x_a' = x_2' - x_1'~,
\end{equation}
且火车前进的速度越快, 缩短的就越多。 而钟变慢效应则是说如果\textbf{火车上同一位置}(同一 $x'$)经历了两个不同事件, 那么这两个事件在地面参考系的时间间隔比在火车参考系的时间间隔要大, 所以有
\begin{equation}
t_2 - t_1 = t_a - t_1 > t_a' - t_1'~.
\end{equation}
绝对时空观中, 以上两式都是取等号的。 代入\autoref{eq_Relat0_3} 就可以发现这两个效应让右边的分子相对绝对时空观变小, 而分母相对变大。 也就是说在绝对时空观中地面上的光速本来应该比火车上的光速大(等式右边比左边大), 经过两个相对论效应一调整后, 光速就保持不变了。 当然这只是一个定性的描述, 当你看到洛伦兹变换的具体公式时, 可以把一些数值带入\autoref{eq_Relat0_3} 验证等号两边的确都等于光速($\pm c$)。

\subsection{地面的尺缩短和钟变慢}
既然所有参考系都是等价的, 那么从火车参考系来看, 相对于地面静止的木棍和钟同样会有尺缩短和钟变慢效应。 这乍看之下是矛盾的, 它们怎么可能互相看对方的木棍都比自己的短, 看对方的钟都比自己慢呢? 如果在火车上放置一根木棍, 在轨道旁放置一根\textbf{同样长度}的木棍, 当它们经过彼此是不就可以比较哪个更长了吗? 注意这里说的同样长度是指火车和地面各自测自己的静止木棍的结果, 这样测出的长度叫做\textbf{固有长度(proper length)},而静止的钟的两个读数之差叫做\textbf{固有时间(proper time)}。 笔者觉得翻译成\textbf{固有时长}会更合适, 因为我们对时间的绝对读数没有兴趣, 只对一段时间的长度有兴趣。

\textbf{谷仓棍子佯谬(barn and pole paradox)}所描述的基本就是上面把两把尺子比较的情景。 这个佯谬描述一个运动的杆子和一个静止的谷仓, 它们的固有长度相同。 根据狭义相对论,当杆子匀速运动时,从谷仓的参考系看,杆子的长度会出现尺缩现象。 这时,杆子足够短,可以让杆子进去后谷仓前后两扇门同时关闭再同时打开。 然而从杆子的参考系看,谷仓是在高速运动的,因此谷仓的长度会出现尺缩,导致谷仓看上去更短。 从这个角度看, 杆子似乎太长而无法允许两扇门同时关闭。 而解决的方法就是利用\textbf{同时}的相对性, 在谷仓参考系来看同时关上再打开的两扇门, 在杆子的参考系中实际上是先后关上打开的, 这样一来就不会产生任何矛盾了。 如果我们在两扇仓门上分别安装一个钟, 并在谷仓的参考系使它们同步, 那么谷仓参考系同时关门就意味着关门时两个钟的读数都相同。 但从杆子参考系看来, 虽然两扇门关闭时, 门上的钟读数仍然相同, 但是这两个钟却是不同步的, 所以两次关门不同时。 在杆子参考系的同一时刻, 杆子进入的那扇门(前门)上的钟读数总是比出去时那扇门(后门)上钟的读数要大, 这就导致后门关闭时前门仍然开着允许一段杆子留在外面, 而后门打开让杆子出去一段后, 前门的钟才走到该关门的读数。

还是回到火车的场景, 为了对同时性如何转换有一个更好的了解, 我们可以在地面参考系和火车参考系中各放置一排静止的钟, 并且用本文开始给出的方法在两参考系钟各自进行校准。 从上面的悖论可以知道(当然也可以从洛伦兹变换直接算出), 地面参考系的观测者如果在某时刻给火车上的那排钟拍一张快照, 那么越靠近车头方向的钟读数就越大。 注意这里所说的\textbf{拍快照}只是一种习惯性的说法, 根据\autoref{sub_Relat0_2} 的规定, 严谨的做法是给地面的每一个钟也安装一个局部传感器, 每个传感器都在自己的钟达到某个读数时读取该位置处火车上的钟的读数, 然后后期再把这些记录收集起来, 绘制成一张 “快照”。 同理, 火车上的人如果给地面上的那排钟拍快照, 会发现地面上的钟越靠近火车尾部的读数越大, 因为地面是在朝火车尾部运动, 两参考系都认为对方的钟沿前进方向读数越来越大。 所以现在我们可以知道\autoref{eq_Relat0_3} 中的 $t_a' > t_2'$, 因为 $x_2' > x_a' = x_1'$, 也就是火车上的观察者认为地面先测量了光的终点位置再测量起点位置(“难怪他们觉得我们缩短了!”)。

所以回到火车和两根木棍的情景, 如果两根木棍经过彼此时, 火车上测量地面的木棍必须根据火车上的时钟同时测量其两端的坐标, 而地面上的人看来, 火车上的这两个测量并不是同时发生的, 于是他们要用自己参考系的时钟重新测量两根木棍, 测出的结果当然就不同了。

至于钟变慢, 我们根据上文的定义, 同样可以说明两个参考系中测量的是完全不同的东西。 地面参考系上的人认为自己的钟都是同步的, 而火车上的钟都不同步。 所以地面上的观测者不能把火车上的一排钟和地面的一排钟直接比较, 只能选定火车上的某个特定的钟进行观测, 记录下它经过地面上每个钟时的事件坐标, 从而得到这个钟走得比地面慢的结论。 如果让火车上的人也把自己的某个钟和地面一排钟做对比, 那么的确也能得到自己这个钟慢的结论。 但对火车上的人来说, 地面的那排钟并不是同步的, 所以他们觉得这么对比的结果意义不大, 他们也想要以自己的那排在他们看来同步的钟和地面的某一个钟对比, 这样做也的确会得到地面的那个钟走得慢的结论。 但由于现在两个参考系做的已经是不是同一个实验, 互相觉得慢也就不矛盾了。

\subsection{双胞胎佯谬}
双胞胎佯谬说, 如果两个双胞胎中一个留在地面, 另一个坐飞船去旅游一圈回来, 那么当他们相见时坐飞船的双胞胎会发现自己比较年轻。 事实上在以上的语境中是不能讨论双胞胎佯谬的, 因为如果地面参考系是惯性系, 而飞船有来有回, 那么飞船参考系就不可能是也一个惯性系。 事实上双胞胎佯谬是半个广义相对论问题, 广义相对论告诉我们, 这个问题中地面观测者观测到的飞船中的钟变慢仍然可以用狭义相对论的方式去计算, 但飞船参考系中观测到的地面上的钟却不能用狭义相对论。 也就是这两个参考系并不是等价的, 牛顿第一定律在地面系仍然成立(不受力的物体仍然匀速直线运动), 但由于飞船需要转向或者变速, 牛顿定律并不成直接成立(会出现所谓的惯性力\upref{Iner})。 所以当两兄弟再次相见时, 他们一致同意坐飞船的那个人比较年轻, 不会得出互相觉得对方年轻的结论。 从逻辑上这也不可能, 因为上文说了同一位置的两个钟比较不会发生歧义。

为了加深理解, 我们不妨在本文的火车的场景中设计一个双生子实验。 如果地面和火车上各有一个双胞胎(, 且当他们相遇



\addTODO{做一个互动演示。}

\addTODO{讨论一下 $t_2', t_a'$ 谁大谁小, 说明一下火车上看地面上静止的木棍同样缩短, 观察地面上的某个钟也同样变慢}


\addTODO{接下来可以另外写一篇, 继续讨论洛伦兹变换的具体形式和意义}

% \addTODO{=========== 以下为回收内容 ==============}

% \addTODO{这么写的确数学太多不像是科普了}
% 假设 $x$ 轴和 $x'$ 轴的方向一样, 火车相对于地面延正方向运动, 那么从火车坐标到地面坐标的洛伦兹变换可以记为
% \begin{equation} \label{eq_Relat0_4}
% \leftgroup{
% &x = \gamma (x' + vt') \\
% &t = \gamma (t' + vx'/c^2)
% }~.
% \end{equation}
% 其中 $\gamma$ 是大于 1 的常数, 火车速度越快, $\gamma$ 越大。 解方程可得逆变换(也就是从地面坐标到火车坐标的变换)为
% \begin{equation}
% \leftgroup{
% &x' = \gamma (x - vt) \\
% &t' = \gamma (t - vx/c^2)
% }~.
% \end{equation}
% 现在来看火车上一个固定的点 $x' = x_0'$ 随着时间 $t'$ 变大选取两个事件 $(x'_0, t'_1)$ 和 $(x'_0, t'_2)$, 如果令它们在地面系得坐标为 $(x_1, t_1)$ 和 $(x_2, t_2)$, 那么带入\autoref{eq_Relat0_4} 可得
% \begin{equation}
% \leftgroup{
% &x_i = \gamma v t_i' + \gamma x_0'\\
% &t_i = \gamma t'_i + \gamma vx_0'/c^2
% } \qquad (i = 1,2)~.
% \end{equation}
% 为了写得更清楚一些,

% 可以看到这一串事件在地面参考系中以速度 $v$ 向右运动。 这是符合上面对相对速度的讨论的。 反之, 如果


% 它的逆变换为(可以通过解二元一次方程解出)



% \addTODO{下面好像设计得有点失败, 似乎应该直接给出洛伦兹变换, 然后解释为什么可以让光速不变公式成立, 再根据公式解释一下尺缩短和钟变慢, 然后搞个互动演示。}

% 现在我们\textbf{似乎}有不同的选择, 一个选择是仍然保持绝对的时间观, 规定两个分母不变, 而让地面的人测量到的被光经过的火车车身长度 $x_2 - x_a$ 相比于火车参考系的 $x_2'-x_1'$ 压缩一些, 使两边的分子也相同。 还记得在一个参考系中测量另一个参考系中的长度是怎么定义的吗? 要在本参考系中的同一时间! 这也是为什么我们定义事件 2, 3 在地面参考系具有相同的时间。 于是我们会得到一个很奇怪的现象, 火车上的同一截车厢, 地面测出来的长度比火车上的人自己测出来的长度要短, 这\textbf{似乎}就得到了著名的尺缩短效应。

% 我们\textbf{似乎}也可以试图仍然顽固地坚持绝对的空间观, 认为那段车身长度不变, 即 $x_2 - x_a = x_2' - x_1'$ 。 而是试图让 $t_2 - t_1$ 比 $t_2' - t_1'$ 大一些。 此时如果只看事件 1 和 3, 它们在火车系的坐标分别为 $(x_1', t_1')$, $(x_1', t_a')$ (坐标不变, 时间增大), 在地面系的坐标分别为 $(x_1, t_1)$, $(x_3, t_2)$。 我们可以把\autoref{eq_Relat0_3} 两个分母的关系记为
% \begin{equation}
% t_2 - t_1 > t_2' - t_1' = (t_2' - t_3') + (t_3' - t_1')
% \end{equation}
% 我们知道其中 $t_2 - t_1$ 和 $t_3' - t_1'$ 都大于零, 所以必须有 $t_2' \ne t_3'$。 这意味着什么? 我们再来单独看事件 3 和 2, 火车中的时空坐标分别为 $(x_1', t_3')$, $(x_2', t_2')$, 地面坐标分别为 $(x_3, t_2)$, $(x_2, t_2)$。 所以 $t_2' \ne t_3'$ 说明在地面系同一时刻不同位置发生的两件事, 在火车系中是不同时的。 那么 2 和 3 到在火车系中谁先呢? 由于 
