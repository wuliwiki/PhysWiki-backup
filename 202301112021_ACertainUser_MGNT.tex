% 磁介质摘要

完全类比于电介质的极化\upref{DLT},我们可以推导出磁介质的极化性质。

\begin{itemize}
\item 在外加磁场下,介质中产生大量总体有序的磁偶极子(顺磁质的磁化\upref{ParaMa}与抗磁质的磁化\upref{DiaMaM})。这使介质的磁偶密度(磁化强度)不再为零$\bvec M$\upref{MaInte}

\item 磁偶极子导致了磁化电流 $\bvec j_s = \curl \bvec M$。\upref{MaInte}
\item 同时,由于电偶的旋转,还产生了极化电流 $\bvec j_P = \pdv{P}{t}$ \upref{PolCur} %似乎这个词条没有说明极化电流的来历

\item 磁化电流、极化电流产生了额外的磁场。磁性介质的存在改变了磁场的分布。\upref{EFIDE} $\bvec B = \bvec B_0 + \bvec B'$, $\curl {\bvec B} = \mu_0 (\bvec j_f + \bvec j_M + \bvec j_p) + \mu_0\epsilon_0 \pdv{\bvec E}{t} = \mu_0 \bvec j_f + \mu_0 \curl \bvec M + \mu_0 \pdv{\bvec P}{t} + \mu_0 \epsilon_0 \pdv{\bvec E}{t}$

\item 为了简化极化电荷的影响,引入“磁场强度”矢量:$\bvec D = \epsilon_0 \bvec E + \bvec P$,并有电位移矢量的高斯定律 $\div \bvec D = \rho_f$ \upref{EFIDE}
\item 在线性电介质中,电偶密度与合电场成线性关系\upref{EFIDE}: $\mathbf P=\chi_{\mathrm e} \varepsilon_{0} \mathbf E$。\footnote{为什么电偶密度与合电场成正比,而不是和外加电场成正比?根据周磊教授的说法,对于任何电荷(不论是介质自带的,还是介质外的自由电荷)而言,他只能感受到最终的合电场,不会(也不能)区分电场的来源,因为这两种电场对电荷产生的效果是相同的;
从另一个角度而言,极化是一个渐进的、需要时间的过程,介质的一部分先极化->产生极化电场->介质的其余部分在已有的极化场和外电场作用下再极化->继续产生极化电场...直到介质被完全极化。幸好在静电学问题中,我们还暂时不需要关心这部分过程。}此时电位移矢量还可以写为 $\bvec D = \epsilon_0 \epsilon_r \bvec E = \epsilon \bvec E $, 其中$\epsilon_r = 1+\chi_{\mathrm e}, \epsilon = \epsilon_0 \epsilon_r$。
\end{itemize}