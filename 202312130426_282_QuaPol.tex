% 二次多项式与二次型
% license Xiao
% type Tutor

\pentry{二次型\upref{QuaFor}, 正定矩阵(实数)\upref{DefMaR}}

所有的 $N$ 元二次(齐次)多项式都可以表示为
\begin{equation}
P(x_1,\dots,x_N) = \sum_{i=1}^N\sum_{j=1}^N a_{i,j}x_i x_j \qquad (a_{i,j}, x_k \in \mathbb R)~,
\end{equation}
其中 $a_{i,j}$ 是对称矩阵 $\mat A$ 的矩阵元。 这是一个二次型\upref{QuaFor}在某组基底下的对称矩阵表示($\bvec x\Tr \mat A \bvec x$)。 由于 $x_i x_j = x_j x_i$, 所以规定 $\mat A$ 为对称矩阵并不影响一般性, 反而可以简化运算。

\subsubsection{极值}
要求二次多项式的极值, 先求驻点:
\begin{equation}
\pdv{x_i}P(x_1,\dots,x_N) = 2\sum_j a_{ij} x_j = 0 \qquad (i = 1,\dots,N)~.
\end{equation}
所以这相当于解齐次线性方程组
\begin{equation}
\mat A \bvec x = \bvec 0~.
\end{equation}
可见它的解集要么是 $\bvec x = \bvec 0$ 一个点, 要么是一个子空间(零空间)\upref{LinEq}。

若矩阵 $\mat A$ 是正定\upref{DefMaR}的, 那么当 $\bvec x \ne \bvec 0$ 时 $\bvec x\Tr \mat A \bvec x > 0$, 即 $\mat A \bvec x \ne \bvec 0$, 可以马上得到只有 $\bvec x = \bvec 0$ 一点是齐次方程组的解, 且该点是全局最小值, 也是唯一一个极值点。 若 $\mat A$ 是负定的, 那么同理 $\bvec x = \bvec 0$ 就是全局最大值。 此时齐次方程组的解只有一个点, 可以推断 $\mat A$ 是满秩的, 可见\textbf{正定(负定)矩阵都是满秩的}。

\subsection{非齐次二次多项式的极值}
\begin{equation}
P(x_1,\dots,x_N) = \sum_{i=1}^N\sum_{j=1}^N a_{i,j}x_i x_j + \sum_{i=1}^N b_i x_i + C~.
\end{equation}
求驻点相当于求解非齐次方程组
\begin{equation}
\mat A \bvec x = -\bvec b~.
\end{equation}
当 $\mat A$ 是正定(负定)时, 由于矩阵是满秩的, 有且仅有一个解, 这个点就是最小(最大)值。
