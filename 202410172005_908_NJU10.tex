% 南京理工大学 2010 年 研究生入学考试试题 普通物理(B)
% license Usr
% type Note

\textbf{声明}:“该内容来源于网络公开资料,不保证真实性,如有侵权请联系管理员”

\subsection{一。填空题(32分,每空2分)}
\begin{enumerate}
    \item 已知一电子的运动方程可表示为 $r = b \cos \omega t\vec{i} + b \sin \omega t\vec{j} + ct\vec{k}$,式中 $a,b$ 为常数,$t$以秒计。$r$以来计,随在$t$时刻,电子的速度为__________,加速度为 ___________。
    \item 一质量为$m$的小球系在长为$L$的细绳的一端,绳的另一端固定于$O$点。先使小球以$v_0$速度做圆周水平匀速运动,然后细绳逐渐缩短,绳始终与运动方向夹角为$\theta$的小球的速度表达式为___________ ,细绳的张力为多大为 ___________。
    \item 设一平面简谐波沿 $x$ 轴正方向传播,已知 $x=0$ 处原点的振动方程为$ y = A \cos(\omega t - \pi/3)$,波速为 $v$。波在 $x=L$ 处终止反射,则$x=x_0$ 处 $(x_0 < L)$ 原点由于反射被引起的振动方程为 ___________ ,$x_0$ 处是波节位置的条件是 $x_0 =$___________。
    \item 如图所示,一沿正 $x$ 方向传播的平面简谐波,波速为 $v = 200 m/s$,波长 $\lambda = 20 m$,则 $x = 0$ 处质点的振动方程为_________,该平面简谐波方程为_________。
\begin{figure}[ht]
\centering
\includegraphics[width=6cm]{./figures/31f0d09795712f3c.png}
\caption{} \label{fig_NJU10_1}
\end{figure}
    \item $2 mol$ 氧气在27°C时的内能等于 ___________,其分子的平均动能是 ___________ ,平均平动动能是 ___________。
    \item 设一个气体分子的密度分布函数为$f(v)$,则单位体积中,$v_1\to v_2$区间内的分子数为 ___________。
    \item 带电量为 $q$ 半径为 $R_1$ 的导体球 $A$,与内、外半径分别为 $R_2$ 和 $R_3$ 接地的同心金属球壳 $B$ 间充满介电常数为 $\varepsilon$ 的介质,构成一球形电容器。则该电容器的电容 $C=$___________。设导体球 $A$ 带电 $q$,则该电容器内任一点 $P$ 处的电场强度 $E=$___________,电容器储存的电能___________。若球壳$B$ 接地,则导体球 $A$ 的电势为 ___________。
\end{enumerate}
\subsection{二、填空题(32分。每空2分)}
1. 半径为 $R$ 的圆环,均匀带电,单位长度的电量为 $\lambda$。 以每秒 $n$ 转绕 $z$ 轴转动, 求环在 $z$ 轴上距环心为 $C$ 的点 $x$ 处的等效磁矩大小为  ___________, 线圈上距 $C$ 为 $x$ 处的任一点 $P$ 的磁感应强度大小为  ___________。

2. 均匀磁场 $B$ 中置一直角边长为 $a$、边有强度为 1 的稳恒电流的等腰直角三角形线圈 $ACD$。 使线圈绕 $AC$ 边匀速转动,线圈平面与磁场方向平行。 如图所示, 现线圈所受的力矩的大小为___________。 在磁力矩作用下,线圈平面绕 $AC$ 边转过 $1/3$ 圈, 磁力矩做的功($I$ 在旋转过程中不变)为___________。

3. 在恒真空场的均匀磁场中,长为 $l$ 的导体棒 $ab$ 以逆时针转的方式绕 $O$ 轴匀速转动, 如图, 则切出电动势的大小为___________,且 ___________ 点的电势为。

4. 在真空中,一平面电磁波的磁场 $B = B_0 \cos \left( \omega \left( t + \frac{z}{c} \right) \right)$, 则该电磁波的电场表达式为  ___________。

