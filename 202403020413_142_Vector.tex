% 【导航】线性代数
% keys 线性代数|几何向量|线性组合|数乘|线性相关|基底|矩阵
% license Xiao
% type Map

\begin{issues}
\issueDraft
\end{issues}
% 原作者是addis,但年代久远没有记录

\subsection{关于此部分}

课程“线性代数”(Linear Algebra)一般指数学系和非数学都需要学习的必修课,开设于低年级(大一/大二),小时百科的线性代数部分基本上涵盖于这门课程。关于现代数学意义下的线性代数,请参考【导航】高级线性代数\upref{mapALA}。

\addTODO{关于解方程}

% Giacomo:整个part应该只有两个目标。1. 作为多变量微积分 2. 用矩阵解线性方程组。

\subsection{列向量、行向量与矩阵}

在高中数学中我们学习了几何向量相关的知识,其中几何向量的坐标给我们了一个理解向量的全新的视角:我们可以把数组称为一个向量。由于我们常常会把它竖着记,这种向量被称为\textbf{列向量}。数组可以和几何向量一样取实数为值,也可以取一些其他的数,比如复数。

\addTODO{数(?)的定义和链接}

实数取值的 $2$ 维(或者 $3$ 维)列向量,等价于选取了坐标系的几何向量——由标准基底的存在,列向量并不是几何向量的推广。几何向量和列向量都是更一般的向量的特殊情况。

% \subsection{行向量与矩阵}

有 $n \times 1$ 的列向量,自然就会有 $1 \times n$ 的行向量,更一般的,我们有 $m \times n$ 的\textbf{矩阵}\upref{Mat}。我们可以定义合适的矩阵(列向量、行向量)之间的\textbf{乘法}关系。


\addTODO{矩阵乘法}

\addTODO{矩阵和几何向量的线性映射的关系}


% 是把标量排成矩形所得到的数学对象。在选定某组基底之后,向量的线性变换可以用其坐标的线性变换表示,并且可以写成矩阵与坐标列向量相乘的形式。

% 词条中要讲逆变换和逆矩阵,这里就算了。

\textbf{旋转矩阵}\upref{Rot2D}可以有两种理解,一是向量绕某个轴相对于当前的正交归一基底转动,其坐标产生了变换,二是向量本身没有变,只是其坐标在两个不同的正交归一基底中不同。这种矩阵的特点是所有列(行)向量都正交归一,所以叫做\textbf{单位正交阵}。单位正交阵的特点是逆矩阵等于转置矩阵。

% Giacomo: 应当移动到其他地方,比如多元微积分导航
% \subsection{向量微积分}
% $N$ 维向量可以作为一个或多个标量的函数(\textbf{向量函数}),可以看成是 $N$ 个普通函数与向量基底的数乘。向量函数同样可以对其自变量求导(或求偏导),也可以积分。不同的是,向量函数还可以进行曲线积分和面积分 %未完成

%介绍几何向量其实是一种抽象的存在,并不需要坐标的辅助就可以定义相加,数乘,内积,叉乘等运算

%然后引入基底的概念,尤其是正交基的概念。然后便是基底转换了。


% 经研究,力学分册绝对不需要用向量空间的概念! 基底,线性组合,线性变换,坐标,等等所有概念都可以通过几何向量讲解! 唯一可能用到其他向量空间的可能性就是傅里叶级数而已!而且傅里叶级数在力学分册中也不会用到啊。
% 正交变换的最高目标估计就是讲明吧旋转矩阵了。
% 讲讲线性方程组的解还是有必要的。
% 要添加矩阵求导法则。

\subsection{线性方程组}

求解解线性方程组\upref{LinEqu}是线性代数学习中的一个主要线索。 虽然我们在中学阶段学过一些简单的技巧, 但要讨论最一般的情况和背后的数学结构, 我们还需要经历一个较为漫长的学习。
%未完成: littlefeng 打算以线性方程组,以及高斯消元法作为线性代数的引入。后续的矩阵初等变换,向量组以及矩阵的秩都和这个知识点息息相关

高斯消元法\upref{GAUSS}是求解线性方程通解的一个重要方法, 适用于所有类型的解, 包括无解, 唯一解和无穷多个解。 另外高斯消元法还引入了线性代数中一种重要的变换即\textbf{行变换}。

对于有唯一解的情况, 另一种方法是使用克拉默法则\upref{kramer}, 但计算量要大许多, 一般不对较大的方程组使用。


% Giacomo:更合理的方式是用矩阵的秩来判断
% TODO:重写这段
% 对于 $N$ 元一次\textbf{齐次方程组}\upref{LinEq2}, 系数矩阵的行列式\upref{Deter}可以用于判断方程的解的个数: 若行列式不为零, 则方程有无穷多个解; 若行列式为零, 则方程有唯一解。 对于非齐次 $N$ 元一次方程, 若行列式不为零, 方程有唯一解; 若行列式为零, 方程有无穷多个解或者无解。 这主要是因为行列式可以用于判断系数方阵中的行或列是否线性无关。 % 链接未完成

% Giacomo:不需要。
% 若想彻底了解线性方程组的解的结构, 我们必须要从矢量空间的角度来理解\upref{LinEq}: 首先我们需要先学习矢量空间\upref{LSpace}, 子空间\upref{SubSpc}, 线性变换\upref{LinMap}, 矩阵的秩\upref{MatRnk}, 然后才能进一步讨论为什么在不同情况下方程组的解会有不同的结构。
