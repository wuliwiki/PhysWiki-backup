% 多项式定理
% 多项式定理|同类项|排列组合|有序数列

\pentry{二项式定理\upref{BiNor}}

我们来吧二项式定理拓展到多项式的情况, 即
\begin{equation}
\qty(\sum_{i = 0}^m a_i)^N = N! \sum_{\{n_i\}}^* \prod_{i = 1}^m \frac{a_i^{n_i}}{n_i!}~.
\end{equation}

\subsection{证明}
首先把指数写成相乘的形式(注意哑标需要各不相同, 否则下面会出错)
\begin{equation}
\qty(\sum_i^m a_i)^N = \qty(\sum_{i_1}^m a_{i_1})\qty(\sum_{i_2}^m a_{i_2})\dots \qty(\sum_{i_N}^m a_{i_N}) = \sum_{i_1}^m \sum_{i_2}^m\dots \sum_{i_N}^m a_{i_1} a_{i_2}\dots a_{i_N}~.
\end{equation}
虽然现在已经拆了括号, 但我们希望能像二项式定理一样写成合并同类项后的形式。 在上式中, 若 $a_{i_1} a_{i_2}\dots a_{i_N}$ ($N$ 个一次项) 中出现了 $n_1$ 个 $a_1$, $n_2$ 个 $a_2$…… $n_m$ 个 $a_m$, 则同类项可以记为 $a_1^{n_1} a_2^{n_2} \dots a_m^{n_m}$, 或用求积符号记为
\begin{equation}
\prod_{i=1}^m a_i^{n_i}~.
\end{equation}
由于一共有 $N$ 项, 必须满足
\begin{equation}
\sum_{i=1}^m n_i = N~,
\end{equation}
用符号 $\{n_i\}$ 表示有序数列 $\{n_1,n_2,\dots, n_m\}$ 的集合。
