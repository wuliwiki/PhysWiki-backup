% 2011 年考研数学试题(数学一)
% keys 考研|数学
% license Copy
% type Tutor
\subsection{选择题}
\begin{enumerate}
\item 曲线 $y=(x-1)(x-2)^2(x-3)^2(x-4)^2$ 的拐点是  ($\quad$)\\
(A)$(1,0)$\\
(B)$(2,0)$\\
(C)$(3,0)$\\
(D)$(4,0)$
\item 设数列 $\{a_n\}$ 单调减少, $\displaystyle \lim_{n\to\infty} a_n=0,S_n=\sum_{k=1}^{n}a_k(n=1,2,\dots)$  无界,则幂级数 $\displaystyle \sum_{n=1}^\infty a_n(x-1)^n$ 的收敛域为($\quad$)\\
(A)$(-1,1]$\\
(B)$[-1,1)$\\
(C)$[0,2)$\\
(D)$(0,2]$
\item  设函数 $f(x)$ 具有二阶连续导数,且 $f(x)>0,f'(0)=0$  ,则函数  $z=f(x)\ln f(y)$ 在点 $(0,0)$ 处取得极小值的一个充分条件是 ($\quad$)\\
(A) $f(0)>1,f''(0)>0$\\
(B) $f(0)>1,f''(0)<0$\\
(C) $f(0)<1,f''(0)>0$\\
(D) $f(0)<1,f''(0)<0$
\item 设 $\displaystyle I=\int_{0}^{\frac{\pi}{4}}\ln (\sin x)\dd{x},J=\int_{0}^{\frac{\pi}{4}}\ln (\cot x)\dd{x},K=\int_{0}^{\frac{\pi}{4}}\ln (\cos x)\dd{x}$ ,则 $I,J,K$ 的大小关系为 ($\quad$)\\
(A) $I<J<K$\\
(B) $I<K<J$\\
(C) $J<I<K$\\
(D) $K<J<I$
\item 设  为3阶矩阵,将  的第2列加到第1列得到矩阵  ,再交换  的

(A)
(B)
(C)
(D)


(A)
(B)
(C)
(D)
\end{enumerate}
