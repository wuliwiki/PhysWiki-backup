% 微分形式
% keys 微分形式|微分几何|外微分|form|1-形式|2-形式|张量|流形|切向量|余切向量|切空间|余切空间

\pentry{对偶空间\upref{DualSp},流形上的切空间\upref{tgSpa}}

\subsection{1-形式}

在流形 $M$ 上给定一个光滑函数 $f$ 和点 $p\in M$ 处的一个切向量 $\bvec{v}$,则由切向量的道路定义容易计算出 $\bvec{v}f(p)$,这个值不依赖于图的选择.不过为了方便理解,既然这是一点处的性质,我们不妨取 $p$ 所在的一个图来考虑,这样就把问题范围从任意流形 $M$ 化为任意欧几里得空间了.

\subsubsection{欧几里得空间一点处的微分形式}



在欧几里得空间 $\mathbb{R}^n$ 中,$\bvec{v}f(p)=\bvec{v}\cdot\nabla f$.我们可以看成是给定 $f$,任意的向量 $\bvec{v}$ 都会被映射为 $\bvec{v}\cdot\nabla f$,这是一个线性映射.不同的 $f$ 有可能对应相同的 $\nabla f$,只要它们全微分的形式是相同的就可以——由此诞生了“微分形式”的概念.

我们用 $\mathbb{R}$ 上的光滑函数举一个例子.考虑函数 $f(x)=x+1$ 和 $g(x)=\E^x$.在 $x=0$ 处,它们的函数值是相同的,斜率也相同,如图\autoref{Forms_fig1} 所示.这样一来,在这个点附近,忽略高阶无穷小的情况下,这两个函数是无法进行区分的.如果两个函数表示两个斜面,那么当你站在 $x=0$ 这一点上的时候,无论是你手表上的高度计读数还是倾斜的感觉,都无法告诉你自己到底站在函数 $f$ 上还是 $g$ 上.

\begin{figure}[ht]
\centering
\includegraphics[width=12cm]{./figures/Forms_1.pdf}
\caption{函数 $f=x+1$ 和 $g=\E^x$ 的例子.} \label{Forms_fig1}
\end{figure}

我们把\textbf{一个点处}函数值和斜率完全相同的全体函数,归为一个等价类,称为这一点处的一个\textbf{微分形式(differential form)}.取这个名字是因为,一个微分形式等价类里包含所有函数,其“微分的形式”都是一样的.比如说,\autoref{Forms_fig1} 中的两个函数,在 $x=0$ 处的微分都是 $\dd f=\dd g=\dd x$.如果在 $x=1$ 处观察,两函数不仅函数值不同,而且“微分的形式”也不一样:$\dd f=\dd x\not=\E\dd x=\dd g$.

上述微分形式的定义,可以拓展到任何 $\mathbb{R}^n$ 上.一般地,如果 $\mathbb{R}^n$ 上的两个光滑函数 $f, g$ 在一点 $p\in\mathbb{R}^n$ 处的函数值相同,并且全微分也相同,那么我们说这两个函数是 $p$ 处同一个微分形式等价类里的元素.直观来说,一个$p$处的微分形式可以用 $\mathbb{R}^n\times\mathbb{R}$ 经过某点 $(p, h(p))$ 的一个平面来代表,所有图像与该平面相切于 $(p, h(p))$ 的函数都属于这个微分形式.在\autoref{Forms_fig1} 的例子中,$p=0, h(p)=1$.

最后,在流形上,由于我们只关心微分形式的“微分的形式”,而不关心具体的函数值,我们也会忽略函数值的差别,而把一点处全微分相同的函数们都归入同一个微分形式中.这是因为在流形上我们把微分形式看成余切向量场,用来和切向量相互作用,函数值是不影响这种作用的.

\subsubsection{流形上的微分形式}

在流形 $M$ 上给定一个光滑函数 $f:M\rightarrow \mathbb{R}$,考虑点 $p\in M$ 处的一个切向量 $\bvec{v}:I\rightarrow M$,我们可以计算出 $f$ 在 $p$ 处沿着 $\bvec{v}$ 方向的方向导数:
\begin{equation}
\bvec{v}f=\lim\limits_{t\rightarrow 0^+}\frac{f(\bvec{v}(t))-f(\bvec{v}(0))}{t}
\end{equation}

由于是求方向导数,因此如果函数 $g$ 和 $f$ 属于 $p$ 处的同一个等价类,那么就有 $\bvec{v}f=\bvec{v}g$.用向量分析的语言来说,$\bvec{v}f$ 就是 $\bvec{v}\cdot\nabla f$,其中 $\nabla f$ 是 $f$ 的梯度向量.$p$ 点处任意微分形式里的全体函数,其在该点的梯度向量都是一样的,因此我们也可以定义这就是微分形式在该点的梯度向量.

我们知道,向量内积实际上是向量和对偶向量的相互作用.这样一来,我们也可以把一点处的一个微分形式看成该点处切向量的一个对偶向量,它的作用是把该点处的切向量映射成一个数字,并且这个映射是线性的.也就是说,一点处的一个微分形式,实际上就是该点处切空间上的一个对偶向量或者叫余切向量.

在欧几里得空间中我们可以很方便地写出一点处任意微分形式的表达:$a_i\dd x^i$.注意这里使用了爱因斯坦求和约定,其中 $a_i\in\mathbb{R}$,而 $x^i$ 是欧几里得空间的坐标.你也可以这么理解,$\dd x^i$ 就是函数 $x^i+c$ 的微分形式,其中 $c$ 是任意常数.这样,$a_i\dd x^i$ 就被视为函数 $f=a_ix^i$ 所在的微分形式了.我们把这个微分形式表示为 $\dd f$.

进一步,流形上的微分形式,不再是指“一点处的微分的形式”或“一点处的一个余切向量”,而是遍及一个区域所有点的“余切向量场”.

\begin{definition}{流形上的微分形式}
给定实流形 $M$.称 $M$ 上的一个\textbf{光滑余切向量场}为一个\textbf{微分形式(differential form)},或者更具体地,一个\textbf{1-形式(1-form)}.

在 $p\in M$ 处的一个余切向量,可以表示为 $\dd f$,其中 $f$ 是定义在 $p$ 的一个邻域中的光滑函数,且对于任意切向量 $\bvec{v}$,都有 $\bvec{v}$ 和 $\dd f$ 的乘积恰为切向量对函数的作用:$\bvec{v}f$.
\end{definition}

\subsection{$k$-形式}
\pentry{外代数\upref{ExtAlg}}

我们已经把微分形式定义为光滑余切向量场了.这样一来,流形 $M$ 上一点 $p$ 处的全体微分形式,构成的余切向量集合,就是一个线性空间 $T^*_pM$.我们自然可以在这个线性空间上定义一个外代数 $\bigwedge T^*_pM$.

\begin{definition}{$k$-形式}
给定实流形 $M$ 上一点 $p$ 处的余切空间 $T^*_pM$,其外积空间 $\bigwedge^kT^*_pM$ 中的任一元素为 $p$ 处的一个\textbf{$k$-形式($k$-form)}.

$M$ 上的一个光滑\footnote{此处“光滑”依然指“在任意坐标图中场系数是光滑函数”.}$k$-形式场\footnote{这是一个 $k$ 阶张量场.},称为 $M$ 上的一个\textbf{$k$-形式}.$M$ 上 $k$-形式的集合,记为 $\Omega^k(M)$.
\end{definition}

流形上的光滑张量场的集合,本身也可以看成一个线性空间.在本节的例子里,流形上 $k$-形式的集合,本身也是一个线性空间,每个 $k$-形式都是该空间里的一个向量.从这个意义上来说,$\Omega^k({M})$ 中的每一个 $k$-形式场也可以看成是 $M$ 上把余切向量场看成一个向量时的 $k$ 次外积.

在度量张量的帮助下,余切向量也可以和切向量进行一一对应,此时我们就可以把余切向量也表示成一个切向量,其和其它切向量的相互作用表现为内积.这样一来,某点上的一个1-形式可以理解为一个箭头;2-形式可以理解为两个箭头张成的面积,注意这个面积是有方向的;类似地,一个 $k$-形式就可以理解为 $k$ 维的有向体积.








