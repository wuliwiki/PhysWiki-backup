% 库仑定律(综述)
% license CCBYSA3
% type Wiki

本文根据 CC-BY-SA 协议转载翻译自维基百科\href{https://en.wikipedia.org/wiki/Coulomb\%27s_law}{相关文章}。

\begin{figure}[ht]
\centering
\includegraphics[width=6cm]{./figures/e58a5f04bd7289ad.png}
\caption{两个点电荷 q1 和 q2 之间的静电力 F 的大小与它们电荷大小的乘积成正比,且与它们之间距离的平方成反比。相同电荷相互排斥,异性电荷相互吸引。} \label{fig_KL_1}
\end{figure}
库仑的反比平方定律,简称库仑定律,是一条物理学的实验定律[1],用于计算静止的两个带电粒子之间的相互作用力。这种电力通常称为静电力或库仑力[2]。尽管这一定律早有认识,但它是在1785年由法国物理学家查尔-奥古斯丁·库仑首次公布的。库仑定律对于电磁学理论的发展至关重要,甚至可能是其起点[1],因为它使得有意义的讨论粒子电荷量成为可能[3]。

该定律指出,两个点电荷之间的吸引或排斥静电力的大小(或绝对值)与它们电荷大小的乘积成正比,且与它们之间距离的平方成反比[4]。库仑发现,相同电荷的物体会相互排斥:

因此,从这三个实验中可以得出结论:两个带有相同类型电荷的球体相互排斥的力,遵循与距离的平方成反比的规律[5]。

库仑还表明,带相反电荷的物体会按照反比平方定律相互吸引:
\[
|F| = k_{\text{e}} \frac{|q_1| |q_2|}{r^2}~
\]
其中,\( k_{\text{e}} \) 是常数,\( q_1 \) 和 \( q_2 \) 是两个电荷的电量,\( r \) 是它们之间的距离。

力的方向沿着连接两个电荷的直线。如果电荷具有相同的符号,静电力使它们相互排斥;如果电荷符号不同,静电力使它们相互吸引。

作为一个反比平方定律,它类似于艾萨克·牛顿的万有引力反比平方定律,但引力总是使物体相互吸引,而静电力则使电荷相互吸引或排斥。此外,万有引力远弱于静电力[2]。库仑定律可以用来推导高斯定律,反之亦然。在静止的单个点电荷情况下,这两个定律是等效的,以不同的方式表达相同的物理规律[6]。这一定律已经被广泛验证,观测结果在从 \( 10^{-16} \) 米到 \( 10^8 \) 米的尺度上都符合该定律[6]。
\subsection{历史}
\begin{figure}[ht]
\centering
\includegraphics[width=6cm]{./figures/a90c2a11cbe06783.png}
\caption{查尔-奥古斯丁·库仑} \label{fig_KL_2}
\end{figure}
地中海周围的古代文化知道某些物体,如琥珀棒,可以通过与猫毛摩擦来吸引轻物体,如羽毛和纸片。米利都的泰利斯大约在公元前600年首次记录了静电现象[7],当时他注意到摩擦可以使一块琥珀吸引小物体[8][9]。

1600年,英国科学家威廉·吉尔伯特对电学和磁学进行了仔细研究,将磁铁效应与摩擦琥珀产生的静电区分开来[8]。他创造了新拉丁词“electricus”(意为“像琥珀一样”或“来自琥珀”,源自希腊语单词 ἤλεκτρον [elektron],意为“琥珀”),用来指代摩擦后能吸引小物体的性质[10]。这一联系促成了英语单词“electric”和“electricity”的诞生,它们首次出现在1646年托马斯·布朗的《伪常识》一书中[11]。

18世纪早期的研究者怀疑电力与重力一样会随距离衰减(即按距离的平方反比衰减),包括丹尼尔·伯努利[12]和亚历山德罗·伏打,他们都测量了电容器板之间的力,以及弗朗茨·艾皮努斯,他在1758年提出了反比平方定律[13]。

基于电荷球体的实验,英国的约瑟夫·普里斯特利是最早提出电力遵循反比平方定律的科学家之一,类似于牛顿的万有引力定律。然而,他并未进一步推广或详细阐述这一点[14]。在1767年,他猜测电荷之间的力随距离的平方反比变化[15][16]。
\begin{figure}[ht]
\centering
\includegraphics[width=6cm]{./figures/20d05c283241d275.png}
\caption{库仑的扭摆天平} \label{fig_KL_3}
\end{figure}
1769年,苏格兰物理学家约翰·罗宾逊宣布,根据他的测量,两个带有相同电荷符号的球体之间的排斥力随距离变化为 \(x^{-2.06}\)[17]。

在1770年代初期,英国的亨利·卡文迪许已经发现了带电物体之间的力与距离和电荷的关系,但并未发表[18]。在他的笔记中,卡文迪许写道:“因此我们可以得出结论,电吸引力和排斥力必定与距离的某种幂次成反比,其中2加1/50次方与2减1/50次方相差无几,而且没有理由认为它与反比平方的比例有所不同。”

最终,1785年,法国物理学家查尔-奥古斯丁·库仑发表了他关于电学和磁学的三篇报告,首次提出了他的定律。这篇出版物对于电磁学理论的发展至关重要[4]。他使用了扭摆天平来研究带电粒子的排斥力和吸引力,并确定了两个点电荷之间的电力大小与电荷的乘积成正比,与它们之间距离的平方成反比。
\subsection{数学形式}
\begin{figure}[ht]
\centering
\includegraphics[width=14.25cm]{./figures/081c4deeec64aa8c.png}
\caption{在图中,向量 \(\mathbf{F}_1\) 是 \(q_1\) 受到的力,向量 \(\mathbf{F}_2\) 是 \(q_2\) 受到的力。当 \(q_1 q_2 > 0\) 时,力是排斥的(如图所示);当 \(q_1 q_2 < 0\) 时,力是吸引的(与图像相反)。力的大小始终是相等的。} \label{fig_KL_4}
\end{figure}
库仑定律表明,在真空中,某一电荷 \(q_1\) 在位置 \(\mathbf{r}_1\) 受到另一个电荷 \(q_2\) 在位置 \(\mathbf{r}_2\) 产生的静电力 \(\mathbf{F}_1\) 为[19]:
\[
\mathbf{F}_1 = \frac{q_1 q_2}{4 \pi \varepsilon_0} \frac{\hat{\mathbf{r}}_{12}}{|\mathbf{r}_{12}|^2}~
\]
其中,\(\mathbf{r}_{12} = \mathbf{r}_1 - \mathbf{r}_2\) 是两个电荷之间的位移向量,\(\hat{\mathbf{r}}_{12}\) 是从 \(q_2\) 指向 \(q_1\) 的单位向量,\(\varepsilon_0\) 是电常数。这里,\(\hat{\mathbf{r}}_{12}\) 用于表示向量的方向。根据牛顿第三定律,电荷 \(q_2\) 受到的静电力 \(\mathbf{F}_2\) 为:
\[
\mathbf{F}_2 = -\mathbf{F}_1~
\]
如果两个电荷具有相同的符号(同种电荷),则乘积 \(q_1 q_2\) 为正,力的方向由 \(\hat{\mathbf{r}}_{12}\) 给出;这时电荷互相排斥。如果电荷符号相反,则乘积 \(q_1 q_2\) 为负,力的方向为 \(-\hat{\mathbf{r}}_{12}\);这时电荷互相吸引[20]。
\subsubsection{离散电荷系统}  
叠加原理使得库仑定律可以扩展到包括任意数量的点电荷。一个点电荷受到多个点电荷系统的作用力,可以通过将每个电荷单独作用于该点电荷的力向量相加得到。最终的力向量与该点的电场向量平行,而该点电荷已被移除。

在真空中,位置为 \(\mathbf{r}\) 的小电荷 \(q\) 受到 \(n\) 个离散电荷系统的作用力 \(\mathbf{F}(\mathbf{r})\) 为[19]:
\[
\mathbf{F}(\mathbf{r}) = \frac{q}{4\pi \varepsilon_0} \sum_{i=1}^{n} q_i \frac{\hat{\mathbf{r}}_i}{|\mathbf{r}_i|^2}~
\]
其中,\(q_i\) 是第 \(i\) 个电荷的电量,\(\mathbf{r}_i\) 是从第 \(i\) 个电荷位置到 \(\mathbf{r}\) 的向量,\(\hat{\mathbf{r}}_i\) 是指向 \(\mathbf{r}_i\) 的单位向量。
\subsubsection{连续电荷分布} 
在这种情况下,线性叠加原理也适用。对于连续电荷分布,对包含电荷的区域进行积分等同于无限求和,将空间中的每个无穷小元看作一个点电荷 \(dq\)。电荷分布通常是线性的、表面分布的或体积分布的。

对于线性电荷分布(如电线中的电荷),其中 \(\lambda(\mathbf{r'})\) 表示位置 \(\mathbf{r'}\) 处单位长度的电荷量,\(d\ell'\) 是一个无穷小的长度元[21]:
\[
dq' = \lambda (\mathbf{r'}) \, d\ell'.~
\]
对于表面电荷分布(如平行板电容器板上的电荷),其中 \(\sigma(\mathbf{r'})\) 表示位置 \(\mathbf{r'}\) 处单位面积的电荷量,\(dA'\) 是一个无穷小的面积元:
\[
dq' = \sigma (\mathbf{r'}) \, dA'.~
\]
对于体积电荷分布(如金属中的电荷),其中 \(\rho(\mathbf{r'})\) 表示位置 \(\mathbf{r'}\) 处单位体积的电荷量,\(dV'\) 是一个无穷小的体积元[20]:
\[
dq' = \rho (\mathbf{r'}) \, dV'.~
\]
在真空中,位置为 \(\mathbf{r}\) 的小测试电荷 \(q\) 所受的力是通过对电荷分布进行积分得到的:
\[
\mathbf{F}(\mathbf{r}) = \frac{q}{4 \pi \varepsilon_0} \int dq' \frac{\mathbf{r} - \mathbf{r'}}{|\mathbf{r} - \mathbf{r'}|^3}.~
\]
库仑定律的“连续电荷”版本不应应用于 \(|\mathbf{r} - \mathbf{r'}| = 0\) 的位置,因为该位置会直接与带电粒子(如电子或质子)的位置重合,这不是经典电场或电势分析的有效位置。现实中电荷总是离散的,“连续电荷”假设仅仅是一个近似,它不应该允许分析 \(|\mathbf{r} - \mathbf{r'}| = 0\) 的情况。
\subsection{库仑常数} 
库仑定律中的比例常数 \(\frac{1}{4 \pi \varepsilon_0}\) 是单位选择的历史结果。[19]: 4–2

常数 \(\varepsilon_0\) 是真空电介质常数。[22] 根据 CODATA 2022 推荐的 \(\varepsilon_0\) 值,[23] 库仑常数 [24] 为:
\[
k_{\text{e}} = \frac{1}{4 \pi \varepsilon_0} = 8.987\ 551\ 7862(14) \times 10^9 \ \mathrm{N \cdot m^2 \cdot C^{-2}}.~
\]
\subsubsection{限制条件}
库仑的反平方定律有效性需要满足以下三个条件:[25]
\begin{enumerate}
\item 电荷必须具有球对称分布(例如,点电荷或带电金属球)。
\item 电荷不能重叠(例如,它们必须是不同的点电荷)。
\item 电荷必须相对于一个非加速的参考系静止。
\end{enumerate}
最后一个条件被称为静电近似。当电荷发生运动时,会引入一个额外的因素,这会改变两个物体之间的作用力。这一额外的力部分被称为磁力。在缓慢运动的情况下,磁力是最小的,库仑定律仍然可以近似地认为是正确的。然而,在这种情况下,一个更精确的近似是韦伯力。当电荷相对运动较快或发生加速时,必须考虑麦克斯韦方程和爱因斯坦的相对论理论。
\subsubsection{电场}  
\begin{figure}[ht]
\centering
\includegraphics[width=6cm]{./figures/b0fb29a55d908d5f.png}
\caption{“如果两个电荷具有相同的符号,它们之间的静电力是排斥的;如果它们的符号不同,它们之间的力是吸引的。”} \label{fig_KL_5}
\end{figure}
电场是一个向量场,它将空间中的每个点与单位测试电荷所受的库仑力相关联。[19] 电荷 \(q_t\) 所受的库仑力 \(\mathbf{F}\) 的强度和方向取决于它所处位置的电场 \(\mathbf{E}\),即:\(\mathbf{F} = q_t \mathbf{E}\)在最简单的情况下,电场被认为仅由一个源点电荷生成。更一般地,电场可以由一组电荷的分布产生,这些电荷通过叠加原理共同贡献。

如果电场是由一个正点电荷 \(q\) 产生的,则电场的方向沿着从源点辐射出去的径向线,即一个正点测试电荷 \(q_t\) 如果被放置在电场中,它将沿着该方向运动。对于负点源电荷,电场方向则是径向指向源点。

电场强度 \(E\) 的大小可以从库仑定律推导出来。通过选择一个点电荷作为源电荷,另一个作为测试电荷,根据库仑定律,单个源点电荷 \(Q\) 在距离 \(r\) 处产生的电场强度为:
\[
|\mathbf{E}| = k_{\text{e}} \frac{|q|}{r^2}~
\]
一个由 \(n\) 个离散电荷 \(q_i\) 组成的系统,电荷位置为 \(\mathbf{r}_i\),在位置 \(\mathbf{r}\) 处产生的电场,其大小和方向由叠加原理给出:
\[
\mathbf{E}(\mathbf{r}) = \frac{1}{4 \pi \varepsilon_0} \sum_{i=1}^{n} q_i \frac{\hat{\mathbf{r}}_i}{|\mathbf{r}_i|^2}~
\]
\subsubsection{原子力}    
库仑定律即使在原子内部也成立,能够正确描述带正电的原子核与每个带负电的电子之间的力。这一定律同样能正确解释将原子结合成分子、以及将原子和分子结合成固体和液体的力。一般来说,随着离子之间距离的增加,吸引力和结合能趋近于零,离子键合变得不那么有利。而当相反电荷的大小增加时,能量增加,离子键合变得更有利。
\subsection{与高斯定律的关系}
\subsubsection{从库仑定律推导高斯定律}  
本节摘自《高斯定律 § 从库仑定律推导高斯定律》。  
[需要引用] 严格来说,仅凭库仑定律无法推导出高斯定律,因为库仑定律仅给出单个静电点电荷所产生的电场。然而,如果再假设电场遵循叠加原理,那么可以从库仑定律推导出高斯定律。叠加原理指出,结果电场是由每个粒子产生的电场的向量和(或者,如果电荷在空间中平滑分布,则是积分)。\\
\textbf{证明大纲}\\
库仑定律指出,静止点电荷所产生的电场为:
\[
\mathbf{E}(\mathbf{r}) = \frac{q}{4 \pi \varepsilon_0} \frac{\mathbf{e_r}}{r^2}~
\]
其中:
\begin{itemize}
\item \(\mathbf{e_r}\) 是径向单位向量,
\item \(r\) 是半径,\(|r|\) 是距离,
\item \(\varepsilon_0\) 是电常数,
\item \(q\) 是粒子的电荷量,假设粒子位于原点。
\end{itemize}