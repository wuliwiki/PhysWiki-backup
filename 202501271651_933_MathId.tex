% 数学归纳法(高中)
% keys 数学归纳法|递推|归纳法
% license Usr
% type Tutor

\begin{issues}
\issueDraft
\end{issues}

\pentry{数列\nref{nod_HsSeFu}}{nod_aa25}

数列可以被视为一个从自然数集合到实数集合的映射,其中每个自然数 $n$ 唯一对应一个实数 $a_n$。数列的通项公式不仅提供了这种对应关系的计算规则,还可以理解为一种筛选机制。筛选的过程就是通过不断验证命题“自然数 $n$ 与其对应的 $a_n$满足通项公式 $a_n = f(n)$ ”是否为真来找到数列的每个值。例如,若通项公式定义为 $a_n = 2n + 1$,选择 $n = 3$ 。设对应的 $a_3 = 6$,代入公式则得 $2 \cdot 3 + 1 = 7 \neq 6$,命题为假,说明 $a_3 = 6$ 不满足公式;设取对应的 $a_3 = 7$,代入验证有 $2 \cdot 3 + 1 = 7$,命题为真,表明 $a_3 = 7$ 满足公式。因此$a_3=7$。

由此可见,通项公式就像一系列逻辑命题,通过逐一验证自然数的取值,从庞大实数集合中筛选出符合定义的值,从而构成数列。这种“逐一验证”的思想可以进一步推广到更广泛的命题形式,包括等式、不等式,以及没有明确代数表达的陈述。只要命题与自然数相关,就可以类比数列的逐一验证过程,设计一个系统化的证明方法。这种证明方法在数学中被称为数学归纳法。数学归纳法不仅在数列的研究中具有重要作用,还广泛应用于多项式、几何等领域的证明。作为高中数学的核心工具之一,数学归纳法能够证明几乎所有与自然数相关的命题,展现出数学的强大力量和广泛适用性。

针对高中考试,数学归纳法通过将复杂的证明过程转化为对最终结果的猜想,提供了一种高效且直观的方法。其特点在于,学生可以绕过繁琐的推导,通过“猜想”直接得出结论。即使跳过了完整的推理过程,仍然可以利用数学归纳法对结果的正确性加以验证。这种方式不仅简化了复杂问题的解决,也为学生提供了应对考试中证明题的有效策略。同时,数学归纳法对“如何猜”提出了更高要求。这一过程体现了学生的数学观察和推断能力,常见的猜想方法包括:通过观察题目中的模式、利用待定系数法、计算几项具体值以总结规律,甚至借助超纲方法来得到初步结论。这些方法不仅是解题的工具,也在训练中帮助学生深化对数学本质的理解。

\subsection{从多米诺骨牌开始}

在介绍数学归纳法之前,先来看一种常见的连锁效应——多米诺骨牌。多米诺骨牌由一块块大小相同的长方体木块组成,每块木块通常只有几厘米高,几毫米厚。这些木块在平坦的桌面上可以独立站立,但由于形状特点,它们极易受力倒下。人们将这些骨牌按照一定的间距排列成一列或更复杂的图案,使得每块骨牌既能独立站立,又能相互影响。当第一块骨牌倒下并恰好撞倒第二块时,第二块继续撞倒第三块,依此类推,整排骨牌便会连续倒下,形成一个连锁反应。

设想一下,如果要描述这一过程,最简单的方案是什么?一个直观的设计是让每块骨牌与下一块保持足够近的距离,并确保倒下的方向能够推倒下一块。这时,只需要推倒第一块骨牌,就可以触发整个过程。这种设计的关键在于,每块骨牌的倒下虽然只影响下一块,但通过前后传递的机制,最终能够让整排骨牌依次倒下。

这种“传递性”的现象正是数学归纳法背后的核心思想。数学归纳法的逻辑和多米诺骨牌非常相似:首先检查命题对某个特定起始点(通常是最小值)是否成立;然后假设命题在任意选取的某个自然数$k$成立,再来验证在此前提下命题对自然数$k+1$是否成立。只要成立,就可以推导出整排骨牌都会倒下,或者说某个规律适用于所有情况。

\begin{definition}{数学归纳法}
对某个与自然数相关的命题$P(n)$,利用\textbf{数学归纳法}证明$P(n)$成立的过程分为三步:
\begin{enumerate}
\item 验证:命题$P(0)$成立成立。
\item 假设:假设 $n = k$ 时,命题$P(k)$成立。
\item 归纳:证明命题在假设的前提下,可以推导命题$P(k+1)$也成立。
\end{enumerate}
如果以上三步都完成,就可以得出结论:该命题对所有自然数$n$都成立。
\end{definition}

虽然数学归纳法的名称中包含“归纳”一词,但它并非通常意义上的\aref{归纳推理}{sub_HsLogi_1},而是一种完全严谨的演绎推理方法。数学归纳法通过递归式的证明,从特殊情况推导出一般性结论。与反证法等其他证明方法相比,数学归纳法更适用于递推关系明确的问题。

从理论上讲,数学归纳法是一种基于公理的模式,其本身的正确性无法被进一步证明,其实,它是定义自然数时的一个基本公理。这意味着,只要自然数的概念成立,数学归纳法就天然适用,并且在与自然数相关的证明中具有不可或缺的地位。

在使用数学归纳法时,需要注意一些常见的误区。首先,基础验证是关键的一步,如果未验证初始情况是否成立,整个证明将缺乏起点支撑,无法完整进行。其次,归纳步骤必须严谨,需从归纳假设  P(k)  完整推导出  P(k+1) ,避免推理过程中的逻辑漏洞或不完整证明。此外,在应用归纳法之前,需要明确命题的适用范围,确保归纳法适用于问题的所有自然数或其子集。

另一个容易被忽视的问题是,证明过程中可能会引入未经验证的假设,或者错误地认为某些条件“显然成立”。这些未经严谨论证的假设可能破坏证明的逻辑性。同时,在推理过程中,还需注意可能存在的特殊情况,例如某些特殊值导致推论不成立。这些隐患可能影响归纳法的应用效果,需要在实际使用中仔细排查。

数学归纳法的意义主要体现在两个方面。一是它确保自然数的完备性,可以验证某一性质对所有自然数都成立,而不必逐一检查每一个自然数。二是它为递归定义提供了严格的验证工具,能够证明递归定义的正确性,从而确保数学推导的严谨性。这些特点使得数学归纳法成为数学证明中极为重要的一种方法。


在使用数学归纳法时,需要注意一些常见的误区。首先,基础验证是关键一步,若未验证初始情况是否成立,证明将缺乏起点支撑,无法完整进行。其次,归纳步骤必须严谨,应严格从归纳假设  $P(k)$  推导  $P(k+1)$ ,避免推理过程中的逻辑漏洞或不完整证明。此外,在使用归纳法之前,需要明确命题的适用范围,确保归纳法应用于正确的情境。另一个容易被忽视的问题是,证明过程中可能引入未经验证的假设或错误认为显然成立的“事实”,从而影响证明的严谨性。同时,在推理过程中,需注意某些特殊值可能导致推论不成立,这些特殊情况也可能成为证明中的隐患。

虽然数学归纳法的名称中包含“归纳”一词,但它并非通常意义上的\aref{归纳推理}{sub_HsLogi_1},而是一种完全严谨的演绎推理方法。它通过递归式证明,从特殊情况推导出一般性结论。与反证法等其他证明方法相比,数学归纳法更适用于递推关系明确的问题。

从理论上讲,数学归纳法是一种基于公理的模式,其本身无法进一步证明,而是自然数定义中的一个基本公理。这意味着只要有自然数存在的场合,就天然存在数学归纳法。

数学归纳法的意义主要体现在两个方面。一是它确保自然数的完备性,可以验证某一性质对所有自然数都成立,而不必逐一检查每一个自然数。二是它为递归定义提供了严格的验证工具,可以证明递归定义的正确性,从而确保数学推导的严谨性。这些特点使得数学归纳法成为数学证明中极为重要的一种方法。

\subsection{实际应用}


所有与自然数有关的命题都可以通过数学归纳法来证明。读者可以自行尝试一下利用数学归纳法证明“\enref{恒等式与恒成立不等式}{HsIden}”中给出的表达式。

数学归纳法的实际应用
前面介绍的数学归纳法比较抽象,尽管过程已经很清晰了,但由于之前从未接触过,具体使用却仍让人感到陌生。下面以几个实例来感受一下数学归纳法的强大。
数学归纳法在以下方面有广泛应用:

\subsubsection{证明恒等式}
下面的例题,曾经应用在\aref{等比数列}{sub_HsGmPg_1}中给出证明。通常,类似的恒等式都可以通过数学归纳法证明。
\begin{example}{证明:对于任意自然数$n$都有$\displaystyle a^{n}-b^{n}=\left(a-b\right)\sum_{i=0}^{n-1}a^{i}b^{n-i}$。}
证明:

当 $n = 0$ 时\footnote{关于成立原因,参见\enref{求和符号}{SumSym}}:
\begin{equation}
a^0 - b^0 = 1-1=0=(a - b)\times0~.
\end{equation}
因此,命题在 $n = 0$ 时成立。

假设命题对 $n = k$ 成立,即:
\begin{equation}
a^k - b^k = (a - b)\left(a^{k-1} + a^{k-2}b + \cdots + b^{k-1}\right)~.
\end{equation}
需证明命题对 $n = k+1$ 成立,代入假设有:
\begin{equation}
\begin{aligned}
a^{k+1} - b^{k+1} &= a^{k+1}-ab^k+ab^k - b^{k+1}\\
&= a \cdot (a^k - b^k) + b^k(a - b)\\
&=a \cdot (a - b)\left(a^{k-1} + a^{k-2}b + \cdots + b^{k-1}\right) + b^k(a - b)\\
&=(a - b)\left[a \cdot \left(a^{k-1} + a^{k-2}b + \cdots + b^{k-1}\right) + b^k\right]\\
&=(a - b)\left(a^{k} + a^{k-1}b + \cdots + ab^{k-1} + b^k\right)~.
\end{aligned}
\end{equation}
与题设形式一致。

综上由数学归纳法可知:
\begin{equation}
a^n - b^n = (a - b)\left(a^{n-1} + a^{n-2}b + \cdots + b^{n-1}\right)~.
\end{equation}
对所有自然数成立。
\end{example}

\subsubsection{不等式恒成立}

有一些问题,结论并不那么显然,在代入处理时,需要利用其他已知条件来辅助放缩。

\begin{example}{证明 $\E^n \geq n+1$ 对所有自然数成立。}
证明:

当 $n = 0$ 时,$\E^0 = 1\geq 0+1$,成立。

假设 $\E^k \geq k+1$ 成立。需要证明 $n = k+1$ 时命题也成立。
由于$\E>2$且$k\geq0$,有:
\begin{equation}
(k + 1) \cdot \E > (k + 1) \cdot 2 = 2k + 2 \geq k + 2~.
\end{equation}
根据假设,可得:
\begin{equation}
\begin{aligned}
\E^{k+1} &= \E \cdot \E^k\\
&\geq \E(k+1)\\
&\geq k+2~.
\end{aligned}
\end{equation}
与命题的形式一致。综上,利用数学归纳法,可以证明$\E^n \geq n+1$对所有整数$n \geq 0$ 都成立。
\end{example}

\subsubsection{整除问题}

由于整除是一个与自然数关系很大的命题,因此很多与整除有关的命题也可以利用数学归纳法来证明。它也成为数论领域的重要工具。

\begin{example}{证明:对任意自然数$n$,$9^{n} - 1$ 能被 $8$ 整除。}
证明:

当 $n = 0$ 时:$9^0 - 1 = 1 - 1 = 0$。由于$0$能被任何整数整除,显然命题在 $n = 0$ 时成立。

假设当 $n = k$ 时,命题成立,这意味着存在整数 $m$,使得:
\begin{equation}
9^k - 1 = 8m~.
\end{equation}
当 $n = k+1$ 时,代入假设,有:
\begin{equation}
\begin{aligned}
9^{k+1} - 1 &= 9\cdot9^{k} - 1\\
&= 9\cdot(8m+1)- 1\\
&= 8\cdot9m+9- 1\\
&= 8\cdot(9m+1)~.
\end{aligned}
\end{equation}
显然,$9^{k+1} - 1$ 是 $8$ 的倍数,所以能被 $8$ 整除。

由数学归纳法可知,对所有正整数$n$,$9^{n} - 1$ 能被 $8$ 整除。
\end{example}
除却上面的例子,如从数列的递推公式得到通项公式、多边形内角和公式以及排列组合或其他算法的递归性质上,都可以窥见数学归纳法的身影。可以这么说,数学归纳法使得很多过去觉得显然的命题,有了逻辑依赖。数学归纳法并不只在做题中有用,通常想要证明一个命题对实数都成立,可以先利用数学归纳法证明它对自然数成立,然后再扩展至有理数,最后利用极限再扩展至实数。因此,数学归纳法是很多命题的源头。



\subsection{*数学归纳法的变式}

下面介绍的方法都是可以从数学归纳法推知的其他类型的数学归纳法。因此,可以看作这些归纳法都是在数学归纳法的前提下给出了一些其他的条件来辅助。由于使用了一些条件,使得证明过程更简单。这些内容在高中阶段完全不涉及,仅作开阔视野。

\subsubsection{从$N$开始的数学归纳法}
高中教科书给出的是这个。
比如从 n = 2 或 n = 0 起步。

\subsubsection{强归纳法}

:假设命题对多个前面情况成立,再推导出后续。

\subsubsection{逆归纳法}
