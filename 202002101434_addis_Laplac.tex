% 拉普拉斯算符
% 梯度|散度|拉普拉斯方程

\pentry{矢量算符\upref{Grad}}

我们令一个标量函数 $u(x, y)$ 的梯度的散度为它的\textbf{拉普拉斯(Laplacian)}, 合成的算符(类比复合函数)叫做\textbf{拉普拉斯算符}, 记为 $\laplacian$.
\begin{equation}
\laplacian u = \div (\grad u)
\end{equation}
也可以记
\begin{equation}
\laplacian = \Nabla \vdot \Nabla
\end{equation}


在直角坐标系中, 散度的两个分量分别为 $\pdv*{u}{x}$ 和 $\pdv*{u}{y}$, 再求旋度, 就是分别对两个分量关于 $x, y$ 再求一次偏导得
\begin{equation}
\laplacian u = \pdv[2]{u}{x} + \pdv[2]{u}{y}
\end{equation}
从形式上, 我们可以写成算符的形式(有点类似矢量的内积)
\begin{equation}
\laplacian = \qty(\uvec x \pdv{x} + \uvec y \pdv{y}) \qty (\uvec x \pdv{x}  + \uvec y \pdv{y})
= \pdv[2]{x} + \pdv[2]{y}
\end{equation}
