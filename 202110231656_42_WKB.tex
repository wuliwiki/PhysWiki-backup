% WKB 近似

\begin{issues}
\issueDraft
\end{issues}
在某些定态问题中,WKB近似方法可以比较容易地求解薛定谔方程.该方法基于将波函数按 $\hbar$ 作幂级数展开,就其本身而言,有两个基本问题:1.在远离转折点处的近似解;2.在转折点处的连接条件.并通过这两个问题求解薛定谔方程.

经典区域
\begin{equation}
\psi(x) \approx \frac{C}{\sqrt{p(x)}} \exp(\pm \I \int p(x) \dd{x})
\end{equation}

隧道区域
\begin{equation}
\psi(x) \approx \frac{C}{\sqrt{\abs{p(x)}}} \exp(\pm \int p(x) \dd{x})
\end{equation}

在二者的转折点, 假设势能为线性函数, 其解是艾里函数 $\opn{Ai}$, 详见 “线性势能的定态薛定谔方程\upref{LinPot}”.

从经典到非经典区域
\begin{equation}
\psi(x) = \leftgroup{
&\frac{B}{\sqrt{p(x)}} \exp(\I \int_{-x}^{x_0} p(x')\dd{x'}) + \frac{C}{\sqrt{p(x)}} \exp(-\I \int_{-x}^{x_0} p(x')\dd{x'}) \qquad &(x < x_0)\\
&\frac{D}{\sqrt{\abs{p(x)}}} \exp(-\int_{x_0}^x \abs{p(x')} \dd{x'})  \qquad &(x > x_0)
}\end{equation}
衔接以后
\begin{equation}
\psi(x) = \leftgroup{
&\frac{2D}{\sqrt{p(x)}} \sin(\int_{x}^{x_0} p(x')\dd{x'} + \frac{\pi}{4}) \quad &(x < x_0)\\
&\frac{D}{\sqrt{\abs{p(x)}}} \exp(-\int_{x_0}^x \abs{p(x')} \dd{x'}) \quad &(x > x_0)
}\end{equation}
\subsection{证明}
\subsubsection{WKB近似解}
薛定谔方程
\begin{equation}\label{WKB_eq2}
\I\hbar\pdv{\psi}{t}=-\frac{\hbar^2}{2m}\Delta\psi+V(\bvec{r})\psi
\end{equation}
的解一般总能写成如下形式
\begin{equation}\label{WKB_eq1}
\psi(\bvec r,t)=A\E^{\I W(\bvec r,t)/\hbar}
\end{equation}
\autoref{WKB_eq1} 代入
\addTODO{Griffiths 例题 Potential well with one vertical wall.}
