% 电偶极子 2
% keys 电偶极子|电磁学|静电|电荷
% license Xiao
% type Tutor

\pentry{电偶极子\upref{eleDpl}}

电偶极子的定义可以拓展到多个电荷的情况或者连续分布的情况
\begin{equation}
\bvec p = \sum_i \bvec r_i q_i~,
\end{equation}
\begin{equation}\label{eq_eleDP2_2}
\bvec p = \int \bvec r \rho(\bvec r) \dd{V}~.
\end{equation}

注意只有被求和或者积分的所有电荷之和为零, 偶极子 $\bvec p$ 才不随参考系改变
\begin{equation}
\sum_i (\bvec r_i + \bvec d) q_i = \sum_i \bvec r_i q_i + \bvec d \sum_i q_i~.
\end{equation}
若电荷之和不为零, 我们可以定义一个和质心性质类似的中心
\begin{equation}
\bvec r_0 = \frac{\sum_i \bvec r_i q_i}{\sum_i q_i}~.
\end{equation}
可以证明这个位置和参考系无关。
\begin{equation}
\frac{\sum_i (\bvec r_i + \bvec d) q_i}{\sum_i q_i} = \bvec r_0 + \bvec d~.
\end{equation}
如果以 $\bvec r_0$ 为原点, 偶极子为零。

那多级展开到底应该关于哪一点进行呢? 笔者认为最好的选择是(想像一个巨大的正电荷左右分别有两个等大反号的小电荷, 中心当然应该是在大电荷上)
\begin{equation}
\bvec r_0 = \frac{\sum_i \bvec r_i \abs{q_i}}{\sum_i \abs{q_i}}~,
\end{equation}
这个位置同样与坐标系选取无关。

\subsection{匀强电场中电偶极子的势能}
\begin{equation}\label{eq_eleDP2_1}
E = -\bvec p \vdot \bvec E~.
\end{equation}
\begin{figure}[ht]
\centering
\includegraphics[width=10cm]{./figures/fc37e7bcd8c35cd0.pdf}
\caption{匀强电场中电偶极子的势能示意图,角度即为$\bvec p$与$\bvec E$的夹角} \label{fig_eleDP2_1}
\end{figure}
\addTODO{推导}
如\autoref{fig_eleDP2_1} 所示,我们选定夹角为$90^\circ$时为势能零点。当角度增大、电偶极子逆时针旋转时,需要外力做功来克服电场力(的力偶)\footnote{如果你对力偶不太熟悉,可以这么想:逆时针旋转过程中,正电荷朝着电场反方向运动,因此电场力阻碍正电荷运动,因此需要外力以克服电场力},因此势能升高;反之,当角度增减小、电偶极子顺时针旋转时,电场力做功,势能降低。
