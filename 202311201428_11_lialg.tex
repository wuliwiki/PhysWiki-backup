% 合同变换
% license Xiao
% type Tutor


\begin{issues}
\issueDraft
\end{issues}

给定域$\mathbb F$上的线性空间$V$,$f$为对称双线性函数:$V\times V\to \mathbb F$,也就是二次型$q(v),v\in V$。再给定$V$上的一组基$\{\boldsymbol e_i\}$,则二次型可以表示为:
\begin{equation}
\eta_{ij}=f(\boldsymbol {\mathrm e}_i,\boldsymbol {\mathrm e}_j)~,
\end{equation}
对于$V$的另一组基$\{\boldsymbol \theta_i\}$,二次型可以表示为$\Theta_{ij}=f(\boldsymbol {\mathrm \theta}_{i},\boldsymbol {\mathrm \theta}_{j})$。设过渡矩阵为$T^i_j,\boldsymbol {\theta}_i=T^j_i\boldsymbol e_j$。那么我们可以看到二次型在不同基下的表示是如何通过过渡矩阵联系在一起的:
\begin{equation}
\begin{aligned}
\Theta_{ij}&=f(\boldsymbol {\mathrm \theta}_{i},\boldsymbol {\mathrm \theta}_{j})\\
&=f(T^r_i\boldsymbol e_r,T^s_j\boldsymbol e_s)\\
&=T^r_i \eta_{rs}T^s_j~,
\end{aligned}
\end{equation}
把上式两端的二次型张量写作(1,1)型矩阵,则有:
$$\Theta^i_j=T^i_r\eta^r_sT^s_j~,$$
写成更清楚的矩阵形式,就是我们的合同变换:$\Theta=T^T\eta T$。可见,合同变换的本质是改变二次型的基。
\begin{definition}{矩阵合同变换}
设A,B为n阶矩阵,若存在可逆矩阵C,使得
\begin{equation}
C^TAC=B~,
\end{equation}
则称矩阵A与B合同,记作$A\simeq B$
\end{definition}
\begin{exercise}{}
证明合同关系是等价关系。即满足
\begin{enumerate}
\item 反身性  $A\simeq A$
\item 对称性  若$A\simeq B$,则$B\simeq A$
\item 传递性  若$A\simeq B,\quad B\simeq C$,则$A\simeq C$
\end{enumerate}
\end{exercise}
研究任何变换下的不变量是很重要的。相对于相似变换,合同变换的不变量很少,只有秩与惯性指数。
由于矩阵C是可逆的,则左乘的逆矩阵和右乘的矩阵C都可以写作一系列初等矩阵的乘积,所以合同变换只是进行了若干个初等列变换与行变换,因而秩不变。惯性指数不变在惯性定理的证明中得以体现。

在合同变换中,有一条很重要的定理:\textbf{实对称矩阵一定能通过合同变换化为对角矩阵}。也即:实数域上的二次型总有标准形。证明思路如下:
考虑矩阵$A^i_j$,现在要消掉第一行第二列的元素b。一共有三种情况:
\begin{enumerate}
\item b=0,不需要消除
\item b≠0,第一行第一列元素a≠0,C为执行ka+b=0的初等列变换。合同变换中配对的行变换把其对称元素,即第二行第一列的元素也化为0.
\item b≠0,a=0,那么第一步合同变换为把第二行的元素加到第一行,把第二列的元素加到第一列,使得a≠0,再执行上一步变换。
\end{enumerate}
以第一个对角元为“参考”,用该思路可以消除第一行和第一列的非对角元元素。其他非对角元素的消除同理。

如果只是对角化,那么对角矩阵有很多可能。然而,有一类合同变换十分特殊,是用正交矩阵对A进行合同变换。因此,这也是相似对角化的过程。
\begin{theorem}{}
实对称矩阵A总可以用正交矩阵进行合同变换,得到对角矩阵$diag(\lambda_1,\lambda_2...\lambda_n)$,各对角元为A的特征值
\end{theorem}  

proof
首先,我们证明n阶实对称矩阵A必有n个实特征值。
我们知道,求特征值的过程是在解特征多项式$|A-\lambda E|$。根据代数学基本定理,n次方程在复数域上必有n个根。因此只需要证明实对称矩阵的特征值皆为实数即可。
设$\boldsymbol x$为任意一个特征向量,$\lambda$为其对应的特征值,那么我们有
\begin{equation}
\begin{aligned}
A\boldsymbol x&=\lambda \boldsymbol x\\
(A\boldsymbol x)^{*T}&=x^{*T}A=\lambda^{*}x^{*T}\\
x^{*T}Ax&=\lambda x^{*T}x=\lambda^{*}x^{*T}x~,
\end{aligned}
\end{equation}