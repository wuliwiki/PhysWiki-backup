% 氢原子电离计算(一阶微扰)
% 氢原子|偶极子|跃迁|微扰

\begin{issues}
\issueDraft
\end{issues}

\pentry{跃迁概率(含时微扰)\upref{HionCr}}

本文使用原子单位制\upref{AU}.

\subsection{长度规范}
归一化的平面波和归一化的氢原子基态为
\begin{equation}
\ket{\bvec k} = (2\pi)^{-3/2} \E^{\I \bvec k \vdot \bvec r}
\qquad \ket{0} = \frac{1}{\sqrt{\pi}} \E^{-r}
\end{equation}
长度规范下的跃迁偶极子, 可以在极坐标系中积分
\begin{equation}
\mel{\bvec k}{\bvec r}{0}
=  \frac{\uvec k}{(2\pi)^{3/2}\sqrt{\pi}} \int_0^{+\infty} \int_0^\pi \E^{-r} \E^{-\I k r \cos\theta} r \cos\theta \cdot 2\pi r^2 \sin\theta \dd{\theta} \dd{r}
\end{equation}
换元, 令 $u = \cos\theta$, 得\footnote{最后一步可通过 Wolfram Alpha 获得}
\begin{equation}\label{HyIon2_eq1}\ali{% 已检查多次
\mel{\bvec k}{\bvec r}{0} &= \frac{1}{\sqrt 2 \pi} \uvec k \int_0^{+\infty} r^3 \E^{-r} \int_{-1}^1 \E^{-\I k r u} u  \dd{u} \cdot \dd{r}\\
&=  \I\frac{\sqrt2 \uvec k}{k\pi}  \int_0^{+\infty} r^2 \E^{-r} \qty[\cos(kr) - \frac{1}{kr}\sin(kr)] \dd{r}\\
&= -\I \frac{8 \sqrt2}{\pi} \frac{\bvec k}{(k^2+1)^3}
}\end{equation}
\addTODO{代入未完成}

\subsection{速度规范}
先看积分
% Merzbacher 19.87 上面一条
\begin{equation}
\int \exp(-\I \bvec k \vdot \bvec r) \exp(-r) \dd[3]{r} = \frac{8\pi }{(k^2 + 1)^2}
\end{equation}
使用算符 $\grad$ 的反厄米性得
\begin{equation}
\begin{aligned}
&\int \exp(-\I \bvec k \vdot \bvec r) \grad \exp(-r) \dd[3]{r}
= -\int [\grad \exp(\I \bvec k \vdot \bvec r)]^* \exp(-r) \dd[3]{r}\\
&= \I \bvec k \int \exp(-\I \bvec k \vdot \bvec r) \exp(-r) \dd[3]{r}
= \I \frac{8 \pi  \bvec k}{(k^2 + 1)^2}
\end{aligned}
\end{equation}
即
\begin{equation}
\mel{\bvec k}{\bvec p}{0}
=  \frac{8 \pi  \bvec k}{(k^2 + 1)^2} = \frac{1}{(2\pi)^{3/2}\sqrt{\pi}}
\end{equation}


如果 $\ket{\bvec k}$ 是库仑函数(能量本征态)应该有(\autoref{DipEle_eq3}~\upref{DipEle})
\begin{equation}
\mel{\bvec k}{\bvec r}{0} = -\frac{\mel{\bvec k}{\grad}{0}}{m\omega_{ij}}
\end{equation}
但实际上
\begin{equation}
\mel{\bvec k}{\bvec r}{0} = -2\frac{\mel{\bvec k}{\grad}{0}}{m\omega_{ij}}
\end{equation}
这说明在使用平面波近似库伦函数时, 长度规范的 transition amplitude 恰好是速度规范的两倍(待验证).

代入\autoref{SIcros_eq8}~\upref{SIcros}($q = -1$)得微分截面为
\begin{equation}
% dipole approximation 下与 Merzbacher 19.87 相同, 但他的据说与实验吻合, 相差应该不大.
\pdv{\sigma}{\Omega} = \frac{32 k}{mc\omega} \frac{(\bvec k \vdot \uvec e)^2}{(k^2 + 1)^4}
\end{equation}
