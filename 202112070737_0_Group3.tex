% 群作用
% keys 群作用|同态|凯莱定理|Cayley定理|群表示
%已完成
\pentry{群的同态与同构\upref{Group2}}

\subsection{群在自身上的作用}

给定一个群 $G$,我们任意拿出一个元素 $a\in G$,用 $a$ 去左乘 $G$ 中的所有元素(包括 $a$ 自己),那么我们可以把 $a$ 看成一种 $G$ 到自身上的映射:$f_a:G\rightarrow G$,使得对于任意 $x\in G$,$f_a(x)=ax$.

群 $G$ 中的每一个元素都可以像这样生成一个映射,把这些映射全部放在一起,我们也可以整体上看成一个映射 $f:G\times G\rightarrow G$,满足:对于任意的 $a, x\in G$,有 $f(a,x)=f_a(x)=ax$.这样的 $G\times G$ 到 $G$ 上的映射,被称为一个\textbf{群作用(group action)}.

按照上述定义得到的作用,通常称为\textbf{左平移}作用;相应地,我们也可以让元素 $a$ 对 $x$ 的作用是 $g_a(x)=xa$;这样的作用被称为\textbf{右平移}.

\subsection{群作用}

更一般地,对于任何集合 $X$,群 $G$ 中每个元素都可以代表 $X\rightarrow X$ 的一个映射.我们当然可以任意规定这些映射,但如果这些映射满足一定条件的话,就会构造出一个很有意思的结构:

\begin{definition}{群作用}
设群 $G$ 和集合 $X$,$G$ 中每个元素都是 $X$ 到自身的映射,记 $g\in G$ 将 $x\in X$ 映射为 $g\cdot x\in X$.如果所有这些映射满足满足下面两条公理:
\begin{itemize}
\item \textbf{结合律}:对于 $g_1, g_2\in G, x\in X$,$(g_1 g_2)\cdot x=g_1\cdot (g_2\cdot x)$.
\item \textbf{单位元是恒等映射}:$G$ 的单位元 $e$ 将任何 $x\in X$ 映射到自身:$e\cdot x=x$.
\end{itemize}

那么我们称群 $G$ \textbf{作用(acts)}于集合 $X$ 上.

\end{definition}

群作用可以像定义里一样记为 $g\cdot x$,也可以记为 $X$ 到 $X$ 的若干映射 $f_g(x)=g\cdot x$,还可以整体上看成 $G\times X$ 到 $X$ 的一个映射 $f(g, x)=g\cdot x$.

\subsection{群作用的例子}

\begin{example}{平移作用}\label{Group3_ex1}
本条开头定义的左平移和右平移,就是群在自身集合上的作用.
\end{example}

由群运算的唯一性(消去律),平移作用是群在自身上的(集合意义上的)双射.因此每个平移作用都可以看成一个\textbf{置换.}这么一来,我们还得到一个重要的性质:

\begin{theorem}{Cayley定理}\label{Group3_the3}
任何群$G$都同构于其自身的置换群$S_G$的一个子群.
\end{theorem}

\autoref{Group3_the3} 是群表示论的基础之一.

\begin{example}{伴随作用}\label{Group3_ex2}
对于任意 $a\in G$,令 $f_a: G\rightarrow G$ 满足 $\forall x\in G, f_a(x)=axa^{-1}$,则这些映射定义了一个群 $G$ 在集合 $G$ 上的作用,称为\textbf{伴随作用}.$f_a(x)$称为$x$的\textbf{共轭元素}.

在\textbf{群的同态与同构}\upref{Group2}词条中我们知道,全体伴随作用构成群的内自同构群,也称共轭自同构群.
\end{example}

\begin{example}{线性变换}\label{Group3_ex3}
参考线性变换\upref{LTrans}.我们已经知道,$n$ 阶非奇异矩阵配上乘法可以构成一个群;相应地,满秩线性变换(可逆线性变换)配上映射的复合运算构成一个群.非奇异矩阵乘法是给定了基向量以后,满秩线性变换的复合的表示.

取 $n$ 维实数向量空间 $X$,那么 $X$ 是向量的集合;$GL(n,\mathbb{R})$ 是 $X$ 上可逆线性变换的群,显然 $GL(n,\mathbb{R})$ 按照通常的线性变换定义,构成了在 $X$ 上的一个作用.
\end{example}

\subsection{群作用的性质}

当我们讨论群 $G$ 在集合 $X$ 上的作用时,一共有两个集合要关心.

如果我们固定 $X$ 中的一个元素 $x$,那么每个 $G$ 中元素 $g$ 都把 $x$ 映射到某个 $f_g(x)\in X$ 上.所有能这样被映射到的元素 $f_g(x)$ 构成了 $X$ 的一个子集,称为元素 $x$ 的\textbf{轨道(orbit)}.我们也可以找出所有不移动 $x$ 的群元素 $g$,即满足 $f_g(x)=x$ 的 $g\in G$,那么所有这样的 $g$ 构成了 $G$ 的一个子群 $F_x$,称为 $x$ 的\textbf{迷向子群(isotropy subgroup)}或者\textbf{稳定子群(stablizer subgroup)}.

\begin{exercise}{迷向子群}\label{Group3_exe1}
证明 $F_x$ 构成群(\autoref{GroupP_ex2}~\upref{GroupP}).
\end{exercise}

如果对于任何 $x\in X$,$x$ 的轨道都是整个 $X$,那么我们称这个作用是\textbf{可递的},此时 $X$ 就是 $G$ 的\textbf{齐性空间}.如果对于任何 $x\in X$,$x$ 的轨道只是 $\{x\}$,那么这个作用就是\textbf{平凡(trivial)}的.

如果对于任何 $x\in X$,任何 $g\in G-\{e\}$,都使得 $g\cdot x\not=x$,那么我们说这个作用是\textbf{有效的}.有效性等价于说任何 $x\in X$ 的迷向子群都是 $\{e\}$.

在之前的例子中,平移作用既是可递的,又是有效的.但是伴随作用不能保证有效性和可递性,具体情况要看群的结构性质.全体可逆线性变换构成的群作用在非零向量空间上,这个作用是可递的,也是有效的——注意一定得是非零向量空间,把零向量排除在外.

\begin{theorem}{}\label{Group3_the1}
设群 $G$ 作用在 $X$ 上,在 $X$ 上定义关系 $\sim$ 如下:$\forall x, y\in X, x\sim y \iff \exists g\in G, g\cdot x=y$,或者说,$x\sim y$ 当且仅当 $y$ 在 $x$ 的轨道里.那么,$\sim$ 是一个等价关系.
\end{theorem}

由群 $G$ 的封闭性和逆元存在性分别可以证明\autoref{Group3_the1} 中关系 $\sim$ 的传递性和对称性.这个定理说明,轨道划分是一种等价类划分.

\begin{definition}{}
设群 $G$ 按照伴随作用,作用在自身上.
\begin{itemize}
\item 对于 $g\in G$,记 $C_g$ 为 $g$ 在伴随作用下的轨道,称 $C_g$ 为 $g$ 的\textbf{共轭类(conjugate class)},每一个 $h\in C_g$ 都称为 $g$ 的\textbf{共轭元素(conjugate)}.

\item 对于 $g\in G$,记 $C_G(g)$ 是 $g$ 在伴随作用下的迷向子群, 称 $C_G(g)$ 为 $g$ 在 $G$ 中的\textbf{中心化子(centralizer)}.

\item 记 $C(G)=\bigcap_{g\in G} C_G(g)$,称为群 $G$ 的\textbf{中心(center)}.
\end{itemize}
\end{definition}

群的中心还有另一个定义:所有可以和 $G$ 中一切元素交换的元素构成的集合,就是 $C(G)$.

\begin{theorem}{轨道-迷向子群定理}\label{Group3_the2}
设群 $G$ 作用在集合 $X$ 上.固定一个 $x\in X$,那么对于 $g, h\in G$,我们有 $g\cdot x=h\cdot x\iff g^{-1}h\cdot x=x\iff g^{-1}h\in F_x\iff h\in gF_x$.也就是说,如果两个群元素 $g, h$ 能够把 $x$ 映射到同一个元素上,那么 $g$ 和 $h$ 在迷向子群 $F_x$ 的同一个左陪集上.
\end{theorem}

\begin{corollary}{}\label{Group3_cor1}
由\autoref{Group3_the2},$|G|/|F_x|=|O_x|$,其中 $O_x$ 是 $x$ 的轨道.
\end{corollary}

\begin{exercise}{Burnside引理}\label{Group3_exe2}
设群 $G$ 作用在集合 $X$ 上.对于给定的 $g\in G$,记 $X^g=\{x\in X|g\cdot x=x\}$,$O_x$ 是 $x\in X$ 的轨道,那么 $|\{O_x|x\in X\}|=\frac{1}{|G|}\sum_{g\in G}|X^g|$.就是说,$X$ 上轨道的数目,等于每个 $g\in G$ 作用后不产生效果的元素数量之平均值.证明此引理(见\autoref{GroupP_ex3}~\upref{GroupP}).
\end{exercise}
