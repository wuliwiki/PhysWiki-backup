% 无线通讯、有线通讯笔记
% license Usr
% type Note

调幅(AM):调幅通过改变无线电波的幅度(即信号的强度或高度),来携带基带信号(例如音频信号)。在调幅中,载波的频率保持不变,而是其幅度根据基带信号的变化而变化。

调频(FM):调频通过改变载波的频率来传输基带信号,而载波的幅度保持不变。基带信号的不同值会导致载波频率的不同程度变化,而这种频率的变化被接收器解码以重建原始信号。

天线的震荡电路只能接受一个频率,那调频信号怎么接收呢?

天线和它的震荡电路确实会对特定的频率范围有更高的灵敏度,这是由天线的物理尺寸和设计的电路参数决定的。这种现象被称为共振,天线在其共振频率上效率最高。然而,天线和它的接收电路通常设计得足夫灵敏,可以接收一个较宽的频率范围,而不是仅限于单一频率。这就是天线能够接收调频(FM)信号的原因之一。
