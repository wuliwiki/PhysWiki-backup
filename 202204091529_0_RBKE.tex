% 刚体的动能、动能定理
% 刚体|动能定理|平动动能|转动动能

\begin{issues}
\issueDraft
\end{issues}

\pentry{刚体的平面运动方程\upref{RBEM}, 柯尼希定理\upref{Konig}}

如果我们把刚体\upref{RigBd}看作质点系, 那么系统中任意两个质点间距离保持不变, 我们可以假设这些质点之间以不可伸长的轻杆相连, 使它们不能相对运动.

\subsection{平动动能与转动动能}
\pentry{柯尼希定理\upref{Konig}}

刚体的转动动能等于 $1/2$ 瞬时转轴的角速度平方乘以关于该轴的转动惯量\upref{RigEng}. 平动动能等于质心的等效动能.

由柯尼希定理, 刚体的动能可以分为\textbf{平动动能}和\textbf{转动动能}两部分. 下面来证明一段时间内, 平动动能的增加等于合外力关于质心位移的做功, 而转动动能的增加等于质心系中合外力矩的做功.
(未完成!)

\subsection{力矩的功率}
先考虑刚体的定点转动, 若瞬时角速度为 $\bvec \omega$, 总力矩为 $\bvec \tau$, 则功率为
\begin{equation}
P = \bvec \tau\vdot \bvec \omega
\end{equation}
推导: 若把刚体看作质点系, 有
\begin{equation}
\begin{aligned}
P &= \sum_i \bvec v_i \vdot \bvec F_i
= \sum_i (\bvec \omega \cross \bvec r_i) \vdot \bvec F_i
= \sum_i (\bvec r_i\cross\bvec F_i)\vdot \bvec \omega\\
&= \sum_i \bvec \tau_i \vdot \bvec \omega = \bvec \tau\vdot \bvec \omega
\end{aligned}
\end{equation}
\addTODO{不够详细, 引用相关公式}

% 图未完成: 不能有力矩, 不能有合力, 只能等大反向且共线.

% 未完成: 哪里讨论一下刚体运动的能量如何拆分成平动动能和转动动能两个部分?
% \begin{example}{}
% 我们还可以从能量的角度来分析\autoref{RBEM_ex1}, 
% \end{example}

\subsection{动能公式}
刚体绕某点转动的动能为
\begin{equation}\label{RBKE_eq1}
T = \frac{1}{2}\bvec \omega \vdot \bvec L = \frac{1}{2}\bvec \omega \mat I \bvec\omega
\end{equation}
证明: 若把刚体看作质点系, 有
\begin{equation}
\frac{1}{2}\bvec \omega \vdot \bvec L = \frac{1}{2} \sum_i m_i (\bvec r_i\cross \bvec v_i)\vdot \bvec \omega = \frac{1}{2} \sum_i m_i (\bvec \omega \cross \bvec r_i)\vdot \bvec v_i = \frac{1}{2} \sum_i m_i \bvec v_i^2 = T
\end{equation}
证毕.
\addTODO{不够详细, 引用相关公式}
