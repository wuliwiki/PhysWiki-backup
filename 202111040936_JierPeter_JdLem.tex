% Jordan 引理
% 复变函数|围道积分|约尔当引理|若尔当引理|若当引理|Jordan's lemma

\pentry{留数定理\upref{ResThe}}



Jordan 引理可以结合\textbf{留数定理}\upref{ResThe},利用围道积分来处理一些复杂的实函数定积分.

\subsection{Jordan引理的表述与证明}

\begin{theorem}{Jordan 引理}
如果$g(z)$是一个连续的\footnote{更一般地,只要要求存在一个半径$r$,使得$g$在“复平面的上半平面”和“以$r$为半径的圆弧之外的区域”的交集里连续,即可.}函数,且总有$\lim\limits_{\abs{z}\to\infty}g(z)=0$,那么对于任何正实数$a$就有
\begin{equation}
\lim\limits_{R\to \infty}\int_{C(R)}g(z)\E^{az\I}\dd z=0
\end{equation}
其中$C(R)$是半径为$R$的半圆弧路径,圆心为原点,坐落于复平面上半平面,路径方向顺逆时针都一样.


\end{theorem}

\textbf{证明}:

$C(R)$上从$\theta_1$到$\theta_2$的一段圆弧上的积分为:

\begin{equation}
\begin{aligned}
\int_{C(R)|_{\theta_1}^{\theta^2}}g(z)\E^{az\I}\dd z&=\int_{\theta_1}^{\theta_2}g( R\E^{\theta\I} )\E^{aR\E^{\theta\I}\I}\frac{\dd R\E^{\theta\I}}{\dd \theta}\dd\theta\\
&=R\I\int_{\theta_1}^{\theta_2}g(R\E^{\theta\I})\E^{(aR\E^{\theta\I}+\theta)\I}\dd\theta\\
&=R\I\int_{\theta_1}^{\theta_2}g(R\E^{\theta\I})\E^{(aR\cos\theta+\theta)\I-aR\sin\theta}\dd\theta
\end{aligned}
\end{equation}

考虑复变函数的柯西不等式:$\abs{\int_{C(R)}f(z)\dd z} \leq \int_{C(R)}\abs{f(z)}\dd z$,可知

\begin{equation}\label{JdLem_eq4}
\begin{aligned}
&\abs{R\I\int_{\theta_1}^{\theta_2}g(R\E^{\theta\I})\E^{(aR\cos\theta+\theta)\I-aR\sin\theta}\dd\theta}\\
=&R\abs{\int_{\theta_1}^{\theta_2}g(R\E^{\theta\I})\E^{(aR\cos\theta+\theta)\I-aR\sin\theta}\dd\theta}\\
\leq&R\int_{\theta_1}^{\theta_2}\abs{g(R\E^{\theta\I})\E^{(aR\cos\theta+\theta)\I-aR\sin\theta}}\dd\theta\\
=&R\int_{\theta_1}^{\theta_2}\abs{g(R\E^{\theta\I})}\E^{-aR\sin\theta}\dd\theta
\end{aligned}
\end{equation}

由于$\lim\limits_{\abs{z}\to\infty}g(z)=0$,对于任何$\epsilon>0$,总存在$R_\epsilon$使得只要$R>R_\epsilon$就有$\abs{g(R\E^{\theta\I}))}<\epsilon$,即此时有
\begin{equation}\label{JdLem_eq3}
R\int_{\theta_1}^{\theta_2}\abs{g(R\E^{\theta\I})}\E^{-aR\sin\theta}\dd\theta  <  R\epsilon\int_{\theta_1}^{\theta_2}\E^{-aR\sin\theta}\dd\theta
\end{equation}

我们希望对整个$C(R)$积分,也就是$\theta_1=0$和$\theta_2=\pi$.但是由于$\E^{-aR\sin\theta}$的对称性,我们只需要计算其中一半的周期:
\begin{equation}\label{JdLem_eq1}
\int_{0}^{\pi}\E^{-aR\sin\theta}\dd\theta=2\int^{\pi/2}_{0}\E^{-aR\sin\theta}\dd\theta
\end{equation}

又由于在$[0, \pi/2]$上,$\sin\theta>\frac{2\theta}{\pi}$,故
\begin{equation}\label{JdLem_eq2}
\int^{\pi/2}_{0}\E^{-aR\sin\theta}\dd\theta<\int^{\pi/2}_{0}\E^{-aR\frac{2\theta}{\pi}}\dd\theta=\frac{\pi}{2aR}(1-\E^{-aR})
\end{equation}

将\autoref{JdLem_eq1} 和\autoref{JdLem_eq2} 代入\autoref{JdLem_eq3} ,再代回\autoref{JdLem_eq4} ,即得

\begin{equation}\label{JdLem_eq5}
\begin{aligned}
&\abs{R\I\int_{\theta_1}^{\theta_2}g(R\E^{\theta\I})\E^{(aR\cos\theta+\theta)\I-aR\sin\theta}\dd\theta}\\
<&\epsilon(1-\E^{-aR})\frac{\pi}{a}\\
<&\frac{\pi}{a}\epsilon
\end{aligned}
\end{equation}

由于$\epsilon$可以任意小,都有足够大的$R$能满足\autoref{JdLem_eq5} ,且绝对值恒为非负数,因此
\begin{equation}
\abs{\lim\limits_{R\to \infty}\int_{C(R)}g(z)\E^{az\I}\dd z}=0
\end{equation}

于是有

\begin{equation}
\lim\limits_{R\to \infty}\int_{C(R)}g(z)\E^{az\I}\dd z=0
\end{equation}


\textbf{证毕}.


\subsection{应用实例}

\begin{example}{}\label{JdLem_ex1}

定义$\sinc$函数\upref{sinc}为
\begin{equation}
\sinc x = 
\leftgroup{
&\frac{\sin x}{x} &\quad & (x \ne 0)\\
&\quad 1 && (x = 0)
}\end{equation}

我们来求$\int_{-\infty}^{\infty}\sinc x\dd x$.

将问题改写为复函数上,对任意$z\in\mathbb{R}$有
\begin{equation}\label{JdLem_eq6}
\sinc z = \Im \frac{\E^{\I z}}{z} \quad  (z \ne 0)
\end{equation}

然后我们考虑$\frac{\E^{\I z}}{z}$在复平面上的围道积分,来解决\autoref{JdLem_eq6} 在实数轴上的积分问题.

和实数轴上的$\sinc x$不同,复平面上的$\frac{\E^{\I z}}{z}$在$z=0$处有一个奇点,而围道上不应该有奇点.


% 这是因为$C(R)$和实数轴上从$-R$到$R$的区间构成了闭合曲线.

\end{example}





























