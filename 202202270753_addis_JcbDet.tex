% 雅可比矩阵和行列式
% 多元微积分|坐标系变化|全微分|混合积|矩阵|雅可比矩阵|行列式|雅可比行列式

\pentry{行列式与体积\upref{DetVol}, 全微分\upref{TDiff}, 三矢量的混合积\upref{TriVM}}

三维直角坐标系 $x,y,z$ 中, 若有坐标系变换
\begin{equation}
\begin{cases}
u = u(x,y,z)\\ v = v(x,y,z)\\ w = w(x,y,z)
\end{cases}
\end{equation}
根据全微分关系 %(链接未完成)
\begin{equation}
\pmat{\dd{u}\\ \dd{v}\\ \dd{w}} =
\pmat{
\pdv*{u}{x} & \pdv*{u}{y}  & \pdv*{u}{z} \\ 
\pdv*{v}{x} & \pdv*{v}{y} & \pdv*{v}{z} \\ 
\pdv*{w}{x} & \pdv*{w}{y} & \pdv*{w}{z} }
\pmat{\dd{x}\\ \dd{y}\\ \dd{z}}
\end{equation}
这里的矩阵记为 $\mat J$ 叫做\textbf{雅可比矩阵(Jacobian matrix)}. 

考虑 $x,y,z$ 坐标系中的一个体积元 $(x,y,z)$-$(x+ \dd{x}, y + \dd{y}, z + \dd{z})$,  一般情况下(不需要是正交曲线坐标系), 体积元为平行六面体, 起点为 $(x,y,z)$  的三条棱对应的矢量分别为
 \begin{equation}
\pmat{\dd{u_1}\\\dd{v_1}\\\dd{w_1}} = 
\mat J\pmat{\dd{x}\\0\\0} = 
\pmat{J_{11}\\J_{21}\\J_{31}} \dd{x}
\end{equation} 
\begin{equation}
\pmat{\dd{u_2}\\\dd{v_2}\\\dd{w_2}} = 
\mat J \pmat{0\\ \dd{y}\\0} = 
\pmat{J_{12}\\J_{22}\\J_{32}} \dd{y}
\end{equation} 
\begin{equation}
\pmat{\dd{u_3}\\\dd{v_3}\\\dd{w_3}} = 
\mat J \pmat{0\\0\\\dd{z}} = 
\pmat{J_{13}\\J_{23}\\J_{33}} \dd{z}
\end{equation} 
由于平行六面体的体积是同一起点三条矢量的混合积\upref{TriVM}, 可以用行列式表示为
\begin{equation}
\dd{V}
= \vmat{
\dd{u_1} & \dd{u_2} & \dd{u_3}\\
\dd{v_1} & \dd{y_2} & \dd{y_3}\\
\dd{z_1} & \dd{z_2} & \dd{z_3}}
= \abs{\mat J} \dd{u}\dd{v}\dd{w}
\end{equation}
其中 $\abs{\mat J}$  叫做\textbf{雅可比行列式(Jacobian determinant)}. 注意这里的体积可能是负值. 例如在二维情况下

