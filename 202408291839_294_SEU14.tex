% 东南大学 2014 年 考研 量子力学
% license Usr
% type Note

\textbf{声明}:“该内容来源于网络公开资料,不保证真实性,如有侵权请联系管理员”

\textbf{1.(15 分)}下列叙述是否正确:\\
(1) 电子自旋态空间是 2 维的\\
(2) 全同费米子系统的波函数具有交换反对称性\\
(3) 氢原子的基态是简并的\\
(4) 宇称不是可观测量\\
(5) 每一个可观测量均可以用一个幺正算符表示。\\

\textbf{2.(15 分)}质量为$m$的粒子作一运动,儿率守恒定理为$\partial \rho/\partial t + \partial j/\partial x = 0$,其中:$\rho(x, t) = |\psi|^2,j(x, t) = -i\hbar/2m \left( \psi^* \partial \psi/\partial x - \psi \partial \psi^*/\partial x \right).$
\begin{enumerate}
\item 若粒子处于定志$\psi = \phi(x) \exp(-iEt/\hbar)$, 试证:$j=c$(与$x,t$无关的常数),.
\item 若自由拉子处于动星本征态$\psi(x, t) = \exp(ipx/\hbar - iEt/\hbar)$, 试证: $j = p/m.$
\end{enumerate}

\textbf{3.(15 分)}试在坐标表象中写出:
\begin{enumerate}
        \item 位置算符 $\hat{x}$ 的本征函数;
        \item 动量算符 $\hat{p}_x$ 的本征函数;
        \item $[\hat{x}, \hat{p}_y,\hat{p}_z]$ 的共同本征函数。
    \end{enumerate}

\textbf{4.(15 分)}一维各向同性谐振子的哈密顿算符为$$\hat{H} = -\frac{\hbar^2}{2m} \left( \frac{\partial^2}{\partial x^2} + \frac{\partial^2}{\partial y^2} \right) + \frac{1}{2} m\omega^2 \left( x^2 + y^2 \right),~$$
试求能量本征值及其简并度。

\textbf{5.(15 分)}假设该体系有两个彼此不对易的守恒量 $\hat{G}$ 和 $\hat{H}$,即
$$\left[\hat{F}, \hat{H}\right] = 0, \quad \left[\hat{G}, \hat{H}\right] = 0, \quad \left[\hat{F}, \hat{G}\right] \neq 0~$$。
试证明该体系至少有一条能级是简并的。

\textbf{6.(15 分)}假设粒子的波函数为
$$\psi(\theta, \phi) = a Y_1(\theta, \phi) + b Y_{20}(\theta, \phi),
(\lvert a \rvert^2 + \lvert b \rvert^2 = 1)~$$。试求:
\begin{enumerate}
    \item $\hat{L}_z$ 的可能值及其平均值;
    \item $\hat{L}^2$ 的可能值及对应的几率。
\end{enumerate}

\textbf{7.(15 分)}质量为$m$的粒子以能量 $E > 0$ 从左入射,碰到势 $V(x) = y\delta(x) \quad (y > 0)$。

\begin{enumerate}
    \item 求入射几率流密度 $j_(i)$,反射几率流密度 $j_(r)$,透射几率流密度 $j_(x)$ 的表达式;
    \item 试证波函数$\psi$满足 $\psi'(0^+) - \psi'(0^-) = \left(2m\gamma/\hbar^2\right) \psi(0)$;
    \item 求透射系数 $t$。
\end{enumerate}

\textbf{8.(15 分)}设体系由 $2$ 个全同粒子组成,每个粒子可能处于 $2$ 个的粒子态 $\psi_1(r)$ 和 $\psi_2(r)$ 中的任意一个。分以下两种情况写出体系可能的波函数:(1)全同玻色子:(2)全同费米子。

\textbf{9.(15 分)}一质量为 $m$,空间位置固定的电子处于沿 $x$ 方向的均匀磁场 $B$ 中,其哈密顿算符(不计空间运动)为
$$H = \left( eB/mc \right) \hat{s}_x,~$$
其中 $\hat{s}$ 为电子自旋角动量算符。已知 $t = 0$ 时电子的自旋态为 $\hat{s}_z$ 的本征态。

本征值为 $h/2\pi$,试求:
\begin{enumerate}
    \item $t > 0$ 时该电子的自旋态。
    \item $t(> 0)$ 时刻 $\hat{s}$ 的平均值。
\end{enumerate}

\textbf{提示:}泡利矩阵为
$$\sigma_x = \begin{pmatrix} 0 & 1 \\ 1 & 0 \end{pmatrix}, \quad \sigma_y = \begin{pmatrix} 0 & -i \\ i & 0 \end{pmatrix}, \quad \sigma_z = \begin{pmatrix} 1 & 0 \\ 0 & -1 \end{pmatrix}.~$$

\textbf{10.(15 分)} 体系未微扰时密度矩阵为 $\hat{\rho}$,微扰后密度矩阵 $\hat{\rho}'(t)$ 为
$$\hat{H}'(t) = \Lambda e^{-iHt/\hbar} \quad (T > 0),~$$
$\hat{H}_0 \lvert n \rangle = E_n \lvert n \rangle$,
$$\langle n' | n \rangle = \delta_{nn'}, \quad \sum_n |n\rangle \langle n| = 1~$$。
已知 $t = -\infty$ 时体系处在 $\hat{H}_0$ 的非简并本征态 $\ket{k}$。则$\lvert \psi(-\infty) \rangle = \lvert k \rangle$  试利用一阶微扰近似计算体系跃迁的几率幅 $C_{nk}(t)$: 
$$C_{nk}^{(1)}(t) = \frac{1}{i\hbar} \int_{-\infty}^{t} dt' \, \mathcal{H}_{nk}'(t') \exp(i\omega_{nk}t'),~$$ 其中 $\hbar\omega_{nk} = E_n - E_k, \ (n \neq k)$。
\begin{enumerate}
        \item 求 $t = 0$ 时刻体系跃迁到本征态 $\ket{n} \ (n \neq k) $ 的几率幅 $C_{nk}^{(1)}(0)$;
        \item 求 $t = 0$ 时刻体系跃迁到本征态 $\ket{n} \ (n \neq k)$ 的几率幅 $C_{nk}^{(1)}(+\infty)$;
        \item 求 $t = +\infty$ 时刻的态 $ \ket{\psi(+\infty)} $(在此小题中取 $T \to \infty$)。
    \end{enumerate}