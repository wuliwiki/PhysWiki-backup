% 盖斯定律与设计路径
% keys 赫斯 赫士 盖斯 Hess 路径 状态量变化

\subsection{盖斯定律}
\begin{theorem}{盖斯定律}
(等压或等容系统中)一个反应,不管是一步完成的还是多步完成的,其热效应总是相同.
\end{theorem}
盖斯定律最早是源于实验观察的经验结论,可以理解为能量守恒在热力学中的另一种体现.

\begin{example}{碳的燃烧}
\begin{figure}[ht]
\centering
\includegraphics[width=8cm]{./figures/Hess_2.png}
\caption{C的两种氧化方式} \label{Hess_fig2}
\end{figure}
一定量的C被氧化为CO2,无论是直接被氧化为CO2,还是先被不完全氧化生成CO、再由CO被氧化为CO2,二者放出的总热量应该相同.
\end{example}

\subsection{盖斯定律的物理内涵}
事实上,盖斯定律有着更为基本、深刻的物理含义.\textbf{根据状态量的特性,我们知道系统状态量的改变与路径无关\upref{StaPro}\upref{statef}}.也就是说,无论C是直接被氧化为CO2,还是先被不完全氧化生成CO、再由CO被氧化为CO2,由于系统的始末状态一致,因此始末状态下系统的焓H相同,因此系统的焓变也相同$$\Delta H=H_2-H_1=Constant$$,又因为系统处于等压状态,$$Q=\Delta H$$,系统的放热自然相同.

\subsection{设计路径}
\begin{figure}[ht]
\centering
\includegraphics[width=12cm]{./figures/Hess_1.png}
\caption{设计路径} \label{Hess_fig1}
\end{figure}

基于盖斯定律的物理内涵,我们可以推广得到“广义盖斯定律”.根据系统状态量的改变与路径无关的特性,我们可以在两个状态间人为设计一条路径,以求解任意过程前后状态量的变化.

设计的路径一般为等压、等容或等温的可逆路径;且可以包括多个步骤,每一步只中改变一个变量.这样方便运用现成结论.

此处的状态量不再局限于焓.只要能找到适合的路径,原则上可以计算任意状态量(U,S,G,H...)的变化.然而,由于过程量(Q,W)的变化与路径有关,一般不可设计路径计算过程量的变化\footnote{除了少数情况}.

\begin{example}{可逆热水器}
\begin{figure}[ht]
\centering
\includegraphics[width=7cm]{./figures/Hess_3.png}
\caption{热水} \label{Hess_fig3}
\end{figure}
使用500K的恒温烤炉加热一杯1mol的水,使其从室温273K加热至沸腾373K,且不考虑相变.求该过程水的熵变.

这个问题粗看无从下手.热力学第二定律阐明的$dS=\delta q/T$只适用于可逆过程.但是,根据系统状态量的改变与路径无关的特性,熵变只和水的chu shi
\end{example}