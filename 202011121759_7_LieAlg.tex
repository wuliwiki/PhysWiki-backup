% 李代数
% keys 李代数
\pentry{域上的代数\upref{AlgFie}, 爱因斯坦求和约定\upref{EinSum}}

\subsection{概念}

李代数是对域上的代数\upref{AlgFie}进行的一种推广.域上的代数是指域上的线性空间配合了一个矢量乘法,使得这个线性空间在矢量乘法下也能构成一个环.李代数也是域上线性空间配合了一个矢量乘法,这个矢量乘法和构成环的乘法几乎一样,但是有一点显著不同:将环乘法的结合律替代为Jacobi结合性.

\begin{definition}{李代数}
给定域$\mathbb{F}$上的一个线性空间$V$.在$V$上定义一个“乘法”运算:对于任意$\bvec{v}_1, \bvec{v}_2\in V$,将它们的运算结果记为$[\bvec{v}_1, \bvec{v}_2]$.称这个结构$(V, [*, *])$为一个\textbf{李代数},如果它满足以下性质:
\begin{itemize}
\item \textbf{封闭性} 对于任意$\bvec{v}_1, \bvec{v}_2\in V$,$[\bvec{v}_1, \bvec{v}_2]\in V$.
\item \textbf{双线性性} 对于任意$k_i, c_i\in \mathbb{F}$和任意$\bvec{v}_i, \bvec{u}_i\in V$,都有$[k_1\bvec{v}_1+k_2\bvec{v}_2, c_1\bvec{u}_1, c_2\bvec{u}_2]=k_1c_1[\bvec{v}_1, \bvec{u}_1]+k_1c_2[\bvec{v}_1, \bvec{u}_2]+k_2c_1[\bvec{v}_2, \bvec{u}_1]+k_2c_2[\bvec{v}_2, \bvec{u}_2]$.
\item \textbf{反对称性} 对于任意$\bvec{v}, \bvec{u}\in V$,有$[\bvec{v}, \bvec{u}]=-[\bvec{u}, \bvec{v}]$.
\item \textbf{Jacobi结合性} 对于任意$\bvec{x}, \bvec{y}, \bvec{z}\in V$,有$[\bvec{x}, ([\bvec{y}, \bvec{z}])]+[\bvec{z}, ([\bvec{x}, \bvec{y}])]+[\bvec{y}, ([\bvec{z}, \bvec{x}])]=0$.
\end{itemize}

由于这个乘法用括号$[*, *]$表示,因此有时也称之为“括积”.
\end{definition}

李代数是对李群的性质进行进一步的抽象,是对李群的一些核心性质的代数刻画.

李代数定义中的双线性性可以通过反对称性和单对称性\footnote{即对于参与运算的双方中某一方线性就可以,比如对后面的线性:$[\bvec{v},a\bvec{x}+b\bvec{y}]=a[\bvec{v}, \bvec{x}]+b[\bvec{v}, \bvec{y}]$.}推出,所以和本书中群的定义一样,是有冗余的定义.如果用爱因斯坦求和约定\upref{EinSum}来表达,双线性性还可以写为$[a_i\bvec{v}_i, b_j\bvec{u}_j]=a_ib_j[\bvec{v}_i, \bvec{u}_j]$.

\begin{definition}{交换李代数}
设$\mathfrak{g}$为域$\mathbb{F}$上的$n$维李代数,那么如果对于任意的$\bvec{x},\bvec{y}\in\mathfrak{g}$,都有$[\bvec{x}, \bvec{y}]=0$,则称这是一个\textbf{交换李代数(commutative Lie algebra)}或\textbf{阿贝尔李代数(abelian Lie algebra)}.
\end{definition}

交换李代数之所以要求括积为零,是因为反对称性和交换性结合必然得到括积为零.

李代数和结合代数的关系极为紧密.事实上,每个结合代数都可以唯一地导出一个李代数,这由以下定理决定:

\begin{theorem}{结合代数导出李代数}
设$\mathfrak{A}$是域$\mathbb{F}$上的一个结合代数,即$\mathfrak{A}$是$\mathbb{F}$上的线性空间,且定义了一个乘法$\times$使得它还成为一个环,那么如果定义$\forall \bvec{x}, \bvec{x}\in\mathfrak{A}, [\bvec{x}, \bvec{x}]=\bvec{x}\times \bvec{x}-\bvec{x}\times \bvec{x}$,则这个括号$[*, *]$配上$\mathfrak{A}$构成一个李代数.
\end{theorem}

\te\bvec{x}tbf{证明}:

封闭性是显然的,因为$\times$是封闭的.

由这个括积的定义,反对称性也是显然的,因为$\bvec{x}\times \bvec{y}-\bvec{y}\times \bvec{x}=-(\bvec{y}\times \bvec{x}-\bvec{x}\times \bvec{y})$恒成立.

由反对称性,我们只需要证明单线性性即可:$\forall \bvec{x}, \bvec{y}, \bvec{z}\in\frak{A}$和$a, b\in\mathbb{F}$,有:

最后是Jacobi结合性:

\begin{equation}\label{LieAlg_eq1}
[\bvec{x}, a\bvec{y}+b\bvec{z}]=a\bvec{x}\times\bvec{y}+b\bvec{x}\times\bvec{z}-a\bvec{y}\times\bvec{x}-b\bvec{z}\times\bvec{x}=a\bvec{x}\bvec{y}+b\bvec{x}\bvec{z}-a\bvec{y}\bvec{x}-b\bvec{z}\bvec{x}
\end{equation}

这里我们省略了乘法符号,以显示紧凑.

由\autoref{LieAlg_eq1} ,轮换$\bvec{x}, \bvec{y}$和$\bvec{z}$的顺序后把得到的三个式子相加,即可以得到Jacobi结合性.

\textbf{证毕}.

知道了每个结合代数可以对应一个李代数后,我们自然会好奇,每个李代数是不是也能对应一个结合代数?答案是肯定的,见\footnote{尚未创作到相关词条}.


\subsection{例子}

结合代数对应李代数这一事实,使得我们很容易联想到一个常见的结合代数:矩阵代数.在本文中,将域$\mathbb{F}$上的$n$阶可逆方阵的集合记为$\opn{gl}(n, \mathbb{F})$,那么这个集合自然构成$\mathbb{F}$上的一个线性空间(以矩阵加法为向量加法),而矩阵乘法则使之构成一个环,因此这是一个结合代数.

\begin{example}{一般线性李代数}
域$\mathbb{F}$上的$\opn{gl}(n, \mathbb{F})$是一个结合代数.由这个结合代数可以导出李代数,其中对于矩阵$\bvec{A}$和$\bvec{B}$,括积的定义为$[\bvec{A}, \bvec{B}]=\bvec{A}\bvec{B}-\bvec{B}\bvec{A}$.该李代数被称为一般线性李代数,名称和一般线性群$\opn{gl}(n, \mathbb{F})$对应.
\end{example}

\begin{example}{特殊线性李代数}
对于域$\mathbb{F}$和正整数$n$,记$\opn{sl}(n, \mathbb{F})$为$\opn{gl}(n, \mathbb{F})$中迹为$0$的矩阵构成的结合代数,则它可以导出一个特殊线性李代数,名称也和特殊线性群$\opn{sl}(n, \mathbb{F})$对应.
\end{example}

\begin{example}{三维向量叉积}
域$\mathbb{R}$上的三维线性空间$\mathbb{R}^3$中,将括积定义为叉积:$\forall \bvec{v}, \bvec{u}\in \mathbb{R}^3$,有$[\bvec{v}, \bvec{u}]=\bvec{v}\times\bvec{u}$.那么这个线性空间配上叉积可以得到一个李代数.
\end{example}

\subsection{结构常数}

李代数的括积的作用是把两个向量映射为一个向量,而且还要求具有双线性性,因此括积实际上是一个三阶张量.如果设$\mathfrak{g}$为域$\mathbb{F}$上的$n$维李代数,而$\{\bvec{x}^i\}$为它的一组基,那么对于任意的$\bvec{x}^i$和$\bvec{x}^j$,存在$C_k^{ij}\in\mathbb{F}$,使得$[\bvec{x}^i, \bvec{x}^j]=C_k^{ij}\bvec{x}^k$.这样的$C_k^{ij}$就是括积张量的三维矩阵表示.

显然,括积张量的三维矩阵表示依赖于基的选取,和所有其它张量一样.当选定基以后,所得到的矩阵称为$\mathfrak{g}$关于基$\{\bvec{x}_i\}$的\textbf{结构常数}.

\begin{theorem}{}\label{LieAlg_the1}
设$\mathfrak{g}$为域$\mathbb{F}$上的$n$维李代数,它在基$\{\bvec{x}^i\}$下的\textbf{结构常数}为$C_k^{ij}$,那么有:
\begin{itemize}
\item $C_k^{ij}=-C^k_{ji}$.
\item $C^l_{ij}C^m_{lk}+C^l_{jk}C^m_{li}+C^l_{ki}C^m_{lj}=0$\footnote{注意使用爱因斯坦求和约定.}.
\end{itemize}
\end{theorem}

定理的证明留给读者,这是直接从李代数的反对称性和Jacobi结合性得来的.

结构常数唯一地对应李代数,也就是说,我们也可以任取一个线性空间,然后通过定义满足\autoref{LieAlg_the1} 的结构常数来定义括积,从而得到一个李代数.

\begin{theorem}{结构常数的变换}
设$\mathfrak{g}$为域$\mathbb{F}$上的$n$维李代数,它在基$\{\bvec{x}^i\}$下的\textbf{结构常数}为$C_k^{ij}$,在基$\{\bvec{y}^i\}$下的\textbf{结构常数}为$D_k^{ij}$,而且有过渡矩阵\upref{TransM}$a^i_j$使得$\bvec{y}^i=a^i_j\bvec{x}^j$,那么有变换法则:
\begin{equation}
D^{ij}_ka^k_j=a^i_ma^j_nC_j^{mn}
\end{equation}

\end{theorem}

\textbf{证明}:

\begin{equation}\label{LieAlg_eq2}
[\bvec{y}^i, \bvec{y}^j]=D^{ij}_k\bvec{y}^k=D^{ij}_ka^k_j\bvec{x}^j
\end{equation}

\begin{equation}\label{LieAlg_eq3}
[\bvec{y}^i, \bvec{y}^j]=[a^i_m\bvec{x}^m, a^j_n\bvec{x}^n]=a^i_ma^j_n[\bvec{x}^m, \bvec{x}^n]=a^i_ma^j_nC_j^{mn}\bvec{x}^j
\end{equation}

比较\autoref{LieAlg_eq2} 和\autoref{LieAlg_eq3} ,消去$\bvec{x}^j$即可得$D^{ij}_ka^k_j=a^i_ma^j_nC_j^{mn}$.
\textbf{证毕}.





