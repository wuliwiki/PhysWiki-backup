% Julia 的数据类型

\begin{issues}
\issueDraft
\end{issues}

\subsection{变量类型}

参考 \href{https://docs.julialang.org/en/v1/manual/types/#man-declared-types}{Julia 文档}.
\begin{itemize}
\item Julia 是动态类型的语言. 具体类型不能互相作为子类, 只能作为抽象类的子类(sub type).
\item Julia 中所有的值都是对象. C++ 中 \verb|float|, \verb|double| 等不是对象, 只有 \verb|struct|, \verb|class| 定义的才是对象. Julia 中的对象\textbf{没有成员函数}.
\item 可以在 literal、 变量、 表达式、 函数定义后面加上 \verb|::变量类型| 用于确认它具有该类型, 如果类型不符会产生异常. 如果 \verb|变量类型| 是抽象的, 那么表达式只能是它的一个子类.
\item 只有 “值” 有类型, 而不是变量具有类型. 变量只是个名字而已
\item 抽象和具体的类都可以使用其他类作为参数
\end{itemize}

常见类型
\verb|Int8|, \verb|UInt8|, \verb|Int16|, \verb|UInt16|, \verb|Int32|, \verb|UInt32|, \verb|Int64|(即 \verb|Int|), \verb|UInt64|, \verb|Int128|, \verb|UInt128|, \verb|Float16|, \verb|Float32|, \verb|Float64|, \verb|ComplexF16|, \verb|ComplexF32| 和 \verb|ComplexF64|

\verb|BigInt| 是任意大小的整数, \verb|BigFloat| 是任意精度浮点数.

\begin{lstlisting}[language=julia]
julia> BigFloat(2.1) # 2.1 here is a Float64
2.100000000000000088817...

julia> BigFloat("2.1") # the closest BigFloat to 2.1
2.099999999...99999986

julia> BigFloat("2.1", RoundUp)
2.100000000...0000021

julia> BigFloat("2.1", precision=128)
2.0999999...995

julia> BigFloat("2.1", RoundUp, precision=128)
2.100000000...00000000000007
\end{lstlisting}

使用 \verb|typeof(var)| 来询问变量类型, 用 \verb|isa(var, type)| 来确定类型. \verb|sizeof(变量或类型)| 来查看类型的字节数

\begin{itemize}
\item \verb|setprecision(二进制精度);| 可以设置运算默认的精度
\item \verb|BigFloat(pi)| 会默认为上面设置的精度, \verb|BigFloat(pi, precision=...)| 可以临时改变精度, 但是 \verb|BigFloat(pi, precision=...)*2| 仍然是上面设置的精度, 因为乘法使用上面设置的精度.
\item \verb|pi| 的加减乘除都得到 \verb|Float64| 类型.
\end{itemize}


\subsubsection{逻辑类型}
\begin{itemize}
\item \verb|Bool| 类型的值可以是 \verb|true| 或者 \verb|false|
\item \verb|any([true, false, ...])| 和 \verb|all([trua, false, ...])|
\end{itemize}

\subsubsection{整数}
\begin{itemize}
\item 整数 literal 的类型取决于它的大小, 例如 \verb|typeof(3)| 是 \verb|Int64|; \verb|typeof(12345678902234567890)| 是 \verb|Int128|, \verb|typeof(1234567890223456789032345678904234567890)| 是 \verb|BigInt|. \verb|typemin(类型)| 和 \verb|typemax(类型)| 可以查看最大最小值.
\item 两个相同类型的 Int 相加时, 如果 overflow 会变成负数. 两个不同类型的 Int 相加时, 较小的类型会先转换为较大的类型. 若 \verb|a = Int8(1)|, \verb|a += 1| 后 \verb|a| 会变为 \verb|Int64|. 因为 \verb|1| 相当于 \verb|Int64(1)|.
\end{itemize}

\subsubsection{浮点数}
\begin{itemize}
\item 任何浮点数 literal 都是 \verb|Float64|.
\item \verb|eps(Float64)| 返回 \verb|2.220446049250313e-16|
\item \verb|NaN| 是 not a number, 可以用 \verb|isnan(x)| 来判断
\end{itemize}

\subsection{无理数}
\begin{itemize}
\item \verb|pi| 的类型是 \verb|Irrational{:π}| 其中 \verb|:π| 的类型是 \verb|Symbol| (又例如 \verb|:abc|, \verb|:a|)注意 \verb|:pi| 和 \verb|:π| 是两个不同的 \verb|Symbol|.
\item \verb|BigFloat(pi, precision=500)| 可以获得 500 位二进制的 pi.
\end{itemize}


\subsection{字符串类型}
\begin{itemize}
\item 任何单个 UTF8 字符都属于 \verb|Char| 类型. \verb|sizeof(Char) = 4|, 格式应该是 UTF32
\item \verb|s = "abcde"| 的类型是 \verb|String|, 注意 \verb|String| 并不是 \verb|Array{Char, 1}|! \verb|s[i]| 可以获取单个 Char, 但是不能赋值, 因为 String 是不可改变的(immutable).
\item 子串替换: \verb|s1 = replace(s, "bc" => "BC")|. \verb|s1| 也可以是 \verb|s| 本身.
\item \verb|s = s1*s2| 拼接字符串(等效地: \verb|s = string(s1, s2)|), \verb|s *= s1| append.
\item \verb|s = s1^3| 重复字符串 3 次, \verb|s ^= 3| 同理.
\item \verb|findfirst("abc", s)| 搜索字符串, 返回一个 \verb|UnitRange{Int64}|, 如 \verb|3:5|.
\item \verb|collect(s)| 把字符串变为 \verb|Vector{Char}|
\end{itemize}


\subsection{抽象类型}
声明抽象类型
\begin{lstlisting}[language=julia]
abstract type 子类名 <: 母类名 end
\end{lstlisting}
声明后, 可以用 \verb|类1 <: 类2| 来判断是否符合该关系, 返回一个 \verb|Bool|.

用 \verb|supertype(类)| 可以查看某个类的母类, \verb|subtypes(类)| 可以查看所有子类.

例子
\begin{lstlisting}[language=julia]
abstract type Number end
abstract type Real     <: Number end
abstract type AbstractFloat <: Real end
abstract type Integer  <: Real end
abstract type Signed   <: Integer end
abstract type Unsigned <: Integer end
\end{lstlisting}
最高级的抽象类是 \verb|Any|, 这里的 \verb|Number| 就是 \verb|Any| 的子类.

\begin{figure}[ht]
\centering
\includegraphics[width=13cm]{./figures/JuType_1.png}
\caption{Julia 自带类型结构} \label{JuType_fig1}
\end{figure}
以及
\begin{lstlisting}[language=julia]
Complex{T<:Real} <: Number
Rational{T<:Integer} <: Real
Irrational{sym} <: AbstractIrrational
\end{lstlisting}


\subsection{原始类型}
\textbf{原始类型(primitive type)}是由 bit 组成的具体类型. Julia 可以自定义原始类型. 定义原始类型的格式为
\begin{lstlisting}[language=julia]
primitive type 类名 比特数 end
primitive type 类名 <: 父类名 比特数 end
\end{lstlisting}
例如
\begin{lstlisting}[language=julia]
primitive type Float16 <: AbstractFloat 16 end
primitive type Float32 <: AbstractFloat 32 end
primitive type Float64 <: AbstractFloat 64 end

primitive type Bool <: Integer 8 end
primitive type Char <: AbstractChar 32 end

primitive type Int8    <: Signed   8 end
primitive type UInt8   <: Unsigned 8 end
primitive type Int16   <: Signed   16 end
primitive type UInt16  <: Unsigned 16 end
primitive type Int32   <: Signed   32 end
primitive type UInt32  <: Unsigned 32 end
primitive type Int64   <: Signed   64 end
primitive type UInt64  <: Unsigned 64 end
primitive type Int128  <: Signed   128 end
primitive type UInt128 <: Unsigned 128 end
\end{lstlisting}

\subsection{复合类型}
\textbf{复合类型(composite)} 类似于 C++ 中的 \verb|struct|
\begin{lstlisting}[language=julia]
struct Foo
    bar # 抽象类型是 Any
    baz::Int
    qux::Float64
end
\end{lstlisting}
创建该类型的对象
\begin{lstlisting}[language=julia]
foo = Foo("Hello, world.", 23, 1.5) # 叫做 constructor
\end{lstlisting}
查看成员名称用 \verb|fieldnames(Foo)|

要获取成员的值, 用 \verb|foo.bar|, \verb|foo.baz|, \verb|foo.qux| 等. 用 \verb|struct| 声明的复合类型值在生成后就不能被改变. 如果 struct 里面有矩阵等, 那么矩阵元仍然是 mutable 的(没有 lower level const).

如果要支持成员改变, 那么用 \verb|mutable struct| 来声明即可. mutable struct 通常存在 heap 中而不是在 stack 中.

\subsection{Type Unions}
\begin{lstlisting}[language=julia]
IntOrString = Union{Int, AbstractString}
1 :: IntOrString
"Hello!" :: IntOrString
1.0 :: IntOrString # 异常
\end{lstlisting}
但是, 这和使用 \verb|<:| 有什么区别? 一个类只可能有一个非 Union 的母类, 如果一个函数的参数想要支持两个母类不同的类怎么办? 那就用 Union 即可.

另外一个应用的例子是 \verb|Union{T, Nothing}| 作为函数参数, 这样这个参数就可以忽略了.

\subsection{含参类型}

复数类型 \verb|ComplexF16|, \verb|ComplexF32| 和 \verb|ComplexF64| 是 \verb|Complex{Float16}|, \verb|Complex{Float32}| 和 \verb|Complex{Float64}| 的别名. \verb|Complex| 就是一个\textbf{含参类型(parametric type)}

例如
\begin{lstlisting}[language=julia]
struct Point{T}
    x::T
    y::T
end
\end{lstlisting}
那么 \verb|Point{Float64}| 等都是合法的具体类型.

\verb|Point| 本身是一个抽象类, 是其具体类的母类
\begin{lstlisting}[language=julia]
julia> Point{Float64} <: Point
true
julia> Point{AbstractString} <: Point
true
\end{lstlisting}
\verb|<:| 相当于一个二元算符, 输出 \verb|Bool|.

例子:
\begin{lstlisting}[language=julia]
function norm(p::Point{Real})
    sqrt(p.x^2 + p.y^2)
end
\end{lstlisting}
constructor
\begin{lstlisting}[language=julia]
p = Point{Float64}(1.0, 2.0)
\end{lstlisting}

\subsection{含参抽象类型}
\textbf{含参抽象类型(Parametric Abstract Types)}
例如
\begin{lstlisting}[language=julia]
abstract type Pointy{T} end
struct Point{T} <: Pointy{T}
    x::T
    y::T
end
\end{lstlisting}

\subsection{Pair}
\begin{itemize}
\item \verb|"abc" => 1.23| 的类型是 \verb|Pair{String, Float64}|, 令 \verb|p = "abc" => 1.23|, 那么 \verb|p.first| 和 \verb|p.second| 分别可以获取两个变量.
\item \verb|Pair| 同样是 immutable 的.
\end{itemize}


\subsection{Tuple}
Tuple 是函数参数列表的抽象. 例如两个元素的 tuple type 类似于如下的含参 struct 类型
\begin{lstlisting}[language=julia]
struct Tuple2{A,B}
    a::A
    b::B
end
\end{lstlisting}
例子
\begin{lstlisting}[language=julia]
julia> typeof((1,"foo",2.5))
Tuple{Int64, String, Float64}
\end{lstlisting}
\begin{lstlisting}[language=julia]
julia> Tuple{Int,AbstractString} <: Tuple{Real,Any}
true

julia> Tuple{Int,AbstractString} <: Tuple{Real,Real}
false
\end{lstlisting}

\begin{itemize}
\item \verb|t = (值1, 值2, 值3, ...)| 就是 \verb|Tuple|
\item 数组的元素类型必须相同, 而 \verb|Tuple| 的元素类型可以不同
\item \verb|t = (1, "3.1", [1 2; 3 4])| 的类型是 \verb|Tuple{Int64, String, Matrix{Int64}}|
\item 数组的 \verb|length()|, \verb|isempty()|, \verb|reverse()| 同样适用于 \verb|Tuple|.
\item \verb|t[i]| 获取某个元素, 但不能对其赋值! Tuple 的元素永远不能改变! Tuple 类型是 immutable 的.
\item named tuple: \verb|x = (a=2, b="abc")|, 类型为 \verb|NamedTuple{(:a, :b), Tuple{Int64, String}}|. 然后可以使用 \verb|x.a| 和 \verb|x.b| 来获取元素.
\item 单个元素的 tuple 如 \verb|(1,)|, 空 tuple 用 \verb|()| 表示, 类型为 \verb|Tuple{}|.
\end{itemize}

\subsection{类似 C++ 的数据结构}
\subsubsection{Set}
\begin{itemize}
\item \verb|s = Set{Int64}()| 或者 \verb|Set{Int64}| 生成某种类型的空 set. \verb|s = Set([1,2,3])| 把数组变为 \verb|Set|
\item \verb|push!(s, 123)| 添加一个元素
\item \verb|empty!(s)| 清空集合.
\item \verb|in(123, s)| 查看某个元素是否存在
\item \verb|union!(s1,s2)| 计算并集, 赋值给 \verb|s1|, \verb|s = union(s1,s2)| 计算并集, 赋值给 \verb|s|.
\end{itemize}

\subsubsection{Dict}
\begin{itemize}
\item \verb|Dict{key类型,Val类型}()| 创建一个字典(哈希表)
\end{itemize}



\subsection{Vararg Tuple Types}

\subsection{Named Tuple Types}

\subsection{Parametric Primitive Types}

\subsection{UnionAll Types}

\subsection{Singleton types}

\subsection{Type{T} type selectors}

\subsection{Type Aliases}
使用等号. 例如
\begin{lstlisting}[language=julia]
# 32-bit system:
julia> UInt
UInt32

# 64-bit system:
julia> UInt
UInt64
\end{lstlisting}
实现方法
\begin{lstlisting}[language=julia]
This is accomplished via the following code in base/boot.jl:

if Int === Int64
    const UInt = UInt64
else
    const UInt = UInt32
end
\end{lstlisting}

\subsection{Operations on Types}
\begin{lstlisting}[language=julia]
julia> isa(1, Int)
true

julia> typeof(Rational{Int})
DataType

julia> typeof(Union{Real,String})
Union

julia> typeof(DataType)
DataType

julia> typeof(Union)
DataType

julia> supertype(Float64)
AbstractFloat

julia> supertype(Number)
Any

julia> supertype(AbstractString)
Any

julia> supertype(Any)
Any
\end{lstlisting}
