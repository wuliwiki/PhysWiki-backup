% 流体运动的描述方法
% keys 欧拉法|拉格朗日法
\subsection{流体与固体的区别}
我们从流体与固体的区别中引入流体运动的描述方法,固体无论处于静止还是运动都可以通过有限的静变形承受剪切力,其形状不易变化,在运动学中,可以只从几何角度来描述物体的位置随时间的变化.

流体在静止时无法通过有限的静变形承受剪切力,在运动状态下虽能产生剪切力,剪切力却不能维持流体内部各质点位置的有序性,其内部各质点在运动中相对位置关系变化可以很大,不能保持规则的几何形状,单从几何角度描述其运动将复杂而困难,需要专门的处理方法.

固体力学中,只着眼于需要描述的运动的物体,物体之外的都叫做环境,物体与环境之间发生力的作用从而改变物体运动状态,这种方法称为拉格朗日法.

流体当然也可以采拉格朗日法,但因为流体几何形状的变化,需要进行复杂的坐标变换.变化的形状难以描述,我们考虑是不是可以研究一个特定不变的空间呢,着眼于流体经过这个空间时发生的变化以及与这个空间的相互作用,是不是也能全面描述流体的运动呢,实践证明这也是可行的,这种方法称为欧拉法.

\subsection{拉格朗日法}

\subsection{欧拉法}