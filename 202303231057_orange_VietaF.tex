% 韦达定理(高等代数)
% 代数方程|根与系数|数论|域|伽罗瓦|伽罗华|维达定理|Vieta's formula

\pentry{一元多项式\upref{OnePol},韦达定理(高中)\upref{VieHi}}

韦达定理描述了多项式根与系数的关系。在中学数学中常见韦达定理最简单的形式,即系数一元二次多项式的根与系数的关系。本词条要讨论的是更一般形式下的韦达定理。

\subsection{定理描述}

韦达定理适用于一般的复系数多项式的根(实系数多项式是它的一个特殊情况),记忆起来也非常方便:

\begin{theorem}{韦达定理}\label{VietaF_the2}
设有复系数多项式 $\sum_{i=0}^n a_i x^i \in \mathbb{C}[x]$ ,据\textbf{代数学基本定理}\upref{BscAlg}知其应有$n$个根(重根按重数记),分别记为$x_1, x_2, \cdots, x_n$。则有:
\begin{equation}\label{VietaF_eq1}
\leftgroup{
    x_1+x_2+\cdots+x_n &= -\frac{a_{n-1}}{a_n}\\
    x_1x_2+x_1x_3+\cdots+x_{n-1}x_n &= \frac{a_{n-2}}{a_n}\\
    &\vdots\\
    x_1x_2\cdots x_n &= (-1)^n\frac{a_0}{a_n}
}
\end{equation}
更一般的,我们有
$$
\sum_{1 \leq k_1 \leq \dots \leq k_i \leq n} x_{k_1} x_{k_2} \cdots x_{k_i} = (-1)^i \frac{a_{n - i}}{a_n}
$$~.

\end{theorem}

\addTODO{Elementary symmetric polynomial}

显然,高中版本的韦达定理只是\autoref{VietaF_the2} 的一个特例:实系数一元二次多项式。

记忆\autoref{VietaF_eq1} 也不难:第$i$个式子就是每$i$个根为一组相乘、所有组相加,结果是$(-1)^i\frac{a_{n-i}}{a_n}$,分母永远是最高次项的系数。





\subsection{定理证明}

按\autoref{VietaF_the2} 题设,知
\begin{equation}
\sum_{i=0}^n a_ix^i = A(x-x_1)(x-x_2)\cdots(x-x_n)
\end{equation}

展开后,比较各项系数得

\begin{equation}\label{VietaF_eq3}
    a_nx^n = Ax^n
\end{equation}
和
\begin{equation}\label{VietaF_eq2}
\leftgroup{
    a_{n-1}x^{n-1} &= A(-x_1-x_2-\cdots-x_n)x^{n-1}\\
    a_{n-2}x^{n-2} &= A(-x_1x_2-x_1x_3-\cdots-x_{n-1}x_n)x^{n-2}\\
    &\vdots\\
    a_{0} &= A(-x_1)(-x_2)\cdots(-x_n)
}
\end{equation}


将\autoref{VietaF_eq3} 代入\autoref{VietaF_eq2} ,整理后即得\autoref{VietaF_eq1} .








