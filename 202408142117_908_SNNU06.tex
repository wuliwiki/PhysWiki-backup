% 陕西师范大学 2006 年 考研 量子力学
% license Usr
% type Note

\textbf{声明}:“该内容来源于网络公开资料,不保证真实性,如有侵权请联系管理员”

\subsection{填空题(本题 10 道题,每小题2分,共 20 分)}
\begin{enumerate}
\item 爱因斯坦提出光量子理论成功解释了光电效应现象,光量子理论的核____________.
\item 德布罗意提出了物质波假说。按这个假说,粒子的能量 $E$、动量 $P$ 与波的波长 $\lambda$ 之间的关系为____________.
\item 微观粒子的状态用波函数描述。若 $\psi(x,y,z,t)$ 与 $\Phi(x,y,z,t)$ 描写同一状态,数的差别是____________.
\item 若微观体系的哈密顿算符 $H$ 不显含时间,则体系的薛定谔方程 
   $$i \hbar \frac{\partial \psi}{\partial t} = -\frac{\hbar^2}{2 \mu} \nabla^2 \psi + U(\vec r) \psi~$$
    的一般解是 $\psi(\mathbf{\vec r},t)$ =____________.
\item 动量算符 $\hat{P} = -i\hbar\nabla$,其本征函数数有两种归一化方式,当本征值取连续谱时,其形式为 ____________.
\item 氢原子处于状态 $\psi_{nlm}(r, \theta, \varphi)$ 时,则在半径 $\vec r$ 到 $\vec r + d\vec r$ 的球壳内找到电 $w_{nl}(\vec r) dr =$____________.
\item 任意态 $\psi(x) = \sum_n c_n \Phi_n(x)$,其中 $\Phi_n(x)$ 是力学量 $\hat{F}$ 的正交归一的本征函数。归一化,则系数 $c_n$ 应满足 $\sum_n |c_n|^2 =$_______.$[\hat{F}, \hat{G}] $=_______.
\item 在非简并态微扰理论中,$\hat{H} = \hat{H}_0 + \hat{H}'$ 该微扰理论运用的条件具体表现为_______.
\end{enumerate}
