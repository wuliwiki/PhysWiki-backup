% 高斯定律(综述)
% license CCBYSA3
% type Wiki

本文根据 CC-BY-SA 协议转载翻译自维基百科\href{https://en.wikipedia.org/wiki/Conservation_of_energy}{相关文章}。

本文讨论的是关于电场的高斯定律。关于其他场的类似定律,请参见磁场的高斯定律和重力的高斯定律。关于与这些定律相关的奥斯特罗格拉德斯基-高斯定理,请参见散度定理。  
请勿将其与高斯法则(Gause's law)混淆。

在物理学中(特别是电磁学),高斯定律(又称高斯通量定理,或有时称为高斯定理)是麦克斯韦方程组之一。它是散度定理的一个应用,将电荷分布与由此产生的电场联系起来。
\subsection{定义}
\begin{figure}[ht]
\centering
\includegraphics[width=8cm]{./figures/95db2b4380930eda.png}
\caption{当由于对称性原因可以找到一个电场沿其均匀的闭合曲面(GS)时,积分形式的高斯定律尤其有用。在这种情况下,电通量是表面积和电场强度的简单乘积,并且与曲面所包围的总电荷成正比。这里正在计算带电球体外部(\( r > R \))和内部(\( r < R \))的电场(见维基学院)。} \label{fig_GSDL_1}
\end{figure}
在积分形式下,高斯定律表述为:电场通过任意封闭曲面的通量与该曲面所包围的电荷成正比,而不论电荷如何分布。尽管仅凭此定律不足以确定包围任意电荷分布的表面上的电场,但在具有对称性导致电场均匀分布的情况下,这是可能的。在没有这种对称性的情况下,可以使用高斯定律的微分形式,其表述为电场的散度与电荷的局部密度成正比。

该定律最早由约瑟夫-路易·拉格朗日于1773年[1][2]提出,后由卡尔·弗里德里希·高斯于1835年在椭球引力的背景下提出。[3] 它是麦克斯韦方程组之一,构成经典电动力学的基础。[注1] 高斯定律可以用于推导库仑定律,[4] 反之亦然。
\subsection{定义}
在积分形式下,高斯定律表述为:电场通过任意封闭曲面的通量与该曲面所包围的电荷成正比,而不论电荷如何分布。尽管仅凭此定律不足以确定包围任意电荷分布的表面上的电场,但在对称性要求电场均匀的情况下,这是可能的。在不存在这种对称性的情况下,可以使用高斯定律的微分形式,其表述为电场的散度与电荷的局部密度成正比。

该定律最早由约瑟夫-路易·拉格朗日于1773年[1][2]提出,随后在1835年由卡尔·弗里德里希·高斯[3]提出,两者都是在椭球引力的背景下提出的。它是麦克斯韦方程组之一,构成经典电动力学的基础。[注1] 高斯定律可以用于推导库仑定律,[4] 反之亦然。
\subsection{定性描述}
用语言描述,高斯定律表述为:

任意假设的封闭曲面的净电通量等于该封闭曲面内所包围的净电荷除以 \(\varepsilon_0\)。该封闭曲面也称为高斯面。[5]

高斯定律在数学上与物理学其他领域中的多条定律有密切的相似性,例如磁学中的高斯定律和重力中的高斯定律。事实上,任何反平方定律都可以用与高斯定律类似的方式表述:例如,高斯定律本质上等价于库仑定律,而重力的高斯定律本质上等价于牛顿的万有引力定律,它们都是反平方定律。

该定律可以使用向量微积分以积分形式和微分形式表示;两者是等价的,因为它们通过散度定理(也称为高斯定理)相关联。这些形式还可以通过两种方式表达:一种是电场 \( E \) 与总电荷之间的关系,另一种是电位移场 \( D \) 与自由电荷之间的关系。[6]
\subsection{包含电场 \( E \) 的方程}
高斯定律可以使用电场 \( E \) 或电位移场 \( D \) 来表述。本节展示了一些包含 \( E \) 的形式;包含 \( D \) 的形式和其他包含 \( E \) 的形式在下方。
\subsubsection{积分形式}
\begin{figure}[ht]
\centering
\includegraphics[width=8cm]{./figures/d34f5443a269bb1a.png}
\caption{任意曲面的电通量与该曲面所包围的总电荷成正比。} \label{fig_GSDL_2}
\end{figure}
高斯定律可以表示为:[6]
\[
\Phi_{E} = \frac{Q}{\varepsilon_{0}}~
\]
其中,\(\Phi_{E}\) 是穿过包围任意体积 \(V\) 的封闭曲面 \(S\) 的电通量,\(Q\) 是体积 \(V\) 内的总电荷,\(\varepsilon_{0}\) 是电常数。电通量 \(\Phi_{E}\) 定义为电场的曲面积分:
\[
\Phi_{E} = \iint_{S} \mathbf{E} \cdot \mathrm{d} \mathbf{A}~
\]
其中,\(\mathbf{E}\) 是电场,\(\mathrm{d} \mathbf{A}\) 是表示曲面微小面积元素的向量,[注2] “\(\cdot\)” 表示两个向量的点积。

在弯曲时空中,电磁场通过封闭曲面的通量表示为
\[
\Phi_{E} = c\iint_{S} F^{\kappa 0} \sqrt{-g} \, \mathrm{d} S_{\kappa}~
\]
其中 \(F^{\kappa 0}\) 是电磁场张量的分量,\(\sqrt{-g}\) 是度规的行列式的平方根,\(\mathrm{d} S_{\kappa}\) 是闭合曲面的微分面积元素。
\begin{figure}[ht]
\centering
\includegraphics[width=8cm]{./figures/40885a4587d08090.png}
\caption{球体内没有包围电荷,其表面的电通量为零。} \label{fig_GSDL_3}
\end{figure}
其中,  
\( c \) 表示光速;  
\( F^{\kappa 0} \) 表示电磁张量的时间分量;  
\( g \) 是度规张量的行列式;  
\(\mathrm{d} S_{\kappa} = \mathrm{d} S^{ij} = \mathrm{d} x^{i} \mathrm{d} x^{j}\) 是围绕电荷 \( Q \) 的二维表面的正交标准元素,  
其中的指标 \( i, j, \kappa = 1, 2, 3 \),且互不相同。[8]

由于通量定义为电场的积分,因此高斯定律的这种表达方式称为积分形式。

在涉及已知电位的导体的问题中,导体外部的电位通过求解拉普拉斯方程获得,可以采用解析或数值方法。然后,电场计算为电位的负梯度。高斯定律使得找到电荷分布成为可能:导体中任一区域的电荷可以通过积分电场得出,从而找到通过一个小盒子的通量,该盒子的边垂直于导体表面,并且电场垂直于表面且在导体内部为零。

逆问题是已知电荷分布并需要计算电场,这个问题要困难得多。给定表面的总通量对电场信息提供很少的线索,通量可以以任意复杂的模式进出表面。

一个例外是问题中存在某种对称性,这使得电场以均匀的方式通过表面。此时,如果已知总通量,就可以在每个点推导出电场。常见的对称性示例包括:柱对称性、平面对称性和球对称性。有关利用这些对称性来计算电场的示例,请参见“高斯面”条目。
\begin{figure}[ht]
\centering
\includegraphics[width=8cm]{./figures/400f9c8a4f36180f.png}
\caption{一个微小的高斯盒子,其边垂直于导体的表面,用于在通过求解拉普拉斯方程计算出电位和电场后找到局部表面电荷。电场在导体的等势面上局部垂直于表面,并在内部为零;根据高斯定律,其通量 \( \pi a^2 \cdot E \) 等于 \( \pi a^2 \cdot \sigma / \varepsilon_0 \)。因此,\(\sigma = \varepsilon_0 E\)。} \label{fig_GSDL_4}
\end{figure}
\subsubsection{微分形式}
根据散度定理,高斯定律也可以写成微分形式:
\[
\nabla \cdot \mathbf{E} = \frac{\rho}{\varepsilon_0}~
\]
其中 \(\nabla \cdot \mathbf{E}\) 是电场的散度,\(\varepsilon_0\) 是真空介电常数,\(\rho\) 是总体积电荷密度(每单位体积的电荷量)。
\subsubsection{积分形式与微分形式的等价性}
积分形式和微分形式在数学上是等价的,根据散度定理。以下是更具体的论证。\\
\textbf{证明概述}\\
高斯定律的积分形式为:
\[
\iint_{S} \mathbf{E} \cdot \mathrm{d} \mathbf{A} = \frac{Q}{\varepsilon_0}~
\]
对于包含电荷 \(Q\) 的任意闭合曲面 \(S\)。根据散度定理,这个方程等价于:
\[
\iiint_{V} \nabla \cdot \mathbf{E} \, \mathrm{d} V = \frac{Q}{\varepsilon_0}~
\]
对于包含电荷 \(Q\) 的任意体积 \(V\)。根据电荷与电荷密度的关系,这个方程等价于:
\[
\iiint_{V} \nabla \cdot \mathbf{E} \, \mathrm{d} V = \iiint_{V} \frac{\rho}{\varepsilon_0} \, \mathrm{d} V~
\]
对于任意体积 \(V\)。为了使该方程在所有可能的体积 \(V\) 内同时成立,必要(且充分)的条件是被积函数在每一点上相等。因此,这个方程等价于:
\[
\nabla \cdot \mathbf{E} = \frac{\rho}{\varepsilon_0}~
\]
因此,积分形式和微分形式是等价的。

\subsection{包含电位移场 \( D \) 的方程}
\subsubsection{自由电荷、束缚电荷和总电荷}  

在最简单的教科书案例中产生的电荷通常被归类为“自由电荷”——例如,静电中的转移电荷,或电容器极板上的电荷。相比之下,“束缚电荷”仅在电介质(可极化材料)的情境中出现。(所有材料在一定程度上都是可极化的。)当这些材料置于外部电场中时,电子仍然束缚于各自的原子,但会响应电场发生微小位移,导致电子在原子的一侧比另一侧更多。所有这些微观位移加起来形成一个宏观的净电荷分布,这就是“束缚电荷”。

尽管在微观层面上所有电荷本质上是相同的,但出于实际原因,人们通常希望将束缚电荷与自由电荷区分开来。因此,更为基础的高斯定律(基于电场 \( E \) 的形式)有时会转换为下述等效形式,仅涉及电位移场 \( D \) 和自由电荷。
\subsubsection{积分形式}

这种高斯定律的表述表示总电荷形式:
\[
\Phi_{D} = Q_{\mathrm{free}}~
\]
其中,\(\Phi_{D}\) 是穿过封闭体积 \(V\) 的曲面 \(S\) 的 \(D\)-场通量,\(Q_{\mathrm{free}}\) 是 \(V\) 中的自由电荷。通量 \(\Phi_{D}\) 的定义类似于电场 \(E\) 穿过 \(S\) 的通量 \(\Phi_{E}\):
\[
\Phi_{D} = \iint_{S} \mathbf{D} \cdot \mathrm{d} \mathbf{A}~
\]
\subsubsection{微分形式}

高斯定律的微分形式,仅涉及自由电荷,表述为:
\[
\nabla \cdot \mathbf{D} = \rho_{\mathrm{free}}~
\]
其中,\(\nabla \cdot \mathbf{D}\) 是电位移场的散度,\(\rho_{\mathrm{free}}\) 是自由电荷密度。
\subsection{总电荷与自由电荷表述的等价性}

\textbf{证明:高斯定律中自由电荷表述与总电荷表述的等价性}

在此证明中,我们将展示以下方程:
\[
\nabla \cdot \mathbf{E} = \frac{\rho}{\varepsilon_0}~
\]
等价于方程:
\[
\nabla \cdot \mathbf{D} = \rho_{\mathrm{free}}~
\]
请注意,我们仅处理微分形式,而不涉及积分形式;然而这已足够,因为根据散度定理,微分形式和积分形式在每种情况下都是等价的。

我们引入极化密度 \( \mathbf{P} \),其与 \( \mathbf{E} \) 和 \( \mathbf{D} \) 的关系如下:
\[
\mathbf{D} = \varepsilon_0 \mathbf{E} + \mathbf{P}~
\]
以及与束缚电荷的关系:
\[
\rho_{\mathrm{bound}} = -\nabla \cdot \mathbf{P}~
\]
现在,考虑以下三个方程:
\[
\begin{aligned}
\rho_{\mathrm{bound}} &= \nabla \cdot (-\mathbf{P}) \\
\rho_{\mathrm{free}} &= \nabla \cdot \mathbf{D} \\
\rho &= \nabla \cdot (\varepsilon_0 \mathbf{E})
\end{aligned}~
\]
关键在于前两个方程之和等于第三个方程。这样,证明完成:第一个方程是根据定义成立的,因此第二个方程成立当且仅当第三个方程成立。因此,第二个方程和第三个方程是等价的,这正是我们想要证明的。
\subsection{线性材料的方程}
在均匀、各向同性、非色散的线性材料中,\(\mathbf{E}\) 和 \(\mathbf{D}\) 之间存在简单关系:
\[
\mathbf{D} = \varepsilon \mathbf{E}~
\]
其中 \(\varepsilon\) 是材料的介电常数。在真空(即自由空间)情况下,\(\varepsilon = \varepsilon_0\)。在这种情况下,高斯定律在积分形式下变为:
\[
\Phi_{E} = \frac{Q_{\mathrm{free}}}{\varepsilon}~
\]
在微分形式下变为:
\[
\nabla \cdot \mathbf{E} = \frac{\rho_{\mathrm{free}}}{\varepsilon}~
\]
\subsection{与库仑定律的关系}
\subsubsection{从库仑定律推导高斯定律  }
严格来说,仅从库仑定律无法推导出高斯定律,因为库仑定律仅适用于单个静电点电荷产生的电场。然而,如果假设电场遵循叠加原理,则可以从库仑定律证明高斯定律。叠加原理表述为:总电场是由每个粒子产生的电场的矢量和(如果电荷在空间中均匀分布,则为积分)。\\
\textbf{证明概述}

库仑定律指出,静止点电荷产生的电场为:
\[
\mathbf{E} (\mathbf{r}) = \frac{q}{4 \pi \varepsilon_0} \frac{\mathbf{e}_r}{r^2}~
\]
其中:
\begin{itemize}
\item \(\mathbf{e}_r\) 是径向单位向量,
\item \(r\) 是半径,即 \(|\mathbf{r}|\),
\item \(\varepsilon_0\) 是电常数,
\item \(q\) 是粒子的电荷,假设该粒子位于原点。
\end{itemize}
利用库仑定律中的表达式,我们通过对空间中每个点 \( \mathbf{s} \) 上的微小电荷在 \( \mathbf{r} \) 点产生的电场进行积分,得到在 \( \mathbf{r} \) 点的总电场:
\[
\mathbf{E} (\mathbf{r}) = \frac{1}{4 \pi \varepsilon_0} \int \frac{\rho (\mathbf{s}) (\mathbf{r} - \mathbf{s})}{|\mathbf{r} - \mathbf{s}|^3} \, \mathrm{d}^3 \mathbf{s}~
\]
其中,\(\rho\) 是电荷密度。接下来对该方程两边在 \( \mathbf{r} \) 处取散度,并利用已知定理[9]
\[
\nabla \cdot \left( \frac{\mathbf{r}}{|\mathbf{r}|^3} \right) = 4 \pi \delta (\mathbf{r})~
\]
其中 \(\delta(\mathbf{r})\) 是狄拉克 δ 函数,得到
\[
\nabla \cdot \mathbf{E} (\mathbf{r}) = \frac{1}{\varepsilon_0} \int \rho (\mathbf{s}) \, \delta (\mathbf{r} - \mathbf{s}) \, \mathrm{d}^3 \mathbf{s}~
\]
利用狄拉克 δ 函数的“筛选特性”,我们得到
\[
\nabla \cdot \mathbf{E} (\mathbf{r}) = \frac{\rho (\mathbf{r})}{\varepsilon_0}~
\]
这就是高斯定律的微分形式。

由于库仑定律仅适用于静止电荷,因此仅根据此推导无法预期高斯定律适用于运动电荷。事实上,高斯定律确实适用于运动电荷,因此在这一点上,高斯定律比库仑定律更具普适性。

\textbf{证明(不使用狄拉克 δ 函数)}

令 \(\Omega \subseteq \mathbb{R}^3\) 为一个有界的开集,并定义
\[
\mathbf{E}_0(\mathbf{r}) = \frac{1}{4\pi \varepsilon_0} \int_{\Omega} \rho(\mathbf{r'}) \frac{\mathbf{r} - \mathbf{r'}}{\|\mathbf{r} - \mathbf{r'}\|^3} \, \mathrm{d} \mathbf{r'} \equiv \frac{1}{4\pi \varepsilon_0} \int_{\Omega} e(\mathbf{r}, \mathbf{r'}) \, \mathrm{d} \mathbf{r'}~
\]
其中,\(\rho(\mathbf{r'})\) 是一个连续函数(电荷密度)。

对于所有 \(\mathbf{r} \neq \mathbf{r'}\) 成立的是:
\[
\nabla_{\mathbf{r}} \cdot \mathbf{e}(\mathbf{r}, \mathbf{r'}) = 0~
\]
现在考虑一个紧集 \(V \subseteq \mathbb{R}^3\),其边界 \(\partial V\) 为分段光滑边界,且满足 \(\Omega \cap V = \emptyset\)。因此,\(\mathbf{e}(\mathbf{r}, \mathbf{r'}) \in C^1(V \times \Omega)\),并根据散度定理:
\[
\oint_{\partial V} \mathbf{E}_0 \cdot d\mathbf{S} = \int_V \nabla \cdot \mathbf{E}_0 \, dV~
\]
由于 \(\mathbf{e}(\mathbf{r}, \mathbf{r'}) \in C^1(V \times \Omega)\),我们有:
\[
\nabla \cdot \mathbf{E}_0(\mathbf{r}) = \frac{1}{4 \pi \varepsilon_0} \int_{\Omega} \nabla_{\mathbf{r}} \cdot \mathbf{e}(\mathbf{r}, \mathbf{r'}) \, \mathrm{d} \mathbf{r'} = 0~
\]
因为 \(\Omega \cap V = \emptyset \Rightarrow \forall \mathbf{r} \in V \ \ \forall \mathbf{r'} \in \Omega \ \ \mathbf{r} \neq \mathbf{r'}\),因此 \(\nabla_{\mathbf{r}} \cdot \mathbf{e}(\mathbf{r}, \mathbf{r'}) = 0\)。

因此,由位于闭合曲面之外的电荷密度产生的电通量为零。

现在考虑 \( \mathbf{r}_0 \in \Omega \),并且 \( B_R(\mathbf{r}_0) \subseteq \Omega \) 为以 \( \mathbf{r}_0 \) 为中心、半径为 \( R \) 的球(因为 \(\Omega\) 是开集,所以这个球存在)。

令 \( \mathbf{E}_{B_R} \) 和 \( \mathbf{E}_C \) 分别为球内和球外产生的电场。那么:
\[
\mathbf{E}_{B_R} = \frac{1}{4 \pi \varepsilon_0} \int_{B_R(\mathbf{r}_0)} e(\mathbf{r}, \mathbf{r'}) \, \mathrm{d} \mathbf{r'}~
\]
\[
\mathbf{E}_C = \frac{1}{4 \pi \varepsilon_0} \int_{\Omega \setminus B_R(\mathbf{r}_0)} e(\mathbf{r}, \mathbf{r'}) \, \mathrm{d} \mathbf{r'}~
\]
并且
\[
\mathbf{E}_{B_R} + \mathbf{E}_C = \mathbf{E}_0~
\]
于是通量 \( \Phi(R) \) 为:
\[
\Phi(R) = \oint_{\partial B_R(\mathbf{r}_0)} \mathbf{E}_0 \cdot d\mathbf{S} = \oint_{\partial B_R(\mathbf{r}_0)} \mathbf{E}_{B_R} \cdot d\mathbf{S} + \oint_{\partial B_R(\mathbf{r}_0)} \mathbf{E}_C \cdot d\mathbf{S} = \oint_{\partial B_R(\mathbf{r}_0)} \mathbf{E}_{B_R} \cdot d\mathbf{S}~
\]
最后一个等式是由于 \((\Omega \setminus B_R(\mathbf{r}_0)) \cap B_R(\mathbf{r}_0) = \emptyset\) 以及上面的论证。

右侧为一个带电球产生的电通量,因此:
\[
\Phi(R) = \frac{Q(R)}{\varepsilon_0} = \frac{1}{\varepsilon_0} \int_{B_R(\mathbf{r}_0)} \rho(\mathbf{r'}) \, \mathrm{d} \mathbf{r'} = \frac{1}{\varepsilon_0} \rho(\mathbf{r'_c}) |B_R(\mathbf{r}_0)|~
\]
其中 \( \mathbf{r'_c} \in B_R(\mathbf{r}_0) \),最后一个等式是根据积分的均值定理得出的。利用夹逼定理和 \(\rho\) 的连续性,可得:
\[
\nabla \cdot \mathbf{E}_0(\mathbf{r}_0) = \lim_{R \to 0} \frac{1}{|B_R(\mathbf{r}_0)|} \Phi(R) = \frac{1}{\varepsilon_0} \rho(\mathbf{r}_0)~
\]
\subsubsection{从高斯定律推导库仑定律}
严格来说,库仑定律无法仅从高斯定律中推导出来,因为高斯定律没有提供关于 \(\mathbf{E}\) 的旋度的信息(参见亥姆霍兹分解和法拉第定律)。然而,如果我们额外假设点电荷产生的电场是球对称的(这个假设在电荷静止时是完全正确的,而在电荷运动时大致正确),则可以从高斯定律证明库仑定律。

\subsubsection{证明概述}
在高斯定律的积分形式中,将 \( S \) 取为以点电荷 \( Q \) 为中心、半径为 \( r \) 的球面,则有:
\[
\oint_{S} \mathbf{E} \cdot d\mathbf{A} = \frac{Q}~{\varepsilon_0}~
\]
根据球对称性的假设,被积函数是一个常数,可以提到积分外。结果为:
\[
4 \pi r^2 \hat{\mathbf{r}} \cdot \mathbf{E}(\mathbf{r}) = \frac{Q}{\varepsilon_0}~
\]
其中,\(\hat{\mathbf{r}}\) 是指向电荷径向外的单位向量。再次根据球对称性,\(\mathbf{E}\) 指向径向方向,因此得到:
\[
\mathbf{E}(\mathbf{r}) = \frac{Q}{4 \pi \varepsilon_0} \frac{\hat{\mathbf{r}}}{r^2}~
\]
这与库仑定律基本相同。因此,库仑定律中电场的反平方关系可以由高斯定律得出。
\subsection{参见}

\begin{itemize}
\item 镜像电荷法
\item 泊松方程唯一性定理
\item 斯蒂格勒法则示例列表
\end{itemize}