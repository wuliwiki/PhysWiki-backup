% 不定积分的常用技巧
% keys 不定积分

\begin{issues}
\issueDraft
\end{issues}

\pentry{不定积分\upref{Int},复数\upref{CplxNo}}

\subsection{一、分项积分法}
分项积分法的原理是不定积分的基本性质\upref{Int}:
$$
\int \left[f(x)+g(x)-h(x)\right]\,\mathrm{d}x=\int f(x)\,\mathrm{d}x+\int g(x)\,\mathrm{d}x-\int h(x)\,\mathrm{d}x~.
$$

\begin{itemize}
\item 多项式的积分等于各个单项式的积分之和
\end{itemize}
\begin{itemize}
\item 分母为多项式,可将其化为简单分式再积分
\end{itemize}

\begin{example}{}
求 $\displaystyle{\int \frac{\,\mathrm{d}x}{x^2-a^2}}$。

\textbf{解:}因式分解,\(x^2-a^2=(x+a)(x-a)\)。

设 $\displaystyle{\frac{1}{x^2-a^2}=\frac{A}{x-a}+\frac{B}{x+a}}$ 于是 \(A(x+a)+B(x-a)=1\)

由掩盖法, $\displaystyle{A=\frac{1}{2a},B=-\frac{1}{2a}}$ , 所以 $\displaystyle{\frac{1}{x^2-a^2}=\frac{1}{2a}\left(\frac{1}{x-a}-\frac{1}{x+a} \right)}$,于是:
\begin{equation}\begin{aligned}
LHS&=\frac{1}{2a}\left(\int\frac{\mathrm{d}x}{x-a}-\int\frac{\mathrm{d}x}{x+a}\right)\\&=\frac{1}{2a}\left(\ln|x-a|-\ln|x+a|\right)+C \\&=\frac{1}{2a}\ln\left|\frac{x-a}{x+a}\right|+C~.
\end{aligned}\end{equation}

更一般地,让我们求
$$
\int\frac{mx+n}{x^2+px+q}\,\mathrm{d}x~.
$$
对分母配方: $\displaystyle{x^2+px+q=\left(x+\frac{p}{2}\right)^2+q-\frac{p^2}{4}}$。

令 $\displaystyle{t=x+\frac{p}{2}}$ ,于是 $\displaystyle{x=t-\frac{p}{2},\mathrm{d}x=\mathrm {d}t}$ ,令 $\displaystyle{q-\frac{p^2}{4}=\pm a^2}$。

令 $\displaystyle{A=m,B=n-\frac{1}{2}mp}$ ,则 \(mx+n=At+B\)。

则 $\displaystyle{LHS=\int\frac{At+B}{t^2\pm a^2}\,\mathrm{d}t =A\int\frac{t\mathrm{d}t}{t^2\pm a^2}+B\int\frac{\mathrm{d}t}{t^2\pm a^2}}$。

首先,$\displaystyle{A\int\frac{t\mathrm{d}t}{t^2\pm a^2}=\frac{A}{2}\int\frac{\mathrm{d}\left(t^2\pm a^2\right)}{t^2\pm a^2}=\frac{A}{2}\ln\left|t^2\pm a^2\right|+C}$。

1. 当 $\displaystyle{q>\frac{p^2}{4}}$ 时,$\displaystyle{B\int\frac{\mathrm{d}t}{t^2+ a^2}=\frac{B}{a}\arctan{\frac{t}{a}}+C}$。

\begin{equation}\begin{aligned}
LHS&=\frac{A}{2}\ln\left|t^2+ a^2\right|+\frac{B}{a}\arctan{\frac{t}{a}}+C \\&=\frac{m}{2}\ln\left|x^2+px+q\right|+\frac{2n-mp}{\sqrt{4q-p^2}}\arctan{\frac{2x+p}{\sqrt{4q-p^2}}}+C~.
\end{aligned}\end{equation}

2. 当 $\displaystyle{q<\frac{p^2}{4}}$ 时:$\displaystyle{B\int\frac{\mathrm{d}t}{t^2- a^2}=\frac{B}{2a}\ln\left|\frac{t-a}{t+a}\right|+C}$。

\begin{equation}
\begin{aligned}
LHS&=\frac{A}{2}\ln\left|t^2- a^2\right|+\frac{B}{2a}\ln\left|\frac{t-a}{t+a}\right|+C \\&=\frac{m}{2}\ln\left|x^2+px+q\right|+\frac{2n-mp}{2\sqrt{p^2-4q}}\ln\left|\frac{x+2p-\sqrt{p^2-4q}}{x+2p+\sqrt{p^2-4q}}\right|+C~.
\end{aligned}
\end{equation}
\end{example}

\begin{itemize}
\item 对于被积函数的分母为二次函数,分子小于二次的情况,普遍可采用上面的公式
\end{itemize}
\begin{itemize}
\item 对于被积函数的分母大于二次的情况,需用待定系数法,将被积函数分解为简单分式之和。待定系数法有以下两种类型:
\end{itemize}
1. 方程两端同次幂的系数相同

2.“掩盖”

\begin{itemize}
\item 若被积函数的分式中分子幂次高于分母幂次,应用长除法,使其变成既约真分式,长除法的大致步骤如下:
\end{itemize}

1. 用分子的最高次项除以分母的最高次项,得到首商,写在横线上对应位置;

2. 分子对应项减去刚才的乘积,得到第一余式写在下面,将分子下一项抄下来;

3. 把第一余式当作被除式,重复上述操作,直至余式次数低于除式次数。
\begin{example}{}
$${\frac{x^3-12x^2-42}{x-3}=x^2-9x-27+\frac{-123}{x-3}}~.$$
\end{example}
\subsection{二、分部积分法}
分部积分的基本原理如下:
$$\int u\,\mathrm{d}v=uv-\int v\,\mathrm{d}u~.$$
更高级地,我们有莱布尼兹公式:
$$
\int uv^{(n+1)}\,\mathrm{d}x=uv^{(n)}-u'v^{(n-1)}+...+(-1)^nu^{(n)}v+(-1)^{n+1}\int u^{(n+1)}v\mathrm{d}x~.
$$
当被积函数的因式之一是多项式时,运用莱布尼兹公式特别方便。

\begin{example}{}
求 $\displaystyle{\int\left(2x^3+3x^2+4x+5\right)e^x\mathrm {d}x}$。

令 $\displaystyle{u=2x^3+3x^2+4x+5,\mathrm{d}v=e^x\,\mathrm{d}x,v=e^x}$,

则 $\displaystyle{u'=6x^2+6x+4,u''=12x+6,u'''=12\,;\,v'=e^x,v''=e^x,v'''=e^x}$。

\begin{equation}
\begin{aligned}
LHS&=(2x^3+3x^2+4x+5)e^x-(6x^2+6x+4)e^x+(12x+6)e^x-12e^x+C \\
&=(2x^3-3x^2+10x-5)e^x+C~.
\end{aligned}
\end{equation}
\end{example}
\begin{corollary}{}
求 $\displaystyle{\int P(x)e^{ax}\,\mathrm{d}x}$ ,其中 $P(x)$ 为 $x$ 的多项式

\textbf{解:}令 $v^{(n+1)}=e^{ax}$ ,则
$$
v^{(n)}=\frac{e^{ax}}{a},v^{(n-1)}=\frac{e^{ax}}{a^2},v^{(n-2)}=\frac{e^{ax}}{a^3}\dots~
$$ 
设 $\deg (P)=n$ ,则:
$$\int P(x)e^{ax}\,\mathrm{d}x=e^{ax}\left(\frac{P}{a}-\frac{P'}{a^2}+\frac{P''}{a^3}-...\right)~.$$
同理, $\displaystyle{\int P(x)\sin bx\,\mathrm{d}x=\sin bx\left(\frac{P'}{b^2}-\frac{P'''}{b^4}+...\right)-\cos bx\left(\frac{P}{b}-\frac{P''}{b^3}+...\right)+C}~.$

同理, $\displaystyle{\int P(x)\cos bx\,\mathrm{d}x=\sin bx\left(\frac{P}{b}-\frac{P''}{b^3}+...\right)+\cos bx\left(\frac{P'}{b^2}-\frac{P'''}{b^4}+...\right)+C}~.$
\end{corollary}
\begin{itemize}
\item 把 1 看作被积函数的因式之一,可以帮助积分
\end{itemize}
\begin{corollary}{}
求 $\displaystyle{\int \ln x\,\mathrm{d}x}$。

\textbf{解:}令 $\displaystyle{u=\ln x\,,\mathrm{d}v=\mathrm{d}x\Rightarrow \mathrm{d}u=\frac{1}{x},v=x}$ 。

则 $\displaystyle{\int \ln x\,\mathrm{d}x=x\ln x-\int x\cdot\frac{\mathrm{d}x}x=x\ln x-x+C}$。
\end{corollary}

\begin{corollary}{}
求 $\displaystyle{\int\arctan x\,\mathrm{d}x}$。

\textbf{解:}令 $\displaystyle{u=\arctan x\,,\mathrm{d}v=\mathrm{d}x\Rightarrow \mathrm{d}u=\frac{\mathrm{d}x}{x^2},v=x}~,$

则 $\displaystyle{\int\arctan x\,\mathrm{d}x=x\cdot \arctan x-\int\frac{x\,\mathrm{d}x}{1+x^2}=x\cdot \arctan x-\frac{1}{2}\ln\left(1+x^2\right)+C}~.$
\end{corollary}

下面我们研究几类常见复合函数的不定积分。
\begin{corollary}{}
研究 $\displaystyle{\int e^{ax}\cos{bx}\,\mathrm{d}x}$ 和 $\displaystyle{\int e^{ax}\sin{bx}\,\mathrm{d}x}$。

\textbf{解:}令 $u=\cos bx$ 或 $u=\sin bx$,$\mathrm{d}v=e^{ax}\mathrm{d}x$,

则 $\mathrm{d}u=-b\sin bx \,\mathrm{d}x$ 或 $\mathrm{d}u=b\cos bx\mathrm{d}x$,$\displaystyle{v=\frac{e^{ax}}{a}}$ ,于是
\begin{equation}
\leftgroup{\int e^{ax}\cos bx\,\mathrm{d}x &=\frac{1}{a}e^{ax}\cos bx+\frac{b}{a}\int e^{ax}\sin bx\,\mathrm{d}x\\
\int e^{ax}\sin bx\,\mathrm{d}x&=\frac{1}{a}e^{ax}\sin bx-\frac{b}{a}\int e^{ax}\cos bx\,\mathrm{d}x~.
}\end{equation}
解得
$$
\leftgroup{\int e^{ax}\cos bx\,\mathrm{d}x&=e^{ax}\frac{a\cos bx+b\sin bx}{a^2+b^2}+C\\ \int e^{ax}\sin bx\,\mathrm{d}x&=e^{ax}\frac{a\sin bx-b\cos bx}{a^2+b^2}+C~.}
$$
\end{corollary}

\begin{corollary}{}
研究 $\displaystyle{\int x^k (\ln x)^n\,\mathrm{d}x}$。

\textbf{解:}首先求 $\displaystyle{\int x^k\ln x\,\mathrm{d}x}$,

令 $\displaystyle{u=\ln x\,,dv=x^k\,\mathrm{d}x\Rightarrow\mathrm{d}u=\frac{\mathrm{d}x}{x},v=\frac{x^{k+1}}{k+1}}$,

则 $\displaystyle{LHS=\frac{x^{k+1}\ln x}{k+1}-\int\frac{x^k\,\mathrm{d}x}{k+1}=\frac{x^{k+1}\ln x}{k+1}-\frac{x^{k+1}}{(k+1)^2}+C}$。

然后,在 $\displaystyle{\int x^k (\ln x)^n\,\mathrm{d}x}$ 中,令 $\displaystyle{u=(\ln x)^n\,,\mathrm{d}v=x^k\,\mathrm{d}x\Rightarrow\,\mathrm{d}u=\frac{n(\ln x)^{n-1}\mathrm{d}x}{x}\,,v=\frac{x^{k+1}}{k+1}}$,

所以 $\displaystyle{\int x^k(\ln x)^n\,\mathrm{d}x=\frac{x^{k+1}}{k+1}(\ln x)^n-\frac{n}{k+1}\int x^k(\ln x)^{n-1}\,\mathrm{d}x}$。

此即原积分式的递推公式。
\end{corollary}

\begin{corollary}{}
研究 $\displaystyle{\int x^n e^{ax}\cos bx\,\mathrm{d}x,\int x^n e^{ax}\sin bx\,\mathrm {d}x}$。

\textbf{解:}令 $\displaystyle{u=x^n\,,\mathrm{d}v=e^{ax}\cos{bx}\,\mathrm{d}x\,\text{或}\,\mathrm{d}v=e^{ax}\sin{bx}\,\mathrm{d}x}$,

则 $\displaystyle{\mathrm{d}u=nx^{n-1}\mathrm{d}x\,,v=e^{ax}\frac{a\cos bx+b\sin bx}{a^2+b^2}\,\text{或}\,v=e^{ax}\frac{a\sin bx-b\cos bx}{a^2+b^2}}$。

令 $\displaystyle{I_n=\int x^n e^{ax}\cos bx\,\mathrm{d}x},J_n=\int x^n e^{ax}\sin bx\,\mathrm{d}x$,则:

$\displaystyle{I_n=x^n e^{ax}\frac{{a\cos bx+b\sin bx}}{a^2+b^2}-\frac{n}{a^2+b^2}\left(aI_{n-1}+bJ_{n-1}\right)+C}$,

$\displaystyle{J_n=x^n e^{ax}\frac{{a\sin bx-b\cos bx}}{a^2+b^2}-\frac{n}{a^2+b^2}\left(aJ_{n-1}-bI_{n-1}\right)+C}$。
\end{corollary}

\subsection{三、换元积分法\upref{IntCV}}
\subsubsection{1. 常用换元}
我们将求不定积分中较为常用的换元总结如下:
\begin{equation}
\int f(ax+b)\,\mathrm{d}x=\frac{1}{a}\int f(ax+b)\,\mathrm{d}\left(ax+b\right)~.
\end{equation}
\begin{equation}
\int x^nf(x^{n+1})\,\mathrm{d}x=\frac{1}{n+1}\int f(x^{n+1})\,\mathrm{d}(x^{n+1})~.
\end{equation}
\begin{equation}
\int \frac{f(\ln x)\,\mathrm{d}x}{x}=\int f(\ln x)\,\mathrm{d}(\ln x)~.
\end{equation}
\begin{equation}
\int e^xf(e^x)\,\mathrm{d}x=\int f(e^x)\,\mathrm{d}(e^x)~.
\end{equation}
\begin{equation}
\int \frac{f(\sqrt{x})\,\mathrm{d}x}{\sqrt{x}}=2\int f(\sqrt{x})\,\mathrm{d}(\sqrt{x})~.
\end{equation}
\begin{equation}
\int \frac{1}{x^n}f\left(\frac{1}{x^n}\right)\,\mathrm{d}x=(1-n)\int f\left(\frac{1}{x^{n-1}}\right)\,\mathrm{d}\left(\frac{1}{x^{n-1}}\right)~.
\end{equation}
\begin{equation}
\int f(\sin x)\cos x\,\mathrm{d}x=\int f(\sin x)\,\mathrm{d}(\sin x)~.
\end{equation}
\begin{equation}
\int f(\cos x)\sin x\,\mathrm{d}x=-\int f(\cos x)\,\mathrm{d}(\cos x)~.
\end{equation}
\begin{equation}
\int f(\tan x)\sec^2x\,\mathrm{d}x=\int f(\tan x)\,\mathrm{d}(\tan x)~.
\end{equation}

\subsubsection{2. 三角换元}
\begin{itemize}
\item $a^2-x^2$ 中,令 $x=a\sin\theta$。
\item $x^2+a^2$ 中,令 $x=a\tan \theta$。
\item $x^2-a^2$ 中,令 $x=a\sec\theta$。
\end{itemize}

\begin{example}{}
求 $\displaystyle{\int \sqrt{a^2-x^2}\,\mathrm{d}x}$ 。

\textbf{解:}令 $\displaystyle{x=a\sin\theta\,,\mathrm{d}x=a\cos\theta\,\mathrm{d}\theta\,,\sqrt{a^2-x^2}=a\cos\theta}$,
\begin{equation}
\begin{aligned}
LHS&=\int a\cos\theta\cdot a\cos\theta\,\mathrm{d}\theta\\ &=a^2\int\cos^2\theta\,\mathrm{d}\theta\\ &=a^2\int\frac{\cos2\theta+1}{2}\,\mathrm{d}\theta\\ &=a^2\left(\frac{\sin2\theta}{4}+\frac{\theta}{2}\right)+C\\ &=\frac{1}{2}a^2\left(\sin\theta\cos\theta+\theta\right)+C\\ &=\frac{1}{2}\left(x\sqrt{a^2-x^2}+a^2\arcsin\frac{x}{a}\right)+C~.
\end{aligned}
\end{equation}
\end{example}

\begin{example}{}
求 $\displaystyle{\int \sqrt{x^2-a^2}\,\mathrm{d}x}$。

令 $\displaystyle{x=a\sec\theta\,,\mathrm{d}x=a\sec\theta\tan\theta\,\mathrm{d}\theta\,,\sqrt{x^2-a^2}=a\tan\theta}$,
\begin{equation}
\begin{aligned}
LHS&=\int a\tan\theta\cdot a\sec\theta\tan\theta\,\mathrm{d}\theta\\ &=a^2\int \tan^2\theta \sec\theta\,\mathrm{d}\theta\\ &=a^2\int (\sec^2\theta-1)\sec\theta\,\mathrm{d}\theta\\ &=a^2\left(\int\sec^3\theta\,\mathrm{d}\theta-\int\sec\theta\,\mathrm{d}\theta\right)\\ &=a^2\left(\frac{1}{2}\sec\theta\tan\theta+\frac{1}2\ln\left|\tan\left(\frac{\theta}{2}+\frac{\pi}{4}\right)\right|-\ln\left|\tan\left(\frac{\theta}{2}+\frac{\pi}{4}\right)\right|\right)+C\\ &=\frac{1}{2}\left(x\sqrt{x^2-a^2}-a^2\ln|x+\sqrt{x^2-a^2}|\right)+C~.
\end{aligned}
\end{equation}
\end{example}

\begin{example}{}
求 $\displaystyle{\int\sqrt{x^2+a^2}\,\mathrm{d}x}$。

令 $x=a\tan \theta\,,\mathrm{d}x=a\sec^2\theta\,\mathrm{d}\theta\,,\sqrt{x^2+a^2}=a\sec\theta$,
\begin{equation}
\begin{aligned}
LHS&=a^2\int \sec^3\theta\,\mathrm{d}\theta\\ &=\frac{a^2}{2}\tan\theta\sec\theta+\frac{a^2}{2}\ln\left|\tan\left(\frac{\theta}{2}+\frac{\pi}{4}\right)\right|+C\\ &=\frac{1}{2}\left(x\sqrt{x^2+a^2}+a^2\ln\left|x+\sqrt{x^2+a^2}\right|\right)+C~.
\end{aligned}
\end{equation}

\end{example}

\subsubsection{3. 万能替换法}
若被积函数由 $\sin x$ 或 $\cos x$ 组成,可令 $\displaystyle{t=\tan\frac{x}{2}\,,x=2\arctan t\,,\mathrm{d}x=\frac{2}{1+t^2}\,\mathrm{d}t}$ ,则:
$$\sin\frac{x}{2}=\frac{t}{\sqrt{1+t^2}}\,,\cos \frac{x}{2}=\frac{1}{\sqrt{1+t^2}}~.$$
使用二倍角公式:
$$\sin x=2\sin\frac{x}{2}\cos\frac{x}{2}=\frac{2t}{1+t^2}~,$$
$$ \cos x=2\cos^2\frac{x}{2}-1=\frac{1-t^2}{1+t^2}~.$$

于是,被积函数就化为 $t$ 的有理函数。

\subsubsection{4. 欧拉换元法}
此类方法适用于计算形如 $\displaystyle{G\left(x,\sqrt{ax^2+bx+c}\right)}$ 的函数的不定积分。

\begin{itemize}
\item \textbf{第一类替换(要求 $a>0$)}
\end{itemize}
令 $\sqrt{ax^2+bx+c}=t-\sqrt{a}x$,

两边平方并消去二次项得: $bx+c=t^2-2\sqrt{a}tx$,

所以 $\displaystyle{x=\frac{t^2-c}{2\sqrt{a}t+b}\,,\sqrt{ax^2+bx+c}=\frac{\sqrt{a}t^2+bt+c\sqrt{a}}{2\sqrt{a}t+b}\,,\mathrm{d}x=2\frac{\sqrt{a}t^2+bt+c\sqrt{a}}{(2\sqrt{a}t+b)^2}\,\mathrm{d}t}$。

\begin{itemize}
\item 第二类替换(要求 $c>0$ )
\end{itemize}
令 $\displaystyle{\sqrt{ax^2+bx+c}=xt+\sqrt{c}}$

两边平方,消去 $c$ 得: $ax+b=t^2x+2\sqrt{c}t$

所以 :

$\displaystyle{x=\frac{2\sqrt{c}t-b}{a-t^2}\,,\sqrt{ax^2+bx+c}=\frac{2\sqrt{c}t-b}{a-t^2}}$

$\displaystyle{t+\sqrt{c}=\frac{\sqrt{c}t^2-bt+a\sqrt{c}}{a-t^2}\,,\mathrm{d}x=2\frac{\sqrt{c}t^2-bt+a\sqrt{c}}{(a-t^2)^2}\mathrm{d}t}$ 

\begin{itemize}
\item 第三类替换(要求 $b^2-4ac>0$)
\end{itemize}
若 $ax^2+bx+c=0$ 有相异实根 $\lambda$ 和 $\mu$

则 $ax^2+bx+c=a(x-\lambda)(x-\mu)$

令 $\sqrt{ax^2+bx+c}=t(x-\lambda)$

两边平方得,约去 $x-\lambda$ 得: $t^2x-ax=-a\mu+\lambda t^2$

于是 $\displaystyle{x=\frac{\lambda t^2-a\mu}{t^2-a^2}\,,t^2=\frac{a(x-\mu)}{x-\lambda}}$ 

于是: $\displaystyle{\sqrt{ax^2+bx+c}=\frac{a(\lambda-\mu)t}{t^2-a}\,,\mathrm{d}x=\frac{2a(\lambda-\mu)t}{(t^2-a)^2}\,\mathrm{d}t}$ 
