% 线性算子代数
% 线性算子|算子代数

\begin{issues}
\issueTODO
\end{issues}

\subsection{线性算子}
\pentry{多重线性映射\upref{MulMap},矩阵与线性映射\upref{MatLS}}
域 $\mathbb{F}$ 上所有从 $n$ 维矢量空间 $V$ 到 $m$ 维矢量空间 $W$ 的线性映射 $f:V\rightarrow W$ 的集合用符号 $\mathcal{L}(V,W)$ (或者 $\mathrm{Hom}(V,W)$) 表示,它仍是一个矢量空间\upref{MulMap},其上的一个线性映射和一个 $m\times n$ 的矩阵一一对应\upref{MatLS}.在 $V=W$ 的情形,矢量空间 $\mathcal{L}(V,W)$ 简记为 $\mathcal{L}(V)$ (或 $\mathrm{End}(V)$),它的向量通常称为\textbf{线性算子}.

\textbf{符号约定:}在线性代数部分,线性算子将用拉丁字母 $\mathcal{A,B,C,\cdots}$ 表示,而在矢量空间 $V$ 的基底 $(\bvec e_i)$ 之下对应的矩阵用字母 $A,B,C,\cdots$ 表示,另一基底 $(\bvec e_i)$ 之下对应矩阵则用字母 $A',B',C',\cdots$ 表示.总是用 $\mathcal{E}=\mathrm{Id}$ 和 $E=(\delta_{ij})$ 表示恒等(单位)映射 $\bvec x\mapsto \bvec x$ .算子 $\mathcal{A}$ 作用在 $\bvec x$ 上的结果简写成 $\mathcal{A}\bvec x$ (代替 $\mathcal{A}(\bvec x)$ ).

线性算子 $\mathcal{B}$ 称为 $\mathcal{A}$ 的\textbf{逆算子},如果 $\mathcal{AB}=\mathcal{BA}=\mathcal{E}$ .算子 $\mathcal{A}$ 的逆算子通常记为 $\mathcal{A}^{-1}$.由\autoref{MatLS2_cor1}~\upref{MatLS2},$\mathcal{A}^{-1}$ 存在等价于 $\mathrm{Ker}\mathcal{A}=0$ 或者 $\mathrm{dim}\;V=\mathrm{dim\;Im}\mathcal{A}$ .$\mathrm{dim\;Ker}\mathcal{A}$ 称为 $\mathcal{A}$ 的\textbf{亏数}.
\begin{example}{零算子}
零算子 $\mathcal{O}$ 把每个向量 $\bvec v\in V$ 都变成零:$\mathrm{rank}\; \mathcal{O}=0$
\end{example}
\begin{example}{投影算子}
$V=U\oplus W$ 
\end{example}