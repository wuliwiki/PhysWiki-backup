% 金属的变形(科普)

\subsection{变形}
正如我们拉伸一根弹簧,弹簧会变形一样;当我们拉伸一根金属棒时,金属棒也会变形.只不过由于金属棒的“弹性系数”很大,以正常人的手劲一般拉不出看得见的变形.

\begin{example}{}
\begin{figure}[ht]
\centering
\includegraphics[width=12cm]{./figures/MetDfm_1.png}
\caption{框架结构}} \label{MetDfm_fig1}
\end{figure}
事实上,与弹簧类似,金属的支持力也源自金属的细微变形...只要在安全的范围内.
\end{example}

根据变形的性质,一般把变形分为两类:弹性变形与塑形变形.顾名思义,弹性变形后撤去外力后金属的形状能恢复原样;而塑形变形后即使撤去外力,金属的形状也不能恢复.塑形变形与弹性变形不是非此即彼,而是相辅相成的.变形可以既包括弹性形变也包括塑形形变.但一般而言,塑形形变只在外力大到超过一定境界时才发生.

\begin{figure}[ht]
\centering
\includegraphics[width=10cm]{./figures/MetDfm_2.png}
\caption{塑性变形与弹性变形示意图} \label{MetDfm_fig2}
\end{figure}

那么,为什么会有两种类型的变形呢?这就涉及到变形的微观原理了.大体而言,弹性变形时原子间的“键”被拉伸,但原子并没有运动到新的位置,因而撤去外力后原子可以回到原位,体现为形状恢复原样;

而塑形变形后,原子间原本的键已经被破坏、原子运动到了新的位置并形成了新的键.因此,塑形变形后金属的形状发生永久改变.

\subsection{塑性变形的微观特性}
\pentry{金属材料结构(科普)\upref{MetInt}}
接下来,我们更细致地探讨一下塑性变形.此处简要探讨塑性形变的主要机制之一,滑移.在本文中,我们先以单晶体为例(即金属中只有一个硕大的晶胞,原子的排列方向都一致)

\subsubsection{位错的运动}
或许你还记得位错\upref{MetInt}的概念.金属的滑移变形机制与位错有着密不可分的关系,位错理论的提出正是为了解释金属的塑性变形.

假设你有一块完整的晶体,现在你要施加外力使其塑性变形.看起来,为了使原子运动到新位置,你得先大力出奇迹、破坏一整面的键.毫无疑问这需要非常大的能量!

可事实上,现实中的金属强度远低于此(大概是按照这种理论计算的1/100至1/1000).怎么回事呢?这时候,假定金属中有一个位错.当位错存在时,奇迹发生了:由于位错的存在,现在上下部分相对运动时,只需断一列的键而不是一面的键.这大大降低了原子运动的难度
\begin{figure}[ht]
\centering
\includegraphics[width=5cm]{./figures/MetDfm_3.png}
\caption{请添加图片描述} \label{MetDfm_fig3}
\end{figure}