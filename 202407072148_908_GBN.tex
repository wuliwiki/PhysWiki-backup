% 尼古拉·哥白尼
% license CCBYSA3
% type Wiki

(本文根据 CC-BY-SA 协议转载自原搜狗科学百科对英文维基百科的翻译)

尼古拉·哥白尼(/koʊˈpɜːrnɪkəs, kə-/;[1][2][3] 波兰语:Mikoaj Kopernik; 德语:Nikolaus Kopernikus;Niklas Koppernigk;1473年2月19日-1543年5月24日)是文艺复兴时期的数学家和天文学家,他构想出一个宇宙模型,将太阳而不是地球置于宇宙的中心,很可能独立于萨摩斯的阿里斯塔克斯,他在大约十八世纪前构想出了这样一个模型。[4]

在1543年他逝世之前,哥白尼模型才在他的书De revolutionibus orbium coelestium (《天体运行论》)中发布,这是科学史上的一件大事,引发了哥白尼革命,为科学革命做出了开创性的贡献。[5]

哥白尼在皇家普鲁士出生和逝世,该地区自1466年以来一直是波兰王国的一部分。他精通多种语言,博学多才,获得了教会法博士学位,同时也是数学家、天文学家、医生、古典学者、翻译家、州长、外交官和经济学家。1517年,他推导出了货币数量理论——经济学中的一个关键概念——1519年,他提出了一个后来被称为格雷欣定律的经济原则。

\subsection{生活}
尼古拉·哥白尼于1473年2月19日出生在波兰王国皇家普鲁士省Toruń(托伦)市。[6]他的父亲是克拉科夫的商人,母亲是一个富有的托伦商人的女儿。[7]尼古拉是四个孩子中最小的。他的兄弟Andreas(安德鲁)在 Frombork(弗龙堡)成为了奥古斯丁教徒。[7]他以母亲的名字命名的妹妹芭芭拉,成为了本笃会修女,在生命的最后几年,成为了Chełmno(库尔姆)修道院的院长;她在1517年后去世。[7]他的妹妹卡塔琳娜嫁给了商人和托伦市议会议员巴特尔·格特纳,留下了五个孩子,哥白尼去世前一直照顾他们。[7]哥白尼从未结过婚,也不知道有没有孩子,但至少从1531年到1539年,他与寄宿管家安娜·席林的关系被瓦尔米亚的两位主教视为丑闻,这两位主教多年来一直规劝他与“情妇”断绝关系。[8]

\subsubsection{1.1 父亲的家庭}
\begin{figure}[ht]
\centering
\includegraphics[width=6cm]{./figures/bb8de03d7ae32025.png}
\caption{托伦出生地(ul。Kopernika 15,左边的)。连同17号的房子(正确),它形成了muze um Mikoaja Kopernika。} \label{fig_GBN_1}
\end{figure}

哥白尼的父亲的家庭可以追溯到奈萨(Neiße)附近西里西亚的一个村庄。这个村庄的名字有不同的拼写,有Kopernik, Copernik, Copernic, Kopernic, Coprirnik,今天被拼写为Koperniki。[9]14世纪,他的家庭成员开始移居到其他西里西亚城市,先是到了波兰首都克拉科夫(1367年),后又移居到托伦(1400年)。[9]哥白尼的父亲-年长的尼古拉来自克拉科夫,很可能是简的儿子。[9]

尼古拉是以他父亲的名字命名的,他父亲第一次作为一个富裕的经营铜的商人出现在记录中,他主要在但泽(Gdańsk)出售铜。[10][11]1458年左右,他从克拉科夫搬到托伦。[12]托伦位于维斯瓦河畔,当时卷入了十三年战争,在这场战争中,波兰王国和普鲁士联邦(一个普鲁士城市、贵族和神职人员的联盟)结盟,与条顿骑士团争夺对该地区的控制权。在这场战争中,像但泽,托伦以及尼古拉·哥白尼的家乡这样的汉萨同盟选择支持波兰国王卡西米尔四世·贾吉隆(Casimir IV Jagiellon),他承诺尊重城市传统的巨大独立,而这正是条顿骑士团所挑战的。尼古拉的父亲积极参与当时的政治活动,支持波兰和城市反对条顿骑士团。[13]1454年,他调解了波兰红衣主教兹比格涅夫·奥列尼基和普鲁士城市之间偿还战争贷款的谈判。[9]在托伦第二次和约时期(1466年),条顿骑士团正式放弃了对其西部省份的所有权利主张,其中,直到波兰第一次(1772年)和第二次(1793年)分区之前,皇家普鲁士仍然是波兰王国的直辖地区。

哥白尼的父亲在1461年至1464年间与天文学家的母亲芭芭拉·瓦岑罗德结婚。[9]他大约死于1483年。[7]

\subsubsection{1.2 母亲的家庭}
\begin{figure}[ht]
\centering
\includegraphics[width=6cm]{./figures/8ce494e56395ed0f.png}
\caption{哥白尼的舅舅,卢卡斯·沃特森罗德二世} \label{fig_GBN_2}
\end{figure}
尼古拉的母亲芭芭拉·瓦岑罗德是一位富有的托伦贵族卢卡斯·瓦岑罗德(1462年已故)和市议员卡塔尔津纳(简·佩考的遗孀)的女儿,市议员在其他来源中被称为为Katarzyna Rüdiger gente Modlibóg(1476年已故)。[7]Modlibógs(莫德里堡家族)是波兰的一个显赫家族,自1271年以来在波兰历史上便广为人知。[14]Watzenrode(瓦岑罗德家族)和哥白尼一家一样,来自西里西亚,靠近Świdnica(施韦迪尼察),1360年后定居托伦。[7]他们很快成为最富有和最有影响力的贵族家庭之一。通过瓦岑罗德斯广泛的婚姻家庭关系,哥白尼与Toruń (托伦), Gdańsk (但泽) 和 Elbląg(埃尔宾)的富裕家庭,以及普鲁士著名的波兰贵族家庭:恰普斯基、泽扬斯基、科诺帕基斯和科西莱克基斯建立了联系。[7]卢卡斯和凯瑟琳有三个孩子:小卢卡斯·瓦岑罗德(1447-1512),他后来成为瓦尔米亚的主教和哥白尼的资助人;芭芭拉是这位天文学家的母亲(1495年后去世);克里斯蒂娜(去世于1502年之前),于1459年与托伦商人兼市长蒂德曼·冯·艾伦结婚。[7]

老卢卡斯·瓦岑罗德是一位富有的商人,1439年至1462年间担任司法法庭庭长,他是条顿骑士团的坚决反对者。[7]1453年,他是托伦的代表团成员,在Grudziądz (格鲁琼兹)会议上策划起义反对他们。[7]在随后的十三年战争中(1454-1466),他通过提供大量的货币补贴来积极支持普鲁士城市的战争努力(他后来只要回了其中的一部分),并在托伦和但泽进行政治活动,亲自在Łasin(塞辛)和 Malbork (马尔堡)参加战斗。[7]他去世于1462年。[7]

小卢卡斯·瓦岑罗德是这位天文学家的舅舅和资助人,他在克拉科夫大学(现为贾吉洛尼亚大学)以及科隆大学和博洛尼亚大学接受教育。他是条顿骑士团的死对头,条顿骑士团的大师曾称他为“魔鬼的化身”。1489年,瓦岑罗德被选为瓦尔米亚(Ermeland, Ermland)的主教,反对国王卡西米尔四世偏爱自己的儿子。国王曾希望自己的儿子担任这一席位。[15]结果,瓦岑罗德与国王争吵不休,直到卡西米尔四世在三年后去世。[16]随后瓦岑罗德与三位相继的波兰君主建立了密切的关系:约翰·艾伯特、亚历山大·贾吉隆和西吉斯蒙德一世。他是每一位统治者的朋友和关键顾问,他的影响极大地加强了瓦尔米亚和波兰本土之间的联系。[17]瓦岑罗德被认为是瓦尔米亚最有权力的人,他的财富、人脉和影响力为哥白尼的教育以及他在弗龙堡大教堂的教士生涯提供了保障。[15]

\subsubsection{1.3 语言}
\begin{figure}[ht]
\centering
\includegraphics[width=6cm]{./figures/1ba6d388047a2af0.png}
\caption{哥白尼的德文信到普鲁士公爵阿尔伯特,为…提供医疗建议乔治·冯·昆海姆(1541)} \label{fig_GBN_3}
\end{figure}
据说,哥白尼能流利地说拉丁语、德语和波兰语;他还会说希腊语和意大利语,并懂一些希伯来语。哥白尼现存的大部分著作是用拉丁语写成的,拉丁语是他在欧洲学术界运用一生的语言。

关于德语是哥白尼的母语的论点是,他出生在一个以讲德语为主的城市,1496年在博洛尼亚学习教会法时,他加入了日耳曼民族(Natio Germanorum)——一个学生组织,根据其1497年的章程,该组织向所有母语为德语的王国和州的学生开放。[18]然而,根据法国哲学家亚历山大·柯瓦雷(Alexandre Koyré)的说法,哥白尼加入日耳曼民族这一行为本身并不意味着哥白尼认为自己是德国人,因为来自普鲁士和西里西亚的学生通常都是这样分类的,这种分类带有某些特权,使得它成为说德语的学生的自然选择,而不管他们的种族或自我认同如何。[18][19]

\subsubsection{1.4 名字}
Kopernik,Copernik,Koppernigk这些姓氏是从1350年开始在克拉科夫以各种拼法记录的,显然是给奈萨公国的Kopernik村的人的(1845年以前是Kopernik,Copernik,Copernik和Koppirnik)。尼古拉·哥白尼的曾祖父被记录为于1386年在克拉科夫获得公民身份。地名Kopernik(现代Koperniki)与波兰语中的dill (koper)和德语中的copper (Kupfer)有不同的联系。虽然后缀-nik(或复数-niki)表示斯拉夫语和波兰语的代理名词。

这一时期常见的是,地名和姓氏的拼写差别很大。哥白尼“对正字法相当漠不关心”。[20]在他的童年时期,大约1480年,他父亲的名字(也就是未来天文学家的名字)在托伦被记录为Niclas Koppernigk。[21]在克拉科夫,他拉丁文的署名为Nicolaus Nicolai de Torunia(尼古拉,托伦尼古拉的儿子)。1496年在博洛尼亚,他在日耳曼Matricula Nobilissimi 大学注册,也就是Natio Germanica Bononiae 的Annales Clarissimae Nacionis Germanorum,相当于Dominus Nicolaus Kopperlingk de Thorn – IX grosseti。[22]在帕多瓦,他的署名变成“Nicolaus Copernik”,随后又改为“Coppernicus”。[20]这位天文学家因此把他的名字拉丁化为Coppernicus,通常用两个“p”(在研究的31个文献中有23个记载如此),[23]但后来在生活中他用了一个“p”。在《天体运行论》的标题页上,雷蒂库斯把这个名字(以名词属格或以所有格的形式)公布为“尼古拉·哥白尼”。

\subsubsection{1.5 教育}
\textbf{在波兰}

父亲去世后,小尼古拉的舅舅小卢卡斯·瓦岑罗德(1447-1512年)带着他长大,让他接受教育,发展事业。[7]瓦岑罗德与波兰著名的知识分子保持着联系,他也是意大利著名的人文主义者和克拉科夫朝臣Filippo Buonaccorsi的朋友。[24]哥白尼童年和教育的早期的原始文件未能得到保存。[7]哥白尼传记作者推测瓦岑罗德首先把年轻的哥白尼送到托伦的圣约翰学校,他本人也曾在这里任教。[7]后来,根据阿米蒂奇的说法,这个男孩在托伦维斯瓦河上游的波兰弗沃茨大教堂学校上学,这里的学生都为考入克拉科夫大学做准备,该所大学也是瓦岑罗德在波兰首都的母校。[25]
\begin{figure}[ht]
\centering
\includegraphics[width=10cm]{./figures/420309e082a8c780.png}
\caption{马尤斯学院在克拉科夫大学哥白尼的波兰文母校} \label{fig_GBN_4}
\end{figure}
在1491-1492年冬季学期,哥白尼以“Nicolaus Nicolai de Thuronia”的名字和他的兄弟安德鲁一起被克拉科夫大学(现在的贾吉洛尼亚大学)录取。[7]哥白尼在克拉科夫天文数学学院的全盛时期开始了他在艺术系的研究(从1491年秋季开始,大概到1495年夏季或秋季),为他后来的数学成就奠定了基础。[7]根据后来一个可靠的说法(Jan Brożek),哥白尼是阿尔伯特·布鲁日斯基的学生,后者当时(从1491年起)是亚里士多德的哲学教授,但在大学之外私下教授天文学;哥白尼开始熟悉布鲁日斯基对乔治·冯·费尔巴哈的《行星理论》的广泛阅读的评论,几乎可以肯定地说,他参加了毕斯库派的伯纳德和萨莫图伊的沃伊切赫·克里帕的讲座,也可能参加了约格沃的扬、沃罗乔的米夏(布雷斯洛)、普涅维的沃伊切赫和奥尔库斯的马尔钦·比利卡的其他天文学讲座。[26]

哥白尼在克拉科夫的研究使他在大学教授的数学天文学(算术、几何、几何光学、宇宙学、理论和计算天文学)中奠定了充分的基础,并对亚里士多德的哲学和自然科学著作(De coelo, 《形而上学》)和阿威洛依(它在将来塑造哥白尼的理论中扮演着重要角色)有了很好的了解,激发了他的学习兴趣,并使他熟悉了人文文化。[15]哥白尼通过独立阅读他在克拉科夫求学时期获得的书籍(欧几里德、哈利·阿本拉格尔、Alfonsine表、约翰尼斯·雷乔蒙塔努斯的Tabulae directionum)拓宽了他从大学讲堂获得的知识;或许,他最早的科学笔记也可以追溯到这一时期,现在部分保存在乌普萨拉大学(Uppsala University)。[15]哥白尼在克拉科夫开始收集大量天文学方面的藏书;这些书后来在16世纪50年代的大洪水期间被瑞典人作为战利品带走,现在存放在乌普萨拉大学图书馆。[27]

哥白尼在克拉科夫的四年对他批判能力的发展中起到了重要作用,他也在这期间开始了对天文学两个“官方”体系中逻辑矛盾的分析——亚里士多德的同心球理论和托勒密偏心圆和本轮理论——对这两个体系的超越和抛弃将是哥白尼自己创立宇宙结构学说的第一步。[15]

大概是在1495年秋天,哥白尼离开克拉科夫去了舅舅瓦岑罗德的教堂,他没有获得学位,瓦岑罗德在1489年被提升为瓦尔米亚的主教王子,不久之后(1495年11月之前),他想让外甥哥白尼接替瓦尔米亚教士职位,前任教士Jan Czanow于1495年8月25日去世而空出这一职位。出于不清楚的原因——可能是因为教会一部分人的反对,他们将此事上诉到罗马——哥白尼的安置被推迟,瓦岑罗德倾向于派他的两个外甥去意大利学习教会法,似乎是为了促进他们的教会事业,从而也加强了他自己在瓦尔米亚教会的影响力。[15]
\begin{figure}[ht]
\centering
\includegraphics[width=6cm]{./figures/8f5348b611e4c308.png}
\caption{圣十字架和圣巴塞洛缪学院教堂在wrocaw} \label{fig_GBN_5}
\end{figure}
哥白尼于1496年年中离开瓦尔米亚——可能是随该教会会长Jerzy Pranghe的随员一同前往意大利——在秋天,可能是10月,哥白尼抵达博洛尼亚,几个月后(1497年1月6日之后),他加入了博洛尼亚大学法律学生的日耳曼民族组织,其中包括来自西里西亚、普鲁士和波美拉尼亚的年轻波兰人以及其他国家的学生。[15]

\textbf{在意大利}

1497年10月20日,哥白尼通过代理人正式继承了两年前授予他的瓦尔米亚教士职位。除此之外,根据1503年1月10日在帕多瓦的一份文件,他将在沃罗斯瓦夫(当时在波希米亚王国)的圣十字学院教堂和圣巴塞洛缪担任一个闲职。尽管哥白尼在1508年11月29日被教皇特许授予以获得更多的圣俸,但是在他的教会生涯中,他不仅没有在教会获得更多的俸禄和更高的地位,而且还在1538年放弃了沃罗斯瓦夫的神职。尚不清楚他是否曾被任命为牧师。[28] 爱德华·罗森声称他不是。[29][30]哥白尼确实接受了一些小任命,这也足以让他胜任教会教士职位。[15] 天主教百科全书认为他很有可能被任命为圣职,因为在1537年,他是瓦尔米亚需要任命的主教席位的四个候选人之一。[31]

哥白尼在1496年秋至1501年春在博洛尼亚待了三年,他似乎不太热衷于研究教会法(他在1503年第二次回到意大利,紧接着他获得了等了七年的的法学博士学位),而更热衷于研究人文学科——可能参加了菲利普·贝罗拉多(Filippo Beroaldo)、安东尼奥·乌尔西奥(Antonio Urceo,又名Codro)、乔瓦尼·加尔松尼(Giovanni Garzoni)和亚历山德罗·阿奇里尼(Alessandro Achillini)的讲座——并研究天文学。他遇到了著名天文学家多梅尼科·玛丽亚·诺瓦拉·达费拉拉( Domenico Maria Novara da Ferrara),成为他的弟子和助手。[15]哥白尼通过阅读乔治·冯·费尔巴哈和约翰尼斯·雷乔蒙塔努斯(威尼斯,1496)的《天文学大成的缩影》(托勒密天文学大成的缩影)发展了新的思想。他通过1497年3月九号在博洛尼亚对金牛座中最亮的恒星毕宿五被月亮遮挡的观测,验证了托勒密月球运动理论中某些特性。人文主义者哥白尼通过仔细阅读希腊和拉丁作家的著述(毕达哥拉斯、萨摩斯的阿里斯塔克斯、克莱门德斯、西塞罗、老普林尼、普卢塔克、菲洛劳斯、赫拉克里德斯、埃克芬托斯、柏拉图),尤其在帕多瓦时期,收集关于古代天文学、宇宙哲学和历法系统的零星历史信息,为他日益增长的疑惑寻求证实。[32]
\begin{figure}[ht]
\centering
\includegraphics[width=6cm]{./figures/0bcff94e19a10aad.png}
\caption{加列拉65号,博洛尼亚,梅尼科·玛丽亚·诺瓦拉的住所。} \label{fig_GBN_6}
\end{figure}
哥白尼在罗马渡过了大赦年1500年,那年春天他和他的兄弟安德鲁来到罗马,无疑是为了在罗马教皇法院当学徒。然而,在这里,他也继续着始于博洛尼亚的天文工作,例如,观察了1500年11月5日至6日晚上的月食。根据雷蒂库斯后来的叙述,哥白尼可能在私下里是以天文学教授的身份向“许多学生和科学界的大师”公开讲授对于当代天文学的数学解决方案的评判,当然这并非是在罗马的萨皮恩扎进行的。[33]

1501年年中,哥白尼在返回瓦尔米亚的途中无疑在博洛尼亚有过短暂停留。在7月28日从教会获得为期两年的学习医学的假期后(因为“他将来可能是有用的医学顾问或者教会的高级教士”),在夏末或秋季,他再次回到意大利,可能由他的兄弟安德鲁和Bernhard Sculteti教士陪同。这一次,他在帕多瓦大学学习,那里以医学学习而闻名,除了在1503年5月至6月短暂访问了费拉拉以通过考试并获得教会法博士学位之外,他从1501年秋季到1503年夏季一直留在帕多瓦。[33]

哥白尼很可能在帕多瓦大学的顶尖教授——巴托洛米奥·达·蒙塔加纳纳、吉罗拉摩·法兰卡斯特罗、加布里埃尔·泽比、亚历山德罗·贝内代蒂——的指导下学习医学,并阅读他当时获得的医学论文,这些论文由瓦莱斯库斯·德·塔兰塔、扬·梅苏、雨果·塞恩西斯、扬·凯瑟姆、阿诺德·德·维拉诺瓦和米歇尔·萨沃纳罗拉撰写,这些论文将形成他后来医学藏书的雏形。[33]

哥白尼必须研究的科目之一是占星术,因为它被认为是医学教育的重要组成部分。[34]然而,与文艺复兴时期的大多数著名天文学家不同,他似乎从未实践过占星术,也从未表达过对占星术的兴趣。[35]

如同在博洛尼亚一样,哥白尼并不局限于他的官方研究。很可能是在帕多瓦的时候,他对希腊文化产生了兴趣。他借助希多罗斯·加沙的语法(1495年)和杰·布·克里斯托纽斯的字典(1499年)熟悉希腊语言和文化,他开始拓展对古典著作的研究,先是从博洛尼亚开始,研究巴西利乌斯·贝萨里昂、洛伦佐·瓦拉等人的著作。似乎也有证据表明,正是在他在帕多瓦逗留期间,一个基于地球运动的世界新体系的想法最终具体化了。[33]随着哥白尼回家的时间临近,1503年春天,他前往费拉拉,1503年5月31日,他在那里通过了规定的考试,被授予教会法博士学位(Nicolaus Copernich de Prusia, Jure Canonico ... et doctoratus)。[36]毫无疑问,就在不久之后(最迟在1503年秋天),他离开意大利回到瓦尔米亚。[33]
\begin{figure}[ht]
\centering
\includegraphics[width=10cm]{./figures/13504257486ffa5c.png}
\caption{“梅尼科·玛丽亚·诺瓦拉的住所就在这儿,他是过去博洛尼亚研究院的教授,波兰数学家兼天文学家尼古拉·哥白尼(NICOLAUS COPERNICUS)在1497-1500年与他的老师进行了杰出的天体观测。于哥白尼诞辰的第五百年放置于这座城市的博洛尼亚大学,博洛尼亚研究院科学院的波兰科学院中。1473 [–] 1973”} \label{fig_GBN_7}
\end{figure}

\subsubsection{1.6 行星观测}
哥白尼对水星进行了三次观测,误差为-3、-15和-1弧分。他对火星进行了一次观测,误差为-24弧分。这四个数据是通过观测火星得到的,误差分别为2弧分、20弧分、77弧分和137弧分。对木星进行了四次观测,误差分别为32、51、-11和25弧分。他对木星进行了四次观测,误差分别为31弧分、20弧分、23弧分和-4弧分。[37]

\subsubsection{1.7 工作}
\begin{figure}[ht]
\centering
\includegraphics[width=10cm]{./figures/44fd8ee43b4d6c25.png}
\caption{天文学家哥白尼,或与上帝的对话,1873年,作者马泰伊科。背景:Frombork大教堂。} \label{fig_GBN_8}
\end{figure}
30岁的哥白尼完成了他在意大利的所有学业后,回到了瓦尔米亚,在那里他将度过余生的40年,除了到克拉科夫和附近的普鲁士城市做过短暂的旅行:Toruń(托伦),Gdańsk(但泽), Elbląg (埃尔宾), Grudziądz (格鲁琼兹), Malbork (马尔堡), Königsberg (克鲁维茨)。[33]

瓦尔米亚的主教辖区享有高度自治,拥有自己的国会(议会)和货币单位(与皇家普鲁士的其他地方一样)以及国库。[38]

哥白尼从1503年到1510年(或者直到他叔叔在1512年3月29日去世)是他叔叔的秘书和医生,住在利兹巴克(海尔斯堡)的主教城堡里,在那里他开始研究他的日心说。他以官方身份参与了他叔叔几乎所有的政治、教会和行政经济职责。从1504年初开始,哥白尼陪同瓦岑罗德参加了在马尔堡和埃尔宾举行的普鲁士皇家国会会议。根据Dobrzycki 和Hajdukiewicz的记载,他“参与了...复杂外交活动的所有重要活动,这些是有野心的政客和政治家在为了有敌意的条顿骑士团和忠于波兰王室之间,维护普鲁士和瓦尔米亚的特殊权利而做的努力”。[33]
\begin{figure}[ht]
\centering
\includegraphics[width=6cm]{./figures/c833c12f0d3cf7d2.png}
\caption{哥白尼对Theophylact Simocattas书信。封面秀盾形纹章关于(从顶部顺时针方向) 波兰立陶宛和克拉科夫。} \label{fig_GBN_11}
\end{figure}
1504-12年间,哥白尼作为他叔叔的随从进行了多次旅行——1504年,去托伦和格但斯克,在波兰国王亚历山大·贾吉隆的见证下参加普鲁士皇家委员会的一次会议;参加在马尔堡(1506年)、埃尔宾 (1507年)和什图姆(1512年)举行的普鲁士国会会议;他可能参加了波兹南会议(1510年)和波兰国王西吉斯蒙德一世在克拉科夫的加冕典礼(1507年)。瓦岑罗德的行程表明哥白尼可能在1509年春天参加了克拉科夫议会。[33]

也许是在后来克拉科夫的活动中,哥白尼将他的一部7世纪拜占庭历史学家西奥菲拉特·西蒙卡特(Theophylact Simocatta)的诗集的希腊文翻译成拉丁文,提交给简·哈雷出版社印刷,这本诗集有85首被称为书信或者信的短诗,据说是在希腊故事中的不同人物之间传递的。它由三部分内容组成——“道德篇”,就人们应该如何生活提供建议;“田园篇”,讲的是牧羊人的生活片段;“爱情篇”,包括很多爱情诗在里面。在每一个话题中这三部分内容都会交替出现。哥白尼把希腊的诗句翻译成了拉丁散文,现在他出版了他的版本Theophilacti scolastici Simocati epistolae morales, rurales et amatoriae interpretatione latina,并将这本书献给他的叔叔,感谢他从他那里得到的一切好处。在这本译作中,在希腊文学是否应该复兴的问题上,哥白尼宣称自己站在人文主义者一边。[39]哥白尼的第一部诗歌作品是一首希腊短诗,很可能是在访问克拉科夫时为约翰内斯·丹蒂斯丘斯就芭芭拉·萨波利亚于1512年与波兰国王老齐格蒙特一世的婚礼而作。[40]

1514年前的某个时候,哥白尼写下了他的日心说的最初大纲,这个大纲只有后来的抄本才有,书名(可能是从抄本上得来的)是Nicolai Copernici de hypothesibus motuum coelestium a se constitutis commentariolus——通常被称为《短论》。这是对世界的日心说机理的简明理论描述,没有数学工具计算,在几何构造的一些重要细节上不同于《天体运行论》;但是它已经是基于关于地球三重运动的相同假设。哥白尼有意识地将《短论》作为他计划出版的书的第一稿,并不打算印刷发行。他只向他最亲密的熟人提供了很少的手稿副本,其中似乎包括几个克拉科夫天文学家,他们曾于1515-30年间一起合作观察日食。第谷·布拉赫将在他自己的著作《新编天文学初阶》中加入《短论》的片段,该论文于1602年在布拉格出版,其依据是他从波希米亚物理学家和天文学家塔德·瓦什·哈耶克那里收到的手稿,后者是雷蒂库斯的朋友。《短论》直到1878年才首次完整出版。[40]
\begin{figure}[ht]
\centering
\includegraphics[width=6cm]{./figures/9138a1f7965db02a.png}
\caption{哥白尼塔在Frombork,他居住和工作的地方;二战后重建} \label{fig_GBN_12}
\end{figure}
1510年或1512年,哥白尼搬到了弗龙堡,这是波罗的海沿岸维斯瓦泻湖西北的一个城镇。1512年4月,他在那里作为瓦尔米业的主教王子参加了洛沙伊宁的法比安的竞选。直到1512年6月初,教会才给哥白尼一个“外部教廷”——大教堂山防御墙外的一所房子。1514年,他购买了弗龙堡要塞城墙内西北处的一座塔。尽管1520年1月条顿骑士团对弗龙堡的突袭摧毁了教会的建筑,哥白尼的天文仪器也可能在突袭中被摧毁,但他会将这两处住所维持到生命的最后。哥白尼在1513年至1516年进行了天文观测,大概是从他的外部教廷进行的;1522-1543年,在一座无明“小塔”(turricula)上,他使用了模仿古代仪器的原始仪器——四分仪、三棱镜、浑天仪。哥白尼在弗龙堡进行了一半以上的他的60多次有记载的天文观测。[40]

哥白尼在弗龙堡永久定居下来,在那里他一直居住到去世,这期间只有在1516-19年和1520-21年被打断,他在瓦尔米亚教会为自己建立了经济和行政中心,这也是瓦尔米亚政治生活的两个主要中心之一。在瓦尔米亚困难的、政治复杂的形势下,对外受到条顿骑士团侵略(条顿军团的攻击;1519-21年的波兰-条顿战争;艾伯特吞并瓦尔米亚的计划),内部受制于强大的分离主义者的压力(瓦尔米亚主教王子的选择;货币改革),他与部分教会的成员提出了一个与波兰王室严格合作的方案,并在其所有的公共活动中表明(保卫他的国家反对骑士团的征服计划;将货币体系与波兰王室统一的提议;支持波兰在瓦尔米亚统治的教会管理中的利益)他是波兰-立陶宛共和国的公民。瓦岑罗德主教舅舅去世后不久,他参加了第二次皮奥特劳·特里布斯基条约(1512年12月7日)的签署,该条约掌控着瓦尔米亚主教的任命,尽管一部分教会成员反对,但他宣布与波兰国王进行忠诚的合作。[40]
\begin{figure}[ht]
\centering
\includegraphics[width=8cm]{./figures/c1b2603cfca8a078.png}
\caption{Frombork大教堂山和防御工事。前景:哥白尼雕像。} \label{fig_GBN_13}
\end{figure}
同年(1512年11月8日之前),哥白尼作为教长皮斯托里亚(magister pistoriae)承担了管理该教会经济企业的责任(他将在1530年再次担任这一职务),自1511年以来他就负责教长之职并负责视察教会的财产。[40]

在1512-1515年间,哥白尼的行政和经济职责并没有分散他对密集观测活动的注意力。他在这一时期对火星和土星的观测结果,特别是1515年对太阳的一系列四次观测,导致了他对地球偏心率的可变性和太阳远地点相对于固定恒星的运动的发现,这在1515-19年促使他对他系统的某些假设进行了第一次修正。他在这一时期进行的一些观测可能与1513年上半年应米德尔堡的保罗·佛松布朗主教的要求提出的儒略历改革有关。他们在第五次拉特兰会议期间在这一问题上的联系后来被纪念在哥白尼的献礼性诗集(Dē revolutionibus orbium coelestium)以及在米德尔堡的保罗的一篇论文Secundum compendium correctionis Calendarii (1516年)中,其中这篇论文提到哥白尼是向会议提交日历修正建议的学者之一。[41]
\begin{figure}[ht]
\centering
\includegraphics[width=8cm]{./figures/388bccfac0dc2bd4.png}
\caption{奥尔兹廷城堡} \label{fig_GBN_14}
\end{figure}
1516年至1521年期间,哥白尼住在Olsztyn(奥尔什丁)城堡,作为瓦尔米亚的经济管理者管辖着Olsztyn(奥尔什丁)和Pieniężno(梅尔萨克)两地。在那里,他写了一份手稿,名为《荒芜的封地的位置》,目的是让勤劳的农民居住在这些封地,从而支撑瓦尔米亚的经济。波兰-条顿战争期间,奥尔什丁被条顿骑士围困时,哥白尼指挥波兰皇家部队保卫奥尔什丁和瓦尔米亚。他还在随后的和平谈判中代表波兰进行谈判。[42]

哥白尼多年来就货币改革向普鲁士皇家议会提出建议,尤其是在16世纪20年代,当时这是普鲁士地区政治中的一个主要问题。[43]1526年,他写了一篇关于货币价值的研究,“Monetae cudendae比率”。他在研究中中阐述了这一理论的早期版本,现在称之为格雷欣定律,即“坏的”(贬值的)货币推动“好的”(未贬值的)货币退出流通(劣币驱逐良币)——这比托马斯·格雷欣还早几十年。他还在1517年创立了货币数量理论,这是至今经济学中的一个主要概念。普鲁士和波兰领导人在试图稳定货币时,广泛阅读了哥白尼关于货币改革的建议。[44]
\begin{figure}[ht]
\centering
\includegraphics[width=8cm]{./figures/936bdd6890274a00.png}
\caption{华沙哥白尼纪念碑由丹麦雕塑家设计阿尔伯特·巴特尔·托瓦尔森} \label{fig_GBN_15}
\end{figure}
1533年,教皇克拉门特七世的秘书约翰·维德曼斯特向教皇和两位红衣主教解释哥白尼的日心说。教皇非常高兴,给了维德曼斯特一份珍贵的礼物。1535年,伯纳德·瓦波斯基给维也纳的一位绅士写了一封信,敦促他出版一本随附的年历,他声称这本年历是哥白尼写的。这是历史记录中唯一提到哥白尼年历的地方。这本“历书”很可能是哥白尼的行星位置表。瓦波斯基的信提到哥白尼关于地球运动的理论。瓦波斯基的要求毫无结果,因为几周后他就去世了。

在瓦尔米亚·毛里求斯·费尔伯亲王主教去世后(1537年7月1日),哥白尼参加了他的继任者约翰内斯·丹蒂斯丘斯的选举(1537年9月20日)。哥白尼是该职位的四名候选人之一,这在蒂德曼·吉斯的提议中有记载;但他的候选资格实际上只是形式上的,因为丹蒂斯丘斯早些时候就被任命为费尔伯的助理主教,而且丹蒂斯丘斯得到了波兰国王西吉斯蒙德一世的支持。[45]起初哥白尼与新的主教王子保持着友好的关系,在1538年春天对他进行了医学上的帮助,并在那个夏天陪同他参观了教会辖区。但是那年秋天,他们的友谊因对哥白尼的管家安娜·席林的怀疑而变得紧张起来,1539年春天丹蒂斯丘斯将他从弗龙堡驱逐出去。[45]

在他年轻的时候,哥白尼作为医师,曾为他的舅舅、兄弟和其他教会成员医治。在后来的几年里,他被安排去照顾那些先后在瓦尔米亚任职的老主教们——毛里求斯·费尔伯和约翰内斯·丹蒂斯丘斯——并且在1539年,他还照顾了在Chełmno(库尔姆)主教的老朋友Tiedemann Giese(蒂德曼·吉塞)。在治疗如此重要的病人时,他有时会向其他医生咨询,其中包括阿尔伯特公爵的医生和通过信函请教的波兰皇家医生。[46]
\begin{figure}[ht]
\centering
\includegraphics[width=6cm]{./figures/2b45b394dee9f492.png}
\caption{哥白尼举着一幅肖像铃兰,发表于Nicolaus Reusners彩色图谱(1587年),根据一个草图由Tobias Stimmer(约1570年),据称基于哥白尼的自画像。这幅肖像成了后来大多数哥白尼描述的基础。[44]} \label{fig_GBN_16}
\end{figure}
1541年春天,阿尔伯特公爵——前条顿骑士团的大师,他将条顿骑士的修道院国转变为路德教和世袭王国,在普鲁士公爵领地,阿尔伯特公爵在向他的叔叔波兰国王西吉斯蒙德一世致敬后,要求哥白尼到柯尼希斯堡去见公爵的顾问乔治·冯·昆海姆,他病得很重,普鲁士医生似乎对此无能为力。哥白尼自愿去了;他在货币改革的谈判中见过冯·昆海姆。哥白尼开始觉得阿尔伯特本人并不是一个如此坏的人;这两个人有许多共同的兴趣爱好。教会欣然同意哥白尼前往,因为它希望与公爵保持良好的关系,尽管他信奉路德教。大约一个月后,病人康复了,哥白尼回到了弗龙堡。有一段时间,他持续收到关于冯·昆海姆病情的报告,并写信给他提供医疗建议。[47]

哥白尼的一些密友变成了新教徒,但哥白尼从未表现出这种倾向。对他的首次打击来自新教徒。定居在埃尔宾的荷兰难民威廉·格纳菲乌斯用拉丁语写了一部喜剧Morosophus(《愚蠢的圣人》),并在自己建立的拉丁学校的舞台上演出。在剧中,哥白尼被讽刺为一个傲慢、冷漠、孤僻的人,他涉足占星术,认为自己受到了上帝的启发,并且有传言说他写了一部巨著,结果这部巨作都要烂在箱子里。[24]

在其他地方,新教徒首先对哥白尼理论的消息做出反应。墨兰顿写道:

有些人认为,那位萨尔马提亚天文学家让地球动起来并阻止太阳运动的荒谬说法很精彩也很正确。事实上,明智的统治者应该阻止这种轻浮的举动。

然而,在1551年,哥白尼去世八年后,天文学家Erasmus Reinhold(伊拉兹马斯·赖因霍尔德)在哥白尼的前军事反对者新教徒Duke Albert(阿尔伯特公爵)的赞助下,出版了一套基于哥白尼工作的天文表《普鲁士表》。天文学家和占星家很快就用它代替了它的前身。[48]

\subsubsection{1.8 日心说}
\begin{figure}[ht]
\centering
\includegraphics[width=6cm]{./figures/a159ce84d15b868f.png}
\caption{"尼古拉·哥白尼Tornaeus Borussus Mathemat。”,1597} \label{fig_GBN_17}
\end{figure}
1514年前的某个时候,哥白尼向朋友们提供了他的《短论》,一份描述他关于日心说的观点的手稿。它包含七个基本假设(详述如下)。[49]此后,他继续收集数据以进行更详细的工作。

大约在1532年,哥白尼基本上完成了他对《天体运行论》手稿的研究;但是,尽管受到他最亲密朋友的敦促,他拒绝公开发表自己的观点,不希望——正如他承认的那样——冒“由于论文新颖和不可理解而暴露自己”被轻蔑的风险。[45]

1533年,约翰·阿尔布雷特·威德曼斯特在罗马发表了一系列概述哥白尼理论的演讲。教皇克拉门特七世和几名天主教红衣主教听到了讲座,并对这一理论感兴趣。1536年11月1日,卡普亚大主教尼古拉·冯·舍恩伯格红衣主教从罗马写信给哥白尼:

几年前,我听说过你的精通,每个人都经常谈到这一点。那时我开始对你有很高的敬意……因为我知道你不仅非常好地掌握了古代天文学家的发现,而且还形成了一种新的宇宙学。你认为地球在运动;太阳占据宇宙中最低的位置,也就是中心位置...因此,我最诚挚地恳求你,最博学的先生,除非我给你带来不便,把你的这一发现告诉学者们,并尽早把你关于宇宙领域的著作连同数据表和其他与这个主题相关的东西寄给我……

那时哥白尼的工作已经接近了它的最终形式,关于他的理论的谣言已经传到全欧洲受过教育的人们耳中。尽管受到许多方面的催促,哥白尼仍然推迟出版他的书,也许是因为害怕批评——他在随后献给教皇保罗三世的杰作中微妙地表达了一种恐惧。学者们在哥白尼的关注是否仅限于可能的天文学和哲学,或者他是否也关注宗教的异议上没有达成一致。

\subsubsection{1.9 这本书}
哥白尼在1539年维滕贝格数学家乔治·约阿希姆·雷蒂库斯到达弗龙堡时,仍在研究《天体运行论》(即使不能确定他是否想出版它)。马丁·路德的亲密神学盟友菲利普·梅兰希顿已经安排雷蒂库斯拜访几位天文学家,并与他们一起研究。雷蒂库斯成为哥白尼的学生,在他身边学习了两年,写了一本书Narratio prima(《第一账户》),概述了哥白尼理论的精髓。1542年,雷蒂库斯发表了哥白尼关于三角学的论文(后来被收录为《天体运行论》第一卷第十三章和第十四章)。[50] 在雷蒂库斯的强大压力下,并在看到他的作品首次受到普遍好评后,哥白尼最终同意将《天体运行论》交给他的密友Chełmno(库尔姆)主教的Tiedemann Giese(蒂德曼·吉塞),由德国Nürnberg(纽伦堡)的德国印刷工约翰内斯·彼得雷乌斯(Johannes Petreius)交给雷蒂库斯印刷。当雷蒂库斯最初监督印刷时,他不得不在印刷完成前离开纽伦堡,他把监督其余印刷的任务交给路德派神学家安德烈亚斯·奥赛德(Andreas Osiander)。[51]

奥赛德增加了一个未经授权和未经签署的序言,为哥白尼的工作辩护,以防那些可能被其新颖假设冒犯的人。他认为,“对于同一运动,有时会提出不同的假设,(因此)天文学家会把最容易理解的假设作为自己的首选。”奥赛德认为,“这些假设不一定是真的,甚至是不可能的。如果他们提供了与观察结果一致的演算,这就足够了。”[52]

\subsubsection{1.10 去世}
\begin{figure}[ht]
\centering
\includegraphics[width=8cm]{./figures/e27406bd0f713fcd.png}
\caption{Frombork大教堂} \label{fig_GBN_18}
\end{figure}
快到1542年的时候,哥白尼被检查出中风瘫痪,于1543年5月24日逝世,享年70岁。传说在他去世的那天,他收到了《天体运行论》的初版,随后便与世长辞。据说他从中风引起的昏迷中醒来,看了看他的书,然后平静地辞世。

据报道哥白尼被埋葬在弗罗姆博克大教堂,在被破坏前,那里一直立着一个1580年的墓志铭;它在1735年被替换。两个多世纪以来,考古学家徒劳地在大教堂寻找哥白尼的遗体。1802年、1909年、1939年寻找它们的努力都白费了。2004年,在历史学家杰西·西科尔斯基的研究指导下,普什图克考古人类学研究所所长Jerzy Gąssowski领导的团队开始了一项新的研究。[53][54]2005年8月,在扫描了大教堂地板后,他们发现了他们认为是哥白尼的遗骸的踪迹。[55]
\begin{figure}[ht]
\centering
\includegraphics[width=6cm]{./figures/7e08a60b2879fcc3.png}
\caption{1735年墓志铭,Frombork大教堂} \label{fig_GBN_19}
\end{figure}
这一发现是在2008年11月3日进一步研究后才宣布的。Gąssowski说他“几乎百分之百肯定这是哥白尼”。[56] 波兰警察中央法医实验室的法医专家达理乌斯·扎德尔上尉用这个头骨重建了一张脸,这张脸与哥白尼自画像上的特征非常相似——包括鼻子骨折和左眼上方的伤疤。[56]这位专家还确定这个头骨属于一个死于70岁左右的人——哥白尼去世时的年龄。[55]

坟墓状况很差,没有找到所有的骨骼残骸;除此之外,他的下颌也不见了。[57]坟墓中发现的骨头的DNA与从哥白尼一本书中提取的头发样本相匹配,这本书保存在瑞典乌普萨拉大学的图书馆里。[54][58]

2010年5月22日,哥白尼举行了第二次弥撒葬礼,由前教皇驻波兰大使、新任命的波兰首席主教约瑟夫·科瓦尔茨克主持。哥白尼的遗体被重新埋葬在弗龙堡大教堂的同一个地方,在那里发现了他的部分头骨和其他骨头。现在一块黑色花岗岩墓碑上的铭文将他视为日心说的创始人,也是教会经典。墓碑上有哥白尼太阳系模型的代表——一个被六颗行星环绕的金色太阳。[59]

\subsection{哥白尼学说}
\subsubsection{2.1 前身}
菲洛劳斯(公元前480-385年)描述了一个天文系统,其中一个中心火(不同于太阳)占据了宇宙的中心,从中心向外,反地球、地球、月亮、太阳、行星和恒星都依次围绕着它旋转。赫拉克里德斯·庞蒂克斯(公元前387-312年)提出地球绕其轴线旋转。萨摩斯的阿里斯塔克斯(公元前310年至公元前230年)是第一个提出地球绕太阳运行理论的人。古希腊天文学家塞琉西亚的塞琉古斯在公元前150年左右发现了阿利斯塔克的日心说系统的进一步数学细节。尽管阿利斯塔克的原稿已经丢失,阿基米德的著作《数沙者》中的一篇参考文献描述了阿利斯塔克提出日心说模型的工作。托马斯·希思对阿基米德文本的英译本如下:

你(“你”指的是King Gelon)现在知道的“宇宙”是大多数天文学家给球体取的名字,球体的中心是地球的中心,而它的半径等于太阳中心和地球中心之间的直线距离。这是你从天文学家那里听到的常见的说法(τά γραφόμενα)。但是阿利斯塔克出版了一本由某些假设组成的书,在书中,作为假设的结果,宇宙比刚才提到的“宇宙”大很多倍。他的假设是,固定恒星和太阳保持不动,地球绕着太阳在一个圆的圆周上旋转,太阳位于轨道的中间,固定恒星的球体与太阳位于同一中心,它是如此之大,以至于他认为地球旋转的圆与固定恒星的距离成比例,就像球体的中心与它的表面成比例一样。

—— 《数沙者》

哥白尼在早期(未出版)的《天体运行论》手稿(现今还存在)中引用了萨摩斯的阿里斯塔克斯的观点,尽管他从最终出版的手稿中删除了参考文献。[60]

哥白尼可能意识到毕达哥拉斯的系统涉及一个移动的地球。毕达哥拉斯体系是被亚里士多德提出的。[61]

哥白尼拥有一份乔治·瓦拉的De expetendis et fugiendis rebus(《预期和逃避》)的副本,其中包括普鲁塔克对阿利斯塔克日静止的引用的翻译。[62]

哥白尼在《天体运行论》中对教皇保罗三世的奉献中——哥白尼希望通过“喋喋不休....对天文学一无所知”来减少他对日心说的批判——这本书的作者写道,在重读所有哲学的时候,在西塞罗和普鲁塔克的书中,他找到了那些“违背天文学家的传统观点,几乎违背常识”的认为地球是移动的少数思想家的文献。

从10世纪开始,在伊斯兰天文学中发展了一种批评托勒密的传统,这种传统在巴士拉的 Al-Shukūk 'alā Baṭalamiyūs(关于托勒密的疑问)的伊本·海塞姆达到顶峰。[63]几位伊斯兰天文学家质疑地球明显的不动性[64][65]和在宇宙中的中心地位。[66]一些人认为地球绕着它的轴旋转,例如阿布·赛义德·阿勒西吉齐(Abu Sa'id al-Sijzi,d. c. 1020)。[67][68]根据阿尔·比鲁尼的说法,阿勒西吉齐根据他的一些同时代人的观点“我们看到的运动是源于地球的运动而不是天空的运动”发明了一个星盘。[68][69]除了阿勒西吉齐之外,其他人也持有这种观点,这一点通过十三世纪阿拉伯文作品的参考文献得到了进一步证实,该文献指出:

根据几何学家或工程师(muhandisī n)的说法,地球一直在做圆周运动,这看似是天空的运动实际上是由于地球的运动,而不是恒星的运动。[68]

在12世纪,努尔·阿德·丁·阿尔比鲁吉提出了托勒密体系的一个完整的替代方案(尽管不是日心说)。[70][71]他宣称托勒密系统是一个虚构的模型,成功预测了行星位置,但不是真实的或实际存在的。[71]13世纪,阿尔·比鲁吉的替代体系在欧洲大部分地区传播开来,对他的观点的争论和驳斥一直持续到16世纪。[71]
\begin{figure}[ht]
\centering
\includegraphics[width=10cm]{./figures/907cdec3036ba967.png}
\caption{图斯双圆} \label{fig_GBN_20}
\end{figure}
由于行星运动的地心模型与哥白尼后来在日心说模型中使用的模型非常相似,数学技术在13至14世纪由莫·阿耶杜丁·乌尔迪、纳西尔·丁·阿尔图斯和伊本·沙特尔发展扩大。[72]哥白尼在同一个行星模型中使用了现在被称为乌尔迪引理和图斯双圆的东西,就像在阿拉伯文献中发现的一样。[73]此外,在大马士革的伊本·沙特尔(公元1375年)的早期著作中发现了哥白尼在《短论》中使用的两个本轮的精确代替。[74]伊本·沙特尔的月球和水星模型也与哥白尼的相同。这导致一些学者认为哥白尼肯定接触到了一些关于那些早期天文学家想法的尚未被确认的工作。[75]由于图斯双圆被哥白尼用于他对数学天文学的重新表述,许多学者认为他在某种程度上意识到了这个想法。有人认为图斯双圆的思想可能已经传播到了欧洲,但是几乎没有留下手稿的痕迹,因为还没有任何的阿拉伯文本翻译成拉丁语文本。[76][77]一种可能的传播途径可能是拜占庭科学,它将阿尔图斯的一些作品从阿拉伯语翻译成拜占庭希腊语。意大利仍然保存着几份包含图斯双圆的拜占庭希腊手稿。[78]其他学者认为哥白尼可以独立于晚期伊斯兰传统发展这些思想。[79]然而,哥白尼在《天体运行论》中引用了一些伊斯兰天文学家的理论和观察,这些天文学家是阿尔·白塔尼、塔比特·伊本·库拉、扎卡利、阿威洛依和阿尔·比鲁吉。[80]

在尼拉坎塔·索马亚吉(1444-1544)的《阿耶波多历算书》中有一个评论,为部分日心说的行星模型开发了一个计算系统,其中行星围绕太阳运行,太阳又围绕地球运行,类似于第谷·布拉尼后来在16世纪后期提出的第谷星系。在坦陀罗山伽(1500年),他进一步修正了他的行星系统,这在数学上比第谷和哥白尼模型更精确地预测了内部行星的日心轨道。

哥白尼在世时在欧洲盛行的理论是托勒密在他的《天文学大成》(大约公元后150年)中发表的理论;地球是宇宙的静止中心。恒星被嵌在一个大的外球面中,这个外球面几乎每天都在快速旋转,而行星、太阳和月球则被嵌在它们自己较小的球体中。托勒密的系统使用了包括本轮、均轮和等径在内的理论来解释这些天体的路径不同于以地球为中心的简单圆形轨道的观察结果。[81]

\subsubsection{2.2 哥白尼}
\begin{figure}[ht]
\centering
\includegraphics[width=6cm]{./figures/7cc288b2a24866bb.png}
\caption{幸存的亲笔签名手稿。} \label{fig_GBN_21}
\end{figure}
\begin{figure}[ht]
\centering
\includegraphics[width=6cm]{./figures/f25a8e4524ea52d3.png}
\caption{第一版印刷版。} \label{fig_GBN_22}
\end{figure}
哥白尼的太阳系日心说理论原理图来自于《天体运行论》[82]

哥白尼关于日心说的主要著作是Dē revolutionibus orbium coelestium(《天体运行论》),发表于他逝世的那一年,1543年。他在1510年已经提出了他的理论。“他写下了一份关于他的新的庄重安排的简短概述,大概也是在1510年,他发给了除了瓦尔米亚以外的至少一名记者。那个人又复制了文件以供进一步分发,想必新的接收者也是这样做的……”[83]

哥白尼的《短论》总结了他的日心说。它列出了该理论所基于的“假设”,如下所示:[84]
\begin{enumerate}
\item 所有天球没有一个中心[85]或者球体。[86]
\item 地球的中心不是宇宙的中心,而是重物体移动的中心和月球球体的中心。
\item 所有的球体都围绕着太阳,就好像它在它们中间一样,因此宇宙的中心靠近太阳。
\item 地球与太阳的距离同苍穹的高度(包含恒星的最外层天体)之比远小于地球半径与太阳距离的比例,即地球到太阳的距离是很小的,以至于与穹苍的高度相比是难以察觉的。
\item 无论天空中出现什么运动,都不是来自天空的任何运动,而是来自地球的运动。地球及其周围的元素在每天的运动中在其固定的极点上完成一次完整的旋转,而天空和最高的天空保持不变。
\item 在我们看来,太阳的运动不是来自它的运动,而是来自地球和我们的球体的运动,我们像其他任何行星一样围绕太阳旋转。那么,地球有不止一个运动。
\item 行星明显的向后和向前运动不是来自它们的运动,而是来自地球的运动。因此,仅仅地球的运动就足以解释天空中如此多明显的不同现象。
\end{enumerate}

《天体运行论》本身被分为六个部分,称为“卷”:[87]
\begin{enumerate}
\item 日心说的总体观点,以及对他的世界观的概述
\item 主要从理论的层面上,介绍了球面天文学的原理和一系列恒星(作为后续书籍中提出的论点的基础)
\item 主要研究太阳的表观运动和相关现象
\item 对月球及其轨道运动的描述
\item 非类地行星经度运动的解释
\item 非类地行星纬度运动的解释
\end{enumerate}

\subsubsection{2.3 继任者}
\begin{figure}[ht]
\centering
\includegraphics[width=8cm]{./figures/b72cf910244cbfed.png}
\caption{哥白尼遗体灵柩于2006年展出Olsztyn} \label{fig_GBN_23}
\end{figure}
乔治·约阿希姆·雷蒂库斯本可以成为哥白尼的继任者,但却没能赶上这个时机。[88]伊拉兹马斯·赖因霍尔德本可以成为他的继任者,但过早去世了。[88]第一个伟大的继任者是第谷·布拉尼[88](尽管他不认为地球绕太阳运行),其次是约翰尼斯·开普勒,[88]他曾在布拉格与第谷合作,并受益于第谷数十年的详细观测数据。[88]

尽管后来日心说几乎被普遍接受(尽管不是本轮或圆形轨道),但是哥白尼的理论最初并不流行。学者们认为,《天体运行论》出版六十年后,全欧洲只有大约15名天文学家支持哥白尼学说:英国的托马斯·迪格斯和托马斯·哈里奥特;意大利的乔尔丹诺·布鲁诺和伽利略·伽利雷;西班牙的迭戈·祖尼加;低地国家的西蒙·斯蒂文;而在德国,最大的群体——乔治·约阿希姆·雷蒂库斯、迈克尔·梅斯特林、克里斯托夫·罗斯曼(后来可能已经放弃)[89]和约翰尼斯·开普勒。其他的可能追随者包括英国人威廉·吉尔伯特还有阿喀琉斯·加塞、乔治·沃格林、瓦伦丁·奥托和蒂德曼·吉斯。[89]

亚瑟·库斯勒在他的畅销书《梦游者》中断言哥白尼的书在第一次出版时没有被广泛阅读。[90]爱德华·罗森尖锐地批评了这一说法,欧文·金格里奇也断然否定了这一说法,他检查了前两个版本中几乎所有幸存下来的副本,并在其中许多版本中发现了作者的大量旁注。金格里奇于2004年在《无人问津之书》一书中发表了他的结论。[91]

当时的知识氛围仍然由亚里士多德哲学和相应的托勒密天文学主导。当时没有理由接受哥白尼理论,除了它的数学简单性(通过避免使用等量来确定行星位置)。第谷·布拉尼的系统(地球静止不动,太阳围绕地球旋转,其他行星围绕太阳旋转)也直接与哥白尼的理论竞争。直到半个世纪后,随着开普勒和伽利略的研究才出现了一些实质性的证据来证明哥白尼学说,从“伽利略阐述惯性原理的时候开始……这个理论有助于解释运动中的物体为什么不会从地球上掉落。”直到艾萨克·牛顿提出万有引力定律和力学定律后,统一了地面力学和天体力学,日心说的观点才被普遍接受。

\subsection{争论}