% 环上的多项式
% 多项式环|长除法

\pentry{一元多项式\upref{OnePol},欧几里得环\upref{EuRing}}

\footnote{本小节节选自《小时百科教材系列》的《代数学》.}
我们知道,函数是一种映射,特指“值域是数字集合”的映射.这里的“数字集合”,通常指任何一个环,换句话说,只要是个环,其元素都可以被视为“数字”.我们熟悉的整数环、有理数域\footnote{域是一类特殊的环.}、实数域、复数域等都是很好的例子.

多项式就是一种极为重要的函数.在微积分中,性质良好的函数(解析函数)都可以被表示为一列多项式函数的极限,或者说总可以用一个多项式函数来逼近它.而多项式的性质较为简单,求导也很容易.我们现在讨论的是代数,所以就不关心可以怎么用求导啊积分啊等手段去处理多项式函数,而是关心这个概念可以怎么在代数上拓展.

先来观察一下我们熟知的多项式吧.作为实变量函数,一个多项式$f(x)$可以表示为:

\begin{equation}
    f(x) = \sum_{i=0}^N a_i x^i
\end{equation}
其中各$a_i$都是实数,而$x^i$是用来抽象表示“任意自变量”的.我们可以给$x$赋值,比如取一个实数$c$,然后令$x=c$,这样就能得到一个\textbf{实数}$f(c)=\sum_{i=0}^N a_i c^i$.但要是不赋值,那

























