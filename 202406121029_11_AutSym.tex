% 有限对称群的性质
% license Usr
% type Tutor

\begin{definition}{}
阶数为1或2的置换,称为\textbf{对合变换(involution)。}
\end{definition}
\begin{theorem}{}
任意有限置换群都可以表示为两个对合变换的复合。
\end{theorem}
\textbf{证明:}
我们首先证明,循环置换可以分解为两个对合变换的复合。

一个n元循环置换可以看作n边形上的旋转,而我们知道,二维空间上的旋转可以分解为两个反射,只要保证两次反射轴的夹角是旋转角度的$1/2$,对合变换就是这种反射变换的置换表示。以正五边形为例,该过程如下所示:
\begin{figure}[ht]
\centering
\includegraphics[width=14cm]{./figures/f32c9320160af59c.png}
\caption{} \label{fig_AutSym_2}
\end{figure}
该循环的分解写作$(12345)=[(23)(14)][(13)(45)]$,显然,$[(13)(45)]$与$[(23)(14)]$就是图中所示的“反射变换”。

由于$n$元置换群总可以拆分成不相交的循环乘积,而每个循环都可以拆成对合之积,则这些对合可以重新组合成两组。如设某置换群可拆分成三个不相交循环之积,用$\sigma$表示对合变换,且下标首字母不同代表不同循环,则有:
\begin{equation}
\begin{aligned}
f=f_1f_2f_3&=[\sigma_{11}\sigma_{12}][\sigma_{21}\sigma_{22}][\sigma_{31}\sigma_{32}]\\
&=[\sigma_{11}\sigma_{21}\sigma_{31}][\sigma_{12}\sigma_{22}\sigma_{32}]\\
&=\sigma'_1\sigma'_2~.
\end{aligned}
\end{equation}
得证。

