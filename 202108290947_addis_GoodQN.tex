% 好量子数

\begin{issues}
\issueOther{举例子}
\end{issues}

\pentry{量子力学的基本假设\upref{QMPos}}

我们知道量子力学中, 可测量的物理量对应厄米算符 $Q$, 本征方程为
\begin{equation}\label{GoodQN_eq1}
Q \ket{\psi_i} = q_i \ket{\psi_i}
\end{equation}

\begin{definition}{}
如果厄米算符 $Q$ 和哈密顿算符 $H$ 对易, 即 $[H, Q] = 0$, 那么 $Q$ 的本征值(或其编号)就被称为\textbf{好量子数(good quantum number)}.
\end{definition}

$[H, Q] = 0$ 说明 $Q$ 是一个守恒量\upref{QMcons}, 也说明两个算符具有一组共同的正交归一的完备本征矢\upref{Commut}.

“好量子数” 这个词常用于不含时微扰\upref{TIPT}理论中的简并情况. 在简并子空间中, 微扰算符的本征值(或其编号)被称为好量子数. 但这只是一种近似, 因为微扰哈密顿 $H'$ 和哈密顿 $H$ 未必对易.
