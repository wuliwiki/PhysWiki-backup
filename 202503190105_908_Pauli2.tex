% 沃尔夫冈·泡利(综述)
% license CCBYSA3
% type Wiki

本文根据 CC-BY-SA 协议转载翻译自维基百科\href{https://en.wikipedia.org/wiki/Wolfgang_Pauli}{相关文章}。

\begin{figure}[ht]
\centering
\includegraphics[width=6cm]{./figures/7d0315d83d1af784.png}
\caption{泡利于1945年} \label{fig_Pauli2_1}
\end{figure}
沃尔夫冈·恩斯特·泡利(Wolfgang Ernst Pauli,/ˈpɔːli/,[5] 德语:[ˈvɔlfɡaŋ ˈpaʊli];1900年4月25日—1958年12月15日)是一位奥地利物理学家,量子力学的先驱。1945年,在阿尔伯特·爱因斯坦的提名下,[6] 泡利因其“通过发现一条新的自然法则——泡利不相容原理(Pauli Exclusion Principle)所做出的决定性贡献”而获得诺贝尔物理学奖。这一发现涉及自旋理论,该理论是物质结构理论的基础。为了保持\(\beta\)衰变中的能量守恒,他提出了一种质量极小、电中性的粒子,其后被恩里科·费米命名为中微子。该粒子最终于1956年被探测到。

泡利就读于维也纳的多布林格文理中学,并于1918年以优异成绩毕业。两个月后,他发表了第一篇论文,内容涉及阿尔伯特·爱因斯坦的广义相对论。随后,他进入慕尼黑大学学习,并在阿诺德·索末菲指导下开展研究。[1]1921年7月,他因其关于电离双原子氢(\(H_2^+\))的量子理论的论文获得博士学位。[2][9]
\subsection{职业生涯}  
索末菲邀请泡利为《数学科学百科全书》撰写相对论综述。在获得博士学位后的两个月内,泡利便完成了这篇长达 237 页的文章。爱因斯坦对此给予高度评价;该文章后来以专著形式出版,并至今仍是该领域的重要参考文献。[10]
\begin{figure}[ht]
\centering
\includegraphics[width=6cm]{./figures/5713325387bd5c08.png}
\caption{沃尔夫冈·泡利用于讲授(1929年)} \label{fig_Pauli2_2}
\end{figure}
泡利曾在哥廷根大学担任马克斯·玻恩的助理一年,随后一年在哥本哈根理论物理研究所(后来的 尼尔斯·玻尔研究所)工作。[1] 从1923年到1928年,他在汉堡大学任教。[11]在此期间,泡利对现代量子力学理论的发展发挥了关键作用,特别是提出了泡利不相容原理以及非相对论性自旋理论。

1928年,泡利被任命为瑞士苏黎世联邦理工学院的理论物理教授。[1]1930年,他获得洛伦兹奖章。[12] 他还曾在1931 年担任密歇根大学客座教授,并于1935年在普林斯顿高等研究院担任访问教授。
\subsubsection{荣格}  
1930年底,就在提出中微子假说后不久,泡利经历了一场个人危机,当时他刚刚离婚,并且他的母亲自杀。1932年1月,他向居住在苏黎世附近的心理学家和心理治疗师卡尔·荣格寻求帮助。荣格立刻开始分析泡利充满原型意象的梦境,并将其纳入自己的研究。泡利随后从科学角度批判性地探讨了荣格理论的认识论基础,这些讨论帮助澄清了荣格的某些概念,尤其是\textbf{共时性}的理论。他们的讨论记录保存在泡利—荣格通信中,后来以《原子与原型》出版。而荣格对泡利400多个梦境的详细分析,则记录在《心理学与炼金术》中。1933 年,泡利出版了他的物理学著作《物理手册》的第二部分,该书被认为是量子物理学领域的权威著作。罗伯特·奥本海默称其为“唯一一本真正成熟的量子力学入门书”。[13]

1938年德国吞并奥地利后,泡利自动成为德国公民,这在1939年第二次世界大战爆发后对他来说成了一个问题。1940年,他曾试图获得瑞士公民身份,以便能够继续留在苏黎世联邦理工学院(ETH),但最终未能成功。[14]
\subsubsection{美国与瑞士} 
1940年,泡利移居美国*,在普林斯顿高等研究院担任理论物理教授。1946 年,战后,他获得美国国籍,随后回到苏黎世,并在那里度过了余生的大部分时间。1949 年,他获得了瑞士公民身份。

1958年,泡利被授予马克斯·普朗克奖章。同年,他被诊断出患有胰腺癌。当他的最后一位助手查尔斯·恩茨在苏黎世红十字医院探望他时,泡利问道:“你看到房间号了吗?” 这是137号房间。泡利一生都在思考精细结构常数的问题,这个无量纲基本常数的值接近 1/137。[15]1958 年12月15日,泡利在那个房间里去世。[16][17]
\subsection{科学研究}
泡利作为物理学家做出了许多重要贡献,主要集中在量子力学领域。然而,他很少发表论文,而更喜欢通过书信与同行交流,如丹麦哥本哈根大学的尼尔斯·玻尔和维尔纳·海森堡,两人都是他亲密的朋友。他的许多想法和研究成果从未正式发表,而只是出现在他的信件中,这些信件常被收件人复制和传播。1921年,泡利与玻尔共同提出了构造原理,该原理描述了电子按照壳层填充的方式构建原子。该名称源自德语“Aufbau”(意为“构建”),因为玻尔同样精通德语。

泡利于1924年提出了一种新的量子自由度(或量子数),其具有两个可能的取值,以解决实验观测到的分子光谱与当时正在发展的量子力学理论之间的不一致性。他提出了泡利不相容原理,这或许是他最重要的贡献。该原理指出,任何两个电子都不能处于相同的量子态,这一态由四个量子数确定,其中包括他所提出的新的二值自由度。电子自旋的概念最初由拉尔夫·克罗尼希提出。一年后,乔治·乌伦贝克和塞缪尔·古兹密特确认了泡利提出的这一自由度即为电子自旋。然而,泡利本人长期拒绝接受这一解释。[18]  

1926 年,在海森堡发表矩阵力学版的现代量子力学后不久,泡利利用该理论推导出了氢原子光谱的观测结果。这一成果对于确立海森堡理论的可信度起到了关键作用。
\begin{figure}[ht]
\centering
\includegraphics[width=6cm]{./figures/b80ac58f66886779.png}
\caption{保罗·狄拉克、沃尔夫冈·泡利与鲁道夫·派尔斯,约 1953 年} \label{fig_Pauli2_3}
\end{figure}
泡利引入了\(2\times 2\)泡利矩阵作为自旋算符的基,从而解决了非相对论性自旋理论的问题。这项工作,包括泡利方程,有时被认为影响了保罗·狄拉克 在建立相对论电子的狄拉克方程时的思路。然而,狄拉克本人表示,他当时是独立地推导出了类似的矩阵。后来,狄拉克在其相对论性费米子自旋理论中,引入了类似但维度更大的(\(4\times4\))自旋矩阵。

1930年,泡利研究\(\beta\)衰变问题。在12月4日致莉泽·迈特纳等人的信中,他以“亲爱的放射性女士们和先生们”开头,并提出了一种尚未被观测到的中性粒子的假设,该粒子的质量极小,不超过质子质量的 1\%,以解释\(\beta\)衰变的连续能谱。1934年,恩里科·费米在其\(\beta\)衰变理论中引入了这一粒子,并将其命名为“neutrino”(意大利语,意为“小中性粒子”)。1956年,弗雷德里克·雷因斯和克莱德·考恩实验上首次确认了中微子的存在,距泡利去世还有两年半。收到消息后,泡利通过电报回应道:“感谢消息。一切都会降临于懂得等待的人。——泡利”。[19]

1940年,泡利重新推导了\textbf{自旋-统计定理},这是量子场论中的一个关键结论。该定理表明:具有半整数自旋的粒子是费米子,而具有整数自旋的粒子是玻色子。

1949年,泡利发表了一篇关于泡利-维拉斯\textbf{重整化}的论文。重整化是指一种数学技术,在计算过程中对无穷的积分进行修正,使其变为有限值,从而确定理论中本质上无穷的物理量(如质量、电荷、波函数)是否能被重新定义,使其变为有限且可计算的一组量,并通过实验值加以校正。这一过程被称为重整化,它不仅消除了量子场论中的无穷大问题,还使得微扰理论中的高阶修正计算成为可能。

泡利多次批评现代进化生物学的综合理论,[20][21]他的一些当代支持者认为表观遗传继承可以支持他当初的论点。[22]

1900年,保罗·德鲁德提出了经典电子在金属固体中运动的第一个理论模型。后来,泡利和其他物理学家对德鲁德的经典模型进行了改进。泡利意识到,金属中的自由电子必须遵守费米–狄拉克统计。基于这一概念,他于1926年提出了顺磁性理论。泡利曾戏称:“Festkörperphysik ist eine Schmutzphysik.”——“固体物理是肮脏的物理。”[23]

1953年,泡利被选为英国皇家学会外籍院士,并在1955年至1957年担任瑞士物理学会主席。[1]1958年,他成为荷兰皇家艺术与科学学院的外籍院士。[24]
\subsection{性格与友谊}
\begin{figure}[ht]
\centering
\includegraphics[width=6cm]{./figures/350bd1a45c23890a.png}
\caption{沃尔夫冈·泡利,约 1924 年} \label{fig_Pauli2_4}
\end{figure}
\textbf{泡利效应}得名于他的一种传闻中的奇异能力——只要他靠近实验设备,这些设备就可能发生故障。泡利本人也意识到自己的这种“名声”,并在泡利效应显现时感到十分愉悦。这些奇怪的现象与他对超心理学的研究相呼应,尤其是他与卡尔·荣格合作探讨的共时性理论。[25]马克斯·玻恩评价泡利:“他唯一能与之相提并论的只有爱因斯坦……甚至可能比爱因斯坦更伟大。” 爱因斯坦则称泡利为自己的“精神继承人”。[26]  

泡利以极端的完美主义著称,这不仅体现在他的个人研究中,也体现在他对同行的工作的严格要求。因此,他在物理学界被誉为“物理学的良知”,是所有同事都需要面对的最严厉的批评者。对于他认为不完善的理论,他常常毫不留情地批评,并用“ganz falsch”(德语,意为“彻底错误”)来形容它。

但这并不是泡利最严厉的批评。他对那些表达晦涩、不清楚到无法检验或评估的理论或论文更为不屑,认为它们伪装成科学,但实际上并不属于科学的范畴。他认为这些理论比错误更糟糕,因为它们甚至无法被证明是错误的。他曾对一篇晦涩难懂的论文发表过著名的评论:“这甚至都算不上错误!”(It is not even wrong!)[1]  

泡利的傲慢性格,从他与另一位著名物理学家保罗·埃伦费斯特初次见面的对话中可见一斑。两人在一次会议上第一次见面。埃伦费斯特早已读过泡利的论文,并对其印象深刻。在交谈了几分钟后,埃伦费斯特说道:“我觉得,我更喜欢你的那篇百科全书文章(关于相对论的综述),胜过喜欢你本人。”泡利随即反驳道:“这可真奇怪,对我来说,恰好相反。”[27]尽管初次见面时交锋激烈,两人后来却成了非常要好的朋友。

从这则故事中,我们可以看到泡利性格中较为温和的一面,该故事出现在关于狄拉克的文章中:  

维尔纳·海森堡在《物理学与超越》(Physics and Beyond, 1971)一书中回忆了1927年索尔维会议上一群年轻参会者的友好交谈,讨论了爱因斯坦和普朗克对宗教的看法。参与讨论的有 沃尔夫冈·泡利、海森堡,以及保罗·狄拉克。狄拉克发表了一番尖锐而清晰的批评,抨击宗教被政治操纵。当海森堡事后将狄拉克的这番话转述给玻尔(Bohr)时,玻尔对其逻辑的清晰性感到十分赞赏。狄拉克在讨论中说道:“我不明白为什么我们要浪费时间讨论宗教。如果我们是诚实的——而作为科学家,诚实正是我们的责任——我们就不能不承认,任何宗教都是一堆毫无真实基础的虚假言论。‘上帝’这一概念完全是人类想象的产物。[……] 我不认同任何宗教神话,至少因为它们彼此矛盾。[……]” 海森堡的态度较为宽容,而泡利在最初发表了一些简短的看法后,便保持沉默。然而,当大家最终询问他的意见时,他开玩笑地说道:“嗯,我想我们的朋友狄拉克也有自己的宗教,而这个宗教的第一条诫命就是:‘上帝不存在,而保罗·狄拉克是他的先知。’” 听到这番话,在场所有人都大笑起来,包括狄拉克自己。[28]

泡利的许多想法和研究成果从未正式发表,而只是出现在他的信件中,这些信件经常被收件人复制并传阅。泡利本人或许并不在意自己的许多工作因此未被署名,但当涉及到海森堡1958年在哥廷根发表的世界知名讲座(探讨他们共同研究的统一场论)时,泡利的态度发生了变化。尤其是新闻稿中将他仅仅称为“海森堡教授的助手”,这让他深感不满,并愤怒地批评海森堡的物理能力。由于这场争议,两人的关系逐渐恶化,最终导致海森堡无视泡利的葬礼,并在自传中写道,泡利的批评过于夸张。然而,最终这一统一场论被证明是不可行的,这也证实了泡利的批评是正确的[29]  

乔治·伽莫夫曾评价道:“要找到一门不使用泡利不相容原理的现代物理学分支,几乎和要找到一个像沃尔夫冈·泡利那样天赋异禀、风趣迷人的人一样困难。”[30]
\subsection{哲学}  
在与卡尔·荣格的讨论中,泡利提出了一种\textbf{本体论理论},后来被称为“泡利–荣格猜想”,被认为是一种双重视角理论。该理论认为,存在一个“心物同源的中性现实”,而心理和物理现象都是从这种更深层的现实中派生出来的。[31]泡利认为,量子物理学的一些元素指向一种更深层的现实,这可能解释心灵与物质之间的鸿沟。他写道:“我们必须假设,在我们的控制范围之外,存在一种宇宙自然秩序,而外在的物质世界和内在的心理意象都服从于它。”[32]  

泡利和荣格都认为,这一现实受某些共同原则(“原型”)所支配,这些原则既表现为心理现象,也表现为物理事件。[33]他们还认为,\textbf{共时性}可能揭示这一潜在现实的某些运作机制。[33][32]  
\subsection{信仰} 
泡利被认为是一位自然神论者和神秘主义者。在《没有时间啰嗦:沃尔夫冈·泡利的科学传记》一书中,他被引用在给科学史学家施缪尔·桑伯斯基的信中写道:“与一神教宗教相对立——但与所有民族的神秘主义(包括犹太神秘主义)一致——我认为,终极现实并非是人格化的存在。”[34][35]
\subsection{个人生活}
\begin{figure}[ht]
\centering
\includegraphics[width=6cm]{./figures/6632be409200d93c.png}
\caption{沃尔夫冈·泡利半身像(1962年)} \label{fig_Pauli2_5}
\end{figure}
1929年,泡利与歌舞表演者凯特·玛格丽特·德普纳结婚。[36]这段婚姻并不幸福,不到一年便以离婚告终。1934年,他再次结婚,妻子是弗朗齐丝卡·贝特朗(Franziska Bertram, 1901–1987)。他们没有子女。  
\subsection{去世}  
1958年12月15日,泡利因胰腺癌去世,享年58岁。[16][17]
\subsection{出版作品}  
\begin{itemize}
\item 泡利 W.《量子力学的一般原理》,Springer,1980 年。  
\item 泡利 W.《物理学讲义》,6 卷本,Dover,2000 年。\\  
  第 1 卷:电动力学\\  
  第 2 卷:光学与电子理论\\  
  第 3 卷:热力学与气体动理论\\  
  第 4 卷:统计力学 \\
  第 5 卷:波动力学\\
  第 6 卷:场量子化中的选题 
\item 泡利 W.《核力的介子理论》,第 2 版,Interscience Publishers,1948 年。  
\item 泡利 W.《相对论》,Dover,1981 年。  
\end{itemize}

\subsection{书目}  
\begin{itemize}
\item 泡利 W., 荣格 C. G.(1955 年):《自然与心灵的解读》,Ishi Press,ISBN 978-4-87187-713-8。  
\item 泡利 W., 荣格 C. G.(2001 年):C. A. Meier(编),《原子与原型:泡利/荣格书信集(1932–1958)》,普林斯顿大学出版社,ISBN 978-0-691012-07-0。  
\end{itemize}
\subsection{参见}  
\begin{itemize}
\item 犹太诺贝尔奖得主名单
\end{itemize}
\subsection{参考文献} 
\begin{enumerate}
\item 派尔斯,鲁道夫(1960)。《沃尔夫冈·恩斯特·泡利(1900–1958)》,《英国皇家学会院士传记回忆录》,第 6 卷,皇家学会(:174–192。doi:10.1098/rsbm.1960.0014,S2CID 62478251。  
\item 沃尔夫冈·泡利在数学系谱计划中的条目。  
\item "马克斯·玻恩"。马克斯·玻恩研究所,检索于 2020 年 11 月 9 日。1922 年……沃尔夫冈·泡利和维尔纳·海森堡是马克斯·玻恩的研究助理。  
\item 布朗,杰拉尔德·E.,李昌焕(2006):《汉斯·贝特及其物理学》,世界科学出版社,ISBN 978-981-256-610-2,第 338 页。  
\item "泡利",《兰登书屋韦氏未删节词典》。  
\item "提名数据库:沃尔夫冈·泡利"。诺贝尔基金会,检索于 2015 年 11 月 17 日。  
\item 恩斯特·马赫与沃尔夫冈·泡利的祖先在布拉格的历史。  
\item "犹太物理学家",检索于 2006 年 9 月 30 日。  
\item 泡利,沃尔夫冈·恩斯特(1921)。《关于氢分子离子模型的研究》,博士论文,慕尼黑大学。  
\item 泡利,W.(1926)。《相对论理论》,克莱因百科全书,第 19 卷,可通过互联网档案馆访问。
\item "汉堡大学与 DESY 共同创立沃尔夫冈·泡利理论物理中心",DESY 汉堡(德语),2013 年 5 月,检索于 2022 年 2 月 14 日。该中心以1923 年至 1928 年在汉堡任教的诺贝尔物理学奖得主泡利命名。  
\item "沃尔夫冈·泡利 – 传记",诺贝尔奖官方网站,检索于 2021 年 6 月 29 日。  
\item 冯·梅恩(2001 年 2 月 1 日):*"沃尔夫冈·泡利",《今日物理》(Physics Today),第 54 卷(2 期):43–48。Bibcode:2001PhT....54b..43M,doi:10.1063/1.1359709。  
\item 查尔斯·保罗·恩茨:《没有时间啰嗦:沃尔夫冈·泡利的科学传记》,2002 年首次出版,2004 年再版,ISBN 978-0-19-856479-9,第 338 页。  
\item M.A. Sherbon:"沃尔夫冈·泡利与精细结构常数",《科学期刊》,第 2 卷,第 3 期,页 148–154(2012 年)。  
\item “由于一种‘卡巴拉式’的巧合,沃尔夫冈·泡利于 1958 年 12 月 15 日在苏黎世红十字医院的 137 号房间去世。” —— 《心灵与精神:查尔斯·恩茨精选论文集》,查尔斯·保罗·恩茨,世界科学出版社,2009 年,ISBN 978-981-281-900-0,第 95 页。  
\item 查尔斯·P·恩茨:"缅怀沃尔夫冈·泡利(1900–1958)",《瑞士物理学报》。  
\item S.A. 古兹密特J.H. 范德华尔斯 译:"电子自旋的发现"。  
\item 查尔斯·恩茨,冯·梅恩(1994 年):"沃尔夫冈·泡利:传记导论",《物理学与哲学文集》,Springer-Verlag,第 19 页。  
\item 泡利,W.(1954 年):"关于无意识观念的科学与认识论方面",《辩证法》,第 8 卷(4 期):283–301。doi:10.1111/j.1746-8361.1954.tb01265.x。  
\item H. Atmanspacher,H. Primas(2006 年):"泡利关于心智与物质的观点在当代科学背景下的意义",《意识研究期刊》,第 13 卷(3 期):5–50。原始 PDF 文件存档于 2009 年 3 月 19 日,检索于 2009 年 2 月 12 日。  
\item "沃尔夫冈·泡利的哲学思想与当代科学会议",由 ETH 组织,于 2007 年 5 月 20–25 日 举行,存档于 Wayback Machine(2014 年 9 月 10 日)。讨论该主题的理查德·约根森论文摘要见 [1],存档于 2015 年 9 月 24 日。
\item Natelson, Douglas(2018 年 12 月 19 日):"评论:凝聚态物理的形象问题",《今日物理》,第 12 期,AIP Publishing:30800。Bibcode:2018PhT..2018l0800N,doi:10.1063/pt.6.3.20181219a,ISSN 1945-0699。  
\item "沃尔夫冈·恩斯特·泡利(1900–1958)",荷兰皇家艺术与科学学院,检索于 2015 年 7 月 26 日。  
\item 哈拉尔德·阿特曼斯帕赫,汉斯·普里马斯(1996 年):"沃尔夫冈·泡利的隐藏面:一位杰出物理学家与深度心理学的非凡相遇",《意识研究期刊》,第 3 期:112–126。  
\item 施库金(2001 年 7 月 30 日):"沃尔夫冈·泡利"。  
\item 奥斯卡·克莱因,引自 贾格迪什·梅赫拉,赫尔穆特·雷琴贝格(Helmut Rechenberg)**(2000 年):*"量子理论的历史发展"*(*"The Historical Development of Quantum Theory"*),Springer,第 488 页,ISBN 978-0-387-95175-1。  
\item 维尔纳·海森堡(Werner Heisenberg)**(1971 年):*"物理学与超越:邂逅与对话"*(*"Physics and Beyond: Encounters and Conversations"*),Harper and Row,第 87 页,ISBN 978-0-06-131622-7。  
\item 阿瑟·I·米勒(Arthur I. Miller)**(2009 年 12 月 10 日):*"泡利与荣格的奇特友谊 – 第 6 部分"*(*"The strange friendship of Pauli and Jung – Part 6"*),CERN,伦敦大学学院(University College London),时长 4–6:00, 8:10–8:50。存档于 2021 年 11 月 17 日。提到泡利最愤怒的一则新闻稿写道:**"海森堡教授及其助手 W. 泡利……"**  
\item 乔治·伽莫夫(George Gamow)**(1966 年):*"震撼物理学的三十年"*(*"Thirty Years That Shook Physics"*),第 79 页。  
\item 哈拉尔德·阿特曼斯帕赫(Harald Atmanspacher)**(2020 年 1 月 1 日):*"泡利–荣格猜想及其相关理论:一种形式上的扩展概述"*(*"The Pauli–Jung Conjecture and Its Relatives: A Formally Augmented Outline"*),《开放哲学》(Open Philosophy),第 3 卷(1 期):527–549。doi:10.1515/opphil-2020-0138,hdl:20.500.11850/448478,S2CID 222005552。
\end{enumerate}