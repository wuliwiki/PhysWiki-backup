% 单纯同调群的计算
% 复形|complex|同调|homology|群|群同态|三角剖分

\pentry{单纯剖分(三角剖分)\upref{Traglt},群同态\upref{Group2}}

\addTODO{未来可能添加更多shi'li}

对一个拓扑空间进行单纯剖分,也就是把这个空间用一个复形来表示.复形上的单纯同调群,就是这个拓扑空间的单纯同调群.本节讨论一些常见拓扑空间的单纯同调群计算,以及一些有助于计算的定理.

\subsubsection{连通复形上的零维单纯同调群}

考虑复形$K$,将其顶点的集合记为$\{a_i\}_{i=1}^n$,那么$K$的零维链群$C_0(K)$、同时也是零维闭链群$Z_0(K)$\footnote{因为零维的链必然闭链.},就是顶点集合的\textbf{自由生成阿贝尔群}.

如果一条零维链为$x_o=\sum m_ia_i$,其中$m_i\in\mathbb{Z}$,则称$\opn{ind} x_0=\sum m_i$为$x_0$的\textbf{指数(index)}.

用指数定义一个\textbf{群同态}$\epsilon: C_0(K)=Z_0(0)\to\mathbb{Z}$\footnote{这句的意思是,$\epsilon$既是$C_0(K)$上的映射,又是$Z_0(K)$上的映射,因为这两个群是一样的.},其中$\epsilon(x)=\opn{ind} x$.

\begin{lemma}{}\label{SHCal_lem1}
$B_0(K)=\opn{ker}\epsilon$.换句话说,$K$上的一条零维(闭)链,“它是边缘链”$\iff$“它的指数为零”.
\end{lemma}

\textbf{证明}:

$\Rightarrow$:一维链$\sum m_{ij}a_ia_j$的边缘就是$\partial\sum m_{ij}a_ia_j=\sum m_{ij}(a_j-a_i)$,其中$m_{ij}=m{ji}$.易得$\opn{ind}\sum m_{ij}(a_j-a_i)=0$.

一个简单的例子是,$a_0a_1+a_0a_2$的边缘就是$a_1-a_0+a_2-a_0$,其指数就是$1-1+1-1=0$.


$\Leftarrow$:如果一个零维链$\sum m_ia_i$指数为零,那么$\sum a_i=0$.我们可以把$m_ia_i$分为$\abs{m_i}$个不同的$\opn{sgn}(m_i)a_i$\footnote{$\opn{sgn}$是符号函数,比如对于$-2a_i$,就可以分成$2$个$-a_i$.},然后就可以一正一负两两配对,互相抵消掉,结果的指数就是零.

一个简单的例子是,$a_1+2a_2-3a_3$的指数为$1+2-3=0$,我们可以把它分成$a_1-a_3, a_2-a_3, a_2-a_3$,每一个都是边缘链,因此$a_1+2a_2-3a_3$是一个边缘链.

\textbf{证毕}.

由\autoref{SHCal_lem1} 和\textbf{群同态基本定理}(\autoref{Group2_exe1}~\upref{Group2}),$H_0(K)=Z_0(K)/B_0(K)=Z_0(K)/\opn{ker}\epsilon=\opn{Im}\epsilon=\mathbb{Z}$.



\subsubsection{圆柱面的单纯同调群}

将圆柱面进行三角剖分,如\autoref{SHCal_fig1} 所示.取有向单形基本组时,可以定义左图中的三角形都以逆时针为正方向.

\begin{figure}[ht]
\centering
\includegraphics[width=10cm]{./figures/SHCal_1.pdf}
\caption{圆柱面的三角剖分.两个图都是三角剖分的示意图,只不过右图将$a_1$和$a_4$对应合并了,此时不得不用曲线段来代替直线段.} \label{SHCal_fig1}
\end{figure}

记这个圆柱面的三角剖分为复形$K$.

由上一小节“连通复形上的零维单纯同调群”知,$H_0(K)=\mathbb{Z}$.

由于非零$2$维链的边缘链,总有$a_1a_2, a_2a_3, a_3a_2$(即\autoref{SHCal_fig1} 右边示意图中的“外边缘”)和/或$a_4a_5, a_5a_6, a_6a_4$(即\autoref{SHCal_fig1} 右边示意图中的“内边缘”)中的项,是无法被抵消掉的,因此$K$上\textbf{不存在}$2$\textbf{维的闭链}.于是,$H_2(K)=0$.

于是我们只需要计算$H_1(K)$.

\begin{lemma}{}\label{SHCal_lem2}
对于本小节所说的圆柱面剖分$K$,$C_1(K)$中的任一链都同调于$a_1a_2, a_2a_3, a_3a_2$(外边缘)、$a_4a_5, a_5a_6, a_6a_4$(内边缘)或者$a_1a_4$中的一个,或者这七个的某一线性组合.
\end{lemma}

\textbf{证明}:

以$a_1a_5$为例.由于同调群是商去了边缘链群的结果,因此“两条闭链同调”$\iff$“两条闭链的差是一个边缘链”.

计算单形$a_1a_2a_5$(看图辅助理解)的边缘后易得,$a_1a_5$和$a_1a_2+a_2a_5$同调\footnote{观察这一结果,可以总结出寻找圆柱面上同调的链的简便方法,即“掐头去尾”.比如这一例子里,$a_1a_2$的尾巴和$a_2a_5$的头部相同,都是$a_2$,那么我们就把它们首尾相连后去掉$a_2$,就得到$a_1a_5$.}.类似地,我们可以给出如下变换过程,每一步变换都保持变换前后的链是同调的:
\begin{equation}\label{SHCal_eq1}
\begin{aligned}
a_1a_5&\to a_1a_2+a_2a_5\\
&\to a_1a_2+a_6a_5+a_2a_6\\
&\to a_1a_2+a_6a_5+a_2a_3+a_3a_6\\&
\to a_1a_2+a_6a_5+a_2a_3+a_4a_6+a_3a_4\\
&\to a_1a_2+a_6a_5+a_2a_3+a_4a_6+a_3a_1+a_1a_4
\end{aligned}
\end{equation}
这就把$a_1a_5$变换为用引理中所说的那七个一维链的组合了.

所有的$a_ia_j$都可以用这个方法变换为那七个链的组合,而一切一维链条都是由$a_ia_j$组合而成的,从而得证.

\textbf{证毕}.

\begin{lemma}{}\label{SHCal_lem3}
\autoref{SHCal_lem2} 中的七个一维链所构成的\textbf{闭链},必然同调于$a_1a_2, a_2a_3, a_3a_1$组合而成的某个闭链.
\end{lemma}

\textbf{证明}:


根据\autoref{SHCal_lem1} ,任何闭链都可以表示为那七个链的组合,又因为是闭链,这种组合一定是$n_1(a_1a_2+a_2a_3+a_3a_1)+n_2(a_4a_5+a_5a_6+a_6a_4)$,其中各$n_i\in\mathbb{Z}$\footnote{注意,没有$a_1a_4$项.}.

仿照\autoref{SHCal_eq1} 的变换过程,容易证明$a_4a_5+a_5a_6+a_6a_4$同调于$a_1a_2+a_2a_3+a_3a_1$.因此任何闭链都可以同调于$(n_1+n_2)(a_1a_2+a_2a_3+a_3a_1)$,得证.


\textbf{证毕}.

\begin{corollary}{圆柱面的一维单纯同调群}
由\autoref{SHCal_lem2} 和\autoref{SHCal_lem3} 知,$K$的任何一维闭链都同调于$k\sum (a_1a_2+a_2a_3+a_3a_1)$,其中$k\in\mathbb{Z}$.类比\textbf{零维的单纯同调群}中“指数”以及用指数定义的群同态,我们也可以定义$K$中一维闭链群到整数群的群同态$\epsilon: Z_1(K)\to \mathbb{Z}$,其中$\epsilon(k\sum (a_1a_2+a_2a_3+a_3a_1))=k$.

由于$a_1a_2+a_2a_3+a_3a_1$不是边缘链,故$Z_1(K)$中的边缘链只能是$0$,即$B_1(K)=\opn{ker}\epsilon$.

故$H_1(K)=Z_1(K)/B_1(K)=Z_1(K)/\opn{ker}\epsilon=\mathbb{Z}$.
\end{corollary}









