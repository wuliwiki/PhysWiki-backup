% 分块矩阵

\pentry{矩阵\upref{Mat}}

把一个矩阵分成若干个子矩阵,称为\textbf{矩阵的分块},将矩阵看作是由子矩阵组成的矩阵,这种矩阵称为\textbf{分块矩阵}.
\[
\mat M=
\begin{pmatrix}
\mat A & \mat B\\
\mat C & \mat D
\end{pmatrix}
\]

容易由矩阵的加法和矩阵的数乘运算定义得到分块矩阵的加法和数乘运算规律

\begin{equation}
\mat A+\mat B=
\begin{pmatrix}
\mat A_1 & \mat A_2\\
\mat A_3 & \mat A_4
\end{pmatrix}
+
\begin{pmatrix}
\mat B_1 & \mat B_2\\
\mat B_3 & \mat B_4
\end{pmatrix}
=
\begin{pmatrix}
\mat A_1\mat B_1 & \mat A_2+\mat B_2\\
\mat A_3 \mat B_3 & \mat A_4+\mat B_4
\end{pmatrix}
\end{equation}