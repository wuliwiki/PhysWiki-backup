% 最大公约数与最小公倍数
% keys 最大公约数|最小公倍数
% license Usr
% type Tutor

\pentry{整除\nref{nod_divisb}}{nod_10ef}
\begin{definition}{最大公约数}
两个不全为零的整数 $a$、$b$ 的\textbf{最大公约数(greatest common divisor)} $d$ 定义为能同时整除 $a$ 与 $b$ 的最大正整数。记作 $d = (a, b)$。类似的可以定义一组数 $\{a_i\}$ 的最大公约数 $(a_1, a_2, \dots, a_k) = (a_1, (a_2, (a_3, \dots)))$。
\end{definition}
显然,若 $a$ 可表示为 $a = 2^{\alpha_1} \times 3^{\alpha_2} \times \cdots \times p_i^{\alpha_i}$,而 $b$ 可表示为 $b = 2^{\beta_1} \times 3^{\beta_2} \times \cdots \times p_j^{\beta_j}$,不失一般性地令 $i \ge j$(否则交换即可),此时令 $\beta_k = 0$ 若 $k > j$,则
\begin{equation}
d = (a, b) = 2^{\min(\alpha_1, \beta_1)} \times 3^{\min(\alpha_2, \beta_2)} \times \cdots \times p_i^{\min(\alpha_i, \beta_i)} ~.
\end{equation}
其中 $p_i$ 是第 $i$ 个素数。

\begin{definition}{互素}
若两个整数的最大公约数为 $1$,就称他们\textbf{互素(coprime)}。
\end{definition}

\begin{definition}{最小公倍数}
两个都全为零的整数 $a$、$b$ 的\textbf{最小公倍数} $l$ 定义为最小的同时是 $a$ 的倍数与 $b$ 的倍数的数。记作 $l = [a, b]$。类似的可以定义一组数 $\{b_i\}$ 的最小公倍数 $[b_1, b_2, \dots, b_l] = [b_1, [b_2, [b_3, \dots]]]$。
\end{definition}
类似的,若 $a$ 可表示为 $a = 2^{\alpha_1} \times 3^{\alpha_2} \times \cdots \times p_i^{\alpha_i}$,而 $b$ 可表示为 $b = 2^{\beta_1} \times 3^{\beta_2} \times \cdots \times p_j^{\beta_j}$,不失一般性地令 $i \ge j$(否则交换即可),此时令 $\beta_k = 0$ 若 $k > j$,则
\begin{equation}
l = [a, b] = 2^{\max(\alpha_1, \beta_1)} \times 3^{\max(\alpha_2, \beta_2)} \times \cdots \times p_i^{\max(\alpha_i, \beta_i)} ~.
\end{equation}
其中 $p_i$ 是第 $i$ 个素数。

\begin{theorem}{最大公约数与最小公倍数的乘积}
显然,对于两正数 $a, b$,$(a, b) \times [a, b] = a \times b$。
\end{theorem}
\textbf{证明}:由于 $\min(a, b) + \max(a, b) =a + b$。考虑若 $a$ 可表示为 $a = 2^{\alpha_1} \times 3^{\alpha_2} \times \cdots \times p_i^{\alpha_i}$,而 $b$ 可表示为 $b = 2^{\beta_1} \times 3^{\beta_2} \times \cdots \times p_j^{\beta_j}$,仍不失一般性地令 $i \ge j$(否则交换即可),此时令 $\beta_k = 0$ 若 $k > j$,则
\begin{equation}
(a, b) \times  [a, b] = 2^{\min(\alpha_1, \beta_1) + \max(\alpha_1, \beta_2)} \times 3^{\min(\alpha_2, \beta_2) + \max(\alpha_2, \beta_2)} \times \cdots \times p_i^{\min(\alpha_i, \beta_i) + \max(\alpha_i, \beta_i)} ~.
\end{equation}
即
\begin{equation}
(a,b) \times [a, b] = 2^{\alpha_1 + \beta_1} \times 3^{\alpha_2 + \beta_2} \times \cdots \times p_i^{\alpha_i + \beta_i} = a \times b ~.
\end{equation}
证毕!

\begin{theorem}{辗转相除}
\begin{equation}
\gcd(a, b) = \gcd(b , a~ \text{mod} ~b)  ~.
\end{equation}
\end{theorem}
\textbf{证明}:若 $a = kb + r$,其中 $r$ 为整数且 $0 \le r < b$,即 $r = a ~\text{mod}~ b$。

设 $d$ 是 $a, b$ 的某公约数,则 $d|a$ 且 $d|b$。而 $r = a - cb$,两边同时除以 $d$ 将得到 $r/d = a/d - kb/d$,而等式右侧将是整数,故左侧也是整数,这指出 $d | r$,故 $d$ 也是 $b, r$ 的公约数。
因(a,b)和(b,a mod b)的公约数相等,则其最大公约数也相等,证毕!