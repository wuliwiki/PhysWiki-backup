% 变分的变换
% 拉格朗日变换|黎曼变换

\pentry{变分\upref{Varia}}
在变分\upref{Varia}一节中,我们得到泛函 
\begin{equation}
J(y)=\int_a^bF(x,y,y')\dd x
\end{equation}
变分的表达式\autoref{Varia_eq5}~\upref{Varia},即
\begin{equation}\label{VarCha_eq1}
\delta J=\int_a^b\qty[F_y(x,y,y')\delta y+F_{y'}(x,y,y')\delta y']\dd x
\end{equation}
利用分部积分\autoref{IntBP_eq1}~\upref{IntBP},可以将上式积分号下只表达为 $\delta y$ (\textbf{拉格朗日变换})或 $\delta y'$ (\textbf{黎曼变换})的线性函数\autoref{LinMap_def1}~\upref{LinMap}.这就是这里说的\textbf{变分的变换}.

\begin{\begin{enumerate}
若在点 $a$ 及 $b$ 上,$\delta y=0$,那么
\item 拉格朗日变换:$\delta J=\int_a^b\qty(F_y-\dv{}{x}F_{y'})\delta y\dd x$
\end{enumerate}


注意,在引出\autoref{VarCha_eq1} 时,仅假定 $y(x)$ 是 $C_1$ 类的,即 $y(x)$ 具有连续微商 $y'(x)$, 而没有假定 $y'(x)$ 可微分.这将使我们看到,拉格朗日变换是不合法的,取而代之的将是黎曼变换. 