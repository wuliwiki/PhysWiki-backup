% 如何自学物理
% license CCBYSA3
% type Tutor

要谈论这个话题,首当其冲的应该是荷兰诺贝尔物理学奖得主 Gerard 't Hooft 的文章了:

\href{https://webspace.science.uu.nl/~hooft101/theorist.html}{How to become a GOOD Theoretical Physicist}

这里附上一个版本的中文翻译: \href{https://xialab.pku.edu.cn/kytdyw1/zdylm.m.jsp?wbtreeid=1011&tstreeid=11956&_t_uid=2945&language=en&homepageuuid=BF649325C5584FC683CE0B601D21AC65&templateuuid=4CC182410BA14FF8B55ED726FB2087FB&producttype=0&_tmode_=99&tsitesapptype=zdylm}{《如何成为一名优秀的理论物理学家》}

我们不妨就这篇文章的核心内容详细讨论。

首先,标题说的虽然是 “理论物理学家”, 但大致来说也\textbf{同样适用于物理学其他研究方向}。 对于公众来说,可能一谈到物理学马上就会想到那些喜闻乐见的理论话题例如宇宙、黑洞、粒子物理等。 这些话题的确非常引人入胜,但远非物理学的全部。 事实上只有一小部分从事物理学研究的人会研究这些领域。

\begin{figure}[ht]
\centering
\includegraphics[width=14.25cm]{./figures/b898880f835f999d.png}
\caption{美国一项调查中物理博士的研究领域占比(参考\href{https://ww2.aip.org/statistics/trends-in-physics-phds}{来源})调查中的,从上到下分别是凝聚态物理、粒子和场、天体物理/宇宙学、原子分子光学、生物物理、核物理、材料/纳米/表面、光学/光子学、计算物理、等离子/聚变物理、应用/工程/能源研究、量子基础/信息理论、复杂系统/统计/非线性/热物理、相对论/引力、软物质/聚合物物理、其他} \label{fig_SdyPhy_1}
\end{figure}

正文首先提到的就是英语,可见英语的重要性。 英语是事实上的科研学术交流语言。在物理领域(以及其他大部分自然科学领域),绝大多数论文都是用英语在欧美期刊上发表的,包括国内的研究者。 哪怕是为数较少的国产 SCI 期刊,也多数是英语或双语的。 可以说一定程度的英语水平是进入物理领域的门槛。


