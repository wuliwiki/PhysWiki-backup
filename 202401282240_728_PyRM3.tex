% Python RoboMaster EP 教程—初始化机器人
% keys Robomaster|机器人
% license Usr
% type Tutor

\pentry{Python 基础\upref{PyFi}}

\begin{issues}
\issueDraft
\end{issues}

在进行与机器人相关的操作之前,需要先初始化机器人对象

首先从 robomaster 包中导入 robot 模块:

\begin{lstlisting}[language=python]
from robomaster import robot
\end{lstlisting}

当SDK运行在多网卡同时使用的设备时(自动获取的ip可能不是与机器人进行连接的ip)需要手动指定 RoboMaster SDK 的本地ip地址

指定ip使用以下语句:

\begin{lstlisting}[language=python]
robomaster.config.LOCAL_IP_STR = "ip地址"
\end{lstlisting}

创建 Robot 类的实例对象 ep_robot, ep_robot 即一个机器人的对象:

\begin{lstlisting}[language=python]
ep_robot = robot.Robot()
\end{lstlisting}

初始化机器人,如果调用初始化方法时不传入任何参数,则使用config.py中配置的默认连接方式(WIFI直连模式) 以及默认的通讯方式(udp通讯)对机器人进行初始化,在本示例中我们手动指定机器人的连接方式为组网模式, 不指定通讯方式使用默认配置:

\begin{lstlisting}[language=python]
ep_robot.initialize(conn_type="sta")
\end{lstlisting}

可以通过以下语句设置默认的连接方式与通讯方式,本例中将默认的连接方式设置为 sta 模式, 默认的通讯方式设置为 tcp 方式:

\begin{lstlisting}[language=python]
config.DEFAULT_CONN_TYPE = "sta"
config.DEFAULT_PROTO_TYPE = "tcp"
\end{lstlisting}

至此,机器人的初始化工作就完成了,接下来可以通过相关接口对机器人进行信息查询、动作控制、多媒体使用等操作, 本文档将在后面的部分对几类接口的使用分别进行介绍
