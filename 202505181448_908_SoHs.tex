% Softmax 函数(综述)
% license CCBYSA3
% type Wiki

本文根据 CC-BY-SA 协议转载翻译自维基百科\href{https://en.wikipedia.org/wiki/Softmax_function}{相关文章}。

Softmax 函数,也称为 softargmax1: 184  或归一化指数函数2: 198 ,能够将一个长度为 K 的实数向量转换为 K 个可能结果的概率分布。它是逻辑函数在多维空间的推广形式,常用于多项式逻辑回归中。Softmax 函数通常作为神经网络中最后一层的激活函数,用于将网络输出归一化为对各个预测类别的概率分布。
\subsection{定义}
Softmax 函数以一个长度为 $K$ 的实数向量 $\mathbf{z}$ 作为输入,并将其归一化为一个概率分布:该分布由 $K$ 个概率值组成,每个概率值与输入中对应元素的指数成正比。也就是说,在应用 Softmax 之前,向量中的某些分量可能为负,或大于 1,且它们的和不一定为 1;但在应用 Softmax 之后,每个分量都将位于区间 $(0, 1)$ 之内,并且所有分量之和为 1,因此可以将它们解释为概率。此外,输入值越大的分量,对应的概率也越大。

标准(单位)Softmax 函数$\sigma: \mathbb{R}^K \to (0,1)^K$,其中 $K > 1$,它接收一个向量$\mathbf{z} = (z_1, \dotsc, z_K) \in \mathbb{R}^K$,并计算输出向量$\sigma(\mathbf{z}) \in (0,1)^K$的每个分量,定义为:
$$
\sigma(\mathbf{z})_i = \frac{e^{z_i}}{\sum_{j=1}^{K} e^{z_j}}.~
$$
换句话说,Softmax 对输入向量 $\mathbf{z}$ 中的每个元素 $z_i$ 应用标准指数函数(即 $e^{z_i}$),然后将所有指数值归一化——即每个指数值除以所有指数值的总和。这个归一化操作保证了输出向量 $\sigma(\mathbf{z})$ 所有分量的和为 1,从而可以被解释为概率分布。

“Softmax”一词来源于指数函数对输入向量中最大值的放大作用。例如,对向量 $(1, 2, 8)$ 进行标准 Softmax 运算,其结果大约为$(0.001, 0.002, 0.997)$,也就是说,几乎所有的权重都被分配给了最大值 8 所在的位置。

一般情况下,Softmax 函数中不一定非要使用自然底数 $e$,可以使用任意大于 0 的底数 $b$。和之前一样:如果 $b > 1$,那么输入向量中较大的分量会对应较大的输出概率;并且当 $b$ 增大时,所得的概率分布将更加集中在最大值所在的位置;相反,如果 $0 < b < 1$,那么输入中较小的分量反而会对应较大的输出概率;随着 $b$ 的减小,概率分布将更多地集中在最小值所在的位置。我们可以写成如下形式:$b = e^{\beta}$,或$b = e^{-\beta}$,其中 $\beta$ 为实数。这将导致 Softmax 函数有如下表达式:
$$
\sigma(\mathbf{z})_i = \frac{e^{\beta z_i}}{\sum_{j=1}^{K} e^{\beta z_j}} \quad \text{或} \quad \sigma(\mathbf{z})_i = \frac{e^{-\beta z_i}}{\sum_{j=1}^{K} e^{-\beta z_j}}, \quad \text{其中 } i = 1, \dotsc, K~
$$
其中与 $\beta$ 的倒数成正比的值,有时被称为温度:$\beta = 1/kT$这里的 $k$ 通常取 1 或玻尔兹曼常数,$T$ 是“温度”。较高的温度(较小的 $\beta$)会使输出分布更均匀(即熵更高,更“随机”);较低的温度(较大的 $\beta$)则会使分布更尖锐,即一个值占主导地位。

在某些领域中,底数 $b$ 是固定的,对应于某种固定的尺度;而在另一些领域中,会改变参数 $\beta$ 或 $T$ 来调整分布的形状。
\subsection{解释}
\subsubsection{平滑的 arg max}
Softmax 函数是 arg max 函数(即返回向量中最大元素索引的函数)的一个平滑近似。尽管如此,“softmax”这个名称可能具有误导性:它并不是最大值函数的平滑近似,而只是arg max的平滑版本。“softmax”一词有时也被用来指代与之密切相关的 LogSumExp 函数,而后者确实是最大值函数的平滑近似。因此,为了更准确地表达其本质,一些人更倾向于使用“softargmax”这一术语,尽管在机器学习领域,“softmax”已经是习惯用法。为避免混淆,本节中使用“softargmax”这一更清晰的表述。

与其将 arg max 看作一个输出为类别索引(如 $1, 2, \dots, n$)的函数,我们可以将其视为一个输出为独热编码的函数(假设最大值是唯一的):
$$
\operatorname{arg,max}(z_1, \dots, z_n) = (y_1, \dots, y_n) = (0, \dots,0,1,0,\dots, 0)~
$$
当且仅当索引 $i$ 是向量$(z_1, \dots, z_n)$的最大值所在位置时,输出坐标 $y_i = 1$,也就是说,$z_i$是该向量的**唯一最大值**。例如,在这种编码下:$\operatorname{arg\,max}(1, 5, 10) = (0, 0, 1)$因为第三个元素是最大值。

这一表示可以推广到存在多个最大值(即多个 $z_i$ 相等且为最大值)的情况。此时,可以将值 1 平均分配给所有最大值所在的位置;形式上,对应位置取值为 $1/k$,其中 $k$ 是最大值的个数。例如:$\operatorname{arg\,max}(1, 5, 5) = (0, 1/2, 1/2)$,因为第二和第三个元素都是最大值。如果所有元素都相等,例如:$\operatorname{arg\,max}(z, \dots, z) = \left( 1/n, \dots, 1/n \right)$表示每个位置都等可能是最大值。具有多个最大值的点 $\mathbf{z}$ 被称为奇异点(singular points或 singularities),它们构成所谓的奇异集——这些是 arg max 函数不连续的点(存在跳跃不连续);而只有一个最大值的点则称为非奇异点或常规点(non-singular或 regular points)。

根据引言中的最后一个表达式,softargmax 是 arg max函数的一个平滑近似:当$\beta \to \infty$时,softargmax 逐点收敛于 arg max。也就是说,对于任意固定的输入向量 $\mathbf{z}$,当 $\beta \to \infty$ 时,有:$\sigma_{\beta}(\mathbf{z}) \to \operatorname{arg\,max}(\mathbf{z})。
$$

---

然而,**softargmax 不以一致方式(uniformly)收敛于 arg max**。这意味着不同的输入点收敛速度不同,甚至可能非常缓慢。

实际上,softargmax 是连续的,而 arg max 在 **奇异集**(即两个或多个坐标相等的位置)上是不连续的。由于连续函数的一致极限也是连续函数,而 arg max 不连续,因此 softargmax 不可能以一致方式收敛于它。

其不一致收敛的原因在于:当输入中两个坐标接近相等(其中一个略大于另一个)时,arg max 的输出会发生剧烈跳跃(从一个位置变到另一个)。例如:

* $\sigma_\beta(1, 1.0001) \to (0, 1)$,
* $\sigma_\beta(1, 0.9999) \to (1, 0)$,
* 而对于完全相等的输入 $(1, 1)$,无论 $\beta$ 为多少,都有:$\sigma_\beta(1, 1) = (1/2, 1/2)$。

这说明:**越接近奇异点 $(x, x)$**,收敛速度越慢。

---

尽管如此,**在非奇异点集(即最大值唯一)上,softargmax 会在紧集上收敛(compact convergence)**。这是一种更强的逐点收敛形式,适用于没有不连续跳变的区域。
