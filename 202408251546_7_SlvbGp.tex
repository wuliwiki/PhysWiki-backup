% 可解群
% keys 可解群|solvable group|导出列|换位子|换位子群|derived series|commutator|commutator group
% license Usr
% type Tutor



可解群在Galois理论中起到关键作用,用于判断代数方程的根式可解性,由此而得名。由于我们现在尚未深入Galois理论,就不讨论何谓“可解”,而仅仅从群结构的角度研究这种群的性质。



\subsection{换位子群}


\begin{definition}{换位子}

给定群$G$以及$x, y\in G$,定义运算
\begin{equation}
[x, y] = x^{-1}y^{-1}xy,~
\end{equation}
称其为元素$x, y$的\textbf{换位子(commutator)}。

\end{definition}


换位子得名的意义很显然,$yx[x, y]=xy$,即把相乘的两个元素位置互换。显然,如果$xy=yx$,则它们的换位子只能是单位元$e$,因此换位子天然和交换性乃至群的中心有深刻联系。


\begin{definition}{换位子群}

给定群$G$及其子群$H$和$K$,称
\begin{equation}
[H, K] = \langle \{[h, k]\mid h\in H, k\in K\} \rangle~
\end{equation}
为$H$和$K$的\textbf{换位子群(commutator subgroup)}或\textbf{导子群(derived subgroup)}。特别地,称$[G, G]$为$G$的换位子群。

\end{definition}

注意,$[H, K]$并不是换位子构成的集合,而是由这个集合生成的子群。换句话说,换位子群中存在不是换位子的元素。

\begin{example}{换位子群中不是换位子的元素}

考虑由四个元素生成的\textbf{自由群}\upref{FreGrp}$\opn{F}_4=\langle x, y, z, w \rangle$,则显然$[x, y][z, w]\in [\opn{F}_4, \opn{F}_4]$,但这个元素本身并不是$\opn{F}_4$的换位子。

\end{example}



























