% 树与林
% keys 树|林|森林
% license Usr
% type Tutor

\pentry{图的顶点度\nref{nod_DGraph},图的连通性\nref{nod_GraCon}}{nod_0546}
林是不含\aref{圈}{def_PatCyc_2}的图,树是\enref{连通}{GraCon}的林。也就是说,林的\aref{连通分支}{def_GraCon_1}都是树。

\begin{definition}{树,林}
设 $G$ 是图。若 $G$ 上无圈,则称 $G$ 为\textbf{林}(forest);若 $G$ 无圈且连通,则称 $G$ 为\textbf{树}(tree)。
\end{definition}


\begin{definition}{叶子,内部点}
设 $G$ 是树,$x\in V(G)$。若 $d_G(v)=1$ (\autoref{def_DGraph_1}),则称 $x$ 是 $G$ 的\textbf{叶子}(leaf)。树中非叶子的点称为\textbf{内部点}。
\end{definition}

\begin{theorem}{}
下面命题是等价的:
\begin{enumerate}
\item $T$ 是树;
\item $T$ 中任意两点被 $T$ 的唯一路连接;
\item $T$ 是极小连通的,即 $T$ 是连通的,但对任一边 $e\in E(T)$,$T-e$ 不连通;
\item $T$ 是极大无圈的,即 $T$ 不包含圈,但任意不相邻的点 $x,y\in V(T)$,$T+xy$ 包含圈(其中 $xy$ 是连接 $xy$ 的边)。
\end{enumerate}

\end{theorem}

\textbf{证明:}
1. $1\Rightarrow 2$:



\textbf{证毕!}






















