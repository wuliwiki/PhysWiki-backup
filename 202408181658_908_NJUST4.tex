% 南京理工大学 2008 量子真题
% license Usr
% type Note

\textbf{声明}:“该内容来源于网络公开资料,不保证真实性,如有侵权请联系管理员”

\subsection{请考生在下列 12题中选作10题,每题15分,满分150分。}
\begin{enumerate}
    \item 1) 当势能 $V(x)$ 改变一个常量 $C$ 时, 即 $V(x) \rightarrow V(x) + C$, 粒子的波函数数中与时间无关部分变化否?粒子的能量本征值变化否?\\
   2) 量子力学有哪些假定?
    \item 利用测不准关系估计一维线性谐振子的基态能量。已知算符 $\hat{F}$ 的不确定度为
    \[
    \overline{(\Delta \hat{F})^2} = \overline{\hat{F}^2}-  \overline{\hat{F}} ^2.~
    \]
    \item 求下列算符对易关系式:\\
    1) $\hat{L}_x, \hat{P}_x - \hat{P}_x \hat{L}_x=?$\\
    2) $\hat{L}_y, \hat{P}_x - \hat{P}_x \hat{L}_y=?$\\
    3) $\hat{L}_z, \hat{P}_x - \hat{P}_x \hat{L}_x=?$\\
    \item 证明:$L = \sqrt{6\hbar}, \, L_z = \pm \hbar$ 的氢原子中的电子,在 $\theta = 45^\circ$ 和 $135^\circ$ 的方向上被发现的概率最大。
    \item 带有电荷 $e$ 的一维线性谐振子,在 $t = 0$ 时处于基态,$t > 0$ 时处于方向沿 $x$ 轴正方向的弱电场 $\epsilon = \epsilon_0 e^{-i\omega t}/l^{1/2}$ 之中($\tau$ 为大于零的常数),试求该谐振子在长时间后处于第一激发态的概率。
    \item 证明:为了保证角动量算符的 $z$ 分量 $L_z = -i\hbar \frac{\partial}{\partial \varphi}$ 是厄密算符,波函数 $\psi(r, \theta, \varphi)$ 必须满足周期性边界条件 $\psi(r, \theta, \varphi) = \psi(r, \theta, \varphi + 2\pi)$。
    \item 在自旋态下 $\chi_{\frac{1}{2}}(s_z) = \begin{pmatrix} 1 \\ 0 \end{pmatrix}$,求 $\overline{\Delta s_x^2}$ 和 $\overline{\Delta s_y^2}$。
\end{enumerate}
