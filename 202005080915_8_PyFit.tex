% SciPy数值计算库(二): 最小二乘法
\pentry{Python 入门\upref{Python}}

\subsection{最小二乘拟合}
最小二乘法是一种数学优化技术.它通过最小化误差的平方和寻找数据的最佳函数匹配.利用最小二乘法可以简便地求得未知的数据,并使得这些求得的数据与实际数据之间误差的平方和为最小.

通常情况下最小二乘法是这样的:
给定一组数,假设是二维数据.$(x_1,y_1),(x_2,y_2),\cdots,(x_n,y_n)$ ,并且 $x,y$是相关的.假设满足方程,

\begin{equation}
y=f(x)
\end{equation}
但是这个方程我们是不知道具体什么样的.我们找到一个近似函数
\begin{equation}
y=g(x)
\end{equation}
y=g(x)

对于每一个点 $(x_i,y_i)$ 有 $y_i=g(x_i)+\epsilon_i$, $\epsilon_i$是残差. 那么总的残差平方和为
\begin{equation}
err=\sum_{i=1}^n(y_i-g(x_i)]^2
\end{equation}
那么我们就是要找到这样的$g(x)$使得如下公式中的$err$函数最小. 这种算法被称之为\textbf{最小二乘拟合}(Least-square fitting).

\verb|scipy|中的子函数库\verb|optimize|已经提供了实现最小二乘拟合算法的函数\verb|leastsq|.下面是用\verb|leastsq|进行数据拟合的一个例子: