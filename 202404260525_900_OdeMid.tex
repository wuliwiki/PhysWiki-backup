% 中点法解常微分方程(组)
% keys 微分方程|数值解|Matlab|微分方程组|ode
% license Xiao
% type Tutor

\pentry{常微分方程(组)的数值解\nref{nod_OdeNum}}{nod_6878}

我们先来尝试用欧拉法解一阶微分方程
\begin{equation}\label{eq_OdeMid_1}
y'(t) = y~.
\end{equation}
令初始条件为 $y(0) = 1$。令步长为 $h = 0.5$, 步数为 $5$, 结果如\autoref{fig_OdeMid_1} 所示(代码见文章最后)。

\begin{figure}[ht]
\centering
\includegraphics[width=7cm]{./figures/bc31b04277d52e41.pdf}
\caption{欧拉法数值解(蓝)和解析解(红)} \label{fig_OdeMid_1}
\end{figure}

% 由? %文章未完成
我们知道该方程的解析解为 $y = \E^{t}$。 对比数值解和解析解, 不难分析出误差产生的原因: 我们仅用每段步长区间左端的导数预测整个区间的曲线增量。 如果我们能利用每个区间中点的导数计算整个区间的增量, 这个预测将会比欧拉法更精确。

考虑微分方程 $y'(t) = f(y, t)$ 在区间 $[t_n, t_{n+1}]$ 的曲线, 若我们已知区间左端的函数值为 $y_n$, 我们可以先用微分近似估计曲线中点的函数值为
\begin{equation}
y \qty(t_n + \frac h2) = y_n + \frac h2 f(y_n, t_n)~.
\end{equation}
然后再求出这个近似中点的导数为
\begin{equation}
y' \qty(t_n + \frac h2) = f \qty[y_n + \frac h2 f(t_n, y_n), t_n + \frac h2]~.
\end{equation}
最后我们利用这个导数估算该区间的曲线增量
\begin{equation}
y_{n+1} = y_n + hy' \qty(t_n + \frac h2) = y_n + h f \qty[y_n + \frac h2 f(t_n, y_n), t_n + \frac h2]~.
\end{equation}
这就是解常微分方程的\textbf{中点法}。

我们再来用中点法取同样的步长计算\autoref{eq_OdeMid_1}, 结果如\autoref{fig_OdeMid_2} 所示。

\begin{figure}[ht]
\centering
\includegraphics[width=7cm]{./figures/a0bf9720ea15be52.pdf}
\caption{中点法数值解(蓝)和解析解(红)} \label{fig_OdeMid_2}
\end{figure}

可见虽然中点法每一步的计算过程比欧拉法稍微复杂一些, 但精度却大大地提高了。 

中点法同样适用于微分方程组, 例如对于常微分方程组
\begin{equation}
\begin{cases}
x'(t) = f(x, y, t)\\
y'(t) = g(x, y, t)
\end{cases}~.
\end{equation}
首先计算近似中点为
\begin{equation}
\begin{cases}
x_{n+1/2} = x_n + \frac{h}{2} f(x_n, y_n, t_n)\\
y_{n+1/2} = y_n + \frac{h}{2} g(x_n, y_n, t_n)
\end{cases}~.
\end{equation}
然后有
\begin{equation}\label{eq_OdeMid_7}
\begin{cases}
x_{n+1} = x_n + h f \qty(x_{n+1/2}, y_{n+1/2}, t_n + h/2)\\
y_{n+1} = y_n + h g \qty(x_{n+1/2}, y_{n+1/2}, t_n + h/2)
\end{cases}~.
\end{equation}

\begin{lstlisting}[language=matlab, caption=odeMid.m]
% 设置参数
N = 6;
h = 0.5;
t = linspace(0, (N-1)*h, N); % 自变量
t0 = linspace(0, (N-1)*h, 100); % 用于画图

% 欧拉法
y = zeros(1,N); % 预赋值
y(1) = 1; % 初值
for ii = 1:N-1
    y(ii+1) = y(ii) + h*y(ii);
end

% 画图
figure;
plot(x,y,'+-');
hold on;
plot(t0, exp(t0));

%中点法
y = zeros(1,N);
y(1) = 1;
for ii = 1:N-1
    y(ii+1) = y(ii) + h*(y(ii) + 0.5*h*y(ii));
end

% 画图
figure;
plot(x,y,'+-');
hold on;
plot(x0, exp(x0));
\end{lstlisting}
