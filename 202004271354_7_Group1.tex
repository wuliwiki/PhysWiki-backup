% 子群和正规子群
\pentry{群的概念\upref{Group}}

\subsection{子群}

\begin{definition}{子群}

给定一个群$(G, \cdot)$,如果集合$G$有一个子集$H$,使得$e\in H$且$H$中的元素在运算$\cdot$下仍然封闭,那么显然\footnote{由于已经知道$(G,\cdot)$构成一个群了,群的四条公理中,结合性、单位元存在性以及逆元存在性都被满足了.}$(H,\cdot)$也构成一个群.称$H$是群$G$的\textbf{子群(subgroup)}.

\end{definition}

虽然群和子群的联系很紧密,但是我们通常还是把它们看作完全不同的集合,只不过可以自然地应用已经存在的群运算来定义子群的运算.这样,将已有的运算直接用在子集上,有时被称作在子集上\textbf{导出}或\textbf{诱导(induce)}了一个运算,有时也称子集上的运算是\textbf{限制在子集上的运算}.比如在定义里,群$H$的运算实际上被认为是和$G$的运算不一样,严格来说应该记为$\cdot|_H$,意思是“限制在$H$上的$\cdot$”. 但是不至于引起混淆的时候,也可以简单记为$\cdot$,并认为是同一个运算.