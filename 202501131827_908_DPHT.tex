% 丢番图(综述)
% license CCBYSA3
% type Wiki

本文根据 CC-BY-SA 协议转载翻译自维基百科\href{https://en.wikipedia.org/wiki/Diophantus#Notes}{相关文章}。

亚历山大的丢番图(约公元200年–约214年出生;约公元284年–约298年去世)是一位希腊数学家,著有两部主要作品:《多边形数论》,该书现存不全,以及《算术学》,分为十三卷,大部分仍然存在,包含了一些通过代数方程求解的算术问题。

丢番图的《算术学》对阿拉伯数学的发展产生了影响,他的方程式也影响了现代抽象代数和计算机科学的研究。他的前五卷完全是代数性质的。此外,最近对丢番图作品的研究表明,他在《算术学》中教授的解题方法,与后来的中世纪阿拉伯代数在概念和整体程序上高度相似。

丢番图是最早认识到正有理数作为数字的数学家之一,通过允许系数和解为分数。他创造了术语 \textbf{παρισότης}(parisotēs)来表示近似等式。这个术语在拉丁语中翻译为 \textbf{adaequalitas},并成为皮埃尔·德·费尔马(Pierre de Fermat)发展出的“等近性”技术,用于求函数的最大值以及曲线的切线。

尽管《算术学》不是最早使用代数符号解决算术问题的作品,但它无疑是最著名的一个,这些问题来源于希腊古代,并且其中的一些问题激发了后来的数学家在分析学和数论领域的研究。现代使用中,丢番图方程指的是带有整数系数的代数方程,目标是寻找其整数解。丢番图几何和丢番图逼近是另外两个以他命名的数论子领域。
\subsection{传记}
丢番图出生于一个希腊家庭,并且已知他在罗马时代的公元200年至214年到284年或298年期间生活在埃及的亚历山大城。[6][8][9][a] 关于丢番图生平的大部分知识来自一部5世纪的希腊数字游戏和谜题选集,由梅特罗多罗斯(Metrodorus)编纂。书中有一个问题(有时被称为他的墓志铭)内容如下:

此地安葬丢番图,令人惊叹。通过代数的艺术,石碑上述说他的年岁:‘上帝赋予他少年时期,生命的一六分之一;青春期更多,成长为胡须浓密的青年,一十二分之一;然后在婚姻前,又度过了七分之一;五年后,迎来了一个跳跃的儿子。可怜的是,这个父亲与智者的亲爱的孩子,在活到父亲生命的一半时,命运将他带走。四年后,他通过数字科学安慰自己的命运,最终结束了生命。’

这个谜题意味着丢番图的年龄 \(x\) 可以表达为:

\[
x =\frac{x}{6}+\frac{x}{12} +\frac{x}{7}+5+\frac{x}{2}+4~
\]

解得 \(x\) 的值为84岁。然而,这些信息的准确性无法确认。

在流行文化中,这个谜题出现在《雷顿教授与潘多拉的盒子》中,作为游戏中最难解的谜题之一,需要通过先解决其他谜题才能解锁。
\subsection{算术}
\begin{figure}[ht]
\centering
\includegraphics[width=6cm]{./figures/43024ff32f6b09ef.png}
\caption{《丢番图算术》拉丁文版由巴谢(Bachet)翻译的标题页(1621年)。} \label{fig_DPHT_1}
\end{figure}
《算术》是丢番图的主要著作,也是希腊数学中关于前现代代数的最重要作品之一。它是一个包含有确定性和不确定性方程数值解的题目集。原本《算术》共有十三卷,但只有六卷保存了下来,尽管有些人认为1968年发现的四本阿拉伯书籍也是丢番图的作品。一些《算术》中的丢番图问题也在阿拉伯文献中找到了踪迹。

需要提到的是,丢番图在解题时从未使用一般的方法。著名的德国数学家赫尔曼·汉克尔曾对丢番图做出如下评论:

“我们的作者(丢番图)没有一点迹象表明他使用了一种通用的、全面的方法;每一个问题都需要一些特殊的方法,而这些方法即使对最相似的问题也行不通。因此,即便现代学者已经研究了丢番图的100个解法,也难以解出第101个问题。”
\subsubsection{历史}
像许多其他希腊数学著作一样,丢番图的著作在西欧黑暗时代被遗忘,因为古希腊语的学习以及总体的文化素养大幅下降。然而,幸存下来的部分希腊文《算术》,和所有传递到近代世界的古希腊文献一样,通过拜占庭学者的抄写得以保存,因此在中世纪的拜占庭学者中是有流传的。拜占庭希腊学者约翰·霍尔塔斯梅诺斯(John Chortasmenos,1370-1437)对丢番图的注释以及早期希腊学者马克西莫斯·普拉奴德斯(Maximos Planudes,1260-1305)的综合性评论得以保存,后者曾在拜占庭君士坦丁堡的科拉修道院编纂了丢番图的版本[16]。此外,《算术》的部分内容可能通过阿拉伯学术传统得以流传(见上文)。1463年,德国数学家雷吉奥蒙塔努斯写道:

“至今没有人将丢番图的十三卷从希腊语翻译成拉丁语,而这些书中隐藏着整个算术学的精华。”

《算术》首次从希腊语翻译成拉丁语是在1570年,由博姆贝利(Bombelli)完成,但翻译本从未出版。然而,博姆贝利借用了其中许多问题并将其用于他自己的《代数学》一书。《算术》的拉丁语初版由希兰德(Xylander)于1575年出版。1621年,由巴谢(Bachet)完成的拉丁文翻译成为了广泛流传的第一版。皮埃尔·德·费马拥有一本副本,研究并在书页边缘做了笔记。1895年,保罗·塔内里(Paul Tannery)对《算术》的拉丁语翻译被托马斯·L·希斯(Thomas L. Heath)称为一次改进,并且在1910年他出版的英文版第二版中使用了这一版本。
\subsubsection{费马和霍尔塔斯梅诺斯的页边注释}
\begin{figure}[ht]
\centering
\includegraphics[width=6cm]{./figures/c913dea01fcaff51.png}
\caption{《算术》第二章第8题(1670年版),附有费马的注释,后来成为费马最后定理。} \label{fig_DPHT_2}
\end{figure}
1621年由巴谢(Bachet)出版的《算术》版本因皮埃尔·德·费马在他副本的页边写下著名的“最后定理”而声名鹊起:

如果一个整数 \( n \) 大于 2,那么方程 \( a^n + b^n = c^n \) 在非零整数 \( a \), \( b \), 和 \( c \) 中没有解。我有一个真正神奇的证明,但这个页边太狭窄,容不下。”

费马的证明从未被找到,且寻找该定理证明的问题未能解决几个世纪。直到1994年,安德鲁·怀尔斯(Andrew Wiles)经过七年的努力终于找到了证明。人们普遍认为费马并没有他所宣称的证明。尽管费马写下这一笔记的原始副本已经失传,费马的儿子却编辑了1670年出版的下一版丢番图著作。尽管该版文本在其他方面不如1621年的版本,但费马的注释——包括“最后定理”——还是在此版本中得以出版。

费马并不是第一个在《算术》页边写下注释的数学家;拜占庭学者约翰·霍尔塔斯梅诺斯(John Chortasmenos,1370–1437)曾在同一个问题旁边写道:“丢番图,你的灵魂与撒旦同在,因为你的其他定理,特别是此定理的难度。[16]
\subsection{其他著作}  
除了《算术》之外,狄奥凡图斯还写了几本其他书,但其中只有少数几本得以保存。
《\subsubsection{命题》}  
狄奥凡图斯提到过一本名为《命题集》(The Porisms 或 Porismata)的书,书中包含一系列引理,但这本书已经完全遗失。[17]

尽管《命题集》已失传,但我们知道其中的三个引理,因为狄奥凡图斯在《算术》中提到了它们。其中一个引理指出,两个有理数立方之差等于另外两个有理数立方之和,即:给定任意的\(a\)和\(b\),且\(a > b\),存在正的有理数\(c\)和\(d\),使得

\(a^3-b^3=c^3+d^3\)。

\subsubsection{多边形数和几何元素}  
狄奥凡图斯还以写作多边形数而闻名,这一主题曾受到毕达哥拉斯及其学派的极大关注。有关多边形数的书籍残片至今仍有保存。[18]

一本名为《几何元素预备》(Preliminaries to the Geometric Elements)的书传统上被归于亚历山大的希罗(Hero of Alexandria)。最近,威尔伯·诺尔(Wilbur Knorr)对其进行了研究,并提出归于希罗的观点是不正确的,真正的作者应是狄奥凡图斯。[19]
\subsection{影响}  
丢番图的著作对历史产生了深远的影响。《算术》的版本对16世纪末至17世纪和18世纪欧洲代数的发展产生了深远的影响。丢番图及其著作也影响了阿拉伯数学,在阿拉伯数学家中享有极高的声誉。丢番图的工作为代数的发展奠定了基础,事实上,现代高级数学很大程度上是基于代数的。[20] 他对印度的影响仍然存在争议。

丢番图被认为是“代数之父”,因为他在数论、数学符号及其在《算术》系列书籍中最早使用的简化符号法方面作出了重要贡献。[2] 然而,这一观点通常会受到争议,因为阿尔-花拉子米也被称为“代数之父”,尽管如此,这两位数学家都为今天代数的发展铺平了道路。
\subsection{丢番图分析}    
今天,丢番图分析是研究整数(整数解)方程的领域,丢番图方程是具有整数系数的多项式方程,要求只有整数解。通常,很难判断给定的丢番图方程是否可解。《算术》中的大多数问题都涉及二次方程。丢番图研究了三种不同类型的二次方程:\(ax^2+bx=c\),\(ax^2=bx+c\)和 \(ax^2+c=bx\)。丢番图之所以有这三种情况,而今天我们只有一种情况,是因为他没有零的概念,并且通过将给定的数字\(a,b,c\)都视为正数来避免负系数。丢番图始终满足于有理数解,而不要求整数解,这意味着他接受分数作为问题的解。丢番图认为负数或无理数的平方根解“无用”、“毫无意义”甚至“荒谬”。举一个具体的例子,他称方程\(4=4x+20\)为“荒谬”,因为它会导致\(x\)的负值。他在求解二次方程时只寻找一个解。没有证据表明丢番图意识到二次方程可能有两个解。他还考虑了同时的二次方程。
\subsection{数学符号}  
参见:算术 § 简化代数法 和 简化代数法  
丢番图在数学符号方面做出了重要的进展,他成为第一个已知使用代数符号和符号表示法的人。在他之前,每个人都完全写出方程式。丢番图引入了一种代数符号表示法,采用了简化的符号来表示频繁出现的运算,以及用于未知数和未知数的幂的缩写。数学历史学家库尔特·福格尔(Kurt Vogel)指出:[21]

丢番图首次引入的符号法,毫无疑问是他自己设计的,提供了一种简短且易于理解的方式来表达方程式……由于“等号”一词也被缩写,丢番图从口头代数迈出了向符号代数的基本一步。

尽管丢番图在符号法方面做出了重要进展,但他仍然缺乏表达更一般方法所需的符号。这使得他的工作更多地关注特定问题,而不是一般情况。丢番图符号法的一些局限性在于,他只有一个未知数的符号,当问题涉及多个未知数时,丢番图不得不用“第一个未知数”、“第二个未知数”等词语来表达。他也没有表示一般数n的符号。我们写作的表达式如 \(\frac{12+6n}{n^2-3}\) 丢番图不得不采用类似这样的表达方式:“……一个六倍数加上十二,结果被除以该数的平方超过三的差。”代数仍然需要很长时间,才能在表达和解决一般性问题时做到简洁明了。
\subsection{参见}  
\begin{itemize}
\item 厄尔多什–丢番图图  
\item 丢番图 II.VIII  
\item 多项式丢番图方程
\end{itemize}
\subsection{注释}  
a.关于丢番图的起源,曾出现过一些边缘理论。在现代,一些作者曾描述他可能是阿拉伯人、犹太人、希腊化的埃及人[10],或希腊化的巴比伦人[11]。还有人甚至声称丢番图是基督教的皈依者。所有这些说法都被认为是没有根据和猜测性的[12][13]。关于他起源的这些误解源于一些混淆(例如与阿拉伯丢番图的混淆)、不同历史时期的合并、将数学问题转化为民族类别,以及种族主义原因[13]。
\subsection{参考文献}  
\begin{enumerate}
\item 古希腊文:Διόφαντος ὁ Ἀλεξανδρεύς,罗马化:Diophantos ho Alexandreus  
\item Carl B. Boyer,《数学史》第二版(Wiley,1991年),第228页  
\item Hettle, Cyrus(2015)。“丢番图《算术》在符号学和数学上的影响”,《人文学数学杂志》,5(1):139–166,doi:10.5642/jhummath.201501.08。  
\item Christianidis, Jean; Megremi, Athanasia(2019)。“追溯代数的早期历史:关于丢番图在讲希腊语世界中的见证(公元4–7世纪)”,《数学史》47:16–38,doi:10.1016/j.hm.2019.02.002。  
\item Katz, Mikhail G.; Schaps, David; Shnider, Steve(2013),“几乎相等:从丢番图到费马及其之后的适当性方法”,《科学视角》,21(3):283–324,arXiv:1210.7750,Bibcode:2012arXiv1210.7750K,doi:10.1162/POSC_a_00101,S2CID 57569974  
\item Research Machines plc.(2004)。“哈钦森科学传记词典”。阿宾登,牛津郡:Helicon Publishing,第312页。丢番图(约公元270-280年),希腊数学家,在解线性数学问题时,发展了代数的早期形式。  
\item D. Mary, R. Flamary, C. Theys 和 C. Aime(2016)。“天文学中的仪器和信号处理数学工具,第78-79卷,2016年”。EAS出版系列,第73–98页。丢番图,亚历山大港的希腊数学家,被称为代数之父。他研究了具有整数系数和整数解的多项式方程,称为丢番图方程。  
\item Boyer, Carl B.(1991)。“希腊数学的复兴与衰退”,《数学史》(第二版),John Wiley & Sons, Inc.第178页,ISBN 0-471-54397-7。在这一时期的开始,也就是后亚历山大港时期,我们看到亚历山大港的主要希腊代数家丢番图,而在这一时期的末期,最后一位重要的希腊几何学家帕普斯出现在历史上。  
\item Cooke, Roger(1997)。“数学的本质”,《数学史:简明课程》,Wiley-Interscience,第7页,ISBN 0-471-18082-3。第三世纪的希腊数学家丢番图在他的著作中对符号的使用做了一些扩展,但与阿卡德人的情况一样,仍然存在相同的缺陷。
\item Victor J. Katz (1998). 《数学史:导论》,第184页。Addison Wesley,ISBN 0-321-01618-1.  
“但我们真正想知道的是,公元前一世纪至五世纪的亚历山大港数学家在多大程度上是希腊人。显然,他们都用希腊语写作,并且是亚历山大港希腊知识圈的一部分。大多数现代研究得出结论认为,希腊社区与埃及社区共存……那么我们是否应当假设托勒密、丢番图、帕普斯和希帕提亚是民族上希腊人,他们的祖先曾经从希腊来到这里,但与埃及人保持隔离?当然,这个问题无法确切回答。但对早期公元时代的纸草文献的研究表明,希腊人和埃及人之间有相当多的通婚……而且已知希腊婚姻合同逐渐趋于与埃及婚姻合同相似。此外,从亚历山大港建立之初,少数埃及人被接纳进城中的特权阶层,承担许多市政职能。当然,在这种情况下,埃及人必须“希腊化”,即接受希腊习俗和希腊语言。考虑到这些亚历山大港数学家活跃在城市建立几百年之后,他们既有可能是民族上埃及人,也有可能保持民族上希腊人的身份。在任何情况下,将他们描绘成纯粹的欧洲面孔是没有根据的,因为没有任何身体特征的描述。”
\item D. M. Burton (1991, 1995). 《数学史》,杜布克,IA(Wm.C. Brown Publishers)。  
“丢番图很可能是一个希腊化的巴比伦人。”
\item Ad Meskens, *Travelling Mathematics: The Fate of Diophantos' Arithmetic* (Springer, 2010),第48页:“自公元1500年以来,距离丢番图去世超过一千年,许多作者对丢番图的生平进行了猜测,认为他是阿拉伯人、犹太人、皈依的希腊人或希腊化的巴比伦人。然而,这些说法都经不起批判性的审视。” 注28:“这里可能与丢番图—阿拉伯人(利巴纽斯的老师)混淆,他生活在罗马皇帝尤利安(弃信者)的统治时期。”  
\item 关于这些说法的分析与驳斥,见:Schappacher, Norbert(2005)。“《亚历山大港的丢番图:一部文本及其历史》”,数学高级研究所。
\item J. Sesiano(1982)。“丢番图《算术》中的第四至第七书在被归于库斯塔·伊本·卢卡的阿拉伯译本中的表现”,纽约/海德堡/柏林:Springer-Verlag,第502页。  
\item Hankel H.,“《古代与中世纪的数学史》,莱比锡,1874年。”(由乌尔里希·利雷赫特翻译,收录于《13世纪的中国数学》,Dover出版公司,纽约,1973年)  
\item Herrin, Judith(2013-03-18)。“*Margins and Metropolis: Authority across the Byzantine Empire*”,普林斯顿大学出版社,第322页,ISBN 978-1400845224。  
\item G. J. Toomer; Reviel Netz. “丢番图”。见:Simon Hornblower; Anthony Spawforth; Esther Eidinow(编),*Oxford Classical Dictionary*(第4版)。  
\item “丢番图传记”。www-history.mcs.st-and.ac.uk。2018年4月10日检索。  
\item Knorr, Wilbur:“*Arithmêtike stoicheiôsis: On Diophantus and Hero of Alexandria*”,《数学历史》,纽约,1993年,Vol.20,第2期,180-192。  
\item Sesiano, Jacques。“丢番图——传记与事实”。《大英百科全书》。2022年8月23日检索。  
\item Kurt Vogel,“*亚历山大港的丢番图*”。见:*Complete Dictionary of Scientific Biography*,Encyclopedia.com,2008年。
\end{enumerate}