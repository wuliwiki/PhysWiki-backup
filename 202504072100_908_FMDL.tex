% 费马大定理(综述)
% license CCBYSA3
% type Wiki

本文根据 CC-BY-SA 协议转载翻译自维基百科\href{https://en.wikipedia.org/wiki/Fermat\%27s_Last_Theorem}{相关文章}。

在数论中,费马大定理(在较早的文献中有时称为“费马猜想”)陈述如下:对于任意整数\( n > 2 \),不存在三个正整数\( a, b, c \)满足方程\( a^n + b^n = c^n \)。而对于\( n = 1 \)和\( n = 2 \)的情形,自古以来就已知存在无穷多个解。\(^\text{[1]}\)

这个命题最早是皮埃尔·德·费马大约于1637年在一本《算术》书的页边空白处提出的。他还写道他已有一个证明,但“这个证明太大,写不下”。尽管费马曾提出的其他未经证明的命题后来被他人证明并被称为“费马定理”(例如费马两平方和定理),但唯独这条“费马大定理”长期无法证明,使人们怀疑费马是否真的拥有一个正确的证明。因此,这个命题长期以来被称为\textbf{猜想}而不是定理。经过数学家长达358年的努力,安德鲁·怀尔斯于1994年首次成功给出了完整证明,并于1995年正式发表。2016年,怀尔斯因其工作获得阿贝尔奖,其成果被称为“一项惊人的突破”。\(^\text{[2]}\)此外,该证明还涵盖了大量谷山–志村猜想的内容,该猜想后来被称为模性定理,它不仅解开了费马大定理之谜,还开辟了许多新领域,并发展出了强大的模性提升技术,对解决众多其他数学难题产生了深远影响。

这个未解难题曾在 19世纪和20世纪极大地推动了代数数论的发展。在整个数学史上,费马大定理是最著名的定理之一。在被证明之前,它还曾被列入《吉尼斯世界纪录》,称为“最难的数学问题”,部分原因是该定理拥有最多数量的失败证明尝试。\(^\text{[3]}\)
\subsection{概述}  
\subsubsection{毕达哥拉斯的起源}
毕达哥拉斯方程\(x^2 + y^2 = z^2\)在正整数\(x\)、\(y\)、\(z\)上有无穷多组解,这些解被称为毕达哥拉斯三元组(最简单的例子是 3、4、5)。大约在 1637年,费马在一本书的页边空白处写道,更一般形式的方程\(a^n + b^n = c^n\)在当\(n > 2\)时,没有正整数解。尽管他声称自己有一个完整的证明,但他未留下任何细节,至今也未有人找到该证明。这个断言是在他去世大约 30 年后才被人发现的。

这个断言后来被称为费马大定理,并在接下来的三个半世纪里始终未被证明。\(^\text{[4]}\)

这一定理最终成为数学史上最著名的未解问题之一。对它的证明尝试促使了数论领域的重大进展,随着时间推移,费马大定理作为数学未解难题的地位也日益突出。
\subsubsection{后续发展与最终解答}
当\(n = 4\)时的特例由费马本人亲自证明,该特例足以说明:如果费马大定理在某个非素数指数\(n\)下不成立,那么它在某个更小的\(n\)下也将不成立,因此只需对素数指数\(n\)进行进一步研究。\(^\text{[注1]}\)在接下来的两个世纪(1637–1839年)中,该猜想仅在素数\(n = 3, 5, 7\)的情况下被证明成立,尽管索菲·热尔曼提出并证明了一种适用于整个素数类的方法,这在当时具有开创性意义。到了19世纪中期,欧内斯特·库默尔进一步拓展了该思路,并证明了该定理对于所有正规素数成立,但对于非正规素数仍需逐个分析。在库默尔工作的基础上,借助复杂的计算机研究,其他数学家将证明的范围扩展到了所有指数为素数且小于四百万的情况。\(^\text{[5]}\)但对于所有整数指数的全面证明依然遥不可及(这意味着大多数数学家认为:该定理要么无法证明、要么极其困难,或者以当时的知识几乎不可能完成)。\(^\text{[6]}\)

大约在1955年,日本数学家志村五郎和谷山丰怀疑椭圆曲线与模形式之间可能存在某种联系——这是数学中的两个完全不同的领域。这个猜想当时被称为“谷山–志村猜想”(后来称为模性定理),它最初是一个独立的数学命题,表面上与费马大定理并无关联。尽管如此,这个猜想被广泛认为本身就具有重要意义和深远影响,但也像费马大定理一样,被认为极其难以证明,几乎无法触及。\(^\text{[7]}\)

