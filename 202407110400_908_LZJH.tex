% 粒子聚合
% license CCBYSA3
% type Wiki

(本文根据 CC-BY-SA 协议转载自原搜狗科学百科对英文维基百科的翻译)
\begin{figure}[ht]
\centering
\includegraphics[width=8cm]{./figures/da338a0a1990f5d4.png}
\caption{上图是粒子聚集的图解。颗粒在稳定的悬浮液中单独分散开,而在不稳定的悬浮液中聚集。随着聚集从早期状态发展到后期状态,聚集物的尺寸增大,最终可能胶化。} \label{fig_LZJH_3}
\end{figure}

颗粒凝聚是指在悬浮液中形成的集合体的过程,代表了导致胶体系统不稳定的一种机制。在此过程中,分散在液相中的颗粒相互粘附,并自发形成不规则的颗粒团、絮状物或聚集体。这种现象也被称为凝结或絮凝,这种悬浮液也被称为不稳定物。颗粒聚集可以通过添加盐类或另一种称为凝结剂或絮凝剂的化学物质来引起。[1] 当凝聚物是由聚合物或聚电解质的加入引起时被称为絮凝,而凝结这一概念包含更广泛的含义。

粒子聚集通常是一个不可逆的过程。一旦颗粒聚集体形成,它们就不会轻易分解。在聚集过程中,聚集物的尺寸会增大,因此它们最终可能会沉淀到容器底部,这种现象被称为沉淀。或者,胶体凝胶可以在浓缩悬浮液中形成,从而改变其流变性质。胶溶是一种向的过程,指颗粒聚集体作为单个颗粒被破坏和分散,这种过程很难自发发生,但可以在搅拌或受剪切力断裂下引起。

胶体颗粒也可能长时间在液体中保持分散,这种分散状态可以保持几天甚至几年。这种现象被称为胶体的稳定性,而这种悬浮液被称为稳定物。稳定的悬浮液可以存在于低盐浓度下,或通过添加稳定剂或有稳定性的化学】品获得。胶体或其他颗粒的稳定性最常用ζ电位来测量。该参数量化了粒子间的斥力测量,这种斥力是抑制颗粒聚集的关键因素。

类似的聚合过程也发生在其他分散的系统中。在乳状液中,粒子也可能与液滴聚结匹配,导致沉淀和乳状液分层。在气溶胶中,同样地,空气中的颗粒也可能聚集并形成更大的团簇,烟灰就是一个例子。

\subsection{早期阶段}
分散良好的胶体悬浮液由单独分离的颗粒组成,颗粒间相互的排斥力使其稳定。当排斥力因为凝结剂的添加而减弱,甚至成为吸引力时,颗粒开始聚集。刚开始,两个单独的$A1$粒子根据以下公式形成$A2$成对物[2]
$A 1 + A 1 \to A 2$
在聚集过程的早期,悬浮液主要含有颗粒单体和一些二聚体。该反应的速率由聚集速率系数$k$表示,因为成对是二阶速率过程,所以该系数的单位是,因为颗粒浓度等于单位体积内()的颗粒数。由于绝对聚集率很难被测量,人们通常参考无量纲稳定比值$W = /k$,其中是快速状态下的聚集率系数,$k$是特定条件下的系数。稳定比在快速状态下接近于1,在慢速状态下增大,在悬浮液稳定时变得非常大。

当粒子之间的相互作用的势能只有纯粹的吸引力时,聚集过程仅受粒子相互扩散(或布朗运动)的限制,又被称作是高速、高频或扩散受限的聚集(DLA)。当相互作用势表现出中间介质的阻止作用时,聚集会因需要多次尝试来克服这类阻止作用而减慢,这又被称作缓慢或反应受限的聚集(RLA)。通过改变盐的浓度、酸碱度或其他添加剂,聚集的速度可以从快到慢进行调节。因为从快速聚集到慢速聚集的转变发生在狭窄的浓度范围内,人们将该范围称为临界凝结浓度(CCC)。
\begin{figure}[ht]
\centering
\includegraphics[width=6cm]{./figures/693158a2989f1f6c.png}
\caption{以上是胶体悬浮液的稳定性与盐浓度关系示意图。} \label{fig_LZJH_1}
\end{figure}
通常,胶体颗粒悬浮在水中。在这种情况下,电荷在粒子的表面累积,并在每个粒子周围形成双电层。[3] 两个逐渐靠近的粒子的扩散层相互重叠,导致了双层排斥力作用的相互势能,最终使得粒子具有稳定性。当悬浮液中加入盐类时,双电层排斥力被屏蔽,范德华引力便占优势,并引起粒子的快速聚集。图中的右边部分显示了典型的稳定比W与电解质浓度的相关性,并由此标出了缓慢和快速聚集的状态。

下表总结了不同负离子净电荷的CCC范围。[4]电荷以单个电荷为基本单位。这种依赖性反映了舒尔茨-哈代法则,[5][6] 该法则指出CCC随着负离子电荷的倒数第六次幂而变化。CCC也多少取决于离子的类型,即使它们携带相同的电荷。这种依赖性可以反映不同的颗粒性质或对颗粒表面的不同离子亲和力。由于大部分粒子带负电荷,因此高价金属阳离子可作为高效凝结剂。
\begin{table}[ht]
\centering
\caption\label{LZJH}
\begin{tabular}{|c|c}
\hline
\textbf{电荷数} & \textbf{CCC(×103摩尔/升)}\\
\hline
1 & 50-300\\
\hline
2 & 2-30\\
\hline
3 & 0.03-0.5\\
\hline
\end{tabular}
\end{table}
带相反电荷的物质的吸附(例如质子,典型的可吸附离子、表面活性剂或聚电解质)可以通过电荷中和,使颗粒悬浮液不稳定,或者通过电荷积累使其稳定,这导致粒子在电荷中和点附近快速聚集,而在远离电荷中和点缓慢聚集。

胶体稳定性的定量解释最初是在DLVO理论中提出的。[2] 这一理论证实了缓慢和快速聚集状态的存在,尽管在缓慢聚集状态下,对盐浓度的依赖性通常被预测为比实验所观察到的强得多。舒尔茨-哈代法则也可以从DLVO理论中推导出来。

胶体的稳定性在其他的机制下也有相同的可能性发生,尤其是涉及聚合物的机制。吸附或接枝聚合物可在颗粒周围形成保护层,产生空间排斥力,并导致空间稳定。聚羧酸酯醚(PCE)就是这种情况,它是最新一代专门设计的化学定制超塑化剂,旨在提高混凝土的可加工性,同时降低其含水量,以改善其性能和耐久性。当聚合物链松散地吸附到颗粒上时,聚合物链可以桥接两个颗粒,并产生桥接力。这种情况被称为桥接絮凝。

当粒子聚集仅由扩散驱动时,被称为围动力学聚集。聚集可以通过分解切应力(例如搅拌)来增强,被称为同向移动的动力学聚集。

\subsection{后期阶段}
\begin{figure}[ht]
\centering
\includegraphics[width=6cm]{./figures/f522f7a5ad033977.png}
\caption{形成的较大聚集体的结构可能会不同。在快速聚集区或DLCA区,聚集区更分散,而在慢速聚集区或RLCA区,聚集区更紧密。} \label{fig_LZJH_2}
\end{figure}
随着聚合过程的继续,会形成更大的集群。这种增长主要是通过不同集群之间的相遇来实现的,因此我们称之为集群和集群的聚合过程。这样得到的聚类是不规则的,但在数据上是自相似的。它们是质量分形的例子,由此它们的质量随着典型尺寸成比例增长,可以被旋转半径的$d$次幂[2]描述:
$$M \propto R_{g}^{d}~$$
其中$d$是质量分形维数。根据聚合速率是快还是慢,可以判断了扩散受限的集群聚合(DLCA)或反应受限的集群聚合(RLCA)。在每一种情况下,集群都有不同的特征。DLCA团簇松散且有分枝($d \approx 1.8$),而RLCA团簇更紧密($d \approx 2.1$)。[7] 在这两种聚合种类下,集群规模分布也不同。DLCA团簇相对单一地分散,而RLCA团簇的大小分布非常广泛。

团簇越大,它们的沉降速度越快。因此,聚集颗粒沉淀物和这种机制提供了一种将它们从悬浮液中分离出来的方法。在较高的颗粒浓度下,生长的团簇可以相互连接,形成颗粒凝胶。这种凝胶是弹性固体,但不同于普通固体的是,它具有非常低的弹性模量。

\subsection{同聚与杂聚}
当聚集发生在由相似的单一分散胶体颗粒组成的悬浮液中时,这种过程被称为

(或同粒凝结)。当聚集发生在由不同胶体颗粒组成的悬浮液中时,这种过程被称为杂聚集(或杂凝聚)。最简单的杂聚过程发生在两种单分散胶体颗粒混合时。在早期阶段,三种类型的成对物可能形成 [8]
$$A + A \to A2 B + B \to B2 A + B \to AB~$$
前两个过程代表了含有颗粒$A$或$B$的纯悬浮液中的均聚,最后一个反应代表了真实的杂聚过程。这些反应可以用聚集系数来描述。例如,当粒子$A$和粒子$B$分别带有正电荷和负电荷时,同聚速率可能较慢,而杂聚速率较快。与同聚不同,杂聚速率随着盐浓度的降低而加快。在这种杂聚过程的后期形成的团簇甚至比在DLCA期间获得的团簇更加分散($d \approx 1.4$)。[9]

杂聚过程的一个重要的特殊情况是颗粒在基底上的沉积。[1] 该过程的早期阶段对应于将单个粒子附着到基底上,基底可以视作另一个大得多的粒子。随后的阶段可以反映出通过颗粒之间的排斥作用来阻挡基底,而吸引作用可以导致多层生长,也称为熟化。这些现象与薄膜或过滤器结垢有关。

\subsection{实验技术}
人们已经开发了许多实验技术来研究粒子聚集。最常用的是基于光的透射或散射的时间分辨技术。[10]

光传输: 透过聚集悬浮液的透射光的变化可以用可见光区的普通分光光度计来研究。随着聚集的进行,介质变得更加混浊,其吸光度增加。吸光度的增加可能与聚集速率常数k相关,并且稳定性比可以从这样的测量中被估计。这项技术的优点是具有简易性。

光散射: 这些技术基于在不同时期探测聚集悬浮液中的散射光。静态光的散射导致了散射强度的变化,而动态光的散射产生导致了明显的流体动力学半径的变化。在聚集的早期阶段,这些量中每一个的变化都与聚集速率常数k成正比。[11]在后期阶段,人们可以获得关于簇的形成信息(例如分形维数)。[7] 光散射技术适用于测量各种尺寸的粒子。人们必须考虑不同的散射效应,因为散射对于较大的颗粒或较大的聚集体变得越来越重要。这种影响在弱混浊悬浮液中可以忽略。人们通过透射率、后向散射技术或散射波光谱学研究了强散射系统中的聚集过程。

单粒子计数: 这种技术提供了极好的分辨率,由许多粒子组成的粒子簇可以被单独分辨。[11] 聚集悬浮液被强制通过狭窄的毛细管颗粒计数器,并通过光散射分析每个聚集物的尺寸。从散射强度,人们可以推断出每个骨料的尺寸,并构建出详细的骨料尺寸分布。如果悬浮液中含有大量的盐类,人们同样可以使用库尔特计数器。随着时间的推移,尺寸分布向更大的聚集转移,从这种变化可以推断出不同聚集的聚集和分散率。该技术的缺点是集料在高剪切下被迫通过狭窄的毛细管,在这些情况下聚集可能会被分散。

间接技术: 由于胶体悬浮液的许多性质取决于悬浮颗粒的聚集状态,各种间接技术也被用来监测颗粒聚集。虽然很难从这些实验中获得关于聚集率或聚集性质的定量信息,但这些实验对于实际应用可能是最有价值的。在这些技术中,沉降测试最为相关。当人们检查一系列含有不同浓度絮凝剂的悬浮液的试管时,稳定的悬浮液通常保持分散,而不稳定的悬浮液则沉淀下来。基于光的散射或透射率来监测悬浮沉降的自动化仪器已经被开发,并且它们可以用于探测颗粒聚集。然而,人们必须认识到,这些技术可能并不总是能正确地反映悬架的实际聚集状态。例如,即使没有聚集,较大的初级粒子也可能沉淀,或者已经形成胶体凝胶的聚集物将保持悬浮。能够监测聚集状态的其他间接技术包括,例如,过滤、流变学、超声波吸收或介电性质。[10]

\subsection{关联}
颗粒聚集是一种普遍现象,它在自然界中自发发生,但在制造业中也被广泛开发。一些例子包括:

三角洲的形成。当携带悬浮泥沙颗粒的河水接触到咸水时,颗粒聚集可能成为河流三角洲形成的原因之一。带电粒子在含盐量低的河流淡水中是稳定的,但会在含盐量高的海水中变得不稳定。在后一种介质中,颗粒聚集,较大的颗粒聚集沉积物,从而形成河流三角洲。

造纸。助留剂被添加到纸浆中以加速纸张的形成。这些助剂是凝固助剂,加速了纤维素纤维和填料颗粒之间的聚集。阳离子聚电解质经常被使用以达成这一目的。

水处理。城市废水的处理通常包括去除固体微粒的步骤。这种分离是通过添加絮凝剂或凝结剂来实现的,这些絮凝剂或凝结剂引起了悬浮固体的聚集。骨料通常通过沉淀分离,产生污水污泥。水处理中常用的絮凝剂包括多价金属离子(如Fe3+或Al3+)、聚电解质或两者的混合物。

奶酪制作。奶酪生产的关键步骤是将牛奶分离成固体凝乳和液体乳清。这种分离是通过酸化牛奶或添加凝乳酶来诱导酪蛋白胶束之间的聚集过程来实现的。这些酸中和了胶束上的羧酸基团并诱导了聚集。

\subsection{参考文献}
[1]
^M. Elimelech, J. Gregory, X. Jia, R. Williams, Particle Deposition and Aggregation: Measurement, Modelling and Simulation, Butterworth-Heinemann, 1998..

[2]
^W. B. Russel, D. A. Saville, W. R. Schowalter,Colloidal Dispersions,Cambridge University Press, 1989..

[3]
^D. F. Evans, H. Wennerstrom, The Colloidal Domain, John Wiley, 1999..

[4]
^Tezak, B.; Matijevic, E.; Schuiz, K. F. (1955). "Coagulation of Hydrophobic Sols in Statu Nascendi. III. The Influence of the Ionic Size and Valency of the Counterion". The Journal of Physical Chemistry. 59 (8): 769–773. doi:10.1021/j150530a018. ISSN 0022-3654..

[5]
^Gold Book IUPAC. Schulze–Hardy rule: "The generalization that the critical coagulation concentration for a typical lyophobic sol is extremely sensitive to the valence of the counter-ions (high valence gives a low critical coagulation concentration)". Source: PAC, 1972, 31, 577 (Manual of Symbols and Terminology for Physicochemical Quantities and Units, Appendix II: Definitions, Terminology and Symbols in Colloid and Surface Chemistry) on page 610..

[6]
^Gold Book IUPAC (1997). Schulze–Hardy rule. IUPAC Compendium of Chemical Terminology 2nd Edition (1997)..

[7]
^M. Y. Lin; H. M. Lindsay; D. A. Weitz; R. C. Ball; R. Klein; P. Meakin (1989). "Universality in colloid aggregation" (PDF). Nature. pp. 360–362. Bibcode:1989Natur.339..360L. doi:10.1038/339360a0..

[8]
^James, Robert O.; Homola, Andrew; Healy, Thomas W. (1977). "Heterocoagulation of amphoteric latex colloids". Journal of the Chemical Society, Faraday Transactions 1: Physical Chemistry in Condensed Phases. 73 (0): 1436. doi:10.1039/f19777301436. ISSN 0300-9599..

[9]
^Kim, Anthony Y; Hauch, Kip D; Berg, John C; Martin, James E; Anderson, Robert A (2003). "Linear chains and chain-like fractals from electrostatic heteroaggregation". Journal of Colloid and Interface Science. 260 (1): 149–159. Bibcode:2003JCIS..260..149K. doi:10.1016/S0021-9797(03)00033-X. ISSN 0021-9797..

[10]
^Gregory, John (2009). "Monitoring particle aggregation processes". Advances in Colloid and Interface Science. 147-148: 109–123. doi:10.1016/j.cis.2008.09.003. ISSN 0001-8686..

[11]
^Holthoff, Helmut; Schmitt, Artur; Fernández-Barbero, Antonio; Borkovec, Michal; Cabrerı́zo-Vı́lchez, Miguel ángel; Schurtenberger, Peter; Hidalgo-álvarez, Roque (1997). "Measurement of Absolute Coagulation Rate Constants for Colloidal Particles: Comparison of Single and Multiparticle Light Scattering Techniques". Journal of Colloid and Interface Science. 192 (2): 463–470. Bibcode:1997JCIS..192..463H. doi:10.1006/jcis.1997.5022. ISSN 0021-9797..