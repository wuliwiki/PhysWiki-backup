% 群函数
% keys 群函数|群函数空间
% license Xiao
% type Tutor

\pentry{群代数与正则表示\nref{nod_gpalg}}{nod_164b}

本节主要介绍群函数、群函数空间,阐明群函数与群表示之间的关系,并介绍类空间等概念,旨在为之后正交定理的证明打下基础。

\subsection{群函数}

相比于普通的函数,群函数与其最重要的区别在于群函数的定义域为群$G$,(注意并非群空间$V_G$)。
\begin{definition}{群函数}
由群$G$到复数域$\mathbb{C}$的一个映射$f$:$G\longmapsto \mathbb{C}$,$\forall g\in G$,$f(g)\in \mathbb{C}$被称为群上的函数,简称群函数。
\end{definition}

\begin{definition}{群函数空间}
群函数空间是所有群函数的任意线性组合所构成的线性空间。其加法规则与数乘规则均遵循复数的相应运算规则。记作$F_G=\{f(*)\}$
\end{definition}

群函数空间的一组较为自然的基底为:$v_i:f(g_\alpha)=\delta_{\alpha i}~.$

回忆群空间的矢量,事实上每一个群空间的矢量都对应着一个群函数。
$\forall v\in V_G$,$v=x_ig_i$都有群函数$f(g_i)=x_i$与之对应,这个对应关系是一个双射,所以群函数空间实际上是与群空间同构的。

还有一组重要的群函数是表示给出的群函数,设$D(g)$为群$G$的一个表示,那么有:$f_{\mu\nu}(g_i)=D_{\mu\nu}(g_i)$给出了一个群函数。这组群函数尤为重要,证明不可约不等价表示按这种方式给出的群函数的正交关系是有限群表示论中很重要的一个定理。

对于群函数空间中矢量的内积关系又如下定义:
$$(f_1(*),f_2(*))=\displaystyle\sum_{g\in G} \overline{f_1(g)}f_2(g)~.$$

这样之前的群函数空间结构便可以构成一个结合代数结构,对于李群,可以将上述求和改为积分。

\subsection{类函数}

类似群函数的定义可以定义类函数。

\begin{definition}{类函数}
由群$G$上的共轭类到复数域$\mathbb{C}$的一个映射$f$:$c^\alpha\longmapsto \mathbb{C}$,$\forall c^\alpha\subseteq G$,$f(c^\alpha)\in \mathbb{C}$被称为类上的函数,简称类函数。
\end{definition}

类似的可以定义类函数空间,并且可以证明其与类空间同构。
