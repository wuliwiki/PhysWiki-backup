% 罗素悖论(综述)
% license CCBYSA3
% type Wiki

本文根据 CC-BY-SA 协议转载翻译自维基百科\href{https://en.wikipedia.org/wiki/Russell\%27s_paradox}{相关文章}。

在数学逻辑中,罗素悖论(也称为罗素反义命题)是由英国哲学家和数学家伯特兰·罗素于1901年提出的一个集合论悖论。罗素悖论表明,任何包含不受限制的理解原理的集合论都会导致矛盾。根据不受限制的理解原理,对于任何足够明确定义的属性,都存在一个集合,包含所有且仅包含具有该属性的对象。设\(R\)为所有不属于自身的集合的集合(这个集合有时被称为“罗素集合”)。如果\(R\)不属于自身,则根据其定义,它必须属于自身;然而,如果它属于自身,那么它就不属于自身,因为它是所有不属于自身的集合的集合。由此产生的矛盾就是罗素悖论。用符号表示如下:

设 
\[
R = \{ x \mid x \notin x \}~
\]
那么
\[
R \in R \iff R \notin R~
\]
罗素还展示了该悖论的一个版本可以在德国哲学家和数学家戈特洛布·弗雷格所构建的公理化系统中推导出来,从而破坏了弗雷格试图将数学归约为逻辑的尝试,并对逻辑主义的程序提出了质疑。避免该悖论的两种有影响力的方式都在1908年提出:罗素的类型理论和泽梅洛集合论。特别是,泽梅洛的公理限制了无限理解原理。随着亚伯拉罕·弗兰克尔的进一步贡献,泽梅洛集合论发展成了现在标准的泽梅洛–弗兰克尔集合论(当包含选择公理时,通常称为ZFC)。罗素和泽梅洛解决悖论的主要区别在于,泽梅洛修改了集合论的公理,同时保持了标准的逻辑语言,而罗素则修改了逻辑语言本身。ZFC的语言,在托拉夫·斯科勒姆的帮助下,最终被证明是第一阶逻辑的语言。[4]

该悖论早在1899年就由德国数学家恩斯特·泽梅洛独立发现。[5]然而,泽梅洛并未发表这一想法,这一想法仅为大卫·希尔伯特、埃德蒙德·胡塞尔和哥廷根大学的其他学者所知。在1890年代末,现代集合论的创始人乔治·康托尔就已经意识到他的理论会导致矛盾,他通过信件告诉希尔伯特和理查德·德德金德。[6]