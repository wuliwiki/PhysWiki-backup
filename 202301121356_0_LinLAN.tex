% 搭建 Linux 局域网

\begin{issues}
\issueDraft
\end{issues}

\subsection{硬件}
\begin{itemize}
\item 交换机(managed, unmanaged)
\item 网线(cat6, 类型需要交换机支持)
\item 笔记本电脑
\end{itemize}

\subsection{wifi 路由器连接}
\begin{itemize}
\item wifi 路由器同样也支持有线连接,可以充当 switch, 还有路由器, 和 DH 什么服务器。
\item 路由器是否会改变机器的 ip? (如果机器太多应该会的)
\end{itemize}

\subsection{switch 连接}
\subsubsection{用 Windows 共享网络}
\begin{itemize}
\item 可以用 Windows 电脑来提供网络, 例如通过 wifi 接互联网, 就再控制面板中找到 view network status and tasks, 点 Wi-Fi 2 (Wifi 名称), 点最下面的 properties, 点 sharing 面板, 勾选 Allow other network users to connect through this computer's Internet connection, 然后在下面选中要分享的 Ethernet 网卡。 下面那个勾也可以打上(... control or disable the shared Internet connection)。 这样就成功了, 链接 switch 的所有电脑都可以上网了。
\item 如果重启 Windows 以后, Linux 连不上, 就把上面的 share 关闭, 连 Linux, 再开一次。
\end{itemize}

\subsubsection{基本操作}
\begin{itemize}
\item Windows 上可以用 cmd 里面的 \verb|ipconfig| 命令来查看 ip。 Linux 上用 \verb|ifconfig|。 如果 Win 上装了 WSL, 那么 WSL 里面看到的 ip 和 cmd 里面是一样的。
\item 每个机器可以自己设置自己的 ip 地址, 也可以从 DH 什么服务器获取设置, 一般好像不用管它。
\item 每台机器上, 要搜索局域网中所有设备, 用 \verb|sudo arp-scan -l --interface=网卡名| 其中 \verb|网卡名| 就是 \verb|ifconfig| 里面显示的。
\end{itemize}

\subsection{ssh 互通}
\begin{itemize}
\item 安装 \verb|openssh-server|, 设置见\upref{SSH}
\end{itemize}

\subsection{共享文件夹}
Linux 详见 “Linux 创建网络文件夹(sshfs 和 NFS)\upref{NFS}”。

Windows Server 自带 NFS, 普通 Windows 可以用第三方的 \href{https://sourceforge.net/projects/freenfs/files/latest/download}{freeNFS}