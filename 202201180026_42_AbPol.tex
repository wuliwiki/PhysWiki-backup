% 绝对极值与相对极值(变分学)
% keys 极值分类|曲线距离|邻区|强极大|弱极大

\pentry{可取曲线(变分学)\upref{DesCur}}

本节将对曲线 $\gamma$ 的函数 $J[\gamma]$ 的极值进行分类,本节内容将有助于引出下一节“变分”的概念.

\subsection{绝对极值}
定义在某可取曲线族中的曲线函数 $J[\gamma]$ 的\textbf{绝对极小值}由此族中曲线 $\gamma_0$ 实现,如果对于这族中任一曲线 $\gamma$ 有
\begin{equation}
J[\gamma]\geq J[\gamma_0]
\end{equation}
同样,可定义绝对极大值.
\subsection{相对极值}
\begin{figure}[ht]
\centering
\includegraphics[width=6cm]{./figures/AbPol_1.pdf}
\caption{相对极值示意图} \label{AbPol_fig1}
\end{figure}

先用一个直观的例子来理解相对极值的概念.如图,从 $A$ 点到 $B$ 点,有几条路可走,水平线上方和下方各有两条路,上方的最短路线 $S_0$ 小于下方的最短路线 $D_0$ .那么,上方路线 $S_0$ 就是连接 $A,B$ 两点的绝对极小值.下方最短路线 $D_0$ 虽然不是绝对极小值,然而它却比连接这两点下方的其它路线都短. 