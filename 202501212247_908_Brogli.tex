% 路易·德布罗意(综述)
% license CCBYSA3
% type Wiki

本文根据 CC-BY-SA 协议转载翻译自维基百科\href{https://en.wikipedia.org/wiki/Louis_de_Broglie}{相关文章}。

\begin{figure}[ht]
\centering
\includegraphics[width=6cm]{./figures/34cce1e349a4d7bf.png}
\caption{德布罗意在1929年} \label{fig_Brogli_1}
\end{figure}
路易·维克托·皮埃尔·雷蒙德,第七代布罗意公爵(法语:[də bʁɔj] 或 [də bʁœj],1892年8月15日-1987年3月19日)是法国物理学家和贵族,他对量子理论做出了开创性贡献。在他1924年的博士论文中,他假设了电子的波动性质,并提出所有物质都有波动特性。这个概念被称为德布罗意假设,是波粒二象性的一个例子,并成为量子力学理论的核心部分。

德布罗意于1929年获得诺贝尔物理学奖,因为物质的波动行为在1927年首次得到了实验验证。

德布罗意发现的粒子波动行为被厄尔温·薛定谔用在他提出的波动力学中。德布罗意的导波概念于1927年在索尔维会议上提出,随后被放弃,转而支持量子力学,直到1952年被大卫·玻姆重新发现并加以完善。

路易·德布罗意于1944年当选为法兰西学院第16位成员,担任法兰西科学院的终身秘书。德布罗意是第一位呼吁建立多国实验室的高级科学家,这一提议最终促成了欧洲核子研究组织(CERN)的成立。
\subsection{传记}  
\subsubsection{家庭与教育}
路易·德布罗意出身于著名的布罗意贵族家族,几百年来,该家族的成员在法国担任重要的军事和政治职务。未来物理学家的父亲路易-阿尔方斯-维克多,第五代布罗意公爵,娶了波琳·达尔梅伊尔,她是拿破仑时代将军菲利普·保尔·塞吉尔伯爵的孙女,而塞吉尔伯爵的妻子是传记作家玛丽·塞勒斯廷·阿梅丽·达尔梅伊尔。他们有五个孩子,除了路易,还有:阿尔贝蒂娜(1872–1946),后来成为卢佩侯爵夫人;莫里斯(1875–1960),后成为著名的实验物理学家;菲利普(1881–1890),在路易出生前两年去世;波琳,潘日女伯爵(1888–1972),后成为著名作家。

路易·德·布罗意出生于法国塞纳-马尔姆地区的迪耶普。作为家中的小儿子,路易在相对孤独的环境中长大,阅读了大量书籍,并且特别喜欢历史,尤其是政治历史。从小他记忆力极好,能准确地背诵剧本中的片段,或者列出法兰西第三共和国的所有内阁部长。因此,人们预测他将来会成为一位伟大的政治家。

德·布罗意原本打算从事人文学科的职业,并获得了历史学的学士学位。此后,他转向数学和物理学,并获得了物理学的学位。第一次世界大战爆发后,他主动为军队提供服务,参与了无线电通讯的开发。
\subsubsection{军服务}
毕业后,路易·德·布罗意加入了工程部队,开始了强制性服役。服役开始于蒙·瓦莱里安堡,但不久后,在他哥哥的提议下,他被调到无线电通讯服务,并在埃菲尔铁塔工作,那里有无线电发射机。路易·德·布罗意在第一次世界大战期间一直服役,主要处理技术性问题。特别是,他与莱昂·布里渥因(Léon Brillouin)和哥哥莫里斯一起,参与了与潜艇的无线通信建设。路易·德·布罗意于1919年8月退役,晋升为上士。后来,这位科学家遗憾地表示,他不得不离开自己真正感兴趣的基础科学问题约六年之久。
\subsubsection{科学与教学生涯}  
他的1924年论文《Recherches sur la théorie des quanta》(《量子理论研究》)提出了他的电子波理论。这个理论包括物质的波粒二象性理论,基于马克斯·普朗克和阿尔伯特·爱因斯坦关于光的研究。这项研究最终得出了德布罗意假说,指出任何运动的粒子或物体都有一个相关的波。德布罗意由此创造了物理学的新领域——波动力学(mécanique ondulatoire),将能量(波)和物质(粒子)的物理学统一在一起。他因发现电子的波动性质而获得了1929年诺贝尔物理学奖。

在他后来的职业生涯中,德布罗意致力于发展波动力学的因果解释,反对主导量子力学理论的完全概率模型;该理论在1950年代由大卫·玻姆进一步完善。此理论后来被称为德布罗意–玻姆理论。

除了严格的科学工作,德布罗意还思考并写作关于科学哲学的内容,包括现代科学发现的价值。1930年,他创办了由埃尔曼出版社出版的书籍系列《Actualités scientifiques et industrielles》(《科学与工业新闻》)。

德布罗意于1933年成为法国科学院院士,并于1942年起担任该院的永久秘书。他曾被邀请加入法国天主教科学家联合会(Le Conseil de l'Union Catholique des Scientifiques Francais),但因他无宗教信仰而拒绝了邀请。1941年,他成为维希法国的国家委员会成员。1944年10月12日,他当选为法国科学院院士,接替数学家埃米尔·皮卡尔。由于在占领期间科学院成员的死亡和监禁以及战争带来的其他影响,学院无法达到选举所需的20名成员法定人数;然而,鉴于特殊情况,17名到场成员的一致选举被接受。在这一事件中,历史上唯一一次,由他自己的兄弟莫里斯(1934年当选)接待他成为新成员。

联合国教科文组织于1952年授予他首届卡林加奖,以表彰他在普及科学知识方面的贡献,1953年4月23日,他被选为皇家学会外籍会员。

1960年,随着他的长兄莫里斯(第六代德布罗意公爵)去世且无继承人,路易成为第七代德布罗意公爵,继承了家族的头衔。

1961年,他被授予法国荣誉军团大十字勋章骑士称号。德布罗意于1945年被任命为法国原子能高级委员会顾问,以表彰他为促进工业与科学的紧密合作所做的努力。他在亨利·庞加莱研究所建立了一个应用力学中心,开展光学、控制论和原子能方面的研究。他激励了国际量子分子科学学会的成立,并成为其早期成员。

路易一生未婚。1987年3月19日,他在卢韦西安斯去世,享年94岁。他的公爵头衔由远房表亲维克多-弗朗索瓦(第八代德布罗意公爵)继承。他的葬礼于1987年3月23日在圣皮埃尔-德-诺伊耶教堂举行。
\subsection{科学活动}
\subsubsection{X射线和光电效应的物理学}
路易·德布罗意的第一批研究(20世纪20年代初)是在他哥哥莫里斯的实验室进行的,涉及光电效应的特点和X射线的性质。这些研究探讨了X射线的吸收现象,并使用玻尔理论对这一现象进行了描述,应用量子原理解释了光电子光谱,并给出了X射线光谱的系统分类。这些X射线光谱的研究对于阐明原子内电子壳层的结构非常重要(光学光谱由外层壳层决定)。因此,与亚历山大·多维利耶(Alexandre Dauvillier)共同进行的实验结果揭示了现有电子分布模型的不足;这些困难后来由埃德蒙·斯通(Edmund Stoner)解决。另一个结果是阐明了索末菲公式在确定X射线光谱中线的位置时的不足;这种差异在发现电子自旋后得以解决。1925年和1926年,列宁格勒物理学家奥列斯特·赫沃尔松(Orest Khvolson)提名德布罗意兄弟因其在X射线领域的工作而获得诺贝尔奖。
\subsubsection{物质与波粒二象性}  
在研究X射线辐射的性质并与哥哥莫里斯讨论这些射线的特性时,莫里斯认为这些射线是一种波和粒子的结合,这促使路易·德布罗意意识到需要建立一个将粒子和波的表现形式联系起来的理论。此外,他还了解了马塞尔·布里渊(Marcel Brillouin)在1919到1922年间提出的原子水动力学模型,并试图将其与玻尔理论的结果联系起来。路易·德布罗意工作的起点是爱因斯坦关于光量子的观点。在他1922年发表的第一篇关于这一主题的文章中,法国科学家将黑体辐射视为光量子气体,并通过经典统计力学,在这种表述框架下推导出了维恩辐射定律。在他的下一篇出版物中,他试图将光量子的概念与干涉和衍射现象结合起来,得出结论认为必须将某种周期性与量子联系起来。在这种情况下,光量子被他解释为具有非常小质量的相对论粒子。[25]