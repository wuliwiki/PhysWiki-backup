% Python 创建模块笔记

\subsection{区分脚本和模块}
有时候我们希望同一个 \verb|.py| 文件既可以作为脚本直接执行也可以作为模块被导入, 并且该文件可以根据这两种方式自动选择执行什么. 这是可以用 \verb|__name__| 来判断:
\begin{lstlisting}[language=python, caption=my\_module.py]
#!/usr/bin/python3
# 执行一些命令(无论作为脚本还是模块都会被运行)
# 被赋值的变量会作为模块的全局变量, 定义的函数会作为模板中的函数

print('欢迎使用 my_module.py')

def plus1(num):
    return num + 1

num = 3.5
num2 = plus1(num)


if __name__ == '__main__':
    # 脚本模式下运行的命令
    print('正在被作为脚本执行,  __name__ 的值为 __main__')
else:
    # 作为模块导入时运行的命令
    print('正在被作为模块导入, __name__ 的值为', __name__)
\end{lstlisting}
