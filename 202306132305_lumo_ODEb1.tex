% 线性微分方程的一般理论
% 常微分方程|ODE|differential equation|常数变易法|朗斯基行列式|齐次|解的结构

\pentry{一阶常微分方程解法:常数变易法\upref{ODEa2}}

\begin{definition}{}
设各 $a_i(x)$ 和 $f(t)$ 都是区间 $[a, b]$ 上的连续函数,则称
\begin{equation}\label{eq_ODEb1_1}
\qty(\frac{\mathrm{d}^n y}{\dd x^n})+a_1\qty(\frac{\mathrm{d}^{n-1} y}{\dd x^{n-1}})+a_2\qty(\frac{\mathrm{d}^{n-2} y}{\dd x^{n-2}})+\cdots+a_ny=f(x)~
\end{equation}
为区间 $[a, b]$ 上的\textbf{线性微分方程(linear differential equation)}。

当 $f(x)=0$ 时,称\autoref{eq_ODEb1_1} 为\textbf{齐次的(homogeneous)},否则称为\textbf{非齐次的(inhomogeneous)}。
\end{definition}

任取 $x_0\in[a, b]$,如果已知该点处 $y, \qty(\frac{\mathrm{d} y}{\dd x}), \qty(\frac{\mathrm{d}^{2} y}{\dd x^{2}}), \cdots, \qty(\frac{\mathrm{d}^{n-1} y}{\dd x^{n-1}})$ 的 $n$ 个值,那么我们可以唯一确定\autoref{eq_ODEb1_1} 满足这些初值的特解。这一解的存在与唯一性定理将在线性方程组相关章节讨论。

\addTODO{引用相关的存在与唯一性定理。}

\subsection{齐次线性微分方程的解的性质与结构}

区间 $[a, b]$ 上的齐次线性方程形如
\begin{equation}\label{eq_ODEb1_2}
\qty(\frac{\mathrm{d}^n y}{\dd x^n})+a_1\qty(\frac{\mathrm{d}^{n-1} y}{\dd x^{n-1}})+a_2\qty(\frac{\mathrm{d}^{n-2} y}{\dd x^{n-2}})+\cdots+a_ny=0~.
\end{equation}

\begin{theorem}{解的线性叠加原理}
如果 $y_1(x), y_2(x), \cdots, y_k(x)$ 是\autoref{eq_ODEb1_2} 的 $k$ 个解,而 $c_1, c_2, \cdots, c_k$ 是 $k$ 个常数,那么 $c_1y_1(x)+c_2y_2(x)+\cdots+c_ky_k(x)$ 也是一个解。
\end{theorem}

证明很简单,利用求导的线性性即可。

线性叠加原理,就是解进行“数乘”和“加法”后仍为解,意味着\autoref{eq_ODEb1_2} 的全体解构成了一个线性空间,称为\textbf{解空间}。如果我们能找到这个线性空间的一组基,也就相当于了解了整个线性空间。这样的基向量,被称为\textbf{基解},一切解都可以表示为它们的线性组合。基解构成的集合,被称为\textbf{基本解组}。

\begin{definition}{线性相关性}
对于区间 $[a, b]$ 上的函数 $y_i(x)$,如果存在一组\textbf{不全为零}的常数 $c_i$,使得 $c_1y_1(x)+\cdots+c_ky_k(x)=0$,那么称 $y_1, \cdots, y_k$ 是\textbf{线性相关}的,否则是\textbf{线性无关}的。
\end{definition}

\begin{theorem}{}\label{the_ODEb1_1}
如果 $y_1(x), y_2(x), \cdots, y_n(x)$ 是\autoref{eq_ODEb1_2} 的 $n$ 个线性无关的解,那么它们构成了一组基解。
\end{theorem}

\autoref{the_ODEb1_1} 证明思路提示:我们已经知道,\autoref{eq_ODEb1_2} 的每个解由 $n$ 个初值唯一确定。在初值点 $x_0$,给定初值条件 $y(x_0), \qty(\frac{\mathrm{d} y}{\dd x})(x_0), \qty(\frac{\mathrm{d}^{2} y}{\dd x^{2}})(x_0), \cdots, \qty(\frac{\mathrm{d}^{n-1} y}{\dd x^{n-1}})(x_0)$,如果 $c_1y_1(x)+\cdots+c_ny_n(x)$ 满足全部 $n$ 个初值条件,那么
\begin{equation}\label{eq_ODEb1_3}
\leftgroup{
&c_1y_1(x_0)+\cdots+c_ny_n(x_0)=y(x_0)\\
&c_1y_1'(x_0)+\cdots+c_ny_n'(x_0)=y'(x_0)\\
&\phantom{123456789012345678}\vdots\\
&c_1\qty(\frac{\mathrm{d}^{n-1} y_1}{\dd x^{n-1}})(x_0)+\cdots+c_n\qty(\frac{\mathrm{d}^{n-1} y_n}{\dd x^{n-1}})(x_0)=y(x_0)~.
}
\end{equation}
而由于 $y_1(x), y_2(x), \cdots, y_n(x)$ 在给定区间上是线性无关的,\autoref{eq_ODEb1_3} 的各方程也就彼此独立
\footnote{\autoref{eq_ODEb1_3} 是关于未知量 $c_i$ 的代数方程组。如果 $\{y_i\}$ 线性相关,那么至少有一个 $y_i$ 可以表示为其它几个 $y_i$ 的线性组合,而 $y'_i, y''_i$ 等也可以表示为相同的线性组合,这就使得\autoref{eq_ODEb1_3} 各方程中至少有一个能被其它方程表示出来,从而不独立。由于初值条件的任意性,这就会导致多数初值条件下\autoref{eq_ODEb1_3} 无法解出 $\{c_i\}$。反过来,$\{y_i\}$ 线性无关,那么\autoref{eq_ODEb1_3} 各方程就彼此独立。}
,因此能够解出唯一的常数组 $\{c_i\}$。“能够解出”意味着“能够把满足初值条件的解表示成 $c_1y_1(x)+\cdots+c_ny_n(x)$”。

如何判断一组 $y_1(x), y_2(x), \cdots, y_n(x)$ 在给定区间 $[a, b]$ 上的线性相关性呢?我们需要借助以下\textbf{朗斯基行列式}。

\begin{definition}{朗斯基行列式}\label{def_ODEb1_1}
对于区间 $[a, b]$ 上存在 $k-1$ 次导函数的 $k$ 个函数 $f_1, f_2, \cdots, f_k$,定义在 $[a, b]$ 上的函数 $W[f_1, f_2, \cdots, f_k](x)$ 为如下行列式:
\begin{equation}
W[f_1, f_2, \cdots, f_k](x)=\vmat{f_1&f_2&\cdots&f_k\\f'_1&f'_2&\cdots&f'_k\\ \vdots&\vdots&\ddots&\vdots\\f_1^{(k-1)}&f_2^{(k-1)}&\cdots&f_k^{(k-1)}} (x)~,
\end{equation}
称之为给定函数组的\textbf{朗斯基行列式(Wronskian)}。
\end{definition}


朗斯基行列式,本质上就是\autoref{eq_ODEb1_3} 的系数矩阵行列式。\autoref{eq_ODEb1_3} 解的存在性和 $\{y_i\}$ 的线性相关性紧密相连,因此我们有以下定理:




\begin{theorem}{}\label{the_ODEb1_2}
对于 $[a, b]$ 上的函数 $f_i(x)$,如果它们在给定区间上线性相关,那么 $W[f_1, f_2, \cdots, f_k](x)$ 在 $[a, b]$ 上恒为 $0$。
\end{theorem}

注意,\autoref{the_ODEb1_2} 的逆定理(W恒为 $0$ 则 $\{f_i\}$ 线性相关)不成立。我们试举一例反例来说明:

\begin{example}{}
考虑 $\mathbb{R}$ 上的函数:
\begin{equation}
f_1(x)=\leftgroup{
    0 \quad(x<0)\\
    \E^{-\frac{1}{x}} \quad (x\geq 0)~.
}
\end{equation}
和
\begin{equation}
f_2(x)=\leftgroup{
    \E^{-\frac{1}{x}} \quad(x<0)\\
    0\quad(x\geq 0)~.
}
\end{equation}

简单计算可得 $W[f_1, f_2]$ 在 $\mathbb{R}$ 上恒为 $0$,但 $f_1, f_2$ 在 $\mathbb{R}$ 上线性无关,而且它们还是无穷次可导的函数,是绝佳的反例。




\end{example}

\begin{theorem}{}\label{the_ODEb1_3}
如果 $y_1(x), y_2(x), \cdots, y_n(x)$ 是\autoref{eq_ODEb1_2} 的 $n$ 个解且它们线性无关,那么它们的朗斯基行列式在 $[a, b]$ 上处处不为 $0$。
\end{theorem}

\autoref{the_ODEb1_3} 的证明思路提示:反设存在某 $x_0\in[a, b]$,使得上述朗斯基行列式为 $0$,那么在 $x_0$ 处的初值条件无法总是给出\autoref{eq_ODEb1_3} 的解,这与解的存在性矛盾。




\subsection{非齐次线性微分方程}

容易验证,\textbf{非齐次}方程\autoref{eq_ODEb1_1} 的解 ${\phi}_1(x), \phi_2(x)$ 和\textbf{齐次}方程\autoref{eq_ODEb1_2} 的解 $\overline{\phi}(x)$ 有如下关系:
\begin{itemize}
\item $\phi_1(x)+\overline{\phi}(x)$ 是非齐次方程\autoref{eq_ODEb1_1} 的解;
\item $\phi_1(x)-\phi_2(x)$ 是齐次方程\autoref{eq_ODEb1_2} 的解。
\end{itemize}

因此,我们容易得到以下定理
\begin{theorem}{非齐次方程的解集}
设 $\phi_1(x), \phi_2(x), \cdots, \phi_n(x)$ 为\autoref{eq_ODEb1_2} 的一组基解,$\varphi(x)$ 为\autoref{eq_ODEb1_1} 的一个特解,那么\autoref{eq_ODEb1_1} 的通解可以表示为
\begin{equation}
c_0\varphi(x)+c_1\phi_1(x)+c_2\phi_2(x)+\cdots+c_n\phi_n(x)~.
\end{equation}
\end{theorem}

知道了\autoref{eq_ODEb1_2} 的基本解组以后,我们可以用\textbf{常数变易法}来解出\autoref{eq_ODEb1_1} .

假设 $\phi_1(x), \phi_2(x), \cdots, \phi_n(x)$ 为\autoref{eq_ODEb1_2} 的一组基解,那么\autoref{eq_ODEb1_2} 的任意一个解都可以表示为 $c_1\phi_1(x)+c_2\phi_2(x)+\cdots+c_n\phi_n(x)$ 的形式。现在我们把各常数 $c_i$ 写成待定函数 $c_i(x)$,使得
\begin{equation}
c_1(x)\phi_1(x)+c_2(x)\phi_2(x)+\cdots+c_n(x)\phi_n(x)~
\end{equation}
是\autoref{eq_ODEb1_1} 的解。

\begin{example}{非齐次线性微分方程的常数变易法}\label{ex_ODEb1_1}
考虑方程
\begin{equation}\label{eq_ODEb1_4}
\frac{\mathrm{d}^2 y}{\dd x^2}+y=\tan x~,
\end{equation}
且已知其齐次形式的基本解组为 $\{\cos x, \sin x\}$。

设\autoref{eq_ODEb1_4} 的通解为
\begin{equation}\label{eq_ODEb1_5}
y=c_1(x)\cos x+c_2(x)\sin x~,
\end{equation}
将\autoref{eq_ODEb1_5} 代入\autoref{eq_ODEb1_4} 得到
\begin{equation}\label{eq_ODEb1_6}
c''_1(x)\cos x+c''_2(x)\sin x-c'_1(x)\sin x+c'_2(x)\cos x=\tan x~.
\end{equation}

\autoref{eq_ODEb1_6} 有两个未知函数,所以没有唯一确定的解。我们可以有多种方法指定第二个约束式子,实践中当然是怎么方便计算怎么来。

通常,我们令
\begin{equation}\label{eq_ODEb1_8}
c'_1(x)\cos x+c'_2(x)\sin x=0~,
\end{equation}
这就意味着 $\frac{\dd y}{\dd x}=-c_1(x)\sin x+c_2(x)\cos x$,进而
\begin{equation}\label{eq_ODEb1_7}
\frac{\mathrm{d}^2 y}{\dd x^2}=-c_1(x)\cos x-c_2(x)\sin x-c'_1(x)\sin x+c'_2(x)\cos x~.
\end{equation}

重新将\autoref{eq_ODEb1_7} 代回\autoref{eq_ODEb1_4} 和\autoref{eq_ODEb1_5} 得到
\begin{equation}\label{eq_ODEb1_9}
-c'_1(x)\sin x+c'_2(x)\cos x=\tan x~.
\end{equation}

联立\autoref{eq_ODEb1_8} 和\autoref{eq_ODEb1_9} ,解得
\begin{equation}
\leftgroup{
c'_1(x)&=\sin^2x/\cos x\\
c'_2(x)&=\sin x~.
}
\end{equation}
解出 $c_1$ 和 $c_2$:
\begin{equation}
\leftgroup{
    c_1(x)&=C_1-\sin x+\ln(\frac{\cos\frac{x}{2}+\sin\frac{x}{2}}{\cos\frac{x}{2}-\sin\frac{x}{2}})\\
    c_2(x)&=C_2-\cos x~,
}
\end{equation}
其中 $C_1, C_2$ 是积分常数。

代回\autoref{eq_ODEb1_5} ,整理后可得\autoref{eq_ODEb1_4} 的通解为
\begin{equation}
y=C_1\cos x+C_2\sin x-\sin 2x+\cos x\ln(\frac{\cos\frac{x}{2}+\sin\frac{x}{2}}{\cos\frac{x}{2}-\sin\frac{x}{2}})~.
\end{equation}




\end{example}


\autoref{ex_ODEb1_1} 中关键的简化步骤是规定了 $c'_1(x)\cos x+c'_2(x)\sin x=0$。事实上,同一个思路在用常数变易法解非齐次方程时很常用,因为它能避免出现 $c_i(x)$ 的高阶导函数。
















