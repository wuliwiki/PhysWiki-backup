% 域的扩张
% keys 域|单扩张|超越扩域|最小多项式|超越元|代数元|扩张次数

\pentry{域\upref{field}}
%%未完成

\subsection{域的单扩张}
如果在一个域中添加不属于域集合的元素,我们可以得到一个更大的集合.要让这个新集合成为域,我们就得定义新元素和原来域中元素相加和相乘的结果;无论怎么定义,这个结果必须满足域的公理.如果在集合中任何元素都无法成为某个运算结果,那么我们就必须再引入新的元素来作为这个结果.以此类推,不停地添加新元素,直到最后不需要添加新元素了,那最后这个集合就是一个新的域,它包含了原来的域.这个域是原来的域的\textbf{扩张},并且包含最初那个新元素的最小的域,因此被称为\textbf{单元素扩张},简称\textbf{单扩张}.

域的单扩张具体是怎么进行的呢?我们将从例子开始说明,最后引入域的单扩张的定义.

\begin{exercise}{有理数域的 $\sqrt{2}$ 扩张}\label{FldExp_exe1}
给定有理数域 $\mathbb{Q}$,则 $\sqrt{2}$ 并不是 $\mathbb{Q}$ 的元素.将 $\sqrt{2}$ 添加进去,那么由于 $\sqrt{2}$ 是实数域 $\mathbb{R}$ 的元素,这提示我们可以把新的运算结果按照 $\mathbb{R}$ 中的运算来定义.这样,$\mathbb{Q}$ 中添加 $\sqrt{2}$ 的单扩域的集合就是 $\{a+b\sqrt{2}|a, b\in\mathbb{Q}\}$.

请验证,任意 $a+b\sqrt{2}$ 都必须在扩域中,而这个集合也满足了加法和乘法的封闭性,所以构成了一个域;换句话说,这个集合是包含 $\sqrt{2}$ 和全体有理数的最小的域.
\end{exercise}

\begin{exercise}{超越扩域}\label{FldExp_exe2}
依然给定有理数域 $\mathbb{Q}$,这次添加的元素是 $\pi$.那么扩域应当是
\begin{equation}
\{\frac{\sum_{i=0}^N a_i\pi^i}{\sum_{j=0}^M b_j\pi^j}|a_i, b_j\in\mathbb{Q}, b_j\text{不全为零,} N, M\in\mathbb{Z}^+\}
\end{equation}
也就是 $\pi$ 的所有有理系数多项式的分式构成的集合,其中分母不是 $0$.

验证这是一个域,并且是包含全体有理数和 $\pi$ 的最小域.注意 $a_i$ 全为 $0$ 的情况意味着什么.

这是 $\mathbb{Q}$ 中添加 $\pi$ 后的单扩张.
\end{exercise}

观察这两个域扩张的特点,我们容易发现,由于加法和乘法的封闭性,添加某个元素后的单扩张,必须包含这个元素的所有多项式,系数取自域中;由于乘法逆元存在性,这些多项式的倒数也得包含进去.因此,域的单扩张可以简单理解为把新元素的多项式都添加进去的过程.

但是两个场景是不太一样的.$\mathbb{Q}$ 中添加 $\pi$ 的过程中,包含了所有 $\pi$ 的有理系数多项式,比如$3.5\pi^4+2\pi^2+0.1\pi-1/3$,但是添加 $\sqrt{2}$ 的时候却只需要最多 $1$ 阶的多项式,像$3.5(\sqrt{2})^3+1$就可以改写为$7\sqrt{2}+1$.这是因为 $\sqrt{2}$ 的 $2$ 次方是一个有理数,落入了 $\mathbb{Q}$ 中,因此 $\sqrt{2}$ 的 $2$ 阶及以上的有理系数多项式都可以表示为最多 $1$ 阶的多项式.

换一种更方便的表述,那就是 $x^2-2$ 这个多项式是阶数最小的能将 $\sqrt{2}$ 映射为 $0$ 的多项式.这就引出\textbf{最小多项式}的概念.

\begin{definition}{最小多项式}
给定一个环 $R$ 和任意元素 $x_0$,其中 $x_0$\textbf{不一定}是 $R$ 中的元素.记
$$R[x_0]=\{\sum_{i=0}^N r_ix_0^i|r_i\in R, N\in\mathbb{Z}^+\}$$称 $R[x_0]$ 是 $x_0$ 在 $R$ 中的\textbf{多项式环}.在 $R[x_0]$ 中,如果存在一个多项式 $f(x_0)=0$,并且任何更低阶的多项式的值都不是 $0$,那么称 $f(x_0)$ 是 $x_0$ 在 $R$ 上的\textbf{最小多项式(minimal polynomial)},或\textbf{极小多项式}.使不存在值为 $0$ 的多项式(这也意味着所有多项式的值甚至都不在 $R$ 中),那么我们说,$x_0$ 在 $R$ 中\textbf{不具有最小多项式}.
\end{definition}

在\autoref{FldExp_exe1} 和\autoref{FldExp_exe2} 中,$\pi$ 和 $\sqrt{2}$ 的本质区别就是:在 $\mathbb{Q}$ 中,$\pi$ 没有最小多项式,但是 $\sqrt{2}$ 具有最小多项式 $x^2-2$.这决定了同一个域 $\mathbb{Q}$ 分别添加 $\pi$ 和 $\sqrt{2}$ 后的不同单扩张.

可是,我们把域抽象地看成满足一定条件的集合时,和所有集合一样,域中的元素叫什么名字并不重要.对于\autoref{FldExp_exe1} 和\autoref{FldExp_exe2} 中的情况,我们其实是借用了 $\mathbb{R}$ 中已有的元素名称来引出两个不同的单扩域的.如果不进行这样的类比引出,而只是单纯地说在 $\mathbb{Q}$ 中加入某个新的元素 $x_0$ 来进行单扩张,那么结果可能是 $\{a+bx_0|a, b\in\mathbb{Q}\}$,可能是 $\{\sum_{i=0}^N a_ix_0^i|a_i\in\mathbb{Q}, N\in\mathbb{Z}^+\}$,可能是 $\{a+bx_0+cx_0^2|a, b, c\in\mathbb{Q}\}$,甚至还可能形式上也是 $\{a+bx_0|a, b\in\mathbb{Q}\}$,只不过 $x_0^2=2/3$ 了.这时该怎么描述域的单扩张呢?

答案是应用最小多项式对单扩张的决定作用.

\begin{definition}{域的单扩张}\label{FldExp_def1}
给定域 $\mathbb{F}$ 和一个元素 $x_0$,其中 $x_0$ 不一定在 $\mathbb{F}$ 中.定义 $\mathbb{F}$ 添加 $x_0$ 的\textbf{单扩张}是集合
\begin{equation}
\{\frac{\sum_{i=0}^N a_ix_0^i}{\sum_{j=0}^M b_jx_0^j}|a_i, b_j\in\mathbb{F},\quad\sum_{j=0}^M b_jx_0^j\not=0,\quad N, M\in\mathbb{Z}^+\}
\end{equation}
记为 $\mathbb{F}(x_0)$;容易验证,这是一个域.
\end{definition}

在\autoref{FldExp_def1} 中,如果 $x_0$ 在 $\mathbb{F}$ 中没有最小多项式,那么 $\mathbb{F}(x_0)$ 中的各元素彼此不同,此时我们称 $x_0$ 是 $\mathbb{F}$ 的\textbf{超越元(transcendental element)}; 

如果 $x_0$ 在 $\mathbb{F}$ 中有最小多项式 $f(x)$,那么称 $x_0$ 是 $\mathbb{F}$ 的\textbf{代数元(algebraic element)},此时在 $\mathbb{F}(x_0)$ 中,若记 $g, h$ 为任意多项式,就有 $g(x_0)$ 和 $g(x_0)+f(x_0)h(x_0)$ 的值相等.这提示我们,在这种情形下,似乎可以把 $\mathbb{F}(x_0)$ 看成是某个超越元 $x$ 的单扩张 $\mathbb{F}(x)$ 集合中,把 $g(x)$ 和 $g(x)+f(x)h(x)$ 认为是等价的,所得到的\textbf{商集}.实际上,严格来说不是商集,而是商集中再把类似$$\frac{\sum_{i=0}^N a_ix_0^i}{f(x)h(x)}$$的元素剔除以后剩下的集合.当然了,$0$ 不要剔除.

\subsection{域的扩张次数}

以上,我们讨论了如何添加一个元素,并根据元素的最小多项式来得到域的单扩张.对一个域进行单扩张以后,再进行单扩张,如此反复,我们可以获得域的多次扩张.


