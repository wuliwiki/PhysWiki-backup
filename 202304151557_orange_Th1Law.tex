% 热力学第一定律
% 热力学第一定律|能量守恒|做功|传热|内能

\begin{issues}
\issueDraft
\end{issues}

\pentry{压力体积图\upref{PVgraf}, 理想气体内能\upref{IdgEng}}

\begin{figure}[ht]
\centering
\includegraphics[width=8cm]{./figures/281a25e352bd0464.pdf}
\caption{热力学第一定律} \label{fig_Th1Law_1}~.
\end{figure}

\begin{theorem}{热力学第一定律}
系统的\textbf{内能} $U$ 增加等于外部对系统传递的热量 $Q$(流入系统为正) 减去系统对外做功 $W$(系统对外做功为正):
\begin{equation}\label{eq_Th1Law_1}
\Delta U = Q - W
\end{equation}
%我记得这个公式有一个适用条件?%
热力学第一定律写成微分形式是
\begin{equation} \label{eq_Th1Law_2}
\dd U = \delta Q - \delta W
\end{equation}
如果仅有体积功,那么
\begin{equation}
\dd U = \delta Q - P \dd V
\end{equation}
\end{theorem}

\footnote{本文参考自朱文涛《简明物理化学》}热力学第一定律是\textbf{能量守恒}在热力学中的体现,在经典热力学中是一条公理。热力学第一定律的另一种表述是:\textbf{第一类永动机}是不可能的。
\addTODO{什么是第一类永动机?链接到永动机词条}

\subsection{内能、功、热}
\textbf{内能U} 指系统包括的总能量,包括(但不限于)系统内分子的平动动能、转动动能、分子间的相互作用能等,是一个状态量。系统的内能有时也用 $E$ 表示。 

\textbf{热Q} 是指系统与环境由于温度差而引起的能量转移,是通过微观粒子的无规则相互作用传递的,是一个过程量。

\textbf{功W\upref{Fwork}} 是指除热之外的其他能量传递形式,一般是由宏观的作用力和宏观位移产生的,例如机械功等于压力乘以体积变化、表面功等于表面张力乘以表面积变化等,也是一个过程量。

\begin{example}{膨胀的气体}

对于活塞容器中克服外界压力膨胀的气体, 把外界压强记为 $P$, 系统体积记为 $V$, 那么系统克服外压、对外做的功可以写成:
\begin{equation}
W = \int_{V_1}^{V_2} P \dd{V}
\end{equation}
\end{example}

\subsection{状态量与过程量}
\pentry{状态量和过程量\upref{StaPro}}
内能 $U$ 只和系统的状态有关, 被称为\textbf{状态量}\upref{StaPro}。 所以 $\Delta U$ 也只与系统的初始和最终的状态有关, 与中间的过程无关。系统发生微小变动时,状态量的微小变化一般记为 $\dd X$。

$Q$ 和 $W$ 和系统变化的过程本身有关,被称为\textbf{过程量}\upref{StaPro}, 也就是说即使系统的初末状态确定, 系统变化的具体过程不一样,也会导致它们的不同。系统发生微小变动时,过程量一般记为 $\delta X$。

% \addTODO{写一些理想气体的例题, 例如 PV 图种, 计算两点间延着不同轨迹的热量}

% 对理想气体\upref{Igas}, 令分子自由度为 $i$, 有
% \begin{equation}
% E = \frac{i}{2}n RT
% \end{equation}
