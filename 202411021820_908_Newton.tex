% 艾萨克·牛顿(综述)
% license CCBYSA3
% type Wiki

本文根据 CC-BY-SA 协议转载翻译自维基百科\href{https://en.wikipedia.org/wiki/Isaac_Newton}{相关文章})

\begin{figure}[ht]
\centering
\includegraphics[width=6cm]{./figures/0c83e6f3dfbe0a8c.png}
\caption{《46岁的牛顿肖像,1689年》} \label{fig_Newton_1}
\end{figure}
艾萨克·牛顿爵士,皇家学会会员(1642年12月25日-1726/27年3月20日[a]),是一位英国博学家,活跃于数学、物理学、天文学、炼金术、神学和写作领域,在他所在的时代被称为自然哲学家。他是科学革命及其后的启蒙运动中的关键人物。他的开创性著作《自然哲学的数学原理》首次出版于1687年,汇集了许多前人的研究成果,奠定了经典力学的基础。牛顿还在光学方面做出了开创性的贡献,并与德国数学家戈特弗里德·威廉·莱布尼茨共同被认为是微积分的创立者,尽管他在莱布尼茨之前几年就已发展了微积分。[10][11]

在《自然哲学的数学原理》中,牛顿制定了运动定律和万有引力定律,这些理论成为数个世纪以来主导性的科学观点,直到相对论的出现。他利用对重力的数学描述推导了开普勒的行星运动定律,解释了潮汐、彗星轨迹、岁差等现象,消除了关于太阳系日心说的疑虑。他展示了地球上的物体和天体的运动可以由相同的原理解释。牛顿推测地球为扁球体,这一推测后来由莫佩尔蒂、拉康达米娜等人的测地测量所证实,使得大多数欧洲科学家信服于牛顿力学的优越性。

他制造了第一个实用的反射望远镜,并基于棱镜将白光分解为可见光谱的颜色的观察,发展出一套精细的颜色理论。他关于光的研究汇集于其极具影响力的著作《光学》中,1704年出版。他提出了一个经验性的冷却定律,这是第一个热传导的表述,首次对声速进行了理论计算,并引入了牛顿流体的概念。此外,他还对电进行了早期研究,他在《光学》一书中的一个设想可以说是电场理论的开端。作为数学家,除了微积分的研究外,他还对幂级数进行了研究,将二项式定理推广至非整数指数,发展出求解函数根的方法,并分类了大部分的三次平面曲线。

牛顿是剑桥大学三一学院的成员,也是剑桥大学的第二任卢卡斯数学教授。他是一位虔诚但非正统的基督徒,私下拒绝三位一体的教义。他拒绝加入英国国教的圣职,这在当时的剑桥大学教员中是少见的。除了数学科学方面的工作之外,牛顿还将大量时间投入到炼金术和《圣经》年代学的研究中,但他在这些领域的大部分作品直到去世后很久才发表。在政治上,他与辉格党有密切联系,并曾在1689-1690年和1701-1702年两次短暂担任剑桥大学的国会议员。1705年,他被安妮女王封为爵士,并在伦敦度过了生命的最后三十年,担任皇家造币厂的监理(1696–1699)和厂长(1699–1727),以及皇家学会会长(1703–1727)。
\subsection{早期生活}  
主要条目:艾萨克·牛顿的早期生活  
艾萨克·牛顿于1642年12月25日(根据当时在英格兰使用的儒略历,公历为1643年1月4日)出生于林肯郡的伍尔斯索普庄园。[17]牛顿的父亲也是名叫艾萨克·牛顿,在他出生前三个月去世。牛顿出生时早产,身体较小;他的母亲汉娜·艾斯考说,他可以放进一个夸脱的杯子里。[18]当牛顿三岁时,母亲再婚,和她的新丈夫巴纳巴斯·史密斯牧师一起生活,留下牛顿由他的外祖母玛格丽·艾斯考(原姓布莱思)照顾。牛顿不喜欢他的继父,并对母亲再婚心存怨恨,这在他19岁之前的一份罪行清单中有所体现:“威胁我的父亲和母亲史密斯要烧掉他们和他们的房子。”[19] 牛顿的母亲在第二次婚姻中生了三个孩子(玛丽、本杰明和汉娜)。[20]
\subsubsection{国王学校}  
牛顿大约在十二岁到十七岁期间,就读于格兰瑟姆的国王学校,该校教授拉丁语和古希腊语,并可能为他打下了坚实的数学基础。1659年10月,他被母亲从学校撤回,回到了伍尔斯索普-拜-科尔斯特沃斯。母亲在第二次丧夫后,试图让他成为一名农民,但他对此职业十分厌恶。国王学校的校长亨利·斯托克斯说服他的母亲让他重返学校。部分出于对校园欺凌者的报复心态,他成为了年级第一的学生,主要通过制作日晷和风车模型而脱颖而出。[24]
\subsubsection{剑桥大学}  
1661年6月,牛顿被剑桥大学的三一学院录取。他的叔叔威廉·艾斯考牧师曾在剑桥学习,推荐他进入该校。在剑桥,牛顿起初以“补助生”的身份入学,通过做杂役来支付学费,直到1664年获得奖学金,覆盖他四年的大学费用,直至获得硕士学位。[25]当时,剑桥的教学以亚里士多德的理论为基础,牛顿与当时的现代哲学家如笛卡尔以及天文学家伽利略·伽利莱和托马斯·斯特里特一起阅读这些作品。他在笔记本中记录了一系列关于机械哲学的“问题”。1665年,他发现了广义二项式定理,并开始发展一个后来成为微积分的数学理论。1665年8月,牛顿在剑桥获得学士学位后,因应对大瘟疫,大学暂时关闭。[26]

尽管他在剑桥大学的学生时代并不显著[27] ,但在接下来的两年里,[28]牛顿在伍尔斯索普的家中进行的私下研究促成了他在微积分、光学和引力定律方面的理论发展。[29][30]

1667年4月,牛顿返回剑桥大学,并在10月被选为三一学院的研究员。[31][32]研究员需要接受圣职并被按立为英国国教牧师,尽管在复辟时期这一要求并未严格执行,符合英格兰教会的声明便足够。他承诺道:“我要么将神学作为我研究的对象,并在这些章程规定的时间(7年)到来时接受圣职,要么就辞去学院职务。”[33]在此之前,他对宗教并没有过多思考,曾两次签署同意《三十九条》,即英格兰教会教义的基础。到1675年,这个问题无法避免,而此时他的非常规观点成为了障碍。[34]

他的学术工作给卢卡斯教授艾萨克·巴罗留下了深刻印象,巴罗渴望发展自己的宗教和行政潜力(他在两年后成为三一学院的院长);在1669年,牛顿接替了他的职位,这距离他获得硕士学位仅一年。卢卡斯教授的任职条款要求持有者不得活跃于教会——可能是为了留出更多时间用于科学研究。牛顿认为这应该使他免于按立的要求,查理二世国王接受了这一论点,因此牛顿的宗教观点与英国国教的正统观念之间的冲突得以避免。[35]

剑桥大学卢卡斯数学教授的职位还包括教授地理的责任。[36][37]在1672年和1681年,牛顿出版了《一般地理》的修订、校正和增补版,这本地理教科书最初由已故的伯纳德·瓦伦纽斯于1650年出版。[38]在《一般地理》中,瓦伦纽斯试图建立一个理论基础,将科学原则与古典地理概念联系起来,并认为地理是科学与应用于量化地球特征的纯数学的结合。[36][39] 虽然不清楚牛顿是否曾讲授地理,但1733年杜格代尔和肖的英文翻译版本中提到,牛顿出版此书是为了让学生在他讲授这一主题时阅读。[36] 《一般地理》被一些人视为地理历史中古代与现代传统的分界线,而牛顿参与后续版本的编辑被认为是这一持久遗产的重要原因之一。[40]

牛顿于1672年被选为皇家学会会员(FRS)。[1]
\subsection{中年}  
\subsubsection{微积分}  
牛顿的工作被认为“显著推动了当时所有研究的数学分支”。[41] 他在这一主题上的研究,通常称为流量(fluxions)或微积分,见于1666年10月的一份手稿,现已被收录在牛顿的数学论文中。[42]他的作品《通过无限项数的方程进行分析》(De analysi per aequationes numero terminorum infinitas)于1669年6月由艾萨克·巴罗发送给约翰·柯林斯,巴罗在同年8月给柯林斯的信中指出,这是“一个非凡天才和在这些领域中精通的作品”。[43]牛顿后来与莱布尼茨发生了关于微积分发展优先权的争论。大多数现代历史学家认为,牛顿和莱布尼茨是独立发展微积分的,尽管他们使用的数学符号大相径庭。然而,已确定牛顿在莱布尼茨之前很早就开始发展微积分。[44][11][45] 莱布尼茨的符号和“微分法”,如今被认为是更加便利的符号,后来被欧洲大陆的数学家采用,并在1820年左右也被英国数学家接受。

他的工作广泛使用基于趋近于零的小量比率的极限值的几何形式微积分:在《自然哲学的数学原理》中,牛顿以“第一和最后比率的方法”这一名称进行了演示,[46] 并解释了为何以这种形式进行阐述,[47]同时也指出“通过这种方式,完成的与无体积法所完成的是相同的。”[48]因此,现代人称《原理》为“一本充满无穷微积分理论和应用的书”,而在牛顿的时代,[49] 几乎所有内容都与这种微积分有关。[50] 他在1684年的《旋转物体运动论》中使用了涉及“一或多个无穷小量阶”的方法,并且在他关于运动的论文中也体现了这一点,[51]“这些论文是在1684年前的二十年内写成的”。[52]

牛顿曾对发表他的微积分持犹豫态度,因为他担心会引发争议和批评。[53]牛顿与瑞士数学家尼古拉·法蒂奥·德·杜伊耶关系密切。1691年,杜伊耶开始撰写牛顿《原理》的新版本,并与莱布尼茨通信。[54]1693年,杜伊耶与牛顿的关系恶化,这本书也未能完成。[55]从1699年开始,皇家学会的其他成员指控莱布尼茨抄袭。[56]1711年,争论进一步升级,皇家学会在一项研究中宣称真正的发现者是牛顿,并将莱布尼茨标记为骗子;后来发现牛顿实际上写下了该研究关于莱布尼茨的结论性评论。由此开始了这场痛苦的争议,影响了牛顿和莱布尼茨两人的生活,直至莱布尼茨于1716年去世。[57]
\begin{figure}[ht]
\centering
\includegraphics[width=6cm]{./figures/d24e0f92cc76e386.png}
\caption{1702年,牛顿的肖像由戈弗雷·奈勒(Godfrey Kneller)绘制。} \label{fig_Newton_2}
\end{figure}
牛顿通常被认为是广义二项式定理的创立者,该定理适用于任何指数。他发现了牛顿恒等式和牛顿法,分类了三次平面曲线(两个变量的三次多项式),对有限差分理论做出了重要贡献,是第一个使用分数指数的人,并采用坐标几何推导丢番图方程的解。他通过对数近似调和级数的部分和(这是欧拉求和公式的前身),并首次自信地使用幂级数及其反演。牛顿对无穷级数的研究受到西蒙·斯特芬的十进制数的启发。[58]
\subsubsection{光学}
\begin{figure}[ht]
\centering
\includegraphics[width=6cm]{./figures/fb2394ae8235a2b3.png}
\caption{牛顿于1672年向皇家学会展示的反射望远镜的复制品(他在1668年制造的第一个望远镜被借给了一位仪器制造商,但之后没有关于其去向的进一步记录)。[59]} \label{fig_Newton_3}
\end{figure}
1666年,牛顿观察到在最小偏转位置,光线通过棱镜后形成的颜色光谱呈椭圆形,即使进入棱镜的光线是圆形的,这意味着棱镜以不同的角度折射不同的颜色。[60][61]这使他得出结论,颜色是光的内在属性——这一点在此之前一直是争论的焦点。

从1670年到1672年,牛顿讲授光学。[62] 在此期间,他研究了光的折射,证明了由棱镜产生的多彩图像(他称之为光谱)可以通过透镜和第二个棱镜重新组合成白光。[63] 现代学术研究表明,牛顿对白光的分析和再合成与粒子炼金术有一定关系。[64] 
\begin{figure}[ht]
\centering
\includegraphics[width=6cm]{./figures/d1e379c1b13d7d69.png}
\caption{牛顿发现的分散棱镜将白光分解成光谱颜色的示意图。} \label{fig_Newton_4}
\end{figure}
通过这项工作,他得出结论,任何折射望远镜的透镜都会受到光的色散影响(色差)。为了证明这一概念,他构建了一台使用反射镜而非透镜作为物镜的望远镜,以绕过这个问题。[66][67]构建这一设计,即今天所称的牛顿望远镜——第一台已知的功能性反射望远镜——涉及解决合适的镜面材料和成型技术的问题。[67]牛顿自己研磨镜面,使用一种高度反射的镜面金属定制合成,并利用牛顿环来判断望远镜光学的质量。到1668年底,[68]他成功制造了这台第一台反射望远镜,长度约为八英寸,能够提供更清晰和更大的图像。1671年,皇家学会要求他演示他的反射望远镜。[69]他们的兴趣鼓励牛顿发表了他的笔记《论颜色》,[70]他后来将其扩展为著作《光学》。当罗伯特·胡克批评牛顿的一些观点时,牛顿感到非常冒犯,以至于他退出了公共辩论。牛顿和胡克在1679年至1680年间有过简短的交流,那时胡克被任命为皇家学会的通信管理者,开始了旨在促使牛顿为皇家学会事务做出贡献的通信,[71]
\begin{figure}[ht]
\centering
\includegraphics[width=6cm]{./figures/8e4d4c58235cd719.png}
\caption{1682年牛顿致威廉·布里格斯的信件的复制品,评论了布里格斯的《新视觉理论》。} \label{fig_Newton_5}
\end{figure}
牛顿认为光是由粒子或微粒组成的,这些粒子在加速进入更密集的介质时会发生折射。他倾向于用声波来解释薄膜的反射和透射的重复模式(《光学》第二卷,第12条),但仍然保留了他的“适合”理论,认为微粒在反射或透射时会受到影响(第13条)。然而,后来的物理学家更倾向于用纯波动的解释来说明光的干涉图样和衍射现象。今天的量子力学、光子和波粒二象性的概念与牛顿对光的理解仅有微小的相似之处。

在1675年的《光的假说》中,牛顿假设以太的存在,以便在粒子之间传递力。[73]与剑桥的柏拉图主义哲学家亨利·摩尔的接触重新激发了他对炼金术的兴趣。他用基于赫尔墨斯吸引与排斥观念的神秘力量取代了以太。约翰·梅纳德·凯恩斯,曾获得牛顿许多关于炼金术的著作,指出“牛顿不是理性时代的第一人:他是最后一位魔法师。”[74]牛顿对科学的贡献无法与他对炼金术的兴趣割裂开来。[73]当时,炼金术和科学之间并没有明确的区分。

在1704年,牛顿出版了《光学》,在书中阐述了他的粒子理论。他认为光由极细微的微粒组成,而普通物质则由更粗大的微粒构成,并推测通过某种炼金术的转化,“粗大物体和光是否可以互相转化……物体是否可以从进入其组成的光粒子中获得大量的活性?”[75]牛顿还构造了一种原始形式的摩擦静电发生器,使用了一个玻璃球。[76]

在他的著作《光学》中,牛顿首次展示了使用棱镜作为光束扩展器的图示,并介绍了多棱镜阵列的应用。[77]大约278年后,多棱镜光束扩展器成为窄线宽可调激光器发展的核心。此外,这些棱镜光束扩展器的使用导致了多棱镜色散理论的产生。[77]

在牛顿之后,许多理论被修正。扬和菲涅尔放弃了牛顿的粒子理论,转而支持惠更斯的波动理论,以展示颜色是光波长的可见表现。科学也逐渐意识到颜色感知与可数学化光学之间的区别。德国诗人兼科学家歌德无法摆脱牛顿的基础理论,但“歌德确实在牛顿的盔甲中找到了一个漏洞……牛顿坚持认为没有颜色的折射是不可能的,因此他认为望远镜的物镜将永远不完美,消色差和折射是不相容的。这一推论后来被多隆证明是错误的。[78]
\subsubsection{重力}
\begin{figure}[ht]
\centering
\includegraphics[width=6cm]{./figures/51b511f20a306b0c.png}
\caption{约翰·范德班克所作的牛顿肖像雕刻。} \label{fig_Newton_6}
\end{figure}
牛顿早在1665年就开始发展他的引力理论。[29][30]1679年,牛顿通过考虑引力及其对行星轨道的影响,重新回到他的天体力学研究,参考了开普勒的行星运动定律。这是受到1679年至1680年与被任命为皇家学会秘书的胡克的一次简短书信交流的刺激,[79]胡克开启了一项旨在促使牛顿为皇家学会事务做出贡献的通信。[71]牛顿对天文学的重新兴趣在1680至1681年的冬季受到了一颗彗星出现的进一步刺激,他与约翰·弗拉姆斯蒂德进行了通信。[80]在与胡克的交流之后,牛顿推导出一个证明,表明行星轨道的椭圆形状源于与半径向量的平方成反比的向心力。牛顿将他的结果以《物体运动论》为题,传达给了埃德蒙·哈雷和皇家学会,这是一篇约九张纸的论文,于1684年12月被抄录入皇家学会的登记册中。[81]这篇论文包含了牛顿发展并扩展形成《原理》的核心内容。
\begin{figure}[ht]
\centering
\includegraphics[width=6cm]{./figures/5dbd8e950865e95f.png}
\caption{牛顿的《原理》自藏本,包含牛顿手写的第二版修正,现在存放于剑桥大学三一学院的温伦图书馆。} \label{fig_Newton_7}
\end{figure}
《原理》于1687年7月5日出版,得到了哈雷的鼓励和财政支持。在这部著作中,牛顿阐述了三条普遍运动定律。这些定律描述了任何物体、作用于其上的力以及由此产生的运动之间的关系,为经典力学奠定了基础。它们对随后不久的工业革命带来了许多进展,并且在200多年内没有得到改进。这些进展中的许多仍然是现代世界中非相对论技术的基础。他使用拉丁词汇“gravitas”(重量)来描述后来被称为重力的效应,并定义了万有引力定律。[82]

在同一部著作中,牛顿提出了一种类似于微积分的几何分析方法,使用“首尾比率”,并首次通过波义耳定律对空气中的声音速度进行了分析性测定。他推断出地球的椭球体形状,并将春分点的进动归因于月球对地球椭圆形的引力作用。他开始研究月球运动中的不规则性,并提供了彗星轨道确定的理论,还有许多其他贡献。[82]牛顿的传记作者大卫·布鲁斯特报道说,将他的引力理论应用于月球运动的复杂性之大,影响了牛顿的健康:“在他处理这一问题的1692至1693年期间,他失去了食欲和睡眠”,并告诉天文学家约翰·马钦,“他只在研究这一主题时才头痛”。据布鲁斯特所述,埃德蒙·哈雷还告诉约翰·康杜伊特,当牛顿被催促完成他的分析时,他“总是回答这让他头痛,常常让他失眠,以至于他再也不想考虑这个问题”。【原文强调】[83]

牛顿明确了他对太阳系的日心观点——这一观点以一种较为现代的方式发展,因为早在1680年代中期,他就意识到太阳相对于太阳系重心的“偏差”。[84]对于牛顿来说,不能准确地认为太阳或任何其他天体是静止的,而应该认为“地球、太阳和所有行星的共同重心被视为世界的中心”,这个重心“要么静止,要么沿直线匀速前进”。(考虑到普遍共识,即无论中心位于何处,都是静止的,牛顿选择了“静止”这一选项。)[85]

牛顿因在科学中引入“神秘力量”而受到批评,因为他假设了一种能够在广阔距离上作用的看不见的力量。[86]在《原理》第二版(1713年)中,牛顿在结尾的《一般学论》中坚决拒绝了这样的批评,写道,现象足以暗示引力的存在,但并未表明其原因,因此构建不由现象所暗示的假设既不必要也不恰当。(在这里,牛顿使用了他著名的表达“Hypotheses non fingo”。)[87]

凭借《原理》,牛顿获得了国际认可。[88]他获得了一批追随者,其中包括瑞士出生的数学家尼古拉·法蒂奥·德·杜伊耶。[89]

在1710年,牛顿找到了78种立方曲线中的72种,并将它们分类为四种类型。[90]1717年,詹姆斯·斯特林在牛顿的帮助下证明了每个立方曲线都是这四种类型之一。牛顿还声称,这四种类型可以通过平面投影从其中一种获得,这一点在1731年得到了证明,距离牛顿去世四年。[91]
\subsection{晚年}   
\subsubsection{皇家铸币厂}
\begin{figure}[ht]
\centering
\includegraphics[width=6cm]{./figures/285008173cc69f79.png}
\caption{1712年牛顿老年的肖像,画家:詹姆斯·索恩希尔爵士。} \label{fig_Newton_8}
\end{figure}
在1690年代,牛顿撰写了一些宗教著作,探讨了《圣经》的字面和象征性解读。他寄给约翰·洛克的一份手稿中,质疑《约翰一书》5章7节——即“约翰的插入句”——与新约原始手稿的一致性,这份手稿直到1785年才得以出版。[92]

牛顿还曾于1689年和1701年担任剑桥大学的英格兰议会议员,但根据一些说法,他的唯一评论是抱怨议会里的冷风,并要求关闭窗户。[93]然而,剑桥的日记作者亚伯拉罕·德·拉·普赖姆注意到,牛顿曾训斥那些通过声称一所房子闹鬼而吓坏当地人的学生。[94]

牛顿于1696年迁往伦敦,担任皇家铸币厂的监护人,这一职位是在威廉三世国王统治期间获得的,得到的是哈利法克斯伯爵查尔斯·蒙塔古的支持,他当时是财政大臣。牛顿负责英国的大规模重铸,曾触怒塔楼的总督卢卡斯勋爵,并为埃德蒙·哈雷争取到了临时切斯特分支的副审计员职位。1699年托马斯·尼尔去世后,牛顿成为最著名的铸币厂厂长,这一职务他一直担任到生命的最后30年。这些任命本是荣誉职务,但牛顿对此十分认真。[95][96]他于1701年辞去了剑桥的职务,利用自己的权力改革货币,惩罚剪币和伪造者。

作为皇家铸币厂的监护人和厂长,牛顿估计在1696年的大重铸中,约20\%的流通硬币是伪造的。伪造货币被视为叛国罪,罪犯将面临绞刑、拖尸和肢解的惩罚。尽管如此,起诉即使是最明显的罪犯也可能极为困难,但牛顿证明了自己能胜任这一任务。[97]

牛顿伪装成酒吧和小酒馆的常客,亲自收集了大量证据。[98]尽管对起诉设置了诸多障碍,并且政府的各个部门之间存在隔阂,但英国法律仍然拥有古老而强大的权威传统。牛顿自己被任命为所有本土县的治安法官。一封关于此事的草稿信件被包含在牛顿的《自然哲学的数学原理》个人第一版中,显示他当时正在进行修改。[99]随后,他在1698年6月至1699年圣诞节之间进行了100多次对证人、线人的交叉审问。牛顿成功起诉了28名伪造者。[100]
\begin{figure}[ht]
\centering
\includegraphics[width=6cm]{./figures/d3126e6cd66f66bd.png}
\caption{林肯郡大戈内比的牛顿家族徽章,后被艾萨克·牛顿爵士使用。[101]} \label{fig_Newton_9}
\end{figure}
牛顿于1703年被任命为皇家学会会长,并成为法国科学院的会员。在皇家学会的职位上,牛顿与皇家天文学家约翰·弗拉姆斯蒂德结下了敌对关系,因为他提前出版了弗拉姆斯蒂德的《不列颠天文学史》,而牛顿在自己的研究中使用了该书。[102]
\subsubsection{爵士封号}  
1705年4月,安妮女王在对剑桥大学三一学院的皇家访问期间,授予牛顿骑士称号。这一封号可能是由于与1705年5月的议会选举相关的政治考虑,而非对牛顿科学工作或作为铸币厂厂长服务的认可。[103]牛顿是继弗朗西斯·培根之后第二位获得骑士称号的科学家。[104]

根据牛顿于1717年9月21日写给国王财政委员会的报告,金币和银币之间的双金属关系在1717年12月22日通过皇家公告发生了变化,禁止将金吉尼兑换超过21个银先令。[105]这无意中导致了银币短缺,因为银币被用来支付进口,而出口则以黄金支付,这实际上使英国从银本位制转向了第一个金本位制。[106]关于他是否有意这样做仍存在争议。有观点认为,牛顿将自己在铸币局的工作视为他炼金术工作的延续。[107]

牛顿投资了南海公司,并在该公司于约1720年崩溃时损失了约20,000英镑(2020年相当于440万英镑)。[109]

在他生命的最后阶段,牛顿与他的侄女及其丈夫一起居住在温彻斯特附近的克兰伯里公园,直到去世。[110]他的半侄女凯瑟琳·巴顿在他位于伦敦杰敏街的住所中担任社交事务的女主人;[111]根据他给她的信件,当她从天花中康复时,他称她为“非常亲爱的叔叔”。[112]
\subsubsection{死亡}
\begin{figure}[ht]
\centering
\includegraphics[width=6cm]{./figures/f13e85a108cd28c2.png}
\caption{牛顿的死亡面具,约1906年拍摄。} \label{fig_Newton_10}
\end{figure}
牛顿于1727年3月20日在伦敦安详去世(旧历1726年3月20日;新历1727年3月31日)。他举行了盛大的葬礼,出席者包括贵族、科学家和哲学家,他被安葬在威斯敏斯特教堂,与众多国王和女王为邻。他是首位被安葬于该教堂的科学家。[113]伏尔泰可能参加了他的葬礼。[114]作为一名单身汉,他在去世前几年将大部分遗产赠予了亲属,去世时没有留遗嘱。他的遗稿由约翰·康杜伊特和凯瑟琳·巴顿继承。[115]

在牛顿去世后不久,制作了他的石膏面具。佛兰德雕塑家约翰·迈克尔·瑞斯布拉克使用该面具创作了牛顿的雕像。[118]现在这面具由皇家学会收藏,并于2012年进行了3D扫描。[119][120]

牛顿的头发在去世后被检查,发现其中含有汞,这可能与他的炼金术追求有关。汞中毒可以解释牛顿晚年的古怪行为。[115]
\subsection{个性}  
尽管有人声称牛顿曾经订婚,但他从未结过婚。法国作家和哲学家伏尔泰在牛顿的葬礼上曾说:“他从未感受过任何激情,不受人类常见弱点的影响,也与女性没有任何交往——这一点是我从陪伴他度过最后时刻的医生和外科医生那里得知的。[122]有一种广泛的看法认为牛顿死时仍是处男,[123]数学家查尔斯·哈顿、[124]经济学家约翰·梅纳德·凯恩斯以及物理学家卡尔·萨根等各类作家对此都有所评论。[125]

牛顿与瑞士数学家尼古拉斯·法蒂奥·德·杜伊耶有着密切的友谊,他们在1689年左右在伦敦相识[89]——他们的一些通信至今仍然保存下来。[126][127]两人的关系在1693年突然且没有解释地结束,正好那时牛顿也经历了一次神经崩溃,[128]包括向他的朋友塞缪尔·佩皮斯和约翰·洛克发送狂野的指控信件。他给洛克的信中指控洛克试图“将他与‘女性’以及其他方式卷入纠纷”。[139]

牛顿对自己的成就相对谦逊,他在1676年2月写给罗伯特·胡克的信中说道:“如果我看得更远,那是因为站在巨人的肩膀上。”[130]有两位作者认为,这句话是在牛顿和胡克因光学发现而争执的时期写的,可能是对胡克(据说身材矮小且驼背)的间接攻击,而不仅仅是谦虚的表达。[131][132]另一方面,关于“站在巨人肩膀上”的广为人知的谚语,曾由17世纪诗人乔治·赫伯特(剑桥大学的前演讲者及三一学院的院士)在《雅库拉·普鲁登图姆》(1651年)中发表,其主要观点是“矮人站在巨人的肩上,看得更远”,因此作为类比,其效果会使牛顿本身而非胡克成为“矮人”。

在后来的回忆录中,牛顿写道:“我不知道在世人眼中我可能显得怎样,但在我自己看来,我似乎只是一个在海滩上玩耍的男孩,偶尔发现一些比平常更光滑的卵石或更美丽的贝壳,而面前的真理大海则仍未被探索。[133]
\subsection{神学}
\subsubsection{宗教观点}
虽然牛顿出生在一个圣公会家庭,但到了三十多岁时,他的基督教信仰如果公之于众,可能不会被主流基督教视为正统。[134]一位历史学家甚至将他称为异端。[135]

到1672年,牛顿开始在笔记本中记录他的神学研究,这些笔记本无人示人,自1972年才对公众开放。[136]他所写的内容中,有超过一半涉及神学和炼金术,而大部分内容从未出版。[136]他的著作显示出对早期教会文献的广泛了解,并表明在界定信经的亚他那修与阿里乌斯的冲突中,他支持了失败的阿里乌斯,后者拒绝了传统的三位一体观。牛顿“认为基督是神与人之间的神圣中介,且在本质上低于创造他的父亲。”[137]他特别对预言感兴趣,但对他而言,“大叛教是三位一体论。”[138]

牛顿曾试图获得两项免除持有人按要求受按立的奖学金,但未能成功。在1675年的最后时刻,他得到了政府的特许,豁免了他及所有未来的卢卡斯教授。[139]

在牛顿看来,崇拜耶稣基督为神是一种偶像崇拜,他认为这是根本的罪。[140]在1999年,历史学家斯蒂芬·D·斯诺贝伦写道:“艾萨克·牛顿是一个异端。但是……他从未公开宣称自己的私人信仰,而这在正统教义看来将是极其激进的。他把自己的信仰隐藏得如此之好,以至于学者们至今仍在解读他的个人信念。”[135]斯诺贝伦得出结论,牛顿至少是索西尼主义的同情者(他拥有并彻底阅读过至少八本索西尼主义书籍),可能是阿里乌斯主义者,几乎可以肯定是反三位一体论者。[135]
\begin{figure}[ht]
\centering
\includegraphics[width=8cm]{./figures/bba96e2da364d071.png}
\caption{牛顿(1795年,细节)由威廉·布莱克创作。牛顿被批判性地描绘为一个‘神圣的几何师’。[141]} \label{fig_Newton_11}
\end{figure}
虽然运动定律和万有引力定律成为牛顿最著名的发现,但他警告不要将这些定律用于将宇宙视为一个单纯的机器,仿佛它类似于一座巨大的时钟。他说:‘因此,重力可以使行星运动,但没有神圣的力量,它永远无法使它们以如同围绕太阳的循环运动。[142]

除了科学上的声誉,牛顿对圣经和早期教父的研究也颇具价值。牛顿撰写了有关文本批评的著作,最著名的包括《两处显著经文的历史记载》和《关于但以理书的预言及圣约翰启示录的观察》。[143]他将耶稣基督的钉十字架日期定在公元33年4月3日,这与一个传统上接受的日期一致。[144]

他相信一个理性内在的世界,但拒绝了莱布尼茨和巴鲁赫·斯宾诺莎所暗含的物质有生命论。这个有序且动态丰富的宇宙可以被理解,并且必须被理解,通过一种积极的理性。在他的通信中,牛顿声称在撰写《原理》时‘我关注的是那些可能引导考虑的人对神的信仰的原则’。[145]他在世界体系中看到了设计的证据:‘在行星系统中如此奇妙的统一性,必须被认为是选择的结果。’但牛顿坚信,由于不稳定性缓慢增长,最终需要神的干预来改革这个系统。[146]为此,莱布尼茨嘲笑他说:‘全能的上帝想不时给他的钟表上发条:否则它将停止运动。看来他没有足够的先见之明来让它成为一个永动机。’[147]

牛顿的立场在他追随者塞缪尔·克拉克的著名通信中得到了有力辩护。一百年后,皮埃尔-西蒙·拉普拉斯的著作《天体力学》自然地解释了为什么行星轨道不需要周期性的神灵干预。[148]拉普拉斯的机械世界观与牛顿的世界观之间的对比是最鲜明的,尤其是考虑到这位法国科学家对拿破仑的著名回答,后者曾批评他在《天体力学》中缺乏造物主的存在:“陛下,我可以不需要这个假设。”[149]
\subsubsection{宗教思想}  
牛顿和罗伯特·波义耳对机械哲学的理解被理性主义的小册子作者宣传为对泛神论者和热情主义者的可行替代方案,且被正统的传教士和如宽容主义者等异议传教士犹豫地接受。[151]科学的清晰与简明被视为对抗迷信热情和无神论威胁的方式,[152]同时,第二波英格兰自然神论者利用牛顿的发现来展示‘自然宗教’的可能性。

对启蒙前‘魔法思维’和基督教神秘元素的攻击,以波义耳的机械宇宙观为基础。牛顿通过数学证明为波义耳的思想提供了完整性,并且,更重要的是,他在推广这些思想方面非常成功。[153]
\subsection{炼金术}
在牛顿的论文中,估计有一千万字的写作,其中约一百万字涉及炼金术。[116]牛顿关于炼金术的许多著作是其他手稿的抄本,并带有他自己的注释。炼金术文本将手艺知识与哲学推测混合在一起,常常隐藏在语言游戏、寓言和意象的层层之下,以保护工艺秘密。[155]牛顿论文中的一些内容可能被教会视为异端。[116][154]\footnote{牛顿不是理性时代的第一人。他是最后一个魔法师,最后一个巴比伦人和苏美尔人,最后一个以与那些开始构建我们智力遗产的人的相同眼光看待可见和智力世界的伟大思想家,时间距今不到一万年。艾萨克·牛顿,一个在1642年圣诞节无父而生的死后出生的孩子,是最后一个可以让东方三博士真诚而适当地致敬的神童。

——约翰·梅纳德·凯恩斯,《牛顿,人》}

1888年,剑桥大学在花费十六年对牛顿的论文进行编目后,保留了少量论文并将其余部分归还给朴茨茅斯伯爵。[156]1936年,一位后裔在苏富比拍卖这些论文。该收藏被拆分并以约9000英镑的总价出售。[157]约翰·梅纳德·凯恩斯是大约三十多位竞标者之一,他在拍卖中获得了部分收藏。凯恩斯随后重新整理了估计约一半的牛顿炼金术论文收藏,并于1946年将其捐赠给剑桥大学。[116][156][158]

牛顿所有已知的炼金术著作目前正在印第安纳大学的一个项目中上线:‘艾萨克·牛顿的化学’[159]并在一本书中进行了总结。[160][161]

牛顿对科学的基本贡献包括引力的量化、发现白光实际上是不可变的光谱颜色的混合以及微积分的形成。然而,牛顿还有另一个更神秘的方面,这一领域的活动跨越了他生命中的约三十年,尽管他在很大程度上将其隐藏于当时的同 contemporaries 和同事面前。我们指的是牛顿在炼金术领域的参与,或在十七世纪英格兰常被称为的“化学”。[159]

2020年6月,邦汉姆拍卖行在线拍卖了两页牛顿关于扬·巴普蒂斯特·范赫尔蒙特关于瘟疫的书《瘟疫》未出版的笔记。根据邦汉姆的说法,牛顿在剑桥时分析这本书,[162]以保护自己免受1665-1666年伦敦瘟疫的影响,这是他已知对瘟疫所做的最重要的书面陈述。关于治疗,牛顿写道‘最佳的疗法是将一只蟾蜍悬挂在烟囱里三天,最后将含有各种昆虫的土壤呕吐到一盘黄色蜡上,不久后就死了。将粉碎的蟾蜍与分泌物和血清混合制成含片,并放置在受影响区域,可以驱散传染病并排出毒素。’[163]
\subsection{遗产}  
\subsubsection{声誉}
\begin{figure}[ht]
\centering
\includegraphics[width=6cm]{./figures/2a06ba03e4def95b.png}
\caption{牛顿在威斯敏斯特大教堂的墓碑,由约翰·迈克尔·赖斯布拉克创作} \label{fig_Newton_12}
\end{figure}
数学家和天文学家约瑟夫-路易斯·拉格朗日常常声称牛顿是有史以来最伟大的天才,[164]并曾补充道牛顿也是‘最幸运的,因为我们无法再找到一个可以建立的世界体系。’英诗人亚历山大·波普写下了著名的碑文:[165]

‘自然和自然的法则隐藏在黑暗中。  
上帝说,让牛顿出现!一切都变为光明。

但这段碑文未被允许刻在牛顿在威斯敏斯特的纪念碑上。所加的碑文如下:[166]

“H. S. E. 艾萨克·牛顿,金马,/ 以几乎神圣的精神,/ 首先展示了行星的运动、形状、/ 彗星的轨迹和海洋的潮汐。/ 以自己的数学为火炬,/ 首先证明了:/ 光线的不同性质,/ 以及由此产生的颜色的特性,/ 这是之前无人猜测过的。/ 自然、古代和圣经的 / 勤勉、聪慧、忠实的阐释者,/ 他在哲学中宣扬全能神的伟大,/ 在道德中展现了福音的简单。/ 让世人庆幸,/ 这样的伟人曾存在于世,/ 是人类的荣耀。/ 生于公元1642年12月25日,/ 逝于公元1726年3月20日。[166]

这里埋葬的是艾萨克·牛顿,爵士,他以几乎神圣的智慧和独特的数学原理,探索了行星的运动和形状、彗星的轨迹、海洋的潮汐、光线的不同性质,以及之前无人想象过的由此产生的颜色的特性。在对自然、古代和圣经的阐释中,他勤奋、聪慧且忠实,通过他的哲学捍卫了全能善良的上帝的伟大,并在他的行为中表达了福音的简单。凡人欢欣鼓舞,庆幸人类中曾有如此伟大的荣光!

在2005年,针对公众和英国皇家学会成员(牛顿曾任会长)进行了一项双重调查,询问牛顿和阿尔伯特·爱因斯坦谁对科学历史的影响更大。结果显示,皇家学会的成员和公众都认为牛顿在整体贡献上更为突出。[167][168]1999年,一项针对100位当时领先物理学家的意见调查中,爱因斯坦被评为“有史以来最伟大的物理学家”,牛顿则位列第二。而由PhysicsWeb网站对普通物理学家的平行调查则将牛顿评为第一。[169][170]新科学家杂志称牛顿为“科学史上最杰出和最神秘的人物”。[172]牛顿被称为“西方科学史上最有影响力的人物”。爱因斯坦在他的书房墙上挂着牛顿的画像,旁边还有迈克尔·法拉第和詹姆斯·克拉克·麦克斯韦的画像。[173]

物理学家列夫·朗道根据生产力在0到5的对数尺度上对物理学家进行了排名。最高的等级0被赋予牛顿,阿尔伯特·爱因斯坦则被评为0.5。量子力学的“创始父亲”尼尔斯·玻尔、维尔纳·海森堡、保罗·狄拉克和厄尔温·薛定谔被评为1。诺贝尔奖获得者和超流体性发现者朗道自评为2。[174]

以牛顿的名字命名的SI导出单位力是牛顿。

伍尔斯索普庄园因是牛顿的出生地和“他发现重力并发展光的折射理论的地方”,被英国历史遗迹保护协会列为一级建筑。[175]

1816年,一颗据说属于牛顿的牙齿在伦敦以730英镑的价格售出,[176]买家是一位贵族,他将其镶嵌在戒指中。[177]《吉尼斯世界纪录2002》将其列为世界上最有价值的牙齿,估计在2001年底价值约为25,000英镑(合35,700美元)。[177]购买者和目前拥有者的身份并未披露。
\subsubsection{苹果事件}
\begin{figure}[ht]
\centering
\includegraphics[width=6cm]{./figures/b4424fd8993f1597.png}
\caption{据称是牛顿苹果树后裔的树木(从上到下):剑桥大学三一学院、剑桥大学植物园,以及阿根廷的巴尔塞罗研究所图书馆花园。} \label{fig_Newton_13}
\end{figure}
牛顿本人经常讲述这个故事,称他是通过观察一只苹果从树上掉落而受到启发,进而形成了他的引力理论。[178][179]这个故事据说是在牛顿的侄女凯瑟琳·巴顿向伏尔泰讲述后传入了大众知识。[180]伏尔泰在他的《史诗诗论》(1727年)中写道:‘艾萨克·牛顿爵士在花园散步时,看到一只苹果从树上掉下,产生了他关于引力系统的第一个想法。’[181][182]

尽管有人说苹果故事是一个神话,他并不是在某一特定时刻得出引力理论,[183] 牛顿的熟人(如威廉·斯图克利,其1752年的手稿已由皇家学会发布)确实证实了这一事件,尽管并非传说中的苹果真的砸到了牛顿的头上。斯图克利在《艾萨克·牛顿爵士的生活回忆录》中记录了他与牛顿在1726年4月15日于肯辛顿的谈话:[184][185][186]

‘我们走进花园,在一些苹果树的阴影下喝茶,只有他和我。在其他谈话中,他告诉我,他当时的情况与当年引力概念进入他脑海时是一样的。“为什么那只苹果总是垂直地落向地面?”他自言自语道:因他坐着沉思,看到一个苹果掉落:“为什么它不横向或向上运动,而是总是朝向地球的中心?肯定是因为地球吸引它。物质中必定存在一种吸引力,而地球物质的总吸引力必然在地球的中心,而不在地球的任何一侧。因此,这只苹果是垂直下落的,或是朝向中心的。如果物质如此吸引物质,那就必须与其数量成比例。因此苹果吸引地球,正如地球吸引苹果。’”

牛顿在皇家铸币厂的助手、牛顿侄女的丈夫约翰·康杜伊特也在写牛顿的生平时描述了这一事件:[187]

1666年,他再次从剑桥退隐回到林肯郡的母亲身边。当他在花园中沉思漫步时,他想到引力的力量(将苹果从树上带到地面)并不局限于地球的一定距离,而是这股力量必须延伸得远远超过通常的想法。他自言自语道:为什么不延伸到月球那么高呢?如果是这样,这一定会影响月球的运动,也许会把它保持在轨道上,于是他开始计算这一假设的效果。

从他的笔记本中可以得知,牛顿在1660年代末期就在思考地球引力是否以反平方比例延伸到月球;然而,他花了二十年的时间才发展出完整的理论。[188] 问题并不是引力是否存在,而是它是否延伸得足够远,以至于可以成为将月球保持在轨道上的力量。牛顿证明,如果力量随距离的平方反比减小,确实可以计算出月球的轨道周期,并得到良好的吻合。他猜测同样的力量负责其他的轨道运动,因此将其命名为“万有引力”。

多个树木声称是牛顿所描述的“那棵”苹果树。格兰瑟姆的国王学校声称该树是由学校购买的,几年前被连根拔起并运输到校长的花园中。现在由国家信托拥有的伍尔斯索普庄园的工作人员对此表示质疑,并声称他们花园中存在的树就是牛顿所描述的那棵。剑桥大学三一学院的主门外可以看到一棵原树的后代[189],它生长在牛顿学习期间居住的房间下方。肯特的布罗格代尔国家水果收藏[190]可以提供来自他们的树的嫁接,这棵树看起来与“肯特花”这种粗肉质的烹饪品种完全相同。[191]
\subsubsection{纪念活动}
\begin{figure}[ht]
\centering
\includegraphics[width=6cm]{./figures/c1e6f8c479f5c97c.png}
\caption{牛顿雕像在牛津大学自然历史博物馆展出} \label{fig_Newton_14}
\end{figure}
牛顿的纪念碑(1731年)位于威斯敏斯特大教堂,位于合唱团入口北侧,靠近他的坟墓,背靠合唱屏幕。[192]该纪念碑由雕塑家迈克尔·瑞斯布拉克(1694–1770)用白色和灰色大理石雕刻,设计由建筑师威廉·肯特完成。纪念碑上描绘了牛顿的雕像,斜卧在一个石棺上,右肘支撑在几本重要的书籍上,左手指向一卷带有数学图案的卷轴。其上方是一个金字塔和一个天球,显示了黄道带和1680年彗星的轨迹。浮雕面板上描绘了天使们使用望远镜和棱镜等仪器。[193]

从1978年到1988年,由哈里·埃克尔斯顿设计的牛顿形象出现在英格兰银行发行的D系列1英镑纸币上(这是英格兰银行发行的最后一批1英镑纸币)。纸币背面展示了牛顿手持一本书,身旁有望远镜、棱镜和太阳系地图。

在牛津大学自然历史博物馆,可以看到一座牛顿雕像,他的目光注视着脚下的苹果。在伦敦的英国图书馆广场上,爱德华多·保罗齐创作的青铜雕像《牛顿》,灵感来自威廉·布莱克的雕刻,创作于1995年,成为广场的主角。在格兰瑟姆市中心,1858年竖立了一座牛顿的青铜雕像,他曾在此上学,这座雕像显眼地矗立在格兰瑟姆市政厅前。[194]

伍尔斯索普庄园的仍然存在的农舍被英国历史遗迹保护协会列为一级建筑,因为它是牛顿的出生地和“他发现重力以及发展光的折射理论的地方”。[175]
\subsection{启蒙时代}  
启蒙哲学家们选择了一段简短的科学前辈历史——主要是伽利略、博伊尔和牛顿——作为他们将自然和自然法的独特概念应用于当时每一个物理和社会领域的指导者和担保人。在这方面,历史的教训和建立在其上的社会结构可以被抛弃。[195]

欧洲启蒙哲学家和启蒙历史学家认为,牛顿《自然哲学的数学原理》的出版是科学革命的一个转折点,并开启了启蒙时代。牛顿基于自然和理性可理解法则的宇宙观成为了启蒙思想的种子之一。[196] 洛克和伏尔泰将自然法的概念应用于主张内在权利的政治体系;重农主义者和亚当·斯密则将自然心理学和自利的观念应用于经济体系;而社会学家们批评当时的社会秩序试图将历史纳入自然进步的模型。蒙博多和塞缪尔·克拉克对牛顿的某些研究持抵制态度,但最终将其合理化,以符合他们对自然的强烈宗教观点。
\subsection{著作}  
\subsubsection{生前出版的著作}  
\begin{itemize}
\item 《通过无穷项数方程的分析》(1669年,1711年出版)[197]  
\item 《自然明显法则与植物生长过程》(未出版,约1671–75年)[198]  
\item 《物体运动论》(1684年)[199]  
\item 《自然哲学的数学原理》(1687年)[200]  
\item 《热量的阶梯:热量的描述与标志》(1701年)[201]  
\item 《光学》(1704年)[202]  
\item 作为铸币厂厂长的报告(1701–1725年)[203]  
\item 《普遍算术》(1707年)[203]
\end{itemize}
\subsubsection{死后出版的著作}  
\begin{itemize}
\item 《世界系统论》(1728年)[203]  
\item 《光学讲座》(1728年)[203]  
\item 《古代王国年表修订版》(1728年)[203]  
\item 《对《但以理书》和《圣约翰启示录》的观察》(1733年)[203]  
\item 《流量法》(1671年,1736年出版)[204]  
\item 《两处显著的圣经篡改的历史记述》(1754年)[203]
\end{itemize}
\subsection{另见}
\begin{itemize}
\item 《牛顿哲学要素》,伏尔泰的著作  
\item 17世纪的多重发现列表  
\item 以艾萨克·牛顿命名的事物列表  
\item 皇家学会历任主席名单
\end{itemize}
\subsection{参考文献}  
\subsubsection{注释}
\begin{enumerate}
\item 在牛顿生前,欧洲使用两种日历:在新教和东正教地区(包括英国)使用的儒略历(“旧历”),以及在罗马天主教欧洲使用的公历(“新历”)。牛顿出生时,公历日期比儒略历日期提前十天;因此,他的出生被记录为旧历的1642年12月25日,但可以转换为新历(现代)日期1643年1月4日。在他去世时,两个日历之间的差异已增加到十一天。此外,他去世的时间是在新历年(1月1日)开始之后,但在旧历新年(3月25日)之前。他的去世日期为1726年3月20日(旧历),但通常调整为1727年。完全转换为新历后的日期为1727年3月31日。[6][自出版来源?]  
\item 这一说法是威廉·斯图克利在1727年写给理查德·米德的关于牛顿的信中提出的。查尔斯·哈顿在18世纪末收集了关于早期科学家的口述传统,他宣称“如果他有结婚的倾向,似乎没有充分的理由使他从未结过婚。更可能的是,他对婚姻状态以及对性别本身存在一种天生的漠不关心。”[121]
\end{enumerate}  
\subsection{引用}  
\begin{enumerate}
\item 皇家学会会员”。伦敦:皇家学会。于2015年3月16日存档。  
\item Feingold, Mordechai. Barrow, Isaac (1630–1677) 于2013年1月29日存档,牛津国家传记词典,牛津大学出版社,2004年9月;在线版,2007年5月。于2009年2月24日检索;在Feingold, Mordechai(1993)中进一步解释。“牛顿、莱布尼茨,以及巴罗:一次重新解读的尝试”。《伊西斯》。84(2):310–338。Bibcode:1993Isis...84..310F。doi:10.1086/356464。ISSN 0021-1753。JSTOR 236236。S2CID 144019197。
\item 《科学传记词典》。注释,第4号。于2005年2月25日存档。  
\item Gjertsen 1986,第[页码需提供]  
\item Kevin C. Knox, Richard Noakes(编辑),《从牛顿到霍金:剑桥大学卢卡斯数学教授的历史》,剑桥大学出版社,2003年,第61页。
\item Thony, Christie (2015). ‘日历混乱,还是牛顿究竟何时去世?’《文艺复兴数学家》。于2015年4月2日存档。于2015年3月20日获取。  
\item Alex, Berezow (2022年2月4日). ‘世界上最聪明的人是谁?’《Big Think》。于2023年9月28日存档。于2023年9月28日获取。  
\item Whiteside, D. T. (1991). ‘《原理》从1664年到1686年的前历史’。《伦敦皇家学会记录与笔记》。45 (1): 11–61. doi:10.1098/rsnr.1991.0002. ISSN 0035-9149. JSTOR 531520. S2CID 145338571。于2023年4月13日存档。于2023年5月8日获取。
\item Gandt, F. D. (2014). 《牛顿的《原理》中的力与几何》。普林斯顿大学出版社,第ix–xii页。ISBN 978-1-4008-6412-6。于2023年7月2日存档。于2023年5月8日获取。  
\item Bos, H. J. M. (1980). Grattan-Guinness, I. (编). 《从微积分到集合论 1630-1910:一部入门历史》(第一版)。普林斯顿大学出版社,第49–50, 54页。ISBN 978-0-691-07082-7。  
\item Sastry, S. Subramanya. 《关于微积分发明的牛顿-莱布尼茨争论》(PDF),威斯康星大学麦迪逊分校,第3页,doi:10.1214/ss/1028905930,原文于2024年1月8日存档(PDF),于2023年10月12日获取。  
\item More, Louis Trenchard (1934). 《艾萨克·牛顿传》。多佛出版公司,第327页。
\item Cheng, K. C.; Fujii, T. (1998). 《艾萨克·牛顿与热传递》。热传递工程,19 (4):9–21。doi:10.1080/01457639808939932。ISSN 0145-7632。  
\item 《大英百科全书:或,艺术、科学及通用文献词典》。第八卷。亚当与查尔斯·布莱克出版社,1855年,第524页。  
\item Sanford, Fernando (1921). 《关于电力的一些早期理论——电的发散理论》。科学月刊,12 (6):544–550。Bibcode:1921SciMo..12..544S。ISSN 0096-3771。  
\item Rowlands, Peter (2017). 《牛顿 - 创新与争议》。世界科学出版社,第109页。ISBN 9781786344045。
\item Hatch, Robert A. (1988). 《艾萨克·牛顿爵士》。原文存档于2022年11月5日。检索于2023年6月13日。  
\item Storr, Anthony (1985年12月). 《艾萨克·牛顿》。英国医学杂志(临床研究版),291 (6511):1779–84。doi:10.1136/bmj.291.6511.1779。JSTOR 29521701。PMC 1419183。PMID 3936583。  
\item Keynes, Milo (2008年9月20日). 《平衡牛顿的心智:他独特的行为与1692-93年的疯狂》。伦敦皇家学会记录与笔记,62 (3):289–300。doi:10.1098/rsnr.2007.0025。JSTOR 20462679。PMID 19244857。  
\item Westfall 1980,第55页。
\item 《牛顿与数学家》,Z. Bechler主编,《当代牛顿研究》(多德雷赫特 1982),第110–111页。  
\item Westfall 1994,第16–19页。  
\item White 1997,第22页。  
\item Westfall 1980,第60–62页。  
\item Westfall 1980,第71、103页。
\item Taylor, Henry Martyn (1911). 《牛顿,艾萨克爵士》。在Chisholm, Hugh(编)。《大英百科全书》第19卷(第11版)。剑桥大学出版社,第583页。  
\item Hoskins, Michael, ed. (1997). 《剑桥插图天文学史》。剑桥大学出版社,第159页。ISBN 978-0-521-41158-5。  
\item 牛顿,艾萨克。《废书》。剑桥大学数字图书馆。于2012年1月8日存档。于2012年1月10日检索。
\item Struik, Dirk J. (1948). 《简明数学史》。多弗出版公司,第151、154页。  
\item McDonald, Kerry (2020年3月27日). 《艾萨克·牛顿如何将大瘟疫中的孤立转变为“奇迹之年”》。fee.org。于2023年9月24日存档。于2023年10月14日检索。  
\item 牛顿,艾萨克(NWTN661I)”。剑桥校友数据库。剑桥大学。
\item Westfall 1980,第178页。  
\item Westfall 1980,第179页。  
\item Westfall 1980,第330–331页。  
\item White 1997,第151页。  
\item Warntz, William (1989). 《牛顿、牛顿派及《地理总论》》。美国地理学会年刊。79 (2): 165–191. doi:10.2307/621272. JSTOR 621272. 于2024年6月9日检索。  
\item Keighren, Innes M. (2006). 《流通的煽动性知识:威廉·麦金托什的“大胆荒谬、研究中的曲解和可憎的谎言”》。在Jöns, Heike; Meusburger, Peter; Heffernan, Michael (主编). 《知识的流动性》。施普林格开放。ISBN 978-3-319-44653-0。
\item Baker, J. N. L. (1955). 《伯恩哈德·瓦雷纽斯的地理》。英国地理学会会刊与论文。21 (21): 51–60. doi:10.2307/621272. JSTOR 621272。  
\item Schuchard, Margret (2008). 《关于《地理总论》及其在英格兰和北美的引入的笔记》。在Schuchard, Margret (主编). 《伯恩哈德·瓦雷纽斯(1622-1650)》。布里尔。第227–237页。ISBN 978-90-04-16363-8。于2024年6月9日检索。  
\item Mayhew, Robert J. (2011). 《地理的谱系》。在Agnew, John A.; Livingstone, David N. (主编). 《SAGE地理知识手册》。SAGE出版公司。ISBN 978-1-4129-1081-1。  
\item Ball 1908,第319页。


\end{enumerate}
