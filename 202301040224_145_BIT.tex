% 树状数组
% keys 树状数组|数据结构|C++

\subsection{基本原理}

若想实现一下两种操作:
\begin{enumerate}
\item 求一个区间内所有元素的和;
\item 修改某个元素的值。
\end{enumerate}

看到求一段序列的和很容易想到前缀和算法,单次查询的时间复杂度为 $\mathcal{O}(1)$,但是修改某个元素的值会影响前缀和数组,最坏为 $\mathcal{O}(n)$。若用普通数组,求一段数的和为 $\mathcal{O}(n)$,修改某个数为 $\mathcal{O}(1)$。若有 $m$ 次询问,两种做法的全局最坏时间复杂度都为 $\mathcal{O}(n \times m)$。树状数组这两种的操作的时间复杂度即不太慢又不太快,单次查询和修改时间复杂度都为 $\mathcal{O}(\log_2 n)$。

树状数组的基本思想来源于二进制拆分优化。对于一个正整数 $x$,它的二进制表示为 $a_{k - 1}, a_{k - 2}, \cdots , a_1, a_0$。可以将 $x$ 用二进制为 $1$ 的位表示出来,$x = 2^{i_1} + 2^{i_2} + \cdots + 2^{i_{k - 1}} + 2^{i_k}$。

其中 $i_1 > i_2 > \cdots > i_k$,可以将 $x$ 划分为 $\mathcal{O}(\left\lceil \log_2 x \right\rceil)$ 个区间。

\begin{enumerate}
\item 长度为 $2^{i_k}$ 的区间 $[x - 2^{i_k} + 1 , x]$;
\item 长度为 $2^{i_{k - 1}}$ 的区间 $[x - 2^{i_k} - 2^{i_{k - 1}} + 1, x - 2^{i_k}]$;
\item 长度为 $2^{i_{k - 2}}$ 的区间 $[x - 2^{i_k} - 2^{i_{k - 1}} -2^{i_{k - 2}} + 1, x - 2^{i_k} - 2^{i_{k - 1}}]$; \\
$\cdots$
\item 长度为 $2^{i_{1}}$ 的区间 $[x - 2^{i_k} - 2^{i_{k - 1}} -2^{i_{k - 2}} - \cdots -2^{i_1} + 1, x - 2^{i_k} - 2^{i_{k - 1}} - \cdots - 2^{i_2}]$。
\end{enumerate}

例如 $x = 7$,可以表示为 $2^2+2^1+2^0$,区间 $[1, 7]$ 可以分解成 $[1, 4]$、$[5, 6]$、$[7, 7]$ 三个区间。长度分别为 $2^2$、$2^1$、$2^0$。将这三个区间分别用二进制表示出来 $[1, 4] = [(1, 100)_2]$、$[5, 6] = [(101, 110)_2]$、$[7, 7] = [(111, 111)]$。可以发现每个区间的长度就是每个区间的右端点\textbf{二进制表示下最后一位 $1$ 及其后边的所有的 $0$。}就拿 $[5, 6]$ 这个区间举例,二进制表示下右端点为 $(110)_2$,最后一位 $1$ 及后面的所有的 $0$ 就是 $(10)_2 = (2)_{10}$,其区间长度正好为 $2$。

进而引出了 $\tt lowbit$ 操作。

$\tt lowbit$ 操作就是求一个数二进制表示下最后一位 $1$ 及其后边的所有的 $0$ 的数值。

\begin{lstlisting}[language=cpp]
int lowbit(x)
{
    return x & -x;
}
\end{lstlisting}

拿 $(20)_{10}$ 来举例,二进制表示下为 $(10100)_2$,最后一位 $1$ 及其后边的所有的 $0$ 就是 $(100)_2$,转化为十进制后就是 $4$,所以若调用 \verb|lowbit(20)|,则会返回 $4$。

树状数组就是基于上述的思想的数据结构,一般是拿树状数组维护一个序列的前缀和。令 $tr_x$ 维护区间 $\texttt{[x-lowbit(x)+1, x]}$ 的和。其结构可以用下图表示出来:

\begin{figure}[ht]
\centering
\includegraphics[width=13cm]{./figures/BIT_2.png}
\caption{树状图} \label{BIT_fig2}
\end{figure}

不难看出其中具有一些性质:

\begin{itemize}
\item 若 $x$ 为奇数,则 $tr_x = a_x$,并且长度都为 $1$。
\item $tr_x$ 的父结点为 $\texttt{tr[x + lowbit(x)]}$。
\item 每个节点 $x$ 的 $tr$ 数组的长度为 $\tt (lowbit(x))$。
\item 树的深度为 $\log_2 n + 1$。
\end{itemize}

\subsubsection{操作一:区间求和}

例如若要计算 $[1, 7]$ 的和,则要加 $tr_7$、$tr_6$、$tr_4$。可以发现,每次将 $x$ 减去 $\tt(lowbit(x))$ 就可以找到前一个要加的结点。所以树状数组维护序列 $1 \sim x$ 代码为:

\begin{lstlisting}[language=cpp]
int ask(int x)
{
    int res = 0;
    for (; x; x -= lowbit(x)) res += tr[x];
    return res;
}
\end{lstlisting}

涉及到的结点最多为 $\log_2 n$,所以时间复杂度最坏为 $\mathcal{O}(\log_2 n)$。若要求 $\sum\limits^r_{i = l}a_i$,类似于前缀和,则直接输出 \verb|ask(r) - ask(l - 1)|。

\subsubsection{操作二:单点修改}

若要将 $a_x$ 加上 $k$,则不断向上找出包含它的结点并且都加 $k$,因为每个结点维护的一个前缀的和。涉及到的结点最多为 $\log_2 n$,所以时间复杂度最坏为 $\mathcal{O}(\log_2 n)$。

\begin{lstlisting}[language=cpp]
void add(int x, int k)
{
    for (; x <= n; x += lowbit(x)) tr[x] += k;
}
\end{lstlisting}

\subsection{树状数组求逆序对}

对于一个序列 $a$,若存在两个数 $i$ 和 $j$,满足 $i < j$ 且 $a_i > a_j$,则 $a_i$ 和 $a_j$ 构成逆序对。

普通的做法怎么计算逆序对呢?可以开一个数组 $t$,初始化全 $0$,维护 $t$ 的前缀和。然后倒序扫描整个序列,每次计算 $a_i - 1$ 的前缀和,然后将 $t_{a_i}$ 加一,因为是倒序扫描,$[1, a_i - 1]$ 的前缀和就是已经出现过的数,并且在原序列中是在 $a_i$ 后面出现的数。所以就可以求出答案。

举个例子:对于一个序列 $a = (3, 4, 2, 5, 1)$。

\begin{lstlisting}[language=cpp]
第一次循环 i = 5,a_5 = 1,前缀和为 0,将 t[1] ++。
        1 2 3 4 5 (下标)
        3 4 2 5 1
t 数组: 1

ans = 0


第二次循环 i = 4,a_4 = 5,1 ~ 4 的前缀和为 1,答案加一,t[5] ++。
        1 2 3 4 5 (下标)
        3 4 2 5 1
t 数组:1       1

ans = 1


第三次循环 i = 3, a_3 = 2, 1 ~ 1 的前缀和为 1,答案加一,将 t[2] ++。
        1 2 3 4 5 (下标)
        3 4 2 5 1
t 数组:1 1     1

ans = 1 + 1


第四次循环 i = 2, a_2 = 4, 1 ~ 3 的前缀和为 2,答案加二,将 t[4] ++。
        1 2 3 4 5 (下标)
        3 4 2 5 1
t 数组:1 1   1 1

ans = 1 + 1 + 2


第五次循环 i = 1, a_1 = 3, 1 ~ 2 的前缀和为 2,答案加二,将 t[3] ++。
        1 2 3 4 5 (下标)
        3 4 2 5 1
t 数组:1 1 1 1 1

ans = 1 + 1 + 2 + 2

所以序列 3 4 2 5 1 的逆序对的数量就为 6
\end{lstlisting}

普通做法求逆序对的操作有:求一段数的前缀和,将某个数加一,树状数组正好能做。

\begin{lstlisting}[language=cpp]
for (int i = n; i; i -- )
{
    ans += ask(a[i] - 1);
    add(a[i], 1);  // 相当于 t[a[i]] ++ 
}
\end{lstlisting}

\subsection{树状数组的扩展应用}

既然树状数组支持区间查询和单点修改,那支不支持单点查询区间修改、区间查询和区间修改呢?答案是可以的,需要用到差分和前缀和的思想。

首先来看区间修改,单点查询。

首先开一个数组 $b$,初始化为 $0$,对于将一段区间加 $x$ 的操作,就将 $b_l$ 加 $x$、$b_{r + 1}$ 减 $x$。用树状数组维护 $b$ 的前缀和,观察一下 $b$ 数组的前缀和对原数组的影响。

可以发现,对于 $l$ 前面的数,$b$ 的前缀和不变,$[l \sim r]$ 的数,$b$ 的前缀和加了 $x$,$r$ 后边的数,$b$ 的前缀和加了先 $x$ 又减了 $x$,相当于没变。这样 $b$ 的前缀和就成了 $a$ 数组的增量。

所以对于区间修改,每次只需执行 \verb|add(l, x), add(r + 1, -x)|。对于单点查询,只需输出 \verb|a[x] + ask(x)|。

\begin{lstlisting}[language=cpp]
原数组:    1 3 4 2 5
b 数组:    0 0 0 0 0

操作一:区间加:2 ~ 4 都加一

原数组:    1 3 4 2 5
b 数组:    0 1 0 0 -1

操作二:查询 a_4 的值 = a_4 + b_4 = 2 + 1 = 3
操作三:查询 a_5 的值 = a_5 + b_5 = 5 + 0 = 5
\end{lstlisting}