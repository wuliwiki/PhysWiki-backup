% 角的概念(高中)
% 高中|角的概念

\subsection{角的概念的推广}
按逆时针方向旋转形成的角叫做\textbf{正角};按顺时针方向旋转形成的角叫做\textbf{负角};如果一条射线从起始位置没有作任何旋转,或终止位置与起始位置重合,我们称这样的角为\textbf{零度角},又称\textbf{零角},记作 $\alpha = 0$

\begin{definition}{999}

\end{definition}

角的终边(除端点外)在第几象限,我们就说这个角是第几象限角.

一般地,所有与角 $\alpha$ 终边相同的角,连同角 $\alpha$ 在内,可构成一个集合
\begin{equation}
S = \begin{Bmatrix} \beta|\beta=\alpha+2k\pi,k \in Z \end{Bmatrix}
\end{equation}

\textsl{注:$2k\pi$ 是弧度制写法,在下面介绍}

\subsection{弧度制}
同样的圆心角所对的弧长与半径之比是常数,我们称这个常数为该角的\textbf{弧度数}.

在单位圆中长度为 $1$ 的弧,所对的圆心角称为 \textbf{$1$ 弧度}角.它的单位符号是 $rad$,读作\textbf{弧度}.

一般地,正角的弧度数是一个正数,负角的弧度数是一个负数,零角的弧度数是 $0$.这种以弧度作为单位来度量角的单位制,叫做\textbf{弧度制}.这种度量方法有效地把角的弧度单位与长度单位统一起来.弧度制确立了角的弧度数与十进制实数间的一一对应关系.

我们知道,圆周率的定义是
\begin{equation}\label{HsAngl_eq2}
\pi = \frac{C}{d}
\end{equation}
在单位圆中,
\begin{equation}\label{HsAngl_eq3}
\begin{aligned}
r &= 1\\
d &= 2
\end{aligned}
\end{equation}
由\autoref{HsAngl_eq2} 和\autoref{HsAngl_eq3} 可得
\begin{equation}
C = 2\pi
\end{equation}
根据弧度数的定义,我们可以得到周角 $\alpha$ 的弧度数
\begin{equation}
\alpha = C / r = 2\pi
\end{equation}

弧度制的优点在于,我们可以快速得到角、半径和弧长的关系
\begin{equation}
\begin{aligned}
l &= \alpha \cdot r \\
\alpha &= \frac{l}{r} \\
r &= \frac{l}{\alpha}
\end{aligned}
\end{equation}
