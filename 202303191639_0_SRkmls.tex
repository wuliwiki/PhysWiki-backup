% 狭义相对论运动学(无质量粒子)
% 狭义相对论|无质量粒子|相对论光学|电磁波

\pentry{世界线和固有时\upref{wdline}}

约定使用东海岸度规 $\eta_{\mu\nu}=\rm{diag}(-1,1,1,1)$ 和自然单位制 $c=\hbar=1$。
\subsection{四速度、四动量、波四矢量}
对于无质量粒子,它的世界线为类光世界线,$V^\mu=\dd x^\mu(\lambda)/\dd \lambda$ 满足 $V^\mu V_\mu = 0$。我们无法用固有时对类光世界线进行参数化,因为在类光世界线上的任意两点的时空间隔都为 $0$。那么我们只能寻求其他的参数化方式。在狭义相对论中,无质量粒子不会受到力的作用,因此其四速度是恒定的,世界线在时空中是一条直线。例如以下就是一条无质量粒子的世界线:
\begin{equation}
x^\mu(\lambda)=b^\mu \lambda,\quad b^\mu=(1,1,0,0)
\end{equation}
容易验证 $V^\mu=(1,1,0,0)$ 满足 $V^\mu V_\mu = 0$。

无质量粒子的能量与波的频率 $\omega$ 成正比,动量与波数 $\bvec k$ 成正比,即\textbf{德布罗意关系}。那么可以定义其四动量为
\begin{equation}
p^\mu=(E,\bvec P)=(\hbar \omega,\hbar \bvec k)=\hbar k^\mu
\end{equation}
$\hbar$ 为\textbf{约化普朗克常数}。在自然单位制下,$p^\mu=(E,\bvec P)=(\omega,\bvec k)=k^\mu$。四动量方向应当与四速度方向一致,即
\begin{equation}
\hat p,\hat k \propto \hat V=\dv{\hat x(\lambda)}{\lambda}
\end{equation}
因此它满足爱因斯坦质能关系
\begin{equation}
p^\mu p_\mu = E^2-|\bvec P|^2=\omega^2-|\bvec k|^2=0
\end{equation}
\subsection{多普勒效应}
考虑一个光源,在它的参考系中光子沿各个方向以频率 $\omega$ 发射。而对于接收器,假设它相对于光源以速度 $\bvec v=-v \hat e_x$ 运动($v>0$ 代表相向,$v<0$ 代表背向运动)。

假设在光源参考系 $S$ 中,以 $\alpha$ 角发射的光子被接收器接收到。那么
假设在观测者静止的参考系 $S'$ 中,在观测者眼中光子是以 $\alpha'$ 角接收到的,即观测到的光子具有波四矢量
\begin{equation}
k'^\mu = \frac{2\pi}{\lambda'}(1,\cos\alpha',\sin\alpha',0)
\end{equation}
$\lambda'$ 为参考系 $S'$ 中接收器观测到的光子波长。

那么可以通过一个 $\bvec v$ 方向的洛伦兹变换
\begin{equation}
\Lambda^\mu{}_\nu=\begin{pmatrix}
\gamma & -v\gamma &  & \\
-v\gamma & \gamma &  & \\
& & 1 & \\
& & & 1
\end{pmatrix}
\end{equation}
那么我们希望
\begin{equation}
k^\mu = \Lambda^\mu{}_\nu k'^\nu = \frac{2\pi}{\lambda}(1,\cos\alpha,\sin\alpha,0)
\end{equation}
转化为求解以下方程组
\begin{equation}
\begin{aligned}
& \gamma\cdot\frac{2\pi}{\lambda'}\left(1-v\cos\alpha'\right)=\frac{2\pi}{\lambda}\\
& \gamma\cdot\frac{2\pi}{\lambda'}\left(\cos\alpha'-v\right)=\frac{2\pi}{\lambda}\cos\alpha\\
& \frac{2\pi}{\lambda'}\sin\alpha' = \frac{2\pi }{\lambda}\sin\alpha
\end{aligned}
\end{equation}
经过适当的化简最终可以得到
\begin{equation}
\begin{aligned}
&\frac{\lambda'}{\lambda}=\gamma(1-v\cos\alpha')\\
&\tan\alpha=\frac{\tan\alpha'}{\gamma(1-v\sec \alpha')}
\end{aligned}
\end{equation}
第一个公式可以换作频率的表达式 $\nu=2\pi/\lambda$:
\begin{equation}
\nu'=\frac{\nu}{\gamma(1-v\cos\alpha')}
\end{equation}
