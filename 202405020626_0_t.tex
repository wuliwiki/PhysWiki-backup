% 【目录】个人目录
% keys 线性代数|数学物理|数学分析|拓扑|量子力学
% license Usr
% type Toc

\subsection{科研}
\enref{氦原子数值解 TDSE 笔记}{HeTDSE},
\enref{氦原子波函数数值分析}{HeAnal},
\enref{氢原子的 streaking 计算}{HyCLC},
\enref{光电离时间延迟:从一维波包到氦原子}{HeAna2},
\enref{氦原子数值解 TDSE 笔记}{HeTDSE},
\enref{氦原子波函数数值分析(笔记)}{HeAnal},
\enref{Berkeley-ECS 方法}{BerECS}

\subsection{线性代数}
\enref{投影算符}{projOp}, \enref{矩阵的迹}{trace}

\subsection{数学物理}
\enref{连带勒让德函数}{AsLgdr}, \enref{球谐函数}{SphHar}, \enref{广义球谐函数}{GenYlm}, \enref{平面波的球谐展开}{Pl2Ylm}, \enref{库仑势能的球谐展开}{PChYlm}, \enref{库仑函数}{CulmF}, \enref{库仑波函数}{CulmWf}, \enref{Wigner 3j 符号}{ThreeJ}, \enref{Wigner 6j 符号}{SixJ}, \enref{Wigner 9j 符号}{NineJ}, \enref{张量积空间}{DirPro}

\subsection{数学分析}
\enref{黎曼积分与勒贝格积分}{Rieman}, \enref{数学分析笔记}{AnalNt}, \enref{泛函分析笔记1}{FnalNt}, \enref{泛函分析笔记2}{FnalN2}, \enref{泛函分析笔记3}{FnalN3}, \enref{泛函分析笔记4}{FnalN4}, \enref{泛函分析笔记5}{FnalN5}

\subsection{代数}
\enref{群论笔记}{GroupN}

\subsection{拓扑学}
\enref{拓扑空间}{Topol}, \enref{流形}{Manif}

\subsection{力学}
\enref{拉普拉斯—龙格—楞次矢量}{LRLvec}, \enref{哈密顿正则方程}{HamCan}, \enref{分析力学笔记}{ClsMec},\enref{工业仿真软件(电磁、流体、多物理)}{ChSciN}

\subsection{电磁}
\enref{电多极展开}{EMulPo}, \enref{拉格朗日电磁势}{EMLagP}

\subsection{相对论}
\enref{《微分几何入门与广义相对论》笔记}{DGGRNt}

\subsection{量子力学}
\enref{全同粒子}{IdPar}, \enref{含时微扰理论}{TDPT}, \enref{量子散射的延迟}{tDelay}, \enref{多通道散射}{MulSct}, \enref{电磁场中的单粒子薛定谔方程}{QMEM}, \enref{长度规范和速度规范}{LVgaug}, \enref{加速度规范}{AccGau}, \enref{Volkov 波函数}{Volkov}, \enref{Keldysh 参数}{keldis}, \enref{密度矩阵}{denMat}
\enref{Adiabatic 笔记}{Adibat},
\enref{Hall 量子力学笔记}{HallQM}

\subsection{计算物理}
\enref{氢原子薛定谔方程数值解}{HyTDSE}, \enref{氢原子球坐标数值解 TDSE}{HTDSE}, \enref{氢原子电离截面}{HionCr},  \enref{单电子原子模型}{SAE}, \enref{物理仿真软件列表(笔记)}{PhySim}

\subsection{科普笔记}
%---------------------------------------
% 这些文章只是列出大纲
\enref{经典力学笔记(科普)}{CMInt},
\enref{天文学笔记(科普)}{AstrIn},
\enref{电磁学笔记(科普)}{EleMag},
\enref{光学笔记(科普)}{OptiIn},
\enref{相对论笔记(科普)}{RelaIn},
\enref{热力学笔记(科普)}{HeatIn},
\enref{原子分子笔记(科普)}{AtomIn},
\enref{宇宙学笔记(科普)}{CosmIn}
