% 电磁脉冲
% license CCBYSA3
% type Wiki

(本文根据 CC-BY-SA 协议转载自原搜狗科学百科对英文维基百科的翻译)

电磁脉冲(Electromagnetic Pulse,EMP),有时也称为瞬变电磁干扰,是电磁能量的短脉冲。这种脉冲的来源可以是自然发生的,也可以是人为的,并且根据来源的不同,可以是辐射场、电场、磁场或传导电流。

电磁脉冲干扰通常会破坏或损坏电子设备,在较高的能量水平下,强电磁脉冲(如雷击)会损坏建筑物和飞机机体的物理结构。电磁脉冲效应的管理是电磁兼容( electromagnetic compatibility,EMC)工程的一个重要分支。

释放高能电磁脉冲的破坏性武器已经被开发出来。

\subsection{一般特性}
电磁脉冲是电磁能量的短脉冲。它的短持续时间意味着它将在一个频率范围内传播。电磁脉冲的典型特征是:
\begin{itemize}
\item 能量类型(辐射、电、磁或传导)。
\item 存在的频率范围或频谱。
\item 脉冲波形:形状、持续时间和振幅。
\end{itemize}
最后两种,频谱和脉冲波形通过傅里叶变换相互关联,可以被视为描述同一脉冲的两种不同方式。
\subsubsection{1.1 能量类型}
电磁脉冲能量可以凭借如下四种形式传递:
\begin{itemize}
\item 电场
\item 磁场
\item 电磁辐射
\item 电传导
\end{itemize}
由于麦克斯韦方程,任何一种形式的电磁能脉冲总是伴随着其他形式电磁能量,然而在典型的脉冲中,一种形式将占主导地位。

一般来说,只有辐射传递方式可以用于长距离传递,其他辐射仅适用于短距离传递。但也有一些例外,比如太阳磁耀斑。
\subsubsection{1.2 频率范围}
电磁能量脉冲通常包括从直流DC(零赫兹)到某个上限的许多频率,这种频率范围取决于辐射源。电磁脉冲的频率范围,有时被称为“直流至日光”,不包括由光学(红外、可见、紫外)和电离(x光和伽马射线)组成的最高频率范围。

某些类型的电磁脉冲事件会留下光学痕迹,如闪电和火花,但这些都是流经空气的电流的副作用,并不属于电磁脉冲本身。
\subsubsection{1.3 脉冲波形}
脉冲波形描述了其瞬时振幅(场强或电流)随时间的变化情况。真实的脉冲往往非常复杂,所以经常使用简化模型。这种模型通常用图表或数学方程来描述。
\begin{figure}[ht]
\centering
\includegraphics[width=14.25cm]{./figures/06d6169b006b519c.png}
\caption{} \label{fig_DCMC_1}
\end{figure}
大多数电磁脉冲都有一个非常尖锐的前沿,迅速积累到最大水平。经典模型是一条双指数曲线,它急剧上升,迅速达到峰值,然后以很慢的速度进行衰减。然而,来自受控开关电路的脉冲通常近似于矩形或“正方形”脉冲的形式。

电磁脉冲事件通常会在周围环境或材料中引发相应的信号。强耦合通常发生在相对较窄的频带上,导致特征阻尼正弦波。在视觉上,在双指数曲线的较长寿命包络内,可以直观的看到高频正弦波的增长和衰减。由于耦合模式的传输特性,阻尼正弦波通常比原始脉冲具有低得多的能量和更窄的频带宽度。实际上,电磁脉冲测试设备经常直接注入这些阻尼正弦波,而不是试图重现高能威胁脉冲。

在脉冲序列中,例如在数字时钟电路中,波形以规则的间隔重复。单个完整的脉冲周期足以描述这样一个有规律的、重复的序列。

\subsection{类型}
电磁脉冲产生于辐射源发出短时间能量脉冲的地方。尽管EMP通常在周围环境中激发相对窄带的阻尼正弦波响应,但能量本质上通常是宽带的。有些类型是作为重复和规则的脉冲序列产生的。

不同类型的电磁脉冲:有来自自然、人为和武器效应的电磁脉冲。

自然电磁脉冲事件的类型包括:
\begin{itemize}
\item 雷电电磁脉冲(LEMP)。放电通常是最初的大电流,至少是百万安培,然后是一系列能量下降的脉冲。
\item 静电放电(ESD),由于两个带电物体非常接近甚至接触。
\item 流星电磁脉冲。由流星体与宇宙飞船碰撞或穿过地球大气层的流星体爆炸破裂而产生的电磁能量的释放。[1][2]
\item 日冕物质抛射(CME)。等离子体爆发和伴随的磁场,从日冕中喷出并释放到太阳风中。有时被称为太阳电磁脉冲。[3]
\end{itemize}

(民用)人为电磁脉冲事件的类型包括:
\begin{itemize}
\item 电路的开关动作,无论是隔离的还是重复的(如脉冲序列)。
\item 当电枢旋转时,内部电触点接通和断开连接时,电动机可以产生一系列脉冲。
\item 汽油发动机点火系统可以在火花塞通电或点火时产生一系列脉冲。
\item 数字电子电路的连续开关动作。
\item 电力线路。这些电压可能高达几千伏,足以损坏保护措施不足的电子设备。
\end{itemize}

军用电磁脉冲的类型包括:
\begin{itemize}
\item 核爆炸产生的核电磁脉冲(NEMP)。另一种变体是高空核电磁脉冲(HEMP),是由于粒子与地球大气和磁场的相互作用所产生二次脉冲。
\item 无核电磁脉冲(NNEMP)武器。
\end{itemize}
\subsubsection{2.1 闪电}
