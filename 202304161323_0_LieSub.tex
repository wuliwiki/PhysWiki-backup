% 李代数的子代数、理想与商代数
% keys 李代数|子代数|理想|正规化子

\begin{issues}
\issueOther{没有定义商代数}
\issueOther{可以考虑拆分}
\end{issues}

\pentry{李代数\upref{LieAlg}}

\subsection{子代数与理想}
和抽象代数中的子群、子环等类比,李代数也可以有次级结构,即子代数。

设 $\mathfrak{m}, \mathfrak{n}$ 是李代数 $\mathfrak{g}$ 的非空子集,定义子集间的运算为 $\mathfrak{m}+\mathfrak{n}=\{M+N|M\in\mathfrak{m}, N\in\mathfrak{n}\}$,以及 $[\mathfrak{m}, \mathfrak{n}]=\{[M, N]|M\in\mathfrak{m}, N\in\mathfrak{n}\}$。那么如果 $\mathfrak{m}, \mathfrak{n}$ 和 $\mathfrak{p}$ 都是 $\mathfrak{g}$ 作为线性空间的子空间,我们容易证明以下性质:

\begin{itemize}
\item $[\mathfrak{m}+\mathfrak{n}, \mathfrak{p}]\subseteq[\mathfrak{m}, \mathfrak{p}]+[\mathfrak{n}, \mathfrak{p}]$;
\item $[\mathfrak{m},\mathfrak{n}]=[\mathfrak{n}, \mathfrak{m}]$;
\item $[\mathfrak{m}, [\mathfrak{n}, \mathfrak{p}]]\subseteq[\mathfrak{n}, [\mathfrak{m}, \mathfrak{p}]]+[\mathfrak{p}, [\mathfrak{n}, \mathfrak{m}]]$。
\end{itemize}

\begin{definition}{李代数的子代数}
若 $\mathfrak{g}$ 是李代数,$\mathfrak{h}$ 是它作为线性空间的子空间,且有 $[\mathfrak{h}, \mathfrak{h}]\subseteq\mathfrak{h}$,那么称 $\mathfrak{h}$ 是 $\mathfrak{g}$ 的\textbf{子代数(sub (Lie) algebra)}。
\end{definition}


尽管李代数并不成环,但是我们也可以仿照环的理想的定义,用吸收律来定义李代数的理想:

\begin{definition}{李代数的理想}
若 $\mathfrak{g}$ 是李代数,$\mathfrak{h}$ 是它作为线性空间的子空间,且有 $[\mathfrak{g}, \mathfrak{h}]=\mathfrak{h}$,那么称 $\mathfrak{h}$ 是 $\mathfrak{g}$ 的\textbf{理想(ideal)}。

显然,$\mathfrak{g}$ 和 $\{0\}$ 都是 $\mathfrak{g}$ 的理想,称为\textbf{平凡理想(trivial ideal)}。
\end{definition}

我们通常把定义中 $[\mathfrak{g}, \mathfrak{h}]=\mathfrak{h}$ 这一条,称为“吸收性”或者说“吸收律”,方便记忆。从定义可以看到,李代数的理想必为其子代数。进一步,理想和子代数还满足以下性质:

\begin{itemize}
\item 如果 $\mathfrak{h}_i$ 是子代数,那么 $\mathfrak{h}_1\cap\mathfrak{h}_2$ 也是子代数。
\item 如果 $\mathfrak{h}_1$ 是子代数而 $\mathfrak{h}_2$ 是理想,那么 $\mathfrak{h}_1+\mathfrak{h}_2$ 是子代数。
\item 如果 $\mathfrak{h}_i$ 是理想,那么 $\mathfrak{h}_1+\mathfrak{h}_2$、$\mathfrak{h}_1\cap\mathfrak{h}_2$ 和 $[\mathfrak{h}_1, \mathfrak{h}_2]$ 都是理想,并且有如下包含关系:\begin{equation}
[\mathfrak{h}_1, \mathfrak{h}_2]\subseteq\mathfrak{h}_1\cap\mathfrak{h}_2\subseteq\mathfrak{h}_i\subseteq\mathfrak{h}_1+\mathfrak{h}_2
\end{equation}

\end{itemize}

证明是很简单的,留作练习。注意最后的包含关系里的符号区别,其中一共三个子集符号。

\begin{example}{理想的例子}
域 $\mathbb{F}$ 上的 $n$ 阶可逆矩阵的集合 $\opn{gl}(n, \mathbb{F})$,构成一个李代数。考虑到对于任意矩阵 $A, B\in\opn{gl}(n, \mathbb{F})$,我们有 $\opn{trace}(AB)=\opn{trace}(A)\opn{trace}(B)=\opn{trace}(B)\opn{trace}(A)=\opn{trace}(BA)$,因此必有 $\opn{trace}([A, B])=0$。这就提示我们去考虑迹为 $0$ 的全体 $n$ 阶方阵的集合,记为 $\mathfrak{t}$,容易验证它就是 $\opn{gl}(n, \mathbb{F})$ 的一个理想。
\end{example}

\begin{exercise}{李代数的中心}
证明对于李代数 $\mathfrak{g}$,其中心(\autoref{def_LieAlg_1}~\upref{LieAlg})$C(\mathfrak{g})$ 构成一个理想。
\end{exercise}


\subsection{子李代数的直和}

\addTODO{如果单独开新词条的话应该加上外直和,和内外半直和}

\begin{definition}{李代数的内直和}
若 $\mathfrak{g}$ 是李代数,$\mathfrak{h}_1, \mathfrak{h}_2$ 是它的子李代数,我们有 $\mathfrak{h}_1 \cap \mathfrak{h}_2 = \{0\}$(作为向量空间是内直积)以及 $[\mathfrak{h}_1, \mathfrak{h}_2] = \{0\}$,那么称 $\mathfrak{h}_1 + \mathfrak{h}_2$ 构成一个新子李代数,记作 $\mathfrak{h}_1 \oplus \mathfrak{h}_2$。
\end{definition}

注意,作为向量空间的内直积和李代数的内直积的符号是一样的,需要根据语境来理解。

\subsection{半单李代数}

和\textbf{单群}\upref{SemGrp}类似,我们可以定义单李代数:

\begin{definition}{单李代数}
给定非交换非平凡李代数 $\mathfrak{g}$,如果它没有非平凡理想,即除了 $\{0\}$ 和 $\mathfrak{g}$ 自身外没有别的理想,那么我们称它为一个\textbf{单李代数(simple Lie algebra)}。
\end{definition}

正如平凡群不是单群,交换李代数($[\cdot, \cdot] = 0$)在一定程度上也很“平凡”,我们也不把它定义成单李代数。


\begin{definition}{半单李代数}\label{def_LieSub_1}
如果李代数 $\mathfrak{g}$ 能被表示成它的单子代数的直和,我们就称之为\textbf{半单李代数(semi-simple Lie algebra)},
\end{definition}

特别的,我们有
\begin{theorem}{半单李代数的中心}
半单李代数的中心是平凡的。
\end{theorem}


% 类似地,如果加一个限制,还有半单李代数的概念:
% \begin{definition}{半单李代数}
% 给定李代数 $\mathfrak{g}$,如果它没有非平凡\textbf{交换}理想,那么我们称它为一个\textbf{半单李代数(semi-simple Lie algebra)}。
% \end{definition}
