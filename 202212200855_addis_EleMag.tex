% 电磁学笔记(科普)

\begin{issues}
\issueDraft
\end{issues}

\subsection{电}
\begin{itemize}
\item 两种电荷: 正电荷, 负电荷。 从微观来讲, 负电荷由电子提供, 正电荷由质子提供。 详见 “原子分子笔记(科普)\upref{AtomIn}”
\item 通常来说, 当物体不带电时, 每一个局部都有相同数量的电子和质子, 它们的电荷大小相等符号相反, 所以从宏观来看物体不带电。
\item 同种电荷相吸, 异种电荷相斥。 这样的力叫做\textbf{库仑力}, 或者\textbf{电场力}。
\item 库仑定律: 两个带电粒子之间的库仑力和它们电荷的乘积成正比, 和距离平方成反比。
\item 库仑力的本质: 带电粒子现在其周围形成电场, 处于该电场中的另一个粒子受到电场的力。
\item 什么是场: 空间(一定范围内)每一个位置都对应一个矢量。
\item 例: 水流的速度场: 每一个位置都对应一个速度矢量, 即该位置水流的速度大小和方向。
\item 电场: 空间中每一点都对应一个电场矢量。 下文的磁场也同理。
\end{itemize}

\subsection{磁}
\begin{itemize}
\item 带电粒子在磁场中运动产生的力叫做\textbf{洛伦兹力}。
\item 洛伦兹力(矢量)取决于: 粒子的电荷(标量)、 粒子的速度(矢量)、粒子所在处的磁场(矢量)。
\end{itemize}
