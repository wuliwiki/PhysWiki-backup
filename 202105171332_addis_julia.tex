% Julia 简介

\begin{issues}
\issueDraft
\end{issues}

\subsection{命令行}
\begin{itemize}
\item 为了区分, 系统的控制行叫做 terminal, 而 julia 的控制行叫做 **REPL** (read-eval-print-loop)
\item \verb|Ctrl + D| 退出或者 \verb|exit()| 退出
\item \verb|ans| 和 Matlab 一样
\end{itemize}

\begin{lstlisting}[language=julia]
println("hello world")
\end{lstlisting}

\subsection{计算器}
\begin{itemize}
\item \verb|b = 2a| 是合法的
\item \verb|2/3| 这种返回小数而不是整数
\item 表示复数如 \verb|1 + 2im|, 类型应该是 \verb|ComplexF64|
\end{itemize}

\subsection{变量}
\begin{itemize}
\item 用 \verb|typeof()| 查看某个变量的类型
\item 类型完全是 dynamic 的, 但可以限制变量类型
\item 字符串用双引号, 单个字符用单引号
\item 变量名可以用 UTF-8, 在一些编辑器中可以用反斜杠 latex 命令打出对应的字符
\item 查看类型的最大和最小值如 \verb|typemax(Int64)|, \verb|typemin(Int64)|
\item julia 自带任意精度类型 \verb|BigInt|,  (底层是 GMP)
\end{itemize}

\subsection{矩阵}
矩阵切割 \verb|Psi[:, j, :]|

\verb|size(Psi, 维度)|

零向量 \verb|zeros(整数)|

随机矩阵
\begin{lstlisting}[language=julia]
rand(ComplexF64, Nr1, Nr2, Npw)
\end{lstlisting}

\subsection{脚本}
\begin{itemize}
\item 在系统控制行用 \verb|julia <file>| 运行脚本, 用 \verb|julia <file> <arg1> <arg2>| 给出 arguments
\item 在 REPL 中运行脚本如 \verb|include("/Users/addis/Desktop/main.jl")|(路径不区分大小写), 也支持反斜杠, 但要转义成 \verb|\\|
\item 在文件中, 如果要确定当前文件是不是主文件, 用 \verb|abspath(PROGRAM_FILE) == @__FILE__|
\end{itemize}

\subsection{画图}
第三方画图包 \verb|Plots|, 无需安装!
\begin{lstlisting}[language=julia]
using Plots
x = 1:10; y = rand(10); # These are the plotting data
plot(x,y)
\end{lstlisting}

\subsection{常用函数}
函数定义
\begin{lstlisting}[language=julia]
function sphere_vol(r)
    return 4/3*pi*r^3
end
\end{lstlisting}

hash (把结果的 hash 输出可以保证计算过程不被优化掉)
\begin{lstlisting}[language=julia]
hash(矩阵)
\end{lstlisting}

当前时间 \verb|time()|
