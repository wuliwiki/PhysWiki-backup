% 博苏克-乌拉姆定理(综述)
% license CCBYSA3
% type Wiki

本文根据 CC-BY-SA 协议转载翻译自维基百科\href{https://en.wikipedia.org/wiki/Borsuk\%E2\%80\%93Ulam_theorem}{相关文章}。

在数学中,博苏克–乌拉姆定理指出:每一个从 $n$ 维球面 $S^n$ 到 $n$ 维欧几里得空间 $\mathbb{R}^n$ 的连续函数,必定存在一对对踵点被映射到同一个点。这里,对踵点指的是位于球面上、从球心看方向完全相反的两点。

形式化表述:若$f: S^n \to \mathbb{R}^n$是一个连续函数,则存在$x \in S^n$使得$f(-x) = f(x)$。

特例说明:当 $ n = 1$:这意味着地球赤道上总存在一对正对着的点,它们的温度相同。这个结论也适用于任何圆周。注意这依赖于温度在空间中连续变化这一假设,而这在现实中不一定成立\(^\text{[1]}\)。

当 $ n = 2 $:可以解释为地球表面在任意时刻总存在一对对踵点,它们的温度和气压完全相同(假设这两个物理量都在空间中连续变化)。

与奇函数等价的其他表述:记$ S^n $为 $n$ 维球面,$ B^n $为 $n$维单位球体:
\begin{itemize}
\item 若  $g: S^n \to \mathbb{R}^n$是一个连续奇函数(即 $g(-x) = -g(x)$),则存在$x \in S^n$使得$g(x) = 0$。
\item 若 $g: B^n \to \mathbb{R}^n$是一个连续函数,且在边界 $S^{n-1}$ 上为奇函数,那么必存在$x \in B^n$使得$g(x) = 0$。
\end{itemize}
\subsection{历史}
据 Matoušek(2003年第25页)记载,博苏克–乌拉姆定理最早的历史性表述出现在 Lyusternik 与 Shnirel'man 于1930年的著作中。第一个正式的证明由卡罗尔·博苏克于1933年给出,他将这一问题的提出归功于斯坦尼斯瓦夫·乌拉姆。自那以后,许多作者都给出了该定理的不同证明方式,这些证明被 Steinlein(1985)汇总整理。
\subsection{等价表述}
以下命题与博苏克–乌拉姆定理是等价的\(^\text{[2]}\):
\subsubsection{关于奇函数的表述}
一个函数 $g$ 被称为奇函数(也称为对极函数或保对极函数),如果对于每一个 $x$,都有
$g(-x) = -g(x)$

博苏克–乌拉姆定理等价于以下两个命题:

1. 每一个连续的奇函数 $g : S^n \to \mathbb{R}^n$ 都有零点。\\
2. 不存在从 $S^n$ 到 $S^{n-1}$ 的连续奇函数。\\

证明:博苏克–乌拉姆定理等价于命题 (1)

($\Longrightarrow$) 如果博苏克–乌拉姆定理成立,那么它当然对奇函数也成立。而对于奇函数 $g$,若有$g(-x) = g(x)$则必须有$g(x) = 0$,因为又有 $g(-x) = -g(x)$。所以每个连续奇函数都有零点。

($\Longleftarrow$) 对于任意连续函数 $f : S^n \to \mathbb{R}^n$,可以构造一个连续奇函数$g(x) = f(x) - f(-x)$如果每个奇函数都有零点,那么 $g$ 存在某个零点 $x$,即$f(x) = f(-x)$,从而得出博苏克–乌拉姆定理成立。

为了证明命题 (1) 与命题 (2) 等价,使用以下连续奇函数:

\begin{itemize}
\item 明显的包含映射$i : S^{n-1} \to \mathbb{R}^n \setminus \{0\}$
\item 辐射投影映射$p : \mathbb{R}^n \setminus \{0\} \to S^{n-1}, \quad x \mapsto \frac{x}{|x|}$
\end{itemize}
证明如下:

((1) ⟹ (2))反设成立:若存在连续奇函数 $f : S^n \to S^{n-1}$,则复合映射 $i \circ f : S^n \to \mathbb{R}^n \setminus \{0\}$ 是一个连续奇函数,这与 (1) 每个连续奇函数都有零点相矛盾(因为值域不含零点)。

((1) ⟸ (2))同样使用反设:若存在连续奇函数 $f : S^n \to \mathbb{R}^n \setminus \{0\}$则 $p \circ f : S^n \to S^{n-1}$ 是连续奇函数,这与 (2) 不存在这样的映射相矛盾。因此,两者等价。
\subsection{证明}
\subsubsection{一维情形}
一维情形可以使用中值定理(IVT)轻松证明。

设 $g(x) = f(x) - f(-x)$ 是一个定义在圆上的实值连续奇函数。随便取一个 $x$。若 $g(x) = 0$,则定理成立。否则,**不妨设** $g(x) > 0$。由于 $g$ 是奇函数,有 $g(-x) = -g(x) < 0$。由中值定理知,存在某点 $y \in S^1$ 使得 $g(y) = 0$,从而有$f(y) = f(-y)$成立,证毕。
\subsubsection{一般情形}
\textbf{代数拓扑证明}

假设存在一个奇连续函数$h : S^n \to S^{n-1}$其中 $n > 2$($n=1$ 的情形已在前面证明,$n=2$ 的情形可用基本覆盖空间理论处理)。由于 $h$ 是奇函数,我们可以考虑将它在对极点作用下传递到商空间中,于是得到一个在实射影空间之间的诱导连续映射:
$h' : \mathbb{RP}^n \to \mathbb{RP}^{n-1}$这个映射在基本群之间诱导同构。根据 Hurewicz 定理,它在系数为 $\mathbb{F}_2$(两个元素的有限域)的上同调环之间诱导如下同态:
$$
H^*(\mathbb{RP}^n; \mathbb{F}_2) = \mathbb{F}_2[a]/(a^{n+1}) \leftarrow H^*(\mathbb{RP}^{n-1}; \mathbb{F}_2) = \mathbb{F}_2[b]/(b^n)~
$$
其中 $b \mapsto a$。但这样一来,左侧的 $b^n = 0$ 被映射为 $a^n \neq 0$,矛盾!

因此,假设不成立,也就是说,不存在这样的奇连续函数 $h$,这就证明了 Borsuk–Ulam 定理。

另外,也可以证明一个更强的结论:任何从 $S^{n-1} \to S^{n-1}$ 的奇映射,其度数为奇数。从这个结论也可以推出 Borsuk–Ulam 定理。

\textbf{组合证明}

Borsuk–Ulam 定理可以由 Tucker 引理推导而来\(^\text{[2][4][5]}\)。

设$g : S^n \to \mathbb{R}^n$是一个连续的奇函数。由于 $g$ 在紧空间上连续,它是一致连续的。因此,对于任意 $\epsilon > 0$,存在 $\delta > 0$,使得当 $S^n$ 上任意两点的距离小于 $\delta$ 时,它们在 $g$ 下的像之间的距离也小于 $\epsilon$。

接下来,对 $S^n$ 进行一个边长至多为 $\delta$ 的**三角剖分**。对每个顶点 $v$ 赋予一个标签 $l(v) \in \{\pm 1, \pm 2, \dots, \pm n\}$,具体规则如下:

\begin{itemize}
\item * 标签的绝对值是使 $g(v)$ 中某坐标的绝对值最大的那个坐标索引:

  $$
  |l(v)| = \arg \max_k |g(v)_k|~
  $$
* 标签的符号为该坐标上的符号:

  $$
  l(v) = \operatorname{sgn}(g(v)_{|l(v)|}) \cdot |l(v)|~
  $$
\end{itemize}

由于 $g$ 是奇函数,即满足 $g(-v) = -g(v)$,所以标签满足:

$$
l(-v) = -l(v)
$$

因此这是一个**奇标记**,可应用 Tucker 引理。根据该引理,存在一对相邻顶点 $u, v$,使得 $l(u) = -l(v)$。

不妨设 $l(u) = 1, l(v) = -1$,这说明:

* 在 $g(u)$ 与 $g(v)$ 中,**第 1 个坐标为最大坐标分量**
* 且 $g(u)_1 > 0$,$g(v)_1 < 0$

由于剖分边长不超过 $\delta$,所以 $g(u)$ 和 $g(v)$ 的欧几里得距离小于等于 $\epsilon$,特别地:

$$
|g(u)_1 - g(v)_1| = |g(u)_1| + |g(v)_1| \leq \epsilon
$$

由于 $g(u)_1 > 0, g(v)_1 < 0$,它们符号相反。

于是有

$$
|g(u)_1| \leq \epsilon
$$

而第 1 个坐标是最大坐标分量,所以对所有 $k \in \{1, \dots, n\}$,都有

$$
|g(u)_k| \leq \epsilon
$$

因此整向量的范数满足

$$
|g(u)| \leq c_n \epsilon
$$

其中 $c_n$ 是一个仅依赖于 $n$ 和所选范数的常数。

由于上面结论对任意 $\epsilon > 0$ 都成立,且 $S^n$ 是紧空间,必然存在某点 $u$,使得

$$
|g(u)| = 0
$$

即

$$
g(u) = 0
$$

从而证明了 Borsuk–Ulam 定理。
