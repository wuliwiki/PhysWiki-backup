% 张量的对称化和交错化
% 对称化|交错化
\pentry{张量的坐标\upref{CofTen},置换群\upref{Perm}}
张量可按选出来的共变或逆变指标的集合来讨论对称性和斜对称性.由于张量的共边和逆变指标的对称性(即共边和逆变或矢量和对偶矢量的区别仅在于原矢量空间的选择),在讨论张量的对称性和斜对称性时可直接将张量限制在 $(p,0)$ 型或 $(0,p)$ 型张量,而不减少一般性.比如:讨论张量 $T^{i_1 i_2\cdots i_p}$ 和 ${S_{i_1}}^{i_2\cdots i_p}$ 在指标 $\{i_1,i_2\}$ 上的对称性质没有任何区别,只需将 $T$ 的上指标 $i_1$ 当作下指标即可移值到张量 $S$.故在本词条里,我们总是将张量限制为 $(p,0)$ 型张量.并且讨论的是这 $p$ 个指标的对称性和斜对称性,而不是它的一部分(因为在讨论某些指标的(斜)对称性质的时候,只需将其它指标当作固定的就变成这里讨论的情形).
\subsection{张量的对称化}
\subsubsection{对称张量}
设 $T\in \mathbb{T}_p^0(V)$ (\autoref{CofTen_def1}~\upref{CofTen}),即
\begin{equation}
T=\sum_{i_1,\cdots,i_p}T_{i_1,\cdots,i_p}e^{i_1}\otimes\cdots\otimes e^{i_p}
\end{equation}
而 $S_p$ 是作用在指标集合 $\{1,2,\cdots,p\}$ 上的 $p$ 阶置换群\upref{Perm}.

对任意置换 $\pi\in S_p$ ,定义映射:
\begin{equation}\label{SIofTe_eq1}
f_\pi(T)(x_1,x_2,\cdots,x_p):=T(x_{\pi1},\cdots,x_{\pi p})
\end{equation}
其中,$x_i$ 是以 $i$ 为指标的矢量.

\begin{theorem}{}
$f_{\pi}(T)\in\mathbb{T}_p^0$ 且
\begin{equation}\label{SIofTe_eq2}
f_{\pi}(T)=\sum_{i_1\cdots i_p}T_{i_{\pi1}\cdots i_{\pi p}}e^{i_1}\otimes\cdots\otimes e^{i_p}
\end{equation}
或等价于
\begin{equation}\label{SIofTe_eq3}
f_{\pi}(T)=\sum_{i_1\cdots i_p}T_{i_1\cdots i_p}e^{i_{\pi^{-1}1}}\otimes\cdots\otimes e^{i_{\pi^{-1}p}}
\end{equation}
\end{theorem}
\textbf{证明:}

1.$f_\pi(T)\in\mathbb{T}_p^0$

不失一般性,设 $\pi k=1$,那么由\autoref{SIofTe_eq1} 
\begin{equation}
\begin{aligned}
f_\pi(T)(\alpha x_1+\beta y_1,x_2,\cdots,x_p)&=T(x_{\pi1},\cdots,\alpha x_1+\beta y_1,\cdots,x_{\pi p})\\
&=\alpha T(x_{\pi1},\cdots,x_1,\cdots,x_{\pi p})+\beta T(x_{\pi1},\cdots,y_1,\cdots,x_{\pi p})\\
&=\alpha f_{\pi}(T)(x_1,x_2,\cdots,x_p)+\beta f_{\pi}(T)(y_1,x_2,\cdots,x_p)
\end{aligned}
\end{equation}

2.\autoref{SIofTe_eq2} ,\autoref{SIofTe_eq3} 的证明:
\begin{equation}
\begin{aligned}
f_{\pi}(T)_{i_1\cdots i_p}&=f_{\pi}(T)(e_{i_1},\cdots,e_{i_p})=T(e_{i_{\pi1}},\cdots,e_{i_{\pi p}})\\
&=T_{i_{\pi1}\cdots i_{\pi p}}
\end{aligned}
\end{equation}


\textbf{证毕!}