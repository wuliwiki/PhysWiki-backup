% 拉格朗日方程和极值问题
% 拉格朗日方程|最小作用量|泛函分析|速降线

\begin{issues}
\issueDraft
\end{issues}

\pentry{哈密顿原理、最小作用量\upref{HamPrn}}

我们之前通过变分法得出, 当位形空间中轨迹 $\{q_i(t)\}$ 的两端固定而中间变化时, 要使作用量 $S[\{q_i(t)\}]$ 取极值, 我们只需要解拉格朗日方程\autoref{HamPrn_eq2}~\upref{HamPrn}即可.

事实上, 拉氏方程不仅能够解出动力学系统随时间的演化, 还能解决泛函分析中一类更广的极值问题, 例如著名的速降线问题. 因为 $L$ 函数除了拉格朗日量以外, 还可以是其他函数.

(未完成:举一些 Goldstein 中的例题)
