% 格林函数法解非齐次偏微分方程

\begin{issues}
\issueDraft
\end{issues}

\pentry{狄拉克 delta 函数\upref{Delta}}

例子: 一根两端固定的xian
\begin{equation}\label{GreenF_eq1}
-\dv[2]{x} y = f(x)
\end{equation}


格林函数法, 先令格林函数 $G(x_0, x)$ 满足
\begin{equation}
-\dv[2]{x} G(x_0, x) = \delta(x - x_0) \qquad (a < x_0 < b)
\end{equation}
以及边界条件 $G(x_0, a) = G(x_0, b) = 0$. 这相当于弦上只有一点受力.

一个连续的受力分布 $f(x)$ 可以分解为许多不同位置的 $x_0$ 的线性组合(积分)
\begin{equation}
f(x) = \int_a^b f(x_0) \delta(x - x_0) \dd{x_0}
\end{equation}
由于\autoref{GreenF_eq1} 的方程是线性的, 那么把 $G(x_0, x)$ 做同样的线性组合就是满足边界条件的解
\begin{equation}
y(x) = \int_a^b G(x_0, x) \delta(x - x_0) \dd{x_0}
\end{equation}
