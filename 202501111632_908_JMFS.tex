% 镜面反射(综述)
% license CCBYSA3
% type Wiki

本文根据 CC-BY-SA 协议转载翻译自维基百科\href{https://en.wikipedia.org/wiki/Specular_reflection}{相关文章}。

镜面反射或规律反射是波(例如光)从表面反射的类似镜面反射的现象。[1]

反射定律指出,反射光线从反射表面射出时,与表面法线的夹角与入射光线的夹角相同,但位于表面法线的对侧,且在由入射光线和反射光线构成的平面内。这一行为最早由亚历山大的赫罗(公元约10年–70年)描述。[2] 后来,阿尔哈森给出了反射定律的完整表述。[3][4][5] 他首次指出,入射光线、反射光线和表面的法线都位于一个垂直于反射面平面的同一平面内。[6][7]

镜面反射可以与漫反射进行对比,后者是光从表面散射到多个方向的现象。