% 埃尔米特型
% 埃尔米特型|Hermitian|正定埃尔米特型
本节半双线性型定义采用物理上习惯的定义,即\autoref{sequil_def1}~\upref{sequil}.
\pentry{半双线性形式\upref{sequil}}
\begin{definition}{埃尔米特型}
称半双线性型是\textbf{埃尔米特型}(Hermitian),若
\begin{equation}\label{HeFor_eq2}
f(\bvec y,\bvec x)=\overline{f(\bvec x,\bvec y)}
\end{equation}
其中横线表共轭复数.
\end{definition}
\begin{example}{埃尔米特型对应矩阵元的性质}
试证明埃尔米特型 $f$ 对应的矩阵 $F$ 的系数满足 $f_{ij}=\overline f_{ji}$.其中 $f_{ij}=f(\bvec e_i,\bvec e_j)$.这就是说 $F^*=F$,其中 $F^*=\overline {F^T}$.

\textbf{证明:}由埃尔米特型定义知
\begin{equation}
\begin{aligned}
\sum_{i,j}\overline{x_i}y_j f_{ij}&=f(\bvec x,\bvec y)=\overline{f(\bvec y,\bvec x)}\\
&=\overline{\sum_{i,j}\overline{y_j} x_i f_{ji}}=\sum_{i,j}y_j\overline{x_i}\overline{f_{ji}}
\end{aligned}
\end{equation}
对比即得 $f_{ij}=\overline{f_{ji}}$.
\end{example}
按照二次型对应的线性型与对应矩阵的命名的惯例(即名为 $name$ 型的线性型对应的矩阵称 $name$ 矩阵),有下面定义
\begin{definition}{埃尔米特矩阵}
称矩阵 $A$ 为\textbf{埃尔米特矩阵},若 $A^*=A$,其中,$A^*=\overline{A^T}$
\end{definition}
\begin{example}{埃尔米特矩阵在不同基底下仍是埃尔米特的}
也就是说要证明对基底 $\bvec e_i$ 下的埃尔米特型 $f$ 对应的矩阵 $F=F^*$ ,要证在基底 $\bvec e'_i$ 下对应的矩阵 $F'$ 仍满足 $F'=F'^*$.
\textbf{证明:}设 $A$ 是基底 $\bvec e_i$ 到基底 $\bvec e'_i$ 的转换矩阵.由 $f$ 是半双线性型,知(\autoref{sequil_ex1}~\upref{sequil})
\begin{equation}
F'=A^*FA
\end{equation}
 故
\begin{equation}
(F')^*=(A^*FA)^*=A^*F^*A=A^*FA=F'
\end{equation}
\end{example}
埃尔米特型 $f(\bvec x,\bvec y)$ 自然对应\textbf{埃尔米特二次型} $f(\bvec x,\bvec x)$ .因为
\begin{equation}
 f(\bvec x,\bvec x)= \overline{f(\bvec x,\bvec x)} 
\end{equation}
所以埃尔米特二次型只取实数值.

\begin{definition}{正定埃尔米特型}

\end{definition}