% 几何与解析几何初步(高中)
% keys 解析几何|初中|角度|三角|几何|初步|坐标系|点斜式
% license Usr
% type Tutor

\begin{issues}
\issueDraft
\end{issues}

本文主要介绍初中出现过,并在高中阶段有重要作用的几何知识。

\subsection{几何基础}

点:
线:
角:直角、锐角、钝角

角度:角的大小

垂直与平行

几何公理:

\subsection{三角形}

三角形是最简单的由线段围城的平面几何形状。

三角形内角和

直角三角形
等腰三角形
等边三角形

全等
相似

\subsection{三角函数}

正弦、余弦、正切的定义

\begin{table}[ht]
\centering
\caption{常见三角函数值}\label{tab_HsGeBa1}
\begin{tabular}{|c|c|c|c|}
\hline
角$\alpha$ & $30^{\circ}$ & $45^{\circ}$ & $60^{\circ}$ \\
\hline
$\sin\alpha$ & $\displaystyle\frac{1}{2}$ & $\displaystyle\frac{\sqrt{2}}{2}$ & $\displaystyle\frac{\sqrt{3}}{2}$ \\
\hline
$\cos\alpha$ & $\displaystyle\frac{\sqrt{3}}{2}$& $\displaystyle\frac{\sqrt{2}}{2}$ &  $\displaystyle\frac{1}{2}$ \\
\hline
$\tan\alpha$ & $\displaystyle\frac{1}{\sqrt{3}}$ & $1$ & $\sqrt{3}$ \\
\hline
\end{tabular}
\end{table}
\subsection{圆}

\textbf{圆(circle)}是平面内与某个定点距离等于定值的所有点组成的图形。其中,该定点称为\textbf{圆心(center)},通常记作 $O$,并以圆心指代圆,称为圆 $O$。定值称为\textbf{半径(radius)},通常记作 $r$。

对于圆上任意不重合的两点 $A$ 和 $B$,有以下概念:
\begin{itemize}
\item \textbf{弦(chord)}是连接圆上任意两点的线段 $AB$。圆心到弦的垂线是该弦的垂直平分线。
\item \textbf{弧(arc)}是圆上两点之间的弧线。两点 $A$ 和 $B$ 将圆分成两条弧,分别称为\textbf{劣弧(minor arc)}和\textbf{优弧(major arc)},其中劣弧的长度不超过半个圆周,优弧的长度大于半个圆周。
\item 顶点位于圆心,且两边与圆相交的角称为\textbf{圆心角(central angle)}。同样,一般包括一个小于$180^\circ$的角和一个大于$180^\circ$的角,二者分别与他们角内夹的弧对应\footnote{有的人分别称这两种为劣角和优角,以求和弧对应,但使用不广泛。劣角在英语中无对应词,优角对应的词为“Reflex Angle”,直译是反射角,对应平面反射的情况。}。
\end{itemize}

在未特别说明的情况下,通常 $\overset{\frown}{AB}$ 表示两点 $A$ 和 $B$ 之间的劣弧。此时,两点 $A$ 和 $B$ 可以唯一确定一条弦 $AB$,一条$\overset{\frown}{AB}$ 和一个圆心角$\angle AOB$。因此可以称他们三者对应,如:$\overset{\frown}{AB}$ 称为弦 $AB$ 所对应的弧,$\angle AOB$是$AB$所对应的圆心角等\footnote{如果有特殊情况,则不一一对应,一条弦对应两段弧,但每个弧仍对应一个圆心角。}。

对于圆上任意不重合的三点 $A,B,C$,有以下概念:
\begin{itemize}
\item 顶点位于圆上,两边与圆相交的角称作\textbf{圆周角(inscribed angle)}。设顶点为$C$,则圆周角$\angle ACB$所夹圆弧$\overset{\frown}{AB}$称作其对应的弧,同弧对应的圆心角是圆周角的两倍,即$\angle AOB=2\angle ACB$。
\item 以三点 $A$、$B$、$C$ 为顶点的三角形称为圆的\textbf{内接三角形(inscribed triangle)},即三角形的三个顶点都在圆上。若三角形的三边分别与圆相切,则称为圆的\textbf{外切三角形(circumscribed triangle)}。
\end{itemize}

通过圆心的弦称为圆的\textbf{直径(diameter)},通常记作$d$,满足:
\begin{equation}
d = 2r~.
\end{equation}
此时,所对弧称为\textbf{半圆(semicircle)},所对圆心角为平角,即$180^\circ$,所对圆周角为直角,即$90^\circ$。因此,由直径参与构成的圆内接三角形一定是直角三角形。


\textbf{圆周率(pi)} $\pi$ 是圆的周长与直径的比值,定义为 $\pi = \frac{C}{d}$。它是一个无理数,近似值为 $\pi \approx 3.14159$。$\pi$ 连接了圆的周长、面积等几何性质,是数学中极为重要的常数。

圆的周长公式为 $C = 2\pi r$,面积公式为 $A = \pi r^2$,其中 $r$ 为半径,$\pi$ 为\textbf{圆周率(pi)}。

\subsection{解析几何基础}

\subsubsection{坐标系}
将所有的实数和直线上的点一一对应,就形成了\textbf{数轴(number line)}。。数轴的定义基于一个确定的原点、单位长度和正方向,这三个因素唯一地确定了数轴在几何中的位置和方向。法国数学家勒内·笛卡尔(René Descartes)在数学研究中,将两条数轴的原点重叠,并将其正交(即相互垂直)放置,创造了\textbf{坐标系(coordinate system)}。这就是初中阶段学习过的\textbf{笛卡尔坐标系(Cartesian coordinate system)},也称为\textbf{直角坐标系(rectangular coordinate system)}。

引入坐标系后,平面上的任何一点都可以通过一个\enref{有序数对}{CartPr} $(x, y)$ 来表示。借助这种表示法,几何形状可以通过数对来分析和研究,这一方式称为\enref{解析几何}{JXJH}。而当数对中的值对应于函数的变量及其结果时,几何图形就成为了函数的图像。因此,坐标系不仅为函数的图像提供了清晰的视觉表达,还使得人们可以通过几何图形直观地观察函数的性质,例如其变化趋势、最大值和最小值等。

通常,直角坐标系中,两条数轴称为$x$轴和$y$轴,且向右的方向为$x$轴的正方向,向上为$y$轴的正方向。数轴将平面分为四个区域,称为\textbf{象限(quadrant)}。其中,第一象限是两个坐标都为正的区域,之后按逆时针方向依次为第二、第三和第四象限。

\subsubsection{常见表达式}

\begin{definition}{直线的点斜式}\label{def_HsGeBa_1}
经过点$(x_0,y_0)$,且斜率为$k$点直线表达式为:
\begin{equation}
y-y_0=k(x-x_0)~.
\end{equation}
\end{definition}

