% 积性函数
% keys 数论函数|积性函数

\pentry{数论函数\upref{NumFun}}

\begin{issues}
\issueAbstract
\end{issues}

\begin{definition}{积性函数}
如果数论函数$f(n)$使
\begin{equation}
f(ab)=f(a)f(b),(a,b)=1~.
\end{equation}
恒成立,则称$f(n)$是\textbf{积性的}。
\end{definition}

\begin{example}{}
$I(n),u(n),e(n),d(n),\sigma(n),\mu(n),\varphi(n),\lambda(n)$都是积性的。
\end{example}

\begin{definition}{完全积性函数}
如果数论函数$f(n)$使
\begin{equation}
f(ab)=f(a)f(b)~.
\end{equation}
恒成立,则称$f(n)$是完全积性的。
\end{definition}

\begin{example}{}
$I(n),u(n),e(n),\lambda(n)$都是完全积性的。
\end{example}

\begin{theorem}{积性函数的性质}
$f(n)$是积性的,则:
\begin{enumerate}
\item $f(1)=1$.
\item $f((a,b))f([a,b])=f(a)f(b)$.
\item 函数$F(n)=\sum\limits_{d|n}f(d)$也是积性的。
\end{enumerate}
\end{theorem}

证明留给读者。