% 电磁场和引力场的作用量
% keys 电磁场|引力场|作用量
% license Usr
% type Tutor

\pentry{电磁力和引力\nref{nod_EleGra}}{nod_9484}

由“电磁力和引力\upref{EleGra}”一节,我们得到了处于外场中的粒子的作用量的两种情况:
\begin{equation}
\begin{aligned}
S_E=&\int \left\{-m\sqrt{-\eta_{\mu\nu} \,\mathrm{d}{x} ^\mu \,\mathrm{d}{x} ^\nu}+A_\mu(x) \,\mathrm{d}{x} ^\mu \right\} ,\\
S_G=&-m\int\sqrt{-g_{\mu\nu}(x) \,\mathrm{d}{x} ^\mu \,\mathrm{d}{x} ^\nu}.
\end{aligned}~
\end{equation}
 $S_E,S_G$ 分别对应于磁场和引力场中的粒子的作用量。

自然会出现这样的问题:场 $A_\mu$ 和 $g_{\mu\nu}$ 是怎样产生的?

物理学应当是粒子和场之间的共舞。场告诉粒子如何运动,粒子反过来产生场。我们已经在“电磁力和引力\upref{EleGra}”一节中描述了粒子在场中的运动,本节将寻找主宰场 $A_\mu(x),g_{\mu\nu}(x)$ 动力学的作用量。

由\autoref{def_CoIn_1},物理应当是关于某一变换不变的。因此找到相关的不变性是首要的。Lorentz不变性必须是满足的(这仅仅是说我们有选择经过(四维)旋转变换下的坐标描述物理的自由)。


\subsection{电磁场的作用量}
\subsubsection{规范不变性}
注意 $S_E$ 中关于 $A_\mu$ 的项
\begin{equation}
\int A_\mu(x) \dd x^\mu.~ 
\end{equation}
若进行下列变换
\begin{equation}\label{eq_EGA_1}
A_\mu(x)\mapsto A_\mu(x)+\partial_\mu\Lambda(x)~,
\end{equation}
则 $S_E$ 改变量为
\begin{equation}
\begin{aligned}
\Delta S_E=&\int\partial_\mu\Lambda(x)\dd x^\mu=\int_{\tau_i}^{\tau_f}\dd \tau\partial_\mu\Lambda(x)\dv{x^\mu}{\tau}\\
=&\int_{\tau_i}^{\tau_f}\dv{\Lambda(x)}{\tau}\\
=&\Lambda(x(\tau_f))-\Lambda(x(\tau_i)).
\end{aligned}~
\end{equation}
取粒子轨迹的端点在远远的过去和未来,并假设 $\Lambda(x)$ 在无穷远处为0(注意 $x$ 无穷远意味着 $t=\int_0^t\dd\tau=\int\sqrt{-\eta_{\mu\nu}\dd x^\mu \dd x^\nu}\rightarrow\infty$,因此 $\tau_f,\tau_i\rightarrow\infty$ 位于无穷远处 ),则 $\Delta S_E=0$。即作用量在变换\autoref{eq_EGA_1} 下不变,\autoref{eq_EGA_1} 称为场 $A_\mu(x)$ 的\textbf{规范变换}。因此,我们发现了作用量在规范变换下具有对称性,称为\textbf{规范不变}。

严格来说,这里讨论的规范对称性并不是一个对称性,而是描述的赘余。$A_\mu(x)$ 和 $A_\mu(x)+\partial_\mu\Lambda(x)$ 描述的是相同的物理。换句话说 $A_\mu(x)$ 包含有非物理的自由的,这可以通过适当的选取 $\Lambda(x)$ 删除。

\subsubsection{Maxwell作用量出现}

既然关于 $A_\mu(x)$ 的作用量是规范不变的,那么描述 $A_\mu$ 的作用量就应当由规范不变的量来构造。显然 $A_\mu$ 本身不是规范不变的,但是场强 $F_{\mu\nu}=\partial_\mu A_\nu-\partial_\nu A_\mu$ 是规范不变的:
\begin{equation}
F_{\mu\nu}\mapsto \partial_\mu[ A_\nu+\partial_\nu\Lambda(x)]-\partial_\nu[ A_\mu+\partial_\nu\Lambda(x)]=F_{\mu\nu}.~
\end{equation}
因此,我们可以通过 $F_{\mu\nu}$ 来构造电动力学的作用量。为确保Lorentz不变性,最简单的可能就是 $F^2=F_{\mu\nu}F^{\mu\nu}$。注意到 $F^2$ 包含有两个时间偏微分算子,

在主导粒子作用量中,动力学变量是粒子的位置 $\vec x(t)$,或更好的记法 $\vec X(t)$,在多粒子情形,则用 $\vec X_a(t)$ 标记,其中 $a$ 是识别不同粒子的标记。因此,主导场 $A_\mu$ 动力学的作用量中,动力学变量是场 $A_\mu(x)=A_\mu(\vec x,t)$ 本身,而 $\vec x$ 仅仅是识别场的不同点的记号。正如在多粒子情形作用量是对 $a$求和,场情形则应是对 $\vec x$ 求和。离散的求和号 $\sum$ 对应连续情形的积分号 $\int\dd{}^3 x$。因此,场 $A_\mu(x)$ 的作用量可写为(为了场运动方程简便起见,往往人们引入 $-\frac{1}{4}$ 的因子)
\begin{equation}
-\frac{1}{4}\int\dd{}^4 x F_{\mu\nu}F^{\mu\nu}.~
\end{equation}

\subsubsection{粒子和电磁场的总作用量}

把电磁场的作用量加入到主导多粒子在场 $A_\mu$ 中的作用量中,便得到粒子和场 $A_\mu$ 的总作用量
\begin{equation}
 S=\left\{\int\sum_a\qty(-m_a\sqrt{-\eta_{\mu\nu} \,\mathrm{d} X_a ^\mu \,\mathrm{d}X_a ^\nu}+e_aA_\mu(X_a) \,\mathrm{d}X_a ^\mu) \right\}-\frac{1}{4}\int\dd{}^4 x F_{\mu\nu}F^{\mu\nu}.~
\end{equation}
其中,第一项代表自由粒子的作用量,第三项代表场 $A_\mu(x)$ 的作用量,而第二项代表场和粒子耦合的作用量。









\subsection{引力场的作用量}

\subsubsection{微分同胚不变性}
注意粒子和引力场耦合的作用量为
\begin{equation}
-m\int\sqrt{-g_{\mu\nu}(x) \,\mathrm{d}{x} ^\mu \,\mathrm{d}{x} ^\nu}~.
\end{equation}
其中 $x$ 的是粒子在时空中的位置。注意到在微分同胚(任意导数存在连续)的坐标变换 $x^\mu(x) \mapsto x^\mu(x')$ 下,张量的缩并不变,因此
\begin{equation}
g_{\mu\nu}(x) \dd x^\mu \dd{x} ^\nu=g_{\rho\sigma}(x') \dd{x'} ^\rho \dd{x'} ^{\sigma}~.
\end{equation}
因此,描述引力场的 $g_{\mu\nu}(x)$ 的作用量应当在坐标的微分同胚变换下不变。这一不变性称为\textbf{微分同胚不变性}。

因此,和电磁场一样,现在我们需要寻求由 $g_{\mu\nu}(x)$ 构造的具有微分同胚不变性的量作为被积函数加入到积分号 $\int\sqrt{-g(x)}\dd{}^4x$ 中($\sqrt{-g(x)}\dd{}^4$ 而不是 $\dd{}^4x$ 的原因在于(直角坐标下)欧氏空间的体积元是在一般空间下由 $\sqrt{-g(x)}\dd{}^4 x$ 取代),以得到引力场的作用量。在微分同胚坐标变换下不变的量是一个标量,因此引力场的作用量具有如下形式
\begin{equation}
S=\int\dd{}^4x\sqrt{-g(x)}A(x)~.
\end{equation}
其中 $A$ 是一个标量,换句话说 $A'(x')=A(x)$。



























