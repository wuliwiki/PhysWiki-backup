% 一阶线性偏微分方程与常微分方程组的等价性
% 线性偏微分方程
\pentry{一般积分\upref{IntGen}}
\footnote{参考斯米尔诺夫《高等数学》第 2 卷第 1 分册}本词条将说明,一阶线性偏微分方程与常微分方程组具有直接的联系:一阶线性偏微分方程求解问题可以化为常微分方程组的求解问题。这也是对应常微分方程组被称为一阶线性偏微分方程的\textbf{特征方程组}的原因。
\subsection{将常微分方程组写成更为对称的形式}
一般的常微分方程组都可写为下面的形式(\autoref{the_GO2SOD_2}~\upref{GO2SOD})
\begin{equation}\label{eq_LPaODE_4}
\dv{y_i}{x}=f_i(x,y_1,\cdots,y_n),\quad i=1,\cdots,n
\end{equation}
这可写为下面等价的形式
\begin{equation}
\dd x=\frac{\dd y_1}{f_1(x,y_1,\cdots,y_n)}=\cdots=\frac{\dd y_n}{f_n(x,y_1,\cdots,y_n)}~.
\end{equation}
为使第一项的分母不为1,可把该式所有的分母都乘上共同的因子。并且为对称起见,将 $x$ 记为 $x_1$,$y_i$ 记为 $x_{i+1}$,上式可写为更具对称性的等价形式:
\begin{equation}\label{eq_LPaODE_1}
\frac{\dd x_1}{X_1}=\cdots=\frac{\dd x_{n+1}}{X_{n+1}}~.
\end{equation}
其中,$X_i$ 是变量 $x_1,\cdots,x_{n+1}$ 的函数。

当常微分方程组写成\autoref{eq_LPaODE_1} 时,可看出这 $n+1$ 个变量是等价的,并没有特定哪个变量作为自变量。在新的记号下,方程组的积分是(\autoref{def_IntGen_1}~\upref{IntGen})
\begin{equation}
\varphi_i(x_1,\cdots,x_{n+1})=C_i,\quad i=1,\cdots,n~.
\end{equation}
\subsection{从常微分方程组到线性偏微分方程}
在词条“一般积分\upref{IntGen}”的末尾,已经知道,把方程组的解代入其积分里面,积分便是常数。设 $\varphi(x_1,\cdots,x_{n+1})$ 是方程组的积分,不失一般性设 $x_1$ 是自变量,而 $x_2,\cdots,x_{n+1}$ 是 $x_1$ 的函数,它们是方程组的解。于是 $\varphi(x_1,\cdots,x_{n+1})=C$ 。这就是说,代入 $x_2,\cdots,x_{n+1}$ 后,应当消去自变量 $x_1$\footnote{这是因为代入后函数值是常数,意味着代入后的函数不显含 $x_1$}。于是 $\varphi$ 对 $x_1$ 进行全微商应等于0:
\begin{equation}
\pdv{\varphi}{x_1}+\pdv{\varphi}{x_2}\dv{x_2}{x_1}+\cdots+\pdv{\varphi}{x_{n+1}}\dv{x_{n+1}}{x_1}=0~,
\end{equation}
或写成
\begin{equation}\label{eq_LPaODE_2}
\pdv{\varphi}{x_1}\dd x_1+\pdv{\varphi}{x_2}\dd{x_2}+\cdots+\pdv{\varphi}{x_{n+1}}\dd{x_{n+1}}=0~.
\end{equation}

由\autoref{eq_LPaODE_1} ,无论代入方程组哪个解,$\dd x_i$ 都与 $X_i$ 的大小成正比,于是\autoref{eq_LPaODE_2} 可写成下列一阶线性偏微分方程:
\begin{equation}\label{eq_LPaODE_3}
X_1\pdv{\varphi}{x_1}+X_2\pdv{\varphi}{x_2}+\cdots+X_{n+1}\pdv{\varphi}{x_{n+1}}=0~.
\end{equation}
根据常微分方程组初始条件的任意性,若我们取方程组的所有解,变量 $x_1,\cdots,x_{n+1}$ 就可能取任意的值,而对方程组任意解,$\varphi$ 都满足\autoref{eq_LPaODE_3} ,所以函数 $\varphi(x_1,\cdots,x_{n+1})$ 应当恒满足\autoref{eq_LPaODE_3} 。于是证得下面定理一部分
\begin{theorem}{}\label{the_LPaODE_1}
 $\varphi(x_1,\cdots,x_{n+1})=C$ 是方程组\autoref{eq_LPaODE_1} 的积分,当且仅当函数 $\varphi(x_1,\cdots,x_{n+1})$ 满足偏微分方程\autoref{eq_LPaODE_3} 。
\end{theorem}
 \textbf{证明:}现在来证明定理的另一部分。设 $\varphi(x_1,\cdots,x_{n+1})$ 是偏微分方程\autoref{eq_LPaODE_3} 的解,现在代入\autoref{eq_LPaODE_4} 的解于 $\varphi$,于是
 \begin{equation}
 \dd\varphi=\pdv{\varphi}{x_1}\dd x_1+\pdv{\varphi}{x_2}\dd{x_2}+\cdots+\pdv{\varphi}{x_{n+1}}\dd{x_{n+1}}~,
 \end{equation}
 等价于
 \begin{equation}
 \dd\varphi=\lambda\qty(X_1\pdv{\varphi}{x_1}+X_2\pdv{\varphi}{x_2}+\cdots+X_{n+1}\pdv{\varphi}{x_{n+1}})~.
 \end{equation}
 其中已设\autoref{eq_LPaODE_1} 的比例系数为 $\lambda$。于是由\autoref{eq_LPaODE_3} 
 \begin{equation}\label{eq_LPaODE_5}
 \dd\varphi=0
 \end{equation}
 而一阶微分具有形式不变性(链接),即将函数的变量当成自变量或自变量的函数,函数的一阶微分形式都是一样的。在现在的情形,$\varphi$ 就是一个自变量的函数,不失一般性设为 $x_1$,于是 \autoref{eq_LPaODE_5} 相当于 $\varphi$ 对 $x_1$ 的微商为0,即 $\varphi$ 不依赖于 $x_1$ 而等于常数,根据微分方程组积分的定义(“一般积分\upref{IntGen}”文末所说的),$\varphi(x_1,\cdots,x_{n+1})$ 是方程组\autoref{eq_LPaODE_1} 的积分。

 \textbf{证毕!}

 \autoref{the_LPaODE_1} 建立了微分方程组\autoref{eq_LPaODE_1} 的积分与一阶线性偏微分方程\autoref{eq_LPaODE_3} 的解的\textbf{等价性}。于是(由“一般积分\upref{IntGen}”文末所说的),下面推论成立
 \begin{corollary}{}\label{cor_LPaODE_1}
 若
 \begin{equation}
 \varphi_i(x_1,\cdots,x_{n+1})=C_i,\quad i=1,\cdots,n
 \end{equation}
 是方程组\autoref{eq_LPaODE_1} 的 $n$ 个无关的积分,则任意函数 $F(\varphi_1,\cdots,\varphi_n)$ 是偏微分方程\autoref{eq_LPaODE_3} 的解。
 \end{corollary}
\subsubsection{一般的一阶线性偏微分方程}
\autoref{eq_LPaODE_3} 是一阶的线性齐次偏微分方程,其自由项为0,对一般的一阶线性偏微分方程,其形式为
\begin{equation}\label{eq_LPaODE_6}
Y_1\pdv{\varphi}{x_1}+Y_2\pdv{\varphi}{x_2}+\cdots+Y_{n+1}=0~.
\end{equation}
其中 $Y_i$ 是 $x_1,\cdots,x_n,\varphi$ 的函数。

设 $\varphi$ 是\autoref{eq_LPaODE_6} 的解,我们考虑下面关于解 $\varphi$的方程构成的族
\begin{equation}\label{eq_LPaODE_7}
\omega(x_1,\cdots,x_n,\varphi)=C~.
\end{equation}
其中 $C$ 是任意的常数,任意的常数 $C$ 使 $x_1,\cdots,x_n,\varphi$ 能够取任意的值。依照隐函数求解法则(链接)
\begin{equation}
\pdv{\varphi}{x_i}=-\frac{\pdv{\omega}{x_i}}{\pdv{\omega}{\varphi}}~,
\end{equation}
上式代入\autoref{eq_LPaODE_6} ,得
\begin{equation}
Y_1\pdv{\omega}{x_1}+Y_2\pdv{\omega}{x_2}+\cdots+Y_{n+1}\pdv{\omega}{\varphi}=0~.
\end{equation}
由于 $x_1,cdots,x_n,\varphi$ 可取任意值,于是 $\omega$ 恒满足上式,由\autoref{the_LPaODE_1} ,解这个偏微分方程可以化为求它对应的常微分方程组。若求得 $\omega$,带入\autoref{eq_LPaODE_7} ,就确定了 $\varphi$。

\textbf{注意:}由\autoref{cor_LPaODE_1} ,偏微分方程的一般解含有任意函数,而常微分方程的解只有任意常数出现。
