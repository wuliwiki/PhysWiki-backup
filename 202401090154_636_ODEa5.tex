% 一阶隐式常微分方程的存在唯一性定理
% keys 隐式方程|ODE|differential euqation|存在唯一
% license Usr
% type Wiki

\begin{issues}
\issueDraft
\end{issues}

\pentry{隐函数定理\upref{impli},皮卡定理\upref{PiLin}}

\subsection{一阶隐式常微分方程的存在唯一性定理}

对于一阶隐式常微分方程 $F(x, y, y')=0$,函数 $F$ 满足:
\begin{enumerate}
\item 在 $(x_0, y_0, y_0')$ 的某个邻域内连续,且关于 $y$、$y'$ 有连续的一阶偏导数;
\item $F(x_0, y_0, y_0')=0$;
\item $F_{y'}'(x_0, y_0, y_0')\neq 0$。
\end{enumerate}
那么,$F(x, y, y') = 0$ 存在唯一的满足 $y(x_0) = y_0, y'(x_0) = y_0' $的、定义在 $[x_0-h, x_0+h]$ 上的函数,其中 $h$ 为一个充分小的正数。

\subsection{证明}

由隐函数定理,这方程为一确定了一个定义在点 $(x_0, y_0)$ 的某邻域 $S$ 上的隐函数 $y'=f(x, y)$,满足
$$F(x, y, f(x,y)) \equiv 0, y_0'=f(x_0, y_0) ~,$$
同时,$f(x, y)$ 在 $S$ 内连续,$f'_y$ 在 $S$ 内连续。其中
$$f'_y(x, y) = - \frac{F'_y(x, y, y')}{F'_{y'}(x, y, y')} ~.$$

接下来引入一个皮卡定理的推论:
\begin{corollary}{皮卡定理推论}
根据微分中值定理,容易证明在某区域内连续的函数,在区域内关于 $y$ 满足李氏局部条件,
\end{corollary}