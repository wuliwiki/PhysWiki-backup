% 【导航】线性代数
% keys 线性代数|几何向量|线性组合|数乘|线性相关|基底|矩阵
% license Xiao
% type Map

\begin{issues}
\issueDraft
\end{issues}
% Giacomo:有没有办法查一下这篇文章的原作者是谁?我很久之前复制的了。我猜是jier

% \pentry{}

\subsection{关于此部分}

课程“线性代数”(Linear Algebra)一般指数学系和非数学都需要学习的必修课,开设于低年级(大一/大二)。小时百科的线性代数部分基本上涵盖于这门课程,但不会引入向量空间等需要用到集合论的现代数学概念。关于现代数学意义下的线性代数,请参考【导航】高级线性代数\upref{mapALA}。

\subsection{几何向量}

线性代数的研究对象是向量和矩阵,而我们最早认识的向量就是\textbf{几何向量},这里我们回顾几何向量的相关概念。

几何向量的存在与坐标系无关,它是一些有长度有方向的箭头。我们把(二维)平面中的向量称为平面向量,(三维)空间中的向量称为空间向量;在高中数学的语境下,我们把(一维)直线上的向量称为标量,但这是不严谨的。

% 对于讨论问题的不同,我们有时仅需要处于同一平面(\textbf{二维空间})的所有几何向量,有时需要\textbf{三维空间}中的所有几何向量,最简单的情况下只需要沿某条线(\textbf{一维空间})的所有几何向量(这时我们可以规定一个正方向,且仅使用几何向量的模长加正负号来表示几何向量以简化书写)。

\addTODO{高中数学中平面向量和空间向量的链接}
% Giacomo:是不是应该把高中数学/物理,改成中学数学/物理?

几何向量有起点(箭尾)和终点(箭头),但我们对几何向量的绝对位置不感兴趣,我们只在乎起点和终点的相对位置,即两个几何向量如果有相同的方向和长度就被视为同一个向量。

几何向量的一些基本运算\upref{GVec} 同样不需要有任何坐标系的概念,\textbf{几何向量相加}按照三角形法则或平行四边形法则即可。
\textbf{几何向量数乘}就是把几何向量的模长乘以一个实数,若乘以正数,方向不变,若乘以负数,取相反方向。 \textbf{几何向量的线性组合}是把若干几何向量分别乘以一个实数再相加得到新的几何向量。

几何向量的\textbf{内积}\upref{Dot}等于一个几何向量在另一个几何向量上的投影长度乘以另一个几何向量的模长得到一个实数,几何向量的\textbf{模长}等于几何向量与自身内积再开方,把几何向量除以自身模长使模长变为单位长度的过程叫做\textbf{归一化}。若两几何向量内积为零,这两个几何向量相互\textbf{正交}\footnote{对于几何向量,正交就是方向垂直,不加区分。}。

三维欧几里得空间中,两几何向量\textbf{叉乘}\upref{Cross}得到的几何向量垂直于两几何向量,模长为一个几何向量在另一个几何向量垂直方向的投影长度乘以另一个几何向量的模长。

为了方便描述几何向量之间的关系,我们选取一些\textbf{线性无关}的几何向量作为所有几何向量的\textbf{基底},使空间中的任何几何向量可以用这些基底的唯一一种线性组合来表示,$N$ 维空间需要 $N$ 个基底向量。一般来说,基底不必互相正交。我们先把这些基底排序,任意几何向量表示成它们的线性组合时,把式中的 $N$ 个系数按照顺序排列,就是该几何向量的\textbf{坐标},通常用列几何向量表示。由于线性组合的唯一性,每个几何向量的坐标是唯一的。

为了方便计算任意几何向量的坐标,往往取\textbf{正交归一}的基底\upref{OrNrB}(所有基底模长为1,任意两基底互相正交)。这样,任意向量的坐标都可以通过与基底的内积得到。

\addTODO{添加相关词条的链接}

\subsection{列向量}

几何向量的坐标让我们可以以一个全新的视角看待向量这个概念,我们可以把数组 $(a_1, \cdots, a_n)$ 称为一个向量$\bvec{a}$;由于我们常常会把它竖着记为
\begin{equation}
\bmat{a_1 \\ \vdots \\ a_n}~,
\end{equation}
这种向量被称为\textbf{列向量},$n$ 被称为 $a$ 的\textbf{维度},$a_i$ 被称为 $a$ 的第 $i$ 坐标\footnote{数学中没有规定一定要从 $1$ 开始计数,也可以从 $0$ 开始。}。对于列向量来说,存在一组特别的基底 $\{\bvec{e}_i\}_{i = 1}^n$,称为\textbf{标准基底},其中 $\bvec{e}_i$ 是第 $i$ 坐标为 $1$,其他坐标为 $0$ 的列向量,因此任何一个列向量都可以写成
\begin{equation}
\bvec{a} = a_1 \bvec{e}_1 + \cdots + a_n \bvec{e}_n~
\end{equation}
的形式。

第 $i$ 坐标 $a_i$ 的取值可以和几何向量一样取 $\mathbb{R}$,也可以取一些其他的数集,比如复数集 $\mathbb{C}$。

\addTODO{数集的定义和链接}

实数取值的 $2$ 维(或者 $3$ 维)列向量,等价于选取了坐标系几何向量;由标准基底的存在,列向量是\textbf{坐标相关的},它并不是几何向量的推广。几何向量和列向量都是更一般的向量的特殊情况。


% Giacomo:线性变换移动到高级线性代数
% \subsection{线性变换}

% 我们可以设计一种规则把某个空间的任意向量对应(\textbf{映射})到另一个向量,叫做\textbf{变换}\footnote{更广义地,变换可以在不同的空间中进行,例如把一个三维空间中的向量映射到一个二维空间中的向量}。如果对于某个变换,任意向量线性组合的变换等于这些向量分别进行该变换再线性组合,这个变换就是\textbf{线性变换}\upref{LTrans}。

\subsection{行向量与矩阵}

\addTODO{行向量}

\textbf{矩阵}\upref{Mat},是把标量排成矩形所得到的数学对象。在选定某组基底之后,向量的线性变换可以用其坐标的线性变换表示,并且可以写成矩阵与坐标列向量相乘的形式。

% 词条中要讲逆变换和逆矩阵,这里就算了。

\textbf{旋转矩阵}\upref{Rot2D}可以有两种理解,一是向量绕某个轴相对于当前的正交归一基底转动,其坐标产生了变换,二是向量本身没有变,只是其坐标在两个不同的正交归一基底中不同。这种矩阵的特点是所有列(行)向量都正交归一,所以叫做\textbf{单位正交阵}。单位正交阵的特点是逆矩阵等于转置矩阵。

% Giacomo: 应当移动到其他地方,比如多元微积分导航
% \subsection{向量微积分}
% $N$ 维向量可以作为一个或多个标量的函数(\textbf{向量函数}),可以看成是 $N$ 个普通函数与向量基底的数乘。向量函数同样可以对其自变量求导(或求偏导),也可以积分。不同的是,向量函数还可以进行曲线积分和面积分 %未完成

%介绍几何向量其实是一种抽象的存在,并不需要坐标的辅助就可以定义相加,数乘,内积,叉乘等运算

%然后引入基底的概念,尤其是正交基的概念。然后便是基底转换了。


% 经研究,力学分册绝对不需要用向量空间的概念! 基底,线性组合,线性变换,坐标,等等所有概念都可以通过几何向量讲解! 唯一可能用到其他向量空间的可能性就是傅里叶级数而已!而且傅里叶级数在力学分册中也不会用到啊。
% 正交变换的最高目标估计就是讲明吧旋转矩阵了。
% 讲讲线性方程组的解还是有必要的。
% 要添加矩阵求导法则。

\subsection{线性方程组}

(抽象的)向量和矩阵可以方便地被用来解\textbf{线性方程组}\upref{LinEqu}。
