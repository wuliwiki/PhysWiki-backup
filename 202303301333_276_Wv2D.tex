% 二维波动方程
% keys 波动方程|张力|面密度|波速|薄膜|散度|拉普拉斯方程

\begin{issues}
\issueDraft
\end{issues}

\pentry{一维波动方程\upref{WEq1D}, 梯度\upref{Grad}, 散度\upref{Divgnc}}

假设我们有一个紧绷的橡皮薄膜, 单位长度的张力为 $\lambda$。 面密度为 $\sigma$, 均为常数。 取一小块微元进行受力分析, 其质量为 $m = \sigma S$。 要分析其受力, 可以将面元的边界划分成许多小段, 每小段给面元施加的竖直方向的力等于 $\lambda$ 乘以法相斜率 $\grad u$ 再乘以该小段长度 $\dd{l}$。 注意只有再假设其振动幅度很小时这才成立, 因为斜率是 $\tan\theta$, 而严格来说我们需要 $\sin\theta$, 类比同一维波动方程的推导\upref{WEq1D}。
\begin{equation}
\pdv[2]{u}{t} = \frac{\lambda}{\sigma S}\oint [\grad u(\bvec r)] \vdot \uvec n(\bvec r) \dd{l}~,
\end{equation}
$\uvec n(\bvec r)$ 为边界在 $\bvec r$ 处向外的法向量。

当 $S \to 0$, 根据散度的定义
\begin{equation}
\frac{1}{S} \oint [\grad u(\bvec r)] \vdot \uvec n(\bvec r) \dd{l} \to \div (\grad u) = \laplacian u~,
\end{equation}
所以波动方程为
\begin{equation}
\laplacian u - \frac{\sigma}{\lambda}\pdv[2]{u}{t} = 0~,
\end{equation}
其中 $v_0 = \sqrt{\lambda/\sigma}$。

特殊地, 当薄膜保持静止不动, 我们就得到了拉普拉斯方程\upref{Laplac}。
