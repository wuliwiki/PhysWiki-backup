% HTML 笔记
% keys html
% license Usr
% type Note

\begin{issues}
\issueDraft
\end{issues}

% html
\begin{lstlisting}[language=none]
<!DOCTYPE html>
<html>
<head>
    <meta charset="UTF-8">
</head>
<body>
    <!-- 内容 -->
</body>
</html>
\end{lstlisting}

\textbf{常识:}
\begin{itemize}
\item 文件头 \verb`<!DOCTYPE html>` 意味着必须使用 html 5。 之前的标准的头更复杂。
\item 所有的 \verb`<script src=""></script>`,无论是外部 js 文件还是直接嵌入代码,无论在 \verb`<head>` 还是 \verb`<body>` 中都是按出现的顺序执行的。 如果 \verb`<head>` 中的代码还没有执行完,就不会渲染 html 页面。
\item attributes 用双引号(最常见),单引号,无引号(不标准)都行,例如 \verb`<a href="https://example.com" title="Visit Example Website">链接文字</a>`。注意用空格隔开。
可以用 base64 来表示图片,代替单独的图片文件 \verb`<img src="data:image/jpeg;base64,一个base64字符串" alt="...">
\end{lstlisting}`。其中 \verb|一个base64字符串| 是直接把整个文件转换成 base64, 在 linux 命令行中可以用 \verb|base64 文件| 做到。该命令默认给每 64 个字符换行,最后一行即使每填满也换行。 若不需要换行用 \verb|base64 -w 0 文件|
\item 任何 tag 都可以加 \verb`id=".."`, 原则上 id 是整个 html 中唯一的。
\item 任何 tag 都可以加 \verb`class=".."`,不要求唯一。 css 可以指定一个 class 或 id 的格式。
\item 这些属性叫做 global attributes。 其他例子如 \verb`style`, \verb`title`(鼠标悬停提示), \verb`hidden`(隐藏内容)
\item 每个 html 都建议在 \verb`<head>` 中放一个图标 \verb`<link rel="icon" href="path/to/your/favicon.ico" type="image/x-icon">`,否则浏览器调试界面会报错找不到图标。
\item \verb`rel` 是 relationship 的缩写,常见的值还有 \verb`stylesheet`
\item 如果目录 \verb`example.com/folder/` 中有 \verb`index.html`, 那么直接输入前者就可以自动加载后者。 如果 html 中需要同目录的文件,只需要 \verb`href="文件名"` 即可加载 \verb`example.com/folder/文件名`, 但如果 url 没有最后的 \verb`/` 就会出错。 为此, 可以在 \verb`<head>` 的一开始用 js 检查并添加最后的 \verb`/`:
\begin{lstlisting}[language=js]
<!-- add a '/' to the end of url if not exist -->
<script>
if (window.location.href.slice(-1) != '/')
    window.location.href = window.location.href + '/';
</script>
\end{lstlisting}
注意改变 \verb`window.location.href` 相当于立即进行一次页面跳转。 另外注意 \verb`window.location.href` 包含了整个 url,包括后面可能的参数, js 就是通过这个属性来 parse 参数的(参考 \verb`URL` 和 \verb`URLSearchParams` 类)。
\end{itemize}

\subsubsection{转义字符}
\begin{itemize}
\item \verb`&nbsp;` 不换行的空格
\item \verb`&amp;` (&).
\item \verb`&lt;` (<).
\item \verb`&gt;` (>).
\item \verb`&quot;` (").
\item \verb`&apos;` (').
\end{itemize}
