% 舒尔引理(Schur's lemma)
% 舒尔引理|Schur|舒尔|Schur's lemma


本章节分为两个部分,前半部分从线性空间的角度进行严格定义并进行证明,后半部分通过矩阵的语言表述并进行证明。

此部分内容参考\cite{维声表示}中定义。

\begin{lemma}{舒尔引理}
设$(\phi,V)$和$(\psi,W)$是群$G$的两个不可约$K-$表示,其中$K$是代数闭域。设$\sigma$是域$K$上有限维线性空间$V$到$W$的一个线性变换使得
\begin{equation}\label{Schlem_eq1}
\psi(g)\sigma =\sigma \phi(g),~\forall g \in G,
\end{equation}
(i)如果$\phi$和$\psi$不等价,则$\sigma=0$; \\
(ii)如果$V=W$并且有$\phi=\psi$,则$\sigma=\lambda1_V$,其中$\lambda$是$K$中某个元素。
\end{lemma}

\textbf{证明}:因为对一切$g\in G$有$\psi(g)\sigma =\sigma \phi(g)$,所以$Ker\sigma $是$V$的$G$不变子空间,而$Im\sigma$是$W$的$G$不变子空间。由于$(\phi,V)$和$(\psi,W)$都不可约,所以只有两种可能的情形:

(1)$Ker\sigma =0$,并且$Im\sigma=W$;

(2)$Ker\sigma =V$,并且$Im\sigma=0$;

在情形(1),$\sigma $是$V$到$W$上的同构。在情形(2),$\sigma=0$。

(i)若$\sigma \neq 0$,则$\sigma$是$V$到$W$上的同构,又由\autoref{Schlem_eq1} 知道,$\phi \approx \psi$。

(ii)若$V=W$并且$\phi=\psi$,则$\sigma$是$V$上的线性变换。因为$K$是代数闭域,所以$\sigma$由特征值$\lambda$。由\autoref{Schlem_eq1} 得,$\sigma$的属于$\lambda$的本征子空间$V_\lambda$是$G$的不变子空间。由于$(\phi,V)$不可约,因此$V_\lambda=V$,从而
\begin{equation}
\sigma=\lambda1_V.
\end{equation}


