% 万有引力
% license CCBYSA3
% type Wiki

(本文根据 CC-BY-SA 协议转载自原搜狗科学百科对英文维基百科的翻译)


牛顿的万有引力定律指出,每个粒子都会以一种与质量的乘积成正比且与它们的中心之间的距离的平方成反比的力吸引宇宙中的所有其他粒子。这是一个由艾萨克·牛顿从所谓的归纳推理经验观察中导出的一般物理定律。[1]它是经典力学的一部分,并详细地阐述在牛顿的自然哲学的数学原理 (“the Principia”) 著作之中,首次发表于1687年7月5日。当牛顿在1686年4月向皇家学会提交未出版文本的第一卷时,罗伯特·胡克声称牛顿从他那里获得了平方反比定律。

在今天的语言中,牛顿定律规定每个点的质量都受到沿与两个点相交的线的作用力吸引的其他点的质量。力与两个质量的乘积成正比,与它们之间距离的平方成反比。[2]

因此,万有引力方程的表达式为:

$F = G \frac{m_1 m_2}{r^2}$

其中 F 是作用在两个物体之间的重力,$m_1$ 和 $m_2$ 是物体的质量,$r$ 是它们的质心之间的距离,$G$ 是重力常数。

1798年,英国科学家亨利·卡文迪许在卡文迪什实验中进行了对牛顿质量间引力理论的第一次测试。[3] 它发生在牛顿的《原理》出版111年后,距他去世大约71年。

牛顿的引力定律类似于电力的库仑定律,用于计算两个带电体之间产生的电力的大小。两者都是平方反比定律,其中力与物体之间距离的平方成反比。库仑定律用两个电荷的乘积代替质量的乘积,用静电常数代替重力常数。

虽然牛顿定律后来被阿尔伯特·爱因斯坦的广义相对论所取代,但在大多数应用中,它仍然被用作为重力效应的一个极好的近似研究。只有当需要极端的准确性时,或者当处理非常强的引力场时才需要考虑相对性的影响因素,例如在极其巨大和致密的物体附近,或者在非常近的距离(例如水星绕太阳的轨道)。

\subsection{历史}

\subsubsection{1.1 早期历史}

最近,马尔迪和里奇奥利在1640年至1650年间证实了自由落体中物体的距离与所用时间之间的平方关系。他们还通过记录钟摆的振荡来计算重力常数。[4]

关于平方反比定律早期历史的现代评估是,“到17世纪70年代末”,关于“重力和距离平方成反比”的假设相当普遍,许多不同的人出于不同的原因提出了这一假设。[5]同一位作者认为罗伯特·胡克做出了重大而开创性的贡献,但是将胡克在反平方点上的优先权要求视为无关紧要,因为除了牛顿和胡克之外,还有其他几个人提出了这一点。相反,他指出“复合天体运动”的想法以及牛顿思想从“离心力”到“向心力”的转变是胡克的重要贡献。

牛顿在其《原理》中赞扬了两个人:布里亚杜斯(Bullialdus,他提出来地球有向太阳的力这一猜想)和博雷利(Borelli,他提出了所有行星都被吸引向太阳)。[6][7]对牛顿主要的影响可能来自于博雷利,他的书牛顿有一副本。[8]

\su