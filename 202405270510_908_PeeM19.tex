% 2019 年考研数学试题(数学一)
% keys 考研|数学
% license Copy
% type Tutor


\textbf{声明}:“该内容来源于网络公开资料,不保证真实性,如有侵权请联系管理员”


\subsection{选择题}
1.当$x \to 0$时,若$x-\tan x$与$x^k$是同阶无穷小,则$k=$ $(\quad)$\\
(A)1 $\quad$ (B)2 $\quad$ (C)3  $\quad$(D)4 $\quad$

2.设函数$f(x)=\leftgroup{x \abs{x}, \quad x \leq 0\\ x \ln x, \quad x>0}$,则$x=0$是$f(x)$的$(\quad)$\\
(A)可导点,极值点 $ \qquad$ (B)不可导点,极值点 \\
(C)可导点,非极值点  $\quad$(D)不可导点,非极值点 

3.设${u_n}$是单调增加的有界数列,则下列级数中收敛的是$(\quad)$\\
(A)$\displaystyle \sum_{n=1}^\infty \frac{u_n}{n} \quad$ 
(B)$\displaystyle \sum_{n=1}^\infty (-1)^n \frac{1}{u_n} \quad$
(C)$\displaystyle \sum_{n=1}^\infty \qty (1- \frac{u_n}{u_{n+1}}) \quad$ 
(D)$\displaystyle \sum_{n=1}^\infty (u_{n+1}^2-u_n ^2)$

4.设函数$Q(x,y)=\frac{x}{y^2}$,如果对上半平面 $(y>0)$ 内的任意有向光滑封闭曲线$C$都有$\oint_c P(x,y)\dd{x}+Q(x,y)\dd{y}=0$,那么函数$P(x,y)$可取为$(\quad)$\\
(A)$ \displaystyle y-\frac{x^2}{y^3} \quad$ (B)$\displaystyle \frac{1}{y}-\frac{x^2}{y^3}\quad$(C)$ \displaystyle \frac{1}{x}-\frac{1}{y}\quad$ (D)$ \displaystyle x-\frac{1}{y}\quad$

5.设$A$是3阶实对称矩阵,$E$是3阶单位矩阵。若$A^2 + A=2E$,且$\abs{A}=4$,则二次型$\mat x\Tr A \mat x$的规范形为$(\quad)$\\
(A)$y_1 ^2+y_2 ^2 +y_3 ^2 \qquad$  (B)$y_1 ^2+y_2 ^2 -y_3 ^2$ \\ 
(C)$y_1 ^2-y_2 ^2 -y_3 ^2 \qquad$  (D)$-y_1 ^2-y_2 ^2 -y_3 ^2$


6.如图1所示,有3张平面两两相交,交线相互平行,它们的方程$$a_{il}x + a_{i2}y + a_{i3}z = d_i \quad (i=1,2,3)~$$组成的线性方程组的系数矩阵和增广矩阵分别记为$\mat A , \overline{ \mat A} $,则 $(\quad)$
\begin{figure}[ht]
\centering
\includegraphics[width=6cm]{./figures/e2f1e6e4669d9ca7.png}
\caption{} \label{fig_PeeM19_1}
\end{figure}\\
(A)$r(\mat A)=2,\quad r(\bar{\mat A})=3 \qquad$ \\
(B)$r(\mat A)=2,\quad r(\bar{\mat A})=2 \qquad$\\
(C)$r(\mat A)=1,\quad r(\bar{\mat A})=2 \qquad$ \\
(D)$r(\mat A)=1,\quad r(\bar{\mat A})=1 \qquad$ 

7.设$A,B$为随机事件,则$P(A)=P(B)$的充分必要条件是 $(\quad)$\\
(A)$P( A \cup B)=P(A)+P(B) \qquad$  (B)$P(AB)=P(A)P(B)$\\
(C)$P(A\bar{ B })=P(B\bar{A}) \qquad \qquad$  (D) $P(AB)=P(\bar {A}\bar {B})$ 

8.设随机变量 $X$ 与 $Y$ 相互独立,且都服从正态分布 $N(\mu,\sigma ^2)$,则$P \{\abs{X-Y}<1\}$ $(\quad)$\\
(A)与$\mu$无关,而与$\sigma ^2$有关 $\qquad$(B)与$\mu$有关,而与$\sigma ^2$无关\\
(C)与$\mu ,\sigma ^2$都有关$\qquad \qquad \qquad$(D)与$\mu ,\sigma ^2$都无关



\subsection{填空题}
1.设函数$f(u)$可导,$z=f(\sin y-\sin x)+xy$ ,则 $\displaystyle \frac{1}{\cos x}\cdot \pdv{z}{x}+\frac{1}{\cos y}\cdot \pdv{z}{y}=$ $(\quad)$。

2.微分方程$2yy'-y^2-2=0$满足条件$y(0)=1$的特解$y=$ $(\quad)$。

3.幂级数$\displaystyle \sum_{n=0}^\infty \frac{(-1)^n}{(2n)!} x^n$在$(0,+\infty)$内的和函数$S(x)=$ $(\quad)$。

4.设 $\Sigma$ 设为曲面 $x^2+y^2+4z^2=4 (z\ge 0)$ 的上侧,则$\int\int_{\Sigma} \sqrt{4-x^2-4z^2}\dd{x}\dd{y}=$ $(\quad)$。

5.设$A=(\alpha_1,\alpha_2,\alpha_3)$为3阶矩阵,若$\alpha_1,\alpha_2$线性无关,且$\alpha_3=-\alpha_1+2\alpha_2$,则线性方程组$Ax=0$的通解为 $(\quad)$。

6.设随机变量$X$的概率密度$f(x)=\leftgroup{\frac{x}{2}, \quad &0<x<2\\  0, \quad &\text{其他}}$ ,$F(x)$为$X$的分布函数,$E(x)$为$X$的数学期望,则$P\{F(X)>E(X)-1\}=$ $(\quad)$。


\subsection{解答题}
1.设函数$y(x)$是微分方程$y'+xy=e^{-\frac{x^2}{2}}$满足条件$y(0)=0$的特解。\\
(1).求$y(x)$;\\
(2).求曲线$y=y(x)$的凹凸区间及拐点。

2.设 $a,b$ 为实数,函数$z=2+ax^2+by^2$在点(3,4)处的方向导数中,沿方向$l=-3i-4j$的方向导数最大,最大值为10。\\
(1).求 $a,b$;\\
(2).求曲面$z=2+ax^2+by^2 \quad(z\geqslant 0)$的面积。

3.求曲线 $y=e^{-x}sinx \quad (x\geqslant 0)$ 与 $x$ 轴之间图形的面积。

4.设$a_n=\int_0^1 x^n \sqrt{1-x^2}\dd{x}\quad (n=0,1,2,\dots)$\\
(1).证明数列 ${a_n}$单调递减,且$\displaystyle a_n=\frac{n-1}{n+2}a_{n-2}\quad (n=2,3,\dots);$\\
(2).求 $\displaystyle \lim_{n \to \infty}\frac{a_n}{a_{n-1}}$

5.设$\Omega$是由锥面 $x^2+(y-z)^2=(1-z)^2 \quad (0\leqslant z \leqslant 1)$ 与平面 $z=0$ 围成的锥体,求$\Omega$的形心坐标。

6.设向量组$\bvec a_1=(1,2,1) \Tr,\bvec a_2=(1,3,2) \Tr,\bvec a_3=(1,a,3) \Tr$为$R^3$的一个基,$\beta=(1,1,1)\Tr$在这个基下的坐标为$(b,c,1) \Tr$。\\
(1).求$a,b,c$;\\
(2)。证明 $\bvec a_2,\bvec a_3,\bvec \beta$ 为 $R^3$ 的一个基,并求 $\bvec a_2,\bvec a_3,\bvec \beta$到$\bvec a_1,\bvec a_2,\bvec a_3$ 的过度矩阵。

7.已知矩阵$\mat A=\pmat{-2 & -2 & 1\\2& x & -2\\0 & 0 &-2}$与$\mat B=\pmat{2 &1 & 0\\0 & -1 & 0\\0 & 0 &y}$相似。\\
(1).求$x,y$;\\
(2).求可逆矩阵$\mat P$使得$\mat{P^{-1}AP}=\mat B$.

8.设随机变量 $X$ 与 $Y$ 相互独立,$X$ 服从参数为1的指数分布,$Y$的概率分布为$P\{Y=-1\}=p,P\{Y=1\}=1-p \quad (0<p<1)$。令$Z=XY$。\\
(1).求$Z$的概率密度;\\
(2).$p$为何值时,$X$ 与 $Z$ 不相关;\\
(3)$X$ 与 $Z$ 是否相互独立?

9.设总体$X$的概率密度为
$$
f(x;\sigma)=\leftgroup{\frac{A}{\sigma}e^{\frac{-(x-\mu)^2}{2\sigma^2}},\quad  x \geqslant \mu\\ 0, \quad  x< \mu}~,
$$
其中 $\mu$ 是已知参数,$\sigma$>0 是未知参数,$A$ 是常数,$X_1,X_2,\dots,X_n$ 是来自总体 $X$ 的简单随机样本。\\
(1).求$A$;\\
(2).求$\sigma^2$的最大似然估计量。
