% 磁偶极矩
% keys 磁偶极矩|静磁学|电流|磁场

\begin{issues}
\issueDraft
\end{issues}
类似于电偶极子\upref{eleDpl},我们可以定义磁偶极子.在经典电动力学中,磁偶极子可以理解为一个带恒定电流的环形小回路.
\begin{figure}[ht]
\centering
\includegraphics[width=5cm]{./figures/Bdipol_1.png}
\caption{磁偶极子示意图} \label{Bdipol_fig1}
\end{figure}

定义他的磁偶极矩$\bvec m$为
\begin{equation}
\bvec m = I \bvec s
\end{equation}
其中$I$是这个回路的电流,$\bvec s = ab\bvec n$是面积元矢量,数值上等于回路的面积,方向由电流方向与右手法则确定.

\subsection{磁偶极子在磁场中所受的力与力矩}
\begin{figure}[ht]
\centering
\includegraphics[width=8cm]{./figures/Bdipol_2.png}
\caption{磁偶极子在磁场中所受的力} \label{Bdipol_fig2}
\end{figure}
由于$\bvec F = I \oint \bvec l \times \bvec B = I (\oint \bvec l) \times \bvec B = 0$,因此具有环形回路的磁偶极子所受合力始终为零.

但是,磁偶极子在磁场中的受合力矩,最终的效果使$\bvec m \parallel \bvec B$

\begin{equation}
\bvec M = \bvec m \times \bvec B
\end{equation}

------
\begin{equation}
\bvec B(\bvec r) = \frac{\mu_0}{4\pi} \frac{1}{r^3} [3(\bvec m \vdot \uvec r) \uvec r - \bvec m]
\end{equation}
