% 麦克斯韦-玻尔兹曼统计(综述)
% license CCBYNCSA3
% type Wiki

本文根据 CC-BY-SA 协议转载翻译自维基百科\href{https://en.wikipedia.org/wiki/Maxwell\%E2\%80\%93Boltzmann_statistics}{相关文章}。

在统计力学中,麦克斯韦–玻尔兹曼统计描述了经典物质粒子在热平衡中不同能级上的分布。当温度足够高或粒子密度足够低,以至于量子效应可以忽略不计时,这一统计方法适用。
\begin{figure}[ht]
\centering
\includegraphics[width=10cm]{./figures/edce00b1e7050d6f.png}
\caption{麦克斯韦–玻尔兹曼统计可用于推导理想气体中粒子速度的麦克斯韦–玻尔兹曼分布。展示了:在 -100°C、20°C 和 600°C 下,106 个氧分子的速度分布。} \label{fig_MBTJ_1}
\end{figure}
对于麦克斯韦–玻尔兹曼统计,能量为\(\varepsilon_i\)的粒子的期望数为:
\[
\langle N_i \rangle = \frac{g_i}{e^{(\varepsilon_i - \mu) / kT}} = \frac{N}{Z} g_i e^{-\varepsilon_i / kT}~
\]
其中:
\begin{itemize}
\item \(\varepsilon_i\)是第\( i \)个能级的能量,
\item \( \langle N_i \rangle \)是能量为\( \varepsilon_i \)的状态集合中粒子的平均数,
\item \( g_i \)是第\( i \)个能级的简并度,即具有能量 \( \varepsilon_i \)的状态的数量,这些状态可以通过其他方式加以区分[注1],
\item \( \mu \)是化学势,
\item \( k \)是玻尔兹曼常数,
\item \( T \)是绝对温度,
\item \( N \)是总粒子数:
  \[
  N = \sum_i N_i~
  \]
\item \( Z \) 是配分函数:
  \[
  Z = \sum_i g_i e^{-\varepsilon_i / kT}~
  \]
\item \( e \) 是欧拉数。
\end{itemize}
等效地,粒子数有时表示为:
\[
\langle N_i \rangle = \frac{1}{e^{(\varepsilon_i - \mu) / kT}} = \frac{N}{Z} e^{-\varepsilon_i / kT}~
\]
其中,索引\( i \)现在指定了一个特定的状态,而不是所有能量为\( \varepsilon_i \)的状态集合,并且配分函数\( Z \)由以下式子给出:
\[
Z = \sum_i e^{-\varepsilon_i / kT}~
\]
\subsection{历史}  
麦克斯韦–玻尔兹曼统计源于麦克斯韦–玻尔兹曼分布,很可能是对基础技术的提炼。该分布最早由麦克斯韦于1860年根据启发式方法推导出来。后来,玻尔兹曼在1870年代对该分布的物理起源进行了重要研究。该分布可以通过最大化系统熵的原理推导出来。
\subsection{与麦克斯韦–玻尔兹曼分布的关系}  
麦克斯韦–玻尔兹曼分布和麦克斯韦–玻尔兹曼统计密切相关。麦克斯韦–玻尔兹曼统计是统计力学中的一个更为一般的原理,用于描述经典粒子处于特定能量状态的概率:
\[
P_i = \frac{e^{-E_i / k_B T}}{Z}~
\]
其中:
\begin{itemize}
\item \( Z \) 是配分函数:
  \[
  Z = \sum_i e^{-E_i / k_B T}~
  \]
\item \( E_i \) 是状态 \( i \) 的能量,
\item \( k_B \) 是玻尔兹曼常数,
\item \( T \) 是绝对温度。
\end{itemize}
