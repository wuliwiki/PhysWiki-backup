% 马尔可夫链蒙特卡洛
% keys 马尔可夫链|蒙特卡洛方法|计算统计学
% license Usr
% type Wiki

\textbf{马尔可夫链蒙特卡洛}(Markov Chain Monte Carlo,简称MCMC)是一种用于从概率分布中抽样的统计方法。这种方法结合了马尔可夫链和蒙特卡洛模拟的思想,被广泛应用于贝叶斯统计、统计物理学、机器学习等领域。

\section{动机}
\subsection{贝叶斯推断 Bayesian Inference}

贝叶斯推断的主要目标是通过结合先验分布和观测数据,利用贝叶斯定理推导参数的后验分布。该定理建立了先验概率 \(P(\theta)\)、似然函数 \(P(D|\theta)\) 和得到的后验概率 \(P(\theta|D)\) 之间的关系。

\begin{equation}
P(\theta|D) = \frac{P(D|\theta) \times P(\theta)}{P(D)} = \frac{P(D|\theta) \times P(\theta)}{\int P(D|\hat{\theta}) \times P(\hat{\theta}) \, d\hat{\theta}}~.
\end{equation}