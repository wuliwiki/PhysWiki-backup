% 银行家算法
% license CCBYSA3
% type Wiki

(本文根据 CC-BY-SA 协议转载自原搜狗科学百科对英文维基百科的翻译)

\textbf{银行家算法}算法(有时称为\textbf{检测算法})是由Edsger Dijkstra开发的资源分配和死锁避免算法,它通过模拟所有资源的预定最大可能量的分配来测试安全性,然后在决定是否允许继续分配之前进行“s状态” ,检查以测试所有其他待处理活动的可能死锁条件。该算法是在THE操作系统的设计过程中开发的,最初在EWD108中以荷兰语描述。[1] 当新进程进入系统时,它必须声明它可能需要的每种资源类型的最大实例数; 显然,该数字可能不会超过系统中的资源总数。 此外,当进程获取其所有请求的资源时,它必须在有限的时间内返回它们。

\subsection{资源}
银行家算法要发挥作用,需要知道三件事:
\begin{itemize}
\item 每个进程可以请求每个资源的最大值
\item 每个进程目前拥有多少分配的资源
\item 系统目前有多少资源可用
\end{itemize}
只有当请求的资源量小于或等于可用的资源量时,才能将资源分配给进程;否则,该过程将一直等到资源可用。

在实际系统中跟踪的一些资源是内存,信号量和接口访问。

银行家算法的名称源于这样一个事实,即该算法可用于银行系统,以确保银行不会耗尽资源,因为银行永远不会以不再满足资金的方式分配资金。 所有客户的需求[2]。 通过使用银行家的算法,银行确保当客户要求资金时,银行永远不会离开安全状态。 如果客户的请求不会导致银行离开安全状态,则会分配现金,否则客户必须等到其他客户存款足够。

实施银行家算法需要维护的基本数据结构:

设$n$是系统中的进程数,$m$是资源类型的数量。。那么我们需要以下数据结构:
\begin{itemize}
\item 可用:长度为m的向量表示每种类型的可用资源数。 如果Available [j] = k,则有k个资源类型为Rj的实例可用。
\item 最多:n×m矩阵定义每个过程的最大需求。 如果Max [i,j] = k,则Pi可以请求最多k个资源类型Rj的实例。
\item 可分配:n×m矩阵定义当前分配给每个进程的每种类型的资源数。 如果Allocation [i,j] = k,则过程Pi当前被分配k个资源类型Rj的实例。
\item 需求:n×m矩阵表示每个进程的剩余资源需求。 如果Need [i,j] = k,则Pi可能需要k个更多的资源类型Rj实例来完成任务。注意:需要[i,j] = Max [i,j] - 分配[i,j]。N = M-A。 实施银行家算法需要维护的基本数据结构:
\end{itemize}

