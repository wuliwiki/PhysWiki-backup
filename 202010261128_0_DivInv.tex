% 散度的逆运算
% keys 散度|库伦定律|不定积分

\begin{theorem}{}
令 $V(\bvec r)$ 为任意标量函数, 则 $V(\bvec r)$ 总能表示为一个矢量场 $\bvec F(\bvec r)$ 的散度, 即
\begin{equation}
V(\bvec r) = \div \bvec F(\bvec r)
\end{equation}
且 $\bvec F(\bvec r)$ 可以通过以下公式计算:
\begin{equation}
\bvec F(\bvec r) \equiv \frac{1}{4\pi}\int \frac{V \bvec R}{R^3} \dd{V'}
\end{equation}
其中 $\bvec r, \bvec r'$ 分别是坐标原点指向三维直角坐标 $(x, y, z)$ 和 $(x', y', z')$ 的位置矢量, $\bvec R = \bvec r' - \bvec r$, $R = \abs{\bvec R}$, 体积分 $\int\dd{V'} = \int\dd{x'}\dd{y'}\dd{z'}$ 的区域是空间中 $\bvec F$ 不为零的区域, $\cross$ 表示矢量叉乘\upref{Cross}.

\end{theorem}
