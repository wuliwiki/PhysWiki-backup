% 幂的定义

\pentry{极限存在的判据、柯西序列\upref{CauSeq}}

在中学数学中, 我们已经学习过正实数的幂的定义. 按照此定义, 给定实数\upref{ReNum}$x>0$和任意实数$y,z$, 幂$x^y$和$x^z$满足如下性质:

\begin{itemize}
\item 对于任意的$x>0$, 均有$x^0=1$.
\item $x^{y+z}=x^y\cdot x^z$.
\item $(x^y)^z=x^{yz}$.
\item 给定$x_1,x_2>0$, 那么$x_1^y\cdot x_2^y=(x_1x_2)^y$.
\item 如果$x>1$, $y<z$, 那么$x^y<x^z$.
\end{itemize}

特别地, 对于正整数$n$, $x^n$就是将$x$自乘$n$次, $x^{-n}$就是$1/x^n$, 而$x^{1/n}$就是$x$的$n$次算术根. 

一个自然的问题是, 满足以上五条性质的运算是否唯一? 答案是肯定的. 严格的实数理论给出了一个构造幂的方法.

\subsection{有理数幂的构造}

首先, 正实数的整数次幂可以通过乘法来直接定义. 它当然满足以上五个要求. 现在要由此出发来构造正实数的有理数次幂.

给定正整数$n$. 根据实数理论, 我们可以严格地构造出任意正实数$x$的$n$次算术根, 也就是满足$y^n=x$的唯一一个正实数$y$. 由于正整数次幂的保序性质, 可见算术根若存在则必然只有唯一一个. 至于实际的构造, 则可以确定一个戴德金分割如下: 分割的下类$L$包含所有非正的有理数, 以及满足$l^n<x$的正有理数$l$; 分割的上类$R$包含所有满足$r^n\geq x$的正有理数. 容易验证它满足戴德金分割的定义, 从而它确定了一个实数$y$. 

为了说明$y^n=x$, 只需要注意到, 实数$y$是下类$L_y$的上确界, 同时也是上类$R_y$的下确界. 所以给定$\varepsilon>0$, 都存在$l\in L_y$和$r\in R_y$使得
\[
y-\varepsilon<l<y,\quad y\leq r<y+\varepsilon.
\]
于是
\[ 
r^n-l^n<(r-l)(r^{n-1}+...+l^{n-1})
<2n(y+\varepsilon)^{n-1}\varepsilon
\]