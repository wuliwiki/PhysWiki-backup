% 负温度
% keys 负温度|核自旋系统|熵

\begin{issues}
\issueDraft
\end{issues}
\pentry{正则系综法\upref{CEsb}}

1951年珀塞耳(Purcell)和庞德(Pound)发现氟化锂(LiF) 晶体中的核自旋系统可以处于负绝对温度状态。
1956年喇姆塞(Ramsey)给予了理论解释。\cite{热统}

这里的温度在统计力学中的严格意义是\autoref{eq_tmp_1}~\upref{tmp}
\begin{equation}
\frac{1}{T}=\left(\frac{\dd S}{\dd E}\right)_{V,N}^{-1}
\end{equation}
有时也将 $E$ 记做 $U$,这都是些符号上的不同使用习惯。在一般系统中,熵随内能的增大而增大(直观的理解就是,系统越热,就越“乱”)。但也存在一些特殊的系统,它们的熵在特定温度范围内随内能的升高而减小,此时由上式得出的温度就是负的。

\subsection{理论简介}
设核自旋量子数为 $j=\frac{1}{2}$,在外磁场中,核磁矩 $\mu$ 相对于外磁场 $H$ 只有平行和反平行两种取向。相应的能量也只有两种取值:$\epsilon_1=-\mu H,\quad \epsilon_2=\mu H$。假设每个能级只有一个量子态(简并度 $g_1=g_2=1$)。由于核自旋彼此之间相互作用很弱,它们组成了\textbf{近独立}\upref{depsys}的\textbf{定域子系}。

设系统的平均能量为 $\bar{E}=N_1\epsilon_1+N_2\epsilon_2=\mu H(N_2-N_1),\ N_1+N_2=N$,当 $N_1=N$ 时所有磁矩都与外磁场方向平行,系统处于能量最低的有序状态;当 $N_2=N$ 时所有磁矩都与外磁场反平行,系统处于能量罪高的有序状态。可以想象随着能量的增大,核自旋系统从有序到无序再转变为有序。根据玻尔兹曼熵公式\footnote{参考玻尔兹曼分布(统计力学)\upref{MBsta}或热力学量的统计表达式(玻尔兹曼分布)\upref{TheSta}的\autoref{eq_TheSta_7}~\upref{TheSta}。},$S=k\ln \Omega$,因此我们可以预见系统的熵随能量的变化曲线是先增大后减小的。
\begin{figure}[ht]
\centering
\includegraphics[width=10cm]{./figures/0b6f27fcd40ac2b4.png}
\caption{参考林宗涵\cite{林宗涵} 的图7.8.4} \label{fig_negtem_1}
\end{figure}
那么根据


子系(单个核构成的子系)的配分函数为
\begin{equation}
\begin{aligned}
Z=\sum_{\lambda}g_\lambda e^{-\beta \epsilon_\lambda}=e^{\beta \epsilon } +e^{-\beta \epsilon}
\end{aligned}
\end{equation}

利用近独立子系的热力学量表达式 \autoref{eq_TheSta_4}~\upref{TheSta} 和 \autoref{eq_TheSta_3}~\upref{TheSta} ,可以由子系的配分函数得到系统的能量和熵的表达式:
\begin{equation}
\begin{aligned}
&\frac{\bar E}{N}=-\pdv{\beta} \ln Z=-\frac{1}{Z}\pdv{Z}{\beta}
=-\epsilon\frac{e^{\beta\epsilon}-e^{-\beta\epsilon}}{e^{\beta\epsilon}+e^{-\beta\epsilon}}\\
&\frac{S}{Nk}
=\ln Z-\beta\pdv{\beta} \ln Z=
\ln \left(e^{\beta\epsilon}+e^{-\beta\epsilon}\right)-
\beta\epsilon \frac{e^{\beta\epsilon}-e^{-\beta\epsilon}}{e^{\beta\epsilon}+e^{-\beta\epsilon}}
\end{aligned}
\end{equation}
