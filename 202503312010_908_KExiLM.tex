% 柯西-黎曼方程(综述)
% license CCBYSA3
% type Wiki

本文根据 CC-BY-SA 协议转载翻译自维基百科\href{https://en.wikipedia.org/wiki/Cauchy\%E2\%80\%93Riemann_equations}{相关文章}。

\begin{figure}[ht]
\centering
\includegraphics[width=8cm]{./figures/9b579a327747d777.png}
\caption{一个向量 \( X \) 在一个域中被复数 \( z \) 乘以后再通过函数 \( f \) 映射,和先通过函数 \( f \) 映射再乘以 \( z \) 的情况进行的视觉对比。如果这两种情况对于所有 \( X \) 和 \( z \) 都导致点最终落在相同的位置,那么函数 \( f \) 满足柯西-黎曼条件。} \label{fig_KExiLM_1}
\end{figure}
在数学的复分析领域,柯西–黎曼方程以奥古斯丁·柯西和伯恩哈德·黎曼命名,由一组二阶偏微分方程组成,这些方程为复变量的复函数可复微分的必要与充分条件。

这些方程是:
\[
\frac{\partial u}{\partial x} = \frac{\partial v}{\partial y} \quad (1a)~
\]

和
\[
\frac{\partial u}{\partial y} = -\frac{\partial v}{\partial x} \quad (1b)~
\]
其中,\(u(x, y)\)和\(v(x, y)\)是实值的二元可微函数。
