% SymPy 符号计算笔记

\begin{issues}
\issueDraft
\end{issues}

\pentry{Python 符号计算简介\upref{SymPy}}

为了书写方便, 本文使用 \verb|from sympy import *|. 所有的变量都假设用 \verb|sympy.symbols()| 声明过.

\subsection{基础}
\begin{itemize}
\item \href{https://docs.sympy.org/latest/reference/index.html}{参考文档}.
\item \verb|import sympy as sm|
\item \verb|x0, x1 = symbols('x0, x1')| 声明变量, 类型为 \verb|sympy.core.symbol.Symbol|
\item \verb|x, y, z = symbols('x:z')|, \verb|x4, x5, x6, x7 = symbols('x4:8')|
\item 或者用 \verb|from sympy.abc import a, b, c, x, y, z, i, j, k, m, n, alpha, beta|
\item \verb|x = S('x')| 也等效于 \verb|x = symbols('x')|
\item 常数如 \verb|pi, E, I, oo| (无穷)
\item 整数为 \verb|Integer(n)| 或者 \verb|numer(n)| 或者 \verb|S(n)|, 类型为 \verb|sympy.core.numbers.Integer|, 也可能是 \verb|sympy.core.numbers.One|
\item 有理数 \verb|Rational(n,m)|. 如果 \verb|n, m| 已经是 \verb|numbers.Integer| 类型, 也可以直接 \verb|n/m|
\item 虚数单位 \verb|I| 的类型是 \verb|sympy.core.numbers.ImaginaryUnit|, 其他虚数和复数都没有专门的类型而是 \verb|I| 和其他实数相乘相加.
\item 函数如 \verb|sin(), asin(), sinh(), exp(), log(), sqrt()|
\item \verb|latex(表达式)| 可以把表达式转换为 latex 代码.
\item \verb|表达式.subs(符号, 表达式或值)| 替换表达式中的符号
\item \verb|表达式.subs({符号1:表达式或值1, 符号2:表达式或值2})| 替换多个符号
\end{itemize}

\subsubsection{检查表达式结构}
\begin{itemize}
\item 虽然表达式的类型有很多, 但都有 \verb|args| 参数.
\item 例如令 \verb|expr = sin(x)**2 + cos(x)**2|, 那么 \verb|expr| 的类型是 \verb|sympy.core.add.Add|, \verb|expr.args| 是 \verb|(cos(x)**2, sin(x)**2)|. \verb|expr.args[0]| 的类型是 \verb|sympy.core.power.Pow|, \verb|expr.args[0].args| 是 \verb|(cos(x), 2)|, 以此类推. 这样就可以生成一个树状结构.
\end{itemize}

\subsection{基本运算}
\begin{itemize}
\item \verb|summation(含i的表达式, (i, 第一个, 最后一个))| (注意求和包含 \verb|最后一个| 的项), 例如 \verb|summation(1/2**n, (n, 1, 2)))| 得 \verb|3/4|, 又例如 \verb|summation(1/2**n, (n, 1, oo))| 得 \verb|1|.
\item \verb|product(含i的表达式, (i, 第一个, 最后一个))| 同理
\end{itemize}

\subsection{线性代数}
\begin{itemize}
\item 矩阵 \verb|mat = Matrix([[x**2, sin(x)], [1, 1/y]])|, 类型为 \verb|sympy.matrices.dense.MutableDenseMatrix|
\item \verb|mat[i,j]| 获取矩阵元.
\item Matrix 永远是二维的, 行主序, \verb|mat[i]| 返回单个矩阵元而不是像 numpy 一样返回一行.
\item \verb|mat[:,j]| 获取一列, 类型还是一样.
\item 其他一些操作和 NumPy\upref{numpy} 数组类似, 如 \verb|len(mat)| 获取元素个数, \verb|shape(mat)| 或者 \verb|mat.shape| 获取形状. 又如 \verb|Matrix([1,2])| 的形状是 \verb|(2,)|, 而 \verb|Matrix([[1,2]])| 的形状是 \verb|(1,2)|.
\item \verb|Matrix(Numpy矩阵)| 也可以生成 sympy 矩阵.
\item \verb|mat.norm()| 返回 2-norm, \verb|mat.norm(n)| 返回 n-norm.
\item 矩阵乘法用 \verb|mat1 * mat2|
\item 点乘 \verb|dot(v1, v2)|
\item \verb|mat.eigenvals()| 求本征值, 返回一个字典, \verb|表达式:简并重数|. \verb|mat.eigenvals(multiple=true)| 返回一个 list, 可能包含重复的表达式. 说明见 \verb|help(mat.eigenvals)|
\item \verb|mat.eigenvects()| 求本征值和本征矢, 返回 tuple 的 list: \verb|[(本征值, 简并数, 本征空间), ...]|
\end{itemize}


\subsection{微积分}
\begin{itemize}
\item 极限如 \verb|limit(sin(x)/x, x, 0)|
\item 求导(偏导)如 \verb|表达式.diff(x)|, 或者 \verb|diff(表达式, x)|
\item 高阶偏导如: \verb|表达式.diff(x,y)|, 或者 \verb|diff(表达式, x, y)| 表示 $\pdv*{f}{x}{y}$. \verb|diff(表达式, x, 3)| 表示三阶导数. \verb|diff(表达式, x, 2, y)| 相当于 \verb|diff(表达式, x, x, y)|
\item 积分如 \verb|integrate(exp(x)*(sin(x) + cos(x)), x)|
\item 多重积分 \verb|integrate(y**2 * x**2, (x,0,2), (y,-1,1))|
\item 格式为 \verb|integrate(表达式, (x, 下限, 上限))|
\item 微分方程 \verb|dsolve(Eq(y(t).diff(t, t) - y(t), exp(t)), y(t))|
\end{itemize}


\subsection{任意精度求值}
\begin{itemize}
\item 通过 \href{https://mpmath.org/}{mpmath} 库完成任意精度计算(和 arb\upref{ArbLib} 是同一个作者, 但是完全用 python 编写)
\item \verb|N(表达式, 有效数字)| 对表达式求值, 例如 \verb|N(pi, 50)|, \verb|N(sin(numer(1)), 50)|
\end{itemize}

\subsection{物理}
\begin{itemize}
\item \verb|import sympy.physics as ph|
\item \verb|import sympy.physics.wigner as wi|
\item CG 系数 \verb|wi.clebsch_gordan(j1, j2, j3, m1, m2, m3)|
\end{itemize}
