% 厘米—克—秒单位制
% keys 厘米|克|秒|单位|量纲|转换|国际单位|达因

\begin{issues}
\issueDraft
\issueOther{高斯单位制和 CGS 单位有什么区别?}
\end{issues}

\pentry{国际单位制\upref{SIunit}, 物理量和单位转换\upref{Units}}

\footnote{参考 Wikipedia \href{https://en.wikipedia.org/wiki/Centimetre-gram-second_system_of_units}{相关页面}.}\textbf{厘米—克—秒单位制(centimetre–gram–second system of units)}也简记为 CGS 单位制. 我们将从 CGS 到 SI 单位制之间的转换常数记为 $\beta_\text{物理量}$, CGS 单位制的物理量符号用角标 $c$ 加以区分. 例如 $x = \beta_x x_c$ 其中 $x$ 是国际单位制的物理量, $x_c$ 是 CGS 单位的物理量.

则 $\beta_x = 0.01\Si{m/cm}$, $\beta_m = 0.001\Si{kg}/\Si{g}$, $\beta_t = 1$. 为了满足 $a_c = \ddot x_c$, 有
\begin{equation}
\beta_a = \beta_x/\beta_s^2 = \beta_x = 0.01 \Si{m/cm}
\end{equation}
为了满足牛顿定律 $F_c = m_ca_c$ 得
\begin{equation}
\beta_F = \beta_m \beta_x/\beta_s^2 = \beta_m\beta_x = 10^{-5} \Si{\frac{kg\cdot m}{g\cdot cm}}
\end{equation}
为了满足 $E = Fs$,得
\begin{equation}
\beta_E = \beta_F \beta_x = \beta_m\beta_x^2 =  10^{-7} \Si{\frac{kg \cdot m^2}{g\cdot cm^2}}
\end{equation}

力的单位叫做\textbf{达因(dyne)}, 记作 $\Si{dyn}$. 注意以上的转换常数本质上全部等于无量纲的 1.
