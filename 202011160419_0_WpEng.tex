% 电磁波包的能谱
% keys 电磁波|波包|能量

\pentry{傅里叶变换(指数)\upref{FTExp}, 平面电磁波的能量叠加\upref{PwvAdd}}

对于真空中的平面波电磁波, 波速恒定为 $c$, 如果知道某点 $x_0$ 处的电场—时间关系 $g(t)$, 如何求波函数 $f(x - ct)$ 呢? 代入 $x = x_0$ 可知 $g(t) = f(x_0 - ct)$, 所以
\begin{equation}
f(x) = g\qty(\frac{x_0 - x}{c})
\end{equation}

当这个波包完整经过一个平面后, 穿过平面的能量密度为(积分上下限为 $\infty$)
\begin{equation}\label{WpEng_eq1}
\sigma_E = \epsilon_0 \int f(x)^2 \dd{x} = \epsilon_0  \int g^2\qty(\frac{x_0 - x}{c}) \dd{x} = c\epsilon_0 \int g^2(u) \dd{u}
\end{equation}
另一种方法是把坡印廷矢量\upref{EBS}对时间积分, 同样能得到该式.

\subsection{能量的频率分布}
根据傅里叶变换的归一化不变性(\autoref{FTExp_eq2}~\upref{FTExp}), 所以若令 $g$ 的傅里叶变换为 $\tilde g$ 则
\begin{equation}
\sigma_E = c\epsilon_0 \int {\tilde g}^2(\omega) \dd{\omega}
\end{equation}
这相当于把波包看作是许多不同频率简谐波的叠加, 总能量密度是每个简谐波能量密度之和. 所以面密度的频率分布为
\begin{equation}\label{WpEng_eq3}
s(\omega) = c\epsilon_0 {\tilde g}^2(\omega)
\end{equation}
考虑到光子能量为 $E = \omega\hbar$, 光子能量分布为.
\begin{equation}\label{WpEng_eq2}
s(E) = \frac{c\epsilon_0}{\hbar} {\tilde g}^2\qty(\frac{E}{\hbar})
\end{equation}

在库仑规范(量子)\upref{CouGau}下, 矢势为 $A(t)$ 对于波包有
\begin{equation}
g(t) = -\dv{A(t)}{t}
\end{equation}
由傅里叶变换的求导公式(\autoref{FTExp_eq3}~\upref{FTExp})得
\begin{equation}
\tilde g(\omega) = -\I \omega \tilde A(\omega)
\end{equation}
\autoref{WpEng_eq3} 得
\begin{equation}
s(\omega) = c\epsilon_0 \omega^2 {\tilde A}^2(\omega)
\end{equation}
