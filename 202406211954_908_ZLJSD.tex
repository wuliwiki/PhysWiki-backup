% 重力加速度
% license CCBYSA3
% type Wiki

(本文根据 CC-BY-SA 协议转载自原搜狗科学百科对英文维基百科的翻译)


重力加速度(又名自由落体加速度)是一个物体受重力作用的情况下所具有的加速度。

通常指地面附近物体受地球引力作用在真空中下落的加速度,记为g。为了便于计算,其近似标准值通常取为980cm/s²或9.8m/s²。在月球、其他行星或星体表面附近物体的下落加速度,则分别称月球重力加速度、某行星或星体重力加速度。

\subsection{对于点质量}

牛顿万有引力定律指出,任何两个质点之间都有一个引力,该引力对每个质点的大小相等,并且两个质点处于同一直线,使它们相互靠近。万有引力公式为:




\subsection{地球重力模型}



\subsection{广义相对论}

在爱因斯坦的广义相对论中,引力是弯曲时空的一个属性,而不是由于物体间传播的力。在爱因斯坦的理论中,质量扭曲了附近的时空,其他粒子沿着时空几何形状决定的轨迹运动。重力是一种虚构的力。 没有重力加速度,因为物体在自由落体中的适当加速度和四加速度为零。自由落体中的物体不会经历加速度,而是沿着弯曲时空上的直线(测地线)运动。

\subsection{参考文献}

[1]
^Fredrick J. Bueche (1975). Introduction to Physics for Scientists and Engineers, 2nd Ed. USA: Von Hoffmann Press. ISBN 978-0-07-008836-8..

[2]
^Brian L. Stevens; Frank L. Lewis (2003). Aircraft Control And Simulation, 2nd Ed. Hoboken, New Jersey: John Wiley & Sons, Inc. ISBN 978-0-471-37145-8..

[3]
^Richard B. Noll; Michael B. McElroy (1974), Models of Mars' Atmosphere [1974], Greenbelt, Maryland: NASA Goddard Space Flight Center, SP-8010..