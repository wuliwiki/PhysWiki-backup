% 线性方程组
% keys 线性方程组|线性代数|二元一次|行变换|齐次|行列式
% license Xiao
% type Tutor

\begin{issues}
\issueMissDepend
\end{issues}

\subsection{低维简单例子}
\subsubsection{二元一次方程组}
在许多实际生活中,我们往往需要解决类似的问题:
%大体完成,还差插入图片与表达的细节优化

小明拿着家长给的 $100$ 元去超市买饮料。超市里有 $3$ 元钱一瓶的可口可乐和 $5$ 元钱一瓶的奶茶,最后他带着 $28$ 瓶饮料回家过暑假了。求问小明买了多少瓶可口可乐,多少瓶奶茶?

这个问题实际上是一个二元一次方程组的问题,我们设小明买了 $x$ 瓶可口可乐,$y$ 瓶奶茶,可以列出一个方程组:\begin{equation}
\leftgroup{
100 &= 3x + 5y & &(a)\\
28 &= x + y & &(b)~.\\
}\end{equation}
一个简单的解决方法是计算 $(a)-3 \cdot (b)$,得到 $16 = 0x + 2y$ 即 $y = 8$,进一步就知道 $x = 20$.

\subsubsection{三元一次方程组}
类似的在三维坐标系里面,考察三个平面:$S_1:x - 3y-2z=3$,$S_2:-2x+y-4z=-9$ 与 $S_3:-x+3y-z=-7$ 的交点。
那么这个问题等价于解决如下三元一次方程组:\begin{equation}
\leftgroup{
x - 3y - 2z &= 3\\
- 2x + y - 4z &= -9\\
- x + 3y - z &= -7~.\\
}\end{equation}
解得\begin{equation}
\leftgroup{
x &= 2\\
y &= -1\\
z &= 1~.\\
}\end{equation}
那么 $(2,-1,1)$ 就是三维坐标系中平面 $S_1$,$S_2$ 与 $S_3$ 的交点。
\subsection{一般情况下的定义}
一般的,形如:
\begin{equation}\label{eq_LinEqu_1}
\leftgroup{
a_{1,1}x_1 + a_{1,2}x_2 + \dots + a_{1,n}x_n &= y_1~,\\
a_{2,1}x_1 + a_{2,2}x_2 + \dots + a_{2,n}x_n &= y_2~,\\
\vdots \\
a_{m,1}x_1 + a_{m,2}x_2 + \dots + a_{m,n}x_n &= y_m~.}
\end{equation}
的等式组统称为线性方程组,也可以根据其未知数的个数称为 $n$ 元一次方程组。

其中 $x_1\dots x_n$ 为 $n$ 个未知量,$y_1\dots y_m$ 与 $a_{1,1} ,a_{1,2}\dots a_{1,n},a_{2,1} \dots a_{n,m}$ 为给定的系数。(形如 $a_{i.j}$ 的系数表示它是方程组中第 $i$ 个方程的 $x_j$ 对应的系数,也即第 $i$ 行第 $j$ 列系数)

\subsection{求解线性方程组}

以刚才的\autoref{eq_LinEqu_1}为例,对于这样一个方程组,我们的做法是“消元”求解,不停地利用上面的行乘以某个bei'sh(这个做法在将来会叫做\textbf{初等行变换})。
