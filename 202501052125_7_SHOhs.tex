% 简谐振子(高中)
% keys 弹簧振子|simple harmonic oscillation|单摆
% license Usr
% type Tutor

\begin{issues}
\issueDraft
\end{issues}

进一步了解简谐振子,请参考\enref{简谐振子(经典力学)}{SHO}。


\subsection{概念}


若一个物体受力所产生的加速度总是指向一个位置,且该加速度的大小正比于物体到这个位置的距离,我们就说这个物体结合它所受的力,构成了一个\textbf{简谐振子},加速度指向的位置被称为该简谐振子的\textbf{平衡位置}。

叫这个名字,首先是因为简谐振子会反复在平衡位置附近做周期性运动,而不会无限远离平衡位置;其次是因为这种周期性运动是最简单的形式,从而被冠以“简单而和谐”之名。我们首先看一些简谐振子的例子,再学着分析它们的运动。




\begin{example}{弹簧振子}

将轻质弹簧的一端固定,另一端连接在一个质量为$m$的物体上,物体可以沿着一个方向运动,弹簧也在这个方向上。此时弹簧和物体构成了一个简谐振子,我们可以称之为“弹簧振子”。

当弹簧处于原长时,物体不受弹簧的作用力,此时物体所在位置就是该简谐振子的平衡位置。当物体到平衡位置的距离为$x$时,物体受到大小为$kx$的拉力,方向指向平衡位置,其中$k$是弹簧的劲度系数。于是,物体的加速度总是指向平衡位置,且其大小正比于到平衡位置的距离:$a=\frac{k}{m}x$。

\end{example}


\begin{example}{单摆}

将长为$L$的轻质杆的一端固定,另一端连接在一个质量为$m$的物体上,让系统在匀强引力场作用下运动。该系统被称为一个“单摆”,它\textbf{不是}简谐振子,但可以近似认为是一个简谐振子。


\begin{figure}[ht]
\centering
\includegraphics[width=10cm]{./figures/8adf458a581996c7.pdf}
\caption{单摆示意图。单摆的一端固定在$\color{Red}O\color{Black}$点,另一端拴着质量为$\color{RoyalBlue}m\color{Black}$的物体,物体可在屏幕平面内运动。由于摆长恒为$L$,物体的运动轨迹是以$\color{Red}O\color{Black}$为圆心的一个圆。重力的方向竖直向下,故单摆处于平衡位置时,杆沿着图中虚线。当单摆偏离平衡位置,杆与平衡位置夹角为$\theta$时,物体所受合力$\color{Orange}\vec{F}$垂直于杆指向下方,大小为$mg\sin\theta$;由几何规律也可知,物体到平衡位置的\textbf{竖直}距离为$L-L\cos\theta$。} \label{fig_SHOhs_1}
\end{figure}


显然,当杆沿着重力方向(竖直方向)下垂的时候,单摆处于平衡位置。设在平衡位置时物体的速度大小为$v_0$,角速度为$\omega_0$;当杆偏离竖直方向的角度为$\theta$时,物体速度为$v_\theta$,角速度为$\omega_\theta$;此时物体比起平衡位置,高度提升了$(1-\cos\theta)L$;因此,根据机械能守恒,可知
\begin{equation}
\frac{1}{2}mv_0^2-\frac{1}{2}mv_\theta^2 = mg(1-\cos\theta)L~. 
\end{equation}
由此可得
\begin{equation}
v_\theta = \sqrt{v_0^2-2gL(1-\cos\theta)} ~. 
\end{equation}

\end{example}



















