% 三角函数(高中)
% 高中|三角函数
\pentry{角的概念(高中)\upref{HsAngl}}
\begin{issues}
\issueDraft
\end{issues}

\subsection{定义}
我们取单位圆上一点 $P(u,v)$,令 $OP$ 与 $x$ 轴夹角为 $\alpha$,则 
\begin{equation}
\begin{aligned}
\cos\alpha &= u,\\
\sin\alpha &= v,\\
\tan\alpha &= \frac{v}{u}
\end{aligned}
\end{equation}
易得,正弦函数和余弦函数的\textbf{定义域为全体实数},正切函数的定义域为 $\begin{Bmatrix}\alpha|\alpha \neq \frac{\pi}{2}+k\pi,k\in Z\end{Bmatrix}$

\subsection{性质}
我们把随自变量的变化呈周期性变化的函数叫作\textbf{周期函数},正周期中最小的一个称为\textbf{最小正周期}.

对于函数 $f(x)$ 为,如果存在非零实数 $T$,对定义域内的任意一个 $x$ 值,都有
\begin{equation}
f(x+T) = f(x)
\end{equation}
我们把 $f(x)$ 称为周期函数,$T$ 称为这个函数的周期.

正弦函数、余弦函数、正切函数的都是周期函数,根据定义易得,正弦函数和余弦函数,周期为 $2k\pi(k\in Z,k\neq0)$,正切函数的周期为 $k\pi(k\in Z,k\neq0)$.

\subsection{图像}
我们可以通过计算机绘制出函数图像
\begin{figure}[ht]
\centering
\includegraphics[width=14.25cm]{./figures/HsTrFu_1.png}
\caption{正弦函数} \label{HsTrFu_fig1}
\end{figure}
\begin{figure}[ht]
\centering
\includegraphics[width=5cm]{./figures/HsTrFu_2.png}
\caption{请添加图片描述} \label{HsTrFu_fig2}
\end{figure}