% 动量定理 动量守恒
% 动量守恒|动量定理|合外力|质点系
\pentry{动量\ 动量定理(单个质点)\upref{PLaw1}, 质点系\upref{PSys}}
\subsection{结论}
系统总动量的变化率等于合外力,所以合外力为零时系统总动量守恒.

\subsection{推导}
任何系统都可以看做质点系,质点系中第 $i$ 个质点可能受到系统内力 $\bvec F_i^{in}$ 或系统外力 $\bvec F_i^{out}$. 由单个质点的动量定理\upref{PLaw1},
\begin{equation}
\dv{t} \bvec p_i = \bvec F_i^{in} + \bvec F_i^{out}
\end{equation}
总动量的变化率为
\begin{equation}
\dv{\bvec P}{t} = \sum_i \dv{t} \bvec p_i  = \sum_i \bvec F_i^{in}  + \sum_i \bvec F_i^{out}
\end{equation}
由“质点系\upref{PSys}” 中的结论, 上式右边第一项求和是系统合内力, 恒为零. 于是我们得到系统的动量定理
\begin{equation}
\dv{\bvec P}{t} = \sum_i \bvec F_i^{out}
\end{equation}
可见当和外力(即等式右边)为零时, 动量 $\bvec P$ 不随时间变化, 也就是\textbf{动量守恒}.

\begin{example}{静止原子核的转变}
一个原来静止的原子核,经放射性衰变,放出一个动量为$9.22×10^{-16}{\rm g\cdot cm/s}$的电子,同时该核在垂直方向上又放出一个动量为$5.33×10^{-16}{\rm g\cdot cm/s}$的中微子.问蜕变后原子核的动量的大小和方向.

解:由于这个静止的原子核在蜕变的全过程中没有受到其他外力,所以对该原子核构成的系统,总动量守恒.即有
$$\bvec p_{\rm B}+\bvec p_{\rm e}+\bvec p_{\rm \nu}=0$$
即有
$$p_{\rm B}=|\bvec p_{\rm B}|=|-\bvec p_{\rm e}-\bvec p_{\rm \nu}|=\sqrt{p_{\rm e}^{2}+p_{\rm \nu}^{2}}=10.65×10^{-16}{\rm g\cdot cm/s}$$
$$\theta=\arctan\frac{5.33}{9.22}=30^\circ$$

(矢量图略)解毕.
\end{example}

\usepackage{pgf,tikz,pgfplots}
\pgfplotsset{compat=1.15}
\usepackage{mathrsfs}
\usetikzlibrary{arrows}
\pagestyle{empty}
\begin{document}
\definecolor{qqwuqq}{rgb}{0.,0.39215686274509803,0.}
\begin{tikzpicture}[line cap=round,line join=round,>=triangle 45,x=0.2051699860393919cm,y=0.3504105947133264cm]
\begin{axis}[
x=0.2051699860393919cm,y=0.3504105947133264cm,
axis lines=middle,
ymajorgrids=true,
xmajorgrids=true,
xmin=-5.0,
xmax=10.0,
ymin=-5.0,
ymax=5.0,
xtick={-4.0,-2.0,...,10.0},
ytick={-4.0,-2.0,...,4.0},]
\clip(-5.,-5.) rectangle (10.,5.);
\draw [shift={(0.,0.)},line width=2.pt,color=qqwuqq,fill=qqwuqq,fill opacity=0.10000000149011612] (0,0) -- (-30.:0.9806855639310398) arc (-30.:0.:0.9806855639310398) -- cycle;
\draw [->,line width=2.pt] (0.,0.) -- (-8.,0.);
\draw [->,line width=2.pt] (0.,0.) -- (2.,3.4641016151377544);
\draw [->,line width=2.pt,dash pattern=on 1pt off 1pt] (0.,0.) -- (8.,0.);
\draw [->,line width=2.pt] (0.,0.) -- (6.,-3.4641016151377553);
\draw [line width=2.pt,dash pattern=on 1pt off 1pt] (2.,3.4641016151377544)-- (8.,0.);
\draw [line width=2.pt,dash pattern=on 1pt off 1pt] (8.,0.)-- (6.,-3.4641016151377553);
\begin{scriptsize}
\draw[color=black] (-6.430170260982736,0.6732383897109115) node {$P_B$};
\draw[color=black] (1.5460723256563864,3.8768112318856343) node {$P_v$};
\draw[color=black] (5.59957265657135,-2.366886858475305) node {$P_e$};
\draw[color=qqwuqq] (1.7912437166391462,-0.16034433963047034) node {$\theta$};
\end{scriptsize}
\end{axis}
\end{tikzpicture}
\end{document}
