% 产生和湮灭
% 多粒子体系|全同粒子|Fock空间
\begin{issues}
\issueDraft
\end{issues}
旧称量子场论的工作是“二次量子化”,但这种描述是不准确的。不仅仅是因为两套体系几乎是同时发展的,更是因为在概念上,量子场论仅仅是对经典场论中的场进行一次量子化,使得场不再是描述幅值随时空变换的函数,而是能描述粒子产生和湮灭的算符。为了算符化,我们需要引入粒子产生和湮灭算符。这是量子场论中的基本概念,也是构筑多粒子体系不可或缺的砖瓦。


\subsection{定义}
用$\ket{0}$来定义能级最低,可视作没有粒子的真空态。用$ a, a^\dagger$来湮灭和产生粒子,即
\begin{equation}
\begin{aligned}
a_{\vec p}\ket{0}&=0\\
a^\dagger_{\vec p}\ket{0}&=\ket{\vec p}
\end{aligned}
\end{equation}
\subsection{传播子与对易关系}
\subsection{Fock空间}
\subsection{多粒子体系的算符}



