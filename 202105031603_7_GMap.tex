% 高斯映射
% Gauss map|形状算子
\pentry{可定向曲面\upref{OriSur}}

\subsection{高斯映射和形状算子}
\addTODO{高斯映射和形状算子的关系可能需要解释(形状算子是高斯映射的微分);自伴性的证明尚缺;Meusnier定理的证明或解释尚缺.}
\begin{definition}{高斯映射}
给定可定向曲面$S\subseteq\mathbb{R}^3$和其一个定向$N$.由于定向的值都是单位向量,因此$N$是一个$S\to S^2$的映射,称为$S$的一个\textbf{高斯映射(Gauss map)}.
\end{definition}

高斯映射是为了研究曲面的内蕴性质而诞生的,在现代微分几何中通常又改用\textbf{形状算子}来描述.我在此提一下形状算子的概念,作为补充.

\begin{definition}{形状算子}
给定流形(曲面)$S$上一点$p$,则$p$处的形状算子$L_p$是一个$T_pS\to T_pS$的一个映射.对于任意切向量$\bvec{v}\in T_pS$,取$\bvec{v}$对应的一条曲线$\alpha(t)$,都有$L_p(\bvec{v})=-\frac{\dd}{\dd t}N(\alpha(t))$,其中$N$是$S$上的一个定向.
\end{definition}


高斯映射的一个关键性质是\textbf{自伴(self-adjoint)},表述如下:

\begin{theorem}{高斯映射的自伴性}
给定高斯映射$N:S\to S^2$,且$\bvec{x}(u, v)$是$S$的一个局部坐标系,则有$N_u\cdot \bvec{x}_v=N_v\cdot \bvec{x}_u$.
\end{theorem}

这一点和形状算子的自伴性是等价的.

\begin{theorem}{形状算子的性质}
给定流形(曲面)$S$上一点$p$处的形状算子$L_p$,则对于$S$上任意切向量场$X, Y$,都有$L_p(X)\cdot Y=D_XY\cdot N$.
\end{theorem}

\subsection{曲率}

\begin{definition}{}
给定一个正则曲面$S$和其上一条正则曲线$C$.任取$p\in C$,记$k$为$C$在$p$处的曲率,$n$为$C$在$p$处的\textbf{单位}法向量,$N$为$S$在$p$处的单位法向量,则定义$k_n=kn\cdot N$为曲线$C$在曲面$S$上的\textbf{法曲率(normal curvature)}.
\end{definition}

法曲率和曲线本身的曲率不一定相同.比如说,考虑在纸面上画一个圆,然后把纸卷成圆柱,那么圆上总有两个点的法曲率为零,但是圆本身的曲率处处不为零.至于是哪两个点,你可以先发挥一下想象力,而接下来介绍的定理可以帮助你验证想象是否准确.

事实上,法曲率的值和曲线本身关系不是特别大,只需要知道曲线在一点处的切线,就可以唯一确定其法曲率了.这一点被表述为以下定理.

\begin{theorem}{Meusnier定理}
给定正则曲面$S$和其上一点$p$,如果$C_1$和$C_2$是两条$S$上过$p$的曲线,且在$p$点处二者有相同的切线,那么二者的法曲率相同.
\end{theorem}

\begin{definition}{主曲率}
给定正则曲面$S$和其上一点$p$,由Meusnier定理,该点处每一个切线方向都唯一对应一个法曲率值.这些法曲率中的最大和最小值被称为$S$在点$p$处的\textbf{主曲率(principal curvature)},对应的方向则是\textbf{主方向(principal direction)}.
\end{definition}

\begin{definition}{曲率曲线}
如果曲面上一条曲线处处都是主方向,那么称其为曲面上的一条\textbf{曲率曲线(line of curvature)}.\footnote{Olinde Rodrigues定理给出了曲率曲线的充要条件.}
\end{definition}

\begin{definition}{高斯曲率与平均曲率}
设曲面$S$在$p$处的主曲率为$k_1$和$k_2$,那么称$K=k_1k_2$为曲面在该点处的\textbf{高斯曲率(Gaussian Curvature)},$\frac{k_1+k_2}{2}$为曲面在该点处的\textbf{平均曲率(mean curvature)}.
\end{definition}

高斯曲率实际上就是形状算子的\textbf{行列式},而平均曲率是形状算子\textbf{迹的一半}.

高斯曲率和形状算子还引出了对曲面上的点的一种分类:

\begin{definition}{}
对于曲面$S$上的一点$p$,我们有:
\begin{itemize}
\item 如果$K$
\end{itemize}
\end{definition}
