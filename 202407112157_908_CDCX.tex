% 超对称性
% license CCBYSA3
% type Wiki

(本文根据 CC-BY-SA 协议转载自原搜狗科学百科对英文维基百科的翻译)

在粒子物理学中,超对称性(SUSY)是一个原理,它提出了基本粒子之间两种基本类别的关系:玻色子具有整数值自旋和费米子具有半整数自旋。[1][2] 超对称性是时空对称的一种类型,是未被发现的粒子物理学的一个可能的候选者,如果被证实是正确的,它被认为是当前粒子物理学中许多问题的一个优秀的解决方案,可以解决当前理论被认为是不完整的各个领域。对标准模型的超对称扩展将通过保证扰动理论中所有阶的二次发散被抵消来解决规范理论中的主要层次问题。

在超对称性中,来自一组中的每一个粒子在另一组中都有一个关联的粒子,称为它的超对称伙伴,其自旋相差半个整数。这些超对称伙伴将是新的未被发现的粒子。例如,有一种粒子叫做“超电子”(超伙伴电子),是电子的玻色子伙伴。在最简单的超对称性理论中,由于完美的“不间断”超对称性,每对超对称伙伴有相同的质量和内部量子数。因为我们希望用现在的设备找到这些“超对称伙伴”,如果超对称性存在,那么它由自发破缺的对称性组成,允许超对称伙伴在质量上有所不同。[3][4][5] 自发破缺超对称性可以解决粒子物理中的许多神秘问题,包括等级问题。

目前没有证据表明超对称性是否正确,或者对当前模型的其他扩展可能更准确。从某种程度上说,这是因为自大约2010年,专门为研究标准模型之外的物理而设计的粒子加速器才开始运行,而且还不知道具体在哪里寻找,也不知道成功搜索所需的能量。

物理学家支持超对称性的主要原因是,目前的理论已知是不完整的,而且它们的局限性是公认的,超对称性将是一些主要问题的有吸引力的解决方案。直接确认将需要在对撞机实验中生产超对称伙伴,例如大型强子对撞机(LHC)。大型强子对撞机(LHC)的第一次运行除了希格斯玻色子之外没有发现以前未知的粒子,希格斯玻色子已经被怀疑是标准模型的一部分,因此也没有超对称性的证据。[6][7] 间接方法包括在已知的标准模型粒子中寻找永久电偶极矩(EDM),这可能发生在标准模型粒子与超对称粒子相互作用时。目前对电子电偶极矩的最佳约束是使其小于10-28 e cm,相当于对TeV标度下新物理的灵敏度,与目前最好的粒子对撞机匹配。[8] 任何基本粒子中的永久电偶极矩都指向违反时间反转的物理现象,因此由CPT定理指向CP不对称性。这种电偶极矩实验也比传统的粒子加速器具有更大的可扩展性,并且随着加速器实验变得越来越昂贵和维护起来越来越复杂,它为检测超出标准模型的物理成分提供了一种实用的替代方法。

这些发现让许多物理学家失望,他们认为超对称性(以及其他依赖它的理论)是迄今为止“新”物理学最有希望的理论,并希望这些实验能有意想不到的结果。[9][10] 前热情支持者米哈伊尔·希夫曼(Mikhail Shifman)甚至敦促理论界寻找新的想法,并接受超对称性是一个失败的理论。[11] 然而,也有人认为这种“自然”危机还为时过早,因为各种计算对质量极限过于乐观,而质量极限将得到基于超对称性的解决方案。[12][13]

\subsection{动机}
超对称性在电弱尺度附近有许多现象学动机,以及任何尺度的超对称性的技术动机。
\subsubsection{1.1 等级问题}
接近电弱尺度的超对称性改善了困扰标准模型的等级问题。[14] 在标准模型中,电弱尺度得到巨大的普朗克尺度量子修正。观察到的弱电标度和普朗克标度之间的等级必须通过非常精细的调谐来实现。另一方面,在超对称理论中,在伙伴和超伙伴之间的普朗克尺度的量子校正相互抵消(由于费米子环有负号)。弱电标度和普朗克标度之间的等级是以自然的方式实现的,没有奇迹般的调谐。
\subsubsection{1.2 规范耦合统一}
规范对称群在高能量时统一的观点被称为大统一理论。然而,在标准模型中,弱耦合、强耦合和电磁耦合在高能量时无法统一。在超对称理论中,对规范耦合的运行进行修改,实现了规范耦合的精确高能量统一。这种改进的运行也为辐射电弱对称性破缺提供了一种自然机制。
\subsubsection{1.3 暗物质}
TeV尺度的超对称性(用离散对称性增强)通常以与热遗迹丰度计算一致的质量尺度提供候选暗物质粒子。[15][16]
\subsubsection{1.4 其他技术动机}
超对称性也是由几个理论问题的解决方案驱动的,这些解决方案通常提供许多理想的数学性质,并确保在高能量下的合理行为。超对称量子场论通常更容易分析,因为更多的问题在数学上变得容易处理。当超对称性被强加为局部对称性时,爱因斯坦的广义相对论被自动包含在内,其结果被称为超重力理论。这也是万物理论、超弦理论和SUSY理论中最受欢迎的候选者的一个必要特征,它可以解释宇宙膨胀的问题。

超对称性的另一个在理论上吸引人的特性是,它为科尔曼-满都拉定理(Coleman–Mandula)提供了唯一的“漏洞”,该定理禁止时空和内部对称性以任何非平凡的方式结合,因为像标准模型这样的量子场理论都具有非常一般假设。哈格-奥普萨斯基-索赫纽斯(Haag–Łopuszański–Sohnius)定理证明了超对称性是时空和内部对称性能够一致结合的唯一方式。[17]

\subsection{历史}
在强子物理学的背景下,宫泽弘毅( Hironari Miyazawa)于1966年首次提出了介子和重子之间的超对称性。这种超对称性不涉及时空,也就是说,它与内部对称性有关,并且被严重破坏。宫泽的工作在当时很大程度上被忽视了。[18][19][20][21]

J. L. Gervais和B. Sakita(在1971年),[22] Yu. A. Golfand和E. P. Likhtman(也是在1971年)和D. V. Volkov和V. P. Akulov (1972年),[23] 在量子场论的背景下独立地重新发现了超对称性,这是时空和基本场的一种全新类型的对称性,它建立了不同量子性质的基本粒子、玻色子和费米子之间的关系,并统一了时空和微观现象的内部对称性。1971年,皮埃尔·雷蒙( Pierre Ramond)、约翰·施瓦兹(John H. Schwarz)和安德烈·奈芙( André Neveu)在弦论早期版本的背景下,首次出现了超对称性,这种超对称性具有一致的李代数分级结构,[24] 而热尔维斯·萨基塔(Gervais−Sakita)再发现正是基于这种结构。

最后,朱利叶斯·韦斯(Julius Wess)和布鲁诺·祖米诺(Bruno Zumino)(1974年)[25] 确定了四维超对称场论的特征重整化特征,并将其定义为显著的量子场论,并且他们和阿卜杜勒·萨拉姆(Abdus Salam )及其同事介绍了早期粒子物理学的应用。超对称性的数学结构(分级李超代数)随后被成功地应用于其他物理课题,从核物理、[26][27] 临界现象、[28] 量子力学到统计物理。它仍然是许多物理学理论的重要组成部分。

标准模型的第一个现实超对称版本是由皮埃尔·费耶特(Pierre Fayet)在1977年提出的,被称为最小超对称标准模型,简称MSSM。它的目的之一是解决等级问题。

\subsection{应用}
\subsubsection{3.1 可能对称群的扩展}
物理学家探索超对称性的一个原因是,它为量子场论中更常见的对称性提供了一个延伸。这些对称性分为庞加莱(Poincaré)群和内对称性,并且科尔曼-满都拉(Coleman–Mandula)定理表明,在某些假设下,S矩阵的对称性必须是具有紧内对称群的庞加莱(Poincaré)群的直接乘积,或者如果没有任何质量间隙,则是具有紧内对称群的共行群的直接乘积。1971年,高尔夫(Golfand)和利赫曼(Likhtman)首次证明庞加莱(Poincaré)代数可以通过引入四个反交换自旋发生器(四维)来扩展,这后来被称为超电荷。1975年,哈格-洛佩斯赞斯基-索赫纽斯(Haag–Lopuszanski–Sohnius)定理分析了所有可能的一般形式的超代数,包括那些具有大量超发生器和中心电荷的超代数。这个扩展的超庞加莱(Poincaré)代数为获得一大类非常重要的超对称场论铺平了道路。

\textbf{超对称性代数}

传统的物理学的对称性是由庞加莱(Poincaré)群的张量表示和内部对称性转换的物体产生的。然而,超对称性是由通过自旋表示变换的物体产生的。根据自旋统计定理,玻色子场交换而费米子场反交换。将这两种场组合成一个代数需要引入$Z_2$分级,在$Z_2$分级下玻色子是偶元素,费米子是奇元素。这样的代数叫做李超代数。

庞加莱(Poincaré)代数最简单的超对称扩展是超庞加莱(Poincaré)代数。用两个Weyl自旋表示,具有以下反交换关系:

