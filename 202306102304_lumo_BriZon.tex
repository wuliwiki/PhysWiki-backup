% 布里渊区
% 第一布里渊区|布里渊区

\begin{issues}
\issueDraft
\end{issues}

\pentry{布拉伐格子、基元与原胞\upref{BraLat},倒格空间\upref{RecLat}}
\subsection{第一布里渊区(Brillouin zone)}
通常取 Wigner- Seitz 原胞为第一布里渊区,它可以反映倒格点阵的对称性:
\begin{figure}[ht]
\centering
\includegraphics[width=14.25cm]{./figures/e54eded0edfb3563.png}
\caption{通常取$\bvec \Gamma$ 为 $\bvec G = 0$点} \label{fig_BriZon_1}
\end{figure}
