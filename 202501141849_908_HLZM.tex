% 阿尔-花拉子米(综述)
% license CCBYSA3
% type Wiki

本文根据 CC-BY-SA 协议转载翻译自维基百科\href{https://en.wikipedia.org/wiki/Al-Khwarizmi}{相关文章}。
\begin{figure}[ht]
\centering
\includegraphics[width=6cm]{./figures/6cab70e0324244c5.png}
\caption{20世纪木刻版画,描绘阿尔-花拉子米} \label{fig_HLZM_1}
\end{figure}
穆罕默德·伊本·穆萨·阿尔-花拉子米(约780年–约850年),或简称阿尔-花拉子米,是一位波斯学者,致力于数学、天文学和地理学的研究,并在阿拉伯语学术领域产生了深远的影响。约在公元820年,他在巴格达的智慧之家工作,该地是阿拔斯王朝的首都。

他在代数领域的代表性著作《阿尔-贾布尔》(《计算的简明书》)编写于813年至833年之间,提出了线性方程和二次方程的系统解法。他在代数方面的成就之一是通过完成平方法来解二次方程,并为此提供了几何证明。由于阿尔-花拉子米是第一个将代数作为独立学科来研究的人,并且引入了“简化”和“平衡”方法(将减去的项转移到方程的另一边,即在方程两边相同项的消去),因此他被誉为代数的奠基人或创始人。英语中的“algebra”一词就来源于他那本著作的简写标题(阿尔-贾布尔,意为“完成”或“重新联合”)。他的名字还衍生出了英语中的“algorism”和“algorithm”以及西班牙语、意大利语和葡萄牙语中的“algoritmo”;以及西班牙语中的“guarismo”和葡萄牙语中的“algarismo”,均表示“数字”。

12世纪时,阿尔-花拉子米关于印度算术的教科书《印度人数字算法》(Algorithmo de Numero Indorum)的拉丁文翻译,规范了印度数字系统,并将基于十进制的位值计数系统引入西方世界。同样,阿尔-贾布尔由英学者罗伯特·查斯特在1145年翻译成拉丁文,并一直作为欧洲大学的主要数学教材,直到16世纪。

阿尔-花拉子米修订了公元2世纪由罗马学者克劳狄乌斯·托勒密撰写的《地理学》,列出了城市和地方的经纬度。他还制作了一套天文表格,并撰写了关于历法的作品,以及关于天文仪器和日晷的研究。阿尔-花拉子米在三角学方面也作出了重要贡献,编制了准确的正弦和余弦表,并制作了第一个正切表。
\subsection{生活}
\begin{figure}[ht]
\centering
\includegraphics[width=6cm]{./figures/27f5f3869a4239e7.png}
\caption{马德里的大学城(Ciudad Universitaria)中穆罕默德·伊本·穆萨·阿尔·胡瓦里兹米的纪念碑} \label{fig_HLZM_2}
\end{figure}
很少有关于穆罕默德·伊本·穆萨·阿尔·胡瓦里兹米(al-Khwarizmi)生活的确切细节。伊本·纳迪姆(Ibn al-Nadim)将他的出生地定为胡瓦尔兹姆(Khwarazm),他通常被认为来自这个地区。作为波斯人,他的名字意味着“来自胡瓦尔兹姆”,这个地区曾是大伊朗的一部分,现在属于土库曼斯坦和乌兹别克斯坦。

塔巴里(al-Tabari)给出的名字是穆罕默德·伊本·穆萨·阿尔·胡瓦里兹米·阿尔·马久西·阿尔·库特鲁布布利(Muḥammad ibn Musá al-Khwārizmī al-Majūsī al-Qūṭrubbullī)。其中的“阿尔·库特鲁布布利”(al-Qutrubbulli)可能表示他来自巴格达附近的库特鲁布尔(Qutrubbul)。然而,罗什迪·拉谢德(Roshdi Rashed)对此表示否认,认为这只是早期手稿的错误。在另一种看法中,大卫·A·金(David A. King)认为他是来自库特鲁布尔地区,因为他被称为“阿尔·胡瓦里兹米·阿尔·库特鲁布布利”,可能是因为他出生在巴格达附近。

关于他的宗教信仰,托默(Toomer)写道,塔巴里给他的另一个称号“阿尔·马久西”(al-Majūsī)似乎表明他是古老的琐罗亚斯德教信徒。这种信仰在那个时代对于伊朗裔人士来说依然有可能,但胡瓦里兹米的《代数》序言表明他是正统的穆斯林,因此“阿尔·马久西”这一称号可能仅仅意味着他的祖先,甚至可能是他年轻时曾是琐罗亚斯德教徒。

伊本·纳迪姆的《历史大典》(Al-Fihrist)中有简短的胡瓦里兹米传记及其著作清单。胡瓦里兹米的工作主要集中在813至833年之间。穆斯林征服波斯后,巴格达成为了科学研究和贸易的中心。大约820年,他被任命为天文学家,并成为智慧之宫(House of Wisdom)图书馆的馆长。智慧之宫是由阿拔斯哈里发阿尔·马蒙(al-Ma'mūn)创立的。胡瓦里兹米研究了包括翻译希腊文和梵文科学手稿在内的各种科学和数学内容。他还是一位历史学家,曾被塔巴里等人引用。

在阿尔·瓦西克(al-Wathiq)统治时期,他据说参与了两次使节任务,其中之一是前往可萨(Khazars)。道格拉斯·莫顿·邓洛普(Douglas Morton Dunlop)建议,穆罕默德·伊本·穆萨·阿尔·胡瓦里兹米可能与穆罕默德·伊本·穆萨·伊本·沙基尔(Muḥammad ibn Mūsā ibn Shākir),即三兄弟班努·穆萨(Banū Mūsā)中的长子是同一个人。
\subsection{贡献}
\begin{figure}[ht]
\centering
\includegraphics[width=6cm]{./figures/dfca5e36d9defc42.png}
\caption{《阿尔·胡瓦里兹米的代数》中的一页} \label{fig_HLZM_3}
\end{figure}
阿尔·胡瓦里兹米在数学、地理、天文学和制图学方面的贡献为代数和三角学的创新奠定了基础。他系统化的线性和二次方程求解方法促成了代数的诞生,而“代数”这一词汇源自他关于这一主题的著作《Al-Jabr》(《完备与平衡计算书》)。

大约在820年写成的《使用印度数字的计算方法》在中东和欧洲传播了印度-阿拉伯数字系统。当该书在12世纪被翻译成拉丁文,名为《Algoritmi de numero Indorum》(阿尔·胡瓦里兹米关于印度算术的著作)时,“算法”这一术语被引入西方世界。

他的一些工作基于波斯和巴比伦天文学、印度数字和希腊数学。

阿尔·胡瓦里兹米对托勒密关于非洲和中东的资料进行了系统化和修正。另一部重要著作是《Kitab surat al-ard》(《地球的图像》;翻译为《地理学》),其中给出了基于托勒密《地理学》中的坐标值,改进了地中海、亚洲和非洲的值。

他还撰写了有关天文仪器如天体仪和日晷的书籍。他参与了一项测定地球周长的项目,并为阿尔·马蒙哈里发绘制了一张世界地图,监督了70位地理学家的工作。当他的著作通过拉丁文翻译在12世纪传播到欧洲时,对欧洲数学的发展产生了深远的影响。


《代数学》(Al-Jabr*,阿拉伯文:الكتاب المختصر في حساب الجبر والمقابلة,*al-Kitāb al-mukhtaṣar fī ḥisāb al-jabr wal-muqābala*)是一本大约在公元820年左右写成的数学书籍。该书在哈里发阿尔·马蒙的鼓励下编写,作为一本关于计算的流行著作,书中充满了许多示例和应用,涵盖了贸易、测量和法律继承等问题。[49] “代数”一词来源于该书中描述的基本方程操作之一(*al-jabr*,意为“恢复”,指的是在方程两边加上一个数以合并或取消项)。该书由罗伯特·查尔斯特(Robert of Chester)于1145年将其翻译为拉丁文《代数与相互平衡之书》(*Liber algebrae et almucabala*),因此形成了“代数”这一术语。杰拉德·克雷莫纳(Gerard of Cremona)也进行过翻译。一本独特的阿拉伯文手稿保存在牛津大学,并由F. 罗斯恩(F. Rosen)于1831年翻译。一份拉丁文翻译本保存在剑桥大学。[50]

该书提供了关于解决二次方程的详尽说明,并讨论了“还原”和“平衡”的基本方法,指的是将方程的项移到方程的另一边,即在方程两边取消相同的项。[51][52]