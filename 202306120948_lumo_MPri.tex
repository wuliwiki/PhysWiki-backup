% 莫培督原理
% 莫培督原理|保守系统
\pentry{端点可变的作用量\upref{EPAct}}
在拉格朗日函数不显含时间时,物理系统的能量守恒。此时,可以建立作用量原理的更简单的形式,这一简单形式的作用量原理就称为\textbf{莫培督原理}。

莫培督原理可描述为:满足能量守恒的系统的演化(真实运动)是使得
\begin{equation}
S_0=\int\sum_i p_i\dd q^i
\end{equation}
取极值的曲线,称 $S_0$ 为\textbf{简约作用量}。
\subsection{推导}
在拉氏量不显含时间时,系统能量守恒
\begin{equation}\label{eq_MPri_1}
H(p,q)=E=\text{常数}~.
\end{equation}
设系统初末位置不变,仅末时刻 $t$ 可变,那么由\autoref{eq_EPAct_1}~\upref{EPAct}
\begin{equation}\label{eq_MPri_2}
\delta S=-H\delta t~,
\end{equation}
\autoref{eq_MPri_1} 代入\autoref{eq_MPri_2} 有
\begin{equation}\label{eq_MPri_4}
\delta S+E\delta t=0~.
\end{equation}
由于作用量 $S$ 可写为
\begin{equation}
S=\int \int\sum_i p_i\dd q^i-E(t-t_0)~,
\end{equation}
记
\begin{equation}\label{eq_MPri_6}
S_0=\int\sum_i p_i\dd q^i~.
\end{equation}
注意到 $E$ 为常数,于是
\begin{equation}\label{eq_MPri_3}
\delta S=\delta S_0-E\delta t~,
\end{equation}
\autoref{eq_MPri_3} 代入\autoref{eq_MPri_4} ,得到
\begin{equation}\label{eq_MPri_7}
\delta S_0=0~.
\end{equation}
\subsection{应用}
利用莫培督原理,可以确定系统的轨道,这可以通过下面看到。

能量 $E$ 是 $q,\dv{q}{t}$ 的函数 
\begin{equation}
E(q,\dv{q}{t})=E~.
\end{equation}
由该方程可以用 $q,\dd q,E$ 来表达 $\dd t$,将 $\dd t$ 代入动量的定义:
\begin{equation}\label{eq_MPri_5}
p_i=\pdv{}{\dot q^i}L\qty(q,\dv{q}{t})~.
\end{equation}
\autoref{eq_MPri_5} 代入 \autoref{eq_MPri_6} ,这时 $S_0$ 中的变量就仅仅是 $q,\dd q, E$ 。那么由\autoref{eq_MPri_7} 就能确定系统的轨道。