% 人工智能导论
% keys 人工智能|机器学习|深度学习
% license Usr
% type Tutor

本文旨在为人工智能部分后续的词条建立基本的概念和总览的图景。

人工智能(Artificial Intelligence,简称AI)是一门研究如何使计算机具有智能行为的科学与技术。它涵盖了一系列的技术、方法和应用,旨在使计算机系统能够模拟、理解和执行人类智能的各种任务。人工智能领域的发展已经走过了几个阶段,从最初的符号主义到现代的机器学习和深度学习。

\subsection{符号主义时代}

人工智能的最初阶段可以追溯到20世纪50年代和60年代,这一时期被称为符号主义时代。研究人员试图通过使用符号和规则来模拟人类智能。这些系统基于专家系统,其中包含了领域专家提供的知识,以解决特定类型的问题。然而,符号主义在处理复杂的、模糊的问题时面临困难,导致了人工智能研究的新方向的产生。

比如,基于规则推理(Rule Base Reasoning,RBR)的方法是一种将专家所掌握的知识和经验转化为规则,通过启发式推理进行推理解的技术。这种方法在解决问题时根据明确的前提条件产生明确的结果,使其推理过程相对清晰。举例来说,对动物的分类规则可以通过IF-THEN语句表示,从而实现对动物种类的判定,如老虎或企鹅。

基于规则的专家系统是早期专家系统的代表,其推理过程相对明确,规则正确时可以得到较为准确的结论。这使得基于规则的专家系统成为一种简单实用、广泛应用的专家系统。尤其对于特定领域的问题,基于规则的方法表现出色,成为解决实际问题的有效手段。

然而,基于规则的专家系统也存在一些缺点。首先,规则的构造高度依赖于专家的经验积累,如果专家的经验不准确,则系统的结果也可能不准确。其次,这类专家系统缺乏自学习能力,更新迭代需要专家经验的不断积累。虽然这两个缺点存在,但从系统开发的角度来看,专家系统是一个持续迭代优化的过程。在实际应用中,不希望专家系统因为自学习而导致结果不可预测的情况,因此依赖专家的经验积累是一个相对可控的方向。