% Linux 包管理笔记(apt, dpkg, snap)

\begin{issues}
\issueDraft
\end{issues}

\subsection{Apt}
\begin{itemize}
\item \verb|apt-show-versions 包1 包2| 可以查看已安装包的版本
\item \verb|apt show 包| 显示包的详细信息, 包括依赖(可以是存在但没有安装的包)
\item \verb|apt policy 包| 显示包的来源
\item \verb|apt-cache rdepends 包| 显示它目前被谁依赖
\item \verb|apt remove 包| 不会清除它依赖的包, 但是会清除依赖它的包!
\item \verb|apt autoremove| 清除没有被依赖的, 被自动安装的包, 比如 \verb|apt remove| 了一个包以后, 就用这个清干净. 也可以 \verb|apt autoremove 包|, 相当于卸载包后再 \verb|autoremove|.
\item 当 apt 出现 conflict 的时候, \verb|aptitude| 通常能提供更详细的信息
\item \verb|apt|, \verb|aptitude| 和 \verb|synaptic| 都使用 \verb|dpkg| 作为后端, 所以它们不会冲突
\item \verb|aptitude| 遇到冲突时会提供多种解决方案(卸载哪些包,安装哪些,升级哪些), 手动选择合适的即可. 这是解决 \verb|you have held broken packages| 的常见办法.
\item \verb|/var/lib/apt/extended_states| 文件储存了哪些包是自动安装的哪些是手动安装的
\item 列出所有手动安装的包 \verb`comm -23 <(apt-mark showmanual | sort -u) <(gzip -dc /var/log/installer/initial-status.gz | sed -n 's/^Package: //p' | sort -u)`
\item 列出所有没有被依赖的包(无论是怎么安装的)(参考\href{https://askubuntu.com/questions/1114733/how-do-i-list-all-packages-that-no-package-depends-on}{这里}) \verb`dpkg-query --show --showformat='${Package}\t${Status}\n' | tac | awk '/installed$/ {print $1}' | xargs apt-cache rdepends --installed | tac | awk '{ if (/^ /) ++deps; else if (!/:$/) { if (!deps) print; deps = 0 } }'`. 注意这里面有一些重要的系统包(包括 linux 内核)如果不是自己装的包要删除时要小心.
\item 或者 \verb`comm -23 <(aptitude search '~i !~M' -F '%p' | sed "s/ *$//" | sort -u) <(gzip -dc /var/log/installer/initial-status.gz | sed -n 's/^Package: //p' | sort -u)` 参考\href{https://askubuntu.com/questions/2389/how-to-list-manually-installed-packages}{这里}.
\item \verb|apt --installed list| 列出所有安装的包
\item \verb|apt/dpkg| 安装时不会覆盖文件, 而是会给出警告并停止安装.
\item \verb|apt install ... -y| 可以自动全选 yes
\item \verb|apt install -f| 可以修复依赖
\item 第三方的 source 保存在 \verb|sudo sh -c 'echo "deb http://apt.postgresql.org/pub/repos/apt $(lsb_release -cs)-pgdg main" > /etc/apt/sources.list.d/pgdg.list'|
\item \verb|apt upgrade| 自动更新所有包, 但并不删除已存在的包, \verb|apt dist-upgrade| 或者等效的 \verb|apt full-upgrade| 更激进, 会删除已有的包, 可能会更新内核, 可能造成一些破坏.
\verb|apt download 包| 可以不安装只下载某个包的 deb (但不会下载它的依赖!)
\item 事实上, \verb|apt install| 命令也是先下载所有需要的以及依赖的 deb 包然后再安装, 这些 deb 包存在 \verb|/var/cache/apt/archives| 目录中.
\item \verb|apt clean| 和 \verb|apt autoclean| 可以清除缓存(例如下载的 deb 安装包等, 后者清的没那么干净)
\end{itemize}

\addTODO{如何指定某个文件夹为 apt 的 source? cuda 安装好像就是这么做的}

\subsubsection{配置 source}
\begin{itemize}
\item 例如在 debian 中, 如果 \verb|apt update| 显示 public key 的问题, 就用
\item \verb|sudo apt-key adv --keyserver keyserver.ubuntu.com --recv-keys DCC9EFBF77E11517|, 其他系统替换成不同网址即可
\end{itemize}

\subsubsection{apt-clone}
\begin{itemize}
\item 一个\href{https://ubunlog.com/en/apt-clone-copia-seguridad-paquetes/}{教程}.
\item \verb|sudo apt-clone clone --with-dpkg-repack 文件夹| 可以把本机所有安装包备份到文件夹中的一个压缩文件.
\item \verb|apt-clone info 备份文件.gz| 可以查看包的信息
\item \verb|sudo apt-clone restore 备份文件.gz| 可以进行恢复(注意一定要相同架构相同 ubuntu 版本)
\item 直接解压备份文件即可得到所有备份的 deb, 可单独用 \verb|sudo apt install xxx.deb| 安装
\item \verb|sudo apt-clone restore 备份文件.gz --destination 某文件夹| 把包安装到某个文件夹(使用 chroot).
\end{itemize}



\subsection{dpkg}
\begin{itemize}
\item \verb`dpkg -l | grep -i 关键字` 可以查找已安装的某个包
\item \verb|sudo dpkg -i 名字.deb| 安装 deb 安装包, 其中 \verb|名字| 可以带 \verb|*|. 完了以后用 \verb|sudo apt install -fy| 安装缺失的依赖包(dpkg 不自动安装依赖).
\item 其实也可以直接用 \verb|sudo apt install *.deb|
\item \verb|sudo apt remove 包| 或者 \verb|dpkg --remove 包| 卸载, \verb|dpkg --purge 包| 可以连配置文件一起卸载.
\item \verb|dpkg -S 文件| 查看该文件是哪个包安装的. 例如 \verb|dpkg -S /bin/gcc|
\item \verb|dpkg -f 文件名.deb field名| 查看 deb 的某个属性, 如果省略 \verb|field名| 则查看所有属性
\item 要卸载某个 \verb|.deb| 包中的所有包, 用 \verb|sudo dpkg -r $(dpkg -f 文件名.deb Package)| (不支持 \verb|*|)
\end{itemize}

\subsubsection{dpkg-deb}
\begin{itemize}
\item \verb|dpkg-deb| 用于操作 deb 安装包(Debian package archive)
\item \verb|dpkg-deb -I xxx.deb| 查看信息
\item \verb|dpkg-deb -c xxx.deb| 列出所有要安装的文件和路径. 这和安装完以后用 \verb|dpkg -L 包名| 列出的文件一样.
\item deb 文件其实就是一个压缩包, 里面包含所有需要安装的文件, 以及一些配置文件.
\end{itemize}


\subsection{Snap}
\begin{itemize}
\item 参考\href{https://www.howtogeek.com/660193/how-to-work-with-snap-packages-on-linux/}{这篇介绍}: snap 基本就是一些微小的 container 每个都包含所有的依赖文件, 使用虚拟硬盘以解决不同的包使用同一个 dependency 的不同版本.
\item \verb|snap list| 列出所有的安装包(注意 Ubuntu Software 中安装的未必都是 snap)
\item \verb|snap find 关键词| 搜索包
\item 或者到 snap 官网, 查找软件, 然后按 install 就会得到安装命令, 例如 \verb|snap install code --classic| 安装 vscode
\item 如果不加 \verb|--classic|, 会出现如下错误 \verb|error: This revision of snap "code" was published using classic confinement and thus may perform arbitrary system changes outside of the security sandbox that snaps are usually confined to, which may put your system at risk. If you understand and want to proceed repeat the command including --classic.|
\item \verb|snap version| 查看版本
\item \verb|snap remove| 删除包
\item \verb|snap run 软件名| 运行某个软件, 也可以直接 \verb|软件名| 运行(在目录 \verb|/usr/snap/bin| 下)
\end{itemize}

\subsection{PPA}
\begin{itemize}
\item 参考\href{https://itsfoss.com/ppa-guide/}{这个}.
\item \verb|/etc/apt/sources.list| 文件会列出软件 repo 的网址
\item 添加 ppa: \verb|sudo add-apt-repository ppa:dr-akulavich/lighttable|. 然后就可以 \verb|update| 和 \verb|install| 了.
\item 这些 ppa 需要在 \href{https://launchpad.net/}{Launchpad} 注册.
\item 添加 ppa 并不会改变 \verb|sources.list|, 而是会在 \verb|/etc/apt/sources.list.d| 添加两个文件, 一个 \verb|.list| 文件, 一个 \verb|.list.save|
\item 添加的网址是 \href{http://ppa.launchpad.net/dr-akulavich/lighttable/ubuntu/}{http://ppa.launchpad.net/dr-akulavich/lighttable/ubuntu/}
\item 有一些 deb 安装包同样会添加 sources.list, 这样以后就可以方便更新了.
\item \verb|Synaptic| 软件可以很方便地查看哪些包是 PPA 安装的, 以及移除 \verb|PPA|.
\end{itemize}
