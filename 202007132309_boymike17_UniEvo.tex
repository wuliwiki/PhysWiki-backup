% 宇宙的演化


宇宙演化可以大约分为两个时期(见表(\ref{tab1})),从创世之初($t=0$) 到创世后3分钟我们可以称为前物质时期,这个时期可以大致分为两个时期:从宇宙诞生后到$10^{-43}$秒,此时宇宙沐浴于$10^{18}$GeV的高温中,四大相互作用被统一在一起.随后到$10^{-34}$秒宇宙在高温中经历暴涨(Inflation),宇宙尺度因子迅速扩大(见图.1).在暴涨快结束时超对称破缺,宇宙中的重元素开始产生.


从$10^{-10}$ 秒到宇宙诞生后3分钟,宇宙暴涨结束,宇宙从$1$ TeV 迅速逐渐冷却到 $0.1 $ MeV.在此期间暴涨所释放的能量把宇宙重新加热,宇宙中的电弱相互作用开始分离,夸克结合成强子,中子冷却下来,宇宙中的核子开始在相互作用下结合并形成,称为原核初合成(BBN);中微子在创世后1秒与其他核子退耦并此后不再与其他物质产生相互作用,一直传播到现在.此时,宇宙开始以辐射为主导,我们也称为辐射主导时期的开始,此外,宇宙中的原初扰动和引力波开始形成.

从创世后3分钟到现在,我们可以称为后物质时期.我们也可以把这个时期大致地分为两个时期:从宇宙诞生后3分钟到380.000年,宇宙一直以辐射为主导.随着宇宙的膨胀光子和物质从密度对等逐渐分离,且光子、物质和原初引力波进行充分的相互作用(recombination),光子逐渐冷却下来形成宇宙辐射背景(CMB),期间暗物质开始形成.

从宇宙诞生后380.000年到现在,宇宙进入物质主导时期,另一部分光子形成了宇宙辐射背景并随着时间演化到现在,一部分光子和原初引力波相互作用后成为B-模极化光子; 一部分原初引力波和中微子并不参与到相互作用中,并随着时间演化直到现在.在此期间,各大星系形成于创世后$10^8$ 年之后,太阳系形成于$8\times 10^9$年.

\begin{figure}[ht]
\centering
\includegraphics[width=10cm]{./figures/UniEvo_1.png}
\caption{请添加图片描述} \label{UniEvo_fig1}
\end{figure}