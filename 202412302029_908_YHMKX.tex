% 约翰·麦卡锡(综述)
% license CCBYSA3
% type Wiki

本文根据 CC-BY-SA 协议转载翻译自维基百科\href{https://en.wikipedia.org/wiki/Maxwell\%27s_equations}{相关文章}。

\begin{figure}[ht]
\centering
\includegraphics[width=6cm]{./figures/b71f34967eb60ef0.png}
\caption{麦卡锡在2006年的一次会议上} \label{fig_YHMKX_1}
\end{figure}
约翰·麦卡锡(John McCarthy,1927年9月4日-2011年10月24日)是一位美国计算机科学家和认知科学家。他是人工智能学科的创始人之一。[1] 他共同撰写了提出“人工智能”(Artificial Intelligence, AI)这一术语的文献,开发了编程语言家族Lisp,深刻影响了ALGOL语言的设计,普及了分时共享技术,并发明了垃圾回收机制。

麦卡锡的大部分职业生涯都在斯坦福大学度过。[2] 他因对人工智能领域的贡献获得了诸多奖项和荣誉,例如1971年的图灵奖、美国国家科学奖章和京都奖。
\subsection{早年生活与教育}
约翰·麦卡锡于1927年9月4日出生在马萨诸塞州波士顿,他的父亲是一位爱尔兰移民,母亲是立陶宛犹太移民。[4] 他的父母分别是约翰·帕特里克·麦卡锡(John Patrick McCarthy)和艾达·格拉特·麦卡锡(Ida Glatt McCarthy)。大萧条期间,家庭多次搬迁,直到父亲在加利福尼亚州洛杉矶找到一份为服装工会(Amalgamated Clothing Workers)担任组织者的工作。他的父亲来自爱尔兰凯里郡的一个小渔村克罗曼(Cromane)。[5] 他的母亲于1957年去世。[6]

麦卡锡的父母在1930年代是共产党的积极成员,他们鼓励孩子学习和批判性思考。在进入高中之前,麦卡锡通过阅读一本名为《十万个为什么》(*100,000 Whys*)的俄文儿童科普书籍,对科学产生了兴趣。[7] 他精通俄语,并在多次访问苏联期间结交了俄罗斯科学家,但在访问东欧集团后与其拉开了距离,并成为一名保守派共和党人。[8]

麦卡锡提早两年从贝尔蒙特高中毕业,[9] 并于1944年被加州理工学院(Caltech)录取。

他展现出数学的早期天赋;在青少年时期,他通过自学加州理工学院使用的大学数学教材掌握了大学数学知识。因此,他进入加州理工学院后得以跳过前两年的数学课程。[10] 他因未参加体育课程而被加州理工学院停学,[11] 后来服役于美国陆军,之后重新被录取,并于1948年获得数学学士学位(BS)。[12]

麦卡锡在加州理工学院听过约翰·冯·诺依曼的讲座,这对他的未来事业产生了重要启发。

麦卡锡在加州理工完成了研究生学习后,前往普林斯顿大学,并于1951年在唐纳德·C·斯宾塞(Donald C. Spencer)的指导下完成了题为《投影算子与偏微分方程》(*Projection operators and partial differential equations*)的博士论文,获得了数学博士学位。[13]
\subsection{学术生涯}

在普林斯顿大学和斯坦福大学短期任职后,麦卡锡于1955年成为达特茅斯学院的助理教授。

一年后,他于1956年秋天转至麻省理工学院(MIT)担任研究员。在麻省理工学院的最后几年,他已经被学生们亲切地称为“约翰叔叔”。[14]

1962年,麦卡锡成为斯坦福大学的正教授,并一直任职至2000年退休。

麦卡锡倡导数学方法,如λ演算,并为人工智能中的常识推理设计了逻辑系统。
\subsection{计算机科学中的贡献}
\begin{figure}[ht]
\centering
\includegraphics[width=6cm]{./figures/3046a488575648f4.png}
\caption{2008年的麦卡锡} \label{fig_YHMKX_2}
\end{figure}
约翰·麦卡锡是人工智能的“创始之父”之一,与艾伦·图灵、马文·明斯基、艾伦·纽厄尔和赫伯特·西蒙齐名。麦卡锡与明斯基、纳撒尼尔·罗切斯特和克劳德·香农在1956年夏天为著名的达特茅斯会议撰写了一份提案,首次提出了“人工智能”这一术语。这次会议标志着人工智能作为一个独立领域的起点。[9][15](明斯基后来于1959年加入麦卡锡在麻省理工学院的团队。)

1958年,他提出了建议接收器(advice taker),这一概念启发了后来关于问答系统和逻辑编程的研究。

在20世纪50年代末,麦卡锡发现原始递归函数可以扩展为对符号表达式进行计算,由此发明了Lisp编程语言。[16] 他发表的关于函数式编程的开创性论文引入了从λ演算语法中借鉴而来的λ符号,这成为后来的编程语言(如Scheme)语义的基础。Lisp在1960年发表后,很快成为人工智能应用的首选编程语言。

1958年,麦卡锡参与了美国计算机协会(ACM)的一个临时语言委员会,该委员会后来参与了ALGOL 60的设计。1959年8月,他提出了递归和条件表达式的使用,这成为ALGOL的一部分。[17] 随后,他加入了国际信息处理联合会(IFIP)的算法语言和演算工作组(Working Group 2.1),参与ALGOL 60和ALGOL 68的标准制定和维护。[19]

大约在1959年,他发明了所谓的垃圾回收(garbage collection)方法,这是一种自动内存管理技术,用于解决Lisp中的内存问题。[20][21]

在麻省理工学院任职期间,他推动了MAC项目的创建;在斯坦福大学工作时,他帮助建立了斯坦福人工智能实验室,该实验室多年来一直是MAC项目的友好竞争对手。

麦卡锡在开发早期三大分时系统(兼容分时系统、BBN分时系统和达特茅斯分时系统)方面发挥了重要作用。他的同事莱斯特·厄内斯特对《洛杉矶时报》说:

“如果不是因为约翰启动了分时系统的发展,互联网可能不会如此迅速地出现。我们不断为分时系统发明新名字,比如服务器……现在我们称之为云计算,但它仍然只是分时系统的延续。约翰是这一切的开端。”[9]  

——艾琳·吴(Elaine Woo)

1961年,他或许是第一个公开提出实用计算(utility computing)概念的人。在麻省理工学院百年校庆的演讲中,他设想分时技术可能带来一种未来,计算能力甚至具体应用程序可以通过像水或电一样的公共事业模式出售。[22][23] 这种“计算机公用事业”的想法在20世纪60年代末非常流行,但在90年代中期逐渐淡出。然而,自2000年以来,这一概念以新的形式重新出现(如应用服务提供商、网格计算和云计算)。

1966年,麦卡锡和他的斯坦福团队编写了一款计算机程序,与苏联团队进行了一系列国际象棋比赛;麦卡锡团队输掉了两局,平局两局(见*科托克-麦卡锡比赛)。

从1978年至1986年,麦卡锡开发了非单调推理的圈定法(circumscription method)。

1982年,他似乎首次提出了太空喷泉(space fountain)的概念。这是一种延伸至太空的塔,通过由地球发射的颗粒流产生的向外力保持垂直。货物可以通过类似传送带的装置沿颗粒流向上运输。[24]
\subsection{其他活动}
麦卡锡经常在Usenet论坛上评论世界事务。他的一些观点可以在他的网站页面“可持续发展”中找到,[25] 该页面旨在“表明人类的物质进步既可取又可持续”。麦卡锡是一个热衷读书的人,乐观主义者,并且是言论自由的坚定支持者。他在Usenet上的最佳互动可以在rec.arts.books的存档中看到。他还积极参加了在帕洛阿尔托举办的旧金山湾区的“rab-fests”,这是r.a.b.读者的聚餐活动。他在斯坦福大学为涉及欧洲民族笑话的言论自由批评进行了辩护。[26]

麦卡锡十分重视数学及其教育。他多年来在Usenet上的签名为:“拒绝做算术的人注定会说胡话”;他的车牌框架上也写着类似的话:“做算术,否则注定要说胡话”。[27][28] 他指导了30位博士生。[29]

他的2001年短篇小说《机器人和婴儿》(*The Robot and the Baby*)以讽刺的方式探讨了机器人是否应该拥有(或模拟拥有)情感的问题,并预见了随后的互联网文化和社交网络的某些方面。[30][31]
\subsection{个人生活}
麦卡锡结过三次婚。他的第二任妻子是维拉·沃森(Vera Watson),她是一名程序员和登山爱好者,1978年在一次全女性登山探险中尝试攀登安纳普尔纳I峰中央峰时遇难。后来,他与斯坦福大学及国际科学研究所(SRI International)的计算机科学家卡罗琳·塔尔科特(Carolyn Talcott)结婚。[32][33]

麦卡锡在斯坦福纪念教堂的一次关于人工智能的演讲中宣布自己是无神论者。[34][35][36] 他成长于共产主义背景,但在1968年苏联入侵捷克斯洛伐克后,访问该地后转变为保守派共和党人。[37] 2011年10月24日,他在斯坦福的家中去世。[38]
\subsection{人工智能哲学}
1979年,麦卡锡发表了一篇名为《为机器赋予心理品质》(Ascribing Mental Qualities to Machines)的文章。[39] 在文中,他写道:“像恒温器这样简单的机器可以说是有信念的,而拥有信念似乎是大多数能够解决问题的机器的一个特征。” 1980年,哲学家约翰·希尔勒(John Searle)通过著名的“中文房间论证”(Chinese Room Argument)[40][15] 对此提出反驳,认为机器不能拥有信念,因为它们没有意识。希尔勒主张机器缺乏意向性。这一争论引发了大量支持双方观点的文献。[示例有待提供]
\subsection{奖项与荣誉}
\begin{itemize}
\item 图灵奖:由美国计算机协会(ACM)授予(1971年)  
\item 京都奖(1988年)  
\item 美国国家科学奖章(数学、统计与计算科学领域,1990年)[41]  
\item 入选\textbf{计算机历史博物馆院士},因“共同创立人工智能(AI)和分时系统领域,以及对数学和计算机科学的重大贡献”(1999年)[42]  
\item 富兰克林奖章(计算机与认知科学领域,由富兰克林研究所颁发,2003年)  
\item 入选\textbf{IEEE智能系统AI名人堂},因“对人工智能和智能系统领域的重大贡献”(2011年)[43]  
\item 被命名为2012年斯坦福工程英雄之一[44] 
\end{itemize} 
\subsection{主要出版作品}
\begin{itemize}
\item McCarthy, J. (1959)。Programs with Common Sense。载于《机械化思维过程会议论文集》,第756–791页,伦敦:皇家文具局。([存档于Wayback Machine](https://web.archive.org/))
\item McCarthy, J. (1960)。Recursive functions of symbolic expressions and their computation by machine。《ACM通讯》(Communications of the ACM),3(4): 184-195。
\item McCarthy, J. (1963a)。A basis for a mathematical theory of computation。载于《计算机编程与形式化系统》。北荷兰出版社(North-Holland)。
\item McCarthy, J. (1963b)。Situations, actions, and causal laws。斯坦福大学技术报告。
\item McCarthy, J., and Hayes, P. J. (1969)。*Some philosophical problems from the standpoint of artificial intelligence*。载于Meltzer, B.和Michie, D.编辑的《机器智能4》(Machine Intelligence 4),爱丁堡大学出版社,第463–502页。([存档于Wayback Machine](https://web.archive.org/))
\item McCarthy, J. (1977)。Epistemological problems of artificial intelligence。载于《国际人工智能联合会议论文集》(IJCAI),第1038–1044页。
\item McCarthy, J. (1980)。Circumscription: A form of non-monotonic reasoning。《人工智能》(Artificial Intelligence),13(1–2): 23–79。doi:10.1016/0004-3702(80)90011-9。
\item McCarthy, J. (1986)。Applications of circumscription to common sense reasoning。《人工智能》(Artificial Intelligence),28(1): 89–116。CiteSeerX 10.1.1.29.5268。doi:10.1016/0004-3702(86)90032-9。
\item McCarthy, J. (1990)。Generality in artificial intelligence。载于Lifschitz, V.编辑的《形式化常识》(Formalizing Common Sense),Ablex出版社,第226–236页。
\item McCarthy, J. (1993)。Notes on formalizing context。载于《国际人工智能联合会议论文集》(IJCAI),第555–562页。
\item McCarthy, J., and Buvac, S. (1997)。Formalizing context: Expanded notes。载于Aliseda, A.; van Glabbeek, R.; 和Westerstahl, D.编辑的《自然语言计算》(Computing Natural Language),斯坦福大学。同时作为斯坦福技术报告STAN-CS-TN-94-13提供。
\item McCarthy, J. (1998)。Elaboration tolerance。载于第四届常识推理逻辑形式化国际研讨会工作论文集。
\item Costello, T., and McCarthy, J. (1999)。Useful counterfactuals。《人工智能电子事务》(Electronic Transactions on Artificial Intelligence),3(A): 51-76。
\item McCarthy, J. (2002)。*Actions and other events in situation calculus*。载于Fensel, D.; Giunchiglia, F.; McGuinness, D.; 和Williams, M.编辑的《KR-2002会议论文集》,第615–628页。
\end{itemize}
\subsection{另见}
\begin{itemize}
\item 克里斯托弗·斯特雷奇(Christopher Strachey),于1959年初申请了分时技术的专利  
\item 丰饶主义(Cornucopian)  
\item 框架问题(Frame problem)  
\item 计算机科学先驱列表(List of pioneers in computer science)  
\item 科托克-麦卡锡国际象棋比赛(Kotok-McCarthy)  
\item 麦卡锡91函数(McCarthy 91 function)  
\item 麦卡锡形式化(McCarthy formalism)  
\item 沃森(计算机)(Watson (computer))  
\end{itemize}
\subsection{参考文献}
\begin{enumerate}
\item Mishlove, Jeffrey(2011年11月3日)。《约翰·麦卡锡(1927-2011):人工智能(完整)——允许思考》。YouTube(视频)。存档于2013年3月24日。检索于2022年8月8日。同名版本存于Ghost Archive。存档于2021年7月31日。检索于2022年8月8日。
\item McCarthy, John.《约翰·麦卡锡教授》(Professor John McCarthy)。jmc.stanford.edu。
\item 《约翰·麦卡锡——图灵奖得主》(John McCarthy – A.M. Turing Award Laureate)。amturing.acm.org。
\item Shasha, Dennis; Lazere, Cathy(1998)。《脱离常规思维:15位伟大计算机科学家的生活与发现》(Out of Their Minds: The Lives and Discoveries of 15 Great Computer Scientists)。Springer出版社,第23页。ISBN 9780387982694。检索于2016年2月27日。
\item 《提出“人工智能”术语的著名学者》。《爱尔兰时报》。检索于2016年1月28日。
\item 《计算机与计算历史:现代计算机的诞生、软件历史、约翰·麦卡锡的LISP语言》。history-computer.com。存档于2020年1月3日。检索于2016年1月28日。
\item Nilsson, Nils J.《传记回忆录》(A Biographical Memoir)(PDF)。美国国家科学院。检索于2022年2月20日。
\item Earnest, Les。《1968年苏联占领捷克斯洛伐克:约翰·麦卡锡目睹的信件》(PDF)。《莱斯特·唐纳德·厄内斯特的回忆录》。存档于2023年6月7日。检索于2022年2月20日。
\item Woo, Elaine(2011年10月28日)。《约翰·麦卡锡去世,享年84岁;人工智能之父》。《洛杉矶时报》。
\item Hayes, Patrick J.; Morgenstern, Leora(2007)。《在约翰·麦卡锡80岁生日之际,纪念他的贡献》。《AI Magazine》,28(4):93–102。存档于2011年9月23日。检索于2010年11月24日。
\item Williams, Sam(2002年3月5日)。《争论AI:21世纪科学的战场》(Arguing A.I.: The Battle for Twenty-first-Century Science)。AtRandom出版社。ISBN 978-0812991802。
\item Lester Earnest。《图灵奖:约翰·麦卡锡,美国——1971》。ACM。检索于2012年9月5日。
\item McCarthy, John(1951年)。《投影算子与偏微分方程》。
\item Levy, Steven。《黑客:计算机革命的英雄》(*Hackers, Heroes of the Computer Revolution)。Gutenberg.org,第34页。
\item Roberts, Jacob(2016)。《思维机器:人工智能的探索》(Thinking Machines: The Search for Artificial Intelligence)。《Distillations》,2(2):14–23。存档于2018年8月19日。检索于2018年3月20日。
\item McCarthy, John(1960年)。《符号表达的递归函数及其机器计算》(Recursive Functions of Symbolic Expressions and Their Computation by Machine)。《ACM通讯》,3(4):184–195。doi:10.1145/367177.367199。S2CID 1489409。
\item McCarthy, John(1959年8月)。《致编辑的一封信》(*Letter to the editor)。《ACM通讯》,2(8):2–3。doi:10.1145/368405.1773349。S2CID 7196706。
\item Jeuring, Johan; Meertens, Lambert; Guttmann, Walter(2016年8月17日)。《IFIP工作组2.1简介》。Foswiki。检索于2020年10月4日。
\item Swierstra, Doaitse; Gibbons, Jeremy; Meertens, Lambert(2011年3月2日)。《ScopeEtc: IFIP21》。Foswiki。检索于2020年10月4日。
\item McCarthy, John(1960年4月)。《符号表达的递归函数及其机器计算》(Recursive functions of symbolic expressions and their computation by machine)。《ACM通讯》,3(4):184–195。doi:10.1145/367177.367199。S2CID 1489409。
\item 符号表达的递归函数及其机器计算,第I部分。存档于2013年10月4日。检索于2013年11月24日。
\item Garfinkel, Simson(1999)。Hal Abelson(编)。《信息社会的建筑师:麻省理工学院计算机科学实验室的三十五年》(Architects of the Information Society, Thirty-Five Years of the Laboratory for Computer Science at MIT)。剑桥:MIT出版社,第1页。ISBN 978-0-262-07196-3。
\item 演讲《分时计算机系统》(Time Sharing Computer Systems),载于《未来的管理与计算机》(Management and the Computer of the Future),编者:Martin Greenberger,1962年出版。1965年以《未来的计算机与世界》重印。
\item McCarthy, John(1994年7月31日)。Space Bridge Short。Usenet新闻组sci.space.tech上的帖子。Google Groups。
\item McCarthy, John(1995年2月4日)。Progress and its sustainability。formal.stanford.edu。存档于2013年10月4日。检索于2013年11月24日。
\item McCarthy, John(1997年5月12日)。Attempt at Censorship of Electronic Libraries at Stanford University in 1989。formal.stanford.edu。检索于2023年12月5日。
\item “拒绝做算术的人注定会说胡话”(Usenet新闻组sci.environment搜索)。
\item 《约翰·麦卡锡,84岁去世;计算机设计的先驱》。《纽约时报》,2011年10月26日。
\item 约翰·麦卡锡学生树,用于计算机历史展览。infolab.Stanford.edu,2012年4月21日。存档于2013年12月2日。检索于2013年11月24日。
\item McCarthy, John(2001年6月28日)。The Robot and the Baby。formal.stanford.edu。存档于2013年10月4日。检索于2013年11月24日。
\item Thomson, Cask J.(2011年10月26日)。《真正的技术创新者D. Ritchie与J. McCarthy的去世——却被史蒂夫·乔布斯的去世所掩盖》。WordsWithMeaning博客。存档于2012年4月26日。
\item Markoff, John(2011年10月25日)。《约翰·麦卡锡,84岁去世;计算机设计的先驱》。《纽约时报》。
\item 卡罗琳·塔尔科特传记(Biography of Carolyn Talcott)。斯坦福大学。存档于2013年12月2日。
\item 计算机先驱讨论无神论与人工智能。2023年1月23日。
\item 关于约翰·麦卡锡(About John McCarthy)。斯坦福大学。存档于2013年10月4日。检索于2013年2月1日。
\item McCarthy, John(2003年3月7日)。*Commentary on World, US, and scientific affairs*。斯坦福大学。存档于2013年10月4日。检索于2013年2月1日。“顺便说一句,我是无神论者。”

- Earnest, Les.《约翰·麦卡锡传记》(*Biographies of John McCarthy*)。斯坦福大学。存档于2016年6月11日。检索于2016年2月14日。

- Myers, Andrew(2011年10月25日)。《斯坦福大学的约翰·麦卡锡,人工智能领域的开创性人物,逝世,享年84岁》。斯坦福大学新闻。检索于2011年10月26日。

- McCarthy, J.(1979)。*Ascribing mental qualities to machines*。载于《人工智能的哲学视角》(*Philosophical perspectives in artificial intelligence*),M. Ringle编辑。Atlantic Highlands, N.J.:Humanities出版社。

- Searle, John R.(1980)。*Minds, brains, and programs*(PDF)。**《行为与脑科学》**,3(3): 417–457。doi:10.1017/s0140525x00005756。S2CID 55303721。

- *总统国家科学奖章:1990年获奖者详情*。美国国家科学基金会,2006年2月14日。检索于2012年9月27日。

- CHM.《约翰·麦卡锡——计算机历史博物馆院士奖获得者》(*John McCarthy – CHM Fellow Award Winner*)。存档于2015年4月3日。检索于2015年3月30日。

- **AI名人堂**(PDF)。**《IEEE智能系统》**,26(4): 5–15。2011年。doi:10.1109/MIS.2011.64。存档于2011年12月16日。检索于2015年9月4日。

- Beckett, Jamie(2012年12月2日)。《斯坦福工程学院命名新的工程英雄》。**斯坦福新闻**。检索于2012年12月2日。
\end{enumerate}