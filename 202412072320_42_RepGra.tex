% 图的矩阵表示
% keys 矩阵表示|图|邻接矩阵|关联矩阵
% license Usr
% type Tutor

\pentry{图\nref{nod_Graph},矩阵及其运算\nref{nod_Mat}}{nod_3f6e}
将图的点集和边集的关系用矩阵来体现,就是所谓的图的矩阵表示。体现点与点相邻关系的矩阵称为邻接矩阵;体现点与边关联关系的矩阵称为关联矩阵。当对某个图选定了权函数成为\enref{赋权图}{GraSpa}时,体现权函数的矩阵称为赋权矩阵。

\subsection{邻接矩阵}
邻接矩阵是用来体现点点相邻关系的,具体地,设图 $G$ 的点集为 $V(G)=\{v_1,\cdots,v_n\}$。若 $G$ 是无向图,则邻接矩阵的第 $(i,j)$ 个元反映了连接 $v_i,v_j$ 的边数。若 $G$ 是有向图,则邻接矩阵的第 $(i,j)$ 个元反映了以 $v_i$ 为起点,$v_j$ 为终点的有向边数。
\begin{definition}{邻接矩阵}
设 $G$ 是图,$V(G)=\{v_1,\cdots,v_n\}$,$A=(a_{ij})$ 是 $n$ 阶矩阵。若 $G$ 是无向图,$a_{ij}$ 等于连接 $v_i,v_j$ 的边

\end{definition}
















