% 费马大定理(综述)
% license CCBYSA3
% type Wiki

本文根据 CC-BY-SA 协议转载翻译自维基百科\href{https://en.wikipedia.org/wiki/Fermat\%27s_Last_Theorem}{相关文章}。

在数论中,费马大定理(在较早的文献中有时称为“费马猜想”)陈述如下:对于任意整数\( n > 2 \),不存在三个正整数\( a, b, c \)满足方程\( a^n + b^n = c^n \)。而对于\( n = 1 \)和\( n = 2 \)的情形,自古以来就已知存在无穷多个解。\(^\text{[1]}\)

这个命题最早是皮埃尔·德·费马大约于1637年在一本《算术》书的页边空白处提出的。他还写道他已有一个证明,但“这个证明太大,写不下”。尽管费马曾提出的其他未经证明的命题后来被他人证明并被称为“费马定理”(例如费马两平方和定理),但唯独这条“费马大定理”长期无法证明,使人们怀疑费马是否真的拥有一个正确的证明。因此,这个命题长期以来被称为\textbf{猜想}而不是定理。经过数学家长达358年的努力,安德鲁·怀尔斯于1994年首次成功给出了完整证明,并于1995年正式发表。2016年,怀尔斯因其工作获得阿贝尔奖,其成果被称为“一项惊人的突破”。\(^\text{[2]}\)此外,该证明还涵盖了大量谷山–志村猜想的内容,该猜想后来被称为模性定理,它不仅解开了费马大定理之谜,还开辟了许多新领域,并发展出了强大的模性提升技术,对解决众多其他数学难题产生了深远影响。

这个未解难题曾在 19世纪和20世纪极大地推动了代数数论的发展。在整个数学史上,费马大定理是最著名的定理之一。在被证明之前,它还曾被列入《吉尼斯世界纪录》,称为“最难的数学问题”,部分原因是该定理拥有最多数量的失败证明尝试。\(^\text{[3]}\)