% 浙江大学 2003 年 考研 量子力学
% license Usr
% type Note

\textbf{声明}:“该内容来源于网络公开资料,不保证真实性,如有侵权请联系管理员”

\subsection{第一题(35 分):}
1. 如果 $\psi_1$ 和 $\psi_2$ 是某一体系含时薛定谔方程的两个解

1) 它们的线性组合 $\psi = a\psi_1 + b\psi_2$,($a, b$ 是常数),是否满足同样的含时薛定谔方程?

2) 若令 $\psi' = \psi_1\psi_2$,你认为 $\psi'$ 是否满足同样的含时薛定谔方程?

2. 质量相同的两个粒子分别在宽度不同的两个一维无限深势阱中,试问势阱中的基态能量低,还是宽势阱中的基态能量低?

3.1) 你是否认识这三个矩阵:
\[\begin{pmatrix}0 & 1 \\\\1 & 0\end{pmatrix}\quad\begin{pmatrix}0 & -i \\\\i & 0\end{pmatrix}\quad\begin{pmatrix}1 & 0 \\\\0 & -1\end{pmatrix}~\]
在量子力学中他们称为什么?

2) 大家知道,$[\hat{x}, \hat{p}] = i\hbar$ 为量子力学中最基本的对易关系(这里 $\hat{x}$ 和 $\hat{p}$ 分别是位置算符和动量算符)

和动量算符,你是否记得角动量 $\hat{L}_x, \hat{L}_y, \hat{L}_z$ 之间的对易关系?请写出来!

3) 请算一下
\[[[\hat{L}_x, \hat{L}_y], \hat{L}_z] + [[\hat{L}_y, \hat{L}_z], \hat{L}_x] + [[\hat{L}_z, \hat{L}_x], \hat{L}_y] = ?~\]
\subsection{第二题(20 分):}

\subsection{第三题(25分):}