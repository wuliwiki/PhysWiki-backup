% 三维直角坐标系中的亥姆霍兹方程
% keys 直角坐标系|亥姆霍兹方程


\pentry{一维齐次亥姆霍兹方程\upref{HmhzEq}, 分离变量法解偏微分方程\upref{SepVar}}
三维直角坐标系中的亥姆霍兹方程为
\begin{equation}\label{RHM_eq2}
\laplacian u+k^2u=0
\end{equation}
这里,$k$ 为常实数。
\subsection{通解}
使用分离变量法进行求解,令
\begin{equation}\label{RHM_eq1}
\begin{aligned}
&u(x,y,z)=X(x)Y(y)Z(z)\\
&k^2=k_x^2+k_y^2+k_z^2
\end{aligned}
\end{equation}
于是方程分解为3个独立的一维齐次亥姆霍兹方程\upref{HmhzEq}
\begin{equation}
\begin{aligned}
\dv[2]{X}{x}+k_x^2X=0\\
\dv[2]{Y}{y}+k_y^2Y=0\\
\dv[2]{X}{x}+k_x^2Z=0
\end{aligned}
\end{equation}
由\autoref{HmhzEq_eq1}~\upref{HmhzEq},其实数解分别为
\begin{equation}
\begin{aligned}
&X(x)=C_1\cos k_xx+C_1'\sin k_xx\\
&Y(y)=C_2\cos k_yy+C_2'\sin k_yy\\
&Z(z)=C_3\cos k_zz+C_3'\sin k_zz
\end{aligned}
\end{equation}
这里,$C_i,C_i'$ 为常实数。由\autoref{RHM_eq1} ,得\autoref{RHM_eq2} 的通解
\begin{equation}\label{RHM_eq3}
u(x,y,z)=(C_1\cos k_xx+C_1'\sin k_xx)(C_2\cos k_yy+C_2'\sin k_yy)(C_3\cos k_zz+C_3'\sin k_zz)
\end{equation}
\begin{example}{矩形波导中的电磁波}
\begin{figure}[ht]
\centering
\includegraphics[width=5cm]{./figures/RHM_1.pdf}
\caption{矩形波导} \label{RHM_fig1}
\end{figure}
在高频电力系统中,为解决电磁波向外辐射的损耗以及与环境的干扰问题,常常采用波导进行电磁波的传输,波导是一根空心金属管,截面通常为矩形或圆形。如\autoref{RHM_fig1} 是一矩形波导,其长宽为 $a,b$,以长边为 $x$ 方向,短边为 $y$ 方向;$z$ 轴沿传播方向。由\autoref{TSEBW_eq3}~\upref{TSEBW},波导内电磁波满足亥姆霍兹方程\autoref{RHM_eq2} ,此时,应以 $\bvec E,\bvec H$ 代替 $u$,我们以计算 $\bvec E$ 进行说明,读者可对 $\bvec H$ 进行类比。注意电磁波沿 $z$ 轴传播,电场应取
\begin{equation}
\bvec E(x,y,z)=\bvec E(x,y)\E^{\I k_z z}
\end{equation}

由刚刚已证明的亥姆霍兹方程的通解\autoref{RHM_eq3} ,并注意现在分离变量不含 $z$ ,易知电场的任一直角分量 $u(x,y)$ 有通解
\begin{equation}
u(x,y)=(C_1\cos k_xx+C_1'\sin k_xx)(C_2\cos k_yy+C_2'\sin k_yy)
\end{equation}

而边界条件为
\begin{equation}
\begin{aligned}
E_y=E_z=0,\pdv{E_x}{x}=0\quad(x=0,a)\\
E_x=E_z=0,\pdv{E_y}{y}=0\quad(y=0,b)
\end{aligned}
\end{equation}
由 $x=0$ 和 $y=0$ 面上的边界条件,得
\begin{equation}
\begin{aligned}
E_x=A\cos k_x x\sin k_yy\E^{\I k_zz} \\
E_x=B\sin k_x x\cos k_yy\E^{\I k_zz} \\
E_x=C\sin k_x x\sin k_yy\E^{\I k_zz} 
\end{aligned}
\end{equation}
 
由 $x=a$ 和 $y=b$ 面上的边界条件,得 
\begin{equation}
k_x=\frac{m\pi}{a},\quad k_y=\frac{m\pi}{b}\quad(m,n=0,1,2,3,\cdots)
\end{equation}
注意 $\laplacian \cdot\bvec E=0$,得
\begin{equation}
k_xA+k_yB+\I k_zC=0
\end{equation}
因此, $A,B,C$ 中只有两个是独立的。对于每一组 $(m,n)$,有两种独立波模。
\end{example}
