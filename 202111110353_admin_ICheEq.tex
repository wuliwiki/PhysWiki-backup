% 理想气体化学平衡条件
% 化学反应|化学平衡条件|理想气体

\pentry{吉布斯自由能\upref{GibbsG},多元系热力学导引\upref{mulTh}}

\addTODO{未完成}

设在一组化学反应中,$i$ 组元的物质的量变动为 $\delta n_i$,它们之间满足一定的比例关系,简写为以下方程:
\begin{equation}
\delta n_i=\nu_i \delta n
\end{equation}

$\nu_i>0$ 表示生成物,$\nu_i<0$ 表示反应物.这一般用来区分化学方程式的左侧和右侧.

在等温等压条件下,化学平衡要求平衡态的吉布斯函数最小,即\textbf{吉布斯判据}(\autoref{GibbsG_eq2}~\upref{GibbsG}).
\begin{equation}
\delta G=\sum_i \mu_i\delta n_i=\delta n\sum_i \nu_i\mu_i
\end{equation}

于是我们得到了化学平衡条件.
\begin{equation}\label{ICheEq_eq2}
\sum_i\nu_i\mu_i=0
\end{equation}

要注意的是,因为化学反应的系统是多元系\upref{mulTh},这里组元 $i$ 的化学势 $\mu_i$ 是偏摩尔吉布斯函数,它是温度、压强以及各组员摩尔分数的函数.

\subsection{理想气体化学平衡}

我们考虑的体系为温度 $T$ 压强 $P$ 的混合理想气体,第 $i$ 个组元的摩尔分数为 $x_i$,则该组元分压为 $x_iP$,化学势为 $\mu_i=g(T,p_i)$,$g$ 为纯 $i$ 组元的化学势,因此根据理想气体吉布斯函数的公式(\autoref{GibbsG_eq3}~\upref{GibbsG}),有下列等式
\begin{equation}\label{ICheEq_eq1}
\mu_i=RT(\phi_i(T)+\ln(x_iP))
\end{equation}

如果具体地把 $\phi_i$ 写出来,那么有(这里小写字母 $h_0,c_p,s_0$ 表示每摩尔的焓常量、定压热容、熵常量)
\begin{equation}
\phi_i=\frac{h_{0,i}}{RT}-\int\frac{\dd T}{RT^2}\int c_{p,i}\dd T-\frac{s_{0,i}}{R}
\end{equation}

混合理想气体的摩尔吉布斯函数为
\begin{equation}
G(T,P,n_1,\cdots,n_k)=\sum_i n_iRT[\phi_i(T)+\ln(x_iP)]
\end{equation}

根据吉布斯判据推出的化学平衡条件\autoref{ICheEq_eq2} ,有
\begin{equation}\label{ICheEq_eq3}
\sum_i\nu_i[\phi_i(T)+\ln (x_iP)]=0
\end{equation}
定义\textbf{定压平衡常量} $K_p(T)$:
\begin{equation}
\ln K_p=-\sum_i\nu_i\phi_i(T)
\end{equation}

那么\autoref{ICheEq_eq3} 可以化简为
\begin{equation}\label{ICheEq_eq4}
\begin{aligned}
&\prod_i (x_iP)^{\nu_i}=K_p(T)\\
&\prod_i x_i^{\nu_i}=P^{-\nu} K_p(T),(\nu=\sum_i\nu_i)
\end{aligned}
\end{equation}

\autoref{ICheEq_eq4} 就是化学反应的平衡条件.当等式不成立,那么化学反应就要进行.根据吉布斯判据,化学反应正向进行的条件是:
\begin{equation}
\begin{aligned}
&\sum_i \nu_i\mu_i<0\\
&\Rightarrow \prod_i x_i^{\nu_i}<P^{-\nu}K_p(T)
\end{aligned}
\end{equation}

\addTODO{例子}

\subsection{化学反映分解度与温度压强的关系}
\addTODO{未完成}