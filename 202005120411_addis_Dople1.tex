% 多普勒效应(一维)

\pentry{平面波\upref{PWave}}

\textbf{多普勒效应(Doppler effect)}是讨论, 当机械波的波源和(或)接收者相对于波的介质运动时, 发射的频率和接收到的频率之间有何关联. 本文不讨论相对论效应, 即假设波速远小于真空中的光速, 另外本文只讨论波源和接收者沿同一直线运动的情况.

\begin{figure}[ht]
\centering
\includegraphics[width=6cm]{./figures/Dople1_1.pdf}
\caption{多普勒效应} \label{Dople1_fig1}
\end{figure}

\begin{example}{}
生活中一种常见的多普勒效应是, 一辆疾驰的车一边鸣笛一边驶过行人, 人听到的音调就会先高后低. 这是因为, 车经过人之前不断靠近人, 经过人后再不断远离人. 可见多普勒效应和运动的速度有关.
\end{example}

在分析多普勒效应时, 一种方便的做法是选取介质为参考系, 例如有均匀的风时, 参考系随风运动. 假设介质处处均匀且静止, 波在介质中传播的速度(\textbf{波速}) $v$ 处处相等, 且与方向无关(\textbf{各向同性}).

令\autoref{Dople1_fig1} 中波源甲运动方程, 即位置关于时间的函数为 $x_1(t)$, 对时间求导得到速度 $v_1(t) = \dot{x}_1(t)$. 同理, 接收者乙位置和速度分别为 $x_2(t)$ 和 $v_2(t)$. 令波源的频率为 $f_1$, 接收者的频率为 $f_2$. 当两点分别以小于波速的速度($\abs{v_1}, \abs{v_2} < v$)匀速运动时, 有
\begin{equation}
\frac{f_2}{f_1} = \frac{v \mp v_2}{v \mp v_1}
\end{equation}
当 $x_1 < x_2$ 时上式取负号, 当 $x_2 < x_1$ 时上式取正号.

\subsection{推导}
本质上, 多普勒效应可以等效为追及问题, 可以想象甲以一定的频率 $f_1$ 向乙发射速度为 $u$ 的子弹, 子弹的位置对应波峰的位置, 两个相邻子弹之间的间距对应波长. 若甲乙相对介质静止不动或者以相同的速度运动, 则乙接收到子弹的频率和甲发射的频率是一样的, 但若甲乙不断靠近, 则接收子弹的频率就会更高, 若不断远离, 则接受的频率就更低.

(推导未完成)

(任意运动的情况又如何?)
