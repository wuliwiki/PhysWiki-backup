% 结合代数(综述)
% license CCBYSA3
% type Wiki

本文根据 CC-BY-SA 协议转载翻译自维基百科\href{https://en.wikipedia.org/wiki/Associative_algebra}{相关文章}。

在数学中,交换环(通常是一个域) $K$ 上的结合代数 $A$ 是一个环 $A$,并且带有一个从 $K$ 到 $A$ 的中心(center)的环同态。因此,它是一种代数结构,包含加法、乘法和数量乘法(即由 $K$ 中元素通过环同态的像所定义的乘法)。加法和乘法运算共同赋予 $A$ 环的结构;加法和数量乘法运算共同赋予 $A$ $K$-模或向量空间的结构。在本文中,我们也使用 $K$-代数 一词来指代 $K$ 上的结合代数。

$K$-代数的一个标准例子是定义在交换环 $K$ 上的方阵环,采用通常的矩阵乘法。

一个交换代数是乘法交换的结合代数,或者等价地,是同时也是交换环的结合代数。

在本文中,假设结合代数都有一个乘法单位元,记作 $1$;为强调这一点,有时称为有单位结合代数。在数学的一些领域中不做这一假设,这类结构称为非有单位结合代数。我们也假设所有的环都是有单位的,并且所有的环同态都是保单位元的。

每个环都是它的中心上的结合代数,也是整数环 $\mathbb{Z}$ 上的结合代数。
\subsection{定义}
设 $R$ 是一个交换环(因此 $R$ 也可以是一个域)。一个结合的 $R$-代数 $A$(或更简单地称为 $R$-代数 $A$)是一个环 $A$,并且同时是一个 $R$-模,满足环加法与模加法是同一个运算,并且数量乘法满足
$$
r \cdot (xy) = (r \cdot x)y = x(r \cdot y)~
$$
对所有 $r \in R$ 和代数中的 $x, y$ 成立。(这个定义意味着代数作为一个环是有单位的,因为假设环必须有乘法单位元。)

等价地,一个结合代数 $A$ 是一个环,并且带有一个从 $R$ 到 $A$ 的中心的环同态。如果 $f$ 是这样的同态,则数量乘法为$(r, x) \mapsto f(r) x$(此处乘法是环乘法);如果给定了数量乘法,则该环同态由$r \mapsto r \cdot 1_A$给出。(另见下文“由环同态导出”一节。)每个环都是一个结合的 $\mathbb{Z}$-代数,其中 $\mathbb{Z}$ 表示整数环。

一个交换代数是一个乘法交换的结合代数,或者等价地,是一个同时也是交换环的结合代数。
\subsubsection{作为模范畴中的幺半群对象}
这个定义等价于说:一个有单位的结合 $R$-代数是 $R$-Mod(即 $R$-模的单(幺)积范畴)中的一个幺半群对象。按照定义,一个环是阿贝尔群范畴中的幺半群对象;因此,结合代数的概念可以通过将阿贝尔群范畴替换为模范畴而得到。

进一步推广这一思想,一些作者将“广义环”定义为某个行为类似于模范畴的其他范畴中的幺半群对象。实际上,这种重新解释使得我们可以避免对代数 $A$ 的元素做显式引用。例如,结合律可以用如下方式表达:根据模张量积的泛性质,乘法(即 $R$-双线性映射)对应于一个唯一的 $R$-线性映射
$$
m: A \otimes_R A \to A~
$$
结合律则对应于以下恒等式:
$$
m \circ (\operatorname{id} \otimes m) = m \circ (m \otimes \operatorname{id})~
$$
\subsubsection{由环同态出发}
一个结合代数本质上等价于一个像落在中心中的环同态。确实,假设给定一个环 $A$ 和一个环同态$\eta : R \to A$其像落在 $A$ 的中心,那么可以通过定义

$$
r \cdot x = \eta(r) x
$$

(对所有 $r \in R$ 和 $x \in A$)使 $A$ 成为一个 $R$-代数。反过来,如果 $A$ 是一个 $R$-代数,取 $x = 1$,同样的公式又定义了一个像落在中心的环同态

$$
\eta : R \to A。
$$

如果一个环是交换的,那么它等于它的中心,因此一个交换的 $R$-代数可以简单地定义为:一个交换环 $A$ 连同一个交换环同态

$$
\eta : R \to A。
$$

上述出现的环同态 $\eta$ 通常称为**结构映射(structure map)**。在交换情形下,可以考虑这样一个范畴:其对象是固定 $R$ 的环同态 $R \to A$,即交换的 $R$-代数;其态射是“在 $R$ 下”的环同态 $A \to A'$,即图

$$
R \to A \to A'
$$

等于

$$
R \to A',
$$

也就是交换环范畴在 $R$ 下的余切范畴(coslice category)。素谱函子 Spec 然后给出这个范畴与 Spec $R$ 上的仿射概形范畴之间的反等价(anti-equivalence)。

如何削弱交换性假设是非交换代数几何以及近来的导出代数几何(derived algebraic geometry)的研究主题。参见:**泛矩阵环(Generic matrix ring)**。
