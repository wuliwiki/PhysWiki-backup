% 布莱斯·帕斯卡(综述)
% license CCBYSA3
% type Wiki

本文根据 CC-BY-SA 协议转载翻译自维基百科\href{https://en.wikipedia.org/wiki/Blaise_Pascal}{相关文章}。

布莱兹·帕斯卡(Blaise Pascal,\(^\text{[a]}\) 1623年6月19日-1662年8月19日)是一位法国数学家、物理学家、发明家、哲学家及天主教作家。

帕斯卡是神童,由担任鲁昂税务官的父亲亲自教育。他最早的数学研究是投影几何,16岁时便撰写了一篇重要的关于圆锥曲线的论文。后来,他与皮埃尔·费马通信探讨概率论,对现代经济学与社会科学的发展产生了深远影响。1642年,他开始从事计算机的先驱性研究,发明了后来被称为“帕斯卡计算器”或“帕斯卡机”的装置,使他成为机械计算器的最早两位发明人之一\(^\text{[6][7]}\)。

与同时代的勒内·笛卡尔一样,帕斯卡也是自然科学和应用科学的先驱。他撰文为科学方法辩护,并提出了若干颇具争议的研究成果。他在流体研究方面作出了重要贡献,推广伊万杰利斯塔·托里拆利的研究成果,澄清了压力和真空的概念。国际单位制中压力单位“帕斯卡”正是以他命名的。1647年,他继托里拆利和伽利略之后,驳斥了亚里士多德与笛卡尔等人所持的“自然界厌恶真空”的观点。

他也被誉为现代公共交通的发明者,因为他在1662年去世前不久创立了“五苏之马车”,这是历史上第一种现代公共交通服务\(^\text{[8]}\)。

1646年,他与妹妹雅克琳一同接受了天主教内部一个被批评者称为詹森主义的宗教运动\(^\text{[9]}\)。1654年末经历一次宗教体验后,他开始撰写有深远影响的哲学与神学作品。他最著名的两部著作都诞生于这一时期:《省函集》和《思想录》。《省函集》以詹森主义者与耶稣会士之间的冲突为背景;而《思想录》中包含了著名的“帕斯卡赌注”,原名为《论机器的演说》\(^\text{[10][11]}\),这是一个以信仰主义为基础、具有概率论性质的论证,主张人应当相信上帝的存在。同年,他还撰写了一部关于“算术三角形”的重要论文。1658至1659年间,他又研究了摆线及其在求解立体体积中的应用。在多年疾病折磨之后,帕斯卡于39岁时在巴黎去世。
\subsection{早年生活与教育}
帕斯卡出生于法国奥弗涅地区的克莱蒙费朗,地处中央高原。他在三岁时失去了母亲安托瓦内特·贝贡。\(^\text{[12]}\)他的父亲艾蒂安·帕斯卡也是一位业余数学家,是当地的法官,同时是“法袍贵族”成员。帕斯卡有两个姐妹,妹妹叫雅克琳,姐姐叫吉尔贝特。
\begin{figure}[ht]
\centering
\includegraphics[width=6cm]{./figures/57329a3ddbf9ceaa.png}
\caption{} \label{fig_BLSpsk_1}
\end{figure}
迁居巴黎
1631年,也就是妻子去世五年后,\(^\text{[13]}\)艾蒂安·帕斯卡带着孩子们搬到了巴黎。这户新到的家庭很快雇佣了女仆路易丝·德福,后者最终成为了这个家庭的重要成员。艾蒂安终身未再婚,决定亲自教育自己的孩子们。

年幼的帕斯卡展现出非凡的智力,特别是在数学和科学方面展现出惊人的天赋。\(^\text{[14]}\)艾蒂安原本试图阻止儿子接触数学;然而在12岁时,帕斯卡凭借自己的努力,用木炭在瓷砖地板上重新推导出了欧几里得的前32条几何命题,因此他得到了《几何原本》的一本副本。\(^\text{[15]}\)

\textbf{关于圆锥曲线的论文}

帕斯卡尤其感兴趣的一本著作是德扎格关于圆锥曲线的研究。沿着德扎格的思路,年仅16岁的帕斯卡撰写了一篇短小的论文《圆锥曲线试作》(法语原名 Essai pour les coniques),用以证明一个被称为“神秘六边形”的命题,并将这篇他人生中第一篇严肃的数学论文寄给了巴黎的梅尔森神父。这一定理今天仍以“帕斯卡定理”之名为人熟知:它断言,若一个六边形内接于一个圆或一般的圆锥曲线中,则其对边延长线的交点三三成对后共线,这条直线称为“帕斯卡线”。

帕斯卡的工作极其早熟,以至于笛卡尔一度坚信是帕斯卡的父亲写下了这篇文章。当梅尔森向他确认确实是帕斯卡之子所作时,笛卡尔冷笑一声轻蔑地回应道:“我并不觉得奇怪,他在圆锥曲线方面给出的证明比古人更为得当”,并补充道:“但关于这一主题,还有些内容是一个16岁的孩子无论如何也不可能想到的。”\(^\text{[16]}\)
\subsubsection{离开巴黎}
在当时的法国,官职是可以买卖的。1631年,艾蒂安·帕斯卡以65,665里弗的价格出售了自己在辅助法院担任的二级主席职务\(^\text{[17]}\)。这笔资金被投资于一项政府债券,虽然不能说奢华,但足以为帕斯卡一家在巴黎提供一种安稳的生活。然而到了1638年,红衣主教黎塞留为了筹措继续打三十年战争的资金,违约了这批政府债券。艾蒂安·帕斯卡的财富于是骤减,从近66,000里弗跌到了不到7,300里弗。

像许多其他人一样,艾蒂安因反对黎塞留的财政政策最终不得不逃离巴黎,留下他的三个孩子由邻居圣托夫人照料。这位圣托夫人美貌动人,过往风流,却也经营着全法国最闪耀、最具文化气息的沙龙之一。直到某次雅克琳在一场儿童剧中表演出色,而黎塞留正好在场观看,艾蒂安才得以被赦免。不久之后,艾蒂安重新获得了红衣主教的青睐,并于1639年被任命为鲁昂市的国王税务专员——而当时该市的税务记录由于多次民变,已是一片混乱。
\subsubsection{帕斯卡计算器}

1642年,为了减轻父亲在税务工作中那无休止的、令人精疲力竭的计算与重新计算(年轻的帕斯卡也参与了这项工作),帕斯卡在尚未满19岁时便设计并制造出一种能进行加法和减法运算的机械计算器,被称为“帕斯卡计算器”或“帕斯卡机”。在现存的八台帕斯卡机中,四台收藏于巴黎的工艺与技术博物馆,另有一台在德国德累斯顿的茨温格宫博物馆展出,均为他原始设计的机械计算器之一\(^\text{[18]}\)。

尽管这些机器是后续400年机械计算技术发展的先驱,从某种意义上说也可被视为计算机工程领域的前身,但这台计算器并未获得商业上的巨大成功。一方面是因为它在实际使用中仍显笨重,更主要的原因可能是其造价极其昂贵,帕斯卡机最终沦为法国乃至欧洲富人手中的玩具与地位象征。帕斯卡在接下来的十年中持续改进其设计,据他自己所述,有大约50台机器是按他的设计制造的\(^\text{[19]}\)。在随后的十年间,他共亲自制造了20台成品机器\(^\text{[20]}\)。
