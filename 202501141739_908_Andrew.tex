% 安德鲁·怀尔斯(综述)
% license CCBYSA3
% type Wiki

本文根据 CC-BY-SA 协议转载翻译自维基百科\href{https://en.wikipedia.org/wiki/Andrew_Wiles}{相关文章}。

\begin{figure}[ht]
\centering
\includegraphics[width=6cm]{./figures/7082cdefe31fa317.png}
\caption{2005年的怀尔斯} \label{fig_Andrew_1}
\end{figure}
安德鲁·约翰·怀尔斯爵士(Sir Andrew John Wiles,1953年4月11日出生)是英国数学家,现任牛津大学皇家学会研究教授,专攻数论。他最著名的成就是证明了费马大定理,因此获得了2016年阿贝尔奖和2017年科普利奖,并于2000年被任命为英国帝国勋章骑士指挥官。2018年,怀尔斯被任命为牛津大学首任Regius数学教授。怀尔斯还是1997年麦克阿瑟学者奖得主。

怀尔斯出生于剑桥,父亲是神学家莫里斯·弗兰克·怀尔斯,母亲是帕特里夏·怀尔斯。在尼日利亚度过了大部分童年后,怀尔斯对数学产生了浓厚兴趣,特别是对费马大定理产生了兴趣。怀尔斯于1974年从牛津大学毕业后,开始致力于统一伽罗瓦表示、椭圆曲线和模形式,最初从巴里·马祖尔对岩泽理论的推广开始。1980年代初期,怀尔斯在剑桥大学工作了几年,然后前往普林斯顿大学,在那里他研究并应用了希尔伯特模形式。1986年,在阅读肯·里贝特关于费马大定理的开创性工作后,怀尔斯着手证明半稳定椭圆曲线的模性定理,这意味着费马大定理得以证明。到1993年,他已经能够说服一位知识渊博的同事相信他找到了费马大定理的证明,尽管随后发现了一个缺陷。1994年9月19日,怀尔斯和他的学生理查德·泰勒通过一番洞察力克服了这个缺陷,并于1995年发表了结果,广受赞誉。

在证明费马大定理的过程中,怀尔斯为数学家们发展了新的工具,使他们能够开始统一不同的思想和定理。他的前学生泰勒和另外三位数学家于2000年证明了完整的模性定理,使用了怀尔斯的工作。在2016年获得阿贝尔奖时,怀尔斯回顾了自己的遗产,并表示他不仅证明了费马大定理,还推动了整个数学领域朝着拉格朗日计划的方向发展,从而统一了数论。
\subsection{教育与早年生活}  
Wiles于1953年4月11日出生在英国剑桥,是Maurice Frank Wiles(1923–2005)和Patricia Wiles( née Mowll)的儿子。1952年至1955年间,他的父亲在剑桥的Ridley Hall担任牧师,后来成为牛津大学的神学讲座教授。[6]  

Wiles的正式学业从他和父母一起居住在尼日利亚时开始。然而,根据他父母写的信件,在他本应上学的最初几个月里,他拒绝去上学。从这个事实中,Wiles自己得出结论,认为在他最早的几年里,他并不热衷于待在学术机构。在2021年接受Nadia Hasnaoui采访时,他表示他相信这些信件,然而他自己却不记得有任何时候不喜欢解决数学问题。[7]  

Wiles曾就读于剑桥的King's College School[8] 和The Leys School, Cambridge[9]。Wiles在1999年接受WGBH-TV采访时提到,他在10岁时从学校回家的路上遇到了《费马最后定理》。他停在当地图书馆,发现了埃里克·坦普尔·贝尔的《最后的问题》一书,讲述了这个定理。[10] 因为这个定理表述如此简单,以至于他,一个10岁的孩子都能理解,但却没人能证明出来,他对它产生了浓厚的兴趣,决定成为第一个证明它的人。然而,他很快意识到自己的知识还不够,因此放弃了童年的梦想,直到33岁时,Ken Ribet在1986年证明了ε猜想,这一猜想与Gerhard Frey早前与费马方程的联系重新把这个梦想带回了他的注意。[11]
\subsection{早期生涯}
1974年,怀尔斯在牛津大学莫顿学院获得了数学学士学位。[6] 怀尔斯的研究生研究由约翰·科茨指导,始于1975年夏季。他们共同研究了椭圆曲线的算术问题,特别是通过岩泽理论的方法来研究复乘法的椭圆曲线。他还与巴里·马祖一起研究了岩泽理论在有理数上的主猜想,随后,他将这一结果推广到完全实数域。[12][13]

1980年,怀尔斯在剑桥大学克莱尔学院获得了博士学位。[3] 在1981年于普林斯顿大学高级研究院的停留之后,怀尔斯成为了普林斯顿大学数学教授。[14]

1985至1986年,怀尔斯成为古根海姆学者,先后在巴黎的高等科学研究所和巴黎高等师范学校工作。[14]

1989年,怀尔斯被选为皇家学会会员。根据他的选举证书,他当时正在“研究附加于希尔伯特模形式的ℓ-代数表示,并已将这些应用于证明完全实数域的环面扩展的‘主猜想’”。[12]
\subsection{费马大定理的证明}
1988年至1990年,怀尔斯是牛津大学皇家学会研究教授,随后他回到了普林斯顿大学。1994年至2009年,怀尔斯担任普林斯顿大学尤金·希金斯教授。

自1986年中期起,基于此前几年由格哈德·弗雷、让-皮埃尔·塞尔和肯·里贝特等人的持续进展,逐渐清楚了费马大定理(即没有三个正整数a、b、c能满足公式 \(a^n + b^n = c^n\) 对于任何大于2的整数n)可以作为模形式定理的一种有限形式的推论来证明(当时该定理尚未被证明,并且被称为“谷山–志村–维尔猜想”)。[15] 模形式定理涉及椭圆曲线,这是怀尔斯的专业领域,并且声明所有此类曲线都有与之相关联的模形式。[16][17] 这些曲线可以看作是数学对象,类似于一个圆环表面的解,如果费马大定理是错误的且存在解,“会产生一种特殊的曲线”。因此,证明该定理就涉及到证明不存在这种曲线。[18]

该猜想被当时的数学家认为是重要的,但极其困难,甚至可能是无法证明的。[19]: 203–205, 223, 226 例如,怀尔斯的前导师约翰·科茨表示,这看起来“几乎不可能证明”,[19]: 226 而肯·里贝特则认为自己是“绝大多数认为[它]完全无法触及的人的一员”,并补充道,“安德鲁·怀尔斯可能是地球上为数不多的几位敢于梦想你实际上可以去证明[它]的人之一。”[19]: 223 

尽管如此,怀尔斯在童年时便对费马大定理产生了浓厚兴趣,他决定接受挑战,至少证明到弗雷的曲线所需的程度。[19]: 226 他将自己所有的研究时间都投入到这个问题上,近乎完全保密,发布的工作只以小段落的形式发布为独立的论文,并且只与妻子分享。[19]: 229–230 

怀尔斯的研究涉及通过反证法证明费马大定理,里贝特在1986年的工作中发现,如果该定理为真,它将有一个椭圆曲线以及相关的模形式。怀尔斯首先假设该定理是错误的,然后基于这个假设,利用里贝特的定理(该定理指出,如果n是素数,则没有这样的椭圆曲线具有模形式,因此不存在费马方程的奇素数反例),矛盾地反驳了谷山–志村–维尔猜想。怀尔斯还证明了该猜想适用于与费马方程相关的特定案例——半稳定椭圆曲线。换句话说,怀尔斯发现谷山–志村–维尔猜想在费马方程的情况下是成立的,里贝特的发现(即如果猜想对半稳定椭圆曲线成立,则费马大定理为真)得到了验证,从而证明了费马大定理。[20][21][15]

1993年6月,怀尔斯在剑桥的一次会议上首次向公众展示了他的证明。纽约时报的吉娜·科拉塔总结了这一讲座如下:

他每周一、周二和周三进行一次讲座,标题为“模形式、椭圆曲线和伽罗瓦表示”。讲座标题没有暗示会讨论费马大定理,里贝特博士说... 最后,在第三次讲座的结束时,怀尔斯博士总结道,他已证明了谷山猜想的一般情况。然后,似乎是作为附带提及,他提到这意味着费马大定理为真。Q.E.D.[18]

1993年8月,发现证明中存在多个问题,涉及到塞尔默群的性质和使用一种称为欧拉系统的工具。[22][23] 怀尔斯尝试并未成功修正他的证明。根据怀尔斯的说法,绕过这一问题的关键想法出现在1994年9月19日,当时他几乎要放弃了。这个绕过方法使用了伽罗瓦表示来替代椭圆曲线,将问题简化为类数公式并解决了它,此外,还使用了维克托·科里瓦金的思想来修正马提亚斯·弗拉赫在岩泽理论中的方法。[23][22] 怀尔斯和他的前学生理查德·泰勒一起,发表了第二篇论文,其中包含了绕过部分,从而完成了证明。这两篇论文于1995年5月在《数学年刊》上专门刊登。[24][25]
\subsection{后期事业}
2011年,怀尔斯重新加入牛津大学,担任皇家学会研究教授。[14]

2018年5月,怀尔斯被任命为牛津大学的Regius数学教授,成为该校历史上的第一位此职位的教授。[4]
\subsection{遗产}  
怀尔斯的工作在数学的多个领域得到了应用。特别是,1999年,他的三位前学生理查德·泰勒、布莱恩·康拉德和弗雷德·戴蒙德,与克里斯托夫·布雷伊尔合作,在怀尔斯的证明基础上证明了完整的模形式定理。[26][15] 怀尔斯的博士生还包括曼朱尔·巴尔加瓦(2014年菲尔兹奖得主)、埃胡德·德·沙利特、瑞塔布拉塔·穆恩希(SSB奖和ICTP拉马努金奖得主)、卡尔·鲁宾(维拉·鲁宾的儿子)、克里斯托弗·斯金纳和维纳亚克·瓦特萨尔(2007年科克塞特–詹姆斯奖得主)。

2016年,在获得阿贝尔奖时,怀尔斯谈到了他对费马大定理的证明:“解决它的方法开辟了攻击当代数学中一个被称为Langlands计划的巨大猜想网络的新途径,这一宏大的愿景试图统一数学的不同分支。它为我们提供了一个全新的视角来看待这一问题。”[5]