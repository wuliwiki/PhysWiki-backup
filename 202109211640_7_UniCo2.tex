% 一致收敛与极限换序
% 一致收敛|换序|limit|微积分|数学分析

\pentry{一致收敛\upref{UniCnv}}

函数、数列等有多个自变量存在时,我们有可能会需要考虑多重极限.比如说对于一个二元函数$f(x, y)$,设它的定义域是$x>0, y>0$,那么我们如何计算$x\to 0, y\to 0$时$f$的极限值呢?我们可以用以下式子来计算:
\begin{equation}\label{UniCo2_eq1}
\lim\limits_{y\to 0}\lim\limits_{x\to 0}f(x, y)
\end{equation}

其中,对于任意$y>0$,$f(x, y)$可以认为是关于$x$的一元函数,这样我们就可以计算出$\lim\limits_{x\to 0}f(x, y)$.对于每个$y>0$,可以定义$g(y)=\lim\limits_{x\to 0}f(x, y)$,因此对二元函数$f$进行第一个求极限后,得到的是一个一元函数$g$.这样,我们同样也能计算出$\lim\limits_{y\to 0}g(y)$,这也就是\autoref{UniCo2_eq1} .

从直观的几何角度来理解,$\lim\limits_{x\to 0}f(x, y)$就像是求出了$x$轴上的一个函数,然后$\lim\limits_{y\to 0}\lim\limits_{x\to 0}f(x, y)$就是这个函数的一个极限.

我们也可以反过来,用
\begin{equation}\label{UniCo2_eq2}
\lim\limits_{x\to 0}\lim\limits_{y\to 0}f(x, y)
\end{equation}
来计算$f$的二重极限,这个时候就相当于先求了$y$轴上的一个函数,再求它的极限.

随之而来的问题是,\autoref{UniCo2_eq1} 和\autoref{UniCo2_eq2} 的值一样吗?换个说法就是,$f(x, y)$的二重极限可以交换次序吗?答案时“不一定”,取决于函数的性质.\autoref{UniCo2_ex1} 就是一个反例.

\begin{example}{}\label{UniCo2_ex1}

\end{example}






















