% 路德维希·玻尔兹曼(综述)
% license CCBYSA3
% type Wiki

本文根据 CC-BY-SA 协议转载翻译自维基百科\href{https://en.wikipedia.org/wiki/Ludwig_Boltzmann}{相关文章}。

\begin{figure}[ht]
\centering
\includegraphics[width=6cm]{./figures/ed66f21add0c5482.png}
\caption{1902年的玻尔兹曼} \label{fig_BRZM_1}
\end{figure}
\textbf{路德维希·爱德华·玻尔兹曼}(Ludwig Eduard Boltzmann,1844年2月20日-1906年9月5日)是奥地利的物理学家和哲学家。他的最大成就包括统计力学的发展和热力学第二定律的统计解释。1877年,他提出了当前的熵定义:\(S = k_{\rm B}\ln\Omega \)其中,Ω是系统能量等于宏观系统能量的微观状态数,解释为衡量系统统计无序度的一个指标。马克斯·普朗克将常数 \( k_B \) 命名为玻尔兹曼常数。

统计力学是现代物理学的基石之一。它描述了宏观观测(如温度和压力)如何与围绕平均值波动的微观参数相关。它将热力学量(如比热容)与微观行为联系起来,而在经典热力学中,唯一可用的方式是为不同材料测量并列出这些量。
\subsection{传记}  
\subsubsection{童年与教育}  
尔兹曼出生在维也纳的郊区厄尔德贝格(Erdberg),来自一个天主教家庭。他的父亲路德维希·乔治·玻尔兹曼(Ludwig Georg Boltzmann)是一名税务官员。他的祖父从柏林迁至维也纳,是一位钟表制造商,而玻尔兹曼的母亲凯瑟琳·保尔恩芬德(Katharina Pauernfeind)则来自萨尔茨堡。玻尔兹曼在家中接受教育,直到十岁才开始正式上学,之后在上奥地利州的林茨市读高中。15岁时,玻尔兹曼的父亲去世。

1863年起,玻尔兹曼在维也纳大学学习数学和物理学。他于1866年获得博士学位,并于1869年获得讲授资格(venia legendi)。玻尔兹曼与物理学研究所所长约瑟夫·斯特凡(Josef Stefan)密切合作,正是斯特凡将玻尔兹曼引入了麦克斯韦的研究成果。
\subsubsection{学术生涯} 
1869年,玻尔兹曼在25岁时,凭借约瑟夫·斯特凡的推荐信,[9] 被任命为格拉茨大学(位于施蒂利亚省)数学物理学全职教授。1869年,他在海德堡与罗伯特·本森(Robert Bunsen)和莱奥·凯尼茨贝格(Leo Königsberger)合作工作了几个月,随后在1871年与古斯塔夫·基尔霍夫(Gustav Kirchhoff)和赫尔曼·冯·亥姆霍兹(Hermann von Helmholtz)在柏林合作。1873年,玻尔兹曼加入维也纳大学,担任数学教授,并在此工作直到1876年。
\begin{figure}[ht]
\centering
\includegraphics[width=8cm]{./figures/2f19a04ecc7fbc4a.png}
\caption{1887年,路德维希·玻尔兹曼与格拉茨的同事们:(站立,左起)能斯特(Nernst)、斯特赖因茨(Streintz)、阿伦纽斯(Arrhenius)、海克(Hiecke);(坐着,左起)奥林格(Aulinger)、艾廷斯豪森(Ettingshausen)、玻尔兹曼(Boltzmann)、克莱门奇奇(Klemenčič)、豪斯曼宁格(Hausmanninger)。} \label{fig_BRZM_2}
\end{figure}
1872年,在女性尚未被允许进入奥地利大学之前,玻尔兹曼遇到了亨丽埃特·冯·艾根特勒(Henriette von Aigentler),她是一位有志成为数学和物理学教师的年轻女性,居住在格拉茨。她曾被拒绝非正式旁听讲座的许可。玻尔兹曼支持她提出上诉,最终她获得了成功。1876年7月17日,路德维希·玻尔兹曼与亨丽埃特结婚,他们育有三位女儿:亨丽埃特(1880年)、伊达(1884年)和埃尔泽(1891年);以及一个儿子,阿图尔·路德维希(1881年)。[10]玻尔兹曼回到格拉茨,担任实验物理学教席。在格拉茨的学生中,有斯万特·阿伦纽斯(Svante Arrhenius)和瓦尔特·能斯特(Walther Nernst)。[11][12] 他在格拉茨度过了14个快乐的年头,并且在那里发展了他关于自然的统计学概念。

玻尔兹曼于1890年被任命为德国巴伐利亚州慕尼黑大学的理论物理学教席。

1894年,玻尔兹曼继承了他的导师约瑟夫·斯特凡(Joseph Stefan)的职位,成为维也纳大学的理论物理学教授。[13]
\subsubsection{最后的岁月与去世}
玻尔兹曼在最后的岁月里投入了大量精力为自己的理论辩护。[14] 他与一些维也纳的同事关系不和,尤其是恩斯特·马赫(Ernst Mach),后者在1895年成为了哲学与科学史教授。那一年,乔治·赫尔姆(Georg Helm)和威廉·奥斯特瓦尔德(Wilhelm Ostwald)在吕贝克的一次会议上提出了他们的能量学观点。他们认为,能量而非物质才是宇宙的主要组成部分。在这场辩论中,玻尔兹曼的立场得到了其他物理学家的支持,尤其是那些支持他原子理论的物理学家。[15] 1900年,玻尔兹曼应威廉·奥斯特瓦尔德的邀请前往莱比锡大学。奥斯特瓦尔德为玻尔兹曼提供了物理学教授的职位,这个职位在古斯塔夫·海因里希·维德曼去世后空缺。由于健康原因,马赫退休后,玻尔兹曼于1902年返回维也纳。[14] 1903年,玻尔兹曼与古斯塔夫·冯·艾舍里奇(Gustav von Escherich)和埃米尔·穆勒(Emil Müller)共同创立了奥地利数学学会。他的学生包括卡尔·普里布拉姆(Karl Přibram)、保罗·埃伦费斯特(Paul Ehrenfest)和莉泽·迈特纳(Lise Meitner)。[14]

在维也纳,玻尔兹曼教授物理学,同时也讲授哲学。玻尔兹曼的自然哲学讲座非常受欢迎,并引起了相当大的关注。他的第一次讲座获得了巨大的成功。尽管选择了最大的讲堂,听众还是站满了楼梯。由于玻尔兹曼哲学讲座的巨大成功,皇帝邀请他参加宫殿的接待[具体时间待定]。[16]

1905年,玻尔兹曼应邀在加利福尼亚大学伯克利分校的夏季学期举办了一系列讲座,他在一篇受欢迎的文章《一位德国教授的埃尔多拉多之旅》中描述了这一经历。[17]

1906年5月,玻尔兹曼的精神状况恶化,院长在信中将其描述为“严重的神经衰弱”。这一症状表明他可能患有今天所诊断的双相情感障碍。[14][18] 四个月后,他在1906年9月5日自杀,死亡方式是上吊。那时他正与妻子和女儿在杜伊诺(位于特里斯特附近,彼时属奥地利)度假。[19][20][21][18] 他被安葬在维也纳的中央公墓(Zentralfriedhof)。他的墓碑上刻有玻尔兹曼的熵公式:\( S = k \cdot \log W \)[14]