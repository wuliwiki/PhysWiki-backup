% 树与林
% keys 树|林|森林
% license Usr
% type Tutor

\pentry{图的顶点度\nref{nod_DGraph},图的连通性\nref{nod_GraCon}}{nod_0546}
林是不含\aref{圈}{def_PatCyc_2}的图,树是\enref{连通}{GraCon}的林。也就是说,林的\aref{连通分支}{def_GraCon_1}都是树。

\begin{definition}{树,林}
设 $G$ 是图。若 $G$ 上无圈,则称 $G$ 为\textbf{林}(forest);若 $G$ 无圈且连通,则称 $G$ 为\textbf{树}(tree)。
\end{definition}


\begin{definition}{叶子,内部点}
设 $G$ 是树,$x\in V(G)$。若 $d_G(v)=1$ (\autoref{def_DGraph_1}),则称 $x$ 是 $G$ 的\textbf{叶子}(leaf)。树中非叶子的点称为\textbf{内部点}。
\end{definition}

由\autoref{ex_PatCyc_1} 可知,林和树都是\aref{简单图}{def_Graph_2}。


\begin{theorem}{}
下面命题是等价的:
\begin{enumerate}
\item $T$ 是树;
\item $T$ 中任意两个不同点被 $T$ 的唯一路连接;
\item $T$ 是极小连通的,即 $T$ 是连通的,但对任一边 $e\in E(T)$,$T-e$(记号见\autoref{def_Graph_5}) 不连通;
\item $T$ 是极大无圈的,即 $T$ 不包含圈,但任意不相邻的点 $x,y\in V(T)$,$T+xy$ 包含圈(其中 $xy$ 是连接 $xy$ 的边)。
\end{enumerate}

\end{theorem}

\textbf{证明:}
1. $1\Rightarrow 2$:设 $\forall x,y\in T$。同于树是连通的,所以存在 $x$ 到 $y$ 的(长为 $l$)路 $R=xe_1x_1\cdots y$。设 $R'=xe'_1x_1'\cdots e'_n y$ 是连接 $x,y$ 的另一(长为 $n$)路。由于 $T$ 无圈,所以链 $xe_1\cdots y|e'_n\cdots e'_1x$ 中在 $|$ 两边一定有两个内点相同,否则该链就是一个圈。由于左边和右边分别是路,因此两个内点只能一个在一边。设这两个内点是 $x_{i},x'_{j}$,于是 $W(x_{i},x'_{j})$ 是 $x_{i}x'_{j}$ 链。

由于 $W(x_{i},x'_{j})$ 端点相同,我们仍可以重复上面的论述。若 $i\neq j$,不是一般性
设 $i>j$,于是至多重复 $l-i+1$ 次后得到链 $x_{l-1}e_ly|e_n'\cdots x'_{i_{l-i+2}}$,继续重复操作,则将出现 $y=y',y'\in R'$。这和 $R'$ 是路矛盾。这一矛盾表明只能是 $x_i=x_i'$。若 $n\neq l$ 则重复 $k-l$ 次又会出现上面的情况,于是只能是 $n=l$。于是重复最多重复 $n-1$ 次,我们就会发现 $x_{n-1}=x'_{n-1}$。

然后对路 $R_1=R-x_{n-1}$ 继续上述操作,可以得到 $x_{n-2}=x_{n-2}'$。如此重复,最终得到 $R=R'$。

2.$2\Rightarrow 3$:由2可知 $T$ 是连通的。$\forall e\in E(T)$,设 其端点为 $x,y$,无圈表明 $x\neq y$。而 $x,y\in V(T-e)$,$xey$ 是连接 $x,y$ 的唯一路,因此 



\textbf{证毕!}






















