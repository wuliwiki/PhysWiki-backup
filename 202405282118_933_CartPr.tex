% 笛卡尔积
% keys 直积|有序数对|笛卡尔积
% license Usr
% type Tutor

在刚刚接触到乘法计算时,作为一个运算结果,“积”是作为与“乘法”相关的概念被引入的。后来,随着对向量的学习的深入,内积和外积逐渐也成为了熟悉的概念,二者分别与点乘($\cdot$)和叉乘($\times$)相对应。或许,“卷积”和“张量积”等概念也偶尔会出现在你的视野中。他们往往是与一个逐渐抽象的“乘法”相对应,说他逐渐抽象,是因为他与我们熟知的数的乘法的样子和计算方法相去甚远。而还称呼它是乘法,是因为某种程度上,它保留了乘法的一些特性。

下面会涉及到一点点的物理知识:在物理上,常常会有通过“乘法”来定义一个新的物理量,比如:功是力与位移的乘积,力矩是力与力臂的乘积,电路中功率是电流与电压的乘积(先忽略这个乘积具体的形式)等。这个新的物理量与原有的两个物理量之间都存在关系。而“加法”往往是在一个概念内部量的多少的计算,基本是不涉及其他概念的。

数学上有一个概念叫作\textbf{直积}(direct product),一般使用它来组合两个同类的已知对象,来定义新对象,比如:集合、群、模、拓扑空间等。而作用在两个集合上的直积便称为\textbf{笛卡尔积}(Cartesian product)。


\textbf{笛卡尔积}是一个集合领域的概念,经常通过对。