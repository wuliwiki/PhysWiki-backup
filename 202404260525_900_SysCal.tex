% Linux 的 System Call 笔记
% license Xiao
% type Note

\begin{issues}
\issueDraft
\end{issues}

\subsection{理论基础}
\begin{itemize}
\item 参考 \href{https://en.wikipedia.org/wiki/Linux_kernel_interfaces}{Linux Kernel Interfaces - Wikipedia}, \href{https://en.wikipedia.org/wiki/System_call}{System Call - Wikipedia}。
\item \href{https://en.wikipedia.org/wiki/User_space_and_kernel_space}{User Space - Wikipedia}: 系统的虚拟内存被划分为 kernel space 和 user space。 前者跑的是系统内核和一些驱动, 剩下的所有程序和另一些驱动都在后者。
\item 系统内核的两种 API: 一个是 \textbf{Linux API("kernel–user space" API)}(user-space 程序需要的), 另一个是 \textbf{"kernel internal" API}(内核内部使用的)。
\item Linux API 由内核的 \textbf{System Call Interface(SCI)} 以及 \textbf{GNU C Library (glibc)} 中的函数构成。 glibc 提供 SCI 的一个 wrapper。
\item 内核大约一共提供 380 个 system call。 常见的如 \verb|open, read, write, close, wait, exec, fork, exit, kill|
\item 一些常用的工具用于诊断程序的 system call 如: \verb|strace, ftrace, truss|
\item 六类 system call: \verb|Process control|, \verb|File management|, \verb|Device management|, \verb|Information maintenance|, \verb|Communication|, \verb|Protection|
\item 类 Unix 系统的 system call 都是在 \verb|kernel mode| 执行的。
\end{itemize}

\subsection{编程实践}
\begin{itemize}
\item 参考\href{https://jameshfisher.com/2018/02/19/how-to-syscall-in-c/}{这篇教程}。
\item \verb|<unistd.h>| 头文件提供各种 system call, 例如 \verb|write(1, "hello, world!\n", 14);|
\item \verb|<sys/syscall.h>| 是更底层的工具, 例如提供函数 \verb|syscall(SYS_write, 1, "hello, world!\n", 14);|。 而 \verb|syscall()| 是用机器码定义的。
\end{itemize}

