% 跃迁概率(一阶微扰)
% keys 一阶微扰理论

\begin{issues}
\issueDraft
\end{issues}

\pentry{类氢原子的波函数\upref{HWF}}

本文使用原子单位制\upref{AU}.一阶微扰理论就是单光子电离, 即两能级之间的能量等于单个光子的能量.

\subsection{长度规范下的微扰跃迁理论}
含时微扰理论(\autoref{TDPT_eq10}~\upref{TDPT}) 为
\begin{equation}\label{HionCr_eq1}
c_i(t) = -\I \int_{-\infty}^t \mel{i}{H'(t)}{j} \E^{\I\omega_{ij} t} \dd{t}
\end{equation}
长度规范\upref{LenGau}中,
\begin{equation}
H'(t) = -q\bvec {\mathcal E}(t) \vdot \bvec r
\end{equation}
$\bvec {\mathcal E}$ 只是 $t$ 的函数, 可以分离
\begin{equation}\label{HionCr_eq9}
\mel{i}{H'(t)}{j} = -q\bvec {\mathcal E}(t) \vdot \mel{i}{\bvec r}{j}
\end{equation}
令电场的傅里叶变换(\autoref{FTExp_eq6}~\upref{FTExp})为
\begin{equation}
\tilde {\bvec {\mathcal E}}(\omega) = \frac{1}{\sqrt{2\pi}} \int_{-\infty}^{\infty} \bvec {\mathcal E}(t) \E^{-\I\omega t} \dd{t}
\end{equation}
则\autoref{HionCr_eq9} 代入\autoref{HionCr_eq1} 得
\begin{equation}\label{HionCr_eq7}
c_i(t) = \I q \mel{i}{\bvec r}{j} \vdot \int_{-\infty}^t \bvec {\mathcal E}(t) \E^{\I\omega_{ij} t} \dd{t} = \I \sqrt{2\pi} q \mel{i}{\bvec r}{j} \vdot \tilde {\bvec {\mathcal E}}(-\omega_{ij})
\end{equation}
令 $\tilde {\bvec {\mathcal E}}(\omega) = \tilde {{\mathcal E}}(\omega)\uvec e$, 跃迁概率为
\begin{equation}\label{HionCr_eq2}
P_{j\to i} = \abs{c_i(t)}^2 = 2\pi q^2 \abs{\uvec e \vdot \mel{i}{\bvec r}{j}}^2 \abs{\tilde {{\mathcal E}}(\omega_{ij})}^2
\end{equation}
结合(\autoref{WpEng_eq3}~\upref{WpEng})
\begin{equation}
s(\omega) = 2c\epsilon_0 \abs{\tilde {{\mathcal E}}(\omega)}^2
\end{equation}
得
\begin{equation}\label{HionCr_eq6}
P_{j\to i} = \frac{\pi q^2}{c\epsilon_0} \abs{\uvec e \vdot\mel{i}{\bvec r}{j}}^2 s(\omega_{ij})
\end{equation}
对于束缚态 $\ket{i}$, $P_{j\to i}$ 是从 $\ket{j}$ 跃迁到 $\ket{i}$ 的概率; 而对于连续态的 $\ket{i}$ (如原子电离), 若 $\ket{i}$ 对应出射方向的渐进动量 $\bvec k$, 那么 $P_{j\to i}$ 是 $\bvec k$ 空间的三维概率密度分布函数\upref{PTCont}.

对氢原子的具体计算见\autoref{HyIon2_eq1}~\upref{HyIon2}.

\subsection{速度规范下的微扰跃迁理论}
\footnote{参考\cite{Merzbacher} 含时微扰相关章节.}速度规范\upref{LVgaug}中,
\begin{equation}
H'(t) = -\frac{q}{m}\bvec A \vdot \bvec p = \frac{\I q}{m}\bvec A \vdot \grad
\end{equation}
$\bvec A$ 只是 $t$ 的函数, 可以分离
\begin{equation}
\mel{i}{H'(t)}{j} = -\frac{q}{m}\bvec A(t) \vdot \mel{i}{\bvec p}{j}
\end{equation}
令
\begin{equation}
\tilde {\bvec A}(\omega) = \frac{1}{\sqrt{2\pi}} \int \bvec A(t) \E^{-\I\omega t} \dd{t}
\end{equation}
代入\autoref{HionCr_eq1} 得
\begin{equation}\label{HionCr_eq4}
c_i(t) = \frac{\I q}{m} \mel{i}{\bvec p}{j} \vdot \int_{-\infty}^t  \bvec A(t) \E^{\I\omega_{ij} t} \dd{t} = \I\sqrt{2\pi}\frac{q}{m} \mel{i}{\bvec p}{j} \vdot \tilde {\bvec A}(-\omega_{ij})
\end{equation}
令 $\tilde {\bvec A}(\omega) = \tilde {A}(\omega)\uvec e$, 则跃迁概率为
\begin{equation}\label{HionCr_eq3}
P_{j\to i} = \abs{c_i(t)}^2 = \frac{2\pi q^2}{m^2} \abs{\uvec e \vdot\mel{i}{\bvec p}{j}}^2 \abs{\tilde {A}(\omega_{ij})}^2
\end{equation}
结合波包的频谱公式(\autoref{WpEng_eq5}~\upref{WpEng} 变为原子单位)
\begin{equation}
s(\omega) = \frac{c}{2\pi} \omega^2 \abs{\tilde {A}(\omega_{ij})}^2
\end{equation}
\begin{equation}\label{HionCr_eq5}
P_{j\to i} = \frac{4\pi^2 q^2}{c m^2 \omega_{ij}^2} \abs{\uvec e \vdot\mel{i}{\bvec p}{j}}^2 s(\omega_{ij})
\end{equation}

\subsubsection{两种规范比较}
注意 $\ket{i}, \ket{j}$ 是没有电磁场时的能量本征态, 波函数与规范无关. 把\autoref{DipEle_eq3}~\upref{DipEle} 和\autoref{WpEng_eq4}~\upref{WpEng} 带入\autoref{HionCr_eq4} 可以证明两种规范等效(\autoref{HionCr_eq7}  等于\autoref{HionCr_eq4}). 但是如果例如 $\ket{i}$ 是平面波, 则不同规范结果不同.

\subsection{电离截面}
截面可以想象成是电磁波传播方向垂直放置的一块面积为 $\sigma(\omega)$ 的面元. 使得原子从电磁波中吸收的功率恰好等于电磁波经过该面元的功率. 对于波包, 单位频率下原子吸收的能量为
\begin{equation}
\dv{E}{\omega} = \sigma(\omega) s(\omega)
\end{equation}
如果再对立体角微分得
\begin{equation}
\pdv{E}{\omega}{\Omega} = \pdv{\sigma}{\Omega} s(\omega)
\end{equation}
当波包经过以后, $\bvec A = 0$, 有 $\omega_{ij} = k^2/(2m) + I_0$, $-I_0$ 是束缚态 $\ket{j}$ 的能量.
\begin{equation}
E = \int \pdv{\sigma}{\Omega} s(\omega) \dd{\frac{k^2}{2m}}\dd{\Omega}
\end{equation}
\begin{equation}
E = \int \omega P_{j\to i}(\bvec k) k^2\dd{\Omega}\dd{k}
\end{equation}
其中 $\omega$ 是光子能量. 对比得
\begin{equation}
\pdv{\sigma}{\Omega} = \frac{km \omega}{s(\omega)} P_{j\to i}(\bvec k)
\end{equation}
长度规范下有(\autoref{HionCr_eq6} )
\begin{equation}
\pdv{\sigma}{\Omega} = \frac{4\pi^2 m\omega k q^2}{c} \abs{\mel{i}{\uvec e \vdot\bvec r}{j}}^2
\end{equation}
速度规范下有(\autoref{HionCr_eq5} )
\begin{equation}\label{HionCr_eq8} % 已验证与 Merzbaucher eq 19.86 完全相同
\pdv{\sigma}{\Omega} = \frac{4\pi^2 k q^2}{c m \omega} \abs{\mel{i}{\uvec e \vdot \bvec p}{j}}^2
\end{equation}
