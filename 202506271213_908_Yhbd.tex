% 约翰·巴丁(综述)
% license CCBYSA3
% type Wiki

本文根据 CC-BY-SA 协议转载翻译自维基百科 \href{https://en.wikipedia.org/wiki/John_Bardeen}{相关文章}。

\begin{figure}[ht]
\centering
\includegraphics[width=6cm]{./figures/c261931e347b9e06.png}
\caption{巴丁于1956年} \label{fig_Yhbd_1}
\end{figure}
约翰·巴丁(John Bardeen,1908年5月23日-1991年1月30日)\(^\text{[1]}\)是一位美国物理学家。他是唯一一位两次获得诺贝尔物理学奖的人:第一次是在1956年,与威廉·肖克利和沃尔特·布拉顿因共同发明晶体管而获奖;第二次是在1972年,与利昂·库珀和约翰·施里弗因提出超导性的微观理论——BCS理论——而再次获奖。\(^\text{[4][5]}\)

巴丁出生并成长于威斯康星州,在威斯康星大学获得电气工程的学士和硕士学位,随后在普林斯顿大学取得物理学博士学位。在参加第二次世界大战后,他曾在贝尔实验室从事研究工作,之后担任伊利诺伊大学教授。

晶体管的发明彻底改变了电子工业,使得从电话到计算机几乎所有现代电子设备的出现成为可能,也由此开启了信息时代。而巴丁在超导性领域的研究成果——为他赢得第二次诺贝尔奖——则被广泛应用于核磁共振光谱(NMR)、医学磁共振成像(MRI)以及超导量子电路等领域。

巴丁是仅有的三位在同一领域获得两次诺贝尔奖的人之一(另外两位是弗雷德里克·桑格和卡尔·巴里·夏普利斯,均为化学奖获得者),也是全球仅有的五位两度获诺贝尔奖的人之一。1990年,《生活》杂志将他评为“20世纪最具影响力的一百位美国人”之一。\(^\text{[6]}\)
\subsection{教育与早年生活}
巴丁于1908年5月23日出生在威斯康星州麦迪逊市。\(^\text{[7]}\)他是查尔斯·巴丁的儿子,后者是威斯康星大学医学院的首任院长。

巴丁就读于麦迪逊的威斯康星大学附属中学。他于1923年毕业,时年仅15岁。\(^\text{[7]}\)他本可以更早毕业,但由于他在另一所高中修读课程以及母亲去世,毕业时间被推迟了。1923年,巴丁进入威斯康星大学。在大学期间,他加入了齐达赛兄弟会,并靠打台球赚取部分会费。他还被接纳为工程荣誉学会Tau Beta Pi的成员。由于不想像父亲那样走学术路线,巴丁选择了工程专业。他还认为工程专业具有良好的就业前景。\(^\text{[8]}\)

巴丁于1928年在威斯康星大学获得电气工程学士学位。\(^\text{[9]}\)尽管他曾休学一年在芝加哥工作,但仍于1928年顺利毕业。\(^\text{[10]}\)他选修了所有令他感兴趣的研究生层次的物理与数学课程,因此用了五年时间完成学业,而不是通常的四年。这也为他提供了时间完成硕士论文,该论文由里奥·J·彼得斯(Leo J. Peters)指导。他于1929年在威斯康星大学获得电气工程硕士学位。\(^\text{[11][9]}\)

巴丁随后继续在威斯康星大学深造,但最终进入了位于匹兹堡的海湾石油公司旗下的研究部门——海湾研究实验室工作。\(^\text{[6]}\)从1930年到1933年,巴丁在该实验室从事磁力和重力勘测数据解释方法的研究工作。\(^\text{[7]}\)他的职位是一名地球物理学家。但随着这项工作逐渐失去吸引力,他申请并被普林斯顿大学的数学研究生项目录取。\(^\text{[8]}\)

作为研究生,巴丁学习了数学和物理。在物理学家尤金·维格纳的指导下,他撰写了一篇关于固体物理问题的论文。尚未完成论文时,他于1935年获得哈佛大学学者协会的初级会员职位。他随后在哈佛度过了三年时间(1935年至1938年),与未来的诺贝尔物理学奖得主约翰·哈斯布鲁克·范弗莱克\(^\text{[12]}\)和珀西·威廉姆斯·布里奇曼合作,研究金属的内聚力与电导问题,并参与了一些关于原子核能级密度的研究工作。他于1936年获得普林斯顿大学数学物理学博士学位。\(^\text{[7]}\)
\subsection{职业与研究经历}
\subsubsection{二战服役}
1941年至1944年间,巴丁在海军军械实验室领导一个研究小组,负责磁性水雷与鱼雷的研发,以及针对水雷和鱼雷的反制技术。在此期间,他的妻子简(Jane)先后生下一子一女:儿子比尔(Bill)出生于1941年,女儿贝齐(Betsy)出生于1944年。[13]
