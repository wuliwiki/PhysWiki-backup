% 抛物线坐标系中的类氢原子定态波函数
% 抛物线坐标系|Kummer-Laplace|本征值|薛定谔方程|类氢原子

\begin{issues}
\issueDraft
\issueOther{画出密度图应该会很漂亮}
\end{issues}

\pentry{定态薛定谔方程\upref{SchEq}, 抛物线坐标系\upref{ParaCr}}

\footnote{参考 \cite{Brandsen} Chap. 3.5 One-electron atoms in parabolic coordinates.}本文使用原子单位制\upref{AU}. 使用相对坐标, 令约化质量\upref{TwoBD}为 $\mu$, 有
\begin{equation}
-\frac{1}{2\mu} \laplacian \psi - \frac{Z}{r} \psi = E\psi
\end{equation}
在抛物线坐标系中变为
\begin{equation}
-\frac{1}{2m} \qty{\frac{4}{\xi + \eta} \qty[\pdv{u}{\xi}\qty(\xi\pdv{\xi}) + \pdv{u}{\eta}\qty(\eta\pdv{\eta})] + \frac{1}{\xi\eta}\pdv[2]{u}{\phi}}\psi - \frac{2Z}{\xi + \eta}\psi = E\psi
\end{equation}
分离变量, 令
\begin{equation}
\psi(\xi, \eta, \phi) = f(\xi) g(\eta) \Phi(\phi)
\end{equation}
和球坐标同理, $\Phi(\phi) = \exp(\I m \phi)$. 令另外两个分离变量常数满足
\begin{equation}
\nu_1 + \nu_2 = Z
\end{equation}
有
\begin{equation}\label{ParaHy_eq1}
\dv{\xi} \qty(\xi \dv{f}{\xi}) + \qty(\frac{\mu E \xi}{2} - \frac{m^2}{4\xi} + \nu_1) f = 0
\end{equation}
\begin{equation}
\dv{\eta}\qty(\eta \dv{g}{\eta}) + \qty(\frac{\mu E\eta}{2} - \frac{m^2}{4\eta} + \nu_2)g = 0
\end{equation}
可以化简为 Kummer-Laplace 微分方程. 解出后, $\nu_1, \nu_2$ 分别对应两个整数 $n_1, n_2$, 称为抛物线量子数, 和主量子数 $n$ 的关系为
\begin{equation}
n = n_1 + n_2 + \abs{m} + 1
\end{equation}
令 $\rho_1 = Z\xi / n$, $\rho_2 = Z\eta/n$, 定态波函数为
\begin{equation}
\begin{aligned}
\psi_{n_1,n_2,m}(\xi,\eta,\phi) &= \frac{(Z/a_\mu)^{3/2}}{\sqrt{\pi} n^2} \sqrt{\frac{n_1!n_2!}{[(n_1+\abs{m})!(n_2+\abs{m})!]^3}} \\
&\times \E^{-(\rho_1+\rho_2)/2}(\rho_1\rho_2)^{\abs{m}/2} L_{n_1 + \abs{m}}^{\abs{m}}(\rho_1) L_{n_2 + \abs{m}}^{\abs{m}}(\rho_2) \E^{\I m\phi}
\end{aligned}
\end{equation}
能量本征值为
\begin{equation}
E_n = -\frac{Z^2}{2n^2}
\end{equation}
