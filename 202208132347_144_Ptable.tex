% 电子轨道与元素周期表
% keys 原子|电子轨道|元素周期表|量子数|周期

\begin{issues}
\issueTODO
\end{issues}

在这里, 用最简单的方式介绍原子的壳层结构,并解释元素周期表如何根据壳层结构分出每个周期. 在玻尔原子模型\upref{BohrMd} 中, 原子轨道如\autoref{Ptable_fig1}.

\begin{figure}[ht]
\centering
\includegraphics[width=4.5cm]{./figures/Ptable_1.pdf}
\caption{电子轨道}\label{Ptable_fig1}
\end{figure}
%现在要把电子放到这些轨道上面来,使电子的总能量最小. 这种状态叫做原子的基态. 理想的状态是,所有电子都在最小的一圈轨道上,但是由于每条轨道只能容纳一定数目的电子,另一些电子不得不进入其他轨道.

\subsection{电子轨道}
%为了解释每条轨道能容纳多少电子,把每个轨道的 “空位” 用一行格子描述. 当一行格子被电子填满时,该轨道就不能容纳更多电子了.
\begin{figure}[ht]
\centering
\includegraphics[width=4cm]{./figures/Ptable_2.pdf}
\caption{用格子描述电子轨道} 
\end{figure}

\begin{table}[ht]
\centering
\caption{电子轨道}\label{Ptable_tab1}
\begin{tabular}{|c|c|c|}
\hline
名称&符号&取值\\
\hline
电子层&主量子数 n& 1(K), 2(L), 3(M), 4(N).. \\
\hline
电子亚层&角量子数 l& 0(S),1(P),2(D),3(F),..., n-1 \\
\hline
亚层的轨道&磁量子数 m & -l, -l+1,...,0,1,2,...,l-1,l \\
\hline
(电子自旋) (这不是轨道参数)&&-1/2, 1/2\\
\hline
\end{tabular}
\end{table}

电子层:如\autoref{Ptable_fig1} 所示,从半径最小的轨道开始,用数字 1,2,3 (或K,L,M,...)等依次命名每条轨道,这些数字也叫主量子数,用 $n$ 来表示. 

电子亚层:一个电子层可以细分为若干个电子亚层,以角量子数$l$计.对于一个主量子数为$n$的电子层,角量子数$l$可以取$0,1,2,...,n-1$(或称$S,P,D,F...$ )等$n$ 个不同的值,即一个主量子数为$n$的电子层可以分为$n$个电子亚层.把行标和列标组合起来, 就能得到任意一个格子的名称,例如第三行第二列的格子叫做 $3P$. 例如,$n=3$的电子层包括$3S, 3P, 3D$三个亚层.(见下)

亚层的轨道:一个电子亚层可以再继续细分为若干个轨道,以磁量子数$m$计.对于一个角量子数为$l$的电子亚层,磁量子数$m$ 可以取 $ - l, - l + 1...0,...l - 1,l$ 等 $(2l + 1)$ 个不同的值,即一个角量子数为$l$的电子亚层可以分为$(2l+1)$个轨道.例如,$l=1$的P亚层包括$P_x, P_y, P_z$三个轨道.

电子自旋:由于不相容原理,一个亚层的轨道中最多只能容纳两个(自旋相反的)电子.

综上所述,每一个电子亚层可以最多容纳$2(2l+1)$个电子,每一个电子层最多容纳$2n^2$ 个电子(见\autoref{Ptable_fig2}).%这也是我们在中学化学中所知的"$2,8,18,...$"电子排布规律的由来.
\subsection{电子轨道的表达法}
现在我们简要说明一下书写电子轨道的一般方法(至于电子\textsl{为什么}这么排布,见下文).

对于一个电子亚层,我们先写电子层(主量子数$n$),再写电子亚层(角量子数$l$,记为$S, P,D,F,...$),最后以上标标注该电子亚层上的电子数量.例如,$3S^2$指$n=3$的电子层的S电子亚层$l=0$上有2个电子.

对于一个原子的电子排布,我们一般按能量从小到大的顺序,依次表示电子轨道.例如,第六号元素C(碳)的电子排布是$1S^2 2S^2 2P^2$.

如果一个原子的周期比较大,我们可以省略内层电子,以相应稀有气体的电子排布代替,称为“原子实”.例如,第十号元素Ne(氖)的电子排布是$1S^2 2S^2 2P^6$,第十一号元素Na(钠)的电子排布$1S^2 2S^2 2P^6 3S^1$,那么Na的电子排布还可记为$[Ne] 3S^1$. 同理,18号元素Ar(氩)的电子排布是$1S^2 2S^2 2P^6 3S^2 3P^6$,那么20号元素Ca(钙)的电子排布也可以记为$[Ar] 4S^2$. (或许你很好奇$n=3$电子层的$l=2 (D)$的电子亚层去哪了,别急,见下文)

这种书写方法不仅让电子的排布更为整洁、也突出了原子的最外层电子排布,而这往往与元素的化学性质密切相关(\textsl{就是对新人不大友好}).

\subsection{电子排序的规律}
到此为止, 每条轨道承载电子的数目已经解释清楚了, 但是应该如何把电子往格子里面放呢? 总的原则是使电子总能量最小(这种状态叫做原子的基态).一般而言,电子能量随$n$与$l$的增大而增大,但这种规则\textbf{十分不严格}(见下),原子轨道的能量事实上非常复杂.

对于氢原子(1 个核外电子), 显然电子应该放在 $1S$ 格子里, 氦原子(2 个核外电子)可以把两个电子都放在 $1S$ 格子里, 从而把 $n=1$ 的轨道填满, 这就是第一周期的两个原子的电子分布. 对于锂原子(3 个核外电子)可以在氦原子的基础上往 $2S$ 格子里放一个电子. 

但奇怪的是, 轨道的能量\textbf{并不是}简单的单调递增, 而是如下图中的箭头所示的 $1S, 2S, 2P, 3S, 3P, 4S, 3D, 4P, 5S\dots$ 的顺序. 
\begin{figure}\label{Ptable_fig2}[ht]
\centering
\includegraphics[width=4.5cm]{./figures/Ptable_3.pdf}
\caption{轨道的填充顺序.格子内的数字表示每个亚层能容纳的电子数, 即 $2(2l + 1)$.} 
\end{figure}

\addTODO{这张图在第四周期以上不成立,见 \cite{GriffQ}}

理论上,$n\ge5$的电子层存在$l\ge4$的电子亚层,但实际的基态原子中,该电子亚层一般不被填充.(至少对于118号及之前的元素,这也是目前大多数元素周期表上所列出的最后一号元素)

\subsubsection{其余的一些规律}
这里列出电子排布的部分规律.不过要注意的是,这些规律不总是成立(至少在前三周期成立).

在填充亚电子层的轨道时,电子一般优先进入空轨道,而不是进入已有一个电子的轨道.
%这可以用来理解部分元素的某些化学性质.例如,N的价电子层有三个未成对电子与一对孤电子对.这三个未成对电子可以分别与H的一个未成对电子成键,形成$NH_3$.由于孤电子对的电子斥力,$N-H$的键角被压缩;这对孤电子对可以作为路易斯碱,进攻其他正电性基团.
\begin{table}[ht]
\centering
\caption{部分元素的2p轨道电子排布示意图}}\label{Ptable_tab2}
\begin{tabular}{|c|c|c|}
\hline
5 B & $\uparrow$ &  & \\
\hline
6 C & $\uparrow$ & $\uparrow$& \\
\hline
7 N & $\uparrow$ & $\uparrow$&$\uparrow$ \\
\hline
8 O & $\uparrow \downarrow$ & $\uparrow$ & $\uparrow$ \\
\hline
9 F & $\uparrow \downarrow$ & $\uparrow \downarrow$ & $\uparrow$ \\
\hline
10 Ne & $\uparrow \downarrow$ & $\uparrow \downarrow$ & $\uparrow \downarrow$ \\
\hline
\end{tabular}
\end{table}
有时,半满、全满的亚层电子排布更稳定.例如,23V(钒)的电子排布是$4S^2 3D^3$,那下一号元素24Cr(铬)的电子排布是$4S^2 3D^4$?非也,由于半满的S、D亚层更稳定,因此24Cr(铬)的电子排布是$4S^1 3D^5$.
\subsection{元素周期表的排序}

%要判断某个原子所在的周期, 就先根据原子序号找出上图中所有装有电子的格子, 

元素原子中最大的主量子数 $n$ (最高的电子层数)确定了该元素所在的周期. 例如 30 号元素Zn(锌), 可以按照上图绿色线条的顺序占满 $1S, 2S, 2P, 3S, 3P, 4S, 3D$ (这些格子能容纳的总电子数刚好是 30). 其中 $4S$ 的主量子数最大,$n=4$, 所以 30 号元素在第四周期. 按照这个规律, 把上图按照周期分类如下.
\begin{figure}[ht]
\centering
\includegraphics[width=8cm]{./figures/Ptable_4.pdf}
\caption{划分周期} 
\end{figure}

再主要根据元素原子最高能量的电子亚层,可以进一步划分元素周期表为s区,p区,d区,f区等.其中s,p区一般合称主族,d,f区合称副族.有人把IB, IIB(11,12)列合称为ds区.
\begin{figure}[ht]
\centering
\includegraphics[width=10cm]{./figures/Ptable_3.png}
\caption{元素周期表的分区}} \label{Ptable_fig3}
\end{figure}
%我不大了解f区诶,感觉好像也不是全按元素原子最高能量的电子亚层分的,也有一点习惯的因素?需要dalao指正

% \addTODO{解释 1s22P2 之类的符号吧}
