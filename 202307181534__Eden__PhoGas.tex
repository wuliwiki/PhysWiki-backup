% 光子气体
% 光子|气体|配分函数|压强|亥姆霍兹自由能
\pentry{近独立子系\upref{depsys},盒中的电磁波\upref{EBBox}}

\begin{issues}
\issueDraft
\end{issues}
光子气体是一类特殊的理想玻色气体。由于某一个光子可以被气体容器壁的分子吸收后再放出多个光子,光子气体的光子数是不守恒的,所以它的化学势 $\mu$ 为 $0$,辐射场的不同频率的能量密度只取决于光子气体的温度 $T$。
\subsection{态密度}
在一维情形下,施加周期性边界条件,一个电磁波振动模式所对应的波矢 $k$ 是 $2\pi/L$ 的倍数,那么波矢空间中 $d k$ 中包括了 $2\cdot \frac{L}{2\pi} dk$ 个振动模式,$2$ 来自于电磁波的两个偏振自由度。推广到三维情形,波矢空间的态密度为 $2\cdot V \frac{dk^3}{(2\pi)^3}=2\cdot Vk^2dk/(2\pi^2)$,前面的系数 $2$ 来自于电磁波的两个偏振自由度。对于电磁波,$k=\omega/c$,所以态密度为 $V \omega^2 d\omega/(\pi^2 c^3)$。

现在考察辐射场的能量 $U$,考察每个频率区间 $[\omega,\omega+d\omega]$ 范围内的辐射场能量 $U(\omega,T)d\omega$。根据玻色分布(\autoref{eq_depsys_1}~\upref{depsys}),每一个振动模式(能级)可以有多个光子占据,光子的平均占据数为 $\frac{1}{\exp(\hbar\omega/kT)-1}$,而每个光子的能量为 $\hbar\omega$,即某个能级上的光子平均能量为 $\hbar\omega/(\exp(\hbar\omega/kT)-1)$。用这一结果乘以态密度,最终可以得到辐射场的能量公式
\begin{equation}
\begin{aligned}
U(\omega,T) d\omega=\frac{V}{\pi^2c^3} \left[\hbar\omega^3/(\exp(\hbar\omega/kT)-1)\right] d\omega~.
\end{aligned}
\end{equation}

对频率积分,就可以得到辐射场的总能量
\begin{equation}
\begin{aligned}
U(T)&=\int U(\omega,T) d\omega=\frac{V}{\pi^2c^3} \int \frac{\hbar\omega^3}{\exp(\hbar\omega/kT)-1} d\omega~,\\
&=\frac{V}{\pi^2c^3 } \frac{(kT)^4}{\hbar^3}\int \frac{z^3dz}{e^z-1}=\frac{V}{\pi^2c^3\hbar^3}(kT)^4\Gamma(4)\zeta(4)\\
&=\frac{V\pi^2}{15c^3\hbar^3}(kT)^4
\end{aligned}
\end{equation}
可以看出辐射场能量正比于温度的四次方。这里运用了公式
\begin{equation}
\int_0^\infty \frac{z^n dz}{e^z-1}=\Gamma(n+1)\zeta(n+1),\quad \zeta(4)=\sum_n \frac{1}{n^4}=\frac{\pi^4}{90}~.
\end{equation}


\subsection{巨正则系综方法}
\pentry{等间隔能级系统(正则系宗)\upref{EqCE}}
\addTODO{这个方法实际上是巨正则系综,光子数不确定,需要用巨配分函数。需要改一下描述。}
我们仍用正则系宗下推导光子气体的总能能量和压强。 已知单模式(单频率)的光子气体配分函数为 $Q(\omega) = 1/[1-\exp(-\omega\hbar\beta)]$, 系统的配分函数为
\begin{equation}
Q = \prod_i Q(\omega_i) = \prod_i \frac{1}{1-\exp(-\omega_i\hbar\beta)}~.
\end{equation}
系统总能量为
\begin{equation}
U = -\pdv{\beta} \ln Q = \sum_i \pdv{\beta} \ln(1 - \E^{-\omega_i \hbar \beta}) = \sum_i \frac{\omega_i \hbar \E^{-\omega_i \hbar\beta}}{1-\E^{-\omega_i \hbar \beta}} = \sum_i \frac{\omega_i \hbar}{\E^{\omega_i\hbar\beta} - 1}~.
\end{equation}
用积分来求和, 模式密度为 $\rho(\omega) = V\omega^2/(\pi^2 c^3)$
\begin{equation}
U = \frac{V}{\pi^2 c^3} \int_0^\infty \frac{\omega^3 \hbar}{\E^{\omega\hbar\beta} - 1}\dd{\omega}~,
\end{equation}
用换元积分法, 令 $x = \omega\hbar\beta$
\begin{equation}
U = \frac{V}{\pi^2c^3 \beta^4 \hbar^3} \int_0^\infty \frac{x}{\E^x - 1}\dd{x} = \frac{\pi^2 V}{15c^3 \hbar^3} (kT)^4~.
\end{equation}
最后的积分可由 Mathematica 完成。

\subsection{压强}
由正则系宗法的步骤, 先求出亥姆霍兹自由能
\begin{equation}
F = -kT \ln Q = kT \sum_i \ln(1 - \E^{-\omega_i \hbar\beta})~.
\end{equation}
利用 $F=G-pV=\mu N-pV=-pV$,可以得到 $p=-F/V$。