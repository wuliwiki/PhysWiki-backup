% ic 集成电路
% license CCBYSA3
% type Wiki

(本文根据 CC-BY-SA 协议转载自原搜狗科学百科对英文维基百科的翻译)

\begin{figure}[ht]
\centering
\includegraphics[width=6cm]{./figures/41b424041814090c.png}
\caption{可擦除可编程只读存储器(EPROM) 集成电路。这些封装具有透明窗口,显示内部的管芯。该窗口用于通过将芯片暴露于紫外光来擦除存储器。} \label{fig_icJCDL_1}
\end{figure}

\textbf{ic集成电路}或者\textbf{单片集成电路}(也称为\textbf{集成电路},\textbf{芯片},或\textbf{微芯片})是一组位于一小片(或“芯片”)半导体材料(通常为硅)上的电子电路。将大量微小的晶体管集成到一个小芯片中使得电路比由分立的电子元件构成的电路小几个数量级、更快、更便宜。集成电路的批量生产能力、可靠性和积木式方法的电路设计使得采用标准化集成电路迅速地取代了使用分立晶体管的设计。集成电路现在几乎被用于所有电子设备,并且已经彻底改变了电子学的世界。计算机、移动电话和其他数字家用电器现在是现代社会结构中不可分割的部分,集成电路的小尺寸和低成本使其成为可能。

20世纪中期半导体器件制造的技术进步使集成电路变得实用。自从20世纪60年代问世以来,芯片的尺寸、速度和容量都有了巨大的进步,这是由越来越多的晶体管安装在相同尺寸的芯片上的技术进步所推动的。现代芯片在人类指甲大小的区域内可能有数十亿个晶体管晶体管。这些进展大致跟随摩尔定律,使得今天的计算机芯片拥有上世纪70年代早期计算机芯片数百万倍的容量和数千倍的速度。

集成电路相对于分立电路有两个主要优势:成本和性能。成本低是因为芯片及其所有组件通过光刻作为一个单元印刷,而不是一次构造一个晶体管。此外,封装集成电路比分立电路使用的材料少得多。性能之所以高,是因为由于集成电路元件体积小且非常接近,它们的切换速度快,功耗相对较小。集成电路的主要缺点是设计它们和制造所需的光掩模的成本高。这种高初始成本意味着集成电路只有在预计到高产量时才是实用的。

\subsection{术语}

集成电路定义为:[1]

一种电路,其中所有或部分电路元件不可分离地关联并电气互连,因此就结构和商业目的而言,它被认为是不可分割的。

符合该定义的电路可以使用许多不同的技术来构建,包括薄膜晶体管、厚膜技术或混合集成电路。不过,一般情况下集成电路指的是单片集成电路。[2][3]

\subsection{发明}

集成电路的早期发展可以追溯到1949年,当时德国工程师沃纳·雅可比[4](西门子)[5]申请了集成电路状半导体放大器的专利[6]示出了公共衬底上的五个晶体管组成的三级放大器。雅可比披露了小巧便宜的助听器作为他专利的典型工业应用。他的专利尚未被报道而立即用于商业用途。

集成电路的概念是由杰弗里·杜默(1909-2002)提出的,一名工作于英国国防部皇家雷达机构的雷达科学家。杜默在公元1952年5月7日华盛顿质量电子元件进展研讨会上向公众提出了这个想法。[7]他公开举办了许多研讨会来宣传他的想法,并在1956年试图建造这样一个电路,但没有成功。

关于集成电路的想法的前身是制造小陶瓷正方形(晶片),每个正方形包含一个小型化的组件。然后,组件可以集成并连线到二维或三维紧凑网格中。这个想法在1957年似乎很有希望,是由杰克·基尔比向美国陆军提出的,并导致了短命的小模块计划(类似于1951年的Tinkertoy项目)。[8][9][10]然而,随着项目势头越来越猛,基尔比提出了一个新的革命性设计: 集成电路。

\begin{figure}[ht]
\centering
\includegraphics[width=6cm]{./figures/85c3599b254c88c9.png}
\caption{杰克·基尔比的原始集成电路} \label{fig_icJCDL_2}
\end{figure}

1958年7月,德州仪器新雇佣的基尔比记录了他关于集成电路的最初想法,并于1958年9月12日成功演示了第一个可工作的集成示例。[11]1959年2月6日,在他的专利申请中,[12]Kilby将他的新设备描述为“一种半导体材料……其中所有电子电路的组件都是完全集成的。”[13]这项新发明的第一个客户是美国空军。[14]

基尔比赢得了2000年的冠物理诺贝尔奖,为了表彰他在集成电路发明中的贡献。[15]2009年,他的工作被命名为 IEEE里程碑。[16]

基尔比之后半年,仙童半导体公司的罗伯特·诺伊斯开发了一种新的集成电路,比基尔比的更实用。诺伊斯的设计由硅制成,而基尔比的芯片由锗制成。诺伊斯将以下原理归功于斯普拉格电气的库尔特·利霍韦克p–n 绝缘结,这也是集成电路背后的关键概念。[17]这种绝缘允许每个晶体管独立工作,尽管它们是同一片硅的一部分。

仙童半导体公司也是第一个拥有自对齐栅极的硅栅集成电路技术的公司,这是所有现代CMOS集成电路的基础。这项技术是由意大利物理学家Federico Faggin在1968年发明的。1970年,他加入了英特尔,发明了第一个单芯片中央处理单元(CPU)微处理器——英特尔 4004,他因此在2010年得到了国家技术和创新奖章。4004是由Busicom的嶋正利和英特尔的泰德·霍夫设计的,但正是Faggin在1970年改进的设计使其成为现实。[18]

\subsection{进展}

集成电路技术的进步,主要是更小的特征和更大的芯片,使得集成电路中晶体管的数量每两年翻一番,这种趋势被称为摩尔定律。这种增加的容量已被用于降低成本和增加功能。一般来说,随着特征尺寸的缩小,集成电路操作的几乎每个方面都得到改善。每个晶体管的成本和每个晶体管的开关功耗下降,而存储容量和速度上升,这是通过丹纳德标度定义的关系实现的。[19]因为速度、容量和功耗的提高对最终用户来说是显而易见的,所以制造商之间在使用更精细的几何结构方面存在激烈的竞争。多年来,晶体管尺寸已经从20世纪70年代早期的10微米减小到2017年的10纳米[20]每单位面积的晶体管数量相应地增加了百万倍。截至2016年,典型的芯片面积从几平方毫米到大约600平方毫米,高达2500万晶体管每平方毫米。

国际半导体技术发展蓝图 (ITRS)多年来预测了特征尺寸的预期缩小和相关领域所需的进展。最终的ITRS于2016年发布,现已被《设备和系统国际路线图》取代。[21]

最初,集成电路严格地说是电子设备。集成电路的成功导致了其他技术的集成,试图获得同样的小尺寸和低成本优势。这些技术包括机械设备、光学和传感器。

\begin{itemize}
\item 电荷耦合器件和与其密切相关的有源像素传感器是对光敏感的芯片。在科学、医学和消费者应用中,它们已经在很大程度上取代了照相胶片。现在每年为手机、平板电脑和数码相机等应用生产数十亿台这样的设备。集成电路的这个子领域获得了2009年诺贝尔奖。
\item 由电力驱动的非常小的机械设备可以集成到芯片上,这种技术被称为微电子机械系统。这些设备是在20世纪80年代后期开发的[22]并且用于各种商业和军事应用。例子包括DLP投影仪,喷码机,和被用于汽车的安全气袋上的加速计和微机电陀螺仪.
\item 自21世纪初以来,将光学功能(光学计算)集成到硅芯片中一直在学术研究和工业上积极进行,使得将光学器件(调制器、检测器、路由)与CMOS电子器件相结合的硅基集成光学收发器成功商业化。[23] 集成光学电路也在开发中,使用了新兴的物理领域,即光子学。
\item 集成电路也正在为在医疗植入物或其他生物电子设备中的传感器的应用而开发。[24]在这种生物环境中必须应用特殊的密封技术,以避免暴露的半导体材料的腐蚀或生物降解。[25]
\end{itemize}

截至2018年,绝大多数晶体管都是使用平坦的二维平面工艺,在硅芯片一侧的单层中制造的。研究人员已经生产了几种有希望的替代品的原型,例如:

\begin{itemize}
\item 堆叠几层晶体管以制造三维集成电路(3DC)的各种方法,例如硅通孔,“单片3D”,[26]堆叠引线接合,[27]和其他方法。
\item 由其他材料制成的晶体管:石墨烯晶体管s .辉钼矿晶体管,碳纳米管场效应晶体管,氮化镓晶体管,类似晶体管纳米线电子器件,有机晶体管等等。
\item 在小硅球的整个表面上制造晶体管。[28][29]
\item 对衬底的修改,通常是为了制造用于柔性显示器或其它柔性电子学的柔性晶体管,可能向卷轴式计算机的方向发展。
\end{itemize}

随着制造越来越小的晶体管变得越来越困难,公司正在使用多晶片模组、三维晶片、 3D与非门、封装在封装上和硅穿孔来提高性能和减小尺寸,而不必减小晶体管的尺寸。[30] [31][32]

\subsection{设计}

设计和开发复杂集成电路的成本相当高,通常高达数千万美元。[33]因此,生产高产量的集成电路产品只有经济意义,因此非经常性工程 (NRE)成本通常分散在数百万个生产单元中。

现代半导体芯片有数十亿个组件,过于复杂,无法手工设计。帮助设计师的软件工具至关重要。电子设计自动化(\textbf{电子设计自动化}),也称为电子计算机辅助设计(\textbf{适应型}),[34]是一类用于设计包括集成电路在内的电子系统的软件工具。这些工具在工程师用来设计和分析整个半导体芯片的设计流程中协同工作。

\subsection{类型}

\begin{figure}[ht]
\centering
\includegraphics[width=6cm]{./figures/be4b53ae591610c1.png}
\caption{DIP 中的CMOS 4511 集成电路} \label{fig_icJCDL_3}
\end{figure}

\begin{figure}[ht]
\centering
\includegraphics[width=6cm]{./figures/461b4a691d1c12d6.png}
\caption{该芯片来自Intel8742,一个8位微控制器,包括一个中央处理器,运行频率12MHz,128字节的RAM ,2048字节的EPROM 和 I/O等都包含在同一芯片中。} \label{fig_icJCDL_4}
\end{figure}

集成电路可以分为模拟,[35] 数字[36]和混合信号,[37]由同一集成电路上的模拟和数字信号组成。

数字集成电路可以在几平方毫米上包含一个[38]到几十亿[39]个逻辑门、触发器、多路复用器和其他电路。与板级集成相比,这些电路的小尺寸允许高速、低功耗,并降低了制造成本。这些数字集成电路,通常是微处理器、 DSP 和微控制器,使用布尔代数来处理“1”和“0”信号。

该芯片来自Intel8742,一个8位微控制器,包括一个中央处理器,运行频率12MHz,128字节的RAM ,2048字节的EPROM 和 I/O等都包含在同一芯片中。
最先进的集成电路有微处理器或“\textbf{内核}”,它可以控制任何电子产品,从个人计算机和蜂窝电话到数字微波炉。数字存储芯片和专用集成电路(ASICs)是对现代信息社会很重要的另外一些集成电路家族的例子。

在20世纪80年代,可编程逻辑设备被开发出来。这些器件包含的电路的逻辑功能和连接性可以由用户编程,而不是由集成电路制造商固定。这允许单个芯片被编程以实现不同的大规模集成电路类型的功能,例如逻辑门,加法器寄存器。可编程性至少有四种形式——可以仅编程一次,可以使用紫外线擦除然后重新编程的设备,可以使用闪寸(重新)编程的设备,和可编程门阵列 (FPGAs),可随时编程,包括在运行期间。目前的FPGA(截至2016年)可以实现相当于数百万个门的功能,运行频率最高可达1GHz。[39]

模拟集成电路,如传感器、电源管理电路和运算放大器,通过处理连续信号来工作。它们执行模拟功能,如放大、有源滤波器、解调和混频。模拟集成电路提供专业设计的模拟电路,而不是从头开始设计和构建困难的模拟电路,从而减轻电路设计者的负担。

集成电路也可以将模拟和数字电路结合在单个芯片上来创建功能,如模数转换器和数模转换器。在20世纪90年代末之前,无线电不能用与微处理器相同的低成本CMOS工艺制造。但是自1998年以来,已经使用CMOS工艺开发了大量无线电芯片。例如英特尔的 DECT 无绳电话,或由Atheros 和其他公司开发的802.11 ( 无线网络)芯片。[40]

现代电子元件分销商通常会对现有的各种集成电路进一步细分:

\subsection{制造业}



\subsubsection{6.1 制造}



\subsubsection{6.2 封装}



\subsection{知识产权}



\subsection{其他发展}



\subsection{世代}



\subsubsection{9.1 SSI、MSI和LSI}



\subsubsection{9.2 超大规模集成电路}



\subsubsection{9.3 ULSI、WSI、SoC和3D-IC}



\subsection{硅标签和涂鸦}



\subsection{集成电路和集成电路系列}



\subsection{参考文献}