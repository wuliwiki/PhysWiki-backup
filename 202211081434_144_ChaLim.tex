% 求极限的一些方法

\begin{issues}
\issueDraft
\end{issues}
\footnote{本文参考了\cite{同济高},\cite{Thomas}与武忠祥的《考研高数》课程.部分例题来自练习册}
尽管实际中极限的求解往往交给电脑解决,但求解极限还是高数考试中的重要环节.本文简要介绍求解极限的一些基本方法与思路.

要强调的是,极限问题千变万化,没有什么题目能只用一种方法就能解决,也没有什么能简单机械地解决任意问题的“万能套路”.你往往需要多次运用各种方法才能得到最终的答案.解决问题的关键最终在于你的积累与灵性(\textsl{多做题}),而不是单纯地“背诵方法”.

\subsection{基础知识}
这里列举一些关于极限计算的基本知识.你需要熟悉这些内容.
\begin{itemize}
\item 极限的四则运算法则
\item 极限与函数连续性:连续函数的极限值就是他在该点处的函数值;极限号可以通过连续函数 $\lim f(g(x)) = f(\lim g(x))$
\item 极限的有理运算性质:
可先提取极限存在且\textbf{非零}的乘数 
$$\lim_{x\to x_0} f(x)g(x) = f(x_0)\lim_{x\to x_0} g(x)$$
也可先提取极限存在的加数
$$\lim_{x\to x_0} (f(x)+g(x)) = f(x_0)+\lim_{x\to x_0} g(x)$$
这能帮助简化计算(为了方便展示,这里假定f(x)连续,但这个条件其实是非必须的)
\item 导数的极限定义:有时,解决极限的办法是通过导数的定义,将其凑成某个函数的导数
\item 三角函数的相关公式
\item 有界函数*无穷小=0
\item 洛必达法则:用来处理$\frac{0}{0}$,$\frac{\infty}{\infty}$等不定式时很方便.\textsl{洛必达虽好,可不要贪杯哦}
\item 泰勒展开:将一点处的函数换成他在该点处的泰勒展开式,从而将复杂函数化为多项式.
\item 等价无穷小:也是用于将复杂的函数代换成简单的函数
\item 几个常用的等式:
$$\lim_{x\to0} \frac{\sin(x)}{x}=1$$
$$\lim_{x\to0} (1+x)^{1/x}=e$$
$$\lim_{x\to0+} x\ln(x)=0$$
$$\lim_{x\to+\infty} \frac{a_nx^n+a_{n-1}x^{n-1}+...}{b_nx^n+b_{n-1}x^{n-1}+...}=\frac{a_n}{b_n}$$
你可以使用非常简单的方法从容地“证明”这些等式,不过最好有个大致印象.
\end{itemize}

\subsection{拼凑与代换}
\textsl{有借有还,再借不难.} 拼凑与代换的核心思路是变形待求的极限式,以配凑出容易解决的形式.当然,为了运用拼凑与代换,\textsl{你先得有个目标、知道“我想凑出什么样的形式”}.这需要你知晓各种常用形式.(常用的等式、等价无穷小...)

拼凑与代换的基本方法包括:
\begin{itemize}
\item 加一项再减去相同的一项
\item 乘一项再除以相同的一项
\item 提取一个公因式
\item 代换自变量
\item ...
\end{itemize}

\begin{example}{}
求解 $\lim_{x \to 0} \frac{\sin(3x)}{x}$.假设我们对等价无穷小、洛必达、泰勒等等等一无所知,我们能用的只有拼凑与代换法.

我们知道一个基本结论$\lim_{x \to 0} \frac{\sin(x)}{x}=1$,因此我们想配凑出类似的形式.因此,我们令 $t=3x, x=t/3$,那么原式化为$\lim_{t \to 0} \frac{\sin(t)}{t/3}=3\lim_{t \to 0} \frac{\sin(t)}{t}=3$.
\end{example}

\begin{example}{}
由于很多结论都适用于$x\to0$的情况,因此常通过代换变量,令自变量趋于0. 有时这种变形会给你一些启发.

例如,$x\to+\infty$,则令 $t=1/x$,那么$t\to0+$; 若$x\to-\infty$,则令 $t=-1/x$,那么$t\to0+$.(我们不喜欢负数,特别当他在根式中时)
\end{example}

\subsection{$\infty+\infty$ 型不定式}
如果两个加数均为不定式($\infty+\infty$),将他们通分,并化为一个分式.

\subsection{$0\cdot\infty$ 型不定式}
把趋于0的因子移动到分母,将整体化为分式 $\frac{\infty}{\infty}$
\begin{example}{}
求解$\lim_{x\to0+} x\ln(x)$

将x移动至分母,化为1/x. 即$\lim_{x\to0+} x\ln(x)=\lim_{x\to0+} \frac{\ln(x)}{1/x}$

随后运用洛必达,即有$-\lim_{x\to0+} \frac{1/x}{1/x^2}=-\lim_{x\to0+} x=0$
\end{example}


\subsection{分式与根式}
\begin{itemize}
\item 有理化.常用于出现根式相加减的情况.
\item 上下同除以最高次的项
\end{itemize}

\subsection{幂指函数}
若待求极限含有幂指函数,则可以运用指数与对数改写式子 $f(x)^{g(x)}=\E^{g(x)\ln(f(x))}$
\begin{example}{}
求解 $\lim (1+f(x))^{g(x)}$,其中$f(x)\to0$,$g(x)\to\infty$.(即$1^\infty$型不定式)

这是一个幂指函数,取对数再取指数.原式化为$\E^{\lim g(x) \ln (1+f(x))}$,运用等价无穷小,$\E^{\lim g(x) f(x)}$.即只要$\lim g(x) f(x)$存在,原式的极限就存在.

有些考研教材上将 $\lim (1+f(x))^{g(x)} = \E^{\lim g(x) f(x) }$($f(x)\to0$,$g(x)\to\infty$,$\lim g(x) f(x)$存在)作为一个基本结论.

当然,你也可以从常见等式 $e=\lim_{x\to0} (1+x)^{1/x}$出发,运用拼凑与代换解决问题,不过这会让问题变得过于繁琐.
\end{example}

\subsection{形式相近的项、变上限积分}
若极限式中出现变上限积分,则可以运用积分中值定理.

若极限式中出现相似的结构,可以构造函数并运用Lagrange中值定理

%极限中出现变上限积分时,自变量往往位于上(下)限,且上下限往往趋于一致.

\begin{example}{}
求解 $\lim_{x\to0} \frac{\int^x_0 \cos(x)\dd x}{x}$

可以运用洛必达法则,原式化为$\lim_{x\to0} \cos(x) = 1$

另一种思路是运用积分中值定理.根据中值定理,$\exists \xi \in (0,x)$, $\int^x_0 \cos(x) \dd x = \cos(\xi) (x-0)$,因此原式化为 $\lim_{x\to0} \frac{\cos(\xi)x}{x} = \lim_{x\to0} \cos(\xi)$.由于$\xi \in (0,x)$,当$x\to0$时,$\xi$\textsl{别无选择,只能也趋向0}(这是可运用中值定理的重要特征;如果$\xi$不趋于一个常数,这道题就无法由中值定理求解).因此,$\lim_{x\to0} \cos(\xi)=1$

尽管在这里微分中值定理看起来更为繁琐,但这是因为\textsl{这道题太简单了}.在另一些问题中,微分中值定理比洛必达更为\textsl{优雅}.
\end{example}

\begin{example}{}
求解 $\lim_{x\to0} \frac{e^{x^2}-e^x}{x}$

注意到分子的两项都是$e^x$式结构,因此设$f(x)=e^x$,原式化为$$\lim_{x\to0} \frac{f(x^2)-f(x)}{x}$$
根据Lagrange中值定理, $$\exists \xi \in (x^2,x), f(x^2)-f(x) = f'(\xi)(x^2-x)$$
所以原式化为 $$\lim_{x\to0} \frac{f'(\xi)(x^2-x)}{x}$$
即$$\lim_{x\to0} e^\xi(x-1)=-\lim_{x\to0} e^\xi=-1$$
\end{example}
