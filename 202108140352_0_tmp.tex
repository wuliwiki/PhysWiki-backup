% 温度 温标
% keys 温度|温标|开尔文温标|绝对温度|摄氏度

\pentry{理想气体状态方程\upref{PVnRT}}

\subsection{理想气体的定义}
对于气体而言, 温度越高意味着气体分子速度越大(\autoref{PVnRT_eq3}~\upref{PVnRT}), 而对于固体, 温度越高说分振动越剧烈. 在理想气体模型中, 我们看到温度与气体分子的平均动能成正比. 
\begin{equation}
\bar E_k = \frac{3}{2} k_B T_K
\end{equation}
这样我们就在微观上定义了\textbf{热力学温标}(单位是开尔文, $K$, 国际单位的一种). 当分子动能为 0 时, 热力学温度就是 $0 \Si{K}$, 即\textbf{绝对零度}. 根据这个定义, 最低的可能温度就是绝对零度\footnote{统计力学中的确有负温度这种说法, 但根据定义, 它的温度反而比任何温度要高.}.

在生活中, 我们一般使用\textbf{摄氏温标}或\textbf{华氏温标}来表示温度, 它们的单位分别记 $^\circ\Si{C}$, $^\circ\Si{F}$, 我们以下用 $T_C$ 和 $T_F$ 表示. 三种温标的转换关系如下
\begin{equation}
T_K = T_C + 273.15^\circ\Si{C}
\end{equation}
\begin{equation}
T_C = \frac{5}{9}(T_F - 32^\circ\Si{F})
\end{equation}
注意一开尔文和一摄氏度的大小一样, 只是相差了一个常数. 零下 $273.15$ 摄氏度就是绝对零度.

\subsection{用熵定义}
在统计力学中,更广义的温度是从熵的角度定义的, 
\frac{1}{T} = \left(\frac{S}{E}\right)_{V,N} 
由此可以推出理想气体的温度定义, 详见任意统计力学教材.
