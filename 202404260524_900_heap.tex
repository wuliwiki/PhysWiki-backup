% 堆
% keys 堆|数据结构|C++
% license Xiao
% type Tutor

堆是一棵完全二叉树,且每个结点上的编号都不小于或不大于其父结点的编号,堆的结点上并没有权值。而二叉堆的结点上具有权值,且满足堆的性质。满足父结点不小于左右两个子结点的二叉堆被称为大根堆,反之为小根堆。大根堆与小根堆都被属于二叉堆。但是一般没有特别指出的堆一般都指二叉堆。

以一个例题来讲解堆。

维护一个集合,初始时集合为空,支持如下几种操作:
\begin{enumerate}
\item 插入一个数 $x$;
\item 输出当前集合中的最小值;
\item 删除当前集合中的最小值;
\item 删除第 $k$ 个插入的数;
\item 修改第 $k$ 个插入的数,将其变为 $x$。
\end{enumerate}

本题要我们实现一个小根堆,根结点是整课树中最小的结点,父结点永远小于等于左右两个子结点。
C++ STL 中的堆只能实现前 $3$ 个操作。
\begin{lstlisting}[language=cpp]
priority_queue<int, vector<int>, greater<int>> heap;   // 定义小根堆的方式
int t;
cin >> t;
while (t -- )
{
    int x;            
    cin >> x;
    heap.push(x);   // 向队尾插入一个数 x
    cout << heap.top() << endl;     // 输出最小值,即队头
    heap.pop();     // 删除最小值,即删除队头
}
\end{lstlisting}

要想实现随机删除和修改,只能用数组来模拟堆,所以我们讲解一下如何使用数组模拟堆。

首先需要一个数组 \verb|h[N]| 用于存储堆中的元素,由于需要在任意位置进行删除和修改,所以需要多开两个数组 \verb|ph[N]| 和 \verb|hp[N]|。

\verb|ph[i]| 的意思是第 $i$ 个插入的数在堆中的下标是什么,而 \verb|hp[i]| 的意思是在堆中下标是 $i$ 的点是第几个插入的。
比如 \verb|ph[1] = a|,\verb|hp[a] = 1|。

因为是随机插入和删除,所以在堆中删除或者插入某个值的话必定要进行调整。分为 up 操作和 $down$ 操作,up 是如果一个数不符合堆的性质就要往上调整,$down$ 也同理,因为某个数不符合堆的性质,就要往下调整。

堆的存储只用开一个一维数组 \verb|h[N]| 就可以了,父结点是 $i$,左子结点就是 $2 \times i$,右子结点就是 $2 \times i + 1$。
$2 \times i$ 可以写成 \verb|u << 1| ,$2 \times i + 1$ 可以写成 \verb?u << 1 | 1?。比如父结点的下标是 $1$,左子结点的下标就是 $2$,右子结点的下标就是 $3$。有一点需要注意:下标必须从 $1$ 开始,如果从 $0$ 开始的话左子结点和右子结点的下标均为 $0$,这样显然是不行的。

插入一个数,就直接在堆的结尾插入一个数就可以了,然后再 up 一遍。

输出最小值直接输出堆顶,一定是最小值。

删除最小值直接删除堆顶不太好删,我们可以把堆的结尾的数与堆顶的数交换,再删除堆的结尾的数(交换完之后堆的结尾的数就是堆顶,即最小值)删除,然后再从根节点 down 一遍,这样就实现了删除最小值。

删除任意一个数与删除最小值类似,都是把堆的结尾的数与要删除的数交换,然后再删除堆的结尾的数,最后要注意进行 up 或 down 操作,为了不判断,干脆两个操作都写上,虽然两个函数都执行了,但其中只有一个函数对堆有影响。

修改任意一个数直接修改就好了,最后也要注意进行 up 或 down 操作。

up 操作与 down 操作的时间复杂度都为 $\mathcal{O}(\log_2 n)$,输出最小值是 $\mathcal{O}(1)$ 的。所以除了输出最小值,其他操作的时间复杂度都为 $\mathcal{O}(\log_2 n)$。

因为有任意删除和修改操作,涉及到第 $k$ 个插入的数,所以不能直接知道第 $k$ 个插入的数在堆中的下标是多少,所以需要写一个独特的 $\mathtt{swap}$ 操作,就有了前面所提到的两个数组 \verb|ph[N]| 和 \verb|hp[N]|。

具体看一下代码:

\begin{lstlisting}[language=cpp]
void heap_swap(int a, int b)
{
    swap(hp[a], hp[b]);
    swap(ph[hp[a]], ph[hp[b]]);
    swap(h[a], h[b]);
}
\end{lstlisting}

这三行的代码的意思是:先交换一下 \verb|hp| 数组,再交换一下对应的 \verb|ph| 数组,最后再交换堆中的元素。

举个例子:
第一个插入的数是 $a$,第二个插入的数是 $b$,所以:

\verb|hp[a] = 1|、\verb|hp[b] = 2|、\verb|ph[hp[a]] = ph[1] = a|、\verb|ph[hp[b]] = ph[2] = b|、\verb|h[1] = 3|、\verb|h[2] = 4|。

先交换一下 \verb|hp| 数组就有:\verb|hp[a] = 2|,\verb|hp[b] = 1|。

再交换一下对应的:\verb|ph| 数组就有:\verb|ph[hp[a]] = ph[2] = a|,\verb|ph[hp[b]] = ph[1] = b|。

最后再交换一下对应的 \verb|h| 数组:\verb|h[1] = 4|,\verb|h[2] = 3|。

具体的可以看一下图片:
\begin{figure}[ht]
\centering
\includegraphics[width=14.25cm]{./figures/a6a8fcd36fc78115.png}
\caption{交换前} \label{fig_heap_1}
\end{figure}

\begin{figure}[ht]
\centering
\includegraphics[width=14.25cm]{./figures/6b91244dbdf27f81.png}
\caption{交换后} \label{fig_heap_2}
\end{figure}


看一下 down 操作:

\begin{lstlisting}[language=cpp]
void down(int u)
{
    int t = u;    // t 为:父结点、左子结点、右子结点三者的最小值
    if ((u << 1) <= cnt && h[u << 1] < h[t]) t = u << 1;
    if ((u << 1 | 1) <= cnt && h[u << 1 | 1] < h[t]) t = u << 1 | 1;
    if (u != t)    // 需要交换
    {
        heap_swap(u, t);
        down(t);
    }
}
\end{lstlisting}

up 操作:

\begin{lstlisting}[language=cpp]

// up 操作,只和我的父结点进行比较
void up(int u)
{
    int t = u;
    if (u / 2 && h[u / 2] > h[t]) t = u / 2;
    if (u != t)
    {
        heap_swap(u, t);
        up(t);
    }
}
\end{lstlisting}

以上就是堆的基本操作了。
