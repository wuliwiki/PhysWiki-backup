% 狄拉克符号
% keys 左矢|右矢|内积

\pentry{内积\upref{InerPd}}

在物理中, 尤其是量子力学中, 我们经常会见到\textbf{狄拉克符号(Dirac notation)}. 矢量空间 $X$ 中, 如果使用狄拉克符号, 那么其中的元素(矢量) $x\in X$ 可以记为 $\ket{x}$, 我们把它叫做\textbf{右矢(ket)}. 对应地, 我们把 $X$ 的对偶空间\upref{DualSp} $X^*$ 中的矢量叫做\textbf{左矢(bra)}, 记为 $\bra{x}$. bra 和 ket 是由英语单词 bracket 拆分而来的. braket 在这里指 $\braket{\phantom{1}}{\phantom{1}}$.

一般来说\footnote{严格来说, 这要求空间 $X$ 是\textbf{完备}% 链接未完成
的, 完备的内积空间又叫\textbf{希尔伯特空间\upref{Hilber}(Hilbert space)}, 是量子力学中主要讨论的矢量空间. 一般来说我们只在希尔伯特空间中使用狄拉克符号.}, 每个 $X^*$ 中的每个左矢和 $X$ 中的右矢是一一对应\upref{map}. 我们说任意 $\ket{x}$ 和对应的 $\bra{x}$ 互为\textbf{对偶}. 但粗略来说我们也可以不需要对偶空间的概念, 而是简单地把 bra 看作一个等待和右边某个 ket 点乘的矢量.

两矢量的加减法记为 $\ket{x} \pm \ket{y}$, $\ket{x}$ 与标量 $\lambda$ 的数乘可以记为 $\lambda\ket{x}$ 或 $\ket{\lambda x}$.

在对偶空间中, 加减法、 标量积、 算符作用同样记为 $\bra{x} \pm \bra{y}$, $\lambda\bra{x}$ 和 $A\bra{x}$. 但要注意的是, 我们将 $\ket{\lambda x}$ 的对偶矢量记为 $\bra{\lambda x}$, 这样一来, 在复矢量空间中把 $\lambda$ 写在左矢的里面和外面意义就变得不一样, 即 % 链接未完成: 链接到 “对偶空间” 中的某公式
\begin{equation}
\bra{\lambda x} = \lambda^* \bra{x}
\end{equation}

如果这个矢量空间中定义了内积, 那么 $X$ 中任意两个矢量 $\ket{x}, \ket{y}$ 的内积用狄拉克符号记为 $\braket{x}{y}$. 我们也可以理解为内积是左矢空间中 $\bra{x}$ 和右矢空间中 $\ket{y}$ 的二元运算.

算符(映射) $A:X\to X$ 作用在 $\ket{x}$ 上同样可以记为 $A\ket{x}$ 或 $\ket{Ax}$. $\ket{Ax}$ 的对偶记为 $\bra{Ax}$ 而不是 $A\bra{x}$ 或 $\bra{xA}$, 详见 “伴随算符\upref{adjoin}”.

\subsection{线性算符}
令 $N$ 维线性空间 $X$ 到 $M$ 维线性空间 $Y$ 的一组\textbf{正交归一基底}分别为 $\{\xi_i\}$ 和 $\{\eta_i\}$. 若线性映射\upref{LinMap}(线性算符) $A:X\to Y$ 表示成矩阵后, 矩阵元为 $A_{i,j}$, 那么该算符也可以用狄拉克符号表示为
\begin{equation}\label{braket_eq1}
A = \sum_{i,j} A_{i,j} \ket{\eta_i}\bra{\xi_j}
\end{equation}
这是因为: 若把 $A$ 作用在任意矢量 $\ket{x} = \sum_k x_k\ket{\xi_k}$ 上, 有
\begin{equation}\label{braket_eq2}
\begin{aligned}
\ket{y} &= A\ket{x} = \sum_{i,j} A_{i,j} \ket{\eta_i}\bra{\xi_j} \sum_k x_k\ket{\xi_k}\\
&= \sum_{i,j, k} A_{i,j} \ket{\eta_i} x_k \delta_{j,k}
= \sum_{i,j} A_{i,j} x_j \ket{\eta_i}
\end{aligned}
\end{equation}
这样就得到了矩阵乘法公式(\autoref{Mat_eq6}~\upref{Mat}) $y_i = \sum_{i,j} A_{i,j} x_j$. 要特别注意\autoref{braket_eq1} 中的基必须是正交归一的, 否则在\autoref{braket_eq2} 中就不可能得到克罗内克 delta 函数\upref{Kronec}.

特殊地, 如果 $A$ 是自映射, 即 $A:X\to X$, 那么只需把以上的 $Y$ 替换为 $X$, $\ket{\eta_i}$ 替换为 $\ket{\xi_i}$ 即可. 下同.

\subsection{矩阵元}
\autoref{braket_eq1} 中线性算符 $A$ 的矩阵元可以表示为 $\braket{\eta_i}{A\xi_j}$, 即先算出 $A\xi_j$ 再与 $\eta_i$ 内积. 但通常为了看起来更对称, 记为
\begin{equation}\label{braket_eq3}
A_{i,j} = \mel{\eta_i}{A}{\xi_j}
\end{equation}
证明见\autoref{MatLS_eq2}~\upref{MatLS}. 同样注意只有正交基底下的矩阵元才能用\autoref{braket_eq3} 表示.
