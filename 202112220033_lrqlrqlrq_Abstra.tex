% 抽象

% 数学|一般|广义|抽象|概念

\begin{issues}
\issueTODO
\end{issues}

我们这里要讨论的是\textbf{抽象(abstract)} 一词, 意为 “抽出一个对象的部分特性做研究而忽略其它特性” 或者 “抽出若干对象的共同特征做研究而抛弃它们的特有特征”,其中“象”取“类比”、“相似”之意.当你在脑海中想象“树”的概念时,只关注树共有的一些关键特质,比如有树叶 有枝干有根系等,至于具体多少树叶、根系如何分布则被忽略了,因此抽象的结果往往是不能画出来的,因为画出来的树都有了具体的特征了,不再是抽象的树.抽象的对立面,是具象.abstract一词是由ab-(向外的)和-tract(拉、拔)构成的,词义和抽象完全一样.当我们说一个事物更抽象时,我们其实也在说它更为 “一般” 或者更 “广义”.


人类最为基础的抽象就是对某事物数量,大小,形状等等特征的概念.比如,质量这一概念是对不同事物共同特征的抽象.不过,质量还能够继续抽象为数学上的实数,同样长度也是如此.因此我们不难看出抽象总是对某个具体对象的特征的描述,并且抽象这一过程是可以一层层深入进行下去的.

我们对具体事物抽象的过程很多时候就是归纳地找出其特征.物质的量自从人类绳结计数起就已经逐步有所触及,同时我们的大脑似乎生来就对我们所处的空间有一定的抽象,方向感和方向这些概念我们仿佛一直就会使用而不用去进行抽象的过程.不过,点线面体这些概念仍然还是需要被抽象出来的.抽象的过程与推理的过程截然不同,抽象更像是一门艺术,极为的依赖于天赋,灵感,创造力,洞见力,和想象力.在抽象的过程中,我们不仅需要敏锐地找到共同特性,并且还需要创造出某种结构或者规则去承载或者描述具体事物的信息.换句话说,我们从具体事物中抽象出来的理论或者概念,不仅仅是一种敏锐的发现,还是一种巧妙的创造.

\begin{figure}[ht]
\centering
\includegraphics[width=5cm]{./figures/Abstra_1.pdf}
\caption{我们可以将一个封闭的围墙抽象为二维的封闭线段} \label{Abstra_fig1}
\end{figure}

这时每条线段所承载的信息是每个墙壁.不过线段这一结构更加本质来说,承载的信息我想大家都能一眼看出就是长度.其实,图中的线段还有个不那么明显,但是尤为重要的性质,那就是方向.没错,不同抽象的概念或者结构可以结合在一起,比如在这里的线段就是长度和方向的结合.

抽象不仅能够方便我们描述具体事物的特征,还能够让我们应用抽象推演出一些巧妙和实用的东西.特别是当我们需要在实际中解决具体问题的时候.比如,我要粉刷围墙,但是我懒得把每条边都量一遍得到周长,那么请问我最少需要知道多少条线段就能够得出围墙周长?

很简单,我们只需要找到一个起点,然后“走”一圈并且记下每条线段的方向(上下左右).一对非垂直方向墙壁数目之和的最小值就是我们必须要测量的数目.比如:上行围墙:$5$,下行围墙:$9$,左行围墙:$2$,右行围墙:$30$,最少测量数目为$7$.

抽象的结构可以被改变,不过前提是要保证信息完整.比如,上面的例子中的线段方向可以只有两个,不过这就要求长度是对应到方向上的并且有正负,才能完整的描述出我们需要的信息.也就是说,我们可以拓展其结构使得该“线段”能够更加广泛的描述其他的事物.比如,方向可以是三维空间的任意方向,不过我们当然不需要无穷多个方向来描述三维空间.事实上,我们只需要$3$个相互垂直的方向,并且还需要再每个方向上确定一个有正负的“长度”量.这就是我们所熟悉几何上的矢量.

很多时候,我们反过来减少抽象的结构,将已经抽象的事物更进一步的抽象,反而能够得到更多,更巧妙,更广泛的描述.
\addTODO{例子:从位移, 速度, 加速度等中抽象出矢量}
