% 丘成桐(综述)
% license CCBYSA3
% type Wiki

本文根据 CC-BY-SA 协议转载翻译自维基百科\href{https://en.wikipedia.org/wiki/Shing-Tung_Yau}{相关文章}。

\begin{figure}[ht]
\centering
\includegraphics[width=6cm]{./figures/fb16385d5284d04b.png}
\caption{} \label{fig_QCT_1}
\end{figure}
丘成桐(Shing-Tung Yau,发音:/jaʊ/;中文:丘成桐;拼音:Qiū Chéngtóng;1949年4月4日出生)是一位中美籍数学家。他是清华大学丘成桐数学科学中心的主任,同时是哈佛大学的名誉教授。直到2022年,丘成桐一直担任哈佛大学威廉·卡斯帕·格劳斯坦数学教授,之后他移居清华大学。

丘成桐1949年出生于汕头,年幼时移居英国香港,1969年移居美国。他因在偏微分方程、卡拉比猜想、正能量定理和蒙热–安培方程等方面的贡献而于1982年获得菲尔兹奖。丘成桐被认为是现代微分几何和几何分析发展的主要贡献者之一。他的工作在凸几何、代数几何、计数几何、镜像对称、广义相对论、弦理论等数学和物理领域产生了深远的影响,同时他的研究也涉及到应用数学、工程学和数值分析等领域。
\subsection{传记}  
丘成桐1949年出生于中华民国广东省汕头市,父母为客家人。[YN19] 他的祖籍是中国嘉应县。[YN19] 他的母亲梁玉兰来自中国梅县区;父亲丘镇英(Chen Ying Chiu)是中华民国国民党学者,涉猎哲学、历史、文学和经济学。[YN19] 他是家中八个孩子中的第五个。[4]

在中国大陆发生共产主义接管时,丘成桐还只有几个月大,他的家人移居到英国香港,并在那里接受教育(除了英语课外),他的学业完全用粤语,而不是父母的母语客家话。[YN19] 他直到1979年,才在华罗庚的邀请下回到大陆,那时中国大陆进入改革开放时代。[YN19] 他们最初住在元朗,1954年搬到沙田。[YN19] 由于失去了所有财产,他们家经济拮据,而他的父亲和第二个姐姐在他十三岁时相继去世。[YN19] 丘成桐开始阅读并欣赏父亲的书籍,变得更加专注于学业。完成培正中学学业后,他于1966至1969年间在香港中文大学学习数学,由于提前毕业,他未获得学位。[YN19] 他将课本留给了他的弟弟丘成栋,后者也决定主修数学。

丘成桐于1969年秋季前往加利福尼亚大学伯克利分校攻读数学博士学位。在寒假期间,他阅读了《微分几何学报》的第一期,并深受约翰·米尔诺(John Milnor)关于几何群体理论的论文启发。[5][YN19] 随后,他提出了普雷斯曼定理的一个推广,并在接下来的学期与布莱恩·劳森(Blaine Lawson)共同进一步发展了这一思想。[6] 基于这项工作,他于1971年获得博士学位,导师是陈省身(Shiing-Shen Chern)。[7]

他在普林斯顿高级研究院度过了一年后,于1972年加入石溪大学担任助理教授。1974年,他成为斯坦福大学的副教授。[8] 1976年,他在加州大学洛杉矶分校(UCLA)担任访问教授,并与物理学家郭玉云结婚,郭玉云是他在伯克利大学攻读研究生时认识的。[8] 1979年,他返回普林斯顿高级研究院,并于1980年成为该院教授。[8] 1984年,他接受了加州大学圣地亚哥分校的讲席教授职位。[9] 1987年,他搬到哈佛大学。[8][10] 2022年4月,丘成桐从哈佛大学退休,担任哈佛大学威廉·卡斯帕·格劳斯坦数学教授名誉教授。[8] 同年,他移居清华大学,担任数学教授。[8][2]