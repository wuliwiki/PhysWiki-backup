% 电磁相互作用
% license CCBYSA3
% type Wiki

(本文根据 CC-BY-SA 协议转载自原搜狗科学百科对英文维基百科的翻译)
\textbf{电磁学}是物理学的一个分支,是对\textbf{电磁力}这种发生在带电粒子之间的物理相互作用的研究。电磁力由电场和磁场所组成的电磁场所介导,它也对电磁辐射(例如可见光)负责,是自然中四种基本互相作用(通常被称为力)之一。其他三种基本相互作用是强相互作用、弱相互作用和万有引力。[1]在高能量下,弱力和电磁力被统一为单一的弱电力。

电磁现象是根据电磁力来定义的(后者有时也被称为洛伦兹力),包括电和磁作为同一现象的不同表现形式。电磁力在决定日常生活中遇到的大多数物体的内部性质方面起着重要作用。原子核和它们的轨道电子之间的电磁吸引力使得原子保持一个整体的状态。电磁力负责原子之间的化学键(分子的成因)和分子间作用力。电磁力控制着所有的化学过程,这些化学过程来源于相邻原子中的电子之间的相互作用。

存在很多电磁场的数学描述。在经典电动力学中,电场被描述为电势和电流。在法拉第定律中,磁场与电磁感应和磁性相关联。麦克斯韦方程组描述了电场和磁场是如何由彼此以及电荷和电流所产生和改变的。

电磁学的理论结果(特别是基于传播“介质”的性质(磁导率和电容率)所建立的光速)导致了阿尔伯特·爱因斯坦于1905年对狭义相对论的发展。

\subsection{理论的历史}
最初,电与磁被认为是两种独立的力。然而,詹姆斯·克拉克·麦克斯韦于1873年出版的《电与磁论文集》改变了物理学家们的这一观点。在麦克斯韦的著作中,正电荷和负电荷的相互作用被证明是由一种力所介导的。这些相互作用产生了四种主要效应,所有这些都已经被实验清楚地证明了:
\begin{enumerate}
\item 电荷之间的互相吸引或者排斥是通过一种反比于它们之间距离平方的力:异种电荷相互吸引,同种电荷相互排斥。
\item 磁极(或单个点的极化状态)以类似正电荷和负电荷的方式相互吸引或排斥,并且总是成对存在:每个北极都与一个南极相连。
\item 电线内部的电流在电线外部产生相应的圆周磁场。它的方向(顺时针或逆时针)取决于导线中电流的方向。
\item 当线圈朝向或远离磁场移动或者磁体朝向或远离线圈移动时,电流便在线圈中产生;电流的方向取决于运动方向。
\end{enumerate}
\begin{figure}[ht]
\centering
\includegraphics[width=8cm]{./figures/7e59008e23b66720.png}
\caption{汉斯·克里斯蒂安·奥斯特。} \label{fig_DCXHZY_1}
\end{figure}
在准备1820年4月21日晚上的演讲时,汉斯·克里斯蒂安·奥斯特做了一个令人惊讶的观察。当他准备材料时,他注意到一个指南针偏离了磁北极,这发生在他使用的电池的电流被打开和关闭时。这种偏转使他相信磁场从携带电流的电线的所有侧面辐射出来,就像光和热一样,并且它证实了电与磁之间的直接关系。

在做出这个发现时,奥斯特没有对这一现象提出任何令人满意的解释,也没有试图用数学框架来表示这一现象。然而,三个月后,他开始了更深入的调查。此后不久,他发表了他的发现,证明电流在流经导线时会产生磁场。磁感应的CGS单位(奥斯特)就是为了纪念他对电磁学领域的贡献而命名的。