% Feshbach 共振
% keys 共振|Feshbach|Hamiltonian|哈密顿算符

相互作用 Hamiltonian,我们考虑一个坐标表象张量积一个 open-closed 的 channel 表象. 这种情况下,可以写为:
\begin{equation}
\hat{H} = -\frac{\hbar^2\nabla^2}{m}\otimes I_2+V_{2\times2}
\end{equation}
其中因为约化所以 $2m\to m$. 此外,
\begin{equation}
V_{2\times2} = \begin{cases}
\left(\begin{matrix}
-V_{\text{open}} & W\\
W & -V_{\text{close}}
\end{matrix}\right) & (r < r_0)\\
\left(\begin{matrix}
0 & 0\\
0 & +\infty
\end{matrix}\right) & (r > r_0)
\end{cases} 
\end{equation}
这其实就是对我们之前的物理要求:在远处close通道有一个远高于能标的渐进值,近处有一个相互作用($W$)但是很小. 它是由超精细结构提供的,比 $|V_{\text{open}}-V_{\text{close}}|$ 小几个量级.

在很前面的时候,我们知道,$E\to0$ 的情况下,波函数可以写为 $\Psi=\frac{r-a_s}{r}|\text{open}\rangle$.

在近处,我们考虑相互作用部分的本征态 $|\pm\rangle$:
\begin{equation}
|+\rangle = \left(\begin{matrix}\cos\theta\\\sin\theta\end{matrix}\right),\quad |-\rangle = \left(\begin{matrix}-\sin\theta\\\cos\theta\end{matrix}\right)
\end{equation}
其中,$\tan2\theta = 2W/|V_{\text{open}}-V_{\text{close}}|$,本征值可以近似认为是 $V_{\text{open}}$ 和 $V_{\text{close}}$. 可以用未归一的波函数来描述这两个本征态各自独立的情况:
\begin{equation}
|\Psi\rangle\propto\frac{1}{r}\left[\sin(q_+r)|+\rangle + A\sin(q_-r)|-\rangle\right]
\end{equation}
其中,$q_+,q_-$ 为该能量下的波矢, $A$ 是一个系数. 由于整体 $E\to0$ 为零能散射,所以 $\hbar^2q^2/m$ 就应该等于负的相互作用本征值,也就是有
\begin{equation}
\frac{\hbar^2q_{+/-}^2}{m} = V_{\text{open}}/V_{\text{close}}
\end{equation}

考虑在 $r\geqslant r_0$ 的位置,体系波函数就不再会包含close成分,也就是说, $\langle\text{close}|\Psi\rangle(r_0) = 0$,从而告诉我们这个系数 $A$ 满足
\begin{equation}
A=-\tan\theta\frac{\sin(q_+r_0)}{\sin(q_-r_0)}
\end{equation}
此外,open态在 $r_0$ 处要连续,不仅是波函数还有其导数. 利用之前得到的关系,可以有
\begin{equation}
\left.\frac{L'}{L}\right|_{r=r_0} = \left.\frac{(r-a_s)'}{r-a_s}\right|_{r=r_0}
\end{equation}
其中
\begin{equation}
L = \sin(q_+r)\cos\theta - \frac{\sin(q_+r_0)}{\sin(q_-r_0)} \sin(q_-r)\tan\theta
\end{equation}
经过化简我们得到
\begin{equation}
\frac{q_+\cos^2\theta}{\tan(q_+r_0)}+\frac{q_-\sin^2\theta}{\tan(q_-r_0)} = \frac{1}{r_0-a_s}
\end{equation}

可以看到,因为 $\theta\to0$,主要成分有 $q_+$ 贡献. 在没有相互作用的时候,我们有
\begin{equation}
\frac{1}{r_0-a_{\text{bg}}} = \frac{q_+}{\tan(q_+r_0)}
\end{equation}
只有在 $\tan(q_-r_0)\to0$ 的时候,第二项才有贡献.
考虑一个体系的束缚态,能量 $E_c$,其满足
\begin{equation}
\left.\sin(qr)\right|_{r=0,r_0} = 0,\quad q = \sqrt{\frac{m(V_c+E_c)}{\hbar^2}}
\end{equation}
在 $E_c\to0$ 的时候, $q_-\sim q$,从而
\begin{equation}
\tan(q_-r_0)\sim (q_- -q) r_0 = \frac{1}{\hbar}\sqrt{m}(\sqrt{V_c}-\sqrt{V_c+E_c}) =-\frac{mE_c}{2\hbar^2 q_-}
\end{equation}
也就是说,第二项约为
\begin{equation}
-\frac{2\hbar^2q_-^2\theta^2}{mr_0}\frac{1}{E_c} = -\gamma\frac{1}{E_c}, \gamma\equiv\frac{2\hbar^2q_-^2\theta^2}{mr_0} 
\end{equation}
此时,
\begin{equation}
\frac{a_s-r_0}{a_{\text{bg}}-r_0} = 1-\frac{\Delta B}{B-B_{\text{res}}}
\end{equation}
其中, $B$ 为施加的磁场,而我们的 $E_c$ 则由磁场调控(因为磁矩不一样): $E_c\to E_c+\delta\mu B$,而 $\Delta B = \frac{\hbar^2}{\delta\mu m}\gamma(a_{\text{bg}}-r_0)$, $B_{\text{res}} = -{E_c}/{\delta\mu}+\Delta B$,都是一些以 $B$ 参数的东西.