% java 入门
% license Xiao
% type Tutor


首先需要安装java,笔者是使用idea软件,下载好idea软件后,再去下载jdk,jdk下载可以去 Oracle 官方网站。jdk最好可以下载比较老的版本,新版本要收费,最好是1.7版本的。

下载好之后我们就可以开始我们的第一个程序啦。首先打开idea软件新建一个项目,然后再右边编辑区输入我们的第一个程序代码
\begin{lstlisting}[language=js]
public class Main{
    public static void main(String[] args){
        System.out.println("Hello Java");
    }
}
\end{lstlisting}
此时文件名为Main.java
如下图\begin{figure}[ht]
\centering
\includegraphics[width=14cm]{./figures/c8f764ebab1a255c.png}
\caption{javacode1} \label{fig_java_1}
\end{figure}
此时,注意在代码的第一句
\begin{lstlisting}[language=js]
public class Main{
\end{lstlisting}
这里的Main应该是和我们的文件名是一致的,不然就会导致编译错误。
\begin{lstlisting}[language=js]
 public static void main(String[] args){
\end{lstlisting}

这句话是 Java 程序的入口点(entry point),表示程序从这里开始执行。让我们逐部分解释。\\
public是访问修饰符:public 表示这个方法是公开的,其他类可以访问和调用这个方法。因为 main 方法是程序的入口点,它必须是 public 的,以便 Java 虚拟机(JVM)可以访问和调用它。
\\
static静态修饰符:static 表示这个方法是静态的,不属于某个对象实例,而是属于类本身。因为程序启动时还没有任何对象被创建,main 方法必须是静态的,以便 JVM 可以在不创建类的实例的情况下调用它。\\
void
返回类型:void 表示这个方法没有返回值。当这个方法执行完毕后,它不会返回任何数据。\\
main
方法名:main 是方法的名字,这是一个特殊的方法名。JVM 会寻找这个名字的方法作为程序的入口点。\\
String[] args
参数:String[] args 是方法的参数。它是一个 String 类型的数组,用来接收从命令行传递给程序的参数。即使你没有传递任何参数,这个参数也会被定义为空数组。\\
方法体
{ } 花括号包围的方法体表示 main 方法的具体实现代码。你可以在这里编写程序的主要逻辑。

这就是一个Hello java的小程序了\begin{figure}[ht]
\centering
\includegraphics[width=14cm]{./figures/c0e1a02a03d0fcd8.png}
\caption{javacode2} \label{fig_java_2}
\end{figure}
这是最后运行得到的结果。