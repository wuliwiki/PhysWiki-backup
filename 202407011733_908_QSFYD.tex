% 轻水反应堆
% license CCBYSA3
% type Wiki

(本文根据 CC-BY-SA 协议转载自原搜狗科学百科对英文维基百科的翻译)

\textbf{轻水反应堆(LWR)}是一种热中子反应堆,它使用普通的水,而不是重水,作为它的冷却剂和中子慢化剂——此外,一种固体形式的裂变元素被用作燃料。热中子反应堆是最常见的核反应堆,轻水反应堆则是最常见的热中子反应堆。

轻水反应堆有三种类型:压水堆(PWR)、沸水堆(BWR)和超临界水堆 (SCWR)。

\subsection{历史}
\subsubsection{1.1 早期概念和实验}
在发现了裂变、慢化以及核链式反应的理论可能性后,早期的实验结果迅速表明,天然铀只能在石墨或重水作为慢化剂的情况下发生持续的链式反应。而世界上第一批反应堆( CP-1 、 X10 等)成功达到临界状态后,铀浓缩开始从理论概念发展到实际应用,以满足曼哈顿计划的目标,即制造核爆炸。

1944年5月,在洛斯阿拉莫斯的低功率(LOPO) 反应堆中,第一批生产出来的浓缩铀达到了临界质量,该反应堆被用来估计制造原子弹的U235的临界质量。[1]LOPO不能被认为是第一座轻水反应堆,因为它的燃料不是固体铀化物,且包裹着耐腐蚀材料中,而是由溶解在水中的硫酸铀酰盐组成。[2]但它是第一个含水均相反应堆,也是第一个以浓缩铀为燃料,以普通水为慢化剂的反应堆。[1]

第二次世界大战末期,根据阿尔文·温伯格的想法,天然铀燃料元素被安置在X10反应堆顶部普通水中的晶格中,以评估中子倍增系数。[3]本实验的目的是确定一个以轻水为慢化剂和冷却剂,并以固体铀为燃料的核反应堆的可行性。结果表明,用低浓缩铀可以达到临界状态。[4]这个实验是迈向轻水反应堆的第一步。

第二次世界大战后,随着浓缩铀的出现,新的反应堆概念变得可行。1946年,尤金·维格纳(Eugene Wigner)和阿尔文·温伯格(Alvin Weinberg)提出并发展了使用浓缩铀作为燃料,并以轻水作为慢化剂和冷却剂的反应堆概念。[3]这个概念是为一个反应堆提出的,其目的是测试材料在中子通量下的行为。这个反应堆——材料测试反应堆(MTR) 建于爱达荷州的 INL ,于1952年3月31日达到临界状态。[5]对于该反应堆的设计,实验是必要的,因此在 ORNL 建造了实物模型(MTR),以评估一回路的液压性能,然后测试其中子学特性。这个MTR模型,后来被称为低强度试验反应堆(LITR),于1950年2月4日达到临界状态,[6]是世界上第一个轻水反应堆。[7]

\subsubsection{1.2 第一座压水堆}
第二次世界大战结束后,美国海军在首长海曼·里科弗(后来的海军上将)的指导下,开始了一项计划,目标是为船舶的提供核推进。它在20世纪50年代早期开发了第一个压水堆,并成功部署了第一艘核潜艇“鹦鹉螺号”(USS Nautilus (SSN-571)。

苏联在20世纪50年代末独立研制了一种名为VVER的压水堆反应堆。虽然在功能上与美国的设计非常相似,但它与西方压水堆也有一定的设计区别。

\subsubsection{1.3 第一座沸水堆}
研究员塞缪尔·恩特尔梅尔二世 (Samuel Untermyer II)在美国国家反应堆试验站(现在的爱达荷国家实验室Idaho National Laboratory )领导了一项名为硼砂实验的系列试验,以开发沸水堆(BWR)。

\subsection{概论}
\begin{figure}[ht]
\centering
\includegraphics[width=8cm]{./figures/ede476a7a9f25af4.png}
\caption{Koeberg 核电站由两个以铀为燃料的压水堆组成} \label{fig_QSFYD_1}
\end{figure}
被称为轻水反应堆(LWR)的核反应堆系列,使用普通水进行冷却和慢化,建造起来往往比其他类型的核反应堆更简单、更便宜;由于这些因素,截至2009年,它们在全世界服役的民用核反应堆和海军推进反应堆中占绝大多数。轻水堆可分为三类——压水堆(PWRs)、沸水堆(BWRs)和超临界水堆(SCWR )。截至2009年,超临界水堆(SCWR) 仍处于假设状态;它属于第四代核电设计技术;但它只是由轻水部分慢化,并表现出快中子反应堆的某些特性。

许多国家在压水堆方面掌握着领先的运行经验,提供反应堆出口的领导国家有几个,美国(提供具有固有安全的AP1000 ,以及一些小规模、模块化、具有固有安全性的压水堆,如 Babcock & Wilcox MPower以及 NuScale MASLWR)、俄罗斯(提供VVER-1000和VVER-1200供出口)、法国(提供阿海珐EPR出口),日本(提供三菱公司设计制造的先进压水堆出口);此外,中国和韩国两者都迅速地上升到压水堆建造国前列,中国正在进行大规模的核电扩张计划,韩国目前正在设计和建造他们的第二代自主核电设计。美国和日本与通用电气(General Electric)和日立(Hitachi)结成联盟,提供先进的沸水反应堆(ABWR)和经济简化的沸水反应堆(ESBWR),用于建设和出口;此外,东芝还为在日本的建筑提供了ABWR改型。西德也曾是BWR的主要建设、运营者。用于发电的其他类型的核反应堆是重水慢化反应堆,加拿大建造(CANDU)和印度建造的(AHWR),英国建造了先进气体冷却反应堆(AGCR),俄罗斯,法国和日本分别建造了液态金属冷却反应器堆(LMFBR),由石墨慢化的水冷反应堆(RBMK or LWGR)仅在俄罗斯和前苏联国家被建造过。

尽管所有这些类型的反应堆的发电能力是相当的,由于上述特点,以及LWR在运行方面的丰富经验,它在绝大多数新的核电厂中受到青睐,并被建设。此外,轻水反应堆构成了为海军核动力船舶提供动力的反应堆的绝大多数。拥有核海军推进能力的五个大国中,有四个国家专门使用轻水反应堆:英国皇家海军,中国人民解放军海军,法国国家海军陆战队和美国海军。只有俄罗斯的海军使用了相对较少的液态金属冷却反应器堆,特别是705型核潜艇,它使用铅铋共晶作为反应堆慢化剂和冷却剂,但绝大多数俄罗斯核动力船舶只使用轻水反应堆。LWR在核动力海军舰艇上几乎专用的原因是这些类型反应堆的固有安全水平。由于轻水在这些反应堆中既用作冷却剂又用作中子慢化剂,如果其中一个反应堆因军事行动而受损,导致反应堆堆芯完整性受损,轻水慢化剂的释放将起到停止核反应和关闭反应堆的作用。这种能力被称为反应性负反应系数。

目前已有的的轻水反应堆堆型包括以下几类:
\begin{itemize}
\item ABWR
\item AP1000
\item APR-1400
\item CPR-1000
\item EPR
\item VVER
\end{itemize}

\subsubsection{2.1 轻水反应堆的统计数据}

国际原子能机构 2009年的数据:[8]
\begin{table}[ht]
\centering
\caption\label{QSFYD}
\begin{tabular}{|c|c}
\hline
正在运行的反应堆数量 & 359\\
\hline
正在建设的反应堆数量 & 27\\
\hline
拥有轻水反应堆的国家数量 & 27\\
\hline
总发电量(千兆瓦)。 & 328.4\\
\hline
\end{tabular}
\end{table}

\subsection{反应堆设计}
轻水反应堆通过受控核裂变产生热量。核反应堆堆芯是核反应堆发生核反应的部分。它主要由核燃料和控制元件组成。这些铅笔般细的核燃料棒,每根大约12英尺(3.7米)长,按数百个分组,被称为燃料组件。在每个燃料棒内部,含有铀或者更常见的是氧化铀,是首尾相连堆叠的。控制元件被称为控制棒,里面充满了像铪或镉这样的元素,它们很容易捕捉中子。当控制棒下降到核心时,它们吸收中子,中子因此不能参与链式反应。相反,当控制棒被提起时,更多的中子会撞击附近燃料棒中可裂变的铀-235或钚-239原子核,链式反应就会加剧。所有这些材料都被封闭在一个充水的钢制压力容器中,叫做反应堆容器。

在沸水堆,裂变产生的热量将水转化为蒸汽,蒸汽直接驱动发电涡轮机。但是在压水堆中,裂变产生的热量通过热交换器传递到次级回路。蒸汽在次级回路中产生,次级回路驱动发电涡轮机。在这两种情况下,蒸汽流过涡轮机后,在冷凝器中会被冷却成水。[9]
\begin{itemize}
\item 沸水堆的动画图
\begin{figure}[ht]
\centering
\includegraphics[width=10cm]{./figures/a7ddc1f49be2598b.png}
\caption\label{fig_QSFYD_2}
\end{figure}

\item 压水堆的动画图
\begin{figure}[ht]
\centering
\includegraphics[width=10cm]{./figures/0e97f52ce566be66.png}
\caption\label{fig_QSFYD_3}
\end{figure}
\end{itemize}
冷却冷凝器所需的水取自附近的河流或海洋。经过冷凝器加热后,温度上升,并被泵重新打入河流或海洋中。另一方面,反应堆冷却水的热量也可以通过冷却塔散发到大气中。美国主要使用轻水反应堆发电,而加拿大使用的是重水反应堆。[10]

\subsubsection{3.1 控制}
\begin{figure}[ht]
\centering
\includegraphics[width=6cm]{./figures/b97b363c915ba553.png}
\caption{压水堆的顶部,控制棒在顶部可见} \label{fig_QSFYD_4}
\end{figure}
控制棒通常组合成控制棒组件——典型的商业压水堆组件有20根控制棒——并插入燃料元件内的导管中。控制棒从堆芯的中心取出或插入,以控制将进一步分裂铀原子的中子数。这反过来又会影响反应堆生成的热能、产生的蒸汽量,从而影响发电量。控制棒从堆芯中部分移除,以允许堆芯发生链式反应。插入控制棒的数量和插入距离可以变化,从而控制反应堆的反应性。

通常还有其他控制反应性的方法。在压水堆设计中,将可溶性的中子吸收剂(通常为硼酸)添加到反应堆冷却剂中,使控制棒在固定功率运行时完全抽出,确保整个堆芯的功率和通量分布均匀。沸水堆设计的操作人员通过改变反应堆循环泵的流量,利用通过堆芯的冷却剂流来控制反应性。通过堆芯的冷却剂流量的增加有利于蒸汽气泡的去除,进而增加了冷却剂/慢化剂的密度,以增加了堆芯功率。