% 总目录
% license CCBYSA3
% type Toc

\input{./others/format}
\renewcommand\thesubsubsection{\thesection.\arabic{subsection}.\arabic{subsubsection}}
\begin{document}
\renewcommand{\thelstlisting}{\arabic{lstlisting}}
\input{./others/MatlabStyle} % 代码格式
\begin{titlepage}
\includepdf{./figures/frontcover.pdf}
\newpage
\includepdf{./figures/dedication.pdf}
\end{titlepage}
\frontmatter % 开始罗马数字页码
\input{./contents/FrontMatters} % 版权声明 & 关于本书
\setcounter{tocdepth}{1}
\tableofcontents % 生成目录
\mainmatter % 开始阿拉伯数字页码
\setcounter{secnumdepth}{3}

\part{科普} \label{prt_PS}
%=================================
\chapter{数学} \label{cpt_1ac}
%---------------------------------------
% \addTODO{例如根号二是无理数、整数和有理数一样多、(想到什么可以添加)}
\entry{抽象(科普)}{Abstra}
\entry{数学结构(科普)}{Struct}
\entry{高尔顿钉板(科普)}{GRDDB}

\chapter{经典力学} \label{cpt_9}
%---------------------------------------
\entry{如何自学物理}{SdyPhy}
\entry{经典力学及其他物理理论}{MecThe}
\entry{经典力学(科普)}{CM0}
\entry{圆周运动:为什么卫星不会掉下来(科普)}{CMintr}
\entry{为什么月球没有被太阳吸走(科普)}{EquivI}
\entry{牛顿第二定律的矢量形式}{New2}
\entry{动量和能量、一维势能曲线(科普)}{CM1}
\entry{平方反比定律(综述)}{PFFBDL}
\entry{傅科摆(科普)}{FouPen}
\entry{角动量(科普)}{AngMo}
% \entry{角动量简介}{CM2}
\entry{网球拍定理(科普)}{Dzhani}
\entry{旋转的陀螺为什么不会倒(科普)}{PrecPS}

\chapter{电与磁} \label{cpt_10}
%---------------------------------------
\entry{电动力学}{EM0}
\entry{静电的基本规律和性质}{EM1}
\entry{荷质比的测定}{Charge}
\entry{右手定则}{RHRul}
\entry{电流产生磁场}{CurMag}
\entry{电路和水路的类比}{EleWat}
\entry{狭义相对论(科普)}{Relat0}

\chapter{量子力学} \label{cpt_11}
%---------------------------------------
\entry{原子的观念(量子序曲)}{AtomId}
\entry{运动的观念(量子序曲)}{QMPre}
\entry{数的观念 (量子序曲)}{QMPre6}
\entry{从天球的音乐到玻尔模型(量子序曲)}{ClBohr}
\entry{波动⼒学 (量子序曲)}{QMPre7}
\entry{永恒的陀螺 (量子序曲)}{QMPre2}
\entry{投影和表示(量子序曲)}{QMPre3}
\entry{线性谐振⼦ (量子序曲)}{QMPre4}
\entry{波动和粒子(量子序曲)}{QMPre5}
\entry{原子吸收光谱(综述)}{AAS}
\entry{量子力学的基本原理(科普)}{QM0}
\entry{量子力学的诠释(综述)}{QMinet}
\entry{量子力学常见问题(科普)}{QmFAQ}
% \entry{原子结构(高中)}{atomH2}
% \entry{量子力学科普视频脚本}{QMvid}

\chapter{光学} \label{cpt_OPTps}
%------------------------------------
\entry{电磁波吸收(综述)}{XSGX}
\entry{多模光纤(综述)}{DMGQ}

\chapter{计算机} \label{cpt_CSps}
%-------------------------------------
\entry{【导航】文件管理}{FileIn}
\entry{集成电路}{icJCDL}
\entry{进制、二进制}{Binary}
\entry{文本文件与字符编码(ASCII、UTF)}{encode}
\entry{计算机文件简介}{ComFil}
\entry{校验和、哈希值}{chkSum}
\entry{计算机文件备份基础(附 python 增量备份脚本)}{SimBac}
% \entry{个人数据安全教程}{PsData}
\entry{如何给文件加密(含 python 加密脚本)}{encryp}
\entry{用网盘增量备份文件}{PanBak}

\chapter{其他} \label{cpt_12}
%---------------------------------------
% \entry{反射}{RefMir}
\entry{折射(综述)}{ZS}
\entry{相对论(综述)}{XDL}
\entry{太阳能电池}{TYNDC}
\entry{天文学常识}{Astro}
\entry{时间的计量}{TimeCa}
\entry{时间的计量 2}{time2}
\entry{材料科学简介}{MSEINT}
\entry{金属材料结构(科普)}{MetInt}
\entry{金属的变形(科普)}{MetDfm}
\entry{能带模型(科普)}{BNDGP}
\entry{芝诺时间}{zeno}
\entry{物质的量与摩尔(高中)}{MOLE}

\part{高中数学} \label{prt_HSmath}
%======================================
\chapter{初中回顾} \label{cpt_a0}
%---------------------------------------
\entry{【导航】高中数学}{HsMOv}
% 一元二次方程
% TODO:配方法解一元二次方程
\entry{二次方程求根公式}{quasol}
\entry{韦达定理(高中)}{VieHi}
% 线性方程组
\entry{线性方程组(高中)}{LinEqu}
%---------------------------------------
\chapter{集合与命题} \label{cpt_a1}
%---------------------------------------
\entry{集合(高中)}{HsSet}
\entry{命题与推理(高中)}{HsLogi}
\entry{复合命题(高中)}{HsCoPr}
\entry{集合的基本运算(高中)}{HsSeOp}
\entry{复数}{CplxNo}
%---------------------------------------
\chapter{函数} \label{cpt_a2}
%---------------------------------------
\entry{函数(高中)}{functi}
\entry{函数的性质(高中)}{HsFunC}
\entry{函数的变换(高中)}{FunTra}
\entry{幂运算与幂函数(高中)}{power}
\entry{指数函数(高中)}{HsExpF}
\entry{对数与对数函数(高中)}{Ln}
\entry{导数(高中)}{HsDerv}
%\entry{反函数(高中)}{InvFun}
%---------------------------------------
\chapter{数列} \label{cpt_a3}
%---------------------------------------
\entry{数列的概念与函数特性(高中)}{HsSeFu}
\entry{等差数列(高中)}{HsAmPg}
\entry{等比数列求和}{HsGmPg}
\entry{等差与等比数列总结(初等数学)}{AGS}
\entry{求和符号(高中)}{SumSym}
%---------------------------------------
\chapter{三角函数} \label{cpt_1}
%---------------------------------------
\entry{角与有向角(高中)}{HsAngl}
\entry{三角函数(高中)}{HsTrFu}
\entry{三角恒等变换(高中)}{HsAnTf}
\entry{三角恒等式}{TriEqv}
\entry{正弦定理、余弦定理}{CosThe}
%---------------------------------------
\chapter{排列、组合和概率与统计}} \label{cpt_2}
%---------------------------------------
% 排列、组合
\entry{阶乘(高中)}{factor}
\entry{求积符号(累乘)}{ProdSy}
\entry{组合(高中)}{HsCb}
\entry{二项式定理(高中)}{HsBN}
\entry{隔板法(排列组合)}{BarCom}
% Giacomo:应当合并,标注超出高中的内容
\entry{排列}{permut}
\entry{组合}{combin}
\entry{二项式定理}{BiNor}

% 概率
\entry{离散型随机变量(高中)}{HsDRV}
\entry{条件概率与事件的独立性(高中)}{HsCpMi}
\entry{随机变量的数字特征(高中)}{HsRdNc}
\entry{概率习题(高中)}{HsPbEc}
%---------------------------------------


\chapter{几何向量} \label{cpt_19}
%---------------------------------------
% Giacomo:几何向量这个板块的基本思路应该是,先用“几何”性质来定义向量的性质,再用向量的“代数性质”给出等价条件。
\entry{几何向量}{GVec}
\entry{几何向量的加法和数乘}{GVecOp}
\entry{几何向量的点乘}{Dot}
\entry{几何向量的线性相关性}{linDpe}
\entry{线性相关与线性组合}{LnDpd2} % 待删除/整合
\entry{几何向量的基底和坐标}{Gvec2}

\entry{几何向量的线性变换}{LTrans}
\entry{平面旋转变换}{Rot2DT}
\entry{平移、旋转与缩放的组合变换}{TrRoSc}

% 矢量叉乘
\entry{几何向量的叉乘}{Cross}
\entry{矢量叉乘分配律的几何证明}{CrossP}
\entry{连续叉乘的化简}{TriCro}


\chapter{平面解析几何} \label{cpt_7}
%---------------------------------------
% Giacomo:顺序有待整理
\entry{解析几何}{JXJH}
\entry{三角形面积、海伦—秦九韶公式}{Heron}
\entry{直线和平面的交点}{LPint}
\entry{点到直线的距离}{P2Line}
\entry{直线和球的交点}{LinSph}
\entry{三角形的外接圆}{SphTri}

\entry{极坐标系}{Polar}

\entry{仿射法在解析几何中的应用}{affine}
\entry{圆锥曲线的极坐标方程}{Cone}
\entry{抛物线}{Para3}
\entry{椭圆}{Elips3}
\entry{双曲线}{Hypb3}
\entry{圆锥曲线和圆锥}{ConSec}
\entry{圆锥曲线(总结)}{conic}
\entry{圆锥曲线的光学性质}{ConOpt}
\entry{圆锥曲线的配极(高中)}{EclPol}

\chapter{立体几何} \label{cpt_7a}
%---------------------------------------
\entry{柱坐标系}{Cylin}
\entry{柱坐标与直角坐标的转换}{CyCar}
\entry{球坐标系}{Sph}
\entry{球坐标与直角坐标的转换}{SphCar}
\entry{球坐标的旋转变换}{SphRot}
\entry{解三棱锥顶角}{PrmSol}
\entry{足球顶点坐标的计算}{FootBl}
\entry{日晷的计算}{SunDia}

\chapter{扩展视野} \label{cpt_ext}
%---------------------------------------
\entry{利萨茹曲线}{Lissaj}
\entry{阿基米德螺线}{ArcSpl}
\entry{摆线}{cycloi}
\entry{手动计算开根号(长除法)}{Hsqrt}
\entry{手动计算开根号(泰勒展开法)}{TSqrt}
\entry{函数的旋转和其他变换}{FunRot}
\entry{抛物线坐标系}{ParaCr}
\entry{椭圆坐标系、椭球坐标系}{EliCor}

\chapter{待整理} \label{cpt_6}
%---------------------------------------

\entry{sinc 函数}{sinc}
\entry{三角函数 n 倍角公式}{sinNx}
\entry{双曲函数}{TrigH}
\entry{利用复数方法证明三角恒等式}{TriCom}
\entry{反三角函数}{InvTri}
\entry{四象限 Arctan 函数(atan2)}{Arctan}
\entry{一元函数的对称与周期性}{shenry}
\entry{韦恩图}{VennD}
\entry{充分必要条件}{SufCnd}
\entry{三矢量的混合积}{TriVM}

\part{高中物理} \label{prt_HSphys}
%======================================
\chapter{基本概念} \label{cpt_HSphysFoundation}
% ----------------------------------
\entry{高中物理导航}{HSPbas}
\entry{科学计数法、数量级(高中)}{OrdMag}

\chapter{经典力学} \label{cpt_8}
%--------------------------------
\entry{机械运动基础(高中)}{HSPM01}
\entry{相互作用(高中)}{HSPM02}
\entry{浮力、阿基米德原理(初中)}{Buoy}
\entry{牛顿运动定律(高中)}{HSPM03}
\entry{曲线运动(高中)}{HSPM04}
\entry{圆周运动(高中)}{HSPM05}
\entry{竖直面内的圆周运动(高中)}{CirVer}
% \entry{简谐振子(高中)}{SHOhs}
\entry{万有引力定律(高中)}{HSPM06}
\entry{功和机械能(高中)}{HSPM07}
\entry{动量(高中)}{HSPM08}
\entry{机械振动(高中)}{HSPM09}

\chapter{电磁学} \label{cpt_HSelect}
%---------------------------------------
\entry{静电场(高中)}{HSPE01}
\entry{静电场的应用(高中)}{HSPE02}
\entry{恒定电流(高中)}{HSPE03}
\entry{电路(高中)}{HSPE04}
\entry{电能(高中)}{HSPE05}
\entry{磁场(高中)}{HSPE06}
\entry{电荷在磁场中的圆周运动(高中)}{CBM}
\entry{楞次定律(综述)}{LCDL}

\chapter{热学}\label{cptHSphysThermal}
%---------------------------------------
\entry{分子动理论(高中)}{thermo}
\entry{热学初步(高中)}{therHS}
\entry{波粒二象性和量子力学初步(高中)}{atomHS}


\part{极简微积分} \label{prt_simcalc}
%======================================
% addis: 不要再合并了,极简微积分就是一个独立的部分,一切按照导航。 如果对应内容要写更严谨的版本,在别的部分新开文章
\chapter{极限} \label{cpt_simLim}
%---------------------------------------
\entry{【导航】极简微积分}{Calc}
\entry{数列的极限(极简微积分)}{Lim0}
\entry{函数的极限(极简微积分)}{FunLim}
\entry{无穷小、无穷大和阶数(极简微积分)}{InfOrd}
\entry{小角极限(极简微积分)}{LimArc}
\entry{自然对数底(极简微积分)}{E}
\entry{切线与割线}{TanL}
\entry{级数(极简微积分)}{SerCal}
\entry{幂级数(极简微积分)}{powerS}

\chapter{导数与微分} \label{cpt_simDer}
%---------------------------------------
% Giacomo:这部分的目录顺序很奇怪,待调整
\entry{导数(简明微积分)}{Der}
\entry{求导法则(简明微积分)}{DerRul}
\entry{复合函数求导、链式法则(极简微积分)}{ChainR}
\entry{反函数求导(极简微积分)}{InvDer}
\entry{一元函数的微分、微分近似(极简微积分)}{Diff}
\entry{高阶导数(极简微积分)}{HigDer}
\entry{基本初等函数的导数(简明微积分)}{FunDer}
\entry{导数与函数极值(简明微积分)}{DerMax}
\entry{泰勒展开(简明微积分)}{Taylor}

\chapter{定积分与不定积分} \label{cpt_simInt}
%---------------------------------------
\entry{定积分(极简微积分)}{DefInt}
\entry{不定积分(简明微积分)}{Int}
\entry{反常积分(简明微积分)}{impro}
\entry{牛顿—莱布尼兹公式(极简微积分)}{NLeib}


\part{一元微积分} \label{prt_calc1}
%======================================
% Giacomo:一元标量函数的微积分
\chapter{极限} \label{cpt_14}
%---------------------------------------
\entry{极限的运算法则}{LimOp}
\entry{求极限的一些方法}{ChaLim}
\entry{夹逼定理}{SquzTh}

\chapter{导数与微分} \label{cpt_15}
%---------------------------------------
\entry{用极值点大致确定函数图像}{DerImg}
\entry{平面曲线的曲率和曲率半径(极简微积分)}{curvat}
\entry{函数的凹凸性(极简微积分)}{CONCV}
% \entry{泰勒级数 2}{Taylr2}
\entry{有限差分}{Diffen}
\entry{导数与差分}{DerDif}
\entry{二项式定理(非整数幂)}{BiNorR}
\entry{曲线的切触}{CofCur}

\chapter{定积分与不定积分} \label{cpt_16}
%---------------------------------------
\entry{换元积分法}{IntCV}
\entry{分部积分法}{IntBP}
\entry{微积分基本定理}{FTcalc}
\entry{积分表}{ITable}
\entry{求定积分的一些方法}{ChaDin}
\entry{不定积分的常用技巧}{intech}
\entry{曲线的长度}{CurLen}

\chapter{常微分方程} \label{cpt_17}
%---------------------------------------
\entry{常微分方程}{ODE}
\entry{一阶线性微分方程}{ODE1}
\entry{一维齐次亥姆霍兹方程}{HmhzEq}
\entry{二阶常系数齐次微分方程}{Ode2}
\entry{二阶常系数非齐次微分方程}{Ode2N}
\entry{欧拉方程(微分方程)}{Eulequ}

\chapter{待归类} \label{cpt_18}
%---------------------------------------
\entry{极坐标中的曲线方程}{PolCrd}
\entry{向量值函数}{VECFUN}
\entry{傅里叶级数(三角)}{FSTri}
\entry{傅里叶级数(指数)}{FSExp}
\entry{三角傅里叶变换}{FTTri}
\entry{傅里叶变换}{FTExp}
\entry{幂指函数之差的极限}{MZHSZC}
\entry{控制理论(综述)}{KZLL}

\part{线性代数} \label{prt_LA1}
%======================================
\chapter{列向量、行向量和矩阵} \label{cpt_20}
%---------------------------------------
\entry{【导航】线性代数}{Vector}
% 列向量
\entry{列向量}{colVec}

% 矩阵
\entry{矩阵及其运算}{Mat}
% \entry{代数矢量}{NumVec}
\entry{逆矩阵}{InvMat} % 可以考虑并入 矩阵 中

\entry{相似变换和相似矩阵}{MatSim}
\entry{转移矩阵}{TransM}

\entry{矩阵的秩}{MatRnk}
% \entry{奇异值分解(SVD)}{SVD}
\entry{秩—零化度定理(矩阵)}{rnMat}
\entry{矩阵的迹}{trace}
\entry{行列式}{Deter}
\entry{行列式的性质}{DetPro}
\entry{行列式唯一性定理}{DetoD}
\entry{行列式与体积}{DetVol}

\entry{矩阵指数}{MatExp} % Giacomo: 标记更改/移动

% 张量
\entry{指标与求和约定}{EinSum}
\entry{张量(向量与矩阵)}{TsrFst}
% \chapter{旋转矩阵} \label{cpt_21a}
% %---------------------------------------
\entry{平面旋转矩阵}{Rot2D}
\entry{叉乘的矩阵形式}{CrosMt}
\entry{三维旋转矩阵}{Rot3D}
\entry{欧拉角}{EulerA}
\entry{四元数与旋转矩阵}{QuatN}
\entry{罗德里格旋转公式、定轴旋转矩阵}{RotA}
\entry{旋转矩阵的导数}{RotDer}

\chapter{线性方程组} \label{cpt_23}
%---------------------------------------
\entry{线性方程组与增广矩阵}{LinEq2}
\entry{高斯消元法解线性方程组}{GAUSS}
\entry{初等矩阵与初等变化}{EleOpt}
\entry{高斯消元法求逆矩阵}{InvMGs}
\entry{线性方程组的解}{LinEq}
\entry{克拉默法则}{kramer}

\chapter{矩阵的本征值} \label{cpt_22}
%---------------------------------------
% Giacomo:要不要移动到 高级线性代数?
\entry{分块矩阵}{BlkMat}
\entry{块对角矩阵}{BlDiag}

\entry{实正定矩阵}{DefMaR}
\entry{复正定矩阵}{DefMat}
\entry{二次多项式与二次型}{QuaPol}
\entry{二次型(线性代数)}{qua}
\entry{对称矩阵}{SymMat}
\entry{厄米矩阵、自伴矩阵}{HerMat}
\entry{正交矩阵、酉矩阵}{UniMat}

\entry{矩阵的本征问题}{MatEig}
% \entry{矩阵的本征值}{eigenM}
% \entry{本征值与本征向量}{eign}
\entry{对称矩阵的本征问题}{SymEig}
\entry{厄米矩阵的本征问题}{HerEig}
\entry{块对角厄米矩阵的本征问题}{BHeig}
\entry{对易厄米矩阵与共同本征矢}{Commut}

% 待处理
\entry{正交矩阵}{lnal05}

\chapter{向量空间和线性映射} \label{cpt_21}
%---------------------------------------
\entry{向量空间}{LSpace}
\entry{泛函与线性泛函}{Funal}
\entry{向量子空间}{SubSpc}
\entry{基底(线性代数)}{VecSpn}
\entry{凸集和凸体}{ConSet}
\entry{齐次凸泛函}{ConFul}
\entry{仿射集}{AffSet}
\entry{商空间(线性代数)}{QuoSpa}
\entry{余维数}{Codim}
\entry{线性泛函的几何意义}{LiFunG}
\entry{直和与补空间(线性代数)}{DirSum}
\entry{乘积空间}{lnal06} % 待处理

\entry{线性映射}{LinMap}
\entry{秩—零化度定理}{MatLS2}
\entry{线性空间的同态与同构}{lnal02}


% 线性映射的矩阵表示
% Giacomo:这个分区不知道怎么改
\entry{矢量空间的表示}{VecRep}
\entry{矩阵与映射}{lnal04}
\entry{线性映射与矩阵的代数关系}{linmat}
\entry{矩阵与线性映射}{MatLS}

% 线性方程组的解的几何/线代解释
\entry{线性方程组的仿射解释}{AS2LF}

\chapter{未归类} \label{cpt_24}
%---------------------------------------
% \entry{矩阵的零空间}{Mat0s}

% \chapter{多项式} \label{cpt_51}
% %---------------------------------------
% Giacomo:需要移动到更合适的地方
% TODO: 数域
\entry{一元多项式}{OnePol}
\entry{韦达定理(高等代数)}{VietaF}
\entry{多项式的结式与判别式}{RDPly}
\entry{带余除法}{DivAlg}
\entry{多项式的整除}{ExDiv}
\entry{辗转相除法}{SucDiv}
\entry{极分解}{PoDe}
\entry{线性变换的极小多项式}{MinPol}
\entry{多项式的可约性质}{RedPol}
\entry{因式分解唯一性定理}{UniFac}
\entry{本原多项式(高等代数)}{PPlyR}
\entry{一般线性变换的 Jordan(若尔当)标准形}{ltrJor }
\entry{幂零线性变换的 Jordan(若尔当)标准形}{Jordan }
\entry{零化多项式}{nullpl  }
\entry{代数学基本定理}{BscAlg}
\entry{反射变换(高等代数)}{ReTr}

% \chapter{其他} \label{cpt_24}
% %---------------------------------------
\entry{范德蒙矩阵、范德蒙行列式}{VandDe}
\entry{Rayleigh-Ritz 方法}{RayRit}

\entry{线性最小二乘法}{LstSqr}
\entry{超定线性方程组的最小二乘法解}{OvrDet}

\part{多元微积分} \label{prt_calc2}
%======================================

\chapter{多元微积分} \label{cpt_25}
%---------------------------------------
% Giacomo: 我仔细想了想没必要细分成那么多情况
\entry{向量值函数的导数}{DerV}
\entry{偏导数(简明微积分)}{ParDer}
% 应融合成一篇
\entry{偏导数(多元矢量值函数)}{VecPdv}
\entry{海森矩阵}{Hesian}
\entry{全微分(简明微积分)}{TDiff}
\entry{复合函数的偏导、链式法则(多元微积分)}{PChain}
\entry{全导数(简明微积分)}{TotDer}
\entry{方向导数}{DerDir}
\entry{梯度、梯度定理}{Grad}
\entry{雅可比矩阵、雅可比行列式}{JcbDet}
\entry{光滑映射(简明微积分)}{SmthM}
\entry{多元泰勒展开}{NDtalr}
\entry{二元函数的极值(简明微积分)}{F2Exm}
\entry{多元函数的极值}{MulPlo}
\entry{偏导与差分}{ParDf}
\entry{单变量矢量值函数的积分}{IntV}
\entry{重积分、面积分、体积分(简明微积分)}{IntN}
\entry{重积分和宇称}{IntPry}
\entry{重积分的换序(简明微积分)}{Fubin0}
\entry{微分形式(简明微积分)}{DForm}
\entry{广义斯托克斯定理(简明微积分)}{Stoke2}
\entry{高阶微分(多元微积分)}{MHDiff}

\chapter{应用} \label{cpt_26}
%---------------------------------------
\entry{极坐标系中单位矢量的偏导}{DPol1}
\entry{立体角}{SolAng}
\entry{高斯积分}{GsInt}
\entry{偏微分算符}{ParOp}
\entry{拉格朗日乘数法、条件极值}{LagMul}
\entry{多维球体的体积}{NSphV}
\entry{用梯度求曲线和曲面的法向量}{GradNr}
\entry{多元函数的傅里叶级数}{NdFuri}
\entry{齐次函数的欧拉定理}{Homeul}
\entry{一阶线性常微分方程组(简明微积分)}{ODEsys}
\entry{高阶线性微分方程的降阶}{ODEb4}

\chapter{矢量分析} \label{cpt_27}
% Giacomo:我不懂矢量分析,就不做具体分类了
% Giacomo:应该专注于三维空间的特例,一般情况应当移动到上面的对应情况
%---------------------------------------
\entry{矢量场(矢量分析)}{Vfield}
\entry{曲面积分、通量}{SurInt}
\entry{线积分、环积分(矢量分析)}{IntL}
\entry{证明闭合曲面的法向量面积分为零}{CSI0}
\entry{矢量算符}{VecOp}
\entry{拉普拉斯算符}{Laplac}
\entry{一种矢量算符的运算方法}{MyNab}
\entry{矢量算符运算法则}{VopEq}
\entry{分部积分的高维拓展}{IntBP2}
\entry{散度、高斯散度定理}{Divgnc}
\entry{散度的逆运算}{DivInv}
\entry{牛顿—莱布尼兹公式(矢量分析)}{NLext}
\entry{旋度(简明微积分)}{Curl}
\entry{旋度的逆运算}{HlmPr2}
\entry{正交曲线坐标系}{CurCor}
\entry{正交曲线坐标系中的重积分}{CrIntN}
\entry{正交曲线坐标系中的矢量算符}{CVecOp}
\entry{斯托克斯定理(矢量分析)}{Stokes}
\entry{调和场(无散无旋场)}{HarmF}
\entry{亥姆霍兹分解}{HelmTh}

\entry{矢量分析总结}{VecAnl}

\part{数学基础} \label{prt_math0}
%======================================


\chapter{集合论} \label{cpt_28}
%---------------------------------------
\entry{【导航】数学基础}{MFnav}
\entry{公理系统}{axioms}
\entry{集合}{Set}
\entry{集合的运算}{OpSet}
% \entry{存在性和任意性}{EA} %待删除
\entry{映射}{map}
\entry{Cantor-Bernstein 定理}{CBTheo}
\entry{集合的基数}{CardiN}
\entry{无穷的概念}{infty}
\entry{公理化集合论}{SetAxi}
\entry{笛卡尔积}{CartPr}

% \chapter{数} \label{cpt_28a}
% %---------------------------------------
\entry{整数}{intger}
\entry{整数集的倍数与整除}{yulan}
% Giacomo:TODO:自然数、有理数、代数数、实数、复数。

% \chapter{序论} \label{cpt_29}
% %---------------------------------------

\entry{二元关系}{Relat}
\entry{序关系}{OrdRel}
\entry{序数、超限数}{OrdNum}
\entry{大型运算符}{LgOper}

\chapter{数理逻辑} \label{cpt_8b}
%---------------------------------------
% \chapter{数理逻辑:命题逻辑} \label{cpt_8bc}
% %---------------------------------------
\entry{命题}{prop}
\entry{命题的连接词}{propco}
\entry{公式(数理逻辑)}{wlform}
\entry{公式的基本等价关系(数理逻辑)}{wffEqR}

% \chapter{数理逻辑:谓词逻辑(一阶逻辑)} \label{cpt_8c}
%---------------------------------------
\entry{个体词、谓词与量词(数理逻辑)}{prelo1}
\entry{量词公式的基本等价关系}{prelo2}
\entry{量词相关的推理规则}{prelo3}

\chapter{其他} \label{cpt_30}
%---------------------------------------
% \chapter{范畴论} \label{cpt_32}
% %---------------------------------------
\entry{范畴论}{Cat}
\entry{范畴}{Catgry}


\part{数学分析} \label{prt_anal}
%======================================
% Giacomo:数学分析板块需要重新梳理
% 1. 数学分析需要引入度量空间,那么要不要引入拓扑空间?
% 2. 实数理论依托于数列,甚至是度量空间的性质,需要穿插进整个part,而不是单独列一个chapter

\chapter{序列与级数} \label{cpt_33}
%---------------------------------------
\entry{实数}{ReNum}
\entry{实数的表示}{ReRep}
\entry{完备公理(戴德金分割)}{Cmplt}
\entry{上确界与下确界}{SupInf}
\entry{实数集的完备公理}{RCompl}
% {代数数} {超越数} {从集合论看实数} {实数空间和欧几里得空间}

\entry{稠密性与完备性}{OrdCom}

\entry{序列}{seq}
\entry{序列的极限(数学分析)}{SeqLim}
\entry{极限存在的判据、柯西序列}{CauSeq}
\entry{子列极限、上极限与下极限}{SubLim}
\entry{极限的一般观点 重极限与累次极限}{MulLim}
\entry{级数(数学分析)}{Series}
\entry{正项级数的收敛性判别}{PosCov}
\entry{绝对收敛与条件收敛}{Convg}
\entry{黎曼重排定理}{RieRes}
\entry{交错级数的收敛性判别}{AltCov}

\entry{自然常数(数学分析)}{exp}
\entry{Euler-Mascheroni 常数}{Masche}
\entry{调和数(基础)}{HarNum}
\entry{反函数定理}{inverf}

\chapter{连续函数} \label{cpt_35}
%---------------------------------------
\entry{极限}{Lim}
\entry{函数的极限}{limfx}
\entry{函数极限的性质}{limff}
\entry{函数的连续性}{contin}
\entry{函数的连续与间断}{confun}
\entry{连续函数的性质}{conff}
\entry{幂函数(数学分析)}{RePw}
% 未完成: 连通与道路连通  连续映射的性质 连续延拓 一致连续 单调性

\chapter{拓扑空间与度量空间} \label{cpt_36}
%---------------------------------------
% {开集与闭集} {实数空间中的紧集}
\entry{实数中的开集和闭集}{realat}
% 重复了
\entry{实数集的拓扑}{ReTop}
\entry{有限覆盖与紧性}{CptRe}

\entry{度量空间}{Metric}
\entry{度量空间中的概念}{Metri2}
\entry{度量空间的连续映射和等距}{ConIso}
\entry{柯西序列、完备度量空间}{cauchy}
\entry{柯西序列的等价}{CauEq}
\entry{空间的完备化}{SpaCom}
\entry{球套定理}{SNesT}
\entry{度量空间的稠密性}{MaDen}
\entry{Baire 定理}{Baire}
\entry{完备空间}{ComSpa} % 重复了!
\entry{巴拿赫不动点定理}{ConMap}

\entry{隐函数定理的不动点证明}{IFTFix}
% 未完成: 贝尔纲定理 $p$-进数 有理数集的赋值


\chapter{导数与微分} \label{cpt_37}
%---------------------------------------
\entry{导数(数学分析)}{Der2}
\entry{凸函数}{ConvFu}
\entry{导数的运算法则}{Der3}
\entry{莱布尼兹公式}{LeiEqu}
% 未完成: 求导法则 映射的微分 临界点与极值 隐函数定理 微分同胚 条件极值 拐点
\entry{微分中值定理}{MeanTh}
\entry{洛必达法则}{LHopiR}
\entry{泰勒公式}{Tayl}
\entry{施勒米希-洛希余项公式}{SchRo}
\entry{幂级数与解析函数}{anal}
\entry{柯西—阿达玛公式}{CHF}
\entry{一致收敛}{UniCnv}
\entry{一致收敛与极限换序}{UniCo2}
\entry{隐函数}{ImpFun}
\entry{一元隐函数的存在及可微定理}{ImFED}

% \chapter{多元函数} \label{cpt_38}
% %---------------------------------------
\entry{偏导数(数学分析)}{ParDif}
\entry{向量函数的微分}{vecdif}
\entry{多元隐函数的存在定理}{Mulmp}
\entry{多元数量函数的隐函数定理}{impli}
\entry{莫尔斯引理}{Morse}

\chapter{黎曼积分} \label{cpt_39}
---------------------------------------
\entry{定积分}{DInt}
\entry{定积分存在条件}{Rieman}
\entry{可积函数}{InFun}
\entry{定积分的性质}{DIntP}
\entry{积分中值定理}{InMT}
% 多元黎曼积分
\entry{狄拉克 delta 函数}{Delta}
\entry{多元狄拉克 delta 函数}{deltaN}
\entry{狄拉克 delta 导函数}{delta2}
\entry{零函数(列)}{F0}
% 未完成: 一致收敛与黎曼积分 牛顿-莱布尼兹公式(连续可微函数) 泰勒余项 反常积分 绝对收敛与条件收敛(反常积分)

% \chapter{傅立叶分析} \label{cpt_40}
% ---------------------------------------
\entry{函数的算符}{DifOp}
\entry{正交函数系}{Fbasis}
\entry{正交函数系 2}{OFS}
\entry{连续正交归一基底与傅里叶变换}{COrNoB}
\entry{傅里叶变换与矢量空间}{FTvec}
\entry{多元傅里叶变换}{NFTran}

% \part{复分析} \label{prt_analC}
% %======================================
\chapter{一元复分析} \label{cpt_41}
%---------------------------------------
\entry{复变函数}{Cplx}
% 未完成:全纯函数与解析函数 复变函数求积分 柯西-古尔萨定理 柯西积分公式
\entry{幂函数(复数)}{CPow}
\entry{指数函数(复数)}{CExp}
\entry{对数函数(复数)}{clog}
\entry{三角函数(复数)}{CTrig}
\entry{复变函数的导数、柯西—黎曼条件}{CauRie}
\entry{解析函数与散度旋度}{HolHar}
\entry{复变函数的积分}{CpxInt}
\entry{牛顿—莱布尼兹公式(复变函数)}{AnaInt}
\entry{柯西积分定理}{CauGou}
\entry{洛朗级数}{LaurSr}
\entry{留数定理}{ResThe}
\entry{Jordan 引理}{JdLem}

\chapter{待整理} \label{cpt_42}
%---------------------------------------
\entry{凸函数(补)}{Convex}
\entry{有界变差}{BouVar}
\entry{黎曼—斯蒂尔切斯积分}{RSInte}
\entry{Gamma 函数}{Gamma}
\entry{Gamma 函数 2}{Gamma2}
\entry{余元公式}{Gama1}
\entry{不完全 Gamma 函数}{IncGam}
\entry{渐近展开}{Asympt}
\entry{拉普拉斯方法}{LapAsm}
% 未完成: 无穷级数
\entry{魏尔施特拉斯逼近定理}{Weiers}

% Giacomo: 或许该移动到常微分方程?
\entry{包络线}{Velope}
\entry{包络和奇解}{EnvSol}

\entry{亥姆霍兹分解 2}{HelDe}

\entry{Euler-Maclaurin 求和公式}{EMSum}

\chapter{笔记与总结} \label{cpt_44}
---------------------------------------
\entry{Rudin 数学分析笔记 1}{AnalNt}
\entry{Rudin 数学分析笔记 2}{AnalN2}
\entry{Rudin 数学分析笔记 3}{AnalN3}
\entry{Rudin 实分析与复分析笔记 1}{AnalN4}
\entry{Rudin 实分析与复分析笔记 2}{AnalN5}


\part{数论} \label{prt_NumberTh}
%=====================================
\chapter{素数与整除} \label{cpt_7d}
%-------------------------------------
\entry{整除}{divisb}
\entry{素数与合数}{prmnt}
\entry{标准型与唯一分解定理}{stafnt}
\entry{渐进估计与阶}{asymeq}
\entry{素数定理}{prithy}
\entry{最大公约数与最小公倍数}{gcdlcm}
\entry{整数模与裴蜀定理}{mobezt}

\chapter{剩余与同余} \label{cpt_7e}
%-------------------------------------
\entry{同余与剩余类}{modres}
\entry{线性同余}{linmod}
\entry{欧拉函数(数论)}{EulFun}
\entry{数论求和记号}{ntsNot}
\entry{数论三角和与高斯和}{ntrtre}
\entry{单位根与本原单位根(数论)}{priru}
\entry{拉马努金和(数论)}{RamSum}
\entry{中国剩余定理}{crtnt}

\chapter{费马小定理与二次剩余} \label{cpt_7f}
%-------------------------------------
\entry{费马小定理与欧拉定理}{fermse}
\entry{二次剩余}{quares}
\entry{Legendre 符号(数论)}{legnnt}
\entry{Wilson 定理}{wilsth}
\entry{二次互反律}{gaqrnt}
\entry{阶与原根}{ordprr}

\chapter{素数相关的解析数论} \label{cpt_8a}
%-------------------------------------
\entry{数论函数 theta 与 psi 的阶}{tpont}
\entry{Bertrand 定理}{bertTh}
\entry{素数定理的证明}{prthpr}

\chapter{其他} \label{cpt_45}
%-------------------------------------
\entry{连分数}{ConFra}
\entry{无理数(数论)}{irrNum}

\entry{数论函数}{NumFun}
\entry{Möbius 函数(数论)}{MbusF}
\entry{积性函数}{MulFun}

\entry{狄利克雷卷积(数论)}{DirCon}


\part{组合数学与离散数学} \label{prt_Combin}
%======================================
% \entry{【导航】组合数学与离散数学}{Comnav}


\chapter{计数组合学} \label{cpt_46}
%---------------------------------------
\entry{计数原理}{ProfPM}
\entry{布置、排列、组合}{APC}
\entry{范德蒙恒等式}{ChExpn}
\entry{逆序数}{InvNum}
\entry{列维—奇维塔符号}{LeviCi}
\entry{Sperner 定理}{SpeThm}
\entry{多项式定理}{PolyNm}


\chapter{图论} \label{cpt_47}
%---------------------------------------
\entry{图}{Graph}

\chapter{其他} \label{cpt_48}
%---------------------------------------
\entry{克罗内克 delta 函数}{Kronec}

\part{概率与统计} \label{prt_Statis}
%======================================
\chapter{基础} \label{cpt_49}
%--------------------------------
\entry{随机变量、概率密度函数}{RandF}
\entry{标准差与方差}{StDevi}
\entry{平均值的不确定度}{MeanS}
\entry{随机变量的变换}{RandCV}
\entry{多变量分布函数}{MulPdf}
\entry{中心极限定理}{CLT}
\entry{生日问题}{Birth}
\entry{三门问题}{MontyH}
\entry{从集合论角度看随机事件}{JNran1}
\entry{贝叶斯公式}{JNran2}
\entry{样本均值与方差}{SaMeVa}
\entry{容斥原理}{inex}

\chapter{概率分布} \label{cpt_50}
%------------------------------------
\entry{高斯分布(正态分布)}{GausPD}
\entry{二项分布}{BiDist}
\entry{泊松分布}{PoisD}
\entry{对数正态分布}{LogNor}
\entry{Weibull 分布}{Weibul}
\entry{抛硬币实验进阶}{CoinEx}
\entry{高尔顿板}{Galton}
\entry{二维随机游走}{RW2D}
\entry{卡方分布}{Chi2}
\entry{贝叶斯概率(综述)}{BYSGL}
\entry{马尔可夫链蒙特卡洛}{MCMC}

% Jier:是否应该整合为抽象代数,放在线性代数后面?另,高等代数从学科分类来说就是线性代数,不应出现在此处。
\part{代数基础} \label{prt_AlgB}
%======================================

\chapter{群} \label{cpt_52}
%---------------------------------------
% Giacomo:缺少交换群(阿贝尔群)的文章
% 应当包括 有限生成阿贝尔群基本定理
\entry{群}{Group}
\entry{群乘法表及重排定理}{groupt}
\entry{子群}{Group1}
\entry{陪集和同余}{coset}
\entry{正规子群和商群}{NormSG}
\entry{共轭与共轭类}{gpcon}
\entry{直积和半直积(群)}{GrpPrd}
\entry{循环群}{cyclic}
\entry{单群}{SemGrp}
\entry{群的同态与同构}{Group2}
\entry{群的扩张}{GrpExt}

\entry{置换群、对称群}{Perm}
\entry{置换的奇偶性}{permu}
\entry{有限对称群的性质}{AutSym}
\entry{有限阿贝尔群基本定理}{Albert}
\entry{群作用}{Group3}
\entry{换位子群}{CmtGrp}
\entry{可解群}{SlvbGp}
\entry{幂零群}{NipGrp}

\entry{一般线性群}{GL}

\entry{自由群}{FreGrp}
\entry{群的自由积}{FrePrd}
\entry{群论中的证明和习题解答}{GroupP}

\chapter{环与域} \label{cpt_53}
%---------------------------------------
\entry{环}{Ring}
\entry{剩余类环}{RRing}

\entry{环的理想和商环}{Ideal}
\entry{环同态}{RingHm}
\entry{半同态}{hlfhom}
\entry{整环}{Domain}
\entry{素理想与极大理想}{Ideals}
\entry{爱森斯坦判别式}{EsstCr}
\entry{真因子树}{FctTre}
\entry{唯一析因环}{UFD}
\entry{主理想整环}{PID}
\entry{欧几里得环}{EuRing}
\entry{多项式环}{RPlynm}

% 域
\entry{环和域}{field}
\entry{素域}{FmLT}
\entry{分式域}{FrcFld}
\entry{四元数}{Quat}
\entry{域的扩张}{FldExp}
\entry{分裂域}{SpltFd}
\entry{有限域}{FntFld}

\chapter{模论} \label{cpt_1a}
%---------------------------------------
\entry{模}{Module}
\entry{模的直和}{DSofMo}


\part{高等线性代数} \label{prt_LA2}
%======================================
\chapter{线性映射与算子} \label{cpt_54}
%---------------------------------------
% \entry{【导航】高级线性代数}{mapALA}
\entry{多线性映射}{MulMap}
\entry{对称/反对称多线性映射}{SASmap}
\entry{斜对称映射}{AntMap} % 重复了
\entry{对偶空间}{DualSp}
\entry{线性无关判别法}{LinInD}
\entry{伴随映射}{AdjMap}
\entry{等距变换}{IsoMe}
\entry{正交变换与对称变换}{ortho}
\entry{向量空间的张量积}{vecTsr}
\entry{向量空间的对称/反对称幂}{vecSAS}
\entry{有限维度下张量积的存在性}{TPofSp}

\entry{线性映射的张量积}{linSW}

\entry{域上的代数}{AlgFie}
\entry{张量代数}{Tsralg}
% TODO:对称代数与多项式环
\entry{外代数}{ExtAlg}

% \chapter{线性算子} \label{cpt_58}
% %---------------------------------------
\entry{线性算子}{LiOper}
\entry{不变子空间}{InvSP}

\entry{算子代数}{SiOAlg}
\entry{本征矢量与本征多项式}{EigVM}
\entry{线性算子对角化的充要条件}{LODia}

\entry{线性算子的张量积}{TPofLO}
\entry{线性算子的行列式}{APdet}
\entry{正规算子}{NmOpt}

\chapter{赋范、内积空间和二次型} \label{cpt_57}
%---------------------------------------
% Giacomo:从现在开始这个part是“纯代数的”,换言之我们不考虑任何“分析”的性质,相关内容全部移动到泛函分析。

\entry{内积、内积空间}{InerPd}
\entry{矢量的模和度量的关系}{SNadM}
\entry{正交归一基底}{OrNrB} % 具体位置有待调整。
\entry{合同变换}{lialg}
\entry{正交分解、投影算符}{projOp}
\entry{柯西—施瓦茨不等式}{CSNeq}
\entry{Cauchy-Schwarz 不等式}{WuChy}
\entry{欧几里得矢量空间}{EuVS}
\entry{欧几里得矢量空间的正交化、同构及正交群}{EVOIOG}
\entry{正交空间与辛空间}{OrSp}
\entry{子空间的正交关系}{OrthSp}

% 复内积空间
% Giacomo:需要统一一下名词
\entry{半双线性形式}{sequil}
\entry{埃尔米特型}{HeFor}
\entry{幺正变换}{Unitar}
\entry{埃尔米特矢量空间(酉空间)}{HVorUV}
\entry{酉群}{UQ}

% 赋范/内积空间中的线性算子
\entry{线性算子的范数}{ONorm}
\entry{向量空间上的范数}{norm}
\entry{对易算符}{Commu}
\entry{厄米共轭算符的映射结构}{RCrank}

% TODO: 谱定理
% TODO: 约旦标准形式

% \chapter{二次型} \label{cpt_59}
% %---------------------------------------
\entry{双线性函数}{BiFunc}
\entry{双线性型}{bilinF}
\entry{二次型}{QuaFor}
\entry{二次型的规范型}{GuaOQu}
\entry{实二次型}{RQuaF}
\entry{正定二次型}{DeQua}
\entry{指数有限度量空间}{EFSp}
\entry{斜对称双线性型的规范型}{OBili}

% 待整理
% \entry{常秩定理}{rankth}

\chapter{张量} \label{cpt_60}
%---------------------------------------
% Giacomo: 张量积相关内容写到《线性空间》章节中,
% 本章节只考虑$(r, s)$型张量这种狭义的张量
% TODO: 大修
\entry{张量}{Tensor}
\entry{协变和逆变}{CoCon}
\entry{张量的分类}{CatTns}
\entry{张量的张量积}{TsrPrd}

\entry{彭罗斯图形表示法}{PnrsGN}
\entry{张量的坐标}{CofTen}
\entry{张量的坐标变换}{TrTnsr}
\entry{结构张量代数}{STAlg}

% Giacomo: 待整改、重命名
\entry{张量积空间}{DirPro}
\entry{张量积空间的算符}{ProdOp}
\entry{张量代数(张量)}{TenAlg}
\entry{张量的对称化和交错化}{SIofTe}
\entry{张量的外积}{WegofT}


\entry{度规张量与指标升降(欧氏空间)}{TofEuc}

% Giacomo:仿射不应该放这里
\chapter{仿射空间} \label{cpt_61}
%---------------------------------------
\entry{仿射空间}{AfSp}
\entry{仿射子空间}{SAfSp}
\entry{仿射群}{AfQ}
\entry{欧几里得空间}{EucSp}
\entry{保距群(欧氏空间)}{CDQ2Ec}
\entry{仿射空间中的曲线坐标系}{CFinAf}
\entry{曲线坐标系下的张量坐标变换(仿射空间)}{TinCur}
\entry{平行移动}{PTinAS}

\chapter{李代数} \label{cpt_62}
%-------------------------------------
\entry{李代数}{LieAlg}
\entry{李代数的子代数、理想与商代数}{LieSub}
\entry{李代数的同态与同构}{LieMor}
\entry{泛包络代数}{UEnAl}

\chapter{其他} \label{cpt_63}
---------------------------------------
\entry{多项式矩阵}{matplo}
\entry{超线性空间}{SVecSp}


\part{拓扑学} \label{prt_Topo}
%=====================================
\chapter{点集拓扑} \label{cpt_64}
%-------------------------------------
\entry{拓扑空间}{Topol}
\entry{点集的内部、外部和边界}{Topo0}
\entry{拓扑空间的收敛序列}{ConvTp}
\entry{连续映射和同胚}{Topo1}
\entry{紧致性}{Topo2}
\entry{连通性}{Topo3}
\entry{道路连通性}{Topo4}
\entry{分离性}{Topo5}
\entry{分离性公理成立的充要条件}{DisAx}
\entry{积拓扑}{Topo6}
\entry{商拓扑}{Topo7}
\entry{拓扑空间之间的运算}{TopSpO}
\entry{映射空间}{Topo8}
\entry{空间偶和带基点空间}{Topo9}
\entry{拓扑群}{TopGrp}
\entry{Tychonoff 定理}{TychT}

\entry{覆叠空间}{CovTop}

\chapter{代数拓扑} \label{cpt_65}
%-------------------------------------
% \chapter{复形} \label{cpt_1b}
% %-------------------------------------
\entry{单纯形与单纯复形}{SimCom}
\entry{可三角剖分空间}{Traglt}

% \chapter{同伦论} \label{cpt_1c}
% %-------------------------------------
\entry{映射的同伦和空间的同伦}{HomT1}
\entry{单连通性}{SmpCn}
\entry{可缩空间}{HomT2}
\entry{基本群}{HomT3}
\entry{基本群的计算}{HomT5}
\entry{高阶同伦群}{HomT4}
\entry{球面的同伦群}{SphHmt}

% \chapter{同调论} \label{cpt_67}
% %-------------------------------------
\entry{复形的单纯同调群}{SimHml}
\entry{单纯同调群的计算}{SHCal}

% by Jier:微分几何应该适合放在拓扑学后头,有点集拓扑和群论知识就可以讲得足够深入了;李代数的引入可能得两条路同时走,“直接从代数引入”和“从流形到李群到李代数的步步抽象”两种方式.
% “函数芽”的概念该如何插入?
%======================================
\part{微分几何} \label{prt_DifGeo}
%======================================
\chapter{古典微分几何} \label{cpt_68}
%-------------------------------------
\entry{【导航】微分几何}{DGnav}
% 部分0: 分析基础
\entry{欧几里得空间中的曲线}{eucur}
% 部分1: 平面曲线
\entry{平面曲线的曲率(古典微分几何)}{cur}
% 部分2: 空间曲线
\entry{三维空间中的曲线}{Curv3D}
% 部分3: 空间曲面
\entry{三维空间中的曲面}{RSurf}
\entry{曲面的切空间(古典微分几何)}{tgSpaE}
\entry{基本型}{FForm}
\entry{可定向曲面}{OriSur}
\entry{超平面的定向}{orihyp}
\entry{高斯映射}{GMap}
\entry{高斯曲率和平均曲率}{GHcurv}
\entry{等距映射与保形映射}{Isomet}
\entry{高斯绝妙定理}{GauTE}
\entry{直纹面(古典微分几何)}{RuSurf}

\chapter{流形} \label{cpt_69}
%-------------------------------------
\entry{流形}{Manif}
\entry{子流形}{SubMnf}
\entry{积流形}{ManPro}
\entry{单位分割}{ParUni}
\entry{切空间(流形)}{tgSpa}
\entry{光滑映射(流形)}{DiffTg}
\entry{抽象指标}{AbsInd}
\entry{切向量场}{Vec}
% \entry{前推}{pfw}
\entry{纤维丛}{Fibre}
\entry{向量丛和切丛}{TanBun}
% 未完成: 余切丛 向量场的流 闭形式与恰当形式
\entry{拉回映射}{PullBk}
\entry{流形上的张量场}{TenMan}
% \entry{向量丛}{VecBun}
\entry{微分形式}{Forms}
\entry{体积形式}{VolFom}
\entry{霍奇星算子}{HodgeO}
\entry{外导数}{ExtDer}
\entry{费罗贝尼乌斯定理}{FrobTh}
% \entry{定向}{Orient}

% \chapter{辛几何} \label{cpt_70}
% %-------------------------------------
\entry{辛流形}{SplcMn}

% \chapter{复几何} \label{cpt_71}
% %-------------------------------------
\entry{复流形}{CMani}

\chapter{黎曼几何} \label{cpt_72}
%-------------------------------------
\entry{黎曼度量与伪黎曼度量}{RiMetr}
% \entry{纳什嵌入定理}{NashEm}
\entry{仿射联络}{affcon}
\entry{协变导数}{CoDer}
\entry{黎曼联络}{RieCon}
\entry{Christoffel 符号}{CrstfS}
\entry{曲率张量场}{RicciC}
% 截面曲率
\entry{高斯恒等式(黎曼几何)}{Egreg}
\entry{联络形式与结构定理}{ConFom}
\entry{测地线}{geodes}
\entry{庞加莱半平面(微分几何计算实例)}{PoiHP}


\chapter{向量丛上的联络} \label{cpt_73}
%-------------------------------------
\entry{联络(向量丛)}{VecCon}
\entry{曲率(向量丛)}{VecCur}
\entry{平行性(向量丛)}{VecPar}
\entry{和乐群(向量丛)}{VecHol}


\chapter{李群和李代数} \label{cpt_74}
%-------------------------------------
\entry{矩阵李群}{MatLG}
\entry{李群}{LieGrp}
\entry{李导数}{lie}
\entry{李群的李代数}{LieGA}
\entry{流形上的代数结构}{MnfAlg}


\part{代数进阶} \label{prt_AlgA}
%======================================
\chapter{有限群论} \label{cpt_75}
---------------------------------------

\entry{Sylow 定理}{Sylow}

\chapter{Galois 理论} \label{cpt_76}
---------------------------------------
\entry{完全域}{CmplD}
\entry{可分扩张}{SprbEx}
\entry{本原元定理}{PrmtEl}
\entry{可分元素的单扩张是可分扩张}{SprbE2}
\entry{纯不可分扩张}{PInsEx}
\entry{正规扩张}{NomEx}
\entry{Galois 扩张}{GExt}
% 应用
% \entry{三次与四次多项式的根}{PlyRtS} % pdf 编译不通过
\entry{分圆多项式和分圆域}{Cycltm}
% Giacomo:这个应该放在“广义伽罗瓦理论”里面
\entry{无穷 Galois 扩张与 Krull 定理}{GExInf}



\chapter{Clifford代数} \label{cpt_7b}
---------------------------------------
\entry{Clifford 代数}{clf01}
\entry{Clifford 代数的基本运算}{clf02}
\entry{伪标量}{Clf03}
\entry{单向量}{Clf04}
\entry{单向量的运算}{Clf05}
\entry{外同态}{Clf06}

% \chapter{交换代数} \label{cpt_77}
%--------------------------------------
% Giacomo:待添加: radical ideal 局部化 C[[x]] 诺特环 (discreet)valuation ring

\chapter{有限群表示论} \label{cpt_78}
---------------------------------------
\entry{群表示}{GrpRep}

% Giacomo: 等有缘人重启这个话题吧
% \part{表示论} \label{prt_Repr}
% %======================================
% \chapter{李代数表示论} \label{cpt_79}
% ---------------------------------------
% 待添加:复化(李代数表示论) 单李代数 复半单李代数的等价定义 Cartan子代数 根系统(root sysmtem)Dynkin图
% Giacomo:我个人认为写Dynkin图之前应该先写Coxeter图,应该是几何群论的内容。

\part{常微分方程} \label{prt_Ode}
%====================================
\chapter{一阶常微分方程} \label{cpt_80}
%---------------------------------------
\entry{常微分方程简介}{ODEint}
\entry{基本知识(常微分方程)}{ODEPr}
\entry{自治系统解的特点}{AuSy}
\entry{平衡位置和圈}{EqliP}

\entry{微分方程 $y^{(N)}=f(x)$}{ynfx}
\entry{记号方法}{Sign}

\entry{化一般常微分方程组为标准方程组(常微分方程)}{GO2SOD}
\entry{一般积分}{IntGen}
\entry{一阶常微分方程解法:变量可分离方程}{ODEa1}
\entry{一阶常微分方程解法:常数变易法}{ODEa2}
\entry{一阶常微分方程解法:恰当方程}{ODEa3}
\entry{一阶隐式常微分方程}{ODEa4}
\entry{一阶隐式常微分方程的存在唯一性定理}{ODEa5}
\entry{判别曲线法求一阶隐式常微分方程的奇解}{ODEa6}
\entry{阿贝尔微分方程恒等式}{AbelID}
\entry{常微分方程的几何图像}{GofODE}
\entry{可微映射的导数}{DoDifM}

\chapter{高阶常微分方程和线性微分方程组} \label{cpt_81}
%---------------------------------------
\entry{线性微分方程的一般理论}{ODEb1}
\entry{常系数线性齐次微分方程}{ODEb2}
\entry{一阶常系数线性微分方程组(常微分方程)}{ODEb3}
\entry{二阶齐次变系数线性微分方程的幂级数解法}{ODE2P}
\entry{拉普拉斯变换}{LapTra}
\entry{拉普拉斯变换的性质}{ProLap}
\entry{拉普拉斯变换与常系数线性微分方程}{ODELap}

\chapter{一般理论} \label{cpt_82}
%---------------------------------------
\entry{压缩映射}{ComMap}
\entry{李普希茨条件}{LipCon}
\entry{皮卡映射}{PicMap}
\entry{常微分方程解的存在、唯一及连续可微定理}{ODEUC}
\entry{皮卡-林德勒夫定理}{PiLin}
\entry{解对参数的连续依赖}{ConDep}
\entry{极大解}{MaxSol}
% 未完成: 皮亚诺存在定理  柯西-科瓦列夫斯卡娅定理
\entry{比较定理}{CompT}
\entry{施图姆—刘维尔理论}{SLthrm}
\entry{正则 S-L 边值问题本征函数零点数量的证明}{SLzPro}
\entry{Liapunov 稳定性(常微分方程)}{ODELia}
\entry{Liapunov 函数(稳定性直接法)}{ODELi2}

% \chapter{常系数方程(组)} \label{cpt_83}
%---------------------------------------
% 未完成: 二维常系数线性常微分方程组的分类  基本解矩阵  特征方程与基本解系 共振与增益 渐近稳定性

% \chapter{微分方程的动力学} \label{cpt_84}
%---------------------------------------
% 未完成: 双曲不动点 哈特曼线性化定理 李雅普诺夫函数 极限圈 庞加莱-本迪克森定理 范德波尔振荡器 洛伦兹系统

\part{实变函数} \label{prt_RealAn}
%======================================
\chapter{测度论} \label{cpt_85}
%---------------------------------------
\entry{集合的极限}{SetLim} % 位置难决定
\entry{集合的测度(实变函数)}{SetMet}
\entry{集合环上的测度}{MsExte}
\entry{可测集合}{MsbSet}
\entry{可测函数}{MsbFun}
\entry{Egoroff 定理}{EgrfTh}
\entry{可测函数的 Lusin 定理}{MsbFSt}
\entry{依测度收敛}{LimMs}
% 测度的概念; 测度空间}{}, 开集与闭集的勒贝格测度, 博雷尔集, 博雷尔测度, 零测集, 外测度: 卡拉泰奥多里构造 测度扩张定理, 勒贝格测度, 长度与面积 \entry{复测度与哈恩分解}

% \chapter{可测函数} \label{cpt_86}
%---------------------------------------
% 简单函数 \entry{可测函数} \entry{分布律} \entry{按测度收敛} \entry{几乎处处收敛} \entry{叶戈洛夫定理} \entry{可测函数按连续函数逼近}

\chapter{勒贝格积分} \label{cpt_87}
---------------------------------------
\entry{Lebesgue 积分}{Lebes1}
\entry{Lebesgue 积分的一些补充性质}{Lebes2}
\entry{Lebesgue 可积的函数}{LIntFn}
\entry{勒贝格控制收敛定理}{LebDC}
% 单调收敛定理 与黎曼积分的关系 控制收敛定理 傅比尼定理 赫尔德不等式与闵科夫斯基不等式 一致可积性 应用: 绝对收敛的含参数积分 换元公式 长度与面积的计算

% \chapter{广义函数} \label{cpt_88}
% %-------------------------------------
\entry{复值测度与广义函数}{GenFun}
% 拉东-尼科蒂姆定理

\chapter{函数} \label{cpt_89}
---------------------------------------
% Giacomo: 不知道该叫什么名字
\entry{赫尔德条件}{HolFun}
% 有界变差函数 绝对连续函数 绝对连续函数的牛顿-莱布尼兹公式
% \entry{扬氏不等式与逼近}  \entry{黑利选择原理} \entry{极大函数}

\part{偏微分方程} \label{prt_PDE}
%======================================
\chapter{偏微分方程的解法} \label{cpt_90}
%---------------------------------------
% Giacomo: 不知道怎么起名,有更合适的可以改
\entry{一阶线性偏微分方程与常微分方程组的等价性}{LPaODE}
\entry{分离变量法解偏微分方程}{SepVar}
\entry{格林函数解线性非齐次微分方程}{GreenF}

\entry{拉普拉斯方程}{LapEq}
\entry{调和函数}{HarFun}
\entry{泊松方程}{PoiEqu}
\entry{球坐标系中的偏微分算符}{SphNab}
\entry{球坐标系中的拉普拉斯方程}{SphLap}
\entry{柱坐标系中的矢量算符}{CylNab}
\entry{柱坐标系中的拉普拉斯方程}{CylLap}

\entry{三维直角坐标系中的亥姆霍兹方程}{RHM}
\entry{球坐标系中的亥姆霍兹方程}{SphHHz}
\entry{柱坐标中的亥姆霍兹方程}{CylHlm}

\entry{分离变量法与张量积空间}{SVarDP}

\entry{(低阶)偏微分方程的分类与特征线}{chacur}

% \chapter{椭圆型偏微分方程} \label{cpt_91}
%---------------------------------------
% 赫尔德空间, 索伯列夫空间 I, 索伯列夫空间 II, 椭圆微分算子, 流形上的椭圆微分算子, 位势积分, 等温坐标, 椭圆正则性理论 I, 椭圆正则性理论 II, 椭圆正则性理论 III, 自伴椭圆微分算子的谱, 外尔律, 听音辨鼓问题, 椭圆复形, 流形上的霍奇分解

\chapter{特殊函数} \label{cpt_92}
%---------------------------------------
\entry{勒让德多项式}{Legen}
\entry{连带勒让德函数}{AsLgdr}
\entry{Hermite 多项式}{HermiP}
\entry{贝塞尔函数}{Bessel}
\entry{球贝塞尔函数}{SphBsl}
\entry{球谐函数}{SphHar}
\entry{实球谐函数}{RYlm}
\entry{球谐函数表}{YlmTab}
\entry{连带拉盖尔多项式}{Laguer}
\entry{双 Gamma 函数}{digama}
\entry{Wigner D 矩阵、球谐函数的旋转}{WigDmt}
\entry{平面波的球谐展开}{Pl2Ylm}
\entry{库仑势能的球谐展开}{PChYlm}
\entry{球谐展开中径向函数的归一化}{FrNorm}
\entry{广义球谐函数}{GenYlm}
\entry{误差函数}{Erf}
\entry{黎曼 zeta 函数}{RiZeta}
\entry{虚误差函数}{Erfi}
\entry{超几何函数}{HypGeo}
\entry{Kummer 函数(1F1)}{Kummer}
\entry{椭圆积分}{EliInt}
\entry{库仑函数}{CulmF}
\entry{艾里函数}{AiryF}
\entry{三角积分}{TriInt}

\part{泛函分析} \label{prt_Funal}
%====================================
\chapter{拓扑向量空间、巴拿赫空间和希尔伯特空间} \label{cpt_93}
%---------------------------------------
\entry{Minkowski 泛函}{MinFul}
\entry{线性泛函的延拓}{ExLina}
\entry{拓扑向量空间}{tvs}
\entry{线性连续泛函}{LinCon}
\entry{共轭空间与代数共轭空间}{ConSpa}
\entry{赋范空间(泛函分析)}{NormV}
\entry{共轭空间中的强拓扑}{STop}
\entry{里斯引理(泛函分析)}{RiLem}
\entry{巴拿赫空间}{banach}
\entry{巴拿赫定理}{BanThm}
% $L^p$空间
% \entry{里斯表示定理I: $L^p$空间} \entry{里斯表示定理II: 博雷尔测度}
% {弱拓扑与弱星拓扑} {可分空间中的弱拓扑} {$L^p$空间中的弱列紧性} {$L^1$空间中的弱列紧性} {测度族的胎紧性} {测度的弱收敛}
% {半范数} {线性泛函} {哈恩-巴拿赫定理} {对偶空间} {弱拓扑} {弱紧与弱列紧} {弗雷歇空间} {巴拿赫定理(续)} {实例}

% \chapter{希尔伯特空间} \label{cpt_95}
% %---------------------------------------
\entry{希尔伯特空间}{Hilber}
% {里斯表示定理III: 希尔伯特空间}
% 拉克斯-米尔格兰姆定理
\entry{装备希尔伯特空间}{RHS}
\entry{宇称算符}{Parity}

% \chapter{广义函数与傅里叶变换} \label{cpt_96}
% %---------------------------------------
\entry{广义函数}{GenFut}
% {试验函数与施瓦兹函数} {广义函数与缓增广义函数} %entry{分布导数} {傅里叶变换} {帕塞瓦尔定理} {卷积}

\chapter{有界算子的谱论} \label{cpt_97}
%---------------------------------------
\entry{有界算子的谱}{BddSpe}
\entry{有界算子的预解式}{BddRsv}
\entry{谱半径}{SpeRad}
% {算子的函数}
\entry{谱投影}{SpePrj}
\entry{例: 有限维方阵}{SpeMat}
% {紧线性算子} {里斯-邵德尔理论} {例: 弗雷德霍姆积分方程} {例: 具有平方可积核的积分方程} {紧自伴算子} {希尔伯特-施密特算子}

%\chapter{扩展: 巴拿赫代数} \label{cpt_98}
% {巴拿赫代数} {元素的谱} {盖尔范德表示}

%\chapter{闭算子的谱论} \label{cpt_99}
% {无界算子} {闭算子} {第一与第二预解公式} {外尔序列} {离散谱与本质谱} {伴随算子} {自伴算子} {弗里德里希扩张} {谱测度} {自伴算子的谱定理} {瑞利-里斯定理}

%\chapter{谱微扰论} \label{cpt_100}

\chapter{一元函数的变分学} \label{cpt_101}
%---------------------------------------
\entry{绝对极值与相对极值(变分学)}{AbPol}
\entry{可取曲线(变分学)}{DesCur}
\entry{变分}{Varia}
\entry{极值的必要条件(变分学)}{PolReq}
\entry{变分的变换(变分学)}{VarCha}
\entry{变分的基本定理(变分学)}{VarDef}
\entry{欧拉方程(变分学)}{ElueEV}
\entry{二次变分}{SecVar}
\entry{极端曲线}{ExtCur}
\entry{端点可变问题}{EPQue}
\entry{斜截条件}{OCCond}
\entry{多元函数泛函的极值}{MulFP}
\entry{条件极值问题与拉格朗日方法}{CPLM}

\chapter{笔记} \label{cpt_102}
%---------------------------------------
\entry{泛函分析笔记 1}{FnalNt}
\entry{泛函分析笔记 2}{FnalN2}
\entry{泛函分析笔记 3}{FnalN3}
\entry{泛函分析笔记 4}{FnalN4}
\entry{泛函分析笔记 5}{FnalN5}

\part{物理学中的数学方法} \label{prt_MathPh}
%======================================
% 来画个大饼(
%   大家有想法的可以来写相关文章啊
% 
%   可以参考 Nakahara 的目录。
%
%   微分流形(有区别于数学文章)
%   拓扑(一些物理当中比较重要的概念)
%   李群李代数,李群李代数的表示
%   纤维丛,示性类
%   
\chapter{群表示论(面向物理,偏应用)} \label{cpt_103}
%--------------------------------------
\entry{群的矩阵表示及实例}{gprep}
\entry{群代数与正则表示}{gpalg}
\entry{表示的约化}{redrep}
\entry{群函数}{gpfun}
\entry{舒尔引理(Schur's lemma)}{Schlem}

\chapter{李群和李代数} \label{cpt_104}
%--------------------------------------
% 矩阵李群
% 伽利略群和它的李代数
\entry{洛伦兹群的李代数}{lielot}
% SU(2) 的表示分类(粒子自旋)
% SO(3) 的表示分类(球谐函数)
% SU(2),SO(3) 的拓扑性质(基本群),以及它与费米子、玻色子分类的关系


\part{经典力学} \label{prt_ClaMec}
%======================================
\chapter{质点运动学} \label{cpt_105}
%--------------------------------------
\entry{物理量和单位转换}{Units}
\entry{无量纲的物理公式}{NoUnit}
\entry{位置矢量、位移}{Disp}
\entry{速度、加速度(一维)}{VnA1}
\entry{速度、加速度}{VnA}
\entry{位移与路程}{DPAS}
\entry{圆周运动的速度}{CMVD}
\entry{圆周运动的加速度}{CMAD}
\entry{匀加速直线运动}{CnstAL}
\entry{匀加速运动}{ConstA}
\entry{曲线运动的加速度}{PCuvMo}
\entry{极坐标中的速度和加速度}{PolA}
\entry{伽利略变换}{GaliTr}
\entry{速度的坐标系变换(无转动)}{Vtrans}
\entry{速度的坐标系变换}{Vtran2}
\entry{加速度的参考系变换}{AccTra}

\chapter{质点动力学} \label{cpt_106}
%---------------------------------------
\entry{力的合成与分解}{Fdecom}
\entry{绳结的受力分析}{Knot}
\entry{牛顿运动定律、惯性系}{New3}
\entry{圆周运动的向心力}{CentrF}
\entry{重力、重量}{Weight}
\entry{功、功率}{Fwork}
\entry{动能、动能定理(单个质点)}{KELaw1}
\entry{势能(简介)}{POTENT}
\entry{力场、保守场、势能}{V}
\entry{状态量和过程量}{StaPro}
\entry{机械能守恒(单个质点)}{ECnst}
\entry{动量、动量定理(单个质点)}{PLaw1}
\entry{角动量、角动量定理、角动量守恒(单个质点)}{AMLaw1}
\entry{质点问题(摘要)}{MPAB}
\entry{简谐振子(经典力学)}{SHO}
\entry{多自由度简谐振子(经典力学)}{MSHO}
\entry{受阻落体}{RFall}
\entry{单摆}{Pend}
\entry{圆锥摆}{ConPen}
\entry{傅科摆的角速度推导}{Fouclt}
\entry{非惯性参考系、惯性力}{Iner}
\entry{滑块和运动斜面问题}{blkSlp}
\entry{离心力}{Centri}
\entry{科里奥利力}{Corio}
\entry{旋转参考系的 “机械能守恒”}{Rconst}
\entry{地球表面的科里奥利力}{ErthCf}
\entry{地球表面的力}{FOE}
\entry{潮汐力}{Tidal}
\entry{相对性原理}{RELTHM}
\entry{位力定理}{CM01}

\chapter{质点系} \label{cpt_108}
%---------------------------------------
\entry{自由度}{DoF}
\entry{质点系}{PSys}
\entry{质心的定义}{CM}
\entry{质心参考系}{CMfram}
\entry{质点系(摘要)}{MPTA}
\entry{质点系的动量}{SysMom}
\entry{动量定理、动量守恒、质心的牛顿第二定律}{PLaw}
\entry{质点系的动能、柯尼希定理}{Konig}
\entry{力矩}{Torque}
\entry{系统的角动量}{AngMom}
\entry{角动量定理、角动量守恒}{AMLaw}
\entry{二体系统}{TwoBD}
\entry{二体碰撞}{TwoCld}

\chapter{刚体} \label{cpt_109}
%---------------------------------------
% \entry{【导航】刚体运动}{RigBod}
\entry{刚体}{RigBd}
\entry{刚体的平面运动(摘要)}{RGAB}
\entry{轻杆模型}{rod}
\entry{木块堆叠问题(里拉斜塔)}{LireTo}
\entry{刚体的静力平衡}{RBSt}
\entry{刚体定轴转动、转动惯量}{RigRot}
\entry{平行轴定理、垂直轴定理、可加性定理}{MIthm}
\entry{常见几何体的转动惯量}{ExMI}
\entry{刚体的平面运动方程}{RBEM}
\entry{惯性张量}{ITensr}
\entry{刚体的惯量主轴}{PrncAx}
\entry{刚体的瞬时转轴、角速度的矢量相加}{InsAx}
\entry{刚体定轴转动 2}{RBrot2}
\entry{纯滚动}{Pscrol}
\entry{转动与平动的类比}{RATC}
\entry{刚体定轴转动的力矩做功、动能、动能定理}{RigEng}
\entry{刚体的动能、动能定理}{RBKE}
\entry{刚体的运动方程}{RBEqM}
\entry{刚体定点旋转的运动方程(欧拉角)}{RigEul}
\entry{刚体运动方程(四元数)}{RBEMQt}
\entry{简单刚体系统的静力学分析}{RGDFA}
\entry{滑动还是翻转}{SDOTN}

\chapter{材料力学} \label{cpt_110}
%---------------------------------------
\entry{刚体的内力}{INTFRC}
\entry{应力}{STRESS}
\entry{斜面应力公式}{CHYSTR}
\entry{主应力}{PRSTR}
\entry{位移与应变}{Strain}
\entry{杨氏模量、泊松比、剪切模量、广义胡克定律的基本形式}{YoungM}
\entry{广义胡克定律(弹性力学本构关系)}{ELACST}
\entry{应力-应变曲线}{SSCUR}
\entry{应力、应变场的唯一性与叠加原理}{SSPRN}

\chapter{软体和流体力学} \label{cpt_111}
%---------------------------------------
\entry{浮力的计算(散度公式)}{Boya2}
\entry{绳的法向压力}{RopeFP}
\entry{悬链线}{Catena}
\entry{流体和固体}{SLG}
\entry{流、流密度}{CrnDen}
\entry{伯努利方程}{Bernul}
\entry{黏度}{viscos}
\entry{流体运动的描述方法}{fluid1}
\entry{物质导数(实质导数)}{fluid2}
\entry{流体力学守恒方程}{fluidC}
\entry{Navier-Stokes 方程}{NSeq}
\entry{流体力学方程组}{fluidE}
% {流体力学的控制方程}
\entry{流体的控制方程}{scp999}

\chapter{振动与波动} \label{cpt_112}
%---------------------------------------
\entry{振动的指数形式}{VbExp}
\entry{能量法解谐振动问题}{EnerVi}
\entry{拍频}{beatno}
\entry{受阻简谐振子}{SHOf}
\entry{简谐振子的品质因数}{SHOq}
\entry{简谐振子受迫运动}{SHOfF}
\entry{弹簧的串联和并联}{Spring}
\entry{共振}{ResoN}
\entry{平面简谐波}{PWave}
\entry{横波与纵波}{LAT}
\entry{驻波}{StaWav}
\entry{波包}{WvPck}
\entry{高斯波包}{GausPk}
\entry{cos2 波包}{cos2wp}
\entry{群速度}{GroupV}
\entry{多普勒效应(一维匀速)}{Dople1}
\entry{多普勒效应}{Dopler}
\entry{一维波动方程}{WEq1D}
% 未完成 边界条件 (两条密度不同的绳子)
\entry{二维波动方程}{Wv2D}
\entry{波的能量}{WaEner}
\entry{波的强度}{WaInte}
\entry{冲击波}{ShoWav}

\chapter{中心力场问题} \label{cpt_113}
%---------------------------------------
\entry{万有引力(综述)}{UG}
\entry{万有引力、引力势能}{Gravty}
\entry{壳层定理、均匀球体的引力场}{SphF}
\entry{开普勒三定律}{Keple}
\entry{中心力场问题}{CenFrc}
\entry{开普勒问题}{CelBd}
\entry{开普勒问题的运动方程}{EqMoKp}
\entry{拉普拉斯—龙格—楞次矢量}{LRLvec}
\entry{轨道方程、比耐公式}{Binet}
\entry{开普勒第一定律的证明}{Keple1}
\entry{开普勒第二和第三定律的证明}{Keple2}
\entry{散射}{Scater}
\entry{卢瑟福散射}{RuthSc}
\entry{闭合轨道的条件}{ClsOrb}
\entry{Bohr-Sommerfeld 原子模型}{BohrEc}
\entry{狭义相对论效应造成的近日进动}{relPre}

\chapter{分析力学} \label{cpt_114}
%---------------------------------------
\entry{欧拉—拉格朗日方程(经典力学)}{Lagrng}
\entry{约束及其分类}{Restri}
\entry{广义力}{LagEqQ}
\entry{虚位移、虚功、虚功原理}{VirWrk}
\entry{拉格朗日方程和极值问题}{LagPrb}
\entry{贝尔特拉米等式}{Beltra}
\entry{最速降线问题}{Brachi}
\entry{单摆(大摆角)}{SinPen}
\entry{双摆和三摆}{Pendu3}
\entry{小振动}{Oscil}
\entry{拉格朗日方程的证明、达朗贝尔原理}{dAlbt}
\entry{最小作用量、哈密顿原理}{HamPrn}
\entry{运动积分}{motint}
\entry{端点可变的作用量}{EPAct}
\entry{莫培督原理}{MPri}
\entry{二体问题(分析力学)}{twoobj}
\entry{勒让德变换}{TrLgdr}
\entry{哈密顿正则方程}{HamCan}
\entry{泊松括号}{poison}
\entry{正则变换}{CanTra}
\entry{正则变换 2}{ClsMec}
\entry{哈密顿—雅可比方程}{HJeq}
\entry{浸渐不变量}{ConAdi}

\chapter{轨道力学} \label{cpt_115}
%---------------------------------------
\entry{二体问题综述}{ConDB}
\entry{轨道参数、时间变量}{OribP}
\entry{限制性三体问题}{TriLim}
\entry{雅可比常量}{JacCon}
\entry{雅可比常量2}{JConst}
\entry{拉格朗日点}{LPoint}
\entry{光子火箭}{PhRoc}
\entry{太空电梯}{SpcLad}

\part{光学} \label{prt_Optics}
%======================================
\chapter{几何光学} \label{cpt_116}
%---------------------------------------
\entry{波面、光线和光束}{Opt1}
\entry{几何光学基本定律}{GeOp2}
\entry{光学系统基本概念}{GeOp3}
\entry{惠更斯原理}{Huygen}
\entry{光的折射、斯涅尔定律}{Snel}
\entry{相移}{PhaSft}
\entry{薄透镜}{ThnLen}

\chapter{波动光学} \label{cpt_117}
%---------------------------------------
\entry{光的电磁波性质}{WaOp1}
\entry{可见光谱}{VisSpt}
\entry{双缝干涉中一个重要极限}{SltLim}
\entry{干涉、光强的余弦平方分布}{IntCos}
\entry{杨氏双缝干涉实验}{Young}
% \entry{衍射}{optics}
% {单缝衍射} {多缝衍射} {薄膜干涉}% 例如牛顿环、楔形薄膜
\entry{劳埃德镜实验}{Lloyd}
\entry{普通光源的发光机理}{LumiMe}
\entry{激光原理}{LaserT}
\entry{单色光}{MonoLi}
% \entry{偏振光}{PolLig}
\entry{高斯光束}{GausBm}
\entry{晶体衍射}{CrysDf}
\entry{高斯光束 3}{GBeam}
\entry{偏振光三维视角代码(Mathematica)}{PLCode}

\part{电动力学} \label{prt_EM}
%======================================
\chapter{静电学} \label{cpt_118}
%---------------------------------------
\entry{厘米—克—秒单位制}{CGS}
\entry{高斯单位制}{GaussU}
\entry{库仑定律}{ClbFrc}
\entry{电场}{Efield}
\entry{电势、电势能}{QEng}
\entry{电偶极子}{eleDpl}
\entry{电偶极子 2}{eleDP2}
\entry{电场的高斯定律}{EGauss}
\entry{电场的高斯定律证明}{EGausP}
\entry{静电场的环路定理}{ELECLD}
\entry{静电势的泊松方程}{EPoiEQ}
\entry{静电边值条件与唯一性定理}{empoi}
\entry{电场的能量}{EEng}
\entry{点电荷模型与连续电荷模型的缺陷}{elecdb}

\chapter{静磁学} \label{cpt_119}
%---------------------------------------
\entry{洛伦兹力}{Lorenz}
\entry{安培力}{FAmp}
\entry{电流}{I}
\entry{电流密度}{Idens}
\entry{磁场}{MagneF}
\entry{磁矩}{MagMom}
\entry{磁偶极矩}{Bdipol}
\entry{磁场的高斯定律}{MagGau}
\entry{磁通量}{BFlux}
\entry{磁场的能量}{BEng}
\entry{安培环路定律(静磁学)}{AmpLaw}
\entry{毕奥—萨伐尔定律}{BioSav}
\entry{磁场中闭合电流的合力}{EBLoop}
\entry{磁场中闭合电流的力矩}{EBTorq}

\chapter{电磁学与麦克斯韦方程组} \label{cpt_120}
%---------------------------------------
\entry{法拉第电磁感应定律}{FaraEB}
\entry{位移电流、安培—麦克斯韦公式}{DisCur}
\entry{电荷守恒、电流连续性方程}{ChgCsv}
\entry{麦克斯韦方程组}{MWEq}
\entry{电磁场标势和矢势}{EMPot}
\entry{电磁场的规范变换}{Gauge}
\entry{洛伦兹规范}{LoGaug}
\entry{库仑规范(电动力学)}{Cgauge}
\entry{磁标势}{elecdy}
\entry{磁矢势}{BvecA}
\entry{电磁场的能量守恒、坡印廷矢量}{EBS}

\chapter{介质} \label{cpt_121}
%---------------------------------------
\entry{导体}{Cndctr}
\entry{导体的静电平衡}{MetEqv}
\entry{霍尔效应}{Hallef}
\entry{电介质摘要}{DLT}
\entry{磁介质摘要}{MGNT}
\entry{电介质的简单模型}{dieleS}
\entry{电介质的微观结构}{Dielec}
\entry{电极化强度}{ElecPo}
\entry{电极化强度与极化电荷的关系}{ElePAP}
\entry{介质中的静电场}{EFIDE}
\entry{极化电流}{PolCur}
\entry{分子电流和分子磁矩}{MoMaMo}
\entry{磁介质}{MagMat}
\entry{磁化强度}{MaInte}
\entry{顺磁质的磁化}{ParaMa}
\entry{抗磁质的磁化}{DiaMaM}
\entry{有磁介质时的安培环路定律}{MaAmpe}
\entry{静电场与静磁场(总结)}{estfid}
\entry{动态电磁场问题(总结)}{dynfld}
\entry{麦克斯韦方程组(介质)}{MWEq1}
\entry{介质的边界条件}{mbdy}
\entry{电磁场波动方程(均匀介质)}{MedWF}
\entry{导体中的电磁波}{MetalW}

\chapter{电路} \label{cpt_122}
%---------------------------------------
\entry{电路}{Circ}
\entry{电压和电动势}{Voltag}
\entry{电阻、欧姆定律、电阻率、电导率}{Resist}
\entry{基尔霍夫电路定律}{Kirch}
\entry{电感}{Induct}
\entry{电容}{Cpctor}
\entry{电阻的串联和并联}{Rcomb}
\entry{电感的串联和并联}{IndCmb}
\entry{电容的串联和并联}{Ccomb}
\entry{LC 振荡电路}{LC}
% \entry{LRC 电路}{LRCcir}
\entry{力电振动类比}{MeElec}
\entry{LC 受迫振荡电路}{EleRes}
\entry{LR 电路}{LRC}
\entry{Y-Δ 变换、星角变换}{Tri2St}
\entry{电容—电阻电路充放电曲线}{RCcurv}
\entry{惠斯通电桥}{WheBrg}
\entry{阻抗、电抗}{impeda}
\entry{电抗、容抗、感抗}{CapRea}

\chapter{电磁波} \label{cpt_123}
%---------------------------------------
\entry{电场波动方程}{EWEq}
\entry{真空中的平面电磁波}{VcPlWv}
\entry{简单的偏振电磁波}{PLREMW}
\entry{平面电磁波的能量叠加}{PwvAdd}
\entry{时谐电磁波}{TSEBW}
\entry{电磁波包的能谱}{WpEng}
\entry{菲涅尔公式、布儒斯特角、临界角、内反射与外反射}{Fresnl}
\entry{盒中的电磁波}{EBBox}
% 边界条件,波动光学简介、波导与谐振腔,TE 波,TM 波……

\chapter{电动力学} \label{cpt_124}
%---------------------------------------
\entry{电多极展开(球坐标)}{EMulPo}
\entry{格林函数与静电边值问题}{EleGr}
\entry{磁多极矩}{edy33}
\entry{电磁场推迟势}{RetPt0}
\entry{电偶极子辐射}{DipRad}
\entry{李纳维谢尔势}{LWP}
\entry{带电粒子的辐射}{chgrad}
\entry{磁单极子}{BMono}
\entry{恩绍定理}{earnsh}
\entry{非齐次亥姆霍兹方程、推迟势}{RetPot}
\entry{磁旋比、玻尔磁子}{BohMag}

\chapter{电磁学与经典场论} \label{cpt_125}
%---------------------------------------
\entry{电磁场的动量守恒、动量流密度张量}{EBP}
\entry{电磁场的参考系变换}{EMRef}
\entry{拉格朗日电磁势}{EMLagP}
\entry{电磁场中粒子的拉氏量}{ElecLS}
\entry{电磁场的作用量}{ElecS}
\entry{电磁场角动量分解}{EMAMSp}
\entry{电磁场张量}{EMFT}
\entry{电磁场的能动张量}{EMtens}

\part{相对论} \label{prt_Relati}
%======================================
\chapter{狭义相对论} \label{cpt_126}
%---------------------------------------
% 需要多加几篇文章进行举例来帮助初学者理解
\entry{光速不变假设的一些误解和历史}{SpeRel}
\entry{事件与尺缩效应}{SRsmt}
\entry{时间的变换与钟慢效应}{SRtime}
\entry{洛伦兹变换}{SRLrtz}
\entry{四维矢量}{vect2}
\entry{斜坐标系}{ObSys}
\entry{斜坐标系表示洛伦兹变换}{SROb}
\entry{自然单位制、普朗克单位制}{NatUni}
\entry{约化光速}{SRc}
\entry{洛伦兹变换的代数推导}{LornzT}
\entry{相对论速度变换}{RelVel}
\entry{相对论加速度变换}{SRAcc}
\entry{时空的四维表示}{SR4Rep}
\entry{闵可夫斯基空间}{MinSpa}
\entry{双生子佯谬}{Twins}
\entry{光的多普勒效应}{RelDop}
\entry{洛伦兹群}{qed1}
\entry{托马斯进动}{TmsPrs}

\chapter{狭义相对论(二)} \label{cpt_127}
%这一节更强调数学上严谨和完备的表达,让读者习惯四矢量、张量的基本运算以及熟悉洛伦兹代数的基本结构
%------------------------------
\entry{协变性和不变性}{CoIn}
\entry{时空的几何}{GeoSpa}
\entry{相对论补全}{Comple}
\entry{光与物质粒子的统一(相对论点粒子的作用量)}{RAct}
\entry{电磁力和引力}{EleGra}
\entry{世界线和固有时}{wdline}
\entry{观测者的标架矢量}{tetrad}
\entry{狭义相对论的运动学(有质量粒子)}{SRkine}
\entry{狭义相对论的运动学(无质量粒子和多普勒效应)}{SRkmls}
\entry{狭义相对论与洛伦兹对称性}{SRrefe}
\entry{协变矢量、逆变矢量与指标升降}{vecspr}

\chapter{相对论动力学} \label{cpt_128}
%---------------------------------------
\entry{相对论动力学假设}{SRDyn}
\entry{闵可夫斯基时空中的能动张量}{SRFld}


\chapter{广义相对论} \label{cpt_129}
%---------------------------------------
\entry{等效原理(综述)}{DXYL}
\entry{等效原理}{EquPri}
\entry{电磁场和引力场的作用量}{EGA}
\entry{引力红移}{grared}
\entry{施瓦西度规下时空的测地线}{grScGe}
\entry{引力的弱场近似}{WeakG}
\entry{爱因斯坦场方程}{EinEqn}
\entry{爱因斯坦场方程(含宇宙项)}{EinEq2}
\entry{因果结构}{Causal}
\entry{广义相对论中的对称性和 Killing 矢量场}{GR}
\entry{线性引力}{LinGra}
\entry{ADM 形式}{ADMF}
\entry{经典统一场论(综述)}{JDTYCL}

\chapter{引力波} \label{cpt_166}
%---------------------------------------
\entry{引力波(综述)}{YLB}
\entry{引力波的几何描述}{Geomet}
\entry{TT 规范}{TTGaug}
\entry{引力波和测试质量的相互作用}{intera}


\part{经典场论} \label{prt_ClaFld}
%======================================
\chapter{基本思想} \label{cpt_130}
%---------------------------------------
\entry{相空间和相流}{PSaPF}
\entry{从分析力学到场论}{CFa1}
\entry{Noether 定理}{NoeTh}
\entry{场与粒子的相互作用}{IntFP}

\chapter{狭义相对论动力学} \label{cpt_131}
%---------------------------------------
\entry{自由粒子拉格朗日函数(狭义相对论)}{FPLSR}

\chapter{电动力学} \label{cpt_132}
\entry{麦克斯韦方程组(外微分形式)}{MWEq2}


\part{量子力学} \label{prt_QM}
%======================================
% 未完成 束缚态的一般性质: 节点数, 对称性 (偶势能的基态是偶函数), 简并性(一维情况不简并)

\chapter{入门} \label{cpt_133}
%---------------------------------------
\entry{【导航】量子力学}{QMmap}
\entry{量子力学的诞生}{QMborn}
\entry{玻尔原子模型}{BohrMd}
\entry{玻尔原子模型(约化质量)}{HRMass}
\entry{氢原子光谱(综述)}{QYZGP}
\entry{原子单位制}{AU}
\entry{指数衰减}{ExpDec}
\entry{精细结构常数}{FinStr}

\chapter{一般理论} \label{cpt_134}
%---------------------------------------
\entry{狄拉克符号}{braket}
\entry{量子力学的基本原理(量子力学)}{QMPrcp}
\entry{量子力学中的基本算符}{OprQM}
\entry{Stern-Gerlach 实验}{SGExp}
\entry{量子力学与矩阵}{QMmat}
\entry{量子力学的算符和本征问题}{QM1}
\entry{算符对易性(量子力学)}{ComOpQ}
\entry{平均值(量子力学)}{QMavg}
\entry{守恒量(量子力学)}{QMcons}
\entry{Ehrenfest 定理}{Ehrenf}
\entry{概率流密度}{PrbJ}
\entry{可观测量的相容性}{MsrCmp}
\entry{不确定性原理}{Uncert}
\entry{Wigner 基本定理}{WigThe}

\chapter{单粒子一维问题} \label{cpt_135}
%---------------------------------------
\entry{平面波的的正交归一化}{EngNor}
\entry{薛定谔方程(单粒子一维)}{TDSE11}
\entry{量子散射(一维)}{Sca1D}
\entry{一维散射态的正交归一化}{ScaNrm}
\entry{定态薛定谔方程(单粒子一维)}{SchEq}
\entry{位置表象和动量表象}{moTDSE}
\entry{好量子数}{GoodQN}
\entry{算符的矩阵表示}{OpMat}

\chapter{单粒子多维问题} \label{cpt_136}
%---------------------------------------
\entry{薛定谔方程(单粒子多维)}{QMndim}
\entry{薛定谔方程 2(单粒子多维)}{TDSE}
\entry{薛定谔方程的分离变量法}{SEsep}
\entry{氢原子的定态薛定谔方程(球坐标)}{RadSE}
\entry{氢原子的含时薛定谔方程(球坐标)}{RYTDSE}
\entry{柱坐标系中的薛定谔方程}{CyliSE}
\entry{轨道角动量(量子力学)}{QOrbAM}
\entry{轨道角动量升降算符归一化}{QLNorm}
\entry{球坐标系中的轨道角动量算符}{SphAM}
\entry{泡利矩阵}{pauliM}
\entry{自旋角动量}{Spin}
\entry{自旋角动量矩阵}{spinMt}
\entry{自旋 1/2 粒子的非相对论波函数}{scheq2}
\entry{角动量的叠加(量子力学)}{AdAngM}
\entry{角动量的叠加 2(量子力学)}{AMAdd}
\entry{算符的指数函数、波函数传播子}{OpExp}
\entry{拉莫尔进动}{Larmor}
\entry{平移算符}{tranOp}
\entry{旋转算符}{rotOp}

\chapter{多粒子问题} \label{cpt_137}
%---------------------------------------
\entry{多体薛定谔方程}{NbdQM}
\entry{全同粒子}{IdPar}
\entry{泡利不相容原理}{PauliE}
\entry{粒子交换算符}{ExchOp}
\entry{量子态的对称化与反对称化}{symetr}
\entry{全同粒子的交换力}{ExchF}
\entry{Clebsch–Gordan 系数}{SphCup}
\entry{Wigner 3j 符号}{ThreeJ}
\entry{Wigner 6j 符号}{SixJ}
\entry{Wigner 9j 符号}{NineJ}

\chapter{定态问题} \label{cpt_138}
%---------------------------------------
\entry{无限深方势阱}{ISW}
\entry{一维有限深方势阱}{FSW}
\entry{方势垒}{SqrPot}
\entry{无限深阶梯势阱}{StpPot}
\entry{有限深不对称方势阱}{AMW}
\entry{阶梯势能散射}{StepV}
\entry{升降算符}{RLop}
\entry{量子简谐振子(升降算符法)}{QSHOop}
\entry{简谐振子升降算符归一化}{QSHOnr}
\entry{量子简谐振子(级数法)}{QSHOxn}
\entry{一维自由粒子(量子)}{FreeP1}
\entry{一维 delta 势能散射}{Dsc1D}
\entry{一维 delta 势能晶格}{DelCry}
\entry{线性势能的定态薛定谔方程}{LinPot}
\entry{不含时微扰理论(量子力学)}{TIPT}
\entry{含连续态的不含时微扰理论(量子力学)}{TIPTc}
\entry{二阶不含时微扰理论(量子力学)}{TIPT2}
% \entry{一维薛定谔方程的格林函数}{adaf}
\entry{WKB 近似}{WKB}
\entry{二维无限深方势阱}{ISW2D}
% \entry{二维有限深方势阱}{FSW2D}
\entry{无限深圆形势阱}{CirISW}
\entry{无限深球势阱}{ISphW}
\entry{有限深球势阱}{FiSph}
\entry{三维量子简谐振子(球坐标系)}{SHOSph}

\chapter{含时问题} \label{cpt_139}
%---------------------------------------
\entry{拉比频率}{RabiF}
\entry{一维自由高斯波包(量子)}{GausWP}
\entry{含时微扰理论(束缚态)}{TDPT}
\entry{含时微扰理论}{TDPTc}
\entry{几种含时微扰}{TDPEx}
\entry{费米黄金法则}{FermGR}
\entry{绝热近似(量子力学)}{AdiaQM}
\entry{含时微扰理论(绝热)}{AdiaPT}

\chapter{量子散射} \label{cpt_140}
%---------------------------------------
\entry{量子散射(单粒子弹性)}{ParWav}
\entry{量子散射的波恩近似}{BornSc}
\entry{球面散射态与平面散射态的转换}{Scatt2}
\entry{Lippmann-Schwinger 方程}{LipSch}
\entry{含时散射的形式理论}{TDFSc}
\entry{库仑散射(量子)}{CulmWf}
\entry{波包的延迟(量子力学)}{tDelay}
\entry{光电离时间延迟:一维波包、氢原子与氦原子}{HeAna2}
% \entry{氢原子的 streaking 计算}{HyCLC}
\entry{多通道散射}{MulSct}
\entry{多通道散射中的绝热基底}{Adibat}
\entry{R-矩阵法(量子力学)}{Rmat}


\chapter{量子力学 2} \label{cpt_141}
%---------------------------------------
\entry{量子力学中的变分法、Rayleigh-Ritz 变分法}{QMVar}
\entry{密度矩阵}{denMat}
\entry{约化密度矩阵}{partra}
\entry{质心系中的多粒子问题}{SECM}
\entry{量子力学的基本假设}{QMPos}
\entry{电磁场中的薛定谔方程及规范变换}{QMEM}
\entry{库仑规范(量子力学)}{CouGau}
\entry{长度规范}{LenGau}
\entry{速度规范}{LVgaug}
\entry{加速度规范}{AccGau}
\entry{Volkov 波函数}{Volkov}
\entry{薛定谔绘景和海森堡绘景}{HsbPic}
\entry{相互作用绘景}{QM06}
\entry{Hartree-Fock 方法}{HarFor}
\entry{Baker-Hausdorff 公式}{BAHA}
\entry{超导唯象解释——伦敦方程}{edy34}
\entry{延迟选择量子擦除实验}{DCQ}

\chapter{量子信息与量子计算} \label{cpt_142}
%---------------------------------------
\entry{量子隐形传态(综述)}{LZYXCT}
\entry{量子比特}{Qubit}
\entry{量子系综}{qEns}
\entry{量子测量}{qmeas}
\entry{量子纠缠 2}{entang}
\entry{迹距离}{Trdist}
\entry{保真度}{fideli}
\entry{von Neumann 熵}{vonNE}
\entry{EPR 佯谬与定域隐变量理论}{EPR}
\entry{Bell 不等式与 CHSH 不等式}{chsh}
\entry{量子信道}{qchanl}
\entry{量子信息守恒}{QIcons}
\entry{量子不可克隆定理}{noClo}
\entry{精确对角化}{ED}

% \entry{量子系统的纯化}{purify}

\chapter{量子力学与量子场论} \label{cpt_143}
%---------------------------------------
\entry{前言}{QFIntro}
\entry{基本概念}{Basics}
\entry{全同粒子的统计}{IdParS}
\entry{近似理论:微扰}{AprPtr}
\entry{角动量 2 (量子力学)}{QMAM}
% 第六章直接从 sakurai 翻译, 后面的一些内容非常零碎跳过
\entry{冷原子基本知识}{UCBas}
\entry{两个原子间的相互作用}{TwoAtF}
\entry{Feshbach 共振}{FeshRs}
\entry{BCS-BEC Crossover 的平均场描述}{BCSBEC}
\entry{BEC 超流}{BECSup}
\entry{绘景变换与时间演化}{picQFT}

\chapter{原子分子物理} \label{cpt_144}
%---------------------------------------
\entry{氢原子基态的波函数}{HWF0}
\entry{氢原子的束缚态波函数}{HWF}
\entry{氢原子波函数分析}{Hanaly}
\entry{氢原子的精细能级结构}{HfineS}
\entry{氢线(21厘米线)}{HydroL}
\entry{类氢原子斯塔克效应(微扰)}{HStark}
\entry{塞曼效应}{ZemEff}
\entry{抛物线坐标系中的类氢原子定态波函数}{ParaHy}
\entry{类氢原子的 Stark 效应(抛物线坐标系)}{HStrk2}
\entry{电磁场中的类氢原子}{EMHydr}
\entry{跃迁概率(一阶微扰)}{HionCr}
\entry{单电子跃迁截面(一阶微扰)}{SIcros}
\entry{氢原子的跃迁偶极子矩阵元、选择定则}{SelRul}
\entry{氢原子的跃迁偶极子矩阵元计算(束缚态之间)}{HDipM}
\entry{氢原子的跃迁偶极子矩阵元(束缚态到连续态)}{HyIon2}
\entry{跃迁偶极子矩阵的三种形式}{DipEle}
\entry{康普顿散射}{Comptn}
\entry{电子轨道与元素周期表}{Ptable}
\entry{能项符号}{TrmSym}
\entry{兰姆位移}{LambSh}
\entry{ponderomotive 能量}{Ponder}
\entry{氢原子隧道电离}{Htunnl}
\entry{Keldysh 参数}{keldis}
\entry{单电子原子模型}{SAE}
\entry{氦原子中的对易算符与能项符号}{HeComu}
\entry{ADK 电离率}{ADKrat}
\entry{Floquet 理论}{Floque}
\entry{布洛赫理论}{Bloch}
\entry{FROG}{Frog}
\entry{惰性气体分子林纳德-琼斯势(量子力学)}{LenJoP}
\entry{双原子分子势能曲线}{dpecs}
\entry{双原子分子莫尔斯势(量子力学)}{MoPoQM}
\entry{碱金属原子(量子力学)}{AlkmQM}
\entry{双原子分子克拉策势(量子力学)}{kratPo}
\entry{分子轨道理论(综述)}{MOFa}

\chapter{固体物理} \label{cpt_145}
%---------------------------------------
\entry{晶格振动导论}{LatVib}
\entry{一维单原子链晶格}{onatom}
\entry{一维双原子链晶格}{twatom}
\entry{晶格动力学}{latdyn}
% \entry{晶向、晶面}{CryDir}
\entry{布拉伐格子、基元与原胞}{BraLat}
\entry{德鲁德模型}{DrudeM}
\entry{索末菲模型}{SMFM}
\entry{晶格热容的爱因斯坦理论}{EScap}
\entry{晶格热容的德拜理论}{Debye}
% 能带论导航
\entry{近自由电子模型}{egasmd}
\entry{紧束缚近似}{tbappx}
\entry{倒格空间}{RecLat}
% 能带密度和费米面
% 晶体能带的对称性
% 表面电子态
\entry{电子运动的准经典模型}{cryele}
\entry{布里渊区}{BriZon}

\part{高等量子力学} \label{prt_QM2}
%======================================
% 可以参考的文献
%   刘觉平 量子力学
%   喀兴林 高等量子力学

\chapter{动力学} \label{cpt_146}
%---------------------------------------
\entry{时间演化算符(量子力学)}{TOprt}
\entry{戴森级数}{Dyser}
\entry{传播子(量子力学)}{PpgtQM}
\entry{路径积分(量子力学)}{PIntQM}

\chapter{多体系统的量子力学} \label{cpt_147}
%---------------------------------------
\entry{二次量子化}{SecQua}
\entry{产生和湮灭}{QMGnr}
\entry{单体算符}{oneopr}
\entry{Hubbard 模型}{hubbar}
\entry{Hartree-Fock 近似}{HFock}
\entry{内禀反常霍尔效应}{anHe}

% 单体算符
% 两体算符
% 简并费米气体
% 等

% \chapter{非相对论量子力学的对称性} \label{cpt_148}
% 主要可以参考刘觉平
% 伽利略变换
% 伽利略群与它的李代数
% 通过对称性确定量子力学的动力学变量

% \chapter{路径积分和不可积相位} \label{cpt_149}
% 路径积分量子化
% 不可积相位和 AB 效应
% 磁单极子和 Dirac 电荷量子化条件
% 绝热定理
% Berry 相位
% 非阿贝尔 Berry 相
% Chern number
\chapter{相对论量子力学} \label{cpt_150}
%---------------------------------------
% 这里是一些到相对论量子力学的初步应用,
% Dirac 方程的初步推导
% Dirac 方程的非相对论近似
\entry{泡利方程}{Pauli}
\entry{电磁场中的狄拉克方程}{DiracE}
\entry{狄拉克方程的非相对论近似}{DiracB}
\entry{自旋轨道耦合}{socpl}
% 经典场论 - 标量场、旋量场、矢量场
% 洛伦兹群的旋量表示
% 庞加莱群的李代数

\part{热力学和统计力学} \label{prt_StaMec}
%======================================
\chapter{热力学} \label{cpt_151}
%---------------------------------------
\entry{【导航】热力学与统计力学}{StatMe}
\entry{压强}{Pressu}
\entry{理想气体状态方程}{PVnRT}
\entry{温度、温标}{tmp}
\entry{理想气体}{Igas}
\entry{理想气体的内能}{IdgEng}
\entry{压强体积图}{PVgraf}
\entry{热平衡、热力学第零定律}{TherEq}
\entry{热传导定律}{Heatco}
\entry{热力学第一定律}{Th1Law}
\entry{态函数}{statef}
\entry{盖斯定律与设计路径}{Hess}
\entry{准静态过程}{Quasta}
\entry{等压过程}{EqPre}
\entry{等体过程}{EqVol}
\entry{等温过程}{EqTemp}
\entry{热容量}{ThCapa}
\entry{绝热过程}{Adiab}
\entry{节流过程}{ttpro}
\entry{卡诺热机}{Carnot}
\entry{热力学第二定律}{Td2Law}
\entry{亥姆霍兹自由能}{HelmF}
\entry{吉布斯自由能}{GibbsG}
\entry{熵的宏观定义}{Entrop}
\entry{热力学关系式}{MWRel}
\entry{熵的宏观表达式}{MacroS}
\entry{范德瓦尔斯气体}{Vand}
\entry{热动平衡判据}{equcri}
% 混合熵
\entry{混合熵的微观解释}{MICMXS}


\chapter{多元系和复相系的热力学} \label{cpt_152}
%---------------------------------------
\entry{多元系热力学导引}{mulTh}
\entry{理想气体分压定律}{PartiP}
\entry{理想气体混合的熵变}{IGME}
\entry{相简介(热力学)}{PHS}
\entry{摩尔量与偏摩尔量}{ParMol}
\entry{相变平衡条件}{PhEquv}
\entry{理想混合物的热力学量}{IMCPTV}
\entry{相变的热力学量变化}{PTTVC2}
\entry{克拉伯龙方程}{Clapey}
\entry{吉布斯相律}{GBPL}
\entry{饱和蒸汽压}{VaporP}
\entry{匀晶相图}{ISOMOR}
\entry{共晶、共析相图}{EUTECT}
\entry{包晶相图}{PERITC}
\entry{化学反应平衡}{chemBl}
\entry{理想气体化学平衡条件}{ICheEq}
\entry{表面张力}{sftens}
\entry{杨氏浸润模型}{YNGMDL}
\entry{Landau-Ginzburg 理论}{LanGin}
% 铁磁相变
% Ising 模型
% 重整化群理论
% 临界现象与普适性
% Landau-Ginzburg 理论

\chapter{物质的微观经典理论} \label{cpt_153}
%---------------------------------------
% 能量均分定理
\entry{气体分子对容器壁的压强}{MolPre}
\entry{分子平均碰壁数}{AvgHit}
\entry{气体分子的速度分布}{VelPdf}
\entry{麦克斯韦—玻尔兹曼分布}{MxwBzm}
\entry{经典形核理论}{NCLT}
\entry{晶核的长大}{GGRW}

\chapter{统计力学} \label{cpt_154}
%---------------------------------------
% 玻色气体
% 费米气体
% 实际气体状态方程(集团展开法)
\entry{统计物理·微观与宏观之间的桥梁}{statsc}
\entry{熵的微观定义与玻尔兹曼公式}{entro2}
\entry{近独立子系、理想玻色气体和费米气体}{depsys}
\entry{负温度}{negtem}
\entry{黑体辐射定律}{BBdLaw}
\entry{维恩位移定律}{WienDs}
\entry{斯特藩—玻尔兹曼定律}{SteBol}
\entry{相空间}{PhSpace}
\entry{刘维尔定理(热力学与统计物理)}{LiouTh}
\entry{理想气体的状态密度(相空间)}{IdSDp}
\entry{理想气体单粒子能级密度}{IdED1}
\entry{玻尔兹曼分布(统计力学)}{MBsta}
\entry{热力学量的统计表达式(玻尔兹曼分布)}{TheSta}
\entry{金属中的自由电子气体}{mfcgas}
\entry{理想气体(微正则系综法)}{IdNCE}
\entry{正则系综法}{CEsb}
\entry{理想气体(正则系宗法)}{IdCE}
\entry{理想气体(巨正则系综法)}{IdMCE}
\entry{等间隔能级系统(正则系宗)}{EqCE}
\entry{巨正则系综法}{MCEsb}
\entry{量子气体(单能级巨正则系综法)}{QGs1ME}
\entry{量子气体(巨正则系宗)}{QGsME}
\entry{光子气体}{PhoGas}
\entry{理想气体的熵:纯微观分析}{IdeaS}
\entry{估算理想气体的熵}{IGS}
\entry{伊辛模型}{Ising}
\entry{玻尔兹曼方程}{BolzEQ}
\entry{统计力学公式}{StatEq}
\entry{玻色爱因斯坦凝聚}{BEC}

\chapter{非平衡态热力学} \label{cpt_155}
%---------------------------------------
\entry{大气密度和压强}{atmDen}
\entry{沸腾}{ebull}
\entry{热传导定律、扩散方程与输运过程}{heatc}
\entry{气体输运过程(微观理论)}{gasTra}
% 玻尔兹曼方程和弛豫时间近似

\part{量子场论} \label{prt_QFT}
%======================================
\chapter{相对论性经典场论} \label{cpt_156}
%---------------------------------------
\entry{引言}{QFT0}
\entry{经典场论基础}{classi}
\entry{薛定谔经典场}{claSch}
\entry{标量场}{qed2}
\entry{Klein-Gordon 方程}{KGeq}
\entry{狄拉克方程}{qed4}
\entry{狄拉克方程的自由粒子解}{diracs}
\entry{狄拉克场}{Dirac}
\entry{狄拉克矩阵}{diracm}
%\entry{洛伦兹群覆盖群 SL(2,C) 的不可约表示}{qed3}
\entry{Weyl 旋量}{Weyl}
\entry{自旋与旋转}{spinqm}
\entry{狄拉克方程无质量费米子}{DiM0}

\chapter{自由场论的正则形式理论} \label{cpt_157}
%---------------------------------------
\entry{标量场的量子化}{quanti}
\entry{时空中的标量场}{scalar}
\entry{因果律}{cau}
\entry{标量场的谱}{spectr}
\entry{克莱因-戈登传播子}{Klein}
\entry{Wick 定理(标量场)}{wick}
\entry{粒子产生}{parti}
\entry{狄拉克场的量子化}{Diracq}
\entry{量子化狄拉克场}{quandi}
\entry{狄拉克场的传播子}{diracD}
\entry{自旋求和}{spinsu}

\chapter{相互作用场论} \label{cpt_158}
%---------------------------------------
\entry{相互作用表象}{Ipic}
\entry{散射理论与 S 矩阵}{Smat}
\entry{LSZ约化公式(标量场)}{LSZ}
\entry{Φ$^4$ 理论的费曼规则}{phi4fy}
\entry{LSZ约化公式(旋量场)}{lszspn}
\entry{LSZ 约化公式(矢量场)}{lszqed}
\entry{QED的费曼规则}{qedfey}
\entry{Yukawa 势}{yukpot}
\entry{Bhabha 散射}{Bhabha}
\entry{Compton 散射}{Compto}
\entry{Ward-Takahashi 等式}{ward}
\entry{路径积分与关联函数(量子力学)}{pathi2}
\entry{QCD 费曼规则}{qcdfey}
% \chapter{标量场路径积分} \label{cpt_1e}
% \chapter{Grassman 数路径积分} \label{cpt_1f}
% \chapter{电磁场路径积分} \label{cpt_2a}
% \chapter{非阿贝尔规范场路径积分} \label{cpt_2b}

% \chapter{路径积分形式理论} \label{cpt_159}
%---------------------------------------

\chapter{重整化理论} \label{cpt_160}
%---------------------------------------
\entry{ϕ$^4$ 理论的重整化(单圈修正)}{phi4c1}
\entry{QED 重整化理论—电子自能和光子自能的单圈修正}{qedlop}
\entry{QED 的重整化理论—顶点函数的单圈修正}{qedvet}
% \chapter{非阿贝尔规范场论} \label{cpt_161}

\chapter{标准模型} \label{cpt_8d}
%---------------------------------------
\entry{粒子物理标准模型}{SMAll}


\part{近代物理} \label{prt_MordPh}
%======================================

\chapter{高能物理} \label{cpt_162}
%---------------------------------------
\entry{基本粒子}{BasPar}
\entry{介子(综述)}{JZ}
\entry{希格斯粒子(综述)}{Higgs}

\chapter{弦理论} \label{cpt_163}
%---------------------------------------
\entry{引力量子化}{QuanGR}
\entry{弦论概述}{STover}
\entry{弦论的种类}{TYPEst}
\entry{BRST 量子化}{BRST}
\entry{RNS 超弦}{RNS}
\entry{AdS/CFT 对偶}{AdSCFT}
\entry{AdS/真实世界}{AdSrea}
% \chapter{经典弦 1} \label{cpt_2c}
%---------------------------------------
% \chapter{经典弦 2} \label{cpt_2e}
%---------------------------------------
% \chapter{弦量子化} \label{cpt_3a}
%---------------------------------------
% \chapter{共形场论 1} \label{cpt_3b}或者叫规范场论?
%---------------------------------------
% \chapter{BRST 量子化} \label{cpt_3c}
%---------------------------------------
% \chapter{RNS 超弦} \label{cpt_3d}
%---------------------------------------
% \chapter{紧致化和T对偶} \label{cpt_3e}
%---------------------------------------
% \chapter{超弦理论(续)} \label{cpt_3f}
%---------------------------------------
% \chapter{超弦理论(总结)} \label{cpt_4a}
%---------------------------------------
% \chapter{第II型弦论} \label{cpt_4b}
%---------------------------------------
% \chapter{Heterotic 弦论} \label{cpt_4d}
%---------------------------------------
% \chapter{D-膜} \label{cpt_4e}
%---------------------------------------
% \chapter{黑洞} \label{cpt_5a}
%---------------------------------------
% \chapter{全息原理和AdS/CFT} \label{cpt_5b}
%---------------------------------------
% \chapter{弦理论和宇宙学} \label{cpt_5c}

\chapter{宇宙学} \label{cpt_164}
%---------------------------------------
\entry{宇宙的演化}{UniEvo}
\entry{宇宙学红移}{CoReSh}
\entry{视界}{horizo}
\entry{宇宙中的距离}{DisCos}
\entry{Friedmann-Robertson-Walker (FRW) 度规}{FRW}
\entry{宇宙学的基本方程}{Frieq}

\entry{暗物质引言}{DMintr}
\entry{星系旋转曲线}{GRotat}
\entry{暗物质在其他星系尺度存在的证据}{DMEvi}
\entry{缺少暗物质的星系}{NoDM}
\entry{兹威基关于暗物质的工作}{DMZWI}
\entry{弱引力透镜}{Weakls}
\entry{暗物质的宇宙学证据}{DMCOSM}
\entry{大尺度结构形成}{LSSFor}
\entry{暗物质的模拟}{DMSimu}
\entry{CMB 声学峰}{CMBAco}
\entry{暗物质属性的宇宙学限制}{CMBBD}
\entry{暗物质在银河系和宇宙中的分布}{DMdist}
\entry{暗物质密度分布}{DMdens}
\entry{暗物质的速度分布}{DMvelo}

\entry{引力原子}{GAtom}

\chapter{宇宙学扰动} \label{cpt_165}
%---------------------------------------
\entry{标量扰动}{ScaPT}
\entry{张量扰动}{TenPT}

% \chapter{非厄米物理} \label{cpt_7c}
\entry{非厄米物理导论}{nonHer}
\entry{非厄米物理中的数学}{nHerMa}

\part{计算机} \label{prt_CS}
%======================================

\chapter{Python 语言} \label{cpt_168}
%---------------------------------------
\entry{【导航】Python}{PyFi}
\entry{Python 简介与安装}{Python}
\entry{Python 的 IDE 安装}{PyIDE}
\entry{Python 快速入门}{PyStar}
\entry{Python 解释器使用}{PyItpt}
% \entry{Python 编码规范}{Pycode}
\entry{Python 词法与分析}{Pynot1}
\entry{Python 基本数据类型与转换}{PyData}
\entry{Python 输入和输出}{Pyio}
\entry{Python 数字}{PyNum}
\entry{Python 文件读写}{PyFile}
\entry{Python 判断与循环}{PyIfFr}
\entry{Python 函数}{PyFunc}
\entry{Python 的类}{PyClas}
\entry{Python 的模块(module)}{PyMod}
\entry{Python 异常处理}{PyExcp}
\entry{Python 的变量}{PyType}
\entry{Python 字符串处理}{PyStr}
\entry{字符串的分片与索引(Python)}{Strsi}
\entry{NumPy 库}{numpy}
\entry{Python Matplotlib 画图}{PyPlot}

\chapter{Java 语言} \label{cpt_170}
%--------------------------------------
\entry{java 入门}{java}


\chapter{C 和 C++ 语言} \label{cpt_171}
%---------------------------------------
\entry{【导航】C++}{CppNav}
\entry{在 Linux 上编译 C/C++ 程序}{linCpp}
\entry{C++ 基础}{Cpp0}
\entry{C++ 函数}{CppFun}
\entry{C++ 中的 SFINAE 技巧}{SFINAE}
% \entry{从图灵机到 C 语言}{TuriMa}
% \entry{逻辑门、布尔运算}{LogicG}

% \chapter{C++ 标准库和其他常用库} \label{cpt_5d}
%---------------------------------

% \chapter{并行计算} \label{cpt_cpar}
%---------------------------------------

% \chapter{GUI} \label{cpt_5e}
%----------------------------------------
\entry{Visual C++ 的简单画图库 MatPlot}{MtPlot}

\chapter{Rust 语言} \label{cpt_222}
%-----------------------------------------
\entry{Rust 安装}{Rustin}

\chapter{数据结构与算法(C++)} \label{cpt_175}
%---------------------------------------
\entry{单链表}{List}
\entry{双链表}{DList}
\entry{栈}{stack}
\entry{单调栈}{Mstack}
\entry{队列}{queue}
\entry{单调队列}{Mqueue}
\entry{trie 树(字典树)}{trie}
\entry{B 树}{Btree}
\entry{堆}{heap}
\entry{二分}{BS}
\entry{并查集}{DSU}
\entry{哈希表}{hash}
\entry{字符串哈希}{Shash}
\entry{二叉树}{tree}
\entry{线段树}{STree}
\entry{广度优先搜索(BFS)}{BFS}
\entry{树与图的广度优先搜索}{BFS2}
\entry{深度优先搜索(DFS)}{DFS}
\entry{树与图的深度优先搜索}{DFS2}
\entry{单源最短路径}{SSSP}
\entry{快速排序}{qsort}
\entry{归并排序}{Msort}
\entry{堆排序}{hsort}
\entry{最小生成树}{Prim}
\entry{快速幂}{qmi}
\entry{双指针算法}{TP}
\entry{双指针法运用}{TPex1}
\entry{Floyd 判圈算法}{FloydC}
\entry{差分}{AD}
\entry{前缀和}{PS}
\entry{背包问题}{dp1}
\entry{树形动态规划}{dp2}
\entry{区间动态规划}{dp3}
\entry{状态压缩动态规划}{dp4}
\entry{A-star 算法}{Astar}
\entry{迭代加深}{ID}
\entry{IDA-star 算法}{IDAs}
\entry{树状数组}{BIT}
\entry{KD-Tree}{KDT}
\entry{Minimax 搜索和 Alpha-Beta 剪枝}{mmsab}
\entry{蒙特卡洛树搜索算法(理论)}{MCTS}
\entry{蒙特卡洛树搜索算法(实现 TicTacToe 机-机对战)}{MCTSci}

\chapter{编译器、调试器、构建工具} \label{cpt_compile}
%-------------------------------------
\entry{g++ 编译器创建静态和动态链接库(Linux 系统)}{gppLib}
\entry{调试 C++ 程序}{gdbcpp}

\chapter{常用开发工具} \label{cpt_tools}
%---------------------------------------
\entry{Git 和 GitHub 入门}{GitHub}
% \entry{Git 命令行教程}{Git0}
\entry{用 Git 备份文件夹}{gitBac}

\chapter{数据库} \label{cpt_db}
%---------------------------------------
\entry{SQL 入门语法}{SQLgrm}
\entry{SQL 入门(SQLite 为例)}{SQLint}
\entry{MySQL安装}{MSQ001}

\chapter{Linux 系统} \label{cpt_Linux}
%---------------------------------------
\entry{Linux 操作系统简介}{Linux0}

\chapter{计算机网络} \label{cpt_173}
%---------------------------------------
% \entry{OpenVPN 笔记}{OpnVPN}
\entry{开放系统互联基本参考模型(OSI模型)}{OSIRM}
\entry{香农定理}{ShanTh}

\chapter{计算机基础概念} \label{cpt_CS0}
%---------------------------------------
% 二进制, 八进制,十六进制
% 文件路径
\entry{计算机语言}{CpLgg}
\entry{原码、反码、补码}{InvCom}
\entry{正则表达式}{regex}

% \chapter{计算机原理} \label{cpt_PC}
%---------------------------------------
\entry{字长}{WordLe}
\entry{算术逻辑单元}{ALU}
\entry{随机存储器}{RAM}

\chapter{计算机图形学} \label{cpt_CG}
%---------------------------------------
\entry{计算机图形学}{cg}
\entry{图像坐标系}{imgFrm}
\entry{建模坐标系(局部坐标系)}{cgmc}
\entry{世界坐标系}{Worcod}
\entry{三维投影}{proj3D}
\entry{相机模型}{CamMdl}
\entry{由图像坐标计算射线}{mn2lin}
\entry{计算 3D 艺术画}{art3D}
\entry{相机的定位}{CamLoc}
\entry{长方形相机定位法}{RecCam}
\entry{解三棱锥 1 (Matlab)}{Pmd1}
\entry{地球测地线计算(Matlab)}{EGDT}
\entry{地球的公转(Matlab 绘图)}{georev}
\entry{足球顶点坐标的计算程序(Matlab)}{foot60}
\entry{刚体的几何运算(Matlab)}{RigBMa}

% \chapter{GPU 编程} \label{cpt_GPU}

\chapter{人工智能与机器学习} \label{cpt_186}
%---------------------------------------
\entry{人工智能导论}{IntrAI}
% \entry{概率密度函数与人工智能概论}{AIstat}
\entry{数据}{datast}
\entry{分类}{Class}
\entry{回归}{Regres}
\entry{欠拟合}{unfit}
\entry{过度拟合}{ovfit}
\entry{模型评估}{MoEva}
\entry{留出法}{holdou}
\entry{交叉验证}{CroVal}
\entry{线性回归}{LiGr}
\entry{精度(机器学习)}{Accu}
\entry{接收者操作特征曲线}{ROC}
\entry{混淆矩阵}{ConMat}
\entry{正则化(机器学习)}{Regula}
% \entry{人工神经元}{neuron}
\entry{神经网络}{NN}
\entry{激活函数}{ActFun}
\entry{梯度下降}{GraDec}
\entry{全连接网络}{FCNN}
\entry{卷积}{Conv}
\entry{卷积神经网络}{ConvNe}
\entry{深度学习 CNN 入门}{CNN1}
\entry{深度学习 CNN 入门 2}{CNN2}
\entry{残差网络}{ResNet}
\entry{稠密连接网络}{DensNe}
\entry{生成对抗网络}{GAN}
\entry{深度卷积生成对抗网络}{DCGAN}
\entry{条件生成对抗网络}{cGAN}
\entry{自编码器}{AE}
\entry{变分自编码器}{VAE}
\entry{去噪扩散概率模型}{DDPM}
\entry{条件去噪扩散概率模型}{cDDPM}
\entry{KL 散度(相对熵)}{KLD}
\entry{Jensen-Shannon 散度}{JSD}
\entry{交叉熵}{CroEnt}
\entry{强化学习}{rl}
\entry{ChatGPT}{ChGPT}
\entry{Transformers 库}{transL}

\chapter{高性能、多线程} \label{cpt_2f}
%---------------------------------------
\entry{C++ 原子变量原子操作笔记}{atomVO}

\chapter{其他} \label{cpt_187}
%---------------------------------------
\entry{信息熵简介}{InfoEn}
\entry{跨平台源代码编辑器安装}{Code}
\entry{LaTeX 结构简介}{latxIn}
\entry{LaTeX 常用符号对照表}{TeXSym}
\entry{安装使用 TeXlive}{TeXliv}
\entry{用 cp 或 rsync 增量备份文件夹}{rsyncB}
\entry{约定式提交}{CvCo}
\entry{Python RoboMaster EP 教程—RoboMaster SDK 安装}{PyRM1}
\entry{Python RoboMaster EP 教程—与机器人建立连接}{PyRM2}
\entry{Python RoboMaster EP 教程—通讯方式}{PyRM4}
\entry{Python RoboMaster EP 教程—初始化机器人}{PyRM3}
\entry{加密货币(数字货币)简介}{crypto}
\entry{使用数字货币钱包}{CryWal}


\part{科学计算} \label{prt_SciCal}
%======================================

\chapter{Matlab 语言} \label{cpt_167}
%---------------------------------------
\entry{【导航】科学计算}{NumPhy}
\entry{Matlab 简介}{Matlab}
\entry{Matlab 的变量与矩阵}{MatVar}
\entry{Matlab 的判断与循环}{MIfFor}
\entry{Matlab 的函数}{MatFun}
\entry{Matlab 画图}{MatPlt}
\entry{Matlab 的程序调试及其他功能}{MatOtr}
\entry{Matlab 解常微分方程组(ode45)}{MatODE}
\entry{Matlab 性能优化(profiling)}{MLprof}
\entry{Matlab 画箭头矢量场图}{MQuivr}
\entry{Matlab 球坐标中的分布图}{MatPol}
\entry{Matlab 符号计算和任意精度计算}{MatSym}
\entry{双精度和变精度浮点数测试(Matlab)}{FltMat}
\entry{Matlab 的稀疏矩阵}{MatSpa}
\entry{Julia 分形}{julias}
\entry{用 Matlab 手动提取图片中的曲线坐标}{plt2xy}
\entry{用 Matlab 制作 gif 动画}{MatGif}
\entry{用 Matlab 生成 mp4 视频}{MatMp4}
\entry{Matlab 的 Table 类型}{MatTab}
\entry{Matlab 调用 C/C++ 语言}{MatlaC}


\chapter{Mathematica 语言} \label{cpt_169}
%--------------------------------------
% 未完成 Wolfram Alpha
\entry{Mathematica 笔记}{Mma}
\entry{Mathematica 文件操作(笔记)}{mmaio}
\entry{Mathematica 脚本模式}{mmacmd}

\chapter{Python 科学计算} \label{cpt_scipy}
%---------------------------------------
\entry{SciPy 数值微分与积分}{SciPy}
\entry{SciPy 最小二乘法}{PyFit}
\entry{SciPy 求解常微分方程组的初值问题}{PyIVP}
\entry{Python 符号计算简介}{SymPy}

\chapter{C/C++ 科学计算} \label{cpt_172}
%---------------------------------------
\entry{数据结构:密矩阵}{MatSto}
\entry{数据结构:带对角矩阵}{BanDmt}
\entry{BLAS 简介}{BLAS}
\entry{SLISC 库简介}{SLISC}
\entry{SLISC 的使用和测试}{SLScom}
\entry{SLISC 的密矩阵类}{SliMat}
\entry{SLISC 矩阵的基本运算}{SliAri}
\entry{SLISC 密矩阵的切割}{SliCut}
\entry{SLISC 库代码生成}{SLSgen}
\entry{SLISC 的 Mcoo 矩阵类}{Mcoo}
\entry{SLISC 的 band 矩阵类}{SliBan}
\entry{SLISC 的 bit 操作工具}{SliBit}
\entry{SLISC 的文件读写}{Sfile}
\entry{SLISC 的计时工具}{SliTim}
\entry{SLISC 的 matt/matb 文件格式}{matb}
\entry{GNU Scientific Library}{GSL}
\entry{Arpack++2 大型本征方程库}{Arpkpp}
\entry{C++ 矩阵类的实现}{CppMat}

% \chapter{高精度计算与符号计算} \label{cpt_174}
%---------------------------------------
\entry{符号计算软件列表}{SymLib}

\chapter{Julia 编程基础(转载)} \label{cpt_177}
%---------------------------------------
\entry{初识 Julia}{JuC1S1}
\entry{Julia 安装和启动}{JuC1S2}
\entry{编写第一个 Julia 程序}{JuC1S3}
\entry{改进第一个 Julia 程序}{JuC1S4}
\entry{Julia 第 1 章小结}{JuC1S5}
\entry{Julia 的 REPL 环境及其用法}{JuC2S1}
\entry{Julia 程序包与环境配置}{JuC2S2}
\entry{Julia 项目的创建与引入}{JuC2S3}
\entry{Julia 第 2 章小结}{JuC2S4}
\entry{Julia 的变量与常量}{JuC3S0}
\entry{Julia 变量的定义}{JuC3S1}
\entry{Julia 变量的命名}{JuC3S2}
\entry{Julia 变量的类型}{JuC3S3}
\entry{Julia 常量}{JuC3S4}
\entry{Julia 第 3 章小结}{JuC3S5}
\entry{Julia 的类型系统}{JuC4S0}
\entry{Julia 类型系统概述}{JuC4S1}
\entry{Julia 的类型与值}{JuC4S2}
\entry{Julia 的两个特殊类型}{JuC4S3}
\entry{Julia 的三种主要类型}{JuC4S4}
\entry{Julia 第 4 章小结}{JuC4S5}
\entry{Julia 的数值与运算}{JuC5S0}
\entry{Julia 的数值类型}{JuC5S1}
\entry{Julia 整数}{JuC5S2}
\entry{Julia 浮点数}{JuC5S3}
\entry{Julia 的复数和有理数}{JuC5S4}
\entry{Julia 常用的数学运算}{JuC5S5}
\entry{Julia 数值类型的提升}{JuC5S6}
\entry{Julia 数学函数速览}{JuC5S7}
\entry{Julia 第 5 章小结}{JuC5S8}
\entry{Julia Unicode 字符}{JuC6S1}
\entry{Julia 字符}{JuC6S2}
\entry{Julia 字符串}{JuC6S3}
\entry{Julia 非常规的字符串值}{JuC6S4}
\entry{Julia 第 6 章小结}{JuC6S5}
\entry{Julia 参数化类型}{JuC7S0}
\entry{Julia 类型的参数化}{JuC7S1}
\entry{Julia 参数化的更多知识}{JuC7S2}
\entry{Julia 容器:元组}{JuC7S3}
\entry{Julia 第 7 章小结}{JuC7S4}
\entry{Julia 字典与集合}{JuC8S0}
\entry{Julia 索引与迭代}{JuC8S1}
\entry{Julia 标准字典}{JuC8S2}
\entry{Julia 集合}{JuC8S3}
\entry{Julia 通用操作}{JuC8S4}
\entry{Julia 第 8 章 小结}{JuC8S5}
\entry{Julia 容器:数组(上)}{JuC9S0}
\entry{Julia 类型}{JuC9S1}
\entry{Julia 数组的表示}{JuC9S2}
\entry{Julia 数组的构造}{JuC9S3}
\entry{Julia 数组的基本要素}{JuC9S4}
\entry{Julia 访问数组元素值}{JuC9S5}
\entry{Julia 修改数组元素值}{JuC9S6}
\entry{Julia 第 9 章 小结}{JuC9S7}
\entry{Julia 广播式的修改}{JuCAS1}
\entry{Julia 元素值的排序}{JuCAS2}
\entry{Julia 数组的拷贝}{JuCAS3}
\entry{Julia 数组的拼接}{JuCAS4}
\entry{Julia 数组的比较}{JuCAS5}
\entry{Julia 再说数组的构造}{JuCAS6}
\entry{Julia 第10章 小结}{JuCAS7}
\entry{Julia 流程控制}{JuCBS0}
\entry{Julia 最简单的代码块}{JuCBS1}
\entry{Julia if 语句}{JuCBS2}
\entry{Julia for 语句}{JuCBS3}
\entry{Julia while 语句}{JuCBS4}
\entry{Julia let 语句}{JuCBS5}
\entry{Julia 错误的报告与处理}{JuCBS6}
\entry{Julia 第11章 小结}{JuCBS7}
\entry{函数与方法}{JuCCS1}
\entry{Julia 中的函数}{JuCCS2}
\entry{Julia 函数的基本编写方式}{JuCCS3}
\entry{Julia 函数的参数}{JuCCS4}
\entry{Julia 函数的结果}{JuCCS5}
\entry{Julia 衍生方法}{JuCCS6}
\entry{Julia 函数的参数化}{JuCCS7}
\entry{Julia do 代码块}{JuCCS8}
\entry{Julia 第 12 章 小结}{JuCCS9}


\chapter{数值计算理论} \label{cpt_178}
%---------------------------------------
% 转载自专栏 https://www.zhihu.com/column/c_1226443594048942080
% 未完成
\entry{数值计算的误差}{NumErr}
\entry{计算机算数}{CmArit}
\entry{数值解线性方程组(入门)}{NLinEq}
\entry{数值解线性方程组(进阶)}{NLinE2}
\entry{数值解线性方程组(高级)}{NLinE3}
\entry{数值解常微分方程(入门)}{NordEq}

\chapter{常用算法} \label{cpt_179}
%---------------------------------------
\entry{二分法(Matlab 示例)}{Bisec}
\entry{多区间二分法}{MBisec}
\entry{冒泡法}{Bubble}
\entry{高斯消元法程序}{GauEli}
\entry{坐标定轴旋转程序(Matlab)}{turnM}
\entry{多项式插值}{InterP}
\entry{多项式插值 2}{PolyIP}
\entry{开根号的数值计算}{sqrtN}
\entry{拉格朗日插值法}{LagraP}
\entry{牛顿求根公式(牛顿迭代法)(Matlab)}{NWTNRT}
\entry{不动点迭代(Matlab 绘图)}{FPIPLT}
\entry{Nelder-Mead 算法}{NelMea}
\entry{最小二乘法拟合函数(Matlab)}{CurFit}
\entry{最小二乘法拟合多项式(Matlab)}{LSpoly}
\entry{数值积分(梯形法)}{NumInt}
\entry{稀疏矩阵}{SprMat}
\entry{函数求值}{SpcFun}
\entry{离散傅里叶变换}{DFT}
\entry{离散正弦变换}{DST}
\entry{Cholesky 分解}{Choles}
\entry{傅里叶变换的数值计算、快速傅里叶变换(Matlab)}{FFTft}
\entry{用傅里叶级数画曲线(Matlab)}{FFTdrw}
\entry{QR 分解}{QRdeco}
\entry{双共轭梯度法解线性方程组(BiCG)}{ConGra}
\entry{二项式定理(非整数幂)的数值验证}{BiNorM}
% \entry{条件数}{ConNum}

% \chapter{其他} \label{cpt_SciEtc}
%-------------------------------
\entry{常用科研软件列表}{SciApp}


% \part{数值计算} \label{prt_CalPhy}
%======================================

% \chapter{偏微分方程数值解} \label{cpt_182}
%---------------------------------------

\part{计算物理} \label{prt_CalQM}
%======================================

\chapter{微分方程数值解} \label{cpt_181}
%---------------------------------------
\entry{简谐振子受迫运动的简单数值计算}{SHOFN}
\entry{天体运动的简单数值计算}{KPNum0}
\entry{常微分方程(组)的数值解}{OdeNum}
\entry{中点法解常微分方程(组)}{OdeMid}
\entry{四阶龙格库塔法(Matlab)}{OdeRK4}
\entry{刚体转动数值模拟(Matlab)}{RBRNum}
\entry{陀螺的数值模拟(Matlab)}{TopNum}
\entry{洛伦兹吸引子}{LrzAct}

\chapter{经典力学的数值计算} \label{cpt_CalCM}
%---------------------------------------
\entry{开普勒问题的数值计算(Matlab)}{KepNum}
\entry{双摆的数值计算(Matlab)}{DbPend}
\entry{天体物理中 N 体问题的数值计算(Matlab)}{NbodyM}
\entry{N 体问题软件(天体物理)}{Nbody}
\entry{拉格朗日方程的数值解(Matlab)}{LagNum}
\entry{Verlet 算法的简单示例(Matlab)}{VRLT}
\entry{一维波动方程的简单数值解(Matlab)}{W1dNum}
\entry{二维波动方程的简单数值解(Matlab)}{wav2dN}
\entry{单缝衍射的模拟(Matlab)}{DiffrN}
\entry{双缝干涉的模拟(Matlab)}{DbSliN}
\entry{多普勒效应与音爆(Matlab 绘图)}{DPLDRW}

\chapter{电磁学的数值计算} \label{cpt_CalEM}
%-------------------------------------------
\entry{点电荷系统的静电场 (MATLAB绘图)}{MLTEPD}

\chapter{统计力学的数值计算} \label{cpt_CalSM}
%-------------------------------------
\entry{简单的扩散与混合模拟器(MATLAB)}{DIFMIX}
\entry{麦克斯韦—玻尔兹曼分布的简单数值模拟}{MBdisN}
\entry{麦克斯韦—玻尔兹曼分布的数值模拟}{MaxwD}

\chapter{一维薛定谔方程数值解} \label{cpt_183}
%---------------------------------------
\entry{试射法解一维有限深势阱束缚态数值解(Matlab)}{BndSho}
\entry{无限深势阱中的高斯波包数值计算(Matlab)}{ISWmat}
\entry{自由高斯波包的动画绘制(Matlab)}{FreeGs}
\entry{无限深势阱中的高斯波包模拟(Matlab)}{wvISW}
\entry{简谐振子中的高斯波包模拟(Matlab)}{SHOgs}
\entry{有限深方势阱束缚态程序(Matlab)}{FSWmat}
\entry{方势垒定态波函数程序(Matlab)}{FSBplt}
\entry{高斯波包的方势垒散射数值计算(Matlab)}{FSBsct}
\entry{一维薛定谔方程不稳定的差分法数值解(Matlab)}{TDSE1N}
\entry{Crank-Nicolson 算法解一维含时薛定谔方程(Matlab)}{CraNic}
\entry{一维有限深方势阱中的光电离模拟(Matlab)}{FSWpi}
\entry{非线性薛定谔方程的数值解法}{NLSEn}

\chapter{氢原子薛定谔方程数值解} \label{cpt_184}
%---------------------------------------
\entry{库仑函数程序(Matlab 和 Mathematica)}{FlCode}
\entry{氢原子波函数 Matlab 画图程序}{Hplot}
\entry{球谐函数数值计算}{YlmNum}
\entry{Gauss-Lobatto 积分}{GLquad}
\entry{FEDVR 网格}{FEDVR}
\entry{Lanczos 算法}{Lanc}
\entry{指数格点}{ExpGrd}
\entry{虚时间法求基态波函数}{ImgT}
\entry{氢原子球坐标薛定谔方程数值解}{HyTDSE}
\entry{氢原子球坐标薛定谔方程数值解 2}{HTDSE}

% \chapter{氦原子薛定谔方程数值解} \label{cpt_185}
%---------------------------------------
% \entry{氦原子数值解 TDSE}{HeTDSE}
% \entry{氦原子的 Berkeley-ECS 方法}{BerECS}
% \entry{氦原子波函数数值分析}{HeAnal}

\part{考研} \label{prt_GradEx}
%======================================

\chapter{数学} \label{cpt_mathTs}
%---------------------------------------
\entry{中国科学院 2012 年考研数学(甲)}{CASM12}
\entry{中国科学院 2013 年考研数学(甲)}{CASM13}
\entry{2010 年考研数学试题(数学一)}{PeeM10}
\entry{2011 年考研数学试题(数学一)}{PeeM11}
\entry{2012 年考研数学试题(数学一)}{PeeM12}
\entry{2013 年考研数学试题(数学一)}{PeeM13}
\entry{2014 年考研数学试题(数学一)}{PeeM14}
\entry{2015 年考研数学试题(数学一)}{PeeM15}
\entry{2016 年考研数学试题(数学一)}{PeeM16}
\entry{2017 年考研数学试题(数学一)}{PeeM17}
\entry{2018 年考研数学试题(数学一)}{PeeM18}
\entry{2019 年考研数学试题(数学一)}{PeeM19}
\entry{北京大学 2020 年强基计划招生考试数学试题}{PKU20}

\chapter{普通物理} \label{cpt_188}
%---------------------------------------
\entry{中国科学院 2003 年考研普通物理 A 卷}{ZKYP03}
\entry{中国科学院 2006 年考研普通物理 A 卷(甲)}{ZKYP06}
\entry{中国科学院 2012 年考研普通物理}{CAS12}
\entry{中国科学院 2013 年考研普通物理}{CAS13}
\entry{中国科学院 2014 年考研普通物理}{CAS14}
\entry{中国科学院 2017 年考研普通物理}{CAS17}
\entry{中国科学院 2018 年考研普通物理}{CAS18}
\entry{中国科学院 2020 年考研普通物理}{CAS20}

\entry{北京师范大学 2012 年考研普通物理}{BNU12}
\entry{北京师范大学 2013 年考研普通物理}{BNU13}
\entry{北京师范大学 2016 年考研普通物理}{BNU16}

\entry{中国科技大学 2013 年考研普通物理}{USTC13}
\entry{中国科技大学 2014 年考研普通物理}{USTC14}
\entry{中国科技大学 2015 年考研普通物理(B)}{USTC15}
\entry{中国科技大学 2016 年考研普通物理}{USTC16}

\entry{复旦大学 2004 量子真题}{FDU04}
\entry{复旦大学 2015 年考研普通物理}{FDU15}
\entry{四川大学 2006 年硕士物理考试试题}{CD06}

\entry{苏州大学 2005 年硕士考试试题}{SD05}
\entry{苏州大学 2008 年硕士考试试题}{SD08}
\entry{苏州大学 2011 年硕士物理考试试题}{SD11}
\entry{苏州大学 2012 年硕士物理考试试题}{SD12}
\entry{苏州大学 2015 年硕士物理考试试题}{SD15}

\entry{首都师范大学 2002 年硕士入学考试试题}{SSD02}
\entry{首都师范大学 2004 年硕士入学考试试卷}{SSD04}
\entry{首都师范大学 2005 年硕士考试试题}{SSD05}
\entry{首都师范大学 2010 年硕士考试试题}{SSD10}
\entry{首都师范大学 2003 年硕士入学物理考试试题(一)}{SSD103}
\entry{首都师范大学 2011 年硕士考试试题}{SSD11}
\entry{首都师范大学 2014 年硕士考试试题}{SSD14}
\entry{首都师范大学 2015 年物理硕士考试试题}{SSD15}
\entry{首都师范大学 2003 年硕士入学物理考试试题(二)}{SSD203}
\entry{首都师范大学 2001 年硕士研究生入学考试实体}{SSDPEE}

\entry{哈尔滨工业大学 2000 年硕士物理考试试题}{HGD00}
\entry{哈尔滨工业大学 2002 年硕士物理考试试题}{HGD02}
\entry{哈尔滨工业大学 2004 年硕士物理考试试题}{HGD04}

\entry{哈尔滨工业大学 2004 年 考研 量子力学}{HIT04}
\entry{哈尔滨工业大学 2005 年 考研 量子力学}{HIT05}
\entry{哈尔滨工业大学 2006 年 考研 量子力学}{HIT06}
\entry{哈尔滨工业大学 2008 年 考研 量子力学}{HIT08}
\entry{哈尔滨工业大学 2010 年 考研 量子力学}{HIT10}
\entry{哈尔滨工业大学 2013 年 考研 量子力学}{HIT13}

\entry{厦门大学 2002 年硕士物理考试试题}{SD02}
\entry{厦门大学 2008 年硕士入学考试物理试题}{XD08}
\entry{厦门大学 2011 年硕士入学物理考试试题}{XD11}
\entry{厦门大学 2012 年硕士入学物理考试试题}{XD12}

\entry{南京大学 2012 年考研普通物理}{NJU12}
\entry{南京大学 2013 年考研普通物理}{NJU13}
\entry{南京大学 2014 年考研普通物理}{NJU14}
\entry{南京大学 2015 年考研普通物理}{NJU15}
\entry{南京大学 2016 年考研普通物理}{NJU16}
\entry{南京大学 2017 年考研普通物理}{NJU17}
\entry{南京大学 2018 年考研普通物理}{NJU18}
\entry{南京大学 2019 年考研普通物理}{NJU19}

\entry{中山大学 2006 年研究生入学物理考试试题}{SY06}
\entry{中山大学 2019 年研究生入学物理考试试题}{SY19}


\entry{北京大学 1999 年 考研 普通物理}{PKU199}
\entry{北京大学 2000 年 考研 普通物理}{PKU200}
\entry{北京大学 2004 年 考研 普通物理}{PKU004}
\entry{北京大学 2007 年 考研 普通物理}{PKU007}
\entry{北京大学 2008 年 考研 普通物理}{PKU008}
\entry{北京大学 2009 年 考研 普通物理}{PKU009}
\entry{北京大学 2011 年 考研 普通物理}{PKU011}
\entry{北京大学 2012 年 考研 普通物理}{PKU012}

\entry{北京大学 1994 年 考研 固体物理}{PKU994}
\entry{北京大学 2000 年 考研 固体物理}{PKU020}
\entry{北京大学 2001 年 考研 固体物理}{PKU201}
\entry{北京大学 2009 年 考研 固体物理}{PKU209}

\entry{北京科技大学 2006 年考研普通物理 A 卷}{BKDP06}
\entry{华东师范大学 2013 年硕士入学考试物理试题}{HDSD13}
\entry{华东师范大学 2014 年考研物理考试试题}{HSD14}
\entry{华东师范大学 2015 年硕士研究生物理考试试题}{HSD15}

\entry{南京理工大学 2004 年 研究生入学考试试题 普通物理(B)}{NJU}
\entry{南京理工大学 2005 年 研究生入学考试试题 普通物理(B)}{NJU05}
\entry{南京理工大学 2006 年 研究生入学考试试题 普通物理(B)}{NIU06}
\entry{南京理工大学 2007 年 研究生入学考试试题 普通物理(A)}{NJU07}
\entry{南京理工大学 2007 年 研究生入学考试试题 普通物理(B)}{NJUB07}
\entry{南京理工大学 2008 年 研究生入学考试试题 普通物理(B)}{NJU08}

\entry{南京理工大学 普通物理 B(845)模拟五套卷 第一套}{NJUD1}
\entry{南京理工大学 普通物理 B(845)模拟五套卷 第二套}{NJUD2}
\entry{南京理工大学 普通物理 B(845)模拟五套卷 第三套}{NJUD3}
\entry{南京理工大学 普通物理 B(845)模拟五套卷 第四套}{NJUD4}
\entry{南京理工大学 普通物理 B(845)模拟五套卷 第五套}{NJUD5}



\chapter{量子力学} \label{cpt_189}
%---------------------------------------------
\entry{首都师范大学 2012 年 考研 量子力学 }{CNU12}

\entry{北京大学 1999 年 考研 量子力学}{PKU99}
\entry{北京大学 2000 年 考研 量子力学}{PKU00}
\entry{北京大学 2003 年 考研 量子力学}{PKU03}
\entry{北京大学 2004 年 考研 量子力学}{PKU04}
\entry{北京大学 2005 年 考研 量子力学}{PKU05}
\entry{北京大学 2007 年 考研 量子力学}{PKU07}
\entry{北京大学 2009 年 考研 量子力学}{PKU09}
\entry{北京大学 2015 年 考研 量子力学}{PKU15}
\entry{北京大学 2017 年 考研 量子力学}{PKU17}

\entry{郑州大学 2006 年 考研 量子力学}{ZZU06}
\entry{郑州大学 2008 年 考研 量子力学}{ZZU08}
\entry{郑州大学 2010 年 考研 量子力学}{ZZU10}
\entry{郑州大学 2003 年 考研 量子力学}{ZZU03}

\entry{厦门大学 2000 年 考研 量子力学}{XMU}
\entry{厦门大学 2001 年 考研 量子力学}{XMU01}
\entry{厦门大学 2002 年 考研 量子力学}{XMU02}
\entry{厦门大学 2005 年 考研 量子力学}{XMU05}
\entry{厦门大学 2006 年 考研 量子力学}{XMU06}
\entry{厦门大学 2007 年 考研 量子力学}{XMU07}
\entry{厦门大学 2008 年 考研 量子力学}{XMU08}
\entry{厦门大学 2009 年 考研 量子力学}{XMU09}
\entry{厦门大学 2011 年 考研 量子力学}{XMU11}
\entry{厦门大学 2012 年 考研 量子力学}{XMU12}
\entry{厦门大学 2013 年 考研 量子力学}{XMU13}
\entry{厦门大学 2014 年 考研 量子力学}{XMU14}

\entry{浙江大学 1998 年 考研 量子力学}{ZJU98}
\entry{浙江大学 1999 年 考研 量子力学}{ZJU99}
\entry{浙江大学 2000 年 考研 量子力学}{ZJU20}
\entry{浙江大学 2001 年 考研 量子力学}{ZJU01}
\entry{浙江大学 2002 年 考研 量子力学}{ZJU02}
\entry{浙江大学 2003 年 考研 量子力学}{ZJU03}
\entry{浙江大学 2004 年 考研 量子力学}{ZJU04}
\entry{浙江大学 2005 年 考研 量子力学}{ZJU05}
\entry{浙江大学 2006 年 考研 量子力学}{ZJU06}
\entry{浙江大学 2007 年 考研 量子力学}{ZJU07}
\entry{浙江大学 2008 年 考研 量子力学}{ZJU08}
\entry{浙江大学 2011 年 考研 量子力学}{ZJU11}
\entry{浙江大学 2012 年 考研 量子力学}{ZJU12}
\entry{浙江大学 2014 年 考研 量子力学}{ZJU14}

\entry{东南大学 2001 年 考研 量子力学}{SEU01}
\entry{东南大学 2002 年 考研 量子力学}{SEU02}
\entry{东南大学 2003 年 考研 量子力学}{SEU03}
\entry{东南大学 2004 年 考研 量子力学}{SEU04}
\entry{东南大学 2005 年 考研 量子力学}{SEU05}
\entry{东南大学 2006 年 考研 量子力学}{SEU06}
\entry{东南大学 2007 年 考研 量子力学}{SEU07}
\entry{东南大学 2008 年 考研 量子力学}{SEU}
\entry{东南大学 2009 年 考研 量子力学}{SEU09}
\entry{东南大学 2010 年 考研 量子力学}{SEU10}
\entry{东南大学 2011 年 考研 量子力学}{SEU11}
\entry{东南大学 2012 年 考研 量子力学}{SEU12}
\entry{东南大学 2013 年 考研 量子力学}{SEU13}
\entry{东南大学 2014 年 考研 量子力学}{SEU14}
\entry{东南大学 2015 年 考研 量子力学}{SEU15}
\entry{东南大学 2016 年 考研 量子力学}{SEU16}
\entry{东南大学 2017 年 考研 量子力学}{SEU17}
\entry{东南大学 2018 年 考研 量子力学}{SEU18}

\entry{华东师范大学 1996 年 考研 量子力学 }{ECNU96}
\entry{华东师范大学 1997 年 考研 量子力学 }{ECNU97}
\entry{华东师范大学 1998 年 考研 量子力学 }{ECNU98}
\entry{华东师范大学 2001 年 考研 量子力学 }{ECNU01}
\entry{华东师范大学 2005 年 考研 量子力学 }{ECNU05}
\entry{华东师范大学 2006 年 考研 量子力学 }{ECNU06}
\entry{华东师范大学 2007 年 考研 量子力学 }{ECNU07}
\entry{华东师范大学 2008 年 考研 量子力学 }{ECNU08}
\entry{华东师范大学 2009 年 考研 量子力学 }{ECNU09}
\entry{华东师范大学 2010  年 考研 量子力学 }{ECNU10}
\entry{华东师范大学 2011 年 考研 量子力学 }{ECNU11}
\entry{华东师范大学 2013 年 考研 量子力学}{ECNU13}
\entry{华东师范大学 2014 年 考研 量子力学 }{ECNU14}
\entry{华东师范大学 2015 年 考研 量子力学 }{ECNU15}

\entry{陕西师范大学 2005 年 考研 量子力学 }{SNNU05}
\entry{陕西师范大学 2006 年 考研 量子力学 }{SNNU06}
\entry{陕西师范大学 2012 年 考研 量子力学 }{SMMU12}
\entry{陕西师范大学 2013 年 考研 量子力学 }{SNNU13}
\entry{陕西师范大学 2015 年 考研 量子力学 }{SNNU15}
\entry{陕西师范大学 2016 年 考研 量子力学 }{SNNU16}

\entry{华中师范大学 2012 年 考研 量子力学 }{CCNU12}

\entry{天津大学 2011 年考研量子力学}{TJU11}
\entry{天津大学 2011 年考研量子力学答案}{TJU11A}
\entry{天津大学 2012 年考研量子力学}{TJU12}
\entry{天津大学 2012 年考研量子力学答案}{TJU12A}
\entry{天津大学 2013 年考研量子力学}{TJU13}
\entry{天津大学 2014 年考研量子力学}{TJU14}
\entry{天津大学 2014 年考研量子力学答案}{TJU14A}
\entry{天津大学 2015 年考研量子力学}{TJU15}
\entry{天津大学 2015 年考研量子力学答案}{TJU15A}
\entry{天津大学 2016 年考研量子力学}{TJU16}
\entry{天津大学 2016 年考研量子力学答案}{TJU16A}
\entry{天津大学 2017 年考研量子力学}{TJU17}
\entry{天津大学 2017 年考研量子力学答案}{TJU17A}

\entry{中国科学院大学 2014 年 考研 量子力学}{UCSA14}
\entry{中国科学院大学 2017 年考研 量子力学}{UCSA17}
\entry{中国科学院大学 2020 年考研 811 量子力学}{UCSA20}
\entry{中国科学院大学 2021 年考研 811 量子力学}{UCAS21}

\entry{复旦大学 1997 量子真题}{FDU97}
\entry{复旦大学 1998 量子真题}{FDU98}
\entry{复旦大学 1999 量子真题}{FDU99}
\entry{复旦大学 2000 量子真题}{FDU00}
\entry{复旦大学 2001 量子真题}{FDU01}
\entry{复旦大学 2006 量子真题}{FDU06}
\entry{复旦大学 2010 量子真题}{FDU10}
\entry{复旦大学 2014 量子真题}{FDU14}

\entry{南京理工大学 2006 量子真题}{NJUST2}
\entry{南京理工大学 2007 量子真题}{NJUST3}
\entry{南京理工大学 2008 量子真题}{NJUST4}
\entry{南京理工大学 2009 量子真题}{NJUST5}
\entry{南京理工大学 2011 量子真题}{NJUST1}
\entry{南京理工大学  2015 量子真题}{NJUST}

\entry{南京航空航天大学 2002 量子真题 }{NUAA}
\entry{南京航空航天大学 2004 量子真题 }{NUAA1}
\entry{南京航空航天大学 2005 量子真题 }{NUAA2}
\entry{南京航空航天大学 2007 量子真题 }{NUAA3}
\entry{南京航空航天大学 2012 量子真题 }{NUAA4}
\entry{南京航空航天大学 2013 量子真题 }{NUAA5}
\entry{南京航空航天大学 2014 量子真题 }{NUAA6}
\entry{南京航空航天大学 2015 量子真题 }{NUAA7}
\entry{南京航空航天大学 2016 量子真题 }{NUAA8}
\entry{南京航空航天大学 2004 量子真题答案 }{NUAA9}
\entry{南京航空航天大学 2007 量子真题答案 }{NUAA10}

\chapter{计算机科学与技术} \label{cpt_190}
%---------------------------------------------
\entry{2009 年计算机学科专业基础综合全国联考卷}{CSN09}
\entry{2010 年计算机学科专业基础综合全国联考卷}{CSN10}
\entry{2011 年计算机学科专业基础综合全国联考卷}{Na11}
\entry{2012 年计算机学科专业基础综合全国联考卷}{CSN12}
\entry{2013 年计算机学科专业基础综合全国联考卷}{CSN13}
\entry{2014 年计算机学科专业基础综合全国联考卷}{CSN14}
\entry{2015 年计算机学科专业基础综合全国联考卷}{CSN15}
\entry{2016 年计算机学科专业基础综合全国联考卷}{CSN16}
\entry{2017 年计算机学科专业基础综合全国联考卷}{CSN17}
\entry{2018 年计算机学科专业基础综合全国联考卷}{CSN18}
\entry{2019 年计算机学科专业基础综合全国联考卷}{CSN19}

\entry{北京航空航天大学 2011 年数据结构与 C 语言程序设计}{BA11}
\entry{北京航空航天大学 2012 年数据结构与 C 语言程序设计}{BAD12}
\entry{北京航空航天大学 2013 年数据结构与 C 语言程序设计}{BAD13}

\entry{中国传媒大学 2013 年数据结构与计算机网络}{DNCC13}
\entry{中国传媒大学 2014 年数据结构与计算机网络}{DNCC14}

\entry{上海海事大学 2011 年数据结构}{SMDS11}
\entry{上海海事大学 2012 年数据结构}{SMDS12}
\entry{上海海事大学 2013 年数据结构}{SMDS13}
\entry{上海海事大学 2014 年数据结构}{SMDS14}

\entry{浙江理工大学 2011 年数据结构}{ZSDS11}
\entry{浙江理工大学 2012 年数据结构}{ZSDS12}
\entry{浙江理工大学 2013 年数据结构}{ZSDS13}

\entry{中山大学 2011 年913专业基础(数据结构)考研真题}{SYDS11}
\entry{中山大学 2012 年913专业基础(数据结构)考研真题}{SYDS12}
\entry{中山大学 2014 年913专业基础(数据结构)考研真题}{SYDS14}
\entry{中山大学 2015 年918专业基础(数据结构)考研真题}{SYDS15}

\entry{暨南大学 2011 年信息科学技术学院830数据结构考研真题}{DSJN11}

\entry{沈阳航空航天大学 2012 年818/数据结构专业综合考研真题}{SHDS12}

\chapter{其他} \label{cpt_19a}
%---------------------------------------------
\entry{南开 2015 电动力学真题}{NKU}
\entry{兰州大学 2017 热力学统计物理真题}{LZU}
\entry{某热力学统计物理考研试题}{RUC12}
\entry{兰州大学 2016 硕士研究生招生初试试题}{LUZ}


\part{搜狗科学百科镜像} \label{prt_sogou}
%=====================================
\chapter{目录} \label{cpt_sg1}
%---------------------------------------
\entry{【目录】搜狗科学百科镜像}{sogou}


\part{附录} \label{prt_Append}
%======================================
\chapter{附录} \label{cpt_191}
%---------------------------------------
\entry{小时百科符号与规范}{Conven}
\entry{希腊字母表}{GreekL}
\entry{常见物理量}{PhyQty}
\entry{国际单位制}{SIunit}
\entry{物理学常数}{Consts}
\entry{国际单位制词头}{UniPre}
\entry{量类和单位}{QCU}
\entry{量类的延拓}{QCC}
\entry{单位制和量纲}{USD}
\entry{量纲与单位制}{dimunt}
\entry{量纲式}{DIMF}
\entry{现象类}{PHEC}
\entry{量纲空间}{DimS}
\entry{小时百科参考项目}{WuliRf}
\entry{小时百科图标}{xwLogo}
\entry{【目录】小时百科项目}{XStoc}


% \part{隐藏内容} \label{prt_Hidden}
% %======================================

% \chapter{数学} \label{cpt_5f}
% %---------------------------------------
% \entry{Quaternion and Rotation}{QuaRot}

% \chapter{力学} \label{cpt_6a}
% %---------------------------------------
% \entry{重心}{CenG}
% \entry{经典系统的线性响应理论}{LiReC1}

% \chapter{光学} \label{cpt_6b}
% %---------------------------------------
% \entry{透镜的主平面和节平面}{LnsPln}

% \chapter{电动力学} \label{cpt_6c}
% %---------------------------------------
% \entry{用狄拉克 delta 函数表示点电荷的散度}{CEfDiv}

% \chapter{量子力学} \label{cpt_6d}
% %---------------------------------------
% \entry{伴随算符}{adjoin}
% \entry{自伴算符、厄米算符}{HerOp}

% \chapter{计算机} \label{cpt_6e}
% %---------------------------------------
% \chapter{其他} \label{cpt_6f}
% \entry{二极管}{Diode}
% \entry{偶极子近似(量子力学)}{DipApr}
% \entry{能均分定理}{EqEng}
% \entry{单缝的夫琅禾费衍射}{FD}
% \entry{分布函数的数值拟合}{FitPdf}
% \entry{有限深势阱中的双粒子}{FSWtwo}
% \entry{函数空间}{FunSpc}
% \entry{伽利略变换}{GaliTr}
% \entry{万有引力和天体运动}{Grav0}
% \entry{同构}{homomo}
% \entry{一维和二维氢原子模型势能}{Hy1D2D}
% \entry{L2 函数空间}{L2FunS}
% \entry{刘维尔定理(热力学与统计物理)}{LiouTh}
% \entry{用 Filebrowser 搭建个人网盘}{Fbrows}
% \entry{命题及其表示法}{Propos}
% \entry{量子霍尔效应}{QHallE}
% \entry{给高中生的量子力学简介}{QMIntr}
% \entry{反射和折射、斯涅尔定律}{Reflec}
% \entry{环形电流的磁场}{RingB}
% \entry{TDSE Open Boundary Condition}{SEopBC}
% \entry{心脏螺旋波}{spiral}
% \entry{【目录】个人目录}{t}
% \entry{Tensor}{TestT}
% \entry{矢量积分}{VecInt}
% \entry{Wolfram Alpha 简介}{WolfAl}
% \entry{半波损失}{WvLost}

\bibli
\end{document}
