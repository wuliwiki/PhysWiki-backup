% 分离性公理成立的充要条件
% keys 分离性公理|充要条件
% license Usr
% type Tutor

\pentry{分离性\nref{nod_Topo5}}{nod_cf78}
拓扑空间是较度量空间更一般的对象,很多度量空间的概念可以拓广到拓扑空间。然而正因为这一一般性,拓扑空间会出现本质上不同于度量空间的情况。例如,有限点集可能不是闭的,收敛序列的极限点可能不唯一等等。为了获得和度量空间更相近的空间来,需要添加一些补充条件。分离性公理\upref{Topo5}就是人们提出的一类重要条件,它衡量了拓扑空间中点的分离程度。本节将给出其中几个分离性公理成立的充要条件。

\subsection{分离性公理成立的充要条件}
第一分离性公理 $T_1$ 是指对拓扑空间中的任意两个不同点,每一点都存在不包含另一点的邻域。这样的拓扑空间被称为\textbf{ $T_1$ 空间}。
\begin{theorem}{$T_1$ 分离性公理成立的充要条件}
第一分离性公理 $T_1$ 成立的充要条件是:空间中任意单点集都是闭的。
\end{theorem}
\textbf{证明:}
\textbf{必要性:}假设第一分离性公理成立,设 $x\neq y$,则存在 $y$ 的邻域 $O_y$,使得
\begin{equation}
x\notin O_y~.
\end{equation}
由闭包的定义,$y\notin [x]$\footnote{为了方便,本文约定单点集 $\{x\}$ 直接写为 $x$。},即任意不同于 $x$ 的点都不在 $[x]$ 中,因此 $[x]=x$。因此任意单点集都是闭的。

\textbf{充分性:}假设任意单点集都是闭的。设 $x\neq y$,则

\textbf{证毕!}







\textbf{证明:}


\textbf{证毕!}









\textbf{证明:}


\textbf{证毕!}




