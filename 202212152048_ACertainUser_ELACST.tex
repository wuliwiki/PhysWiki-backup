% 本构关系(弹性力学)

\begin{issues}
\issueDraft
\end{issues}
%工程量似乎有点巨大...先占位

之前我们已经分别探讨了应力\upref{STRESS}与应变\upref{Strain}.但是如何联系二者起来呢?二者的关联正由材料的本构关系描述,即力如何使材料变形.

以下是各向同性的线弹性材料的本构关系.本构关系一共包括$6$个独立的方程.需要两个物理量以构建本构关系,例如杨氏模量$E$与泊松比$\nu$.

\begin{equation}
\sigma_{ij}=\frac{E}{1+\nu}\varepsilon_{ij}+\delta_{ij}\frac{\nu E}{(1+\nu)(1-2\nu)}\sum_k\varepsilon_{kk}\qquad i,j=1,2,3
\end{equation}
或
\begin{equation}
\varepsilon_{ij}=\frac{1+\nu}{E}\sigma_{ij}-\delta_{ij}\frac{\nu}{E}\sum_k\sigma_{kk} \qquad i,j=1,2,3
\end{equation}

一个各向同性的线弹性材料
