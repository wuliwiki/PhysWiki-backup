% 线性映射 线性变换

\pentry{矢量空间\upref{LSpace}}

\subsection{线性映射}
\begin{definition}{线性映射}\label{LinMap_def1}
给定域$\mathbb{F}$上的线性空间$V$和$U$.如果有映射$f:V\rightarrow U$满足,对于任意的向量$\bvec{v}_1, \bvec{v}_2\in V$和标量$a_1, a_2\in\mathbb{F}$,都有$f(a_1\bvec{v}_1+a_2\bvec{v}_2)=a_1f(\bvec{v}_1)+a_2f(\bvec{v}_2)$,那么称$f$是$V$到$U$的一个\textbf{线性映射(linear map)}.
\end{definition}


\autoref{LinMap_def1} 的内涵比看上去广一些.对于任意的一组有限个向量$\{\bvec{v}_i\}\subseteq V$和一组对应的标量$\{a_i\}\subseteq\mathbb{F}$,都有$f(\sum_i a_i\bvec{v}_i)=\sum_i a_if(\bvec{v}_i)$.

如果$\{\bvec{e}_i\}_{i=1}^n$是$V$的一组基,那么任意$\bvec{v}\in V$都可以唯一地表示为$\bvec{v}=\sum_i c_i\bvec{e}_i$的形式,其中$c_i\in\mathbb{F}$.这样,由于线性性,$f(\bvec{v})=\sum_ic_if(\bvec{e}_i)$.也就是说,只需要知道了基向量被映射到哪里,也就可以计算出任意向量映射到哪里.于是和线性函数一样,确定一个线性映射的时候,我们最多只能自由选择基向量映射到哪里,只不过这里的函数值不再是数字,而是$U$中的向量.

在矢量空间的表示\upref{VecRep}中我们还会看到,选定两个空间的基以后,一个线性映射也可以看成是多个线性函数的排列,因此线性映射和线性函数性质很相似.

线性函数是一种特殊的线性映射, $V$ 上的所有线性函数组成的矢量空间叫做对偶空间\upref{DualSp}.

\subsection{线性变换}
线性空间$V$上的一个线性变换$T$,是指把$V$中的任意向量$\bvec{v}$映射为另一个向量$T\bvec{v}$的\textbf{操作}.这个操作满足线性性,因此被称作线性变换;线性性的好处在于我们只需要讨论基向量被映射到哪里,就知道了任何向量会被映射到哪里.
\addTODO{提一下线性和线性映射的区别}

用线性映射\upref{LinMap}的语言来说,线性变换就是$V$到自身的一个线性映射.

\begin{definition}{线性变换}
给定线性空间$V$,如果$T$是$V$到$V$上的线性映射,那么称$T$是一个$V$上的\textbf{线性变换(linear transformation)}.$T$将$\bvec{v}\in V$映射为$T\bvec{v}$.
\end{definition}
