% 信息科学(综述)
% license CCBYSA3
% type Wiki

本文根据 CC-BY-SA 协议转载翻译自维基百科\href{https://en.wikipedia.org/wiki/Information_science}{相关文章}。

\begin{figure}[ht]
\centering
\includegraphics[width=10cm]{./figures/15c7d956dfa4ab1f.png}
\caption{从元数据领域获取洞察的各种方法论方法的可视化} \label{fig_INCE_1}
\end{figure}
信息科学[1][2][3] 是一门主要关注信息的分析、收集、分类、处理、存储、检索、传输、传播和保护的学术领域。[4] 该领域内外的从业者不仅研究知识在组织中的应用与使用,还研究人与组织以及任何现有信息系统之间的互动,旨在创建、替代、改进或理解信息系统。

从历史上看,信息科学与信息学、计算机科学、数据科学、心理学、技术、文献学、图书馆学、医疗保健和情报机构等领域有关联。[5] 然而,信息科学也包括了诸如档案学、认知科学、商业、法律、语言学、博物馆学、管理学、数学、哲学、公共政策和社会科学等多种领域的内容。
\subsection{基础} 
\subsubsection{范围与方法} 
信息科学专注于从利益相关者的角度理解问题,然后根据需要应用信息和其他技术。换句话说,它首先解决的是系统性问题,而不是系统内部的单个技术组件。在这一点上,可以将信息科学视为对技术决定论的回应,技术决定论认为技术“按照自身的规律发展,发挥自身的潜力,仅受到可用物质资源和开发者创造力的限制。因此,它必须被视为一个自治系统,控制并最终渗透社会的所有其他子系统。”[6]

许多大学拥有专门研究信息科学的学院、部门或学校,而众多信息科学学者则在传播学、医疗保健、计算机科学、法学和社会学等学科中工作。一些机构已成立信息学院联盟(参见信息学院列表),除此之外,许多其他机构也具有全面的信息研究方向。

在信息科学领域,2013年时的当前问题包括:
\begin{itemize}
\item 科学中的人机交互
\item 协同软件
\item 语义网
\item 价值敏感设计
\item 迭代设计过程
\item 人们生成、使用和寻找信息的方式
\end{itemize}
\subsubsection{定义}  
“信息科学”这一术语的首次已知使用是在1955年。[7] 信息科学的早期定义(追溯到1968年,美国文献学会将其名称更改为美国信息科学与技术学会的那一年)指出:

信息科学是研究信息的属性和行为、控制信息流动的力量,以及处理信息以实现最佳可访问性和可用性的方法的学科。它关注与信息的生成、收集、组织、存储、检索、解释、传输、转化和利用相关的知识体系。这包括自然和人工系统中信息表现的真实性,使用编码进行高效信息传递,以及研究信息处理设备和技术,如计算机及其编程系统。它是一门跨学科的科学,源自并与数学、逻辑学、语言学、心理学、计算机技术、运筹学、图形艺术、通讯、管理以及其他类似领域相关。它既有纯科学的成分,探索这一主题而不考虑其应用,也有应用科学的成分,开发服务和产品。(Borko 1968,第3页)[8]

\textbf{相关术语}

一些作者将信息学(Informatics)视为信息科学的同义词。这种说法尤其在与A. I. 米哈伊洛夫(A. I. Mikhailov)及其他苏联作者在1960年代中期提出的概念相关时尤为常见。米哈伊洛夫学派将信息学视为与科学信息研究相关的学科。[9] 由于信息学领域快速发展且跨学科的特点,使得信息学的定义很难精确定义。依赖于用于从数据中提取有意义信息的工具特性的定义,正在信息学的学术项目中出现。[10]

区域差异和国际术语使得这个问题更加复杂。有些人[哪一些?]指出,今天所称为“信息学”的许多内容曾经被称为“信息科学”——至少在医学信息学等领域中是这样。例如,当图书馆学家开始使用“信息科学”一词来指代他们的工作时,“信息学”这一术语便应运而生:
\begin{itemize}
\item 在美国,信息学作为计算机科学家为区分自己与图书馆学的工作所做的回应;
\item 在英国,信息学作为研究自然信息处理系统以及人工或工程化信息处理系统的科学术语。
\end{itemize}
另一个被讨论为“信息研究”的同义词的术语是“信息系统”。Brian Campbell Vickery的《信息系统》(1973)将信息系统纳入信息科学(IS)范畴。[11] 另一方面,Ellis、Allen和Wilson(1999)提供了一项文献计量学调查,描述了“信息科学”和“信息系统”这两个不同领域之间的关系。[12]