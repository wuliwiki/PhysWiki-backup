% 换位子群
% keys 换位子|commutator|导子群|交换子|交换子群|群论|正规子群

\pentry{正规子群\upref{NormSG}}


\begin{definition}{换位子}\label{CmtGrp_def1}
给定群$G$中的任意元素$a$和$b$,则称元素$b^{-1}a^{-1}ba$为一个\textbf{换位子(commutator)},或者\textbf{导子}.
\end{definition}

给\autoref{CmtGrp_def1} 中的概念取换位子这一名称,是因为它作用在两个元素的乘积上可以交换其乘积顺序:$(ab)(b^{-1}a^{-1}ba)=ba$.它可以用来更细致地刻画群的交换性,而不仅仅是粗糙的“交换\不交换”的描述.具体来说,我们需要用\textbf{换位子群}来描述群的交换性:

\begin{definition}{换位子群}
给定群$G$.$G$的全体换位子所\textbf{生成}的子群,称为$G$的\textbf{换位子群(commutator group)},记为$G'$或$[G, G]$.
\end{definition}

注意,全体换位子构成的集合,通常不是一个群,因此换位子群并不仅仅是换位子构成的,而是换位子以及换位子的乘积构成的.比如说,形如$b^{-1}a^{-1}bad^{-1}c^{-1}dc$的元素可能不是一个换位子,但它是换位子群中的元素.

\begin{theorem}{}
给定群$G$,则$G'\triangleleft G$.
\end{theorem}

\textbf{证明}:

换位子群中的元素都可以写成换位子相乘的形式:$x_1x_2x_3\cdots x_n$,其中各$x_i$为换位子.因此,要证明$g^{-1}G'g=G'$,只需要证明$g^{-1}x_1g$是一个换位子即可,这样$g^{-1}x_1x_2x_3\cdots x_ng=g^{-1}x_1gg^{-1}x_2gg^{-1}x_3gg^{-1}\cdots gg^{-1}x_ng$也是换位子的乘积,故在换位子群中.

对于任意$g\in G$,都有
\begin{equation}
\begin{aligned}
g^{-1}b^{-1}a^{-1}bag&=g^{-1}b^{-1}gg^{-1}a^{-1}gg^{-1}bgg^{-1}ag\\
&=(g^{-1}bg)^{-1}(g^{-1}ag)^{-1}(g^{-1}bg)(g^{-1}ag)
\end{aligned}
\end{equation}
也是一个换位子.

\textbf{证毕}.

换位子群刻画交换性的方式由以下\autoref{CmtGrp_the1} 描述:

\begin{theorem}{}\label{CmtGrp_the1}

\end{theorem}














