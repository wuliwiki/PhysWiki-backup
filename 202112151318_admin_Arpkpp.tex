% Arpack++2 大型本征方程库

\begin{issues}
\issueDraft
\end{issues}

\pentry{C++ 基础\upref{Cpp0}}

\href{https://www.caam.rice.edu/software/ARPACK/}{Arpack} 最初是一个 Fortran 语言编写的, 用于求解大型本征方程的程序库.
\begin{itemize}
\item 它可以求出指定值附近的若干本征值和本征矢.
\item 它可以不需要提供矩阵本身而是提供矩阵—矢量乘法. 也就是你只需要提供一个函数 \verb|v = F(u)| 给 Arpack 即可, 其中 \verb|v| 就是矩阵乘以 \verb|u| 的结果.
\end{itemize}

\href{http://www.ime.unicamp.br/~chico/arpack++/}{Arpack++} 是 Arpack 的 C++ 版本程序库, 但实际上只是一个 C++ 接口. 官网提供的版本已经多年没有维护(最后更新于 2000 年, 版本 1.2), 用目前的 \verb|g++| 编译器无法正常编译. 麻省理工在 GitHub 上维护了一个新版本: \href{https://github.com/m-reuter/arpackpp}{Arpack++2}. 本文使用的是当前最新的 release 2.3.

\subsection{Ubuntu/Debian 安装}
Ubuntu/Debian 中可以直接用 apt 安装, 如果没有装 gfortran 要先 \verb|apt install gfortran|. 然后安装 \verb|apt install libarpack++2-dev|, 其实我们只需要它安装的二进制文件和 dependency. 注意 dependency 中会自动安装 OpenBlas, 如果已经装了其他版本的 Blas 如 CBLAS 可以将其手动卸载\footnote{OpenBlas 和 CBLAS 的接口不完全相同, 前者的函数使用 \lstinline|double *| 输入复数, 而后者使用 \verb|void *|, 兼容性更强. 例如本书的 SLISC0 库用 OpenBlas 就会编译出错(待更新).}: \verb|sudo apt purge libopenblas-dev|.

安装好后, 可以用 \verb|dpkg -L libarpack++2-dev| 查看安装的文件, 其中头文件和文档可以删除, 使用 github 上的头文件和文档. doc 中的 pdf 文档是重要的参考, 其中第二章第一个例子在 \verb|examples/areig/nonsym/simple.cc| 中, 在该目录运行 \verb|make simple| 即可编译. 用 \verb|./simple| 运行.

\subsection{使用}
第三章讲解了 Classes that require user-defined matrix-vector products, 即不直接提供矩阵而是提供矩阵—矢量乘法.