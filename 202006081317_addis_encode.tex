% 字符编码

\subsection{文本文件}
我们这里讨论的是\textbf{文本文件(text file)}, 也就是 Windows 中为人熟知的 txt 拓展名文件. 文本文件只能储存字符, 也就是说文件中的信息只能是 “有几个字符” 以及 “每个字符是什么”. 所以文本文件本质上不包括字体,字号,下划线等等的其他信息. 一个文件是否是文本文件是由它的储存格式决定的, 而不是由拓展名决定的. 一些其他的文件拓展名如 html, xml, md, 以及大部分变成语言的代码文件都是文本文件.

我们把所有不是文本文件的文件统称为二进制文件(binary file), 因为从原理上来说任何文件都是以二进制的形式储存的. 例如 Word 文档保存的 doc 或 docx 拓展名文件就是二进制文件, 因为里面用其特定的格式储存了许多其他信息.  又例如图片文件 jpg, png, 视频文件 mp4 等也都是二进制文件.

文本文件的一个基本的问题是, 字符以什么形式保存? 从概念上来说, 每个字符对应一个整数, 我们把对应的规则叫做\textbf{编码(encoding)}. 例如在著名的 ascii 编码中, 128 个字符被一一对应到 0-127 的整数, 这些字符包括大写和小写字母, 数字, 常见标点, 以及一些格式上的符号如空格, 换行符, 制表符等. ascii 编码可以满足通常的英语写作需求, 但显然不支持其他语言如中文. 下文会看到中文的文本文档通常以 UTF-8 或者 GB2312 编码(常见于中文版的 Windows)储存. 遗憾的是, 一般来说文本文档中不会声明使用什么编码, 所以如果储存和打开文件的默认编码方式不一致, 就会导致出现乱码.

这里我们使用 VScode 编辑器为例进行介绍.

当我们在在中版的 Windows 操作系统中新建一个 txt 并输入一些中文的时候, 默认编码是 GB2312, GB 代表“国标”. 但 Linux 或 MacOS 中的文本文件

\subsection{CR 和 LF}
Windows 默认使用 CRLF, MacOS 默认用 CR, Unix 系统一般用 LF. 高级一些的编辑器会自动检测

\subsection{Unicode}
Unicode 支持许多语言.
