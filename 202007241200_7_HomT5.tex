% 基本群的计算
\pentry{基本群\upref{HomT3},群的自由积\upref{FrePrd}}


虽然把各阶同伦群都考虑进去以后可以很详细地刻画空间的伦型,但是高阶同伦群大多极其难计算.本节简单介绍一阶同伦群,即基本群的计算方法.

\subsection{基本群计算定理}

我们列举一些方便基本群计算的定理如下.

\begin{theorem}{积空间的基本群}
给定拓扑空间$X$和$Y$,则$\pi_1(X\times Y)=\pi_1(X)\times\pi_1(Y)$.
\end{theorem}
%需要附证明吗?直观来看这个定理很显然,证明无非就是严格把直觉描述清楚.

\begin{theorem}{Seifert-van Kampen定理}
设拓扑空间$X$可以被它的两个开集$U_1$和$U_2$覆盖,即$X=U_1\cup U_2$;若$U_1$,$U_2$和$U_1\cap U_2$都是道路连通空间,取$U_1\cap U_2$中一个点作为基点来构造各空间的基本群.设$f_i:U_1\cap U_2\rightarrow U_i$为恒等嵌入,即$\forall x\in U_1\cap U_2, f_i(x)=x\in U_i$.如果用$f_i$来定义$\pi_1(U_1\cap U_2)$到$\pi_(U_i)$上的同态,那么有:$\pi_1(X)=\pi_1(U_1)*_{\pi_1(U_1\cap U_2)}\pi_1(U_2)$.
\end{theorem}

Seifert-van Kampen定理难以简洁表达,不过如果能充分理解\textbf{群的自由积}\upref{FrePrd},应该容易理解该定理.不过,我们可以考虑该定理的弱化版本,此版本用处也很广泛:

\begin{theorem}{弱化版Seifert-van Kampen定理}
设拓扑空间$X$可以被它的两个开集$U_1$和$U_2$覆盖,并且$U_1\cap U_2$是单连通(\autoref{HomT3_ex1}~\upref{HomT3})的,那么有$\pi_1(X)=\pi_1(U_1)*\pi_1(U_2)$.
\end{theorem}

\begin{theorem}{锥空间的基本群}
设$X$是任意非空的拓扑空间,则$\pi_1(\widetilde{C}X)=\{e\}$,即为只有一个元素的平凡群.
\end{theorem}

锥空间的基本群总是平凡群,也就是说所有回路都是保基点同伦的.这是因为,首先锥空间一定是道路连通空间,因此基点可以任意选择,不妨选为锥顶点;其次,任何一条回路都可以通过各点沿着锥空间的$I$分量连续地收缩到锥顶点上,从而和恒等于锥顶点的回路同伦.



\subsection{基本群计算实例}

\pentry{复数\upref{CplxNo},覆叠空间\upref{CovTop}}

\begin{example}{$S^1$的基本群}
将$S^1$看成复平面上的单位圆$\{\E^{2\pi\I t}\in\mathbb{C}|t\in\mathbb{R}\}=\{x+y\I\in\mathbb{C}|x=\cos{2\pi t}, y=\sin{2\pi t}\}$.取通常的实度量空间$\mathbb{R}$,建立映射$p:\mathbb{R}\rightarrow S^1$,其中$p(t)=\E^{2\pi\I t}$,则$p$是一个覆叠映射.

把$S^1$看成$\mathbb{R}$的商拓扑空间,其中等价关系$\sim$为:$x\sim y\iff \abs{x-y}\in\mathbb{Z}$,即把实数轴绕到圆上.对于圆周上任意一个点$\E^{2\pi\I t_0}$,可以取典范邻域$U_{t_0}=\{\E^{2\pi\I t}|t\in(t_0-1/4, t_0+1/4)\}$.这个典范邻域中$1/4$的选择是任意的,换成任何小于$1/2$的正数也可以.

取$S^1$的基点为$p(0)=1$,设$S^1$中有一条道路$f:I\rightarrow S^1$.我们现在尝试把$f$\textbf{提升}为$\mathbb{R}$中的道路$\tilde{f}:I\rightarrow\mathbb{R}$,使得$f=p\cdot\tilde{f}$.
\end{example}

\begin{example}{和$S^1$有关的空间}
由于$\pi_1(S^1)=\mathbb{Z}$,结合以上定理,我们可以轻松计算如下空间的基本群:
\begin{itemize}
\item 甜甜圈空间可以表示为$S^1\times S^1$,因此其基本群为$\pi_1(S^1\times S^1)=\mathbb{Z}\times\mathbb{Z}$.
\item 高维甜甜圈$S^1\times\cdots\times S^1$的基本群是$\mathbb{Z}\times\cdots\times\mathbb{Z}$.
\item 二阶圈图$S^1\vee S^1$(\autoref{Topo9_def1}~\upref{Topo9})的基本群是$\mathbb{Z}*\mathbb{Z}$.
\item 高阶圈图$S^1\vee\cdots\vee S^1$的基本群是$\mathbb{Z}*\cdots*\mathbb{Z}$.
\end{itemize}

\end{example}



