% 多项式插值 2
% license Xiao
% type Tutor

\begin{issues}
\issueTODO
\issueOther{和拉格朗日插值法\upref{LagraP} 是否需要合并?}
\end{issues}
\pentry{范德蒙德行列式\upref{Sample}}
\subsection{什么是插值}
插值法,就是在函数表达式未知的情况下,通过已知点处的函数值来估计未知点处函数值的方法。\begin{definition}{插值函数}
已知函数 $f(x)$ 在 $\left[a,b\right] $ 上有定义,且已经测得在 $x_0,x_1,\dots, x_n$ 处的函数值为
\begin{equation}
y_i = f(x_i) \qquad (i = 1,2, \dots, n)~.
\end{equation}
如果存在一个简单易算的函数 $p(x)$, 使得
\begin{equation}
\label{eq_PolyIP_1}
p(x_i) = f(x_i)\qquad (i = 1,2, \dots, n)~,
\end{equation}
则称 $p(x)$ 是 $f(x)$ 的插值函数。
\end{definition}

插值法是数值计算中一种常用且基础的方法,在数据挖掘、数学建模等诸多领域都有广泛的应用。

\subsection{多项式插值}
当\autoref{eq_PolyIP_1} 中的简单函数 $p(x)$ 是多项式函数时,该插值方法就叫\textbf{多项式插值}, $p(x)$ 被称为\textbf{插值多项式}。当有 $n$ 个插值点时,插值多项式的次数一定\textbf{小于等于} $n$ 次。
\begin{theorem}{插值多项式的唯一性定理}\label{the_PolyIP_1}
满足上述条件的多项式 $p(x)$ 存在唯一。

\end{theorem}
\textbf{证明:} 由Vandermonde矩阵为非奇异矩阵易证。

由 \autoref{the_PolyIP_1} 可知,对一组给定的插值点以及对应的函数值,由插值法得到的多项式\textbf{存在且唯一}。因此,下面介绍的两种插值方法只是推导不同,但其最终得到的插值多项式是一致的。
\subsubsection{Lagrange插值}

\begin{definition}{Lagrange基函数}
设 $l_k(x)$是 $n$ 次多项式,在节点 $x_0, x_1,\cdots,x_n$ 上满足:
\begin{equation}
l_k(x_i) = 
\begin{cases}
1\ &(i = k)\\
0 \ &(i \neq k)~,
\end{cases}
\end{equation}
则称 $l_k(x)$ 为节点 $x_0,x_1,\cdots,x_n$ 上的 $n$ 次 Lagrange插值基函数。
\end{definition}
根据定义,由构造法易得:
\begin{equation}
l_k(x) = \frac{(x-x_0)\cdots(x-x_{k-1})(x-x_{k+1})\cdots (x-x_n)}{(x_k-x_0)\cdots(x_k-x_{k-1})(x_k-x_{k+1})\cdots(x_k-x_n)}~.
\end{equation}
已知Lagrange基函数,构造插值函数的过程就十分简单了。由于我们已知函数在 $x_0,x_1,\cdots,x_n$ 处的函数值,只要让每一个 $x_i$ 对应的基函数乘上相应的 $f(x_i)$ 再相加,就可以得到最终的插值多项式函数:
\begin{equation}
p(x) = \sum_{i = 0}^n f(x_i)l_i(x) ~.
\end{equation}
\subsubsection{Lagrange插值的MATLAB实现}
这里以 $y = \frac{1}{1+x^2}$ 在 $[-5,5]$的n次Lagrange插值函数为例展示Lagrange插值的具体实现。
\begin{lstlisting}[language=matlab]
function f = LagrangeIP(n)
%   [-5,5]内的Lagrange等距点插值,多项式次数为n,插值函数为y = 1./(1+x.*x);
x = -5:0.1:5;
y = 1./(1+x.*x);
x1 = linspace(-5,5,n+1);%插值节点
y1 = 1./(1+x1.*x1);
f = 0;%f为插值形成的函数
for i = 1:n+1
    L = 1;%L为插值基函数
    for j = 1:n+1
        if j~=i
            L = L.*(x-x1(j))./(x1(i)-x1(j));
        end
    end
    L = y1(i).*L;
    f = f+L;
end
plot(x,f,x,y);
\end{lstlisting}
\subsubsection{Newton插值}

\subsubsection{Runge 现象}
如果想要使得在取无限密的等分点时,拉格朗日插值得到的函数能够最终趋于原函数,拉格朗日插值要求函数不仅仅在这插值范围内解析,还要求在复数域上解析,否则在函数边界就会出现\textbf{Runge 现象(龙格现象)}。

以函数
$$y = f(x) = \frac1{1+16 x^2} ~,$$
举例,该函数本身图像为:
\begin{figure}[ht]
\centering
\includegraphics[width=6cm]{./figures/78c869353bce9d97.png}
\caption{$f(x)$ 之图像} \label{fig_PolyIP_1}
\end{figure}
在其中等间隔取点,随着点数的变多,插值函数在边缘的“抖动”越来越剧烈,如下图:
\begin{figure}[ht]
\centering
\includegraphics[width=6cm]{./figures/1941d5aa89ad4887.png}
\caption{取 $4$ 个红色点位插值得到图像} \label{fig_PolyIP_2}
\end{figure}
\begin{figure}[ht]
\centering
\includegraphics[width=6cm]{./figures/2ee81c4a4e2c5103.png}
\caption{取 $7$ 个红色点位插值得到图像} \label{fig_PolyIP_3}
\end{figure}
\begin{figure}[ht]
\centering
\includegraphics[width=6cm]{./figures/7034ddb6c52000bc.png}
\caption{取 $10$ 个红色点位插值得到图像} \label{fig_PolyIP_4}
\end{figure}
这现象便是龙格现象,原因是该函数在插值区间附近的复数域不解析。生成这图像使用的是 python 的 matplotlib 库,代码如下:

\begin{lstlisting}[language=python]
import numpy as np
from tqdm import tqdm
from sympy import Symbol as symSym
from sympy import Poly as symPoly
from sympy import expand as polyExpand
from matplotlib import pyplot as plt



class LagrangeInter:
    def __init__(self, xs: [list, np.array], ys: [list, np.array]):
        self.x = np.asarray(xs, dtype=np.float64)
        self.y = np.asarray(ys, dtype=np.float64)
        
        if len(self.x) > 1:
            if len(self.x) == len(self.y):
                self.n = len(self.x)
            else:
                return Exception("不能对空点集进行 Lagrange 插值")
        else:
            return Exception("x, y 数量不对应")
        
        self.poly = None
        self.polyCoff = None
        self.coffOrder = None
    
    def interp(self):
        t = symSym("t")
        self.poly = 0
        for i in tqdm(range(self.n)):
            basis_fun = self.y[i]
            for j in range(0, i):
                basis_fun *= (t - self.x[j]) / (self.x[i] - self.x[j])
            for j in range(i + 1, self.n):
                basis_fun *= (t - self.x[j]) / (self.x[i] - self.x[j])
            self.poly += basis_fun
            self.poly = polyExpand(self.poly)
            poly = symPoly(self.poly, t)
            self.polyCoff = poly.coeffs()
            self.coffOrder = poly.monoms()
    
    def calc(self, x0: [list, np.array]):
        x0 = np.asarray(x0, dtype=np.float64)
        y0 = np.zeros(len(x0))
        t = self.poly.free_symbols.pop()
        
        for i in tqdm(range(len(x0))):
            y0[i] = self.poly.evalf(subs={t: x0[i]})
        return y0



def func(x: float) -> float:
    return 1/(1+ 16*x*x)



if __name__ == "__main__":
    a = np.linspace(-3, 3, 10, endpoint=True)
    b = [func(x) for x in a]
    x0 = np.arange(-3, 3, 0.01)
    lagInterp = LagrangeInter(a, b)
    lagInterp.interp()
    y0 = lagInterp.calc(x0)
    plt.plot(x0, y0, 'bo')
    plt.plot(a, b, 'ro')
    plt.show()
\end{lstlisting}