% 个人目录
% keys 线性代数|数学物理|数学分析|拓扑|量子力学
% license Usr
% type Toc

\subsection{科研}
\enref{氦原子数值解 TDSE 笔记}{HeTDSE},
\enref{氦原子波函数数值分析}{HeAnal},
\enref{氢原子的 streaking 计算}{HyCLC},
\enref{光电离时间延迟:从一维波包到氦原子}{HeAna2},
\enref{氦原子数值解 TDSE 笔记}{HeTDSE},
\enref{氦原子波函数数值分析(笔记)}{HeAnal},
\enref{Berkeley-ECS 方法}{BerECS}

\subsection{线性代数}
\enref{投影算符}{projOp}, 矩阵的迹\upref{trace}

\subsection{数学物理}
连带勒让德函数\upref{AsLgdr}, 球谐函数\upref{SphHar}, 广义球谐函数\upref{GenYlm}, 平面波的球谐展开\upref{Pl2Ylm}, 库仑势能的球谐展开\upref{PChYlm}, 库仑函数\upref{CulmF}, 库仑波函数\upref{CulmWf}, Wigner 3j 符号\upref{ThreeJ}, Wigner 6j 符号\upref{SixJ}, Wigner 9j 符号\upref{NineJ}, 张量积空间\upref{DirPro}

\subsection{数学分析}
黎曼积分与勒贝格积分\upref{Rieman}, 数学分析笔记\upref{AnalNt}, 泛函分析笔记1\upref{FnalNt}, 泛函分析笔记2\upref{FnalN2}, 泛函分析笔记3\upref{FnalN3}, 泛函分析笔记4\upref{FnalN4}, 泛函分析笔记5\upref{FnalN5}

\subsection{代数}
群论笔记\upref{GroupN}

\subsection{拓扑学}
拓扑空间\upref{Topol}, 流形\upref{Manif}

\subsection{力学}
拉普拉斯—龙格—楞次矢量\upref{LRLvec}, 哈密顿正则方程\upref{HamCan}, 分析力学笔记\upref{ClsMec},工业仿真软件(电磁、流体、多物理)\upref{ChSciN}

\subsection{电磁}
电多极展开\upref{EMulPo}, 拉格朗日电磁势\upref{EMLagP}

\subsection{相对论}
《微分几何入门与广义相对论》笔记\upref{DGGRNt}

\subsection{量子力学}
全同粒子\upref{IdPar}, 含时微扰理论\upref{TDPT}, 量子散射的延迟\upref{tDelay}, 多通道散射\upref{MulSct}, 电磁场中的单粒子薛定谔方程\upref{QMEM}, 长度规范和速度规范\upref{LVgaug}, 加速度规范\upref{AccGau}, Volkov 波函数\upref{Volkov}, Keldysh 参数\upref{keldis}, 密度矩阵\upref{denMat}
Adiabatic 笔记\upref{Adibat},
Hall 量子力学笔记\upref{HallQM}

\subsection{计算物理}
氢原子薛定谔方程数值解\upref{HyTDSE}, 氢原子球坐标数值解 TDSE\upref{HTDSE}, 氢原子电离截面\upref{HionCr},  单电子原子模型\upref{SAE}, 物理仿真软件列表(笔记)\upref{PhySim}

\subsection{科普笔记}
%---------------------------------------
% 这些文章只是列出大纲
经典力学笔记(科普)\upref{CMInt},
天文学笔记(科普)\upref{AstrIn},
电磁学笔记(科普)\upref{EleMag},
光学笔记(科普)\upref{OptiIn},
相对论笔记(科普)\upref{RelaIn},
热力学笔记(科普)\upref{HeatIn},
原子分子笔记(科普)\upref{AtomIn},
宇宙学笔记(科普)\upref{CosmIn}

\subsection{其他}
%----------------------------------------
三维重建笔记\upref{TDcon}
