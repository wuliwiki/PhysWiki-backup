% 天津大学 2017 年考研量子力学答案
% 考研|天津大学|量子力学|2017|答案

\begin{issues}
\issueDraft
\end{issues}


\subsection{ }
\begin{enumerate}
\item 
(1)由题可知势能 $\overline{U}=-\frac{e^2}{r}$.
\begin{equation}
\begin{aligned}
\overline{U}=&\iiint \psi^{*} \overline{U} \psi \dd{\tau}\\
=&-\frac{e^2}{\pi a^{3}_{0}}\int^{\pi}_{0}\int^{2\pi}_{0}\int^{\infty}_{0} \frac{1}{r}e^{-\frac{2r}{a_0}}r^{2}\sin{\theta} \dd{r}\dd{\theta}\dd{\varphi}\\
=&-\frac{e^{2}}{\pi a^{3}_{0}}\int^{\pi}_{0}\int^{2\pi}_{0}\int^{\infty}_{0} e^{-\frac{2r}{a_0}}r\sin{\theta} \dd{r}\dd{\theta}\dd{\varphi}\\
=&-\frac{4 e^2}{a^{3}_{0}}\int^{\infty}_{0}e^{-\frac{2r}{a_0}}r\dd{r}\\
=&-\frac{4e^{2}}{a^{3}_{0}}(\frac{a_{0}}{2})^2\\
=&-\frac{e^{2}}{a_{0}}
\end{aligned}
\end{equation}
(2)电子 $r+dr$ 在球壳内出现的几率为:\\
\begin{equation}
\begin{aligned}
w(r)\dd{r}=&\int^{\pi}_{0}\int^{2\pi}_{0} \lvert \psi(r,\theta,\varphi) \rvert \sin{\theta}\dd{\theta}\dd{\varphi}\dd{r}\\
=&\frac{4}{a^{3}}e^{-\frac{2r}{a_0}}r^2 \dd{r}\\
\end{aligned}
\end{equation}
% $w(r)=\frac{4}{a^{3}}e^{-\frac{2r}{a_0}}r^2 $
\begin{equation}
\begin{aligned}
\frac{\dd{w(r)}}{\dd{r}}=\frac{4}{a^{3}}e^{-\frac{2r}{a_0}}r^2 
\end{aligned}
\end{equation}
令 $\frac{\dd{w(r)}}{\dd{r}}=0,\Longrightarrow r_1 = 0,r_2 = \infty,r_3 = a_0$,因为 $\frac{dd^{2}{w(r)}}{\dd{r^{2}}}|_{r = a_{}} < 0$,所以 $r = a_0$ 为最概然半径.

\item
答:康普顿散射是光子与电子做弹性碰撞,在 $X$ 射线通过实物物质发生散射的实验时,除原波长的光外还产生了大于原波长的 $X$ 光,借助光电理论,才可以得到这是由于光子与电子发生碰撞后频率变小的缘故,从而证实了光具有粒子性.

\item 
(1)三维转子的能级为:$E = \frac{l(l+1)}{2I}$,简并度为:$2l+1$.\\
(2)平面转子设沿 $z$ 轴方向,$\hat{H} = \frac{\hat{l}^{2}_{z}}{2I}$,能级为:$E = \frac{m^{2} \hbar^{2}}{2I} ,m = 0 , \pm 1 , \pm 2$,除了 $m = 0$ 外,能级都是二重简并.
\end{enumerate}
\subsection{ }
\begin{enumerate}
\item 
\begin{equation}
\begin{aligned}
(\vec{r} \times \vec{L} + \vec{L} \times \vec{r})_{x} =& yL_{z}-zL_{y}+L_{y}z-L_{z}y \\
%=& [y,xp_{y}-yp_{x}]+[zp_{x}-xp_{z},z] \\
%=& x[y,p_{y}]+[]
\end{aligned}
\end{equation}
\item 对于 $\vec{p} \times \vec{L} + \vec{L} \times \vec{p}$ 有:
\begin{equation}
\begin{aligned}
(\vec{p} \times \vec{L} + \vec{L} \times \vec{p})_{y} =& \vec{p}_{y} \vec{L}_{z} - \vec{p}_{z} \vec{L}_{y} + \vec{L}_{y} \vec{p}_{z} - \vec{L}_{z} \vec{p}_{y} \\
=& [\vec{p}_{y},\vec{L}_{z}] + [\vec{L}_{y},\vec{p}_{z}] \\
=& [\vec{p}_{y},x\vec{p}_{y} - y\vec{p}_{x}] + [z\vec{p}_{x}-x\vec{p}_{z},\vec{p}_{z}] \\
=& -[\vec{p}_{y},y]\vec{p}_{x} + [z,\vec{p}_{z}]\vec{p}_{x} \\
=& 2i\hbar \vec{p}_{x}
\end{aligned}
\end{equation}
同理可得:\\
$(\vec{p} \times \vec{L} + \vec{L} \times \vec{p})_{y} =2i\hbar \vec{p}_{y} $ \\
$(\vec{p} \times \vec{L} + \vec{L} \times \vec{p})_{z} =2i\hbar \vec{p}_{z} $ \\
所以:$\vec{p} \times \vec{L} + \vec{L} \times \vec{p} = 2i\hbar \vec{p} $
\end{enumerate}

\subsection{ }
在一维无限深势阱 $V_{x} = \leftgroup{
    & 0 \qquad 0\leqslant x \leqslant a \\
    & \infty \qquad x<0,x>a \\
}$ 的基础上,把 $H' = x \quad \frac{a}{2} < x < a $ 看作微扰,一维无限深势阱的本征函数和能量为:\\
\begin{align}
\psi^{0}_{n}=& \sqrt{\frac{2}{a}} \sin{\frac{n\pi x}{a}}\\
E^{0}_{n}=& \frac{n^{2}\pi{2}\hbar^{2}}{2Ma^2}
\end{align}
能量的一级修正为:\\
\begin{equation}
\begin{aligned}
E^{(1)}_{n}=&\int^{a}_{\frac{a}{2}} \sqrt{\frac{2}{a}} \sin{\frac{n\pi x}{a}}(x)\sqrt{\frac{2}{a}} \sin{\frac{n\pi x}{a}} \dd{x} \\
=& \frac{2}{a} \int^{a}_{\frac{a}{2}} x\sin^{2}{\frac{n\pi x}{a}} \dd{x} \\
=& \frac{2}{a}\int^{a}_{\frac{a}{2}} x\Bigl[\frac{1-\cos{\frac{2n \pi}{a}}}{2}\dd x\Bigr] \\
=& \frac{3a}{8}
\end{aligned}
\end{equation}

\begin{equation}
\begin{aligned}
H'_{mn} =& \int^{a}_{\frac{a}{2}} \sqrt{\frac{2}{a}} \sin{\frac{m \pi x}{a}} (x) \sqrt{\frac{2}{a} }\sin{\frac{n \pi x}{a}} \dd x \\
=& \frac{2}{a} \int^{a}_{\frac{a}{2}} x\Bigl[\frac{1}{2} \cos{\frac{(m-n)\pi x}{a}} - \frac{1}{2} \cos{\frac{(m+n) \pi x}{a}}\Bigr] \dd x \\
=& \frac{1}{a} \Bigl[\frac{a}{(m-n)\pi} \sin{\frac{(m-n)\pi x}{a}}\Big|^{a}_{\frac{a}{2}} - \frac{a}{(m+n)\pi} \sin{\frac{(m+n)\pi x}{a}}\Big|^{a}_{\frac{a}{2}}\Bigr] \\
=&
\end{aligned}
\end{equation}
\subsection{ }
设整体在$x$轴上可自由运动,谐振子的哈密顿量为:$\hat{H} = -\frac{\hbar^{2}}{2m} \nabla^{2}+\frac{1}{2}m\omega^{2}x^{2} $ \\

薛定谔方程为:$\hat{H}\psi_{x} = E\psi_{x}$ \\

$E=(n+\frac{1}{2} \hbar \omega)$ \\

$\psi = Ne^{-\frac{\alpha^{2}x^{2}}{2}}H_{n}(\alpha x)$
\subsection{ }
\begin{enumerate}
\item 由题意可得:\\
\begin{equation}
\begin{aligned}
H =& aS_{1x}S_{2x}+aS_{1y}S_{2y}+bS_{1z}S_{2z}+aS_{1z}S_{2z}-aS_{1y}S_{2y} \\
=& a(S_{1x}S_{2x}+S_{1y}S_{2y}+S_{1z}S_{2z})+(b-a)S_{z} \\
=& aS_{1}S_{2}+(b-a)S_z \\
=& \frac{a}{2}(s^{2}-\frac{3}{2}\hbar)+(b-a)m\hbar
\end{aligned}
\end{equation}
因为能量本征态是$S^{2}$和$S_{z}$的共同本征函数$\chi_{sm}$.\\
$\because s_{1}=\frac{1}{2},s_{2}=\frac{1}{2},\therefore s=1,0 $ \\
$\therefore E_{sm}=\frac{a}{2}[s(s+1)\hbar^{2}-\frac{3}{2}\hbar^{2}]+(b-a)m\hbar $ \\
$\therefore E_{11}=\frac{1}{4}a\hbar^{2}+(b-a)\hbar $ \\
$\therefore E_{1-1}=\frac{1}{4}a\hbar^{2}-(b-a)\hbar $ \\
$\therefore E_{10}=\frac{1}{4}a\hbar^{2} $ \\
$\therefore E_{00}=-\frac{3}{4}a\hbar^{2} $ \\
$\therefore \chi_{11}=\chi_{\frac{1}{2}}(s_{1z})\chi_{\frac{1}{2}}(s_{2z}) $ \\
$\chi_{1-1}=\chi_{-\frac{1}{2}}(s_{1z})\chi_{-\frac{1}{2}}(s_{2z}) $ \\
$\chi_{10}=\frac{1}{\sqrt{2}}[\chi_{\frac{1}{2}}(s_{1z})\chi_{-\frac{1}{2}}(s_{2z})+\chi_{\frac{1}{2}}(s_{2z})\chi_{-\frac{1}{2}}(s_{1z})] $ \\
$\chi_{00}=\frac{1}{\sqrt{2}}[\chi_{\frac{1}{2}}(s_{1z})\chi_{-\frac{1}{2}}(s_{2z})-\chi_{\frac{1}{2}}(s_{2z})\chi_{-\frac{1}{2}}(s_{1z})] $ 
\item 此时:\\
$\ket{1,0}=\chi_{s}=\frac{1}{\sqrt{2}}[\chi_{\frac{1}{2}}(s_{1z})\chi_{-\frac{1}{2}}(s_{2z})+\chi_{\frac{1}{2}}(s_{2z})\chi_{-\frac{1}{2}}(s_{1z})] $ \\
$\ket{0,0}=\chi_{A}=\frac{1}{\sqrt{2}}[\chi_{\frac{1}{2}}(s_{1z})\chi_{-\frac{1}{2}}(s_{2z})-\chi_{\frac{1}{2}}(s_{2z})\chi_{-\frac{1}{2}}(s_{1z})] $
\end{enumerate}