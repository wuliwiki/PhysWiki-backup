% 约束及其分类
% 约束|完整约束|非完整约束|约束方程|定常约束

\begin{issues}
\issueDraft
\issueTODO
\end{issues}

\pentry{矢量力学}

牛顿力学(或称矢量力学),原则上已经可以处理一切经典力学的问题,只需列出所有质点的受力之后进行求解.但是这种方法也有不足之处,比如在面对复杂约束时方程将变得难以求解,矢量的特性更给求解带来复杂度.为此,拉格朗日在前人的基础上提出了\textbf{分析力学(Analytical Mechanics)}.

在分析力学中,我们使用标量(拉格朗日量)来描述一个系统,系统的演化由拉格朗日方程来决定.为了进一步介绍有关内容,我们需要先叙述系统受到的约束及其分类.

\textbf{约束(Restrict)},顾名思义就是对系统中的每个质点的坐标及其速度所设的约束条件.一般我们认为对加速度不会有约束.描述约束条件的方程称为约束方程.

\begin{example}{铰接在地面上的轻杆}
未完成
\end{example}

在上例中,我们发现约束方程可写成f(x,y,z)=0的形式.一般地,假设一个系统中共有N个质点,其笛卡尔坐标依次记为$x
_1$,$x_2$,…$x_3N$,如果约束方程可写成f($x_1$,$x_2$,…$x_3N$,$\dot x_1$,$\dot x_2$,…$\dot x_3N$,t)=0的形式,则我们称这个约束是\textbf{不可解}的.如果方程左侧与前式同,右侧为 $\leq 0(\geqslant 0)$,那么我们称它是可解的.可解约束不是我们的讨论重点.

如果一个不可解约束方程中不含速度(又称几何约束),或是含速度(又称微分约束),但可以积分成不含速度的方程,那么我们称这个约束是\textbf{完整(Complete)}的.相应地,含速度且无法积分的约束被称为\textbf{非完整的(Imcomplete)}

\begin{example}{举一个可积含速度约束的例子}
未完成
\end{example}

下面我们再来看一个例子.

\begin{example}{环上的小球}
未完成
\end{example}

在上例中,球在二维平面内运动,拥有x,y两个坐标.让我们试想,如果在满足“球穿在环上“的约束的同时,让球的坐标发生一个微小变化.由于这个位移是我们设想出来的,而不一定是下一刻球的真实位移,因此将其称为\textbf{虚位移}.记x方向的虚位移为 $\delta x$,记 y 方向的虚位移为 $\delta y$,以示与通常的dx、dy不同,则这两个变化需要满足 $x\delta x+y\delta y=0$.显然,两个虚位移中只有一个是独立的.

一般地,我们将一个系统被约束方程限制后可以独立改变的虚位移的个数称为该系统的\textbf{自由度}.

\begin{example}{经典的非完整约束:纯滚圆盘}
未完成
\end{example}

从上例中我们看到,非完整约束一般复杂而难以求解.并且,其中并没有包含太多的新物理.因此,以后我们将只讨论完整约束,即约束方程中只含坐标与时间的简单情形.对非完整约束的详细讨论,请见马尔契夫的书.

如果约束方程不含时间,则称为\textbf{定常}的.反之,则称为非定常的.

\begin{example}{简单的非定常约束:加速直线运动的斜面}
未完成
\end{example}