% 金属材料科普(草稿)

\subsection{原子与晶体}
如果你的视力足够好\footnote{事实上,光的波长一般大于此(可见光的波长约为300-700nm,因此凭光学显微镜是不可能看到如此细小的结构等.这也是为什么我们发明了电子显微镜)},可以看见纳米级别(大概$10^-9m =10^-6 mm$)的金属微观结构,那么你会发现金属好像是由大量原子层层叠叠有序堆积起来的.

\begin{figure}[ht]
\centering
\includegraphics[width=5cm]{./figures/MetInt_1.png}
\caption{晶体中原子的排列示意图} \label{MetInt_fig1}
\end{figure}

\begin{definition}{晶体}
原子(或分子)在三维空间按一定规律作周期性排列而形成的固体
\end{definition}

根据金属种类的不同,堆积的具体方式也不同.既然晶体中的原子排列是周期性重复的,自然我们就能找出其中一个最小的重复单元以代表这种排列方式的特征,我们称之为晶胞.例如铁的晶胞有点像这样:
\begin{definition}{晶胞}
能够完全反应晶体几何特征的最小单元
\end{definition}
