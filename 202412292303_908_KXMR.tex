% 亨德里克·卡西米尔(综述)
% license Usr
% type Wiki

本文根据 CC-BY-SA 协议转载翻译自维基百科\href{https://en.wikipedia.org/wiki/Hendrik_Casimir}{相关文章}。

\begin{figure}[ht]
\centering
\includegraphics[width=6cm]{./figures/e3c81feb01c9da0b.png}
\caption{1958年的卡西米尔} \label{fig_KXMR_1}
\end{figure}
亨德里克·布鲁赫特·盖尔哈德·卡西米尔 (Hendrik Brugt Gerhard Casimir) ForMemRS(1909年7月15日-2000年5月4日)是一位荷兰物理学家,对量子力学和量子电动力学领域作出了重要贡献。他因其关于**卡西米尔效应**的研究而最为人知,该效应描述了由于电磁场的量子涨落,在真空中两块无电荷平板之间的吸引力。

卡西米尔还因他与C.J.戈特尔(C. J. Gorter)于1934年共同提出的超导体**双流体模型**而闻名。
\subsection{传记}
卡西米尔于1909年7月15日出生。[1] 他在**莱顿大学**师从保罗·埃伦费斯特(Paul Ehrenfest),并于1931年获得博士学位。[4] 他的博士论文涉及刚性旋转体的量子力学及分子旋转的群论。[5] 在此期间,他还在哥本哈根与尼尔斯·玻尔(Niels Bohr)合作,帮助玻尔支持他关于“枪手效应”的假说,甚至在校园进行模拟对决以证明假设。[6]

1932年至1933年中期,卡西米尔在苏黎世的瑞士联邦理工学院担任沃尔夫冈·泡利(Wolfgang Pauli)的助手。此期间,他研究了电子的相对论理论,尤其是评估束缚电子中克莱因-日下方程的偏差。为解决问题,他发明了许多数学工具,其中一种被称为“卡西米尔技巧”,这是粒子相互作用计算中使用狄拉克矩阵的迹形成与投影的常见方法。

1938年,卡西米尔成为莱顿大学的物理学教授。在此期间,他积极研究热传导与电传导,并对毫开尔文温度的实现作出了贡献。

在二战期间的1942年,卡西米尔加入位于荷兰埃因霍温的飞利浦物理实验室(NatLab)。[7] 1945年,他撰写了一篇关于拉尔斯·昂萨格(Lars Onsager)微观可逆性原理的重要论文。他于1946年成为飞利浦NatLab的联合主任,并在1956年成为飞利浦董事会成员。[8] 1972年,他从飞利浦退休。[9]

尽管卡西米尔的大部分职业生涯是在工业领域度过,但他仍是荷兰杰出的理论物理学家之一。从1931年至1950年,他在研究期间的科学贡献包括纯数学(李群,1931年)、超精细结构与核四极矩的计算(1935年)、低温物理学、磁性、超导体的热力学、顺磁松弛(1935年至1942年)以及昂萨格不可逆现象理论的应用(1942年至1950年)。他协助创立了欧洲物理学会,并于1972年至1975年担任主席。1979年,他在CERN成立25周年庆典上发表了主旨演讲。他于1946年成为荷兰皇家艺术与科学学院的成员。[10]

在飞利浦NatLab期间,1948年,卡西米尔与德克·波尔德(Dirk Polder)合作,预测了导体平板之间的量子力学吸引力,即如今所知的卡西米尔效应,这一效应在微机电系统(MEMS)等领域具有重要影响。

卡西米尔曾获得6个由荷兰以外的大学授予的荣誉博士学位,并赢得了众多奖项和荣誉,包括1976年由工业研究院授予的著名IRI奖章。他是美国国家工程院的外籍院士,并于1982年获得威廉·埃克斯纳奖章。[11] 他还被选为美国艺术与科学学院、美国国家科学院和美国哲学学会的成员。[12][13][14]
\subsection{出版作品}
\begin{itemize}
\item Casimir, H. B. G.(1940年)。《磁性与极低温》。英国剑桥:剑桥大学出版社。
\item H. B. G. Casimir,《随意的现实:半个世纪的科学》(Haphazard Reality: half a century of science)。纽约:Harper & Row出版社,1983年;卡西米尔的英文自传。ISBN 0-06-015028-9。
\item H. B. G. Casimir,《现实的偶然性:半个世纪的物理学》(Het toeval van de werkelijkheid: Een halve eeuw natuurkunde)。阿姆斯特丹:Meulenhof出版社,1992年;卡西米尔的荷兰文自传。ISBN 90-290-9709-4。
\item Casimir, H. B. G.; Polder, D.(1948年2月15日)。《延迟对伦敦-范德瓦尔斯力的影响》。《物理评论》。73 (4)。美国物理学会(APS):360–372。Bibcode:1948PhRv...73..360C。doi:10.1103/physrev.73.360。ISSN 0031-899X。
\item H. B. G. Casimir,《关于两块完美导电板之间的吸引力》。《荷兰皇家艺术与科学学院会议录》,第51卷,第793–795页(1948年)。
\item H. B. G. Casimir 和 J. Ubbink,《表面效应》(The Skin Effect)。《飞利浦技术评论》,第28卷,第300–315页(1967年)。
\end{itemize}