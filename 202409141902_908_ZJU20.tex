% 浙江大学 2000 年 考研 量子力学
% license Usr
% type Note

\textbf{声明}:“该内容来源于网络公开资料,不保证真实性,如有侵权请联系管理员”

\subsection{第一题:(20 分)}
\begin{enumerate}
  \item 下列说法哪个是正确的?不正确的说法给予修正。
  \begin{itemize}
    \item a. 量子力学适用于微观体系,而经典力学适用于宏观体系。
    \item b. 电子是粒子,又是波。
    \item c. 电子是粒子,不是波。
    \item d. 电子是波,不是粒子。
  \end{itemize}
  
  \item a. 厄米算符的定义是什么?算符 $x \frac{d}{dx}$ 是否厄米?
  \begin{itemize}
    \item b. 等式 $e^{\hat g} \cdot e^{\hat f} = e^{\hat g+\hat f}$ 何时成立?何时不成立?
  \end{itemize}
  
  \item 若太阳为一黑体,人所能感受到的太阳光能量的最大波长为 $\lambda_m = 0.48 \\, \mu m$,太阳半径 $R = 7.0 \times 10^8 \\, \text{m}$,太阳质量 $m = 2 \times 10^{30} \\ \text{kg}$,试估算太阳质量由于热辐射而损耗 1\% 所需要的时间。(斯特藩常数 $\sigma = 5.67 \times 10^{12}  \text{W} / (\text{cm}^2  \text{K}^4)$)
\end{enumerate}
\subsection{第二题:(20分)}
若有一粒子,质量为 $m$,在有限深势阱 $V(x) = \begin{cases}
0, & |x| \leq a \\\\
V_0, & |x| > a
\end{cases}$ 中运动,$V_0$ 为某一正常数。

\begin{enumerate}
  \item 试推导出其能量本征值所满足的方程。
  \item 如何求能量本征值?试作出求解本征值的草图。
  \item 若粒子不作一维运动,而是三维运动,$V(r) = \begin{cases}
0, & 0 < r < a \\\\
V_0, & r \geq a
\end{cases}$,试求出至少存在一个本征能的条件。
\end{enumerate}