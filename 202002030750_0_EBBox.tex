% 盒中的电磁波
% 电磁波|麦克斯韦方程|波动方程|驻波

空间中一个电阻不计的金属盒中有电磁波. 金属盒的大小为 $(0 \leqslant x \leqslant a, 0 \leqslant y \leqslant b, 0 \leqslant z \leqslant c)$, 电场的波动方程%(链接未完成)
为
\begin{equation}
\laplacian \vdot \bvec E = \frac{1}{c^2} \pdv[2]{\bvec E}{t}
\end{equation}  
矢量相等的充要条件是三个分量分别相等
\begin{equation}
\laplacian E_x = \frac{1}{c^2} \pdv[2]{E_x}{t} \qquad
\laplacian E_y = \frac{1}{c^2} \pdv[2]{E_x}{t} \qquad
\laplacian E_z = \frac{1}{c^2} \pdv[2]{E_x}{t}
\end{equation}   
下面以 $E_x$ 为例, 用分离变量法得出通解.
先令 $E_x = X(x) Y(y) Z(z) T(t)$. 代入上式, 两边同除 $X(x) Y(y) Z(z) T(t)$ 得
\begin{equation}
\left. \dv[2]{X}{x} \middle/ X + \dv[2]{Y}{y} \middle/ Y + \dv[2]{Z}{z} \middle/ Z  = \frac{1}{c^2}  \dv[2]{T}{t} \middle/ T\right.
\end{equation}
由于上式每一项都是一个独立变量的函数, 所以每一项都等于一个常数. 令这些常数为
\begin{equation}\begin{aligned}
&\left. \dv[2]{X}{x} \middle/ X \right. = -k_x^2 \qquad
\left. \dv[2]{Y}{y} \middle/ Y \right. = -k_y^2\\
&\left. \dv[2]{Z}{z} \middle/ Z \right. = -k_z^2 \qquad
\frac{1}{c^2} \left. \dv[2]{T}{t} \middle/ T \right. = -\omega^2
\end{aligned}\end{equation}
(取负号是因为我们只对三角函数解感兴趣, 指数函数解在这里无关)代入上式, 这些常数满足
\begin{equation}
k_x^2 + k_y^2 + k_z^2 = \omega ^2
\end{equation} 
上面三式的通解是
\begin{equation}
\leftgroup{
X = C_1\cos(k_x x) + C_2\sin(k_x x)\\
Y = C_3\cos(k_y y) + C_4\sin(k_y y)\\
Z = C_5\cos(k_z z) + C_6\sin(k_z z)
}\end{equation} 
时间函数的解取 $T = C\cos(\omega t)$ (时间函数的相位不重要)
由理想导体的电磁场边界条件%(未完成, 注意包含以下第二条)
\begin{equation}
E_{//} = 0  \qquad  \pdv{E_\bot}{n} = 0
\end{equation}  
$\pdv*{E_x}{x} = 0$ ( $x \to a$ 时);  $E_x = 0$ ($y \to b$ 或$z \to c$ 时). 把上面的通解带入条件, 得
\begin{equation}
X = C_1\cos(\frac{n_x \pi}{a} x)
\qquad
Y = C_4\sin(\frac{n_y \pi}{b} y)
\qquad
Z = C_6\sin(\frac{n_z \pi}{c} z)
\end{equation}  
三个变量相乘, 令 $C_1 C_4 C_6 = E_{x0}$,  得
\begin{equation}
E_x = E_{x0} \cos(\frac{n_x \pi}{a} x) \sin(\frac{n_y \pi}{b} y) \sin(\frac{n_z \pi}{c} z)
\end{equation} 
同理对 $E_y$,  $E_z$ 分析, 得到电场的三个分量在盒内的分布
\begin{equation}
\leftgroup{
E_x = E_{x0}\cos(\frac{n_x \pi}{a} x) \sin(\frac{n_y \pi}{b} y) \sin(\frac{n_z \pi}{c} z)\\
E_y = E_{y0}\sin(\frac{n_x \pi}{a} x) \cos(\frac{n_y \pi}{b} y) \sin(\frac{n_z \pi}{c} z)\\
E_z = E_{z0}\sin(\frac{n_x \pi}{a} x) \sin(\frac{n_y \pi}{b} y) \cos(\frac{n_z \pi}{c} z)
}\end{equation} 
\begin{equation}
T = C \cos(\omega t)
\end{equation}
且满足 $\omega  = \pi \sqrt{n_x^2/a^2 + n_y^2/b^2 + n_z^2/c^2}$
特殊地, 当盒子是立方体的时候, $a = b = c = L$ 时,$\omega  = \pi \sqrt{n_x^2 + n_y^2 + n_z^2}/L $.   


%(未完成)
%貌似等一下还会有一个限制条件的(电场散度为零! 磁场散度也为零! 还有磁场的边界条件!),所以电场三个分量的振动幅度并不是独立的, 要求 ${E_x}{n_x} + {E_y}{n_y} + {E_z}{n_z}$, 可能还有更多要求!

%为什么世界如此复杂啊!

