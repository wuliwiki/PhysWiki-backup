% 正交分解 投影算符
% keys 内积|投影|厄米算符|自伴算符|正交分解

\pentry{正交子空间\upref{OrthSp}}

本词条只讨论有限维向量空间.

\begin{theorem}{正交分解}\label{projOp_the1}
令 $M$ 为 $N$ 维内积空间\upref{InerPd} $X$ 的子空间. 那么任意 $u\in X$ 有唯一的正交分解:
\begin{equation}\label{projOp_eq1}
u = v + w \qquad (v\in M, w\in M^\bot)
\end{equation}
其中 $M^\bot$ 是 $M$ 在 $X$ 中的正交补空间(\autoref{OrthSp_def1}~\upref{OrthSp}).
\end{theorem}
根据正交子空间的定义, 分解后的两个向量正交: $\braket{v}{w} = 0$, 故有勾股定理(\autoref{InerPd_eq1}~\upref{InerPd})
\begin{equation}
\norm{u}^2 = \norm{v}^2 + \norm{w}^2
\end{equation}
其中范数使用内积定义 $\norm{u} = \sqrt{\braket{u}{u}}$.

\begin{example}{三维几何矢量的投影}
三维几何向量\upref{GVec} 空间 $X$ 中, 一个过原点的平面是它的一个子空间 $M \subset X$, 该平面过原点的法线就是它的正交补空间 $M^\bot$.  任意 $\bvec u \in X$ 可以唯一地正交分解成 $M$ 上的一个向量 $\bvec v$ 和法线上的一个矢量 $\bvec w$, 且满足勾股定理 $\bvec u^2 = \bvec v^2 + \bvec w^2$.
\addTODO{图}
这只是\autoref{projOp_the1} 一个很显然的例子, 若想象 4 维几何矢量空间中矢量正交分解到两个 2 维平面上, 则唯一性就没那么显然了.
\end{example}

对每个子空间 $M\subseteq X$, 我们定义对应的\textbf{投影算符} $P$: 根据\autoref{projOp_eq1} 将每个 $u\in X$ 映射到 $v\in M$.

\begin{theorem}{}
令投影算符将 $N$ 维空间 $X$ 中的矢量投影到其子空间 $A$ 中, 若 $\qty{a_i}$ 是 $A$ 的一组正交归一基底. 那么投影算符可以用狄拉克符号\upref{braket}表示为
\begin{equation}
P = \sum_i \ket{a_i}\bra{a_i}
\end{equation}
\end{theorem}
由该定理易证\autoref{projOp_the1} (将 $u$ 分别分解到 $M$ 和 $M^\bot$ 的正交归一基底上, 这个投影存在且唯一).



\subsection{投影算符是厄米算符}
投影算符 $P$ 是厄米算符(也叫自伴算符), 即对任意 $u, v\in X$ 满足 $\braket{u}{Pv} = \braket{Pu}{v}$.

投影算符有 $0$ 和 $1$ 两个本征值. 对应的本征矢分别是 $M$ 和 $M^\bot$ 空间中的所有矢量.

\subsubsection{证明}
对任意 $u, v\in X$, 有
\begin{equation}
\braket{u}{Pv} = \sum_i \\braket{u}{a_i}\braket{a_i}{v}
\end{equation}
\begin{equation}
\braket{Pu}{v} = \overline {\braket{v}{Pu}} = \sum_i \overline{\\braket{v}{a_i}}\overline{\braket{a_i}{u}} = \sum_i \\braket{u}{a_i}\braket{a_i}{v} = \braket{u}{Pv}
\end{equation}
证毕.
