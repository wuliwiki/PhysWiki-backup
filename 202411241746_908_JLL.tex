% 伽利略·伽利莱(综述)
% license CCBYSA3
% type Wiki

本文根据 CC-BY-SA 协议转载翻译自维基百科\href{https://en.wikipedia.org/wiki/Galileo_Galilei}{相关文章}。

\begin{figure}[ht]
\centering
\includegraphics[width=8cm]{./figures/f16d6423773a5687.png}
\caption{约1640年的肖像画} \label{fig_JLL_1}
\end{figure}
伽利略·迪·文琴佐·博纳尤蒂·德·伽利略(\textbf{Galileo di Vincenzo Bonaiuti de' Galilei},1564年2月15日-1642年1月8日),通常简称为伽利略·伽利莱(\textbf{Galileo Galilei},/ˌɡælɪˈleɪoʊ ˌɡælɪˈleɪ/,美国英语中也读作 /ˌɡælɪˈliːoʊ -/;意大利语:[ɡaliˈlɛːo ɡaliˈlɛːi]),或单名称为伽利略(\textbf{Galileo}),是一位佛罗伦萨的天文学家、物理学家和工程师,有时被描述为“博学多才之人”。他出生于当时属于佛罗伦萨公国的比萨市。伽利略被誉为观测天文学之父、现代经典物理学之父、科学方法之父以及现代科学之父。

伽利略研究了速度与加速度、重力与自由落体、相对性原理、惯性、抛体运动等,并在应用科学和技术领域开展了工作,描述了摆的特性和“静水天平”。他是文艺复兴早期温度计(即热测量仪)的开发者之一,还发明了多种军用罗盘。通过他改进的望远镜,伽利略观测到了银河的恒星、金星的位相、木星的四大卫星、土星的光环、月球的陨石坑以及太阳黑子。他还制作了一种早期显微镜。

伽利略对哥白尼日心说的支持遭到了天主教会内部和一些天文学家的反对。1615年,罗马宗教裁判所对这一问题进行了调查,得出结论认为他的观点与当时普遍接受的《圣经》解释相矛盾。[9][10][11]

后来,伽利略在《两大世界体系的对话》(1632年)中为自己的观点辩护,但书中似乎对教皇乌尔班八世进行了攻击和嘲讽,这使伽利略疏远了教皇及之前一直大力支持他的耶稣会士。[9] 他因此受到宗教裁判所的审判,被认定为“严重可疑的异端”,并被迫公开认错。伽利略此后被软禁在家中度过余生。[12][13] 在此期间,他撰写了《两种新科学》(1638年),主要探讨了运动学和材料强度问题。[14]
\subsection{早年生活与家庭}  
伽利略于1564年2月15日出生在比萨(当时属于佛罗伦萨公国),是文琴佐·伽利莱和朱莉亚·阿曼纳蒂的长子。他的父亲文琴佐是著名的鲁特琴演奏家、作曲家和音乐理论家,他的母亲朱莉亚是当地一位显赫商人的女儿。两人于1562年结婚,当时文琴佐42岁,而朱莉亚24岁。伽利略自己也成为了一名出色的鲁特琴演奏家,并可能从父亲那里早早学会了对权威的怀疑态度。[15][16]  

伽利略的五个兄弟姐妹中,有三个在婴儿期幸存下来。他最小的弟弟米开朗基罗(或称米开拉尼奥洛)也成为了一名鲁特琴演奏家和作曲家,但这一生都给伽利略带来了经济负担。[17] 米开朗基罗未能履行父亲承诺的嫁妆分担责任,这导致伽利略的妹夫们试图通过法律途径追讨欠款。此外,米开朗基罗有时还需要向伽利略借钱,以支持他的音乐事业和旅行。这些经济压力可能促使伽利略早年就产生了发明一些能够带来额外收入的装置的想法。[18]  

伽利略八岁时,家人搬到了佛罗伦萨,但他被留在比萨,由穆齐奥·泰达尔迪照顾了两年。十岁时,他离开比萨,与家人团聚在佛罗伦萨,并开始接受雅科波·博尔吉尼的指导。[15] 从1575年至1578年,他在佛罗伦萨东南约30公里的瓦隆布罗萨修道院接受教育,特别是在逻辑方面的学习。[19][20]
\subsubsection{名字}  
伽利略通常只用他的名字来称呼自己。在当时的意大利,姓氏并非必需,而他的名字“伽利略”(Galileo)与他有时使用的家族姓氏“伽利莱”(Galilei)源于同一祖先。无论是他的名字还是姓氏,最终都可追溯到他的祖先伽利略·博纳尤蒂(Galileo Bonaiuti),一位15世纪佛罗伦萨的重要医生、教授和政治家。[21] 伽利略·博纳尤蒂被安葬在佛罗伦萨的圣十字大教堂,这也是大约200年后伽利略·伽利莱的安葬地。[22]  

当伽利略使用多于一个名字时,他有时称自己为“伽利略·伽利莱·林切奥”(Galileo Galilei Linceo),以表示他是林切伊学院(Accademia dei Lincei)的一员。该学院是教皇国成立的一所精英科学组织。在16世纪中期的托斯卡纳,长子通常以父母的姓氏命名为名字。[23] 因此,伽利略·伽利莱的名字未必是专门为了纪念他的祖先伽利略·博纳尤蒂。意大利男性名“伽利略”(Galileo)及其派生的姓氏“伽利莱”(Galilei)来源于拉丁语“Galilaeus”,意为“加利利人”。[24][21]  

伽利略名字和姓氏的圣经渊源后来成为一个著名双关语的主题。[25] 1614年,在伽利略事件期间,伽利略的一位反对者、多米尼加会士托马索·卡奇尼(Tommaso Caccini)发表了一篇颇具争议且影响深远的布道词。他在布道中引用《使徒行传》1:11说道:“加利利人哪,你们为什么站着望天呢?”(可能含有对伽利略的讽刺)。[citation needed]  
\subsubsection{子女}  
\begin{figure}[ht]
\centering
\includegraphics[width=6cm]{./figures/79d02887461111a7.png}
\caption{据认为是伽利略长女维尔吉尼亚的肖像,她对父亲特别忠诚。} \label{fig_JLL_2}
\end{figure}
尽管伽利略是一位虔诚的天主教徒,[26] 他与玛丽娜·甘巴(Marina Gamba)未婚生育了三个孩子。他们育有两个女儿:维尔吉尼亚(Virginia,生于1600年)和利维娅(Livia,生于1601年),以及一个儿子:文琴佐(Vincenzo,生于1606年)。[27]  

由于孩子们是非婚生的,伽利略认为两个女儿难以嫁人,且无法承担高昂的经济支持或嫁妆费用。这些费用可能会使伽利略重蹈帮助两位姐妹嫁人的经济困境。[28] 因此,她们唯一合适的选择就是修道生活。两位女儿都被阿尔切特里的圣马太修道院接纳,并在此度过了余生。[29]  

维尔吉尼亚进入修道院后改名为玛利亚·切莱斯特(Maria Celeste)。她于1634年4月2日去世,并与伽利略合葬在佛罗伦萨的圣十字大教堂。利维娅进入修道院后改名为阿尔坎杰拉修女(Sister Arcangela),大部分时间身体都不好。伽利略的儿子文琴佐后来被确认为合法继承人,并与塞斯蒂莉亚·博基内里(Sestilia Bocchineri)结婚。[30]  
\subsection{职业生涯与首次科学贡献}
伽利略年轻时曾认真考虑成为一名神职人员,但在父亲的敦促下,他于1580年进入比萨大学学习医学。[31] 他受到了佛罗伦萨的吉罗拉莫·博罗(Girolamo Borro)和弗朗切斯科·博纳米奇(Francesco Buonamici)讲座的影响。[20] 1581年,在学习医学时,他注意到一个吊灯在空气流动的作用下以大小不同的弧度摆动。他发现,通过与自己的心跳比较,无论吊灯摆动的幅度多大,摆动一次所需的时间似乎相同。回到家后,他用两根等长的摆进行了实验,一个摆幅较大,另一个摆幅较小,发现它们的摆动时间相同。直到近一百年后,克里斯蒂安·惠更斯(Christiaan Huygens)的研究才将摆的等时性用于制造精准的计时器。[32]  

在此之前,伽利略被刻意远离数学,因为医生的收入高于数学家。然而,在偶然听了一场几何学讲座后,他说服了不情愿的父亲,让他改学数学和自然哲学,而不是医学。[32] 他发明了一种热测量仪(thermoscope),这是温度计的前身。1586年,他出版了一本小册子,介绍他设计的静水天平,这一发明使他首次引起学术界的关注。此外,伽利略还学习了绘画艺术(\textbf{disegno}),1588年在佛罗伦萨的美术学院(\textbf{Accademia delle Arti del Disegno})担任透视法与明暗法的讲师。同年,他应佛罗伦萨学院邀请发表了两场讲座,题为《但丁地狱的形状、位置与大小》,试图提出关于但丁地狱的严格宇宙模型。[33] 受城市艺术传统和文艺复兴艺术家作品的启发,伽利略形成了美学思维。在美术学院任教期间,他与佛罗伦萨画家奇戈利(Cigoli)建立了终生的友谊。[34][35]  

1589年,伽利略被任命为比萨大学的数学教授。1591年,他的父亲去世,他承担起照顾弟弟米开朗基罗的责任。1592年,他转到帕多瓦大学,教授几何学、力学和天文学,直至1610年。[36] 在此期间,伽利略在基础科学(如运动学和天文学)以及应用科学(如材料强度和望远镜的开创性研究)方面取得了重要发现。他的兴趣广泛,还研究了占星术,这在当时是一门与数学、天文学和医学紧密相关的学科。[37][38]
\subsubsection{天文学} 
\textbf{开普勒超新星}  

第谷·布拉赫(Tycho Brahe)及其他人曾观测过1572年的超新星。1605年1月15日,奥塔维奥·布伦佐尼(Ottavio Brenzoni)写信给伽利略,提到了1572年的超新星以及1601年较暗的新星。伽利略在1604年观测并讨论了开普勒超新星。由于这些新星没有显示出可检测的日视差,伽利略得出结论认为它们是遥远的恒星,因此推翻了亚里士多德关于天穹不变性的观点。[39]

\textbf{折射望远镜}
\begin{figure}[ht]
\centering
\includegraphics[width=8cm]{./figures/534f6baa94d041ab.png}
\caption{伽利略的“cannocchiali”望远镜,展于佛罗伦萨伽利略博物馆} \label{fig_JLL_3}
\end{figure}
或许仅仅基于对荷兰人汉斯·李普希(Hans Lippershey)在1608年尝试申请专利的首个实用望远镜的描述,[40] 伽利略在次年制作了一台放大倍率约为3倍的望远镜。他随后改进了设计,制造出放大倍率高达约30倍的版本。[41] 使用伽利略望远镜,观察者可以看到放大且正立的地面影像——这就是通常所说的地面望远镜或单筒望远镜。他还用它来观测天空;一段时间内,他是少数能够制作适合天文观测望远镜的人之一。1609年8月25日,他向威尼斯人展示了一台早期望远镜,放大倍率约为8倍或9倍。  

伽利略的望远镜也成为他的一个盈利副业,他将望远镜卖给商人,后者发现它在海上导航和贸易中都非常有用。伽利略于1610年3月出版了一本简短的论文,题为《星际信使》(**Sidereus Nuncius**),记录了他最初的天文望远镜观测成果。[42]

\textbf{月球}
\begin{figure}[ht]
\centering
\includegraphics[width=6cm]{./figures/7cfb4912b973d48d.png}
\caption{《星际信使》中月球的插图,1610年在威尼斯出版} \label{fig_JLL_4}
\end{figure}
1609年11月30日,伽利略将他的望远镜对准了月球。[43] 虽然他并非第一个通过望远镜观察月球的人(英国数学家托马斯·哈里奥特(Thomas Harriot)早在四个月前就进行了观察,但他仅记录到“奇怪的斑点”),[44] 但伽利略是第一个推测月球表面不规则阴影是由月球上的山脉和陨石坑遮挡光线所致的人。在他的研究中,他还绘制了月球地形图,并估算了山脉的高度。月球并非如亚里士多德所宣称的那样是一颗透明而完美的球体,也不像但丁所描述的那样是“第一颗行星”,一颗“升入天穹的永恒明珠”。  

伽利略有时被认为在1632年发现了月球纬度上的天平动,[45] 尽管托马斯·哈里奥特或威廉·吉尔伯特(William Gilbert)可能早已做出这一发现。[46]  

伽利略的朋友、画家奇戈利(Cigoli)在他的某幅画作中包含了一幅逼真的月球描绘,这可能是基于他通过自己望远镜的观察所得。[34]

\textbf{木星的卫星} 

1610年1月7日,伽利略通过望远镜观测到木星附近有三个他当时描述为“极其微小、完全不可见的小恒星”,它们位于木星附近,并与木星在一条直线上。[47] 随后几晚的观测显示,这些“恒星”相对于木星的位置在不断变化,这种现象无法用它们是真正固定的恒星来解释。1月10日,伽利略注意到其中一个“恒星”消失了,他推测这是因为它被木星遮挡了。在接下来的几天里,他得出结论,这些天体正在围绕木星运行:他发现了木星四大卫星中的三个。[48]  

伽利略在1月13日发现了第四颗卫星。他将这四颗卫星命名为“美第奇之星”(Medicean stars),以此向未来的资助人托斯卡纳大公科西莫二世·德·美第奇(Cosimo II de' Medici)及其三位兄弟致敬。[49] 不过,后来天文学家为了纪念伽利略,将这些卫星改名为“伽利略卫星”。这些卫星实际上也在1610年1月8日被西蒙·马里乌斯(Simon Marius)独立发现,并在1614年出版的《木星的世界》(\textbf{Mundus Iovialis})中被马里乌斯命名为\textbf{木卫一(Io)}、\textbf{木卫二(Europa)}、\textbf{木卫三(Ganymede)} 和 \textbf{木卫四(Callisto)}。[50]
\begin{figure}[ht]
\centering
\includegraphics[width=8cm]{./figures/9e52a2a199660cbe.png}
\caption{1684年绘制的法国地图,展示了早期地图的轮廓(浅色线条)与使用木星卫星作为精确时间参考进行的新测量结果(较粗线条)之间的对比。} \label{fig_JLL_5}
\end{figure}
伽利略对木星卫星的观测在天文学界引发了争议:一颗行星有更小的天体围绕其运行,这与亚里士多德宇宙学的原则不符,后者认为所有天体都应围绕地球运转。[51][52] 起初,许多天文学家和哲学家拒绝相信伽利略能够发现这样的现象。[53][54] 更加复杂的是,其他天文学家很难验证伽利略的观测结果。当伽利略在博洛尼亚展示望远镜时,与会者难以看到木星的卫星。其中一人,马丁·霍尔基(Martin Horky),注意到通过望远镜观察某些固定恒星(如室女座的角宿一)时,它们会显得是双星。他认为这证明了望远镜在观察天体时具有误导性,从而对木星卫星的存在表示怀疑。[55][56]  

然而,罗马的克里斯托弗·克拉维乌斯(Christopher Clavius)天文台确认了伽利略的观测结果。尽管对如何解释这些现象仍存疑问,但当伽利略次年访问罗马时,他受到了英雄般的欢迎。[57] 在接下来的18个月里,伽利略持续观测这些卫星,到1611年年中,他得出了这些卫星轨道周期的极其准确的估计值——这是约翰内斯·开普勒(Johannes Kepler)曾认为不可能完成的壮举。[58][59]  

伽利略还看到了其发现的实际用途。在海上确定船只的东西位置需要船上的时钟与本初子午线的时钟同步。解决这个经度问题对于航行安全至关重要,因此西班牙和后来荷兰都设立了巨额奖励来鼓励这一问题的解决。由于伽利略发现的卫星的掩食现象相对频繁且时间可以被非常准确地预测,它们可以用来校准船上的时钟。伽利略申请了这些奖励。然而,从船上观测这些卫星过于困难,但这种方法被用于陆地测量,包括重新绘制法国的地图。[60]: 15–16 [61]

\textbf{金星的相位}
\begin{figure}[ht]
\centering
\includegraphics[width=14.25cm]{./figures/5532f8a9bd158eb0.png}
\caption{1610年,伽利略·伽利莱通过望远镜观察到金星展现出相位变化,尽管它在地球的天空中始终靠近太阳(第一张图)。这一发现证明了金星围绕太阳运行,而不是围绕地球运行,这与哥白尼的日心模型的预测相符,并否定了当时普遍接受的地心模型(第二张图)。} \label{fig_JLL_6}
\end{figure}
从1610年9月起,伽利略观察到金星呈现出一整套类似月球的相位变化。尼古拉·哥白尼(Nicolaus Copernicus)提出的日心模型预测,金星应该会展现所有相位,因为金星绕太阳的轨道会使其被照亮的半球在金星位于太阳的对侧时面对地球,而在金星位于太阳的地球侧时背向地球。在托勒密的地心模型中,任何行星的轨道都不可能与承载太阳的球壳相交。因此,传统上金星的轨道被完全置于太阳的近侧,在这种情况下金星只能呈现新月和朔相。也可以将其轨道完全置于太阳的远侧,这样金星只能呈现盈凸相和满相。

伽利略通过望远镜观察到金星的弦月相、盈凸相和满相后,托勒密的地心模型变得无法成立。在17世纪早期,由于这一发现,大多数天文学家转而支持某种形式的地-日心混合行星模型,[62][63] 如第谷模型(Tychonic model)、卡佩拉模型(Capellan model)和扩展卡佩拉模型(Extended Capellan model),这些模型有的包含每日自转的地球,有的则没有。这些模型解释了金星的相位变化,同时避免了完全日心模型关于恒星视差的“驳斥”。因此,伽利略对金星相位的发现是从完全地心模型向通过地-日心混合模型过渡到完全日心模型的两阶段转变中最具实证意义的贡献之一。[citation needed]

\textbf{土星与海王星}  

1610年,伽利略观察到土星,他最初误将土星的光环视为行星,[64] 认为它是一个由三个天体组成的系统。当他后来再次观测时,发现土星的光环正对地球,导致他以为两个“天体”消失了。1616年,他再次观察到光环的重新出现,这使他更加困惑。[65]  

1612年,伽利略观察到海王星。在他的笔记中,这颗行星被记录为许多微弱、不引人注目的恒星之一。他没有意识到这是一颗行星,但注意到它相对于背景恒星的运动,然后失去了对它的进一步跟踪。[66]  

\textbf{太阳黑子}  

伽利略通过肉眼和望远镜研究了太阳黑子。[67] 它们的存在再次挑战了正统亚里士多德天体物理学中“天体不变且完美”的理论。1612年至1613年,弗朗切斯科·西兹(Francesco Sizzi)及其他人观察到太阳黑子的轨迹出现明显的年度变化,[68] 这一现象成为反对托勒密体系和第谷地日混合体系的强有力证据。[c] 关于太阳黑子的发现优先权以及其解释的争议,使伽利略与耶稣会士克里斯托夫·谢纳(Christoph Scheiner)之间展开了一场长期且激烈的争论。谢纳曾将他的发现告知马克·维尔瑟(Mark Welser),后者向伽利略询问他的意见。两人都未意识到约翰内斯·法布里修斯(Johannes Fabricius)早期的观察和关于太阳黑子的发表。[72]  

\textbf{银河系与恒星}  
伽利略观察到银河系,发现其并非人们之前认为的模糊云气,而是由无数密集的恒星组成,使其从地球上看似为云状。他还定位了许多肉眼无法看到的遥远恒星,并在1617年观察了大熊座中的双星——北斗七星的 Mizar。[73]  

在《星际信使》中,伽利略报告说恒星在望远镜中仅显示为光点,外观几乎没有改变,而行星则在望远镜中呈现为圆盘。但随后在《论太阳黑子的书信》中,他报告说望远镜揭示了恒星和行星的形状都“非常圆”。从那时起,他继续报告望远镜显示了恒星的圆形,并测量了恒星的视直径,大约为几角秒。[74][75] 他还设计了一种不用望远镜测量恒星视直径的方法。在《两大世界体系的对话》中,他描述了这一方法:用一根细绳挡在恒星的视线中,测量能够完全遮挡恒星的最大距离。通过测量这个距离和绳子的宽度,他可以计算恒星在观测点的夹角。[76][77][78]  

在《对话》中,伽利略报告说他测得一等星的视直径不超过5角秒,而六等星的视直径约为5/6角秒。和他同时代的大多数天文学家一样,伽利略没有意识到他测量的恒星视直径是虚假的,主要由衍射和大气扰动引起,并不代表恒星的真实大小。然而,伽利略的数值比第谷等人此前对最亮恒星视直径的估计要小得多,这使他能够反驳反对哥白尼理论的论点,例如第谷提出的“恒星必须极其巨大以至于其周年视差不可探测”的说法。[79][80][81] 西蒙·马里乌斯(Simon Marius)、乔瓦尼·巴蒂斯塔·里乔利(Giovanni Battista Riccioli)和马丁努斯·霍滕修斯(Martinus Hortensius)等天文学家也对恒星进行了类似测量,马里乌斯和里乔利得出结论,恒星的视直径虽然较小,但仍不足以完全回答第谷的论点。[82][83]
\subsubsection{潮汐理论}
\begin{figure}[ht]
\centering
\includegraphics[width=6cm]{./figures/72e99e7fa5604283.png}
\caption{伽利略·伽利莱,由弗朗切斯科·波尔恰绘制的肖像} \label{fig_JLL_7}
\end{figure}
\textbf{红衣主教贝拉明于1615年写道,哥白尼体系无法成立,除非有“真正的物理证明表明太阳不是绕地球运转,而是地球绕太阳运转”。}[84] 伽利略认为,他关于潮汐的理论能够提供这种证据。[85] 这一理论对他而言如此重要,以至于他最初打算将《两大世界体系的对话》命名为《海潮涨落的对话》。[86] 不过,宗教裁判所下令将标题中的潮汐参考移除。[citation needed]

对于伽利略而言,潮汐是由于地球表面某一点随着地球的自转和绕太阳的公转而加速或减速时,海水在海洋中来回晃动所引起的。他在1616年首次向红衣主教奥尔西尼(Cardinal Orsini)提交了他的潮汐理论。[87] 他的理论首次揭示了海洋盆地形状对潮汐大小和时间的重要性。例如,他正确解释了为什么亚得里亚海中部的潮汐几乎可以忽略不计,而两端的潮汐却更为显著。然而,作为关于潮汐成因的一般理论,他的观点是失败的。[citation needed]

如果他的理论是正确的,每天只会有一次涨潮。然而,伽利略和他的同时代人知道这一点不足,因为威尼斯每天有两次涨潮,间隔约12小时。伽利略将这一异常归因于几个次要原因,包括海洋的形状、深度以及其他因素。[88][89] 阿尔伯特·爱因斯坦后来表示,伽利略发展了他的“引人入胜的论点”,并因对地球运动的物理证明的渴望而不加批判地接受了它们。[90]  

伽利略还否定了从古代以及其同时代的约翰内斯·开普勒提出的月球引力导致潮汐的观点。[91] 他同样对开普勒提出的行星椭圆轨道理论不感兴趣。[92][93] 伽利略继续支持他的潮汐理论,并认为这是地球运动的最终证明。[94]
\subsubsection{关于彗星的争论与《观察者》} 
1619年,伽利略与耶稣会罗马学院(Collegio Romano)数学教授奥拉齐奥·格拉西(Father Orazio Grassi)神父陷入了一场争论。这场争论最初围绕彗星的本质展开,但到伽利略在1623年发表《观察者》(\textbf{Il Saggiatore})——这一争论的最后一击时,已演变为一场关于科学本质的更广泛争议。书籍的标题页称伽利略为哲学家和托斯卡纳大公的“首席数学家”(\textbf{Matematico Primario})。[95]  

由于《观察者》包含了大量伽利略关于科学实践的观点,这本书被视为他的科学宣言。[96][97] 1619年初,格拉西神父匿名发表了一本小册子,标题为《1618年三颗彗星的天文争议》(\textbf{An Astronomical Disputation on the Three Comets of the Year 1618}),[98] 讨论了前一年11月末出现的一颗彗星的本质。格拉西认为,这颗彗星是一个炽热的天体,以恒定的距离沿大圆的一段轨道运动,[99][100] 由于它在天空中的移动速度比月亮慢,因此它的距离必定比月亮更远。[citation needed]  

格拉西的观点和结论在一篇名为《论彗星的对话》(\textbf{Discourse on Comets})的文章中受到了批评。[101] 该文章以伽利略的弟子、佛罗伦萨律师马里奥·圭杜奇(Mario Guiducci)的名义发表,但主要由伽利略撰写。[102] 伽利略和圭杜奇并未对彗星的本质提出明确的理论,[103][104] 他们提出的一些假说后来被证明是错误的。(当时,第谷·布拉赫已经提出了对彗星研究的正确方法。)在文章的开篇,伽利略和圭杜奇对耶稣会士克里斯托夫·谢纳(Christoph Scheiner)进行了毫无必要的侮辱,[105][106][107] 并在文中多次贬低罗马学院的教授们。[105] 耶稣会士对此感到冒犯,[105][104] 格拉西很快以“洛塔里奥·萨西奥·西根萨诺”(Lothario Sarsio Sigensano)的笔名发表了一篇回应文章《天文学与哲学的天平》(\textbf{The Astronomical and Philosophical Balance}),[108][109] 假称这是他的一位学生所作。[citation needed]  

《观察者》是伽利略对《天文学的天平》的强烈回应。[101] 它被广泛认为是争论文学的杰作,[110][111] 在书中,“萨西”的论点被毫不留情地嘲讽和批评。[112] 这本书广受赞誉,特别令新任教皇乌尔班八世感到满意,因为伽利略将其献给了他。[113] 在过去十年中,巴贝里尼(即后来的乌尔班八世)曾在罗马支持伽利略及其所在的林切伊学院(Lincean Academy)。[114]  

伽利略与格拉西的争论永久地疏远了许多耶稣会士,[115] 伽利略和他的朋友们确信是耶稣会士促成了他后来受到的谴责,[116] 尽管对此的确凿证据并不充分。[117][118]
\subsubsection{关于日心说的争议}
\begin{figure}[ht]
\centering
\includegraphics[width=6cm]{./figures/55ff24265526737a.png}
\caption{克里斯蒂亚诺·班蒂1857年的画作《伽利略面对罗马宗教裁判所》} \label{fig_JLL_8}
\end{figure}
在伽利略与教会发生冲突的时期,多数受过教育的人都支持亚里士多德的地心说,认为地球是宇宙的中心和所有天体的轨道中心,或者支持第谷·布拉赫的混合体系,该体系将地心说和日心说结合起来。[119][120] 对日心说以及伽利略相关著作的反对意见结合了宗教和科学的观点。宗教反对主要来自《圣经》中暗示地球静止不动的经文。[d] 科学反对则来自第谷,他认为如果日心说是正确的,那么应该可以观察到周年恒星视差,但当时并未发现这种现象。[e] 阿里斯塔克斯和哥白尼曾正确地推测,由于恒星距离遥远,视差可以忽略不计。然而,第谷反驳道,既然恒星看起来具有可测量的角大小,那么如果它们如此遥远,它们的大小必须远远超过太阳,甚至大于地球轨道。[123] 直到后来,天文学家才意识到恒星的视亮度是由一种光学现象——爱里斑(Airy disk)——引起的,恒星的视亮度与其亮度有关,而非其真实物理大小(参见\textbf{Magnitude#History})。[123]  

伽利略根据1609年的天文观测为日心说辩护。1613年12月,佛罗伦萨的大公夫人克里斯蒂娜(Christina)以《圣经》为依据,对伽利略的朋友和追随者贝内代托·卡斯特利(Benedetto Castelli)提出了关于地球运动的反对意见。[f] 受此事件的启发,伽利略写了一封信给卡斯特利,论证日心说实际上并不违反《圣经》文本,并指出《圣经》的权威在于信仰和道德,而非科学。这封信未被发表,但广泛传播。[124] 两年后,伽利略写了一封给克里斯蒂娜的信,将他先前用八页阐述的论点扩展至四十页。[125]  

到1615年,伽利略关于日心说的著作被尼科洛·洛里尼神父(Father Niccolò Lorini)提交给了罗马宗教裁判所,洛里尼声称伽利略及其追随者试图重新解释《圣经》,[d] 这被视为违反特伦特会议的行为,并被认为带有危险的基督教新教倾向。[126] 洛里尼特别引用了伽利略写给卡斯特利的信。[127] 伽利略前往罗马为自己及其观点辩护。1616年初,弗朗切斯科·因戈利(Francesco Ingoli)发起了与伽利略的争论,向伽利略递交了一篇驳斥哥白尼体系的论文。伽利略后来表示,他认为这篇论文在随后的反日心说行动中起了重要作用。[128] 因戈利可能受宗教裁判所委托撰写关于此争议的专家意见,其论文成为宗教裁判所采取行动的依据。[129]  

论文集中提出了18个针对日心说的物理和数学论据,主要借鉴了第谷·布拉赫的观点,尤其是关于日心说需要恒星远大于太阳的假设。[g] 此外,该论文还包括四个神学论据,但因戈利建议伽利略关注物理和数学论据,他并未提及伽利略的圣经观点。[131]

1616年2月,宗教裁判所的委员会宣布日心说在哲学上是“愚蠢而荒谬的”,在神学上则是“正式的异端”,因为它在许多方面明确地与《圣经》的字面意义相矛盾。宗教裁判所认为,关于地球运动的观点“在哲学上得到了相同的判断,并且……在神学真理方面,至少是信仰上的错误”。[132] 教皇保罗五世指示红衣主教贝拉明将这一裁定通知伽利略,并命令他放弃日心说。2月26日,伽利略被召到贝拉明的住所,接到命令:“完全放弃……太阳静止于世界中心而地球运动的观点,并从此不得以任何方式持有、教授或辩护该观点,无论是口头还是书面形式。”[133] 教皇的禁书目录会还下令禁止阅读哥白尼的《天体运行论》和其他日心说著作,直至进行修改。[133]  

在接下来的十年里,伽利略远离了这一争议。1623年,红衣主教马费奥·巴贝里尼当选为教皇乌尔班八世,这鼓励了伽利略重新开始撰写关于这一主题的书。巴贝里尼是伽利略的朋友和崇拜者,并曾反对1616年对伽利略的警告。伽利略最终完成的著作《两大世界体系的对话》于1632年出版,获得了宗教裁判所的正式批准和教皇的许可。[134]
\begin{figure}[ht]
\centering
\includegraphics[width=6cm]{./figures/d9b6990f5fee1170.png}
\caption{尤斯图斯·苏斯特曼斯1635年绘制的伽利略肖像} \label{fig_JLL_9}
\end{figure}
早先,教皇乌尔班八世曾亲自要求伽利略在书中列出支持和反对日心说的论点,并注意不要宣扬日心说。然而,无论是无意还是故意,《两大世界体系的对话》中支持亚里士多德地心说的角色辛普利乔(Simplicio)经常在争论中暴露自己的错误,有时显得像个傻瓜。尽管伽利略在书的前言中声明,该角色的名字来源于一位著名的亚里士多德哲学家(拉丁文为*Simplicius*,意大利语为“Simplicio”),但在意大利语中,“Simplicio”也带有“愚蠢的人”或“简单之人”的含义。[135][136] 这种对辛普利乔的描写使《两大世界体系的对话》看起来像一本宣扬日心说的书:对亚里士多德地心说的攻击和对哥白尼理论的辩护。[citation needed]  

大多数历史学家认为,伽利略并非出于恶意,他对书引发的反应感到措手不及。[h] 然而,教皇并未轻视书中可能包含的对他的公众嘲弄,也无法接受明显对哥白尼体系的支持。

伽利略疏远了他最重要和最有权势的支持者之一——教皇,并于1632年9月被召往罗马为自己的著作辩护。[140] 他最终于1633年2月抵达罗马,并被带到宗教裁判官文琴佐·马库拉尼(Vincenzo Maculani)面前接受指控。在整个审判过程中,伽利略坚决主张自己自1616年以来一直忠实地履行了不持有任何被禁止观点的承诺,并最初否认他曾为这些观点辩护。然而,他最终被劝说承认,虽然这并非他的本意,但《两大世界体系的对话》的读者确实可能会得出这是为哥白尼理论辩护的印象。鉴于伽利略的否认缺乏可信性,即他在1616年后从未持有哥白尼观点,也从未打算在《对话》中为其辩护,他在1633年7月的最终审问中被威胁,如不说出真相将遭受酷刑,但他尽管受到威胁,仍坚持否认。[141][142][143]  

宗教裁判所的判决于6月22日宣布,主要包括三项内容:  
\begin{itemize}
\item 伽利略被认定“严重涉嫌异端”(尽管他从未被正式指控为异端,因此免除了身体惩罚),[144] 具体指他持有以下观点:太阳静止于宇宙中心;地球不是宇宙中心并且在运动;以及在某观点被宣告与《圣经》相悖后仍可以认为其是可能的并为之辩护。他被要求“放弃、诅咒并厌恶”这些观点。[145][146][147][148]  
\item 他被判处在宗教裁判所的意愿下接受正式监禁。[149] 次日,这一判决被改为软禁,他在余生中一直处于这种状态。[150]  
\item 他的问题作品《对话》被禁;此外,未在审判中宣布的一项行动是禁止出版他所有的作品,包括他未来可能撰写的任何著作。[151][152]
\end{itemize}
\begin{figure}[ht]
\centering
\includegraphics[width=6cm]{./figures/0613373b2536bed9.png}
\caption{原本归于穆里略创作的肖像画,描绘了伽利略凝视着监狱墙上刻下的文字“E pur si muove”(“然而它在移动”,此图中无法辨认)。然而,关于这幅画的归属及其相关叙述后来遭到了质疑。} \label{fig_JLL_10}
\end{figure}
根据流行的传说,在伽利略放弃其“地球绕太阳运动”理论之后,他据说低声说出了反抗性的短语“然而它在移动”(\textbf{E pur si muove})。据称,一幅1640年代的画作,由西班牙画家巴托洛梅·埃斯特班·穆里略(Bartolomé Esteban Murillo)或其画派的一位艺术家创作,其中隐藏的文字直到1911年修复时才显现,描绘了伽利略在监禁中凝视着地牢墙上写着“E pur si muove”的画面。最早已知关于这一传说的文字记载出现在他去世一个世纪之后。基于这幅画作,斯蒂尔曼·德雷克(Stillman Drake)写道:“现在毫无疑问,这句著名的话在伽利略去世之前就已被归于他了。”[153] 然而,天体物理学家马里奥·利维奥(Mario Livio)的深入调查表明,这幅画作很可能是1837年由佛兰芒画家罗曼-尤金·范·马尔代亨(Roman-Eugene Van Maldeghem)创作的一幅画的复制品。[154]  

在伽利略的一段软禁期间,他与友好的锡耶纳大主教阿斯卡尼奥·皮科洛米尼(Ascanio Piccolomini)共处,随后于1634年获准返回佛罗伦萨附近阿尔切特里的别墅,在那里他在软禁中度过了部分时光。伽利略被要求每周诵读《七首忏悔诗篇》(\textbf{Seven Penitential Psalms})一次,持续三年。然而,他的女儿玛利亚·切莱斯特(Maria Celeste)获得教会许可后,主动替他完成了这一任务。[155]  

伽利略在软禁期间完成了他最优秀的著作之一《两种新科学》(\textbf{Two New Sciences})。在这部作品中,他总结了四十年前所做的关于两门学科(现称运动学和材料强度学)的研究。为了避开审查制度,该书在荷兰出版。这部著作受到了阿尔伯特·爱因斯坦的高度赞扬。[156] 因此,伽利略常被称为“现代物理学之父”。1638年,伽利略完全失明,并患上了严重的疝气和失眠症,因此获准前往佛罗伦萨寻求医疗建议。[14]  

达娃·索贝尔(Dava Sobel)认为,在伽利略1633年因异端审判和判决之前,教皇乌尔班八世因宫廷阴谋和国家问题而分心,开始担忧受到迫害或生命威胁。在这种背景下,伽利略的问题被宫廷内部人士和伽利略的敌人提到教皇面前。因被指责在捍卫教会方面表现软弱,乌尔班出于愤怒和恐惧而对伽利略作出了反应。[157] 马里奥·利维奥(Mario Livio)则将伽利略及其发现置于现代科学和社会背景之下。他特别指出,伽利略事件在某种程度上与现代的“科学否认”现象有相似之处。[158]
\subsection{逝世}
\begin{figure}[ht]
\centering
\includegraphics[width=8cm]{./figures/a54b4eda264b6d70.png}
\caption{伽利略之墓,位于佛罗伦萨圣十字大教堂} \label{fig_JLL_11}
\end{figure}
伽利略在去世前一直接待访客,直到1642年1月8日因发烧和心悸去世,享年77岁。[14][159] 托斯卡纳大公费迪南多二世希望将他葬在佛罗伦萨圣十字大教堂的主教堂内,安放在他父亲和其他祖先的墓旁,并为他建造一座大理石陵墓以示敬意。[160][161]  

然而,由于教皇乌尔班八世及其侄子红衣主教弗朗切斯科·巴贝里尼(Cardinal Francesco Barberini)的反对,这一计划被取消。[160][161][162] 原因是伽利略曾因“严重涉嫌异端”而被天主教会定罪。[163] 他被安葬在大教堂南侧耳堂通向圣器室的一条走廊尽头,靠近初学者小礼拜堂的一个小房间内。[160][164]  

1737年,在为纪念伽利略建造的纪念碑竣工后,他的遗体被重新迁葬至大教堂的主堂。[165][166] 在迁葬过程中,他的遗骸中被取走了三根手指和一颗牙齿。[167] 其中一根手指目前陈列于意大利佛罗伦萨的伽利略博物馆(Museo Galileo)。[168]
\begin{figure}[ht]
\centering
\includegraphics[width=6cm]{./figures/a3be0c2db4eca121.png}
\caption{伽利略右手的中指} \label{fig_JLL_12}
\end{figure}
\subsection{科学贡献}  
“这些以及其他数量不少、同样值得了解的事实,我已经成功证明;更重要的是,我认为,通过我的工作,仅作为这门广阔而卓越科学的开端,为那些比我更敏锐的头脑开启了探索其深远角落的途径和方法。”

—— 伽利略·伽利莱,《两种新科学》
\subsubsection{科学方法} 
伽利略通过实验与数学的创新结合,对运动科学做出了原创性贡献。[169] 当时更典型的科学研究是威廉·吉尔伯特(William Gilbert)关于磁学和电学的定性研究。伽利略的父亲文琴佐·伽利莱(Vincenzo Galilei)是一位鲁特琴演奏家和音乐理论家,他通过实验可能确立了物理学中最古老的非线性关系之一:对于拉紧的弦,其音高与张力的平方根成比例。[170] 这些观察属于毕达哥拉斯音乐传统的框架,这一传统对乐器制造者来说十分熟知,其中包括一个事实:将琴弦按整数比例分割会产生和谐音阶。因此,数学早已在一定程度上与音乐和物理科学相关,而年轻的伽利略能看到自己父亲的观察拓展了这一传统。[171]  

伽利略是最早明确指出自然规律具有数学性质的现代思想家之一。在《观察者》(*The Assayer*)中,他写道:“哲学写在这本伟大的书中,即宇宙……它用数学的语言写成,其字符是三角形、圆形及其他几何图形;……”[172] 他的数学分析是后期经院自然哲学家传统的进一步发展,这一传统是伽利略在学习哲学时掌握的。[173] 他的工作标志着科学最终从哲学和宗教中分离的又一步,这一分离是人类思想史上的重大进展。他经常愿意根据观察结果调整自己的观点。  

为了进行实验,伽利略必须建立长度和时间的标准,以便在不同时间和实验室中进行的测量可以以可重复的方式进行比较。这为使用归纳推理验证数学定律提供了可靠的基础。[citation needed] 伽利略对数学、理论物理学和实验物理学之间的适当关系展现出了现代理解。他理解抛物线既可以通过圆锥曲线的视角理解,也可以通过纵坐标(y)随横坐标(x)的平方变化的关系来理解。伽利略进一步断言,在没有空气阻力或其他干扰的情况下,抛物线是均匀加速物体的理想弹道轨迹。他承认这一理论的适用性有限,并从理论上指出,若投射物的轨迹规模可与地球相当,则不可能是抛物线。[174][175][176] 然而,他仍然坚持认为,在其时代火炮射程范围内,投射物轨迹与抛物线的偏差非常小。[174][177][178]
\subsubsection{天文学}
伽利略通过他的折射望远镜,于1609年末观察到月球表面并不平滑。[34] 次年初,他观测到了木星的四大卫星。[49] 同年晚些时候,他观察到金星的相位变化——这为日心说提供了证明,同时还观察了土星,但当时他误以为土星的光环是另外两颗行星。[64] 1612年,他观测到海王星并注意到其运动,但并未识别其为一颗行星。[66]  

伽利略研究了太阳黑子,[67] 银河系,并对恒星进行了多项观察,包括如何在没有望远镜的情况下测量恒星的视直径。[76][77][78]  

1619年,他创造了“极光”(\textbf{Aurora Borealis})这一术语,源自罗马的黎明女神(Aurora)和希腊语中北风的名称,用来描述当太阳风粒子激发磁层时在北方和南方天空中出现的光。[179]
\subsubsection{工程学}
\begin{figure}[ht]
\centering
\includegraphics[width=8cm]{./figures/d0e19a1ff2e7e18f.png}
\caption{鲁本斯1602-1606年的画作《在曼图亚友人圈中的自画像》。伽利略是左侧第三人。画中远处描绘了北极光的景象。} \label{fig_JLL_13}
\end{figure}
伽利略在现代所谓的工程学领域做出了许多贡献,与纯粹的物理学有所区别。在1595年至1598年间,伽利略设计并改进了一种几何和军用罗盘,适用于炮手和测量员的使用。这种罗盘是在尼科洛·塔塔利亚(Niccolò Tartaglia)和圭多巴尔多·德尔·蒙特(Guidobaldo del Monte)早期设计的仪器基础上进一步发展而成。对于炮手来说,它不仅提供了一种更安全且更精确的抬高火炮的方法,还能快速计算不同大小和材料的炮弹所需的火药量。作为几何工具,这种罗盘可以用来构造任意正多边形,计算任意多边形或圆扇形的面积,以及进行多种其他计算。  

在伽利略的指导下,仪器制造商马尔切安东尼奥·马佐莱尼(Marc'Antonio Mazzoleni)生产了100多件这种罗盘。伽利略将其以50里拉的价格出售,并附带一本由他撰写的使用手册,同时还开设了一门关于使用罗盘的课程,收费为120里拉。[180]
\begin{figure}[ht]
\centering
\includegraphics[width=8cm]{./figures/6ef345fc18f73198.png}
\caption{伽利略的几何和军用罗盘,据认为由他的专属仪器制造师马尔切安东尼奥·马佐莱尼于约1604年制作。} \label{fig_JLL_14}
\end{figure}
1593年,伽利略制作了一种温度计,通过气泡中空气的膨胀和收缩推动连接管中的水流动。[citation needed]  

1609年,伽利略与英国人托马斯·哈里奥特(Thomas Harriot)及其他人一起,成为最早使用折射望远镜观察恒星、行星或卫星的人之一。“望远镜”(\textbf{telescope})这一名称是希腊数学家乔瓦尼·德米西亚尼(Giovanni Demisiani)为伽利略的仪器创造的。[181][182] 这一命名是在1611年费德里科·切西亲王(Prince Federico Cesi)为接纳伽利略加入其林切伊学院(\textbf{Accademia dei Lincei})而举办的一场宴会上提出的。[183]  

1610年,伽利略利用望远镜近距离放大观察昆虫的部分结构。[184][185] 到1624年,他已使用复合显微镜。同年5月,他将其中一台仪器赠送给佐伦红衣主教(Cardinal Zollern),由其转赠给巴伐利亚公爵;[186] 同年9月,他又将另一台赠予切西亲王。[187] 一年后,林切伊学院的另一位成员乔瓦尼·法伯(Giovanni Faber)以希腊词 \textbf{μικρόν}(micron,意为“小”)和 \textbf{σκοπεῖν}(skopein,意为“观察”)为伽利略的发明命名为“显微镜”(\textbf{microscope}),以对应“望远镜”的命名方式。[188][189]  

1625年,利用伽利略显微镜制作的昆虫插图被出版,这似乎是复合显微镜使用的首次清晰记录。[187]  
\begin{figure}[ht]
\centering
\includegraphics[width=6cm]{./figures/ebdf158b052cdf11.png}
\caption{已知最早的摆钟设计,由伽利略·伽利莱构思。} \label{fig_JLL_15}
\end{figure}
1612年,伽利略在确定木星卫星的轨道周期后提出,如果能够充分准确地掌握它们的轨道信息,可以将其位置作为通用时钟,从而实现经度的测定。在他余生中,他时不时地研究这个问题,但实践中存在许多困难。该方法首次由乔瓦尼·多梅尼科·卡西尼(Giovanni Domenico Cassini)于1681年成功应用,并随后广泛用于大型土地测量。例如,这一方法被用于绘制法国地图,后来又由泽布伦·派克(Zebulon Pike)在1806年绘制美国中西部地图时采用。在海上导航中,由于精密的望远镜观测更加困难,经度问题最终通过约翰·哈里森(John Harrison)发明的便携式海洋计时器得以解决。[190]  

在晚年完全失明时,伽利略设计了一种用于摆钟的擒纵机构(称为“伽利略擒纵器”),尽管直到17世纪50年代克里斯蒂安·惠更斯(Christiaan Huygens)制造出第一个完全可操作的摆钟后,这一设计才被付诸实施。[citation needed]  

伽利略曾多次受邀就工程方案提供建议,以缓解河流洪水问题。1630年,马里奥·圭杜奇(Mario Guiducci)可能在确保伽利略被咨询比森齐奥河(Bisenzio River)附近开辟新河道的计划中发挥了重要作用。[191]  

普通滚珠轴承的问题在于滚珠会彼此摩擦,从而产生额外的摩擦力。通过将每个滚珠装入一个笼子中可以减少这种摩擦。这种“笼式”滚珠轴承最早由伽利略在17世纪描述。[192]
\subsubsection{物理学}
\begin{figure}[ht]
\centering
\includegraphics[width=8cm]{./figures/b94a075b33164854.png}
\caption{《伽利略与维维亚尼》,作者提托·莱西,1892年} \label{fig_JLL_16}
\end{figure}
伽利略在物体运动方面的理论和实验研究,与开普勒和勒内·笛卡尔的独立研究一起,为艾萨克·牛顿发展经典力学奠定了基础。

\textbf{摆}  
\begin{figure}[ht]
\centering
\includegraphics[width=6cm]{./figures/c82c52ed8aef65ee.png}
\caption{比萨大教堂的穹顶及“伽利略的灯”} \label{fig_JLL_17}
\end{figure}
伽利略进行了多次与摆相关的实验。人们普遍认为(得益于文琴佐·维维亚尼的传记)这些实验始于他在比萨大教堂观察青铜吊灯的摆动,并用自己的脉搏作为计时器。伽利略对摆的最早记录兴趣见于他死后出版的笔记《论运动》(*On Motion*),[193] 但后来的实验被描述在他的《两种新科学》中。伽利略声称一个简单的摆是等时的,即无论摆动幅度如何,摆动的时间总是相同。实际上,这只是近似成立,[194] 正如克里斯蒂安·惠更斯(Christiaan Huygens)后来发现的那样。此外,伽利略还发现,摆的周期的平方与摆长成正比。

\textbf{声音频率}  

伽利略较少因这一领域而为人所知,但他仍被认为是最早理解声音频率的人之一。他通过以不同速度刮动凿子,发现声音的音高与凿子跳动的间距相关,这实际上是频率的度量。

\textbf{水泵}  

到17世纪,水泵的设计已改进到能够产生可测量的真空,但这一现象当时尚未被完全理解。已知的是,吸水泵无法将水抽至某一高度以上:根据约1635年的测量,这个高度为18佛罗伦萨码,约为34英尺(10米)。[195] 这一限制对灌溉工程、矿井排水以及托斯卡纳大公计划建造的装饰性喷泉构成了问题,因此大公委托伽利略研究这一问题。在他的《两种新科学》(1638年)中,伽利略提出了一个错误的解释,认为水泵提升的水柱在超过34英尺后会因自身重量而断裂。[195]  

\textbf{光速}   

1638年,伽利略描述了一种测量光速的实验方法:安排两个观察者,各持一盏带有遮光板的灯笼,在一定距离内观察对方的灯光。第一个观察者打开他的灯笼的遮光板,当第二个观察者看到光时,立即打开自己的灯笼的遮光板。第一个观察者从打开遮光板到看到第二个灯笼的光所经历的时间,表明光在两人之间往返所需的时间。伽利略报告说,当他在不到一英里的距离内尝试这一实验时,无法确定光是否是瞬间出现的。[196] 在伽利略去世到1667年之间,佛罗伦萨实验学院(Accademia del Cimento)的成员重复了这一实验,距离约为一英里,结果同样无结论。[197] 后来证明,光速远远快于这种方法所能测量的范围。  

\textbf{伽利略不变性} 

伽利略提出了相对性基本原理,即物理定律在任何以恒定速度沿直线运动的系统中都是相同的,无论其具体速度或方向如何。因此,不存在绝对的运动或绝对的静止。这个原理为牛顿运动定律提供了基本框架,并成为爱因斯坦狭义相对论的核心基础。  

\subsubsection{伽利略不变性} 
伽利略提出了相对性的基本原理,即物理定律在任何以恒定速度沿直线运动的系统中都是相同的,无论其具体速度或方向如何。因此,不存在绝对的运动或绝对的静止。这一原理为牛顿运动定律提供了基本框架,并且是爱因斯坦狭义相对论的核心基础。

\subsubsection{下落的物体}  
\textbf{约翰·菲洛波努斯、尼科尔·奥雷斯姆与多明戈·德·索托}  

不同重量的物体以相同速度下落的观点可能早在罗马哲学家卢克莱修(Lucretius)时期就被提出过。[198] 关于不同重量但尺寸相近的物体以相同速度下落的观察结果,在6世纪约翰·菲洛波努斯(John Philoponus)的著作中有记载,而伽利略对此有所了解。[199][200]  

14世纪,尼科尔·奥雷斯姆(Nicole Oresme)推导出了均匀加速运动的平方定律,[201][202] 而16世纪,多明戈·德·索托(Domingo de Soto)提出物体在均质介质中下落时会呈现均匀加速。[203] 然而,德·索托并未预见伽利略关于下落物体理论中的许多修正和细化。例如,他没有像伽利略那样认识到,只有在真空中物体才会以严格的均匀加速度下落,否则在其他情况下,物体最终会达到一个均匀的终端速度。

\textbf{代尔夫特塔实验 } 

1586年,西蒙·斯蒂文(Simon Stevin,通常称为斯蒂维努斯)和扬·科内茨·德·赫鲁特(Jan Cornets de Groot)从荷兰代尔夫特市的新教堂(Nieuwe Kerk)上投下铅球。实验表明,大小相同但质量不同的物体以相同的速度下落。[32][204]  

虽然代尔夫特塔实验取得了成功,但其科学严谨性不及后来的一些实验。斯蒂文被迫依赖听觉反馈(球体撞击木质平台时产生的声音)来推断球体以相同速度下落。相比之下,这一实验的影响力不如伽利略·伽利莱更为扎实的研究以及他1589年著名的比萨斜塔思想实验。

\textbf{比萨斜塔实验}

伽利略的学生文琴佐·维维亚尼(Vincenzo Viviani)在其传记中写道,伽利略曾从比萨斜塔上投下由相同材料制成但质量不同的球体,以证明它们的下落时间与质量无关。[205] 这一实验与亚里士多德的教义相反:亚里士多德认为重物比轻物下落得更快,其速度与重量成正比。[206][207]  

尽管这一故事在大众叙述中被广泛传颂,但伽利略本人并未对这一实验有过任何记载,历史学家普遍认为这最多是一个并未真正发生的思想实验。[208] 唯一的例外是斯蒂尔曼·德雷克(Stillman Drake),[209] 他认为这一实验确实以维维亚尼所描述的方式或类似方式进行过。然而,伽利略对下落物体的大多数实验实际上是在斜面上进行的,这种方法能够大大减少时间测量和空气阻力的问题。[210]

\textbf{《两种新科学》} 

在1638年的《两种新科学》中,伽利略塑造的角色萨尔维亚蒂(Salviati),被广泛视为伽利略的代言人,他主张所有不同重量的物体在真空中都会以相同的有限速度下落。萨尔维亚蒂还认为,可以通过比较空气中用铅和软木制作的摆锤运动来实验性地验证这一点,这些摆锤虽然重量不同,但在其他方面是相似的。[citation needed]  

\textbf{时间平方定律}  

伽利略提出,下落的物体会以均匀加速度下落,只要它所通过介质的阻力可以忽略不计,或者在极限情况下,物体在真空中下落。[211][212] 他还推导出了从静止状态开始的均匀加速运动中物体所经过的距离的正确运动学定律,即该距离与经过时间的平方成正比(d∝t²)。[203][213] 伽利略使用几何构造和数学上精确的语言表达了时间平方定律,以符合当时的学术标准。(后来由其他人以代数形式重新表达了该定律。)[citation needed]

\subsubsection{惯性}  

伽利略还得出结论,在没有任何阻力的情况下,物体会保持其速度不变,[214] 这与普遍接受的亚里士多德假说相矛盾。亚里士多德认为,物体只能在所谓的“强迫”、“非自然”或“受力”运动中保持运动,前提是有一个持续施加作用的“推动者”存在。[215] 关于惯性的哲学思想早在约翰·菲洛波努斯(John Philoponus)和让·布里丹(Jean Buridan)的研究中就有所提出。伽利略在《两种新科学》中写道:[216][217]  

> 设想有一个粒子在没有摩擦的水平平面上被抛射;然后,根据前几页更详细的解释,我们知道,如果这个平面没有边界,这个粒子将在此平面上以均匀且永恒的运动继续移动。

——伽利略·伽利莱,《两种新科学》,第四天  

地球表面如果消除了所有的不平整,就可以被视为这样一个平面。[218] 这一思想被纳入了牛顿运动定律(第一定律),但在运动方向上有所不同:牛顿认为运动方向是直线,而伽利略认为是圆周运动(例如,行星绕太阳的运动,伽利略认为这种运动发生在没有引力作用的情况下,这与牛顿的观点不同)。根据迪克斯特豪伊斯(Dijksterhuis)的观点,伽利略关于惯性的圆周运动倾向的概念,与他对哥白尼体系的信念密切相关。[219]
\subsubsection{数学} 
虽然伽利略将数学应用于实验物理的方式具有创新性,但他所使用的数学方法是当时的标准方法,包括大量逆比例平方根法的例子,这一方法可追溯至斐波那契和阿基米德。他的分析和证明在很大程度上依赖于欧多克斯比例理论,这一理论在欧几里得《几何原本》的第五卷中有所阐述。这一理论仅在一个世纪前才得以传播,得益于塔塔利亚(Tartaglia)等人的准确翻译;但到伽利略晚年,它已逐渐被笛卡尔的代数方法取代。  

如今被称为“伽利略悖论”的概念并非由他首创。他提出的解决方案,即无限数目无法比较,现已被认为不再具有实用价值。[220]
\subsection{遗产}  
\subsubsection{教会的后续重新评估} 
伽利略事件在伽利略去世后逐渐被遗忘,争议也随之平息。宗教裁判所对重新印刷伽利略作品的禁令在1718年被解除,当时获准在佛罗伦萨出版其作品的一个版本(不包括被禁止的《对话》)。[221] 1741年,教皇本笃十四世批准出版伽利略科学著作全集的一个版本,其中包括经过轻微审查的《对话》版本。[223][222] 1758年,提倡日心说的著作被从禁书目录中全面移除,但未经审查的《对话》和哥白尼的《天体运行论》仍然被禁止。[224][222] 到1835年,这些作品最终从禁书目录中移除,所有教会对日心说的官方反对痕迹也随之消失。[225][226]  

19世纪早期,新教争论者重新激起了对伽利略事件的兴趣,他们利用这一事件(以及如西班牙宗教裁判所和平地说的神话等其他事件)攻击罗马天主教会。[9] 自此,对这一事件的兴趣时起时落。1939年,教皇庇护十二世在其当选教皇数月内对宗座科学院发表的首次讲话中,将伽利略描述为“研究中最无畏的英雄之一……不惧道路上的障碍和风险,也不畏惧带来死亡的威胁”。[227] 他的长期顾问罗伯特·莱伯教授(Robert Leiber)写道:“庇护十二世非常小心,不愿过早关闭任何(通向科学的)大门。他对此非常积极,并对伽利略事件感到遗憾。”[228]  

1990年2月15日,红衣主教拉青格(后来的教皇本笃十六世)在罗马大学发表演讲时引用了对伽利略事件的一些现代观点,并称其为“一个典型案例,让我们看到现代科学和技术的自我怀疑有多深”。[229][230][231] 他引用了哲学家保罗·费耶阿本德(Paul Feyerabend)的观点,后者认为:“伽利略时代的教会比伽利略本人更接近理性,同时也考虑到了伽利略教学的伦理和社会后果。对伽利略的判决是理性且公正的,而推翻这一判决只能基于政治上的便利。”[231] 红衣主教并未明确表示他是否同意费耶阿本德的观点,但他表示:“基于这样的观点构建草率的辩护是愚蠢的。”[231]  

1992年10月31日,教皇若望·保禄二世承认宗教裁判所在谴责伽利略坚持地球围绕太阳运行的问题上犯了错误。他表示:“当时谴责伽利略的神学家没有认识到《圣经》与其解释之间的正式区别。”[232]  

2008年3月,宗座科学院院长尼古拉·卡比博(Nicola Cabibbo)宣布计划在梵蒂冈城墙内为伽利略树立一座雕像以示纪念。[233] 同年12月,在纪念伽利略最早通过望远镜进行天文观测的400周年活动中,教皇本笃十六世赞扬了伽利略对天文学的贡献。[234] 然而,一个月后,宗座文化委员会主席詹弗兰科·拉瓦西(Gianfranco Ravasi)透露,在梵蒂冈建立伽利略雕像的计划已被搁置。[235]
\subsubsection{对现代科学的影响}
\begin{figure}[ht]
\centering
\includegraphics[width=6cm]{./figures/7a337654d7435c7b.png}
\caption{伽利略向威尼斯总督展示如何使用望远镜(朱塞佩·贝尔蒂尼1858年的壁画)} \label{fig_JLL_18}
\end{figure}
根据斯蒂芬·霍金的说法,伽利略可能比任何其他人都对现代科学的诞生负有更大的责任,[236] 而阿尔伯特·爱因斯坦则称他为“现代科学之父”。[237][238]  

伽利略的天文发现以及对哥白尼理论的研究为他留下了持久的遗产,其中包括将伽利略发现的木星四大卫星(木卫一、木卫二、木卫三和木卫四)分类为“伽利略卫星”。其他科学项目和原理也以伽利略命名,包括伽利略号航天器(\textbf{Galileo spacecraft})。[239]  

部分由于2009年是伽利略首次用望远镜记录天文观测的400周年,联合国将该年定为国际天文学年。[240]  
\subsection{著作}  
\begin{figure}[ht]
\centering
\includegraphics[width=6cm]{./figures/06c6d1b43e18a0f1.png}
\caption{佛罗伦萨乌菲兹美术馆外的雕像} \label{fig_JLL_19}
\end{figure} 
伽利略早期关于科学仪器的著作包括1586年撰写的《小天平》(\textbf{La Billancetta}),描述了一种能够在空气或水中准确称重的天平,[241] 以及1606年印刷出版的《几何和军用罗盘操作》(\textbf{Le Operazioni del Compasso Geometrico et Militare}),介绍几何和军用罗盘的使用方法。[242]  

他早期关于动力学、运动科学和力学的研究包括1590年前后的比萨著作《论运动》(\textbf{De Motu})和约1600年的帕多瓦著作《力学》(\textbf{Le Meccaniche})。前者基于亚里士多德-阿基米德的流体动力学,认为物体在流体中的下落速度与其比重超过流体的程度成正比,因此在真空中,物体的下落速度与其比重成正比。它还采纳了菲洛波努斯的动力理论,其中动力会自我耗散,自由下落的物体在初期加速后会达到一个终端速度。[citation needed]  

\textbf{《星际信使》}  
伽利略1610年的《星际信使》(\textbf{Sidereus Nuncius})是第一部基于望远镜观测的科学论文,其中报告了以下发现:  

\begin{itemize}
\item 伽利略卫星  
\item 月球表面的粗糙  
\item 大量肉眼无法看到的恒星,尤其是那些构成银河的恒星  
\item 行星与恒星外观的差异——行星呈现为小圆盘,而恒星则是未放大的光点
\end{itemize}  

\textbf{《论太阳黑子》}  1613年,伽利略发表了《论太阳黑子》(\textbf{Letters on Sunspots}),提出太阳和天体并非不可腐朽。[243] 该书还记录了他1610年通过望远镜观测到的金星的全相变化,以及土星神秘的“附加物”及其更令人困惑的随后的消失。  \textbf{《致大公夫人克里斯蒂娜的信》}1615年,伽利略完成了一份名为《致大公夫人克里斯蒂娜的信》(\textbf{Letter to the Grand Duchess Christina})的手稿,直到1636年才以印刷形式出版。这封信是他写给卡斯特利的信的修订版。后者因倡导哥白尼理论既符合物理事实又符合《圣经》而被宗教裁判所指控为对神学的干预。[244]\textbf{《潮汐论》}  1616年,在宗教裁判所命令伽利略不得持有或捍卫哥白尼立场后,他撰写了《潮汐论》(\textbf{Discorso sul flusso e il reflusso del mare}),基于哥白尼地球理论,以私人信件形式写给奥尔西尼红衣主教。[245]\textbf{《论彗星》}1619年,伽利略的学生马里奥·圭杜奇(Mario Guiducci)发表了一篇以伽利略为主要作者的讲座《论彗星》(\textbf{Discorso Delle Comete}),反驳了耶稣会对彗星的解释。[246]  

\textbf{《观察者》}1623年,伽利略出版了《观察者》(\textbf{Il Saggiatore}),批评基于亚里士多德权威的理论,提倡实验和科学思想的数学表达。这本书大获成功,甚至在基督教会的高层中获得支持。[247]\textbf{《两大世界体系的对话》}在《观察者》取得成功后,伽利略于1632年出版了《两大世界体系的对话》(\textbf{Dialogo sopra i due massimi sistemi del mondo})。尽管他尽力遵守宗教裁判所1616年的指示,但书中支持哥白尼理论和非地心模型的内容导致他被审判并禁止出版。\textbf{《两种新科学》}尽管出版受禁,伽利略于1638年在宗教裁判所管辖之外的荷兰出版了《两种新科学(\textbf{Discorsi e Dimostrazioni Matematiche, intorno a due nuove scienze})。[citation needed]  
\begin{figure}[ht]
\centering
\includegraphics[width=6cm]{./figures/1e8b6125d7dcd58c.png}
\caption{皮奥·费迪(Pio Fedi,1815–1892)创作的伽利略雕像,位于贝尔法斯特女王大学兰扬楼内。这座雕像由威廉·惠特拉爵士(Sir William Whitla,1890–1919年担任药物学教授)从意大利带回并捐赠给该大学。} \label{fig_JLL_20}
\end{figure}
\subsubsection{已出版的著作}  

伽利略的主要著作为:  

\begin{itemize}
\item 《小天平》 (The Little Balance) (1586;意大利语:La Bilancetta)  
\item 《论运动》 (On Motion) (约1590;拉丁语:De Motu Antiquiora)[249]  
\item 《力学》 (Mechanics) (约1600;意大利语:Le Meccaniche)  
\item 《几何和军用罗盘操作》 (The Operations of Geometrical and Military Compass) (1606;意大利语:Le operazioni del compasso geometrico et militare)  
\item 《星际信使》 (The Starry Messenger) (1610;拉丁语:Sidereus Nuncius)  
\item 《论浮体》 (Discourse on Floating Bodies) (1612;意大利语:Discorso intorno alle cose che stanno in su l'acqua, o che in quella si muovono*,“关于漂浮或移动于水中的物体的讨论”)  
\item 《太阳黑子的历史与论证》 (History and Demonstration Concerning Sunspots) (1613;意大利语:Istoria e dimostrazioni intorno alle macchie solari,基于《三封关于太阳黑子的信》Tre lettere sulle macchie solari,1612)  
\item 《致大公夫人克里斯蒂娜的信》 (Letter to the Grand Duchess Christina) (1615;1636年出版)  
\item 《潮汐论》 (Discourse on the Tides) (1616;意大利语:Discorso del flusso e reflusso del mare)  
\item 《论彗星》 (Discourse on the Comets) (1619;意大利语:Discorso delle Comete)  
\item 《观察者》 (The Assayer) (1623;意大利语:Il Saggiatore*)  
\item 《两大世界体系的对话》 (Dialogue Concerning the Two Chief World Systems) (1632;意大利语:Dialogo sopra i due massimi sistemi del mondo*)  
\item 《两种新科学的讨论与数学论证》 (Discourses and Mathematical Demonstrations Relating to Two New Sciences) (1638;意大利语:Discorsi e Dimostrazioni Matematiche, intorno a due nuove scienze)  
\end{itemize}
\subsubsection{私人藏书}  
伽利略·伽利莱在他生命的最后几年里,在佛罗伦萨郊外的伊尔乔耶洛别墅(Villa Il Gioiello)保存了至少598本藏书(其中560本已被确认)。[250] 在被软禁期间,他被禁止撰写或出版自己的思想。然而,他仍然接待访客直至去世,通过这些访客,他不断获得来自北欧的最新科学著作。[251]  

伽利略的遗嘱并未提及他的藏书和手稿。只有在他去世后,才制作了一份详细的清单。当时,他的大部分财产,包括藏书,都传给了他的儿子文琴佐·伽利莱(Vincenzo Galilei Jr.)。文琴佐于1649年去世后,这些藏书由他的妻子塞斯蒂利娅·博基内里(Sestilia Bocchineri)继承。[251]  

伽利略的书籍、私人文稿和未编辑的手稿后来由他的前助手兼学生文琴佐·维维亚尼(Vincenzo Viviani)收集,他的目的是将老师的作品以出版形式保存下来。然而,这一项目从未实现。在他的遗嘱中,维维亚尼将藏书的一部分捐赠给佛罗伦萨圣玛利亚新医院(Hospital of Santa Maria Nuova),那里已拥有一个庞大的图书馆。这些藏书的价值未被认识到,一些重复的副本被分散到其他图书馆,例如锡耶纳的公共图书馆 Biblioteca Comunale degli Intronati。后来,为了使图书馆的收藏更加专业化,与医学无关的书籍被转移到马利亚贝基亚纳图书馆(Biblioteca Magliabechiana),这一图书馆后来成为佛罗伦萨国家中央图书馆(Biblioteca Nazionale Centrale di Firenze)的基础。[251]  

维维亚尼藏书中的一小部分,包括伽利略及其同时代人埃万杰利斯塔·托里拆利(Evangelista Torricelli)和贝内代托·卡斯特利(Benedetto Castelli)的手稿,传给了他的侄子雅各波·潘扎尼尼修士(Abbot Jacopo Panzanini)。这一小部分藏书在潘扎尼尼去世后由他的侄孙卡洛(Carlo)和安杰洛·潘扎尼尼(Angelo Panzanini)继承。然而,这些藏书未被妥善保护,继承者的仆人将其中的一些书籍当作废纸出售。大约在1750年,佛罗伦萨参议员乔瓦尼·巴蒂斯塔·克莱门特·德内利(Giovanni Battista Clemente de'Nelli)听说此事后,从店主手中购买了这些书籍和手稿,并从潘扎尼尼兄弟那里买下了维维亚尼藏书的剩余部分。正如内利在回忆录中所述:“我能如此便宜地获得这一无价之宝,完全是因为出售这些书籍的人对其价值一无所知。”  

这批藏书由内利保管,直到他1793年去世。他的儿子们意识到父亲手稿的价值,试图将剩余的藏书出售给法国政府。托斯卡纳大公费迪南多三世(Ferdinand III)干预了这次出售,并买下了整个藏书。这批手稿、印刷书籍和私人文献被存放在佛罗伦萨的帕拉蒂纳图书馆(Biblioteca Palatina),并于1861年与马利亚贝基亚纳图书馆合并。[251]  
\subsection{另见}  
\begin{itemize}
\item  天主教会与科学(Catholic Church and science)  
\item 秒摆(Seconds pendulum)  
\item 伽利略的纪念馆(Tribune of Galileo)  
\item 伊尔乔耶洛别墅(Villa Il Gioiello)
\end{itemize}  
\subsection{注释}  

\begin{itemize}
\item a. 即,肉眼不可见。  
\item b. 在卡佩拉模型中,只有水星和金星绕太阳运行,而在其扩展版本中,例如由里乔利(Riccioli)阐述的模型,火星也绕太阳运行,但木星和土星的轨道以地球为中心。  
\item c. 在地心体系中,太阳黑子运动的表观年度变化只能通过太阳自转轴的极其复杂的岁差来解释。[69][70][71] 然而,这一问题并不适用于由第谷的学生隆哥蒙塔努斯(Longomontanus)提出的修正版本体系中,该体系假设地球自转,并且可以像哥白尼体系一样解释太阳黑子的表观运动。  
\item d. 此类经文包括《诗篇》93:1、96:10 和《历代志上》16:30,其中提到“世界也已稳固,不得动摇”。同样,《诗篇》104:5 写道:“他(主)立定了地基,使地不再移动。”此外,《传道书》1:5 写道:“太阳升起,太阳落下,急忙回到升起之地。”《约书亚记》10:14 写道:“太阳停在基遍上……”[121]  
\item e. 詹姆斯·布拉德利(James Bradley)于1729年1月发现的光行差是地球运动的首个决定性证据,从而支持了阿里斯塔克斯、哥白尼和开普勒的理论;这一发现于1729年1月宣布。[122] 第二个证据由弗里德里希·贝塞尔(Friedrich Bessel)于1838年提出。  
\item f. 根据莫里斯·菲诺奇亚罗(Maurice Finocchiaro)的说法,这是出于好奇心,以友好和优雅的方式进行的。[124]  
\item g. 英戈利(Ingoli)写道,日心理论中恒星的巨大距离“清楚地表明……恒星的大小可能超过地球轨道圆的大小”。[130]  
\item h. 德雷克(Drake)认为,辛普利乔(Simplicio)的角色是以亚里士多德哲学家洛多维科·德勒·科隆贝(Lodovico delle Colombe)和切萨雷·克雷莫尼尼(Cesare Cremonini)为原型,而非教皇。[137] 他还认为,要求伽利略在《对话》中加入教皇的论点,使伽利略别无选择,只能将其放入辛普利乔之口。[138] 即使通常对伽利略颇为苛刻的阿瑟·凯斯特勒(Arthur Koestler)在《梦行者》(The Sleepwalkers)中也表示,教皇怀疑伽利略故意将辛普利乔塑造成教皇的讽刺形象,“这种说法是不正确的”。[139]
\end{itemize}  

\subsection{参考文献} 
\subsubsection{引文} \\
1. 《科学:终极视觉指南》(Science: The Definitive Visual Guide),DK出版,2009年,第83页。ISBN 978-0-7566-6490-9。\\  
2. 德雷克(Drake),1978年,第1页。\\ 
3. Modinos, A. (2013)。从亚里士多德到薛定谔:物理学的好奇心,本科物理讲义(From Aristotle to Schrödinger: The Curiosity of Physics, Undergraduate Lecture Notes in Physics),斯普林格出版社,第43页。ISBN 978-3-319-00750-2。\\  
4. Singer, C. (1941)。19世纪科学的简史(A Short History of Science to the Nineteenth Century),Clarendon出版社,第217页。\\ 
5. Whitehouse, D. (2009)。文艺复兴的天才:伽利略及其对现代科学的遗产(Renaissance Genius: Galileo Galilei & His Legacy to Modern Science),Sterling出版社,第219页。ISBN 978-1-4027-6977-1。\\
6. 《托马斯·霍布斯:批评性评估》(Thomas Hobbes: Critical Assessments),卷1,Preston King,1993年,第59页。\\
7. Disraeli, I. (1835)。文学的奇趣(Curiosities of Literature),W. Pearson & Co,第371页。\\
8. Valleriani, Matteo (2010)。伽利略工程师(Galileo Engineer),斯普林格出版社,第160页。ISBN 978-90-481-8644-0。\\
9. Hannam, J. (2009)。第329–344页。\\
10. Sharratt, M. (1994)。第127–131页。\\
11. Finocchiaro, M. (2010)。第74页。\\
12. Finocchiaro, M. (1997)。第47页。\\
13. Hilliam, D. (2005)。第96页。\\
14. Carney, J. E. (2000)。文艺复兴与宗教改革,1500–1620(Renaissance and Reformation, 1500–1620)。\\  
15. O'Connor, J. J.; Robertson, E. F.伽利略·伽利莱(Galileo Galilei)。数学史档案(*MacTutor History of Mathematics archive),圣安德鲁斯大学,苏格兰,2007年7月24日检索。\\
16. Gribbin, J. (2008),第26页。\\
17. Gribbin, J. (2008),第30页。\\
18. Gribbin, J. (2008),第31页。\\
19. Gribbin, J. (2009)。科学的历史,1543–2001(Science. A History. 1543–2001)。伦敦:企鹅出版集团,第107页。ISBN 978-0-14-104222-0。\\ 
20. Gilbert, N. W. (1963)。伽利略与帕多瓦学派(Galileo and the School of Padua)。哲学史杂志(Journal of the History of Philosophy),第1卷第2期,第223–231页。DOI: 10.1353/hph.2008.1474。ISSN 0022-5053。S2CID 144276512。\\
21. Sobel, D. (2000),第16页。\\
22. Robin Santos Doak,伽利略:天文学家与物学家(Galileo: Astronomer and Physicist),Capstone出版社,2005年,第89页。\\
23. Sobel, D. (2000),第13页。\\
24. 《伽利略的条目》(Galilean)。《世纪词典与百科全书》(The Century Dictionary and Encyclopedia),第3卷。纽约:世纪出版社,1903年(1889年),第2436页。  
25. Finocchiaro, M. (1989),第300页,第330页。\\
26. Sharratt, M. (1994),第17页,第213页。\\
27. Rosen, J.; Gothard, L. Q. (2009)。物理科学百科全书(Encyclopedia of Physical Science)。纽约:Infobase出版,第268页。ISBN 978-0-8160-7011-4。\\
28. Gribbin, J. (2008),第42页。\\
29. Sobel, D. (2000),第5页。\\
30. Pedersen, O. (1985). 伽利略的宗教信仰(Galileo's Religion)。收录于:Coyne, G.; Heller, M.; Życiński, J.(编辑),《伽利略事件:信仰与科学的会面》(The Galileo Affair: A Meeting of Faith and Science),梵蒂冈城:梵蒂冈天文台出版社,第75–102页。Bibcode:1985gamf.conf...75P。OCLC 16831024。\\
31. Reston, J. (2000),第3–14页。\\
32. Asimov, Isaac (1964). 阿西莫夫的科学与技术传记百科全书(Asimov's Biographical Encyclopedia of Science and Technology)。ISBN978-0-385-17771-9。\\  
33. Len Fisher (2016年2月16日)。伽利略、但丁·阿利吉耶里以及如何计算地狱的维度(Galileo, Dante Alighieri, and how to calculate the dimensions of hell)。澳大利亚广播公司(Australian Broadcasting Corporation)。2022年1月9日检索。\\
34. Ostrow, Steven F. (1996年6月)。奇戈利的圣母无染原罪与伽利略的月亮:17世纪早期罗马的天文学与圣母(Cigoli's Immacolata and Galileo's Moon: Astronomy and the Virgin in early seicento Rome)。MutualArt。2020年9月27日检索。\\ 
35. Panofsky, Erwin (1956). 伽利略作为艺术的批评家:审美态度与科学思想(Galileo as a Critic of the Arts: Aesthetic Attitude and Scientific Thought)。《Isis》, 47 (1): 3–15。doi:10.1086/348450。JSTOR 227542。S2CID 145451645。\\
36. Sharratt, M. (1994),第45–66页。\\
37. Rutkin, H. D. 伽利略、占星术与科学革命:另一种视角(Galileo, Astrology, and the Scientific Revolution: Another Look)。斯坦福大学科学与技术历史与哲学项目,2007年4月15日检索。\\
38. Battistini, Andrea (2018). 作为实践占星家的伽利略(Galileo as Practising Astrologer)。《天文学史期刊》(Journal for the History of Astronomy), 49 (3): 388–391。Bibcode:2018JHA....49..345。doi:10.1177/0021828618793218。S2CID 220119861。2020年12月30日检索。\\
39. Kollerstrom, N. (2004年10月)。伽利略与新星(Galileo and the new star)。《Astronomy Now》, 第18卷(第10期): 58–59页。Bibcode:2004AsNow..18j..58K。ISSN 0951-9726。2017年2月20日检索。\\
40. King, H. (2003),第30–32页。\\
41. Drake, S. (1990),第133–134页。\\
42. Sharratt, M. (1994),第1–2页。\\
43. Edgerton, S. (2009),第159页。\\
44. Edgerton, S. (2009),第155页。\\
45. Jacqueline Bergeron(编辑)(2013)。Highlights of Astronomy: As Presented at the XXIst General Assembly of the IAU, 1991,斯普林格科学与商业媒体,第521页。ISBN 978-94-011-2828-5。\\
46. Stephen Pumfrey (2009年4月15日)。哈里奥特的月球地图:新解读(Harriot's maps of the Moon: new interpretations)。《皇家学会纪事与记录》(*Notes and Records of the Royal Society), 第63卷(第2期): 163–168页。doi:10.1098/rsnr.2008.0062。\\
47. Drake, S. (1978),第146页。\\
48. Drake, S. (1978),第152页。\\
49. Sharratt, M. (1994),第17页。\\
50. Pasachoff, J. M. (2015年5月)。西蒙·马里乌斯的《木星的世界》:伽利略阴影下的400周年纪念(Simon Marius's Mundus Iovialis: 400th Anniversary in Galileo's Shadow)。《天文学史期刊》(Journal for the History of Astronomy),第46卷(第2期): 218–234页。Bibcode:2015JHA....46..218P。doi:10.1177/0021828615585493。S2CID 120470649。\\
51. Linton, C. M. (2004),第98页,第205页。\\
52. Drake, S. (1978),第157页。\\
53. Drake, S. (1978),第158–168页。\\
54. Sharratt, M.(1994),第18–19页。\\
55. Feyerabend, P. (1975),第88–89页。\\
56. Naess, A. (2004),第57页。\\
57. Hannam, J. (2009),第313页。\\
58. Drake, S. (1978),第168页。\\
59. Sharratt, M. (1994),第93页。\\
60. Edwin Danson (2006). 称量世界(Weighing the World),牛津大学出版社。ISBN 0-19-518169-7。\\
61. 解决经度问题:木星的卫星(Solving Longitude: Jupiter's Moons),格林尼治皇家博物馆,2014年10月16日。\\
62. Thoren, V. E. (1989),第8页。\\
63. Hoskin, M. (1999),第117页。\\
64. Cain, Fraser (2008年7月3日)。土星的历史(History of Saturn),《今日宇宙》(Universe Today)。原文存档于2012年1月26日。2020年10月5日检索。\\
65. Baalke, Ron.土星环的历史背景(Historical Background of Saturn's Rings),喷气推进实验室,加州理工学院,NASA,2007年3月11日检索。\\
66. Drake, S.; Kowal, C. T. (1980)。\\
67. Vaquero, J. M.; Vázquez, M. (2010)。历史记录中的太阳(The Sun Recorded Through History),斯普林格出版社。第2章,第77页:“伽利略用肉眼观察到的大型太阳黑子的图像,显示在1612年8月19日、20日和21日的观察记录中,每个人都看到了相同的现象。” \\
68. Drake, S. (1978),第209页。\\
69. Linton, C. M. (2004),第212页。\\
70. Sharratt, M. (1994),第166页。\\
71. Drake, S. (1970),第191–196页。\\
72. Gribbin, J. (2008),第40页。\\
73. Ondra, L. (2004),第72–73页。\\
74. Graney, C. M. (2010),第455页。\\
75. Graney, C. M. & Grayson, T. (2011),第353页。\\
76. Van Helden, A. (1985),第75页。\\
77. Chalmers, A. F. (1999),第25页。\\
78. Galilei, G. (1953),第361–362页。\\ 
79. Finocchiaro, M. A. (1989),第167–176页。\\
80. Galilei, G. (1953),第359–360页。\\
81. Ondra, L. (2004),第74–75页。\\
82. Graney, C. M. (2010),第454–462页。\\
83. Graney, C. M. & Grayson, T.(2011),第352–355页。\\
84. Finocchiaro, M. A. (1989),第67–69页。\\
85. Naylor, R. (2007)。伽利略的潮汐理论(Galileo's Tidal Theory)。《Isis》, 第98卷(第1期): 1–22页。Bibcode:2007Isis...98....1N。doi:10.1086/512829。PMID 17539198。S2CID 46174715。\\
86. Finocchiaro, M. A. (1989),第354页。\\
87. Finocchiaro, M. A. (1989),第119–133页。\\
88. Finocchiaro, M. A. (1989),第127–131页。\\
89. Galilei, G. (1953),第432–436页。\\
90. Einstein, A. (1953),序言第xvii页。\\
91. Galilei, G. (1953),第462页。\\
92. Voelkel, J. R. (2001)。开普勒《新天文学》的构成(The Composition of Kepler's Astronomia Nova)。普林斯顿大学出版社,第74页。\\
93. Stillman Drake. 关于伽利略及科学史与哲学的论文,第1卷(Essays on Galileo and the History and Philosophy of Science, Volume 1)。多伦多大学出版社,1999年,第343页。\\
94. 关于两个世界体系的对话(Dialogue Concerning the Two Chief World Systems),第四giornata。\\
95. 作为科学家的职业 | 伽利略·伽利莱(1564–1642)(Career as a Scientist | Galileo Galilei (1564–1642))。www.storiespreschool.com,2023年8月31日检索。\\
96. Drake, S. (1960),第vii页,xxiii–xxiv页。\\
97. Sharratt, M. (1994),第139–140页。\\
98. Grassi, O. (1960a)。\\
99. Drake, S. (1978),第268页。\\
100. Grassi, O. (1960a),第16页。\\
101. Galilei, G.; Guiducci, M. (1960)。\\
102. Drake, S. (1960),第xvi页。\\
103. Drake, S. (1957),第222页。\\
104. Drake, S.(1960),第xvii页。\\
105. Sharratt, M. (1994),第135页。\\
106. Drake, S. (1960),第xii页。\\
107. Galilei, G.; Guiducci, M. (1960),第24页。\\
108. Grassi, O. (1960b)。\\
109. Drake, S. (1978),第494页。\\
110. Sharratt, M. (1994),第137页。\\
111. Drake, S. (1957),第227页。\\
112. Sharratt, M. (1994),第138–142页。\\
113. Drake, S. (1960),第xix页。\\
114. Alexander, A. (2014)。无穷小:一种危险的数学理论如何塑造现代世界(Infinitesimal: How a Dangerous Mathematical Theory Shaped the Modern World)。《科学美国人》(*Scientific American)/ Farrar, Straus and Giroux,第131页。ISBN 978-0-374-17681-5。\\
115. Drake, S. (1960),第vii页。\\
116. Sharratt, M. (1994),第175页。\\
117. Sharratt, M. (1994),第175–178页。\\
118. Blackwell, R. (2006),第30页。\\
119. Hannam, J. (2009),第303–316页。\\
120. Blackwell, R. (1991)。伽利略、贝拉明与《圣经》(Galileo, Bellarmine, and the Bible)。圣母大学出版社,第25页。ISBN 978-0-268-01024-9。\\
121. Brodrick, J. (1965),第95页。\\
122. Bradley, James (1728)。来自牛津大学詹姆斯·布拉德利萨维兰天文学教授致埃德蒙·哈雷博士的信:关于固定星体新发现运动的报告(A Letter ... Giving an Account of a New Discovered Motion of the Fix'd Stars)。《皇家学会哲学会刊》(Philosophical Transactions of the Royal Society of London),第35卷:637–661页。\\
123. Graney, C. M. & Danielson, D. R. (2014)。\\
124. Finocchiaro, M. A. (1989),第27–28页。\\
125. Finocchiaro, M. A. (1989)。\\
126. Langford, J. A. (1998),第56–57页。\\
127. Finocchiaro, M. A. (1989),第28页,第134页。\\
128. Graney, C. M. (2015),第68–69页。\\
129. Finocchiaro, M. A. (2010),第72页。\\
130. Graney, C. M. (2015),第71页。\\
131. Graney, C. M. (2015),第66–76页,164–175页,187–195页。\\
132. Finocchiaro, M.西切斯特大学——天文学史;讲义笔记:选自《伽利略事件:文献历史》(*West Chester University – History of Astronomy; Lecture notes: Texts from The Galileo Affair: A Documentary History)。西切斯特大学,ESS 362/562。原文存档于2007年9月30日。2014年2月18日检索。\\
133. Heilbron, J. L. (2010),第218页。\\
134. 教皇乌尔班八世传记(Pope Urban VIII Biography)。伽利略项目(Galileo Project)。\\
135. Finocchiaro, M. A. (1997),第82页。\\ 
136. Moss, J. D. & Wallace, W. A. (2003),第11页。\\
137. Drake, S. (1978),第355页。\\
138. Drake, S. (1953),第491页。\\
139. Koestler, A. (1990),第483页。\\
140. Lindberg, D.超越战争与和平:基督教与科学相遇的再评价(Beyond War and Peace: A Reappraisal of the Encounter between Christianity and Science*)。\\
141. Sharratt, M. (1994),第171–175页。\\
142. Heilbron, J. L. (2010),第308–317页。\\
143. Gingerich, O. (1992),第117–118页。\\
144. Numbers, R. L.(编)。《伽利略入狱及科学与宗教的其他神话》(*Galileo Goes to Jail and Other Myths About Science and Religion*)。哈佛大学出版社,2009年,第77页。 \\
145. Fantoli, A. (2005),第139页。\\
146. Finocchiaro, M. A. (1989),第288–293页。\\
147. Fantoli, A. (2005),第140页。\\
148. Heilbron, J. L. (2005),第282–284页。\\
149. Finocchiaro, M. A. (1989),第38页,第291页,第306页。\\
150. 伽利略·伽利莱(Galileo Galilei),《斯坦福哲学百科全书》(Stanford Encyclopedia of Philosophy),简要传记。\\
151. Drake, S. (1978),第367页。\\
152. Sharratt, M. (1994),第184页。\\
153. Drake, S. (1978),第356–357页。\\
154. Livio, M. (2020)。伽利略真的说过“然而它在转动”吗?现代侦探故事("Did Galileo Truly Say, 'And Yet It Moves'? A Modern Detective Story")。《伽利略期刊》(Galilaeana),第XVII卷(第17期):第289页。doi:10.1400/280789。\\
155. Shea, W. (2006年1月)。伽利略事件(The Galileo Affair)(未发表的作品)。科学、理性与信仰研究组(CRYF)。2010年9月12日检索。\\
156. “伽利略……是现代物理学——甚至现代科学的奠基人” ——阿尔伯特·爱因斯坦,引用于斯蒂芬·霍金编《站在巨人的肩膀上》(On the Shoulders of Giants),第398页。\\
157. Sobel, D. (2000),第232–234页。\\
158. Livio, M. (2020)。伽利略与科学否认者(Galileo and the Science Deniers)。纽约:Simon & Schuster。ISBN 978-1-5011-9473-3。\\
159. Gerard, J. (1913)。伽利略·伽利莱(Galileo Galilei),收录于Herbermann, Charles(编)。《天主教百科全书》(Catholic Encyclopedia)。纽约:Robert Appleton Company。\\
160. Shea, W.; Artigas, M. (2003),第199页。\\
161. Sobel, D. (2000),第378页。\\
162. Sharratt, M. (1994),第207页。\\
163. 伽利略的纪念性陵墓(Monumental tomb of Galileo)。佛罗伦萨科学史研究所与博物馆。2010年2月15日检索。\\
164. Sobel, D. (2000),第380页。\\
165. Shea, W.; Artigas, M. (2003),第200页。\\
166. Sobel, D. (2000),第380–384页。\\
167. 第七室伽利略图像与遗物区(Section of Room VII Galilean iconography and relics),伽利略博物馆。2011年5月27日在线访问。\\
168. 伽利略右手的中指*(*Middle finger of Galileo's right hand),伽利略博物馆。2011年5月27日在线访问。\\
169. Sharratt, M. (1994),第204–205页。\\
170. Cohen, H. F. (1984)。量化音乐:音乐的科学*(*Quantifying Music: The Science of Music at)。斯普林格出版社,第78–84页。ISBN 978-90-277-1637-8。\\
171. Field, J. V. (2005)。皮耶罗·德拉·弗朗切斯卡:数学家的艺术(Piero Della Francesca: A Mathematician's Art)。耶鲁大学出版社,第317–320页。ISBN 978-0-300-10342-7。\\
172. Drake, S. (1957),第237–238页。\\
173. Wallace, W. A. (1984)。\\
174. Sharratt, M. (1994),第202–204页。\\
175. Galilei, G. (1954),第250–252页。\\
176. Favaro, A. (1890),第274–275页。\\
177. Galilei, G. (1954),第252页。\\
178. Favaro, A. (1890),第275页。\\
179. 极光被伽利略命名于1619年(The Aurora Borealis was named by Galileo in 1619)。《BBC天气》(BBC Weather)。2017年10月25日,2023年12月7日检索。\\
180. Reston, J. (2000),第56页。\\
181. Sobel, D. (2000),第43页。\\
182. Drake, S. (1978),第196页。\\
183. Rosen, E. (1947)。望远镜的命名(The Naming of the Telescope)。\\
184. Drake, S. (1978),第163–164页。\\
185. Favaro, A. (1890),第163页。\\
186. Drake, S. (1978),第289页。\\
187. Drake, S. (1978),第286页。\\
188. 伽利略的显微镜(Il microscopio di Galileo),来源于Brunelleschi网站。2008年4月9日存档。\\
189. Van Helden, A. 伽利略时间线(Galileo Timeline)。伽利略项目(Galileo Project),1995年最后更新,2007年8月28日检索。\\
190. 经度:一个孤独天才解决他时代最大科学问题的真实故事(Longitude: the true story of a lone genius who solved the greatest scientific problem of his time),Dava Sobel,企鹅出版社,1996年。ISBN 978-0-14-025879-0。\\
191. Cesare S. Maffioli (2008)。伽利略、Guiducci与工程师Bartolotti在比森齐奥河的工作(Galileo, Guiducci and the Engineer Bartolotti on the Bisenzio River)。发表于academia.edu,Galilaeana (V)。2017年8月11日检索。\\
192. Justin Corfield (2014)。菲利普·沃恩(1794)(Vaughan, Philip (fl. 1794))。收录于Kenneth E. Hendrickson III(编)。世界历史中的工业革命百科全书(The Encyclopedia of the Industrial Revolution in World History),第3卷。兰汉姆(美国马里兰州):Rowman & Littlefield,第1008页。ISBN 978-0-8108-8888-3。  
> “沃恩仍被认为是滚珠轴承的发明者,尽管...一些尼米船(Nemi ships)可追溯至公元40年,已经将它们纳入设计,而达·芬奇则被认为首先提出了滚珠轴承的原理,尽管他并未在自己的发明中使用该原理。另一位意大利人伽利略描述了捕捉滚珠的用法。”\\
193. Galilei, Galileo; Drabkin, I. E.; Drake, Stillman (1960)。论运动与力学(On Motion and On Mechanics)。麦迪逊:威斯康星大学出版社,第108页。\\
194. Newton, R. G. (2004)。伽利略的摆:从时间节奏到物质的制造(Galileo's Pendulum: From the Rhythm of Time to the Making of Matter*)。哈佛大学出版社,第51页。ISBN 978-0-674-01331-5。\\
195. Gillespie, C. C. (1960)。客观性边缘:科学思想史中的一篇随笔(The Edge of Objectivity: An Essay in the History of Scientific Ideas)。普林斯顿大学出版社,第99–100页。\\
196. Galileo Galilei,两种新科学(Two New Sciences),(麦迪逊:威斯康星大学出版社,1974),第50页。\\
197. I. Bernard Cohen,罗默与首次测定光速(Roemer and the First Determination of the Velocity of Light (1676)),《Isis》,第31卷(1940):第327–379页。\\
198. 卢克莱修,《物性论》(De rerum natura),第II卷,第225–229页;相关段落见:Lane Cooper,《亚里士多德、伽利略与比萨斜塔》(Aristotle, Galileo, and the Tower of Pisa),伊萨卡,纽约州:康奈尔大学出版社(Cornell University Press),1935年,第49页。\\
199. Hannam, J.(2009). 第305–306页。\\
200. Lemons, Don S.,《绘制物理学:2600年的探索,从泰勒斯到希格斯》(*Drawing Physics: 2,600 Years of Discovery From Thales to Higgs*),麻省理工学院出版社(MIT Press),2017年,第80页。\\
201. Clagett, M. (1968). 第561页。\\
202. Grant, E. (1996). 第103页。\\
203. Sharratt, M. (1994). 第198页。\\
204. 西蒙·斯蒂文,《水静力学原理》(De Beghinselen des Waterwichts),包含水静力学的练习序言及水静力学原理和称量法的附录,荷兰莱顿:Christoffel Plantijn,1586年。该书中记载了斯蒂文与扬·科尔内茨·德·格鲁特(Jan Cornets de Groot)从代尔夫特教堂塔上投掷铅球的实验;相关段落见:E.J. Dijksterhuis(编),《西蒙·斯蒂文的主要作品》(The Principal Works of Simon Stevin),荷兰阿姆斯特丹:C.V. Swets & Zeitlinger,1955年,第1卷,第509、511页。\\
205. Drake, Stillman(1978). 第19–20页。\\
206. Drake, Stillman (1978). 第9页。\\
207. Sharratt, M. (1994). 第31页。\\
208. Groleau, R.,《伽利略的天体之战》(Galileo's Battle for the Heavens),2002年7月,PBS。Ball, P. (2005年6月30日)。*科学历史:纠正记录,2005年6月30日,发表于《印度报》(The Hindu),金奈。于2014年6月20日从原始文献存档检索,2007年10月31日访问。\\
