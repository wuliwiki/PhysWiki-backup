% 欧拉函数(数论)
% keys 数论|欧拉函数
% license Usr
% type Tutor

\pentry{数论函数\nref{nod_NumFun}}{nod_24cb}

前面在数论函数中已经提到过欧拉函数,但欧拉函数相关内容很丰富,下面做一些展开。
首先回顾欧拉函数(Euler's Function/Euler's Totient Function)的定义:
\begin{definition}{欧拉函数}
欧拉函数 $\varphi(n)$ 表示 $n$ 以内的非零自然数中与 $n$ 互质的数的个数。也可以表示为,
$$\varphi(n) = \sum_{1 \le d \le n, \gcd(d, n) = 1} 1  ~.$$
其中 $\gcd$ 为求最大公约数,故可以用 $\gcd(d, n)=1$ 表示要求 $d$ 与 $n$ 互质。
\end{definition}

\subsection{素数与素数幂次的欧拉函数}
考察一个数论函数,我们经常按如下顺序考察:
\begin{enumerate}
\item 素数,
\item 素数幂,
\item 非零自然数。
\end{enumerate}
