% 无量纲的物理公式
% keys 量纲|公式简化

\pentry{万有引力\upref{Gravty}}

与数学公式不同,物理公式中的物理量通常既包含数值又包含量纲,如 $x = 10\Si{cm}$. 但有时候(例如数值计算时)我们需要将含有量纲的公式转换为不含量纲的物理公式(因为计算机程序中的变量不含量纲),且公式中的常量(如万有引力常数 $G$)越少越好以尽可能减少计算量. 我们通过几个例子来说明转换过程.

\begin{example}{牛顿第二定律}\label{NoUnit_ex1}
国际单位制下的牛顿第二定律公式为
\begin{equation}
F = ma
\end{equation}
我们先定义几个含量纲的常量例如 $\beta_{F} = 1\Si{N}$,$\beta_{m} = 1\Si{kg}$,$\beta_{a} = 1\Si{m/s^2}$. 把不含量纲的的力,质量和加速度分别为 $F_a, m_a, a_a$, 则有 $F = F_a \beta_F$, $m = m_a \beta_{m}$, $a = m_a \beta_{a}$. 代入上式得
\begin{equation}
F_a = \frac{\beta_m \beta_a}{\beta_F} m_a  a_a
\end{equation}
由于国际单位定义 $1\Si{N} = 1\Si{kg}\cdot\Si{m/s^2}$, 上式中 $\beta_m \beta_a/\beta_F = 1$, 所以不含量纲的牛顿第二定律为
\begin{equation}\label{NoUnit_eq3}
F_a = m_a a_a
\end{equation}
以上的做法看起来似乎并没有什么意义, 这是因为我们把每个常量 $\beta$ 都定义为一个相应的国际单位. 事实上,只要保证 $\beta_m \beta_a/\beta_F = 1$, 这三个常量是可以任取的. 例如令 $\beta_m = 1\Si{g}$, $\beta_a = 1\Si{cm/s^2}$, $\beta_F = 10^{-5}\Si{N}$,上式仍然成立.
\end{example}

\begin{example}{万有引力公式}
国际单位下的万有引力公式为
\begin{equation}
F = G\frac{Mm}{r^2}
\end{equation}
其中 $G \approx 6.674\times 10^{-11} \Si{m^3 kg^{-1} s^{-2}}$. 在\autoref{NoUnit_ex1} 的基础上定义 $r = \beta_x r_a$,则有
\begin{equation}
F_a = \frac{G\beta_m^2}{\beta_F \beta_x^2} \frac{M_a m_a}{r_a^2}
\end{equation}
我们若想让新的不含量纲的万有引力公式也不含 $G$ 以减少计算量,即
\begin{equation}\label{NoUnit_eq6}
F_a = \frac{M_a m_a}{r_a^2}
\end{equation}
只需令所有的 $\beta$ 满足 $G\beta_m^2/(\beta_F\beta_x^2) = 1$, 例如 $\beta_x = 1\Si{m}$, $\beta_m = 1.224\times 10^5 \Si{kg}$, $\beta_F = 1\Si{N}$.
\end{example}

注意无量纲的物理公式和有量纲的物理公式(无论用什么量纲)存在本质的不同, 例如在\autoref{NoUnit_eq3} 中, 如果受力物体的质量恰好等于 $\beta_m$, 那么公式可以直接写为
\begin{equation}\label{NoUnit_eq7}
F_a = a_a
\end{equation}
而这种写法对含量纲的公式来说是错误的, 因为量纲不同的两个物理量不可以相等(或相加).

\subsection{无量纲公式转换为含量纲公式}
对某个物理量 $Q$, 有
\begin{equation}
Q = \beta_Q Q_a
\end{equation}
要把无量纲公式变为含量纲公式, 就把式中所有 $Q_a$ 用 $Q/\beta_Q$ 替换即可. 以\autoref{NoUnit_eq6} 为例, 替换后得
\begin{equation}
F = \frac{\beta_F\beta_x^2}{\beta_m^2} \frac{Mm}{r^2} = G\frac{Mm}{r^2}
\end{equation}
