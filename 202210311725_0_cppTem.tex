% C++ 模板笔记

\begin{issues}
\issueDraft
\end{issues}

\begin{itemize}
\item template function 的自变量不允许自动类型转换 (除非用 \verb`<>` 声明 template 类型)
\item 如果 function template 的 return type 是 template type 且没有出现在 parameter list, 那么使用该函数的时候就必须要在函数名后面加 \verb`<>` 至少声明 return type
\item 在函数后面加 \verb`<>` 声明 template parameter 的时候, 可以只声明前几个, 把剩下的留给编译器 deduce.
\item 如果 Derived class template 中 overload 了 Base class template 的 member function, 那么需要声明 \verb`using <Base>::<fun>` 才可以使用 Base class template 中的 overload 函数.
\item \verb`if constexpr` 加上 \verb`<typetrait>` 可以对不同的类型执行不同的代码 (没有 overhead).

\subsection{Template Function Overloading}
【这里的笔记过时了, 用 SFINAE 就好】
在普通的 function overloading 的基础上:
\item 所有可以 instantiate 的 template 函数也都是 viable 的.
\item 如果存在多于一个 best match, 就先在其中选非 template 的 function, 如果没有就选最 spetialized template function, 如果还没有就报错 (ambiguous).
\item 什么样的含有 template type 的 function argument 算是更 specialized 的呢? 目前已知 \verb|T *| 和 \verb|vector<T>| 肯定比 T 更 specialled. 猜测大概是能接受的变量类型越少越 specialize.

\subsubsection{Meta programming}
\item constexpr 函数是在编译时计算的, 其实就是 meta-programming.
\item 而且 constexpr 函数既可以用在编译时也可以用在运行时.
\item \verb`if constexpr` 加上 \verb`<typetrait>` 可以对不同的类型执行不同的代码 (没有 overhead).
\item \verb`<typetrait>` 里面有用的模板有 \verb`is_float<T>`, \verb`is_same<T1, T2>`
\begin{lstlisting}[language=cpp]
template <class T, class T1>
inline void divide_equals_vs(T *v, const T1 &s, Long_I N)
{
	constexpr std::is_floating_point<T> is_float;
	if constexpr (is_float) {
		times_equals_vs(v, 1./s, N);
	}
	else {
		for (Long i = 0; i < N; ++i)
			v[i] /= s;
	}
}
\end{lstlisting}
\item constexpr 函数和 constructor 会自动 inline, 即可以无视 one definition rule.
\item \verb`if constexpr()` 是编译时完成的, 相当于 \verb`#ifdef`, \verb`#endif` 导致的条件编译, 所以没有编译到的部分即使对当前的情况有语法错误也没有关系. 例如下面这段 SLISC 中的代码, 当 T1, T2 是 Vector<> 或 Matrix<> 的时候, 即使 \verb`dim1()`, \verb`dim2()`, \verb`dim3()` 没有定义也不会出错!
\begin{lstlisting}[language=cpp]
template <class T1, class T2>
Bool shape_cmp(const T1 &v1, const T2 &v2)
{
	if constexpr (T1::ndims() == 1 && T2::ndims() == 1) {
		return v1.size() == v2.size();
	}
	else if constexpr (T1::ndims() == 2 && T2::ndims() == 2) {
		return v1.nrows() == v2.nrows() && v1.ncols() == v2.ncols();
	}
	else if constexpr (T1::ndims() == 3 && T2::ndims() == 3) {
		return v1.dim1() == v2.dim1() && v1.dim2() == v2.dim2()
			&& v1.dim3() == v2.dim3();
	}
	return false;
}
\end{lstlisting}
\item 感觉在 class 里面定义 \verb`enum{};` 的做法完全可以被 \verb`static constexpr` 函数取代.
\begin{lstlisting}[language=cpp]
constexpr long myfac(long i)
{
	if (i == 0)
		return 1;
	return myfac(i - 1)*i;
}

int main()
{
	constexpr long i = myfac(4);
	int a[i]; // totally ok!
	long j = 5;
	long k = myfac(j); // still ok!
}
\end{lstlisting}
\item \verb`std::common_type<T1, T2, ...>::type` 返回一个共同的类型 \verb`T` 使得所有的 \verb`T1, T2...` 都可以 implicitly 转换成 T. 例如可以定义两种类型的加法为
\begin{lstlisting}[language=cpp]
template <class T1, class T2>
std::common_type<T1, T2, ...>::type myplus(const T1 &x, const T2 &y)
{
	return x + y;
}
\end{lstlisting}
该函数 argument 如果输入 \verb`complex<double>` 和 \verb`double` 就可以返回前者, 输入 \verb`double` 和 \verb`int` 也可以返回前者.
\item 模板也可以有默认变量, 或者任意个数变量 \verb|...|
\end{itemize}
