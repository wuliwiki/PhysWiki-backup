% 电流
% 电流|电荷|截面

\begin{issues}
\issueDraft
\end{issues}

\pentry{导数\upref{Der}}
\begin{figure}[ht]
\centering
\includegraphics[width=10cm]{./figures/I_1.pdf}
\caption{电流的经典模型:电流是导线中载流子(电荷)的定向运动.本图以正电荷为载流子.} \label{I_fig1}
\end{figure}

若在导线中取一个任意横截面,并定义一个正方向,通过该截面的电流定义为
\begin{equation}\label{I_eq1}
I = \dv{q}{t}
\end{equation}
其中 $\dd{q}$ 为 $\dd{t}$ 时间内由正方向通过截面的净电荷量\footnote{具体来说,正电荷向正方向流动和负电荷向负方向流动,电流都取正号,反之取负号.}.

考虑一根电荷线密度为 $\lambda$ 的导线中电荷都以速度 $v$ 沿正方向运动,时间 $\dd{t}$ 内通过的长度为 $v\dd{t}$,通过的电荷为 $\lambda v\dd{t}$,所以电流为
\begin{equation}
I = \lambda v
\end{equation}
