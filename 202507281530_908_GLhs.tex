% 格林函数(综述)
% license CCBYSA3
% type Wiki

本文根据 CC-BY-SA 协议转载翻译自维基百科\href{https://en.wikipedia.org/wiki/Green\%27s_function}{相关文章}。

\begin{figure}[ht]
\centering
\includegraphics[width=8cm]{./figures/36e3d9728324115d.png}
\caption{如果我们已知某个微分方程在点源作用下的解 $G(x, x')$,即满足$\hat{L}(x)G(x,x') = \delta(x - x')$的格林函数,其中 $\hat{L}(x)$ 是一个线性微分算子,那么我们可以通过叠加原理构造出任意源项 $f(x)$ 下的解:$u(x) = \int f(x') G(x, x')\, dx'$从而求得方程$\hat{L}(x) u(x) = f(x)$的解。简而言之,如果你知道格林函数,就可以用它通过积分方式表示出任意源函数 $f(x)$ 下的解 $u(x)$。} \label{fig_GLhs_1}
\end{figure}
在数学中,格林函数(Green's function,亦称 Green 函数)是定义在某一给定初始条件或边界条件下的非齐次线性微分算子的冲激响应。

这意味着,如果 $L$ 是一个线性微分算子,那么:
\begin{itemize}
\item 格林函数$G$ 是满足方程 $LG = \delta$ 的解,其中 $\delta$ 是狄拉克δ函数;
\item 初值问题 $Ly = f$ 的解为 $G * f$(即格林函数与 $f$ 的卷积)。
\end{itemize}
通过叠加原理,对于一个线性常微分方程(ODE)$Ly = f$,可以先对每个源点 $s$ 求解 $LG = \delta_s$,由于源项可以看作多个δ函数的叠加,根据线性算子 $L$ 的线性性,最终的解也是各个格林函数的线性叠加。

格林函数得名于英国数学家乔治·格林,他在 1820 年代首次提出了这一概念。在现代关于线性偏微分方程的研究中,格林函数更多被视为基本解的一种方式来研究。

在多体理论中,这一术语也广泛用于物理学,特别是在量子场论、空气动力学、空气声学、电动力学、地震学和统计场论中,用来表示各种类型的关联函数,即便它们并不总是符合严格的数学定义。在量子场论中,格林函数还扮演着传播子的角色。

