% 潮汐力

\begin{issues}
\issueDraft
\end{issues}
\begin{figure}[ht]
\centering
\includegraphics[width=8cm]{./figures/Tidal_1.pdf}
\caption{潮汐力分布示意图(地球参考系)} \label{Tidal_fig1}
\end{figure}

我们先以月球对地球的潮汐力为例.由于月球到地球各点的距离不同,根据$$F_\text{引力}=\frac{GMm}{r^2}$$,所以月球对地球各点的引力也是不同的.潮汐现象正是由于引力的这种细微差异.

\begin{figure}[ht]
\centering
\includegraphics[width=12cm]{./figures/Tidal_2.pdf}
\caption{月球对地球各点的引力不同} \label{Tidal_fig2}
\end{figure}
如\autoref{Tidal_fig1} 所示,正对月球的地球表面A点、地球质心、背对月球的B点感受到的


绘制地球表面潮汐力分布的octave/matlab代码:
\begin{lstlisting}[language=matlab]
clc
clear

[x y z] = sphere(20);
xs = -60;
scatter3(xs,0,0);

u = x - xs;
v = y;
w = z;

r = sqrt(u.^2+v.^2+w.^2);
mag = 10^4./(r.^2);

u = -mag.*u./r;
v = -mag.*v./r;
w = -mag.*w./r;
u = u + 10^4/(xs^2); %地球参考系这一非惯性参考系中的离心力

hold on
axis equal
axis off
surf(x,y,z,'FaceColor','none','EdgeColor',[0.8 0.8 0.8]);
axis([-1.2 1.2 -1.2 1.2 -1.2 1.2])
quiver3(x,y,z,u,v,w);
view(-30,30);
hold off

\end{lstlisting}