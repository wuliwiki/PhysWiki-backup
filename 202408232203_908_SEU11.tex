% 东南大学 2011 年 考研 量子力学
% license Usr
% type Note

\textbf{声明}:“该内容来源于网络公开资料,不保证真实性,如有侵权请联系管理员”

\textbf{1.(15 分)}以下叙述是否正确:

\begin{enumerate}
    \item  仅当体系处在定态时,守恒量的平均值才不随时间变化;
    \item  若厄密算符 $\hat{A}$ 与 $\hat{B}$ 不对易,则它们一定没有共同本征态;
    \item  一维谐振子的量子态空间是无穷维的;
    \item  如体系处于力学量 $\hat{A}$ 的本征态,则测量 $\hat{A}$ 必会得到一个确定值;
    \item  空间平移对称性导致宇称守恒。
\end{enumerate}

\textbf{2.(15 分)}质量为 $m$ 的粒子处于一维无限深方势井中,$V(x) = 0$,$(|x| < a/2)$;$V(x) = \infty$,$(|x| > a/2)$。试求能量本征值和归一化的本征函数。

\textbf{3.(15 分)}一维谐振子的哈密顿量为 
$$\hat{H} = \hat{p}^2/2m + m\omega^2 \hat{x}^2/2,~$$ 
定义
$$\hat{a} = (\alpha \hat{x} + i \hat{p}/\hbar\alpha)/\sqrt{2}, \quad (\alpha = \sqrt{m \omega / \hbar}).~$$,试证明:
\begin{enumerate}
    \item  $[\hat{a}, \hat{a}^\dagger] = 1$;
    \item  $\hat{H} = (\hat{N} + 1/2)\hbar \omega$,其中 $\hat{N}$ 定义为 $\hat{N} = \hat{a}^\dagger \hat{a}$;
    \item  若 $|\hat{n}\rangle$ 为 $\hat{N}$ 的本征态,即 $\hat{N}|\hat{n}\rangle = n|\hat{n}\rangle$,则 $\hat{a}^\dagger|\hat{n}\rangle$ 也是 $\hat{N}$ 的本征态,并求相应的本征值。
\end{enumerate}

\textbf{4.(15 分)}一维谐振子哈密顿量为 
$$\hat{H} = \frac{\hat{p}^2}{2m} + \frac{m\omega^2 \hat{x}^2}{2},~$$ 
定义 $\hat{X}(t) = e^{i\hat{H}t/\hbar} \hat{x} e^{-i\hat{H}t/\hbar}$,试证明:
\begin{enumerate}
    \item  $\frac{d\hat{X}(t)}{dt} = \frac{\hat{P}(t)}{m}, \frac{d\hat{P}(t)}{dt} = -m\omega^2 \hat{X}(t)$;
    \item  $\hat{X}(t) = \hat{x} \cos(\omega t) + \frac{\hat{p}}{m\omega} \sin(\omega t)$;
    \item  $\hat{P}(t) = \hat{p} \cos(\omega t) - m\omega \hat{x} \sin(\omega t)$。
\end{enumerate}

\textbf{5.(15 分)}设 $|\hat{n}\rangle$ 是某体系的一个非简并的能量本征态,试证明 $|\hat{n}\rangle$ 一定是该体系所有守恒量的共同本征态。