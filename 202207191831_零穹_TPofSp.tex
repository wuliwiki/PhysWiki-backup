% 空间的张量积
% 空间的张量积


\pentry{张量积\upref{TsrPrd}}
矢量空间的张量积在微分几何,群表示论,数学物理中有各式各样的作用.
\subsection{矢量空间的张量积}
\begin{definition}{空间的张量积}\label{TPofSp_def1}
设 $V_1,V_2,T$ 是域 $\mathbb F$ 上的矢量空间,$\sigma:V_1\times V_2\rightarrow T$ 是由 $V_1,V_2$ 是个双线性映射.若对域 $\mathbb F$ 上任意矢量空间 $U$ ,任一双线性映射 $\varphi:V_1\times V_2\rightarrow U$, 都存在唯一的线性映射 $\psi:T\rightarrow U$ 使
\begin{equation}
\varphi=\psi\sigma
\end{equation}
那么对 $(\sigma,T)$ 就称为 $V_1$ 与 $V_2$ 的\textbf{张量积}.
\end{definition}

\begin{lemma}{}\label{TPofSp_lem2}
设 $V_1,V_2$ 是域 $\mathbb F$ 上的矢量空间,$V_1$ 的维数是 $n$,$e_1,\cdots,e_n$ 是 $V_1$ 的一组基.对 $V_1$ 中任意 $n$ 个元素 $\alpha_1,\cdots,\alpha_n$ ,存在唯一的线性映射 $f:V_1\rightarrow V_2$ 使
\begin{equation}
f(e_i)=\alpha_i,\quad i=1,\cdots,n
\end{equation}
\end{lemma}
\textbf{证明:}若有另一线性映射 $f_1$,使得
\begin{equation}
f_1(e_i)=\alpha_i,\quad i=1,\cdots ,n
\end{equation}
那么对任意 $v=\sum_\limits{i}v^ie_i\in V_1$,有
\begin{equation}
f_1(v)=\sum_{i}v^if_1(e_i)=\sum_{i}v^i\alpha_i=\sum_{i}v^if(e_i)=f(v)
\end{equation}
即 $f_1=f$.

\textbf{证毕!}
\begin{lemma}{}\label{TPofSp_lem1}
设 $V_1,V_2$ 分别是 $n$ 维和 $m$ 维的矢量空间, $\{\mu^i\}$,$\{\nu^i\}$ 分别是 $V_1^*,V_2^*$ 的基,那么
\begin{equation}\label{TPofSp_eq1}
\mu^i\otimes\nu^j,\quad i=1,\cdots,n,\quad j=1,\cdots,m
\end{equation}
 是 $\mathcal L(V_1,V_2;\mathbb F)$ \upref{MulMap}的一组基.
\end{lemma}
\textbf{证明:}先证明 \autoref{TPofSp_eq1} 之间线性无关:若
\begin{equation}
\sum_{i,j}\lambda_{ij}\mu^i\otimes\nu^j=0
\end{equation}
右边的 $0$ 是矢量空间 $\mathcal L(V_1,V_2;\mathbb F)$
中的零元(注意,由张量积定义 $\mu^i\otimes\nu^j\in \mathcal L(V_1,V_2;\mathbb F)$ ).于是
\begin{equation}
\lambda_{kl}=\qty(\sum_{i,j}\lambda_{ij}\mu^i\otimes\nu^j)(\mu_k,\nu_l)=0(\mu_k,\nu_l)=0
\end{equation}
其中,最后的 $0$ 是域 $\mathbb F$ 的零元,$\{\mu_k\},\{\nu_l\}$ 分别是 $V_1,V_2$ 的与 $\{\mu^i\}$,$\{\nu^i\}$ 对偶的基.所以得到\autoref{TPofSp_eq1} 的线性无关性.

又 $\mathcal L(V_1,V_2;\mathbb F)$ 的维数为 $nm$,由基的定义(\autoref{VecSpn_def2}~\upref{VecSpn}),所以\autoref{TPofSp_eq1} 就是它的一组基.

\textbf{证毕!}

由\autoref{TPofSp_lem1} ,容易证得下面的推论
\begin{corollary}{}\label{TPofSp_cor1}
映射
\begin{equation}
\sigma(f,g):=f\otimes g,\quad f\in V_1^*,g\in V_2^*
\end{equation}
是空间 $\mathcal L(V_1,V_2;\mathbb F)$ 中的一个矢量,即 $V_1^*,V_2^*$ 上的一双线性型.
\end{corollary}
注意:上述的引理和推论,将矢量空间和其对偶空间互调仍然成立,因为彼此是对方的对偶空间(对偶空间的对称性).

\begin{theorem}{}\label{TPofSp_the1}
设 $V_1,V_2$ 是域 $\mathbb F$ 上有限维矢量空间,则矢量空间 $V_1,V_2$ 的张量积存在,且在同构的意义下是唯一的.
\end{theorem}
\textbf{证明:}先证明定理前一部分:

由\autoref{TPofSp_cor1} ,映射 $\sigma:V_1\times V_2\rightarrow\mathcal L(V_1^*,V_2^*;\mathbb F) $:
\begin{equation}\label{TPofSp_eq7}
\sigma(v_1,v_2)=v_1\otimes v_2,\quad v_1\in V_1,v_2\in V_2 
\end{equation}
是  $\mathcal L(V_1^*,V_2^*;\mathbb F)$ 中的矢量.

而由 \autoref{TPofSp_lem1} ,若 $\{\mu_i\},\{\nu_i\}$ 分别是 $V_1,V_2$ 的基,$n=\dim V_1,m=\dim V_2$ .那么
\begin{equation}\label{TPofSp_eq2}
\sigma(\mu_i,\nu_j)=\mu_i\otimes \nu_j,\quad i=1,\cdots,n,\quad j=1,\cdots,m
\end{equation}
是 $\mathcal L(V_1^*,V_2^*;\mathbb F)$ 的一组基.

现在证明,对 $(\sigma,\mathcal L(V_1^*,V_2^*;\mathbb F))$ 就是 $V_1,V_2$ 的张量积.设 $U$ 是域 $\mathbb F$ 上任一线性空间,
\begin{equation}\label{TPofSp_eq3}
\begin{aligned}
&\varphi:V_1\times V_2\rightarrow U;\\
&\varphi(\mu_i,\nu_j)=\varphi_{ij},\quad i=1,\cdots,n,\quad j=1,\cdots m
\end{aligned}
\end{equation}
是任一双线性映射.由\autoref{TPofSp_lem2} ,存在\textbf{唯一}的线性映射 $\psi:\mathcal L(V_1^*,V_2^*;\mathbb F)\rightarrow U$,使
\begin{equation}\label{TPofSp_eq4}
\psi(\mu_i\otimes \nu_j)=\varphi_{ij},\quad i=1,\cdots,n,\quad j=1,\cdots m
\end{equation}

\autoref{TPofSp_eq2} 带入\autoref{TPofSp_eq4}  ,得
\begin{equation}
\psi\sigma(\mu_i,\nu_j)=\varphi_{ij}=\varphi(\mu_i,\nu_j)
\end{equation}
于是对任意 $v_1\in V_1,v_2\in V_2$,由于双线性型,显然
\begin{equation}
\psi\sigma(v_1,v_2)=\varphi(v_1,v_2)
\end{equation}
即 $\psi\sigma=\varphi$.

由空间张量基的定义(\autoref{TPofSp_def1} ),\textbf{证得}对 $(\sigma,\mathcal L(V_1^*,V_2^*,\mathbb F))$ 就是 $V_1,V_2$ 的张量积.

定义第二部分的证明:设对 $(\sigma,T),(\sigma', T')$ 是 $V_1,V_2$ 的两个张量积,于是存在唯一的线性映射 $\psi:T\rightarrow T',\psi':T'\rightarrow T$ ,使得
\begin{equation}
\psi\sigma=\sigma',\quad\psi'\sigma'=\sigma
\end{equation}
于是
\begin{equation}
\psi'\psi\sigma=\sigma=e_T\sigma
\end{equation}
其中,$e_T$ 是 $T$ 到 $T$ 的恒等映射.

由于 $\psi,\psi'$ 的唯一性
\begin{equation}\label{TPofSp_eq5}
\psi'\psi=e_T
\end{equation}
同理
\begin{equation}\label{TPofSp_eq6}
\psi\psi'=e_{T'}
\end{equation}
首先,两个线性映射 $\psi,\psi'$ 的复合 $\psi'\psi$ 仍是线性映射,其次,\autoref{TPofSp_eq5} ,\autoref{TPofSp_eq6} 表明 $\psi,\psi'$ 都是双射(因为上两式表明二者都有左逆和右逆,所以必为双射),即 $T,T'$ 同构(矢量空间中的同构映射是线性的双射).

\textbf{证毕!}

现在容易证明,空间张量积中的映射 $\sigma$ 是个满映射.因为由上面证明知道,$\mathcal L(V_1^*,V_2^*;\mathbb F)$ 是空间 $V_1,V_2$ 的张量积中的 $T$.而张量积在同构的意义下是唯一的,所以只需证明 $\sigma:V_1\times V_2\rightarrow \mathcal L(V_1^*,V_2^*;\mathbb F)$ 是个满映射.由\autoref{TPofSp_lem1} ,$\{e_i\otimes e'_j\}$ 是 $\mathcal L(V_1^*,V_2^*;\mathbb F)$ 的一组基,其中 $\{e_i\},\{e'_j\}$ 分别是 $V_1,V_2$ 的一组基.所以 $\forall t\in \mathcal L(V_1^*,V_2^*;\mathbb F)$,都可写成
\begin{equation}
t=t^{ij}e_i\otimes e'_j
\end{equation}
那么任意的 $v=v^i e_i\in V_1,v'=v'^j e'_j\in V_2$ 只要满足 $v^iv'^j=t^{ij}$ 就有
\begin{equation}
v\otimes v'=t
\end{equation}
由\autoref{TPofSp_eq7} $\sigma(v,v')=t$.故 $\sigma$ 的满射性得证!

由\autoref{TPofSp_the1} 证明过程可看出,下面定理成立
\begin{theorem}{}
矢量空间 $V_1,V_2$ 的张量积存在,并且在同构意义下就是 $(\sigma,\mathcal L(V_1^*,V_2^*))$,其中
\begin{equation}
\sigma(v_1,v_2)=v_1\otimes v_2
\end{equation}
\end{theorem}
 
 因此,将 $V_1,V_2$ 的张量积中的 $T$ 记作 $T=V_1\otimes V_2$ 是恰当的.

\begin{definition}{}
若 $(\sigma ,T)$ 是 $V_1,V_2$ 的张量积,则记 $T=V_1\otimes V_2$.
\end{definition}
\subsection{空间张量积的同构}
由于维数相同的矢量空间必同构,由\autoref{TPofSp_lem1} ,$V_1\otimes V_2$ 和 $V_2\otimes V_1$ 的维数相同,同样,$(V_1\otimes V_2)\otimes V_3$ 和 $V_1\otimes( V_2\otimes V_3)$ 维数也相同.那么它们各自必同构,下面定理给出了具体的同构.
\begin{theorem}{}\label{TPofSp_the2}
设 $U,V,W$ 是域 $\mathbb F$ 上有限维矢量空间.则有同构映射 $\psi_1:(U\otimes V)\otimes W\rightarrow U\otimes(V\otimes W)$ 和 $\psi_2:U\otimes V\rightarrow V\otimes U$,使得
\begin{equation}
\begin{aligned}
\psi_1((u\otimes v)\otimes w)&=u\otimes(v\otimes w)\\
\psi_2(u\otimes v)&=v\otimes u
\end{aligned}
\end{equation}
\end{theorem}
\textbf{证明}: 设 $\{\epsilon_i\},\{\eta_j\},\{\xi_k\}$ 分别是 $U,V,W$ 的基底,那么 $\psi_1,\psi_2$ 分别将 $(U\otimes V)\otimes W$ 和 $U\otimes V$ 的基矢
\begin{equation}
(\epsilon_i\otimes\eta_j)\otimes\xi_k,\quad \epsilon_i\otimes\eta_j
\end{equation}

映到 $U\otimes (V\otimes W)$ 和 $V\otimes U$ 的基矢 
\begin{equation}
\epsilon_i\otimes(\eta_j\otimes\xi_k),\quad \eta_j\otimes\epsilon_i
\end{equation}

那么,如果证明了 $\psi_1,\psi_2$ 的线性性质,就能得到 $\psi_1,\psi_2$ 的同构性.因为从上面容易看出 $\psi_1,\psi_2$ 分别将两空间基底一一对应,如果它们还是线性映射,那么各空间中的矢量在 $\psi_1,\psi_2$ 的作用下坐标和原来相同,这时就容易知道 $\psi_1,\psi_2$ 的双射性.

对
\begin{equation}
\begin{aligned}
&\forall u=u^i\epsilon_i\in U,\quad v=v^j\eta_j\in V,\quad w=w^k\xi_k\in W\\
&\forall u'=u'^i\epsilon_i\in U,\quad v'=v'^j\eta_j\in V,\quad w'=w'^k\xi_k\in W\\
&\forall \alpha,\beta \in\mathbb F
\end{aligned}
\end{equation}
由纯量与张量积的乘法定律(\autoref{TsrPrd_the1}~\upref{TsrPrd}),有
\begin{equation}
\begin{aligned}
&\psi_1(\alpha(u\otimes v)\otimes w+\beta(u'\otimes v')\otimes w')\\
&=\psi_1\qty(\qty(\alpha u^iv^jw^k+\beta u'^iv'^jw'^k)(\epsilon_i\otimes\eta_j)\otimes\xi_k)\\
&=\qty(\alpha u^iv^jw^k+\beta u'^iv'^jw'^k)\epsilon_i\otimes(\eta_j\otimes\xi_k)\\
&=\alpha u\otimes (v\otimes w)+\beta u'\otimes (v'\otimes w')\\
&=\alpha\psi_1((u\otimes v)\otimes w)+\beta\psi_1((u'\otimes v')\otimes w')\\
\\
&\psi_2\qty(\alpha u\otimes v+\beta u'\otimes v')=\psi_2\qty(\qty(\alpha u^iv^j+\beta u'^iv'^j)\epsilon_i\otimes\eta_j)\\
&=\qty(\alpha u^iv^j+\beta u'^iv'^j)\eta_j\otimes\epsilon_i\\
&=\alpha v\otimes u+\beta v'\otimes u'\\
&=\alpha\psi_2(u\otimes v)+\beta\psi_2(u'\otimes v')
\end{aligned}
\end{equation}
于是证得  $\psi_1,\psi_2$ 的线性性质.

\textbf{证毕!}

\autoref{TPofSp_the2} 给出了空间张量积的结合律,因此对多个空间的张量积,可以不适用括弧.

另外,下面定理也成立:
\begin{theorem}{}
设 $U,V,W$ 是域 $\mathbb F$ 上有限维矢量空间.则有同构映射
\begin{equation}
\begin{aligned}
&\psi_1:(U\oplus V)\otimes W\rightarrow (U\otimes W)\oplus (V\otimes W) \\
&\psi_2:U\otimes(V\oplus W)\rightarrow (U\otimes V)\oplus (U\otimes W)
\end{aligned}
\end{equation}
使得
\begin{equation}\label{TPofSp_eq8}
\begin{aligned}
&\psi_1((u+v)\otimes w)=(u\otimes w)+ (v\otimes w)\\
&\psi_2(u\otimes (v+ w))=(u\otimes v)+ (u\otimes w) 
\end{aligned}
\end{equation}
\end{theorem}
\textbf{证明:}由张量积对加法的分配律(\autoref{TsrPrd_the1}~\upref{TsrPrd}),\autoref{TPofSp_eq8} 第一式相当于
\begin{equation}
\begin{aligned}
\psi_1((u+v)\otimes w)&=(u+v)\otimes w\\
\psi_2(u\otimes(v+w))&=u\otimes(v+w)
\end{aligned}
\end{equation}
%\begin{equation}
%\begin{aligned}
%&\psi_1(\alpha(u+v)\otimes w+\beta(u'+v')\otimes w')\\
%&=\psi_1\qty(\qty((\alpha u^iw^k+\beta u'^iw'^k)\epsilon_i+(\alpha v^jw^k+\beta v'^jw'^k)\eta_j)\otimes\xi_k)\\
%&=(\alpha u^iw^k+\beta u'^iw'^k)\epsilon_i\otimes\xi_k+(\alpha v^jw^k+\beta v'^jw'^k)\eta_j\otimes\xi_k\\
%&=(\alpha u\otimes w+\beta u'\otimes w')+(\alpha v\otimes w+\beta v'\otimes w')\\
%&=\alpha(u+v)\otimes w+\beta(u'+v')\otimes w'\\
%&=\alpha\psi_1((u+v)\otimes w)+\beta\psi((u'+v')\otimes w')
%\end{aligned}
%\end{equation}
即 $\psi_1,\psi_2$ 是恒等映射,而空间 $(U\oplus V)\otimes W$ 和空间 $(U\otimes W)\oplus (V\otimes W) $ 基底(在映射 $\sigma(v,w)=v\otimes w$ 下)显然相同,所以这两空间可看成同一空间,那么恒等映射当然是同构映射.

\textbf{证毕!}
