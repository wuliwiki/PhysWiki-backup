% Sylow定理
% 西罗定理|西罗子群|Sylow子群|群论|有限群

\pentry{群作用\upref{Group3}}

\addTODO{加入目录}

拉格朗日定理(\autoref{coset_the2}~\upref{coset})揭示了子群阶数的特点.可惜的是,其逆命题“如果$n\mid \abs{G}$则$G$总有阶数为$n$的子群”是不成立的.比如说,交错群$A_4$就没有$6$阶的子群——你可以动手验证这一点.

但是,挪威数学家Peter Ludwig Sylow于1872年发表的文章\footnote{L. Sylow, \href{https://eudml.org/doc/156588}{Théorèmes sur les groupes de substitutions}, Mathematische Annalen 5 (1872), 584–594. 对此文的英文翻译见 \href{http://www.maths.qmul.ac.uk/~raw/pubs_files/Sylow.pdf}{Theorems on groups of substitutions}.}指出,在$n$是素数或者素数的幂时,拉格朗日定理逆命题是成立的.同时他还发现了所谓的Sylow子群全都是彼此共轭的.

举个例子:考虑阶数为$300$的群$G$,对$300$进行素因子分解得$300=2^2\cdot 3^2\cdot 5^2$,那么阶数为$2^2$的子群总存在,且它们彼此共轭.不过,$2$阶子群虽存在,却不能总保证所有$2$阶子群彼此共轭.


\begin{definition}{Sylow子群}\label{Sylow_def1}
给定群$G$和它的子群$H$.如果$\abs{H}=p^k$,其中$p$是素数,且$p^{k+1}\not\mid\abs{G}$,那么称$H$是$G$的一个$p$\textbf{-Sylow 子群},或\textbf{Sylow-}$p$\textbf{子群},或直接统称为\textbf{Sylow子群}.
\end{definition}



\begin{exercise}{}
考虑循环群$C_{12}$,求它的所有Sylow子群.
\end{exercise}

\begin{example}{}
考虑置换群$S_4$.则
\begin{equation}
C=\{1, \pmat{1&2}\pmat{3&4}, \pmat{1&3}\pmat{2&4}, \pmat{1&4}\pmat{2&3}, \pmat{1&2}, \pmat{3&4}, \pmat{1&3&2&4}, \pmat{1&4&2&3}\}
\end{equation}
是它的一个Sylow-$2$子群.

$C$是$\pmat{1&2}\pmat{3&4}$的中心,因此我们可以类似构造出$S_4$的剩下两个Sylow-$2$子群,分别是$\pmat{1&3}\pmat{2&4}$和$\pmat{1&4}\pmat{2&3}$的中心.这三个Sylow-$2$子群的交集是$V_4=\{1, \pmat{1&2}\pmat{3&4}, \pmat{1&3}\pmat{2&4}, \pmat{1&4}\pmat{2&3}\}$.

如果把一个正方形的四个顶点顺时针依次编号为$1, 3, 2, 4$,则不难看出$C$实际上就是正方形的对称群$D_4$.类似地,$S_4$的每个Sylow-$2$子群都同构于$D_4$.

\end{example}

Sylow定理通常被拆分成三个部分来表述.

\begin{theorem}{Sylow第一定理}
取\textbf{有限群}$G$.如果素数$p\mid\abs{G}$,那么$G$一定有一个Sylow-$p$子群$H_p$.
\end{theorem}

\textbf{证明}:

令$\abs{G}=p^km$,其中$p\not\mid m$.

先证明当$G$是阿贝尔群时,$p$阶子群存在.

任取$g\in G$,如果$g$的阶(\autoref{coset_def1}~\upref{coset})$\opn{ord}g$是$p$的倍数$np$,那么$g^n$生成的循环群就是$p$阶的.否则,设$\opn{ord}g=r$,取$g$生成的子群$H_g$,它是$G$的正规子群(因为$G$阿贝尔),于是$\abs{G/H_g}=p^km/r$.在$G/H_g$中再任挑一个元素,如果该元素的阶\textbf{不是}$p$,同样求出其循环群后用$\abs{G/H_g}$去除掉这个循环群,得到商群.易证有限步后,总能得到一个阶数为$p$\textbf{的倍数}的元素\footnote{注意,这里说的\textbf{阶数}逻辑上有跳步.比如说,在$G/H_g$中找到了一个$h$,其阶数为$sp$,那这只能说明$h^{sp}\in H_g$.不过由于$h^{sp}$本身也是有限阶的,因此$h$在$G$中的阶数也确实是$p$的倍数.}.于是这个元素的一个幂生成的循环群是$p$阶的.

下设$G$是任意的有限群,$Z(G)$是它的中心.设定理对于阶数小于$p^km$的群都成立,然后分类进行归纳讨论.

如果$p\mid \abs{Z(G)}$,那么由于$Z(G)$是阿贝尔群,故存在$G$的$p$阶子群$A$.由于$A\subseteq Z(G)$,故$A\vartriangleleft G$.于是得到商群$G/A$,其阶数为$p^{k-1}m$.由归纳假设,$G/A$有一个$p^{k-1}$阶Sylow子群,因此这个子群是$G$的$p^k$阶子群,从而是$G$的Sylow-$p$子群.

如果$p\not\mid \abs{Z}(G)$,那么根据共轭类等式(The Class Equation\autoref{Group3_the4}~\upref{Group3}),
\begin{equation}
\abs{G} = \abs{Z(G)} + \sum_{C\in O, \abs{C}>1} \abs{C}
\end{equation}
其中$O$是$G$在伴随作用下全体轨道构成的集合.

由于$p\mid \abs{G}$,$p\not\mid \abs{Z(G)}$,故存在$\abs{C}$使得$p\not\mid \abs{C}$,即存在$x\in G$使得$p\not\mid \abs{C_x}$,故$\abs{C_x}\mid m$.由\autoref{Group3_cor3}~\upref{Group3},得$\abs{C_G(x)}=p^km/\abs{C_x}=p^ks<\abs{G}$\footnote{$C_G(x)$即$x$的中心化子,或者说$x$在伴随作用下的迷向子群.},其中$s>1$.因此由归纳假设,$C_G(x)$有一个$p^k$阶子群$A$,从而$G$有个Sylow-$p$子群$A$.

\textbf{证毕}.





\begin{theorem}{Sylow第二定理}
取\textbf{有限群}$G$,它的所有Sylow-$p$子群彼此共轭.
\end{theorem}

\textbf{证明}:

令$\abs{G}=p^km$,其中$p\not\mid m$.

任取$G$的两个Sylow-$p$子群$A$和$B$,记$G/B$是$B$在$G$中的左陪集构成的集合\footnote{不是商群,因为不能保证$B\vartriangleleft G$.}.考虑群$A$作用在集合$G/B$上,方式是\textbf{左伴随作用}.

据\autoref{Group3_the5}~\upref{Group3},
\begin{equation}
\abs{G/B} = \abs{\opn{Fix}_A(G/B)} \opn{mod} p
\end{equation}

由Sylow子群的定义,知$\abs{G/B}=m$.故$m\abs{\opn{Fix}_A(G/B)} \opn{mod} p$,意味着$\abs{\opn{Fix}_A(G/B)}\neq 0$,即存在不动点$gB\in G/B$.

因此$AgB=bB$


\textbf{证毕}.




\begin{theorem}{Sylow第三定理}
取\textbf{有限群}$G$,设$n_p$是它的所有Sylow-$p$子群的数目.令$\abs{G}=p^km$,其中$p\not\mid m$.

那么$n_p\equiv 1\opn{mod} p$,或者说$p\mid n_p-1$;且$n_p\mid m$.
\end{theorem}

\textbf{证明}:

令$\abs{G}=p^km$,其中$p\not\mid m$.



\textbf{证毕}.




















