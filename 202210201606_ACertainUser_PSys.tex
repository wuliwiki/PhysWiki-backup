% 质点系
% 质点|牛顿第三定律|合外力|内力

\pentry{牛顿第三定律\upref{New3}}
\begin{figure}[ht]
\centering
\includegraphics[width=5cm]{./figures/PSys_1.pdf}
\caption{质点系} \label{PSys_fig1}
\end{figure}

在考虑多个物体构成的系统时, 我们有时候可以把每个物体都近似为一个质点, 这样我们就得到了由有限个质点构成的系统, 简称为\textbf{质点系}或者\textbf{质点组}.

令质点系中有 $N$ 个质点, 每个质点的受力都可以分为两类, 一是系统外界物体给该质点的力, 称为\textbf{外力}, 二是来自系统内其他质点的力, 称为\textbf{内力}. 对第 $i$ 个质点, 以下将它受到的所有外力和内力之和分别记为 $\bvec F_i^{out}$ 和 $\bvec F_i^{in}$, 即单个质点的\textbf{合内力}以及\textbf{合外力}, 所以单个质点所受的合力为
\begin{equation}
\bvec F_i = \bvec F_i^{in} + \bvec F_i^{out}
\end{equation}

系统中所有质点所受的\textbf{合力}等于\textbf{合内力}加\textbf{合外力}\footnote{角标 “tot” 表示 total.}
\begin{equation}
\bvec F_{tot} = \sum_i \bvec F_i = \sum_i^N \bvec F_i^{in} + \sum_i^N \bvec F_i^{out} = \bvec F_{tot}^{in} + \bvec F_{tot}^{out}
\end{equation}
若将第 $j$ 个质点对第 $i$ 个质点的内力记为 $\bvec F_{j\to i}$ 则上式中
\begin{equation}
\sum_i \bvec F_i^{in} = \sum_{i,j}^{i \ne j} \bvec F_{j \to i}
\end{equation}
任意两个质点 $k$ 和 $l$ 对该求和的贡献是一对相互作用力 $\bvec F_{k \to l} + \bvec F_{l \to k}$,而根据牛顿第三定律\upref{New3},相互作用力之和为零.所以上式求和为零. 所以, 质点系中\textbf{合内力为零}, 系统所受合力等于合外力
\begin{equation}
\bvec F_{tot} = \bvec F_{tot}^{out} = \sum_i^N \bvec F_i^{out}
\end{equation}
