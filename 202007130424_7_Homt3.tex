% 基本群

本节我们将介绍拓扑空间上的一个重要的代数结构,这将会是我们研究同伦(或者说拓扑空间的连续变化)的有力工具.这是一个群,被称为\textbf{基本群}.

我们首先将简单讨论基本群的想法是怎么来的,从直观上定义一个道路运算开始,z

\subsection{基本群的构造过程}
\subsubsection{道路类的积}

一个自然的想法是,研究拓扑空间中的道路之间的运算,称这个运算为道路之间的积.这种运算首先要满足封闭性,即两个道路的积还是一条道路.最简单的情况,就是两条道路首尾相连而成一条新的道路.因此,我们首先定义道路之间的积为首尾相连:

\begin{definition}{道路的积}
给定拓扑空间$X$及其上的两条道路:$f, g:I\rightarrow X$,并设$f(1)=g(0)$,即$f$的终点是$g$的起点.定义$f$和$g$的积为$f*g=h$,其中\begin{equation}h=\leftgroup{f(2t),\quad &t\in[0, 1/2]\\g(2t-1),\quad &t\in[1/2, 1]}\end{equation}
显然,$h$也是一条道路,并且其起点是$f$的起点、终点是$g$的终点.
\end{definition}

如上定义的道路$f$和$g$的积,虽然满足封闭性,但是并不满足结合性,这用一个简单的例子就可以说明:

\begin{exercise}{道路的积不满足结合性}
取通常的一维欧几里得空间$\mathbb{R}$.设$\mathbb{R}$上有三条道路,$f=t$,$g=t+1$和$h=t+2$.请验证,$(f*g)*h\not=f*(g*h)$.
\end{exercise}

简单来说,道路的积不满足结合律,是因为虽然$(f*g)*h$和$f*(g*h)$的轨迹是重合的,但是道路上的点走过这条轨迹的“速度”不一样,因此不被视作一条道路.如果我们能把$(f*g)*h$和$f*(g*h)$归入同一个等价类里,转而研究这种道路等价类上的运算,那么结合律就可以满足了.

\begin{definition}{道路类}
给定拓扑空间$X$.如果两条道路:$f, g:I\rightarrow X$满足$f(0)=g(0), f(1)=g(1)$(即有相同的起点和终点),并且$f\cong g$,那么将它们归入同一个等价类.这样划分出的等价类,称为$X$上的\textbf{道路类(path class)},记为$[f]$,其中$f$是$[f]$中任意一条道路.
\end{definition}

显然,如果两条道路轨迹相同,那么它们一定是在同一个道路类里的,这就解决了结合性的问题.

\begin{definition}{道路类的积}
给定拓扑空间$X$和其上的两个道路类$[f]$和$[g]$.记$[f]$和$[g]$之间的积为
\end{definition}







