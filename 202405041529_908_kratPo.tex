% 双原子分子克拉策势(量子力学)
% keys Kratzer|克拉策势|克拉测势
% license Xiao
% type Tutor

\pentry{薛定谔方程(单粒子多维)\nref{nod_QMndim},球谐函数 \nref{nod_SphHar}}{nod_9c14}

克拉策势(Kratzer's molecular potential,又写作 Kratzer's Potential 或克拉测势等),是另一个描述双原子分子的势能模型,表示为:
\begin{equation}
V(r) = -2D [a/r - a^2/(2r^2)] ~.
\end{equation}
是对于一般角量子数 $l$ 的,而\enref{莫尔斯势\upref{MoPoQM}的求解过程不难发现其实是对应角量子数 $l=0$ 的情况。其中 $r$ 是两原子间距而 $a$ 是平衡核间距,有势能最小值 $V(a) = -D$。下面给出一些经典取值\footnote{表格数据来源:《量子力学 I》,顾樵。}
\begin{table}[ht]
\centering
\caption{经典分子参数}\label{tab_kratPo1}
\begin{tabular}{|c|c|c|c|c|}
\hline
分子 & $D/\Si{eV}$ & $a/\mathring{\text{A}}$,埃米 & $\mu/\Si{amu}$,原子质量单位\\
\hline
CO & 10.8451 & 1.128 & 6.8606 \\
\hline
NO & 8.0438 & 1.151 & 7.4684 \\
\hline
O$_2$ & 5.1567 & 1.208 & 7.9974 \\
\hline
I$_2$ & 1.5818 & 2.662 & 63.4522 \\
\hline
\end{tabular}
\end{table}

\subsection{简谐近似}
类似于林纳德-琼斯势\upref{LenJoP}和摩尔斯势\upref{MoPoQM}的方法,仍先考虑简谐近似。有
\begin{equation}
\begin{aligned}
\epsilon = \eval{V}_{r=a} &= -D ~, \\
k = \eval{\dv{^2 V}{r^2}}_{r=a} &= \frac{2D}{a^2} ~.
\end{aligned}
\end{equation}
对于体系的约化质量 $\widetilde M = m_1m_2/(m_1+m_2)$,有
$$\omega^2 = \frac{k}{\widetilde M} = \frac{2D}{\widetilde M a^2} ~.$$

\subsection{简谐近似势的量子化}
体系的本征能量为
\begin{equation}
E_\nu = \hbar \omega\left(\nu + \frac12\right) - D, \ (\nu = 0, 1, 2, \cdots) ~.
\end{equation}

\subsection{精确解}
体系的薛定谔方程为
\begin{equation}
\laplacian \psi + \frac{2 \widetilde M}{\hbar^2} [E - V(r)] \psi = 0~.
\end{equation}
考虑一个典型的解法——球谐函数展开:考虑波函数 $\psi = R(r) Y_{l, m} (\theta, \phi)$。则对于径向波函数 $R(r)$ 应满足微分方程
\begin{equation}
\frac1R \dv{}{r}\left(r^2 \dv{R}{r}\right) + \left( \frac{2\widetilde M r^2}{\hbar^2} \left[E-V(r)\right] \right) = l(l+1), \ (l = 0, 1, 2, \cdots) ~.
\end{equation}
若令 $R(r) = u(r)/r$,取 $\beta^2=2\widetilde M a^2 D/(\hbar^2)$,则等价于微分方程
\begin{equation}
\dv{^2 u}{r^2} + \left[\frac{2\widetilde M}{\hbar^2} \left(E + \frac{2aD}{r}\right) - \frac{\beta^2 + l(l+1)}{r^2}\right] u = 0 ~.
\end{equation}

\subsubsection{解这微分方程}
\pentry{碱金属原子\nref{nod_AlkmQM}}{nod_68df}
我们发现这微分方程有类似于碱金属原子讨论过的微分方程(\autoref{eq_AlkmQM_1}~\upref{AlkmQM})的形式。取两系数
\begin{equation}
e' = \sqrt{2aD},\ l'(l'+1) = \beta^2 + l(l+1) ~,
\end{equation}
即
$$l' = -\frac12 + \sqrt{\beta^2 + \left(l + \frac12\right)^2} ~,$$
就有
\begin{equation}
\dv{^2 u}{r^2} + \left[ \frac{2 \widetilde M}{\hbar^2} \left(E + \frac{e'^2}{r}\right)  - \frac{l'(l'+1)}{r^2}\right] u = 0 ~.
\end{equation}
与碱金属原子讨论过的微分方程形式相同。

对应的,量子化条件 $n' = \nu + l' + 1, \ (\nu = 0, 1, 2, \cdots)$。其中,$n' > 1$ 一般不是整数。分子体系的能级是
\begin{equation}
E_{\nu, l} = -\frac{\widetilde M e'^4}{2 \hbar^2} \left(\frac1{n'}\right)^2 = -\beta^2 D \left(\frac{1}{n'}\right)^2 ~.
\end{equation}

\subsection{基态性质}
对于基态而言 $\nu=0, l=0$。故基态能量
\begin{equation}
E_{0, 0} = -\beta^2 D \left(\frac12 + \sqrt{\beta^2 + \left(\frac12\right)^2}\right)^{-2} ~.
\end{equation}


而径向概率密度
\begin{equation}
W(\rho) = \frac{2^{2n'}}{n' \Gamma{(2n')}} \rho^{2n'} \exp(-2\rho) ~, \ \left(\rho = \frac{r}{n'a'}\right) ~.
\end{equation}
最可几值于 $\dv{W(\rho)}{\rho} = 0$ 时,有
\begin{equation}
\rho = n' ~.
\end{equation}
则两原子最可几间隔 $r = a (n'/\beta)^2$。

下面给出一些经典的分子的计算出的参数\footnote{表格数据来源:同上。《量子力学 I》,顾樵。}
\begin{table}[ht]
\centering
\caption{经典分子参数 2}\label{tab_kratPo2}
\begin{tabular}{|c|c|c|c|}
\hline
分子 & $\beta$ & $E_{0,0} / \Si{eV}$ & $r/\Si{\mathring{A}}$ \\
\hline
CO & 212.826 & -10.7943 & 1.133 \\
\hline
NO & 195.103 & -8.0027 & 1.157 \\
\hline
O$_2$ & 169.686 & -5.1264 & 1.215 \\
\hline
I$_2$ & 583.344 & -1.5791 & 2.667 \\
\hline
\end{tabular}
\end{table}
