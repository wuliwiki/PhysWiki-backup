% 薛定谔方程(综述)
% license CCBYSA3
% type Wiki

本文根据 CC-BY-SA 协议转载翻译自维基百科\href{https://en.wikipedia.org/wiki/Schr\%C3\%B6dinger_equation}{相关文章}。

薛定谔方程是一个偏微分方程,用于描述非相对论量子力学体系的波函数的演化过程。\(^\text{[1]: 1–2 }\) 它的发现是量子力学发展史上的一个重要里程碑。该方程以奥地利物理学家埃尔温·薛定谔的名字命名。他于1925年提出该方程,并于1926年发表,从而奠定了其后获得1933年诺贝尔物理学奖的工作基础。\(^\text{[2][3]}\)

在概念上,薛定谔方程是量子力学中对应于经典力学中牛顿第二定律的表达。给定一组已知的初始条件,牛顿第二定律可以用数学方式预测一个物理系统随时间演化的轨迹。薛定谔方程则给出了波函数随时间的演化规律,而波函数是对一个孤立物理系统的量子力学描述。该方程是薛定谔在路易·德布罗意提出“所有物质都具有伴随的物质波”这一假设的基础上提出的。薛定谔方程成功预测了与实验观测一致的原子束缚态。\(^\text{[4]: II:268 }\)

薛定谔方程并不是研究量子力学系统和进行预测的唯一方法。量子力学的其他表述方式还包括维尔纳·海森堡提出的矩阵力学,以及主要由理查德·费曼发展的路径积分表述。在比较这些方法时,使用薛定谔方程的方式有时被称为“波动力学”。

薛定谔提出的方程是非相对论性的,因为它在时间上是一阶导数,而在空间上是二阶导数,因此空间与时间在方程中并不对等。保罗·狄拉克将狭义相对论与量子力学结合成一个统一的表述形式,在非相对论极限下会简化为薛定谔方程。这就是狄拉克方程,它在空间和时间上都只包含一阶导数。

另一个偏微分方程,即克莱因–戈尔登方程,虽然是一个相对论性的波动方程,但在描述概率密度时出现了问题:概率密度可能为负值,这在物理上是不可接受的。狄拉克通过对克莱因–戈尔登算符进行所谓的“开平方”处理,引入了狄拉克矩阵,从而解决了这一问题。

在现代物理的语境中,克莱因–戈尔登方程用于描述无自旋粒子,而狄拉克方程则用于描述自旋为1/2的粒子。
\subsection{定义}
\subsubsection{预备知识}
在物理或化学的入门课程中,通常会以一种仅需掌握基础微积分(特别是关于空间与时间的导数)的概念和符号就能理解的方式来介绍薛定谔方程。薛定谔方程的一个特例,是针对一维空间中单个非相对论粒子的位置空间形式,其表达如下:
$$
i\hbar \frac{\partial}{\partial t}\Psi(x,t) = \left[ -\frac{\hbar^2}{2m} \frac{\partial^2}{\partial x^2} + V(x,t) \right] \Psi(x,t)~
$$
在这个方程中,$\Psi(x, t)$ 是**波函数**,即为每个时刻 $t$ 下的每个位置 $x$ 分配一个复数值的函数;$m$ 是粒子的质量;$V(x, t)$ 是势能函数,用来表示粒子所处环境中的势场【5: 第74页】;$i$ 是虚数单位;$\hbar$ 是约化普朗克常数,其单位为作用量(能量乘以时间)\(^\text{[5]: 10 }\)。
