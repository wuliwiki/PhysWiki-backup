% 卡尔·弗里德里希·高斯(综述)
% license CCBYSA3
% type Wiki

本文根据 CC-BY-SA 协议转载翻译自维基百科\href{https://en.wikipedia.org/wiki/Carl_Friedrich_Gauss}{相关文章}。

\begin{figure}[ht]
\centering
\includegraphics[width=6cm]{./figures/6c9aaaacb17b4d2e.png}
\caption{由克里斯蒂安·阿尔布雷希特·延森绘制的肖像,1840年(由戈特利布·比尔曼复制,1887年)} \label{fig_KRGS_3}
\end{figure}
约翰·卡尔·弗里德里希·高斯(德语:Gauß [kaʁl ˈfʁiːdʁɪç ˈɡaʊs];拉丁语:Carolus Fridericus Gauss;1777年4月30日–1855年2月23日)是德国数学家、天文学家、测量学家和物理学家,对数学和科学的多个领域做出了重要贡献。他自1807年起成为哥廷根天文台台长和天文学教授,直至1855年去世。高斯被广泛认为是历史上最伟大的数学家之一。

在哥廷根大学学习期间,他提出了多个数学定理。高斯以私人学者的身份完成了他的代表作《算术研究》和《天体运动理论》。他给出了代数基本定理的第二个和第三个完整证明,对数论作出了贡献,并发展了二次和三次二次型的理论。

高斯在发现冥王星作为矮行星的工作中发挥了重要作用。他关于受到大行星影响的行星状物体的运动的研究,导致了高斯引力常数和最小二乘法的引入,而高斯在Adrien-Marie Legendre发表之前就已发现了这一方法。高斯与其他学者一起负责了1820年至1844年期间对汉诺威王国的大规模地理测量和弧长测量项目;他是地球物理学的创始人之一,并提出了磁学的基本原理。他实际工作的成果包括1821年发明了太阳能标,1833年发明了磁力计,以及与威廉·爱德华·韦伯一起于1833年发明了第一台电磁电报机。

高斯是第一个发现并研究非欧几里得几何的人,并且他自己创造了这个术语。他还在约160年前就发展了快速傅里叶变换,比约翰·图基和詹姆斯·库利提前了数十年。

高斯拒绝发表未完成的工作,留下了多部未完成的作品,交由后人编辑。他认为学习的过程,而非拥有知识本身,才是最令人享受的。高斯曾坦言自己不喜欢教学,但他的一些学生后来成为了有影响力的数学家,如理查德·德德金德和伯恩哈德·黎曼。
\subsection{传记}  
\subsubsection{青年时期与教育}
\begin{figure}[ht]
\centering
\includegraphics[width=6cm]{./figures/b21e0aea7920ceb9.png}
\caption{布伦瑞克出生之家(在第二次世界大战中被摧毁)} \label{fig_KRGS_1}
\end{figure}
高斯于1777年4月30日出生在布伦瑞克公国(今德国下萨克森州的布伦瑞克市)。他的家庭社会地位相对较低。父亲盖布哈德·迪特里希·高斯(1744-1808)从事过屠夫、砖瓦匠、园丁和丧葬基金的财务工作。高斯曾形容他的父亲是一个正直而受人尊敬的人,但在家中则是粗暴且专横的。他的父亲擅长写作和计算,而高斯的继母多萝西娅几乎是文盲。高斯有一个从父亲第一次婚姻中生的哥哥。

高斯在数学方面是一个天才儿童。当他的小学老师注意到他的智力时,便将他推荐给布伦瑞克公爵。公爵将他送到当地的卡罗林学院学习,并在那里从1792年到1795年学习,埃伯哈德·奥古斯特·威廉·冯·齐默曼是他的老师之一。之后,公爵为他提供了在哥廷根大学学习数学、科学和古典语言的资源,直到1798年为止。高斯的数学教授是亚伯拉罕·戈特尔夫·凯斯特纳,高斯称他为“诗人中的数学大师,数学家中的诗人”,因为凯斯特纳有许多讽刺性诗句。天文学是由卡尔·费利克斯·赛弗教授的,毕业后高斯与赛弗保持通信;奥尔伯斯和高斯在他们的信件中取笑了他。另一方面,高斯对他的物理学老师乔治·克里斯托夫·李希滕贝格和基督教·戈特洛布·海恩的古典学课程给予高度评价,他愉快地参加了海恩的讲座。这段时间的同学包括约翰·弗里德里希·本岑贝格、法尔卡斯·博尔亚伊和海因里希·威廉·布兰德斯。

他可能是一个自学成才的数学学生,因为他独立重新发现了几条定理。1796年,他解决了一个自古希腊以来困扰数学家的几何问题,确定了哪些规则多边形可以通过圆规和直尺作图。这一发现最终使高斯选择了数学而不是语言学作为职业。高斯的数学日记,记录了他从1796年到1814年间的许多数学成果的简短备注,显示出他许多思想的萌芽,这些思想最终成为他数学巨著《算术研究》(1801)的基础。
\begin{figure}[ht]
\centering
\includegraphics[width=6cm]{./figures/ca9e9a3b21b12fba.png}
\caption{高斯在哥廷根作为学生时的住所} \label{fig_KRGS_2}
\end{figure}
\subsubsection{私人学者}  
高斯于1799年获得哲学博士学位,这一事实有时被误称为在哥廷根大学毕业,实际上是应布伦瑞克公爵的特别请求,从赫尔姆施塔特大学(公国唯一的州立大学)毕业的。约翰·弗里德里希·法夫评审了他的博士论文,高斯在没有进一步口试的情况下以缺席方式获得了学位。随后,公爵为他提供了作为私人学者在布伦瑞克的生活费用。高斯因此拒绝了圣彼得堡的俄罗斯科学院和兰茨胡特大学的邀请。后来,公爵在1804年承诺为他在布伦瑞克建立一个天文台。建筑师彼得·约瑟夫·克拉赫设计了初步的方案,但由于拿破仑战争,这些计划被取消:公爵在1806年的耶拿战役中阵亡。次年,公国被废除,高斯的经济支持也随之停止。

当高斯在世纪初的几年里计算小行星轨道时,他与不来梅和利连塔尔的天文界建立了联系,特别是与威廉·奥尔伯斯、卡尔·路德维希·哈丁和弗里德里希·威廉·贝塞尔等人,他们是“天体警察”这一非正式天文学家小组的一部分。该小组的目标之一是发现更多的行星。他们收集了小行星和彗星的数据,为高斯后来在其天文学巨著《天体运动论》(1809年)中发表的轨道研究提供了基础。
\subsubsection{哥廷根大学教授}
\begin{figure}[ht]
\centering
\includegraphics[width=8cm]{./figures/27ae38e6ad98b0b0.png}
\caption{约1800年的旧哥廷根天文台} \label{fig_KRGS_4}
\end{figure}
1807年11月,高斯应召到哥廷根大学,时该校隶属于新成立的西法利亚王国,由 Jérôme Bonaparte 统治,担任天文学教授兼天文台主任,并一直担任此职务直到1855年去世。很快,他就面临了西法利亚政府要求支付两千法郎作为战争贡献的要求,这笔费用他无力支付。奥尔伯斯和拉普拉斯都希望帮助他支付这笔费用,但高斯拒绝了他们的援助。最终,一位来自法兰克福的匿名人士——后来被发现是大公子达尔贝格——支付了这笔款项。

高斯接管了这座已有60年历史的天文台,该天文台由选帝侯乔治二世于1748年建立,建在一座改建过的防御塔上,仪器可用,但部分已过时。选帝侯乔治三世自1802年起原则上批准了新天文台的建设,西法利亚政府也继续进行规划,但高斯直到1816年9月才得以搬到新的工作地点。他获得了新的现代化仪器,包括从Repsold和Reichenbach公司购买的两台经纬仪和从弗劳恩霍夫购买的日心仪。

高斯的科学活动,除了纯数学外,大致可以分为三个阶段:19世纪前二十年以天文学为主,第三十年为测地学,第四十年则主要以物理学,特别是磁学为主。

高斯毫不掩饰自己对讲授学术课程的反感。但自从在哥廷根大学开始学术生涯以来,他一直持续讲授直到1854年。他经常抱怨教学的负担,觉得这是浪费时间。另一方面,他偶尔也会称某些学生才华横溢。他的大多数讲座涉及天文学、测地学和应用数学,仅有三次讲授纯数学的课程。高斯的学生中,有些人成为著名的数学家、物理学家和天文学家,如莫里茨·坎托尔、德德金德、迪尔克森、恩克、古尔德、海涅、克林克费乌斯、库普费尔、利斯廷、莫比乌斯、尼古莱、黎曼、里特、谢林、谢尔克、舒马赫、冯·斯陶特、斯特恩、乌尔辛;在地球科学领域,如萨托留斯·冯·瓦尔特豪森和瓦普厄斯。

高斯没有写过任何教科书,也不喜欢科学内容的普及。他唯一的普及尝试是他关于复活节日期的著作(1800/1802年)和1836年的《地磁学与磁力计》论文。高斯的论文和著作完全用拉丁文或德语发表。他的拉丁文写作风格古典,但使用了一些当代数学家所设定的常见修饰。
\begin{figure}[ht]
\centering
\includegraphics[width=8cm]{./figures/573bd7843f14cb7f.png}
\caption{1816年新的哥廷根天文台;高斯的起居室位于西翼(右侧)} \label{fig_KRGS_6}
\end{figure}
在1808年哥廷根大学的开学讲座中,高斯宣称,天文学的唯一任务是通过强大的微积分工具获得可靠的观察和结果。 在大学期间,他有其他讲师随行,负责他学科的教学工作,完成教育计划,其中包括数学家蒂博(Thibaut)及其讲座、物理学家迈耶(Mayer),以其教科书而闻名,以及自1831年起接替他的魏伯(Weber),在天文台则有哈丁(Harding),他主讲实践天文学。当天文台建成后,高斯住进了新天文台的西翼,哈丁住进东翼。他们曾是朋友,但随着时间的推移,关系疏远,可能是因为——正如一些传记作者推测——高斯希望和哈丁平起平坐,只能是他的助手或观察员。高斯几乎独自使用新的经纬仪,并将它们与哈丁隔离,除了极少数的联合观测。
\begin{figure}[ht]
\centering
\includegraphics[width=6cm]{./figures/d490526cca326556.png}
\caption{威廉·韦伯和海因里希·埃瓦尔德(前排),作为哥廷根七贤的成员} \label{fig_KRGS_7}
\end{figure}
布伦德尔(Brendel)将高斯的天文活动按时间顺序划分为七个阶段,其中自1820年以来被视为“天文活动较低的时期”。新建的、设备齐全的天文台并未像其他天文台那样高效运作;高斯的天文研究具有单人企业的特征,缺乏长期的观察计划,直到哈丁于1834年去世后,大学才为他设立了助手职位。

尽管如此,高斯两次拒绝了通过接受来自1810年和1825年柏林的邀请,成为普鲁士科学院的正式成员,而不承担讲课责任,以及1810年莱比锡大学和1842年维也纳大学的邀请,可能是因为家庭的经济困境。高斯的薪水从1810年的1000里希塔尔升至1824年的2400里希塔尔,晚年他成为大学薪酬最高的教授之一。
\begin{figure}[ht]
\centering
\includegraphics[width=8cm]{./figures/8e5041a341a9bab4.png}
\caption{高斯临终时(1855年)(菲利普·佩特里拍摄的银版照)} \label{fig_KRGS_8}
\end{figure}
在1810年,高斯的同事和朋友弗里德里希·威廉·贝塞尔(Friedrich Wilhelm Bessel)因缺乏学术头衔而在柯尼斯堡大学遇到困境时,高斯为他提供了荣誉博士学位,来自哥廷根大学哲学系。高斯也曾为索菲·热尔曼(Sophie Germain)提供过荣誉学位的推荐,但是在她去世前不久,所以她未能获得该学位。他还成功支持了数学家戈特霍尔德·艾森斯坦(Gotthold Eisenstein)在柏林的工作。

高斯忠诚于汉诺威王室。在威廉四世国王于1837年去世后,新国王厄尔内斯特·奥古斯都(Ernest Augustus)废除了1833年的宪法。七名教授,后来被称为“哥廷根七人”,对此进行了抗议,其中包括他的朋友和合作者威廉·魏伯(Wilhelm Weber)和高斯的女婿海因里希·埃瓦尔德(Heinrich Ewald)。所有人都被解职,其中三人被驱逐,但埃瓦尔德和魏伯得以留在哥廷根。高斯对此争执深感痛心,但认为无法帮助他们。

高斯参与了学术管理:三次被选为哲学系的院长。作为大学遗孀抚恤金基金的负责人,他处理了精算学,并写了一份关于稳定福利的策略报告。他还被任命为哥廷根皇家科学院的院长,担任了九年。

高斯即使在年老体衰、患有痛风和普遍不满的情况下,依然保持着思想上的活跃。1855年2月23日,他因心脏病发作在哥廷根去世;并葬于那里的阿尔巴尼公墓。高斯的女婿海因里希·埃瓦尔德和高斯的亲密朋友兼传记作家沃尔夫冈·萨托里乌斯·冯·瓦尔特斯豪森为他在葬礼上作了悼词。

高斯是一位成功的投资者,通过股票和证券积累了可观的财富,最终财富价值超过15万塔尔。高斯去世后,他的房间里发现了约18,000塔尔的藏款。
\subsubsection{高斯的大脑}  
高斯去世后的第二天,他的大脑被取出、保存并由鲁道夫·瓦格纳进行研究,发现其质量略高于平均值,为1492克(3.29磅)。瓦格纳的儿子赫尔曼,一位地理学家,在他的博士论文中估算了大脑的面积为219,588平方毫米(340.362平方英寸)。2013年,一位来自哥廷根马克斯·普朗克生物物理化学研究所的神经生物学家发现,由于标签错误,高斯的大脑在第一次研究后很快就与几个月后在哥廷根去世的医生康拉德·海因里希·福克斯的大脑混淆。进一步的研究显示,两者的大脑并无显著异常。因此,直到1998年,所有关于高斯大脑的研究(除了鲁道夫和赫尔曼·瓦格纳的首次研究)实际上都指的是福克斯的大脑。
\subsubsection{家庭}
\begin{figure}[ht]
\centering
\includegraphics[width=6cm]{./figures/d1ed993b231bc6d7.png}
\caption{哥萨斯的第二任妻子,威尔赫尔米娜·瓦尔德克} \label{fig_KRGS_5}
\end{figure}
高斯于1805年10月9日在布伦瑞克的圣凯瑟琳教堂与约翰娜·奥斯托夫结婚。他们有两个儿子和一个女儿:约瑟夫(1806–1873)、威尔赫尔米娜(1808–1840)和路易斯(1809–1810)。约翰娜于1809年10月11日去世,路易斯出生一个月后也去世。高斯为孩子们起名时,分别以第一颗小行星的发现者朱塞佩·皮亚齐、威廉·奥尔伯斯和卡尔·路德维希·哈丁的名字命名。

1810年8月4日,高斯与约翰娜的朋友威尔赫尔米娜(米娜)·瓦尔德克结婚,他们有了三个孩子:尤金(后来的尤金)(1811–1896)、威廉(后来的威廉)(1813–1879)和特雷莎(1816–1864)。米娜·高斯在1831年9月12日去世,之前她病重了十多年。此后,特雷莎接管了家庭并照顾高斯直到他去世;她父亲去世后,她嫁给了演员康斯坦丁·斯陶芬诺。她的妹妹威尔赫尔米娜嫁给了东方学者海因里希·埃瓦尔德。高斯的母亲多萝西娅从1817年起住在他家中,直到1839年去世。

长子约瑟夫在还是中学生时,曾在1821年夏天的测量工作中作为助手帮助父亲。短暂在大学学习后,约瑟夫于1824年加入了汉诺威军队,并在1829年再次参与测量工作。1830年代,他负责扩展测量网络到王国西部地区。凭借他的测量资格,他离开了军队,并作为皇家汉诺威国家铁路公司总监从事铁路建设。1836年,他曾在美国研究了几个月的铁路系统。

尤金于1830年9月离开哥廷根,移民到美国,加入军队服役五年。之后,他在美国中西部为美国毛皮公司工作。后来,他移居密苏里州,成为一名成功的商人。威廉娶了天文学家贝塞尔的侄女;随后他搬到密苏里州,开始做农场主,后来在圣路易斯的制鞋业中变得富有。尤金和威廉在美国有许多后代,而留在德国的高斯后代全部来自约瑟夫,因为他的女儿们没有子嗣。
\subsubsection{个性}  
\textbf{学者}
\begin{figure}[ht]
\centering
\includegraphics[width=8cm]{./figures/9d218617e444768f.png}
\caption{学生画他的数学教授:高斯(1795年)画的亚伯拉罕·戈特赫尔夫·凯斯特纳的讽刺画} \label{fig_KRGS_9}
\end{figure}
在19世纪的前二十年,哥萨斯是德国唯一重要的数学家,可以与当时法国的数学领袖相媲美;他的《算术研究》是第一本被翻译成法语的德国数学著作。

哥萨斯“走在新发展前沿”,自1799年起开始有文献记载的研究,凭借丰富的新思想和严谨的证明方法。[71]与之前的数学家如莱昂哈德·欧拉不同,后者让读者参与思考过程,展现一些错误的偏差,哥萨斯则引入了一种全新的风格,直接且完整的解释,避免了让读者理解作者思路的尝试。[73]
\begin{figure}[ht]
\centering
\includegraphics[width=6cm]{./figures/a58c371101fa76f1.png}
\caption{学生画他的数学教授:高斯由他的学生约翰·本尼迪克特·利斯廷(1830年)所画} \label{fig_KRGS_10}
\end{figure}
哥萨斯是第一个恢复了我们在古代所钦佩的严谨证明的方法,而这种方法在前一时期由于对新发展的过度关注而被不当忽视。

——克莱因 1894年,第101页  
然而,他对自己推广了一种完全不同的理想,在一封写给法卡什·博尔亚的信中,他写道:

“不是知识,而是学习的过程;不是拥有,而是到达的过程,才是带来最大享受的。当我已弄清并穷尽一个课题时,我便会离开它,进入黑暗中再度探索。”

——邓宁顿 2004年,第416页  
他死后的论文、科学日记以及他自己教材中的简短注释显示,哥萨斯在很大程度上是通过经验的方式进行工作。[75][76][77]他一生忙碌且充满热情,擅长计算,通常能够以惊人的速度进行计算,大多数情况下不进行精确控制,但通过巧妙的估算来检验结果。[78]尽管如此,他的计算并不总是无误的。[79]他通过使用高效工具来应对庞大的工作量。[80]哥萨斯使用了大量的数学表格,检验它们的精确性,并为个人使用构建了新的表格。[81]他还开发了有效计算的新工具,例如高斯消元法。[82]有趣的是,他通常会进行比实际所需精度更高的计算,并为实际应用准备比实际需要更多小数位的表格。[83]很可能,这种方法为他在数论中发现定理提供了大量的材料。[79][83]
\begin{figure}[ht]
\centering
\includegraphics[width=6cm]{./figures/10a55cdd9806a8ec.png}
\caption{高斯的印章及其座右铭“Pauca sed Matura”(少而精)} \label{fig_KRGS_11}
\end{figure}
高斯拒绝发表他认为不完整或无法经受批评的作品。这种完美主义与他个人印章上的座右铭“Pauca sed Matura”(“少而精”)相一致。许多同事鼓励他公开新的想法,有时如果他拖得太久,认为他应该发表时,他们会对他提出批评。高斯为自己辩护,声称想法的初步发现很容易,但将这些想法整理成一份可公开的成果对他而言是一个艰巨的任务,要么是因为缺乏时间,要么是因为“心境不宁”。尽管如此,他还是在各种期刊上发表了许多紧急内容的简短通讯,同时也留下了相当可观的文献遗产。高斯称数学为“科学之王”,算术为“数学之王”,并且据说曾经认为,要成为一名一流的数学家,必须立即理解欧拉公式作为一个标杆。

在某些情况下,高斯声称某些学者的想法早已在他自己的脑海中。因此,他关于“发现第一,出版第二”的优先权概念与他的科学同时代人有所不同。与他在呈现数学思想时的完美主义相比,他因引用文献时的粗心大意而受到批评。他为自己辩解,认为引用文献必须以非常完整的方式进行,涉及到所有重要的前人作者,这些人不应该被忽视;但这种引用方式需要对科学史有深入了解,并且比他愿意花的时间要多。

\textbf{私人生活}  

高斯去世后不久,他的朋友萨托里乌斯出版了第一本传记(1856年),以相当热情的风格写成。萨托里乌斯将他视为一个平静且具有进取心的人,具有孩子般的谦逊,但也拥有“铁的性格”和坚定不移的精神力量。除了亲近的人圈子,其他人则认为他是一个内敛且难以接近的人,“像一位奥林匹斯的神坐在科学的顶峰上”。他的同时代人一致认为高斯是一个性格复杂的人。他常常拒绝接受赞美。有时他的访客会因他的脾气暴躁而感到不悦,但过了一会儿,他的情绪会发生变化,成为一个迷人且开明的主人。高斯厌恶争论性格的人;他和同事豪斯曼一起反对让尤斯图斯·李比希在哥廷根大学担任教授,“因为他总是参与一些争论”。
\begin{figure}[ht]
\centering
\includegraphics[width=8cm]{./figures/082120e5db2658d6.png}
\caption{高斯1808至1816年的住所在二楼} \label{fig_KRGS_12}
\end{figure}
高斯的一生受到了家庭重大问题的影响。当他的第一任妻子约翰娜在第三个孩子出生后不久突然去世时,他在给已故妻子的最后一封信中表达了自己的悲痛,这封信以古代悲歌的风格写成,是高斯最为个人化的遗存文件。情况在第二任妻子米娜因结核病折磨身体长达13年后愈发严重;他的两个女儿也都患上了同样的疾病。高斯自己很少暗示自己内心的痛苦:在1831年12月的一封信中,他向贝塞尔提到自己是“最惨痛的家庭苦难的受害者”。

由于妻子的病情,两个较小的儿子被送往远离哥廷根的切尔,接受了几年的教育。他的大儿子约瑟夫在服役超过二十年后,最终仅以一名薪水微薄的中尉军官身份结束了军旅生涯,尽管他在测量学方面有相当的造诣。即使结婚后,他依然需要父亲的经济支持。二儿子尤金也具有和父亲相当的计算和语言天赋,但性格活跃且有时反叛。他本想学习语言学,而高斯希望他成为一名律师。尤金在公共场合陷入了债务危机并引发丑闻后,于1830年9月在戏剧性情况下突然离开哥廷根,通过不来梅移民美国。他很快挥霍了所带的钱,之后父亲拒绝再给予任何经济支持。最小的儿子威廉本想从事农业管理工作,但由于难以获得合适的教育,最终也选择了移民。只有高斯最小的女儿特雷莎陪伴在他晚年左右。

在晚年,高斯养成了收集各种数字数据的习惯,是否有用并不重要,比如他家到哥廷根某些地方的路径数,或某人活过的天数;他在1851年12月祝贺洪堡,以此来庆祝洪堡和艾萨克·牛顿死时的年龄相同,计算单位是天数。

与他出色的拉丁语知识相似,他也精通现代语言。在62岁时,他开始自学俄语,很可能是为了理解来自俄罗斯的科学著作,其中包括罗巴切夫斯基关于非欧几何的著作。高斯阅读了古典和现代文学,并能以原文阅读英语和法语作品。他最喜欢的英语作家是沃尔特·斯科特,最喜欢的德国作家是让·保尔。高斯喜欢唱歌,常常去听音乐会。他是一个热衷的报纸读者,在晚年,他每天中午都会去大学的学术沙龙。

高斯对哲学兴趣不大,曾讽刺过“自称形而上学家”的人们,指的是当时的自然哲学派的支持者。

高斯具有“贵族般的,彻底保守的天性”,对人们的智慧和道德缺乏敬重,秉持着“世界愿意被欺骗”的格言。他不喜欢拿破仑及其制度,对一切暴力和革命都感到恐惧。因此,他谴责了1848年革命中的方法,尽管他同意其中的一些目标,如统一德国的理念。至于政治制度,他对宪政制度的评价较低,批评当时的议员们缺乏知识和逻辑错误。

一些高斯的传记作家推测他的宗教信仰。他曾说过“上帝进行算术”,并表示“我成功了——不是因为我努力工作,而是上帝的恩典”。高斯是路德教会的成员,像北德大部分人一样。似乎他并不完全相信所有教义,也不完全字面理解圣经。萨托里乌斯提到高斯的宗教宽容,并认为他“对真理的渴求”和他的正义感受到了宗教信仰的激励。
\subsection{科学工作}  
\subsubsection{代数与数论}  
\textbf{代数基本定理}
\begin{figure}[ht]
\centering
\includegraphics[width=6cm]{./figures/a6872b087560a2dd.png}
\caption{纪念高斯200周年的德国邮票:复平面或高斯平面} \label{fig_KRGS_13}
\end{figure}
在他1799年的博士论文中,高斯证明了代数基本定理,该定理表明每个具有复系数的非恒定一元多项式至少有一个复数根。此前,包括让·勒朗·达朗贝尔在内的数学家曾给出过错误的证明,高斯的论文中批判了达朗贝尔的工作。此后,他又给出了三种其他的证明,最后一种证明于1849年完成,通常被认为是严格的。他的这些尝试在此过程中大大澄清了复数的概念。[110]

\textbf{《算术研究》}

在《算术研究》的序言中,高斯将他在数论上的工作开始时间定为1795年。通过研究费马、欧拉、拉格朗日和勒让德等前辈数学家的工作,他意识到这些学者已经发现了许多他自己才刚刚得到的结论。[111]《算术研究》从1798年开始写作,于1801年出版,巩固了数论作为一门学科,并涵盖了初等数论和代数数论的内容。在书中,他引入了三重横线符号(≡)表示同余,并利用该符号清晰地呈现了模运算。[112]该书讨论了唯一因数分解定理和模n的原根问题。在主要章节中,高斯给出了二次互反律的前两种证明[113],并发展了二元[114]和三元二次型的理论[115]。

《算术研究》包括了高斯二次型合成定理,以及整数作为三个平方和的表示个数的枚举。作为他关于三个平方定理的几乎直接推论,他证明了费马多边形数定理的三角形情形,即当n = 3时的情况。[116]在第五章末尾,高斯给出了一些关于类数的分析结果,这些结果没有证明,显示出高斯在1801年时已经知道了类数公式。[117][118]

在最后一章中,高斯通过将一个几何问题归结为代数问题,给出了用直尺和圆规构造正17边形(17边形)的证明。[119]他展示了,如果正多边形的边数是2的幂次,或是2的幂次与任意数量的不同费马素数的积,则该正多边形是可构造的。在同一章节中,他还给出了某些三次多项式在有限域上的解的个数的结果,这实际上是对椭圆曲线上整数点的计数。[120]一章未完成的第八章在他去世后才在遗留文件中被发现,这些内容是在1797至1799年间完成的。[121][122]

\textbf{进一步的研究}  

高斯的第一个结果之一是1792年通过经验发现的猜想——后来的素数定理——该定理通过使用积分对数估计素数的数量。[123][o]  

当奥尔伯斯在1816年鼓励高斯竞选法国科学院的奖项,证明费马大定理时,高斯拒绝了,因为他对这一问题的评价较低。然而,在他留下的作品中发现了一篇未注明日期的短文,其中包含了对费马大定理在n = 3和n = 5情况下的证明。[125] n = 3的特殊情况早在莱昂哈德·欧拉时期就已被证明,但高斯发展出了一种更加简洁的证明方法,利用了艾森斯坦整数;尽管证明更为一般,但比实整数情况下的证明要简单。[126]  

高斯在1831年通过证明三维空间中球体的最大堆积密度是在球心形成一个立方体面心排列时给出的,从而为解决开普勒猜想做出了贡献。[127]这是他在回顾路德维希·奥古斯特·泽伯关于正三元二次型约化理论的书籍时发现的。[128] 发现泽伯的证明存在一些不足后,他简化了其中的许多论证,证明了中心猜想,并指出该定理等价于规则排列下的开普勒猜想。[129]  

在关于四次剩余的两篇论文(1828年,1832年)中,高斯引入了高斯整数环\(\mathbb{Z}[i]\)
并证明它是一个唯一因子分解域。[130] 他还推广了一些关键的算术概念,如费马小定理和高斯引理。引入这个环的主要目的是提出四次互反律[130]——正如高斯所发现的,复整数环是这类更高互反律的自然背景。[131]  

在第二篇论文中,他陈述了四次互反律的一般法则,并证明了其中几个特例。在1818年他发布的一篇文章中,包含了他对二次互反律的第五和第六次证明,他宣称这些证明的技巧(高斯和)可以应用于证明更高的互反律。[132]
\subsubsection{分析}  
高斯的第一个发现之一是两个正实数的算术-几何平均数(AGM)的概念。[133] 他在1798至1799年间通过兰登变换发现了它与椭圆积分的关系,一篇日记记录了高斯常数与二次椭圆函数的联系,这一结果高斯表示“将无疑打开一个全新的分析领域”。[134] 他还早期开始研究复分析基础的更正式问题,并且从1811年给贝塞尔的信件中可以看出,他已经知道了“复分析基本定理”——柯西积分定理——并理解了在围绕极点积分时复残差的概念。[120][135]

欧拉的五边形数定理,以及他对AGM和二次椭圆函数的其他研究,促使他获得了许多关于雅可比θ函数的结果,[120] 最终在1808年发现了后来的雅可比三重积恒等式,这包括了欧拉定理作为一个特例。[136] 他的工作表明,从1808年起他已经知道椭圆函数的3阶、5阶和7阶模变换。[137][p][q]  

他在遗稿中的几篇数学片段表明,他知道现代模形式理论的部分内容。[120] 在他关于两个复数的多值AGM的研究中,他发现了AGM的无数个值与其两个“最简单值”之间的深刻联系。[134] 在他未发表的著作中,他认识并画出了模群基本域的关键概念草图。[139][140] 高斯的其中一幅此类草图是对单位圆盘的“等边”双曲三角形镶嵌的描绘,这些三角形的所有角度都等于 
\(\pi /4\)。[141]

高斯在分析领域的洞察力的一个例子是他关于圆分割的原则可以应用于分割二次椭圆曲线的神秘评论,这启发了阿贝尔关于二次椭圆分割的定理。[r] 另一个例子是他在《Summatio quarundam serierum singularium》(1811)中关于二次高斯和符号判定的研究,他通过引入二项式系数的q-类比并使用几种原始恒等式对其进行操作,成功解决了这个主要问题,这些恒等式似乎源自他在椭圆函数理论方面的工作;然而,高斯以一种正式的方式表达他的论证,未揭示出其在椭圆函数理论中的来源,直到后来如雅可比和埃尔米特等数学家的工作才揭示了他的论证的核心。[142]

在《Disquisitiones generales circa series infinitam...》(1813)中,他首次系统地处理了广义超几何函数\(F(\alpha, \beta, \gamma, x)\)并证明了当时已知的许多函数都是超几何函数的特例。[143] 这项工作是数学史上首次对无限级数的收敛性进行精确探讨。[144] 此外,它还涉及了作为超几何函数比值而产生的无限连分数,现在被称为高斯连分数。[145]

1823年,高斯因其关于共形映射的论文获得了丹麦学会奖,该论文包含了与复分析领域相关的若干发展。[146] 高斯指出,复平面中的保持角度的映射必须是复分析函数,并使用后来的贝尔特拉米方程证明了在分析曲面上存在等温坐标。该论文以共形映射到球面和旋转椭球体的例子作为结尾。[147]

\textbf{数值分析}  

高斯经常通过归纳法从他通过经验收集的数值数据中推导定理。[77] 因此,使用高效的算法来促进计算对于他的研究至关重要,他在数值分析方面做出了许多贡献,其中包括在1816年发布的高斯求积法。[148]

在1823年写给格尔林的私人信件中,[149] 高斯描述了使用高斯-赛德尔法求解4x4线性方程组的过程——这是一种求解线性方程组的“间接”迭代方法,并推荐它优于通常的“直接消元法”,特别是对于方程数量超过两个的系统。[150]

高斯在1805年计算帕拉斯星和朱诺星的轨道时发明了一种算法,用于计算现在称为离散傅里叶变换(Discrete Fourier Transform, DFT),这比库利和图基在160年后发现的库利-图基快速傅里叶变换(Cooley–Tukey FFT)算法要早。[151] 他将其作为一种三角插值方法进行开发,但相关论文《Theoria Interpolationis Methodo Nova Tractata》直到1876年才出版,这是高斯去世后发布的;这篇论文早于约瑟夫·傅里叶于1807年首次介绍的相关研究。[153]
\subsubsection{时间表}  
继博士论文之后的第一篇出版物是关于复活节日期的确定(1800年),这是一个基础的数学问题。高斯旨在为没有任何教会或天文学年表知识的人提供最便捷的算法,因此避免使用通常需要的术语,如黄金数、复历、太阳周期、星期字母以及任何宗教含义。[154] 传记作者曾推测高斯为何会处理这个问题,但从历史背景来看,或许是可以理解的。自16世纪以来,儒略历被改为格里历在神圣罗马帝国内引起了混乱,并且直到1700年德国才完成这一更替,届时删除了11天的差距,但在复活节日期的计算上,天主教和新教地区之间依然存在差异。1776年的进一步协议统一了不同宗派的计算方式;因此,在像布伦瑞克公国这样的新教国家,1777年的复活节——距高斯出生五周——是第一个按新方法计算的复活节。[155] 更替过程中的公众困扰可能构成了高斯家族对此问题混淆的历史背景(参见章节:轶事)。由于与复活节规定相关,关于逾越节日期的论文随后于1802年发表。[156]
\subsubsection{天文学}
\begin{figure}[ht]
\centering
\includegraphics[width=6cm]{./figures/2e87bd19be233e67.png}
\caption{卡尔·弗里德里希·高斯,1803年,约翰·克里斯蒂安·奥古斯特·施瓦茨画作} \label{fig_KRGS_14}
\end{figure}
1801年1月1日,意大利天文学家朱塞佩·皮亚兹发现了一个新的天体,依据所谓的提图斯–博德定律,他推测这是长期寻找的位于火星和木星之间的行星,并命名为谷神星(Ceres)。[157] 但他只能追踪这个天体短暂的时间,直到它消失在太阳的光辉后。那时的数学工具不足以从有限的数据推算出其重新出现的位置。高斯解决了这个问题,并预测了该天体可能的重新发现位置在1801年12月。最终,弗朗茨·泽维尔·冯·扎赫(Franz Xaver von Zach)于12月7日和31日在哥达,亨利·奥尔伯斯(Heinrich Olbers)于1月1日和2日在不来梅,独立地在预定位置附近发现了这个天体,偏差仅为半度。[158][s]

高斯的方法得出了一一个8次方的方程,其中一个解为地球的轨道已知。接下来,依据物理条件将所寻找的解与其余六个解分开。在这项工作中,高斯使用了他为此目的创造的综合近似方法。[159]

谷神星的发现促使高斯提出了行星小天体在大行星引力干扰下的运动理论,最终于1809年以《天体运动理论——绕太阳运行的圆锥曲线截面》为名发表。[160] 这篇论文引入了高斯引力常数。[33]

自从新的小行星被发现后,高斯便开始研究它们轨道元素的摄动。首先,他用类似拉普拉斯的方法分析了谷神星,但他最喜欢的天体是帕拉斯,因为它具有很大的偏心率和轨道倾角,而拉普拉斯的方法无法应用。高斯使用了自己的工具:算术–几何平均数、超几何函数和插值法。[161] 1812年,他发现帕拉斯与木星有18:7的轨道共振;高斯以密码形式给出了这个结果,并在给奥尔伯斯和贝塞尔的信中才明确其含义。[162][163][t] 在经过多年的工作后,他在1816年完成了这项研究,但没有得出他认为足够的结果。这标志着他在理论天文学领域活动的结束。[165]
\begin{figure}[ht]
\centering
\includegraphics[width=8cm]{./figures/0b1ce75abe7ea7cd.png}
\caption{ Göttingen天文台从西北方向看(由弗里德里希·贝泽曼绘制,约1835年)} \label{fig_KRGS_15}
\end{figure}
高斯对帕拉斯扰动的研究成果之一是《Determinatitio Attractionis...》(1818年),该方法是理论天文学中的一种方法,后来被称为“椭圆环法”。它引入了一个平均概念,其中轨道上的行星被一个虚拟环替代,这个环的质量密度与行星沿相应轨道弧段所需的时间成比例。[166] 高斯展示了如何计算这样一个椭圆环的引力吸引力的方法,这个方法包括多个步骤;其中之一涉及直接应用算术-几何平均(AGM)算法来计算椭圆积分。[167]

虽然高斯的理论天文学贡献到此为止,但他在观察天文学方面的实际活动持续并贯穿了他的整个职业生涯。早在1799年初,高斯就开始研究利用月球视差来确定经度的问题,他为此开发了比当时常用的公式更为便捷的计算方法。[168] 在被任命为天文台台长后,他重视与贝塞尔的天文常数通信。高斯本人提供了关于岁差和天体偏差、太阳坐标以及折射的表格。[169] 他还对球面几何学做出了许多贡献,并在此背景下解决了一些关于星象导航的实际问题。[170] 他发表了大量的天文观测,主要涉及小行星和彗星;他的最后一次观测是1851年7月28日的日全食。[171]
\subsubsection{误差理论}  
高斯可能在计算谷神星的轨道时使用了最小二乘法,以最小化测量误差的影响。[88] 该方法最早由阿德里安-玛丽·勒让德于1805年发布,但高斯在《天体运动理论》(1809年)中声称,他自1794年或1795年起就开始使用该方法。[172][173][174] 在统计学史上,这种争议被称为“最小二乘法发现的优先权争议”。[88] 高斯证明了在假设误差服从正态分布的情况下,该方法在所有线性无偏估计量中具有最低的采样方差(高斯-马尔科夫定理),这一成果出现在两篇论文《观测误差的最小组合理论》(1823年)中。[175]  

在第一篇论文中,他证明了高斯不等式(一种切比雪夫类型的不等式)适用于单峰分布,并未证明另一个关于四阶矩的不等式(高斯-温克勒不等式的特例)。[176] 他推导了样本方差的方差的上下界。在第二篇论文中,高斯描述了递归最小二乘法。高斯关于误差理论的工作在多个方向上被测量学家弗里德里希·罗伯特·赫尔梅特扩展至高斯-赫尔梅特模型。[177]

高斯还为概率理论中的一些问题做出了贡献,尽管这些问题与误差理论没有直接关系。一个例子出现在他的日记中,他试图描述一个随机数在区间(0,1)内均匀分布的连分数展开项的渐近分布。他从高斯映射的遍历性发现中推导出了这个分布,现在被称为高斯-库兹明分布。高斯的解决方案是连分数度量理论中的第一个结果。[178]
\subsubsection{大地弧测量与大地测量调查}
\begin{figure}[ht]
\centering
\includegraphics[width=6cm]{./figures/ede346ff6e7561f8.png}
\caption{1820年5月9日乔治四世国王下达的三角测量项目命令(下方附有恩斯特·冯·明斯特伯爵的额外签名)} \label{fig_KRGS_16}
\end{figure}
自1799年以来,高斯便忙于大地测量问题,当时他协助卡尔·路德维希·冯·莱科克进行在威斯特法伦地区的测量计算。[179] 自1804年起,他在布伦瑞克[180] 和哥廷根[181] 自学了部分大地测量实践,使用的是六分仪。

自1816年起,高斯的前学生海因里希·克里斯蒂安·舒马赫(当时是哥本哈根大学的教授,居住在汉堡附近的阿尔托纳,作为天文台的负责人)开始进行日德兰半岛的三角测量,从北部的斯卡根到南部的劳恩堡。[u] 该项目不仅为制图提供了基础,还旨在确定终端站点之间的大地弧长。大地弧长的数据被用于确定地球大地水准面的尺寸,较长的大地弧距离提供了更精确的结果。舒马赫请求高斯继续向南扩展这项工作,覆盖下萨克森王国的区域;高斯在短暂犹豫后同意了。最终,1820年5月,乔治四世国王下令高斯继续这项工作。[182]
\begin{figure}[ht]
\centering
\includegraphics[width=8cm]{./figures/00049dbede8f0e00.png}
\caption{太阳测距仪} \label{fig_KRGS_17}
\end{figure}
测量一条大地弧线需要精确的天文测定,至少要有两点的确定位置。高斯和舒马赫利用两座天文台的有利位置,这两座天文台几乎位于相同的经度线上,一座位于哥廷根,另一座位于舒马赫家的花园中的阿尔托纳。两人使用各自的仪器和一个拉姆斯登的天顶仪进行纬度测量,天顶仪被运送到两个天文台。[183][v]

高斯和舒马赫早在1818年10月便已经确定了吕讷堡、汉堡和劳恩堡之间的一些角度,为大地连接做准备。[184] 在1821年到1825年夏季期间,高斯亲自指导了从南部的图林根到北部易北河的三角测量工作。霍赫·哈根山、图林根森林中的格罗瑟·因塞尔斯贝格山和哈茨山脉中的布罗肯山之间的三角形是高斯测量过的最大一个,最大边长为107公里(66.5英里)。在几乎没有显著自然山峰或人工建筑物的稀疏人口的吕讷堡荒原,高斯在寻找合适的三角测量点时遇到了困难,有时不得不砍开通往这些点的道路。[155][185]
\begin{figure}[ht]
\centering
\includegraphics[width=6cm]{./figures/d5231a060b23c288.png}
\caption{高斯的副太阳测距仪,一种配有附加镜面的特劳顿六分仪} \label{fig_KRGS_18}
\end{figure}
为了定位信号,高斯发明了一种新仪器,配有可移动的镜子和小型望远镜,将太阳光反射到三角测量点上,并将其命名为日光仪(heliotrope)。[186] 另一种适用的构造是带有附加镜子的六分仪,高斯将其命名为副日光仪(vice heliotrope)。[187] 高斯得到了汉诺威军队士兵的帮助,其中包括他的长子约瑟夫。高斯参加了1820年舒马赫在汉堡附近布拉克村的基准线(布拉克基准线)测量,并将这一结果用于汉诺威三角测量的评估。[188]

此外,项目还得出了地球椭球形的扁率的更精确值。[189][w] 高斯开发了地球椭球的通用横向墨卡托投影(他称之为共形投影),用于在平面图表中表示大地测量数据。[191]

当大地弧测量完成后,高斯开始了向西扩展三角测量的工作,旨在通过1828年3月25日的皇家法令,进行整个汉诺威王国的测量。[192] 实际工作由三名军官负责,其中包括中尉约瑟夫·高斯。所有数据的评估由高斯亲自进行,他应用了诸如最小二乘法和消去法等数学发明。项目于1844年完成,高斯将最终报告提交给政府;他的投影方法直到1866年才被编辑出版。[193][194]

1828年,在研究纬度差异时,高斯首次定义了地球形状的物理近似,即地球表面在所有地方都垂直于重力方向;[195] 后来,他的博士生约翰·本尼迪克特·利斯廷称此为“大地水准面”。[196]
\subsubsection{微分几何}
汉诺威的测量工作激发了高斯对微分几何和拓扑学的兴趣,这些数学领域研究曲线和曲面。这使他在1828年发表了一篇论文,标志着现代微分几何学的诞生。该论文不同于传统的将曲面看作是二维变量函数的笛卡尔图形的处理方式,开启了从“内在”角度探索曲面的研究,即从一个受限于在曲面上运动的二维存在者的角度来研究曲面。由此,《卓越定理》确立了高斯曲率这一概念的性质。非正式地说,该定理表明,曲面的曲率可以完全通过在曲面上测量角度和距离来确定,而与曲面在三维或二维空间中的嵌入方式无关。[197]

《卓越定理》引出了将曲面抽象为双重延展流形的概念;它澄清了流形的内在性质(度量)与其在外部空间中的物理实现之间的区别。其后果是,不同高斯曲率的曲面之间无法进行等距变换。这意味着,球面或椭球体不能在不失真的情况下变换为平面,这给地理地图投影设计带来了一个基本问题。[197] 本文的一部分致力于对测地线的深入研究。特别是,高斯证明了局部的高斯—博内定理(Gauss–Bonnet theorem)对于测地三角形的适用性,并将勒让德定理关于球面三角形的定理推广到具有连续曲率的任意曲面上的测地三角形;他发现,“足够小”的测地三角形的角度与相同边长的平面三角形的角度的偏差,仅取决于三角形顶点处曲面的曲率值,而与三角形内部曲面的行为无关。[198]

高斯1828年的论文中缺乏测地曲率的概念。然而,在一篇之前未发表的手稿中,这篇手稿很可能是在1822到1825年间写成的,他引入了“侧向曲率”(德语:Seitenkrümmung)这一术语,并证明了其在等距变换下的不变性,这一结果后来由费迪南德·明丁(Ferdinand Minding)获得,并于1830年发表。这篇高斯的论文包含了他关于总曲率的引理的核心内容,但也包含了它的推广,由皮埃尔·奥西安·博内(Pierre Ossian Bonnet)于1848年发现并证明,后来被称为高斯—博内定理。[199]
\subsubsection{非欧几里得几何}
\begin{figure}[ht]
\centering
\includegraphics[width=6cm]{./figures/5f6ed8db5b07f363.png}
\caption{塞格弗里德·本迪克森(Siegfried Bendixen)绘制的石版画(1828年)} \label{fig_KRGS_19}
\end{figure}
在高斯生前,关于欧几里得几何中的平行公设展开了激烈的讨论。[200] 许多努力尝试在欧几里得公理体系框架内证明这一公设,同时一些数学家讨论了没有平行公设的几何体系的可能性。[201] 高斯自1790年代起就思考几何的基础问题,但在1810年代,他意识到没有平行公设的非欧几里得几何可以解决这一问题。[202][200] 在1824年写给弗朗茨·陶里努斯(Franz Taurinus)的一封信中,他简要而易懂地概述了他所称的“非欧几里得几何”,[203] 但他强烈禁止陶里努斯使用这一概念。[202] 高斯被认为是第一个发现并研究非欧几里得几何的人,甚至还创造了这一术语。[204][203][205]

非欧几里得几何的第一篇数学历史上的出版物由尼古拉·罗巴切夫斯基(Nikolai Lobachevsky)于1829年和扬诺什·博尔亚伊(Janos Bolyai)于1832年发表。[201] 在接下来的几年里,高斯写下了他关于这一主题的想法,但并没有发表,以避免对当时的科学讨论产生影响。[202][206] 高斯在给他的父亲和大学朋友法尔卡什·博尔亚伊(Farkas Bolyai)的一封信中称赞了扬诺什·博尔亚伊的想法,[207] 并声称这些想法与他几十年前的思想是一致的。[202][208] 然而,尚不清楚他在多大程度上先于罗巴切夫斯基和博尔亚伊,因为他的信件中的评论只是模糊和晦涩的。[201]

萨托里乌斯(Sartorius)在1856年首次提到高斯关于非欧几里得几何的工作,但直到《高斯文集》第八卷(1900年)中左存文件的出版,才展示了高斯在这一问题上的思想,而此时非欧几里得几何尚处于争议讨论的阶段。[202]
\subsubsection{早期拓扑学}  
高斯还是拓扑学的早期先驱之一,拓扑学在他生前被称为“几何位置学”(Geometria Situs)。他在1799年首次证明的代数基本定理中,包含了一个本质上属于拓扑学的论证;五十年后,他在该定理的第四个证明中进一步发展了这个拓扑学的论证。[209]
\begin{figure}[ht]
\centering
\includegraphics[width=6cm]{./figures/b6317e16a16c9445.png}
\caption{海因里希·赫瑟曼(Heinrich Hesemann)雕刻的高斯半身像(1855年)[x]} \label{fig_KRGS_20}
\end{figure}
高斯在他的天文学工作中于1804年再次遇到拓扑学的概念,当时他确定了天球上彗星和小行星可能出现的区域的界限,并将其称为“黄道”。他发现,如果地球和彗星的轨道是相连的,那么从拓扑学的角度来看,黄道实际上是整个天球。1848年,在发现小行星7号“伊里斯”时,他发表了对黄道的进一步定性讨论。[210]

在高斯1820年至1830年的信件中,他对与“几何位置学”紧密相关的主题进行了深入思考,并逐渐意识到该领域的语义困难。来自这一时期的片段显示,他曾尝试对“区段图形”进行分类,这些图形是有限个横向自交点的封闭平面曲线,也可以是结的平面投影。[211] 为此,他设计了一个符号体系,即高斯代码,这在某种意义上捕捉了区段图形的特征。[212][213]

在1833年的一个片段中,高斯通过某个双重积分定义了两个空间曲线的连结数,并首次为拓扑现象提供了一个解析的表述。在同一笔记中,他 lamented拓扑学中进展缓慢,并指出该领域的一个核心问题将是“计算两个闭合或无限曲线的交织”。他在那一时期的笔记本中还透露出他对其他拓扑对象(如辫子和纠结)的思考。[210]

高斯在后期对新兴的拓扑学领域的影响,主要通过偶尔的评论和与莫比乌斯及列斯廷的口头交流,他高度评价这一领域。[214]
\subsubsection{高斯的小数学成就}  
高斯应用复数概念,以一种新的简洁方式解决了著名的数学问题。例如,在1836年关于三元形式的几何学方面及其在晶体学中的应用的简短论文中,[215] 他陈述了轴测投影的基本定理,该定理告诉我们如何通过复数将三维立方体准确地表示在二维平面上。[216] 他将球体的旋转描述为某些线性分式变换作用于扩展复平面,[217] 并证明了几何定理,即三角形的三高线总是交于一个单一的正交中心。[218]

高斯对约翰·纳皮尔的《奇异五角星》(Pentagramma mirificum)——一种特定的球面五角星——进行了数十年的研究;[219] 他从不同角度接近这一问题,并逐渐全面理解了其几何、代数和分析方面的内容。[220] 特别地,1843年他陈述并证明了几个定理,这些定理将椭圆函数、纳皮尔球面五边形和庞塞莱平面五边形联系起来。[221]

此外,他还为在给定四边形内构造最大面积的椭圆问题提供了解决方案,[222][223] 并发现了一个关于五边形面积计算的惊人结果。[224][225]
\subsubsection{磁学与电报学}  
\textbf{地磁学}
\begin{figure}[ht]
\centering
\includegraphics[width=6cm]{./figures/626b1242c2acda83.png}
\caption{高斯-韦伯纪念碑,位于哥廷根,由费迪南德·哈茨尔(Ferdinand Hartzer)于1899年创作。} \label{fig_KRGS_21}
\end{figure}
高斯自1803年起便对磁学产生了兴趣。[226] 在1826年亚历山大·冯·洪堡访问哥廷根后,两位科学家开始了关于地磁学的深入研究,部分是独立进行的,部分则是富有成效的合作。[227] 1828年,在德国自然科学家和医师学会会议期间,洪堡邀请高斯做客柏林,在那里他结识了物理学家威廉·韦伯。[228]

当韦伯于1831年在高斯的推荐下接替约翰·托比亚斯·迈耶成为哥廷根的物理学教授时,两人开始了富有成果的合作,推动了磁学的新知识发展,并提出了通过质量、电荷和时间来表示磁学单位的理论。[229] 他们共同成立了磁学协会(德语:Magnetischer Verein),这是一个国际性工作小组,多个天文台共同参与,支持在世界各地按照统一方法、在规定的日期内进行地球磁场的测量,活动持续至1836至1841年。[230]
\begin{figure}[ht]
\centering
\includegraphics[width=8cm]{./figures/beb92f66f51c8110.png}
\caption{高斯-韦伯磁力计} \label{fig_KRGS_22}
\end{figure}
1836年,洪堡建议建立一个全球范围的地磁站网,并通过信件向英国皇家学会会长萨塞克斯公爵提出此建议;他提议使用他的测量方法,在标准化条件下进行磁场测量。[231][232] 这项倡议与其他发起人共同推动,最终形成了一个名为“磁学征途”的全球计划,由爱德华·萨宾(Edward Sabine)负责。观测的日期、时间和间隔提前确定,使用哥廷根标准时间作为标准。[233] 该计划涉及全球五大洲的61个观测站。高斯和韦伯创办了一个系列出版该计划成果,六卷成果集于1837至1843年间出版。韦伯因哥廷根七贤事件的后果于1843年离开哥廷根,标志着磁学协会活动的结束。[230]

效仿洪堡的做法,高斯下令在天文台的花园中建立一座磁学观测台,但在仪器设备上,科学家们意见不一;高斯倾向于使用固定仪器,他认为这样能获得更精确的结果,而洪堡则习惯使用可移动的仪器。高斯对磁偏角、倾角和磁场强度的时间和空间变化感兴趣,但他区别对待了洪堡提出的磁强度概念,将其划分为“水平强度”和“垂直强度”。他与韦伯一起发展了测量磁场强度分量的方法,并设计了一种合适的磁力计,用于测量地球磁场的绝对值,而不再是依赖于仪器的相对值。[230][234] 该磁力计的精度约是以前仪器的十倍。通过这项工作,高斯成为第一个通过基本机械量推导非机械量的科学家。[233]

高斯在1839年提出了《地球磁学通论》,他认为这篇论文描述了磁力的本质;根据费利克斯·克莱因的观点,这项工作实际上是通过球面调和函数展示观测数据,而不是提出一种物理理论。[235] 该理论预测了地球上存在两个磁极,从而使汉斯滕提出的四极理论不再成立,[236] 并且该数据使得确定磁极位置具有相当高的精确度。[237]

高斯的研究影响了俄罗斯地球物理学的开端,当他的学生之一阿道夫·西奥多·库普费尔在圣彼得堡建立了磁学观测台时,便效仿了哥廷根的天文台。同样,伊凡·西蒙诺夫在喀山也建立了类似的观测台。[236]

\textbf{电磁学}

\begin{figure}[ht]
\centering
\includegraphics[width=6cm]{./figures/98bc336e48b9db2d.png}
\caption{哥廷根市区规划图,标出了电报线路的走向} \label{fig_KRGS_23}
\end{figure}
汉斯·克里斯蒂安·厄尔斯特(Hans Christian Ørsted)在电磁学上的发现以及迈克尔·法拉第(Michael Faraday)在电磁感应方面的研究引起了高斯的关注。[238] 高斯和韦伯为分支电路找到了一些规则,后来这些规则被古斯塔夫·基尔霍夫(Gustav Kirchhoff)独立发现,并首次发表,后被命名为基尔霍夫电路定律,[239] 他们还对电磁学进行了研究。他们于1833年建造了第一台电机电报机,韦伯本人将天文台与市中心的物理研究所连接起来,[y] 但他们并未关心该发明的商业化发展。[240][241]

高斯在电磁学中的主要理论兴趣体现在他尝试制定支配电磁感应的定量规律。在这些年的笔记本中,他记录了几项创新的公式;他发现了向量势函数的概念(该概念在1845年由弗朗茨·恩斯特·诺伊曼(Franz Ernst Neumann)独立重新发现),并且在1835年1月,他写下了一个等同于法拉第定律的“感应定律”,该定律指出,给定空间点的电动势等于该函数随时间变化的瞬时速率。[242][243]

高斯曾试图找到一个统一的定律,用于描述静电学、电动力学、电磁学和电磁感应的远程效应,类似于牛顿的引力定律,[244] 但他的尝试最终以“悲剧性的失败”告终。[233]
\subsubsection{势理论}
自从艾萨克·牛顿理论上证明地球和旋转的恒星呈现非球形以来,椭球体的引力问题在数学天文学中变得非常重要。在他关于势理论的第一篇论文《Theoria attractionis...》(1813年)中,高斯为均匀三轴椭球体在空间中每个点的引力提供了一个闭式表达式。[245] 与之前的麦克劳林、拉普拉斯和拉格朗日的研究不同,高斯的新解直接以椭圆积分的形式处理了引力问题。在这个过程中,他还证明并应用了所谓的高斯定理的一些特殊情况,这些定理是矢量分析中的基础。[246]

在《关于吸引力和排斥力的通用定理》(1840年)中,高斯为磁势理论奠定了基础,基于拉格朗日、拉普拉斯和泊松的理论;[235] 似乎他并未了解乔治·格林在这个领域的早期工作。[238] 然而,高斯从未能够为磁学提供任何理由,也未能像牛顿在引力理论中所做的那样,提出类似的磁学理论,使科学家们能够预测未来的地磁效应。[233]
\subsubsection{光学}
高斯的计算使得汉堡的仪器制造商约翰·乔治·雷普索尔德在1810年能够构造出一种新的消色差透镜系统。一个主要问题是,对所使用玻璃类型的折射率和色散的知识不准确,这也是其他困难之一。[247] 在1817年发表的一篇短文中,高斯解决了双透镜中消除色差的问题,并计算了为了最小化色差所需的形状和折射系数的调整。他的工作引起了光学家卡尔·奥古斯特·冯·施泰因海尔的注意,后者在1860年推出了基于高斯计算的消色差施泰因海尔双透镜。[248] 高斯在几何光学方面的许多成果仅散见于他的信件和手记中。[249]

在《光学研究》(1840年)中,高斯首次对在旁轴近似下成像的形成进行了系统分析(高斯光学)。[250] 他仅通过光学系统的主点来表征在旁轴近似下的光学系统,[251] 并推导出了高斯透镜公式,该公式在透镜厚度方面没有限制。[252][253]
\subsubsection{力学} 
高斯在力学方面的首次工作涉及地球的旋转。当他的大学朋友本岑贝格在1802年进行实验,测定下落物体偏离垂直方向的程度时,这一现象今天被称为科里奥利力效应,他请高斯基于理论计算出相应的数值以与实验数据进行比较。高斯为运动建立了一套基本方程体系,结果与本岑贝格的实验数据相当吻合,本岑贝格将高斯的理论作为附录添加到他关于自由落体实验的书中。[254]

1851年,当福科通过他的摆实验公开演示地球的旋转时,格尔林向高斯请求进一步的解释。这促使高斯设计了一种新的装置,用比福科的摆更短的摆绳进行演示。摆的振荡通过配有垂直刻度的读数望远镜观察,且摆上固定了一面镜子。这在高斯与格尔林的通信中有所描述,韦伯也在1853年用这一装置进行了一些实验,但没有公布数据。[255][256]

高斯在1829年提出的最小约束原理,作为一个普遍的概念,用来克服力学中静力学与动力学的划分,结合了达朗贝尔原理与拉格朗日虚功原理,并展示了与最小二乘法方法的类比。[257]
\subsubsection{计量学} 
1828年,高斯被任命为汉诺威王国度量衡委员会的负责人。他负责制定长度和度量标准。高斯亲自处理了繁琐的测量工作,并对机械准备工作给出了详细的指示。[155] 在与舒马赫的通信中,他描述了高精度刻度的新想法。[258] 他在1841年将关于汉诺威英尺和磅的最终报告提交给政府。这项工作通过1836年颁布的一项法律获得了超越地区性的意义,该法律将汉诺威的度量衡与英国的度量衡联系了起来。[155]
\subsubsection{轶事}
\begin{figure}[ht]
\centering
\includegraphics[width=8cm]{./figures/4de10b5a4494f14a.png}
\caption{1777年5月4日高斯洗礼的教区登记,后附上出生日期} \label{fig_KRGS_24}
\end{figure}
关于高斯早期天才的故事有许多。高斯的母亲从未记录过他的出生日期,只记得他是在一个星期三出生的,距离升天节(复活节后39天)相差八天。高斯后来通过计算复活节的日期,解决了关于自己出生日期的这个谜题,并推导出计算过去和未来年份复活节日期的方法。[259]

在沃尔夫冈·萨托留斯·冯·瓦尔特沙森(Wolfgang Sartorius von Waltershausen)关于高斯的纪念文献中,讲述了一个关于三岁高斯的故事,他纠正了父亲的数学错误。最著名的故事,也是萨托留斯所讲述的,发生在一次学校练习中:教师Büttner和他的助手马丁·巴尔特尔斯(Martin Bartels)要求学生们加总一个算术级数。在大约一百名学生中,高斯是第一个以显著的优势正确解答问题的人。[260][8] 尽管(或因为)萨托留斯没有给出细节,随着时间的推移,关于这个故事有了更多版本,且细节越来越多,关于级数的性质—最常见的是加总从1到100的所有整数—以及课堂上的具体情况。[261][z]

还有一个故事讲述了高斯三岁时纠正了父亲账目中的一个错误。[263][264]
\subsubsection{荣誉与奖项}
\begin{figure}[ht]
\centering
\includegraphics[width=6cm]{./figures/8fb4487bbcc15b08.png}
\caption{高斯的科普利奖(1838年)} \label{fig_KRGS_25}
\end{figure}
高斯于1802年首次获得科学学会会员资格,由俄罗斯科学院授予。[265] 随后他还获得了以下学会的会员资格(通讯会员、外籍会员或正式会员):哥廷根科学院(1802/1807),[266] 法国科学院(1804/1820),[267] 伦敦皇家学会(1804),[268] 柏林普鲁士皇家科学院(1810),[269] 维罗纳国家科学院(1810),[270] 爱丁堡皇家学会(1820),[271] 慕尼黑巴伐利亚科学院(1820),[272] 哥本哈根丹麦皇家学会(1821),[273] 伦敦皇家天文学会(1821),[274] 瑞典皇家科学院(1821),[273] 波士顿美国艺术与科学院(1822),[275] 布拉格波希米亚皇家科学院(1833),[276] 比利时皇家科学院、文学与美术学院(1841/1845),[277] 乌普萨拉皇家科学院(1843),[276] 都柏林爱尔兰皇家学会(1843),[276] 荷兰皇家学会(1845/1851),[278] 马德里西班牙皇家科学院(1850),[279] 俄罗斯地理学会(1851),[280] 维也纳帝国科学院(1848),[280] 美国哲学学会(1853),[281] 剑桥哲学学会,[280] 以及哈勒姆荷兰皇家科学学会。[282][283]

卡赞大学和布拉格大学哲学系在1848年任命他为名誉会员。[282]

高斯于1809年获得法国科学院的拉兰德奖,以表彰他在行星理论及通过仅三次观测确定行星轨道的研究成果,[284] 1823年获得丹麦科学院奖,以表彰他关于共形投影的论文,[276] 并于1838年获得皇家学会的科普利奖章,以表彰“他在磁学方面的发明和数学研究”[283][285][33]。

高斯于1837年被授予法国荣誉军团勋章,[286] 并在1842年普鲁士“Pour le Merite”勋章(文官类)设立时成为首批成员之一。[287] 他还获得了西法利亚王冠勋章(1810),[283] 丹麦丹麦勋章(1817),[283] 汉诺威皇家古尔菲勋章(1815),[283] 瑞典极星勋章(1844),[288] 亨利狮子勋章(1849),[288] 以及巴伐利亚马克西米连科学与艺术勋章(1853)。[280]

汉诺威国王授予他荣誉头衔“宫廷顾问”(1816)[51] 和“秘密宫廷顾问”(1845)。在他获得金博士学位50周年纪念时(1949年),他被授予布伦瑞克和哥廷根两个城市的名誉市民称号。[280] 他去世后不久,汉诺威国王乔治五世下令发行了一枚纪念勋章,背面铭文写道:“献给数学王子”。[289]

“哥廷根高斯学会”于1964年成立,致力于研究卡尔·弗里德里希·高斯及相关人物的生平与工作,并编辑《高斯学会通讯》(Mitteilungen der Gauss-Gesellschaft)。[290]
\subsubsection{名称与纪念活动}
\begin{itemize}
\item 以卡尔·弗里德里希·高斯命名的事物列表
\end{itemize}
\subsection{精选著作}  
\subsubsection{数学与天文学}
\begin{itemize}
\item 1799年: 《Demonstratio nova theorematis omnem functionem algebraicam rationalem integram unius variabilis in factores reales primi vel secundi gradus resolvi posse》 [每个一变量代数有理整数函数可以分解为一阶或二阶实因子的定理的新证明]。赫尔姆施塔特:C. G. Fleckeisen。(关于代数学基本定理的博士论文,赫尔姆施塔特大学)原版书籍  
\item 1816年: 《Demonstratio nova altera theorematis omnem functionem algebraicam rationalem integram unius variabilis in factores reales primi vel secundi gradus resolvi posse》 [每个一变量代数有理整数函数可以分解为一阶或二阶实因子的定理的另一新证明]。《哥廷根皇家科学院会报·数学类》 3: 107–134。原版  
\item 1816年: 《Theorematis de resolubilitate functionum algebraicarum integrarum in factores reales demonstratio tertia》 [关于代数有理整数函数分解为实因子的第三种证明]。《哥廷根皇家科学院会报·数学类》 3: 135–142。原版  
\item 1850年: 《Beiträge zur Theorie der algebraischen Gleichungen》 [代数方程理论贡献]。《哥廷根皇家科学院会报》 4: 34–35。原版(1849年讲座)  
\item 《Die vier Gauss'schen Beweise für die Zerlegung ganzer algebraischer Funktionen in reelle Faktoren ersten und zweiten Grades》(1799–1849) [高斯的四个代数基本定理证明]。由Netto翻译。莱比锡:威廉·恩格尔曼出版社,1890年。(德文)  
\item 1800年: 《Berechnung des Osterfestes》 [复活节日期的计算]。《月刊通讯:地球和天文学的促进》 2: 121–130。原版
\item 1801: Disquisitiones Arithmeticae. Leipzig: Gerh. Fleischer jun.
\item Gauss, Carl Friedrich (1986). Disquisitiones Arithmeticae & other papers on number theory. Translated by Clarke, Arthur A. (2nd, corrected ed.). New York: Springer. doi:10.1007/978-1-4939-7560-0. ISBN 978-0-387-96254-2. (translated from the second German edition, Göttingen 1860)
\item 1802: "Berechnung des jüdischen Osterfestes" [Calculation of Jewish Easter]. Monatliche Correspondenz zur Beförderung der Erd- und Himmelskunde (in German). 5: 435–437. Original
\item 1804: "Über die Grenzen der geocentrischen Oerter der Planeten" [On the limits of the geocentric places of the planets]. Monatliche Correspondenz zur Beförderung der Erd- und Himmelskunde (in German). 10: 171–193. Original (on the Zodiacus)
\item 1808: "Theorematis arithmetici demonstratio nova". Commentationes Societatis Regiae Scientiarum Gottingensis. Comm. Math. 16: 69–74. Original (Introduces Gauss's lemma, uses it in the third proof of quadratic reciprocity)
\item 1808: Methodus peculiaris elevationem poli determinandi (in Latin). Göttingen.
\item 1809: Theoria motus corporum coelestium in sectionibus conicis solem ambientium (in Latin). Hamburg: Friedrich Perthes & Johann Heinrich Besser. Original book
\item Theory of the Motion of Heavenly Bodies Moving about the Sun in Conic Sections. Translated by Davis, Charles Henry. Little, Brown & Co. 1857.
\item 1811: "Disquisitio de elementis ellipticis Palladis ex oppositionibus annorum 1803, 1804, 1805, 1806, 1807, 1808, 1809". Commentationes Societatis Regiae Scientiarum Gottingensis Recentiores. Comm. Math. 1: 1–26. Original (from 1810) (Orbit of Pallas)
\item 1811年: 《Summatio quarundam serierum singularium》 [某些特殊级数的和]。《哥廷根皇家科学院会报·数学类》 1: 1–40。原版(来自1808年)(确定二次高斯和的符号,使用此方法给出了二次互反律的第四个证明)
\item 1813年: 《Disquisitiones generales circa seriem infinitam 1 + ... + 等等》 [关于无穷级数的总研究]。《哥廷根皇家科学院会报·数学类》 2: 1–42。原版(来自1812年,包含高斯连分数)
\item 1816年: 《Methodus nova integralium valores per approximationem inveniendi》 [通过逼近法求积分值的新方法]。《哥廷根皇家科学院会报·数学类》 3: 39–76。原版(来自1814年)
\item 1818年: 《Theorematis fundamentalis in doctrina de residuis quadraticis demonstrationes et ampliationes novae》 [关于二次剩余理论的基本定理的证明及扩展]。《哥廷根皇家科学院会报·数学类》 4: 3–20。原版(来自1817年)(二次互反律的第五和第六个证明)
\item 1818年: 《Determinatio attractionis, quam in punctum positionis datae exerceret planeta, si eius massa per totam orbitam, ratione temporis, quo singulae partes describuntur, uniformiter esset dispertita》 [确定行星在给定位置点上施加的引力,如果其质量沿整个轨道均匀分布]。《哥廷根皇家科学院会报·数学类》 4: 21–48。原版(唯一提及的——大部分未出版——关于算术-几何平均数算法的工作)
\item 1823年: 《Theoria combinationis observationum erroribus minimis obnoxiae. Pars Prior》 [关于组合观测误差最小理论。前篇]。《哥廷根皇家科学院会报·数学类》 5: 33–62。原版(来自1821年)
\item 1823年: 《Theoria combinationis observationum erroribus minimis obnoxiae. Pars Posterior》 [关于组合观测误差最小理论。后篇]。《哥廷根皇家科学院会报·数学类》 5: 63–90。原版
\item 1825年: 《Allgemeine Auflösung der Aufgabe die Theile einer gegebnen Fläche auf einer andern gegebnen Fläche so abzubilden dass die Abbildung dem Abgebildeten in den kleinsten Theilen ähnlich wird》 [关于将给定平面上的部分映射到另一个给定平面,使得映射在最小部分上相似的一般解法]。《天文研究》 3. 阿尔托纳。(1822年关于共形映射的获奖论文)
\item 1828年: 《Bestimmung des Breitenunterschiedes zwischen den Sternwarten von Göttingen und Altona durch Beobachtungen am Ramsdenschen Zenithsector》 [通过使用拉姆斯登天顶仪观测,确定哥廷根和阿尔托纳天文台之间的纬度差异](德文)。哥廷根:Vandenhoeck und Ruprecht。1828年。原版书籍
\item 1828年: 高斯,卡尔·弗里德里希(1828年)。《Supplementum theoriae combinationis observationum erroribus minimis obnoxiae》 [关于组合观测误差最小理论的补充]。《哥廷根皇家科学院会报·数学类》 6: 57–98。Bibcode:1828stco.book.....G。(来自1826年)
\item 高斯,卡尔·弗里德里希;斯图尔特,G. W.(1995年)。《Theory of the Combination of Observations Least Subject to Errors. Part One, Part Two, Supplement》 [关于组合观测误差最小理论,第一部分、第二部分、补充篇](应用数学经典系列)。由G. W. Stewart翻译。费城:工业与应用数学学会。doi:10.1137/1.9781611971248。ISBN 978-0-89871-347-3。(三篇关于概率计算的论文,作为高斯误差传播法则的基础)
\item 1828年: 《Disquisitiones generales circa superficies curvas》 [关于曲面的一般研究]。《哥廷根皇家科学院会报·数学类》 6: 99–146。原版(来自1827年)
\item 《General Investigations of Curved Surfaces》 [关于曲面的一般研究](PDF)。由J. C. Morehead和A. M. Hiltebeitel翻译。普林斯顿大学图书馆。1902年。
\item 1828年: 《Theoria residuorum biquadraticorum, Commentatio prima》 [四次剩余理论,第一篇]。《哥廷根皇家科学院会报·数学类》 6: 27–56。原版(来自1825年)
\item 1832年: 《Theoria residuorum biquadraticorum, Commentatio secunda》 [四次剩余理论,第二篇]。《哥廷根皇家科学院会报·数学类》 7: 89–148。原版(来自1831年)(引入高斯整数,陈述(无证明)四次互反律,证明1 + i的补充法则)
\item 1845年: 《Untersuchungen über Gegenstände der Höheren Geodäsie. Erste Abhandlung》 [关于高等大地测量学的研究。第一篇]。《哥廷根皇家科学院会报》第二卷,1842–1844年:3–46。原版(来自1843年)
\item 1847年: 《Untersuchungen über Gegenstände der Höheren Geodäsie. Zweite Abhandlung》 [关于高等大地测量学的研究。第二篇]。《哥廷根皇家科学院会报》第三卷,1845–1847年:3–44。原版(来自1846年)
\item 1848年: 高斯(1848年)。《Schreiben des Herrn Geheimen Hofrathes Gauss an den Herausgeber》 [秘密宫廷顾问高斯先生致编辑的信]。《天文消息》 (德文)27: 1–3。Bibcode:1848AN.....27....1G。doi:10.1002/asna.18480270102。原版
\item 1903年: 《Wissenschaftliches Tagebuch》 [科学日记](由费利克斯·克莱因编,1903年)。《数学年刊》(拉丁文和德文)。57: 1–34。doi:10.1007/BF01449013。S2CID 119641638。(来自1847年,关于黄道带)
\item 杰里米·格雷(1984年)。《A commentary on Gauss's mathematical diary, 1796–1814》 [关于高斯数学日记1796–1814的评论]。《数学展示》 2: 97–130。
\end{itemize}
\subsubsection{物理学}
\begin{itemize}
\item 1804年: 《Fundamentalgleichungen für die Bewegung schwerer Körper auf der Erde》 [地球上重物运动的基本方程](原书:本岑贝格,约翰·弗里德里希。《关于自由落体定律、空气阻力和地球自转的实验》)。多特蒙德:Gebrüder Mallinckrodt。第363–371页。原版。
\item 1813年: 《Theoria attractionis corporum sphaeroidicorum ellipticorum homogeneorum methodo nova tractata》 [均质椭球体的引力理论以新方法处理]。《哥廷根皇家科学院会报·数学类》 2: 1–24。原版(包含高斯的向量分析定理)。
\item 1817年: 《Ueber die achromatischen Doppelobjective besonders in Rücksicht der vollkommnern Aufhebung der Farbenzerstreuung》 [关于消色差双透镜,特别是考虑到色散的更完全消除]。《天文学与相关科学杂志》 IV: 345–351。(德文)
\item 1829年: 《Über ein neues allgemeines Grundgesetz der Mechanik》 [关于一种新的普遍力学基本定律]。《纯粹与应用数学期刊》 1829年(4):232–235。doi:10.1515/crll.1829.4.232。S2CID 199545985。
\item 1830年: 《Principia generalia theoriae figurae fluidorum in statu aequilibrii》 [流体平衡态下的流体形态理论的一般原理]。《哥廷根皇家科学院会报·数学类》 7: 39–88。原版(来自1829年)。
\item 1841年: 《Intensitas vis magneticae terrestris ad mensuram absolutam revocata》 [地球磁力强度的绝对测量]。《哥廷根皇家科学院会报·数学类》 8: 3–44。原版(来自1832年)。
\item 《The Intensity of the Earth's Magnetic Force Reduced to Absolute Measurement》 [地球磁力强度的绝对测量]。由苏珊·P·约翰逊翻译。
\item 1836年: 《Erdmagnetismus und Magnetometer》 [地磁与磁力计](原书:H.C.舒马赫(编)。《1836年年鉴》(德文)。第1836卷。蒂宾根:J.G.Cotta出版公司。第1–47页。)
\item 1840年: 《Allgemeine Lehrsätze in Beziehung auf die im verkehrten Verhältnis des Quadrats der Entfernung wirkenden Anziehungs- und Abstoßungskräfte》 [关于以平方距离的倒数比例作用的引力与排斥力的普遍定理](原书:《关于以平方距离的倒数比例作用的引力与排斥力的普遍定理》(德文)。莱比锡:Weidmannsche书店。1840年)。
\item 1843年: 《Dioptrische Untersuchungen》 [光学研究]。《哥廷根皇家科学院会报》(德文)。第一卷:1–34。原版(来自1840年)。
\end{itemize}

\textbf{与威廉·韦伯(Wilhelm Weber)共同著作}

\begin{itemize}
\item 1837–1839年: 《Resultate aus den Beobachtungen des magnetischen Vereins im Jahre 1836–1838》 [来自1836至1838年磁学协会观测结果](德文)。哥廷根:Dieterichsche书店。
\item 1840–1843年: 《Resultate aus den Beobachtungen des magnetischen Vereins im Jahre 1839–1841》 [来自1839至1841年磁学协会观测结果](德文)。莱比锡:Weidmannsche出版书店。
\item 1840年: 《Atlas des Erdmagnetismus nach den Elementen der Theorie entworfen》 [根据理论要素绘制的地磁图集]。作为《来自磁学协会观测结果》的补充(德文)。莱比锡:Weidmannsche出版书店。
\end{itemize}
\subsubsection{《全集》}
\begin{itemize}
\item 普鲁士王国科学院(Königlich Preußische Akademie der Wissenschaften)编辑(1863–1933)。《卡尔·弗里德里希·高斯作品集》(Carl Friedrich Gauss. Werke)(拉丁语和德语)。第1–12卷。哥廷根:(多家出版社)。(包括未出版的文献遗产)
\end{itemize}
\subsubsection{《信件》}
\begin{itemize}
\item 普鲁士王国科学院编辑(1880年)。《高斯与贝塞尔之间的信件》(Briefwechsel zwischen Gauss und Bessel)(德语)。莱比锡:威廉·恩格尔曼出版社。(1804年12月到1844年8月的信件)
\item 埃里希·舍恩贝格,阿尔丰斯·佩里克(1955年)。《卡尔·弗里德里希·高斯与弗朗茨·威廉·贝塞尔的未公开信件》。巴伐利亚科学院数学与自然科学班,新的系列,第71号(德语)。慕尼黑:巴伐利亚科学院出版社,5–21页。(1835年2月到1848年1月写给博古斯拉夫斯基的信件)
\item 弗里德海姆·施韦敏,编(2014年)。《卡尔·弗里德里希·高斯与约翰·埃勒特·博德的信件往来》。天文历史学报(Acta Historica Astronomica)(德语)。第53卷。莱比锡:学术出版社,ISBN 978-3-944913-43-8。(1802年2月到1826年10月的信件)
\item 弗朗茨·施密特,保罗·斯塔克尔,编(1899年)。《卡尔·弗里德里希·高斯与沃尔夫冈·博尔亚伊的信件往来》(Briefwechsel zwischen Carl Friedrich Gauss und Wolfgang Bolyai)(德语)。莱比锡:B.G.特布纳出版社。(1797年9月到1853年2月的信件;附有其他通信者的信件)
\item 阿克塞尔·维特曼,编(2018年)。《尽管如此,卡尔·弗里德里希·高斯与约翰·弗朗茨·恩克的信件往来》(Obgleich und indeßen. Der Briefwechsel zwischen Carl Friedrich Gauss und Johann Franz Encke)(德语)。雷马根:凯塞尔出版社,ISBN 978-3945941379。(1810年6月到1854年6月的信件)
\item 克莱门斯·谢费尔,编(1927年)。《卡尔·弗里德里希·高斯与克里斯蒂安·路德维希·格尔林的信件往来》(Briefwechsel zwischen Carl Friedrich Gauss und Christian Ludwig Gerling)(德语)。柏林:奥托·埃尔斯纳出版社。(1810年6月到1854年6月的信件)
\item 卡尔·克里斯蒂安·布鲁恩斯,编(1877年)。《亚历山大·冯·洪堡与高斯的信件往来》(Briefe zwischen A. v. Humboldt und Gauss)(德语)。莱比锡:威廉·恩格尔曼出版社。(1807年7月到1854年12月的信件;附有其他通信者的信件)
\item 卡琳·赖希,埃琳娜·鲁萨诺娃(2018年)。《卡尔·克雷尔与地磁学:他与卡尔·弗里德里希·高斯的信件及其历史背景》。自然科学、数学与医学历史委员会出版物,第68号(德语)。维也纳:奥地利科学院出版社。(1835年到1843年间的信件)
\item 西奥·杰拉尔迪,编(1959年)。《卡尔·弗里德里希·高斯与卡尔·路德维希·冯·莱科克的信件往来》。哥廷根科学院学报,数学与物理学班,第4号(德语)。哥廷根:范登霍克与鲁普雷希特出版社,37–63页。(1799年2月到1800年9月的信件)
\item 埃里克·G·福布斯(1971年)。《卡尔·弗里德里希·高斯与内维尔·马斯凯林牧师(1802–1805)的信件往来》。科学年鉴,第27卷(第3期):213–237,DOI:10.1080/00033797100203767。
\item 克利福德·坎宁安(2004年)。《卡尔·弗里德里希·高斯与内维尔·马斯凯林牧师(1802–1805)的失落信件的发现》。科学年鉴,第61卷(第4期):469–481,DOI:10.1080/00033790310001660164。
\item 卡尔·施林,编(1900年)。《奥尔伯斯与高斯的信件往来:第一部分。威廉·奥尔伯斯——他的生平与作品》(Briefwechsel zwischen Olbers und Gauss: Erste Abtheilung. Wilhelm Olbers. Sein Leben und seine Werke)。柏林:尤利乌斯·施普林格出版社。(1802年1月到1819年10月的信件)
\item 卡尔·施林,编(1909年)。《奥尔伯斯与高斯的信件往来:第二部分。威廉·奥尔伯斯——他的生平与作品》(Briefwechsel zwischen Olbers und Gauss: Zweite Abtheilung. Wilhelm Olbers. Sein Leben und seine Werke)。柏林:尤利乌斯·施普林格出版社。(1820年1月到1839年5月的信件;附有其他通信者的信件)
\item 克里斯蒂安·奥古斯特·弗里德里希·彼得斯,编(1860–1865年)。《高斯与H.C.舒马赫的信件往来》(Briefwechsel zwischen C. F. Gauss und H. C. Schumacher)(德语)。阿尔托纳:古斯塔夫·埃施出版社。
\item 第1+2卷(1808年4月到1836年3月的信件)
\item 第3+4卷(1836年3月到1845年4月的信件)
\item 第5+6卷(1845年4月到1850年11月的信件)
\item 汉斯·波瑟,编(1987年)。《卡尔·弗里德里希·高斯与埃伯哈德·奥古斯特·齐默曼的信件往来》。哥廷根科学院学报,数学与物理学班,系列3,第39号(德语)。哥廷根:范登霍克与鲁普雷希特出版社,ISBN 978-3525821169。(1795年到1815年间的信件)
\end{itemize}
哥廷根科学院提供了卡尔·弗里德里希·高斯所有已知信件的完整收藏,并且可以在线访问。[34] 文学遗产由哥廷根州立大学图书馆保存并提供。[291] 来自卡尔·弗里德里希·高斯及其家族成员的书面材料也可以在布伦瑞克市档案馆找到。[292]
\subsection{参考文献}  
\subsubsection{注释}\\  
a. 卡尔·弗里德里希·高斯的时代,卡罗林大学是布伦瑞克技术大学的前身,但当时它并不等同于大学。[7] \\
b. 曾有一次高斯画了一幅讲座场景,教授凯斯特纳在简单的计算中出现了错误。[10]\\  
c. 这个错误在例如Marsden (1977)中有出现。[17]\\  
d. 高斯宣布了195场讲座,其中70\%是关于天文学的,15\%是关于数学的,9\%是关于测量学的,6\%是关于物理学的。[33]\\  
e. 信件索引显示,贝杰明·古尔德可能是最后一位与高斯通信的人,他在1855年2月13日给高斯写了一封信。这是一封告别信,但不确定信件是否及时送达收件人。[34]\\  
f. 高斯去世后,在他的文稿中发现了一篇关于帕拉斯扰动的法语文章,可能是为法国科学院的一个奖学金竞赛写的贡献。[36]\\  
g. 《天体运动理论》于1806年完成了德文版本,但应编辑弗里德里希·克里斯托夫·佩尔特斯的要求,高斯将其翻译成了拉丁文。[37]\\  
h. 高斯和哈丁在他们的通信中仅仅含蓄地提到了这个个人问题。一封写给舒马赫的信表明,高斯曾试图摆脱他的同事,并为他寻找一个哥廷根以外的新职位,但没有结果。除此之外,高斯的岳母夏洛特·瓦尔德克曾恳求奥尔伯斯试图为高斯提供一个远离哥廷根的职位。[42]\\  
i. 高斯的第一位助手是本杰明·戈尔德施密特,第二位是威廉·克林克弗斯,他后来成为了高斯的继任者之一。[33]\\  
j. 贝塞尔从未接受过大学教育。[46][47]\\  
k. 在这次旅行中,他遇到了地理学家费尔迪南德·鲁道夫·哈斯勒,哈斯勒是卡尔·弗里德里希·高斯的科学通信员。[65][66]\\  
l. 根据博尔雅伊的手写匈牙利文文本的下方,高斯故意通过添加错误的加法,刻画了凯斯特纳的形象。\\  
m. 他在1795年借的第一本大学图书馆的书是塞缪尔·理查森的小说《克拉丽莎》。[101]\\  
n. 政治背景是德意志邦联的复杂局面,邦联由39个几乎独立的国家组成,其中三个国家的君主分别是其他国家的国王(荷兰、丹麦、联合王国),而普鲁士王国和奥斯曼帝国的疆域则大范围跨越邦联的边界。\\  
o. 高斯后来在给恩克的信中详细讲述了这一故事。[124]\\  
p. 后来,这些变换在1824年由勒让德给出(3阶),1829年由雅可比给出(5阶),1837年由松克给出(7阶及其他阶)。\\  
q. 在1828年写给贝塞尔的一封信中,高斯评论道:“阿贝尔先生[……]提前做了我应该做的,并且使我不再需要在这些问题上费力……”[138]\\  
r. 这句话出现在《算术研究》1801年版第7章第335条中。\\  
s. 要准确识别固定星座中的一个天体为行星,至少需要进行两次带时间间隔的观察。\\  
t. 布伦德尔(1929年)认为这个密码无法解密,但实际上解码非常简单。[162][164] \\ 
u. 劳恩堡是荷尔斯泰因公国的最南端城市,那个地方由丹麦国王以个人联邦的方式统治。\\  
v. 这个拉姆斯登扇形仪是由军械局借出的,早先由威廉·马奇在英国大三角测量中使用过。[183]\\  
w. 瓦尔贝克(1820年)提供的1/302.78值经过改进为1/298.39;这一计算由哥廷根大学的私人讲师爱德华·施密特完成。[190]\\  
x. 赫塞曼还为高斯做了一个死亡面具。[39]\\ 
y. 1845年,一场雷暴损坏了电缆。[240]\\  
z. 一些作者,如约瑟夫·J·罗特曼,质疑这是否曾经发生过。[262]\\  
aa. 字面翻译:宫廷秘密顾问\\  
ab. 高斯于1832年12月将文本呈交给哥廷根科学院,并于1833年发布了少数副本的拉丁文预印本。随后该文本很快被翻译并用德语和法语出版。拉丁文完整文本于1841年出版。[230]\\
\subsubsection{引用文献}  
\begin{enumerate}
\item Axel D. Wittmann, Inna V. Oreshina (2009). "关于詹森的卡尔·弗里德里希·高斯画作". 《高斯学会通讯》 (46): 57–61.  
\item Duden – 《发音词典》 (第7版). 柏林:Dudenverlag. 2015 [1962]. 第246, 381, 391页. ISBN 978-3-411-04067-4.  
\item Krech, Eva-Maria; Stock, Eberhard; Hirschfeld, Ursula; Anders, Lutz-Christian (2009). 《德国语言发音词典》. 柏林:Walter de Gruyter. 第402, 520, 529页. ISBN 978-3-11-018202-6.  
\item Borch, Rudolf (1929). 《数学家卡尔·弗里德里希·高斯的家谱》. 《著名德国人的家谱》 (第1卷). 德国人物与家族历史中心. 第63–65页.  
\item Dunnington 2004, 第8页.  
\item Dunnington 2004, 第8–9页.  
\item Dunnington 2004, 第17页.  
\item Schlesinger 1933, 第10页.  
\item Dunnington 2004, 第14页.  
\item Ullrich, Peter (2005). "卡尔·弗里德里希·高斯的来源、学校和学习经历". In Mittler, Elmar (编辑). 《如何解开闪电之谜——卡尔·弗里德里希·高斯在哥廷根》 (PDF). 哥廷根图书馆文献 (第30卷). 下萨克森州立与大学图书馆. 第17–29页. ISBN 3-930457-72-5.  
\item Dunnington, Waldo (1927). "高斯诞辰150周年纪念". 《科学月刊》. 24 (5): 402–414. Bibcode:1927SciMo..24..402D. JSTOR 7912. 2008年2月26日存档. 也可以在线查看 "高斯诞辰150周年纪念". 2014年2月23日获取. 综合传记文章.  
\item Dunnington 2004, 第24页.  
\item Dunnington 2004, 第26页.  
\item Wattenberg, Diedrich (1994). 《威廉·奥尔伯斯与他同时代天文学家的信件》. 斯图加特:GNT – 自然科学与技术历史出版社. 第41页. ISBN 3-928186-19-1.  
\item Dunnington 2004, 第28页.  
\item Dunnington 2004, 第37页.  
\item Marsden, Brian G. (1977年8月1日). "卡尔·弗里德里希·高斯,天文学家". 《加拿大皇家天文学会杂志》. 71: 309–323. Bibcode:1977JRASC..71..309M. ISSN 0035-872X.  
\item Reich, Karin (2000). "高斯的学生们". 《高斯学会哥廷根通讯》 (37): 33–62.  
\item Beuermann, Klaus (2005). "卡尔·弗里德里希·高斯与哥廷根天文台" (PDF). In Beuermann, Klaus (编辑). 《关于新天文台建设的原则,参照哥廷根大学天文台的情况,作者:乔治·海因里希·博赫》. 哥廷根:哥廷根大学出版社. 第37–45页. ISBN 3-938616-02-4.  
\item Michling, Horst (1966). "关于在布伦瑞克建立高斯天文台的计划". 《高斯学会哥廷根通讯》 (3): 24.  
\item Dunnington 2004, 第50, 54–55, 74–77页.  
\item Dunnington 2004, 第91–92页.  
\item Dunnington 2004, 第85–87页.  
\item Dunnington 2004, 第86–87页.  
\item Brendel 1929, 第81–82页.  
\item Brendel 1929, 第49页.  
\item Brendel 1929, 第83页.  
\item Brendel 1929, 第84页.  
\item Brendel 1929, 第119页.  
\item Brendel 1929, 第56页.  
31. Klein 1979, 第7页.  
32. Dunnington 2004, 第405–410页.  
33. Wittmann, Axel (2005). "卡尔·弗里德里希·高斯与他作为天文学家的工作". In Mittler, Elmar (编辑). 《如何解开闪电之谜——卡尔·弗里德里希·高斯在哥廷根》 (PDF). 哥廷根图书馆文献 (第30卷). 下萨克森州立与大学图书馆. 第131–149页. ISBN 3-930457-72-5.  
34. "卡尔·弗里德里希·高斯的完整通信". 哥廷根科学院. 2023年3月10日获取.  
35. Biermann, Kurt-R. (1966). "关于卡尔·弗里德里希·高斯与弗里德里希·威廉·贝塞尔的关系". 《高斯学会哥廷根通讯》 (3): 7–20.  
36. Brendel 1929, 第211页.  
37. Dunnington 2004, 第90页.  
38. Dunnington 2004, 第37–38页.  
39. Dunnington 2004, 第324页.  
40. Cantor, Moritz (1894). Thibaut, Bernhard Friedrich. 《德国百科全书》 (第37卷). 莱比锡:Duncker & Humblot. 第745–746页.  
41. Folkerts, Menso (1990). Mayer, Johann Tobias. 《新德国传记》 (第16卷). Duncker & Humblot. 第530页.  
42. Küssner, Martha (1978). "弗里德里希·威廉·贝塞尔与哥廷根的关系及对他的记忆". 《高斯学会哥廷根通讯》 (15): 3–19.  
43. Brendel 1929, 第106–108页.  
44. Brendel 1929, 第7, 128页.  
45. Gerardy, Theo (1966). "卡尔·弗里德里希·高斯与他的儿子们". 《高斯学会哥廷根通讯》 (3): 25–35.  
46. Hamel, Jürgen (1984). 《弗里德里希·威廉·贝塞尔》。莱比锡:BSB B.G.Teubner出版公司. 第29页.  
47. Dunnington 2004, 第76页.  
48. Mackinnon, Nick (1990). "索菲·热尔曼,或者说,高斯是女权主义者吗?". 《数学公报》. 74 (470). 数学协会: 346–351. doi:10.2307/3618130. JSTOR 3618130. S2CID 126102577.  
49. Biermann, Kurt-R. (1964). "哥特霍尔德·艾森斯坦". 《纯粹与应用数学杂志》. 214: 19–30. doi:10.1515/crll.1964.214-215.19.  
50. Dunnington 2004, 第195–200页.
\end{enumerate}