% 电磁场的作用量
% keys 电磁场|作用量|拉氏量

\pentry{拉格朗日电磁势\upref{EMLagP},电磁场张量\upref{EMFT},电磁场中粒子的拉氏量\upref{ElecLS},经典场论基础\upref{classi}}

我们继续使用自然单位制,令 $\mu_0=\epsilon_0=c=1$ 来简化表达.依照习惯,上下标使用希腊字母如 $\mu, \nu$ 时,取值范围为 $\{0, 1, 2, 3\}$;使用拉丁字母如 $i, j$ 时,取值范围为 $\{1, 2, 3\}$.

\subsection{自由电磁场的作用量}
我们先考虑没有电流电荷密度分布的时空,仅考虑自由电磁场的作用量.我们要求电磁场的作用量满足洛伦兹不变性,是个洛伦兹标量,它可以由\textbf{电磁场张量}\upref{EMFT}来构造:
\begin{equation}
S=\int {\mathcal L} {\dd}^4 x=-\frac{1}{16\pi}\int F^{\mu\nu}F_{\mu\nu}{\dd}^4 x 
\end{equation}
其中 $\mathcal L=-\frac{1}{16\pi}F^{\mu\nu}F_{\mu\nu}$ 为自由电磁场的拉氏密度.
根据最小作用量原理,$\delta S=0$,可以得到
\begin{equation}
\delta S=-\frac{1}{16\pi}
\end{equation}