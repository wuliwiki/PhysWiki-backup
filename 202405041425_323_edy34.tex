% 超导唯象解释——伦敦方程
% keys 麦克斯韦|库仑规范|伦敦方程|伦敦规范
% license Usr
% type Tutor

\pentry{麦克斯韦方程\nref{nod_MWEq}, 库仑规范\nref{nod_Cgauge}}{nod_7c3f}

本文为唯象论,不涉及超导本质。考虑超导本质需用量子的角度去解释即 BCS 理论。
\subsection{伦敦第一方程}
假定超导体中有两种载流电子——正常传到电子与超导电子,对于超导电子有
\begin{equation}
\pdv{t}\bvec J_s=\alpha\bvec E \label{eq_edy34_1}~.
\end{equation}
其中 $\bvec J_s$ 代表超导体中的超导电流密度,$\alpha=n\dfrac {n_se^2}m$,该方程理解,在超导电子运动速度远小于光速 $c$ 的情况下,磁力对超导电子的影响忽略,此时由 $\bvec F=m \bvec a$ 写在电场中的形式 $m\pdv{v}{t}=-e\bvec E$ 推出。
实验证明与超导体的相关性质吻合。为第二方程引出给出先决条件 $E=0$,否则电子速度会不断上升。
\subsection{伦敦第二方程}
\begin{equation}
\curl\bvec J_s=-\alpha\bvec B \label{eq_edy34_2}~,
\end{equation}
由\autoref{eq_edy34_1} 取旋度加之 $\curl E=\pdv{B}{t}$ 推得。
\subsection{伦敦规范}
在库仑规范的前提下,矢势 $\bvec A$ 并不唯一,为了使矢势 $\bvec A$ 唯一确定,而在超导体表面 $S$ 上引入限定 $\bvec A$ 的法向分量为 $0$,即
\begin{equation}
\div \bvec A=0~,
\end{equation}
\begin{equation}
\bvec e_n \cdot \bvec A|_s=0~.
\end{equation}
\subsection{伦敦方程解释超导现象}\label{sub_edy34_1}
\begin{theorem}{迈斯纳效应}
当材料处于超导态时,随着进入导体内部深度的增加,磁场迅速衰减,磁场主要存在于导体表面一定厚度的薄层内。
\end{theorem}
恒定情形时(正常电子所导致的电流为零,超导电流与深度有关。)对于超导体内的磁场和电流满足的麦克斯韦—伦敦方程为
\begin{equation}
\div \bvec B=0\label{eq_edy34_3}~,
\end{equation}
\begin{equation}
\curl \bvec B=\mu_0 \bvec J_s\label{eq_edy34_4}~,
\end{equation}
\begin{equation}
\div \bvec J_s=0\label{eq_edy34_5}~,
\end{equation}
\begin{equation}
\curl \bvec J_s=-\alpha\bvec B~.
\end{equation}
将\autoref{eq_edy34_4} 取旋度,按照矢量乘积展开,同时代入\autoref{eq_edy34_3} 与\autoref{eq_edy34_2} 得
\begin{equation}
\laplacian \bvec B=\dfrac{1}{\lambda_L^2}\bvec B \label{eq_edy34_7}~.
\end{equation}
其中 $\lambda_L= 1/\sqrt{\mu_0\alpha}=\sqrt{\dfrac m {\mu_0n_se^2}}$,由此可以看出磁场 $\bvec B$ 随着位置变化迅速变化。方程解亦可能形式为磁场随着位置变化迅速上升或下降,但当考虑现实情况迈斯纳效应时,可以确定真解。

求解超导电流,通过对\autoref{eq_edy34_2} 取旋度,及\autoref{eq_edy34_4} 和\autoref{eq_edy34_3} 的代入得
\begin{equation}
\laplacian\bvec J_s=\dfrac 1 {\lambda_L^2}\bvec J_s~.
\end{equation}
同\autoref{eq_edy34_7} 相同形式,故而可以得到相同的结果,即超导电流分布在超导体表面。
\subs

