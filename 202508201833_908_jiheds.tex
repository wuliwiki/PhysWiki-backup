% 几何代数(综合)
% license CCBYSA3
% type Wiki

本文根据 CC-BY-SA 协议转载翻译自维基百科\href{https://en.wikipedia.org/wiki/Geometric_algebra}{相关文章}。

在数学中,几何代数(geometric algebra,亦称为Clifford 代数)是一种能够表示和操作几何对象(如向量)的代数。几何代数建立在两个基本运算之上:加法与几何乘积。向量的乘法会生成更高维的对象,称为多向量。与其他用于处理几何对象的形式体系相比,几何代数的一个显著特征是它支持向量的除法(尽管通常不能对所有元素都定义),并且允许不同维度对象之间的加法。

几何乘积最早由赫尔曼·格拉斯曼简要提出\(^\text{[1]}\),他当时主要致力于发展与之密切相关的外代数。1878 年,威廉·金顿·克利福德在格拉斯曼工作的基础上大幅扩展,形成了如今通常被称为Clifford 代数的体系(尽管克利福德本人选择称其为“几何代数”)。克利福德将 Clifford 代数及其乘法定义为格拉斯曼代数与哈密顿四元数代数的统一体。进一步引入格拉斯曼外积的对偶,可以得到Grassmann–Cayley 代数。在 1990 年代末期,基于平面的几何代数以及共形几何代数(CGA)分别为欧几里得几何和经典几何提供了框架\(^\text{[2]}\)。在实际应用中,这些运算及其若干衍生操作,使得代数中的元素、子空间与运算可以自然地对应到几何解释。然而,数十年来几何代数一度被忽视,其发展被当时新兴的向量分析(主要用于描述电磁学)所掩盖。直到 1960 年代,大卫·赫斯特尼斯再次推广“几何代数”这一术语,并强调它在相对论物理学中的重要性\(^\text{[3]}\)。

标量和向量在几何代数中具有其通常的解释,并且构成了代数中的不同子空间。双向量提供了一种比三维向量分析中的伪向量更自然的表示方式。伪向量通常通过叉乘得到,用来描述有向面积、有向旋转角、力矩、角动量以及磁场等物理量。三向量可以表示有向体积,依此类推。几何代数中的一种元素称为刀片,它可用于表示子空间以及到该子空间的正交投影。旋转与反射也能作为代数的元素来表示。与传统的向量代数不同,几何代数能够自然而然地适应任意维度以及任意二次型,例如相对论中的情况。

在物理学中,几何代数的应用实例包括时空代数(以及较少使用的物理空间代数)。几何分析是几何代数的扩展,它结合了微分与积分,可以用来建立其他理论,例如复分析与微分几何——其方法是使用 Clifford 代数来取代传统的微分形式。

几何代数的推广者中,最著名的是大卫·赫斯特尼斯\(^\text{[4]}\)和克里斯·多兰\(^\text{[5]}\),他们认为几何代数应当成为物理学的首选数学框架。支持者声称,它在许多领域中(包括经典力学、量子力学、电磁理论以及相对论)都能提供简洁而直观的描述\(^\text{[6]}\)。此外,几何代数还在计算机图形学\(^\text{[7]}\)与机器人学中找到了计算工具的用途。
\subsection{定义与记号}
几何代数有多种不同的定义方式。赫斯特尼斯最初采用的是一种公理化方法,\(^\text{[8]}\) 这种方法“充满几何意义”,并且与普遍的 Clifford 代数等价。\(^\text{[9]}\)设 $V$ 是定义在一个域 $F$ 上的有限维向量空间,并带有一个对称双线性型(即内积,\(^\text{[b]}\)例如欧几里得或洛伦兹度量)$g : V \times V \to F $,则二次空间 $(V, g)$ 的几何代数就是 Clifford 代数$\operatorname{Cl}(V, g)$,其元素称为多向量。Clifford 代数通常被定义为张量代数的商代数。然而这种定义过于抽象,因此下列给出一种不依赖抽象代数的定义。

\textbf{定义}

一个带幺元的结合代数 $\operatorname{Cl}(V, g)$,若满足以下条件,则称为二次空间 $(V, g)$ 的 Clifford 代数\(^\text{[10]}\):
\begin{itemize}
\item 它包含 $F$ 与 $V$ 作为互不相交的子空间;
\item 对于任意 $a \in V$,有$a^{2} = g(a,a)\,1 $;
\item $V$ 生成了整个 $\operatorname{Cl}(V, g)$ 作为代数;
\item $\operatorname{Cl}(V, g)$ 不能由 $V$ 的任何真子空间生成。
\end{itemize}
为了涵盖退化对称双线性型的情况,最后一个条件必须加以修正。\(^\text{[c]}\)可以证明,这些条件唯一地刻画了几何积。

在本文的其余部分中,仅考虑实数情形,即$F = \mathbb{R}$.记号$\mathcal{G}(p,q)$(分别为$ \mathcal{G}(p,q,r)$)将用来表示其双线性型 $g$ 的符号(signature)为 $(p,q)$(分别为 $(p,q,r)$)的几何代数。

代数中的乘法称为几何积,而其所包含的外代数中的乘法称为外积(exterior product,常被称为楔积或外积\(^\text{[d]}\))。按照惯例,这两者的记号分别是:几何积:并列书写(即省略任何显式的乘号);外积:使用符号 $\wedge$。

上面给出的几何代数的定义仍然有些抽象,因此我们在这里总结一下几何积的性质。对于任意多向量$A, B, C \in \mathcal{G}(p,q)$,有:
\begin{itemize}
\item 封闭性:$AB \in \mathcal{G}(p,q)$
\item 单位元的存在:$1A = A1 = A,$其中 $1$ 是单位元。
\item 结合律:$A(BC) = (AB)C$
\item 分配律:$A(B+C) = AB + AC, \quad (B+C)A = BA + CA$
\item 平方性质:$a^2 = g(a,a)\,1, \quad a \in V$
\end{itemize}
至于外积,它也具有相同的性质,但最后一条被替换为:$a \wedge a = 0, \quad a \in V$

请注意,在上述最后一个性质中,如果 $g$ 不是正定的,那么实数 $g(a,a)$ 不必是非负的。几何积的一个重要性质是存在可逆元素。对于一个向量$a$,如果$a^{2} \neq 0$,那么其逆元 $a^{-1}$ 存在,并且$a^{-1} = g(a,a)^{-1} a$。然而,代数中的一个非零元素并不一定都有乘法逆元。例如,如果 $u \in V$ 且$u^{2} = 1$,那么元素$\frac{1}{2}(1+u)$既是一个非平凡的幂等元,又是一个非零的零因子,因此没有逆元\(^\text{[e]}\))。

通常,我们会把 $\mathbb{R}$ 与 $V$ 分别与它们在自然嵌入$\mathbb{R} \to \mathcal{G}(p,q), \quad V \to \mathcal{G}(p,q)$下的像识别起来。在本文中,这种识别被默认采用。自始至终,“标量”与“向量”这两个术语,分别指的是 $\mathbb{R}$ 与 $V$ 的元素(以及它们在该嵌入下的像)。
\subsubsection{几何积}
\begin{figure}[ht]
\centering
\includegraphics[width=6cm]{./figures/1c7641680b7d820a.png}
\caption{给定两个向量 $a$ 和 $b$,若它们的几何积 $ab$ 是反对易的,则说明它们是垂直的(上图),因为$a \cdot b = 0$;若它们的几何积是对易的,则说明它们是平行的(下图),因为$a \wedge b = 0$。} \label{fig_jiheds_1}
\end{figure}
对于向量$a$和$b$,任意两个向量的几何积可以写成对称部分与反对称部分的和:
$$
ab = \tfrac{1}{2}(ab + ba) + \tfrac{1}{2}(ab - ba).~
$$
因此,我们可以将向量的内积定义为:
$$
a \cdot b := g(a, b),~
$$
这样,对称部分就可以写为:
$$
\tfrac{1}{2}(ab + ba) 
= \tfrac{1}{2}\left((a+b)^{2} - a^{2} - b^{2}\right) 
= a \cdot b.~
$$
反过来说,双线性型 $g$ 完全由代数结构决定。反对称部分则是两个向量的外积,即包含在其中的外代数的乘积:
$$
a \wedge b := \tfrac{1}{2}(ab - ba) = -(b \wedge a).~
$$
于是,通过简单相加:$ab = a \cdot b + a \wedge b$这就是几何积的非广义(即向量形式)的表达式。
\begin{figure}[ht]
\centering
\includegraphics[width=10cm]{./figures/49ce6c2075f8ffb1.png}
\caption{实外代数中次数为$n$的元素的几何解释:当 $n=0$ 时,表示一个带符号的点;当$n=1$时,表示一个有向线段(即向量);当$n=2$时,表示一个有向平面元;当$n=3$时,表示一个有向体积。$n$个向量的外积可以形象化为任意一个$n$ 维形状(例如 $n$-平行多胞体、$n$-椭球体);其大小对应于超体积,而其取向由其 $(n-1)$ 维边界的方向,以及内部位于哪一侧所决定\(^\text{[14][15]}\)。} \label{fig_jiheds_2}
\end{figure}
内积与外积与标准向量代数中的一些熟悉概念相关。几何上,当两个向量$a$ 和 $b$ 的几何积等于它们的内积时,它们是平行的;而当几何积等于它们的外积时,它们是垂直的。在一个几何代数中,如果任意非零向量的平方均为正数,那么两个向量的内积就可以看作是标准向量代数中的点积;而两个向量的外积则可以理解为由它们构成的平行四边形的带符号面积。在三维空间中,若二次型是正定的,则两个向量的叉积与它们的外积密切相关。

大多数有研究价值的几何代数都具有非退化二次型。如果二次型完全退化,那么任意两个向量的内积恒为零,此时几何代数就会简化为一个外代数。除非特别说明,本文只讨论非退化几何代数。

外积可以自然地扩展为代数中任意两个元素之间的结合的双线性二元运算,并满足以下恒等式:
$$
1 \wedge a_i = a_i \wedge 1 = a_i~
$$
$$
a_1 \wedge a_2 \wedge \cdots \wedge a_r 
= \frac{1}{r!}\sum_{\sigma \in S_r}\operatorname{sgn}(\sigma)\, a_{\sigma(1)}a_{\sigma(2)}\cdots a_{\sigma(r)},~
$$
其中,求和是对所有指标的排列进行的,$\operatorname{sgn}(\sigma)$ 表示排列 $\sigma$ 的符号,$a_i$ 是向量(而不是代数中的一般元素)。
由于代数中的每个元素都可以表示为这种形式的乘积之和,这就为代数中任意两个元素定义了外积。由定义可知,外积构成一个交替代数。

Clifford 代数的等价结构方程为\(^\text{[16][17]}\):
$$
a_1 a_2 a_3 \dots a_n
=\sum_{i=0}^{\lfloor \frac{n}{2} \rfloor}\;
\sum_{\mu \in \mathcal{C}}
(-1)^k \, \operatorname{Pf}\!\bigl(a_{\mu_1}\cdot a_{\mu_2},\dots,a_{\mu_{2i-1}}\cdot a_{\mu_{2i}}\bigr)\;
a_{\mu_{2i+1}} \wedge \dots \wedge a_{\mu_n},~
$$
其中,$\operatorname{Pf}(A)$ 是矩阵 $A$ 的Pfaffian(帕夫算子);
$\mathcal{C}={\binom{n}{2i}}$ 表示将 $n$ 个指标划分为 $2i$ 个和 $n-2i$ 个部分的所有组合;
$k$ 表示该组合的奇偶性。

Pfaffian 为外代数提供了一种度量。正如 Claude Chevalley 指出的那样,当二次型为零时,Clifford 代数会退化为外代数。\(^\text{[18]}\)从几何角度看,可以通过单形来发展 Clifford 代数,这样就能理解 Pfaffian 所起的作用。\(^\text{[19]}\)这种推导方式也使 Pascal 三角形与单形之间建立了更好的联系,因为它为 Pascal 三角形第一列的“1”提供了一个解释。
\subsubsection{刀、阶与基}
一个由 $r$ 个线性无关向量的外积构成的多向量称为刀,并称它的阶为 $r$\(^\text{[f]}\)。一个由若干个同阶 $r$ 的刀相加得到的多向量称为(同质的)$r$ 阶多向量。根据几何代数的公理及其封闭性,每一个几何代数中的多向量都可以表示为若干刀的和。

设有一组 $r$ 个线性无关向量 $\{a_{1}, \ldots, a_{r}\}$,它们张成向量空间的一个 $r$ 维子空间。由此可以定义一个实对称矩阵(类似于 Gram 矩阵):
$$
[\mathbf{A}]_{ij} = a_i \cdot a_j~
$$
根据谱定理,矩阵 $\mathbf{A}$ 可以通过一个正交矩阵 $\mathbf{O}$ 对角化为对角矩阵 $\mathbf{D}$:
$$
\sum_{k,l} [\mathbf{O}]_{ik}\,[\mathbf{A}]_{kl}\,[\mathbf{O}^{\mathrm{T}}]_{lj}
= \sum_{k,l} [\mathbf{O}]_{ik}\,[\mathbf{O}]_{jl}\,[\mathbf{A}]_{kl}
= [\mathbf{D}]_{ij}~
$$
由此定义一组新的向量 $\{e_{1}, \ldots, e_{r}\}$,称为正交基向量,它们由正交矩阵变换得到:
$$
e_i = \sum_{j} [\mathbf{O}]_{ij}\, a_j~
$$
由于正交变换保持内积,因此有$e_i \cdot e_j = [\mathbf{D}]_{ij}$于是 $\{e_1, \ldots, e_r\}$ 互相正交。换句话说,两个不同的向量 $e_i \neq e_j$ 的几何积完全由它们的外积决定,更一般地:
$$
\begin{aligned}
e_1 e_2 \cdots e_r 
&= e_1 \wedge e_2 \wedge \cdots \wedge e_r \\
&=\left(\sum_{j} [\mathbf{O}]_{1j} a_j\right) \wedge 
  \left(\sum_{j} [\mathbf{O}]_{2j} a_j\right) \wedge \cdots \wedge 
  \left(\sum_{j} [\mathbf{O}]_{rj} a_j\right) \\
&= (\det \mathbf{O}) \, a_1 \wedge a_2 \wedge \cdots \wedge a_r
\end{aligned}~
$$
因此,每一个 $r$ 阶刀都可以写成 $r$ 个向量的外积。更一般地,如果允许退化几何代数,那么正交矩阵被一个分块矩阵所取代,其中非退化部分仍是正交的,而对角矩阵在退化维度上含有零值。若将非退化子空间中的新向量按如下方式归一化:
$$
\widehat{e_i} = \frac{1}{\sqrt{|e_i \cdot e_i|}} \, e_i,~
$$
那么这些归一化后的向量必须平方为 $+1$ 或 $-1$。根据Sylvester 惯性定理,对角矩阵中平方为 $+1$ 的向量总数,以及平方为 $-1$ 的向量总数都是不变量。由此推广,记平方为 $+1$ 的基向量数为 $p$,平方为 $-1$ 的基向量数为 $q$,它们都是不变量。(若考虑退化情况,平方为零的基向量数也同样是不变量,并且可能非零。)我们将这种代数记作 $\mathcal{G}(p,q)$。例如:$\mathcal{G}(3,0)$ 对应三维欧几里得空间,$\mathcal{G}(1,3)$ 对应相对论时空,$\mathcal{G}(4,1)$ 对应三维空间的共形几何代数。

所有可能的由 $n$ 个正交基向量按递增顺序相乘所形成的积(包括作为空积的 $1$),构成了整个几何代数的一个基(类似于PBW 定理的情形)。例如,几何代数 $\mathcal{G}(3,0)$ 的一个基为:
$$
\{1, e_{1}, e_{2}, e_{3}, e_{1}e_{2}, e_{2}e_{3}, e_{3}e_{1}, e_{1}e_{2}e_{3}\}~
$$
以这种方式形成的基称为几何代数的标准基。对于向量空间 $V$ 的任一组正交基,都将生成另一组标准基。每一个标准基包含 $2^n$ 个元素。几何代数中的任意多向量都可以表示为标准基元素的线性组合。若标准基元素记作 $\{B_i \mid i \in S\}$,其中 $S$ 是一个指标集合,则任意两个多向量的几何积为:
$$
\left(\sum_{i} \alpha_i B_i\right)\left(\sum_{j} \beta_j B_j\right)
= \sum_{i,j} \alpha_i \beta_j B_i B_j .~
$$
术语“$k$-向量”常用来描述只包含单一阶次元素的多向量。在高维空间中,有些这样的多向量并不是刀(即不能分解为 $k$ 个向量的外积)。例如,在 $\mathcal{G}(4,0)$ 中:$e_1 \wedge e_2 + e_3 \wedge e_4$就不能被分解。通常,这类代数元素没有直接的几何对象解释,但它们可能表示某些几何量,例如旋转。只有 $0$-、$1$-、$(n-1)$- 和 $n$-向量在 $n$ 维空间中总是刀。
\subsubsection{向量子}
一个$k$-versor是可以表示为 $k$ 个可逆向量的几何积的多向量。\(^\text{[g][21]}\)四元数中的单位四元数(最初由哈密顿称为 versors)可以与三维空间中的转子对应,就像二维实转子包含了复数一样;具体细节可参见 Dorst\(^\text{[22]}\)。

一些作者使用“versor 积”一词来指代经常出现的一种情况:即一个操作数被“夹在”两个算子之间。旋转和反射(包括它们的外射outermorphism)的描述就是这种“夹心”形式的例子。这些外射具有特别简单的代数形式。\(^\text{[h]}\) 具体来说,形如

$V \to V : a \mapsto RaR^{-1}$的向量映射可以扩展为外射:$\mathcal{G}(V) \to \mathcal{G}(V) : A \mapsto RAR^{-1}$.

由于算子和操作数都是 versor,因此还可能存在其他形式的例子,例如旋转一个转子、或者反射一个旋量——只要能为这些运算赋予某种几何或物理意义即可。

根据Cartan–Dieudonné 定理,每一个等距变换都可以表示为对超平面的反射。而既然复合的反射能产生旋转,那么正交变换就是 versor。

在群论意义上,对于一个实的、非退化的几何代数 $\mathcal{G}(p,q)$,若将群 $\mathcal{G}^{\times}$ 识别为几何代数中所有可逆元素的群,Lundholm 给出了一个证明,说明“versor 群”$\{v_{1} v_{2}\cdots v_{k} \in \mathcal{G} \mid v_{i} \in V^{\times}\}$(即所有可逆 versor 的集合)等同于Lipschitz 群 $\Gamma$(又称 Clifford 群,不过 Lundholm 并不提倡这种叫法)\(^\text{[23]}\)。
\subsubsection{Lipschitz 群的子群}
我们将次数对合记作 $\widehat{S}$,转置记作 $\widetilde{S}$。

虽然Lipschitz 群(定义为$\{S \in \mathcal{G}^{\times} \mid \widehat{S} V S^{-1} \subseteq V\}$)和versor 群(定义为$\Bigl\{\prod_{i=0}^{k} v_i \ \Bigm| \ v_i \in V^{\times}, \ k \in \mathbb{N}\Bigr\}$)有不同的定义,但它们实际上是同一个群。Lundholm 将Pin 群$\operatorname{Pin}$、Spin 群$\operatorname{Spin}$、以及正旋子群$\operatorname{Spin}^{+}$ 定义为 Lipschitz 群的子群。\(^\text{[24]}\)
\begin{figure}[ht]
\centering
\includegraphics[width=10cm]{./figures/1321ca65dbb25328.png}
\caption{} \label{fig_jiheds_3}
\end{figure}
多种关于旋量的分析都使用几何代数(GA)作为表示方法。\(^\text{[25]}\)
\subsubsection{分级投影}
可以通过几何积自然诱导出的外积,在几何代数上建立一个$\mathbb{Z}$-分级向量空间结构。

由于在正交向量上几何积与外积相等,因此可以方便地利用一个正交基$\{e_{1}, \ldots , e_{n}\}$来构造这种分级。

几何代数中是1的数量倍的元素为0 次(grade 0),称为标量。在 $\{e_{1}, \ldots , e_{n}\}$ 张成空间中的元素为1 次(grade 1),即通常的向量。在 $\{e_{i}e_{j}\mid 1\leq i<j\leq n\}$ 张成空间中的元素为2 次(grade 2),称为双向量。这一术语可以继续推广,直到最高次的$n$-向量。另一种叫法是:$n$-向量称为伪标量;$(n-1)$-向量称为伪向量;以此类推。代数中的许多元素并不属于这种分级体系,因为它们是不同次数元素的和,这类元素称为混合次数。多向量的分级与最初选取的基无关。

需要注意的是,这是一种向量空间的分级,但不是代数的分级。因为一个 $r$-刃($r$-blade)与一个 $s$-刃的乘积,包含在从 0 次到 $(r+s)$-次刃的张成空间中,所以几何代数实际上是一个滤代数。

一个多向量 $A$ 可以通过分级投影算子$\langle A\rangle_r$ 来分解,该算子输出 $A$ 的$r$ 次部分。因此:
$$
A = \sum_{r=0}^{n} \langle A \rangle_{r}~
$$
例如,两个向量的几何积:$ab = a \cdot b + a \wedge b = \langle ab \rangle_{0} + \langle ab \rangle_{2}$因为 $\langle ab \rangle_{0} = a \cdot b$,$\langle ab \rangle_{2} = a \wedge b$,而对除 $0$ 与 $2$ 之外的 $i$,都有 $\langle ab \rangle_{i} = 0$。

一个多向量 $A$ 也可以分解为偶次部分与奇次部分,分别是所有偶次数分量与所有奇次数分量的和:
$$
A^{[0]} = \langle A \rangle_{0} + \langle A \rangle_{2} + \langle A \rangle_{4} + \cdots~
$$
$$
A^{[1]} = \langle A \rangle_{1} + \langle A \rangle_{3} + \langle A \rangle_{5} + \cdots~
$$
这可以看作是把原本的 $\mathbf{Z}$-分级向量空间退化为 $\mathbf{Z}_2$-分级向量空间的结果。几何积遵守这种更粗的分级,因此几何代数不仅是一个 $\mathbf{Z}_2$-分级向量空间,还是一个$\mathbf{Z}_2$-分级代数(又称超代数,superalgebra)。

若限制在偶次部分,两个偶元素的乘积仍然是偶的。这意味着偶多向量形成一个偶子代数。一个 $n$ 维几何代数的偶子代数,与一个 $(n-1)$ 维的完整几何代数代数同构(但不保留滤过结构或分级结构)。例如:$\mathcal{G}^{[0]}(2,0) \cong \mathcal{G}(0,1)$和$\mathcal{G}^{[0]}(1,3) \cong \mathcal{G}(3,0)$
\subsubsection{空间的表示}
几何代数用刀刃来表示向量空间 $V$ 的子空间,因此它们与来自 $V$ 的向量共存于同一个代数中。一个 $k$ 维子空间 $W \subseteq V$,可以通过选取一个正交基 $\{b_{1}, b_{2}, \ldots, b_{k}\}$,并使用几何积构造出刀刃$
D = b_{1} b_{2} \cdots b_{k}$.存在多个刀刃可以表示同一个子空间 $W$;凡是表示 $W$ 的刀刃,都是 $D$ 的标量倍。它们可以分成两类:$D$ 的正倍数与负倍数。正倍数被称为与 $D$ 具有相同的定向,而负倍数则具有相反的定向。

刀刃之所以重要,是因为几何操作(如投影、旋转和反射)依赖于外积下的可分解性,而这种可分解性由受限类别的 $n$-刀刃所提供,而当 $n \geq 4$ 时,一般的 $n$-级多向量并不具备这种性质。
