% Julia数值类型的提升
% 数值类型 提升

本文授权转载自郝林的 《Julia 编程基础》. 原文链接:\href{https://github.com/hyper0x/JuliaBasics/blob/master/book/ch05.md}{第 5 章 数值与运算}.


\subsection{5.6 数值类型的提升}

Julia 中有一个辅助系统,叫做类型提升系统.它可以将数学运算符操作的多个值统一地转换为某个公共类型的值,以便运算的顺利进行.我们下面就简要地说明一下这个辅助系统的应用和作用.关于公共类型的解释也会在其中.

在 Julia 中,数学运算符其实也是用函数实现的.就拿用于二元加的运算符`+`来说,它的一个衍生方法的定义是这样的:

```julia
+(x::Float64, y::Float64) = add_float(x, y)
```

这个定义向我们揭示了两个细节.第一个细节就是我刚刚说的,数学运算符是由函数实现的.不仅如此,针对每一类可操作的数值,Julia 都定义了相应的衍生方法.第二个细节是,数学运算符操作的多个值必须是同一个类型的.你可能会有疑问,那为什么我们编写的像`1 + 2.0`这样的运算依然可以顺利进行呢?实际上,这恰恰得益于 Julia 的类型提升系统.我们来看该系统中的一个定义:

```julia
+(x::Number, y::Number) = +(promote(x,y)...)
```

这个衍生方法的两个参数的类型都是`Number`.这就意味着,只要参与二元加的操作数都是数值且它们的类型不同,该运算就会被分派到这个方法上.如果两个数值的类型相同,那么二元加运算就会被分派到像前一个定义那样的方法上.

请注意,这个衍生方法的定义中有一个对`promote`函数的调用.这个函数其实就代表了类型提升系统的核心算法.我们可以在 REPL 环境中输入表达式`promote(1, 2.0)`并回车.其结果如下:

```julia
julia> promote(1, 2.0)
(1.0, 2.0)

julia> typeof(ans)
Tuple{Float64,Float64}

julia> 
```

我们都知道,在 64 位的计算机系统中,字面量`1`的类型一定是`Int64`,而字面量`2.0`的类型肯定是`Float64`.由此,在那个调用`promote`函数后得到的元组中,包含了转换自参数值`1`的、`Float64`类型的数值`1.0`,以及保持原样的、`Float64`类型的数值`2.0`.这正是类型提升系统所起到的作用.它一般会先找到能够无损地表示输入值的某个公共类型,然后把这些值都转换为此公共类型的值(通常通过调用`convert`函数实现),最后输出这些类型统一的值.

在一般情况下,如果参数值列表中只包含了整数和有理数,那么`promote`函数就会把这些参数值都转换为有理数.倘若参数值列表中存在浮点数(但不存在复数),那么这个函数就会把这些参数值都转换为适当类型的浮点数.一旦参数值列表中有复数,那该函数就一定会返回适当类型的复数的元组.另一方面,如果这些参数值的类型只是在宽度上所有不同(如`Int64`和`Int8`、`Float16`和`Float32`等等),那么`promote`函数就会把它们都转换为宽度较大的那个类型的值.

我们倒是不用死记硬背这些规则.因为有一个名叫`promote_type`的函数,它可以接受若干个类型字面量并返回它们的公共类型.例如:

```julia
julia> promote_type(Int64, Float64)
Float64

julia> promote_type(Int64, Int8)
Int64

julia> promote_type(Float16, Float32)
Float32

julia> 
```

请注意,我们一直在说的是多个类型的公共类型,而不是多个类型的共同超类型.这两者之间并没有任何关联.如果你确实想得到两个类型的共同超类型,那么可以调用`typejoin`函数.例如,调用表达式`typejoin(Int, Float64)`的求值结果会是`Real`.

好了,不论细节如何,经过前文所述的处理之后,这些数值就可以交给普通的运算符实现方法进行操作了.就像这样:

```julia
julia> +(promote(1, 2.0)...)
3.0

julia> 
```

这里对`+`函数的调用会被分派到我们在前面展示的那个针对`Float64`类型的衍生方法上.

解释一下,符号`...`的作用是,把紧挨在它左边的那个值中的所有元素值(如元组`(1.0, 2.0)`中的`1.0`和`2.0`)都平铺开来,并让这些元素值都成为传入外层函数(如`+`函数)的独立参数值.所以,调用表达式`+((1.0, 2.0)...)`就相当于`+(1.0, 2.0)`.

至于什么是元组,你现在可以简单地把它理解为由圆括号包裹的、可承载若干值的容器.函数在同时返回多个值的时候通常就会用这种数据结构呈现.在后面讲参数化类型的那一章里有对元组的详细说明.

除了以上讲的这些,Julia 的类型提升系统还有一个很重要的作用,那就是:让我们可以编写自己的类型提升规则,以自定义数学运算的部分行为,尤其是在操作数的类型不同的时候.例如,若我们想让整数和浮点数的运算结果变成`BigFloat`类型的值,则可以这样做:

```julia
julia> import Base.promote_rule

julia> promote_rule(::Type{Int64}, ::Type{Float64}) = BigFloat
promote_rule (generic function with 137 methods)

julia>
```

第一行代码是一条导入语句.简单来说,我们在编写某个函数的衍生方法的时候必须先导入这个函数.第二行代码就是我编写的衍生方法.由于与之相关的一些背景知识我们还没有讲到,所以你看不太懂也没有关系.在这里,你只要关注这行代码中的`Int64`、`Float64`和`BigFloat`就可以了.前两个都代表了操作数的类型,而后一个则代表了它们的公共类型.这正是在定义操作数类型和公共类型的对应关系.

现在,我们再次执行之前的代码:

```julia
julia> promote(1, 2.0)
(1.0, 2.0)

julia> typeof(ans)
Tuple{BigFloat,BigFloat}

julia> 
```

可以看到,这次调用`promote`函数后得到的元组包含了两个`BigFloat`类型的值.这就说明我们刚刚编写的类型提升规则已经生效了.当然,修改 Julia 内置的类型提升规则是比较危险的.因为这可能会改变已有代码的基本行为,并且会明显地降低程序的稳定性,所以还是要谨慎为之.但对于我们自己搭建的数值类型体系来讲,这一特性的潜力是非常可观的.

总之,Julia 的类型提升系统辅助维护着数学运算的具体实现.其中有着大量的默认规则,并确保着常规运算的有效性.但同时,它也允许我们自定义类型提升的规则,以满足自己的特殊需要.