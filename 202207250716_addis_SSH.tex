% SSH 笔记

\begin{issues}
\issueDraft
\end{issues}

## 密码连接 ssh
* 安装 openssh-server `sudo apt install openssh-server`
* 修改 sshd 设置文件 `/etc/ssh/sshd_config` (注意不是 `ssh_config`): 将 `PasswordAuthentication` 改为 `yes`, 最后一行加上 `AllowUsers 用户名`, 用户名是 linux 的用户名
* 改完设置文件后 `sudo service ssh restart` 才会生效
* 查看 service 是否在运行用  `sudo service ssh status`
* 可以试试自己连接自己, 即 `ssh localhost`, 如果成功, 就可以了

## 使用 key 连接 ssh
* 参考 https://www.digitalocean.com/community/tutorials/ssh-essentials-working-with-ssh-servers-clients-and-keys
* 用 `ssh-keygen` 生成 key, 使用默认路径 `~/.ssh/`, 会生成 `id_rsa` 和 `id_rsa.pub` 两个文件. 后者用于加密而不能解密, 是 public key, 可以给任何人. 前者是 private key, 用于解密, 绝对不能给别人.
* 要先登录 host, 要把 public key (只有一行内容) append 到服务器中的 `~/.ssh/authorized_keys` 文件.
* 一般来说如果用 key 的话 server 的 `/etc/ssh/sshd_config` 是不需要设置的
* 其中 append 的操作也可以用 `ssh-copy-id -i /path/to/key.pub SERVERNAME` 会更方便, 如果事先允许密码登录的话
* ssh 用 key 来登录的原理大概就是, client 告诉 server 我要用哪个用户名登录, server 去用户名的 `~/.ssh/authorized_keys` 中寻找 public key, 然后用其加密一个随机字符串给 client, client 用 private key 解密, 将字符串的 MD5 hash 发给 server, server 一对照 MD5 如果吻合, 开始连接.  (其实我猜 host 还需要发送一个密钥用于 ssh 通讯, 要不然 client 发给 host 的东西就无法加密了)
* 现在 ssh 的话应该就不需要密码了
* client 上可以写一个设置文件 `~/.ssh/config`, 设置 host 的名字, 就可以用名字而不是 ip 登录了
```
Host [SERVERNAME]
Hostname [ip-or-domain-of-server]
User [USERNAME]
PubKeyAuthentication yes
IdentityFile [./path/to/key]
```
最后一条强制使用 key 登录而不用密码.

* 如果说用 key `ssh` 的时候说 `permission id_rsa too open`, 用 `chmod 700 id_rsa*` 即可, 其中 7 变为二进制是 `111` 对应 `rwx` 的三个开关都打开, 0 对应 `000` 没有任何权限, 所以 700 就是 owner 有所有权限, 其他用户没有任何权限
* 虽然 public key 中最后有 `用户名@主机`, 其实这个好像没有任何影响, private key 和 public key 都可以复制到许多不同的机器中, 让这些机器两两之间互相 ssh, 而它们的 `~/.ssh/authorized_keys` 中只需要有一条
* ssh 登录一个新主机时, 如果 `known_host` 没有, 就会让确认, 确认以后就会把新主机的 id 加入该文件, 如果这个 id 以后变了(例如主机重装了 `openssh-server`) 再登录就会出错(为了安全), 这时只需要将 `known_host` 中该主机的记录删除即可. 也可以直接把整个文件删除, 但这样就所有 ssh 登录都要再确认一次了.

## ssh 语法
* `ssh xxx@xxx "命令"` 可以直接将某个命令在 host 上执行并退出, 而不需要先进入 host 的 shell, 执行命令再 exit. 用 `&&` 可以将多个命令写在一起依次执行.
* `ssh xxx@xxx "cat > remote_file" < local_file` 可以直接将本地文件用 ssh 直接传到 host, 而不需要用 sftp. 也可以用 piping, 例如 `cat local_file | ssh xxx@xxx "cat > remote_file"`, 效果一样
* `ssh xxx@xxx "cat /path/to/remote_file" > /path/to/local_file` 可以将 host 的文件传到本地
