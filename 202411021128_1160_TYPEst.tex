% 弦论的种类
% keys 对偶|M 理论|Yang-Mills 理论|p 模
% license Usr
% type Tutor

\begin{issues}
\issueNeedCite
\issueMissDepend
\end{issues}

\subsection{五种弦论}

弦论有六种,包括一种玻色弦和五种超弦,五种超弦通过\textbf{对偶(duality)}联系在一起。这五种弦论分别是

1. 玻色弦论
这个弦论里面只有玻色子。没有超对称。因为这种弦论里面没有费米子,且存在快子(虚质量粒子,代表理论不稳定),所以这种弦论不能描述物质。这只是一个玩具理论。它包含了开弦,闭弦。需要26维的时空。

2. 第一型弦论
这个版本的弦论包含了玻色子和费米子。粒子相互作用包含了N=1超对称和规范群SO(32)。超弦理论均为十维。

3. 第二A型弦论
这个版本的弦论同样包含了超对称,开弦和闭弦,超对称为N=(1,1),因此这个理论里面费米子不是chiral的。第二A型弦论的端点能够attach到高维物体(D-膜)上面,稳定的D-膜维数为奇数(p为偶数)。

4. 第二B型弦论
跟二A型弦论差不多,只不过N=(2,0),因此这种弦论里面的费米子是chiral的。稳定的D-膜维数为偶数(p为奇数)。

5. Heterotic $E_8\times E_8$弦论
包含了N=1超对称,只允许闭弦。规范群是$E_8\times E_8$。左行和右行模式需要不同的时空维度,其中玻色部分紧化到10维。杂化弦被认为有可能得到现实世界的物理。

6. Heterotic SO(32)弦论
跟Heterotic $E_8\times E_8$弦论差不多,只不过规范群是SO(32)。

\subsection{M-理论}
不同的duality把这几种不同的弦论联系在了一起。这些duality的名字分别是S duality和T duality。所以一个大胆的猜测是,在这五种弦论的背后,还有一个更加基本的理论:M理论。

\subsection{p-膜}
p-膜的意思是,具有p个空间维度的物体。D-膜在弦论里面非常重要,因为开弦的端点能够attach到它上面。比如Yang-Mills理论,就涉及到attach到D-膜的开弦。 引力并不attach到D-膜上。这也就解释了为什么引力这样弱。

\subsection{高维}
这个idea首先是Kaluza和Klein首先提出来的。 在五维的空间,引力和电磁力可以很好地统一到同一个框架下。 他们提出了compactification这个idea。
意思就是,当额外的维度足够小的时候,多余的维度就探测不到了。 
