% 厦门大学 2001 年 考研 量子力学
% license Usr
% type Note

\textbf{声明}:“该内容来源于网络公开资料,不保证真实性,如有侵权请联系管理员”

\subsection{(30分)回答和计算下列问题}
\begin{enumerate}
    \item 为什么表示方位量的算符必须是厄米算符?
    \item 什么是简并度?氢原子能级的简并度是多少?(不考虑电子自旋)
    \item 全同粒子体系波函数对于交换两个全同粒子具有对称性,试问:这种对称性会不会随时间改变?
    \item 电子在均匀电场 $\vec{\epsilon}=(0, \epsilon, 0)$ 中运动,单哈密顿量为:
      \[      \hat{H} = \frac{\hat{p}^2}{2m} + e\epsilon y ~\]
      判断下列那些量是不守恒量:
      \[      \hat{P}_x,\ \hat{P}_y,\\  \hat{L}_x,\\ \hat{L}_y,\\  ~\]
    \item 写出线性谐振子偶极跃迁的选择定则。
    \item $\hat{L}_x$ 与 $\hat{L}_y$ 相互不对易,试问:它们会不会有共同本征函数?若有,写出具体函数表达式。($\hat{L}_x,\\ \hat{L}_y$ 分别为轨道角动量在 $x, y$ 方向的分量)
  \end{enumerate}
  
\subsection{(20分)}
一粒子在一维无限深方势阱中运动
  \[  U(x) =  \begin{cases}    0, & 0 < x < a \\\\    \infty, & \text{其他区域}  \end{cases} ~\]
  \begin{enumerate}
    \item 试求粒子的能级和对应的归一化波函数。
    \item 若状态用波函数 $x(t) = \frac{4}{\sqrt{a}} \sin\left( \frac{\pi x}{a} \right) \cos^2\left( \frac{\pi x}{a} \right)$ 描述,求粒子能量的可能测量值及相应几率。
  \end{enumerate}
\subsection{(15分)}