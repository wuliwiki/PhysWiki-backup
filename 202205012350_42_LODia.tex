% 线性算子对角化的充要条件
% keys 对角化|充要条件
\pentry{特征矢量与特征多项式\upref{EigVM}}
\begin{definition}{}
 $n$ 维矢量空间 $V$ 中,若有一基底,使得在该基底下,线性算子 $\mathcal{A}$ 对应的矩阵 $A$ 取对角形式
 \begin{equation}
 A=\begin{pmatrix}
 \lambda_1&0&\cdots&0\\
 0&\lambda_2&\cdots&0\\
 \vdots&\vdots&\cdots&\vdots\\
 0&0&\cdots&\lambda_n
 \end{pmatrix}
 \end{equation}
 则称算子 $\mathcal{A}$ 是\textbf{可对角化}的.
\end{definition}
\begin{theorem}{线性算子可对角化的充要条件}
定义在域 $\mathbb{F}$ 上的 $n$ 维矢量空间,其上的线性算子 $\mathcal{A}$ 可对角化的充要条件为:$\mathcal{A}$ 的特征多项式\upref{EigVM} $\mathrm{det}{\mathcal{A}-t \mathcal{E}}$ 的 所有根都在 $\mathbb{F}$ 上,且每个特征值 $\lambda$ 的几何重数等于代数重数\upref{EigVM}.


\end{theorem}