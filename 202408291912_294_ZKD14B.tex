% 中国科学技术大学 2014 年考研普通物理 B 考试试题
% keys 中国科学技术大学|2014年|考研|物理B
% license Copy
% type Tutor

\textbf{声明}:“该内容来源于网络公开资料,不保证真实性,如有侵权请联系管理员”
\begin{enumerate}
\item (15分)离水面高度为h的岸上有人用绳索拉船靠岸。人以匀速率v拉绳求当船离岸的距离为s的时候,船的速度和加速度。
2.(20分)在一铅直面内有一个光滑轨道,左边是一个上升的曲线,右边是足够长的水平直线,二者平滑连接,如图所示。现有A、B两个质点,B在水平轨道上静止,A在曲线部分高h处由静止滑下,与B发生完全弹性碰撞。碰后仍可返回上升到曲线轨道某处,并再度下滑,已知A、B两质点的质量分别为m和m,。求至少发生两次碰撞的条件。
题2图
一个自转着的密度均匀的球形行星(行星半径为r,没有大气),在它赤道上的点的速率为v,赤道上的重力加速度,是两极处重力加速度g的213,问粒子从极点的逃逸速度是多大?在此行星表面有一质点A以某一速度沿着行星表面的切线方向发射出来,之后便围绕该行星做椭圆轨道运动,距离行星最远时的距离为R(其中R>r为A到行星中心的距离),另一质点B在距离行星为R的轨道上围绕该行星做圆周运动,求A和B周期之比。
一个圆环穿在光滑水平绝缘杆上,当条形磁铁的 N极向右靠近圆环时,圆环会向什么方向运动?为什么?分别就圆环是铁环、铝环和塑料三种情形讨论
一半径为R,厚度为h(h<<R)的均匀介质圆板被均匀极化,极化强度P平行于板面,求面极化电荷密度和圆板中心的退极化场强。
一同轴电缆,内、外分别是半径为a、b的导体薄圆筒,通以电流1,之间充满绝对磁导率为山的介质。求
(1)磁化电流分布;
电缆单位长度的自感系数;2
外圆简单位面积所受的磁作用力。
考虑一个多电子原子,其电子组态为1s22s22p63s?3p63d104s?4p4d:
(1)这个原子是否处于基态?如果不是基态,那题6图么基态的电子组态又是什么样的?(2)写出该原子基态的原子态符号;如果该原子经过斯特恩-盖拉赫的实验装置,它分裂为几条?说明理由。
分别按照LS耦合和耦合确定下列组态所能构成的全部原子态:
(1) npnd;
(2)ndn'd (n<n')
\end{enumerate}