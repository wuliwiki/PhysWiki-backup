% 密度矩阵(综述)
% license CCBYSA3
% type Wiki


在量子力学中,密度矩阵(或密度算符)是描述物理系统集体(即使该集体仅包含一个系统)作为量子态的矩阵。它允许通过博恩规则计算对集体中各系统进行测量后的结果概率。密度矩阵是比常见的状态向量或波函数更一般的概念:状态向量或波函数只能表示纯态,而密度矩阵也能表示混合集体(有时模糊地称为混合态)。混合集体在量子力学中有两种情况:
\begin{enumerate}
\item 当系统的准备导致集体中有多个纯态时,需要处理可能准备的统计分布;
2. 当想要描述一个与另一个系统纠缠的物理系统时,而不描述它们的组合态;这种情况通常出现在系统与某些环境交互时(例如去相干)。在这种情况下,纠缠系统的密度矩阵不同于由多个纯态组成的集体的密度矩阵,后者在测量时会给出相同的统计结果。
\end{enumerate}
因此,密度矩阵在处理混合集体的量子力学领域中是至关重要的工具,如量子统计力学、开放量子系统和量子信息等。