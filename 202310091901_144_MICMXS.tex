% 混合熵的微观解释
% license CCBY4
% type Tutor


\pentry{理想气体混合的熵变\upref{IGME}}

\footnote{参考书目:Gaskel et al. Thermodynamics of Materials}
在理想气体混合的熵变\upref{IGME}我们已经知道了如何从经典的方法计算混合熵:
\begin{equation}
\Delta_{mix} S = - R \sum n_i \ln x_i~,
\end{equation}
现在,我们尝试用统计力学的观点,从微观角度论证为什么混合熵的公式是这样的。

想象我们有$N_A$个$A$球,$N_B$个$B$球,现在将他们混合在一起,那么会有几种可能的混合方式?幸运的是,统计力学假设所有微观粒子都是相同的,因此我们不需要考虑球的摆放顺序,那么,问题简化为一个\textsl{gao}

