% 爱因斯坦场方程
% 引力场|引力|gravity|场方程|Ricci张量|测地线|geodesic|广义相对论|相对论|relativity|时空|spacetime|弯曲|曲率

\pentry{测地线\upref{geodes},尘埃云的能动张量\upref{SRFld}}

\subsection{能动张量}

爱因斯坦场方程的引出过程中,我们考虑的是最简单的宏观模型,即\textbf{尘埃云的能动张量}\upref{SRFld}中所介绍的\textbf{理想流体}.对于任何物质,其四动量的各分量,都会随着参考系的不同而有不同取值,因此这些量只能是流形上某种量在具体坐标系中的坐标分量而已.这样一来,在流形上描述能量质量分布的量,就不能是简单的光滑函数,或者说标量场,而只能是更高阶的张量场.能描述理想流体四动量分布的张量,可以是四动量本身,也可以是能动张量,而我们会选择能动张量,这样才能和Ricci曲率张量的阶数吻合.

能动张量能如何描述物质分布的性质呢?

\begin{definition}{能量密度}
令$T_{ab}$为一理想流体在时空流形上的能动张量场,$u^a$是该流体在各时空点处局部质心系的四速度,则$T_{ab}u^au^b$是该流体在该处的\textbf{能量密度}.
\end{definition}

这一定义是合理的.考虑到局部质心系中$u^a=\pmat{1&0&0&0}\Tr$,根据理想流体的能动张量的定义,可以计算出$T_{ab}u^au^b$就是能量密度分布,而这是张量场的抽象指标分布,不依赖于参考系的选择,所以是合理的推广表达.

\subsubsection{守恒定律}

理想流体的能动张量,满足方程
\begin{equation}\label{EinEqn_eq1}
\partial^aT_{ab}=0
\end{equation}

\autoref{EinEqn_eq1} 的理解可以类比向量场的散度$\partial_av^a$.如果向量场$v^a$描述了某种东西的流,比如说流体的质量流,那么$\partial_av^a$就是流体质量的流失速率,在$\partial_av^a=0$的地方自然就有质量密度守恒.

根据指标的升降法则,对于任何张量$\bvec{T}$,我们定义$\partial^a\bvec{T}=\eta^{ab}\partial_a\bvec{T}$,这样子我们就可以把\autoref{EinEqn_eq1} 理解为$\eta^{ai}\partial_i(T_b)^a$,相当于用余切向量场$T_b$取代了光滑函数$v$,形式不变,同样表达了某种量的守恒.

什么东西守恒呢?

利用\autoref{SRFld_eq4}~\upref{SRFld},$T_{ab}=\rho U_aU_b+P g_{ab}+P U_aU_b$,把\autoref{EinEqn_eq1} 展开,我们能得到:
\begin{equation}\label{EinEqn_eq2}
\begin{aligned}
0=\partial^aT_{ab}=&\partial^a(\rho U_aU_b+P g_{ab}+P U_aU_b)\\
=&U_aU_b\partial^a\rho+\rho U_b\partial^aU_a+\rho U_a\partial^aU_b+\\
&g_{ab}\partial^ag_{ab}+P\partial^ag_{ab}+\\
&U_aU_b\partial^aP+P U_b\partial^aU_a+P U_a\partial^aU_b
\end{aligned}
\end{equation}

\autoref{EinEqn_eq2} 最右边很长,我们可以把它分成两部分,分别是在度量$\eta_{ab}$下垂直和平行于$U^b$的项.显然,其中的$U_aU_b\partial^a\rho+\rho U_b\partial^aU_a+P U_b\partial^aU_a$是平行于$U^b$的,因为它们都是在$U_b$前面乘以一个光滑函数的形式;我们断言,剩下的部分,是垂直于$U^b$的,即与$U^b$的内积为$0$.

这是因为,剩下的那部分还可以再分成两部分,一个是$(P+\rho)U_a\partial^aU_b$,一个是$(\eta_{ab}+U_aU_b)\partial^aP$.考虑到$U^b$按定义必须是单位向量,因此$\partial_aU^b$必定与$U^a$垂直,即$U^a\partial_aU^b=0$,因此$U^b(P+\rho)U_a\partial^aU_b=0$;再由洛伦兹度规的定义,$U^aU_b=1$,因此$U^b(\eta_{ab}+U_aU_b)=U_a-U_a=0$.综上,剩下这部分与$U^b$的乘积就是$0$\footnote{特别要注意的是$U^b(\eta_{ab}+U_aU_b)$这一部分,如果取$\eta_{ab}=\opn{diag}\pmat{1&-1&-1&-1}$,那它就不再垂直于$U^b$了.}.

将\autoref{EinEqn_eq2} 最右边按上述讨论分成垂直和平行于$U^b$的两部分后,就分别得到了两个独立的等式(适当进行指标升降以获得更直观的表达):

\begin{equation}\label{EinEqn_eq3}
U^a\partial_a\rho+(\rho+P)\partial^aU_a=0
\end{equation}

\begin{equation}\label{EinEqn_eq4}
(P+\rho)U^a\partial_aU_b+(\eta_{ab}+U_aU_b)\partial^aP=0
\end{equation}

回归经典极限时,三速度$\bvec{v}$很小,以至于$U^\mu=\pmat{1, \bvec{v}}$,而流体之间的压强也远小于密度,那么\autoref{EinEqn_eq3} 和\autoref{EinEqn_eq4} 就分别变成











