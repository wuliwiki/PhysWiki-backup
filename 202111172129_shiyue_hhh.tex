% hhh

Exercise 9.1 仅考虑电子自旋,其受到沿x轴方向的均匀磁场B的作用,哈密顿量为
$$
\hat{H}=\frac{e B}{\mu c} \hat{s}_{x}=\frac{e B \hbar}{2 \mu c} \hat{\sigma}_{x}
$$
当$t=0$时,电子处于自旋向上的状态,($\hat{s}_z$的本征值为$\hbar/2$),求$t>0$时,自旋各分量的平均值.

设哈密顿量的本征值为$E$,本征态为$\psi$,即
\begin{align}
	\hat{H}\psi =E\psi \,\,\quad
	\frac{eB}{2\mu c}\hat{\sigma}_x\psi =E\psi
\end{align}
$\psi$可由$\chi _{\frac{1}{2}}$和$\chi _{-\frac{1}{2}}$展开
\begin{align}
	\psi =c_1\chi _{\frac{1}{2}}+c_2\chi _{-\frac{1}{2}}
\end{align}

接下来求$\hat{\sigma}_x$的本征态和本征值:

把(2)带入(1)中
$$
\hat{\sigma}_x\left( c_1\chi _{\frac{1}{2}}+c_2\chi _{-\frac{1}{2}} \right) =E\left( c_1\chi _{\frac{1}{2}}+cc_2\chi _{-\frac{1}{2}} \right)
$$
由于
$$
\left( \begin{matrix}{}
	0&		1\\
	1&		0\\
\end{matrix} \right) \left( \begin{array}{l}
	1\\
	0\\
\end{array} \right) =\left( \begin{array}{l}
	0\\
	1\\
\end{array} \right) \quad \left( \begin{matrix}{}
	0&		1\\
	1&		0\\
\end{matrix} \right) \left( \begin{array}{l}
	0\\
	1\\
\end{array} \right) =\left( \begin{array}{l}
	1\\
	0\\
\end{array} \right) 
$$
$$
\hat{\sigma}_x\chi _{\frac{1}{2}}=\chi _{-\frac{1}{2}}\quad \hat{\sigma}_x\chi _{-\frac{1}{2}}=\chi _{\frac{1}{2}}
$$
整理得
\begin{align}
\left( c_2-Ec_1 \right) \chi _{\frac{1}{2}}+\left( c_1-Ec_2 \right) \chi _{-\frac{1}{2}}=0
\end{align}
所以
$$
c_2=Ec_1\quad	c_1=Ec_2\\
$$
$$E=\pm 1$$
当$E=1$时,
$$
\left( c_2-c_1 \right) \chi _{\frac{1}{2}}+\left( c_1-c_2 \right) \chi _{-\frac{1}{2}}=0$$$$
c_1=c_2
$$
又由于正交归一条件$\abs{c_1}^2+\abs{c_2}=1$得
$$
c_1=c_2=\frac{1}{\sqrt{2}}
$$
所以
$$
\psi_1=\frac{1}{\sqrt{2}}\chi _{\frac{1}{2}}+\frac{1}{\sqrt{2}}\chi _{-\frac{1}{2}}
$$
当$E=-1$时,
$$
\left( c_2+c_1 \right) \chi _{\frac{1}{2}}+\left( c_1+c_2 \right) \chi _{-\frac{1}{2}}=0$$$$
c_1=-c_2
$$
又由于正交归一条件$\abs{c_1}^2+\abs{c_2}=1$得
$$
c_1=-c_2=\frac{1}{\sqrt{2}}
$$
所以
$$
\psi_2=\frac{1}{\sqrt{2}}\chi _{\frac{1}{2}}-\frac{1}{\sqrt{2}}\chi _{-\frac{1}{2}}
$$

综上可知,$\hat{H}$的值及对应得本征态为
\begin{align}
E_1&=\frac{eB\hbar}{2\mu c}=\hbar \omega\\
E_2&=-\frac{eB\hbar}{2\mu c}=-\hbar \omega\\
w=\frac{eB}{2\mu c}
\end{align}

Exercise 9.2 在$(\hat{l}^2,\hat{l}_z)$表象中,$l=1$的子空间是几维的?求$\hat{l}_x$在该空间中的矩阵形式,并求出$\hat{l}_x$的本征值与本征态.