% 中国传媒大学 2014 年数据结构与计算机网络
% 中国传媒大学 2014  数据结构 计算机网络

答题说明:答案一律写在答题纸上,不需抄题,标明题号即可,答在试题上无效.

\subsection{一、单项选择题}
(每小题2分,共50分)

1.下列关于栈和队列说法中,正确的是(    ). \\
A.消除递归不一定需要使用栈 \\
B.对同一输入序列进行两组不同的合法入栈和出栈组合操作,所得的输出序列也一定相同 \\
C.通常使用队列来处理函数或过程处理 \\
D.队列和栈是运算受限的线性表,只允许在表的两端进行运算

2.一个栈的入栈序列是1,2,3,4,5,则栈的不可能的输出序列是( ). \\
A. 5,4,3,2,1,  $\qquad$ B.4,5,3,2,1 $\qquad$ C.4,3,5,1,2, $\qquad$ D.1,2,3,4,5

3.已知栈的输入序列为$1$,$2$,$3$,...$n$,输出序列为$p_1$, $p_2$, $p_3$, ..., $p_n$,若$p_1=3$,则$p_2$的值为(    ). \\
A.一定是2 $\qquad$ B.-定是1 $\qquad$ C.可能是1 $\qquad$ D.可能是2

4.循环队列用数组A[0..m-1]存放其元素值,已知其头尾指针分别为front和rear,则当前元素个数为( ). \\
A. (rear-front+m) MOD m $\qquad$ B. rear-front+1 \\
C. rear-front-1 $\qquad$ D. rear-front

5.已知有一维数组$A[0..m*n-1]$,若要对应为$m$行、$n$列的矩阵,将元素$A[k](0\leqslant k<m*n)$表示成矩阵的第$i$行、第$j$列的元素$(0\leqslant i<m, 0\leqslant j<n)$,则下面的对应关系是(    ). \\
A. i=k/n, j=k\%m $\qquad$ B. i=k/m, j=k\%m \\
C. i=k/n, j=k\%n $\qquad$ D. i=k/m, j=k\%n

6.设有一个$10$阶的对称矩阵$A$,采用压缩存储方式,以行序为主存储,$a_{1,1}$. 为第一元素,其存储地址为$1$,每个元素占一个地址空间,则$a_{8,5}$的地址是(). \\
A.13 $\qquad$ B.33 $\qquad$ C.18 $\qquad$ D.40

7.含有$n$个结点的三叉树的最小高度是(    ). \\
A.$n$ $\qquad$ B.$\lfloor n/3 \rfloor$ $\qquad$ C.$\lfloor log_3n\rfloor+1$ $\qquad$ D.$\lceil log_3(2n+1)\rceil$

8.在一棵具有n个结点的二叉树中,所有结点的空子树个数等于(    ). \\
A. n $\qquad$ B. n-1 $\qquad$ C. n十1 $\qquad$ D.2*n

9.在常用的描述二叉排序树的存储结构中,关键字值最大的结点是(    ). \\
A.左指针一定为空 $\qquad$ B.右指针一定为空 \\
C.左右指针均为空 $\qquad$ D.左右指针均不为空

10. 由权值为9、2、5、7的四个叶子构造一棵哈夫曼树,该树的带权路径长度为(    ) \\
A.23 $\qquad$ B.37 $\qquad$ C.44 $\qquad$ D.46

11. 若一个具有n个结点、k条边的非连通无向图是一个森林(n>k), 则该森林中必有树的数目是(    ). \\
A. k $\qquad$ B. n $\qquad$ C. n-k $\qquad$ D. n+k

12.采用邻接表存储的图的广度优先遍历算法类似于树的(    ). \\
A.中根遍历 $\qquad$ B.先根遍历 $\qquad$ C.后根遍历 $\qquad$ D.按层次遍历

13.在有向图$G$的拓扑序列中,若顶点$V_i$在顶点$V_j$之前,则下列情形不可能出现的是(    ). \\
A. $G$中有弧$<V_i,V_j>$ $\qquad$ B. $G$中有一条从$V_i$到$V_j$的路径 \\
C. $G$中没有弧$<V_i,V_j>$ $\qquad$ D. $G$中有一条从$V_j$到$V_i$的路径

14.有一个长度为12的有序表,按折半查找法对该表进行查找,在表内各元素等概率情况下,查找成功所需的平均比较次数是(    ). \\
A.37/12 $\qquad$ B.35/12 $\qquad$ C.39/12 $\qquad$ D.43/12

15.假设有k个关键字互为同义词,若用线性探查法把这k个关键字存入,至少要进行的探查次数是(    ). \\
A. k-I $\qquad$ B. k $\qquad$ C. k+1 $\qquad$ D. k(k+1)/2

16.下列序列中,满足堆定义的是(    ). \\
A. (100, 86, 48,73,35, 39,42,57, 66, 21) \\
B. (12, 70, 33, 65,24, 56, 48, 92, 86,33) \\
C. (103, 97,56,38,66, 23,42,12, 30,52, 6, 26) \\
D. (5, 56, 20, 23, 40, 38, 29, 61, 36,76, 28, 100)

17.对于一个长度为$n$的任意表进行排序,至少需要进行的比较次数是( ). \\
A. $O(m)$ $\qquad$ B. $O(n^2)$ $\qquad$ C.$O(logn)$ $\qquad$ D. $O(nlogn)$

18.关于网络分层结构,下列说法正确的是(    ). \\
A.某一层可以使用其上一层提供的服务而不需知道服务是如何实现的 \\
B.层次划分越多,灵活性越好,协议效率也越高 \\
C.由于结构彼此分离,实现和维护更加困难 \\
D.当某一层发生变化时,只要接口关系不变,以上或以下的各层均不受影响

19.不受电磁干扰或噪声影响的介质是(    ). \\
A.双绞线 $\qquad$ B.光纤 $\qquad$ C.同轴电缆 $\qquad$ D.微波

20.要在带宽为4kHz的信道上用2秒钟发送80kb的数据块,按照香农定理,信道的信噪比最小应为多少? (    ) \\
A.1023 $\qquad$ B.200 $\qquad$ C.1005 $\qquad$ D.600

21.若数据链路层采用选择重传协议,发送方已经发送了编号为0~6的帧.当计时器超时时,只有5号帧的确认还没有返回,则发送方需要重发的帧数为(    ). \\
A.1 $\qquad$ B.2 $\qquad$ C.3 $\qquad$ D.7

22.下面技术无法使10Mbps以太网升级到100Mbps和1Gbps的是( ). \\
A.采用帧扩展和帧突发技术 \\
B.传输介质使用高速光纤 \\
C.帧长保持不变,网络跨距增加 \\
D.使用以太网交换机,引入全双工流量控制协议

23.位于不同子网中的主机之间进行相互通信,下面说法中正确的是( ). \\
A.源站点可以直接进行ARP广播得到目的站的硬件地址 \\
B.路由器在转发IP数据报时,重新封装源硬件地址和目的硬件地址 \\
C.路由器在转发IP数据报时,重新封装目的IP地址和目的IP硬件地址 \\
D.路由器在转发IP数据报时,重新封装源IP地址和目的IP地址

24.在TCP/IP网络上,主机及主机上运行的程序可以用( ) 来标识. \\
A. IP地址,MAC地址 $\qquad$ B.端口号,IP 地址 \\
C. IP 地址,主机地址 $\qquad$ D. IP 地址,端口号

25.标准的URL组成:服务器类型、主机名和路径及( ). \\
A.文件名 $\qquad$ B.浏览器 \\
C.客户名 $\qquad$ D.进程名

\subsection{二、综合应用题}
26~34小题,共100分.

26. (10分) 现有一个解决无向连通图的最小生成树的一种方法如下: \\
\begin{lstlisting}[language=cpp]
将图中所有边按权重从大到小排序为(e1,e2,...,em);
i=1;
while (所剩边数>=顶点数) {
    从图中删去ei;
    若图不再连通,则恢复ei;
    i=i+1;
}
\end{lstlisting}
请问上述方法能否求得原图的最小生成树?若该方法可行,请证明之;否则请举例说明.

27. (10分)采用散列函数$H(k)=3\times k \quad MOD \quad 13$并用线性探测开放地址法处理冲突,在散列地址空间$[0..12]$中对关键字序列$22$, $41$, $53$, $46$, $30$, $13$, $1$, $67$, $51$; \\
(1)构造散列表(画示意图); \\
(2)装填因子; \\
(3)等概率情况下查找成功的平均查找长度; \\
(4)等概率情况下查找失败的平均查找长度.

28. (12 分)已知数组$A[1..n]$的元素类型为整型$int$, 设计一个时间和空间上尽可能高效的算法,将其调整为左右两部分,左边所有元素为负整数,右边所有元素为正整数.不要求对这些元素排序. \\
(1)给出算法的基本设计思想; \\
(2)根据设计思想,采用$C$或$C++$语言表述算法,关键之处给出注释; \\
(3)说明你所设计算法的时间复杂度和空间复杂度.

29. (12 分)已知带头结点的单链表$H$,写一算法将其数据结点逆序链接,即线性表$(a_1,..,a_n)$逆置为$(a_n,..,a_1)$. \\
(1)给出算法的基本设计思想. \\
(2)根据设计思想,采用$C$或$C++$语言描述算法,关键之处给出注释.

30. (12 分)假设二叉树采用二叉链表存储结构存储,设计一个算法,利用结点的右孩子指针$rchild$将一棵二叉树的叶子结点按从左往右的顺序串成一个单链表. \\
(1)给出算法的基本设计思想. \\
(2)根据设计思想,采用$C$或$C++$语言描述算法,关键之处给出注释.

31. (8分)若构造一个CSMACD总线网,速率为100Mbps,电缆总长度1km,中间用一个中继器连接,电缆中的信号传播速度是200000km/s,信号经过中继器产生2Hs时延.试求出数据帧的最小长度.

32. (14 分)一个公司有两个部门,销售部和财务部,销售部有28台PC,财务部.有15台PC.现在,公司申请了一个C类地址221.156.18.0,规划的网络拓扑如下图所示,试解答如下问题. \\
(1)给出合理的子网规划,并说明理由. \\
(2)为两个部门各分配一个子网地址,并为两个路由器的接口和各台PC分配IP地址. \\
(3)如果路由器R1和R2都采用了RIP作为路由选择协议,请给出当稳定运行之后,R1和R2的路由表.
\begin{figure}[ht]
\centering
\includegraphics[width=12cm]{./figures/DNCC14_1.png}
\caption{第32题图} \label{DNCC14_fig1}
\end{figure}

33. (12分)假设主机A已向主机B发送了一个序号为70的报文段,并己在超时计数器超时之前收到了主机B的确认.现主机A向主机B又连续发送了两个TCP报文段p、q,其序号分别为90、130.请回答以下问题,并写出解答过程. \\
(1) p报文段携带了多少字节的数据? \\
(2)主机B收到p报文段后发回的确认中的确认号应当是多少? \\
(3)如果主机B收到q报文段后发回的确认中的确认号是180, 试问A发送的q报文段中的数据有多少字节? \\
(4)如果A发送的p报文段丢失了,但q报文段到达了B. B在q报文段到达后向A发送确认.试问这个确认号应为多少?

34.(10分)基于万维网的电子邮件系统有什么特点?在传送邮件时使用什么协议?