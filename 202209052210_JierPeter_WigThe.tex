% Wigner基本定理
% 对称表示定理|量子力学|射线空间|ray space|对称变换


\pentry{量子力学的基本原理(量子力学)\upref{QMPrcp}}


%本文主要参考文献为\href{https://arxiv.org/abs/math/9808033}{AN ALGEBRAIC APPROACH TO WIGNER’S UNITARY-ANTIUNITARY THEOREM},作者LAJOS MOLNAR.


Wigner基本定理是说,量子态的对称变换一定能表示成希尔伯特空间中的对称算子或反对称算子.在Weinberg的量子场论第一卷\cite{WeinbergQFT1}中也将其称为“对称表示定理(symmetry representation theorem)”.

\subsection{定理描述}

量子力学认为量子态构成一个\textbf{希尔伯特空间}\upref{Hilber},其中互为倍数的量子态是等价的(\autoref{QMPrcp_def2}~\upref{QMPrcp}).为了方便,我们可以限定只讨论模为$1$的那些量子态,也可以说我们讨论的不是矢量,而是\textbf{射线(ray)}.当然,如果觉得射线的语言不好理解,也可以只考虑归一化的态矢,不影响对定理的理解和表述.

希尔伯特空间$H$中的射线构成$H$的一个\textbf{商空间}\footnote{见\autoref{Relat_sub1}~\upref{Relat}中的“商集”概念.}$\mathbb{P}H$,定义为${\displaystyle \mathbb {P} H=H\setminus \{0\}/{\approx }}$.其中等价关系$\approx$就是量子态等价的定义,即$\ket{s_1}\approx\ket{s_2}\iff \exists c\in\mathbb{C}, \ket{s_1}=c\ket{s_2}$.

量子态$\ket{s}$所在的射线,或者说能表示这一量子态的所有右矢的集合,记为$\ket{\bar{s}}$\footnote{Wikipedia的\href{https://en.wikipedia.org/wiki/Wigner_theorem}{Wigner基本定理
}词条中,用波函数$\Psi$来表示一个右矢(笔者建议不要有这个坏习惯,波函数和右矢不等价.可以用$\ket{\Psi}$表示波函数$\Psi$对应的右矢,二者的关系为$\Psi=\braket{x}{\Psi}$,其中$\ket{x}$是位置算符的本征右矢.),而把$\Psi$所在的射线记为$\underline{\Psi}$.}.射线的“内积”由归一化矢量定义:
\begin{equation}
\braket{\bar{s}_1}{\bar{s}_2} = \frac{\braket{s_1}{s_2}}{\sqrt{\braket{s_1}{s_1}\braket{s_2}{s_2}}}
\end{equation}

有了上述概念,就可以定义\textbf{对称变换}:

\begin{definition}{对称变换}
给定希尔伯特空间$H$及其射线空间$\mathbb{P}H$,如果映射$f:\mathbb{P}H\to\mathbb{P}H$满足:对于任意$\ket{\bar{s}_i}\in\mathbb{P}H$,都有
\begin{equation}
\abs{\braket{\bar{s}_1}{\bar{s}_2}}^2 = \abs{\braket{f(\bar{s}_1)}{f(\bar{s}_2)}}^2
\end{equation}
那么称$f$是一个\textbf{对称变换(symmetry transformation)}.
\end{definition}

注意对称变换的名称,“对称”是一个名词,而非形容词.

显然,对称变换是保\textbf{跃迁概率}的,因为两个态之间的跃迁振幅定义正是其内积,而概率是振幅的模方.

有了对称变换的概念,就可以表述Wigner基本定理了.

\begin{theorem}{Wigber基本定理}
任何对称变换都可以表示成物理态Hilbert空间上的线性算符$\mathcal{Q}$,且这个算符要么是线性且幺正的:
\begin{equation}
\ali{
    \mathcal{Q}(\xi\ket{a}+\eta\ket{b}) &= \xi\mathcal{Q}\ket{a}+\eta\mathcal{Q}\ket{b}\\
    \bra{a}\mathcal{Q}^\dagger\mathcal{Q}\ket{b} &= \braket{a}{b}
}
\end{equation}
要么是共轭线性且反幺正的:
\begin{equation}
\ali{
    \mathcal{Q}(\xi\ket{a}+\eta\ket{b}) &= \xi^*\mathcal{Q}\ket{a}+\eta^*\mathcal{Q}\ket{b}\\
    \bra{a}\mathcal{Q}^\dagger\mathcal{Q}\ket{b} &= \braket{a}{b}^*
}
\end{equation}

\end{theorem}



\subsection{定理证明}

考虑态空间$H$中的一个归一化正交完备集$\{\ket{s_\alpha}\}$,即$\braket{s_\alpha}{s_\alpha}=1$,$\braket{s_\alpha}{s_\beta}=\delta_{\alpha\beta}$,且任意态矢量都可以表示成$\sum_\alpha c_\alpha\ket{s_\alpha}$(离散情况)或$\int c(\alpha)$























