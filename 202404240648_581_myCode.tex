% 小时百科程序框架简介
% license Xiao
% type Tutor

网页版百科分为两个界面:
\item \textbf{原界面} \verb|wuli.wiki/online| 由静态 html 网页构成,每个文章的网址如 \verb|wuli.wiki/online/AU.html| 其中 AU 是文章的 id, 每个文章都有唯一的 id, 即使文章被重命名也不会变。  \textbf{新界面} \verb|wuli.wiki/book| 相当于在原界面上套壳,每天准点爬取 \verb|wuli.wiki/online/| 中的内容进行更新, 增加了评论,搜索等功能, 目录也更美观。 新界面中文章的网址如 \verb|wuli.wiki/book/AU|。
文章编辑器 \verb|wuli.wiki/editor| 是一个相对独立的模块,百科文章的作者使用编辑器用 LaTeX 语言写作, 编辑器把 LaTeX 源码转为 \verb|wuli.wiki/online/| 目录中的 html 页面。

所有代码通过 \href{https://github.com/wuliwiki}{GitHub 账号}管理。 以下是各个仓库的功能
\begin{itemize}
\item \href{https://github.com/MacroUniverse/littleshi.cn}{littleshi.cn}: 网站中所有静态页面, 服务器路径 \verb|/var/www/littleshi.cn|, 其中的 \verb`online` 子目录就是所有文章的静态 html 放置的地方, 通过 url \verb`wuli.wiki/online/xxx.html` 直接访问。
\item \href{https://github.com/MacroUniverse/PhysWiki}{PhysWiki}: 百科文章的源代码, 使用 LaTeX 语言, 支持用 TeXlive 直接编译 pdf, 服务器路径 \verb|/var/www/PhysWiki|。 编辑器 \verb`wuli.wiki/editor` 编辑文章后, 把 LaTeX 源码保存到 \verb`PhysWiki/contents/xxx.tex` 文件, 其中 \verb`xxx` 就是每个文章的 id。 然后编辑器调用一个 C++ 命令行程序 \verb`PhysWikiScan` 把 tex 文件转换为 \verb`online` 目录的 html 文件。
\item \href{https://github.com/MacroUniverse/PhysWikiScan}{PhysWikiScan}: 一个后台 C++ 命令行程序, 用于把 PhysWiki 中的 LaTeX 源码文件转为 littleshi.cn 中的静态的百科 html 页面, 可以单个转换也可以批量转换。服务器路径 \verb|/var/www/PhysWikiScan|
\item \href{https://github.com/MacroUniverse/littleshi.cn-server}{littleshi.cn-server}: 百科编辑器 \verb`wuli.wiki/editor` 的源码, 调用 PhysWikiScan 程序。 服务器路径 \verb|/var/www/editor|, 使用 node.js 开发, 已封装到 docker 中运行。
\item \href{https://github.com/MacroUniverse/PhysWiki-backup}{PhysWiki-backup}: 用于存放百科文章备份,每个文章页面根据该备份记录给作者排序。服务器路径 \verb|/var/www/PhysWiki-backup|
\item \href{https://github.com/MacroUniverse/wuliwiki-web}{wuliwiki-web}:网站中除了文章编辑器外的所有动态页面,以及后台。 当前在服务器中用 docker 运行。
\item \href{https://github.com/MacroUniverse/wuliwiki-app}{wuliwiki-app}: 百科移动端 app, 用 flutter 开发, 支持安卓、iPhone、iPad。 经费不足已经停止开发。
\end{itemize}

\subsection{框架}
\begin{itemize}
\item 编辑器和论坛前端都是 React 的 MUI
\item 后端 Python/Django/Redis/PstgreSQL/RESTful/RabbitMQ
\end{itemize}
