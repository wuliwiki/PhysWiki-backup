% LLVM IR 笔记
% license Xiao
% type Note

\begin{issues}
\issueDraft
\end{issues}

\pentry{LLVM 笔记\upref{LLVMnt}}

\textbf{LLVM IR (Intermediate Representation)} 是 LLVM 的一个 portable 的中间语言。 使用 LLVM 的编译器会先把高级语言编译成 IR, 然后再做进一步优化。 LLVM IR 的语法和汇编语言类似。

一个完整的例子, 可以编译运行
\begin{lstlisting}[language=none,caption=test.ll]
; Declare the main function
define i32 @main() {
    ; Allocate stack space for a single 32-bit integer in the stack memory
    ; `var` is a pointer
    %var = alloca i32

    ; Store the value 42 into the variable on memory stack
    store i32 42, i32* %var

    ; Load the value of the variable into a [virtual] register
    %val = load i32, i32* %var

    ; Return the value of the register
    ret i32 %val
}
\end{lstlisting}

编译:
\begin{lstlisting}[language=bash]
llvm-as test.ll -o test.bc # 编译成 bitcode
clang test.bc -o test # 编译成可执行文件
./test # 执行
echo $? # 查看返回值(exit status)
\end{lstlisting}

\begin{itemize}
\item \textbf{Static Single Assignment (SSA)}:每个变量只能赋值一次!
\item  在使用一个变量之前, 通常需要把它加载到 register 中, 而不是直接 dereference。
\item 例如算数以及 return 都需要先加载到 register 中.
\end{itemize}

以下是一个函数例子: 计算两个整数相加。
\begin{lstlisting}[language=none]
define i32 @add(i32 %a, i32 %b) {
  %sum = add i32 %a, %b
  ret i32 %sum
}
\end{lstlisting}

调用该函数
\begin{lstlisting}[language=none]
define i32 @main() {
    ; set a=10, b=20 in registers
    %a = add i32 0, 10
    %b = add i32 0, 20
    %sum = call i32 @add(i32 %a, i32 %b)
    ret i32 %sum
}
\end{lstlisting}

其中对 \verb`a, b` 的赋值也可以用
\begin{lstlisting}[language=none]
%a = or i32 10, 0
%b = or i32 20, 0
\end{lstlisting}

第三种方法是从内存 stack 中加载到 register
\begin{lstlisting}[language=none]
define i32 @main() {
    %a = alloca i32
    %b = alloca i32
    store i32 10, i32* %a
    store i32 20, i32* %b
    %a_val = load i32, i32* %a
    %b_val = load i32, i32* %b
    %sum = call i32 @add(i32 %a_val, i32 %b_val)
    ret i32 %sum
}
\end{lstlisting}

\subsection{判断}
\begin{itemize}
\item \verb`br i1 <cond>, label <iftrue>, label <iffalse>` 用于判断, \verb`i1` 是一个 1bit 整数,也就是 bool。 如果条件为真, 就跳到标签 \verb`<iftrue>`, 否则跳到 \verb`<iffalse>`
\item 例如, \verb`%cond = icmp eq i32 %x, %y`, 然后 \verb`br i1 %cond, label %equals, label %notequals`
\end{itemize}


\subsection{循环}
\begin{lstlisting}[language=none]
define void @countUpTo(i32 %limit) {
entry:
    ; Initialize loop variable
    %i = alloca i32
    store i32 0, i32* %i

    ; Jump to the loop condition check
    br label %loop.cond

loop.cond:
    ; Load loop variable
    %i.val = load i32, i32* %i

    ; Check if the loop variable is less than the limit
    %cond = icmp slt i32 %i.val, %limit
    br i1 %cond, label %loop.body, label %loop.end

loop.body:
    ; Increment the loop variable
    %next.val = add i32 %i.val, 1
    store i32 %next.val, i32* %i

    ; Jump back to the loop condition
    br label %loop.cond

loop.end:
    ; Exit point of the loop
    ret void
}
\end{lstlisting}
