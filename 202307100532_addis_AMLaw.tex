% 角动量定理、角动量守恒
% 角动量|角动量定理|角动量守恒|力矩|牛顿第二定律

\pentry{动量定理 动量守恒\upref{PLaw}, 系统的角动量\upref{AngMom}}
\textbf{角动量定理}可以表示为
\begin{equation}\label{eq_AMLaw_1}
\bvec \tau = \dv{\bvec L}{t}~,
\end{equation}
即系统总角动量(矢量) $\bvec L$ 对时间的变化率等于所有外力矩\upref{Torque}的矢量和(\textbf{合外力矩}) $\bvec \tau$。角动量定理可以类比动量定理\upref{PLaw}, 其中角动量与系统动量 $\bvec p$ 对应, 合外力矩与合外力 $\bvec F$ 对应。 注意一般情况下, $\bvec L$ 和 $\bvec \tau$ 的方向不一定相同, 只有在例如刚体绕固定轴转动\upref{RigRot}的特殊情况时, 二者才相同。 \autoref{eq_AMLaw_1} 的证明见下文。

作为一个简单的情况, 我们来看刚体的定轴转动\upref{RigRot}。 此时刚体的转动惯量以及合外力矩的方向都固定的转轴平行\footnote{这可以类比质点做直线运动的动量定理\upref{PLaw}, 动量和力都沿同一方向。}, 于是我们可以规定一个正方向, 把 $\bvec L, \bvec \tau$ 用实数表示: 当实数为正, 则矢量指向正方向, 反之则指向反方向。

然而一般情况下, $\bvec L$ 和 $\bvec \tau$ 的方向都可以随时间改变, 且二者方向未必相同。 这可以类比对质点的曲线运动(如圆周运动)使用动量定理: 加速度不仅取决于速度大小的变化, 还取决于速度方向的变化; 动量和力的方向也未必相同。
\begin{example}{陀螺的进动}\label{ex_AMLaw_2}
\begin{figure}[ht]
\centering
\includegraphics[width=8.5cm]{./figures/6734a2ae910afd2c.pdf}
\caption{陀螺的进动}\label{fig_AMLaw_2}
\end{figure}

如\autoref{fig_AMLaw_2} (左), 陀螺旋转时, 若它的轴与竖直方向有一定倾角, 轴会绕一个竖直轴缓慢旋转, 这种现象被称为\textbf{进动(precession)}。 为了便于分析, 我们先假设陀螺进动的角速度比陀螺自转的角速度要慢得多。 这样, 我们就可以认为陀螺的角动量 $\bvec L$ 与陀螺的轴平行。

陀螺在进动过程中, 角动量大小不变, 但方向不断变化, 所以角动量变化率 $\dv*{\bvec L}{\bvec t}$ 不为零。 令 $\bvec L$ 的起点为原点, $\bvec L$ 末端在圆形轨迹上运动。 $\dv*{\bvec L}{\bvec t}$ 的方向始终沿着该圆形轨迹的切线方向。 根据角动量定理, 陀螺所受的力矩 $\bvec \tau$ 也具有同样的大小和方向。

那么这个力矩是如何产生的呢? 我们对陀螺进行受力分析如\autoref{fig_AMLaw_2} (右), 要计算陀螺所受力矩, 我们取轴的底端为原点, 假设陀螺的轴没有质量, 则地面对陀螺的支持力 $N$ 产生的力矩为零, 而重力产生的力矩为 $\bvec \tau = \bvec r_0 \cross (m\bvec g)$, 其大小为 $mgr_0\sin\theta$, 方向垂直纸面向里, 恰好符合陀螺进动的要求。

比较违反直觉的地方在于, 陀螺受到的重力是延使陀螺倾倒的方向施加的, 然而陀螺不但丝毫不会倾倒(如果不计摩擦), 反而其重心会向着与重力垂直的方向移动。 要具体计算陀螺进动的快慢, 我们还需要知道角动量和陀螺自转的角速度的关系, 见\autoref{exe_RigRot_1}~\upref{RigRot}。

答案:
\begin{equation}
\Omega = \frac{mgr_0}{I\omega}~.
\end{equation}
\end{example}

\subsection{角动量守恒}
当\autoref{eq_AMLaw_1} 中系统受合外力矩为零时, 角动量变化率为零, 即系统总角动量不随时间变化。 这时我们说该系统\textbf{角动量守恒(conservation of angular momentum)}。

\begin{example}{面团碰撞}
在没有引力的太空中, 两个面团在各自质心系中的角动量分别为 $\bvec L_1$ 和 $\bvec L_2$。 他们的质心在某惯性系 $S$ 的 $x$ 轴上相向运动然后相撞并融为一体, 求碰撞后的角动量。

解: 根据角动量的质心系分解(\autoref{eq_AngMom_3}~\upref{AngMom}), 在惯性系 $S$ 中他们的角动量仍然为 $\bvec L_1$ 和 $\bvec L_2$。 把两个面团看成一个系统, 总角动量为 $\bvec L = \bvec L_1 + \bvec L_2$。 由于不受任何外力(矩), 根据角动量守恒, 相撞后总角动量仍然为 $\bvec L$。
\end{example}

\begin{exercise}{面团碰撞 2}
在没有引力的太空中, 质量为 $m_1, m_2$ 的两个面团在各自质心系中的角动量分别为 $\bvec L_1$ 和 $\bvec L_2$。 他们的质心分别在某惯性系 $S$ 的 $x$ 轴和 $y = 1$ 直线上运动, $x$ 方向速度分别为 $v_1, v_2$。 若它们相撞并融为一体, 求碰撞后的角动量。
\end{exercise}

\subsection{角动量分量守恒}
在一些情况下, 我们不能完全保证合外力矩为零, 而只能得出合外力矩在某个方向 $\uvec e$ 的投影(即分量)为零
\begin{equation}
\bvec \tau \vdot \uvec e \equiv 0~.
\end{equation}
把\autoref{eq_AMLaw_1} 两边同时点乘 $\uvec e$, 得
\begin{equation}
\dv{t} (\bvec L \vdot \bvec e) = 0~.
\end{equation}
这就说明 $\bvec L$ 在 $\uvec e$ 方向得投影不随时间变化, 即角动量分量守恒。

角动量分量守恒最常见的例子是刚体绕固定轴的无摩擦转动\upref{RigRot}: 若不施加额外的力矩, 轴只可能给刚体施加垂直轴方向的力矩, 所以刚体在平行轴方向的角动量分量守恒。

\begin{example}{单车轮与转椅实验}
小明开始时坐在静止的无摩擦转椅上, 两手握住一个单车轮的轴的两端, 单车轮在水平面上转动。 这时小明将单车轮上下翻转(仍保持转动), 问小明与转椅会如何转动?

假设开始时车轮的角动量向上, 那么翻转后车轮的角动量向下, 即角动量增量向下。 由于竖直方向的角动量分量守恒,小明的身体和转椅的角动量必须有一个向上的增量, 所以转椅最后的旋转方向与轮子开始时的旋转方向相同。
\end{example}

\begin{exercise}{}
若陀螺上有两个相同的转盘逆向旋转,陀螺是否能保持平衡?
\end{exercise}

\subsection{证明角动量定理}
证明可类比系统的动量定理\upref{PLaw}。 我们已经知道单个质点的角动量,而任何物体都可以划分成若干足够小的微元,每个微元可以看成一个质点。令第 $i$ 个质点的位矢为 $\bvec r_i$, 角动量为 $\bvec L_i$,力矩为 $\bvec \tau_i$,单个质点的角动量定理\upref{AMLaw1} 为
\begin{equation}
\dv{\bvec L_i}{t} = \bvec \tau_i = \bvec \tau_i^{in} + \bvec \tau_i^{out}~,
\end{equation}
其中 $\bvec \tau_i^{in}$ 和 $\bvec \tau_i^{out}$ 为质点 $i$ 受到的系统内所有其他质点的力矩和来自系统外的所有力矩。将该式对所有 $i$ 求和,得到总角动量 $\bvec L$ 变化率
\begin{equation}
\dv{\bvec L}{t} =\sum_i\dv{\bvec L_i}{t} = \sum_i\bvec \tau_i^{in} + \sum_i\bvec \tau_i^{out}~.
\end{equation}
现在我们只需证明质点系的合内力矩为零即可
\begin{equation}
\sum_i\bvec \tau_i^{in} = \sum_i \qty(\bvec r_i\cross\sum_j^{j\ne i}\bvec F_{j\to i}) = \sum_{i,j}^{i\ne j} \bvec r_i\cross\bvec F_{j\to i}~,
\end{equation}
其中 $\bvec F_{j\to i}$ 是质点 $j$ 对质点 $i$ 的力。现在只考虑任意两个质点 $k$ 和 $l$,在求和中的贡献为
\begin{equation}
\bvec r_k\cross\bvec F_{l\to k} + \bvec r_l\cross\bvec F_{k\to l} \equiv \bvec \tau_{l\to k}+ \bvec \tau_{k\to l}~,
\end{equation}
即 $k$ 对 $l$ 的力矩加 $l$ 对 $k$ 的力矩(两质点的和内力矩)。所以若能证明任意两质点的和内力矩为零,则质点系的合内力矩为零。

我们先来看几何证明。如\autoref{fig_AMLaw_1}, 根据定义, 力矩的大小等于力的模长乘以力臂的长度\upref{Torque}, 而一对相互作用力的大小相同, 又由于二者共线, 力臂也重合, 所以两个力矩大小相等。 但是两个力矩的方向一个是顺时针(指向纸内), 一个是逆时针(指向纸外), 所以两力矩互相抵消, 相加为零。

\begin{figure}[ht]
\centering
\includegraphics[width=4.5cm]{./figures/07fe87f671f76c96.pdf}
\caption{两质点的相互作用力对总力矩贡献为零}\label{fig_AMLaw_1}
\end{figure}

再在看代数的方法:我们先沿着两质点的连线写出相互作用力 $\bvec F_{l\to k} = \alpha(\bvec r_k - \bvec r_l)$, $\bvec F_{k\to l} = \alpha(\bvec r_l - \bvec r_k)$, 其中 $\alpha$ 是一个常数。 直接计算两力矩和得
\begin{equation}
\bvec r_k \cross (\bvec r_k - \bvec r_l)\alpha + \bvec r_l \cross (\bvec r_l - \bvec r_k)\alpha = 0~.
\end{equation}
证毕。
