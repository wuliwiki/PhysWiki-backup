% x86-64 笔记

\begin{issues}
\issueDraft
\end{issues}

参考 Wikipedia \href{https://en.wikipedia.org/wiki/X86-64}{相关页面}。

x86 \textbf{指令集(instruction set)}, 或者\textbf{指令集架构(ISA, instruction set architecture)}, 分为 32 位和 64 位两个版本。

现在最常见的是 64 位, 称为 x86-64 (也叫 x64, x86_64, AMD64, 和 Intel 64)

比较老的 32 位版本, 称为 IA-32 (Intel Architecture, 32-bit, 也叫 i386)


The x86-64 architecture provides 16 general-purpose registers, each of which is 64 bits wide. The 16 registers are named as follows:

RAX, RBX, RCX, RDX: General-purpose registers for arithmetic and data manipulation
RSI, RDI, RBP, RSP: General-purpose registers for storing data and addressing memory
R8-R15: Additional general-purpose registers that can be used for arithmetic and data manipulation