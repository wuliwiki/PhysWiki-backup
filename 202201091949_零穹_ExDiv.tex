% 多项式的整除
% 因式|倍式|整除

\begin{issues}
\issueTODO
\end{issues}

\pentry{带余除法\upref{DivAlg}}
有了带余除法\upref{DivAlg} ,我们自然会想到像整数一样来定义多项式的整除、因式、倍式等概念.同时,带余除法还给出了多项式整除性的一个判别法.
\begin{definition}{整除、因式、倍式}\label{ExDiv_def1}
设 $f(x),g(x)$ 为数域 $\mathbb{F}$ 上的多项式,若存在数域 $\mathbb{F}$ 上的多项式 $q(x)$ 使得 $f(x)=q(x)g(x)$,则称 $g(x)$ \textbf{整除} $f(x)$,记作 $g(x)|f(x)$,并称 $g(x)$ 为 $f(x)$ 的\textbf{因式},$f(x)$ 为 $g(x)$ 的\textbf{倍式}.
\end{definition}
由\autoref{ExDiv_def1} ,容易验证:
\begin{enumerate}
\item 任意多项式一定整除自身:$f(x)|f(x)$;
\item 零多项式只整除零多项式:$0|0$;
\item 任意一个非零多项式一定整除零多项式:$f(x)|0$;
\item 零次多项式整除任意多项式:$c|f(x)$,其中 $c\neq0\in\mathbb{F}$.
\end{enumerate}
根据这4条性质,规定:零次多项式,以及 $f(x)$ 的与其次数相同的因式,称为 $f(x)$ 的\textbf{平凡因式};零多项式,以及 $f(x)$的与其次数相同的倍式,称为 $f(x)$ 的\textbf{平凡倍式}.

有了整除的概念,考虑到带余除法,自然容易想到,两个多项式能够整除是其余式为0.因此,下面定理是显然的
\begin{theorem}{}
设 $f(x),g(x)$ 为数域 $\mathbb{F}$ 上的两个多项式,$g(x)\neq 0$ ,则 $g(x)|f(x)$ 的充要条件是 $g(x)$ 除 $f(x)$ 的余式为0,即存在多项式 $q(x)$ 使得 $f(x)=q(x)g(x)$.
\end{theorem}
\begin{definition}{多项式的商}
若 $g(x)|f(x)$,则 $g(x)$ 除 $f(x)$ 的\textbf{商} $q(x)$ 可表示 
\begin{equation}
q(x)=\frac{f(x)}{g(x)}
\end{equation}

\end{definition}