% 艾兹赫尔·戴克斯特拉(综述)
% license CCBYSA3
% type Wiki

本文根据 CC-BY-SA 协议转载翻译自维基百科\href{https://en.wikipedia.org/wiki/Edsger_W._Dijkstra}{相关文章}。

\begin{figure}[ht]
\centering
\includegraphics[width=6cm]{./figures/b98db333d3a6a579.png}
\caption{2002年的Dijkstra。} \label{fig_ACHR_1}
\end{figure}
Edsger Wybe Dijkstra(/ˈdaɪkstrə/ DYKE-strə;荷兰语:[ˈɛtsxər ˈʋibə ˈdɛikstraː] ⓘ;1930年5月11日-2002年8月6日)是荷兰计算机科学家、程序员、软件工程师、数学家和科学散文家。

Dijkstra出生于荷兰鹿特丹,他学习了数学和物理学,之后在莱顿大学攻读理论物理学。阿德里安·范·温格尔丹(Adriaan van Wijngaarden)曾邀请他担任荷兰第一位计算机程序员,在阿姆斯特丹的数学中心工作,任职时间为1952年到1962年。他在1956年提出并解决了最短路径问题,并于1960年与同事贾普·A·宗内费尔德(Jaap A. Zonneveld)共同开发了编程语言ALGOL 60的首个编译器。1962年,他搬到埃因霍温,随后又移居到纽嫩,在埃因霍温技术大学数学系担任教授。20世纪60年代末,他构建了THE多任务操作系统,该系统通过使用基于软件的分页虚拟内存,影响了后续操作系统的设计。1973年8月,Dijkstra加入了Burroughs公司,成为唯一的研究员。在Burroughs的这段时间里,他的研究文章产出最为丰盛,撰写了近500篇《EWD》系列文件,其中大部分为技术报告,供特定小组私下传阅。

1984年,Dijkstra接受了德克萨斯大学奥斯汀分校计算机科学系的施伦伯杰百年讲席,并在奥斯汀工作,直到1999年11月退休。退休后,他和妻子返回原住地纽嫩,直至2002年8月6日因长期与癌症斗争去世。

Dijkstra因其在开发结构化编程语言方面的基础性贡献而获得了1972年的图灵奖。临终前,他因在程序计算自稳定性方面的研究获得了ACM PODC影响力论文奖,这项奖项在次年以“Dijkstra奖”命名,以纪念他。