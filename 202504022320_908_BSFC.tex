% 泊松方程(综述)
% license CCBYSA3
% type Wiki

本文根据 CC-BY-SA 协议转载翻译自维基百科\href{https://en.wikipedia.org/wiki/Poisson\%27s_equation}{相关文章}。

泊松方程是一个在理论物理中广泛应用的椭圆型偏微分方程。例如,泊松方程的解是由给定的电荷或质量密度分布引起的势场;一旦知道了势场,就可以计算出相应的静电或引力(力)场。它是拉普拉斯方程的推广,后者在物理学中也经常出现。该方程以法国数学家和物理学家西蒙·丹尼斯·泊松的名字命名,泊松于1823年发布了这个方程。\(^\text{[1][2]}\)
\subsection{方程的表述}  
泊松方程是  
\[
\Delta \varphi = f,~
\]
其中,\(\Delta\)是拉普拉斯算子,\(f\)和\(\varphi\)是流形上的实值或复值函数。通常,给定\(f\),而求解\(\varphi\)。当流形是欧几里得空间时,拉普拉斯算子通常表示为\(\nabla^2\),因此泊松方程通常写作  
\[
\nabla^2 \varphi = f.~
\]
在三维笛卡尔坐标系中,它的形式为  
\[
\left( \frac{\partial^2}{\partial x^2} + \frac{\partial^2}{\partial y^2} + \frac{\partial^2}{\partial z^2} \right) \varphi (x, y, z) = f (x, y, z).~
\]
当\(f = 0\)恒成立时,我们得到拉普拉斯方程。
泊松方程可以通过格林函数求解:  
\[
\varphi(\mathbf{r}) = - \iiint \frac{f(\mathbf{r'})}{4 \pi |\mathbf{r} - \mathbf{r'}|} \, \mathrm{d}^3 r',~
\]
其中积分是对整个空间进行的。泊松方程的格林函数的详细说明见于筛选泊松方程的相关文章。还有多种数值求解方法,如松弛法,这是一种迭代算法。