% 如何给文件加密(含 python 加密脚本)
% license Usr
% type Tutor

\pentry{Python 基础\upref{Py}}

以下给出一些常用的 python 加密函数,可以加密单个文件,加密字符串,加密一个文件夹中所有文件/文件名。 注意该脚本只能在 linux 环境使用(因为需要调用命令行的 \verb`openssl`)。在 Windows 中可以通过 WSL\upref{WSLnt} 中使用该脚本。 事实上也可以用 \verb|pyOpenSSL| 模块来增加可移植性。

加密算法为 \verb`aes-256-cbc`, \verb`-nosalt` 意味着同样的数据和密码生成的加密文件一摸一样(以支持网盘的 “秒传” 功能)。若去掉该选项可能会让加密变得更安全一些。

\begin{lstlisting}[language=python,caption=encrypt.py]
import os
import sys
import random
import subprocess
import shutil

file_extension = '.enc'

# encrypt a file
def encrypt(file_to_encrypt, encrypted_file, password):
	command = [
		'openssl', 'aes-256-cbc', '-nosalt', '-pbkdf2',
		'-in', file_to_encrypt,
		'-out', encrypted_file,
		'-pass', 'pass:' + password
	]
	result = subprocess.run(command)
	if result.returncode != 0:
		print("encryption failed!")
		raise RuntimeError('encryption failed!')

# decrypt a file
def decrypt(file_to_decrypt, decrypted_file, password):
	command = [
		'openssl', 'aes-256-cbc', '-nosalt', '-pbkdf2', '-d',
		'-in', file_to_decrypt,
		'-out', decrypted_file,
		'-pass', 'pass:' + password
	]
	result = subprocess.run(command)
	if result.returncode != 0:
		print("encryption failed!")
		raise RuntimeError('decryption failed!')

# encrypt a string to base64 (no `\n`, with `=` padding)
def encrypt_str(string_to_encrypt, password):
	command = [
		'openssl', 'enc', '-base64', '-A', '-e', '-aes-256-cbc',
		'-nosalt', '-pbkdf2', '-pass', 'pass:' + password
	]
	result = subprocess.run(command, input=string_to_encrypt, text=True,
						 stdout=subprocess.PIPE, stderr=subprocess.PIPE)
	if result.returncode != 0:
		raise RuntimeError('encryption failed!')
	return result.stdout

# decrypt a base64 string (no `\n`, with `=` padding)
def decrypt_str(string_to_decrypt, password):
	command = [
		'openssl', 'enc', '-base64', '-A', '-d', '-aes-256-cbc',
		'-nosalt', '-pbkdf2', '-pass', 'pass:' + password
	]
	result = subprocess.run(command, input=string_to_decrypt+'\n', text=True,
						 stdout=subprocess.PIPE, stderr=subprocess.PIPE)
	if result.returncode != 0:
		raise RuntimeError('decryption failed!')
	return result.stdout

# encrypt files/folders in folder to a new folder named `prefix + directory`
def encrypt_files_in_folder(directory, prefix, password):
	for root, dirs, files in os.walk(directory, topdown=False):
		for name in files:
			path = os.path.join(root, name)
			root_new = prefix + root
			if not os.path.exists(root_new):
				os.makedirs(root_new)
			new_path = os.path.join(root_new, name)
			print(new_path)
			encrypt(path, new_path, password)
		for name in dirs:
			path = os.path.join(root, name)
			root_new = prefix + root
			new_path = os.path.join(root_new, name)
			print(new_path)
			if not os.path.exists(new_path):
				os.makedirs(new_path)

# decrypt files in folder to a new folder named `prefix + directory`
def decrypt_files_in_folder(directory, prefix, password):
	dic_file = 'enc-long-name-dic.txt'
	for root, dirs, files in os.walk(directory, topdown=False):
		for name in files:
			if (name == dic_file):
				continue
			path = os.path.join(root, name)
			root_new = prefix + root
			if not os.path.exists(root_new):
				os.makedirs(root_new)
			new_path = os.path.join(root_new, name)
			print(new_path)
			decrypt(path, new_path, password)
		for name in dirs:
			path = os.path.join(root, name)
			root_new = prefix + root
			new_path = os.path.join(root_new, name)
			print(new_path)
			if not os.path.exists(new_path):
				os.makedirs(new_path)

# encrypt names of files and subfolders inside a folder recursively (rename)
# will add `file_extension`, and skip files already with it
# for files with name too long, rename it to
#   `long-name-<id>.<file_extension>`,
# then use a `dic_file` to map to the actual
#   encrypted name `*.<file_extension>`
def encrypt_names_in_folder(directory, password):
	dic_file = 'enc-long-name-dic.txt'
	dic_path = os.path.join(directory, dic_file)
	file = open(dic_path, 'a')
	hex_chars = '0123456789abcdef'
	for root, dirs, files in os.walk(directory, topdown=False):
		for name in files:
			if (name[-4:] == file_extension or name == dic_file):
				continue
			path_old = os.path.join(root, name)
			name_new = encrypt_str(name, password).replace('/',
				'-') + file_extension
			if len(name_new) > 220: # windows filename max size is 224
				random_hex_string = ''.join(random.choice(hex_chars) for _ in range(32))
				name_short = 'long-name-' + random_hex_string
					+ file_extension
				file.write(name_short + ' ' + name_new + '\n')
				name_new = name_short
			print(path_old, ' -> ', name_new)
			os.rename(path_old, os.path.join(root, name_new))
		for name in dirs:
			if (name[-4:] == file_extension):
				continue
			path_old = os.path.join(root, name)
			name_new = encrypt_str(name, password).replace('/',
				'-') + file_extension
			if len(name_new) > 250:
				random_hex_string = ''.join(random.choice(hex_chars) for _ in range(32))
				name_short = 'long-name-' + random_hex_string
					+ file_extension
				file.write(name_short + ' ' + name_new + '\n')
				name_new = name_short
				
			print(path_old, ' -> ', name_new)
			os.rename(path_old, os.path.join(root, name_new))
	file.close()

# decrypt names of files and subfolders inside a folder recursively (rename)
# will only process files with extension file_extension
def decrypt_names_in_folder(directory, password):
	dic_file = 'enc-long-name-dic.txt'
	dic_path = os.path.join(directory, dic_file)
	long_names = {}
	if os.path.exists(dic_path):
		with open(dic_path, 'r') as file:
			for line in file:
				parts = line.strip().split(' ')
				long_names[parts[0]] = parts[1]
				# print('[' + parts[0] + '] -> [' + parts[1] + ']')
				
	for root, dirs, files in os.walk(directory, topdown=False):
		for name in files:
			if (name[-4:] != file_extension):
			 	continue
			path_old = os.path.join(root, name)
			if (name[:10] == 'long-name-'):
				try:
					name = long_names[name]
				except Exception as e:
					print('Error: enc-long-name-dic.txt key not found (will skip): ' + name)
					continue
			str64 = name[:-4].replace('-', '/')
			try:
				name_new = decrypt_str(str64, password)
			except Exception as e:
				print('Error: string decryption failed: ' + str64)
				continue
			print(path_old, ' -> ', name_new)
			os.rename(path_old, os.path.join(root, name_new))
		for name in dirs:
			if (name[-4:] != file_extension):
			 	continue
			path_old = os.path.join(root, name)
			if (name[:10] == 'long-name-'):
				try:
					name = long_names[name]
				except Exception as e:
					print('Error: enc-long-name-dic.txt key not found (will skip): ' + name)
					continue
			str64 = name[:-4].replace('-', '/')
			try:
				name_new = decrypt_str(str64, password)
			except Exception as e:
				print('Error: string decryption failed: ' + str64)
				continue
			print(path_old, ' -> ', name_new)
			os.rename(path_old, os.path.join(root, name_new))
	# if os.path.exists(dic_path):
		# os.remove(dic_path)

# encrypt names and data of files and subfolders inside a folder recursively
# and save to a new folder `prefix + directory`
def encrypt_folder(directory, prefix, password):
	encrypt_files_in_folder(directory, prefix, password)
	encrypt_names_in_folder(prefix + directory, password)

# decrypt names and data of files and subfolders inside a folder recursively
# and save to a new folder `prefix + directory`
def decrypt_folder(directory, prefix, password):
	decrypt_files_in_folder(directory, prefix, password)
	decrypt_names_in_folder(prefix + directory, password)
\end{lstlisting}
