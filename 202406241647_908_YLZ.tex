% 引力子
% license CCBYSA3
% type Wiki

(本文根据 CC-BY-SA 协议转载自原搜狗科学百科对英文维基百科的翻译)

\begin{figure}[ht]
\centering
\includegraphics[width=6cm]{./figures/285d0c18599e748d.png}
\caption{请添加图片标题} \label{fig_YLZ_1}
\end{figure}

在量子引力理论中,\textbf{引力子}是假设的关于重力的量子,一个调节重力的基本粒子。由于广义相对论的重归一化是一个突出的数学问题,因此没有引力子的完整量子场论。在弦理论中,引力子是基本弦的无质量状态,被认为是量子引力的一致理论。

如果引力子存在,它应该是无质量的,因为引力的作用范围很长,并且近似以光速传播。引力子一定是spin-2玻色子,因为引力的来源是应力-能量张量,一个二阶张量(与电磁的spin-2光子相比,其来源是四电流一阶张量)。此外,可以证明任何无质量的spin-2场都会产生与引力无法区分的力,因为无质量的spin-2场会像引力相互作用相同的方式耦合到应力-能量张量。这个结果表明,一个无质量的spin-2粒子一定是引力子。[1]

\subsection{理论}
据推测,重力相互作用是由一种尚未发现的基本粒子介导的,这种粒子被称为引力子。另外三个已知的自然力是由基本粒子介导的:光子的电磁作用,胶子的强相互作用以及了W、Z玻色子的弱相互作用。所有这三种力似乎都具有粒子物理学的标准模型。在经典极限条件下,一个成功的引力子理论可以简化为广义相对论,在弱场极限下,它可以简化为牛顿万有引力定律。[2][3][4]

引力子这个术语最初是由苏联物理学家Dmitrii Blokhintsev和F. Gal'perin于1934年创造的。[5]

\subsubsection{1.1 引力子和重正化}
当描述引力子相互作用时,费曼图的经典理论以及诸如one-loop图的半经典修正表现的非常平常。然而,具有至少两个回路的费曼图会导致紫外线发散。因为与量子电动力学和Yang-Mills理论模型不同,量子化的广义相对论不是摄动的,因而不能被微扰地重新归一化。因此,物理学家通过摄动方法计算粒子发射或吸收引力子的概率,得到了无法计算的答案,该理论失去了预测的准确性。这些问题和互补近似框架是证明需要一个比量子化广义相对论更统一的理论来描述在普朗克尺度附近行为的基础。

\subsubsection{1.2 与其他力的比较}
就像其他力的载流子一样,引力在广义相对论中起着决定事件发生的时空的作用。在某些描述中,能量改变了时空本身的“形状”,而重力是这种形状的结果,乍一看似乎很难与粒子之间作用的力相提并论。[6]由于该理论的亚纯不变性不允许将任何特定的时空背景选作“真实”时空背景,因此广义相对论被认为与背景无关。相反,标准模型不是与背景无关的,Minkowski空间作为固定的时空背景享有特殊地位。[7]为了调和这些差异,引入了一个量子引力理论。[8]这个理论是否应该独立于背景是一个悬而未决的问题。这个问题的答案将决定我们对引力在宇宙命运中所起的特定作用的理解。[9]

\subsubsection{纯理论中的引力子}
弦理论预测了引力子的存在及其明确定义的相互作用。扰动弦理论中的引力子是处于非常特殊的低能量振动状态的闭合弦。引力子在弦论中的散射可以根据AdS / CFT的对应关系进行分析,根据保形场论中的相关函数来计算,也可以根据矩阵论来计算。

弦理论中引力子的一个特征是,作为没有端点的闭合弦,它们不会束缚在膜上,可以在它们之间自由移动。如果我们生活在一个膜上(正如膜理论所假设的那样),引力子从膜“泄漏”到高维空间可以解释为什么引力是如此的小,而来自与我们相邻的其他膜的引力子可以为暗物质提供潜在的解释。然而,如果引力子在膜之间完全自由移动,这将过分稀释引力,导致违反牛顿平方反比定律。为了解决这个问题,丽莎·蓝道尔发现了具有自身引力的三膜,阻止引力子的自由漂移,从而导致我们观察到的重力稀释,同时大致保持了牛顿平方反比定律。[10]

艾哈迈德·法拉赫·阿里(Ahmed Farag Ali)和索利亚·达斯(Saurya Das)提出的理论将量子力学校正(使用Bohm轨迹)添加到广义相对论测地线中。如果引力子的质量很小但不为零,则可以解释宇宙常数而无需暗能量,从而解决了体积小的问题。[11]该理论用于解释宇宙常数的微小性,并在2014年Gravity Research Foundation 论文大赛上获得了荣誉奖。[12]此外,该理论还在2015年Gravity Research Foundation 论文竞赛中获得了优秀奖,用于自然解释由于所提出的量子修正而观察到的宇宙的大尺度均匀性和各向同性。[13]

\subsection{能量和波长}
虽然引力子被认为是无质量的,但它们仍然会像其他任何量子粒子一样携带能量。光子能量和胶子能量也由无质量的粒子携带。目前,尚不清楚哪些变量可以确定引力子能量,即单个引力子携带的能量。

或者,如果引力子是大质量的,则对引力波的分析产生了引力子质量的新上限。引力子的康普顿波长至少是$1.6\times10^{16}m$,或约1.6 光年,对应于不大于$7.7\times10^{-23} eV/c^{2}$。[14]波长和质能之间的关系是用普朗克-爱因斯坦关系式计算的,该公式将电磁波长与光子能量联系起来。然而,如果引力子是引力波的量子,那么引力子的波长和相应的粒子能量之间的关系从根本上不同于光子,因为引力子的康普顿波长不等于引力波的波长。相反,下界引力子的康普顿波长大约是GW170104事件的引力波长(约1,700公里)的$9\times10^{9}$倍。该报告[14]没有详细说明这一比例的来源。引力子可能不是引力波的量子,或者这两种现象以不同的方式相关联。

\subsection{实验观察}
尽管没有任何基本法律禁止对单个引力子进行明确的检测,但是使用任何物理上合理的检测器都是不可能的。[15]原因是引力子与物质相互作用的横截面极低。即使在最有利的条件下,例如,将与木星质量相同且效率为100\%的探测器,放置在围绕中子星的紧密轨道上,也只能每十年观察一个引力子。将这些事件与中微子的背景区分开来是不可能的,因为所需中微子屏蔽的尺寸将确保其塌陷成一个黑洞。[15]

激光干涉引力波天文台和处女座的协同观测直接检测到了引力波。[16][17][18]其他人则推测,由于粒子相互作用产生相干态,引力子散射会产生引力波。[19]虽然这些实验不能检测单个引力子,但它们可能提供引力子某些性质的信息。[20]例如,如果引力波的传播速度慢于$c$(真空中的光速),这意味着引力子有质量(但是,在非零质量密度的区域中,如果是可检测的,则引力波的传播速度必须比$c$小。)。[21]最近对引力波的观测表明引力子的质量上限为$1.2\times10^{-22} eV/c^{2}$。[16]对星系运动学的天文观测,特别是星系旋转问题和修正牛顿动力学,可能会涉及到非零质量的引力子。[22]

\subsection{困难和未决问题}
大多数包含引力子的理论都存在严重的问题。试图通过添加引力子来扩展标准模型或其他量子场论,在接近或超过普朗克尺度的能量上遇到了很大的理论难题。这是因为量子效应引起的不定式;技术上,引力不是可重正化的。由于古典广义相对论和量子力学在这样的能量下似乎是不相容的,因此从理论上讲,这种情况是不成立的。一种可能的解决方案是采用弦理论。弦论是重力的量子理论,从某种意义上讲,它们在低能量下可归结为经典广义相对论和场论,但实际上它们属于量子力学领域,且具有引力子,并且在数学上被认为是一致的。[23]

\subsection{参考文献}
[1]
^关于广义相对论的几何推导和(非几何的)自旋2场推导的比较,参见Misner, C. W.; Thorne, K. S.; Wheeler, J. A. (1973). Gravitation. W. H. Freeman. ISBN 0-7167-0344-0..

[2]
^Feynman, R. P.; Morinigo, F. B.; Wagner, W. G.; Hatfield, B. (1995). Feynman Lectures on Gravitation. Addison-Wesley. ISBN 0-201-62734-5..

[3]
^Zee, A. (2003). Quantum Field Theory in a Nutshell. Princeton University Press. ISBN 0-691-01019-6..

[4]
^Randall, L. (2005). Warped Passages: Unraveling the Universe's Hidden Dimensions. Ecco Press. ISBN 0-06-053108-8..

[5]
^Blokhintsev, D. I.; Gal'perin, F. M. (1934). "Гипотеза нейтрино и закон сохранения энергии" [Neutrino hypothesis and conservation of energy]. Pod Znamenem Marxisma (in 俄语). 6: 147–157..

[6]
^请参阅上的其他文章广义相对论,重力场,引力波等等.

[7]
^Colosi, D.; et al. (2005). "Background independence in a nutshell: The dynamics of a tetrahedron". Classical and Quantum Gravity. 22 (14): 2971–2989. arXiv:gr-qc/0408079. Bibcode:2005CQGra..22.2971C. doi:10.1088/0264-9381/22/14/008..

[8]
^Witten, E. (1993). "Quantum Background Independence In String Theory". arXiv:hep-th/9306122..

[9]
^Smolin, L. (2005). "The case for background independence". arXiv:hep-th/0507235..

[10]
^Michio kaku(2006年)平行世界——另类宇宙的科学和我们在宇宙中的未来。双日。第218-221页。ISBN 978-0385509862。.

[11]
^Ali, Ahmed Farag (2014). "Cosmology from quantum potential". Physics Letters B. 741: 276–279. arXiv:1404.3093v3. Bibcode:2015PhLB..741..276F. doi:10.1016/j.physletb.2014.12.057..

[12]
^Das, Saurya (2014). "Cosmic coincidence or graviton mass?". International Journal of Modern Physics D. 23 (12): 1442017. arXiv:1405.4011. Bibcode:2014IJMPD..2342017D. doi:10.1142/S0218271814420176..

[13]
^Das, Saurya (2015). "Bose–Einstein condensation as an alternative to inflation". International Journal of Modern Physics D. 24 (12): 1544001–219. arXiv:1509.02658. Bibcode:2015IJMPD..2444001D. doi:10.1142/S0218271815440010..

[14]
^B. P. Abbott; et al. (LIGO Scientific Collaboration and Virgo Collaboration) (1 June 2017). "GW170104: Observation of a 50-Solar-Mass Binary Black Hole Coalescence at Redshift 0.2". Physical Review Letters. 118: 221101. arXiv:1706.01812. Bibcode:2017PhRvL.118v1101A. doi:10.1103/PhysRevLett.118.221101..

[15]
^Rothman, T.; Boughn, S. (2006). "Can Gravitons be Detected?". Foundations of Physics. 36 (12): 1801–1825. arXiv:gr-qc/0601043. Bibcode:2006FoPh...36.1801R. doi:10.1007/s10701-006-9081-9..

[16]
^Abbott, B. P. et al. (LIGO Scientific Collaboration and Virgo Collaboration) (2016). "Observation of Gravitational Waves from a Binary Black Hole Merger". Physical Review Letters. 116 (6): 061102. arXiv:1602.03837. Bibcode:2016PhRvL.116f1102A. doi:10.1103/PhysRevLett.116.061102. PMID 26918975..

[17]
^Castelvecchi, Davide; Witze, Witze (February 11, 2016). "Einstein's gravitational waves found at last". Nature News. doi:10.1038/nature.2016.19361..

[18]
^"Gravitational waves detected 100 years after Einstein's prediction | NSF - National Science Foundation". www.nsf.gov. Retrieved 2016-02-11..

[19]
^Senatore, L.; Silverstein, E.; Zaldarriaga, M. (2014). "New sources of gravitational waves during inflation". Journal of Cosmology and Astroparticle Physics. 2014 (8): 016. arXiv:1109.0542. Bibcode:2014JCAP...08..016S. doi:10.1088/1475-7516/2014/08/016..

[20]
^Dyson, Freeman (8 October 2013). "Is a Graviton Detectable?". International Journal of Modern Physics A. 28 (25): 1330041–1–1330035–14. Bibcode:2013IJMPA..2830041D. doi:10.1142/S0217751X1330041X..

[21]
^Will, C. M. (1998). "Bounding the mass of the graviton using gravitational-wave observations of inspiralling compact binaries" (PDF). Physical Review D. 57 (4): 2061–2068. arXiv:gr-qc/9709011. Bibcode:1998PhRvD..57.2061W. doi:10.1103/PhysRevD.57.2061..

[22]
^Trippe,S. (2013),"银河尺度上引力相互作用的简化处理",J. Kor。阿斯顿。足球。4641岁。arXiv:1211.4692.

[23]
^Sokal, A. (July 22, 1996). "Don't Pull the String Yet on Superstring Theory". The New York Times. Retrieved March 26, 2010..