% 激活函数
% 激活函数

\textbf{激活函数}(Activation function)是人工神经元计算流程的最后一步,跟随在仿射变换之后的一个非线性变换。神经元中的仿射变换是带有参数的,参数的值是模型训练时学习而来的。而激活函数往往是一个无参数的固定的非线性变换,它决定着一个神经元输出的值的范围。

神经网络中能够采用的激活函数种类繁多,往往须要根据实际应用场景做选择。设激活函数为$g$的输入为$x$,输出为$y$,有$y=g(x)$。


\subsection{恒等函数}
显然,恒等函数的表达式为:
\begin{equation}
g(x)=x
\end{equation}
导数为:
g'(x)=1

\subsection{S型函数}
S型函数又称为Sigmoid函数。表达式为:
\begin{equation}
y=g(x)=\frac{1}{1+e^x}~
\end{equation}

\subsection{双曲正切函数}
双曲正切函数表达式为:
\begin{equation}
y=g(x)=tanh(x)=\frac{e^x-e^{-x}}{e^x+e^{-x}}~
\end{equation}




在现代深度神经网络的架构中,最常用的激活函数是\textbf{线性整流单元}(Rectified linear unit, ReLu)。