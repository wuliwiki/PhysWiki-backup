% 实数中的开集和闭集
% 数学分析



\pentry{度量空间\upref{Metric}}

从实数理论中我们可以知道,有理数集和实数集都是域,同时他们都是良序的.以三个集合的描绘角度来说,它们的代数结构,序结构都是一致的,而第三个拓扑结构才是这两个集合的真正不同之处.我们要在一个足够普遍的空间上进行,但是我的能力有限,不能揭示出其中朴素的思想.总而言之,我们只需要一个有距离的空间——度量空间上去讨论.

利用度量空间上的距离函数,我们可以定义有界集.

\begin{definition}{有界集}
设$E$是度量空间$X$的子集,如果$\exists{p}\in{X}$使得$\exists{M}\in{\mathbb{R}}$,$\forall{q}\in{E}$,总是满足
\[d(p,q)<M\]
则称$E$是$X$上的有界集.
\end{definition}

\textbf{注意}:这里的有界的概念是度量空间的子集中的概念,和有序域中的的上下界以及上下确界是不同的.很快我们会发现它们之间的联系.

\begin{definition}{邻域}

\end{definition}


\begin{definition}{极限点}

\end{definition}



\begin{definition}{内点}

\end{definition}


\begin{definition}{开集}

\end{definition}


\begin{definition}{闭集}

\end{definition}


\begin{definition}{完备集}

\end{definition}

\begin{definition}{闭包}
\end{definition}

\begin{definition}{相对开集}

\end{definition}