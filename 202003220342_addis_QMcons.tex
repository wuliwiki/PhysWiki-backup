% 守恒量(量子力学)
% 量子力学|守恒|本征值|能量守恒

\pentry{平均值\upref{QMavg}, 无限深势阱\upref{ISW}}

在经典力学中, 某个时刻测量系统的某个量 $Q$ 可以得到唯一确定的值. 如果这个值不随时间变化, 就说它\textbf{守恒}. 而在量子力学中, 这个值(算符的本征值)是不确定的, 那守恒量由该如何理解呢? 比如无限深势阱中两个相同的粒子处于相同的状态(相同的波函数), 测得的能量却可能各不相同, 是否意味着能量不守恒?

答案是否定的, 在量子力学中, 我们定义若对系统的任意状态测量某个物理量得到任意本征值的概率都不随时间改变, 这个量就是守恒的. 例如在无限深势阱中任意随时间变化的波包, 测得每个能量本征值的概率都不随时间变化, 这就说明能量守恒. 由这个定义可知, 守恒量的平均值也不随时间变化.

\subsection{能量守恒}

\begin{theorem}{能量守恒}
若哈密顿算符 $H$ 不随时间变化, 就有能量守恒.
\end{theorem}
证明: 令能量的所有本征态为(为了方便起见, 假设本征态是离散的) $\{\ket{u_i}\}$, 系统状态随时间的演化可以记为($C_i$ 为常数, 由初始状态决定)
\begin{equation}
\ket{v(t)} = \sum_i C_i \E^{-\I E_i t} \ket{u_i}
\end{equation}
令能量 $E_j$ 对应的所有本征态的角标的集合为 $D_j$, 那么测得 $E_j$ 的概率为
\begin{equation}
P(E_j) = \sum_{i \in D_j} \abs{C_i}^2
\end{equation}
不随时间变化, 所以有能量守恒. 证毕.

\begin{example}{}
能量守恒的简单系统如: 无限深势阱\upref{ISW}, 简谐振子\upref{QSHOop}, 自由粒子% 未完成
, 有限深势阱\upref{FiSph}, 阶梯势能\upref{StpPot}等. 这些系统的哈密顿算符都不随时间变化。 将任意时刻的波函数投影到能量的本征态上, 得到的系数的模方都是相同的.
\end{example}

\subsection{其他守恒量}

\pentry{算符对易与共同本征矢\upref{OpComu}}

\begin{theorem}{对易与守恒量}
令 $H$ 为哈密顿算符, $Q$ 是另一个物理量的算符. 以下两个命题互为充分必要条件充分必要条件\upref{SufCnd}
\begin{enumerate}
\item $Q$ 对应一个守恒量
\item $\comm{H}{Q} = 0$
\end{enumerate}
\end{theorem}

我们还是从离散本征态的简单情况来证明这个定理, 且假设哈密顿算符不含时\footnote{如果含时, 可以将时间分割成许多小区间, 假设每个小区间中不随时间变化}. 首先由 2 证明 1, 若两算符对称, 则它们必有一组完备的共同本征矢 $\{\ket{u_i}\}$, 满足
\begin{equation}
\begin{cases}
H \ket{u_i} = E_i \ket{u_i}\\
Q \ket{u_i} = q_i \ket{u_i}
\end{cases}
\end{equation}

系统状态随时间的变化可以记为
\begin{equation}
\ket{v(t)} = \sum_i C_i \E^{-\I E_i t} \ket{u_i}
\end{equation}
其中 $C_i$ 为常数. 令集合 $D_j$ 包含 $q_j$ 对应的所有基底的角标. 于是, 任意时刻测到 $q_j$ 的概率等于相关系数的模方和
\begin{equation}
P(q_j) = \sum_{i \in D_j} \abs{C_i \E^{-\I E_i t}}^2 = \sum_{i \in D_j} \abs{C_i}^2
\end{equation}
也是常数. 证毕.

由 1 证明 2 要更麻烦一些. 我们新开一节, 不感兴趣的读者可以跳过.

\subsection{证明守恒量算符与哈密顿算符对易}
先假设能量的本征态基底为 $\{\ket{u_i}\}$, 系统状态随时间变化为
\begin{equation}
\ket{v(t)} = \sum_i C_i \E^{-\I E_i t} \ket{u_i}
\end{equation}
令算符 $Q$ 任意本征值 $q_m$ 对应的本征矢子空间的一组基底为 $\{\ket{v_k}\}$, 那么任意时刻测得 $q_m$ 的概率为
\begin{equation}
\begin{aligned}
P(q_m) &= \sum_k \abs{\bra{v_k} \sum_i C_i \E^{-\I E_i t} \ket{u_i}}^2\\
&= \sum_k \qty(\sum_i C_i \E^{\I E_i t} \braket{v_k}{u_i})\Her \qty(\sum_j C_j \E^{-\I E_j t} \braket{v_k}{u_j})\\
&= \sum_{i,i'} \qty( C_i^* C_j \sum_k \braket{u_i}{v_k} \braket{v_k}{u_j}) \E^{\I (E_i - E_j) t}
\end{aligned}
\end{equation}
若 $Q$ 守恒, 则要求对于任何 $C_i$ 可能的取值, 该式的结果都与时间无关. 所以对任意 $i \ne j$ 都有
\begin{equation}
\sum_k \braket{u_i}{v_k} \braket{v_k}{u_j} = 0
\end{equation}
稍加思考可以得出要么 $\ket{u_j}$ 在 $q_m$ 子空间中, 要么 $\ket{u_j}$ 在 $q_m$ 子空间上的投影为 0. 所以任何 $q_m$ 子空间(假设是 $N_m$ 维)同样可以由 $N_m$ 个 $\ket{u_i}$ 张成. 所以可以找到 $H$ 和 $Q$ 的一套共同本征矢, 所以两算符对易. 证毕.
