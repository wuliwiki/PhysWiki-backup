% 链、路和圈
% keys 链|路|圈
% license Usr
% type Tutor

\pentry{图\nref{nod_Graph}}{nod_9c36}
本节介绍图中一类特殊的结构————链,而路是一类特殊的链,圈是一类特殊的路。

\subsection{链}
生活中我们把环环相扣的东西叫做链,图论中把点边交叉组成的序列称作链(不同位置的点和边可以相同)。

\begin{definition}{链}
设 $G=(V,E,\varphi)$ 是一个\enref{图}{Graph}, $v_i\in V,e_j\in E,i=0,\cdots,k,j=1\cdots n$ 分别是其上的 $n+1$ 个点和 $n$ 条边。若序列\footnote{注意因为点边的记号很好区分,往往省略了序列中相邻两元间的逗号}
\begin{equation}
W=v_0 e_1v_1e_2\cdots v_{k-1} e_kv_k,\quad v_i\neq v_j,i\neq j,~
\end{equation}
对所有的 $i=0,\cdots,k$ 满足 $v_{i-1},v_{i}\in e_{i}$,则称 $W$ 是 $G$ 中的一个\textbf{链}。 
\end{definition}














