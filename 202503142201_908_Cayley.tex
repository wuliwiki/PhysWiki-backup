% 阿瑟·凯利(综述)
% license CCBYSA3
% type Wiki

本文根据 CC-BY-SA 协议转载翻译自维基百科\href{https://en.wikipedia.org/wiki/Arthur_Cayley}{相关文章}。

\begin{figure}[ht]
\centering
\includegraphics[width=6cm]{./figures/54015c4f6823f2b0.png}
\caption{} \label{fig_Cayley_1}
\end{figure}
阿瑟·凯利(Arthur Cayley,FRS,1821年8月16日-1895年1月26日)是英国数学家,主要从事代数方面的研究。他帮助创立了现代英国纯数学学派,并在剑桥大学三一学院担任教授长达35年。

他提出了如今被称为凯利-哈密顿定理的观点——即每个方阵都是其自身特征多项式的根,并验证了2阶和3阶矩阵的情况。[1]他是第一个定义抽象群概念的人,抽象群是满足某些运算规律的集合,[2]区别于艾瓦里斯特·伽罗瓦(Évariste Galois)对置换群的定义。在群论中,凯利表、凯利图以及凯利定理都以他命名,而在组合数学中,也有凯利公式以纪念他。
\subsection{早年生活}  
阿瑟·凯利于1821年8月16日出生在英国伦敦的里士满。他的父亲亨利·凯利是航空工程师乔治·凯利的远亲,来自约克郡的一个古老家族,并作为商人定居在俄罗斯圣彼得堡。他的母亲是玛丽亚·安托尼娅·道蒂(Maria Antonia Doughty),威廉·道蒂的女儿。根据一些作家的说法,她是俄罗斯人,但她父亲的名字表明她有英格兰血统。他的兄弟是语言学家查尔斯·巴戈特·凯利。阿瑟在圣彼得堡度过了他生命的最初八年。1829年,他的父母定居在伦敦的布莱克希斯,阿瑟在那里上了一所私立学校。

14岁时,他被送到国王学院学校。年轻的凯利喜欢复杂的数学问题,学校的校长注意到他在数学方面的天赋,并建议父亲不要让儿子从事商业工作,而是送他去剑桥大学学习。
\subsection{教育}  
17岁时,凯利开始在剑桥大学三一学院学习,在那里他在希腊语、法语、德语、意大利语以及数学方面表现出色。此时,分析学会的事业已取得胜利,剑桥数学杂志由格雷戈里(Gregory)和罗伯特·莱斯利·埃利斯(Robert Leslie Ellis)创办。20岁时,凯利向该杂志投稿了三篇论文,这些论文的主题受到了他阅读约瑟夫·路易·拉格朗日的《解析力学》和拉普拉斯的一些著作的启发。

凯利在剑桥的导师是乔治·皮科克(George Peacock),他的私人教练是威廉·霍普金斯(William Hopkins)。他通过赢得高级学者(Senior Wrangler)和第一个史密斯奖学金(Smith's Prize)顺利完成了本科课程。接下来,他计划获得硕士学位,并通过竞争考试赢得奖学金。[3]他继续在剑桥大学居住了四年,在此期间,他接收了一些学生,但他的主要工作是为数学杂志准备28篇论文。
\subsection{法律事业}  
由于奖学金的任期有限,他必须选择一个职业;像德·摩根一样,凯利选择了法律,并于1846年4月20日以24岁的年龄被录取为林肯律师协会(Lincoln's Inn)的成员。[4]他专攻不动产转让。在参加律师资格考试期间,他曾前往都柏林聆听威廉·罗文·汉密尔顿关于四元数的讲座。

他的朋友J.J. 西尔维斯特(J. J. Sylvester),比他大五岁,曾在剑桥大学与他为同学,当时是一名常驻伦敦的精算师;他们常一起在林肯律师协会的院子里散步,讨论不变量和协变理论。在这十四年间,凯利创作了大约二三百篇论文。[5]


