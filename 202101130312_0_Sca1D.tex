% 一维散射(量子)

\begin{issues}
\issueDraft
\end{issues}

\pentry{含时薛定谔方程\upref{TDSE}}
\addTODO{这个是更一般的介绍, 自由粒子只是 $V(x) \equiv 0$ 的特例而已.}
一个波包 $\psi(x, t)$, 若我们知道 $t = 0$ 时的函数, 如何求波函数接下来的\textbf{演化(propagation)}呢? 从概念上, 我们可以直接把它代入薛定谔方程进行求解
\begin{equation}
\frac{1}{2m}\pdv[2]{t}\psi(x, t) + V(x) \psi(x, t) = \I \pdv{t} \psi(x, t)
\end{equation}
即使没有解析解, 也可以通过数值方法, 依次求出每个时间步长 $t_n$ ($n = 1, 2, \dots$)的波函数 $\psi(x, t_n)$. % 连接未完成

但对于我们有更好的办法, 就是求出不同的动量本征态 $\psi_k$, 然后由于薛定谔方程是线性的(未完成).

\addTODO{然后, 各种散射势能引用该词条. 我们至少要先说明散射态(本征态)有什么用, 再去求解吧.}
