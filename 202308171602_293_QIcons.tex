% 量子信息守恒
% keys 量子不可克隆定理|量子不可删除定理
% license Xiao
% type Tutor

在这一章中,我们将会讨论量子信息中的两个基本原理:不可克隆定理和不可删除定理。它们有时也被冠名为量子信息守恒原理,因为它们告诉了我们量子信息相比于经典信息的不同之处:它不可被复制,也不可被删除。这样的特性在很大程度上成为了量子通信、量子密码学中的无条件安全性的来源,而且也导致了量子纠错等任务相比于经典信息处理困难得多。

\subsection{量子不可克隆定理}

在日常生活中,我们经常需要对信息进行拷贝、复制或者说克隆。这一点可以通过手动抄写、复印机、文件拷贝等办法进行处理。

不管怎么样,在信息的复制过程中,我们需要将想要复制的信息(这里用$\ket{\psi}\rangle$来标记)和一份空白的信息(这里用$\ket{0}\rangle$来标记)放进一个克隆机器中,这台机器便会输出两份信息$\ket{\psi}\rangle\ket{\psi}\rangle$。注意我们要求$\ket{\psi}\rangle$是任意的。不然这也称不上是一个克隆机器,只是一个把$\ket{\psi}\rangle$写入到$\ket{0}\rangle$的机器罢了。我们同样要求克隆机不会破坏初始输入的信息。

对于经典信息来说,$\ket{\psi}\rangle=0,1$,在这种情况下显然是存在克隆机的。只需要让它执行这样的程序:
\begin{itemize}
\item 如果$\ket{\psi}\rangle=0$,那么在$\ket{0}\rangle$写入0。
\item 如果$\ket{\psi}\rangle=1$,那么在$\ket{1}\rangle$写入1。
\end{itemize}
这也是拷贝文件时的处理逻辑。

下面的问题是:对于量子信息(也就是$\ket{\psi}\rangle=\ket{\psi}$)是否也存在这种克隆机呢?具体来说,是否有这样一个幺正演化$U$,使得对于任意的输入$\ket{\psi}$,都有
\begin{equation}
U\ket{\psi}\ket{0}=\ket{\psi}\ket{\psi}~.
\end{equation}
成立?

量子不可克隆定理告诉我们,这样的$U$并不存在。

\textbf{证明:}假如存在着这样的$U$,那么对于任意两个态$\ket{\psi_1},\ket{\psi_2}$来说,有
\begin{equation}
\begin{aligned}
U\ket{\psi_1}\ket{0}&=\ket{\psi_1}\ket{\psi_1}\\
U\ket{\psi_2}\ket{0}&=\ket{\psi_2}\ket{\psi_2}~.
\end{aligned}
\end{equation}
那么有
\begin{equation}
\braket{\psi_1}{\psi_2}=\bra{\psi_1}\bra{0}U^\dagger U\ket{\psi_2}\ket{0}=\bra{\psi_1}\bra{\psi_1}\ket{\psi_2}\ket{\psi_2}=\braket{\psi_1}{\psi_2}^2~.
\end{equation}

该式只在$\braket{\psi_1}{\psi_2}=0,1$时成立。而这与$\ket{\psi_1},\ket{\psi_2}$的任意性相矛盾。因此并不存在这样的$U$。

值得注意的是,量子不可克隆定理并没有否认经典信息的克隆操作。经典信息可以被编码为$\ket{0}\to0,\ket{1}\to1$,而且整个过程中不允许有态的叠加出现。那么此时显然CNOT门就能够起到克隆信息的作用。因此这个故事也告诉了我们,如果把量子信息全部限制为某一个测量基矢下的本征态上,那它就会和经典信息没有差别。

\subsection{量子不可删除定理}

经典信息不但可以复制,同时还可以被删除。比如生活中我们会在电脑中删除某些文件,也可以把纸质文件放进碎纸机中进行粉碎。

不过严格定义到底什么是对信息的删除需要保持谨慎。比如说即使我们粉碎了一份文件,但是这份文件的信息被扩散到了环境中,有心人仍然可以通过收集碎屑来复原这份文件。换句话说,环境所处的状态会和初态的信息有关。在这种情况下我们不认为这是对信息的删除。那什么被认为是信息的删除呢?逻辑电路里的置零操作(在逻辑上)就是一种对信息的删除:不论本来比特处于什么状态,一律将其变成0。

因此对信息删除的定义应该是:不论系统初态如何,输出的末态都是一个相同的态,而且环境的状态也和被删除的系统状态无关。

那么现在的问题是,对于量子信息来说,我们能否完成对信息的删除呢?很容易可以验证并不可以。