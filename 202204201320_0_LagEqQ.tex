% 拉格朗日方程(额外广义力)

\begin{issues}
\issueTODO
\end{issues}

\pentry{欧拉—拉格朗日方程\upref{Lagrng}}

在之前的讨论中, 所有的系统外力都可以纳入势能项 $V$ 中. 这些里一般来说是保守力, 然而许多情况中还可能出现种种非保守力, 例如摩擦力等, 较难用势能描述. 这时引入\textbf{广义力(generalized force)}的概念会更方便. 若系统的广义坐标为 $q_i$, 则每个广义坐标对应的广义力定义为(上标 $e$ 表示 extra)
\begin{equation}
Q_i^{(e)}(q, \dot q, t) = \sum_j \bvec F_j^{(e)} \vdot \pdv{\bvec r_j}{q_i} \qquad (i=1,\dots,N)
\end{equation}
其中 $\bvec F_j$ 是除约束力的系统外力, 作用点为 $\bvec r_j$. 从虚功的角度, 广义力等于虚功对广义坐标的导数. 和之前一样, 为了书写方便式中 $q$ 代表 $q_1, \dots, q_N$, $\dot q$ 也同理.
\addTODO{虚功和虚功原理在哪里介绍?}

若非约束力 $\bvec F_j^{(e)}$ 不包含于势能项 $V$, 那么拉格朗日方程(\autoref{Lagrng_eq1}~\upref{Lagrng})可以拓展为
\begin{equation}\label{LagEqQ_eq1}
\dv{t} \pdv{L}{\dot q_i} = \pdv{L}{q_i} + Q_i^{(e)}
\qquad (i=1,\dots,N)
\end{equation}
证明见 “拉格朗日方程的证明、达朗贝尔定理\upref{dAlbt}”.

事实上, 在证明中会发现 $\pdv*{L}{q_i} = \pdv*{T}{q_i} + \pdv*{V}{q_i}$ 的第二项同样也是广义力
\begin{equation}
Q_i^{(V)} = \pdv{V}{q_i} = \sum_j \bvec F_j^{(V)} \vdot \pdv{\bvec r_j}{q_i}
\end{equation}
那么如果令总广义力为
\begin{equation}
Q_i = Q_i^{(V)} + Q_i^{(e)}
\end{equation}
那么拉格朗日方程\autoref{LagEqQ_eq1} 也可以完全抛开势能的概念写成
\begin{equation}\label{LagEqQ_eq2}
\dv{t} \pdv{L}{\dot q_i} = \pdv{T}{q_i} + Q_i
\qquad (i=1,\dots,N)
\end{equation}


\begin{example}{受阻耦合弹簧振子}
\autoref{Lagrng_ex2}~\upref{Lagrng} 中, 若两个滑块受到于速度相反, 大小与速度平方成正比的摩擦力, 比例系数为 $\alpha$, 求运动方程.

这里有一点微妙的地方在于摩擦力是取决于运动方向的. 把 $\dot X$ 的正负号记为 $\opn{sgn}(\dot X)$, 那么广义力为
\begin{equation}
Q_x = -2\alpha\opn{sgn}(\dot x) \dot x^2~, \qquad
Q_X = -2\alpha\opn{sgn}(\dot X) \dot X^2
\end{equation}
求得运动方程为
\begin{equation}
m\ddot x =  - 3kx - 4\alpha \opn{sgn}(\dot x)\dot x^2~, \qquad
m\ddot X =  - kX - \alpha\opn{sgn}(\dot X)\dot X^2
\end{equation}
\end{example}
