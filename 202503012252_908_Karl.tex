% 卡尔·威尔施特拉斯
% license CCBYSA3
% type Wiki

本文根据 CC-BY-SA 协议转载翻译自维基百科\href{https://en.wikipedia.org/wiki/Karl_Weierstrass}{相关文章}。
\begin{figure}[ht]
\centering
\includegraphics[width=6cm]{./figures/f068c4d82f5a5984.png}
\caption{} \label{fig_Karl_1}
\end{figure}
卡尔·西奥多·威尔斯特拉斯(/ˈvaɪərˌstrɑːs, -ˌʃtrɑːs/;德语:Weierstraß [ˈvaɪɐʃtʁaːs];1815年10月31日 – 1897年2月19日)是一位德国数学家,常被称为“现代分析学之父”。尽管他未获得大学学位,但他学习了数学并接受了教师培训,最终教授数学、物理学、植物学和体育学。[3] 他后来获得了荣誉博士学位,并成为柏林大学的数学教授。

威尔斯特拉斯做出了许多贡献,其中包括形式化函数连续性的定义和复分析,证明了中值定理和博尔查诺–威尔斯特拉斯定理,并利用后者研究闭合有界区间上连续函数的性质。
\subsection{传记}
威尔斯特拉斯出生于西法兰西省恩尼格尔洛附近的一个名为奥斯滕费尔德的村庄,来自一个天主教家庭。[4]

卡尔·威尔斯特拉斯是威廉·威尔斯特拉斯和西奥多拉·冯德福尔斯特的儿子,前者是政府官员,后者是天主教的莱茵兰人。他对数学的兴趣始于他在帕德博恩的提奥多里亚努姆中学时。毕业后,他被送往波恩大学,准备为政府工作;为此,他的学习方向应该是法律、经济和金融,这与他自己希望学习数学的愿望发生了直接冲突。他通过忽视计划中的课程,继续私下学习数学,最终导致他未能获得学位而离开了大学。

威尔斯特拉斯继续在明斯特学院(当时以数学闻名)学习数学,且他的父亲为他争取到了一所明斯特师范学校的位置;他在那里的努力最终使他获得了该市的教师资格。在这一学习期间,威尔斯特拉斯听取了克里斯托夫·古德曼的讲座,并对椭圆函数产生了兴趣。

1843年,他在西普鲁士的德意志克罗内教书,1848年起,他在布劳恩斯贝格的霍西亚努姆中学任教。[5] 除了数学,他还教授物理学、植物学和体育学。[4] 在某个时期,威尔斯特拉斯可能与他朋友卡尔·威廉·博尔哈特的寡妇有一个私生子("弗朗茨")。[6][有争议 – 讨论]

1850年后,威尔斯特拉斯经历了长期的疾病,但他依然能够发表具有足够质量和原创性的数学文章,令他赢得了声誉和殊荣。 Königsberg大学于1854年3月31日授予他荣誉博士学位。1856年,他在柏林的工业学院(该学院原为培养技术工人的机构,后来与建筑学院合并,成为位于查尔滕堡的技术高等学府,现为柏林工业大学)担任教职。1864年,他成为了柏林弗里德里希·威廉大学的教授,该校后来成为柏林洪堡大学。

1870年,55岁的威尔斯特拉斯遇到了索菲亚·科瓦列夫斯卡娅,并在未能为她争取到大学入学资格后,开始私下辅导她。他们之间建立了富有成效的智力关系,并且个人关系也远超普通的师生关系。他辅导了她四年,视她为自己最好的学生,帮助她获得了海德堡大学的博士学位,且无需进行口头答辩。

从1870年到1891年她去世期间,科瓦列夫斯卡娅与威尔斯特拉斯保持通信。当他得知她去世的消息时,他烧毁了她的信件。大约150封他写给她的信件被保留下来。雷因哈德·贝林教授发现了她在1883年抵达斯德哥尔摩并被任命为斯德哥尔摩大学私人讲师时写给威尔斯特拉斯的信件草稿。[7]

威尔斯特拉斯在生命的最后三年行动不便,并于1897年2月19日在柏林因肺炎去世。[8]
\subsection{数学贡献}
\subsubsection{微积分的健全性}  
威尔斯特拉斯对微积分的健全性非常感兴趣,当时微积分的基础定义尚不明确,因此许多重要的定理无法以足够的严谨性证明。尽管博尔扎诺早在1817年就已经发展出了一个相对严谨的极限定义(甚至可能更早),但他的工作在多年后才为大多数数学界所知,而许多数学家对于极限和函数的连续性只有模糊的定义。

\(\Delta-\epsilon\)证明的基本思想可以说最早出现在柯西在1820年代的作品中。[9][10] 柯西并没有明确区分区间上的连续性和一致连续性。特别是在他1821年出版的《分析课程》中,柯西认为(逐点)连续函数的(逐点)极限本身是(逐点)连续的,而这一说法在一般情况下是错误的。正确的说法是,连续函数的一致极限是连续的(此外,一致连续函数的一致极限是一致连续的)。这一结果需要一致收敛的概念,而一致收敛首次在威尔斯特拉斯的导师克里斯托夫·古德曼的1838年论文中被注意到,古德曼观察到这一现象,但并未给出定义或进一步阐述。威尔斯特拉斯看到了这一概念的重要性,并且对其进行了形式化并广泛应用于微积分的基础中。