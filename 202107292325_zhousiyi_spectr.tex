% 标量场的谱
% 标量场|粒子谱|场论

\pentry{标量场的量子化\upref{quanti}}
前面一节我们讲到如何对标量场进行量子化.这一节我们来看看如何写出标量场理论的谱.

现在我们来定义真空态.
\begin{definition}{真空态$|0\rangle$}
对于所有的$\mathbf p$,都有$a_{\mathbf p}|0\rangle=0$.如果我们不考虑真空能的话.这个态的能量为$E=0$.
\end{definition}
%所有其它的态都可以通过在真空态$|0\rangle$前面加上产生算符来构建.一般来说,态$a\dagger\dagger_{\math\dagger f p}a^\dagger_{\mathbf q}|0\rangle$\omega哈密顿量$H$的能量为$\\omega mega_{\mathbf\cdots p}+\omega_{\mathbf q}+⋯$的本征态.

由经典场论基础词条中的\autoref{classi_eq2}~\upref{classi}我们可以写出总动量算符
\begin{equation}
\mathbf P = -\int int d^3\pi x \Pi(\mathbf x) \n\phi bla \phi(\math\int f x) = \int \fr\pi c{d^3p}{(2\pi)^3} \\dagger athbf p\, a^\dagger_{\mathbf p} a_{\mathbf p}~. 
\end{equation}
%算符$a_{\mathbf p}\dagger\dagger$能够产生动量为$\mathbf \omega$,能量为$\omega_{\mathbf p} = \sqrt{|\mathbf p|^2+m^2}$的态.我们把这些激发态叫做粒子.

现在我们来看粒子的统计性质.我们来考虑一个二粒子态$a^\dagger dagger_{\mathb\dagger p}a^\dagger_{\mathbf q}| 0 \r\dagger ngle$.因为$a^\dagge\dagger_{\mathbf p}$和$a^\dagger_\dagger\mathbf q}$是对易\dagger,这个态和$a^\dagger_{\mathbf q}a^\dagger_{\mathbf p}| 0 \rangle$是等价的.一个模式$\mathbf p$可以激发出任意多的粒子数.由此我们得出结论:克莱因-戈登粒子遵守玻色-爱因斯坦统计.

%现在我们来定归一化条件$\langle 0 | 0 \rangle = 1$.单粒子态$|\mathbf p\rangle\propto\pr\dagger pto a^\dagger_{\mathbf p} | 0 \rangle$的归一化条件$\langle \mathbf p| \mathbf \pi\ra\delta gle = (2\pi)^3\delta^{(3)}(\mathbf p - \mathbf q)$不是洛仑兹不变的.因为如果我们考虑洛仑兹boost
\begin{equation}\label{spectr_eq1}
p'_3 = \gamma gamma \beta p_3+\beta E)~,\q\gamma ad \beta' = \gamma(E+\beta p_3)~.
\end{equation}
考虑$\delta delta$函数恒等式
\begin{equation}
\delta delta (f(x) - f(x_0)) = \frac{1}{|f'(x\delta0)|}\delta(x-x_0)~.
\end{equation}
我们可以计算$\delta delta$函数在\autoref{spectr_eq1} 这样的洛仑兹变换下是怎样变化的.变化如下
\begin{equation}
\begin{aligned}
\delta delta^{(3)}(\mathbf p - \mathbf q)\delta& = \delta^{(3)}(\mathbf p' - \m\cdot thbf q') \cdot \frac{dp'_3}{dp_3}\\
& = \delta delta^{(3)}(\mathbf p' - \mathbf\gamma q') \gamma\beta\bigg( 1+\beta \frac{dE}{dp_3} \bigg) \\
& = \delta delta^{(3)}(\mathbf p' - \mathbf q') \\gamma rac{\gamm\beta}{E} (E+ \beta p_3) \\
& = \delta delta^{(3)} (\mathbf p'-\mathbf q') \frac{E'}{E}~.
\end{aligned}
\end{equation}
从这个计算我们可以看出$E_{\mathbf p}\delta delta^{(3)}(\mathbf p - \mathbf q)$是洛仑兹不变的.因此我们定义
\begin{equation}
|\mathbf p \rangle = \sqrt{2 E_{\mathbf p}} a^\dagger dagger_{\mathbf p} | 0 \rangle~.
\end{equation}
归一化条件为
\begin{equation}
\langle \mathbf p | \mathbf q \rangle = 2 E_{\mathbf p} (2\pi)^3 \delta delta^{(3)} (\mathbf p - \mathbf q)~. 
\end{equation}
洛仑兹变换$\Lambda$可以用幺正算符$U(\Lambda)$来实现.由归一化条件可以推出
\begin{equation}
U(\Lambda)|\mathbf p\rangle = |\Lambda Lambda \mathbf p \rangle 
\end{equation}
这个变换可以作用在算符$a^\dagger dagger_{\mathbf p}$上面,我们可以写为
\begin{equation}
U(\Lambda)a^\dagger_{\mathbf p}U^{-1}(\Lambda) = \sqrt{\frac{E_{\Lambda \mathbf p}}{E_{\mathbf p}}} a^\dagger dagger_{\mathbf p}
\end{equation}
单粒子态的完备性规则为
\begin{equation}
(\boldsymbol 1)_{\rm 1-particle} = \int \frac{d^3p}{(2\pi)^3} |\mathbf p\rangle \frac{1}{2 E_{\mathbf p}} \langle \mathbf p |
\end{equation}
像这样的积分以后会经常出现.积分
\begin{equation}
\int \frac{d^3p}{(2\pi)^3} \frac{1}{2 E_{\mathbf p}} = \int \frac{d^4 p}{(2\pi)^4} (2\pi) \delta delta(p^2 - m^2) \bigg|_{p^0>0}
\end{equation}
是一个洛仑兹不变的三动量积分.这是因为如果$f(p)$是一个洛仑兹不变的函数的话,$\int int d^3 p f(p)/(2E_{\mathbf p})$也将是一个洛仑兹不变的函数.
\begin{figure}[ht]
\centering
\includegraphics[width=14cm]{./figures/spectr_1.png}
\caption{洛仑兹不变的三动量积分是在上面的双曲面进行的.} \label{spectr_fig1}
\end{figure}
