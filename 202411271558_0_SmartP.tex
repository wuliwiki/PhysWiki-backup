% C++ 的智能指针笔记
% license Xiao
% type Note

\begin{issues}
\issueDraft
\end{issues}

\subsection{unique\_ptr}
\begin{itemize}
\item \href{https://en.cppreference.com/w/cpp/memory/unique_ptr}{文档}。
\item \verb`unique_ptr<类> p(类指针)`, 其中 \verb`类指针` 可以由 \verb`new 类(构造参数)` 产生,指向一个需要 \verb`delete` 的对象。
\item \verb`unique_ptr<类,删除函数类> p(类指针, 删除函数)`。 \verb`删除函数` 是 \verb`删除函数类` 的一个对象(例如 C 函数指针)。
\item \verb`删除函数类` 和 \verb`删除函数` 可以省略, 省略后默认用 \verb`delete 类指针` 或 \verb`delete[] 类数组`,实际上是 \verb`std::default_delete<T>`(对数组进行了特化)。
\item \verb`删除函数` 需要可以像 \verb`void 函数名(类* p)` 一样使用。 它几乎可以是任何 callable(例如具有 \verb`operator()` 的对象)
\item \verb`unique_ptr` 的 destructor 会调用 \verb`删除函数(类指针)` 释放内存。
\item 一般指针的操作 \verb`*, ->, []`, 也可以对 \verb`unique_ptr` 使用。
\item 必须初始化, 不存在 copy assignment, 只有 move assignment。 \verb`p2 = move(p1)` 可以转移。
\item \verb`p.get()` 返回 \verb`类指针`。 \verb`p.get_deleter()` 返回 \verb`删除函数` 的 ref。
\item \verb`if(p)` 可以判断是否在管理一个对象。
\item \verb`p.reset(新指针)` 调用原来对象的删除函数, 并管理新的对象。
\end{itemize}
\begin{lstlisting}[language=cpp]
// 创建(无法指定释放器,默认用 delete 释放)
std::unique_ptr<int> ptr = std::make_unique<int>(42);
std::unique_ptr<int> ptr2 = std::move(ptr1); // 转移所有权
int* rawPtr = ptr.release(); // 释放所有权并返回裸指针(对象不会被释放)
ptr.reset(); // 手动释放对象

// 自定义释放器
struct Deleter {
    void operator()(int* ptr) {
        std::cout << "Deleting resource: " << *ptr << std::endl;
        delete ptr;
    }
};

// 用法
std::unique_ptr<int, Deleter> ptr(new int(42));
\end{lstlisting}


\subsection{shared\_ptr}
\begin{itemize}
\item 可以使用 \verb`p2 = p1` 复制, 只有所有指向同一对象的 \verb`shared_ptr` 的 destructor 都被调用, 才会调用删除函数。
\item \verb`p.use_count()` 返回有几个 \verb`shared_ptr` 正在使用。
\end{itemize}

\subsection{weak\_ptr}
\begin{itemize}
\item 可以从 \verb`shared_ptr`, 或者 \verb`weak_ptr` 初始化
\item \verb`p.lock()` 创建出管理对象的一个新的 \verb`shared_ptr`
\item \verb`p.expired()` 可以检查对象是否已经被删除。
\item 可用于防止循环依赖导致的内存泄漏, 例如:
\begin{lstlisting}[language=cpp]
#include <iostream>
#include <memory>

class Node {
public:
    std::shared_ptr<Node> next;
    std::weak_ptr<Node> prev; // weak_ptr to avoid circular reference

    ~Node() {
        std::cout << "Node destroyed" << std::endl;
    }
};

int main() {
    std::shared_ptr<Node> node1 = std::make_shared<Node>();
    std::shared_ptr<Node> node2 = std::make_shared<Node>();

    node1->next = node2;
    // This would create a circular reference if prev were a shared_ptr
    node2->prev = node1;

    std::cout << "node1 use count: " << node1.use_count() << std::endl;  // 1
    std::cout << "node2 use count: " << node2.use_count() << std::endl;  // 2

    return 0;
}
\end{lstlisting}
\end{itemize}
