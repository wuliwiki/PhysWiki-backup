% 最小二乘法数值拟合(多项式)

\pentry{最小二乘法\upref{LstSqr}}

\begin{figure}[ht]
\centering
\includegraphics[width=8cm]{./figures/LSpoly_1.png}
\caption{多项式拟合正弦函数} \label{LSpoly_fig1}
\end{figure}

以下代码使用\autoref{LstSqr_eq2}~\upref{LstSqr} 解出多项式拟合的最小二乘法系数 $c_i$.
\begin{equation}
p(x) = c_0 + c_1 x + c_2 x^2 + \dots
\end{equation}

\begin{lstlisting}[language=matlab, caption=ls_poly.m]
% c = [..., c2, c1, c0]
function c = ls_poly(x, y, order)
Sx = zeros(1, 2*order+1);
Sx(1) = numel(x);
for n=1:2*order
    Sx(n+1) = sum(x.^n);
end

Sxy = zeros(order+1, 1);
for n=0:order
    Sxy(n+1) = sum(y .* x.^n);
end
Sxy = flip(Sxy);

A = zeros(order+1, order+1);
for i = 0:order
    for j = 0:order
        A(order-i+1,order-j+1) = Sx(i+j+1);
    end
end
c = A \ Sxy;
end
\end{lstlisting}

\begin{lstlisting}[language=matlab]
% ls_poly_demo
x = linspace(-10, 10, 100);
y = sin(x);
figure; plot(x, y);
c = ls_poly(x, y, 15);
y1 = polyval(c, x);
hold on; plot(x, y1);
axis([-10, 10, -1, 1]);
\end{lstlisting}
运行结果如\autoref{LSpoly_fig1}.
