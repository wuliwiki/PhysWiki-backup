% 高中物理导航
% keys 高中物理|运动学|动力学|科普
% license Usr
% type Map

\subsection{引言}

一切物理理论,都可以分为“运动学”和“动力学”两个部分。

\textbf{运动学}就是\textbf{“描述物体所处状态”的方式},如经典力学认为一个物体所处的状态就是欧几里得空间中的一个点,而在量子力学则认为物体所处的状态是一个向量空间中的向量,等等。有了描述物体所处状态的方式,也就能区分不同的状态,从而描述物体状态的变化规律。在\textbf{经典力学}中,物体的状态被称为\textbf{位置},随着时间流逝,一个物体的位置变化就被称作\textbf{机械运动}。

\textbf{动力学}则是\textbf{物体状态改变的规律},如经典力学中的牛顿三定律,量子力学中的薛定谔方程,等等。牛顿三定律引入了一个\textbf{假想的}概念,\textbf{力}。一个受合外力为$0$的粒子,应该做匀速直线运动\footnote{注意,静止也算匀速直线运动。}。这就排除了孤立粒子一会儿加速、一会儿减速的可能性。

没有任何物理理论是真理,它们都是人类用数学模型模拟自然规律的方式。运动学是理论的基础,而动力学则是理论的核心:有了运动学才能描述物体状态的改变,而动力学规定了这些状态改变的规律。

一个优秀的理论,应该尽可能贴合实验现象,如经典力学能准确预言火星的轨道,指导火星探测器顺利登陆;它还能根据天王星的轨道扰动,预言应该存在一个尚未观测到的大行星,之后根据理论预言的轨道真的找到了海王星。这些对现实宇宙的成功预言,说明它就是一个优秀的理论。

经典力学也有瑕疵,如它预言的水星轨道与实际观测大相径庭,直到出现广义相对论,才让理论计算非常精确地贴合了水星轨道。不能因为经典力学在水星轨道上犯了错就认为它是错的,因为经典力学成功预言了几乎所有日常能想象到的现象;应该说这是经典力学的\textbf{局限性}。也不能说广义相对论推翻了经典力学,因为广义相对论在经典力学成功的地方也同样成功\footnote{理论预言不可能完美,比如牛顿力学计算的火星轨道,和观测差个几厘米没有任何影响。不同的理论对同一个现象的预言也不可能完全一致,如广义相对论计算的火星轨道就和牛顿力学有差异,但这个差异可能比几厘米还小,因此可以认为它们都成功预言了火星轨道。},只是广义相对论成功的地方更多一些,这种情况下我们应该说,广义相对论\textbf{替代}了经典力学。假如经典力学成功的地方广义相对论不成功,反之经典力学不成功的地方广义相对论成功了,那么这两个理论都没有替代对方,而是\textbf{互补}\footnote{量子力学诞生之前,对黑体辐射的描述有两种理论,一个叫维恩定律,一个叫瑞利-金斯定律,前者在短波段和实验吻合,对长波段和实验相去甚远;后者则恰好相反。这两个定律就是\textbf{互补}的。后来出现的普朗克辐射定律则吻合了所有波段的实验结果,因此普朗克定律\textbf{替代}了前两个定律。}。

为什么中学教育不从更成功的广义相对论开始,而是要教授经典力学呢?因为经典力学的思维方式是后续理论的基础,无论是广义相对论还是量子力学,都是从经典力学中受到启发而发展出来的;同时,广义相对论要求学习者熟悉微分几何,对高中生来说要求太高了。教育有自己的规律,并不是一味地教授“更正确”的理论。





\subsection{高中物理导航}


高中物理中所有部分都以经典力学为基础,即所有运动学都是经典力学的“欧几里得空间”模型。虽然课本上提到了相对论\footnote{狭义相对论的运动学是“洛伦兹空间”模型,广义相对论的运动学则是“洛伦兹流形”模型。},但只是作为拓展的孤立知识,与其它部分没有逻辑联系。




\subsubsection{经典力学}

经典力学,又称牛顿力学,其对运动的描述可参见\enref{机械运动基础(高中)}{HSPM01},动力学的基本规律可参见\enref{相互作用(高中)}{HSPM02}和\enref{牛顿运动定律(高中)}{HSPM03}。了解了运动学和动力学的基本概念后,可参见






















