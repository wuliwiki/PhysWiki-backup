% 介质的边界条件
% 边界条件 介质

\begin{issues}
\issueTODO 需要添加A的边界条件;补充相应的证明
\end{issues}
\pentry{麦克斯韦方程组\upref{MWEq}, 麦克斯韦方程组(介质)\upref{MWEq1}, 曲面积分、通量\upref{SurInt}, 线积分\upref{IntL}}

\footnote{本文参考自\cite{GriffE}与周磊教授的讲义,http://fdjpkc.fudan.edu.cn/d200927/2009/0314/c8569a14801/page.htm}
在解决场与势在电、磁介质边界的问题时,微分形式的麦克斯韦方程组不再适用,但积分形式的麦克斯韦方程组仍然适用.(边界处的场可以是不连续的)

本文中,$\sigma_f$指自由面电荷密度,$\sigma$指总面电荷密度,即包括所有自由电荷与因介质极化而产生的感应电荷.在计算时,使用势的边界条件计算,往往比使用场更为简便.

\subsection{电场}

\subsubsection{E场}
\begin{equation}
E^\perp_{above} - E^\perp_{below} = \frac{\sigma}{\epsilon_0}
\end{equation}
\begin{equation}
E^\parallel_{above} - E^\parallel_{below} = 0
\end{equation}

\subsubsection{D场}
\begin{equation}
D^\perp_{above} - D^\perp_{below} = \sigma_f
\end{equation}

\subsubsection{电势 $\varphi$}
\begin{equation}
\varphi_{above}-\varphi_{below}=0
\end{equation}
\begin{equation}
\pdv{\varphi_{above}}{n} - \pdv{\varphi_{below}}{n}  = -\frac{\sigma}{\epsilon_0}
\end{equation}
\begin{equation}
\epsilon_{above}\pdv{\varphi_{above}}{n} - \epsilon_{below}\pdv{\varphi_{below}}{n}  = -\sigma_f
\end{equation}

\subsection{磁场}

\subsubsection{B场}
\begin{equation}
B^\perp_{above} - B^\perp_{below} = 0
\end{equation}
\begin{equation}
\bvec B^\parallel_{above} - \bvec B^\parallel_{below} = \mu_0\bvec K \times \hat n 
\end{equation}
K: 面电流密度
\subsubsection{H场}
\begin{equation}
\bvec H^\parallel_{above} - \bvec H^\parallel_{below} = \bvec K_f \times \hat n 
\end{equation}
\subsubsection{磁标势$\varphi$ (如果有定义)}
\begin{equation}
\varphi_{above}-\varphi_{below}=0
\end{equation}
\begin{equation}
\mu_{above}\pdv{\varphi_{above}}{n} - \mu_{below}\pdv{\varphi_{below}}{n}  = 0
\end{equation}
