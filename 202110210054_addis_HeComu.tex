% 氦原子中的对易算符

\pentry{类氢原子的束缚态\upref{HWF}}
总哈密顿算
\begin{equation}
H = H_1 + H_2 + V_{12} \qquad H_i = K_i + \frac{L_i^2}{2r^2}
\end{equation}
$K_i$ 是纯径向算符(只和 $r_i$ 有关), 所有的 $L$ 都是纯角向算符(只和角度有关\upref{SphAM}). 不同电子的任意算符可对易, 纯径向算符和纯角向算符可对易.

所以 $L_1^2, L_2^2, L^2, L_z$ 两两对易, $L_1^2, L_2^2, L_{1z}, L_{2z}$ 也两两对易.

最复杂的是 $V_{12}$, 既有径向也有角向, 且耦合两个电子(\autoref{HeTDSE_eq5}~\upref{HeTDSE})
\begin{equation}
[V_{12}, L^2] = [V_{12}, M] = 0
\end{equation}
从经典力学的角度来看这是成立的.

已经数值验证: $\mel*{l'_1,l'_2,L',M'}{\mathcal Y_{l,l}^{0,0}}{l_1,l_2,L,M} = \delta_{L,L'}\delta_{M,M'}$. 说明 $H$ 只会耦合不同的 $l_1,l_2$ 而不会耦合不同的 $L,M$. $H$ 的其他部分不会耦合任何不同的分波. 根据\autoref{Commut_the1}~\upref{Commut} 这说明 $H,L^2,L_z$ 两两对易, 可以在每个 $L,M$ 本征子空间中分别求解能量本征值.

但是注意 $H$ 和 $L_1, L_2, L_{1z}, L_{2z}$ 之一并不对易, 因为 $H$ 会耦合不同的 $(l_1,l_2)$ 组合. 从经典力学上这也符合直觉, 因为两个粒子的角动量不各自守恒.



