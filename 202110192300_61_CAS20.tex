% 中国科学院 2020 年考研普通物理
% 中国科学院|考研|普通物理

\begin{issues}
\issueDraft
\issueTODO
\end{issues}



\subsection{单项选择题(共32分,每小题4分)}
\begin{enumerate}
\item 简谐振动微分方程$\ddot{x}+{\omega}^2{x}=0$满足的变换不变性不包括\\
A. 时间反演\\
B. 时间平移\\
C. 空间反射\\
D. 空间加速平移\\

\item 在高空走钢丝者手持长杆作为辅助,所应用的物理原理是\\
A. 长杆很重,走钢丝者手持长杆有一种踏实的感觉;\\
B. 长杆的转动惯量对运动的平衡作用;\\
C. 长杆对分的平衡能力;\\
D. 走钢丝者的体重加上长杆的重量,增加了运动的惯性.\\

\item 对于惯性系$S$,一飞船以船头向前且速度为0.8$c$($c$为光速)匀速运动.飞船的静长为$l$,飞船的船头朝船尾发出一束光,并被位于船尾的接收器接收到,$S$系中的观察者认为从船头发出光到船尾接收到光的时间间隔为\\
A. \\
B. \\
C. \\
D. \\

\item 三个点电荷分别位于边长为$a$的正三角形的三个顶点,它们的电荷量分别为$zzzzzzzzzzz$,$zzzzzzzzzz$和$-4q$.真空介电常数为$\epsilon_{0}$,则这个系统的总静电能为(设相距无穷远时相互作用能为零)\\
A. $-5q^{2}12\pi \epsilon_{0} a$\\
B. $-5q^{2}14\pi \epsilon_{0} a$\\
C. $-7q^{2}14\pi \epsilon_{0} a$\\
D. $-7q^{2}12\pi \epsilon_{0} a$\\

\item 将一点电荷从无穷远处沿径向逐渐靠近到一带电的均匀材质导体球表面,下列论述正确的时\\
A. 如果点电荷与导体球所带总电荷同号,移动过程中两者之间总是相互排斥;如果点电荷与导体球所带总电荷异号,移动过程中两者之间总是相互吸引\\
B. 如果点电荷与导体球所带总电荷同号,移动过程中两者之间从相互排斥变成相互吸引;如果点电荷与导体球所带总电荷异号,移动过程中两者之间总是相互吸引\\
C. 如果点电荷与导体球所带总电荷同号,移动过程中两者之间总是相互排斥;如果点电荷与导体球所带总电荷异号,移动过程中两者之间从相互吸引变成相互排斥\\
D. 根据电量大小的不同,(A)、(B)或(C)三种情况都可能出现\\

\item 有一半径为$R$的单匝圆线圈,通以顺时针电流$I$.若将该导线弯成匝数$N=2$的平面线圈,导线总长度保持不变,通以同样大小的电流$I$,并且在两匝线圈中电流方向仍同为顺时针,则线圈中心的磁感应强度和线圈的磁矩分别是原来的\\
A. 4倍和1/8\\
B. \\
C. \\
D. \\
\end{enumerate}