% 相对论补全和推广
% keys 相对论|补全|四矢量|数量流
% license Usr
% type Tutor

\pentry{协变性和不变性\nref{nod_CoIn}}{nod_029b}
在狭义相对论中,Lorentz变换是给出了两个参考系间的坐标变换关系。特别地,给出了惯性系间的坐标变换关系。狭义相对论的假设之一是,物理定律在所有惯性系中都是相同的。根据协变性和不变性的定义(\autoref{def_CoIn_1}),这就是说,描述狭义相对论的基本方程被假设在Lorentz变换下是协变的,因而狭义相对论的物理在Lorentz变换下是不变的。因此,所有的物理量在Lorentz变换下必须以一种定义良好的方式变换。而在非相对论下的物理量在相对论的框架下必须进行补全,物理定律应当推广到相对论情形。
\subsection{物理量的相对论补全}
Lorentz变换是四维时空中一个基底到另一个基底对应的转换矩阵。因此,四维时空的矢量坐标在不同基底下的转换关系必须由Lorentz变换连接。$\dd x^\mu=(\dd t,\dd{\vec x})$ 是四维时空中的矢量,即在相对论框架下,三维空间中的矢量(简称3-矢量) $\dd{\vec x}$ 必须和三维空间的标量(简称3-标量) $\dd t$ 一起统一成四维空间的矢量(简称4-矢量),才是狭义相对论下具有良好定义的矢量。

\subsubsection{速度补全}
已知3-速度定义为 $\vec v_N:=\dv{\vec x}{t}$($N$ 可理解为Newton定义的),这是一个3-矢量除以一个3-标量,而 $\dd t$ 恰巧是4-矢量的时间分量。容易验证,$\bvec v_N$ 在Lorentz变换下并不像4-矢量任何分量一样变换。

相反,考虑3-矢量 $\dd{\vec x}$ 除以一个Lorentz标量(在Lorentz变换下数值不变的量)$\dd\tau$ ($\dd \tau^2:=-\eta_{\mu\nu}\dd x^{\mu}\dd x^\nu=\dd t^2-\dd {\vec x^2}$)。显然,这样得到的量是4-矢量 $v^\mu:=\dv{x^\mu}{\tau}$ 的空间分量 $v^i$。因此,若定义 $\vec v:=\dv{\vec x}{\tau}$,则 $\vec v$ 的相对论补全是4-矢量 $v^\mu$。

需要澄清的是,$\vec v_N$ 和 $\vec v$ 是两个不同的量。出现在因子 $\sqrt{1-v_N^2}$ 中的是$\vec v_N$ (不是 $\vec v$)。当 $\vec v_N^2\leq1$时,$\vec v^2$ 取值范围遍及 $0$ 到 $\infty$。容易造成混乱的是,人们通常省略掉 $\vec v_N$ 中的下标 $N$,这也是我们常采用的标准做法,只需特别注意这里的提醒。

\subsubsection{动量补全和动量守恒推广}
质量 $m$ 乘以速度 $v^\mu$,得到质量为 $m$ ,速度为 $v^\mu$ 的粒子的4-动量 $p^\mu:=mv^\mu=m\dv{x^\mu}{\tau}$。非相对论的3-动量守恒 $\vec p_N=m\vec v_N$ 强烈表明4-动量 $p^\mu$ 也是守恒的。事实上由 $\vec p_N$ 为常矢量,得 $\vec v_N$ 为常矢量(因为是Lorentz标量 $m$ 是常量),从而 
\begin{equation}
p^\mu=\qty(\frac{m}{\sqrt{1-v_N^2}},\frac{m\vec v_N}{\sqrt{1-v_N^2}})~
\end{equation}
即为常矢量。上式即是3-动量守恒在相对论4-动量守恒的推广。

展开 $p^0=\frac{m}{\sqrt{1-v_N^2}}$,得 $p^0=m+\frac{1}{2}m\vec v_N^2+o(v_N^2)$,其中 $o(v_N^2)$ 表示 $v_N^2$ 的高阶无穷小量。明显的,第二项是Newton动能,因此 $p^0$ 要理解为质量为 $m$ 的粒子的能量。这表明,即使是静止粒子也具有能量,若恢复光速 $c$(以上讨论已经假定 $c=1$了),则静止粒子的能量就是 $mc^2$。

使用 $\dd\tau^2$ 的定义,得
\begin{equation}\label{eq_Comple_1}
p^2=\eta_{\mu\nu}p^\mu p^\nu=m^2\eta_{\mu\nu}\dv{x^\mu}{\tau}\dv{x^\nu}{\tau}=-m^2.~
\end{equation}
该式就是所谓的\textbf{质壳条件}(mass shell condition),它给出了 $p$ 的一个约束。\autoref{eq_Comple_1} 对任意粒子(包含 $m=0$),在任一参考系下都是不变的。


\subsubsection{粒子数密度补全}

现在我们考虑粒子数密度 $n(t,\vec x)$ (单位体积中的粒子数)的相对论补全。在3维空间的旋转变换下,$n(t,\vec x)$ 是旋转不变量,因为旋转不会改变体积和体积内的粒子数。因此,一个天真的想法是粒子数密度的相对论补全是一个Lorentz标量。然而,事实上并非如此。

从





















