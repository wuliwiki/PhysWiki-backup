% 约翰内斯·开普勒(综述)
% license CCBYSA3
% type Wiki

本文根据 CC-BY-SA 协议转载翻译自维基百科\href{https://en.wikipedia.org/wiki/Johannes_Kepler}{相关文章}。

\begin{figure}[ht]
\centering
\includegraphics[width=6cm]{./figures/cc1aaaf12433e376.png}
\caption{由奥古斯特·科勒(August Köhler)于约1910年绘制的肖像,基于1627年的原作。} \label{fig_KPL1_2}
\end{figure}
约翰内斯·开普勒(Johannes Kepler,/ˈkɛplər/;德语:[joˈhanəs ˈkɛplɐ, -nɛs -] ⓘ;1571年12月27日–1630年11月15日)是德国天文学家、数学家、占星家、自然哲学家以及音乐作家。他是17世纪科学革命的关键人物,以行星运动定律最为人知,并且以《新天文学》(Astronomia nova)、《世界和谐论》(Harmonice Mundi)和《哥白尼天文学概要》(Epitome Astronomiae Copernicanae)等著作影响了以艾萨克·牛顿为代表的科学家,为牛顿的万有引力理论提供了基础之一。开普勒的工作具有多样性和深远影响,使他成为现代天文学、科学方法、自然科学和现代科学的奠基人之一。他被誉为“科幻小说之父”,因为他的小说《梦境》(Somnium)。 

开普勒曾是格拉茨一所神学院的数学教师,在那里他成为了汉斯·乌尔里希·冯·埃根贝格(Prince Hans Ulrich von Eggenberg)王子的合作者。后来,他成为天文学家第谷·布拉赫(Tycho Brahe)在布拉格的助手,并最终成为神圣罗马帝国皇帝鲁道夫二世及其继任者马提亚斯和斐迪南二世的皇家数学家。他还曾在林茨教授数学,并且是沃尔斯坦将军的顾问。此外,开普勒在光学领域做出了基础性贡献,被誉为现代光学之父,尤其以《光学天文学》为代表。他还发明了改进版的折射望远镜——开普勒望远镜,成为现代折射望远镜的基础,同时改进了伽利略·伽利莱的望远镜设计,伽利略在他的著作中提到了开普勒的发现。

开普勒生活在一个天文学和占星学没有明确界限的时代,但天文学(作为自由艺术中的一门数学分支)和物理学(作为自然哲学的一门分支)之间却有着明显的区分。开普勒还将宗教论证和推理融入到他的工作中,受宗教信仰的激励,他认为上帝按照可以通过理性之光理解的智能计划创造了这个世界。开普勒将他的新天文学描述为“天体物理学”,作为“对亚里士多德《形而上学》的探索”,并作为“对亚里士多德《天论》的补充”,通过将天文学视为普遍数学物理学的一部分,开普勒彻底改造了古代的物理宇宙学传统。
\subsection{早期生活}  
\subsubsection{童年(1571年–1590年)}
开普勒的出生地,魏尔德施塔特  
开普勒于1571年12月27日出生在魏尔德施塔特的自由帝国城市(现为德国巴登-符腾堡州斯图加特地区的一部分)。他的祖父塞巴尔德·开普勒曾是该市的市长。到约翰内斯出生时,开普勒家族的财富已开始衰退。他的父亲海因里希·开普勒以雇佣兵的身份维持生计,在约翰内斯五岁时离开了家人,据信他在荷兰的八十年战争中去世。他的母亲凯瑟琳娜·古尔登曼是位酒馆老板的女儿,也是一个治疗师和草药师。约翰内斯有六个兄弟姐妹,其中两个兄弟和一个姐妹活到了成年。由于早产,他自称小时候身体虚弱且多病。然而,他常常在祖父的酒馆里给旅客留下深刻的印象,展现出他非凡的数学才能。[22]

他在很小的时候就接触了天文学,并发展出了对它的浓厚兴趣,这份热情贯穿了他的一生。六岁时,他观察到了1577年的大彗星,并写道他“被母亲带到一个高处去观看它。”[23] 1580年,九岁时,他又观察到了一个天文现象——月全食,他记得“被叫到户外”去观看,月亮“显得非常红。”[23] 然而,童年的天花让他留下了视力虚弱和双手残疾的问题,这限制了他在天文观测方面的能力。[24]
\begin{figure}[ht]
\centering
\includegraphics[width=6cm]{./figures/73a80a0cf2cfd6a7.png}
\caption{开普勒的出生地,位于韦尔德施塔特} \label{fig_KPL1_1}
\end{figure}
1589年,在经历了文法学校、拉丁学校和毛尔布龙神学院的学习后,开普勒进入了图宾根大学的图宾根神学院。在那里,他在维图斯·穆勒的指导下学习哲学,[25] 并在雅各布·赫尔布兰德(菲利普·梅兰希顿在维滕贝格的学生)的指导下学习神学。赫尔布兰德也曾教过迈克尔·梅斯特林,直到1590年他成为图宾根大学的校长。[26] 他证明自己是一个出色的数学家,并因精湛的占星术而赢得了声誉,为同学们制作星座图。在图宾根大学担任数学教授的迈克尔·梅斯特林(1583年至1631年)指导下,他学习了托勒密的行星运动体系和哥白尼的行星运动体系。那时,他成为了哥白尼主义者。在一次学生辩论中,他从理论和神学角度捍卫了日心说,主张太阳是宇宙中运动力的主要来源。[27] 尽管他渴望成为路德宗教会的牧师,但由于其信仰与《和协议》相悖,他未能获得圣职。[28] 在学业接近尾声时,开普勒被推荐到格拉茨的基督教学校担任数学和天文学教师。他在1594年4月接受了这个职位,当时他22岁。[29]
\subsubsection{格拉茨(1594–1600年)}
\begin{figure}[ht]
\centering
\includegraphics[width=8cm]{./figures/0f766dbc7312ae9f.png}
\caption{小时候,开普勒目睹了1577年的大彗星,这一事件引起了全欧洲天文学家的关注。} \label{fig_KPL1_3}
\end{figure}
在结束他在图宾根的学业之前,开普勒接受了在格拉茨(现位于奥地利斯蒂里亚州)的新教学校教数学的工作,接替乔治·斯塔迪乌斯的职位(1594–1600年期间)。在这段时间里,他发布了许多官方的日历和预言,这些增强了他作为占星家的声誉。尽管开普勒对占星术有着复杂的感情,并且贬低了许多占星家的传统做法,他仍然深信宇宙与个体之间存在某种联系。他最终将自己在学生时代的一些想法写成了《宇宙神秘学》(1596年),这本书在他到达格拉茨一年多后出版。

1595年12月,开普勒遇到了巴巴拉·穆勒(Barbara Müller),一位23岁的寡妇(已经结过两次婚)并且有一个年轻的女儿,名叫瑞吉娜·洛伦茨(Regina Lorenz),他开始追求她。穆勒是她已故丈夫们遗产的继承人,也是一个成功的磨坊主的女儿。穆勒的父亲约布斯最初反对这桩婚事。尽管开普勒继承了祖父的贵族身份,但开普勒的贫困使得他成为一个不被接受的婚配对象。在开普勒完成《宇宙神秘学》的工作后,约布斯最终同意了婚事,但在开普勒去处理出版细节期间,这段婚约几乎破裂。然而,帮助撮合这桩婚姻的 protestant 官员们施加了压力,迫使穆勒家族履行他们的承诺。巴巴拉和约翰内斯于1597年4月27日结婚。

在婚后的头几年,开普勒夫妇有了两个孩子(海因里希和苏珊娜),但两人都在婴儿时期去世。1602年,他们有了一个女儿(苏珊娜);1604年,生了一个儿子(弗里德里希);1607年,又有了另一个儿子(路德维希)。