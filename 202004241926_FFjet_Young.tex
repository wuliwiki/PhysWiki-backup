% 杨氏双缝干涉实验

用两个点波源作光的干涉实验的典型代表, 是杨氏实验.杨氏实验的装置如下图所示,
\begin{figure}[ht]
\centering
\includegraphics[width=8cm]{./figures/Young_1.png}
\caption{杨氏双缝实验} \label{Young_fig1}
\end{figure}
\begin{figure}[ht]
\centering
\includegraphics[width=7cm]{./figures/Young_2.png}
\caption{杨氏双缝实验} \label{Young_fig2}
\end{figure}
在普通单色光源前面放一个开有小孔$S$的屏,作为单色点光源.在$S $的照明范围内再放一个开有两个小孔$S_1,S_2$的屏.按惠更斯原理,$S$将作为两个次波源向前发射次波(球面波),形成交叠的波场.在较远的地方放置一接收屏,屏上可以观测到一组几乎是平行的直线条纹\autoref{Young_fig1}(b)\autoref{Young_fig2}(b) .

但这样的亮度太低了.为了提高干涉条纹的亮度,实际中$S,S_1,S_2$用三个互相平行的狭缝,并且可不用屏幕接收,而代之以目镜直接观测.激光出现以后,人们可以用氨氖激光束直接照明双孔,在屏幕上即可获得一套相当明显的干涉条纹,供许多人同时观看.现在来分析,用普通光源做杨氏实验时,由双孔出射的两束光波之间的相位差.设$SS_1=R_1,SS_2=R_2$,用$\varphi_0$代表点波源$S$的初相位,则次波源$S_1,S_2$的初相位分别为
\begin{equation}
\varphi_{10}=\varphi_{0}+\frac{2 \pi}{\lambda} R_{1}, \quad \varphi_{20}=\varphi_{0}+\frac{2 \pi}{\lambda} R_{2}
\end{equation}
从而
\begin{equation}
\varphi_{10}-\varphi_{20}=\frac{2 \pi}{\lambda}\left(R_{1}-R_{2}\right)
\end{equation}
由此可见,两次波之间的相位差与$\varphi_0$无关,即使中$\varphi_0$变了,相位差中$\varphi_{10}-\varphi_{20}$也不变.下面来作一些计算.

令双孔间距为$d$, 屏幕与双孔屏间的距离为$D$,屏幕上横向观测范围为$X$,我们设$d^{2} \ll D^{2}$(这被称为\textbf{远场条件}), $X^{2} \ll D^{2}$(傍轴条件).设$S_1$、$S_2$离$S$的距离相等,即$R_1=R_2$, 从而$\varphi_{10}=\varphi_{20}$,可取二者皆为$0$.如\autoref{Young_fig1},从线段$S_1S_2$的中点$O$作$z$轴垂直于双孔屏和接收屏.设接收屏上点$P$的横向距离为$X$,$OP$与$z$轴的夹角为$\theta$.在远场条件下可认为$S_1P\parallel S_2P$
与z 轴的夹角为(J. 在远场条件下可认为S, P II S2P , 在傍轴条件下可认为
OP 也与它们平行.自s, 作OP 和S2P 的垂线交S2P 于N, 则瓦讨近似等
于光程差: