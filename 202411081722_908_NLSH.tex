% 能量守恒定律(综述)
% license CCBYSA3
% type Wiki

本文根据 CC-BY-SA 协议转载翻译自维基百科\href{https://en.wikipedia.org/wiki/Conservation_of_energy}{相关文章}。

能量守恒定律表明,孤立系统的总能量保持不变;即它随着时间的推移是守恒的。[1] 对于封闭系统,该原理表明系统内的总能量只能通过能量的进出而改变。能量既不能被创造,也不能被销毁;它只能从一种形式转化或转移为另一种形式。例如,当一根炸药爆炸时,化学能转化为动能。如果把爆炸中释放的所有形式的能量相加,如碎片的动能和势能以及热和声,就可以得到炸药燃烧过程中化学能的精确减少量。

在经典物理中,能量守恒与质量守恒是不同的。然而,狭义相对论表明质量和能量是相互关联的,反之亦然,其关系由方程 \( E = mc^2 \) 表示,即质能等价,现在科学认为整体的质能是守恒的。从理论上讲,这意味着任何具有质量的物体本身可以转化为纯能量,反之亦然。然而,人们认为这只有在最极端的物理条件下才可能发生,例如宇宙大爆炸后的极短时间内,或黑洞发射霍金辐射时。

根据驻行动原理,通过诺特定理可以严格证明能量守恒是连续时间平移对称性的结果;即物理定律不随时间改变这一事实。

能量守恒定律的一个结果是第一类永动机不可能存在;也就是说,没有外部能量供应的系统不能向其周围环境提供无限的能量。[2] 根据能量的定义,能量守恒在宇宙尺度上可以被认为在广义相对论中被违背。[3]
\subsection{历史}
早在公元前550年左右,米利都的泰勒斯等古代哲学家就隐约意识到某种构成万物的基础物质的守恒。然而,他们的理论与我们今天所知的“质能”并无直接关联(例如,泰勒斯认为这一基础物质是水)。恩培多克勒(公元前490–430年)写道,在他由四种根本元素(土、气、水、火)构成的宇宙体系中,“没有事物生成或毁灭”;[4] 相反,这些元素经历不断的重新排列。另一方面,伊壁鸠鲁(约公元前350年)认为宇宙中的一切都由不可分割的物质单元构成——即原子的古老先驱——他也有一些关于守恒的观点,认为“万物的总和一直是现在的样子,并将永远保持如此”。[5]

1605年,佛兰德科学家西蒙·斯蒂文基于永动机不可能存在的原则,解决了若干静力学问题。

1639年,伽利略发表了对几种情况的分析——包括著名的“中断摆”——可以用现代语言描述为势能与动能的守恒转换。他指出,运动物体上升的高度等于其下降的高度,并据此推断出惯性的概念。这个观察的显著之处在于,在无摩擦表面上,运动物体上升的高度与表面的形状无关。

1669年,克里斯蒂安·惠更斯发表了他的碰撞定律。在他列出的碰撞前后保持不变的量中,包括物体线性动量的总和和动能的总和。然而,当时尚未理解弹性碰撞和非弹性碰撞之间的差别。这导致后来的研究者在争论哪种守恒量更为基础。在他的著作《摆钟》中,他更清晰地阐述了运动物体上升高度的概念,并将此观点与永动机不可能性联系起来。惠更斯对摆运动动力学的研究基于一个简单的原则:重物的重心无法自我提升。
\begin{figure}[ht]
\centering
\includegraphics[width=6cm]{./figures/3d8704027bff5214.png}
\caption{戈特弗里德·莱布尼茨} \label{fig_NLSH_1}
\end{figure}
在1676年至1689年间,戈特弗里德·莱布尼茨首次尝试对与运动(动能)相关的能量进行数学表述。基于惠更斯的碰撞研究,莱布尼茨注意到在许多机械系统中(由若干质量为 \( m_i \) 的物体组成,每个物体的速度为 \( v_i \)),
\[
\sum _{i}m_{i}v_{i}^{2}~
\]
在各质量不相互作用的情况下是守恒的。他称这个量为系统的“活力”或“生命力”。该原理精确地描述了无摩擦情况下动能近似守恒的情形。当时许多物理学家,包括艾萨克·牛顿在内,认为动量守恒(在有摩擦的系统中也成立)是定义的活力:
\[
\sum _{i}m_{i}v_{i}~
\]
后来证明,在适当条件下(如弹性碰撞),这两个量可以同时守恒。

1687年,艾萨克·牛顿发表了《自然哲学的数学原理》,其中提出了他的运动定律。该书围绕力和动量的概念展开。然而,研究者们很快认识到,书中提出的原则虽然适用于点质量,但不足以解决刚体和流体的运动问题,还需要其他原则。

到1690年代,莱布尼茨主张活力守恒和动量守恒削弱了当时流行的交互主义二元论哲学学说。(在19世纪,随着人们对能量守恒的理解加深,莱布尼茨的基本论点得到了广泛认可。一些现代学者继续特别通过守恒理论来批判二元论,而另一些学者则将这一论点纳入关于因果封闭性的更普遍论证中。)[6]
\begin{figure}[ht]
\centering
\includegraphics[width=6cm]{./figures/ab34f1710db0b732.png}
\caption{丹尼尔·伯努利} \label{fig_NLSH_2}
\end{figure}
活力守恒定律由父子二人约翰·伯努利和丹尼尔·伯努利所倡导。约翰在1715年全面阐明了静力学中虚功原理的普遍性,而丹尼尔在他1738年出版的《流体动力学》一书中,基于这一单一的活力守恒原则。丹尼尔对流水活力损失的研究使他提出了伯努利原理,该原理断言损失与流体动力压力变化成正比。丹尼尔还为水力机械提出了功和效率的概念;他提出了气体的动理论,并将气体分子动能与气体温度联系起来。

大陆物理学家对活力的关注最终导致了控制力学的驻定性原理的发现,如达朗贝尔原理、拉格朗日力学和哈密顿力学的表述。