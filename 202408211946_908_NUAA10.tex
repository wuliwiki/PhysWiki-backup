% 南京航空航天大学 2007 量子真题答案
% license Usr
% type Note

\textbf{声明}:“该内容来源于网络公开资料,不保证真实性,如有侵权请联系管理员”

\subsection{一}
\subsubsection{1.}
解:设厄米函数$\hat{F}$的本征值为$\lambda$,$\hat{F}\psi = \lambda \psi$\\
在厄米算符定义式$$\int \psi^* \hat{F} \phi d\tau = \int (\hat{F} \psi)^* \phi d\tau~$$中,令$\phi=\psi$,则$$\lambda \int \psi^* \varphi d\tau = \lambda^* \int \psi^* \psi d\tau~$$\\
$\therefore \lambda=\lambda^*$得证。
\subsubsection{2.}
解:它取极小值的条件为
$$\frac{\partial E}{\partial \overline{(\Delta x)^2}} = 0~$$
由此得出
$$\overline{(\Delta x)^2} = \frac{\hbar}{2 m \omega}~$$
用此值代入(3)式, 可知
$$E \geq \frac{1}{2} \hbar \omega~$$
所以讲振子基态能量
$$E = \frac{1}{2} \hbar \omega~$$
由于一维谐振子势具有对坐标原点的反射对称性,我们有
$$ x = 0, \quad  p = 0~$$
因而
$$ \overline{\Delta x^2} =\overline{x^2} -\overline{x}^2 =\overline{x^2}~ $$
$$ \overline{\Delta p^2} =\overline{p^2} - \overline{p}^2 =\overline{p^2}~ $$
所以在能量本征态下
$$ E = \frac{\overline{p^2}}{2m} + \frac{1}{2} m \omega^2\overline{x^2}= \frac{\overline{\Delta p^2}}{2m} + \frac{1}{2} m \omega^2 \overline{\Delta x^2}~$$
按不确定性关系
$$ (\overline{\Delta x)^2}.\overline{(\Delta p)^2}\geq \frac{\hbar^2}{4}~$$
所以
$$ E \geq \frac{\hbar^2}{8m (\Delta x)^2} + \frac{1}{2} m \omega^2 \overline{(\Delta x)^2}~$$
\subsection{二}
解:一个质量为 $m$ 的粒子在一维无限深势阱 $(0 \leq x \leq a)$ 中运动, $t = 0$ 时刻的初始波函数为
$$\psi(x, 0) = \sqrt{\frac{8}{5a}} \left(1 + \cos \frac{\pi x}{a}\right)\sin \frac{\pi x}{a}~$$\\
(1) 在后来某一时刻 $t_0$ 的波函数是什么?\\
(2) 体系在 $t = 0$ 和 $t = t_0$ 时的平均能量是多少?\\
(3) 在 $t = t_0$ 时, 在势阱左半部 $(0 \leq x \leq \frac{a}{2})$ 发现粒子的概率是多少?\\
解:(1) 无穷深方势阱中粒子的定态波函数为 $\psi_n = \sqrt{\frac{2}{a}} \sin \frac{n\pi x}{a}$, 相应的能为 
$$E_n = \frac{n^2 \pi^2 \hbar^2}{2ma^2}~$$
将 $t = 0$ 时刻粒子的初态波函数用这些定态波函数展开
$$\psi(x, 0) = \sum_n A_n \psi_n~$$
$$\psi(x, 0) = A_1 \psi_1 + A_2 \psi_2~$$
其中 
$$A_1 = -\sqrt{\frac{4}{5}} A_2 = \sqrt{\frac{1}{5}}~$$
$$\psi(x, 0) = \sqrt{\frac{4}{5}}\left(\sqrt{\frac{2}{a}}\sin \frac{\pi x}{a}\right)+\frac{1}{5}\left(\sqrt{\frac{2}{a}}\sin \frac{\pi x}{a}\right)~$$
$t=t_0$时刻粒子的波函数
\begin{align}
\psi(x,t_0) &= e^{-\frac{iE_1t_0}{\hbar}} \psi(x, 0) = e^{-\frac{iE_1t_0}{\hbar}} A_1 \psi_1 + e^{-\frac{iE_2t_0}{\hbar}} A_2 \psi_2 \\
&= e^{-\frac{iE_1t_0}{\hbar}} \sqrt{\frac{4}{5}}\left(\sqrt{\frac{2}{a}}\sin \frac{\pi x}{a}\right)+ e^{-\frac{iE_2t_0}{\hbar}}\frac{1}{5}\left(\sqrt{\frac{2}{a}}\sin \frac{\pi x}{a}\right)~
\end{align}
(2) $t=0$ 时, $\langle E \rangle = \int \psi^*(x, 0) \hat{H} \psi(x, 0) \,dx = \frac{4}{5} E_1 + \frac{1}{5} E_2 = \frac{4\pi^2 \hbar^2}{5ma^2} t - t_0$时,
同理可得, 能量与t=0时相同。
(3)
$$
\begin{aligned}
\int_{0}^{\infty} \psi^*(x, t_0) \psi(x, t_0) \, dx &= \int_{0}^{a} \left( e^{-iE_1 t_0/\hbar} A_1 \psi_1 + e^{-iE_2 t_0/\hbar} A_2 \psi_2 \right) \left( e^{-iE_1 t_0/\hbar} A_1 \psi_1 + e^{-iE_2 t_0/\hbar} A_2 \psi_2 \right)^* dx \\
&= \int_{0}^{a} \left( A_1^2 \psi_1^2 + A_2^2 \psi_2^2 + 2 A_1 A_2 \psi_1 \psi_2 \cos \left( \frac{E_2 - E_1 t_0}{\hbar} \right) \right) dx \\
&= \int_{0}^{a} \left( \frac{8}{5a} \sin^2 \frac{\pi x}{a} + \frac{2}{5a} \sin^2 \frac{2\pi x}{a} + \frac{8}{5a} \sin \frac{\pi x}{a} \sin \frac{2\pi x}{a} \cos \frac{3\pi^2 \hbar t_0}{2ma^2} \right) dx \\
&= \int_{0}^{a} \left[\frac{4}{5a} \left( 1 - \cos \frac{2\pi x}{a} \right) + \frac{1}{5a} \left( 1 - \cos \frac{4\pi x}{a} \right)\\
+ \frac{4}{5a} \left(\cos \frac{\pi x}{a} \cos \frac{3\pi x}{a}\right) \cos \frac{3\pi^2 \hbar t_0}{2ma^2} \right] dx \\
&= \frac{1}{2} + \frac{16}{15\pi} \cos \frac{3\pi^2 \hbar t_0}{2ma^2}.
\end{aligned}~$$
\subsection{三}
解:\begin{equation}
H' = \lambda \delta \left( x - \frac{a}{2} \right)~
\end{equation}

\begin{equation}
E_n^{(0)} = n^2\pi^2 \hbar^2/2\mu a^2, \quad \psi_n^{(0)} = \sqrt{\frac{2}{a}} \sin \frac{n \pi}{a}x~
\end{equation}

求到二级,矩阵元一般形式
\begin{equation}
\langle n | H' | 1 \rangle = \frac{2}{a} \int_0^a \sin \frac{n \pi x}{a} \lambda \delta \left( x - \frac{a}{2} \right) \sin \frac{l\pi x}{a} dx = \frac{2\lambda}{a} \sin \frac{n \pi}{2} \sin \frac{l\pi}{2}~
\end{equation}

\textbf{基态:} $n=1$,一级修正
\begin{equation}
E_1^{(1)} = \langle 1 | H' | 1 \rangle = \frac{2\lambda}{a}~
\end{equation}

\textbf{二级修正}
\begin{equation}
E_1^{(2)} = \sum_{l \neq 1} \frac{|\langle 1 | H' | l \rangle |^2}{E_1^{(0)} - E_l^{(0)}} = \sum_{l \neq 1} \frac{4\lambda^2}{\pi^2 l^2} \times \frac{\sin^2 \frac{l \pi}{2}}{E_1^{(0)}(1 - l^2)} = \frac{8 \mu \lambda^2}{\pi^2 \hbar^2} \sum_{l \neq 1} \frac{1 - \cos l \pi}{2(l^2 - 1)}
= \frac{8 \mu \lambda^2}{\pi^2 \hbar^2} \sum_{l \neq 1} \frac{1 - (-1)^l}{2(l^2 - 1)}~
\end{equation}\\
(1)当$l$为偶数时,$1-(-1)'=0$,这时$E_1^{2}= 0$\\
(2)当 $l$ 为奇数时,令 $l=2k+1, k=1,2,3,\dots,$ 上式给出
\begin{equation}
E_1^{(2)} = -\frac{8\mu\lambda^2}{\pi^4 \hbar^2} \sum_{k=1}^\infty \frac{1}{4k(k+1)} = -\frac{2 \mu \lambda^2}{\pi^4 \hbar^2} \sum_{k=1}^\infty \left(\frac{1}{k} - \frac{1}{k+1}\right) = -\frac{2 \mu \lambda^2}{\pi^4 \hbar^2}~
\end{equation}

所以
\begin{equation}
E_1 = \frac{\pi^2 \hbar^2}{2\mu a^2} + \frac{2\lambda}{a} + \frac{2 \mu \lambda^2}{\pi^2 \hbar^2}~
\end{equation}

由 $|E_1^{(2)}| \ll |E_1^{(1)}| \ll |E_1^{(0)}|$,可得 $\lambda \ll \pi^2 \hbar^2/\mu a$
\subsection{四}
解:
偶极跃迁,$\mathbf{H'} \propto X,Y, \text{或} Z$,视偏振方向而定。

由
\begin{equation}
\begin{aligned}
    &x = r\sin\theta \cos\varphi \\\\
    &y = r\sin\theta \sin\varphi \\\\
    &z = r\cos\theta 
\end{aligned}~
\end{equation}

和球谐函数的递推关系
$$\cos\theta Y_{lm}(\theta, \varphi) = a_1 Y_{l-1,m} + a_2 Y_{l+1,m}~$$

$$\sin \theta Y_{lm}(\theta, \varphi) = b_1 Y_{l-1,m\pm1} + b_2 Y_{l+1,m\pm1}~$$

可得极化矩阵元$H'_{n\epsilon m, n'\epsilon'm'} \neq 0$ 的条件:
$$\Delta l = \pm1, \Delta m = 0, \pm1~$$\\
径向函数不构成选择定则, $\therefore \text{无}\Delta n$ 的规则。\\
又由自旋李矩阵的正交归一性:} $\alpha = \\beta = 0, \alpha' = 1 \text{和} \beta' = 1,$\\
可得自旋选择定则:
$\Delta S = 0$
\subsection{五}
解: (1) 设  $\hat{T} = A \hat{s}_y + B \hat{s}_z$,则在$\hat{s}_x, \hat{s}_y, \hat{s}_z$ 表象中有
$$\hat{T} = \frac{h}{2}\begin{pmatrix}0 & B - iA \\\\B + iA & 0\end{pmatrix}~$$

设本征值为$\lambda \frac{h}{2}$, 有
$$\det\begin{pmatrix}B - \lambda & -iA \\\\iA & -B - \lambda\end{pmatrix}= 0 \Rightarrow \lambda = \pm \sqrt{A^2 + B^2}~$$
设归一化的本征态为$(ab),a^2+b^2=1$则由本征方程
$$\begin{pmatrix}B  & -iA \\\\iA & -B \end{pmatrix}\begin{pmatrix}a \\\\b\end{pmatrix}= \lambda\begin{pmatrix}a \\\\b\end{pmatrix}~$$

可以解出本征态为
$$\Psi_{\pm} = \frac{1}{\sqrt{2}}\begin{pmatrix}\frac{1}{\sqrt{A^2 + (B \pm \sqrt{A^2 + B^2})^2}} \\\\\frac{iA}{B \pm \sqrt{A^2 + B^2}}\end{pmatrix}~$$

(2) 在$\hat{s}_x, \hat{s}_y, \hat{s}_z$的表象中, $\hat{s}_z$ 的本征态为(见 7.2 题)
$$\hat{s}_z = \pm \frac{h}{2} = \pm \frac{1}{\sqrt{2}}\begin{pmatrix}1 \\\\1\end{pmatrix}~$$
