% 双线性函数
% license Usr
% type Tutor
本文采用爱因斯坦求和约定。

\begin{definition}{}
设$V$是域$\mathbb F$上的线性空间,映射$f:V\times V\rightarrow\mathbb F$若满足对于任意$\bvec x,\bvec y,\bvec z \in V,a,b\in \mathbb F$有:
\begin{equation}
\begin{aligned}
f(a\bvec x+b\bvec y,\bvec z)&=af(\bvec x,\bvec z)+bf(\bvec y,\bvec z)=0\\
f(\bvec z,a\bvec x+b\bvec y)&=af(\bvec z,\bvec  x)+bf(\bvec z,\bvec y)=0
\end{aligned}~,
\end{equation}
则称$f$是$V$上的一个双线性函数。
\end{definition}

显然,当固定一个向量不变时,双线性函数就是$V$上的一个线性函数。

双线性函数可以用矩阵表示。设$\{\bvec e_i\}$为$V$上的一组基,任意向量$\bvec x=a^i\bvec e_i,\bvec y=b^i\bvec e_i$,则$f(a^i\bvec e_i,b^j\bvec e_j)=a^ib^jf(\bvec e_i,\bvec e_j)=\bvec x^TA\bvec y$。称$A$是$f$在基$\{\bvec e_i\}$下的\textbf{度量矩阵}。
\begin{example}{}
\begin{equation}
\bvec x^TA\bvec y=\begin{pmatrix}
 a_1 &a_2  &a_3
\end{pmatrix}
\begin{pmatrix}
  f(\bvec e_1,\bvec e_1)& f(\bvec e_1,\bvec e_2) &f(\bvec e_1,\bvec e_3) \\
 f(\bvec e_2,\bvec e_1) &f(\bvec e_2,\bvec e_2)  &f(\bvec e_2,\bvec e_3) \\
 f(\bvec e_3,\bvec e_1) & f(\bvec e_3,\bvec e_2) &f(\bvec e_3,\bvec e_3)
\end{pmatrix}
 \begin{pmatrix}
 b^1 &b^2  &b^3
\end{pmatrix}~.
\end{equation}
\end{example}
\begin{example}{}
欧几里得内积是特殊的度量矩阵,表示为单位矩阵$\boldsymbol E$
\end{example}
。在前文我们已经知道,相似变换可以改变线性映射和向量的表示。“度量矩阵”的意义便在于使得向量内积在不同基下不变。