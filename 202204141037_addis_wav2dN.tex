% 二维波动方程的简单数值解(Matlab)

\begin{issues}
\issueDraft
\end{issues}

\pentry{二维波动方程\upref{Wv2D}}

\begin{figure}[ht]
\centering
\includegraphics[width=14cm]{./figures/wav2dN_2.png}
\caption{波的干涉, 动画见\href{https://wuli.wiki/apps/wav2D.html}{这里}.} \label{wav2dN_fig2}
\end{figure}

\begin{equation}
\laplacian u - \frac{1}{v_0^2}\pdv[2]{u}{t} = 0
\end{equation}

\begin{equation}
\pdv[2]{u}{t} = v_0^2\laplacian u
\end{equation}

使用差分法\upref{DerDif} 求解常微分方程, 时间差分得
\begin{equation}
u(t_{i+1}) = v_0^2\Delta t^2\laplacian u(t_i) + 2u(t_{i}) - u(t_{i-1})
\end{equation}
其中, 空间差分得
\begin{equation}
\laplacian u(x_i,y_i) = \frac{u(x_{i+1}, y_i) + u(x_{i-1}, y_i) - u(x_{i}, y_i)}{\Delta x^2}
+ \frac{u(x_i, y_{i+1}) + u(x_i, y_{i-1}) - u(x_i, y_{i})}{\Delta y^2}
\end{equation}

边界条件: 反射, 行波边界条件(类比\autoref{W1dNum_eq1}~\upref{W1dNum}):

对平面波 $u = A\sin(\bvec k\vdot\bvec r - \omega t)$, 有
\begin{equation}
v_0\grad u = -\uvec k \pdv{z}{t}
\end{equation}
所以
\begin{equation}
\pdv{z}{t} = v_0\abs{\grad u}
\end{equation}


以下 Matlab 代码中使用了 
\begin{lstlisting}[language=matlab, caption=wav2D.m]
% 二维波动方程的简单数值解
% 使用简单的差分法
function wav2D
% === 参数设置 ====
bc = 'o'; % 边界条件: [d] Dirichlet, [n] Neumann, [o] Open
v0 = 1;
Nplot = 10;
tmin = 0; tmax = 20; Nt = 1001;
xmin = -10; xmax = 10; Nx = 200;
ymin = -10; ymax = 10; Ny = 200;
crange = [-0.4,0.4];
% ================
close all;
t = linspace(tmin, tmax, Nt); dt = (tmax-tmin)/(Nt-1);
x = linspace(xmin, xmax, Nx); dx = (xmax-xmin)/(Nx-1);
y = linspace(ymin, ymax, Ny); dy = (ymax-ymin)/(Ny-1);
% [X, Y] = ndgrid(x, y);
Z = zeros(Nx, Ny, 3);
figure; set(gcf, 'Unit', 'Normalized', 'Position', [0.1,0.1,0.4,0.4]);
for it = 1:Nt
    Z(100,100,2) = 2*sin(5*t(it));
    d2Z = laplacian(Z(:,:,2), dx, dy);
    if bc == 'd'
        Z(:,:,3) = v0^2*dt^2*d2Z + 2*Z(:,:,2) - Z(:,:,1);
    elseif bc == 'o'
        Z(2:end-1,2:end-1,3) = v0^2*dt^2*d2Z(2:end-1, 2:end-1) +...
            2*Z(2:end-1,2:end-1,2) - Z(2:end-1,2:end-1,1);
        Z(:,1,3) = Z(:,1,2) + v0*dt*(Z(:,2,2)-Z(:,1,2))/dy;
        Z(:,end,3) = Z(:,end,2) - v0*dt*(Z(:,end,2)-Z(:,end-1,2))/dy;
    end
    if mod(it, Nplot) == 0
        clf; surfCart(x, y, Z(:,:,3)); caxis(crange);
        colormap jet; shading interp; axis equal;
        xlabel x; ylabel y; view(78, 32);
        axis([xmin,xmax,ymin,ymax,-2,2]);
        title(['t = ' num2str(t(it), '%.2f')]);
        saveas(gcf, [num2str(it/Nplot) '.png']);
    end
    Z(:,:,1) = Z(:,:,2); Z(:,:,2) = Z(:,:,3);
end
end

% laplacian of Z(ix, iy)
% finite difference
% boundary condition (not included in Z) is 0
function d2Z = laplacian(Z, dx, dy)
N1 = size(Z, 1); N2 = size(Z, 2);
z1 = zeros(N1, 1); z2 = zeros(1, N2);
d2Z_1 = ([Z(2:end, :); z2] + [z2; Z(1:end-1, :)] - 2*Z)/dx^2;
d2Z_2 = ([Z(:, 2:end) z1] + [z1 Z(:, 1:end-1)] - 2*Z)/dy^2;
d2Z = d2Z_1 + d2Z_2;
end

% my version of surf()
% X, Y, Z are 2D matrices
% the color of each facet is determined by the value
%                   of the upper-left grid point
% there will be 1 facet for each element of Z,
%                    and one more grid-point at
% the end of each dimension
function h = surfCart(x,y,Z,varargin)
[X,Y] = ndgrid([x,2*x(end)-x(end-1)],[y,2*y(end)-y(end-1)]);
Z = [Z, Z(:,end); Z(end,:), Z(end,end)];
h = surf(X,Y,Z,varargin{:});
shading flat;
view(90,90);
axis([min(X(:)), max(X(:)), min(Y(:)), max(Y(:)), ...
   [-1, 1]*max(1, max(abs(Z(:))))]);
xlabel x; ylabel y; zlabel z;
colorbar;
set(datacursormode(gcf), 'UpdateFcn', @datatip);
cameratoolbar;
end
\end{lstlisting}
