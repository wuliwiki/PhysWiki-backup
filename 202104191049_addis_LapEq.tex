% 拉普拉斯方程与调和函数
% 梯度|散度|拉普拉斯方程|偏微分方程|分离变量法

\pentry{梯度\upref{Grad}, 散度\upref{Divgnc}, 偏微分方程(未完成)}

若令 $\mathbb R^N$ 上某区域的实函数 $u(\bvec r)$ 的拉普拉斯等于零, 那么我们就得到了一个偏微分方程, 即\textbf{拉普拉斯方程(laplacian equation)}
\begin{equation}
\laplacian u(\bvec r) = 0
\end{equation}
从物理上, 二元函数的拉普拉斯方程可以理解为一个静止的, 不受外力的薄膜\upref{Wv2D}所满足的方程. 要得到方程的解, 我们需要规定一些边界条件. 常见的条件是给定一个区域, 然后给出 $u(\bvec r)$ 在边界上的函数值.

\begin{theorem}{刘维尔定理}
$\mathbb R^N$ 上的调和函数有界当且仅当它是常数.
\end{theorem}
\addTODO{证明}

\begin{theorem}{最大值定理}
$\mathbb R^N$ 上一个区域内的调和函数的最大值总出现在该区域的边界处.
\end{theorem}
