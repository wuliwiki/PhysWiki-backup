% 皮卡-林德勒夫定理

皮卡-林德勒夫定理 (Picard-Lindelöf theorem) 是分析数学中的一个基本定理, 又称为柯西-李普希茨定理 (Cauchy-Lipschitz theorem). 它断言: 常微分方程 (组) 的初值问题只需要满足一些非常宽泛的条件, 就是唯一可解的. 

由于许多经典物理问题都可以化归为常微分方程组, 所以皮卡-林德勒夫定理可以用来说明这些物理问题的决定论 (deterministic) 特性: \textbf{给定了系统的初始状态之后, 系统的演化就唯一确定了.}

对于不满足皮卡-林德勒夫定理条件的常微分方程组, 尚有皮亚诺存在定理. 后者无法保证解的唯一性.

\subsection{定理的表述与辨析}
\begin{theorem}{皮卡-林德勒夫定理}
设$I\subset\mathbb{R}$是开区间, $X$是巴拿赫空间\upref{banach}, $U\subset X$是开集. 设有连续映射$f:U\times I\to X$, 对于$X$变量满足局部李普希茨条件, 即对于任意$x_0\in U$, $t_0\in I$, 都存在$x_0$的小邻域$\bar B_X(x_0,R)\subset U$和$t_0$的小邻域$[t_0-r,t_0+r]\subset I$, 以及一个正数$L>0$, 使得对于任何$x_1,x_2\in \bar B_X(x_0,R)$和$t\in[t_0-r,t_0+r]$, 都有
$$
|f(x_1,t)-f(x_2,t)|_X\leq L|x_1-x_2|_X.
$$

则对于任何$t_0\in I$, $x_0\in U$, 都存在一个正数$T>0$, 使得常微分方程的初值问题
$$
\frac{d}{dt}u(t)=f(u(t),t),\quad u(t_0)=x_0
$$
在区间$[t_0-T,t_0+T]\cap I$上有唯一解.
\end{theorem}

虽然定理的精确表述有点繁琐, 但它背后的意思很简单: \textbf{对于常微分方程}
\begin{equation}\label{PiLin_eq1}
\frac{d}{dt}u(t)=f(u(t),t)
\end{equation}
\textbf{只要右边的函数$f$满足李普希茨条件, 那么它的初值问题就唯一可解.}

在实际应用中, 空间$X$一般都是实数空间$\mathbb{R}^n$. 这时候 \autoref{PiLin_eq1} 就是有$n$个未知函数的常微分方程组. 对于形如
$$
y^{(n)}(t)=F(t,y(t),y'(t),...,y^{(n-1)}(t))
$$
的$n$阶方程, 只要命
$$
u(t)=\left(\begin{array}{c}
y(t)\\
y'(t)\\
...\\
y^{(n-1)}(t)
\end{array}
\right),\quad
f(u(t),t)=\left(\begin{array}{c}
u_2(t)\\
u_3(t)\\
...\\
F\left(t,u_1(t),u_2(t),...,u_n(t)\right)
\end{array}
\right),
$$
就得到了有$n$个未知函数的常微分方程组. 这表示: \textbf{对于$n$阶常微分方程, 如果要确定它的一个特解, 一般来说需要给定它的直到$n-1$阶导数在某点处的值.}

\subsection{证明}
对于给定的$t_0\in I$和$x_0\in U$, 就取定理表述中的邻域$\bar B_X(x_0,R)\subset U$和$[t_0-r,t_0+r]\subset I$. 映射$f$在$\bar B_X(x_0,R)\times[t_0-r,t_0+r]$上是有界的, 不妨设它的上界为$M$. 对于