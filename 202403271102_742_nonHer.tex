% 非厄米物理导论
% keys 非厄米
% license Usr
% type Tutor

量子物理学的一大假设是,给定系统的哈密顿量是厄米的。厄米性保证了孤立系统的概率守恒,并且保证了一个量子态的能量期望值是实数。但是,在自然界中,由于我们总是会出现一些粒子,能量,或者信息流出给定的希尔伯特空间的情况,所以非厄米物理的研究变得十分重要。因为凡是实际的系统,都会出现耗散现象,这时候就需要处理哈密顿量是非厄米的情况。

关于开放系统的研究,最早可以追溯到伽莫夫等人的工作。他们考虑了活动的原子核的衰变现象,并且给出了这个衰变所对应的非厄米哈密顿量。从这里可以推出,由于衰变导致的流流出原子核的过程。后来,在原子分子物理中,根据这个思路,研究者又开发出了新的理论框架,被称为Feshbach投影,或者是Cohen-Tannoudgi投影方法。

另一个关于开放系统的理论研究框架主要是