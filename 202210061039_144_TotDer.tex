% 全导数(多元标量函数)
% 多元微积分|多元函数|全微分|全导数

\pentry{复合函数的偏导\upref{PChain}}

若多元函数包含若干个变量(以下以 $f(x,y,t)$ 为例),我们知道它可以对其中任意一个变量求偏导, 即 $\pdv*{f}{x}$,  $\pdv*{f}{y}$,  $\pdv*{f}{t}$, 注意求偏导时其余两个变量不变.现在若把 $x,y$ 看做 $t$ 的函数, 那么 $f$ 归根结底也是 $t$ 的函数 $f[x(t),y(t),t]$, 我们可以将其对 $t$ 求导.为了强调这与对 $t$ 求偏导有所不同,我们把得到的函数叫做\textbf{全导数}.

与偏微分中的链式法的推导类似,我们先来看函数的全微分\upref{TDiff}
\begin{equation}
\dd{f} = \pdv{f}{x} \dd{x} + \pdv{f}{y} \dd{y} + \pdv{f}{t} \dd{t}
\end{equation}
而根据 $x,y$ 与 $t$ 的微分关系
\begin{equation}
\dd{x} = \dv{x}{t} \dd{t} \qquad  \dd{y} = \dv{y}{t} \dd{t}
\end{equation}
代入上式得
\begin{equation}
\dd{f} = \qty( \pdv{f}{x}\dv{x}{t} + \pdv{f}{y}\dv{y}{t} + \pdv{f}{t} )\dd{t}
\end{equation}
而若把 $f$ 看做 $t$ 的一元函数,又应该有全微分关系
\begin{equation}
\dd{f} = \dv{f}{t} \dd{t}
\end{equation}
对比以上两式可得 $f$ 关于 $t$ 的全导数为
\begin{equation}\label{TotDer_eq1}
\dv{f}{t} = \pdv{f}{x}\dv{x}{t} + \pdv{f}{y}\dv{y}{t} + \pdv{f}{t}
\end{equation}

% 未完成 此处应该有例子
% 未完成 要介绍显含 $t$ 和不显含 $t$ 的区别
