% 可测函数
% 叶葛洛夫定理|Egoroff|Lusin|测度|Lebesgue积分|实变函数|广义实数

为了引入新的积分思想,我们首先要讨论可测函数的相关概念.

\subsection{广义实数}

为了方便将来的讨论,我们将实数扩充为“广义实数”,即在实数集合中再添加$\pm\infty$这两个元素.

$\pm\infty$的运算规则简述如下:

\begin{equation}
(\pm\infty)+( \pm \infty)=\pm \infty
\end{equation}

\begin{equation}
(\pm\infty)(\pm\infty)=+\infty
\end{equation}

\begin{equation}
(\pm\infty)(\mp\infty)=-\infty
\end{equation}

设$a, b, c$都是实数,且$b<0<c$,则还有:

\begin{equation}
a+(\pm\infty)=\pm\infty
\end{equation}

\begin{equation}
b(\pm\infty)=\mp\infty
\end{equation}

\begin{equation}
c(\pm\infty)=\pm\infty
\end{equation}

而$(\pm\infty)+(\mp\infty)$是不允许的运算,多数情况下$0\cdot(\pm\infty)$也是不允许的.

这样一来,我们就可以不再局限于实数集,而是在广义实数集上定义函数.



\subsection{可测函数}

可测函数是构造Lebesgue积分思想的砖块,就像“柱子的面积”是Riemann积分思想的砖块一样.

\begin{definition}{可测函数}
设$f$是$E\subseteq\mathbb{R}^n$上的函数(值域为广义实数集),其定义域$E$为一个可测集.如果对于任意实数$y_0$,都有$\{x\in E|f(x)\geq y_0\}$是可测集,那么称$f$是$E$上的\textbf{可测函数(measurable function)}.
\end{definition}


可测函数要保持的性质是,在对值域进行分划的时候,各分划区间dui


\begin{theorem}{}

\end{theorem}










