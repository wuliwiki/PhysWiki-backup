% 北京大学 2009 年 考研 普通物理
% license Usr
% type Note

\textbf{声明}:“该内容来源于网络公开资料,不保证真实性,如有侵权请联系管理员”
\subsection{小题}
\begin{enumerate}
\item 写出以$T,V$为自变量的热力学基本方程及与其相应的麦氏关系,
\item 朗道二级相变理论的基本假设有哪些?
\item 什么是连续相变?它有什么特点?
\item 写出玻尔兹曼微分积分方程的弛豫时间近似,并指明各项的意义
\item 写出吉布斯相律。
\item 什么是经典极限条件
\item 什么是等概率原理。
\item 什么条件下可引入磁标势?
\item 宇宙飞船进入大气层后会出现电磁波信号的“黑区”,请用电介质介电常数的谐振子理论解释之
\item 写出静电场的唯一性定理。
\item 写出电介质中的麦克斯韦方程组。
\item 写出电磁场的规范变换。
\item 写出矢势的推迟势公式
\item 什么是经典电动力学的局限性:
\end{enumerate}
\subsection{大题}
\begin{enumerate}
\item 1.电子处于 $x-y$ 系的 $(0, a)$ 处,$x'-y'$ 系以速度 $v$ 相对于 $x-y$ 系沿 $x$ 轴正向运动。求在 $x'-y'$ 系中观察到的电磁场,并用 $x'-y'$ 系中的 $x', y', t'$ 表示。(题目中给出了 $B, E$ 的变换公式,可用此公式计算)
\item 平面电磁波 $E = E_0 \cdot \mathbf{e_1} \cdot \exp(ikx - \omega t)$ ($E_0$ 是振幅,$\mathbf{e_1}$ 是某方向的基矢,$kx$ 是矢量点乘)照射在一个自由电子上,电子吸收该电磁波,并辐射出去。①忽略辐射阻尼力,写出电子的运动方程。②入射波的偏振方向为 $\mathbf{e_1}$,出射波的偏振方向为 $\mathbf{e_2}$,设 $\mathbf{e_1}$ 与 $\mathbf{e_2}$ 的夹角为 $\langle \mathbf{e_1}, \mathbf{e_2} \rangle$,求电子对该电磁波的微分散射截面,并对$\langle\mathbf{e_1}, \mathbf{e_2} \rangle$ 求平均。
\item 一个Fermi 子系统,其中粒子的磁矩在外磁场中能量为“B、-μB.求:①分别求出粒子磁矩取向与磁场方向相同和相反的粒子数N+和N-②)该系统的Fermi能级③求系统的磁矩,保留B的一阶,并证明该近视下磁导率与B无关。④求系统的磁矩,保留至B的二阶,并求出磁导率。
\item 有一个二维吸附面,若粒子吸附在其上,则能量由零变为-ε,同时可以做二维运动(即总能量为e与二维运动的能量之和)。设总粒子数为N0,吸附在其上的为N
\end{enumerate}
