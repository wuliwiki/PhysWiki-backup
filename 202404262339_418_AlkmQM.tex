% 碱金属原子(量子力学)
% keys 碱金属|量子力学
% license Usr
% type Tutor

\pentry{薛定谔方程(单粒子多维)\nref{nod_QMndim},球谐函数 \nref{nod_SphHar},类氢原子的束缚态\nref{nod_HWF}}{nod_060d}

碱金属原子本质是由价电子(最外层电子)来决定性质的。价电子所受到原子实(原子核、内层电子)作用的势函数描述为
\begin{equation}
V(r) = -e^2/r - \lambda a e^2/r^2, \ (0 < \lambda \le 1/8) ~.
\end{equation}
是一个中心势场。则薛定谔方程是可分离变量的,使得波函数可以表示为球谐函数与径向解的乘积的形式——$\psi = Y_{l, m}(\theta, \phi) R(r)$。考虑求解径向波函数,应满足微分方程
\begin{equation}
\frac{1}{R} \left(\frac{1}{r^2} \dv{R}{r}\right) + \frac{2mr^2}{\hbar^2} \left[E - V(r)\right] = l(l+1), \ (l = 0, 1, 2, \cdots)~.
\end{equation}

观察到第一项,不难想到常规技巧,设 $R(r) = u(r) /r$,从而可化为
\begin{equation}
\dv{^2 u}{r^2} + \left[\frac{2m}{\hbar^2}\left(E + e^2/r\right) + \left(\frac{2 \lambda}{r^2} - \frac{l(l+1)}{r^2}\right)\right]u = 0 ~.
\end{equation}
若取 $l'(l'+1) = l(l+1) - 2\lambda$,即
$$l' = -\frac12 + \left(l + \frac12\right) \sqrt{1 - \frac{8\lambda}{(2l+1)^2}} ~,$$
就有
\begin{equation}
\dv{^2 u}{r^2} + \left[\frac{2m}{\hbar^2}\left(E + e^2/r\right) - \frac{l'(l'+1)}{r^2}\right]u = 0 ~.
\end{equation}
这类似于(类)氢原子的情况\autoref{eq_HWF_11}~\upref{HWF}。

类比于类氢原子,对于碱金属原子有价电子
$$n' = n_r + l' + 1, \ (n_R = 0, 1, 2, \cdots) ~.$$
其中 $n' \ge 1/2$ 一般不是整数。而价电子能级
\begin{equation}
E_{n'} = -\frac{me^4}{2\hbar^2} \left(\frac{1}{n'}\right)^2 ~.
\end{equation}
