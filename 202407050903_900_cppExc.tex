% C++ 异常处理(笔记)
% keys throw|c++|cpp|exit|异常|try|catch
% license Xiao
% type Note

\begin{issues}
\issueDraft
\end{issues}

在初学编程时, 遇到错误我们往往就直接用 \verb`exit()` 终止程序。 但有时候我们不希望程序终止, 而是希望程序自行对错误进行一定的处理。 C 语言的常见办法是把函数的返回值(通常是整型)作为错误代码, \verb`0` 代表成功, 其他值对应不同类型的错误。 然而函数调用是重重嵌套的, 将错误代码层层传递是一件很麻烦的事情, 另外, 把错误处理和函数调用的语法分离开也可以使代码的结构更明显。 当然 C 还有另外一种办法就是通过一个全局变量(通常叫做 \verb`errno`)来传递错误信息, 但它只能是一个整数类型, 如果你自定义许多其他的全局变量来传递错误信息,常常会导致混乱(使用全局变量一般是不太好的习惯,会导致代码非常难以调试)。

在 C++ 中有专门的异常处理机制, 一般使用 \verb`throw`(抛出某种类型的错误), \verb`try`(检测某段代码的运行) 和 \verb`catch`(处理某种类型的错误) 三个关键词完成。 来看一个简单的例子。

\begin{lstlisting}[language=cpp]
#include <iostream>
#include <string>
using namespace std;

struct err_info { string where, what;};

void fun2()
{
	err_info e;
	e.where = "fun2()"; e.what = "something wrong!";
	throw e;
}

void fun1() { fun2(); }

int main () {
	try { fun1(); }
	catch (err_info e) {
		cout << "where: " << e.where << endl;
		cout << "what: " << e.what << endl; 
	}
}
\end{lstlisting}
程序中 \verb`main()` 调用 \verb`fun1()`, \verb`fun1()` 接着调用 \verb`fun2()`, 而 \verb`fun2()` 必然会出现一个异常, 抛出了一个类型为 \verb`err_info` 的对象 \verb`e`。 这时无论 \verb`fun2()` 是否运行完成都会终止, 并把 \verb`e` 传给 \verb`fun1()`, 而 \verb`fun1()` 并没有处理这个异常的代码, 所以 \verb`fun1()` 同样也终止运行, 并把 \verb`e` 回传给 \verb`main()`。 由于 \verb`main()` 中存在处理 \verb`err_info` 类型错误的代码(\verb`catch (err_info e)`), 所以就会执行相应的错误处理。
