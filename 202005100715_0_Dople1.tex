% 多普勒效应(一维)

多普勒效应是讨论, 当机械波的发设者与接收者相对于波的介质运动时, 发射的频率和接收到的频率之间有何关联.

我们在介质的参考系中, 假设介质处处均匀且静止, 波速 $u$ 处处相等.

先来讨论一维的情况. 令波源甲运动方程(位置关于时间的函数)为 $x_1(t)$, 速度为(一点表示对时间求导) $v_1(t) = \dot{x_1}(t)$. 同样, 接收者乙位置和速度分别为 $x_2(t)$ 和 $v_2(t)$.

本质上, 多普勒效应可以等效为追及问题, 可以想象甲以一定的频率 $f_1$ 向乙发射速度为 $u$ 的子弹, 子弹的位置对应波峰的位置, 两个相邻子弹之间的间距对应波长. 若甲乙静止不动, 则乙接受到子弹的频率和甲
