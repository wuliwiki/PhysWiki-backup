% 线性规划
% keys 线性规划|单纯形法
% license Usr
% type Tutor
\pentry{线性方程组(高中)\nref{nod_LinEqu}}{nod_9aef}

1939年,苏联数学家Kantorovich出版了《生产组织与计划中的线性规划模型》一书,为列宁格勒胶合板厂的计划任务建立了一个线性规划的数学模型,为用数学方法解决管理并使二者结合做出了开创性的工作。后来,由于战争的需要,美国经济学家Koopmans重新独立地研究运输问题,并很快看到了线性规划在经济学中应用的意义。此后线性规划也被广泛应用于军事、经济等各方面。鉴于他们在线性规划方面的突出贡献,1975年的诺贝尔经济学奖授予了他们。1947年美国数学家Dantzig提出了求解一般线性规划问题的方法——单纯性法,之后线性

在线性不等式或等式限制下,使得某一线性目标取得最大(或最小)的问题。
数学模型:










