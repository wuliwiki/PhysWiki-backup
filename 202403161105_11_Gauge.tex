% 电磁场的规范变换
% keys 库仑规范|洛伦兹规范|标量势能|矢量势能|麦克斯韦方程组
% license Xiao
% type Tutor

\begin{issues}
\issueDraft
\end{issues}

\pentry{电磁场标势和矢势\nref{nod_EMPot}}{nod_4d5d}

\footnote{参考 Wikipedia \href{https://en.wikipedia.org/wiki/Gauge_fixing}{相关页面}。}虽然标势和矢势可以唯一确定电磁场,但是同一个电磁场却可以由不同的标势和矢势得到。由于有物理意义 令 $\bvec A, \varphi$ 对应的电磁场为 $\bvec E, \bvec B$; $\bvec A', \varphi'$ 对应的电磁场为 $\bvec E', \bvec B'$, 我们希望得到某类从 $\bvec A, \varphi$ 到 $\bvec A', \varphi'$ 的变换使得 $\bvec E' = \bvec E$, $\bvec B' = \bvec B$。

首先给矢势(\autoref{eq_EMPot_2}~\upref{EMPot})加上某个函数 $\lambda(\bvec r, t)$ 的梯度, 令 $\bvec A' = \bvec A + \grad\lambda$ 磁场不变:
\begin{equation}
\bvec B' = \curl \bvec A' = \curl (\bvec A + \grad \lambda) = \curl \bvec A = \bvec B~.
\end{equation}
其中 $\curl(\grad \lambda) \equiv 0$ 对任意标量场都成立, 这是梯度和散度的基本性质(引用未完成)。 如果 $\lambda$ 随时间变化,\autoref{eq_EMPot_1}~\upref{EMPot} 中的电场会改变。 因此我们需要同时修正 $\varphi$, 才能确保变换后电场也不改变。 可以发现只需要令 $\varphi' = \varphi - \pdv*{\lambda}{t}$ 即可
\begin{equation}
\bvec E' = -\grad \varphi' - \pdv{\bvec A'}{t} = -\grad \qty(\varphi - \pdv{\lambda}{t}) - \pdv{t} (\bvec A + \grad \lambda) = \bvec E~.
\end{equation}


这种“保持电磁场不变时,对势 $\pmat{\phi, \bvec{A}}$ 进行的变换”被称为\textbf{规范变换(gauge transformation)}:
\begin{equation}\label{eq_Gauge_3}
\leftgroup{
&\bvec A' = \bvec A + \grad \lambda\\
&\varphi' = \varphi - \pdv{\lambda}{t}~.
}\end{equation}
任何产生相同电磁场的两组标势矢势都可以通过规范变换联系起来。

常见的两种规范是\textbf{库仑规范\upref{Cgauge}}和\textbf{洛伦兹规范\upref{LoGaug}}。
