% 磁化强度

为了表征磁介质磁化的程度,与讨论电介质时定义极化强度一样引进一个物理量,叫做\textbf{磁化强度(magnetization intensity)}. 在被磁化后的磁介质内,任取一体积元$\Delta V$. 在这体积元中所有分子固有磁矩的矢量和$\sum \mathbf{m}_{mole}$加上附加磁矩的矢量和$\sum \Delta\mathbf{m}_{mole}$与该体积元的比值,即单位体积内分子磁矩的矢量和,称为\textbf{磁化强度},用$\mathbf M$表示.即
\begin{equation}
\mathbf M=\frac{\sum \mathbf m_{mole}+\sum \Delta \mathbf m_{mole}}{\Delta V}
\end{equation}
对于顺磁质,$\sum \Delta\mathbf{m}_{mole}$可以忽略;对于抗磁质,$\sum \mathbf{m}_{mole}=0$;对于真空,$\mathbf M=0$.如果在介质中各点的$\mathbf M $相同,就称磁介质被均匀磁化.在国际单位制中,$\mathbf M$的单位是$\rm A/m$.

