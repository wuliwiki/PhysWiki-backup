% 戈特洛布·弗雷格(综述)
% license CCBYSA3
% type Wiki

本文根据 CC-BY-SA 协议转载翻译自维基百科\href{https://en.wikipedia.org/wiki/Gottlob_Frege}{相关文章}。

\begin{figure}[ht]
\centering
\includegraphics[width=6cm]{./figures/49ab91a025d7e7b7.png}
\caption{弗雷格,约1879年} \label{fig_Gottlo_1}
\end{figure}
弗里德里希·路德维希·戈特洛布·弗雷格[7] (Friedrich Ludwig Gottlob Frege,1848年11月8日-1925年7月26日)是德国哲学家、逻辑学家和数学家。他曾担任耶拿大学的数学教授,被许多人视为分析哲学的奠基人,专注于语言哲学、逻辑学和数学哲学。尽管他在生前几乎未受到关注,但朱塞佩·皮亚诺(Giuseppe Peano,1858–1932)、伯特兰·罗素(Bertrand Russell,1872–1970)以及在某种程度上路德维希·维特根斯坦(Ludwig Wittgenstein,1889–1951)将他的工作介绍给了后来的哲学家。弗雷格被广泛认为是自亚里士多德以来最伟大的逻辑学家,也是有史以来最深刻的数学哲学家之一。[8] 

他的贡献包括在《概念文字》(Begriffsschrift)中发展了现代逻辑,以及在数学基础方面的工作。他的著作《算术基础》是逻辑主义项目的开创性文本,迈克尔·杜梅特(Michael Dummett)将其视为语言学转向的标志。弗雷格的哲学论文《论意义与指称》和《思想》也被广泛引用。前者论证了两种不同的意义类型和描述主义。在《算术基础》和《思想》中,弗雷格分别在关于数字和命题的问题上主张与心理主义或形式主义对立的柏拉图主义。
\subsection{生活}  
\subsubsection{童年(1848–1869)}  
弗雷格于1848年出生在维斯马,梅克伦堡-什未林(今天属于梅克伦堡-前波美拉尼亚)。他的父亲卡尔(卡尔)·亚历山大·弗雷格(1809–1866)是女子中学的共同创始人和校长,直至去世。卡尔去世后,学校由弗雷格的母亲奥古斯特·威尔赫尔米娜·索菲·弗雷格(原姓比亚洛布洛茨基,1815年1月12日–1898年10月14日)领导;她的母亲是奥古斯特·阿玛利亚·玛丽亚·巴尔霍恩,菲利普·梅兰希通的后裔;她的父亲是约翰·海因里希·西格弗里德·比亚洛布洛茨基,来自一个17世纪离开波兰的波兰贵族家庭。[9][10] 弗雷格是路德宗信徒。[11]

在童年时期,弗雷格接触到了一些将指导他未来科学事业的哲学思想。例如,他的父亲编写了一本面向9至13岁儿童的德语教材,名为《Hülfsbuch zum Unterrichte in der deutschen Sprache für Kinder von 9 bis 13 Jahren》(第二版,维斯马,1850年;第三版,维斯马和路德维希斯卢斯特:辛斯托夫出版社,1862年),该书的第一部分讨论了语言的结构和逻辑。

弗雷格在维斯马的大城市学校(Große Stadtschule Wismar)学习,并于1869年毕业。[12] 数学和自然科学教师古斯塔夫·阿道夫·利奥·萨克斯(Gustav Adolf Leo Sachse,1843–1909),他同时也是一位诗人,在决定弗雷格未来的科学事业方面发挥了重要作用,鼓励他继续在自己的母校耶拿大学深造。[13]
\subsubsection{大学学习(1869–1874)}  
弗雷格于1869年春季以北德意志联邦公民身份入学耶拿大学。在四个学期的学习中,他参加了大约二十门讲座课程,其中大多数是关于数学和物理学的。他最重要的老师是恩斯特·卡尔·阿贝(Ernst Karl Abbe,1840–1905;物理学家、数学家和发明家)。阿贝讲授了引力理论、电流学和电动力学、复分析的复变函数理论、物理应用、机械学的各个分支以及固体力学。阿贝不仅是弗雷格的老师,还是他值得信赖的朋友,并且作为卡尔·蔡司光学制造公司(Carl Zeiss AG)的总监,他有能力推动弗雷格的职业生涯。弗雷格毕业后,他们开始了更紧密的通信。[citation needed]

他其他值得注意的大学老师包括克里斯蒂安·菲利普·卡尔·斯内尔(Christian Philipp Karl Snell,1806–1886;授课科目:几何学中的微积分分析应用、平面解析几何、解析力学、光学、力学的物理基础);赫尔曼·卡尔·尤利乌斯·特劳戈特·谢费尔(Hermann Karl Julius Traugott Schaeffer,1824–1900;授课科目:解析几何、应用物理学、代数分析、电报及其他电子机器);以及哲学家库诺·费舍尔(Kuno Fischer,1824–1907;康德主义和批判哲学)。[citation needed]

从1871年开始,弗雷格在哥廷根继续他的学业,哥廷根是德语地区数学领域的领先大学,他参加了鲁道夫·弗里德里希·阿尔弗雷德·克莱布施(Rudolf Friedrich Alfred Clebsch,1833–1872;解析几何)、恩斯特·克里斯蒂安·尤利乌斯·舍林(Ernst Christian Julius Schering,1824–1897;函数理论)、威廉·爱德华·韦伯(Wilhelm Eduard Weber,1804–1891;物理学研究、应用物理)、爱德华·里克(Eduard Riecke,1845–1915;电学理论)和赫尔曼·洛策(Hermann Lotze,1817–1881;宗教哲学)的讲座。成熟弗雷格的许多哲学理论与洛策相似;是否弗雷格的观点直接受到洛策讲座的影响,一直是学术讨论的主题。[citation needed]

1873年,弗雷格在恩斯特·克里斯蒂安·尤利乌斯·舍林的指导下获得了博士学位,论文题目为《Über eine geometrische Darstellung der imaginären Gebilde in der Ebene》(《关于平面中虚构形态的几何表示》),他在其中旨在解决几何学中的一些基本问题,如射影几何中无限远(虚构)点的数学解释。[citation needed]

弗雷格于1887年3月14日娶了玛格丽特·卡塔琳娜·索菲亚·安娜·丽泽贝格(Margarete Katharina Sophia Anna Lieseberg,1856年2月15日–1904年6月25日)。[12] 夫妻二人至少有两个孩子,不幸的是,他们都在幼年时去世。多年后,他们收养了一个儿子,阿尔弗雷德。然而,关于弗雷格的家庭生活,知之甚少。[14]
