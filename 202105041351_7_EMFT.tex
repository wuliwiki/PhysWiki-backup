% 电磁场张量
% 场张量|相对论|电磁场|张量变换|麦克斯韦张量|法拉第张量|麦克斯韦双矢量


\pentry{张量的分类\upref{CatTns}, 电动力学}

一个参考系中的电磁场需要用六个实函数数来刻画,三个用来刻画电场,三个用来刻画磁场.六个实数太过复杂,我们希望寻求一种简单的方式来简化表达.把六个实数合成一个对象的方法,最直接的当然是使用一个六维向量——不过这样并不能带来实质上的简化.实践中我们使用的其实是一个反对称张量场,用它来表示电磁场.

和向量一样,任何张量只有给定了空间的基,才有“坐标分量”的概念.在狭义相对论里,时空是一个线性空间,事件的时空坐标随着基的不同而不同,而不同的基就代表不同的观察者,事件的坐标分量就是观察者的测量值.

电磁张量也是一样的,它本身不会因观察者而改变,给定









