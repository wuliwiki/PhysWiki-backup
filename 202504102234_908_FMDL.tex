% 费马大定理(综述)
% license CCBYSA3
% type Wiki

本文根据 CC-BY-SA 协议转载翻译自维基百科\href{https://en.wikipedia.org/wiki/Fermat\%27s_Last_Theorem}{相关文章}。

在数论中,费马大定理(在较早的文献中有时称为“费马猜想”)陈述如下:对于任意整数\( n > 2 \),不存在三个正整数\( a, b, c \)满足方程\( a^n + b^n = c^n \)。而对于\( n = 1 \)和\( n = 2 \)的情形,自古以来就已知存在无穷多个解。\(^\text{[1]}\)

这个命题最早是皮埃尔·德·费马大约于1637年在一本《算术》书的页边空白处提出的。他还写道他已有一个证明,但“这个证明太大,写不下”。尽管费马曾提出的其他未经证明的命题后来被他人证明并被称为“费马定理”(例如费马两平方和定理),但唯独这条“费马大定理”长期无法证明,使人们怀疑费马是否真的拥有一个正确的证明。因此,这个命题长期以来被称为\textbf{猜想}而不是定理。经过数学家长达358年的努力,安德鲁·怀尔斯于1994年首次成功给出了完整证明,并于1995年正式发表。2016年,怀尔斯因其工作获得阿贝尔奖,其成果被称为“一项惊人的突破”。\(^\text{[2]}\)此外,该证明还涵盖了大量谷山–志村猜想的内容,该猜想后来被称为模性定理,它不仅解开了费马大定理之谜,还开辟了许多新领域,并发展出了强大的模性提升技术,对解决众多其他数学难题产生了深远影响。

这个未解难题曾在 19世纪和20世纪极大地推动了代数数论的发展。在整个数学史上,费马大定理是最著名的定理之一。在被证明之前,它还曾被列入《吉尼斯世界纪录》,称为“最难的数学问题”,部分原因是该定理拥有最多数量的失败证明尝试。\(^\text{[3]}\)
\subsection{概述}  
\subsubsection{毕达哥拉斯的起源}
毕达哥拉斯方程\(x^2 + y^2 = z^2\)在正整数\(x\)、\(y\)、\(z\)上有无穷多组解,这些解被称为毕达哥拉斯三元组(最简单的例子是 3、4、5)。大约在 1637年,费马在一本书的页边空白处写道,更一般形式的方程\(a^n + b^n = c^n\)在当\(n > 2\)时,没有正整数解。尽管他声称自己有一个完整的证明,但他未留下任何细节,至今也未有人找到该证明。这个断言是在他去世大约 30 年后才被人发现的。

这个断言后来被称为费马大定理,并在接下来的三个半世纪里始终未被证明。\(^\text{[4]}\)

这一定理最终成为数学史上最著名的未解问题之一。对它的证明尝试促使了数论领域的重大进展,随着时间推移,费马大定理作为数学未解难题的地位也日益突出。
\subsubsection{后续发展与最终解答}
当\(n = 4\)时的特例由费马本人亲自证明,该特例足以说明:如果费马大定理在某个非素数指数\(n\)下不成立,那么它在某个更小的\(n\)下也将不成立,因此只需对素数指数\(n\)进行进一步研究。\(^\text{[注1]}\)在接下来的两个世纪(1637–1839年)中,该猜想仅在素数\(n = 3, 5, 7\)的情况下被证明成立,尽管索菲·热尔曼提出并证明了一种适用于整个素数类的方法,这在当时具有开创性意义。到了19世纪中期,欧内斯特·库默尔进一步拓展了该思路,并证明了该定理对于所有正规素数成立,但对于非正规素数仍需逐个分析。在库默尔工作的基础上,借助复杂的计算机研究,其他数学家将证明的范围扩展到了所有指数为素数且小于四百万的情况。\(^\text{[5]}\)但对于所有整数指数的全面证明依然遥不可及(这意味着大多数数学家认为:该定理要么无法证明、要么极其困难,或者以当时的知识几乎不可能完成)。\(^\text{[6]}\)

大约在1955年,日本数学家志村五郎和谷山丰怀疑椭圆曲线与模形式之间可能存在某种联系——这是数学中的两个完全不同的领域。这个猜想当时被称为“谷山–志村猜想”(后来称为模性定理),它最初是一个独立的数学命题,表面上与费马大定理并无关联。尽管如此,这个猜想被广泛认为本身就具有重要意义和深远影响,但也像费马大定理一样,被认为极其难以证明,几乎无法触及。\(^\text{[7]}\)

1984年,盖尔哈德·弗雷注意到这两个原本毫不相关、尚未解决的问题之间似乎存在某种联系。弗雷给出了一个初步的推理框架,表明这种联系是可能被证明的。随后,让-皮埃尔·塞尔证明了除一个关键部分(即“ε-猜想”)以外的所有部分。1986年,肯·里贝特在塞尔的基础上,完整证明了这两个问题之间的密切联系(参见:里贝特定理与弗雷曲线)。\(^\text{[2]}\)这些由弗雷、塞尔和里贝特发表的论文表明:如果能证明谷山–志村猜想对至少半稳定椭圆曲线成立,那么费马大定理也将随之被证明。这个逻辑关系如下:任何能够反驳费马大定理的解,也可以被用来反驳谷山–志村猜想。因此,如果模性定理(谷山–志村猜想的证明)被确认无误,就意味着不会存在与费马大定理相矛盾的解,也就是说,费马大定理必定成立。

尽管这两个问题在当时都被视为极其艰难、几乎“完全无法触及”的数学难题,\(^\text{[2]}\) 但这是第一次明确提出了一条有可能将费马大定理推广至所有整数指数并完成证明的路径。与费马大定理不同的是,谷山–志村猜想当时正是一个活跃的研究领域,也被认为更有可能在当代数学框架中被攻克。\(^\text{[8]}\)然而,数学界普遍的看法却是:这项联系虽然美妙,但更凸显了谷山–志村猜想的难以证明性。[9] 数学家约翰·科茨的评论代表了当时的主流观点:\(^\text{[9]}\)

“我本人当时非常怀疑费马大定理与谷山–志村猜想之间的这种美妙联系能真正带来什么成果,因为说实话,我并不认为谷山–志村猜想是可以被证明的。尽管这个问题非常优美,但看起来根本不可能被真正证明。我必须承认,我当时认为自己这一生都不太可能看到它被证明。”

当听说里贝特已经证明了弗雷所提出的联系是正确的后,英国数学家安德鲁·怀尔斯决定尝试证明谷山–志村猜想,以此作为证明费马大定理的途径。怀尔斯从小就对费马大定理着迷,并在椭圆曲线及相关领域拥有扎实背景。从1987年起,怀尔斯秘密地独自研究该问题,历时六年。1993年,他成功地证明了谷山–志村猜想中足以推导出费马大定理的部分。怀尔斯的论文在篇幅和理论深度上都非常庞大。在论文同行评审过程中,人们在其中一部分发现了一个错误。为了修复这个缺陷,怀尔斯又花了一年时间,并与他的前学生理查德·泰勒合作,最终解决了这个问题。因此,最终于1995年发表的完整证明中,还附有一篇由两人共同完成的小型论文,用以证明修正部分的有效性。怀尔斯的成就得到了广泛报道,并通过多本书籍和电视节目向大众传播。谷山–志村–韦伊猜想其余尚未证明的部分,在1996至2001年间由其他数学家在怀尔斯工作的基础上陆续完成,最终全部得以证明,该猜想也因此被称为模性定理。\(^\text{[10][11][12]}\)怀尔斯因其对数学的杰出贡献获得了多项荣誉和奖项,其中包括2016年的阿贝尔奖。\(^\text{[13][14][15]}\)
\subsubsection{费马大定理的等价表述}
有几种不同的方式可以表述费马大定理,这些表述在数学上与原问题的陈述是等价的。

为了表述这些方式,我们使用以下符号:设\( \mathbf{N} \)为自然数集合\( 1, 2, 3, \dots \),设\( \mathbf{Z} \)为整数集合\( 0, \pm 1, \pm 2, \dots \),设\( \mathbf{Q} \)为有理数集合\(a/b\),其中\( a \)和\( b \)是整数且\( b \neq 0 \)。接下来我们将称\( x^n + y^n = z^n \)的解,其中一个或多个\( x, y, z \)为零的解为平凡解。所有三个数都非零的解将称为非平凡解。

为了比较,我们从原始表述开始。

\begin{itemize}
\item \textbf{原始表述}:设 \( n, x, y, z \in N \)(意味着 \( n, x, y, z \) 都是正整数)且 \( n > 2 \),方程 \( x^n + y^n = z^n \) 没有解。\\\\  
大多数关于该主题的处理方法都是这样表述的。它也通常在\( \mathbf{Z} \)上表述:\(^\text{[16]}\)
\item \textbf{等价表述 1}:方程 \( x^n + y^n = z^n \),其中整数 \( n \geq 3 \),没有非平凡解 \( x, y, z \in \mathbf{Z} \)。\\\\  
如果 \( n \) 是偶数,等价性是显而易见的。如果 \( n \) 是奇数且 \( x, y, z \) 都是负数,那么我们可以用 \( -x, -y, -z \) 替换 \( x, y, z \) 来得到\( \mathbf{N} \)中的解。如果其中两个是负数,那必须是 \( x \) 和 \( z \) 或 \( y \) 和 \( z \)。如果 \( x, z \) 是负数而 \( y \) 是正数,那么我们可以重新排列得到 \( (-z)^n + y^n = (-x)^n \),从而得到\( \mathbf{N} \)中的解;另一个情况也可以类似处理。现在,如果只有一个是负数,它必须是 \( x \) 或 \( y \)。如果 \( x \) 是负数,且 \( y \) 和 \( z \) 是正数,那么可以重新排列得到 \( (-x)^n + z^n = y^n \),再次得到\( \mathbf{N} \)中的解;如果 \( y \) 是负数,结果是对称的。因此,在所有情况下,\( \mathbf{Z} \)中的非平凡解也意味着 \( \mathbf{N} \)中存在解,这就是原问题的表述。
\item \textbf{等价表述 2}:方程 \( x^n + y^n = z^n \),其中整数 \( n \geq 3 \),没有非平凡解 \( x, y, z \in \mathbf{Q}) \)。\\\\ 
这是因为 \( x, y, z \) 的指数相等(为 \( n \)),因此如果在\( \mathbf{Q} \) 中有解,那么可以通过适当的最小公倍数进行乘法运算,将解转化为\( \mathbf{Z} \)中的解,从而得到\( \mathbf{N} \)中的解。
\item \textbf{等价表述 3}:方程 \( x^n + y^n = 1 \),其中整数 \( n \geq 3 \),没有非平凡解 \( x, y \in \mathbf{Q} \)。\\  
方程 \( x^n + y^n = z^n \) 的一个非平凡解 \( a, b, c \in \mathbf{Z} \) 会给出方程 \( v^n + w^n = 1 \) 的非平凡解 \( a/c, b/c \in \mathbf{Q}\)。反过来,方程 \( v^n + w^n = 1 \) 中的一个解 \( a/b, c/d \in \mathbf{Q}\) 会给出方程 \( x^n + y^n = z^n \) 的非平凡解 \( ad, cb, bd \)。\\\\
这个最后的表述尤其有益,因为它将问题从关于三维空间中的曲面的研究,转化为关于二维空间中曲线的研究。此外,它允许在域\( \mathbf{Q} \)上进行操作,而不是在环 \( \mathbf{Z} \)上进行操作;域比环具有更多的结构,这使得对其元素的分析更加深入。
\item \textbf{等价表述 4} – 与椭圆曲线的联系:如果 \( a, b, c \) 是方程 \( a^p + b^p = c^p \) 的非平凡解,其中 \( p \) 为奇素数,则方程 \( y^2 = x(x - ap)(x + bp) \)(Frey 曲线)将是一个没有模形式的椭圆曲线。\(^\text{[17]}\)\\\\  
通过利用 Ribet 定理检查这条椭圆曲线,可以发现它没有模形式。然而,Andrew Wiles 的证明表明,任何形如 \( y^2 = x(x - an)(x + bn) \) 的方程确实具有模形式。因此,任何 \( x^p + y^p = z^p \)(其中 \( p \) 为奇素数)的非平凡解将会产生矛盾,从而证明没有非平凡解存在。\(^\text{[18]}\)
\end{itemize}
换句话说,任何可能与费马大定理相矛盾的解,也可以用来与模形式定理相矛盾。因此,如果模形式定理被证明为真,那么也可以推导出费马大定理不存在矛盾。正如上文所述,这一等价表述的发现对费马大定理最终的解决至关重要,因为它提供了一种可以“一次性”对所有数字进行“攻克”的方法。
\subsection{数学史}  
\subsubsection{毕达哥拉斯和丢番图}  
\textbf{毕达哥拉斯三元组 } 

在古代,人们已经知道,边长比为 3:4:5 的三角形其一个角是直角。这在建筑和早期几何学中得到了应用。人们还知道,这是一个普遍规律的例子:任何三角形,如果两边的长度各自平方后相加(\( 3^2 + 4^2 = 9 + 16 = 25 \)),等于第三边的平方(\( 5^2 = 25 \)),那么这个三角形也是直角三角形。这就是现在所称的毕达哥拉斯定理,满足这一条件的三个数称为毕达哥拉斯三元组;这两者都以古希腊的毕达哥拉斯命名。例子包括 (3, 4, 5) 和 (5, 12, 13)。这样的三元组有无数个,\(^\text{[19]}\)生成这些三元组的方法在许多文化中都被研究过,从巴比伦人开始,\(^\text{[20]}\)然后是古希腊、中国和印度的数学家们。\(^\text{[1]}\)数学上,毕达哥拉斯三元组的定义是满足方程\( a^2 + b^2 = c^2 \)的三个整数\(^\text{[21]}\)\((a,b,c)\)。

\textbf{丢番图方程}  

费马方程\( x^n + y^n = z^n \)的正整数解是丢番图方程的一个例子,\(^\text{[22]}\) 该方程以公元3世纪的亚历山大数学家丢番图命名,他研究了这些方程并发展了求解某些类型丢番图方程的方法。典型的丢番图问题是找到两个整数\( x \)和\( y \),使得它们的和以及它们的平方和分别等于给定的两个数\( A \)和\( B \):
\[
A = x + y~
\]
\[
B = x^2 + y^2.~
\]
丢番图的主要著作是《算术》,其中只有一部分得以保存。\(^\text{[23]}\)费马关于最后定理的猜想灵感来源于阅读《算术》的新版本,\(^\text{[24]}\)该版本由Claude Bachet于1621年翻译成拉丁文并出版。\(^\text{[25][26]
}\)

丢番图方程已经研究了数千年。例如,二次丢番图方程\( x^2 + y^2 = z^2 \) 的解由毕达哥拉斯三元组给出,最早由巴比伦人(约公元前1800年)解决。\(^\text{[27]}\)线性丢番图方程的解,例如\( 26x + 65y = 13 \),可以使用欧几里得算法(约公元前5世纪)求得。\(^\text{[28]}\)从代数的角度来看,许多丢番图方程的形式与费马最后定理的方程相似,因为它们没有交叉项混合两个字母,但没有共享它的特定性质。例如,已知存在无数个正整数\( x, y, z \),使得\( x^n + y^n = z^m \),其中\( n \)和\( m \)是互质的自然数。\(^\text{[注2]}\)
\subsubsection{费马的猜想}
\begin{figure}[ht]
\centering
\includegraphics[width=6cm]{./figures/ea2ec1c6d24ec545.png}
\caption{《算术》1621年版中的问题II.8。右侧是费马所谓的“最后定理”证明的边注,因边缘过小而无法容纳该证明。} \label{fig_FMDL_1}
\end{figure}
《算术》中的问题 II.8 问的是如何将一个给定的平方数分解为两个其他的平方数;换句话说,给定一个有理数\( k \),找到有理数\( u \)和\( v \),使得 \( k^2 = u^2 + v^2 \)。丢番图展示了如何解决这个平方和问题,当\( k = 4 \)时,解为\( u = \frac{16}{5} \)和 \( v = \frac{12}{5} \)。\(^\text{[29]}\)

大约在1637年,费马在他那本《算术》副本的边缘写下了他的最后定理,紧接着丢番图的平方和问题旁边:\(^\text{[30][31][32]}\)

"Cubum autem in duos cubos, aut quadratoquadratum in duos quadratoquadratos & generaliter nullam in infinitum ultra quadratum potestatem in duos eiusdem nominis fas est dividere cuius rei demonstrationem mirabilem sane detexi. Hanc marginis exiguitas non caperet."

“将一个立方数分解为两个立方数,或者一个四次方分解为两个四次方,或者一般来说,将任何大于二次方的幂分解为两个相同幂次的数,这是不可能的。我已经发现了一个真正奇妙的证明,但这个边缘太窄,无法容纳它。”\(^\text{[33][34]}\)

费马去世后(1665年),他的儿子克莱芒-塞缪尔·费马出版了这本书的新版本(1670年),并增加了他父亲的评论。\(^\text{v}\)虽然当时这并不是真正的定理(即没有证明的数学陈述),但这个边注随着时间的推移被称为费马的最后定理,\(^\text{[30]}\)因为它是费马提出的最后一个未被证明的定理。\(^\text{[36][37]}\)

目前尚不清楚费马是否确实为所有指数\( n \)找到了有效的证明,但似乎不太可能。仅有一份与此相关的证明流传下来,即\( n = 4 \)的情况,正如在 § 特定指数的证明 部分所描述的。

虽然费马将\( n = 4 \)和\( n = 3 \)的情况作为挑战题目给他的数学通信者们,如马林·梅尔森、布莱兹·帕斯卡和约翰·沃利斯,\(^\text{[38]}\)他从未提出过一般情况。\(^\text{[39]}\)此外,在他生命的最后三十年里,费马再也没有写过关于他“真正奇妙的证明”的任何内容,也从未将其发表。范·德·波尔滕\(^\text{[39]}\)建议,尽管缺乏证明并不重要,但缺乏挑战意味着费马意识到自己没有证明;他引用了魏尔\(^\text{[40]}\)的话,认为费马可能曾一时自欺,产生了一个无法挽回的错误想法。费马可能在这种“奇妙证明”中使用的技巧至今未知。

怀尔斯和泰勒的证明依赖于20世纪的技术。\(^\text{[41]}\)与此相比,费马的证明必须是初等的,因为那时的数学知识有限。

虽然哈维·弗里德曼的宏大猜想暗示,任何可证明的定理(包括费马最后定理)都可以仅使用“初等函数算术”来证明,但这样的证明在技术上可能被称为“初等的”,并且可能涉及数百万个步骤,因此远远超出费马所能提供的证明长度。
\subsubsection{特定指数的证明}
\textbf{指数 = 4}  

费马留下的唯一相关证明使用了无限下降法,证明了具有整数边的直角三角形的面积永远不可能等于某个整数的平方。\(^\text{[42][43][44]}\)他的证明等价于证明方程
\[
x^4 - y^4 = z^2~
\]
在整数中没有原始解(没有互质解)。反过来,这证明了费马最后定理在\( n = 4 \)的情况下成立,因为方程\( a^4 + b^4 = c^4 \)可以写成\( c^4 - b^4 = (a^2)^2 \)。

关于 \( n = 4 \) 情况的替代证明后来由以下数学家发展:\(^\text{[45]}\)弗雷尼克尔·德·贝西(1676),\(^\text{[46]}\)莱昂哈德·欧拉(1738),\(^\text{[47]}\)卡乌斯勒(1802),\(^\text{[48]}\)彼得·巴洛(1811),\(^\text{[49]}\)阿德里安-玛丽·勒让德(1830),\(^\text{[50]}\)绍皮斯(1825),\(^\text{[51]}\)奥尔里·泰尔凯姆(1846),\(^\text{[52]}\)约瑟夫·伯特朗(1851),\(^\text{[53]}\)维克多·勒贝格(1853, 1859, 1862),\(^\text{[54]}\)特奥菲尔·佩平(1883),\(^\text{[55]}\) 塔费尔马赫(1893),\(^\text{[56]}\)大卫·希尔伯特(1897),\(^\text{[57]}\)本兹(1901),\(^\text{[58]}\)甘比奥利(1901),\(^\text{[59]}\)利奥波尔德·克罗内克(1901),\(^\text{[60]}\)班格(1905),\(^\text{[61]}\)索默(1907),\(^\text{[62]}\)博塔里(1908),\(^\text{[63]}\)卡雷尔·里赫利克(1910),\(^\text{[64]}\)努茨霍恩(1912),\(^\text{[65]}\)罗伯特·卡迈克尔(1913),\(^\text{[66]}\)汉考克(1931),\(^\text{[67]}\)乔治·弗朗切努(1966),\(^\text{[68]}\)格兰特与佩雷拉(1999),\(^\text{[69]}\)巴巴拉(2007),\(^\text{[70]}\)和多兰(2011)。\(^\text{[71]}\)

\textbf{其他指数}  

在费马证明了特例\( n = 4 \)后,所有\( n \)的一般证明只需证明定理对于所有奇素数指数成立。\(^\text{[72]}\)换句话说,只需要证明方程\( a^n + b^n = c^n \)在\( n \)是奇素数时没有正整数解(\( a, b, c \))。这是因为,对于给定的\( n \),一个解\( (a, b, c)\)等价于所有\( n \)的因子的解。举例来说,设\( n \)被分解为\( d \)和\( e \),即\( n = de \)。则一般方程
\[
a^n + b^n = c^n~
\]
意味着\( (a^d, b^d, c^d) \)是指数\( e \)的解
\[
(a^d)^e + (b^d)^e = (c^d)^e.~
\]
因此,为了证明费马方程在\( n > 2 \)时没有解,只需证明它在每个\( n \) 的至少一个素因子上没有解即可。每个整数\( n > 2 \)都可以被 4 或某个奇素数(或两者)整除。因此,如果费马的最后定理能在\( n = 4 \)和所有奇素数\( p \)上证明成立,那么它就能为所有\( n \)证明成立。

在其猜想后的两个世纪(1637–1839),费马的最后定理在三个奇素数指数\( p = 3, 5, 7 \)上被证明。\( p = 3 \)的情况最早由阿布·马哈茂德·霍贾迪(10世纪)提出,但他对定理的证明是错误的。\(^\text{[73][74]}\)1770年,莱昂哈德·欧拉给出了\( p = 3 \)的证明,\(^\text{[75]}\)但他通过无限下降法\(^\text{[76]}\)的证明存在一个重大漏洞。\(^\text{[77][78][79]}\)然而,由于欧拉本人在其他工作中已证明了完成证明所需的引理,因此他通常被认为是第一个提供证明的人。\(^\text{[44][80][81]}\)独立的证明分别由以下数学家发表:\(^\text{[82]}\)卡乌斯勒(1802),\(^\text{[48]}\)勒让德(1823, 1830),\(^\text{[50][83]}\)卡尔佐拉里(1855),\(^\text{[84]}\)加布里埃尔·拉梅(1865),\(^\text{[85]}\)彼得·古斯特里·泰特(1872),\(^\text{[86]}\)西格蒙德·冈瑟(1878),\(^\text{[87]}\) 甘比奥利(1901),\(^\text{[59]}\)克雷(1909),\(^\text{[88]}\)里赫利克(1910),\(^\text{[64]}\)斯托克豪斯(1910),\(^\text{[89]}\)卡迈克尔(1915),\(^\text{[90]}\)约翰内斯·范·德·科尔普特(1915),\(^\text{[91]}\)阿克塞尔·图厄(1917),\(^\text{[92]}\)和杜阿尔特(1944)。\(^\text{[93]}\)

\( p = 5 \)的情况约在1825年由勒让德和彼得·古斯塔夫·勒让·狄利克雷独立证明。\(^\text{[94][95][96][44][97]}\)曹·弗里德里希·高斯(1875年,死后)、\(^\text{[99]}\)勒贝格(1843年)、\(^\text{[100]}\)拉梅(1847年)、\(^\text{[101]}\)甘比奥利(1901年)、\(^\text{[59][102]}\)韦雷布鲁索夫(1905年)、\(^\text{[103]}\)里赫利克(1910年)、\(^\text{[104]}\)范·德·科尔普特(1915年)\(^\text{[91]}\)和吉·特尔贾尼安(1987年)\(^\text{[105]}\)也提出了替代证明。

\( p = 7 \)的情况由拉梅于1839年证明。\(^\text{[106][107][44][97]}\)他相当复杂的证明在1840年由勒贝格简化,\(^\text{[109]}\)而更简洁的证明\(^\text{[110]}\)则由安杰洛·杰诺基于1864年、1874年和1876年发布。\(^\text{[111]}\)替代证明由特奥菲尔·佩平(1876年)\(^\text{[112]}\)和埃德蒙·梅耶(1897年)\(^\text{[113]}\)提出。

费马最后定理也在指数\( n = 6 \)、\( n = 10 \)和\( n = 14 \)的情况下得到了证明。关于\( n = 6 \)的证明由卡乌斯勒\(^\text{[48]}\)、图厄\(^\text{[114]}\)、塔费尔马赫\(^\text{[115]}\)、林德\(^\text{[116]}\)、卡普费尔尔\(^\text{[117]}\)、斯威夫特\(^\text{[118]}\)和布鲁什\(^\text{[119]}\)发布。同样,狄利克雷\(^\text{[120]}\)和特尔贾尼安\(^\text{[121]}\)各自证明了\( n = 14 \)的情况,而卡普费尔尔\(^\text{[117]}\)和布鲁什\(^\text{[119]}\)各自证明了\( n = 10 \)的情况。严格来说,这些证明是多余的,因为这些情况分别可以通过\( n = 3 \)、\( n = 5 \)和\( n = 7 \)的证明推导出来。然而,这些偶数指数的证明与奇数指数的证明不同。狄利克雷对于\( n = 14 \)的证明发表于1832年,早于拉梅在1839年证明\( n = 7 \)的证明。\(^\text{[122]}\)

所有特定指数的证明都使用了费马的无限下降法,\(^\text{[citation needed]}\)无论是其原始形式,还是在椭圆曲线或阿贝尔流形上的下降形式。然而,细节和辅助论证往往是临时的,并且与所考虑的具体指数相关。\(^\text{[123]}\)随着\( p \)的增大,这些证明变得越来越复杂,因此似乎不太可能通过在个别指数的证明基础上推导出费马最后定理的一般情况。\(^\text{[123]}\)尽管尼尔斯·亨里克·阿贝尔和彼得·巴洛在19世纪初发表了一些关于费马最后定理的普遍结果,\(^\text{[124][125]}\)但对一般定理的首次重要工作是由索菲·热尔曼完成的。\(^\text{[126]}\)
\subsubsection{早期的现代突破}  
\textbf{索菲·热尔曼} 

在19世纪初,索菲·热尔曼发展了几种新颖的方法,试图证明费马最后定理对所有指数成立。\(^\text{[127]}\)首先,她定义了一组辅助素数\( \theta \),它是通过公式\( \theta = 2hp + 1 \)从素数指数\( p \)构造的,其中\( h \)是一个不被3整除的整数。她证明了,如果没有整数的 \( p \)次幂在模\( \theta \)下是相邻的(即非连续性条件),那么\( \theta \)必须整除积\( xyz \)。她的目标是通过数学归纳法证明,对于任何给定的\( p \),无穷多个辅助素数\( \theta \)满足非连续性条件,从而整除\( xyz \);由于积\( xyz \)最多只能有有限数量的素因子,这样的证明就可以确立费马最后定理。尽管她发展了许多技术来确立非连续性条件,但她未能实现她的战略目标。她还努力为给定指数 \( p \) 的费马方程的解设定下限,其中一个修改版由阿德里安-玛丽·勒让德出版。作为这项后期工作的副产品,她证明了索菲·热尔曼定理,验证了费马最后定理的第一个情况(即\( p \)不整除\( xyz \)的情况),对于每个小于270的奇素数指数成立,\(^\text{[127][128]}\)并且对于所有素数\( p \),只要\( 2p + 1 \)、\( 4p + 1 \)、\( 8p + 1 \)、\( 10p + 1 \)、\( 14p + 1 \) 和 \( 16p + 1 \)中至少有一个是素数(特别地,\( 2p + 1 \)是素数的素数\( p \)被称为索菲·热尔曼素数)。热尔曼未能成功证明费马最后定理的第一个情况对于所有偶数指数成立,特别是对于\( n = 2p \),该情况在1977年由吉·特尔贾尼安证明。\(^\text{[129]}\)1985年,莱昂纳德·阿德尔曼、罗杰·希思-布朗和埃蒂安·福夫里证明了费马最后定理的第一个情况对于无穷多个奇素数 \( p \)成立。\(^\text{[130]}\)

\textbf{恩斯特·库默与理想数的理论}  

1847年,加布里埃尔·拉梅提出了一个基于在复数中因式分解方程\( x^p + y^p = z^p \)的费马最后定理证明,特别是基于数1的根构造的循环域。他的证明失败了,因为他错误地假设这些复数可以像整数一样唯一地分解为素数。这个缺陷很快被约瑟夫·柳维尔指出,柳维尔后来阅读了恩斯特·库默的论文,证明了这一唯一分解失败的问题。

库默的任务是确定是否可以将循环域推广到包括新的素数,从而恢复唯一分解的性质。他通过发展理想数理论成功地完成了这一任务。

(人们常说库默因为对费马最后定理的兴趣而得出了他的“理想复数”;甚至有一个故事常被讲述,说库默像拉梅一样认为他已经证明了费马最后定理,直到勒让·狄利克雷告诉他他的论证依赖于唯一分解;但这个故事最早由库尔特·亨塞尔在1910年讲述,证据表明它很可能来源于亨塞尔某些来源的混淆。哈罗德·爱德华兹表示,认为库默主要对费马最后定理感兴趣的观点“无疑是错误的”。\(^\text{[131]}\)请参见理想数的历史。)

通过使用拉梅概述的一般方法,库默证明了费马最后定理对于所有常规素数的两个情况。然而,他无法证明定理对于那些在理论上大约39\%情况下出现的特殊素数(不规则素数);低于270的唯一不规则素数是37、59、67、101、103、131、149、157、233、257和263。

\textbf{莫尔德尔猜想}

在1920年代,路易斯·莫尔德尔提出了一个猜想,暗示如果指数\( n \)大于二,那么费马方程最多只有有限个非平凡的原始整数解。\(^\text{[132][133]}\)这个猜想在1983年由格尔德·法尔廷斯证明,\(^\text{[134]}\)现在被称为法尔廷斯定理。

\textbf{计算研究}  

在20世纪后半叶,计算方法被用来扩展库默的方法,以处理不规则素数。1954年,哈里·范迪弗使用SWAC计算机证明了费马最后定理对于所有小于2521的素数成立。\(^\text{[135]}\)到1978年,塞缪尔·瓦格斯塔夫将这一结果扩展到所有小于125,000的素数。\(^\text{[136]}\)到1993年,费马最后定理已被证明对所有小于四百万的素数成立。\(^\text{[5]}\)

然而,尽管有这些努力和成果,费马最后定理的证明仍然不存在。由于其性质,个别指数的证明永远无法证明一般情况:即使所有指数都在一个极大的数字\( X \)以下得到了验证,仍然可能存在一个超过\( X \)的更高指数,对于该指数,定理可能不成立。(这与一些过去的猜想相似,如斯库伊斯数的情况,不能排除在这个猜想中出现类似的情况。)
\subsection{与椭圆曲线的联系}  
最终导致费马最后定理成功证明的策略来源于“惊人的”\(^\text{[137]: 211}\) 塔尼亚马–岛村–魏尔猜想,该猜想大约在1955年提出——许多数学家认为这个猜想几乎不可能证明,\(^\text{[137]: 223 }\)并且在1980年代,由格哈德·弗雷、让-皮埃尔·塞尔和肯·里贝特将其与费马方程联系起来。通过在1994年完成该猜想的部分证明,安德鲁·怀尔斯最终成功证明了费马最后定理,并为其他数学家提供了通往完整证明的道路,这一证明现在被称为模形式定理。

\textbf{塔尼亚马–岛村–魏尔猜想 } 

大约在1955年,日本数学家岛村五郎和谷山丰观察到椭圆曲线和模形式这两个看似完全不同的数学分支之间可能存在联系。由此产生的模形式定理(当时称为塔尼亚马–岛村猜想)指出,每个椭圆曲线都是模的,意味着它可以与一个唯一的模形式相关联。

最初,这一联系被认为不太可能或高度推测,但当数论学家安德烈·魏尔找到支持这一联系的证据时,这一猜想被更加认真地对待,尽管魏尔并没有证明它;因此,该猜想通常被称为塔尼亚马–岛村–魏尔猜想。\(^\text{[137]: 211–215  }\)

即便在引起广泛关注之后,这一猜想仍然被当时的数学家视为极其困难,甚至可能无法证明。\(^\text{[137]: 203–205, 223, 226}\) 例如,怀尔斯的博士导师约翰·科茨表示,这似乎是“根本不可能实际证明的”,\(^\text{[137]: 226}\) 而肯·里贝特则认为自己“是绝大多数认为它完全无法接近的人的一员”,并补充说:“安德鲁·怀尔斯可能是地球上为数不多的几位敢于梦想自己能去证明它的人之一。”\(^\text{[137]: 223 }\)

\textbf{里贝特定理与弗雷曲线}  

主条目:弗雷曲线 和 里贝特定理  
1984年,格哈德·弗雷注意到费马方程与模形式定理之间的联系,当时模形式定理仍然是一个猜想。如果费马方程对于指数\( p > 2 \)有任何解(\( a, b, c \)),那么可以证明,半稳定的椭圆曲线(现在被称为弗雷-赫勒戈阿尔什曲线\(^\text{[note 3]}\))
\[
y^2 = x(x - ap)(x + bp)~
\]
会具有如此特殊的性质,以至于它不太可能是模形式的。\(^\text{[138]}\)这与模形式定理相冲突,后者断言所有椭圆曲线都是模的。因此,弗雷观察到,塔尼亚马–岛村–魏尔猜想的证明可能同时也能证明费马最后定理。\(^\text{[139][140]}\)通过对立推理,费马最后定理的反证或驳斥将驳斥塔尼亚马–岛村–魏尔猜想。

简而言之,弗雷证明了,如果他对方程的直觉是正确的,那么任何一组能够反驳费马最后定理的四个数字(\( a, b, c, n \)),也可以用来反驳塔尼亚马–岛村–魏尔猜想。因此,如果后者为真,前者就不能被反驳,也必须为真。

按照这一策略,证明费马最后定理需要两个步骤。首先,需要证明模形式定理,或至少证明它对于包括弗雷方程在内的椭圆曲线类型(即半稳定椭圆曲线)成立。这在当时的数学家看来几乎是无法证明的。\(^\text{[137]: 203–205, 223, 226}\) 其次,需要证明弗雷的直觉是正确的:即如果以这样的方式构造椭圆曲线,使用的是费马方程的解集,那么得到的椭圆曲线不可能是模形式的。弗雷证明了这一点是合理的,但并没有给出完整的证明。缺失的部分(即所谓的“ε猜想”,现称为里贝特定理)被让-皮埃尔·塞尔发现,他也给出了几乎完整的证明,弗雷所提示的联系最终在1986年由肯·里贝特证明。\(^\text{[141]}\)

在弗雷、塞尔和里贝特的工作之后,情况是这样的:
\begin{itemize}
\item 费马最后定理需要为所有素数指数\( n \)证明。
\item 如果模形式定理对于半稳定椭圆曲线得到了证明,那么这意味着所有半稳定椭圆曲线必须是模形式的。
\item 里贝特定理表明,任何一个费马方程的素数解都可以用来创建一个不可能是模形式的半稳定椭圆曲线;
\item 只有在这两个陈述都为真的情况下,费马方程没有解(因为那时无法创建这样的曲线),这正是费马最后定理所说的。由于里贝特定理已经证明,这意味着模形式定理的证明将自动证明费马最后定理也为真。
\end{itemize}
\subsubsection{怀尔斯的普遍证明}
\begin{figure}[ht]
\centering
\includegraphics[width=6cm]{./figures/26fc7697185b9923.png}
\caption{英国数学家安德鲁·怀尔斯} \label{fig_FMDL_2}
\end{figure}
1986年,里贝特证明了ε猜想,实现了弗雷提出的两个目标中的第一个。得知里贝特的成功后,英国数学家安德鲁·怀尔斯——他从小对费马最后定理充满兴趣,并且曾研究过椭圆曲线——决定致力于实现第二个目标:为半稳定椭圆曲线证明模形式定理的特例(当时被称为塔尼亚马–岛村猜想)。\(^\text{[142][143]}\)

怀尔斯在近乎完全保密的情况下,花费了六年的时间从事这一任务,通过将之前的工作分割成小部分作为单独的论文发布,并且只与妻子分享他的进展,来掩饰自己的努力。\(^\text{[137]: 229–230}\) 他最初的研究建议通过归纳法来证明\(^\text{[137]: 230–232, 249–252 }\),并且他在初步工作和第一次重大突破时依赖了伽罗瓦理论\(^\text{[137]: 251–253, 259 }\),然后在1990年至1991年左右,当现有方法似乎无法解决问题时,他转而尝试扩展水平岩泽理论进行归纳论证。\(^\text{[137]: 258–259}\) 然而,到1991年中期,岩泽理论似乎也未能触及问题的核心。\(^\text{[137]: 259–260 [144][145]}\)为此,他向同事求助,寻找任何前沿研究和新技术,并发现了由维克托·科利亚金和马蒂亚斯·弗拉赫最近开发的欧拉系统,这一系统似乎是“量身定制”的,适合他证明中的归纳部分。\(^\text{[137]: 260–261}\) 怀尔斯研究并扩展了这一方法,结果证明是有效的。由于他的工作广泛依赖这一新方法,这对于数学界和怀尔斯来说都是新的,在1993年1月,他请求普林斯顿的同事尼克·卡茨帮助他检查推理中的微妙错误。他们当时的结论是,怀尔斯使用的技术似乎是正确的。\(^\text{[137]: 261–265 [146][147]}\)

到1993年5月中旬,怀尔斯准备告诉他的妻子,他认为自己已经解决了费马最后定理的证明问题,\(^\text{[137]: 265}\) 并且到6月,他感到足够自信,准备在1993年6月21日至23日于艾萨克·牛顿数学科学研究所举办的三场讲座中展示他的结果。\(^\text{[148][149]}\) 具体来说,怀尔斯展示了他对半稳定椭圆曲线的塔尼亚马–岛村猜想的证明;结合里贝特对ε猜想的证明,这就暗示了费马最后定理。然而,在同行评审过程中,证明中的一个关键点被发现是错误的。该错误涉及对一个特定群的阶数的界限。这个错误被几位审阅怀尔斯手稿的数学家发现,包括卡茨(作为审稿人)\(^\text{[150]}\),他在1993年8月23日提醒了怀尔斯。\(^\text{[151]}\)

这个错误并不会使他的工作变得毫无价值:怀尔斯工作的每一部分都具有重要性和创新性,他在工作过程中创造的许多发展和技术也同样重要,只有其中一部分受到了影响。\(^\text{[137]: 289, 296–297}\) 然而,在没有证明这一部分的情况下,费马最后定理实际上并没有被证明。怀尔斯花了近一年的时间试图修复他的证明,最初是自己修复,后来与他的前学生理查德·泰勒合作,仍然没有成功。\(^\text{[152][153][154]}\)到1993年底,谣言开始传播,怀尔斯的证明在审查下失败了,但具体的严重程度尚不清楚。数学家们开始施压怀尔斯,要求他公开他的工作,无论是否完整,这样更广泛的社区可以探索并利用他所取得的任何成就。但问题并没有像最初看起来那么微不足道,而是变得非常重要,远比预期的严重,也不容易解决。\(^\text{[155]}\)