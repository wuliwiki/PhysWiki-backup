% 南京理工大学 2010 年 研究生入学考试试题 普通物理(B)
% license Usr
% type Note

\textbf{声明}:“该内容来源于网络公开资料,不保证真实性,如有侵权请联系管理员”

\subsection{一。填空题(32分,每空2分)}
\begin{enumerate}
    \item 已知一电子的运动方程可表示为 $r = b \cos \omega t\vec{i} + b \sin \omega t\vec{j} + ct\vec{k}$,式中 $a,b$ 为常数,$t$以秒计。$r$以来计,随在$t$时刻,电子的速度为__________,加速度为 ___________。
    \item 一质量为$m$的小球系在长为$L$的细绳的一端,绳的另一端固定于$O$点。先使小球以$v_0$速度做圆周水平匀速运动,然后细绳逐渐缩短,绳始终与运动方向夹角为$\theta$的小球的速度表达式为___________ ,细绳的张力为多大为 ___________。
    \item 设一平面简谐波沿 $x$ 轴正方向传播,已知 $x=0$ 处原点的振动方程为$ y = A \cos(\omega t - \pi/3)$,波速为 $v$。波在 $x=L$ 处终止反射,则$x=x_0$ 处 $(x_0 < L)$ 原点由于反射被引起的振动方程为 ___________ ,$x_0$ 处是波节位置的条件是 $x_0 =$___________。
    \item 如图所示,一沿正 $x$ 方向传播的平面简谐波,波速为 $v = 200 m/s$,波长 $\lambda = 20 m$,则 $x = 0$ 处质点的振动方程为_________,该平面简谐波方程为_________。
\begin{figure}[ht]
\centering
\includegraphics[width=6cm]{./figures/31f0d09795712f3c.png}
\caption{} \label{fig_NJU10_1}
\end{figure}
    \item $2 mol$ 氧气在27°C时的内能等于 ___________,其分子的平均动能是 ___________ ,平均平动动能是 ___________。
    \item 设一个气体分子的密度分布函数为$f(v)$,则单位体积中,$v_1\to v_2$区间内的分子数为 ___________。
    \item 带电量为 $q$ 半径为 $R_1$ 的导体球 $A$,与内、外半径分别为 $R_2$ 和 $R_3$ 接地的同心金属球壳 $B$ 间充满介电常数为 $\varepsilon$ 的介质,构成一球形电容器。则该电容器的电容 $C=$___________。设导体球 $A$ 带电 $q$,则该电容器内任一点 $P$ 处的电场强度 $E=$___________,电容器储存的电能___________。若球壳$B$ 接地,则导体球 $A$ 的电势为 ___________。
\end{enumerate}
\subsection{二、填空题(32分。每空2分)}
1. 半径为 $R$ 的圆环,均匀带电,单位长度的电量为 $\lambda$。 以每秒 $n$ 转绕 通过环心并与环面需直的轴作等速转动。则环的等效磁矩大小为___________,轴线上距坏心为$x$处的任一点$p$的磁感应强度大小为___________。
\begin{figure}[ht]
\centering
\includegraphics[width=6cm]{./figures/9eca76d12caabf5d.png}
\caption{} \label{fig_NJU10_2}
\end{figure}
2. 均匀磁场 $B$ 中置一直角边长为 $a$通有强度为 $I$ 的稳恒电流的等腰直角三角形线圈 $ACD$。 使线圈绕 $AC$ 边匀速转动,线圈平面与磁场方向平行。 如图所示, 现线圈所受的力矩的大小为___________。 在磁力矩作用下,线圈平面绕 $AC$ 边转过 $\pi/3$ 圈, 磁力矩做的功($I$ 在旋转过程中不变)为___________。
\begin{figure}[ht]
\centering
\includegraphics[width=6cm]{./figures/1b69f57a20f285d4.png}
\caption{} \label{fig_NJU10_3}
\end{figure}
3. 在恒真空场的均匀磁场中,长为 $l$ 的导体棒 $ab$ 以$\omega$逻时针绕$a$点勾速转动, 如图, 则切出电动势的大小为___________,且 ___________ 点的电势为。
\begin{figure}[ht]
\centering
\includegraphics[width=6cm]{./figures/bc3edc382260c554.png}
\caption{} \label{fig_NJU10_4}
\end{figure}
4. 在真空中,一平面电磁波的磁场 $B = B_0 = B_0 \cos \left( \omega (t + \frac{z}{c} )\right) \\, (T)$。则该电磁波的传播方向为___________,电场强度为___________。

5.用氮-氖撒光器发出的波长为$632.8nm$的单色光徽牛领环实验,已知所用平凸透镜的鱼率半径为$10.0m$,平面直径为$3.0cm$,则能观察列___________,条暗坏。若把整个装置放入水中$(\eta=1.33)$,总观察到___________,条暗坏:

6、两个偷报化方向相互看宜的偏报片平行放置,组合战正交偏报片。现让光强为$I_0$的一東自然光垂直射入谈正交偏探片,则透射光强为___________。若在阿偏振片之间放入第三块偷报片。其偏化方向与第一块偷振片的偏振化方向央角为 30”,则透射光强为___________。

7、动能$E=1.53MV$的电子运动的德布罗意波长为___________。

8、处于第一灏发态的氢原子的势能为___________,其核外电子绕核运动的动能为___________。

9. 已知一维无限深势井中粒子的波函数为: $\psi_n(x) = \sqrt{\frac{2}{a}} \sin{\frac{n \pi }{a}}x$. 设 $n = 1$ 时, 粒子在 $x = \frac{a}{3}$ 处出现的概率密度为: ___________。
\subsection{三、(12分)}
如图所示,水平桌面上。一长为$l = 1.0m$:质量为$m_1 = 3.0kg$的匀质烟杆,一端围定于$O$点,细杆可经过$O$点竖直轴在水平桌面上标动烟杆与桌面间的摩操系数为$\mu=0.20$,开始时杆静止。现有一质量为 $m_2 = 20g$速度 $v_0 = 400m/s$,沿水平方向以与杆成$\theta = 30^\circ$常射入杆的中点日留在杆内。来:
(1)擅击后杆开始神动的角速度大小:
(2)子弹射入后,细扦所受摩擦力矩:
(3)烟杆的角加速度
\begin{figure}[ht]
\centering
\includegraphics[width=6cm]{./figures/24f68f40a0af4a39.png}
\caption{} \label{fig_NJU10_5}
\end{figure}
\subsection{四、(12分)}
如图,一质量$m=2.0kg$的物体沿$x$轴做简谐振动,振幅为$0.12m$,周期为$2t$初始时$x_0=0.06m$并向$x$独正向运动。求:
(1)物体的运动方程:
(2)物体从初始时刻运动到平衡位置所需要最短时间;
(3)物体在平衡位置时所具有的机械能:
\begin{figure}[ht]
\centering
\includegraphics[width=6cm]{./figures/c2c396c185e026c7.png}
\caption{} \label{fig_NJU10_6}
\end{figure}
\subsection{五. (18 分)}
一摩尔的双原子理想气体,初始时压强为 $2a_0$,体积为 $20V_0$。先等压膨胀至体积变为 2 倍,再等容冷却至原来温度,最后等温压缩回到初态。
\begin{enumerate}
    \item 作出该过程中 $p-V$ 图;
    \item 求气体在各过程中的功;
    \item 该循环的效率。
\end{enumerate}
\subsection{六、(12分)}
一无限长均匀带电圆柱,体电荷密度为 $\rho$,截面半径为 $R$。
\begin{enumerate}
    \item 用高斯定理求出柱体内外电场强度分布;
    \item 求出柱体内外的电势分布(以柱面为电势零点)。
\end{enumerate}
\subsection{七、(10分)}
如图为矩形,总匝数为 $N$ 的通有电流 $I$ 螺绕环,尺寸如图所示。求:

(1) 环内磁感应强度的分布;

(2) 通过螺绕环截面(图中阴影区)的磁通量。
\begin{figure}[ht]
\centering
\includegraphics[width=6cm]{./figures/e5cc39f91bf9a0cf.png}
\caption{} \label{fig_NJU10_7}
\end{figure}
\subsection{八、(10分)}
一螺绕环,横截面的半径为 $a$,中心线的半径为 $R$,$R \gg a$。其上由表面绝缘的导线均匀地绕密绕两个线圈,一个 $N_1$ 匝,另一个 $N_2$ 匝。求:
\begin{enumerate}
    \item 两线圈的自感 $L_1$ 和 $L_2$;
    \item 两线圈的互感 $M$;
\end{enumerate}
\begin{figure}[ht]
\centering
\includegraphics[width=6cm]{./figures/f0966fa00d0f6220.png}
\caption{} \label{fig_NJU10_8}
\end{figure}
\subsection{九、(10分)}
光栅每厘米有 $2500$ 条纹线,且刻痕宽度 $b$ 是缝宽 $a$ 的3倍。若以 $\lambda_1 = 600 nm$ 的单色光垂直入射到光源上,求:
\begin{enumerate}
    \item 光栅常数;
    \item 在单缝衍射的中央明纹区域内,最多可见到多少条主极大明纹;
    \item 若用另一波长为 $\lambda_2$ 的单色光垂直入射,发现其第3级与 $\lambda_1$ 的第2级主极大明纹重合,$\lambda_2$ 的量值为多少?
\end{enumerate}
\subsection{十、(10分)}
$\pi$ 介子,相对静止时测得其平均寿命 $\tau_0 = 1.8 \times 10^{-8} \\ \mathrm{s}$,若使其以 $v = 0.6c$ 的速度离开加速器,求:

(1) 从实验室观测,$\pi$ 介子的平均寿命;

(2) 若在寿命为 $\tau_0$ 的 $\pi$ 介子上测量,能测到的实验室后退的距离。

附常用物理常数:

\begin{itemize}
    \item 电子静止质量 $m_0 = 9.1 \times 10^{-31} \\ \mathrm(kg)$
    \item 普朗克常数 $h = 6.63 \times 10^{-34} \\ \mathrm(J \cdot s)$
    \item 普适气体常数 $R = 8.31 \\ \mathrm(J/mol \cdot K)$
    \item 引力常量 $G = 6.67 \times 10^{-11} \\ \mathrm{N \cdot m^2 / kg^2}$
    \item 电子电量 $e = 1.6 \times 10^{-19} \\ \mathrm(C)$
    \item 真空中光速 $c = 3 \times 10^8 \\ \mathrm(m/s)$
    \item 玻尔兹曼常量 $k = 1.38 \times 10^{-23} \\ \mathrm(J/K)$
    \item 真空电容率 $\varepsilon_0 = 8.85 \times 10^{-12} \mathrm C^{-2}N^{-1}m^{-2}$
\end{itemize}
