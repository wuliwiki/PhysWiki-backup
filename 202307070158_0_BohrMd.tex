% 玻尔原子模型
% keys 玻尔|原子|半经典|氢原子|类氢原子

\pentry{圆周运动\upref{CMAD}, 库仑力\upref{ClbFrc}}

\begin{figure}[ht]
\centering
\includegraphics[width=5cm]{./figures/578c496b45038c58.pdf}
\caption{玻尔原子模型} \label{fig_BohrMd_1}
\end{figure}

\textbf{玻尔原子模型(Bohr Model)}(\autoref{fig_BohrMd_1})是量子力学发展的早期被提出的一种解释\textbf{类氢原子}光谱的模型。 该模型中, 我们假设原子核具有 $Z$ 个正电荷。 对于氢原子 $\mathrm{H}$ 有 $Z = 1$, 氦离子 $\mathrm{He}^+$ 有 $Z = 2$, 锂离子 $\mathrm{Li}^{++}$ 有 $Z = 3$ 等等。

由于原子核质量远大于电子, 我们先\textbf{假设原子核固定不动}, 若要考虑原子核运动使用相对坐标和约化质量\upref{TwoBD} $\mu$ 代替电子质量 $m$ 即可(\autoref{eq_HRMass_1}~\upref{HRMass})。
唯一的核外电子按照牛顿力学和库仑定律运动, 再人为地加上一个条件(\textbf{量子化条件})使电子的轨道角动量只能取一些特定的(\textbf{离散}的)值。 这样, 轨道的半径也只能取离散的值, 每个半径 $r_n$ 对应一个机械能(动能加势能) $E_n$, 我们把这些能量叫做\textbf{能级}。 如果电子从一条能量较高的轨道跃迁到另一条能量较低的轨道, 那么一个光子将被产生, 带走两个轨道的能量差。 反之, 如果恰好有一个入射光子的能量是两条轨道机械能之差, 那这个光子就可以被低能量轨道的电子吸收, 使其跃迁到高能量轨道。

虽然这个模型成功地解释了氢原子各个能级的能量以及氢原子光谱%未完成, 介绍一下
, 但它却并不是完全按照量子力学的的方法来计算的。 按照(现代的)量子力学, 电子需要用波函数描述, 波函数由薛定谔方程计算得到, 所以不具有经典力学中“轨道” 的概念。

根据玻尔模型, 氢原子的各个能级的能量为
\begin{equation}
E_n =  - \frac{m e^4}{32 \pi^2 \epsilon_0^2 \hbar ^2} \frac{Z^2}{n^2} \approx - 13.6\Si{eV} \frac{Z^2}{n^2}
\qquad (n = 1, 2, \dots)~.
\end{equation}
可见 $n$ 越大, 能级越高。 我们把 $n = 1$ 的状态叫做\textbf{基态}, 其他状态叫做\textbf{激发态}。 公式中的常数因子($13.6\Si{eV}$)叫做\textbf{里德堡能量(Rydberg energy)}, 也就是玻尔模型中的基态能量。

各能级的轨道半径如下。 特殊地, 我们把氢原子基态($Z = 1$, $n = 1$) 的电子轨道的半径叫做\textbf{玻尔半径(Bohr radius)}, 记为 $a_0$。
\begin{equation}\label{eq_BohrMd_1}
r_n = a_0 \frac{n^2}{Z}
\qquad (n = 1, 2, \dots)~,
\end{equation}
\begin{equation}\label{eq_BohrMd_3}
a_0 = \frac{4\pi \epsilon_0 \hbar^2}{m_e e^2} \approx 5.292\e{-11}\Si{m}~.
\end{equation}


\subsection{能级公式推导}
% 未完成, 应该给出 $r_n$ 的公式
所有原子或离子中最简单的一类叫\textbf{类氢原子},类氢原子只有一个核外电子,以及一个带 $Z$ 个元电荷的原子核。以下的计算假设二者为质点和点电荷,原子核不动,电子绕原子核做圆周运动。运用经典力学和库仑力公式,可求出电子在不同半径下做圆周运动的能量。 库仑定律与牛顿定律(圆周运动)分别为
\begin{equation}
F = \frac{1}{4\pi \epsilon_0} \frac{(Ze)e}{r^2}~,
\qquad
F = ma = m\frac{v^2}{r}~.
\end{equation}
解得电子速度平方为
\begin{equation}\label{eq_BohrMd_2}
v^2 = \frac{e^2 Z}{4\pi \epsilon_0 mr}~.
\end{equation}
动能与势能分别为
\begin{equation}
E_K = \frac12 m v^2 = \frac{1}{8\pi\epsilon_0} \frac{Z e^2}{r}~,
\qquad
E_P =  -\frac{1}{4\pi\epsilon_0} \frac{Ze^2}{r} = -2 E_k~.
\end{equation}   
总能量为
\begin{equation}\label{eq_BohrMd_4}
E = E_K + E_P =  -\frac{Z e^2}{8\pi\epsilon_0 r}~.
\end{equation}
到此为止,我们还没有用到量子力学。然而这样的模型与真实的类氢原子相比有两个致命的缺陷: 第一,根据电动力学,圆周运动的电子会向外辐射电磁波,能量减少,最终坠入原子核; 第二,该模型允许氢原子的能量具有连续值(因为 $r$ 可连续变化),而实验中氢原子只能放出特定能量的光子,说明只能取特定的能量,即存在离散的\textbf{能级},我们把能级由低到高记为 $E$  $(n = 1,2,3\dots)$。 

以上矛盾说明微观世界的粒子不遵守经典力学和电磁学。玻尔为了解释实验,在经典力学和电磁学上加入了一个条件: 角动量量子化。

以原子核为原点,电子轨道平面的法向量为 $z$ 轴,由于电子的位矢 $\bvec r$ 与动量 $\bvec p$ 始终垂直,电子的角动量为
\begin{equation}
\bvec L = \bvec r \cross \bvec p = mvr \uvec z~,
\end{equation}
玻尔引入的角动量量子化条件为
\begin{equation}\label{eq_BohrMd_6}
L = mvr = n\hbar~,
\end{equation}
其中 $n$ 可以取任意正整数, $\hbar$ 为\textbf{约化普朗克常量}
\begin{equation}\label{eq_BohrMd_7}
\hbar  = \frac{h}{2\pi}~,
\end{equation}
该条件也可以等效理解为驻波条件,即允许的圆形轨道长度是德布罗意波% 在量子力学基本假设中提一下德布罗意波长, 未完成,引用
长的整数倍。
\begin{equation}\label{eq_BohrMd_8}
2\pi r  = \frac{h}{mv} n~.
\end{equation}
注意\autoref{eq_BohrMd_6} 与\autoref{eq_BohrMd_8} 等效。把\autoref{eq_BohrMd_2} 代入了该条件,解得可能的轨道半径为\autoref{eq_BohrMd_1}。 注意轨道与 $n^2$ 成正比, 和 $Z$ 成反比。

将 $r_n$(\autoref{eq_BohrMd_1}) 代入\autoref{eq_BohrMd_4}, 得到能级表达式为
\begin{equation}\label{eq_BohrMd_11}
E_n =  - \frac{mZ^2 e^4}{32\pi^2\epsilon_0^2 \hbar ^2} \frac{1}{n^2} \approx  - 13.6\Si{eV}\frac{Z^2}{n^2}~.
\end{equation}
对氢原子, 有 $Z = 1$, 最低的能级为 $n = 1$, 所以氢原子\textbf{基态}的能级 $E_0$ 约为 $-13.6\Si{eV}$。 这是一个著名的常数(若使用原子单位\upref{AU}, 这个值恰好是 $-1/2$)。

将 $r_n$ 代入\autoref{eq_BohrMd_2} 还可以得到电子速度为
\begin{equation}\label{eq_BohrMd_10}
v_n = \frac{Z e^2}{4\pi\epsilon_0\hbar} \frac{1}{n}~.
\end{equation}

\subsection{玻尔原子模型的局限性}
从\autoref{eq_BohrMd_6} 可以看出, 玻尔原子模型中角动量和能级是一一对应的。 而事实上现代的理论和实验告诉我们氢原子的每个能级都具有许多不同的角动量量子态, 对给定的 $n$, 角动量有 $n$ 个不同的值, 使用量子数 $l$ 来区分这些态, 对应的轨道角动量大小为\upref{QOrbAM}
\begin{equation}
L = \sqrt{l(l+1)}\hbar \qquad (l = 0,\dots,n-1)~.
\end{equation}

其次, 玻尔模型认为角动量矢量 $\bvec L$ 的方向可以是任意的, 也就是对某个给定能级的许多取向随机的氢原子测量某个给定方向的角动量分量可以得到连续的值, 但事实上只能得到 $2l+1$ 个分立的值:
\begin{equation}
L_z = m\hbar \qquad (m = -l,\dots,l)~.
\end{equation}
所以算下来每个能级 $n$ 共有 $1+3+\dots+[2(n-1)+1] = n^2$ 个不同的角动量状态(包括大小和取向)。 这是玻尔模型完全无法解释的。


