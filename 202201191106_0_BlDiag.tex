% 块对角矩阵

\pentry{分块矩阵\upref{BlkMat}, 矩阵与线性映射\upref{MatLS}, 子空间的直和、补空间\upref{DirSum}}

\textbf{块对角矩阵(block diagonal matrix)}一般是指由方形对角块组成的分块方阵, 例如
\begin{equation}\label{BlDiag_eq1}
\mat A = \pmat{1 & 1 & 0 & 0 & 0 & 0\\ 1 & 1 & 0 & 0 & 0 & 0\\ 0 & 0 & 2 & 0 & 0 & 0\\ 0 & 0 & 0 & 3 & 3 & 3 \\ 0 & 0 & 0 & 3 & 3 & 3\\ 0 & 0 & 0 & 3 & 3 & 3}
\end{equation}
也就是除了对角线上的方块(包括单个对角元)外, 其他矩阵元都为零. 我们把对角线上的方块称为\textbf{对角块(diagonal block)}, 注意对角块未必要求所有矩阵元都非零.

当我们把该矩阵乘以一个列矢量时, 可以把列矢量也进行同样的划分, 例如

下面来看如何把块对角矩阵与线性映射联系起来:
\begin{theorem}{}\label{BlDiag_the1}
若 $N$ 维线性空间 $V$ 中有若干 $N_i$ 维子空间 $V_i$($i=1,\dots,n$), 满足
\begin{equation}
V = V_1 \oplus V_2 \oplus \dots \oplus V_n
\end{equation}
每个 $V_i$ 的一组基底为 $v_{i,j}$ ($j=1,\dots,N_i$). 那么所有的 $v_{i,j}$ 是 $V$ 的一组基底(未必需要是正交归一的). 定义基底的顺序为
\begin{equation}
v_{1,1},\dots, v_{1,N_1}, v_{2,1}, \dots, v_{2,N_2}, \dots
\end{equation}
此时若线性映射 $A: V\to V$ 在每个 $V_i$ 中都闭合. 那么 $A$ 关于这组基底的矩阵就是块对角矩阵, 第 $i$ 块的大小为 $N_i\times N_i$. 第 $i$ 块对角块就是算符 $A:V_i\to V_i$ 关于基底 $v_{i,j}$ ($j=1,\dots,N_i$) 的矩阵.
\end{theorem}

证明留作习题. 请试着用\autoref{BlDiag_eq1} 的矩阵作为\autoref{BlDiag_the1} 的例子.
