% 惯性约束核聚变
% license CCBYSA3
% type Wiki

(本文根据 CC-BY-SA 协议转载自原搜狗科学百科对英文维基百科的翻译)

\textbf{惯性约束核聚变(ICF)}是一种聚变能试图发起的研究核聚变通过加热和压缩燃料靶进行的反应,通常为通常含有以下物质的混合物的颗粒形式氘和氚。典型的燃料球大约有针头那么大,大约有10个毫克燃料。

为了压缩和加热燃料,能量使用高能激光束、电子或离子传递到靶的外层,尽管出于各种原因,几乎所有ICF装置截至2015年使用激光。受热的外层向外爆炸,对目标的其余部分产生反作用力,向内加速,压缩目标。该过程旨在产生穿过目标向内传播的冲击波。一组足够强大的冲击波可以压缩和加热中心的燃料,以至于发生聚变反应。

ICF是聚变能量研究的两个主要分支之一,另一个是磁约束聚变。当ICF在20世纪70年代初首次提出时,它似乎是一种实用的发电方法,并且该领域蓬勃发展。20世纪70年代和80年代的实验表明,这些装置的效率比预期的低得多,达到点火并不容易。在20世纪80年代和90年代,为了理解高强度激光和等离子体的复杂相互作用,进行了许多实验。这些导致了新机器的设计,更大,最终达到点火能量。

最大的可操作ICF实验是美国的国家点火装置(NIF),它是利用早期实验数十年的经验设计的。然而,像那些早期的实验一样,NIF没有达到点火,并且截至2015年,正在产生大约1⁄3所需的能量水平。[1]

\subsection{描述}
\subsubsection{1.1 基本融合}
\begin{figure}[ht]
\centering
\includegraphics[width=6cm]{./figures/2941b793a05597d5.png}
\caption{间接驱动激光器ICF使用黑腔室其内表面从任一侧用激光束锥照射,以用平滑的高强度X射线在内部浸泡融合微胶囊。可以看到最高能量的X射线通过环空器泄漏,这里用橙色/红色表示。} \label{fig_GXYS_1}
\end{figure}
聚变反应结合了较轻的原子,例如氢一起形成更大的。通常这些反应发生在如此高的温度下离子ed,他们的电子被高温剥去;因此,核聚变通常被描述为“原子核”,而不是“原子”。

原子核带正电,因此由于静电力相互排斥。克服这种排斥需要大量的能量,这被称为库仑障壁或者聚变势垒能量。一般来说,导致较轻的原子核融合需要较少的能量,因为它们的电荷较少,因此势垒能量较低,当它们融合时,会释放更多的能量。随着原子核质量的增加,有一个点是反应不再释放净能量——克服能量屏障所需的能量大于最终聚变反应释放的能量。

从能源角度来看,最好的燃料是氘和氚的一对一混合;两者都是氢的重同位素。D-T(氘和氚)混合物的屏障很低,因为它的中子和质子比率很高。原子核中中性中子的存在有助于通过核力将它们拉在一起,而带正电荷的质子的存在通过静电力将原子核推开。氚是任何稳定或中度不稳定核素中中子与质子比率最高的核素之一——两个中子和一个质子。添加质子或移除中子会增加能量屏障。

在标准条件下的D-T混合物不进行融合;在核力将原子核拉在一起形成稳定的集合之前,必须将它们压在一起。即使在太阳炎热、密集的中心,平均质子在融合之前也会存在数十亿年。[2]对于实际的聚变发电系统,必须通过将燃料加热到数千万度,和/或压缩到巨大的压力来大幅提高燃料利用率。任何特定燃料熔化所需的温度和压力称为劳森准则。自20世纪50年代第一颗氢弹诞生以来,这些条件就已经为人所知。在地球上遇到劳森判据是极其困难的,这也解释了为什么聚变研究花了许多年才达到目前的高技术水平。[3]
\subsubsection{1.2 ICF作用机制}
在氢弹中,聚变燃料用单独的裂变炸弹压缩和加热(参见Teller-Ulam设计)中。多种机制将裂变“初级”爆炸的能量转移到聚变燃料中。一个主要的机制是,初级粒子发出的x射线闪光被捕获在炸弹的工程外壳内,导致外壳和炸弹之间的体积充满x射线“气体”。这些x射线均匀地照射到聚变区的外部,即“次级”,迅速加热它,直到它向外爆炸。这种向外的吹出导致次级线圈的其余部分向内压缩,直到达到聚变反应开始的温度和密度。

裂变炸弹的要求使得这种方法不适合发电。这种引爆器不仅生产成本过高,而且制造这种炸弹的最小尺寸也是可以的,大致由所使用的钚燃料的临界质量来定义。一般来说,建造产量小于约1千吨的核装置似乎很困难,而聚变二次堆会增加产量。这使得从爆炸中提取能量成为一个困难的工程问题; PACER 项目研究了工程问题的解决方案,但也证明了成本在经济上是不可行的。

PACER的参与者之一,约翰·努科尔斯(John nuckols)开始探索随着次级粒子的尺寸减小,启动聚变反应所需的初级粒子的尺寸发生了什么变化。他发现,当次级粒子达到毫克大小时,激发它所需的能量落入兆焦范围。这远远低于原子弹的需求,因为原子弹的主爆炸源在万亿焦耳射程内,相当于大约6盎司的TNT炸药。

这在经济上是不可行的,这种设备的成本将超过它所生产的电能的价值。然而,有许多其他设备可能能够重复传递这种能量水平。这导致了使用一种设备将能量“束”到聚变燃料上,确保机械分离的想法。到20世纪60年代中期,激光器似乎将发展到可以获得所需能量水平的程度。

通常ICF系统使用单个激光器驾驶员其光束被分成许多光束,这些光束随后被分别放大一万亿倍或更多。这些被许多镜子送入反应室(称为目标室),镜子的位置是为了均匀地照射整个表面上的目标。驾驶员施加的热量导致目标的外层爆炸,就像氢弹燃料缸的外层被裂变装置的x光照射时一样。

材料从表面爆炸导致内部剩余的材料被巨大的力向内驱动,最终坍缩成一个微小的近球形球。在现代惯性约束聚变装置中,产生的燃料混合物的密度高达铅密度的100倍,约为1000克/厘米3。这个密度不足以独自创造任何有用的聚变速率。然而,在燃料崩溃的过程中,冲击波也形成并以高速进入燃料的中心。当他们遇到从燃料中心的另一侧进入的对手时,那个点的密度会提高很多。

给定正确的条件,在被冲击波高度压缩的区域中的融合率可以释放出大量高能的α粒子α。由于周围燃料的高密度,它们在被“热化”之前只移动很短的距离,将能量作为热量损失给燃料。这种额外的能量将在加热的燃料中引起额外的聚变反应,释放出更多的高能粒子。这个过程从中心向外扩展,导致一种自我维持的烧伤,称为点火。
\begin{figure}[ht]
\centering
\includegraphics[width=14.25cm]{./figures/f5ecab750c2b7c82.png}
\caption{激光惯性约束聚变阶段示意图。蓝色箭头代表辐射;橙色被吹走了;紫色是向内传输的热能。 Laser beams or laser-produced X-rays rapidly heat the surface of the fusion target, forming a surrounding plasma envelope. Fuel is compressed by the rocket-like blowoff of the hot surface material. During the final part of the capsule implosion, the fuel core reaches 20 times the density of lead and ignites at 100,000,000 ˚C. Thermonuclear burn spreads rapidly through the compressed fuel, yielding many times the input energy.} \label{fig_GXYS_2}
\end{figure}
\subsubsection{1.3 成功的问题}
自20世纪70年代早期实验以来,提高惯性约束聚变性能的主要问题是向靶输送能量、控制内爆燃料的对称性、防止燃料过早加热(在达到最大密度之前)、通过流体动力学不稳定性防止热燃料和冷燃料过早混合以及在压缩燃料中心形成“紧密”冲击波会聚。

为了将冲击波聚焦在目标的中心,必须使目标具有极高的精度和球形,其表面(内部和外部)的像差不超过几微米。同样,激光束的瞄准必须非常精确,并且光束必须同时到达目标上的所有点。然而,光束定时是一个相对简单的问题,并且通过在光束的光路中使用延迟线来实现皮秒级的定时精度来解决。困扰内爆靶实现高对称性和高温度/密度的另一个主要问题是所谓的“束-束”不平衡和束各向异性。这些问题分别是,由一个光束传递的能量可能高于或低于撞击目标的其他光束,以及光束直径撞击目标内的“热点”,这导致目标表面上的不均匀压缩,从而形成瑞利-泰勒不稳定性[4]在燃料中过早混合,并在最大压缩时降低加热效率。由于冲击波的形成,里克特迈耶-梅什科夫不稳定性也在此过程中形成。
\begin{figure}[ht]
\centering
\includegraphics[width=10cm]{./figures/5d114c9228a96cb3.png}
\caption{惯性约束聚变靶是一种填充泡沫的圆柱形靶,带有机械扰动,被新星激光压缩。这张照片拍摄于1995年。形象秀目标的压缩,以及瑞利-泰勒不稳定性的增长。[4]} \label{fig_GXYS_3}
\end{figure}