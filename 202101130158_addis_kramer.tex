% 克拉默法则

\pentry{行列式\upref{Deter}}

\footnote{参考 Wikipedia \href{https://en.wikipedia.org/wiki/Cramer's_rule}{相关页面}.}\textbf{克拉默法则(Kramer's rule)}是一种直接用行列式解线性方程组的方法. 把线性方程组记为矩阵乘法\upref{Mat}的形式
\begin{equation}
\mat A \bvec x = \bvec b
\end{equation}
其中 $\mat A$ 为系数矩阵. 当 $\mat A$ 为方阵且行列式 $\det{\mat A} \ne 0$ 时, 方程有唯一解(见 “线性方程组解的结构\upref{LinEq}”). 该解可以用克拉默法则直接写出:
\begin{equation}
x_i = \frac{\det{\mat A_i}}{\det{\mat A}}
\end{equation}
其中 $\mat A_i$ 是把 $\mat A$ 的第 $i$ 列替换为 $\bvec b$ 而来.
