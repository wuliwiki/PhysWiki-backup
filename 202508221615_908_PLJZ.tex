% 泡利矩阵(综述)
% license CCBYSA3
% type Sum

本文根据 CC-BY-SA 协议转载翻译自维基百科\href{https://en.wikipedia.org/wiki/Pauli_matrices}{相关文章}。

\begin{figure}[ht]
\centering
\includegraphics[width=6cm]{./figures/943b6a85e9584185.png}
\caption{沃尔夫冈·泡利(1900–1958),约摄于 1924 年。泡利因提出泡利不相容原理而在 1945 年获颁诺贝尔物理学奖**,提名人是阿尔伯特·爱因斯坦。} \label{fig_PLJZ_1}
\end{figure}
在数学物理和数学中,泡利矩阵是一组三个$2\times2$ 的复矩阵,它们具有迹为零、厄米特、自反和酉的性质。这些矩阵通常用希腊字母$\sigma$(σ)表示,在涉及同位旋对称性时,有时也用$\tau$(τ)表示。
$$
\sigma_1 = \sigma_x = 
\begin{pmatrix}
0 & 1 \\
1 & 0
\end{pmatrix},
\quad
\sigma_2 = \sigma_y = 
\begin{pmatrix}
0 & -i \\
i & 0
\end{pmatrix},
\quad
\sigma_3 = \sigma_z = 
\begin{pmatrix}
1 & 0 \\
0 & -1
\end{pmatrix}.~
$$
这些矩阵因物理学家 沃尔夫冈·泡利而得名。在量子力学中,它们出现在泡利方程中,用于描述粒子的自旋与外部电磁场相互作用的情况。它们还可以用来表示两种偏振滤光片的相互作用状态,例如水平/垂直偏振、45 度偏振(左/右)以及圆偏振(左/右)的状态。

每个泡利矩阵都是厄米特矩阵。再加上单位矩阵 $I$(有时被视为第零个泡利矩阵$\sigma_0$),这些矩阵在实数加法下构成了所有 $2 \times 2$ 厄米特矩阵的向量空间的基。这意味着,任何 $2 \times 2$ 厄米特矩阵都可以唯一地写成泡利矩阵的实数线性组合。

泡利矩阵满足以下有用的乘积关系:
$$
\sigma_i \sigma_j = \delta_{ij} + i \, \epsilon_{ijk} \, \sigma_k,~
$$
其中 $\delta_{ij}$ 是克罗内克 δ,$\epsilon_{ijk}$ 是三阶 Levi-Civita 符号。由于厄米特算符在量子力学中表示可观测量,因此泡利矩阵张成了复二维希尔伯特空间中所有可观测量的空间。在泡利的研究语境中,$\sigma_k$ 表示三维欧几里得空间 $\mathbb{R}^3$ 中沿第 $k$ 个坐标轴方向的自旋对应的可观测量。

此外,当泡利矩阵乘以 $i$ 使之成为反厄米特矩阵后,它们还可以生成李代数意义下的变换:矩阵 $i\sigma_1, i\sigma_2, i\sigma_3$ 构成了实李代数 $\mathfrak{su}(2)$ 的基,并通过指数映射生成特殊酉群 $SU(2)$。由三个位移矩阵 $\sigma_1, \sigma_2, \sigma_3$ 所生成的代数同构于 $\mathbb{R}^3$ 的克利福德代数;而由 $i\sigma_1, i\sigma_2, i\sigma_3$ 生成的带幺结合代数则与四元数代数$\mathbb{H}$ 完全同构。
\subsection{代数性质}
所有三个位相矩阵都可以压缩成一个表达式:
$$
\sigma_j =
\begin{pmatrix}
\delta_{j3} & \delta_{j1} - i\delta_{j2} \\
\delta_{j1} + i\delta_{j2} & -\delta_{j3}
\end{pmatrix},~
$$
其中,$\delta_{jk}$ 是克罗内克δ函数,当 $j = k$ 时取值为 $+1$,否则为 $0$。这个表达式的好处在于,可以通过代入 $j = 1, 2, 3$ 来“选择”三个位相矩阵中的任意一个,这在需要进行代数推导但不针对某个具体矩阵时非常实用。

这些矩阵具有自反性:
$$
\sigma_1^2 = \sigma_2^2 = \sigma_3^2 = -i\,\sigma_1\sigma_2\sigma_3 =
\begin{pmatrix}
1 & 0 \\
0 & 1
\end{pmatrix}
= I,~
$$
其中 $I$ 是单位矩阵。

位相矩阵的行列式和迹为:
$$
\det(\sigma_j) = -1, \quad
\operatorname{tr}(\sigma_j) = 0,~
$$
由此可知,每个矩阵 $\sigma_j$ 的特征值为 $+1$ 和 $-1$。

当将单位矩阵 $I$(有时记作 $\sigma_0$)包括在内时,位相矩阵与 $I$ 一起构成:实数域 $\mathbb{R}$ 上 $2 \times 2$ 厄米矩阵空间 $\mathcal{H}_2$的一个 Hilbert–Schmidt 正交基;复数域 $\mathbb{C}$ 上所有 $2 \times 2$ 矩阵空间 $\mathcal{M}_{2,2}(\mathbb{C})$的一个正交基。
\subsubsection{对易与反对易关系}
\textbf{对易关系}

保利矩阵满足如下对易关系:
$$
[\sigma_j, \sigma_k] = 2i \, \varepsilon_{jkl} \, \sigma_l,~
$$
其中 $\varepsilon_{jkl}$ 是Levi-Civita 符号。


这些对易关系表明,保利矩阵是以下李代数表示的生成元:
$$
(\mathbb{R}^3, \times) \;\cong\; \mathfrak{su}(2) \;\cong\; \mathfrak{so}(3),~
$$
即它们与三维实数空间 $\mathbb{R}^3$ 的叉积代数、$\mathfrak{su}(2)$ 李代数和 $\mathfrak{so}(3)$ 李代数互相同构。

\textbf{反对易关系}

保利矩阵还满足以下反对易关系:
$$
\{\sigma_j, \sigma_k\} = 2 \delta_{jk} I,~
$$
其中,$\{\sigma_j, \sigma_k\}$ 定义为:$\sigma_j \sigma_k + \sigma_k \sigma_j$,$\delta_{jk}$ 是 Kronecker δ 符号,$I$ 表示$2 \times 2$ 单位矩阵。


这些反对易关系说明,保利矩阵是三维实空间$\mathbb{R}^3$Clifford 代数$\mathrm{Cl}_3(\mathbb{R})$ 的一个表示的生成元。

利用 Clifford 代数的常规构造方式,可以定义 $\mathfrak{so}(3)$ 李代数的生成元:$\sigma_{jk} = \tfrac{1}{4}[\sigma_j, \sigma_k]$,从而重新得到上文的对易关系(仅相差一个数值因子)。

下面给出一些显式的对易子和反对易子的示例:
\begin{figure}[ht]
\centering
\includegraphics[width=10cm]{./figures/5eeeab185b454847.png}
\caption{} \label{fig_PLJZ_2}
\end{figure}
\subsubsection{特征向量与特征值}
每一个(厄米的)泡利矩阵都有两个特征值:$+1$ 和 $-1$。相应的归一化特征向量为:
$$
\psi_{x+} = \frac{1}{\sqrt{2}}
\begin{bmatrix}
1\\
1
\end{bmatrix},
\quad
\psi_{x-} = \frac{1}{\sqrt{2}}
\begin{bmatrix}
1\\
-1
\end{bmatrix},~
$$
$$
\psi_{y+} = \frac{1}{\sqrt{2}}
\begin{bmatrix}
1\\
i
\end{bmatrix},
\quad
\psi_{y-} = \frac{1}{\sqrt{2}}
\begin{bmatrix}
1\\
-i
\end{bmatrix},~
$$
$$
\psi_{z+} =
\begin{bmatrix}
1\\
0
\end{bmatrix},
\quad
\psi_{z-} =
\begin{bmatrix}
0\\
1
\end{bmatrix}.~
$$
\subsection{泡利向量}
泡利向量定义为\(^\text{[b]}\):
$$
\vec{\sigma} = \sigma_1 \hat{x}_1 + \sigma_2 \hat{x}_2 + \sigma_3 \hat{x}_3,~
$$
其中 $\hat{x}_1$、$\hat{x}_2$、$\hat{x}_3$ 分别是更常见的 $\hat{x}$、$\hat{y}$、$\hat{z}$ 的等价记法。

泡利向量提供了一种将三维向量基底映射到泡利矩阵基底的机制\(^\text{[2]}\),如下所示:
$$
\vec{a} \cdot \vec{\sigma} 
= \sum_{k, l} a_k \sigma_\ell (\hat{x}_k \cdot \hat{x}_\ell) 
= \sum_k a_k \sigma_k
= 
\begin{pmatrix}
a_3 & a_1 - i a_2 \\
a_1 + i a_2 & -a_3
\end{pmatrix}.~
$$
更正式地说,该映射定义了从 $\mathbb{R}^3$ 到迹为零的厄米 $2\times 2$ 矩阵向量空间的映射。该映射通过矩阵函数编码了 $\mathbb{R}^3$ 作为赋范向量空间和李代数(其中向量叉乘作为李括号)的结构,使得该映射成为一个李代数同构。
从表示论的角度看,泡利矩阵因此可视为交织算子。

另一种视角下,泡利向量也可以看作是一个值为 $2\times 2$ 厄米迹为零矩阵的对偶向量,即:$\text{Mat}_{2\times 2}(\mathbb{C}) \otimes (\mathbb{R}^3)^*$,
它将向量 $\vec{a}$ 映射为:$\vec{a} \mapsto \vec{a} \cdot \vec{\sigma}$.
\subsubsection{完备性关系}
向量 $\vec{a}$ 的每一个分量都可以从矩阵中恢复出来(见下方的完备性关系):
$$
\frac{1}{2} \operatorname{tr} \bigl( (\vec{a} \cdot \vec{\sigma}) \vec{\sigma} \bigr) = \vec{a}.~
$$
这实际上是映射$\vec{a} \;\mapsto\; \vec{a} \cdot \vec{\sigma}$的逆映射,从而明确表明该映射是一个双射。
\subsubsection{行列式}
范数可以通过行列式(差一个负号)给出:
$$
\det(\vec{a} \cdot \vec{\sigma}) = -\vec{a} \cdot \vec{a} = -|\vec{a}|^2.~
$$
接着,考虑一个$SU(2)$矩阵 $U$ 在这类矩阵空间上的共轭作用:
$$
U * (\vec{a} \cdot \vec{\sigma}) := U\,(\vec{a} \cdot \vec{\sigma})\,U^{-1},~
$$
我们可以得到:$\det(U * (\vec{a} \cdot \vec{\sigma})) = \det(\vec{a} \cdot \vec{\sigma})$,并且 $U * (\vec{a} \cdot \vec{\sigma})$ 仍然是厄米且无迹的。因此我们可以定义:$U * (\vec{a} \cdot \vec{\sigma}) = \vec{a}' \cdot \vec{\sigma}$,其中,$\vec{a}'$ 与 $\vec{a}$ 具有相同的范数,于是我们可以把 $U$ 解释为三维空间中的一次旋转。事实上,矩阵 $U$ 的特殊限制还意味着该旋转是保持方向的。这就允许我们定义一个映射:$R: \mathrm{SU}(2) \;\longrightarrow\; \mathrm{SO}(3)$其具体形式为:
$$
U * (\vec{a} \cdot \vec{\sigma}) = \vec{a}' \cdot \vec{\sigma} := (R(U)\,\vec{a}) \cdot \vec{\sigma},~
$$
其中 $R(U) \in \mathrm{SO}(3)$。这个映射就是 $\mathrm{SU}(2)$ 对 $\mathrm{SO}(3)$ 的双覆盖的具体实现,因此说明了:$\mathrm{SU}(2) \;\cong\; \mathrm{Spin}(3)$.此外,$R(U)$ 的各个分量可以通过如下迹运算恢复出来:
$$
R(U)_{ij} = \frac{1}{2} \operatorname{tr}(\sigma_i U \sigma_j U^{-1})~
$$
\subsubsection{叉积}
叉积可以由矩阵对易子(差一个 $2i$ 的因子)表示:
$$
[\vec{a} \cdot \vec{\sigma},\; \vec{b} \cdot \vec{\sigma}] = 2i\,(\vec{a} \times \vec{b}) \cdot \vec{\sigma}.~
$$
实际上,范数的存在源于 $\mathbb{R}^3$ 本身是一个李代数(参见 Killing 型)。

这种叉积还可以用于证明上述映射保持空间方向的性质。
\subsubsection{特征值与特征向量}
矩阵$\vec{a} \cdot \vec{\sigma}$的特征值为$\pm |\vec{a}|$.这可以直接从矩阵的无迹性以及显式计算行列式得出。

从更抽象的角度来看,即便不计算行列式(而计算行列式需要用到泡利矩阵的具体性质),也能通过如下关系推导出该结果:$(\vec{a} \cdot \vec{\sigma})^2 - |\vec{a}|^2 = 0$,因为该式可以因式分解为:$(\vec{a} \cdot \vec{\sigma} - |\vec{a}|)(\vec{a} \cdot \vec{\sigma} + |\vec{a}|) = 0$.线性代数中的一个标准结论是:如果一个线性映射满足由若干不相同一次因式组成的多项式方程,那么该映射是可对角化的。因此,这意味着$\vec{a} \cdot \vec{\sigma}$是可对角化的,其特征值为$\pm |\vec{a}|$.此外,由于$\vec{a} \cdot \vec{\sigma}$是无迹矩阵,它必然有且只有一个正特征值 $+|\vec{a}|$ 和一个负特征值 $-|\vec{a}|$。

它的归一化特征向量是:
$$
\psi_{+} = 
\frac{1}{\sqrt{2\left|{\vec{a}}\right|\left(a_3+\left|{\vec{a}}\right|\right)}} 
\begin{bmatrix}
a_3 + \left|{\vec{a}}\right| \\
a_1 + i a_2
\end{bmatrix},
\qquad
\psi_{-} =
\frac{1}{\sqrt{2\left|{\vec{a}}\right|\left(a_3+\left|{\vec{a}}\right|\right)}} 
\begin{bmatrix}
i a_2 - a_1 \\
a_3 + \left|{\vec{a}}\right|
\end{bmatrix}.~
$$
这些表达式在 $a_3 \to -|{\vec{a}}|$ 时会出现奇异性。可以通过令${\vec{a}} = |{\vec{a}}|(\epsilon, 0, -(1-\epsilon^2/2))$并取极限 $\epsilon \to 0$ 来修正,这样可以得到 $\sigma_z$ 的正确特征向量 $(0,1)$ 和 $(1,0)$。

另一种方法是使用球坐标:${\vec{a}} = a(\sin\vartheta\cos\varphi, \sin\vartheta\sin\varphi, \cos\vartheta)$,这样可以得到特征向量:
$$
\psi_+ = \big(\cos(\vartheta/2), \ \sin(\vartheta/2)\exp(i\varphi)\big),~
$$
以及
$$
\psi_- = \big(-\sin(\vartheta/2)\exp(-i\varphi), \ \cos(\vartheta/2)\big)~
$$
\subsubsection{泡利四矢量}
在旋量理论中使用的泡利四矢量写作$\sigma^\mu$其分量为
$$
\sigma^\mu = (I, {\vec{\sigma}}).~
$$
这定义了一个从 $\mathbb{R}^{1,3}$ 到厄米矩阵向量空间的映射:
$$
x_\mu \;\mapsto\; x_\mu \sigma^\mu ,~
$$
该映射的行列式也编码了闵可夫斯基度规(采用“多数负号”的约定):
$$
\det(x_\mu \sigma^\mu) = \eta(x, x) .~
$$
该四矢量还满足一个完备性关系。为此,方便起见,引入第二个泡利四矢量:
$$
\bar{\sigma}^\mu = (I, -{\vec{\sigma}}) .~
$$
并允许使用闵可夫斯基度规张量进行指标升降。这样,关系可以写为:
$$
x_\nu = \frac{1}{2} \operatorname{tr} \left( \bar{\sigma}_\nu \, (x_\mu \sigma^\mu) \right).~
$$
与泡利三矢量的情形类似,我们也可以找到一个在$\mathbb{R}^{1,3}$上作为等距变换**作用的矩阵群。在这种情况下,该矩阵群是:$\mathrm{SL}(2, \mathbb{C})$,因此我们有:$\mathrm{SL}(2, \mathbb{C}) \;\cong\; \mathrm{Spin}(1,3)$.与前面的推导类似,对于$S \in \mathrm{SL}(2, \mathbb{C})$可以显式写出对应洛伦兹变换矩阵的分量为:
$$
\Lambda(S)^{\mu}{}_{\nu} = 
\frac{1}{2} 
\operatorname{tr}\!\left( \bar{\sigma}_\nu \, S \, \sigma^\mu \, S^\dagger \right).~
$$
实际上,行列式的这个性质可以从$\sigma^\mu$的迹运算特性抽象地推出。对于 $2\times 2$矩阵,成立如下恒等式:
$$
\det(A+B) = \det(A) + \det(B) + \operatorname{tr}(A)\operatorname{tr}(B) - \operatorname{tr}(AB).~
$$
也就是说,“交叉项”可以通过迹来表示。当$A, B$分别取不同的$\sigma^\mu$时,这些交叉项会消失。于是可以得到(显式写出求和):$\det\!\left( \sum_{\mu} x_\mu \sigma^\mu \right)= \sum_{\mu} \det\!\left( x_\mu \sigma^\mu \right).$由于这些矩阵都是$2\times 2$阶矩阵,所以该式进一步化简为:$\sum_{\mu} x_\mu^2 \det(\sigma^\mu)=\eta(x, x)$,即行列式自然反映了闵可夫斯基度规 $\eta(x,x)$ 的结构。
\subsubsection{与点积和叉积的关系}
泡利矢量优雅地将对易关系和反对易关系映射为相应的向量运算。将对易子与反对易子相加,可得:
$$
\begin{aligned}
[\sigma_j, \sigma_k] + \{\sigma_j, \sigma_k\}
&= (\sigma_j \sigma_k - \sigma_k \sigma_j) + (\sigma_j \sigma_k + \sigma_k \sigma_j) \\
2 i \, \varepsilon_{jk\ell} \, \sigma_\ell + 2 \delta_{jk} I
&= 2 \sigma_j \sigma_k
\end{aligned}~
$$
因此:
$$
\sigma_j \sigma_k = \delta_{jk} I + i \varepsilon_{jk\ell} \, \sigma_\ell \;.~
$$

