% 视界
% license Usr
% type Tutor

\begin{issues}
\issueTODO
部分词条需要引用该词条。
\end{issues}

\subsection{可观测宇宙}
宇宙伊始至今,光在这漫漫长途中走过的共动距离为:
\begin{equation}
\int_{0}^{r_{0}} \frac{d r}{\sqrt{1-k (r/R)^{2}}}=\int_{0}^{t_{0}} \frac{c d t}{a(t)}~.
\end{equation}
在把尺度因子归一化后,物理距离便等同于共动距离。这是我们原则上能探测的宇宙大小,故称为\textbf{可观测宇宙(observarable universe)}。因为光速最快,所以大于该距离的粒子并不能影响到此时的我们,因此这个距离又被叫作\textbf{粒子视界(particle horizon)}。不过实际上,光子在退耦前与其他带电粒子作用频繁,整个宇宙都是不透光的,我们能测得的
