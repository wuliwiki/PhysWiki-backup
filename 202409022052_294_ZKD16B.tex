% 中国科学技术大学 2016 年考研普通物理 B 考试试题
% keys 中国科学技术大学|考研|物理|2016年
% license Copy
% type Tutor

\textbf{声明}:“该内容来源于网络公开资料,不保证真实性,如有侵权请联系管理员”
\begin{enumerate}
\item 长为、质量为M的均质木板,可绕穿过一端的光滑水平轴0转动,开始时用细线拉着木板的另一端使其静止于水平位置,然后剪断细线。
(1)试求细线刚剪断时作用于轴上的力;(2)试求当木板通过竖直位置时作用于轴上的力。
2.(20分)在光滑水平地板上滑动的弹性立方体撞到竖直墙上,撞击时立方体的一个面与墙平行。墙与立方体之间的摩擦系数为μ。碰撞前立方体速度方向与墙的夹角为a,问碰撞后夹角变为多大?
一个粒子在势能V(r)=αr"+βr"所确定的中心力作用下作平面运
动,a、B、p以及q均为常数,且p<q。已知该粒子可沿着螺旋线r=c0’所确定的轨道运动,其中c为常数。试确定常数p和q的数值。
4.(15 分)两个相同细金属圆环同轴放置,带电量分别为+0和-Q(O>0)。设圆环半径为a,两圆心A和B相距2d,以轴线为x轴,轴线在两圆环之间的中点 0为原点,
(1)求轴线上任一点的电势 U(x),并定性画出 U~x曲线
求轴线上任一点的电场E(x),并定性画出 E~x曲线;
(20分)极板尺寸相同的两个平板空气电容器充以相同的电量0。第一个电容器两极板的间距是第二个电容器的两倍。设第一个电容器的电容值为 Co,
(1)求两个电容器各自的静电能;
(2)如果将第二个电容器插入第一个电容器两极板的中央,所有极板保持平行,求整个体系的静电能。
(20分)孤立导体球半径为a,充电到电势,球绕其一直径以角速度@旋转。求
(1)球心的磁场强度;
(2)旋转球的磁矩。
(15分)求氢原子中:(1)电子在n=1轨道上运动时相应的电流值大小;(2)n=1和100时轨道中心处的磁场强度:(3)n=1和100时电子分别感受到的原子核的电场大小。已知
碳原子 2p3s'P→2p2p'D,的跃迁发出波长约为2000埃的谱线,若在弱的外磁场中,该谱线将如何分裂?设外磁场B=0.100T,试求在垂直于磁场方向观察各谱线与原谱线的波长差是多少?
9.(13 分)分别写出单电子2s、2p、3d、4f的原子态符号、总角动量大小和可能的5.Z取值
\end{enumerate}