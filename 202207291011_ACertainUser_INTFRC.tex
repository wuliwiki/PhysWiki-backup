% 刚体的内力
\pentry{简单刚体系统的静力学分析\upref{RGDFA}}
在材料力学中,我们关心一个刚体内部的受力情况,即刚体的内力.本文简要探讨平面杆件某截面处内力的分析方法.

\begin{example}{强度理论}
强度理论认为,材料失效的原因是由于材料内部某处的内力超过了材料所能承受的极限.因此,分析材料内力对于设计或选用可靠的材料起重要作用.
\end{example}

\subsection{截面法 Section Method}
分析材料内力的常用方法是“截面法”,该方法非常的通俗易懂.

0. 先确定作用在该刚体上的外力\upref{RGDFA}.

1. 在需要计算内力的截面处,假想切割刚体,将其“一分为二”.
\begin{figure}[ht]
\centering
\includegraphics[width=8cm]{./figures/INTFRC_1.png}
\caption{假想切割刚体} \label{INTFRC_fig1}
\end{figure}

2. 选取左半段(或右半段)刚体.此时刚体截面处的内力变成了“外力”.类似于钉子模型\upref{RGDFA},材料内力的效果是一个任意方向的力与一个力偶.同样的,此时还不能确定力的方向,先标注为两个互相垂直的力
\begin{figure}[ht]
\centering
\includegraphics[width=8cm]{./figures/INTFRC_2.png}
\caption{画出截面处的内力} \label{INTFRC_fig2}
\end{figure}

3. 根据力的平衡条件\upref{RGDFA},即可解出内力.
