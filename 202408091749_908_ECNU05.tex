% 华东师范大学 2005 年 考研 量子力学
% license Usr
% type Note

\textbf{声明}:“该内容来源于网络公开资料,不保证真实性,如有侵权请联系管理员”

\subsection{一、判断题(判断以下说法的正误,分别用“十”和“-”标示出来。本题共有6小题,每小题2分,满分12分)}
\begin{enumerate}
\item 光电效应实验主要体现了光场的粒子性。
\item 在粒子双缝衍射实验中,能够在不破坏粒子本身状态的情况下,确定粒子从哪个缝穿过。
\item 两个厄密算符的乘积也必然是厄密算符。
\item 海森堡不确定关系表明:不管将来的测量技术如何改进,同时测量一个微观粒子的坐标和动量是不可能的。
\item 设$\Omega$为对应力学量只的算符,则对量子态$\psi(\vec x)$进行关于力学量$\Omega$的测量,
所得$\Omega$的实测值必是算符$\hat\Omega$的本征值之一
\item 根据全同粒子交换对称性可知,全同粒子系统的波函数必须是全对称的。
\end{enumerate}
