% 电偶极子辐射

\begin{issues}
\issueDraft
\end{issues}

\pentry{电磁场推迟势\upref{RetPt0}}

\footnote{参考 \cite{GriffE}.}令原点处的电偶极子为
\begin{equation}
\bvec p(t) = p_0 \cos(\omega t) \uvec z
\end{equation}
使用洛伦兹规范, 在 $r \gg \lambda$ 的近似下
\begin{equation}
\varphi(r, \theta, t) = -\frac{p_0\omega}{4\pi\epsilon_0 c} \qty(\frac{\cos\theta}{r}) \sin[\omega(t - r/c)]
\end{equation}
\begin{equation}
\bvec A(r, \theta, t) = -\frac{\mu_0 p_0 \omega}{4\pi r} \sin[\omega(t - r/c)]\uvec z
\end{equation}
进而得
\begin{equation}
\bvec E = -\frac{\mu_0 p_0\omega^2}{4\pi} \qty(\frac{\sin\theta}{r})\cos[\omega(t - r/c)]\uvec \theta
\end{equation}
\begin{equation}
\bvec B = -\frac{\mu_0 p_0\omega^2}{4\pi c} \qty(\frac{\sin\theta}{r})\cos[\omega(t - r/c)]\uvec \phi
\end{equation}
其中用到了近似
\begin{equation}
d \ll \frac{\lambda}{2\pi} \ll r
\end{equation}

\subsubsection{辐射功率}
\begin{equation}
\bvec s(\bvec r, t) = \frac{1}{\mu_0} \bvec E \cross \bvec B = \frac{\mu_0p_0^2\omega^4}{16\pi^2c} \frac{\sin^2\theta}{r^2} \cos^2[\omega(t - r/c)] \uvec r
\end{equation}
时间平均值为
\begin{equation}
\ev{\bvec s(\bvec r)} = \frac{1}{\mu_0} \bvec E \cross \bvec B
= \frac{\mu_0p_0^2\omega^4}{32\pi^2c} \frac{\sin^2\theta}{r^2} \uvec r
\end{equation}
总平均功率为
\begin{equation}
\ev{P} = \oint \bvec s \vdot \dd{\bvec a} = \frac{\mu_0 p_0^2 \omega^4}{12\pi c}
\end{equation}
