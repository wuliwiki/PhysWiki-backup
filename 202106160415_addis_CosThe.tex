% 余弦定理

\begin{figure}[ht]
\centering
\includegraphics[width=4cm]{./figures/CosThe_1.pdf}
\caption{余弦定理示例三角形} \label{CosThe_fig1}
\end{figure}

\footnote{参考 Wikipedia \href{https://en.wikipedia.org/wiki/Law_of_cosines}{相关页面}.}\textbf{余弦定理(law of cosines)}是指:三角形(\autoref{CosThe_fig1} )任何一条边的平方等于其它两边平方的和减去这两边与它们夹角的余弦的积的两倍.
\begin{equation}\label{CosThe_eq1}
c^2=a^2 + b^2 - 2ab\cos C
\end{equation}


%\addTODO{图: 画个平行四边形, $\bvec c = \bvec a + \bvec b, \bvec d = \bvec a - \bvec b$}
\subsection{推导}
\begin{figure}[ht]
\centering
\includegraphics[width=5cm]{./figures/CosThe_2.pdf}
\caption{余弦定理的证明} \label{CosThe_fig2}
\end{figure}
我们在直角坐标系中来推导,如\autoref{CosThe_fig2} ,以 $C$点为原点,$B$ 点在 $x$ 轴正方向上,则各点坐标为:
\begin{equation}
C(0,0),\quad B(a,0),\quad A(b\cos C,b\sin C)
\end{equation}
那么由两点之间的距离公式,有
\begin{equation}
c^2=\overline{AB}^2=(b\cos C-a)^2+(b\cos C-0)^2=a^2+b^2-2ab\cos C
\end{equation}

\subsection{推导 2}
\pentry{几何矢量的内积\upref{Dot}}
该推导不依赖坐标. 利用几何矢量点乘的分配律, 以及点乘的几何定义 $\bvec a \vdot \bvec b = ab\cos\theta$:
\begin{equation}
\begin{aligned}
(\bvec a \pm \bvec b)^2 &= \bvec a^2 + \bvec b^2 \pm 2\bvec a \vdot \bvec b\\
&= a^2 + b^2 \pm 2ab \cos\theta
\end{aligned}
\end{equation}
