% Matlab 的判断与循环
% Matlab|循环|数值计算

\pentry{Matlab 的变量与矩阵\upref{MatVar}}

\subsection{脚本文件}
在讲解更复杂的程序结构前,我们先来看脚本文件.\textbf{脚本(script)文件} 是包含若干个指令的文件,文件后缀名为 “.m”.脚本文件可以单独执行,也在其他文件或 Command Window 中被调用(注意需要将所在文件夹添加到搜索路径). 后者相当于把被调用脚本的代码直接插入到调用指令处,调用指令就是脚本文件的文件名.脚本中的每条命令后面应该加分号以隐藏输出结果,若需要输出,用 \x{disp} 函数.
\begin{lstlisting}[language=MatlabCom]
>> disp('good'); a = 3; disp(['a = ', num2str(a)])
good
a = 3
\end{lstlisting}
在脚本文件中,可以在行首或命令后用百分号 \x{\%} 进行\textbf{注释(comment)}\footnote{截止到 Matlab 2017b, 在英文版 Matlab IDE 中, 任何中文注释都会在 Matlab 重启后变为乱码. 若要使用中文注释, 建议使用中文操作系统和中文 Matlab.}.注释是程序的说明,使程序更易读,但在执行程序时会被忽略(\autoref{Matlab_fig1}).

\subsection{判断结构}
现在来看一段代码(脚本文件). 要生成新的脚本文件, 可以在 Editor 菜单(\autoref{Matlab_fig1}\upref{Matlab}) 的左边单击 New, 然后选 Script, 或者用快捷键 Ctrl+N. 在生成的 Editor 中输入以下代码

\begin{lstlisting}[language=MyMatlab]
a = rand(1,1); b = 0.5;
if a > b % a 较大
    disp('a is larger');
else % b 较大
    disp('b is larger');
end
\end{lstlisting}

这段程序用 \x{rand} 函数随机生成一个从 \x{0} 到 \x{1} 的数,如果随机数大于 \x{0.5} 则输出第一段文字,否则输出第二段文字. 不难猜测出这里的 \x{if} 用于判断,如果条件满足,则只执行 \x{if} 和 \x{else} 之间的指令.如果条件不满足,则只执行 \x{else} 到 \x{end} 的指令. 注意 \x{else} 语句可以不在判断结构中出现, 若不出现, 当判断条件不满足时程序将直接执行 \x{end} 后面的代码.

要执行该代码, 在 Editor 菜单中单击 Run 图标(绿色三角形), 如果代码没有保存, Matlab 会先弹出保存对话框. 再次强调文件必须保存在 Matlab 的搜索路径下.

\x{elseif} 语句可用于在判断结构中产生多个分支, 如

\begin{lstlisting}[language=MyMatlab]
a = rand(1,1);
if a > 0.9
    disp('a in (0.9, 1]');
elseif a > 0.6
    disp('a in (0.6, 0.9]');
elseif a > 0.3
    disp('a in (0.3, 0.6]');
else
    disp('a in [0, 0.3]');
end
\end{lstlisting}

这个程序用于判断随机数 \x{a} 的区间. 若 \x{if} 的条件判断成功, 判断结构就只执行 \x{if} 到第一个 \x{elseif} 之间的命令. 若 \x{if} 判断失败, 程序就继续判断第一个 \x{elseif} 中的条件, 若判断成功, 就只执行第一个 \x{elseif} 到第二个 \x{elseif} 之间的命令, 以此类推. 如果 \x{if} 和所有的 \x{elseif} 条件都判断失败, 则执行 \x{else} 后面的命令.

\subsection{循环结构}
我们先来看 \x{for} 循环

\begin{lstlisting}[language=MyMatlab]
for ii = 1:3
    disp(['ii^2 = ' num2str(ii^2)]);
end
\end{lstlisting}

运行结果为
\begin{lstlisting}[language=MatlabCom]
ii^2 = 1
ii^2 = 4
ii^2 = 9
\end{lstlisting}
容易看出这段代码被执行了 3 次,\textbf{循环变量} \x{ii} 按顺序取 \x{1:3} 中的一个矩阵元.注意选取 \x{ii} 作为变量名是为了与虚数单位区分,当然也可以选择其他变量名.再来看一个稍复杂的循环

\begin{lstlisting}[language=MyMatlab]
Nx = 5;
x = zeros(1,Nx);  % 预赋值
x(1) = 2;
for ii = 2:numel(x)
    x(ii) = x(ii-1)^2;
end
disp(['x = ' num2str(x)])
\end{lstlisting}

在循环开始前 \x{x(1)} 被赋值为 2,在循环中,第 \x{ii} 个矩阵元 依次被赋值为第 \x{ii-1} 个矩阵元的平方.运行结果为
\begin{lstlisting}[language=MatlabCom]
x = 2  4  16  256  65536
\end{lstlisting}
注意在循环前用 \x{zero} 对矩阵进行了\textbf{预赋值(preallocation)}.预赋值不是必须的,但如果不进行预赋值,每次循环矩阵的尺寸都要改变,会导致程序运行变慢.另外注意循环中不允许给循环变量赋值.

再来看另一种循环叫做 \x{while} 循环. 下面来看一个例程, 输出 100 以内的斐波那契数列($a_1 = 1, a_2 = 1, a_{n+1} = a_{n} + a_{n-1}$).

\Code{fibonacci}

\x{while} 结构在每个循环开始会判断 \x{while} 后面的条件, 如果条件成立, 则进行一次循环, 否则退出循环. 以上的程序中由于我们事先并不知道我们要进行几次循环, 所以选用 \x{while}, 当最后一项大于 100 时, 循环终止. 运行结果为(每个数占一行):
\begin{lstlisting}[language=MatlabCom]
1  1  2  3  5  8  13  21  34  55  89
\end{lstlisting}

在 \x{for} 循环或 \x{while} 循环的内部, 使用 \x{continue} 命令可以直接进入下一个循环(\x{while} 的仍然要先判断条件), 使用 \x{break} 命令可以跳出循环. 以下例程计算 100 以内的斐波那契数列的所有奇数项

\Code{fibonacciOdd}

先来看第 4 行, \x{double} 类型的非零数在这里会自动转换为 \x{logical} 类型的 1 (\x{true}), 只有 \x{double} 类型的 0 才会转换为 \x{logical} 类型的 0 (\x{false}). 乍看之下, \x{while} 循环将永远执行下去(称为\textbf{死循环}), 然而第 9 行的 \x{break} 在 \x{a3 > 100} 时就会使程序跳出循环. 如果 \x{a3 <= 100} 且为偶数, 则第 10 行的 \x{elseif} 判断为真, \x{continue} 命令被执行, 程序将直接跳过之后的 \x{disp} 函数直接进入下一个循环, 所以数列的偶数项都不会被输出. 程序的运行结果为(每个数占一行)
\begin{lstlisting}[language=MatlabCom]
1  1  3  5  13  21  55  89
\end{lstlisting}

\subsection{return 命令}
在一个脚本文件的任何地方, 如果 \x{return} 命令被执行, 则程序将结束该脚本文件的执行. 如果该脚本文件是被单独执行的, 程序将终止. 如果该脚本文件是被其他脚本文件或函数文件调用的, 程序将继续执行调用命令的下一个命令.



