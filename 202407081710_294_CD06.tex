% 四川大学 2006 年硕士物理考试试题
% keys 四川大学|考研|物理|2006年
% license Copy
% type Tutor
\subsection{简答题}
\begin{enumerate}
\item 将一个带电$+q$、半径为$R$的大导体球$B$移近一个半径为$r$而不带电的小导体球$A$,试判断下列说法是否正确?并说明理由。\\
(1)$B$球电势高于$A$球。\\
(2)以无限远为电势零点,$A$球的电势小于0。\\
(3)在$B$球表面附近任一点的场强等于$\displaystyle \frac{q}{4\pi R^2}$
\item 指出下列有关电场强度与电势$U$的关系的说法是否正确?并简要说明原因。\\
(1)已知某点的就可以确定该点的$U$。\\
(2)已知某点的$U$就可以确定该点的$\vec E$。\\
(3)$\vec E$不变的空间,$U$也一定不变。\\
(4)$U$值相等的曲面上,$\vec E$值不一定相等。
\item 一根通有 $20A $电流的无限长细直导线,放在磁感应强度为$B=10^{-3}T$的均匀外磁场中,导线与外磁场正交。试确定磁感应强度为零的各点的位置。
\end{enumerate}
\subsection{应用题}
\begin{enumerate}
\item 如图1所示,有一弯成$\theta$角的金属架 COD放在磁场中,磁感强度$\vec B$的方向垂直于金属架 COD所在平面。一导体杆MN垂直于 OD边,并在金属架上以恒定速度$\bar v$向右滑动,$\bar v$与MN垂直.设$t=0$时,$x=0$.分别求下列两种情形下,框架内的感应电动势$E_i$:\\
(1)磁场分布均匀,且$\vec B$不随时间改变。\\
\begin{figure}[ht]
\centering
\includegraphics[width=8cm]{./figures/6071f7b6ca21b4a5.png}
\caption{} \label{fig_CD06_1}
\end{figure}
(2)磁场随时间和空间都变化,且磁场强度$B=Kx\sin \omega t$。
\item (\textbf{凝聚杰物理、光学、生物医学物理、应用电子技术专业考生必作})\\
如图2所示,一半径为$R$的均匀带正电圆环,其电荷线密度为$\lambda$。在其轴线上有A、B两点,它们与环心的距离分别为$\displaystyle \bar{OA}=\sqrt{3}R,\bar{OB}=\sqrt{8}R$,一质量为 $m$、电荷为$q$的粒子从A点运动到B点,求在此过程中电场力所作的功。
\begin{figure}[ht]
\centering
\includegraphics[width=8cm]{./figures/0e9d1c146b9fedfc.png}
\caption{} \label{fig_CD06_2}
\end{figure}
\item (\textbf{凝聚态物理、光学、生物医学物理、应用电子技术专业考生必作})\\
如图3所示,一半径为$R$的均匀带电无限长直圆筒,面电荷密度为$\sigma$.该筒以角速度$\omega$绕其轴线匀速旋转。试求圆筒内部的磁感强度。
\begin{figure}[ht]
\centering
\includegraphics[width=8cm]{./figures/a90dd2c1cf450c6c.png}
\caption{} \label{fig_CD06_3}
\end{figure}
\item (\textbf{通论物理、粒子物理与原子核物理、原子分子物理专业必作})\\
如图4所示,一平行板电容器,极板面积为$S$,两板间距离为$d$,其中充有两种各向同性均匀电介质,相对介电常量分别为$\varepsilon_{r1}$和$\varepsilon_{r2}$,且各占一半体积。试证该电容器的电容为\begin{equation}
C=\frac{s_0S}{2d}(\varepsilon_{r1}+\varepsilon_{r2})~
\end{equation}
并说明该电容器相当于左、右两部分作为单独的电容器的并联。
\item (\textbf{理论物理、粒子物理与原子核物理、原子分子物理专业必作})\\
载有稳恒电流$I_1$的无限长直导线(看成刚体)下用一劲度系数为$$K的轻质弹簧挂一载有稳恒电流$I_2$的矩形线圈。设长直导线通电前弹簧长度为$L_0$,通电后矩形线圈将向下移动一段距离,求当磁场对线圈作的功满足A=/hha/2n时,线圈、弹簧、地球组成的系统的势能变化(忽略感应电流对上的影响).
\end{enumerate}