% 角动量
% 角动量

从“角动量”的词义上看,角动量可理解成转动一定角度产生的动量,即角动量首先是一个动量,其次这个动量是由物体的转动产生的.我们知道,物体的动量是由于物体的运动产生的,而描述运动的物理量是速度.在物理当中,常常用物理量的英文单词首字母来表示某个物理量.比如用 $\bvec v$ 表示物理运动的速度.“动量是由物体的运动产生的”表达了动量必须包含两个特性:物体和运动.而运动是由速度 $\bvec v$ 来描述的,那动量必须包含的另一特性“物体”该怎么描述呢?当说到一个物体,不考虑它的几何特性,那你会想到物体的什么属性?在力学当中,所有物体都有个重要的物理量,就是质量,用 $m$表示.所以我们可用质量 $m$ 来描述“物体”这个概念.有了 $m$ 和 $\bvec v$,要由这两部件构成动量,该怎么组合呢?