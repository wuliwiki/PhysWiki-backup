% 格奥尔格·欧姆(综述)
% license CCBYSA3
% type Wiki

本文根据 CC-BY-SA 协议转载翻译自维基百科\href{https://en.wikipedia.org/wiki/Georg_Ohm}{相关文章}。

\begin{figure}[ht]
\centering
\includegraphics[width=6cm]{./figures/33a59a3a2faf0fbf.png}
\caption{乔治·西蒙·欧姆  1789年3月16日  埃尔朗根,勃兰登堡-拜罗伊特(现为德国)} \label{fig_GOM_1}
\end{figure}

乔治·西蒙·欧姆(Georg Simon Ohm,/oʊm/;德语:[ˈɡeːɔʁk ˈʔoːm];1789年3月16日 – 1854年7月6日)是德国物理学家和数学家。作为一名学校教师,欧姆开始研究由意大利科学家亚历山德罗·伏打发明的新的电化学电池。通过使用他自己制作的设备,欧姆发现导体两端施加的电位差(电压)与产生的电流之间存在直接的正比关系。这个关系被称为欧姆定律,而电阻的国际单位“欧姆”(Ω)也以他的名字命名。
\subsection{传记}
\subsubsection{早年生活}  
乔治·西蒙·欧姆出生于一个新教家庭,地点是埃尔朗根,勃兰登堡-拜罗伊特(当时属于神圣罗马帝国)。他是锁匠约翰·沃尔夫冈·欧姆和埃尔朗根裁缝的女儿玛丽亚·伊丽莎白·贝克的儿子。虽然他的父母并未接受正式教育,但欧姆的父亲是一位受人尊敬的人,他通过自学达到了较高的学识水平,并能够通过自己的教导为儿子们提供优良的教育。[4] 在家族的七个孩子中,只有三人活到了成年:乔治·西蒙、他的弟弟马丁(后来成为一位著名数学家)和他的妹妹伊丽莎白·巴巴拉。他的母亲在他十岁时去世。

从小,乔治和马丁就由父亲亲自教育,父亲将他们培养到较高的数学、物理、化学和哲学水平。乔治·西蒙从十一岁到十五岁就读于埃尔朗根中学,但在学校的科学训练上并未得到很多指导,这与他和马丁在父亲那里接受的启发性教育形成了鲜明对比。这一特点使欧姆家族与伯努利家族相似,正如埃尔朗根大学的教授卡尔·克里斯蒂安·冯·朗斯多夫所指出的。
\subsubsection{大学生涯}
Georg Ohm的父亲担心儿子浪费了教育机会,于是将Ohm送往瑞士。1806年9月,Ohm接受了在Gottstadt bei Nidau的一所学校担任数学教师的职位。

1809年初,Karl Christian von Langsdorf离开了埃尔兰根大学,去海德堡大学任职。Ohm希望能在Langsdorf的指导下重新开始数学学习。然而,Langsdorf建议Ohm独立进行数学研究,并建议他阅读欧拉、拉普拉斯和拉克鲁瓦的著作。Ohm有些不情愿地接受了这个建议,但他于1809年3月离开了Gottstatt修道院的教职,转而成为Neuchâtel的私人家教。在两年时间里,他一边担任家教,一边按照Langsdorf的建议继续私人学习数学。直到1811年4月,他才返回埃尔兰根大学。
\subsubsection{教学生涯} 
欧姆的学术研究为他攻读博士学位做了准备,他于1811年10月25日从埃尔朗根大学获得博士学位。此后,他立即加入该校数学系,担任讲师,但由于前景不佳,他在三个学期后离开了。作为讲师,他的薪水无法维持生计。巴伐利亚政府向他提供了一个在班贝格一所质量较差的学校教授数学和物理的职位,欧姆于1813年1月接受了这个职位。对工作不满的欧姆开始编写一本基础几何教材,试图证明自己的能力。那所学校在1816年2月关闭。随后,巴伐利亚政府将欧姆派往班贝格的一所过于拥挤的学校,协助数学教学。


在班贝格的工作结束后,欧姆将他完成的手稿寄给了普鲁士国王威廉三世。国王对欧姆的书籍感到满意,并在1817年9月11日为欧姆提供了科隆耶稣会文理中学的职位。这所学校以良好的科学教育而著称,欧姆不仅需要教授数学,还需要教授物理。该校的物理实验室设备齐全,使得欧姆能够开始进行物理实验。作为一名锁匠的儿子,欧姆在机械装置方面有一定的实践经验。

欧姆于1827年出版了《电池电路的数学研究》(*Die galvanische Kette, mathematisch bearbeitet*)。然而,欧姆所在的学院并未欣赏他的研究成果,于是他辞去了职位。随后,他向纽伦堡的工艺学校提出申请,并成功被聘用。欧姆于1833年到达纽伦堡工艺学校,并于1852年成为慕尼黑大学实验物理学教授。

1849年,欧姆出版了《分子物理学》(*Beiträge zur Molecular-Physik*)。在这部作品的序言中,他表示希望能写出第二卷和第三卷,“如果上帝赐予我长寿的话,第四卷也许能完成”。然而,在发现书中所记载的一项原创发现已被一位瑞典科学家预见之后,他决定不再出版,称:“这一事件使我对‘人算不如天算’这一说法有了全新的、深刻的理解。最初推动我进行此项研究的计划已经消失无踪,而一个新的计划,非我所预见,却在不经意间得以完成。”

欧姆于1854年在慕尼黑去世,并葬于旧南方墓地。他的家族信件被编成一本德文书籍,其中显示他曾在信件的签名处写道“Gott befohlen, G S Ohm”,意思是“托付于上帝”。