% 费马大定理(综述)
% license CCBYSA3
% type Wiki

本文根据 CC-BY-SA 协议转载翻译自维基百科\href{https://en.wikipedia.org/wiki/Fermat\%27s_Last_Theorem}{相关文章}。

在数论中,费马大定理(在较早的文献中有时称为“费马猜想”)陈述如下:对于任意整数\( n > 2 \),不存在三个正整数\( a, b, c \)满足方程\( a^n + b^n = c^n \)。而对于\( n = 1 \)和\( n = 2 \)的情形,自古以来就已知存在无穷多个解。\(^\text{[1]}\)

这个命题最早是皮埃尔·德·费马大约于1637年在一本《算术》书的页边空白处提出的。他还写道他已有一个证明,但“这个证明太大,写不下”。尽管费马曾提出的其他未经证明的命题后来被他人证明并被称为“费马定理”(例如费马两平方和定理),但唯独这条“费马大定理”长期无法证明,使人们怀疑费马是否真的拥有一个正确的证明。因此,这个命题长期以来被称为\textbf{猜想}而不是定理。经过数学家长达358年的努力,安德鲁·怀尔斯于1994年首次成功给出了完整证明,并于1995年正式发表。2016年,怀尔斯因其工作获得阿贝尔奖,其成果被称为“一项惊人的突破”。\(^\text{[2]}\)此外,该证明还涵盖了大量谷山–志村猜想的内容,该猜想后来被称为模性定理,它不仅解开了费马大定理之谜,还开辟了许多新领域,并发展出了强大的模性提升技术,对解决众多其他数学难题产生了深远影响。

这个未解难题曾在 19世纪和20世纪极大地推动了代数数论的发展。在整个数学史上,费马大定理是最著名的定理之一。在被证明之前,它还曾被列入《吉尼斯世界纪录》,称为“最难的数学问题”,部分原因是该定理拥有最多数量的失败证明尝试。\(^\text{[3]}\)
\subsection{概述}  
\subsubsection{毕达哥拉斯的起源}
毕达哥拉斯方程\(x^2 + y^2 = z^2\)在正整数\(x\)、\(y\)、\(z\)上有无穷多组解,这些解被称为毕达哥拉斯三元组(最简单的例子是 3、4、5)。大约在 1637年,费马在一本书的页边空白处写道,更一般形式的方程\(a^n + b^n = c^n\)在当\(n > 2\)时,没有正整数解。尽管他声称自己有一个完整的证明,但他未留下任何细节,至今也未有人找到该证明。这个断言是在他去世大约 30 年后才被人发现的。

这个断言后来被称为费马大定理,并在接下来的三个半世纪里始终未被证明。\(^\text{[4]}\)

这一定理最终成为数学史上最著名的未解问题之一。对它的证明尝试促使了数论领域的重大进展,随着时间推移,费马大定理作为数学未解难题的地位也日益突出。
\subsubsection{后续发展与最终解答}
当\(n = 4\)时的特例由费马本人亲自证明,该特例足以说明:如果费马大定理在某个非素数指数\(n\)下不成立,那么它在某个更小的\(n\)下也将不成立,因此只需对素数指数\(n\)进行进一步研究。\(^\text{[注1]}\)在接下来的两个世纪(1637–1839年)中,该猜想仅在素数\(n = 3, 5, 7\)的情况下被证明成立,尽管索菲·热尔曼提出并证明了一种适用于整个素数类的方法,这在当时具有开创性意义。到了19世纪中期,欧内斯特·库默尔进一步拓展了该思路,并证明了该定理对于所有正规素数成立,但对于非正规素数仍需逐个分析。在库默尔工作的基础上,借助复杂的计算机研究,其他数学家将证明的范围扩展到了所有指数为素数且小于四百万的情况。\(^\text{[5]}\)但对于所有整数指数的全面证明依然遥不可及(这意味着大多数数学家认为:该定理要么无法证明、要么极其困难,或者以当时的知识几乎不可能完成)。\(^\text{[6]}\)

大约在1955年,日本数学家志村五郎和谷山丰怀疑椭圆曲线与模形式之间可能存在某种联系——这是数学中的两个完全不同的领域。这个猜想当时被称为“谷山–志村猜想”(后来称为模性定理),它最初是一个独立的数学命题,表面上与费马大定理并无关联。尽管如此,这个猜想被广泛认为本身就具有重要意义和深远影响,但也像费马大定理一样,被认为极其难以证明,几乎无法触及。\(^\text{[7]}\)

1984年,盖尔哈德·弗雷注意到这两个原本毫不相关、尚未解决的问题之间似乎存在某种联系。弗雷给出了一个初步的推理框架,表明这种联系是可能被证明的。随后,让-皮埃尔·塞尔证明了除一个关键部分(即“\(\varepsilon\)-猜想”)以外的所有部分。1986年,肯·里贝特在塞尔的基础上,完整证明了这两个问题之间的密切联系(参见:里贝特定理与弗雷曲线)。\(^\text{[2]}\)这些由弗雷、塞尔和里贝特发表的论文表明:如果能证明谷山–志村猜想对至少半稳定椭圆曲线成立,那么费马大定理也将随之被证明。这个逻辑关系如下:任何能够反驳费马大定理的解,也可以被用来反驳谷山–志村猜想。因此,如果模性定理(谷山–志村猜想的证明)被确认无误,就意味着不会存在与费马大定理相矛盾的解,也就是说,费马大定理必定成立。

尽管这两个问题在当时都被视为极其艰难、几乎“完全无法触及”的数学难题,\(^\text{[2]}\) 但这是第一次明确提出了一条有可能将费马大定理推广至所有整数指数并完成证明的路径。与费马大定理不同的是,谷山–志村猜想当时正是一个活跃的研究领域,也被认为更有可能在当代数学框架中被攻克。\(^\text{[8]}\)然而,数学界普遍的看法却是:这项联系虽然美妙,但更凸显了谷山–志村猜想的难以证明性。[9] 数学家约翰·科茨的评论代表了当时的主流观点:\(^\text{[9]}\)

“我本人当时非常怀疑费马大定理与谷山–志村猜想之间的这种美妙联系能真正带来什么成果,因为说实话,我并不认为谷山–志村猜想是可以被证明的。尽管这个问题非常优美,但看起来根本不可能被真正证明。我必须承认,我当时认为自己这一生都不太可能看到它被证明。”

当听说里贝特已经证明了弗雷所提出的联系是正确的后,英国数学家安德鲁·怀尔斯决定尝试证明谷山–志村猜想,以此作为证明费马大定理的途径。怀尔斯从小就对费马大定理着迷,并在椭圆曲线及相关领域拥有扎实背景。从1987年起,怀尔斯秘密地独自研究该问题,历时六年。1993年,他成功地证明了谷山–志村猜想中足以推导出费马大定理的部分。怀尔斯的论文在篇幅和理论深度上都非常庞大。在论文同行评审过程中,人们在其中一部分发现了一个错误。为了修复这个缺陷,怀尔斯又花了一年时间,并与他的前学生理查德·泰勒合作,最终解决了这个问题。因此,最终于1995年发表的完整证明中,还附有一篇由两人共同完成的小型论文,用以证明修正部分的有效性。怀尔斯的成就得到了广泛报道,并通过多本书籍和电视节目向大众传播。谷山–志村–韦伊猜想其余尚未证明的部分,在1996至2001年间由其他数学家在怀尔斯工作的基础上陆续完成,最终全部得以证明,该猜想也因此被称为模性定理。\(^\text{[10][11][12]}\)怀尔斯因其对数学的杰出贡献获得了多项荣誉和奖项,其中包括2016年的阿贝尔奖。\(^\text{[13][14][15]}\)
\subsubsection{费马大定理的等价表述}
有几种不同的方式可以表述费马大定理,这些表述在数学上与原问题的陈述是等价的。

为了表述这些方式,我们使用以下符号:设\( \mathbf{N} \)为自然数集合\( 1, 2, 3, \dots \),设\( \mathbf{Z} \)为整数集合\( 0, \pm 1, \pm 2, \dots \),设\( \mathbf{Q} \)为有理数集合\(a/b\),其中\( a \)和\( b \)是整数且\( b \neq 0 \)。接下来我们将称\( x^n + y^n = z^n \)的解,其中一个或多个\( x, y, z \)为零的解为平凡解。所有三个数都非零的解将称为非平凡解。

为了比较,我们从原始表述开始。

\begin{itemize}
\item \textbf{原始表述}:设 \( n, x, y, z \in N \)(意味着 \( n, x, y, z \) 都是正整数)且 \( n > 2 \),方程 \( x^n + y^n = z^n \) 没有解。\\\\  
大多数关于该主题的处理方法都是这样表述的。它也通常在\( \mathbf{Z} \)上表述:\(^\text{[16]}\)
\item \textbf{等价表述 1}:方程 \( x^n + y^n = z^n \),其中整数 \( n \geq 3 \),没有非平凡解 \( x, y, z \in \mathbf{Z} \)。\\\\  
如果 \( n \) 是偶数,等价性是显而易见的。如果 \( n \) 是奇数且 \( x, y, z \) 都是负数,那么我们可以用 \( -x, -y, -z \) 替换 \( x, y, z \) 来得到\( \mathbf{N} \)中的解。如果其中两个是负数,那必须是 \( x \) 和 \( z \) 或 \( y \) 和 \( z \)。如果 \( x, z \) 是负数而 \( y \) 是正数,那么我们可以重新排列得到 \( (-z)^n + y^n = (-x)^n \),从而得到\( \mathbf{N} \)中的解;另一个情况也可以类似处理。现在,如果只有一个是负数,它必须是 \( x \) 或 \( y \)。如果 \( x \) 是负数,且 \( y \) 和 \( z \) 是正数,那么可以重新排列得到 \( (-x)^n + z^n = y^n \),再次得到\( \mathbf{N} \)中的解;如果 \( y \) 是负数,结果是对称的。因此,在所有情况下,\( \mathbf{Z} \)中的非平凡解也意味着 \( \mathbf{N} \)中存在解,这就是原问题的表述。
\item \textbf{等价表述 2}:方程 \( x^n + y^n = z^n \),其中整数 \( n \geq 3 \),没有非平凡解 \( x, y, z \in \mathbf{Q}) \)。\\\\ 
这是因为 \( x, y, z \) 的指数相等(为 \( n \)),因此如果在\( \mathbf{Q} \) 中有解,那么可以通过适当的最小公倍数进行乘法运算,将解转化为\( \mathbf{Z} \)中的解,从而得到\( \mathbf{N} \)中的解。
\item \textbf{等价表述 3}:方程 \( x^n + y^n = 1 \),其中整数 \( n \geq 3 \),没有非平凡解 \( x, y \in \mathbf{Q} \)。\\  
方程 \( x^n + y^n = z^n \) 的一个非平凡解 \( a, b, c \in \mathbf{Z} \) 会给出方程 \( v^n + w^n = 1 \) 的非平凡解 \( a/c, b/c \in \mathbf{Q}\)。反过来,方程 \( v^n + w^n = 1 \) 中的一个解 \( a/b, c/d \in \mathbf{Q}\) 会给出方程 \( x^n + y^n = z^n \) 的非平凡解 \( ad, cb, bd \)。\\\\
这个最后的表述尤其有益,因为它将问题从关于三维空间中的曲面的研究,转化为关于二维空间中曲线的研究。此外,它允许在域\( \mathbf{Q} \)上进行操作,而不是在环 \( \mathbf{Z} \)上进行操作;域比环具有更多的结构,这使得对其元素的分析更加深入。
\item \textbf{等价表述 4} – 与椭圆曲线的联系:如果 \( a, b, c \) 是方程 \( a^p + b^p = c^p \) 的非平凡解,其中 \( p \) 为奇素数,则方程 \( y^2 = x(x - ap)(x + bp) \)(Frey 曲线)将是一个没有模形式的椭圆曲线。\(^\text{[17]}\)\\\\  
通过利用 Ribet 定理检查这条椭圆曲线,可以发现它没有模形式。然而,Andrew Wiles 的证明表明,任何形如 \( y^2 = x(x - an)(x + bn) \) 的方程确实具有模形式。因此,任何 \( x^p + y^p = z^p \)(其中 \( p \) 为奇素数)的非平凡解将会产生矛盾,从而证明没有非平凡解存在。\(^\text{[18]}\)
\end{itemize}
换句话说,任何可能与费马大定理相矛盾的解,也可以用来与模形式定理相矛盾。因此,如果模形式定理被证明为真,那么也可以推导出费马大定理不存在矛盾。正如上文所述,这一等价表述的发现对费马大定理最终的解决至关重要,因为它提供了一种可以“一次性”对所有数字进行“攻克”的方法。
\subsection{数学史}  
\subsubsection{毕达哥拉斯和丢番图}  
\textbf{毕达哥拉斯三元组 } 

在古代,人们已经知道,边长比为 3:4:5 的三角形其一个角是直角。这在建筑和早期几何学中得到了应用。人们还知道,这是一个普遍规律的例子:任何三角形,如果两边的长度各自平方后相加(\( 3^2 + 4^2 = 9 + 16 = 25 \)),等于第三边的平方(\( 5^2 = 25 \)),那么这个三角形也是直角三角形。这就是现在所称的毕达哥拉斯定理,满足这一条件的三个数称为毕达哥拉斯三元组;这两者都以古希腊的毕达哥拉斯命名。例子包括 (3, 4, 5) 和 (5, 12, 13)。这样的三元组有无数个,\(^\text{[19]}\)生成这些三元组的方法在许多文化中都被研究过,从巴比伦人开始,\(^\text{[20]}\)然后是古希腊、中国和印度的数学家们。\(^\text{[1]}\)数学上,毕达哥拉斯三元组的定义是满足方程\( a^2 + b^2 = c^2 \)的三个整数\(^\text{[21]}\)\((a,b,c)\)。

\textbf{丢番图方程}  

费马方程\( x^n + y^n = z^n \)的正整数解是丢番图方程的一个例子,\(^\text{[22]}\) 该方程以公元3世纪的亚历山大数学家丢番图命名,他研究了这些方程并发展了求解某些类型丢番图方程的方法。典型的丢番图问题是找到两个整数\( x \)和\( y \),使得它们的和以及它们的平方和分别等于给定的两个数\( A \)和\( B \):
\[
A = x + y~
\]
\[
B = x^2 + y^2.~
\]
丢番图的主要著作是《算术》,其中只有一部分得以保存。\(^\text{[23]}\)费马关于最后定理的猜想灵感来源于阅读《算术》的新版本,\(^\text{[24]}\)该版本由Claude Bachet于1621年翻译成拉丁文并出版。\(^\text{[25][26]
}\)

丢番图方程已经研究了数千年。例如,二次丢番图方程\( x^2 + y^2 = z^2 \) 的解由毕达哥拉斯三元组给出,最早由巴比伦人(约公元前1800年)解决。\(^\text{[27]}\)线性丢番图方程的解,例如\( 26x + 65y = 13 \),可以使用欧几里得算法(约公元前5世纪)求得。\(^\text{[28]}\)从代数的角度来看,许多丢番图方程的形式与费马最后定理的方程相似,因为它们没有交叉项混合两个字母,但没有共享它的特定性质。例如,已知存在无数个正整数\( x, y, z \),使得\( x^n + y^n = z^m \),其中\( n \)和\( m \)是互质的自然数。\(^\text{[注2]}\)
\subsubsection{费马的猜想}
\begin{figure}[ht]
\centering
\includegraphics[width=6cm]{./figures/ea2ec1c6d24ec545.png}
\caption{《算术》1621年版中的问题II.8。右侧是费马所谓的“最后定理”证明的边注,因边缘过小而无法容纳该证明。} \label{fig_FMDL_1}
\end{figure}
《算术》中的问题 II.8 问的是如何将一个给定的平方数分解为两个其他的平方数;换句话说,给定一个有理数\( k \),找到有理数\( u \)和\( v \),使得 \( k^2 = u^2 + v^2 \)。丢番图展示了如何解决这个平方和问题,当\( k = 4 \)时,解为\( u = \frac{16}{5} \)和 \( v = \frac{12}{5} \)。\(^\text{[29]}\)

大约在1637年,费马在他那本《算术》副本的边缘写下了他的最后定理,紧接着丢番图的平方和问题旁边:\(^\text{[30][31][32]}\)

"Cubum autem in duos cubos, aut quadratoquadratum in duos quadratoquadratos & generaliter nullam in infinitum ultra quadratum potestatem in duos eiusdem nominis fas est dividere cuius rei demonstrationem mirabilem sane detexi. Hanc marginis exiguitas non caperet."

“将一个立方数分解为两个立方数,或者一个四次方分解为两个四次方,或者一般来说,将任何大于二次方的幂分解为两个相同幂次的数,这是不可能的。我已经发现了一个真正奇妙的证明,但这个边缘太窄,无法容纳它。”\(^\text{[33][34]}\)

费马去世后(1665年),他的儿子克莱芒-塞缪尔·费马出版了这本书的新版本(1670年),并增加了他父亲的评论。\(^\text{v}\)虽然当时这并不是真正的定理(即没有证明的数学陈述),但这个边注随着时间的推移被称为费马的最后定理,\(^\text{[30]}\)因为它是费马提出的最后一个未被证明的定理。\(^\text{[36][37]}\)

目前尚不清楚费马是否确实为所有指数\( n \)找到了有效的证明,但似乎不太可能。仅有一份与此相关的证明流传下来,即\( n = 4 \)的情况,正如在 § 特定指数的证明 部分所描述的。

虽然费马将\( n = 4 \)和\( n = 3 \)的情况作为挑战题目给他的数学通信者们,如马林·梅尔森、布莱兹·帕斯卡和约翰·沃利斯,\(^\text{[38]}\)他从未提出过一般情况。\(^\text{[39]}\)此外,在他生命的最后三十年里,费马再也没有写过关于他“真正奇妙的证明”的任何内容,也从未将其发表。范·德·波尔滕\(^\text{[39]}\)建议,尽管缺乏证明并不重要,但缺乏挑战意味着费马意识到自己没有证明;他引用了魏尔\(^\text{[40]}\)的话,认为费马可能曾一时自欺,产生了一个无法挽回的错误想法。费马可能在这种“奇妙证明”中使用的技巧至今未知。

怀尔斯和泰勒的证明依赖于20世纪的技术。\(^\text{[41]}\)与此相比,费马的证明必须是初等的,因为那时的数学知识有限。

虽然哈维·弗里德曼的宏大猜想暗示,任何可证明的定理(包括费马最后定理)都可以仅使用“初等函数算术”来证明,但这样的证明在技术上可能被称为“初等的”,并且可能涉及数百万个步骤,因此远远超出费马所能提供的证明长度。
\subsubsection{特定指数的证明}
\textbf{指数 = 4}  

费马留下的唯一相关证明使用了无限下降法,证明了具有整数边的直角三角形的面积永远不可能等于某个整数的平方。\(^\text{[42][43][44]}\)他的证明等价于证明方程
\[
x^4 - y^4 = z^2~
\]
在整数中没有原始解(没有互质解)。反过来,这证明了费马最后定理在\( n = 4 \)的情况下成立,因为方程\( a^4 + b^4 = c^4 \)可以写成\( c^4 - b^4 = (a^2)^2 \)。

关于 \( n = 4 \) 情况的替代证明后来由以下数学家发展:\(^\text{[45]}\)弗雷尼克尔·德·贝西(1676),\(^\text{[46]}\)莱昂哈德·欧拉(1738),\(^\text{[47]}\)卡乌斯勒(1802),\(^\text{[48]}\)彼得·巴洛(1811),\(^\text{[49]}\)阿德里安-玛丽·勒让德(1830),\(^\text{[50]}\)绍皮斯(1825),\(^\text{[51]}\)奥尔里·泰尔凯姆(1846),\(^\text{[52]}\)约瑟夫·伯特朗(1851),\(^\text{[53]}\)维克多·勒贝格(1853, 1859, 1862),\(^\text{[54]}\)特奥菲尔·佩平(1883),\(^\text{[55]}\) 塔费尔马赫(1893),\(^\text{[56]}\)大卫·希尔伯特(1897),\(^\text{[57]}\)本兹(1901),\(^\text{[58]}\)甘比奥利(1901),\(^\text{[59]}\)利奥波尔德·克罗内克(1901),\(^\text{[60]}\)班格(1905),\(^\text{[61]}\)索默(1907),\(^\text{[62]}\)博塔里(1908),\(^\text{[63]}\)卡雷尔·里赫利克(1910),\(^\text{[64]}\)努茨霍恩(1912),\(^\text{[65]}\)罗伯特·卡迈克尔(1913),\(^\text{[66]}\)汉考克(1931),\(^\text{[67]}\)乔治·弗朗切努(1966),\(^\text{[68]}\)格兰特与佩雷拉(1999),\(^\text{[69]}\)巴巴拉(2007),\(^\text{[70]}\)和多兰(2011)。\(^\text{[71]}\)

\textbf{其他指数}  

在费马证明了特例\( n = 4 \)后,所有\( n \)的一般证明只需证明定理对于所有奇素数指数成立。\(^\text{[72]}\)换句话说,只需要证明方程\( a^n + b^n = c^n \)在\( n \)是奇素数时没有正整数解(\( a, b, c \))。这是因为,对于给定的\( n \),一个解\( (a, b, c)\)等价于所有\( n \)的因子的解。举例来说,设\( n \)被分解为\( d \)和\( e \),即\( n = de \)。则一般方程
\[
a^n + b^n = c^n~
\]
意味着\( (a^d, b^d, c^d) \)是指数\( e \)的解
\[
(a^d)^e + (b^d)^e = (c^d)^e.~
\]
因此,为了证明费马方程在\( n > 2 \)时没有解,只需证明它在每个\( n \) 的至少一个素因子上没有解即可。每个整数\( n > 2 \)都可以被 4 或某个奇素数(或两者)整除。因此,如果费马的最后定理能在\( n = 4 \)和所有奇素数\( p \)上证明成立,那么它就能为所有\( n \)证明成立。

在其猜想后的两个世纪(1637–1839),费马的最后定理在三个奇素数指数\( p = 3, 5, 7 \)上被证明。\( p = 3 \)的情况最早由阿布·马哈茂德·霍贾迪(10世纪)提出,但他对定理的证明是错误的。\(^\text{[73][74]}\)1770年,莱昂哈德·欧拉给出了\( p = 3 \)的证明,\(^\text{[75]}\)但他通过无限下降法\(^\text{[76]}\)的证明存在一个重大漏洞。\(^\text{[77][78][79]}\)然而,由于欧拉本人在其他工作中已证明了完成证明所需的引理,因此他通常被认为是第一个提供证明的人。\(^\text{[44][80][81]}\)独立的证明分别由以下数学家发表:\(^\text{[82]}\)卡乌斯勒(1802),\(^\text{[48]}\)勒让德(1823, 1830),\(^\text{[50][83]}\)卡尔佐拉里(1855),\(^\text{[84]}\)加布里埃尔·拉梅(1865),\(^\text{[85]}\)彼得·古斯特里·泰特(1872),\(^\text{[86]}\)西格蒙德·冈瑟(1878),\(^\text{[87]}\) 甘比奥利(1901),\(^\text{[59]}\)克雷(1909),\(^\text{[88]}\)里赫利克(1910),\(^\text{[64]}\)斯托克豪斯(1910),\(^\text{[89]}\)卡迈克尔(1915),\(^\text{[90]}\)约翰内斯·范·德·科尔普特(1915),\(^\text{[91]}\)阿克塞尔·图厄(1917),\(^\text{[92]}\)和杜阿尔特(1944)。\(^\text{[93]}\)

\( p = 5 \)的情况约在1825年由勒让德和彼得·古斯塔夫·勒让·狄利克雷独立证明。\(^\text{[94][95][96][44][97]}\)曹·弗里德里希·高斯(1875年,死后)、\(^\text{[99]}\)勒贝格(1843年)、\(^\text{[100]}\)拉梅(1847年)、\(^\text{[101]}\)甘比奥利(1901年)、\(^\text{[59][102]}\)韦雷布鲁索夫(1905年)、\(^\text{[103]}\)里赫利克(1910年)、\(^\text{[104]}\)范·德·科尔普特(1915年)\(^\text{[91]}\)和吉·特尔贾尼安(1987年)\(^\text{[105]}\)也提出了替代证明。

\( p = 7 \)的情况由拉梅于1839年证明。\(^\text{[106][107][44][97]}\)他相当复杂的证明在1840年由勒贝格简化,\(^\text{[109]}\)而更简洁的证明\(^\text{[110]}\)则由安杰洛·杰诺基于1864年、1874年和1876年发布。\(^\text{[111]}\)替代证明由特奥菲尔·佩平(1876年)\(^\text{[112]}\)和埃德蒙·梅耶(1897年)\(^\text{[113]}\)提出。

费马最后定理也在指数\( n = 6 \)、\( n = 10 \)和\( n = 14 \)的情况下得到了证明。关于\( n = 6 \)的证明由卡乌斯勒\(^\text{[48]}\)、图厄\(^\text{[114]}\)、塔费尔马赫\(^\text{[115]}\)、林德\(^\text{[116]}\)、卡普费尔尔\(^\text{[117]}\)、斯威夫特\(^\text{[118]}\)和布鲁什\(^\text{[119]}\)发布。同样,狄利克雷\(^\text{[120]}\)和特尔贾尼安\(^\text{[121]}\)各自证明了\( n = 14 \)的情况,而卡普费尔尔\(^\text{[117]}\)和布鲁什\(^\text{[119]}\)各自证明了\( n = 10 \)的情况。严格来说,这些证明是多余的,因为这些情况分别可以通过\( n = 3 \)、\( n = 5 \)和\( n = 7 \)的证明推导出来。然而,这些偶数指数的证明与奇数指数的证明不同。狄利克雷对于\( n = 14 \)的证明发表于1832年,早于拉梅在1839年证明\( n = 7 \)的证明。\(^\text{[122]}\)

所有特定指数的证明都使用了费马的无限下降法,\(^\text{[citation needed]}\)无论是其原始形式,还是在椭圆曲线或阿贝尔流形上的下降形式。然而,细节和辅助论证往往是临时的,并且与所考虑的具体指数相关。\(^\text{[123]}\)随着\( p \)的增大,这些证明变得越来越复杂,因此似乎不太可能通过在个别指数的证明基础上推导出费马最后定理的一般情况。\(^\text{[123]}\)尽管尼尔斯·亨里克·阿贝尔和彼得·巴洛在19世纪初发表了一些关于费马最后定理的普遍结果,\(^\text{[124][125]}\)但对一般定理的首次重要工作是由索菲·热尔曼完成的。\(^\text{[126]}\)
\subsubsection{早期的现代突破}  
\textbf{索菲·热尔曼} 

在19世纪初,索菲·热尔曼发展了几种新颖的方法,试图证明费马最后定理对所有指数成立。\(^\text{[127]}\)首先,她定义了一组辅助素数\( \theta \),它是通过公式\( \theta = 2hp + 1 \)从素数指数\( p \)构造的,其中\( h \)是一个不被3整除的整数。她证明了,如果没有整数的 \( p \)次幂在模\( \theta \)下是相邻的(即非连续性条件),那么\( \theta \)必须整除积\( xyz \)。她的目标是通过数学归纳法证明,对于任何给定的\( p \),无穷多个辅助素数\( \theta \)满足非连续性条件,从而整除\( xyz \);由于积\( xyz \)最多只能有有限数量的素因子,这样的证明就可以确立费马最后定理。尽管她发展了许多技术来确立非连续性条件,但她未能实现她的战略目标。她还努力为给定指数 \( p \) 的费马方程的解设定下限,其中一个修改版由阿德里安-玛丽·勒让德出版。作为这项后期工作的副产品,她证明了索菲·热尔曼定理,验证了费马最后定理的第一个情况(即\( p \)不整除\( xyz \)的情况),对于每个小于270的奇素数指数成立,\(^\text{[127][128]}\)并且对于所有素数\( p \),只要\( 2p + 1 \)、\( 4p + 1 \)、\( 8p + 1 \)、\( 10p + 1 \)、\( 14p + 1 \) 和 \( 16p + 1 \)中至少有一个是素数(特别地,\( 2p + 1 \)是素数的素数\( p \)被称为索菲·热尔曼素数)。热尔曼未能成功证明费马最后定理的第一个情况对于所有偶数指数成立,特别是对于\( n = 2p \),该情况在1977年由吉·特尔贾尼安证明。\(^\text{[129]}\)1985年,莱昂纳德·阿德尔曼、罗杰·希思-布朗和埃蒂安·福夫里证明了费马最后定理的第一个情况对于无穷多个奇素数 \( p \)成立。\(^\text{[130]}\)

\textbf{恩斯特·库默与理想数的理论}  

1847年,加布里埃尔·拉梅提出了一个基于在复数中因式分解方程\( x^p + y^p = z^p \)的费马最后定理证明,特别是基于数1的根构造的循环域。他的证明失败了,因为他错误地假设这些复数可以像整数一样唯一地分解为素数。这个缺陷很快被约瑟夫·柳维尔指出,柳维尔后来阅读了恩斯特·库默的论文,证明了这一唯一分解失败的问题。

库默的任务是确定是否可以将循环域推广到包括新的素数,从而恢复唯一分解的性质。他通过发展理想数理论成功地完成了这一任务。

(人们常说库默因为对费马最后定理的兴趣而得出了他的“理想复数”;甚至有一个故事常被讲述,说库默像拉梅一样认为他已经证明了费马最后定理,直到勒让·狄利克雷告诉他他的论证依赖于唯一分解;但这个故事最早由库尔特·亨塞尔在1910年讲述,证据表明它很可能来源于亨塞尔某些来源的混淆。哈罗德·爱德华兹表示,认为库默主要对费马最后定理感兴趣的观点“无疑是错误的”。\(^\text{[131]}\)请参见理想数的历史。)

通过使用拉梅概述的一般方法,库默证明了费马最后定理对于所有常规素数的两个情况。然而,他无法证明定理对于那些在理论上大约39\%情况下出现的特殊素数(不规则素数);低于270的唯一不规则素数是37、59、67、101、103、131、149、157、233、257和263。

\textbf{莫尔德尔猜想}

在1920年代,路易斯·莫尔德尔提出了一个猜想,暗示如果指数\( n \)大于二,那么费马方程最多只有有限个非平凡的原始整数解。\(^\text{[132][133]}\)这个猜想在1983年由格尔德·法尔廷斯证明,\(^\text{[134]}\)现在被称为法尔廷斯定理。

\textbf{计算研究}  

在20世纪后半叶,计算方法被用来扩展库默的方法,以处理不规则素数。1954年,哈里·范迪弗使用SWAC计算机证明了费马最后定理对于所有小于2521的素数成立。\(^\text{[135]}\)到1978年,塞缪尔·瓦格斯塔夫将这一结果扩展到所有小于125,000的素数。\(^\text{[136]}\)到1993年,费马最后定理已被证明对所有小于四百万的素数成立。\(^\text{[5]}\)

然而,尽管有这些努力和成果,费马最后定理的证明仍然不存在。由于其性质,个别指数的证明永远无法证明一般情况:即使所有指数都在一个极大的数字\( X \)以下得到了验证,仍然可能存在一个超过\( X \)的更高指数,对于该指数,定理可能不成立。(这与一些过去的猜想相似,如斯库伊斯数的情况,不能排除在这个猜想中出现类似的情况。)
\subsection{与椭圆曲线的联系}  
最终导致费马最后定理成功证明的策略来源于“惊人的”\(^\text{[137]: 211}\) 塔尼亚马–岛村–魏尔猜想,该猜想大约在1955年提出——许多数学家认为这个猜想几乎不可能证明,\(^\text{[137]: 223 }\)并且在1980年代,由格哈德·弗雷、让-皮埃尔·塞尔和肯·里贝特将其与费马方程联系起来。通过在1994年完成该猜想的部分证明,安德鲁·怀尔斯最终成功证明了费马最后定理,并为其他数学家提供了通往完整证明的道路,这一证明现在被称为模形式定理。

\textbf{塔尼亚马–岛村–魏尔猜想 } 

大约在1955年,日本数学家岛村五郎和谷山丰观察到椭圆曲线和模形式这两个看似完全不同的数学分支之间可能存在联系。由此产生的模形式定理(当时称为塔尼亚马–岛村猜想)指出,每个椭圆曲线都是模的,意味着它可以与一个唯一的模形式相关联。

最初,这一联系被认为不太可能或高度推测,但当数论学家安德烈·魏尔找到支持这一联系的证据时,这一猜想被更加认真地对待,尽管魏尔并没有证明它;因此,该猜想通常被称为塔尼亚马–岛村–魏尔猜想。\(^\text{[137]: 211–215  }\)

即便在引起广泛关注之后,这一猜想仍然被当时的数学家视为极其困难,甚至可能无法证明。\(^\text{[137]: 203–205, 223, 226}\) 例如,怀尔斯的博士导师约翰·科茨表示,这似乎是“根本不可能实际证明的”,\(^\text{[137]: 226}\) 而肯·里贝特则认为自己“是绝大多数认为它完全无法接近的人的一员”,并补充说:“安德鲁·怀尔斯可能是地球上为数不多的几位敢于梦想自己能去证明它的人之一。”\(^\text{[137]: 223 }\)

\textbf{里贝特定理与弗雷曲线}  

主条目:弗雷曲线 和 里贝特定理  
1984年,格哈德·弗雷注意到费马方程与模形式定理之间的联系,当时模形式定理仍然是一个猜想。如果费马方程对于指数\( p > 2 \)有任何解(\( a, b, c \)),那么可以证明,半稳定的椭圆曲线(现在被称为弗雷-赫勒戈阿尔什曲线\(^\text{[note 3]}\))
\[
y^2 = x(x - ap)(x + bp)~
\]
会具有如此特殊的性质,以至于它不太可能是模形式的。\(^\text{[138]}\)这与模形式定理相冲突,后者断言所有椭圆曲线都是模的。因此,弗雷观察到,塔尼亚马–岛村–魏尔猜想的证明可能同时也能证明费马最后定理。\(^\text{[139][140]}\)通过对立推理,费马最后定理的反证或驳斥将驳斥塔尼亚马–岛村–魏尔猜想。

简而言之,弗雷证明了,如果他对方程的直觉是正确的,那么任何一组能够反驳费马最后定理的四个数字(\( a, b, c, n \)),也可以用来反驳塔尼亚马–岛村–魏尔猜想。因此,如果后者为真,前者就不能被反驳,也必须为真。

按照这一策略,证明费马最后定理需要两个步骤。首先,需要证明模形式定理,或至少证明它对于包括弗雷方程在内的椭圆曲线类型(即半稳定椭圆曲线)成立。这在当时的数学家看来几乎是无法证明的。\(^\text{[137]: 203–205, 223, 226}\) 其次,需要证明弗雷的直觉是正确的:即如果以这样的方式构造椭圆曲线,使用的是费马方程的解集,那么得到的椭圆曲线不可能是模形式的。弗雷证明了这一点是合理的,但并没有给出完整的证明。缺失的部分(即所谓的“ε猜想”,现称为里贝特定理)被让-皮埃尔·塞尔发现,他也给出了几乎完整的证明,弗雷所提示的联系最终在1986年由肯·里贝特证明。\(^\text{[141]}\)

在弗雷、塞尔和里贝特的工作之后,情况是这样的:
\begin{itemize}
\item 费马最后定理需要为所有素数指数\( n \)证明。
\item 如果模形式定理对于半稳定椭圆曲线得到了证明,那么这意味着所有半稳定椭圆曲线必须是模形式的。
\item 里贝特定理表明,任何一个费马方程的素数解都可以用来创建一个不可能是模形式的半稳定椭圆曲线;
\item 只有在这两个陈述都为真的情况下,费马方程没有解(因为那时无法创建这样的曲线),这正是费马最后定理所说的。由于里贝特定理已经证明,这意味着模形式定理的证明将自动证明费马最后定理也为真。
\end{itemize}
\subsubsection{怀尔斯的普遍证明}
\begin{figure}[ht]
\centering
\includegraphics[width=6cm]{./figures/26fc7697185b9923.png}
\caption{英国数学家安德鲁·怀尔斯} \label{fig_FMDL_2}
\end{figure}
1986年,里贝特证明了ε猜想,实现了弗雷提出的两个目标中的第一个。得知里贝特的成功后,英国数学家安德鲁·怀尔斯——他从小对费马最后定理充满兴趣,并且曾研究过椭圆曲线——决定致力于实现第二个目标:为半稳定椭圆曲线证明模形式定理的特例(当时被称为塔尼亚马–岛村猜想)。\(^\text{[142][143]}\)

怀尔斯在近乎完全保密的情况下,花费了六年的时间从事这一任务,通过将之前的工作分割成小部分作为单独的论文发布,并且只与妻子分享他的进展,来掩饰自己的努力。\(^\text{[137]: 229–230}\) 他最初的研究建议通过归纳法来证明\(^\text{[137]: 230–232, 249–252 }\),并且他在初步工作和第一次重大突破时依赖了伽罗瓦理论\(^\text{[137]: 251–253, 259 }\),然后在1990年至1991年左右,当现有方法似乎无法解决问题时,他转而尝试扩展水平岩泽理论进行归纳论证。\(^\text{[137]: 258–259}\) 然而,到1991年中期,岩泽理论似乎也未能触及问题的核心。\(^\text{[137]: 259–260 [144][145]}\)为此,他向同事求助,寻找任何前沿研究和新技术,并发现了由维克托·科利亚金和马蒂亚斯·弗拉赫最近开发的欧拉系统,这一系统似乎是“量身定制”的,适合他证明中的归纳部分。\(^\text{[137]: 260–261}\) 怀尔斯研究并扩展了这一方法,结果证明是有效的。由于他的工作广泛依赖这一新方法,这对于数学界和怀尔斯来说都是新的,在1993年1月,他请求普林斯顿的同事尼克·卡茨帮助他检查推理中的微妙错误。他们当时的结论是,怀尔斯使用的技术似乎是正确的。\(^\text{[137]: 261–265 [146][147]}\)

到1993年5月中旬,怀尔斯准备告诉他的妻子,他认为自己已经解决了费马最后定理的证明问题,\(^\text{[137]: 265}\) 并且到6月,他感到足够自信,准备在1993年6月21日至23日于艾萨克·牛顿数学科学研究所举办的三场讲座中展示他的结果。\(^\text{[148][149]}\) 具体来说,怀尔斯展示了他对半稳定椭圆曲线的塔尼亚马–岛村猜想的证明;结合里贝特对\(\varepsilon\)猜想的证明,这就暗示了费马最后定理。然而,在同行评审过程中,证明中的一个关键点被发现是错误的。该错误涉及对一个特定群的阶数的界限。这个错误被几位审阅怀尔斯手稿的数学家发现,包括卡茨(作为审稿人)\(^\text{[150]}\),他在1993年8月23日提醒了怀尔斯。\(^\text{[151]}\)

这个错误并不会使他的工作变得毫无价值:怀尔斯工作的每一部分都具有重要性和创新性,他在工作过程中创造的许多发展和技术也同样重要,只有其中一部分受到了影响。\(^\text{[137]: 289, 296–297}\) 然而,在没有证明这一部分的情况下,费马最后定理实际上并没有被证明。怀尔斯花了近一年的时间试图修复他的证明,最初是自己修复,后来与他的前学生理查德·泰勒合作,仍然没有成功。\(^\text{[152][153][154]}\)到1993年底,谣言开始传播,怀尔斯的证明在审查下失败了,但具体的严重程度尚不清楚。数学家们开始施压怀尔斯,要求他公开他的工作,无论是否完整,这样更广泛的社区可以探索并利用他所取得的任何成就。但问题并没有像最初看起来那么微不足道,而是变得非常重要,远比预期的严重,也不容易解决。\(^\text{[155]}\)

怀尔斯表示,在1994年9月19日早晨,他几乎要放弃了,几乎已经接受自己失败的事实,并准备发布他的工作,让其他人能够在此基础上继续并修正错误。他补充道,当时他正在做最后的检查,试图理解为什么他的办法不能奏效,这时他突然有了一个灵感:科利亚金–弗拉赫方法无法直接奏效的具体原因,也意味着,如果他利用从科利亚金–弗拉赫方法中获得的经验加强他的岩泽理论的原始尝试,岩泽理论就能够奏效。通过用一个方法中的工具修复另一个方法,将解决所有未在他已审阅的论文中证明的情况。\(^\text{[152][156]}\)他后来描述说,岩泽理论和科利亚金–弗拉赫方法各自单独都不够充分,但结合起来,它们可以强大到克服最后的难关。\(^\text{[152]}\)

“我坐在我的书桌前,研究科利亚金–弗拉赫方法。并不是因为我相信我可以让它奏效,而是我认为至少可以解释为什么它不起作用。突然,我有了一个不可思议的启示。我意识到,科利亚金–弗拉赫方法虽然不起作用,但它正是我三年前原始岩泽理论能够奏效的关键。因此,从科利亚金–弗拉赫的灰烬中似乎升起了问题的真正答案。它是如此难以形容的美丽,既简单又优雅。我无法理解自己是怎么错过的,我就那样盯着它,难以置信地看了二十分钟。然后,在白天,我走遍了整个系,每次都回来看看它是否还在。它还在。我无法抑制自己,激动不已。这是我职业生涯中最重要的时刻。以后我做的任何事都不会像那时那样重要。”

——安德鲁·怀尔斯,西蒙·辛格引用\(^\text{[157]}\)

1994年10月24日,怀尔斯提交了两篇手稿,《模形式椭圆曲线与费马最后定理》\(^\text{[158][159]}\)和《某些赫克代数的环论性质》\(^\text{[160]}\),后者由泰勒合著,并证明了为证明主要论文中修正步骤所需的某些条件得到了满足。这两篇论文经过审核后,于1995年5月刊登在《数学年刊》上。证明中通过将变形环与赫克代数相识别(现称为\(R=T\)定理)来证明模形式提升定理,这一方法在代数数论中产生了深远的影响。

这两篇论文确立了半稳定椭圆曲线的模形式定理,这是证明费马最后定理的最后一步,距其猜想已过去358年。
\subsubsection{后续发展}  
完整的塔尼亚马–岛村–魏尔猜想最终由戴蒙德(1996年)\(^\text{[10]}\)、康拉德等人(1999年)\(^\text{[11]}\)和布吕尔等人(2001年)\(^\text{[12]}\)证明,他们在怀尔斯工作的基础上,逐步攻克了剩余的情况,直到最终证明了完整的结果。现在完全证明的猜想被称为模形式定理。

数论中与费马最后定理类似的其他几个定理也可以通过相同的推理得出,使用模形式定理。例如:没有立方数可以写成两个互质的n次方的和,且\( n \geq 3 \)。(其中\( n = 3 \)的情况已由欧拉知道。)
\subsection{与其他问题及推广的关系}
费马大定理考虑的是费马方程的解:\( a^n + b^n = c^n \),其中 \( a \),\( b \),和\( c \)是正整数,\( n \)是大于 2 的整数。费马方程有多个推广,允许指数\( n \)为负整数或有理数,或者考虑三个不同的指数。
\subsubsection{推广的费马方程}
推广的费马方程通过考虑满足下列方程的正整数解\( a \),\( b \),\( c \),\( m \),\( n \),\( k \)来推广费马大定理的表述\(^\text{[161]}\):
\[
a^m + b^n = c^k.~
\]
特别地,指数\( m \),\( n \),\( k \)不必相等,而费马大定理则考虑了\( m = n = k \)的情况。

比尔猜想(也称为莫尔丁猜想和泰伊德曼-扎吉尔猜想)\(^\text{[162]}\)断言,在正整数\( a \),\( b \),\( c \),\( m \),\( n \),\( k \)中,没有解满足推广的费马方程,且 \( a \),\( b \),和 \( c \)互质,且所有的\( m \),\( n \),\( k \)都大于 2。\(^\text{[166]}\)

费马-卡塔兰猜想通过结合卡塔兰猜想的思想来推广费马大定理。\(^\text{[167][168]}\)该猜想断言,推广的费马方程只有有限个解\( (a, b, c, m, n, k) \),其中\( (a^m, b^n, c^k) \) 是不同的三元组,且\( a \),\( b \),\( c \) 是正整数且互质,\( m \),\( n \),\( k \)是满足以下不等式的正整数:
\[
\frac{1}{m} + \frac{1}{n} + \frac{1}{k} < 1.~
\]
该陈述涉及解集的有限性,因为目前已知有 10 个解。\(^\text{[161]}\)

\subsubsection{逆费马方程}
当我们允许指数\( n \)为整数的倒数,即 \( n = 1/m \)(其中 \( m \)为某个整数)时,我们得到逆费马方程 \( a^{1/m} + b^{1/m} = c^{1/m} \)。这个方程的所有解在 1992 年由亨德里克·伦斯特拉(Hendrik Lenstra)计算得出。\(^\text{[169]}\)在要求\( m \)次根为实数且为正的情况下,所有解由以下公式给出:\(^\text{[170]}\)
\[
a = rs^m~
\]
\[
b = rt^m~
\]
\[
c = r(s+t)^m~
\]
其中\( r \),\( s \),\( t \)为正整数且\( s \) 和 \( t \)互质。
\subsubsection{有理数指数}
对于 Diophantine 方程\( a^{n/m} + b^{n/m} = c^{n/m} \),其中\( n \neq 1 \),Bennett、Glass 和 Székely 在 2004 年证明,对于\( n > 2 \),如果 \( n \) 和 \( m \) 互质,则当且仅当\( 6 \) 整除 \( m \),且 \( a^{1/m} \),\( b^{1/m} \),和 \( c^{1/m} \)是同一个实数的不同复数 6 次根时,方程才有整数解。\(^\text{[171]}\)
\subsubsection{负整数指数}

\textbf{\( n = -1 \)}

所有光学方程\( a^{-1} + b^{-1} = c^{-1} \)的原始整数解(即那些\( a \),\( b \),和\( c \)没有公共质因数的解)可以写为\(^\text{[172]}\):
\[
a = mk + m^2,~
\]
\[
b = mk + k^2,~
\]
\[
c = mk~
\]
其中\( m \)和\( k \)是正整数且互质。

\textbf{\( n = -2 \)}

当\( n = -2 \)时,也有无穷多解,这些解在几何上可以通过具有整数边长和整数高到斜边的直角三角形来解释。\(^\text{[173][174]}\)所有原始解\( a^{-2} + b^{-2} = d^{-2} \)由以下公式给出:
\[
a = (v^2 - u^2)(v^2 + u^2),~
\]
\[
b = 2uv(v^2 + u^2),~
\]
\[
d = 2uv(v^2 - u^2),~
\]
其中\( u \)和\( v \)是互质的正整数且\( v > u \)。几何解释是,\( a \)和\( b \)是直角三角形的整数边,而\( d \)是到斜边的整数高。然后,斜边本身是整数:
\[
c = (v^2 + u^2)^2,~
\]
因此,\( (a, b, c) \)是一个毕达哥拉斯三元组。

\textbf{\( n < -2 \)}

对于\( n < -2 \)的整数解,方程\( a^n + b^n = c^n \)没有解。如果存在解,可以将方程两边同时乘以\( |a|^n |b|^n |c|^n \),得到:\((bc)^n + (ac)^n = (ab)^n\),这与费马大定理矛盾,因此不可能有解。
\subsubsection{abc 猜想}
\(abc\)猜想大致表述为,如果三个正整数\( a \),\( b \)和\( c \)(因此得名)是互质的并且满足 \( a + b = c \),那么\( abc \)的根\( d \)通常不会比\( c \)小得太多。特别地,\(abc\)猜想的最标准表述意味着,对于足够大的\( n \),它可以推导出费马大定理。\(^\text{[175][176][177]}\) 修改版的 Szpiro 猜想与\(abc\)猜想等价,因此也具有相同的推论。\(^\text{[178][177]}\)\(abc\)猜想的有效版本,或修改版 Szpiro 猜想的有效版本,直接推导出费马大定理。\(^\text{[177]}\)
\subsection{奖项和错误的证明}
在1816年和1850年,法国科学院分别为费马大定理提供了一个完整证明的奖金。\(^\text{[179][180]}\)1857年,学院为库默因其关于理想数的研究授予了3000法郎和一枚金奖章,尽管他并未提交任何证明申请该奖项。\(^\text{[179]}\) 1883年,布鲁塞尔科学院又提供了一个奖金。\(^\text{[181]}\)

1908年,德国工业家和业余数学家保罗·沃尔夫斯凯尔将10万金马克(当时是巨额的金额)捐赠给哥廷根科学院,作为对费马大定理完整证明的奖金。\(^\text{[182][183]}\)1908年6月27日,学院发布了九条关于奖励奖金的规则。除此之外,这些规则要求证明必须发表在同行评审的期刊上;奖金在出版后两年才会颁发;并且在2007年9月13日之后将不再颁发奖金,约在比赛开始100年后。\(^\text{[184]}\)威尔斯在1997年6月27日领取了沃尔夫斯凯尔奖金,当时价值5万美元。\(^\text{[185]}\)2016年3月,威尔斯因“通过半稳定椭圆曲线的模形式猜想震惊性地证明了费马大定理,开创了数论的新纪元”,获得了挪威政府颁发的阿贝尔奖,奖金为60万欧元。\(^\text{[186]}\)

在威尔斯的证明之前,沃尔夫斯凯尔委员会收到了数千份错误的证明,信件总长度约为10英尺(3.0米)。\(^\text{[187]}\)在第一个年份(1907–1908) alone,提交了621个尝试的证明,尽管到了1970年代,提交的证明数量减少到了每月约3到4个尝试。有一些说法称,埃德蒙德·兰道倾向于使用一种特殊的预打印表格来处理这些证明,在表格中,第一个错误的位置是空白的,留给他的研究生填写。\(^\text{[188]}\)根据沃尔夫斯凯尔的审稿人F. Schlichting的说法,大多数证明都基于学校里教授的基本方法,且常常是由“有技术教育背景但职业失败的人”提交的。\(^\text{[189]}\)数学史学家霍华德·伊夫斯曾说:“费马大定理具有一个特殊的区别,那就是它是数学问题中发布错误证明最多的。”\(^\text{[181]}\)
\subsection{在大众文化中}
该定理在科学之外的流行使其被描述为获得了“数学界最稀有的荣誉之一:在流行文化中的独特地位。”\(^\text{[190]}\)
\begin{figure}[ht]
\centering
\includegraphics[width=6cm]{./figures/dc193083aeceb573.png}
\caption{捷克邮票纪念威尔斯的证明} \label{fig_FMDL_3}
\end{figure}
亚瑟·波尔吉斯1954年的短篇小说《魔鬼与西蒙·弗拉格》讲述了一位数学家与魔鬼达成交易,要求魔鬼在二十四小时内无法证明费马大定理。\(^\text{[191]}\)

在《辛普森一家》的一集《常青大道的巫师》中,霍默·辛普森在黑板上写下了方程式\( 3987^{12} + 4365^{12} = 4472^{12} \),看起来像是费马大定理的反例。这个方程是错误的,但如果在具有10个有效数字的计算器上输入,它看起来是正确的。\(^\text{[192]}\)

在《星际迷航:下一代》的一集《皇家酒店》中,皮卡德船长提到费马大定理在24世纪仍未被证明。该定理的证明在该集首播五年后发布。\(^\text{[193]}\)
\subsection{另见}
\begin{itemize}
\item 欧拉的幂和猜想  
\item 不可能性证明  
\item 幂和,相关猜想和定理列表  
\item 瓦尔–孙–孙素数
\end{itemize}
\subsection{脚注}  
\begin{enumerate}
\item 如果指数 \( n \) 不是质数或 4,则可以将 \( n \) 写为两个较小整数的乘积(\( n = PQ \)),其中 \( P \) 是大于 2 的质数,那么对于 \( a \)、\( b \) 和 \( c \),有 \( a^n = a^{PQ} = (a^Q)^P \)。也就是说,等效解也必须存在,对于小于 \( n \) 的质数幂 \( P \);或者 \( n \) 会是一个大于 4 的 2 的幂,写成 \( n = 4Q \),那么相同的论证也成立。  
\item 例如,\(((jr + 1)^s)^r + (j(jr + 1)^s)^r = (jr + 1)^{rs+1}\)。  
\item 这条椭圆曲线最早由 Yves Hellegouarch 于1960年代提出,但他并未强调其非模性。更多详情请参见 Hellegouarch, Yves (2001),《费马–威尔斯数学邀请》,学术出版社,ISBN 978-0-12-339251-0。
\end{enumerate}
\subsection{参考文献}
\begin{enumerate}
\item 辛格(Singh),第18–20页
\item “2016年阿贝尔奖——完整引文”。存档于2020年5月20日。检索于2016年3月16日。
\item 《科学与技术》。《吉尼斯世界纪录大全》。吉尼斯出版有限公司,1995年。ISBN 9780965238304。
\item 奈杰尔·波士顿(Nigel Boston)。《费马大定理的证明》(PDF)。第5页。
\item 布勒(Buhler J)、克兰德尔(Crandell R)、恩瓦尔(Ernvall R)、梅岑凯拉(Metsänkylä T)(1993年)。《四百万以内的不规则素数与分圆不变量》。《计算数学》。61(203)。美国数学学会:151–153。Bibcode:1993MaCom..61..151B。doi:10.2307/2152942。JSTOR 2152942。
\item 辛格,第223页
\item 辛格(1997年),第203–205页、第223页、第226页
\item 辛格,第144页引述怀尔斯(Wiles)对此消息的反应:“我如遭电击。那一刻我明白,我的人生轨迹将彻底改变——因为这意味着,要证明费马大定理,我只需证明谷山-志村猜想。我的童年梦想从此成了一项值得全力以赴的工作。”
\item 辛格,第144页
\item 弗雷德·戴蒙德(Diamond, Fred)(1996年7月)。《论形变环与赫克环》。《数学年刊》。144(1):137–166。doi:10.2307/2118586。JSTOR 2118586。
\item 布莱恩·康拉德(Conrad, Brian)、弗雷德·戴蒙德(Diamond, Fred)、理查德·泰勒(Taylor, Richard)(1999年)。《某些潜在Barsotti-Tate伽罗瓦表示的模性》。《美国数学会刊》。12(2):521–567。doi:10.1090/S0894-0347-99-00287-8。ISSN 0894-0347。
\item 克里斯托夫·布鲁伊(Breuil, Christophe)、布莱恩·康拉德(Conrad, Brian)、弗雷德·戴蒙德(Diamond, Fred)、理查德·泰勒(Taylor, Richard)(2001年5月15日)。《论Q上椭圆曲线的模性:狂野3进制练习》。《美国数学会刊》。14(4):843–939。doi:10.1090/S0894-0347-01-00370-8。ISSN 0894-0347。
\item Castelvecchi, Davide (2016年3月15日)。 "费马大定理为安德鲁·威尔斯赢得阿贝尔奖"。《自然》。531(7594):287。Bibcode:2016Natur.531..287C。doi:10.1038/nature.2016.19552。PMID 26983518。S2CID 4383161。  
\item 英国数学家安德鲁·威尔斯获得阿贝尔数学奖 – 《华盛顿邮报》。  
\item 300年的数学问题被解决,教授赢得70万美元奖金 – CNN.com。  
\item Weisstein, Eric W. "费马大定理"。《数学世界 – Wolfram Web资源》。检索日期:2021年5月7日。  
\item Wiles, Andrew (1995)。《模形式椭圆曲线与费马大定理》(PDF)。《数学年刊》。141(3):448。doi:10.2307/2118559。JSTOR 2118559。OCLC 37032255。原文存档(PDF)于2011年5月10日。检索日期:2003年8月11日。Frey的建议,在以下定理的符号中,是展示(假设的)椭圆曲线 \( y^2 = x(x + up)(x - vp) \) 不能是模形式。  
\item Ribet, Ken (1990)。《来自模形式的Gal(Q/Q)的模表示》(PDF)。《数学发明》。100(2):432。Bibcode:1990InMat.100..431R。doi:10.1007/BF01231195。hdl:10338.dmlcz/147454。MR 1047143。S2CID 120614740。  
\item Stillwell, J (2003)。《数论元素》。纽约:Springer-Verlag。第110–112页。ISBN 0-387-95587-9。检索日期:2016年3月17日。
\item Aczel 1996, 第13–15页  
\item Stark 1978, 第151–155页  
\item Stark 1978, 第145–146页  
\item Singh, 第50–51页  
\item Stark 1978, 第145页  
\item Aczel 1996, 第44–45页  
\item Singh, 第56–58页  
\item Aczel 1996, 第14–15页  
\item Stark 1978, 第44–47页  
\item Friberg 2007, 第333–334页  
\item Dickson 1919, 第731页  
\item Singh, 第60–62页  
\item Aczel 1996, 第9页  
\item T. Heath, 《亚历山大·狄奥凡图斯》第二版,剑桥大学出版社,1910年,Dover出版,纽约,1964年,第144–145页  
\item Manin & Panchishkin 2007, 第341页  
\item Singh, 第62–66页  
\item Singh, 第67页  
\item Aczel 1996, 第10页  
\item Ribenboim, 第13、24页
\item 范德波尔滕(van der Poorten),《注释与按语》1.2节,第5页
\item 安德烈·韦伊(André Weil)(1984年)。《数论:历史进路——从汉谟拉比到勒让德》(Number Theory: An approach through history. From Hammurapi to Legendre)。瑞士巴塞尔:Birkhäuser出版社。第104页。
\item BBC纪录片。[失效的YouTube链接]
\item L·弗里曼(Freeman L)(2005年5月12日)。《费马的一个证明》("Fermat's One Proof")。检索于2009年5月23日。
\item 迪克森(Dickson,1919年),第615–616页
\item 阿策尔(Aczel,1996年),第44页
\item 里本博伊姆(Ribenboim),第15–24页
\item 弗雷尼克尔·德贝西(Frénicle de Bessy),《整数直角三角形论》(Traité des Triangles Rectangles en Nombres)第一卷,1676年,巴黎。重印于《皇家科学院纪要》(Mém. Acad. Roy. Sci.)第5卷,1666–1699页(1729年)。
\item 欧拉(Euler L,1738年)。《若干算术定理的证明》("Theorematum quorundam arithmeticorum demonstrationes")。《圣彼得堡科学院新评论》(Novi Commentarii Academiae Scientiarum Petropolitanae)第10卷:125–146页。重印于《欧拉全集》(Opera omnia)第一辑《算术论文集》("Commentationes Arithmeticae")第一卷,第38–58页,莱比锡:Teubner出版社(1915年)。
\item 考斯勒(Kausler CF,1802年)。《关于两个立方数之和或差不能为立方数的新证明》("Nova demonstratio theorematis nec summam, nec differentiam duorum cuborum cubum esse posse")。《圣彼得堡皇家科学院新学报》(Novi Acta Academiae Scientiarum Imperialis Petropolitanae)第13卷:245–253页。
\item 巴洛(Barlow P,1811年)。《数论基础研究》(An Elementary Investigation of Theory of Numbers)。伦敦圣保罗教堂庭院:J·约翰逊出版社。第144–145页。
\item 勒让德(Legendre AM,1830年)。《数论》(第二卷)(第三版)。巴黎:Firmin Didot Frères出版社。1955年由A·Blanchard出版社(巴黎)重印。
\item 绍皮斯(Schopis,1825年)。《不定解析中的若干定理》(Einige Sätze aus der unbestimmten Analytik)。贡宾嫩:Programm出版社。
\item 泰尔凯姆(Terquem O,1846年)。《关于数的幂的定理》("Théorèmes sur les puissances des nombres")。《数学新年鉴》(Nouvelles Annales de Mathématiques)第5卷:70–87页。
\item Bertrand J (1851)。《代数学基础教程》。《巴黎:哈切特出版社》。第217–230页,第395页。  
\item Lebesgue VA (1853)。"解二次方程 \( z^2 = x^4 \pm 2^my^4, z^2 = 2^mx^4 - y^4, 2^mz^2 = x^4 \pm y^4 \)"。《纯数学与应用数学期刊》。18:73–86。  
Lebesgue VA (1859)。《数值分析练习》。《巴黎:Leiber et Faraguet出版社》。第83–84页,第89页。  
Lebesgue VA (1862)。《数论引论》。《巴黎:Mallet-Bachelier出版社》。第71–73页。  
\item Pepin T (1883)。"关于不定方程 \( ax^4 + by^4 = cz^2 \) 的研究"。《意大利国家科学院学报》。物理、数学与自然科学类。Lincei学报。第九辑。数学与应用。36:34–70。  
\item A. Tafelmacher (1893)。"关于方程 \( x^4 + y^4 = z^4 \)"。《智利大学学报》。84:307–320。  
\item Hilbert D (1897)。"代数数域的理论"。《德国数学家协会年报》。4:175–546。1965年再版于《全集》第I卷,纽约:切尔西出版社。  
\item Bendz TR (1901)。《狄奥范廷方程 \( x^n + y^n = z^n \) 的研究》(博士论文)。乌普萨拉:Almqvist & Wiksells出版社。  
\item Gambioli D (1901)。"关于费马最后定理的书目纪实"。《数学期刊》。16:145–192。  
\item Kronecker L (1901)。《数论讲座》第I卷。《莱比锡:Teubner出版社》。第35–38页。1978年纽约:Springer-Verlag出版社再版。  
\item Bang A (1905)。"关于方程 \( x^4 - y^4 = z^4 \) 无有理解的证明"。《数学新期刊》。16B:31–35。JSTOR 24528323。  
\item Sommer J (1907)。《数论讲座》。《莱比锡:Teubner出版社》。  
\item Bottari A (1908)。"毕达哥拉斯方程的整数解及其在证明数论定理中的应用"。《数学期刊》。23:104–110。  
\item Rychlik K (1910)。"费马最后定理对于 \( n = 4 \) 和 \( n = 3 \) 的研究(捷克语)"。《捷克数学与物理学期刊》。39:65–86。  
\item Nutzhorn F (1912)。"不定方程 \( x^4 + y^4 = z^4 \)"。《数学新期刊》。23B:33–38。  
\item Carmichael RD (1913)。"关于某些狄奥范廷方程及方程组不可能解的研究"。《美国数学月刊》。20(7)。美国数学会:213–221。doi:10.2307/2974106。JSTOR 2974106。  
\item Hancock H (1931)。《代数数理论基础》第I卷。纽约:麦克米兰出版社。
\item Gheorghe Vrănceanu (1966)。“费马定理在 \( n = 4 \) 的情况下”。《数学报A系列》。71:334–335。1977年再版于《数学著作》第4卷,第202–205页,布加勒斯特:罗马尼亚社会主义共和国科学院出版社。  
\item Grant, Mike, 和 Perella, Malcolm, "下降到无理数",《数学杂志》83,1999年7月,第263–267页。  
\item Barbara, Roy, "费马最后定理在 \( n = 4 \) 的情况下",《数学杂志》91,2007年7月,第260–262页。  
\item Dolan, Stan, "费马的无限下降法",《数学杂志》95,2011年7月,第269–271页。  
\item Ribenboim,第1–2页  
\item Dickson 1919,第545页  
\item O'Connor, John J.; Robertson, Edmund F. "阿布·马哈穆德·哈米德·伊本·阿尔·基赫德·阿尔·胡詹迪"。MacTutor数学史档案。圣安德鲁斯大学。  
\item Euler L (1770)《代数完全指南》,皇家科学院,圣彼得堡。  
\item Freeman L (2005年5月22日)。“费马最后定理:\( n = 3 \) 的证明”。检索于2009年5月23日。  
\item Ribenboim,第24–25页  
\item Mordell 1921,第6–8页  
\item Edwards 1996,第39–40页  
\item Edwards 1996,第40,52–54页  
\item J. J. Mačys (2007)。“关于欧拉的假设证明”。《数学笔记》。82(3–4):352–356。doi:10.1134/S0001434607090088。MR 2364600。S2CID 121798358。
\item Ribenboim,第33页,第37–41页  
\item Legendre AM (1823)。“关于一些不定分析问题的研究,特别是关于费马定理”。《皇家科学院学报》。6:1–60。1825年作为《数论理论》第二版的“第二补充”再版,出版商为Courcier(巴黎)。1909年也再版于《斯芬克斯-俄狄浦斯》,第4期,97–128页。  
\item Calzolari L (1855)。证明费马定理关于不定方程 \( x^n + y^n = z^n \) 的尝试。费拉拉。  
\item Lamé G (1865)。“关于立方二项式 \( x^3 \pm y^3 \) 的研究”。《科学院周报》。61:921–924,961–965。  
\item Tait PG (1872)。“数学笔记”。《爱丁堡皇家学会会报》。7:144。doi:10.1017/s0370164600041857。  
\item Günther, S. (1878)。“关于不定方程 \( x^3 + y^3 = a^3 \)”。《布拉格皇家波希米亚科学院会议报告》。1878-1880年:112–120。  
\item Krey, H. (1909)。“新的算术定理证明”。《数学自然科学通讯》。6(12):179–180。  
\item Stockhaus H (1910)。费马定理证明的贡献。莱比锡:Brandstetter出版社。  
\item Carmichael RD (1915)。《狄奥范廷分析》。纽约:Wiley出版社。  
\item van der Corput JG (1915)。“一些二次型和一些不定方程”。《新数学档案》。11:45–75。
\item Thue, Axel (1917)。“证明方程 \( A^3 + B^3 = C^3 \) 在不同的整数 \( A \),\( B \),和 \( C \) 中无解”。《数学与自然科学档案》。34(15):3–7。再版于《精选数学论文》(1977),奥斯陆:大学出版社,第555–559页。  
\item Duarte FJ (1944)。“关于方程 \( x^3 + y^3 + z^3 = 0 \)”。《物理、数学与自然科学学报》(加拉加斯)。8:971–979。  
\item Freeman L (2005年10月28日)。“费马最后定理:\( n = 5 \) 的证明”。检索于2009年5月23日。  
\item Ribenboim,第49页  
\item Mordell 1921,第8–9页  
\item Singh,第106页  
\item Ribenboim,第55–57页
\item Gauss CF (1875)。“新立方分解理论”。《复数理论,作品第II卷(第二版)》。哥廷根皇家科学院, 第387–391页。(死后出版)  
\item Lebesgue VA (1843)。“关于不定方程 \( x^5 + y^5 = az^5 \) 的新定理”。《纯数学与应用数学期刊》。8:49–70。  
\item Lamé G (1847)。“关于复杂数解方程 \( A^5 + B^5 + C^5 = 0 \) 的记忆”。《纯数学与应用数学期刊》。12:137–171。  
\item Gambioli D (1903–1904)。“关于费马最后定理”。《毕达哥拉斯》。10:11–13,第41–42页。  
\item Werebrusow AS (1905)。“关于方程 \( x^5 + y^5 = Az^5 \) 的研究(俄文)”。《莫斯科数学汇编》。25:466–473。  
\item Rychlik K (1910)。“费马最后定理对于 \( n = 5 \) 的研究(捷克语)”。《捷克数学期刊》。39:185–195,第305–317页。  
\item Terjanian G (1987)。“关于V. A. Lebesgue的问题”。《傅里叶学院年报》。37(3):19–37。doi:10.5802/aif.1096。  
\item Ribenboim,第57–63页  
\item Mordell 1921,第8页  
\item Lamé G (1839)。“关于费马最后定理的记忆”。《科学院周报》。9:45–46。  
Lamé G (1840)。“不定分析记忆,证明方程 \( x^7 + y^7 = z^7 \) 在整数中不可能有解”。《纯数学与应用数学期刊》。5:195–211。  
\item Lebesgue VA (1840)。“证明方程 \( x^7 + y^7 + z^7 = 0 \) 在整数中无解”。《纯数学与应用数学期刊》。5:276–279,第348–349页。
\item Freeman L (2006年1月18日)。“费马最后定理:\( n = 7 \) 的证明”。检索于2009年5月23日。  
\item Genocchi A (1864)。“关于方程 \( x^7 + y^7 + z^7 = 0 \)”。《纯数学与应用数学年鉴》。6:287–288。doi:10.1007/bf03198884。S2CID 124916552。  
Genocchi A (1874)。“关于某些双重等式不可能性的研究”。《科学院周报》。78:433–436。  
Genocchi A (1876)。“拉梅定理的推广:关于方程 \( x^7 + y^7 + z^7 = 0 \) 不可能性的证明”。《科学院周报》。82:910–913。  
\item Pepin T (1876)。“关于方程 \( x^7 + y^7 + z^7 = 0 \) 不可能性的研究”。《科学院周报》。82:676–679,第743–747页。  
\item Maillet E (1897)。“关于不定方程 \( ax^\lambda t + by^\lambda t = cz^\lambda t \)”。法国科学进步协会,第26次会议报告,第二部分。26:156–168。  
\item Thue A (1896)。“关于某些不定方程的可解性”。《挪威皇家科学学会学报》。7。再版于《精选数学论文》,第19–30页,奥斯陆:大学出版社(1977)。  
\item Tafelmacher WLA (1897)。“方程 \( x^3 + y^3 = z^2 \):费马定理在六次方的情况下的新证明”。《智利大学学报》。97:63–80。
\item Lind B (1909)。“一些数论定理”。《数学与物理学档案》。15:368–369。  
\item Kapferer H (1913)。“证明费马定理对于指数 6 和 10 的情况”。《数学与物理学档案》。21:143–146。  
\item Swift E (1914)。“问题 206 的解答”。《美国数学月刊》。21(7):238–239。doi:10.2307/2972379。JSTOR 2972379。  
\item Breusch R (1960)。“费马最后定理对于 \( n = 6, n = 10 \) 的简单证明”。《数学杂志》。33(5):279–281。doi:10.2307/3029800。JSTOR 3029800。  
\item Dirichlet PGL (1832)。“费马定理对于 14 次方的证明”。《纯数学与应用数学期刊》。9:390–393。再版于《全集》第I卷,第189–194页,柏林:G. Reimer(1889);1969年纽约:切尔西出版社再版。  
\item Terjanian G (1974)。“关于方程 \( x^{14} + y^{14} = z^{14} \) 在整数中的解”。《数学科学学报》。第二系列。98:91–95。  
\item Edwards 1996,第73–74页  
\item Edwards 1996,第74页  
\item Dickson 1919,第733页  
\item Ribenboim P (1979)。 《费马最后定理的13讲》。纽约:Springer Verlag。第51–54页。ISBN 978-0-387-90432-0。  
\item Singh,第97–109页
\item Laubenbacher R, Pengelley D (2007)。“这是我找到的:索菲·热尔曼证明费马最后定理的宏大计划”(PDF)。原文存档于2013年4月5日。检索于2009年5月19日。  
\item Aczel 1996,第57页  
\item Terjanian, G. (1977)。“关于方程 \( x^{2p} + y^{2p} = z^{2p} \)”。《科学院公报,A-B系列》。285:973–975。  
\item Adleman LM, Heath-Brown DR (1985年6月)。“费马最后定理的第一个案例”。《数学发明》。79(2)。柏林:Springer出版社:409–416。Bibcode:1985InMat..79..409A。doi:10.1007/BF01388981。S2CID 122537472。  
\item Edwards 1996,第79页  
\item Aczel 1996,第84–88页  
\item Singh,第232–234页  
\item Faltings G (1983)。“关于代数多样体在数域上的有限性定理”。《数学发明》。73(3):349–366。Bibcode:1983InMat..73..349F。doi:10.1007/BF01388432。S2CID 121049418。  
\item Ribenboim P (1979)。 《费马最后定理的13讲》。纽约:Springer Verlag。第202页。ISBN 978-0-387-90432-0。
\item Wagstaff SS Jr. (1978)。“至125000的非规律素数”。《计算数学》。32(142)。美国数学会:583–591。doi:10.2307/2006167。JSTOR 2006167。(PDF)存档于2012年10月24日于Wayback Machine。  
\item 费马最后定理,Simon Singh,1997年,ISBN 1-85702-521-0。  
\item Frey G (1986)。“稳定椭圆曲线与某些狄奥范廷方程之间的联系”。《萨拉温大学学报。数学系列》。1:1–40。  
\item Singh,第194–198页  
\item Aczel 1996,第109–114页  
\item Ribet, Ken (1990)。“关于从模形式派生的Gal(Q/Q)的模表示”(PDF)。《数学发明》。100(2):431–476。Bibcode:1990InMat.100..431R。doi:10.1007/BF01231195。hdl:10338.dmlcz/147454。MR 1047143。S2CID 120614740。  
\item Singh,第205页  
\item Aczel 1996,第117–118页  
\item Singh,第237–238页  
\item Aczel 1996,第121–122页  
\item Singh,第239–243页  
\item Aczel 1996,第122–125页
\item Singh,第244–253页  
\item Aczel 1996,第1–4页,第126–128页  
\item Aczel 1996,第128–130页  
\item Singh,第257页  
\item Singh,第269–277页  
\item 《一年后,数学证明仍然存在难题》 1994年6月28日  
\item 1994年6月26日–7月2日;《一年后,费马难题仍未完全证明》 1994年7月3日  
\item Singh,第175–185页  
\item Aczel 1996,第132–134页  
\item Singh,第186–187页(文本已简化)  
\item Wiles, Andrew (1995)。“模形式椭圆曲线与费马大定理”(PDF)。《数学年刊》。141(3):443–551。doi:10.2307/2118559。JSTOR 2118559。OCLC 37032255。原文存档(PDF)于2003年6月28日。  
\item 《模形式椭圆曲线与费马大定理》(PDF)。  
\item Taylor R, Wiles A (1995)。“某些赫克代数的环论性质”。《数学年刊》。141(3):553–572。doi:10.2307/2118560。JSTOR 2118560。OCLC 37032255。原文存档于2001年11月27日。
\item Barrow-Green, June; Leader, Imre; Gowers, Timothy (2008)。 《普林斯顿数学指南》。普林斯顿大学出版社。第361–362页。ISBN 9781400830398。  
\item "Mauldin / Tijdeman-Zagier 猜想"。《素数谜题》。检索于2016年10月1日。  
\item Elkies, Noam D. (2007)。"数论的ABC"(PDF)。《哈佛大学数学评论》。1(1)。  
\item Michel Waldschmidt (2004)。“开放的狄奥范廷问题”。《莫斯科数学期刊》。4:245–305。arXiv:math/0312440。doi:10.17323/1609-4514-2004-4-1-245-305。S2CID 11845578。  
\item Crandall, Richard; Pomerance, Carl (2000)。《素数:计算视角》。Springer出版社。第417页。ISBN 978-0387-25282-7。  
\item "比尔猜想"。美国数学学会。检索于2016年8月21日。  
\item Cai, Tianxin; Chen, Deyi; Zhang, Yong (2015)。"费马大定理的新推广"。《数论杂志》。149:33–45。arXiv:1310.0897。doi:10.1016/j.jnt.2014.09.014。S2CID 119732583。  
\item Mihailescu, Preda (2007)。"卡塔兰-费马猜想的一个圆分式研究"。《哥廷根数学》。  
\item Lenstra Jr. H.W. (1992)。"关于逆费马方程"。《离散数学》。106–107:329–331。doi:10.1016/0012-365x(92)90561-s。

\item Newman M (1981)。“一个激进的狄奥范廷方程”。《数论杂志》。13(4):495–498。doi:10.1016/0022-314x(81)90040-8。  
\item Bennett, Curtis D.; Glass, A. M. W.; Székely, Gábor J. (2004)。“有理指数下的费马最后定理”。《美国数学月刊》。111(4):322–329。doi:10.2307/4145241。JSTOR 4145241。MR 2057186。  
\item Dickson 1919,第688–691页  
\item Voles, Roger (1999年7月)。“方程 \( a^{-2} + b^{-2} = d^{-2} \) 的整数解”。《数学杂志》。83(497):269–271。doi:10.2307/3619056。JSTOR 3619056。S2CID 123267065。  
\item Richinick, Jennifer (2008年7月)。“倒立的毕达哥拉斯定理”。《数学杂志》。92:313–317。doi:10.1017/S0025557200183275。S2CID 125989951。  
\item Lang, Serge (2002)。 《代数》。研究生数学教材。第211卷。Springer-Verlag纽约出版社。第196页。  
\item Elkies, Noam (1991)。“ABC暗示莫尔代尔”。《国际数学研究通知》。1991(7):99–109。doi:10.1155/S1073792891000144。我们的证明推广了已知的“有效ABC \(\rightarrow\) 终极费马”的推论,这是ABC猜想的最初动机。
\item Granville, Andrew; Tucker, Thomas (2002)。“这就像 abc 一样简单”(PDF)。《美国数学学会通告》。49(10):1224–1231。  
\item Oesterlé, Joseph (1988)。“费马‘定理’的新方法”。《阿斯特里斯克》。布尔巴基研讨会第694次报告(161):165–186。ISSN 0303-1179。MR 0992208。  
\item Aczel 1996,第69页  
\item Singh,第105页  
\item Koshy T (2001)。《基础数论及应用》。纽约:学术出版社。第544页。ISBN 978-0-12-421171-1。  
\item Singh,第120–125页,第131–133页,第295–296页  
\item Aczel 1996,第70页  
\item Singh,第120–125页  
\item Singh,第284页  
\item "阿贝尔奖颁奖辞 2016"。《阿贝尔奖》。阿贝尔奖委员会。2016年3月。原文存档于2020年5月20日。检索于2016年3月16日。  
\item Singh,第295页  
\item 《轮子、生活与其他数学娱乐》,马丁·加德纳  
\item Singh,第295–296页  
\item Garmon, Jay (2006年2月21日)。“极客趣闻:神话背后的数学”。TechRepublic。检索于2022年5月21日。
\end{enumerate}