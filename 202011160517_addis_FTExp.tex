% 傅里叶变换(指数)
% 微积分|傅里叶级数|傅里叶变换|三角傅里叶变换|指数傅里叶变换

\begin{issues}
\issueAbstract
\end{issues}

\pentry{傅里叶级数(指数)\upref{FSExp},傅里叶变换(三角)\upref{FTTri}}

用三角傅里叶变换 \upref{FTTri} 中同样的方法可把指数傅里叶级数拓展为指数傅里叶变换
\begin{align}
g(k) &= \frac{1}{\sqrt{2\pi }} \int_{-\infty }^{+\infty } f(x)\E^{-\I kx} \dd{x} \\
f(x) &= \frac{1}{\sqrt{2\pi }} \int_{-\infty }^{+\infty } g(k)\E^{\I kx} \dd{k} \label{FTExp_eq1}
\end{align}
特殊地,当 $f(x)$ 为实函数时,$g(k)$ 的实部是偶函数,虚部是奇函数.

\subsection{实数函数的情况}

如果实函数 $f(x)$ 的复数傅里叶变换为 $g(k)$, 即 $g(k)$ 需要满足什么条件才能使 $f(x)$ 是实数呢?我们从\autoref{FTExp_eq1} 开始入手.
\begin{equation}\begin{aligned}
f(x) &= \frac{1}{\sqrt{2\pi }} \int_0^\infty [g(k)\E^{\I kx} + g(-k)\E^{-\I kx}] \dd{k} \\
&= \frac{1}{\sqrt{2\pi }} \int_0^\infty [g(k)+g(-k)]\cos(kx)\dd{k}\\
&\quad + \frac{\I}{\sqrt{2\pi }} \int_0^\infty [g(k)-g(-k)]\sin(kx) \dd{k}
\end{aligned}\end{equation}
从傅里叶变换(三角函数)% 链接未完成
我们已知对实数函数,方括号项都必须为实函数,即上式第一个方括号中的虚部为零,第二个方括号中的实部为零,即
\begin{equation}\begin{aligned}
g_{Re}(-k) &= g_{Re}(k)\\
g_{Im}(-k) &= -g_{Im}(k)
\end{aligned}\end{equation}
所以结论是,当 $f(x)$ 为实函数时,$g(k)$ 的实部是偶函数,虚部是奇函数. 或用共轭记为
\begin{equation}\label{FTExp_eq5}
g(-k) = g(k)^*
\end{equation}
所以 $\abs{g(-k)} = \abs{g(k)}$ 即频谱是偶函数.所以对于实数函数,我们只需要 $k>0$ 的频谱.这与三角傅里叶变换的情况一致.

% 举例未完成

\subsection{性质}
为了书写方便我们用\textbf{算符} $\mathcal F$ 和 $\mathcal F^{-1}$ 表示傅里叶变换和反变换, 即 $\mathcal F f = g$ 以及 $\mathcal F^{-1} g = f$. 算符在这里可以看作 “函数的函数”, 即自变量和函数值都是函数.
\begin{equation}\label{FTExp_eq4}
\mathcal F [f(x) \E^{\I k_0 x}] = g(k - k_0)
\end{equation}
\begin{equation}
\mathcal F^{-1} [g(k) \E^{-\I k x_0}] = f(x - x_0)
\end{equation}
也就是说, 给函数乘以 $\E^{\I k_0 x}$ 因子再做傅里叶变换, 等于先对函数做傅里叶变换, 再向右平移 $k_0$. 给函数乘以 $\E^{-\I k x_0}$ 因子再做反傅里叶变换, 等于先对函数做反傅里叶变换, 再向右平移 $x_0$. 证明显然, 留做习题.

变换前后模长不变
\begin{equation}\label{FTExp_eq2}
\int g(k)^* g(k) \dd{k} = \int f(x)^* f(x) \dd{x}
\end{equation}

导数的傅里叶变换
\begin{equation}\label{FTExp_eq3}
\mathcal F [f'(x)] = \I k \mathcal F[f(x)]
\end{equation}

\subsection{性质的证明}

\textbf{证明}\autoref{FTExp_eq2} : 把傅里叶变换看成傅里叶级数在 $l \to \infty$ 时的极限, 使用\autoref{FSExp_eq5}~\upref{FSExp}, 右边的求和在极限下变为积分即可证明. 详细过程留做习题.

\textbf{证明}\autoref{FTExp_eq3} : 对\autoref{FTExp_eq1} 关于 $x$ 求导得
\begin{equation}
f'(x) = \frac{1}{\sqrt{2\pi }} \int_{-\infty }^{+\infty } [\I kg(k)]\E^{\I kx} \dd{k}
\end{equation}
把方括号看作一整个 $k$ 的函数, 那么上式对应的反变换为
\begin{equation}
\I kg(k) = \frac{1}{\sqrt{2\pi }} \int_{-\infty }^{+\infty } f'(x) \E^{-\I kx} \dd{x} = \mathcal F [f'(x)]
\end{equation}
其中 $g(k) = F [f(x)]$, 证毕.
\addTODO{证明其他性质}
