% Julia 的函数

\begin{issues}
\issueDraft
\end{issues}

\begin{itemize}
\item 函数名字中可以包含 \verb|!|, 但不能在首字母。 也可以在任何位置包含 unicode
\item 快速定义函数: \verb|f(x, y) = x + y;| (assignment form)
\item 通常来说, 函数名后面加 \verb|!| 表示函数参数会被改变。 例如 \verb|v = [3,2,1]; sort(v)| 返回排好的数组, 但 \verb|v| 不改变。 \verb|sort!(v)| 直接改变 \verb|v|。
\item 算符都是函数, 例如 \verb|1 + 2 + 3 + ...| 相当于 \verb|+(1, 2, 3, + ...)|
\end{itemize}


\begin{table}[ht]
\centering
\caption{算符和对应的函数}\label{JuFunc_tab1}
\begin{tabular}{|c|c|}
\hline
Expression & Calls \\
\hline
\verb|[A B C ...]| & \verb|hcat| \\
\hline
\verb|[A; B; C; ...]| & \verb|vcat| \\
\hline
\verb|[A B; C D; ...]| & \verb|hvcat| \\
\hline
\verb|A'| & \verb|adjoint| \\
\hline
\verb|A[i]| & \verb|getindex| \\
\hline
\verb|A[i] = x| & \verb|setindex!| \\
\hline
\verb|A.n| & \verb|getproperty| \\
\hline
\verb|A.n = x| & \verb|setproperty!| \\
\hline
\end{tabular}
\end{table}

\begin{itemize}
\item \verb|[1,2,3]'| 返回的是 \verb|LinearAlgebra.Adjoint{Int64, Vector{Int64}}|, Julia 真心牛逼。
\item 可以看出, Julia 可以把若干算符合并为一个函数, 这立马解决了 C++ expresion template 才能搞定的痛点。 Julia 真心牛逼。
\item \verb|@which sin(pi)| 可以返回调用的 \verb|sin| 的定义(具体到行)。
\item \verb|function g(x,y)::变量类型| 可以限制返回的变量类型。
\item \verb|fun(a,b,x...)| 可以定义任意变量个数的函数。 在函数体内 \verb|x| 是一个 tuple, 包括后面的所有变量。
\item \verb|return 表达式| 用于返回一个值。 如果不使用, 那就返回最后一个表达式的值, 如果最后一个表达式没有值(如 \verb|println()|), 那就返回 \verb|nothing|。
\end{itemize}

\subsubsection{默认值与命名值}
\begin{itemize}
\item \verb|fun(a, b=2; c=3, d) = a + b + c - d;| 中, \verb|b| 的默认值是 \verb|2|, 含有默认值的普通变量必须放在所有普通变量的最后。 \verb|;| 后面是命名变量, 每一个都可以有或没有默认值。
\end{itemize}


\subsubsection{匿名函数}
\begin{itemize}
\item \verb|f = x -> x^2 +1|, 类型是 \verb|var"#1#2"|, 是 \verb|Function| 的子类。
\end{itemize}

\subsection{常用函数}
\begin{itemize}
\item \verb|sort| 和 \verb|sort!|
\item \verb|map(函数, 数组)| 对数组中每个元素使用函数。
\end{itemize}
