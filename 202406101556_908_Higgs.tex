% 希格斯粒子(科普)
% license CCBYSA3
% type Wiki

(本文根据 CC-BY-SA 协议转载自原搜狗科学百科对英文维基百科的翻译)

  希格斯玻色子是粒子物理标准模型中的基本粒子,由粒子物理理论中的希格斯场量子激发产生。它是以物理学家彼得·希格斯(Peter Higgs)的名字命名的,Higgs在1964年与其他五位科学家一起提出了这种粒子存在的机制。它的存在于2012年由欧洲核子中心(CERN)的ATLAS实验组和CMS实验组基于大型强子对撞机(LHC)上的对撞实验的联合确认。

2013年12月10日,彼得·希格斯(Peter Higgs)和弗朗索瓦·恩格勒(François Englert)两位物理学家,因为他们的理论预测被授予诺贝尔物理学奖。虽然希格斯的名字已经与这个理论(希格斯机制)联系在一起,但在1960年至1972年间,还有其他的一些研究人员在这个问题的不同方面作出了自己独立的贡献。

主流媒体经常将希格斯玻色子称为“\textbf{上帝粒子}”,源于1993年一本关于这个话题的书[1] ,但是许多物理学家,包括希格斯本人,都认为这个绰号是有些夸大的。

\subsection{介绍}

\subsubsection{1.1标准模型}

标准模型是目前物理学家广泛接受的用来解释基本粒子之间力的性质的理论框架,可以用于理解除了引力(一个独立的理论,广义相对论,用于引力理论)之外的已知宇宙中的几乎所有事物。在这个模型中,自然界的基本力来自于我们宇宙的规范不变性和对称性。力是由被称为规范玻色子的粒子传递的[2][3]。

在标准模型中,希格斯粒子是自旋为零的玻色子,没有电荷,也没有色荷。它也非常不稳定,几乎立即衰变为其它粒子。希格斯场是标量场,有两个中性的和两个带电的分量,它们形成了弱同位旋SU(2)对称性的复二重态。希格斯场的势是“墨西哥帽形”的。这导致场在它的基态任何地方都有非零值(包括其他的“空"空间),结果,在非常高的能量之下,电弱相互作用的弱同位旋对称性的破缺。(技术上非零期望值将拉格朗日量中的汤川耦合项转换为质量项。)当这种情况发生时,希格斯场的三个分量被SU(2)和U(1)规范玻色子(“希格斯机制”)吸收,成为现在有质量的传递弱力W及Z玻色子的纵向分量。其余的电中性分量要么表现为希格斯粒子,要么可以单独与费米子耦合(通过汤川耦合),促使这些粒子也获得质量[4]。

\subsubsection{1.2规范玻色子质量问题}

场论在理解电磁场和强力方面取得了巨大成功,但是到了1960年左右,所有试图建立一个将电磁相互作用和弱相互作用结合在一起的 规范不变 的弱力理论都失败了,规范场论的声誉从此受到了巨大的影响。问题在于规范场论中的对称性要求电磁力的规范玻色子(光子)和弱力的规范玻色子(W和Z玻色子)都应该具有零质量。虽然光子确实没有质量,但实验表明弱玻色子有质量[5]。 这意味着要么规范不变性是不正确的理论,要么还有其它未知的理论赋予这些粒子质量,但是所有试图提出能够解决这个问题的理论的尝试似乎都遇到新的理论上的问题。

到20世纪50年代末,物理学家还没有解决这些问题,这些问题是发展成熟的粒子物理学理论遇到的重大障碍。

\textbf{对称性破缺}

到20世纪60年代初,物理学家已经意识到,在某些条件下,至少在某些物理领域,给定的对称定律可能并不总是被遵循 。这就是所谓的对称破缺,并在20世纪50年代末被南部阳一郎(Yoichiro Nambu)证实。对称破缺会导致意想不到的结果。1962年,超导物理学家菲利普·安德森(Philip Anderson)写了一篇研究粒子物理学中对称性破缺的论文,提出对称性破缺可能是解决粒子物理学中规范不变性问题所缺少的一部分。如果电弱对称不知何故被打破了,这也许可以解释为什么电磁玻色子没有质量,而弱玻色子有质量,并解决了问题。不久之后,在1963年,理论上证明这是可能的,至少对于一些有限的(非相对论性的)情况。

\subsubsection{1.3希格斯机制}

继1962年和1963年的论文之后,1964年三组研究人员在《物理评论快报》(PRL)上分别独立发表了对称破缺的论文,这些论文具有相似的结论,适用于所有情况,而不仅仅是一些有限的情况。他们表明,如果宇宙中存在一种不寻常的场,使得电弱对称性的条件将被“打破”,从而使一些基本粒子将获得质量。产生这种机制所需要的场(当时纯粹是假设性的)被称为希格斯场 (以研究人员之一彼得·希格斯(Peter Higgs)的名字命名)以及它导致对称性破缺的机制,称为希格斯机制。这个场的一个关键特征是不像其他已知的场真空期望值在Φ为0处,而是在Φ不为0处能量更少。因此,希格斯场各处具有非零值(或 真空期望值) 。这是第一个能够证明在规范不变理论中,弱规范玻色子如何在对称性的情况下仍然具有质量的假设。

虽然这些想法最初并没有得到广泛的支持或关注,但到了1972年,它们已经发展成为一个全面的理论,并被证明能够给出“合理的”结果,准确地描述当时已知的粒子,并以异常准确的精度预测了随后几年发现的其他几个粒子。 在20世纪70年代,这些理论迅速形成了粒子物理学的标准模型。虽然目前还没有任何直接证据证明希格斯场存在,但即使没有该场存在的证据,其预测的准确性也让科学家相信该理论可能是真的。到了1980年代,希格斯场是否存在,以及整个标准模型是否正确的问题,已经被认为是粒子物理学中最重要的未解问题之一。

\textbf{希格斯场}

根据标准模型,这个所需的场(即 希格斯场)存在于全空间中,并且破坏了电弱相互作用的某些对称性 。通过希格斯机制,这个场使得弱规范玻色子在低于一个极高温度值之下具有质量。当弱玻色子获得质量时,它们只能存在很短的距离 。此外,后来人们意识到,同样的场也会以不同的方式解释为什么物质的其他基本部分(包括电子和夸克)具有质量。

几十年来,科学家们无法确定希格斯场是否存在,因为还没有探测它所需的技术。如果希格斯场确实存在,那么它将不同于任何其他已知的基本领域,但也有可能这些关键思想,甚至整个标准模型,在某种程度上是不正确的 。只有发现希格斯玻色子和希格斯场的存在才能解决了这个问题。

与电磁场等其他已知场不同,希格斯场是标量场,在真空中具有非零常数值。希格斯场的存在成为粒子物理学标准模型中最后一个未经证实的部分,几十年来被认为是“粒子物理学的中心问题”。

这个场的存在,现在被实验研究证实,解释了为什么一些基本粒子有质量,尽管它们的相互作用所遵循的对称性意味着它们应该是无质量的。它还解决了其他几个长期存在的难题,比如为什么弱力范围极小。

尽管希格斯场在任何地方都不是零,其影响无处不在,但证明它的存在绝非易事。原则上,它可以通过检测它的激发态来证明它的存在(希格斯玻色子),但是这些是非常难以生产和检测的。这个重要的基本问题经历了40年的探索,以及迄今为止世界上最昂贵和最复杂的实验设施之一——欧洲核子中心的大型强子对撞机的建造[6] ,就是试图产生希格斯玻色子和其他粒子用于观察和研究。2012年7月4日,CERN宣布发现了一个质量在125~ 127 GeV/c2 之间的新粒子。物理学家怀疑这是希格斯玻色子[7] 。从那以后,这个粒子被证明符合标准模型预测希格斯粒子的许多方面表现、相互作用和衰变,并且具有希格斯玻色子的两个基本属性偶宇称和零自旋。 这也意味着它是自然界中发现的第一个基本标量粒子 。截至2018年,深入研究表明,该粒子的行为仍然符合标准模型对希格斯玻色子的预测。需要进行更多的研究来更精确地验证所发现的粒子是否具有预测的所有性质,或者如一些理论所描述的,是否存在多个希格斯玻色子[8]。

\textbf{希格斯玻色子}

理论假设的希格斯机制做出了数个准确的预测 然而,为了证实它的存在,寻找与”希格斯玻色子”相匹配的粒子,人们进行了广泛的研究 。由于产生,希格斯玻色子所需要的能量,以及即使能量足够,希格斯玻色子的产生也非常罕见,因此很难探测到希格斯玻色子。所以,几十年后,希格斯玻色子存在的第一个证据才被发现。能够寻找希格斯玻色子的粒子对撞机、探测器和计算机花费了30多年时间(约1980年至2010年)才研制出来。

到2013年3月,希格斯玻色子的存在得到了确认,因此,某种类型的希格斯场存在于全空间的概念得到了强有力的支持 。目前正在利用LHC收集的更多数据,进一步研究希格斯场的本质和性质。

\textbf{解释}

各种各样的类比被用来描述希格斯场和玻色子,包括与众所周知的对称破缺效应的类比,例如彩虹和棱镜,电场,波纹,和宏观物体在介质中移动的阻力(例如人们在人群中移动或一些物体在糖浆或糖蜜中移动)。然而,基于简单的运动阻力的类比是不准确的,因为希格斯场不是通过抵抗运动来工作的。

\subsection{意义}

\subsubsection{2.1粒子物理学}

\textbf{标准模型的验证}

希格斯玻色子通过质量产生机制验证了标准模型。随着对其属性进行更精确的测量,一些相关的理论可以得以继续发展或被排除正确的可能性。随着测量场的行为和相互作用的实验手段的发展,这个基本场可能会得到更好的理解。如果希格斯场没有被发现,标准模型将需要被修改或被其他理论取代。

与此相关的是,物理学家普遍认为有超越标准模型的“新”物理的存在,标准模型在某些时候会需要新的内容或者失效。希格斯玻色子的发现,以及在LHC上其他对撞实验,为物理学家分析标准模型失效的部分提供了一个敏感的工具,并且可以提供相当多的证据指导研究人员进入未来的理论发展。

\textbf{弱电相互作用的对称破缺}

在极高的温度之下,电弱对称破缺导致电弱相互作用部分表现为由有质量规范玻色子携带的短程弱力。这种对称破缺是原子和其他结构形成所必需的,也是恒星(如太阳)核反应所必需的。希格斯场在这种对称破缺中起到主要作用。

\textbf{粒子质量的获得}

希格斯场在夸克和带电轻子(通过汤川耦合)获得质量和W和Z规范玻色子(通过希格斯机制)获得质量方面起着关键作用。

值得注意的是,希格斯场不是凭空“创造”质量的(这将违反能量守恒定律),希格斯场也不对所有粒子的质量负责。例如,重子(复合粒子,如质子和中子)的质量的大约99%是由于量子色动力学结合能获得的,这是夸克的动能和在重子内部传递强相互作用的无质量胶子的能量之和[9] 。在基于希格斯玻色子的理论中,“质量”的性质是当基本粒子与希格斯场相互作用(“耦合”)时转移到基本粒子的势能的一种表现,该粒子以能量的形式具有该质量[10]。

\textbf{标量场与标准模型的扩展}

希格斯场是唯一被发现的标量场(自旋为0);标准模型中其它所有的场都是自旋½的费米子或自旋为1玻色子。当希格斯玻色子被发现时,欧洲核子中心总干事罗尔夫-迪特尔·霍耶尔(Rolf-Dieter Heuer)认为,证明标量场的存在几乎和希格斯粒子在确定其他粒子质量中的作用一样重要。它表明,其他理论从暴胀到暗能量假设的标量场,也可能存在[11][12]。