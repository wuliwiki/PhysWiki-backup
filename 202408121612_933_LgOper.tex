% 大型运算符
% keys 运算符|求和|求积|极限|积分|并集|交集|最值
% license Xiao
% type Wiki

\begin{issues}
\issueDraft
\end{issues}

这篇文章会介绍大型运算符,他们往往代表了一个过程。下面会主要介绍他们的过程、记法以及一些常用的性质。

\subsection{累计运算过程}

累计运算过程一般包括:
\begin{itemize}
\item 析取符号:$\bigvee$
\item 合取符号:$\bigwedge$
\item 并集符号:$\bigcup$
\item 交集符号:$\bigcap$
\item 求和符号:$\sum$
\item 求积符号:$\prod$
\item 上积符号:$\coprod$
\item 积分符号:$\int$
\end{itemize}


\subsubsection{记法}

\subsubsection{无穷情况}

有时会出现,下限位置为$-\infty$,上限位置为$+\infty$等情况,比如级数等情况。这种情况并非实指去到对应点,而是指取相应极限,即:

\begin{equation}
\sum_{i=-\infty}^n a_i:= \lim_{a\to-\infty}\sum_{i=a}^n a_i.~
\end{equation}

\begin{equation}
\sum_{i=0}^{+\infty} a_i:= \lim_{a\to+\infty}\sum_{i=0}^a a_i.~
\end{equation}

特殊地:
\begin{equation}
\sum_{i=-\infty}^{+\infty} a_i:= \lim_{a\to+\infty\atop b\to-\infty}\sum_{i=b}^a a_i.~
\end{equation}

\begin{example}{设等比级数通项$a_n=a_1q^{n-1}$,用求和符号表示等比级数和}
$$
\sum_{i=1}^{+\infty} a_i=
\leftgroup{
{a_1\over 1-q},\qquad |q|<1\\  
\text{发散},\qquad |q|\geq1
} ~.
$$
\end{example}

\subsubsection{指标集记法}

上面介绍的求和过程,指标集都是$\mathbb{Z}$与某个区间的交集(这样可以保证自增),对于某些不需要自增或无法使用自增的场合,可以直接给定指标集,求和过程记作:

\begin{equation}
\sum_{i\in I} a_i~.
\end{equation}
其中,$I$为指标集。特别的,若$I$为空集,一般称为“空和”,并定义为0,即:
\begin{equation}
\sum_{i\in \varnothing} a_i=0~.
\end{equation}

同时,若通过上下文可以明确指标集$I$或只为表述记号的运算特点而不强调指标集时,在不引起歧义的情况下,为了方便可以直接记作:

\begin{equation}
\sum_i a_i.~
\end{equation}

这样使用时通常不会涉及指标的运算。
\subsubsection{通用性质}
这一部份性质是
\subsection{求和运算}
\subsubsection{积分近似}

对于求和项数量非常多的情况,可以利用黎曼和的思想来近似化简,将求和过程转化成\enref{定积分}{DefInt}。

\begin{equation}
\sum_{i=a}^b f(i) \approx \int_a^b f(x) \, dx.~
\end{equation}


\subsubsection{多个求和符号的运算技巧}

有一些场合会出现一个表达式中出现多个求和符号的情形。一如上面所说,展开仍然是通用的方法,但过于复杂会让人望而却步。下面的运算技巧绝对会让化简过程如虎添翼。

交换求和次序

类似于一个表格中,不论是先求行和再求和,还是先求列和再求和,结果都是将所有的数字求一遍。

\begin{equation}
\sum_{i,j} a_{ij}=\sum_{i} \sum_{j} a_{ij} = \sum_{j} \sum_{i} a_{ij}.~
\end{equation}
其中 $\sum\limits_{i,j}$ 表示按任意顺序遍历所有$(i,j)$。

双重求和

两个求和符号的积可以展开为它们每一项组合的积的和。

\begin{equation}
\qty(\sum_i a_i) \qty(\sum_j b_j) = \sum_{i,j} a_i b_j = \sum_i \qty(a_i \sum_j b_j) = \sum_j \qty(b_j \sum_i a_i)~.
\end{equation}

注意:第一个等号从左到右没有条件,但从右到左时需要保证每个变量的独立性,即$a_i$中不包含$j$,$b_j$中不包含$i$。

\begin{example}{用求和符号计算数列$\{a_n\}$的和的平方}
$$\qty(\sum_i a_i)^2 = \sum_{i,j} a_i a_j = \sum_i a_i^2 + 2\sum_{i<j} a_i a_j~.$$
注意区分求和指标,第二步如果写成 $\sum\limits_{i,i} a_i a_i$ 将产生混乱。
\end{example}
\subsection{极限运算过程}

$\lim$

\subsection{极值运算过程}

\begin{itemize}
\item 上确界:$\sup$
\item 下确界:$\inf$
\item 最大值:$\max$
\item 最小值:$\min$
\end{itemize}
