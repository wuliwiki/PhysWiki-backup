% 傅科摆
% keys Fouclt|傅科摆|科里奥利力|几何推导

\pentry{单摆\upref{Pend}}

傅科摆是首个直接证明地球自转的实验. 试想如果把一个不受任何阻力的单摆放在地球的北极, 那么地球每自转一定角度, 单摆的摆平面不变, 所以以地球为参考系观察, 摆平面将反方向转动, 这样就能证明地球在自转. 现实中, 为了能克服阻力和微扰长时间摆动, 通常使用质量较大, 摆臂较长的摆作为傅科摆.

但若傅科摆被放在北纬 $\alpha$ 角处, 摆平面的将会以怎样的角速度转动呢? 事实证明, 若令地球自转的角速度为 $\omega_0$, 则单摆相对地面转动的角速度 $\omega$ 将等于

\begin{equation}
\omega = \omega_0 \sin\alpha
\end{equation}

\subsection{傅科摆角速度的一种几何推导}
\pentry{球坐标系\upref{Sph}, 连续叉乘的化简\upref{TriCro}}
设 $\bvec R$ 为地心指向傅科摆的矢量, $\hat {\bvec R}$ 是其单位矢量, 当地纬度为 $\alpha$, 地轴指向北的单位矢量为 $\uvec k$, 有
\begin{equation}
\uvec k\vdot \uvec R = \cos(\pi/2 - \alpha) = \sin\alpha
\end{equation}

若把任意矢量 $\bvec P$ 围绕某单位矢量 $\uvec M$ 以右手定则旋转角微元 $\dd{\theta}$, 有
\begin{equation}\label{Fouclt_eq2}
\dd{\bvec P} = \uvec M \cross \bvec P \dd{\theta}
\end{equation}
开始时, 令傅科摆在最低点的速度方向的单位向量为 $\uvec A$ ($\uvec A\vdot \bvec R = 0$), 在傅科摆下方的水平地面上标记单位向量 $\uvec B$, 使开始时 $\uvec B = \uvec A$. 当傅科摆随地球在准静止状态下移动位移 $\dd{\bvec s}$ ($\dd{\bvec s} \vdot\bvec R = 0$)后, 由\autoref{Fouclt_eq2} 可得

\begin{equation}
\dd{\uvec A} = \uvec M\cross \uvec A\vdot \dd{\theta} = 
\frac{\bvec R\cross \dd{\bvec s}}{\abs{\bvec R\cross \dd{\bvec s}}} \cross \uvec A \frac{\dd{s}}{R}
\end{equation}
注意这只是一个比较符合物理直觉的假设, 这里并不给出证明. 当地球转动 $\dd{\theta}$ 时, 上式中 $\dd{\bvec s} = \uvec k \cross \bvec R \dd{\theta}$, 而地面上的标记 $\uvec B$ 也围绕地轴转动, 所以 $\dd{\uvec B} = \uvec k \cross \uvec A \dd{\theta}$.

下面计算 $\dd{\uvec A} - \dd{\uvec B}$. 因为 $\bvec R\vdot \dd{\bvec s} = 0$, 所以 $\abs{\bvec R \cross \dd{\bvec s}} = R\dd{s}$, 所以
\begin{equation}\ali{
\dd{\uvec A} &= \frac{\bvec R\cross \dd{\bvec s}}{R^2}\cross\uvec A =
\frac{1}{R^2}\bvec R\cross(\uvec k\cross\bvec R \dd{\theta})\cross \uvec A\\
&= \uvec R\cross(\uvec k\cross\uvec R)\cross\uvec A \dd{\theta} =
[(\uvec R\vdot\uvec R)\uvec k - (\uvec R\vdot \uvec k)\uvec R] \cross \uvec A \dd{\theta}\\
&= (\uvec k - \uvec R\sin\alpha) \cross \uvec A \dd{\theta}
}\end{equation}
其中第二行使用了 “连续叉乘的化简\upref{TriCro}”.
\begin{equation}
\dd{\uvec A} - \dd{\uvec B} = (\uvec k - \uvec R\sin\alpha)\cross \uvec A \dd{\theta} - \uvec k \cross \uvec A \dd{\theta} = -\sin\alpha \uvec R \cross \uvec A \dd{\theta}
\end{equation}
所以地球转过 $\dd{\theta}$ 角以后, $\uvec A$ 与 $\uvec B$ 之间的夹角为
\begin{equation}
\dd{\gamma} = \abs{\dd{\uvec A} - \dd{\uvec B}} = \sin\alpha \dd{\theta}
\end{equation}
两边除以 $\dd{t}$ 得角速度
\begin{equation}
\omega = \omega_0 \sin\alpha
\end{equation}
