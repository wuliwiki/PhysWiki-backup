% 二分
% 二分|算法|C++

二分(查找)是个很强的算法,二分也是个很麻烦的算法,因为有很多边界问题.

据说,只有 $10 \%$ 的程序员能写对二分.二分的写法有很多,选一种自己习惯的写法就行.Um_Nik 的名言:“停止学习无用算法,二分查找解决问题!”

二分的本质不是单调性,有单调性的问题一定能二分,但能二分的问题可以不具有单调性.因此二分的本质是\textbf{二段性}.具体来讲,一段区间,使得左半边满足一个性质,右半边不满足、左半边不满足一个性质,而右半边满足.所以可以通过二分查找来找到这个分界点.

二分查找算法每次把区间一分为二使搜索范围缩小一半,所以二分的时间复杂度为 $\mathcal{O}(\log_2 n)$.

以一个具体的例子看二分:

给定一个按照升序排列的长度为 $n$ 的整数数组 $a$,以及 $q$ 个查询.
对于每个查询,返回一个元素 $x$ 的起始位置和终止位置(位置从 $0$ 开始计数).如果数组中不存在该元素,则返回 \verb|-1 -1|.

样例:
\begin{lstlisting}[language=cpp]
左半边输入,右半边输出
6 3
1 2 2 3 3 4
3             3 4
4             5 5
5            -1 -1
\end{lstlisting}

假设要二分一下 $3$ 的起始位置,设置二分分界点为 $\dfrac{l+r}{2}$,每次判断一下二分条件,再划分边界.

这里的二分条件有两种情况.

\begin{enumerate}
\item 判断一下分界点是不是不小于 $3$,如果判断条件成立,则说明答案在左半边区间,则应该把右半边区间缩小.相反(分界点小于 $3$),则说明答案在右半边区间,则应该把左半边区间缩小.
\item 判断一下分界点是不是小于 $3$,如果判断条件成立,则说明答案在右半边区间,则应该把左半边区间缩小.相反(分界点不小于 $3$),则说明答案在左半边区间,则应该把右半边区间缩小.
\end{enumerate}
可见这两种情况是相对应的,可以根据自身习惯选择其中一种写法.

\begin{lstlisting}[language=cpp]
// 二分查找 x 的起始位置

int l = 0, r = n - 1;

// 写法一
while (l < r)
{
	int mid = l + r >> 1;  // (l + r) / 2
    if (a[mid] >= x) r = mid;
    else l = mid + 1;
}

// 写法二
while (l < r)
{
    int mid = l + r >> 1;
    if (q[mid] < x) l = mid + 1;
    else r = mid;
}
\end{lstlisting}

\begin{figure}[ht]
\centering
\includegraphics[width=14.25cm]{./figures/BS_1.png}
\caption{查找起始位置} \label{BS_fig1}
\end{figure}


二分查找 $x$ 的终止位置也与查找起始位置类似.

这里的二分条件同样也有两种情况.

\begin{enumerate}
\item 判断一下分界点是不是不大于 $3$,如果判断条件成立,则说明答案在右半边区间,应该把左半边区间缩小.相反缩小右半边区间.
\item 判断一下分界点是不是大于 $3$,如果判断条件成立,则说明答案在左半边区间,应该把右半边区间缩小.相反缩小左半边区间.
\end{enumerate}
\begin{lstlisting}[language=cpp]
// 二分查找 x 的终止位置

int l = 0, r = n - 1;

// 写法一
while (l < r)
{
	int mid = l + r >> 1;  // (l + r) / 2
    if (a[mid] <= x) l = mid;
    else r = mid - 1;
}

// 写法二
while (l < r)
{
    int mid = l + r + 1 >> 1;
    if (q[mid] > x) r = mid - 1;
    else l = mid;
}
\end{lstlisting}



还有一点边界情况需要注意,当二分查找 $x$ 的终止位置时,分界点要多加 $1$.

简单的证明一下:

因为当有 $l, r$ 两个位置只差一的时候,有 $l = r - 1, r = l + 1$,当不补上加一的时候,计算分界点时 $\frac{r - 1 + l + 1}{2} = \frac{l+r}{2}=l$(C++ 除法是下取整),当判断条件分界点不大于 $x$ 成立时,执行 \verb|l = mid| 之后 \verb|l = mid = l|,所以就会陷入死循环中.所以需要补上加一.

