% 南京理工大学 2009 年 研究生入学考试试题 普通物理(B)
% license Usr
% type Note

\textbf{声明}:“该内容来源于网络公开资料,不保证真实性,如有侵权请联系管理员”

\subsection{一、填空题(每题 2 分,共 30 分)}
1. 一半径为 $a$ 的金属球带电 $Q$,其周围充满介电常数为$\xi$ 的均匀无限大介质,则金属球内部的电场能量为__________,金属球外内部的电场能量为______________。

2. 一无限长载流直线 $I$ 与一载流矩形回路共面,其尺寸如图所示,则载流线圈受到的力矩大小为___________;电流 $I$ 激发的磁场通过回路的磁通量________。

3. 互感系数的物理意义是________。

4. 油轮漏油$(n=1.2)$入海,在海面上形成一大片油膜,如有人在膜厚为 480nm的油膜上空的飞机上垂直往下看,能看到 $\lambda=$_________$nm$ 的光;如有人在 45 度方向往油膜看,又能看到 $\lambda=$___________$nm$ 的光。(海水的折射率为 1.33)

5. 一部分偏振光由线偏振光和自然光组成,让该部分偏振光经过一可旋转的线偏振片后,得到 $Imax/Imin=5/2$,则该部分偏振光中线偏振光的比例为________;如用自然光通过,则 $Imax/Imin=$______________。

6. 宽度为 $a$ 的一维无限深势阱中,粒子的波函数为:$\psi_\pi(x)=\sqrt{\frac{2}{a}}\sin\frac{n \pi}{a}x$ ,则该粒子在势阱中出现的几率密度表达式为 $P=$______________,若 $n=2$ 时,粒子在$x=$________________处出现的概率最大。

7. 电子的静止质量是 $m_0$,当电子以 $v=0.8c$ 的速度运动时,它的运动动能为____________,总能量为________________。

8. 在氢原子光谱的莱曼系$(n=1)$中,最短波长为_______$nm$,最长波长为_______$nm$。
\subsection{二、填空题(每空 2 分,共 30 分)}
1. 一质点作直线运动,运动方程为 x=3+3t2-t3(t>0)(SI 制),则该质点 t=1s
时,v=___________;在 t=_______时,质点开始作减速直线运动。
2. 有一孤立球形天体绕过球心的自转轴转动,其初始转动惯量为 I0,初始转动
到那个能为 Ek0。若干年后由于自身收缩致使其转动惯量减少为 I0/2,则此刻其
自转角速度大小为_____________,其转动动能的改变量△Ek=___________。
3. 一质量为 m、长为 l 的均匀直尺,立于桌面 O 点,直尺可绕 O 点转动。该系统
对 O 轴的转动惯量 I=__________,直尺在竖直位置由静止开始转动,直尺刚到
地之前的角速度 大小为_____________,此时直尺的角加速度 β 大小为_____
_____________。
4. 某种理想气体的密度为 ρ,摩尔质量为 μ,其最概然(最可几)速率为 vρ,
则该气体的分子数密度 n=_________,其压强 P=________________。
5. t=27°C时,1mol 氧气分子的内能为______________;t=27°C时,1mol 氧气分
子的平均总动能为___________________。
6. 一倔强系数为 k 的轻质弹簧,一端系以质量为 m 的小球,此系统振动时,振
动频率是_________________。若将弹簧截成相同的两端,取其一段系住小球
m,则系统的振动频率是__________________。
7. 一半径为 R 的导体球的球心为 O 点,电量为 q,在距 O 点 2R 处的电场强度
E=_____________,在距 O 点 R/2 处的电势 U=______________