% 泊松括号
% keys 哈密顿量|泊松括号|守恒量

\begin{issues}
\issueDraft
\end{issues}

\pentry{哈密顿正则方程\upref{HamCan}}

给定函数 $u(q, p, t)$ 和 $v(q, p, t)$, 定义泊松括号为
\begin{equation}
\pb{u}{v} = \sum_i \pdv{u}{q_i}\pdv{v}{p_i} - \pdv{v}{q_i}\pdv{u}{p_i}
\end{equation}
容易证明
\begin{equation}
\pb{u}{v} = -\pv{v}{u}
\end{equation}

对任意不显含时的物理量 $\omega (q,p)$ 都有
\begin{equation}\label{poison_eq1}
\dot \omega  = \sum_i \qty(\pdv{\omega}{q_i} \dot q_i + \pdv{\omega}{p_i} \dot p_i)
= \sum_i \qty(\pdv{\omega}{q_i} \pdv{H}{p_i} - \pdv{H}{q_i} \pdv{\omega}{p_i})
= \pb{\omega}{H} 
\end{equation}
所以若泊松括号恒等于零, 则该物理量守恒.

同理, 当 $\omega (q,p,t)$ 显含时间时有
\begin{equation}
\dot \omega  =  \pb{\omega}{H}  + \pdv{\omega}{t}
\end{equation}


量子力学中的对易算符对应泊松括号. 该式对应量子力学中的算符平均值演化方程.
