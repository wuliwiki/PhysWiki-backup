% 电磁感应
% license CCBYSA3
% type Wiki

(本文根据 CC-BY-SA 协议转载自原搜狗科学百科对英文维基百科的翻译)
\begin{figure}[ht]
\centering
\includegraphics[width=8cm]{./figures/c17b904609835bc2.png}
\caption{法拉第的实验显示了线圈之间的感应:液体电池(右)提供流过小线圈的电流(一)创造一个磁场。当线圈静止时,不会感应出电流。但是当小线圈移入或移出大线圈时(二),通过大线圈的磁通量变化,感应出由检流计检测到的电流(G)。[1]} \label{fig_DCGY_1}
\end{figure}

\textbf{电磁感应}是在变化的磁场中跨越电导体产生的电动势(即电压)。

麦可·法拉第通常被认为是1831年感应的发现,和詹姆斯·克拉克·麦克斯韦数学上将其描述为法拉第感应定律。楞次定律描述感应场的方向。法拉第定律后来被推广为麦克斯韦-法拉第方程,这四个方程之一麦克斯韦方程在他的理论中电磁。

电磁感应有许多应用,包括电气元件如电感器和变压器,以及设备如电动机和发电机。
\subsection{基本概念}
1831年,一位叫迈克尔·法拉第的科学家发现了磁与电之间的相互联系和转化关系。只要穿过闭合电路的磁通量发生变化,闭合电路中就会产生感应电流。这种利用磁场产生电流的现象称为电磁感应(Electromagnetic induction),产生的电流叫做感应电流。

电磁感应现象的产生条件有两点(缺一不可)。
\begin{itemize}
\item 闭合电路。

\item 穿过闭合电路的磁通量发生变化。
\end{itemize}

让磁通量发生变化的方法有两种。一种方法是让闭合电路中的导体在磁场中做切割磁感线的运动;另一种方法是让磁场在导体内运动。
\subsubsection{1.1 磁通量}
设在匀强磁场中有一个与磁场方向垂直的平面,磁场的磁感应强度为 $ B $,平面的面积为 $S $。

(1) 定义:在匀强磁场中,磁感应强度 $ B $) 与垂直磁场方向的面积 $ S $ 的乘积,叫做穿过这个面的磁通量,简称磁通。

(2) 定义式:
$\varphi = B \cdot S$

当平面与磁场方向不垂直时
$\varphi = B \cdot S = BS \cos \theta$

(( $\theta$) 为平面的垂线与磁场方向的夹角)

(3) 物理意义

垂直穿过某个面的磁感线条数表示穿过这个面的磁通量。

(4) 单位:在国际单位制中,磁通量的单位是韦伯,简称韦,符号是 Wb。

1 Wb = 1 T·m2 = 1 V·s。

(5) 标量性:磁通量是标量,但是有正负之分。
\subsubsection{1.2 现象}
(1)电磁感应现象:闭合电路中的一部分导体做切割磁感线运动,电路中产生感应电流。
\begin{figure}[ht]
\centering
\includegraphics[width=8cm]{./figures/1ff263d55b2e4e63.png}
\caption{电磁感应现象} \label{fig_DCGY_2}
\end{figure}
(2)感应电流:在电磁感应现象中产生的电流。

(3)产生电磁感应现象的条件:

①两种不同表述

a.闭合电路中的一部分导体与磁场发生相对运动

b.穿过闭合电路的磁场发生变化

②两种表述的比较和统一

a.两种情况产生感应电流的根本原因不同

闭合电路中的一部分导体与磁场发生相对运动时,是导体中的自由电子随导体一起运动,受到的洛伦兹力的一个分力使自由电子发生定向移动形成电流,这种情况产生的电流有时称为动生电流。

穿过闭合电路的磁场发生变化时,根据电磁场理论,变化的磁场周围产生电场,电场使导体中的自由电子定向移动形成电流,这种情况产生的电流称为感应电流或感生电流。

b.两种表述的统一

两种表述可统一为穿过闭合电路的磁通量发生变化。

③产生电磁感应现象的条件

不论用什么方法,只要穿过闭合电路的磁通量发生变化,闭合电路中就有电流产生。

条件:a.闭合电路;b.一部分导体 ; c.做切割磁感线运动
\subsubsection{1.3 能量的转化}
能的转化守恒定律是自然界普遍规律,同样也适用于电磁感应现象。
\subsubsection{1.4 感应电动势}
(1)定义:在电磁感应现象中产生的电动势,叫做感应电动势。方向是由低电势指向高电势。

(2)产生感应电动势的条件:穿过回路的磁通量发生变化。

(3)物理意义:感应电动势是反映电磁感应现象本质的物理量。

(4)方向规定:内电路中的感应电流方向,为感应电动势方向。

(5)反电动势:在电动机转动时,线圈中也会产生感应电动势,这个感应电动势总要削弱电源电动势的作用,这个电动势称为反电动势。
\subsection{历史}
\begin{figure}[ht]
\centering
\includegraphics[width=8cm]{./figures/722fd17ce20fe82b.png}
\caption{法拉第铁环装置的示意图。左线圈磁通量的变化会在右线圈中感应出电流。[1]} \label{fig_DCGY_3}
\end{figure}

\begin{figure}[ht]
\centering
\includegraphics[width=6cm]{./figures/b89500168fa1715a.png}
\caption{法拉第圆盘(见单极发生器)} \label{fig_DCGY_4}
\end{figure}
电磁感应是由麦可·法拉第于1831年发现的。[1][2]它是由约瑟夫·亨利于1832年独立发现的。[3][4]

在法拉第的第一次实验演示中(1831年8月29日),他将两条电线缠绕在铁环或“圆环”的相对两侧(这种布置类似于现代的环形变压器)。[来源请求]基于他对电磁体的理解,他预计当电流开始在一根导线中流动时,一种波将穿过环,并在相对侧产生一些电效应。他将一根电线插入检流计,并在将另一根电线连接到电池时观察它。当他将电线连接到电池上时,他看到了一个瞬时电流,他称之为“电波”,当他断开电池时,他看到了另一个瞬时电流。[5]这种感应是由于电池连接和断开时发生的磁通量变化。[6]两个月内,法拉第发现了电磁感应的其他几种表现形式。例如,当他快速将条形磁铁推入和拉出线圈时,他看到了瞬态电流,并且他通过用滑动的电导线旋转条形磁铁附近的铜盘(“法拉第盘”)产生了稳定的( DC )电流。[7]

法拉第用一个他称之为磁力线的概念解释了电磁感应。然而,当时的科学家普遍拒绝接受他的理论观点,主要是因为它们不是数学公式。[8]一个例外是詹姆斯·克拉克·麦克斯韦,他使用法拉第的思想作为他定量电磁理论的基础。[8][9][10]在麦克斯韦的模型中,电磁感应的时变方面被表示为微分方程,奥利弗·亥维赛称之为法拉第定律,尽管它与法拉第的原始公式略有不同,并且没有描述。Heaviside的版本(见下面的)是今天在被称为麦克斯韦方程组的方程组中公认的形式。

1834年,海因里希·楞次制定了以他的名字命名的定律来描述“通过电路的通量”。楞次定律给出了由电磁感应产生的感应电动势和电流的方向。
\subsection{理论}
\subsubsection{3.1 法拉第感应定律和楞次定律}