% 速度规范
% 长度规范|速度规范|波函数|规范变换|薛定谔方程|麦克斯韦方程组

\pentry{长度规范\upref{LenGau}}

和长度规范中的思路一样, 我们只在使用偶极子近似时讨论长度规范. 用角标 $V$ 表示速度规范, 先从规范不变的哈密顿算符(\autoref{LenGau_eq2}~\upref{LenGau})出发
\begin{equation}
H_V = H_0 - \frac{q}{2m} (\bvec A_V \vdot \bvec p + \bvec p \vdot \bvec A_V)
+ \frac{q^2}{2m} \bvec A_V^2 + q \varphi_V
\end{equation}

对库仑规范使用规范变换
\begin{equation}\label{LVgaug_eq3}
\Psi_C(\bvec r, t) = \exp(\I q\chi_V)\Psi_V(\bvec r, t)
\end{equation}
\begin{equation}\label{LVgaug_eq4}
\chi_V(t) = -\frac{q}{2m} \int_{-\infty}^t \bvec A_C^2(t') \dd{t'}
\end{equation}
使用规范变换(\autoref{QMEM_eq5}~\upref{QMEM})
\begin{equation}\label{LVgaug_eq1}
\bvec A_V = \bvec A_C - \grad \chi_V = \bvec A_C
\end{equation}
可见速度规范和库仑规范中的矢势相同, 注意广义动量(\autoref{QMEM_eq6}~\upref{QMEM})也和库伦规范的相同, 以下统一记为 $\bvec A_C$.
\begin{equation}
\bvec p_V = \bvec p_C =  m \bvec v + q\bvec A_{C} = -\I \grad
\end{equation}

再看标势的变换:
\begin{equation}\label{LVgaug_eq5}
\varphi_V = \varphi_C + \pdv{\chi_V}{t} = - \frac{q}{2m} \bvec A_C^2
\end{equation}
()和()带入()可以消除 $\bvec A^2$ 项得
\begin{equation}
H^V = H_0 - \frac{q}{m} \bvec A \vdot \bvec p
\end{equation}
薛定谔方程为
\begin{equation}
H^V \Psi^V = \I \pdv{t} \Psi^V
\end{equation}
这种规范叫做\textbf{速度规范(velocity gauge)}.

与长度规范的关系见\autoref{LenGau_eq3}~\upref{LenGau}
\begin{equation}
\Psi^V = \exp[\I q(\chi^L - \chi^V)]\Psi^L = \exp[-\I\frac{q^2}{2m}\int_{-\infty}^t \bvec A^2(t')\dd{t'} + \I q \bvec A\vdot \bvec r] \Psi^L
\end{equation}
