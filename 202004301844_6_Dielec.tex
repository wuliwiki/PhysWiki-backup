% 电介质

首先来了解一下何为电介质.
\begin{definition}{电介质}
\textbf{电介质(dielectric)}是电阻率很大、导电能力很差的物质,其主要特征在于它的原子或分子中的电子与原子核的结合力很强,电子处于束缚状态.当电介质处在电场中时,在电介质中,不论是原子中的电子、还是分子中的离子或是晶体点阵上的带电粒子,在电场的作用下都会在原子大小的范刚内移动,当达到静电平衡时,在电介质表面层或在体内会出现极化电荷,这个现象称作电介质的\textbf{极化(polarization)}.
\end{definition}
下面就研究电场与电介质间的相互作用,从而说明电介质的一些性质