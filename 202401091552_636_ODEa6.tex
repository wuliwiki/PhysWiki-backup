% 判别曲线法求一阶隐式常微分方程的奇解
% keys 判别曲线|ODE|隐式常微分方程|奇解|包络
% license Usr
% type Wiki

\begin{issues}
\issueDraft
\end{issues}

\pentry{包络和奇解\upref{EnvSol},一阶隐式常微分方程的存在唯一性定理\upref{ODEa5}}
一般的一阶隐式常微分方程,往往可能会出现对于一阶隐式常微分方程的存在唯一性定理\upref{ODEa5}的判定中,条件 $(3)$ 的不满足。也就是可能出现 $F(x,y,y'), F'_y(x,y,y'), F'_{y'}(x,y,y')$ 连续,但在某处 $F(x,y,y')=0$ 的情况。在这一点处解的唯一性\textbf{可能}不成立,从而 $F(x,y,y')=0$ 可能有奇解产生。

由奇解性质可以知道,奇解就是由通解构成的曲线族的包络线,由包络线的求法引出了求一阶隐式常微分方程的奇解的以下两种求法。
\subsection{$p$-判别曲线法}\label{sub_ODEa6_1}
$p$-判别曲线法的思路是直接求包络线,再检验包络线是否是原方程的解。
\begin{definition}{$p$-判别曲线}
若方程 $F(x,y,y')=0$ 有奇解,则这奇解必定满足两方程:
\begin{equation}\label{eq_ODEa6_1}
F(x,y,y')=0, F'_{y'}(x,y,y')=0 ~.
\end{equation}
记 $y'=p$,有方程组:
\begin{equation}\label{eq_ODEa6_2}
\left \{
\begin{aligned}
F(x,y,p) &= 0~, \\
F'_{p}(x,y,p) &= 0~.
\end{aligned}
\right .
\end{equation}
由这方程组确定的曲线(其中 $p$ 为参数,可以消元),称为方程 $F(x,y,y')=0$ 的 \textbf{$p$-判别曲线}。
\end{definition}
显然,若原方程有奇解,则一定包含在方程组\autoref{eq_ODEa6_2}所确定的 $p$-判别曲线中,但 $p$-判别曲线不一定是原方程的解。求解出 $p$-判别曲线后应当做以下两个检验:
\begin{itemize}
\item 检验某条 $p$-判别曲线是否是原方程的解;
\item 检验这条曲线上的各点,是否至少有原方程的另一条积分曲线相切。
\end{itemize}

\subsection{$c$-判别曲线法}\label{sub_ODEa6_2}
$c$-判别曲线法的思路是先求出通解的曲线族,再求这曲线族的包络。