% test

% !TEX program = xelatex
% 使用 texlive 完整编译:
% xelatex -> bibtex -> xelatex -> xelatex
% 攻读博士学位科研计划书  LaTeX 模板
\documentclass{phdproposal}
% 进行个人信息设置
\suptitle{XX大学 202X 年攻读博士学位}
\title{科学研究计划}
\author{某~某~某}    % 作者姓名
\date{2\,0\,2\,X~年~1\,2~月}  % 报名时间
\college{学~院~名~称}  % 未用
\major{计~算~数~学}  % 专业名称
\study{微~分~方~程~数~值~解}
\instructor{某~某~某}  % 导师姓名

% 添加自己要用的其他宏包
\usepackage[numbers]{natbib}
\usepackage{subfig}
% \usepackage{xltxtra}

% --- 证明结束黑框 ----
% \renewcommand{\qedsymbol}{$\blacksquare$}

% --- 设置英文字体 -----
\usepackage{newtxtext}  % for text fonts

% --- 设置数学字体 -----
% \usepackage{newtxmath}
% \usepackage{mathptmx}

% --- 直接插入 pdf 文件 ----
% \usepackage{pdfpages}

% --- 自定义命令 -----
\newcommand{\CC}{\ensuremath{\mathbb{C}}}
\newcommand{\RR}{\ensuremath{\mathbb{R}}}
\newcommand{\A}{\mathcal{A}}
\newcommand{\ii}{\bm{\mathrm{i}}\,}  % 虚部
\newcommand{\md}{\mathrm{d}\,}
\newcommand{\bA}{\boldsymbol{A}}
\newcommand{\red}[1]{\textcolor{red}{#1}}

%\renewcommand{\theequation}{\arabic{section}.\arabic{equation}}
\renewcommand{\contentsname}{攻读博士学位科学研究计划书}


\begin{document}

    \frontmatter

    % 生成标题页
    \maketitle

    % \thispagestyle{empty}
    % 需要 pdfpages 宏包
    % \includepdf[pages=-]{pdfname.pdf}


    \clearpage   % 结束上一页
    \pagenumbering{Roman} % 摘要页码为大写罗马数字
	
    % 生成目录(自定义的命令)
    % 使用方法: \maketoc[nopagenum/pagenum/pagenumtoc]
    % 其中: nopagenum指目录没有页码(默认值);pagenum指目录有页码;
    % pagenumtoc指目录有页码, 且目录两字出现在目录中
    % 请注意在合适的位置放置\pagenumbering{numstyle}使用新的页码
    \maketoc[nopagenum]


	\clearpage  % 结束上一页
	\pagenumbering{arabic}  % 正文页码为阿拉伯数字

    \mainmatter

%%%%%%%%%%%%%%%%%%%%%%%%%%%%%% 正文内容从这里开始  %%%%%%%%%%%%%%%%%%%%%%%%%%%%%


\chapter{考生基本情况}

某某某, 男, XXXX 年 XX 月生, 20XX年9月-至今就读于某某大学某某专业, 攻读硕士学位, 导师为某某教授, 研究方向为XXXXX. 硕士期间获得的主要成果如下:


\section{学术论文}

\begin{enumerate}[label=(\arabic*)]
	\item 第一项
	\item 第二项
\end{enumerate}

\section{项目参与情况}

\begin{enumerate}[label=(\arabic*)]
	\item 第一项
	\item 第二项
\end{enumerate}

\section{获奖与荣誉情况}

\begin{enumerate}[label=(\arabic*)]
	\item 第一项
	\item 第二项
\end{enumerate}

\section{学术交流}

\begin{enumerate}[label=(\arabic*)]
	\item 第一项
	\item 第二项
\end{enumerate}

\section{毕业论文题目及主要内容}

这是一大段文字这是一大段文字这是一大段文字这是一大段文字这是一大段文字这是一大段文字这是一大段文字这是一大段文字这是一大段文字这是一大段文字这是一大段文字这是一大段文字这是一大段文字这是一大段文字这是一大段文字这是一大段文字这是一大段文字这是一大段文字这是一大段文字这是一大段文字这是一大段文字这是一大段文字这是一大段文字这是一大段文字这是一大段文字这是一大段文字这是一大段文字这是一大段文字这是一大段文字.




%%%%%%%%%%%%%%%%%%%%%%%%% 攻读博士学位期间拟开展研究的课题  %%%%%%%%%%%%%%%%%%%%%%%%%%%%

\chapter{攻读博士学位期间拟开展研究的课题}

\section{课题名称及摘要}

\textbf{课题名称:} 课题的题目课题的题目课题的题目课题的题目课题的题目

\textbf{摘要:} 课题的摘要课题的摘要课题的摘要课题的摘要课题的摘要课题的摘要课题的摘要课题的摘要课题的摘要.
主要包括:1) 课题的摘要课题的摘要课题的摘要课题的摘要课题的摘要课题的摘要;2) 课题的摘要课题的摘要课题的摘要课题的摘要课题的摘要; 3) 课题的摘要课题的摘要课题的摘要课题的摘要课题的摘要课题的摘要.

\section{课题的研究背景}

这是一大段文字这是一大段文字这是一大段文字这是一大段文字这是一大段文字这是一大段文字这是一大段文字这是一大段文字这是一大段文字这是一大段文字这是一大段文字这是一大段文字这是一大段文字这是一大段文字这是一大段文字这是一大段文字这是一大段文字这是一大段文字这是一大段文字.


\section{研究目的和现实意义}
\subsection{研究的现实意义}

这是一大段文字这是一大段文字这是一大段文字这是一大段文字这是一大段文字这是一大段文字这是一大段文字这是一大段文字这是一大段文字这是一大段文字这是一大段文字这是一大段文字这是一大段文字这是一大段文字这是一大段文字这是一大段文字这是一大段文字这是一大段文字这是一大段文字.



\subsection{研究内容及目的}

这是一大段文字这是一大段文字这是一大段文字这是一大段文字这是一大段文字这是一大段文字这是一大段文字这是一大段文字这是一大段文字这是一大段文字这是一大段文字这是一大段文字这是一大段文字这是一大段文字这是一大段文字这是一大段文字这是一大段文字这是一大段文字这是一大段文字.



\section{国内外研究现状述评} %国内外研究现状述评 (文献综述)


自定义了一个命令 \verb|\red{文字}| 可以\red{加红文字}, 可以在论文修改阶段方便标记.

这是一个引用的示例 \cite{Adams1975}、\cite{LiLiu1997}和 \cite{Shen1994,Tadmor2012,TreWei2014}.

这是一大段文字这是一大段文字中英文混排 Numerical Methods.

数学公式的使用请参考公式手册 symbols-a4, 或者 《一份(不太)简短的 \LaTeX~2$\varepsilon$ 介绍》 (lshort-zh-cn).

自定义命令表示的几个数学符号 $\RR$, $\CC$, $\A$, $\ii$, $\md$, $\bA$.

在文中行内公式可以这么写: $a^2+b^2=c^2$, 这是勾股定理, 它还可以表示为 $c=\sqrt{a^2+b^2}$, 还可以让公式单独一段并且加上编号
\begin{equation}\label{eqn:trifun}
\sin^2{\theta}+\cos^2{\theta}=1.
\end{equation}
还可以通过添加标签在正文中引用公式, 如等式~\eqref{eqn:trifun} 或者 \ref{eqn:trifun}.


\section{研究内容与研究方法}

\subsection{研究内容}

读者可能阅读过其它手册或者资料, 知道 LaTeX 提供了 eqnarray 环境. 它按照等号左边—等号—等号右边呈三列对齐, 但等号周围的空隙过大, 加上公式编号等一些 bug, 目前已不推荐使用. (摘自 lshort-zh-cn)



\section{研究框架及提纲}

\begin{definition}
这是一个定义.
\end{definition}

\begin{lemma}[Lemma] \label{lemma1}
这是一个引理.
\end{lemma}

\begin{theorem}[Theorem]
这是一个定理.
\end{theorem}
\begin{proof}
这是证明环境.
\end{proof}

\begin{remark}\label{remark1}
这是一个 remark.
\end{remark}

\begin{example}
这是一个例子.
\end{example}


\section{研究步骤及进度}

\subsection{研究步骤}

\begin{enumerate}[label=(\arabic*)]
	\item 第一项
	\item 第二项
    \item 第三项
\end{enumerate}


\subsection{研究进度}

\begin{enumerate}
\setlength{\itemsep}{0pt}
  \item [] 第一阶段:
  \item [] 第二阶段:
  \item [] 第三阶段:
  \item [] 第四阶段:
  \item [] 第五阶段:
\end{enumerate}



\section{预期成果}

本研究计划预期研究成果主要体现在以下几个方面:
\begin{enumerate}[label=(\arabic*)]
\setlength{\itemsep}{0pt}
  \item 第一方面;
  \item 第二方面;
  \item 第三方面.
\end{enumerate}




%%%%%%%%%%%%%%%%%%%%%%%%%%%%%%%%%%%%%%%%%%%%%%%%%%%%%%%%%%%%%%%%%%%%%%%%%%

\backmatter  % 结束章节自动编号

%%%%%%%%%%%%%%%%%%%%%%%% 生成参考文献  %%%%%%%%%%%%%%%%%%%%%%%%%%%%%%%%%%%%%

% \nocite{*}  % 可以暂时显示全部参考文献, 包括未引用的

% 使用方法:\bibliography{参考文件1文件名, 参考文献2文件名, ...}
% 参考文献格式可选  plain, abbrv, unsrt, siam
% \bibliographystyle{abbrv}
% \bibliographystyle{thuthesis-numeric}
\bibliographystyle{shnuthesis-numeric}
\bibliography{mybib}


%%%%%%%%%%%%%%%%%%%%%%%% 附录  %%%%%%%%%%%%%%%%%%%%%%%%%%%%%%%%%%%%%%%

% 添加附录, 如不需要可以注释掉
\appendix

%\setcounter{page}{1} % 如果需要可以自行重置页码
\renewcommand{\chaptermark}[1]{\markboth{#1}{}}
\chapter{附录 ~ 这是一个附录}
\section*{1. 第一小节}

这里是附录环境, 手动设置了 chapter 和 section 的样式.

\section*{2. 第二小节}

这里是附录环境, 手动设置了 chapter 和 section 的样式.



\end{document}



% !TEX program = xelatex
% 使用 texlive 完整编译:
% xelatex -> bibtex -> xelatex -> xelatex
% 攻读博士学位科研计划书  LaTeX 模板
\documentclass{phdproposal}
% 进行个人信息设置
\suptitle{XX大学 202X 年攻读博士学位}
\title{科学研究计划}
\author{某~某~某}    % 作者姓名
\date{2\,0\,2\,X~年~1\,2~月}  % 报名时间
\college{学~院~名~称}  % 未用
\major{计~算~数~学}  % 专业名称
\study{微~分~方~程~数~值~解}
\instructor{某~某~某}  % 导师姓名

% 添加自己要用的其他宏包
\usepackage[numbers]{natbib}
\usepackage{subfig}
% \usepackage{xltxtra}

% --- 证明结束黑框 ----
% \renewcommand{\qedsymbol}{$\blacksquare$}

% --- 设置英文字体 -----
\usepackage{newtxtext}  % for text fonts

% --- 设置数学字体 -----
% \usepackage{newtxmath}
% \usepackage{mathptmx}

% --- 直接插入 pdf 文件 ----
% \usepackage{pdfpages}

% --- 自定义命令 -----
\newcommand{\CC}{\ensuremath{\mathbb{C}}}
\newcommand{\RR}{\ensuremath{\mathbb{R}}}
\newcommand{\A}{\mathcal{A}}
\newcommand{\ii}{\bm{\mathrm{i}}\,}  % 虚部
\newcommand{\md}{\mathrm{d}\,}
\newcommand{\bA}{\boldsymbol{A}}
\newcommand{\red}[1]{\textcolor{red}{#1}}

%\renewcommand{\theequation}{\arabic{section}.\arabic{equation}}
\renewcommand{\contentsname}{攻读博士学位科学研究计划书}


\begin{document}

    \frontmatter

    % 生成标题页
    \maketitle

    % \thispagestyle{empty}
    % 需要 pdfpages 宏包
    % \includepdf[pages=-]{pdfname.pdf}


    \clearpage   % 结束上一页
    \pagenumbering{Roman} % 摘要页码为大写罗马数字
	
    % 生成目录(自定义的命令)
    % 使用方法: \maketoc[nopagenum/pagenum/pagenumtoc]
    % 其中: nopagenum指目录没有页码(默认值);pagenum指目录有页码;
    % pagenumtoc指目录有页码, 且目录两字出现在目录中
    % 请注意在合适的位置放置\pagenumbering{numstyle}使用新的页码
    \maketoc[nopagenum]


	\clearpage  % 结束上一页
	\pagenumbering{arabic}  % 正文页码为阿拉伯数字

    \mainmatter

%%%%%%%%%%%%%%%%%%%%%%%%%%%%%% 正文内容从这里开始  %%%%%%%%%%%%%%%%%%%%%%%%%%%%%


\chapter{考生基本情况}

某某某, 男, XXXX 年 XX 月生, 20XX年9月-至今就读于某某大学某某专业, 攻读硕士学位, 导师为某某教授, 研究方向为XXXXX. 硕士期间获得的主要成果如下:


\section{学术论文}

\begin{enumerate}[label=(\arabic*)]
	\item 第一项
	\item 第二项
\end{enumerate}

\section{项目参与情况}

\begin{enumerate}[label=(\arabic*)]
	\item 第一项
	\item 第二项
\end{enumerate}

\section{获奖与荣誉情况}

\begin{enumerate}[label=(\arabic*)]
	\item 第一项
	\item 第二项
\end{enumerate}

\section{学术交流}

\begin{enumerate}[label=(\arabic*)]
	\item 第一项
	\item 第二项
\end{enumerate}

\section{毕业论文题目及主要内容}

这是一大段文字这是一大段文字这是一大段文字这是一大段文字这是一大段文字这是一大段文字这是一大段文字这是一大段文字这是一大段文字这是一大段文字这是一大段文字这是一大段文字这是一大段文字这是一大段文字这是一大段文字这是一大段文字这是一大段文字这是一大段文字这是一大段文字这是一大段文字这是一大段文字这是一大段文字这是一大段文字这是一大段文字这是一大段文字这是一大段文字这是一大段文字这是一大段文字这是一大段文字.




%%%%%%%%%%%%%%%%%%%%%%%%% 攻读博士学位期间拟开展研究的课题  %%%%%%%%%%%%%%%%%%%%%%%%%%%%

\chapter{攻读博士学位期间拟开展研究的课题}

\section{课题名称及摘要}

\textbf{课题名称:} 课题的题目课题的题目课题的题目课题的题目课题的题目

\textbf{摘要:} 课题的摘要课题的摘要课题的摘要课题的摘要课题的摘要课题的摘要课题的摘要课题的摘要课题的摘要.
主要包括:1) 课题的摘要课题的摘要课题的摘要课题的摘要课题的摘要课题的摘要;2) 课题的摘要课题的摘要课题的摘要课题的摘要课题的摘要; 3) 课题的摘要课题的摘要课题的摘要课题的摘要课题的摘要课题的摘要.

\section{课题的研究背景}

这是一大段文字这是一大段文字这是一大段文字这是一大段文字这是一大段文字这是一大段文字这是一大段文字这是一大段文字这是一大段文字这是一大段文字这是一大段文字这是一大段文字这是一大段文字这是一大段文字这是一大段文字这是一大段文字这是一大段文字这是一大段文字这是一大段文字.


\section{研究目的和现实意义}
\subsection{研究的现实意义}

这是一大段文字这是一大段文字这是一大段文字这是一大段文字这是一大段文字这是一大段文字这是一大段文字这是一大段文字这是一大段文字这是一大段文字这是一大段文字这是一大段文字这是一大段文字这是一大段文字这是一大段文字这是一大段文字这是一大段文字这是一大段文字这是一大段文字.



\subsection{研究内容及目的}

这是一大段文字这是一大段文字这是一大段文字这是一大段文字这是一大段文字这是一大段文字这是一大段文字这是一大段文字这是一大段文字这是一大段文字这是一大段文字这是一大段文字这是一大段文字这是一大段文字这是一大段文字这是一大段文字这是一大段文字这是一大段文字这是一大段文字.



\section{国内外研究现状述评} %国内外研究现状述评 (文献综述)


自定义了一个命令 \verb|\red{文字}| 可以\red{加红文字}, 可以在论文修改阶段方便标记.

这是一个引用的示例 \cite{Adams1975}、\cite{LiLiu1997}和 \cite{Shen1994,Tadmor2012,TreWei2014}.

这是一大段文字这是一大段文字中英文混排 Numerical Methods.

数学公式的使用请参考公式手册 symbols-a4, 或者 《一份(不太)简短的 \LaTeX~2$\varepsilon$ 介绍》 (lshort-zh-cn).

自定义命令表示的几个数学符号 $\RR$, $\CC$, $\A$, $\ii$, $\md$, $\bA$.

在文中行内公式可以这么写: $a^2+b^2=c^2$, 这是勾股定理, 它还可以表示为 $c=\sqrt{a^2+b^2}$, 还可以让公式单独一段并且加上编号
\begin{equation}\label{eqn:trifun}
\sin^2{\theta}+\cos^2{\theta}=1.
\end{equation}
还可以通过添加标签在正文中引用公式, 如等式~\eqref{eqn:trifun} 或者 \ref{eqn:trifun}.


\section{研究内容与研究方法}

\subsection{研究内容}

读者可能阅读过其它手册或者资料, 知道 LaTeX 提供了 eqnarray 环境. 它按照等号左边—等号—等号右边呈三列对齐, 但等号周围的空隙过大, 加上公式编号等一些 bug, 目前已不推荐使用. (摘自 lshort-zh-cn)



\section{研究框架及提纲}

\begin{definition}
这是一个定义.
\end{definition}

\begin{lemma}[Lemma] \label{lemma1}
这是一个引理.
\end{lemma}

\begin{theorem}[Theorem]
这是一个定理.
\end{theorem}
\begin{proof}
这是证明环境.
\end{proof}

\begin{remark}\label{remark1}
这是一个 remark.
\end{remark}

\begin{example}
这是一个例子.
\end{example}


\section{研究步骤及进度}

\subsection{研究步骤}

\begin{enumerate}[label=(\arabic*)]
	\item 第一项
	\item 第二项
    \item 第三项
\end{enumerate}


\subsection{研究进度}

\begin{enumerate}
\setlength{\itemsep}{0pt}
  \item [] 第一阶段:
  \item [] 第二阶段:
  \item [] 第三阶段:
  \item [] 第四阶段:
  \item [] 第五阶段:
\end{enumerate}



\section{预期成果}

本研究计划预期研究成果主要体现在以下几个方面:
\begin{enumerate}[label=(\arabic*)]
\setlength{\itemsep}{0pt}
  \item 第一方面;
  \item 第二方面;
  \item 第三方面.
\end{enumerate}




%%%%%%%%%%%%%%%%%%%%%%%%%%%%%%%%%%%%%%%%%%%%%%%%%%%%%%%%%%%%%%%%%%%%%%%%%%

\backmatter  % 结束章节自动编号

%%%%%%%%%%%%%%%%%%%%%%%% 生成参考文献  %%%%%%%%%%%%%%%%%%%%%%%%%%%%%%%%%%%%%

% \nocite{*}  % 可以暂时显示全部参考文献, 包括未引用的

% 使用方法:\bibliography{参考文件1文件名, 参考文献2文件名, ...}
% 参考文献格式可选  plain, abbrv, unsrt, siam
% \bibliographystyle{abbrv}
% \bibliographystyle{thuthesis-numeric}
\bibliographystyle{shnuthesis-numeric}
\bibliography{mybib}


%%%%%%%%%%%%%%%%%%%%%%%% 附录  %%%%%%%%%%%%%%%%%%%%%%%%%%%%%%%%%%%%%%%

% 添加附录, 如不需要可以注释掉
\appendix

%\setcounter{page}{1} % 如果需要可以自行重置页码
\renewcommand{\chaptermark}[1]{\markboth{#1}{}}
\chapter{附录 ~ 这是一个附录}
\section*{1. 第一小节}

这里是附录环境, 手动设置了 chapter 和 section 的样式.

\section*{2. 第二小节}

这里是附录环境, 手动设置了 chapter 和 section 的样式.



\end{document}



