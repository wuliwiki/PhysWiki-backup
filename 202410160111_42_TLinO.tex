% 拓扑线性空间中的线性算子
% keys 线性算子
% license Usr
% type Tutor

\pentry{拓扑向量空间\nref{nod_tvs}}{nod_c6e1}
线性空间中由\enref{线性算子}{LiOper}的定义,拓扑线性空间是线性空间,因此在拓扑线性空间中就有线性算子的定义。此外,拓扑线性空间的拓扑性质使得其上的线性算子可以定义连续性,而算子的像和核有开闭性的讨论。

\begin{definition}{线性算子}
设 $E,E_1$ 是两个(定义在域 $\mathbb F$上)\enref{拓扑线性空间}{tvs},$D_A\subset E,$ 若映射 $A:D_A\rightarrow E_1$ 满足:
\begin{enumerate}
\item \textbf{可加性:}$A(x+y)=A(x)+A(y),\quad x,y\in E$;
\item \textbf{齐次性:}$A(\alpha x)=\alpha A(x),\quad x\in E,\alpha\in \mathbb F$
\end{enumerate}
则称 $A$ 是 $E$ 到 $E_1$ 的\textbf{线性算子}。
  
\end{definition}

\textbf{注意:}一般地,不能假定 $D_A=E$,然而总可以假定 $D_A$ 是线性流形,即 $x,y\in D_A$,则 $\alpha x+\beta y\in D_A$ 对任意 $\alpha,\beta\in\mathbb F$ 恒成立(因为总可以通过 $A(\alpha x+\beta y)=\alpha A(x)+\beta A(y)$定义 $A$ 在 $\alpha x+\beta y$ 的值)。

\begin{definition}{连续}
设 $A:D_A\rightarrow E_1$ 是 $E$ 到 $E_1$ 的线性算子,若对 $y_0=Ax_0$ ($x_0\in D_A$) 的任意邻域 $V$,存在 $x_0$ 的邻域 $U$,使得 $A(U\cap D_A)\subset V$,则称 $A$ 在 $x_0$ 是\textbf{连续的}。若 $A$ 在 $D_A$ 上处处连续,则称 $A$ \textbf{连续}。
\end{definition}
