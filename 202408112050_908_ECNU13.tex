% 华东师范大学 2013 年 考研 量子力学
% license Usr
% type Note

\textbf{声明}:“该内容来源于网络公开资料,不保证真实性,如有侵权请联系管理员”

\subsection{简答题(每小题7分,共56分)}
\begin{enumerate}
\item 为什么波函数$\psi$,必定是复函数?
\item 试叙述量子力学基本假设一测量共设的要点。
\item $\psi(x,t)$ 和 $e^{i\theta}\psi(\overline x,t)$ 是否代表同一个量子态?并说明为什么,其$\phi(\overline z)$ 是实函数。
\item 力学量之间的对易关系是否具有传递性?即:如果$A$与$B$对易,$B$与$C$对易成立,是否必有$A$与$C$对易成立?试用举例来证明你的结论。
\item 两个有限深方势阱深度相同,但宽度不同,与窄的势阱哪一个束缚态的个数多?为什么?
现有三种系统,其能级与其量子数$n$,的关系分别是正比于$n^2,n^{-2}$以及与$n$满足线性关系,请举例指出它们对应的分别可能是什么系统?
\item 什么是反常Zeeman效应?产生该效应的根源是什么?
\item 证明 $\text{exp}(i\theta \vec{\sigma}) = \cos \theta + i \sin \theta (\vec{n} \cdot \vec{\sigma})$,其中 $\vec{\sigma} = (\sigma_x, \sigma_y, \sigma_z)$ 表示 Pauli 矩阵,$\vec{n}$ 为单位向量。
\end{enumerate}
\subsection{(本题16 分)}
已知某微观体系的力学量 $A$ 有两个归一化本征态 $\psi_1$ 和 $\psi_2$,相应的本征值为 $a_1$ 和 $a_2$。力学量 $B$ 也有两个归一化本征态 $\phi_1$ 和 $\phi_2$,相应的本征值为 $b_1$ 和$b_2$。两种本征态之间存在如下关系:
$$\psi_1 = \frac{3\phi_1 + 4i\phi_2}{5}, \quad \psi_2 = \frac{4\phi_1 - 3i\phi_2}{5}~$$
当对某个态测量 $A$ 后得到 $a_1$,然后再测量 $B$,接着再测量 $A$,试求第二次测量 $A$ 仍然得到 $a_1$ 的几率。
\subsection{(本题 18 分)}
一根长为$l$的无质量的绳子一端固定,另一端系质点$m$。在重力作用下,质点在整直平面内摆动
\begin{enumerate}
\item 写出质点运动的量子体系哈密顿量
\item 在小角近似下求系统的能级。
\end{enumerate}
\subsection{【本题20分)}
一个一维谐振子圆频率为$\omega$,处在相干态$|z\rangle$上,其中$|z\rangle$是该谐振子湮灭算符的本征态,
$$a |z\rangle = z |z\rangle~$$
$z$是复数,试求出该谐振子的以下两个量
\begin{enumerate}
\item 平均能量
\item 能量均方差
\end{enumerate}
\subsection{(本题 20分)}