% 多变量分布函数
% 概率分布|统计|分布函数|概率

% 未完成: 名字改为多维概率分布函数

\pentry{概率分布函数\upref{RandF}, 重积分\upref{IntN}}

\subsection{二维分布函数}
我们以打靶为例来引入二维分布函数, 以及在直角坐标系和极坐标系中如何分析各种平均值. 我们在靶上建立 $x$-$y$ 直角坐标系, 类比一维概率分布函数的定义, 二维分布可以用一个二元函数表示为 $f(x, y)$. 子弹落在一个长方形区域内(令 $x \in [x_1, x_2]$, $y \in [y_1, y_2]$) 的概率可以用二重积分表示
\begin{equation}
P = \int_{y_1}^{y_2} \int_{x_1}^{x_2} f(x, y) \dd{x} \dd{y}
\end{equation}
从几何上理解, $f(x, y)$ 可以看作一张三维空间中的曲面, 而这个二重积分则表示曲面和 $x$-$y$ 平面之间的一个体积(\autoref{MulPdf_fig1}).

\begin{figure}[ht]
\centering
\includegraphics[width=7cm]{./figures/MulPdf_1.png}
\caption{请添加图片描述} \label{MulPdf_fig1}
\end{figure}

类比一维情况, 概率归一化条件为
\begin{equation}
\int_{-\infty}^{+\infty} \int_{-\infty}^{+\infty} f(x, y) \dd{x} \dd{y} = 1
\end{equation}
由于概率的量纲是 1, $f(x, y)$ 的国际单位量纲就是 $\Si{m^{-2}}$, 即面积的倒数.

位置矢量的 $x$ 和 $y$ 分量平均值分别为
\begin{equation}\label{MulPdf_eq2}
\ev{x} = \iint x f(x, y) \dd{x}\dd{y}
\end{equation}
\begin{equation}
\ev{y} = \iint y f(x, y) \dd{x}\dd{y}
\end{equation}

位置矢量模长的平均值
\begin{equation}
\ev{r} = \iint \sqrt{x^2 + y^2} f(x, y) \dd{x}\dd{y}
\end{equation}

位置矢量模长平方的平均值
\begin{equation}
\ev{r^2} = \iint (x^2 + y^2) f(x, y) \dd{x}\dd{y}
\end{equation}

当射击次数 $N \to \infty$ 时, 这些量也可以表示为求和的形式, 例如
\begin{equation}
\ev{r^2} = \frac{1}{N}\sum_{i=1}^N (x_i^2 + y_i^2)
\end{equation}

% 习题未完成

\subsection{概率分布的极坐标表示}
\pentry{极坐标系\upref{Polar}}

除了直角坐标系外, 我们也可以使用极坐标\upref{Polar} 表示分布函数, 记为\footnote{在数学上, 该式两边是两个不同的二元函数, 应该使用不同的函数名如 $g(r, \theta) = f(x, y)$, 但在物理上, 我们有时为了方便不以区分}
\begin{equation}
f(r, \theta) \equiv f(x, y)
\end{equation}
那么归一化条件变为
\begin{equation}\label{MulPdf_eq1}
\int_0^\infty \int_{0}^{2\pi} f(r, \theta)\,\, r\dd{\theta}\dd{r} = 1
\end{equation}
以上的平均值也容易变为极坐标中的积分, 例如
\begin{equation}\label{MulPdf_eq3}
\ev{r^2} = \int_0^\infty \int_{0}^{2\pi} r^2 f(r, \theta)\,\, r\dd{\theta}\dd{r}
= \int_0^\infty \int_{0}^{2\pi} r^3 f(r, \theta) \dd{\theta}\dd{r}
\end{equation}

% 习题未完成

\subsection{圆对称分布}
继续打靶的例子, 我们接下来讨论圆对称的概率分布(在三维情况下叫做球对称), 即 $f(r, \theta)$ 的值只和 $r$ 有关而与 $\theta$ 无关. 为了书写方便可记 $f(r) \equiv f(r, \theta)$. 归一化条件\autoref{MulPdf_eq1} 化简为(先完成关于 $\theta$ 积分)
\begin{equation}
2\pi \int_0^\infty r f(r) \dd{r} = 1
\end{equation}

一个容易混淆的概念是, $f(r)$ 并不是变量 $r$ 的概率分布函数. 也就是说 $\int_a^b f(r) \dd{r}$ 并不是子弹落在圆环 $r \in [a, b]$ 内的概率. 若将 $r$ 的概率分布函数后者记为 $F(r)$, 这个概率应该是
\begin{equation}\label{MulPdf_eq4}
P_{ab} = \int_a^b F(r) \dd{r} = \int_a^b 2\pi r f(r) \dd{r}
\end{equation}
由于这个定积分在任何区间都相等, 两个被积函数必须相等, 即
\begin{equation}\label{MulPdf_eq5}
F(r) = 2\pi r f(r)
\end{equation}

这时以上的平均值也可以得到进一步化简, 例如\autoref{MulPdf_eq3} 变为
\begin{equation}
\ev{r^2} = \int_0^\infty r^2 F(r) \dd{r}
= 2\pi \int_0^\infty r^3 f(r) \dd{r}
\end{equation}
