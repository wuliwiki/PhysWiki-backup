% 字符串的分片与索引(Python)
% keys 字符串|分片|索引

\pentry{Python 基本变量类型\upref{PyType}}
\subsection{分片}
\begin{definition}{分片}

\textbf{字符串}可以通过string[x] 的方式进行索引、分片,也就是加一个[](像不像一把刀).字符串的分片(slice)实际上可以看作是从字符串这个大面包\autoref{Strsi_fig1} 中找出来你要吃的那片美味(复制出来一小段你要的长度),放在你的嘴里(储存在另一个地方),而不会对字符串这个源文件改动.分片获得的每个字符串可以看作是原字符串的一个副本(面包片).
\begin{figure}[ht]
\centering
\includegraphics[width=10cm]{./figures/Strsi_1.png}
\caption{切片} \label{Strsi_fig1}
\end{figure}

\end{definition}

我们来看一段程序:
\begin{lstlisting}[language=python]
name = 'My name is Mike'
print(name[0])
'M'
print(name[-4])
'M'
print(name[11:14]) # 从第一个到第十四个,第十四个不包括在内
'Mik'
print(name[11:15]) # 从第一个到第十五个,第十五个不包括在内
'Mike'
print(name[5:])  #代表着从编号为5的字符到结束的字符串分片.
'me is Mike'
print(name[:5]) #从编号为0的字符开始到编号为5但不包含第5个字符
'My na'
\end{lstlisting}
\textbf{\textsl{:}}两边分别代表着字符串的分割从哪里开始,并到哪里结束.我们不妨列个表格\autoref{Strsi_tab1} 来说明字符的对应关系
\begin{table}[ht]
\centering
\caption{对应关系}\label{Strsi_tab1}
\begin{tabular}{|c|c|c|c|c|c|c|c|c|c|c|c|c|c|c|c|}
\hline
 字符  & M & y &   & n & a & m & e &   & i & s &   & M & i & k &e\\
\hline
 序号  & 0 & 1 & 2 & 3 & 4 & 5 & 6 & 7 & 8 & 9 & 10 & 11 & 12 & 13 & 14\\
\hline
 反序  & -15 & -14 & -13 & -12 & -11 & -10 & -9 & -8 & -7 & -6 & -5 & -4 & -3 & -2 & -1\\
\hline
\end{tabular}
\end{table}
\begin{example}{”朋友中的魔鬼“}
这个文字小游戏代码如下:
\begin{lstlisting}[language=python]
word = 'friends'
find_the_evil_in_your_friends = word[0]+word[2:4]+word[-3:-1] 
print(find_the_evil_in_your_friends)
\end{lstlisting}
没有问题的话,你会得到结果:
\begin{lstlisting}[language=python]
fiend
\end{lstlisting}
也就发现了朋友中的魔鬼.你get到了吗?
\end{example}
\begin{example}{密码遮盖}
我们来看这样一个场景:很多时候我们使用手机号在网站注册时,为了保证用户的信息安全性,通常账户信息只会显示后四位,其余的用\textbf{“*”}来代替,让我们利用所学来试试吧!
\begin{lstlisting}[language=python]
p_number = '13755878256'
hide_number = phone_number.replace(phone_number[:9],'*' * 9)
print(hide_number)
\end{lstlisting}
其中我们使用了一个新的字符串方法\textbf{\textsl{replace()}}进行“遮挡”.replace方法的括号中,第一个 \verb|p_ number [: 9]| 代表要被替换掉的部分,后面的‘* ' *9表示将要替换的字符,然后把*乘以9,显示9个*.你会得到这样的结果: ********8526\footnote{此处电话随手打的,大家不要有任何的好奇心}
你懂了吗?

\end{example}

\subsection{索引}
索引小胖猫可以快速帮你找出你爱吃的面包哟,如下代码我们来模拟手机通讯簿中的电话号码联想功能(简陋的模拟)

\begin{lstlisting}[language=python]
s = '132'
na = '1386-132-0706'
nb = '1321-282-5046'
print(s+' is at '+ str(na.find(s))+' to '+ str(na.find(s)+len(s))+' of na')
print(s+' is at '+ str(nb.find(s))+' to '+ str(nb.find(s)+len(s))+' of nb')
\end{lstlisting}
你会得到结果:

\textsl{\textbf{132 is at 5 to 8 of na}}

\textsl{\textbf{132 is at 0 to 3 of nb}}
\subsection{附注:}
\begin{figure}[ht]
\centering
\includegraphics[width=10cm]{./figures/Strsi_2.png}
\caption{填空} \label{Strsi_fig2}
\end{figure}
这样的填空题我们屡见不鲜,当字符串中有多个这样的“空”需要填写的时候,我们可以使用.format()进行批处理,它的基本使用方法有如下几种,输入代码:
\begin{lstlisting}[language=python]
print('{} word is what she needs. '.format('This'))
print('{} word is {} she needs. '.format('This','what'))
print('{p} word is what she needs. '.format('This'))
print('{jih} word is {wul} she needs. '.format('This','what'))
print('{0} word is {1} she needs. '.format('This','what'))
print('{1} word is {0} she needs. '.format('what','This'))
\end{lstlisting}
所有的结果都为:

\textsl{\textbf{This word is what she needs.}}

好了,到这里你就掌握了变量和字符串的基本概念和常用方法.加油!
