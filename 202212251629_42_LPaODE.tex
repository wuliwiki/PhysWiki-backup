% 一阶线性偏微分方程与常微分方程组的等价性
% keys 线性偏微分方程
\pentry{一般积分\upref{IntGen}}
本词条将说明,一阶线性偏微分方程与常微分方程组具有直接的联系:一阶线性偏微分方程求解问题可以化为常微分方程组的求解问题。这也是对应常微分方程组被称为一阶线性偏微分方程的\textbf{特征方程组}的原因。
\subsection{将常微分方程组写成更为对称的形式}
一般的常微分方程组都可写为下面的形式(\autoref{GO2SOD_the2}~\upref{GO2SOD})
\begin{equation}
\dv{y_i}{x}=f_i(x,y_1,\cdots,y_n),\quad i=1,\cdots,n
\end{equation}
这可写为下面等价的形式
\begin{equation}
\dd x=\frac{\dd y_1}{f_1(x,y_1,\cdots,y_n)}=\cdots=\frac{\dd y_n}{f_n(x,y_1,\cdots,y_n)}
\end{equation}
为使第一项的分母不为1,可把该式所有的分母都乘上共同的因子。并且为对称起见,将 $x$ 记为 $x_1$,$y_i$ 记为 $x_{i+1}$,上式可写为更具对称性的等价形式:
\begin{equation}\label{LPaODE_eq1}
\frac{\dd x_1}{X_1}=\cdots=\frac{\dd x_{n+1}}{X_{n+1}}
\end{equation}
其中,$X_i$ 是变量 $x_1,\cdots,x_{n+1}$ 的函数。

当常微分方程组写成\autoref{LPaODE_eq1} 时,可看出这 $n+1$ 个变量是等价的,并没有特定哪个变量作为自变量。在新的记号下,方程组的积分是(\autoref{IntGen_def1}~\upref{IntGen})
\begin{equation}
\varphi_i(x_1,\cdots,x_{n+1})=C_i,\quad i=1,\cdots,n
\end{equation}
