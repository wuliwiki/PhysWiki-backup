% 隐函数
% 隐函数|显函数|长方体

\subsection{一元隐函数}
假定二变元 $x$ 及 $y$ 的值用方程联系着,若把这方程的一切项移到左边,则得到一般的形式
\begin{equation}\label{ImpFun_eq1}
F(x,y)=0
\end{equation}
此处 $F(x,y)$ 是在某一区域给定的二元函数.

若函数 $y=f(x)$ 由\autoref{ImpFun_eq1} 给定,但未解出,则称其为\textbf{隐函数};若 $y$ 对 $x$ 的关系 $y=f(x)$ 被(直接)解出来,就称为\textbf{显函数}.
\subsection{多元隐函数}
同一元隐函数一样,若多个变元的方程
\begin{equation}
F(x_1,\cdots,x_n,y)=0
\end{equation}
确定 $y$ 为 $n$ 个变元 $x_1,\cdots,x_n$ 的函数 $y=f(x_1,\cdots,x_n)$,若未解出具体的表达式,则 $y=f(x_1,\cdots,x_n)$ 称为\textbf{$n$ 元隐函数}. 