% 黑体辐射定律
% keys 黑体辐射|普朗克|电磁波

\begin{issues}
\issueDraft
\issueNeedCite
\end{issues}

在非绝对零度的温度下,任何物体都能辐射出电磁波(热辐射),同时也能吸收外来电磁波.假想一种黑体,它能 $100\%$ 地吸收所有辐射在其上的电磁波.并且为了能够达到热平衡,黑体也不断地辐射出能量.为了研究黑体辐射,我们希望能测定温度 $T$ 下黑体辐射的\textbf{能量密度}\footnote{单位体积单位频率间隔内黑体辐射的能量.} $S_\nu(\nu,T)$ 与辐射频率 $\nu$ 和温度 $T$ 之间的关系.

\subsection{维恩定律与瑞利金斯公式}
Wien(1894)从经典统计出发总结黑体辐射经验规律,得到了黑体辐射能量密度的公式:
\begin{equation}
S_\nu(\nu,T)\dd \nu=C_1 \frac{\nu^3}{c^3}e^{-C_2\nu/T}\dd \nu
\end{equation}
其中 $c$ 是真空中的光速,$C_1,C_2$ 是经验常数.该公式只在高频区适用.

瑞利(1900)和金斯(1905)则将空腔中的辐射场视为电磁驻波振子的集合,利用 Maxwell-Boltzmann 分布律与能量连续分布的观念导出
\begin{equation}
S_\nu(\nu,T)\dd \nu=\frac{8\pi\nu^2}{c^3}kT\dd \nu
\end{equation}


\subsection{黑体辐射定律}
在上述两个经验规律的基础上,普朗克(1900)提出了黑体辐射定律:
\begin{equation}
S_\nu(\nu,T) = \frac{8\pi h}{c^3}\frac{\nu^3}{\E^{h\nu/(k_B T)} - 1}
\end{equation}
如果要计算波长的分布, 根据随机变量的变换\upref{RandCV}, 由 $\abs{S_\lambda(\lambda) \dd{\lambda}} = \abs{S_\nu(\nu)\dd{\nu}}$ 得
\begin{equation}\label{BBdLaw_eq1}
S_\lambda(\lambda,T) = \frac{c}{\lambda^2}S_\nu\qty(\frac{c}{\lambda}) =
\frac{8\pi ch}{\lambda^5} \frac{1}{\E^{hc/(k_B T\lambda)} - 1}
\end{equation}

单位面积单位频率单位立体角\upref{SolAng}的功率
\begin{equation}\label{BBdLaw_eq2}
B(\nu) = \frac{c}{4\pi}S_\nu(\nu) = \frac{2h}{c^2} \frac{\nu^3}{\E^{h\nu/(k_B T)} - 1}
\end{equation}
在黑体内部, 辐射是各向同性的, 但在黑体表面, 对于给定的一个平面微元, $B(\nu)$ 是垂直于平面的值, 与法向量夹角为 $\theta$ 的方向的值为 $B(\nu)\cos\theta$.

每个能级($n\omega\hbar$)的平均能量
\begin{equation}
E(\nu) = \frac{h\nu}{\E^{h\nu/(k_B T)} - 1}
\end{equation}

态密度
\begin{equation}
\rho(\nu) = \frac{8\pi}{c^3}\nu^2
\end{equation}

\subsection{推导}
\pentry{盒中的电磁波\upref{EBBox}}
\addTODO{..}
