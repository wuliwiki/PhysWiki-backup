% 角动量守恒

	\subsection{cuFFT}
    
		Bei der cuFFT Library handelt es sich um die state-of-the-art Implementierung einer GPU parallelisierten FFT. Sie wurde im Wesentlichen von der \textit{Fastest Fourier Transformation in the West} (FFTW) inspiriert, unterscheidet sich aber in einem wichtigen Punkt (siehe \ref{fftw}). Zur Benutzung muss der Header \textit{cufft.h} (bzw. \textit{cufftXt.h} für Xt Funktionalität) inkludiert und mit \textit{libcufft} gelinkt werden.
		
		cuFFT unterstützt FFTs in bis zu drei Dimensionen von Komplex nach Komplex (CUFFT{\_}C2C), Real nach Komplex (CUFFT{\_}R2C) und Komplex nach Real (CUFFT{\_}C2R). R und C bezeichnen dabei einfache Präzision. Für die doppelte verwendet man D und Z.
	
		Zunächst müssen Arrays eines speziellen cuFFT Datentyps (\li`cufftComplex`. \li`cufftReal`, \li`cufftDouble`,\\
		\li`cufftDoubleComplex`) erstellt und wie gewohnt in den \gls{global Memory} kopiert werden:
		
		\begin{lstlisting}[caption=cuFFT: komplexer Datentyp]
		int dim = {Nx, Ny, Nz};
		cufftComplex *data;
		cudaMalloc(&data, Nx*Ny*Nz * sizeof(cufftComplex));
		\end{lstlisting}
		
		Ein Wert lässt sich dann wie in einem multidimensionalen Array von zweikomponentigen Datenstrukturen setzen, z.B. \li`data[x][y][z].x = 5.0f; data[x][y][z].y = 3.0f;` für $5 + i\cdot 3$.
	
		Im nächsten Schritt muss ein \Gls{Handle} für das Programm, ein sogenannter Plan erstellt werden. Dies ist notwendig, da abhängig von der Beschaffenheit der Daten vorab bereits verschiedene Berechnungen durchgeführt werden müssen, z.B. die Größe der \Glspl{Block} oder welcher Algorithmus benutzt werden soll. 
		
		\begin{lstlisting}[caption=cuFFT: Pläne]
		cufftHandle plan;
		/////////in einer Dimension /////////
		cufftPlan1d(&plan, Nx, CUFFT_C2C, 1);	
		/////////in zwei Dimensionen/////////
		cufftPlan2d(&plan, Nx, Ny, CUFFT_C2C, 1);	
						
		/////////in drei Dimensionen/////////
		cufftPlan3d(&plan, Nx, Ny, Nz, CUFFT_C2C, 1);
		\end{lstlisting}
		
		Der letzte Wert bezeichnet die Batch Größe (später mehr).
	    
		Dieser Plan muss nun ausgeführt und evtl. das Ergebnis zur Ausgabe zurück kopiert werden.
		
		\begin{lstlisting}[caption=cuFFT: Ausführen]
		cufftExecC2C(plan, data, data, CUFFT_FORWARD);
		cufftDestroy(plan);
		cudaFree(data);
		\end{lstlisting}
		
		Input und Output Array sind hier identisch, es handelt sich also um eine in-place Transformation. Das letzte Keyword bezeichnet die Art der Transformation. CUFFT\_ INVERSE bezeichnet die inverse Transformation. Da Input und Output Array nicht zwingend vom selben Datentyp sein müssen, können ihre Größen abweichen. Tabelle \ref{tab6:size} zeigt einen Vergleich.
		
		\begin{table}[h]
		    \centering
		    \begin{tabular}{llll}
			    \toprule
        			Dimension & FFT & Inputgröße & Outputgröße \\ \midrule
        		    	1D & C2C & $N_x$ \li`cufftComplex` & $N_x$ \li`cufftComplex` \\
			    1D & C2R & $\left \lfloor{\frac{N_x}{2}}\right \rfloor + 1$ \li`cufftComplex` & $N_x$ \li`cufftReal` \\
		        	1D & R2C & $N_x$ \li`cufftReal` & $\left \lfloor{\frac{N_x}{2}}\right \rfloor + 1$ \li`cufftComplex` \\ \hline
        			2D & C2C & $N_x\cdot N_y$ \li`cufftComplex` & $N_x\cdot N_y$ \li`cufftComplex` \\
		        	2D & C2R & $N_x\left \lfloor{\frac{N_y}{2}}\right \rfloor + 1$ \li`cufftComplex` & $N_x\cdot N_y$ \li`cufftReal` \\
        			2D & R2C & $N_x\cdot N_y$ \li`cufftReal` & $N_x\left \lfloor{\frac{N_y}{2}}\right \rfloor + 1$ \li`cufftComplex` \\ \hline
		        	3D & C2C & $N_x\cdot N_y\cdot N_z$ \li`cufftComplex` & $N_x\cdot N_y\cdot N_z$ \li`cufftComplex` \\
        			3D & C2R & $N_x\cdot N_y\left \lfloor{\frac{N_z}{2}}\right \rfloor + 1$ \li`cufftComplex` & $N_x\cdot N_y\cdot N_z$ \li`cufftReal` \\
		        	3D & R2C & $N_x\cdot N_y\cdot N_z$ \li`cufftReal` & $N_x\cdot N_y\left \lfloor{\frac{N_z}{2}}\right \rfloor + 1$ \li`cufftComplex` \\ \bottomrule
        		\end{tabular}
		    \caption[cuFFT Arraygrößen]{Größenvergleich von Input und Output Arrays}
		    \label{tab6:size}
		\end{table}
	
		Nvidia stellt auch hier eine Dokumentation im gewohnten Format zur Verfügung. \autocite{cufftDoc}
	
		 	\subsubsection{cuFFT v.s. FFTW}\label{fftw}
		 	Die \textit{Fastest Fourier Transformation in the West} (FFTW) hat sich wegen ihrer freien Zugänglichkeit und ihrer effizienten Parallelisierung in OpenMP und MPI für Prozessoren als de-facto Standard durchgesetzt. Sie unterscheidet sich in einem wesentlichen Punkt gegenüber cuFFT: FFTW stellt viele Pläne zur Verfügung und führt sie auf eine Art aus, cuFFT stellt nur drei Pläne zur Verfügung und benutzt mehrere Methoden zur Ausführung \autocite{FFTW05}. Tabelle \ref{tab6:fftw} zeigt einige Unterschiede.
		 	\begin{table}[h]
		 	\centering
		 	\begin{tabular}{ll}
		 		\toprule
		 		FFTW & cuFFT \\ \midrule
		 		\li`fftw_plan_dft_1d()`, \li`fftw_plan_dft_r2c_1d()`, \li`fftw_plan_dft_c2r_1d()` & \li`cufftPlan1d()` \\
		 		\li`fftw_plan_dft_2d()`, \li`fftw_plan_dft_r2c_2d()`, \li`fftw_plan_dft_c2r_2d()` & \li`cufftPlan2d()` \\
		 		\li`fftw_plan_dft_3d()`, \li`fftw_plan_dft_r2c_3d()`, \li`fftw_plan_dft_c2r_3d()` & \li`cufftPlan3d()` \\
		 		\li`fftw_plan_dft()`, \li`fftw_plan_dft_r2c()`, \li`fftw_plan_dft_c2r()` & \li`cufftPlanMany()` \\
		 		\li`fftw_plan_many_dft()`, \li`fftw_plan_many_dft_r2c()`, \li`fftw_plan_many_dft_c2r()` & \li`cufftPlanMany()` \\
		 		\li`fftw_execute()` & \multirowcell{6}{\li`cufftExecC2C()`\\ \li`cufftExecZ2Z()`\\ \li`cufftExecR2C()`\\ \li`cufftExecD2Z()`\\ \li`cufftExecC2R()`\\ \li`cufftExecZ2D()`} \\
		 		& \\
		 		& \\
		 		& \\
		 		& \\
		 		& \\
		 		\li`fftw_destroy_plan()` & \li`cufftDestroy()` \\ \bottomrule
		 	\end{tabular}
		 	\caption[cuFFT v.s. FFTW]{Unterschiede zwischen FFTW und cuFFT}
		 	\label{tab6:fftw}
		 	\end{table}
		 	
		 	Eine eindimensionale Transformation würde mit FFTW in der OpenMP-parallelen Variante so aussehen:
		 	\begin{lstlisting}[caption=FFTW Beispiel] 
		 	fftw_complex *in = (fftw_complex*) fftw_malloc(sizeof(fftw_complex)*N);
		 	//Komplexen Vektor belegen...
			fftw_init_threads();
			fftw_plan_with_nthreads(omp_get_max_threads());	
			fftw_plan p = fftw_plan_dft_1d(N, in, in, FFTW_FORWARD, FFTW_ESTIMATE); 
			fftw_execute(p); 
			
			fftw_destroy_plan(p);
			fftw_free(in);
			fftw_cleanup_threads();	 	
		 	\end{lstlisting}
		 	
			Abbildung \ref{fig6:fft} zeigt einen Laufzeitvergleich.
		
			Mehr Informationen lassen sich in der Dokumentation nachlesen. \autocite{fftwDoc}
			
			\subsubsection{cuFFTW}
			Um eine Umstellung zu erleichtern, wurde ein \Gls{API} programmiert, das manche FFTW Funktionen auf cuFFT abbildet. Dazu muss lediglich der Header gegen \textit{cufftw.h} ausgetauscht und mit \textit{libcufftw} gelinkt werden. Kapitel 7 in der Dokumentation zeigt alle implementierten Funktionen. \autocite{cufftDoc}		
			
		\subsection{clFFT}
		Es existiert eine Open-Source Implementierung der FFT in OpenCL von AMD. Sie wurde zwar für AMD GPUs optimiert lässt sich aber prinzipiell auf allen benutzen. Diese steht unter \url{https://github.com/clMathLibraries/clFFT} zur Verfügung und muss explizit nach Anleitung kompiliert werden. Zur Benutzung muss der Header \textit{clfft.h} inkludiert und mit \textit{libclFFT} (und wie gewohnt mit \textit{libOpenCL}) gelinkt werden. Eine eindimensionale Transformation würde mit clFFT so aussehen:
		
		\begin{lstlisting}[caption=clFFT Beispiel]
		cl_int err, N = ...;
		float *X = (float*)malloc(N*2 * sizeof(*X));
		//Komplexen Vektor belegen...
		
		cl_context ctx = ...;
		cl_command_queue queue = ...;
		
		clfftSetupData fftSetup;
		clfftInitSetupData(&fftSetup);
		clfftSetup(&fftSetup);

		cl_mem bufX = clCreateBuffer(ctx, CL_MEM_READ_WRITE, 
			N*2 * sizeof(*X), NULL, &err);
		clEnqueueWriteBuffer(queue, bufX, CL_TRUE, 0, 
			N*2 * sizeof(*X), X, 0, NULL, NULL);
		
		clfftDim dim = CLFFT_1D;
		size_t clLengths[1] = {N};
		clfftPlanHandle planHandle;
		clfftCreateDefaultPlan(&planHandle, ctx, dim, clLengths);

		clfftSetPlanPrecision(planHandle, CLFFT_SINGLE);
		clfftSetLayout(planHandle, CLFFT_COMPLEX_INTERLEAVED, 
            CLFFT_COMPLEX_INTERLEAVED);
		clfftSetResultLocation(planHandle, CLFFT_INPLACE);

		clfftBakePlan(planHandle, 1, &queue, NULL, NULL);
		clfftEnqueueTransform(planHandle, CLFFT_FORWARD, 1, 
			&queue, 0, NULL, NULL, &bufX, NULL, NULL);
		
		.........
		
		clfftDestroyPlan(&planHandle);
		clfftTeardown();
		\end{lstlisting}		
		
		Abbildung \ref{fig6:fft} zeigt einen Laufzeitvergleich.	
					
		\begin{figure}[h]
  			\centering
			\begin{tikzpicture}
    			    \begin{loglogaxis}[xlabel={vector size $n$}, ylabel={computation time / msec.}, legend pos=outer north east, legend style={cells={align=left}}]
      		        \addplot [draw=UR@color@12!50!black, mark=*, only marks, mark options={scale=1}, fill=UR@color@12]   table[x index=0, y index=1]{chapter6/plots/fft.dat};
      		        \addlegendentry{OpenMP FFTW}
%
    	  		        \addplot [draw=gray!50!black, mark=*, only marks, mark options={scale=1}, fill=gray!50!white] table[x index=0, y index=2]{chapter6/plots/fft.dat};
    	  		        \addlegendentry{clFFT}
%
      		        \addplot [draw=UR@color@12!30!black, mark=*, only marks, mark options={scale=1}, fill=UR@color@12!30!white] table[x index=0, y index=3]{chapter6/plots/fft.dat};
      		        \addlegendentry{cuFFT}  
		   	    \end{loglogaxis}
            \end{tikzpicture}
  			\caption[Vergleich von FFTs]{Laufzeitvergleich von \textit{Complex-to-Complex} FFTs desselben $n$-dimensionalen Vektors (log-log). Zum Einsatz kam eine \textit{Nvidia GTX 1060} sowie ein Intel i7-8700K 6$\times$4.8GHz.}
  			\label{fig6:fft}
		\end{figure}
		
		Mehr Informationen zum \Gls{API} lassen sich in der Dokumentation nachlesen. \autocite{clfftDoc}