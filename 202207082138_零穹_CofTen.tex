% 张量的坐标
% 张量坐标|坐标转换关系

\pentry{张量积\upref{TsrPrd}}
进行张量分析往往需要选择空间的基底,并用坐标去刻画张量.在矢量空间 $V$ 和 $V^*$ 中选择相互对偶的基底(\autoref{DualSp_sub1}~\upref{DualSp})
\begin{equation}
V=\langle e_1,\cdots ,e_n\rangle,\quad V^*=\langle e^1,\cdots,e^n\rangle
\end{equation}
这里,按照惯例,空间 $V$ 中的基底指标排列在下方,$V^*$ 中则在上方.而在对应的坐标中,指标的排列则是对立的,即若 $x\in V,f\in V^*$ ,则 $x=\sum_{i}x^i e_i,f=\sum_{i}f_ie^i$

