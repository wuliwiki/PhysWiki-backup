% 判别曲线法求一阶隐式常微分方程的奇解
% keys 判别曲线|ODE|隐式常微分方程|奇解|包络
% license Usr
% type Wiki

\begin{issues}
\issueDraft
\end{issues}

\pentry{包络和奇解\upref{EnvSol},一阶隐式常微分方程的存在唯一性定理\upref{ODEa5}}
一般的一阶隐式常微分方程,往往可能会出现对于一阶隐式常微分方程的存在唯一性定理\upref{ODEa5}的判定中,条件 $(3)$ 的不满足。也就是可能出现 $F(x,y,y'), F'_y(x,y,y'), F'_{y'}(x,y,y')$ 连续,但在某处 $F(x,y,y')=0$ 的情况。在这一点处解的唯一性\textbf{可能}不成立,从而 $F(x,y,y')=0$ 可能有奇解产生。

由奇解性质可以知道,奇解就是由通解构成的曲线族的包络线,由包络线的求法引出了求一阶隐式常微分方程的奇解的以下两种求法。
\subsection{$p$-判别曲线法}\label{sub_ODEa6_1}
$p$-判别曲线法的思路是直接求包络线,再检验包络线是否是原方程的解。

若方程 $F(x,y,y')=0$ 有奇解,则这奇解必定满足两方程:
\begin{equation}\label{eq_ODEa6_1}
F(x,y,y')=0, F'_{y'}(x,y,y')=0 ~.
\end{equation}

\subsection{$c$-判别曲线法}\label{sub_ODEa6_2}
$c$-判别曲线法的思路是先求出通解的曲线族,再求这曲线族的包络。