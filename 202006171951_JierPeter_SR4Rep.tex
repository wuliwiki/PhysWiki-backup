% 四位移、四速度和四加速度

\pentry{斜坐标表示洛伦兹变换\upref{SROb},坐标变换与过渡矩阵}
%需要补充“过渡矩阵”词条
%未完成

\subsection{概念}
\subsubsection{四位置和四位移}
把时间坐标和空间坐标都看成时空的坐标,那么一个事件在时空中的位置就被称为其\textbf{四位置(4-position)}.四位置本身是一个向量,但其坐标表示取决于所选取的惯性参考系.同一个四位置在不同惯性系中的坐标,可以利用洛伦兹矩阵作为过渡矩阵来相互转化.

两个事件之间的四位置之差,称为这两个事件之间的\textbf{四位移(4-displacement)}.

\subsubsection{四速度}

假设某质点在三维空间中运动.经典物理中认为,质点轨迹上某一点的速度是一个向量,其方向与该点处轨迹相切.但速度的大小具体是多少,三维轨迹完全没有提供足够的信息.同样的轨迹完全可以是用不同的瞬时速度来走过的.

为了完全从几何角度描述该质点的速度,我们可以把视角提高到四维时空,将三维的轨迹拉升到四维,这样就有充足的信息来描述质点的速度大小了,而三维轨迹就是四维轨迹在三维空间中的投影.取四维轨迹上某一点的切向量$\vec{v}$,使其在时间轴上的投影为$1$(单位时间长度),那么$\vec{v}$在三维空间中的投影就是速度.

这个例子启发我们研究四维速度比研究三维速度更加全面,由此有了以下概念:

\begin{definition}{四速度}

一个质点的\textbf{四速度(4-velocity)},定义为质点的四维运动轨迹的切向量$\vec{v}$,满足$\vec{v}$在瞬时自身系中的时间轴投影长度是$1$.

\end{definition}

四速度的概念可以应用在经典力学中,也可以应用于相对论.

\subsubsection{四加速度}

\begin{definition}{四加速度}

一个质点的\textbf{四加速度(4-acceleration)},定义为质点的四速度对固有时间(瞬时自身系中的时间)求导的结果.

\end{definition}

\subsection{时空间隔}

描述经典力学的时空,是一个四维欧几里得空间.一个$n$维欧几里得空间,是$n$个实度量空间的乘积空间,记为$\mathbb{R}^n$.从线性空间的角度来理解,$n$维欧几里得空间可以看成是$n$个相互垂直的单位向量所张成的空间.在一个欧几里得空间中,两点之间的距离用勾股定理来定义.比如说,在四维欧几里得空间中,$(t_1, x_1, y_1, z_1)$和$(t_2, x_2, y_2, z_2)$之间的距离是$$\sqrt{(t_1-t_2)^2+(x_1-x_2)^2+(y_1-y_2)^2+(z_1-z_2)^2}$$

在欧几里得空间中,“距离”是一个不变量.不管你用什么样的参考系\footnote{不同的参考系对应不同的基向量,但必须是标准正交基.}来描述,两个点之间的距离都是一样的.欧几里得空间之间的变换,是伽利略变换,因此以上定义的勾股距离也被称为一个伽利略不变量.

在洛伦兹变换下,勾股距离不再是不变量.对于$(t_1, x_1, y_1, z_1)$和$(t_2, x_2, y_2, z_2)$,在洛伦兹变换下保持不变的是另一个量:$$\sqrt{-(t_1-t_2)^2+(x_1-x_2)^2+(y_1-y_2)^2+(z_1-z_2)^2}$$.
%笔者休息后继续创作

