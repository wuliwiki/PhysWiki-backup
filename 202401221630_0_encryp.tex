% 如何给文件加密(含 python 加密脚本)
% license Usr
% type Tutor

\pentry{Python 基础\upref{Py}}

以下给出一些常用的 python 加密函数,可以加密单个文件,加密字符串,加密一个文件夹中所有文件/文件名。 注意该脚本只能在 linux 环境使用(因为需要调用命令行的 \verb`openssl`)。在 Windows 中可以通过 WSL\upref{WSLnt} 中使用该脚本。 事实上也可以用 \verb|pyOpenSSL| 模块来增加可移植性。

加密算法为 \verb`aes-256-cbc`, \verb`-nosalt` 意味着同样的数据和密码生成的加密文件一摸一样(以支持网盘的 “秒传” 功能)。若去掉该选项可能会让加密变得更安全一些。

完整程序下载见 \href{https://github.com/MacroUniverse/MyPythonLibrary/tree/master/encrypt}{GitHub} 或\href{https://pan.baidu.com/s/1y4Asx-oS4ShGlN9kIyRZeg?pwd=3q1d#list/path=\%252F}{网盘分享}

\subsection{使用说明}
\begin{itemize}
\item 目前无需依赖任何第三方 python 库, 但需要在 Linux 命令行安装 \verb`openssl`, 例如 \verb`sudo apt install openssl`。
\item 压缩包解压后, 在 \verb`=== what to run ===` 下面输入想要运行的函数, 例如想加密一个字符串,用 \verb`print(encrypt_str('一个字符串', password))`。 注意不要替换 \verb`password`, 脚本运行后会提示手动输入密码
\item 在命令行中 \verb`cd` 到脚本所在路径, 运行 \verb`./encrypt.py` 即可。
\item 
\end{itemize}
