% 正交变换与对称变换
% license Usr
% type Tutor


\begin{issues}
\issueDraft 全文替换向量
\end{issues}

本文用“$(x,y)$”表示对任意两个向量作内积。
\subsection{正交变换}
\begin{definition}{}
定义实内积空间$V$上的满射$\mathcal A$。对于任意$(x,y)\in V$,若有:
\begin{equation}
(x,y)=(\mathcal A x,\mathcal A y)~,
\end{equation}
则称$\mathcal A$是$V$上的\textbf{正交变换}。也就是说,正交变换是保内积不变的满射,从而保向量长度和向量之间的夹角不变。
\end{definition}
实际上,正交变换是线性变换。
\begin{theorem}{}
实内积空间$V$上的正交变换$\mathcal A$一定是线性映射。
\end{theorem}
\textbf{证明:}
\begin{equation}
\begin{aligned}
|\mathcal A(x+y)-(\mathcal Ax+\mathcal Ay)|^2&=(x+y)^2+x^2+y^2-2\left(\mathcal A(x+y),\mathcal Ax+\mathcal Ay\right)\\
&=(x+y)^2+x^2+y^2-2(x+y,x)-2(x+y,y)\\
&=0
\end{aligned}
~,\end{equation}

因而$\mathcal A(x+y)=\mathcal Ax+\mathcal Ay$。同理可得$\mathcal A(kx)=k\mathcal A(x)$。

设$\{\bvec e_i\}^k_{i=1}$是$V$上的一组基,由线性性可知,$\{\mathcal A\bvec e_i\}^k_{i=1}$也是线性无关组。又因为$\mathcal A$是满射,所以$\{\mathcal A\bvec e_i\}^k_{i=1}$是$V$上的一组基,则该线性映射既单又满,是“同构映射”。由正交变换保内积可知,$\mathcal A$\textbf{把标准正交基映射为标准正交基。}

总结上述讨论,易证:$\mathcal A$是正交变换$\Longleftrightarrow \mathcal A$\textbf{把标准正交基映射为标准正交基。}后者是正交矩阵的定义,可见正交矩阵是正交变换的矩阵表示。\textbf{正交矩阵保二次型不变的性质等价于正交变换的内积定义,}\footnote{广义内积即二次型决定的对称双线性函数:$(x,y)\equiv \frac{1}{2}(q(x+y)-q(x)-q(y))$}或者说——

$\mathcal A$是正交变换$\Longleftrightarrow $对任意$x\in V,|\mathcal Ax|=|x|$。

令$O(n)$表示$n$维线性空间$V$上全体正交变换的集合。由定义可知,该集合具有如下性质:
\begin{enumerate}
\item 封闭性。若$A,B\in O(n)$,则$AB\in O(n)$;
\item 结合性。若$A,B,C\in O(n)$,则$A(BC)=(AB)C$;
\item 单位元存在性。$I\in O(n)$;
\item 逆元存在性。若$A\in O(n)$,则$A^{-1}\in O(n)$;
\end{enumerate}
所以,正交变换构成一个群。

在欧几里得空间中,正交矩阵的行列式为$1$或者$-1$,常称行列式为$1$的正交变换为\textbf{第一类的(旋转)};行列式为$-1$的正交变换则是\textbf{第二类的}。
\subsubsection{正交变换的特征值}
由上述讨论可知,正交变换若有特征向量,则特征值的模长为$1$,从而保证向量的模长不变。
\begin{corollary}{}
 设$\lambda$为正交变换$A$的特征值,则$|\lambda|=1$。若$V$定义在实数域上,$\lambda=\pm 1$。
\end{corollary}
\begin{corollary}{}
若$V$是奇数维线性空间,$A$必有特征值$1$。
\end{corollary}
\textbf{证明:}
实数域上易证推论2,下面讨论复数域的情况。由于特征值是特征多项式的根,且一元多项式方程里复根成对出现(若$\lambda_0$为特征值,则$\lambda_0^{*}$也为特征值),所以$A$有偶数个模为$1$的复根,这些特征值连乘为$1$。结合行列式等于特征值连乘这一\autoref{the_MatEig_3}~\upref{MatEig},可知剩下奇数个连乘结果为1的实特征值,得证。
\subsubsection{正交变换的块对角形式}
为了简化正交矩阵的形式,我们先来证明两个常用结论。
\begin{lemma}{}\label{lem_ortho_1}
设$f$为\textbf{实线性空间}$V$上的线性映射,存在$f$的不变子空间$W\subseteq V$,且$\opn{dim}W\in\{1,2\}$。
\end{lemma}
\textbf{证明:\footnote{参考Jier Peter的《代数学基础》}}

若$f$有若干个实特征值,则其对应的特征向量是$V$上的一维不变子空间。

将$n$维$V$复化为$U$。若$\{\bvec e_i\}$为$V$上的一组基,则任意$x\in U$可表示为:
\begin{equation}
x=a^i\bvec e_i+\mathrm i b^j\bvec e_j=\bvec u+\mathrm i\bvec v~.
\end{equation}
其中$a^i,b^i\in \mathbb R,\bvec u,\bvec v\in V$。显然$U\supseteq V$。
在复数域上,$f$有$n$个特征值。设有$k$个实特征值,则对应的特征向量是$V$上的一维不变子空间。设复特征值表示为$a+\mathrm ib$,对应特征向量表示为$x+\mathrm i y\,(x,y\in V)$。利用$f$的线性可得:
\begin{equation}
\begin{aligned}
f(x+\mathrm i y)&=(a+\mathrm i b)(x+\mathrm i y)\\
&=(ax-by)+\mathrm i(ay+bx)\\
&=f(x)+\mathrm if(y)
\end{aligned}~.
\end{equation}
即:
\begin{equation}\label{eq_ortho_1}
\begin{aligned}
f(x)&=ax-by\\
f(y)&=ay+bx
\end{aligned}~,
\end{equation}
所以$\opn{Span}\{x,y\}$为$f$在$V$上的一个二维不变子空间。
\begin{lemma}{}\label{lem_ortho_3}
若正交变换$A$有复根,对应特征向量$x=\bvec u+\mathrm i\bvec v$。则$\bvec u,\bvec v$是相互正交的特征向量。令$W=\opn{Span}\{\bvec u,\bvec v\}$,则有:
\begin{equation}\label{eq_ortho_2}
A|_W=\left[\begin{array}{rr}
\cos \phi & \sin \phi \\
-\sin \phi & \cos \phi
\end{array}\right]~.
\end{equation}
\end{lemma}
\textbf{证明:}
设复特征值$\lambda=a+b\mathrm i=\opn{cos}\phi+\mathrm i \opn{sin}\phi$,代入\autoref{eq_ortho_1} 便可得形式为\autoref{eq_ortho_2} 的$A|_W$。接下来只需要证明实部和虚部分量是标准正交即可。为方便计,把转置操作表示为'。

由于$kx$依然是$A$的特征向量,我们可以归一化其\textbf{实部分量},使得$\bvec u^2=1$。则
\begin{equation}\label{eq_ortho_3}
\begin{aligned}
x^2&=(\bvec u+\mathrm i\bvec v)^2\\
&=1-\bvec v^2+2\mathrm i \bvec u\cdot\bvec v
\end{aligned}~.
\end{equation}

由于$A'=A^{-1}$且$Ax=\lambda x$,则$A'x=A^{-1}x=\frac{1}{\lambda}x=\lambda^* x$。左乘特征向量的转置得:
\begin{equation}
x'A'x=\lambda^* x'x=(Ax)'x=\lambda x'x~,
\end{equation}
因此$x^2=x'x=0$。代入\autoref{eq_ortho_3} 可得:
\begin{equation}
\bvec v^2=1,\,\bvec u\cdot \bvec v=0~,
\end{equation}
得证。
这两条引理说明,正交变换$A$的每个复特征值都对应一个二维不变子空间,且$A$限制在该子空间上的形式总如\autoref{eq_ortho_2} 所示。若$A$为二阶矩阵且实数域上无特征根,则$A$唯一表示为该形式,乘以任意向量相当于旋转该向量。
\begin{exercise}{}
写出二阶正交矩阵的所有可能形式。
\end{exercise}
\begin{lemma}{}\label{lem_ortho_2}
设$A$为$V$上的正交变换,若$W$为其不变子空间,则$W^{\bot}$也是其不变子空间。
\end{lemma}
设$\{\bvec e_i\}$为$W$上的一组标准正交基,并扩展到全空间,使得$\opn{Span}\{\bvec {\theta}_i\}$张成$W^{\bot}$。由题设知$\{A\bvec e_i\}$依然张成$W$,由于$(A\bvec {\theta}_i,A\bvec e_i)=0$,因此$A\bvec{\theta}_i\in W^{\bot}$,得证。

由定理3可知,在$\{\bvec e_i\}\cup\{\bvec {\theta}_i\}$下,$A$有块对角形式:$A=A|_W\oplus A|_{W^{\bot}}$。实际上,该块对角形式可以进一步“细化”。
\begin{theorem}{}
设$A$为n维\textbf{实线性空间}$V$上的正交变换。则在$V$内存在一组\textbf{标准正交基},使得$A$具有如下块对角形式:
\begin{equation}
A=\left[\begin{array}{lllllll}
\lambda_1 & & & & & & \\
& \lambda_2 & & & & & \\
& & \ddots & & & 0 & \\
& & & \lambda_k & & & \\
& & & & S_1 & & \\
& & 0 & & & \ddots & \\
& & & & & & S_l
\end{array}\right]~,
\end{equation}
其中$\lambda_i=\pm 1(i=1,2...k)$,$S_i$为\autoref{eq_ortho_2} ,$\phi$由$\phi_i$代替。
\end{theorem}
\textbf{证明:}
若$n=1$,矩阵为一实数,设为a。由$(ax,ax)=x^2$可得,$a=\pm 1$。

若$n=2$且$A$没有特征值,由\autoref{lem_ortho_1} 可知,$A$在$V$上必有一二维不变子空间,$A$的形式就是旋转矩阵。
如果$A$有一特征值$\lambda_1$且$Ax=\lambda_1 x$,设单位向量$y\cdot x=0$,由\autoref{lem_ortho_2} 知,$y$也是$A$的特征向量。设$Ay=\lambda_2 y$,则
\begin{equation}
A=\left[\begin{array}{rr}
\lambda_1&0\\
0&\lambda_2
\end{array}
\right]~.
\end{equation}

若$A$无特征值,由\autoref{lem_ortho_3} 知其形式即旋转矩阵。

接下来设$n>2$。若$A$无特征值,则必有一二维不变子空间,设为$S$,$A|_{S}$为\autoref{eq_ortho_2}。$A|_{S^{\bot}}$为正交矩阵,若$\lambda_1$为其特征值,则必定有一维不变子空间,设对应的特征向量为$x_1$,则可以扩展为$S^{\bot}$上的标准正交基,设$x_2$的正交补为$S_1$,则
\begin{equation}
\left[\begin{array}{rrr}
A|_{S}&0&0\\
0&\lambda_1&0\\
0&0&A|_{S_1}
\end{array}
\right]~.
\end{equation}
则$\lambda_1$也是$A$的特征值,矛盾。故$A$由若干个形式如\autoref{eq_ortho_2} 的二阶矩阵直和而成。

若$A$有特征值,令其为$\lambda'_1$,则其特征向量可以扩展为全空间的标准正交基。在该特征向量的正交补空间讨论,若可以找到另一特征值$\lambda'_2$,则对应的特征向量又可以扩展为该正交补上的基矢组,经过相似变换得到$A$的新形式,$\lambda'_1$和$\lambda'_2$为对角元。以此类推,最后得到该定理的形式,$k$为$A$的特征值数量。
\subsection{对称变换}
\begin{definition}{}
设$\mathcal B$为实线性空间上的线性变换,若对于任意$x,y\in V$有
\begin{equation}\label{eq_ortho_4}
(x,\mathcal By)=(\mathcal Bx,y)~,
\end{equation}
则称$B$为对称变换。
\end{definition}

若$V$为n维实线性空间,$B$为对称变换的矩阵表示,则由\autoref{eq_ortho_4} 得:
\begin{equation}
x^TBy=x^TB^Ty~.
\end{equation}
左乘$x$可得,
\begin{equation}
By=B^Ty~.
\end{equation}
因此,$B=B^T$,对称变换的矩阵表示是对称矩阵。
\begin{theorem}{}
$n$维欧式空间上的线性变换$\mathcal B$是对称变换当且仅当$\mathcal B$在任意标准正交基下的矩阵表示为对称矩阵。

\end{theorem}