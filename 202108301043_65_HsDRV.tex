% 离散型随机变量(高中)
% 高中|离散型随机变量

\subsection{定义}
试验结果可以用一个变量 $X$ 来表示,并且 $X$ 是随着试验的结果的不同变化的,我们把这样的变量 $X$ 叫做一个\textbf{随机变量(random variable)}.随机变量常用大写字母 $X,Y,\cdots$ 表示.

如果随机变量  $X$ 的所有可能的取值都能一一列举出来,则称 $X$ 为\textbf{离散型随机变量(discrete random variable)}.

\subsection{离散型随机变量的分布列}
要掌握一个离散型随机变量 $X$ 的取值规律,必须知道:
\begin{enumerate}
\item $X$ 所有可能取的值 $x_1$,$x_2$,$\cdots$ ,$x_n$ .
\item $X$ 取每一个值 $x_i$ 的概率 $p_1$,$p_2$,$\cdots$,$p_n$.
\end{enumerate}
这就是说,需要列出下表:

\begin{table}[ht]
\centering
\caption{分布列}\label{HsDRV_tab1}
\begin{tabular}{|c|c|c|c|c|c|c|}
\hline
$X$ & $x_1$ & $x_2$ & $\cdots$ & $x_i$ & $\cdots$ & $x_n$ \\
\hline
$P$ & $p_1$ & $p_2$ & $\cdots$ & $p_i$ & $\cdots$ & $p_n$ \\
\hline
\end{tabular}
\end{table}
我们称这个表为离散型随机变量 $X$ 的\textbf{概率分布(probability distribution)},或称为离散型随机变量 $X$ 的\textbf{分布列(distribution series)}.

如果随机变量 $X$ 的分布列为

\begin{table}[ht]
\centering
\caption{二点分布}\label{HsDRV_tab2}
\begin{tabular}{|c|c|c|}
\hline
$X$ & $1$ & $0$ \\
\hline
$P$ & $p$ & $q$ \\
\hline
\end{tabular}
\end{table}

\begin{table}[ht]
\centering
\caption{请输入表格标题}\label{HsDRV_tab3}
\begin{tabular}{|c|c|c|}
\hline
$X$ & $1$ & $0$ \\
\hline
$P$ & $p$ & $q$ \\
\hline
\end{tabular}
\end{table}