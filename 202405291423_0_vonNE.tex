% von Neumann 熵
% keys von Neumann entropy|von Neumann熵|冯诺依曼熵|纠缠熵
% license Xiao
% type Tutor

\begin{issues}
\issueTODO
\issueMissDepend(香农熵)
\end{issues}

\pentry{密度矩阵\nref{nod_denMat}}{nod_bc59}
%此处还应该运用香农熵的预备知识,但在我编辑这条消息的时候还没有相对应的文章,如果以后有了记得加上。

\footnote{参考了\cite{量子信息}和Wikipedia\href{https://en.wikipedia.org/wiki/Von_Neumann_entropy}{相关界面}}
von Neumann 熵的形式来自于 Shannon 熵。

\subsection{von Neumann 熵和量子相对熵}

\begin{definition}{von Neumann熵}\label{def_vonNE_1}
对于一个给定的密度矩阵,von Neumann熵 $S\left(\rho\right)$的定义为:

\begin{equation}
S\left( \rho \right) = \opn{tr}\left( - \rho \log \rho \right)~.
\end{equation}

如果$\left\{ \lambda_1,~\lambda_2,~\cdots \lambda_N \right\}$是$\rho$的本征值,那么:

\begin{equation}
S\left(\rho\right) = \sum_i^N \lambda_i \log \lambda_i~.
\end{equation}

上式中应注意我们定义$0\log0 = 0$来规避发散。

\end{definition}

值得注意的是,\autoref{def_vonNE_1} 中 von Neumann 熵的底数并未指明,在某些文献中底数被定义为$2$,在某些文献中底数则被定义为$\E$,请读者在阅读相关文献时自行判断,在大多数情况下两者并无任何本质区别,仅仅相差一个倍数,所以我们仅在涉及到von Neumann熵的具体数值的时候,比如讨论纠缠熵的连续性时指明底数。

von Neumann 度量了一个混态的密度矩阵的“混乱程度”,正如\enref{约hua{partra}中提到,如果一个大系统的纯态对其中的某一个子系统取偏迹,同时如果得到了一个混态而非纯态,那么代表该子系统与剩余部分存在纠缠,这时求完偏迹的密度矩阵的von Neumann 熵就给出了一个度量纠缠的方法,这既是其纠缠熵名字的由来。

\subsubsection{量子相对熵}

量子相对熵类似于经典相对熵,$\rho$到$\sigma$的量子相对熵定义为:

\begin{equation}
S\left(\rho || \sigma\right) = \opn{tr}\left(\rho \log \rho\right) - \opn{tr}\left(\rho \log \sigma\right)~.
\end{equation}

接下来我们证明 \textbf{Klein 不等式},也就是量子相对熵非负:
\begin{equation}\label{eq_vonNE_2}
S\left(\rho || \sigma\right) \geqslant 0~.
\end{equation}

假设$\rho = \sum\limits_ip_i\ketbra{u_i}{u_i}$,$\sigma = \sum\limits_jq_j\ketbra{v_j}{v_j}$。

带入量子相对熵的表达式:

\begin{equation}
\begin{aligned}
S\left(\rho || \sigma\right) &= \sum_k \bra{u_k}\left( \sum_ip_i\ketbra{u_i}{u_i}\sum_j\log p_j\ketbra{u_j}{u_j} - \sum_ip_i\ketbra{u_i}{u_i}\sum_j\log q_j\ketbra{v_j}{v_j} \right)\ket{u_k} \\
&= \sum_{i,j,k}\left(\delta_{ik}\delta_{ij}\delta_{jk}p_i\log p_j - \delta_{ik}p_i \braket{u_i}{v_j}\braket{v_j}{u_k}\log q_j \right) \\
&= \sum_i\left( p_i\log p_i - \sum_jp_i\braket{u_i}{v_j}\braket{v_j}{u_i}\log q_j  \right) \\
&= \sum_i\left(p_i\log p_i - \sum_j p_i\abs{\braket{u_i}{v_j}}^2\log q_j \right) \\
&= \sum_ip_i\left(\log p_i - \sum_j P_{ij}\log q_j\right)~.
\end{aligned}~
\end{equation}

其中$P_{ij} = \abs{\braket{u_i}{v_j}}^2 \geqslant 0$。易得$\sum\limits_i P_{ij} = \sum\limits_j P_{ij} = 1$($u_i$在另一组标准正交基$\left\{v_j\right\}$下展开系数模方和为1)。

考虑对数函数的凹凸性,则$\sum\limits_j P_{ij}\log q_j \leqslant \log\left(\sum\limits_j P_{ij}q_j\right) = \log \left(r_i\right)$。当且仅当$P$矩阵为置换阵时,不等式取等号。

且$\sum\limits_i r_i = \sum\limits_{ij}P_{ij}q_j = \sum\limits_jq_j = 1$,即$\left\{r_i\right\}$可视作一概率分布。

则有:

\begin{equation}\label{eq_vonNE_1}
\begin{aligned}
S\left(\rho || \sigma\right) &= \sum_i p_i\left(\log p_i - \sum_j P_{ij}\log q_j\right) \\ 
&\geqslant\sum_i p_i\left(\log p_i - \log r_i\right) \\
&= \sum_i p_i\log\frac{p_i}{r_i}~.
\end{aligned}~
\end{equation}

可以看出\autoref{eq_vonNE_1} 最后的形式是概率分布$\left\{p_i\right\}$对概率分布$\left\{r_i\right\}$的经典相对熵,由经典相对熵的非负性有$\sum\limits_i p_i \log\frac{p_i}{r_i}\geqslant 0$。

则$S\left(\rho || \sigma\right)\geqslant 0$。由此我们证明了量子相对熵是非负的。

\subsection{von Neumann熵的性质}

von Neumann熵有以下几条性质:

\subsubsection{纯态密度矩阵的von Neumann熵为0}
密度矩阵$\rho$的von Neumann熵当且仅当$\rho$表示纯态时为0。我们在定义量子态的 von Neumann熵时\autoref{def_vonNE_1} ,规定了$0\log 0 = 0$,纯态在计算von Neumann熵时仅会出现$1\log 1$和$0\log 0$项,均为0,则纯态的纠缠熵也为0。

\subsubsection{von Neumann熵存在上限}

von Neumann熵存在上限,$d$维的希尔伯特空间中的量子态的von Neumann熵的最大值为为$\log d$,当且仅当$\rho = \frac{1}{d}I$时取到最大值。

对于这个上限的证明并不复杂,得益于我们已经在前文中证明了量子相对熵非负,所以我们仅需要计算一个量子态与$\frac{1}{d}I$之间的量子相对熵即可。
\begin{equation}
\begin{aligned}
S\left(\rho \left|\left| \frac{1}{d}I\right.\right.\right) &= \opn{tr}\left(\rho\log\rho\right) - \opn{tr}\left( \rho\log\left(\frac{1}{d}I\right) \right) \\
&= -S\left(\rho\right) + \log d \opn{tr}\left(\rho\right) \\
&= \log d - S\left(\rho\right)  \\
&\geqslant 0
\end{aligned}~
\end{equation}

所以我们可以得到$S\left(\rho\right) \leqslant \log d$

\subsubsection{纯态在两个子区域上的约化密度矩阵的von Neumann熵相等}

如果复合系统$AB$总体处于纯态,$\rho_A$和$\rho_B$分别为其在$A$区域和在$B$区域分别的约化密度矩阵,那么其von Neumann熵相等。

对于定义在$AB$两区域上的纯态$\ket{\psi}$,我们取维数较大的$N$维子空间称为空间$A$,我们总可以取$A$区域的一组正交基将其展开,写成:
$$\ket{\psi} = \sum\limits_i^N\ket{a_i}\ket{u_i} = \begin{pmatrix}
\ket{a_1}&\ket{a_2}&\cdots&\ket{a_N}
\end{pmatrix}\begin{pmatrix}
\ket{u_1}\\ \ket{u_2}\\ \vdots\\ \ket{u_M}
\end{pmatrix}~.$$
上式中$\left\{\ket{a_i}\right\}$是$A$区域上的一组正交归一基矢量,而$\left\{\ket{u_i}\right\}$则是一组定义在$B$区域上的,不一定正交也不一定归一的矢量。而我们总可以通过$\left\{\ket{u_i}\right\}$来张成一个$N$维线性空间(如果矢量不足$N$个,则需引入额外的基矢量来补足),我们将这组新的基地记作$\left\{\ket{b_i}\right\}$。则存在$N\times N$的$A$矩阵,使得:
$$\begin{pmatrix}
\ket{u_1}\\ \ket{u_2}\\ \vdots\\ \ket{u_N}
\end{pmatrix} = A
\begin{pmatrix}
\ket{b_1}\\ \ket{b_2}\\ \vdots\\ \ket{b_N}
\end{pmatrix}~.$$
因此有:
$$\ket{\psi} = \begin{pmatrix}
\ket{a_1}&\ket{a_2}&\cdots&\ket{a_N}
\end{pmatrix}A\begin{pmatrix}
\ket{b_1} \\ \ket{b_2} \\ \vdots \\ \ket{b_N}
\end{pmatrix}~.$$
考虑矩阵$A$的奇异值分解,$A = U\Sigma D^\dagger$,其中,$U$和$D$为幺正矩阵,而$\Sigma$为半正定对角矩阵,则有:
$$\ket{\psi} = \begin{pmatrix}
\ket{a_1}&\ket{a_2}&\cdots&\ket{a_N}
\end{pmatrix}U \Sigma D^\dagger\begin{pmatrix}
\ket{b_1} \\ \ket{b_2} \\ \vdots \\ \ket{b_N}
\end{pmatrix}~.$$
由于$U$和$D$的幺正特性,则其于原本的标准正交基相乘之后得到的仍然是一组标准正交基,即
$$\begin{pmatrix}
\ket{\tilde{a}_1}&\ket{\tilde{a}_2} & \cdots & \ket{\tilde{a}_N}
\end{pmatrix} = 
\begin{pmatrix}
\ket{a_1}&\ket{a_2}&\cdots &\ket{a_N}
\end{pmatrix}U~$$
和
$$\begin{pmatrix}
\ket{\tilde{b}_1} & \ket{\tilde{b}_2} & \cdots & \ket{\tilde{b}_N}
\end{pmatrix} = \begin{pmatrix}
\ket{b_1}&\ket{b_2}& \cdots &\ket{b_N}
\end{pmatrix}D~$$
都是标准正交基,那么我们记$\Sigma$的第$i$个对角元是$\sigma_i$,那么我们则可以写出:
$$\ket{\psi} = \sum_i \sigma_i \ket{\tilde{a}_i}\ket{\tilde{b}_i}~.$$
写道这里已经很容易看到,$\ket{\psi}$在以$\left\{\ket{\tilde{a}_i}\right\}$和$\left\{\ket{\tilde{b}_i}\right\}$为基矢分别求取约化密度矩阵时,得到的是相同的对角矩阵的形式,因此自然von Neumann熵也相同。

\subsubsection{加和的密度矩阵的von Neumann熵}

若$\left\{p_i\right\}$是概率分布,而$\rho_i$位于相互正交的空间上,那么有:
$$S\left(\sum\limits_i p_i\rho_i\right) = \sum\limits_i p_i S\left(\rho_i\right) - \sum\limits_i p_i \log p_i~.$$

由于$\rho_i$都是位于相互正交的空间上,所以其均是相互对易的,可同时对角化的,记$\rho_i$的第$j$个本征值为$\lambda_i^j$,有$\sum\limits_j \lambda_i^j = 1$,同时由于$\sum\limits_i p_i = 1$,则$\sum\limits_{ij}p_i\lambda_i^j = 1$,则事实上,$\sum\limits_i p_i\rho_i$实际上得到了一个本征值为$p_i\lambda_i^j$(i,j任意取值)的密度矩阵,所以有:

\begin{equation}
\begin{aligned}
S\left( \sum_i p_i\rho_i \right) &= -\sum_{i,j}p_i\lambda_i^j\log\left(p_i \lambda_i^j\right) \\
&= -\sum_{i,j}p_i\lambda_i^j\log p_i - \sum_{i,j}p_i\lambda_i^j\log\lambda_i^j \\
&= -\sum_i p_i\log p_i - \sum_i p_i\sum_j\lambda_i^j\log\lambda_i^j \\
&= \sum_i p_i S\left(\rho_i\right) - \sum_i p_i\log p_i~.
\end{aligned}~
\end{equation}


\subsubsection{直积态的von Neumann熵}

对于$\rho$和$\sigma$的直积态$\rho\otimes\sigma$,其纠缠熵为$S\left(\rho\otimes \sigma\right) = S\left(\rho\right) + S\left(\sigma\right)$。

假设$\rho$的本征值为$\lambda_i^1,~i\in \left\{1,2,\cdots N^1\right\}$,$\sigma$的本征值为$\lambda_i^2,~i \in \left\{1,2,\cdots N^2\right\}$。那么直积态$\rho\otimes \sigma$的本征值就为$\lambda_i^1\lambda_j^2,~i\in\left\{1,2,\cdots N^1\right\},~j\in\left\{1,2,\cdots N^2\right\}$共$N^1\times N^2$个本征值。

那么类似上一条性质的,我们有:
\begin{equation}
\begin{aligned}
S\left(\rho\otimes\sigma\right) &= -\sum_{i,j}\lambda_i^1\lambda_j^2\log\lambda_i^1\lambda_j^2 \\
&= -\sum_{i,j}\lambda_i^1\lambda_j^2\log\lambda_i^1 - \sum_{i,j}\lambda_i^1\lambda_j^2\log\lambda_j^2 \\
&= -\sum_i \lambda_i^1\log\lambda_i^1 - \sum_i\lambda_i^2\log\lambda_i^2 \\
&= S\left(\rho\right) + S\left(\sigma\right)~.
\end{aligned}~
\end{equation}
由此得证。

\subsubsection{纠缠熵的次可加性}

纠缠熵的次可加性说的是不等式:

\begin{equation}
S\left(A,B\right) \leqslant S\left(A\right) + S\left(B\right) ~.
\end{equation}
他的证明是 klein 不等式\autoref{eq_vonNE_2} 的简单应用。 记AB总系统上密度矩阵为$\rho_{AB}$,$A$
和$B$分别两个子系统上的密度矩阵为$\rho_A$和$\rho_B$,我们令 klein 不等式中的$\rho = \rho_{AB}$,$\sigma = \rho_A\otimes \rho_B$。那么有:

\begin{equation}
\begin{aligned}
S\left(\rho_{AB}\left|\left|\rho_A\otimes\rho_B\right.\right.\right)&= \opn{tr}\left(\rho_{AB}\log\rho_{AB}\right) - \opn{tr}\left(\rho_{AB}\log\left(\rho_A\otimes\rho_B\right)\right) \\
&= -S\left(A,B\right) - \opn{tr}\left(\rho_{AB}\log\left(\rho_A\otimes I_B\right) + \rho_{AB}\log\left(I_A\otimes\rho_B\right)\right) \\
&= -S\left(A,B\right) - \opn{tr}\left(\rho_A\log\rho_A\right) - \opn{tr}\left(\rho_B\log\rho_B\right) \\
&= -S\left(A,B\right) + S\left(A\right) + S\left(B\right) \\
&\geqslant 0~.
\end{aligned}~
\end{equation}
由此可证$S\left(A,B\right)\leqslant S\left(A\right) + \left(B\right)$,其取等号的条件也即是 klein 不等式取等号的条件,即$\rho_{AB} = \rho_A\otimes \rho_B$,总系统是子系统的直积态时取等号,这恰好与上一条性质相符。

\subsubsection{纠缠熵的三角不等式}

纠缠熵的三角不等式,也被称为 Araki-Lieb 不等式,指的是:
\begin{equation}
S\left(A,B\right)\geqslant\abs{S\left(A\right) - S\left(B\right)}~.
\end{equation}
除$A$,$B$系统外,我们额外引入系统$C$来纯化系统$A$和$B$。此时由于总系统是纯态,$S\left(A,B,C\right) = 0$,则有$S\left(A\right) = S\left(B,C\right)$和$S\left(B\right) = S\left(A,C\right)$。

分别考虑$A$,$C$区域的次可加性和$B$,$C$区域的次可加性,则有:
\begin{equation}
\begin{aligned}
S\left(A,C\right)&\leqslant S\left(A\right)+S\left(C\right) \\
S\left(B,C\right)&\leqslant S\left(B\right)+S\left(C\right)~.
\end{aligned}~
\end{equation}

也可写作:

\begin{equation}
\begin{aligned}
S\left(B\right)&\leqslant S\left(A\right) + S\left(A,B\right) \\
S\left(A\right)&\leqslant S\left(B\right) + S\left(A,B\right)~.
\end{aligned}~
\end{equation}

整理得:

\begin{equation}
\begin{aligned}
S\left(A,B\right)&\geqslant S\left(B\right) - S\left(A\right) \\
S\left(A,B\right)&\geqslant S\left(A\right) - S\left(B\right)~.
\end{aligned}~
\end{equation}

写在一起就是$S\left(A,B\right)\geqslant\abs{S\left(A\right) - S\left(B\right)}$。

\subsection{纠缠熵的连续性}

借助\enref{迹距离}{Trdist}作为密度矩阵的度量,我们可以讨论纠缠熵的连续性。

纠缠熵的连续性由 Fannes 不等式保证。

\begin{theorem}{Fannes不等式}
设$\rho$和$\sigma$是两个密度矩阵,$T\left(\rho,\sigma\right)$是$\rho$和$\sigma$之间的迹距离。若$T\left(\rho,\sigma\right) \leqslant \frac{1}{e}$,则有:

\begin{equation}
\abs{S\left(\rho\right) - S\left(\sigma\right)} \leqslant 2T\left(\rho,\sigma\right)\log_2 d - 2T\left(\rho,\sigma\right) \log_2\left[ 2 T\left(\rho,\sigma\right)\right]~.
\end{equation}
其中$d$表示希尔伯特空间的维度。

\end{theorem}

而对于更大的$T\left(\rho,\sigma\right)$,有弱化版的不等式:

\begin{equation}
\abs{S\left(\rho\right) - S\left(\sigma\right)} \leqslant 2 T\left(\rho,\sigma\right)\log d + \frac{1}{e\ln 2}~.
\end{equation}

在\cite{量子信息}中给出了证明,而在\href{https://arxiv.org/pdf/quant-ph/0610146.pdf}{论文}中给出了该不等式更强的形式:

\begin{equation}
\abs{S\left(\rho\right) - S\left(\sigma\right)} \leqslant T\left(\rho,\sigma\right)\log_2\left(d-1\right) + H\left(\left(T\left(\rho,\sigma\right),1-T\left(\rho,\sigma\right)\right)\right)~.
\end{equation}

其中$H(p)$是香农熵。

可见由Fannes不等式可得,在$\forall 0<\epsilon,~\exists \delta>0$,当$T\left(\rho,\sigma\right)<\delta$时,有$\abs{S\left(\rho\right) - S\left(\sigma\right)}<\epsilon$,
其中,$\delta$取$\epsilon = x\log_2\left(d-1\right) + H\left(\left(x,1-x\right)\right)$的较小的解,当解不存在时,取$f(x)=x\log_2\left(d-1\right) + H\left(\left(x,1-x\right)\right)$的极值点横坐标。

由此可以说Fannes不等式给出了在迹距离的度量下,纠缠熵的连续性。