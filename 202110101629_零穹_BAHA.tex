% Baker-Hausdorff公式
% Baker|Hausdorff|定理


Baker-Hausdorff公式是一个相当有用的公式.在数学上,它可用于给出李群-李代数对应的深层结果的相对简单的证明;在量子力学中,它可实现系统哈密顿量在薛定谔绘景和海森堡绘景的转换,并在微扰论中也有诸多应用.本节将给出该公式的一个证明和由它导出的一些重要的结果.

Baker-Hausdorff公式是指
\begin{equation}
\begin{aligned}
\E ^{A}B\E^{-A}=\sum_{n=0}^{\infty}\frac{1}{n!}A_n\\
A_n\equiv\underbrace{[A,[A,\cdots,[A}_{n\text{个}},B]]\cdots]
\end{aligned}
\end{equation}

\subsection{证明}
\begin{lemma}{}
\begin{equation}\label{BAHA_eq1}
A_n\equiv\underbrace{[A,[A,\cdots,[A}_{n\text{个}},B]]=\sum_{m=0}^{n}(-1)^{m}C_{n}^{m}A^mBA^{n-m}
\end{equation}
\end{lemma}
\textbf{证明:}
$A_0,A_1$显然成立:
\begin{equation}
\begin{aligned}
A_0&=B=\sum_{m=0}^{0}(-1)^{m}C_{0}^{m}A^mBA^{0-m}\\
A_1=&[A,B]=AB-BA=\sum_{m=0}^{1}(-1)^{m}C_{1}^{m}A^mBA^{1-m}
\end{aligned}
\end{equation}
假设对 $n=k-1$ 时\autoref{BAHA_eq1} 成立,则
\begin{equation}
\begin{aligned}
A_k&=[A,A_{k-1}]=AA_{k-1}-A_{k-1}A\\
&=A\sum_{m=0}^{k-1}(-1)^{m}C_{k-1}^{m}A^mBA^{k-1-m}-\qty(\sum_{m=0}^{k-1}(-1)^{m}C_{k-1}^{m}A^mBA^{k-1-m})A\\
&=\sum_{m=0}^{k-1}(-1)^{m}C_{k-1}^{m}A^{m+1}BA^{k-1-m}-\sum_{m=0}^{k-1}(-1)^{m}C_{k-1}^{m}A^mBA^{k-m}\\
&=(-1)^{k-1}A^{k}B+\sum_{m=0}^{k-2}(-1)^m\qty(C_{k-1}^{m}+C_{k-1}^{m+1})A^{m+1}BA^{k-1-m}-BA^k\\
&=(-1)^{k-1}A^{k}B+\sum_{m=0}^{k-2}(-1)^mC_{k}^{m+1})A^{m+1}BA^{k-1-m}-BA^k
\end{aligned}
\end{equation}
