% 线性连续泛函
% keys 线性|连续|泛函
% license Usr
% type Tutor

\pentry{拓扑向量空间\nref{nod_tvs},泛函与线性泛函\nref{nod_Funal}}{nod_5b23}
拓扑线性(或拓扑向量)空间的拓扑空间性质表明其上映射的连续性具有基本的重要性,而其线性空间性表明线性映射具有基本的重要性。因此,拓扑线性空间上的线性连续映射具有基本的重要性。特别,对拓扑线性空间上的泛函,线性连续泛函具有基本的重要性。
\begin{definition}{连续}
设 $f$ 是拓扑线性空间 $E$ 上的\enref{线性泛函}{Funal}。若对任意 $x_0\in E,0<\epsilon\in\mathbb R$,存在 $x_0$ 的邻域 $U$,使得 $x\in U$ 就有
\begin{equation}\label{eq_LinCon_1}
\abs{f(x)-f(x_0)}<\epsilon,~
\end{equation}
则称 $f$ 是\textbf{线性连续泛函}。
\end{definition}

一般定义映射的连续,往往是先定义映射在一点的连续。然而上面的定义是对每一点都连续,而没有事先定义在一点的连续。事实上,在拓扑线性空间中,线性映射在一点连续必定在全空间连续。这由下面的定理指定。
\begin{theorem}{在一点处连续的线性泛函处处连续}
设线性泛函 $f$ 在某一点 $x\in E$ 处连续,则 $f$ 必定在 $E$ 上处处连续。
\end{theorem}
\textbf{证明:}由\autoref{the_tvs_1},设 $U$ 是 $x$ 的满足\autoref{eq_LinCon_1} 的邻域,于是 $U-x$ 是零邻域,从而 $V=U-x+y=U+(y-x)$ 是(任一点) $y$ 的邻域。因此,若 $z\in V$,则 $z+x-y\in U$,进而
\begin{equation}
\abs{f(z)-f(y)}=\abs{f(z-y+x)-f(x)}<\epsilon.~
\end{equation}
即验证线性泛函在点 $y$ 的连续性,只需要验证在某一点 $x$ 的连续性即可。

\textbf{证毕!}

\begin{theorem}{有限维的线性泛函必连续}
设 $E$ 是有限维的拓扑向量空间,则 $E$ 中的任何lian'x
\end{theorem}




