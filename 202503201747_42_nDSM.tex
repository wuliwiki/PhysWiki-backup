% N 维球的度规
% keys n维球|高维球|度规
% license Usr
% type Tutor

\pentry{度规张量与指标升降\nref{nod_TofEuc}}{nod_2f31}
$N$ 维球是 $N+1$ 维空间中的超球面,其中的“超”字在数学上定义为 $N$ 维空间中的 $N-1$ 为曲面。因此,$N$ 维球作为三维空间球面的推广,代表着 $N$ 维空间中到某一(称为球心的)点距离恒定的所有点的全体。通常球心取为坐标原点。本节将推导球坐标下 $N$ 维球的\enref{度规}{TofEuc}。

\subsection{$N+1$ 维空间的球坐标}
\begin{definition}{笛卡尔坐标}\label{def_nDSM_1}
在 $N+1$ 空间中,若在坐标 $(x^1,\cdots,x^{N+1})$ 下,线元 $\dd s^2$ 可写为
\begin{equation}
\dd s^2=\dd x^i+\cdots+\dd x^{N+1},~
\end{equation}
 则称坐标 $(x^1,\cdots,x^{N+1})$ 为\textbf{笛卡尔坐标}。
\end{definition}

高维空间中的球坐标可以通过三维空间的球坐标推广得到。在三维空间中,笛卡尔坐标 $(x,y,z)$ 和球坐标 $(r,\theta,\varphi)$ 的关系具有这样的几何图像:$r$ 代表(由坐标原点指向对应点的矢量)对应点径矢的长度, $\theta$ 是径矢与 $z$ 轴的夹角,$\varphi$ 是径矢在 $x-y$ 平面的投影与 $x$ 轴的夹角。由此得到两坐标系统的转换关系
\begin{equation}
\begin{aligned}
&x=r\sin\theta\cos\varphi,\\
&y=r\sin\theta\sin\varphi,\\
&z=r\cos\theta.\\
\end{aligned}~
\end{equation}
推广到 $N+1$ 维空间中,则 $N+1$ 维球坐标 $(r,\theta^1,\cdots,\theta^{N})$ 和笛卡尔坐标 $(x^1,\cdots,x^{N+1})$ 具有这样的联系:$r$ 代表点径矢的大小,$\theta^{N}$ 代表径矢和 $x^{N+1}$ 轴的夹角,$\theta^{N-1}$ 代表径矢在垂直与 $x^{N+1}$ 的超曲面上的投影和 $x^{N}$ 的夹角, $\theta^{N-2}$ 代表径矢在垂直于 $x^{N+1}$ 的投影矢量,再次投影在垂直于 $x^{N+1},x^{N}$ 轴的平面上,得到的投影矢量和 $x^{N-1}$ 的夹角,其它球坐标以此类推。

因此可得 $n+1$ 空间的球坐标和笛卡尔坐标的关系。其可以总结在下面的球坐标的定义中。
\begin{definition}{高维空间的球坐标}\label{def_nDSM_2}
是 $(x^1,\cdots,x^{N+1})$ 是 $N+1$ 维空间的笛卡尔坐标,则称如下定义的坐标 $(r,\theta,\varphi)$ 为该空间上的\textbf{球坐标}。
\begin{equation}
\begin{aligned}
&x^{N+1}=r\cos \theta^N,\\
&x^{N}=r\sin \theta^N\cos\theta^{N-1},\\
&x^{N-1}=r\sin \theta^N\sin\theta^{N-1}\cos\theta^{N-2},\\
&\cdots\\
&x^{2}=r\sin \theta^N\sin\theta^{N-1}\ldots\sin\theta^2\cos\theta^1,\\
&x^{1}=r\sin \theta^N\sin\theta^{N-1}\ldots\sin\theta^2\sin\theta^1.\\
\end{aligned}~
\end{equation}
\end{definition}

细心的读者可能会有疑问:在\autoref{def_nDSM_1} 中定义笛卡尔坐标时,没有用到其它坐标,而在\autoref{def_nDSM_2} 中定义球坐标时却用到了笛卡尔坐标。难不成笛卡尔坐标相比其它坐标而言具有一种特权性?事实是,我们这里的定义只是起到教学的作用。球坐标和其它所有的坐标都是平等的,

















