% AdS/CFT 对偶
% keys 对偶
% license Usr
% type Wiki

AdS/CFT对偶这个概念起源于超弦理论。超弦理论是统一自然界四大相互作用力的头号选手。这四种相互作用力分别是强相互作用力,弱相互作用力,电磁相互作用力和引力。

AdS/CFT的具体陈述是:
强耦合的4维规范理论=5维AdS空间中的引力理论

AdS/CFT说明了四维空间的理论和五维空间的理论是相互联系着的。由此可见,AdS/CFT常常被称为全息理论。在光学的全息照相理论中,三维的物体可以在二维的平面上成一个全息相。同样的道理,全息理论把一个五维的理论和四维的理论联系在了一起。

规范理论能描述强,弱,电磁相互作用,但不能描述引力。比方说,电磁力可以由U(1)规范理论描述,强力可以由U(3)规范理论,也就是我们说的量子色动力学,也就是QCD来描述。这三种力背后的理论都是规范理论。但是要对强耦合的规范理论进行具体的计算是非常困难的事情。当耦合非常强的时候,我们对这个理论就还不够了解。但是我们可以换一种思路。可以试着把这个强耦合的问题转化成一个弯曲空间中的弱耦合问题进行研究。我们一般会考虑转化成AdS时空中的一个问题来进行研究。

AdS时空也就是所谓的Anti de Sitter(反德西特)时空。球是一个具有常数的正曲率的球面空间。与此对应的AdS空间是一个具有常数负曲率的时空。De Sitter(德西特)是一个荷兰的天文学家。在1917年,他找到了爱因斯坦方程的一个常数正曲率的解,也就是我们常说的de Sitter(德西特)空间。这就是解释了为什么常数负曲率的时空被称之为反德西特空间了。AdS时空有一个很自然定义的空间的边界。我们的规范理论就存在于这个边界之上。

一般来说,对偶的意思是,两个理论看上去很不一样,但是内在却有联系甚至是完全等价的。在AdS/CFT对偶里面,规范理论和引力理论看起来非常不一样,甚至是这两个理论所处的时空也完全不同。但是,在对偶里面,一个理论是弱耦合,另一个理论是强耦合。这会导致两种结果:

1.强弱对偶解释了为什么两个表面上看起来并不相同的理论在对偶下是等价的。当规范理论是强耦合的时候,再去使用弱耦合的规范理论相同的量就不妥当了。我们应该用另外的量去描述这个理论。对偶建议我们在弱耦合的引力理论中寻找合适的量。

2.对偶让两个完全不同的理论联系在了一起。这在概念上非常有趣,并且在实际应用上也非常有价值。因为即使规范理论是强耦合的,我们可以把这个规范理论对偶到弱耦合的引力理论来研究,这让分析更简便一些。

以上适用于零温的情况。在有限温度的情况下,对偶的表述应该修改为:有限温度下强耦合的规范理论=AdS黑洞的引力理论

在引力理论里面出现了黑洞,这是由于黑洞也是一个热力学系统。由于霍金辐射,黑洞具有温度。所以我们可以用有限温度的AdS/CFT去分析非平衡现象。

利用黑洞,我们可以看出quan'xi'yuan'li

