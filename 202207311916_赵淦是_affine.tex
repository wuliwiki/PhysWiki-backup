% 仿射变换在解析几何中的应用
% 仿射变换 圆锥曲线
\begin{definition}{仿射变换}
设椭圆\,\(\frac{x^2}{a^2}+\frac{y^2}{b^2}=1\),其中\,\(a>b>0\),置变换:
$$x'=\frac{x}{a},y'=\frac{y}{b}$$
则椭圆化为单位圆\,\(C:x'^2+y'^2=1\)
\end{definition}
届时,我们可以就可以抛开繁琐的代数计算,运用几何性质解决问题.此前,我们先介绍仿射变换的几个性质.
\begin{lemma}{}
变换后,平面内任意一条直线的斜率变为原来的\,\(\frac{a}{b}\)
\end{lemma}
\begin{lemma}{}
变换后,平面上任意区域的面积变为原来的\,\(\frac1{ab}\)
\end{lemma}
\begin{lemma}{}
变换后,线段中点依然是线段中点;关于坐标轴对称的元素依然关于坐标轴对称;平面区域的重心保持不变
\end{lemma}
\begin{lemma}{}
变换前后,平行关系保持不变
\end{lemma}
\begin{lemma}{}
变换前后,平行线段的长度比保持不变
\end{lemma}
\begin{corollary}{1}
设椭圆 \(\frac{x^2}{a^2}+\frac{y^2}{b^2}\),直线 \(l\) 交椭圆于点 \(A\) 和\,\(B\),点 \(P(x_0,y_0)\) 为线段 \(AB\) 的中点,求直线斜率
\begin{figure}[ht]
\centering
\includegraphics[width=10cm]{./figures/affine_1.png}
\caption{请添加图片描述} \label{affine_fig1}
\end{figure}

解:作变换 \(x'=\frac{x}{a},y'=\frac{y}{b}\) ,则椭圆化为单位圆 \(C:x'^2+y'^2=1\),且 \(P'\left(\frac{x_0}{a},\frac{y_0}{b}\right)\)
故 \(k_{OP'}=\frac{ay_0}{bx_0}\)
由性质3,\(P'\) 为 \(A'B'\) 的中点
在圆中,由垂径定理,\(OP'\bot A'B'\)
而 \(k_{A'B'}\cdot k_{OP'}=-1\) ,得 \(k_{A'B'}=-\frac{bx_0}{ay_0}\)
由上述性质 1, \(k_l=\frac{b}{a}\cdot k_{A'B'}=-\frac{b^2x_0}{a^2y_0}\)
\end{corollary}