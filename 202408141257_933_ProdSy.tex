% 求积符号(累乘)
% license Xiao
% type Tutor

\begin{issues}
\issueDraft
\end{issues}

类似\enref{求和符号}{SumSym}, 若有数列 $\{a_n\}$,即 $a_1, a_2, \dots$ 那么把它的前 $N$ 项相乘可得
\begin{equation}
\prod_{n = 1}^N a_n = a_1\times a_2\times \dots a_N~.
\end{equation}

其他记法参见\enref{大型运算符}{LgOper}。

\subsection{性质}

乘法结合律
\begin{equation}
\prod_i a_i \prod_i b_i =  \prod_{i} a_i b_i~.
\end{equation}

幂运算
\begin{equation}
\qty(\prod_i a_i)^m =  \prod_i a_i^m~.
\end{equation}

提取常数
\begin{equation}
\prod_{i=1}^n (C \cdot a_i) = C^n \cdot \prod_{i=1}^n a_i.~
\end{equation}

\subsection{与求和符号的转化}
\addTODO{对数加连接}

若$\forall a_i > 0$,则利用对数的性质可以得到:

\begin{equation}
\ln \left( \prod_i a_i \right) = \sum_i \ln(a_i).~
\end{equation}

两边取指数也可以得到转换的另一种形式:
\begin{equation}
\prod_i a_i = \exp({\displaystyle\sum_i \ln {a_i}})~.
\end{equation}

