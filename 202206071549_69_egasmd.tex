% 近自由电子模型
% 自由电子气体|布洛赫波|晶格|能带

\pentry{金属中的自由电子气体\upref{mfcgas},布洛赫理论\upref{Bloch}}

在自由电子模型中,我们将电子视作了满足周期性边界条件的平面波(理想气体单粒子能级密度\upref{IdED1}),也就是将电子气体视作了理想气体,来采用相应的能态密度 $D(\epsilon)$.在这样的近似下,我们得到了重要的结果,例如不同能量电子的分布,电子气体对金属热容的贡献等.但这种模型许多严重的缺陷,例如它不能解释为什么有些固体是金属,而有些固体是绝缘体;它不能解释固体丰富的热学、电学或光学性质.为此,我们需要的是\textbf{近自由电子模型},将固体中周期势场作为微扰项考虑进来,考察周期势场下的电子波函数的行为及能带关系.近自由电子模型几乎给出了关于金属中电子行为所有的定性问题的答案.
\subsection{近自由电子的哈密顿量}
设周期性势场 为 $V(\bvec x)$,它包括固体中固定不动的离子实产生的库伦势,以及固体中的巡游电子在 $\bvec x$ 处产生的平均势能.由于我们采用平均势作为近自由电子所受到的势能的近似,我们常称这种方法为\textbf{平均场近似}.
那么我们可以将固体中一个电子的哈密顿量可以近似地表达为
\begin{equation}
H=-\frac{\hbar^2}{2m}\nabla^2+V(\bvec x)=H_0+V(\bvec x)
\end{equation}
$H_0$ 是自由电子哈密顿量.一般而言 $V$ 可以写作 $V$ 在全空间的平均值 $\overline V$ 与 $V(\bvec x)-\overline V$ 的差,即
\begin{equation}
H=H_0+\overline V+(V(\bvec x)-\overline V)
\end{equation}
由此可以将 $V(\bvec x)-\overline V$ 视为微扰项,利用微扰法求解近自由电子的波函数的近似解,即完全哈密顿量 $H$ 的本征值.

由于晶格的周期性,势能函数 $V(\bvec x)$ 也满足周期性条件
\begin{equation}
V(\bvec x)=V(\bvec x+\bvec R_n)
\end{equation}
\subsection{带隙的来源}
让我们以一维单原子链晶格为例,设晶格常数为 $a$,周期性势场满足 $V(x)=V(x-a)$.那么单粒子哈密顿量的本征波函数 $\phi$ 满足
\begin{equation}\label{egasmd_eq1}
\left(-\frac{\hbar^2}{2m}\frac{\dd{}^2}{\dd x^2}+V(x)\right)\phi(x)=E\phi(x)
\end{equation}
将 $V-\overline V$ 视作微扰,令 $\phi(x)=\phi_0(x)+\phi_1(x)+\cdots$,其中 $\phi_1(x)$ 为一阶微扰.
\begin{equation}
-\frac{\hbar^2}{2m}\frac{\dd{}^2}{\dd x^2} \phi_0(x)=(E-\overline V)\phi_0(x)
\end{equation}
解得
\begin{equation}
\phi_0(x)=\frac{1}{\sqrt{L}}e^{ikx}
\end{equation}
对应的能量为 $E=\frac{\hbar^2 k}{2m}+\overline V$.将这个波数为 $k$ 的波函数记为 $|k\rangle$.

回到初始的方程\autoref{egasmd_eq1},在微扰场作用下,对于波数 $k$,能量 $E_k^{0}=\frac{\hbar^2 k}{2m}+\overline V$ 受到微扰,波函数也受到微扰.其他的 $|k'\rangle$ 态会对 $|k\rangle$ 态造成微扰.我们利用一阶不含时微扰理论(量子力学)\upref{TIPT}求解能量的一级修正:
\begin{equation}
E_k^{(1)}=\sum_{k'}\frac{|\langle k'||}{E_k^0-E_{k'}^0}
\end{equation}

\subsection{微扰论与布洛赫函数}