% Julia编程基础
% keys Julia 编程 基础 书 目录

本文转载自郝林的 《Julia 编程基础》.


- 第一部分 预备知识 \\
    - [第 1 章 起步](ch01.md) \\
        - 1.1 初识 Julia \\
        - 1.2 安装和启动 \\
        - 1.3 编写第一个程序 \\
        - 1.4 改进第一个程序 \\
        - 1.5 小结 \\
    - [第 2 章 编程环境](ch02.md) \\
        - 2.1 REPL 环境及其用法 \\
        - 2.2 程序包与环境配置 \\
        - 2.3 项目的创建与引入 \\
        - 2.4 小结 \\
- 第二部分 基本的定义 \\
    - [第 3 章 变量与常量](ch03.md) \\
        - 3.1 变量的定义 \\
        - 3.2 变量的命名 \\
        - 3.3 变量的类型 \\
        - 3.4 常量 \\
        - 3.5 小结 \\
    - [第 4 章 类型系统](ch04.md) \\
        - 4.1 概述 \\
        - 4.2 类型与值 \\
        - 4.3 两个特殊类型 \\
        - 4.4 三种主要类型 \\
        - 4.5 小结 \\
    - [第 5 章 数值与运算](ch05.md) \\
        - 5.1 数值的类型 \\
        - 5.2 整数 \\
        - 5.3 浮点数 \\
        - 5.4 复数和有理数 \\
        - 5.5 常用的数学运算 \\
        - 5.6 数值类型的提升 \\
        - 5.7 数学函数速览 \\
        - 5.8 小结 \\
    - [第 6 章 字符和字符串](ch06.md) \\
        - 6.1 Unicode 字符 \\
        - 6.2 字符 \\
        - 6.3 字符串 \\
        - 6.4 非常规的字符串值 \\
        - 6.5 小结 \\
    - [第 7 章 参数化类型](ch07.md) \\
        - 7.1 类型的参数化 \\
        - 7.2 参数化的更多知识 \\
        - 7.3 容器:元组 \\
        - 7.4 小结 \\
    - [第 8 章 容器:字典与集合](ch08.md) \\
        - 8.1 索引与迭代 \\
        - 8.2 标准字典 \\
        - 8.3 集合 \\
        - 8.4 通用操作 \\
        - 8.5 小结 \\
    - [第 9 章 容器:数组(上)](ch09.md) \\
        - 9.1 类型 \\
        - 9.2 数组的表示 \\
        - 9.3 数组的构造 \\
        - 9.4 数组的基本要素 \\
        - 9.5 访问元素值 \\
        - 9.6 修改元素值 \\
        - 9.7 小结 \\
    - [第 10 章 容器:数组(下)](ch10.md) \\
        - 10.1 广播式的修改 \\
        - 10.2 元素值的排序 \\
        - 10.3 数组的拷贝 \\
        - 10.4 数组的拼接 \\
        - 10.5 数组的比较 \\
        - 10.6 再说数组的构造 \\
        - 10.7 小结 \\
- 第三部分 代码的组织 \\
    - [第 11 章 流程控制](ch11.md) \\
        - 11.1 最简单的代码块 \\
        - 11.2 if 语句 \\
        - 11.3 for 语句 \\
        - 11.4 while 语句 \\
        - 11.5 let 语句 \\
        - 11.6 错误的报告与处理 \\
        - 11.7 小结 \\
    - [第 12 章 函数与方法](ch12.md) \\
        - 12.1 什么是函数 \\
        - 12.2 Julia 中的函数 \\
        - 12.3 基本的编写方式 \\
        - 12.4 函数的参数 \\
        - 12.5 函数的结果 \\
        - 12.6 衍生方法 \\
        - 12.7 函数的参数化 \\
        - 12.8 do 代码块 \\
        - 12.9 小结  \\
    - [第 13 章 模块](#) \\
        - 13.1 程序中的模块 \\
        - 13.2 模块的导入与使用 \\
        - 13.3 模块的引入路径 \\
        - 13.4 标准模块 \\
        - 13.5 小结 \\
- 第四部分 进阶之门 \\
    - [第 14 章 接口编程](#) \\
        - 14.1 什么是接口编程 \\
        - 14.2 接口编程的好处 \\
        - 14.3 接口编程的运用 \\
        - 14.4 两种接口 \\
        - 14.5 小结 \\
    - [第 15 章 关于元编程](#) \\
        - 15.1 什么是元编程 \\
        - 15.2 元编程入门 \\
        - 15.3 进阶用法 \\
        - 15.4 元编程利器:宏 \\
        - 15.5 小结 \\
- 附录 \\
    - [附录 1:重要名词解释](#) \\
    - [附录 2:系统环境变量](#) \\
 \\


原文链接: \href{https://github.com/hyper0x/JuliaBasics/blob/master/book/CATALOG.md}{全书目录}
