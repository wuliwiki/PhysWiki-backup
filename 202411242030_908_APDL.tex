% 安培力定律(综述)
% license CCBYSA3
% type Wiki

本文根据 CC-BY-SA 协议转载翻译自维基百科\href{https://en.wikipedia.org/wiki/Amp\%C3\%A8re\%27s_force_law}{相关文章}。

在静磁学中,两根载流导线之间的吸引或排斥力(见下方第一幅图)通常被称为安培力定律。这种力的物理来源是每根导线根据毕奥-萨伐尔定律产生磁场,而另一根导线则根据洛伦兹力定律受到磁力作用。

**公式**

**特殊情况:两条直平行导线**

安培力定律中最为人熟知且最简单的例子(在2019年5月20日之前[1])用来定义电流的国际单位制(SI)单位**安培**。该定律表述,两条直平行导线之间的每单位长度的磁力为:

\[
\frac{F_m}{L} = 2k_{\text{A}} \frac{I_1 I_2}{r},
\]

其中:
- \( k_{\text{A}} \) 是由毕奥-萨伐尔定律定义的磁力常数;
- \( \frac{F_m}{L} \) 是单位长度上的总磁力(在较短导线上,较长导线被近似为相对于较短导线无限长);
- \( r \) 是两导线之间的距离;
- \( I_1 \) 和 \( I_2 \) 是两导线中传输的直流电流。

这种公式在以下情况下是良好的近似:
1. 如果一根导线的长度远大于另一根,可以将较长的导线近似为无限长;
2. 如果两根导线之间的距离相对于导线的长度较小(使得无限长导线近似成立),但同时相对于导线的直径又较大(使得导线可以被近似为无限细的线)。

\( k_{\text{A}} \) 的值取决于所选的单位系统,而 \( k_{\text{A}} \) 的值决定了电流单位的大小。

在国际单位制(SI)中[2][3]:

\[
k_{\text{A}} = \frac{\mu_0}{4\pi},
\]

其中:
- \( \mu_0 \) 是磁常数(在国际单位中,称为真空磁导率)。

SI单位中,磁常数的值为:
\[
\mu_0 = 1.25663706212(19) \times 10^{-6} \, \text{H/m}.
\]