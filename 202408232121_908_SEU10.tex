% 东南大学 2010 年 考研 量子力学
% license Usr
% type Note

\textbf{声明}:“该内容来源于网络公开资料,不保证真实性,如有侵权请联系管理员”

\textbf{1.(15 分)}设粒子能级 $E_n$ 的简并度为 $f_n$,归一化的能量本征函数为 $\phi_{n\alpha}(r)$ ($\alpha = 1,2,\dots,f_n$),$t = 0$ 时刻粒子的归一化波函数为 $\psi(r,0) = \sum_{n\alpha} c_{n\alpha} \phi_{n\alpha}(r)$,试求:

\begin{enumerate}
    \item $t$ 时刻粒子的波函数 $\psi(r,t)$;
    \item $t$ 时刻粒子的能量平均值 $ \overline{H} $。
\end{enumerate}

\textbf{2.(15 分)}设 $\psi_1(x)$ 和 $\psi_2(x)$ 均为与能级 $E$ 对应的能量本征函数,试证:
\begin{enumerate}
    \item $\psi_1 \psi_2' - \psi_1' \psi_2 = c$ (常数);
    \item 若 $\psi_1$ 和 $\psi_2$ 均为束缚态波函数,则 $\psi_1 \psi_2' = \psi_1' \psi_2$。
\end{enumerate}

\textbf{3.(15 分)}质量为 $m$ 的粒子处于一维势场中 $V(x) = 0, (0 < x < a)$, $V(x) = \infty, (x < 0, x > a)$,试求能量本征值和归一化的能量本征函数。

\textbf{4.(15 分)}设粒子的能量 $E > V_0$ 从左入射,碰到势场 $V(x) = 0, (x < 0)$, $V(x) = V_0, (x > a)$,在 $0 < x < a$ 的区域 $V(x)$ 是连续有限的函数,反射系数和透射系数分别为 $r$ 和 $t$。试证明:$r + t = 1$。

提示:几率流密度公式为 $j(x) = -(i\hbar/2m)(\psi^*\psi'  - \psi\psi^{*'})$。

\textbf{5.(15 分)}试用测不准关系估计以下体系的基态能量:
\begin{enumerate}
    \item 质量为 $m$ 的粒子处于长度为 $a$ 的一维无限深方势阱中;
    \item 频率为 $\omega$ 的一维谐振子。
\end{enumerate}

\textbf{6.(15 分)}设平面转子的哈密顿量为 $\hat{H} = \hat{l}_z^2/2I$, ($\hat{l}_z = -i\hbar \partial/ \partial \varphi$),试求归一化的能量本征函数、能量本征值、能级简并度。

\textbf{7.(15 分)}设量子体系的哈密顿量为 $\hat{H}$,力学量算符 $\hat{A}$ 不显含时间 $t$,$t$ 时刻的量子态为 $\psi(t)$,$\hat{A}$ 的平均值为 $\overline{A}(t)$,求证:
\[
i\hbar d\overline{A}(t)/dt = \overline{[\hat{A}, \hat{H}]}~
\]

\textbf{8.(15 分)}电子总角动量为 $\hat{j} = \hat{l} + \hat{s}$,设 $|l, j, m_j \rangle$ 为 $\hat{l}^2$, $\hat{j}^2$, $\hat{j}_z$ 的共同本征态:

\begin{enumerate}
    \item 在此态下,$\hat{l}^2$, $\hat{j}^2$, $\hat{j}_z$ 的本征值各是多少?
    \item 利用 $\hat{j}^2 = \hat{l}^2 + \hat{s}^2 + 2\hat{l} \cdot \hat{s}$,证明 $|l j m_j\rangle$ 也是 $\hat{l} \cdot \hat{s}$ 的本征态,并求本征值。
\end{enumerate}

\textbf{9.(15 分)}设缺金属原子的价电子处于中心力场 $V(r)$ 中,哈密顿量为 $H_0 = p^2/2\mu + V(r)$,守恒量完全集合 $\{H_0, L^2, L_z, s_z\}$ 的共同本征态为 $|n l m\rangle$,能级为 $E_{nl}$:

\begin{enumerate}
    \item  若沿 $z$ 方向外加强磁场 $B_z$,则价电子的哈密顿算符变为 $\hat{H} = H_0 + \omega L_z$,其中 $\omega = eB/2\mu c$,试求相应的能量本征态和能量本征值。
    \item  求存在强磁场 $B$ 时,原子从 $P$ 态 ($l = 1$) 跃迁到 $S$ 态 ($l = 0$) 的光谱线频率 (跃迁过程中保持 $n_r$ 不变)。
\end{enumerate}

\textbf{10.(15 分)}