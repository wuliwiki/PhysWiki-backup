% QED的重整化理论-顶点函数的单圈修正
% QED的费曼规则|顶点函数|形状因子

\pentry{QED的重整化理论-电子自能和光子自能的单圈修正\upref{qedlop},Ward-Takahashi 等式\upref{ward}}

在电子自能和光子自能的单圈修正\upref{qedlop} 中,我们对裸拉氏量中的场量进行缩放得到了重整化的拉氏量
\begin{equation}
\begin{aligned}
\mathcal{L} &= Z_2\bar\psi (i\not\partial - m_0)\psi - \frac{1}{4}Z_3 (F^{\mu\nu})^2-e_0Z_2Z_3^{1/2} \bar\psi \gamma^\mu\psi A_\mu\\
&=\bar\psi (iZ_2\not\partial - Z_m m)\psi - \frac{1}{4}Z_3 (F^{\mu\nu})^2-Z_1 e \bar\psi \gamma^\mu\psi A_\mu
\end{aligned}
\end{equation}
其中我们重新定义了 $Z_1e=e_0 Z_2 Z_3^{1/3},Z_m m=Z_2 m_0$,$m,e$ 为物理质量和物理电荷。我们可以在适当的重整化条件下确定各个重整化常数。

\subsection{$Z_1=Z_2$}
让我们从 Ward 等式和裸拉氏量出发给出 $Z_1=Z_2$ 的一个证明。考虑两条电子外线、一条光子外线所组成的三点函数
\begin{equation}
k_\mu \mathcal{M}^\mu(k,p,p+k)=e_0(\mathcal{M}^{\mu}(p)- \mathcal{M}^{\mu}(k))
\end{equation}
利用三点函数 Feynman 规则(\autoref{qedfey_the1}~\upref{qedfey}),并且保留电子外线的正规传播子(它有在壳极点行为),并截去光子外线的正规传播子(也就是直接连在截肢图上,这不影响 Ward 等式的正确性,因为只有直接连在截肢图上才能对 $k_\mu \mathcal{M}^\mu$ 有贡献),将光子外线贡献用 $k_\mu$ 替代。我们总是可以将三点函数的 Feynman 图表示为 QED 正规顶点和两条电子外腿图的组合:
\begin{equation}\label{qedvet_eq2}
k_\mu \mathcal{M}^\mu(k,p,p+k) = S(p+k) [-ie_0k_\mu \Gamma^\mu(p+k,p)] S(p) =e_0[S(p)-S(p+k)]
\end{equation}
其中
\begin{equation}
S(p)=iZ_2/(\not p-m+i\epsilon)+(\text{regular term at } p^2=m^2)
\end{equation}
为电子的正规传播子(两点编时格林函数),$Z_2$ 为裸拉氏量的正规传播子在 $\not p=m$ 处的留数。对\autoref{qedvet_eq2} 两边除以 $S(p)S(p+k)$,可以得到
\begin{equation}
-ik_\mu \Gamma^\mu(p+k,p) = [S(p+k)^{-1}-S(p)^{-1}]= Z_2^{-1}\not k
\end{equation}
根据 LSZ 约化公式(\upref{lszqed}和\upref{lszspn}),三点函数所对应的 Feynman 振幅为
\begin{equation}
i\mathcal{M} = Z_2(Z_3)^{1/2}\epsilon_\mu(k)\bar u(p+k)[-ie_0\Gamma^\mu] u(p)=\epsilon_\mu(k)\bar u(p+k)[-i e Z_1\Gamma^\mu] u(p)
\end{equation}
由于 $|\mathcal{M}|^2$ 是可观测量,且在非相对论极限下,从上式出发应当正确地得到库伦定律。非相对论极限下的计算告诉我们,在 $k\rightarrow 0$ 的极限下,$Z_1 \Gamma^\mu(k=0) = \gamma^\mu$。代入


\subsection{顶点函数的洛伦兹结构}
我们常常将 $i\mathcal{M}(e^-(p)\gamma(q)\rightarrow e^-(p'))=\bar u(p') [-ie\Gamma^\mu(p,p')]u(p) \epsilon_\mu(q)$ 称作 QED 的正规顶点,它对应于两条电子外线、一条光子外线的所有截肢费曼图的贡献之和。将 $[-ie\Gamma^\mu(p,p')]$ 称为 QED 的顶点函数,它仍保持着四矢量的洛伦兹结构。

让我们来先分析 $\Gamma^\mu$ 的洛伦兹结构。由于它服从洛伦兹四矢量的变换规则,且是 $4\times 4$ 矩阵,它一定具有以下的形式:
\begin{equation}
\begin{aligned}
\Gamma^\mu=\gamma^\mu\cdot A + ({p'}^\mu + p^\mu)\cdot B + ({p'}^\mu - p^\mu) \cdot C
\end{aligned}
\end{equation}
其中 $A,B,C$ 是 Dirac 代数中的元素,可以表示为 $1,\gamma^\mu,\sigma^{\mu\nu}=\frac{i}{2}[\sigma^\mu,\sigma^\nu],\gamma^\mu\gamma^5,\gamma^5$ 这 $16$ 个元素的线性组合,且它们是洛伦兹标量。又因为在旋量 QED 中,任何阶 Feynman 图的任何量中都没有出现 $\gamma^5$,所以我们只有 $1,\not p^{(1)},p_\mu^{(1)} p_\nu^{(2)} \sigma^{\mu\nu}=\frac{i}{2}[\not p^{(1)},\not p^{(2)}]$ 这几种可能。$A,B,C$ 作为洛伦兹标量矩阵,可以由这组基底展开。

现在假设电子外线在壳,$p^2=p'^2=m^2$,那么可以利用 $\bar u(p')(\not p'-m)= (\not p-m)u(p)=0$,将 $A,B,C$ 中所有 $\not p,\not p'$ 都可以被替换为 $m$,如果多个 $\not p,\not p'$ 相乘也可以通过等式 $\not p\not p'=2p\cdot p'-\not p'\not p$ 的方式交换位置最后被替换为单位矩阵。因此最终 $A,B,C$ 可以表示成洛伦兹标量函数乘以单位矩阵的形式。由于外动量只有 $p,p',q=p'-p$,洛伦兹标量一定是关于独立标量 $p^2,{p'}^2,p\cdot p'$ 的函数。$p^2={p'}^2=m^2$ 是常数,$q^2=(p'-p)^2 = -2p'p$,因此 $A,B,C$ 都可以表示为 $q^2$ 的函数。

我们再来考察 Ward 等式(或者规范对称性)对顶点函数的限制。Ward 等式要求
\begin{equation}
\begin{aligned}
&\bar u(p')\Gamma^\mu u(p) (p'_\mu-p_\mu) = 0\\
&\Rightarrow \bar u(p')\left[A(q^2)(\not p'-\not p)+ C(q^2)(p'-p)^2\right] u(p)=\bar u(p')C(q^2)q^2 u(p)=0
\end{aligned}
\end{equation}
Ward 等式对于不在壳的虚光子也成立,这说明 $C(q^2) = 0$。综合前面的讨论,$\Gamma^\mu$ 可以表示成
\begin{equation}\label{qedvet_eq1}
\begin{aligned}
\Gamma^\mu = \gamma^\mu\cdot A(q^2) + ({p'}^\mu + p^\mu)\cdot B(q^2),\quad p^2=(p')^2=m^2
\end{aligned}
\end{equation}
或者我们也可以利用 Gordon 等式:
\begin{equation}
\bar u(p') \left[\frac{{p'}^\mu+p^\mu}{2m}+\frac{i\sigma^{\mu\nu}q_\nu}{2m}\right]u(p)=\bar u(p')\gamma^\mu u(p)
\end{equation}
可以将\autoref{qedvet_eq1} 改写为另一种形式
\begin{equation}
\Gamma^\mu = \gamma^\mu F_1(q^2) + \frac{i\sigma^{\mu\nu}q_\nu}{2m} F_2(q^2),\quad p^2=(p')^2=m^2
\end{equation}
$F_1(q^2) = A+2mB,F_2(q^2)=-2mB$ 被称为 QED 正规顶点的\textbf{形状因子}。
\subsection{顶点函数的单圈修正}
下面我们所进行的讨论都是基于重整化的拉氏量。
\begin{equation}
\mathcal{L}=\bar\psi (iZ_2\not\partial - Z_m m)\psi - \frac{1}{4}Z_3 (F^{\mu\nu})^2-Z_1 e \bar\psi \gamma^\mu\psi A_\mu
\end{equation}
类似电子自能和光子自能的单圈修正\upref{qedlop}中的讨论,我们先在 OS 重整化方案下进行讨论,用维数正规化处理发散积分。根据重整化条件,我们希望当电子光子外线在壳、且光子四动量为 $0$ 时,有
\begin{equation}
-ie\Gamma^\mu(p,p')|_{q=0,p^2={p'}^2=m^2} = -ie\gamma^\mu
\end{equation}
这表明在充分大的空间尺度上测得的物理电荷量为 $e_{\text{ph}}=e$。重整化条件要求 $F_1(q^2=0)=1$,于是我们可以通过圈图修正计算形状因子 $F_1(q^2)$ 来确定重整化常数 $Z_1 = 1+\delta_1$。

