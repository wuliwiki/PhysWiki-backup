% 电子
% license CCBYSA3
% type Wiki

(本文根据 CC-BY-SA 协议转载自原搜狗科学百科对英文维基百科的翻译)

\textbf{电子}是一种亚原子粒子,符号是$e^-$或者β−,其电荷量是负一个单位的基本电荷。[1]电子属于第一代轻子族,且通常被认为是基本粒子,因为它们没有已知的成分或亚结构。电子的质量大约是质子质量的1/1836。 电子的量子力学性质包括半整数值的内禀角动量(自旋),单位为约化普朗克常数ħ。根据泡利不相容原理,作为费米子,没有两个电子可以占据相同的量子态。像所有基本粒子一样,电子表现出波粒二象性:它们可以与其他粒子发生碰撞,也可以像光一样发生衍射。比其他粒子(例如中子和质子)更容易通过实验观察到,因为电子的质量更低,对于给定的能量电子拥有更大的德布罗意波长。

电子在许多物理现象中起着重要作用,如电学、磁学、化学和热导率,电子还参与重力、电磁相互作用和弱相互作用。[2]因为电子带有电荷,所以它周围有一个电场,如果电子相对于某个观察者发生运动,这个观察者将观察到它产生一个磁场。其他来源所产生的电磁场将影响电子的运动,这种影响由洛伦兹力定律所描述。当电子被加速时,它们以光子的形式辐射或吸收能量。实验室仪器能够通过使用电磁场囚禁单个电子和电子等离子体。特殊的望远镜可以探测外层空间的电子等离子体。电子涉及许多应用,例如电子学、焊接、阴极射线管、电子显微镜、放射疗法、激光器、气体电离探测器和粒子加速器。

电子与其他亚原子粒子的相互作用在化学和核物理等领域都很有意义。原子核内带正电荷的质子和没有原子核的负电荷电子之间的库仑力相互作用允许它们共同组成原子。电离或负电荷电子与正电荷原子核之间的电荷量差异改变了原子系统的结合能。两个或多个原子之间的电子交换或共享是化学键形成的主要原因。1838年,英国自然哲学家理查德·拉明首先假设了一个不可分割的电荷量的概念来解释原子的化学性质。爱尔兰物理学家乔治·约翰斯顿·斯通尼在1891年将这种电荷命名为“电子”,而约瑟夫·汤姆孙和他的英国物理学家团队在1897年将它确定为粒子。电子也可以参与核反应,例如恒星中的核合成,在其中电子被称为贝塔粒子。电子可以通过放射性同位素的β衰变以及高能碰撞产生,后者的一个例子发生在宇宙射线进入大气层的时候。电子的反粒子被称为正电子;它与电子拥有很多相同的性质,除了它携带有与电子相反的电荷以及其他的荷。当一个电子与正电子碰撞时,两个粒子可以发生湮灭,并产生γ射线光子。