% Matlab 球坐标中的分布图
% keys Matlab|分布图|曲面|网格
% license Xiao
% type Tutor

\begin{issues}
\issueAbstract
\end{issues}

\begin{figure}[ht]
\centering
\includegraphics[width=7cm]{./figures/3d39f81dd2f543ec.png}
\caption{示例 1} \label{fig_MatPol_1}
\end{figure}
\begin{figure}[ht]
\centering
\includegraphics[width=9cm]{./figures/dda39e8200a02432.png}
\caption{示例 2} \label{fig_MatPol_2}
\end{figure}

我们可以用 Matlab 内建的 \verb|surf(X,Y,Z,Val)| 函数在空间中画出任意曲面。 其中 \verb|X,Y,Z,Val| 是尺寸相同的 2 维矩阵, 分别代表曲面上每个网格的各点。
\begin{lstlisting}[language=matlab, caption=surfSph.m]
% surf() in spherical coordinate
% usage 1. R,Th,Ph,Val are 2D matrices in same shape
% usage 2. R,Th,Ph are vector,vector,scalar
% usage 3. R,Th,Ph are vector,scalar,vector
function h = surfSph(R,Th,Ph,Val,varargin)
% resize
if isvector(R) && isvector(Th)
    [R,Th] = ndgrid(R, Th);
end
if isvector(R) && isvector(Ph)
    [R,Ph] = ndgrid(R,Ph);
end
if isscalar(Th)
    Th = Th*ones(size(R));
end
if isscalar(Ph)
    Ph = Ph*ones(size(R));
end
if ~isequal(size(R),size(Th)) || ...
    ~isequal(size(Th),size(Ph)) || ...
    ~isequal(size(Ph),size(Val))
    error('wrong shape for Th, Ph, Val');
end
% plot
[X,Y,Z] = Sph2Cart(R, Th, Ph);
h = surf(X,Y,Z,Val,varargin{:});
view(0,0);
shading flat; axis equal;
xlabel x; ylabel y; zlabel z;
Rmax = max(R(:));
axis([-1, 1, -1, 1, -1, 1]*Rmax);
set(datacursormode(gcf), 'UpdateFcn', @datatip);
cameratoolbar;
end
\end{lstlisting}

\subsubsection{示例 1}
\begin{lstlisting}[language=matlab, caption=surfSph\_demo1.m]
Nr = 200; Nth = 150;
R = linspace(0, 10, Nr);
Th = linspace(0, 2*pi, Nth);
[R, Th] = ndgrid(R, Th);
Ph = 0;
Val = cos(R) .* cos(Th).^2;
figure; surfSph(R,Th,Ph,Val);
title('cos(r)cos^2(\theta)');
view(0, 0);
\end{lstlisting}
运行结果如\autoref{fig_MatPol_1}。

\subsubsection{示例 2}
\begin{lstlisting}[language=matlab, caption=surfSph\_demo2.m]
Nr = 200; Nth = 150;
R = linspace(0, 10, Nr);
Ph = linspace(0, 2*pi, Nth);
[R, Ph] = ndgrid(R, Ph);
Val = cos(R) .* cos(Ph).^2;
figure; surfSph(R,pi/6,Ph,Val);
hold on; surfSph(R,pi/3,Ph,Val);
surfSph(R,pi/2,Ph,Val);
xlabel x; ylabel y; zlabel z;
title('cos(r)cos^2(\phi)');
view(30, 16);
\end{lstlisting}
运行结果如\autoref{fig_MatPol_2}。
