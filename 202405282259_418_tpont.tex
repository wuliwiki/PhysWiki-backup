% 数论函数 theta 与 psi 函数的阶
% license Usr
% type Tutor

\pentry{数论函数\nref{nod_NumFun},渐进估计与阶\nref{nod_asymeq}}{nod_5d92}
\subsection{阶}
\begin{theorem}{}
函数 $\vartheta(x)$ 与 $\psi(x)$ 的阶是 $x$:

对于足够大的 $x$($x \ge 2$),
\begin{equation}
Ax < \vartheta(x) < Bx, ~ Cx < \psi(x) < Dx ~ ~.
\end{equation}

\end{theorem}

\subsection{证明}
我们首先引入一个引理并予以证明。
\begin{lemma}{}
\begin{equation}
\psi(x) = \vartheta(x) + \mathcal O\left( x^{1/2} (\ln x)^{1/2} \right) ~.
\end{equation}

\end{lemma}
\textbf{证明}:考虑 $\psi(x)$ 的定义,由于 $p^2 \le x$, $p^3 \le x$, $\dots$ 等价于 $p \le x^{1/2}$, $p \le x^{1/3}$, $\dots$,