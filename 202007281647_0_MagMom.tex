% 磁矩

\pentry{安培力\upref{FAmp}}

当我们把一个通有电流的线圈放置在磁场中时, 这个线圈往往回受到安培力产生的力矩. 本词条讨论如何计算一些简单的情况, 先看一道例题.

\begin{example}{匀强磁场中的长方形线圈}
假设一个粗细可以忽略不计的长方形电流环路被放置在匀强磁场 $\bvec B$ 中(图未完成), 两条边的边长分别为 $a, b$, 电流为 $I$. 线圈平面的法向量和磁场夹角为 $\theta$, 边长为 $a$ 边始终垂直于磁场. 求
\end{example}



\begin{equation}\label{MagMom_eq1}
\bvec \mu = I \bvec A
\end{equation}
其中 $\bvec A$ 为过回路

产生的力矩为
\begin{equation}
\bvec \tau = \bvec \mu \times \bvec B
\end{equation}

\subsection{旋转的电荷}
对于绕轴做圆周运动的点电荷, 令角速度为 $\omega$, 等效电流为 $q/T$, $T = 2\pi/\omega$ 是转动周期, 圆周面积 $A = \pi r^2$, 代入\autoref{MagMom_eq1} 得
\begin{equation}
\bvec \mu = \frac{q}{2} \omega r^2
\end{equation}
