% 线性方程组解的结构
% 线性方程组|矢量空间|秩|矩阵

\begin{issues}
\issueDraft
\end{issues}

\pentry{矩阵与线性变换\upref{MatLS}, 高斯消元法\upref{GAUSS}, 矩阵的秩\upref{MatRnk}}

线性方程组可以记为
\begin{equation}\label{LinEq_eq1}
\mat A \bvec x = \bvec y
\end{equation}
其中 $\mat A$ 是 $M \times N$ 的矩阵, $\bvec x$ 是 $N$ 维列矢量, $\bvec y$ 是 $M$ 维列矢量, $\mat A \bvec x$ 表示矩阵与列矢量相乘(\autoref{Mat_eq4}~\upref{Mat}). $\mat A$ 和 $\bvec y$ 是已知的, $\bvec x$ 是未知的, 被称为方程组的\textbf{解(solution)}.

\begin{figure}[ht]
\centering
\includegraphics[width=14cm]{./figures/LinEq_1.png}
\caption{Ax=b解的性质}} \label{LinEq_fig1}
\end{figure}

从矢量空间\upref{LSpace}的角度来看, $\bvec x$ 是一个 $N$ 维矢量空间(以下称为 $X$ 空间)中一个矢量关于某组基底的坐标, $\bvec y$ 是一个 $M$ 维矢量空间(以下称为 $Y$ 空间)中一个矢量关于某组基底的坐标. 矩阵 $\mat A$ 可以将 $X$ 空间中的任意矢量映射到 $Y$ 后的坐标.

我们知道 $\mat A$ 的第 $i$ 列代表的矢量就是 $X$ 空间中的第 $i$ 个基底映射到 $Y$ 空间的对应矢量. 我们把 $A$ 的 $N$ 列对应的 $N$ 个矢量记为 $\{\bvec \alpha_i\}$. 先来看一个定理

\subsection{满秩方阵}
我们知道矩阵的秩 $R$ 等于线性无关的行数或列数\upref{MatRnk}, 下面来根据秩来分类讨论方程组的解空间结构. 最简单的情况是 $\mat A$ 为满秩, 即 $R = M = N$. 这时由于 $\{\bvec \alpha_i\}$ 两两线性无关, 它们可以作为 $Y$ 空间的一组基底, 与 $X$ 空间的基底一一对应. 那么这个映射既是单射又是满射.%未完成:引用
对于 $Y$ 空间的任意矢量 $\bvec y$, $X$ 空间都存在唯一的解 $\bvec x$. 特殊地,当 $\bvec y = \bvec 0$ 时(即方程是\textbf{齐次}的),唯一解就是 $\bvec x = \bvec 0$.

\subsection{$R = M < N$}\label{LinEq_sub1}

当 $\mat A$ 的秩等于 $M$ 且小于 $N$ 时, 映射变为从 $N$ 维空间到更小的 $M$ 维空间. 即非单射: 虽然任意的 $\bvec x$ 仍然映射到唯一的 $\bvec y$, 但任意的 $\bvec y$ 却对应无穷多个 $\bvec x$. 

% 举例未完成:三维矢量投影到二维矢量

当方程是齐次的时候, 零空间(\autoref{LinMap_the1}~\upref{LinMap})$X_0$ 是 $N- M$ 维的(为什么?). 这种情况下,我们希望能解出零空间的 $N - M$ 个基底,使得这组基底的任意线性组合都是齐次方程的解.

对于非齐次方程, 我们可以先求对应的齐次方程组的零空间的一组基底,再求出非齐次方程的任意一个解(\textbf{特解}), 那么非齐次方程组的\textbf{解集}(所有解的集合)就等于零空间中的所有矢量与特解相加. 注意非齐次方程的解集并不构成一个矢量空间, 因为它不包含零矢量($\bvec x = \bvec 0$ 总是对应 $\bvec y = \bvec 0$, 所以不可能是非齐次方程组的解),解集中若干矢量的线性组合也不一定仍然属于解集.
\addTODO{证明}

\subsection{$R < M$}
当 $R < M$ 时, $\{\bvec \alpha_i\}$ 中只有 $R$ 个线性无关, 它们在 $Y$ 空间中张成\upref{VecSpn}一个 $R$ 维子空间 $Y_0$. 如果 $\bvec y$ 在 $Y_0$ 中(可以通过 $\bvec y$ 是否与 $\{\bvec \alpha_i\}$ 线性无关来判断), 方程组就存在解, 如果落在子空间外, 方程组就无解.

\footnote{see also 《线性变换与矩阵的代数关系》一节的 \autoref{linmat_the2}~\upref{linmat}}
