% 乔治·斯托克斯(综述)
% license CCBYSA3
% type Wiki

本文根据 CC-BY-SA 协议转载翻译自维基百科\href{https://en.wikipedia.org/wiki/Sir_George_Stokes,_1st_Baronet}{相关文章}。


乔治·加布里埃尔·斯托克斯爵士,第一代从男爵(/stoʊks/;1819年8月13日-1903年2月1日),是爱尔兰数学家和物理学家。斯托克斯出生于爱尔兰斯莱戈郡,在剑桥大学度过了整个职业生涯,并在1849年至1903年去世期间担任卢卡斯数学教授长达54年,是该职位任期最长的持有者。

作为物理学家,斯托克斯在流体力学领域作出了开创性的贡献,包括纳维-斯托克斯方程;在物理光学方面,他的研究涵盖偏振和荧光等现象。作为数学家,他普及了矢量微积分中的“斯托克斯定理”,并对渐近展开理论作出了贡献。斯托克斯与菲利克斯·霍普-塞勒一道,首次揭示了血红蛋白的携氧功能,并展示了血红蛋白溶液通气后所产生的颜色变化。

1889年,斯托克斯被英国君主封为从男爵。1893年,他因“在物理科学领域的研究与发现”获得当时全球最负盛名的科学奖项——皇家学会的科普利奖章。他曾于1887年至1892年在英国下议院担任剑桥大学选区的议员,隶属保守党。斯托克斯还于1885年至1890年担任皇家学会会长,并曾短暂出任剑桥大学彭布罗克学院院长。由于他的大量通信往来以及担任皇家学会秘书期间的工作,他被称为维多利亚时代科学的大门守卫者,其贡献远远超越了他发表的论文本身\(^\text{[1]}\)。
\subsection{传记}
乔治·斯托克斯是加布里埃尔·斯托克斯牧师(1834年去世)的幼子。加布里埃尔是爱尔兰圣公会的牧师,担任斯莱戈郡斯克林的教区牧师;其母为伊丽莎白·霍顿,是约翰·霍顿牧师的女儿。斯托克斯的家庭生活深受父亲福音派新教信仰的影响,他的三个兄弟也都进入教会,其中最杰出的是阿马郡副主教约翰·惠特利·斯托克斯\(^\text{[2]}\)。斯托克斯自幼对新教信仰虔诚,他在斯克林的童年经历也对他后来的研究方向产生了重大影响,尤其是他选择流体力学作为研究领域\(^\text{[3]}\)。他的女儿伊莎贝拉·汉弗莱斯曾写道,她父亲“告诉我,他小时候在斯莱戈海岸游泳时差点被一股大浪卷走,这第一次引起了他对波浪的兴趣”\(^\text{[4]}\)。

约翰与乔治兄弟情深,乔治在都柏林上学期间就住在约翰家中。在所有家庭成员中,他与妹妹伊丽莎白关系最为亲近。家族中回忆其母亲是“美丽但非常严厉”的。斯托克斯在斯克林、都柏林和布里斯托尔的学校接受教育,1837年,他进入剑桥大学彭布罗克学院。四年后,他以“高级数学状元”和史密斯一等奖毕业,凭此成就他被选为该学院研究员\(^\text{[5]}\)。

根据当时学院章程,斯托克斯于1857年结婚时必须辞去研究员职务。十二年后,在新章程下,他重新当选为研究员,并一直担任此职直到1902年。在他83岁生日的前一天,他被选为彭布罗克学院院长。然而他担任院长时间不长,于次年(1903年)2月1日在剑桥去世,并被安葬于米尔路公墓\(^\text{[6]}\)。他在西敏寺北侧走廊也有一块纪念碑\(^\text{[7]}\)。
