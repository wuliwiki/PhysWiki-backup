% C++ 标准库常用容器

\begin{issues}
\issueDraft
\end{issues}

\subsection{pair}
\begin{itemize}
\item 例如 \verb|pair<string, int> p("abc", 123);|, \verb|p.first|, \verb|p.second|
\item 当且仅当 \verb|p.first, p.second| 都相等, \verb|==| 才会相等.
\item 比较大小时, 仅需要 \verb|<| 对两种类型都有定义. 若 \verb|p.first < q.first| 则 \verb|p < q|. 若 \verb|p.first == q.first|(根据 \verb|<| 来确定, 不需要使用 \verb|==|), 且 \verb|p.second < q.second|, 也有 \verb|p < q|.
\end{itemize}

\subsection{unordered_map}
\begin{itemize}
\item \verb|unordered_map<key类型, val类型>|
\item \verb|umap.count(key)| 如果存在返回 1, 否则返回 0, 不可以用多个 key.
\item \verb|key| 如果不存在, \verb|umap[key]| 返回默认值 (例如 \verb|int| 返回 \verb|0|)
\item \verb|key| 即使不存在, 也可以视为 lvalue, 例如直接用 \verb|umap[key]++|, \verb|&umap[key]| 之类的. 用了之后就存在了.
\item 对每个 element 循环用 \verb|for (auto &e : umap.)|, 每个 \verb|e| 是一个 \verb|pair<>|, 分别是 key, value.
\item 如果要用 \verb|pair| 作为 key, 定义以下类函数 \verb|hash_pair|, 并声明 \verb|unordered_map<key类型, val类型, hash_combine>|
\begin{lstlisting}[language=cpp]
template<class T> // from boost library
inline void hash_combine(size_t &seed, const T &v) {
    seed ^= hash<T>{}(v) + 0x9e3779b9 + (seed << 6) + (seed >> 2);
}

struct hash_pair { // similar to std::hash, for pair<>
    template<class T, class T1>
    size_t operator()(const pair<T,T1> &a) const {
        size_t h = 0;
        hash_combine(h, a.first);
        hash_combine(h, a.second);
        return h;
    }
};
\end{lstlisting}
\item unordered_map 会先用 hash 函数查找 key, 如果有 hash collilsion 也没关系, 会进一步对比区分.
\end{itemize}

\subsection{unordered_set}
\begin{itemize}
\item 和 \verb|unordered_map| 类似, 不过只有 key 么有 value
\item \verb|unordered_set<key类型, hash类型(默认 std::hash), ...>|
\end{itemize}

\subsection{queue}
\begin{itemize}
\item \verb|queue<class T, class Container = deque<T>>| 像排队一样, 后面进, 前面出. 不支持随机访问. 如果要 print, 可以复制一个, 然后边 print 边 pop. 事实上, \verb|deque| 是可以 iterate 以及随机访问的.
\item 成员函数: \verb|empty, size, front, back, push, emplace, pop, swap|, swap 交换两个 queue 的内容: \verb|p.swap(q);|, 相当于 \verb|std::swap(p, q)|. queue 本身并不实现这些功能, 只是通过调用 \verb|Container| 的成员函数来实现(container adaptor).
\item 其中 \verb|Container| 类型至少应该支持 \verb|empty, size, front, back, push_back, pop_front|, 上一条的功能都是通过调用这些实现的.
\end{itemize}

\subsection{stack}
\begin{itemize}
\item \verb|stack <class T, class Container = deque<T>>|
\item 同样是 container adaptor, 成员函数: \verb|empty, size, top, push, emplace, pop, swap|
\item \verb|Container| 至少支持的成员函数: \verb|empty, size, back, push_back, pop_back|
\item 不支持随机访问.
\end{itemize}

\subsection{deque}
\begin{itemize}
\item double ended queue: \verb|deque<class T, class Alloc = allocator<T> >|
\item 成员函数(基本是 \verb|vector| 的拓展版): \verb|begin, end, size, max_size, resize, empty, operator[], front, back, push_front, pop_back, pop_front, pop_back, insert, erase, swap, clear, emplace, emplace_front, emplace_back|
\end{itemize}
