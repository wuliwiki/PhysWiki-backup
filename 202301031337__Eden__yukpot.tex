% Yukawa 势
% Yukawa 势|Yukawa 理论|非相对论近似|强相互作用

\pentry{散射理论与 S 矩阵\upref{Smat}}
Yukawa 提出一个费米子与标量粒子耦合的相互作用理论,它的作用量可以写为
\begin{equation}
\begin{aligned}
\mathcal{L}=\bar\psi (i\gamma^\mu-m) \partial_\mu \psi+
\frac{1}{2}(\partial^\mu \phi \partial_\mu \phi + m_\phi^2 \phi^2) + g\bar\psi\phi\psi
\end{aligned}
\end{equation}
其中 $g\bar\psi \phi\psi$ 意味着费米子与标量粒子耦合的相互作用,其耦合常数为 $g$。Yukawa 用这一理论给出了核子间强相互作用的一种解释。在非相对论极限下,两个费米子(或反费米子)之间的非相对论势能为
\begin{equation}
\begin{aligned}
V(r)=-\frac{g^2}{4\pi} \frac{1}{r} e^{- m_\phi r}
\end{aligned}
\end{equation}
这是一个吸引势。由于有一个 $e^{-m_\phi r}$ 的随距离增加而减小的因子,Yukawa 相互作用是一个短程的相互作用。下面我们通过计算一阶树图的费曼矩阵元,利用 Born 近似给出非相对论情形下 Yukawa 势的结果。

\subsection{$e^-e^-\rightarrow e^-e^-$ 的 Yukawa 势}

我们先计算两体散射过程 $\ket{{\bvec p},s,+;{\bvec k},r,+}\rightarrow \ket{{\bvec p}',s',+;{\bvec k}',r',+}$ 的费曼矩阵元忽略高圈图的贡献,我们只保留其树图的贡献。在这里 $+$ 代表费米子,$-$ 代表反费米子。要注意 $\ket{{\bvec p}',s',+;{\bvec k}',r',+}$ 对应的左矢是 $\bra{\bvec p',s',+;\bvec k',r',+}=\bra 0 a_{\bvec k'}^{r'} a_{\bvec p'}^{s'}$,在我们计算过程中产生湮灭算符的顺序对结果的正负号非常重要。

最低阶的树图有两个,利用 Wick 定理写出其对应的两种 wick 缩并,再利用 Feynman 规则写出它对 Feynman 矩阵元的贡献:
\begin{equation}\label{yukpot_eq1}
\begin{aligned}
\notag i\mathcal{M}&=
\frac{(-ig)^2}{2!}\int \dd[4]{x} \int  \dd[4]{y} 
{
\langle\overset{1}{\bvec p',s',+};\overset{2}{\bvec k',r',+}|
(\overset{3}\phi  \overset{1}{\bar{\psi}} \overset{4}\psi)_x 
(\overset{3}\phi \overset{2}{\bar{\psi}} \overset{5}\psi)_y
|\overset{4}{\bvec p,s,+};\overset{5}{\bvec k,r,+} \rangle
}+(x\leftrightarrow y)\\
&\quad +
\frac{(-ig)^2}{2!}\int \dd[4]{x} \int  \dd[4]{y} 
\langle\overset{1}{\bvec p',s',+};\overset{2}{\bvec k',r',+}|
(\overset{3}\phi  \overset{1}{\bar{\psi}} \overset{4}\psi)_x 
(\overset{3}\phi \overset{2}{\bar{\psi}} \overset{5}\psi)_y
|\overset{5}{\bvec p,s,+};\overset{4}{\bvec k,r,+} \rangle
+(x\leftrightarrow y)\\
&=(-ig)^2\left(
 \bar u^{s'}(p')u^s(p) \frac{i}{(p'-p)^2-m_{\phi}^2} \bar u^{r'}(k') u^r(k)
\right.\\
&\left.\quad 
-\bar u^{s'}(p')u^r(k) \frac{i}{(p'-k)^2-m_{\phi}^2} \bar u^{r'}(k') u^{s}(p)  \right)
\end{aligned}
\end{equation}
在非相对论极限下,两个费米子之间是可区分的,那么第二项可以被忽略。此外由于 $|\bvec p|\ll 1$,有如下的近似:$p=(m,{\bvec p})+O(|{\bvec p}|^2),k=(m,{\bvec k})+O(|{\bvec k}|^2),\cdots$,$(p-k)^2=-|{\bvec p}-{\bvec k}|^2+O(|{\bvec p}|^4)$,因此对于 Dirac 旋量\autoref{diracs_eq4}~\upref{diracs},其非相对论近似为
\begin{equation}
\begin{aligned}
&u^r(k)=\sqrt{m} \begin{pmatrix}\xi^r\\ \xi^r\end{pmatrix}+O(|{\bvec p}|),\quad  \xi^{r\dagger}\xi^s=\delta_{rs}\\
&\bar u^s(p') u^r(p)=2m \delta_{rs}
\end{aligned}
\end{equation}
代入\autoref{yukpot_eq1} 我们得到
\begin{equation}\label{yukpot_eq2}
\begin{aligned}
i\mathcal{M} = i g^2 (2m)^2 \frac{1}{|{\bvec p}'-{\bvec p}|^2+m_\phi^2}\delta_{s' s}\delta_{r' r}
\end{aligned}
\end{equation}
这意味着非相对论极限下两个费米子的自旋不会发生改变。下面我们用这一结果来求解非相对论势 $V(\bvec x)$:它是非相对论势能算符 $V$ 的坐标表象 $V\ket{\bvec x}=V(\bvec x)\ket{\bvec x}$。

注意到 $\mathcal{S}$ 矩阵元的定义为 $\mathcal{S}_{\beta\alpha}=\delta_{\beta\alpha}+i\mathcal{T}_{\beta\alpha}$,而 $i\mathcal{T}=i\mathcal{M}(2\pi)^4 \delta^4(p'+k'-p-k)$, $i\mathcal{T} = \bra{{\bvec p}',{\bvec k}'}iT\ket{{\bvec p},{\bvec k}}$。根据一阶玻恩近似,$T$ 近似为非相对论势能算符 $V$ 再乘以 $-2\pi\delta(E_{\bvec p'}-E_{\bvec p})$(\autoref{Smat_eq2}~\upref{Smat} 式以及玻恩近似可得。)。为了求出势能算符的坐标表象的表达式,我们还需要注意这里的入态和初态的归一化问题:$\langle{\bvec p}'|{\bvec p}\rangle=2m\cdot (2\pi)^3 \delta({\bvec p}'-{\bvec p}) = 2m \langle{\bvec p}'|{\bvec p}\rangle_{NR}$('NR' 表示非相对论情形下粒子波函数的归一化),于是对于入态和初态都是双粒子的情形,我们有
\begin{equation}
\begin{aligned} 
i\mathcal{T}&=(2m)^2\bra{{\bvec p}',{\bvec k}'}iT\ket{{\bvec p},{\bvec k}}_{NR}\\
& =-(2m)^2\bra{{\bvec p}',{\bvec k}'}iV\ket{{\bvec p},{\bvec k}}_{NR} \cdot (2\pi) \delta(E_{{\bvec p}'}-E_{\bvec p})
\\
& = -i(2m)^2 \tilde{V}({\bvec p}'-{\bvec p})(2\pi)^3 \delta({\bvec p}'+{\bvec k}'-{\bvec p}-{\bvec k})\cdot (2\pi) \delta(E_{{\bvec p}'}-E_{\bvec p})\\
&=-i(2m)^2 \tilde{V}(\bvec q) (2\pi)^4 \delta^4(p'+k'-p-k),\quad \bvec q={\bvec p}'-{\bvec p}
\end{aligned}
\end{equation}
上式中 $\tilde V(\bvec q)$ 是非相对论情形下的势能算符 $V(\bvec x)$ 的动量表象,联系\autoref{yukpot_eq2} 可得:
\begin{equation}
\begin{aligned}
\tilde{V}(\bvec q)&=\frac{-g^2}{|\bvec q|^2+m_\phi^2}\\
V({\bvec x})&=\int \frac{\dd[3]{\bvec q}}{(2\pi)^3} \frac{-g^2}{|\bvec q|^2+m_\phi^2}e^{i\bvec q\cdot {\bvec x}} =\frac{-g^2}{4\pi^2}\int_0^\infty \dd q q^2\int_{-1}^1 \dd t \frac{e^{iq|{\bvec x}|t}}{|\bvec q|^2+m_\phi^2}\\
&= \frac{-g^2}{4\pi^2}\int_0^\infty \dd q q^2\frac{e^{iq|{\bvec x}|}-e^{-iq|{\bvec x}|}}{iq|{\bvec x}|} \frac{1}{q^2+m_\phi^2}\\
&=\frac{ig^2}{4\pi^2|{\bvec x}|}\int_{-\infty}^\infty \dd q \frac{qe^{iq|{\bvec x}|}}{q^2+m_\phi^2}\\
&=-\frac{g^2}{4\pi}\frac{1}{|{\bvec x}|} e^{-m_\phi |{\bvec x}|}
\end{aligned}
\end{equation}
这是一个吸引的 Yukawa 势,可以看到它与库仑势的形式非常相像。
\subsection{$e^+e^-\rightarrow e^+e^-$ 的 Yukawa 势}
我们再来考察两体散射过程 $\ket{{\bvec p},s,+;{\bvec k},r,-}\rightarrow \ket{{\bvec p}',s',+;{\bvec k}',r',-}$,其中 $-$ 表示 $e^+$。类似前面的计算,这里的 Feynman 矩阵元为
\begin{equation}
\begin{aligned}
i\mathcal{M}&=(-ig)^2 \left(
-\bar u^{s'}(p')u^s(p) \frac{i}{|p'-p|^2-m_\phi^2} \bar v^{r}(k) v^{r'}(k')\right.\\
&\left.\quad + \bar u^{s'}(p')v^{r'}(k') \frac{i}{|p+k|^2-m_\phi^2} \bar v^{r}(k) u^s(p)\right)
\end{aligned}
\end{equation}
其中第二项对应的 Feynman 图是正负电子对湮灭成光子再产生正负电子对。在非相对论极限下,$|p'-p|\ll |p+k|$,因此我们可以忽略第二项。利用 $\bar v^s(p') v^r(p)=-2m \delta_{rs}$,上式的结果与 $e^-e^-\rightarrow e^- e^-$ 过程的 Feynman 矩阵元是一样的,我们可以得到吸引的 Yukawa 势。
\[
V({\bvec x})=-\frac{g^2}{4\pi}\frac{1}{|{\bvec x}|} e^{-m_\phi |{\bvec x}|}
\]