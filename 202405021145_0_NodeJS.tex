% Node.JS 笔记
% license Xiao
% type Note

\begin{issues}
\issueDraft
\end{issues}

\pentry{JavaScript 入门笔记\nref{nod_JS}}{nod_3689}

\subsection{安装}
\begin{itemize}
\item \href{https://nodejs.org/en/download/package-manager/}{下载安装包}
\item Linux 用 \href{https://github.com/nodesource/distributions/blob/master/README.md#installation-instructions}{Node source}
\item 安装完成以后, 而已用 \verb`node -v` 和 \verb`npm -v` 查看版本。 目前是 v16.5.0 和 7.19.1
\end{itemize}
\begin{lstlisting}[language=bash]
# Using Ubuntu
curl -fsSL https://deb.nodesource.com/setup_21.x | sudo -E bash - &&\
sudo apt-get install -y nodejs

# Using Debian, as root
curl -fsSL https://deb.nodesource.com/setup_21.x | bash - &&\
apt-get install -y nodejs
\end{lstlisting}
ru'gu

\subsection{hello world}
\begin{itemize}
\item 参考\href{https://medium.com/@adnanrahic/hello-world-app-with-node-js-and-express-c1eb7cfa8a30}{这篇文章}。
\item node.js 可以在服务器端运行 javascript。 \textbf{npm} 是 node package manager
\item 在 linux 下安装 nodejs 和 npm: \verb|sudo apt install nodejs npm|
\item 创建一个项目文件夹, 在该文件夹中打开 terminal, 初始化: \verb|npm init|。 \verb|npm init| 这个命令的唯一作用就是生成 \verb`package.json` 文件, 你也可以自己写这个文件。
\item 按照 \verb|npm init| 提示填写, \verb|entry point| 填 \verb`app.js`. 不想填可以按回车跳过。 输入的内容都会在 \verb|package.json| 中,以后想改可以随时改。
\begin{lstlisting}[language=none,caption=package.json 示例]
{
  "name": "node",
  "version": "1.0.0",
  "description": "",
  "main": "index.js",
  "scripts": {
    "test": "echo \"Error: no test specified\" && exit 1"
  },
  "author": "addis",
  "license": "ISC",
  "dependencies": {
    "express": "^4.18.2"
  }
}
\end{lstlisting}
\item \verb|express| 包是一个 minimalist web framework
\item 安装 express 包: \verb|npm install express --save|, 其中 \verb|--save| 会把安装的包保存到 \verb|package.json| 中的依赖列表。 另外会生成 \verb|package-lock.json| (指定所有依赖包的版本) 以及程序安装目录 \verb|node_modules|。
\item 过一段时间再把上面流程做一次, 会发现 \verb|package-lock.json| 中的版本号改变了, 这是因为网上的包是不断更新版本的, 这就是该文件的作用。
\item 创建程序文件 \verb|app.js|, 就是刚才输入的 entry point
\begin{lstlisting}[language=js]
var express = require('express');
var app = express();
app.get('/', function(req, res) {
  res.send('Hello World!');
});
app.listen(3000, function() {
  console.log('Example app listening on port 3000!');
});
\end{lstlisting}
\item 运行 \verb|node app.js|
\item 成功的话, 打开浏览器输入 \href{http://localhost:3000/}{http://localhost:3000/} 就会显示 \verb|Hello World!|。
\end{itemize}
