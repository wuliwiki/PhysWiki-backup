% 戴森级数
% 戴森级数|时间演化算符

\pentry{时间演化算符(量子力学)\upref{TOprt}}

\subsection{戴森级数}

考虑时间演化算符所满足的微分方程式:

\begin{equation}
i\hbar \frac{d}{dt}\hat U (t,t_0) = \hat H(t) \hat U (t,t_0)
\end{equation}

将其转化为积分方程,且考虑$\lim_\limits{t\rightarrow 0}\hat U=\hat I$,则有:

\begin{equation}
\hat U (t,t_0) =\hat I - \frac{i}{\hbar}\int^t_{t_0} \hat H(t_1) \hat U (t_1,t_0)dt_1
\end{equation}

将$\hat{U}$不断迭代,则有:

\begin{align}
\hat U (t,t_0) &=\hat I - \frac{i}{\hbar}\int^t_{t_0} \hat H(t_1) \hat U (t_1,t_0)dt_1 \\
&=\hat I - \frac{i}{\hbar}\int^t_{t_0} \hat H(t_1) dt_1 - (\frac{i}{\hbar})^2\int^t_{t_0}\int^{t_1}_{t_0} \hat H(t_1)H(t_2) \hat U (t_1,t_0)dt_1dt_2  \\
&= \cdots \\
&=\hat I +\sum^\limits{\infty}_\limits{n=1}(-\frac{i}{\hbar})^n\int^{t}_{t_0}\int^{t_1}_{t_0}\cdots\int^{t_{n-1}}_{t_0}H(t_1)H(t_2)\cdots H(t_n)dt_n\cdots dt_2dt_1 
\end{align}

上式即被称为戴森级数。

\subsection{时序算符}

考虑算符$\hat T$,其作用为:

