% 常见几何体的转动惯量
% 刚体|转动惯量|圆环|球壳|球体|圆柱体|长方体

\pentry{刚体定轴转动、转动惯量\upref{RigRot}}

\begin{figure}[ht]
\centering
\includegraphics[width=12cm]{./figures/ExMI_1.pdf}
\caption{常见几何体的转动惯量,虚线为转轴,物体质量 $M$ 均匀分布, $R$ 为几何体的半径或红线标注的长度。}\label{ExMI_fig1}
\end{figure}

\addTODO{薄圆盘共面轴:画图}

一个通用的结论是: 若把刚体在延轴方向复制任意多次, 其总质量 $M$ 相应变大但转动惯量公式不变。 例如\autoref{ExMI_fig1} 中的圆柱体可以看作薄圆盘延轴的方向叠加无限多次, 又例如薄长方体(共面轴)可以看作细棒(中心轴)延轴的方向叠加无限多次。 这是因为如果两个物体转动惯量分别为 $I_1 = \alpha M_1 R^2$ 和 $I_2 = \alpha M_2 R^2$, 总质量 $M = M_1 + M_2$, 那么总转动惯量为 $I = I_1 + I_2 = \alpha M R^2$, 系数 $\alpha$ 不变。

\subsection{细圆环、薄圆柱环}
细圆环和薄圆柱环的所有质量与转轴的距离都为 $R$,可以看成许多质点的叠加,每个质点的转动惯量为 $m_i R^2$,所以
\begin{equation}\label{ExMI_eq4}
I = \sum_i m_i R^2 = M R^2~.
\end{equation}

\subsection{ 细棒(端点轴)}
细棒的线密度为 $\lambda  = M/L$,如果划分成长度为 $\Delta r$ 的小段,第 $i$ 段距离转轴 $r_i$, 有
\begin{equation}\label{ExMI_eq3}
I = \lim_{\Delta r \to 0}\sum_i \lambda\Delta r \cdot r_i^2 =  \int_0^L \lambda r^2 \dd{r} = \frac{1}{3}\lambda L^3 = \frac{1}{3}M L^2~.
\end{equation}

\subsection{细棒(中心轴)}
细棒(中心轴)可以看做两个等质量的细棒(端点轴),质量都为 $M_1$,每个具有转动惯量(\autoref{ExMI_eq3} ) $M_1 R^2/3$, 乘以 $2$ 得总转动惯量为($M=2M_1$)
\begin{equation}\label{ExMI_eq2}
I = \frac{1}{3} MR^2 = \frac{1}{12}ML^2~,
\end{equation}
其中 $L=2R$。 由此可以看出,\textbf{若一个物体可以拆分成转动惯量相同的若干部分,那么转动惯量公式不变}。

\subsection{薄长方体(共面轴)}
薄长方体(共面轴)可以看成许多细棒(中心轴)组成,所以转动惯量的系数仍然为
\begin{equation}\label{ExMI_eq8} 
I = \frac{1}{3} MR^2 = \frac{1}{12}ML^2~.
\end{equation}

\subsection{薄圆盘、圆柱}
薄圆盘可以看做许多宽度为 $\Delta r$ 的细圆环组成\footnote{然而薄圆盘不能看做由许多过圆心的细棒组成,因为这样面密度就是不均匀的。另外注意每个细环的转动惯量并不相同(因为半径各不相同),所以不能直接用圆环的转动惯量公式。},质量面密度为 $\sigma  = M/(\pi R^2)$,第 $i$ 个圆环的半径为 $r_i$,面积为 $2\pi r_i \Delta {r}$,总转动惯量为
\begin{equation}
I = \int_0^R r^2 \dd{m}  = \int_0^R r^2 \sigma  \cdot 2\pi r \dd{r}  = 2\pi \sigma \int_0^R r^3 \dd{r}~,
\end{equation}
也可以在极坐标极坐标\upref{Polar}中直接根据定义写出积分
\begin{equation}
I = \int {r^2}\sigma \dd{s}  = \int_0^{2\pi } \int_0^R \sigma r^2 \cdot r \dd{r}\dd{\theta}  = 2\pi \sigma \int_0^R r^3 \dd{r}  = \frac12\sigma \pi R^2 R^2 = \frac12 M R^2~,
\end{equation}
圆柱可看做由许多相同的薄圆盘组成,转动惯量系数相同。

\subsection{薄圆盘(共面轴)}
使用垂直轴定理\autoref{MIthm_eq2}~\upref{MIthm} 可得, 薄圆盘(垂直轴)的转动惯量是薄圆盘(共面轴)的两倍, 所以
\begin{equation}
I = \frac{1}{4}MR^2~.
\end{equation}
注意垂直轴定理只适用于薄片, 所以圆柱(共面轴)并不能这么计算。

\subsection{薄球壳}
球壳可以看做由许多细圆环组成,质量面密度为 $\sigma  = M/(4\pi R^2)$,球坐标中,令第 $i$ 个圆环对应的极角为 $\theta$,宽度为 $R \dd{\theta}$,面积为 $\dd{s_i} = 2\pi R\sin\theta \cdot R \dd{\theta}$,半径为 $r_{\bot} = R\sin\theta$,总转动惯量为
\begin{equation}
\ali{
I &= \int_0^\pi r_\bot^2 \dd{m}  = \int R^2 \sin^2 \theta \cdot \sigma  \cdot 2\pi R\sin\theta \cdot R \dd{\theta} \\
&= 2\pi \sigma R^4 \int \sin^3 \theta_i \dd{\theta}  = 2\pi \sigma R^4 \int_0^\pi \sin^3 \theta \dd{\theta}~, % 未完成! 积分有例题吗?
}\end{equation}
也可以在球坐标中直接写出球面积分
\begin{equation}
\ali{
I &= \int r^2 \sigma \dd{s}  = \int_0^{2\pi} \int_0^\pi  (R\sin \theta)^2 \sigma R^2\sin \theta \dd{\theta} \dd{\phi}   = 2\pi \sigma R^4 \int_0^\pi  \sin^3 \theta \dd{\theta}\\
&= 2\pi \sigma R^4\int_{-1}^1 (1 - \cos^2 \theta ) \dd{\cos \theta}  = \frac23 (\sigma 4\pi R^2) R^2 = \frac23 M R^2~,
}\end{equation}
其中对 $\theta$ 的积分使用了换元积分法。%(链接未完成,最好有链接到例题) 

\subsection{球体}
球体可以看做由许多薄球壳组成,体密度为 $\rho  = M/(4\pi R^3/3)$,令第 $i$ 个球壳半径为 $r$,厚度为 $\dd{r}$,体积为 $4\pi r^2 \dd{r}$,总转动惯量为
\begin{equation}\label{ExMI_eq5}
I = \int_0^R \frac23 r^2 \dd{m}  = \frac23 \rho \int_0^R r^2 \dd{V}  = \frac{2M}{R^3} \int_0^R r^4 \dd{r} = \frac{2}{5} M R^2~,
\end{equation}
也可以在球坐标中直接体积分
\begin{equation}
\ali{
I &= \int (r\sin \theta )^2 \dd{m}  = \int_0^{2\pi } \int_0^\pi  \int_0^R (r\sin \theta )^2\sigma r^2 \sin \theta \dd{r}\dd{\theta}\dd{\phi}\\
&= \frac{3M}{2R^3}\int_0^\pi  \sin^3\theta  \dd{\theta}  \int_0^R r^4 \dd{r}  =\frac{2M}{R^3}\int_0^R r^4 \dd{r} = \frac{2}{5} M R^2~,
}\end{equation}
其中对 $\theta$ 的积分使用了换元积分法。%(链接未完成,最好有链接到例题)

\subsection{薄长方体(垂直轴)}
由“薄长方体(共面轴)”可知两个共面方向的转动惯量分别为(\autoref{ExMI_eq8} ) $MR_1^2/3$ 和 $MR_2^2/3$,使用垂直轴定理\autoref{MIthm_eq2}~\upref{MIthm} 可得关于垂直轴的转动惯量为二者之和
\begin{equation}\label{ExMI_eq1}
I = \frac13 M(R_1^2 + R_2^2) = \frac{1}{12} M(L_1^2 + L_2^2)~,
\end{equation}
其中 $L_1 = 2R_1$, $L_2 = 2R_2$ 分别是两条边长。

\subsection{长方体}
另外由于长方体可以看作许多薄片延轴方向叠加, 其转动惯量公式也相同
\begin{equation}\label{ExMI_eq9}
I = \frac13 M(R_1^2 + R_2^2) = \frac{1}{12} M(L_1^2 + L_2^2)~,
\end{equation}
其中 $L_1 = 2R_1$, $L_2 = 2R_2$ 分别是长方体垂直于转轴的两条边长。
