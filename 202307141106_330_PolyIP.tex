% 多项式插值
\subsection{什么是插值}
插值法,就是在函数表达式未知的情况下,通过已知点处的函数值来估计未知点处函数值的方法。\begin{definition}{插值函数}
已知函数 $f(x)$ 在 $\left[a,b\right] $ 上有定义,且已经测得在 $x_0,x_1,\cdots, x_n$ 处的函数值为 $y_0 = f(x_0),y_1 = f(x_1), \cdots, y_n = f(x_n)$, 如果存在一个简单易算的函数 $p(x)$, 使得\begin{equation}
\label{eq_PolyIP_1}
p(x_i) = f(x_i)~, i = 1,2, \cdots, n~
\end{equation}

则称 $p(x)$ 是 $f(x)$ 的插值函数。
\end{definition}

插值法是数值计算中一种常用且基础的方法,在数据挖掘、数学建模等诸多领域都有广泛的应用。

\subsection{多项式插值}
当\autoref{eq_PolyIP_1} 中的简单函数 $p(x)$ 是多项式函数时,该插值方法就叫\textbf{多项式插值}, $p(x)$ 被称为\textbf{插值多项式}。当有 $n$ 个插值点时,插值多项式的次数一定\textbf{小于等于} $n$ 次。
\begin{theorem}{插值多项式的唯一性定理}
满足上述条件的多项式 $p(x)$ 存在唯一。

\end{theorem}
\textbf{证明:} 由Vandermonde矩阵为非奇异矩阵易证。
\subsubsection{Lagrange插值}
\begin{definition}{Lagrange基函数}

\end{definition}
