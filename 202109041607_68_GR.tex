% 广义相对论中的对称性和Killing矢量场
% keys 关键词1:对称性|关键词2:Killing矢量场|关键词3:史瓦西时空

## 一:物理中的对称性
对称性是物理学中的一条基本定律,物理学中对它的通俗解释是:如果一个物理定律在某些操作下保持不变,我们就说这个物理定律在这些操作下具有某种不变性,也即对称性.在经典力学中,关于对称性有一条著名的诺特定理:如果一个系统具有某种连续对称性,则该系统存在一个相应的守恒荷,即守恒律.例如:若一个系统存在平移不变对称性,该系统的能量守恒;若系统存在旋转对称不变性,对应的该系统角动量守恒.在场论中,我们可以以两种视角理解对称性:系统的运动方程不变和作用量保持不变.场论中将对称性分为两类:一种是时空对称性,时空坐标在变换时相应的场也在变,如庞加莱变换;另一种是内禀对称性,时空坐标不变,但场在变,如$U_{(1)}$相位对称性.在广义相对论中,时空几何存在一种重要的对称性,其中Killing矢量场刚好描述这种时空对称性,每一个Killing矢量场确定度规的一个对称变换,对应地存在一个守恒荷.本文将慢慢引入Killing矢量场的物理意义,推导Killing方程并且介绍它在广义相对论史瓦西度规中的具体应用.
## 二:Lie导数和Lie括号
广义相对论的数学基础是微分几何和微分流形,所谓的流形就是可用$\mathbb{R}^{n}$即n维欧式空间“粘”出来的空间,它局部像$\mathbb{R}^{n}$.研究广义相对论要基于微分几何的数学语言,其中Lie导数是微分几何的一种结构,描述时空对称性的Killing矢量场的数学语言是基于李导数而来的.什么是李导数,李导数和Killing矢量场之间有什么关联,下面将对其作仔细介绍.
#### 1:Lie导数定义
给定某矢量场$\zeta^{\mu}(x)$,其中$\epsilon$是个无穷小量,$\zeta^{\mu}(x)$是该矢量场$\overrightarrow{\zeta}$在x方向的分量大小.在两点间建立一种映射关系:$\zeta^{\mu}(x):x^{\mu}\rightarrow x'^{\mu}=x^{\mu}+\epsilon\zeta^{\mu}$,其中$\delta x^{\mu}=\epsilon\zeta^{\mu}$,$\frac{d x^{\mu}}{\epsilon}=\zeta^{\mu}$.在流形任意张量场$T(x)$建立一种导数如下所示:
$$\begin{equation}
\mathcal{L}_{\zeta}T=\lim_{\epsilon\rightarrow0}\frac{T(x')-T(x\rightarrow x')}{\epsilon}\equiv \frac{\delta T}{\epsilon}
\end{equation}$$
以上为张量场$T(x)$沿矢量场$\zeta^{\mu}$的李导数.
#### 2:标量场Lie导数
对于一个标量场f,其中$f(x\rightarrow x')=f(x)$.标量场f沿矢量场$\zeta^{\mu}$的李导数为:
$$\begin{aligned}
\mathcal{L}_{\zeta} f &=\lim _{\epsilon \rightarrow 0} \frac{f(x')-f(x \rightarrow x')}{\epsilon}=\lim _{\epsilon \rightarrow 0} \frac{f(x')-f(x)}{\epsilon} \\
&=\lim _{\epsilon\rightarrow 0} \frac{\partial_{\mu} fd x^{\mu}}{\epsilon}=\zeta^{\mu}\partial_{\mu}f \\
&=\zeta(f)=\nabla_{\zeta}f
\end{aligned}$$
可以得出:标量场的李导数等于其沿$\zeta^{\mu}$方向的协变导数.
#### 3:切矢量场的Lie导数
对于切矢量场$A^{\mu}(x)$,其中$A^{\mu}(x)$是矢量场$\overrightarrow{A(x)}$在x方向的分量大小:
$$\begin{aligned}
&A^{\mu}_{(x)}=\frac{d x^{\mu}}{d \lambda}\\
&A^{\mu}\left(x \rightarrow x^{\prime}\right)=\frac{d x^{\prime \mu}}{d \lambda}=\frac{d\left(x^{\mu}+\epsilon \zeta^{\mu}\right)}{d \lambda}=\frac{d x^{\mu}}{d \lambda}+\epsilon\frac{d \zeta^{\mu}}{d \lambda} \\
&=A^{\mu}(x)+\epsilon \frac{d x^{\nu}}{d \lambda} \cdot \frac{d \zeta^{\mu}}{dx^{\nu}} =A^{\mu}_{(x)}+\epsilon A_{(x)}^{\nu}\partial_{\nu}\zeta^{\mu}
\end{aligned}$$
沿矢量场$\zeta^{\mu}$的李导数为:
$$\begin{aligned}
\mathcal{L}_{\zeta} A^{\mu} &=\lim _{\epsilon \rightarrow 0} \frac{A^{\mu}(x')-A^{\mu}(x \rightarrow x')}{\epsilon} \\
&=\lim _{\epsilon \rightarrow 0} \frac{A^{\mu}(x')-A^{\mu}_{(x)}-\epsilon A_{(x)}^{\nu}\partial_{\nu}\zeta^{\mu}}{\epsilon} \\
&=\lim _{\epsilon \rightarrow 0} \frac{\partial_{\nu} A^{\mu}(x) d x^{\nu}}{\epsilon}-A^{\nu}(x) \partial_{\nu} \zeta^{\mu} \\
&=\zeta^{\nu} \partial_{\nu} A^{\mu}-A^{\nu} \partial_{\nu} \zeta^{\mu}
\end{aligned}$$
对于一个黎曼流形,它定义的几何条件是无挠性.则沿矢量场$\zeta^{\mu}$的李导数为:$\mathcal{L}_{\zeta} A^{\mu}=\zeta^{\nu} \nabla_{\nu} A^{\mu}-A^{\nu} \nabla_{\nu} \zeta^{\mu}$.
#### 4:Lie括号
对于一个完整矢量,此时$\overrightarrow{A}=A^{\mu}\overrightarrow{e}_{\mu}$.$\overrightarrow{A}$则沿矢量场$\zeta^{\mu}$的李导数为:
$$\begin{aligned}
\mathcal{L}_{\zeta} \overrightarrow{A} &=\left(\mathcal{L}_{\zeta} A^{\mu}\right)\overrightarrow{e}_{\mu}  \\
&=\left(\zeta^{\nu} \partial_{\nu} A^{\mu}-A^{\nu} \partial_{\nu} \zeta^{\mu}\right) \overrightarrow{e}_{\mu} =\overrightarrow{\zeta} \overrightarrow{A}-\overrightarrow{A} \overrightarrow{\zeta} \\
&=\zeta^{\mu} \partial_{\overrightarrow{e}_{\mu}} \overrightarrow{A}-A^{\mu} \partial_{\overrightarrow{e}_{\mu}} \overrightarrow{\zeta}\\
&=\partial_{\overrightarrow{\zeta}} \overrightarrow{A}-\partial_{\overrightarrow{A}} \overrightarrow{\zeta}:=[\zeta, A]
\end{aligned}$$
上式中引入的括号为李括号,同理,在无挠率的黎曼流形下,$\overrightarrow{A}$则沿矢量场$\zeta^{\mu}$的李导数为:$\mathcal{L}_{\zeta} \overrightarrow{A}=\nabla_{\overrightarrow{\zeta}} \overrightarrow{A}-\nabla_{\overrightarrow{A}} \overrightarrow{\zeta}$.
#### 5:任意阶张量的Lie导数
$A^{\mu}$是矢量场$\overrightarrow{A}$的分量大小,$B_{\mu}$是矢量场$\overrightarrow{B}$的分量大小,两个矢量场分量的缩并$A^{\mu}B_{\mu}$为一个标量.根据莱布尼茨法则,$A^{\mu}B_{\mu}$沿矢量场$\zeta^{\mu}$的李导数为:
$$\begin{aligned}
\mathcal{L}_{\zeta}\left(A^{\mu} B_{\mu}\right) &=\zeta^{\nu} \partial_{\nu}\left(A^{\mu} B_{\mu}\right) \\
&=\zeta^{\nu} (\partial_{\nu} A^{\mu}) B_{\mu}+\zeta^{\nu} A^{\mu} \partial_{\nu} B_{\mu} \\
&=(\mathcal{L}_{\zeta} A^{\mu}) B_{\mu}+A^{\mu} \mathcal{L}_{\zeta} B_{\mu} \\
&=\zeta^{\nu} \partial_{\nu} A^{\mu} B_{\mu}-A^{\nu} \partial_{\nu} \zeta^{\mu} B_{\mu}+A^{\mu} \mathcal{L}_{\zeta} B_{\mu }
\end{aligned}$$
求得出在无挠率的黎曼流形下协变矢量的李导数为:
$$\begin{aligned}
\mathcal{L}_{\zeta} B_{\mu} &=\zeta^{\nu} \partial_{\nu} B_{\mu}+(\partial_{\mu} \zeta^{\nu}) B_{\nu} \\
&=\zeta^{\nu} \nabla_{\nu} B_{\mu}+(\nabla_{\mu} \zeta^{\nu}) B_{\nu}
\end{aligned}$$
对比切矢量场$A^{\mu}$沿矢量场$\zeta^{\mu}$的李导数和切矢量场$B_{\mu}$沿矢量场$\zeta^{\mu}$的李导数可得:对于任意阶张量沿矢量场$\zeta^{\mu}$的李导数,可写成如下形式:
$$\begin{aligned}
\mathcal{L}_{\zeta} A_{\nu}^{\mu} &=\zeta^{\rho} \partial_{\rho} A_{\nu}^{\mu}-A_{\nu}^{\rho} \partial_{\rho} \zeta^{\mu}+A_{\rho}^{\mu} \partial_{\nu} \zeta^{\rho} \\
&=\zeta^{\rho} \nabla_{\rho} A_{\nu}^{\mu}-A_{\nu}^{\rho} \nabla_{\rho} \zeta^{\mu}+A_{\rho}^{\mu} \nabla_{\nu} \zeta^{\rho}
\end{aligned}$$
#### 6:度规场的Lie导数
度规$g_{\mu \nu}$是个二阶张量,利用黎曼流形度规兼容性$\nabla_{\rho} g_{\mu \nu}=0$性质和以上任意阶张量的Lie导数等式,可求得度规张量场$g_{\mu \nu}$沿矢量场$\zeta^{\mu}$的李导数:
$$\begin{aligned}
\mathcal{L}_{\zeta} g_{\mu \nu} 
&=\zeta^{\rho} \partial_{\rho} g_{\mu \nu}+g_{\rho \nu} \partial_{\mu} \zeta^{\rho}+g_{\mu \rho} \partial_{\nu} \zeta^{\rho} \\
&=\zeta^{\rho} \nabla_{\rho} g_{\mu \nu}+g_{\rho \nu} \nabla_{\mu} \zeta^{\rho}+g_{\mu \rho} \nabla_{\nu} \zeta^{\rho} \\
&=\nabla_{\mu} \zeta_{\nu}+\nabla_{\nu} \zeta_{\mu} 
\end{aligned}$$
## 三:Killing矢量场和Killing等式
#### 1:Killing矢量场
如果矢量场$\zeta$恰好为局域坐标的某个基矢,这时$\zeta=\partial_{\rho}=\delta_{\rho}^{\mu} \partial_{\mu}$,$\zeta^{\mu}=\delta_{\rho}^{\mu}$,度规张量场$g_{\mu \nu}$沿矢量场$\zeta^{\mu}$的李导数变为:$\mathcal{L}_{\zeta} g_{\mu \nu} =\mathcal{L}_{\partial_{\rho}} g_{\mu \nu}=\partial_{\rho} g_{\mu \nu}$.若度规张量场$g_{\mu \nu}$沿矢量场$\zeta^{\mu}$的李导数为0,即$\mathcal{L}_{\zeta} g_{\mu \nu} =0$,这时称$\zeta$为Killing矢量场.容易看出,Killing矢量场本质上是个保度规场.
#### 2:时空的对称性和Killing方程
对于闵式时空,其度规为常数.时空线元$$\begin{aligned}
d s^{2}=\eta_{\mu \nu} \mathrm{d} x^{\mu} \mathrm{d} x^{\nu}=-\mathrm{d} t^{2}+\mathrm{d} x^{2}+\mathrm{d} y^{2}+\mathrm{d} z^{2}
\end{aligned}$$对其坐标作平移变换$x^{\mu}\rightarrow x^{\mu}+a^{\mu}$和洛伦兹变换$x^{\mu}=\Lambda^{\mu}_{\nu} x^{\nu}$,会发现时空间隔$d s^{2}$不变,闵式时空有个庞加莱对称性.庞加莱变换是一种等度规变换,它具有保度规性质.闵式时空对应的庞加莱群一共有十个生成元,4个平移生成元,3个转动生成元,3个boost生成元,每个生成元对应相应的守恒量.由于Killing矢量场本质上是个保度规场,平坦时空的庞加莱变换是等度规变换,那么庞加莱群对应的每个群生成元对应一个Killing矢量.<br>
对于广义的弯曲时空,其时空线元$d s^{2}=g_{\mu \nu} \mathrm{d} x^{\mu} \mathrm{d} x^{\nu}$.若在$\sigma_{*}$矢量场上$\partial_{\sigma_{*}} g_{\mu \nu}=0$,那么该时空在该坐标变换下$x^{\sigma_{*}} \rightarrow x^{\sigma_{*}}+a^{\sigma_{*}}$有个对称性.很明显史瓦西时空具有该对称性,选定该对称性的时空为背景.引力是一种弯曲时空的几何效应,测试粒子在弯曲时空背景下沿测地线运动.测地线方程为$p^{\lambda} \nabla_{\lambda} p^{\mu}=0$和$p^{\lambda} \nabla_{\lambda} p_{\mu}=0$,对方程进行分解得$p^{\lambda} \partial_{\lambda} p_{\mu}-\Gamma_{\lambda \mu}^{\sigma} p^{\lambda} p_{\sigma}=0$.其中$$\begin{aligned}
p^{\lambda} \partial_{\lambda} p_{\mu}=m \frac{d x^{\lambda}}{d \tau} \partial_{\lambda} p_{\mu}=m \frac{d p_{\mu}}{d \tau}\end{aligned}$$
$$\begin{aligned}
\Gamma_{\lambda \mu}^{\sigma} p^{\lambda} p_{\sigma} &=\frac{1}{2} g^{\sigma \nu}\left(\partial_{\lambda} g_{\mu \nu}+\partial_{\mu} g_{\nu\lambda}-\partial_{\nu} g_{\lambda \mu}\right) p^{\lambda} p_{\sigma} \\
&=\frac{1}{2}\left(\partial_{\lambda} g_{\mu \nu}+\partial_{\mu} g_{v \lambda}-\partial_{\nu} g_{\lambda \mu}\right) p^{\lambda} p^{\nu} \\
&=\frac{1}{2}\left(\partial_{\mu} g_{\nu \lambda}\right) p^{\lambda} p^{\nu}
\end{aligned}$$
可以得出$m\frac{d p_{\mu}}{d \tau}=\frac{1}{2}\left(\partial_{\mu} g_{\nu \lambda}\right) p^{\lambda} p^{\nu}$.若$\partial_{\sigma_{*}} g_{\mu \nu}=0$,很明显$g_{\mu \nu}$与$x^{\sigma_{*}}$无关,在$x^{\sigma_{*}}$方向上四维动量p守恒,即$\frac{d p_{\sigma_{*}}}{d \tau}=0$.根据以上Killing矢量场的定义,若$\partial_{\sigma_{*}} g_{\mu \nu}=0$,则$K=\partial_{\sigma_{*}}$是个沿$x^{\sigma_{*}}$方向上的Killing矢量场,即沿Killing矢量场方向四维动量p守恒.<br>
$K^{\mu}=(\partial_{\sigma_{*}})^{\mu}=\delta_{\sigma_{*}}^{\mu}$,$p_{\sigma_{*}}=K^{\mu}p_{\mu}=K_{\mu}p^{\mu}$.由于$\frac{d p_{\sigma_{*}}}{d \tau}=0$,再根据测试粒子测地线方程可得:$p^{\mu} \nabla_{\mu} p_{\sigma_{*}}=p^{\mu} \nabla_{\mu}(K_{\nu}p^{\nu}) =0$.根据莱布尼茨法则对$p^{\mu} \nabla_{\mu}(K_{\nu}p^{\nu})$进行展开得;
$$\begin{aligned}
p^{\mu} \nabla_{\mu}(K_{\nu}p^{\nu})=K_{\nu}p^{\mu} \nabla_{\mu}p^{\nu}+p^{\mu} p^{\nu}\nabla_{\mu}K_{\nu}\\
=p^{\mu} p^{\nu}\nabla_{\mu}K_{\nu}=p^{\mu} p^{\nu}\nabla(_{\mu}K_{\nu})=0
\end{aligned}$$
上式的第三项中利用了测地线方程$p^{\mu} \nabla_{\mu}p^{\nu}=0$,最后一项$p^{\mu} p^{\nu}\nabla_{\mu}K_{\nu}=p^{\mu} p^{\nu}\nabla(_{\mu}K_{\nu})$利用了$p^{\mu} p^{\nu}$的对称性质.最后求得出$\nabla(_{\mu}K_{\nu})=0$,即Killing等式.
## 四:Killing矢量场在史瓦西对称时空中的应用
史瓦西度规描述的是一个静态,球对称的史瓦西时空,其表达式为:
$$d s^{2}=-\left(1-\frac{2 G M}{r}\right) \mathrm{d} t^{2}+\left(1-\frac{2 G M}{r}\right)^{-1} \mathrm{~d} r^{2}+r^{2} d \Omega^{2}$$,其中$$d \Omega^{2}=\mathrm{d} \theta^{2}+\sin ^{2} \theta \mathrm{d} \phi^{2}$$
容易发现其度规$\partial_{t} g_{tt}=0$,$\partial_{\phi} g_{\phi\phi}=0$.根据Killing矢量场本质上是个保度规场定义,可知史瓦西时空有两个Killing矢量场:一个是沿时间轴方向的$\partial_{t}$,另一个是沿转动轴方向的$\partial_{\phi}$.Killing矢量场:$K=\partial_{t}$,其坐标分量$K^{\mu}=\left(\partial_{t}\right)^{\mu}=(1,0,0,0)$;$R=\partial_{\phi}$,其坐标分量$R^{\mu}=\left(\partial_{\phi}\right)^{\mu}=(0,0,0,1)$.$K_{\mu}=g_{\mu \nu}K^{\nu}=(-1+\frac{2 G M}{r},0,0,0)$,$R_{\mu}=g_{\mu \nu}R^{\nu}=(0,0,0,r^{2}\sin ^{2} \theta )$.每个Killing矢量场对应一个守恒量,沿时间轴方向的Killing矢量场对应该时空背景下的能量守恒,沿转动轴方向的Killing矢量场对应史瓦西时空背景下的角动量守恒.沿时间方向的守恒量$P_{t}=K_{\mu}P^{\mu}=-E$,系统能量:$$E=-K_{\mu} \frac{d x^{\mu}}{d \lambda}=\left(1-\frac{2 G M}{r}\right) \frac{d t}{d \lambda}$$
沿转动轴方向的守恒量$P_{\phi}=R_{\mu}P^{\mu}=L$,系统角动量:$$L=R_{\mu} \frac{d x^{\mu}}{d \lambda}=r^{2} \frac{d \phi}{d \lambda}$$
测试粒子在史瓦西背景下运动,其能量和角动量守恒.
#### 参考资料:
1:Spacetime and Geometry 作者:Sean Carroll<br>
2:General Relativity   作者:Robert M. Wald<br>
3:微分几何入门与广义相对论  作者:梁灿彬<br>