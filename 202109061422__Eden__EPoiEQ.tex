% 静电势的泊松方程
% 静电学|静电势|泊松方程|电介质

\pentry{球坐标系中的拉普拉斯方程\upref{SphLap},麦克斯韦方程组(介质)\upref{MWEq1}}

\subsection{泊松方程}
我们首先考虑均匀、各向同性的线性电介质中的静电问题.设其电容率为 $\epsilon$(即相对介电常数 $\epsilon_r$ 乘以 $\epsilon_0$).根据介质中的麦克斯韦方程组 \upref{MWEq1},电场与电极化强度需要满足以下方程:
\begin{align}
&\nabla \cdot \bvec D=\rho,\ \ \nabla \times \bvec E=0,\\
&\bvec D=\epsilon \bvec E
\end{align}
式中 $\rho$ 表示空间的自由电荷密度.由于 $\bvec E$ 无旋,我们引入静电势 $\phi$:
\begin{align}
E=-\nabla \phi
\end{align}
由此可以得到泊松方程:
\begin{align}
\nabla^2 \phi(\bvec x)=-\rho(\bvec x)/\epsilon
\end{align}
如果所考虑的区域自由电荷密度 $\rho(\bvec x)\equiv 0$,那么静电势满足拉普拉斯方程:
\begin{align}
\nabla^2 \phi(\bvec x)=0
\end{align}

\subsection{泊松方程的解}
\subsubsection{无边界情况下}
如果求解泊松方程的问题是在没有边界的无穷大空间中,同时空间中的电荷分布为已知,那么泊松方程的解可以简单写出:
\begin{align}
\end{equation}