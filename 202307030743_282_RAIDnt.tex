% Linux 硬盘 RAID 阵列笔记

\begin{issues}
\issueDraft
\end{issues}

\begin{itemize}
\item 也可以使用自带 RAID 功能的文件系统如 ZFS\upref{ZFS} 可能更方便。
\item \verb|mdadm| 命令可以创建 raid
\item 参考\href{https://www.digitalocean.com/community/tutorials/how-to-create-raid-arrays-with-mdadm-on-ubuntu-22-04}{这个教程}
\item 参考这个\href{https://www.raid-calculator.com/}{计算器(理论数值)}。
\item RAID 0 (striping)
\item RAID 1 (mirroring)
\item RAID 5 最常用的 stripping + 单重冗余, 冗余均分于各个磁盘
\item RAID 6 双重冗余(允许两个磁盘同时故障)
\item RAID10 其实就是把硬盘两两 RAID1 镜像, 然后用 RAID0 提速扩容。
\end{itemize}
\addTODO{国产 yotamaster 4 bay 测评}
