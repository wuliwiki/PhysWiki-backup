% 编程语言理论(综述)
% license CCBYSA3
% type Wiki

本文根据 CC-BY-SA 协议转载翻译自维基百科\href{https://en.wikipedia.org/wiki/Programming_language_theory}{相关文章}。

编程语言理论(PLT)是计算机科学的一个分支,研究编程语言这一形式语言的设计、实现、分析、特征描述和分类。编程语言理论与数学、软件工程和语言学等领域密切相关。该领域有许多学术会议和期刊。
\begin{figure}[ht]
\centering
\includegraphics[width=6cm]{./figures/7180d92ac9151835.png}
\caption{小写希腊字母 λ(lambda)是编程语言理论领域的一个非官方符号。[citation needed] 这一用法源自 λ 演算,这是一种由阿隆佐·丘奇(Alonzo Church)在1930年代提出的计算模型,并被编程语言研究人员广泛使用。它出现在经典著作《计算机程序的构造与解释》的封面上,以及1975至1980年间由杰拉尔德·杰伊·萨斯曼(Gerald Jay Sussman)和盖伊·斯蒂尔(Guy Steele)编写的《Lambda Papers》一书的标题中,后者是 Scheme 编程语言的开发者。[行话]} \label{fig_BCYY_1}
\end{figure}
\subsection{历史}  
在某些方面,编程语言理论的历史甚至早于编程语言本身的发展。λ 演算(lambda calculus)由阿隆佐·丘奇(Alonzo Church)和斯蒂芬·科尔·克莱尼(Stephen Cole Kleene)在1930年代提出,被一些人认为是世界上第一个编程语言,尽管它的初衷是用来建模计算,而不是作为程序员向计算机系统描述算法的工具。许多现代的函数式编程语言被描述为在 λ 演算之上提供了一个“薄薄的表面层”,[2] 并且许多语言可以通过 λ 演算来简单地描述。

第一个发明的编程语言是 Plankalkül,它由康拉德·祖泽(Konrad Zuse)在1940年代设计,但直到1972年才为公众所知(并且直到1998年才实现)。第一个广为人知且成功的高级编程语言是 FORTRAN(表示“公式翻译”),由IBM研究团队在1954至1957年间开发,约翰·巴克斯(John Backus)领导。FORTRAN的成功促成了一个科学家委员会的形成,旨在开发一种“通用”计算机语言;他们的努力结果是ALGOL 58。与此同时,麻省理工学院的约翰·麦卡锡(John McCarthy)开发了 Lisp,它是首个源自学术界且成功的编程语言。随着这些初步努力的成功,编程语言在1960年代及以后成为了一个活跃的研究主题。
\subsubsection{时间轴}  
自那时以来,编程语言理论历史中的一些关键事件:

\textbf{1950年代}  
\begin{itemize}
\item 诺姆·乔姆斯基(Noam Chomsky)在语言学领域提出了乔姆斯基层次结构,这一发现直接影响了编程语言理论和计算机科学的其他分支。
\end{itemize}

\textbf{1960年代}  
\begin{itemize}
\item 1962年,Ole-Johan Dahl 和 Kristen Nygaard 开发了 Simula 语言,它被广泛认为是第一个面向对象编程语言的例子;Simula 还引入了协程(coroutine)的概念。  
\item 1964年,彼得·兰丁(Peter Landin)首次意识到教堂的 λ 演算可以用于建模编程语言。他提出了 SECD 机器,它“解释”λ表达式。  
\item 1965年,兰丁引入了 J 操作符,本质上是一种延续(continuation)形式。  
\item 1966年,兰丁在他的文章《The Next 700 Programming Languages》中提出了 ISWIM,它是一种抽象的计算机编程语言,对设计导致 Haskell 编程语言的语言产生了影响。  
\item 1966年,科拉多·博姆(Corrado Böhm)提出了编程语言 CUCH(Curry-Church)。[3]  
\item 1967年,克里斯托弗·斯特雷奇(Christopher Strachey)发布了具有影响力的讲义《Fundamental Concepts in Programming Languages》,引入了术语 R 值、L 值、参数多态性和临时多态性。  
\item 1969年,J·罗杰·辛德利(J. Roger Hindley)发布了《The Principal Type-Scheme of an Object in Combinatory Logic》,后续推广为辛德利-米尔纳类型推导算法。  
\item 1969年,托尼·霍尔(Tony Hoare)提出了霍尔逻辑(Hoare logic),一种公理化语义形式。  
\item 1969年,威廉·阿尔文·霍华德(William Alvin Howard)观察到“高级”证明系统,称为自然推理(natural deduction),可以直接在其直觉主义版本中解释为 λ 演算的类型变体。这被称为库里-霍华德对应(Curry–Howard correspondence)。
\end{itemize}
\textbf{1970年代}  
\begin{itemize}
\item 1970年,达纳·斯科特(Dana Scott)首次发布了他关于指示语义学(denotational semantics)的工作。  
\item 1972年,逻辑编程和 Prolog 的开发使得计算机程序可以用数学逻辑表示。  
由艾伦·凯(Alan Kay)领导的一组科学家在施乐 PARC 开发了 Smalltalk,这是一种面向对象的语言,因其创新的开发环境而广为人知。  
\item 1974年,约翰·C·雷诺兹(John C. Reynolds)发现了系统 F,它早在1971年就已被数学逻辑学家让-伊夫·吉拉尔(Jean-Yves Girard)发现。  
\item 1975年起,杰拉尔德·杰伊·萨斯曼(Gerald Jay Sussman)和盖伊·斯蒂尔(Guy Steele)开发了 Scheme 编程语言,这是一种 Lisp 方言,融入了词法作用域、统一命名空间以及来自演员模型的元素,包括一类优先的延续(first-class continuations)。  
\item 在1977年的图灵奖讲座上,巴克斯(Backus)批评了工业语言的现状,并提出了一类新的编程语言,现在被称为函数级编程语言。  
\item 1977年,戈登·普洛特金(Gordon Plotkin)提出了《编程可计算函数》(Programming Computable Functions),一种抽象类型的函数式语言。  
\item 1978年,罗宾·米尔纳(Robin Milner)为 ML 引入了辛德利-米尔纳类型推导算法。类型理论作为一种学科应用于编程语言,这一应用在多年里带来了类型理论的巨大进展。
\end{itemize}

\subsubsection{1980年代}  
\begin{itemize}
\item 1981年,戈登·普洛特金发布了关于结构化操作语义学(structured operational semantics)的论文。  
\item 1988年,吉尔·卡恩(Gilles Kahn)发布了关于自然语义学的论文。  
\item 出现了过程演算(process calculi),如罗宾·米尔纳(Robin Milner)的《通讯系统演算》(Calculus of Communicating Systems)和C·A·R·霍尔(C. A. R. Hoare)的《通讯顺序过程模型》(Communicating Sequential Processes),以及其他类似的并发模型,如卡尔·休伊特(Carl Hewitt)的演员模型(actor model)。  
\item 1985年,《Miranda》的发布激发了学术界对懒惰求值的纯函数式编程语言的兴趣。一个委员会成立以定义开放标准,最终导致1990年发布 Haskell 1.0 标准。  
\item 伯特兰·梅耶(Bertrand Meyer)创建了契约设计法(Design by Contract),并将其纳入了 Eiffel 编程语言中。
\end{itemize}
\subsubsection{1990年代}
\begin{itemize}
\item 格雷戈尔·基察尔斯(Gregor Kiczales)、吉姆·德里维埃尔(Jim Des Rivieres)和丹尼尔·G·博布罗(Daniel G. Bobrow)出版了《元对象协议的艺术》(The Art of the Metaobject Protocol)。  
\item 尤金·莫吉(Eugenio Moggi)和菲利普·沃德勒(Philip Wadler)引入了单子(monads)用于结构化函数式编程语言中的程序。
\end{itemize}
\subsection{子学科与相关领域}  
编程语言理论(PLT)涉及多个学科,这些学科或属于编程语言理论的范畴,或对其产生深远影响;其中许多领域存在显著的交叉。此外,PLT还借用了数学的其他分支,包括可计算性理论、范畴理论和集合论。
\subsubsection{形式语义学}  
形式语义学是计算机程序和编程语言行为的形式规范。描述计算机程序“语义”或“意义”的三种常见方法是指示语义学(denotational semantics)、操作语义学(operational semantics)和公理化语义学(axiomatic semantics)。
\subsubsection{类型理论} 
类型理论是研究类型系统的学科;类型系统是“通过根据程序短语所计算的值类型对短语进行分类,以证明某些程序行为缺失的可处理语法方法”。许多编程语言的特征由其类型系统的特点决定。
\subsubsection{程序分析与转换}  
程序分析是检查程序并确定关键特征(如是否缺少某些类型的程序错误)的通用问题。程序转换是将一个形式(语言)的程序转换为另一种形式的过程。
\subsubsection{编程语言的比较分析} 
编程语言的比较分析旨在根据其特征将编程语言分类;编程语言的广泛分类通常称为编程范式。
\subsubsection{泛型编程与元编程} 
元编程是生成高阶程序的过程,这些程序在执行时会生成程序(可能是另一种语言,或原始语言的一个子集)作为结果。
\subsubsection{领域特定语言}  
领域特定语言是为了高效解决特定领域问题而构建的语言。
\subsubsection{编译器构造}    
编译器理论是编写编译器(或更一般的翻译器)的理论;编译器是将用一种语言编写的程序转换为另一种形式的程序。编译器的动作通常分为语法分析(扫描与解析)、语义分析(确定程序应该做什么)、优化(根据某种度量提高程序性能;通常是执行速度)和代码生成(生成并输出等效程序,通常是CPU指令集的目标语言)。
\subsubsection{运行时系统}  
运行时系统指的是开发编程语言运行环境及其组件的过程,包括虚拟机、垃圾回收和外部函数接口等。
\subsection{期刊、出版物与会议}  
会议是展示编程语言研究的主要平台。最著名的会议包括编程语言原理研讨会(POPL)、编程语言设计与实现会议(PLDI)、国际函数式编程会议(ICFP)、国际面向对象编程、系统、语言与应用会议(OOPSLA)和国际编程语言与操作系统架构支持会议(ASPLOS)。

一些重要的发布PLT研究的期刊包括ACM《编程语言与系统交易》(TOPLAS)、《函数式编程期刊》(JFP)、《函数式与逻辑编程期刊》和《高阶与符号计算》。
\subsection{另见}  
\begin{itemize}
\item SIGPLAN  
\item 高级编程语言
\end{itemize}
\subsection{参考文献}  
\begin{enumerate}
\item Abelson, Harold (1996). 《计算机程序的结构与解释》. Gerald Jay Sussman, Julie Sussman (第二版). 剑桥,马萨诸塞州:麻省理工学院出版社。ISBN 0-262-01153-0。OCLC 34576857。  
\item “计算模型” 。wiki.c2.com。2014年12月3日。原文存档于2020年11月30日。  
\item C. Böhm 和 W. Gross (1996)。 《CUCH介绍》。见E. R. Caianiello(编辑),《自动机理论》,第35-64页。  
\item Benjamin C. Pierce. 2002. 《类型与编程语言》。麻省理工学院出版社,剑桥,马萨诸塞州,美国。
\end{enumerate}