% 电场的能量
% 电场|电场能|能量密度

\pentry{电势、电势能\upref{QEng}}
\footnote{本文参考 \cite{GriffE} 与周磊教授的电动力学讲义.}
如果我们认为电荷之间的交互作用是经由电场实现的(而不是令人费解的电荷间超距作用),那么系统的电势能实则储存于电场之中;或者说,电场本身具有能量,而电势能是电场能量的体现.我们不妨把电场含有的能量称为\textbf{电场能}.以下假定空间中不存在电介质,只存在\textsl{孤独的电荷们与一望无际的真空}.

电场能与电场强度的关系:
\begin{equation}\label{EEng_eq2}
E_p = \frac12 \epsilon_0 \int \bvec E(\bvec r)^2 \dd[3]{r}
\end{equation}

因此,单位体积的电场能量密度为
\begin{equation}
u = \frac12 \epsilon_0 \bvec E^2
\end{equation}

\subsubsection{固有能与相互作用能}
我们知道电场的叠加原理\upref{Efield}:当多个电荷存在时,总电场等于各个电荷各自的电场之和
$$
\bvec E = \bvec E_1 + \bvec E_2 +... 
$$
因此,
$$
\begin{aligned}
u &= \frac12 \epsilon_0 (\bvec E_1 + \bvec E_2 +... )^2\\
&=\frac12 \epsilon_0(E_1^2+E_2^2+...)+\frac12 \epsilon_0 (2\bvec E_1 \bvec E_2+2\bvec E_1 \bvec E_3+...)
\end{aligned}
$$
第一项可称为电荷的“固有能”,因为这部分能量是电荷自身的性质,与其他电荷无关;而第二项可称为“相互作用能”,因为这部分能量源于电荷间的相互作用.

由于能量密度正比于电场强度的平方,因此电场能量\textbf{不满足}叠加原理.从能量角度上看,这是因为往已有电场中再加入电荷时,需要额外的外力做功已克服电场力;因此,加入新电荷后.系统能量的增加不仅是新加入电荷的固有能,还有新加入电荷与已有电荷的交互作用能.
\begin{exercise}{电荷的相互作用能}
在空间中仅有两个带 $+e$ 的点电荷,分别位于 $\bvec x_1=(0,0,0)$ 和 $\bvec x_2=(0,0,r)$.则他们在全空间激发的电场的相互作用能为
\begin{equation}
\begin{aligned}
u&=\int \dd[3]{x} \epsilon_0 \bvec E_1\cdot \bvec E_2\\
&=\int \dd[3]{x} \epsilon_0 \frac{e}{4\pi\epsilon_0(\bvec x-\bvec x_1)^2}\cdot \frac{e}{4\pi\epsilon_0(\bvec x-\bvec x_2)^2}\\
&=\frac{e^2}{(4\pi)^2\epsilon_0} \int\dd[3]{x} \frac{1}{(\bvec x-\bvec x_1)^2(\bvec x-\bvec x_2)^2}
\end{aligned}
\end{equation}
试通过采用球坐标系\upref{Sph}的三维积分,证明上式的结果是
\begin{equation}
u=\frac{e^2}{4\pi \epsilon|\bvec x_1-\bvec x_2|}
\end{equation}
\end{exercise}

\subsubsection{点电荷模型与连续电荷模型的缺陷}
在上面的讨论中,有一个巧妙的\textsl{陷阱}(具体参见\cite{GriffE}).当我们在计算离散电荷的电势能\upref{QEng}时,我们事实上忽略了点电荷的固有能:因为不幸的是,点电荷的“固有能”是发散的.在将其推广至连续电荷分布时,这个问题被悄悄地解决了.这实际上是因为,在推广至连续电荷时,我们将物理系统的“微观”信息给抹去了,例如我们不再关心带电粒子的微观结构,不再关心电荷的微观分布,而是在一个更\textbf{粗粒化}的层面上去考察一个大量电荷分布的宏观物理量.这样的结果虽然是忽视了物理系统的微观信息,但这样的处理在普遍情况下是正确的.

因为我们是从粗粒化的角度去考察我们的物理系统,所以在距离很小的两点荷之间的相互作用的信息并没有被我们考虑进来,在粗粒化的过程中我们实际上\textbf{丢失了}那些微观的物理信息.例如当我们在考虑一个中性的材料,虽然在宏观角度上看它是不带电的(那么根据连续电荷模型,我们计算得到的电场能量应当为 $0$),然而当我们真正地去考察其微观结构,我们会发现材料是由许多带正电的离子、许多被离子束缚的电子、或自由电子组成的.当我们真正计算这些微观层面上电荷间的相互作用,我们会发现他们的累加不为 $0$.这一部分能量通常被我们称为\textbf{化学能},它是材料的一个属性.



\subsection{电场能量公式的证明}

\subsubsection{基于电容的不严谨推导}
\pentry{电容\upref{Cpctor}}
我们来回顾一下真空中平行板电容器(\autoref{Cpctor_ex2}~\upref{Cpctor})的能量
\begin{equation}
E_p = \frac12 CV^2 = \frac12 \epsilon_0 \frac Sd (Ed)^2 = \frac 12 \epsilon_0 \tau E^2
\end{equation}
其中 $\tau = Sd$ 为平行板间长方体的体积. 这条公式容易让我们联想到电势能储存在电场之中. 在电动力学中, 这种理解是正确的.

\subsubsection{更严格的推导}
\pentry{分部积分的高维拓展\upref{IntBP2},电场的高斯定律\upref{EGauss}}

将电场的高斯定律\upref{EGauss}
\begin{equation}
\rho = \epsilon_0\div \bvec E
\end{equation}
代入连续分布电荷系统的电势能 \autoref{QEng_eq8}~\upref{QEng}
\begin{equation}
E_p = \frac 12 \int V(\bvec r) \rho(\bvec r) \dd[3]{r}
\end{equation}
得
\begin{equation}
E_p = \frac{\epsilon_0}2 \int V (\div \bvec E) \dd[3]{r}
\end{equation}
由 3 维分部积分\autoref{IntBP2_eq1}~\upref{IntBP2}得
\begin{equation}
E_p = \frac{\epsilon_0}2 \oint \bvec E V \dd{\bvec s} - \frac{\epsilon_0}{2} \int \bvec E \vdot \grad V \dd{V}
\end{equation}
当我们把积分区域拓展到无限大时, 右边第一个面积分趋于 0(我们假设电荷与场只存在于有限区域内,因此在无穷远处电场与电势都趋近$0$). 右边第二个积分可以利用电场与电势的关系 $\grad V = -\bvec E$(\autoref{QEng_eq13}~\upref{QEng}). 带入可得\autoref{EEng_eq2}. 证毕.
