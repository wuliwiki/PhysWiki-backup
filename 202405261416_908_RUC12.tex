% 热力学统计物理
% license Usr
% type Art
一\textbf{已知} $U = U(T, V) = f(T, V)$,$P = f(T)$.

(1) 证明
&\left(\pdv{U}{T}\right)_T=T, \left(\pdv{S}{V}\right)_V=-P~.

(2) 用麦克斯韦关系求 $f$ 与 $T$ 的函数关系。

二
现有一液体质量为 $m_1$,温度为 $T_1$ 与同一种物理质量为 $m_2$,温度为 $T_2$,在绝热等压下混合。

(1) 混合温度

(2) 求熵变,证明熵变是非负的。

三
一物体有 $N$ 个粒子,每个粒子有三个能级,分别为 $-\varepsilon, \varepsilon, 0 (\varepsilon > 0)$,试按照玻尔兹曼分布,求:

(1) 配分函数

(2) 最低能级的粒子数及在高温、低温极限下的近似值

(3) 热容量 $C_v$ 及在高温、低温下的极限值

(4) 熵 $s$ 及在高温、低温下的近似值

四光子的能量为 $\epsilon = cp$,在能量腔中,求

(1) 平均粒子数

证明 $\frac{C_V}{Nk}$ 是常数
求 $\lambda$ 与 $T$ 的关系,验证 $n \lambda^3$ 能否满足经典极限条件。($k=1.38 \times 10^{-38} \ \text{J/K}$, $T=300 \text{k}$, $c=3 \times 10^8 \text{m/s}$, $m=9.1 \times 10^{-31} \text{kg}$ 计算时只需求出 $n \lambda^3$ 数量级即可)



对于费米气体,已知参数为 $N, V, T, \epsilon = b p^\alpha$,不考虑自旋,求:


态密度 $D(\epsilon)$
当 $a=1$ 和 $a=2$ 时,分别求费米能级
$U$ 与 $T$ 的关系,高温、低温下的极限
$C_v$ 与 $T$ 的关系,高温、低温下的极限及物态方程