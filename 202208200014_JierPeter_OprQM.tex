% 量子力学中的基本算符
% 动量算符|角动量算符|哈密顿算符|哈密顿量|能量算符|生成元
\addTODO{加入目录.}

\pentry{经典力学,量子力学的基本原理(量子力学)\upref{QMPrcp}}

本文中,$\hbar=1$.

在\textbf{量子力学的基本原理(量子力学)}的\autoref{QMPrcp_ex1}~\upref{QMPrcp} 和\autoref{QMPrcp_ex2} 中,我们不加证明地给出了动量、能量(哈密顿)、角动量算符在给定表象下的形式,相当于进行了定义.本文将讨论如何从经典力学中导出这几个算符的定义.

\subsection{无穷小算符}

考虑无穷小算符总是有益的,因为微分线性近似,而线性的东西很简单.

如果要求一个算符在时间趋于$0$的时候趋于恒等算符,即连续性,那么对于\textbf{无穷小}自变量$\varepsilon$后,这个算符总可以写为
\begin{equation}\label{OprQM_eq1}
U(\varepsilon) = 1-\I G\varepsilon
\end{equation}
我们称$G$是$U$的生成元.

当$G$是一个厄米算符时,有
\begin{equation}
U^{\dagger}(\varepsilon)U(\varepsilon) = (1-\I G\varepsilon)(1+\I G\varepsilon) = 1+o(\varepsilon)
\end{equation}
其中$o(\varepsilon)$表示比$\varepsilon$更高阶的无穷小.

因此,$G$是厄米算符$\implies$ $U(\varepsilon)$是\textbf{幺正}算符.

量子态归一化要求可观测量是幺正的,因此可观测量的生成元应该是厄米算符.



\subsection{位置算符}

位置算符的本征矢都是位置精确给定的态,本征值即对应的位置,因此位置算符是$x$或$\bvec{x}$,即空间坐标.

处于$x_0$或$\pmat{x_0, y_0, z_0}$的位置本征矢,在位置表象下的波函数为$\delta(x-x_0)$或$\delta(x-x_0)\delta(y-y_0)\delta(z-z_0)$.



\subsection{动量算符}

\pentry{平移算符\upref{tranOp}}

由经典力学,动量是平移生成元,\autoref{OprQM_eq1} 中的$G$应是动量算符.

又由\autoref{tranOp_eq1}~\upref{tranOp},可知一维无穷小平移算符为
\begin{equation}
P(\dd x) = \exp(-\dd x \cdot \partial_x) = 1-\dd x\partial_x
\end{equation}

代回\autoref{OprQM_eq1} ,注意$P(\dd x)$相当于$U(\varepsilon)$,即可得到一维动量算符:
\begin{equation}
p  = -\I\partial_x
\end{equation}





















