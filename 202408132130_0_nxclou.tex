% NextCloud 笔记
% license Usr
% type Note

\begin{itemize}
\item \href{https://nextcloud.com/install/}{官方安装页面(推荐用 AOI 即 docker)}
\begin{lstlisting}[language=bash]
# For Linux and without a web server
# or reverse proxy (like Apache, Nginx, Caddy,
# Cloudflare Tunnel and else) already in place:
sudo docker run \
--init \
--sig-proxy=false \
--name nextcloud-aio-mastercontainer \
--restart always \
--publish 80:80 \
--publish 8080:8080 \
--publish 8443:8443 \
--volume nextcloud_aio_mastercontainer:/mnt/docker-aio-config \
--volume /var/run/docker.sock:/var/run/docker.sock:ro \
nextcloud/all-in-one:latest
\end{lstlisting}
\item 其中 80 和 844 的两行可以删掉
\item \verb`https://ip地址或localhost:5002` 然后跳过安全警告即可访问
\item 命令中本机映射的文件夹在 docker 的默认映射文件夹 \verb`/var/lib/docker/volumes/nextcloud_aio_mastercontainer/`
\item 初始界面千万不要跳过去,记下密码!如果忘了密码可以在映射文件夹中的设置文件查看 \verb`_data/data/configuration.json`
\end{itemize}

彻底卸载 nextcloud docker (\href{https://github.com/nextcloud/all-in-one#how-to-properly-reset-the-instance}{参考}):
\begin{lstlisting}[language=bash]
sudo docker stop nextcloud-aio-mastercontainer
sudo docker stop nextcloud-aio-domaincheck
sudo docker network rm nextcloud-aio
sudo docker volume rm nextcloud_aio_mastercontainer
# 检查是否还有残留 volume
sudo docker volume ls --filter "dangling=true"
sudo docker volume ls --format {{.Name}}
\end{lstlisting}
