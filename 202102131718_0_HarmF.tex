% 调和场
% keys 散度|旋度|调和场|拉普拉斯

\begin{issues}
\issueDraft
\end{issues}

我们把散度和旋度都为零的场称为\textbf{调和场}. 注意这并不是一个常用的数学名词, 笔者只在个别中文教材中见过. 由于旋度为零, 积分与路径无关, 必定可以定义势函数 $u$, 而调和场就是其梯度
\begin{equation}
\bvec f = \div u
\end{equation}
要保证散度 $\div f$ 为零, 就要求 $u$ 是一个调和函数:
\begin{equation}
\laplacian u = 0
\end{equation}
所以调和场的充分必要条件是存在拉普拉斯为零的势函数.
