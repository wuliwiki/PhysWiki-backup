% 笛卡尔积
% keys 直积|有序数对|笛卡尔积
% license Usr
% type Tutor

在刚刚接触到乘法计算时,作为一个运算结果,“积”是作为与“乘法”相关的概念被引入的。后来,随着对向量的学习的深入,内积和外积逐渐也成为了熟悉的概念,二者分别与点乘($\cdot$)和叉乘($\times$)相对应。或许,“卷积”和“张量积”等概念也偶尔会出现在你的视野中。他们往往是与一个逐渐抽象的“乘法”相对应,说他逐渐抽象,是因为他与我们熟知的数的乘法的样子和计算方法相去甚远。而还称呼它是乘法,是因为某种程度上,它保留了乘法的一些特性。

下面会涉及到一点点的物理知识:在物理上,常常会有通过“乘法”来定义一个新的物理量,比如:功是力与位移的乘积,力矩是力与力臂的乘积,电路中功率是电流与电压的乘积(先忽略这个乘积具体的形式)等。这个新的物理量与原有的两个物理量之间都存在关系。而“加法”往往是在一个概念内部量的多少的计算,基本是不涉及其他概念的。

数学上有一个概念叫作\textbf{直积}(direct product),一般使用它来组合两个同类的已知对象,来定义新对象,比如:集合、群、模、拓扑空间等。而作用在两个集合上的直积便称为\textbf{笛卡尔积}(Cartesian product)。

\subsection{有序对}
有序对(ordered pair)是集合论和关系理论中的一个基本概念。

\begin{definition}{有序对}
在集合论中,一个有序对(a, b)是一种数据结构,表示两个元素的顺序对。
\end{definition}

唯一性和顺序性
最常见的定义方法是通过库拉托夫斯基对(Kuratoswki pair):

\[ (a, b) = \{\{a\}, \{a, b\}\} \]

这种定义方式确保了有序对的,即:

\[ (a, b) = (c, d) \iff a = c \text{ 且 } b = d \]

### 2. 有序数对的性质

- **唯一性**:通过库拉托夫斯基对的定义,有序对 \((a, b)\) 的定义是唯一的。这确保了有序对的精确表示。
- **顺序性**:有序对中的元素顺序是固定的,即 \((a, b) \neq (b, a)\) 除非 \(a = b\)。

### 3. 有序数对的应用

有序数对在很多数学领域中有广泛的应用:

- **关系和函数**:一个二元关系 \(R\) 可以看作是有序对的集合。例如,函数 \(f : A \to B\) 可以看作是满足 \(f(a) = b\) 的有序对 \((a, b)\) 的集合。
- **笛卡尔积**:集合 \(A\) 和 \(B\) 的笛卡尔积 \(A \times B\) 是所有有序对 \((a, b)\) 的集合,其中 \(a \in A\) 且 \(b \in B\)。
- **拓扑学**:在拓扑学中,点对的定义和处理也是通过有序对来实现的。

### 4. 深入研究

以下是一些资源和概念,可以帮助你更深入地理解有序数对的定义和应用:

#### 4.1 教科书和参考书
- **《Naive Set Theory》 by Paul R. Halmos**:这本书详细介绍了集合论的基础,包括有序对的定义和性质。
- **《Elements of Set Theory》 by Herbert B. Enderton**:这本书提供了更加系统和形式化的集合论介绍。

#### 4.2 相关概念
- **集合论的公理化**:了解Zermelo-Fraenkel集合论(ZF)和选择公理(AC)对理解有序对的定义是非常重要的。
- **关系和函数的表示**:研究有序对在关系和函数中的应用,可以加深对这个概念的理解。
- **笛卡尔积和拓扑空间**:学习笛卡尔积和拓扑空间中的点对定义,可以帮助理解有序对在不同数学领域中的应用。

### 参考资料
1. Halmos, Paul R. *Naive Set Theory*. Springer, 1974.
2. Enderton, Herbert B. *Elements of Set Theory*. Academic Press, 1977.
3. Wikipedia: [Ordered Pair](https://en.wikipedia.org/wiki/Ordered_pair)
4. MathWorld: [Ordered Pair](https://mathworld.wolfram.com/OrderedPair.html)

这些资源和概念将帮助你系统地理解和研究有序数对的定义和应用。
\subsection{笛卡尔积}

\textbf{笛卡尔积}是一个集合领域的概念,经常通过对。

