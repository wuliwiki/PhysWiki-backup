% 2015 年考研数学试题(数学一)
% keys 考研|数学
% license Copy
% type Tutor
\subsection{选择题}
\begin{enumerate}($\quad$)
\item 设函数 $f(x)$ 在 $(-\infty,+\infty)$ 上连续,其2阶导函数 $f''(x)$  的图形如右图所示,则曲线 $y=f(x)$ 的拐点个数为 ($\quad$)\\
(A)$\quad$0  $\quad$  (B)$\quad$1 $\quad$  (C)$\quad$2  $\quad$(D)$\quad$3 
\item 设 $y=\frac{1}{2}e^{2x}+(x-\frac{1}{3})e^x$ 是二阶常系数非齐次线性微分方程 $y''+ay'+by=ce^x$  的一个特解,则($\quad$)\\
(A) $a=-3,b=2,c=-1$\\
(B) $a=3,b=2,c=-1$\\
(C) $a=-3,b=2,c=1$\\
(D) $a=3,b=2,c=1$
\item 若级数 $\displaystyle \sum_{n=1}^\infty a_n $ 条件收敛,则 $x=\sqrt{3}$ 与 $x=3$  依次为幂级数 $\displaystyle \sum_{n=1}^\infty na_n(x-1)^n$  的($\quad$)
(A) 收敛点,收敛点\\
(B) 收敛点,发散点\\
(C) 发散点,收敛点\\
(D) 发散点,发散点
\item  设 $D$ 是第一象限中的曲线 $2xy=1,4xy=1$   与直线 $y=x,y=\sqrt{3}$  围成的平面区域,函数 $f(x,y)$   在 $D$ 上连续,则 $\displaystyle {\int \int}_D f(x,y)\dd{x}\dd{y}$=($\quad$)\\
(A)$\displaystyle  \int_\frac{\pi}{4}^\frac{\pi}{3}\dd{\theta}\int_\frac{1}{2\sin 2\theta}^\frac{1}{\sin 2\theta}f(r\cos \theta,r\sin \theta)r\dd{r}$\\
(B)$\displaystyle  \int_\frac{\pi}{4}^\frac{\pi}{3}\dd{\theta}\int_\frac{1}{\sqrt{2\sin 2\theta}}^\frac{1}{\sqrt{\sin 2\theta}}f(r\cos \theta,r\sin \theta)r\dd{r}$\\
(C) $\displaystyle  \int_\frac{\pi}{4}^\frac{\pi}{3}\dd{\theta}\int_\frac{1}{2\sin 2\theta}^\frac{1}{\sin 2\theta}f(r\cos \theta,r\sin \theta)\dd{r}$\\
(D)$\displaystyle  \int_\frac{\pi}{4}^\frac{\pi}{3}\dd{\theta}\int_\frac{1}{\sqrt{2\sin 2\theta}}^\frac{1}{\sqrt{\sin 2\theta}}f(r\cos \theta,r\sin \theta)\dd{r}$
\item 设矩阵 $\mat A=\pmat{1&1&1\\1&2&a\\1&4&a^2},\mat b=\pmat{1\\d\\d^2}$  若集合 $\Omega={1,2}$ 则线性方程组 $\mat{ Ax=b}$  有无穷多解的充分必要条件为($\quad$)\\
(A) $a\notin \Omega,d\notin \Omega$ \\
(B) $a\notin \Omega,d\in \Omega$\\
(C) $a\in \Omega,d\notin \Omega$\\
(D) $a\in \Omega,d\in \Omega$\\
\item 设二次型 $f(x_1,x_2,x_3)$ 在正交变换  $x=Py$ 下的标准型为 $2y_1^2+y_2^2-y_3^2$,其中 $P=(e_1,e_2,e_3)$ 。若 $Q=(e_1,-e_3,e_2)$ ,则 $f(x_1,x_2,x_3)$  在正交变换 $x=Qy$  下的标准形为($\quad$)\\
(A) $2y_1^2-y_2^2+y_3^2$\\
(B) $2y_1^2+y_2^2-y_3^2$\\
(C) $2y_1^2-y_2^2-y_3^2$\\
(D) $2y_1^2+y_2^2+y_3^2$
\item 若 $A,B$ 为任意两个随机事件,则($\quad$)\\
(A) $P(AB)\le P(A)P(B)$\\
(B) $P(AB) \ge P(A)P(B)$\\
(C) $\displaystyle P(AB)\le \frac{P(A)+P(B)}{2}$\\
(D) $\displaystyle P(AB)\ge \frac{P(A)+P(B)}{2}$
\item  设随机变量 $X,Y$ 不相关,且 $E(X)=2,E(Y)=1,D(X)=3$  ,则 $E[X(X+Y-2)]$=($\quad$)\\
(A) -3\\
(B) 3\\
(C) -5\\
(D) 5
\end{enumerate}
\subsection{填空题}
\begin{enumerate}
\item $\displaystyle \lim_{x \to 0} \frac{\ln(\cos x)}{x^2}$=($\quad$)
\item $\displaystyle \int_\frac{-\pi}{2}^\frac{\pi}{2}(\frac{\sin x}{1+\cos x}+\abs{x})\dd{x}$
\end{enumerate}
