% 2009 年计算机学科专业基础综合全国联考卷
% keys 2009年 计算机 考研 全国卷

\subsection{一、单项选择题}
第1~40 小题,每小题2 分,共80 分.下列每题给出的四个选项中,只有一个选项最符合试题要求.

1. 为解决计算机主机与打印机之间速度不匹配问题,通常设置一个打印数据缓冲区,主机将要输出的数据依次写入该缓冲区,而打印机则依次从该缓冲区中取出数据.该缓冲区的逻辑结构应该是. \\
A.栈 $\qquad$ B.队列 $\qquad$ C.树 $\qquad$ D.图

2. 设栈S和队列Q的初始状态均为空,元素a,b,c,d,e,f,g依次进入栈S.若每个元素出栈后立即进入队列Q,且7个元素出队的顺序是b,d,c,f,e,a,g,则栈S的容量至少是. \\
A.1 $\qquad$ B.2 $\qquad$ C.3 $\qquad$ D.4

3. 给定二叉树如图所示.设N 代表二叉树的根,L 代表根结点的左子树,R 代表根结点的右子树.若遍历后的结点序列是3,1,7,5,6,2,4,则其遍历方式是. \\
\begin{figure}[ht]
\centering
\includegraphics[width=5cm]{./figures/CSN09_1.png}
\caption{第3题图} \label{CSN09_fig1}
\end{figure}
A.LRN $\qquad$ B.NRL $\qquad$ C.RLN $\qquad$ D.RNL

4. 下列二叉排序树中,满足平衡二叉树定义的是______.\\
\begin{figure}[ht]
\centering
\includegraphics[width=14.25cm]{./figures/CSN09_2.png}
\caption{第4题图} \label{CSN09_fig2}
\end{figure}

5. 已知一棵完全二叉树的第6 层(设根为第1 层)有8 个叶结点,则该完全二叉树的结点个数最多是______. \\
A. 39 $\qquad$ B.52 $\qquad$ C.111 $\qquad$ D.119

6. 将森林转换为对应的二叉树,若在二叉树中,结点u是结点v的父结点的父结点,则在原来的森林中,u和v可能具有的关系是______. \\
Ⅰ.父子关系 $\qquad$ Ⅱ.兄弟关系 \\
Ⅲ.u的父结点与v的父结点是兄弟关系 \\
A. 只有Ⅱ $\qquad$ B.Ⅰ和Ⅱ $\qquad$ C.Ⅰ和Ⅲ $\qquad$ D.Ⅰ、Ⅱ和Ⅲ

7. 下列关于无向连通图特性的叙述中,正确的是______. \\
I. 所有顶点的度之和为偶数 \\
II. 边数大于顶点个数减1 \\
III. 至少有一个顶点的度为1 \\
A. 只有Ⅰ $\qquad$ B.只有Ⅱ $\qquad$ C.Ⅰ和Ⅱ $\qquad$ D.Ⅰ和Ⅲ

8. 下列叙述中,\textbf{不}符合m阶B树定义要求的是______. \\
A.根节点最多有m棵子树 $\qquad$ B.所有叶结点都在同一层上 \\
C.各结点内关键字均升序或降序排列 $\qquad$ D.叶结点之间通过指针链接

9. 已知关键字序列5,8,12,19,28,20,15,22 是小根堆(最小堆),插入关键字3,调整后得到的小根堆是______. \\
A.3,5,12,8,28,20,15,22,19 \\
B.3,5,12,19,20,15,22,8,28 \\
C.3,8,12,5,20,15,22,28,19 \\
D.3,12,5,8,28,20,15,22,19 \\

10. 若数据元素序列11,12,13,7,8,9,23,4,5 是采用下列排序方法之一得到的第二趟排序后的结果,
则该排序算法只能是______. \\
A.起泡排序 $\qquad$ B.插入排序 $\qquad$ C.选择排序 $\qquad$ D.二路归并排序

11. 冯•诺依曼计算机中指令和数据均以二进制形式存放在存储器中,CPU区分它们的依据是. \\
A.指令操作码的译码结果 $\qquad$ B.指令和数据的寻址方式 \\
C.指令周期的不同阶段 $\qquad$ D.指令和数据所在的存储单元

12. 一个C语言程序在一台32位机器上运行.程序中定义了三个变量x、y和z,其中x和z为int型,y为short型.当x=127,y=-9时,执行赋值语句z=x+y 后,x、y和z的值分别是. \\
A.x=0000007FH,y=FFF9H,z=00000076H \\
B. x=0000007FH,y=FFF9H,z=FFFF0076H \\
C. x=0000007FH,y=FFF7H,z=FFFF0076H \\
D. x=0000007FH,y=FFF7H,z=00000076H

13. 浮点数加、减运算过程一般包括对阶、尾数运算、规格化、舍入和判溢出等步骤.设浮点数的阶码和尾数均采用补码表示,且位数分别为5位和7位(均含2位符号位).若有两个数X=27×29/32,Y=25×5/8,则用浮点加法计算X+Y 的最终结果是. \\
A.00111 1100010 $\qquad$ B.00111 0100010 \\
C.01000 0010001 $\qquad$ D.发生溢出

14. 某计算机的Cache共有16块,采用2路组相联映射方式(即每组2块).每个主存块大小为32字节,按字节编址.主存129 号单元所在主存块应装入到的Cache组号是. \\
A.0 $\qquad$ B.1 $\qquad$ C.4 $\qquad$ D.6

15. 某计算机主存容量为64KB,其中ROM 区为4KB,其余为RAM 区,按字节编址.现要用2K×8 位的ROM芯片和4K×4 位的RAM 芯片来设计该存储器,则需要上述规格的ROM 芯片数和RAM 芯片数分别是. \\
A.1、15 $\qquad$ B.2、15 $\qquad$ C.1、30 $\qquad$ D.2、30

16. 某机器字长16位,主存按字节编址,转移指令采用相对寻址,由两个字节组成,第一字节为操作码字段,第二字节为相对位移量字段.假定取指令时,每取一个字节PC 自动加1.若某转移指令所在主存地址为2000H,相对位移量字段的内容为06H,则该转移指令成功转移后的目标地址是_____. \\
A.2006H $\qquad$ B.2007H $\qquad$ C.2008H $\qquad$ D.2009H

17. 下列关于 RISC 的叙述中,\textbf{错误}的是. \\
A.RISC 普遍采用微程序控制器 \\
B.RISC 大多数指令在一个时钟周期内完成 \\
C.RISC 的内部通用寄存器数量相对CISC多 \\
D.RISC 的指令数、寻址方式和指令格式种类相对CISC少

18. 某计算机的指令流水线由四个功能段组成,指令流经各功能段的时间(忽略各功能段之间的缓存时间)分别为90 ns、80ns、70ns、和60ns,则该计算机的CPU时钟周期至少是. \\
A.90 ns $\qquad$ B.80 ns $\qquad$ C.70 ns $\qquad$ D.60 ns

19. 相对于微程序控制器,硬布线控制器的特点是. \\
A.指令执行速度慢,指令功能的修改和扩展容易 \\
B.指令执行速度慢,指令功能的修改和扩展难 \\
C.指令执行速度快,指令功能的修改和扩展容易 \\
D.指令执行速度快,指令功能的修改和扩展难

20. 假设某系统总线在一个总线周期中并行传输4 字节信息,一个总线周期占用2 个时钟周期,总线时钟频率为10MHz,则总线带宽是______. \\
A.10MB/S $\qquad$ B.20MB/S $\qquad$ C.40MB/S $\qquad$ D.80MB/S

21. 假设某计算机的存储系统由Cache和主存组成,某程序执行过程中访存1000次,其中访问Cache缺失(未命中)50次,则Cache的命中率是. \\
A.5\% $\qquad$ B.9.5\% $\qquad$ C.50\% $\qquad$ D.95\%

22. 下列选项中,能引起外部中断的事件是. \\
A.键盘输入 $\qquad$ B.除数为0 \\
C.浮点运算下溢  $\qquad$ D.访存缺页

23. 单处理机系统中,可并行的是. \\
Ⅰ 进程与进程 $\qquad$ Ⅱ 处理机与设备 $\qquad$ Ⅲ 处理机与通道  $\qquad$ Ⅳ 设备与设备 \\
A.Ⅰ、Ⅱ和Ⅲ $\qquad$ B.Ⅰ、Ⅱ和Ⅳ \\
C.Ⅰ、Ⅲ和Ⅳ $\qquad$ D.Ⅱ、Ⅲ和Ⅳ

24. 下列进程调度算法中,综合考虑进程等待时间和执行时间的是______. \\
A.时间片轮转调度算法 $\qquad$ B.短进程优先调度算法 \\
C.先来先服务调度算法 $\qquad$ D.高响应比优先调度算法

25. 某计算机系统中有8台打印机,由K个进程竞争使用,每个进程最多需要3台打印机.该系统可能会发生死锁的K的最小值是______. \\
A.2 $\qquad$ B.3 $\qquad$ C.4 $\qquad$ D.5

26. 分区分配内存管理方式的主要保护措施是______. \\
A.界地址保护 $\qquad$ B.程序代码保护 $\qquad$ C.数据保护 $\qquad$ D.栈保护

27. 一个分段存储管理系统中,地址长度为32位,其中段号占8位,则最大段长是______. \\
A.$2^8$字节 $\qquad$ B.$2^{16}$字节 $\qquad$ C.$2^{24}$字节 $\qquad$ D.$2^{32}$字节

28. 下列文件物理结构中,适合随机访问且易于文件扩展的是______. \\
A.连续结构 $\qquad$ B.索引结构 \\
C.链式结构且磁盘块定长 $\qquad$ D.链式结构且磁盘块变长

29. 假设磁头当前位于第105道,正在向磁道序号增加的方向移动.现有一个磁道访问请求序列为35,45,12,68,110,180,170,195,采用SCAN 调度(电梯调度)算法得到的磁道访问序列是______. \\
A.110,170,180,195,68,45,35,12 $\qquad$ B.110,68,45,35,12,170,180,195 \\
C.110,170,180,195,12,35,45,68 $\qquad$ D.12,35,45,68,110,170,180,195

30. 文件系统中,文件访问控制信息存储的合理位置是______. \\
A.文件控制块 $\qquad$ B.文件分配表 $\qquad$ C.用户口令表 $\qquad$ D.系统注册表 \\

31. 设文件F1的当前引用计数值为1,先建立F1的符号链接(软链接)文件F2,再建立F1的硬链接文件F3,然后删除F1.此时,F2和F3的引用计数值分别是______. \\
A. 0、1 $\qquad$ B.1、1 $\qquad$ C.1、2 $\qquad$ D.2、1

32. 程序员利用系统调用打开I/O 设备时,通常使用的设备标识是______. \\
A.逻辑设备名 $\qquad$ B.物理设备名 \\
C.主设备号 $\qquad$ D.从设备号

33. 在OSI参考模型中,自下而上第一个提供端到端服务的层次是______. \\
A.数据链路层 $\qquad$ B.传输层 $\qquad$ C.会话层 $\qquad$ D.应用层

34. 在无噪声情况下,若某通信链路的带宽为3kHz,采用4 个相位,每个相位具有4 种振幅的QAM调制技术,则该通信链路的最大数据传输速率是______. \\
A.12kbps $\qquad$ B.24 kbps $\qquad$ C.48 kbps $\qquad$ D.96 kbps

35. 数据链路层采用后退N 帧(GBN)协议,发送方已经发送了编号为0~7 的帧.当计时器超时时,若发送方只收到0、2、3 号帧的确认,则发送方需要重发的帧数是______. \\
A.2 $\qquad$ B.3 $\qquad$ C.4 $\qquad$ D.5

36. 以太网交换机进行转发决策时使用的PDU 地址是______. \\
A.目的物理地址 $\qquad$ B.目的IP 地址 \\
C.源物理地址 $\qquad$ D.源IP 地址

37. 在一个采用CSMA/CD协议的网络中,传输介质是一根完整的电缆,传输速率为1Gbps,电缆中的信号传播速度是200 000km/s .若最小数据帧长度减少800 比特,则最远的两个站点之间的距离至少需要______. \\
A.增加160m $\qquad$ B.增加80m \\
C.减少160m $\qquad$ D.减少80m

38. 主机甲与主机乙之间已建立一个TCP 连接,主机甲向主机乙发送了两个连续的TCP 段,分别包含300 字节和500 字节的有效载荷,第一个段的序列号为200,主机乙正确接收到两个段后,发送给主机甲的确认序列号是______. \\
A.500 $\qquad$ B.700 $\qquad$ C.800 $\qquad$ D.1000

39. 一个TCP连接总是以1KB的最大段长发送TCP 段,发送方有足够多的数据要发送.当拥塞窗口为16KB时发生了超时,如果接下来的4 个RTT(往返时间)时间内的TCP 段的传输都是成功的,那么当第4 个RTT 时间内发送的所有TCP 段都得到肯定应答时,拥塞窗口大小是______. \\
A.7 KB $\qquad$ B.8 KB $\qquad$ C.9 KB $\qquad$ D.16 KB

40. FTP 客户和服务器间传递FTP 命令时,使用的连接是______. \\
A.建立在TCP之上的控制连接 $\qquad$ B.建立在TCP之上的数据连接 \\
C.建立在UDP之上的控制连接 $\qquad$ D.建立在UDP之上的数据连接

\subsection{二、综合应用题}
第41~47题,共70分.

41. (10 分)带权图(权值非负,表示边连接的两顶点间的距离)的最短路径问题是找出从初始顶点到目标顶点之间的一条最短路径.假设从初始顶点到目标顶点之间存在路径,现有一种解决该问题的方法: \\
① 设最短路径初始时仅包含初始顶点,令当前顶点u为初始顶点; \\
② 选择离u最近且尚未在最短路径中的一个顶点v,加入到最短路径中,修改当前顶点u=v; \\
③ 重复步骤②,直到u是目标顶点时为止. \\
请问上述方法能否求得最短路径?若该方法可行,请证明之;否则,请举例说明.

42. (15 分)已知一个带有表头结点的单链表,结点结构为:
\begin{table}[ht]
\centering
\caption{链表结构}\label{CSN09_tab1}
\begin{tabular}{|c|c|}
\hline
data & link \\
\hline
\end{tabular}
\end{table}
假设该链表只给出了头指针list.在不改变链表的前提下,请设计一个尽可能高效的算法,查找链表中倒数第k 个位置上的结点( k 为正整数).若查找成功,算法输出该结点的data 域的值,并返回1;否则,只返回0.要求: \\
⑴ 描述算法的基本设计思想; \\
⑵ 描述算法的详细实现步骤; \\
⑶ 根据设计思想和实现步骤,采用程序设计语言描述算法(使用C、C++或Java 语言实现),关键之处请给出简要注释.