% 伯恩哈德·黎曼(综述)
% license CCBYSA3
% type Wiki

本文根据 CC-BY-SA 协议转载翻译自维基百科\href{https://en.wikipedia.org/wiki/Bernhard_Riemann}{相关文章}。

\begin{figure}[ht]
\centering
\includegraphics[width=6cm]{./figures/e77e2d063c32a0a3.png}
\caption{黎曼,大约1863年} \label{fig_BEHDLM_1}
\end{figure}
乔治·弗里德里希·伯恩哈德·黎曼(德语:[ˈɡeːɔʁk ˈfʁiːdʁɪç ˈbɛʁnhaʁt ˈʁiːman] ⓘ;1826年9月17日 – 1866年7月20日)是德国数学家,对分析学、数论和微分几何学做出了深远的贡献。在实分析领域,他最著名的是首次严格提出的黎曼积分及其在傅里叶级数方面的工作。他在复分析方面的贡献,尤其是引入了黎曼曲面,为复分析的自然几何处理开辟了新天地。他1859年关于素数计数函数的论文,其中包含了黎曼猜想的原始表述,被认为是分析数论的基础性论文。通过在微分几何学方面的开创性贡献,黎曼为广义相对论的数学奠定了基础。[3] 许多人认为他是历史上最伟大的数学家之一。[4][5]
\subsection{传记}  
\subsubsection{早年}  
黎曼于1826年9月17日出生在布雷泽伦茨(Breselenz),一个位于汉诺威王国丹嫩贝尔格附近的村庄。他的父亲弗里德里希·伯恩哈德·黎曼是布雷泽伦茨的贫穷路德宗牧师,曾参加拿破仑战争。他的母亲夏洛特·艾贝尔于1846年去世。黎曼是六个孩子中的第二个。黎曼从小展现出卓越的数学才能,如出色的计算能力,但他也患有羞怯和害怕公开演讲的症状。
\subsubsection{教育}  
1840年,黎曼前往汉诺威与祖母同住,并就读于文法学校(相当于中学),因为他所在的村庄没有这样的学校。1842年,祖母去世后,黎曼转学至吕讷堡的约翰纽姆中学。在那里,黎曼集中特别学习圣经,但他经常被数学分心。他的老师们惊讶于他的数学运算能力,黎曼经常超越老师的知识水平。1846年,19岁的黎曼开始学习语言学和基督教神学,打算成为一名牧师,并帮助家庭经济。

在1846年春天,他的父亲积攒了一些钱,将黎曼送到哥廷根大学,原计划学习神学并获得神学学位。然而,一到哥廷根,黎曼便开始在卡尔·弗里德里希·高斯的指导下学习数学(尤其是高斯的最小二乘法讲座)。高斯建议黎曼放弃神学事业,转而进入数学领域;得到父亲同意后,黎曼于1847年转学至柏林大学。在柏林学习期间,卡尔·古斯塔夫·雅各布·雅可比、彼得·古斯塔夫·勒让·狄利克雷、雅各布·斯坦纳和戈特霍尔德·艾森斯坦等数学家为他授课。黎曼在柏林待了两年,直到1849年才返回哥廷根。
\subsubsection{学术生涯}  
黎曼于1854年首次讲授课程,创立了黎曼几何学领域,为阿尔伯特·爱因斯坦的广义相对论奠定了基础。[7] 1857年,曾尝试将黎曼晋升为哥廷根大学的特任教授,尽管未能成功,但该尝试促使黎曼最终获得了定期薪水。1859年,随着狄利克雷(当时担任高斯的讲席教授)去世,黎曼被提升为哥廷根大学数学系主任。他也是第一个提出使用三维以上的维度来描述物理现实的人。[8][7]

1862年,他与埃莉丝·科赫结婚;他们的女儿伊达·施林于1862年12月22日出生。[9]
\subsubsection{新教家庭与意大利的死亡}
黎曼在1866年汉诺威和普鲁士军队在哥廷根交战时逃离了哥廷根。[10] 他在第三次前往意大利的途中死于结核病,地点是塞拉斯卡(现为马焦雷湖上的一个小村庄,隶属于维尔巴尼亚),他被埋葬在比甘佐洛的墓地(维尔巴尼亚)。  
黎曼是一个虔诚的基督徒,出生于一位新教牧师的家庭,他将自己作为数学家的生活视为服务上帝的一种方式。在他的一生中,他始终坚持自己的基督教信仰,并认为这是他生活中最重要的部分。在他去世时,他正在与妻子一起诵念《主祷文》,并在他们尚未念完时去世。[11] 与此同时,在哥廷根,他的女管家丢弃了他办公室中的一些文件,包括许多未发表的作品。黎曼拒绝发表不完整的工作,因此一些深刻的见解可能已经丧失。[10]

黎曼位于比甘佐洛(意大利)的墓碑上刻有《罗马书》8:28的经文:[12]

在神的怀抱中安息
乔治·弗里德里希·伯恩哈德·黎曼
哥廷根大学教授
生于1826年9月17日,布雷塞伦茨
死于1866年7月20日,塞拉斯卡
对于那些爱神的人,一切都必定互相效力,为最好的结果
\subsection{黎曼几何学}
黎曼的已发表作品开启了将分析与几何相结合的研究领域。随后,这些领域成为了黎曼几何学、代数几何学和复流形理论的主要组成部分。黎曼曲面的理论由费利克斯·克莱因和特别是阿道夫·赫尔维茨进一步阐述。这一数学领域是拓扑学的基础部分,并且至今仍在数学物理学中以新颖的方式得到应用。

1853年,高斯要求他的学生黎曼准备一篇关于几何基础的Habilitationsschrift。在经过数月的研究后,黎曼发展了他的高维理论,并于1854年6月10日在哥廷根发表了题为《几何学基础假设探讨》(Ueber die Hypothesen, welche der Geometrie zu Grunde liegen)的讲座。直到1868年,在黎曼去世两年后,德德金才将其出版。早期的接受情况似乎较为缓慢,但现在它被公认为几何学中最重要的著作之一。

这篇著作奠定了黎曼几何学的基础。黎曼找到了将高维流形的微分几何扩展到n维的方法,而这一方法是高斯在他的《著名定理》中所证明的。基本对象被称为黎曼度量和黎曼曲率张量。对于二维的曲面情况,曲率可以在每个点上归结为一个数(标量),具有恒定正曲率或负曲率的曲面是非欧几何的模型。

黎曼度量是每个空间点上的一组数字(即张量),它允许在任何轨迹中测量速度,其积分给出了轨迹端点之间的距离。例如,黎曼发现,在四维空间中,每个点需要十个数字来描述流形上的距离和曲率,无论流形如何扭曲。
\subsection{复变函数论}
在他的博士论文中,黎曼通过黎曼曲面为复变函数论奠定了几何基础,借此多值函数,如对数函数(具有无限个分支)或平方根函数(具有两个分支),可以转化为单值函数。在这些曲面上,复函数是调和函数(即它们满足拉普拉斯方程,从而也满足柯西-黎曼方程),并且可以通过它们的奇点位置和曲面的拓扑结构来描述。黎曼曲面的拓扑“基因”由以下公式给出:\(g = \frac{w}{2} - n + 1\)其中曲面有 \(n\) 个叶片在 \(w\) 个分支点处汇合。对于 \(g > 1\),黎曼曲面有 \((3g - 3)\) 个参数(即“模量”)。

黎曼在这一领域的贡献很多。著名的黎曼映射定理指出,复平面中的一个单连通域是“双全纯同构”的(即存在一个双射,该双射是全纯的,且具有全纯逆函数),它要么等同于复平面 \(\mathbb{C}\),要么等同于单位圆的内部。这个定理对黎曼曲面的推广就是著名的统一化定理,这一理论在19世纪由亨利·庞加莱和费利克斯·克莱因证明。这里,严格的证明是在数学工具(特别是拓扑学)发展后才给出的。对于黎曼曲面上函数存在性的证明,黎曼使用了一种最小化条件,他称之为狄利克雷原理。卡尔·魏尔斯特拉斯发现了这个证明中的一个漏洞:黎曼没有注意到他的假设(即最小值的存在性)可能并不成立;函数空间可能不完备,因此最小值的存在性不能得到保证。通过大卫·希尔伯特在变分法中的工作,狄利克雷原理最终得到了建立。魏尔斯特拉斯对黎曼非常赞赏,特别是黎曼的阿贝尔函数理论。当黎曼的工作发表时,魏尔斯特拉斯撤回了他在《克雷尔期刊》上的文章,并未发表。黎曼在1859年访问魏尔斯特拉斯时,两人有了很好的理解。魏尔斯特拉斯鼓励他的学生赫尔曼·阿曼杜斯·施瓦茨在复分析中寻找狄利克雷原理的替代方法,施瓦茨成功地做到了这一点。阿诺德·索末菲尔德的一个轶事显示了当时的数学家们对黎曼新思想的困难理解。1870年,魏尔斯特拉斯带着黎曼的论文去瑞吉度假,并抱怨它很难理解。物理学家赫尔曼·冯·亥姆霍兹在一夜之间帮助他完成了工作,第二天回来说,这些内容“很自然”且“非常易懂”。

其他的亮点包括黎曼在黎曼曲面上的阿贝尔函数和θ函数的研究。自1857年以来,黎曼与魏尔斯特拉斯一直竞争,解决阿贝尔积分的雅可比反问题,这是椭圆积分的一个推广。黎曼使用多变量的θ函数,并将问题简化为确定这些θ函数的零点。黎曼还研究了周期矩阵,并通过“黎曼周期关系”对其进行了表征(对称的、实部为负)。由费迪南德·乔治·弗罗贝纽斯和所罗门·莱夫谢茨证明,这一关系的有效性等价于通过θ函数将\(\mathbb{C}^n/\Omega\)(其中\(\Omega\)是周期矩阵的格子)嵌入投影空间。对于某些 \(n\) 的值,这就是黎曼曲面的雅可比流形,属于阿贝尔流形的一个例子。

许多数学家,如阿尔弗雷德·克莱布施,进一步发展了黎曼在代数曲线方面的工作。这些理论依赖于在黎曼曲面上定义的一个函数的性质。例如,黎曼-罗赫定理(罗赫是黎曼的学生)描述了黎曼曲面上一组线性无关的不同ials的数量(其中包括对零点和极点的已知条件)。

根据德特雷夫·劳茨(Detlef Laugwitz)的说法,自动性函数首次出现在关于带电圆柱上拉普拉斯方程的文章中。然而,黎曼在1859年关于超几何函数的讲座中使用了这样的函数用于保形映射(例如将拓扑三角形映射到圆上),或在他关于最小曲面的论著中使用。
\subsection{实变函数论}
在实变函数论领域,黎曼在他的Habilitation论文中发现了黎曼积分。他证明了每个分段连续的函数都是可积的。此外,斯蒂尔捷斯积分也源自哥廷根的数学家,因此这两者一起被称为黎曼-斯蒂尔捷斯积分。

在他关于傅里叶级数的Habilitation论文中,黎曼继承了他导师狄利克雷的工作,证明了黎曼可积的函数可以通过傅里叶级数“表示”。狄利克雷已证明了这一点对于连续、分段可微的函数(即具有可数多个不可微点)。黎曼给出了一个傅里叶级数的例子,这个傅里叶级数表示一个连续的、几乎处处不可微的函数,这种情况狄利克雷并未涵盖。他还证明了黎曼-勒贝格引理:如果一个函数可以通过傅里叶级数表示,那么傅里叶系数在大 \(n\) 时趋于零。

黎曼的论文也是乔治·康托尔关于傅里叶级数研究的起点,这一研究促成了集合论的诞生。

黎曼还在1857年使用复分析方法研究了超几何微分方程,并通过描述奇点周围闭合路径的行为(由单值矩阵描述)展示了解决方案。通过已知的单值矩阵证明此类微分方程的存在性是希尔伯特问题之一。
\subsection{数论}
黎曼对现代解析数论做出了几项著名的贡献。在他唯一一篇关于数论的短文中,他研究了现在以他名字命名的黎曼ζ函数,确立了它在理解素数分布中的重要性。黎曼猜想是他关于该函数性质的一系列猜想中的一个。

黎曼的工作中还有许多其他有趣的发展。他证明了ζ函数的函数方程(这一方程早已由莱昂哈德·欧拉知道),其背后隐藏着一个θ函数。通过对在实部为1/2的非平凡零点上对该近似函数的求和,他给出了\(\pi(x)\)的一个精确的“显式公式”。

黎曼知道帕夫努季·切比雪夫关于素数定理的研究。1852年,他曾访问过狄利克雷。
\subsection{著作}
黎曼的著作包括:
\begin{itemize}
\item 1851年 – 《变动复数函数的一般理论基础》,博士论文,哥廷根,1851年。
\item 1857年 – 《阿贝尔函数的理论》,《纯粹与应用数学杂志》, 第54卷,第101-155页。
\item 1859年 – 《给定大小下素数的数量》,收录于《普鲁士科学院月报》,柏林,1859年11月,第671页起。包含黎曼猜想。《给定大小下素数的数量》。(维基文库)手稿的复制品,已存档于Wayback Machine,Clay Mathematics。
\item 1861年 – 《数学论文,尝试回答由巴黎科学院提出的问题》,提交给巴黎科学院的奖学金竞赛。
\item 1867年 – 《一个函数能否由三角级数表示》,《哥廷根皇家科学学会学报》第13卷。
\item 1868年 – 《几何学基础假设探讨》,《哥廷根皇家科学学会学报》,1868年。翻译:EMIS,PDF文件《On the hypotheses which lie at the foundation of geometry》,由W.K. Clifford翻译,《自然》杂志,1873年第8期,183页 – 重新出版于Clifford的《数学文集》,伦敦1882年(MacMillan);纽约1968年(Chelsea)http://www.emis.de/classics/Riemann/。也收录于Ewald, William B.(编),1996年《从康德到希尔伯特:数学基础的文献源》2卷,牛津大学出版社:652–661页。
\item 1876年 – 《伯恩哈德·黎曼的数学著作与科学遗产》,由海因里希·韦伯(Heinrich Weber)在理查德·德德金(Richard Dedekind)协助下编辑,莱比锡:B.G. Teubner,1876年,第2版1892年,1953年Dover再版(包含马克斯·诺特(Max Noether)和威廉·维廷格(Wilhelm Wirtinger)贡献,Teubner,1902年)。后续版本《伯恩哈德·黎曼全集:完整的德文版》,编辑:海因里希·韦伯、理查德·德德金、马克斯·诺特、威廉·维廷格、汉斯·莱维(Hans Lewy)。纽约:Dover出版公司,1953年、1981年、2017年。
\item 1876年 – 《重力、电力与磁性》,汉诺威:卡尔·哈滕多夫(Karl Hattendorff)。
\item 1882年 – 《偏微分方程讲义》第3版,布伦斯维克,1882年。
\item 1901年 – 《黎曼讲义中的数学物理学偏微分方程》,PDF版在Wikimedia Commons上可用。存档于archive.org: Riemann, Bernhard (1901). Weber, Heinrich Martin (ed.). "Die partiellen differential-gleichungen der mathematischen physik nach Riemann's Vorlesungen". archive.org. Friedrich Vieweg und Sohn. 于2022年6月1日检索。
\item 2004年 – 《黎曼全集》,肯德里克出版社,Heber City, UT,ISBN 978-0-9740427-2-5,MR 2121437
\end{itemize}
\subsection{另见}
\begin{itemize}
\item 以伯恩哈德·黎曼命名的事物列表
\item 非欧几何
\item 《给定大小下素数的数量》,黎曼1859年的论文,介绍了复数ζ函数
\end{itemize}