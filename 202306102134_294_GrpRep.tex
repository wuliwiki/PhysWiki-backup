% 群表示
% keys 表示|群论|置换群|线性空间|同态

% \begin{issues}
% \end{issues}

\pentry{群作用\upref{Group3}, 域上的代数\upref{AlgFie}}
\subsection{表示}

字面上讲,\textbf{表示(representation)}是用一个易于理解的表达方法来描述一些数学对象。比如说,对于学龄前的小朋友,想理解$1+1=2$ 的概念可能过于抽象,不利于理解,这个时候可以使用“一个苹果加一个苹果等于两个苹果”来表示相同的概念,会更容易理解。

数学上讲,\textbf{表示论}是研究对称性的学科,它把一个抽象的代数结构的元素映射成一个相对具体线性变换。其中群是表示论所研究的最简单的代数结构,这个学科就被称为\textbf{群表示论}。\cite{GTM222}\cite{维声表示}

\subsection{群的表示}

% Giacomo: 这什么?
% 
% 最直观和易于讨论的群,莫过于置换群。事实上,当伽罗华(Galois)第一次提出群的概念的时候,并没有像我们现代理论那样高度抽象和严格;他主要都在讨论置换群的性质。遗憾的是,当年的数学家们都迷惑于伽罗华研究这种东西的意义何在。
% 
% 我们使用置换群来尝试表示任意的群。
% 
% \begin{definition}{群在置换群上的表示}
% 设有群 $G$ 和一个置换群 $S_n$。如果存在\textbf{同态}$\phi: G\rightarrow S_n$,那么我们称 $\phi$ 是群 $G$ 在 $S_n$ 上的一个\textbf{表示}。
% \end{definition}
% 
% 我们也可以使用线性空间来对群进行表示,利用线性空间的线性变换。

\begin{definition}{群的(线性)表示}
设有群 $G$ 和一个线性空间 $V$,记 $V$ 上的全体可逆线性变换为 $\opn{GL}(V)$\footnote{一般线性群\autoref{def_GL_1}~\upref{GL}}。如果存在\textbf{同态}$\phi: G\rightarrow \opn{GL}(V)$,那么我们称 $(V, \phi)$ 是群 $G$ 在 $V$ 上的一个\textbf{群表示}。在不会产生歧义时,我们把 $\phi(g)(v)$ 简记做 $g \cdot v$。
\end{definition}

注:有些时候我们会直接称 $V$ 是群表示,此时群表示是一个代数结构;有时候我们又会称 $\rho$ 为群表示,此时群表示是一个函数。为了避免歧义,可以称 $V$ 为\textbf{表示空间}, $\rho$ 为\textbf{表示映射}。

群表示是一种特殊的群作用\autoref{def_Group3_1}~\upref{Group3},而且一个群作用可以诱导出一个与之相关的群表示。

在选定线性空间的基后,群的线性变换与表示矩阵相对应,有变换规则:$P(g)v_\nu=D(g)_{\nu\mu}v_\mu$

作为一个例子,有循环群$C_3$的一个一维表示为:$D(e)=1$,$D(a)=\rm{e}^{\frac{2\pi i}{3}}$,$D(a^2)=\rm{e}^{\frac{4\pi i}{3}}$,可见群的表示矩阵的乘法规则与群乘法表相同。

% Giacomo: 要不要移动到 群代数与正则表示\upref{gpalg} 里?

\begin{definition}{形式代数}
设有域 $\mathbb{F}$ 和集合 $S$, 我们可以定义一个 $\mathbb{F}$-代数,其元素为 $S$ 中元素的形式线性组合,记做
$$
\mathbb{F}[S]: = \left\{ \sum a_i s_i \mid \text{有限个非零} a_i \right\}~.
$$
\end{definition}
\addTODO{名字不确定}
\addTODO{例子}

\begin{definition}{群作用诱导的群表示}
设有群 $G$ 到集合 $S$ 的群作用 $\rho: G \to \opn{Aut}(S)$,我们可以得到一个 $G$ 在 $\mathbb{F}[S]$ 上表示 $(\mathbb{F}[S], \phi)$,
$$\begin{aligned}
\phi: G &\to \opn{GL}(\mathbb{F}[S]) \\
\phi(g)(\sum a_i s_i) &= \sum a_i \rho(g)(s_i)~.
\end{aligned}$$
\end{definition}
\addTODO{例子}

\subsection{等变映射(同态)}

就像我们研究所有的代数结构一样,我们也要为群的表示之间定义“相等”的概念,也就是线性表示之间的同态。类似于所有代数结构上的同态,线性表示的同态也是保持“运算结果”,也就是表示对群的作用不变的映射。

\begin{definition}{等变映射(同态映射),同构映射}
$G$ 的两个表示 $(V, \phi), (W, \psi)$ 之间的一个\textbf{等变映射(同态)}为 $f: (V, \phi) \to (W, \psi)$,
$$
\psi(g) \circ f = f \circ \phi(g)~,
$$
或者更简单的记做
$$
g \cdot f(v) = f(g \cdot v)~.
$$

如果 $f$ 可逆的话,$f$ 被称为\textbf{同构映射};

如果两个表示之间存在一个同构映射的话,就称这两个表示是\textbf{同构的}。
\end{definition}
% Giacomo: 感觉这样写百科没意思啊

\addTODO{商表示、不可约表示}

\subsection{子表示}
\addTODO{这里的内容可能有问题, 待讨论}


类似所有的代数结构,我们可以引入在$V$的子空间上引入子表示的概念,但是显然需要对子空间有一定的限制条件。

\begin{definition}{不变子空间}
设$(V,\varphi)$是群$G$的一个表示。$V$的一个子空间$U$如果是线性变换$\varphi(g)$的不变子空间,$\forall g\in G$,即对任意$g\in G$有$\varphi(g)(U)\subseteq U$,那么称$U$是\textbf{表示$\varphi$的不变子空间}或\textbf{$G$不变子空间}。
\end{definition}
\addTODO{参考一般的不变子空间\upref{InvSP}}

从而在不变子空间$U$上,$\varphi(g)|U$仍是线性变换且可逆(不是不变就没法可逆),我们就可以得到相应的子表示。

\begin{definition}{子表示}
设$(V,\varphi)$是群$G$的一个表示。$U\neq\{0\}$是$G$不变子空间,令
\begin{equation}
\varphi_U(g) := \varphi(g)|U,\quad\forall g\in G~,
\end{equation}
则得到一个表示$(U,\varphi_U)$,称为$\varphi$的一个\textbf{子表示}。
\end{definition}
