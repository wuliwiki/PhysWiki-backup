% 幂级数与解析函数

\pentry{泰勒公式\upref{Tayl}}

按照泰勒公式, 一个在定义域内无穷次可微的函数在任何一点都可以用它的泰勒级数的部分和进行逼近:
$$
\begin{aligned}
f(x)=&f(x_0)+\frac{f'(x_0)}{1!}(x-x_0)+\frac{f''(x_0)}{2!}(x-x_0)^2+\cdots\\
&+\frac{f^{(n)}(x_0)}{n!}(x-x_0)^n+o((x-x_0)^n).
\end{aligned}
$$
然而, 如果执意要将右端的有限和扩展为无穷级数, 那么立刻就会出现两个问题: 这个级数收敛吗? 如果收敛, 它能够收敛到左端吗?

一般来讲, 答案都是否定的. 

\begin{theorem}{博雷尔 (Borel) 定理}
设$\{a_n\}$是任意复数序列. 则存在一个光滑函数$f:(-1,1)\to\mathbb{C}$, 使得$f^{(n)}(0)=a_n$.
\end{theorem}
博雷尔定理告诉我们: 任何复数序列都能够成为某个光滑函数在某一点处的导数值序列. 由此构成的泰勒级数
$$
\sum_{n=0}^\infty\frac{a_n}{n!}x^n
$$
当然可能发散, 比如取$a_n=(n!)^2$.

即便泰勒级数是收敛的, 它也不一定能够收敛到被展开的函数. 一个典型的例子是
$$
f(x)=\exp\left(-\frac{1}{x^2}\right),
$$
这里补充定义$f(0)=0$. 直接计算可以看出$f^{(n)}(0)=0$, 所以它的泰勒级数恒为零. 因此, $f(x)$的泰勒级数部分和同$f(x)$自身的偏差永远是$f(x)$本身.

\begin{exercise}{}
用归纳法证明: 对于正整数$n$, 有一个多项式$P_n$使得
$$
f^{(n)}(x)=\exp\left(-\frac{1}{x^2}\right)P_n\left(\frac{1}{x}\right).
$$
例如, 
$$
f'(x)=\frac{2}{x^3}\exp\left(-\frac{1}{x^2}\right).
$$
由此证明$f^{(n)}(0)=0$.
\end{exercise}