% 2014 年计算机学科专业基础综合全国联考卷
% keys 2014 计算机 考研 真题 全国卷

\subsection{一、单项选择题}
第1~40 小题,每小题2 分,共80 分.下列每题给出的四个选项中,只有一个选项最符合试题要求.

1.下列程序段的时间复杂度是:
\begin{lstlisting}[language=cpp]
count=0;
for(k=1;k<=n;k*=2)
  for(j=1;j<=n;j++)
    count++;
\end{lstlisting}
A.$O(log2n)$ $\quad$ B.$O(n)$ $\quad$ C.$O(nlog2n)$ $\quad$ D.$O(n2)$

2.假设栈初始为空,将中缀表达式$a/b+(c*d-e*f)/g$转换为等价的后缀表达式的过程中,当扫描到$f$时,栈中的元素依次是. \\
A.$+ ( * -$  $\quad$ B.$+ ( - *$  $\quad$ C.$/ + ( * - *$  $\quad$ D.$/ + - *$

3.循环队列放在一维数组$A[0...M-1]$中,$end1$指向队头元素,$end2$指向队尾元素的后一个位置.假设队列两端均可进行入队和出队操作,队列中最多能容纳$M-1$ 个元素.初始时为空.下列判断队空和队满的条件中,\textbf{正确}的是. \\
A.队空:end1 == end2; 队满:end1 == (end2+1)mod M \\
B.队空:end1 == end2; 队满:end2 == (end1+1)mod (M-1) \\
C.队空:end2 == (end1+1)mod M; 队满:end1 == (end2+1)mod M \\
D.队空:end1 == (end2+1)mod M;队满:end2 == (end1+1)mod (M-1)

4.若对如下的二叉树进行中序线索化,则结点$x$的左、右线索指向的结点分别是:
\begin{figure}[ht]
\centering
\includegraphics[width=5cm]{./figures/CSN14_1.png}
\caption{第3题图} \label{CSN14_fig1}
\end{figure}
A.e、c $\quad$ B.e、a $\quad$ C.d、c $\quad$ D.b、a

5.将森林F转换为对应的二叉树T,F中叶结点的个数等于. \\
A.T 中叶结点的个数 $\quad$ B.T 中度为1 的结点个数 \\
C.T 中左孩子指针为空的结点个数 $\quad$ D.T 中右孩子指针为空的结点个数

6.5个字符有如下4种编码方案,\textbf{不是}前缀编码的是: \\
A.01,0000,0001,001,1 $\quad$ B.011,000,001,010,1 \\
C.000,001,010,011,100 $\quad$ D.0,100,110,1110,1100

7.对如下所示的有向图进行拓扑排序,得到的拓扑序列可能是: \\
A.3,1,2,4,5,6 $\quad$ B.3,1,2,4,6,5 \\
C.3,1,4,2,5,6 $\quad$ D.3,1,4,2,6,5
\begin{figure}[ht]
\centering
\includegraphics[width=10cm]{./figures/CSN14_2.png}
\caption{第7题图} \label{CSN14_fig2}
\end{figure}

8.用哈希(散列)方法处理冲突(碰撞)时可能出现堆积(聚集)现象,下列选项中,会受堆积现象直接影响的是 \\
A.存储效率 $\quad$ B.散列函数 $\quad$ C.装填(装载)因子 $\quad$ D.平均查找长度

9.在一棵具有15个关键字的4阶B树中,含关键字的结点个数最多是: \\
A.5 $\quad$ B.6 $\quad$ C.10 $\quad$ D.15

10.用希尔排序方法对一个数据序列进行排序时,若第1趟排序结果为9,1,4,13,7,8,20,23,15,则该趟排序采用的增量(间隔)可能是 \\
A.2 $\quad$ B.3 $\quad$ C.4 $\quad$ D.5

11.下列选项中,不可能是快速排序第2趟排序结果的是 \\
A.2,3,5,4,6,7,9 $\quad$ B.2,7,5,6,4,3,9 $\quad$ C.3,2,5,4,7,6,9 $\quad$ D.4,2,3,5,7,6,9

12.程序P在机器M上的执行时间是20秒,编译优化后,P执行的指令数减少到原来的70\%,而CPI增加到原来的1.2倍,则P在M上的执行时间是 \\
A.8.4秒 $\quad$ B.11.7秒 $\quad$ C.14秒 $\quad$ D.16.8秒

13.若x=103,y=-25,则下列表达式采用8位定点补码运算实现时,会发生溢出的是 .A.x+y B.-x+y C.x-y D.-x-y 14.float型数据据常用IEEE754单精度浮点格式表示.假设两个float型变量x和y分别存放在32位寄存器f1和f2中,若(f1)=CC90 0000H,(f2)=B0C0 0000H,则x和y之间的关系为 .A.x<y且符号相同B.x<y且符号不同C.x>y且符号相同D.x>y且符号不同15.某容量为256MB的存储器由若干4M×8位的DRAM芯片构成,该DRAM芯片的地址引脚和数据引脚总数是 .A.19 B.22 C.30 D.36 16.采用指令Cache与数据Cache分离的主要目的是 .A.降低Cache的缺失损失B.提高Cache的命中率C.降低CPU平均访存时间D.减少指令流水线资源冲突17.某计算机有16个通用寄存器,采用32位定长指令字,操作码字段(含寻址方式位)为8位,Store指令的源操作数和目的操作数分别采用寄存器直接寻址和基址寻址方式.若基址寄存器可使用任一通用寄存器,且偏移量用补码表示,则Store指令中偏移量的取值范围是 .A.-32768 ~ +32767 B.-32767 ~ +32768 C.-65536 ~ +65535 D.-65535 ~ +65536