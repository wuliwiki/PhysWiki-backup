% 詹姆斯·弗兰克(综述)
% license CCBYSA3
% type Wiki

本文根据 CC-BY-SA 协议转载翻译自维基百科 \href{https://en.wikipedia.org/wiki/James_Franck}{相关文章}。

詹姆斯·弗兰克(James Franck,[德语发音:[ˈdʒɛɪ̯ms ˈfʁaŋk] ⓘ;1882年8月26日-1964年5月21日)是一位德裔美国物理学家,因“发现了电子撞击原子时所遵循的规律”而与古斯塔夫·赫兹共同获得1925年诺贝尔物理学奖。\(^\text{[2]}\)他于1906年在柏林腓特烈·威廉大学(即柏林大学)获得博士学位,1911年完成教授资格论文,并在该校讲授课程直至1918年,其间升任特别教授。第一次世界大战期间,他以志愿者身份加入德军服役,1917年在一次毒气攻击中重伤,获授一等铁十字勋章。

弗兰克后来成为普鲁士科学院物理化学研究所(即凯撒·威廉物理化学研究所)物理部主任。1920年,弗兰克被任命为哥廷根大学实验物理学正式教授兼第二实验物理研究所所长。在哥廷根期间,他与理论物理研究所所长马克斯·玻恩合作开展量子物理研究,他的工作包括著名的弗兰克–赫兹实验,这是对玻尔原子模型的重要验证。他还积极推动女性在物理领域的发展,著名的包括莉泽·迈特纳、赫尔塔·斯波纳和希尔德·莱维。

1933年纳粹党在德国上台后,弗兰克为抗议对同行学者的解职,辞去了自己的职务。他协助弗雷德里克·林德曼帮助被解职的犹太科学家在海外寻找工作机会,随后于1933年11月离开德国。在丹麦尼尔斯·玻尔研究所工作一年后,他移居美国,先在巴尔的摩的约翰斯·霍普金斯大学工作,后转至芝加哥大学。在此期间,他对光合作用产生了兴趣。

二战期间,弗兰克参与了曼哈顿计划,担任冶金实验室化学部主任。他还担任原子弹政治与社会问题委员会主席,最著名的成果是主持编写《弗兰克报告》,建议在对日本城市使用原子弹前应进行警告,不应直接使用原子弹。
\subsection{早年生活}
詹姆斯·弗兰克于1882年8月26日出生在德国汉堡的一个犹太家庭,是银行家雅各布·弗兰克和妻子丽贝卡(娘家姓纳胡姆·德鲁克尔 [Nachum Drucker])的第二个孩子和第一个儿子。\(^\text{[3]}\)他有一个姐姐宝拉(Paula)和一个弟弟罗伯特·伯纳德。\(^\text{[4]}\)他的父亲是一位虔诚的宗教人士,而母亲则来自拉比世家。\(^\text{[3]}\)弗兰克在汉堡完成了小学学业,并于1891年开始就读于威廉文理中学,当时这是一所男校。\(^\text{[4]}\)

当时汉堡尚无大学,打算继续升学的学生必须前往德国其他地区的22所大学之一就读。弗兰克原打算学习法律和经济学,于1901年进入拥有著名法学院的海德堡大学。\(^\text{[5]}\)虽然他参加了法律课程,但他对科学课程更感兴趣。在那里,他遇见了马克斯·玻恩(Max Born),并与他建立了终生友谊。在玻恩的帮助下,他成功说服父母允许他转而学习物理和化学。\(^\text{[6]}\)弗兰克在海德堡期间听过利奥·科尼希斯伯格和格奥尔格·康托尔的数学课程,但由于海德堡在自然科学方面实力较弱,他决定转学到柏林的腓特烈·威廉大学继续深造。\(^\text{[5]}\)

在柏林期间,弗兰克聆听了马克斯·普朗克(Max Planck)和埃米尔·瓦尔堡(Emil Warburg)的课程。[7] 1904年7月28日,他在施普雷河(Spree River)救起了一对溺水的儿童。[7] 在瓦尔堡的指导下,他攻读哲学博士(Dr. Phil.)学位,[8] 瓦尔堡建议他研究电晕放电,但弗兰克认为该课题过于复杂,于是更换了论文研究方向。[9] 他将论文命名为《尖端放电中电荷载流子迁移率研究》(*Über die Beweglichkeit der Ladungsträger der Spitzenentladung*),[10] 并随后发表在《物理年鉴》(*Annalen der Physik*)上。[11]

完成论文后,弗兰克需要履行被推迟的兵役。他于1906年10月1日入伍,加入第一电报营(1st Telegraph Battalion)。同年12月,他在骑马时发生小事故,被判定不适合服役而退伍。1907年,他在法兰克福物理学会(Physikalische Verein)担任助理,但并不喜欢这份工作,于是不久便返回了柏林腓特烈·威廉大学。[12] 在一次音乐会上,弗兰克结识了瑞典钢琴家英格丽德·约瑟夫森(Ingrid Josephson)。他们于1907年12月23日在瑞典哥德堡举行婚礼。两人育有两个女儿,分别是1909年出生的达格玛(Dagmar,昵称 Daggie),以及1912年出生的伊丽莎白(Elisabeth,昵称 Lisa)。[13]
