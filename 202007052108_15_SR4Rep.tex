% 时空的四维表示
% 四位置|四位移|四速度|四加速度|世界线

\pentry{斜坐标表示洛伦兹变换\upref{SROb},坐标变换与过渡矩阵}
%需要补充“过渡矩阵”词条
%基本完成 
\subsection{概念}
\subsubsection{四位置和四位移}
把时间坐标和空间坐标都看成时空的坐标,那么一个事件在时空中的位置就被称为其\textbf{四位置(4-position)}.四位置本身是一个向量,但其坐标表示取决于所选取的惯性参考系.同一个四位置在不同惯性系中的坐标,可以利用洛伦兹矩阵作为过渡矩阵来相互转化.

两个事件之间的四位置之差,称为这两个事件之间的\textbf{四位移(4-displacement)}.

\subsubsection{四速度}

假设某质点在三维空间中运动.经典物理中认为,质点轨迹上某一点的速度是一个向量,其方向与该点处轨迹相切.但速度的大小具体是多少,三维轨迹完全没有提供足够的信息.同样的轨迹完全可以是用不同的瞬时速度来走过的.

为了完全从几何角度描述该质点的速度,我们可以把视角提高到四维时空,将三维的轨迹拉升到四维,这样就有充足的信息来描述质点的速度大小了,而三维轨迹就是四维轨迹在三维空间中的投影.取四维轨迹上某一点的切向量$\bvec{v}$,使其在时间轴上的投影为$1$(单位时间长度),那么$\bvec{v}$在三维空间中的投影就是速度.

这个例子启发我们研究四维速度比研究三维速度更加全面,由此有了以下概念:

\begin{definition}{四速度}

一个质点的\textbf{四速度(4-velocity)},定义为质点的四维运动轨迹的切向量$\bvec{v}$,满足$\bvec{v}$在瞬时自身系中的时间轴投影长度是$1$. 

\end{definition}

四速度的概念可以应用在经典力学中,也可以应用于相对论.

相对论框架下,在某惯性参考系里的观察者看来,一个质点的四速度的时间分量,就是这个质点系所处的参考系的“时间流逝速度”,即质点的手表和观察者的手表的转动速度之比\footnote{由于我们采用$c=1$的约定,速度是一个无量纲量.}.如果质点相对观察者静止,那么质点的四速度就是$(1,0,0,0)$,即非零分量只有时间分量,且时间流逝速度为$1$.

如果在经典力学框架下讨论四速度,那么其时间分量就永远是$1$,这其实意味着经典情况下不存在钟慢效应.

\subsubsection{四加速度}

\begin{definition}{四加速度}

一个质点的\textbf{四加速度(4-acceleration)},定义为质点的四速度对固有时间(瞬时自身系中的时间)求导的结果.

\end{definition}

\subsection{四位移、四速度与四加速度的性质}

\subsubsection{四位移的不变性}

\begin{exercise}{}

设有惯性参考系$K_1$,另一惯性系$K_2$以速度$\bvec{v}=\pmat{v, 0, 0}^T$相对$K_1$运动.请通过计算证明:对于两个事件$A$和$B$而言,它们在两个参考系中的四位移都是一样的.

\end{exercise}

\subsubsection{四速度的表达}

设有惯性参考系$K_1$,一个质点在$K_1$中某点的速度是$\bvec{u}=\pmat{u, 0, 0}^T$,那么它在$K_1$的四速度$\bvec{U}$的\textbf{方向}和$\pmat{1, u, 0, 0}^T$一致\footnote{也就是和瞬时自身系$K_2$的时间轴重合.},设$\bvec{U}=k\cdot\pmat{1, u, 0, 0}^T$.则由于$\bvec{U}$对应瞬时自身系$K_2$中单位时间的长度,再考虑到$K_2$的坐标轴在$K_1$中的表示具有\textbf{拉伸比例}$\sqrt{(1+u^2)}/\sqrt{(1-u^2)}$,知$\bvec{U}$在$K_1$中的长度为$\sqrt{(1+u^2)}/\sqrt{(1-u^2)}$,其在时间轴上的投影就是$1/\sqrt{(1-u^2)}=\gamma$.

因此,速度为$\bvec{u}=\pmat{u, 0, 0}^T$的质点,其四速度是$\bvec{U}=\gamma\pmat{1, u, 0, 0}^T=\pmat{\gamma, \gamma u, 0, 0}^T=\pmat{1/\sqrt{1-u^2}, u/\sqrt{1-u^2}, 0, 0}^T$.

\subsubsection{四加速度}

题设同“四加速度的表达”,设质点在某点的四加速度是$\bvec{A}=\dd{\bvec{U}}/\dd\tau$,其中$\tau$是质点的瞬时自身系$K_2$中的固有时.记$\bvec{U}=(t, \bvec{u})$,$\frac{\dd{\bvec{u}}}{\dd{t}}=\bvec{a}$.由于在$K_1$中$\dd{t}/\dd{\tau}=1/\sqrt{1-u^2}=\gamma$,再考虑到分部积分和$\dd{\gamma}/\dd{t}=\bvec{a}\cdot\bvec{u}/\gamma^3$,可以得:

\begin{equation}
\begin{aligned}
%
\bvec{A}&=\frac{\dd{\bvec{U}}}{\dd{\tau}}\\&=\frac{\dd{\bvec{U}}}{\dd{t}}\cdot\frac{\dd{t}}{\dd{\tau}}\\&=(\gamma^4\bvec{a}\cdot\bvec{u}, \gamma^2\bvec{a}+\gamma^4(\bvec{a}\cdot\bvec{u})\bvec{u})\\&=(\gamma^4\bvec{a}\cdot\bvec{u}, \gamma^4(\bvec{a}+(\bvec{a}\times\bvec{u})\times\bvec{u})
%
\end{aligned}
\end{equation}

\subsection{世界线}

我们知道,一个事件被看作是四维时空中的一个点,它有给定的时间和空间坐标.如果我们考虑一个粒子的运动,其在时空中的轨迹就是一条曲线.这条曲线就被称为粒子的\textbf{世界线(world line)}.


