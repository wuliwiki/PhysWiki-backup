% 物理教材推荐

\begin{issues}
\issueDraft
\end{issues}

\subsection{科普}
\begin{figure}[ht]
\centering
\includegraphics[width=5cm]{./figures/11821116458c7691.png}
\caption{请添加图片描述} \label{fig_PhText_4}
\end{figure}
这是我(高中)看过的第一本英文物理书,简单介绍了力学,电磁学,光学,以及相对论。该书所需要的数学大部分在高中以下,但讲物理的思路却较为新颖,例如从对称和守恒量(诺特定理)开始讲力学,要知道要认真分析守恒量和对称的关系,需要设计到理论力学。我当时看这本书主要是用来学专业英语:我把书中碰到的每个生词记了个滚瓜烂熟,对一些常用物理词汇有了一定的了解。事实上,这本书最大最大的优点还在于可以免费从官网 lightandmatter.com 上 下载 pdf!(如果速度比较慢也可以 点此下载 )

\begin{figure}[ht]
\centering
\includegraphics[width=5cm]{./figures/44ecff2dd0194051.png}
\caption{请添加图片描述} \label{fig_PhText_9}
\end{figure}
这与第一本是同一个系列的,同样用简单的数学简单介绍了狭义相对论,量子力学以及原子等,同样提供免费下载,我也同样把里面的所有生词背了下来。现在书的内容已经整合到 Light and Matter 这本将近一千页的书中,网站上找不到原书了。 Light and Matter 的官网下载点 这里 (或者从本站下载点 这里 )。这里还提供原书的下载(点 这里 )。

\subsection{热力学}
\begin{figure}[ht]
\centering
\includegraphics[width=5cm]{./figures/2a22243618f7a163.png}
\caption{请添加图片描述} \label{fig_PhText_1}
\end{figure}
虽然我大四在 KSU 上的热力学与统计力学课是 Kittle 的教材,但个人觉得这本书写得更好,更适合自学。这本书名字叫做热力学,但还是有相当大的篇幅在讲统计力学。注意热力学是对系统宏观规律(温度,压强,体积)的描述,而统计力学作为四大力学之一,试图从微观上解释热力学规律的本质原因。

\subsection{数学}

\begin{figure}[ht]
\centering
\includegraphics[width=5cm]{./figures/38808837c3cea6d4.png}
\caption{请添加图片描述} \label{fig_PhText_2}
\end{figure}
在推荐电动力学之前,我想应该先推荐一本矢量微积分的小册子。这本书我同样在高中就精读完并把生词牢记,而且居然还在最后发现了当时梦寐以求的开普勒三定律证明。

\begin{figure}[ht]
\centering
\includegraphics[width=5cm]{./figures/7b0c7f407cda6ae6.png}
\caption{请添加图片描述} \label{fig_PhText_3}
\end{figure}
美国经典研究生力学教材(精读过前半本), 介绍了拉格朗日和哈密顿力学,


\begin{figure}[ht]
\centering
\includegraphics[width=5cm]{./figures/7334405da2e7491d.png}
\caption{请添加图片描述} \label{fig_PhText_5}
\end{figure}
电动力学(Electrodynamics)是四大力学之一,也是物理专业的本科必修课。而 David Griffiths 的电动力学,作为他的三部曲(还有量子力学和粒子物理)之一,已经成为了美国物理本科的电动力学标准教材,甚至还有中文译本。教材前面详细介绍了书中用到的矢量微积分(这也是美国教材的亮点之一,尽量降低阅读的门槛),所以有一定高中数学基础读起来应该不会太难。 后面还介绍了狭义相对论中的电动力学。我高中读完了这本书的前三章,本科在 KSU 交换的时候电动力学恰好使用这本教材,就把剩下的也仔细读完了。

\subsection{量子力学}
\begin{figure}[ht]
\centering
\includegraphics[width=5cm]{./figures/ac22e6d4a0451afa.png}
\caption{请添加图片描述} \label{fig_PhText_6}
\end{figure}
这是 Griffiths 老先生的又一本经典之作,这本书并不像 Shankar 的书那样先从经典量子力学的理论框架开始讲(因为那需要较强的线性代数功底),而是用波函数和平均值引入动量算符等。说实话,我个人还是喜欢先讲理论框架。

\begin{figure}[ht]
\centering
\includegraphics[width=5cm]{./figures/585f07dc95c05f72.png}
\caption{请添加图片描述} \label{fig_PhText_7}
\end{figure}
线性代数基本上可以看做是量子力学的数学基础,这本书虽然有些偏向数学专业(注重理论结构和严谨的证明),但在以后学习经典量子力学的理论构架时,我庆幸大二的时候把这本书从头到尾啃了下来

\begin{figure}[ht]
\centering
\includegraphics[width=5cm]{./figures/f7741eaeb381bfdd.png}
\caption{请添加图片描述} \label{fig_PhText_8}
\end{figure}
较经典的研究生量子力学教材(我部分精读过), 也适合高年级的本科生阅读。 建议读完 Griffiths 的量子力学以后再看。 比起后者来, 这本书在原理上讲得更系统, 自然对线性代数的要求也较高(书的第一章简介了所需数学)。

