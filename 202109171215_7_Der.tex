% 导数
% 微积分|导数|导函数|切线|极限

% 未完成: 举几个例子,说明常数导数为零, 直线导数为定值等!
\pentry{切线与割线\upref{TanL}, 函数\upref{functi},极限\upref{Lim}}

\subsection{导数的直观几何理解}

一个一元函数 $y = f(x)$, 在直角坐标系中表示为一条曲线.在这个曲线的光滑部分取一点 $A$, 并作其切线\upref{TanL}.

\begin{figure}[ht]
\centering
\includegraphics[width=5cm]{./figures/Der_1.pdf}
\caption{点 $A$ 的切线}
\end{figure}


若切线存在,该切线与 $x$ 轴的夹角 的正切值 $\theta$ 就叫点 $A$ 的导数.当函数在 $A$ 点递增时,可能的取值为 $\theta \in (0,\pi/2)$, 即 $\tan \theta  \in (0, + \infty)$. 递减时,取 $\theta  \in (-\pi/2,0)$, 即 $\tan \theta \in (-\infty ,0)$. 当切线水平时,$\theta  = \tan \theta  = 0$. 

若函数曲线在 $x$ 的某一开区间内的每一点都可导, 则这个区间上每一个 $x$ 对应一个导数.将其写成关于 $x$ 的函数 $g(x)$,  $g(x)$  就是该区间上的 \textbf{导函数}. 通常将导函数记为以下的一种(后3种记号的来源见下文)

\begin{equation}\label{Der_eq1}
f'(x),\quad [f(x)]',\quad \dv{y}{x},\quad \dv{f}{x},\quad \dv{x}f(x)
\end{equation}
在物理中, 常常在物理量上方加一点表示对时间求导(注意仅限于对时间求导), 例如 $\dot f(t) = \dv*{f(t)}{t}$.

若切线不存在(例如折线的棱角处,但也有其他更复杂的情况), 我们说点 $A$ 不可导. 如果某区间内是 “光滑” 的, 那么的该区间内处处可导.

\begin{figure}[ht]
\centering
\includegraphics[width=5cm]{./figures/Der_3.pdf}
\caption{棱角处不可导}
\end{figure}

若函数曲线在某一点附近是光滑的,那么在这点附近取一小段,当这一段取得足够小,可以近似认为它是线段且与切线重合(如下图). 以这条线段为斜边,作一直角三角形,令其底边长为 $\dd{x}$ (在微积分中,通常把非常小的一段 $\Delta x$ 记为 $\dd{x}$,  $\dd{x}$ 是一不能分割的整体符号,而不是两个量相乘),竖直边的边长为 $\dd{y}$ (当函数递增时, $\dd{y}$ 取正值,反之取负值).根据上面导数的定义,$\dv*{y}{x} = \tan \theta $ 就是函数的导数.所以导数通常表示为 $\dv*{y}{x}$, 导数的倒数则为 $\dv*{x}{y}$. 

\begin{figure}[ht]
\centering
\includegraphics[width=14cm]{./figures/Der_2.pdf}
\caption{将切点放大,会发现切线和曲线在切点附近“重合”}
\end{figure}

由上面的讨论可得,当 $x$ 增加一小段 $\Delta x$ 时,$y$ 轴的增量约为 $\Delta y \approx f'(x)\Delta x$,且当 $\Delta x$ 越小,这条式子就越精确成立, 记为 $\dd{y} = f'(x) \dd{x}$.这个关系就叫函数的微分.

\subsection{导数的严谨理解}
% 导数的代数理解就是: 一个量关于另一个量的变化率. 例如质点直线运动时,速度的大小就是其路程对时间的导数.把这种描述用极限\upref{Lim}表达出来就是
% \begin{equation}\label{Der_eq2}
% f'(x) = \lim_{\Delta x \to 0} \frac{f(x + \Delta x) - f(x)}{\Delta x}
% \end{equation}
% 在图3的右图中,$\Delta x$ 的始末位置并不非常重要,既可以从 $x$ 取到 $x + \Delta x$, 也可以从 $x - \Delta x$  取到 $x$ 等等( 因为当 $\Delta x$ 非常小的时候,$x$ 附近的曲线基本处处跟切线重合,它们的斜率都是一样的). 所以导数的定义也有其他类似的形式

% \begin{equation}
% f'(x) = \lim_{\Delta x \to 0} \frac{f(x) - f(x - \Delta x)}{\Delta x} = \lim_{\Delta x \to 0} \frac{f(x + \Delta x) - f(x - \Delta x)}{2\Delta x}
% \end{equation}
% 虽然上面用到了诸如“近似”等词,但根据定义,极限都是精确的.

% 例子: 速度 加速度(一维)\upref{VnA1}.

切线的几何描述并不严谨和完整,初学者可能会有很多疑问,比如割线的两个点靠近切点的时候,如果其中一个点到切点的距离始终是另一个点到切点距离的两倍,那它们还算同时达到切点吗?

实际上,在\textbf{极限}\upref{Lim}中我们强调过,极限的概念并不是简单地等同,而是趋近的性质.用割线去接近切线也是一种极限,不存在“达到切点”的情况,只能是越来越接近切点.

和其它类型的极限一样,割线趋近的过程强调“任意性”.为了方便描述,我们需要把一根切线表示为数字.最直接的办法就是用切线的斜率来表示.

\begin{definition}{斜率}
对于用$f(x)=ax+b$描述的直线,定义其斜率为$a$.
\end{definition}

由于我们给定了切点,因此割线接近切线的过程中,肯定也会接近切点,我们只需要考察割线的斜率是不是收敛就行.由于斜率是一个数字,这就使得我们又获得了一个数列,而我们对于数列的极限是很熟悉的了.

\begin{definition}{导数}
考虑实函数$f(x)$,给定实数$x_0$.对于\textbf{一对}数列$\{a_n\}$、$\{b_n\}$,令$\lim\limits_{n\to\infty}a_n=x_0$、$\lim\limits_{n\to\infty}b_n=x_0$,且各$a_n\not=b_n$\footnote{这一条是保证总能画出割线,因为两点确定一条直线嘛,$a_n=b_n$的话这条割线就画不出来了.}.这样,每一对点$(a_n, f(a_n))$、$ (b_n, f(b_n))$都能唯一确定一条割线,也就确定了割线的斜率$d_n$.

如果对于\textbf{任意}的上述数列对$\{a_n\}$、$\{b_n\}$,其生成的$d_n$都收敛到同一个实数$A$上,那么该实数$A$就是$f(x)$在$x_0$处的\textbf{切线斜率},也称\textbf{导数(derivative)}.
\end{definition}

我们来看几个例子,加深理解.

\begin{example}{}
考虑函数$f(x)=2x+1$.容易验证,无论怎么选一对数列,它们生成的割线斜率都是$2$.因此由定义,$f(x)$在任何一个点处的导数值都是$2$.
\end{example}

\begin{example}{}\label{Der_ex1}
考虑函数$f(x)=x^2$,尝试计算其在$x=2$处的导数值\footnote{$x=2$处即点$(2, 4)$.也可以说$y=4$且$x>0$处,但肯定是$x=2$处的说法更方便.}.

取数列$a_n=2+g_n$和$b_n=2+h_n$,其中$g_n$和$h_n$都趋近于$0$,且各$g_n\not=h_n$.那么$a_n$和$b_n$所确定的割线就通过点$(2+g_n, 4+4g_n+g_n^2)$和点$(2+h_n, 4+4h_n+h_n^2)$,因此割线斜率是
\begin{equation}
\begin{aligned}
d_n&=\frac{4+4g_n+g_n^2-4-4h_n-h_n^2}{g_n-h_n}\\
&=\frac{4(g_n-h_n)+(g_n+h_n)(g_n-h_n)}{g_n-h_n}\\
&=4+g_n+h_n
\end{aligned}
\end{equation}
由于随着$n$的增大,$g_n$和$h_n$都趋于零,因此$d_n$趋于$4$.

又由于我们没有具体约束$g_n$和$h_n$,使得上述讨论适用于一切可用于构造趋近于给定点割线的数列$a_n$和$b_n$,因此满足\textbf{任意}性,因此$f(x)=x^2$在$x=2$处的导数值就是$4$.
\end{example}

\begin{exercise}{}
考虑函数$f(x)=\abs{x}$,画出这个函数的图像,并证明其在$x=0$处没有导数.
\end{exercise}

\begin{exercise}{}
考虑函数$f(x)=[x]$,定义为$[x]$是小于等于$x$的、最接近$x$的整数.比如,$[\pi]=3$,$[e]=2$,$[4.99]=4$, $[-4.99]=-5$.

画出这个函数的图像,并证明其在横坐标为整数的点处没有导数.
\end{exercise}








\subsection{导函数及其计算}

我们上面讨论的是对于函数$f(x)$,在给定点求其导数.但是如果我们把所有点的导数都求出来(“不存在导数”也是一种结果),那么我们就得到了一个新的函数,就叫做$f(x)$的\textbf{导函数},记为$f'(x)$.

如果我们能把导函数的表达式计算出来,那么就可以直接代值去计算各处的导数,没必要挨个像\autoref{Der_ex1} 那样进行一番冗长的运算了.为了实现这个目的,我们要先研究一番几个基本的函数.

\begin{example}{多项式函数的导数}
多项式是形如$f(x)=a_0+a_1x+a_2x^2+\cdots+a_nx^n$的函数.为了计算其导函数,我们首先要考虑最简单的多项式,$x^n$.

对于任意实数$x_0$,取数列$a_n=x_0+g_n$和$b_n=x_0+h_n$,其中$g_n$和$h_n$都趋近于$0$,且各$g_n\not=h_n$.那么$a_n$和$b_n$分别确定点$(x_0+g_n, (x_0+g_n)^n)$和$(x_0+h_n, ((x_0+h_n)^n)$,割线斜率也就是$d_n=\frac{(x_0+g_n)^n-(x_0+h_n)^n}{g_n-h_n}$.

应用二项式定理$(a+b)^n=a^n+b^n+$
\end{example}







