% 正则表达式
% 正则表达式|VScode|C++|搜索|匹配|文本

\begin{issues}
\issueDraft
\issueOther{如何表示一行的开始?}
\end{issues}

\pentry{文本文件与字符编码\upref{encode}}

\footnote{参考 Matlab 文档\href{https://www.mathworks.com/help/matlab/ref/regexp.html}{相关页面}。}在文本文件中搜索内容的时候, 有时候想要的是某种格式而不是某些具体的字符, 例如要搜索 “*月*日 消费 ** 元”, 找到匹配项后需要选中这几个字(以便进行替换等操作), 又或者仅选中 “元” 前面的数值(以便进行统计等)。 理论上我们可以通过编程解决这个问题, 但更简单地, 可以用一种广为使用的表达式来达到同样的效果, 就是下面要介绍的\textbf{正则表达式(regular expresion)}。

正则表达式在许多软件中都被支持, 例如在常用的文本编辑器(如 VScode), 搜索软件(如 Fileseek), 和大部分编程语言(如 C++, python, Matlab)中都有很好的支持。

\subsubsection{字符匹配}
\begin{table}[ht]
\centering
\caption{字符匹配}\label{regex_tab1}
\begin{tabular}{|c|c|c|}
\hline
符号 & 说明 & 例子 \\
\hline
\verb|.| & 匹配单个任意字符, 包括空格回车等 & \verb|.at| 可以匹配 \verb|bat|, \verb|cat|, \verb|hat| 等 \\
\hline
\verb|[...]| & 匹配方括号中的任意一个字符。 如果要表示一个范围的字符可以用 \verb|-| 连接 & \verb|[bc]ase| 可以匹配 \verb|base| 和 \verb|case|; 又例如 \verb|[a-z0-9]| 可以匹配任意一个小写字母或数字 \\
\hline
\verb|[^...]| & 匹配任何除方括号中以外的字符 & 例如 \verb|[^b]ase| 不能匹配 \verb|base| 但可以匹配 \verb|case| \\
\hline
\verb|\w| & 等效于 \verb|[a-zA-Z_0-9]| &  \\
\hline
\verb|\W| & 等效于 \verb|[^a-zA-Z_0-9]| &  \\
\hline
\verb|\s| & 空白字符,等效于 \verb|[ \f\n\r\t\v]| (注意第一个字符是空格, 剩下的符号是不同功能的空格, 见下文) &  \\
\hline
\verb|\S| & 非空白字符,等效于 \verb|[^ \f\n\r\t\v]| & \\
\hline
\verb|\d| & 一个数字(digit), 等效于 \verb|[0-9]| & \\
\hline
\verb|\D| & 一个非数字, 等效于 \verb|[^0-9]| & \\
\hline
\end{tabular}
\end{table}

\begin{itemize}
\item \verb`\a` Alarm (beep)
\item \verb`\b` Backspace
\item \verb`\f` Form feed
\item \verb`\n` New line
\item \verb`\r` Carriage return
\item \verb`\t` Horizontal tab
\item \verb`\v` Vertical tab
\item \verb`\特殊字符` 例如 \verb`\?` 表示问号, \verb`\*` 表示星号, \verb`\\` 表示反斜杠
\item \verb`\oN` 或 \verb`\o{N}`, 用 8 进制指定字符
\item \verb`\xN` 或 \verb`\x{N}`, 用 16 进制指定字符
\end{itemize}

\subsubsection{重复匹配}
在\autoref{regex_tab1} 的命令后面可以加上如下 \textbf{quantifier}
\begin{itemize}
\item \verb|?| 表示 0 或 1 次重复
\item \verb|+| 表示 1 或者若干次重复
\item \verb|*| 表示 0 或者若干次重复
\item \verb|{n}| 表示 n 次重复
\item \verb|{m,n}| 表示 m 到 n 次重复
\item \verb|{m,}| 表示 m 次或以上重复
\item 以上 quantifier 后面加 \verb|?| 可以匹配尽量短的内容, 例如 \verb|abc.+?def|
\end{itemize}

\subsubsection{Group}
\begin{itemize}
\item \verb|(expr)| 可以把表达式 group 到一起并 match, 例如 \verb|abc(\d+)def| 寻找 \verb|abc\d+def| 并且 match \verb|\d+| 部分
\item 在 VScode 中, 用小括号 group 起来的部分, 在替换的时候可以用 \verb|$1|, \verb|$2| 等来表示。
\item \verb|(?:expr)| 可以把表达式 group 到一起, 例如 \verb|(?:abc){2}| 搜索 "abcabc"
\item \verb`(expr1|expr2)` 匹配其中一个
\end{itemize}


\subsection{例子}
\begin{itemize}
\item 在 VS code 中, 要把 \verb|[链接名](网址)| 替换为 \verb|\href{网址}{链接名}|, 搜索正则表达式 \verb|\[(.*?)\]\((.*?)\)|, 替换为为 \verb|\href{$2}{$1}|。
\end{itemize}

