% 二阶齐次变系数线性微分方程的幂级数解法
% keys 幂级数|ODE|常微分方程|differential equation|二阶微分方程

\pentry{幂级数\upref{powerS},常系数线性齐次微分方程\upref{ODEb2}}

\addTODO{是不是还要加上差分方程作为预备知识?}

\subsection{从例子出发}

从微积分学中我们知道,许多函数是可以表示为幂级数的形式:$f(x)=a_0+a_1x+a_2x^2+\cdots$.幂级数良好的性质可以用于解二阶微分方程.

我们先看一个简单的实例.遵循微积分学的习惯,我们这里以$x$为自变量了.

\begin{example}{}\label{ODE2P_ex1}
考虑方程
\begin{equation}\label{ODE2P_eq4}
\frac{\mathrm{d}^2 y}{\dd x^2}-2x\frac{\dd y}{\dd x}-4y=0
\end{equation}
在初始条件
\begin{equation}\label{ODE2P_eq1}
\leftgroup{
    y(0)&=0\\
    y(1)&=1
}
\end{equation}
下的\textbf{特解}.

尝试设
\begin{equation}\label{ODE2P_eq3}
y(x)=a_0+a_1x+a_2x^2+\cdots=\sum\limits_{i=0}^\infty a_ix^i
\end{equation}

首先代入初值条件\autoref{ODE2P_eq1} ,得到$a_0=0, a_1=1$.

接着,考虑到
\begin{equation}\label{ODE2P_eq2}
\leftgroup{
    &\frac{\mathrm{d}^2 y}{\dd x^2}=2a_2+6a_3x+12a_4x^2+\cdots=\sum\limits_{k=0}^\infty (k+1)(k+2)a_{k+2}x^k\\
    &2x\frac{\dd y}{\dd x}=2a_1x+4a_2x^2+6a_3x^3+\cdots=\cdots=\sum\limits_{k=1}^\infty 2ka_kx^k
}
\end{equation}

将\autoref{ODE2P_eq2} 和\autoref{ODE2P_eq3} 代回\autoref{ODE2P_eq4} ,比较各$x^k$的系数,得到
\begin{equation}
(k+1)(k+2)a_{k+2}=(2k+4)a_k
\end{equation}
整理得
\begin{equation}
a_{k+2}=\frac{2}{k+1}a_k
\end{equation}

这是一个二阶\textbf{差分方程}.

由于$a_0=0$,故$a_{2k}=0$对所有$k$成立.我们只需要考虑奇数项即可.

令$b_k=a_{2k-1}$\footnote{反过来就是$a_k=b_{\frac{k+1}{2}}$.},则我们有$b_1=a_1=1$和$b_{\frac{k+3}{2}}=\frac{2}{k+1}b_{\frac{k+1}{2}}$;换个写法,就是$b_{k+1}=\frac{1}{k}b_k$.

因此,
\begin{equation}
b_k=\frac{1}{(k-1)!}
\end{equation}

进而
\begin{equation}
\begin{aligned}
y&=b_1x+b_2x^3+b_3x^5+\cdots\\
 &=x\sum\limits_{k=1}^\infty b_kx^{2k-2}\\
 &=x\sum\limits_{k=1}^\infty \frac{x^{(2k-2)}}{(k-1)!}\\
 &=x\sum\limits_{k=0}^\infty \frac{x^{2k}}{k!}\\
 &=x\E^{x^2} 
\end{aligned}
\end{equation}

\end{example}

\autoref{ODE2P_ex1} 中“假设解为$x$的幂级数,通过比较系数来求出特解”的方法,被称为\textbf{幂级数解法}.


\subsection{幂级数解法}

\autoref{ODE2P_ex1} 和我们之前所解的方程不一样,\autoref{ODE2P_eq4} 中的系数$2x$不再是一个常数,而是$x$的函数,这使得我们应对\textbf{常系数}方程的方法无效了.对于二阶变系数方程,幂级数解法是很有用的.

哪些方程能应用幂级数解法呢?幂级数解的收敛区间又是否能覆盖所要求解的区间呢?这些问题有完善的解答,但由于较为深入,我们在此只给出重要的结论.

我们所考虑的方程是形如
\begin{equation}\label{ODE2P_eq5}
\qty(\frac{\mathrm{d}^2}{\dd x^2}+p(x)\frac{\dd}{\dd x}+q(x))y(x)=0
\end{equation}
在$x=0$处的\textbf{特解}.

$x=x_0$处的特解,可以通过变量代换$t=x-x_0$,来化为$y(t)$的方程在$t=0$处的特解问题.

\begin{theorem}{}
如果\autoref{ODE2P_eq5} 中的$p(x)$和$q(x)$都可以写为$x$的幂级数形式,
\end{theorem}










