% 微积分基本定理
% keys 微积分基本定理|牛顿莱布尼茨公式
% license CCBY3
% type Tutor

\begin{issues}
\issueTODO
\issueMissDepend
\issueNeedCite
\end{issues}

\pentry{导数\nref{nod_Sample}}{nod_098a}

\textbf{微积分基本定理}是微积分领域的一个非常重要的定理,从它的名字你就可以看出来,他叫“基本定理”。而如此重要的原因是它将原本分开进行研究的微分和积分联系了起来,使得人们有了一个新的视角来统一审视这两种运算。而对于二者早已融合的当下,接触微积分的学习者往往在接触了这个定理的时候,早已接受了“微分与积分互为逆运算”的观念,这令他们感到困惑:“这不是显然的吗?”因此,现在请暂时忘记关于微分和积分的联系,回到尚未发现这个基本定理的时候,通过穿越建立这个基本定理的过程,来感受它的奇妙。

\subsection{前提}

现在我们已经知道的内容包括极限运算中的夹逼定理,微分运算中导数的定义,以及积分运算中关于上下限的运算和积分中值定理。w
\subsubsection{导数的定义}

\subsection{证明过程}

\subsection{定理内容}

\begin{theorem}{微积分第一基本定理}
若$f(x)$在[a,b]上连续,则$f(x)$在[a,b]上的变上限积分是$f(x)$的一个原函数,即:
$${d\over\mathrm {d}x}\int_{a}^{x}f(t)\mathrm {d}t=f(x)$$
\end{theorem}

\begin{theorem}{微积分第二基本定理}
若$f(x)$在[a,b]上连续,则对$f(x)$的任意一个原函数$F(x)$,$f(x)$在[a,b]上的定积分的值为$F(x)$在区间端点处的函数值的差,即:
$$\int_{a}^{b} f(x)\mathrm {d}x=F(b)-F(a)$$
\end{theorem}

注意这里有几点需要注意的内容:
\begin{enumerate}
\item 证明的过程中没有针对特定的积分定义(如黎曼积分、勒贝格积分等),因此这个定理其实是对这些定义下的积分都成立的。
\item 微积分第二基本定理一般也称作\textbf{牛顿-莱布尼茨公式}(Newton–Leibniz formula)。它也是斯托克斯定理在一维情况下的特例。
\end{enumerate}