% 南京理工大学 2004 年 研究生入学考试试题 普通物理(B)
% license Usr
% type Note

\textbf{声明}:“该内容来源于网络公开资料,不保证真实性,如有侵权请联系管理员”

\subsection{填空题(每空2分,总共32分)}

1. 已知质点作半径为 \( R \) 的匀加速率圆周运动,其角位置 \( \theta = \theta_0 + \omega_0 t + bt^2/2 \),其中 \( \theta_0 \), \( \omega_0 \), \( b \) 均为常数,则质点在 \( t \) 时刻的速率为 $\underline{\hspace{2cm}}$,\( t \) 时刻的法向加速度大小为 $\underline{\hspace{2cm}}$,\( t \) 时刻的切向加速度大小为$\underline{\hspace{2cm}}$。

2. 已知一质点作简谐振动,其振动方程为 \( y = 0.1 \cos (100\pi t + \pi/3) \) (米),则 \( t = 1 \) 秒时质点的振动速度大小为$\underline{\hspace{2cm}}$ 米/秒,振动加速度大小为 $\underline{\hspace{2cm}}$ 米/秒\(^2\)。

3. 已知一沿 $ +x $ 方向传播的简谐波的波动方程为$y = 0.05 \cos (100\pi t - 20x + \pi/3)$ (米)。则该简谐波的传播速度为$\underline{\hspace{2cm}}$,波长为$\underline{\hspace{2cm}}$ 在波线上同一时刻相位差为 $\pi/3$ 的两点间的距离为$\underline{\hspace{2cm}}$

4.一半径为$R$的金属球,带电$Q$,则球中心0点的电场强度为$\underline{\hspace{2cm}}$,其电势为球内任一点的电势为$\underline{\hspace{2cm}}$。

5.一半径为R的圆环状载流回路,由N匝线圈组成,设一匝线中通过电流1,则圆环中圆环铀线上距圆环中心 x心处的磁感应强度大小为,方向为处的磁感应强度大小为