% 哈尔滨工业大学 2004 年硕士物理考试试题
% keys 哈尔滨工业大学|考研|物理|2004年
% license Copy
% type Tutor


\textbf{声明}:“该内容来源于网络公开资料,不保证真实性,如有侵权请联系管理员”
\begin{enumerate}
\item 如图所示,透镜的中心厚度为$\bar{OO'}=20cm$,折射率为$n=1.5$,前后表面的曲率半径分别是$20cm$和$40cm$,透镜的后表面镀铝反射膜,在前表面左方距 $O$点 $20cm$ 远的$Q$点放置一个高度为 $1mm $的小物$y$,在傍轴条件下求透镜最后成像的位置、高度、大小、虚实和放缩情况?
\begin{figure}[ht]
\centering
\includegraphics[width=8cm]{./figures/f44e6b31ffc567ba.png}
\caption{} \label{fig_HGD04_2}
\end{figure}
\item 一束在$x-z$平面、沿与$z$轴成如图的$\theta$角方向传播的波长为$\lambda$的平面波和一束点波源$Q$位于$Z$轴上、且与坐标原点$o$相距$a$的发散球面波发生干涉,在傍轴条件下求$z=0$平面上干涉条纹的形状方程及间距公式,并用文字说明干涉条纹的位置和形状。
\begin{figure}[ht]
\centering
\includegraphics[width=8cm]{./figures/f9f3d28dd3f1326e.png}
\caption{} \label{fig_HGD04_1}
\end{figure}
\item 杨氏干涉装置中的$S$点光源发出波长为$\lambda=6000A$°的单色光,间距为$d=0.4mm$的双缝$S_1$和$S_2$对称分布于光轴两侧,衍射屏与观察屏的距离为$D=100cm$,一个焦距为$f=10cm$的薄透镜$L$置于衍射屏和观察屏之间,若薄透镜与衍射屏的距离分别为(1)$A=8cm$和(2)$A=10cm$,在傍轴条件下分别求观察屏$\Sigma$上这两种情况的干涉条纹的形状和间距?
\begin{figure}[ht]
\centering
\includegraphics[width=10cm]{./figures/d5243ec28980c5d0.png}
\caption{} \label{fig_HGD04_3}
\end{figure}
\item 如图所示,衍射屏上有四条平行透光狭缝,缝宽都是$a$,缝间不透明部分的宽度分别是$a,2a,a$,求单色平行光正入射到该衍射屏上时的夫琅和费衍射强度分布?
\begin{figure}[ht]
\centering
\includegraphics[width=12cm]{./figures/a09d127ab15f5461.png}
\caption{} \label{fig_HGD04_4}
\end{figure}
\item 如图所示,在两个透振方向互相垂直的偏振片$P_1$和$P_2$之间插入一块$\Delta=3\pi/4$的波晶片$K$,波晶片的光轴方向与偏振片$P_1$和$P_2$透振方向的夹角分别为如图所示的$30$°和$60$°。一束强度为$I_0$的单色平行自然光垂直入射到该装置上,忽略吸收和反射等光损耗,分别求透射光在I,II,III区里的偏振态(画出偏振态图)和光强度?
\begin{figure}[ht]
\centering
\includegraphics[width=10cm]{./figures/c7296c4c79582e93.png}
\caption{} \label{fig_HGD04_5}
\end{figure}
\item (1)一半径为$R$的圆环上均匀分布着总电量为$+Q$的电荷,试计算环心处的电场强度和电势(设无穷远电势为零)。\\
(2)今有一电荷$-Q$被限制在沿环的轴心线上(垂直于圆环平面)移动,试证明在环心附近处$-Q$电荷将作何种运动。
\item 将两个导体嵌于电导率为$\sigma$,介电常数为$\varepsilon$的介质中,导体之间的电阻为$R$,试求导体间电容。
\item 一对半径分别为$a,b(a<b)$的同心金属球中间填充了电导率为$\sigma$的均匀介质。设在$t=0$时,内球上突然出现了总电荷$Q$。\\
(1)计算介质中的电流。\\
(2)计算此电流产生的焦耳热。\\
(3)证明电荷重新分布而减少的静电能与焦耳热相等。
\item 如果要在一个半径为$R$的球体内产生均匀磁场$B$,需要一个怎样的球面电流分布。
\item 一个长方形线圈宽为$ a$、长为$ b$,以角速度 $\omega$绕$PQ$轴转动,如右图所示。一个随时间变化的均匀磁场$B=B_0\sin \omega t$垂直于$t=0$时的线圈平面。求线圈内的感生电动势,并指明其变化的频率为多少。
 \begin{figure}[ht]
 \centering
 \includegraphics[width=8cm]{./figures/1f5e44881c3f470e.png}
 \caption{} \label{fig_HGD04_6}
 \end{figure}
\end{enumerate}