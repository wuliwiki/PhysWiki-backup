% 百科创作指导

\subsection{预备知识}
\begin{itemize}
\item 我们重视百科的自洽性, 所以几乎所有词条都需要有预备知识(用 \verb|\pentry| 命令实现), 预备知识相当于 “必备知识”, 如果里面的内容不掌握, 读者阅读词条内容就会遇到困难.
\item 但一些拓展或者选读的相关词条不需要作为预备知识, 例如 “详见……”.
\item 一般来说我们假设读者具有普通高中生数理水平. 任何超出该水平的内容都需要再 “预备知识” 中有所体现. 如果百科存在低于该水平的内容, 也需要加入预备知识.
\item 在添加预备知识时, 先浏览一下里面的内容, 确保它包含当前词条所需内容.
\item 如果你写的内容在百科中找不到预备知识所需的内容, 应该把词条标记为 “缺少预备知识” (在 \verb|issues| 环境中插入 \verb|\issueMissDepend|)
\end{itemize}
