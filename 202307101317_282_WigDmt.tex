% Wigner D 矩阵、球谐函数的旋转
% keys 球谐函数|旋转|Wigner D 矩阵

\begin{issues}
\issueDraft
\end{issues}

\pentry{球谐函数\upref{SphHar}, 正交矩阵、酉矩阵\upref{UniMat}, 欧拉角\upref{EulerA}}

\footnote{参考 Wikipedia \href{https://en.wikipedia.org/wiki/Wigner_D-matrix}{相关页面}。}若我们把球谐函数 $Y_{l, m}(\uvec r)$ 绕原点进行某种旋转, 得到的函数可以表示成球谐函数的线性组合, 且只需要同一个 $l$ 子空间中的球谐函数, 若将旋转算符(主动)记为 $\mathcal{R}$, 则
\begin{equation}
\mathcal R\ket{l, m} = \sum_{m'} \ket{l, m'} \mel{l, m'}{\mathcal R}{l, m}~,
\end{equation}
我们把系数矩阵称为 \textbf{Wigner D 矩阵}
\begin{equation}
D_{m', m}^l = \mel{l, m'}{\mathcal R}{l, m}~.
\end{equation}
由于旋转算符是正交算符, 所以 Wigner D 矩阵是\textbf{酉矩阵}\upref{UniMat}。 注意该矩阵一般是复数的,因为作为基底的球谐函数一般是复数的。

\subsection{表达式}
一种定义旋转的方法就是利用 $zyz$ 欧拉角\upref{EulerA}, 即先将 $\ket{l, m}$ 绕 $z$ 轴逆时针旋转 $\gamma$ 角, 再绕 $y$ 轴逆时针旋转 $\beta$ 角, 最后绕 $z$ 轴旋转 $\alpha$ 角。 我们将旋转算符记为 $\mathcal R(\alpha, \beta, \gamma)$。 由球谐函数的定义\autoref{eq_SphHar_1}~\upref{SphHar}, 第一个旋转得到 $\exp(-\I m \gamma)\ket{l, m}$。

第二个旋转为
\begin{equation}
\mathcal R(0, \beta, 0) \ket{l, m} = \sum_{m'} \mel{l, m'}{\mathcal R(0, \beta, 0)}{l, m} \ket{l, m'}~.
\end{equation}
我们把系数矩阵称为 \textbf{Wigner d 矩阵}, 是 Wigner D 矩阵的一个特例。
\begin{equation}
d_{m', m}^l(\beta) = D_{m', m}^l(0, \beta, 0) = \mel{l, m'}{\mathcal R(0, \beta, 0)}{l, m}~.
\end{equation}
其表达式为(推导略)
\begin{equation}\ali{
d_{m', m}^l (\beta) = &\sqrt{(l + m')! (l - m')! (l + m)! (l - m)!} \ \ \times\\
&\sum_s \frac{(-1)^{m' - m + s} \qty(\cos \frac{\beta}{2})^{2l + m - m' - 2s} \qty(\sin\frac{\beta}{2})^{m' - m + 2s}}{(l + m - s)! s! (m' - m + s)! (l - m' - s)!}
}~.\end{equation}
其中 $s$ 的取值范围需要保证被阶乘的数为非负。

第二次旋转完以后我们得到若干个 $\ket{l, m'}$ 的线性组合, 要再次绕 $z$ 轴旋转 $\alpha$ 角, 只需要把它们分别乘以 $\exp(-\I m' \alpha)$ 即可。 所以 Wigner D 矩阵的完整表达式为
\begin{equation}
D_{m', m}^l(\alpha, \beta, \gamma) = \E^{-\I m' \alpha} d_{m', m}^l(\beta) \E^{-\I m \gamma}~.
\end{equation}
