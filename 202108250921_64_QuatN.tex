% 四元数与旋转矩阵
% keys 线性代数|矩阵|绕轴旋转矩阵|旋转矩阵|四元数|基底|基底变换矩阵

\begin{issues}
\issueAbstract
\issueTODO
\end{issues}

\pentry{罗德里格旋转公式\upref{RotA}, 旋转矩阵的导数\upref{RotDer},四元数\upref{Quat}}

四元数可以用来简洁地表示三维空间中的旋转,极大地减少了计算量.

三维空间中的一个向量表示为标部为 $0$ 的四元数:$v=\pmat{0, \bvec{v}}$.如果我们绕着一个单位向量 $\hat{\bvec{n}}$ 把 $\bvec{v}$ 旋转一个角度 $\theta$,所得的结果应该是哪个向量呢?取四元数 $q=\pmat{\cos{\frac{\theta}{2}}, \hat{\bvec{n}}\sin{\frac{\theta}{2}}}$,那么旋转后的向量就可以表示为
\begin{equation}
qvq^{-1}
\end{equation}

比如说,取 $v=(0, 1, 0, 0)$,它代表一个 $x$ 轴上的单位向量.如果想要把它绕着 $z$ 轴上的单位向量转 $\pi/2$,那么结果向量的四元数表示应该是 $(0, 0, 1, 0)$.按照我们定义的规则,旋转表示为四元数 $q=(\sqrt{2}/2, 0, 0, \sqrt{2}/2)$,按照四元数的乘法规则易得 $qvq^{-1}=(0,0,1,0)$.

\subsection{证明}

取四元数 $q=\pmat{\cos{\frac{\theta}{2}}, \hat{\bvec{n}}\sin{\frac{\theta}{2}}}$ 和 $v=(0, \bvec{v})$,直接计算得:

\begin{equation}
\begin{aligned}
qvq^{-1}&=\pmat{\cos{\frac{\theta}{2}}, \hat{\bvec{n}}\sin{\frac{\theta}{2}}}\cdot (0, \bvec{v})\cdot q^{-1}\\&
=(-\bvec{v}\cdot\hat{\bvec{n}}\sin{\frac{\theta}{2}}, \bvec{v}\cos{\frac{\theta}{2}}+\hat{\bvec{n}}\times\bvec{v}\sin{\frac{\theta}{2}})\cdot (\cos{\frac{\theta}{2}}, -\hat{\bvec{n}}\sin{\frac{\theta}{2}})\\&
=(0, \cos^2\frac{\theta}{2}\bvec{v}+\cos\frac{\theta}{2}\sin\frac{\theta}{2}\hat{\bvec{n}}\times\bvec{v}+(\hat{\bvec{n}}\cdot\bvec{v})\sin^2\frac{\theta}{2}\hat{\bvec{n}}\\&+\cos\frac{\theta}{2}\sin\frac{\theta}{2}\hat{\bvec{n}}\times\bvec{v}-\sin^2\frac{\theta}{2}\hat{\bvec{n}}\times(\hat{\bvec{n}}\times\bvec{v}))\\&
=(0, \cos^2\frac{\theta}{2}\bvec{v}-\sin^2\frac{\theta}{2}\hat{\bvec{n}}\times(\hat{\bvec{n}}\times\bvec{v})+\sin\theta\hat{\bvec{n}}\times\bvec{v}\\&+(\hat{\bvec{n}}\cdot\bvec{v})\sin^2\frac{\theta}{2}\hat{\bvec{n}})
\end{aligned}
\end{equation}

如果取 $\bvec{v}\perp\hat{\bvec{n}}$,考虑到此时 $\hat{\bvec{n}}\times(\hat{\bvec{n}}\times\bvec{v})=-\bvec{v}$ 且 $\hat{\bvec{n}}\cdot\bvec{v}=0$,那么上式可以化简为
\begin{equation}
\begin{aligned}
qvq^{-1}&=(0, \cos\theta\bvec{v}+\sin\theta\hat{\bvec{n}}\times\bvec{v})
\end{aligned}
\end{equation}

这正是 $\bvec{v}$ 绕 $\hat{\bvec{n}}$ 旋转 $\theta$ 的结果.

如果取 $\bvec{v}\parallel\hat{\bvec{n}}$,考虑到此时 $\hat{\bvec{n}}\times\bvec{v}=0$ 且 $(\hat{\bvec{n}}\cdot\bvec{v})\bvec{v}=\bvec{v}$,那么上式化为
\begin{equation}
\begin{aligned}
qvq^{-1}&=(0, \cos^2\frac{\theta}{2}\bvec{v}+\sin^2\frac{\theta}{2}\bvec{v})=(0, \bvec{v})=v
\end{aligned}
\end{equation}

这也是 $\bvec{v}$ 绕 $\hat{\bvec{n}}$ 旋转 $\theta$ 的结果.

综上所述,取任意向量 $\bvec{v}$,把它分为垂直和平行于旋转轴的 $\bvec{v}_\perp$ 和 $\bvec{v}_\parallel$,分别对应四元数 $v_\perp$ 和 $v_\parallel$,那么 $q v_\perp q^{-1}$ 和 $q v_\parallel q^{-1}$ 分别对应这两个向量旋转后的结果.故取 $v=v_\perp+v_\parallel$ 为 $\bvec{v}$ 对应的四元数后可由四元数乘法分配律得到,$qvq^{-1}$ 对应 $\bvec{v}$ 旋转后的向量.

一点补充:考虑到四元数限定向量方向时退化为复数(见\textbf{四元数}\upref{Quat}的讨论),故 $\bvec{v}\parallel\hat{\bvec{n}}$ 的情况其实可以直接套用复数的交换性来得到,即 $qvq^{-1}=qq^{-1}v=v$.






\subsection{旋转矩阵}
我们可以用\textbf{四元数(quaternions )} $\bvec q = [s, \bvec v]$ 来表示绕轴旋转矩阵, 其中
\begin{equation}
s = \cos(\frac{\theta}{2}) \qquad v = \abs{\bvec v} = \sin(\frac{\theta}{2})\uvec A
\end{equation}
则绕轴旋转矩阵\upref{RotA}可以表示为
\begin{equation}\label{QuatN_eq2}
\mat R(\theta) =
\begin{pmatrix}
1 - 2v_y^2 - 2v_z^2 & 2v_xv_y - 2sv_z  & 2v_x v_z + 2s v_y\\
2v_x v_y + 2sv_z & 1 - 2v_x^2 - 2v_z^2 & 2v_y v_z - 2s v_x\\
2v_x v_z - 2s v_y & 2v_y v_z + 2s v_x & 1 - 2v_x^2 - 2v_y^2
\end{pmatrix}
\end{equation}

四元数的乘法运算可以表示两个旋转矩阵相乘, 即把两次旋转合并为一次旋转
\begin{equation}
[s_1, \bvec v_1] [s_2, \bvec v_2] = [s_1 s_2 - \bvec v_1 \vdot \bvec v_2, s_1 \bvec v_2 + s_2 \bvec v_1 + \bvec v_1 \cross \bvec v_2]
\end{equation}
% 推导未完成

\subsection{导数}
若从坐标系 B 到坐标系 A 的基底变换矩阵为 $\mat R$, 当 B 绕原点以角速度 $\bvec \omega$ 旋转时有(\autoref{RotDer_eq4}~\upref{RotDer})
\begin{equation}
\dv{\mat R}{t} = \mat \Omega \mat R
\end{equation}
其中 $\mat\Omega$ 乘以任意位置矢量 $\bvec r$ 等于 $\bvec \omega \cross \bvec r$
\begin{equation}
\mat \Omega = \pmat{
0 & -\omega_z & \omega_y\\
\omega_z & 0 & -\omega_x\\
-\omega_y & \omega_x & 0
}\end{equation}
若旋转矩阵 $\mat R$ 对应的四元数为 $\bvec q$, 则
\begin{equation}\label{QuatN_eq1}
\dot {\bvec q}(t) = \frac12 [0, \bvec \omega(t)] \bvec q(t)
\end{equation}

% 推导大概就是旋转微小角度然后取极限
