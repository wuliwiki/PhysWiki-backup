% 伊本·海什木(综述)
% license CCBYSA3
% type Wiki

本文根据 CC-BY-SA 协议转载翻译自维基百科 \href{https://en.wikipedia.org/wiki/Ibn_al-Haytham}{相关文章}。

哈桑·伊本·海赛姆(Ḥasan Ibn al-Haytham,拉丁化名为 Alhazen,/ælˈhæzən/;全名:阿布·阿里·哈桑·伊本·哈桑·伊本·海赛姆,阿拉伯语:أبو علي، الحسن بن الحسن بن الهيثم;约965年—约1040年),是伊斯兰黄金时代的一位中世纪数学家、天文学家和物理学家,来自今天的伊拉克地区。[6][7][8][9]他被誉为“现代光学之父”,[10][11][12] 尤其在光学原理与视觉感知领域做出了重要贡献。他最具影响力的著作是《光学书》(阿拉伯语:كتاب المناظر,*Kitāb al-Manāẓir),写于1011年至1021年之间,现存有其拉丁文译本。[13]在科学革命时期,艾萨克·牛顿、约翰内斯·开普勒、克里斯蒂安·惠更斯和伽利略·伽利莱等人经常引用海赛姆的著作。

伊本·海赛姆是第一个正确解释视觉理论的人,[14] 他提出视觉是在大脑中形成的,并指出视觉具有主观性,会受到个体经验的影响。[15] 他还首次提出了光在折射时走最短时间路径的原理,这一原理后来被称为费马原理。[16]在镜学和透镜学领域,他通过对反射、折射以及光线成像性质的研究做出了重大贡献。[17][18]伊本·海赛姆还是最早倡导假设必须通过可验证程序或数学推理支持的实验来检验的人之一——在文艺复兴科学家出现前五个世纪,他就已是科学方法的早期奠基者,[19][20][21][22] 有时他被称为世界上“第一位真正的科学家”。[12]
此外,他还是一位博学多才的学者,著述涵盖哲学、神学和医学等多个领域。[23]
