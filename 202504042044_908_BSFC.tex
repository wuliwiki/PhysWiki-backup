% 泊松方程(综述)
% license CCBYSA3
% type Wiki

本文根据 CC-BY-SA 协议转载翻译自维基百科\href{https://en.wikipedia.org/wiki/Poisson\%27s_equation}{相关文章}。

泊松方程是一个在理论物理中广泛应用的椭圆型偏微分方程。例如,泊松方程的解是由给定的电荷或质量密度分布引起的势场;一旦知道了势场,就可以计算出相应的静电或引力(力)场。它是拉普拉斯方程的推广,后者在物理学中也经常出现。该方程以法国数学家和物理学家西蒙·丹尼斯·泊松的名字命名,泊松于1823年发布了这个方程。\(^\text{[1][2]}\)
\subsection{方程的表述}  
泊松方程是  
\[
\Delta \varphi = f,~
\]
其中,\(\Delta\)是拉普拉斯算子,\(f\)和\(\varphi\)是流形上的实值或复值函数。通常,给定\(f\),而求解\(\varphi\)。当流形是欧几里得空间时,拉普拉斯算子通常表示为\(\nabla^2\),因此泊松方程通常写作  
\[
\nabla^2 \varphi = f.~
\]
在三维笛卡尔坐标系中,它的形式为  
\[
\left( \frac{\partial^2}{\partial x^2} + \frac{\partial^2}{\partial y^2} + \frac{\partial^2}{\partial z^2} \right) \varphi (x, y, z) = f (x, y, z).~
\]
当\(f = 0\)恒成立时,我们得到拉普拉斯方程。
泊松方程可以通过格林函数求解:  
\[
\varphi(\mathbf{r}) = - \iiint \frac{f(\mathbf{r'})}{4 \pi |\mathbf{r} - \mathbf{r'}|} \, \mathrm{d}^3 r',~
\]
其中积分是对整个空间进行的。泊松方程的格林函数的详细说明见于筛选泊松方程的相关文章。还有多种数值求解方法,如松弛法,这是一种迭代算法。
\subsection{物理和工程中的应用}  
\subsubsection{牛顿引力}   
在由于吸引的质量物体密度为\(\rho\)的引力场\(g\)的情况下,可以使用引力的高斯定律的微分形式来得到相应的引力泊松方程。引力的高斯定律为  
\[
\nabla \cdot \mathbf{g} = -4\pi G \rho.~
\]
由于引力场是保守的(且无旋的),它可以用标量势\(\varphi\)来表示:  
\[
\mathbf{g} = - \nabla \varphi.~
\]
将这一表达式代入高斯定律,得到  
\[
\nabla \cdot (-\nabla \varphi) = -4\pi G \rho,~
\]  
从而得到引力的泊松方程:  
\[
\nabla^2 \varphi = 4\pi G \rho.~
\]
如果质量密度为零,泊松方程就化简为拉普拉斯方程。对应的格林函数可以用来计算从中心点质量\(m\)距离\(r\)处的势(即基本解)。在三维中,势为  
\[
\varphi (r) = \frac{-Gm}{r},~
\]  
这等价于牛顿的万有引力定律。
\subsubsection{静电学}
许多静电学中的问题由泊松方程所支配,该方程将电势\(\varphi\)与自由电荷密度\(\rho_f\)联系起来,这在导体等系统中尤为常见。

泊松方程的数学细节通常采用国际单位制(SI单位,区别于高斯单位制)来表达,描述了自由电荷的分布如何在某一区域内产生静电势。

从电学的高斯定律(也是麦克斯韦方程组之一)的微分形式出发,有:\(\nabla \cdot \mathbf{D} = \rho_f\),其中\(\nabla \cdot\)是散度算子,\(\mathbf{D}\)是电位移场,\(\rho_f\)是自由电荷密度(表示从外部引入的电荷)。

假设介质是线性、各向同性且均匀的(见极化密度),我们有本构关系:\(\mathbf{D} = \varepsilon \mathbf{E}\),其中\(\varepsilon\)是介质的介电常数,\(\mathbf{E}\)是电场强度。

将此代入高斯定律,并假设介电常数\(\varepsilon\)在感兴趣的区域内是空间常数,可以得到\(\nabla \cdot \mathbf{E} = \frac{\rho_f}{\varepsilon}\).在静电学中,我们假设不存在磁场(以下推导在存在恒定磁场时也成立)\(^\text{[3]}\)。因此,我们有\(\nabla \times \mathbf{E} = 0\),其中\(\nabla \times\)是旋度算子。这个方程意味着我们可以将电场表示为标量函数\(\varphi\)(称为电势)的梯度,因为任何梯度的旋度为零。因此,我们可以写作\\mathbf{E} = -\nabla \varphi,~
\]  
引入负号是为了将\(\varphi\)识别为单位电荷的电势能。

在这种情况下,泊松方程的推导是直接的。将电场的势梯度代入,得到  
\[
\nabla \cdot \mathbf{E} = \nabla \cdot (-\nabla \varphi) = -\nabla^2 \varphi = \frac{\rho_f}{\varepsilon},~
\]  
直接得出静电学中的泊松方程:
\[
\nabla^2 \varphi = -\frac{\rho_f}{\varepsilon}.~
\]
指定泊松方程来求解势函数需要知道电荷密度分布。如果电荷密度为零,则得到拉普拉斯方程。如果电荷密度遵循玻尔兹曼分布,则得到泊松-玻尔兹曼方程。泊松-玻尔兹曼方程在发展德拜-赫克尔稀溶液电解质理论中起着重要作用。

使用格林函数,从一个中心点电荷\(Q\)距离\(r\)处的势(即基本解)为  
\[
\varphi(r) = \frac{Q}{4\pi \varepsilon r},~
\]  
这就是库仑定律的静电学表达式。(出于历史原因,并且与上面引力模型不同,\(4\pi\)因子出现在这里,而不是高斯定律中。)

上述讨论假设磁场不随时间变化。即使磁场随时间变化,只要使用库仑规,仍然会得到相同的泊松方程。在这个更一般的情况下,计算\(\varphi\)不再足以计算电场\(\mathbf{E}\),因为\(\mathbf{E}\)还依赖于磁矢势\(\mathbf{A}\),它必须独立计算。有关\(\varphi\)和\(\mathbf{A}\)在麦克斯韦方程中的应用以及如何在这种情况下得到适当的泊松方程的更多内容,请参见麦克斯韦方程的势函数形式。

\textbf{高斯电荷密度的势}

如果存在一个静态的球对称高斯电荷密度  
\[
\rho_f(r) = \frac{Q}{\sigma^3 (2\pi)^{3/2}}\, e^{-r^2 / (2\sigma^2)},~
\]  
其中\(Q\)是总电荷,那么泊松方程  
\[
\nabla^2 \varphi = -\frac{\rho_f}{\varepsilon}~
\]  
的解为  
\[
\varphi(r) = \frac{1}{4\pi \varepsilon} \cdot \frac{Q}{r} \cdot \operatorname{erf} \left( \frac{r}{\sqrt{2}\sigma} \right),~
\]  
其中\(\operatorname{erf}(x)\)是误差函数。该解可以通过直接计算\(\nabla^2 \varphi\)来验证。

需要注意的是,当\(r\)远大于\(\sigma\)时,\(\operatorname{erf}(r / \sqrt{2} \sigma)\)趋近于 1,[6] 这时电势\(\varphi(r)\)会趋近于点电荷的电势形式:
\[
\varphi \approx \frac{1}{4\pi \varepsilon} \cdot \frac{Q}{r},~
\]  
这是符合预期的结果。此外,误差函数随着其自变量的增大而迅速接近 1;在实际应用中,当\(r > 3\sigma\)时,相对误差已小于千分之一。[6]