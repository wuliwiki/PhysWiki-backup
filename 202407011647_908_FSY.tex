% 放射源
% license CCBYSA3
% type Wiki

(本文根据 CC-BY-SA 协议转载自原搜狗科学百科对英文维基百科的翻译)

放射源是指发出电离辐射的已知数量放射性核素,典型的几种辐射类型有γ射线、$\alpha$粒子、$\beta$粒子和中子辐射。

放射源可用于辐照(这里辐射对目标材料起到了重要的电离作用),可以作为用于辐射测量过程和辐射防护仪器校准的辐射计量源,也用于工业过程测量,如造纸和钢铁工业中的厚度测量。放射源可以密封在容器中(高穿透性辐射)或者沉积在表面上(弱穿透性辐射),或者将它们置于流体中。

作为一种辐射源,它们被应用于放射治疗医学和工业中,例如工业射线成像、食品辐射、消毒、灭虫和聚氯乙烯的辐射交联。

放射性核素是根据它们发射的辐射类型和特征、发射强度和衰变半衰期来分类的。常见的放射性核素包括钴-60,[1] 铱-192,[2] 和锶-90。[3] 在国际单位制(SI)中,放射源活度的测量单位是贝克勒尔,但是在部分地区(比如美国)仍然使用历史上的单位:居里。尽管美国国家标准技术研究所(NIST)强烈建议使用国际单位制。[4] 出于健康卫生的目的,在欧盟国际单位制是强制使用的。

\subsection{密封源}
许多放射源是密封的,这意味着它们要么永久地完全包裹在胶囊中,要么牢固地结合在表面上。胶囊通常由不锈钢、钛、铂或其他惰性金属制成。[5] 密封源的使用消除了由于操作不当造成的放射性物质扩散到环境中的几乎所有风险,[5] 但是容器并不是用于衰减辐射,因此需要进一步的屏蔽来辐射防护。[6] 在不需要把放射源用化学或者物理手段包含在液体或者气体的应用场合,密封源几乎都可以适用。

\subsubsection{1.1 密封源的分类[7]}
国际原子能机构(IAEA)根据密封源相对于最低危险源的活度对其进行分类(其中危险源可能对人类造成重大伤害)。使用的比率是A/D,其中A是放射源活度,D是最小危险活度。
\begin{table}[ht]
\centering
\caption\label{FSY}
\begin{tabular}{|c|c}
\hline
\textbf{种类} & \textbf{A/D}\\
\hline
1 & ≥1000\\
\hline
2 & 10–1000\\
\hline
3 & 1–10\\
\hline
4 & 0.01–1\\
\hline
5 & <0.01\\
\hline
\end{tabular}
\end{table}
