% 万有引力定律(高中)
% 开普勒定律|万有引力|天体|宇宙速度|同步卫星

\begin{issues}
\issueDraft
\issueTODO
\end{issues}

\subsection{开普勒定律}

开普勒定律是开普勒根据对太阳系中行星运动的观测数据总结出来的,是一个普适定律,也适用于其他天体绕中心天体运动的情况,如卫星围绕地球的运动等.

第一定律:所有行星绕太阳运动的轨道都是椭圆,太阳在椭圆的一个焦点上.

第二定律:对任意一个行星来说,它与太阳的连线在相等时间内扫过相等的面积.

第三定律:所有行星的轨道半长轴($a$)的三次方跟它的公转周期($T$)的二次方之比都相等.表达式为
\begin{equation}\label{HSPM06_eq1}
\frac{a^3}{T^2}=k
\end{equation}
要注意的是,\autoref{HSPM06_eq1} 中代表“都相等“的比值$k$,是针对围绕同一中心天体运动的所有天体而言的,因为$k$的大小与中心天体的质量有关.

\subsection{万有引力定律}

内容:任意两个质点都相互吸引,这个引力的大小与两质点的质量的乘积成正比,与两质点的距离的平方成反比.

万有引力的大小为
\begin{equation}
F=G\frac{m_1m_2}{r^2}
\end{equation}
$m_1$、$m_2$分别为两质点的质量,$r$为两质点间的距离,$G$为引力常量,且$G=6.67\times 10^{-11}\mathrm{N\cdot m^2/kg^2}$

对于两个不能被视为质点的物体,它们之间的引力等于组成其中一个物体的所有质点与组成另一个物体的所有质点之间的所有引力的合力(矢量和).当然,这种情况已经属于大学阶段研究的问题了,在高中阶段,以处理可视作质点的问题为主.

\subsection{地球对物体的万有引力与物体所受重力}

首先,我们把地球近似看成半径为$R$、质量为$M$且均匀分布的球体.由于地球自转,地球上的物体会随之做圆周运动,则地球对物体的万有引力可分解为提供物体圆周运动的向心力和物体所受的重力.可见由于地球自转,万有引力和重力是存在差异的.

\subsubsection{地球表面的重力}

在地球两极处,向心力为零,此时物体所受重力等于万有引力,方向指向地心.

在赤道上,由$F=m\omega^2R$可知,向心力达到最大值,且与万有引力、重力同向,此时物体所受的重力最小,也指向地心.

在地球其他地面位置处,物体受到的万有引力、重力及其做圆周运动的向心力遵循平行四边形定则,重力小于万有引力且不指向地心,随纬度增大而增大.

\subsubsection{距地表高$h$处的重力(重力加速度)}

忽略地球自转时,在地球表面:
\begin{equation}\label{HSPM06_eq2}
mg=G\frac{Mm}{R^2}
\end{equation}

距离地表$h$高度处:
\begin{equation}
mg'=G\frac{Mm}{(R+h)^2}
\end{equation}

\begin{equation}
g'=\frac{R}{R+h}^2g
\end{equation}

可见高度$h$越大,重力(重力加速度)越小

\subsubsection{距地表深$d$处的重力(重力加速度)}

距离地表$d$深度处,已知质量分布均匀的球壳对壳内物体的引力为零(证明参考 壳层定理),只有地球内部半径为$R-d$的球形区域对物体有引力的作用,则:
\begin{equation}\label{HSPM06_eq3}
mg'=G\frac{M'm}{(R-d)^2}
\end{equation}

通过密度的计算可知:
\begin{equation}\label{HSPM06_eq4}
\frac{M}{R^3}=\frac{M'}{(R-d)^3}
\end{equation}

结合\autoref{HSPM06_eq2} \autoref{HSPM06_eq3} \autoref{HSPM06_eq4} 可得
\begin{equation}
g'=\frac{R-d}{R}g
\end{equation}

可见深度$d$越大,重力(重力加速度)越小