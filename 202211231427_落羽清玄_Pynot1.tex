% Python 注释
% Python|注释

\begin{issues}
\issueTODO
\end{issues}

\pentry{Python 输入 & 输出\upref{Pyio}}

\begin{definition}{注释}
\begin{itemize}
\item 注释可用于解释 Python 代码.
\item 确保对模块, 函数, 方法和行内注释使用正确的风格.
\item 注释可用于提高代码的可读性.
\end{itemize}
\end{definition}

\subsection{单行注释}
Python 中单行注释以 # 开头,例如:
\begin{lstlisting}[language=python]
# 这是一个注释,注意标准格式要在#后面加一个空格再加上注释
print("你好,世界!")  # 这是一个注释,注意在代码尾部的注释标准格式要在#的前面在再加两个空格
\end{lstlisting}

\subsection{多行注释}
Python 实际上没有多行注释的语法.
要添加多行注释,您可以为每行插入一个#
\begin{lstlisting}[language=python]
#这是一个注释
#这是一个注释
#这是一个注释
print("你好,世界!")
\end{lstlisting}

或者,以不完全符合预期的方式,您可以使用多行字符串.由于 Python 将忽略未分配给变量的字符串文字,因此您可以在代码中添加多行字符串,并在其中添加注释
\subsubsection{单引号(''')}
\begin{lstlisting}[language=python]
'''
这是多行注释,用三个单引号
这是多行注释,用三个单引号 
这是多行注释,用三个单引号
'''
print("Hello, World!")
\end{lstlisting}

\subsubsection{双引号(""")}
\begin{lstlisting}[language=python]
"""
这是多行注释,用三个双引号
这是多行注释,用三个双引号 
这是多行注释,用三个双引号
"""
print("Hello, World!")
\end{lstlisting}

\subsection{#!/usr/bin/python3 }
仅仅在\textbf{linux}或\textbf{unix}系统下有作用,在\textbf{windows}下无论在代码里加什么都无法直接运行一个文件名后缀为.py的脚本,因为在windows下文件名对文件的打开方式起了决定性作用.
\begin{lstlisting}[language=python]
#!/usr/bin/env python
或者
#!/usr/bin/python
\end{lstlisting}

\subsubsection{如何理解}
把这一行语句拆成两部分.第一部分是 \textbf{#!};第二部分是 \textbf{/usr/bin/python} 或者 \textbf{/usr/bin/env python} .关于 \textbf{#!} 这个符号,其实它是有名字的,叫做 Shebang 或者Sha-bang ,有的翻译组将它译作 释伴,即“解释伴随行”的简称,同时又是Shebang的音译.

Shebang通常出现在类Unix系统的脚本中第一行,作为前两个字符.在Shebang之后,可以有一个或数个空白字符,后接解释器的绝对路径,用于指明执行这个脚本文件的解释器

\begin{lstlisting}[language=bash]
... # python
等阶
... # usr/bin/python
\end{lstlisting}

如果不加 \textbf{#!} 的话,你每次执行这个脚本时,都得这样 python xx.py

有没有一种方式?可以省去每次都加 python 呢?

当然有,你可以文件头里加上#!/usr/bin/python ,那么当这个文件有可执行权限 时,只直接写这个脚本文件,就像下面这

\subsection{coding = utf-8 }
\begin{lstlisting}[language=python]
# -*- coding: UTF-8 -*-
或者
# coding=utf-8
\end{lstlisting}
\begin{itemize}
\item 作用:将编码格式改为utf-8格式
\item Python版本:Python2中默认的编码格式为ASCII码格式,Python3中默认的编码格式为UTF-8格式.
\item 使用原因:在ASCII码格式下不能出现中文字符(代码或注释都不可以),否则就会报错,只有在utf-8格式下才能正常编译运行.所以在Python2版本下,只要出现中文,一定要加上这句头部声明.
\item 代码的可移植性:Python3的默认格式就是utf-8,这句话对它就没有意义了,但是考虑到代码的可移植性,还是建议加上这句话.
\end{itemize}

