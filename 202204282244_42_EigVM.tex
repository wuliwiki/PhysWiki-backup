% 特征矢量与特征多项式
% 1维不变子空间|特征矢量|特征值|特征多项式
\pentry{不变子空间\upref{InvSP}}
在矩阵的本征方程\upref{MatEig}一节中,从矩阵的角度介绍了矩阵的本征矢量与本征多项式的概念,本节将从线性算子的角度介绍这一概念.

在矩阵理论中,我们知道相似矩阵的本征向量与本征多项式都相同,但相似矩阵之间在形式上是不同的.然而在线代算子代数的语言下,所有相似的矩阵都对应一个线性算子,而相似矩阵中的每一矩阵只不过是同一线性算子在某一基底下的表现形式\autoref{LiOper_sub1}~\upref{LiOper}.也就是说,在线性算子的语言下,相似的矩阵都是同一个东西,那么它们在本质上就该是相同的,这也是对“本征矢量”和“本征多项式”中“本征”一词的理解.在本节,为了表示在算子代数之下讨论同一概念,我们用“特征”一次代替“本征”.
\subsection{特征矢量与特征值}
算子 $\mathcal{A}$ 的一维不变子空间\upref{InvSP}中的非零矢量都被称为 $\mathcal{A}$ 的\textbf{特征矢量}.显然,如果 $\bvec x$ 是特征矢量,就有
\begin{equation}
\mathcal{A}\bvec x=\lambda \bvec x,\quad \lambda\in\mathbb{F}
\end{equation}
其中,$\lambda$ 称为算子 $\mathcal{A}$ 的对应特征矢量 $\bvec x$ 的\textbf{特征值}.