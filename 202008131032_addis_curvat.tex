% 曲率 曲率半径

\pentry{切线\upref{TanL}}

我们来看一个平面上的一个光滑曲线(即处处存在切线), 我们如何描述它某点处的弯曲程度呢? 一种常用方法是在这点附近取曲线的一小段, 然后做一个尽量与它吻合的圆, 当这小段的长度趋近于 0 时, 这个圆可以唯一确定. 我们把这个圆叫做\textbf{密切圆(osculating circle)}, 把密切圆的半径叫做曲线在该点的\textbf{曲率半径(radius of curvature)}, 曲率半径的倒数叫做\textbf{曲率(curvature)}.

我们先来看一个半径为 $R$ 的圆的一小段圆弧, 令其长度为 $\Delta l$. 作这段圆弧两端的切线, 令它们的夹角为 $\Delta \theta$, 那么显然满足 $R \theta = \Delta l$. 同理, 对于任意光滑曲线上长度为 $\Delta l$ 的一段, 我们也可以做相同的处理, 但需要令 $\Delta l \to 0$
\begin{equation}
R = \lim_{\Delta l \to 0} \frac{\Delta l}{\Delta \theta}
\end{equation}
曲率的具体的计算公式取决于使用什么方式定义曲线, 例如使用直角坐标系还是极坐标系, 是否使用参数方程. 下面来一一介绍.

\subsection{直角坐标系的函数曲线}
\pentry{高阶导数\upref{HigDer}, 一元函数的微分\upref{Diff}}

平面上曲线的最常见描述方式就是通过定义函数 $y = f(x)$ ($f:\mathbb R \to \mathbb R$). 我们可以通过导数\upref{Der}计算曲线上某点切线关于 $x$ 轴的夹角 $\theta$.
\begin{equation}
f'(x) = \tan \theta
\end{equation}
我们再对该式做微分, 把 $\theta$ 看成 $x$ 得函数得
\begin{equation}
\dd{f'(x)} = f''(x) \dd{x} = \frac{1}{\cos^2\theta} \dd{\theta} = (1 + \tan^2\theta) \dd{\theta}
\end{equation}
曲线长度的微分为
\begin{equation}
\dd{l} = \frac{\dd{x}}{\cos\theta}
\end{equation}

所以
\begin{equation}
\dd{\theta} = \frac{f''(x)}{1 + f'(x)^2}\dd{x}
\end{equation}

