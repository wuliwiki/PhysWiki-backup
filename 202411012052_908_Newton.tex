% 艾萨克·牛顿(综述)
% license CCBYSA3
% type Wiki

本文根据 CC-BY-SA 协议转载翻译自维基百科\href{https://en.wikipedia.org/wiki/Isaac_Newton}{相关文章})

\begin{figure}[ht]
\centering
\includegraphics[width=6cm]{./figures/0c83e6f3dfbe0a8c.png}
\caption{《46岁的牛顿肖像,1689年》} \label{fig_Newton_1}
\end{figure}
艾萨克·牛顿爵士,皇家学会会员(1642年12月25日-1726/27年3月20日[a]),是一位英国博学家,活跃于数学、物理学、天文学、炼金术、神学和写作领域,在他所在的时代被称为自然哲学家。他是科学革命及其后的启蒙运动中的关键人物。他的开创性著作《自然哲学的数学原理》首次出版于1687年,汇集了许多前人的研究成果,奠定了经典力学的基础。牛顿还在光学方面做出了开创性的贡献,并与德国数学家戈特弗里德·威廉·莱布尼茨共同被认为是微积分的创立者,尽管他在莱布尼茨之前几年就已发展了微积分。[10][11]

在《自然哲学的数学原理》中,牛顿制定了运动定律和万有引力定律,这些理论成为数个世纪以来主导性的科学观点,直到相对论的出现。他利用对重力的数学描述推导了开普勒的行星运动定律,解释了潮汐、彗星轨迹、岁差等现象,消除了关于太阳系日心说的疑虑。他展示了地球上的物体和天体的运动可以由相同的原理解释。牛顿推测地球为扁球体,这一推测后来由莫佩尔蒂、拉康达米娜等人的测地测量所证实,使得大多数欧洲科学家信服于牛顿力学的优越性。

他制造了第一个实用的反射望远镜,并基于棱镜将白光分解为可见光谱的颜色的观察,发展出一套精细的颜色理论。他关于光的研究汇集于其极具影响力的著作《光学》中,1704年出版。他提出了一个经验性的冷却定律,这是第一个热传导的表述,首次对声速进行了理论计算,并引入了牛顿流体的概念。此外,他还对电进行了早期研究,他在《光学》一书中的一个设想可以说是电场理论的开端。作为数学家,除了微积分的研究外,他还对幂级数进行了研究,将二项式定理推广至非整数指数,发展出求解函数根的方法,并分类了大部分的三次平面曲线。

牛顿是剑桥大学三一学院的成员,也是剑桥大学的第二任卢卡斯数学教授。他是一位虔诚但非正统的基督徒,私下拒绝三位一体的教义。他拒绝加入英国国教的圣职,这在当时的剑桥大学教员中是少见的。除了数学科学方面的工作之外,牛顿还将大量时间投入到炼金术和《圣经》年代学的研究中,但他在这些领域的大部分作品直到去世后很久才发表。在政治上,他与辉格党有密切联系,并曾在1689-1690年和1701-1702年两次短暂担任剑桥大学的国会议员。1705年,他被安妮女王封为爵士,并在伦敦度过了生命的最后三十年,担任皇家造币厂的监理(1696–1699)和厂长(1699–1727),以及皇家学会会长(1703–1727)。
\subsection{早期生活}  
主要条目:艾萨克·牛顿的早期生活  
艾萨克·牛顿于1642年12月25日(根据当时在英格兰使用的儒略历,公历为1643年1月4日)出生于林肯郡的伍尔斯索普庄园。[17]牛顿的父亲也是名叫艾萨克·牛顿,在他出生前三个月去世。牛顿出生时早产,身体较小;他的母亲汉娜·艾斯考说,他可以放进一个夸脱的杯子里。[18]当牛顿三岁时,母亲再婚,和她的新丈夫巴纳巴斯·史密斯牧师一起生活,留下牛顿由他的外祖母玛格丽·艾斯考(原姓布莱思)照顾。牛顿不喜欢他的继父,并对母亲再婚心存怨恨,这在他19岁之前的一份罪行清单中有所体现:“威胁我的父亲和母亲史密斯要烧掉他们和他们的房子。”[19] 牛顿的母亲在第二次婚姻中生了三个孩子(玛丽、本杰明和汉娜)。[20]
\subsubsection{国王学校}  
牛顿大约在十二岁到十七岁期间,就读于格兰瑟姆的国王学校,该校教授拉丁语和古希腊语,并可能为他打下了坚实的数学基础。1659年10月,他被母亲从学校撤回,回到了伍尔斯索普-拜-科尔斯特沃斯。母亲在第二次丧夫后,试图让他成为一名农民,但他对此职业十分厌恶。国王学校的校长亨利·斯托克斯说服他的母亲让他重返学校。部分出于对校园欺凌者的报复心态,他成为了年级第一的学生,主要通过制作日晷和风车模型而脱颖而出。[24]
\subsubsection{剑桥大学}  
1661年6月,牛顿被剑桥大学的三一学院录取。他的叔叔威廉·艾斯考牧师曾在剑桥学习,推荐他进入该校。在剑桥,牛顿起初以“补助生”的身份入学,通过做杂役来支付学费,直到1664年获得奖学金,覆盖他四年的大学费用,直至获得硕士学位。[25]当时,剑桥的教学以亚里士多德的理论为基础,牛顿与当时的现代哲学家如笛卡尔以及天文学家伽利略·伽利莱和托马斯·斯特里特一起阅读这些作品。他在笔记本中记录了一系列关于机械哲学的“问题”。1665年,他发现了广义二项式定理,并开始发展一个后来成为微积分的数学理论。1665年8月,牛顿在剑桥获得学士学位后,因应对大瘟疫,大学暂时关闭。[26]

尽管他在剑桥大学的学生时代并不显著[27] ,但在接下来的两年里,[28]牛顿在伍尔斯索普的家中进行的私下研究促成了他在微积分、光学和引力定律方面的理论发展。[29][30]

1667年4月,牛顿返回剑桥大学,并在10月被选为三一学院的研究员。[31][32]研究员需要接受圣职并被按立为英国国教牧师,尽管在复辟时期这一要求并未严格执行,符合英格兰教会的声明便足够。他承诺道:“我要么将神学作为我研究的对象,并在这些章程规定的时间(7年)到来时接受圣职,要么就辞去学院职务。”[33]在此之前,他对宗教并没有过多思考,曾两次签署同意《三十九条》,即英格兰教会教义的基础。到1675年,这个问题无法避免,而此时他的非常规观点成为了障碍。[34]

他的学术工作给卢卡斯教授艾萨克·巴罗留下了深刻印象,巴罗渴望发展自己的宗教和行政潜力(他在两年后成为三一学院的院长);在1669年,牛顿接替了他的职位,这距离他获得硕士学位仅一年。卢卡斯教授的任职条款要求持有者不得活跃于教会——可能是为了留出更多时间用于科学研究。牛顿认为这应该使他免于按立的要求,查理二世国王接受了这一论点,因此牛顿的宗教观点与英国国教的正统观念之间的冲突得以避免。[35]

剑桥大学卢卡斯数学教授的职位还包括教授地理的责任。[36][37]在1672年和1681年,牛顿出版了《一般地理》的修订、校正和增补版,这本地理教科书最初由已故的伯纳德·瓦伦纽斯于1650年出版。[38]在《一般地理》中,瓦伦纽斯试图建立一个理论基础,将科学原则与古典地理概念联系起来,并认为地理是科学与应用于量化地球特征的纯数学的结合。[36][39] 虽然不清楚牛顿是否曾讲授地理,但1733年杜格代尔和肖的英文翻译版本中提到,牛顿出版此书是为了让学生在他讲授这一主题时阅读。[36] 《一般地理》被一些人视为地理历史中古代与现代传统的分界线,而牛顿参与后续版本的编辑被认为是这一持久遗产的重要原因之一。[40]

牛顿于1672年被选为皇家学会会员(FRS)。[1]
\subsection{中年}  
\subsubsection{微积分}  
牛顿的工作被认为“显著推动了当时所有研究的数学分支”。[41] 他在这一主题上的研究,通常称为流量(fluxions)或微积分,见于1666年10月的一份手稿,现已被收录在牛顿的数学论文中。[42]他的作品《通过无限项数的方程进行分析》(De analysi per aequationes numero terminorum infinitas)于1669年6月由艾萨克·巴罗发送给约翰·柯林斯,巴罗在同年8月给柯林斯的信中指出,这是“一个非凡天才和在这些领域中精通的作品”。[43]牛顿后来与莱布尼茨发生了关于微积分发展优先权的争论。大多数现代历史学家认为,牛顿和莱布尼茨是独立发展微积分的,尽管他们使用的数学符号大相径庭。然而,已确定牛顿在莱布尼茨之前很早就开始发展微积分。[44][11][45] 莱布尼茨的符号和“微分法”,如今被认为是更加便利的符号,后来被欧洲大陆的数学家采用,并在1820年左右也被英国数学家接受。

他的工作广泛使用基于趋近于零的小量比率的极限值的几何形式微积分:在《自然哲学的数学原理》中,牛顿以“第一和最后比率的方法”这一名称进行了演示,[46] 并解释了为何以这种形式进行阐述,[47]同时也指出“通过这种方式,完成的与无体积法所完成的是相同的。”[48]因此,现代人称《原理》为“一本充满无穷微积分理论和应用的书”,而在牛顿的时代,[49] 几乎所有内容都与这种微积分有关。[50] 他在1684年的《旋转物体运动论》中使用了涉及“一或多个无穷小量阶”的方法,并且在他关于运动的论文中也体现了这一点,[51]“这些论文是在1684年前的二十年内写成的”。[52]

牛顿曾对发表他的微积分持犹豫态度,因为他担心会引发争议和批评。[53]牛顿与瑞士数学家尼古拉·法蒂奥·德·杜伊耶关系密切。1691年,杜伊耶开始撰写牛顿《原理》的新版本,并与莱布尼茨通信。[54]1693年,杜伊耶与牛顿的关系恶化,这本书也未能完成。[55]从1699年开始,皇家学会的其他成员指控莱布尼茨抄袭。[56]1711年,争论进一步升级,皇家学会在一项研究中宣称真正的发现者是牛顿,并将莱布尼茨标记为骗子;后来发现牛顿实际上写下了该研究关于莱布尼茨的结论性评论。由此开始了这场痛苦的争议,影响了牛顿和莱布尼茨两人的生活,直至莱布尼茨于1716年去世。[57]
\begin{figure}[ht]
\centering
\includegraphics[width=6cm]{./figures/d24e0f92cc76e386.png}
\caption{1702年,牛顿的肖像由戈弗雷·奈勒(Godfrey Kneller)绘制。} \label{fig_Newton_2}
\end{figure}
牛顿通常被认为是广义二项式定理的创立者,该定理适用于任何指数。他发现了牛顿恒等式和牛顿法,分类了三次平面曲线(两个变量的三次多项式),对有限差分理论做出了重要贡献,是第一个使用分数指数的人,并采用坐标几何推导丢番图方程的解。他通过对数近似调和级数的部分和(这是欧拉求和公式的前身),并首次自信地使用幂级数及其反演。牛顿对无穷级数的研究受到西蒙·斯特芬的十进制数的启发。[58]
\subsubsection{光学}
