% 线性映射的张量积
% keys 对称积|反对称积|对称幂|反对称幂
% license Xiao
% type Wiki

\begin{issues}
\issueDraft
\issueOther{可以对照张量的对称化和交错化\upref{SIofTe}进行阅读}
\end{issues}
\pentry{向量空间的张量积\upref{TPofSp},线性映射\upref{LinMap}}

\subsection{线性映射的张量积}

两个线性映射 $f: V \to W$,$g: V' \to W'$ 之间可以定义它们的张量积\footnote{更严格的写法应该是 $\sum_i v_i \otimes v_i' \mapsto \sum_i f(v_i) \otimes g(v_i')$,不过本文的所有映射都是线性映射,所以只需要定义一组基的线性变换即可。}
\begin{equation}
\begin{aligned}
f \otimes g: V \otimes V' &\to W \otimes W', \\
v \otimes v' &\mapsto f(v) \otimes g(v')~.
\end{aligned}
\end{equation}


\subsubsection{线性映射的空间的张量积}

从另一个角度来说,全体 $V$ 到 $W$ 的线性映射的集合 $L(V, W)$ 是一个向量空间(参考\autoref{sub_LinMap_1}~\upref{LinMap})



\subsection{线性映射的对称/反对称幂}

\pentry{向量空间的对称/反对称幂\upref{vecSAS}}