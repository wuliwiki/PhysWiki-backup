% 对称化与反对称化

\pentry{粒子交换算符\upref{ExchOp}}

如果给定一个多粒子波函数处于对称或反对称子空间中, 我们可以将其\textbf{对称化(symmetrize)}或\textbf{反对称化(antisymmetrize)}. 先用粒子交换算符表示\autoref{IdPar_ex1}~\upref{IdPar}的过程.

\begin{example}{}\label{symetr_ex2}
假设单个粒子态空间的一组单位正交基底是 $\qty{\ket{i}}$, 两个粒子某时刻的态矢可以用单个张量积表示 $\ket{i}\ket{j}$ 且 . 若 $i = j$, 则显然这个态已经是对称的.  若 $i \ne j$, 则态矢是不对称的, 我们可以通过乘以 $(1 \pm P_{12})/\sqrt{2}$ 来对称或反对称化(分别取正号和负号)
\begin{equation}\label{symetr_eq5}
\frac{1}{\sqrt{2}}(1 \pm P_{12})\ket{i}\ket{j} = \frac{1}{\sqrt{2}}(\ket{i}\ket{j} \pm \ket{j}\ket{i})
\end{equation}
由于等式中的两项也是正交归一的, 我们需要另乘归一化系数 $1/\sqrt 2$. 可以验证, 该式就是 $P_{1,2}$ 的本征矢, 本征值分别为 $\pm 1$. 这种方法利用了交换算符的性质\autoref{ExchOp_eq7}~\upref{ExchOp}.

更一般地, 我们也可以不要求 $\ket{i}\ket{j}$ 正交, 但这样\autoref{symetr_eq5} 中的两项也变得不正交, 我们就要重新计算归一化系数了.
\end{example}

在该例中, 我们把 $(1 \pm P_{12})/\sqrt{2}$ 称为\textbf{对称化算符(symmetrizer)}或\textbf{反对称化算符(antisymmetrizer)}.

\subsection{多粒子对称化}
若将\autoref{symetr_ex2} 中的两个粒子改为 $N$ 个粒子, 如何将状态 $\ket{i_1}\ket{i_2}\dots\ket{i_N}$ (反)对称化呢? 稍加思考会发现, 若要对称化, 我们只需要将 $\ket{i_1},\ket{i_2},\dots,\ket{i_N}$ 的所有不同的排列相加再归一化即可. 如果这 $N$ 个单粒子态都是不同的, 那么一共有 $N!$ 种排列. 我们用 $p_n(i)$ 来表示, 例如
\begin{table}[ht]
\centering
\caption{$N = 3$ 的 6 种排列}\label{symetr_tab2}
\begin{tabular}{|c|c|c|c|c|c|c|}
\hline
  & $p_1(i)$ & $p_2(i)$ & $p_3(i)$ & $p_4(i)$ & $p_5(i)$ & $p_6(i)$ \\
\hline
$i=1$ & 1 & 1 & 2 & 2 & 3 & 3 \\
\hline
$i=2$ & 2 & 3 & 1 & 3 & 1 & 2 \\
\hline
$i=3$ & 3 & 2 & 3 & 1 & 2 & 1 \\
\hline
\end{tabular}
\end{table}
则正交化的结果为
\begin{equation}
\frac{1}{N!}\sum_{n = 0}^{N!} \ket{p_n(1)}\ket{p_n(2)}\dots\ket{p_n(N)}
\end{equation}

但如果其中有一些是相同的, 情况就要 那么就有 $N!/(n_1! n_2! \dots n_M)$.

\subsection{多粒子反对称化,斯莱特行列式}


\addTODO{未完成}

斯莱特行列式(Slater determinant)

双粒子空间中一个不对称的态矢可以表示为
\begin{equation}
\ket{v} = \sum_{i,j} c_{i,j} \ket{i}\ket{j}
\end{equation}

\subsection{子空间的维度}
假设单个粒子态所在空间是 $M$ 维的(例如只考虑自旋空间), 基底记为 $\qty{\ket{i}}$. 当有 $N$ 个费米子时, 总状态处于 $M^N$ 维空间的反对称子空间中. 这个空间的维数是多少呢?  
\addTODO{$C_M^N$}

% 我们先将它用基底展开为
% \begin{equation}
% \ket{\chi} = \sum_{i_1,\dots,i_N} c_{i_1,\dots,i_N}\ket{i_1}\dots\ket{i_N}
% \end{equation}
% 加上反对称条件后, 独立系数的个数就是空间的维数. 可以证明

