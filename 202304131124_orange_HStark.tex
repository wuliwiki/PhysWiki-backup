% 类氢原子斯塔克效应(微扰)

\begin{issues}
\issueDraft
\end{issues}

\pentry{不含时微扰理论\upref{TIPT}}

微扰理论($\mathcal{E_z}$ 是 $z$ 方向电场):
\begin{equation}
H' = \mathcal{E_z} z~.
\end{equation}
矩阵元为
\begin{equation}\label{eq_HStark_1}
H'_{l',l} = \mathcal{E_z}\mel{\psi_{n,l',m}}{z}{\psi_{n,l,m}}
\end{equation}


\begin{example}{氢原子 $n=2$ 的斯塔克效应}
先考虑 $n=2$, $m=0$ 的情况, 这是一个二维希尔伯特子空间。 根据\autoref{tab_HDipM_1}~\upref{HDipM}, \autoref{eq_HStark_1} 为
\begin{equation}
\mat H' = -3\mathcal{E_z}\pmat{0 & 1\\ 1 & 0}
\end{equation}
本征值为 $E_{\pm}^1 = \mp 3\mathcal{E_z}$, 好本征态为 $\ket{2\pm} = (\ket{20} \pm \ket{21})/{\sqrt 2}$, 也被称为 \textbf{Stark 态}。

\begin{figure}[ht]
\centering
\includegraphics[width=8cm]{./figures/9ef123ae9a2374aa.png}
\caption{$\ket{2+}$ 的概率密度函数的 $x$-$z$ 切面, 可见电子向下偏移, 电场向上为正, 所以本征能量变小。 $\ket{2-}$ 态是此图上下翻转, 本征能量变大。} \label{fig_HStark_1}
\end{figure}

不要以为\autoref{fig_HStark_1} 是外电场扭曲波函数的结果, $\ket{2\pm}$ 本身就是无电场的氢原子 $n=2$ 本征态。 施加了电场后波函数反而需要进一步修正。

从经典电磁学角度来理解, 电偶极子在电场中的能量(\autoref{eq_eleDP2_1}~\upref{eleDP2})等于 $-d_z \mathcal{E}_z$, 其中 $d_z$ 是 $z$ 方向电偶极子
\begin{equation}
d_z^{(\pm)} = \mel{2\pm}{z}{2\pm} = \pm 3
\end{equation}
\end{example}

\addTODO{如果初始时, 波函数处于 $n=2$ 子空间的任意状态, 例如 $\ket{20}$, 那么当逐渐施加电场后, 波函数会如何变化?}
