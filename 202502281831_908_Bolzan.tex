% 伯恩哈德·波尔查诺(综述)
% license CCBYSA3
% type Wiki

本文根据 CC-BY-SA 协议转载翻译自维基百科\href{https://en.wikipedia.org/wiki/Bernard_Bolzano}{相关文章}。

\begin{figure}[ht]
\centering
\includegraphics[width=6cm]{./figures/f44771c523b92573.png}
\caption{} \label{fig_Bolzan_1}
\end{figure}
伯纳德·博尔扎诺(UK: /bɒlˈtsɑːnoʊ/, US: /boʊltˈsɑː-, boʊlˈzɑː-/;德语:[bɔlˈtsaːno];意大利语:[bolˈtsaːno];原名伯纳尔杜斯·普拉西杜斯·约翰·内波穆克·博尔扎诺;1781年10月5日–1848年12月18日)是捷克数学家、逻辑学家、哲学家、神学家和天主教神父,具有意大利血统,以其自由主义观点而著称。

博尔扎诺使用德语写作,这是他的母语。[6] 大部分他的工作是在他去世后才获得广泛关注。
\subsection{家庭}  
博尔扎诺是两位虔诚天主教徒的儿子。他的父亲,伯纳德·庞培乌斯·博尔扎诺,是一位意大利人,曾移居布拉格,在那里娶了来自布拉格讲德语的毛雷尔家族的玛丽亚·凯瑟莉亚·毛雷尔。他们的十二个孩子中只有两个活到成年。
\subsection{职业生涯}  
博尔扎诺十岁时进入布拉格的皮亚尔修道士中学,期间从1791年到1796年就读于该校。[7]

博尔扎诺于1796年进入布拉格大学,学习数学、哲学和物理学。从1800年起,他开始学习神学,并于1804年成为一名天主教神父。1805年,他被任命为布拉格大学宗教哲学新设立的讲座教授。[5] 他不仅在宗教领域,甚至在哲学领域也是一位受欢迎的讲师,并于1818年被选为哲学系系主任。

博尔扎诺因他关于军事主义的社会浪费以及战争不必要性的教义,疏远了许多教职员工和教会领袖。他主张全面改革教育、社会和经济制度,旨在将国家的利益引导向和平,而非国家之间的武装冲突。他的政治信仰,虽然他经常与他人分享,最终被奥地利当局视为过于自由。在1819年12月24日,他因拒绝撤回自己的信仰而被解除教授职务,并被流放到乡村,之后将精力投入到他关于社会、宗教、哲学和数学的著作中。

尽管被禁止在主流期刊上发表文章作为流放的条件,博尔扎诺依然继续发展他的思想,并通过个人或在东欧不太知名的期刊上发表他的作品。1842年,他回到布拉格,并于1848年去世。