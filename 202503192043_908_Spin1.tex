% 自旋(综述)
% license CCBYSA3
% type Wiki

本文根据 CC-BY-SA 协议转载翻译自维基百科\href{https://en.wikipedia.org/wiki/Spin_(physics)}{相关文章}。

自旋是基本粒子所具有的一种内禀角动量形式,因此,诸如强子、原子核和原子等复合粒子也具有自旋。[1][2]: 183–184  自旋是量子化的,并且描述自旋相互作用的准确模型需要相对论量子力学或量子场论。

电子自旋角动量的存在是通过实验推断出来的,例如斯特恩-盖拉赫实验,其中银原子被观察到具有两个可能的离散角动量状态,尽管它们没有轨道角动量。[3] 相对论性的自旋-统计定理将电子自旋的量子化与泡利不相容原理联系起来:不相容性的观测结果意味着自旋为半整数,而半整数自旋的观测结果又意味着不相容性。

在数学上,自旋可以用向量来描述(例如光子),也可以用旋量或双旋量(bispinor)来描述(例如电子)。旋量和双旋量在某些方面与向量类似:它们具有确定的大小,并且在旋转下发生变化;然而,它们的“方向”采用了一种非传统的方式。所有同类的基本粒子具有相同大小的自旋角动量,尽管其方向可以变化。这些特性通过赋予粒子一个自旋量子数来表示。[2]: 183–184 

自旋的国际单位制(SI)单位与经典角动量相同(即 \(N\cdot m\cdot s\),\(j\cdot s\) 或\(kg\cdot m^2\cdot s^{-1}\))。在量子力学中,角动量和自旋角动量具有离散的值,并且它们的大小与普朗克常数成比例。在实际应用中,自旋通常通过将自旋角动量除以约化普朗克常数\(\hbar\)来表示为无量纲的自旋量子数。通常,“自旋量子数”也直接被称为“自旋”。
\subsection{模型}
\subsubsection{旋转的带电质量}  
最早的电子自旋模型设想电子是一个旋转的带电质量,但当对该模型进行详细检验时,它无法成立:所需的空间分布不符合对电子半径的限制;此外,所需的旋转速度超过了光速。[4] 在标准模型中,所有基本粒子都被视为“点状”粒子:它们的作用是通过周围的场来体现的。[5] 任何基于质量旋转的自旋模型都需要与这一观点保持一致。
\subsubsection{泡利的“经典上无法描述的二值性” } 
沃尔夫冈·泡利是量子自旋历史上的核心人物,他最初拒绝将他为解释实验观察结果而引入的“自由度”与旋转联系起来。他称之为“经典上无法描述的二值性”。后来,他接受了自旋与角动量有关的观点,但坚持将自旋视为一种抽象属性。[6] 这一方法使泡利能够推导出他的基本泡利不相容原理,该证明后来被称为自旋-统计定理。[7] 从历史的角度来看,泡利对自旋的抽象处理方式及其证明方法开启了现代粒子物理学的时代,在这个时代,由对称性推导出的抽象量子属性占据主导地位,而具体的物理解释则变得次要甚至可有可无。[6]
\subsubsection{经典场的循环}
最早的经典自旋模型假设一个围绕轴旋转的小型刚性粒子,这一设想符合“自旋”一词的日常用法。角动量也可以通过经典场来计算。[8][9]: 63  通过应用弗雷德里克·贝林方特计算场角动量的方法,汉斯·C·奥哈尼安证明了“自旋本质上是一种波动特性……由电子波场中的电荷循环流动所产生”。[10] 这一相同的自旋概念也可以应用于水中的重力波:“自旋由水粒子在亚波长尺度上的圆周运动产生”。[11]

与允许连续角动量值的经典波场循环不同,量子波场仅允许离散角动量值。[10] 因此,向自旋态的能量转移或从自旋态的能量转移总是以固定的量子阶跃方式发生。仅有少数阶跃是允许的:在许多定性讨论中,可以忽略自旋量子波场的复杂性,而只考虑系统属性,以“整数”或“半整数”自旋模型来讨论,如下文所述的量子数部分所示。
\subsubsection{狄拉克的相对论电子}
对电子的自旋性质进行定量计算需要使用狄拉克的相对论波动方程。[7]
\subsection{自旋与轨道角动量的关系 } 
顾名思义,自旋最初被设想为粒子绕某个轴的旋转。从历史上看,轨道角动量与粒子的轨道运动相关。[12]: 131  尽管基于力学模型的名称沿用至今,但其物理解释已不再适用。量子化从根本上改变了自旋和轨道角动量的性质。  

由于基本粒子是点状的,对它们而言自旋运动并没有明确的定义。然而,自旋意味着粒子的相位依赖于角度,其关系为\(e^{iS\theta}\)其中,\(\theta\)是绕与自旋\(S\)平行的轴旋转的角度。这一性质等价于量子力学中动量的相位依赖性在位置上的体现,以及轨道角动量的相位依赖性在角位置上的体现。

根据埃伦费斯特定理,角速度等于哈密顿量对其共轭动量的导数,而共轭动量即总角动量算符 \( J = L + S \)。因此,如果哈密顿量\( H \)依赖于自旋\( S \),则\(\frac{\partial H}{\partial S}\)必须非零;因此,在经典力学中,哈密顿量中自旋的存在将导致实际的角速度,并因此产生实际的物理旋转——即相位角 \( \theta \) 随时间的变化。然而,对于自由电子而言,这一观点是否成立仍存在争议。由于电子的自旋大小 \( |S|^2 \) 是常数\(\frac{1}{2} \hbar\)有人可能认为,如果它不能改变,那么其偏导数(\(\partial \))就不存在。因此,是否必须在哈密顿量中包含这一项,以及经典力学的这一特性是否延伸到量子力学,都取决于具体的解释方式。(任何粒子的内禀自旋角动量\( S \)都是一个量子数,它来源于狄拉克方程数学解中的“旋量”,而不是像轨道角动量\( L \)那样更接近物理概念的量。) 尽管如此,自旋仍然出现在狄拉克方程中,因此,将电子视为狄拉克场进行处理时,它的相对论哈密顿量可以被解释为包含对自旋\( S \)的依赖。[9]
\subsection{量子数} 
自旋遵循角动量量子化的数学规律。自旋角动量的具体特性包括:  
\begin{itemize}
\item 自旋量子数可以取半整数或整数值。  
\item 尽管自旋的方向可以改变,但基本粒子的自旋大小无法改变。  
\item 带电粒子的自旋与磁偶极矩相关,其\(g\)因子不同于 1。(在经典物理中,这意味着对于一个旋转的物体,其内部的电荷分布和质量分布不同。[4])
\end{itemize}
自旋量子数的常规定义为\(s = \frac{n}{2}\)其中,\( n \)可以是任何非负整数。因此,\( s \)的允许值为 0、\( \frac{1}{2} \)、1、\( \frac{3}{2} \)、2,依此类推。基本粒子的自旋量子数\( s \)仅取决于粒子的类型,无法以任何已知方式改变(相较之下,自旋方向可以改变,如下所述)。任何物理系统的自旋角动量\( S \)都是量子化的,其允许值为:  
\[
S = \hbar \sqrt{s(s+1)} = \frac{h}{2\pi} \sqrt{\frac{n}{2} \cdot \frac{(n+2)}{2}} = \frac{h}{4\pi} \sqrt{n(n+2)}~
\]  
其中,\( h \)是普朗克常数,\( \hbar = \frac{h}{2\pi} \)是约化普朗克常数。相比之下,轨道角动量的自旋量子数\( s \)只能取整数值,即\( n \)仅能取偶数。
\subsubsection{费米子与玻色子} 
自旋取半整数(如 \( \frac{1}{2} \)、\( \frac{3}{2} \)、\( \frac{5}{2} \))的粒子称为费米子,而自旋取整数(如 0、1、2)的粒子称为玻色子。这两类粒子遵循不同的规则,并在我们周围的世界中扮演不同的角色。两者的一个关键区别在于:费米子遵守泡利不相容原理,即两个相同的费米子不能同时具有相同的量子数(大致上,意味着它们不能占据相同的位置、速度和自旋方向)。费米子服从费米–狄拉克统计。相比之下,玻色子遵循玻色–爱因斯坦统计,不受此限制,因此可以“聚集”在相同的量子态中。此外,复合粒子的自旋可以与其组成粒子的自旋不同。例如,氦-4 原子在基态时的自旋为 0,表现为一个玻色子,尽管构成它的夸克和电子本身都是费米子。

这一特性带来了一些深远的影响: 
\begin{itemize}
\item 夸克和轻子(包括电子和中微子)构成了传统意义上的物质,它们都是自旋为 \( \frac{1}{2} \) 的费米子。人们常说“物质占据空间”,实际上是由于泡利不相容原理作用于这些粒子,使得费米子不能处于相同的量子态。进一步压缩物质会要求电子占据相同的能级,因此会产生一种类似压力的作用(有时称为电子简并压力),阻止费米子过度接近。目前尚未发现其他自旋值(\( \frac{3}{2} \)、\( \frac{5}{2} \) 等)的基本费米子。  
\item 传递相互作用的基本粒子都是自旋为 1 的玻色子,包括:光子(传递电磁相互作用) 胶子(传递强相互作用)、\(W\)和\(Z\)玻色子(传递弱相互作用)玻色子能够占据相同的量子态,这一特性在多个物理现象中起关键作用,例如:激光(使大量光子具有相同的量子数,如相同方向和频率)超流氦(由于氦-4 原子作为整体是玻色子)超导性(电子虽然是费米子,但配对后形成的\textbf{库珀对}可视为一个复合玻色子)其他自旋(0、2、3 等)的基本玻色子在历史上尚未被观测到,但它们在理论上得到了充分研究,并在相关的主流理论中被广泛接受。例如:引力子——量子引力理论预测的自旋为 2 的粒子。希格斯玻色子——解释电弱对称性破缺的粒子,自旋为 0。自 2013 年以来,实验已确认希格斯玻色子(自旋 0)确实存在。[13] 它是自然界中已知的首个标量基本粒子(自旋 0)。  
\item 原子核具有核自旋,其取值可以是半整数或整数,因此原子核可以表现为费米子或玻色子。
\end{itemize}
\subsubsection{自旋-统计定理}  
自旋-统计定理将粒子分为两类:玻色子和费米子。其中,玻色子服从玻色-爱因斯坦统计,费米子服从费米-狄拉克统计(并因此遵守泡利不相容原理)。具体来说,该定理要求:自旋为半整数的粒子必须遵守泡利不相容原理。自旋为整数的粒子不受该原理的限制。例如,电子的自旋为半整数(\(\frac{1}{2}\)),因此它是费米子,遵守泡利不相容原理;而光子的自旋为整数(1),因此它是玻色子,不受该原理限制。该定理由沃尔夫冈·泡利于1940年推导出,它基于量子力学和狭义相对论。泡利曾将自旋与统计之间的联系描述为“狭义相对论理论最重要的应用之一”。[14]