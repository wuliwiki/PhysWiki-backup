% Julia 解释器笔记

\begin{issues}
\issueDraft
\end{issues}

本文是关于 Julia 解释器的原理: 它使用了哪些技术, 可以使得它作为一门动态语言能达到编译语言的性能。

\begin{itemize}
\item \textbf{JIT} 编译器: 所有的代码都经过 JIT 编译, 官方称为 \textbf{just-ahead-of-time}。
\item \textbf{Multiple Dispatch (MD)}: 可以基于参数的类型生成不同的专门代码。
\item \textbf{Type Inference}: 自动推导变量类型。
\item 内建并行, 包括分布计算
\item 高效内存管理: 减少内存分配, 增加内存重复利用。
\item 基于 \textbf{LLVM}: Julia 编译器先把 Julia 语言变为 Julia IR (和 LLVM IR 相似), 再变为 LLVM IR\upref{llvmIR}, 最后交给 LLVM 进行优化。
\item Julia IR 使用 \textbf{Single Static Assignment (SSA)} 形式, 把 julia 代码的控制流表示为 \textbf{directed acyclic graph (DAG)}。 Julia IR 中的变量类型都是明确的。
\item \verb|LLVM.jl| 包可以把 julia 代码直接生成 LLVM IR, 或者把 Julia IR 转为 LLVM IR。
\item 一些预备知识: 编译原理, 编译器设计, LLVM, Julia 背后原理。
\end{itemize}

\subsection{把 Julia 代码编译成 LLVM IR 代码}
首先安装 \verb|using Pkg; Pkg.add("LLVM")|
\begin{lstlisting}[language=julia]
using LLVM
function add(x::Int, y::Int)::Int
    return x + y
end
llvm_ir = @code_llvm add(1, 2) # 生成 LLVM IR
\end{lstlisting}
生成的代码如下
\begin{lstlisting}[language=none]
;  @ REPL[3]:1 within `add'
define i64 @julia_add_652(i64 signext %0, i64 signext %1) {
top:
;  @ REPL[3]:2 within `add'
; ┌ @ int.jl:87 within `+'
   %2 = add i64 %1, %0
; └
  ret i64 %2
}
\end{lstlisting}
所以上面参数 \verb|(1,2)| 的作用是提供自变量的类型。
