% 张量代数(张量)
% keys 张量代数|对称代数
% license Xiao
% type Tutor

\begin{issues}
\issueOther{子节3相应位置缺多项式代数的链接}
\end{issues}

% Giacomo: 张量代数是张量幂的直和,与共变逆变无关,这篇文章应该是讲 $T(V)$ 而不是 $T(V^*)$ 的。
% 预备知识应该是:张量积 和 直和

\pentry{直和(线性空间)\nref{nod_DirSum},张量的对称化和交错化\nref{nod_SIofTe}}{nod_fbf8}
这一节将用 $\mathbb T_1^0(V)$ 来构造一个代数 $\mathbb T(V^*)$,其上的乘法为张量积。之所以用 $\mathbb T(V^*)$ 而不用 $T(V)$ 是因为 $\mathbb T_1^0(V)=V^*$,暗示着 $\mathbb T(V^*)$ 的每一子空间 $\mathbb T_p^0(V)$ 是 $p$ 个 $V^*$ 的张量积 。 

如下文所示,所寻求的代数为
\begin{equation}
\mathbb T(V^*)=\mathbb F\oplus\mathbb T_1^0(V)\oplus\mathbb T_2^0(V)\oplus\cdots~
\end{equation}
或记
\begin{equation}
\mathbb T(V^*)=\bigoplus_{p=0}^\infty\mathbb T_p^0(V)~.
\end{equation}
\subsection{共变张量代数}
由矢量空间的\enref{张量积}{TPofSp}知道, $\mathbb T_p^0(V)$ 上矢量和 $\mathbb T_q^0(V)$ 上矢量的张量积所在的空间为  $\mathbb T_{p+q}^0(V)$。所以要 $\mathbb T_1^0(V)$ 构造的代数其乘法为张量积,那么该代数需包含矢量空间 $\mathbb T_2^0(V)$ 。于是该代数需包含子空间 $\mathbb T_1^0(V)\oplus\mathbb T_2^0(V) $

 考虑到张量积的单位元为 $1\in\mathbb F$,所以 1在该代数上,而代数的矢量空间性质要求 $\mathbb F$ 也在该代数上。所以该代数需包含子空间
\begin{equation}
\mathbb F\oplus\mathbb T_1^0(V)\oplus\mathbb T_2^0(V) ~.
\end{equation} 
同理,继续将该矢量空间上进行张量积,可得该代数包含子空间
\begin{equation}
\mathbb F\oplus\mathbb T_1^0(V)\oplus\mathbb T_2^0(V)\oplus\mathbb T_3^0(V)\oplus\mathbb T_4^0(V)~.
\end{equation}
重复这一过程,便得所需的代数为
\begin{equation}
\mathbb F\oplus\mathbb T_1^0(V)\oplus\mathbb T_2^0(V)\oplus\cdots~
\end{equation}
由\enref{直和}{DirSum}的性质知道,该代数上的矢量 $f$ 可记作
\begin{equation}\label{eq_TenAlg_5}
f=\sum_{i=0}^\infty f_i=(f_0,f_1,f_2,\cdots) \quad f_i\in\mathbb T_i^0(V)~,
\end{equation}
其中 $\mathbb T_0^0(V)=\mathbb F$。

设 $f=\sum\limits_ i f_i,g=\sum\limits_ i g_i$。显然,该代数上的\textbf{加法}为
\begin{equation}\label{eq_TenAlg_1}
f+g=\sum_{i}f_i+g_i=(f_0+g_0,f_1+g_1,\cdots)~,
\end{equation}
\textbf{乘法}便是
\begin{equation}\label{eq_TenAlg_2}
f\otimes g=\sum_{i,j}f_i\otimes g_j=\sum_k h_k~,
\end{equation}
其中 $h_k=\sum\limits_{i=0}^k f_i\otimes g_{k-i}$。

乘法\autoref{eq_TenAlg_2} 的结合性和纯量与张量积的乘法定律直接由张量积的运算性质(\autoref{the_TsrPrd_1}~\upref{TsrPrd})得到。

\begin{definition}{共变张量代数}
称代数
\begin{equation}\label{eq_TenAlg_3}
\mathbb T(V^*)=\mathbb F\oplus\mathbb T_1^0(V)\oplus\mathbb T_2^0(V)\oplus\cdots~
\end{equation}
为矢量空间 $V$ 上的\textbf{共变张量代数},其上的加法和乘法分别由\autoref{eq_TenAlg_1} ,\autoref{eq_TenAlg_2} 定义。
\end{definition}

容易知道,代数 $\mathbb T(V^*)$ 是个无穷维结合代数。

正如多个 “+” 的求和用 $\sum$ 表示,多个 “$\oplus$” 的直和用 “$\bigoplus$” 表示。所以\autoref{eq_TenAlg_3} 可写为
\begin{equation}
\mathbb T(V^*)=\bigoplus_{p=0}^\infty\mathbb T_p^0(V)~.
\end{equation}
 
\subsection{反变张量代数}
同样的方法可构造 $\mathbb T^1_0(V)$ 上的代数 $\mathbb T(V)$(乘法为张量积)。
\begin{definition}{反变张量代数}
称代数
\begin{equation}\label{eq_TenAlg_4}
\mathbb T(V)=\mathbb F\oplus\mathbb T_0^1(V)\oplus\mathbb T_0^2(V)\oplus\cdots~
\end{equation}
为矢量空间 $V$ 上的\textbf{反变张量代数}。
\end{definition}

同样,\autoref{eq_TenAlg_4} 可记为
\begin{equation}
\mathbb T(V)=\bigoplus_{i=0}^\infty \mathbb T_0^p(V)~.
\end{equation}

\subsection{其它说明}
由于 $\mathbb T_p^+(V)$ 和 $\Lambda^P(V^*)$ 分别是 $\mathbb T_p^0(V)$ 上的对称和反对称的子空间,那么
\begin{equation}
\begin{aligned}
&\mathbb T^+(V^*)=\mathbb F\oplus\mathbb T_1^+(V)\oplus\mathbb T_2^+(V)\oplus\cdots\\
&\Lambda(V^*)=\mathbb F\oplus \Lambda^1(V^*)\oplus\Lambda^2(V^*)\oplus\cdots
\end{aligned}~
\end{equation}
是 $\mathbb T(V^*)$ 上的子空间。但它们并不是 $\mathbb T(V^*)$ 的子代数,因为(反)对称张量的张量积一般不再是(反)对称张量,也就是不满足封闭性条件。这就是说,可以在它们上引进一个运算,使其各自成为一个结合代数。对对称子空间的外直和 $\mathbb T^+(V^*)$,它就是通常的多项式代数(链接);对斜对称张量的外直和 $\Lambda(V^*)$,参见\enref{外代数}{ExtAlg}。
