% 导数(高中)
% keys 导数|高中|变化|求导
% license Usr
% type Tutor

\begin{issues}
\issueDraft
\end{issues}

\subsection{理解导数}

\subsection{导数的定义}
一点的\textbf{导数}
\begin{equation}
f'(x_0)=\lim_{x_1\to x_0}{f(x_1)-f(x_0)\over x_1-x_0}~.
\end{equation}

导数也是一个对应关系,即每个自变量都对应一个导数,因此他也是一个函数,这个函数称为\textbf{导函数}(不引起歧义时,简称为导数)。导函数和原本的函数是一一对应的,因此可以根据定义或求导方法,来求一个函数的导函数,这个过程就是\textbf{求导}。

\subsubsection{记号}
由于历史发展和人们长久以来的使用习惯,导数这个极为常见的数学运算逐渐衍生出了许多不同的记法。这些记法不仅仅是使用者的偏好选择,还与特定领域的需求和表达习惯密切相关。既为方便计算和推导,也为强调不同的数学概念。了解这些符号的使用,有助于理解导数这个运算,另外在未来见到时,也不至陌生,不要求完全掌握,看个眼熟就好。下面的符号针对函数$y=f(x)$:
\begin{itemize}
\item 拉格朗日记法——$y'$或$f'(x)$,好处是记法比较简洁,便于书写,缺点是难以表达较为复杂的关系。高中数学主要采用这种表示法。
\item 莱布尼茨记法——$\displaystyle\frac{\dd y}{\dd x}$  或  $\displaystyle\frac{\dd}{\dd x}f(x)$,好处是在进行某些复杂运算时,分子与分母可以直接按照乘除法的规则来进行运算,降低推导的复杂度。另外,也在形式上代表着变化率。在大学阶段的数学领域主要采用这种表示方法。
\item 牛顿记法——$\dot{y}$。由于在物理学中,时间是一个较为特别的变量,一般用这种方法来表示某个变量相对于时间的导数。基本只在物理学领域使用。
\item 重导数记法——$f_x$,这种记法简洁紧凑,又在出现复杂关系时,避免了拉格朗日记法的问题。在偏微分方程和张量分析中常用。
\item 欧拉记法——$Df(x)$,采用$D$运算符。主要在大学阶段的微分方程中使用,好处是$D^n f(x)$可以直接修改$n$来表示进行几次求导运算。另外,将导数视为一种运算符,便于与其他运算符进行组合,适合处理复杂的微分运算。
\item 差分导数——$\Delta f(x)$,对于定义域是离散的函数(一般是$\mathbb{Z}$),通常会用这样的符号来表示它的导数,称为\textbf{差分}。
\end{itemize}

在高中阶段,一般只要求使用拉格朗日记法,并且不建议使用其他记法。其余的记法可以这样理解,$Df(x)$用$D$运算符代替了$\displaystyle\frac{\dd}{\dd x}f(x)$中的$\displaystyle\frac{\dd}{\dd x}$,$f_x$则是把$\displaystyle\frac{\dd}{\dd x}f(x)$中最重要的两部分$f,x$拿出来显示。$\Delta f(x)$与$Df(x)$只是因为定义域不同,处理方法不同。$\dot{y}$和$y'$异曲同工,只是更着眼于“时间”。

\subsection{求导法则}

为记录方便,下面记$u=f(x),v=g(x),u'=f'(x),v'=g'(x)$。

\begin{itemize}
\item 加减法:$(u\pm v)'=u'\pm v'$
\item 乘法:$(uv)'=u'v+uv'$
\item 除法:$\displaystyle\left(\frac{u}{v}\right)'=\frac{u'v-uv'}{v^2}$
\item 复合函数:$(f(v))'=f'(v)v'$
\end{itemize}

\subsection{基本初等函数的导数推导}

\subsection{对照表}

这里将常见的函数与导数对照表列出如下,方便查询。具体介绍需查看每个函数自己的页面。

\begin{table}[ht]
\centering
\caption{高中常见函数及其导数}\label{tab_HsDerv1}
\begin{tabular}{|c|c|c|}
\hline
\textbf{函数名称}     & \textbf{函数 $f(x)$}     & \textbf{导函数 $f'(x)$}     \\ \hline
幂函数&$x^n$                    & $n x^{n-1}$                \\ \hline
反比例函数&$\displaystyle\frac{1}{x}$             & $\displaystyle-\frac{1}{x^2}$           \\ \hline
指数函数(e为底)&$e^x$                     & $e^x$                      \\ \hline
对数函数(e为底)&$\ln(x)$                  & $\displaystyle\frac{1}{x}$              \\ \hline
指数函数&$a^x$                     & $a^x\ln a $                      \\ \hline
对数函数&$\log_a(x)$                  & $\displaystyle \frac{1}{x\ln a}$              \\ \hline
正弦函数&$\sin(x)$                 & $\cos(x)$                  \\ \hline
余弦函数&$\cos(x)$                 & $-\sin(x)$                 \\ \hline
正切函数&$\tan(x)$                 & $\displaystyle \frac{1}{\cos^2(x)}$                \\ \hline
\end{tabular}
\end{table}

