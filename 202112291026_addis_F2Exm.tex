% 二元函数的极值(简明微积分)
% 多元微积分|二元函数|导数|方向导数|极值

\pentry{导数与函数极值\upref{DerMax}, 方向导数\upref{DerDir}}

\footnote{本文参考: \cite{同济高} 下册的 “多元函数的极值及其求法” 一节.}类似一元函数, 二元函数的极值与其偏导数密切相关. 以下讨论中, 我们假设在某区域内二元函数的一阶偏导处处存在(即函数曲面处处光滑). 如果二元函数 $f(x,y)$ 在某点 $(x_i, y_i)$ 处对 $x, y$ 的偏导数都为零, 那么 $(x_i, y_i)$ 就叫做函数 $f(x,y)$ 的\textbf{驻点}. 根据\autoref{DerDir_eq7}~\upref{DerDir}, 驻点处各个方向的方向导数\upref{DerDir}也都为零.

我们先来定义二元函数的\textbf{极值点}, 以驻点为圆心在 $xy$ 平面上作一个圆形区域, 若当半径足够小时, $f(x_i, y_i)$ 是该圆形区域的最大值或最小值, 那么该驻点就是极大值点或极小值点. 与一元函数类似, 驻点不一定是极值点. 例如 $f(x,y) = xy$ 在坐标原点的两个一阶偏导都为零, 但原点并不是极值点. 为了判断驻点是不是极值点, 也需要用到二阶偏导(假设驻点处的各个二阶偏导都存在). 如果满足
\begin{equation}\label{F2Exm_eq1}
\pdv[2]{f}{x}\pdv[2]{f}{y} - \qty(\pdv{f}{x}{y})^2 > 0
\end{equation}
则驻点是极值点. 如果 $\pdv*[2]{f}{x}$ 和 $\pdv*[2]{f}{y}$ 都大于零\footnote{根据\autoref{F2Exm_eq1}, $\pdv*[2]{f}{x}$ 和 $\pdv*[2]{f}{y}$ 的乘积大于零, 所以只需要任意一个大于零, 另外一个就必定大于零. 一个小于零, 另一个也必小于零.}, 则极值为极小值, 若都小于零, 则极值为极大值.

注意\autoref{F2Exm_eq1} 只是存在极值的充分非必要条件\upref{SufCnd}. 也就是说存在一些极值点不满足\autoref{F2Exm_eq1}. 例如 $f(x, y) = x^4 + y^4$ 在原点处的极值点. 当\autoref{F2Exm_eq1} 左边小于零时, 必定不是极值点, 等于零时可能是也可能不是, 需要用高阶导数进一步判断, 这里暂时不讨论.

\subsection{推导}

类比一元函数的证明, 要证明二元函数的某点是极值点, 就要证明该点的任意二阶方向导数都大于零或都小于零\footnote{否则延一个方向前进函数值会越来越大, 而延另一个方向前进函数值会越来越小, 这个点就不是极值点}. 令某方向为 $\uvec n = \uvec x \cos\theta + \uvec y \sin\theta$, 由\autoref{DerDir_eq7}~\upref{DerDir} 得该方向的方向导数为
\begin{equation}
\qty(\cos\theta\pdv{x} + \sin\theta\pdv{y}) f
\end{equation}
再次求方向导数得二阶方向导数为
\begin{equation}
\qty(\cos\theta\pdv{x} + \sin\theta\pdv{y})^2 f
= \pdv[2]{f}{x} \cos^2\theta + 2\pdv{f}{x}{y} \sin\theta\cos\theta + \pdv[2]{f}{y} \sin^2\theta
\end{equation}
如果你还不习惯看算符的平方, 可以把上式的括号项平方看做两个括号项, 依次作用在函数上. 以极小值为例, 令上式恒大于零, 并除以 $\cos^2\theta$ 得
\begin{equation}
\pdv[2]{f}{y} \tan^2\theta + 2\pdv{f}{x}{y} \tan\theta + \pdv[2]{f}{x} > 0
\end{equation}
上式左边是关于 $\tan\theta$ 的二次函数, 若要恒大于零, 则二次项系数要大于零, 且判别式需小于零, 立即可得\autoref{F2Exm_eq1} . 同理可得极大值条件.

当判别式(\autoref{F2Exm_eq1} )小于零时, 必然存在不同方向的二阶方向导数具有相反的符号, 所以必定不是极值点. 而当判别式等于零时, 存在某些方向的二阶导数为零, 无法判断是否为极值点.
