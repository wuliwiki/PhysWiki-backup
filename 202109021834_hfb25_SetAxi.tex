% 集合(公理化)
% 集合|公理|公理系统
\begin{issues}
\issueTODO
\end{issues}


\pentry{集合\upref{Set}}
\subsection{产生原因}
通常来说,我们都采取朴素集合论的观点,认为集合是一个最基本的数学概念,不需要严格的定义.但是,\textbf{罗素悖论(antinomy of Russell)}使得这一观念受到挑战.罗素悖论可以用集合的语言叙述为:是否存在一个集合$A=\{x|x\notin A\}$?

显然,如果认为$x\in A$,那么根据定义,$x\notin A$;反过来,如果认为$x\notin A$,根据定义又有$x\in A$.