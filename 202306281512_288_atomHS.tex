% 原子结构和波粒二象性(高中)
% keys 必修三能量子|普朗克黑体辐射|光电效应|原子核式结构|玻尔模型|粒子波动性|量子力学初步

\begin{issues}
\issueTODO
\end{issues}

%\pentry{功和机械能\upref{HSPM07},分子动力学\upref{thermo} }

\subsection{普朗克黑体辐射理论}
\subsubsection{黑体辐射}
在了解什么是黑体辐射之前,必须先明确黑体的定义。\textbf{黑体}指的是一种可以完全吸收入射的各种波长的电磁波而不将其反射的物体。我们常认为一个带有小孔的空腔可以作为一个黑体,因为如果有电磁波从小孔入射,在其中不管发生了多少次的反射和折射,都很难从空腔再度射出,因此满足对于黑体的定义。

一个常见的误区是,黑体虽然不能\textsl{反射},却仍然可以向外\textsl{辐射}电磁波,这是由于黑体在吸收能量的时候具有一定温度。这种辐射称之为\textbf{黑体辐射}。更进一步的,对于一般材料的热辐射,不仅仅和所选材料的温度有关,还和材料的种类以及表面状态相关,但是在黑体的情况下,黑体辐射的电磁波强度按照波长的分布仅仅和黑体的温度有关。

\subsubsection{黑体辐射的实验规律}
利用现代化设备,可以测出黑体辐射电磁波的强度按波长分布情况。实验表明,随着温度的升高,各个波长的辐射强度都有增加,这是符合我们直观认识的;与此同时,辐射强度的极大值会向着波长较短的方向进行移动。

这种现象该如何从微观上去进行解释呢?我们知道,物体存在着不停运动的带电微粒,每个带电微粒的振动都会产生变换的电磁场,这是物体电磁辐射的来源,而微粒的运动又和热密切相关,因此即可把温度和电磁辐射相互关联起来。德国物理学家维恩和英国物理学家瑞利分别提出了辐射强度按波长分布的理论公式。维恩公式在短波区和实验非常接近,但是在长波区域偏差较大;瑞利的结果在长波区和实验接近,但是在短波区会预测处无限大的辐射强度,称之为\textbf{紫外灾难}。显然和实验结果是不符合的。

那有没有换一种和全部实验结果相符合的理论公式呢?德国物理学家普朗克刚提出了普朗克公式解决了这一问题,普朗克公式可写成如下形式:
$$u(\lambda)=\dfrac{8\pi hc}{\lambda^5}\cdot\dfrac{1}{e^{hc/\lambda k_BT}-1}~.$$其中,$u(\lambda)$是能流密度,$h$是普朗克常量,$k_B$是玻尔兹曼常数。

\subsubsection{能量子}
从普朗克公式即可看出,普朗克引入了一个在经典物理中从未涉及过的常量$h$到对于黑体辐射的解释中,那么,对应于这一个新常量,是否有一些新的物理特性呢?答案是肯定的,如果想要得到这个公式,就需要先假定组成黑体的振动带电微粒所具有的能量并不是连续的,而是某一个最小能量单元$\epsilon_0$的整数倍。这个最小的能量单元称之为\textbf{能量子},表达式可以写作$$\epsilon_0=h\nu ~,$$其中$\nu$是带电微粒的振动频率,也即带电微粒吸收或者辐射电磁波的频率。另外,该公式也将能量子和我们新引入的常量也联系在了一起,$h=6.62607015\times 10^{-34}\Si{J\cdot s}$,是普朗克常量。

这个假设和我们的日常经验十分不同,在日常观察中,以动能为例,如果想让一个确定的物体具有更大的动能,则需要稍微增加一点它的速度,而速度是连续的,因此该物体所具有的动能也应该是连续的。但是这些认知在宏观世界才成立,等物理学深入到微观的量子世界,根据普朗克的假设,只有微观粒子的能量是分立的,也即所谓\textbf{量子化}的,才能解释微观世界的诸多现象。能量是否连续,也是微观和宏观世界物理规律最重要的区别之一。从观测的角度来说,在宏观世界中,即使每一个组成宏观物体的微观微粒能量都是分立的,但是由于数目极其巨大,每一个能量子的微粒比较于宏观的能量又极其微小,因此在观测中仍然会表现为连续的能量。

% 补充图片 黑体辐射的实验规律 P68
\subsection{光电效应}
\subsection{原子核式结构}
\subsection{波尔模型}
\subsection{粒子波动性}
\subsection{量子力学初步}

%% 画图时间
% 
%% 错别字纠正
% 