% 两体问题(用分析力学方法求解)
% keys 两体问题|开普勒问题|卢瑟福散射

\pentry{拉格朗日方程\upref{Lagrng}}

两体问题研究的对象是两个可以看成质点的物体,质量分别为 $m_1,m_2$,位矢分别为 $\bvec r_1,\bvec r_2$.它们之间的相互作用势是 $V(r)=V(|\bvec r_1-\bvec r_2|)$,也就是说只和两者的距离有关.开普勒问题\upref{CelBd}、卢瑟福散射\upref{RuthSc} 都属于两体问题.

在这个词条中我们将用分析力学的方法来解决两体问题.设 $\bvec r_c$ 为质心位置,设 $\bvec r$ 为它们的相对位置,那么有
\begin{equation}
\left\{
\begin{aligned}
&\bvec r_c=\frac{m_1\bvec r_1+ m_2\bvec r_2}{m_1+m_2}\\
&\bvec r=\bvec r_2-\bvec r_1
\end{aligned}
\right.
\end{equation}
那么体系的动能为质心动能加上系统相对于质心参考系的动能.
\begin{equation}
\begin{aligned}
T&=\frac{1}{2}m_1 \dot{\bvec r_1}^2+\frac{1}{2}m_2 \dot{\bvec r_2}^2\\
&=\frac{1}{2}(m_1+m_2)\dot{\bvec r_c}^2+\frac{1}{2}\frac{m_1m_2}{m_1+m_2}\dot{\bvec r}^2\\
&=\frac{1}{2}M \dot{\bvec{\bvec r_c}}^2+\frac{1}{2}\mu \dot{\bvec{\bvec r}}^2
\end{aligned}
\end{equation}