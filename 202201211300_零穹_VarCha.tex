% 变分的变换
% 拉格朗日变换|黎曼变换

\pentry{变分\upref{Varia}}
在变分\upref{Varia}一节中,我们得到泛函 
\begin{equation}
J(y)=\int_a^bF(x,y,y')\dd x
\end{equation}
变分的表达式\autoref{Varia_eq5}~\upref{Varia},即
\begin{equation}\label{VarCha_eq1}
\delta J=\int_a^b\qty[F_y(x,y,y')\delta y+F_{y'}(x,y,y')\delta y']\dd x
\end{equation}
利用分部积分\autoref{IntBP_eq1}~\upref{IntBP},可以将上式积分号下只表达为 $\delta y$ (\textbf{拉格朗日变换})或 $\delta y'$ (\textbf{黎曼变换})的线性函数\autoref{LinMap_def1}~\upref{LinMap}.这就是这里说的\textbf{变分的变换}.

若在点 $a$ 及 $b$ 上,$\delta y=0$,那么:
\begin{enumerate}
\item \textbf{拉格朗日变换:}
\begin{equation}\label{VarCha_eq3}
\delta J=\int_a^b\qty(F_y-\dv{}{x}F_{y'})\delta y\dd x
\end{equation}
\item \textbf{黎曼变换:}
\begin{equation}
\delta J=\int_a^b\qty(F_{y'}-N)\delta y'\dd x ,\quad where \;N=\int_a^bF_y\dd x
\end{equation}
\end{enumerate}


注意,在引出\autoref{VarCha_eq1} 时,仅假定 $y(x)$ 是 $C_1$ 类的,即 $y(x)$ 具有连续微商 $y'(x)$, 而没有假定 $y'(x)$ 可微分.这将使我们看到,拉格朗日变换是不合法的,取而代之的将是黎曼变换. 

\subsection{证明}
\subsubsection{拉格朗日变换}
利用分部积分,有
\begin{equation}\label{VarCha_eq2}
\int_a^bF_{y'}\delta y'\dd x=\qty[F_{y'}\delta y]_a^b-\int_a^b\delta y\dv{}{x}F_{y'}\dd x
\end{equation}
这里使用了\autoref{Varia_eq6}~\upref{Varia}
\begin{equation}
\delta y'=\overline{y}'-y'=(\overline{y}-y)'=(\delta y)'
\end{equation}
当 $\delta y$ 在 $a,b$ 处为0时,\autoref{VarCha_eq2} 第一项为0,代入\autoref{VarCha_eq1} 即得\autoref{VarCha_eq3} .

\tetxbf{但是},在进行分部积分时,
\subsubsection{黎曼变换}
