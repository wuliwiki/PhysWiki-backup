% 多项式的根式解
% Galois理论|求根公式|五次方程|5次方程|Galois群


\pentry{Galois扩张\upref{GExt}}

\subsection{古典数学难题}

代数方程的根式解表达,是古典代数学中的经典难题.根式解即用任意多项式的系数进行有限次加、减、乘、除以及开根号运算,得到其根.比如,二次多项式$ax^2+bx+c$的根总可以表达为$\frac{-b\pm\sqrt{b^2-4ac}}{2a}$.

得到二次多项式求根公式的方法很简单,配方即可.

三次多项式的求根则复杂许多.

1494年,意大利数学家卢卡·帕乔利(Lusa Pacioli)发表了他近三十年心血的结晶,《算术、几何、比及比例概要》(\textsl{Summa de arithmetica, geometria, Proportioni et proportionalita})\footnote{关于这本书的介绍,可参考https://en.wikipedia.org/wiki/Summa_de_arithmetica.},有时也被翻译为《数学大全》或者《算术大全》.在这本书里,他列出了两种无法解出的三次代数方程:
\begin{equation}\label{PlyRtS_eq1}
n=ax+bx^3
\end{equation}
\begin{equation}
n=ax^2+bx^3
\end{equation}

但是就在约一个世纪后,一个名叫希皮奥内·德尔·费罗(Scipione del Ferro)的意大利数学家就发现了\autoref{PlyRtS_eq1} 的解法.这个人很“自闭”,他不喜欢公开交流思想,只喜欢和自己的朋友或学生交流——这大概就是为什么没多少人记得他.所幸,费罗在三次方程求根公式上的成果被记录在他的笔记本上,在他1526年去世后由女婿哈尼瓦·纳威(Hannival Nave)继承了,这位女婿也是个数学家.

戏剧的是,在费罗去世之前,还秘密将\autoref{PlyRtS_eq1} 的解法传给了他的学生,安东尼奥·玛丽亚·菲奥利(Antonio Maria del Fiore).目前英文维基上都没有此人的介绍\footnote{参考资料:https://es.wikipedia.org/wiki/Antonio_Maria_del_Fiore(意大利语),https://second.wiki/wiki/antonio_maria_del_fiore.他的名字Fiore也可写作Fior , Fióre ,Flòrido 或者 Floridus.},而现代数学科普书《代数的历史:人类对未知量的不舍追踪》中对菲奥利的介绍只是“威尼斯人”和“\textbf{数学才能平庸之辈”}.与其说他是个数学家,倒不如说是个客观上促进了代数变革的\textbf{商人}.

拿到方程解的秘密后,他开始琢磨怎么捞钱.当时的意大利北部有的是营销的机会,因为学者们很难得到赞助,大学薪水不太理想,还没有终身教职制度,大家巴不得有路子来宣传自己.

于是又一个人物登场了:尼科洛·塔尔塔利亚(Niccolò Tartaglia).塔尔塔利亚13岁遭遇法国军队屠杀,活是活了下来,但是下颌严重创伤,从此变得口吃——所以人们叫他“Tartaglia”——口吃的人.没错,那个时代就这样,外号也能变成姓氏.1530年,塔尔塔利亚开始和一个数学老师交流他关于三次方程的一些成果.菲奥利不知道怎么听说了这些消息,也不知道谁给他的自信——可能是老师留下的秘籍给了他自信?——总之,他向塔尔塔利亚发出挑战





















