% Julia 第11章 小结
% keys 第11章 小结

本文授权转载自郝林的 《Julia 编程基础》. 原文链接:\href{https://github.com/hyper0x/JuliaBasics/blob/master/book/ch11.md}{第 11 章 流程控制}.


\textbf{11.7 小结}

在本章,我们主要讲的是控制Julia程序的执行流程的基本方式.这包括,最简单的并列表达式和\verb|begin|代码块、可以在一定的条件下执行代码的\verb|if|语句、可以对一些对象进行迭代的\verb|for|语句、可以重复地执行某段代码的\verb|while|语句,以及比较纯粹但在局部变量的定义上很有特点的\verb|let|语句.还有,我们在最后详细阐释的\verb|try|语句.

这些代码块各自都有很鲜明的特点,并且大都也有自己的特殊编写规则.比如,虽然\verb|if|语句不会自成一个作用域,但我们若想在之后正常地访问到其中的变量,就要确保那个变量在每一个分支中都有定义.又比如,一条\verb|for|语句可以同时迭代多个对象,这与使用多条嵌套在一起的\verb|for|语句分别迭代多个对象存在着一些细节上的差异,并且各有千秋.还比如,我们需要在编写\verb|while|语句的时候特别注意死循环的问题,而且在大多数情况下都需要使用\verb|continue|语句和\verb|break|语句对它的执行流程进行干预.等等.

有了上述的这些代码块,再加上之前讲过的各种程序定义,我们就可以去编写相对高级一些的Julia程序了.但是这还不够.要想编写出完整度高、重用性强、模块化的应用程序,我们还必须学会编写函数.虽然我们在之前已经见过函数很多次了,并且也一起编写过一些函数,但是那并不成体系.在下一章,我会为你系统化地讲述Julia中的函数.