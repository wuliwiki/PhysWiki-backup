% Ehrenfest 定理
% keys 量子力学|Ehrenfest
% license Usr
% type Wiki


\pentry{量子力学中的基本算符\nref{nod_ecd8}}{nod_d4e4}
\begin{theorem}{Ehrenfest 定理}
\end{theorem}
\begin{equation}
\frac{\partial\langle A\rangle}{\partial t}=\left\langle\frac{\partial A}{\partial t}\right\rangle+\frac{i}{\hbar}\langle[H, A]\rangle~.
\end{equation}
\begin{issues}
\issueTODO 定理的推导
\end{issues}

如果可观察量$A$不随时间演化。且与哈密顿算符对易:
\begin{equation}
\frac{\partial A}{\partial t}=0,\qquad[H, A]=0~,
\end{equation}
那么$A$的期望值不随时间演化,物理上一个可观测量$A$,如果满足
\begin{equation}
\frac{\partial\langle A\rangle}{\partial t}=0~,
\end{equation}
被称为守恒量。