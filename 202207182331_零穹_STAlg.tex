% 结构张量代数
% 代数|结构张量|结构常数
\pentry{张量的坐标变换\upref{TrTnsr},域上的代数\upref{AlgFie}}
在线性算子代数\upref{LiOper} 一节提到,代数是一个同时是个环的矢量空间,或者由代数的定义直接获得.那么,要使一个矢量空间构成一个代数,就得赋予矢量空间环的特性,即任意二矢量可进行乘法运算且该乘法对加法满足分配律.由于矢量都可由一组基表示,那么任意二矢量可作乘法及对加法满足分配律的要求,就变成只需规定基矢量之间的乘法.由运算的封闭性,作乘法得到的矢量仍能用基表示,这样的基矢量之间的乘法得到的矢量在该组基下的坐标就称为\textbf{结构常数},这样只需要求乘法满足结合律,矢量空间便是一个环了,于是就将矢量空间构造成了一个代数.可以证明,结构常数是某一个 $(2,1)$ 型张量的坐标,这个张量就称为\textbf{结构张量}. 

一句话来说就是:结构张量使得一个矢量空间具有了代数结构.

\subsection{结构常数}
设 $V$ 是域 $\mathbb F$ 上的矢量空间,$\{e_i\}$ 是它的一个基.那么任意的元都可表示成 (使用爱因斯坦求和约定\upref{EinSum})
\begin{equation}
x^i e_i
\end{equation}
的形式.为使任意二矢量能进行乘法运算,且乘法对加法满足分配律,那么只需规定
\begin{equation}
e_i*e_j=\gamma_{ij}^k e_k
\end{equation}
