% GDB 笔记(C++)

\pentry{调试 C++ 程序\upref{gdbcpp}}

\subsection{笔记}
\begin{itemize}
\item \verb|g++/gcc| 编译时用 \verb`-g` 选项以用于 debug
\item \verb|r|, \verb|p|, \verb|f|, \verb|s|, \verb|n|, \verb|ptype| (检查 typename 或者变量的类型)
\item \verb|catch throw| 可以在任何异常发生时暂停
\item 要 debug 某程序, 用 \verb`gdb <exename>`, 无需 \verb`./`. 也可以先进入 \verb`gdb` 然后用 \verb`file <exename>`
\item \verb`gdb -w` 开启 gdb 自带 gui, 如果存在. 强烈推荐 \verb`guigdb` 项目(见我的相关笔记), 可以在浏览器中显示一个 gdb 的 gui, 也支持 intel 的 \verb`gdb-ia`
\item \verb`gdb -tui` 或 \verb`gdbtui` 开启 Text User Interface, 可以同时显示代码, 汇编, registers, gdb 命令等等.
\item gdb 中的命令可以被缩写为前几个字母, 如果没有歧义
\item \verb`quit` or \verb`q` 退出 gdb
\item \verb`catch throw` 命令可以在 (c++) 程序调用 throw 的时候 break
\item \verb`break <line#>`  (快捷键 \verb`b <line#>`) 指定主文件的行号设置 breakpoing
\item \verb`break <file>:<line#>` 指定文件和行号设置 breakpoing
\item \verb`break <subroutine>` 在函数体第一行设置 breakpoing
\item 设置 breakpoint 的条件如 \verb`break <line#> if i == 99`
\item 给指定编号的 breakpoint 加条件如 \verb`cond 3 i == 99`
\item \verb`info breakpoints` 显示所有 breakpoint (快捷键 \verb`info b`)
\item 如果 breakpoint 被设置在不能执行的行, 那么遇到 breakpoint 后将停在下一个可执行的行
\item \verb`run` or \verb`r` 运行程序, 遇到 breakpoint 为止
\item 如果程序需要用 \verb`<` 输入 input 文件, 打开 gdb 的时候不能指定 input 文件, 而是在 gdb 内使用 \verb`run < inputfle`.
\item \verb`next` (快捷键 \verb`n`)运行到下一行 
\item \verb`step` (快捷键 \verb`s`) 进入函数
\item \verb`print <var/expr>` (快捷键 \verb`p`) 来显示某个变量或执行某个命令, 只有在没有被优化掉的函数才可以在调试时执行, 所以为了使函数可以被执行, 可以把函数定义放到单独的 cpp 文件中, 而不要用 \verb`inline`
\item \verb`print` 调用的函数不能有默认变量, 函数 overload 大部分情况也不行.
\item \verb`print` 字符串的时候会打出一堆东西不方便, 用 \verb`p str.c_str()` 和 \verb`p str.size()` 即可
\item \verb`frame` 显示当前的文件, 行号, 函数, 和代码
\item \verb`where` 用于询问当前执行到哪里(等效于 \verb`backtrace` 即 \verb`bt`)
\item \verb`list` (快捷键 \verb`l`) 显示 10 行周围的代码, 反复使用可以连续列出下文代码
\item \verb`continue` (快捷键 \verb`c`), 继续运行直到遇到 breakpoint
\item \verb`clear` 清除当前行的 breakpoint
\item \verb`delete <number>` 清除编号为 \verb`<number>` 的 breakpoint
\item \verb`delete` 清除所有 breakpoint
\item use ignore \verb`<break#> <times>` to ignore the breakpoint \verb`<break#>` for \verb`<times>` times
\item \verb`finish` 命令跳出当前函数 (快捷键不是 \verb`f`)
\item \verb`f` 可以显示当前的行号和代码
\item \verb`until <number>` 执行到指定行
\item \verb`backtrace` 或 \verb`bt` 显示每层正在执行的子程序的信息 (stack frame). 这样就可以追踪当前的 function 是被哪一行调用, 直到 main() 程序

\item 在 gdb 内可以用 \verb`shell <command>` 来输入命令行命令 \verb`<command>` 而无需退出 gdb 或者暂停程序
\item 在 gdb 内可以直接用 \verb`make` 命令而无需在前面加 \verb`shell`
\item 回车用于重复上一条命令, 一些命令如 run 不会被回车重复. 如果 list 被回车重复, 会 list 接下来的代码
\item Tab 键可以自动填充命令名, 文件名, 函数名等. 如果有多种选择, 按两次 Tab 就可以列出所有的可能.
\end{itemize}

\subsection{Fortran}
\begin{itemize}
\item 在 intel debugger \verb`gdb-ia` 中, \verb`V(1:2,1:2,3,4)` 这种语法是可以用的.
\item 如果 \verb`p V(1,1,1:2,1:2)` 提示超过了 \verb`max-value-size`, 就用 \verb`set max-value-size unlimited` 就可以解决
\end{itemize}

\subsection{GDB server}
\begin{itemize}
\item 在远程机器(target)上运行要调试的程序和 gdb server, 在本地机器(host)上运行 gdb.
\item \href{https://stackoverflow.com/questions/69176457/the-difference-between-gdbserver-and-remote-gdb}{为什么可以用 ssh 还要用 gdb server?} 其实还有一点就是比如远程系统不支持 GUI, 而我们想使用例如 CLion 或者 gdbgui 的 GUI 界面来调试程序.
\item 一个\href{https://www.thegeekstuff.com/2014/04/gdbserver-example/}{简单的教程}.
\item CLion 如何\href{https://www.jetbrains.com/help/clion/remote-debug.html}{远程调试}.
\item 在服务器安装 \verb|sudo apt install gdbserver| (一般已经随系统安装). 用 \verb|-g| 选项编译一个程序, 然后运行例如 \verb|gdbserver localhost:2000 main.x| (其中 \verb|localhost| 不需要替换为真正的 ip).
\item 在本地安装 \verb|gdb|, 直接运行, 不需要编译相同的 \verb|main.x| (可能会导致 breakpoing 不能使用), 但这么做的缺点是运行前要加载一大堆远程系统的 so 文件. 然后 \verb|target remote xxx.xxx.xxx.xxx:2000|. 接下来就可以正常使用了 \verb|gdb| 的各种命令了. 如果显示 \verb|run| 命令不能用, 就直接 \verb|continue| 即可.
\item 运行完以后, 服务器的 \verb|gdbserver| 会自动结束, 本地的 \verb|gdb| 也会断开链接. 需要再次在服务器中运行并连接.
\item TODO: 如果有 input file 怎么办?
\end{itemize}
