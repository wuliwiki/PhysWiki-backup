% 数值解线性方程组(进阶)

\begin{issues}
\issueDraft
\end{issues}

\pentry{数值解线性方程组(入门)\upref{NLinEq}}

上一节我们介绍了求解线性方程组的基本方法:高斯消元法和 LU 分解. 然而,由于篇幅原因,我故意遗留了一个很关键的点,这一节我们就来仔细讨论一下.首先, 看下面的例子:

\subsubsection{例子 1}

求解线性方程组  $Ax=b$  ,其中  $A=\begin{bmatrix} \epsilon & 1\\ 1 & 1 \end{bmatrix}$  ,  $b=\begin{bmatrix}  1\\  2 \end{bmatrix}$  .很简单的可以求出解为:  $x_1=\frac{1}{1-\epsilon}$, $x_2=\frac{1-2\epsilon}{1-\epsilon}$.

如果按照上一节中高斯消元的求解步骤,并考虑一个特殊情况  $\epsilon=10^{-17}$  

\begin{enumerate}
\item 对  $A$  进行 LU 分解,可以得到  $L=\begin{bmatrix} 1& 0\\ \frac{1}{\epsilon} & 1 \end{bmatrix}$  ,  $U=\begin{bmatrix} \epsilon & 1\\ 0 & 1-\frac{1}{\epsilon} \end{bmatrix}$.
\item  求解  $Ly=b$  ,即  $\begin{bmatrix} 1& 0\\ \frac{1}{\epsilon} & 1 \end{bmatrix} \begin{bmatrix} y_1\\ y_2 \end{bmatrix}= \begin{bmatrix} 1\\ 2 \end{bmatrix}$  ,依次得到  $y_1=1$   $y_2=2-\frac{1}{\epsilon}$  .但是,由于	\textbf{机器精度}的原因,计算机在这个浮点运算过程中会得到  $y_2=-10^{17}$  .关于机器精度,可以参考 “计算机算数\upref{CmArit}”. 当然,这个结果和  $y_2$  的精确解差距不大,还是可以接受的.
\item 求解  $Ux=y$  ,即  $\begin{bmatrix} \epsilon& 1\\ 0 &1-\frac{1}{\epsilon}  \end{bmatrix} \begin{bmatrix} x_1\\ x_2 \end{bmatrix}= \begin{bmatrix} 1\\ 2-\frac{1}{\epsilon} \end{bmatrix}$  ,可以得到  $x_2=\frac{2-\frac{1}{\epsilon}}{1-\frac{1}{\epsilon}}$  .即使我们不使用上面一步得到的  $y_2=-10^{17}$  ,这里也会继续因为机器误差解得  $x_2=1$  .有兴趣的小伙伴,可以尝试直接用计算机验证这个浮点运算的结果.代入求解  $x_1$  ,即  $\epsilon x_1+1=1$  .那么,最终的解为  $x_1=0$ ! 而如果我们用计算机直接运算之前的解析解表达式,可以得到  $x_1=1$  .显然,这才是机器误差允许范围内的正确解.
\end{enumerate}

当然,如果你是用 Matlab 或者 Python 自带的线性方程组求解器来求解上面的问题,并不会得到  $x_1=0$  这样的错误解.

\subsubsection{解释}

事实上,问题出在了我们的高斯消元算法上,这个算法目前还并不完整.我们在前一节讨论的整个算法都是基于解析过程,也就是说,假设所有的运算过程都是精确完成的,没有任何误差.但是,在浮点数运算中,这个条件并不能完全满足,机器误差会伴随着每一步运算.而且,随着问题的复杂度增加,运算量加大,机器误差会不断积累. \textbf{因此,科学计算中一个重要的研究内容就是如何控制误差,使其始终保持在一个相对较小的范围.}

\subsubsection{改进}

那么,如何完善我们的LU分解算法呢?我们不妨尝试在分解之前,将  $A$  的两行对调,同样,为了结果的一致性,  $b $  的两行也要对换.这样得到的新的LU分解结果也和之前的有所变化:  $\bar{L}=\begin{bmatrix} 1& 0\\ {\epsilon} & 1 \end{bmatrix}$  ,  $\bar{U}=\begin{bmatrix} 1& 1\\ 0 & 1-{\epsilon} \end{bmatrix}$  .继续按照前面例子中的求解过程,可以得到  $\begin{bmatrix} 1& 1\\ 0 &1-{\epsilon}  \end{bmatrix} \begin{bmatrix} x_1\\ x_2 \end{bmatrix}= \begin{bmatrix} 2\\ 1-2{\epsilon} \end{bmatrix}$  ,(有兴趣的同学可以按照上面例子的三步,自己算一遍.)这样,当  $\epsilon=10^{-17}$  时,  $x_1=1$  ,  $x_2=1$  ,非常接近解析解.

\subsubsection{分析}

我们将这个对调两行的过程用矩阵乘法的形式表示,即  $PAx=Pb$  ,其中  $P=\begin{bmatrix} 0&1\\\ 1&0 \end{bmatrix}$  .因此,我们改进后得到的LU分解事实上就是  $\bar{L}\bar{U}=PA$  .

那为何这样的对调之后,运算误差就被有效的控制住了呢?为此,我们要回到高斯消元的基础思路:

对于第  $k$  步,将 $A$  的第  $k$  行乘以一个系数使其可以正好消去下面各行(  $i=k+1,...,n$  )的第  $k$  列的元素,这个系数应为  $\frac{a_{i,k}}{a_{k,k}}$  ,注意这个系数在消元的过程中会乘以  $A$  的第  $k $  行中每一个元素.如果这个系数大于1,则第  $k $  行中元素的机器误差会被放大,反之则会被缩小.前面的例子1正是因为消元系数  $\frac{a_{2,1}}{a_{1,1}}=\frac{1}{\epsilon}=10^{17}$  ,因此将机器误差放大.

解决

那么,如果我们可以在每一步消元时,都让这个系数  $\frac{a_{i,k}}{a_{k,k}}$ 小于1,那么就可以保证机器误差至少不会由于上面的原因被放大.这也正是	\textbf{改进}办法可以成功的根本原因.因此交换了两行,使得这个消元系数由原来的  $10^{17}$  变成了  $10^{-17}$  .这类方法被称作Pivoting(中文翻译似乎叫寻找主元法).

当然,Pivoting 的策略有很多,包括完全 Pivoting 和部分 Pivoting.

\textbf{完全 Pivoting},就是在第  $k$ 步开始以前,找到第  $k$  到  $n$  行第  $k$  到  $n$  列元素中最大的那一个,通过一次行交换和一次列交换将它和  $a_{k,k}$  互换,然后进行消元.

\textbf{部分 Pivoting}, 仅仅在第 $k$ 列或者第 $k$ 行中寻找最大的元素.

事实上,这些方法已被广泛应用到了几乎所有软件包的LU分解中,有兴趣的同学可以去查看 Matlab 的 \verb|lu| 函数和 scipy 的\verb|linalg.lu| 函数.它们不仅会求出 $L$ 和 $U$, 还会给出相应的 $P$ 矩阵.
