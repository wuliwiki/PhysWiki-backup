% 拉普拉斯方法

\pentry{极值\upref{DerMax}, 积分换元\upref{IntCV}, 高斯积分\upref{GsInt}, 渐近展开\upref{Asympt},  Gamma函数\upref{Gamma}}

在分析数学中,拉普拉斯渐近方法是一种计算含参数积分的渐近展开式的办法. 所考察的积分一般具有如下形式:
\begin{equation}\label{LapAsm_eq1}
L(s)=\int_a^b\E^{-s\Phi(x)}\psi(x)\,\dd x
\end{equation}
其中 $s$ 是正的实参数,考察当 $s\to+\infty$ 时积分的变化趋势. 拉普拉斯方法背后的想法很简单: 如果函数 $\Phi$ 在某点处达到极小值, 那么当 $s$ 充分大时, 只有极小值点附近的贡献才比较可观, 其余部分相比起来都要小得多.

\subsection{Watson 引理}
拉普拉斯方法基于 Watson 引理. 它本身也是很有用的.

\begin{lemma}{Watson 引理}

设函数$\phi:X\to Y$在$\mathbb{R}^+$上连续,序列$\left\{\lambda_n\right\}_0^\infty\left(\lambda_n>0\right)$单调递增,且存在$\delta$右邻域$B_{+}(0,\delta)\;(0<\delta<\pi/2)$,使得
\begin{equation}\label{LapAsm_eq2}
  \phi\left(t\right)\sim\sum_{n=0}^{\infty}{c_nt^{\lambda_n-1}}\quad t\to0
\end{equation}

若有常数$c>0$,满足$\phi\left(t\right)=O\left(\E^{ct}\right),t\to+\infty$,则当$s\to+\infty,z\in{\left\{z:\left|\arg{z}\right|\le\pi/2-\delta\right\}}$时,有
\begin{equation}\label{LapAsm_eq3}
\int_{\mathbb{R}^+}\E^{-zt}\phi(t)\dd t\sim\sum_{n=0}^{\infty}{c_n}\frac{\Gamma(\lambda_n)}{z^{\lambda_n}}  
\end{equation}
\end{lemma}

证:
由Laplace的理论可知, 
\[
F\left(z\right)=\int_{\mathbb{R}^+}{\E^{-zt}\phi\left(t\right)}\,\dd t
\]
若函数$\phi$满足定理中提到的三个条件,则$F(z)$变换对于$\operatorname{Re} z>c$是存在的.即:当$\operatorname{Re} z>c$时,积分收敛.注意 \autoref{LapAsm_eq2} 意味着存在正整数$N$,当$n>N$时,对一切的$t\in B_{+}(0,\delta)$,总有
\[
  \left|\phi\left(t\right)-\sum_{n=0}^{N-1}{c_nt^{\lambda_n-1}}\right|\le Mt^{\lambda_N-1}
\]
其中$M>0$且为常数.同时 \autoref{LapAsm_eq3} 表明当时,存在正整数$N$,当$n>N$时,对一切的$t\in B_{+}(0,\delta)$,总有
\[
\left|\phi\left(t\right)-\sum_{n=0}^{N-1}{c_nt^{\lambda_n-1}}\right|\le K\E^{ct}t^{\lambda_N-1}
\]
其中$K>0$且为常数.于是就有
\[
\left|\int_{\mathbb{R}^+}{\E^{-zt}\phi\left(t\right)}\,\dd t-\sum_{n=0}^{N-1}{c_n\int_{\mathbb{R}^+}{\E^{-zt}t^{\lambda_n-1}}\,\dd t}\right|
\le K\int_{\mathbb{R}^+}\E^{[-\operatorname{Re}z-c]t}t^{\lambda_N-1}\,\dd t
\]
注意到当$\operatorname{Re}z>0$时
\[
\int_{\mathbb{R}^+}{\E^{-zt}t^{\lambda_n-1}}\,\dd t
=\frac{1}{z^{\lambda_n}}\int_{\mathbb{R}^+}{\E^{-u}u^{\lambda_n-1}}\,\dd u
=\frac{\Gamma\left(\lambda_n\right)}{z^{\lambda_n}}
\]
因此
\[
\left|F(z)-\sum_{n=0}^{N-1}{c_n\frac{\Gamma\left(\lambda_n\right)}{z^{\lambda_n}}}\right|\le K {\Gamma(\lambda_{N})\over \operatorname{[Re}z-c]^{\lambda_N}}
=K{\Gamma(\lambda_{N})\over |z|^{\lambda_N}}\left(|z|\over \operatorname{Re}z-c\right)^{\lambda_N}
\]
由于$z\in{\left\{z:\left|\arg{z}\right|\le\pi/2-\delta\right\}}$,于是$\operatorname{Re}z\geq |z|\sin \delta$.这表明当$\left|z\right|$足够大时,$\operatorname{Re}z-c\geqslant1/2|z|\sin \delta$.因此
\[
F\left(z\right)-\sum_{n=0}^{N-1}{c_n\frac{\Gamma\left(\lambda_n\right)}{z^{\lambda_n}}}=O\left(z^{-\lambda_N}\right)
\]
这就证明了Watson's 引理.


\subsection{拉普拉斯方法}
拉普拉斯方法是将 Watson 引理用于\autoref{LapAsm_eq1} 得到的. 作出如下假设:

\begin{enumerate}
\item 函数 $\Phi,\psi:X\to Y$ 都是光滑函数.

\item 函数 $\Phi(x)$ 在积分区间$I=[a,b]$内部仅有一个严格的极小值点 $x_0\in(a,b)$,同时$\Phi''(x_0)>0$, 且 $x_0$ 是整个区间上的最小值点.

\item \autoref{LapAsm_eq1} 对于 $s=s_0$ 绝对收敛.
\end{enumerate}

\begin{figure}[ht]
\centering
\includegraphics[width=12cm]{./figures/LapAsm_1.pdf}
\caption{函数$\Phi$示意图} \label{LapAsm_fig1}
\end{figure}

于是 $\Phi'(x_0)=0$. 由泰勒公式可知, 存在正数 $\delta>0$ ,使得对于任意的 $x\in B(x_0,\delta)$  
\[
  \Phi(x)=\Phi(x_0)+\frac{1}{2}\Phi''(x_0)(x-x_0)^2+\varphi(x)(x-x_0)^3
\]

其中存在正数$M>0$,使得函数 $|\varphi|\leq M$ . 当 $s$ 充分大时, 邻域 $B(x_0,\delta)$ 以外的积分的贡献比 $s$ 的任何负幂次衰减得都快:根据假定 2,当 $x\in I\setminus B(x_0,\delta)$ 时, 存在正数 $\varepsilon_0>0$ 使得 $\Phi(x)>\Phi(x_0)+\varepsilon_0$, 从而邻域 $B(x_0,\delta)$ 之外的积分估计为
\[
  \left|\int_{I\setminus B(x_0,\delta)}\psi(x)\E^{-s\Phi(x)}\,\dd x\right|
  \leq \E^{-\epsilon_0(s-s_0)}\int_I|\psi(x)|\E^{-s_0\Phi(x)}\,\dd x
  \leq \E^{-s_0\Phi(x_0)}\E^{-\varepsilon_0s}\int_{I}|\psi(x)|\,\dd x
\]
而在邻域 $B(x_0,\delta)$ 内, 可作如下换元: $y^2=\Phi(x)-\Phi(x_0)$, 即
\[
  y=\sqrt{\frac{\Phi''(x_0)}{2}}(x-x_0)\left(1+\frac{2}{\Phi''(x_0)}(x-x_0)\varphi(x)\right)^{1/2}.
\]
令 $\alpha<\beta$分别等于$\pm\sqrt{\Phi(x_0\pm\delta)-\Phi(x_0)}$, 又设当$y\to0$时,函数$\tilde{\psi}$有泰勒展开
\begin{equation}\label{LapAsm_eq4}
\tilde{\psi}(y):=\psi(x(y))\frac{\dd}{\dd x}[x(y)]\sim c_0+c_1y+\cdots \quad 
\end{equation}
(这里显然有 $c_0=\psi(x_0)\sqrt{2/\Phi''(x_0)}$), 则当$s\to+\infty$
$$
\begin{aligned}
\int_{B(x_0,\delta)}\psi(x)\E^{-s\Phi(x)}\,\dd x
&=\E^{-s\Phi(x_0)}\int_{\alpha}^{\beta}\tilde \psi(y)\E^{-sy^2}\,\dd y\\
&\sim\E^{-s\Phi(x_0)}\int_{0}^{\varepsilon_0}[\tilde \psi(y)+\tilde \psi(-y)]\E^{-sy^2}\,\dd y
\end{aligned}
$$
应用 Watson 引理, 最后终于得到\autoref{LapAsm_eq1} 当$s\to+\infty$时的渐近展开:
\begin{equation}\label{LapAsm_eq5}
  L(s)\sim \E^{-s\Phi(x_0)}\sum_{n=0}^\infty c_{2n}\Gamma\left(n+\frac{1}{2}\right)s^{-n-1/2}
\end{equation}
其中 $c_n$ 由\autoref{LapAsm_eq4} 给出. 特别地, 由于 $\Gamma(1/2)=\sqrt{\pi}$, \autoref{LapAsm_eq1} 渐近公式的首项是
\[
L(s)=\E^{-s\Phi(x_0)}\left(As^{-1/2}+O(s^{-1/2})\right)\quad s\to+\infty
\]
其中
\[
  A=\sqrt{2\pi}\frac{\psi(x_0)}{(\Phi''(x_0))^{1/2}}
\]

\subsection{斯特林公式}
拉普拉斯方法最基本的应用就是导出 $\Gamma$ 函数的\textbf{斯特林公式(Stirling formula)}. 按照定义,
\[
  \Gamma(s+1)
  =\int_0^\infty x^s\E^{-x}\,\dd x
  =e^{(s+1)\log s}\int_0^\infty \E^{-s(x-\log x)}\,\dd x.
\]
最后一步换元 $x\mapsto sx$. 最后这个积分刚好符合上一小节所要求的三条假设:这里 $\psi(x)\equiv1$,$\Phi(x)=x-\log x$.于是 $x=1$ 是 $\Phi(x)$ 唯一的极小值点,同时也是唯一的最小值点.代入\autoref{LapAsm_eq5}, 就得到渐近公式
\[
  \Gamma(s+1)\sim\sqrt{2\pi s}\left(\frac{s}{\E}\right)^s\quad s\to+\infty
\]
\autoref{LapAsm_eq5} 还给出了更精细的渐近级数:
\[
  \Gamma(s+1)
  =\sqrt{2\pi s}\left({s\over \E}\right)^s
  \left(
   1
   +{1\over12s}
   +{1\over288 s^2}
   -{139\over51840 s^3}
   -{571\over2488320 s^4}
   + \cdots
  \right)\quad s\to+\infty
\]
毫无疑问这是一个可以对大的 $n$ 来近似计算 $n!$ 的公式.
