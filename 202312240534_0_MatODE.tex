% Matlab 阶常微分方程组(ode45)
% license Usr
% type Tutor

Matlab 自带的 \verb`ode45` 解算器使用成熟的变步长的 4 阶 和 5 阶 Runge-Kutta 算法(该算法详见\upref{OdeRK4})。 下面我们演示如何用它求解一个简单的微分方程组
\begin{equation}
\leftgroup{
&\dot x = -\I (x + \alpha t y)~,\\
&\dot y = -\I (\alpha t x + y)~.
}\end{equation}
其中 $\alpha$ 为常数。 为了告诉解算器我们需要求解的方程, 可以

\begin{figure}[ht]
\centering
\includegraphics[width=12cm]{./figures/a7512df93f4fa9a5.png}
\caption{\verb|ode45_test.m| 的运行结果} \label{fig_MatODE_2}
\end{figure}

\begin{lstlisting}[language=matlab,caption=ode45\_test.m]
function ode45_test
close all;
% 初始条件和alpha的值
x0 = 1;  % 您需要提供x(0)的值
y0 = 0;  % 您需要提供y(0)的值
alpha = 0.01;  % 您需要提供alpha的值

% 时间跨度
tspan = [0 30];  % 从0到10秒,根据需要进行调整

% 初始状态向量
y0 = [x0; y0];

% 调用ODE求解器
[t, y] = ode45(@(t,y) odefunc(t, y, alpha), tspan, y0);

% 绘制结果
figure;
plot(t, abs(y(:,1)).^2, 'r'); hold on; grid on;
plot(t, angle(y(:,1)), 'b');
plot(t, abs(y(:,2)).^2, 'r--');
plot(t, angle(y(:,2)), 'b--');
title('解微分方程组');
xlabel('时间 t');
ylabel('解 x 和 y');
legend({'|x(t)|^2', 'Arg x(t)', '|y(t)|^2', 'Arg y(t)'});
end

function dydt = odefunc(t, y, alpha)
    mat = -1i*[1 alpha*t; alpha*t 1];
    dydt = mat * y;
end
\end{lstlisting}

