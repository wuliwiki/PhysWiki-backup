% 群代数
% 群代数|群空间
\pentry{群矩阵表示及实例\upref{gprep}}
\begin{issues}
\issueDraft
\end{issues}

\begin{definition}{群空间}
对于有限群$G=\{g_1,g_2...g_m\}$,设$V_G$为群元在复数域$\mathbb{C}$是的所有线性叠加的集合:

\begin{equation}
V_G=\{\displaystyle\sum_\nu x_\nu g_\nu|x_\nu \in \mathbb{C},g_\nu \in G\}
\end{equation}

在这个基础上我们可以定义加法和数乘。

设$x=\displaystyle\sum_\nu x_\nu g_\nu$,$y=\displaystyle\sum_\mu y_\mu g_\mu$,$a\in \mathbb{C}$,则有:
\begin{align}
x+y&=\displaystyle\sum_\nu x_\nu g_\nu+\displaystyle\sum_\mu y_\mu g_\mu=\displaystyle\sum_\nu(x_\nu+y_\nu)g_\nu\\
ax&=\displaystyle\sum_\nu (ax_\nu) g_\nu
\end{align}

这样显然构成了一个$m$维的线性空间,$m$为群$G$的阶数。

线性空间的一组基为$\{g_1,g_2...g_m\}$,称为自然基底。

\end{definition}

在定义完群空间后我们进一步定义群空间中的代数乘法,使其构成一个代数,也就是本节标题——群代数。

\begin{definition}{群代数}
设$V_G$为群$G$的群空间,且有$x=\displaystyle\sum_\nu x_\nu g_\nu$,$y=\displaystyle\sum_\mu y_\mu g_\mu$,我们定义其乘法规则为:

\begin{equation}
x*y=\displaystyle\sum_{\nu\mu}x_\nu g_\nu y_\mu g_\mu=
\displaystyle\sum_{\nu\mu}(x_\nu y_\mu) (g_\nu g_\mu)
\end{equation}

其中$g_\nu g_\mu$一项依照群乘法表的惩罚规则给出结果。

在这样的惩罚规则下$V_G$构成了复数域$\mathbb{C}$上的一个结合代数,称为$A_G$。

\end{definition}

注:群代数的结合律来自于群元的结合律。

在群代数的视角下,不同的