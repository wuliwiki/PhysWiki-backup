% 齐次凸泛函
% keys 凸性|泛函
% license Usr
% type Tutor
\pentry{泛函与线性泛函\nref{nod_Funal}}{nod_63ed}

齐次凸泛函是与凸集紧密联系的概念,共有的“凸”也表达了这一印象。以后我们将看到,非负齐次凸泛函与其\enref{核}{ConSet}含0点的\enref{凸体}{ConSet}是一一对应的。当然,凸泛函可以看作是\enref{凸函数}{ConvFu}在一般线性空间中的推广。

\begin{definition}{凸泛函}
设 $L$ 是实线性空间。$L$ 上的\enref{泛函}{Funal} $p$ 称为\textbf{凸的},是指对任意 $x,y\in L,0\leq\alpha\leq1$,成立
\begin{equation}
p(\alpha x+(1-\alpha)y)\leq \alpha p(x)+(1-\alpha)p(y).~
\end{equation}
\end{definition}

\begin{definition}{齐次}
$L$ 上的泛函 $p$ 称为\textbf{正齐次的},是指对所有的 $x\in L,\alpha >0$,
\begin{equation}
p(\alpha x)=\alpha p(x).~
\end{equation}

\end{definition}
