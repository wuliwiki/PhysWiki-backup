% 位移与路程
\pentry{曲线的长度\upref{CurLen} , 位置矢量、位移\upref{Disp} , 速度、加速度\upref{VnA},}

我们已经处理了质点位矢与位移的问题。接下来我们要问的是,质点从$t_0$至$t_1$这段时间之内运动的距离(或者说,路程)是多少?
\begin{figure}[ht]
\centering
\includegraphics[width=5cm]{./figures/DPAS_1.pdf}
\caption{请添加图片描述} \label{DPAS_fig1}
\end{figure}
首先可以\textbf{排除} $s = \abs{\bvec r(t_1) - \bvec r(t_0)}  $。如图,这段时间内位移的模长显然不等于质点通过的路程长度。

这就需要我们复习一下我们计算曲线长度时所用的套路。将这段时间分成一段段很小的时间间隔,当$\Delta t \to 0$时,这段小时间内位移的模长就近似等于路程。

如图所示,应该有$$s=\sum \abs{\Delta \bvec r}$$
亦即$$
\begin{aligned}
s&=\sum \abs{\bvec v \Delta t}\\
&=\sum \abs{\bvec v} \Delta t\\
&=\int _{t_0}^{t_1} \abs{\bvec v} \dd t\\
\end{aligned}
$$
展开为分量形式(假定$\bvec v = (x',y',z')^T$,其中$x',y',z'$都是关于$t$的函数):
$$
s = \int _{t_0}^{t_1} \sqrt{x'^2+y'^2+z'^2} \dd t
$$
