% 原码、反码、补码
% keys 编码|位运算|ALU|计算机组成原理
% license Usr
% type Wiki
\begin{issues}
\issueDraft
\end{issues}

\pentry{数字电路_运算器\nref{nod_Sample}}{nod_eb6a}

\subsection{原码(True form)}

原码即“未经更改”的码,是指一个二进制数左边加上符号位后所得到的码,且当二进制数大于0时,符号位为0;二进制数小于0时,符号位为1;二进制数等于0时,符号位可以为0或1(+0/-0)。

使用n位原码表示\textbf{有符号数}时,范围是 $-(2^{n-1}-1)\sim +(2^{n-1}-1)$。 当n=8时,这个范围就是$-127\sim +127 $

表示\textbf{无符号数}时,由于不需要考虑数的正负,就不需要用一位来表示符号位,n位机器数全部用来表示是数值,这时表示数的范围就是
$0\sim 2^{n}-1$。当$n=8$时,这个范围就是$0\sim 255$

\subsubsection{优点}
简单直观,原码易于人类理解和计算。

\subsubsection{缺点}
原码不能直接被用于运算。
对于加法运算,例如,数学上,1+(-1)=0,但用原码进行运算时,00000001+10000001=10000010,该结果对应数值为-2。显然出错了;
对于减法运算,原码减法需要先将减数取反加 1,才能得到正确的数学结果。

也就是说,原码的符号位不能直接参与运算,必须和其他位分开,这就增加了硬件的开销和复杂性。

\subsection{反码}
在反码表示法中,正数的反码与其原码相同;负数的反码则是将原码(除符号位外)的每一位取反。反码解决了一些原码在运算上的问题,但仍然存在如负零的表示以及加法运算中的进位问题。

\subsubsection{优点}
可以直接用于减法运算。

与原码表示相同符号的数,反码也相同。

\subsubsection{缺点}
不能直接用于加法运算。
0 的反码为其自身,这与其他数不同。


\subsection{补码}





\subsection{为什么要使用补码}

补码的规则看起来很怪,但对数字电路设计或者说计算机的位运算来说却是友好的编码方式。

\textbf{换一种角度理解:}

在设计一个编码规范时,我们可以把编码后的数围成一个圆(类似一个时钟),我们希望在这个钟表上:
\begin{enumerate}
\item 
无冲突,每个数值的位置都是独一无二,
\item 
连续性,每次运算(+1)相当于时钟顺时针移动一个单位,
\end{enumerate}

从上述角度看,补码是一种很和谐的编码。







% 参考 https://www.cnblogs.com/zhangziqiu/archive/2011/03/30/computercode.html remove by lzq
