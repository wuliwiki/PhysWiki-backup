% 简谐振子的品质因数
% 简谐振子|品质因数|阻尼|共振

\pentry{简谐振子受迫运动\upref{SHOfF}}

简谐振子的品质因数为
\begin{equation}
Q = \frac{k}{\alpha \omega_m}
\end{equation}
其中 $k$ 为弹性系数, $\alpha$ 为线性阻尼, $\omega_m$ 为共振角频率.

弹簧振子的品质因数有几种定义方法,它们是等效的

\subsection{从能量定义}
\begin{equation}
Q = \frac{2\pi E}{W}
\end{equation}
其中 $E$ 是当 $\omega = \omega_m$ 时弹簧振子做简谐运动的总能量, $W$ 是 $\omega = \omega_m$ 时 $f(t)$ 在一个周期内给弹簧振子做的功,等于阻力在一个周期内消耗的能量.下面根据该定义推导.

当达到共振频率时($\omega = \omega_m$), 外力一周期做功
\begin{equation}
W = \int_0^{2\pi/\omega} \Re[y'(t)] \Re[f(t)] \dd{t}
\end{equation}

令 $A = A_x + \I A_y$, 则
\begin{equation}
\Re[y'(x)] = \Re[\I\omega (A_x + \I A_y)(\cos\omega t + \I \sin\omega t)] = -\omega A_y \cos\omega t - \omega A_x \sin\omega t
\end{equation}
\begin{equation}
\Re f(t) = B\cos \omega t
\end{equation}
代入方程,得
\begin{equation}
W = -\omega B \int_0^{2\pi/\omega} (A_y \cos\omega t + A_x \sin\omega t) \cos\omega t \dd{t} = -B A_y \pi
\end{equation}
另外弹簧振子的总能量为 $E = k\abs{A}^2/2$, 所以品质因数为
\begin{equation}\label{SHOq_eq1}
Q = \frac{2\pi E}{W} = -\frac{k\abs{A}^2}{B A_y}
\end{equation}
此时
\begin{equation}
A = \frac{B}{m(\omega_0^2 - \omega^2) + \I \alpha\omega} = \frac{B}{\frac{\alpha^2}{2m} + \I\omega \alpha}
\end{equation}
所以
\begin{equation}
\abs{A} = \frac{\abs{B}}{\sqrt{\alpha^2\omega^2 + \frac{\alpha^4}{4m^2}}}
\end{equation}
\begin{equation}
A_y = \Im[A] = \frac{-B\omega\alpha}{\alpha^2\omega^2 + \frac{\alpha^2}{4m^2}}
\end{equation}
代入\autoref{SHOq_eq1},得品质因数为
\begin{equation}
Q = \frac{k}{\alpha\omega_m}
\end{equation}

\subsection{从幅频曲线定义}
当线性阻尼 $\alpha$ 很小时,品质因数为
\begin{equation}
Q = \frac{f_m}{\Delta {f}} = \frac{\omega_m}{\Delta \omega}
\end{equation}
其中 $\Delta\omega$ 是系统的能量—频率曲线的半高宽(FWHM).%未完成
由此可以看出品质因数越大,幅频曲线就越尖锐越细.

由于弹簧振子能量为 $E = k\abs{A}_{max}^2/2$, $\Delta \omega$ 也是幅频曲线的 “$1/\sqrt{2}$ 高宽”. 即满足
\begin{equation}\label{SHOq_eq2}
\abs{A} \geqslant \frac{1}{\sqrt{2}} \abs{A}_{max}
\end{equation}
的 $\omega$ 区间的宽度.

由于
\begin{equation}
\omega_m = \sqrt{\frac{k}{m} - \frac{\alpha^2}{2m^2}} = \sqrt{\omega_0^2 - \frac{\alpha^2}{2m^2}}
\end{equation}
当阻力系数 $\alpha$ 很小时, 弹簧振子的共振频率接近于无阻力时的频率.根据(未完成),
\begin{equation}
\omega_m = \sqrt{\omega_0^2 - \frac{\alpha^2}{2m^2}} = \omega_0 - \frac{\alpha^2}{4\omega_0 m^2}
\end{equation}
所以可以认为 $\omega_m \approx \omega_0$.

把幅频关系
\begin{equation}
\abs{A} = \frac{\abs{B}}{m\sqrt{(\omega^2 - \omega_m^2)^2 + (\omega_0^4 - \omega_m^4)}}
\end{equation}
代入\autoref{SHOq_eq2} 得
\begin{equation}
(\omega^2 - \omega_m^2)^2 + (\omega_0^4 - \omega_m^4) = 2(\omega_0^4 - \omega_m^4)
\end{equation}
解得
\begin{equation}
\omega^2 = \omega_m^2 \pm \sqrt{\omega_0^4 - \omega_m^4}
\end{equation}
所以区间以 $\omega_0$ 为中心且
\begin{equation}
\Delta(\omega^2) = 2\sqrt{\omega_0^4 - \omega_m^4}
\end{equation}
而 $\Delta(\omega^2) \approx 2\omega_m \Delta \omega$, 所以
\begin{equation}
\Delta \omega =  \frac{\Delta(\omega^2)}{2\omega_m} = \frac{\sqrt{\omega_0^4 - \omega_m^4}}{\omega_m}
\end{equation}
其中
\begin{equation}
\sqrt{\omega_0^4 - \omega_m^4} = \sqrt{(\omega_0^2 - \omega_m^2)(\omega_0^2 + \omega_m^2)}
\approx \sqrt{\frac{\alpha^2}{2m^2} \cdot 2\omega_m^2} = \frac{\alpha\omega_m}{m}
\end{equation}
所以 $\Delta \omega = {\alpha}/{m}$. 所以
\begin{equation}
Q = \frac{\omega_m}{\Delta \omega} \approx \frac{\omega_0}{\Delta \omega} = \frac{m\omega_0}{\alpha} = \frac{m}{\alpha} \sqrt{\frac{k}{m}} = \frac{k}{\alpha \omega_0}
\approx \frac{k}{\alpha\omega_m}
\end{equation}
与第一种定义结论一致.
