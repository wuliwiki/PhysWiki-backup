% Julia 的类型与值
% 类型 值

本文授权转载自郝林的 《Julia 编程基础》. 原文链接:\href{https://github.com/hyper0x/JuliaBasics/blob/master/book/ch04.md}{第 4 章 类型系统}.


\subsubsection{4.2 类型与值}

我们在前面提到了子类型(subtype)这个概念.与之相对的概念是超类型(supertype).比如说,\verb|Integer|类型是\verb|Signed|类型的直接超类型,并且还是\verb|Int64|类型的间接超类型.如果用操作符\verb|<:|来表示的话,那就是:\verb|Int64 <: Signed <: Integer|.

实际上,Julia中预定义的所有类型共同构成了一幅具有层次的类型图.这幅类型图中的类型之间都是有关系的.更具体地说,它们要么存在着直接或间接的继承关系,要么有着共同的超类型.

每一个Julia程序都会使用甚至定义一些类型.正因为如此,我们的程序才与Julia的类型系统关联在了一起.可以说,我们在编写程序时总会使用到Julia的类型图,并且有时候(即在自定义类型时)还会对这幅图进行扩展.我们定义的所有类型都会有一个超类型,即使我们没有显式地指定它.如此一来,我们的类型就与Julia原有的类型图联系在一起了.

我们之前说过,Julia代码中的任何值都是有类型的.或者说,Julia程序中的每一个值都分别是其所属类型的一个实例.不仅如此,每一个值也都分别是其所属类型的所有超类型的一个实例.例如:

\begin{lstlisting}[language=julia]
julia> 10::Int64, 10::Signed, 10::Integer
(10, 10, 10)

julia> 
\end{lstlisting}

可以看到,上例中的3个类型断言都成功了.也就是说,\verb|10|这个值既是\verb|Int64|类型的一个实例,也是\verb|Signed|类型和\verb|Integer|类型的一个实例.

此外,Julia代码中所有的值都是对象(object).但与那些传统的支持面向对象编程的语言不同,Julia中的对象(或者说这些对象所属的类型)并不会包含或关联任何方法.恰恰相反,一个函数会用它的衍生方法去尽量适应被操作的对象.这正是由Julia的多重分派机制来控制的.

再次强调,Julia中只有值才有类型,而变量本身是没有类型的.一个变量代表的只是一个标识符与某个值之间的绑定关系.