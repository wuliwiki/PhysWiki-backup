% X 射线衍射技术
% license CCBYSA3
% type Wiki

(本文根据 CC-BY-SA 协议转载自原搜狗科学百科对英文维基百科的翻译)

\textbf{X射线衍射法}是一种揭示材料和薄膜的晶体结构、化学成分和物理性质的无损分析技术。这些技术是基于观察X射线照射到样品上的散射强度,它是入射和散射角度、偏振、波长或能量的函数。

需要注意的是,X射线衍射现在通常被认为是X射线散射的的一个子集,其中散射是弹性的,而散射对象是晶体,因此所得图案包含由 X射线晶体学分析的尖锐斑点(如图所示)。然而,散射和衍射都是相关的普遍现象,它们之间并不总是存在区别。因此,1963年Guinier的经典著作[1]《晶体、不完美晶体和非晶中的X射线衍射》”,所以“衍射”在当时显然并不局限于晶体。

\subsection{散射技术}
\subsubsection{1.1 弹性散射}
\begin{itemize}
\item $X$射线衍射或者更具体地说是广角$X$射线衍射(WAXD)
\item 小角$X$射线散射(SAXS)通过测量散射角$2\theta$接近0°时的散射强度来探测纳米到微米范围内的结构。
\item $X$射线反射率是一种分析技术,用于确定单层和多层薄膜的厚度、粗糙度和密度。
\item 广角$X$射线散射 (WAXS),一种专注于散射角$2\theta$大于5°的技术。
\end{itemize}
\subsubsection{1.2 非弹性x光散射(IXS)}