% 厦门大学 2002 年硕士物理考试试题
% keys 厦门大学|考研|物理|2002年
% license Copy
% type Tutor

\textbf{声明}:“该内容来源于网络公开资料,不保证真实性,如有侵权请联系管理员”

\begin{enumerate}
\item 一质盘为$m=2kg$的质点在XY平面内运动,运动方程为:$\bar r =[(-t^3+2t)\bar i+(t^2-3t)\bar j]m$。\\
(1)$t=2$(s)时质点所受的合力$\bar F$=($\qquad$)\\
(2)$t=0$(S)至$t=2$(s)时间内$\bar F$的量$\bar I$=($\qquad$)\\
(3)$t=0$(s)至$t=2$(s)时间内产所作的功$W$=($\qquad$)
\item 一平面简谐波以波速$v=60(m/s)$在密度为$\rho=2.0(kg/m^3)$的弹性介质中沿Y轴负方向传播,已知平衡位置在$y=20(m)$处的$P$点的振动曲线如图所示。\\
(1)该平面简谐波的表达式为($\qquad$)\\
(2)该平面简谐波的能流密度大小为($\qquad$)
\item 某气体在$T=27$°$K,p=1.00atm$时的密度$\rho=1.30kg.m^{-3}$,则该气体单位体积分子数$n$=($\qquad$),摩尔质量$\mu$=($\qquad$),分子平均平动动能$\bar \varepsilon$=($\qquad$),平均速率$\bar v$=($\qquad$)。
\item 基尔雀夫第一定律的基础是($\qquad$),基尔萑夫第二定律的基础是($\qquad$)。
\item 电路如图所示,已知电阻$R=40\Omega$,三个电流表的读数分别为$I_1=4.0A,I_2=3.0A,I=6.0A$,则元件乙的功率为($\qquad$)。
\begin{figure}[ht]
\centering
\includegraphics[width=6cm]{./figures/2af4dd599f1c29c7.png}
\caption{} \label{fig_SD02_1}
\end{figure}
\item 利用消反射膜来消除玻璃镜头的反射光时,通常膜的折射率n应($\qquad$)(小于或大于),玻璃的折射率:膜的最小厚度应等于($\qquad$)。
\item 自然光投射到两片叠在一起的偏振片上,若偏振片是理想的,(1)当透射光是入射光强度的三分之一时,这两偏振片的透光方向的夹角应:($\qquad$)。(2)当透射光是最大透射光强的三分之一时,这两偏振为片的透光方向的夹角应为($\qquad$)。
\item 如图,长为$l$,质是可忽略的杆上周定有两质盘均为$m$的小球A,B,A 球位于杆中间,B球位于杆的一端,系统自水平位置以零初速绕过杆的另一端点O的转轴沿竖直而自由下摆。求:\\
(1)当杆与水平夹角为$\theta$时,$B$球的速度和加速度;\\
(2)此时杆对$B$球的作用力;\\
(3)系统运动到整直位置时,$B$球与一静止的小球$C$发生对心弹性碰撞,碰撤后$B$球(连同$A$球)刚好静止,求小球$C$的质量。
\end{enumerate}
