% 牛顿运动定律、惯性系
% 牛顿定律|惯性系|运动定律|第二定律|第三定律

\pentry{速度、加速度\upref{VnA}, 力的合成与分解\upref{Fdecom}, 速度的参考系变换\upref{Vtrans}}

牛顿的三定律可表述如下, 为了避免讨论物体的质心及转动, 这里我们只讨论质点。
\begin{itemize}
\item \textbf{第一定律}:不受力或受合力为零的质点做匀速直线运动或静止。
\item \textbf{第二定律}:质点所受合外力等于质点的质量乘以加速度。
\item \textbf{第三定律}:两质点的相互作用力等大反向。
\end{itemize}

\subsection{第一定律}

牛顿第一定律的作用是定义\textbf{惯性坐标系(inertial frame of reference, 简称惯性系)}:定义满足牛顿第一定律的参考系就是惯性系, 且惯性系存在。

\begin{corollary}{推论}
相对某惯性系静止或匀速运动且没有相对转动的参考系也是惯性系,否则不是惯性系。
\end{corollary}
\textbf{证明}:若已知 $A$ 系为惯性系,$B$ 系相对 $A$ 系的平移速度为 $\bvec v_{AB}$, 质点在两系中的瞬时速度分别记为 $\bvec v_A, \bvec v_B$, 则由高中的 “绝对速度 = 相对速度 + 牵连速度”(\autoref{Vtrans_eq1}~\upref{Vtrans})得
\begin{equation}
\bvec v_{B} = \bvec v_{A} + \bvec v_{AB}~.
\end{equation}
由于 $A$ 是惯性系, 所以任何不受力的质点 $\bvec v_{A}$ 都不随时间变化。 若 $B$ 与 $A$ 之间静止或匀速运动且没有相对转动, 那么相对速度 $\bvec v_{AB}$ 也不随时间变化, 所以任何不受力的质点在 $B$ 中的速度 $\bvec v_{B}$ 也不随时间变化, 所以 $B$ 也是惯性系。 但若两系之间有任何相对的加速度(包括转动和加速平移),那么 $\bvec v_{AB}$ 将随时间或位置变化, 也就不能保证对任意的不受力质点 $\bvec v_B$ 都不随时间变化, 所以 $B$ 系就不是惯性系。

\subsection{第二定律}
\pentry{牛顿第二定律的矢量形式\upref{New2}}

牛顿第二定律只能在惯性系中使用, 在非惯性系中需要用惯性力\upref{Iner} 对合力进行修正才能成立。用矢量 $\bvec F$ 表示合力,牛顿第二定律记为
\begin{equation}\label{New3_eq1}
\bvec F = m\bvec a~.
\end{equation}
高中物理只强调圆周运动和直线运动, 所以一般 $F$ 和 $a$ 都记为标量, 请读者不要局限在高中思维\upref{New2}。 这是一条矢量表达式, 加速度矢量 $\bvec a$ 是位置矢量 $\bvec r$ 关于时间的二阶导数, 或者速度矢量 $\bvec v$ 关于时间的导数。 加速度和速度不必共线也不必垂直。

\subsubsection{动量定理}
牛顿本人对第二定律的表述使用了动量定理\upref{PLaw1},记质点的动量为 $\bvec p$,则
\begin{equation}\label{New3_eq2}
\bvec F = \dv{\bvec p}{t}~.
\end{equation}
在经典力学中,由于质量不发生变化,\autoref{New3_eq1} 和\autoref{New3_eq2} 是等效的,但令人惊讶的是,牛顿所用的形式在狭义相对论中仍然成立\footnote{在狭义相对论中,动量的定义有所不同。},而\autoref{New3_eq1} 却不成立。

\subsection{第三定律}
广义来说,牛顿第三定律就是动量守恒定律\upref{PLaw}。牛顿第三定律在任何参考系中都适用,但是要注意两点。第一,在非惯性系中,由于惯性力作为一个数学上的修正,并不是真正的力,所以不存在反作用力。第二,在考虑电磁力时,由于电磁场可能具有动量,所以动量守恒定律要求所有物体与电磁场的动量之和守恒,而不仅仅是质点的总动量守恒。在考虑两带电粒子的相互作用力时,若假设粒子的运动速度较慢,则磁场可以忽略,电磁场动量始终为零,此时两粒子的总动量守恒,相互作用力等大反向。
