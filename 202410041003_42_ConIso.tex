% 度量空间的连续映射和等距
% keys 连续|度量空间|等距
% license Usr
% type Tutor

\pentry{度量空间\nref{nod_Metric}}{nod_9cd3}
连续映射在分析学和拓扑学中都有定义。从逻辑上来说,度量空间是拓扑空间的特殊情形,分析学中的函数(数)空间是度量空间的特殊情形。因此,更好的学习方式是从底层理论开始学的。例如,“连续映射”的概念应当以拓扑学中的概念为最基本的概念,其它的情形的定义只不过是这一基本概念的特殊情形。然而,从现实来说,底层框架仅仅给出了构造世界的最基本构件,而生命(意识)的诞生需要在基本框架上附加更复杂的结构。因此从生命(意识)的认识来说,一开始接触到框架本身就是嵌套了额外的复杂结构,而这对于生命(意识)来说则更加感性具体。正如连续映射,从生命(意识)的认识来说,分析学中的函数的连续性则更加感性具体。这也解释为什么探寻真理的过程需要不断的进行抽象,寻找出其中更普遍的结构。因此,于我们而言,更合理的学习方式是从具有复杂结构的理论开始学习,再循序渐进的到更基本的理论中去。

因此,对连续映射的认识不是从拓扑空间中开始,而是以分析学中连续函数作为起点,再到达度量空间,最后才是拓扑空间。本节将从度量空间的视角去认识连续性的概念。

\subsection{连续映射}
函数的连续概念通过数与数之差的绝对值来定义,这一绝对值度量了两数之间的距离。在度量空间中,有距离的概念,因此,可以类似的建立起度量空间中连续映射的概念。
\begin{definition}{连续映射}
设 $(X_1,d_1),(X_2,d_2)$ 是两度量空间。

\end{definition}









