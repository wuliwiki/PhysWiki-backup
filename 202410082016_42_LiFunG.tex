% 线性泛函的几何意义
% keys 线性泛函|几何意义|超平面
% license Usr
% type Tutor

\pentry{余维数\nref{nod_Codim},泛函与线性泛函\nref{nod_Funal}}{nod_d70e}
在有限维线性空间中,一个线性方程和一个超曲面一一对应(\enref{线性方程组的仿射解释}{AS2LF}),这在无穷维的线性空间中仍然成立。具体的,在任一的线性空间中,一个非平凡线性泛函(即不恒为零)和一个不通过坐标原点的超曲面一一对应。这便是线性泛函的几何意义。

本文将始终假定 $\mathbb F$ 是定义线性空间 $L$ 的域。

\subsection{零子空间}
\begin{definition}{零子空间,核}
设 $f$ 是线性空间 $L$ 上不恒为零的线性泛函。则
\begin{equation}
\{x|f(x)=0,x\in L\}~,
\end{equation}
称为 $L$ 的(关于 $f$) 的\textbf{零子空间}或线性泛函 $f$ 的\textbf{核},记作 $\ker f$。
\end{definition}

称 $\ker f$ 为“子空间” 是因为若 $f(x)=f(y)=0$,则 
\begin{equation}
f(\alpha x+\beta y)=\alpha f(x)+\beta f(y)=0.~
\end{equation}


\begin{lemma}{}\label{lem_LiFunG_1}
设 $f$ 是线性空间 $L$ 的非平凡线性泛函,则任一 $x\in L$,对固定的满足 $f(x_0)\neq0$ 的 $x_0$,存为唯一 $\alpha\in\mathbb F,y\in \ker f$,使得
\begin{equation}
x=\alpha x_0+y.~
\end{equation}

在 $f,x_0$ 确定的情况下, $\alpha=\frac{f(x)}{f(x_0)}$
\end{lemma}

\textbf{证明:}\textbf{存在性:}任一 $x\in L$,令 $y=x-\frac{f(x)}{f(x_0)}x_0$,则 $f(y)=0$,即 $y\in\ker f$。因此任一 $x\in L$,存在 $\alpha\in\mathbb F,y\in \ker f$,使得
\begin{equation}\label{eq_LiFunG_1}
x=\alpha x_0+y.~
\end{equation}

\textbf{唯一性:}设 $\alpha x_0+y=\alpha' x_0+y'$,则 $(\alpha-\alpha')x_0=y'-y$。若 $\alpha=\alpha'$,则 $y'=y$,唯一性成立;若 $\alpha\neq\alpha'$,则 $x_0=\frac{y'-y}{\alpha-\alpha'}\in\ker f$,这和 $f(x_0)\neq0$矛盾,因此只能是 $\alpha=\alpha'$。 

\textbf{证毕!}

\begin{corollary}{}
设 $f$ 是线性空间 $L$ 的非平凡线性泛函, 则 $x$ 等价于 $x'$,当且将当 $f(x)=f(x')$。
\end{corollary}
\textbf{证明:}由于 $f$ 不恒为0,因此存在 $x_0$,使得 $f(x_0)\neq0$。由\autoref{lem_LiFunG_1} ,任意 $x,x'$,可表示为
\begin{equation}
x=\alpha x_0+y,\quad x'=\alpha' x_0+y'.~
\end{equation}
$x,x'$ 等价的条件是 $x-x'=(\alpha'-\alpha)x_0+(y-y')\in \ker f$,这要求 $\alpha'=\alpha$。此时 $f(x)-f(x')=f(y)-f(y')=0$,即 $f(x)=f(x')$。

\textbf{证毕!}


\begin{corollary}{ $\mathrm{codim} f=1$ }
零子空间 $\ker f$ 的\enref{余维数}{Codim}为 $1$。
\end{corollary}

\textbf{证明:}需要证明商空间 $L/\ker f$ 的维数为1,即存在非零矢量 $[x_0]\neq [0]\in L/\ker f$,使得任一 $\xi\in L/\ker f$,都有 $\xi=\alpha[x_0],\alpha\in\mathbb F$。证明如下:

由\autoref{lem_LiFunG_1} ,任一 $\xi\in L/\ker f$ 的代表元 $x$ 可写为 $x=\alpha x_0+y,y\in\ker f,x_0\notin\ker f$。又 $\alpha x_0$ 和 $x$ 等价,因此 $\alpha x_0\in \xi$,即每一 $\xi$ 都有形如 $\alpha x_0$ 的代表元。即 $[\xi]=\alpha[x_0]$。 

\textbf{证毕!}







\begin{corollary}{}
设 $f,g$ 是线性空间 $L$ 的非平凡线性泛函,则若 $\ker f=\ker g$,那么恒有 $g(x)=\frac{f(x_0)}{g(x_0)}f(x)$。即在不计常数因子的情况下,在 $\ker f$ 定义的线性泛函是唯一的。 
\end{corollary}
\textbf{证明:}取 $x_0$,使得 $f(x_0)\neq0$。那么可断定 $g(x_0)\neq0$。事实上,由\autoref{lem_LiFunG_1} 
\begin{equation}
\begin{aligned}
x=\frac{f(x)}{f(x_0)}x_0+y,&\quad y\in\ker f=\ker g,\\
&\Downarrow\\
g(x)=\frac{f(x)}{f(x_0)}g(x_0)&+g(y)=\frac{f(x)}{f(x_0)}g(x_0).
\end{aligned}~
\end{equation}
若 $g(x_0)=0$,那么 $g(x)\equiv0$,这与 $g$ 的非平凡性矛盾。由上式可知命题成立。

\textbf{证毕!}

\begin{definition}{超平面}
设 $L'$ 是 $L$ 的余维数为 1 的任意子空间,则称空间 $L'$ 的任一剩余类(\autoref{def_QuoSpa_2})为(与子空间 $L'$平行)的\textbf{超平面}。
\end{definition}
由剩余类的定义可知,与 $L'$ 平行的超平面 $M'$,是从 $L'$ 平移任一向量 $x_0\in L$ 所得到的集 $M'=x_0+L'$。



\begin{theorem}{}
定义在 $L$ 上的所有非平凡线性泛函与 $L$ 中不通过坐标原点的所有超平面是一一对应的。
\end{theorem}

\textbf{证明:} 若 $f$ 是空间 $L$ 上的非平凡线性泛函,则集 $M_f=\{x|f(x)=c\neq 0\}$ 是与子空间 $\ker f$ 平行的超平面:事实上,选定任一满足 $f(x_0)=c$ 的元素 $x_0$。于是由\autoref{lem_LiFunG_1} ,任一 $x\in M_f$ 可表为 $x=\frac{f(x)}{f(x_0)}x_0+y=x_0+y,y\in\ker f$。反过来, $x_0+\ker f\subset M_f$。于是 $M_f=x_0+\ker f$。

若 $M'$ 是与子空间 $L'$ 平行、且不通过原点的任一超曲面,则存在唯一的线性泛函,使得 $M'=\{x|f(x)=c\neq0\}$:设 $M'=L'+x_0,x_0\in L$/

\textbf{证毕!}



