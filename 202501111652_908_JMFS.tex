% 镜面反射(综述)
% license CCBYSA3
% type Wiki

本文根据 CC-BY-SA 协议转载翻译自维基百科\href{https://en.wikipedia.org/wiki/Specular_reflection}{相关文章}。

\begin{figure}[ht]
\centering
\includegraphics[width=6cm]{./figures/faeab0afd20b44ba.png}
\caption{镜面反射的共面条件,其中\(\theta_i = \theta_r\)} \label{fig_JMFS_1}
\end{figure}
镜面反射或规律反射是波(例如光)从表面反射的类似镜面反射的现象。[1]

反射定律指出,反射光线从反射表面射出时,与表面法线的夹角与入射光线的夹角相同,但位于表面法线的对侧,且在由入射光线和反射光线构成的平面内。这一行为最早由亚历山大的赫罗(公元约10年–70年)描述。[2] 后来,阿尔哈森给出了反射定律的完整表述。[3][4][5] 他首次指出,入射光线、反射光线和表面的法线都位于一个垂直于反射面平面的同一平面内。[6][7]

镜面反射可以与漫反射进行对比,后者是光从表面散射到多个方向的现象。
\subsection{反射定律}
\begin{figure}[ht]
\centering
\includegraphics[width=6cm]{./figures/1d039160dbdddcc6.png}
\caption{静止水面的反射是镜面反射的一个例子。} \label{fig_JMFS_2}
\end{figure}
当光遇到材料的边界时,它会受到该材料对电磁波的光学和电子响应函数的影响。光学过程,包括反射和折射,通常由边界两侧折射率的差异来表达,而反射率和吸收率则是由于材料的电子结构而产生的响应的实部和虚部。[8] 这些过程在传输中的参与程度是光的频率或波长、其偏振状态以及入射角的函数。一般而言,反射随着入射角的增加而增加,并且随着边界吸收率的增加而增强。菲涅尔方程描述了光学边界的物理过程。

反射可以是镜面反射(或镜面式反射)和漫反射。镜面反射将所有从给定方向到达的光反射回相同的角度,而漫反射则将光反射到广泛的方向。这个区别可以通过涂有光泽漆和哑光漆的表面来说明。哑光漆几乎完全表现出漫反射行为,而光泽漆则表现出较大比例的镜面反射行为。由非吸收性粉末(如石膏)构成的表面几乎可以作为完美的漫反射体,而抛光的金属物体则能够非常有效地进行镜面反射。镜子的反射材料通常是铝或银。
\begin{figure}[ht]
\centering
\includegraphics[width=6cm]{./figures/50207ae39b989f1b.png}
\caption{湿润金属球的镜面反射} \label{fig_JMFS_3}
\end{figure}
光在空间中传播时,表现为电磁场的波前。光线的方向由波前法线(波法线)决定。当光线遇到表面时,波法线与表面法线之间的夹角称为入射角,两个方向所定义的平面为入射平面。入射光线的反射也发生在入射平面内。

反射定律指出,光线的反射角等于入射角,且入射方向、表面法线和反射方向共面。

当光垂直入射到表面时,它会沿源方向直接反射回来。

反射现象源自平面波在平坦边界上的衍射。当边界的尺寸远大于波长时,边界处的电磁场仅在镜面方向上完全同相振荡。
\begin{figure}[ht]
\centering
\includegraphics[width=6cm]{./figures/1423b28efcc15a4d.png}
\caption{大理石球的漫反射} \label{fig_JMFS_4}
\end{figure}
\subsubsection{向量形式}
反射定律也可以通过线性代数等效地表达。反射光线的方向由入射向量和表面法线向量决定。给定从光源到表面的入射方向 \(\mathbf {\hat {d}}_{\mathrm {i}}\) 和表面法线方向 \(\mathbf {\hat {d}}_{\mathrm {n}}\),镜面反射方向 \(\mathbf {\hat {d}}_{\mathrm {s}}\)(所有单位向量)为:[9][10]
\[
\mathbf {\hat {d}}_{\mathrm {s}} = \mathbf {\hat {d}}_{\mathrm {i}} - 2 \mathbf {\hat {d}}_{\mathrm {n}} \left( \mathbf {\hat {d}}_{\mathrm {n}} \cdot \mathbf {\hat {d}}_{\mathrm {i}} \right)~
\]
其中,\(\mathbf {\hat {d}}_{\mathrm {n}} \cdot \mathbf {\hat {d}}_{\mathrm {i}}\) 是通过点积得到的标量。不同的作者可能会用不同的符号来定义入射方向和反射方向。假设这些欧几里得向量以列向量的形式表示,方程可以等效地表达为矩阵-向量乘法形式:[11]
\[
\mathbf {\hat {d}}_{\mathrm {s}} = \mathbf {R} \; \mathbf {\hat {d}}_{\mathrm {i}}~
\]
其中,\(\mathbf {R}\) 是所谓的霍斯特变换矩阵,定义为:
\[
\mathbf {R} = \mathbf {I} - 2 \mathbf {\hat {d}}_{\mathrm {n}} \mathbf {\hat {d}}_{\mathrm {n}}^{\mathrm {T}}~
\]
其中 \(\mathbf {I}\) 是单位矩阵,\(\mathbf {\hat {d}}_{\mathrm {n}} \mathbf {\hat {d}}_{\mathrm {n}}^{\mathrm {T}}\) 是法线向量的外积。