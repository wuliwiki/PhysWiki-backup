% 拉格朗日力学 (综述)
% license CCBYSA3
% type Wiki

本文根据 CC-BY-SA 协议转载翻译自维基百科\href{https://en.wikipedia.org/wiki/Lagrangian_mechanics}{相关文章})

在物理学中,拉格朗日力学是一种基于平稳作用原理(也称为最小作用原理)的经典力学表述。它由意大利-法国数学家和天文学家约瑟夫-路易·拉格朗日在1760年向都灵科学学院的演讲中提出,并在1788年他的巨著《解析力学》中达到巅峰。

拉格朗日力学将一个力学系统描述为由构型空间 \( M \) 和其中的一个光滑函数 \( L \)(称为拉格朗日量)组成的对 \( (M, L) \)。对于许多系统,\( L = T - V \),其中 \( T \) 和 \( V \) 分别为系统的动能和势能。

平稳作用原理要求由 \( L \) 推导出的系统的作用泛函在系统的时间演化过程中保持在一个平稳点(最大值、最小值或鞍点)。这一约束条件使得可以利用拉格朗日方程计算系统的运动方程。
\subsection{简介}
\begin{figure}[ht]
\centering
\includegraphics[width=6cm]{./figures/dd4713438c2a5e46.png}
\caption{“约瑟夫-路易·拉格朗日 (1736–1813)”} \label{fig_LGL_1}
\end{figure}
一个珠子被限制在无摩擦的细线上运动。细线对珠子施加一个反作用力 \( C \),以保持其在细线上。此时的非约束力 \( N \) 是重力。注意,珠子在细线上的初始位置可能会导致不同的运动。

简单摆。由于杆是刚性的,摆锤的位置受到方程 \( f(x, y) = 0 \) 的约束,约束力 \( C \) 是杆中的拉力。同样,此时的非约束力 \( N \) 也是重力。

牛顿定律和力的概念通常是讲授机械系统的起点【5】。这种方法适用于许多问题,但对于一些问题,计算变得极其复杂【6】。例如,在计算一个带珍珠的圆环在水平面上滚动的运动时,时间变化的约束力(例如圆环的角速度、珍珠相对于圆环的运动)使得利用牛顿方程难以确定圆环的运动【7】。拉格朗日力学采用能量而非力作为基本成分【5】,从而得到更抽象的方程,能够处理更复杂的问题【6】。

特别是,拉格朗日的方法是为每个物体的位移和速度建立独立的广义坐标,从而写出一个拉格朗日量(系统的总动能减去势能)的通式,并对所有可能的粒子运动路径进行求和,得到‘作用’的公式,通过极小化作用得到一组广义方程。这一求和的量在粒子实际采取的路径上被最小化。这一选择消除了将约束力引入最终广义方程组的必要性。这样,方程减少了,因为不需要直接计算给定时刻约束对粒子的影响【7】。
\begin{figure}[ht]
\centering
\includegraphics[width=7cm]{./figures/6b06deede1a9c42f.png}
\caption{“珠子被限制在无摩擦的细线上运动。细线对珠子施加一个反作用力 \( C \) 以使其保持在细线上。在这种情况下,非约束力 \( N \) 是重力。注意,珠子在细线上的初始位置可能会导致不同的运动。”} \label{fig_LGL_2}
\end{figure}
对于各种物理系统,如果一个物体的大小和形状可以忽略不计,将其视为质点是一种有用的简化。对于由质量为 \( m_1, m_2, \dots, m_N \) 的 \( N \) 个质点组成的系统,每个质点都有一个位置向量,分别记为 \( r_1, r_2, \dots, r_N \)。笛卡尔坐标系通常足够,因此 \( r_1 = (x_1, y_1, z_1), r_2 = (x_2, y_2, z_2) \) 等。在三维空间中,每个位置向量需要三个坐标来唯一确定一个点的位置,因此需要 \( 3N \) 个坐标来唯一确定系统的构型。这些坐标是粒子在空间中的具体位置;空间中的一个通用点写作 \( r = (x, y, z) \)。每个粒子的速度是其沿运动路径的移动速度,为其位置的时间导数,因此
\[
\mathbf{v}_1 = \frac{d\mathbf{r}_1}{dt}, \quad \mathbf{v}_2 = \frac{d\mathbf{r}_2}{dt}, \dots, \quad \mathbf{v}_N = \frac{d\mathbf{r}_N}{dt}.~
\]
在牛顿力学中,运动方程由牛顿定律给出。第二定律“合力等于质量乘加速度”适用于每个粒子,
\[
\sum \mathbf{F} = m \frac{d^2 \mathbf{r}}{dt^2},~
\]
对于一个三维的 \( N \) 粒子系统,这样的方程组成了关于粒子位置的 \( 3N \) 个二阶常微分方程,需要求解。