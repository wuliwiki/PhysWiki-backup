% 仿射集
% keys 仿射|直线
% license Xiao
% type Tutor

\begin{issues}
\issueTODO
\end{issues}

\pentry{基底(线性代数)\nref{nod_VecSpn}}{nod_6f36}

\addTODO{增加一个简介}
仿射集是矢量空间中的一类子集,它起源于线性代数和几何学的共同发展。它是研究向量空间中点的集合性质的一种方式。在更广泛的背景下,仿射几何已经成为几何学的一个重要分支。

\begin{definition}{仿射组合与仿射集}\label{def_AffSet_1}
取向量空间 $V$ (记其域为 $\mathbb{F}$) 中的两点 $x_1, x_2 \in V$ 的线性组合 $\theta x_1 + (1 - \theta) x_2$ 被称为 $x_1, x_2$ 的\textbf{仿射组合};更一般的,系数和为一(即$\sum_i a_i = 1$)的线性组合 $\sum_i a_i x_i$ 被称为 $x_1, \dots, x_n$ 的\textbf{仿射组合}。

向量空间的子集 $C \subseteq V$ 被称为\textbf{仿射集}(affine set),意味着 $C$ 中的任意仿射组合都在 $C$ 中;等价的,我们只需要考虑任意两个向量的仿射组合即可(和向量子空间的情况一样)。
\end{definition}

\begin{figure}[ht]
\centering
\includegraphics[width=12cm]{./figures/a3c7ce94fa5d9e0a.png}
\caption{仿射集示意图} \label{fig_AffSet_1}
\end{figure}

图$1$表示的是一条穿过$x_1,x_2$两点的直线。当$0 \leq \theta \leq 1$,形成图中直线加粗的部分;反之,形成直线上细线表示的部分。

从几何上看,仿射集仍然是平直的,或者说“线性的”:

\begin{theorem}{}
对任意的仿射集 $C \subseteq V$,存在唯一的向量子空间 $U \subseteq V$,使得对任意的 $x \in C$,我们有
\begin{equation}
C = x + U = \{ x + v \mid v \in U\}~.
\end{equation}
\end{theorem}

\textbf{证明:}第一步,取一点 $x_0 \in C$,我们定义
\begin{equation}
U: = \{ x - x_0 \mid x \in C \}~,
\end{equation}
要证明它是一个向量子空间:
\begin{enumerate}
\item $0_V = x_0 - x_0 \in U$,
\item 对任意的 $a, b \in \mathbb{F}$,$x, y \in C$,
    \begin{equation}
    \begin{aligned}
    a (x - x_0) + b (y - x_0) &= a x + b y - (a + b - 1) x_0 - x_0 \\
    &= a x + b y + (1 - (a + b)) x_0 - x_0 \\
    \end{aligned}~
    \end{equation}
$a x + b y + (1 - (a + b)) x_0$ 是 $x, y, x_0$ 的仿射组合,因此 $a (x - x_0) + b (y - x_0) \in U$,证得。
\end{enumerate}

第二步,要证明对任意的 $x \in C$,$C = x + U$:首先证明 $\subseteq$,对任意的 $y \in C$,我们有
\begin{equation}
y = x + (y - x_0) - (x - x_0); ~
\end{equation}
现在证明 $\supseteq$,对任意的$u  = z - x_0 \in U$,我们有
\begin{equation}
x + u = x + z + (-1) x_0 ~
\end{equation}
是 $x, z, x_0$ 得仿射组合,证得。

第三步,要证明 $U$ 不依赖于 $x_0$ 的选取——对任意的 $x' \in C$,定义 $U': = \{x - x' \mid x \in C\}$,我们要证明 $U' = U$,处于对称性我们只需要证明 $U' \subseteq U$:取 $x \in C$,
\begin{equation}
x - x' = (x + x_0 - x') - x_0 ~
\end{equation}
$x + x_0 - x'$ 是 $x, x_0, x'$的仿射组合,因此证得。

第四步,证明$U$的唯一性:假设 $V$ 存在另外一个向量子空间 $U'$ 满足对任意 $x \in C$,$C = x + U'$,取 $x = x_0$,那么 $x_0 + U' = x_0 + U \implies U' = U$。

\textbf{证毕!}

因此我们可以定义仿射集的维度
\begin{definition}{维度}
对于仿射集 $C \subseteq V$,我们定义它的\textbf{维度}为它对应的向量空间的维度。

特别的,一维的仿射集被称为\textbf{仿射直线},二维的被称为\textbf{仿射平面},余一维的被称为\textbf{仿射超平面}。
\end{definition}

在向量空间中谈直线、平面、超平面时,我们有时指的是向量子空间,有时指的是仿射子空间,要注意分辨。

\begin{theorem}{}\label{the_AffSet_1}
对于向量空间的子集 $C \subseteq V$,$C$为仿射集当且仅当过 $C$ 中任意不同的两点的(仿射)直线仍然在 $C$ 中。
\end{theorem}

\textbf{证明:}\textbf{必要性:}假设 $C$ 是仿射集。那么任意由 $x_1,x_2\in C$ 确定的直线为 $x_1+(1-k)x_2,k\in\mathbb F$,即是 $x_1,x_2$ 的仿射组合,因此由仿射集的定义(\autoref{def_AffSet_1} ),该直线属于 $C$。

\textbf{充分性:}设 $C$ 是过其上任意两点的直线都在 $C$ 中的集。那么任意 $x_i\in C,i=1,\cdots ,n$,对任意满足 $\sum_{i}a_i=1$ 的 $a_i\in\mathbb F$,成立(不失一般性,设 $a_1\neq1$)
\begin{equation}\label{eq_AffSet_1}
\sum_{i=1}^n a_ix_i=(1-a_1)\sum_{i=2}^n \frac{a_i}{1-a_1}x_i+a_1x_1~.
\end{equation}
由数学归纳法,设对 $n=k-1$ 命题成立。那么 $\sum_{i=2}^n \frac{a_i}{1-a_1}x_i\in C$,又\autoref{eq_AffSet_1} 是 $\sum_{i=2}^n \frac{a_i}{1-a_1}x_i$ 和 $x_1$ 的仿射组合,因此 $\sum_{i=1}^n a_ix_i\in C$。由仿射集的定义,$C$ 是仿射集

\textbf{证毕!}


\begin{corollary}{}\label{cor_AffSet_1}
设 $U\subset V$ 是线性空间 $V$ 的子空间,$x\in V$,则 $x+U$ 是仿射集。 
\end{corollary}

\textbf{证明:}任意 $x_1,x_2\in x+U$,设 $x_1=x+y_1,x_2=x+y_2$,其中 $y_1,y_2\in U$。则过 $x_1,x_2$ 的直线为
\begin{equation}
\{x_1+k(x_2-x_2)|k\in\mathbb F\}=\{x+(y_1+k(y_2-y_1))|k\in\mathbb F\}\subset x+U~.
\end{equation}
因为 $x_1,x_2$ 任意,所以由\autoref{the_AffSet_1} ,命题得证!

\textbf{证毕!}




