% 多模光纤
% license CCBYSA3
% type Wiki
(本文根据 CC-BY-SA 协议转载自原搜狗科学百科对英文维基百科的翻译)

多模光纤是一种主要用于短距离通信的光纤。典型的多模式链路在长达$600$米($2000$英尺)的链路长度上具有$10 Mbit/s$至$10 Gbit/s$的数据传输速率。多模光纤具有相当大的纤芯直径,这就使得其能够传播多种光模式,并且由于模式色散而限制了传输链路的最大长度。

\subsection{应用}
用于多模光纤通信的设备比用于单模光纤的设备便宜。[1] 典型的传输速度与传输距离的极限关系为在传输距离高达$2km$情况下速度极限为$100 Mbit/s$,在$1000m$的情况下速度极限为$1 Gbit/s$,而$550m$的情况下速度极限为$10 Gbit/s$。

由于其高容量和可靠性,多模光纤通常在建筑物中作为主要的通信应用工具。越来越多的用户正收获到通过光纤将桌面或区域间联系起来所带来的用户们之间越来越近的通信交互的好处。符合标准的体系结构(如集中布线和光纤到电信机柜)使用户能够通过将电子设备集中在电信机房而不是在每层都安装有源电子设备来利用光纤的距离传输能力。

多模光纤用于向/从微型光纤光谱设备(光谱仪、光源和采样附件)传输光信号,并在第一台便携式光谱仪的开发中发挥了重要作用。

多模光纤也用于高功率光的传输,如激光焊接。
\subsection{与单模光纤的比较}
多模光纤和单模光纤的主要区别在于前者的纤芯直径要大得多,通常为$50-100$微米;且其能够传播更大波长的光。由于纤芯大,及大数值孔径的可能性,多模光纤比单模光纤具有更高的“聚光”能力。实际上,较大的纤芯尺寸简化了连接,还允许使用低成本电子器件,例如工作在850纳米和1300纳米波长的发光二极管(LEDs)和垂直腔面发射激光器(VCSELs)(电信中使用的单模光纤通常工作在1310或1550纳米 [2])。然而,与单模光纤相比,多模光纤带宽-距离乘积限制更低。因为多模光纤比单模光纤具有更大的纤芯尺寸,所以它支持多种传播模式;因此,多模光纤受到模式色散的限制,而单模却不受。

与多模光纤一起使用的LED光源有时会产生一系列不同波长的光波,并且每个波长的波以不同的速度传播。这种色散是对多模光纤电缆可用长度的另一种限制。相比之下,用于驱动单模光纤的激光器则产生单一波长的相干光。由于模式色散,多模光纤比单模光纤具有更高的脉冲展宽速率,这就限制了多模光纤的信息传输能力。

单模光纤通常用于高精度的科学研究,因为将光限制在单一传播模式下,可以使其聚焦在一个强烈的衍射受限的点上。

电缆的外皮颜色有时可用于区分多模电缆和单模电缆。在非军事领域中应用的TIA-598C标准建仪电缆外皮的颜色具体取决于所适用的光纤的种类,单模光纤使用黄色外皮,多模光纤使用橙色或浅绿色外皮。 一些供应商使用紫色来区分更高性能的OM4通信光纤和其他类型的光纤。[3]

