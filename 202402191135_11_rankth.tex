% 常秩定理
% license Usr
% type Tutor

\begin{theorem}{}
设$U\subset \mathbb R^m$是点$x_0$的邻域。若$f:U\rightarrow R^n$是$ C^p(U)(p>1)$类映射且在该邻域内每一点有相同的秩$k$\footnote{为切映射$Df$的秩},则存在点$x_0$和点$f(x_0)=y_0$的邻域$O(x_0)$和$O(y_0)$以及定义在对应邻域的$C^p$类\textbf{微分同胚:}
\begin{equation}
\begin{aligned}
\phi:O(x_0) \rightarrow O_(u_0)
\psi:O(y_0) \rightarrow O_(v_0)~.
\end{aligned}
\end{equation}
映射$v=\psi\circ f\circ \phi^{-1}$具有如下典则形式:
\begin{equation}
v=(v^1,v^2...v^n)=(u^1,u^2...u^k...0,0,...0)~.
\end{equation}
\end{theorem}
