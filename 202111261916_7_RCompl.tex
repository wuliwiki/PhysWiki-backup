% 实数的完备公理
% 戴德金分割|区间套|闭区间套|确界原理|单调有界定理|有限覆盖定理|聚点定理|致密性定理|柯西收敛准则|Cauchy收敛准则




我们常见到一种争论,$0.\dot{9}$到底等不等于$1$.实际上这个问题无法证明,而是被当作定义实数的公理之一,我们称之为完备性公理.你可以留意一下各种各样所谓的“证明”,认为$0.\dot{9}\neq 1$的论证通常都是否认了完备公理,而认为$0.\dot{9}=1$的都默认了完备公理.

用$0.\dot{9}=1$来当作完备公理很不好用,我们通常使用本节介绍的的完备公理来描述实数的完备性,这些公理彼此是等价的,并且都可以推出$0.\dot{9}$.

\subsection{完备公理的表述}

由于几个完备公理是等价的,可以互推,因此实际建立理论时只挑其中一个作为公理体系的一部分,其它的都当作定理,这也导致我们也常把这几条公理称为“定理”.因此,我们使用“定理”的格式来列举这几条完备公理.

\begin{theorem}{确界原理}
实数集的任何子集,必有上确界(见\textbf{上确界与下确界}\upref{SupInf}).
\end{theorem}

\begin{theorem}{单调有界收敛定理}
单调有界数列必有极限.
\end{theorem}

\begin{theorem}{(闭)区间套定理}
设$a_n$是单调递增数列,$b_n$是单调递减数列,$b_n-a_n$恒为正数且收敛到$0$,且$[a_{n+1}, b_{n+1}]$都是$[a_n, b_n]$的真子集.称这样的集合$\{[a_n, b_n]_{n=1}^\infty\}$为一个\textbf{(闭)区间套}.

对于任意的区间套$\{[a_n, b_n]\}$,存在唯一的实数$x_0$使得$x_0\in [a_n, b_n]$对任意正整数$n$成立.
\end{theorem}















