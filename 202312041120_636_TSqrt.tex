% 手动计算开根号(泰勒展开法)
% license MIT
% type Wiki

\pentry{泰勒展开\upref{Taylor}}

\begin{theorem}{幂函数的泰勒展开}

$$(1+x)^k =  1 + \frac{k}{1!} x + \frac{k(k-1)}{2!} x^2 + \frac{k(k-1)(k-2)}{3!} x^3 + \cdots~.$$

\end{theorem}

泰勒展开是一种常用的“拟合”方法,对于一个幂函数:$\left(1+x\right) ^k$,当 $x$ 极小的时候(可以表示为 $x \rightarrow 0$),就会有:

$$(1+x)^k \approx 1 + kx + \frac{k(k-1)}{2!}x ~,$$

te'bi

我们的手动计算根号的这种方法就是由此得来。

下面举例子来更明确的说明如何使用这种方法计算:
\begin{example}{平方根的手动计算}
例如对于 $\sqrt{101}$,我们要考虑将其拆为 $\sqrt{1+ \Delta x}$ 的形式(其中 $\Delta x$ 极小)。
我们找到一个距离 $101$ 最近的平方数,即 $10^2 = 100$,之后将这 $100$ 提出,之后就一定可以得到一个 $$
$$\sqrt{101} = \sqrt{100} \times \sqrt{\frac{101}{100}}  = 10 \sqrt{1+\frac{1}{100}}~,$$
也就可以考虑对
\end{example}