% C++ 标准库笔记
% license Xiao
% type Note

\subsection{string}
详见 “C++ 字符串笔记\upref{CppStr}”。

\subsection{vector}
更多容器详见 “C++ 标准库常用容器\upref{STLstr}”
\begin{itemize}
\item \verb`v.size()` 检查大小
\item \verb`v.empty()` 检查是否为空
\item range based for loop \verb`for (auto &i : v)`
\item iterator 类似指针, \verb`auto a = v.begin()`, 生成第一个元素的 iterator, \verb`v.end()` 生成最后一个元素后的一个 iterator. \verb`*a` 获取 a 指向的元素, \verb`*(a + n)` 获取 a 后面的第 n 个元素. \verb`==` 只有在 iterator 指向同一个 vector 的同一个元素时才成立.
\end{itemize}

\subsection{ctime}
\begin{itemize}
\item \verb`clock()` 用于测量 CPU 时, 而不是真正的时间. 如果在 ubuntu 下用 OpenMP, 时间将是所有 CPU 的累加.
\begin{lstlisting}[language=cpp]
clock_t start, stop;
start = clock();
...
stop = clock();
\end{lstlisting}
\item \verb`time()` 用于测量从 1970 年某时刻起流逝的秒数, 但仅限于整数秒.
若要精确测量物理时间, 用 chrono, 这个头文件用起来要复杂得多, 见我在 nr3plus.h 中写的 tic(), toc() 函数.
\end{itemize}

\subsection{algorithm}
\begin{itemize}
\item 提供 \verb`max()`, \verb`min()`, \verb`swap()` 交换两个变量值。
\end{itemize}


\subsection{stdlib}
\begin{itemize}
\item \verb`int system("命令")` 在调用程序的环境(如 linux 的 bash,windows 的 cmd)中执行命令,例如获得文件列表 \verb`ls`, 当前目录 \verb`pwd` 等。 但这些命令返回的值不能直接获得,需要 redirect 到文件再读取文件。
\item \verb`int remove("文件名")` 可以删除文件。
\end{itemize}

\subsection{std::variant}
\begin{itemize}
\item 需要 c++17 标准
\item \verb|sizeof(variant)| 会比类型列表中最大的 sizeof 稍大。
\item \verb|if (std::holds_alternative<int>(变量))| 用于判断 \verb|variant| 的类型。
\item 获取值: \verb|std::get<int>(变量)| 或者 \verb|std::get<2>(变量)| (数字表示第几个类型),注意不能转换 \verb|&变量| 指针。 类型检查失败会抛出 \verb|std::bad_variant_access|
\item \verb|std::get_if<int>(&v)| 可以返回特定类型的指针, 如果不是 \verb|int| 则返回 \verb|nullptr|。
\item 如果用 \verb|if (holds_alternative<int>(变量)) get<int>(...);| 那么会检查两次类型,所以 \verb|get_if| 是更好的选择,后者是 \verb|noexcept|,抛出错误也会让程序变慢。
\item 支持 \verb|=| 赋值, \verb|>,<,|
\end{itemize}
\begin{lstlisting}[language=cpp]
#include <variant>
using namespace std;

using Operand = variant<int, double>;

Operand performOperation(const Operand& left,
    const Operand& right, const string& op) {
    // Visitor that applies the operation
    return visit([&](auto&& arg1, auto&& arg2) -> Operand {
        if (op == "+") return arg1 + arg2;
        else if (op == "-") return arg1 - arg2;
        else if (op == "*") return arg1 * arg2;
        else if (op == "/") {
            if (arg2 == 0) {
                cout << "Cannot divide by zero!" << endl;
                return 0;
            }
            return arg1 / arg2;
        }
        else {
            cout << "Unsupported operation" << endl;
            return 0;
        }
    }, left, right);
}
\end{lstlisting}
\begin{itemize}
\item \verb|lambda| 函数被调用的时候, \verb|arg1, arg2| 的类型会是 \verb|variant| 的具体类型。
\item 为什么不用 \verb`visit` 直接写 lambda 里面的东西? 因为这些代码编译时,stack 中的每个变量类型和长度都是确定的,\verb|+-*/| 调用的具体算法也是确定的,不可能同一个机器码在运行时用不同的类型。 所以不用 \verb|visit| 就必须手写 \verb|if, else|, 以及手动把 \verb|variant| cast 成不同的类型。\upref{StaDyn}
\item 而 \verb|std::visit| 相当于用 \verb|if, else if| 遍历 lambda 中的所有组合,并自动把变量 cast 成动态的类型。 lambda 中的操作需要适用与所有组合或者使用静态类型判断,否则就会编译报错。
\end{itemize}

\subsection{std::any}
\begin{itemize}
\item 需要 c++17 标准
\end{itemize}
\begin{itemize}
\item \verb|any| 可以是任何类型, 所以不可能用 \verb|std::visit| 遍历,需要自己手动判断。
\item \verb|sizeof(any)| g++ 测试是 16 字节,但也可能更多。
\item \verb|std::any::reset()| 可以清空它的值
\item \verb|std::any::has_value()| 判断是否有值
\item 判断类型: \verb|if (a.type() == typeid(int))|
\item 获取值 \verb|std::any_cast<int>(a)|,若错误抛出 \verb|std::bad_any_cast|
\item 获取指针 \verb|std::any_cast<int>(&a)|, 若错误返回 \verb|nullptr|。
\end{itemize}
