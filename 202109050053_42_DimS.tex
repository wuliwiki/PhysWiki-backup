% 量纲空间
% keys 量纲空间

量纲式\upref{DIMF}
\autoref{DIMF_the1}~\upref{DIMF}告诉我们,在选定一个有 $l$ 个基本量类的单位制族后,每个量类 $\tilde{\boldsymbol{A}}$ 便对应于 $l$ 个实数(量纲指数) $\sigma_1,\cdots,\sigma_l$ ,可记作 $\tilde{\boldsymbol{A}}=(\sigma_1,\cdots,\sigma_l)$ .反之,任意给定 $l$ 个实数列 $(\sigma_1,\cdots,\sigma_l)$ ,可把它看作量纲指数以得到一个“量类”.于是“量类”与实数列 $(\sigma_1,\cdots,\sigma_l)$ 有一个一一对应的关系.
\begin{definition}{}
对于有 $l$ 个基本量类的单位制族 $\tilde{\mathscr{Z}}$,l维的矢量空间 $\mathscr{L}$ 称为定义在 $\tilde{\mathscr{Z}}$ 上的\textbf{量纲空间}.
\end{definition}
量纲空间中的每个点 $(\sigma_1,\cdots,\sigma_l)$ 对应一个 "量类",称之为\textbf{数学量类},为区别于物理量类(字母上加 $~ 表示),