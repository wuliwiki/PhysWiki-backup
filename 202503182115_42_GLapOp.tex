% 高维弯曲空间中的拉普拉斯算符
% keys 高维空间|弯曲空间|拉普拉斯算符
% license Usr
% type Map

\pentry{拉普拉斯算符\nref{nod_Laplac}}{nod_057a}
\enref{三维空间中的拉普拉斯算符}{Laplac} $\Delta$ 由下式所定义:
\begin{equation}
\Delta u:=\Nabla\cdot (\Nabla u)=\frac{\partial^{2}{u}}{\partial{x}^{2}} + \frac{\partial^{2}{u}}{\partial{y}^{2}} + \frac{\partial^{2}{u}}{\partial{z}^{2}}.~
\end{equation}
其中 $u$ 是空间坐标的函数。拉普拉斯算符在求解许多物理问题时都会出现,包括真空中电势满足的\enref{泊松方程}{EPoiEQ},\enref{定态薛定谔方程}{SchEq}和线性化的爱因斯坦场方程(作为d'$\Lambda$)。








