% 从集合论角度看随机事件
% 集合论|概率论|随机样本
\pentry{定积分\upref{DefInt}}
在概率论问题中,我们通常要处理各种各样的时间。比如,我们要问,当掷出一个骰子时,点数大于四的概率是多少?或者当掷出两个骰子,一个点数大于二,另一个点数小于三的概率是多少?在这里我们将处理各种事件的想法与集合论中集合的运算相对应,来看待如何从集合运算视角处理概率论中的问题。
\subsection{样本空间}
我们定义在随机试验过程中,每次获得的一个数据称为一个样本,或称一个样本点。所有可能的样本点所构成的集合称为样本空间
\begin{equation}
S = \{e_1,e_2,...,e_n\}~.
\end{equation}
我们称抽样过程中的某一个事件是由一组样本点所构成的子集。定义事件$A$
\begin{equation}
A = \{e_1,e_2,..e_i\}~.
\end{equation}
其中当$A=\{\emptyset\}$时,我们称A为不可能事件。而当$A=S$时,我们称A为必然事件。如果$A$中只包含一个样本,则称$A$为
在这里我们多说几句这个定义的一些想法,我们在

