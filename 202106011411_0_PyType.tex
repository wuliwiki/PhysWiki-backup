% Python 基本变量类型
% Python|变量类型|整数|浮点|字符

\begin{issues}
\issueDraft
\end{issues}

Python 基本变量类型有(未完成), 我们可以用 \verb|type()| 函数查看某个变量的类型. 例如执行
\begin{lstlisting}[language=python]
n = 123; x = 3.14; print(type(n)); print(type(x))
\end{lstlisting}
或者用 \verb|is| 判断变量是否为某个类型
\begin{lstlisting}[language=python]
type(i) is int # true
\end{lstlisting}
结果为
\begin{lstlisting}
<class 'int'>
<class 'float'>;
\end{lstlisting}

\subsection{整数}
与一些编译语言不同, Python 的整数类型(integer)没有长度限制(除超出了内存大小). 例如
\begin{lstlisting}[language=python]
print(12345678901234567890123456789 + 1)
\end{lstlisting}
的结果为 \verb|12345678901234567890123456790|.

默认情况下整数用十进制表示, 如果需要输入 \textbf{2 进制(binary)}, 可以在前面加 \verb|0b| 或 \verb|0B|, 例如 \verb|0b1001| 表示 10 进制的 \verb|9|. 类似地, \verb|0o| 或 \verb|0O| 开头表示 \textbf{8 进制(octal)}; \verb|0x| 或 \verb|0X| 开头的表示 \textbf{16 进制(hexadecimal)}, 16 进制中的 10 到 15 分别用大写或小写字母 \verb|a| 到 \verb|f| 表示. 例如 \verb|0xff| 表示十进制的 \verb|255|. 不同进制的整数同样没有长度限制.

\subsection{类型转换}
转换格式为 \verb|类型(变量)|. 例如 \verb|int('123')| 会把字符串 \verb|'123'| 变为整数 \verb|123|.

\subsection{变量和对象}
和 C 语言类似的语言不同, Python 中的变量和对象(object)是分开考虑的. 对象可以理解为内存上的一小段某类型的数据, 而变量是指向对象的一个个名称. 例如两个变量可以指向同一个对象, 要判断变量 \verb|a| 和 \verb|b| 是否指向同一个对象, 用 \verb|a is b|, 返回一个 \verb|bool|.
