% 西梅翁·德尼·泊松(综述)
% license CCBYSA3
% type Wiki

本文根据 CC-BY-SA 协议转载翻译自维基百科 \href{https://en.wikipedia.org/wiki/Sim\%C3\%A9on_Denis_Poisson}{相关文章}。

西缅·德尼·泊松男爵(Baron Siméon Denis Poisson,/pwɑːˈsɒ̃/,美式亦作 /ˈpwɑːsɒn/,法语发音:[si.me.ɔ̃ də.ni pwa.sɔ̃];1781年6月21日-1840年4月25日)是一位法国数学家和物理学家,研究领域包括统计学、复分析、偏微分方程、变分法、解析力学、电磁学、热力学、弹性力学与流体力学。此外,他在试图反驳奥古斯丁·让·菲涅耳的波动理论时,意外预测了“阿拉戈斑点”的存在。
\subsection{生平}
泊松出生于法国卢瓦雷省皮蒂维耶(今属卢瓦雷省),是法国陆军军官西缅·泊松的儿子。

1798年,他以年级第一的成绩进入巴黎综合理工学院,很快便引起校内教授们的注意,并被允许自由选择研究方向。在入学不到两年的最后一年学习中,他发表了两篇论文:一篇是关于埃蒂安·贝祖消元法的研究,另一篇是关于有限差分方程积分个数的探讨。这些成果令教授们极为赞赏,以至于他在1800年未参加毕业考试的情况下被特别准许毕业。 第二篇论文由西尔维斯特-弗朗索瓦·拉克鲁瓦和阿德里安-玛丽·勒让德审阅,并推荐发表在《外国学者文集》上——这对当时年仅十八岁的泊松来说是史无前例的荣誉。

这一成功迅速使泊松进入了科学界。约瑟夫-路易·拉格朗日在综合理工学院开设函数论课程,泊松是他的听众之一,很早便发现了他的才华,并成为他的朋友。同时,泊松亦追随皮埃尔-西蒙·拉普拉斯的学术道路,后者几乎将他视为自己的儿子。

此后,泊松的余生一直在巴黎附近的索镇从事科研工作,撰写并发表了大量学术成果,并陆续担任了多个教育机构的重要职位,直至去世。[4]

