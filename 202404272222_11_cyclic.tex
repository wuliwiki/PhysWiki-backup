% 循环群
% license Usr
% type Tutor
\subsection{循环群}
\begin{definition}{}
若群$G$的元素都是由某个元素生成,即$G=<a>$,则称群$G$为\textbf{循环群}(cyclic group),称$a$是循环群的生成元。
\end{definition}

显然,循环群的群元都可以表示为生成元的整数次幂,因此循环群实际上是阿贝尔群。
\begin{example}{}
整数群$\mathbb Z$是有无限元素的循环群,群乘法为加法。
\end{example}
\begin{example}{}
模$n$同余类$\mathbb Z_n$。
\end{example}
\begin{example}{}
群$G=\{-1,-\mathrm i,1,\mathrm i\}=<\mathrm i>$。
\end{example}
循环群的形式看似没有什么规律,但我们可以借助同构来缩减研究对象。
\begin{theorem}{}
无限循环群同构于整数加群;$n$元有限循环群同构于$\{\mathbb Z_n;+\}$。
\end{theorem}
\textbf{proof.}\footnote{参考《抽象代数》,邓少强祝,朱富海著。}

设$G$是无限循环群,建立$\mathbb Z\rightarrow G$的同态映射,使得对于任意$n\in \mathbb Z$,都有$f(n)=a^n$。根据群同态基本定理\autoref{exe_Group2_1}~\upref{Group2},我们有$\mathbb Z/\opn{ker}f\cong G$。由于$\mathbb Z$的正规子群都是$n\mathbb Z,n\in \mathbb N$,因此模$n$同余类与$G$同构。当该$n=0$时,对应无限循环群;当$m\neq 0$时,$\mathbb Z_n$与n元循环群同构。

因为有限循环群可以继承整数群的乘法,因此还是一个环。可以证明,$\mathbb Z_n$环上的零因子是$n$的因子。所以,如果$n$是素数,那么这个环就是\textbf{无零因子交换幺环}了,我们一般简称其为\textbf{整环}。
\begin{theorem}{}
有限整环必是域。
\end{theorem}
只要证明任意环元都有逆在环内即可。
因为是有限整环,假设生成元为$a$,则由封闭性知对于每个非零同余类都必有$a^m=a^n,m\neq n$。设$m> n$,因为$a^{m-n}a^n=1\cdot a^n$,所以$a^{m-n}=1$\footnote{若对于环上元素有$ab=cb$且$b\neq 0$,则$(a-c)b=0$。由于整环没有零因子,所以$a=c$,即消去律对整环必然成立。}。因此,$a^{m-n-1}$为$a$的逆元,$a^{k(m-n-1)}$是$a^k$的逆元,证毕。

\subsection{循环群的子群结构}
\begin{theorem}{}
循环群的子群必是循环群。
\end{theorem}
\begin{theorem}{}
设$G$是$n$元循环群,若$d|n$,则$G$内存在唯一一个$d$阶子群。
\end{theorem}
\begin{theorem}{}
若群$G$的不同子群阶数不同,则$G$是循环群。
\end{theorem}