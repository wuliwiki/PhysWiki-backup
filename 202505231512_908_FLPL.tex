% 菲利普·莱纳德(综述)
% license CCBYSA3
% type Wiki

本文根据 CC-BY-SA 协议转载翻译自维基百科\href{https://en.wikipedia.org/wiki/Philipp_Lenard}{相关文章}。

菲利普·爱德华·安东·冯·莱纳德(德语发音:[ˈfɪlɪp ˈleːnaʁt],匈牙利语:Lénárd Fülöp Eduárd Antal,1862年6月7日-1947年5月20日)是一位匈牙利裔德国物理学家,因“在阴极射线研究中取得的成果”以及发现其多种性质,于1905年获得诺贝尔物理学奖。

他最重要的贡献之一是对光电效应的实验验证:他发现,从阴极中逸出的电子的能量(速度)只依赖于入射光的频率,而与其强度无关。

莱纳德是民族主义者和反犹主义者;他是纳粹意识形态的积极支持者,早在1920年代便支持阿道夫·希特勒,并在纳粹时期成为“德国物理学”运动的重要榜样。值得注意的是,他将阿尔伯特·爱因斯坦的科学贡献贬称为“犹太物理学”。

\subsection{早年生活与工作}
菲利普·莱纳德于1862年6月7日出生在匈牙利王国的普雷斯堡(当时称为 Pozsony,即今日斯洛伐克的布拉迪斯拉发)。莱纳德家族最早在17世纪来自蒂罗尔,而他母亲的家族则源自巴登;他的父母都是讲德语的。[5] 父亲名为菲利普·冯·莱纳德,是普雷斯堡的一位葡萄酒商人;母亲名为安东妮·鲍曼。[6] 莱纳德在大多为日耳曼血统的祖先中也有一些马扎尔人血统。年幼的莱纳德曾就读于波若尼皇家天主教高级文理中学(Pozsonyi Királyi Katolikus Főgymnasium,今称 Gamča),据他在自传中记述,这段经历给他留下了深刻印象,尤其是他老师维吉尔·克拉特的个人魅力。[7]
1880年,他先后在维也纳和布达佩斯学习物理和化学。[7]
