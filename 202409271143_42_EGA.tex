% 电磁场和引力场的作用量
% keys 电磁场|引力场|作用量
% license Usr
% type Tutor

\pentry{电磁力和引力\nref{nod_EleGra}}{nod_9484}

由“电磁力和引力\upref{EleGra}”一节,我们得到了处于外场中的粒子的作用量的两种情况:
\begin{equation}
\begin{aligned}
S_E=&\int \left\{-m\sqrt{-\eta_{\mu\nu} \,\mathrm{d}{x} ^\mu \,\mathrm{d}{x} ^\nu}+A_\mu(x) \,\mathrm{d}{x} ^\mu \right\} ,\\
S_G=&-m\int\sqrt{-g_{\mu\nu}(x) \,\mathrm{d}{x} ^\mu \,\mathrm{d}{x} ^\nu}.
\end{aligned}~
\end{equation}
 $S_E,S_G$ 分别对应于磁场和引力场中的粒子的作用量。

自然会出现这样的问题:场 $A_\mu$ 和 $g_{\mu\nu}$ 是怎样产生的?

物理学应当是粒子和场之间的共舞。场告诉粒子如何运动,粒子反过来产生场。我们已经在“电磁力和引力\upref{EleGra}”一节中描述了粒子在场中的运动,本节将寻找主宰场 $A_\mu(x),g_{\mu\nu}(x)$ 动力学的作用量。

由\autoref{def_CoIn_1},物理应当是关于某一变换不变的。因此找到相关的不变性是首要的。Lorentz不变性必须是满足的(这仅仅是说我们有选择经过(四维)旋转变换下的坐标描述物理的自由)。

\subsection{电磁场的作用量}









