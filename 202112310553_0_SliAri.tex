% SLISC 矩阵的基本运算
% keys C++|SLISC|矩阵|矢量|运算|求和|共轭|点乘|内积

\begin{issues}
\issueDraft
\end{issues}

\pentry{SLISC 的矢量和矩阵\upref{SliMat}}

这里简单列出 \verb|arithmetic.h| 中的函数, 用于对矩阵进行基本运算.
\addTODO{给一个具体的例程}

\verb|Long size(T_I v)| 相当于 \verb|std::vector| 的 \verb|.size()|, 但返回 \verb|Long| 类型, 这是为了避免使用 \verb|unsigned| 类型.

\verb|Tr sum_abs(T_I v)| 绝对值求和 $\sum_i \abs{v_i}$

\verb|Ts max(T_I v)|, \verb|Ts min(T_I v)| 最大元素和最小元素

\verb|Ts max(Long_O ind, T_I v)|, \verb|Ts min(Long_O ind, T_I v)| 最大元素和最小元素(同时求出其位置)

\verb|Tr max_abs(T_I v)| 最大绝对值

\verb|void mod(T_O v, T1_I v1, Ts1_I s)| 求余

\verb|void mod(T1_O v, Ts1_I s)| 求余

\verb|void real(T_O v, T1_I v1)| 提取实部

\verb|void imag(T_O v, T1_I v1)| 提取虚部

\verb|void abs(T_IO v)| 求绝对值

\verb|void abs(T_O v, T1_I v1)| 求绝对值

\verb|Tret sum(T_I v)| 求和

\verb|Tret prod(T_I v)| 求积

\verb|Tr norm2(T_I v)| 绝对值平方和

\verb|Tr norm(T_I a)| 绝对值平方和再开根号

\verb|void resize_cpy(Tv_IO v, Long_I N, T_I val = 0)| 矢量改变尺寸保留数据(多出数据初始化为 0)

\verb|void resize_cpy(Tv_IO v, Long_I N0, Long_I N1, T_I val = 0)| 矩阵改变尺寸保留数据(多出数据初始化为 0)

\verb|void resize_cpy(Tv_IO v, Long_I N0, Long_I N1, Long_I N2, T_I val = 0)| 3D 改变尺寸保留数据(多出数据初始化为 0)

\verb|void linspace(Tv_O v, Ts_I first, Ts_I last)| 等间距赋值

\verb|void flip(Tv_IO v)| 矢量翻转

\verb|void flip(T_O v, T1_I v1)| 矢量翻转

\verb|void reorder(Tv_O v, To_I order)| 矢量重新排序

\verb|void trans(T_IO v)| 方矩阵转置

\verb|void trans(T_O v, T1_I v1)| 矩阵转置

\verb|void conj(T_IO v)| 共轭

\verb|void conj(T_O v, T1_I v1)| 共轭

\verb|void her(T_IO v)| 厄米共轭

\verb|void her(T_O v, T1_I v1)| 厄米共轭

\verb|void operator+=(T_IO v, Ts_I s)|, \verb|void operator-=(T_IO v, Ts_I s)|, \verb|void operator*=(T_IO v, Ts_I s)|, \verb|void operator/=(T_IO v, Ts_I s)| 矩阵和标量的四则运算

\verb|void operator+=(T_O &v, T1_I v1)|, \verb|void operator+=(T_O &v, T1_I v1)|, \verb|void operator-=(T_O &v, T1_I v1)|, \verb|void operator*=(T_O &v, T1_I v1)|, \verb|void operator/=(T_O &v, T1_I v1)| 矩阵和矩阵的四则运算

\verb|void plus(T_O v, T1_I v1, Ts2_I s)|, \verb|void minus(T_O v, T1_I v1, Ts2_I s)|, \verb|void times(T_O v, T1_I v1, Ts2_I s)|, \verb|void divide(T_O v, T1_I v1, Ts2_I s)| 矩阵和标量的四则运算

\verb|void pow(T_IO v, Ts_I s)| 幂运算

\verb|void pow(T_O v, T1_I v1, Ts_I s)| 幂运算

\verb|T dot(T1_I v1, T2_I v2)| 点乘(内积)

\verb|void cumsum(T_O v, T1_I v1)| 累积求和

\verb|void mul(T_O y, T1_I a, T2_I x)| 矩阵乘矢量

\verb|void mul(T_IO y, T1_I a, T2_I x, Ts1_I alpha, Ts_I beta)| 矩阵乘矢量(慢)

\verb|void mul(T_O &y, T1_I x, T2_I a)| 行矢量乘矩阵(慢)

\verb|void mul_gen(Ty_O y, Ta_I a, Tx_I &x)| 一般矩阵乘矢量(使用 BLAS)

\verb|void mul_gen(Ty_O &y, Ta_I a, Tx_I x, Tsa_I alpha = 1, Tsa_I beta = 0)| 一般矩阵乘矢量(使用 BLAS)

\verb|void mul_sym(Ty_IO &y, Ta_I a, Tx_I x, Tsa_I alpha = 1, Tsa_I beta = 0)| 对称矩阵乘矢量(使用 BLAS)

\verb|void uniq_elm(T_IO v)| 找出不相同的元素(自动 resize)

\verb|void uniq_rows(T_O a, T1_I a1)| 找出不相同的行

\verb|void exp(T_IO v)| 指数函数

\verb|void exp(T_O v, T1_I v1)| 指数函数
