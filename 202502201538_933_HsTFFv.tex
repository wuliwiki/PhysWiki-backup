% 函数视角下的三角函数(高中)
% keys 函数|三角函数|周期|性质
% license Usr
% type Tutor

\begin{issues}
\issueDraft
\end{issues}

\pentry{三角函数\nref{nod_HsTrFu},函数\nref{nod_functi},函数的性质\nref{nod_HsFunC},导数\nref{nod_HsDerv}}{nod_5a43}

在前面,我们已经学习了三角函数的定义,并基于这些定义推导出了诱导公式以及三角函数之间的基本恒等关系。这些推导主要依赖于三角函数的几何定义,如单位圆和直角三角形等工具,帮助我们理解它们的对称性和变换规律。

然而,三角函数不仅仅是一种几何工具,它们也是典型的数学函数,具有一般函数所具备的性质,如周期性、单调性、对称性等。因此,接下来将从函数的视角进一步探讨三角函数,分析它们的图像、变化趋势以及在不同数学背景下的应用。

需要注意的是,几何视角和函数视角并非相互独立,而是互相补充的。几何定义有助于建立直观理解,而函数分析则揭示了三角函数的整体结构及其运算规律。在不同的数学问题中,选择合适的视角能够使问题的解决更加高效,因此理解这些视角之间的联系至关重要。下文主要针对高中涉及的正弦、余弦与正切函数。其余三者由于与此三个有倒数关系,本文会适当提及,但不会给出更细致的讲解和推导。

有一个非常有名的动画,在每次有人介绍三角函数时都会出现,就是左侧是一个点在单位圆上运动,表明着任意角的各个取值,然后在单位圆的右侧和上侧,分别基于定义,将角的大小和对应线段的长度画在另一个坐标系中,自然地引出各个三角函数的函数图像。纵然,这样的动画能够直观地展现他们的关系,但更希望读者能够自己在脑海中将这个过程演练出来。当看到函数图像时,能够自动地反映出这个圆上点旋转的过程,反过来,当看到某些圆周运动的点时,能够自动地在脑海中画出它对应的图像。这对未来从更多视角理解三角函数相关的内容有好处。

\subsection{前文提及的性质总结}

\subsubsection{定义域}

正弦函数和余弦函数的\textbf{定义域为全体实数},正切函数的定义域为 $\begin{Bmatrix}\alpha|\alpha \neq \frac{\pi}{2}+k\pi,k\in Z\end{Bmatrix}~.$

\subsection{周期性}

正弦函数、余弦函数、正切函数的都是周期函数,根据定义易得,正弦函数和余弦函数,周期为 $2k\pi(k\in Z,k\neq0)$,正切函数的周期为 $k\pi(k\in Z,k\neq0)$.


\subsection{图像}
根据前面的推导,可以得到基本三角函数的图像如下图。
\begin{figure}[ht]
\centering
\includegraphics[width=14.25cm]{./figures/14fd66d8d1e6e0b5.png}
\caption{$\sin x$和$\cos x$} \label{fig_HsTFFv_1}
\end{figure}
可以看出正弦函数和余弦函数是定义域为 $R$ 值域为 $[-1,1]$ 最小正周期 $T = 2\pi$ 的周期函数。

\begin{figure}[ht]
\centering
\includegraphics[width=14.25cm]{./figures/6f97182187b36e36.png}
\caption{正切函数与余切函数} \label{fig_HsTFFv_3}
\end{figure}

作为扩展,下面也给出正割函数与余割函数的函数图像,他们的性质均可通过与正弦和余弦的关系分析得到,此处不予赘述。

\begin{figure}[ht]
\centering
\includegraphics[width=14.25cm]{./figures/56f93ee1a7fb0faa.png}
\caption{正割函数与余割函数} \label{fig_HsTFFv_2}
\end{figure}

\subsection{从三角函数推广得到的}

\subsection{导数}