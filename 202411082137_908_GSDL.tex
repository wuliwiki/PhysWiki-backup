% 高斯定律(综述)
% license CCBYSA3
% type Wiki

本文根据 CC-BY-SA 协议转载翻译自维基百科\href{https://en.wikipedia.org/wiki/Conservation_of_energy}{相关文章}。

本文讨论的是关于电场的高斯定律。关于其他场的类似定律,请参见磁场的高斯定律和重力的高斯定律。关于与这些定律相关的奥斯特罗格拉德斯基-高斯定理,请参见散度定理。  
请勿将其与高斯法则(Gause's law)混淆。

在物理学中(特别是电磁学),高斯定律(又称高斯通量定理,或有时称为高斯定理)是麦克斯韦方程组之一。它是散度定理的一个应用,将电荷分布与由此产生的电场联系起来。
\subsection{定义}
\begin{figure}[ht]
\centering
\includegraphics[width=8cm]{./figures/95db2b4380930eda.png}
\caption{当由于对称性原因可以找到一个电场沿其均匀的闭合曲面(GS)时,积分形式的高斯定律尤其有用。在这种情况下,电通量是表面积和电场强度的简单乘积,并且与曲面所包围的总电荷成正比。这里正在计算带电球体外部(\( r > R \))和内部(\( r < R \))的电场(见维基学院)。} \label{fig_GSDL_1}
\end{figure}
在积分形式下,高斯定律表述为:电场通过任意封闭曲面的通量与该曲面所包围的电荷成正比,而不论电荷如何分布。尽管仅凭此定律不足以确定包围任意电荷分布的表面上的电场,但在具有对称性导致电场均匀分布的情况下,这是可能的。在没有这种对称性的情况下,可以使用高斯定律的微分形式,其表述为电场的散度与电荷的局部密度成正比。

该定律最早由约瑟夫-路易·拉格朗日于1773年[1][2]提出,后由卡尔·弗里德里希·高斯于1835年在椭球引力的背景下提出。[3] 它是麦克斯韦方程组之一,构成经典电动力学的基础。[注1] 高斯定律可以用于推导库仑定律,[4] 反之亦然。
\subsection{定义}
在积分形式下,高斯定律表述为:电场通过任意封闭曲面的通量与该曲面所包围的电荷成正比,而不论电荷如何分布。尽管仅凭此定律不足以确定包围任意电荷分布的表面上的电场,但在对称性要求电场均匀的情况下,这是可能的。在不存在这种对称性的情况下,可以使用高斯定律的微分形式,其表述为电场的散度与电荷的局部密度成正比。

该定律最早由约瑟夫-路易·拉格朗日于1773年[1][2]提出,随后在1835年由卡尔·弗里德里希·高斯[3]提出,两者都是在椭球引力的背景下提出的。它是麦克斯韦方程组之一,构成经典电动力学的基础。[注1] 高斯定律可以用于推导库仑定律,[4] 反之亦然。
\subsection{定性描述}
用语言描述,高斯定律表述为:

任意假设的封闭曲面的净电通量等于该封闭曲面内所包围的净电荷除以 \(\varepsilon_0\)。该封闭曲面也称为高斯面。[5]

高斯定律在数学上与物理学其他领域中的多条定律有密切的相似性,例如磁学中的高斯定律和重力中的高斯定律。事实上,任何反平方定律都可以用与高斯定律类似的方式表述:例如,高斯定律本质上等价于库仑定律,而重力的高斯定律本质上等价于牛顿的万有引力定律,它们都是反平方定律。

该定律可以使用向量微积分以积分形式和微分形式表示;两者是等价的,因为它们通过散度定理(也称为高斯定理)相关联。这些形式还可以通过两种方式表达:一种是电场 \( E \) 与总电荷之间的关系,另一种是电位移场 \( D \) 与自由电荷之间的关系。[6]
\subsection{包含电场 \( E \) 的方程}
高斯定律可以使用电场 \( E \) 或电位移场 \( D \) 来表述。本节展示了一些包含 \( E \) 的形式;包含 \( D \) 的形式和其他包含 \( E \) 的形式在下方。
\subsubsection{积分形式}
\begin{figure}[ht]
\centering
\includegraphics[width=8cm]{./figures/d34f5443a269bb1a.png}
\caption{任意曲面的电通量与该曲面所包围的总电荷成正比。} \label{fig_GSDL_2}
\end{figure}
高斯定律可以表示为:[6]
\[
\Phi_{E} = \frac{Q}{\varepsilon_{0}}~
\]
其中,\(\Phi_{E}\) 是穿过包围任意体积 \(V\) 的封闭曲面 \(S\) 的电通量,\(Q\) 是体积 \(V\) 内的总电荷,\(\varepsilon_{0}\) 是电常数。电通量 \(\Phi_{E}\) 定义为电场的曲面积分:
\[
\Phi_{E} = \iint_{S} \mathbf{E} \cdot \mathrm{d} \mathbf{A}~
\]
其中,\(\mathbf{E}\) 是电场,\(\mathrm{d} \mathbf{A}\) 是表示曲面微小面积元素的向量,[注2] “\(\cdot\)” 表示两个向量的点积。

在弯曲时空中,电磁场通过封闭曲面的通量表示为
\[
\Phi_{E} = c\iint_{S} F^{\kappa 0} \sqrt{-g} \, \mathrm{d} S_{\kappa}~
\]
其中 \(F^{\kappa 0}\) 是电磁场张量的分量,\(\sqrt{-g}\) 是度规的行列式的平方根,\(\mathrm{d} S_{\kappa}\) 是闭合曲面的微分面积元素。
\begin{figure}[ht]
\centering
\includegraphics[width=8cm]{./figures/40885a4587d08090.png}
\caption{球体内没有包围电荷,其表面的电通量为零。} \label{fig_GSDL_3}
\end{figure}
其中,  
\( c \) 表示光速;  
\( F^{\kappa 0} \) 表示电磁张量的时间分量;  
\( g \) 是度规张量的行列式;  
\(\mathrm{d} S_{\kappa} = \mathrm{d} S^{ij} = \mathrm{d} x^{i} \mathrm{d} x^{j}\) 是围绕电荷 \( Q \) 的二维表面的正交标准元素,  
其中的指标 \( i, j, \kappa = 1, 2, 3 \),且互不相同。[8]

由于通量定义为电场的积分,因此高斯定律的这种表达方式称为积分形式。

在涉及已知电位的导体的问题中,导体外部的电位通过求解拉普拉斯方程获得,可以采用解析或数值方法。然后,电场计算为电位的负梯度。高斯定律使得找到电荷分布成为可能:导体中任一区域的电荷可以通过积分电场得出,从而找到通过一个小盒子的通量,该盒子的边垂直于导体表面,并且电场垂直于表面且在导体内部为零。

逆问题是已知电荷分布并需要计算电场,这个问题要困难得多。给定表面的总通量对电场信息提供很少的线索,通量可以以任意复杂的模式进出表面。

一个例外是问题中存在某种对称性,这使得电场以均匀的方式通过表面。此时,如果已知总通量,就可以在每个点推导出电场。常见的对称性示例包括:柱对称性、平面对称性和球对称性。有关利用这些对称性来计算电场的示例,请参见“高斯面”条目。
\begin{figure}[ht]
\centering
\includegraphics[width=8cm]{./figures/400f9c8a4f36180f.png}
\caption{一个微小的高斯盒子,其边垂直于导体的表面,用于在通过求解拉普拉斯方程计算出电位和电场后找到局部表面电荷。电场在导体的等势面上局部垂直于表面,并在内部为零;根据高斯定律,其通量 \( \pi a^2 \cdot E \) 等于 \( \pi a^2 \cdot \sigma / \varepsilon_0 \)。因此,\(\sigma = \varepsilon_0 E\)。} \label{fig_GSDL_4}
\end{figure}
\subsubsection{微分形式}
根据散度定理,高斯定律也可以写成微分形式:
\[
\nabla \cdot \mathbf{E} = \frac{\rho}{\varepsilon_0}~
\]
其中 \(\nabla \cdot \mathbf{E}\) 是电场的散度,\(\varepsilon_0\) 是真空介电常数,\(\rho\) 是总体积电荷密度(每单位体积的电荷量)。