% 线性算子的反对称幂与行列式
% license Xiao
% type Tutor

\begin{issues}
\issueDraft
\issueOther{可以对照线性算子的张量积\upref{TPofLO}进行阅读}
\end{issues}

\pentry{线性算子的张量积\upref{TPofLO},向量空间的对称/反对称幂\upref{vecSAS},不变子空间\upref{InvSP}}

\subsection{线性算子的反对称幂}

\textbf{二次幂}(平方)是张量积的一种特殊情况,具体而言,考虑线性映射 $f: V \to W$,它和它自己的张量积为
\begin{equation}
f \otimes f: V \otimes V \to W \otimes W~.
\end{equation}

更一般的,$f$ 的 $n$ 阶张量幂为
\begin{equation}
f^{\otimes n}: V^{\otimes n} \to W^{\otimes n}~.
\end{equation}

如果 $f$ 是一个线性算子,即 $f: V \to V$,我们发现fan du ci e h $$

% 对于线性映射 $f: V \to W$,我们可以定义它的反对称幂,二阶幂为线性映射\footnote{更严格的写法应该是 $\sum_i v_i \wedge v_i' \mapsto \sum_i f(v_i) \wedge f(v_i')$,不过本文的所有映射都是线性映射,所以只需要定义一组基的线性变换即可。}
% \begin{equation}
% \begin{aligned}
% {\large \wedge}^2 f: {\large \wedge}^2 V &\to {\large \wedge}^2 W, \\
% v \wedge v' &\mapsto f(v) \wedge f(v')~.
% \end{aligned}
% \end{equation}

% 我们知道 $v \wedge v' = - v' \wedge v$,因此需要验证这个定义是良定义的:
% \begin{equation}
% ({\large \wedge}^2 f)(- v' \wedge v) = - f(v') \wedge f(v) = f(v) \wedge f(v') = ({\large \wedge}^2 f)(v \wedge v')~.
% \end{equation}

% 更一般的我们可以定义$n$阶幂:
% \begin{equation}
% \begin{aligned}
% {\large \wedge}^n f: {\large \wedge}^n V &\to {\large \wedge}^n W, \\
% v_1 \wedge \cdots \wedge v_n &\mapsto f(v_1) \wedge \cdots \wedge f(v_n)~.
% \end{aligned}
% \end{equation}

\subsection{行列式}

