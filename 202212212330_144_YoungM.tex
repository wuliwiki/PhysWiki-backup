% 杨氏模量、泊松比、剪切模量、广义胡克定律的基本形式
% 杨氏模量|压强|长度|弹性材料|张力

\begin{issues}
\issueDraft
\end{issues}

\footnote{本文参考\href{https://en.wikipedia.org/wiki/Young's_modulus}{维基百科}相关页面与Callister的Material Science and Engineering An Introduction。}
在本文中,我们假定材料是弹性的、线性的、各向同性的,且所有变形都发生在弹性限度内。

\subsection{杨氏模量;广义胡克定律(单向拉伸)}
\begin{figure}[ht]
\centering
\includegraphics[width=8cm]{./figures/YoungM_1.pdf}
\caption{在外力下材料变形} \label{YoungM_fig1}
\end{figure}

如同弹簧的胡克定律$$F=k \Delta x$$,当材料被垂直于截面的单向外力拉伸时,我们有广义胡克定律
\begin{equation}
\sigma = E \varepsilon
\end{equation}。

其中$\sigma$是应力\upref{STRESS},$\varepsilon$是应变\upref{Strain},$E$被称为\textbf{杨氏模量(Young's modulus)},也叫\textbf{杨氏模数}或者\textbf{弹性模量}, 是用于衡量材料弹性的力学属性,物理含义类似弹簧的劲度系数$k$。杨氏模量具有压强的量纲(国际单位:帕斯卡\upref{SIunit})。

如果外力在截面上分布均匀,那么应力化为$\sigma=\frac{F}{A}$,应变为$\varepsilon=\frac{\Delta l}{l_0}$,此时广义胡克定律简化为:
\begin{equation}\label{YoungM_eq1}
E = \frac{FL_0}{A\Delta L}
\end{equation}
其中 $F$ 是受力, $L_0$ 是原长, $A$ 是横截面积, $\Delta L$ 是伸长或压缩的长度。通常我们假设 $\Delta L \ll L_0$, 因为和弹簧一样, 过大的形变会导致非线性效应。

\begin{example}{}
一根横截面直径为 $2r = 2\Si{mm}$ 的钢丝, 松弛长度为 $L_0 = 1\Si{m}$, 已知该钢丝的杨氏模量为 $E = 200\Si{GPa}$, 要将其拉长 $\Delta L = 1\Si{mm}$ 需要在两端施加多大的力?

解: 横截面为 $A = \pi r^2$, 将各个量代入\autoref{YoungM_eq1} 解的张力为 $F = 628.3\Si{N}$。
\end{example}

\subsection{泊松比}
玩过橡皮条的你会发现,当你在一个方向上拉伸橡皮时,橡皮往往会在另外两个方向上缩短。这种“收缩现象”是普遍的,也是弹性力学的另一个基本假设。%似乎没有官方的名称?。
这意味着,如\autoref{YoungM_fig1} 所示,材料即使只在$x$方向上受力,在$y,z$方向上也会产生变形。
\begin{equation}
-\varepsilon_z= \nu \varepsilon_x
\end{equation}
负号表示变形的方向,$\nu$被称为柏松比 (Poisson's Ratio)\footnote{一般$\nu$为正值,并根据弹性力学的相关理论可确定$\nu$的理论范围。在一些具有特殊结构的材料中,$\nu$可以是负数。}。%需要补充这个范围,我有点忘了...

\subsection{剪切模量;广义胡克定律(单向剪切)}
\begin{figure}[ht]
\centering
\includegraphics[width=8cm]{./figures/YoungM_2.pdf}
\caption{材料发生剪切变形} \label{YoungM_fig2}
\end{figure}
当材料受到一组平行于截面的剪切外力时,材料也会发生形变,只不过发生的是剪切变形。类似地,有
\begin{equation}
\tau=2G\varepsilon
\end{equation}
其中$\tau$是剪应力\upref{STRESS},$\varepsilon$是切应变\upref{Strain},$G$是剪切模量,或第一Lame常数。

剪切模量并不是独立的物理量,他取决于材料的杨氏模量与泊松比。
\begin{equation}
G = \frac{E}{2(1+\nu)}
\end{equation}

在工程上,这个公式也记为$\tau=G\gamma$。$\gamma$是工程剪应变\upref{Strain},工程剪应变仅仅是相应切应变的$2$倍。
