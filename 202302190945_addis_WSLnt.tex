% WSL 笔记

\begin{issues}
\issueDraft
\end{issues}

\subsection{cmd 中的 wsl 命令}
\begin{itemize}
\item 详细文档参考 https://docs.microsoft.com/en-us/windows/wsl/
\item `wsl --help` 获取帮助
\item `wsl --shutdown` 关闭所有 distro 和虚拟机
\item `wsl --status` 显示 wsl 状态
\item `wsl --update` 更新
\item `wsl --set-default-version 1 或 2` 切换默认的 wsl 版本
\item `wsl --set-version Ubuntu-22.04 2` 把某个已安装的 distro 在 WSL1 和 2 之间转换, 如果 windows 没有开 Virtual Machine Platform, 会提示且不做转换。 (两三分钟就好了)。 注意这并不会把整个 linux 的文件系统都变成 ext4 的虚拟硬盘(可以用 windirtree 验证)。 其实还是另外新装 wsl2 版本的 ubuntu 比较好, 因为 wsl1 没有 snap store 和正常的 systemd systemctl 等。 可以用下面的方法把某个 wsl1 distro 重命名, 再安装相同的 wsl2 distro (没试过)。
\item `wsl --set-default [distro]` 选择默认的 distro
\item `wsl --list` 列出所有 distro `wsl --list --verbose` 显示详细信息
\item `wsl --list --online` 列出所有可下载的 distro
\item `wsl --install -d` 安装某个 distro
\item WSL 虚拟机的默认路径是 `C:/Users/addis/AppData/Local/Packages/CanonicalGroup...distro版本号/`
\item `wsl --unregister ...` 删除某个 distro 以及它的所有文件

## 镜像导出导入
\item `wsl --export <distribution name> <export file name>` 可以导出 wsl。 例如 `wsl --export Ubuntu ubuntu.tar` 导出文件会存到当前路径。
\item export 和 import 可以把同一个 distro 安装多份或者重命名。
\item `wsl --import <new distribution name> <install location> <export file name> --version 1或2` 例如 `wsl --import Ubuntu-20.04-WSL1-20220904 C:\Users\addis\ C:\Users\addis\Desktop\Ubuntu-20.04-WSL1-20220904.tar --version 1或2` 注意 import 以后默认会以 root 身份登录, 可以设置 `.bashrc` 自动 `su` 成某个用户。
\item `No3/WSL-images/Ubuntu-20.04-WSL2-20220905.tar` 是从 Miranda 的 Win11 备份的镜像, 亲测在 Surface 上可以直接 import, X11 可用(但是 CLion 好卡)
\item import wsl 以后, 会默认用 root 登录, 先试试 xeyes 可不可以。 可以的话 `cp ~/.Xautority /home/addis/` 然后 `su addis` 再试试 xeyes 可不可以。 可以的话就把这两个命令添加到 `/root/.bashrc` 中。
\end{itemize}
