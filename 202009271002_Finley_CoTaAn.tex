% 列联表独立性检验和拟合优度检验
\pentry{机器学习数据类型\upref{DatTyp}}
分类数据是对事物进行分类的结果,它虽然是用数值表示,但是数值仅仅反映对象的不同特征,其大小没有意义.分类数据的结果是频数,对其进行统计分析主要利用$\chi^2$分布.
\subsection{$\chi^2$统计量}
$\chi^2$统计量可用于测定2个分类变量之间的相关程度.用$f_o$表示观察值频数,$f_e$表示期望值频数,则:
\begin{equation}
\chi^2 =  \sum \frac {(f_o-f_e)^2}{f_e}
\end{equation}
利用$\chi^2$统计量,可以对分类数据进行拟合优度检验和独立性检验.
\subsection{拟合优度实验}
拟合优度实验依据总体分布,计算出各类别的期望频数,与观察频数进行对比,判断两者是否有显著差异,从而对分类变量进行分析.
\subsubsection{原假设和备择假设}
$H0$:观察频数与期望频数一致
$H1$:观察频数与期望频数不一致
\subsubsection{检验统计量}
$\chi^2$统计量
自由度为$df=R−1$,$R$为分类变量的类型的个数.
在假设检验中,我们在二项分布总体、大样本情况下,对总体比例采用$z$检验:
\begin{equation}
z = \frac {p-\pi_0}{\sqrt{
    frac{\pi_0(1-\pi_0)}{n}
}}
\end{equation}
对于总体比例,同样可以使用拟合优度检验(比例可视为2个类别的分类变量).z检验只能针对二项分布问题,而χ2检验既可以分析二项分布,也可以分析多项分布(对总体的多个比例的假设进行检验).