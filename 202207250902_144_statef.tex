% 态函数
% 态函数|状态量|内能|熵
\pentry{热力学第一定律\upref{Th1Law},全微分\upref{TDiff}}

\subsection{态函数}
如果某个函数只和系统的热力学参量(例如体积V,温度T,压强p等)有关,也就是只和系统状态有关,我们称它为\textbf{态函数}.

系统态函数的改变量只与起点和终点系统的状态有关,与系统是如何从一个状态变化至另一个状态的具体路径无关.热力学研究的就是热力学系统的态函数之间的关系.

\subsection{"状态公理"}
%好像没有正式的表述?感觉这么叫挺合适的(
系统的各热力学参量间并不是互相独立地.事实上,几个宏观的热力学参量就能完整地刻画一个\textbf{热力学平衡系统}.以下假定系统不处于外场\footnote{例如处于重力场下的大气,压强会随高度的变化而变化,情况肯定会有所不同,}之中、系统中的物质种类已知.

\subsubsection{一元单相系统(无外场)}
对于这样一个一元单相平衡系统,只需两个热力学强度量(一般使用压强 $P$ 和温度 $T$ )与该物质的物质的量,便足以刻画这个系统的一切宏观特征(即可原则上导出U,S,H等其余热力学量...).
\begin{equation}
X=f(T,P,n)
\end{equation}

\begin{example}{E的全微分形式}
这样一来,对于一定量的物质(不一定是理想气体,可以是一般气体,甚至液体和固体),我们可以把内能 $E$ 写成 $T$ 和 $P$ 的函数 $E(T,P)$,因此可以写出 $E$ 的全微分形式:
\begin{equation}
\dd E=\left(\frac{\partial E}{\partial T}\right)_p \dd T + \left(\frac{\partial E}{\partial P}\right)_T \dd P
\end{equation}
这个全微分形式刻画的是,在一个改变温度和压强的\textbf{微小的准静态过程}\upref{Quasta}中\footnote{也就是说,在这个过程中,系统每时每刻都是平衡态,$E$ 总是态函数.要注意的是,态函数的定义只对一个热力学平衡系统有效},$E$ 的改变量与 $\dd T$ 和 $\dd P$ 成线性关系.而偏导数描绘的是比例系数.如果我们能知道两个偏导数,利用全微分关系式,我们可以刻画任意准静态过程的内能改变量.
\end{example}

\subsubsection{多元单相系统(无外场)}
如果系统中还包括多种物质,那么除$p$, $T$外,还需知道各组分的物质的量$n$,才可刻画这个系统的一切宏观特征.
\begin{equation}
X=f(T,P,n_1, n_2, ...)
\end{equation}

\subsubsection{多元多相系统(无外场)}
在这种情况下,可以分别对每个相运用“状态公理”.
\begin{equation}
X_1=f(T,P,n_1, n_2, ...)
\end{equation}
\begin{equation}
X_2=f(T,P,n_1, n_2, ...)
\end{equation}
$$...$$

\subsection{热力学第一定律与熵}
\pentry{熵\upref{Entrop}}
熵是表征系统混乱程度的一个态函数,一个系统熵越小,则越有序,这会在熵的纯微观分析中\upref{IdeaS}会进行解释.熵的宏观定义是,可逆过程中 $\dd Q/T$ 的积分\upref{Entrop},也就是说 $\dd S=\dd Q/T$.所以,\textbf{等熵过程就是绝热过程\upref{Adiab}}.

我们把热力学第一定律中 $\dd Q$ 改写为 $T\dd S$,$\dd W$ 改写为 $p\dd V$(这里我们只考虑体积功).那么有\footnote{
此处的全微分式是在可逆过程中成立的.由热力学第二定律\upref{Td2Law},对于不可逆的热力学过程,有 $\dd Q<T\dd S$.所以代入第一定律可以得到 $T\dd S\ge \dd E+\dd W=\dd E+P\dd V$,等号在可逆过程中成立.
}

\begin{equation}
\dd E=T\dd S-P\dd V
\end{equation}

于是 $E$ 可以看成是熵 $S$ 和体积 $V$ 的函数,由全微分式,可得 $T=\left(\frac{\partial E}{\partial S}\right)_V$,$P=-\left(\frac{\partial E}{\partial V}\right)_S$.第二个式子表明压强就是绝热过程中内能随体积增大而减小的量.

上面的结论也告诉我们一个神奇的事实,对一个过程量(例如做功,吸热),如果将它乘以某个函数,它可能能成为态函数.虽然 $\dd W$ 是过程量,但 $\dd W/P=\dd V$ 是全微分($V$ 是态函数).$\dd Q/T=\dd S$ 也是全微分,其中 $S$ 为热力学熵\upref{Entrop}.
