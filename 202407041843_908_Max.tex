% 马克斯·普朗克
% license CCBYSA3
% type Wiki

(本文根据 CC-BY-SA 协议转载自原搜狗科学百科对英文维基百科的翻译)
\begin{figure}[ht]
\centering
\includegraphics[width=6cm]{./figures/8bbcc6f4aa654f88.png}
\caption{马克思·普朗克} \label{fig_Max_7}
\end{figure}
马克斯·卡尔·恩斯特·路德维希·普朗克(德语:[ˈplaŋk];[1] English: /ˈplæŋk/;[2] 1858年4月23日 – 1947年10月4日),德国理论物理学家,1918年因发现能量量子获得诺贝尔物理学奖。[3]

普朗克对理论物理做出了许多贡献,但他作为物理学家的名声主要取决于他作为量子理论创始人的角色,[4]这彻底改变了人类对原子和亚原子过程的理解。1948年,德国科学机构凯泽·威廉学会(普朗克曾两次担任主席)更名为马克斯·普朗克学会(MPS)。议员现在包括代表广泛科学方向的83个机构。

\subsection{生活和事业}
\begin{figure}[ht]
\centering
\includegraphics[width=10cm]{./figures/25347aee0b1e0c17.png}
\caption{马克思普朗克十岁时的签名} \label{fig_Max_1}
\end{figure}
普朗克来自一个传统的知识分子家庭。他的曾祖父和祖父都是哥廷根的神学教授;他的父亲是基尔大学和慕尼黑大学的法学教授[5]。他的一个叔叔也是法官。[6]

普朗克生于荷斯坦基尔,家承约翰·朱利叶斯·威廉·普朗克和他的第二任妻子艾玛·帕齐希。他受洗时的名字是卡尔·凯尔希纳·马克思·普朗克;他的名字,马克思 (现已过时的变体马库斯或者仅仅是马克思的错误写法,实际上是马克西米兰的简写)被表示为“称谓名称”。[7]然而,到了十岁,他就用马克思这个名字签了名并且用这个名字度过余生。[8]

普朗克是这个家庭的第六个孩子,尽管他的两个兄弟姐妹来自他父亲的第一次婚姻。在普朗克早年时期,战争很普遍 他最早的记忆是普鲁士和奥地利军队在1864年进军基尔。[6]1867年,全家搬到了慕尼黑,普朗克就读于马克西米利安体育学校,在那里他接受了数学家赫尔曼·穆勒的指导,赫尔曼对青年很感兴趣,教他天文学、力学和数学。普朗克最早是从穆勒那里学到能量守恒原理的。普朗克早在17岁就毕业了。[9]这就是普朗克第一次接触物理领域的方式。

普朗克在音乐方面很有天赋。他上唱歌课,弹钢琴,管风琴和大提琴,还创作歌曲和歌剧。然而,他选择学习物理而不是音乐。
\begin{figure}[ht]
\centering
\includegraphics[width=6cm]{./figures/9dd227bb43a347f2.png}
\caption{878年,年轻时期的普朗克} \label{fig_Max_2}
\end{figure}
慕尼黑物理学教授菲利普·冯·乔利(Philipp von Jolly)建议普朗克不要进入物理学领域,他说,“在这个领域,几乎所有的东西都已经被发现,剩下的只是填补一些空白。”[10]普朗克回答说,他不想发现新的东西,只是想了解这个领域的已知基础,于是他于1874年在慕尼黑大学开始了研究。在乔利的监督下,普朗克完成了他科学生涯中唯一的实验,研究氢气在加热的铂中的扩散,但他随后转移到了理论物理方向。

1877年,他与物理学家赫尔曼·赫尔姆霍茨、古斯塔夫·基尔霍夫和数学家卡尔·魏尔斯特拉斯一起在柏林弗里德里希·威廉大学学习了一年。他写道,赫尔姆霍茨从来没有做好充分准备,说话缓慢,不断计算错误,令听众厌烦,而基尔霍夫却在枯燥乏味的讲座中精心准备并发言。他很快就和赫尔姆霍茨成了好朋友。在那里,他进行了一个主要是自学克劳修斯著作的项目,这使他选择热力学作为他的研究领域。

1878年10月,普朗克通过了资格考试,在1879年2月完成了关于论机械热理论的第二原理 论文答辩。他曾在慕尼黑的学校短暂教过数学和物理。

到1880年,普朗克获得了欧洲最高的两个学位。第一个是博士学位,他完成了详细阐述他的研究和热力学理论的论文。[6] 然后他提交了他的论文,题目是垂直温度下的各向同性卡尔珀 (不同温度下各向同性体的平衡态),这使他获得了适应训练论文资格。

\subsubsection{1.1 学业生涯}
随着他的适应训练论文的完成,普朗克成为了慕尼黑的一名无薪私人教师(德国学术等级相当于讲师/助理教授),直到他被提供一个学术职位。虽然他最初被学术界忽视,但他推进了他在热理论领域的工作,并发现了一个又一个与吉布斯相同的热力学形式,却他没有意识到这一点。克劳修斯关于熵的思想在他的工作中占据了核心地位。

1885年4月,基尔大学任命普朗克为理论物理副教授。接下来是熵及其处理的进一步研究,特别是在物理化学中的应用。在1897年,他发表了他的热力学论文 。[11]他为斯凡特·奥古斯特·阿伦尼乌斯的电解离理论提供了热力学基础。

1889年,他被任命为柏林弗里德里希-威廉-大学基尔霍夫职位的继任者,[12]大概是由于赫尔姆霍茨的调解,并在1892年成为一名正教授。1907年,普朗克获得了玻尔兹曼在维也纳的职位,但被拒绝留在柏林。1909年,作为一名柏林大学教授,他被邀请成为纽约哥伦比亚大学理论物理的欧内斯特·肯普顿·亚当斯讲师。哥伦比亚大学教授A. P .威尔斯翻译并共同出版了他的一系列讲座。[13]他于1926年1月10日从柏林退休,[14]由埃尔温·薛定谔继任。[15]

\subsubsection{1.2 家庭}
1887年3月,普朗克娶了一个校友的妹妹玛丽默克(1861-1909),并和她一起搬进基尔的一间转租公寓。他们有四个孩子:卡尔 (1888-1916),双胞胎艾玛 (1889-1919)和格雷特 (1889-1917),和欧文 (1893-1945)。

在柏林公寓后,普朗克一家住在柏林-格吕内瓦尔德(柏林)旺根海姆大街21号的一栋别墅里。其他几位来自柏林大学的教授住在附近,其中包括神学家阿道夫·冯·哈那克,他成为普朗克的密友。不久,普朗克之家成为了一个社会和文化中心。许多著名的科学家,如阿尔伯特·爱因斯坦、奥托·哈恩和莉泽·迈特纳都是常客。在赫尔姆霍茨的家里已经建立了联合表演音乐的传统。

几年快乐后,玛丽·普朗克于1909年7月去世,可能是死于肺结核。1911年3月,普朗克娶了他的第二任妻子玛格·冯·赫斯林(1882-1948);12月,他的第五个孩子赫尔曼出生。

第一次世界大战期间,普朗克的第二个儿子欧文于1914年被法国人俘虏,而他的大儿子卡尔在凡尔登阵亡。格雷特在1917年生第一个孩子时去世。两年后,她的姐姐嫁给了格雷特的鳏夫,也以同样的方式去世了。两个孙女都幸存了下来,并以她们母亲的名字命名。普朗克坚忍地忍受了这些损失。

1945年1月,与他关系特别密切的欧文被纳粹人民法院判处死刑,因为他参与了1944年7月暗杀希特勒的未遂行动。欧文于1945年1月23日被处决。[16]

\subsubsection{1.3 柏林大学教授}
\begin{figure}[ht]
\centering
\includegraphics[width=6cm]{./figures/c3b865fe6023c01e.png}
\caption{柏林洪堡大学的牌匾:“马克斯普朗克,作为基本行动量子h 的发现者,1889年至1928年在这座建筑中授课。”} \label{fig_Max_3}
\end{figure}
作为柏林弗里德里希-威廉-大学的教授,普朗克加入了当地的物理学会。他后来写道:“在那些日子里,我基本上是那里唯一的理论物理学家,那里的事情对我来说并不那么容易,因为我开始提到熵,但这并不十分流行,因为它被视为数学幽灵”。[17]由于他的倡议,1898年德国各地方物理学会合并形成了德国物理学会(Deutsche Physikalische Gesellschaft,DPG);从1905年到1909年,普朗克担任主席。
普朗克开始了一个为期六个学期的理论物理讲座课程,根据莉泽·迈特纳的说法,“枯燥,有点客观”,不使用笔记,从不犯错,从不动摇;“我听过的最好的讲师”根据一位英国参与者詹姆斯·r·帕廷顿的说法,他继续说:“总是有许多人站在房间周围。由于教室很热,而且相当近,有些听众会不时地倒在地板上,但这并没有干扰讲座”。普朗克没有建立真正的“学校”;他的研究生只有20人左右,其中包括:

1897年马克斯·亚伯拉罕(1875-1922)

1903年马克斯·冯·劳厄(1879-1960)

1904年莫里茨·施利克(1882-1936)

1906年瓦尔特·迈斯纳(1882-1974)

1907 弗里茨·赖克特(1883-1960)

1912年沃尔特·肖特基(1886-1976)

1914年瓦尔特·博特(1891-1957)  [18]

\subsubsection{1.4 黑体辐射}
1894年,普朗克把注意力转向了黑体辐射问题。他受电力公司委托,用最少的能量从灯泡中产生最大的光。基尔霍夫在1859年提出了这个问题:“黑体(一种完美的吸收体,也称为空腔辐射体)发出的电磁辐射的强度如何取决于辐射的频率(即光的颜色)和身体的温度?”。这个问题已经通过实验进行了探索,但没有理论处理与实验值一致。威廉·维恩提出了维恩定律,它正确地预测了高频下的行为,但在低频下却失败了。解决这个问题的另一种方法是,在低频时,与实验结果一致,但在高频时,产生了后来被称为“紫外线灾难”的现象。然而,与许多教科书相反,这并不是普朗克的动机。[19]

普朗克于1899年提出了第一个解决这个问题的方案,这是从普朗克所谓的“基本无序原理”开始的,这使得他可以从许多关于理想振荡器熵的假设中推导出维恩定律,从而产生了所谓的维恩-普朗克定律。很快人们发现,令普朗克沮丧的是,实验证据根本没有证实新定律。普朗克修改了他的方法,推导出著名的普朗克黑体辐射定律的第一个版本,该定律很好地描述了实验观察到的黑体光谱。它最初是在1900年10月19日的一次会议上被提出的,并于1901年发表。第一个推导不包括能量量子化,也不使用统计力学,他对此持厌恶态度。1900年11月,普朗克修改了第一种方法,依靠玻尔兹曼对热力学第二定律的统计解释,获得了对辐射定律背后原理的更基本理解。由于普朗克对玻尔兹曼方法的这种解释的哲学和物理含义深感怀疑,他求助于这些解释,正如他后来所说的,“是一种绝望的行为...我准备牺牲我以前对物理的任何信念。”[19]

他的新推导于1900年12月14日提交给了DPG,其核心假设是电磁能量只能以量子化的形式发射,换句话说,能量只能是基本单位的倍数:
$$E=h\nu ~$$
$h$是普朗克常数,也称为普朗克作用量 (于1899年引入),$\nu $ 是辐射的频率。注意这里讨论的基本能源单位由$h\nu$共同作用而不仅仅是通过$\nu$来体现。物理学家现在称这些量子光子和频率光子$\nu$  会有自己独特的能量。该频率的总能量等于$h\nu$乘以该频率下光子的数量。

起初,普朗克认为量子化只是“一个纯粹的形式假设”...其实我并没有多想……”;如今,这一与经典物理学不相容的假设被认为是量子物理学的诞生和普朗克职业生涯中最伟大的智力成就 (1877年,路德维希·玻尔兹曼在一篇理论论文中讨论了物理系统的能量状态可能是离散的)。普朗克常数的发现使他能够定义一套新的通用物理单位 (如普朗克长度和普朗克质量),所有这些单位都基于基本物理常数,而许多量子理论都是基于这些基本物理常数。为了表彰普朗克对物理学新分支的重要贡献,他获得了1918年诺贝尔物理学奖 (实际上他是在1919年获得的)。[20][21]

随后,普朗克试图理解能量量子的含义,但无济于事。“我试图以某种方式将作用量重新纳入经典理论的徒劳无功的尝试延续了几年,给我带来了很多麻烦。”甚至几年后,其他物理学家如瑞利、金斯和洛伦兹将普朗克常数设置为零,以便与经典物理保持一致,但普朗克很清楚这个常数具有精确的非零值。“我无法理解金斯的固执——他是一个永远不应该存在的理论家的例子,就像黑格尔对哲学的解释一样。如果与事实不符,那就更糟了。"[22]

梅克斯·玻恩写道:“他天生是一个保守的人;他对革命一无所知,对猜测持完全怀疑的态度。然而,他对从事实中进行逻辑推理的强制力的信念是如此坚定,以至于他毫不退缩地宣布了一个震撼物理学的最具革命性的想法。”[23]
\begin{figure}[ht]
\centering
\includegraphics[width=6cm]{./figures/e7a740d777ed3802.png}
\caption{普朗克于1918年因为他在量子理论方面的工作获得诺贝尔物理学奖} \label{fig_Max_4}
\end{figure}

\subsubsection{1.5 爱因斯坦和相对论}
1905年,阿尔伯特·爱因斯坦的三篇划时代的论文发表在杂志物理年刊上。普朗克是少数几个立即认识到狭义相对论重要性的人之一。由于他的影响,这个理论很快在德国被广泛接受。普朗克对狭义相对论的扩展也有很大贡献。例如,他用经典行为来改写这个理论。[23]

爱因斯坦的光假说量 (光子),基于1887年海因里希·赫兹对光电效应的发现 (以及菲利普·勒纳德的进一步研究),最初被普朗克否定。他不愿意完全抛弃麦克斯韦的电动力学理论。“光的理论不是在几十年后,而是在几个世纪后,被抛回到一个时代,在这个时代,克里斯蒂安·惠更斯敢于与艾萨克·牛顿的强大发射理论作斗争……”

1910年,爱因斯坦指出低温比热的异常行为是另一个无法用经典物理学解释的现象。普朗克和能斯特试图澄清越来越多的矛盾,组织了第一次索尔维会议 (布鲁塞尔,1911)。在这次会议上,爱因斯坦说服了普朗克。

与此同时,普朗克被任命为柏林大学的院长,由此他有能力把爱因斯坦叫到柏林,为他设立一个新的教授职位 (1914年)。很快,两位科学家成了亲密的朋友,经常见面一起演奏音乐。

\subsubsection{1.6 第一次世界大战}
第一次世界大战开始时,普朗克感受到了公众的普遍兴奋,他写道,“除了可怕的事情之外,还有出乎意料的伟大和美好的事情:通过所有政党的统一,顺利解决最困难的国内政治问题...赞美一切美好和高尚的事物。”[24][25]

尽管如此,普朗克避免了极端的民族主义。1915年,当意大利即将加入同盟国时,他成功地投票支持了意大利的一篇科学论文,该论文获得了普鲁士科学院奖(Prussian Academy of Sciences)颁发的奖项。普朗克是普鲁士科学院的四位常任主席之一。

普朗克还签署了臭名昭著的“93知识分子宣言”,一本关于论战性战争宣传的小册子 (而爱因斯坦保留了一种严格的和平主义态度,这几乎导致了他的监禁,幸免于他的瑞士公民身份)。但在1915年,普朗克在与荷兰物理学家洛伦兹几次会面后,撤销了宣言的部分内容。然后在1916年,他签署了一份反对德国的声明 annexationism。

\subsubsection{1.7 战后与魏玛共和国}
在动荡的战后年代,普朗克,当时德国物理学的最高权威,向他的同事们发出了“坚持不懈,继续努力”的口号。

1920年10月,他和弗里茨·哈伯建立了德国科学应急组织,旨在为科学研究提供财政支持。该组织发放的很大一部分资金是在国外筹集的。

普朗克还在柏林大学、普鲁士科学院、德国物理学会和凯泽·威廉学会(1948年成为马克斯·普朗克学会)担任领导职务。在此期间,德国的经济状况非常糟糕,他几乎无法进行研究。1926年,普朗克成为荷兰皇家艺术与科学学院的外国成员。[26]

在两次世界大战期间,普朗克成为了诺贝尔和平奖获得者古斯塔夫·施特雷泽曼的政党德意志党的成员,该党渴望国内政策的自由主义目标,而不是全世界政治的修正主义目标。

普朗克不同意实行普选制,后来表示纳粹独裁是“群众统治上升”的结果。[27]

\subsubsection{1.8 量子力学}
\begin{figure}[ht]
\centering
\includegraphics[width=10cm]{./figures/12dcaac68fbdbd41.png}
\caption{从左到右:W.能斯托,A.爱因斯坦,M.普朗克,R. A. 米利根 和 冯·劳厄于1931年11月11日在柏林的冯·劳厄的晚宴上} \label{fig_Max_5}
\end{figure}
在20世纪20年代末,玻尔、海森堡和泡利研究出了量子力学的哥本哈根解释,但遭到普朗克、薛定谔、劳厄和爱因斯坦的拒绝。普朗克预计,波动力学很快会让量子理论——他自己的孩子——变得没有必要。然而事实并非如此。进一步的工作只巩固了量子理论,即使与他和爱因斯坦的哲学厌恶相对立。普朗克从年轻时与旧观点的斗争中体验到了自己早期观察的真理:“新的科学真理不会通过说服其对手并让他们看到光明而获胜,而是因为其对手最终会死去,并且新一代熟悉它的人会成长。”[28]

\subsubsection{1.9 纳粹独裁和第二次世界大战}
当纳粹1933年掌权时,普朗克已经74岁了。他目睹了许多犹太朋友和同事被开除出他们的职位并受到羞辱,数百名科学家从纳粹德国移民出去。他再次试图“坚持不懈,继续工作”,并要求考虑移民的科学家留在德国。尽管如此,他还是帮助他的侄子,经济学家赫尔曼·克拉诺德在被捕后移民到了伦敦。[29]他希望危机会很快平息,政治局势会有所改善。

“如果你今天能召集30位这样的绅士,那么明天就会有150位其他人来反对它,因为他们渴望取代其他人的位置。”[30]在普朗克的领导下,凯撒威廉协会(KWG)避免了与纳粹政权的公开冲突,除了关于弗里茨·哈伯的问题。普朗克试图与阿道夫·希特勒讨论这个问题,但没有成功。第二年,1934年,哈伯在流亡中去世。

一年后,自1930年以来一直担任KWG总统的普朗克以一种有点挑衅的方式为哈伯组织了一次官方纪念会议。他还成功秘密地让一些犹太科学家继续在KWG的研究所工作了几年。1936年,他的KWG总统任期结束,纳粹政府迫使他不再寻求连任。

随着德国的政治气候逐渐变得更加敌对,德国物理(Deutsche Physik,又称“雅利安物理”)的著名倡导者约翰尼斯·斯塔克抨击普朗克、索末菲和海森堡继续教授爱因斯坦的理论,称他们为“白人犹太人”。纳粹政府科学办公室开始调查普朗克的祖先,声称他是“1/16犹太人”,但普朗克本人否认了这一点。[31]
\begin{figure}[ht]
\centering
\includegraphics[width=6cm]{./figures/ba681d5b12b2a056.png}
\caption{马克斯普朗克在哥廷根的坟墓} \label{fig_Max_6}
\end{figure}
1938年,普朗克庆祝了他的80岁生日。在庆祝活动中,法国物理学家路易·德布罗意获得了马克斯·普朗克奖章奖(1928年由DPG创立为最高奖章)。1938年底,普鲁士学院失去了它剩余的独立性,被纳粹分子接管了(一体化)。普朗克辞去总统职务以示抗议。他继续频繁旅行,发表了许多公开演讲,比如他的宗教和科学演讲,五年后,他已经足够健康,可以攀登阿尔卑斯山3000米的山峰。

第二次世界大战期间,越来越多的盟军针对柏林的轰炸任务迫使普朗克和他的妻子暂时离开城市,住在农村。1942年,他写道:“在我心中,一种强烈的愿望已经增长,那就是坚持生活在这场危机中,并活得足够长,能够见证一个转折点,一个新崛起的开始。”1944年2月,他在柏林的家被一次空袭彻底摧毁,摧毁了他所有的科学记录和通信。他在农村的撤退受到了双方盟军快速推进的威胁。

1944年,普朗克的儿子欧文在7月20日暗杀希特勒未遂后被盖世太保逮捕。1944年10月,欧文被人民法院审判并判处死刑。欧文于1945年1月被绞死在柏林的波兰监狱。他儿子的死摧毁了普朗克的大部分生存意志。[32]战争结束后,普朗克带着他的第二任妻子和儿子来到了一个亲戚家,普朗克于1947年10月4日在那里去世。他的坟墓位于哥廷根的老墓地。[33]

\subsection{宗教观点}
普朗克是德国路德教会的成员。[34]然而,普朗克对其他观点和宗教非常宽容。[35]在1937年的一次题为“宗教与自然科学”的讲座中,他提出了这些符号和仪式的重要性,这些符号和仪式与信徒崇拜上帝的能力直接相关,但必须注意,这些符号提供了对神性的不完美的说明。他批评无神论者专注于嘲笑这种符号,同时警告信徒对这种符号重要性的过度估计。[36]

他对所有宗教都很宽容和友好。尽管普朗克仍然留在路德教会,但他并没有宣扬基督教或圣经的观点。普朗克认为“对奇迹的信念必须在科学事实稳步前进之前一步一步地产生,它的彻底失败无疑是时间问题。” [37]

普朗克在他1937年的讲座“宗教与自然”中表达了上帝无处不在的观点,并认为“难以理解的神性是通过符号的神圣来传达的。”他认为无神论者过于重视仅仅是象征的东西。普朗克从1920年到去世一直担任牧师,他相信全能、无所不知、仁慈的上帝(尽管不一定是个人的上帝)。科学和宗教都发起了一场“反对怀疑主义和教条主义,反对不信和迷信的不懈斗争”,目标是“向着上帝!”[37]

马克斯·普朗克在1944年说,“作为一个把一生都奉献给最清醒的科学和物质研究的人,作为我对原子研究的结果,我可以告诉你的有这么多: 没有物质本身。所有物质的起源和存在都是依靠一种力量,这种力量使原子的粒子振动,并将原子的这个最微小的太阳系保持在一起。我们必须假设在这种力量的背后存在着一种有意识和智慧的精神(原始精神)。这种精神是所有物质的矩阵。”[38]

普朗克认为这位科学家是一个富有想象力和基督教信仰的人。他说:“宗教和科学都需要对上帝的信仰。对信徒来说,上帝是开始,对物理学家来说,他是所有考虑的终点……对前者来说,他是基础,对后者来说,他是每个广义世界观大厦的王冠”。[39]

另一方面,普朗克写道“..相信意味着承认为真理”,自然知识在无可争议的安全轨道上不断前进,这使得一个受过自然科学训练的人完全不可能认识到许多违背自然规律的非常事件的报告是建立在真理的基础上的,奇迹仍然被普遍认为是宗教教义的基本支持和确认,而这些以前被认为是纯粹和简单的事实,没有疑问或批评。对奇迹的信念必须在无情而可靠地推进科学之前一步步后退,我们不能怀疑奇迹迟早会彻底消失。”[40]

晚年,普朗克对上帝的看法是自然神论者的看法。[41]例如,在他去世前六个月,有谣言说普朗克已经皈依天主教,但当被问及是什么促使他做出这一举动时,他宣称,尽管他一直都是虔诚的宗教信徒,但他并不相信“个人的上帝,更不用说基督教的上帝。”[42]

\subsection{出版物}
\begin{itemize}
\item Planck, M. (1900a). "Über eine Verbesserung der Wienschen Spektralgleichung". Verhandlungen der Deutschen Physikalischen Gesellschaft. \textbf{2}: 202–204. 翻译成 ter Haar, D. (1967). "On an Improvement of Wien's Equation for the Spectrum". The Old Quantum Theory (PDF). Pergamon Press. pp. 79–81. LCCN 66029628.
\item Planck, M. (1900b). "Zur Theorie des Gesetzes der Energieverteilung im Normalspectrum". Verhandlungen der Deutschen Physikalischen Gesellschaft. \textbf{2}: 237. 翻译成 ter Haar, D. (1967). "On the Theory of the Energy Distribution Law of the Normal Spectrum". The Old Quantum Theory (PDF). Pergamon Press. p. 82. LCCN 66029628.
\item Planck, M. (1900c). "Entropie und Temperatur strahlender Wärme" [Entropy and Temperature of Radiant Heat]. Annalen der Physik. \textbf{306} (4): 719–737. Bibcode:1900AnP...306..719P. doi:10.1002/andp.19003060410.
\item Planck, M. (1900d). "Über irreversible Strahlungsvorgänge" [On Irreversible Radiation Processes]. Annalen der Physik. \textbf{306} (1): 69–122. Bibcode:1900AnP...306...69P. doi:10.1002/andp.19003060105.
\item Planck, M. (1901). "Über das Gesetz der Energieverteilung im Normalspektrum". Annalen der Physik. 309 (3): 553–563. Bibcode:1901AnP...\textbf{309}..553P. doi:10.1002/andp.19013090310. 翻译成 Ando, K. "On the Law of Distribution of Energy in the Normal Spectrum" (PDF). Archived from the original (PDF) on 6 October 2011. Retrieved 13 October 2011.
\item Planck, M. (1903). Treatise on Thermodynamics. Ogg, A. (transl.). London: Longmans, Green & Co. OL 7246691M.
\item Planck, M. (1906). Vorlesungen über die Theorie der Wärmestrahlung. Leipzig: J.A. Barth. LCCN 07004527.
\item Planck, M. (1914). The Theory of Heat Radiation. Masius, M. (transl.) (2nd ed.). P. Blakiston's Son & Co. OL 7154661M.
\item Planck, M. (1915). Eight Lectures on Theoretical Physics. Wills, A. P. (transl.). Dover Publications. ISBN 0-486-69730-4.
\item Planck, M. (1943). "Zur Geschichte der Auffindung des physikalischen Wirkungsquantums". Naturwissenschaften. 31 (14–15): 153–159. Bibcode:1943NW.....\textbf{31}..153P. doi:10.1007/BF01475738.
\end{itemize}

\subsection{参考文献}
[1]
^"Planck". Random House Webster's Unabridged Dictionary..

[2]
^"Planck's constant". Cambridge Dictionary..

[3]
^The Nobel Prize in Physics 1918. Nobelprize.org. Retrieved on 2011-07-05..

[4]
^Fraenkel, Abraham (2016). Recollections of a Jewish Mathematician in Germany. Basel, Switzerland: Birkhäuser. p. 96. ISBN 9783319308456..

[5]
^Dilip, James (2017). A Gestalt Aether Theory on the Nature of Light and Related Phenomena: A New Physics. Educreation Publishing. p. 6. ISBN 9781545703182..

[6]
^Weir, Jane (2009). Max Planck: Revolutionary Physicist (in 英语). Capstone. ISBN 9780756540739..

[7]
^Christoph Seidler, Gestatten, Marx Planck Archived 29 6月 2011 at the Wayback Machine, Spiegel Online, 24 April 2008.

[8]
^Press release of the Max Planck Society about Max Planck's name..

[9]
^Encyclopædia Britannica: Max Planck.

[10]
^Lightman, Alan P. (2005). The discoveries: great breakthroughs in twentieth-century science, including the original papers. Toronto: Alfred A. Knopf Canada. p. 8. ISBN 0-676-97789-8..

[11]
^Planck, Max (1897). Vorlesungen über Thermodynamik. Leipzig: Verlag Von Veit & Company. Retrieved 27 June 2012. English translation: Planck, Max (1903). Treatise on Thermodynamics. London: Longmans, Green, and Company. Retrieved 27 June 2012..

[12]
^"Max Planck - Biographical". Nobelprize.org. Nobel Prize Organisation. Retrieved 26 February 2017..

[13]
^Jacques Hadamard (1915). Four lectures on mathematics: delivered at Columbia University in 1911. Columbia University Press. pp. 7–. Retrieved 5 July 2011..

[14]
^"Max Planck — Humboldt-Universität zu Berlin". www.hu-berlin.de (in 英语). Retrieved 15 May 2016..

[15]
^"Erwin Schrödinger — Humboldt-Universität zu Berlin". www.hu-berlin.de (in 英语). Retrieved 15 May 2016..

[16]
^Jürgen Heideking; Christof Mauch (5 October 1998). American Intelligence and the German Resistance to Hitler: A Documentary History. Westview Press. pp. 361–. ISBN 978-0-8133-3636-7. Retrieved 5 July 2011..

[17]
^Verband Deutscher Elektrotechniker; Elektrotechnischer Verein (Berlin, Germany) (1948). "ETZ: Elektrotechnische Zeitschrift: Ausg. A." ETZ: Elektrotechnische Zeitschrift (in German). VDE-Verlag. 69 (A).CS1 maint: Unrecognized language (link), Snipped extract.

[18]
^"Max Planck - The Mathematics Genealogy Project". www.genealogy.math.ndsu.nodak.edu. Retrieved 2017-06-05..

[19]
^For a solid approach to the complexity of Planck's intellectual motivations for the quantum, for his reluctant acceptance of its implications, see Helge Kragh, Max Planck: the reluctant revolutionary, Physics World. December 2000..

[20]
^Kragh, Helge (1 December 2000), Max Planck: the reluctant revolutionary, PhysicsWorld.com.

[21]
^"The Nobel Prize in Physics 1918". www.nobelprize.org. Retrieved 2017-06-11..

[22]
^Heilbron, 2000, page 8.

[23]
^Born, M. (1948). "Max Karl Ernst Ludwig Planck. 1858–1947". Obituary Notices of Fellows of the Royal Society. 6 (17): 161–188. doi:10.1098/rsbm.1948.0024..

[24]
^Heilbron, 2000, page 72.

[25]
^Evans, James; Thorndike, Alan S. (2007). Quantum mechanics at the crossroads: new perspectives from history, philosophy and physics. Springer. p. 31. ISBN 3-540-32663-4. Extract of page 31.

[26]
^"Max Karl Ernst Ludwig Planck (1858–1947)". Royal Netherlands Academy of Arts and Sciences. Retrieved 4 August 2015..

[27]
^Scully, Robert J.; Scully, Marlan O. (2007). The demon and the quantum: from the pythagorean mystics to Maxwell's demon and quantum mystery. Wiley-VCH. p. 90. ISBN 3-527-40688-3., Chapter 7, p 90.

[28]
^Quoted in Thomas Kuhn, The Structure of Scientific Revolutions (1970 ed.): p. 150..

[29]
^"Johanna Kranold Stein". Ithaca Journal. Legacy.com. Retrieved 10 October 2016..

[30]
^In a slightly different translation, Hahn remembers Planck saying: "If you bring together 30 such men today, then tomorrow 150 will come to denounce them because they want to take their places." This translated quote is found in: Heilbron, 2000, p. 150. Heilbron, at the end of the paragraph, on p. 151, cites the following references to Hahn’s writings: Otto Hahn Einige persönliche Erinnerungen an Max Planck MPG, Mitteilungen (1957) p. 244, and Otto Hahn My Life (Herder and Herder, 1970) p. 140..

[31]
^Heilbron, 2000, page 191.

[32]
^"Max Karl Ernst Ludwig Planck". Archived from the original on 12 May 2008. Retrieved 2010-06-17.CS1 maint: BOT: original-url status unknown (link).

[33]
^Max Planck's Grave at Göttingen, Germany, Youtube, retrieved 2016-01-04.

[34]
^Erich Dinkler, Planck, Max, in Die Religion in Geschichte und Gegenwart, Third Edition, Volume V, Tübingen (Germany), 1961, col. 404-405.

[35]
^The Religious Affiliation of Physicist Max Planck. adherents.com. Retrieved on 2011-07-05..

[36]
^The Life Max Planck. encyclopedia.com. Retrieved on 2012-03-07..

[37]
^"The religion of Max Planck, physicist". www.adherents.com. Retrieved 2018-03-20..

[38]
^Das Wesen der Materie [The Nature of Matter], speech at Florence, Italy (1944) (from Archiv zur Geschichte der Max-Planck-Gesellschaft, Abt. Va, Rep. 11 Planck, Nr. 1797).

[39]
^Religion and Natural Science (Lecture Given 1937) Scientific Autobiography and Other Papers, trans. F. Gaynor (New York, 1949), pp. 184 (from http://en.wikiquote.org/wiki/Max_Planck).

[40]
^Max Planck, Scientific Autobiography and Other Papers.

[41]
^J. L. Heilbron (1986). The Dilemmas of an Upright Man: Max Planck and the Fortunes of German Science. Harvard University Press. p. 198. ISBN 9780674004399. On the other side, Church spokesmen could scarcely become enthusiastic about Planck's deism, which omitted all reference to established religions and had no more doctrinal content than Einstein's Judaism. It seemed useful therefore to paint the lily, to improve the lesson of Planck's life for the use of proselytizers and to associate the deanthropomorphizer of science with a belief in a traditional Godhead..

[42]
^Heilbron, 2000, page 198.