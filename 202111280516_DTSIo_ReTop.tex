% 实数集的拓扑

\pentry{实数\upref{ReNum}}

\subsection{开集与闭集}

首先给出如下定义.

\begin{definition}{开集与闭集}
设$x$是实数. 任意包含$x$的开区间都称作$x$的一个开邻域 (open neighbourhood).

实数集$\mathbb{R}$的子集$U$称为开集 (open set), 如果对于任意$x\in U$, 都存在$x$的开邻域$V_x$使得$V_x\subset U$. 

实数集$\mathbb{R}$的子集$C$称为闭集 (closed set), 如果$\mathbb{R}\setminus C$是开集.

规定空集既是开集也是闭集.
\end{definition}

容易证明如下性质:

\begin{theorem}{开集和闭集的运算}
任意多个开集的并集仍然是开集. 有限多个开集的交集仍然是开集.

等价地, 任意多个闭集的交集仍然是闭集. 有限多个闭集的并集仍然是闭集.
\end{theorem}

\begin{exercise}{}
证明这个定理. 提示: 设$\{U_\alpha\}_{\alpha\in A}$是一族开集, 那么若$x\in \cup_{\alpha\in A}U_\alpha$, 则必定有一$\alpha$使得$x\in U_\alpha$. 如果$U_1,...,U_n$是有限多个开集, $x\in\cap_{k=1}^nU_k$, 而$V_x^k$是$x$的包含在$U_x^k$中的开邻域, 那么$\cap_{k=1}^nV_x^k$还是$x$的开邻域.
\end{exercise}

\begin{exercise}{}
在证明"有限多开集的交集还是开集"时, "有限"这个条件究竟被用在哪里? 可以参考下面的反例.
\end{exercise}

\begin{example}{一些反例}
无限多个开集的交集不一定是开集. 例如, 设开区间$U_k=(-1/k,1/k)$, 那么$\cap_{k=1}^\infty=\{0\}$. 相应地, 无限多个闭集的并集也不一定是闭集, 例如, 设闭区间$I_k=[0,1-1/k]$, 则$\cup_{k=1}^\infty I_k=[0,1)$, 它不是开集也不是闭集.
\end{example}

粗略地说, 在一个集合上给定拓扑, 就是给定一个衡量元素之间的"远近关系"的尺度. 在实数集$\mathbb{R}$中, 一个给定的实数$x$的全体开邻域就划定了距离这个实数的"远近关系". 

\subsection{开集的结构}
在实数集$\mathbb{R}$中, 开集的结构可以被清楚地刻画出来. 首先引入一个定义: 包含于非空开集$G\subset\mathbb{R}$中的开区间$(a,b)$称为一个分支 (component), 如果端点$a,b\notin G$. 容易看出, 任何非空开集中的两个分支一定不相交.

\begin{theorem}{实数集中开集的结构}
每一个非空开集$G\subset\mathbb{R}$都是至多可数个分支的并集. 
\end{theorem}
\textbf{证明.} 首先注意到任何一点$x\in\mathbb{R}$都一定属于某个分支$U$: 这个分支是所有包含在$G$中且包含$x$的开区间的并集. 为了说明它符合分支的定义, 首先注意到$U$当然是个区间. 进一步, 可以反设, 例如, $U$的左端点$a\in G$; 那么有$a$的开邻域$U_a\subset G$, 从而$U_a\cup U$也是包含$x$的区间, 但它严格包含了$U$, 同$U$的定义相违背.

接下来, 按照这个推理, 注意到$G$中的任何有理数$r$都属于某个分支$U_r$. 这些分支或者不相交, 或者重合. 由此, $G$的全体分支被$G$中所包含的you'li'sh

\subsection{距离, 极限点与闭包}

