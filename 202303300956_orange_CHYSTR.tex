% 斜面应力公式

\pentry{应力\upref{STRESS}}
\footnote{本文参考了陆明万的《弹性力学》。}

\begin{figure}[ht]
\centering
\includegraphics[width=5cm]{./figures/CHYSTR_1.pdf}
\caption{斜切面$\bvec n$上的应力$\bvec \sigma_\nu$。$\bvec n$是面的单位法向量} \label{CHYSTR_fig1}
\end{figure}

在应力\upref{STRESS}中,我们已经处理了微元体上的应力问题。那么新的问题来了,如\autoref{CHYSTR_fig1} 所示,假设我们已经在某个坐标系中算出了微元体的各个应力,如果我们在微元体中取一个切面,那么要如何计算这个切面上的应力?

\textsl{跨界大牛}柯西给出了一个漂亮的回答,因此该公式也称柯西应力公式。柯西应力公式写成矩阵乘法\upref{Mat}的形式:
\begin{equation}\label{CHYSTR_eq1}
\begin{pmatrix}
\sigma_{\nu,1}\\
\sigma_{\nu,2}\\
\sigma_{\nu,3}\\
\end{pmatrix}
=
\begin{pmatrix}
\sigma_{11} & \tau_{12} & \tau_{13} \\
\tau_{21} & \sigma_{22} & \tau_{23} \\
\tau_{31} & \tau_{32} & \sigma_{33} \\
\end{pmatrix}
\begin{pmatrix}
n_1\\
n_2\\
n_3\\
\end{pmatrix}~.
\end{equation}
还可以写成更紧凑的张量乘法形式,
\begin{equation}
\bvec \sigma_\nu = \sigma \cdot n ~,
\end{equation}
或指标形式
\begin{equation}
\bvec (\bvec \sigma_\nu)_i = \sum_j \sigma_{ij} n_j~.
\end{equation}
因为已经被投影,此处的$\bvec \sigma_\nu$已经是一个矢量,而不是原先的张量。此外,斜面上的应力$\bvec \sigma_\nu$不一定垂直于该平面,而可以朝任意方向。该公式的论证方法大致是在三角形台中运用力的平衡方程,此处按下不表。

\begin{figure}[ht]
\centering
\includegraphics[width=5cm]{./figures/CHYSTR_2.pdf}
\caption{$\sigma_\nu$中各个分量的含义} \label{CHYSTR_fig2}
\end{figure}

回过头来,我们思考一下,我们依据上文方法写出的$\sigma_\nu=
\begin{pmatrix}
\sigma_{\nu,1}\\
\sigma_{\nu,2}\\
\sigma_{\nu,3}\\
\end{pmatrix}$中各个分量的含义是什么?显然(但又有点让人困惑),各个分量仍基于原先的坐标系,如\autoref{CHYSTR_fig2} 所示。

\begin{figure}[ht]
\centering
\includegraphics[width=5cm]{./figures/CHYSTR_3.pdf}
\caption{$\sigma_\nu$的斜面表示} \label{CHYSTR_fig3}
\end{figure}

我们写出斜面所受应力,总希望他和斜面有点关系(而不是原先的坐标系);因此,我们得想办法把它改写为和斜面有关系的、更“直观”的形式,例如表达为沿斜面法向与切向的应力。

只需再做一步简单的向量投影就能实现这个目的:法向应力$\bvec \sigma_n$即是$\bvec \sigma_\nu$在$\bvec n$上的投影:
\begin{equation}
\bvec \sigma_n = (\bvec \sigma_\nu \cdot \bvec n)  \hat{\bvec n}~.
\end{equation}
他的数值大小写成张量乘法的形式
\begin{equation}
\sigma_n = n \cdot \sigma \cdot n 
\qquad \sigma_n =\sum_i \sum_j \sigma_{ij} n_i n_j~,
\end{equation}
而切向应力就是
\begin{equation}
\bvec \tau = \bvec \sigma_\nu - \bvec \sigma_n~.
\end{equation}

至此我们已经解决了文章开头的问题,剩下的只是繁琐的数学计算。不过,我们还可以额外思考另一个问题:如\autoref{CHYSTR_eq1} 所示,柯西斜面应力公式相当于一个矩阵乘向量的矩阵乘法。
$$\sigma_\nu = \sigma \cdot n~.$$
如果你的数理直觉足够敏锐,就会意识到这里存在一个矩阵的本征值与本征向量问题\upref{MatEig}:存不存在一种$n_0$,使
$$
\sigma \cdot n_0 = \lambda n_0~.
$$?
更进一步的思考后,你会发现如果$n_0$存在,那么$\bvec \sigma_\nu = \lambda \bvec n_0$将平行于$\bvec n_0$,也即在这个面上将只有正应力。那么,$n_0$将是一个很特殊的平面。到这一步,你其实就已经想明白了主应力\upref{PRSTR}问题。

