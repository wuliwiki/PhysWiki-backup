% 热平衡 热力学第零定律
% 热平衡|热力学第零定律|温度|容器

\begin{issues}
\issueDraft
\end{issues}

\subsection{热平衡}
热力学研究的对象是一个由大量微观粒子(分子或其他粒子)组成的一个宏观物质系统(例如一个绝热容器中的气体).经验指出,一个孤立系统(与外界没有物质和能量交换的系统)若放置得足够久,将会达到这样一种状态——系统的各种\textbf{宏观性质}(例如温度、压强、化学势等物理性质和化学性质)在长时间内部发生任何变化.这称为热力学平衡态.

一般来说,热平衡的具体要求有热学平衡、化学平衡和力学平衡.简单地来说,就是系统内温度处处相等(,压强处处相等,化学组成处处相同.这些都是宏观可观测的性质,例如,加入系统的两个
\addTODO{弛豫时间}
\addTODO{态函数}

\subsection{热力学第零定律}
当我们提及温度,我们会认为它是度量了一个系统的冷热程度的物理量,或者说这个物理量衡量了系统自发放热的能力(温度越高,那么它的放热能力应当越强.但这些都是基于经验的“直觉”,并非温度的定义.要考虑温度,我们必须思考热平衡系统的“属性”以及不同系统之间的关系.

为了阐释清楚温度概念,人们提出了\textbf{热力学第零定律}:
若物体 $A$ 与物体 $C$ 达到热平衡(将它们接触时没有“热量”传递,表征为宏观性质没有发生变化), 且物体 $B$ 与物体 $C$ 也达到热平衡, 那么 $A$ 和 $B$ 之间同样有热平衡.上面的 $A,B,C$ 可以替换成装有气体的导热容器,定律仍然成立.

这意味着我们可以引入一个物理量(称它为“温度”),用同一个“温度”值来标定一切处于热平衡的的物质.因此热平衡定律指明了比较温度的方法.

\addTODO{温度计的概念}
\addTODO{热力学温标}
