% 莱昂哈德·欧拉(综述)
% license CCBYSA3
% type Wiki

本文根据 CC-BY-SA 协议转载翻译自维基百科\href{https://en.wikipedia.org/wiki/Leonhard_Euler}{相关文章}。

莱昂哈德·欧拉(Leonhard Euler,发音:/ˈɔɪlər/ OY-lər;[b] 德语:[ˈleːɔnhaʁt ˈʔɔʏlɐ] ⓘ,瑞士标准德语:[ˈleɔnhard ˈɔʏlər];1707年4月15日 – 1783年9月18日)是瑞士数学家、物理学家、天文学家、地理学家、逻辑学家和工程师。他是图论和拓扑学的创始人,并在其他多个数学分支(如解析数论、复分析和微积分)中做出了开创性和深远的发现。他引入了许多现代数学术语和符号,包括数学函数的概念。他还以在力学、流体动力学、光学、天文学和音乐理论等领域的贡献而闻名。

欧拉被认为是历史上最伟大、最多产的数学家之一,也是18世纪最伟大的数学家。许多在欧拉去世后才产生的伟大数学家都承认他在这一领域的重要性,正如他们的名言所示:皮埃尔-西蒙·拉普拉斯曾通过一句话表达欧拉对数学的影响:“读欧拉,读欧拉,他是我们的导师。”卡尔·弗里德里希·高斯写道:“研究欧拉的作品将是学习数学各个领域的最佳学校,其他任何东西都无法替代它。”欧拉的866篇论文和他的信件正在被收集成《欧拉全集》(Opera Omnia Leonhard Euler),完成后将包含81卷四开本。欧拉大部分成年生活都在俄罗斯圣彼得堡和普鲁士首都柏林度过。

欧拉还被认为是第一个推广使用希腊字母π(小写pi)来表示圆的周长与直径的比率的人,以及第一个使用符号f(x)来表示函数值的人。他还使用字母i表示虚数单位√(-1),使用希腊字母Σ(大写sigma)表示求和,使用希腊字母Δ(大写delta)表示有限差分,使用小写字母表示三角形的边,使用大写字母表示角度。他给出了常数e的定义,它是自然对数的底数,现在被称为欧拉数。

欧拉还被认为是第一个发展图论的人(部分因为他解决了“柯尼斯堡七桥问题”,这也被认为是拓扑学的第一个实际应用)。他还因解决多个未解的数论和分析学问题而声名远播,包括著名的巴塞尔问题。欧拉还被誉为发现了多面体的顶点和面数之和减去边数等于2,这个数字现在被称为欧拉示性数。在物理学领域,欧拉将牛顿的物理定律重新表述为新的定律,并在他的两卷本著作《力学》中更好地解释了刚体的运动。他还为固体物体的弹性变形研究做出了重大贡献。
\subsection{早期生活}
莱昂哈德·欧拉于1707年4月15日出生在巴塞尔,父亲保罗三世·欧拉是改革宗教会的牧师,母亲玛格丽特(娘家姓布鲁克尔),她的祖先包括许多著名的古典学者。[16] 欧拉是家中的长子,下面有两个妹妹,安娜·玛利亚和玛利亚·玛格达莱娜,以及一个弟弟约翰·海因里希。[17][16] 在欧拉出生后不久,欧拉一家从巴塞尔搬到瑞士的里恩镇,父亲成为当地教堂的牧师,而欧拉则在这里度过了大部分童年。[16]

从小,欧拉就接受了父亲的数学教育,父亲曾在巴塞尔大学向雅各布·伯努利学习过一些课程。大约在八岁时,欧拉被送到外祖母家生活,并在巴塞尔的拉丁学校就读。此外,他还接受了年轻神学家约翰内斯·布尔卡特的私人辅导,布尔卡特对数学有浓厚的兴趣。[16]

1720年,欧拉13岁时进入巴塞尔大学。[7] 在当时,年轻就读大学并不罕见。[16] 数学基础课程由已故的雅各布·伯努利的弟弟约翰·伯努利教授(雅各布·伯努利曾是欧拉父亲的老师)讲授。约翰·伯努利和欧拉很快熟识。欧拉在自传中描述了伯努利的教导:[18]

“著名教授约翰·伯努利……特别乐意帮助我在数学科学上取得进展。然而,他由于日程繁忙,拒绝了私人的授课请求。然而,他给了我一个更为有益的建议,就是让我自己去掌握一些更为艰深的数学书籍,并以极大的勤奋把它们研读一遍,如果遇到疑难或困难,他每周六下午都会开放时间免费为我解答,他如此慷慨地评论我的问题,以至于当他解答我一个疑问时,十个疑问随之消失,这无疑是数学科学进步的最佳方法。”

正是在这段时间里,在伯努利的支持下,欧拉得到了父亲的同意,决定成为一名数学家,而不是继续成为牧师。[19][20]

1723年,欧拉获得哲学硕士学位,并撰写了一篇论文,比较了笛卡尔和牛顿的哲学。[16] 此后,他又进入了巴塞尔大学的神学系。[20]

1726年,欧拉完成了一篇关于声音传播的论文《De Sono》[21][22],试图通过这篇论文获得巴塞尔大学的职位,但未成功。[23] 1727年,欧拉第一次参加了巴黎科学院的奖学金竞赛(该竞赛自1720年起每年举办,后来改为每两年一次)[24]。当年题目是寻找在船上最好的桅杆安放方式。皮埃尔·布盖,后被称为“海军建筑学之父”,获得了第一名,欧拉则获得了第二名。[25] 在接下来的几年里,欧拉共参加了15次该竞赛,赢得了其中的12次。[24][25]
\subsection{职业生涯}
\subsubsection{圣彼得堡}