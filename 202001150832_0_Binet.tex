% 轨道方程 比耐公式
% keys 轨道方程|比耐公式

\pentry{中心力场问题\upref{CenFrc}}
我们来看 “中心力场问题\upref{CenFrc}” 中得到的两条运动方程(\autoref{CenFrc_eq5} 和\autoref{CenFrc_eq4})
\begin{align}
\ddot{r} - r \dot\theta^2 &= F(r)/m \label{Binet_eq1}\\
mr^2\dot \theta &= L \label{Binet_eq2}
\end{align}
为了得到极坐标中 $r(\theta)$ 的微分方程(\textbf{轨道方程}), 我们以下用\autoref{Binet_eq2} 消去\autoref{Binet_eq1} 中的 $t$. 首先可以把 $r$ 看做复合函数 $r[\theta(t)]$, 再用链式法则\upref{ChainR}处理\autoref{Binet_eq1} 的第一项
\begin{equation}\label{Binet_eq10}\ali{
\ddot{r} & = \dv{t} \qty( \dv{r}{t} ) = \dv{t} \qty( \dv{r}{\theta} \dv{\theta}{t} ) = \dv{\theta}\qty( \dv{r}{\theta} ) \qty( \dv{\theta}{t} )^2 + \dv{r}{\theta}\dv[2]{\theta}{t}\\
& = \dv[2]{r}{\theta} \qty( \dv{\theta}{t} )^2 + \dv{r}{\theta}\dv{\theta} \qty( \dv{\theta}{t} )\dv{\theta}{t}
}\end{equation}
然后把\autoref{Binet_eq2} 代入\autoref{Binet_eq1} 消去所有 $\dot\theta = \dv*{\theta}{t}$, 得到 $r$ 关于 $\theta$ 的微分方程
\begin{equation}
\dv[2]{r}{\theta} \qty( \frac{L}{r^2} )^2 + \dv{r}{\theta}\dv{\theta} \qty( \frac{L}{r^2} )\frac{L}{r^2} - r \qty( \frac{L}{r^2} )^2 =  m^2 F(r)
\end{equation}
即
\begin{equation}\label{Binet_eq5}
\dv[2]{r}{\theta} + r^2\dv{r}{\theta}\dv{\theta} \qty( \frac{1}{r^2} ) - r =  \frac{m^2 r^4}{L^2} F(r)
\end{equation}
这就是轨道方程. 这个方程比较复杂, 但可以通过换元法% 未完成:介绍微分方程的换元
化为十分简洁的形式.令
\begin{equation}\label{Binet_eq13}
u \equiv \frac{1}{r}
\end{equation}
代入\autoref{Binet_eq5},  得到 $u$ 关于 $\theta $ 的微分方程
\begin{equation}\label{Binet_eq15}
\dv[2]{u}{\theta} + u = -\frac{m}{L^2 u^2} F\qty(\frac 1u)
\end{equation}
这个二阶微分方程被称为\textbf{比耐公式}.
