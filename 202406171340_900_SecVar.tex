% 二次变分
% keys 二次变分
% license Xiao
% type Tutor


\pentry{变分\nref{nod_Varia}}{nod_1920}
在数学分析中,一次(或阶)微分为0是函数取极值的必要条件,为了获得极值是极大值或极小值的信息,可以研究二次微分。同样的,在泛函中,一次变分为0是泛函取极值的必要条件,而研究二次变分可以获得极值是极大值或极小值的信息。

泛函 $J(y)=\int_a^bF(x,y,y')\dd x$ 的\textbf{二次变分} $\delta^2J$ 是指
\begin{equation}
\delta^2J=\frac{1}{2}\int_a^b(F_{yy}\delta y^2+2F_{yy'}\delta y\delta y'+F_{y'y'}\delta y'^2)\dd x~.
\end{equation}
\subsection{二次变分的引入}
设 
\begin{equation}
J(y)=\int_a^bF(x,y,y')\dd x~
\end{equation}
为定义在有固定端点的 $C_1$ 类曲线上的泛函。

应用泰勒公式并引用符号\autoref{eq_Varia_6} 
\begin{equation}
\delta y(x)=\overline{y}(x)-y(x)~.
\end{equation}
此处 $\delta y(a)=\delta y(b)=0$。则
\begin{equation}
\begin{aligned}
J(\overline{y})-J(y)&=\int_a^b[F(x,\overline{y},\overline{y}')-F(x,y,y')]\dd x\\
&=\int_a^b\qty[F'_y\delta y+F'_{y'}\delta y'+\frac{1}{2}\qty(\tilde{F}''_{yy}\delta y^2+2\tilde{F}''_{yy'}\delta y\delta y'+\tilde{F}''_{y'y'}\delta y'^2)]\dd x~.
\end{aligned}
\end{equation}
其中, $\tilde{F}_{yy}=F_{yy}(x,y+\theta_1\delta y,y'+\theta_2\delta y'),\;(\abs{\theta_1},\abs{\theta_2}\leq1)$,$\tilde{F}_{yy},F_{yy'},F_{y'y'}$ 类似。由于 $F$ 对其变量二阶连续,当一级距离\autoref{def_AbPol_1}  $r(\overline{y},y)$ 充分小时,就有
\begin{equation}
\tilde F_{yy}=F_{yy}+\epsilon_1~,\quad \tilde F_{yy'}=F_{yy'}+\epsilon_2~,\quad \tilde F_{yy}=F_{y'y'}+\epsilon_3~,
\end{equation}
其中 $\epsilon_1,\epsilon_2,\epsilon_3$ 随 $r(\overline{y},y)$ 趋于零。所以
\begin{equation}\label{eq_SecVar_1}
J(\overline{y})-J(y)=\int_a^b(F'_y\delta y+F'_{y'}\delta y')\dd x+\frac{1}{2}\int_a^b(F''_{yy}\delta y^2+2F''_{yy'}\delta y\delta y'+F''_{y'y'}\delta y'^2)\dd x+\epsilon~.
\end{equation}
其中, $\epsilon=\int_a^b(\epsilon_1\delta y^2+2\epsilon_2\delta y\delta y'+\epsilon_3\delta y'^2)\dd x$
。因为 $\abs{2\delta y\delta y'}\leq \delta y^2+\delta y'^2$,所以 $\abs{\epsilon}\leq \int_a^b(\epsilon_3\delta y^2+\epsilon_5\delta y'^2)\dd x$ ,其中 $\epsilon_3,\epsilon_4$ 随 $r(\overline{y},y)$ 而一致趋于0.但是
\begin{equation}
\abs{\delta y}\leq r(\overline{y},y)~,\quad \abs{\delta y'}\leq r(\overline{y},y)~.
\end{equation}
因此, $\abs{\epsilon}\leq (\epsilon_4+\epsilon_5)(b-a)r(\overline{y},y)^2$。从而可见,$\epsilon$ 是比 $r(\overline{y},y)$ 更高阶的无穷小。\autoref{eq_SecVar_1} 中略去这一项,就有
\begin{equation}
J(\overline{y})-J(y)\approx\delta J+\delta^2J~.
\end{equation}
其中,
\begin{equation}
\begin{aligned}
&\delta J=\int_a^b(F'_y\delta y+F'_{y'}\delta y')\dd x~,\\
&\delta^2 J=\frac{1}{2}\int_a^b(F''_{yy}\delta y^2+2F''_{yy'}\delta y\delta y'+F''_{y'y'}\delta y'^2)\dd x~.
\end{aligned}
\end{equation}
与二阶微分类似, $\delta^2J$ 被称为\textbf{二次(阶)变分}。
