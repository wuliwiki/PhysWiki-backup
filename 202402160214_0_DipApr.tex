% 偶极子近似(量子力学)
% license Xiao
% type Tutor

\begin{issues}
\issueDraft
\end{issues}

量子力学中, 偶极子近似假设整个空间中,外电场 $\mathcal{E}$ 和矢量势 $\bvec A$ 处处相同,仅随时间变化。

该近似常用于原子分子在激光中的变化。 当它们的尺寸远小于激光波长时,效果较好。

根据 $\bvec B = \curl \bvec A$, 外磁场处处为零。

\addTODO{这是否法拉第电磁感应矛盾? 毕竟 $\pdv*{\mathcal E}{t}\ne 0$。 不矛盾, 论证它可以忽略。}
