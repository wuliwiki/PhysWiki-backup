% 黎曼度量与伪黎曼度量
% keys 度量|metric|流形|微分几何|相对论|闵可夫斯基时空|relativity|minkowski|riemann|riemannian metric|洛伦兹度量|lorentz
% license Xiao
% type Tutor
\pentry{内积\nref{nod_InerPd},切丛\nref{nod_TanBun},余切丛(未完成),张量场\nref{nod_TenMan}}{nod_be3a}

本节采用爱因斯坦求和约定。

黎曼度量(Riemannian metric)或伪黎曼度量(pseudo-Riemannian metric)是黎曼几何或伪黎曼几何所要求的基本结构。赋予黎曼度量/伪黎曼度量的微分流形被称为黎曼流形/伪黎曼流形,它们既是几何学研究的对象,也是广义相对论得以展开的舞台。

这里将一直设 $M$ 是一个 $n$ 维实微分流形。

使用其它表述的黎曼度量另见\textbf{黎曼联络}的\autoref{sub_RieCon_1}。

\subsection{黎曼度量}
\begin{definition}{黎曼度量\footnote{更适合初学者的定义见\enref{黎曼联络}{RieCon}文章。}}
$M$ 上的一个黎曼度量 $g$ 是指丛 $T^*(M)\otimes T^*(M)$ 的一个对称的正定截面。等价地,给出黎曼度量 $g$,就相当于在每一点 $p$ 的切空间 $T_pM$ 上指定一个内积 $g_p$。指定了黎曼度量的微分流形称为黎曼流形。有时也会把黎曼度量记为 $\langle\cdot,\cdot\rangle_p$.
\end{definition}
给定局部坐标系 $\{x^i\}$ 后,黎曼度量 $g$ 的局部表达式是
$$
g_{ij}(x)dx^i\otimes dx^j~,
$$
这里 $g_{ij}(x)=g_{ji}(x)$,而且对于任何向量 $(X^i)_{i=1}^n\neq0$ 都有
$$
g_{ij}X^iX^j>0~.
$$

有了黎曼度量,就可以谈论诸如长度/面积/体积之类的度量性质了。例如,设 $\gamma:[a,b]\to M$ 是黎曼流形 $(M,g)$ 上的一条道路,则定义其长度为
$$
L[\gamma]:=\int_{a}^b \sqrt{g_{\gamma(t)}(\gamma'(t),\gamma'(t))}dt~.
$$
如果用局部坐标系给出曲线的局部参数方程 $x(t)=(x^i(t))_{i=1}^n$,则
$$
L[\gamma]:=\int_{a}^b \sqrt{g_{ij}(x(t))(\dot x^i(t),\dot x^j(t))}dt~.
$$
容易验证:曲线的长度不依赖于其参数化的方式。
\subsubsection{正切-余切同构}
通常用$(g,M)$表示赋予了黎曼度量的流形。对于任意黎曼流形,我们总可以定义一个丛映射$\widetilde g:TM\rightarrow T^*M$,对任意两个切向量场$X,Y$满足
\begin{equation}
g(X,Y)=\widetilde g(X)Y~.
\end{equation}

\subsection{伪黎曼度量}
\pentry{伴随映射\nref{nod_AdjMap}}{nod_2d17}
相对于黎曼度量,伪黎曼度量把对截面的正定性要求放松至只要满足非退化性即可。伪黎曼度量是一个\textbf{光滑、对称、非退化的二阶张量场}。“非退化”与线性代数的定义相同,即有下述等价条件:
\begin{itemize}
\item 在同构的欧几里得空间里,伪黎曼度量的矩阵表示是可逆的。
\item 在同构的欧几里得空间里,伪黎曼度量$g(\bvec x,\bvec y)$对于任意$\bvec y\in V$成立$\Longleftrightarrow\,\bvec x=0$
\item $g$可以诱导同构映射$\sigma_g:V\rightarrow V^*$,满足$\sigma_g(\bvec x)\bvec y=g(\bvec x,\bvec y)$。
\end{itemize}
\begin{example}{洛伦兹度量}

\end{example}