% 斯坦尼斯瓦夫·乌拉姆(综述)
% license CCBYSA3
% type Wiki

本文根据 CC-BY-SA 协议转载翻译自维基百科\href{https://en.wikipedia.org/wiki/Stanis\%C5\%82aw_Ulam}{相关文章}。
\begin{figure}[ht]
\centering
\includegraphics[width=6cm]{./figures/15377c1b183c5f57.png}
\caption{乌拉姆在洛斯阿拉莫斯} \label{fig_Ulam_1}
\end{figure}
斯坦尼斯瓦夫·马尔钦·乌拉姆(波兰语:[sta'ɲiswaf 'mart͡ɕin 'ulam];1909年4月13日 – 1984年5月13日)是波兰数学家、核物理学家和计算机科学家。他参与了曼哈顿计划,提出了特勒–乌拉姆热核武器设计,发现了细胞自动机的概念,发明了蒙特卡罗计算方法,并提出了核脉冲推进技术。在纯数学和应用数学领域,他证明了多个定理并提出了若干猜想。

乌拉姆出生于奥匈帝国利沃夫的一个富裕的波兰犹太家庭;他在利沃夫工艺大学学习数学,并于1933年在卡齐米日·库拉特科夫斯基(Kazimierz Kuratowski)和弗沃季米日·斯托热克(Włodzimierz Stożek)的指导下获得博士学位。[1] 1935年,乌拉姆在华沙遇到了约翰·冯·诺依曼,后者邀请他到新泽西州普林斯顿的高等研究院待几个月。从1936年到1939年,他每年夏天都回波兰,学年则在马萨诸塞州剑桥的哈佛大学度过,在那里他致力于建立关于遍历理论的重要成果。1939年8月20日,他和17岁的弟弟亚当·乌拉姆一起最后一次乘船前往美国。1940年,他成为威斯康星大学麦迪逊分校的助理教授,并于1941年成为美国公民。

1943年10月,乌拉姆收到了汉斯·贝特的邀请,加入位于新墨西哥州洛斯阿拉莫斯的曼哈顿计划秘密实验室。在那里,他负责进行流体动力学计算,以预测爆炸透镜在内爆型武器中的行为。他被分配到爱德华·泰勒的团队,在泰勒和恩里科·费米的指导下,参与了泰勒的“超级”炸弹项目。战后,他离开洛斯阿拉莫斯,成为南加州大学的副教授,但在1946年回到洛斯阿拉莫斯,继续从事热核武器的研究。在一群女性“计算员”的帮助下,他发现泰勒的“超级”设计不可行。1951年1月,乌拉姆和泰勒共同提出了泰勒–乌拉姆设计,这一设计成为所有热核武器的基础。

乌拉姆考虑了火箭核推进的问题,这一问题由“罗孚计划”(Project Rover)进行研究。他提出了一种替代“罗孚计划”核热火箭的方法,即利用小规模的核爆炸进行推进,这一方案后来成为了“猎鹰计划”(Project Orion)。与费米、约翰·帕斯塔(John Pasta)和玛丽·青果(Mary Tsingou)一起,乌拉姆研究了著名的费米–帕斯塔–乌拉姆–青果问题(Fermi–Pasta–Ulam–Tsingou problem),这一问题成为了非线性科学领域的启发来源。他可能最为人知的是意识到,电子计算机使得将统计方法应用于没有已知解的函数变得可行。随着计算机的发展,蒙特卡罗方法(Monte Carlo method)已成为解决许多问题的常见且标准的方法。
\subsection{波兰}  
乌拉姆于1909年4月13日出生在加利西亚的莱姆堡(Lemberg)。当时,加利西亚属于奥匈帝国的加利西亚和洛多梅里亚王国,波兰人称其为奥地利分治区。1918年,莱姆堡成为新恢复的波兰第二共和国的一部分,并重新取回了其波兰名字——利沃夫(Lwów)。

乌拉姆家族是一个富裕的波兰犹太家庭,从事银行业、工业和其他专业工作。乌拉姆的直系家庭“生活富足,但并不算非常富有”。他的父亲,约瑟夫·乌拉姆(Józef Ulam),出生在利沃夫,是一名律师;母亲安娜(Anna, née Auerbach)出生于斯特里(Stryj)。他的叔叔米哈乌·乌拉姆(Michał Ulam)是一名建筑师、建筑承包商和木材工业家。从1916年到1918年,约瑟夫的家庭曾暂时居住在维也纳。返回后,利沃夫成为波兰–乌克兰战争的中心,期间该市遭遇了乌克兰的围困。
\begin{figure}[ht]
\centering
\includegraphics[width=6cm]{./figures/d11840c23803e51e.png}
\caption{位于乌克兰利沃夫的苏格兰咖啡馆大楼现在是Szkocka餐厅和酒吧的所在地(该餐厅以原苏格兰咖啡馆命名)。} \label{fig_Ulam_2}
\end{figure}
1919年,乌拉姆进入了利沃夫第七中学,并于1927年毕业。[10] 随后,他在利沃夫理工学院学习数学。在卡兹米日·库拉托夫斯基的指导下,他于1932年获得文学硕士学位,并于1933年获得科学博士学位。[9][11] 在1929年,年仅20岁的乌拉姆在《数学基础》杂志上发表了他的第一篇论文《关于集合的函数》。[11] 从1931年到1935年,他前往并在维尔纽斯(今立陶宛首都)、维也纳、苏黎世、巴黎和英国剑桥学习,在那里他结识了G·H·哈迪和苏布拉马尼扬·钱德拉塞卡。[12]

乌拉姆与斯坦尼斯瓦夫·马祖尔、马克·卡茨、弗沃季米日·斯托热克、卡兹米日·库拉托夫斯基等人一起,是利沃夫数学学派的成员。该学派的创始人是胡戈·施泰因豪斯和斯特凡·巴纳赫,他们是雅努什·卡兹米日大学的教授。这些数学家常常在苏格兰咖啡馆聚会,讨论他们的问题,这些问题被收录在《苏格兰书》中,这是由巴纳赫的妻子提供的一本厚重的笔记本。乌拉姆是这本书的主要贡献者之一。在1935年至1941年间记录的193个问题中,他作为唯一作者贡献了40个问题,又与巴纳赫和马祖尔一起合作贡献了11个问题,并与其他人共同贡献了15个问题。1957年,他从施泰因豪斯那里收到了这本幸存下来的书,并将其翻译成了英语。[13] 1981年,乌拉姆的朋友R·丹尼尔·莫尔丁发布了扩展版和注释版。[14]
\subsection{移居美国}  
1935年,乌拉姆在华沙遇见的约翰·冯·诺依曼邀请他前往新泽西州普林斯顿的高等研究院待几个月。那年12月,乌拉姆启程前往美国。在普林斯顿,他参加了讲座和研讨会,听到了奥斯瓦尔德·维布伦、詹姆斯·亚历山大和阿尔伯特·爱因斯坦的演讲。在冯·诺依曼家的一次茶话会上,他遇到了G·D·伯克霍夫,伯克霍夫建议他申请哈佛大学学者协会的职位。[9] 根据伯克霍夫的建议,乌拉姆从1936年到1939年夏季在波兰度过,学年则在马萨诸塞州剑桥的哈佛大学度过,并与约翰·C·奥克斯托比合作,研究了遍历理论的相关成果。这些成果于1941年发表在《数学年刊》上。[10][15] 1938年,乌拉姆的母亲安娜·汉娜·乌拉姆(原名奥尔巴赫)因癌症去世。

1939年8月20日,在格丁尼亚,约瑟夫·乌拉姆与他的兄弟希蒙一起,将他的两个儿子,斯坦尼斯瓦夫和17岁的亚当,送上了一艘前往美国的船。[9] 十一天后,德国人入侵了波兰。两个月内,德国完成了对西波兰的占领,苏联则入侵并占领了东波兰。两年内,约瑟夫·乌拉姆和他的家人,包括斯坦尼斯瓦夫的妹妹斯特凡尼亚·乌拉姆,成为了大屠杀的受害者,胡戈·斯坦豪斯则藏匿起来,卡齐米日·库拉特科夫斯基在华沙的地下大学讲课,弗沃季米日·斯托日克和他的两个儿子在利沃夫教授大屠杀中丧生,最后一个问题被记录在《苏格兰书》里。斯特凡·巴纳赫通过在鲁道夫·韦格尔的伤寒研究所喂食虱子,幸存于纳粹占领期间。1963年,亚当·乌拉姆(他在哈佛大学成为了著名的克里姆林学家)收到了乔治·沃尔斯基的信件,[17] 沃尔斯基曾在波兰军队叛逃后藏匿在约瑟夫·乌拉姆的家中。这段回忆记录了1939年末利沃夫混乱的景象。[18] 晚年,乌拉姆自称“是一个不可知论者。有时我深思那些对我而言不可见的力量。当我几乎接近上帝的概念时,我立刻因这个世界的恐怖而感到疏离,他似乎容忍这些恐怖”。[19]

1940年,在比尔科夫的推荐下,乌拉姆成为威斯康星大学麦迪逊分校的助理教授。在这里,他于1941年成为美国公民。[9] 同年,他与弗朗索瓦丝·阿隆结婚。[10] 她曾是霍利奥克学院的法国交换生,他们在剑桥相识。他们有一个女儿,克莱尔。在麦迪逊,乌拉姆结识了他的朋友和同事C·J·埃弗雷特,他们合作发表了多篇论文。[20]
\subsection{曼哈顿计划}
\begin{figure}[ht]
\centering
\includegraphics[width=6cm]{./figures/d0b3ffc64d56e78d.png}
\caption{乌拉姆在洛斯阿拉莫斯国家实验室的身份证照片} \label{fig_Ulam_3}
\end{figure}
1943年初,乌拉姆请求冯·诺依曼为他找到一份战争工作。10月,他收到了一个邀请函,邀请他加入一个位于新墨西哥州圣菲附近的未透露名称的项目。[9] 这封信是由汉斯·贝特签署的,他被罗伯特·奥本海默任命为洛斯阿拉莫斯国家实验室的理论部门负责人。[21] 乌拉姆对这个地方一无所知,于是他借了一本新墨西哥州的指南。在借书卡上,他发现了他在威斯康星大学的同事们——琼·欣顿、大卫·弗里施和约瑟夫·麦基本的名字,他们都神秘地消失了。[9] 这就是乌拉姆与曼哈顿计划的第一次接触,曼哈顿计划是美国在二战期间为了制造原子弹而开展的工作。[22]
\subsubsection{湍流塌缩的水动力学计算}  
乌拉姆于1944年2月抵达洛斯阿拉莫斯后几周,项目遭遇了一场危机。1944年4月,埃米里奥·塞格雷发现,反应堆中制造的钚无法用于像“瘦子”那样的枪式钚武器,该武器与同时开发的铀武器“原子男孩”并行,而“原子男孩”最后被投放到广岛。这一问题威胁到对汉福德现场新建反应堆的大量投资,并使得缓慢的铀同位素分离成为准备适用于武器的裂变材料的唯一途径。为应对这一问题,奥本海默在8月实施了一次大规模的实验室重组,将重点放在开发内爆型武器上,并任命乔治·基斯季亚科夫斯基为内爆部门负责人。他是哈佛大学的教授,也是精确使用爆炸物的专家。[23]

内爆的基本概念是使用化学炸药将一块裂变材料压缩到临界质量,在那里中子倍增导致核链式反应,释放大量能量。赛斯·内德迈尔曾研究过圆柱形内爆配置,但约翰·冯·诺依曼具有使用装甲穿透弹药中形状炸药的经验,他是球形内爆驱动爆炸透镜的强烈倡导者。他意识到内爆压缩钚的对称性和速度是关键问题,[23]并邀请乌拉姆协助设计能够提供几乎球形内爆的透镜配置。在内爆中,由于巨大的压力和高温,固体材料的行为更像流体。这意味着需要进行水动力学计算来预测并最小化可能破坏核爆炸的非对称性。对于这些计算,乌拉姆说:

水动力学问题表述简单,但计算起来非常困难——不仅是细节上的,甚至是数量级上的。在这次讨论中,我强调了纯粹的实用主义,并认为必须通过简单的蛮力方法获得问题的启发式概览,而不是依赖大量的数值计算。[9]

尽管当时的设施非常原始,乌拉姆和冯·诺依曼还是进行了数值计算,最终得出了一个令人满意的设计。这促使他们在洛斯阿拉莫斯推动了强大的计算能力的建设,这一工作始于战争期间,[24]并延续到了冷战时期,至今仍然存在。[25] 奥托·弗里施回忆道,乌拉姆是“一位才华横溢的波兰拓扑学家,拥有迷人的法国妻子。他立刻告诉我,他是一个纯粹的数学家,已经堕落到自己的最新论文中居然出现了带小数点的数字!”[26]
\subsubsection{分支和乘法过程的统计学 } 
即使是在链式反应中,中子倍增的固有统计波动,也会对聚爆速度和对称性产生影响。1944年11月,David Hawkins 和 Ulam 在一篇题为《乘法过程理论》的报告中讨论了这个问题[27]。这篇报告利用了概率生成函数,也是关于分支和乘法过程统计学的早期文献之一[28]。1948年,Ulam 和 Everett 扩展了这一研究的范围[29]。

在曼哈顿计划初期,恩里科·费米将注意力集中在利用反应堆生产钚上。1944年9月,他到达洛斯阿拉莫斯,不久后他成功恢复了第一座汉福德反应堆的正常运行,该反应堆曾因氙同位素污染而停运。[30] 在费米到达后不久,泰勒的‘超级’原子弹小组(乌拉姆是其中一员)被调到由费米领导的新部门。[31] 费米和乌拉姆建立了密切的合作关系,战后这一关系取得了丰硕的成果。[32]
\subsection{战后洛斯阿拉莫斯}  
1945年9月,乌拉姆离开洛斯阿拉莫斯,成为洛杉矶南加州大学的副教授。1946年1月,他患上了急性脑炎,生命垂危,但通过紧急脑部手术得以缓解。在恢复期间,许多朋友前来探望,其中包括来自洛斯阿拉莫斯的尼古拉斯·梅特罗波利斯和著名数学家保罗·厄尔德什[33],厄尔德什说道:“斯坦,你和以前一样。”[9] 这让乌拉姆感到宽慰,因为他当时担心自己的智力状态,因为在危机期间他失去了说话的能力。另一位朋友,吉安-卡洛·罗塔,在1987年的一篇文章中指出,这次脑炎发作改变了乌拉姆的性格:此后,他从严格的纯数学转向更具推测性的数学猜想,关注数学在物理学和生物学中的应用;罗塔还引用了乌拉姆的前合作者保罗·斯坦的说法,称乌拉姆此后在穿着上变得不那么讲究,约翰·奥克斯托比则指出,乌拉姆在患脑炎之前可以连续几个小时进行计算,而在与罗塔合作时,他甚至不愿解一个二次方程。[34] 这一说法未被乌拉姆的妻子弗朗索瓦丝·阿龙·乌拉姆接受。[35]

到1946年4月底,乌拉姆已康复到足以参加在洛斯阿拉莫斯举行的一个秘密会议,讨论热核武器。与会者包括乌拉姆、冯·诺依曼、梅特罗波利斯、泰勒、斯坦·弗兰克尔以及其他人。在曼哈顿计划期间,泰勒的努力一直集中在基于核聚变的‘超级’武器的开发上,而不是实用裂变炸弹的开发。在广泛讨论之后,与会者达成共识,认为他的想法值得进一步探索。几周后,乌拉姆收到了梅特罗波利斯和新任理论部主任罗伯特·D·里希特迈尔的邀请,提供了一个薪资更高的职位,乌拉姆一家人便回到了洛斯阿拉莫斯。[36]
\subsubsection{蒙特卡洛方法}
\begin{figure}[ht]
\centering
\includegraphics[width=6cm]{./figures/feeb318326d46f16.png}
\caption{斯坦·乌拉姆拿着FERMIAC} \label{fig_Ulam_4}
\end{figure}
在战争后期,在冯·诺依曼的支持下,弗兰克尔和梅特罗波利斯开始对第一台通用电子计算机——位于马里兰州阿伯丁试验场的ENIAC进行计算。乌拉姆在返回洛斯阿拉莫斯后不久,参与了对这些计算结果的审查。[37] 早些时候,在从手术中恢复期间,乌拉姆在玩接龙时曾想过通过玩几百局游戏来统计估算成功结果的概率。[38] 有了ENIAC,他意识到计算机的可用性使得这种统计方法变得非常实际。约翰·冯·诺依曼立即看到了这一洞察的重要性。1947年3月,他提出了一种统计方法来解决中子在可裂变物质中的扩散问题。[39] 由于乌拉姆经常提到他的叔叔米哈乌·乌拉姆,“他总是得去蒙特卡洛”去赌博,梅特罗波利斯便将这种统计方法称为“蒙特卡洛方法”。[37] 梅特罗波利斯和乌拉姆于1949年发布了关于蒙特卡洛方法的第一篇非机密论文。[40]

费米得知乌拉姆的突破后,设计了一种模拟计算机,称为蒙特卡洛电车,后来被称为FERMIAC。这种装置对中子的随机扩散进行了机械模拟。随着计算机在速度和可编程性方面的改进,这些方法变得更加实用。特别是,许多在现代大规模并行超级计算机上进行的蒙特卡洛计算是非常适合并行处理的应用,其结果可以非常精确。[25]
\subsubsection{泰勒–乌拉姆设计}  
1949年8月29日,苏联进行了首次裂变炸弹测试,即RDS-1。在拉夫连季·贝利亚的监督下,该炸弹旨在复制美国的努力,这款武器几乎与‘胖子’炸弹相同,因为它的设计基于间谍克劳斯·弗克斯、西奥多·霍尔和大卫·格林格拉斯提供的信息。作为回应,1950年1月31日,哈里·S·杜鲁门总统宣布启动一个紧急计划,开发聚变炸弹。[41]

为了倡导积极的发展计划,厄内斯特·劳伦斯和路易斯·阿尔瓦雷斯来到洛斯阿拉莫斯,与实验室主任诺里斯·布拉德伯里、乔治·伽莫夫、爱德华·泰勒和乌拉姆进行了磋商。很快,泰勒、伽莫夫和乌拉姆成为布拉德伯里任命的一个短命委员会的成员,研究该问题,泰勒担任委员会主席。[9] 当时,关于使用裂变武器引发聚变反应的研究自1942年以来一直在进行,但设计仍然基本沿用了泰勒最初提出的方案。他的概念是将氚和/或氘与裂变炸弹放置在接近的位置,希望炸弹爆炸时释放的热量和强烈的中子通量能够点燃自持的聚变反应。氢的这些同位素的反应之所以受到关注,是因为它们在聚变时释放的单位质量燃料的能量要远大于重核裂变释放的能量。[42]
\begin{figure}[ht]
\centering
\includegraphics[width=6cm]{./figures/5cdc6db6f9970615.png}
\caption{艾薇·迈克(Ivy Mike),泰勒–乌拉姆设计的首次完整测试(分级聚变炸弹),于1952年11月1日爆炸,威力为1040万吨当量。} \label{fig_Ulam_5}
\end{figure}
由于基于泰勒概念的计算结果令人沮丧,许多科学家认为这无法导致成功的武器,而另一些人则出于道德和经济原因反对继续进行。因此,曼哈顿计划的几位高级人物反对发展这一方案,包括贝特和奥本海默。[43] 为了澄清情况,乌拉姆和冯·诺依曼决定进行新的计算,确定泰勒的方法是否可行。为了进行这些研究,冯·诺依曼决定使用电子计算机:位于阿伯丁的ENIAC,普林斯顿的新计算机MANIAC,以及在洛斯阿拉莫斯建设中的它的双胞胎。乌拉姆邀请埃弗雷特采用一种完全不同的方法,这种方法以物理直觉为指导。弗朗索瓦丝·乌拉姆是[44] 一群“计算员”之一,她们在机械计算器上进行艰巨而广泛的热核场景计算,并通过埃弗雷特的计算尺进行补充和确认。乌拉姆和费米进一步合作分析这些场景。结果表明,在可行的配置中,热核反应不会点燃,如果点燃,也不会是自持的。乌拉姆利用他在组合数学方面的专长,分析了氘的链式反应,这比铀和钚的反应复杂得多,他得出结论,考虑到泰勒所考虑的(低)密度,不会发生自持链式反应。[45] 1950年底,这些结论得到了冯·诺依曼的结果确认。[35][46]

1951年1月,乌拉姆提出了另一个想法:利用核爆炸的机械冲击来压缩聚变燃料。在妻子的建议下,[35] 乌拉姆在告诉泰勒之前,先与布拉德伯里和马克讨论了这一想法。[47] 几乎立刻,泰勒看到了这个想法的优点,但指出裂变炸弹发出的软X射线会比机械冲击更强烈地压缩热核燃料,并建议增强这种效果的方法。1951年3月9日,泰勒和乌拉姆提交了一份联合报告,描述了这些创新。[48] 几周后,泰勒建议将一个裂变杆或圆柱体放置在聚变燃料的中心。这个“火花塞”的引爆[49] 将有助于启动并增强聚变反应。基于这些想法的设计,被称为分级辐射内爆,已成为制造热核武器的标准方法。它通常被称为“泰勒–乌拉姆设计”。[50]
\begin{figure}[ht]
\centering
\includegraphics[width=6cm]{./figures/5737eeb6e26d43c3.png}
\caption{‘香肠’装置是迈克核试验中的一部分(威力为1040万吨),该试验在恩纽埃塔环礁进行。该试验是‘常春藤行动’的一部分。‘香肠’是第一次真正测试的氢弹,意味着它是基于泰勒–乌拉姆的分级辐射内爆原理建造的第一种热核装置。} \label{fig_Ulam_6}
\end{figure}
1951年9月,在与布拉德伯里和其他科学家发生一系列分歧后,泰勒辞去了洛斯阿拉莫斯的职务,返回芝加哥大学。[51] 大约在同一时间,乌拉姆请假作为哈佛大学的访问教授,讲授一个学期。[52] 尽管泰勒和乌拉姆共同提交了他们的设计报告[48],并共同申请了专利[22],他们很快就卷入了一场关于谁应获得功劳的争论。[47] 战后,贝特返回康奈尔大学,但他作为顾问深度参与了热核武器的发展。1954年,他写了一篇关于氢弹历史的文章[53],在其中他表达了自己的观点,认为两人都为这一突破做出了非常重要的贡献。其他参与其中的人,包括马克和费米,也持有这种平衡的观点,但泰勒坚持试图淡化乌拉姆的作用。[54] “氢弹制造出来后,”贝特回忆道,“记者开始称泰勒为氢弹之父。为了历史的准确性,我认为更精确的说法是乌拉姆是父亲,因为他提供了种子,而泰勒是母亲,因为他与孩子始终在一起。至于我,我想我是接生婆。”[55]

在基本的聚变反应得到确认,并且有了可行的设计后,洛斯阿拉莫斯测试热核装置再没有任何障碍。1952年11月1日,首次热核爆炸发生,当时‘艾薇·迈克’在恩纽埃塔环礁的美国太平洋试验场引爆。这个装置使用液态氘作为聚变燃料,体积庞大,完全无法作为武器使用。然而,它的成功验证了泰勒–乌拉姆设计,并刺激了实用武器的密集开发。[52]
\subsubsection{“费米–帕斯塔–乌拉姆–青狗问题 }
当乌拉姆返回洛斯阿拉莫斯时,他将注意力从武器设计转向了利用计算机研究物理和数学中的问题。与约翰·帕斯塔合作,后者帮助梅特罗波利斯于1952年3月使MANIAC投入使用,他在一份名为《高速度计算机上的数学物理问题启发式研究》的报告中探讨了这些想法,该报告于1953年6月9日提交。报告讨论了几种无法在传统解析方法框架内解决的问题:流体的膨胀、引力系统中的旋转运动、磁力线和流体动力学不稳定性。[56]

不久,帕斯塔和乌拉姆在MANIAC上积累了电子计算的经验,到了这时,恩里科·费米已经习惯了在芝加哥大学度过学年,在洛斯阿拉莫斯度过夏季的日程安排。在这些夏季访问期间,帕斯塔、乌拉姆和MANIAC小组的程序员玛丽·青狗与他一起研究了经典问题的一个变种:一串由弹簧连接的质量块,这些弹簧施加的力与质量块从平衡位置的位移成线性比例。[57] 费米提议在这个力上添加一个非线性成分,可以选择使其与位移的平方或立方成正比,或者与一个更复杂的“断线性”函数成正比。这个附加项就是费米–帕斯塔–乌拉姆–青狗问题的关键元素,该问题通常被缩写为FPUT。[58][59]

一个经典的弹簧系统可以通过振动模式来描述,这些模式类似于拉紧的小提琴弦上出现的谐波。如果系统开始时处于某个特定模式,那么其他模式中的振动就不会发展。加入非线性成分后,费米预期一个模式中的能量会逐渐转移到其他模式,最终在所有模式之间均匀分配。最初,当系统被初始化并将所有能量集中在最低模式时,确实出现了这样的现象,但很久以后,几乎所有的能量又周期性地重新出现在最低模式中。[59] 这种行为与预期的能量均分不同。直到1965年,Kruskal和Zabusky通过适当的数学变换证明,该系统可以用Korteweg–de Vries方程来描述,而该方程是具有孤立子解的非线性偏微分方程的原型。这意味着FPUT行为可以通过孤立子来理解。[60]
\subsubsection{核推进}
\begin{figure}[ht]
\centering
\includegraphics[width=6cm]{./figures/f2c2160822228bc6.png}
\caption{艺术家对NASA参考设计的构想,展示了由核推进提供动力的‘猎户座’项目航天器。} \label{fig_Ulam_7}
\end{figure}
从1955年开始,乌拉姆和弗雷德里克·赖尼斯考虑了飞机和火箭的核推进。[61] 这是一个具有吸引力的可能性,因为核能每单位质量的燃料比化学燃料高出百万倍。从1955年到1972年,他们的想法在‘罗孚计划’期间得到了追求,该计划探索了使用核反应堆为火箭提供动力。[62] 1958年1月22日,在一次关于‘通过核能进行外层空间推进’的国会委员会听证会上,参议员约翰·O·帕斯托尔提问时,乌拉姆回答说:“人类未来的发展在某种程度上与走出地球是不可避免的。”[63]

乌拉姆和C·J·埃弗雷特还提出,与‘罗孚计划’的连续加热火箭排气不同,他们建议利用小型核爆炸进行推进。[64] ‘猎户座’计划就是对这一想法的研究。该计划始于1958年,并于1965年结束,原因是1963年的《部分核试验禁令条约》禁止了大气层内和外层空间中的核武器试验。[65] 该项目的工作由物理学家弗里曼·戴森领导,他在文章《一个项目的终结》中评论了结束‘猎户座’计划的决定。[66]

1957年,布拉德伯里任命乌拉姆和约翰·H·曼利为实验室主任的研究顾问。这些新设立的职位与各个部门负责人处于相同的行政级别,乌拉姆一直担任这一职位,直到他从洛斯阿拉莫斯退休。在这个职位上,他能够影响并指导多个部门的项目,包括理论、物理、化学、冶金、武器、健康、罗孚计划等。[62]

除了这些活动,乌拉姆还继续发表技术报告和研究论文。其中一篇介绍了费米–乌拉姆模型,这是费米关于宇宙射线加速理论的扩展。[67] 另一篇与保罗·斯坦和玛丽·辛古合作,题为《二次变换》,是对混沌理论的早期研究,并被认为是首次出版使用‘混沌行为’这一术语。[68][69]
\subsection{重返学术界}
\begin{figure}[ht]
\centering
\includegraphics[width=6cm]{./figures/8137bc270dc71a90.png}
\caption{当正整数沿乌拉姆螺旋排列时,质数(用点表示)往往沿对角线聚集。} \label{fig_Ulam_8}
\end{figure}
在洛斯阿拉莫斯的岁月里,乌拉姆曾于1951年至1952年担任哈佛大学的访问教授,1956年至1957年担任麻省理工学院的访问教授,1963年担任加利福尼亚大学圣地亚哥分校的访问教授,1961年至1962年以及1965年至1967年担任科罗拉多大学博尔德分校的访问教授。1967年,乌拉姆的最后一个职位成为永久职务,他被任命为科罗拉多大学数学系的教授兼系主任。他在圣菲保留了一处住所,这使得他可以方便地在夏天作为顾问回到洛斯阿拉莫斯。[70] 他是美国艺术与科学学院、美国国家科学院和美国哲学学会的选举会员。[71][72][73]

在科罗拉多州,他与朋友们加莫夫、里希特迈尔和霍金斯重新聚首,乌拉姆的研究兴趣转向了生物学。1968年,科罗拉多大学医学院认识到这一方向,任命乌拉姆为生物数学教授,并一直担任该职位直到去世。与洛斯阿拉莫斯的同事罗伯特·施兰特一起,他发表了一篇报告《一些关于进化过程速率的数值建模的初步尝试》,将他早期关于分支过程的理论应用于进化问题。[74] 另一篇报告与威廉·贝耶、坦普尔·F·史密斯和M·L·斯坦合作,题为《生物学中的度量》,介绍了关于数值分类学和进化距离的新理念。[75]

1975年,乌拉姆从科罗拉多大学退休后,开始在佛罗里达大学度过冬季学期,他在那里担任研究生研究教授。1976年,他被波兰流亡政府授予波兰恢复勋章的指挥十字星勋章。[76] 除了在1982至1983年间在加利福尼亚大学戴维斯分校和1980至1984年间在洛克菲勒大学的学术休假外,[70] 这种夏季待在科罗拉多和洛斯阿拉莫斯,冬季待在佛罗里达的生活模式一直持续,直到乌拉姆于1984年5月13日因心脏病突发去世,享年73岁。[3] 保罗·厄尔德什提到:“他在没有恐惧和痛苦的情况下突然因心脏衰竭去世,在他仍能证明和推测时。”[33] 1987年,弗朗索瓦兹·乌拉姆将他的档案存放在费城的美国哲学会图书馆。[77] 她继续住在圣菲,直到2011年去世,享年93岁。弗朗索瓦兹和她的丈夫一同被安葬在巴黎的蒙帕纳斯公墓,和她的家人一起。[78][79]
\subsection{经济学的挑战}  
直到第二次世界大战结束,阿尔弗雷德·马歇尔及其门徒主导着经济理论。随着冷战的到来,理论发生了变化,强调市场经济是优越的,也是唯一明智的方式。在保罗·塞缪尔森的《经济学:导论分析》(1948年)中,亚当·斯密的“看不见的手”只是一个脚注。在后来的版本中,它成为了核心主题。正如塞缪尔森回忆的那样,这一切都受到了斯坦尼斯瓦夫·乌拉姆的挑战:

“几年前……我和数学家斯坦尼斯瓦夫·乌拉姆一起在哈佛的学者协会。乌拉姆,后来成为蒙特卡罗方法的创始人之一和氢弹的共同发现者……他常常拿我开玩笑说,‘给我举出一条社会科学中既真实又非平凡的命题。’这是我总是失败的测试。但现在,三十年后……我想到了一个合适的答案:里卡多的比较优势理论……它在逻辑上真实,这一点在数学家面前不需要争辩;它并不平凡,这一点由成千上万重要且聪明的人证明,他们要么无法自己理解这一理论,要么在被解释之后仍然不相信它。”[80][81]
\subsection{‘奇点’一词的创造}  
乌拉姆在1958年报告中提到,他曾与约翰·冯·诺依曼讨论过‘技术进步的加速和人类生活的变化,给人一种在种族历史中接近某种本质奇点的印象,越过这个奇点之后,人类事务就无法像我们所知道的那样继续下去’。[82]这也是‘奇点’一词的首次记录使用。
\subsection{影响与遗产}  
乌拉姆参与了氢弹的研发,作为洛斯阿拉莫斯实验室核项目的一部分。从他作为学生在1929年发表第一篇论文开始,直到去世,乌拉姆一直在不断撰写数学方面的论文。乌拉姆的出版物列表包括超过150篇论文。[10] 他的论文涉及的主题包括:集合论(包括可测基数和抽象测度)、拓扑学、泛函分析、变换理论、遍历理论、群论、投影代数、数论、组合数学和图论。[83]

这些工作的显著成果包括:
\begin{itemize}
\item Borsuk–Ulam定理  
\item Mazur–Ulam定理  
\item Kuratowski–Ulam定理  
\item Hyers–Ulam–Rassias稳定性  
\item 幸运数  
\item 乌拉姆螺旋  
\item 乌拉姆猜想(数论中的)  
\item 乌拉姆猜想(图论中的)  
\item 乌拉姆的打包猜想  
\item 乌拉姆游戏  
\item 乌拉姆矩阵  
\item 乌拉姆数
\end{itemize}
乌拉姆在热核武器的发展中发挥了关键作用。根据弗朗索瓦兹·乌拉姆的说法:“斯坦会安慰我说,除了意外之外,氢弹使得核战争变得不可能。”[35] 1980年,乌拉姆和他的妻子出现在电视纪录片《三位一体之后的日子》中。[84]
\begin{figure}[ht]
\centering
\includegraphics[width=6cm]{./figures/eff3584ca6530677.png}
\caption{一段展示幸运数筛法的动画。红色数字是幸运数。} \label{fig_Ulam_9}
\end{figure}
蒙特卡洛方法已成为一种无处不在的标准计算方法,广泛应用于大量科学问题。[85] 除了物理学和数学问题外,该方法还被应用于金融、社会科学、[86] 环境风险评估、[87] 语言学、[88] 放射治疗、[89] 以及体育领域。[90]

费米–帕斯塔–乌拉姆–青谷问题不仅被誉为“实验数学的诞生”,[59] 还激发了非线性科学这一广泛领域。在他的利连费尔德奖讲座中,David K. Campbell 提到了这一关系,并描述了 FPUT 如何催生了混沌、孤立子和动力学系统的思想。[91] 1980年,洛斯阿拉莫斯实验室的实验室主任 Donald Kerr,在乌拉姆和 Mark Kac 的大力支持下,[92] 创立了非线性研究中心 (CNLS)。[93] 1985年,CNLS 启动了 Stanislaw M. Ulam 卓越学者计划,提供年度奖项,支持一位知名科学家在洛斯阿拉莫斯进行为期一年的研究。[94]

FPUT 原始论文的五十周年纪念是 2005 年 3 月《混沌》期刊的主题,[95] 也是 CNLS 第25届国际年会的主题。[96] 密西西比大学和佛罗里达大学支持了《乌拉姆季刊》,[97] 该季刊自 1992 年至 1996 年期间活跃,是最早的在线数学期刊之一。[98] 自 1998 年以来,佛罗里达大学数学系每年举办乌拉姆学术讲座,[99] 并且在 2009 年 3 月举行了乌拉姆百年纪念大会。[100]

乌拉姆在分子生物学中关于非欧几里得距离度量的研究,对序列分析做出了重大贡献[101],他的理论生物学贡献被认为是细胞自动机理论、种群生物学、模式识别和生物计量学发展的分水岭(然而,大卫·桑科夫(David Sankoff)挑战了沃尔特的结论,认为乌拉姆对序列比对方法的早期发展影响有限[102])。同事们指出,他的一些最伟大的贡献在于清晰地识别待解决的问题,并提出了通用的解决方法。[103]

1987年,洛斯阿拉莫斯发布了特别版的《科学》期刊,概述了乌拉姆的成就[104],该期刊于1989年出版为书籍《从基数到混沌》。类似地,1990年,加利福尼亚大学出版社出版了乌拉姆及其洛斯阿拉莫斯合作者的数学报告集:《类比之间的类比》[105]。在他的职业生涯中,乌拉姆获得了新墨西哥大学、威斯康星大学和匹兹堡大学的荣誉学位。[9]

2021年,德国电影导演托尔斯滕·克莱因根据书籍《数学家的冒险》改编了一部关于乌拉姆生活的电影。

乌拉姆是纽约警察局情报与反恐副专员丽贝卡·威纳(Rebecca Weiner)的祖父。[106][107]
\subsection{参考书目}
\begin{itemize}
\item Kac, Mark; Ulam, Stanisław (1968). *Mathematics and Logic: Retrospect and Prospects*. New York: Praeger. ISBN 978-0-486-67085-0. OCLC 24847821.
\item Ulam, Stanisław (1974). Beyer, W. A.; Mycielski, J.; Rota, G.-C. (eds.). *Sets, Numbers, and Universes: Selected Works*. Mathematicians of Our Time. Vol. 9. The MIT Press, Cambridge, Mass. London. ISBN 978-0-262-02108-1. MR 0441664.
\item Ulam, Stanisław (1960). *A Collection of Mathematical Problems*. New York: Interscience Publishers. OCLC 526673.
\item Ulam, Stanisław (1983). *Adventures of a Mathematician*. New York: Charles Scribner's Sons. ISBN 978-0-684-14391-0. OCLC 1528346. (自传)
\item Ulam, Stanisław (1986). *Science, Computers, and People: From the Tree of Mathematics*. Boston: Birkhauser. ISBN 978-3-7643-3276-1. OCLC 11260216.
\item Ulam, Stanisław; Ulam, Françoise (1990). *Analogies Between Analogies: The Mathematical Reports of S.M. Ulam and his Los Alamos Collaborators*. Berkeley: University of California Press. ISBN 978-0-520-05290-1. OCLC 20318499.
\end{itemize}
\subsection{另见}
\begin{itemize}
\item 德语:《Abenteuer eines Mathematikers》(英文标题:*Adventures of a Mathematician*),关于斯坦尼斯瓦夫·乌拉姆的传记电影,改编自他的自传。
\item 波兰数学家列表
\item 波兰物理学家列表
\item 以斯坦尼斯瓦夫·乌拉姆命名的事物列表
\item 波兰科学与技术时间表
\end{itemize}
\subsection{参考文献}
\begin{enumerate}
\item "Mathematics Genealogy Project: Stanisław Marcin Ulam". *Mathematics Genealogy Project*. 访问日期:2022年5月17日。
\item Chartrand, Gary; Zhang, Ping (2013). *A First Course in Graph Theory*. Courier Corporation. 第78页. ISBN 978-0-486-29730-9.
\item Sullivan, Walter (1984年5月15日). "Stanislaw Ulam, Theorist on Hydrogen Bomb". *纽约时报*. 访问日期:2013年5月30日。
\item "Stanislaw Ulam | Biography, Facts, & Spiral". *大英百科全书*. 访问日期:2021年4月11日。
\item Ulam, S. M (1983). *Adventures of a Mathematician*. New York: Charles Scribner's Sons. 第9-15页. ISBN 9780684143910. OCLC 1528346.
\item Ulam, Adam Bruno (2002). *Understanding the Cold War: a historian's personal reflections*. New Brunswick, NJ: Transaction Publishers. 第19页. ISBN 9780765808851. OCLC 48122759. 访问日期:2011年12月28日。
\item Ulam, Molly (2000年6月25日). "Ulam Family of Lwow; Auerbachs of Vienna". *Genforum*. 访问日期:2011年10月10日。
\item "Genealogy of Michael Ulam". *GENi*. 2011年5月24日. 访问日期:2011年10月12日。
\item Ulam, Francoise (1987). "Vita: Excerpts from Adventures of a Mathematician" (PDF). *洛斯阿拉莫斯国家实验室*. 存档于2009年1月14日的原文 (PDF). 访问日期:2011年10月7日。
\item Ciesielski, Kryzystof; Thermistocles Rassias (2009). "On Stan Ulam and His Mathematics" (PDF). *澳大利亚数学分析与应用期刊*. 访问日期:2011年10月10日。第6卷,第1期,第1-9页,2009年。
\item Andrzej M. Kobos (1999). "Mędrzec większy niż życie" [A Sage Greater Than Life]. *Zwoje* (波兰语). 第3卷,第16期. 存档于2009年3月6日的原文. 访问日期:2013年5月10日。
\item Ulam, S. M (1983). *Adventures of a Mathematician*. New York: Charles Scribner's Sons. 第56-60页. ISBN 9780684143910. OCLC 1528346.
\item Ulam, Stanislaw (2002年11月). "Preface to the 'Scottish Book'". *Turnbull WWW Server*. 圣安德鲁斯大学数学与计算科学学院. 存档于2012年5月3日的原文。
\item Mauldin, R. Daniel (1981). *The Scottish Book*. Birkhauser. 第268页. ISBN 9783764330453. OCLC 7553633. 访问日期:2011年12月4日。
\item "Obituary for John C, Oxtoby". *纽约时报*. 1991年1月5日. 访问日期:2011年10月10日。
\item "Obituary for Adam Ulam". *哈佛大学公报*. 2000年4月6日. 访问日期:2011年10月10日。
\item Volsky, George (1963年12月23日). "Letter about Jozef Ulam". *Anxiously from Lwow*. Adam Ulam. 存档于2013年5月17日的原文. 访问日期:2013年5月24日。
\item "Lwow lives on at Leopolis Press". *The Hook*. 2002年11月14日. 存档于2015年7月1日的原文. 访问日期:2011年10月10日。
\item Budrewicz, Olgierd (1977). *The melting-pot revisited: twenty well-known USers of Polish background*. Interpress. 第36页. 访问日期:2012年9月11日。
\item Ulam, S. M (1983). *Adventures of a Mathematician*. New York: Charles Scribner's Sons. 第125-130页,第174页. ISBN 9780684143910. OCLC 1528346.
\item Ulam, S. M (1983). *Adventures of a Mathematician*. New York: Charles Scribner's Sons. 第143-147页. ISBN 9780684143910. OCLC 1528346.
\item "Staff biography of Stanislaw Ulam". 洛斯阿拉莫斯国家实验室. 访问日期:2011年10月22日。
\item Hoddeson, Lillian; Henriksen, Paul W.; Meade, Roger A.; Westfall, Catherine L. (1993). *Critical Assembly: A Technical History of Los Alamos During the Oppenheimer Years, 1943–1945*. New York: Cambridge University Press. 第130-137页. ISBN 978-0-521-44132-2. OCLC 26764320.
\item "Supercomputing". *History @ Los Alamos*. 洛斯阿拉莫斯国家实验室. 访问日期:2011年10月24日。
\item "From Calculators to Computers". *History @ Los Alamos*. 洛斯阿拉莫斯国家实验室. 访问日期:2011年10月24日。
\item Frisch, Otto (1974年4月). "Somebody Turned the Sun on with a Switch". *原子科学家公报*. 第30卷,第4期:17. Bibcode:1974BuAtS..30d..12F. doi:10.1080/00963402.1974.11458102. 访问日期:2013年5月29日。
\item Lehmann, Christopher (2002年3月4日). "Obituary of David Hawkins". *纽约时报*. 访问日期:2011年10月14日。
\item Hawkins, D.; S. Ulam (1944年11月14日). "Theory of Multiplicative Processes" (PDF). LANL报告 LA-171. 访问日期:2011年10月13日。
\item Ulam, S.; Everett, C. J (1948年6月7日). "Multiplicative Systems in Several Variables I, II, III". *LANL报告*. 加利福尼亚大学出版社. 访问日期:2011年10月13日。
\item Hewlett, Richard G.; Anderson, Oscar E. (1962). *The New World, 1939–1946* (PDF). University Park: 宾夕法尼亚州立大学出版社. 第304-307页. ISBN 978-0-520-07186-5. OCLC 637004643. (PDF)已归档于2022年10月9日。
\item Ulam, S. M (1983). *Adventures of a Mathematician*. New York: Charles Scribner's Sons. 第152-153页. ISBN 9780684143910. OCLC 1528346.
\item Ulam, S. M (1983). *Adventures of a Mathematician*. New York: Charles Scribner's Sons. 第162-157页. ISBN 9780684143910. OCLC 1528346.
\item Erdős, Paul (1985). "Ulam, the man and the mathematician" (PDF). *Journal of Graph Theory*. 第9卷,第4期:445–449. doi:10.1002/jgt.3190090402.
\item Rota, Gian-Carlo. "Stan Ulam: The Lost Cafe" (PDF). *Los Alamos Science*, 第15期,1987年. 访问日期:2011年10月22日。
\item Ulam, Françoise (1991). *Postscript to Adventures of a Mathematician*. 伯克利:加利福尼亚大学出版社. ISBN 978-0-520-07154-4.
\item Ulam, S. M (1983). *Adventures of a Mathematician*. New York: Charles Scribner's Sons. 第184-187页. ISBN 9780684143910. OCLC 1528346.
\item Metropolis, Nicholas (1987). "The Beginnings of the Monte Carlo Method" (PDF). *Los Alamos Science*, 第15期. 访问日期:2011年10月22日。
\item Eckhardt, Roger (1987). "Stan Ulam, John von Neumann, and the Monte Carlo method" (PDF). *Los Alamos Science*, 第15期. 访问日期:2011年10月22日。
\item Richtmyer, D.; J. Pasta; S. Ulam (1947年4月9日). "Statistical Methods in Neutron Diffusion" (PDF). *LANL报告 LAMS-551*. 访问日期:2011年10月23日。
\item Metropolis, Nicholas; Stanislaw Ulam (1949). "The Monte Carlo method" (PDF). *Journal of the American Statistical Association*. 44(247): 335–341. doi:10.1080/01621459.1949.10483310. JSTOR 2280232. PMID 18139350. 访问日期:2011年11月21日。
\item Hewlett, Richard G.; Duncan, Francis (1969). *Atomic Shield, Vol. II, 1947–1952. A History of the United States Atomic Energy Commission*. University Park, Pennsylvania: 宾夕法尼亚州立大学出版社. 第406–409页. ISBN 978-0-520-07187-2。
\item Rhodes, Richard (1995). *Dark Sun: The Making of the Hydrogen Bomb*. New York: Simon & Schuster. 第248页. ISBN 978-0-684-80400-2。
\item Hewlett, Richard G.; Duncan, Francis (1969). *Atomic Shield, 1947–1952. A History of the United States Atomic Energy Commission*. University Park: 宾夕法尼亚州立大学出版社. 第380–385页. ISBN 978-0-520-07187-2. OCLC 3717478。
\item Ulam, S. M (1983). *Adventures of a Mathematician*. New York: Charles Scribner's Sons. 第215页. ISBN 9780684143910. OCLC 1528346.
\item Peter Galison (1996). "5: Computer Simulations and the Trading Zone". 在 Peter Galison, David J. Stump (编辑). *The Disunity of Science: Boundaries, Contexts, and Power*. 斯坦福大学出版社. 第135页. ISBN 9780804725620.
\item Rhodes, Richard (1995). *Dark Sun: The Making of the Hydrogen Bomb*. New York: Simon & Schuster. 第422–424页. ISBN 978-0-684-80400-2.
\item "Staff biography of J. Carson Mark". 洛斯阿拉莫斯国家实验室. 原文存档于2012年7月16日. 访问日期:2011年10月22日。
\item Teller, E.; Ulam, S. (1951年3月9日). "On Heterocatalytic Detonations I. - Hydrodynamic Lenses and Radiation Mirrors" (PDF). 洛斯阿拉莫斯国家实验室. LAMS-1225. 原文存档于2012年3月1日. 访问日期:2022年4月4日。
\item Teller, E. (1951年4月4日), "A New Thermonuclear device", 技术报告 LAMS-1230, 洛斯阿拉莫斯国家实验室。
\item Rhodes, Richard (1995). *Dark Sun: The Making of the Hydrogen Bomb*. New York: Simon & Schuster. 第455–464页. ISBN 978-0-684-80400-2。
\item Hewlett, Richard G.; Duncan, Francis (1969). *Atomic Shield, 1947–1952. A History of the United States Atomic Energy Commission*. University Park: Pennsylvania State University Press. 第554–556页. ISBN 978-0-520-07187-2. OCLC 3717478.
\item Ulam, S. M (1983). *Adventures of a Mathematician*. New York: Charles Scribner's Sons. 第220–224页. ISBN 9780684143910. OCLC 1528346.
\item Bethe, Hans A. (1982年秋). "Reprinting of 1954 article: Comments on the History of the H-Bomb" (PDF). *Los Alamos Science*, No 6. 洛斯阿拉莫斯国家实验室. 访问日期:2011年11月3日.
\item Uchii, Soshichi (2003年7月22日). "Review of Edward Teller's Memoirs". *PHS Newsletter*, 52. 访问日期:2012年8月13日.
\item Schweber, S. S. (2000). *In the Shadow of the Bomb: Bethe, Oppenheimer, and the Moral Responsibility of the Scientist*. 普林斯顿: 普林斯顿大学出版社. 第166页. ISBN 978-0-691-04989-2.
\item Pasta, John; S. Ulam (1953年3月9日). "Heuristic studies in problems of mathematical physics" (PDF). *LANL报告* LA-1557. 访问日期:2011年11月21日.
\item Dauxois, Thierry (2008). "Fermi, Pasta, Ulam, and a mysterious lady" (PDF). *Physics Today*. 6 (1): 55–57. arXiv:0801.1590. Bibcode:2008PhT....61a..55D. doi:10.1063/1.2835154. S2CID 118607235. 已归档 (PDF),原文存档于2022年10月9日. 访问日期:2017年5月7日.
\item Fermi, E.; J. Pasta; S. Ulam (1955年5月). "Studies of Nonlinear Problems I" (PDF). *LANL报告* LA-1940. 访问日期:2011年11月21日.
\item Porter, Mason A.; Zabusky, Norman J.; Hu, Bambi; Campbell, David K. (2009年5–6月). "Fermi, Pasta, Ulam and the Birth of Experimental Mathematics" (PDF). *American Scientist*. 97 (3): 214–221. doi:10.1511/2009.78.214. 访问日期:2011年11月20日.
\item Lindley, David (2013年2月8日). "Focus: Landmarks – Computer Simulations Led to Discovery of Solitons". *Physics*. 6 (15): 15. Bibcode:2013PhyOJ...6...15L. doi:10.1103/Physics.6.15.
\item Longmier, C.; F. Reines; S. Ulam (1955年8月). "Some Schemes for Nuclear Propulsion" (PDF). *LANL报告* LAMS-2186. 访问日期:2011年11月24日.
\item Ulam, S. M (1983年). *Adventures of a Mathematician*. 纽约: Charles Scribner's Sons. pp. 249–250. ISBN 9780684143910. OCLC 1528346.
\item Schreiber, R. E.; Ulam, Stanislaw M.; Bradbury, Norris (1958年). "US Congress, Joint Committee on Atomic Energy: hearing on 22 January 1958". *Outer Space Propulsion by Nuclear Energy*. 美国政府印刷局. p. 47. 访问日期:2011年11月25日.
\item Everett, C. J.; S. M. Ulam (1955年8月). "On a Method of Propulsion of Projectiles by Means of External Nuclear Explosions" (PDF). *LANL报告* LAMS-1955. 访问日期:2011年11月24日.
\item "History of Project Orion". *The Story of Orion*. OrionDrive.com. 2008–2009. 访问日期:2011年10月7日.
\item Dyson, Freeman (1965年7月9日). "Death of a Project". *Science*. 149 (3680): 141–144. Bibcode:1965Sci...149..141D. doi:10.1126/science.149.3680.141. PMID 17734490. S2CID 39761976.
\item Ulam, S. M. (1961年), "On Some Statistical Properties of Dynamical Systems", *Proceedings of the 4th Berkeley Symposium on Mathematical Statistics and Probability*, 伯克利: University of California Press.
\item Abraham, Ralph (2011年7月9日). "Image Entropy for Discrete Dynamical Systems" (PDF). *加利福尼亚大学圣克鲁兹分校*. 访问日期:2013年5月30日.
\item Stein, P. R.; Stanislaw M. Ulam (1959年3月). "Quadratic Transformations. Part I" (PDF). *LANL报告* LA-2305. 洛斯阿拉莫斯国家实验室. 访问日期:2011年11月26日.
\item "Stanislaw Ulam". *美国物理学会*. 从原文归档于2015年7月2日. 访问日期:2013年5月14日.
\item "Stanislaw Marcin Ulam". *美国艺术与科学院*. 访问日期:2022年9月21日.
\item "S. M. Ulam". *美国国家科学院*. 访问日期:2022年9月21日.
\item "APS Member History". *美国哲学学会*. 访问日期:2022年9月21日.
\item Schrandt, Robert G.; Stanislaw M. Ulam (1970年12月). "Some Elementary Attempts at Numerical Modeling of Problems Concerning Rates of Evolutionary Processes" (PDF). *LANL报告* LA-4246. 洛斯阿拉莫斯国家实验室. 访问日期:2011年11月26日.
\item Beyer, William A.; Temple F. Smith; M. L. Stein; Stanislaw M. Ulam (1972年8月). "Metrics in Biology, an Introduction" (PDF). *LANL报告* LA-4973. 洛斯阿拉莫斯国家实验室. 访问日期:2011年11月26日.
\item "Komunikat o nadaniu Orderu Odrodzenia Polski" (PDF). *波兰共和国公报* (第23页,第6号). 1976年12月31日. 从原文归档于2018年4月24日. 访问日期:2023年7月25日.
\item "Stanislaw M. Ulam Papers". *美国哲学学会*. 访问日期:2013年5月14日.
\item "Françoise Ulam Obituary". *Santa Fe, New Mexican*. 2011年4月30日. 访问日期:2011年12月12日.
\item "Stanisław Ulam" (PDF) (法语). 从原文归档于2022年10月9日. 访问日期:2015年10月29日.
\item "Comparative Advantage". *世界贸易组织*. 访问日期:2021年3月10日.
\item The Collected Scientific Papers of Paul A. Samuelson, vol. iii, p. 683, MIT Press, 1966.
\item Ulam, Stanislaw (1958年5月). "Tribute to John von Neumann" (PDF). *美国数学学会公告*. 64, #3, part 2: 5. 从原文归档于2021年2月15日. 访问日期:2018年11月7日.
\item "Publications of Stanislaw M. Ulam" (PDF). *洛斯阿拉莫斯科学*. 第15期,1987年. 洛斯阿拉莫斯国家实验室. 访问日期:2011年12月6日.
\item *The Day After Trinity* 在IMDb上的页面.
\item Eckhardt, Roger (1987年). "Stan Ulam, John von Neumann, and the Monte Carlo Method" (PDF). *洛斯阿拉莫斯科学*. 洛斯阿拉莫斯国家实验室. 访问日期:2016年3月11日.
\item Casey, Thomas M. (2011年6月). "课程描述:社会科学家的蒙特卡洛方法". *政治与社会研究跨大学联盟*. 密歇根大学. 访问日期:2011年12月9日.
\item Poulter, Susan R. (1998年冬季). "蒙特卡洛模拟在环境风险评估中的应用" (PDF). *风险:健康、安全与环境*. 新罕布什尔大学. 从原文归档于2016年3月6日. 访问日期:2012年9月13日.
\item Klein, Sheldon (1966年5月23日). "使用蒙特卡洛技术的语言历史变化" (PDF). *机械翻译与计算语言学*. 9 (3 和 4): 67–81. 从原文归档于2011年10月16日. 访问日期:2011年12月9日.
\item Earl, M. A.; L. M. Ma (2002年3月12日). "电子束在外部磁场作用下的剂量增强:一项蒙特卡洛研究". *医学物理学*. 29 (4): 484–492. Bibcode:2002MedPh..29..484E. doi:10.1118/1.1461374. PMID 11991119. 访问日期:2011年12月9日.
\item Ludwig, John (2011年11月). "蒙特卡洛模拟的Big10比赛". *ludwig.com*. 访问日期:2011年12月9日.
\item Campbell, Donald H. (2010年3月17日). "非线性科学的诞生" (PDF). *美国物理学会*. 访问日期:2011年12月8日.
\item "CNLS: 向Martin Kruskal和Alwyn Scott致敬". *洛斯阿拉莫斯国家实验室*. 2007年. 访问日期:2011年12月8日.
\item "非线性研究中心历史". *洛斯阿拉莫斯国家实验室*. 访问日期:2011年12月8日.
\item "Ulam学者计划,非线性研究中心". *洛斯阿拉莫斯国家实验室*. 访问日期:2011年12月8日.
\item "焦点专题:Fermi-Pasta-Ulam问题——前50年". *混沌*. 15 (1). 2005年3月. 从原文归档于2012年5月3日. 访问日期:2011年12月9日.
\item "Fermi-Pasta-Ulam问题50周年:遗产、影响与展望". *CLNS第25届国际会议*. *洛斯阿拉莫斯国家实验室*. 2005年5月16日至20日. 访问日期:2011年12月9日.
\item "Ulam季刊主页". *佛罗里达大学*. 访问日期:2011年12月24日.
\item Dix, Julio G. (2004年6月25–27日). "运行免费电子期刊的若干方面" (PDF), 载于 Becker, Hans (编), *电子出版的新发展*, 斯德哥尔摩:欧洲数学家大会;ECM4卫星会议, 第41–43页, ISBN 978-3-88127-107-3, 访问日期:2013年1月5日.
\item "Ulam座谈会演讲者名单". *佛罗里达大学数学系*. 访问日期:2011年12月24日.
\item "Ulam百年纪念会议". *佛罗里达大学*. 2009年3月10–11日. 从原文归档于2012年4月24日. 访问日期:2011年12月24日.
\item Goad, Walter B (1987). "序列分析:Ulam对分子遗传学的贡献" (PDF). *洛斯阿拉莫斯科学*. *洛斯阿拉莫斯国家实验室*. 访问日期:2011年12月28日.
\item Sankoff, David (2000). "动态规划在计算生物学中的早期引入". *生物信息学*. 16 (1): 41–47. doi:10.1093/bioinformatics/16.1.41. PMID 10812476.
\item Beyer, William A.; Peter H. Sellers; Michael S. Waterman (1985). "Stanislaw M. Ulam对理论生物学的贡献" (PDF). *数学物理学报通讯*. 10 (2–3): 231–242. Bibcode:1985LMaPh..10..231B. CiteSeerX 10.1.1.78.4790. doi:10.1007/bf00398163. S2CID 2791811. 从原文归档于2011年9月27日. 访问日期:2011年12月5日.
\item Cooper, Necia Grant. "Stanislaw Ulam 1909–1984". *Los Alamos Science*, 第15期, 1987年. *洛斯阿拉莫斯国家实验室*. 访问日期:2011年12月6日.
\item Ulam, S. M. (1990). A. R. Bednarek; Françoise Ulam (编). *类比之间的类比*. 伯克利:加利福尼亚大学出版社. ISBN 978-0-520-05290-1. 访问日期:2011年12月24日.
\item Cramer, Maria (2023年8月13日). "纽约警察局新任情报局长接管秘密部门". *纽约时报*. 访问日期:2023年8月13日.
\item "警察局长Caban任命Rebecca Weiner为NYPD情报与反恐副局长". *www.nyc.gov*. 2023年7月19日. 访问日期:2023年8月13日.
\end{enumerate}
\subsection{外部链接}
\begin{itemize}
\item 1979年斯坦尼斯瓦夫·乌拉姆(Stanislaus Ulam)与马丁·谢尔温(Martin Sherwin)的音频采访 *曼哈顿计划之声*。
\item 1965年斯坦尼斯瓦夫·乌拉姆(Stanislaus Ulam)与理查德·罗兹(Richard Rhodes)的音频采访 *曼哈顿计划之声*。
\item "斯坦尼斯瓦夫·M·乌拉姆的出版物"(PDF)。 *洛斯阿拉莫斯科学*(特刊):313,1987年。ISSN 0273-7116。存档(PDF)自2022年10月9日。
\item 《冯·诺依曼:数学与计算的互动》在YouTube上的视频 – 1976年关于第一次国际计算历史研究会议的讲座。
\end{itemize}