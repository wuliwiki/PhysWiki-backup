% 菲利普·莱纳德(综述)
% license CCBYSA3
% type Wiki

本文根据 CC-BY-SA 协议转载翻译自维基百科\href{https://en.wikipedia.org/wiki/Philipp_Lenard}{相关文章}。

菲利普·爱德华·安东·冯·莱纳德(德语发音:[ˈfɪlɪp ˈleːnaʁt],匈牙利语:Lénárd Fülöp Eduárd Antal,1862年6月7日-1947年5月20日)是一位匈牙利裔德国物理学家,因“在阴极射线研究中取得的成果”以及发现其多种性质,于1905年获得诺贝尔物理学奖。

他最重要的贡献之一是对光电效应的实验验证:他发现,从阴极中逸出的电子的能量(速度)只依赖于入射光的频率,而与其强度无关。

莱纳德是民族主义者和反犹主义者;他是纳粹意识形态的积极支持者,早在1920年代便支持阿道夫·希特勒,并在纳粹时期成为“德国物理学”运动的重要榜样。值得注意的是,他将阿尔伯特·爱因斯坦的科学贡献贬称为“犹太物理学”。

\subsection{早年生活与工作}
菲利普·莱纳德于1862年6月7日出生在匈牙利王国的普雷斯堡(当时称为 Pozsony,即今日斯洛伐克的布拉迪斯拉发)。莱纳德家族最早在17世纪来自蒂罗尔,而他母亲的家族则源自巴登;他的父母都是讲德语的。[5] 父亲名为菲利普·冯·莱纳德,是普雷斯堡的一位葡萄酒商人;母亲名为安东妮·鲍曼。[6] 莱纳德在大多为日耳曼血统的祖先中也有一些马扎尔人血统。年幼的莱纳德曾就读于波若尼皇家天主教高级文理中学(Pozsonyi Királyi Katolikus Főgymnasium,今称 Gamča),据他在自传中记述,这段经历给他留下了深刻印象,尤其是他老师维吉尔·克拉特的个人魅力。[7]
1880年,他先后在维也纳和布达佩斯学习物理和化学。[7]

1882年,莱纳德离开布达佩斯,返回普雷斯堡。但在1883年,他前往海德堡,因为他申请布达佩斯大学助教职位遭到拒绝。在海德堡,他师从著名的罗伯特·本生,其间曾在柏林跟随赫尔曼·冯·亥姆霍兹学习了一个学期。他还曾师从格奥尔格·赫尔曼·昆克[1],并于1886年获得博士学位。[1][8]1887年,他回到布达佩斯,在洛兰·厄特沃什手下担任演示员。[7]此后他曾在亚琛、波恩、布雷斯劳、海德堡(1896–1898年)和基尔(1898–1907年)任教,并最终于1907年回到海德堡大学,出任菲利普·莱纳德研究所所长。莱纳德于1905年当选为瑞典皇家科学院院士,1907年成为匈牙利科学院院士。[7] 他早期的研究涉及磷光和荧光现象以及火焰的导电性等课题。
\subsection{对物理学的贡献}
\subsubsection{光电效应研究}
\begin{figure}[ht]
\centering
\includegraphics[width=6cm]{./figures/6f04b289d4a626c3.png}
\caption{} \label{fig_FLPL_1}
\end{figure}
作为物理学家,莱纳德的主要贡献在于对阴极射线的研究,他从 1888 年开始涉足这一领域。在他的研究之前,阴极射线是通过一种原始的、部分抽真空的玻璃管产生的,这种玻璃管中装有金属电极,可施加高电压。然而,这种装置使得阴极射线的研究极为困难,因为射线被封闭在玻璃管中,不易接近,同时射线也会受到玻璃管内空气分子的影响。为了解决这些问题,莱纳德设计出一种方法:在玻璃管上制作金属小窗,这些小窗既足够厚以承受内外压力差,又足够薄以允许射线穿过。借助这样的“窗户”,他可以将阴极射线引出到实验室环境中,或者引入另一个完全抽真空的腔体中。这种窗后来被称为莱纳德窗。借助这些窗,他可以更方便地检测阴极射线,并通过涂有磷光材料的纸张来测量其强度。[9]特别地,他采用了一种名为十五烷基对甲苯酮的材料作为阴极射线探测剂,这种材料在探测阴极射线方面非常有效,但不幸的是,它对X射线不具荧光反应。这对莱纳德来说是个遗憾。
当伦琴试图复现莱纳德的实验时,发现莱纳德已经买下了市面上所有的十五烷基对甲苯酮,他不得不用氰酸铂钡(替代。恰巧的是,这种替代材料对紫外线和X射线都很敏感,从而让伦琴发现了X射线。[10]
