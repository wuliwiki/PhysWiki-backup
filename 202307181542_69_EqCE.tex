% 等间隔能级系统(正则系宗)
% keys 配分函数|统计力学|正则系综

\begin{issues}
\issueDraft
\end{issues}

\addTODO{我觉得这一块的推导应该是巨正则系综的范畴,要引入化学势 $\mu$。}

\pentry{玻尔兹曼因子% 未完成,
配分函数 %未完成
}

\subsection{结论}
若一个系统的能量只能取一系列离散的值(能级),但相邻能级间距恰好为 $\varepsilon$,那么该系统在温度 $\tau$ 达到热平衡时,平均能量为
\begin{equation}
\ev{\varepsilon} = \frac{\varepsilon}{\E^{\varepsilon/\tau} - 1}~.
\end{equation}

\subsection{推导1}
令第 $n$ 个能级的能量为 $\varepsilon_n$,能量的平均值为
\begin{equation}\label{eq_EqCE_2}
\ev{\varepsilon} = \frac{\sum\limits_{n = 0}^\infty  \varepsilon_n \exp(-\varepsilon_n/\tau)}{\sum\limits_{n = 0}^\infty \exp(-\varepsilon_n/\tau)}~.
\end{equation}
其中 $\sum\limits_{n = 0}^\infty \exp(-\varepsilon_n/\tau)$ 就是配分函数 $Q$。

对于等间距能级,假设等间距能级 $\varepsilon_n = n\varepsilon$ (也可以假设 $\varepsilon_n = \varepsilon_0 + n\varepsilon $,上式的分子分母都多出一个因子 $\exp(-\varepsilon_0/\tau)$,最后的结果相同)。首先化简配分函数
\begin{equation}
Q = \sum_{n = 0}^\infty \E^{-n\varepsilon /\tau}  = \sum_{n = 0}^\infty (\E^{-\varepsilon/\tau})^n~.
\end{equation}
由于 $\E^{-\varepsilon/\tau} < 1$,根据等比数列求和公式 %未完成
\begin{equation}\label{eq_EqCE_4}~.
Q = \frac{1}{1 - \E^{-\varepsilon /\tau}}
\end{equation}
再化简分子
\begin{equation}\label{eq_EqCE_5}
\sum_{n = 0}^\infty  n\varepsilon \exp(-n\varepsilon /\tau)  = \varepsilon \sum_{n = 0}^\infty  n(\E^{-\varepsilon/\tau})^n~.
\end{equation}
令 $x = \E^{-\varepsilon/\tau}$,有 $x < 1$。同样根据等比数列求和公式
\begin{equation}
\sum_{n = 0}^\infty  n x^n = x\dv{x}\sum_{n = 0}^\infty x^n = x\dv{x} \qty(\frac{1}{1 - x}) = \frac{x}{(1 - x)^2}~.
\end{equation}
把 $x$ 换成 $\E^{-\varepsilon/\tau}$,\autoref{eq_EqCE_5} 变为
\begin{equation}
\frac{\varepsilon \E^{-\varepsilon/\tau}}{(1 - \E^{-\varepsilon/\tau})^2}~.
\end{equation}
把分子分母代入平均值公式\autoref{eq_EqCE_2} 得到最后结论
\begin{equation}
\ev{\varepsilon} = \left. \frac{\varepsilon \E^{-\varepsilon/\tau}}{(1 - \E^{-\varepsilon/\tau})^2} \middle/ \frac{1}{1 - \E^{-\varepsilon/\tau}}  \right. = \frac{\varepsilon \E^{-\varepsilon/\tau}}{1 - \E^{-\varepsilon/\tau}} = \frac{\varepsilon}{\E^{\varepsilon/\tau} - 1}~.
\end{equation}

\subsection{推导2}
由\autoref{eq_EqCE_4} 已知配分函数
\begin{equation}
Q = \frac{1}{1 - \E^{-\varepsilon/\tau}} = \frac{1}{1 - \E^{-\varepsilon\beta}}~.
\end{equation}
我们也可以直接用能量均值公式
\begin{equation}
\ev{\varepsilon} = -\pdv{\beta} \ln Q
= \pdv{\beta} \ln (1 - \E^{-\varepsilon\beta}) = \frac{\varepsilon \E^{-\varepsilon\beta}}{1 - \E^{-\varepsilon\beta}} = \frac{\varepsilon}{\E^{\varepsilon /\tau} - 1}~.
\end{equation}
结果相同。
