% 真因子树
% 环|因式分解|唯一析因环|因子|素理想|极大理想

\pentry{整环\upref{Domain}}

真因子树的概念,是笔者优化了“因子链”的概念而得出的一套描述因式分解理论的框架。

\subsection{概念的描述}

\begin{definition}{真因子}
给定整环 $R$,对于 $r, s\in R$,如果 $s|r$ 且 $r\not{|}s$,那么称 $r$ 是 $s$ 的真因子。
\end{definition}

\begin{definition}{单位}\label{def_FctTre_3}
给定整环 $R$,对于 $u\in R$,如果 $u^{-1}$ 是存在的,那么称 $u$ 是 $R$ 的一个\textbf{单位(unit)}。$R$ 中全体单位的集合,记为 $U$。
\end{definition}

显然,如果有单位 $u$ 使得 $r=us$,那么 $r$ 和 $s$ 互相不是真因子。我们将这样的 $r, s$ 视为等价的:

\begin{definition}{}\label{def_FctTre_2}
给定整环 $R$,定义集合 $R$ 上的一个等价关系:对于 $r, s\in R$,$r$ 等价于 $s$ 当且仅当存在单位 $u$ 使得 $r=su$。等价的元素视为同一个元素,或者说把每个等价类看成一个元素,得到的集合是 $R$ 模去该等价关系的商集\footnote{见\textbf{二元关系}\upref{Relat}。},记为 $R_u$。
\end{definition}

举例来说,在\textbf{整数环}$\mathbb{Z}$ 上,对于任意正整数 $n$,我们把它等价于 $-n$。

%——————————————————————————————————————————————————————————————————————————————————————————————————————————————————

有了 $R_u$ 的概念,就可以定义本节核心的概念了:真因子树。

\begin{definition}{真因子树}
给定整环 $R$,对于 $r\in R$,如果存在非单位的 $a, b\in R$ 使得 $ab=r$,那么可以从 $r$ 画两个箭头分别指向 $a$ 和 $b$,而 $\{a, b\}$ 就是 $r$ 的一个因子分解;同样,如果 $a$ 和 $b$ 可以继续分解为其它非单位元素之积,那么也可以继续画出箭头指向它们对应的因子分解。如是反复,直到不能继续进行下去为止,所获得的整个结构称为 $r$ 的一棵\textbf{真因子树}。

每个从 $r$ 开始,出发一路指向末端的路径,称为 $r$ 的一个\textbf{枝条},枝条中涉及到的箭头数量,称为枝条的\textbf{长度}。一棵真因子树中最长的枝条的长度,称为这棵树的\textbf{高度}。特别地,如果 $r$ 无法进行分解,也就是说它的树只包含 $r$ 本身,那么定义这棵树的高度为 $0$。

一棵树中从元素 $a$ 到元素 $b$ 的\textbf{路径},其长度定义为这条路径上包含的箭头数量。

$r$ 的真因子树一般不止一棵。
\end{definition}

%要画图说明;说明等价关系是什么,以及真因子树长什么样。


\subsection{用真因子树进行描述}

以下讨论限制在整环 $R$ 的集合上。利用真因子树的语言来直接翻译各种概念的方式如下:

\begin{definition}{}\label{def_FctTre_1}
\begin{itemize}
\item 不可约元素:在某一棵树中为末端。
\item 素元素:$p$ 是素元素,当且仅当对于任意 $a, b\in R$,如果 $p$ 在 $ab$ 的\textbf{某棵}真因子树上,那么 $p$ 必在 $a$ 或 $b$ 的\textbf{某棵}真因子树上。
\item 有限析因性:对于任意 $r$,存在一个正整数 $N_r$,使得 $r$ 的任意枝条长度不超过 $N_r$。
\item 唯一析因性:有限析因,且对于任意 $r$,其任何两棵树的末端元素构成的集合都是一样的。
\end{itemize}
\end{definition}

\begin{theorem}{有限析因时素元素的等价定义}\label{the_FctTre_1}
设 $p$ 是整环 $R$ 中的一个元素,且$R$具有有限析因性。则$p$是素元素,当且仅当对于任意 $a$,如果 $p|a$,那么 $p$ 必在 $a$ 的每一棵树上。
\end{theorem}

应用素元素的定义,不难证明该定理。


\begin{theorem}{素元素必是不可约元素}\label{the_FctTre_2}
整环 $R$ 中的素元素都是不可约元素。
\end{theorem}

\textbf{证明}:

反设素元素 $p\in R$ 不是末端,那么就可以得到 $p$ 的\textbf{长度不为零}的一棵树,其第一级分解为 $p=ab$;而由于素元素的定义,它又必须在以 $a$ 和 $b$ 为起点的某棵树上,从而在自己的后面,而这是不可能的。因此反设不成立,素元素必是末端。

\textbf{证毕}。

\begin{exercise}{}\label{exe_FctTre_1}
利用\autoref{the_FctTre_1} ,证明\autoref{the_FctTre_2} 。
\end{exercise}

今后我们也会引用真因子树的概念来方便阐释因式分解相关的问题。

