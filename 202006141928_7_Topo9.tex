% 空间偶和带基点空间
\pentry{映射空间\upref{Topo8}}

\begin{definition}{空间偶}
\begin{itemize}
\item 把拓扑空间$X$和它的一个子空间$A$绑定,记为$(X, A)$,称其为一个\textbf{空间偶}.
\item 如果一个连续映射$f:X\rightarrow Y$还满足$f(A)\subseteq B$,即限制在$A$上时可以认为$f:A\rightarrow B$,那么我们也称这是一个空间偶之间的映射$f:(X, A)\rightarrow(Y, B)$.
\item 空间偶的映射空间(Y, B)^{(X, A)}是所有满足$f(A)\subseteq B$的连续映射$f:X\rightarrow Y$的集合.显然这是$Y^X$的子集,因此我们定义其拓扑为$Y^X$的子拓扑.
\item 乘积空间偶$(X, A)\times(Y,B)$定义为$(X\times Y, (X\times B)\cup(A\times Y))$.特别地,$X\cong(X, \varnothing)$,所以由$X\times(Y, B)=(X\times Y, X\times B)$.


\end{itemize}
\end{definition}

\begin{definition}{带基点空间}

取拓扑空间$X$中的一个点$x_0$.称空间偶$(X, \{x_0\})=(X, x_0)$是一个\textbf{带基点空间},称$x_0$为这个带基点空间的\textbf{基点}.

\end{definition}
