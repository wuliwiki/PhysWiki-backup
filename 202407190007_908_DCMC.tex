% 电磁脉冲
% license CCBYSA3
% type Wiki

(本文根据 CC-BY-SA 协议转载自原搜狗科学百科对英文维基百科的翻译)

电磁脉冲(Electromagnetic Pulse,EMP),有时也称为瞬变电磁干扰,是电磁能量的短脉冲。这种脉冲的来源可以是自然发生的,也可以是人为的,并且根据来源的不同,可以是辐射场、电场、磁场或传导电流。

电磁脉冲干扰通常会破坏或损坏电子设备,在较高的能量水平下,强电磁脉冲(如雷击)会损坏建筑物和飞机机体的物理结构。电磁脉冲效应的管理是电磁兼容( electromagnetic compatibility,EMC)工程的一个重要分支。

释放高能电磁脉冲的破坏性武器已经被开发出来。

\subsection{一般特性}
电磁脉冲是电磁能量的短脉冲。它的短持续时间意味着它将在一个频率范围内传播。电磁脉冲的典型特征是:
\begin{itemize}
\item 能量类型(辐射、电、磁或传导)。
\item 存在的频率范围或频谱。
\item 脉冲波形:形状、持续时间和振幅。
\end{itemize}
最后两种,频谱和脉冲波形通过傅里叶变换相互关联,可以被视为描述同一脉冲的两种不同方式。
\subsubsection{1.1 能量类型}
电磁脉冲能量可以凭借如下四种形式传递:
\begin{itemize}
\item 电场
\item 磁场
\item 电磁辐射
\item 电传导
\end{itemize}
由于麦克斯韦方程,任何一种形式的电磁能脉冲总是伴随着其他形式电磁能量,然而在典型的脉冲中,一种形式将占主导地位。

一般来说,只有辐射传递方式可以用于长距离传递,其他辐射仅适用于短距离传递。但也有一些例外,比如太阳磁耀斑。
\subsubsection{1.2 频率范围}
电磁能量脉冲通常包括从直流DC(零赫兹)到某个上限的许多频率,这种频率范围取决于辐射源。电磁脉冲的频率范围,有时被称为“直流至日光”,不包括由光学(红外、可见、紫外)和电离(x光和伽马射线)组成的最高频率范围。

某些类型的电磁脉冲事件会留下光学痕迹,如闪电和火花,但这些都是流经空气的电流的副作用,并不属于电磁脉冲本身。
\subsubsection{1.3 脉冲波形}
脉冲波形描述了其瞬时振幅(场强或电流)随时间的变化情况。真实的脉冲往往非常复杂,所以经常使用简化模型。这种模型通常用图表或数学方程来描述。
\begin{figure}[ht]
\centering
\includegraphics[width=14.25cm]{./figures/06d6169b006b519c.png}
\caption{} \label{fig_DCMC_1}
\end{figure}