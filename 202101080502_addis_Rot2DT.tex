% 平面旋转变换
% 线性代数|平面旋转变换|三角恒等式|坐标系|极坐标系

\pentry{三角恒等式\upref{TriEqv}, 极坐标系\upref{Polar}}

\begin{figure}[ht]
\centering
\includegraphics[width=5cm]{./figures/Rot2DT_1.pdf}
\caption{把矢量绕原点旋转 $\alpha$ 角} \label{Rot2DT_fig1}
\end{figure}

已知直角坐标系中一点 $P(x,y)$, $P$ 绕原点逆时针旋转 $\alpha $ 角($\alpha  \in R$) 之后变为 $P'(x',y')$ 则有
\begin{align}\label{Rot2DT_eq1}
x' &= (\cos \alpha)x + (- \sin \alpha)y \\
\label{Rot2DT_eq2}
y' &= (\sin \alpha)x + (\cos \alpha)y
\end{align}
其逆变换如下,即已知 $P'(x',y')$ 求 $P(x,y)$ 
\begin{align}\label{Rot2DT_eq3}
x &= ( \cos \alpha  )x' + ( \sin \alpha  )y' \\
\label{Rot2DT_eq4}
y &= ( - \sin \alpha)x' + ( \cos \alpha )y'
\end{align}

\subsubsection{绕任点旋转}


\subsection{推导}

平面上一点 $P(x,y)$ 也可以用极坐标 $(r, \theta)$ 表示,一般情况下令极点与原点重合,极径与 $x$ 轴重合,则有
\begin{equation}
x = r\cos \theta \qquad y = r\sin \theta 
\end{equation}     
把点 $P$ 绕原点逆时针旋转 $\alpha $ 角变为 $P'$, 则 $P'$ 极坐标为 $(r, \theta  + \alpha)$. 根据上式计算为 $P'$ 的直角坐标 $(x', y')$ 并用两角和公式(\autoref{TriEqv_eq1}~\upref{TriEqv})化简如下
\begin{align}
x' &= r\cos(\theta  + \alpha) = r\cos\theta \cos\alpha  - r\sin\theta \sin\alpha  = x\cos\alpha  - y\sin\alpha \\
y' &= r\sin(\theta  + \alpha) = r\sin\theta \cos\alpha  + r\cos\theta \sin\alpha  = x\sin\alpha  + y\cos\alpha 
\end{align} 
这就证明了\autoref{Rot2DT_eq1} 和\autoref{Rot2DT_eq2} 两式.

若要证\autoref{Rot2DT_eq3} 和\autoref{Rot2DT_eq4} 有两种方法.一是将\autoref{Rot2DT_eq1} 和\autoref{Rot2DT_eq2} 式中的 $x, y$ 看成未知数,解二元一次方程组.另一种方法的思路是,既然 $P$ 逆时针旋转 $\alpha $ 角为 $P'$, 那么把 $P'$ 顺时针旋转 $\alpha$ 角可得到 $P$. 而“顺时针旋转 $\alpha$ 角”就是“逆时针旋转 $-\alpha $ 角”.把变换\autoref{Rot2DT_eq1} 和\autoref{Rot2DT_eq2} 中的 $\alpha$ 换为 $-\alpha$ 再化简得
 \begin{align}
x &= \cos(-\alpha) x' - \sin(-\alpha) y' = \cos\alpha x' + \sin\alpha y'\\
y &= \sin(-\alpha) x' + \cos(-\alpha) y' =  -\sin\alpha x' + \cos\alpha y'
\end{align}
证毕.
