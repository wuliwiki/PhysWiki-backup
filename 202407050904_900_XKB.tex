% Linux 键盘设置(XKB)笔记
% license Xiao
% type Note

\begin{issues}
\issueDraft
\end{issues}

\begin{itemize}
\item Linux 中所有的 GUI 功能都使用 X windows 系统, 通过 X server 显示。 X server 会在重新 login 以后重启。
\item 注意 Ubuntu (20.04 亲测, 参考\href{https://manpages.ubuntu.com/manpages/focal/en/man5/keyboard.5.html}{这里})设置 xkb config 文件无效的! 需要在 \verb`/etc/default/keyboard` 里面设置(同样重新 login 以后生效)。 例如添加一行 \verb`XKBOPTIONS=ctrl:swapcaps` 对调 Ctrl 和 Capslock 按键。
\item 这相当于 \verb`setxkbmap -option "ctrl:swapcaps"`, 但后者只对当前的 \verb`X session` 有效。
\item 要恢复,用 \verb`setxkbmap -option`
\item 显示 xkb 当前的所有设置 \verb`setxkbmap -print -verbose 10`
\item 更多设置参考\href{https://gist.github.com/jatcwang/ae3b7019f219b8cdc6798329108c9aee}{这里}。
\item 另一个是参考\href{https://www.linux.com/training-tutorials/hacking-your-linux-keyboard-xkb/}{这里}。 设置文件 \verb`/usr/share/X11/xorg.conf.d/10-keyboard.conf`:
\begin{lstlisting}[language=none]
Section "InputClass"
        Identifier "system-keyboard"
        MatchIsKeyboard "on"
        Option "XkbLayout" "us"
        Option "XkbModel" "pc104"
        Option "XkbOptions" "ctrl:swapcaps"
EndSection
\end{lstlisting}
\end{itemize}
