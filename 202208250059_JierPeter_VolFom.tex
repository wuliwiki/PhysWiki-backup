% 体积形式
% 微分几何|微分形式|体积|微积分|流形|calculus|manifold|differential form|differential geometry


\pentry{微分形式\upref{Forms}}


体积形式是微分流形上一种微分形式,可以用于计算切平面附近区域的大小,进而实现积分运算.

\begin{definition}{体积形式}
给定$n$维微分流形$M$,则其上一个处处非零的$n$-形式即为一个\textbf{体积形式(volume form)}.
\end{definition}

\begin{example}{二维欧几里得空间}
设$M$是二维欧几里得空间$\mathbb{R}^2$,给定两个\textbf{坐标函数}以确定一个(局部)坐标系.设在这个坐标系中,欧几里得度量被表示为$g_{ab}$.

这里举一个坐标函数与坐标系的例子:给定一点$o$和从这一点出发的一根射线$R$,再规定一个逆时针方向.任取$p\in M$,定义第一个坐标函数$r(p)$为$p$到给定点的距离,第二个坐标函数$\theta(p)$为射线$R$沿着规定方向转到与射线$op$重合时所经过的弧度,取值范围为$[0, 2\pi)$.上述定义的坐标函数所得到的坐标系,就是我们熟知的极坐标系.在极坐标系中,欧几里得度量为
\begin{equation}
g_{ab}=\pmat{
    1&0\\
    0&r^2
}
\end{equation}


如果用以坐标函数$r, \theta$确定的余切向量$\dd r$和$\dd \theta$构成余切空间的基,那么切空间的对偶基由$\partial_r=\cos\theta\partial_x+\sin\theta\partial_y$和$\partial_\theta=\frac{-\sin\theta}{r}\partial_x+\frac{\cos\theta}{r}\partial_y$构成.



设坐标函数$x$和$y$诱导出的是我们熟知的\textbf{标准正交基},其中$x$轴的非负半轴与上述极坐标轴重合.





而$2$-形式$\dd r\wedge\dd \theta$对这两个切向量的作用为\footnote{推导留给读者,注意按定义,$\dd f(\partial_i)=\partial_i f$,其中$i\in\{x, y\}$,$f\in\{r, \theta\}$.余切向量的外积定义见\autoref{ExtAlg_sub1}~\upref{ExtAlg}.}
\begin{equation}
\dd r\wedge\dd \theta(\partial_x, \partial_y) = \frac{1}{r}
\end{equation}

其几何意义可以如此理解:按照通常的面积(即二维的体积)定义,给定一个无穷小量$\varepsilon$,则切向量$\varepsilon\partial_r$和$\varepsilon\partial_\theta$所张成的小正方形的面积为
\begin{equation}
\frac{1}{r}\cdot \varepsilon^2
\end{equation}

前面那个$1/r$不利于计算面积,从而不利于计算积分.因此,我们定义


\end{example}






\begin{theorem}{}
当且仅当流形可定向时,流形上存在处处非零的体积形式.
\end{theorem}






















