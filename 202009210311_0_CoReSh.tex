% 宇宙学红移
% keys FRW 度规|红移|哈勃常数

\begin{issues}
\issueMissDepend
\end{issues}

\pentry{FRW 度规 \upref{FRW}}
由于物体在宇宙中传播的过程中,宇宙也在加速膨胀,所以为了准确的测量物体在宇宙传播过程中的物理量,我们需要引进宇宙学红移的概念.

\subsection{光子的红移}
从量子力学的描述中,光子的波长可以定义为$\lambda=\frac{h}{p}$.当光在$t_1$时刻发射以波长$\lambda_1$发射,于$t_0$时刻被接收所观测到的波长为
\begin{equation}
\lambda_0=\frac{a(t_0)}{a(t_1)}\lambda_1,\label{CoReSh_eq1}
\end{equation}
其中$a(t)$为$t$时刻的宇宙尺度因子.因为宇宙在加速膨胀$a(t_0)>a(t_1)$, 所以容易得$\lambda_0>\lambda_1$.

\subsection{红移因子}
为了计算方便,我们可以通过定义从星系发出的,经过一段时间到达地球后被观测的光的红移为\textbf{红移因子(redshift parametre)}
\addTODO{补充及引用 “光的多普勒效应” 词条}
\begin{equation}
z=\frac{\lambda_0-\lambda_1}{\lambda_1},
\end{equation}
显然从\autoref{CoReSh_eq1} 我们可以推出
\begin{equation}
1+z=\frac{a(t_0)}{a(t)}=\frac{1}{a(t_1)}. \label{CoReSh_eq2}
\end{equation}
一般地我们设现在的宇宙尺度因子 $a(t_0)=1$.

\subsection{哈勃常数}
我们把 $t_1$ 时刻的宇宙尺度因子 $a(t_1)$ 以现在的时刻 $t_0$ 为原点作泰勒展开,可得
\begin{equation}
a(t_1)=a(t_0)(1+(t-t_0)H_0+\cdots)
\end{equation}

从\autoref{CoReSh_eq2} 我们可以看出$z=H_0(t_0-t_1)$,这里我们定义了\textbf{哈勃常数(Hubble Constant)} $H_0$. 显然我们可以发现红移参量与光走过的距离 $d=c(t_0-t_1)$ 成正比
\begin{equation}
z\simeq\frac{H_0d}{c}.
\end{equation}

哈勃常数常常被定义为以下数值
\begin{equation}
H_0 \equiv100 h {\rm kms}^{-1}{\rm Mpc}^{-1},
\end{equation}
其中$h$测得的数值为\footnote{12Planck 2013 Results – Cosmological Parameters [arXiv:1303.5076]}
\begin{equation}
h\sim 0.67 \pm 0.01.
\end{equation}