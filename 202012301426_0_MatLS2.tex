% 矩阵与线性映射 2

\pentry{补空间\upref{DirSum}}

\begin{theorem}{}
令线性映射为 $A:X\to Y$ 对应的矩阵为 $\mat A$. $X$ 是 $N_X$ 维矢量空间, $A$ 的零空间为 $X_0$, 是 $X$ 的子空间. 令 $\mat A$ 的秩为 $R$, 那么
\begin{equation}
N_X = N_0 + R
\end{equation}
\end{theorem}

该定理揭示了线性映射的结构(\autoref{MatLS2_fig1} ). 首先根据零空间的定义, 其中的每个矢量都被 $A$ 映射到 $Y$ 空间中的零矢量, 即 $A(X_0) = 0$. 令 $X_1$ 为 $X_0$ 在 $X$ 中的任意一个补空间
\begin{figure}[ht]
\centering
\includegraphics[width=9cm]{./figures/MatLS2_1.pdf}
\caption{子空间的映射. $X_0$ 是零空间, $X_1$ 是 $X_0$ 在 $X$ 中的补空间.} \label{MatLS2_fig1}
\end{figure}


\subsubsection{证明}


