% 热传导定律

如果气体内各部分的品度不同,从温度较高处向温度较低处,将有热量的传递,这一现象叫做热传导(heat conduction)现象

如图所示,$Ox$轴是气体温度变化最大的方向,在这个方向上气体温度的空间变化率$\mathrm dT/\mathrm dx$,叫做\textbf{温度梯度}.设$\Delta S$为垂直于$Ox $轴的某指定平面的面积.实验证明,在单位时间内,从温度较高的一侧,通过这一平面,向温度较低的一侧所传递的热量,与这一平面所在处的温度梯度成正比,同时也与面积$\Delta S$成正比,即
\begin{equation}
\frac{\Delta Q}{\Delta t}=-\kappa \frac{\mathrm{d} T}{\mathrm{d} x} \Delta S
\end{equation}
比例系数$\kappa$叫做热导率(thermal conductivity).式中负号表示热量传递的方向是从高温处传到低温处,和温度梯度的方向是相反的热导率
的单位是$\rm W /(m \cdot K)$实验测得,在$0°C$ 时,氢的
热导率为0. 168 WI (m·K) , 氧力2. 42x IO -' W/
(m·K), 空气为2. 23 x 10 -1 W / (m·K) 在
100°C 时,水气的热导率为2. J8x 10-'W/ (m·
K) 显然,气体的热导率都很小,所以,当气体中
不存在对流时,气体可用作很好的绝热材料
在气体动理论中,气体热传导现象是这样解
释的:在温度较高的热层中,分子平均动能较大;
T, >TI
图5-12 热传导现象
而在温度较低的冷层中,分子平均动能较小由于冷热两层分子的互相掺和与相互碰撞的结
果,从热层到冷层出现热运动能量的净捡运捡运的热运动能量,对单原子气体来说,只是分
子的平动动能;而对多原子气体来说,还包含转动和挔动的能量在内