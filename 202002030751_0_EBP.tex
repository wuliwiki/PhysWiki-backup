% 电磁场的动量守恒 动量流密度张量
% 动量守恒|麦克斯韦应力张量|动量流密度

\pentry{能流密度,%未完成
 张量,%未完成
张量的散度%未完成
}
\subsection{结论}

假设电磁场动量守恒,则动量流密度张量为
\begin{equation}
T_{ij} = \epsilon_0 \qty(\frac12 \bvec E^2 \delta_{ij} - E_i E_j) + \frac{1}{\mu_0} \qty(\frac12 \bvec B^2 \delta_{ij} - B_i B_j)
\end{equation} 
该张量也成为\textbf{麦克斯韦应力张量(Maxwell Stress Tensor)}

\subsection{推导}	

能量是标量,所以能流密度就是矢量.但动量本身就是矢量,要如何表示动量流密度呢? 
我们可以分析动量在某方向分量的流密度.根据张量的散度%未完成:引用
\begin{equation}
(\div \ten T)_j = \sum_i \pdv{x_i} T_{ji}
\end{equation}
假设电磁场满足动量守恒,在闭合空间中,有 “转换速率+流出速率+增加速率= 0”(类比电磁场的能量守恒公式).则电磁场的动量守恒会有
\begin{equation}\label{EBP_eq1}
\bvec f + \div \ten T + \mu_0 \epsilon_0 \pdv{\bvec s}{t} = \bvec 0
\end{equation} 
由广义洛伦兹力%未完成:引用
计算电荷的受力密度 $\bvec f$ %未完成:引用
\begin{equation}
\bvec f = \rho (\bvec E + \bvec v \cross \bvec B) = \rho \bvec E + \bvec j \cross \bvec B
\qquad (\bvec j = \rho \bvec v)
\end{equation} 
由于\autoref{EBP_eq1} 的后两项是电磁场的量,不能含有关于电荷的量,所以接下来要通过麦克斯韦方程组 %未完成:引用
把电荷密度 $\rho$ 和电流密度 $\bvec j$ 替换成电磁场.
\begin{equation}
\rho  = \epsilon_0 \div \bvec E \qquad
\qty(\div \bvec E = \frac{\rho}{\epsilon_0})
\end{equation}
\begin{equation}
\bvec j = \frac{1}{\mu_0} \curl \bvec B - \epsilon_0 \pdv{\bvec E}{t}
\qquad \qty(\curl \bvec B = \mu_0 \bvec j + \epsilon_0 \mu_0 \pdv{\bvec E}{t})
\end{equation}
代入上式,得
\begin{equation}
\bvec f = \epsilon_0 (\div \bvec E) \bvec E + \frac{1}{\mu_0} (\curl \bvec B) \cross \bvec B - \epsilon_0 \pdv{\bvec E}{t} \cross \bvec B
\end{equation} 
其中 
\begin{equation}\ali{
\pdv{\bvec E}{t} \cross \bvec B &= \pdv{t} (\bvec E \cross \bvec B) - \bvec E \cross \pdv{\bvec B}{t}\\ 
&= \pdv{t} (\bvec E \cross \bvec B) - (\curl \bvec E) \cross \bvec E
\qquad \qty(\curl \bvec E =  - \pdv{\bvec B}{t})
}\end{equation} 
代入上式得
\begin{equation}\ali{
\bvec f &= \epsilon_0 (\div \bvec E) \bvec E + \frac{1}{\mu_0} (\curl \bvec B) \cross \bvec B + \epsilon_0 (\curl \bvec E) \cross \bvec E - \epsilon_0 \pdv{t} (\bvec E \cross \bvec B)\\
&= \epsilon_0 [ (\div \bvec E)\bvec E + (\curl \bvec E) \cross \bvec E ] + \frac{1}{\mu_0} (\curl \bvec B) \cross \bvec B - \epsilon_0 \mu_0 \pdv{\bvec s}{t}
} \end{equation} 
为了使式中电磁场的公式更加对称,不妨加上一项
\begin{equation}
\frac{1}{\mu_0} (\div \bvec B)\bvec B = \bvec 0
\qquad (\div \bvec B = 0)
\end{equation} 
得
\begin{equation}
\bvec f = \epsilon_0 [(\div \bvec E)\bvec E + (\curl \bvec E) \cross \bvec E] + \frac{1}{\mu_0} [ (\div \bvec B)\bvec B + (\curl \bvec B) \cross \bvec B] - \epsilon_0 \mu_0 \pdv{\bvec s}{t}
\end{equation}  
一般来说,凡是出现两个连续的叉乘要尽量化成内积,下面计算 $(\curl \bvec E) \cross \bvec E$. 
由吉布斯算子(劈形算符)的相关公式
\begin{equation}
\grad (\bvec A \vdot \bvec B) = \bvec A \cross (\curl \bvec B) + \bvec B \cross (\curl \bvec A) + (\bvec A\vdot\bvec\nabla )\bvec B + (\bvec B\vdot\bvec\nabla)\bvec A
\end{equation} 
令 $\bvec A = \bvec B = \bvec E$,得
\begin{equation}
\grad (\bvec E^2) = 2\bvec E \cross (\curl \bvec E) + 2(\bvec E\vdot\bvec\nabla )\bvec E
\end{equation} 
即 $(\curl \bvec E) \cross \bvec E = (\bvec E\vdot\bvec\nabla)\bvec E - \grad (\bvec E^2)/2$
同理得
\begin{equation}
(\curl \bvec B) \cross \bvec B = (\bvec B\vdot\bvec\nabla )\bvec B - \frac12 \grad (\bvec B^2)
\end{equation} 
代入得
\begin{equation}\ali{
\bvec f = &\epsilon_0 [ (\curl E)\bvec E + (\bvec E\vdot\bvec\nabla)\bvec E - \frac12 \grad ( \bvec E^2) ]\\
&+ \frac{1}{\mu_0} [ (\div \bvec B)\bvec B + (\bvec B\vdot\bvec\nabla)\bvec B - \frac12 \grad (\bvec B^2) ] - \epsilon_0 \mu_0 \pdv{\bvec s}{t}
}\end{equation} 

与\autoref{EBP_eq1} 对比,可以看出动量流密度张量的散度为
\begin{equation}\ali{
\div \ten T =  &-\epsilon_0 [ ( \div \bvec E )\bvec E + (\bvec E\vdot\bvec\nabla )\bvec E - \frac12 \grad (\bvec E^2)]\\
&-\frac{1}{\mu_0} [(\div \bvec B)\bvec B + (\bvec B\vdot\bvec\nabla)\bvec B - \frac12 \grad (\bvec B^2)]
}\end{equation} 
接下来由二阶张量的散度计算公式,通过对比系数,就可以求出动量流密度张量 $\ten T$ (三阶矩阵).

下面把等式右边的部分用求和符号表示(求和符号是张量分析中最常见的符号,只有熟练运用才能学好张量分析).下面推导用到了克罗内克 $\delta$ 函数 %未完成:引用
,且定义任意矢量加上下标 表示第 个分量,例如
\begin{equation}
\bvec A_j = \leftgroup{A_x\quad (j = 1)\\ A_y\quad (j = 2)\\ A_z\quad (j = 3)} \qquad
x_j = \leftgroup{x\quad (j = 1)\\ y\quad (j = 2)\\ z\quad (j = 3)}
\end{equation} 
$(\div \bvec E)\bvec E + (\bvec E\vdot\bvec\nabla)\bvec E - \frac12 \grad (\bvec E^2)$ 是一个矢量,它的第 $j$ 个分量为
\begin{equation}\ali{
&\phantom{={}} [(\div \bvec E)\bvec E + (\bvec E\vdot\bvec\nabla)\bvec E - \frac12 \grad (\bvec E^2)]_j\\
&= \sum_i \pdv{E_i}{x_i} E_j + \sum_i E_i\pdv{E_j}{x_i} - \frac12 \sum_i \pdv{\bvec E^2}{x_i} \delta_{ij} \\
&= \sum_i \qty(\pdv{E_i}{x_i} E_j + {E_i} \pdv{E_j}{x_i} - \frac12 \pdv{\bvec E^2}{x_i} \delta_{ij}) \\
&= \sum_i \qty(\pdv{x_i} (E_i E_j) - \frac12 \pdv{\bvec E^2}{x_i}\delta_{ij} ) \\
&= \sum_i \pdv{x_i} \qty( E_i E_j - \frac12 \bvec E^2 \delta_{ij})
}\end{equation} 
同理
\begin{equation}
\qty[(\div \bvec B)\bvec B + (\bvec B\vdot\bvec\nabla)\bvec B - \frac12 \grad (\bvec B^2)]_j = \sum_i \pdv{x_i} \qty(B_i B_j - \frac12 \bvec B^2 \delta_{ij}) 
\end{equation} 
所以
\begin{equation}
(\div \ten T)_j = \sum_i \pdv{x_i} \qty[\epsilon_0 \qty(\frac12 \bvec E^2 \delta_{ij} - E_i E_j) + \frac{1}{\mu_0} \qty(\frac12 \bvec B^2 \delta_{ij} - B_i B_j)]
\end{equation} 
而由张量散度的定义
\begin{equation}
(\div \ten T)_j = \sum_i \pdv{x_i} T_{ij}
\end{equation} 
得到动量流密度张量为
\begin{equation}
T_{ij} = \epsilon_0 \qty(\frac12 \bvec E^2 \delta_{ij} - E_i E_j) + \frac{1}{\mu_0} \qty(\frac12 \bvec B^2 \delta_{ij} - B_i B_j)
\end{equation} 

理论上,在 $\ten T$ 上面加上任意一个满足 $\div \ten T'  = \bvec 0$ 的张量场,均可以使电磁场动量守恒,但是若规定无穷远处动量流密度为零,则可以证明 ${T'_{ij}} = 0$. 
 
 
 