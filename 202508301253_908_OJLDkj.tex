% 欧几里得空间(综述)
% license CCBYSA3
% type Wiki

本文根据 CC-BY-SA 协议转载翻译自维基百科\href{https://en.wikipedia.org/wiki/Euclidean_space}{相关文章}

\begin{figure}[ht]
\centering
\includegraphics[width=6cm]{./figures/24da64db04284f76.png}
\caption{} \label{fig_OJLDkj_1}
\end{figure}
欧几里得空间是几何学中的基本空间,用来表示物理空间。最初在欧几里得的《几何原本》中,它指的是三维欧几里得几何空间;但在现代数学中,欧几里得空间可以是任意正整数维的空间$n$,当需要明确维数时,称为$n$维欧几里得空间。当$n=1$或$n=2$ 时,通常分别称为欧几里得直线和欧几里得平面。\(^\text{[1]}\)“欧几里得”这一限定词用来区分欧几里得空间与后来的物理学和现代数学中研究的其他类型的空间。

古希腊几何学家引入欧几里得空间来模拟物理空间。他们的工作由古希腊数学家欧几里得汇编成《几何原本》\(^\text{[2]}\)。该书的一大创新是从少数基本性质(称为公设)出发,将空间的所有性质都作为定理加以证明。这些公设有的被认为是不言自明的(例如:“通过两点可以作且仅可以作一条直线”),有的则看似无法证明(例如平行公设)。
