% C 语言文件处理笔记
% license Usr
% type Note

\begin{itemize}
\item \verb`FILE *fopen(const char *filename, const char *mode);` 创建新文件,读取失败返回 \verb`NULL`。
\item \verb`int fgetc(FILE *stream)` 读取一个字符
\item \verb`int fgets(FILE *stream)` 读取一个字符
\item \verb`fscanf, fgets` 读取文件
\item \verb`size_t fwrite(const void *ptr, size_t size, size_t nmemb, FILE *stream)` 把字符串写入文件
\item \verb`fprintf` 写文件(具有类似 \verb`printf` 的格式化)
\item \verb`fseek(file, start_ind, SEEK_SET), rewind` 把文件指针移动到文件的某个位置。
\item \verb`long ftell(FILE *stream)` 获取文件指针
\item \verb`int fclose(FILE *stream)` 关闭文件(成功返回 0,否则返回 EOF)
\item \verb`"r"`: 只读 (文件必须存在).
\item \verb`"w"`: 只写 (新建或覆盖文件).
\item \verb`"a"`: 最后写 (不存在则新建).
\item \verb`"r+"`: 读写 (文件必须存在).
\item \verb`"w+"`: 读写 (新建或覆盖文件).
\item \verb`"a+"`: 最后写和读 (不存在则新建).
\item \verb`r` 模式的例子:
\begin{lstlisting}[language=cpp]
#include <stdio.h>
#include <stdlib.h>

int main() {
    FILE *file;
    char *buffer;
    // start index (will start from the 6th char)
    int start_ind = 5;
    // Length of the string to read
    int length = 10;

    // Open the file in read mode
    file = fopen("example.txt", "r");
    if (file == NULL) {
        perror("Error opening file");
        return 1;
    }

    // Allocate memory for the string
    buffer = (char *)malloc((length + 1) * sizeof(char));
    if (buffer == NULL) {
        perror("Memory allocation error");
        fclose(file);
        return 1;
    }

    // Move the file pointer to the start index
    fseek(file, start_ind, SEEK_SET);

    // Read the bytes from start_ind
    size_t bytesRead = fread(buffer, sizeof(char), length, file);

    // Null-terminate the buffer if any data was read
    if (bytesRead > 0)
        buffer[bytesRead] = '\0';

    // Check if we reached EOF
    if (feof(file))
        printf("End of file reached.\n");
    // Check for read errors
    else if (ferror(file))
        perror("Error reading file");

    // Print the string
    printf("Read string: %s\n", buffer);

    // Free the allocated memory and close the file
    free(buffer);
    fclose(file);

    return 0;
}
\end{lstlisting}
\item \verb`size_t fread(void *ptr, size_t size, size_t nmemb, FILE *stream)` 读取指定 \verb`size*nmemb` 字节的内容到 \verb`ptr`(不会在结尾写 \verb`\0`!需要手动添加)。 函数返回成功读取的元素个数,如果返回值小于 \verb`nmemb`,说明达到了文件末尾,或发生了错误。 读完后,当前位置会指向下一个未读的字符。
\item \verb`char *fgets(char *str, int n, FILE *stream)` 是 \verb`fread()` 的简化版,返回 \verb`\0` 结尾的字符串。
\item \verb`size_t fwrite(const void *ptr, size_t size, size_t nmemb, FILE *stream)`
\item \verb`w+` 模式的例子:
\begin{lstlisting}[language=cpp]
#include <stdio.h>
#include <stdlib.h>

int main() {
    FILE *file;
    
    // Open the file in "w+" mode
    // (write and read, truncate if exists)
    file = fopen("example.txt", "wb+");
    if (file == NULL) {
        printf("Error opening file!\n");
        return 1;
    }

    // Step 1: Write to the file
    fputs("Hello, World!\n", file);
    fputs("This is a test file.\n", file);
    fputs("End of writing.\n", file);

    // Step 2: Jump to the beginning of the file using fseek()
    fseek(file, 0, SEEK_SET);

    // Step 3: Read from the file and print its
    // contents to the console
    char buffer[100];
    printf("Reading the file contents:\n");
    while (fgets(buffer, sizeof(buffer), file) != NULL) {
        printf("%s", buffer);
    }

    // Step 4: Move the file pointer to a specific location, modify contents
    fseek(file, 7, SEEK_SET); // Jump to the 8th character
    fputs("Universe", file);  // Overwrite "World" with "Universe"

    // Step 5: Jump to the beginning again to read the updated content
    fseek(file, 0, SEEK_SET);
    printf("\nUpdated file contents:\n");
    while (fgets(buffer, sizeof(buffer), file) != NULL) {
        printf("%s", buffer);
    }

    // Close the file
    fclose(file);
    return 0;
}
\end{lstlisting}
\item \verb`int fseek(FILE *stream, long offset, int whence);` 可以跳到文件的某个位置。 例如 \verb`fseek(file, 7, SEEK_SET);` 相对于文件开头, \verb`fseek(file, -7, SEEK_CUR);` 相对当前位置, \verb`SEEK_END` 相对于文件 EOF。 如果成功,\verb`fseek()` 返回 0。如果返回非 0 则失败(例如超出范围)。
\item \verb`offset` 可以叫做\textbf{文件位置指示器 file position indicator},也称为\textbf{文件指针}, 但 \verb`FILE *` 也叫做文件指针, 需要根据语境区分!
\item \verb`void perror(const char *str)` 打印错误信息到 \verb`stderr` 并根据全局变量 \verb`errno` 在后面附上 \verb`: 具体错误原因(如文件不存在,权限不够)`。
\item \verb`int fgetc(file)` 获取一个字节。 如果因为超出文件结尾没有获取到,则返回 EOF。
\item \verb`EOF` 是 \verb`stdio.h` 中的一个宏,通常是 -1 但不能保证。
\item 例如要获取文件最后的三个字节,用 \verb`fseek(file, -3, SEEK_END)` 然后 \verb`fgetc(file)` 三次即可。
\item ASCII 表不存在 EOF,EOF 是一个概念不是一个字符。
\item \verb`long ftell(FILE *stream)` 可以查看当前位置,会返回 \verb`fseek(file, 位置, SEEK_SET)` 设置的位置。
\end{itemize}

