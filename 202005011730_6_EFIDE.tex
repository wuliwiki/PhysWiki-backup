% 介质中的静电场

\pentry{电极化强度与极化电荷的关系\upref{ElePAP}}

如果把激发外电场的原有电荷系称为自由电荷,并用$\mathbf E_0$表示它们所激发的电场强度,而用$\mathbf E'$表示极化过程完成之后极化电荷所激发的电场强度.那么,空间任一点最终的合电场强度$\mathbf E $应是上述两类电荷所激发电场强度的矢量和,即
\begin{equation}
\mathbf E=\mathbf E_{0}+\mathbf E^{\prime}
\end{equation}
由于在电介质中,自由电荷的电场与极化电荷的电场的方向总是相反,所以在电介质中的合电场强度$\mathbf E $与外电场强度$\mathbf E_0$相比显著地削弱了.

对于各向同性线性电介质,电极化强度$\mathbf P $和介质内部的合电场强度$\mathbf E $的关系为
\begin{equation} 
\mathbf P=\chi_{\mathrm e} \varepsilon_{0} \mathbf E
\end{equation}
式中的比例因数$\chi_{\mathrm{e}}$和电介质的性质有关,叫做电介质的\textbf{电极化率(electric susceptibility)},是量纲为$1 $的量.

\begin{figure}[ht]
\centering
\includegraphics[width=6cm]{./figures/EFIDE_1.pdf}
\caption{电介质中的电场强度} \label{EFIDE_fig1}
\end{figure}
为了定量地了解电介质内部电场强度被削弱的情况,我们讨论如下特例:\autoref{EFIDE_fig1}表示在两块“无限大“极板间充有电极化率为儿的均匀电介质,设两极
板上的自由电荷面密度为切-0' 电介质表面上的极化电荷面密度为土er' 自由电