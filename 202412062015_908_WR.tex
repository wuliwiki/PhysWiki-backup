% 赫尔曼·外尔(综述)
% license CCBYSA3
% type Wiki

本文根据 CC-BY-SA 协议转载翻译自维基百科\href{https://en.wikipedia.org/wiki/Hermann_Weyl#Weyl_equation}{相关文章}。

\begin{figure}[ht]
\centering
\includegraphics[width=6cm]{./figures/b6bfe45a42d757ab.png}
\caption{赫尔曼·克劳斯·雨果·外尔  1885年11月9日  德国帝国,埃尔姆斯霍恩} \label{fig_WR_1}
\end{figure}
赫尔曼·克劳斯·雨果·外尔(Hermann Klaus Hugo Weyl,ForMemRS,德语发音:[vaɪl];1885年11月9日-1955年12月8日)是一位德国数学家、理论物理学家、逻辑学家和哲学家。尽管他的大部分职业生涯是在瑞士苏黎世和美国新泽西州普林斯顿度过的,他依然被认为是哥廷根大学数学传统的一部分,该传统由卡尔·弗里德里希·高斯、大卫·希尔伯特和赫尔曼·闵可夫斯基代表。

外尔的研究在理论物理学和纯数学领域(例如数论)具有重要意义。他是20世纪最具影响力的数学家之一,也是早期普林斯顿高等研究院的重要成员。[4][5]

外尔在空间、时间、物质、哲学、逻辑、对称性以及数学史等领域作出了非凡的贡献。他是最早设想将广义相对论与电磁学定律结合起来的人之一。弗里曼·戴森(Freeman Dyson)曾写道,外尔是唯一一个可以与19世纪“最后的伟大通才数学家”亨利·庞加莱和大卫·希尔伯特相提并论的人。迈克尔·阿蒂亚(Michael Atiyah)特别指出,每当他研究一个数学主题时,总会发现外尔早已涉足其中。[7]

