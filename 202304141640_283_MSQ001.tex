% MySQL安装
% 数据库|MySQL

\subsection{什么是MySQL}

MySQL是一个开源的关系型数据库管理系统(RDBMS),采用客户-服务器模式,它被用来创建和管理基于关系模型的数据库。同时它也是世界上最流行的开放源码数据库,它为许多访问量最大的应用程序提供动力,包括Facebook、Twitter、Netflix、Uber、Airbnb、Shopify和Booking.com等等。

\subsubsection{相关术语}

\begin{figure}[ht]
\centering
\includegraphics[width=10cm]{./figures/f862111f2d629142.png}
\caption{关系型数据库示例} \label{fig_MSQ001_2}
\end{figure}
\begin{itemize}
\item 数据库(Database):一种用于存储和管理数据的软件系统,它可以将数据组织成不同的数据表。
\item 数据表(Table):一种二维的数据结构,它由行(Row)和列(Column)组成。行表示数据表中的一条记录,列表示数据表中的一个字段或属性。
\item 属性(Attribute):数据表中的一个特征或特性,它可以描述或区分数据表中的记录。例如,学生信息表可以包含学号,姓名,性别,年龄等属性,每一行代表一个学生的信息,每一列代表一个属性的值。
\item 表头(Header):数据表中第一行,它显示了每一列的属性名称。
\item 值(Value):数据表中除了表头以外的每个单元格中的内容,它表示了某个实体或对象在某个属性上的具体信息。
\item 键(Key):数据表中用于标识或关联记录的一种特殊的属性,它可以分为主键和外键。
\item 主键(Primary key): 一个列或一个列的组合,唯一地标识了表中的每一行。
\item 外键(Foreign key): 引用另一个表的主键的列或列的组合,以建立表之间的关系。
\item 索引(Index): 一种数据结构,通过创建指向行的指针来提高表的数据检索操作的速度。
\item 查询(Query): 根据某些标准和条件,从一个或多个表中请求数据的语句。
\item SQL: 结构化查询语言,用于定义、操作和查询RDBMS中的数据的标准语言。
\item 规范化(Normalization): 组织数据库中的数据的过程,以减少冗余和提高数据的完整性。
\item 连接(Join): 基于一个共同的列或条件,将两个或更多的表的数据结合起来的一种操作。
\item 视图(View): 一个显示查询结果的虚拟表,不需要实际存储数据。
\item 事务(Transaction): 一个逻辑上的工作单位,包括对数据库的一个或多个操作,如插入、更新、删除或选择。一个事务必须作为一个整体被提交或回滚,以确保数据的一致性和可靠性。
\end{itemize}

\subsection{MySQL的安装}

\subsubsection{下载}
官网:\href{https://www.mysql.com/downloads/}{MySQL :: MySQL Downloads}
进入官网后选择社区版下载(免费)。
\begin{figure}[ht]
\centering
\includegraphics[width=14cm]{./figures/3e67cee5aaef58d3.png}
\caption{下载主界面} \label{fig_MSQ001_3}
\end{figure}

\begin{figure}[ht]
\centering
\includegraphics[width=14cm]{./figures/4957d551da9f6610.png}
\caption{选择社区版下载} \label{fig_MSQ001_4}
\end{figure}
找到自己系统对应的版本选择一个安装包下载。
\begin{figure}[ht]
\centering
\includegraphics[width=14cm]{./figures/726b25d6d56b1802.png}
\caption{找到自己系统对应的版本点击下载} \label{fig_MSQ001_5}
\end{figure}

\begin{figure}[ht]
\centering
\includegraphics[width=14cm]{./figures/e902903c3bce3e93.png}
\caption{选择一个安装包进行下载} \label{fig_MSQ001_6}
\end{figure}

\begin{figure}[ht]
\centering
\includegraphics[width=14cm]{./figures/499e3bf505e453a0.png}
\caption{点击此处直接下载} \label{fig_MSQ001_7}
\end{figure}

\subsubsection{安装}
由于本机在此之前已经完成过安装,因此在这里展示customer安装方式。当然也可选择Full进行全量安装。
\begin{figure}[ht]
\centering
\includegraphics[width=14cm]{./figures/19e06b51783780d9.png}
\caption{选择customer方式安装自己想要的扩展} \label{fig_MSQ001_8}
\end{figure}

接下来选择安装 MySQL Server,MySQL Workbench以及其他自己想要的products。
\begin{figure}[ht]
\centering
\includegraphics[width=14cm]{./figures/5e3ac3b2cd46032e.png}
\caption{安装MySQL Server以及MySQL Workbench} \label{fig_MSQ001_9}
\end{figure}

选择存储路径(默认会安装在C盘,建议修改)
\begin{figure}[ht]
\centering
\includegraphics[width=14cm]{./figures/a39ebc37e19cfb23.png}
\caption{安装路径修改} \label{fig_MSQ001_10}
\end{figure}

\begin{figure}[ht]
\centering
\includegraphics[width=14cm]{./figures/9e2fba4d097e4015.png}
\caption{点击Execute执行安装} \label{fig_MSQ001_11}
\end{figure}

接下来对server进行配置,第一部分保持默认状态即可,无需修改
\begin{figure}[ht]
\centering
\includegraphics[width=14cm]{./figures/5a46b200641563d4.png}
\caption{此部分无需修改} \label{fig_MSQ001_12}
\end{figure}

这一部分需要选择第二种身份认证方式,因为caching_sha2_password基本不受第三方客户端支持,当然如果误选了在安装完成后也可以修改。
