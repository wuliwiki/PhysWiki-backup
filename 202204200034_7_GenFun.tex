% 测度与广义函数
% Lebesgue测度|广义函数|绝对连续|Lebesgue积分|狄拉克函数|狄拉克测度|Dirac函数|Dirac测度

\pentry{Lebesgue 积分\upref{Lebes1},映射\upref{map}}

在\textbf{集合的测度(实变函数)}\upref{SetMet}中,我们了解了Lebesgue外测度.它是将“开集的体积”进行推广而得来的,其定义思路也有明显的物理对应:用容器去衡量被容纳物的大小.具象的概念虽然容易想象,但因其具象也束缚了概念范围,所以我们在得到一个新的概念时,总会想把它的核心特征抽离出来,抽象出一个更一般的概念,看看能不能引申出有趣的理论.

Lebesgue外测度该怎么抽象呢?抛去其定义的方式不谈,它就是用来衡量“集合的体积”的,对不对?体积是一个数字,那我们就把测度看成是给各集合赋予一个数字,也就是“集合函数”\footnote{注意这个术语的意思:集合函数是指把集合映射到数字上的映射.值域是数字的映射,通常又称为函数.}.再考虑一些集合体积所具有的性质,我们可以构造出这样一个定义:

\begin{definition}{非负测度}\label{GenFun_def1}
设$S$是一个集合,$\mathcal{A}$是由$S$的子集构成的一个$\sigma$-代数\footnote{即用$\mathcal{A}$中元素进行任意多次的交、并、差、补等运算,结果仍在$\mathcal{A}$中.}.称映射$\mu:\mathcal{A}\to [0, +\infty]$是$(S, \mathcal{A})$上的一个\textbf{非负测度(non-negative measure)},如果它满足:
\begin{enumerate}
\item $\mu(\varnothing)=0$;\\
\item 对于两两不交的至多可数个$A_i\in\mathcal{A}$,有
\begin{equation}
\mu\qty(\bigcup_i A_i) = \sum_i \mu(A_i)
\end{equation}
\end{enumerate}

通常将\textbf{非负测度}简称为\textbf{测度(measure)}.

三元组$(S, \mathcal{A}, \mu)$也被称为一个\textbf{测度空间(measure space)}.

\end{definition}

注意,定义中测度$\mu$的值域是$[0, +\infty]$而非$[0, +\infty)$,意味着测度值也可以取广义实数$+\infty$.由于测度值非负,因此在不至于混淆的时候,也可以用$\infty$代替$+\infty$.

有非负测度,意味着测度的概念还可以继续推广:

\begin{definition}{复值测度}
将非负测度的定义域从$[0, +\infty]$改为$\mathbb{C}$,即得到\textbf{复值测度}.
\end{definition}

\begin{example}{计数测度}
设$S$为任意集合,定义$(S, 2^S)$上的一个测度$\mu$如下:
\begin{equation}
\mu(A)=
\leftgroup{
    \abs{A},\quad \abs{A}<\aleph_0\\
    +\infty,\quad \abs{A}\geq\aleph_0
}
\end{equation}
称其为$(S, 2^S)$上的\textbf{计数测度}.
\end{example}

\autoref{GenFun_def1} 比Lebesgue测度的定义抽象多了,因此有必要举一个和Lebesgue测度截然不同的例子,来体现其“更一般更抽象”的特点:

\begin{example}{Dirac测度}
令$\mathcal{B}$为$\mathbb{R}^n$中的Borel集合\footnote{即用开区间进行任意多次交、并、差、补等运算获得的集族.}.取$\mathbb{R}^n$中的一点$x_0$,定义$(\mathbb{R}^n, \mathcal{B})$上的测度$\sigma_{x_0}$为:
\begin{equation}
\sigma_{x_0}(A)=\leftgroup{
    1, \quad x_0\in A\\
    0, \quad x_0\not\in A
}
\end{equation}
称之为\textbf{Dirac测度}.
\end{example}

Dirac测度显然连平移不变性都不具备,和Lebesgue外测度截然不同.你可能也注意到了,Dirac测度和Dirac函数看起来很相似——没错,Dirac函数本质上就不是一个函数,而是一个测度.但这话乍一听很奇怪,函数是给集合上每个点赋予一个数字,测度是给各子集赋予数字,两者是怎么联系起来的呢?这就需要利用Lebesgue外测度作为脚手架,来衔接二者了.

\begin{definition}{测度的绝对连续}
设$\mu$和$\nu$都是$(S, \mathcal{A})$上的测度,且只要$\nu(A)=0$,就必有$\mu(0)$,那么称测度$\mu$关于测度$\nu$绝对连续,记为$\mu\ll$
\end{definition}

\addTODO{需补充或引用\textbf{Radon-Nikodym定理}相关词条.}



















