% 分圆域
% 分圆域|cyclotomy|域扩张|单位根|本原根|原根|primitive element

\addTODO{预备知识待确定}



$n$次\textbf{单位根}即形如$x^n-1$的多项式在$\mathbb{C}$中的根,可以理解为$1$的$n$次根.$\pm 1$都是$1$的$2^n$次根,$\pm \I$都是$1$的$4$次单位根,而$\omega=\frac{1}{2}\qty(-1+\I\sqrt{3})$是$1$的$3$次根.

本节将使用Galois理论来处理单位根及其最小多项式对有理数域的扩张,作为Galois理论的应用.


\subsection{分圆多项式}

任意$n$次单位根都形如$\exp{(2\pi\I\cdot  k/n)}$,其中$k$取从$1$到$k$的全体整数,即包括了所有$n$次单位根.如$4$次单位根构成的集合是$\{1, -1, \I, -\I\}$,而这四个元素又可以两两分组:$\pm 1$都可以是$2$次单位根\footnote{更细节些,$1$也是$1$次单位根.},$\pm \I$是$4$次单位根.显然两组性质是不同的:$\mathbb{Q}(\pm 1)=\mathbb{Q}\subsetneq \mathbb{Q}(\pm\I)$.我们将这分类表述为以下定义:

\begin{definition}{本原单位根}

当$n$与$k$互素时,称单位根$\exp{(2\pi\I\cdot  k/n)}$是\textbf{本原(primitive)}的.

\end{definition}

由互素的概念易知,$n$次\textbf{本原单位根}的全体幂,包含了全体$n$次\textbf{单位根},即$n$次本原单位根至少要自乘$n$次才能变成$1$;而非本原的单位根则不然,如$3$次单位根自乘$3$次即可得到$1$,但它同时也是$6$次单位根.因此由分裂域的知识可知,非本原的单位根,其最小多项式不可能是$x^n-1$.为此,我们需要明确本原根所属的最小多项式.

\begin{definition}{分圆多项式}

设$\{\varepsilon_i\}_{i\in S}$是全体$n$次\textbf{本原单位根}的集合,则称
\begin{equation}
\Phi_n(x)\in \mathbb{C}[x] = \prod_{i\in S}(x-\varepsilon_i)
\end{equation}
为$n$次\textbf{分圆多项式(cyclotomic polynomials)}.

\end{definition}

取“分圆”之名,是指其将单位圆分割.英文cyclotomy取自“cycl-”和“-tomy”的组合,cycl-指圆形的、环形的、旋转的,而-tomy源自古希腊语“tomia”或其变形“tomnein”




\begin{example}{分圆多项式的例子}\label{Cycltm_ex1}

\begin{equation}
\Phi_1(x) = x-1
\end{equation}

\begin{equation}\label{Cycltm_eq2}
\Phi_2(x) = x+1
\end{equation}

\begin{equation}
\Phi_3(x) = (x^3-1)/(x-1) = x^2+x+1
\end{equation}

\begin{equation}
\Phi_4(x) = (x^4-1)/(x^2-1) = x^2+1
\end{equation}

\begin{equation}
\Phi_5(x) = (x^5-1)/(x-1) = x^4+x^3+x^2+x+1
\end{equation}

% 注意后三个例子里,都除以了一个多项式.为什么这么做?

\end{example}

\begin{exercise}{}
请尝试算几个\autoref{Cycltm_ex1} 中未列出的分圆多项式的例子.
\end{exercise}

下面列出分圆多项式的几个基本性质.

\begin{theorem}{}
任取正整数$n\geq 2, q\geq 2$,则$\Phi_n(x)\in\mathbb{R}[x]$,且$\Phi_n(q)>q-1$.
\end{theorem}

\textbf{证明}:

首先证明$\Phi_n(x)\in\mathbb{R}[x]$.

任取小于$n$的正整数$k$,则$k$与$n$互素当且仅当$n-k$与$n$互素.因此,$\exp{(2\pi\I\cdot  k/n)}$是本原的,当且仅当$\exp{(2\pi\I\cdot  (n-k)/n)}$是本原的.所以本原元素$\varepsilon_k$的共轭$\varepsilon_{n-k}$还是本原元素,从而
\begin{equation}\label{Cycltm_eq1}
(x-\varepsilon_k)(x-\varepsilon_{n-k})=x^2-(\varepsilon_k+\varepsilon_{n-k})x+\varepsilon_n\varepsilon_{n-k}
\end{equation}
是\textbf{实系数}多项式.于是分圆多项式是由若干形如\autoref{Cycltm_eq1} 或者形如$x+1$\footnote{即本原单位元的共轭就是自身的情况,而该情况只有\autoref{Cycltm_eq2} 一种,所以$n>3$的情况都是本原元素成对出现的.}的多项式相乘而得,从而$\Phi_n(x)\in\mathbb{R}[x]$.

由于$q\geq 2$,且各$\Phi_n$的本原根$\varepsilon_k$满足$\abs{\varepsilon_k}=1$,$\varepsilon_k\neq 1$.因此$\abs{q-\varepsilon_k}>\abs{q-1}=q-1$,从而得$\Phi_k(q)>q-1$.


\textbf{证毕}.




\begin{theorem}{}
\begin{equation}
x^n-1 = \prod_{d\mid n}\Phi_d(x)
\end{equation}

等价表述:不同的次的\textbf{本原单位根}之间无交集,且全体满足$d\mid n$的$d$次\textbf{本原单位根}构成的集合,恰为全体$n$次\textbf{单位根}之集合.
\end{theorem}

\textbf{证明}:

\textbf{一个}$d_i$次\textbf{本原单位根}的幂遍历所有$d_i$次单位根,因此“一个$d_1$的本原单位根也是$d_2$的本原单位根”,当且仅当“$d_1=d_2$”,当且仅当“$d_1$和$d_2$的本原单位根完全一样”.于是得证“不同的次的\textbf{本原单位根}之间无交集”.

任取一个$n$次单位根$\varepsilon_k=\exp{(2\pi\I\cdot  k/n)}$,取$d=\opn{gcd}(n, k)$\footnote{即$n$与$k$的最大公因子(greatest common divisor).},则$\varepsilon_k$是一个$d$次本原单位根.从而任意$n$次单位根都是某个$d$次本原单位根,满足$d\mid n$,结合上一段的结论得证定理.

\textbf{证毕}.



\begin{theorem}{}
$\Phi_n(x)$是首一整系数多项式.
\end{theorem}

\textbf{证明}:



\textbf{证毕}.

























