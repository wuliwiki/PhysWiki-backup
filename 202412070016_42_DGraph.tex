% 图的顶点度
% keys 顶点度
% license Usr
% type Tutor

\pentry{图\nref{nod_Graph}}{nod_2e68}

图的某一顶点的度是指与它关联的边数。由于图论的早期主要研究的是无向图,因此顶点度的概念往往专门用于无向图。然而,为了一般化起见我们将它定义在任一图上。

\begin{definition}{顶点度,出度,入度,孤立点}
设 $G=(V,E,\varphi)$ 是图,$v\in V$。则 $G$ 中与 $v$ \aref{关联}{def_Graph_1}的边数称为 $v$ 在 $G$ 中的\textbf{度}(degree),记作 $d_G(v)$。$G$ 中以 $v$ 为起点的有向边数称为 $v$ 在 $G$ 中的\textbf{出度}(out degree),记作 $d_G^+(v)$。$G$ 中以 $d$ 为终点的有向边数称为 $v$ 在 $G$ 中的\textbf{入度}(in degree),记作 $d_G^-(v)$。度为 $d$ 的点称为 $d$ 度点(d degree vertex)。零度点称为\textbf{孤立点}(isolated vertex)。
\end{definition}

\begin{example}{}
设 $G$ 是有向图。试证明 $d_G(v)=d_G^+(v)+d_G^-(v)-r$,其中 $r$ 是过 $v$ 的环数。

\textbf{证明:}设 $E^+$ 是 $G$ 中以 $v$ 为起点的边集,$E^-$ 是 $G$ 中以 $v$ 为终点的边集。则 $E_0=E^+\cap E^-$ 是过 $v$ 的环集。由于
$E^+\backslash E_0, E_0, E^-\backslash E_0$ 是 $E(v)$ 的划分,因此
\begin{equation}
\abs{E(v)}=\abs{E^+\backslash E_0}+\abs{E_0}+\abs{E^-\backslash E_0}.~
\end{equation}
又 $\abs{E(v)}=d_G(v),\abs{E^+\backslash E_0}=d_G^+(v)-r, \abs{E_0}=r, \abs{E^-\backslash E_0}=d_G^-(v)-r$,所以
\begin{equation}
d_G(v)=d_G^+(v)+d_G^-(v)-r.~
\end{equation}
\end{example}
图的所有点当中最大的度和最小的度人们习惯用记号 $\Delta,\delta$ 分别来标记它。
\begin{definition}{最大度,最小度,平均度}
设 $G$ 是图。则称 $G$ 中顶点度的最大者称为 $G$ 的\textbf{最大度}(maximum degree),记作 $\Delta(G)$;而 $G$ 中顶点度的最小者称为\textbf{最小度}(minimum degree),记作 $\delta(G)$;所有点的度之和除于点数称为\textbf{平均度}(average degree),记作 $d(G)$。即
\begin{equation}
\begin{aligned}
&\Delta(G):=\max\{d_G(v)|v\in V\}, \\
&\delta(G):=\min\{d_G(v)|v\in V\},\\
&d(G):=\frac{1}{\abs{V(G)}}\sum_{v\in V}d(v).
\end{aligned}~
\end{equation}
\end{definition}

\begin{exercise}{}
试证明不等式:$\delta(G)\leq d(G)\leq \Delta(G)$。
\end{exercise}


当所有点具有相同的度时,人们习惯用正则这个词来形容图。
\begin{definition}{正则}
若图 $G$ 中的所有点都具有相同的度 $k$,则称 $G$ 是 \textbf{$k$ 正则的}($k$ regular),简称\textbf{正则的}(regular)。特别的,$3$ 正则图称为 \textbf{立方图}(cubic graph)。
\end{definition}

\begin{theorem}{}
设 $G$ 试图,则其边集和点集个数具有如下关系:
\begin{equation}
\abs{E(G)}=\frac{1}{2}d(G)\abs{V(G)}.~
\end{equation}
\end{theorem}
\textbf{证明:}方法1:对所有的顶点度求和,由于一边有两个端点,计算一个点的度计数一次边,所以每条边刚好被计算两次。因此
\begin{equation}\label{eq_DGraph_1}
\abs{E}=\frac{1}{2}\sum_{v\in V}d(v)=\frac{1}{2}d(G)\abs{V}.~
\end{equation}

\textbf{方法2:}方法1具有较大的“人为”部分,或者说形式上不严谨,相当于用了一推所谓的人能思考的逻辑,直接给出了\autoref{eq_DGraph_1}。为了形式严密起见,下面使用更精细的证明。

考虑边及其端点组成的偶对 $(e,v)$ 的全体
\begin{equation}
A=\{(e,v)|v\in e,e\in E(G)\}.~
\end{equation}
对每一边 $e$ 定义
\begin{equation}
\tilde e=\{(e,v)|v\in e,v\in V(G)\}.~
\end{equation}
对每一点 $v$ 定义
\begin{equation}
\tilde v=\{(e,v)|v\in e,e\in E(G)\}.~
\end{equation}
由于 
\begin{equation}
\bigcup_{e\in E}\tilde e=A=\bigcup_{v\in V}\tilde v,~
\end{equation}
且不同的 $e_1,e_2\in E(G),v_1,v_2\in V(G)$,$\tilde e_1\cap\tilde e_2=\varnothing,\tilde v_1\cap\tilde v_2=\varnothing$
所以(不相交集合的基数和等于它们的并集的基数)
\begin{equation}
\sum_{e\in E}\abs{\tilde e}=\sum_{v\in V}\abs{\tilde v}.~
\end{equation}
将 $\abs{\tilde e}=2,\forall e\in E,\abs{\tilde v}=d_G(v)$ 带入上式即得\autoref{eq_DGraph_1} 。

\textbf{证毕!}















