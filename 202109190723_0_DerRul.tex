% 求导法则(简明微积分)
% 微积分|导数|求导法则|线性运算

\pentry{基本初等函数的导数\upref{FunDer}}

如果需要求导的函数可以看做若干个已知导函数的函数(如基本初等函数)经过四则运算或复合得到的, 那么我们可以直接使用一系列求导法则对其求导

\subsubsection{四则运算}
\begin{equation}\label{DerRul_eq5}
[ f(x) \pm g(x) ]' = f'(x) \pm g'(x)
\end{equation}
\begin{equation}\label{DerRul_eq6}
[ f(x)g(x) ]' = f'(x)g(x) + f(x)g'(x) 
\end{equation}
\begin{equation}\label{DerRul_eq3}
\qty[ \frac{f(x)}{g(x)} ]'  = \frac{f'(x)g(x) - g'(x)f(x)}{g(x)^2}
\end{equation}

\subsubsection{复合函数}
\begin{equation}\label{DerRul_eq4}
f[g(x)]' = f'[g(x)]g'(x)
\end{equation}
详见“一元复合函数求导(链式法则)\upref{ChainR}”

\subsubsection{反函数}
\begin{equation}
[f^{-1}(x)]' = \frac{1}{f'[f^{-1}(x)]} 
\end{equation}
详见“反函数求导\upref{InvDer}”.

\subsection{线性}
我们先来证明\autoref{DerRul_eq5}. 对求导而言, \textbf{线性}是指若干\textbf{函数线性组合}(即把若干个函数分别乘以常数再相加)\footnote{线性组合是线性代数中的概念, 线性代数中, 线性空间都可以做线性组合, 例如几何矢量(\autoref{GVecOp_eq1}~\upref{GVecOp}). 特定函数的集合可以看作线性空间(\autoref{LSpace_ex2}~\upref{LSpace}).}的求导等于对这些函数先分别求导再进行同样的线性组合. 由于函数加减法属于函数线性组合的两种简单情况, 这里只需要证明求导是一种\textbf{线性运算} 即可.  令若干常数为 $c_i$, 若干可导函数为 $f_i(x)$, 根据导数的定义, 这些函数线性组合的导数为
\begin{equation}\label{DerRul_eq1}\ali{
\dv{x}\sum_i c_i f_i(x) &= \lim_{h\to 0} \qty[\sum_i c_i f_i(x+h) - \sum_i c_i f_i(x) ]/h\\
& =  \sum_i c_i \lim_{h\to 0} [f_i(x+h) - f_i(x)]/h\\
& = \sum_i c_i f_i'(x)
}\end{equation}
证毕.

\begin{example}{对函数 $f(x) = 5\sin x + 3x^2$ 求导}
这里的 $f(x)$ 可以看做三角函数 $\sin x$ 函数和幂函数 $x^2$ 的线性组合, 二者都是基本初等函数, 导数分别为 $\cos x$ 和 $2x$, 由于求导是线性运算, 我们只需要对两个函数各自的导函数进行同样的线性组合即可
\begin{equation}
f'(x) = 5 \sin' x + 3(x^2)' = 5 \cos x + 3(2x) = 5\cos x + 6x
\end{equation}
\end{example}

\subsection{乘积法则}
现在证明\autoref{DerRul_eq6}, 令两函数分别为 $f(x)$ 和 $g(x)$, 现在求 $f(x) g(x)$ 的导函数. 由导数的定义\upref{Der}得
\begin{equation}\label{DerRul_eq2}
[f(x)g(x)]' = \lim_{h\to 0} \frac{f(x+h)g(x+h) - f(x)g(x)}{h}
\end{equation}

\begin{figure}[ht]
\centering
\includegraphics[width=6.0cm]{./figures/DerRul_1.pdf}
\caption{乘积法则的几何理解} \label{DerRul_fig1}
\end{figure}

从几何上来看(\autoref{DerRul_fig1} ), 我们可以把 $f(x), g(x)$ 看做一个矩形的两条边长, 它们的乘积 $f(x)g(x)$ 就是矩形的面积, 而 $f(x+h)g(x+h)$ 不妨看做是另一个矩形的面积. 当 $x$ 增加 $h$ 后, 令 $f$ 和 $g$ 分别增加 $\Delta f$ 和 $\Delta g$ (图中假设他们大于零,其他情况同理), 那么矩形面积的增量可以分解为三部分
\begin{equation}
f(x+h)g(x+h) - f(x)g(x) = \Delta f g + f \Delta g + \Delta f\Delta g
\end{equation}
代入\autoref{DerRul_eq2} 得
\begin{equation}
[f(x)g(x)]' = \lim_{h\to 0} \frac{\Delta f}{h} g + f \lim_{h\to 0} \frac{\Delta g}{h} + \lim_{h\to 0}\frac{\Delta f\Delta g}{h}
\end{equation}
容易看出前两项中的两个极限分别是 $f,g$ 的导数, 而第三个极限中, 两个增量的乘积变小的速度远比前两个极限中要快(我们把它叫做二阶无穷小), 所以第三个极限为零. 从几何上来看, 这是因为随着 $h$ 变小, 右上角小矩形的面积 $\Delta f\Delta g$ 比起两条长矩形的面积可以忽略不计.

\subsection{商法则}
要证明\autoref{DerRul_eq3}, 我们可以把 $f(x)/g(x)$ 看成 $f(x)$ 乘以 $1/g(x)$, 这样就可以直接使用乘积法则了. 但是如何对 $1/g(x)$ 求导呢? 我们可以将其看做 $h(x) = 1/x$ 和 $g(x)$ 的复合函数 $h[g(x)]$. 其中 $h'(x) = -1/x^2$(\autoref{FunDer_eq3}~\upref{FunDer}). 根据\autoref{DerRul_eq4}, 有
\begin{equation}
\qty[\frac{1}{g(x)}]' = h'[g(x)]g'(x) = -\frac{g'(x)}{g^2(x)}
\end{equation}
最后使用乘积法则
\begin{equation}
\begin{aligned}
\qty[f(x) \frac{1}{g(x)}]' &= f'(x) \frac{1}{g(x)} - f(x) \frac{g'(x)}{g^2(x)}\\
&= \qty[ \frac{f(x)}{g(x)} ]'  = \frac{f'(x)g(x) - g'(x)f(x)}{g(x)^2}
\end{aligned}
\end{equation}
证毕.
