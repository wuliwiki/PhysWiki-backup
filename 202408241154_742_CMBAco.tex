% CMB 声学峰
% license Usr
% type Tutor

在这一部分,我们讨论宇宙微波背景(CMB)如何携带暗物质存在的印记。主要的可观测性是CMB功率谱:通过执行光子温度场(同一图的顶部左侧的各向异性)的球面傅里叶变换,然后通过平均每个角动量的$2l+1$个方向来计算方差。这个方差被测量并与膨胀宇宙大爆炸宇宙学的预测进行比较:这种比较允许我们推断宇宙的内容,包括暗物质的存在。

更精确地说,CMB峰值是由于重子/光子流体的声学振荡。这些声学峰的位置取决于暗物质密度(较少的暗物质意味着辐射-物质平等较晚),它们的幅度取决于相对于普通物质的暗物质的相对数量(只有后者经历声学振荡)。全局拟合发现,这些和其他宇宙学可观测性可以通过包括暗物质的标准宇宙学模型很好地再现(具有具有高斯初始扰动的$\Lambda CDM$模型),并允许确定其宇宙学参数的值。这提供了目前可用的暗物质密度的最准确确定。

下面我们扩展讨论并半定量地展示暗物质如何影响CMB功率谱的形状。我们仅关注主要与暗物质相关的特征; 

CMB功率谱与物质功率谱$P(k)$具有相同的直观含义,但现在是针对光子各向异性的。大致上,$C_l$是大约为$\theta \sim \pi /l$的各向异性量。角尺度$\theta$大致对应于波数$k \sim l H_0$,其中$H_0$是当前的哈勃常数。CMB功率谱随着$l$的增加而下降,这是由于“Silk阻尼”,即小尺度结构由于光子扩散而平滑。然而,第二和第三CMB峰值大致相等。这表明,如果能够去除Silk阻尼,第三个峰值将特别突出,高于第二个峰值。这就是暗物质的足迹,正如我们接下来所说明的。

在宇宙中特定位置$r$和时间$t$的光子流体中的不均匀性由函数$\Theta(r, t, \hat{\mathbf{p}}) \equiv \delta T/T$表示,或者等价地,由其傅里叶变换$\Theta(k, t, \hat{\mathbf{p}})$表示。这里$T$是光子温度,$\hat{\mathbf{p}}$是光子方向。这个函数服从以下玻尔兹曼演化方程:
\begin{equation}\label{eq_CMBAco_1} 
\dot{\Theta} - i \frac{k}{a}\mu  \Theta + \dot{\Psi} - i \frac{k}{a} \mu  \Phi = - \dot{\tau}[\Theta_0 - \Theta + \mu v_{b k}]~. 
\end{equation}
其中$\mu = \hat{\mathbf{k}} \cdot \hat{\mathbf{p}}$是角变量,$\Phi$是牛顿势,$\Psi$是曲率扰动,出现在度规$d s^2 = -(1 + 2\Phi)dt^2 + a^2 d x^2(1 + 2\Psi)$中。最后,$v_{b k}$是重子物质的速度扰动。左侧的项描述了光子如何在引力背景下移动;右侧的项描述了光子如何与重子物质进行电磁相互作用:光学深度$\tau$稍后定义。将$\Theta$展开为关于Legendre多项式$P_l$的乘积项,这些乘积项表示为$\Theta_l$,并定义为:
\begin{equation}
\Theta_l = i^l\int_{-1}^{1} d\mu \, \frac{1}{2} P_l(\mu) \Theta(\mu)~.
\end{equation}
单极子$\Theta_0(k, t)$和偶极子$\Theta_1(k, t)$,分别描述了光子流体的整体密度不均匀性和光子流体的速度,而更高阶的矩则具有不那么直观的含义。换句话说,光子流体中的扰动演化可以通过整个函数$\Theta(k, t, \hat{\mathbf p})$或通过无限数组的乘积$\Theta_l(k, t)$来描述,这些乘积服从从\autoref{eq_CMBAco_1} 导出的无限组(耦合)方程。我们现在讨论这些函数在宇宙的不同阶段所服从的方程。一个关键时刻是重组时代,$a_{recomb} \sim 1100$,即电子、质子和中子在等离子体中形成束缚的氢和氦原子的时刻。光子在这一点之前和之后的演化非常不同。在重组之前,光子不断地在带电粒子上散射,形成一个紧密结合的重子-光子流体,这个流体是非相对论性的。对于这样的流体,玻尔兹曼方程可以截断,即只需要引入上述的前两个矩$\Theta_0(k, t)$和$\Theta_1(k, t)$,而更高阶的矩可以忽略。重组之后,光子从最后散射面自由流出来。它们是一个相对论性流体,因此需要重新引入更高阶的多极矩,我们将在后面进行讨论。

在紧密结合的极限下,光子流体中扰动的演化由$\Theta_0$和$\Theta_1$的两个耦合玻尔兹曼方程控制,这些方程又与物质的演化方程耦合。系统如下:
\begin{equation}
\begin{aligned}
& \dot{\delta}_k - i \frac{k}{a} v_k + 3 \dot{\Psi}_k = 0~. \\
& a \dot{v}_k + \dot{a} v_k - ik \Phi_k = 0~. 
\end{aligned} \quad \quad \text{暗物质}
\end{equation} 
\begin{equation}\label{eq_CMBAco_2}
\begin{aligned} 
&\dot{\delta}_{bk} - ik \frac{a}{\mu} v_{bk} + 3 \dot{\Psi}_k = 0~. \\
& a \dot{v}_{bk} + \dot{a} v_{bk} - ik \Phi_k = a  \frac{\dot{\tau}}{R}[3 i \Theta_1 + v_{bk}]~.
\end{aligned} \quad \quad \text{重子物质}
\end{equation}
\begin{equation}\label{eq_CMBAco_3}
\begin{aligned} 
&\dot{\Theta}_0 -  \frac{k}{a} \Theta_1 + \dot{\Psi}_k = 0~. \\
& \dot{\Theta}_1 +  \frac{k}{3 a} \Theta_0 +  \frac{k}{3 a} \Phi_k = \dot{\tau} \bigg[\Theta_1 -\frac{ i v_{bk}}{3}\bigg]~. 
\end{aligned} \quad \quad \text{光子}
\end{equation} 
我们没有写出中微子和引力的额外方程,为了简单起见。这里$R = 3\rho_b/4\rho_\gamma$,$\tau = \int d\eta n_e \sigma_T a$ 是光度,用电子的数密度$n_e$,汤姆森截面$\sigma_T$和固有时$\eta$表示(见附录C)。物质(重子或暗物质)的方程应该看起来很熟悉:它们是\autoref{eq_LSSFor_6} 中导出的,有一些差异:我们在这里考虑了完整的相对论处理,因此,曲率扰动$\Psi_k$现在出现了(牛顿势$\Phi$仍然存在);我们通过汤姆森散射项明确了重子和光子之间的耦合(\autoref{eq_CMBAco_2} 的右侧)。正如预期的那样,暗物质和重子物质之间的唯一区别是与光子的耦合。\autoref{eq_CMBAco_3} 的右侧又是汤姆森散射项,它将光子与重子耦合起来。

上述方程表明,光子以两种方式与物质耦合:通过电磁作用,如汤姆森散射所表示,以及通过引力作用,如所有流体的引力势$\Phi$和$\Psi$在方程中所表示。与引力的耦合解释了光子在爬出由物质创造的引力势阱时发生红移(或蓝移)的物理效应。关键是,CMB光子分别对提供引力的物质和也带电的物质敏感。这构成了它测量宇宙中暗物质和重子物质密度的能力。重新排列上述方程,可以定性地说明这在实践中是如何工作的。

在大的光度$\dot \tau \rightarrow \infty$的紧密结合极限下,$v_b$的方程可以近似解为$v_{bk} \simeq -3 i\Theta_1$,而重子速度$v_b$的方程可以近似重构为:
\begin{equation}
\bigg[\Theta_1 - \frac{i v_{bk}}{3}\bigg]\dot \tau\simeq -R \bigg[\dot \Theta_1 +H \Theta_1 +\frac{k}{3 a}\Phi_k\bigg]~.
\end{equation}
这可以用来从光子的第二个方程\autoref{eq_CMBAco_3} 中消除物质得到:
\begin{equation}
\dot{\Theta}_1 + H \frac{R}{1 + R} \Theta_1 + \frac{k}{3a} (1 + R) \Theta_0 = -\frac{k}{3a} \Phi_k~. 
\end{equation} 
为了方便起见,我们将时间导数转换为关于共形时间η的导数,我们用撇号表示:Θ₀和Θ₁的方程如下:
\[ \Theta_0' - k \Theta_1 = -\Psi_k'~. \]
\[ \Theta_1' + \frac{a'}{a} \frac{R}{1 + R} \Theta_1 + \frac{k}{3(1 + R)} \Theta_0 = -\frac{k}{3} \Phi_k~. \]
这些可以组合成一个关于光子单极子Θ₀的二阶方程,通过对第一个方程关于η求导:
\begin{equation}
\Theta_0'' + \frac{a'}{a} \frac{R}{1 + R} \Theta_0' + \frac{k^2}{3(1 + R)} \Theta_0 = -\frac{k^2}{3} \Phi_k - a' \frac{R}{1 + R} \Psi_k' - \Psi_k''~. 
\end{equation}
这是熟悉的阻尼受迫谐振子方程。阻尼(左侧的第二项)通常是由宇宙的膨胀引起的,编码在前因子 \( a'/a = aH \) 中。为了简单起见,我们将在后面忽略这个项。如果我们也可以忽略受迫项(右侧的项),方程将读作:
\[ \Theta_0'' + \frac{k^2}{3(1 + R)} \Theta_0 = 0~. \]
给出 \( \Theta_0 = c_1 \cos(k\eta/v_s) + c_2 \sin(k\eta/v_s) \),其中 \( v_s = 1/\sqrt{3(1 + R)} \) 是光子-重子流体的声速(如果重子密度可以忽略,它给出了相对论性流体的标准声速,\( v_s \rightarrow 1/\sqrt{3} \))。因此,$\Theta_0$的解是$k\eta$的振荡函数,所有峰值和谷值都具有相同的幅度(对于正弦和余弦函数来说这是显而易见的)。重要的是,$k$的振荡周期依赖于$v_s$,后者又通过$R$依赖于$\rho_b$。因此,CMB中的声学峰之间的间隔将受到早期宇宙中重子物质数量的影响。

当包括重力提供的受迫项时,Θ₀不再围绕0振荡,而是围绕受迫项提供的偏移振荡。例如,仅包括Φ_k项的贡献,近似解变为:

\[ \Theta_0 = c_1 \cos(k\eta/vs) + c_2 \sin(k\eta/vs) - (1 + R) \Phi_k~. \]

稍微提前一点,我们预计我们将对诸如此类的函数进行“平方”运算。做完之后,声学峰不再具有相同的高度。这正是我们在现今CMB相关性中观察到的特性,因此也在Cℓ中。也就是说,引力受迫项(方程(1.28)的右侧),取决于暗物质的数量,控制着Θ²₀(k)中偶数和奇数峰值之间的相对高度(因此在CMB各向异性中),提供了宇宙中总暗物质的测量。

自由流区

为了继续叙述,我们需要考虑重组之后发生的事情,即光子自由流射的区域。为此,我们回到方程(1.19),我们可以重新排列为:

\[ \Theta' - (ik\mu - \tau')\Theta = \hat{S} ~. \]

其中源函数 \( \hat{S} \) 定义为 \( \hat{S} = -ik\mu\Phi - \tau'\Theta_0 \)。为了得到这里,我们已经转换到共形时间,暂时忽略了引力势的时间依赖性,并且忽略了重子物质密度的偶极子 \( vbk \)。然后我们找到了方程(1.31)在共形时间中的解,并将其展开为勒让德多项式(这会导致球贝塞尔函数 \( j_\ell \) 出现)。我们最终得到(见[20])以下结果,这提供了现今(即在 \( \eta_0 \) 时)光子场的多极值:

\[ \Theta_\ell(k, \eta_0) = \int_0^{\eta_0} d\eta \, g(\eta) \left[ \Theta_0(k, \eta) + \Phi(k, \eta) \right] j_\ell[k(\eta_0 - \eta)] ~. \]

这里的可见度函数 \( g(\eta) = -\tau' \exp[-\tau(\eta)] \) 给出了光子在共形时间 \( \eta \) 上最后一次散射的概率。它在 \( \eta \approx \eta_{recomb} \) 处达到峰值:对于 \( \eta \ll \eta_{recomb} \),光子继续在紧密耦合的光子-重子流体中散射的可能性非常大,而对于 \( \eta \gg \eta_{recomb} \),宇宙是透明的。因此,在这两种极限情况下,\( g(\eta) \to 0 \),并且可以很好地近似为:

\[ \Theta_\ell(k, \eta_0) \approx \left[ \Theta_0(k, \eta_{recomb}) + \Phi(k, \eta_{recomb}) \right] j_\ell[k(\eta_0 - \eta_{recomb})] ~. \]

这个解具有直观的含义。光子扰动在重组之后不会增长——宇宙中的引力势太弱,无法捕获光子,这些光子基本上自由流射,给我们提供了在最后散射表面上的宇宙快照。CMB温度场由动能和势能之和控制,\( \Theta_0(k, \eta_{recomb}) + \Phi(k, \eta_{recomb}) = -\delta(k, \eta_{recomb})/6 \),即光子在现今的温度包括了它们需要从重组时期的势阱中爬出来的事实。此外,光子总能量中的扰动由暗物质扰动 \( \delta \) 控制(负号表示,因为现在较热的区域对应于重组时期暗物质流体中的低密度区域)。球贝塞尔函数 \( j_\ell[k(\eta_0 - \eta_{recomb})] \) 仅仅编码了从具有波数 k 的纯平面波在角度尺度 \( 1/\ell \) 上的各向异性有多少。对于大 \( \ell \),\( j_\ell(x) \to (x/\ell)^{\ell-1/2}/\ell \),因此只有在 \( \ell \sim k\eta_0 \) 时才对 CMB 各向异性有相关贡献(可以忽略 \( \eta_{recomb} \),因为 \( \eta_0 \gg \eta_{recomb} \))。



