% 热力学第一定律
% 热力学第一定律|能量守恒|做功|传热|内能

\begin{issues}
\issueDraft
\end{issues}

\pentry{压力体积图\upref{PVgraf}, 理想气体内能\upref{IdgEng}}

热力学第一定律是能量守恒在热力学中的形式.热量是指当热学系统出现温度差时引起的能量转移的一个度量.热力学第一定律表明,外部对系统传递的热量等于系统对外做功加上系统的内能增加:
\begin{equation}\label{Th1Law_eq1}
\Delta Q = W + \Delta E
\end{equation}
我们称压强 $p$ 和体积 $V$ 为系统的力学参量,其中 $p$ 为强度量,$V$ 为广延量.$E$ 为系统的内能,有时也用字母 $U$ 表示.热力学第一定律的另一种表述是:\textbf{第一类永动机}是不可能造成的.


在力学相互作用过程中,系统和外界之间转移的能量就是功.对于容器中的气体,设其压强为 $p$,体积为 $V$,那么对外做功可以写成积分形式:
\begin{equation}
W = \int p \dd{V}
\end{equation}
热力学第一定律写成微分形式是
\begin{equation}\label{Th1Law_eq2}
\dd E=\delta Q-\delta W=\delta Q-p\dd V
\end{equation}
$Q$ 和 $W$ 前用的是 $\delta$ 符号而不是全微分符号,是因为 $Q$ 和 $W$ 和系统变化的过程本身有关.$E$ 前面用的是全微分符号,是因为 $E$ 本身代表气体系统的内能函数,是一个态函数,而 $\Delta E$ 只与初始和最终的系统状态有关.$E$ 是系统状态的函数,我们称它为 \textbf{态函数},而 $W$ 和 $Q$ 并不是系统状态的函数,它们用来描述在一个系统变化过程中功和热量的传递,是一个 \textbf{过程量}\upref{StaPro}.

虽然,$\delta W$ 是过程量,但 $\delta W/p=\dd V$ 是全微分($V$ 是态函数).$\delta Q/T=\dd S$ 也是全微分,其中 $S$ 为热力学熵\upref{Entrop}.

\subsection{内能和态函数}
如果某个函数只和系统的热力学参量有关,也就是只和系统状态有关,我们称它为\textbf{态函数}.热力学研究的就是热力学系统的态函数之间的关系.

我们可以用几个宏观的热力学参量来完整地刻画一个热力学平衡系统.例如,对于一个无外场的孤立气体系统,压强 $p$ 和温度 $T$ 足以刻画这个气体系统的一切宏观特征.对于理想气体,有状态方程 $pV=nRT$,压强 $p$ 和温度 $T$ 足以描绘整个理想气体系统(对任意热力学系统也有类似的结论).因此可以写出 $E$ 的全微分形式:
\begin{equation}
\dd E=\left(\frac{\partial E}{\partial T}\right)_p \dd T + \left(\frac{\partial E}{\partial p}\right)_T \dd p
\end{equation}

如果将 $E$ 看成是熵 $S$ \upref{Entrop} 和体积 $V$ 的函数,则可以写成
\begin{equation}
E=T\dd S-p\dd V
\end{equation}

这里 $S=\left(\frac{\partial E}{\partial T}\right)_V$,除此以外熵还有统计物理的定义.如果这个全微分刚好对应系统的一个可逆过程,那么我们可以看出 $\delta W$ 就是 $p\dd V$,由热力学第一定律,就有 $\delta Q=T\dd S$,这给出了熵的另一个定义——对于可逆过程 $\delta Q/T$ 的积分.

对于不可逆的热力学过程,有 $\delta W<p\dd V$,$\delta Q<T\dd S$.

对理想气体\upref{Igas}, 令分子自由度为 $i$, 有
\begin{equation}
E = \frac{i}{2}n RT
\end{equation}

\addTODO{需要加一个范德瓦尔斯气体的词条,作为经典例子}