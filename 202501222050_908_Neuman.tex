% 约翰·冯诺依曼(综述)
% license CCBYSA3
% type Wiki

本文根据 CC-BY-SA 协议转载翻译自维基百科\href{https://en.wikipedia.org/wiki/John_von_Neumann}{相关文章}。

\begin{figure}[ht]
\centering
\includegraphics[width=6cm]{./figures/c4c9226c223e913e.png}
\caption{冯·诺依曼在1940年代} \label{fig_Neuman_1}
\end{figure}
约翰·冯·诺依曼(John von Neumann,1903年12月28日—1957年2月8日)是一位匈牙利裔美国数学家、物理学家、计算机科学家和工程师。冯·诺依曼可能是他那个时代涵盖面最广泛的数学家之一,他将纯粹科学和应用科学相结合,并对许多领域作出了重要贡献,包括数学、物理学、经济学、计算机学和统计学。他是量子物理学数学框架建设的先驱,在泛函分析和博弈论的发展中也做出了突出贡献,提出或规范了包括细胞自动机、通用构造器和数字计算机等概念。他对自我复制结构的分析,早于DNA结构的发现。

在第二次世界大战期间,冯·诺依曼参与了曼哈顿计划,他开发了用于爆炸透镜的数学模型,这些透镜在内爆型核武器中起到了重要作用。战前和战后,他为许多组织提供咨询服务,包括科学研究与发展办公室、陆军弹道研究实验室、武装部队特殊武器计划和橡树岭国家实验室等。在1950年代的巅峰时期,他主持了多个国防部委员会,包括战略导弹评估委员会和洲际弹道导弹科学顾问委员会。他还是负责全国所有原子能开发的影响力巨大的原子能委员会的成员。在与伯纳德·施里弗和特雷弗·加德纳的合作中,他在美国首个洲际弹道导弹(ICBM)项目的设计和开发中扮演了关键角色。那时,他被认为是美国核武器方面的顶尖专家,也是美国国防部的首席防御科学家。

冯·诺依曼的贡献和智力能力得到了物理学、数学及其他领域同事的高度赞扬。他所获得的荣誉包括自由勋章以及以他名字命名的月球陨石坑。
\subsection{生活与教育}  
\subsubsection{家庭背景}  
冯·诺依曼于1903年12月28日出生在匈牙利王国的布达佩斯(当时是奥匈帝国的一部分),出生于一个富裕的、不信教的犹太家庭。他的出生名为Neumann János Lajos。在匈牙利语中,姓氏排在前面,而他的名字相当于英语中的John Louis。

他是家中三个兄弟中的长子,两个弟弟分别是米哈伊(Mihály,迈克尔)和尼克拉斯(Miklós,尼古拉斯)。他的父亲Neumann Miksa(Max von Neumann)是一位银行家,拥有法学博士学位。父亲于1880年代末从佩奇(Pécs)搬到布达佩斯。Miksa的父亲和祖父均出生于匈牙利北部的翁德(Ond,现在是泽伦茨(Szerencs)的一部分),位于泽普伦(Zemplén)县。冯·诺依曼的母亲是Kann Margit(玛格丽特·坎);她的父母是Kann Jákab和Meisels Katalin,来自Meisels家族。坎家三代人居住在布达佩斯坎-赫勒事务所上方的宽敞公寓里;冯·诺依曼的家庭占据了顶层的18个房间。

1913年2月20日,弗朗茨·约瑟夫皇帝因冯·诺依曼的父亲为奥匈帝国的服务,将他提升为匈牙利贵族。因此,诺依曼家族获得了“Margittai”这一世袭称号,意为“来自马吉塔”(如今的罗马尼亚马尔吉塔)。家族与该镇并无实际联系,这个称号是为了纪念玛格丽特而选择的,他们的家族徽章也描绘了三朵雏菊。冯·诺依曼János成为了“margittai Neumann János”(约翰·诺依曼·德·马吉塔),后来他将其更改为德语的“Johann von Neumann”。
\subsubsection{天才儿童}  
冯·诺依曼是一个天才儿童,六岁时就能在脑海中计算两个八位数的除法,并能用古希腊语交流。他和他的兄弟、表兄弟们由家庭教师教授。冯·诺依曼的父亲认为,除了母语匈牙利语之外,掌握其他语言是非常重要的,因此孩子们都接受了英语、法语、德语和意大利语的辅导。到八岁时,冯·诺依曼已经熟悉微积分,十二岁时他读过了布雷尔(Borel)的《函数论》(La Théorie des Fonctions)。他还对历史感兴趣,阅读了威廉·昂肯(Wilhelm Oncken)的46卷世界历史丛书《普通历史的个别阐述》(Allgemeine Geschichte in Einzeldarstellungen)。公寓中的一间房间被改成了图书馆和阅览室。

冯·诺依曼于1914年进入了路德教会的法索里·厄尔文吉库斯高中(Lutheran Fasori Evangélikus Gimnázium)。尤金·维格纳(Eugene Wigner)比冯·诺依曼大一岁,成为了冯·诺依曼在学校的朋友。

尽管冯·诺依曼的父亲坚持让他按年龄进入相应的年级,但他同意雇佣私人教师为冯·诺依曼提供更高水平的教学。15岁时,他开始在数学分析家加博尔·塞戈(Gábor Szegő)的指导下学习高阶微积分。塞戈第一次见到冯·诺依曼时,被他的数学才能和学习速度震惊得无以复加,正如他的妻子回忆的那样,塞戈带着泪水回到家中。19岁时,冯·诺依曼已发表了两篇重要的数学论文,其中第二篇给出了现代的序数定义,取代了乔治·康托尔(Georg Cantor)原有的定义。在完成高中学业后,冯·诺依曼申请并获得了厄尔托斯奖(Eötvös Prize),这是匈牙利的一项国家数学奖项。
\subsubsection{大学学习}
根据他的朋友西奥多·冯·卡门(Theodore von Kármán)的说法,冯·诺依曼的父亲希望约翰能够继承他的事业进入工业界,并要求冯·卡门劝说冯·诺依曼不要选择数学。冯·诺依曼和父亲决定,最好的职业道路是化学工程。冯·诺依曼对化学工程并不太了解,因此安排他在柏林大学参加为期两年的非学位化学课程,之后参加了苏黎世联邦理工学院(ETH Zurich)的入学考试,并在1923年9月通过了考试。与此同时,冯·诺依曼还进入了帕兹玛尼·彼得大学(当时称为布达佩斯大学),作为数学博士候选人。他的博士论文是关于康托尔集论的公理化。

1926年,他从苏黎世联邦理工学院获得化学工程学位,并同时以优异成绩通过了布达佩斯大学数学博士的最终考试(辅修实验物理学和化学)。

随后,他凭借洛克菲勒基金会的奖学金前往哥廷根大学,师从大卫·希尔伯特(David Hilbert)学习数学。赫尔曼·外尔(Hermann Weyl)回忆道,在1926至1927年冬季,冯·诺依曼、艾米·诺特(Emmy Noether)和他常常在课后一起走在“哥廷根寒冷、湿滑、被雨水打湿的街道上”,讨论超复数系统及其表示方法。
\subsection{职业与私生活}
\begin{figure}[ht]
\centering
\includegraphics[width=8cm]{./figures/970bcb0b4439142d.png}
\caption{摘自1928年和1928/29学年《弗里德里希·威廉大学柏林校历》的内容,宣布冯·诺依曼的讲座主题包括:函数理论II、公理化集合论和数学逻辑、数学座谈会、量子力学的最新研究综述、数学物理的特殊函数以及希尔伯特的证明理论。他还讲授了相对论理论、集合论、积分方程以及无限多个变量的分析。} \label{fig_Neuman_2}
\end{figure}
冯·诺依曼于1927年12月13日完成了他的“资格讲师”(Habilitation)资格认证,并于1928年开始在柏林大学担任资格讲师(Privatdozent)。他成为该大学历史上最年轻的资格讲师。他开始几乎每月撰写一篇重要的数学论文。1929年,他短暂地成为汉堡大学的资格讲师,因为在那里成为终身教授的前景更好;然后,他在同年10月转至普林斯顿大学,担任数学物理学的访问讲师。

冯·诺依曼于1930年接受了天主教洗礼。不久后,他与玛丽埃塔·科韦西结婚,玛丽埃塔曾在布达佩斯大学学习经济学。冯·诺依曼和玛丽埃塔育有一女,玛丽娜,生于1935年;她后来成为一名教授。夫妻俩于1937年11月2日离婚。1938年11月17日,冯·诺依曼与克拉拉·丹(Klára Dán)结婚。

1933年,冯·诺依曼接受了新泽西州高等研究院的终身教授职位,当时该机构原计划任命赫尔曼·外尔的计划似乎未能实现。1939年,冯·诺依曼的母亲、兄弟和亲家一同前往美国。冯·诺依曼将自己的名字改为英文名“John”,但保留了德式贵族姓氏“von Neumann”。冯·诺依曼于1937年成为美国公民,并立即尝试成为美国陆军后备军官团的一名中尉。他通过了考试,但因年龄问题被拒绝。

克拉拉与约翰·冯·诺依曼在当地学术界十分活跃。他们位于西科特路的白色木板房屋是普林斯顿最大的私人住宅之一。冯·诺依曼总是穿着正式的西装。他喜欢犹太语和“有点不雅”的幽默。在普林斯顿,他曾因播放极其响亮的德国进行曲音乐而受到投诉。冯·诺依曼在嘈杂、混乱的环境中做出了他的一些最佳工作。根据丘吉尔·艾森哈特(Churchill Eisenhart)的说法,冯·诺依曼曾参加过派对直到凌晨,然后在8:30准时开始讲座。

冯·诺依曼以乐于为各个层次的学者提供科学和数学建议而闻名。维格纳(Wigner)曾写道,他可能比任何其他现代数学家都更“随意地”指导了更多的工作。他的女儿写道,冯·诺依曼非常关心自己在两个方面的遗产:他的生活和他对世界的知识贡献的持久性。

许多人认为冯·诺依曼是一个出色的委员会主席,他在个人或组织事务上较为宽容,但在技术问题上则表现得非常坚持。赫伯特·约克(Herbert York)描述了冯·诺依曼参与的许多“冯·诺依曼委员会”,认为它们“在风格和成果上都非常出色”。冯·诺依曼主持的委员会与必要的军事或企业实体密切合作,这种工作方式成为了所有空军远程导弹项目的蓝图。许多曾与冯·诺依曼接触的人对他与军事及权力结构的关系感到困惑。斯坦尼斯瓦夫·乌拉姆(Stanisław Ulam)怀疑冯·诺依曼对那些能够影响他人思维和决策的人或组织有着隐秘的钦佩。

冯·诺依曼还保持着他年轻时学过的语言能力。他能流利地使用匈牙利语、法语、德语和英语,并保持一定的意大利语、犹太语、拉丁语和古希腊语会话水平。他的西班牙语水平较差。他对古代历史充满热情,并具备百科全书般的知识,喜欢用原文阅读古希腊历史学家的著作。乌拉姆怀疑这些可能影响了冯·诺依曼对未来事件的看法,以及他对人性和社会运行方式的理解。

冯·诺依曼在美国最亲密的朋友是数学家斯坦尼斯瓦夫·乌拉姆。冯·诺依曼认为,他的大部分数学思维是直觉性的;他常常在睡觉时带着一个未解答的问题,醒来后便知道了答案。乌拉姆指出,冯·诺依曼的思维方式可能并非视觉性的,而更偏向听觉性。乌拉姆回忆道:“完全不论他对抽象智慧的喜爱,他对更接地气的喜剧和幽默有着强烈的欣赏(可以说几乎是渴望)。”
\subsubsection{疾病与死亡}
\begin{figure}[ht]
\centering
\includegraphics[width=6cm]{./figures/4dbffa775d803dc1.png}
\caption{冯·诺依曼的墓碑} \label{fig_Neuman_3}
\end{figure}
1955年,冯·诺依曼的锁骨附近发现了一个肿块,最终被确诊为源自骨骼、胰腺或前列腺的癌症。(尽管普遍认为肿瘤已经发生了转移,但不同的资料来源对于原发性癌症的位置存在不同看法。)这恶性肿瘤可能是由于在洛斯阿拉莫斯国家实验室接触辐射所致。随着死亡临近,他请求了一位神父,尽管这位神父后来回忆道,冯·诺依曼在接受最后的圣事时几乎没有得到安慰——他仍然对死亡感到恐惧,且无法接受它。关于冯·诺依曼的宗教观点,据说他曾说:“只要非信徒可能面临永恒的诅咒,那么在最后时刻成为信徒更为合逻辑,”这是指帕斯卡尔的赌注。他曾对母亲倾诉:“或许必须有一个上帝。如果有的话,许多事情比没有上帝时更容易解释。”

他于1957年2月8日在沃尔特·里德陆军医疗医院去世,享年53岁,并被埋葬在普林斯顿墓地。
\subsection{数学}  
\subsubsection{集合论}
\begin{figure}[ht]
\centering
\includegraphics[width=6cm]{./figures/4efc733fe9f8afc1.png}
\caption{导致NBG集合论的各种方法的历史} \label{fig_Neuman_4}
\end{figure}
20世纪初,试图将数学建立在朴素集合论基础上的努力遭遇了挫折,原因是拉塞尔悖论(即“所有不属于自己的集合的集合”)的出现。解决集合论充分公理化的问题,大约在二十年后由恩斯特·采尔梅洛(Ernst Zermelo)和亚伯拉罕·弗兰克尔(Abraham Fraenkel)间接解决。采尔梅洛-弗兰克尔集合论提供了一系列原理,允许构造日常数学实践中使用的集合,但并未明确排除存在自包含集合的可能性。在他1925年的博士论文中,冯·诺依曼展示了两种排除这种集合的技巧——基础公理和类的概念。

基础公理提出,每个集合都可以通过采尔梅洛-弗兰克尔原理,从底部向上按顺序构造。如果一个集合属于另一个集合,那么第一个集合必须先于第二个集合出现在这个顺序中。这排除了集合自包含的可能性。为了证明添加这个新公理不会产生矛盾,冯·诺依曼引入了内模型方法,这成为集合论中的一个重要证明工具。

解决自包含集合问题的第二种方法以“类”的概念为基础,并定义一个集合为属于其他类的类,而定义一个适当类为不属于其他类的类。在采尔梅洛-弗兰克尔方法中,公理阻止了构造一个包含所有不属于自己的集合的集合。而在冯·诺依曼的方法中,可以构造所有不属于自己的集合的类,但它是一个适当类,而不是集合。

总体而言,冯·诺依曼在集合论方面的主要成就之一是“集合论的公理化和(与之相关的)优雅的序数与基数理论,以及通过超限归纳法严格表述定义原理”。

\textbf{冯·诺依曼悖论}  

在费利克斯·豪斯多夫(Felix Hausdorff)于1914年提出的豪斯多夫悖论的基础上,斯特凡·巴拿赫(Stefan Banach)和阿尔弗雷德·塔尔斯基(Alfred Tarski)于1924年展示了如何将一个三维球体分割成不相交的集合,然后通过平移和旋转这些集合,形成两个完全相同的球体副本;这就是著名的巴拿赫–塔尔斯基悖论。他们还证明了一个二维圆盘没有这样的悖论性分解。然而,在1929年,冯·诺依曼将圆盘分割成有限多个部分,并通过面积保持的仿射变换,而非平移和旋转,将这些部分重新排列成两个圆盘。这个结果依赖于找到仿射变换的自由群,这是一个重要的技术,后来冯·诺依曼在他的测度论工作中进一步扩展了这一方法。
\subsubsection{证明论} 
冯·诺依曼对集合论的贡献使得该集合理论的公理系统避免了早期系统的矛盾,并成为数学基础的可用框架,尽管没有证明其一致性。下一个问题是:它是否为所有可以在其中提出的数学问题提供了确定的答案,还是可以通过添加更强的公理来改进,从而能够证明更广泛类别的定理。[95]

到1927年,冯·诺依曼开始参与哥廷根的讨论,是否可以从佩阿诺公理中推导出基本的算术。他在阿克曼(Ackermann)的工作基础上,开始尝试用希尔伯特学派的有限方法证明一阶算术的一致性。他成功地证明了自然数算术的一个片段的一致性(通过对归纳法的限制)。[97] 他继续寻找使用证明论方法证明经典数学一致性的更一般的证明。[98]

关于这一系统是否是确定性的,1930年9月,在第二届精确科学认识论大会上给出了一个强烈的否定答案。在大会上,库尔特·哥德尔宣布了他的第一不完备定理:常见的公理系统是不完备的,意味着它们无法证明在其语言中表达的每一个真理。此外,这些系统的任何一致扩展必然仍然是不完备的。[99] 在大会上,冯·诺依曼建议哥德尔应该尝试将他的结果转化为关于整数的不可判定命题。[100]

不到一个月后,冯·诺依曼向哥德尔传达了他定理的一个有趣结论:常见的公理系统无法证明它们自身的一致性。[99] 哥德尔回复说,他已经发现了这个结论,也就是他的第二不完备定理,并且他将发送包含这两个结果的文章的预印本,但该文章从未公开。[101][102][103] 冯·诺依曼在下一封信中承认了哥德尔的优先权。[104] 然而,冯·诺依曼的证明方法与哥德尔不同,他还认为第二不完备定理对希尔伯特计划的打击要比哥德尔认为的要大得多。[105][106] 这一发现大大改变了他对数学严格性的看法,冯·诺依曼停止了对数学基础和元数学的研究,转而专注于与应用相关的问题。[107]
\subsubsection{遍历理论}  
在1932年发表的一系列论文中,冯·诺依曼对遍历理论做出了基础性贡献,遍历理论是数学的一个分支,涉及具有不变测度的动态系统的状态。[108] 对于冯·诺依曼在1932年发表的遍历理论论文,保罗·哈尔莫斯(Paul Halmos)曾写道,即使“冯·诺依曼再也没有做过其他任何事情,这些论文也足以确保他获得数学上的不朽声誉”。[109] 到那时,冯·诺依曼已经写过关于算子理论的文章,这些工作的应用在他的均值遍历定理中发挥了重要作用。[110]

该定理涉及任意的单参数酉群 \(t \to V_{t}\)并声明,对于希尔伯特空间中的每个向量 \(\phi\)
,\(\lim_{T \to \infty} \frac{1}{T} \int_0^T V_{t}(\phi) \, dt\)在由希尔伯特范数定义的度量意义下是存在的,并且是一个向量 \(\psi\),使得\(V_{t}(\psi) = \psi\)对于所有\(t\)。这一结论在第一篇论文中得到了证明。在第二篇论文中,冯·诺依曼认为他的结果足以应用于与玻尔兹曼的遍历假设相关的物理应用。他还指出,遍历性尚未实现,并将其孤立出来,作为未来研究的课题。[111]

同年晚些时候,他发表了另一篇具有影响力的论文,开始系统地研究遍历性。他给出了并证明了一个分解定理,展示了实数轴上的遍历测度保持作用是所有测度保持作用的基本构建块,所有其他测度保持作用都可以由此构建。这篇论文中给出并证明了几个其他关键定理。这些结果与保罗·哈尔莫斯的另一篇论文共同,在数学的其他领域有着重要应用。[111][112]
\subsubsection{测度理论}  
在测度理论中,n维欧几里得空间 \( \mathbb{R}^n \) 的“测度问题”可以表述为:“是否存在一个正的、归一化的、不变的、可加的集合函数,定义在 \( \mathbb{R}^n \) 所有子集的类上?”[113] 费利克斯·豪斯多夫(Felix Hausdorff)和斯特凡·巴拿赫(Stefan Banach)的工作暗示,当 \( n = 1 \) 或 \( n = 2 \) 时,测度问题有正解,而在其他情况下则为负解(因为有班纳赫–塔尔斯基悖论)。冯·诺依曼的工作认为,“这个问题本质上是群论性质的”:测度的存在可以通过观察给定空间的变换群的性质来确定。对于最多二维的空间,正解存在;对于更高维的空间,负解存在,原因在于欧几里得群对于最多二维的空间是可解群,而对于高维空间则不是可解的。“因此,根据冯·诺依曼的观点,是群的变化才决定了差异,而不是空间的变化。”[114] 大约在1942年,他告诉多萝西·马哈拉姆(Dorothy Maharam)如何证明每个完全 σ-有限测度空间都有一个乘法提升;他没有发表这个证明,后来她提出了一个新的证明。[115]

在冯·诺依曼的许多论文中,他采用的论证方法被认为比结果本身更为重要。为了预示他日后对算子代数中的维数理论的研究,冯·诺依曼使用了有限分解等价的结果,并将测度问题重新表述为关于函数的问题。[116] 冯·诺依曼对测度理论的主要贡献之一是,他写了一篇论文回答哈尔(Haar)提出的问题,关于是否存在一个所有有界函数的代数,这些函数定义在实数轴上,并且它们构成“几乎处处相等的可测有界函数类的代表系统”。[117] 他证明了这个问题的正解,并在后来的论文中与斯通(Stone)讨论了这个问题的各种推广和代数方面。[118] 他还通过新的方法证明了各种一般类型测度的分解存在性。冯·诺依曼还通过使用函数的平均值,给出了哈尔测度唯一性的一个新证明,尽管这个方法仅适用于紧群。[117] 他不得不创造全新的技术,将此应用到局部紧群。[119] 他还为拉东–尼科迪姆定理提供了一个新的巧妙证明。[120] 他在高等研究所的测度理论讲义是当时美国关于该主题的重要知识来源,后来被出版。[121][122][123]