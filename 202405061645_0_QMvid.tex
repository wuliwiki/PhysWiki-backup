% 量子力学科普视频脚本
% license Xiao
% type Art

\begin{issues}
\issueDraft
\end{issues}



\pentry{量子力学的基本原理(科普)\nref{nod_QM0}}{nod_5f89}

解说:现代的量子力学如何描述微观粒子的运动? 为了简单我们先看直线运动。 经典力学中,根据牛顿第一定律,不受力的小球保持静止或匀速直线运动。

动画:画出一个数轴(动画拉长方式出现,有刻度线,配 click 音效),小球静止在原点(渐变出现), 一个火柴人出现,踢一脚(画出受力箭头突然变长然后变短,速度箭头突然变长然后固定不变),匀速向右运动, 镜头跟随。

画面暂停: 解说:如果我们不考虑小球的形状和旋转,可以把它抽象成一个运动的有质量的点(画面把球缩成一点)。 经典力学告诉我们, 每个时刻质点的状态由位置和速度完全确定。

解说:如果我们想要描述电子、质子等基本粒子的运动,我们就需要使用量子力学。 在量子力学中,微观粒子不再适合用确定的位置和速度描述, 而是需要把它看成一个波动。

解说:生活中处处可见波动现象,使用一根橡皮绳,就可以探索波动的各种规律。

动画:天上掉下一根绳子,弯曲躺在地上,右边超出屏幕,小人捡起绳子在地上的左端,拉进绳子,抖动几下停下来,绳子出现向右运动的波,消失在屏幕右边。 又胡乱抖了几下,出现了各种形状各异的波消失在屏幕右边。

解说:在橡皮绳的一头振动几次,就可以产生向右传递的波。 这些波看起来像一个个包,我们把它叫做波包。波包的特点是长度有限。 如果我们假设绳子右边无限长,那么这些波包会一直保持固定的形状向右匀速运动。

解说:如果在橡皮绳的一头以固定的频率振动,那么经过一段时间绳子就形成了一列完美的简谐波。
(动画根据解说)

解说:

解说:如果把绳的另一端固定,那么这些波包会发生反射。(动画根据解说)




波函数描述。 而波函数通常以波包的形状出现。

动画:渐变成开始的坐标系,高斯波包的实部渐变出现,不画虚部和模长。

解说:波函数通常记为 $\psi(x)$, 把每个坐标 $x$ 对应到函数值 $y = \psi(x)$。

动画:画出一个从数轴向上指向曲线的箭头, 快速向右移动, 坐标轴下面跟随显示坐标, 箭头上方画水平线到 $y$ 轴,显示 $y$ 坐标。

解说:事实上波函数的值是一个复数, 刚才画出的只是函数值的实数部分,我们还可以用另一条曲线表示虚数部分,再用一条曲线表示复数的模长。 模长曲线就是某个位置振动幅度的大小。

动画:根据解说

解说:当时间开始流逝,波包的整体向右移动,同时它的形状也会发生一些改变。

动画:根据解说

解说:这样一个匀速运动的波包对应的是经典力学中向右匀速运动的质点,它不受任何外力作用。

动画:上下画两个数轴,上面演示波包向右运动,下面演示小球向右运动。

解说:

% addis: 这不是我们要做的量子视频,可能以后做一个关于科学的科普的视频的时候可以用到

% \subsubsection{Jier的稿子}

% 黑屏,字幕:【科学是一种认知体系,一切以“观察、猜想、验证”三步循环来认知事物的方式,都是科学。】【Science is a system of cognition. The way of science, is the cycle of "observation, conjecture, and examination". 】

% 动画:主角出现,坐在苹果树下打盹。一颗苹果砸中他的脑袋,惊醒了他。他捡起苹果,摸着被砸痛的位置,脑袋上冒出心理活动气泡,随着内容展开,气泡范围逐渐变大,直到整个屏幕都是气泡的内容。气泡内容如下:空旷的世界里只有一个苹果和苹果下方的地面,然后出现一个二维直角坐标系,作为空间坐标,但不要标出刻度和数值;同时苹果中心出现一个点,代表苹果的位置。接着,从苹果中心点出发,延伸出两条平行于坐标轴的虚线,与坐标轴相交时停止,并表示出苹果当前的空间坐标值。苹果中心点处出现一根箭头,表示重力,旁边加上字母G。出现箭头的时候,苹果开始自由落体运动,虚线和坐标值同步变化,直到苹果落地。苹果落地有回弹效果,回弹过程中要同步画出地面对苹果的形变压力。当苹果最终静止后,压力和重力的箭头等大反向,二者存在两秒钟左右,然后像俄罗斯方块一样同步闪烁后消失。

% 同步配音:【一颗苹果打破了我宁静的午后时光。好奇心驱使我,开始猜想它为什么会这么运动。也许有一个东西在改变苹果的运动状态,我管它叫“力”。受的力越大,苹果的速度改变就越快;受的合外力为零,苹果应该做匀速直线运动。】【That completely ordinary day was broken by an apple. I was driven to conjecture a theory to explain why it moves the way it did. I imagined there's a thing that affects the apple's status of movement, namely "force"。The greater the force, the quicker the apple's velocity changes; hence when the net force is zero, the apple should remain in the same speed, moving along a straight line. 】

% 动画:主角本人入镜,捡起苹果,抬在手里,这次不需要标记苹果受力。一把大号尺子从地面长出来,竖在主角身旁,使得苹果的中心点可以和尺子上的刻度重合。主角松手让苹果下落,和之前一样边下落边表示出其空间坐标值变化。下落过程中取几个瞬间定格,表现尺子上的刻度值和此刻竖直坐标轴上的数值一样。

% 同步配音:【我可以解释描述苹果的那两个数字是怎么测量得到的:我发明了一个装置,取名“尺子”。竖轴上的数值,就是尺子上和苹果中心点重合的刻度值。测量是必要的,因为“观察”和“验证”两个步骤都需要和现实世界互动,我必须解释我的理论与现实的关联。】【I can explain how to measure the two numbers: I invented this device, namely a "ruler". It's necessary to measure, since the two steps "observation" and "examination" involve interacting with the reality. I have to explain how my theory reflects the reality. 】

% 动画:苹果出现半透明的分身,按之前自由落体的直线路径下落到地面;然后再出现一个分身,沿着任意的曲线下落到地面;再出现一个分身,沿着任意的曲线运动到任意位置。三个分身的轨迹都要画出来,像苹果拖出的尾巴;后两个分身轨迹旁边要画上问号。

% 同步配音:【我的理论有很多问题,比如我没有解释什么叫“直线”。】【There remain quite some flaws in my theory, say, I never explained what a "straight line" is. 】

% 动画:镜头迅速拉远,表现为:苹果缩小到看不见,平直的地面逐渐弯曲,直到画面中出现了地球和月球。与此同时,坐标系的坐标原点也逐渐向下移动,最终落在地月质心处。地球和月球绕着质心,做角速度相同的匀速圆周运动。从地球内部伸出一个巨大的火箭,尾部喷火,飞向月球,绕月一周后落在月球上。

% 同步配音:【但是问题不大,我的理论预言了物体运动的规律,也能解释这些规律如何体现在现实中,比如火箭如何飞到月球上。于是人们按照我的理论设计了一枚火箭,成功落到了月球上指定的位置。】【But that's not a big deal. My theory predicts the movement of an object, and how it looks like in the reality, how a rocket can land on the Moon, for example. People designed a rocket, and it landed on the Moon precicely as theory predicted. 】

% 动画:主角昂首挺胸,走在地面上,背后是模糊的群众在欢呼,扔出彩带。

% 同步配音:【】
