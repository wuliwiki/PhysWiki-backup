% 刚体的瞬时转轴、角速度的矢量相加
% 转轴|刚体|陀螺|角速度

\pentry{刚体\upref{RigBd}, 速度的参考系变换\upref{Vtran2}}

\addTODO{首先说明,一般来说两次旋转是不对易的, 所以旋转并不能表示为矢量。 但是经过以下分析, 微小旋转可以表示为矢量。}

\subsection{刚体的瞬时转轴}
当刚体绕某个固定点做任意转动时, 我们在每个时刻仍然能找到一个\textbf{瞬时转轴}以及沿转轴的瞬时角速度矢量 $\bvec \omega$。

\begin{theorem}{刚体的瞬时转轴}
若刚体绕某个固定点做任意转动, 任意时刻都存在过该点的一条直线, 使得刚体上任意落在直线上的点的(瞬时)速度为零, 且任意刚体上任意速度为零的点都落在直线上。 我们把这条直线叫做刚体的\textbf{瞬时转轴}。 如果某时刻刚体不是静止的, 那么此时\textbf{瞬时转轴}就是唯一的。
\end{theorem}
根据该定理, 如果刚体不静止, 我们只需要在某时刻找到刚体上瞬时速度为 0 的任意两个不同点, 就可以过这两点作出瞬时转轴, 并确保该直线上的所有点瞬时速度都为 0。

\begin{example}{进动陀螺的瞬时转轴}\label{InsAx_ex2}
在陀螺进动的例子(\autoref{AMLaw_ex2}~\upref{AMLaw}) 中, 我们可能会认为陀螺的瞬时转轴就是陀螺的对称轴。 但陀螺的轴时时刻刻都在运动, 除了与地面接触的点外, 任意一点的瞬时速度都不为 0。 所以进动陀螺的对称轴并不是瞬时转轴。

假设陀螺的自转和进动的方向相同(都是顺时针或逆时针), 我们可以知道真正的瞬时转轴同样经过其对称轴与地面的交点, 但会在对称轴的上方。 我们只需要找到陀螺圆盘表面上瞬时速度为 0 的一点即可求出瞬时转轴的倾角。 具体计算见\autoref{InsAx_ex1}。
\end{example}

\subsection{角速度的矢量相加}

我们首先约定以下的黑体字母表示几何矢量\upref{GVec}而非坐标, 与参考系无关。 在 $S'$ 参考系中, 刚体绕原点以角速度 $\bvec \omega'$ 旋转, 而 $S'$ 参考系相对于 $S$ 参考系(它们原点重合)绕原点以角速度 $\bvec \omega_r$ 旋转, 那么可以证明刚体相对于 $S$ 参考系的角速度矢量就是
\begin{equation}\label{InsAx_eq1}
\bvec \omega = \bvec \omega' + \bvec \omega_r
\end{equation}

\textbf{证明}:可以直接使用 “速度的参考系变换\upref{Vtran2}” 中的\autoref{Vtran2_eq2} : 设某时刻刚体上任意一点的位置矢量为 $\bvec r$, 那么该点在 $S'$ 系中速度为(\autoref{CMVD_eq5}~\upref{CMVD}) $\bvec v' = \bvec\omega'\cross\bvec r$, 在 $S$ 系中速度为 $\bvec v = \bvec\omega\cross\bvec r$。 而该时刻 $S'$ 系中位于 $\bvec r$ 的固定点相对于 $S$ 系的速度为 $\bvec v_r = \bvec\omega_r\cross\bvec r$。 于是代入 $\bvec v = \bvec v' + \bvec v_r$ 有
\begin{equation}
\bvec\omega\cross\bvec r = \bvec v' + \bvec v_r = \bvec\omega'\cross\bvec r + \bvec\omega_r\cross\bvec r
= (\bvec\omega' + \bvec\omega_r)\cross\bvec r
\end{equation}
该式对任何 $\bvec r$ 都成立, 所以\autoref{InsAx_eq1} 显然成立。

\begin{example}{进动陀螺的角速度}\label{InsAx_ex1}
\autoref{InsAx_ex2} 中, 令陀螺的对称轴在 $S'$ 系中静止, 但与 $z$ 轴正方向有一锐角夹角。 陀螺相对 $S'$ 系绕对称轴以角速度 $\bvec \omega'$ 旋转。 再令 $S'$ 绕 $S$ 关于 $z$ 轴旋转的角速度为 $\bvec \omega_r$ 指向 $z$ 轴正方向, 那么在 $S$ 系中, 陀螺的角速度为
\begin{equation}
\bvec \omega = \bvec \omega' + \bvec \omega_r
\end{equation}
其方向就是陀螺相对于 $S$ 系的瞬时转轴的方向。 根据矢量加法的平行四边形法则\upref{GVecOp}, 瞬时转轴的确比陀螺对称轴上仰一些。

如果把陀螺的进动取反方向, 那么 $\bvec \omega_r$ 则指向 $-z$ 方向, 得到的瞬时转轴就会比陀螺对称轴下倾一些。
\end{example}
