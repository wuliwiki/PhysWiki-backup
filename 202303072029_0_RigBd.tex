% 刚体
% 刚体|越疏|自由度|转动|平移

\pentry{质心\upref{CM}, 自由度\upref{DoF}}

当我们要考虑一个物体的质量分布带来的力学效应时, 就不能再将其简化为一个质点。 许多情况下我们考虑的物体在某过程中形变较小可忽略不计, 这时我们就可以忽略它运动过程中的的任何形变, 从而大大简化问题。 我们把这种模型叫做\textbf{刚体}。 在分析刚体时, 我们通常把刚体看做是质点系。 要这么做, 我们可以把刚体划分为无限多个体积无限小的微元, 再把每个微元近似为一个质量相同的质点即可。 

\subsection{刚体的自由度}

在没有任何约束的情况下, 三维空间中每个质点有 3 个\textbf{自由度}, 即用三个完全独立的变量才能完全确定位置, 所以 $N$ 个质点组成的质点系共有 $3N$ 个自由度。 然而完全确定一个刚体的位置只需要 6 个变量, 这是因为刚体模型通过假设“任意两个质点之间距离不变”, 给质点系的位置施加了 $3N - 6$ 个约束条件。 如何得出 6 个自由度呢? 我们可以假设第一个质点有 3 个自由度, 第二个质点由于要与第一个质点保持距离不变, 只有 $3 - 1 = 2$ 个自由度, 而第三个质点要与前两个质点保持距离不变, 只有 $3 - 2 = 1$ 个自由度。 有了前三个质点后(假设它们不共线), 剩下所有质点的位置都可以由与这三个质点的距离确定, 所以任何刚体都有 $3 + 2 + 1 = 6$ 个自由度。

我们也可以这么划分 6 个自由度: 令其中 3 个决定刚体上某点的位置, 2 个决定该点到刚体上另一点的矢量的方向(球坐标\upref{Sph}中的两个角度), 最后一个决定刚体绕该矢量旋转的角度。

若刚体只能绕固定点旋转, 那么刚体就只有 3 个转动自由度。 若在此基础上, 刚体只能绕固定轴旋转, 那它就只剩一个自由度了(转过的角度)。

\subsection{刚体的运动}
如果已知刚体的受力(可以关于时间变化), 如何计算刚体的运动呢? 我们可以分别计算刚体所受的所有外力的矢量和, 称为\textbf{合外力}, 以及所有外力关于刚体质心质心\upref{CM}的矢量和, 称为\textbf{合外力矩}。 然后通过质点系的动量定理\upref{PLaw}, 角动量定理角动量定理\upref{AMLaw}以及刚体的刚体的转动惯量\upref{RigRot}或者惯性张量\upref{ITensr}就可以得到刚体在每个时刻的姿态。

如果刚体在旋转过程中转轴始终保持在同一个方向, 那么计算过程将十分简单, 因为不需要使用惯性张量, 具体过程见 “刚体的平面运动方程\upref{RBEM}”。 但如果刚体转动时转轴可能发生改变, 那么就必须使用惯性张量和参考系变换, 使计算更为复杂, 详见 “刚体的运动方程\upref{RBEqM}”。
