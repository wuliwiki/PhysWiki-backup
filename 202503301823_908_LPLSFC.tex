% 拉普拉斯方程(综述)
% license CCBYSA3
% type Wiki

本文根据 CC-BY-SA 协议转载翻译自维基百科\href{https://en.wikipedia.org/wiki/Laplace\%27s_equation}{相关文章}。

在数学和物理学中,拉普拉斯方程是一个二阶偏微分方程,得名于皮埃尔-西蒙·拉普拉斯,他在1786年首次研究了其性质。通常写作:
\[
\nabla^2 f = 0~
\]
或
\[
\Delta f = 0~
\]
其中\(\Delta = \nabla \cdot \nabla = \nabla^2\)是拉普拉斯算符,\(^\text{[注1]}\)\(\nabla\cdot\)是散度算符(也表示为“div”),\(\nabla\)是梯度算符(也表示为“grad”),\(f(x, y, z)\)是一个二次可微的实值函数。因此,拉普拉斯算符将标量函数映射到另一个标量函数。”