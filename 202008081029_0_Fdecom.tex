% 力的分解合成 合力

\pentry{几何矢量\upref{GVec}}

在经典力学中, 力的分解与合成可以看作一个基本假设. 这个假设是牛顿运动定律\upref{New3}的基础.

当若干个力 $\bvec F_i$ ($i = 1, 2, \dots, N$)作用在同一个质点上时, 等效于一个力 $\bvec F$ 作用在同一个质点上.
\begin{equation}\label{Fdecom_eq1}
\bvec F = \sum_i \bvec F_i
\end{equation}

我们把 $\bvec F$ 叫做 $N$ 个 $\bvec F_i$ 的\textbf{合力}, 每个 $\bvec F_i$ 叫做一个\textbf{分力}. \autoref{Fdecom_eq1} 从左到右的过程叫做\textbf{力的分解}, 从右到左的过程叫做\textbf{力的合成}.

这里所说的 “等效” 可以指这个质点受力后的运动情况, 也可以指物体发生的形变, 例如该质点固定在弹簧上, 弹簧发生的形变.

\subsubsection{多次分解}
注意在\autoref{Fdecom_eq1} 中我们甚至可以进行多次分解, 即继续令某个(或每个)
\begin{equation}
\bvec F_i = \sum_j \bvec F_{i,j}
\end{equation}
那么一个 $\bvec F$ 就可以分解为
\begin{equation}
\bvec F = \sum_{i,j} \bvec F_{i,j}
\end{equation}
这仍然符合分解的定义, 即一个力矢量表示为多个力矢量相加, 本质上并无不同.
