% 紧束缚近似
% 紧束缚|能带理论

\pentry{金属中的自由电子气体\upref{mfcgas},布洛赫理论\upref{Bloch}}

在固体物理学中,除了用近自由电子近似\upref{egasmd}分析电子能带结构,还有一种方法,被称为\textbf{紧束缚近似}在一些场合的问题下是一个很好的近似,并且在计算和分析能带上能体现出一些优势.

在紧束缚近似模型中,一般假定电子受到了原子实很强的束缚作用,当它靠近一个原子实时,它的势场近似为库伦势.因此电子在原子实附近的行为和原子轨道波函数(类氢原子的束缚态\upref{HWF})有一定的相关性.
\begin{figure}[ht]
\centering
\includegraphics[width=8cm]{./figures/tbappx_1.png}、
\caption{一维晶格的势场} \label{tbappx_fig2}
\end{figure}


由于势场是所有原子实产生的库伦势的叠加,所以在紧束缚近似模型中,假定在原子实附近的电子主要受该原子实的库伦势的影响,而将其他的原子实的库伦势对它的作用视为微扰.原子轨道波函数的能级是分立的,第 $n$ 个轨道对应 $n^2$ 个简并的态(不考虑自旋).但如果引入其他原子实的势场微扰,这些能级就会发生劈裂,如下图所示.