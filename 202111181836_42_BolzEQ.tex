% 玻尔兹曼方程
% keys 玻尔兹曼方程

刘维尔定理\upref{LiouTh}
玻尔兹曼方程是动理学理论的奠基者路德维希 $\cdot$ 玻尔兹曼于1872年首先推导出来的,其可表示为下面的积分微分方程的形式:
\begin{equation}
\pdv{f}{t}+\bvec{v}\vdot\nabla f=\int\omega\qty(f'f_1'-ff_1)\dd\Gamma_1\dd\Gamma'\dd\Gamma_1'
\end{equation}
式中,我们用$\Gamma$ 表示分布函数所依赖的变量中除分子质心坐标(和时间 $t$ )以外的一切变量总体. $f,f'$ 是气体分子在其相空间的分布函数 $f(t,\bvec r,\Gamma)$,本文规定函数 $f$ 的附标均对应于其变量 $\Gamma$ 的附标,即$f=f(t,\bvec r,\Gamma),f'=f(t,\bvec r,\Gamma')$,等等.$\omega=\omega(\Gamma',\Gamma_1';\Gamma,\Gamma_1)$是其所有变量的函数,其对应两分子初值为 $\Gamma$ 和 $\Gamma_1$ 而结果为 $\Gamma'$ 和 $\Gamma_1'$ 的碰撞(该碰撞简记为 $\Gamma,\Gamma_1\rightarrow\Gamma',\Gamma_1'$ ). 