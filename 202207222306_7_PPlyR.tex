% 本原多项式(线性代数)
% 本原多项式|有理系数多项式|整系数多项式|线性代数|高等代数|多项式|primitive polynomial

任何一个有理系数多项式乘以一个整数,总能得到一个整系数多项式,且二者的根完全一样;类似地,任何整系数多项式,如果其系数有公共整数因子,那也可以用这个因子去除该多项式,得到的还是整系数多项式,且根不变.

综上,研究有理系数多项式的根时,可以把焦点完全集中在下述“本原多项式”上.


\begin{definition}{本原多项式}\label{PPlyR_def1}

若整系数多项式$f(x)$的各系数之最大公因子是$1$,则称$f(x)$为一个\textbf{本原多项式(primitive polynomial)}.

\end{definition}

显然,每个有理系数多项式都唯一对应一个本原多项式.首项系数为$1$的多项式(简称为首一多项式)必为本原多项式.

下面给出有关本原多项式基本性质的两个引理:

\begin{lemma}{}
若$f$和$g$都是本原多项式,则$fg$也是本原多项式.
\end{lemma}

\textbf{证明}:

设$f(x) = \sum_{i=0}^n a_ix^i$,$g(x) = \sum_{j=0}^m b_jx^j$.由\autoref{PPlyR_def1} 

\textbf{证毕}.


















