% 用傅里叶级数画画(Matlab)

\begin{issues}
\issueDraft
\end{issues}

\pentry{傅里叶级数(指数)\upref{FSExp}}

复平面上一条任意的连续曲线, 都可以看成复值函数 $z(t)$. 把它展开为傅里叶级数, 那么级数的每一项就代表复平面上的一个做圆周运动的矢量.

\begin{lstlisting}[language=matlab, caption=FFTplt]
close all;
% ==== 参数设置 ====
load logo_curve.mat x y;
Nt = 1000;
% ================
x = (x - mean(x))*481/290;
y = y - mean(y);
N = numel(x);
z = x + 1i*y;
[C, w] = FFT(z, 1, 1);
C = sqrt(2*pi)/N * C(:); w = w(:);

t = linspace(1,N,Nt);
z1 = sum(C.*exp(1i*w.*t), 1);
figure; plot(x, y); hold on;
plot(real(z1),imag(z1));

th = linspace(0, 2*pi, 200);
costh = cos(th); sinth = sin(th);
N1 = round(N/2-1);
[~, order] = sort(abs(C), 'descend');
C1 = C(order);
w1 = w(order);
z1 = sum(C1.*exp(1i*w1.*t), 1);
plot(real(z1),imag(z1));

for it = 1:Nt
    z1 = cumsum(C1.*exp(1i*w1*t(it)), 1);
    clf;
    plot(x, y); hold on; axis equal;
    for i = 1:2*N1
        A = abs(z1(i+1)-z1(i)); x0 = real(z1(i)); y0 = imag(z1(i));
        plot(x0+A*costh, y0+A*sinth, 'g');
    end
    plot(real(z1), imag(z1), '.-');
    scatter(real(z1(end)), imag(z1(end)), 'o');
    axis([-0.7, 0.7, -0.4, 0.7]);
    saveas(gcf, [num2str(it), '.png']);
end
\end{lstlisting}
