% 集合的运算
% keys 交|并|对偶原理|de Morgan定理
% license Xiao
% type Tutor

\pentry{集合\upref{Set}}
通过 “集合\upref{Set}” 的预备知识,我们应该知道,集合是由元素构成的。这是说,在一开始讨论集合的时候,必须先行给出一个集合,才能继续进行有关的讨论。这是说当你说集合 $A$ 时,我们已经知道了它的元素。具体而言,当谈论集合 $A$ 时,你应该马上将其理解成 $A=\{a_1,\cdots,a_n,\cdots\}$,其中花括号里的是 $A$ 的元素。既然提到集合相当于指出它的元素,那同时就能判断哪些使它的元素,哪些不是它的元素,$a$ 是 $A$ 的元素记作 $a\in A$,$a$ 不是 $A$ 的元素记作 $a\notin A$,这些都是在提到集合的时候就同时表明了的。有了集合后,一个重要的任务是其上运算的定义及运算规则,这便是本节要介绍的。
\subsection{与 $\land$ 和 或 $\lor$}
同样在“集合\upref{Set}”里,已经知道集合间可以进行运算 $\cap$(交)、$\cup$(并)。现在来给出它们的具体定义。严格的定义需要用到两个逻辑学概念 $\land$(读作“与”) 和 $\lor$(读作 “或”)。它们仅仅代表两个函数,下面将记 $Z_2=\{0,1\}$。
\begin{definition}{与}
称函数 $f:Z_2\times Z_2\rightarrow Z_2$ 为\textbf{与},若
\begin{equation}
f(0,0)=0,\quad f(0,1)=0,\quad f(1,0)=0,\quad f(1,1)=1~.
\end{equation}
并将与函数 $f$ 记作 $\land$。
\end{definition}
\begin{definition}{或}
称函数 $f:Z_2\times Z_2\rightarrow Z_2$ 为\textbf{或},若
\begin{equation}
f(0,0)=0,\quad f(0,1)=1,\quad f(1,0)=1,\quad f(1,1)=1~.
\end{equation}
并将或函数 $f$ 记作 $\lor$ 。
\end{definition}
\subsection{属于$\in$的数值化}
在精确定义集合的交和并前,还需要引入一种特定的函数,它将元素和集合的关系数值化了,不妨称为集合上的坐标函数。
\begin{definition}{}
设 $X$ 是集合,则称函数 $f_X$ 是集合 $X$ 上的\textbf{坐标},若
\begin{equation}
f_X(x)=\left\{\begin{aligned}
&1,\quad x\in X,\\
&0,\quad x\notin X~.
\end{aligned}\right.
\end{equation}
并称 $f_X(x)$ 是元素 $x$ 关于 $X$ 的\textbf{坐标}。
\end{definition}
元素的坐标其实就表达了元素是否属于对应的集合。