% 汇编语言笔记(GAS, x86-64)

\begin{issues}
\issueDraft
\end{issues}

参考\href{https://cs.lmu.edu/~ray/notes/gasexamples/}{这个}。

一个例子:
\begin{lstlisting}[language=none]
        .global _start

        .text
_start:
        # write(1, message, 13)
        mov     $1, %rax            # system call 1 is write
        mov     $1, %rdi            # file handle 1 is stdout
        mov     $message, %rsi      # address of string to output
        mov     $13, %rdx           # number of bytes
        syscall                     # invoke operating system to do the write

        # exit(0)
        mov     $60, %rax           # system call 60 is exit
        xor     %rdi, %rdi          # we want return code 0
        syscall                     # invoke operating system to exit
message:
        .ascii  "Hello, world\n"
\end{lstlisting}

编译: \verb|gcc -c hello.s && ld hello.o && ./a.out|, 运行: \verb|./hello.x|

\begin{itemize}
\item assembler 是汇编语言的编译器
\item 汇编语言的语法依赖于 CPU 架构和具体的 assembler。
\item \verb|gcc| 使用的 assembler 是 \textbf{GNU Assembler (GAS)}
\item 注释用 \verb|#|
\item 缩进没有意义
\item \verb|movl $4, %eax| 把 register \verb|eax| 的值设为 4。 \verb|%| 表示 register, \verb|$| 表示 immediate value, 可以用十进制也可以 16 进制 \verb|0x4|。
\item \verb|int 0x80| 开始 software interrupt, 进行 system call。
\item \verb|mov eax, 1| 是 system call \verb|exit()|
\item \verb|xor %ebx, %ebx| 可以把 register \verb|ebx| 变为 0。
\item \verb|$| 和 \verb|%| 是可以省略的
\item \verb|e?x| 是 32-bit register, \verb|r?x| 是 64-bit 的。
\item 在 32 位系统上, \verb|mov| 等效于 \verb|movl|, 64 位系统上等效于 \verb|movq|。
\item \verb|.section .text| 用于声明一个 section, text 说明这是执行程序。 \verb|.section| 可以省略。
\item \verb|syscall| 一般代替 \verb|int 0x80| 进行 systemcall, 后者已经过时了。
\item \verb|标签:| 是一个 label, 用于指定代码的某个位置, 用于跳转。
\item \verb|jnz 标签| 是条件跳转, 
\end{itemize}

\subsubsection{gdb}
\begin{itemize}
\item 编译时加上选项 \verb|-g|, 就可以用 \verb|gdb ./hello.x| 来调试程序。
\item \verb|b main| 可以在 main 的第一行设置 break point, \verb|r| 跑程序。
\item \verb|p $eax| 可以查看 register 的值。
\item 在用 gdb 调试 c/c++ 语言时, \verb|disassemble| 命令可以看到当前的汇编码。
\item 要直接把 c 编译成汇编码, 用 \verb|gcc -S main.c|, 但是生成汇编码的可读性比较差(比手写的复杂)。
\end{itemize}
