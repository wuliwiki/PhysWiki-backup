% 布拉利-福尔蒂悖论(综述)
% license CCBYSA3
% type Wiki

本文根据 CC-BY-SA 协议转载翻译自维基百科\href{https://en.wikipedia.org/wiki/Burali-Forti_paradox}{相关文章}。

在集合论中,布拉利-福尔提悖论展示了构造“所有序数的集合”会导致矛盾,因此证明了在允许构造该集合的系统中存在反论证。该悖论以切萨雷·布拉利-福尔提命名,他在1897年发表了一篇论文,证明了一个定理,这个定理在他未曾意识到的情况下,与乔治·康托尔先前证明的结果相矛盾。伯特兰·罗素随后注意到了这一矛盾,并在1903年出版的《数学原理》一书中提到了这一点,称这一矛盾是由布拉利-福尔提的论文启发的,因此该悖论以布拉利-福尔提的名字命名。
\subsection{用冯·诺依曼序数表述}
我们将通过反证法来证明这一点。

令Ω是由所有序数构成的集合。
$\Omega$是传递的,因为对于Ω的每一个元素x(它是一个序数,可以是任何序数)和x的每一个元素y(即根据冯·诺依曼序数的定义,对于每个序数y < x),我们有y是Ω的一个元素,因为根据该序数构造的定义,任何序数只包含序数。
Ω是按成员关系良序的,因为它的所有元素也都按此关系良序。
因此,根据步骤2和3,我们知道Ω是一个序数类,并且根据步骤1,它也是一个序数,因为所有作为集合的序数类也是序数。
这意味着Ω是Ω的一个元素。
根据冯·诺依曼序数的定义,$\Omega$< $\Omega$等同于$\Omega$是Ω$\Omega$的一个元素。这个结论由步骤5证明。
但是,没有任何一个序数类小于它自己,包括Ω,因为根据步骤4(Ω是一个序数类),即Ω ≮ Ω。
我们从Ω作为集合的前提推导出了两个矛盾的命题($\Omega< \Omega$ 和 Ω ≮ Ω),因此我们证明了Ω不是一个集合。



好的,下面是用LaTeX表示的公式翻译内容:

---

**Stated in terms of von Neumann ordinals**

We will prove this by contradiction.

Let $\Omega$ be a set consisting of all ordinal numbers.
$\Omega$ is transitive because for every element $x$ of $\Omega$ (which is an ordinal number and can be any ordinal number) and every element $y$ of $x$ (i.e., under the definition of von Neumann ordinals, for every ordinal number $y < x$), we have that $y$ is an element of $\Omega$ because any ordinal number contains only ordinal numbers, by the definition of this ordinal construction.
$\Omega$ is well-ordered by the membership relation because all its elements are also well-ordered by this relation.
So, by steps 2 and 3, we have that $\Omega$ is an ordinal class and also, by step 1, an ordinal number, because all ordinal classes that are sets are also ordinal numbers.
This implies that $\Omega \in \Omega$.
Under the definition of von Neumann ordinals, $\Omega < \Omega$ is the same as $\Omega \in \Omega$. This latter statement is proven by step 5.
But no ordinal class is less than itself, including $\Omega$, because of step 4 ($\Omega$ is an ordinal class), i.e. $\Omega \not\prec \Omega$.
We have deduced two contradictory propositions ($\Omega < \Omega$ and $\Omega \not\prec \Omega$) from the sethood of $\Omega$ and, therefore, disproved that $\Omega$ is a set.

---
