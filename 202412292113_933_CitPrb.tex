% 一些反直觉的概率问题
% keys 反直觉|三门|性别比|生日悖论|患病可能|贝叶斯
% license Usr
% type Wiki

概率学中涉及许多日常生活中常见的问题。然而,人类的直觉通常不适合处理概率,导致许多概率问题的结论显得与直觉不符,甚至令人惊讶。以下将介绍六个具有代表性的概率问题,这些问题通过不同的视角展示了概率的反直觉性及其背后的逻辑。

\begin{example}{抽签中的生日悖论}
在一个由 23 人组成的房间中,至少有两个人的生日(仅考虑月日,不包括年份)相同的概率是多少?
\end{example}

人们的直觉往往会认为,23 人远小于一年的天数(365 天),因此至少两人生日相同的可能性应该很低。然而,这种看似小概率的事件,实际上在人数较少时就已经接近一半,约为 $50.7\%$。

通过计算至少两人生日相同的补集(即所有人生日都不同的概率),可以间接求解问题。具体步骤如下:

假设一年有 365 天,第一个人的生日可以是任意一天,因此有 365 种可能;第二个人的生日需要与第一个不同,则有 364 种可能;第三个人的生日需与前两人不同,则有 363 种可能……依此类推。所有人生日都不同的概率为:
\begin{equation}
P_{\text{不同}} = \frac{365}{365} \cdot \frac{364}{365} \cdot \frac{363}{365} \cdots \frac{343}{365}~.
\end{equation}

至少两人生日相同的概率可以通过补集计算得到:
\begin{equation}
P(\text{相同}) = 1 - P(\text{不同})~.
\end{equation}


这一结果之所以显得反直觉,核心在于概率的“非线性增长”。人们往往直观地将概率理解为线性累加,但在实际问题中,事件之间的组合复杂性和交互性显著提升了总体概率。这种现象颠覆了人们关于概率直观增长规律的认知。

\subsection{某种生育策略下的男女比例}

\subsubsection{问题与结论}

问题:

一个村子中的人,因为特别喜欢男孩,统一采用下面的生育策略:如果生的是女孩的话,就再生一个孩子,直到生出一个男孩为止。这样许多年后,这个村子的男女比例是多少?

结论:

男女比为1:1。

\subsubsection{论证过程}


假设最大生育数量是$M$,男孩出生的可能性为$p\in[0,1]$,则有:

\begin{table}[ht]
\centering
\caption{生育可能性表}\label{tab_CitPrb1}
\begin{tabular}{|c|c|c|c|}
\hline
孩子数 & 情况的概率 & 女孩数 & 男孩数 \\
\hline
1 & $p$ & 0 & 1 \\
\hline
2 & $p(1-p)$ & 1 & 1 \\
\hline
3 & $p(1-p)^2$ & 2 & 1 \\
\hline
$\vdots$ & $\vdots$ & $\vdots$ & $\vdots$ \\
\hline
$M-1$ & $p(1-p)^{M-2}$ & $M-2$ & 1 \\
\hline
$M$ & $p(1-p)^{M-1}$ & $M-1$ & 1 \\
\hline
$M$ & $(1-p)^M$ & $M$ & 0 \\
\hline
\end{tabular}
\end{table}
则在这种生育策略下,每对家长所生女孩的个数期望:
\begin{equation}
{1-p\over p}\left(1-(1-p)^M\right)~.
\end{equation}

每对家长所生男孩的个数期望:
\begin{equation}
1-(1-p)^M~.
\end{equation}

男女比:
\begin{equation}
p\over 1-p~.
\end{equation}

每对家长所生孩子的个数期望:
\begin{equation}
{1\over p}\left(1-(1-p)^M\right)~.
\end{equation}

孩子的个数期望与出生可能性和最大生育数量相关,而生育比与限定人们生育的最大数量无关。一般认为男孩与女孩的出生可能相同,即$p=1\over2$,这时可以发现男女比是$1:1$。

最终一定是有一个男孩会出生的,预期出生的数量则是男生出生率的倒数。

问题就在于,尽管女生的数量会随着女生的出生而依次递增,但这种情况发生的概率却在指数减少。

\subsection{性别推断}

\subsubsection{问题与结论}

问题:

老王家有两个孩子,其中一个孩子是男孩,那么另一个孩子也是男孩的概率是多少?

结论:
“其中一个孩子是男孩”存在语义理解的区别:如果是随机选一个孩子知道他是男孩,则概率是$50\%$;如果是已知至少有一个男孩,则为$\displaystyle 1\over 3$。

\subsubsection{论证过程}
假设两个孩子的性别独立,每个孩子是男孩或女孩的概率都是 $\frac{1}{2}$。
家庭的可能组合为:
	•	男孩-男孩
	•	男孩-女孩
	•	女孩-男孩
	•	女孩-女孩(排除,因为至少有一个是男孩)

剩下的三种可能性中,只有一种是两个都是男孩。因此概率是 $\frac{1}{3}$。

\subsection{三门问题}

\subsubsection{问题与结论}

问题:

在一个游戏中,你面前有三个门,其中一扇后面有车,另外两扇后面是山羊。选手选定一个门后,主持人会打开另一个有山羊的门,并问选手是否愿意换门。请问:换门是否会增加赢得汽车的概率?

结论:

换门会增加赢得汽车的概率。

\subsubsection{论证过程}


解析:
	•	初始选门时,选到汽车的概率是 $\frac{1}{3}$,选到山羊的概率是 $\frac{2}{3}$。
	•	主持人打开一扇山羊的门后,如果最初选的是山羊(概率 $\frac{2}{3}$),换门必定赢。
	•	换门的成功概率是 $\frac{2}{3}$,不换的成功概率是 $\frac{1}{3}$。

\subsection{贝叶斯医生问题}

\subsubsection{问题与结论}

问题:

假设有一种疾病,患病概率是 $1\%$,诊断测试的准确率为 $99\%$(即真阳性率和真阴性率均为 $99\%$)。现在一个随机人测试结果为阳性,他实际患病的概率是多少?

结论:

约 $50\%$。

\subsubsection{论证过程}
    
运用贝叶斯定理:
	•	患病的先验概率:$P(\text{患病}) = 0.01$。
	•	未患病的先验概率:$P(\text{未患病}) = 0.99$。
	•	阳性结果的条件概率:
	•	患病者阳性概率 $P(\text{阳性}|\text{患病}) = 0.99$。
	•	未患病者阳性概率 $P(\text{阳性}|\text{未患病}) = 0.01$。

通过贝叶斯公式计算患病的后验概率:
$$
P(\text{患病}|\text{阳性}) = \frac{P(\text{阳性}|\text{患病}) \cdot P(\text{患病})}{P(\text{阳性})}~.
$$

总阳性概率 $P(\text{阳性})$ 为:
$$
P(\text{阳性}) = P(\text{阳性}|\text{患病}) \cdot P(\text{患病}) + P(\text{阳性}|\text{未患病}) \cdot P(\text{未患病})~.
$$

代入计算得实际患病概率接近 $50\%$。

\subsubsection{启发}

患病率过低的病,即使是阳性,患病率也并不高。要求检测假阳性的概率必须与患病率接近。实际生活中并不会随机选取一个病人,而是通过医生诊断,来提升做检测的人患病的概率,从而保证结论可靠。


\subsection{圆内随机选点的直径问题}

\subsubsection{问题与结论}
问题:
在一个圆内随机选择两个点,连接它们构成一条线段。请问这条线段是圆直径的概率是多少?

直觉分析

很多人会认为,这条线段成为直径的概率非常低,因为直径是一条特殊的直线,随机选点后能正好碰到直径似乎是小概率事件。

然而,实际概率的结果和选点的方式密切相关!这是一个非常经典且反直觉的问题。


\subsubsection{论证过程}
解答方式 1:几何概率分析

要使一条线段成为直径,两点必须刚好在圆的直径两端。关键在于:
	•	随机选点“如何定义”?
不同的“随机”选法会导致完全不同的概率结果。

以下是两种常见的选法及其概率计算:

方式一:点均匀分布在圆内
	•	在圆内任意选择两个点,假设点是均匀分布的。
	•	对于圆的任意一条直径,只有当两点恰好位于直径的两端时,这条线段才是直径。

在这种分布下,概率为 0。
因为随机选点时,两点恰好位于直径两端的可能性是无穷小的(属于测度为 0 的事件)。

方式二:点均匀分布在圆周上
	•	在圆周上随机选两个点,构成的线段是直径的条件是:两点相隔 $180^\circ$。
	•	圆周上的两点位置可以用角度描述,第一点固定后,第二点均匀分布在 $[0^\circ, 360^\circ)$ 上。

两点相隔 $180^\circ$ 的概率为:
$$
P = \frac{1}{360^\circ} \cdot 2 = \frac{1}{2}~.
$$
所以,在这种情况下,概率为 $\frac{1}{2}$。

结论
	1.	如果点均匀分布在圆内:概率为 $0$(几何意义上)。
	2.	如果点均匀分布在圆周上:概率为 $\frac{1}{2}$。

这是一个很经典的概率问题,明确随机分布的方式是关键!如果还有疑问,欢迎继续讨论~