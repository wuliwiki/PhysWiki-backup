% 柯西序列的等价
% keys 柯西序列|等价
% license Usr
% type Tutor

\pentry{柯西序列、完备度量空间\nref{nod_cauchy},二元关系\nref{nod_Relat}}{nod_70ae}
在柯西序列上可以定义等价关系(\autoref{def_Relat_1}),而通过等价关系可以从原集合得到一个商集,下面将表明,这个由柯西序列得到的商集也具有度量空间的结构。

\begin{theorem}{}
设 $(X,d)$ 是任一度量空间, $\{x_n\},\{x'_n\}$ 是 $X$ 上的两个柯西序列(\autoref{def_cauchy_3})。定义柯西序列上的二元关系 $R$:即若
\begin{equation}
\lim_{n\rightarrow\infty}d(x_n,x_n')=0,~
\end{equation}
则称 $\{x_n\},\{x_n'\}$ 具有关系 $R$,并记为 $\{x_n\}R\{x_n'\}$。那么关系 $R$ 是等价关系。
\end{theorem}
\textbf{证明:}
自反性 $\{x_n\}R\{x_n\}$:由于 $d(x_n,x_n)=0$ 对所有 $n$ 成立,因此 $\lim\limits_{n\rightarrow\infty}d(x_n,x_n)=0$ 成立。

对称性 $\{x_n\}R\{x_n'\}\Rightarrow\{x_n'\}R\{x_n\}$:由 $d$ 的对称性,$d(x_n,x_n')=d(x_n',x_n)$,因此由 $\{x_n\}R\{x_n'\}$ 得 
\begin{equation}
\lim_{n\rightarrow\infty}d(x_n',x_n)=\lim_{n\rightarrow\infty}d(x_n,x_n')=0.~
\end{equation}

传递性 $\{x_n\}R\{x_n'\},\{x'_n\}R\{y_n\}\Rightarrow\{x_n\}R\{y_n\}$:因为 $\{x_n\}R\{x_n'\},\{x'_n\}R\{y_n\}$,所以对任一 $\epsilon>0$,存在 $N,N'$,使得只要 $n\geq N,n'\geq N'$,就有
\begin{equation}
d(x_n,x_n')<\epsilon/2,\quad d(x_{n'}',y_{n'})<\epsilon/2.~
\end{equation}
取 $N_1=\max\{N,N'\}$,则只要 $n>N_1$,就有
\begin{equation}
d(x_n,y_n)<d(x_n,x'_n)+d(x'_n,y_n)<\epsilon/2+\epsilon/2=\epsilon.~
\end{equation}
即 $\lim\limits_{n\rightarrow\infty}d(x_n,y_n)=0$。

\textbf{证毕!}

由上面的定理,可以定义柯西序列的等价关系。
\begin{definition}{}
设 $(X,d)$ 是任一度量空间, $\{x_n\},\{x'_n\}$ 是 $X$ 上的两个柯西序列(\autoref{def_cauchy_3})。称 $\{x_n\}$ 和 $\{x_n'\}$ 是\textbf{等价}的,并记作 $\{x_n\}\sim\{x'_n\}$ 若
\begin{equation}
\lim_{n\rightarrow\infty}d(x_n,x_n')=0.~
\end{equation}
\end{definition}

\begin{theorem}{}
设 $C_X$ 是度量空间 $X$ 上的所有柯西序列构成的集族,则由等价关系 $\sim$ 定义的商空间 $X^*:=C_X/\sim$ 上通过如下定义的距离 $d^*$ 是良好的:
\begin{equation}
d^*(x^*,y^*):=\lim_{n\rightarrow\infty} d(x_n,y_n),\quad x^*,y^*\in X^*,\{x_n\}\in x^*,\{y_n\}\in y^*.~
\end{equation}
即 $d^*(x^*,y^*)$ 的取值不依赖于 $\{x_n\},\{y_n\}$ 的选取且存在。
\end{theorem}
\textbf{证明:}由不等式\autoref{eq_ConIso_1} 以及序列 $\{x_n\},\{y_n\}$ 是柯西序列知,对于充分大的 $n,m$,成立
\begin{equation}
\abs{d(x_n,y_n)-d(x_m,y_m)}\leq d(x_n,x_m)+d(y_n,y_m)<\epsilon.~
\end{equation}
于是序列 $\{s_n\}$ 是实数构成的柯西序列,其中 $s_n:=d(x_n,y_n)$。由Cauchy准则(\autoref{the_CauSeq_1}),$\{s_n\}$ 有极限。这个极限不依赖于 $\{x_n\}\in x^*,\{y_n\}\in y^*$ 的选取。事实上,设
\begin{equation}
\{x_n\},\{x_n'\}\in x^*,\quad \
\end{equation}



\textbf{证毕!}





