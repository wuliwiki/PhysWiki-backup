% 堆
% 堆|数据结构|C++

堆是一个以一颗完全二叉树的结构存储值的数据结构.

堆分为\textbf{大根堆}和\textbf{小根堆},大根堆的意思是,父节点的值都大于左右两个儿子的值,小根堆则相反.C++ STL 中的 \verb|priority_queue| 优先队列默认就是大根堆.

以一个例题来讲解堆.

维护一个集合,初始时集合为空,支持如下几种操作:
\begin{enumerate}
\item 插入一个数 $x$;
\item 输出当前集合中的最小值;
\item 删除当前集合中的最小值;
\item 删除第 $k$ 个插入的数;
\item 修改第 $k$ 个插入的数,将其变为 $x$.
\end{enumerate}

C++ STL 中的堆只能实现前 $3$ 个操作.
\begin{lstlisting}[language=cpp]
priority_queue<int, vector<int>, greater<int>> heap;   // 定义小根堆的方式
int t;
cin >> t;
while (t -- )
{
    string s;
    int x;
    cin >> s;
            
    cin >> x;
    heap.push(x);   // 插入一个数 x

    cout << heap.top() << endl;     // 输出最小值,即栈顶
    heap.pop();     // 删除最小值,即删除栈顶
}
\end{lstlisting}

要想实现随机删除和修改,只能用数组来模拟堆,所以我们讲解一下如何使用数组模拟堆.

首先需要一个数组 \verb|h[M]| 用于存储堆中的元素,由于需要在任意位置进行删除和修改,所以需要多开两个数组 \verb|ph[N]| 和 \verb|hp[N]|,