% 函数(高中)
% keys 函数|定义域|值域|二元函数
% license Usr
% type Tutor

\pentry{集合(高中)\nref{nod_HsSet},集合的基本关系(高中)\nref{nod_HsSeOp}}{nod_19dc}

\begin{issues}
\issueDraft
\end{issues}

% 高中的 函数 不应该需要 映射 作为预备知识
% 本篇文章的预设读者是,对于函数的定义感到熟悉但又有些模糊,希望进一步了解函数的高中学生。

在小学学习数学时,老师会先教你数数,然后引导你学习加减法等基本运算。到了初中,老师会让你理解边、角等几何的基本概念,再逐步带领你探索平面图形内部的线与角之间的关系。在高中阶段,主要的研究基础就是实数集。通过前面的学习,相信你已经掌握了集合和元素的知识,积累了一些理解新概念的经验,是时候开始探讨数集之间的关系——函数。函数将成为你理解和分析数学问题的重要工具。

\subsection{接触函数}

其实,你在初中时就已经接触过函数了。你或许还记得正比例函数、反比例函数以及二次函数。在初中的数学学习中,主要关注这三个具体函数的性质。

例如,




其实,初中时你就已经接触过函数了,你应该学过正比例函数,反比例函数,还有二次函数在初中时学习的函数主要是针对每一个具体的函数研究它的性质,比如正比例函数是一条直线,反比例函数是两条双曲线,然后你还知道反比例函数在的表达式,然后根据你知道带一个点可以进去求职,然后还有二次函数,有一些有两个他和X轴有两个焦点,他有一个对称轴,他还有一个最低点这个时候。

我想你先回想一下最开始接触函数的概念的时候是怎么做的,那个时候老师给了你一个表达式,然后给了你一个表格,然后让你用这个表达式带入X的值去算每一个的值,然后再在平面直角坐标系上个序描出这些点的值,最后再把它们连起来形成一个函数的图像同时把这个表达式认为就是函数的图像,现在回忆结束,让我们来看一看高中里面是怎样看待函数的。

高中里面的函数与初中最大的不同,就是最直接的不同,就是它把它的表达式不再具体的写了出来,而是用一个符号F ( X来代替这个( X 70表达的是一个含有X的表达式的意思,这种表示方法在具有参数和带入参数的时候特别的好用,比如:



函数最开始是用于研究曲线的工具。高中时期函数往往出现在平面直角坐标系上,不同的函数就对应着不同的曲线,你也许有些困惑——说不清二者之间的区别和联系。比如$f(x)=x^2$,对于每一个$x_1$,都能够找到对应的$f(x_1)$,进而得到平面上对应的点$(x_1,f(x_1))$,这是它们的联系。而区别在于,不应认为曲线就是我们口中的函数,只能认为,一个函数可以确定一条曲线。

在高中时期,一个函数通常指的是一个计算式,输入一个数字,然后通过计算式计算以输出一个数字。数学家们在上述内涵中,进一步地抽象得到更加广泛的定义,以此让函数这一概念能够更加精确、更加广泛的描述事物。

我们可以将函数认作是对输入和输出的关系的描述,我们所见到的\textsl{计算式}就是告诉我们如果将输入得到输出的方法。用工厂来比喻,我们向自动化工厂的入料口中添加原材料,经过一系列复杂的加工得到了产品,其中的加工方法、流程就可以粗略地看作是一个函数。又或者更加具体的说,让若干水果变成水果沙拉的函数是,一份水果沙拉的制作方法+我们灵巧的双手。在原有的计算式的理解上更进一步,得到了计算机学科中对函数的理解。
\begin{enumerate}
\item 从内容上,我们将输入、输出从数字拓宽到了更多事物
\item 从数量上,我们不局限于单一输入,而是能够同时输入多种事物
\end{enumerate}

在数学中,我们将函数的概念进一步抽象。抛开对象之间如何实现转换的过程,而仅仅在两个事物之间建立对应关系——将集合中的全体元素,向另一个集合中的元素建立对应法则,一个对应法则就称为一个函数。
$$f:\{1,2,3,4\}\to\{1,2,3,4,5\}~.$$
其中具体的对应法则为$f(1)=2,f(2)=3,f(3)=1,f(4)=1$,这样我们就打造了一个函数$f$。其中左侧的集合$\{1,2,3,4\}$称为\textbf{定义域},右侧的集合$\{1,2,3,4,5\}$称为\textbf{上域},全体定义域经过对应关系得到的全体元素的集合为$\{1,2,3\}$,称为\textbf{值域或像}。

\begin{figure}[ht]
\centering
\includegraphics[width=5cm]{./figures/720d887f9539bb73.png}
\caption{函数} \label{fig_functi_1}
\end{figure}
\begin{definition}{函数}
对于两个非空数集$X$和$Y$,$f$是一个二元关系,如果对于每个$x\in X$都存在唯一的$y\in Y$与它对应,则称$f$是一个定义在$X$上的\textbf{函数}(function),记作:
\begin{equation}
y=f(x)~.
\end{equation}

\end{definition}



初等函数
性质好


\subsection{函数的表示方法}

你应该还记得集合的表示方法分为枚举法、描述法和图示法,这三者各有优势,需要根据情况使用。函数也一样。下面是函数的\addTODO{几}种表示方法

\subsubsection{表达式法}

通常最关键的区别是f这个表达式。所以一般也会直接用f来代指函数,即直接记作f(x)。

\subsubsection{图像法}

\subsection{函数的性质}

函数具有一些性质,有一些在高中会接触到,有一些不会接触到。
以后我们会看到一些用\enref{极限}{Lim}和\enref{导数}{Der}描述的性质。 例如\enref{连续性}{contin}, 一致连续 % \addTODO{链接}
, 可导。

\subsection{复合函数}
\addTODO{复合函数}

\subsection{*映射}

请注意,这一部分的内容已经完全从高中数学中删除,但由于其地位重要及为了了解一些相关概念,这里介绍一下映射及相关的概念。其实,刚才学习的函数特指的是数集之间的关系,如果不限定集合中的元素是数字的话,就可以把“函数”的概念推广,得到“映射”。

\begin{definition}{映射}
,称为\textbf{映射}(map),记作:
\begin{equation}
f:A\to B\qquad\text{或者}\qquad f:a\mapsto b~.
\end{equation}
其中:
\begin{itemize}
\item A称为定义域\item a称为原像\item B称为值域\item b称为像
\end{itemize}
\end{definition}

由于所有的数集都是集合,所以函数都是映射,那么也可以用映射的记号表示函数(高中阶段不使用)。例如:$f: \mathbb R \to \mathbb R$表示一个定义域和值域都为实数集的函数;$f:x \mapsto y$表示一个自变量为$x$,函数值为$y$的函数。

根据定义,映射已经限定了所有的$x$都必须能找到对应的值,因此从映射的结果看,$Y$有四种可能:
\begin{itemize}
\item 存在一些$y$是没有对应(0个)的$x$的,剩下都只对应一个(单射)
\item 每个$y$都只对应一个(1个)$x$的(双射)
\item 存在一些$y$对应了好几个(多个)$x$的,剩下都只对应一个(满射)
\item 既存在一些$y$是没有对应的$x$的,又存在一些$y$对应了好几个$x$的
\end{itemize}

由于前三个比较容易研究,因此分别称为单射、双射、满射,理解清楚会对反函数的学习有帮助。下面给出这三个概念的具体定义,如果不理解,只参照刚才的文字或后面给出的图片形象记忆就可以了,高中这里并不要求掌握每个概念。

\begin{definition}{单射、满射、双射}
设$f:A\to{B}$,若:
\begin{itemize}
\item 对任意$a_1,a_2\in{A}$,$a_1\not={a_2}$,都有$f(a_1)\not={f(a_2)}$,则称$f$是一个\textbf{单射}(injective)。
\item 任意$b\in{B}$,都存在$a\in{A}$,使得$f(a)=b$。则称$f$是一个\textbf{满射}(surjective)。
\item $f$既是单射又是满射,则称$f$是一个\textbf{双射}或者\textbf{一一对应}。
\end{itemize}
\end{definition}

\begin{figure}[ht]
\centering
\includegraphics[width=3cm]{./figures/f6cfb71bb0c378ef.png}
\caption{单射}\label{fig_functi_2}
\end{figure}
\begin{figure}[ht]
\centering
\includegraphics[width=3cm]{./figures/3031fee516997db1.png}
\caption{满射} \label{fig_functi_3}
\end{figure}
\begin{figure}[ht]
\centering
\includegraphics[width=3cm]{./figures/9fef429051c64955.png}
\caption{双射} \label{fig_functi_4}
\end{figure}
\subsection{总结}

阅读完上面的内容,相信你已经开始对函数这个新朋友有一些熟悉的感觉了。实际上,函数的概念涵盖了非常广泛的领域,它不仅是一个数学工具,更代表了一种深刻的思维方式——找寻基本的单元,研究彼此的关系\footnote{更底层的抽象是两个元素之间或两个集合之间的“关系”。注意这里的“关系”并非语言上的泛指,而有严格的数学定义。函数是一种特殊的关系,而由于函数的概念大多数人都接触过,而且函数的英语本身也有功能的意思,有很多场合用函数来代指映射、变换等并非函数的关系,但这些概念高中不会接触到,此处只是提示,希望你在未来遇到时注意。}。而因为数字是我们最为熟悉的元素,数字构成的集合(数集)也是我们最为熟悉的集合,所以研究数集之间关系的函数在人们的认知中占据了特别重要的地位。函数这种概念不仅限于数学中的数集,它在各种领域中都有着广泛的应用。

从现在开始,直到你未来的大学、研究生学习,你都会遇到各种各样不同类型的函数。无论是在物理、经济学还是计算机科学中,函数都是用来描述复杂关系的核心。函数不仅仅是一个数学符号,它是一种让你能够用更抽象、更广阔的视角去理解世界的方法。通过将一个特定的过程或现象抽象成一个函数,并利用函数的性质去分析和预测其行为,你可以更深入地理解事物的动态变化。在生活中,你可能会发现,很多问题都可以通过函数的视角来处理和解决。这种方法将成为你未来学习和工作的基石,是帮助你分析、理解和解决问题的强大工具。
