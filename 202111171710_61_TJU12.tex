% 天津大学 2012 年考研量子力学
% 考研|天津大学|量子力学|2012

\subsection{30分}
\begin{enumerate}
\item 频率为$\omega$的谐振子处于状态$\varPsi (x)=\frac{1}{\sqrt{5}}\varPsi_{0} (x)-\sqrt{\frac{2}{5}}\varPsi_{1} (x)-A\varPsi_2 (x)$,将其归一化,并求能量平均值.
\item 波函数为什么可以归一化.
\item 三维空间转子的哈密顿量是$H=\frac{L^{2}}{2I}$,能量简并度是多少?
\end{enumerate}
\subsection{30分}
能量为$m$的粒子,处在区间$[0,a]$无限深势阱中运动,归一化函数为$\varPsi (x)=x(x-a)\sqrt{\frac{30}{a^{5}}}$.\\
(1)计算粒子处于某个本征态的概率;\\
(2)写出任一时刻的波函数.
\subsection{30分}
一个体系的哈密顿量是$H=[p^{2}_{x}+(p_{y}-qB_{x}^{2})+p^{2}_{z}]/2m$,其中$p_{x}$、$p_{y}$、$p_{z}$分别为三个方向动量算符,求体系的能级和波函数.
\subsection{30分}
非简谐振子哈密顿量是$H=H_{0}+H'$,$H_{0}=\frac{p^{2}}{2m}+\frac{1}{2}m\omega^{2}x^{2}$,$H'=\lambda x^{2}$,求微扰哈密顿量的矩阵元和能量一级近似.
\subsection{30分}
中子的自旋为$\frac{1}{2}$,磁矩可写为$gS$,$S$为自旋算符.
\begin{enumerate}
\item 一个中子在$x$方向的外磁场$B$中运动,求体系的能级和波函数;
\item 若三个中子的哈密顿量是$A(S_{1}+S_{2})\cdot S_{3}$,其中$A$为常数,$S_{1}$、$S_{2}$、$S_{3}$是三个中子的自旋角动量,求体系能级.
\end{enumerate}