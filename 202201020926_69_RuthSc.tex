% 卢瑟福散射
% 碰撞参量|双曲线|经典散射

\begin{issues}
\issueDraft
\end{issues}

\pentry{开普勒问题\upref{CelBd}, 散射\upref{Scater}}

卢瑟福通过用 $\alpha$ 粒子轰击金箔,否定了汤姆孙的葡萄干面包模型.卢瑟福惊讶地发现,每 20000 个粒子中,有 1 个 $\alpha$ 粒子会被反弹回去,这是汤姆孙的理论无法解释的.根据卢瑟福的设想,原子内部应该有一个体积很小的区域聚集了所有的正电荷,而负电荷则围绕着它在转,这使得 $\alpha$ 粒子轰击金箔这个两体问题的相互作用可以近似为库仑相互作用.卢瑟福因此提出原子的行星轨道模型,为原子结构的研究作出巨大贡献,于1908年获得诺贝尔化学奖.
\subsection{概要}
定义\textbf{碰撞参量}为双曲线的渐近线到焦点的距离.在卢瑟福散射问题中,双曲线是粒子 2 的轨迹,焦点则是粒子 1 的位置.(在随粒子 1 平动的参考系中)设粒子 2 从无穷远的距离以 $\bvec v_0$ 的速度靠近粒子 1,开始运动所在直线(即渐近线)与粒子 2 的距离为 $b$ ,这就是\textbf{碰撞参量}).由双曲线的性质,碰撞参量等于双曲线的参数 $b$.

对于经典散射问题,\textbf{微分截面}\upref{Scater}等于
\begin{equation}
\dv{\sigma}{\Omega} = \frac{b \dd{b}\dd{\phi} }{\sin \theta \dd{\theta} \dd{\phi} } = \frac{b}{\sin \theta }\dv{b}{\theta}
\end{equation}
由双曲线性质,偏射角满足
\begin{equation}
\cot{\frac{\theta }{2}}= \frac{b}{a}
\end{equation}
由反开普勒问题  $E = kQq/(2a)$,消去 $a$ 得
\begin{equation}
b = \frac{kQq}{2E}\cot {\frac{\theta }{2}}
\end{equation}
求导代入微分截面得
\begin{equation}\label{RuthSc_eq1}
\dv{\sigma}{\Omega} = \qty[ \frac{kQq}{4E \sin[2](\theta /2)} ]^2
\end{equation}

\subsection{推导}
\pentry{开普勒第一定律的证明\upref{Keple1},轨道方程、比耐公式\upref{Binet}}
在上面的推导中我们利用了轨道形状为双曲线这一事实.在这里我们用比耐公式\autoref{Binet_eq3}~\upref{Binet}给出推导.设粒子 1 质量 $m_1$,带电荷 $q_1$;粒子 2 质量 $m_2$,带电荷 $q_2$.在粒子 1 的参考系中考察粒子 2 的运动,并设约化质量 $\mu=m_1m_2/(m_1+m_2)$.两个粒子都是带正电荷的粒子,那么它们之间就有排斥力,相互作用势为
\begin{equation}
V(\rho)=\frac{q_1q_2}{4\pi\epsilon_0\rho}=\frac{k}{\rho}
\end{equation}
相互作用势与 $\rho$ 成反比,因此还可以用求解开普勒问题的方法计算.不同的是这里 $k$ 没有负号.
由比耐公式(一阶微分方程形式):
\begin{equation}
\begin{aligned}
&\qty(\frac{\dd u}{\dd \phi})^2=\frac{2\mu}{J^2}\qty(E-V(1/u)))-u^2
\end{aligned}
\end{equation}
可以得到
\begin{equation}
\begin{aligned}
\left|\frac{\dd u}{\dd \phi}\right|&=\sqrt{-u^2-\frac{2\mu k}{J^2}u+\frac{2\mu E}{J^2}}\\
&=\sqrt{-\qty(u+\frac{\mu k}{J^2})^2+\frac{2\mu E}{J^2}+\frac{\mu^2 k^2}{J^4}}
\end{aligned}
\end{equation}
该一阶偏微分方程的解的形式为
\begin{equation}
u+\frac{\mu k}{J^2}=\alpha\cos(\phi-\beta)
\end{equation}
可以解得
\begin{equation}
\begin{aligned}
\alpha=\sqrt{\frac{2\mu E}{J^2}+\frac{\mu^2 k^2}{J^4}}\\
\end{aligned}
\end{equation}
这样就求得了 $u$ 关于 $\phi$ 的表达式.最后将 $u$ 用 $\rho=1/u$ 表示,得到
\begin{equation}
\rho=\frac{p}{1+e\cos(\phi-\beta)}
\end{equation}
其中 $e$ 为轨道的偏心率(或者称离心率).$p,e$ 由下式给出:
\begin{equation}
\begin{aligned}
&p=-\frac{J^2}{\mu k}\\
&e=-\frac{J^2}{\mu k}\alpha=-\sqrt{1+\frac{2J^2E}{\mu k^2}}
\end{aligned}
\end{equation}
\begin{equation}
\rho=\frac{p}{1+e\cos(\phi-\beta)}=\frac{J^2/\mu k}{\sqrt{1+2J^2E/\mu k^2}\cos(\phi-\beta)-1}
\end{equation}
为使分母 $>0$,$\phi$ 有一定取值范围:
\begin{equation}
\cos(\phi-\beta)>\frac{1}{|e|}
\Rightarrow \phi \in (\phi_1,\phi_2)
\end{equation}
粒子从无穷远飞来,到与另一粒子距离极小时开始返回,再飞回无穷远.轨道的形状为双曲线.\textbf{散射角}(粒子的偏转角度)$\theta$ 满足为
\begin{equation}
\cot\frac{\theta}{2}=\tan\frac{|\phi_1-\phi_2|}{2} =\sqrt{e^2-1}=\sqrt{\frac{2J^2E}{\mu k^2}}
\end{equation}

定义\textbf{碰撞参量}为双曲线的渐近线到焦点的距离.在卢瑟福散射问题中,双曲线是粒子 2 的轨迹,焦点则是粒子 1 的位置.(在随粒子 1 平动的参考系中)设粒子 2 从无穷远的距离以 $\bvec v_0$ 的速度靠近粒子 1,开始运动所在直线(即渐近线)与粒子 2 的距离为 $b$ ,这就是\textbf{碰撞参量}),那么角动量为 $J=\mu bv_0$.能量为 $E=\mu v_0^2/2$.因此可以得到散射角 $\theta$ 与 $b,v_0$ 的关系式:
\begin{equation}\label{RuthSc_eq2}
\cot \frac{\theta}{2}=\frac{\mu bv_0^2}{k}
\end{equation}
如果发射大量相同速度粒子 2,探测被粒子 1 “弹”回的各个方向上的粒子数——不同的碰撞距离将导致不同的散射角.微分散射截面的信息往往反应粒子间相互作用的信息,以帮助人们对粒子的内部结构进行猜测,或对已有的猜想进行实验验证.根据微分散射截面的定义(散射\upref{Scater}),$\dd \Omega=2\pi \sin\theta \dd \theta$,$\dd \sigma=2\pi b\dd b$,有
\begin{equation}
\frac{\dd \sigma}{\dd \Omega}=\frac{b}{\sin \theta}\left|\frac{\dd b}{\dd \theta}\right|
\end{equation}
对\autoref{RuthSc_eq2} 两边微分,得到 $\dd b$ 和 $\dd \theta$ 的关系:
\begin{equation}
-\frac{\dd \theta/2}{\sin^2(\theta/2)}=\mu v_0^2\dd b/k\Rightarrow \left|\frac{\dd b}{\dd \theta}\right|=\frac{k}{2\mu v_0^2\sin^2(\theta/2)}
\end{equation}
那么
\begin{equation}
\begin{aligned}
\frac{\dd \sigma}{\dd \Omega}&=\frac{\cot(\theta/2)k/(\mu v_0^2)}{\sin\theta}\cdot \frac{k}{2\mu v_0^2\sin^2(\theta/2)}=\frac{k^2}{4\mu ^2v_0^4\sin^4(\theta/2)}
\\
&=\frac{k^2}{16E^2\sin^4(\theta/2)}
\end{aligned}
\end{equation}
其中 $k=k_eq_1q_2=q_1q_2/(4\pi\epsilon_0)$.
