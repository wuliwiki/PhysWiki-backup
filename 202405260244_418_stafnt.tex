% 标准型(分解)与唯一分解定理
% keys 数论|标准型|分解|唯一分解定理|算数基本定理|欧几里得第一定理|Euclid 第一定理
% license Usr
% type Tutor

\pentry{素数与合数\nref{nod_prmnt}}{nod_8c50}
我们对素数与合数一节中讨论到的\autoref{the_prmnt_1}~\upref{prmnt}中的分解结果 $n = p_1 \times p_2\times \cdots \times p_{m+1}$,进行合并,使得相同的 $p_i$ 表示为次方的形式,就得到了 $n$ 的标准型。
\begin{definition}{标准型}
对于任意大于 $1$ 的正整数 $n$,将其表示为各个素数的次方的乘积的形式
\begin{equation}
n = p_1^{a_1} \times p_2^{a_2} \times \cdots \times p_k^{a_k} ~ (a_1 > 0, a_2 > 0 , \dots, p_1 < p_2<\dots) ~,
\end{equation}
其中各个 $p$ 都是素数。这就称 $n$ 被表示为了\textbf{标准型(standard form)},而这个乘积表示 $p_1^{a_1} \times p_2^{a_2} \times \cdots \times p_k^{a_k}$ 就称为 $n$ 的\textbf{标准型}。
\end{definition}

而标准型是唯一的,这被是唯一分解定理。
\begin{theorem}{唯一分解定理}
对于任意大于 $1$ 的正整数,其标准型都是唯一的。即忽略各素数顺序的前提下, $n$ 只能用唯一一种方式表示为各个素数的乘积。
\end{theorem}

而唯一分解定理是欧几里得第一定理的推论。
\begin{theorem}{欧几里得第一定理}

\end{theorem}