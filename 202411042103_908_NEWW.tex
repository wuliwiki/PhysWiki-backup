% 牛顿万有引力定律(综述)
% license CCBYSA3
% type Wiki


本文根据 CC-BY-SA 协议转载翻译自维基百科\href{https://en.wikipedia.org/wiki/Newton\%27s_law_of_universal_gravitation}{相关文章}。

牛顿的万有引力定律指出,宇宙中的每一个粒子都以一种力吸引着其他粒子,这种力与它们质量的乘积成正比,与它们中心之间距离的平方成反比。相隔的物体互相吸引,就好像它们的所有质量都集中在它们的中心一样。该定律的发表被称为“第一次伟大统一”,因为它标志着地球上重力现象与已知的天文行为的统一。[1][2][3]

这是一个从经验观察中通过艾萨克·牛顿所称的归纳推理得出的普遍物理定律。[4] 它是经典力学的一部分,最早由牛顿在其著作《自然哲学的数学原理》(简称《原理》)中提出,该书首次出版于1687年7月5日。

因此,万有引力的方程形式为:
\[
F = G \frac{{m_1 m_2}}{{r^2}}~
\]
其中,\( F \) 是作用在两个物体之间的引力,\( m_1 \) 和 \( m_2 \) 是两个物体的质量,\( r \) 是它们质心之间的距离,\( G \) 是引力常数。

对牛顿万有引力定律的首次实验室测试是由英国科学家亨利·卡文迪许在1798年进行的卡文迪许实验。[5] 这次实验发生在牛顿的《原理》发表111年后,大约在他去世71年后。

牛顿的引力定律与库仑的电力定律相似,后者用于计算两个带电体之间产生的电力的大小。它们都是平方反比定律,即力与两个物体之间距离的平方成反比。库仑定律中用电荷代替质量,并使用不同的常数。

牛顿的引力定律后来被阿尔伯特·爱因斯坦的广义相对论所取代,但引力常数的普适性依然存在,而且该定律在大多数应用中仍被用作计算引力效应的极好近似。只有在需要极高精度时,或在处理非常强的引力场(如在极其巨大和密集的物体附近)或小距离(例如水星绕太阳的轨道)时,才需要使用相对论。
\subsection{历史}   
大约在1600年,科学方法开始扎根。勒内·笛卡尔从一个更基础的视角重新出发,发展了独立于神学的物质和作用的观念。伽利略·伽利莱写下了关于自由落体和滚动物体的实验测量的著作。约翰内斯·开普勒的行星运动定律总结了第谷·布拉赫的天文观测。[6]: 132  

大约在1666年,艾萨克·牛顿提出了开普勒定律也必须适用于月亮围绕地球的轨道,进而适用于地球上的所有物体的想法。该分析需要假设引力作用如同地球的所有质量集中在其中心一样,这是当时未经证实的猜想。他对月球轨道周期的计算值与已知值相差16\%。到1680年,地球直径的新测量值使他的轨道周期计算值与已知值相差缩小至1.6\%,更重要的是,牛顿找到了对他之前猜想的证明。[7]: 201  

1687年,牛顿发表了他的《自然哲学的数学原理》(Principia),将他的运动定律与新的数学分析相结合,解释了开普勒的经验结果。[6]: 134 他的解释形式是一条万有引力定律:任何两个物体之间的吸引力与它们的质量成正比,与它们的距离平方成反比。[8]: 28 牛顿的原始公式为:
\[
\text{重力} \propto \frac{\text{物体1的质量} \times \text{物体2的质量}}{\text{中心间距离}^2}~
\]
其中符号 ∝ 表示“成比例”。要将其转化为一个等式,需要一个乘数或常数,以在任何质量值或距离下给出正确的引力大小(即引力常数)。牛顿需要准确测量这个常数以证明他的平方反比定律。当牛顿在1686年4月向皇家学会提交尚未出版的《自然哲学的数学原理》第一卷时,罗伯特·胡克声称牛顿是从他那里获得了平方反比定律,最终被认为是无稽之谈。[7]: 204
\subsubsection{牛顿“至今未知的原因”}
虽然牛顿在他的巨著中成功地阐述了引力定律,但他对方程式暗示的“超距作用”概念深感不安。1692年,在写给本特利的第三封信中,他写道:“一个物体可以通过真空对另一个物体施加作用,而无需任何其他物质来传递它们之间的作用力,这对我来说是极其荒谬的。我相信,没有哪个在哲学问题上有合格思维能力的人会落入这种荒谬中。”

他从未“指明这种力量的原因”。在所有其他情况下,他利用运动现象来解释作用于物体的各种力的起源,但在引力的情况下,他无法通过实验识别产生引力的运动(尽管他在1675年和1717年提出了两种机械假说)。此外,他甚至拒绝就这种力的原因提出假设,因为他认为这违背了严谨的科学。他感叹道,“哲学家们迄今为止徒劳地尝试在自然中寻找”引力的源头,因为他“通过许多理由”确信存在“至今未知的原因”,这些原因是所有“自然现象”的基础。这些基本现象至今仍在研究之中,尽管假说层出不穷,但尚未找到最终的答案。而在牛顿1713年《自然哲学的数学原理》第二版的总注中,他写道:“我至今未能从现象中发现这些引力性质的原因,我不编造假说……重力确实存在并按照我所解释的定律作用,这已足以解释所有天体的运动。”[9]
\subsection{现代形式}
用现代语言表述,这一定律如下:
每个点质量通过沿两点连线的作用力吸引每一个其他点质量。该力与两个质量的乘积成正比,与它们之间距离的平方成反比:[10] 
\begin{figure}[ht]
\centering
\includegraphics[width=8cm]{./figures/dc0c3d83fbf22afe.png}
\caption{} \label{fig_NEWW_1}
\end{figure}
\[
F = G \frac{m_1 m_2}{r^2}~
\]
其中:
\begin{itemize}
\item \( F \) 是两个质量间的引力;
\item \( G \) 是牛顿引力常数(6.674×10⁻¹¹ m³·kg⁻¹·s⁻²);
\item \( m_1 \) 是第一个质量;
\item \( m_2 \) 是第二个质量;
\item \( r \) 是两个质量中心之间的距离。
\end{itemize}
