% 统计力学(综述)
% license CCBYSA3
% type Wiki


在物理学中,\textbf{统计力学}是一种数学框架,将统计方法和概率理论应用于大量微观实体的集合。有时也称为\textbf{统计物理}或\textbf{统计热力学},其应用包括物理学、生物学、化学、神经科学、计算机科学、信息理论和社会学等多个领域。其主要目的是通过研究原子运动所遵循的物理规律,阐明物质在宏观集合状态下的性质。

统计力学源于经典热力学的发展,成功地解释了宏观物理属性(如温度、压力和热容量),将其与微观参数联系起来。这些微观参数围绕平均值波动,并以概率分布为特征。

虽然经典热力学主要关注\textbf{热力学平衡},但统计力学在\textbf{非平衡统计力学}中得到了广泛应用,用于微观建模不可逆过程的速度,这些过程由不平衡驱动。例如,化学反应以及粒子和热的流动。\textbf{涨落-耗散定理}是将非平衡统计力学应用于研究多粒子系统中最简单的非平衡状态(即稳态电流流动)时获得的基本理论。
**历史

1738年,瑞士物理学家兼数学家**丹尼尔·伯努利**发表了《流体动力学》(*Hydrodynamica*),奠定了气体动理论的基础。在这项工作中,伯努利提出了一个至今仍在使用的观点:气体由大量分子组成,这些分子向各个方向运动,它们对表面的撞击导致了我们感受到的气体压力,而我们感受到的热量只是它们运动的动能。[9]

统计力学领域的创立通常归功于以下三位物理学家:

1. **路德维希·玻尔兹曼**,他发展了关于熵的微观状态集合的基本解释;
2. **詹姆斯·克拉克·麦克斯韦**,他建立了微观状态概率分布的模型;
3. **乔赛亚·威拉德·吉布斯**,他在1884年首次命名了这一领域。

1859年,在阅读鲁道夫·克劳修斯关于分子扩散的论文后,苏格兰物理学家**詹姆斯·克拉克·麦克斯韦**提出了**麦克斯韦分子速率分布**,描述了具有特定速度范围的分子比例。这是物理学中的首个统计规律。[10][11] 麦克斯韦还首次从力学角度论证了分子碰撞会导致温度的均匀化,从而趋于平衡。[12] 五年后,即1864年,年轻的维也纳学生**路德维希·玻尔兹曼**阅读了麦克斯韦的论文,并在其后大部分生涯中进一步发展了这一领域。

**统计力学**在19世纪70年代由玻尔兹曼的研究正式开创,他的大部分研究成果于1896年的《气体理论讲义》(*Lectures on Gas Theory*)中发表。[13] 玻尔兹曼关于热力学统计解释的原创论文,包括H定理、输运理论、热平衡、气体状态方程等主题,分布在维也纳科学院和其他学会的约2,000页的论文中。他引入了**平衡统计系综**的概念,并首次研究了**非平衡统计力学**,并提出了H定理。