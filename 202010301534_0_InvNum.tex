% 逆序数
% 逆序数|排列

\begin{issues}
\issueDraft
\end{issues}

\pentry{排列\upref{permut}}

\footnote{参考 Wikipedia \href{https://en.wikipedia.org/wiki/Inversion_(discrete_mathematics)}{相关页面}.}我们把集合 $\qty{1,\dots,N}$ 的第 $n$ 种排列记为 $p_n$, 该排列的元素按照顺序分别记为 $p_{n,1}, \dots, p_{n,N}$. 对于任意 $i < j$, 如果满足 $p_{n,i} > p_{n,j}$ 我们就把 $i, j$ 或者 $p_{n,i}, p_{n,j}$ 称为排列 $p_n$ 的一个\textbf{逆序对(inversion)}. 一个排列中所有逆序对的个数就叫\textbf{逆序数(inversion number)}.

在行列式和 Levi-Civita 符号中, 我们只对逆序数的奇偶性感兴趣. 可以证明交换排列中任意两个数, 逆序数奇偶性改变. 

\addTODO{证明}
在排列 $p_n$ 中, 考虑 $p_{n,i}$ 和 $p_{n,j}$ 两个元素的置换. 不妨假设 $i < j$ 且 $p_{n,i} < p_{n,j}$. 我们把所有的逆序对用弧线连起来, 弧线的条数就是逆序数. 首先, 该置换只可能影响连接到 $p_{n,i}$ 或 $p_{n,j}$ 的线, 所以下面我们只考虑这些连线. 我们先看连接到 $p_{n,i}$ 左边或 $p_{n,j}$ 右边的那些线, 置换不会他们的数量产生任何影响, 所以也可以排除在外. 再来看连接到 $p_{n,i}$ 和 $p_{n,j}$ 之间某个元素 $p_{n, k}$ ($i < k < j$) 的线, 如果它的值比 $p_{n,i}$ 和 $p_{n,j}$ 都大或者都小, 也不会受置换影响. 只有对于每个满足 $p_{n,i} < p_{n,k} < p_{n,j}$ 的 $p_{n,k}$, 逆序数会加 2.
