% Friedmann-Robertson-Walker (FRW) 度规
% keys 度规|尺度因子|FRW|时空弯曲|宇宙学
% license Usr
% type Tutor

\pentry{度规,张量分析}{nod_4424} % \addTODO{链接}
宇宙学基本假设,即各向同性和均匀性对度规作出了约束。可以证明,要满足这样的条件,加以空间膨胀的事实,球坐标下的度规形式必须为:
\begin{equation}
\mathrm{d}s^2=g_{\mu\nu}\mathrm{d}x^{\mu}\mathrm{d}x^{\nu}=-\mathrm{d}t^2+a(t)^2 \left( \frac{1}{1-k(r/R)^2}\mathrm{d}r^2+r^2 \mathrm{d} \Omega^2\right)~,
\end{equation}
并称之为\textbf{Friedmann-Robertson-Walker (FRW) 度规}。其中 $t$ 为时间坐标,$r$ 为空间某一点到原点的\textbf{共动距离},也就是随着宇宙膨胀一起变动的空间坐标系,在\textbf{共动坐标系(co-moving frame)}上,任意两点之间的距离始终不变。另外$\mathrm{d} \Omega^2 =\mathrm{d} \theta^2 + \sin^2\theta\mathrm{d} \phi^2 $,$k$ 对应空间的弯曲性质——
\begin{itemize}
\item 若$k=1$,共动空间为三维球,因此常称这样的宇宙是\textbf{闭的(closed)};
\item 若$k=0$,显然空间部分为欧式空间,称该宇宙为\textbf{平直(flat)}的;
\item 若$k=-1$,空间为双曲面,因此常称这样的宇宙是\textbf{开放的(open)}。
\end{itemize}
 另将$a(t)$ 称作\textbf{尺度因子(scalar factor)},任意两点之间的实际距离为$\bvec r'_{phy}=a(t)\bvec r$。
根据最新的观测数据,目前的宇宙可被认为是平坦的。

需要注意度规形式对$a$的对称性,易证如果$a\rightarrow \lambda a,r\rightarrow r/\lambda, R\rightarrow R/\lambda$,度规形式也保持不变。所以通常rescale使得现今的尺度因子取值为1

