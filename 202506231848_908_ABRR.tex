% 阿贝尔-鲁菲尼定理(综述)
% license CCBYSA3
% type Wiki

本文根据 CC-BY-SA 协议转载翻译自维基百科 \href{https://en.wikipedia.org/wiki/Abel\%E2\%80\%93Ruffini_theorem}{相关文章}。

在数学中,阿贝尔–鲁芬尼定理(也称为阿贝尔不可能性定理)指出:对于一般的五次及更高次数的多项式方程,无法通过根式(即有限次加减乘除和开方)求出其解。这里的“一般”是指将方程的系数看作不定元,并对其进行运算。

该定理以保罗·鲁芬尼和尼尔斯·亨里克·阿贝尔的名字命名。鲁芬尼于1799年给出了一个不完整的证明(该证明在1813年被完善,并被柯西接受),而阿贝尔在1824年提供了完整的证明。

“阿贝尔–鲁芬尼定理”也常指一个稍强的命题:存在某些五次或更高次数的方程无法用根式求解。这个结论虽不直接出现在阿贝尔的定理陈述中,但可以从他的证明推导而来——他的证明基于这样一个事实:某些由方程系数组成的多项式不是零多项式。这个更强的结论也可以直接从伽罗瓦理论中的“一个不可解的五次方程例子”中得出。

伽罗瓦理论还暗示:
$$
x^5 - x - 1 = 0~
$$
是最简单的不能用根式求解的方程。而且,几乎所有的五次及以上次数的多项式都无法用根式求解。

这种在五次及以上多项式中求解的“不可能性”,与低次数的情况形成鲜明对比:二次方程有求根公式,三次方程有三次公式,四次方程有四次公式,但五次及以上则无通解公式。
\subsection{背景}
二次多项式方程可以通过求根公式求解,这种方法自古代起就已为人所知。同样,三次方程的求解公式(三次公式)和四次方程的求解公式(四次公式)也分别在16世纪被发现。自那时起,一个根本性的问题便是:更高次数的方程是否也可以用类似的方法求解。

在 17 世纪,数学家们提出了一个重要的断言:每一个正次数的多项式方程都有解(即使该解可能是复数),但直到 19 世纪初这一断言才被彻底证明。这就是著名的代数学基本定理。不过,该定理并不提供计算这些解的具体方法,虽然我们现在知道很多数值近似解法,可以以任意精度逼近所有解。

从 16 世纪到 19 世纪初,代数学的核心问题一直是:寻找一个通用的公式,用于解五次及更高次数的多项式方程。这也就是为何“代数学基本定理”曾一度被理解为“寻找求解公式”的意思。所谓“公式”,指的是用根式表达解,即仅使用方程系数,以及加、减、乘、除和开n次根等代数运算构成的表达式。

阿贝尔–鲁菲尼定理(Abel–Ruffini 定理)证明了通解形式的根式求解是不可能的。然而,这个不可能性并不意味着所有高次方程都不能用根式求解。恰恰相反,对于任意次数,确实存在可以用根式解出的特定方程。例如,方程$x^n - 1 = 0$对于任意的 $n$,都可以通过根式求解;再如由圆分多项式定义的方程,其所有解都可以用根式表达。

阿贝尔在其定理的证明中并没有明确指出某个特定方程无法用根式解出。事实上,他定理的表述并不排除如下的可能性:“每一个具体的五次方程也许都可以用某种特别的公式解出”。\(^\text{[5]}\)也就是说,阿贝尔定理的陈述本身并不等同于我们现在熟知的“存在不可根式求解的具体方程”这个结论。

不过,从阿贝尔的证明过程来看,似乎可以推导出这样的结论:确实存在某些不能用根式解的具体方程。这是因为阿贝尔的证明利用了一个事实:某些系数的多项式不恒为零,而有限多个非零多项式在某些变量取值下可以同时不为零,从而得到无法根式解的实例。

阿贝尔发表其证明后不久,埃瓦里斯特·伽罗瓦引入了现在被称为伽罗瓦理论的理论框架,它可以判断任意一个具体多项式方程是否可以用根式求解。在电子计算机普及之前,这仍是一个纯理论的问题。如今,借助现代计算机和软件,人们可以判断次数超过 100 的多项式是否可以用根式解出。\(^\text{[6]}\)不过,即便一个方程可以用根式求解,其求解过程也可能极其复杂。哪怕是五次方程,其根式表达式往往过于庞大,几乎没有实际意义。
\subsection{证明}
阿贝尔–鲁菲尼定理的证明早于伽罗瓦理论的建立。然而,伽罗瓦理论使这一问题更加清晰易懂,因此现代的定理证明通常基于伽罗瓦理论,而阿贝尔和鲁菲尼的原始证明如今主要作为历史文献被呈现。\(^\text{[1][7][8][9]}\)

基于伽罗瓦理论的证明通常包含以下四个主要步骤:
1. 用域论描述可解方程:首先要从域扩张的角度刻画一个多项式是否可以用根式求解。
2. 利用伽罗瓦对应:通过伽罗瓦群与其对应域扩张之间的对应关系,将“可根式求解”的条件转化为伽罗瓦群是可解群。
3. 证明对称群在五次及以上不可解:证明当次数为五或更高时,对应的对称群 $S_n$ 是不可解群,因此不能表示为一系列阿贝尔群扩张。
4. 构造具有对称伽罗瓦群的多项式:指出确实存在一些五次及以上的多项式,其伽罗瓦群就是对称群 $S_n$,因此这些多项式无法用根式求解。
\subsubsection{代数解与域论}
一个多项式方程的代数解,是指使用四则运算(加、减、乘、除)和开方运算构成的表达式。可以将这样的表达式看作是一个计算过程的描述:从方程的系数出发,依次计算出一系列数值,直到得到方程的解。

在这个计算过程中,每一步都可以考虑包含目前为止所有计算结果的最小域。只有在进行n 次开方运算的步骤时,这个域才会发生变化。

因此,一个代数解对应于如下形式的一系列域的扩张:
$$
F_0 \subseteq F_1 \subseteq \cdots \subseteq F_k~
$$
并且存在元素 $x_i \in F_i$,使得:
$$
F_i = F_{i-1}(x_i) \quad \text{且} \quad x_i^{n_i} \in F_{i-1} \quad \text{其中} \quad n_i > 1~
$$
也就是说,每一步都是在前一域中添加一个 $n_i$ 次根,从而得到新的域。如果存在这样一个域链,使得 $F_k$ 包含了多项式方程的一个解,那么就说这个方程有一个代数解。

为了确保这些扩张是正规扩张,这在伽罗瓦理论中是基本前提,必须对上述域链做进一步精炼:如果某一步中的 $F_{i-1}$ 不包含所有 $n_i$ 次单位根(即复数单位根),那么我们需要引入一个扩张域 $K_i$,它是通过向 $F_{i-1}$ 添加一个原始单位根得到的,并重新定义:$F_i = K_i(x_i)$

综上所述:如果一个方程能通过根式表示其解,那么它对应着一条逐步增加的域扩张链;在这条链中,每一扩张都是一个可正规化的扩张,且其伽罗瓦群是循环群;反过来,如果能构造出这样的域链,则说明该多项式可以用根式求解。

因此,证明一个方程是否可以用根式解的关键,就是判断是否存在这样一条域链,而这正是伽罗瓦理论所提供的工具。
\subsubsection{伽罗瓦对换关系}
伽罗瓦对换关系确立了正规扩域 $F/E$ 的子扩域与其伽罗瓦群的子群之间的一一对应关系。

这一对应关系如下:给定一个域 $K$,满足 $E \subseteq K \subseteq F$,它对应的子群是$\operatorname{Gal}(F/K)$即保持 $K$ 不变的 $F$ 上自同构的集合。反过来,给定 $\operatorname{Gal}(F/E)$ 的一个子群 $H$,它对应的子域是$F^H$即所有在 $H$ 的作用下保持不变的 $F$ 中的元素所构成的子域。

结合前一节的结论:一个多项式方程可以用根式解,当且仅当其分裂域(包含该方程所有根的最小扩域)的伽罗瓦群是可解群。一个群 $G$ 被称为“可解的”,如果它有一个子群链:$G = G_0 \triangleright G_1 \triangleright \cdots \triangleright G_k = \{1\}$其中每一个子群 $G_{i+1}$ 在 $G_i$ 中是正规子群,且对应的商群 $G_i / G_{i+1}$ 是循环群(或更常见地,阿贝尔群,两者在有限群的情况下是等价的——由有限阿贝尔群基本结构定理保证)。

因此,为了证明 Abel–Ruffini 定理,只需完成两个关键步骤:1. 证明对称群 $S_5$ 不是可解群;2. 证明确实存在伽罗瓦群为 $S_5$ 的多项式。

这两点结合在一起,说明了某些五次及以上的多项式方程,其解无法用根式表示,从而完成 Abel–Ruffini 不可解性定理的证明。

\subsubsection{可解的对称群}
当 $n > 4$ 时,n 阶对称群 $S_n$(即所有 n 个元素的置换所组成的群)仅有一个非平凡正规子群,即其对应的交错群 $A_n$。(参见“对称群 § 正规子群”)而对于 $n > 4$,交错群$A_n$ 是单群,这意味着它不含任何非平凡的正规子群,并且它也不是阿贝尔群。这两个事实结合起来意味着:$A_n$ 不是可解群;$S_n$ 也因此不是可解群(因为它的唯一非平凡正规子群 $A_n$ 已经不是可解的了)。

因此,Abel–Ruffini 定理的关键结论之一如下:一旦存在某个多项式的伽罗瓦群为 $S_n$,其中 $n > 4$,那么这个多项式就不可能用根式解。而确实存在这类多项式(下一节会展示一个例子),因此 Abel–Ruffini 定理得证。
\subsubsection{具有对称伽罗瓦群的多项式}
\textbf{一般方程}

$n$次的一般或通用多项式方程指的是如下形式的方程:
$$
x^n + a_1 x^{n-1} + \cdots + a_{n-1}x + a_n = 0~
$$
其中 $a_1, \ldots, a_n$ 是互不相同的不定元(即变量)。这个方程是定义在域$F = \mathbb{Q}(a_1, \ldots, a_n)$上的,该域是以 $a_1, \ldots, a_n$ 为变量、系数为有理数的有理函数域。原始的 Abel–Ruffini 定理断言,对于 $n > 4$,这个方程不能用根式来求解。根据前述各节内容,这一结论来源于以下事实:该方程在域 $F$ 上的伽罗瓦群是对称群$\mathcal{S}_n$即该伽罗瓦群是该方程的分裂域(包含方程所有根的最小扩域)中那些固定 $F$ 中元素的域自同构所构成的群。

为了证明伽罗瓦群是 $\mathcal{S}_n$,从根出发更为简单。设 $x_1, \ldots, x_n$ 是新的不定元,表示该多项式的根,考虑以下多项式:
$$
P(x) = x^n + b_1 x^{n-1} + \cdots + b_{n-1}x + b_n = (x - x_1)(x - x_2) \cdots (x - x_n)~
$$
设 $H = \mathbb{Q}(x_1, \ldots, x_n)$ 是以这些根为变量的有理函数域,$K = \mathbb{Q}(b_1, \ldots, b_n)$ 是由 $P(x)$ 的系数生成的子域。对 $x_i$ 的任意置换都会诱导 $H$ 的一个自同构。由韦达公式可知,$K$ 中的每一个元素都是 $x_i$ 的对称函数,因此会被所有这些置换所保持不变。因此,伽罗瓦群$\operatorname{Gal}(H / K)$就是对称群 $\mathcal{S}_n$。

对称多项式基本定理说明:系数 $b_i$ 是代数独立的,因此从每个 $a_i$ 映射到对应的 $b_i$ 可定义一个域同构 $F \to K$。这意味着我们可以将$P(x) = 0$视为一个一般方程。这就完成了对一般方程的伽罗瓦群是对称群的证明,也即完成了原始 Abel–Ruffini 定理的证明:对于 $n > 4$,一般多项式方程无法用根式求解。

\textbf{显式例子}

方程$x^5 - x - 1 = 0$是不可用根式求解的,下面将解释原因。

设 $q = x^5 - x - 1$,设 $G$ 是其伽罗瓦群,它对 $q$ 的所有复数根组成的集合有忠实作用。给根编号后,可以将 $G$ 看作对称群 $\mathcal{S}_5$ 的一个子群。

由于$q \bmod 2 = (x^2 + x + 1)(x^3 + x^2 + 1)$在有限域 $\mathbb{F}_2[x]$ 中分解为两个不同不可约多项式,因此伽罗瓦群 $G$ 包含一个置换 $g$,其由两个不相交的循环组成,循环长度分别为 2 和 3(一般来说,当一个首一整数多项式在模某个素数下分解成多个不同的不可约因子时,这些因子的次数对应于伽罗瓦群中某个置换的循环长度)。于是 $G$ 也包含 $g^3$,这是一个对换。又因为$q \bmod 3$在 $\mathbb{F}_3[x]$ 中是不可约的,因此根据相同原理,$G$ 包含一个 5-循环。由于 5 是质数,在 $\mathcal{S}_5$ 中,一个对换和一个 5-循环即可生成整个群(见对称群 § 生成元与关系)。因此$G = \mathcal{S}_5$
而对称群 $\mathcal{S}_5$ 是不可解的群,因此方程$x^5 - x - 1 = 0$无法用根式解出。
\subsection{凯莱的辅助方程}
判断一个特定的五次方程是否可以用根式解出,可以使用凯莱的辅助方程。这是一个六次一元多项式,其系数是原始通用五次方程系数的多项式。具体来说,如果一个不可约的五次方程是可以用根式求解的,当且仅当将该方程的系数代入凯莱的辅助方程后,所得到的这个六次多项式具有有理根。这可以通过有理根定理很容易地进行检验。
\subsection{历史}
大约在1770年,约瑟夫·路易·拉格朗日开始奠定基础,统一了此前用于求解方程的各种不同方法,并将这些方法与置换群理论联系起来,提出了拉格朗日辅助函数的概念。\(^\text{[10]}\)拉格朗日的这一创新性工作是伽罗瓦理论的先驱,它虽然未能为五次及更高次数的方程找到解法,但已暗示这类解可能根本不存在。不过,这一理论当时还不能提供确凿的证据。第一个猜想五次方程可能无法用根式解出的人是卡尔·弗里德里希·高斯。他在1798年写成的著作《算术研究》(Disquisitiones Arithmeticae,1801年才出版)第359节中写道:“几乎可以肯定,这个问题不是挑战现代分析方法的能力,而是提出了一个不可能完成的任务。”第二年,他在自己的论文中又写道:“在众多几何学家的努力之下,人们越来越少地期望能以代数方式解决一般方程,这种解决方式似乎是不可能且自相矛盾的。”他还补充说:“也许要严格证明五次方程无法以根式解出并不那么困难。我将在别处更详细地阐述我的研究。”然而,高斯实际上从未就该问题发表更多内容。\(^\text{[1]}\)
\begin{figure}[ht]
\centering
\includegraphics[width=8cm]{./figures/0ec84a3227de4e4d.png}
\caption{} \label{fig_ABRR_1}
\end{figure}
该定理最早由保罗·鲁菲尼于1799年接近证明。\(^\text{[11]}\)他将自己的证明寄给多位数学家以寻求认可,其中包括拉格朗日(未予回复)以及奥古斯丁-路易·柯西,后者给他回信称:“您关于一般方程解法的论文,是一篇我一直认为数学家应当牢记的作品,在我看来,它确凿地证明了四次以上一般方程在代数上不可解。”\(^\text{[12]}\)然而,普遍而言,鲁菲尼的证明并未被广泛视为有说服力。阿贝尔写道: “在我之前,唯一试图证明一般方程代数不可解的人(如果我没记错的话)是数学家鲁菲尼。但他的论文过于复杂,以至于很难判断其论证是否有效。在我看来,他的论证并不完全令人信服。”\(^\text{[12][13]}\)

后来人们发现,鲁菲尼的证明实际上并不完整。他假设他处理的所有根式都可以通过基本的域运算从多项式的根中表达出来;用现代术语来说,他假设这些根式都属于该多项式的**分裂域。要理解这实际上是一个额外的假设,我们可以看一个例子:

令多项式$P(x) = x^3 - 15x - 20$根据卡尔达诺公式,其一个根(实际上三个根都可以)可以表示为$\sqrt[3]{10 + 5i} + \sqrt[3]{10 - 5i}$然而,从函数值来看:$P(-3) < 0$ $P(-2) > 0$ $P(-1) < 0$ $P(5) > 0$可知该多项式的三个根 $r_1, r_2, r_3$ 全部是**实数**,因此其生成的有理数扩域 $\mathbb{Q}(r_1, r_2, r_3) \subset \mathbb{R}$。但这就意味着,复数 $10 \pm 5i$ 并不在这个域中。换言之,根式所涉及的复数并不总属于分裂域。柯西要么没有注意到鲁菲尼的这个假设,要么认为这只是一个不重要的小问题,但大多数历史学家认为,这一证明直到阿贝尔证明了“自然无理性定理”后才算真正完整,该定理指出在处理一般多项式时,鲁菲尼的假设是成立的。\(^\text{[8][14]}\)因此,阿贝尔–鲁菲尼定理通常归功于尼尔斯·阿贝尔。他在1824年发表了只用六页纸压缩写成的完整证明。\(^\text{[3]}\)(阿贝尔为了省纸和印刷费,采用极为简练的写作风格,论文是他自己出资印刷的。\(^\text{[9]}\)后来在1826年,他又发表了一个更加详细的版本。\(^\text{[4]}\)

证明一般五次(及更高次)方程无法用根式求解,并没有完全解决这一问题,因为阿贝尔–鲁菲尼定理并未提供必要且充分的条件来准确判断哪些五次(及更高次)方程无法用根式求解。阿贝尔在1829年去世时,正致力于这一完整刻画的研究工作。\(^\text{[15]}\)

据纳森·雅各布森所说:“鲁菲尼和阿贝尔的证明 […] 很快就被这一研究方向的巅峰成果所取代:伽罗瓦在方程可解性理论中的发现。”\(^\text{[16]}\)1830年,年仅18岁的伽罗瓦向巴黎科学院提交了一篇关于根式可解性理论的论文,但最终于1831年被拒绝,理由是论述过于简略,并且给出的条件是基于方程的根,而不是系数。伽罗瓦了解鲁菲尼和阿贝尔的贡献,因为他写道:“如今,一个普遍的真理是,大于四次的一般方程不能用根式解出……尽管几何学家忽视了阿贝尔和鲁菲尼的证明,这一真理还是(通过耳闻)广为流传了。”1832年,伽罗瓦去世,他的论文《关于方程可由根式解出的条件的回忆录》)\(^\text{[17]}\)直到1846年才由约瑟夫·刘维尔发表,并附带他自己的解释。\(^\text{[15]}\)在正式发表之前,刘维尔曾于1843年7月4日在学院的一次演讲中宣布了伽罗瓦的成果。\(^\text{[5]}\)1845年,皮埃尔·旺策尔发表了对阿贝尔证明的简化版本。\(^\text{[18]}\)他在发表时已知伽罗瓦的研究,并指出阿贝尔的证明只适用于一般多项式,而伽罗瓦的方法则可用于构造具体的五次方程,其根无法由其系数通过根式表示。

1963年,弗拉基米尔·阿诺尔德发现了阿贝尔–鲁菲尼定理的一个拓扑证明,\(^\text{[19][20]}\)这成为后来拓扑伽罗瓦理论的起点。\(^\text{[21]}\)
\subsection{参考文献}
\begin{enumerate}
\item Ayoub, Raymond G.(1980年),《保罗·鲁菲尼对五次方程的贡献》,载《精确科学史档案》,第22卷第3期,第253–277页,doi:10.1007/BF00357046,JSTOR 41133596,MR 0606270,S2CID 123447349,Zbl 0471.01008
\item Ruffini, Paolo(1813年),《关于一般代数方程解法的思考》,意大利语,由Società Tipografica印行。
\item Abel, Niels Henrik(1881)[1824],《关于代数方程的回忆录,证明了一般五次方程不可用代数方法解出》,载:Sylow, Ludwig;Lie, Sophus 编,《尼尔斯·亨里克·阿贝尔全集》(法语),第一卷(第二版),Grøndahl & Søn,第28–33页(PDF)
\item Abel, Niels Henrik(1881)[1826],《关于超过四次的一般代数方程无法代数求解的证明》,载:Sylow, Ludwig;Lie, Sophus 编,《尼尔斯·亨里克·阿贝尔全集》(法语),第一卷(第二版),Grøndahl & Søn,第66–87页(PDF)
\item Stewart, Ian(2015),《伽罗瓦理论导论》,第四版,CRC出版社,ISBN 978-1-4822-4582-0
\item Fieker, Claus;Klüners, Jürgen(2014),《有理数多项式的伽罗瓦群计算》,《伦敦数学学会计算与数学期刊》,第17卷第1期,第141–158页,arXiv:1211.3588,doi:10.1112/S1461157013000302,MR 3230862
\item Rosen, Michael I.(1995),《尼尔斯·亨里克·阿贝尔与五次方程》,《美国数学月刊》,第102卷第6期,第495–505页,doi:10.2307/2974763,JSTOR 2974763,MR 1336636,Zbl 0836.01015
\item Tignol, Jean-Pierre(2016),《鲁菲尼与阿贝尔关于一般方程的研究》,《伽罗瓦的代数方程理论》(第二版),世界科学出版社,ISBN 978-981-4704-69-4,Zbl 1333.12001
\item Pesic, Peter(2004),《阿贝尔的证明:一篇关于数学不可解性的起源与意义的随笔》,剑桥:麻省理工学院出版社,ISBN 0-262-66182-9,Zbl 1166.01010
\item Lagrange, Joseph-Louis(1869)[1771],《关于代数方程解法的反思》,载 Serret, Joseph-Alfred 编,《拉格朗日全集》第三卷,Gauthier-Villars,第205–421页
\item Ruffini, Paolo(1799),《一般方程理论,证明高于四次的一般代数方程无法解出》,意大利语,S. Tommaso d'Aquino印刷所
\item Kiernan, B. Melvin(1971),《伽罗瓦理论的发展:从拉格朗日到阿廷》,《精确科学史档案》,第8卷第1/2期,第40–154页,doi:10.1007/BF00327219,JSTOR 41133337,MR 1554154,S2CID 121442989
\item Abel, Niels Henrik(1881)[1828],《关于代数方程的代数解法》,载:Sylow, Ludwig;Lie, Sophus 编,《尼尔斯·亨里克·阿贝尔全集》(法语),第二卷(第二版),Grøndahl & Søn,第217–243页(PDF)
\item Stewart, Ian(2015),《伽罗瓦理论的核心思想》,《伽罗瓦理论》,第四版,CRC出版社,ISBN 978-1-4822-4582-0
\item Tignol, Jean-Pierre(2016),《伽罗瓦》,《伽罗瓦的代数方程理论》(第二版),世界科学出版社,ISBN 978-981-4704-69-4,Zbl 1333.12001
\item Jacobson, Nathan(2009),《方程的伽罗瓦理论》,《基础代数学》第1卷(第二版),多佛出版社,ISBN 978-0-486-47189-1
\item Galois, Évariste(1846),《关于方程可由根式解出的条件的回忆录》,《纯与应用数学期刊》(法语),第十一卷,第417–433页(PDF)
\item Wantzel, Pierre(1845),《证明所有代数方程都不能用根式解的不可解性》,《新数学年鉴》(法语),第四期,第57–65页
\item Alekseev, Valeriy B.(2004),《阿贝尔定理的问题与解答:根据阿诺德教授的讲义》,Kluwer学术出版社,ISBN 1-4020-2186-0,MR 2110624,Zbl 1065.12001
\item Goldmakher, Leo,《阿诺德关于五次不可解性的初等证明》(PDF)
\item Khovanskii, Askold(2014),《拓扑伽罗瓦理论:有限形式下方程的可解性与不可解性》,《施普林格数学专著》,施普林格出版社,doi:10.1007/978-3-642-38871-2,ISBN 978-3-642-38870-5
\end{enumerate}