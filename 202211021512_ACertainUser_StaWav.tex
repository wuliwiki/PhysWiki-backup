% 驻波

\begin{issues}
\issueDraft
\end{issues}

\pentry{平面简谐波\upref{PWave}}

\subsection{一维驻波}
想象一根紧绷的弦的两端固定, 那么弦上显然不可能出现稳定的简谐波\upref{PWave}, 因为简谐波要求两个端点也必须随做简谐运动\footnote{“两端固定” 这个条件在解波动方程\upref{WEq1D}的语境下被称为边界条件, 所以前面讨论的简谐波作为波动方程的解, 不满足所要求的边界条件}.

一般来说, 波都是可叠加的, 所以我们不妨看看两列振幅、频率、波长都相同而方向相反的简谐波
$$
\begin{aligned}
f_1(x,t)&=\sin(kx+\omega t)\\
f_2(x,t)&=\sin(kx-\omega t)\\
\end{aligned}
$$
叠加后能否满足条件,即
$$f=f_1+f_2=\sin(kx+\omega t)+\sin(kx-\omega t)$$
根据和差化积\upref{TriEqv}公式,
\begin{equation}
f(x,t)=-2\sin(\omega t)\sin(kx)
\end{equation}
形如
\begin{equation}\label{StaWav_eq1}
f(x,t)=A\sin(\omega t)\sin(kx)
\end{equation}
的波可以理解为单一频率的驻波.

\subsubsection{波节与波腹}
\autoref{StaWav_eq1} 
驻波中从不振动的点为波节,而振幅最大的点为波腹.


\addTODO{介绍: 波节、波腹、驻波的波动方程; 如何计算一根线所支持的振动频率、 谐波是什么. 大学物理程度即可, 不要讲太深}
