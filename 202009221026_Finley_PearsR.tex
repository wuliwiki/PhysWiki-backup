% 皮尔逊相关系数

\verb | \issueTODO 皮尔逊相关系数|

在统计学中,皮尔逊相关系数(Pearson correlation coefficient),又称皮尔逊积矩相关系数(Pearson product-moment correlation coefficient,简称 PPMCC或PCCs),是用于度量两个变量X和Y之间的相关(线性相关),其值介于-1与1之间.
\begin{definition}{皮尔逊相关系数}\label{PearsR_def1}
两个变量之间的皮尔逊相关系数定义为两个变量之间的协方差和标准差的商:
\begin{equation}
\rho_{X,Y}=\frac{cov(X,Y)}{\sigma_{X}\sigma_{Y}}=\frac{E[(X-\mu X)(Y-\mu Y)]}{\sigma_{X}\sigma_{Y}}
\end{equation}
上式定义了总体相关系数,常用希腊小写字母$\rho_{X,Y}$表示,估算样本的协方差和标准差,可得到皮尔逊相关系数,常用英文小写字母$r$代表:
\begin{equation}
r=\frac{\sum_{i=1}^n\left(X_i-\overline X\right)\left(Y_i-\overline Y\right)}{\sqrt{\sum_{i=1}^n\left(X_i-\overline X\right)^2}\sqrt{\sum_{i=1}^n\left(Y_i-\overline Y\right)^2}}
\end{equation}
\end{definition}
\subsection{数学特性}
总体和样本皮尔逊系数的绝对值小于或等于1.如果样本数据点精确的落在直线上(计算样本皮尔逊系数的情况),或者双变量分布完全在直线上(计算总体皮尔逊系数的情况),则相关系数等于1或-1.皮尔逊系数是对称的: $corr(Y,X)=corr(X,Y)$.
皮尔逊相关系数有一个重要的数学特性是,因两个变量的位置和尺度的变化并不会引起该系数的改变,即它该变化的不变量(由符号确定).也就是说,我们如果把$X$移动到$a+bX$和把$Y$移动到$c+dY$,其中a、b、c和d是常数,并不会改变两个变量的相关系数(该结论在总体和样本皮尔逊相关系数中都成立).