% JSON 笔记

\begin{issues}
\issueDraft
\end{issues}

\subsection{常识}
\begin{itemize}
\item JSON 全称是 \textbf{JavaScript Object Notation}
\item JSON 和 xml 格式常用于从网站 api 获取数据。
\end{itemize}

\subsection{格式}
一个例子
%[language=json]  % pdf 不支持 json
\begin{lstlisting}[language=none]
{"menu": {
  "id": "file",
  "value": "File",
  "popup": {
    "menuitem": [
      {"value": "New", "onclick": "CreateNewDoc()"},
      {"value": "Open", "onclick": "OpenDoc()"},
      {"value": "Close", "onclick": "CloseDoc()"}
    ]
  }
}}
\end{lstlisting}

\begin{itemize}
\item object 由若干对\textbf{无序}\footnote{也就是 parser 读取文件时并不会保留文件中的顺序}的 key-val 组成: \verb|{"key" : val, "key" : val, ...}|
\item value 的类型有 \verb|null|, \verb|string|, \verb|number| (整数或浮点), \verb|bool|, \verb|object|, \verb|array|
\item array 由若干\textbf{有序}的 value 组成: \verb|[val1, val2, val3]| 每一个 val 都可以是任意类型, 包括 object 和 array。
\item 理论上整个 json 文件可以是任意单个 value, 但一般是一个 object。
\end{itemize}
