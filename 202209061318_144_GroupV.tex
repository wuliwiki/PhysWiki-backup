% 群速度
% keys 相速度|群速度

\begin{issues}
\issueDraft
\end{issues}

\pentry{拍频\upref{beatno}}

当波速不随频率变化时, 波形整体移动.

波速是频率的函数

\begin{equation}
v_g = \dv{\omega}{k}
\end{equation}

(未完成:推导一下量子力学中中心动量为 $k$ 的波包的相速度为什么是 $2k$)

\subsection{形象的推导}
\begin{equation}
k, k+\Delta k, \Delta k, v_1 = \omega/k, v_2 = (\omega+\Delta\omega)/(k+\Delta k)
\end{equation}
\begin{equation}
\Delta v = \frac{\Delta \omega}{k} - \frac{\omega}{k^2}\Delta k
\end{equation}
\begin{equation}
v_g = v_1 + \frac{\Delta \lambda}{\Delta v} = v_1 - \frac{2\pi\Delta k}{k^2\Delta v}
\end{equation}

\subsection{公式推导}
\begin{figure}[ht]
\centering
\includegraphics[width=14cm]{./figures/GroupV_1.png}
\caption{两束频率、速度略有差异的波的叠加} \label{GroupV_fig1}
\end{figure}

\begin{figure}[ht]
\centering
\includegraphics[width=5cm]{./figures/GroupV_2.png}
\caption{请添加图片描述} \label{GroupV_fig2}
\end{figure}
两个振幅相同的平面波
\begin{equation}
\begin{aligned}
&f_1(x,t) = A\cos(k_1 x - \omega_1 t + \phi_1)\\
&f_2(x,t) = A\cos(k_2 x - \omega_2 t + \phi_2)
\end{aligned}
\end{equation}
根据和差化积(\autoref{TriEqv_eq9}~\upref{TriEqv})
\begin{equation}
\begin{aligned}
f_1(x,t) + f_2(x,t) &= 2 A \cos \qty(\frac{k_2-k_1}{2}x - \frac{\omega_2-\omega_1}{2}t + \frac{\phi_2 - \phi_1}{2})\\
& \times \cos \qty(\frac{k_1+k_2}{2}x - \frac{\omega_{2}+\omega_{1}}{2}t + \frac{\phi_1 + \phi_2}{2})
\end{aligned}
\end{equation}
第一个 $\cos$ 是波包, 第二个 $\cos$ 是载波. 当频率很接近时, 从第一项可知波包的速度为
\begin{equation}
v_g = \dv{\omega}{k}
\end{equation}
