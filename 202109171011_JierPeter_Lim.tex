% 极限
% 微积分|极限|数列极限|函数极限|无穷小

\subsection{数列的极限}

微积分的核心概念是极限,而极限最基础的情形是数列的极限.数列是离散的,比较容易理解,而所有与极限有关的概念也都可以从数列的极限拓展得到.

先来看一个数列的例子.

\begin{example}{}\label{Lim_ex1}
我们都知道 $\pi$ 是一个无理数,所以 $\pi$ 的小数部分是无限多的.目前用计算机,已经可以将 $\pi$ 精确地计算到小数点后数亿位.然而在实际应用中,往往只用取前几位小数的近似即可.下面给出一个数列,定义第 $n$ 项是 $\pi$ 的前 $n$ 位小数近似(不考虑四舍五入),即
\begin{equation}
a_0 = 3,\,\, a_1 = 3.1,\,\, a_2 = 3.14,\,\, a_3 = 3.141,\,\dots.
\end{equation}
\end{example}

这个数列显而易见的性质,就是当 $n$ \textbf{趋于无穷}时,$a_n$ 趋(近)于 $\pi$. 无穷通常用符号 $\infty$ 来表示(像“8”横过来写).我们把这类过程叫做\textbf{极限}.以上这种情况,用极限符号表示,就是
\begin{equation}
\lim_{n \to \infty } {a_n} = \pi 
\end{equation}
这里 $\lim$ 是极限(limit)的意思,下方用箭头表示某个量变化的趋势\footnote{$\lim\limits_{n \to \infty }$ 在这里相当于一个“操作”,叫\textbf{算符(operator)}, 它作用在数列$a_n$ 上,把数列变成一个数,即该数列的极限.}. 算符的 “输出” 就是一个数( $a_n$ 的极限值).所以不要误以为这条式子是说当 $n = \infty$ 时,$a_n=\pi$ \footnote{有两个理由可以说明这种理解不正确:首先,按定义,每个$a_n$都是有理数,而$\pi$是无理数,所以不应该有任何一个$a_n=\pi$;其次,$\infty$不是一个实数,不存在$n=\infty$的说法.这里的$n\to\infty$只是表示$n$的增大是没有限制的.},而要理解成数列 $a_n$ 经过算符 $\lim\limits_{n \to \infty }$ 的作用以后,得出其极限是 $\pi$. 类比函数 $\sin x = y$,并不是说 $x=y$, 而是说 $x$ 经过正弦函数作用后等于 $y$. 

所以从概念上来说,极限中的“趋于” 和“等于” 是不同的.趋于是数列整体的性质,而不是单个数字的性质.我们可以像这样粗略理解“趋近”:
\begin{itemize}
\item 越来越接近,但不一定相等
\item (在不相等的情况下)只有更近,没有最近
\end{itemize}

对极限来说,第2点成立是非常必要的.但是怎样能说明 “没有最近”呢?可以看出,当 $n$ 越大,$a_n$ 越接近 $\pi$, 它们的 “距离”,可以用 $\abs{a_n - \pi}$ 来表示.也就是说,对任何一个 $a_n$, 如果所对应的距离 $\abs{a_n - \pi } \ne 0$, 总能找到一个更大的数 $m>n$, 使 $\abs{a_m - \pi} < \abs{a_n - \pi}$ (也就是$a_m$比$a_n$更靠近$\pi$),并且要求$a_m$之后的所有项也都能满足这一条件.只有这样,才能从数学上说明上面两个意思.这就是极限思想的精髓.根据这个思想,下面可以写出数列极限的定义.



\begin{definition}{数列的极限}
考虑数列$\{a_n\}$.若存在一个实数$A$,使得对于\textbf{任意}给定的\textbf{正实数} $\varepsilon > 0$(无论它有多么小),总存在正整数 $N_\epsilon$, 使得对于所有编号 $n>N_\epsilon$ ,都有 $\abs{a_n - A} < \varepsilon$ ($A$ 为常数) 成立,那么数列 $a_n$ 的极限就是 $A$.

将“数列$\{a_n\}$的极限是$A$”表示为$\lim\limits_{n\to\infty}a_n=A$.
\end{definition}
 

% 在命题中,通常把 “任意” 用 “ $\forall$” (for all) 表示,把 “存在” 用 “$\exists $” (there exist(s), there is (are))表示.即“ 对 $\forall \varepsilon>0$, $\exists N$, 当 $n>N$ 时,有 $\abs{a_n - A} < \varepsilon$”. 

由于以上讨论中 $\lim$ 作用的对象是数列,那么箭头右边只能是 $\infty$ (准确来说应该是正无穷 $+\infty$, 但是由于数列的项一般是正的,所以正号省略了).

把定义套用到上面的\autoref{Lim_ex1} 中, 如果要求 $\abs{a_n - \pi} < 10^{-3}$ (给定 $\varepsilon  = 10^{-3}$),只要令 $N=3$ (当然也可以令 $N=4, N=5$, 等) 就可以保证第 $N$ 项后面所有的项都满足要求. 一般地如果给定 $\varepsilon  = b\e{-q}  (b > 1)$, 就令 $N = q$, 第 $N$ 项以后的项就满足要求.根据定义,这就意味着 $\lim\limits_{n \to \infty } a_n = \pi$. 

我们来看几个简单的例题,加深一下印象.

\begin{exercise}{}
考虑数列$a_n=\frac{1}{2^n}$.根据定义,证明$\lim\limits_{n\to\infty}a_n=0$.
\end{exercise}

\begin{exercise}{}\label{Lim_exe1}
考虑数列$a_n=(-1)^n$.这个数列存在极限吗?
\end{exercise}

\autoref{Lim_exe1} 的数列是不存在极限的,因为它的值在$\pm 1$之间反复横跳,也就是说对于任何实数$A$,$\abs{a_n-A}$都只有最多两个值,而且其中一个肯定非零.这就导致如果我们把$\epsilon>0$取得足够小(小于两个$\abs{a_n-A}$中比较大的那个),那么不管$N$多大,总有$n>N$使得$\abs{a_n-A}>\epsilon$.结果就是这个数列没有任何极限值.我们把这种情况称为\textbf{发散}.

\begin{definition}{数列的敛散性}
如果一个数列$\{a_n\}$不存在极限,就称它是\textbf{发散(divergent)}的.如果$\{a_n\}$存在极限,则称它是\textbf{收敛(convergent)}的.
\end{definition}

\subsection{函数的极限}
实函数$f(x)$可以看成是一种“连续”的数列,只不过把元素编号从离散的$n$改为连续的$x$.类比数列的极限, 我们也可以定义\textbf{函数在正无穷的极限} $\lim\limits_{x\to +\infty} f(x) = A$.

\begin{definition}{函数的极限}
考虑实函数$f(x)$.若对于任意$\epsilon>0$,总存在正实数$X_\epsilon$,使得对于所有$x>X_\epsilon$,
\end{definition}

与数列不同的是, 对于函数我们还可以定义\textbf{函数在负无穷的极限} $\lim\limits_{x\to -\infty} f(x)$(把以上定义的 $>$ 号改成 $<$ 号即可).

另外可以定义 \textbf{$f(x)$ 在 $x_0$ 处的极限} $A$, 即“ 对 $\forall \varepsilon > 0$, $\exists \delta > 0$, 当 $\abs{x - x_0} < \delta$ 时,有 $\abs{f(x) - A} < \varepsilon$”. 注意 $f(x)$ 不需要在 $x_0$ 处有定义.

\begin{example}{}
求函数在某个值处的极限时, 通常可以直接代入数值计算, 如
\begin{equation}
\lim_{x\to 1} 2x + 1 = 3 \qquad \lim_{x\to 2}\frac{x + 1}{x + 2} = \frac34
\end{equation}

当无穷大与常数相加时, 可以忽略常数, 如
\begin{equation}
\lim_{x\to +\infty} \frac{x + 1}{2x + 2} = \lim_{x\to +\infty} \frac{x}{2x} = \frac12
\end{equation}
\end{example}

\subsection{无穷小的阶}
如果令 $x\to 0$, 我们就说 $x$ 是\textbf{无穷小}. 但一些无穷小会更快地趋近于 $0$, 若 $x$ 的某个函数 $\alpha(x)$ 满足
\begin{equation}
\lim_{x\to 0} \frac{\alpha(x)}{x} = 0
\end{equation}
那 $\alpha(x)$ 就是 $x$ 的\textbf{高阶无穷小}. 若
\begin{equation}
\lim_{x\to 0} \frac{\alpha(x)}{x^n} \ne 0
\end{equation}
则称 $\alpha(x)$ 为 $x$ 的 $n$ 阶无穷小. 例如, $c x^n$ ($c$ 为常数)就是 $x$ 的 \textbf{$n$ 阶无穷小}, 记为 $\order{x^n}$.

在求极限时, 若高阶无穷小与低阶无穷小相加, 通常可以忽略高阶无穷小. 另外由定义不难推出
\begin{equation}
\order{x^n} x^m = \order{x^{n + m}} \qquad (m > -n)
\end{equation}

在物理中, 当我们用一个函数 $g(x)$ 来近似另一个函数 $f(x)$ 并记为 $f(x) = g(x) + \order{h^n}$ 时(这里 $x$ 是函数的自变量, $h$ 是函数表达式中一个较小的常数), 就说 $g(x)$ 的误差为 $\order{h^n}$.
