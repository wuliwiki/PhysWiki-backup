% 混淆矩阵
% keys 分类|机器学习|统计学
% license Xiao
% type Tutor

\pentry{分类\nref{nod_Class}}{nod_554a}

在机器学习中,\textbf{混淆矩阵}(Confusion Matrix)是一种通过表格可视化的方式呈现分类模型性能的常用工具,能够显示出模型预测值与实际标签之间的对应关系。顾名思义,混淆矩阵能够方便地看出模型是否将两个不同的类混淆了(比如把一个类错误地判定为另一类)以及混淆的数量有多少。

要弄清楚混淆矩阵,首先必须了解以下基本概念。

\begin{enumerate}
\item True Positive(TP):真正类。样本的真实类别是正类,并且模型也将其判定为正类。
\item False Negative(FN):假负类。样本的真实类别是正类,但模型将其判定为负类。
\item False Positive(FP):假正类。样本的真实类别是负类,但模型将其判定为正类。
\item True Negative(TN):真负类。样本的真实类别是负类,并且模型将其判定为负类。 
\end{enumerate}

对于二分类问题而言,混淆矩阵包含两行、两列,一共四个单元格。列(行)分别表示分类器预测的值,行(列)分别表示实际的值。如\autoref{tab_ConMat_1} 所示。

\begin{table}[ht]
\centering
\caption{混淆矩阵基本模式}\label{tab_ConMat_1}
\begin{tabular}{|c|c|c|}
\hline
 & 预测为正类 (Positive)& 预测为负类 (Negative)\\
\hline
实际为正类(Positive) & 真正类 (TP)& 假负类 (FN) \\
\hline
实际为负类 (Negative)& 假正类 (FP) & 真负类 (TN)\\
\hline
\end{tabular}
\end{table}

举个例子,现在有一个训练好的二元分类器,用于判断给定图片上的动物是马还是羊。假设,有一个图片数据集,一共14张图片,其中9只为羊,5只为马。假设用0表示羊,1表示马。样本情况可以表示为\autoref{tab_ConMat1} 。

\begin{table}[ht]
\centering
\caption{样本表}\label{tab_ConMat1}
\begin{tabular}{|c|c|c|c|c|c|c|c|c|c|c|c|c|c|c|}
\hline
样本编号 & 1 & 2 & 3 & 4 & 5 & 6 & 7 & 8 & 9 & 10 & 11 & 12 & 13 & 14 \\
\hline
实际类别 & 0 & 0 & 0 & 0 & 0 & 0 & 0 & 0 & 0 & 1 & 1 & 1 & 1 & 1 \\
\hline
\end{tabular}
\end{table}

现在用训练好的分类器来做判断,有可能产生下面的结果。

\begin{table}[ht]
\centering
\caption{样本分类表}\label{tab_ConMat2}
\begin{tabular}{|c|c|c|c|c|c|c|c|c|c|c|c|c|c|c|}
\hline
样本编号 & 1 & 2 & 3 & 4 & 5 & 6 & 7 & 8 & 9 & 10 & 11 & 12 & 13 & 14 \\
\hline
实际类别 & 0 & 0 & 0 & 0 & 0 & 0 & 0 & 0 & 0 & 1 & 1 & 1 & 1 & 1 \\
\hline
预测类别 & 0 & 1 & 0 & 1 & 0 & 1 & 0 & 0 & 0 & 1 & 1 & 1 & 0 & 1 \\
\hline
\end{tabular}
\end{table}

从\autoref{tab_ConMat2} 中可以看出,实际有9只羊,模型预测正确了6只(预测为羊),预测错了3只(预测为马)。马实际上有5匹,模型预测正确了4只(预测为马),预测错了1匹(预测为羊)。把结论写下来,就形成了如\autoref{tab_ConMat_2} 所示的混淆矩阵。
\begin{table}[ht]
\centering
\caption{混淆矩阵例子}\label{tab_ConMat_2}
\begin{tabular}{|c|c|c|}
\hline
 & 预测为马 & 预测为羊 \\
\hline
实际为马 & 4 & 1 \\
\hline
实际为羊 & 3 & 6 \\
\hline
\end{tabular}
\end{table}

设样本总数用N表示,本例中N=14。显然,当只给定混淆矩阵时,也可以从中算出样本总数:N=TP+FN+FP+TN=14。由混淆矩阵,我们可以得出对于模型的多个常规的评价指标。

\textbf{精确率}(Accuracy),或者称\textbf{精度}\upref{Accu}:最常用的分类性能指标。可以用来表示模型的分类精度,即模型识别正确的个数/样本的总个数。

本例模型精度 = (TP+TN)/N=(4+6)/14=10/14

\textbf{准确率}(Precision),又称\textbf{查准率}:表示在模型判定为正类的样本中,真正为正类的样本所占的比例。

本例准确率 = TP/(TP+FP)=4/(4+3)=4/7

\textbf{召回率}(Recall),又称\textbf{查全率}:在实际正样本中,模型判定正确的数量。

本例召回率=TP/(TP+FN)=4/(4+1)=4/5

\textbf{特异度}(Specificity):实际为负类的样本中被模型正确判定为负类的比例。

本例特异度=TN/(TN+FP)=6/(6+3)=2/3

\textbf{F1分数}(F1 score):准确率和召回率的调和平均数。

本例F1分数 = $ 2 \times \frac{}{}Precision \times Recall / (Precision + Recall) = 2 \times (4/7) $