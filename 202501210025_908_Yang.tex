% 杨-米尔斯理论(综述)
% license CCBYSA3
% type Wiki

本文根据 CC-BY-SA 协议转载翻译自维基百科\href{https://en.wikipedia.org/wiki/Yang\%E2\%80\%93Mills_theory}{相关文章}。

杨-米尔斯理论是由杨振宁和罗伯特·米尔斯于1953年提出的一种量子场论,用于描述核结合力,也广泛用于描述类似的理论。杨-米尔斯理论是一种基于特殊酉群 SU(n) 或更一般的紧李群的规范理论。杨-米尔斯理论试图利用这些非阿贝尔李群来描述基本粒子的行为,是电磁力和弱力的统一(即 U(1) × SU(2))以及量子色动力学(强力理论,基于 SU(3))的核心。因此,它构成了粒子物理学标准模型理解的基础。
\subsection{历史与定性描述}  
\subsubsection{电动力学中的规范理论}  
所有已知的基本相互作用都可以通过规范理论来描述,但这一点的确立花费了几十年的时间。[2] 赫尔曼·外尔的开创性工作始于1915年,当时他的同事艾米·诺特证明了每一个守恒的物理量都有一个匹配的对称性,并最终在1928年出版了他的著作,将对称性几何理论(群论)应用于量子力学。[3]: 194  外尔将诺特定理中相关的对称性命名为“规范对称性”,类比于铁路轨距中的距离标准化。

1922年,厄尔温·薛定谔在工作于薛定谔方程之前的三年,连接了外尔的群概念与电子电荷。薛定谔展示了群 U(1) 会在电磁场中产生一个相位变化 \( e^{i\theta} \),该相位变化与电荷守恒相匹配。[3]: 198  随着量子电动力学在1930年代和1940年代的发展,U(1) 群变换在其中起到了核心作用。许多物理学家认为,必定存在一种与核子动力学类似的理论,特别是杨振宁对这种可能性十分痴迷。
\subsubsection{杨和米尔斯发现了核力规范理论}
杨的核心思想是寻找一个在核物理学中与电荷相似的守恒量,并利用它来发展一个与电动力学相对应的规范理论。他选择了等向量自旋的守恒,这是一种量子数,用来区分中子和质子,但他在理论上未能取得进展。[3]: 200  1953年夏天,杨在普林斯顿休假时,遇到了一位可以提供帮助的合作者:罗伯特·米尔斯。正如米尔斯自己所描述:

“在1953–1954学年,杨是布鲁克海文国家实验室的访问学者...我也在布鲁克海文...并且被分配到和杨同一个办公室。杨曾多次表现出他对刚开始职业生涯的物理学家的慷慨,他告诉我关于普适化规范不变性的想法,我们讨论了很长时间...我能够为讨论做出一些贡献,特别是在量子化过程方面,并且在推导形式上做了一些小的贡献;然而,关键的想法是杨的。”[4]

1953年夏天,杨和米尔斯将阿贝尔群(例如量子电动力学)的规范理论概念扩展到非阿贝尔群,选择了SU(2)群来解释涉及强相互作用的碰撞中等向量自旋的守恒。杨在1954年2月在普林斯顿的工作报告遭到了保利的挑战,保利询问了基于规范不变性概念所发展的场中质量问题。[3]: 202  保利知道这是一个问题,因为他曾研究过应用规范不变性,尽管他选择没有发表,认为该理论中的无质量激发是“非物理的‘阴影粒子’”。[2]: 13  杨和米尔斯于1954年10月发表了论文;在论文的最后,他们承认:

我们接下来要讨论的是b量子的质量问题,但我们没有找到一个令人满意的答案。[5]

这一无质量激发的非物理问题阻碍了进一步的进展。[3]

这个想法被搁置直到1960年,当时在无质量理论中通过对称破缺使粒子获得质量的概念被提出,最初是由杰弗里·戈德斯通、南部阳一郎和乔凡尼·乔纳-拉西尼奥提出的。这一概念促使了杨-米尔斯理论研究的重大重启,并成功地在电弱统一和量子色动力学(QCD)的公式化中取得了进展。电弱相互作用由规范群SU(2) × U(1)描述,而QCD则是SU(3)杨-米尔斯理论。电弱SU(2) × U(1)的无质量规范玻色子在自发对称破缺后混合,产生了三个弱相互作用的质量玻色子(W+、W−和Z0)以及仍然是无质量的光子场。光子场的动力学及其与物质的相互作用,反过来又由量子电动力学的U(1)规范理论支配。标准模型通过对称群SU(3) × SU(2) × U(1)将强相互作用与统一的电弱相互作用(将弱相互作用与电磁相互作用统一)结合起来。在当前的时代,强相互作用与电弱相互作用并未统一,但从耦合常数的跑动观察来看,人们相信[citation needed]它们在极高能量下会收敛到一个单一值。

在量子色动力学的低能现象学尚未完全理解,这主要是因为强耦合理论的管理困难。这可能是为什么束缚问题尚未被理论证明的原因,尽管它是一个一致的实验观察结果。这也表明,QCD低能束缚是一个具有重大相关性的数学问题,因此杨-米尔斯存在性与质量间隙问题成为了千年奖难题。
\subsubsection{在非阿贝尔规范理论的平行工作}
1953年,在一封私人信件中,沃尔夫冈·泡利提出了一个六维的爱因斯坦场方程理论,扩展了卡鲁扎(Kaluza)、克莱因(Klein)、福克(Fock)等人提出的五维理论,涉及到一个更高维度的内部空间。[6]然而,没有证据表明泡利发展了规范场的拉格朗日量或其量子化。由于泡利发现他的理论“导致了一些相当不真实的影像粒子”,他决定不正式发表他的结果。[6]尽管泡利没有发表他的六维理论,但他在1953年11月在苏黎世进行了两次关于该理论的研讨会讲座。[6]

1954年1月,剑桥大学的研究生罗纳德·肖(Ronald Shaw)也为核力发展了一个非阿贝尔规范理论。[7]然而,为了保持规范不变性,该理论需要无质量的粒子。由于当时没有已知的无质量粒子,肖和他的导师阿卜杜斯·萨拉姆(Abdus Salam)决定不公开他们的工作。[7]在杨和米尔斯于1954年10月发表论文后不久,萨拉姆鼓励肖发表他的工作以标记他的贡献。肖拒绝了,而是将其作为他1956年发布的博士论文的一章。[8][9]
\subsection{数学概述}
杨–米尔斯理论是具有非阿贝尔对称群的规范理论的特殊例子,其拉格朗日量给出。
\begin{figure}[ht]
\centering
\includegraphics[width=10cm]{./figures/08c066f4b1d9312e.png}
\caption{在 \( \mathbb{R}^4 \) 的 \( (x_1, x_2) \)-切片上,BPST 质点的 \( dx_1 \otimes \sigma_3 \) 系数,其中 \( \sigma_3 \) 是第三个保利矩阵(左上)。\( dx_2 \otimes \sigma_3 \) 系数(右上)。这些系数决定了 BPST 质点 \( A \) 的限制,其中 \( g=2 \)、\( \rho=1 \)、\( z=0 \),限制在这个切片上。对应的场强度以 \( z=0 \) 为中心(左下)。一个 BPST 质点场强度的可视化表示,质点中心 \( z \) 位于 \( \mathbb{R}^4 \) 的紧致化 \( S^4 \) 上(右下)。BPST 质点是 \( \mathbb{R}^4 \) 上杨–米尔斯方程的经典质点解。} \label{fig_Yang_1}
\end{figure}
\[\displaystyle \ {\mathcal {L}}_{\mathrm {gf} }=-{\tfrac {1}{2}}\operatorname {tr} (F^{2})=-{\tfrac {1}{4}}F^{a\mu \nu }F_{\mu \nu }^{a}~\] 
与李代数的生成元 \(\displaystyle \ T^{a}\),其索引为 a,对应于 F-量(曲率或场强形式),满足
\[\displaystyle \ \operatorname {tr} \left(T^{a}\ T^{b}\right)={\tfrac {1}{2}}\delta ^{ab}\ ,\qquad \left[T^{a},\ T^{b}\right]=i\ f^{abc}\ T^{c}~\]. 
这里的 \(f^{abc}\) 是李代数的结构常数(如果李代数的生成元按规范化,使得 \(\displaystyle \ \operatorname {tr} (T^{a}\ T^{b})\) 与 \(\displaystyle \ \delta ^{ab}\)成比例,则\(f^{abc}\)是完全反对称的),协变导数定义为
\[\displaystyle \ D_{\mu }=I\ \partial _{\mu }-i\ g\ T^{a}\ A_{\mu }^{a}~\]
其中 I 是单位矩阵(与生成元的大小相匹配),{\displaystyle \ A_{\mu }^{a}\ } 是矢量势,g 是耦合常数。在四维空间中,耦合常数 g 是一个纯数字,并且对于 SU(n) 群体,有\(\displaystyle \ a,b,c=1\ldots n^{2}-1\)”

关系
\[\displaystyle \ F_{\mu \nu }^{a}=\partial _{\mu }A_{\nu }^{a}-\partial _{\nu }A_{\mu }^{a}+g\ f^{abc}\ A_{\mu }^{b}\ A_{\nu }^{c}~\] 
可以通过交换子推导得出
\[\displaystyle \ \left[D_{\mu },D_{\nu }\right]=-i\ g\ T^{a}\ F_{\mu \nu }^{a}~\]. 
该场具有自相互作用的性质,得到的运动方程被称为半线性的,因为非线性项既有导数项也没有导数项。这意味着,只有通过小的非线性扰动理论才能处理此理论。

注意,“上标”(“反变”)和“下标”(“共变”)的矢量或张量分量之间的转换对于 a 指数来说是平凡的(例如,\(\displaystyle \ f^{abc}=f_{abc}\)),而对于 μ 和 ν 来说是非平凡的,例如对应于通常的洛伦兹符号,\(\displaystyle \ \eta _{\mu \nu }={\rm {diag}}(+---)\)

从给定的拉格朗日量中,可以导出运动方程:
\[\displaystyle \ \partial ^{\mu }F_{\mu \nu }^{a}+g\ f^{abc}\ A^{\mu b}\ F_{\mu \nu }^{c}=0.~\]
将\(\displaystyle \ F_{\mu \nu }=T^{a}F_{\mu \nu }^{a}\) ,  
这些可以重写为
\[\displaystyle \ \left(D^{\mu }F_{\mu \nu }\right)^{a}=0.~\]
一个比安奇恒等式成立:
\[\displaystyle \ \left(D_{\mu }\ F_{\nu \kappa }\right)^{a}+\left(D_{\kappa }\ F_{\mu \nu }\right)^{a}+\left(D_{\nu }\ F_{\kappa \mu }\right)^{a}=0~\]
这等价于雅可比恒等式:
\[\displaystyle \ \left[D_{\mu },\left[D_{\nu },D_{\kappa }\right]\right]+\left[D_{\kappa },\left[D_{\mu },D_{\nu }\right]\right]+\left[D_{\nu },\left[D_{\kappa },D_{\mu }\right]\right]=0~\]
因为:
\(\displaystyle \ \left[D_{\mu },F_{\nu \kappa }^{a}\right]=D_{\mu }\ F_{\nu \kappa }^{a}\)
定义对偶场强张量:
\(\displaystyle \ {\tilde {F}}^{\mu \nu }={\tfrac {1}{2}}\varepsilon ^{\mu \nu \rho \sigma }F_{\rho \sigma }\)
那么,比安奇恒等式可以重写为:
\[\displaystyle \ D_{\mu }{\tilde {F}}^{\mu \nu }=0.~\]
一个源项:\(\displaystyle \ J_{\mu }^{a}\)进入运动方程为:
\[\displaystyle \ \partial ^{\mu }F_{\mu \nu }^{a}+g\ f^{abc}\ A^{b\mu }\ F_{\mu \nu }^{c}=-J_{\nu }^{a}.~\]
注意,电流在规范群变换下必须适当变化。

我们在这里对耦合常数的物理维度做一些评论。在 D 维中,场的尺度为\(\displaystyle \ \left[A\right]=\left[L^{\left({\tfrac {2-D}{2}}\right)}\right]\),因此耦合常数的尺度必须为\(\displaystyle \ \left[g^{2}\right]=\left[L^{\left(D-4\right)}\right]\)这意味着,对于维度大于四的情况,杨–米尔斯理论是不可重正化的。此外,当 D = 4 时,耦合常数是无量纲的,且场和耦合的平方具有与无质量四次标量场理论中场和耦合相同的维度。因此,这些理论在经典层次上共享尺度不变性。
\subsection{量子化}  
量子化杨–米尔斯理论的一种方法是通过泛函方法,即路径积分。我们引入 n 点函数的生成泛函为
\[\displaystyle \ Z[j]=\int [\mathrm {d} A]\ \exp \left[-{\tfrac {i}{2}}\int \mathrm {d} ^{4}x\ \operatorname {tr} \left(F^{\mu \nu }\ F_{\mu \nu }\right)+i\ \int \mathrm {d} ^{4}x\ j_{\mu }^{a}(x)\ A^{a\mu }(x)\right]~\]
但这个积分本身没有意义,因为由于规范自由度,势矢量可以任意选择。这个问题在量子电动力学中已被认识到,但由于规范群的非阿贝尔性质,这里变得更加严重。Ludvig Faddeev 和 Victor Popov 提出了一个解决方案,通过引入一个幽灵场(参见 Faddeev–Popov 幽灵),该场具有非物理特性,因为尽管它符合费米–狄拉克统计,它是一个复标量场,违反了自旋–统计定理。因此,我们可以将生成泛函写为。
\[\displaystyle {\begin{aligned}Z[j,{\bar {\varepsilon }},\varepsilon ]&=\int [\mathrm {d} \ A][\mathrm {d} \ {\bar {c}}][\mathrm {d} \ c]\ \exp {\Bigl \{}i\ S_{F}\ \left[\partial A,A\right]+i\ S_{gf}\left[\partial A\right]+i\ S_{g}\left[\partial c,\partial {\bar {c}},c,{\bar {c}},A\right]{\Bigr \}}\\&\exp \left\{i\int \mathrm {d} ^{4}x\ j_{\mu }^{a}(x)A^{a\mu }(x)+i\int \mathrm {d} ^{4}x\ \left[{\bar {c}}^{a}(x)\ \varepsilon ^{a}(x)+{\bar {\varepsilon }}^{a}(x)\ c^{a}(x)\right]\right\}\end{aligned}}~\]
为:
\[\displaystyle S_{F}=-{\tfrac {1}{2}}\int \mathrm {d} ^{4}x\ \operatorname {tr} \left(F^{\mu \nu }\ F_{\mu \nu }\right)~\] 
对于场,  
\[\displaystyle S_{gf}=-{\frac {1}{2\xi }}\int \mathrm {d} ^{4}x\ (\partial \cdot A)^{2}~\] 
对于规范固定,
\[\displaystyle S_{g}=-\int \mathrm {d} ^{4}x\ \left({\bar {c}}^{a}\ \partial _{\mu }\partial ^{\mu }c^{a}+g\ {\bar {c}}^{a}\ f^{abc}\ \partial _{\mu }\ A^{b\mu }\ c^{c}\right)~\] 
对于幽灵。  
这是常用来推导费曼规则的表达式(见费曼图)。在这里,我们使用了幽灵场 \(c_a\),而 \(\xi\) 用来固定量子化时的规范选择。从这个泛函中得到的费曼规则如下。
\begin{figure}[ht]
\centering
\includegraphics[width=10cm]{./figures/e2e9d197a4fcbca9.png}
\caption{} \label{fig_Yang_3}
\end{figure}

这些费曼图规则可以通过将上述生成泛函重写为
\[\displaystyle {\begin{aligned}Z[j,{\bar {\varepsilon }},\varepsilon ]&=\exp \left(-i\ g\int \mathrm {d} ^{4}x\ {\frac {\delta }{i\ \delta \ {\bar {\varepsilon }}^{a}(x)}}\ f^{abc}\partial _{\mu }\ {\frac {i\ \delta }{\delta \ j_{\mu }^{b}(x)}}\ {\frac {i\ \delta }{\delta \ \varepsilon ^{c}(x)}}\right)\\&\qquad \times \exp \left(-i\ g\int \mathrm {d} ^{4}x\ f^{abc}\partial _{\mu }{\frac {i\ \delta }{\delta \ j_{\nu }^{a}(x)}}{\frac {i\ \delta }{\delta \ j_{\mu }^{b}(x)}}\ {\frac {i\ \delta }{\delta \ j^{c\nu }(x)}}\right)\\&\qquad \qquad \times \exp \left(-i\ {\frac {g^{2}}{4}}\int \mathrm {d} ^{4}x\ f^{abc}\ f^{ars}{\frac {i\ \delta }{\delta \ j_{\mu }^{b}(x)}}\ {\frac {i\ \delta }{\delta \ j_{\nu }^{c}(x)}}\ {\frac {\ i\delta }{\delta \ j^{r\mu }(x)}}{\frac {i\ \delta }{\delta \ j^{s\nu }(x)}}\right)\\&\qquad \qquad \qquad \times Z_{0}[j,{\bar {\varepsilon }},\varepsilon ]\end{aligned}}~\]
其中,
\[\displaystyle Z_{0}[j,{\bar {\varepsilon }},\varepsilon ]=\exp \left(-\int \mathrm {d} ^{4}x\ \mathrm {d} ^{4}y\ {\bar {\varepsilon }}^{a}(x)\ C^{ab}(x-y)\ \varepsilon ^{b}(y)\right)\exp \left({\tfrac {1}{2}}\int \mathrm {d} ^{4}x\ \mathrm {d} ^{4}y\ j_{\mu }^{a}(x)\ D^{ab\mu \nu }(x-y)\ j_{\nu }^{b}(y)\right)~\]
是自由理论的生成泛函。通过在 \(g\) 上展开并计算泛函导数,我们能够获得所有的 n 点函数,使用微扰理论。利用 LSZ 还原公式,我们可以从 n 点函数得到对应的过程幅度、截面和衰变率。该理论是重整化的,并且在任何微扰理论阶数下,修正是有限的。

对于量子电动力学,由于规范群是阿贝尔群,幽灵场会解耦。这可以从规范场和幽灵场之间的耦合看出:\(\displaystyle {\bar {c}}^{a}\ f^{abc}\ \partial _{\mu }A^{b\mu }\ c^{c}\)对于阿贝尔群的情况,所有结构常数\(\displaystyle f^{abc}\)都为零,因此没有耦合。在非阿贝尔群的情况下,幽灵场作为一种有用的方式出现,用于重写量子场论,并且对理论的可观测量(如截面或衰变率)没有物理影响。

杨-米尔斯理论得到的最重要的结果之一是渐近自由性。这个结果可以通过假设耦合常数 \(g\) 很小(即小的非线性效应),如高能情况下,并应用微扰理论来得到。这个结果的相关性在于,描述强相互作用和渐近自由性的杨-米尔斯理论能够妥善处理来自深度非弹性散射的实验结果。

为了获得杨-米尔斯理论在高能下的行为,从而证明渐近自由性,假设耦合常数很小并应用微扰理论。这在紫外极限下事后得到验证。在相反的极限——红外极限中,情况则完全不同,因为耦合常数过大,微扰理论在这种情况下不再可靠。研究面临的主要困难就是如何在低能下管理这理论。这才是有趣的情况,因为它本质上涉及对强子物质的描述,更广泛地说,涉及所有观察到的胶子和夸克束缚态及其束缚(见强子)。在这一极限下研究该理论的最常用方法是尝试在计算机上求解(见格点规范理论)。在这种情况下,需要大量计算资源来确保获得正确的无限体积极限(更小的格点间距)。这就是必须与之对比的极限。较小的间距和较大的耦合常数并不是独立的,需要更多的计算资源来处理每一种情况。截至今天,对于强子谱和胶子及幽灵传播子的计算,情况似乎令人满意,但胶球和混合谱依然是一个有争议的问题,因为实验上观察到这类外来态的情况尚未明确。事实上,\(\sigma\) 谐振子[10][11]在这些格点计算中没有被观察到,并且已有对立的解释被提出。这是一个热点讨论问题。
\subsection{开放问题}
杨-米尔斯理论在物理学界获得广泛认可是在1972年,当时杰拉尔德·‘t Hooft根据他的导师马丁努斯·费尔特曼的理论框架,解决了该理论的重整化问题。[12] 即使该理论描述的规范玻色子是有质量的,如电弱理论中的情况,只要质量是由希格斯机制“获得”的,即是通过希格斯机制产生的,重整化仍然成立。

杨-米尔斯理论的数学领域是一个非常活跃的研究领域,例如,通过西蒙·唐纳森的工作,获得了四维流形上可微结构的不变量。此外,杨-米尔斯理论领域已被列入克雷数学研究所的“千禧年奖难题”中。这个奖项问题,特别是涉及纯杨-米尔斯理论(即没有物质场)的最低激发具有相对于真空态的有限质量间隙的猜想证明。与此猜想相关的另一个开放问题是,在存在额外费米子的情况下,证明束缚性质。

在物理学中,杨-米尔斯理论的研究通常不是从微扰分析或解析方法开始的,而是近年来从数值方法的系统应用开始,特别是在格点规范理论中。
\subsection{参见}
\begin{itemize}
\item 阿哈罗诺夫-玻姆效应
\item 库仑规范
\item 变形赫尔米特杨-米尔斯方程
\item 规范协变导数
\item 规范理论(数学)
\item 赫尔米特杨-米尔斯方程
\item 卡鲁扎-克莱因理论
\item 格点规范理论
\item 洛伦兹规范
\item n = 4 超对称杨-米尔斯理论
\item 传播子
\item 量子规范理论
\item 标准模型的场论表述
\item 物理学中的对称性
\item 二维杨-米尔斯理论
\item 韦尔规范
\item 杨-米尔斯方程
\item F-杨-米尔斯方程
\item Bi-杨-米尔斯方程
\item 杨-米尔斯存在性与质量间隙
\item 杨-米尔斯-希格斯方程
\end{itemize}
\subsection{参考文献}
\begin{enumerate}
\item “杨-米尔斯与质量间隙”。克莱数学研究所。检索于 2024-04-09。
\item O’Raifeartaigh, Lochlainn; Straumann, Norbert (2000-01-01). “规范理论:历史起源与一些现代发展”。《现代物理学评论》。72 (1): 1–23. doi:10.1103/RevModPhys.72.1. ISSN 0034-6861。
\item Baggott, J.E. (2013). 《量子故事:40个时刻的历史》(第三版)。英国牛津:牛津大学出版社。ISBN 978-0-19-956684-6。
\item  Gray, Jeremy; Wilson, Robin (2012-12-06). 《数学对话:来自《数学智能体》的精选》。施普林格科学与商业媒体。第63页。ISBN 9781461301950 – 通过Google Books。
\item Yang, C.N.; Mills, R. (1954). “同位旋的守恒与同位规范不变性”。《物理评论》。96 (1): 191–195. Bibcode:1954PhRv...96..191Y. doi:10.1103/PhysRev.96.191。
\item Straumann, N. (2000). “论保罗在1953年发明非阿贝尔卡鲁扎-克莱因理论”。arXiv:gr-qc/0012054。
\item Atiyah, M. (2017). “罗纳德·肖 1929–2016 由迈克尔·阿提亚(1954年)”。《三一学院年度记录》(纪念)。2017: 137–146。
\item Shaw, Ronald (1956年9月). 《粒子类型问题及其他对基本粒子理论的贡献》(博士论文)。剑桥大学。第3章,第34-46页。
\item Fraser, Gordon (2008). 《宇宙的愤怒:阿卜杜斯·萨拉姆 – 第一个获得诺贝尔奖的穆斯林科学家》。英国牛津:牛津大学出版社。第117页。ISBN 978-0199208463。
\item Caprini, I.; Colangelo, G.; Leutwyler, H. (2006). "QCD中最低共振的质量和宽度"。《物理评论快报》。96 (13): 132001. arXiv:hep-ph/0512364. Bibcode:2006PhRvL..96m2001C. doi:10.1103/PhysRevLett.96.132001. PMID 16711979. S2CID 42504317。 
\item Yndurain, F.J.; Garcia-Martin, R.; Pelaez, J.R. (2007). "低能量下ππ同位旋S波的实验现状:f0(600)极点与散射长度"。《物理评论D》。76 (7): 074034. arXiv:hep-ph/0701025. Bibcode:2007PhRvD..76g4034G. doi:10.1103/PhysRevD.76.074034. S2CID 119434312。
\item 't Hooft, G.; Veltman, M. (1972). "规范场的正则化与重整化"。《核物理B》。44 (1): 189–213. Bibcode:1972NuPhB..44..189T. doi:10.1016/0550-3213(72)90279-9. hdl:1874/4845。
\end{enumerate}