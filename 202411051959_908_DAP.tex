% 达朗贝尔原理(综述)
% license CCBYSA3
% type Wiki

本文根据 CC-BY-SA 协议转载翻译自维基百科\href{https://en.wikipedia.org/wiki/D\%27Alembert\%27s_principle}{相关文章}。

达朗贝尔原理,也称为拉格朗日–达朗贝尔原理,是经典运动定律的基本表述。它以其发现者法国物理学家和数学家让·勒朗·达朗贝尔以及意大利-法国数学家约瑟夫·路易·拉格朗日命名。达朗贝尔原理通过引入惯性力,将虚功原理从静力系统推广到动力系统,当这些惯性力与系统中的外力相加时,系统达到动态平衡状态。[1][2]

达朗贝尔原理可以应用于依赖于速度的运动约束情形。[1]: 92  该原理不适用于不可逆位移(如滑动摩擦),并且需要更一般的不可逆性说明。[3][4]
\subsection{原理陈述}
该原理表述为:作用在一个由大量粒子组成的系统上的力与系统本身的动量的时间导数之间的差的和,在投影到与系统约束一致的任何虚位移上时,其总和为零。[需要澄清] 因此,用数学符号表示,达朗贝尔原理写作:
\[
\sum_{i} \left( \mathbf{F}_i - m_i \dot{\mathbf{v}}_i - \dot{m}_i \mathbf{v}_i \right) \cdot \delta \mathbf{r}_i = 0,~
\]
其中:
\begin{itemize}
\item \( i \) 是一个整数,用于表示(通过下标)系统中对应于特定粒子的变量,
\item \( \mathbf{F}_i \) 是作用在第 \( i \) 个粒子上的总外力(不包括约束力),
\item \( m_i \) 是第 \( i \) 个粒子的质量,
\item \( \mathbf{v}_i \) 是第 \( i \) 个粒子的速度,
\item \( \delta \mathbf{r}_i \) 是与约束一致的第 \( i \) 个粒子的虚位移。
\end{itemize}
牛顿的点符号用于表示相对于时间的导数。上述方程通常被称为达朗贝尔原理,但这种变分形式首次是由约瑟夫·路易·拉格朗日书写的。[5] 达朗贝尔的贡献在于证明在一个动态系统的整体中,约束力消失。也就是说,广义力 \( \mathbf{Q}_j \) 不需要包含约束力。该原理等价于更为繁琐的高斯最小约束原理。