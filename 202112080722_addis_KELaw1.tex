% 动能、动能定理(单个质点)
% 动能|动能定理|功|功率|牛顿第二定律

\pentry{功\ 功率\upref{Fwork},牛顿第二定律\upref{New3}}

令质点的质量为 $m$,速度\upref{VnA}为 $\bvec v$,则质点的\textbf{动能}定义为
\begin{equation}
E_k = \frac12 m\bvec v^2 = \frac12 mv^2
\end{equation}
注意这里的 $\bvec v^2 = \bvec v \vdot \bvec v$ 表示速度矢量和自身的内积\upref{InerPd}, 结果等于其模长的平方.

\begin{theorem}{质点动能定理}
\textbf{一段时间内质点动能的变化等于合外力对质点做的功}.从变化率(即时间导数)的角度来看,动能定理也可以表述为\textbf{质点的动能变化率等于合外力对质点的功率}.
\end{theorem}

\subsection{推导}
力对质点做功的功率\upref{Fwork}为
\begin{equation}\label{KELaw1_eq2}
P = \dv{W}{t} =  \bvec F\vdot \dv{\bvec r}{t} = \bvec F\vdot\bvec v
\end{equation}
再来看动能的变化率
\begin{equation}
\dv{t} E_k = \frac12 m\dv{t} (\bvec v\vdot\bvec v)
\end{equation}
由“ 矢量内积的求导\upref{DerV}” \autoref{DerV_eq5},$\dv*{(\bvec v\vdot\bvec v)}{t} = 2\bvec v\vdot \dv*{\bvec v}{t} = 2\bvec v\vdot\bvec a$,上式变为
\begin{equation}\label{KELaw1_eq4}
\dv{t} E_k = m\bvec a\vdot\bvec v = \bvec F\vdot\bvec v
\end{equation}
最后一步使用了牛顿第二定律(\autoref{New3_eq1}~\upref{New3}). 注意\autoref{KELaw1_eq2} 与\autoref{KELaw1_eq4} 相等,所以动能变化率等于合外力的功率.