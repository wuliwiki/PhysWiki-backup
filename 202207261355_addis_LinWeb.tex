% Linux 网络基础笔记

\begin{issues}
\issueDraft
\end{issues}

\begin{itemize}
\item \verb`sudo ifconfig` 显示网卡信息, 以及 ip 地址 (inet addr)
\item \verb`ifconfig 网卡名 up` 启动网卡, \verb`ifconfig 网卡名 down` 关闭网卡, 可以用这两个命令重启网卡
\item \verb`netplan` 是 ubuntu 18.04 的默认管理网络设置的程序, 比如设置 hdcp, 静态 ip, 掩码, 网关等, 设置完成以后用 \verb`sudo netplan apply` 可以生效. 但有时候还需要重启网卡才能成功.
\item \verb`sudo hpclient` 显示 ip 地址
\item \verb`sudo /etc/init.d/networking restart` 重启网络服务 
\item 要从某个网址下载文件, 只需安装 wget 软件即可, 使用方法如 \verb`wget http://...` 文件会下载到当前文件夹
\item \verb`ping` 可以用来检查是否有网络连接, 比如 \verb`ping google.com` 也可以用来查看某个域名的 ip 地址, 也可以直接使用 ip 地址如 \verb`ping 8.8.8.8` (8.8.8.8 是谷歌的主要 DNS 服务器)
\item 如果连不上网, 可以参考\href{https://upcloud.com/community/tutorials/troubleshoot-network-connectivity-linux-server/}{这篇文章}的步骤调试
\item 如果想 ping 某个端口, 用 \verb|telnet IP 端口号|
\end{itemize}
