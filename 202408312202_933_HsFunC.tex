% 函数的变换与性质
% keys 函数|变换|平移|旋转|伸缩|单调|对称|奇偶性|初等函数|周期|
% license Usr
% type Tutor

\subsection{函数的变换}

在实际生活中,变换的概念无处不在。比如,调整照片的大小、改变音乐的速度、甚至是地图应用中缩放和旋转视图,这些操作都与函数变换的原理息息相关。

函数有几种变换方式:平移、旋转、伸缩

\subsection{函数的性质}

函数具有一些性质,有一些在高中会接触到,有一些不会接触到。
以后我们会看到一些用\enref{极限}{Lim}和\enref{导数}{Der}描述的性质。 例如 % \addTODO{链接}
, 可导。
\subsubsection{零点}
\begin{definition}{零点}

\end{definition}
\subsubsection{单调性}

\begin{definition}{单调性}

\end{definition}

\subsubsection{反函数}

\begin{definition}{反函数}

\textbf{反函数}(inverse function,也译作\textbf{逆函数})

\end{definition}

\subsubsection{对称性}

对称性分为两种,一种是轴对称性,一种是中心对称性,这两个性质在初中就有接触过。

\begin{definition}{中心对称}
中心对称
\end{definition}

\begin{definition}{轴对称}
轴对称
\end{definition}

有两个比较特殊的对称性称为奇偶性。

\begin{definition}{奇偶性}
如果函数$f(x)$的定义域为$D$,且$D$是关于$0$对称的。若对任意的$x\in D$,有:
\begin{itemize}
\item $f(x)=f(-x)$,则称$f(x)$是偶函数。
\item $f(x)=-f(-x)$或$-f(x)=f(-x)$,则称$f(x)$是奇函数。
\end{itemize}
\end{definition}

\subsubsection{周期性}

\begin{definition}{周期}

\end{definition}

还有一些性质是高中不会涉及到的,此处给出:
\begin{itemize}
\item \enref{连续性}{contin}, 一致连续
\end{itemize}


\subsection{特殊的函数}

\subsubsection{初等函数}

性质好

基本初等函数:
\begin{itemize}
\item 常值函数
\item 幂函数
\item 指数函数
\item 对数函数
\item 三角函数
\end{itemize}

\subsubsection{分段函数}

绝对值函数

取整函数

狄利克雷函数