% 中微子
% license CCBYSA3
% type Wiki

(本文根据 CC-BY-SA 协议转载自原搜狗科学百科对英文维基百科的翻译)

中微子(/nuːˈtriːnoʊ/或/njuːˈtriːnoʊ/)(由希腊字母ν表示)是费米子(一种具有半整数自旋的基本粒子),它仅通过弱力和引力参与相互作用。[1][2]中微子之所以如此命名,是因为它是电中性的,并且因为它的静止质量非常小,以致于人们长期以来认为它为零。中微子的质量比其他已知的基本粒子小得多。弱力的范围非常短,引力相互作用非常弱,中微子作为轻子不参与强相互作用。因此,中微子通常能畅通无阻地穿过普通物质,无法被探测到。[1][2]

弱相互作用产生三味轻子类型之一的中微子:电子中微子(νe), μ子中微子(νμ),或τ中微子(ντ)并伴随相应的带电轻子。虽然中微子长期以来被认为是无质量的,但现在已经知道有三个不同微小值的离散中微子质量,但它们并不与这三种味道唯一对应。一个带有特定味道的中微子是所有三种质量状态的特定量子叠加。因此,中微子在飞行中在不同的味道之间振荡。例如,在β衰变反应中产生的电子中微子可能在远处的探测器中作为μ子或τ中微子相互作用。尽管截至2016年,只有三个质量值的平方差是已知的, 但宇宙学观察表明,三个质量的总和必须小于电子质量的百万分之一。

对于每一个中微子,也存在一个相应的反粒子,称为反中微子,它也有半整数自旋,没有电荷。它们与中微子的区别在于轻子数和手性符号相反。为了使总的轻子数守恒,在核β衰变中,电子中微子只与正电子(反电子)或电子反中微子一起出现,电子反中微子与电子或电子中微子一起出现。

中微子是由各种放射性衰变产生的,包括原子核或强子的$\beta$衰变、核反应,如发生在恒星核心或核反应堆、核弹或粒子加速器中的核反应、超新星爆发期间、中子星自旋期间以及加速粒子束或宇宙射线撞击原子时。地球附近的大多数中微子来自太阳的核反应。在地球附近,垂直于太阳方向,每平方厘米每秒大约有650亿($6.5\times10^{10}$)太阳中微子穿过。[3]

为了研究中微子,可以用核反应堆和粒子加速器人工制造中微子。有大量涉及中微子的研究活动,目标包括确定三个中微子质量值,测量轻子区的CP破坏程度(导致轻子发生);并寻找粒子物理标准模型之外的物理证据,如无中微子双$\beta$衰变,这将是轻子数守恒破坏的证据。中微子也可以用于地球内部的断层摄影。[4][5]





































































