% Crank-Nicolson 算法解一维含时薛定谔方程
% 算法|薛定谔方程|数值解|偏微分方程

\begin{issues}
\issueDraft
\end{issues}

\pentry{薛定谔方程\upref{TDSE}}

\footnote{参考 \cite{NR3}.}薛定谔方程为
\begin{equation}
-\frac12 \pdv[2]{\psi}{x} + V\psi = \I \pdv{\psi}{t}
\end{equation}
用 Crank-Nicolson 或 Caley scheme\footnote{二者是一回事, 见 \cite{NR3} 19.2 节.} 得到的结果是
\begin{equation}\label{CraNic_eq2}
\qty(1+\frac{\I}{2}\mat H\Delta t)\bvec\psi_{n+1} = \qty(1-\frac{\I}{2}\mat H\Delta t)\bvec\psi_n
\end{equation}
其中 $\bvec\psi_n$ 是时刻 $t_n$ 的波函数列矢量(已知), $\bvec\psi_{n+1}$ 为时刻 $t_{n+1}$ 的未知矢量.

但事实上, 还可以继续减少计算量. 将\autoref{CraNic_eq2} 整理后得
\begin{equation}\label{CraNic_eq5}
\qty(\frac12 + \frac{\I}{4}\mat H\Delta t)\qty(\bvec\psi_{n+1}+\bvec\psi_n) = \bvec\psi_n
\end{equation}
解这个方程, 再减去 $\bvec \psi_n$ 即可.

\subsection{等间距网格}
对于等间距网格, 可以用差分法计算二阶导数, 表示为矩阵有
\begin{equation}
\mat D_2 = \frac{1}{\Delta x^2}\pmat{-2 & 1 & 0 & 0 & \dots\\
1 & -2 & 1 & 0 & \dots\\
0 & 1 & -2 & 1 & \dots\\
0 & 0 & 1 & -2 & \dots\\
\vdots & \vdots & \vdots & \vdots & \ddots}
\end{equation}
那么 $\mat H = -\mat D_2/(2m)$ + \mat V.

\subsection{虚时间}
使用虚时间后, \autoref{CraNic_eq2} 和\autoref{CraNic_eq5} 分别变为
\begin{equation}
\qty(1+\frac12\mat H\Delta t)\psi_{n+1} = \qty(1-\frac12\mat H\Delta t)\psi_n
\end{equation}
\begin{equation}
\qty(\frac12 + \frac14\mat H\Delta t)\qty(\psi_{n+1}+\psi_n) = \psi_n
\end{equation}

(代码未完成)
