% 马约拉纳方程(综述)
% license CCBYSA3
% type Wiki

本文根据 CC-BY-SA 协议转载翻译自维基百科\href{https://en.wikipedia.org/wiki/Majorana_equation}{相关文章}。

在物理学中,\textbf{Majorana 方程}是一种相对论波动方程。它以意大利物理学家埃托雷·马约拉纳(Ettore Majorana)的名字命名,他于1937年提出这一方程用于描述那些自身即为反粒子的费米子。依据这一方程的粒子被称为\textbf{Majorana 粒子}。然而,如今这个术语涵盖了更广泛的意义,指任何(可能是非相对论的)自身为反粒子的费米子,因此这些粒子必然是电中性的。

有理论提议认为,具有质量的中微子可以用 Majorana 粒子来描述;标准模型的各种扩展允许这种可能性。关于 Majorana 粒子的文章中包含了实验搜索的最新进展,包括中微子的相关细节。而本文则主要关注该理论的数学发展,特别是其离散和连续对称性。离散对称性包括\textbf{电荷共轭}、\textbf{宇称变换}和\textbf{时间反演};连续对称性为\textbf{洛伦兹不变性}。

电荷共轭在其中扮演了重要角色,这是使得 Majorana 粒子能够被描述为电中性的关键对称性。一个特别值得注意的特性是,电中性允许对左右手螺旋场的全局相位进行自由选择。这意味着,在没有显式限制这些相位的情况下,Majorana 场天然是 CP 破坏的。电中性带来的另一个特性是,左右手螺旋场可以被赋予不同的质量。换句话说,\textbf{电荷}是洛伦兹不变量,同时也是运动常数;而\textbf{手性}则是洛伦兹不变量,但对于具有质量的场来说不是运动常数。因此,电中性的场受到的约束比带电场更少。在电荷共轭作用下,这两个自由的全局相位出现在质量项中(因为它们是洛伦兹不变量),因此 Majorana 质量被描述为一个复矩阵,而不是一个单一数值。

简而言之,Majorana 方程的离散对称性远比 Dirac 方程复杂。在 Dirac 方程中,电荷 
\( U(1) \) 对称性约束并消除了这些自由度,而在 Majorana 方程中,这些自由度得以保留。
\subsection{定义} 
Majorana 方程可以以几种不同的形式表示:
\begin{itemize}
\item 作为 Dirac 方程的形式,其中 Dirac 算符是纯粹厄米的,从而得到纯实数解。  
\item 作为一个算符,将一个四分量自旋子与其电荷共轭相关联。  
\item 作为一个作用于复数二分量自旋子的 2×2 微分方程,类似于 Weyl 方程,但具有适当的洛伦兹协变质量项。
\end{itemize}  
这三种形式是等价的,可以相互推导。每种形式提供了对方程性质的略有不同的理解。第一种形式强调可以找到纯实数解。第二种形式阐明了电荷共轭的作用。第三种形式则提供了与洛伦兹群表示理论最直接的联系。
\subsubsection{纯实数四分量形式}
常规的起始点是表述为“Dirac 方程可以写成厄米形式”,当伽马矩阵采用 Majorana 表示时。Dirac 方程可以写成如下形式:[6]
\[
\left( -i \frac{\partial}{\partial t} - i \hat{\alpha} \cdot \nabla + \beta m \right) \psi = 0~
\]
其中,\(\hat{\alpha}\) 是纯实数的 4×4 对称矩阵,而 \(\beta\) 是纯虚数的斜对称矩阵;这要求确保括号内的算符是厄米算符。在这种情况下,可以找到该方程的纯实数 4-自旋子解,这些解就是 Majorana 自旋子。
\subsubsection{电荷共轭四分量形式} 
Majorana 方程为:
\[
i \, \partial \!\!\!{\big /} \psi - m \, \psi_c = 0~
\]
其中,导数算符 \(\partial \!\!\!{\big /}\) 采用费曼斜线符号表示,包括伽马矩阵以及对自旋子分量的求和。自旋子 \(\psi_c\) 是 \(\psi\) 的电荷共轭。根据构造,电荷共轭必然由以下公式给出:
\[
\psi_c = \eta_c \, C \, {\overline{\psi}}^{\mathsf{T}}~
\]
其中,\((\cdot)^{\mathsf{T}}\) 表示转置,\(\eta_c\) 是一个任意相位因子,满足 \(|\eta_c| = 1\),通常取 \(\eta_c = 1\),而 \(C\) 是 4×4 矩阵,即电荷共轭矩阵。矩阵 \(C\) 的表示依赖于伽马矩阵的选择。根据惯例,共轭自旋子写作:
\[
{\overline{\psi}} = \psi^\dagger \, \gamma^0~
\]
由电荷共轭矩阵 \(C\) 可以推导出若干代数恒等式。[a] 其中之一表明,在任何伽马矩阵的表示下(包括 Dirac、Weyl 和 Majorana 表示),有:
\[
C \, \gamma_\mu = - \gamma_\mu^{\mathsf{T}} \, C~
\]
因此可以写为:
\[
\psi_c = - \eta_c \, \gamma^0 \, C \, \psi^*~
\]
其中,\(\psi^*\) 是 \(\psi\) 的复共轭。电荷共轭矩阵 \(C\) 还具有以下性质:
\[
C^{-1} = C^\dagger = C^{\mathsf{T}} = -C~
\]
在所有表示下(Dirac、手性、Majorana)。由此,通过一些代数运算,可以得到等效方程:
\[
i \, \partial \!\!\!{\big /} \psi_c - m \, \psi = 0~
\]