% 阿贝尔-鲁菲尼定理(综述)
% license CCBYSA3
% type Wiki

本文根据 CC-BY-SA 协议转载翻译自维基百科 \href{https://en.wikipedia.org/wiki/Abel\%E2\%80\%93Ruffini_theorem}{相关文章}。

在数学中,阿贝尔–鲁芬尼定理(也称为阿贝尔不可能性定理)指出:对于一般的五次及更高次数的多项式方程,无法通过根式(即有限次加减乘除和开方)求出其解。这里的“一般”是指将方程的系数看作不定元,并对其进行运算。

该定理以保罗·鲁芬尼和尼尔斯·亨里克·阿贝尔的名字命名。鲁芬尼于1799年给出了一个不完整的证明(该证明在1813年被完善,并被柯西接受),而阿贝尔在1824年提供了完整的证明。

“阿贝尔–鲁芬尼定理”也常指一个稍强的命题:存在某些五次或更高次数的方程无法用根式求解。这个结论虽不直接出现在阿贝尔的定理陈述中,但可以从他的证明推导而来——他的证明基于这样一个事实:某些由方程系数组成的多项式不是零多项式。这个更强的结论也可以直接从伽罗瓦理论中的“一个不可解的五次方程例子”中得出。

伽罗瓦理论还暗示:
$$
x^5 - x - 1 = 0~
$$
是最简单的不能用根式求解的方程。而且,几乎所有的五次及以上次数的多项式都无法用根式求解。

这种在五次及以上多项式中求解的“不可能性”,与低次数的情况形成鲜明对比:二次方程有求根公式,三次方程有三次公式,四次方程有四次公式,但五次及以上则无通解公式。
