% 公司简介
% keys 小时科技|小时百科|理工教材
% license Xiao
% type Tutor

广州小时科技有限公司成立于 2020 年 8 月, 是一家新兴的互联网创业公司。 我们致力于提供面向国内理工科相关专业本科生和研究生的优质网络学习和交流平台。

\subsection{主打产品}
公司目前主打产品为小时百科(\href{https://wuli.wiki}{wuli.wiki})。 小时百科目前主要提供数学、物理、计算机等专业的本科和研究生核心课程原创中文内容, 适用于课前预习、课后辅导,以及考研、科研等场景。 百科在形式上以文字图片为主, \href{https://wuli.wiki/apps/}{动画与互动演示}为辅,同时提供\href{https://wuli.wiki/tree/}{知识树状图}等思维工具。 与传统的百科网站不同,小时百科更偏向教学,注重知识的系统性,连贯性,强调依赖关系,同时配有例题和习题,使其更适合用于自学。\href{https://wuli.wiki/online/}{网页版百科}包含目前所有图文内容。

\begin{figure}[ht]
\centering
\includegraphics[width=14.25cm]{./figures/d6aa6ea5dda175c2.png}
\caption{小时百科\href{https://wuli.wiki/tree/}{知识树状图}} \label{fig_Compny_2}
\end{figure}

百科的多数内容创作由多名志愿者和签约作者共同进行, 创作者多为国内外著名高校(包括清华、北大、中科大、中科院、麻省理工等)的在读研究生或优秀本科生。 满足一定资质的创作者提交申请后方可进行创作, 创作内容经过内部审阅后发布上线, 同时也根据读者反馈不断完善。除签约创作外,也支持使用免费开源协议进行公益创作,详见\href{https://wuli.wiki/forum/f9ec7f8e-ca37-4278-a77e-ba5c0e40e115}{加入创作页面}。

我们的目标是建立中文互联网上专业易用的理工科知识库,以及专业的学术交流社区和个人展示平台。 目前已上线文章约 2000 个,共计约 201 万字, 折合纸质书约 6000 页。\footnote{基于 2023 年 10 月数据。}

更详细的介绍见 “关于小时百科\upref{about}”。

\subsection{核心技术}
\begin{itemize}
\item 经过专业技术团队研发,我们自主研发了基于学术排版语言 LaTeX 的\href{https://wuli.wiki/editor/}{在线文章编辑器},生成出版级排版效果的网页和 pdf。\footnote{由于版权原因,最新完整版 pdf 暂不提供下载,历史版本下载见\href{https://wuli.wiki/forum/}{讨论版}的相关说明。} 
\item 我们正在用同一套编辑器技术给用户建立个人百科知识库和个人主页,如需试用见\href{https://wuli.wiki/forum/}{讨论版}左侧菜单(需注册登录)。
\item 互动插件制作, 以及各种演示动画, 详见\href{http://wuli.wiki/apps}{这里}。
\end{itemize}

\begin{figure}[ht]
\centering
\includegraphics[width=14.25cm]{./figures/784e78f094665235.png}
\caption{小时百科编辑器} \label{fig_Compny_1}
\end{figure}
