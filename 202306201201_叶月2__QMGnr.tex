% 产生和湮灭
% 多粒子体系|全同粒子|Fock空间
\begin{issues}
\issueDraft
\end{issues}
旧称量子场论的工作是“二次量子化”,但这种描述是不准确的。不仅仅是因为两套体系几乎是同时发展的,更是因为在概念上,量子场论仅仅是对经典场论中的场进行一次量子化,使得场不再是描述幅值随时空变换的函数,而是能描述粒子产生和湮灭的算符。为了算符化,我们需要引入粒子产生和湮灭算符。这是量子场论中的基本概念,也是构筑多粒子体系不可或缺的砖瓦。


\subsection{定义}
用$\ket{0}$来定义能级最低,可视作没有粒子的真空态。用$ a, a^\dagger$来湮灭和产生粒子,即
\begin{equation}
\begin{aligned}
a_{\vec p}\ket{0}&=0\\
a^\dagger_{\vec p}\ket{0}&=\ket{\vec p}
\end{aligned}
\end{equation}
在扩展理论的过程中,我们需要格外留心原本理论中不变量的适用条件。在量子力学里,为了符合物理现实和数学理论,可观测量,比如力学算符本征值或者传播振幅是不随“参考系”而改变的,这里的参考系是不同表象或者绘景。然而量子场论里往往需要研究的是洛伦兹不变量,原本的$\braket{\vec q}{\vec p}$并不是洛伦兹不变量。

可以证明,洛伦兹不变测度$\int\mathrm d^4 p=\int\frac{\mathrm d^3 p}{2E_{\vec p}}$,因而$2E_{\vec p}\delta^3 (\vec p-\vec q)$才是洛伦兹不变量。于是在场论里,不变"传播振幅"可修改为
\begin{equation}
\braket{ q}{ p}=(2 \pi)^3 2 E_{\vec q}\delta^3 (\vec q-\vec p)~,
\end{equation}
其中$\ket{p}=\sqrt{2 E_\vec p}\ket p$。
\subsection{传播子与对易关系}
在没有产生和湮灭算符之前,场方程的通解是函数的线性叠加。比如众所周知的$K-G$方程
(\partial_\mu\partial^\mu+m^2)\phi=0
\subsection{Fock空间}
\subsection{多粒子体系的算符}



