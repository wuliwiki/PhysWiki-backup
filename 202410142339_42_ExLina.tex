% 线性泛函的延拓
% keys 延拓|泛函|Hahn-Banach定理
% license Usr
% type Tutor

\pentry{齐次凸泛函\nref{nod_ConFul}}{nod_4b3f}
在分析学中函数的延拓是一类重要的问题,而函数空间可以用线性空间来描述,因此函数的延拓是对线性泛函的延拓。本文将讲述在线性泛函延拓的一般概念,及在线性泛函延拓的整个问题中起着重要作用的定理——Hahn-Banach定理。

\begin{definition}{延拓}
设 $L$ 是实线性空间,而 $L_0$ 是它的某一子空间,$f_0$ 是在 $L_0$ 上定义的线性泛函。若 $f$ 是在 $L$ 上定义的线性泛函,且对 $\forall x\in L_0$,成立 $f(x)=f(x_0)$,则称 $f$ 是 $f_0$ (在全空间 $L$ 上)的\textbf{延拓}。
\end{definition}


\begin{theorem}{Hahn-Banach}
设 $p$ 是线性空间 $L$ 上的\enref{齐次凸泛函}{ConFul},$L_0$ 是 $L$ 中的线性子空间。如果 $f_0$ 是 $L_0$ 上的线性泛函,且从属于 $p$ 在 $L_0$ 上的限制,即在 $L_0$ 上
\begin{equation}
f_0(x)\leq q(x),~
\end{equation}
则 $f_0$ 在 $L$ 上的延拓 $f$ 存在,且 $f$ 从属于 $p$。
\end{theorem}



