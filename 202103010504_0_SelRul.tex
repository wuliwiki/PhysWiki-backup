% 氢原子的选择定则
% keys 选择定则|宇称|3j 符号

\begin{issues}
\issueAbstract
\end{issues}

\pentry{3j 符号\upref{ThreeJ}, 球谐函数\upref{SphHar}, 微扰理论\upref{TDPT}}

矩阵元的选择定则(什么时候为零)决定了某两个径向波函数的微分方程是否耦合. 如果没有耦合, 一阶微扰的跃迁概率为零, 但有可能多阶微扰仍然存在耦合. 即从初态耦合到中间态, 中间态再耦合到末态. $n$ 次耦合需要用至少 $n$ 阶的微扰理论. 如果高阶微扰也被禁止, 那么就是绝对禁止的. % 未完成: 不确定是不是这么定义的

\subsection{利用 3j 符号的对称性推导}
\pentry{3j 符号\upref{ThreeJ}}
相比与算符对易, 3j 符号的好处是不仅能得到选择定则,还可以直接算出偶极子矩阵元的具体值而无需手动积分\footnote{当然, 手动 3j 符号也比较繁琐, 可以借助 Wolfram Alpha 或 Mathematica, Matlab 中我也写了一个程序(同样可以符号计算), 还没放进百科.}. 长度规范\upref{LenGau}中电场哈密顿量为 $\bvec{\mathcal{E}}\vdot \bvec r$. 角向的选择定则可以由\textbf{跃迁偶极子矩阵元(transition dipole matrix element)} $\mel{Y_{l,m}}{\bvec r}{Y_{l',m'}}$ 来决定
\begin{equation}
\bvec{\mathcal{E}}\vdot \mel{Y_{l,m}}{\bvec r}{Y_{l',m'}} = \mathcal{E}_x \mel{Y_{l,m}}{x}{Y_{l',m'}} + \mathcal{E}_y \mel{Y_{l,m}}{y}{Y_{l',m'}} + \mathcal{E}_z \mel{Y_{l,m}}{z}{Y_{l',m'}}
\end{equation}
来决定. 而(\autoref{SphHar_eq4}~\upref{SphHar})
\begin{equation}
x = \sqrt{\frac{2\pi}{3}} r (Y_{1,-1} - Y_{1,1}) \qquad
y = \I \sqrt{\frac{2\pi}{3}} r (Y_{1,-1}+Y_{1,1}) \qquad
z = \sqrt{\frac{4\pi}{3}} rY_{1,0}
\end{equation}

使用\autoref{SphCup_eq1}~\upref{SphCup} 以及 3j 符号的对称性就可以得到选择定则
\begin{equation}\label{SelRul_eq1}
\mel{Y_{l,m}}{Y_{1,m_1}}{Y_{l',m'}} = \sqrt{\frac{3(2l+1)(2l'+1)}{4\pi}} \pmat{l & 1 & l'\\ 0 & 0 & 0}\pmat{l & 1 & l'\\ -m & m_1 & m'}
\end{equation}
由选择定则 $-m + m_1 + m' = 0$ 得 $\Delta m = -m_1$. 所以
\begin{equation}
\Delta m =
\begin{cases}
0 & (\text{电场只延 $z$ 方向}) \\
0, \pm 1 & (\text{其他方向电场})
\end{cases}
\end{equation}

由三角约束($\abs{l-l'} \leqslant 1 \leqslant l + l'$)得 $\Delta l = 0, \pm 1$. 但由\autoref{ThreeJ_eq8}~\upref{ThreeJ} 得 $l + l' + 1$ 为奇数时\autoref{SelRul_eq1} 为零, 所以只能有
\begin{equation}
 \Delta l = \pm 1
\end{equation}
这是两个常见的选择定则.

注意符合以上选择定则的偶极子矩阵元仍然有可能为零. 注意 3j 符号还有其他选择定则.
\addTODO{可能有误,需检查}

\subsubsection{物理意义}
$z$ 方向的电场不会改变电子 $z$ 方向的角动量, 所以 $L_z$ 守恒\upref{QMcons}, $\Delta m$ 为 0. 另外由于光子的角动量量子数为 $l=1$ (链接未完成), 所以 $\Delta l = 0, \pm 1$, 但 $\Delta l = 0$ 仍然需要对称性来排除.
