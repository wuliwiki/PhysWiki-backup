% 质点系的动量

\pentry{动量\ 动量定理(单个质点)\upref{PLaw1}, 质点系\upref{PSys}, 矢量的导数\upref{DerV}}

质点系的总动量为
\begin{equation}\label{SysMom_eq1}
\bvec p = \sum_i m_i \bvec v_i = \sum_i m_i \dot{\bvec r}_i = \dv{t}  \sum_i m_i \bvec r_i
\end{equation}
由质心的定义(\autoref{CM_eq1}\upref{CM}) 
\begin{equation}
\sum_i m_i \bvec r_i = M \bvec r_c
\end{equation}
其中 $\bvec r_c$ 为质心的位置, $M = \sum_i m_i$ 为质点系的总质量. 对时间求导并带入\autoref{SysMom_eq1} 得
\begin{equation}\label{SysMom_eq2}
\bvec p = M\bvec v_c
\end{equation}
其中 $\bvec v_c = \dot{\bvec r}_c$ 是质心的速度.

\autoref{SysMom_eq2} 告诉我们一个简洁的结论: 在求一个系统的总动量时, 我们可以把它看作质心处质点.

\begin{example}{滚动的圆盘}
一个质量为 $M$ 均匀圆盘在地面延直线滚动, 若将其分割为许多小份, 使用 $\sum_i m_i \bvec v_i$ (或者用积分形式)求总动量会比较麻烦. 但如果直接用\autoref{SysMom_eq2}, 我们可以马上写出它的总动量为 $\bvec p = M\bvec v$, 
\end{example}
