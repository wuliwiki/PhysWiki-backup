% 玻尔兹曼因子(综述)
% license CCBYNCSA3
% type Wiki

本文根据 CC-BY-SA 协议转载翻译自维基百科\href{https://en.wikipedia.org/wiki/Boltzmann_distribution}{相关文章}。

在统计力学和数学中,**玻尔兹曼分布**(也称为**吉布斯分布**[1])是一种概率分布或概率测度,用于描述一个系统处于某一状态的概率,它是该状态能量和系统温度的函数。该分布的形式为:

\[
p_i \propto \exp\left(-\frac{\varepsilon_i}{kT}\right)
\]

其中:
- \( p_i \) 是系统处于状态 \( i \) 的概率;
- \( \exp \) 是指数函数;
- \( \varepsilon_i \) 是该状态的能量;
- 常数 \( kT \) 是玻尔兹曼常数 \( k \) 与热力学温度 \( T \) 的乘积。

符号 \( \propto \) 表示**成比例关系**(关于比例常数的具体内容,见“分布”一节)。