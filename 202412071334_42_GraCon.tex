% 图的连通性
% keys 连通
% license Usr
% type Tutor
\pentry{链、路、圈、回\nref{nod_PatCyc}}{nod_11cb}
图上两点的连通性是指有以这两点为端点的\enref{路}{PatCyc}存在,而连通图是指任意两点都连通的图。

\begin{definition}{连通}
设 $G$ 是图,$x,y\in V(G)$。若存在连接 $x,y$ 的路,则称 $x,y$ 是\textbf{连通的}(connected)。
\end{definition}

\begin{theorem}{}
连通关系是图上的\enref{等价关系}{Relat}。
\end{theorem}
\textbf{证明:}
1.\textbf{自反性:} 设 $x,x$ 连通,那么 $x,x$ 连通。

2.\textbf{对称性:} 设 $x,y$ 连通。于是存在路 $xe_1\cdots e_m y$,而 $ye_m\cdots e_1 x$ 显然也是路,所以 $y,x$ 连通。

3.\textbf{传递性:}设 $x,y$ 连通,$y,z$ 连通。于是存在路 $xe_1\cdots e_m y$ 和 $ye_{m+1}\cdots e_{n}z$。于是
\begin{equation}
xe_1\cdots e_m ye_{m+1}\cdots e_{n}z~
\end{equation}
是连接 $x,z$ 的链。由\autoref{the_PatCyc_1},存在连接 $x,z$ 的链,即 $x,z$ 连通。

\textbf{证毕!}
由于连通关系是 $V(G)$ 的等价关系,因此其可以将 $V(G)$ 分成不相交的等价类,每一类都称为 $$
\begin{corollary}{}

\end{corollary}












