% 第四代反应堆
% license CCBYSA3
% type Wiki

(本文根据 CC-BY-SA 协议转载自原搜狗科学百科对英文维基百科的翻译)

\textbf{第四代反应堆}是由第四代核能系统国际论坛目前正在研究的一系列用于商业应用的核反应堆设计方案。器技术准备水平从需要示范的水平到具有经济竞争力的实施水平之间存在差异。[1]它们的设计理念是为了实现各种各样的目标,其中包括提高安全性、提高可持续性、提高反应堆运行效率以及降低安装成本、运行成本等。

目前最先进的第四代反应堆设计为钠冷快堆,多年来它获得了最大份额的资金,并且运行了许多示范设施。 该设计的第四代主要内容是为反应堆开发出一种可持续的闭式燃料循环。 熔盐堆是一种较不成熟的技术,但在六种第四代反应堆设计方案中被认为可能具有最大的固有安全性。[2][3]超高温反应堆被设计在更高的温度下运行。这允许在反应堆运行期间可进行高温电解,从而有效地生产氢气以及合成碳中性燃料[1]

这六种第四代反应堆设计中的大多数需要到2020-2030年期间才能用于商业建设。[4]目前,世界上运行的大多数反应堆被认为是第二代反应堆系统,因为绝大多数第一代系统在一段时间前已经退役,截至2014年,只有少数第三代反应堆在运行。第五代反应堆是指纯理论的反应堆,在短期时间内尚不可行,因此相应的研发资金相当有限。

\subsection{历史}
第四代核能系统国际论坛(GIF)是“一项国际性的合作努力,旨在开展必要的研究和开发,以实现下一代核能系统的可行性和性能。”[5]它成立于2001年。目前,第四代核能系统国际论坛的活跃成员包括:澳大利亚,加拿大、中国、欧洲原子能共同体(Euratom)、法国、日本、俄罗斯、南非、韩国、瑞士和美国。非活跃成员包括:阿根廷、巴西和英国。[6]瑞士于2002年加入,欧洲原子能共同体(Euratom)于2003年加入,中国和俄罗斯于2006年加入,澳大利亚[7]2016年加入。其余国家是创始成员国。[6]

第36届GIF会议于2013年11月在布鲁塞尔举行。[8][9]《第四代核能系统技术路线图更新》于2014年1月发表,详细介绍了未来十年的研发目标。[10]每个论坛成员正在研究的反应堆设计的细目已经公布。[11]

据报道,2018年1月,“世界上第一个第四代反应堆压力容器盖的首次安装”已在HTR项目中完成。[12]

\subsection{反应堆类型}
最初考虑了许多反应器类型;但是为了将重点放在最有前景的技术和那些最有可能实现第四代计划目标的技术上,删减了许多名单上的方案。[4]三个设计方案名义上是热中子反应堆,四个是快中子反应堆。超高温反应堆(VHTR)也正在研究中,它可为氢生产的过程中提供高质量的热能。快中子反应堆提供了燃烧锕系元素以进一步减少废物的可能性,并且能够“产生比它们消耗的更多的燃料”。这些系统在可持续性、安全性和可靠性、经济性、防扩散能力(视情况而定)和物理保护方面取得了重大进展。

\subsubsection{2.1 热中子反应堆}
热中子反应堆是使用慢中子或者热中子的核反应堆。中子慢化剂用于减缓裂变所释放的中子,使它们更有可能被燃料俘获。

\textbf{超高温堆}

\begin{figure}[ht]
\centering
\includegraphics[width=8cm]{./figures/bb468d9b88c246d0.png}
\caption{高温堆} \label{fig_FYD_1}
\end{figure}

\textbf{超高温反应堆的}概念是使用了石墨作为中子慢化剂以、铀作为燃料,使用氦或者熔盐作为冷却剂。该反应堆的设计出口温度为1000℃。反应堆堆芯可以是棱柱状块体或者球床反应堆设计。高温使得诸如工艺热或通过热化学碘-硫工艺生产氢气的应用成为可能。超高温反应堆也具有固有安全性。

第一座超高温反应堆的计划建设是南非的PBMR ( 床模块化反应堆)。但在2010年2月,政府中断了对该项目的资助。[13]该项目成本的显著增加以及担忧可能出现的意外技术问题令潜在投资者和客户望而却步。

2012年,中国开始建造一座200兆瓦的高温球床反应堆,作为HTR-10反应堆的延续。[14]

同样在2012年,作为下一代核电厂竞争的一部分,Idaho National Laboratory 批准了一项类似于阿海珐(Areva)棱柱状块体安塔雷斯(Antares)反应堆的设计,作为选定的HTGR,将于2021年作为原型堆部署。它与通用原子(General Atomics) 的燃气轮机模块化氦反应堆和西屋的球床模块高温气冷堆竞争。[15]

X-energy 获得了美国能源部(Department of Energy)一份为期五年、价值5300万美元的先进反应堆概念合作协议(Advanced Reactor Concept cooperation Agreement),用于推进其反应堆开发的各项内容。[16] Xe-100 是球床模块化反应堆,将产生200MWt和大约76MWe。标准Xe-100“四包式工厂可生产约300 MWe,占地仅13英亩。Xe-100的所有组件都可通过公路进行运输,并且在项目现场进行安装,而不是建造,从而可大大简化施工。[17]

\textbf{熔盐反应堆(MSR)}

\begin{figure}[ht]
\centering
\includegraphics[width=8cm]{./figures/0a53dbafe1d63342.png}
\caption{熔盐堆(MSR)} \label{fig_FYD_2}
\end{figure}

\textbf{熔盐反应堆}[18]是一种核反应堆,其主冷却剂,甚至是燃料本身都是熔盐混合物。对于这种类型的反应堆,科研人员已经提出了许多设计方案,并建造了一些原型堆。早期的概念以及当前的许多概念都在于核燃料方案的选择,四氟化铀(UF4)或四氟化钍(ThF4)溶解在熔融的氟化物盐中。该流体将流入一个石墨充当慢化剂的核心,从而达到临界状态。当前的许多概念依赖于分散在石墨基体中的燃料,其中熔融盐提供了一个低压、高温冷却的环境。

第四代反应堆中的熔盐反应堆应该称为超热中子反应堆,而不是热中子反应堆,因为堆芯中导致其燃料发生裂变事件的中子平均速度比热中子更快[19]

MSR的工作原理可用于热中子反应堆、超热中子反应堆和快中子反应堆。自2005年以来,焦点已转向快速频谱MSR (MSFR)。[20]

虽然大多数MSR的设计主要源自20世纪60年代的熔盐反应堆实验(MSRE),熔盐堆相关技术的发展包括概念上的双流反应堆设计以及使用铅作为冷却介质,但熔盐燃料通常为金属氯化物,例如三氯化钚,从而提高“核废料”封闭式燃料循环能力。其他显著不同于MSRE的方法包括MOLTEX提出的稳定盐反应堆(SSR)概念,该概念将熔融盐包裹在数百个已经在核工业中得到广泛应用的普通固体燃料棒中。2015年,总部位于英国的咨询公司Energy Process development发现,后一种英国设计被认为是最具竞争力的小型模块化反应堆。[21][22]

MSR的另一个显著特点是有可能使用热光谱核废料燃烧器。传统上认为,只有快速反应堆被认为是可行的利用或减少核废料。Seaborg Technologies在2015年春季发布的白皮书中首次展示了热废热燃烧器的概念可行性。[23]热废物的燃烧是通过用钍代替乏燃料中的一小部分铀来实现的。超铀元素(例如钚和镅)的净生产量率降低到消耗率以下,从而减少了核储存问题的规模,且不涉及核扩散以及与快堆相关的其他技术问题。

\textbf{超临界水冷反应堆(SCWR)}

\begin{figure}[ht]
\centering
\includegraphics[width=8cm]{./figures/cbed4f48e66bc5db.png}
\caption{v} \label{fig_FYD_3}
\end{figure}

\textbf{超临界水堆(SCWR)}[18]是一个减温水反应堆概念,由于导致燃料内裂变事件的中子平均速度比热中子快,因此它被更准确地称为超热中子反应堆而不是热中子反应堆。它使用超临界水作为工作流体。超临界水冷反应堆基本上是在较高的压力和温度下运行的轻水反应堆(LWR),具有直接的单程热交换循环。正如最常见的设想,它将在直接循环中运行,与沸水堆(BWR )很相似,但是由于它使用超临界水(不要与临界质量混淆)作为工作流体,因此它将只存在一个水相,这使得超临界热交换方法更类似于压水堆( PWR )。它可以在比当前压水堆和沸水堆高得多的温度下运行。

超临界水冷反应堆(SCWRs)是有前途的先进核系统,因为它们具有高的热效率(其效率约为45\%,而目前轻水反应堆的效率约为33\%)和显著的设备简化。

超临界水反应堆的主要任务是生产低成本的电力。它基于两种成熟的技术,一种是轻水堆,这是世界上最常用的发电反应堆,另一项是过热的化石燃料锅炉,世界各地也在大量使用这种锅炉。13个国家的32个组织正在进行超临界水反应堆的相关研究。

由于超临界水反应堆是水反应堆,它们与沸水堆和轻水堆一样,都具有蒸汽爆炸和放射性蒸汽释放的危险,并且需要极其昂贵的重型压力容器、管道、阀门和泵。由于超临界水反应堆在较高的温度下运行,这些共同的问题对超临界水反应堆来说,本质上更加严重。

正在开发中的超临界水反应堆设计是VVER -1700/393(VVER-SCWR或VVER-SKD),这是俄罗斯的一个超临界水冷反应堆,具有双入口堆芯,其增值比为0.95。[24]

\subsubsection{2.2 快中子反应堆}
快中子反应堆直接利用裂变释放出的快中子,不进行慢化。与热中子反应堆不同,快中子反应堆可以被配置为”\textbf{烧伤}“,或裂变,所有的锕系元素,并给予足够的时间,因此大大降低了当前世界热中子轻水反应堆产生的乏核燃料中锕系元素的分数,从而结束了核燃料循环。或者,如果配置不同,它们也可以\textbf{类型}锕系元素燃料比它们消耗的要多。

\textbf{气冷快中子反应堆}

\begin{figure}[ht]
\centering
\includegraphics[width=8cm]{./figures/103cb2902b09e106.png}
\caption{气冷快中子反应堆} \label{fig_FYD_4}
\end{figure}

\textbf{气冷快堆(GFR)}[18]系统具有快中子能谱和封闭的燃料循环,可有效地转换富铀和处理锕系元素。该反应堆采用氦气冷却,出口温度为850℃。 是超高温反应堆(VHTR)向更可持续的燃料循环的发展。它将直接使用一个布雷顿循环燃气轮机,热效率很高。目前正在考虑几种燃料形式,因为它们有可能在非常高的温度下工作,并确保裂变产物的良好保留:复合陶瓷燃料、先进的燃料颗粒或锕系化合物的陶瓷包覆元素。并且正在考虑基于销或板基燃料组件或棱柱块的堆芯配置。

欧洲可持续核工业倡议(European Sustainable Nuclear Industrial Initiative)正在为第三代(Generation IV)反应堆系统提供资金,其中一个是名为Allegro的气冷快堆(100 MW(t)),将在中欧或东欧国家建造,预计2018年开始建设。[25]]欧洲中部Visegrad集团致力于这项技术的研发。[26]德国、英国和法国的研究机构完成了一项为期3年的工业规模设计合作研究,称为GoFastR。[27]它们是由欧盟第七届FWP框架方案资助的,其目标是制造一个可持续的VHTR。[28]

\textbf{钠冷快中子反应堆(南斯拉夫)}

\begin{figure}[ht]
\centering
\includegraphics[width=8cm]{./figures/7a5308d9f65dc5f5.png}
\caption{钠冷快中子反应堆泳池设计公司} \label{fig_FYD_7}
\end{figure}

两个最大的商业钠冷快堆都在俄罗斯,BN-600(600兆瓦)和BN-800(800兆瓦)。迄今为止运行的最大的反应堆是发电量超过1200兆瓦的Superphenix反应堆,该反应堆在法国成功运行了数年,直到1996年退役。在印度,快中子增殖试验堆在1985年10月达到临界状态。2002年9月,FBTR的燃料燃烧效率首次达到每公吨铀10万兆瓦日(MWd/MTU)的标准。这被认为是印度增殖反应堆技术的一个重要里程碑。根据快速增殖堆原型堆FBTR的运行经验,一座500 MWe钠冷式快速反应堆的建设造价567.7亿卢比(约9亿美元),预计到2019年6月将达到关键水平。在PFBR之后,还将有6个商用快中子增殖反应堆(CFBRs),每个反应堆的容量为600兆瓦。

第四代钠冷快中子反应堆(南斯拉夫)[18]是一个建立在两个现有钠冷快堆工程基础上的项目,即氧化物燃料快速增殖反应堆和金属燃料整体快速反应堆。

目标是通过钚元素再生和消除超铀元素同位素,使之离开堆芯来提高反应堆中铀的使用效率。反应堆堆芯运行在快中子未经慢化的环境,设计允许任何超铀同位素被消耗(在某些情况下用作燃料)。除了从废物循环中去除长半衰期超铀酸盐的好处外,当反应堆过热时,SFR燃料会膨胀,链式反应会自动减慢。因此,它具有固有安全性。[29]

\begin{figure}[ht]
\centering
\includegraphics[width=8cm]{./figures/84ca578b2b8bd07c.png}
\caption{可持续燃料循环是在20世纪90年代提出的整体快堆概念(颜色),一个动画的热解技术也是可用的。[19]} \label{fig_FYD_5}
\end{figure}
一种SFR反应堆的概念是由液态钠冷却,由铀和钚的金属合金或乏燃料(轻水反应堆的“核废料”)提供燃料。SFR燃料包含在钢包层中,在组成燃料组件的包层元件之间的空间中填充液态钠。SFR的设计挑战之一是处理钠的风险,钠一旦与水接触就会发生爆炸反应。然而,使用液态金属代替水作为冷却剂可以使系统在常压下工作,降低泄漏的风险。

欧洲可持续核工业计划正在资助三个第四代反应堆系统,其中一个是钠冷快堆ASTRID,用于工业示范的先进钠技术反应堆。阿海珐、 CEA 和 EDF 正在与英国合作设计。[31][32]阿斯特丽德的额定功率约为600兆瓦,拟在法国建造,靠近凤凰号反应堆。建设的最终决定将于2019年作出[25]

位于福建三明附近的中国首个商业规模800兆瓦的快中子反应堆将是SFR。2009年签署了一项协议,该协议将要求俄罗斯的 BN-800反应堆设计完成后出售给中国,这将是商业规模的快中子反应堆首次出口。[33]BN-800反应堆于2014年开始运行。

许多第四代SFR的前身存在于世界各地,华盛顿州汉福德区的400 MWe 快速通量测试设施成功运行了十年。

20 MWe EBR II在爱达荷州国家实验室成功运行了30多年,直到1994年被关闭。

通用日立公司的PRISM反应器是1984年至1994年由阿尔贡国家实验室(Argonne National Laboratory)开发的一体化快堆(IFR)技术的现代化和商业化应用。PRISM的主要目的是将重点放在燃烧其他反应堆的乏燃料上,而不是再生新的燃料。作为掩埋乏燃料/废物的替代方案,该设计减少了乏燃料中可裂变元素的半衰期,同时主要作为副产品发电。
\begin{figure}[ht]
\centering
\includegraphics[width=8cm]{./figures/9ab77f4d64e7ae65.png}
\caption{《国际财务报告准则》概念(黑白文本更清晰)} \label{fig_FYD_6}
\end{figure}

\textbf{铅冷快中子堆 (LFR)}

\begin{figure}[ht]
\centering
\includegraphics[width=8cm]{./figures/fb07444604360a41.png}
\caption{铅冷却快堆(LFR)} \label{fig_FYD_8}
\end{figure}
\textbf{铅冷却快堆}[18]具有快中子能谱,铅或铅 / 铋 低共熔的 ( LBE )液态金属冷却反应器,具有封闭的燃料循环。选项包括一系列工厂等级,包括50至150的“电池” 兆瓦的电力,具有很长的换料间隔,一个模块化系统额定在300到400兆瓦,一个大型工厂选择1200兆瓦(术语电池指长寿命、工厂制造的芯,而不是任何电化学能量转换的规定)。这种燃料是金属或氮基的,含有丰富的铀和超铀酸盐。LFR采用自然对流冷却,反应堆出口冷却剂温度为550℃,先进材料的冷却温度可能高达800℃。较高的温度使热化学法生产氢成为可能。

欧洲可持续核工业计划正在资助三个第四代反应堆系统,其中一个是铅冷快堆,也是一个的亚临界反应堆,称为MYRRHA100兆瓦(吨),将在比利时建造。预计2014年后开始建造,工业规模版本称为阿尔佛雷德,计划在2017年后建造。2009年3月,密耳拉的一个叫做Guinevere低功耗模型在 Mol 启动。[25]2012年,研究小组报告称,Guinevere已经投入运营。[34]

另外两个正在开发的铅冷却快堆是SVBR-100,这是由设计局设计的模块化100兆瓦铅铋冷却快中子反应堆概念Gidropress在俄罗斯和欧洲BREST-OD-300(Lead-cooled fast reactor) 300兆瓦,将在SVBR-100之后开发,计划于2016-2020年建造,它将去除围绕堆芯的再生覆盖层以取代钠冷却BN-600反应堆,旨在增强防核扩散能力。[24]

\subsection{优点和缺点}
相对于目前的核电厂技术,第四代反应堆的好处包括:
\begin{itemize}
\item 核废料仅在在几个世纪而不是几千年内保持放射性[35]
\item 等量核燃料产生的能量是原有的100-300倍[36]
\item 范围更广的燃料,甚至是未封装的原始燃料(非卵石型 MSR , LFTR )。
\item 在一些反应堆中,在发电过程中消耗现有核废料的能力,即封闭的核燃料循环。这加强了将核能视为可再生能源的论点。
\item 改进的操作安全特性,例如(取决于设计)避免加压操作、固有安全(无动力、非指令)反应堆停堆、避免水冷以及相关的失水(泄漏或沸腾)、制氢/爆炸和冷却水污染风险。
\end{itemize}
核反应堆在运行过程中不排放二氧化碳,尽管与所有低碳能源一样,如果在建设过程中使用非碳中性的能源(如化石燃料)或二氧化碳排放水泥,采矿和建设阶段也会导致二氧化碳排放2012年耶鲁大学发表在《工业生态学杂志》上的一篇评论文章分析了核能排放的二氧化碳生命周期评估(LCA),

文献表明,核电的生命周期温室气体(温室气体)排放量仅为传统化石能源的一小部分,可与可再生技术相媲美。

虽然本文主要处理的是来自第二代反应堆的数据,并没有分析到2050年正在建设的第三代反应堆的二氧化碳排放量,但是总结了发展中反应堆技术的生命周期评估结果。

LCA文献中对FBRs[快中子增殖反应堆]进行了评估。评估这一潜在未来技术的文献报告了温室气体排放的生命周期中值……与LWRs(第二代轻水反应堆)类似或更低,声称消耗很少或根本不消耗铀矿。

钠冷快堆的一个特定风险与使用金属钠作为冷却剂有关。一旦发生泄漏,钠就会与水发生爆炸反应。修复漏洞也可能被证明是危险的,因此最便宜的惰性气体氩气也用于防止钠氧化。氩气和氦气一样,可以取代空气中的氧气,并可能引起缺氧问题,因此工人们可能会面临这种额外的风险。这是一个相关的问题,如在日本津浦文殊的循环类型事件 ——原型快中子增殖反应堆。[37]使用铅或熔盐可以降低冷却剂的反应性,并在发生泄漏时允许较高的冷却温度和较低的压力,从而缓解这一问题。与钠相比,铅的缺点是粘度高、密度高、热容低以及放射性中子活化产物更多。

在许多情况下,人们已经积累了大量经验,对第四代设计概念进行了大量证明。例如,Fort St. Vrain电站和HTR-10反应堆与拟议的第四代VHTR设计相似,而池式EBR-II、Phenix、BN-600和BN-800反应堆与正在设计的第四代钠冷快堆相似。

然而,核工程师大卫·洛赫鲍姆认为,由于反应堆操作人员对新设计缺乏经验,最初的安全风险可能会更大。人类也会犯错”。[38]正如一位美国研究实验室主任所说,“新反应堆的制造、建造、运行和维护将面临一条陡峭的学习曲线:先进技术将具有更高的事故和错误风险。技术可能会被证明,但人不会”。[38]

\subsection{设计表}
\begin{table}[ht]
\centering
\caption{第四代反应堆设计总结 [39]}\label{FYD}
\begin{tabular}{|c|c|c|c|c|c|c}
\hline
\textbf{系统} & \textbf{中子谱} & \textbf{冷冻剂} & \textbf{温度(摄氏度)} & \textbf{燃料循环} & \textbf{尺寸(兆瓦)} & \textbf{示例开发者}\\
\hline
VHTR & 热中子 & 氦 & 900–1000 & 开式 & 250–300 & JAEA ( HTTR )、清华大学 ( HTR-10 )、X-energy [40]\\
\hline
瑞士法郎 & 快终止 & 钠 & 550 & 闭式	30–150,300–1500,1000–2000 & 泰拉能源(TWR),东芝(4S),通用电气日立核能公司(棱镜),OKBM Afrikantov(BN-1200)
\hline
SCWR & 热中子或快中子 & 水 & 510–625 & 开式或闭式 & 300–700,1000–1500 & \\
\hline
GFR & 快中子 & 氦 & 850 & 闭式 & 1200 & Energy Multiplier Module\\
\hline
LFR & 快终中子 & 铅 & 480–800 & 闭式 & 20–180,300–1200,600–1000 & \\
\hline
MSR	快中子或热中子	氟化物或氯化物盐	700–800	闭式	250,1000	seabor Technologies 、泰拉能源、极乐空间工业、 Moltex Energy 、氟锂铍能源( LFTR )、跨原子能源、钍技术解决方案(富士MSR )、陆地能源 ( IMSR )、美国南方电力公司[40]