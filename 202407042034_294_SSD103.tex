% 首都师范大学 2003 年硕士入学物理考试试题(一)
% keys 首都师范大学|考研|物理|2003年
% license Copy
% type Tutor
\begin{enumerate}
\item 傲微幅振动的单摆悬吊在电梯的天花板上。求电梯加速上升和匀速上升两种情况下的单摆振动周期差。
\item 小球自一斜面上运动到底部,下落高度为h。分两种情况考虑:\\
(1)如果将小球视为不受摩擦力的质点;\\
(2)将小球视为无滑滚动的刚体,求在这两种情况下的小球质心位于底部时的速度差。
\item 孤立系中有两个质量任意的质点,它们的初始速度不在一直线上,它们相互间的作用力满足万有引力定律。说明系统满足什么守恒定律,以及两个质点的运动情况。
\item 一均匀带电细棒弯成半径为R的半园环,棒上总电量为+Q。求:圆心o处的电场强
度 $\vec{E}$ 和电位U。\begin{figure}[ht]
\centering
\includegraphics[width=8cm]{./figures/df40f4f0232383a7.png}
\caption{} \label{fig_SSD103_1}
\end{figure}
\item 一个正点电荷Q放在一个内半径为$R_1$,外半径为$R_2$的电介质球壳中心(如图所示),电介质的相对介电常数为$\varepsilon_r$,求:\\
(1)$r<R_1,R_1<r<R_2,r<R_2$,各空间的电位移矢量$\vec D$、电场强度矢量$\vec{E}$、极化强度矢量 $\vec P$ 分布。\\
(2)$r<R_1$空间的点位分布及 $r=R_2$ 面上的极化电荷密度 $\sigma$'。\\
(3)储存在 $R_1<r<R_2$ ,空间的静电场能量。
\begin{figure}[ht]
\centering
\includegraphics[width=8cm]{./figures/39280c599d24c023.png}
\caption{} \label{fig_SSD103_2}
\end{figure}
\item 一无限长载流导线弯成图示形状,电流为I。$\frac{2}{3}$圆弧的半径为 $R$,圆心在o点。求:该载流导线在圆心处产生的磁感应强度 $\vec B$。
\i
\end{enumerate}
