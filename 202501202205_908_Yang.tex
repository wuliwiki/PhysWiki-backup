% 杨-米尔斯理论(综述)
% license CCBYSA3
% type Wiki

本文根据 CC-BY-SA 协议转载翻译自维基百科\href{https://en.wikipedia.org/wiki/Yang\%E2\%80\%93Mills_theory}{相关文章}。

杨-米尔斯理论是由杨振宁和罗伯特·米尔斯于1953年提出的一种量子场论,用于描述核结合力,也广泛用于描述类似的理论。杨-米尔斯理论是一种基于特殊酉群 SU(n) 或更一般的紧李群的规范理论。杨-米尔斯理论试图利用这些非阿贝尔李群来描述基本粒子的行为,是电磁力和弱力的统一(即 U(1) × SU(2))以及量子色动力学(强力理论,基于 SU(3))的核心。因此,它构成了粒子物理学标准模型理解的基础。
\subsection{历史与定性描述}  
\subsubsection{电动力学中的规范理论}  
所有已知的基本相互作用都可以通过规范理论来描述,但这一点的确立花费了几十年的时间。[2] 赫尔曼·外尔的开创性工作始于1915年,当时他的同事艾米·诺特证明了每一个守恒的物理量都有一个匹配的对称性,并最终在1928年出版了他的著作,将对称性几何理论(群论)应用于量子力学。[3]: 194  外尔将诺特定理中相关的对称性命名为“规范对称性”,类比于铁路轨距中的距离标准化。

1922年,厄尔温·薛定谔在工作于薛定谔方程之前的三年,连接了外尔的群概念与电子电荷。薛定谔展示了群 U(1) 会在电磁场中产生一个相位变化 \( e^{i\theta} \),该相位变化与电荷守恒相匹配。[3]: 198  随着量子电动力学在1930年代和1940年代的发展,U(1) 群变换在其中起到了核心作用。许多物理学家认为,必定存在一种与核子动力学类似的理论,特别是杨振宁对这种可能性十分痴迷。
\subsubsection{杨和米尔斯发现了核力规范理论}
杨的核心思想是寻找一个在核物理学中与电荷相似的守恒量,并利用它来发展一个与电动力学相对应的规范理论。他选择了等向量自旋的守恒,这是一种量子数,用来区分中子和质子,但他在理论上未能取得进展。[3]: 200  1953年夏天,杨在普林斯顿休假时,遇到了一位可以提供帮助的合作者:罗伯特·米尔斯。正如米尔斯自己所描述:

“在1953–1954学年,杨是布鲁克海文国家实验室的访问学者...我也在布鲁克海文...并且被分配到和杨同一个办公室。杨曾多次表现出他对刚开始职业生涯的物理学家的慷慨,他告诉我关于普适化规范不变性的想法,我们讨论了很长时间...我能够为讨论做出一些贡献,特别是在量子化过程方面,并且在推导形式上做了一些小的贡献;然而,关键的想法是杨的。”[4]

1953年夏天,杨和米尔斯将阿贝尔群(例如量子电动力学)的规范理论概念扩展到非阿贝尔群,选择了SU(2)群来解释涉及强相互作用的碰撞中等向量自旋的守恒。杨在1954年2月在普林斯顿的工作报告遭到了保利的挑战,保利询问了基于规范不变性概念所发展的场中质量问题。[3]: 202  保利知道这是一个问题,因为他曾研究过应用规范不变性,尽管他选择没有发表,认为该理论中的无质量激发是“非物理的‘阴影粒子’”。[2]: 13  杨和米尔斯于1954年10月发表了论文;在论文的最后,他们承认:

我们接下来要讨论的是b量子的质量问题,但我们没有找到一个令人满意的答案。[5]

这一无质量激发的非物理问题阻碍了进一步的进展。[3]

这个想法被搁置直到1960年,当时在无质量理论中通过对称破缺使粒子获得质量的概念被提出,最初是由杰弗里·戈德斯通、南部阳一郎和乔凡尼·乔纳-拉西尼奥提出的。这一概念促使了杨-米尔斯理论研究的重大重启,并成功地在电弱统一和量子色动力学(QCD)的公式化中取得了进展。电弱相互作用由规范群SU(2) × U(1)描述,而QCD则是SU(3)杨-米尔斯理论。电弱SU(2) × U(1)的无质量规范玻色子在自发对称破缺后混合,产生了三个弱相互作用的质量玻色子(W+、W−和Z0)以及仍然是无质量的光子场。光子场的动力学及其与物质的相互作用,反过来又由量子电动力学的U(1)规范理论支配。标准模型通过对称群SU(3) × SU(2) × U(1)将强相互作用与统一的电弱相互作用(将弱相互作用与电磁相互作用统一)结合起来。在当前的时代,强相互作用与电弱相互作用并未统一,但从耦合常数的跑动观察来看,人们相信[citation needed]它们在极高能量下会收敛到一个单一值。

在量子色动力学的低能现象学尚未完全理解,这主要是因为强耦合理论的管理困难。这可能是为什么束缚问题尚未被理论证明的原因,尽管它是一个一致的实验观察结果。这也表明,QCD低能束缚是一个具有重大相关性的数学问题,因此杨-米尔斯存在性与质量间隙问题成为了千年奖难题。
\subsubsection{在非阿贝尔规范理论的平行工作}
1953年,在一封私人信件中,沃尔夫冈·泡利提出了一个六维的爱因斯坦场方程理论,扩展了卡鲁扎(Kaluza)、克莱因(Klein)、福克(Fock)等人提出的五维理论,涉及到一个更高维度的内部空间。[6]然而,没有证据表明泡利发展了规范场的拉格朗日量或其量子化。由于泡利发现他的理论“导致了一些相当不真实的影像粒子”,他决定不正式发表他的结果。[6]尽管泡利没有发表他的六维理论,但他在1953年11月在苏黎世进行了两次关于该理论的研讨会讲座。[6]

1954年1月,剑桥大学的研究生罗纳德·肖(Ronald Shaw)也为核力发展了一个非阿贝尔规范理论。[7]然而,为了保持规范不变性,该理论需要无质量的粒子。由于当时没有已知的无质量粒子,肖和他的导师阿卜杜斯·萨拉姆(Abdus Salam)决定不公开他们的工作。[7]在杨和米尔斯于1954年10月发表论文后不久,萨拉姆鼓励肖发表他的工作以标记他的贡献。肖拒绝了,而是将其作为他1956年发布的博士论文的一章。[8][9]


杨–米尔斯理论是具有非阿贝尔对称群的规范理论的特殊例子,其拉格朗日量给出。
\begin{figure}[ht]
\centering
\includegraphics[width=10cm]{./figures/08c066f4b1d9312e.png}
\caption{} \label{fig_Yang_1}
\end{figure}