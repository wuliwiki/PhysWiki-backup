% 树形动态规划
% 树|动态规划|dp

树形 $\texttt{dp}$ 顾名思义就是在树上做 $\texttt{dp}$.通常是从根节点开始遍历整棵树,在\textbf{回溯}的时候\textbf{从子节点往上更新父节点}的信息.对于特殊的节点,如根节点或叶子节点需要进行特殊的处理.因为树的遍历需要用到递归,所以树形 $\texttt{dp}$ 一般是根据递归实现的.所以树形 $\texttt{dp}$ 较为抽象,可以画图理解.

树形 $\texttt{dp}$ 的框架:

\begin{lstlisting}[language=cpp]
void dfs(int u, int father) // father 是 u 节点的父节点
{
    for (int i = h[u]; ~i; i = ne[i])   // 遍历每条边
    {
        int j = e[i];
        if (j == father) continue;  // 如果是双向边需要特判,不往回重复搜索
        dfs(j); // 递归搜索
        f[j] <-- f[u] // 回溯的时候,用子节点更新父节点
    }
}
\end{lstlisting}

\href{https://www.luogu.com.cn/problem/P1352}{例题 $1$:没有上司的舞会}

简明题意:有 $n$ 节点构成一课树,每个节点有一个值 $w_i$,要求整棵树的权值最大值,如果一个节点的父节点加进了答案,那么这个节点就不能加进答案.

这是一个树的模型,因此可以通过树形 $\texttt{dp}$ 来求解.

\begin{itemize}
\item 状态表示:\\
    (1):$f(u, 0)$:表示从 $u$ 的子树中选,且\textbf{不选} $u$ 的子节点权值最大值.\\
    (2):$f(u, 1)$:表示从 $u$ 的子树中选,且\textbf{选择} $u$ 的子节点权值最大值.

\end{itemize}

\begin{itemize}
\item 状态计算:\\
    (1):$f(u, 0) = \sum_{s_i \in \text{son}(u)} \max(f(s_i, 0), f(s_i, 1))$ \\
    (2):$f(u, 1) = \sum_{s_i \in \text{son}(u)} f(s_i, 0)$
\end{itemize}

因为没选 $u$ 这个节点,那么子节点可选可不选,因此求两者的最大值,如果选择了 $u$ 这个节点,那么子节点一定不能选.

每次 $\texttt{dfs}$ 的时候初始化每个节点的 $f(u, 1) = w_u$,从下往上递归计算,递归结束的时候,根节点就是答案.可见树形 $\texttt{dp}$ 的状态转移方程不止 $1$ 个,通常需要分类讨论.

\textbf{时间复杂度:}$\mathcal{O}(n)$.

\textbf{C++ 代码:}

\begin{lstlisting}[language=cpp]
void dfs(int u)
{
    f[u][1] = w[u];    // 初始化每个节点选择自己的权值
    for (int i = h[u]; ~i; i = ne[i]) // 遍历 u 的每条边
    {
        int j = e[i];
        dfs(j);
        
        f[u][1] += f[j][0];  // 选择 u
        f[u][0] += max(f[j][1], f[j][0]); // 不选 u
    }

    return; // 回溯
}

\end{lstlisting}

