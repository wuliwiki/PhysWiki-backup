% 线性泛函的几何意义
% keys 线性泛函|几何意义|超平面
% license Usr
% type Tutor

\pentry{余维数\nref{nod_Codim},泛函与线性泛函\nref{nod_Funal}}{nod_d70e}
在有限维线性空间中,一个线性方程和一个超曲面一一对应(\enref{线性方程组的仿射解释}{AS2LF}),这在无穷维的线性空间中仍然成立。具体的,在任一的线性空间中,一个非平凡线性泛函(即不恒为零)和一个不通过坐标原点的超曲面一一对应。这便是线性泛函的几何意义。

本文将始终假定 $\mathbb F$ 是定义线性空间 $L$ 的域。

\subsection{零子空间}
\begin{definition}{零子空间,核}
设 $f$ 是线性空间 $L$ 上不恒为零的线性泛函。则
\begin{equation}
\{x|f(x)=0,x\in L\}~,
\end{equation}
称为 $L$ 的(关于 $f$) 的\textbf{零子空间}或线性泛函 $f$ 的\textbf{核},记作 $\ker f$。
\end{definition}

称 $\ker f$ 为“子空间” 是因为若 $f(x)=f(y)=0$,则 
\begin{equation}
f(\alpha x+\beta y)=\alpha f(x)+\beta f(y)=0.~
\end{equation}


\begin{lemma}{}
设 $f$ 是线性空间 $L$ 的非平凡线性泛函,则任一 $x\in L$,对固定的满足 $f(x_0)\neq0$ 的 $x_0$,存为唯一 $\alpha\in\mathbb F,y\in \ker f$,使得
\begin{equation}
x=\alpha x_0+y.~
\end{equation}
\end{lemma}

\textbf{证明:}\textbf{存在性:}任一 $x\in L$,令 $y=x-f(x)\frac{x_0}{f(x_0)}$,则 $f(y)=0$,即 $y\in\ker f$。因此任一 $x\in L$,可

\textbf{唯一性:}

\textbf{证毕!}



\begin{theorem}{ $\mathrm{codim} f=1$ }
零子空间 $\ker f$ 的\enref{余维数}{Codim}为 $1$。
\end{theorem}

\textbf{证明:}需要证明商空间 $L/\ker f$ 的维数为1,即存在非零矢量 $[x_0]\neq [0]\in L/\ker f$,使得任一 $[x]\in L/\ker f$,都有 $[x]=\alpha[x_0],\alpha\in\mathbb F$。或者对每一 $[x]$,都有 $y\in[x]$,使得 $y=\alpha x_0$。  证明如下

由于 $f$ 不恒为0,因此存在 $x_0$,使得 $f(x_0)\neq0$。不失一般性,设 $f(x_0)=1$(否则用 $\frac{x_0}{f(x_0)}$ 替代 $x_0$)。


\textbf{证毕!}







\textbf{证明:}



\textbf{证毕!}







