% 刘维尔定理(综述)
% license CCBYSA3
% type Wiki

本文根据 CC-BY-SA 协议转载翻译自维基百科\href{https://en.wikipedia.org/wiki/Liouville\%27s_theorem_(Hamiltonian)}{相关文章}。

在物理学中,李维尔定理(Liouville's theorem)以法国数学家约瑟夫·李维尔(Joseph Liouville)命名,是经典统计力学和哈密顿力学中的一个关键定理。该定理断言,系统的相空间分布函数在系统的轨迹上是恒定的——即在相空间中,围绕给定系统点的系统点的密度随着时间变化保持不变。这种时间不变的密度在统计力学中称为经典的先验概率。[1]

李维尔定理适用于守恒系统,即在没有摩擦效应或摩擦效应可以忽略的系统。对于这种系统的普遍数学表述是保持度量不变的动力系统。当系统具有可以解释为位置和动量的自由度时,李维尔定理适用;并非所有保持度量不变的动力系统都有这些自由度,但哈密顿系统有。在共轭位置和动量坐标的数学框架中,李维尔定理在辛几何的数学设置下得到了描述。李维尔定理忽略了化学反应的可能性,在化学反应中,粒子的总数可能随时间变化,或能量可能转移到内部自由度。李维尔定理有扩展形式,可以涵盖这些广义的情形,包括随机系统。[2]


\begin{figure}[ht]
\centering
\includegraphics[width=6cm]{./figures/773e5ba82edae5e7.png}
\caption{经典系统集群在相空间中的演化(上图)。每个系统由一个质量粒子组成,处于一维势阱中(红色曲线,下图)。虽然单个系统成员的运动由哈密顿方程给出,李维尔方程则描述了整个分布的流动。该运动类似于染料在不可压缩流体中的流动。} \label{fig_LWR_1}
\end{figure}
李维尔方程描述了相空间分布函数的时间演化。尽管该方程通常被称为“李维尔方程”,但约西亚·威拉德·吉布斯(Josiah Willard Gibbs)是第一个认识到这个方程作为统计力学基本方程重要性的人。[3][4] 它之所以被称为李维尔方程,是因为其对非正则系统的推导使用了李维尔在1838年首次推导出的一个恒等式。[5][6] 考虑一个哈密顿动力学系统,具有正则坐标 \( q_i \) 和共轭动量 \( p_i \),其中 \( i = 1, \dots , n \)。那么,相空间分布函数 \( \rho(p,q) \) 决定了系统出现在微小相空间体积 \( \mathrm{d}^n q \, \mathrm{d}^n p \) 中的概率 \( \rho(p,q) \, \mathrm{d}^n q \, \mathrm{d}^n p \)。

李维尔方程描述了 \( \rho(p,q;t) \) 在时间 \( t \) 中的演化:
\[
\frac{d\rho}{dt} = \frac{\partial \rho}{\partial t} + \sum_{i=1}^{n} \left( \frac{\partial \rho}{\partial q_i} \dot{q}_i + \frac{\partial \rho}{\partial p_i} \dot{p}_i \right) = 0.~
\]
时间导数用点表示,并根据系统的哈密顿方程进行求值。这个方程表明相空间中的密度守恒(这是吉布斯对该定理的称呼)。李维尔定理表明: