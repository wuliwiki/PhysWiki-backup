% 正交空间与辛空间
% license Usr
% type Tutor
\begin{definition}{}
设$V$是域$\mathbb F$上的线性空间,$f$是$V$上的双线性函数,
\begin{itemize}
\item 若$f$是\textbf{对称的},称$(V,f)$是\textbf{正交空间};
\item 若$f$是\textbf{反对称的},称$(V,f)$是\textbf{辛空间};
\end{itemize}
$f$则为\textbf{广义内积},$(V,f)$为广义内积空间。对任意$\bvec x,\bvec y\in V$,简记广义内积为$(\bvec x,\bvec y)\equiv f(\bvec x,\bvec y)$
\end{definition}
引入广义内积的概念后,向量关系不再是直观的几何关系。在欧几里得空间下,由于内积是正定对称双线性型,我们可以定义两个点之间的距离,向量长度和角度。然而在广义内积下,这些概念无法再定义。譬如一般定义长度$l_{\bvec x}=\sqrt{\bvec x^2}\ge 0$,但在辛空间内向量内积都为0,不再具有区分性。

因此,两向量正交不再由角度定义,而是采取内积为$0$的定义,因此正交关系具有对称。在平面几何里,两向量正交则线性无关。但在广义内积空间内并不一定成立,比如退化的正交空间内总有与自身内积为0的向量。

\begin{definition}{}
设$V$是域$\mathbb F$的线性空间,$S$是其子空间。
\begin{itemize}
\item 设$\bvec x,\bvec y\in V$,若
\end{itemize}
\end{definition}
