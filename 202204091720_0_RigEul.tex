% 刚体定点旋转的运动方程(欧拉角)

\begin{issues}
\issueDraft
\end{issues}

\pentry{刚体的运动方程\upref{RBEqM}, 欧拉角\upref{EulerA}}

首先摆放刚体, 使得三个主轴分别与 $x,y,z$ 轴重合. 然后使用 $z$-$y$-$z$ 欧拉角描述刚体每个时刻的位置. 若已知角速度, 欧拉角的时间导数为(\autoref{EulerA_eq2}~\upref{EulerA})
\begin{equation}\label{RigEul_eq1}
\dot\psi = \omega_r + \omega_\theta \cot\theta,\qquad
\dot\theta = \omega_\phi,\qquad
\dot\phi = -\frac{\omega_\theta}{\sin\theta}
\end{equation}
球坐标中三个方向构建的坐标系与刚体三个主轴 $\uvec e_1, \uvec e_2, \uvec e_3$ ($\uvec e_1 = \uvec r$)所构建的体坐标系之间还需要经过绕 $\uvec r$ 轴的 $\psi$ 角旋转变换.
\begin{equation}
\pmat{\omega_1\\\omega_2\\\omega_3} = \pmat{1 & 0 & 0\\ 0 & C_\psi & S_\psi\\ 0 & -S_\psi & C_\psi}\pmat{\omega_r\\\omega_\theta\\\omega_\phi}
\end{equation}
在体坐标系中使用(\autoref{RBEqM_eq6}~\upref{RBEqM})得
\begin{equation}\label{RigEul_eq2}
\dot{\bvec \omega} = \mat I_0^{-1} (\bvec \tau  - \bvec\omega\cross \mat I_0 \bvec\omega)
\end{equation}
其中
\begin{equation}
\bvec\omega\cross \mat I_0 \bvec\omega = (I_3-I_2)\omega_3\omega_2\uvec r + (I_1-I_3)\omega_1\omega_3\uvec e_2 + (I_2-I_1)\omega_1\omega_2\uvec e_3
\end{equation}

这样, \autoref{RigEul_eq1} 和\autoref{RigEul_eq2} 就组成了一个六元一阶常微分方程组.
