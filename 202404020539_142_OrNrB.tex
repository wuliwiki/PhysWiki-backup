% 正交归一基底
% keys 线性代数|向量|克罗内克 delta 函数
% license Xiao
% type Tutor

\begin{issues}
\issueTODO
\end{issues}

\pentry{基底(线性代数)\nref{nod_VecSpn},内积、内积空间\nref{nod_InerPd}}{nod_f662}

\addTODO{目前写的是有限维度版本,需要调整成一般的内积空间}

\subsection{正交归一基底}

我们已经知道了向量基底的概念, 如果内积空间中一组向量基底中的每个向量模长都为 $1$ 且每两个向量都正交, 则我们把这组基底称为\textbf{正交归一基底(orthonormal bases)}, 也叫\textbf{单位正交基}。 若把这组正交归一基底记为 $\uvec x_1,\uvec x_2\dots\uvec x_N$, 则正交归一可以用内积表示为
\begin{equation}\label{eq_OrNrB_3}
\langle \uvec x_i, \uvec x_j \rangle = \delta_{ij} ~,\qquad (i,j = 1,\dots, N)~.
\end{equation}
其中 $\delta_{ij}$ 是克罗内克 delta 函数\upref{Kronec}。 简单来说就是任意两个不同的基底点乘为零, 以及任意基底与自身的点乘为 1。

单位正交基是某个空间的基底\upref{Gvec2}, 所以该空间中任意向量 $\bvec v$ 在单位正交基上的展开
\begin{equation}\label{eq_OrNrB_1}
\bvec v = \sum_{i = 1}^N c_i \,\uvec x_i ~,\qquad (i = 1,\dots, N)~.
\end{equation}
其中 $c_i$ 就是 $\bvec v$ 的各个坐标。 由正交归一性可以证明(见下文)
\begin{equation}\label{eq_OrNrB_4}
c_i = \langle \bvec v, \uvec x_i \rangle ~,\qquad (i = 1,\dots, N)~.
\end{equation}
或者把\autoref{eq_OrNrB_1} 记为
\begin{equation}\label{eq_OrNrB_5}
\bvec v = \sum_{i = 1}^N (\langle \bvec v, \uvec x_i \rangle) \,\uvec x_i~, \qquad (i = 1,\dots, N)~.
\end{equation}

最常见的例子就是几何向量在直角坐标系的 $\uvec x, \uvec y, \uvec z$ 三个单位正交向量上的展开:
\begin{equation}
\bvec v = (\langle \bvec v, \uvec x \rangle)\,\uvec x + (\langle \bvec v, \uvec y \rangle)\,\uvec y + (\langle \bvec v, \uvec z \rangle)\,\uvec z = x \,\uvec x + y \,\uvec y + z \,\uvec z~.
\end{equation} 

\subsubsection{证明}
我们来证明\autoref{eq_OrNrB_4} : 用 $\uvec x_k$ 同时乘以\autoref{eq_OrNrB_1} 两边,得
\begin{equation}\label{eq_OrNrB_2}
\langle \bvec v, \uvec x_k \rangle = \sum_{i = 1}^N  c_i \langle \uvec x_i, \uvec x_k \rangle = \sum_{i = 1}^N c_i \delta_{ik}  = c_k ~,\qquad (k = 1, \dots, N)~.
\end{equation}
最后一步使用了\autoref{eq_Kronec_1}~\upref{Kronec}。 证毕。

\autoref{eq_OrNrB_2} 的过程可以看作是用 “点乘 $\uvec x_k$” 的操作把\autoref{eq_OrNrB_2} 求和中需要的一项筛选出来。 注意只有使用正交归一基底才可以进行这样的筛选。 一个反例是斜坐标系(\autoref{ex_Gvec2_1}~\upref{Gvec2})不能使用\autoref{eq_OrNrB_1} 计算坐标, 因为它的基底不正交。


\subsection{施密特正交归一化}\label{sub_OrNrB_1}

\addTODO{目前写的是有限维度版本,需要增加无限维度的情况}

若在 $M$ 维向量空间中任意给出 $N \leqslant M$ 个线性无关的向量, 如何得到一组正交归一化的基底呢? 我们可以用\textbf{施密特正交归一化(Schmidt orthonormalization)}。 先看一个二维的例子

\begin{example}{二维空间中的几何向量}
已知两个几何向量 $\bvec v_1, \bvec v_2$ 坐标分别为 $(2, 1)$, $(1, 2)$。 这两个向量不共线, 说明它们线性无关。 但容易看出它们既不归一也不正交, 下面来进行施密特正交归一化。

先把 $\bvec v_1$ 归一化, 并记为 $\bvec u_1$
\begin{equation}
\bvec u_1 = \frac{\bvec v_1}{\norm{\bvec v_1}} = \frac{1}{\sqrt{5}} (2, 1)~.
\end{equation}
然后, 用内积来计算 $\bvec v_2$ 在 $\bvec v_1$ 方向的投影长度
\begin{equation}
\langle \bvec v_2, \bvec u_1 \rangle = \frac{4}{\sqrt{5}}~,
\end{equation}
所以 $\bvec v_2$ 在平行于 $\bvec v_1$ 方向的分量为
\begin{equation}
\bvec v_2^\parallel =  (\langle \bvec v_2, \bvec u_1 \rangle)\bvec u_1 = \frac{4}{5} (2, 1)~.
\end{equation}
将 $\bvec v_2$ 减去和 $\bvec v_1$ 平行的分量, 就是和 $\bvec v_1$ 垂直的分量
\begin{equation}
\bvec v_2^\bot = \bvec v_2 - \bvec v_2^\parallel = \qty(-\frac35, \frac65)~,
\end{equation}
归一化并记为
\begin{equation}
\bvec u_2 = \frac{\bvec v_2^\bot}{\norm{\bvec v_2^\bot}} = \frac{1}{3\sqrt{5}} (-3, 6)~.
\end{equation}
现在可以验证, 基底 $\bvec u_1$ 和 $\bvec u_2$ 是正交归一的, 即 $\norm{\bvec u_1} = \norm{\bvec u_2} = 1$, 且 $\langle \bvec u_1, \bvec u_2 \rangle = 0$。
\end{example}

若给出 $M$ 维向量空间中的 $N$ 个($N \leqslant M$)线性无关向量
\begin{enumerate}
\item 将第 1 个向量归一化得到第 1 个基底
\item 将第 2 个向量分解为与第 1 个向量平行和垂直的两个分量, 并将垂直分量归一化得到第 2 个基底
\item 将第 3 个向量分解为三个部分, 即分别平行于前两个基底的分量和一个垂直分量, 并将垂直分量归一化得到第 3 个基底
\item 对第 $n = 4, \dots , N$ 个基底重复该步骤, 得到第 $n$ 个基底
\end{enumerate}

用公式来表示这个过程, 就是:
\begin{equation}
\bvec v_1^\bot = \bvec v_1~.
\end{equation}
\begin{equation}
\bvec u_i = \frac{\bvec v_i^\bot}{\norm{\bvec v_i^\bot}} \qquad (i = 1, \dots ,N)~,
\end{equation}
\begin{equation}\label{eq_OrNrB_6}
\bvec v_i^\parallel = \sum _{j=1}^{i-1} (\langle \bvec v_i, \bvec u_j \rangle) \bvec u_j \qquad (i = 2, \dots N)~,
\end{equation}
\begin{equation}\label{eq_OrNrB_7}
\bvec v_i^\bot = \bvec v_i - \bvec v_i^\parallel \qquad (i = 2, \dots N)~.
\end{equation}

\begin{exercise}{}
对三维空间中的向量 $(2, 1, 1)$, $(1, 2, 1)$ 和 $(1, 1, 2)$ 进行施密特正交归一化。
\end{exercise}

\subsubsection{推导}
这里来解释\autoref{eq_OrNrB_6} 和\autoref{eq_OrNrB_7}。 我们假设已经知道 $i-1$ 个正交归一的向量, 由于 $N$ 维空间中必然存在 $N$ 个正交归一基底, 我们可以设剩下 $u_i$ 的也已经知道(或者可以任意取)。 于是 $v_i$ 可以用基底展开为
\begin{equation}
\bvec v_i = \sum _{j=1}^N c_j \bvec u_j~.
\end{equation}
\autoref{eq_OrNrB_6} 得到前 $i-1$ 项之和
\begin{equation}
\bvec v_i^\parallel = \sum _{j=1}^{i-1} c_j\bvec u_j~,
\end{equation}
所以\autoref{eq_OrNrB_7} 就是第 $i$ 项到第 $N$ 项之和
\begin{equation}
\bvec v_i^\bot = \sum _{j=i}^{N} c_j\bvec u_j~.
\end{equation}
所以对 $j = 1, \dots , i-1$, 都有 $\langle \bvec v_i^\bot, \bvec u_j \rangle = 0$。 也就是说和已有的 $i-1$ 个基底都正交。