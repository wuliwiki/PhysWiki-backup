% 计算机技术与软件专业技术资格(水平)考试

\begin{issues}
\issueDraft
\end{issues}

俗称软考, 是国内水平最高的计算机资格考试。

\subsection{初级}
\begin{itemize}
\item 程序员:计算机相关基础知识;基本数据结构和常用算法;C程序设计语言以及C++。Java中的一种程序设计语言。
\item 网络管理员:计算机系统、网络操作系统、数据通信的基础知识;计算机网络的相关知识;以太网的性能、特点、组网方法及简单管理;主流操作系统的安装、设置和管理方法;Web网站的建立、管理与维护方法;交换机和路由器的基本配置。
\item 信息技术处理员:信息技术的基本概念;计算机的组成、各主要部件的功能和性能指标;操作系统和文件管理的基本概念和基本操作;文字处理、电子表格、演示文稿和数据库应用的基本知识和基本操作;Internet及其常用软件的基本操作。
\item 信息系统远行管理员:计算机系统的组成及主要设备的基本性能指标;操作系统、数据库系统、计算机网络的基础知识;多媒体设备、电子办公设备的安装、配置和使用;信息处理基本操作;信息化及信息系统开发的基本知识。
\end{itemize}

\subsection{中级}
\begin{itemize}
\item 软件评测师:操作系统、数据库、中间件、程序设计语言、计算机网络基础知识;软件工程知识;软件质量及软件质量管理基础知识;软件测试标准、测试技术及方法;软件测试项目管理知识。
\item 软件设计师:计算机相关基础知识;常用数据结构和常用算法;C程序设计语言,以及C++、Java中的一种程序设计语言;软件工程、软件过程改进和软件开发项目管理的基础知识;软件设计的方法和技术。
\item 网络工程师:计算机系统、网络操作系统、数据通信的基础知识;计算机网络的相关知识,包括计算机网络体系结构的网络协议、计算机网络互联技术、网络管理的基本原理和操作方法、网络安全机制和安全协议;网络系统的性能测试和优化技术,以及可靠性设计技术;网络新技术及其发展趋势。
\item 多媒体应用设计师:多媒体计算机的系统结构;多媒体数据获取、处理及输出技术;多媒体数据压缩编码及其适用的国际标准;多媒体应用系统的创作过程,报考数字音频编辑、图形的绘制、动画和视频的制作、多媒体制作工具的使用等。
\item 嵌入式系统设计师:嵌入式系统的硬软件基础知识;嵌入式系统需求分析方法;嵌入式系统设计与开发的方法及步骤;嵌入式系统实施、运行、维护知识;软件过程改进和软件开发项目管理等软件工程基础知识;系统的安全性、可靠性、信息技术标准以及有观法律法规的基本知识。
\item 电子商务设计师:电子商务基本模式、模式创新及发展趋势;电子商务交易的一般流程;电子支付概念理想的物流技术和供应链技术;电子商务网站的运用、维护、和管理;电子商务相关的经济学和管理学基本原理、法律法规等。
\item 系统集成项目管理师:信息系统集成项目管理知识、方法和工具;系统集成项目管理工程师职业道德要求;信息化知识;信息安全知识与安全管理体系。
\item 信息系统监理师:信息系统工程师监理知识、方法和工具;信息系统工程监理师的职业道德要求;信息系统工程监理的有关政策、法律、法规、标准和规范。
\item 数据库系统工程师:数据库系统基本概念及关系理论;常用的大型数据库管理系统的应用技术;数据库应用系统的设计方法和开发过程;数据库系统的管理和维护方法。
\item 信息系统管理工程师:信息化和信息系统集成知识;信息系统开发的基础过程与方法;信息系统管理维护的知识、工具与方法。
\item 信息安全工程师:信息安全的基本知识;密码学的基础知识与应用技术;计算机安全防护与检测技术;网络安全防护与处理技术;数字水印在版权保护中的应用技术;信息安全相关的法律法规和管理规定。
\end{itemize}

\subsection{高级}
\begin{itemize}
\item 信息系统项目管理师:信息系统项目管理知识和方法;项目整体绩效评估方法;常用项目管理工具;信息系统相关法律法规、技术标准与规范。
\item 系统分析师:信息系统开发所需的综合技术知识,包括硬件、软件、网络、数据库等;信息系统开发过程和方法;信息系统开发标准;信息安全的相关知识与技术。
\item 系统架构设计师:计算机硬软件知识;信息系统开发过程和开发标准;主流的中间件和应用服务器平台;软件系统建模和系统架构设计基本技术;计算机安全技术、安全策略、安全管理知识。
\item 网络规划设计师:数据通信、计算机网络、计算机系统的基本原理;网络计算环境与网络应用;各类网络产品及其应用规范;网络安全和信息安全技术、安全产品及其应用规范;应用项目管理的方法和工具实施网络工程项目。
\item 系统规划与管理师:IT战略规划知识;信息技术服务知识;IT服务规划设计、部署实施、运营管理、持续改进、监督管理、服务营销;团队建设与管理的方法和技术;标准化相关知识。
\end{itemize}
