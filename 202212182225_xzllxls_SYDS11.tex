% 中山大学 2011 年913专业基础(数据结构)考研真题

\subsection{一、单项选择题(每题2分,共40分)}

1.算法复杂度通常是表达算法在最坏情况下所需要的计算量,$O(1)$的含义是( ) \\
(A).算法执行1步就完成 \\
(B).算法执行1秒钟就完成 \\
(C).解决执行常数步就完成 \\
(D).算法执行可变步数就完成

2.在数据结构中,按逻辑结构可把数据结构分为( ) \\
(A).静态结构和动态结构 \\
(B).线性结构和非线性结构 \\
(C).顺序结构和链式结构 \\
(D).内部结构和外部结构

3.在数据结构中,可用存储顺序代表逻辑顺序的数据结构为( ) \\
(A). Hash表 \\
(B).二叉搜索树 \\
(C).链式结构 \\
(D).顺序结构

4. 对链式存储的正确描述是( ) \\
(A).结点之间是连续存储的 \\
(B).各结点的地址由小到大 \\
(C).各结点类型可以不一致 \\
(D).结点内单元是连续存储的

5. 在下列关于“串”的陈述中,正确的说明是( ) \\
(A).串是一种特殊的线性表 \\
(B).串中元素只能是字母 \\
(C).串的长度必须大于零 \\
(D).空串就是空白串

6.关于堆栈的正确描述是( ) \\
(A). FILO \\
(B). FIFO \\
(C).只能用数组来实现 \\
(D).可以修改栈中元素的数据

7. 假设循环队列的长度为QSize. 当队列非空时,从其队列头取出数据后,其队头下标Front的变化为() \\
(A). Front = Front+ 1 \\
(B). Front = (Front + 1) \% 100 \\
(C). Front = (Front+ 1) \% QSize \\
(D). Front = Front \% Qsize + 1

8. 假设Head是带头结点单向循环链的头结点指针,判断其为空的条件是( ) \\
(A). Head.next = NULL \\
(B). Head~>next == Head \\
(C). Head->next = NULL \\
(D). Head = NULL

9.设A[)][m]为一个对称矩阵, 数组下标从[0[0)开始。 为了节省存储,将其下三角部分按行存放在一维数组B0.m-1], m=n(n+1)2, 对下三角部分中任一元素4.fi≥D, 它在一-维数组 B的下标k值是( ) \\
(A). i(i-1)/2+j \\
(B). (-1)2+j-1 \\
(C). (i+1)/2+j-1 \\
(D). i(+1)2+j

10. 假设二又树的根结点为第$0$层,那么,其第$i$层($i\geqslant0$)的结点数最多为( ) \\
(A).$2i$ \\
(B).$2^i$ \\
(C).$2^{i+1}-1$ \\
(D).$2^{i+1}$

11. 若一棵二叉树的后序和中序序列分别是dbefca和dbaef,则其先序序列是( ) \\
(A). adbefc \\
(B). abdcfe \\
(C). adbcef \\
(D). abdcef

12.用一维数组来存储满二叉树,若数组下标从0开始,则元素下标为k的右子结点下标是( )(不考虑数组下标的越界问题) \\
(A). $2k+1$ \\
(B). $2k+2$ \\
(C). $ \lfloor k/2 \rfloor $ \\
(D). $ \lceil k/2 \rceil $ \\

13. 假设LTree和RTree是二叉搜索树Tree的左右子树,H(T)表示树T的高度。若树Tree是AVL树,则( ) \\
(A). H(LTree) - H(RTree)= 0 \\
(B). HCLTree)- H(RTree) < 1 \\
(C). H(LTree) - H(RTree) <= 1 \\
(D). H(LTree) - H(RTree) <= 1

14. 对$n$个结点和$e$条边的无向图(无环),其邻接矩阵中零元素的个数为( ) \\
(A). $e$ \\
(B). $2e$  \\
(C). $n^2-e$ \\
(D). $n^2-2e$

15. 用邻接矩阵存储有n个顶点和e条边的有向图,则删除与某个顶点相邻的所有边的时间复杂度是() \\
(A). $O(n)$ \\
(B). $O(e)$ \\
(C). $O(n+e)$ \\
(D). $O(ne)$

16. 下列排序算法中,时间复杂度最差的是( ) \\
(A).选择排序 \\
(B).桶(基数)排序 \\
(C).快速排序 \\
(D).堆排序

17. 基于比较的排序算法对n个数进行排序的比较次数下界为( ) \\
(A). $O(logn)$ \\
(B). $O(m)$  \\
(C). $O(nlogn)$ \\
(D). $O(n^2)$

18. 在下列存储条件下,( )是最适合 使用折半查找算法来进行查找操作。 \\
(A).顺序存储 \\
(B).链式存储 \\
(C).散列存储 \\
(D).数据有序且顺序存储

19. 在下列算法中,求图最小生成树的算法是( ) \\
(A). DFS算法 \\
(B). KMP算法 \\
(C). Prim算法 \\
(D). Djkstra算法

20. 若结点的存储地址与其关键字之间存在某种映射关系,则称这种存储结构为( ) \\
(A).顺序存储结构 $\qquad$ (B). 链式存储结构 \\
(C).散列存储结构 $\qquad$ (D). 索引存储结构

\subsection{二、解答题(每题10分,共50分)}

1.假设有如图1所示的图 \\
(1)写出图1的邻接矩阵; \\
(2)根据邻接矩阵从顶点a出发进行宽度(或广度)优先遍历,画出相应的宽度优先遍历树(同一个结点的邻接结点按结点序号大小为序)。
\begin{figure}[ht]
\centering
\includegraphics[width=5cm]{./figures/SYDS11_1.png}
\caption{第二1题图} \label{SYDS11_fig1}
\end{figure}

2.简单描述求图最小生成树的Kruskal算法(克鲁斯卡尔算法)的基本思想,并按步骤列出图2的最小生成树的求解过程。
\begin{figure}[ht]
\centering
\includegraphics[width=8cm]{./figures/SYDS11_2.png}
\caption{第二2题图} \label{SYDS11_fig2}
\end{figure}

3.简单叙述快速排序的思想,在“第一个元素为支点”前提下按步骤列出下列序列的排序过程。 \\
待排序的数值序列: 45,12,56,87,34,78


4. 已知有下列13个元素的散列表:
\begin{figure}[ht]
\centering
\includegraphics[width=12cm]{./figures/SYDS11_3.png}
\caption{第二4题图} \label{SYDS11_fig3}
\end{figure}
其散列函数为h(key)=key\%m (m=13),处理冲突的方法为双重散列法,探查序列为: \\
$h=(h(key)+i*(key)\%m$  $\qquad$ $i=0.1.... m-1$, 其中:$h'(key)=key\%11+1$ \\
间:对表中关键字35进行查找时,所需进行的比较次数为多少?依次写出每次的计算公式和值。s.假设设在通信中,字符a, b,c,d, e,,g出现的频率如下: \\
a:20\% b: 7\% $\quad$ c:16\% $\quad$ d: 27\% e: 7\% $\quad$ f: 10\% $\quad$ g: 13% \\
(1)根据Huffman算法(赫夫曼算法)画出其赫夫曼树: \\
(2)给出每个字母所对应的赫夫曼编码,规定:结点左分支边上标0,右分支边上标I;  \\
(3)计算其加权路径的长度WPL.

\subsection{三、阅读理解题,按空白编号填写相应的C语言语句,以实现函数功能。(每空2分,每题10分,共30分)}

1.排队是日常生活中常见的一种现象,比如:在商店排队付款。当第一位顾客完成付款离开后,其他顾客依次前移。下面用数据结构中的队列来模拟这种排队现象。
\begin{lstlisting}[language=cpp]
#define QUEUE 40
struct Queue {
    int queue[QUEUE];
    int Rear;
    int Rear;    // Rear 记录队列尾
};
\end{lstlisting}
(1)初始化队列Q
\begin{lstlisting}[language=cpp]
void InitQueue(Queue *Q)
{
    Q->Rear=-1;
}
\end{lstlisting}

(2)入队操作EnQueu(Q, dat):若队列Q已满,返回0, 否则,把数据data加入队列Q,并返回1
\begin{lstlisting}[language=cpp]
int EnQueue(Queue *Q, int *data)
{
    if(__(1)__) return 0;
    __(2)__;
    Q->queue[Rear] = data;
    return 1;
}
\end{lstlisting}

(3)出队操作DeQueue(Q, dat):若队列Q为空,则返回0,否则,把队头元素存入地址参数data,然后从队列Q中去除该队头元素,并返回1.
\begin{lstlisting}[language=cpp]
int DeQueue(Queue *Q, int *data)
{
    if(Q->Rear = -1) return 0;
    *data=___ (3)___;
    for(i=0; i<Q->Rear; i++) __ (4)__;
    __(5)__;
    return 1;
}
\end{lstlisting}

2.假设有两个堆栈共享一个存储空间,其有关定义如下:
\begin{lstlisting}[language=cpp]
#define SIZE 50
struct Stacks {
    int Elements[SIZE];
    int Topl, Top2;    //Top1和Top2分别记录二个栈的栈顶
};
\end{lstlisting}
这二个堆栈在某个时刻的状态如下图所示。
\begin{figure}[ht]
\centering
\includegraphics[width=12cm]{./figures/SYDS11_6.png}
\caption{第三2题图} \label{SYDS11_fig6}
\end{figure}
(1)初始化堆栈
\begin{lstlisting}[language=cpp]
void InitStacks(Stacks *stack)
{
    stack->Topl = ____ (1)___;
    stack->Top2 = ____ (2)___;
}
\end{lstlisting}
(2)堆栈1(左堆栈)压栈操作
\begin{lstlisting}[language=cpp]
int pushl(Stacks *stack, int data)
{
    if( _ (3)_ ) return 0;
    stack->Top1++;
    Elements[stack->Top1]=data;
    return 1;
}
\end{lstlisting}
(3)堆栈2(右堆栈)出栈操作,并把栈顶元素的值赋给指针变量data所指向的存储单元
\begin{lstlisting}[language=cpp]
BOOL pop2(Stacks *stack, int *data)
{
    if(___(4)___) return 0;
    *data = Elements[stack->Top2];
    ___(5)___;
    return 1;
}
\end{lstlisting}

3.假设二叉树$T=<T_L,root,T_R>$的深度定义如下: \\
$Depth(T)=\leftgroup{&0, & T\text{是空树} \\ &1, & T\text{的根结点是叶结点} \\ & max(Depth(TL), Depth(TR)) & \text{其他}}$
\\
已知二叉树的结点定义如下:
\begin{lstlisting}[language=cpp]
struct BNode {
    int Key;
    struct BNode *LChild, *RChild;
};
\end{lstlisting}
函数Depth(root)是求以结点root为根的二叉树深度。
\begin{lstlisting}[language=cpp]
int Depth(BNode *root)
{
    int dl, d2;
    if(root==__(1)__) return 0;
    if(___(2)___) return 1;
    d1=____(3)____;
    d2=____(4)____;
    return(___(5)___?d1:d2);
}
\end{lstlisting}

\subsection{四、算法设计题(每题15分,共30分)}
用C语言或类C语言实现下面函数的功能。

1.假设用链表表示集合,集合链表的结点定义如下:
\begin{lstlisting}[language=cpp]
struct Set {
    int element;
    struct Set next
};
\end{lstlisting}
例如: A={2.1,3},B={}, 集合A和B的存储形式如下图所示。
\begin{figure}[ht]
\centering
\includegraphics[width=12cm]{./figures/SYDS11_8.png}
\caption{第四1题图} \label{SYDS11_fig8}
\end{figure}
试实现集合的下列二个操作:
(1) Set Intersection(Set *A, Set *B),其功能是返回集合A和B交集的首结点地址(10 分) \\
(2) int Cardinality(Set *A),其功能是返回集合A中的元素个数,即:求|AI (5 分) \\
例如有下列语句:
\begin{lstlisting}[language=cpp]
Set *A, *B, *C;
int NumC;
......    //集合A和B的值由其它集合操作获得
C = Intersection(A, B);    //C=A交B
NumC = Cardinality(C);     // NumC=|C|
\end{lstlisting}

2.已知二叉树的结点定义如下:
\begin{lstlisting}[language=cpp]
struct BNode {
    int Key;
    struct BNode *LChild, *RChild;
};
\end{lstlisting}
编写函数TraveralByLeve(BNode *ro0), 其功能是“按层”遍历以结点roo为根的二叉树,并输出每个结点中Key的信息。 \\
在函数描述中可直接使用下列队列功能(如果需要的话,仅供参考)
\begin{lstlisting}[language=bash]
Queue:队列类型定义符
InitQueue(Queue *Q):初始化队列Q为空队列
EnQueue(Queue *Q, BNode *node);:把指针node入队列Q
BNode *DeQueue(Queue *Q):若队列Q为空,则返回NULL,否则,返回队头元素,并从队列Q中删除该队头元索
int QueueEmpty(Queue *Q):若队列Q为空,则返回1,否则,返回0
\end{lstlisting}