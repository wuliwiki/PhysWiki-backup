% 应力

\begin{issues}
\issueDraft
\issueTODO
\end{issues}

\footnote{本文参考自P. Beer的Mechanics of Materials}

对于一块材料,我们很容易用截面法\upref{INTFRC}分析出材料某一截面上的受力情况;但是,截面法只能告诉我们整个截面的“总内力”,却不能告诉我们截面上“具体一处”的受力.事实上,在不少情况下,截面各处的受力是不一样的.

\subsection{微元体与应力}
为了更好地处理截面上一处的受力,类似于微积分\upref{IntN}中“划分小块体积”的思想,我们假定材料是由无数小正方形块组成的\footnote{这要求材料是“无限可分”的,并且每一小块还能维持物理性质不变.这当然是不“现实”\upref{MetInt}的,不过在初步的学习中,这是一个好的简化近似.},每一小块被称作“微元体 Element”.

\begin{figure}[ht]
\centering
\includegraphics[width=10cm]{./figures/STRESS_1.png}
\caption{三维微元体.仿自P. Beer的Mechanics of Materials} \label{STRESS_fig1}
\end{figure}

\subsubsection{三维情况}
如\autoref{STRESS_fig1} ,微元体的每一个面上可以受3个力(的分量),包括一个垂直于表面的力与两个平行于表面的力.定义垂直于表面的“力”为正应力$\sigma$、平行于表面的“力”为切应力$\tau$.

\begin{definition}{正应力与切应力}
正应力:$$\sigma_{ii} =\dv{F_{ii}}{A}, \dd F_{ii} = \sigma_{ii} \dd A$$

切应力:$$\tau_{ij}=\dv{F_{ij}}{A}, \dd F_{ij} = \tau_{ij} \dd A$$
\footnote{有时不在符号上区分$\sigma$与$\tau$,并统称为应力}

类似于微积分的思想,当微元体足够小时,微元体表面上不同处的受力大小也趋于一致.

i表示这个力的作用面的法方向,j表示这个力的方向\footnote{不同作者可能选取不同的约定}.
\end{definition}

看起来,一个微元体上共有$3\times6=18$个力的分量;但对微元体运用刚体的静力平衡\upref{RBSt},可以证明事实上\textbf{一个微元体上只有6个相互独立的力}.

一个微元体的受力情况可以记为一个三阶矩阵,注意这个矩阵是\textbf{对称}的:
\begin{equation}
\mat \sigma=
\begin{bmatrix}
\sigma_{xx} & \tau_{xy} & \tau_{xz} \\
\tau_{yx} & \sigma_{yy} & \tau_{yz} \\
\tau_{zx} & \tau_{zy} & \sigma_{zz} \\
\end{bmatrix}
\end{equation}
这六个力可以分别选取 $\sigma_{xx}, \sigma_{yy},\sigma_{zz}, \tau_{xy}, \tau_{xz},  \tau_{yz}$.

\subsubsection{二维情况}
\begin{figure}[ht]
\centering
\includegraphics[width=5cm]{./figures/STRESS_2.png}
\caption{二维微元体.仿自P. Beer的Mechanics of Materials} \label{STRESS_fig2}
\end{figure}

二维情况下的微元体更为简单.每一个面上只受一个正应力与一个切应力,共受$2\times4=8$个力,但只有3个相互独立的力.此时受力情况可以记为一个二阶矩阵,这个矩阵也是\textbf{对称}的:
\begin{equation}
\mat \sigma=
\begin{bmatrix}
\sigma_{xx} & \tau_{xy}\\
\tau_{yx} & \sigma_{yy}\\
\end{bmatrix}
\end{equation}
这三个力可以分别选取 $\sigma_{xx}, \sigma_{yy}, \tau_{xy}$.

\subsection{应力的宏观效果}
那微元体上的应力是如何和截面上的总内力联系起来呢?答案是总内力等于各微元体应力的累和.
%我的理解是这样的,不知有无dalao斧正
例如,宏观拉力 $F = \iint \sigma_x dA$
\begin{figure}[ht]
\centering
\includegraphics[width=10cm]{./figures/STRESS_3.png}
\caption{拉力} \label{STRESS_fig3}
\end{figure}

力偶 $F = \iint y\sigma_x dA$
\begin{figure}[ht]
\centering
\includegraphics[width=10cm]{./figures/STRESS_4.png}
\caption{力偶} \label{STRESS_fig4}
\end{figure}

此外,叠加原理依旧适用.假如某截面处的内力既包括拉力、又包括力偶,那某点处的内应力是拉力、力偶单独存在时的内应力之和:
\begin{figure}[ht]
\centering
\includegraphics[width=10cm]{./figures/STRESS_5.png}
\caption{叠加原理.仿自P. Beer的Mechanics of Materials} \label{STRESS_fig5}
\end{figure}
