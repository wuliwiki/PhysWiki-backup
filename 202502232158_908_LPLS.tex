% 皮埃尔-西蒙·拉普拉斯(综述)
% license CCBYSA3
% type Wiki

本文根据 CC-BY-SA 协议转载翻译自维基百科\href{https://en.wikipedia.org/wiki/Pierre-Simon_Laplace}{相关文章}。

\begin{figure}[ht]
\centering
\includegraphics[width=6cm]{./figures/12e7e5c72d2ec562.png}
\caption{皮埃尔-西蒙·拉普拉斯作为法兰西第一帝国参议院的参议长} \label{fig_LPLS_1}
\end{figure}
皮埃尔-西蒙·拉普拉斯(Pierre-Simon, Marquis de Laplace,1749年3月23日-1827年3月5日)是法国学者,他的工作对工程学、数学、统计学、物理学、天文学和哲学的发展具有重要意义。他在五卷本《天体力学》(Mécanique céleste)(1799–1825)中总结并扩展了前人的工作。这部著作将经典力学的几何学研究转化为基于微积分的研究,从而开辟了更广泛的研究领域。[2]拉普拉斯还推广并进一步确认了艾萨克·牛顿爵士的工作。在统计学中,贝叶斯概率解释主要是由拉普拉斯发展而来的。[3]

拉普拉斯提出了拉普拉斯方程,并开创了拉普拉斯变换,这在数学物理的许多分支中都有应用,而数学物理领域正是他主导发展的一个重要领域。广泛应用于数学中的拉普拉斯算子也以他的名字命名。他重新阐述并发展了太阳系起源的星云假说,是最早提出类似黑洞概念的科学家之一,[4] 斯蒂芬·霍金曾表示:“拉普拉斯基本上预言了黑洞的存在”。[1]他提出了拉普拉斯恶魔,这是一个假设的全能预测智慧。他还改进了牛顿关于声速的计算,得出了更精确的测量结果。[5]

拉普拉斯被认为是历史上最伟大的科学家之一。有时被称为“法国的牛顿”或“法国的牛顿”,他被描述为拥有卓越的数学天赋,超越了几乎所有同时代的人。[6] 1785年,拉普拉斯曾担任拿破仑从巴黎军事学院毕业时的考官。[7] 拉普拉斯于1806年成为帝国伯爵,并在1817年波旁王朝复辟后被封为侯爵。
\subsection{早年}
\begin{figure}[ht]
\centering
\includegraphics[width=6cm]{./figures/1abf9bfb2f2cc079.png}
\caption{皮埃尔-西蒙·拉普拉斯的肖像,作者:约翰·恩斯特·海因修斯(1775年)} \label{fig_LPLS_2}
\end{figure}
拉普拉斯生活中的一些细节不为人知,因为1925年与他的曾曾孙科尔贝尔-拉普拉斯伯爵的家族 château(位于利厄市附近的圣朱利安·德·梅约)一起被烧毁。还有一些记录早在1871年,当他的位于巴黎附近阿尔居伊的住所被掠夺时就已被销毁。[8]

拉普拉斯于1749年3月23日出生在诺曼底的博蒙-昂-奥热(Beaumont-en-Auge),这个小村庄位于Pont l'Évêque以西约四英里处。根据W. W. Rouse Ball的说法,[9] 他的父亲皮埃尔·德·拉普拉斯(Pierre de Laplace)拥有并耕种着马尔基斯的小庄园。他的曾叔父奥利维尔·德·拉普拉斯(Maitre Oliver de Laplace)曾担任皇家外科医生。似乎他从一名学生逐渐升职为博蒙学校的监督员;但在获得了给达朗贝尔(d'Alembert)的介绍信后,他前往巴黎以谋求更好的发展。然而,卡尔·皮尔逊(Karl Pearson)对Rouse Ball叙述中的不准确之处进行了严厉批评,并指出:

的确,在拉普拉斯时代,卡昂可能是诺曼底所有城市中最具知识活力的地方。正是在这里,拉普拉斯接受了教育,并暂时担任教授。也是在这里,他写下了发表在都灵皇家学会《混合文集》中的第一篇论文,卷四,1766-1769年,至少比他22或23岁时前往巴黎的1771年早了两年。因此,在他不到20岁时,他已经与位于都灵的拉格朗日保持联系。他并不是一个仅有农民背景的未经教化的乡村小伙子,贸然前往巴黎!1765年,16岁的拉普拉斯离开了博蒙的“奥尔良公爵学校”,前往卡昂大学,他似乎在这里学习了五年,并且是斯芬克斯社团的成员。博蒙的军事学校直到1776年才取代了这所旧学校。

他的父母,皮埃尔·拉普拉斯和玛丽-安妮·索雄,来自较为富裕的家庭。拉普拉斯家族至少在1750年之前一直从事农业,但皮埃尔·拉普拉斯(父亲)还是一位苹果酒商人和博蒙镇的市政官。

皮埃尔·西蒙·拉普拉斯曾就读于该村的一所由本笃会修道院经营的学校,他的父亲原本打算让他成为天主教会的神职人员。16岁时,为了进一步实现父亲的意图,他被送往卡昂大学学习神学。[10]

在大学期间,他受到两位热衷数学的教师克里斯托夫·加德布莱和皮埃尔·勒卡努的指导,正是他们激发了他对该学科的热情。在这里,拉普拉斯作为数学家的才华迅速得到了认可,并且在仍在卡昂时,他写下了《关于无穷小和有限差分的积分计算的论文》。这篇论文也标志着拉普拉斯与拉格朗日之间的第一次通信。拉格朗日比拉普拉斯年长十三岁,最近在他故乡都灵创办了一本名为《都灵文集》(Miscellanea Taurinensia)的期刊,许多他的早期作品都刊登在其中,拉普拉斯的论文正是在该系列的第四卷中发表的。大约在这个时候,拉普拉斯意识到自己并没有成为神职人员的天赋,于是决定成为一名职业数学家。有些资料指出,他此后与教会决裂并成为无神论者。[需要引用] 拉普拉斯没有获得神学学位,而是带着勒卡努写给让·勒朗·达朗贝尔的介绍信前往巴黎,而达朗贝尔在当时是科学界的权威。[10][11]

根据他的曾曾孙的说法,[8] 达朗贝尔最初对拉普拉斯的接待并不好,为了摆脱他,他给了拉普拉斯一本厚厚的数学书,说让他读完后再回来。几天后,当拉普拉斯回来时,达朗贝尔更为冷淡,并且没有掩饰他对拉普拉斯不可能读懂并理解那本书的看法。但在与拉普拉斯交谈后,达朗贝尔意识到拉普拉斯确实理解了书中的内容,从那时起,他开始对拉普拉斯给予指导。

另一种说法是,拉普拉斯在一夜之间解决了达朗贝尔要求他提交的下周的问题,然后在接下来的夜晚又解决了一个更难的问题。达朗贝尔对此印象深刻,并推荐他进入军事学院任教。[12]

有了稳定的收入和不太苛求的教学任务,拉普拉斯开始全身心投入原创研究,在接下来的十七年里(1771年至1787年),他在天文学领域创作了大量原创性作品。[13]
\begin{figure}[ht]
\centering
\includegraphics[width=6cm]{./figures/61f188bb0217db12.png}
\caption{《拉瓦锡和拉普拉斯的热量计》,《伦敦百科全书》,1801年} \label{fig_LPLS_6}
\end{figure}
从1780年到1784年,拉普拉斯与法国化学家安托万·拉瓦锡(Antoine Lavoisier)合作进行了一些实验研究,为这些任务设计了自己的设备。[14] 1783年,他们共同发表了论文《关于热的回忆录》,在其中讨论了分子运动的动理论。[15] 在他们的实验中,他们测量了不同物体的比热和金属在温度升高时的膨胀情况。他们还测量了乙醇和醚在压力下的沸点。

拉普拉斯进一步给孔多塞侯爵留下了深刻的印象,早在1771年,拉普拉斯便觉得自己有资格加入法国科学院。然而,那一年,学籍被授予了亚历山大-泰奥菲尔·范德蒙德(Alexandre-Théophile Vandermonde),1772年则授予了雅克·安托万·约瑟夫·库桑(Jacques Antoine Joseph Cousin)。拉普拉斯对此感到不满,1773年初,达朗贝尔写信给位于柏林的拉格朗日,询问是否可以为拉普拉斯找到一个职位。然而,孔多塞于2月成为了科学院的常任秘书,拉普拉斯在3月31日被选为院士候补成员,时年24岁。[16] 1773年,拉普拉斯在法国科学院前宣读了他的关于行星运动不变性的论文。同年3月,他被选为学院成员,之后他在该学术机构进行了大部分的科学研究。[17]

在1788年3月15日,拉普拉斯(Laplace)在39岁时娶了来自贝桑松一个“良好”家庭的18岁女孩玛丽-夏洛特·德·库尔蒂·德·罗曼热(Marie-Charlotte de Courty de Romanges)。婚礼在巴黎的圣叙尔皮斯教堂举行。夫妻俩有一个儿子,查尔斯-埃米尔(1789–1874),和一个女儿,索菲-苏珊娜(1792–1813)。
\subsection{分析、概率与天文稳定性}
拉普拉斯的早期发表作品始于1771年,主要涉及微分方程和有限差分,但他已经开始思考概率和统计的数学及哲学概念。[22] 然而,在1773年当选为法兰西科学院成员之前,他已经草拟了两篇论文,这些论文为他建立了声誉。第一篇论文《通过事件的概率推测因果关系》最终于1774年发表,第二篇论文则在1776年发表,进一步阐述了他的统计思想,并开始了他对天体力学和太阳系稳定性的系统研究。这两个学科始终在他的思维中紧密相连。“拉普拉斯将概率作为修正知识缺陷的工具。”[23] 拉普拉斯关于概率和统计的工作将在下文中讨论,包括他关于概率解析理论的成熟工作。
\subsubsection{太阳系的稳定性}  
艾萨克·牛顿爵士于1687年出版了《自然哲学的数学原理》(Philosophiæ Naturalis Principia Mathematica),在其中他根据自己的运动定律和万有引力定律推导出了开普勒定律,描述了行星的运动。然而,尽管牛顿在私下里已经发展了微积分方法,他的所有公开作品仍然使用繁琐的几何推理,这种方法并不适合解释行星间相互作用的更微妙的高阶效应。牛顿本人曾怀疑是否有可能找到整个问题的数学解法,甚至认为定期的神力干预是保证太阳系稳定所必需的。摒弃神力干预的假设将成为拉普拉斯科学生涯中的一项重大任务。[24] 现在普遍认为,尽管拉普拉斯的方法对理论的发展至关重要,但仅凭这些方法并不足够精确以证明太阳系的稳定性;今天我们理解太阳系在精细尺度上通常是混乱的,尽管在粗尺度上目前相对稳定。[25]: 83, 93

天文学中的一个特定问题是木星的轨道似乎在缩小,而土星的轨道在扩展。这个问题曾被莱昂哈德·欧拉于1748年和约瑟夫·路易·拉格朗日于1763年尝试解决,但都未成功。[26] 1776年,拉普拉斯发表了一篇论文,首次探讨了假设的光以太或不瞬时作用的引力定律可能的影响。最终,他回到了对牛顿引力的知识投资。[27] 欧拉和拉格朗日通过忽略运动方程中的小项,做出了一个实际的近似。拉普拉斯注意到,尽管这些项本身很小,但当它们在时间上积累时,可能会变得重要。拉普拉斯将他的分析扩展到了更高阶的项,直到包括三次项。通过这种更精确的分析,拉普拉斯得出结论,任何两颗行星和太阳必须处于相互平衡中,从而开启了他对太阳系稳定性的研究。[28] 杰拉尔德·詹姆斯·惠特罗称这一成就为“自牛顿以来物理天文学中最重要的进展”。[24]

拉普拉斯对所有科学都有广泛的知识,并主导了法兰西科学院的所有讨论。[29] 拉普拉斯似乎仅将分析视为解决物理问题的一种手段,尽管他发明所需分析的能力几乎是惊人的。只要他的结果是正确的,他对解释他得出这些结果的过程几乎不做任何努力;他从不研究过程中的优雅或对称性,对他来说,只要能以任何方式解决他所讨论的特定问题就足够了。[13]
\subsection{潮汐动力学}    
\subsubsection{潮汐的动态理论}  
虽然牛顿通过描述潮汐产生力来解释潮汐现象,伯努利则描述了地球上水体对潮汐势的静态反应,拉普拉斯于1775年发展出的潮汐动态理论[30] 描述了海洋对潮汐力的真实反应。[31] 拉普拉斯的海洋潮汐理论考虑了摩擦、共振和海洋盆地的自然周期。它预测了全球海洋盆地中大的安菲德罗米克系统,并解释了实际观测到的海洋潮汐。[32][33]

基于太阳和月球引力梯度的平衡理论,忽略了地球的自转、大陆的影响及其他重要因素,无法解释真实的海洋潮汐。[34][35][36][32][37][38][39][40][41]
\begin{figure}[ht]
\centering
\includegraphics[width=8cm]{./figures/e256f0da20f057c3.png}
\caption{} \label{fig_LPLS_3}
\end{figure}
由于测量结果已证实该理论,现在许多现象有了可能的解释,例如潮汐如何与深海脊和海山链相互作用,从而产生深层漩涡,将营养物质从深海输送到海面。[42] 平衡潮汐理论计算出的潮汐波高度不到半米,而动态理论则解释了潮汐为什么能达到15米。[43] 卫星观测证实了动态理论的准确性,全球的潮汐现已被测量到几厘米的误差范围内。[44][45] 来自CHAMP卫星的测量结果与基于TOPEX数据的模型非常吻合。[46][47][48] 全球潮汐的准确模型对研究至关重要,因为在计算重力和海平面变化时,必须从测量中去除潮汐引起的变化。[49]
\subsubsection{拉普拉斯的潮汐方程}
\begin{figure}[ht]
\centering
\includegraphics[width=6cm]{./figures/9b2d5a988b79470b.png}
\caption{A. 月球引力势:这表示从北半球上方看,月球正位于北纬30°(或南纬30°)的上空。} \label{fig_LPLS_4}
\end{figure}
\begin{figure}[ht]
\centering
\includegraphics[width=6cm]{./figures/aad7e166278f0243.png}
\caption{B. 这个视图显示的是与视图A相差180°的相同引力势。 从北半球上方看,红色朝上,蓝色朝下。} \label{fig_LPLS_5}
\end{figure}
1776年,拉普拉斯提出了一组描述潮汐流动的线性偏微分方程,潮汐流被描述为一种条形流动的二维薄层流动。引入了科氏效应以及由引力引起的侧向强迫力。拉普拉斯通过简化流体动力学方程获得了这些方程,但它们也可以通过拉格朗日方程从能量积分推导出来。

对于平均厚度为 \( D \) 的流体薄层,垂直潮汐升高 \( \zeta \) 以及水平方向速度分量 \( u \) 和 \( v \)(分别在纬度 \( \varphi \) 和经度 \( \lambda \) 方向上)满足拉普拉斯的潮汐方程:[50]
\[
\frac{\partial \zeta}{\partial t} + \frac{1}{a \cos(\varphi)} \left[ \frac{\partial}{\partial \lambda} (uD) + \frac{\partial}{\partial \varphi} \left( vD \cos(\varphi) \right) \right] = 0,~
\]
\[
\frac{\partial u}{\partial t} - v \left( 2 \Omega \sin(\varphi) \right) + \frac{1}{a \cos(\varphi)} \frac{\partial}{\partial \lambda} \left( g \zeta + U \right) = 0 \quad \text{and}~
\]
\[
\frac{\partial v}{\partial t} + u \left( 2 \Omega \sin(\varphi) \right) + \frac{1}{a} \frac{\partial}{\partial \varphi} \left( g \zeta + U \right) = 0,~
\]
其中,\( \Omega \) 是行星自转的角频率,\( g \) 是行星在平均海洋表面的重力加速度,\( a \) 是行星半径,\( U \) 是外部引力潮汐强迫势。

威廉·汤姆森(开尔文勋爵)使用旋度重新写出了拉普拉斯的动量项,从而得出了一个涡度方程。在某些条件下,这可以进一步重写为涡度守恒方程。
\subsection{关于地球的形状}  
在1784年至1787年期间,他发表了一些具有非凡影响力的论文。其中最重要的一篇是在1783年宣读的,并于1784年作为《行星的运动与椭圆形状理论》第二部分再版,同时也收录在《天体力学》第三卷中。在这项工作中,拉普拉斯完全确定了一个椭球体对其外部粒子的引力。这一成果因引入了球面调和函数或拉普拉斯系数而值得记住,同时也促进了我们现在所称的引力势在天体力学中的应用发展。
\subsubsection{球面调和函数}
在1783年,阿德里安-马里·勒让德在一篇提交给法兰西科学院的论文中引入了现在所称为关联勒让德函数。[13] 如果平面中的两个点具有极坐标 \((r, \theta)\) 和 \((r', \theta')\),其中 \(r' \geq r\),则通过基本的代数变换,可以将这两个点之间距离 \(d\) 的倒数表示为:
\[
\frac{1}{d} = \frac{1}{r'} \left[ 1 - 2 \cos(\theta' - \theta) \frac{r}{r'} + \left( \frac{r}{r'} \right)^2 \right]^{-\frac{1}{2}}.~
\]
这个表达式可以使用牛顿的广义二项式定理展开为 \(r/r'\) 的幂级数,得到:
\[
\frac{1}{d} = \frac{1}{r'} \sum_{k=0}^{\infty} P_k^0 (\cos(\theta' - \theta)) \left( \frac{r}{r'} \right)^k.~
\]
函数序列 \( P_k^0(\cos \varphi) \) 是所谓的“关联勒让德函数”集合,它们的实用性来源于这样一个事实:圆上每个点的函数都可以展开为这些函数的级数。[13]

拉普拉斯在几乎没有给予勒让德应有的信用的情况下,将这一结果扩展到三维空间,从而得出了一个更一般的函数集——球面调和函数或拉普拉斯系数。后一术语现在不常使用。[13]:第340页及后续
\subsubsection{势理论}  
这篇论文还因发展了标量势的概念而值得注意。[13] 在现代语言中,作用于物体的引力是一个向量,具有大小和方向。势函数是一个标量函数,它定义了这些向量如何变化。与向量函数相比,标量函数在计算和概念上更容易处理。

亚历克西·克莱劳特(Alexis Clairaut)在1743年首次提出了这一概念,当时他在处理一个类似的问题,尽管他使用的是牛顿类型的几何推理。拉普拉斯将克莱劳特的工作描述为“最美丽的数学成果之一”。[51] 然而,罗素·鲍尔(Rouse Ball)声称,这个想法“是从约瑟夫·路易·拉格朗日(Joseph Louis Lagrange)那里借用的,后者在1773年、1777年和1780年的论文中使用了这一概念”。[13] “势”这一术语本身是丹尼尔·伯努利(Daniel Bernoulli)提出的,他在1738年的《流体力学》论文中引入了这一术语。然而,根据罗素·鲍尔的说法,直到乔治·格林(George Green)在1828年发表的《数学分析在电磁理论中的应用》一文中,才真正使用了“势函数”这一术语(指代拉普拉斯意义上的空间坐标函数 \( V \))。[52][53]

拉普拉斯将微积分的语言应用于势函数,并证明它总是满足以下微分方程:[13]
\[
\nabla^2 V = \frac{\partial^2 V}{\partial x^2} + \frac{\partial^2 V}{\partial y^2} + \frac{\partial^2 V}{\partial z^2} = 0.~
\]
类似的结果早些年由莱昂哈德·欧拉(Leonhard Euler)得到,针对的是流体的速度势。[54][55]

拉普拉斯随后关于引力的研究基于这一结果。量 \(\nabla^2 V\) 被称为 \(V\) 的集中度,其在任意点的值表示该点 \(V\) 值相对于该点邻域中的平均值的“过剩”[56]。拉普拉斯方程,作为泊松方程的特例,在数学物理中无处不在。势的概念出现在流体动力学、电磁学以及其他领域。罗素·鲍尔(Rouse Ball)推测,这可以被看作是康德(Kant)感知理论中“先天形式”的一种“外在标志”[13]。

球面调和函数对拉普拉斯方程的实际解至关重要。拉普拉斯方程在球坐标系下(如用于绘制天空图的坐标系统)可以通过分离变量法简化为一个依赖于与中心点距离的径向部分和一个角度或球面部分。球面部分的解可以表示为拉普拉斯的球面调和函数级数,从而简化实际计算。
\subsection{行星和月球不等式}
\subsubsection{木星-土星大不等式}
\begin{figure}[ht]
\centering
\includegraphics[width=6cm]{./figures/e385c4f85dad3ba3.png}
\caption{德拉布尔《木星卫星历书》("Tables écliptiques des satellites de Jupiter")1817年版的标题页,其中在标题中提到了拉普拉斯的贡献。} \label{fig_LPLS_7}
\end{figure}
\begin{figure}[ht]
\centering
\includegraphics[width=6cm]{./figures/39e35371ddae04b5.png}
\caption{德拉布尔《木星卫星历书》("Tables écliptiques des satellites de Jupiter")1817年版中的表格——这些计算受到了拉普拉斯先前发现的影响。} \label{fig_LPLS_8}
\end{figure}
拉普拉斯在1784年、1785年和1786年分三部分提出了关于行星不等式的论文。该论文主要涉及识别和解释现在称为“木星-土星大不等式”的扰动。拉普拉斯解决了长期以来在研究和预测这些行星运动中的一个问题。他首先通过一般性考虑,证明了两个行星的相互作用永远不会导致它们轨道的偏心率和倾斜度发生大的变化;然而,更重要的是,他还指出,由于木星和土星的平均运动接近成比关系,这种特殊性在木星-土星系统中产生了。[6][57]

在这个背景下,"成比关系"意味着两个行星的平均运动比率几乎等于一对小整数之间的比率。土星围绕太阳的轨道周期几乎等于木星的五个周期。相应的,平均运动的倍数差(2nJ − 5nS)对应一个近900年的周期,这个周期在与这个相同周期的小扰动力的积分中作为一个小的除数出现。因此,具有这个周期的积分扰动非常大,土星的轨道经度扰动约为0.8°,木星约为0.3°。

拉普拉斯在1788年和1789年的两篇论文中进一步发展了这些关于行星运动的定理,但借助拉普拉斯的发现,木星和土星的运动表终于可以更加精确地制作。基于拉普拉斯的理论,德拉布尔(Delambre)计算出了他的天文表。[13]
\subsubsection{书籍}
拉普拉斯现在将自己定下任务,编写一部作品,旨在“为太阳系这一伟大的力学问题提供完整的解决方案,并使理论与观测尽可能吻合,以至于经验公式不再在天文表中占有一席之地。” 其成果体现在《世界体系概述》(Exposition du système du monde)和《天体力学》(Mécanique céleste)中。

前者于1796年出版,提供了对现象的一般解释,但省略了所有细节。它包含了天文学历史的概述。这个概述为作者赢得了进入法兰西科学院40名成员的荣誉,并被普遍认为是法国语言文学的杰作之一,尽管它在处理后期天文学内容时并不完全可靠。

拉普拉斯发展了行星系统形成的星云假说,该假说最早由伊曼纽尔·斯威登堡(Emanuel Swedenborg)提出,并由伊曼纽尔·康德(Immanuel Kant)扩展。这个假说至今仍然是研究行星系统起源的最广泛接受的模型。根据拉普拉斯对这一假说的描述,太阳系由一个旋转的发光气体球体演化而来,这个气体球体围绕其质心轴旋转。随着其冷却,这个气体球体收缩,连续的环带从其外缘脱落。这些环带又冷却,最终凝聚成行星,而太阳则代表着剩下的中央核心。根据这一观点,拉普拉斯预测,较远的行星将比靠近太阳的行星更古老。

如前所述,星云假说的概念早在1755年就由康德提出,[58]他还建议了“流星聚集”和潮汐摩擦作为影响太阳系形成的原因。拉普拉斯很可能知道这一点,但像他那个时代的许多作家一样,他通常不引用他人的工作。[8]

拉普拉斯对太阳系的分析性讨论载于他的《天体力学》(Mécanique céleste),该书分为五卷。前两卷于1799年出版,内容包括行星运动的计算方法、行星形状的确定以及潮汐问题的求解方法。[6] 第三卷和第四卷分别于1802年和1805年出版,包含了这些方法的应用以及若干天文表。第五卷于1825年出版,主要是历史性的,但作为附录给出了拉普拉斯最新研究的成果。《天体力学》包含了拉普拉斯的许多个人研究,但其中许多结果是借用其他作家的成果,且几乎没有提及来源。该书的结论,被历史学家描述为其他作家和拉普拉斯一百年努力的组织性成果,拉普拉斯却将其呈现为仅属于他自己的发现。[13]

让-巴蒂斯特·比奥(Jean-Baptiste Biot),在协助拉普拉斯修订《天体力学》时表示,拉普拉斯自己常常无法回忆起推理过程中的细节,如果确信结论是正确的,他便满足于插入一句话:“Il est àisé à voir que...” (“显而易见的是……”)。《天体力学》不仅是牛顿《自然哲学的数学原理》的微分学语言翻译,而且完成了牛顿未能填补的部分细节。这部著作在费利克斯·蒂塞朗(Félix Tisserand)的《天体力学教程》(1889–1896)中得到了更精细的呈现,但拉普拉斯的这部著作依然是标准的权威之作。[13] 在1784至1787年间,拉普拉斯发表了一些具有卓越影响力的论文。其中重要的论文之一发表于1784年,并在《天体力学》第三卷中再版。[需要引用] 在这部作品中,他完全确定了一个椭球体对其外部粒子的引力。这一工作因将“势”引入分析中而闻名,这一数学概念在物理科学中具有广泛的应用。
\subsection{光学}  
拉普拉斯支持牛顿的光子理论。在《天体力学》第四版中,拉普拉斯假设短程分子力是负责光子折射的原因。[59] 拉普拉斯和埃蒂安-路易·马吕斯(Étienne-Louis Malus)还证明了,光的双折射现象可以通过最小作用原理从光子中恢复出惠更斯原理。[60]

然而,1815年,奥古斯丁-让·弗雷内尔(Augustin-Jean Fresnel)提出了一种新的衍射波动理论,并在弗朗索瓦·阿拉戈(François Arago)的帮助下将其提交给法国科学院的委员会。拉普拉斯是委员会成员之一,最终他们授予弗雷内尔奖项,以表彰他的新理论。[61]: I.108 
\subsubsection{引力对光的影响}  
通过使用光子理论,拉普拉斯还接近于提出黑洞的概念。他认为引力可以影响光,并且可能存在一些巨大的恒星,其引力如此之大,以至于光也无法从其表面逃逸(参见逃逸速度)。[62][1][63][64] 然而,这一见解超前于当时的科学发展,因而在科学史上并未发挥作用。[65]
\subsection{阿尔库伊}
\begin{figure}[ht]
\centering
\includegraphics[width=6cm]{./figures/8ba12415755a6576.png}
\caption{拉普拉斯位于巴黎南部阿尔库伊的住所。} \label{fig_LPLS_9}
\end{figure}
“1806年,拉普拉斯在阿尔库伊购买了一座房子,当时阿尔库伊还是一个村庄,尚未并入巴黎的都市区。化学家克劳德·路易·贝尔托莱特是他的邻居——他们的花园没有隔开[66]——两人形成了一个非正式的科学圈子,后来被称为阿尔库伊学会。由于与拿破仑的亲近关系,拉普拉斯和贝尔托莱特实际上掌控了科学界的晋升和更高职位的入选。该学会建立了一个复杂的赞助金字塔[67]。1806年,拉普拉斯还被选为瑞典皇家科学院的外籍会员。”