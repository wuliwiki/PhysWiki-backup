% 拉普拉斯变换
% 拉普拉斯变换|拉普拉斯积分|拉普拉斯函数|收敛因子

拉普拉斯变换常用于微分方程的初始值问题,即已知某个物理量在初始时刻 $t=0$ 的值 $f(0)$,而求解它在初始时刻之后的变化情况 $f(t)$。至于它在初始时刻之前的值,我们就让它都等于 $0$,也就是说
\begin{equation}
f(t)=0 \quad(t<0)
\end{equation}
为了获得较宽的变换条件,构造一个函数 $g(t)$,
\begin{equation}
g(t)=\mathrm{e}^{-\sigma t} f(t)
\end{equation}
这里 $e^{-\sigma t}$ 为收敛因子。我们需要选一个充分大的正实数 $\sigma$,用来保证 $g(t) $ 在区间 $(-\infty,+\infty)$ 上绝对可积。于是,可以对 $g(t) $ 做傅里叶变换:
\begin{equation}
G(\omega)=\frac{1}{2 \pi} \int_{-\infty}^{\infty} g(t) \mathrm{e}^{-i\omega t} \mathrm{d} t=\frac{1}{2 \pi} \int_{0}^{\infty} f(t) \mathrm{e}^{-(\sigma+\mathrm{i} \omega) t} \mathrm{d} t
\end{equation}
将 $\sigma+i \omega$ 记作 $p$,并将 $G(\omega)$ 改记作 $\bar f(p) / 2 \pi$ 则
\begin{equation} \label{LapTra_eq1}
\bar{f}(p)=\int_{0}^{\infty} f(t) \mathrm{e}^{-p t} \mathrm{d} t
\end{equation}
其中积分
\begin{equation}
\int_{0}^{\infty} f(t) \mathrm{e}^{-p t} \mathrm{d} t
\end{equation}
称为\textbf{拉普拉斯积分}。$\bar f(p)$ 称为 $f(t)$ 的\textbf{拉普拉斯函数}。\autoref{LapTra_eq1} 称为\textbf{拉普拉斯变换}(简称\textbf{拉氏变换}),$\mathrm e^{-pt}$ 是拉普拉斯变换的\textbf{核}。

$G(\omega)$ 的傅里叶逆变换是:
\begin{equation}
g(t)=\int_{-\infty}^{\infty} G(\omega) \mathrm{e}^{\mathrm{i} \omega t} \mathrm{d} \omega=\frac{1}{2 \pi} \int_{-\infty}^{\infty} \bar{f}(\sigma+\mathrm{i} \omega) \mathrm{e}^{\mathrm{i} \omega t} \mathrm{d} \omega
\end{equation}
即
\begin{equation}
f(t)=\frac{1}{2 \pi} \int_{-\infty}^{\infty} \bar{f}(\sigma+\mathrm{i} \omega) \mathrm{e}^{(\sigma+\mathrm{i} \omega)t} \mathrm{d} \omega
\end{equation}
由于 $\sigma+\mathrm i\omega=p$,可得 $\mathrm d\omega = 1/\mathrm i\mathrm dp$。所以有
\begin{equation}
f(t)=\frac{1}{2 \pi \mathrm{i}} \int_{\sigma-\mathrm{i} \infty}^{\sigma+\mathrm{i} \infty} \bar{f}(p) \mathrm{e}^{\mathrm{i} p} \mathrm{d} p
\end{equation}

$\bar f(p)$ 又称为\textbf{像函数},而 $f(t)$ 称为\textbf{原函数}。它们之间的关系通常简单地用记号表示为:
\begin{equation}
\begin{array}{l}\bar{f}(p)=\mathscr{S}[f(t)] \\ f(t)=\mathscr{L}^{-1}[\bar{f}(p)]\end{array}
\end{equation}

我们来看几个例题熟悉一下它的运算。

\begin{example}{}
求 $\mathscr L[1]$。

在 $\Re p>0$(即 $\sigma>0$)的半平面上,
\begin{equation}
\int_{0}^{\infty} 1 \cdot e^{-p t} d t=\frac{1}{p}
\end{equation}
所以
\begin{equation}
\mathscr{L}[1]=\frac{1}{p} \quad(\operatorname{Re} p>0)
\end{equation}
\end{example}

\begin{example}{}
求 $\mathscr L[t]$。

在 $\Re p>0$ 的半平面上
\begin{equation}
\begin{aligned} \int_{0}^{\infty} t \mathrm{e}^{-p t} \mathrm{d} t &=-\frac{1}{p} \int_{0}^{\infty} t \mathrm{d}\left(\mathrm{e}^{-p t}\right) \\ &=-\frac{1}{p}\left[t \mathrm{e}^{-p t}\right]_{0}^{\infty}+\frac{1}{p} \int_{0}^{\infty} \mathrm{e}^{-p t} \mathrm{d} t \\ &=\frac{1}{p} \int_{0}^{\infty} \mathrm{e}^{-p t} \mathrm{d} t=\frac{1}{p^{2}} \end{aligned}
\end{equation}
所以
\begin{equation}
\mathscr{L}[t]=\frac{1}{p^{2}} \quad(\operatorname{Re} p>0)
\end{equation}
\end{example}

\begin{example}{}
求 $\mathscr L[\mathrm e^{st}]$,$s$ 为常数。

在 $\operatorname{Re} p>\operatorname{Re} s$ 的半平面上,
\begin{equation}
\begin{aligned} \int_{0}^{\infty} \mathrm{e}^{s t} \mathrm{e}^{-p t} \mathrm{d} t &=\int_{0}^{\infty} \mathrm{e}^{-(p-s) t} \mathrm{d} t=-\frac{1}{p-s}\left[\mathrm{e}^{-(p-s) t}\right]_{0}^{\infty} \\ &=\frac{1}{p-s} \end{aligned}
\end{equation}
所以
\begin{equation}
\mathscr{L}\left[\mathrm{e}^{st}\right]=\frac{1}{p-s} \quad(\operatorname{Re} p>\operatorname{Re} s)
\end{equation}
\end{example}

\begin{example}{}
求 $\mathscr L[t\mathrm e^{st}]$,$s$ 为常数。

在 $\operatorname{Re} p>\operatorname{Re} s$ 的半平面上,
\begin{equation}
\begin{aligned} \int_{0}^{\infty} t \mathrm{e}^{s t} \mathrm{e}^{-p t} \mathrm{d} t &=-\frac{1}{p-s} \int_{0}^{\infty} t \mathrm{d}\left[\mathrm{e}^{-(p-s) t}\right] \\ &=-\frac{1}{p-s}\left\{\left[t \mathrm{e}^{-(p-s) t}\right]_{0}^{\infty}-\int_{0}^{\infty} \mathrm{e}^{-(p-s) t} \mathrm{d} t\right\} \\ &=\frac{1}{(p-s)^{2}} \end{aligned}
\end{equation}
所以
\begin{equation}
\mathscr{S}\left[t \mathrm{e}^{s t}\right]=\frac{1}{(p-s)^{2}} \quad(\operatorname{Re} p>\operatorname{Re} s)
\end{equation}
同理
\begin{equation}
\mathscr{L}\left[t^{n} \mathrm{e}^{n}\right]=\frac{n !}{(p-s)^{n+1}}
\end{equation}
\end{example}
