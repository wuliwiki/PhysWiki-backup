% 皮卡映射
% keys 皮卡映射|逐次皮卡近似
% license Xiao
% type Tutor

\begin{issues}
\issueDraft
\end{issues}

\pentry{定积分\upref{DInt},映射\upref{map}}
通过曲线 $\varphi$ 构造出一新曲线 $f(\varphi)$ ,使得新曲线上每一点 $f(\varphi(t))$ 的切线平行于曲线 $\varphi$ 上点 $\varphi(t)$ 处给定的向量。具体来说,设 $U$ 是 $\mathbb R^n$ 的一区域,其上每一点 $x$ 都定义了一个依赖于时间的向量 $v(x,t)$ (或称为 $U$ 上定义了依赖于时间的向量场 $v$)。给定 $U$ 中的曲线 $\varphi:I\rightarrow U$($I$ 为 $t$ 轴上一区间),那么向量场 $v$ 在曲线上每一点 $\varphi(t)$ 对应的向量为 $v(\varphi(t),t)$。于是新曲线 $f(\varphi)$ 是这样的曲线,其在每一 $\tau$ 时的点 $f(\varphi(\tau))$ 的切向量满足 $\dv{}{t}\big|_{t=\tau}(f\circ \varphi)=v(\varphi(\tau),\tau)$\footnote{当然,平行说明还有个系数,但是这里特指系数为1}。描述这样的曲线 $\varphi$ 到新曲线 $f(\varphi)$ 的映射\upref{map}称为\textbf{皮卡映射(Picard 映射)}。
\subsection{皮卡映射}
\begin{definition}{皮卡映射}\label{def_PicMap_1}
设 $U$ 是 $\mathbb R^{n+1}$ 的一区域,在其上定义了一向量场\footnote{在 $U\in\mathbb R^{n+1}$ 上定义了向量场相当于在 $\mathbb R^n$ 中一区域定义了依赖于时间 $t\in\mathbb R$ 的向量场。前者表述的空间称为后者的扩张空间(\autoref{sub_PSaPF_1}~\upref{PSaPF})。}(\autoref{def_GofODE_6}~\upref{GofODE}) 
\begin{equation}
v:U\rightarrow\mathbb TU~.
\end{equation}
$\varphi$ 是 $U$ 中满足 $\varphi(t_0)=x_0$ 的曲线,则称映射
\begin{equation}\label{eq_PicMap_1}
(A\varphi)(t)\equiv x_0+\int_{t_0}^{t}v(\varphi(\tau),\tau)\dd \tau~
\end{equation}
为曲线 $\varphi$ 的\textbf{皮卡映射(Picard Map)}。
\end{definition}
\autoref{eq_PicMap_1} 的微分形式为
\begin{equation}\label{eq_PicMap_2}
\dv{}{t}((A\varphi)(t))=v(\varphi(t),t)~.
\end{equation}
所以\autoref{eq_PicMap_1} 等价于初始条件满足 $(A\varphi)(t_0)=x_0$ 的微分方程(\autoref{eq_PicMap_2}). 由向量场 $v$ 确定的微分方程的定义(\autoref{def_GofODE_7}~\upref{GofODE}),于是\textbf{$\varphi$是满足初始条件 $\varphi(t_0)=x_0$ 的由向量场 $v$ 确定的微分方程的解,当且仅当 $\varphi=A\varphi$。}由映射不动点的定义,\textbf{向量场对应微分方程的解就是皮卡映射的不动点}。这可以描述为下面定理
\begin{theorem}{微分方程的解是皮卡映射的不动点}\label{the_PicMap_1}
设$v$ 是区域 $U\in\mathbb R^{n+1}$ 中定义的向量场,则 $\varphi$ 是 $v$ 确定的微分方程
\begin{equation}\label{eq_PicMap_3}
\dot x=v(x,t)~
\end{equation}
满足初始条件 $\varphi(t_0)=x_0$ 的解,当且仅当 $\varphi$ 是皮卡映射(\autoref{def_PicMap_1} )的不动点,即 $A\varphi=\varphi$。
\end{theorem}
于是,微分方程有解,当且仅当皮卡映射存在不动点。
\subsection{逐次皮卡近似}
逐次皮卡近似是指连续施行皮卡映射。考虑逐次皮卡近似是因为其和微分方程的解存在有着对应关系。
\begin{definition}{逐次皮卡近似}
设 $A$ 是皮卡映射,则称序列
\begin{equation}
\{A^n\varphi\}=\{\varphi, A\varphi, A^2\varphi,\cdots,\}~ 
\end{equation}
为初始曲线为 $\varphi$ 的\textbf{逐次皮卡近似}。
\end{definition}
\begin{theorem}{}
若皮卡映射是定义在连续可微曲线的完备空间上l,且逐次皮卡近似收敛,则微分方程\autoref{eq_PicMap_3} 有解。
\end{theorem}
\textbf{证明:}
\begin{equation}
\lim_{n\rightarrow\infty}A^n\varphi\overset{\text{收敛性}}{=}\lim_{n\rightarrow\infty}A^{n+1}\varphi \overset{\text{完备性}}{=}A\lim_{n\rightarrow\infty}A^n\varphi~.
\end{equation}
于是 $\lim\limits_{n\rightarrow\infty}A^n\varphi$ 是 $A$ 的不动点,由\autoref{the_PicMap_1} ,微分方程\autoref{eq_PicMap_3} 有解 $\lim\limits_{n\rightarrow\infty}A^n\varphi$。


\textbf{证毕!}
\begin{example}{指数函数}
设向量场定义为 $v(x,t)=x$ ,且给定曲线 $\varphi(t)=x_0$,取 $t_0=0$,于是此时皮卡近似为
\begin{equation}
\begin{aligned}
\varphi(t)&=x_0,\\
A\varphi(t)&=x_0+\int_{0}^{t}x_0\dd t=x_0(1+t),\\
A^2\varphi(t)&=x_0+\int_{0}^{t}x_0(1+t)\dd t=x_0(1+t+\frac{1}{2}t^2),\\
&\vdots\\
A^n\varphi(t)&=x_0\sum_{i=0}^n\frac{t^i}{i!}\\
\lim_{n\rightarrow\infty}A^n\varphi&=e^tx_0
~.
\end{aligned}
\end{equation}
只需假设 $A$ 的定义域上的曲线都是连续可微的(这是容易做到的),由于
\end{example}