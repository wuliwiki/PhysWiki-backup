% 阿兰·图灵
% license CCBYSA3
% type Wiki

(本文根据 CC-BY-SA 协议转载自原搜狗科学百科对英文维基百科的翻译)

艾伦·麦席森·图灵·奥博·弗莱斯(英语:Alan Mathison Turing,/ˈtjʊərɪŋ/,又译阿兰·图灵,Turing也常翻译成涂林或者杜林,1912年6月23日——1954年6月7日),英国数学家、计算机科学家、逻辑学家、密码分析学家、哲学家和理论生物学家。图灵在理论计算机科学的发展中具有很大的影响力,他为算法和计算的概念提供了一个形式化的图灵机,图灵机可以被认为是一个通用计算机的模型,为现代计算机的逻辑工作方式奠定了基础,[1]同时也被称为理论计算机科学和人工智能之父。[2]

第二次世界大战期间,图灵在破解截获的编码信息方面发挥了关键作用,这些信息使盟军能够在许多关键战役中击败纳粹,包括大西洋战役。[3][4]从长远意义上讲,据估计这项工作缩短了欧洲战争两年多,挽救了1400多万人的生命。[3]1952年因同性恋行为被起诉,拉博彻修正案规定“严重猥亵”在英国是刑事犯罪。他接受了化学阉割治疗,用DES替代监狱。1954年,图灵死于氰化物中毒,享年41岁。[5]2013年12月24日,在英国司法大臣克里斯·格雷灵的要求下,英国女王伊丽莎白二世向图灵颁发了皇家赦免。

\subsection{ 早期生活}
\subsubsection{1.1 家庭}
图灵出生在伦敦迈达谷,[6]而他的父亲朱利叶斯·马西森·图灵(1873-1947年)则离开了他在查特拉布尔(Chatrapur)的印度公务员队伍,当时是马德拉斯(Madras)总统任期,目前在印度奥迪沙邦( Odisha)。[6][7]图灵的父亲是牧师约翰·罗伯特·图灵牧师的儿子,他来自一个位于荷兰的苏格兰商人家庭,其中包括一名准男爵。图灵的母亲,朱利叶斯(Julius)的妻子,是埃塞尔·萨拉·图灵(内·斯通尼,1881-1976),[6]爱马德拉斯铁路公司(Madras Railways)总工程师爱德华·沃勒·斯通尼的女儿。斯通尼一家是蒂珀雷里县和龙福德县的新教盎格鲁爱尔兰贵族家庭,而埃塞尔本人童年大部分时间都在克莱尔县(County Clare)度过。[8]

朱利叶斯与ICS的合作将这个家庭带到了英属印度,他的祖父曾在那里的孟加拉军队中担任将军。然而,朱利叶斯和埃塞尔都想让他们的孩子在英国长大,所以他们搬到了伦敦的迈达山谷,[9]艾伦·图灵于1912年6月23日出生在那里,正如他出生的房子(后来的Colonnade酒店)外面的蓝色匾所记录的那样[10][11] [6][12]。图灵有一个哥哥约翰(约翰·德莫特·图灵爵士的父亲,图灵男爵的第12个男爵)。[13]

图灵父亲的公务员委员会仍然很活跃,在图灵童年时期,图灵的父母在英国黑斯廷斯( Hastings)和印度之间旅行,[14]留下他们的两个儿子和一对退休的陆军夫妇住在一起。在黑斯廷斯,图灵住在上梅兹山的巴斯顿小屋(Baston Lodge, Upper Maze Hill),圣莱昂纳斯(St Leonards-on-Sea),现在有一块蓝色的匾,[15]这块匾于图灵诞生一百周年的2012年6月23日揭幕。[16]

在早年,图灵展示了天才的迹象,后来他又突出地展示了这一点。[17]1927年,他的父母在吉尔福德(Guildford)买了一栋房子,图灵在学校放假期间住在那里。该位置还标有蓝色斑块。[18]
\subsubsection{1.2 学校}
图灵的父母在他六岁的时候就给他注册了圣迈克尔学校(St Michael's),这是一所位于海边的查尔斯路(Charles Road)20号的日制学校。女校长很早就发现了他的才华,他后来的许多老师也是如此。

1922年1月至1926年1月间,图灵在哈斯赫斯特预备学校(Hazelhurst Preparatory School)接受教育,这所学校是苏塞克斯郡(Sussex)(现为东苏塞克斯郡)弗兰特村的一所独立学校。[19]1926年,13岁时,他继续就读于舍伯恩(Sherborne) 学校,这是一所位于多塞特郡舍伯恩集镇的独立寄宿学校。开学的第一天正好赶上1926年英国总罢工,但他下定决心要参加,以至于他独自骑自行车从南安普敦到谢伯恩60英里(97公里),并在一家旅店过夜。[20]

图灵对数学和科学的天生爱好并没有赢得舍伯恩的一些老师的尊重,舍伯恩的老师们对教育的定义更加强调经典。他的校长写信给他的父母:“我希望他不会两头落空。如果他要留在公立学校,他必须以接受教育为目标。如果他是一名科学专家,那他在公立学校是对时间的一种浪费”。[21]尽管没有得到老师们的认可,但是图灵继续在他热爱的研究中表现出非凡的能力。1927年,他甚至在连初等微积分没接触过的基础上就解决了高级问题;1928年,16岁的图灵读到了阿尔伯特·爱因斯坦的作品;他不仅领会了这一点,而且有可能从一篇从未明确阐述过的文章中推断出爱因斯坦对牛顿运动定律的质疑。[22]

舍伯恩毕业后,图灵从1931年到1934年在剑桥国王学院学习,[6]并在那里获得了一级数学荣誉。1935年,他22岁时,凭借一篇证明中心极限定理的论文,被选为国王学院的研究员。[23]委员会不知道,这个定理已经在1922年被贾尔·瓦尔德马尔·林德伯格(Jarl Waldemar Lindeberg)证明了。[24]2012年6月23日,该学院的一块蓝色牌匾在他诞辰100周年之际揭幕,现已安装在国王阅兵式上的学院凯恩斯大楼(Keynes Building)。[25][26]
\subsubsection{1.3 朋友}
在舍伯恩,图灵和他的同学克里斯托弗·莫克姆(Christopher Morcom,1911-1930)建立了重要的友谊,克里斯托弗·莫克姆被描述为图灵的“初恋”。他们的关系为图灵未来的成就提供了灵感,但由于莫克姆于1930年2月死于牛结核病的并发症,这一关系被打断了。牛结核病是几年前喝了受感染的牛奶后感染的。[27][28][29]

这件事给图灵带来了巨大的悲痛。他通过更加努力地研究他与莫克姆共享的科学和数学主题来应对自己的悲伤。在给莫克姆的母亲的一封信中图灵说:

我相信我在任何地方都找不到另一个如此聪明、如此迷人、如此不可思议的伴侣。我认为我对我的工作和天文学(他向我介绍了天文学)的兴趣是可以和他分享的,我想他对我也有一点相同的感觉...我知道我必须把尽可能多的精力投入到我的工作中,就像他活着一样,因为那是他希望我做的。

莫克姆去世后很久,图灵和莫克姆的母亲一直保持着联系,她给图灵送礼物,而他通常在莫克姆生日那天写信。莫克姆去世三周年的前一天(1933年2月12日),他写信给莫克姆夫人:

我希望你收到这封信时会想起克里斯。我也会的,这封信只是想告诉你,明天我会想到克里斯和你。我确信他现在和曾经在这里时一样快乐。你亲爱的艾伦。

一些人推测,莫克姆的死是图灵无神论和唯物主义的原因。显然,在生命的这一点上,他仍然相信精神(独立于肉体,在死亡中幸存)的概念。图灵在后来的一封信中说:

就我个人而言,我相信精神确实永远与物质联系在一起,但肯定不是由同一种身体联系在一起的...至于精神和身体之间的实际联系,我认为身体可以抓住一个“精神”,当身体活着和清醒时,两者是紧密相连的。当身体睡着的时候,我无法猜测会发生什么,但是当身体死亡的时候,身体的“机制”就消失了,灵魂迟早会找到一个新的身体,也许是马上。

\subsection{研究成就}
图灵在科学、特别在数理逻辑和计算机科学方面,取得了举世瞩目的成就,他的一些科学成果,构成了现代计算机技术的基础。
\subsubsection{2.1 可计算性研究}
\begin{figure}[ht]
\centering
\includegraphics[width=8cm]{./figures/e29f60892c491fbe.png}
\caption{剑桥国王学院。图灵于1931年成为这里的学生,并在1935年成为研究员。微机室是以他的名字命名的。} \label{fig_ALTL_1}
\end{figure}
1936年,图灵发表了他的论文《关于可计算的数字,及其在Entscheidungsproblem中的应用》(1936)[30],在这篇论文中,图灵重新表述了库尔特·哥德尔(Kurt Gödel's)1931年关于证明和计算极限的结果,用后来被称为图灵机的形式和简单的假设装置取代了哥德尔的通用算术形式语言。决策问题最初是由德国数学家戴维·希尔伯特在1928年提出的。图灵证明了他的“通用计算机器”能够执行任何可以想象的数学计算,如果它可以表示为一种算法的话。他继续证明决策问题没有解决方案,首先证明图灵机的停机问题是不可判定的:不可能通过算法决定图灵机是否会停机。

虽然图灵的证明是在阿隆佐·邱奇(Alonzo Church's)用λ演算进行等价证明后不久发表的,[31]但图灵的方法比丘奇的方法更容易理解和直观。[32]它还包含了“通用机器”(现在被称为通用图灵机)的概念,认为这样的机器可以执行任何其他计算机器的任务(实际上教会的λ演算也可以)。根据丘奇-图灵理论,图灵机和λ演算能够计算任何可计算的东西。约翰·冯·诺依曼(John von Neumann)承认现代计算机的核心概念源于图灵的论文。[33]迄今为止,图灵机是计算理论的一个中心研究对象。

从1936年9月到1938年7月,图灵第二年大部分时间在普林斯顿大学教堂学习,[34]除了纯粹的数学工作之外,他还研究密码学,并建立了机电二进制乘法器的四个阶段中的三个[53]。1938年6月,他获得了普林斯顿数学系的博士学位;[35]他的论文《基于序数的逻辑系统》引入了序数逻辑的概念和相对计算的概念,[36][37]在这两个概念中,图灵机被所谓的预言所增强,从而可以研究图灵机无法解决的问题。约翰·冯·诺依曼想雇他做博士后助理,但他回到了英国。[38]

图灵回到剑桥后,他参加了1939年路德维希·维特斯坦根(Ludwig Wittgenstein)关于数学基础的讲座。[39]讲座逐字逐句地进行了重构,包括图灵和其他学生的感叹词,以及学生笔记。[40]图灵和维特根斯坦争论不休,图灵为形式主义辩护,维特根斯坦提出他的观点,即数学不发现任何绝对真理,而是发明了它们。[41]
\subsubsection{2.2 密码分析学}
\begin{figure}[ht]
\centering
\includegraphics[width=8cm]{./figures/66f4e52ad7bf9342.png}
\caption{布莱奇利公园的马场里有两间小屋。图灵1939年和1940年在这里工作,后来搬到8号小屋。} \label{fig_ALTL_2}
\end{figure}
在第二次世界大战期间,图灵是布莱切利公园破解德国密码的主要参与者。历史学家和战时破译员阿萨·布里格斯(Asa Briggs)说过,“你需要非凡的才能,你需要布莱切利的天才,图灵就是那个天才。”[42]

从1938年9月开始,图灵一直在英国密码破译组织——政府法典和塞弗尔学校(GC&CS)兼职工作。他和GC&CS高级密码破译员迪利·诺克斯一起专注于密码分析。[43]在1939年7月华沙会议上,波兰密码局向英国和法国提供了英格玛转子接线的细节以及他们解密英格玛密码信息的方法,此后不久,图灵和诺克斯开始研究一种不那么脆弱的方法来解决这个问题。[44]波兰的方法依赖于一个不安全的指标程序,德国人很可能会改变这个程序,他们在1940年5月就这样做了。图灵的方法更普遍,使用基于婴儿床的解密,为此他产生了炸弹的功能规范(波兰炸弹的改进)。[45]

1939年9月4日,英国对德宣战的第二天,图灵向GC&CS战时车站布莱切利公园报到。[46]指定炸弹是图灵在战争期间取得的五大密码分析进展中的第一个。其他的是:推导德国海军使用的指示程序;制定一个统计程序,以便更有效地利用被称为班布里斯穆斯的炸弹;开发了一个程序来计算Lorenz SZ 40/42 (Tunny)命名为Turingery的车轮的凸轮设置,并在战争即将结束时,在Hanslope Park开发了一个便携式安全语音扰频器,代号为Delilah。

图灵通过使用统计技术优化代码破解过程中不同可能性的试验,对这个课题做出了创新性贡献。他写了两篇讨论数学方法的论文,题目分别是《概率在密码学中的应用》和《重复统计论文》。这两篇论文对GC&CS及其继任者GCHQ非常有价值,直到2012年4月,也就是他出生一百周年前夕,才被公布到英国国家档案馆。一位GCHQ数学家,“他自称只是理查德”,当时说,内容被限制了大约70年的事实证明了它们的重要性,以及它们与战后密码分析的相关性:

他说,内容受到限制的事实“表明它在我们学科的基础上有多么重要”。...论文详细介绍了如何使用“数学分析来尝试并确定哪些设置更有可能,以便能够尽快尝试。”...理查德说,GCHQ现在已经从这两份文件中“榨出了汁液”,并且“很高兴它们被公开”。

图灵在布莱切利公园以古怪著称。他被他的同事们称为“教授”,他关于谜的论文被称为“教授的书”。[47]根据历史学家罗纳德·列文的说法,与图灵一起工作的密码分析师杰克·古德这么评价他:

每年六月的第一周,他会患上严重的花粉热,他会戴着防毒面具骑车去办公室防止花粉传播。他的自行车有一个毛病:链条会定期脱落。他会数踏板转动的次数,然后及时从自行车上下来,用手调整链条,而不是修理它。他的另一个怪癖是他把杯子锁在散热管上以防被偷。

图灵是一名优秀的长跑运动员,在布莱切利工作时,他偶尔会在需要开会时跑40英里(64公里)到伦敦,[48]他有能力达到世界级的马拉松标准。[49][50]图灵参加了1948年英国奥林匹克队的选拔,但他因受伤而受阻。他的马拉松预赛时间仅比英国银牌得主托马斯·理查兹(Thomas Richards')的奥林匹克比赛时间慢11分钟,后者为2小时35分钟。他是沃尔顿体育俱乐部最好的跑步者,这是当他独自跑步时通过团体时发现的事实。[51][52][53]

1946年,图灵因其战时服务被乔治六世国王(King George VI)任命为大英帝国勋章(OBE)的一名军官,但他的工作多年来一直保密。[54]
\subsubsection{2.3 炸弹机}
\begin{figure}[ht]
\centering
\includegraphics[width=8cm]{./figures/8888b94e60068354.png}
\caption{布莱奇利公园的国家计算机博物馆。图为一个正在工作的完整轰炸机复制品。} \label{fig_ALTL_3}
\end{figure}
在到达布莱切利公园后的几周内,[46]图灵专注于一种名为“bombe”的机电机器,这种机器可以比波兰炸弹kryptologiczna更有效地破解“谜”。这个炸弹,由数学家戈登·韦尔奇曼Gordon Welchman)提出一个改进,成为主要工具之一,也是主要的自动化工具,用来攻击英格玛加密的信息。[55]

炸弹使用合适的婴儿床(可能明文的片段)搜索用于英格玛(Enigma)消息的可能正确设置(即转子顺序、转子设置和插板设置)。[56]对于转子的每一种可能的设置(对于四转子U型船,大约有1019个状态,或者1022个状态)[80]炸弹执行了一系列基于婴儿床的逻辑推理,并实现了机电一体化。

当矛盾发生时,炸弹被探测到,并排除了这种设置,转到下一个。大多数可能的设置会导致矛盾并被丢弃,只留下少数需要详细调查。当一个加密的字母变成相同的明文字母时,就会发生矛盾,这在英格玛中是不可能的。第一枚炸弹于1940年3月18日安装。[57]

到1941年底,图灵和他的密码分析师戈登·韦尔奇曼、休·亚历山大和斯图尔特·米尔纳-巴里(Gordon Welchman, Hugh Alexander and Stuart Milner-Barry )都感到沮丧。在波兰人工作的基础上,他们建立了一个很好的解密英格玛信号的工作系统,但是他们有限的人员和炸弹意味着他们不能翻译所有的信号。夏季,他们取得了相当大的成功,航运损失降至每月不到10万吨;然而,他们迫切需要更多的资源来跟上德国的调整。他们试图通过适当的渠道吸引更多的人和资助更多的炸弹制造商,但失败了。[58]

10月28日,他们直接写信给温斯顿·丘吉尔(Winston Churchill),解释他们的困难,图灵是第一个被指名的人。他们强调,他们的需求与部队的大量人力和财力相比是多么微不足道,与他们能够向部队提供的援助水平相比又是多么微不足道。[58]正如图灵的传记作者安德鲁·霍奇斯(Andrew Hodges)后来写道,“这封信有一种电效应。”[59]丘吉尔给伊斯梅尔将军(General Ismay)写了一份备忘录,上面写道:“今天行动。确保他们把他们想要的一切放在最优先的位置,并向我报告已经完成了。”11月18日,特勤局局长报告说,正在采取一切可能的措施。[59]布莱切利公园的密码学家不知道首相的反应,但正如米尔纳-巴里(Milner-Barry)回忆的那样,“我们所注意到的是,几乎从那天起,崎岖的道路奇迹般地变得平坦起来。”[60]到战争结束时,已有两百多枚炸弹投入使用。[61]
\begin{figure}[ht]
\centering
\includegraphics[width=8cm]{./figures/b4a82ee8c68173f8.png}
\caption{斯蒂芬·凯特尔在布莱奇利公园的图灵雕像,由西德尼·弗兰克委托,由50万块威尔士石板建造。.[34]} \label{fig_ALTL_4}
\end{figure}
\subsubsection{2.4 8号小屋和海军之谜}
图灵决定解决德国海军“英格玛”这个特别困难的问题,“因为没有其他人对此做任何事,我可以独享它”。[62]1939年12月,图灵解决了海军指示系统的关键部分,这个部分比其他部门使用的指示系统更复杂。[62][63]

同一天晚上,他还想到了班布里斯穆斯(Banburismus)的想法,这是一种序列统计技术(亚伯拉罕·瓦尔德后来称之为序列分析),用来帮助破解海军之谜,“尽管我不确定它在实践中是否行得通,事实上,直到有几天真的破解了,我也不确定。”[62]为此,他发明了一种衡量证据权重的方法,称之为禁令。Banburismus可以排除英格玛转子的某些序列,从而大大减少测试炸弹设置所需的时间。[64]后来,这种利用决策层(十分之一的决策层)积累足够证据权重的连续过程被用于洛伦兹密码的密码分析。[65]

图灵于1942年11月前往美国,[66]并与美国海军密码分析师合作在华盛顿建造海军英格玛和邦贝;他还参观了他们在俄亥俄州代顿的计算机实验室。

图灵对美国炸弹设计的反应并不热烈:

美国炸弹计划将生产336枚炸弹,每个车轮订单一枚。我过去常常对这个节目暗示的Bombe hut例程的概念暗自微笑,但我认为指出我们不会真的那样使用它们,不会达到任何特定的目的。 他们的测试(换向器)很难被认为是结论性的,因为他们没有测试电子站发现装置的反弹。似乎没有人被告知关于杆或offiziers或banburismus,除非他们真的要做些什么。

在这次旅行中,他还在贝尔实验室协助开发了安全语音设备。[67]他于1943年3月回到布莱切利公园。休·亚历山大不在的时候,他正式担任了8号小屋的负责人,尽管亚历山大实际上已经当了一段时间的负责人(图灵对这一部分的日常运作不感兴趣)。图灵成为布莱切利公园密码分析的总顾问。[68]

亚历山大写道图灵的贡献:

毫无疑问,图灵的工作是8号小屋成功的最大因素。在早期,他是唯一认为这个问题值得解决的密码学家,他不仅主要负责小屋内的主要理论工作,还与韦尔奇曼和基恩分享了炸弹发明的主要功劳。很难说任何人都是“绝对不可或缺的”,但是如果说任何人对8号小屋来说都是不可或缺的,那就是图灵。当经验和例行公事使一切变得简单时,先驱的工作总是被遗忘,我们8号小屋的许多人觉得图灵的贡献从未被外界完全意识到。
\subsubsection{2.5 图灵格林}
1942年7月,图灵发明了一种被称为图灵格林(Turingery) (或戏谑的 jokingly Turingismus)[69]的技术,用于对抗德国新的盖海姆施雷伯(秘密作家)机器产生的洛伦兹密码信息。这是布莱切利公园的电传打字机转子密码附件,代号为Tunny。Turingery是一种车轮断裂的方法,即计算Tunny车轮凸轮设置的程序。[70]他还将通尼团队介绍给汤米·弗劳尔斯,汤米·弗劳尔斯在马克斯·纽曼的指导下,继续建造巨像计算机,这是世界上第一台可编程数字电子计算机,它取代了更简单的先前机器(希斯·罗宾逊,Heath Robinson),其优越的速度使统计解密技术能够有效地应用于信息。[98]有人错误地说图灵是巨像计算机设计中的关键人物。图灵格林和班布里斯穆斯的统计方法无疑为洛伦兹密码的密码分析提供了思路,[71][72]但他并没有直接参与“巨像”的发展。[73]
\subsubsection{2.6 黛利拉}
继他在美国贝尔实验室的工作之后,[74]图灵在电话系统中寻求语音电子加密的想法。在战争后期,他搬家是为了在汉斯洛普公园(Hanslope Park)为特勤局的无线电安全局工作。在公园里,他在工程师唐纳德·贝利(Donald Bayley)的帮助下进一步发展了他的电子学知识。他们一起设计和建造了一台代号为黛利拉(Delilah)的便携式安全语音通信机。[75]这台机器是为不同的应用而设计的,但它缺乏长距离无线电传输的能力。无论如何,黛利拉完成得太晚了,不能在战争中使用。虽然这个系统工作得很好,图灵通过加密和解密温斯顿·丘吉尔的一段录音向官员们展示了这个系统,但是黛利拉并没有被采用。[76]图灵还就SIGSALY的开发咨询了贝尔实验室,SIGSALY是一种在战争后期使用的安全语音系统。
\subsubsection{2.7 人工智能}
\begin{figure}[ht]
\centering
\includegraphics[width=8cm]{./figures/7860e2b99ef4d2bf.png}
\caption{牌匾,汉普顿大街78号} \label{fig_ALTL_5}
\end{figure}
1945年至1947年间,图灵住在伦敦汉普顿,[77]在国家物理实验室从事自动计算引擎的设计。1946年2月19日,他发表了一篇论文,这是第一个详细设计的存储程序计算机。[78]冯·诺依曼的不完整的教育真空科学报告初稿早于图灵的论文,但它不太详细,根据不良贷款数学部门主管约翰·沃默斯利(John R. Womersley)的说法,它“包含了许多图灵博士自己的想法”。[79]尽管ACE是一个可行的设计,但是布莱切利公园战时工作的秘密导致了项目启动的延误,他也醒悟了。1947年末,他回到剑桥休假一年,在此期间,他创作了一部关于智能机械的开创性著作,这本书在他有生之年没有出版过。[80]当他在剑桥的时候,Pilot ACE正是在他不在的时候建造。它在1950年5月10日执行了它的第一个程序,后来世界各地的许多计算机都归功于它,包括英国的电动杜斯(Electric DEUCE)和美国的本迪克斯(Bendix G-15)。图灵的完整版本ACE直到他死后才被制造出来。[81]
\subsubsection{2.8 图灵测试}
根据杜塞尔多夫根舍(Genscher, Düsseldorf)出版的马克斯·普朗克物理研究所(MPIP)德国计算机先驱海因茨·比尔(Heinz Billing)的回忆录,图灵和康拉德·楚泽之间有一次会面。[82]它于1947年在哥廷根举行。审讯采取座谈会的形式。参与者包括沃默斯利、图灵、来自英国的波特和一些德国研究人员,如苏泽、沃尔特和比尔。

1948年,图灵被任命为曼彻斯特维多利亚大学( Victoria University of Manchester)数学系的读者。一年后,他成为了计算机实验室的副主任,在那里他为最早的存储程序计算机之一——曼彻斯特马克1号——开发软件。图灵为这台机器编写了第一版《程序员手册》,并被费朗蒂(Ferranti)聘请为他们商业化机器费朗蒂马克1的开发顾问。费朗蒂一直向他支付咨询费,直到他去世。[83]在此期间,他继续在数学方面做更抽象的工作,在《计算机械与智能》(Mind,1950年10月)中,图灵提出了人工智能的问题,并提出了一个后来被称为图灵测试的实验,试图定义一个被称为“智能”的机器标准。这个想法是,如果一个人类询问者不能通过对话将计算机与人类区分开来,那么计算机可以被称为“思考”。[84]图灵在论文中建议,与其建立一个模拟成人大脑的程序,不如制作一个更简单的程序来模拟儿童的大脑,然后让其接受教育。图灵测试的反向形式在互联网上被广泛使用;验证码测试旨在确定用户是人还是计算机。

1948年,图灵和他以前的大学同事钱博诺尼(D.G. Champernowne)一起,开始为一台尚不存在的计算机编写象棋程序。到1950年,该计划完成,并被称为涡轮增压器。[85]1952年,他试图在费朗蒂马克1上实现它,但由于缺乏足够的电力,计算机无法执行该程序。相反,图灵通过翻阅算法页面并在棋盘上执行指令来“运行”程序,每次移动大约需要半个小时。比赛被记录下来了。[86]根据加里·卡斯帕罗夫(Garry Kasparov)的说法,图灵的程序“玩了一个可识别的象棋游戏”[87] 这个程序输给了图灵的同事阿利克·格伦尼,尽管据说它赢了一场与尚珀诺尼的妻子伊莎贝尔(Isabel)的比赛。[88]

他的图灵测试是对人工智能争论的一个重要的、典型的挑衅性的、持久的贡献,这场争论持续了半个多世纪。[89]
\subsubsection{2.9 数理生物学}
