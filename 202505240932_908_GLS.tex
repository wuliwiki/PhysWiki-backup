% 格罗斯–皮塔耶夫斯基方程(综述)
% license CCBYSA3
% type Wiki

本文根据 CC-BY-SA 协议转载翻译自维基百科\href{https://en.wikipedia.org/wiki/Gross\%E2\%80\%93Pitaevskii_equation}{相关文章}。

格罗斯–皮塔耶夫斯基方程(Gross–Pitaevskii equation,简称 GPE,以尤金·P·格罗斯和列夫·彼得罗维奇·皮塔耶夫斯基命名)在哈特里–福克近似与伪势相互作用模型的基础上,用来描述由相同玻色子组成的量子体系的基态。

玻色–爱因斯坦凝聚是指一类玻色子气体,其所有粒子处于同一个量子态中,因此可以由相同的波函数描述。一个自由量子粒子可由单粒子薛定谔方程描述。对于现实气体中粒子间的相互作用,则需使用多体薛定谔方程加以考虑。在哈特里–福克近似中,整个由 $N$ 个玻色子组成的体系的总波函数 $\Psi$ 被认为是单粒子波函数 $\psi$ 的乘积:
$$
\Psi(\mathbf{r}_1, \mathbf{r}_2, \dots, \mathbf{r}_N) = \psi(\mathbf{r}_1)\psi(\mathbf{r}_2) \dots \psi(\mathbf{r}_N)~
$$
其中,$\mathbf{r}_i$ 是第 $i$ 个玻色子的坐标。如果气体中粒子之间的平均间距大于散射长度(即处于所谓的稀薄极限),那么可以用伪势来近似表示方程中真实的相互作用势能。在足够低的温度下,德布罗意波长远大于玻色子之间相互作用的作用范围,[3] 此时的散射过程可以很好地用 s 波散射(即偏波分析中的 $\ell = 0$,也称为硬球势)来近似描述。在这种情况下,该体系的伪势模型哈密顿量可以写为:
$$
H = \sum_{i=1}^{N} \left( -\frac{\hbar^2}{2m} \frac{\partial^2}{\partial \mathbf{r}_i^2} + V(\mathbf{r}_i) \right) + \sum_{i<j} \frac{4\pi \hbar^2 a_s}{m} \delta(\mathbf{r}_i - \mathbf{r}_j)~
$$

其中,$m$ 是玻色子的质量,$V$ 是外部势,$a_s$ 是玻色子之间的 s 波散射长度,$\delta(\mathbf{r})$ 是狄拉克 δ 函数。

变分法表明,如果单粒子波函数满足以下格罗斯–皮塔耶夫斯基方程:
$$
\left(-\frac{\hbar^2}{2m} \frac{\partial^2}{\partial \mathbf{r}^2} + V(\mathbf{r}) + \frac{4\pi \hbar^2 a_s}{m} |\psi(\mathbf{r})|^2 \right) \psi(\mathbf{r}) = \mu \psi(\mathbf{r})~
$$
那么总波函数在满足归一化条件$\int dV\,|\Psi|^2 = N$的前提下,可以最小化该模型哈密顿量的期望值。因此,这样的单粒子波函数描述了体系的基态。

格罗斯–皮塔耶夫斯基方程(GPE)是描述玻色–爱因斯坦凝聚态中基态单粒子波函数的模型方程。它在形式上类似于金兹堡–朗道方程,有时也被称为非线性薛定谔方程。

格罗斯–皮塔耶夫斯基方程的非线性性来源于粒子之间的相互作用:若将该方程中的相互作用耦合常数设为零(见下一节),便可还原为描述陷阱势中单个粒子的线性薛定谔方程。

通常认为格罗斯–皮塔耶夫斯基方程适用于弱相互作用区域。尽管如此,即使在该区域内,它有时也无法再现某些重要的物理现象。[4][5] 若要研究超出弱相互作用极限的玻色–爱因斯坦凝聚体,则需要引入李–黄–杨修正项。[6][7] 或者,在一维系统中,可以使用精确方法,例如 Lieb–Liniger 模型,[8] 或使用扩展形式的方程,例如 Lieb–Liniger–Gross–Pitaevskii 方程,[9](有时也被称为修正的[10]或广义的非线性薛定谔方程[11])。
\subsection{方程形式}
该方程具有薛定谔方程的形式,但多了一个相互作用项。耦合常数 $g$ 与两个相互作用玻色子的 s 波散射长度 $a_s$ 成正比:
$$
g = \frac{4\pi \hbar^2 a_s}{m}~
$$
其中 $\hbar$ 是约化普朗克常数,$m$ 是玻色子的质量。

能量密度为:
$$
\mathcal{E} = \frac{\hbar^2}{2m} |\nabla \Psi(\mathbf{r})|^2 + V(\mathbf{r}) |\Psi(\mathbf{r})|^2 + \frac{1}{2} g |\Psi(\mathbf{r})|^4~
$$
其中 $\Psi$ 是波函数(或称序参量),$V$ 是外部势(例如谐振势)。对于粒子数守恒的系统,其定态格罗斯–皮塔耶夫斯基方程为:
$$
\mu \Psi(\mathbf{r}) = \left( -\frac{\hbar^2}{2m} \nabla^2 + V(\mathbf{r}) + g |\Psi(\mathbf{r})|^2 \right) \Psi(\mathbf{r})~
$$
其中 $\mu$ 是化学势,它由以下条件确定,即粒子数 $N$ 与波函数之间的关系为:
$$
N = \int |\Psi(\mathbf{r})|^2 \, d^3 r~
$$
通过定态格罗斯–皮塔耶夫斯基方程,我们可以求解玻色–爱因斯坦凝聚体在不同外部势(例如谐振势)下的结构。

时间依赖的格罗斯–皮塔耶夫斯基方程为:
$$
i\hbar \frac{\partial \Psi(\mathbf{r}, t)}{\partial t} = \left( -\frac{\hbar^2}{2m} \nabla^2 + V(\mathbf{r}) + g |\Psi(\mathbf{r}, t)|^2 \right) \Psi(\mathbf{r}, t).~
$$
通过这个方程,我们可以研究玻色–爱因斯坦凝聚体的动力学行为。它常被用来分析陷阱中的气体的集体激发模式。
\subsection{解的情况}
由于格罗斯–皮塔耶夫斯基方程是一个非线性偏微分方程,因此其**精确解很难求得**。因此,通常需要通过各种近似方法来求解。
\subsubsection{精确解}
\textbf{自由粒子情形}

最简单的精确解是自由粒子解,对应于外势为零,即 $V(\mathbf{r}) = 0$:
$$
\Psi(\mathbf{r}) = \sqrt{\frac{N}{V}} e^{i\mathbf{k} \cdot \mathbf{r}}.~
$$
这个解通常被称为**哈特里解(Hartree solution)**。虽然它确实满足格罗斯–皮塔耶夫斯基方程,但由于粒子间的相互作用,它会在能谱中留下一个能隙:
$$
E(\mathbf{k}) = N\left[ \frac{\hbar^2 k^2}{2m} + g \frac{N}{2V} \right].~
$$
根据Hugenholtz–Pines 定理[12],在存在排斥相互作用的情况下,相互作用的玻色气体不应出现能隙。
