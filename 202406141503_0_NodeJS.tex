% Node.JS 笔记
% license Xiao
% type Note

\pentry{JavaScript 入门笔记\nref{nod_JS}}{nod_3689}

\subsection{安装}
\subsubsection{Ubuntu 安装}
\begin{itemize}
\item \href{https://nodejs.org/en/download/package-manager/}{下载安装包}
\item Linux 用 \href{https://github.com/nodesource/distributions/blob/master/README.md#installation-instructions}{Node source}
\item 安装完成以后, 而已用 \verb`node -v` 和 \verb`npm -v` 查看版本。 目前是 v16.5.0 和 7.19.1
\end{itemize}
\begin{lstlisting}[language=bash]
# Using Ubuntu
curl -fsSL https://deb.nodesource.com/setup_21.x | sudo -E bash - &&\
sudo apt-get install -y nodejs

# Using Debian, as root
curl -fsSL https://deb.nodesource.com/setup_21.x | bash - &&\
apt-get install -y nodejs
\end{lstlisting}
如果有依赖问题,可以把上面的 \verb`21.x` 改成 \verb`20.x` 或更低。

在 Ubuntu 18.04 上以上方法全部失败,因为 GLibC 版本过低,node 11.15.0 以上都用不了。 而 node 官网下载的 11.15.0 二进制包(如 node-v11.15.0-linux-x64.tar.xz)的 \verb`npm -v` 又提示 \verb`ERROR: npm is known not to run on Node.js v11.15.0`。 Ubuntu 很难支持多版本 GLibC 或者将其升级。 解决办法只有更新 Ubuntu 或用 docker 了。

\href{https://docs.npmjs.com/downloading-and-installing-node-js-and-npm}{npm 的下载页面}强烈建议使用 \verb`nvm`(Node Version Manager)来管理 nodejs 和 npm 的版本(可以随时切换不同版本)。

\subsubsection{Windows 安装}
\begin{itemize}
\item 可以直接下载 prebuilt binary (无需安装),也可以下载 msi \href{https://nodejs.org/en/download/prebuilt-installer}{安装包}。
\item 运行以后会显示以下命令行窗口,里面可以直接输入 js 命令,用于互动测试一些语法之类的。 你可以把整个 js 文件中的所有命令都粘贴进去运行。
\begin{figure}[ht]
\centering
\includegraphics[width=6cm]{./figures/f404aaf7652202e5.png}
\caption{Node.js REPL (Read-Eval-Print Loop)} \label{fig_NodeJS_1}
\item 如果要运行一个现成的 \verb`my.js` 文件,除了粘贴进去也可以在 Cmd 或者 PowerShell 中运行 \verb`node my.js`
\item Cmd 或 PowerShell 也可以运行 \verb`npm --version` 查看 npm 是否成功安装。
\end{figure}
\end{itemize}

\subsubsection{macOS 安装}
\begin{itemize}
\item 安装包下载安装后,重启命令行,直接打 \verb`node --version` 和 \verb`npm --version` 即可。
\end{itemize}

\subsection{node}
\begin{itemize}
\item node 可以直接把 js 程序像 python 和 bash 脚本一样运行:
\begin{lstlisting}[language=js,caption=test.js]
#!/usr/bin/env node
console.log("Hello, world!");
\end{lstlisting}
\end{itemize}
然后直接用 \verb`./test.js` 即可。 这里的 \verb`console.log` 输出到 \verb`stdout`。 或者直接 \verb`node test.js`(会自动忽略第一行)。

\subsection{npm}
\begin{itemize}
\item 参考\href{https://medium.com/@adnanrahic/hello-world-app-with-node-js-and-express-c1eb7cfa8a30}{这篇文章}。
\item \textbf{npm} 是 node package manager,相当于 python 的 pip。 用 npm 安装的包可以在 node 运行的脚本中使用。
\item 创建一个项目文件夹, 在该文件夹中打开 terminal, 初始化: \verb|npm init|。 这个命令的唯一作用就是生成 \verb`package.json` 文件, 你也可以自己写这个文件。
\item 按照 \verb|npm init| 提示填写, \verb|entry point| 填 \verb`app.js`. 不想填可以按回车跳过。 输入的内容都会在 \verb|package.json| 中,以后想改可以随时改。
\begin{lstlisting}[language=none,caption=package.json 示例]
{
  "name": "node",
  "version": "1.0.0",
  "description": "",
  "main": "index.js",
  "scripts": {
    "test": "echo \"Error: no test specified\" && exit 1"
  },
  "author": "addis",
  "license": "ISC",
  "dependencies": {
    "express": "^4.18.2"
  }
}
\end{lstlisting}
\item \verb|express| 包是一个 minimalist web framework,一般用于本地 Web Server 对程序进行调试。(如果直接双击 html 文件,浏览器的 url 中使用的是 file 协议而不是 http,会遇到许多问题。
\item 安装 express 包: \verb|npm install express --save|, 其中 \verb|--save| 会把安装的包保存到 \verb|package.json| 中的依赖列表。 另外会生成 \verb|package-lock.json| (指定所有依赖包的版本) 以及程序安装目录 \verb|node_modules|。
\item 过一段时间再把上面流程做一次, 会发现 \verb|package-lock.json| 中的版本号改变了, 这是因为网上的包是不断更新版本的, 这就是该文件的作用。
\item 创建程序文件 \verb|app.js|, 就是刚才输入的 entry point
\begin{lstlisting}[language=js]
var express = require('express');
var app = express();
app.get('/', function(req, res) {
  res.send('Hello World!');
});
app.listen(3000, function() {
  console.log('Example app listening on port 3000!');
});
\end{lstlisting}
\item 运行 \verb|node app.js|
\item 成功的话, 打开浏览器输入 \href{http://localhost:3000/}{http://localhost:3000/} 就会显示 \verb|Hello World!|。
\end{itemize}

\subsection{包管理}

上面 \verb`app.js` 中的 \verb`require()` 函数是 node 特有的,前端不能用。
node 查找包的顺序是:
\begin{enumerate}
\item 内建包, 如 \verb`fs, http`
\item 当前目录的 \verb`package.json` 中的 \verb`main`
\item 当前目录的 \verb`index.js, index.json, index.node`
\item 当前目录的 \verb`node_modules` 文件夹
\item 上级目录的 \verb`node_modules` 文件夹,直到根目录
\end{enumerate}

\begin{itemize}
\item 
\end{itemize}
