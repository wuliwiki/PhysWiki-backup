% 深度学习(综述)
% license CCBYSA3
% type Wiki

本文根据 CC-BY-SA 协议转载翻译自维基百科\href{https://en.wikipedia.org/wiki/Deep_learning}{相关文章}。

\begin{figure}[ht]
\centering
\includegraphics[width=8cm]{./figures/b06d12296b7d8e2d.png}
\caption{在深度学习中以多层抽象表示图像[1]} \label{fig_SDXX_1}
\end{figure}
深度学习是机器学习的一个子领域,专注于利用神经网络执行分类、回归和表示学习等任务。该领域受到生物神经科学的启发,核心在于将人工神经元堆叠成多层,并通过“训练”使其能够处理数据。“深度”一词指的是网络中使用了多层结构,这些层数从三层到几百甚至几千层不等。深度学习的方法可以是监督学习、半监督学习或无监督学习。[2]

常见的深度学习网络架构包括全连接网络(Fully Connected Networks)、深度信念网络(Deep Belief Networks)、循环神经网络(Recurrent Neural Networks)、卷积神经网络(Convolutional Neural Networks)、生成对抗网络(Generative Adversarial Networks)、Transformer以及神经辐射场(Neural Radiance Fields)。这些架构已应用于计算机视觉、语音识别、自然语言处理、机器翻译、生物信息学、药物设计、医学图像分析、气候科学、材料检测以及棋盘游戏程序等领域,并在许多情况下取得了与人类专家相当甚至超越的表现。[3][4][5]

早期形式的神经网络受到生物系统中信息处理和分布式通信节点的启发,特别是人脑。然而,当前的神经网络并不旨在模拟生物体的大脑功能,在这一方面通常被认为是低质量的模型。[6]