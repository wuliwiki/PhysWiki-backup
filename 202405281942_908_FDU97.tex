% 复旦大学 1997 量子真题
% license Usr
% type Note

\textbf{声明}:“该内容来源于网络公开资料,不保证真实性,如有侵权请联系管理员”

\begin{enumerate}
    \item设体系处于 $\psi = C_1 \psi_1 + C_2 \psi_2$ 态 ($\psi_1$ 和 $\psi_2$ 正交归一化,即 $\langle \psi_1 | \psi_1 \rangle = \langle \psi_2 | \psi_2 \rangle = 1$ ),求
    \begin{itemize}
    \item a) $|C_1|^2$ 的可能测值及相应几率;(5分)
    \item b) $|C_2|^2$ 的可能测值及相应几率;(5分)
    \item c) $|C_1|^2 + |C_2|^2$ 的可能测值及相应几率。(10分)
   \end{itemize}
 
    (2)对于氢原子基态,计算 $\Delta X \cdot \Delta P_x$。(20分)\\
    (3)在 $\psi$ 态中,求 $Y$ 的本征态。(20分)\\
    (4)电子自旋计算。略去原子核,假定有如下 $\vec{B} = B_0 \vec{e}_z$,在 $t=0$ 时刻,电子自旋向上,并附加弱磁场 $\vec{B}_{ex} = B_1 \sin \omega t \vec{e}_x$。试用一级含时微扰证明对于:小 $\lambdabar$,到自旋朝下态的跃迁发生在 $\omega = \omega_0$ 处。略去非谐振项,计算 $\omega = \omega_0$,处到自旋向下态的跃迁几率(20分) 
     \end{enumerate}
     
    (5)粒子在Yukawa 势阱 $V(r) = -V_0 \frac{e^{-r/a}}{(r/a)}, \left( V_0 > 0, a > 0 \right)$
    中运动,用径向波函数 $R(r) = e^{-\beta r}$,求基态能量。$\beta$ 是待定参数。(20分)(写出 $\beta$ 满足的方程)
    
    附带用式:
$\Delta L_y \psi_{lm} = \hbar \sqrt{l(l+1)-m(m\pm 1)} Y_{lm \pm 1}$\\
$\text{球坐标 } (r, \theta, \varphi):$
$\nabla^2 = \frac{1}{r^2} \frac{\partial}{\partial r} \left( r^2 \frac{\partial}{\partial r} \right) + \frac{1}{r^2 \sin \theta} \frac{\partial}{\partial \theta} \left( \sin \theta \frac{\partial}{\partial \theta} \right) + \frac{1}{r^2 \sin^2 \theta} \frac{\partial^2}{\partial \varphi^2}$



