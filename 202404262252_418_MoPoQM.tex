% 双原子分子莫尔斯势(量子力学)
% keys 莫尔斯势|摩尔斯势
% license Usr
% type Tutor

\pentry{薛定谔方程(单粒子多维)\nref{nod_QMndim},Kummer 函数(1F1)\nref{nod_Kummer}}{nod_67ff}

双原子分子中两原子的相互作用可以理解为是一个弹簧两端连接着两个振子。考虑设两原子在某时刻相距 $r$,存在某 $r_0$ 称为平衡核间距,当 $r>r_0$ 时他们之间互相吸引、$r<r_0$ 时他们之间互相排斥。这将使得原子间距在 $r_0$ 附近振荡。

\subsection{莫尔斯势}
莫尔斯势(Morse Potential,又写作摩尔斯势)表示为:
\begin{equation}
V(r) = D [\exp(-2ax) - 2\exp(-ax)] \ , \left(x \equiv \frac{r-r_0}{r_0}\right)~.
\end{equation}
参数 $D$ 与 $a$ 一般都是正数,$D$ 一般有能量量纲、$a$ 一般是无量纲数。莫尔斯势在 $r=r_0$ 时有极小值 $V(r)=-D$。

与林纳德-琼斯势相比较,莫尔斯势在 $r = 0$ 即 $x = -1$ 时是有限值 $V(0) = D(e^{2a}-2e^{a})$。而在 $r \rightarrow +\infty$ 时,$V(r) \rightarrow 0$。

考虑几个经典双原子分子,用 $\widetilde{M} = m_1m_2/(m_1+m_2)$ 描述约化质量,$\Si{cm}^{-1}$ 量纲描述能量,
\begin{table}[ht]
\centering
\caption{经典分子参数}\label{tab_MoPoQM1}
\begin{tabular}{|c|c|c|c|}
\hline
分子 & $\frac{\hbar^2}{2 \widetilde{M} r_0^2}/\Si{cm}^{-1}$ & $D/\Si{cm}^{-1}$ & $a$ \\
\hline
$\text{H}_2$ & 60.8296 & 38292 & 1.440 \\
\hline
$\text{I}_2$ & 0.0374 & 12550 & 4.954 \\
\hline
$\text{HCl}$ & 10.5930 & 37244 & 2.380 \\
\hline
\end{tabular}
\end{table}
$\Si{cm}^{-1}$ 单位对应在这波数(波长的倒数)下光子能量。转换到 $\Si{eV}$ 是:
$$E(\Si{eV}) = E(\Si{cm}^{-1}) \times 1.2398 \times 10^{-4} ~.$$

\subsection{简谐近似}
模仿讨论林纳德-琼斯势中讨论简谐近似(\autoref{sub_LenJoP_1}~\upref{LenJoP})的方法。可以得到:
\begin{equation}
\begin{aligned}
\epsilon = \eval{v}_{r = r_0}& = -D ,\\
k = \eval{\dv{^2 V(r)}{r^2}}_{r = r_0}& = \frac{2a^2 D}{r_0^2} ~.
\end{aligned}
\end{equation}
从而 $V(r) = a^2 D[(r-r_0)/r_0]^2 - D$。而 
$$\omega^2 = \frac{k}{\widetilde M} = \frac{2a^2D}{\widetilde Mr_0^2} ~.$$

\subsection{简谐近似势的量子化}
振动能级为
$$E_\nu = \hbar \omega (\nu + \frac12) - D , \ (\nu = 0, 1, 2, \cdots) ~.$$

\subsection{精确解}
体系波函数应满足三维薛定谔方程的描述,考虑径向波函数 $u(r)$ 应满足
\begin{equation}
-\frac{\hbar^2}{2 \widetilde M} \dv{^2 u}{r^2} + V(r) u = Eu ~,
\end{equation}
讨论束缚态的情况($E<0$),$u$ 随 $x$ 的变化为
\begin{equation}
\dv{^2 u}{x^2} + [-\beta^2 + 2 \gamma^2 \exp(-ax) - \gamma^2 \exp(-2ax)]u = 0 ~,
\end{equation}
其中 $\beta^2 = -2\widetilde M r_0^2 E/(\hbar^2)$,$\gamma^2 = 2\widetilde M r_0^2 D/(\hbar^2)$。引入 $\xi = \eta \exp(-ax)$,$\eta = 2 \gamma/a$。则方程又可化为
\begin{equation}
\xi^2 \dv{^2 u}{\xi^2} + \xi \dv{u}{\xi} + (-(\beta/a)^2 + \xi\eta/2 - \xi^2/4)u = 0 ~~
\end{equation}

这式存在两奇点 $\xi = 0$ 与 $\xi \rightarrow \infty$。在 $\xi = 0$ 时方程有渐进形式
\begin{equation}
\xi \dv{u}{\xi} - (\beta/a)^2 u = 0 ~,
\end{equation}
从而 $u$ 正比于 $\xi^{(\beta/a)^2}$。

类似的,在 $\xi \rightarrow \infty$ 时,渐进形式是
\begin{equation}
\dv{^2 u}{\xi^2} - u/4 = 0 ~.
\end{equation}
其通解为 $u \approx \exp(\xi/2) + \exp(-\xi/2)$。从而可以将 $u$ 表示为
\begin{equation}
u(\xi) = \xi^{(\beta/a)^2} \exp(-\xi/2) y(\xi) ~.
\end{equation}
代回就可以得到 $y(\xi)$ 满足 Kummer 微分方程\autoref{eq_Kummer_1}~\upref{Kummer}
\begin{equation}
\xi \dv{^2 y}{\xi^2} + [(2(\beta/a) + 1) - \xi] \dv{y}{\xi} - \frac{(2(\beta/a) + 1)-\eta}{2}y=0 ~.
\end{equation}
特别的,当 $\xi = 0$ 即 $r \rightarrow \infty$ 时 $u(0) = 0$,为满足条件,用合流超几何函数可以将解表示为
\begin{equation}
u(\xi) = A \xi^{(\beta/a)^2} \exp(-\xi/2) {}_1F_1\left(\frac{(2 (\beta/a) + 1) - \eta}{2}, 2(\beta/a)+1;\xi\right)~.
\end{equation}
当取 $\frac{(2 (\beta/a) + 1) - \eta}{2} = -\nu, \ (\nu = 0, 1, 2, \cdots)$ 时,合流超几何函数变为 $\nu$ 次多项式,故体系有本征解
\begin{equation}
u_\nu(x) = C_\nu \exp\left[- (\beta x + \eta e^{-ax}/2)\right] {}_1F_1 \left(-\nu, 2(\beta/a)+1; \eta e^{-ax}\right)~.
\end{equation}
其中 $C_\nu = A \eta^\nu$ 是归一化系数,$\nu$ 是振动量子数。
\subsubsection{体系的本征能量}
接下来考虑体系的本征能量。$\nu$ 收到束缚态中的限制,体系能量小于 $0$ 但不低于势能最小值:$-D < E < 0$,即 $0 < \beta < \gamma$。这使得有限制:
\begin{equation}
\frac{1-\eta}2 < \frac{(2 (\beta/a) + 1) - \eta}2 < \frac12, 1 < 2 (\beta/a) + 1 <  1 + \eta ~.
\end{equation}
从而 $\nu$ 有限制 $\nu < V_m = \frac{1}{2} (\eta-1)$。而体系的能量可以表示为
\begin{equation}
E_\nu = -D + \frac{\hbar^2}{2 \widetilde M r_0^2} \left[2a\gamma\left(\nu+\frac12\right) - a^2 \left(\nu + \frac12\right)^2\right] ~,
\end{equation}
即
\begin{equation}
E_\nu = -D + \hbar \omega \left[\left(\nu+\frac12\right) - \frac1\eta \left(\nu + \frac12\right)^2\right], \ \left(\nu < \frac12 (\eta - 1)\right) ~.
\end{equation}

不难发现能级间隔 $\Delta E_\nu = E_\nu - E_{\nu - 1} = \hbar \omega (1 - 2 \nu / \eta)$,  $(\nu = 1, 2, 3, \cdots)$。
