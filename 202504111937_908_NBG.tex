% 冯·诺伊曼-博内斯-哥德尔集合论(综述)
% license CCBYSA3
% type Wiki

本文根据 CC-BY-SA 协议转载翻译自维基百科\href{https://en.wikipedia.org/wiki/Von_Neumann\%E2\%80\%93Bernays\%E2\%80\%93G\%C3\%B6del_set_theory}{相关文章}。

在数学基础中,冯·诺依曼–伯奈斯–哥德尔集合论(NBG)是一种公理化集合论,是泽梅洛–弗兰克尔–选择公理集合论(ZFC)的保守扩展。NBG 引入了“类”的概念,类是由公式定义的集合,其量词仅对集合进行量化。NBG 可以定义比集合更大的类,例如所有集合的类和所有序数的类。摩尔斯–凯利集合论(MK)允许通过量词对类进行量化的公式来定义类。NBG 是有限公理化的,而 ZFC 和 MK 则不是。

NBG 的一个关键定理是类存在定理,它声明,对于每个量词仅对集合进行量化的公式,都存在一个类,该类包含满足该公式的集合。这个类是通过用类逐步构造公式来构建的。由于所有集合论公式都是由两种原子公式(成员关系和相等性)和有限多的逻辑符号构成,因此只需要有限多的公理来构建满足这些公式的类。这就是为什么 NBG 是有限公理化的原因。类还用于其他构造、处理集合论悖论,并用于表述全局选择公理,该公理比 ZFC 的选择公理要强。

约翰·冯·诺依曼在 1925 年将类引入集合论。他的理论的原始概念是函数和参数。利用这些概念,他定义了类和集合。\(^\text{[1]}\)保罗·伯奈斯通过将类和集合作为原始概念重新表述了冯·诺依曼的理论。\(^\text{[2]}\)库尔特·哥德尔简化了伯奈斯的理论,用于他对选择公理和广义连续统假设相对一致性的证明。\(^\text{[3]}\)