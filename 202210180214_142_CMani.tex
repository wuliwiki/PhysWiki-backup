% 复流形
% keys 复流形|流形|微分几何

\pentry{流形\upref{Manif}}

\subsection{复流形}

将光滑流形\autoref{Manif_def3}~\upref{Manif}定义中的 $\mathbb{R}^n$ 替换为 $\mathbb{C}^n$,图册中的“光滑映射”替换为“全纯映射”(复解析映射), 我们就得到复流形的定义.

\addTODO{$\mathbb{C}^n \to \mathbb{C}^m$ 的全纯函数}

\begin{definition}{复图和复图册}
$N$ 是一个 $n$ 维拓扑流形,如果存在开集 $U \in \mathcal{T}_N$ 和拓扑同胚映射 $\varphi: U \rightarrow \tilde{U} \subseteq \mathbb{C}^n$,其中 $\tilde{U}$ 是 $\mathbb{C}^n$ 的一个开子集,那么称 $(U,\varphi)$ 是 $N$ 上的一张\textbf{复图}.如果图的一个集合 $\mathcal{A}=\{(U_\alpha, \varphi_\alpha)\}$ 覆盖了 $N$,即 $\bigcup\{U_\alpha\}=N$,那么称这个集合 $\mathcal{A}$ 是一个\textbf{复图册}.
\end{definition}

特别需要注意由于 $\mathbb{C}$ 和它的真开子集(比如开球)不是全纯同胚的(这和实数的情况不相同),不是所有的坐标图 $\varphi: U \rightarrow \tilde{U} \subseteq \mathbb{C}^n$ 都能改写成 $\varphi': U' \rightarrow \mathbb{C}^n$ 的形式的.实际上,$\mathbb{C}$无法嵌入到复环面(complex tori) $\mathbb{C} / \Lambda$ 中.



\begin{definition}{全纯相容}
考虑一个拓扑流形 $N$ 的两个复图 $(U, \varphi)$ 和 $(V, \phi)$.如果 $U \cap V \neq \varnothing$,且 $\varphi \circ \phi^{-1}: \phi(V) \rightarrow \varphi(U)$ 和 $\phi \circ \varphi^{-1}: \varphi(U) \rightarrow \phi(V)$ 都是全纯(复解析)映射,那么我们称这两个图是\textbf{全纯相容的(compatible)}.
\end{definition}

\begin{definition}{复(全纯)流形}\label{CMani_def1}
一个拓扑流形 $N$ 和加上其上一组全纯相容的复图册 $\mathcal{A}$,被称为一个\textbf{复流形(complex (holomorphic) manifold)} $(N, \mathcal{A})$.
\end{definition}

\begin{theorem}{}
复流形都是可定向实流形.
\end{theorem}

\addTODO{证明}

\begin{definition}{全纯映射(复流形)}
给定复流形 $M, N$, 一个映射 $f: M \to N$ 在点 $p \in M$ 处\textbf{全纯}(holomorphic 或称\textbf{解析} analytic),如果存在 点$p$附近的坐标图 $\phi: U \ni p \to \tilde{U} \subseteq \mathbb{C}^m$,和点 $f(p)$ 附近的坐标图 $\psi: V \ni f(p) \to \tilde{V} \subseteq \mathbb{C}^n$,使得函数
$$
\psi \circ f \circ \phi^{-1}: \tilde{U} \subseteq \mathbb{C}^m \to \tilde{V} \subseteq \mathbb{C}^n
$$
在 $\phi(x)$ 点处全纯,如图;

$f$ 被称作\textbf{全纯函数},如果它在任意点处都全纯.
\end{definition}

\addTODO{交换图}

\addTODO{切向量和余切向量}

\subsection{解析集}

\subsection{近复流形}

\begin{definition}{(近)复结构(向量空间)}
一个实向量空间 $V$ 上的一个\textbf{(近)复结构}是一个(实)线性映射 $j: V \to V$ 满足 $j \circ j = - \text{id}_V$.
\end{definition}

\begin{theorem}{}
一个实向量空间上存在近复结构,当切仅当它的维度是偶数.
\end{theorem}

\begin{definition}{近复结构(实流形)}
一个实流形 $M$ 上的一个\textbf{近复结构}是它的(实)切空间上的“近复结构场”,即一个 $(1, 1)$ 型张量场 $J \in \Gamma(M, T_1^1 M)$ 满足对任意一点 $p$, 
$$
J|_p: T_p(M) \to T_p(M)
$$
是一个向量空间上的近复结构.
\end{definition}

\addTODO{没有定义复流形的切空间没法往下写了}




