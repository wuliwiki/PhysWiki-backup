% 正交曲线坐标系
% 多元微积分|坐标系|柱坐标系|球坐标系|矢量|内积|内积|导数|偏导数|曲线坐标系|正交曲线坐标系

\begin{issues}
\issueTODO
\end{issues}

\pentry{柱坐标系\upref{Cylin}, 球坐标与直角坐标的转换\upref{SphCar}}

\footnote{本文参考 Wikipedia \href{https://en.wikipedia.org/wiki/Curvilinear_coordinates}{相关页面}.}如果 $u, v, w$ 是三维空间中某曲线坐标系的三个坐标, 空间任意一点的位置矢量\upref{Disp} $\bvec r$ 都是它们的函数 $\bvec r(u, v, w)$. 那么定义任意一点处三个单位矢量为
\begin{equation}\label{CurCor_eq8}
\uvec u = \frac{\pdv*{\bvec r}{u}}{\abs{\pdv*{\bvec r}{u}}}\qquad
\uvec v = \frac{\pdv*{\bvec r}{v}}{\abs{\pdv*{\bvec r}{v}}}\qquad
\uvec w = \frac{\pdv*{\bvec r}{w}}{\abs{\pdv*{\bvec r}{w}}}
\end{equation}
注意一般来说, 这三个矢量会随着 $\bvec r$ 改变. 形象地说: 当我们分别只把 $u, v, w$ 增加一点时, $\bvec r$ 会分别沿 $\uvec u, \uvec v, \uvec w$ 方向移动(请以球坐标系和柱坐标系为例思考).

若对空间中任意一点, \autoref{CurCor_eq8} 中的三个矢量都两两正交, 那么这个曲线坐标系就是\textbf{正交曲线坐标系(orthogonal curvilinear coordinate system)}. 常见的例子除了球坐标系\upref{Sph}, 柱坐标系\upref{Cylin} 还有抛物线坐标系\upref{ParaCr}和椭圆坐标系\upref{EliCor}. 直角坐标系也可以看作是正交曲线坐标系的一个特例.

\begin{example}{球坐标系}
以下我们用\autoref{CurCor_eq8} 重新推导球坐标系中的三个单位矢量(\autoref{SphCar_eq3}~\upref{SphCar}), 并证明它是一个正交曲线坐标系.

球坐标系中, 位置矢量可以表示为
\begin{equation}
\bvec r = r \uvec r = r\sin\theta\cos\phi\,\uvec x + r\sin\theta\sin\phi\,\uvec y + r\cos\theta\uvec z
\end{equation}
同样, 球坐标系的三个单位矢量由三个坐标增加的方向确定
\begin{equation}\label{CurCor_eq9}
\leftgroup{
\pdv{\bvec r}{r} &= \sin\theta\cos\phi\,\uvec x + \sin\theta\sin\phi\,\uvec y + \cos\theta\,\uvec z\\
\pdv{\bvec r}{\theta} &= r\cos\theta\cos\phi\,\uvec x + r\cos\theta\sin\phi\,\uvec y - r\sin\theta\,\uvec z\\
\pdv{\bvec r}{\phi} &= -r\sin\theta\sin\phi\,\uvec x + r\sin\theta\cos\phi\,\uvec y
}\end{equation}
归一化得三个单位矢量为
\begin{equation}\label{CurCor_eq10}
\begin{cases}
\uvec r = \sin\theta\cos\phi\,\uvec x + \sin\theta\sin\phi\,\uvec y + \cos\theta\,\uvec z\\
\uvec \theta = \cos\theta\cos\phi\,\uvec x + \cos\theta\sin\phi\,\uvec y - \sin\theta\,\uvec z\\
\uvec\phi = -\sin\phi\,\uvec x + \cos\phi\,\uvec y
\end{cases}
\end{equation}
不难验证对除原点外的任意坐标点, 这三个单位矢量两两间内积为零, 即两两垂直, 所以球坐标系属于正交曲线坐标系.
\end{example}

\subsection{矢量场}
举例: $r\uvec r$, $r\uvec \phi$ 等. 以及和直角坐标的转换.

\subsection{坐标线, 坐标面}
正如开头所说,若空间任意一点 $M$ 的坐标可用三个新的变量 $u,v,w$ 来表示的话,则 $(u,v,w)$ 称为点 $M$ 的\textbf{曲线坐标}.给定新变量以常数值 $A,B,C$,就得到三族\textbf{曲坐标面}.将 $\bvec r(u,v,w)$ 写成分量的形式
\begin{equation}
x=\varphi_1(u,v,w),\quad y=\psi_1(u,v,w),\quad z=\omega_1(u,v,w)
\end{equation}
或写成 $u,v,w$ 的形式
\begin{equation}
u=\varphi(x,y,z),\quad v=\psi(x,y,z),\quad w=\omega(x,y,z)
\end{equation}
则这些新的曲坐标面在坐标系 $x,y,z$ 中的方程为
\begin{equation}
\varphi(x,y,z)=A,\quad \psi(x,y,z)=B,\quad \omega(x,y,z)=C
\end{equation}

由不同族中取出任意两个曲坐标面,它们相交于一线
\begin{equation}
\psi(x,y,z)=B_0,\quad \omega(x,y,z)=C_0
\end{equation}
其中, $B_0,C_0$ 为确定的常数,这条线只有 $u$ 变化,这样的线叫作 \textbf{坐标线} $u$ .类似的方法还可得到坐标线 $v,w$.

我们知道,对于含参曲线 $\bvec r(t)$,$\dv{\bvec r}{t}$ 是其切线方向向量,而 $\dv{\bvec r}{t}/\abs{\dv{\bvec r}{t}}$ 是其切线单位向量.而对于曲坐标线 $u,v,w$ ,$u,v,w$ 分别是其参数,所以曲坐标线 $u,v,w$ 的单位向量便是\autoref{CurCor_eq8} 中的 $\uvec u,\uvec v,\uvec w$.所以这三个向量两两正交,便意味着三族曲坐标线的切线两两正交,所以称其为正交曲线坐标系.

三条坐标线的qi之间两两正交,这意味着其中任一条坐标线的法向量必在另两个曲线坐标构成的曲坐标面
\addTODO{坐标线上任意一点的法向量都是其余两个坐标, 坐标面上任意一点的法向量都是该坐标的法向量}

\subsection{线元, 面积元, 体积元}
未完成, 另见 “正交曲线坐标系中的重积分\upref{CrIntN}”.
