% 个体词与谓词(数理逻辑)
% keys 个体词|谓词
% license Usr
% type Tutor

\begin{definition}{个体词}
独立存在的个体称为\textbf{个体词},可以是\textbf{个体常量}或\textbf{个体变量}。
\begin{itemize}
\item \textbf{个体常量}表示特定的或具体的个体,是不变的,一般用带或不带下标的小写字母 $a, b, c, \dots, a_1, a_2, \dots$ 表示。
\item \textbf{个体变量}表示泛指的或抽象的、不定的个体,是变化的,一般用带或不带下标的小写字母 $x, y, z, \dots, x_1, x_2, \dots$ 表示。
\end{itemize}

\end{definition}
\begin{definition}{论域}
所有个体词的取值范围称为\textbf{论域}(或\textbf{取值域}),常用字母 $D$ 表示。
\end{definition}


\begin{definition}{谓词}
在原子命题中,可以独立存在的客体(句子中的主语、宾语等),成为个体词。而用以刻画客体的心智或者客体之间关系即为谓词。
\end{definition}
