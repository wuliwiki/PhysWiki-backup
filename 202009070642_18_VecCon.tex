% 联络 (向量丛)
\pentry{流形\upref{Manif}, 向量场, 向量丛}

本节采用爱因斯坦求和约定.

\subsection{定义}
设$M$是$n$维微分流形, $E$是其上秩为$k$的光滑向量丛. $E$上的一个\textbf{联络 (connection) }是指一个映射$D:\Gamma(E)\otimes TM\to\Gamma(E)$, 或者等价地$D:\Gamma(E)\to T^*M\otimes\Gamma(E)$, 满足如下条件:

\begin{enumerate}
\item 对于截面$\xi\in\Gamma(E)$和任何切向量场$X$, $X\to D_X\xi$是$C^\infty(M)$-线性的, $\xi\to D_X\xi$是实线性的, 即
$$
D_{fX_1+fX_2}\xi=fD_{X_1}\xi+gD_{X_2}\xi,\,f,g\in C^\infty(M).
$$
\item 对于截面$\xi\in\Gamma(E)$和任何切向量场$X$, $\xi\to D_X\xi$是实线性的, 即对于实数$c_1,c_2$有
$$
D_X(c_1\xi_1+c_2\xi_2)=c_1D_X\xi_1+c_2D_X\xi_2).
$$
\item 对于任何$f\in C^\infty(M)$, 都有莱布尼兹律:
$$
D_X(f\xi)=X(f)\xi+fD_X\xi.
$$
\end{enumerate}

直观上说, 在向量丛上给定联络, 就是给定一个"符合张量规律的导数运算". 

\subsection{联络形式}
设$D$是向量丛$E$上的联络. 设$\{e_i\}_{i=1}^n,\{\theta^i\}_{i=1}^n,\{s_\alpha\}_{\alpha=1}^k$分别是$TM,T^*M,\Gamma(E)$的局部光滑标架, 其中$\{e_i\}$和$\{\theta^i\}$是对偶标架. 在这些局部光滑标架之下, 按照联络的定义, 有
$$
D_X\xi=X_i\cdot\left[e_i(\xi^\alpha)s_\alpha+\xi^\alpha D_{e_i}s_\alpha\right].
$$
若命$D_{e_i}s_\alpha=\Gamma_{\alpha i}^\beta s_\beta$ (有时称系数$\Gamma^\alpha_{\beta i}$为克氏符, Christoffel symbol), 则有
$$
D_X\xi=X_i\cdot\left[e_i(\xi^\alpha)+\Gamma_{\beta i}^\alpha\xi^\beta \right]s_\alpha.
$$
于是1-形式的矩阵
$$
\omega=(\omega_\beta^\alpha)=(\Gamma_{\beta i}^\alpha\theta^i)
$$
is called the \emph{matrix of connection 1-forms}. Thus,
$$D\xi=(d\xi^\beta+\omega_\alpha^\beta\xi^\alpha)\otimes s_\beta.$$
If $\{s'_\beta\}$ is another local frame, such that $s_\beta=b_\beta^\alpha s_\alpha$, then
$$b_\alpha^\gamma{\omega'}_{\beta}^{\alpha}\otimes s_\gamma={\omega'}_{\beta}^{\alpha}s'_\alpha=Ds'_\beta=(db_\beta^\gamma+b_\beta^\alpha\omega_\alpha^\gamma)\otimes s_\gamma.$$
Hence $\omega'=db\cdot b^{-1}+b\cdot\omega\cdot b^{-1}$. This is the transformation formula of the connection 1-forms under the transformation of local frames. The following theorem follows directly from this formula:

\begin{theorem}{}
Given an open covering $\mathfrak{U}$ of $M$, such that on each $U\in\mathfrak{U}$ a local frame $\{e^U_i\}$ of $TM$ and a local frame $\{s^U_\alpha\}$ of $E$ are defined, and a $k\times k$ matrix $\omega^U$ of 1-forms is given. If for any $U,V\in\mathfrak{U}$ such that $U\cap V\neq0$, the following formula holds on $U\cap V$:
$$\omega^V=db^V_U\cdot \{b_V^U\}^{-1}+b^V_U\cdot\omega^U\cdot \{b^V_U\}^{-1},$$
where $b^V_U$ is the transformation matrix from $\{s^U_\alpha\}$ to $\{s^V_\alpha\}$, then there exists a unique connection on $E$, whose local connection 1-forms coincide with $\{\omega^U\}$.
\end{theorem}

Also, the following theorem on normal coordinates holds for any connection:

\begin{theorem}{}
Given any $p\in M$ and a connection $D$ on $E\to M$, there exists a local frame of $E$ around $p$ such that under this coordinate, $\omega(p)=0$.
\end{theorem}
\textbf{证明大意} Fix a frame $\{s_\alpha\}$, under which
$$\omega^\beta_\alpha=\Gamma_{\alpha i}^\beta dx^i.$$
Let $\{x^i\}$ be local coordinates around $p$ with $x^i(p)=0$. With
$$b_\alpha^\beta=\delta_\alpha^\beta-\Gamma_{\alpha i}^\beta x^i$$
the new frame $s'_\alpha=b_\alpha^\beta s_\beta$. The result follows from the transformation formula, since $db(p)=-\omega(p)$.

Finally, a connection $D$ on $E$ induces a connection (which is still denoted by $D$) on $E^*$ by
$$d\langle s^*,s\rangle=\langle Ds^*,s\rangle+\langle s^*,Ds\rangle.$$
If $\{s_\alpha\}$ is a local frame for $E$ and $\{s^{*\gamma}\}$ is the dual frame, then the connection 1-forms of $D$ on $E^*$ are
$$\omega^{*\beta}_\alpha=\langle Ds^{*\beta},s_\alpha\rangle=-\langle s^{*\beta},Ds_\alpha\rangle=-\omega^\beta_\alpha,$$
or $Ds^{*\beta}=-\omega^\beta_\alpha s^{*\alpha}$. If a section $s^*\in\Gamma(E^*)$ has local expression $s^*=\eta_\beta s^{*\beta}$, then
$$Ds^*=(d\eta_\beta-\omega_\beta^\alpha\eta_\alpha)\otimes s^{*\beta}.$$
If $D_1,D_2$ are connections on bundles $E_1,E_2$ respectively, then they induce connections on $E_1\oplus E_2$ and $E_1\otimes E_2$ by
$$D(s_1\oplus s_2)=Ds_1\oplus Ds_2,\,D(s_1\otimes s_2)=Ds_1\otimes s_2+s_1\otimes Ds_2.$$