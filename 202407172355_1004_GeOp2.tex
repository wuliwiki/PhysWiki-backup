% 几何光学基本定律
% license Usr
% type Tutor

\pentry{波面、光线和光束\nref{nod_Opt1}}{nod_ece1}

几何光学是以实验定律为基础发展出来的理论。历史上,人们通过实验观察光的传播路径,总结形成了多个实验定律,如光的直线传播定律、光的独立传播定律、折射定律和反射定律、费马原理与马吕斯定理。

\subsection{光的直线传播定律}

\textbf{在各向同性的均匀介质中,光沿直线传播,这就是光的直线传播定律。}在日常生活中,各种障碍物大小、各种孔径尺寸远远比光的波长大得多,衍射现象极不明显,可以忽略不计,可以简单应用光的直线传播来分析宏观光现象,如影子的形成等。

\subsection{光的独立传播定律}

\textbf{从不同光源发出的光线,以不同方向经过介质某点,各光线互不影响,这就是光的独立传播定律。}利用这条定律,可以让我们对光线传播规律的研究大大简化,即只需要关心某一研究对象光线的传播,而不考虑其他光线。注意,这条定律依然只用于分析宏观光现象,详见光的干涉相关内容。

\subsection{光的折射定律和反射定律}

\subsubsection{内容介绍}

光在传播过程中遇到两种不同介质构成的界面时,在界面上会部分反射,部分折射。光的反射定律和折射定律描述了反射与折射前后光传播方向的不同。

历史上,光的反射定律和折射定律首先由实验得到,后来,人们通过电磁学理论进行了严格推导,也得出了它们(详见波动光学部分)。

\begin{figure}[ht]
\centering
\includegraphics[width=6cm]{./figures/5c284730dbf3e287.png}
\caption{反射和折射定律} \label{fig_GeOp2_1}
\end{figure}

图1中,界面两侧介质折射率分别为 $n$ 和 $n'$。入射光线在界面入射点为 $O$,虚线为过 $O$ 点的法线,我们将入射光线和法线所在的平面称为入射面,则反射光线和折射光线均在入射面内(证明见波动光学内容,在此不证)。三条光线的传播方向可以使用它们与法线的夹角来描述。如图,我们将入射光线、折射光线和反射光线与法线夹角规定为 $I$、$I'$、$I''$。进一步地,将光线沿锐角转向法线,如果为顺时针,则夹角为正,反之,夹角为负。

此时,我们可以将折射定律表示为:

\begin{equation}
nsin(I)=n'sin(I') ~.
\end{equation}

将反射定律表示为:

\begin{equation}
I=-I'' ~.
\end{equation}

观察 $(1)$ 式,我们可以看出,令 $n'=-n$,即得 $I=-I''$,即 $(2)$ 式。这表明,反射定律是折射定律的特殊情况,可以使用统一的方法来研究折射光学现象与反射光学现象。

再观察 $(1)$ 式与 $(2)$ 式,我们完全可以将它们写成:

\begin{equation}
n'sin(I')=nsin(I) ~.
\end{equation}

\begin{equation}
-I''=I ~.
\end{equation}

这意味着,光线沿着折射光线出射的反方向入射经过界面折射后,沿原来入射光线的反方向出射,对于反射光线也是如此。这就是所谓的\textbf{光路可逆}。

\subsubsection{矢量形式}

重新考察图1的光路传播,定义三个矢量 $\bvec A$、$\bvec A'$ 和 $\bvec A''$ 与法向单位矢量 $\bvec n^0$:

\begin{equation}
\begin{aligned}
& \left| \bvec A \right| = \left| \bvec A'' \right| = n ~, \\
& \left| \bvec A' \right| = n' ~.
\end{aligned}
\end{equation}

它们的方向分别沿着入射光线、折射光线和反射光线的传播方向。

\begin{figure}[ht]
\centering
\includegraphics[width=6cm]{./figures/7a31b2c6eb481fb5.png}
\caption{反射和折射定律矢量关系} \label{fig_GeOp2_2}
\end{figure}

由于三线共面,我们有:
\begin{equation}
\begin{aligned}
& \bvec A' - \bvec A = \Gamma _ t \bvec n^0 ~, \\
& \bvec A'' - \bvec A = \Gamma _ r \bvec n^0 ~.
\end{aligned}
\end{equation}

等式两边同时点乘 $\bvec n^0$:

\begin{equation}
\begin{aligned}
& \bvec n^0 \cdot \bvec A' - \bvec n^0 \cdot \bvec A = \Gamma _ t ~, \\
& \bvec n^0 \cdot \bvec A'' - \bvec n^0 \cdot \bvec A = \Gamma _ r ~.
\end{aligned}
\end{equation}

而:

\begin{equation}
\begin{aligned}
& \bvec n^0 \cdot \bvec A = ncos(I) ~, \\
& \bvec n^0 \cdot \bvec A' = n'cos(I') ~, \\
& \bvec n^0 \cdot \bvec A'' = -ncos(I'') ~.
\end{aligned}
\end{equation}

代入得:

\begin{equation}
\begin{aligned}
& \Gamma _ t = \sqrt{n'^2 - n^2 + (\bvec n^0 \cdot \bvec A)^2} - \bvec n^0 \cdot \bvec A ~, \\
& \Gamma _ r = -2 \bvec n^0 \cdot \bvec A ~.
\end{aligned}
\end{equation}

所以 $\bvec A'$ 和 $\bvec A''$ 用 $\bvec A$ 表示为:

\begin{equation}
\begin{aligned}
& \bvec A' = \bvec A + \Gamma _ t \bvec n^0 = \bvec A + (\sqrt{n'^2 - n^2 + (\bvec n^0 \cdot \bvec A)^2} - \bvec n^0 \cdot \bvec A)\bvec n^0 ~, \\
& \bvec A'' = \bvec A + \Gamma _ r \bvec n^0 = \bvec A - 2(\bvec n^0 \cdot \bvec A) \bvec n^0 ~.
\end{aligned}
\end{equation}

\subsubsection{全反射}

光从光密介质进入光疏介质时,在两种介质的界面上会出现所谓的\textbf{全反射}现象,即 $I'=0$,没有折射光,光线被界面完全反射的情形。

对于式 $(1)$,将 $I'=0$ 代入,则此时的入射角 $I$ 即为发生全反射的临界角 $\theta_c$:

\begin{equation}
\theta _c=I=arcsin(\frac{n'}{n}) ~.
\end{equation}

当入射角大于临界角时,光线会被界面完全反射。\footnote{实际上,光并没有被“完全”反射,在折射介质中极靠近入射点的地方依然存在着光场,详见波动光学衰逝波相关内容。}

\subsection{费马原理}

\subsection{马吕斯定理}

马吕斯定理指出,\textbf{垂直于入射波面的入射光线束,经过任意次反射和折射,出射光线束仍然垂直于出射波面,并且在入射波面和出射波面间的所有光路光程都相等。}

通过电磁学理论分析,马吕斯定理是显而易见的:同一个光源发出的入射波面,经过反射和折射得到的出射波面,两者相位差是恒定的。波的行进带来相位差,于是光程也恒定。
