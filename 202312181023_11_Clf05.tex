% 单向量的运算
% license Xiao
% type Tutor


\begin{issues}
\issueTODO
\end{issues}

% 可加入式五对向量空间的证明。

注:本文参考 Jier Peter 的《代数学基础》。

在上一节,我们介绍了单向量的基本运算。本节着重讨论外积、对偶以及左内积运算在单向量集合上的性质,即这些运算在单向量集合上是否封闭,以及运算结果与子空间的关系。需要注意的是,左内积以及对偶运算往往需要我们取一组正交基,因为用集合语言讨论非常方便,所以对于退化二次型,我们需要仔细检验某些结论是否成立。
\subsubsection{互反基}
由于几何代数定义在线性空间上,所以我们也可以定义“对偶基”的概念。稍后我们可以发现,对单向量取对偶后其子空间和“对偶基”的关系。

\begin{definition}{}
给定\textbf{非退化}的几何代数$\mathcal G(V,q)$,$\{e_1,e_2...e_k\}$是$V$上的一组基,则可以定义该基的\textbf{互反基(reciprocal basis)}$\{e^1,e^2...e^k\}$,使得
\begin{equation}
e^i*e_j=\delta^i_j~.
\end{equation}
\end{definition}
reciprocal也可翻译为“互逆”、“对偶”和“倒易”等。
与对偶空间的概念相似,无论是定义中的基向量组还是互反基,都不要求正交性。

不过非退化的几何代数里总存在标准正交基。在取了标准正交基后,其互反基实际上就是这组基本身:$e^i=e_i$。因此,此时互反基相当于利用二次型,将对偶空间同构回原空间。
假设$\{x_i\}$为一般基向量组,考虑过渡矩阵为$T^i_j$的基变换:$y_i=T^j_ix_j$,由互反基与普通基的关系得:
\begin{equation}
y^j=S^j_ix^i~,
\end{equation}
其中$S^j_i$为$T^j_i$的逆矩阵。

\subsubsection{运算封闭性}
利用互反基的概念,我们可以表示任意单向量的对偶。下面我们来验证,如果$A$是单向量,那么$A^c=\bar A^{\perp}$
\begin{theorem}{}\label{the_Clf05_1}
给定\textbf{非退化}的几何代数$G(V,q)$。任取\textbf{非零}单向量$A=v_1\wedge v_2...\wedge v_k$,由于其非零,可以拓展为$V$上的一组基$\{v_1,v_2...v_k,v_{k+1}...v_n\}$,则有
\begin{equation}
A^c\propto v^{k+1}\wedge v^{k+2}...\wedge v^n~,
\end{equation}
\end{theorem}
其中,$\{v^i\}$是$\{v_i\}$的互反基。

proof.
由几何代数非退化,可以取$\bar A$上的一组标准正交基$\{e_i\}^k_{i=1}$,并扩展为全空间的标准正交基$\{e_i\}^n_{i=1}$。由互反基定义得,$v^{k+m}$垂直于$\bar A$.则
\begin{equation}
\begin{aligned}
A&\propto e_1\wedge e_2...\wedge e_k\\
v^{k+1}\wedge v^{k+2}...\wedge v^n&\propto e^{k+1}\wedge e^{k+2}\wedge...\wedge e^n\\
&=e_{k+1}\wedge e_{k+2}\wedge...\wedge e_n~,
\end{aligned}
\end{equation}
对$A$取对偶,由定义得
\begin{equation}
\begin{aligned}
A^c&=AI^{-1}\\
&\propto AI\\
&\propto e_1 e_2... e_k e_1 e_2... e_n\\
&=e^{k+1} e^{k+2}... e^n\\
&=e^{k+1}\wedge e^{k+2}\wedge...\wedge e^n~,
\end{aligned}
\end{equation}
得证。

由上述定理可知,单向量的对偶还是单向量。读者也可以进一步证明,单向量集合在其他运算下也是封闭的。
\begin{corollary}{}
给定非退化的几何代数$\mathcal G(V,q)$。任取两个单向量$A$,$B$,则$A\wedge B$与$A\vee B$也是单向量
\end{corollary}
\begin{corollary}{}\label{cor_Clf05_1}
给定非退化的几何代数$\mathcal G(V,q)$。任取两个单向量$A$,$B$,则$A\llcorner B$也是单向量。
\end{corollary}
第二个推论的证明需要用到\autoref{the_clf02_1}~\upref{clf02}$A\llcorner B=(A\wedge B^c)^c$,并结合本节\autoref{the_Clf05_1}。
但实际上,即使是非退化的几何代数,该推论也成立。
\begin{corollary}{}\label{cor_Clf05_2}
若$A,B$为几何代数$\mathcal G(V,q)$的两个单向量,则$A\llcorner B$也是单向量。
\end{corollary}
proof.
设$A=v_1\wedge v_2\wedge...\wedge v_n$,由基本推论\autoref{eq_clf02_1}~\upref{clf02}得,上式可以写为$A\llcorner B=(v_1\wedge v_2...)\llcorner B=(v_1\llcorner (v_2\llcorner...(v_n\llcorner B))$,因此本推论只需要$v_n\llcorner B$成立。
把$q$写为对角矩阵的形式,通过合同变换使得不为零的对角元都在前$k$列,并设$v_n=\sum \limits^{n}_{i=1}a^i e_i,u=\sum \limits^{k}_{i=1}a^i e_i$,其中$\{e_i\}$为标准正交基。

代入计算可得:$v_n\llcorner B=u\llcorner B$,该结果是外积的线性组合。

改变二次型,使得第$k$到第$n$列对角元为1,命名该几何代数为$\mathcal G(V,p)$,与$\mathcal G(V,q)$共享标准正交基。建立线性同构为basis和单向量的恒等映射。因而我们有
\begin{equation}\label{eq_Clf05_1}
u\llcorner_q B=u\llcorner _p B~.
\end{equation}
结合\autoref{cor_Clf05_1} ,可得上式为单向量。


\subsubsection{与子空间关系}
\autoref{the_Clf05_1} 的另一种阐述方式为
\begin{theorem}{}
取$A\in \mathcal B^{\bullet}(V,q)$,若$q$非退化,有
\begin{equation}
\bar A=\bar A^{\perp}~.
\end{equation}
\end{theorem}
现在给出一些运算与子空间的关系。
\begin{theorem}{}
取$A,B\in \mathcal B^{\bullet}(V,q)$,则有
\begin{equation}
\left\{\begin{aligned}
A \wedge B \neq 0 & \Longrightarrow \overline{A \wedge B}=\bar{A} \oplus \bar{B} \\
\bar{A}+\bar{B}=V & \Longrightarrow \overline{A \vee B}=\bar{A} \cap \bar{B} \\
\bar{A} \subseteq \bar{B} & \Longrightarrow A\llcorner B=A B \\
A\llcorner B \neq 0 & \Longrightarrow \overline{A\llcorner B}=\bar{A}^{\perp} \cap \bar{B} \\
\bar{A} \cap \bar{B}^{\perp} \neq\{0\} & \Longrightarrow A\llcorner B=0 .
\end{aligned}\right.~
\end{equation}
\end{theorem}
对于第一条,我们由左边可知,$A,B$各自对应的线性无关组合在一起也是线性无关的。若$\bar A,\bar B$有非0元素为$v$,可知$-v$也是交集元素,则合在一起的基矢组线性相关。

对于第二条,设$\bar A$对应基矢组$\{e_i\}^k_{i=1}$,$\bar B$对应基矢组$\{e_i\}^n_{i=r+1}$,即交集部分为组$\{e_i\}^k_{i=r+1}$。根据定义有
\begin{equation}
\begin{aligned}
A \vee B &=(A^c\wedge B^c)^c\\
&\propto\left(\bigwedge\limits^n_{i=k+1}e_i\wedge \bigwedge\limits^r_{i=1}e_i\right)^c\\
&=\bigwedge\limits^k_{i=r+1}e_i~,
\end{aligned}
\end{equation}
因此第二条成立。

读者可以验证第三条。
第四条的证明思路仿照\autoref{cor_Clf05_2} 。由于$A\llcorner B=(v_1\wedge v_2...)\llcorner B=(v_1\llcorner (v_2\llcorner...(v_n\llcorner B))$,我们只消验证第四条对任意$v\in V$都成立即可。

非退化的情况下把$\bar B$内的一组正交基$\{e_i\}^k_{i=1}$扩展为全空间的正交基,然后证明该式成立。

下面证明二次型退化的情况下该式依然成立。
令$q$为该线性空间的标准二次型,前k个对角元不为0,其他的都为0。令$p$为把$q$中为0对角元改为1的结果。外积与二次型无关,所以可以建立从$\mathcal G(V,q)$到$\mathcal G(V,p)$的线性同构(恒等映射),该同构保外积形式不变。令$\{e_i\}^n_{i=1}$为这两个代数的公共正交基,$v=a^ie_i,u=\sum \limits ^k_{i=1}a^ie_i$=。则我们有:
\begin{equation}
v_{\llcorner q} B=u_{\llcorner q}B=u_{\llcorner p}B~,
\end{equation}
由\autoref{cor_Clf05_2} 可得,上式为单向量,且由前面的证明可知,非退化下其张成的空间为$\bar u^{\perp}\cap\bar B$。最后我们只需要证明:$u^{\perp}|_p=v^{\perp}|_q$。

设$x=\{b^ie_i|x\in u^{\perp}|_p\}$,则
\begin{equation}
\begin{aligned}
u\cdot x|_p&=\sum \limits^k_{i=1}a^i b^i p(e_i)=0\\
& \Leftrightarrow \sum \limits^n_{i=1}a^i b^i q(e_i)=0\\
& \Leftrightarrow u^{\perp}|_p=v^{\perp}|_q~.
\end{aligned}
\end{equation}
