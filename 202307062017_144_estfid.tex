% 静电场与静磁场(摘要)

\begin{issues}
\issueDraft
\end{issues}

\subsection{电荷、电流与电荷守恒}
什么是电荷?按照Landau富有深意的话来说,电荷是“粒子与电磁场相互作用的强度”。不过,对于初学者来说,你只要知道电荷同质量一样,也是粒子/物质的一种性质。之所以引入电荷的概念,是因为人们发现粒子间除了有源于质量的引力交互作用外,还有另一种交互作用,其强度无关但与物质的另一种性质有关。人们把这种性质称为“电荷”,这种交互称为“静电力”。

同质量一样,电荷是守恒的\upref{ChgCsv}。也就是说,如果一个区域内有电荷的流出,那么这个区域内的电荷量就会相应地减小。我们将定向运动的电荷的称为电流\upref{I}。
$$
\oint \bvec j \vdot \dd{\bvec s}  =  - \dv{t} \int \rho \dd{V}~,
$$
或者
$$
\div \bvec j + \pdv{\rho}{t} = 0~.
$$
在静场条件下, 空间中的电荷密度以及电流密度都不随时间变化, 所以有
$$
\div \bvec j = 0~,
$$
或者
$$\sum_i I_i = 0~.$$
其中,电流流入、流出的方向由正负号标记。这个结论还被称为基尔霍夫第一定律\upref{Kirch}

\subsection{静电场与静磁场}
\begin{table}[ht]
\centering
\caption{静电场与静磁场}\label{tab_estfid1}
\begin{tabular}{|c|c|c|}
\hline
* & 电场 $\bvec E$ \upref{Efield} & 磁场 $\bvec B$\upref{MagneF} \\
\hline
场源 & 电荷 $q$ \upref{Efield}& 电流(运动的电荷) $\bvec j$ \upref{I}\\
\hline
场源产生的场 & $$\dd \bvec E (\bvec r) = \frac{1}{4 \pi \epsilon_0} \frac{\dd q}{R^2} \bvec {\hat R}~ $$
其中 $\bvec r$是场点,$\bvec r'$是场源位置,$\bvec R = \bvec r - \bvec r'$是场源指向场点的矢量,$\bvec{\hat R}$是相应的单位矢量。\upref{Efield} 不大严格地说,$\dd \bvec E$是由单个小电荷 $\dd q$ 产生的电场。
& $$\dd \bvec B(\bvec r) = \frac{\mu_0}{4\pi} \frac{I \dd{\bvec r'} \cross \uvec R}{R^2}~$$ 比奥萨伐尔定律\upref{BioSav}; $\dd \bvec B$是由一小段电流 $I \dd \bvec r'$ 产生的电场。\footnote{不同于静电场中可以任意摆放电荷,在静磁场中我们不能“任意摆放”电流,而必须时电流成环,或者延伸到无穷远处。假如设计的“电路”不成环,那么根据电荷守恒\upref{ChgCsv},区域内的电荷密度必定变化,从而不再是静场问题。这也是为什么这个公式实际上不能准确描述“单个运动电荷的磁场”。}\\
\hline
散度方程 & 
$$\oint \bvec E \vdot \dd{\bvec s} = \frac{1}{\epsilon_0}\int \rho \dd{V} = \frac{Q}{\epsilon_0}~$$
$$\div \bvec E = \frac{\rho}{\epsilon_0}~$$ 电场的高斯定律\upref{EGauss}
&
$$\oint \bvec B \vdot \dd{\bvec s} = 0~$$
$$\div \bvec B = 0~$$ 磁场的高斯定律\upref{MagGau}\\
\hline
旋度方程 & 
$$ \oint \bvec E \vdot \dd{\bvec l} = \bvec 0~$$
$$ \curl \bvec E = \bvec 0 ~$$ 静电场的环路定理\upref{ELECLD}
 &
$$\oint \bvec B \vdot \dd{\bvec l} = \mu_0 \int \bvec j \vdot \dd{\bvec s}=\mu_0 I ~$$ 
$$\curl \bvec B = \mu_0 \bvec j~$$ 静磁场的环路定理(专业术语:安培环路定律) \upref{AmpLaw}\\
\hline 
\end{tabular}
\end{table}

\subsection{电(标)势与磁矢势}
基于数学与物理意义上的考量,有时使用势的概念,会比场更为简洁、深刻。
\begin{table}[ht]
\centering
\caption{电(标)势与磁矢势}\label{tab_estfid2}
\begin{tabular}{|c|c|c|}
\hline
* & 电场 $\bvec E$ & 磁场 $\bvec B$ \\
\hline
势 & $$\varphi~$$  电势\upref{QEng},标量函数& $$\bvec A~$$  磁矢势\upref{BvecA},矢量函数\\
\hline
势与场 & $$\bvec E = -\grad \varphi~$$ & $$\bvec B = \curl \bvec A~$$ \\
\hline
势的任意性 & $$\varphi += \lambda~$$ $\lambda$是常数 & $$\bvec A += \grad \lambda~$$ $\lambda$是标量函数 \\
\hline
\end{tabular}
\end{table}

由于电荷直接感受到的是场而不是势,所以只要能得到相同的场,势的取值可以具有一定的任意性。例如,由于 $\bvec E = -\grad \varphi$,即使势加上一个常数后,仍有 $\bvec E = -\grad (\varphi+\lambda) = -\grad \varphi - \grad \lambda = -\grad \varphi$,即相应的电场也不会变化。因此,电势总能相差一个常数。这就有点像我们做不定积分时,总会得到一个积分常数 $C$;或者对函数求导时,常数项不会改变导函数。
