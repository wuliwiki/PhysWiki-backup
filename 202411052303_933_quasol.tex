% 因式分解与一元二次方程(高中)
% license Xiao
% type Tutor
\pentry{函数回顾\nref{nod_HsFunB}}{nod_b9ee}

本篇将回顾初中阶段学习的代数运算,包括\textbf{因式分解(factorization)}和\textbf{一元二次方程(quadratic equation)}的内容。这些知识构成了高中数学的重要基础,是理解更复杂代数运算和解析几何的关键。高中学习中,因式分解帮助理解多项式的结构,简化复杂表达式,而一元二次方程则用于解决一系列涉及二次关系的函数、解析几何问题。

\subsection{因式分解}

\textbf{单项式(monomial)}是由常数系数与变量的非负整数次幂的乘积构成的数学表达式。一个单项式可能包含一个变量或多个变量,但它们的幂次必须是非负整数。例如,$3x^2y$ 和 $7z$ 是单项式,而包含其他运算(如 $\sin x$、$\cos x$、$2^x$)或非整数次幂(如 $\displaystyle x^{1/2} = \sqrt{x}$、$\displaystyle x^{-1} = \frac{1}{x}$)的表达式均不是单项式。\textbf{多项式(polynomial)}是由一个或多个单项式通过加法连接组成的表达式。不含变量的单项式称为\textbf{常数项},如 $5$ 或 $-3$,这类单项式在表达式中不会随变量的变化而变化。

\textbf{因式分解(factorization)}是将一个多项式分解为多个多项式或单项式的乘积的操作,这些构成乘积的式子称为\textbf{因式(factor)}。因式分解是解决多项式方程的一个重要工具,因为多项式方程可以转化为“多项式 = 0”的形式。利用因式分解,方程可以进一步简化为多个因式的乘积等于零,而乘积为零的情况下,任一因式可以等于零,从而降低了方程求解的难度。

通常,分解所得的每个因式应为\textbf{不可约的(irreducible)},即不能进一步分解。高中阶段常见的不可约的因式包括:单项式、一次多项式(形如$ax+b$)以及符合\aref{条件}{eq_quasol_4}的二次多项式(形如$ax^2+bx+c$)。其余的因式可以通过\textbf{爱森斯坦判别法(Eisenstein's criterion)}来判定:
对多项式$a_n x^n + a_{n-1}x^{n-1} + \cdots + a_0$  ,若存在一个素数  $p$  使得 :
\begin{itemize}
\item $a_{n-1},a_{n-2},\dots,a_0$能被$p$整除;
\item $a_n$不能被$p$整除;
\item $a_0$不能被$p^2$整除。
\end{itemize}
则这个多项式在有理数范围内是不可约的。这个判别法本身只作为扩展视野,但它揭示了因式分解与质数计算的重要关系。这使得因式分解本质上和质因数分解分不开,是一个比较困难的问题。在高中阶段一般只会涉及到简单的分解,这是基本功需要非常熟练。因式分解的方法包括:
\begin{itemize}
\item 提取公因式,如:$4x^2+2x\to2x(2x+1)$;
\item 分组分解,有时需要添项或拆项,如:$ x^3 + x^2 + x + 1\to (x^3  + x) + (x^2+ 1)\to(x^2 + 1)(x + 1) $
\item 十字相乘法
\end{itemize}

\textbf{十字相乘法(cross method)}的实际操作是对二次三项式$ax^2 + bx + c$则需要将$a,c$分别分解为两个数的乘积,交叉分别相乘之后的和为$b$。\addTODO{原理图}

它背后的依据是:
\begin{equation}
(ax+b)(cx+d)=acx^2+(ad+bc)x+bd~.
\end{equation}


\subsection{一元二次方程的根}

一元二次方程指形如
\begin{equation}
ax^2+bx+c=0 \qquad (a\neq 0)~.
\end{equation}
的方程,它可以通过“配平方”得到如下形式:
\begin{equation}\label{eq_quasol_3}
\begin{split}
 & ax^2+bx+c = 0 \\ 
\iff&x^2+2{b\over 2a} x+\left({b\over 2a}\right)^2 = \left({b\over 2a}\right)^2-{c\over a} \\ 
\iff&\left(x+{b\over 2a}\right)^2 = {b^2-4ac\over 4a^2} \\ 
\end{split}~.
\end{equation}

从函数的视角来看,如果设函数$y=ax^2+bx+c$,那么求解方程$ax^2+bx+c=0$即化为寻找所有使$y=0$成立的点。

\subsubsection{存在性判定}

由于\autoref{eq_quasol_3} 左侧及右侧分母是一个平方形式,因此$x$的值只与右侧分子的符号相关,定义其为判别式:
\begin{equation}\label{eq_quasol_4}
\Delta = b^2-4ac~.
\end{equation}

则:
\begin{itemize}
\item $\Delta > 0$时,方程有实数解,为两个不同的实根。
\item $\Delta = 0$时,方程有实数解,为两个相同的实根。\footnote{这里称两个相同实根而非一个实根,是根据\aref{代数基本定理}{the_HsEquN_1}。}
\item $\Delta < 0$时,方程无实数解。
\end{itemize}

从函数的视角看\autoref{eq_quasol_3} 的变化,其实就是相当于将函数水平平移到了以$y$轴为对称轴的位置上。而方程左侧是一个开口向上,对称轴为$y$轴的二次函数,右侧则是它与$y$轴的交点。那么交点的正负自然意味着函数与$x$轴交点的性质。

\begin{figure}[ht]
\centering
\includegraphics[width=14cm]{./figures/652b2f4efe76656f.pdf}
\caption{$f(x)$示意图。从左到右为$\Delta > 0, \Delta = 0, \Delta < 0$} \label{fig_quasol_1}
\end{figure}

\subsubsection{求根公式}

对于\autoref{eq_quasol_3} ,方程有解时,两个根分别为:
\begin{equation}\label{eq_quasol_1}
x_1=\frac{-b+\sqrt{\Delta}}{2a}\qquad x_2=\frac{-b-\sqrt{\Delta}}{2a}~.
\end{equation}
或统一写作
\begin{equation}
x_{1,2} = \frac{-b \pm \sqrt{b^2 - 4ac}}{2a}~.
\end{equation}
这被称为\textbf{二次方程的求根公式(quadratic formula)}。

\autoref{eq_quasol_1} 中,若$\Delta = 0$,则$\displaystyle x_1=x_2=-\frac{b}{2a}$与之前推理的结论相符。$\displaystyle x=-\frac{b}{2a}$也正是二次函数的对称轴,两个零点如果存在则一定关于该轴对称。


\subsubsection{快速解法}

求根公式尽管能适用于所有一元二次方程,且形式固定,但由于其计算过程繁琐,实际做题中可以根据具体情况简化操作。

对于可以快速因式分解的方程,可直接得到:
$$k(x-a)(x-b)=0\Rightarrow x_1=a, x_2=b~.$$
\begin{figure}[ht]
\centering
\includegraphics[width=5cm]{./figures/fe527c666ebfd775.pdf}
\caption{$f(x)=(x-a)(x-b)$示意图} \label{fig_quasol_2}
\end{figure}

对于$a=1$,$b=2n$为偶数的一元二次方程$x^2+2nx+c=0$,可以直接配方得到:
\begin{equation}
(x+n)^2=m\qquad(m=n^2-c)~.
\end{equation}
自然两个根就是
\begin{equation}
x_1=\sqrt{m}-n\qquad x_2=-\sqrt{m}-n~.
\end{equation}
这与求根公式的推导过程相同,但计算上利用特殊条件快一些。

\subsection{韦达定理}

\textbf{韦达定理(Vieta's formulas)}是法国数学家弗朗索瓦·韦达(François Viète)发现的一组描述代数方程的根和方程系数的关系的公式,因此也称\textbf{根与系数关系}。

\begin{theorem}{韦达定理(二次情况)}
设方程 $a x^2 + b x + c = 0\quad(a \neq 0)$的两个根为 $x_1$ 和 $x_2$,则它们满足:
$$\begin{aligned}
x_1 + x_2 &= -\frac{b}{a} ~,\\
x_1 x_2 &= \frac{c}{a}~.
\end{aligned}$$
\end{theorem}

下面给出证明:

对方程 $a x^2 + b x + c = 0\quad(a \neq 0)$,设其有两个根 $x_1$ 和 $x_2$,则有:
\begin{equation}\label{eq_quasol_2}
\begin{split}
ax^2+bx+c &= 0 \\ 
&=a(x-x_1)(x-x_2)\\
&=ax^2-a(x_1+x_2)x+ax_1x_2
\end{split}~.
\end{equation}
由于\autoref{eq_quasol_2} 为恒成立的代数式,所以两侧对应变量的系数相等,从而有:
\begin{equation}
\begin{cases}
-a(x_1+x_2)=b\\
ax_1x_2=c
\end{cases}
\implies
\begin{cases}
x_1 + x_2 &= \displaystyle-\frac{b}{a} \\
x_1 x_2 &= \displaystyle\frac{c}{a}
\end{cases}.~
\end{equation}

证毕。

这是一种常用的证明方法,即恒成立的代数方程,两侧对应位置的系数相等。韦达定理使得面对方程时,即使不求解具体的解的形式,也可以描述两个解之间的关系,这在\enref{解析几何}{JXJH}部分会有非常重要的应用。另外,对于任意$n$次代数方程的情况,有:
\begin{theorem}{韦达定理}
对于$n$次代数方程$a_0x^n+a_1x^{n-1}+\cdots+a_n=0\quad(a_0\neq0)$,若存在$n$个根$x_1,x_2,\cdots,x_n$,则有:
\begin{equation}
\begin{split}
&x_1+x_2+\cdots +x_n=-{a_1\over a_0}\\
&(x_1x_2+x_1x_3+\cdots+ x_1x_n)+(x_2x_3+\cdots +x_2x_n)+\cdots+x_{n-1}x_n={a_2\over a_0}\\
&\vdots\\
&x_1x_2\cdots x_n=(-1)^n{a_n\over a_0}
\end{split}~.
\end{equation}
也可以统一记作\footnote{此处使用的是\enref{求和符号}{SumSym}和\enref{求积符号}{ProdSy}。如果不理解可以跳过。}:
\begin{equation}
\sum_{1\le i_1 < i_2 < \cdots < i_k\le n} \left(\prod_{j = 1}^k r_{i_j}\right)=(-1)^k\frac{a_{n-k}}{a_n}~.
\end{equation}
\end{theorem}

它的证明方式与之前完全相同,可以自己尝试做一下,在计数原理部分会有更透彻的讲解。