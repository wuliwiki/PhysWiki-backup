% 网球拍定理(科普)
% keys 网球拍定理|旋转|转动惯量|Tennis racket theorem|贾尼别科夫效应|Dzhanibekov Effect
% license Usr
% type Art

\pentry{角动量(科普)\upref{AngMo}}



\subsection{现象描述}


\textbf{中间轴定理},又称\textbf{网球拍定理}或\textbf{贾尼别科夫效应(Dzhanibekov Effect)},收录于于法国数学家、物理学家潘索(Louis Poinsot)于1834年出版的\textsl{Théorie Nouvelle de la Rotation des Corps}(《转动物体的新理论》),并由苏联航天员贾尼别科夫(Vladimir Dzhanibekov)在1985年观测到实验证据。


称之为“网球拍”定理,是因为抛掷网球拍的时候能观察到相应的现象,但我们在家里用一本书或者一部手机就能完成这个实验。以手机为例,将手机正常握持,然后抛向空中(请在安全环境下做这个实验,如在床上,以防手机摔坏),使得手机绕如\autoref{fig_Dzhani_2} 所示的轴在空中旋转一周后落回手中。绝大多数情况下,落回手中的手机不再是正面朝上,而是背面朝上。再抛掷一次,手机旋转一周后落回手中,又会变成正面朝上。


%\begin{figure}[ht]
%\centering
%\includegraphics[width=7cm]{./figures/edb6af44a3d38d05.pdf}
%\caption{网球拍定理示意图。图中手机是正面朝上。将手机抛向空中,使得手机绕图中虚线轴旋转一周后落回手中,则通常手机变成背面朝上。} \label{fig_Dzhani_1}
%\end{figure}



\begin{figure}[ht]
\centering
\includegraphics[width=12cm]{./figures/28ef0f94d7aeadab.pdf}
\caption{网球拍定理示意图。图中手机是正面朝上。将手机抛向空中,使得手机绕图中\textbf{左边}的虚线轴旋转一周后落回手中,则通常手机变成背面朝上。} \label{fig_Dzhani_2}
\end{figure}

仔细观察抛掷过程会发现,扔手机的时候手给手机施加的力不对称,使得手机在空中也会绕如\autoref{fig_Dzhani_2} 右边的轴旋转,这才造成了最终的翻转。如果你抛掷后手机只转了半圈就落回来,那么也很容易造成一次翻转,使得手机仍然正面朝上,但上下颠倒了。即便是双手握住手机,尽可能对称地用力,仍难以避免手机旋转的同时绕着\autoref{fig_Dzhani_2} 右边的轴翻转。


贾尼别科夫在太空中观察到的现象则更为明显。如\autoref{fig_Dzhani_3} 所示,蝴蝶形螺帽的两个“翅膀”是为了方便手动上紧或松开螺帽设计的。快速松开螺帽的时候,螺帽脱离螺丝后仍然会绕着图中的红色虚线旋转。在地球上,螺帽飞出后会直接落地,因而很难观察到特别的现象;但在太空中,由于螺帽脱离后能在空中保持旋转姿态,从而能观察到奇特的现象:螺帽看似稳定地旋转一会儿后,会突然翻转,再继续稳定地旋转后又翻转回来,如此反复。


具体的翻转效果见\autoref{fig_Dzhani_3} 的描述,或参见\href{https://www.bilibili.com/video/BV1jy4y1Y7CU/?share_source=copy_web&vd_source=5d336c88ab0583d823dccd3a9651b27b}{萌萌战队的视频}和\href{https://www.bilibili.com/video/BV12K411S7Gc/?spm_id_from=333.337.search-card.all.click&vd_source=0ae31f148e26bb548391e861d5f13610}{帆雨动画的视频}。


\begin{figure}[ht]
\centering
\includegraphics[width=6cm]{./figures/074d90d4f05d77fc.pdf}
\caption{贾尼别科夫的蝴蝶形螺帽实验示意图。图中是一个蝴蝶形螺帽,两个“翅膀”朝上,让螺帽绕着红色虚线旋转,则螺帽会在旋转时突然翻转,变成翅膀朝下,继续旋转,过一会儿又突然翻转,变成翅膀朝上,如此循环往复。} \label{fig_Dzhani_3}
\end{figure}



抛掷网球拍、手机时的翻转效果,和贾尼别科夫观察到的螺帽旋转过程中周期性的翻转,是源于相同的机制,这便是本节要介绍的贾尼别科夫效应。为了理解这一现象,我们需要铺垫一些关于旋转的知识。





\subsection{角动量}


\pentry{矢量叉乘\upref{Cross}}

本节介绍角动量的概念,已经熟悉此概念的读者可以跳过。


\subsubsection{提出角动量概念的动机}


我们知道,一个系统不受外力的时候,总动量是守恒的。尽管系统中的若干质点之间可能相互作用,导致各质点的动量改变,但总量依然守恒。

这是因为按照动量的定义,每个质点的动量是$mv$,其中$m$是质点的质量,$v$是它此刻的速度。在经典力学中,质点的质量不变,那么引起动量改变的只能是速度的变化,因此动量的改变率正是质量乘以速度的改变率,即$ma$,其中$a$是速度的改变率,即加速度。根据\textbf{牛顿第二定律},$ma$正是质点所受的外力。由\textbf{牛顿第三定律},两个质点相互作用的时候,彼此所受来自对方的力刚好方向相反,大小相同,因此它们给彼此造成的动量改变速率也相同,但方向相反,正好互相抵消,从而总动量的改变速率为$0$。对更多质点也是如此。


动量守恒为解决问题带来了很多便利。于是,在研究转动的时候,我们也希望找到一种和旋转有关的不变量,这样才有可能方便讨论问题。


和平动不一样,转动依赖于参考点,即“绕某点旋转”或者“绕某轴旋转”。因此,和旋转有关的不变量也应该以参考点来定义。


\begin{figure}[ht]
\centering
\includegraphics[width=14cm]{./figures/8ba4779afcd1aba8.pdf}
\caption{角动量的动机示意图。图中点$X$做匀速直线运动,$P$是参考点。$X$和$P$的连线,在相同的两个时间间隔内扫过的面积相同。} \label{fig_Dzhani_5}
\end{figure}


如\autoref{fig_Dzhani_5} ,点$X$不受外力,做匀速直线运动。以点$P$为参考点,$X$和$P$的连线随着$X$的运动,在一定的时间间隔内扫过一个三角形面积。图中面积$S_1$和$S_2$所对应的\textbf{时间间隔相同},则\textbf{两个面积相同}。这是因为$X$做匀速直线运动,相同时间内运动的距离相同,即两个三角形的底边长相同;两个三角形的高都是点$P$到$X$的运动所在直线的垂线段的高,因而也相同。



换个视角看,则是当$X$运动时,连线$PX$扫过的面积逐渐增长,但这个\textbf{面积的增长速率}是不变的。由于扫过的三角形的高是不变的,因此增长速率由底边长的增长速率决定,而这个增长速率恰为点$X$的运动速率。因此,面积的增长速率为$rv$,其中$r$是$P$到$X$的运动直线的垂线段长度,$v$是$X$的运动速率。


这个不变量,就是定义角动量的动机。



\subsubsection{角动量的概念}





表示扫过面积的增长速率还有一个办法:$\bvec{r}\times \bvec{v}$,其中$\bvec{r}$是以点$P$为起点、点$X$为终点的\textbf{几何向量}\upref{GVec},$\bvec{v}$是$X$的速度。由此,我们就可以简洁地定义角动量了。


\begin{definition}{角动量}

给定参考点$P$和一个质点。若质点的质量为$m$,则当点$P$到质点的位移向量是$\bvec{r}$、质点的运动速度是$\bvec{v}$时,质点\textbf{关于参考点}$P$的\textbf{角动量(angular momentum)}定义为
\begin{equation}
\bvec{L} = m\bvec{r}\times \bvec{v}~, 
\end{equation}
或等价地,
\begin{equation}
\bvec{L} = \bvec{r}\times\bvec{p}~, 
\end{equation}
其中$\bvec{p}$是质点的\textbf{动量}。

\end{definition}



角动量相当于“旋转版本的动量”,它刻画了物体关于参考点的旋转运动状态。角动量是一个\textbf{向量},它描绘了物体绕参考点旋转的方向,可以用\textbf{右手定则}来记忆。伸出你的右手,四指握拳,大拇指竖直向上,比出“点赞”的手势,则一个沿着四指弯曲方向旋转的质点的角动量之方向,就是大拇指指向的方向。

自行车向前运动时,其轮子上某处质点关于轮轴中心的角动量方向是垂直轮盘向左的;地球自西向东自转,则赤道上的质点关于\textbf{地心}的角动量是沿着自转轴向北的;但不在赤道上的质点,其关于地心的角动量就不是沿着自转轴向北的,而是和地球表面\textbf{相切},如\autoref{fig_Dzhani_6} 所示。


\begin{figure}[ht]
\centering
\includegraphics[width=10cm]{./figures/d0390d410dd5b5bf.pdf}
\caption{地球上随地球自转的质点关于地心的角动量示意图。点$Q$在赤道上,其角动量方向沿着地球自转轴向北;点$P$不在赤道上,其角动量方向就和自转轴方向不平行。但无论何种情况,质点的角动量都和地面相切,且与地球自转轴在同一平面。} \label{fig_Dzhani_6}
\end{figure}





\subsubsection{参考点与参考轴}



旋转既可以关于参考点定义,也可以关于参考轴定义。
























