% 微正则系统(综述)
% license CCBYSA3
% type Wiki

本文根据 CC-BY-SA 协议转载翻译自维基百科\href{https://en.wikipedia.org/wiki/Microcanonical_ensemble}{相关文章}。

在统计力学中,微正则系综是一个统计系综,表示总能量被精确指定的机械系统的可能状态。[1] 假设系统是孤立的,即它不能与环境交换能量或粒子,因此(根据能量守恒)系统的能量随着时间不发生变化。

微正则系综的主要宏观变量是系统中的总粒子数(符号:\( N \))、系统的体积(符号:\( V \))以及系统中的总能量(符号:\( E \))。这些变量在系综中被假设为常数。因此,微正则系综有时被称为 NVE 系综。

简单来说,微正则系综通过为每个能量落在以\( E \)为中心的范围内的微观状态分配相等的概率来定义。所有其他微观状态的概率为零。由于概率总和必须为 1,因此概率 \( P \) 是能量范围内微观状态数\( W \)的倒数,
\[
P = \frac{1}{W},~
\]
然后能量范围的宽度被逐渐缩小,直到它变得无限窄,仍然以\( E \)为中心。在这个过程中,当宽度趋于零时,得到微正则系综。[1]