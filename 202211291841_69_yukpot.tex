% Yukawa 势
% Yukawa 势|Yukawa 理论|非相对论近似|强相互作用

\subsection{Yukawa potential for $e^-e^-\rightarrow e^-e^-$}

Calculate the tree-level diagram of $\ket{{\bvec p},s,+;{\bvec k},r,+}\rightarrow \ket{{\bvec p}',s',+;{\bvec k}',r',+}$. $+$ denotes the fermion and $-$ denotes the antifermion, $\bra{\bvec p',s',+;\bvec k',r',+}=\bra 0 a_{\bvec k'}^{r'} a_{\bvec p'}^{s'}$.
\begin{equation}
\begin{aligned}
\notag i\mathcal{M}&=
\frac{(-ig)^2}{2!}\int \dd[4]x \int  \dd[4]y 
{
\langle\overset{1}{\bvec p',s',+};\overset{2}{\bvec k',r',+}|
(\overset{3}\phi  \overset{1}{\bar{\psi}} \overset{4}\psi)_x 
(\overset{3}\phi \overset{2}{\bar{\psi}} \overset{5}\psi)_y
|\overset{4}{\bvec p,s,+};\overset{5}{\bvec k,r,+} \rangle
}+(x\leftrightarrow y)\\
&\quad +
\frac{(-ig)^2}{2!}\int \dd[4]x \int  \dd[4]y 
\langle\overset{1}{\bvec p',s',+};\overset{2}{\bvec k',r',+}|
(\overset{3}\phi  \overset{1}{\bar{\psi}} \overset{4}\psi)_x 
(\overset{3}\phi \overset{2}{\bar{\psi}} \overset{5}\psi)_y
|\overset{5}{\bvec p,s,+};\overset{4}{\bvec k,r,+} \rangle
+(x\leftrightarrow y)\\
&=(-ig)^2\left(
 \bar u^{s'}(p')u^s(p) \frac{i}{(p'-p)^2-m_{\phi}^2} \bar u^{r'}(k') u^r(k)
\right.\\
&\left.\quad 
-\bar u^{s'}(p')u^r(k) \frac{i}{(p'-k)^2-m_{\phi}^2} \bar u^{r'}(k') u^{s}(p)  \right)
\end{aligned}
\end{equation}
In non-relativistic limit, the two fermions are distinguishable, so we ignore the second term. In non-relativistic limit, $p=(m,{\bvec p})+O(|{\bvec p}|^2),k=(m,{\bvec k})+O(|{\bvec k}|^2),\cdots$,$(p-k)^2=-|{\bvec p}-{\bvec k}|^2+O(|{\bvec p}|^4)$, we have \begin{align*}
&u^r(k)=\sqrt{m} \begin{pmatrix}\xi^r\\ \xi^r\end{pmatrix}+O(|{\bvec p}|),\quad  \xi^{r\dagger}\xi^s=\delta_{rs}\\
&\bar u^s(p') u^r(p)=2m \delta_{rs}
\end{align*}
we obtain
\begin{equation}
\begin{aligned}
i\mathcal{M} = i g^2 (2m)^2 \frac{1}{|{\bvec p}'-{\bvec p}|^2+m_\phi^2}
\end{aligned}
\end{equation}
$i\mathcal{T}=i\mathcal{M}(2\pi)^4 \delta^4(p'+k'-p-k)$, $i\mathcal{T} = \bra{{\bvec p}',{\bvec k}'}iT\ket{{\bvec p},{\bvec k}}$. Use Born approximation, and consider the normalization: $\langle{\bvec p}'|{\bvec p}\rangle=2m\cdot (2\pi)^3 \delta({\bvec p}'-{\bvec p}) = 2m \langle{\bvec p}'|{\bvec p}\rangle_{NR}$('NR' means the normalization scheme in non-relativistic case), we have
\begin{equation}
\begin{aligned} 
i\mathcal{T}&=(2m)^2\bra{{\bvec p}',{\bvec k}'}iT\ket{{\bvec p},{\bvec k}}_{NR}\\
& =-(2m)^2\bra{{\bvec p}',{\bvec k}'}iV\ket{{\bvec p},{\bvec k}}_{NR} \cdot (2\pi) \delta(E_{{\bvec p}'}-E_{\bvec p})
\\
& = -i(2m)^2 \tilde{V}({\bvec p}'-{\bvec p})(2\pi)^3 \delta({\bvec p}'+{\bvec k}'-{\bvec p}-{\bvec k})\cdot (2\pi) \delta(E_{{\bvec p}'}-E_{\bvec p})\\
&=-i(2m)^2 \tilde{V}(\bvec q) (2\pi)^4 \delta^4(p'+k'-p-k),\quad \bvec q={\bvec p}'-{\bvec p}
\end{aligned}
\end{equation}
Finally, we obtain
\begin{equation}
\begin{aligned}
\tilde{V}(\bvec q)&=\frac{-g^2}{|\bvec q|^2+m_\phi^2}\\
V({\bvec x})&=\int \frac{\dd[3]\bvec q}{(2\pi)^3} \frac{-g^2}{|\bvec q|^2+m_\phi^2}e^{i\bvec q\cdot {\bvec x}} =\frac{-g^2}{4\pi^2}\int_0^\infty \dd q q^2\int_{-1}^1 \dd t \frac{e^{iq|{\bvec x}|t}}{|\bvec q|^2+m_\phi^2}\\
&= \frac{-g^2}{4\pi^2}\int_0^\infty \dd q q^2\frac{e^{iq|{\bvec x}|}-e^{-iq|{\bvec x}|}}{iq|{\bvec x}|} \frac{1}{q^2+m_\phi^2}\\
&=\frac{ig^2}{4\pi^2|{\bvec x}|}\int_{-\infty}^\infty \dd q \frac{qe^{iq|{\bvec x}|}}{q^2+m_\phi^2}\\
&=-\frac{g^2}{4\pi}\frac{1}{|{\bvec x}|} e^{-m_\phi |{\bvec x}|}
\end{aligned}
\end{equation}
This is an attractive "Yukawa potential".
\subsection{Yukawa potential for $e^+e^-\rightarrow e^+e^-$}
$\ket{{\bvec p},s,+;{\bvec k},r,-}\rightarrow \ket{{\bvec p}',s',+;{\bvec k}',r',-}$, $-$ denotes the antifermion $e^+$. Similarly, we have
\begin{equation}
\begin{aligned}
i\mathcal{M}&=(-ig)^2 \left(
-\bar u^{s'}(p')u^s(p) \frac{i}{|p'-p|^2-m_\phi^2} \bar v^{r}(k) v^{r'}(k')\right.\\
&\left.\quad + \bar u^{s'}(p')v^{r'}(k') \frac{i}{|p+k|^2-m_\phi^2} \bar v^{r}(k) u^s(p)\right)
\end{aligned}
\end{equation}
In non-relativistic limit, $|p'-p|\ll |p+k|$, so we ignore the second term. Use $\bar v^s(p') v^r(p)=-2m \delta_{rs}$, the result is the same as $e^-e^-\rightarrow e^- e^-$, we can obtain an attractive Yukawa potential 
\[
V({\bvec x})=-\frac{g^2}{4\pi}\frac{1}{|{\bvec x}|} e^{-m_\phi |{\bvec x}|}
\]