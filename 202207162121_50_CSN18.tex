% 2018 年计算机学科专业基础综合全国联考卷
% keys 2018 年计算机学科专业基础综合全国联考卷

\subsection{一、单项选择题}
第1~40小题,每小题2分,共80分.下列每题给出的四个选项中,只有一个选项最符合试题要求.

1.若栈$S_1$中保存整数,栈$S_2$中保存运算符,函数$F()$依次执行下述各步操作: \\
(1) 从$S_1$中依次弹出两个操作数$a$和$b$; \\
(2) 从$S_2$中弹出一个运算符$op$; \\
(3) 执行相应的运算$b$ $op$ $a$; \\
(4) 将运算结果压人$S_1$中. \\
假定$S_1$中的操作数依次是$5$, $8$, $3$, $2$($2$在栈顶),$S_2$中的运算符依次是$*$, $-$, $+$ ($+$在栈顶).调用$3$次$F()$后, $S_1$栈顶保存的值是. \\
A. -15 $\quad$  B. 15 $\quad$ C. -20 $\quad$ D. 20

2. 现有队列$Q$与栈$S$,初始时$Q$中的元素依次是$1$, $2$, $3$, $4$, $5$, $6$($1$在队头),$S$为空.若仅允许下列$3$种操作:①出队并输出出队元素;②出队并将出队元素人栈;③出栈并输出出栈元素,则不能得到的输出序列是. \\
A . 1, 2, 5, 6, 4, 3 $\quad$ B. 2, 3, 4, 5, 6, 1 \\
C. 3, 4, 5, 6, 1, 2 $\quad$ D. 6, 5, 4, 3, 2, 1

3. 设有一个$12\times12$的对称矩阵M,将其上三角部分的元素$m_{i, j}$($1\leqslant i\leqslant j\leqslant12$)按$12$行优先存人$C$语言的一维数组$N$中,元素$m_{6,6}$在$N$中的下标是. \\
A . 50 $\quad$ B. 51 $\quad$ C. 55 $\quad$ D. 66

4. 设一棵非空完全二叉树$T$的所有叶结点均位于同一层,且每个非叶结点都有$2$个子结点.若$T$有$k$个叶结点,则$T$的结点总数是. \\
A . 2k-1 $\quad$ B. 2k $\quad$ C. k2 $\quad$ D. 2k-1

5. 已知字符集{a, b, c, d, e, f} ,若各字符出现的次数分别为6, 3, 8, 2, 10, 4 ,则对应字符集中
各字符的哈夫曼编码可能是. \\
A . 00, 1011, 01, 1010, 11, 100 B. 00, 100, 110, 000, 0010, 01 \\
C. 10, 1011, 11, 0011, 00, 010 D. 0011, 10, 11, 0010, 01, 000

6. 已知二叉排序树如下图所示,元素之间应满足的大小关系是.
\begin{figure}[ht]
\centering
\includegraphics[width=12.5cm]{./figures/CSN18_1.png}
\caption{第6题图} \label{CSN18_fig1}
\end{figure}
A. $x_1<x_2<x_5$ $\quad$ B. $x_1<x_4<x_5$ $\quad$ C. $x_3<x_5<x_4$ $\quad$ D. $x_4<x_3<x_5$

7. 下列选项中,不是如下有向图的拓扑序列的是 \\
\begin{figure}[ht]
\centering
\includegraphics[width=14.25cm]{./figures/CSN18_2.png}
\caption{第7题图} \label{CSN18_fig2}
\end{figure}
A. 1, 5, 2, 3, 6, 4  $\quad$ B. 5, 1, 2, 6, 3, 4 \\
C. 5, 1, 2, 3, 6, 4 $\quad$ D. 5, 2, 1, 6, 3, 4

8. 高度为5的3阶B树含有的关键字个数至少是. \\
A. 15 $\quad$ B. 31 $\quad$ C. 62 $\quad$ D. 242

9. 现有长度为$7$、初始为空的散列表$HT$,散列函数$H(k)=k\%7$,用线性探测再散列法解决冲突.将关键字$22$,$43$,$15$依次插人到$HT$后,查找成功的平均查找长度是. \\
A. 1.5 $\quad$ B. 1.6 $\quad$ C. 2 $\quad$ D. 3

10. 对初始数据序列(8, 3, 9, 11, 2, 1, 4, 7, 5, 10, 6)进行希尔排序.若第一趟排序结果为(1,
3, 7, 5, 2, 6, 4, 9, 11, 10, 8),第二趟排序结果为(1, 2, 6, 4, 3, 7, 5, 8, 11, 10, 9),则两趟排序采用的增量(间隔)依次是. \\
A. 3, 1 $\quad$ B. 3,2 $\quad$ C. 5,2 $\quad$ D. 5,3

11. 在将数据序列(6, 1, 5, 9, 8, 4, 7)建成大根堆时,正确的序列变化过程是. \\
A. 6,1,7,9,8,4,5 → 6,9,7,1,8,4,5 → 9,6,7,1,8,4,5 → 9,8,7,1,6,4,5 \\
B. 6,9,5,1,8,4,7 → 6,9,7,1,8,4,5 → 9,6,7,1,8,4,5 → 9,8,7,1,6,4,5 \\
C. 6,9,5,1,8,4,7 → 9,6,5,1,8,4,7 → 9,6,7,1,8,4,5 → 9,8,7,1,6,4,5 \\
D. 6,1,7,9,8,4,5 → 7,1,6,9,8,4,5 → 7,9,6,1,8,4,5 → 9,7,6,1,8,4,5 → 9,8,6,1,7,4,5

12. 冯?诺依曼结构计算机中数据采用二进制编码表示,其主要原因是. \\
Ⅰ .二进制的运算规则简单 \\
Ⅱ .制造两个稳态的物理器件较容易 \\
Ⅲ .便于用逻辑门电路实现算术运算 \\
A. 仅Ⅰ、Ⅱ $\quad$ B. 仅Ⅰ、Ⅲ  $\quad$ C. 仅Ⅱ、Ⅲ $\quad$ D.Ⅰ、Ⅱ和Ⅲ

13. 假定带符号整数采用补码表示, 若int 型变量x 和y 的机器数分别是FFFF FFDFH 和00000041H ,则x、y 的值以及x - y 的机器数分别是. \\
A. x = -65, y = 41, x -y 的机器数溢出 \\
B. x = -33, y = 65, x-y 的机器数为FFFF FF9DH \\
C. x = -33, y = 65, x-y 的机器数为FFFF FF9EH \\
D. x = -65, y = 41, x-y 的机器数为FFFF FF96H

14. $IEEE$ $754$单精度浮点格式表示的数中,最小的规格化正数是. \\
A . $1.0\times2^{-126}$ $\quad$ B. $1.0\times2^{-127}$ $\quad$ C. $1.0\times2^{-128}$ $\quad$ D. $1.0\times2^{-149}$

15. 某32位计算机按字节编址,采用小端(Little Endian) 方式.若语令“ int i = 0; ”对应指令的机器代码为 “C7 45 FC 00 00 00 00 ,”则语句 “int i = - 64;”对应指令的机器代码是 . \\
A . C7 45 FC C0 FF FF FF $\quad$ B. C7 45 FC 0C FF FF FF \\
C. C7 45 FC FF FF FF C0 $\quad$ D. C7 45 FC FF FF FF 0C

16. 整数$x$的机器数为$1101$ $1000$,分别对$x$进行逻辑右移$1$位和算术右移$1$位操作, 得到的机器数各是. \\
A . 1110 1100、1110 1100 $\quad$ B. 0110 1100、1110 1100 \\
C. 1110 1100、0110 1100 $\quad$ D. 0110 1100、0110 1100

17. 假定DRAM芯片中存储阵列的行数为r、列数为c,对于一个2K×1 位的DRAM 芯片,为保证其地址引脚数最少,并尽量减少刷新开销,则r、c 的取值分别是. \\
A . 2048、1 $\quad$ B. 64 、32 $\quad$ C. 32、64 $\quad$ D. 1、2048

18. 按字节编址的计算机中,某double 型数组A 的首地址为2000H ,使用变址寻址和循环结构访问数组A,保存数组下标的变址寄存器初值为0,每次循环取一个数组元素,其偏移地址为变址值乘以sizeof(double) ,取完后变址寄存器内容自动加1.若某次循环所取元素的地址为2100H,则进入该次循环时变址寄存器的内容是. \\
A . 25 B. 32 C. 64 D. 100

19. 减法指令 “sub R1, R2, R3的功”能为 “(R1)-(R2)→ R3”,该指令执行后将生成进位 /借位标志CF 和溢出标志OF.若( R1)= FFFF FFFFH ,(R2)= FFFF FFF0H ,则该减法指令执行后, CF 与OF 分别为. \\
A . CF=0, OF=0 $\quad$ B. CF=1, OF=0 \\
C. CF=0, 0F=1 $\quad$ D. CF=1, OF=1

20. 若某计算机最复杂指令的执行需要完成5个子功能,分别由功能部件A~E 实现,各功能部件所需时间分别为80ps、50ps、50ps、70ps 和50ps,采用流水线方式执行指令,流水段寄存器延时为20ps,则CPU 时钟周期至少为. \\
A . 60 ps $\quad$ B. 70 ps $\quad$ C. 80 ps $\quad$ D. 100 ps

21. 下列选项中,可提高同步总线数据传输率的是. \\
Ⅰ .增加总线宽度Ⅱ.提高总线工作频率 \\
Ⅲ .支持突发传输Ⅳ.采用地址/数据线复用 \\
A . 仅Ⅰ、Ⅱ $\quad$ B. 仅Ⅰ、Ⅱ、Ⅲ \\
C. 仅Ⅲ、Ⅳ $\quad$ D.Ⅰ、Ⅱ、Ⅲ和Ⅳ

22. 下列关于外部I/O中断的叙述中,正确的是. \\
A . 中断控制器按所接收中断请求的先后次序进行中断优先级排队 \\
B. CPU 响应中断时,通过执行中断隐指令完成通用寄存器的保护 \\
C. CPU 只有在处于中断允许状态时,才能响应外部设备的中断请求 \\
D. 有中断请求时, CPU 立即暂停当前指令执行,转去执行中断服务程序

23. 下列关于多任务操作系统的叙述中,正确的是. \\
Ⅰ . 具有并发和并行的特点 \\
Ⅱ . 需要实现对共享资源的保护 \\
Ⅲ . 需要运行在多CPU 的硬件平台上 \\
A. 仅Ⅰ $\quad$ B. 仅Ⅱ $\quad$ C. 仅Ⅰ、Ⅱ $\quad$ D. Ⅰ、Ⅱ、Ⅲ

24. 某系统采用基于优先权的非抢占式进程调度策略,完成一次进程调度和进程切换的系统时间开销为1μs.在T时刻就绪队列中有3 个进程P1、P2 和P3,其在就绪队列中的等待时间、需要的CPU 时间和优先权如下表所示.
\begin{table}[ht]
\centering
\caption{第24题表}\label{CSN18_tab1}
\begin{tabular}{|c|c|c|c|}
\hline
进程 & 等待时间 & 需要的CPU时间 & 优先权 \\
\hline
$P_1$ & 30μs & 12μs & 10μs \\
\hline
$P_2$ & 15μs & 24μs & 30μs \\
\hline
$P_3$ & 18μs & 36μs & 20μs \\
\hline
\end{tabular}
\end{table}
若优先权值大的进程优先获得CPU,从T时刻起系统开始进程调度, 则系统的平均周转时间为 \\
A. 54μs $\quad$ B. 73μs $\quad$ C. 74μs $\quad$ D. 75μs

25. 属于同一进程的两个线程thread1和thread2并发执行, 共享初值为0的全局变量x.thread1和thread2实现对全局变量x加1的机器级代码描述如下. \\
\begin{table}[ht]
\centering
\caption{第25题表}\label{CSN18_tab2}
\begin{tabular}{|c|c|}
\hline
thread1 & thread2 \\
\hline
$mov \quad R1,x \qquad // (x) \rightarrow R1$ & $mov \quad R2, x \qquad // (x) \rightarrow R2$ \\
\hline
$inc \quad R1 \qquad // (R1)+1 \rightarrow R1$ & $inc \quad R2 \qquad // (R2)+1 \rightarrow R2$ \\
\hline
$mov \quad x,R1 \qquad // (R1) \rightarrow x$ & $mov \quad x,R2 \qquad // (R2) \rightarrow x$ \\
\hline
\end{tabular}
\end{table}
在所有可能的指令执行序列中,使x 的值为2 的序列个数是. \\
A. 1 $\qquad$ B. 2 $\qquad$ C. 3 $\qquad$ D. 4

26. 假设系统中有4 个同类资源,进程P1、P2 和P3 需要的资源数分别为4、3 和1,P1、P2和P3 已申请到的资源数分别为2、1 和0,则执行安全性检测算法的结果是. \\
A. 不存在安全序列,系统处于不安全状态 \\
B. 存在多个安全序列,系统处于安全状态 \\
C. 存在唯一安全序列P3、P1、P2,系统处于安全状态 \\
D. 存在唯一安全序列P3、P2、P1,系统处于安全状态

27. 下列选项中,可能导致当前进程P 阻塞的事件是. \\
Ⅰ . 进程P 申请临界资源 \\
Ⅱ . 进程P 从磁盘读数据 \\
Ⅲ . 系统将CPU 分配给高优先权的进程 \\
A. 仅Ⅰ $\qquad$ B. 仅Ⅱ $\qquad$ C. 仅Ⅰ、Ⅱ $\qquad$ D. Ⅰ、Ⅱ、Ⅲ

28. 若x 是管程内的条件变量,则当进程执行x.wait() 时所做的工作是. \\
A. 实现对变量x 的互斥访问 \\
B. 唤醒一个在x 上阻塞的进程 \\
C. 根据x 的值判断该进程是否进人阻塞状态 \\
D. 阻塞该进程,并将之插入x 的阻塞队列中

29. 当定时器产生时钟中断后,由时钟中断服务程序更新的部分内容是. \\
Ⅰ .内核中时钟变量的值 \\
Ⅱ .当前进程占用CPU 的时间 \\
Ⅲ .当前进程在时间片内的剩余执行时间 \\
A. 仅Ⅰ、Ⅱ $\qquad$ B. 仅Ⅱ、Ⅲ $\qquad$ C. 仅Ⅰ、Ⅲ $\qquad$ D. Ⅰ、Ⅱ、Ⅲ

30. 系统总是访问磁盘的某个磁道而不响应对其他磁道的访问请求, 这种现象称为磁臂黏着.下列磁盘调度算法中,不会导致磁臂粘着的是. \\
A . 先来先服务( FCFS) B. 最短寻道时间优先( SSTF) \\
C. 扫描算法( SCAN ) D. 循环扫描算法( CSCAN )

31. 下列优化方法中,可以提高文件访问速度的是. \\
Ⅰ . 提前读Ⅱ . 为文件分配连续的簇 \\
Ⅲ . 延迟写Ⅳ .采用磁盘高速缓存 \\
A . 仅Ⅰ、Ⅱ $\quad$ B. 仅Ⅱ、Ⅲ \\
C. 仅Ⅰ、Ⅲ、Ⅳ $\quad$ D.Ⅰ、Ⅱ、Ⅲ、Ⅳ

32. 在下列同步机制中,可以实现让权等待的是. \\
A . Peterson 方法 $\quad$ B. swap 指令 \\
C. 信号量方法 $\quad$ D. TestAndSet 指令

33. 下列TCP/IP应用层协议中,可以使用传输层无连接服务的是. \\
A . FTP $\quad$ B. DNS $\quad$ C. SMTP $\quad$ D. HTTP

34. 下列选项中,不属于物理层接口规范定义范畴的是. \\
A . 接口形状 $\quad$ B. 引脚功能 $\quad$ C. 物理地址 $\quad$ D. 信号电平

35. IEEE 802.11 无线局域网的MAC 协议CSMA/CA 进行信道预约的方法是. \\
A . 发送确认帧 $\quad$ B. 采用二进制指数退避 \\
C. 使用多个MAC 地址 $\quad$ D. 交换RTS 与CTS 帧

36. 主机甲采用停-等协议向主机乙发送数据,数据传输速率是3 kbps,单向传播延时是200ms,忽略确认帧的传输延时.当信道利用率等于40\%时,数据帧的长度为. \\
A . 240 比特 $\quad$ B. 400 比特 $\quad$ C. 480 比特 $\quad$ D. 800 比特

37. 路由器R 通过以太网交换机S1 和S2 连接两个网络, R 的接口、主机H1 和H2 的IP 地址与MAC 地址如下图所示.若H1 向H2 发送1 个IP 分组P,则H1 发出的封装P 的以太网帧的 目的MAC 地址、H2收到的封装P的以太网帧的源MAC地址分别是.
\begin{figure}[ht]
\centering
\includegraphics[width=14.25cm]{./figures/CSN18_3.png}
\caption{第37题图} \label{CSN18_fig3}
\end{figure}