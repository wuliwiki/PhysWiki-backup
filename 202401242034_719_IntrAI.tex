% 人工智能导论
% keys 人工智能|机器学习|深度学习
% license Usr
% type Tutor

本文旨在为人工智能部分后续的词条建立基本的概念和总览的图景。

\textbf{人工智能}(Artificial Intelligence,简称AI)是一门研究如何使计算机具有智能行为的科学与技术。它涵盖了一系列的技术、方法和应用,旨在\textbf{使计算机系统能够模拟、理解和执行人类智能的各种任务。}人工智能领域的发展已经走过了几个阶段,从最初的符号主义到现代的机器学习和深度学习。

\subsection{符号主义和搜索算法}

人工智能的最初阶段可以追溯到20世纪50年代和60年代,这一时期被称为符号主义时代。研究人员试图通过使用符号和规则来模拟人类智能。这些系统基于专家系统,其中包含了领域专家提供的知识,以解决特定类型的问题。然而,符号主义在处理复杂的、模糊的问题时面临困难,导致了人工智能研究的新方向的产生。

比如,\textbf{基于规则推理}(Rule Base Reasoning,RBR)的方法是一种将专家所掌握的知识和经验转化为规则,通过启发式推理进行推理解的技术。这种方法在解决问题时根据明确的前提条件产生明确的结果,使其推理过程相对清晰。举例来说,对动物的分类规则可以通过IF-THEN语句表示,从而实现对动物种类的判定,如老虎或企鹅。

基于规则的专家系统是早期专家系统的代表,其推理过程相对明确,规则正确时可以得到较为准确的结论。这使得基于规则的专家系统成为一种简单实用、广泛应用的专家系统。尤其对于特定领域的问题,基于规则的方法表现出色,成为解决实际问题的有效手段。

然而,基于规则的专家系统也存在一些缺点。首先,规则的构造高度依赖于专家的经验积累,如果专家的经验不准确,则系统的结果也可能不准确。其次,这类专家系统缺乏自学习能力,更新迭代需要专家经验的不断积累。虽然这两个缺点存在,但从系统开发的角度来看,专家系统是一个持续迭代优化的过程。在实际应用中,不希望专家系统因为自学习而导致结果不可预测的情况,因此依赖专家的经验积累是一个相对可控的方向。

同时期的还有搜索算法,\textbf{搜索算法更侧重于问题空间的系统搜索,而基于规则推理更专注于通过已有规则的逻辑推理解决问题。}搜索算法在当时游戏中的人工智能有着广泛的应用,比如:路径规划、游戏策略、排序等问题。实际上传统的人工智能最初解决的问题大多都是游戏问题。在接下来一些词条介绍传统人工智能算法的细节时,常会涉及人工智能玩棋牌之类益智游戏的算法。

以下是一些常见的搜索算法:

深度优先搜索(Depth-First Search,DFS): 从起始状态开始,沿着一个路径一直探索到底,直到找到目标状态或无法继续。DFS使用堆栈来存储路径信息。

广度优先搜索(Breadth-First Search,BFS): 从起始状态开始,逐层地扩展搜索,先探索离起始状态最近的节点。BFS使用队列来存储待探索的节点。

启发式搜索: 使用启发函数(Heuristic Function)来评估每个状态的“好坏”,并优先探索那些看起来更有可能达到目标的状态。著名的启发式搜索算法包括A*算法。

A*算法: 结合了深度优先搜索和广度优先搜索的优势,使用启发函数来估计从当前状态到目标状态的代价,并选择代价最小的路径。A*算法保证找到最短路径。

\subsection{机器学习}

随着计算能力的提升和数据的增加,人工智能进入了一个新的阶段。研究者们逐渐转向机器学习,这是一种让计算机通过学习经验来改进性能的方法。机器学习的基本思想是通过训练模型来识别模式,并从数据中提取知识。监督学习、无监督学习和强化学习等不同类型的机器学习算法开始崭露头角。

机器学习中的几个关键概念包括:

\textbf{数据表示}:数据需要以计算机可以处理的形式表示,常见的有表格、向量等。良好的数据表示对于构建一个有效的机器学习系统至关重要。线性代数中向量、矩阵、特征值分解等概念,为表达数据与模型,以及设计算法奠定了框架。例如,多数机器学习算法会将特征转换为向量表示,并基于向量空间中的几何关系进行建模。