% 电磁推进
% license CCBYSA3
% type Wiki

(本文根据 CC-BY-SA 协议转载自原搜狗科学百科对英文维基百科的翻译)

电磁推进是利用流动的电流和磁场加速物体的一种方法。电流被用来产生一个与运动方向相反的磁场,或者给一个磁场提供电能,然后该磁场就会受到排斥力作用。当电流在磁场中流经导体时,产生的电磁力,也就是洛伦兹力,会将导体推向垂直于导体和磁场的方向。这种排斥力是电磁推进系统中产生推进力的成因。电磁推进这个术语可以用它的两个组成部分来描述:电磁——用电产生磁场,推进——推进物体的过程。当流体(液体或气体)用作移动导体时,推进系统可以称为磁流体动力驱动。尽管电磁推进和电动机推进都使用磁场和流动电流,但两者之间的一个关键区别是:电磁推进所用的电能不用于产生运动的旋转动能。

电磁推进的科学理论并不是起源于特定一个人,却应用于许多不同的领域。自从1897年约翰·芒罗发表他的虚构故事《金星之旅》开始,人们就一直梦想着使用磁体推进物体,这一想法一直持续到今天。[1]目前电磁推进的应用可以在磁悬浮列车和军用轨道炮上看到。其他应用尚未广泛使用或仍在开发中,其中包括低轨卫星的离子推进器以及船舶和潜艇的磁流体动力驱动系统。