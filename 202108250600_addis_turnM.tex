% 坐标绕轴旋转程序(Matlab)

\begin{issues}
\issueNeedCite
\end{issues}



\begin{lstlisting}[language=matlab]
% 格式 0
% [X1,Y1,Z1] = turn(X,Y,Z,A,theta)  把空间点绕原点以 A 方向为轴逆时针转动theta角
% [X1,Y1,Z1] = turn(X,Y,Z,D1,D2)  把空间点由 D1 方向绕原点转到 D2 方向
% numel(X) = numel(Y) = numel(Z)
% numel(D1) = numel(D2) = numel(A) = 3
% numel(theta) = 1

% 格式 1
% P1 = turn(P,A,theta) 把空间点绕原点以 A 方向为轴逆时针转动 theta 角
% P1 = turn(P,D1,D2)  把空间点由 D1 方向绕原点转到 D2 方向
% P, P1 的每行是一个点, size(P) = size(P1) = [N,3]

function [X1, Y1, Z1] = turn(X, Y, Z, A, theta)

% 格式检查
flag = 0; % flag = 0 表示格式 0, flag = 1 表示格式 1
if nargout == 1 && nargin == 3  % 格式检查
    A = Y; theta = Z;
    Y = X(:,2); Z = X(:,3); X = X(:,1);
    flag = 1;
elseif numel(X) ~= numel(Y) || numel(Y) ~= numel(Z) %格式 0 要求 xyz 元素个数相同即可
    error('numel of x, y, z is not equal')
end

if length(A(:)) ~= 3 % 两种格式都要求 A 是一个矢量
    error('length A is not 3');
end

if numel(theta) == 3
    D1 = vunit(A);
    D2 = vunit(theta);
    A = cross(D1,D2);
    C = dot(D1,D2);
    S = sqrt(1-C^2);
elseif numel(theta) == 1
    C = cos(theta);
    S = sin(theta);
else
    error('numel(theta)~=1 or numel(A2)~=3'); 
end

normA = sqrt(A(1)^2 + A(2)^2 + A(3)^3);
if normA == 0
    X1 = X; Y1 = Y; Z1 = Z;
    return;
end
A0 = A / normA;
Ax = A0(1); Ay = A0(2); Az = A0(3);
C1 = 1 - C;

M = ...  % 见小时百科旋转矩阵
[
C1*Ax*Ax+C      C1*Ax*Ay-Az*S   C1*Ax*Az+Ay*S
C1*Ay*Ax+Az*S  C1*Ay*Ay+C       C1*Ay*Az-Ax*S
C1*Az*Ax-Ay*S   C1*Az*Ay+Ax*S  C1*Az*Az+C 
];

XYZ1 = M*[X(:)';Y(:)';Z(:)'];

if flag == 0 % 原格式
    X1 = zeros(size(X));
    Y1 = zeros(size(Y));
    Z1 = zeros(size(Z));

    X1(:)=XYZ1(1,:);
    Y1(:)=XYZ1(2,:);
    Z1(:)=XYZ1(3,:);
else  % 新格式
    X1 = XYZ1';
    Y1 = []; Z1 = [];
end
end
\end{lstlisting}
