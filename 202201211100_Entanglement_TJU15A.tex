% 天津大学 2015 年考研量子力学答案
% 考研|天津大学|量子力学|2015|答案

\begin{issues}
\issueDraft
\issueTODO
\end{issues}

\subsection{ }
由归一化条件$\displaystyle (\sqrt{\frac{1}{3}}A)^{2} + (\sqrt{\frac{2}{3}}A)^{2} = 1 $可得到$A=1$.故归一化函数为:\\

$\displaystyle \psi(x) = \sqrt{\frac{1}{3}}\phi_{210}(x)+\sqrt{\frac{2}{3}} \phi_{310}(x)$ \\

\subsection{ }
\begin{enumerate}
\item 由题可得:
\begin{equation}
\begin{aligned}
\left[L^{2},L_{x}\right] =& [ L^{2}_{x} + L^{2}_{y} + L^{2}_{z} , L_{x}] \\
=& 0 + [L^{2}_{y},L_{x}] + [L^{2}_{z},L_{x}] \\
=& 0 - L_y[L_y,L_x]+[L_y,L_x]L_y \\
=& 0 - i\hbar L_y L_z - i\hbar L_z L_y + i\hbar L_z L_y + i\hbar L_y L_z \\
=& 0
\end{aligned}
\end{equation}
同理可得:
\begin{equation}
\begin{aligned}
\left[\hat{L}_+,\hat{L}_z\right]Y_{lm}(\theta ,\phi) =& \left[\hat{L}_{x}+i\hat{L}_{y} ,\hat{L}_z \right]Y_{lm}(\theta ,\phi) \\
=& \left[\hat{L}_x ,\hat{L}_z \right]Y_{lm}(\theta ,\phi) + i\left[\hat{L}_x ,\hat{L}_z \right]Y_{lm}(\theta ,\phi) \\
=& -\hbar \hat{L}_{+}Y_{lm}(\theta ,\phi)
\end{aligned}
\end{equation}

\begin{equation}
\begin{aligned}
\left[\hat{L}_{-},\hat{L}_{z}\right]Y_{lm}(\theta ,\phi) =& \left[\hat{L}_{x}-i\hat{L}_{y} ,\hat{L}_z \right]Y_{lm}(\theta ,\phi) \\
=& \left[\hat{L}_x ,\hat{L}_z \right]Y_{lm}(\theta ,\phi) - i\left[\hat{L}_x ,\hat{L}_z \right]Y_{lm}(\theta ,\phi) \\
=& \hbar \hat{L}_{-}Y_{lm}(\theta ,\phi)
\end{aligned}
\end{equation}

\end{enumerate}