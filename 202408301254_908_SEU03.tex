% 东南大学 2003 年 考研 量子力学
% license Usr
% type Note

\textbf{声明}:“该内容来源于网络公开资料,不保证真实性,如有侵权请联系管理员”

\subsection{[30分] 简答题}
 \begin{enumerate}
        \item 从列举实验得知,辐射或力学体系的能量 $E$ 以何种性质存在?
        \item 在电场力产生之前,一大问题是如何摆脱原子束缚辐射。在电场力产生之后,一大问题是如何使处于激发态的原子失去能量,这种转变是如何使激发态辐射衰变的?
        \item 任何反常量子效应要经过解释之。
        \item 马克效应表象中的不含有 Schrödinger 矩阵。
        \item 马克现象的后续是转移值与转变。
        \item 这一组量子 Hamiltonian 是 $H = \frac{p^2}{2m} + V(x)$,描述表象特性,以及从中确定其。
    \end{enumerate}

\subsection{[20分]}
设 $\psi_1$ 和 $\psi_2$ 是 Schrödinger 方程的解,证明
\[
\frac{d}{dt} \int d^3r \, \psi_1^* (\vec{r}, t) \psi_2 (\vec{r}, t) = 0.~
\]

\subsection{[10分]}
设粒子处于一维无限深势阱中:
\[
V(x) =
\begin{cases}
0, & |x| < a_2 \\
\infty, & |x| > a_2
\end{cases}~
\]
处于基态 ($n=1$) 求粒子的动量分布。

\subsection{[10分]}
设粒子处于 $Y_{lm}(\theta, \varphi)$ 状态下,求 $\overline{(\Delta L_x)^2}$ 和$\overline{(\Delta L_y)^2}$。

\subsection{[10分]}
证明:体系的任何状态下,甚厄密算符的平均值必为实数。

\subsection{[15分]}
设有两个自由粒子都处于动量本征值(本征值为$\hbar \vec{r}_\alpha, \ \hbar \vec{r}_\beta$)。分三种情况讨论空间的相对距离的几率分布:\\
(1) 两个非全同粒子。\\
(2) 两个费本子。\\
(3) 两个玻色子。

\subsection{[15分]}
荷电为$\delta$的拉子在均匀外磁场$B$中运动,求能量本征值和本征函数。
