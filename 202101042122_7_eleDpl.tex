% 电偶极子
% keys 偶极子|电场|电荷

% 这是一级词条, 先介绍两个点电荷

\pentry{电场\upref{Efield}}

\subsection{电偶极子}

在电磁学和基础的电动力学框架下,“电荷”可以被认为是一种粒子,它没有大小,同时具有一个特征,该特征用标量(数字)来刻画,也就是电荷.电荷能产生电场,同时由于电荷是一个标量,没有方向之分,这种电场也就是球对称的.

现在我们介绍的模型可以被认为是一种新的粒子,也没有大小,同时具有一个特征,称作电偶极矩,但是该特征用一个向量来刻画.电偶极子也能产生电场,但是由于电偶极矩是向量,有方向之分,这种电场就不再是球对称的了.我们从实际模型出发,来导入电偶极子的概念.

\begin{definition}{物理电偶极子}
在电磁学框架下,给定两个等量异号的电荷,假设它们的电荷量分别是$\pm q$,其中$q\geq 0$.如果负电荷到正电荷的位移向量是$\bvec{r}$,那么我们把这两个粒子构成的系统看成一个叫\textbf{物理电偶极子(physical electric dipole)}的对象.物理电偶极子电荷量为$0$,但是它有另一个性质叫\textbf{电偶极矩(electric dipole moment)}来取代电荷的概念.电偶极矩定义为$q\bvec{r}$,是一个向量.
\end{definition}

电磁学框架下的“粒子”,是没有大小但可以具有确定空间位置的一个存在.我们希望用同样的视角来看待电偶极子,把它也当成一个没有大小但是具有确定空间位置的一个存在.物理电偶极子显然不能达到这个效果,因为它必须有长度,正负电荷的位置是不一样的——如果一样,电偶极矩就为$0$了.不过,如果我们考虑问题的空间尺度远大于电偶极子的长度,那么我们可以近似地看成正负电荷的位置是一样的,这样就可以确定电偶极子的位置了.更进一步,为了形式上简洁,我们抽象出了一个原本不存在的新概念,理想电偶极子.

\begin{definition}{理想电偶极子}
在电磁学框架下引入新的粒子,称作\textbf{理想电偶极子(ideal electric dipole)}.理想电偶极子是一个点粒子,其标量电荷量为$0$,但是有一个向量\textbf{电偶极矩(electric dipole moment)}.理想电偶极子能产生电场,该电场的分布取决于电偶极矩的大小和方向.
\end{definition}

理想电偶极子是物理电偶极子的抽象,抹去了“大小”的概念.我们希望的是,在可以忽略物理电偶极子长度的时候,我们可以直接把它们视为理想电偶极子.这就使得理想电偶极子的性质需要从物理电偶极子来导出了.这些性质包括理想电偶极子是如何产生电场的.导出的方法是,我们首先计算物理电偶极子的电场分布,这是电磁学框架下可以做到的;然后在\textbf{保持电偶极矩不变}的情况下\textbf{缩短两电荷间的距离}.正负电荷的间距趋于$0$时的电场分布,就是理想电偶极子的电场分布了.

我们接下来就应用这个思想来推导电偶极子产生的电场和电势.


\subsection{物理电偶极子的电场}

令空间中两个位置不同的点电荷具有等量的异号电荷, 则他们构成一对\textbf{电偶极子(electric dipole)}. 令他们的电荷量分别为 $q_1$ 和 $q_2$ ($q_1 + q_2 = 0$), 位置矢量分别为 $\bvec r_1$ 和 $\bvec r_2$, 则它们的总电场为两个电荷各自电场的矢量和(见\autoref{Efield_eq2}~\upref{Efield})

\begin{equation}
\bvec E(\bvec r) = \frac{1}{4\pi\epsilon_0} \qty(\frac{q_1(\bvec r - \bvec r_1)}{\abs{\bvec r - \bvec r_1}^3} + \frac{q_2(\bvec r - \bvec r_2)}{\abs{\bvec r - \bvec r_2}^3})
\end{equation}
总电势同样是两个点电荷的电势之和% 链接未完成: N 个点电荷的电势
\begin{equation}
V(\bvec r) = \frac{1}{4\pi\epsilon_0} \qty(\frac{q_1}{\abs{\bvec r - \bvec r_1}} + \frac{q_2}{\abs{\bvec r - \bvec r_2}})
\end{equation}

我们常常讨论的是偶极子远处的电势和电场分布, 即 $r \gg \abs{\bvec r_2 - \bvec r_1}$ 的情况. 定义\textbf{电偶极矩(electric dipole moment)}为
\begin{equation}\label{eleDpl_eq1}
\bvec p = q_1 \bvec r_1 + q_2 \bvec r_2 = q_2 (\bvec r_2 - \bvec r_1)
\end{equation}
则电势分布为
\begin{equation}
V_d(\bvec r) = \frac{1}{4\pi\epsilon_0} \frac{\bvec p\vdot \bvec r}{r^3} = \frac{1}{4\pi\epsilon_0} \frac{\bvec p \vdot \uvec r}{r^2}
\end{equation}
由此可以求出远处的电场分布
\begin{equation}
\bvec E_d(\bvec r) = \frac{1}{4\pi\epsilon_0} \frac{1}{r^3} [3(\bvec p \vdot \uvec r) \uvec r - \bvec p]
\end{equation}
注意这两个量分别随 $r$ 的平方反比和三次方反比下降.

\subsection{偶极子远处电势推导}

\subsection{偶极子远处电场推导}

\begin{equation}
\bvec E_d(\bvec r) = \div V_d(\bvec r) = \frac{1}{4\pi\epsilon_0} \div  \frac{\bvec p\vdot \bvec r}{r^3}
\end{equation}
