% 乔赛亚·威拉德·吉布斯(综述)
% license CCBYSA3
% type Wiki

本文根据 CC-BY-SA 协议转载翻译自维基百科\href{https://en.wikipedia.org/wiki/Josiah_Willard_Gibbs}{相关文章}。

约西亚·威拉德·吉布斯(Josiah Willard Gibbs,/ɡɪbz/,1839年2月11日-1903年4月28日)是一位美国机械工程师和科学家,在物理学、化学和数学领域做出了基础性的理论贡献。他关于热力学应用的研究在将物理化学转变为一门严谨的演绎科学方面起到了关键作用。吉布斯与詹姆斯·克拉克·麦克斯韦和路德维希·玻尔兹曼一起创立了统计力学(该术语由他提出),将热力学定律解释为由大量粒子组成的物理系统可能状态集合的统计特性所导致的结果。吉布斯还研究了麦克斯韦方程在物理光学问题中的应用。作为数学家,他独立于英国科学家奥利弗·赫维赛德(后者在同一时期进行了类似的工作)创建了现代向量分析,并在傅里叶分析理论中描述了“吉布斯现象”。

1863年,耶鲁大学授予吉布斯美国首个工程学博士学位。在欧洲度过三年后,吉布斯将余生的职业生涯都奉献给了耶鲁大学,自1871年起担任数学物理学教授,直到1903年去世。他在相对孤立的环境中工作,成为美国最早获得国际声誉的理论科学家之一,曾被阿尔伯特·爱因斯坦称为“美国历史上最伟大的头脑”。

1901年,吉布斯因其在数学物理方面的贡献,获得了当时国际科学界最高荣誉——由伦敦皇家学会颁发的科普利奖章。

评论家和传记作家们曾指出,吉布斯宁静而孤独的新英格兰生活方式与其思想在国际上的巨大影响之间形成了鲜明对比。尽管他的研究几乎完全是理论性的,但随着20世纪上半叶工业化学的发展,吉布斯成果的实际价值逐渐显现。正如罗伯特·A·密立根所言,在纯科学领域,吉布斯“对于统计力学和热力学的贡献,就如拉普拉斯之于天体力学,麦克斯韦之于电动力学——他几乎将这个领域构建成一个完整的理论体系”。
\subsection{传记}
\subsubsection{家庭背景}
\begin{figure}[ht]
\centering
\includegraphics[width=6cm]{./figures/0453fffcb9e69ba5.png}
\caption{青年时期的威拉德·吉布斯} \label{fig_QSY_1}
\end{figure}
吉布斯出生于康涅狄格州纽黑文。他出身于一个古老的“洋基”家族,自17世纪以来,该家族不断涌现出杰出的美国牧师和学者。他是家中五个孩子中排行第四的孩子,也是父亲约西亚·威拉德·吉布斯与母亲玛丽·安娜(娘家姓范·克里夫,Mary Anna,née Van Cleve)唯一的儿子。在父系方面,他是塞缪尔·威拉德的后代,后者于1701年至1707年间曾担任哈佛学院代理校长。在母系方面,他的一位祖先是乔纳森·迪金森牧师,新泽西学院(后来的普林斯顿大学)首任校长。
“约西亚·威拉德”这个名字在吉布斯家族中代代相传,他与父亲及其他一些家族成员都使用这个名字。它源自他的一位祖先——18世纪曾任马萨诸塞湾省国务秘书的约西亚·威拉德。他的祖母默西·普雷斯科特·吉布斯(Mercy (Prescott) Gibbs)是丽贝卡·米诺特·普雷斯科特·舍曼的妹妹,而后者是美国开国元勋罗杰·舍曼的妻子。因此,吉布斯是舍曼家族的第二代近亲,也与后来涉及“阿米斯塔德号案件”的罗杰·舍曼·鲍德温是堂表亲。

吉布斯的父亲在家庭和学术界通常被称为“约西亚”,而他本人则被称为“威拉德”(Willard)。约西亚·吉布斯是一位语言学家和神学家,自1824年起担任耶鲁大学神学院的圣经文学教授,直到1861年去世。他如今最广为人知的事迹,是作为废奴主义者,在“阿米斯塔德号”事件中为非洲船员找到口译员,使他们能在审判中作证,讲述自己反抗被贩卖为奴的经历。
\subsubsection{教育经历}
威拉德·吉布斯在霍普金斯学校接受教育,并于1854年15岁时进入耶鲁学院。在耶鲁,吉布斯因数学和拉丁语方面的优异成绩而获得奖项,并于1858年以班级前列的成绩毕业。他随后留在耶鲁,成为谢菲尔德科学学院的研究生。19岁时,也就是他刚从本科毕业不久,吉布斯被选入康涅狄格艺术与科学学院,这是一个由耶鲁大学教师为主组成的学术机构。这一时期留下的文献资料相对较少,因此很难精确还原吉布斯早期职业生涯的细节。据传记作者推测,吉布斯在耶鲁大学及康涅狄格学院的主要导师与支持者,很可能是天文学家兼数学家休伯特·安森·牛顿,他是当时研究流星领域的权威,也一直是吉布斯的终生朋友和知己。1861年吉布斯父亲去世后,他继承了一笔足以维持经济独立的遗产。

年轻时期的吉布斯长期受到反复发作的肺部疾病困扰,医生担心他可能容易感染肺结核——他的母亲便因这种病去世。他还患有散光,而当时眼科医生对这种病的治疗尚不熟悉,因此吉布斯不得不自行诊断,并亲自打磨适合自己的眼镜镜片。虽然在后来,他只在阅读或从事近距离工作时才佩戴眼镜,但他体质虚弱以及视力不佳,很可能是他在1861至1865年的南北战争期间没有主动参军的原因。他也未被征召入伍,而是一直留在耶鲁大学度过了整个战时期间。

1863年,吉布斯获得了美国授予的首个工程学博士(PhD)学位,其论文题为《论直齿轮中齿的形状》,他在其中运用几何方法研究齿轮的最优设计。耶鲁大学在1861年成为美国首所提供博士学位的大学,而吉布斯的博士学位是美国在所有学科中授予的第五个博士学位。
\begin{figure}[ht]
\centering
\includegraphics[width=6cm]{./figures/20f35aef807fa103.png}
\caption{吉布斯在耶鲁担任讲师期间的照片\(^\text{[17]}\)} \label{fig_QSY_2}
\end{figure}
\subsubsection{职业生涯,1863–1873年}
获得博士学位后,吉布斯被任命为耶鲁学院的讲师,任期三年。前两年他教授拉丁语,第三年则讲授“自然哲学”(即物理学)。

1866年,吉布斯获得了一项铁路制动器的专利,并在康涅狄格艺术与科学学院发表了一篇题为《长度单位的适当量级》的论文,提出了一套使力学中所使用的度量单位系统更加合理化的方案。

讲师任期结束后,吉布斯与两位姐妹一同前往欧洲旅行。1866至1867年的冬季,他们在巴黎度过,吉布斯在那里听取了索邦大学和法兰西学院的讲座,授课人包括著名数学科学家约瑟夫·刘维尔和米歇尔·沙尔。

由于学习强度过大,吉布斯染上重感冒,医生担心他患上肺结核,建议他前往里维埃拉休养。他与姐妹在那里待了几个月,最终完全康复。

随后,吉布斯前往柏林,听取了数学家卡尔·魏尔施特拉斯和利奥波德·克罗内克以及化学家海因里希·古斯塔夫·马格努斯的课程。1867年8月,吉布斯的妹妹朱莉娅在柏林与亚迪森·范·内姆结婚,后者是吉布斯在耶鲁的同班同学。新婚夫妇返回纽黑文,而吉布斯与妹妹安娜继续留在德国。

在海德堡,吉布斯接触到了物理学家古斯塔夫·基尔霍夫、赫尔曼·冯·亥姆霍兹(Hermann von Helmholtz)以及化学家罗伯特·本生(Robert Bunsen)的研究成果。当时,德国学术界在自然科学领域,尤其是在化学与热力学方面处于世界领先地位。

1869年6月,吉布斯返回耶鲁,并短期教授工程学生法语。据推测,他也大约在此时设计了一种新的蒸汽机调速器,这是他在机械工程领域最后一次重要研究。

1871年,吉布斯被任命为耶鲁大学的数学物理学教授,这是美国首个此类教授职位。当时的吉布斯有经济独立来源,尚未发表任何著作,因此学校安排他只教授研究生课程,并以无薪身份聘用他。
\subsubsection{职业生涯,1863–1873}
毕业后,吉布斯被任命为耶鲁学院的讲师,任期三年。前三年中,他前两年教授拉丁语,第三年教授“自然哲学”(即物理学)。1866年,吉布斯为一种铁路制动器的设计申请了专利,并在康涅狄格艺术与科学学院宣读了一篇题为《长度单位的适当量级》的论文,提出了一个使力学中所用度量单位体系更为合理化的方案。

讲师任期结束后,吉布斯与两位姐妹前往欧洲旅行。1866至1867年的冬季,他们在巴黎度过,吉布斯在那里听取了索邦大学和法兰西学院的讲座,授课者包括著名数学科学家约瑟夫·刘维尔和米歇尔·沙尔。由于学习强度极大,吉布斯染上重感冒,医生担心他患上肺结核,建议他前往里维埃拉(地中海沿岸)休养。他与姐妹在那里度过了几个月,最终完全康复。

随后,吉布斯前往柏林,听取了数学家卡尔·魏尔施特拉斯、利奥波德·克罗内克以及化学家海因里希·古斯塔夫·马格努斯的课程。1867年8月,吉布斯的妹妹朱莉娅在柏林与亚迪森·范·内姆结婚,后者是吉布斯在耶鲁的同班同学。新婚夫妇返回纽黑文,吉布斯与妹妹安娜则继续留在德国。在海德堡,吉布斯接触到物理学家古斯塔夫·基尔霍夫和赫尔曼·冯·亥姆霍兹以及化学家罗伯特·本生的研究。当时,德国学界在自然科学,尤其是化学与热力学方面处于世界领先地位。

1869年6月,吉布斯返回耶鲁,曾短期为工程专业学生教授法语。据推测,他也大约在这段时间设计了一种新的蒸汽机调速器,这是他在机械工程领域的最后一项重要研究。1871年,吉布斯被任命为耶鲁大学数学物理学教授,这是美国历史上首个此类教授职位。当时吉布斯有经济来源,尚未发表任何学术论文,因此学校安排他专门教授研究生课程,并以无薪身份聘任他。
\subsubsection{职业生涯,1873–1880 年}
\begin{figure}[ht]
\centering
\includegraphics[width=6cm]{./figures/28f1c6e64a8baec8.png}
\caption{麦克斯韦为构建一个基于吉布斯所定义的水的热力学曲面模型而绘制的恒温线和恒压线草图(见“麦克斯韦的热力学曲面”)。} \label{fig_QSY_3}
\end{figure}
吉布斯于1873年发表了他的第一部作品。他关于热力学量几何表示法的论文刊登在《康涅狄格艺术与科学学院学报》上。这些论文引入了不同类型的相图,这些图形工具是他在研究过程中最喜欢的想象辅助工具——相比之下,他并不依赖麦克斯韦在构建其电磁理论时所使用的机械模型,因为后者可能无法完全代表所对应的物理现象。尽管该期刊的读者很少有人能够理解吉布斯的研究内容,他仍将论文的抽印本寄给欧洲的同行,其中一位是剑桥大学的詹姆斯·克拉克·麦克斯韦,后者对此热情回应。麦克斯韦甚至亲手用黏土制作了一个模型,以展示吉布斯的构造。他随后还制作了两个石膏模型,其中一个寄送给了吉布斯。该石膏模型如今陈列在耶鲁大学物理系。

麦克斯韦在1875年出版的新版《热学理论》中专门增设了一章介绍吉布斯的研究成果。他在伦敦化学学会的一次讲座中讲解了吉布斯图示方法的实用性,并在他为《大英百科全书》撰写的“图示”条目中也提到了这项工作。然而,他与吉布斯可能展开合作的希望随着麦克斯韦于1879年英年早逝(享年48岁)而终结。纽黑文后来流传起这样一句玩笑话:“世上只有一个人能看懂吉布斯的论文,那就是麦克斯韦——可他已经去世了。”

随后,吉布斯将他的热力学分析扩展到多相化学系统(即包含多种物质形态的系统),并探讨了各种具体应用。他将这些研究成果整理为一篇题为《非均相物质的平衡》的专著,由康涅狄格艺术与科学学院出版,分两部分分别于1875年和1878年发表。这部作品长约三百页,共包含整整七百个编号的数学方程。开篇引用了鲁道夫·克劳修斯的一句名言,这句话后来被视为热力学第一定律和第二定律的简明表述:“世界的能量是恒定的;世界的熵趋于最大。”

吉布斯的这部专著以严谨而巧妙的方式将他的热力学技术应用于物理化学现象的解释,将原本零散的事实与观察加以统一并相互关联。这部作品被誉为“热力学的《原理》”,也是一部“几乎没有限制的巨著”。它坚实地奠定了物理化学的基础。

将该专著翻译成德文的威廉·奥斯特瓦尔德称吉布斯为“化学能学的奠基人”。据现代评论者指出:

“这一著作的发表被公认为化学史上具有首要意义的事件……尽管如此,这部作品的价值在相当多年后才广泛为人所知。这种延迟主要源于其高度数学化的表达形式与严密的演绎过程,使得阅读极为困难,尤其对那些恰恰最相关的实验化学学生而言更是如此。”
——J. J. O'Connor 与 E. F. Robertson,1997年

吉布斯一直无薪工作,直到1880年马里兰州巴尔的摩的新成立的约翰斯·霍普金斯大学向他提供了一份年薪3,000美元的职位。作为回应,耶鲁大学为他开出每年2,000美元的薪资,吉布斯对此表示满意并接受了。

1879年,吉布斯推导出了吉布斯–阿佩尔运动方程,该方程在1900年被保罗·埃米尔·阿佩尔重新发现。
\subsubsection{职业生涯,1880–1903 年}
\begin{figure}[ht]
\centering
\includegraphics[width=6cm]{./figures/2c0717e94ba4c38a.png}
\caption{耶鲁大学的斯隆物理实验室,1882年至1931年间位于现今乔纳森·爱德华兹学院所在地。吉布斯的办公室位于二楼,在图中塔楼右侧的位置。} \label{fig_QSY_4}
\end{figure}
从1880年到1884年,吉布斯致力于将赫尔曼·格拉斯曼提出的外代数发展为一种适合物理学家使用的向量分析工具。为此目的,吉布斯区分了两个向量之间的点乘和叉乘,并引入了“二向量”的概念。与此类似的研究几乎在同一时期也由英国数理物理学家兼工程师奥利弗·赫维赛德独立开展。吉布斯试图说服其他物理学家,相较于当时英国科学界广泛使用的威廉·罗恩·哈密顿四元数方法,向量方法更加简洁实用。因此,在19世纪90年代初,他与彼得·格思里·泰特等人围绕这一问题,在《自然》杂志上展开了一场争论。

吉布斯关于向量分析的讲义笔记于1881年和1884年以私印形式印制,仅供其学生使用。后来,这些内容由埃德温·比德韦尔·威尔逊整理改编为教科书《向量分析》,于1901年出版。这本书有助于推广今天在电动力学和流体力学中广泛使用的“nabla”(∇)符号表示法。在其他数学研究中,吉布斯在傅里叶级数理论中重新发现了后来被称为“吉布斯现象”的现象(而他本人以及后来的学者并未意识到,早在五十年前,一位名不见经传的英国数学家亨利·威尔布拉罕就曾描述过这一现象)。
\begin{figure}[ht]
\centering
\includegraphics[width=10cm]{./figures/384d09c1bfe8ee06.png}
\caption{正弦积分函数,用于描述阶跃函数在实数轴上的傅里叶级数所产生的吉布斯现象中的超调现象。} \label{fig_QSY_5}
\end{figure}
从1882年到1889年,吉布斯发表了五篇关于物理光学的论文,研究双折射及其他光学现象,并为麦克斯韦的电磁光理论辩护,反对开尔文勋爵等人提出的机械光理论。在光学研究中,正如在热力学研究中一样,吉布斯有意避免对物质的微观结构进行臆测,而是有意识地将研究问题限定在那些可以通过广泛的一般性原理和实验可证实的事实来解决的范围内。他采用的方法极具原创性,所得结果也有力地证明了麦克斯韦电磁理论的正确性。

吉布斯创造了“统计力学”这一术语,并在描述物理系统的数学框架中引入了一些关键概念,包括化学势(1876年)和统计系综(1902年)等。吉布斯通过多粒子系统的统计性质推导出热力学定律,并将这项重要成果总结于其极具影响力的教科书《统计力学的基本原理》中,该书于1902年出版,即他去世前一年。

吉布斯性格内向,专注于学术工作,这使得他与学生的交流相对有限。他最主要的门生是埃德温·比德韦尔·威尔逊,但他也曾表示:“除了在课堂上,我很少见到吉布斯。他有个习惯,在下午接近尾声时,会在斯隆实验室与家之间的街道上散步——这是他在工作与晚餐之间的锻炼——偶尔可以在那时遇见他。”尽管如此,吉布斯仍曾指导欧文·费雪于1891年完成了一篇关于数理经济学的博士论文。吉布斯去世后,费雪出资资助了其《文集》的出版。另一位著名学生是李·德福雷斯特,他后来成为无线电技术的先驱。

吉布斯于1903年4月28日在纽黑文去世,终年64岁,死因为急性肠梗阻。两天后在他位于海街121号的住所举行了葬礼,他的遗体安葬在附近的格罗夫街公墓。同年5月,耶鲁大学在斯隆实验室举办了纪念会。英国著名物理学家J. J. 汤姆孙(J. J. Thomson)亲临出席,并发表了简短的悼词。
\subsubsection{个人生活与性格}
吉布斯终身未婚,一直与妹妹朱莉娅及其丈夫亚迪森·范·内姆(Addison Van Name)同住在他儿时的家中。范·内姆是耶鲁大学图书馆馆长。除了他惯常的夏季假期——早期在阿迪朗达克山区(纽约州金谷),后来在白山地区(新罕布什尔州因特瓦尔)——以及1866至1869年在欧洲的旅居之外,吉布斯几乎从未离开过纽黑文。他在大一结束时加入了耶鲁学院教堂(属于公理会),并在此后一直定期参加礼拜。在总统选举中,吉布斯通常投票支持共和党候选人,但像其他“穆格温派”一样,他对政党机器政治日益增长的腐败感到担忧,因此在1884年大选中转而支持保守派民主党人格罗弗·克利夫兰。关于他的宗教信仰或政治立场,所知甚少,因为他对这些话题一向保持沉默。

吉布斯并没有留下大量的私人信件,他的许多信件后来也遗失或被销毁。除了与其科研相关的技术性著作外,他一生仅发表过两篇其他类型的文章:一篇是为鲁道夫·克劳修斯撰写的简短讣告(克劳修斯是热力学数学理论的奠基人之一),另一篇则是为他在耶鲁的导师休伯特·安森·牛顿(H. A. Newton)撰写的较长传记回忆录。

在埃德温·比德韦尔·威尔逊看来:

“吉布斯并非为了个人声望而推销自己,也不是科学的鼓吹者;他是一个学者,出身于一个古老的学术世家,生活在科研尚未变成‘功利研究’的年代……吉布斯不是怪人,没有引人注目的言行举止,而是一个和善而庄重的绅士。”
——E. B. 威尔逊,1931年

根据林德·惠勒的描述——他曾是吉布斯在耶鲁的学生——晚年时的吉布斯:

“总是衣着整洁,外出通常戴一顶毡帽,从未表现出人们常以为与天才密不可分的那些身体怪癖或行为特征……他态度亲切但不张扬,清晰传达出他本性中的朴素与真诚。”
——林德·惠勒,1951年

他是一位谨慎的投资者和理财人,1903年去世时,其遗产估值为10万美元(相当于今天约350万美元)。多年来,他担任母校霍普金斯学校的受托人、秘书及财务主管。美国总统切斯特·A·阿瑟曾任命他为国家电气会议的委员之一,该会议于1884年9月在费城召开,吉布斯还主持了其中一场会议。吉布斯是一位热情而技艺娴熟的骑手,人们常常在纽黑文看到他驾驶着妹妹的马车出行。

在《美国科学杂志》刊登的一篇讣告中,吉布斯的学生亨利·A·邦斯特德如此评价其人格:

“他举止谦逊,待人亲切和蔼,从不表现出不耐烦或恼怒,毫无低俗的个人野心,也不曾有丝毫想要抬高自己的欲望。他几乎完美地体现了无私、基督徒绅士的理想。在那些了解他的人心中,他卓越的智识成就永远不会掩盖他高尚而庄严的人生风范。”

——H. A. 邦斯特德,1903年
\subsection{主要科学贡献}
\subsubsection{化学与电化学热力学}
\begin{figure}[ht]
\centering
\includegraphics[width=6cm]{./figures/fd638984c3ac94ed.png}
\caption{吉布斯于1873年发表的论文中所绘制的自由能图示。图中展示了一个恒定体积的平面,该平面通过点 A,代表物体的初始状态。曲线 MN 是“耗散能量曲面”的截面。AD 和 AE 分别表示初始状态下的能量(ε)和熵(η)。AB 是“可用能量”(即现在所称的亥姆霍兹自由能),而 AC 则是“熵容量”(即在不改变能量和体积的前提下,熵可增加的量)。} \label{fig_QSY_6}
\end{figure}
吉布斯在19世纪70年代发表的论文中引入了这样一个观点:将系统的内能 $U$ 表示为熵 $S$ 的函数,除了通常使用的状态变量体积 $V$、压强 $p$ 和温度 $T$ 之外。他还引入了化学势$\mu$ 的概念,将其定义为在熵和体积保持不变的条件下,系统中某种化学物质的分子数 $N$ 增加所引起的内能 $U$ 的变化率。

因此,正是吉布斯首次将热力学第一定律和第二定律结合起来,用如下形式表达一个封闭系统内能的无穷小变化 $\mathrm{d}U$:
$$
\mathrm{d}U = T\,\mathrm{d}S - p\,\mathrm{d}V + \sum_i \mu_i\,\mathrm{d}N_i~
$$
其中,$T$ 是绝对温度,$p$ 是压强,$\mathrm{d}S$ 是熵的无穷小变化,$\mathrm{d}V$ 是体积的无穷小变化。最后一项是对所有参与化学反应的化学物种求和:每一种物种的**化学势** $\mu_i$ 与其摩尔数的无穷小变化 $\mathrm{d}N_i$ 的乘积之和。

通过对上述表达式进行勒让德变换,吉布斯定义了焓$H$ 和吉布斯自由能$G$ 的概念:
$$
G(p, T) = H - T S~
$$
与此相对,亥姆霍兹自由能$A$ 的表达式为:
$$
A(v, T) = U - T S~
$$
其中:

当一个化学反应的吉布斯自由能变化为负时,该反应将自发进行。当化学系统处于平衡状态时,吉布斯自由能的变化为零。平衡常数与反应物处于标准态时的自由能变化之间的关系为:
$$
\Delta G^\ominus = -RT \ln K^\ominus~
$$
化学势通常被定义为偏摩尔吉布斯自由能,其表达式为:
$$
\mu_i = \left( \frac{\partial G}{\partial N_i} \right)_{T, P, N_{j \ne i}}~
$$
即在温度 $T$、压强 $P$ 以及其他组分摩尔数不变的条件下,吉布斯自由能对第 $i$ 种组分摩尔数的偏导数。吉布斯还推导出了后来被称为吉布斯–杜亥姆方程的关系式。

在一个以电动势 $\mathcal{E}$ 和转移电荷量 $Q$ 为特征的电化学反应中,吉布斯的基本方程变为:
$$
\mathrm{d}U = T\mathrm{d}S - p\mathrm{d}V + \mathcal{E}\,\mathrm{d}Q~
$$
\begin{figure}[ht]
\centering
\includegraphics[width=8cm]{./figures/ce6f3818e11edaef.png}
\caption{用于研究铁–氮体系相律的装置,美国固定氮研究实验室,1930年} \label{fig_QSY_7}
\end{figure}
吉布斯于1874年至1878年发表的论文《非均相物质的平衡》如今被视为化学发展史上的一个里程碑。在这篇论文中,吉布斯为多种传输现象建立了严密的数学理论,包括吸附、电化学以及流体混合物中的马兰戈尼效应。

他还提出了著名的相律:
$$
F = C - P + 2~
$$
其中:$F$ 表示在平衡状态下可以独立控制的变量个数,$C$ 为系统中组分(成分)数,$P$ 为存在的相数(不同物质状态,如气、液、固等)。相律在冶金学、矿物学和岩石学等多个领域都具有重要应用价值,也可用于物理化学中的多种研究问题。
\subsubsection{统计力学}
吉布斯与詹姆斯·克拉克·麦克斯韦和路德维希·玻尔兹曼一道,共同创立了“统计力学”这一学科,该术语正是由吉布斯提出,用以指代这样一个理论物理分支:通过研究由大量粒子组成的系统在所有可能物理状态组成的系综中的统计分布,从而解释其观察到的热力学性质。

他引入了“力学系统的相”这一概念,并据此定义了微正则系综、正则系综和巨正则系综,这些都与吉布斯测度相关,从而比麦克斯韦和玻尔兹曼更进一步地提出了适用于多粒子系统统计性质的更一般化的数学框架。

吉布斯还推广了玻尔兹曼对熵 $S$ 的统计解释,他将任意系综的熵定义为:
$$
S = -k_B \sum_i p_i \ln p_i~
$$
其中:$k_B$ 是玻尔兹曼常数,求和范围为系统所有可能的微观状态 $i$,$p_i$ 是系统处于状态 $i$ 的概率。这一公式即“吉布斯熵公式”,后来在克劳德·香农的信息论中发挥了核心作用,因此它也常被视为现代热力学信息论解释的基础。

根据亨利·庞加莱于1904年的评论,尽管麦克斯韦和玻尔兹曼此前已用概率的语言解释了宏观物理过程的不可逆性,但“看得最清楚的人,是吉布斯——尽管他的书《统计力学的基本原理》有些难读,因此阅读者寥寥。”吉布斯对不可逆性的分析,以及他对玻尔兹曼 $H$ 定理和遍历假设的表述,对20世纪的数学物理学产生了深远影响。

吉布斯非常清楚,将能量均分定理应用于大型经典粒子系统时,无法解释对固体和气体比热的实验测量结果。他因此指出,将热力学建立在关于物质结构的假设之上是有风险的。吉布斯提出的统计力学框架基于宏观上不可区分的微观状态组成的系综,即便在后来人们发现自然界的微观规律遵循量子力学(而非吉布斯和其同时代人所知的经典力学)之后,这一框架仍几乎可以完整保留。他对著名的“吉布斯佯谬”(Gibbs paradox,即关于气体混合熵变的问题)所做的解决方案,如今常被引用为量子力学中粒子不可区分性原则的预示。
\subsubsection{向量分析}
\begin{figure}[ht]
\centering
\includegraphics[width=6cm]{./figures/9482c21fe12bedc6.png}
\caption{图示展示了两向量叉积的大小和方向,使用吉布斯引入的符号表示法。} \label{fig_QSY_8}
\end{figure}
包括麦克斯韦在内的英国科学家,曾依赖哈密顿的四元数来表达具有大小和方向的三维物理量(如电场和磁场)的动力学性质。然而,吉布斯借助W\.K. 克利福德在其著作《动力学要义》(Elements of Dynamic, 1888)中的分析指出:四元数的乘积实际上可以分解为两个部分:一个是一维的标量,另一个是三维的向量。因此,使用四元数会引入一些数学上的复杂性和冗余,而这些本可以通过更简洁的方式避免,从而简化教学和应用。在耶鲁大学的课堂笔记中,吉布斯为一对向量定义了两种不同的乘积形式:点乘(或称标量积)和叉乘(或称向量积),并引入了现在广泛使用的符号表示法。通过E.B. 威尔逊根据吉布斯笔记编写、于1901年出版的教科书《向量分析》,吉布斯对今天在电动力学和流体力学中仍广泛使用的向量微积分技术的发展作出了重要贡献。

在19世纪70年代末从事向量分析研究时,吉布斯发现自己的方法与格拉斯曼在其“多重代数”中所采用的方法非常相似。随后,吉布斯开始积极宣传格拉斯曼的工作,强调其理论不仅更为一般化,而且在时间上早于哈密顿的四元数代数。为了确立格拉斯曼思想的优先权,吉布斯说服了格拉斯曼的继承人,在德国出版了格拉斯曼于1840年提交给柏林大学的论文《潮汐理论》。在这篇论文中,格拉斯曼首次引入了后来被称为向量空间(或线性空间)的概念。

正如吉布斯在19世纪80至90年代所主张的那样,四元数最终几乎被物理学家完全弃用,转而采用他和奥利弗·赫维赛德各自独立发展的向量方法。吉布斯还将其向量方法应用于行星和彗星轨道的确定。他还提出了互为倒向的三向量组的概念,这一概念后来在晶体学中证明具有重要意义。
\subsubsection{物理光学}
\begin{figure}[ht]
\centering
\includegraphics[width=6cm]{./figures/e78fa4800cb14d6e.png}
\caption{方解石晶体会使光产生双折射现象(即“光的双重折射”),而吉布斯用麦克斯韦的电磁方程成功解释了这一现象。} \label{fig_QSY_9}
\end{figure}
尽管吉布斯在物理光学方面的研究如今不像他的其他工作那样广为人知,但它通过将麦克斯韦方程应用于双折射、色散和光学活性等光学过程的理论,确实为经典电磁学作出了重要贡献。

在这项工作中,吉布斯证明,这些光学现象可以完全由麦克斯韦方程来解释,而无需对物质的微观结构或电磁波传播所依赖介质(即所谓的“以太”)作出任何特别假设。吉布斯还强调,麦克斯韦方程自动保证了电磁波中纵波的不存在——这是解释光的观测特性所必需的。这种性质源自麦克斯韦方程如今所谓的“规范不变性”;而在像开尔文勋爵那样的光的机械理论中,这种结果必须通过对以太性质的人为设定才能强加出来。

在他最后一篇关于物理光学的论文中,吉布斯写道:

“就电学理论而言,它的优势在于无需发明假说,而只需应用电学科学所提供的定律。倘若不将光的运动视为电现象,我们很难解释介质的电学性质与光学性质之间的种种吻合。”

——J. W. 吉布斯,1889年

不久之后,海因里希·赫兹在德国通过实验证实了光的电磁本质。
