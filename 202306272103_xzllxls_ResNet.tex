% 残差网络
% 残差 网络 卷积网络 变体 ResNet Residual learning

\textbf{残差网络}(Resdual network, ResNet)是卷积神经网络的一种常见的变体。与原始的卷积网络所不同的是,残差网络学习的是源数据到源数据与标签之间的残差的映射。

在卷积网络的实践中,人们发现了退化(degradation)现象。随着网络深度的增加,模型的准确率会降低,也就是说,较深的网络模型反而比浅层网络有更大的训练误差和测试误差[1]。文献[1]提出的解决方法是增加一条表示恒等映射的短连接(shortcut connections),将浅层网络输出的值通过恒等映射直接传递到深层网络上,从而保证深层网络的误差至少不会超过浅层网络。

假设一个残差网络的输入张量为$\bvec x$,标签值为$\bvec y$,残差网络学习的是$F: \bvec x \rightarrow \bvec y-x$。数学表达式为:
\begin{equation}
\bvec y=F(\bvec  x;\bvec  w)+\bvec x
\end{equation}
其中,$\bvec x$和$\bvec y$分别为输入和输出张量,$F$为残差映射,$\bvec w$为权值,其值是学习而来的。

为了表示上述数学模型,网络结构方面,采用卷积网络来表示残差映射,短连接来表示关于$x$的恒等映射,最后再将$x$与$F(x)$相加。此处的加法是元素级别(element wise)对应的张量相加,因此,$x$和$F(X)$的维度要相同,否则无法运算。如下图所示:
\begin{figure}[ht]
\centering
\includegraphics[width=12cm]{./figures/5d61fa5ce1ced1f0.png}
\caption{残差网络示意图 [1]} \label{fig_ResNet}
\end{figure}


\textbf{参考文献:}
\begin{enumerate}
\item S. R. Kaiming He Xiangyu Zhang, “Deep Residual Learning for Image Recognition,” in Proceedings of the IEEE conference on computer vision and pattern recognition, 2016, pp. 770–778.
\end{enumerate}