% 勒让德变换


勒让德变换的定义很多,有的定义虽然严谨却不易理解.本节采用其最方便理解的方式来定义.

\subsection{定义}

\subsubsection{单自变量勒让德变换}
\begin{definition}{}
给定一个实单自变量函数$f(t)$,假设它是一个凸函数,即其导函数$f'(t)$严格单调.将$f'(t)$重命名为一个新的实变量$s$,如果存在一个实函数$g(s)$,使得$g'(s)=t$,那么称$g(s)$是$f(t)$的\textbf{勒让德变换(Legendre transformation)}.
\end{definition}

简单来说,勒让德变换就是有两个函数,$f(t)$和$g(s)$,它们使用不同的自变量,但是各自求导以后却能得到对方的自变量:$f'(t)=s$,$g'(s)=t$.这个关系是对称的,因此$f(t)$也是$g(s)$的勒让德变换.使用勒让德变换,可以把凸函数$f(t)$所表示的某些性质改用$g(s)$来表示,相当于用新自变量描述同样的信息.

\subsubsection{与分部积分的关系}

假设$f(t)$和$g(s)$互为彼此的勒让德变换.由分部积分可得,
\begin{equation}
\begin{aligned}
\dd{(ts)}&=s\dd{t}+t\dd{s}\\&=f'(t)\dd{t}+g'(s)\dd{s}\quad\text{这一步由勒让德变换得}\\&=\dd{f}+\dd{g}
\end{aligned}
\end{equation}

\subsubsection{多自变量勒让德变换}

有了分部积分的理解,我们就可以自然地推广出多自变量的勒让德变换.

\begin{definition}{}
给定一个实函数$f(t_1, t_2, \cdots, t_n)$,假设对于任意$i=1, 2, \cdots, n$,都有$\pdv{t_i}f$是关于$t_i$的凸函数.
\end{definition}

如何求$f(t)$勒让德变换呢?如果已知$f(t)$的表达式,那么就可以计算出$f'(t)=s$,由定义,$f'(t)$是一个单调函数,因此存在反函数$F(s)=t$.由于$g'(s)=t=F(s)$,我们就可以写出:
\begin{equation}
g(s)=\int t \dd{s}=\int F(s) \dd{s}
\end{equation}

许多情形下,我们并不知道$f(t)$的显式表达,这个方法就可能并不适用.


