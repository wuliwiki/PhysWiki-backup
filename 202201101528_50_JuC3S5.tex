% 第3章小结
% 第3章 小结

本文授权转载自郝林的 《Julia 编程基础》. 原文链接:\href{https://github.com/hyper0x/JuliaBasics/blob/master/book/ch03.md}{第3章:变量与常量}.


\subsubsection{3.5 小结}

我们在这一章主要讲的是变量.我们首先从可以与变量绑定在一起的对象——值——讲起.在 Julia 中,任何值都是有类型的.我们在定义一个变量的时候,可以为它添加标注类型(涉及到操作符\verb|::|),也可以让 Julia 自行推断其类型.我们推荐前者的做法,因为那样做会对程序的性能有好处.不过要注意,我们是无法为全局变量添加类型标注的.

代表变量的标识符、赋值符号\verb|=|和代表值的字面量共同组成了变量的定义.它把一个值与一个标识符(或称变量名)绑定在了一起.此后,这个标识符就是那个值的代表.

Julia 对变量名有所限制,但是这种限制还是很宽松的.大部分 Unicode 字符都可以被包含在内.不过,变量名不能与 Julia 中任何一个单一的(或者说单词的)关键字重复,也不能与处在同一个作用域中的其他程序定义的名称重复.

一旦理解了变量,常量的概念就很好理解了.因为 Julia 的常量就是一种特殊的变量.只不过,常量只能被定义在全局,而不能被定义在局部.另外,我们需要特别注意对常量的重新定义.

除了上述知识,我们还在本章顺便说明了程序定义、代码块、作用域等概念的含义.你最好记住它们,因为我们在后面还会有所提及.

值、变量以及类型是 Julia 中最基础的概念.没有了它们,我们就无法编写 Julia 程序.别着急,我们马上就会讲到第三个基础概念——类型.