% 狄拉克方程(综述)
% license CCBYSA3
% type Wiki

本文根据 CC-BY-SA 协议转载翻译自维基百科\href{https://en.wikipedia.org/wiki/Dirac_equation}{相关文章}。

在粒子物理学中,狄拉克方程是由英国物理学家保罗·狄拉克于1928年推导出的相对论波动方程。在其自由形式或包括电磁相互作用的情况下,它描述了所有自旋为1/2的有质量粒子,被称为“狄拉克粒子”,例如电子和夸克,这些粒子具有宇称对称性。它与量子力学原理和相对论的特殊理论一致,[1]并且是第一个在量子力学中完全考虑特殊相对论的理论。通过严格地解释氢谱的精细结构,它得到了验证。它在标准模型的构建中变得至关重要。[2]

该方程还暗示了一种新的物质形式——反物质,之前未曾被怀疑或观察到,几年的实验验证了这一点。它还为保罗的自旋现象学理论中引入多个分量波函数提供了理论依据。狄拉克理论中的波函数是四个复数值的向量(称为双自旋数),其中两个在非相对论极限下类似于保罗波函数,而与描述单一复数值波函数的薛定谔方程不同。此外,在零质量极限下,狄拉克方程简化为魏尔方程。

在量子场论的背景下,狄拉克方程被重新解释为描述与自旋1/2粒子相对应的量子场。

狄拉克没有完全意识到自己结果的重要性;然而,他关于自旋的解释——作为量子力学和相对论结合的结果——以及最终发现正电子,代表了理论物理学的伟大胜利之一。这一成就被认为与牛顿、麦克斯韦和爱因斯坦的工作相提并论。[3] 有些物理学家认为这方程是“现代物理学的真正种子”。[4] 该方程还被描述为“相对论量子力学的核心”,并且有人说“这方程可能是所有量子力学中最重要的方程”。[5]

狄拉克方程被刻在威斯敏斯特大教堂的地板上的一块纪念牌上。该纪念牌于1995年11月13日揭幕,纪念狄拉克的一生。[6]

\subsection{历史}
狄拉克方程在狄拉克最初提出的形式是:[7]:291 [8]
\[
\left( \beta mc^{2} + c \sum_{n=1}^{3} \alpha_{n} p_{n} \right) \psi(x,t) = i\hbar \frac{\partial \psi(x,t)}{\partial t}~
\]
其中,\(\psi(x,t)\) 是具有静止质量 \(m\) 的电子的波函数,\(x\) 和 \(t\) 为时空坐标,\(p_1, p_2, p_3\) 是动量的分量,被理解为薛定谔方程中的动量算符。\(c\) 是光速,\(\hbar\) 是约化普朗克常数;这些基本物理常数分别反映了特殊相对论和量子力学。

狄拉克提出此方程的目的是解释相对论性运动电子的行为,从而使得原子可以以与相对论一致的方式进行处理。他希望通过这种方式引入的修正可能对原子光谱的问题有所帮助。

在此之前,试图使旧量子理论与相对论理论兼容的努力——这些努力基于将电子在原子核周围可能非圆形轨道中存储的角动量离散化——都失败了,而海森堡、保利、约旦、薛定谔和狄拉克自己提出的新量子力学还未充分发展以处理这一问题。尽管狄拉克最初的目的已得到满足,但他的方程对物质结构有着更深远的影响,并引入了新的数学对象类,这些对象现在是基础物理学的重要元素。

方程中的新元素是四个 4 × 4 矩阵 \(\alpha_1, \alpha_2, \alpha_3\) 和 \(\beta\),以及四分量波函数 \(\psi\)。\(\psi\) 有四个分量,因为在配置空间中的任意一点处对其求值是一个双自旋数。它被解释为自旋向上的电子、自旋向下的电子、自旋向上的正电子和自旋向下的正电子的叠加。

这四个 4 × 4 矩阵 \(\alpha_k\) 和 \(\beta\) 都是厄米矩阵,并且是自反的:
\[
\alpha_i^2 = \beta^2 = I_4~
\]
它们互相反对易:
\[
\alpha_i \alpha_j + \alpha_j \alpha_i = 0 \quad (i \neq j)~
\]
\[
\alpha_i \beta + \beta \alpha_i = 0~
\]
这些矩阵及波函数的形式具有深刻的数学意义。伽马矩阵所表示的代数结构早在50年前由英国数学家 W. K. Clifford 创造。反过来,Clifford 的思想源于19世纪中期德国数学家赫尔曼·格拉斯曼在其《线性展开理论》(\textbf{Lineare Ausdehnungslehre})中的工作。











































































































































































































































































































































































































































































































































































































































































