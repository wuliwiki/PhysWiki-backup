% 反常积分
% keys 积分|定积分|微积分
\begin{issues}
\issueTODO
\issueDraft
\end{issues}

\pentry{定积分\upref{DefInt}}
\subsection{积分限为无穷的反常积分}

\subsubsection{引入}
在许多物理、数学问题中我们会遇到一些积分到无穷的积分。例如 势能(简介)\upref{POTENT} 中,我们定义势能是
$$E_P=\int ^\infty_a \bvec F_{in} \cdot \dd \bvec r$$

你可能会想,“功是力在位移上的作用量,如果物体在力的作用下运动了无穷远,那么力岂不做了无穷多的功?那这个结论还有什么意义?”结果真的是这样吗?

我们先看一个简单的例子:$\int^a_1 \frac{1}{x^2} \dd x$ (许多有心力遵循平方反比律),并令$a\to+\infty$。

\begin{table}[ht]
\centering
\caption{积分结果}\label{impro_tab1}
\begin{tabular}{|c|c|}
\hline
$a$的值 & $\int^a_1 \frac{1}{x^2} \dd x$的值 \\
\hline
$a=10$ & $0.9000$ \\
\hline
$a=100$ & $0.9900$ \\
\hline
$a=1000$ & $0.9990$ \\
\hline
$a=10000$& $0.9999$\\
\hline
\end{tabular}
\end{table}
看起来,随着$a$的增大,$\int^a_1 \frac{1}{x^2} \dd x$似乎并没有趋向无穷,而是趋向某一个确定的值$1$。这也就启发我们,即使积分区间趋向无穷大,积分结果仍然可能是有意义的。这就要求我们扩展定积分的定义,\textsl{玩一种很新的积分}。

\subsubsection{定义}
\begin{definition}{反常积分}
设函数 $f(x)$ 在 $[a, +\infty)$ 上连续(或分段连续),对于任意 $t>a$,积分 $\displaystyle \int^t_af(x)\mathrm{d} x$ 存在,则定义 $\displaystyle \int ^{+\infty}_a f(x)\mathrm{d} x=\lim_{t\rightarrow+\infty }\int _a^{+\infty}f(x)\mathrm{d} x$,并称 $\displaystyle \int ^{+\infty}_a f(x)\mathrm{d} x $ 为 $f(x)$ 在 $[a, +\infty)$ 的反常积分。

如果 $\displaystyle \lim_{t\rightarrow+\infty }\int _a^{+\infty}f(x)\mathrm{d} x$ 存在,则称反常积分 $\displaystyle \int ^{+\infty}_a f(x)\mathrm{d} x$ \textbf{收敛},且该极限值称为反常积分的值;若此极限不存在,则称反常积分 $\displaystyle \lim_{t\rightarrow+\infty }\int _a^{+\infty}f(x)\mathrm{d} x$ \textbf{发散}。
\end{definition}

同样地,可以定义在 $(-\infty,b]$ 上的连续函数 $f(x)$ 的反常积分为 $\displaystyle \int ^b _{-\infty}f(x)\mathrm{d} x=\lim_{t\rightarrow-\infty }\int ^b _{-\infty}f(x)\mathrm{d} x$

对于定义在 $(-\infty,+\infty )$ 上的连续函数 $f(x)$ 的反常积分 $\displaystyle \int ^{+\infty}_{-\infty}f(x)\mathrm{d} x$ 作如下定义
$$\displaystyle \int ^{+\infty}_{-\infty}f(x)\mathrm{d} x=\displaystyle \int ^{+\infty}_c f(x)\mathrm{d} x+\displaystyle \int ^c _{-\infty}f(x)\mathrm{d} x$$
其中 $c$ 是任意实数。当且仅当等式右边的两个积分同时收敛时,称反常积分 $\displaystyle \int ^{+\infty}_{-\infty}f(x)\mathrm{d} x$ 收敛,且右端两个积分值的和称为反常积分 $\displaystyle \int ^{+\infty}_{-\infty}f(x)\mathrm{d} x$ 的值;否则称 $\displaystyle \int ^{+\infty}_{-\infty}f(x)\mathrm{d} x$ 发散。

\subsubsection{计算}
为了计算反常积分,我们仅需要简单地推广牛顿—莱布尼兹公式\upref{NLeib}。
\begin{equation}
\displaystyle \int ^{+\infty}_a f(x)\mathrm{d} x=\lim_{t\rightarrow+\infty }\int _a^{t}f(x)\mathrm{d} x
=\lim_{t\rightarrow+\infty }F(t) - F(a)
\end{equation}
其中$F(x)$是$f(x)$的原函数。

换句话说,如果$\lim_{t\rightarrow+\infty } F(t)$存在,那么原反常积分就存在。

\begin{example}{计算$\int^{+\infty}_1 \frac{1}{x^2} \dd x$}
使用推广的牛顿—莱布尼兹公式计算上文中的$\int^{+\infty}_1 \frac{1}{x^2} \dd x$

很显然,$f(x)=\frac{1}{x^2}$的原函数是$F(x)=-\frac{1}{x}$,那么
$$\int^{+\infty}_1 \frac{1}{x^2} \dd x=\lim_{x\rightarrow+\infty } (-\frac{1}{x}) - (-\frac{1}{1})=1$$

因此,随着$a$增大、积分结果趋向于$1$并不是巧合。
\end{example}

\begin{example}{计算$\int^{+\infty}_1 \frac{1}{x} \dd x$}
$$\int^{+\infty}_1 \frac{1}{x} \dd x=\lim_{x\rightarrow+\infty } {\ln(x)} - \ln(1)$$
很遗憾,$\lim_{x\rightarrow+\infty } {\ln(x)} $是发散的,因此这个反常积分也是发散的。
\end{example}

