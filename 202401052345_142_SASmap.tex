% 对称/反对称多线性映射
% license Usr
% type Wiki

\begin{issues}
\issueTODO
\end{issues}

\pentry{多线性映射\upref{MulMap},置换的奇偶性\upref{permu},群作用\upref{Group3}}

\subsection{对称/反对称多线性映射}

回忆\autoref{ex_Group3_6}~\upref{Group3}中我们定义的 $S_k$ 置换作用,
\begin{equation}
\begin{aligned}
R_\pi: V^{\times k} &\to V^{\times k}~, \\
(v_1, \cdots, v_k) &\mapsto (v_{\pi(1)}, \cdots, v_{\pi(k)})~.
\end{aligned}
\end{equation}

\addTODO{将 $G$-不变映射的定义改写到 $G$-空间后更改引用}
\begin{definition}{对称多线性映射}
我们称一个多线性映射 $f: V^{\times k} \to W$ 为\textbf{对称多线性映射}(简称\textbf{对称 $k$-映射})如果它是 $S_k$-不变的(参考\autoref{def_GrpRep_1}~\upref{GrpRep}),即对任意置换 $\pi \in S_k$,
\begin{equation}
f(R_\pi(v)) = f(v)~.
\end{equation}

由于对称群 $S_k$ 由对换生成(见对称群\upref{Perm}),我们可以把它简化为,交换任意两项 $v_i, v_j$ 后函数值保持不变:
\begin{equation}
f(\cdots, v_j, \cdots, v_i, \cdots) = f(\cdots, v_i, \cdots, v_j, \cdots)~.
\end{equation}

如果 $f: V^{\times k} \to \mathbb{F}$ 是一个多线性型(多线性函数),我们称它为\textbf{对称多线性型}。
\end{definition}

\addTODO{例子:内积}

% Giacomo: 对称/反对称多线性型没有特殊之处。
% \begin{definition}{对称多线性型}\label{def_SASmap_1}
%
% 如果对任意 $\bvec{v}_1,\cdots\bvec{v}_p \in V$ 及任意置换\upref{Perm} $\pi\in S_p$ ,都有
% \begin{equation}
% f(\bvec{v}_{\pi(1)},\bvec{v}_{\pi(2)},\cdots,\bvec{v}_{\pi(p)})=f(\bvec{v}_1,\bvec{v}_2,\cdots,\bvec{v}_p)~,
% \end{equation}
% 而 $f$ 称为\textbf{斜对称的}(或反对称)。如果
% \begin{equation}
% f(\bvec{v}_{\pi(1)},\bvec{v}_{\pi(2)},\cdots,\bvec{v}_{\pi(p)})=\epsilon_\pi f(\bvec{v}_1,\bvec{v}_2,\cdots,\bvec{v}_p)~,
% \end{equation}
% 其中 $\epsilon_\pi$ 是置换的符号(偶置换取1,奇置换取负)\upref{permu}。
% \end{definition}

\begin{definition}{反对称多线性映射}\label{def_SASmap_1}
类似的我们可以定义的\textbf{反对称多线性映射} (简称\textbf{反对称 $k$-映射})$f: V^{\times k} \to W$ 如果它满足 $\pi \in S_k$,
\begin{equation}
f(R_\pi(v)) = \opn{sgn}(g) f(v)~.
\end{equation}
其中对于偶置换$\opn{sgn}{\sigma} = 1$、奇置换$\opn{sgn}{\sigma} = -1$;由于对称群的性质,我们可以把它简化为,交换任意两项 $v_i, v_j$ 后函数值相反:
\begin{equation}
f(\cdots, v_j, \cdots, v_i, \cdots) = - f(\cdots, v_i, \cdots, v_j, \cdots)~.
\end{equation}

如果 $f: V^{\times k} \to \mathbb{F}$ 是一个多线性型(多线性函数),我们称它为\textbf{反对称多线性型};全体反对称 $k$-形式的集合记做 $\Omega^k(V)$ (或者 $A^k(V)$)。

反对称亦称\textbf{斜对称}、\textbf{交错的}。
\end{definition}

\subsection{微分形式}

我们把向量空间 $V$ 上的反对称 $k$-形式称为\textbf{微分形式},因为我们可以定义微分算子
\begin{equation}
\begin{aligned}
\dd: \Omega^k(V) ^\to \Omega^{k + 1}(V)
\end{aligned}~
\end{equation}
