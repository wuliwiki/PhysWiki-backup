% 端点可变问题
% 端点可变|泛函极值|全微分

\begin{issues}
\issueTODO
\end{issues}

\pentry{极端曲线\upref{ExtCur}}
\subsection{问题的提出}在前面的问题中,总是取以两个固定点为端点的曲线作为可取曲线\upref{DesCur}.现在,我们研究以更广泛的曲线作为可取曲线的泛函极值问题.

设函数 $F(x,y,y')$ 满足普通的连续性及可微性条件;此外,在 $xOy$ 平面上,给定两个 $C_1$ 类的曲线 $\varphi$ 及 $\psi$:
\begin{equation}
y=\varphi(x),\quad y=\psi(x)
\end{equation}
在此规定下,我们的问题可叙述如下:

取端点分别在曲线 $\varphi$ 及曲线 $\psi$ 上的 $C_1$ 类中全体曲线 $\gamma$,作为可取曲线族.现在要求泛函
\begin{equation}\label{EPQue_eq1}
J(\gamma)=\int_\gamma F(x,y,y')\dd x
\end{equation}
的极值,其中积分是沿曲线 $\gamma$ 而取的.

注意到:若某一以 $A$ 和 $B$ 为端点的曲线 $\gamma_0$ 是本问题的解,即 $\gamma_0$ 给积分\autoref{EPQue_eq1} 以极值,那么这一曲线 $\gamma_0$ 也在连接 $A_0$ 及 $B_0$ 的所有 $C_1$ 类曲线中,给 $J$ 以极值.由最简单问题时的欧拉定理\upref{ElueEV},曲线 $\gamma_0$ 满足欧拉方程
\begin{equation}
F_y-\dv{}{x}F_{y'}=0
\end{equation}

这就是说,若以可取曲线族中的极端曲线构成新的可取曲线族,得到的使 $J$ 取极值的曲线仍是同一条曲线.
\subsection{$J(\gamma)$ 对极端曲线的微分} 
\begin{theorem}{}
泛函 $J$ \autoref{EPQue_eq1} 对极端曲线(其起点和终点坐标分别为 $x_0,y_0;x_1,y_1$)的微分为
\begin{equation}
\begin{aligned}
&\pdv{J}{x_0}=-(F-y'F_{y'})^{(0)},\quad \pdv{J}{y_0}=-F_{y'}^{(0)}\\
&\pdv{J}{x_1}=(F-y'F_{y'})^{(1)},\quad \pdv{J}{y_1}=F_{y'}^{(1)}\\
&\dd J=-[(F-y'F_{y'})^{(0)}\delta_{x_0}+ F_{y'}^{(0)}\delta y_0]+[(F-y'F_{y'})^{(1)}\delta_{x_1}+F_{y'}^{(1)}\delta y_1]
\end{aligned}
\end{equation}
其中,指标 $^{(0)},^{(1)}$ 在这里及以后(特指整个变分学)都表示对应的函数值是取在弧的起点及终点上.
\end{theorem}
\subsubsection{证明}