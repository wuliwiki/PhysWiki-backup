% 量子力学常见问题解答
% license Usr
% type Tutor

\begin{issues}
\issueDraft
\end{issues}

这里的量子力学指的是非量子场论的,也就是只考虑一次量子化。
\addTODO{考虑把以下一些内容放到量子力学的基本原理(科普)\upref{QM0} 中}

\begin{itemize}
\item 问:量子是什么?和粒子有什么关系?
\item 答:不过是翻译成中文恰好包含一同一个汉字罢了,英文中 particle(粒子) 和 quantum(量子) 并没有什么共同的词根在里面。 “量子”,或者说 “量子现象” 是对在经典力学中连续的,但在量子力学中许多离散\textbf{现象}的一种笼统称呼,并不是构成某种物质的粒子,也不是只某个具体的现象。粒子是量子力学中物质的不可分割的基本组成部分,是物质。但注意经典力学中描述的基本粒子也是不可分割的质点,所以在量子场论出现以前粒子的不连续性还算不上是量子力学特有的 “量子现象”。量子力学(非场论)并没有明确把二者联系起来。在量子场论中基本粒子是可以通过量子场激发出来的,所以粒子在产生和湮灭中的不可分割性本身也可以看作是一种量子现象。
\item 问:原子中电子处于高能级不稳定,从而跃迁至低能级,并以光子的形式放出能量。但是,进一步地,
• 原子处于高能级时,为什么会不稳定?“高能级”必然会导致“不稳定”吗?二者有真正的因果关系吗?
• 为什么原子、分子的跃迁,能量通常以光子的形式,而不是其他的形式放出?
• 为什么跃迁会导致系统光子数目的增加,其辐射光子的内在机制又是什么?
——这些问题,高中教科书中并没有提到。这也是我在高中阶段对原子物理最大的困惑。
\item 答:1.对一个不受外界干扰的真空中的孤立的高能级原子来说,自发辐射是导致不稳定的唯一因素,你可以理解为真空并不空而是存在一些“零点扰动”,这扰动导致了自发辐射。这个扰动是量子力学(非场论)所解释不了的。

2.若一个高能级原子受到外部电磁场的作用,它可能存在受激辐射或者激发到更高的能态,这是量子力学(非场论)可以解释的,但在能级改变后,为什么释放或吸收光子,是量子力学(非场论)解释不了的。

3.量子力学(非场论)甚至没有光子的概念,但是有质量的粒子在许多情况下倾向于(而不是严格规定)吸收或释放整数倍的 hv(v 是经典电磁场的频率),这已经使得光子的概念呼之欲出了。

4.为什么是光子:经典物理中光本质是电磁波,而运动的电荷会发出电磁波。量子场论也认为同样的现象是存在的,只不过具体过程和经典物理中大不相同。笔者对量子场论了解不多,但基本粒子要释放能量无非就是释放若干个费米子或者玻色子,不存在宏观系统中放热之类的能量释放。而原子能级跃迁释放的那点能量根本不可能激发出一个有质量的粒子(根据 mc^2 这需要大得多的能量释放),而公认的无质量粒子就只有光子和胶子,后者主要伴随强相互作用出现,能级跃迁只涉及电磁相互作用,所以释放光子就自然而然了。

补充/更正:基本粒子要释放能量不一定要产生新的基本粒子,也可以把能量以动能或者势能的形式传给已有基本粒子。所以原子发生能级跃迁的第三种原因就是和其他原子或者已经存在的其他基本粒子发生碰撞。例如在原子气体中,这种碰撞是大量随机发生的,这既可能导致高能态的原子跃迁到低能态,也可能反过来。所以在一定温度的气体中这会达到一种平衡,使得处于每个能级的原子数量最终几乎保持不变。又例如实验上除了用电磁波轰击原子外,也可以用电子或质子等有质量的粒子轰击原子,用于探索原子(或分子)的各种性质。
\addTODO{可以考虑专门写一个原子的科普}
\item 问:波函数就是粒子本身吗?
\item 答:波函数完整描述了微观粒子的所有状态,但并不是粒子本身。就像经典力学中质点的质量,位置,动量决定了它在某个时刻的一切状态,这三个量并不是粒子本身。 不能认为波函数的大小就是粒子大小,更不能认为波函数的概率密度就是粒子的质量密度或者电荷密度。带电粒子之间的相互作用是哈密顿算符中的一项而不是根据概率波的分布而决定的。这本来就是现有理论。但现有理论也没有说除了波函数以外还需要什么额外的信息,事实上 EPR 告诉我们没有这样的信息。

答主或者现有理论也从来没说过会出现半个粒子的情况。波函数并不只包含概率信息,而是包含粒子的一切信息,包括动量能量分布和一切可观测物理量的分布。

毕竟他只是几个数字,数字当然不是物质。波函数也是一个道理,函数也同样不是物质本身。所以你还是没有表达清楚你到底要反驳什么

至于除了波函数给出的信息之外,粒子“本身”到底是无大小的点还是小球还是三角形,这目前并没有任何实验可以证明或者证伪,所以讨论它没有意义。
\end{itemize}
