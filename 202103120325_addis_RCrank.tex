% 厄米共轭算符的映射结构
% 线性无关|列秩|行秩|零空间|解空间|正交归一|基底

\pentry{矩阵与矢量空间\upref{MatLS}, 正交子空间\upref{OrthSp}, 线性映射的结构\upref{MatLS2}, 厄米共轭\upref{HerMat}}

\begin{figure}[ht]
\centering
\includegraphics[width=10cm]{./figures/RCrank_1.pdf}
\caption{\autoref{RCrank_the1} 中的韦恩图, 三角形表示线性空间, $X_0\cap X_1$ 和 $Y_0\cap  Y_1$ 都是零向量.} \label{RCrank_fig1}
\end{figure}

\begin{theorem}{}\label{RCrank_the1}
令 $X, Y$ 为有限维线性空间, 线性算符 $A:X \to Y$ 的零空间(\autoref{LinMap_the1}~\upref{LinMap})为 $X_0$, 其厄米共轭\upref{HerMat}(也叫自伴算符) $A\Her: Y \to X$ 的零空间为 $Y_0$; 令 $Y_1 = A(X)$, $X_1 = A\Her(Y)$. 那么 $X_0, X_1$ 是 $X$ 中的正交补(\autoref{OrthSp_def1}~\upref{OrthSp}), $Y_0, Y_1$ 是 $Y$ 中的正交补. 即
\begin{equation}
X = X_0 \oplus X_1 \qquad X_0 \perp X_1
\end{equation}
\begin{equation}
Y = Y_0 \oplus Y_1 \qquad Y_0 \perp Y_1
\end{equation}
其中 $\oplus$ 表示直和\upref{DirSum}, $\perp$ 表示两空间正交\upref{OrthSp}.
\end{theorem}
证明见下文. 根据\autoref{MatLS2_the1}~\upref{MatLS2}, \autoref{RCrank_the1} 中 $X_1, Y_1$ 有一一对应关系(见\autoref{RCrank_fig1} ):
\begin{equation}\label{RCrank_eq3}
A(X_1) = Y_1 \qquad A\Her (Y_1) = X_1
\end{equation}

\begin{corollary}{}
矩阵的行秩等于列秩\upref{MatRnk}.
\end{corollary}
证明: 令矩阵 $\mat A$ 的算符为 $A$, 该矩阵的列秩就是 $Y_1$ 的维数. 而 $\mat A$ 的行秩等于 $\mat A\Her$ 的列秩, 也就是 $X_1$ 的维数. 由于 $X_1, Y_1$ 有一一对应关系(\autoref{RCrank_eq3}), 所以维数相同. 证毕.

\subsection{证明}
以下证明\autoref{RCrank_the1} 中 $Y_1$ 的正交补就是 $Y_0$ ($X_0, X_1$ 的证明同理). 令 $Y_1$ 的正交补为 $Y'_0$, 那么 $y \in Y'_0$ 的充分必要条件是 $y$ 和 $Y_1$ 中所有矢量都正交, 即
\begin{equation}\label{RCrank_eq1}
\braket{Ax}{y} = 0 \qquad (\forall x \in X)
\end{equation}
而根据零空间的定义(\autoref{LinMap_the1}~\upref{LinMap}) $y \in Y_0$ 的充分必要条件是 $A\Her y = 0$, 即
\begin{equation}\label{RCrank_eq2}
\braket*{x}{A\Her y} = 0 \qquad (\forall x \in X)
\end{equation}
而根据(链接未完成)\autoref{RCrank_eq1} 和\autoref{RCrank_eq2} 等效. 所以 $Y'_0 = Y_0$. 证毕.
