% 等差数列(高中)
% 高中|等差数列

\begin{issues}
\issueDraft
\end{issues}

\pentry{数列的概念与函数特性(高中)\upref{HsSeFu}}
\subsection{定义}
从第2项起,每一项与前一项的差是同一个常数,我们称这样的数列为\textbf{等差数列},称这个常数为等差数列的\textbf{公差},通常用字母 $d$ 表示.

\textsl{注:常数列也是等比数列.}

\subsection{通项}
如果等差数列 $\begin{Bmatrix} a_n \end{Bmatrix}$的首项是 $a_1$,公差是 $d$,那么根据等差数列的定义可得
\begin{equation}
\begin{aligned}
&a_1 = a_1,\\
&a_2 = a_1 + d,\\
&a_3 = a_2 + d = a_1 + 2d,\\
&\cdots \\
&a_n = a_{n-1} + d = a_1 + (n - 1)d
\end{aligned}
\end{equation}

当 $n = 1$ 时
\begin{equation}
a_1 = a_1 + (1 - 1)d = a_1
\end{equation}
也就是说这个公式对 $n = 1$ 同样适用.

综上,等差数列通项公式为
\begin{equation}
a_n = a_1 + (n - 1)d
\end{equation}

\textsl{注:这里需要说明一下,带入$n = 1$验算的原因是,我们推算的是$n > 1$时的通项公式,不能说明对首项成立.正如上一节所说,不是所有数列都能写出通项公式,在题目中,经常会出现首项不符合其余项通项公式的情况.}

\subsection{等差中项}
如果在 $a$ 和 $b$ 中间插入一个数 $A$,使 $a,A,b$ 成等差数列,那么 $A$ 叫作 $a$ 与 $b$ 的\textbf{等差中项}.

如果 $A$ 是 $a$ 与 $b$ 的等差中项,那么
\begin{equation}
A - a = b - A
\end{equation}
\begin{equation}
A = \frac{a+b}{2}
\end{equation}

\textsl{注:等差中项是等差数列的重要考点,大部分考察等差数列的题目都会考察等差中项.}

\subsection{前 $n$ 项的和}
等差数列求和的思路非常简单,设一个等差数列
\begin{equation}
a_1,a_2,a_3\cdots,a_n
\end{equation}
我们将这个数列倒序排列
\begin{equation}
a_n,a_{n-1},\cdots,a_1
\end{equation}
则两个数列的和为
\begin{equation}
\begin{aligned}
S &= a_1 + a_2 + \cdots + a_{n-1} + a_n \\
S &= a_n + a_{n - 1} + \cdots + a_2 + a_1
\end{aligned}
\end{equation}
由\autoref{}