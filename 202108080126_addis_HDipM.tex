% 氢原子的跃迁偶极子矩阵元列表

\pentry{氢原子的跃迁偶极子矩阵元和选择定则\upref{SelRul}}

\subsection{$z$ 方向跃迁偶极子矩阵元}
\autoref{HDipM_tab1} 给出了 $\mel{\psi_{n',l',m'}}{z}{\psi_{n,l,m}}$, 由于这是一个对称矩阵, 只给出矩阵的上半三角.
\begin{table}[ht]
\centering
\caption{$\mel{\psi_{n',l',0}}{z}{\psi_{n,l,0}}$ 的上半三角}\label{HDipM_tab1}
\begin{tabular}{|c|c|c|c|c|c|c|c|c|c|c|}
\hline
$\ket{n,l,m}$ & $\ket{1,0,0}$ & $\ket{2,0,0}$ & $\ket{2,1,0}$ & $\ket{3,0,0}$ & $\ket{3,1,0}$ & $\ket{3,2,0}$ & $\ket{4,0,0}$ &  $\ket{4,1,0}$ & $\ket{4,2,0}$ & $\ket{4,3,0}$ \\
\hline
$\ket{1,0,0}$ & 0 & 0 & $\frac{128\sqrt 2}{243}$ & 0 & $\frac{27}{64\sqrt 2}$ & 0 & 0 & $\frac{6144}{15625 \sqrt 5}$ & 0 & 0\\
\hline
$\ket{2,0,0}$ &  & 0 & $-3$ & 0 & $\frac{27648}{15625}$ & 0 & 0 & $\frac{512\sqrt{10}}{2187}$ & 0 & 0 \\
\hline
$\ket{2,1,0}$ &  &  & 0 & $\frac{3456\sqrt 6}{15625}$ & 0 & $\frac{110592\sqrt 3}{78125}$ & $\frac{1024\sqrt 2}{6561}$ & 0 & $\frac{4096\sqrt 2}{6561}$ & 0\\
\hline
$\ket{3,0,0}$ &  &  &  & 0 & $-3\sqrt 6$ & 0 & 0 & $\frac{4700160 \sqrt{15}}{5764801}$ & 0 & 0\\
\hline
$\ket{3,1,0}$ &  &  &  & & 0 & $-3 \sqrt 3$ & $\frac{5750784 \sqrt 2}{5764801}$ & 0 & $\frac{15925248 \sqrt 2}{5764801}$ & 0\\
\hline
$\ket{3,2,0}$ &  &  &  &  &  & 0 & 0 & $\frac{3538944}{5764801}\sqrt{\frac 65}$ & 0 & $\frac{191102976}{40353607}\sqrt{\frac 65}$ \\
\hline
$\ket{4,0,0}$ &  &  &  &  &  &  & 0 & $-6\sqrt 5$ & 0 & 0 \\
\hline
$\ket{4,1,0}$ &  &  &  &  &  &  &  & 0 & $-\frac{24}{\sqrt 5}$ & 0\\
\hline
$\ket{4,2,0}$ &  &  &  &  &  &  &  &  & 0 & $-\frac{18}{\sqrt 5}$ \\
\hline
$\ket{4,3,0}$ &  &  &  &  &  &  &  &  &  & 0\\
\hline
\end{tabular}
\end{table}

这可以用于计算类氢原子斯塔克效应\upref{HStark}以及跃迁率(未完成)等.

Mathematica 代码(请自行修改矩阵尺寸和循环范围), \verb|HydrogenR| 函数见类氢原子的束缚态\upref{HWF}.
\begin{lstlisting}[language=Mathematica]
DipoleZ[Z_, n1_, l1_, m1_, n2_, l2_, 
   m2_] := (-1)^
    m1 Sqrt[(2 l1 + 1) (2 l2 + 1)] ThreeJSymbol[{l1, 0}, {1, 0}, {l2, 
     0}] Integrate[
    HydrogenR[Z, n1, l1, r]\[Conjugate] HydrogenR[Z, n2, l2, 
      r] r^3, {r, 0, +\[Infinity]}] ThreeJSymbol[{l1, -m1}, {1, 
     0}, {l2, m2}];
d = ConstantArray[0, {10, 10}];
i1 = 0; i2 = 0;
For[n1 = 1, n1 <= 4, n1++, For[l1 = 0, l1 < n1, l1++,
  ++i1; i2 = 0;
  For[n2 = 1, n2 <= 4, n2++, For[l2 = 0, l2 < n2, l2++,
    ++i2;
    d[[i1, i2]] = DipoleZ[1, n1, l1, 0, n2, l2, 0];
  ]]
]];
Print[d // MatrixForm];
\end{lstlisting}
