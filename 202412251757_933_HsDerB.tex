% 导数的计算(高中)
% keys 导数|求导
% license Usr
% type Tutor

\begin{issues}
\issueDraft
\end{issues}

\pentry{导数\nref{nod_HsDerv}}{nod_ed15}

为了更高效地计算函数的导数,数学家在实践中总结和发展了许多求导方法。这些方法不仅大大简化了计算过程,还将求导从一个过程提升为独立的数学运算。在高中阶段,能够快速且熟练地使用这些求导方法是数学学习的基础。然而,高中教材中对求导公式的处理通常是直接给出结果,供学生直接应用,而未深入说明这些公式的推导过程。这种方式虽然足以应对考试和解题,但在数学的逻辑性和严谨性上有所欠缺。

为了为对数学推导感兴趣的读者提供进一步探索的机会,本文将在总结教材中公式的基础上,尝试在高中生能够接受的范围内,尽量给出这些公式的推导和证明过程。需要强调的是,这些推导并非高中学习的必备内容,而是面向那些希望了解数学本质、追求逻辑严谨性的学习者。通过这些推导,读者不仅能更深入地理解公式的来源,还能从中感受到数学推理的魅力,以及逻辑思维的力量。

\subsection{总结对照表}

下面先介绍导数的运算法则,为记录方便,记$f=f(x),g=g(x),f'=f'(x),g'=g'(x)$。
\begin{table}[ht]
\centering
\caption{导数运算法则}\label{tab_HsDerB1}
\begin{tabular}{|c|c|}
\hline
和差 & $(f\pm g)'=f'\pm g'$ \\
\hline
积 & $(fg)'=f'g+fg'$ \\
\hline
商 & $\displaystyle\left(\frac{f}{g}\right)'=\frac{f'g-fg'}{g^2}$ \\
\hline
倒数 & $\displaystyle\left(\frac{1}{g}\right)'=-\frac{g'}{g^2}$ \\
\hline
复合函数 & $(f(g))'=f'(g)g'$ \\
\hline
\end{tabular}
\end{table}

关于导数的运算法则,有几点需要注意的:
\begin{itemize}
\item 加法法则体现了导数运算的线性性质。
\item 初学者很容易混淆,将数字计算的习惯代入求导,错认为$(f \cdot g)'= f'\cdot g'$,$\displaystyle\left(\frac{f}{g}\right)'=\frac{f'}{g'}$。
\item 商法则的适用条件是分母$g\neq0$ 且导数$g'$存在。
\item 商法则和倒数法则中的$g^2$指的是将分母原本的函数平方,完全书写则应为$[g(x)]^2$,而非$g^2(x)$\footnote{不排除某些不严谨的场合会有人这样用。$g^2(x)$一般是表示复合函数$g\left(g(x)\right)$。}。
\item 倒数法则是商法则$f=1$时的特例。
\item 复合函数求导也被称为\textbf{链式法则},
\end{itemize}


这里将常见的函数与导数对照表列出如下,方便查询。具体介绍需查看每个函数自己的页面。

\begin{table}[ht]
\centering
\caption{高中常见函数及其导数}\label{tab_HsDerB2}
\begin{tabular}{|c|c|c|}
\hline
\textbf{函数名称}     & \textbf{函数 $f(x)$}     & \textbf{导函数 $f'(x)$}     \\ \hline
幂函数&$x^n$                    & $n x^{n-1}$                \\ \hline
反比例函数&$\displaystyle\frac{1}{x}$             & $\displaystyle-\frac{1}{x^2}$           \\ \hline
指数函数(e为底)&$e^x$                     & $e^x$                      \\ \hline
对数函数(e为底)&$\ln(x)$                  & $\displaystyle\frac{1}{x}$              \\ \hline
指数函数&$a^x$                     & $a^x\ln a $                      \\ \hline
对数函数&$\log_a(x)$                  & $\displaystyle \frac{1}{x\ln a}$              \\ \hline
正弦函数&$\sin(x)$                 & $\cos(x)$                  \\ \hline
余弦函数&$\cos(x)$                 & $-\sin(x)$                 \\ \hline
正切函数&$\tan(x)$                 & $\displaystyle \frac{1}{\cos^2(x)}$                \\ \hline
\end{tabular}
\end{table}

\subsection{求导法则推导}

在下面的所有推导之前,先回顾一下导数的\aref{定义}{def_HsDerv_1},这尽管不


\subsubsection{加法法则}

这个求导法则是符合生活中的直观认知的,是生活中常用的。如果两个人分别以一定的速度完成各自的任务,合起来的速度就是两者速度的相加。

\subsubsection{乘法法则}
可以想象一个池塘,水面上漂着一片荷叶。假设池塘面积和荷叶的覆盖率都在增长,那么整个覆盖面积的变化不仅取决于池塘的变化,还取决于荷叶本身的变化。

\subsection{基本初等函数的导数推导}

