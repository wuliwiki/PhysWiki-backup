% 群作用
% keys 群作用|同态
%未完成
\pentry{群同态\upref{Group2}}

\subsection{群在自身上的作用}

给定一个群$G$,我们任意拿出一个元素$a\in G$,用$a$去左乘$G$中的所有元素(包括$a$自己),那么我们可以把$a$看成一种$G$到自身上的映射:$f_a:G\rightarrow G$,使得对于任意$x\in G$,$f_a(x)=ax$.

群$G$中的每一个元素都可以像这样生成一个映射,把这些映射全部放在一起,我们也可以整体上看成一个映射$f:G\times G\rightarrow G$,满足:对于任意的$a, x\in G$,有$f(a,x)=f_a(x)=ax$.这样的$G\times G$到$G$上的映射,被称为一个\textbf{群作用(group action)}.

按照上述定义得到的作用,通常称为\textbf{左平移}作用;相应地,我们也可以让元素$a$对$x$的作用是$g_a(x)=xa$;这样的作用被称为\textbf{右平移}.

\subsection{群作用}

更一般地,对于任何集合$X$,群$G$中每个元素都可以代表$X\rightarrow X$的一个映射.我们当然可以任意规定这些映射,但如果这些映射满足一定条件的话,就会构造出一个很有意思的结构:

\begin{definition}{群作用}
设群$G$和集合$X$,$G$中每个元素都是$X$到自身的映射,记$g\in G$将$x\in X$映射为$g\cdot x\in X$.如果所有这些映射满足满足下面两条公理:
\begin{itemize}
\item \textbf{结合律}:对于$g_1, g_2\in G, x\in X$,$(gh)\cdot x=g\cdot (h\cdot x)$.
\item \textbf{单位元是恒等映射}:$G$的单位元$e$将任何$x\in X$映射到自身:$e\cdot x=x$.
\end{itemize}

那么我们称群$G$ \textbf{作用(acts)}于集合$X$上.

\end{definition}

群作用可以像定义里一样记为$g\cdot x$,也可以记为$X$到$X$的若干映射$f_g(x)=g\cdot x$,还可以整体上看成$G\times X$到$X$的一个映射$f(g, x)=g\cdot x$.

\subsection{群作用的例子}

\begin{example}{平移作用}\label{Group3_ex1}
本条开头定义的左平移和右平移,就是群在自身集合上的作用.
\end{example}

\begin{example}{伴随作用}\label{Group3_ex2}
对于任意$a\in G$,令$f_a: G\rightarrow G$满足$\forall x\in G, f_a(x)=axa^{-1}$,则这些映射定义了一个群$G$在集合$G$上的作用,称为\textbf{伴随作用}.
\end{example}

\begin{example}{线性变换}\label{Group3_ex3}
参考线性变换\upref{LTrans}.我们已经知道,$n$阶非奇异矩阵配上乘法可以构成一个群;相应地,满秩线性变换(可逆线性变换)配上映射的复合运算构成一个群.非奇异矩阵乘法是给定了基向量以后,满秩线性变换的复合的表示.

取$n$维实数向量空间$X$,那么$X$是向量的集合;$GL(n,\mathbb{R})$是$X$上可逆线性变换的群,显然$GL(n,\mathbb{R})$按照通常的线性变换定义,构成了在$X$上的一个作用.
\end{example}

\subsection{群作用的性质}

当我们讨论群$G$在集合$X$上的作用时,一共有两个集合要关心.

如果我们固定$X$中的一个元素$x$,那么每个$G$中元素$g$都把$x$映射到某个$f_g(x)\in X$上.所有能这样被映射到的元素$f_g(x)$构成了$X$的一个子集,称为元素$x$的\textbf{轨道(orbit)}.我们也可以找出所有不移动$x$的群元素$g$,即满足$f_g(x)=x$的$g\in G$,那么所有这样的$g$构成了$G$的一个子群$F_x$,称为$x$的\textbf{迷向子群(isotropy group)}.

\begin{exercise}{迷向子群}\label{Group3_exe1}
证明$F_x$构成群.\footnote{见\autoref{GroupP_ex2}.}
\end{exercise}

如果对于任何$x\in X$,都有$x$的轨道是整个$X$,那么我们称这个作用是\textbf{可递的},此时$X$就是$G$的\textbf{齐性空间}.如果对于任何$x\in X$,$x$的轨道只是$\{x\}$,那么这个作用就是\textbf{平凡(trivial)}的.

如果对于任何$x\in X$,任何$g\in G-\{e\}$,都使得$g\cdot x\not=x$,那么
