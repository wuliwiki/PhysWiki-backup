% Python 解释器
% keys Python|解释器

\begin{issues}
\issueTODO
\end{issues}

\pentry{计算机语言 — 脚本语言(解释型语言)\upref{CpLgg}}

Python解释器由编译器和虚拟机构成,编译器将源代码转换成字节码,然后再通Python虚拟机来逐行执行这些字节码.

\subsubsection{python程序执行过程:}

\begin{enumerate}
\item 执行 .py 文件,就会启动python解释器

\item 编译器将源文件解释成字节码

\item 虚拟机将字节码转化成机器语言,与操作系统交互

\item 程序运行结束后,将字节码存到pyc文件,便于后续直接执行
\end{enumerate}

\subsubsection{python解释器有很多种:}

\begin{itemize}
\item \textbf{CPython:}\textbf{C语言开发},使用最广,默认的解释器,新的语言特性通常在此率先添加.

\item \textbf{IPython:}\textbf{基于CPython之上}的交互式解释器

\item \textbf{PyPy:}完全使用 \textbf{Python 语言编写}的 Python 实现.它支持多个其他实现所没有的高级特性,例如非栈式支持和 JIT 编译器等.此项目的目标之一是通过允许方便地修改解释器 (因为它是用 Python 编写的),鼓励该对语言本身进行试验.想了解详情可访问 PyPy 项目主页.

\item \textbf{Jython:}以 \textbf{Java 语言编写}的 Python 实现.此实现可以作为 Java 应用的一个脚本语言,或者可以用来创建需要 Java 类库支持的应用.想了解更多信息可访问 Jython 网站.

\item \textbf{IronPython:}另一个 \textbf{.NET} 的 Python 实现,与 Python.NET 不同点在于它是生成 IL 的完全 Python 实现,并且将 Python 代码直接编译为 .NET 程序集.它的创造者就是当初创造 Jython 的 Jim Hugunin.想了解详情可访问 IronPython 网站.

\item \textbf{Python for .NET:}此实现实际上使用了 \textbf{CPython} 实现,但是属于 .NET 托管应用并且可以引入 .NET 类库.它的创造者是 Brian Lloyd.想了解详情可访问 Python for .NET 主页.
\end{itemize}