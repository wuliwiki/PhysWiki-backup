% 等差数列(高中)
% keys 高中|等差数
% license Usr
% type Tutor

\begin{issues}
\issueDraft
\end{issues}

\pentry{数列(高中)\nref{nod_HsSeFu}}{nod_53a2}

不知道你是否注意过,人们在上台阶时,一般会一阶一阶地走;有的人喜欢两阶两阶地跨,总是隔着一阶;还有的人更大胆,三阶三阶地跨。这些看似随意的方式,其实都有一个固定的模式——每次跨的台阶数是恒定的。这种固定的规律,就像一条隐形的线,把每次的动作串联在一起,形成了数学中所说的等差数列。

等差数列是最简单的数列之一,早在小学阶段就已经接触过了——如果从 $1$ 开始选取到  $n$  的  $n$  个自然数,它们按顺序排列就构成了一个等差数列,而这个数列的求和方法更是耳熟能详。相传,高斯小时候被老师要求计算从 1 加到 100 的总和,他敏锐地发现了其中的规律:可以将首项和末项配对,这样所有的项都变成了相同的和,再乘以项数的一半就能快速得到结果。这种求和方式总结为一句顺口溜:“\textbf{首项加末项,乘以项数除以 2}”,至今仍为许多人津津乐道。

在接下来的内容中,将探索等差数列及其相关的数学性质,有以下几点需要关注:首先是等差数列本身的结构与特性,其次是它的求和公式及相关推导过程。此外,还将探讨等差数列与函数的关系,理解它在更广泛数学背景中的地位。作为学习数列的第一步,等差数列也为研究其他数列提供了一个范例,通过它可以学习如何分析数列的构成、关注哪些信息,以及用哪些方法来进行研究。这不仅是对数列知识的学习,也是一种数学思维方式的培养。

\subsection{等差数列}

从台阶的故事中可以看出,不论是每次跨一阶、两阶还是三阶,这些方式都有一个共同的特点:每次跨的台阶数是固定的,即两次动作之间的差始终相等。这种规律不仅出现在上台阶这样的情景中,还广泛存在于其他现象中,例如每天增加固定的储蓄金额,或者不断调整音量时每次调节相同的数值。无论这些场景多么不同,它们的本质都是一种“均匀增加或减少”的过程,由固定的差值串联起来。

\begin{definition}{等差数列}
如果数列 $\{a_n\}$ 满足对于 $n > 1$ 的所有项,每一项与前一项的差为同一个常数 $d$,则称 $\{a_n\}$ 为\textbf{等差数列(arithmetic sequence)},$d$ 称为$\{a_n\}$的\textbf{公差(common difference)},即等差数列满足递推公式
\begin{equation}
a_{n}=a_{n-1}+d\qquad(n>1)~.
\end{equation}
\end{definition}

特别地,之前提到的常数列是 $d = 0$ 的等差数列。如果公差 $d$ 为负,数列的值会逐渐减少,例如 $10, 8, 6, \dots$。

有了递推公式,自然要研究一下是否可以得到通项公式,下面介绍两种推导等差数列通项公式的方法:

方法一:迭代递推公式

当$n>1$ 时,利用递推公式  $a_n = a_{n-1} + d$ 逐步将数列的前一项用递推公式展开,可以得到:
\begin{equation}\label{eq_HsAmPg_6}
\begin{aligned}
a_n &= a_{n-1} + d \\
&= a_{n-2}+d + d\\
&\cdots \\
&= a_2 + (n-2)d\\
&=  a_1 + (n-1)d~.
\end{aligned}
\end{equation}

方法二:错项相减法

它是累加法的一个特例,核心思想是利用数列的结构性(如相邻项有规律的差或比),通过累加将多项式或等式中的部分消去,从而简化计算。实际操作时,首先根据递推公式写出$n-1$个等式:
\begin{equation}\label{eq_HsAmPg_7}
\begin{split}
        &        &    & & a_2& -a_1 & = d, \\
        &        &a_3 & &-a_2&      & = d, \\
        &  a_4   &-a_3& &    &      & = d, \\
        &        &    & &    &      &\cdots \\
a_n -   &a_{n-1} &    & &    &      & = d~.
\end{split}
\end{equation}
然后把将这些等式的左侧和右侧分别相加,得到:
\begin{equation}
(a_2 - a_1) + (a_3 - a_2) + (a_4 - a_3) + \cdots + (a_n - a_{n-1}) = d + d +d+ \cdots + d~.
\end{equation}
将左侧的括号打开,这时由于相邻项的抵消,剩余的部分只有$a_n-a_1$,左侧则是$n-1$项和,即:
\begin{equation}\label{eq_HsAmPg_8}
a_n -a_1=(n-1)d~.
\end{equation}

\autoref{eq_HsAmPg_6} 的迭代推导与 \autoref{eq_HsAmPg_7} 通过差的累加推导,分别代表了研究数列的两种重要构造方法。前者通过递推公式逐步展开,直接得到通项公式;后者通过相邻项的差累加,从整体上揭示了数列的内在规律。这两种方法虽然路径不同,但都最终将第  $n $ 项表示为首项 $ a_1$  和公差  $d$  的线性关系。在得出这一公式后,还需要记得验证其是否符合边界条件。

当  $n = 1$  时,代入\autoref{eq_HsAmPg_6} 得:
\begin{equation}
a_1 = a_1 + (1 - 1)d = a_1~,
\end{equation}
这个结果表明,这个关系对 $n = 1$ 的边界条件依然成立,因此最终得到等差数列的通项公式。
\begin{corollary}{等差数列通项公式}
对等差数列$\{a_n\}$,其通项公式为:
\begin{equation}\label{eq_HsAmPg_5}
a_n = a_1 + (n - 1)d~.
\end{equation}
其中,$a_1$ 是首项,$d$ 是公差,$n$ 是项数。
\end{corollary}
\addTODO{至此。}
如果能够证明一个数列满足定义,也就是任意相邻两项的差是定值,那么就能确定他是等差数列,同时也能确定公差。对于任何等差数列,如果确定了是等差数列,那么公差和首项就能唯一确定,也就是说两个数列如果()对应相等,那么他们就是同一个数列。而可以看出,尽管首项对确认等差数列很重要,但更重要的是公差。
\begin{example}{证明:若$\{a_n\}$通项公式形式为$a_n=kn+b$,则其为等差数列。}
对于$n>1$,有:
\begin{equation}
a_n-a_{n-1}=kn+b-k(n-1)-b=k~.
\end{equation}
说明$\{a_n\}$是以$k$为公差的等差数列,通项公式带入$n=1$有,$a_1=b+k$。
\end{example}

事实上,知道任意一项都可以通过公差来推出所有其他项,满足:
\begin{equation}
a_n=a_k+(n-k)d~.
\end{equation}
通过这个关系还可以得到另一个关系:
\begin{equation}\label{eq_HsAmPg_1}
d={a_n-a_k\over n-k}~.
\end{equation}
这像极了\aref{平均变化率}{def_HsFunC_3}或者\aref{斜率}{def_HsFunC_5}的定义。其实,如果将等差数列的通项公式变形可以得到:
\begin{equation}
a_n=dn+(a_1-d)~.
\end{equation}
在\enref{数列}{HsSeFu}中曾介绍过,这个形式,其实就是一次函数的离散形式,也就是说,在直角坐标系中,将等差数列的项数$n$作为横坐标,数列的值$a_n$作为纵坐标,点$(n,a_n)$分布在一条直线上。


根据\autoref{eq_HsAmPg_1} 还能得到等差数列的一个重要性质,取四个整数,满足$m+n=p+q$,即$p-n=m-q$,那么根据\autoref{eq_HsAmPg_1} 有:
\begin{equation}
{a_p-a_n\over p-n}=d={a_m-a_q\over m-q}~.
\end{equation}
两侧分母相等,化简得到推论。
\begin{corollary}{}\label{cor_HsAmPg_1}
对于等差数列$\{a_n\}$,若满足$m+n=p+q$,则有:
\begin{equation}
a_m+a_n=a_p+a_q~.
\end{equation}
\end{corollary}

\subsubsection{等差中项}

如果在 $a$ 和 $b$ 中间插入一个数 $A$,使 $a,A,b$ 成等差数列,那么 $A$ 叫作 $a$ 与 $b$ 的\textbf{等差中项}.

如果 $A$ 是 $a$ 与 $b$ 的等差中项,那么
\begin{equation}
A - a = b - A~,
\end{equation}
\begin{equation}
A = \frac{a+b}{2}~.
\end{equation}


\subsection{等差数列的数列和}

研究完数列的性质后,现在来研究等差数列的数列和。在学习数列时,曾提及过数列和的一个\aref{性质}{eq_HsSeFu_1},据此将数列两次排列后相加有:
\begin{equation}\label{eq_HsAmPg_4}
\begin{split}
2S_n &= (a_1 + a_2 + \cdots + a_n)+(a_n + a_{n-1} + \cdots + a_1)\\
&=(a_1+a_{n})+(a_2+a_{n-1}) +\cdots +(a_n+a_1)~.
\end{split}
\end{equation}
观察\autoref{eq_HsAmPg_4} ,每对括号中的两项下标之和均为 $n+1$,根据\autoref{cor_HsAmPg_1} ,这意味着括号内的和均等于 $a_1 + a_n$,设数列共有 $n$ 项,则共有 $n$ 对这样的和,因此:
\begin{equation}\label{eq_HsAmPg_2}
2S = n \cdot (a_1+a_n)\implies S = \frac{n\cdot(a_1+a_n)}{2}~.
\end{equation}

这一公式表明,等差数列的和可以通过的方法计算。这与经典的高斯求和公式完全一致。
对于更特殊的从1开始的自然数列,(高斯求和的故事),得到的“首项加末项,乘以项数除以2”,于此处也是契合的。

接下来,将\aref{等差数列的通项公式}{eq_HsAmPg_5}代入\autoref{eq_HsAmPg_2} 可以得到等差数列和的通项公式。
\begin{corollary}{等差数列和的通项公式}
对等差数列$\{a_n\}$,其数列和$\{S_n\}$的通项公式为:
\begin{equation}\label{eq_HsAmPg_3}
S_n = na_1+\frac{n(n-1)}{2}d~.
\end{equation}
其中,$a_1,d$ 是等差数列的首项和公差,$n$ 是项数。
\end{corollary}

\autoref{eq_HsAmPg_3} 可以变形得到
\begin{equation}
S_n = \frac{d}{2}n^2+\left(a_1-\frac{d}{2}\right)n~.
\end{equation}

如果看作函数,则是一个过原点的二次函数。这与认知相符,显然如果这个和是零项和时应该为$0$。


\begin{example}{若$\{a_n\}$的数列和为$S_n=An^2+Bn+C,(ABC\neq0)$,判断$\{a_n\}$是否为等差数列。}\label{ex_HsAmPg_1}
答:

$\{a_n\}$不是等差数列。

解析:

首先,由数列的和 $S_n$ 的定义可得首项为$a_1=S_1=A+B+C$。

对于$n>1$,根据 $a_n = S_n - S_{n-1}$,有
\begin{equation}
\begin{split}
a_n &= S_n-S_{n-1}\\
&=An^2+Bn+C-A(n-1)^2-B(n-1)-C\\
&=(2n+1)A+B~.
\end{split}
\end{equation}

分别计算相邻两项的差:

\begin{equation}
\begin{split}
a_2 - a_1 &= (2A \cdot 2 + A + B) - (A + B + C) \\
&= 4A + A + B - A - B - C \\
&= 2A - C~.
\end{split}
\end{equation}

对$n>2$,两项之差为
\begin{equation}
a_n-a_{n-1} = (2n+1)A+B-(2n-1)A-B=2A~.
\end{equation}

由此可以看出,从 $a_2$ 开始,数列 $\{a_n\}$ 的相邻两项之间的差是固定的,即 $2A$,因此后续项构成等差数列。然而,$a_2 - a_1 = 2A - C$ 与 $a_n - a_{n-1} = 2A$ 不相等,说明首项与后续项之间的差异导致 $\{a_n\}$ 整体不构成等差数列。

进一步分析,若 $\{a_n\}$ 是等差数列,则 $S_n$ 必须具有 $S_n = An^2 + Bn$ 的形式(即 $C = 0$)。由于 $C \neq 0$,因此 $\{a_n\}$ 不是等差数列。
\end{example}

在\autoref{ex_HsAmPg_1} 中,尽管数列 ${a_n}$ 的后续项具有等差性,但由于边界条件的特殊性(首项与后续项之间存在差异),整体判断表明 ${a_n}$ 并非等差数列。边界条件在问题中起到关键作用,其直接影响了解答的完整和正确。在处理计算题时,必须特别注意边界条件是否成立,以避免因忽视边界条件而导致的错误结论。

\textbf{等差数列与函数}

求数列和的最小值

若$a_0<0,d>0$,则一定有$a_1<a_2<\dots<a_{k}<0$。

验证$a_{k+1} =0$且$a_{k+m}>0(m\in\mathbb{N}^+)$,或$a_{k+m}>0(m\in\mathbb{N})$。

因此$S_n$的最小値$S_k=S_{k+1}=S$或$S_k=S$
