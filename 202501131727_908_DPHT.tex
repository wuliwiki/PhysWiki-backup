% 丢番图(综述)
% license CCBYSA3
% type Wiki

本文根据 CC-BY-SA 协议转载翻译自维基百科\href{https://en.wikipedia.org/wiki/Diophantus#Notes}{相关文章}。

亚历山大的丢番图(约公元200年–约214年出生;约公元284年–约298年去世)是一位希腊数学家,著有两部主要作品:《多边形数论》,该书现存不全,以及《算术学》,分为十三卷,大部分仍然存在,包含了一些通过代数方程求解的算术问题。

丢番图的《算术学》对阿拉伯数学的发展产生了影响,他的方程式也影响了现代抽象代数和计算机科学的研究。他的前五卷完全是代数性质的。此外,最近对丢番图作品的研究表明,他在《算术学》中教授的解题方法,与后来的中世纪阿拉伯代数在概念和整体程序上高度相似。

丢番图是最早认识到正有理数作为数字的数学家之一,通过允许系数和解为分数。他创造了术语 \textbf{παρισότης}(parisotēs)来表示近似等式。这个术语在拉丁语中翻译为 \textbf{adaequalitas},并成为皮埃尔·德·费尔马(Pierre de Fermat)发展出的“等近性”技术,用于求函数的最大值以及曲线的切线。

尽管《算术学》不是最早使用代数符号解决算术问题的作品,但它无疑是最著名的一个,这些问题来源于希腊古代,并且其中的一些问题激发了后来的数学家在分析学和数论领域的研究。现代使用中,丢番图方程指的是带有整数系数的代数方程,目标是寻找其整数解。丢番图几何和丢番图逼近是另外两个以他命名的数论子领域。
\subsection{传记}
丢番图出生于一个希腊家庭,并且已知他在罗马时代的公元200年至214年到284年或298年期间生活在埃及的亚历山大城。[6][8][9][a] 关于丢番图生平的大部分知识来自一部5世纪的希腊数字游戏和谜题选集,由梅特罗多罗斯(Metrodorus)编纂。书中有一个问题(有时被称为他的墓志铭)内容如下:

此地安葬丢番图,令人惊叹。通过代数的艺术,石碑上述说他的年岁:‘上帝赋予他少年时期,生命的一六分之一;青春期更多,成长为胡须浓密的青年,一十二分之一;然后在婚姻前,又度过了七分之一;五年后,迎来了一个跳跃的儿子。可怜的是,这个父亲与智者的亲爱的孩子,在活到父亲生命的一半时,命运将他带走。四年后,他通过数字科学安慰自己的命运,最终结束了生命。’

这个谜题意味着丢番图的年龄 \(x\) 可以表达为:

\[
x =\frac{x}{6}+\frac{x}{12} +\frac{x}{7}+5+\frac{x}{2}+4~
\]

解得 \(x\) 的值为84岁。然而,这些信息的准确性无法确认。

在流行文化中,这个谜题出现在《雷顿教授与潘多拉的盒子》中,作为游戏中最难解的谜题之一,需要通过先解决其他谜题才能解锁。
\subsection{算术}
《算术》是丢番图的主要著作,也是希腊数学中关于前现代代数的最重要作品之一。它是一个包含有确定性和不确定性方程数值解的题目集。原本《算术》共有十三卷,但只有六卷保存了下来,尽管有些人认为1968年发现的四本阿拉伯书籍也是丢番图的作品。一些《算术》中的丢番图问题也在阿拉伯文献中找到了踪迹。

需要提到的是,丢番图在解题时从未使用一般的方法。著名的德国数学家赫尔曼·汉克尔曾对丢番图做出如下评论:

“我们的作者(丢番图)没有一点迹象表明他使用了一种通用的、全面的方法;每一个问题都需要一些特殊的方法,而这些方法即使对最相似的问题也行不通。因此,即便现代学者已经研究了丢番图的100个解法,也难以解出第101个问题。”

历史

像许多其他希腊数学著作一样,丢番图的著作在西欧黑暗时代被遗忘,因为古希腊语的学习以及总体的文化素养大幅下降。然而,幸存下来的部分希腊文《算术》,和所有传递到近代世界的古希腊文献一样,通过拜占庭学者的抄写得以保存,因此在中世纪的拜占庭学者中是有流传的。拜占庭希腊学者约翰·霍尔塔斯梅诺斯(John Chortasmenos,1370-1437)对丢番图的注释以及早期希腊学者马克西莫斯·普拉奴德斯(Maximos Planudes,1260-1305)的综合性评论得以保存,后者曾在拜占庭君士坦丁堡的科拉修道院编纂了丢番图的版本[16]。此外,《算术》的部分内容可能通过阿拉伯学术传统得以流传(见上文)。1463年,德国数学家雷吉奥蒙塔努斯写道:

“至今没有人将丢番图的十三卷从希腊语翻译成拉丁语,而这些书中隐藏着整个算术学的精华。”

《算术》首次从希腊语翻译成拉丁语是在1570年,由博姆贝利(Bombelli)完成,但翻译本从未出版。然而,博姆贝利借用了其中许多问题并将其用于他自己的《代数学》一书。《算术》的拉丁语初版由希兰德(Xylander)于1575年出版。1621年,由巴谢(Bachet)完成的拉丁文翻译成为了广泛流传的第一版。皮埃尔·德·费马拥有一本副本,研究并在书页边缘做了笔记。1895年,保罗·塔内里(Paul Tannery)对《算术》的拉丁语翻译被托马斯·L·希斯(Thomas L. Heath)称为一次改进,并且在1910年他出版的英文版第二版中使用了这一版本。