% 暗物质的模拟
% license Usr
% type Tutor

 

\subsection{非线性增长的不均匀性:数值模拟}

上一节通过分析地讨论了由于引力吸引,原始密度扰动 \( \delta_k \) 如何增长,这是在微扰极限 \( \delta_k \ll 1 \) 下的情况。当过密区域最终达到 \( \delta_k \sim 1 \) 时,微扰计算不再适用。此时,开始形成引力束缚系统:银河系子晕、星系、星系团等。较小的结构首先形成,因为在较小尺度 \( 1/k \) 上的不均匀性 \( \delta_k(t) \) 更大(见上文)。较大的结构后来形成,通过吸积扩散的暗物质和通过预先存在的较小结构的合并形成(这个过程被称为层次结构形成)。动力学仍然由描述暗物质的非碰撞玻尔兹曼方程描述,以欧拉形式写在方程中,但现在非线性效应变得重要。因此,通过计算机模拟来研究超过 \( \delta_k \sim 1 \) 的不均匀性增长是必要的。

这些模拟是一项了不起的成就,因为它们能够覆盖巨大的范围:时间上大约 \( 10^{13} \) 年(从 \( z \approx 100 \) 时的初始条件,
大约 \( t \approx 10 \) Myr 到今天的 \( z \approx 0 \)),空间上大约7个数量级(一些最近的模拟处理的体积大约是 $10 Gpc^3$,具有 \( \sim \) $kpc$ 的分辨率),密度对比度上大约11个数量级(从在重组时 \( \delta_k \approx 10^{-5} \) 到今天大尺度结构中典型的密度对比度 \( \delta_k \approx 10^6 \)),以及质量上大约30个数量级(Wang等人(2020年)的最近模拟能够同时解析地球质量的晕,
大约 \( 10^{-6} M_{\odot} \),和星系团质量的晕,大约 \( 10^{14} M_{\odot} \),并且它们可以在总模拟质量 \( 10^{19} M_{\odot} \) 中追踪 \( 10^{-11} M_{\odot} \) 的元素,这要归功于多变焦技术)。实践中所做的是在宇宙膨胀的背景下模拟大量 \( N \) 个质量点(因此得名N体模拟)的引力运动,从典型的原始不均匀性的实现开始。现代计算机可以为 \( N \sim 10^{12} \) 个质量点解决牛顿运动方程。尽管基本暗物质粒子的数量要多得多,但由于万有引力的普遍性,这足以正确模拟比单个元素大的尺度上的物理过程,其中每个元素的质量为 \( M/N \),\( M \) 是计算体积中的总质量。从定性的角度来看,这样的模拟产生了展示暗物质如何聚集成丝状、墙状和薄煎饼状,它们合并成由空洞分隔的晕,产生通常所说的宇宙网的图像和视频。例如,它们展示了像银河系这样总质量大约 \( 10^{12} M_{\odot} \) 的天体主要在红移 \( z \approx 1 \) 时形成,而目前大约20\%的暗物质仍然扩散。在定量层面上,模拟允许预测形成结构的统计特性:质量分布(晕质量函数)、典型形状(球形或椭圆形)、典型密度剖面 \( \rho(r) \)、暗物质速度分布及其方差 \( \sigma^2(r) \)。这里的主张更加普遍:暗物质假设导致这些模拟的定性和定量结果与当前宇宙中大尺度结构的观测特性总体一致。正如我们已经多次暗示的,暗物质是塑造和支撑宇宙中可见物质分布的“宇宙脚手架”。数值模拟在测试其存在方面是至关重要的,因为它们提供了理论模型与实际天文观测之间的基本联系。

讨论到目前为止涉及的是仅包含暗物质的数值模拟,自20世纪80年代以来一直在进行。普通物质(即“重子物质”)比暗物质少,但其影响非常重要,应该被包括在内,这是最近模拟(称为重子或流体动力学模拟)开始做的,试图实现精度。详细讨论这些模拟如何模拟重子物质及其相互作用超出了我们综述的重点。一般来说,这些模拟需要考虑许多不同的物理过程,在不同的时间尺度上,对于非常大的速度、密度、温度、粘度范围等。成分包括:气体流体动力学、气体冷却、辐射、磁场(由于它们显著贡献于压力;它们在计算机代码中的放大,可能使它们更复杂的种子变得不重要)、恒星形成和恒星死亡,以及恒星喷射物(包括超新星:产生的宇宙射线对压力有贡献,在短时标内通过将物质从局部中心移动到周围广阔区域显著改变引力势),超大质量黑洞(存在于大质量星系中;它们的种子是未知的,它们可能通过获取气体,可能通过合并而增长),等等。尘埃、热传导和粘度被认为不那么重要。结果,计算机代码最近开始产生现实的星系。虽然细节有所不同(例如,螺旋星系的形成在很大程度上取决于恒星反馈如何调节恒星形成),但不同的代码目前似乎在一些预测上达成了共识。总的来说,似乎ΛCDM模型包括重子物质可以解释星系形成,并且已经确定了形成星系的基本物理过程。然而,这些代码确实包含自由可调参数,包括控制数值近似的参数,如空间分辨率和所谓的“子网格物理”的建模。因此,存在通常的风险,可能(过度)拟合数据以适应不完整甚至不正确的模型,其中调整现象学参数可能模仿其他某些东西的效果。数值模拟还包括暗能量(DE),至少以加速膨胀的宇宙的背景形式出现,因此在晚期进入哈勃参数的背景。也模拟了DE的替代模型。结构形成的影响可能是巨大的,但并不总是容易与其他退化效应区分开来。

最后,我们提到数值模拟可以用来研究暗物质的性质,即探索偏离假设的普通冷、无碰撞暗物质的偏差。例如,已经进行了模拟温暗物质、自相互作用暗物质和模糊暗物质的数值模拟,有或没有重子物质。