% 共轭与共轭类
% 共轭|共轭类|共轭元素|等价关系|conjugate

\begin{issues}
\issueDraft
\end{issues}
\begin{definition}{共轭}
若有群元$d,f\in G$,且$\exists g\in G$使得$gdg^{-1}=f$,则称群元$d$与群元$f$共轭。记作$d$~$f$。
\end{definition}

共轭关系是一个等价关系,满足自反率、对称率和传递率:\\
自反率:$g=ggg^{-1}$,则$g$~$g$。 \\
对称率:若$d$~$f$,则$\exists g\in G$使得$gdg^{-1}=f$,那么$d=g^{-1}fg=
g^{-1}f(g^{-1})^{-1}$,则$f$~$d$ \\
传递率:若$d$~$f$,$f$~$h$,则$\exists g_1,g_2\in G$,使$h=g_2fg_2^{-1}=
g_2g_1dg_1^{-1}g_2^{-1}$ $=g_2g_1d(g_2g_1)^{-1}$,则有$d$~$h$。

这一概念与矩阵中的相似矩阵类似。

\begin{definition}{共轭类}
群中所有相互共轭的元素的集合称为共轭类。
\end{definition}