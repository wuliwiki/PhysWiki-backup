% 图的矩阵表示
% keys 矩阵表示|图|邻接矩阵|关联矩阵
% license Usr
% type Tutor

\pentry{图\nref{nod_Graph},矩阵及其运算\nref{nod_Mat}}{nod_3f6e}
将\enref{图}{Graph}的点集和边集的关系用矩阵来体现,就是所谓的图的矩阵表示。体现点与点相邻关系的矩阵称为邻接矩阵;体现点与边关联关系的矩阵称为关联矩阵。当对某个图选定了权函数成为\enref{赋权图}{GraSpa}时,体现权函数的矩阵称为赋权矩阵。

\subsection{邻接矩阵}
邻接矩阵是用来体现点点相邻关系的,具体地,设图 $G$ 的点集为 $V(G)=\{v_1,\cdots,v_n\}$。若 $G$ 是无向图,则邻接矩阵的第 $(i,j)$ 个元反映了连接 $v_i,v_j$ 的边数。若 $G$ 是有向图,则邻接矩阵的第 $(i,j)$ 个元反映了以 $v_i$ 为起点,$v_j$ 为终点的有向边数。
\begin{definition}{邻接矩阵}
设 $G$ 是图,$V(G)=\{v_1,\cdots,v_n\}$,$A=(a_{ij})$ 是 $n$ 阶矩阵。若 $G$ 是无向图,$a_{ij}$ 等于连接 $v_i,v_j$ 的边数;若 $G$ 是有向图,$a_{ij}$ 等于以 $v_i$ 为起点,$v_j$ 为终点的有向边数。则称 $A$ 为 $G$ 的\textbf{邻接矩阵}(adjacency matrix)。
\end{definition}


\subsection{关联矩阵}

关联矩阵是用来体现点边的关联关系的,具体地,设图 $G$ 的点集和边集分别为 
\begin{equation}
V(G)=\{v_1,\cdots,v_n\},E(G)=\{e_1,\cdots,e_m\}.~
\end{equation}
若 $G$ 是无向图,则关联矩阵的第 $(i,j)$ 个元反映了 $v_i,e_j$ 是否关联。若 $G$ 是有向图,则关联矩阵的第 $(i,j)$ 个元反映了以 $v_i$ 是 $e_j$ 的起点还是终点。
\begin{definition}{关联矩阵}
设 $G$ 是图,$V(G)=\{v_1,\cdots,v_n\},E(G)=\{e_1,\cdots,e_m\}$,$A=(a_{ij})$ 是 $n\times m$ 的矩阵。若 $G$ 是无向图,
\begin{equation}
a_{ij}=\left\{\begin{aligned}
&1, v_i\in e_j,\\
&0, v_i\not\in e_j;
\end{aligned}\right.~
\end{equation}
若 $G$ 是有向图,
\begin{equation}
a_{ij}=\left\{\begin{aligned}
&1, &v_i\text{是} e_j\text{起点} ,\\
&-1, &v_i\text{是} e_j\text{终点} ,\\
&0,&v_i\not\in e_j.
\end{aligned}\right.~
\end{equation}
则称 $A$ 为 $G$ 的\textbf{关联矩阵}(incidence matrix)。
\end{definition}


\subsection{赋权矩阵}

赋权矩阵是用来体现赋权函数的,具体地,设图 $G$ 的点集为 $V(G)=\{v_1,\cdots,v_n\}$,$\omega_1:V(G)\rightarrow\mathbb R,\omega_2:E(G)\rightarrow\mathbb R$ 分别是点赋权函数和边赋权函数,则点赋权矩阵的第 $(i,j)$ 个元反映了以 $v_i$ 是 $e_j$ 的起点还是终点。
\begin{definition}{邻接矩阵}
设 $G$ 是图,$V(G)=\{v_1,\cdots,v_n\}$,$A=(a_{ij})$ 是 $n$ 阶矩阵。若 $G$ 是无向图,$a_{ij}$ 等于连接 $v_i,v_j$ 的边数;若 $G$ 是有向图,$a_{ij}$ 等于以 $v_i$ 为起点,$v_j$ 为终点的有向边数。则称 $A$ 为 $G$ 的\textbf{邻接矩阵}(adjacency matrix)。
\end{definition}









