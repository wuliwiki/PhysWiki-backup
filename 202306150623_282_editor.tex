% 编辑器使用说明
% 小时百科|在线编辑器|latex

欢迎使用小时百科/云笔记在线编辑器。
\begin{figure}[ht]
\centering
\includegraphics[width=13cm]{./figures/19c6cc6482ff004d.png}
\caption{编辑器截图(\href{https://wuli.wiki/apps/editor.gif}{查看 GIF 动画})} \label{fig_editor_3}
\end{figure}

\subsection{如果你已经会 LaTeX}
我们必须强调本编辑器只是在 MathJax\footnote{\href{https://www.mathjax.org/}{MathJax} 是一个在网页上显示 LaTeX 公式的插件, 知乎等网站都使用它显示公式。} 的基础上做了一个简单的拓展, 所以\textbf{如果你直接把别处的 LaTeX 代码复制进来几乎肯定会遇到严重的问题}。 我们在 MathJax 的基础上另外实现了一些基本命令如 \verb|\subsection{}|, \verb|\textbf{}|, 常用环境的简单格式以及对他们的引用等。 公式以外绝大部分支持的命令都可以在菜单栏上找到。 最后, 编辑器\textbf{仅支持百科的模板}, 整个百科是一个 \verb|document| 环境, 主文件是 \verb|main.tex|, 其他所有 tex 文件都将作为一个 \verb|section| 插入 main.tex 中(使用我们定义的 \verb|\entry{词条名}{词条标签}|), 不需要手动添加 \verb|\section{词条名}|。

以下大部分不是 LaTeX 的教程而是编辑器的教程, 所以同样建议你看一看。 可能\textbf{大大提高编辑效率}的功能有: 引用按钮(自动添加 \verb|\label| 和 \verb|\autoref| 来引用公式图表等), 自动补全和符号替换(可以在设置里面自定义), 以及各种快捷键(\autoref{tab_editor_1} )。

\subsection{零基础视频教程}
\begin{enumerate}
\item 编辑器使用入门(\href{https://www.bilibili.com/video/av87698355/}{B 站}, \href{https://zhuanlan.zhihu.com/p/105869878}{知乎}, \href{https://www.youtube.com/watch?v=AN2tXNanD9U&t=1s}{YouTube})
\item 提高编辑效率 (未完成): 使用代码补全, 符号显示, 自动引用, 实时预览, 符号显示, 直接 Ctrl+V 粘贴图片。
\item 其他功能(未完成): 编译 pdf, 转载到知乎等。
\end{enumerate}

\subsection{账号和访问}
目前该编辑器同时用于\href{http://wuli.wiki/editor/}{编辑小时百科}以及\href{http://wuli.wiki/note/}{小时云笔记}两个板块。 它们的功能几乎相同, 但百科有合作编辑功能而云笔记暂时没有; 百科需要申请成为作者才可以进行编辑, 而云笔记对所有注册用户开放。 云笔记目前已经进入稳定阶段, 数据会永久保留, 请放心使用。

\subsection{编辑百科}
\begin{itemize}
\item 百科志愿者申请成功后默认拥有编辑词条的权限, 但却不能直接发布词条, 而是需要拥有发布权限的用户审核通过后发布。
\item 若一个词条处于 “编辑中” 的状态, 则其他用户无法修改该词条(除非有 “介入编辑” 权限), 只能以只读模式打开。
\item 新建或修改的词条只有通过审核并被发布后才可以重新被其他用户重新编辑, 否则视为 “编辑中” 状态。
\end{itemize}

\subsection{编辑器简介}

\begin{itemize}
\item 【新】增加了全局查找功能, 在菜单栏上点击相应按钮, 然后输入关键词即可, 默认支持 regex, 区分大小写。 该功能是通过命令行的 \verb|git grep| 命令实现的, 具体命令为 \verb|git grep --no-index 用户输入|, 详细功能参考\href{https://git-scm.com/docs/git-grep}{官方文档}。
\item 【新】新增了直接粘贴图片功能, 例如先截图, 再在代码窗口按 \verb|Ctrl+V|, 可以自动上传剪切板中的图片并创建图片环境。
\item 【新】新增了实时预览功能, 即改变代码时预览自动更新。 该功能仍然有一些小 bug 待修复, 如果遇到问题可以在设置面板中关闭。 在非实时预览模式下, 编辑过程中用快捷键 \verb|Ctrl+S| 可以保存并刷新预览(建议经常刷新, 便于定位错误)。 也可以用工具栏的保存图标。
\item 【语言】本编辑器使用 LaTeX 语言的一个子集, 一个简单的 LaTeX 介绍见 \href{https://wuli.wiki/online/latxIn.html}{LaTeX 结构简介}。 绝大部分支持的命令都可以通过工具栏插入, 所有支持的命令见 “词条示例\upref{Sample}”。
\item 【自定义命令】我们在模板中用 \verb|\newcommand{}{}| 加入了一些自定义命令, 但不会覆盖原有的 LaTeX 命令。 若希望加入新的自定义命令, 请与管理员协商, 也可以使用下文的 “自动补全” 功能作为代替。
\item 【生成 pdf】 仅限在\href{http://wuli.wiki/note/}{小时云笔记}中使用。 点击菜单栏的 “下载” 按钮可以下载所有 tex 文件以及图片, 然后用 TeXlive 的 XeLaTeX 编译 \verb|main.tex| 即可(需要编译 2 到 4 次,取决于 pdf 的页数), 推荐使用 TeXlive 2019 或者更新版本。 TeXlive 的使用详见 “安装使用 TeXlive\upref{TeXliv}”。 在 Linux 环境中也可以直接用 \verb|make| 命令编译(会自动编译足够的次数)。
\item 【转载到知乎】 仅限在\href{http://wuli.wiki/note/}{小时云笔记}中使用, 点击 “导出 Markdown 文件” 按钮生成与知乎兼容的 md 文件(矢量图和表格等暂不兼容), 在知乎的回答或文章编辑器中直接导入该文件即可。
\item 【开始使用】进入编辑器初始页面后, 点 “新建词条” 图标(红色加号)可以新建文件, 点 “打开词条” 图标可以编辑已有文件。

\end{itemize}

\subsection{公式}
\begin{itemize}
\item 公式环境支持大部分 LaTeX 命令, 严格来说是所有 \href{https://www.mathjax.org/}{MathJax} 支持的命令。
\item 一个简单的公式编辑器见\href{https://www.codecogs.com/latex/eqneditor.php}{这里} (不建议使用, 建议练习手动输入)。
\item 一个简单的 TeX/LaTeX 公式入门教程见\href{https://chaoli.club/index.php/211}{超理论坛}。
\item 本编辑器额外支持支持部分 \href{http://mirrors.ibiblio.org/CTAN/macros/latex/contrib/physics/physics.pdf}{Physics 宏包}中的命令,以及百科模板中自定义的快捷命令(见 “词条示例\upref{Sample}”)。
\item 行内公式插入到两个美元符号之间, 如 \verb|$a^2+b^2=c^2$| 显示为 $a^2 + b^2 = c^2$。
\item 独立公式只能用 \verb|equation| 环境(推荐), \verb|align| 环境或者 \verb|gather| 环境。 \verb|equation| 环境可以通过工具栏的公式图标插入, 也可以打 \verb|\beq| 然后按 Tab 键或者回车插入, 如
\begin{equation}\label{eq_editor_1}
a^2 + b^2 = c^2~,
\end{equation}
\item 公式中所有常用的和自定义的命令见 “词条示例\upref{Sample}”。
\item 为增加代码可读性, 公式中一些命令会显示为对应的符号(如希腊字母, 求和符号, 不等号等), 注意这不会影响源码(复制时得到的也是命令而不是符号)。 设置面板(齿轮图标)可以选择关闭该功能。
\item 工具栏中的 “内部引用” 按钮可以引用同一页面的公式并生成链接(图片表格等同理), 如 “\autoref{eq_editor_1}”。 “外部引用” 按钮可以引用其他词条的公式或图表。
\end{itemize}

\subsection{文件结构}

整个百科(或用户笔记)是 LaTeX 的一个 \verb|\document| 环境, 目录中每个 “部分” 是一个 \verb|\part|, 每个 “章” 是一个 \verb|\chapter|, 每个词条是一个 \verb|\section|, 词条中蓝色的小标题是 \verb|\subsection|, 黑色的小标题是 \verb|\subsubsection|。 编辑器打开的一个词条文件就是一个 section 的内容(不需要 \verb|\section| 命令)。 用 TeXlive 编译 pdf 的时候所有词条文件都会通过 \verb|\input{xxx.tex}| 插入到主文件 main.tex 中。

网页版的百科词条目录由 \verb|main.tex| 文件生成, 所以必须修改该文件并发布才能更新目录。 否则虽然页面可以访问但却不会出现在目录中。

每个词条文件(后缀名为 tex)都有一个独一无二的文件名(不区分大小写), 可以将通过将光标停留在编辑器中的 tab 上查看或者通过地址栏的 url 查看。

\begin{figure}[ht]
\centering
\includegraphics[width=4cm]{./figures/89cb63348bbde05a.png}
\caption{查看词条文件名} \label{fig_editor_2}
\end{figure}

每个词条(section) 的 label 与文件名相同, 转换后输出的网页文件(html)也有相同的文件名, 可以在浏览器的地址栏中看到(例如本文的 LaTeX 文件是 \verb|editor.tex|, label 是 \verb|editor|, 转换成网页为 \verb|editor.html|)。

\subsection{编辑器说明}
\begin{table}[ht]
\centering
\caption{编辑器快捷键}\label{tab_editor_1}
\begin{tabular}{|c|c|c|c|}
\hline
保存词条 & \verb|Ctrl| + \verb|S| & 打开词条 & \verb|Ctrl| + \verb|O| \\
\hline
新建词条 & \verb|Ctrl| + \verb|Alt| + \verb|N| & 关闭词条 & \verb|Ctrl| + \verb|Alt| + \verb|W| \\
\hline
查找文本 & \verb|Ctrl| + \verb|F| & 替换文本 & \verb|Ctrl| + \verb|H| \\
\hline
增大字号 & \verb|Shift| + \verb|Alt| + \verb|+| & 减小字号 & \verb|Shift| + \verb|Alt| + \verb|-| \\
\hline
显示编辑器选项 & \verb|Ctrl| + \verb|Q| & 跳转到某行 & \verb|Ctrl| + \verb|G| \\
\hline
向左缩进 & \verb|Ctrl| + \verb|[| & 向右缩进 & \verb|Ctrl| + \verb|]| \\
\hline
关闭不保存 & \verb|Shift| + \verb|点击关闭| &  &  \\
\hline
\end{tabular}
\end{table}

\begin{itemize}
\item 【新】快捷功能: 若剪切板有图片(例如截图以后), 在编辑器中直接用 \verb|Ctrl+V| 就可以上传该图片。
\item 将光标停留在任意按钮上都会出现提示说明按钮的名称。 要新建词条, 点击红色的加号按钮, 根据提示新建即可。 要打开已有词条, 点击最右边的打开, 搜索需要的词条即可。
\item 正文中请使用中文标点, 编辑器会自动把空心句号替换为全角实心句号(如果你在使用笔记功能,可以选择在设置中关闭这个功)。
\item 编辑器会将有改动的词条每隔 5 分钟备份一次, 可以用 “恢复” 按钮恢复历史版本。
\item 双击词条的 tab 也可以关闭词条。
\item 编辑器支持各种自动引用(被引用对象没有 label 时会自动插入 label), 工具栏上的\textbf{内部引用}按钮可以引用同一词条的公式, 图表, 例题等环境。 \textbf{外部引用}按钮可以引用其他词条的各种环境。
\item 如果要在网页预览和 LaTex 代码之间跳转到对应位置, 可以通过搜索关键词实现。 例如在预览窗口复制一段文字, 在编辑窗口搜索就可以跳转到对应内容。
\item 任何时候打出反斜杠会自动提示可以自动补全的命令, 用上下键选择, 用 Tab 键或回车确认。 候选词未必是从最左边开始匹配, 例如打 \verb|\bf| 按 tab 就会得到 \verb|\textbf{}|。
\item 如果自动补全带括号, 例如 \verb|\frac{}{}|, 补全后光标会自动进入第一个大括号, 再次按 Tab 光标会跳到第二个括号, 再按 Tab 光标会跳到第二个大括号外。
\item 打 \verb|\beq| 按 Tab 会自动出现 \verb|\begin{equation}...\end{equation}|, 其他环境也同理。
\item 用光标选中一串字符后按下加粗按钮, 这串字符会自动插入 \verb|\textbf{}| 中。 同样, 选中字符串后输入 \verb|(| 等括号, 这个字符串会自动插入 \verb|()| 中。 表示行内公式的 \verb|$$| 也支持该操作。 用 “对齐” 按钮添加 \verb|aligned| 环境同理。
\item 编辑器在生成网页时会将编辑器中的 LaTex 代码转换为通用的 LaTex 代码(网页中右键点击公式获得), 即不需要自定义命令和额外宏包, 可以在任何支持 LaTeX 公式的环境使用(如知乎的公式编辑器)。
\item 编辑器支持一键转载到知乎(有少量不兼容, 例如表格和代码), 使用菜单栏的 “导出 Markdown 文件” 按钮, 然后将导出的文件上传到知乎即可。 该功能只能在编辑个人笔记时使用, 未经允许请勿转载百科内容。
\item 若要搜索整个百科的 LaTex 源码, 用\href{https://github.com/MacroUniverse/PhysWiki-log/tree/master/contents}{这个页面}, 若要搜索所有作者的编辑历史, 用\href{https://github.com/MacroUniverse/PhysWiki-backup}{这个页面}。
\end{itemize}

\subsection{编辑器设置}
编辑器中的 “设置” 按钮(齿轮图标)可以添加 “自动补全” 规则。 “自动补全” 如上文所描述的, 在输入 LaTeX 命令的过程中, 候选框会显示可以补全的命令, 用上下键选择命令, 然后用 tab 键或回车键补全。 补全规则的格式说明可以点击设置面板中的帮助按钮获得。

“符号替换” 功能是指, 在 LaTeX 公式中输入一些命令时, 编辑器会自动将其显示为对应的符号, 例如 \verb|\alpha| 显示为 \lstinline|α|, \verb|\sum| 显示为 \lstinline|∑| 等。 这样做是为了增加源码的可读性, 注意这只是一种视觉效果, 不会影响源码本身。 设置面板中的也可以添加 “符号替换” 规则或者将其关闭。

\addTODO{详细介绍设置面板的其他按钮}
\addTODO{tab 可以拖动}
