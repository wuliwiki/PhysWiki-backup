% 小时百科的类似项目

以下列出一些可供百科参考的项目.

\subsection{内容}
\begin{itemize}
\item \href{https://www.matongxue.com/}{马同学} 高数、线代、概率统计、机器学习等教程
\item \href{https://www.bananaspace.org/wiki/}{香蕉空间} 中文数学社区(百科、讲义、讨论室) CC BY-SA 4.0
\item \href{https://brilliant.org/}{Brilliant} 含有数学分析等内容的百科, 配有例题习题. 还有很多互动插件.
\item \href{http://hyperphysics.phy-astr.gsu.edu/}{Hyperphysics}
\item \href{https://openstax.org/}{OpenStax} CC-BY 协议! 没有 SA
\item \href{http://www.astro.uvic.ca/~tatum/index.php}{Physics topics by Tatum} 含有一些天文内容
\item \href{http://mathmu.github.io/MTCAS/RecentChanges.html}{MaTHmu} 清华学生国产符号计算软件, 有很多数值/符号计算文档, 还有 \href{https://github.com/maTHmU/MTCAS}{Mathematica Theory of Computer Algebra System.pdf}
\item \href{https://www.termonline.cn/index}{术语在线} 对照中英文术语
\item \href{https://oi-wiki.org/}{信息学奥林匹克竞赛} 致力于成为一个免费开放且持续更新的编程竞赛知识整合站点
\item \href{http://www.myliushu.com/}{刘叔物理} 高中物理网站
\item \href{https://fungrim.org/}{The Mathematical Functions Grimoire}
\item \href{https://mathworld.wolfram.com/}{Wolfram Mathworld}
\end{itemize}

\subsection{互动演示}
\begin{itemize}
\item \href{https://phet.colorado.edu/}{PhET} 老牌互动插件, 创始人 Carl Wieman 跟我合影过
\item \href{https://ophysics.com/}{ophysics} 翻译到百科
\item \href{https://www.nobook.com/}{nobook} 国产 PhET 虚拟实验室
\item \href{https://www.falstad.com/}{pfalstad} 比较应和的网页模拟
\item \href{http://www.cs.cornell.edu/courses/cs5643/2010sp/}{计算机 CG 课程}
\item \href{https://demonstrations.wolfram.com/topics.php?PhysicalSciences#5}{Wolfram Demonstration Project}
\item \href{https://github.com/3b1b/manim}{Manim} 3b1b 视频使用的视频制作 python 库
\end{itemize}

\subsection{公众号}
\begin{itemize}
\item 数林广记 JierPeter 提到的, 数学非科普
\item 周思益书签里面有一些抖音快手知乎的科普号
\end{itemize}

\subsection{计算机}
\begin{itemize}
\item \href{https://pythonnumericalmethods.berkeley.edu/notebooks/Index.html}{Python Numerical Methods}
\item \href{http://www.drhuang.com/}{黄博士网} 非常简单的百科和类似 Wolfram Alpha 的云符号计算
\item \href{https://www.latexlive.com/}{latexlive}
\item \href{https://www.qikegu.com/}{奇客谷}
\item \href{https://harrypotter.fandom.com/zh/wiki/Special:%E7%94%A8%E6%88%B7%E8%B4%A1%E7%8C%AE/Laoxie.H}{Fandom} 使用 MediaWiki 的百科网站
\item \href{http://c.biancheng.net/}{C语言中文网}
\end{itemize}

% \subsection{公开课}
% \begin{itemize}
% \item 网易公开课、腾讯课堂、华为学院
% \item edx、udecity、khan、 3b1b、 coursera
% \end{itemize}
