% 2017 年考研数学试题(数学一)
% keys 考研|数学
% license Xiao
% type Tutor

\subsection{选择题}
1.若函数 $f(x)=\leftgroup{& \frac{1-\cos \sqrt{x}}{ax},\quad  & x>0 \\ & b,\quad & x \le 0}$ 在 $x=0$ 处连续,则 $(\quad)$\\
(A)$ab=\frac{1}{2} \quad$
(B)$ab=-\frac{1}{2} \quad$
(c)$ab=0 \quad$
(D)$ab=2 \quad$

2.设函数 $f(x)$ 可导,且 $f(x)f'(x)>0$ ,则 $(\quad)$ \\
(A)$f(1)>f(-1) \qquad$
(B)$f(1)<f(-1) \quad$\\
(c)$\abs{f(1)}>\abs{f(-1)} \quad$
(D)$\abs{f(1)}<\abs{f(-1)} $

3.函数 $f(x,y,z)=x^2y+z^2$ 在点 $(1,2,0)$ 处沿向量 $n=(1,2,2)$ 的方向导数为 $(\quad)$\\
(A)$12 \quad$
(B)$6 \quad$
(c)$4 \quad$
(D)$2 \quad$


4.甲,乙两人赛跑,及时开始时,甲在乙前方10(单位:m)处,图中,实线表示甲的速度曲线 $v=v_1(t)$(单位:m/s),虚线表示已的速度曲线 $v=v_2(t)$ ,三块阴影部分面积的数值依次是10,20,3。计时开始后乙追上甲的时刻记为 $t_0$ (单位:s),则 $(\quad)$\\
\begin{figure}[ht]
\centering
\includegraphics[width=6cm]{./figures/fff65d0b7592b974.png}
\caption{} \label{fig_PeeM17_1}
\end{figure}
(A)$t_0=10 \quad$
(B)$15<t_0<20 \quad $
(c)$t_0=25\quad$
(D)$t_0>25$

5.设 $\bvec \alpha$ 为 $n$ 维单位列向量,$\mat E$ 为 $n$ 阶单位矩阵,则 $(\quad)$\\
(A)$\mat E- \bvec \alpha \bvec \alpha \Tr$不可逆 $\quad$
(B)$\mat E+ \bvec \alpha \bvec \alpha \Tr$不可逆\\
(c)$\mat E+ 2\bvec \alpha \bvec \alpha \Tr$不可逆 $\quad$
(D)$\mat E- 2\bvec \alpha \bvec \alpha \Tr$不可逆

6.已知矩阵 $\mat A=\pmat{2&0&0\\0&2&1\\0&0&1},\mat B=\pmat{2&1&0\\0&2&0\\0&0&1},\mat C=\pmat{1&0&0\\0&2&0\\0&0&2}$,则 $(\quad)$\\
(A)$A$与$C$ 相似, $B$ 与 $C$ 相似 $\quad$
(B)$A$与$C$ 相似, $B$ 与 $C$ 不相似 \\
(c)$A$与$C$ 不相似, $B$ 与 $C$ 相似 $\quad$
(D)$A$与$C$ 不相似, $B$ 与 $C$ 不相似 $\quad$

7.设 $A,B$ 为随机事件,若 $0<P(A)<1,0<P(B)<1$ ,则 $P(A|B)>P(A|\bar{B})$  的充分必要条件是 $(\quad)$\\
(A)$P(B|A)>P(B|\bar{A}) \quad$
(B)$P(B|A)<P(B|\bar{A}) \quad$\\
(c)$P(\bar{B}|A)>P(B|\bar{A}) \quad$
(D)$P(\bar{B}|A)<P(B|\bar{A}) \quad$

8.设 $X_1,X_2,\dots,X_n \quad (n \ge 2)$ 为来自总体 $N(\mu,1)$ 的简单随机样本,记 $\displaystyle \bar{X}=\frac{1}{n} \sum_{i=1}^n X_i$ ,则下列结论中不正确的是 $(\quad)$\\ 
(A)$\displaystyle \sum_{i=1}^n (X_i-\mu)$服从 $\chi^2$ 分布 $\quad$
(B)$2(X_n-X_1)^2$ 服从 $\chi^2$ 分布\\
(c)$\displaystyle \sum_{i=1}^n (X_i-\bar{X})$服从 $\chi^2$ 分布 $\quad$
(D)$n(\bar{X}-\mu)^2$ 服从  $\chi^2$ 分布

\subsection{填空题}
1.已知函数 $ \displaystyle f(x)=\frac{1}{1+x^2}$  ,则   $f^{(3)}(0)=(\quad)$。

2.微分方程 $y''+2y'+3y=0$  的通解为 $y=(\quad)$。

3.若曲线积分 $\displaystyle \int_L \frac{x\dd{x-ay\dd{y}}}{x^2+y^2-1}$ 在区域 $D=\{(x,y)|x^2+y^2<1\}$ 内与路径无关,则 $a=(\quad)$。

4.幂级数 $\displaystyle \sum_{n=1}^\infty (-1)^{n-1} nx^{n-1}$  在区间 $(-1,1) $内的和函数 $S(x)=(\quad)$ 。

5.设矩阵 $\mat A=\pmat{1&0&1\\1&1&2\\0&1&1}$,$\mat \alpha_1,\mat \alpha_2,\mat \alpha_3$  为线性无关的3维列向量组,则向量组 $\mat A\mat \alpha_1,\mat A \mat \alpha_2,\mat A \mat \alpha_3$ 的秩为 $(\quad)$。

6.设随机变量 $X$  的分布函数为 $\displaystyle F(x)=0.5\Phi(x)+0.5\Phi(\frac{x-4}{2})$  ,其中 $\Phi(x)$ 为标准正态分布函数,则 $E(X)=(\quad)$。


\subsection{解答题}
1.设函数 $f(u,v)$  具有2阶连续偏导数, $y=f(e^x,\cos x)$  ,求 
$\displaystyle \left.\frac{\mathrm{d} y}{\mathrm{~d} x}\right|_{x=0},\left.\frac{\mathrm{d}^2 y}{\mathrm{~d} x^2}\right|_{x=0}$。

2.求 $\displaystyle \lim_{n \to \infty} \sum_{k=1}^n \frac{k}{n^2} \ln(1+\frac{k}{n})$。

3.已知函数 $y(x)$ 由方程 $x^3+y^3-3x+3y-2=0$  确定,求 $y(x)$ 的极值。

4.设函数  $f(x)$ 在区间 $[0,1]$  上具有2阶导数,且 $\displaystyle f(1)>0,\lim_{x \to 0^+}\frac{f(x)}{x}<0$ ,  证明:\\
(1).方程 $f(x)=0$  在区间 $(0,1)$ 内至少存在一个实根;\\
(2).方程 $f(x)f''(x)+[f'(x)]^2=0$ 在区间 $(0,1)$ 内至少存在两个不同实根。

5.设薄片型物体 $S$ 是圆锥面  $z=\sqrt{x^2+y^2}$ 被柱面 $z^2=2x$  割下的有限部分,其上任一点的密度为$\mu(x,y,z)=9\sqrt{x^2+y^2+z^2}$   。记圆锥面与柱面的交线为 $C$。\\
(1).求 $C$ 在 $xOy$  平面上的投影曲线的方程;\\
(2).求 $S$ 的质量 $M$ 。

6.设3阶矩阵 $\mat A=(\bvec \alpha_1,\bvec \alpha_2,\bvec \alpha_3)$  有3个不同的特征值,且 $\bvec \alpha_3=\bvec \alpha_1+2\bvec \alpha_2$ 。\\
(1).证明 $r(\mat A)=2$ ;\\
(2).设 $\bvec{\beta=\alpha_1+\alpha_2+\alpha_3}$  ,求方程组 $\bvec{Ax=\beta}$  的通解。

7.设二次型  $f(x_1,x_2,x_3)=2x_1^2-x_1^2+ax_3^2+2x_1x_2-8x_1x_3+2x_2x_3$ 在正交变换  $x=Qy$ 下的标准形为 $\lambda_1y_1^2+\lambda_2y_2^2$  ,求 $a$ 的值及一个正交矩阵 $Q$ 。

8.设随机变量 $X,Y$  相互独立,且 $X$  的概率分布为 $P\{X=0\}=p\{X=2\}=\frac{1}{2}$  , $Y$ 的概率密度为 $f(y)=\leftgroup{&2y,\quad & 0<y<1,\\&0,\quad &\text{其他}}$ 。\\
(1).求 $P\{Y \le E(Y)\}$ ;\\
(2).求 $Z=X+Y$ 的概率密度。

9.某工程师为了解一台天平的精度,用该天平对一物体的质量做 $n$ 次测量,该物体的质量 $\mu$ 是已知的。设 $n$ 次测量结果 $X_1,X_2,\dots,X_n$  相互独立且均服从正态分布 $N(\mu,\sigma^2)$ ,该工程师记录的是 $n$  次测量的绝对误差 $Z_i=\abs{X_i-\mu} \quad (i=1,2,\dots,n)$  。利用  4 估计  。
