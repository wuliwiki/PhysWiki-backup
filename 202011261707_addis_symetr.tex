% 对称化与反对称化

\pentry{粒子交换算符\upref{ExchOp}}

如果给定一个多粒子波函数处于对称或反对称子空间之外, 特定情况下我们可以将其\textbf{对称化(symmetrize)}或\textbf{反对称化(antisymmetrize)}. 先用粒子交换算符表示\autoref{IdPar_ex1}~\upref{IdPar}的过程.

\begin{example}{}\label{symetr_ex2}
假设单个粒子态空间的一组单位正交基底是 $\qty{\ket{i}}$, 两个粒子某时刻的态矢可以用单个张量积表示 $\ket{i}\ket{j}$ 且 . 若 $i = j$, 则显然这个态已经是对称的.  若 $i \ne j$, 则态矢是不对称的, 我们可以通过乘以 $(1 \pm P_{12})/\sqrt{2}$ 来对称或反对称化(分别取正号和负号)
\begin{equation}\label{symetr_eq5}
\frac{1}{\sqrt{2}}(1 \pm P_{12})\ket{i}\ket{j} = \frac{1}{\sqrt{2}}(\ket{i}\ket{j} \pm \ket{j}\ket{i})
\end{equation}
由于等式中的两项也是正交归一的, 我们需要另乘归一化系数 $1/\sqrt 2$. 可以验证, 该式就是 $P_{1,2}$ 的本征矢, 本征值分别为 $\pm 1$. 这种方法利用了交换算符的性质\autoref{ExchOp_eq7}~\upref{ExchOp}.

更一般地, 我们也可以不要求 $\ket{i}\ket{j}$ 正交, 但这样\autoref{symetr_eq5} 中的两项也变得不正交, 我们就要重新计算归一化系数了.
\end{example}

在该例中, 我们把 $(1 \pm P_{12})/\sqrt{2}$ 称为\textbf{对称化算符(symmetrizer)}或\textbf{反对称化算符(antisymmetrizer)}.

\subsection{多粒子对称化}
若将\autoref{symetr_ex2} 中的两个粒子改为 $N$ 个粒子, 如何将状态 $\ket{i_1}\ket{i_2}\dots\ket{i_N}$ (反)对称化呢? 稍加思考会发现, 若要对称化, 我们只需要将 $\ket{i_1},\ket{i_2},\dots,\ket{i_N}$ 的所有不同的排列相加再归一化即可. 如果这 $N$ 个单粒子态都是不同的, 那么一共有 $N!$ 种排列. 我们用 $p_n(i)$ 来表示, 例如
\begin{table}[ht]
\centering
\caption{$N = 3$ 的 6 种排列}\label{symetr_tab2}
\begin{tabular}{|c|c|c|c|c|c|c|}
\hline
  & $p_1(i)$ & $p_2(i)$ & $p_3(i)$ & $p_4(i)$ & $p_5(i)$ & $p_6(i)$ \\
\hline
$i=1$ & 1 & 1 & 2 & 2 & 3 & 3 \\
\hline
$i=2$ & 2 & 3 & 1 & 3 & 1 & 2 \\
\hline
$i=3$ & 3 & 2 & 3 & 1 & 2 & 1 \\
\hline
\end{tabular}
\end{table}
则正交化的结果为
\begin{equation}\label{symetr_eq1}
\frac{1}{\sqrt{N!}}\sum_{n = 0}^{N!} \ket{p_n(1)}\ket{p_n(2)}\dots\ket{p_n(N)}
\end{equation}
为了验证这是一个对称态, 我们可以用例如 $P_{i,j}$ 作用在上面, 这相当于把\autoref{symetr_tab2} 中的第 $i,j$ 行调换, 容易得出这不会改变\autoref{symetr_eq1}.

如果 $\ket{i_1},\dots,\ket{i_N}$ 中出现重复, 情况就要更复杂一些. 令其中只有 $M < N$ 种不同的单粒子态, 重复的次数分别是 $n_1, \dots, n_M$, 有 $\sum_i n_i = N$. 这样一来不同的排列减少至 $N!/(n_1! n_2! \dots n_M!)$ 种,% 未完成: 链接到排列例题
 我们仍然可以写出类似\autoref{symetr_eq1} 的表达式, 但求和只有 $N!/(n_1! n_2! \dots n_M!)$ 个正交归一的项, 所以归一化系数也变为 $\sqrt{n_1! n_2! \dots n_M!/N!}$.

\subsubsection{对称化算符}
要把对称化用算符表示出来也不难, 我们可以对每一种不同的排列 $p_n(1), p_n(2), \dots$ 都定义一个对称化算符 $P_n$. 定义反对称化算符为(求和的项数同样取决于是否出现重复)
\begin{equation}\label{symetr_eq2}
\mathcal S = 1 + \sum_n P_n
\end{equation}
这样对称化就可以优雅地表示为 $\mathcal S \ket{i_1}\dots\ket{i_N}$.

\subsection{多粒子反对称化}
我们同样也可以使用排列算符对 $N$ 粒子态 $\ket{i_1}\ket{i_2}\dots\ket{i_N}$ 进行反对称化, 但前提是我们必须要求其中 $N$ 个单粒子态都是不同的. 这时因为如果 $\ket{j} = \ket{k}$, 那么无论如何排列, 得到的态关于交换算符 $P_{j,k}$ 都是对称的. 这就是泡利不相容原理\upref{PauliE}.

由\autoref{symetr_eq5} 的启发, 我们可以尝试改变\autoref{symetr_eq1} 或\autoref{symetr_eq2} 中一些项的正负号来达到反对称化. 事实上行列式的定义\upref{Deter}中已经给出了我们需要规则, 即使用逆序数的奇偶性来决定正负号. 对称化的结果可以用行列式记为
\begin{equation}\label{symetr_eq3}
\vmat{
\ket{i_1} & \ket{i_2} & \dots & \ket{i_N}\\
\ket{i_1} & \ket{i_2} & \dots & \ket{i_N}\\
\dots & \dots  & \dots & \dots \\
\ket{i_1} & \ket{i_2} & \dots & \ket{i_N}
}
\end{equation}
这个行列式被称为\textbf{斯莱特行列式(Slater determinant)}.

要证明反对称性很简单, 任意交换算符 $P_{i,j}$ 作用在\autoref{symetr_eq3} , 相当于把行列式的两行置换, 而根据行列式的性质(\autoref{DetPro_the6}~\upref{DetPro}), 这会使结果取相反数. 证毕.

通过在\autoref{symetr_eq2} 中根据 $P_n$ 的逆序数在其前面适当添加正负号 $S_n$(见\autoref{Deter_eq4}~\upref{Deter}), 我们也可以定义反对称化算符
\begin{equation}
\mathcal A = 1 + \sum_n S_n P_n
\end{equation}

\subsection{子空间的维度}
先来看有限维的情况, 假设单个粒子态所在空间是 $M$ 维的(例如只考虑自旋空间, $M = 2s+1$), 那么只有当粒子数 $N \le M$ 时, 才能允许全同费米子的态矢. 通过基底的不同组合\upref{combin}以及对称化, 我们一共可以找到 $C_M^N$ 种不同的反对称基底, 所以\footnote{这不是证明只是对结论的一个简单描述.}反对称子空间是 $C_M^N$ 维的. 要得到这 $C_M^N$ 个基底, 我们只需要从 $M$ 个单粒子基底中无顺序不重复地选出 $N$ 个($\ket{i_1}, \dots, \ket{i_N}$). 每选一次就可以得到一个对称基底 $\mathcal A \ket{i_1}\dots\ket{i_N}$, 它们张成整个反对称子空间. 物理上, 这意味着 $N$ 个全同费米子占据 $M$ 个状态, 每种可能就是一个基底.

对于玻色子, 我们不要求 $N \le M$, 因为多个玻色子可以处于同一个单粒子态. 可以证明对称子空间的维数等于 “从 $M$ 个态中无序地选 $N$ 个, 允许重复” 的个数, 但遗憾的是我们不能把这个数写成一个简洁的公式. 对称子空间基底的构建也同理, 每次选取出 $N$ 个态后都可以构建一个基底 $\mathcal S \ket{i_1}\dots\ket{i_N}$.
