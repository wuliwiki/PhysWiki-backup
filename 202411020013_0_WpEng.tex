% 电磁波包的能谱
% keys 电磁波|波包|能量|频谱|能谱|光谱
% license Xiao
% type Tutor

\pentry{傅里叶变换(指数)\nref{nod_FTExp}, 平面电磁波的能量叠加\nref{nod_PwvAdd}}{nod_2e21}

对于真空中的平面波电磁波,沿 $x$ 轴传播, 波速恒定为 $c$, 如果知道某点 $x_0$ 处的电场—时间关系 $g(t)$, 如何求波函数 $f(x - ct)$ 呢? 代入 $x = x_0$ 可知 $g(t) = f(x_0 - ct)$, 所以
\begin{equation}
f(x) = g\qty(\frac{x_0 - x}{c})~.
\end{equation}

当这个波包完整穿过一个 $y$-$z$ 平面后, 穿过平面的\textbf{能量面密度} $\sigma_E$ 等于能量体密度(\autoref{eq_VcPlWv_2})在传播方向的积分(积分上下限为 $\pm\infty$)
\begin{equation}\label{eq_WpEng_1}
\sigma_E = \epsilon_0 \int f(x)^2 \dd{x} = \epsilon_0  \int g^2\qty(\frac{x_0 - x}{c}) \dd{x} = c\epsilon_0 \int g^2(u) \dd{u}~.
\end{equation}
另一种方法是把\enref{坡印廷矢量}{EBS}对时间积分, 同样能得到该式。

\subsection{能量的频率分布}
根据傅里叶变换的归一化不变性(\autoref{eq_FTExp_2}), 若令 $g$ 的傅里叶变换为 $\tilde g$ 则\footnote{第二个等号中, 由\autoref{eq_FTExp_5} 得 $\abs{\tilde g(-\omega)}^2 = \abs{\tilde g(\omega)}^2$, 所以负半轴的积分与正半轴相等。}
\begin{equation}
\sigma_E = c\epsilon_0 \int_{-\infty}^{+\infty} \abs{\tilde g(\omega)}^2 \dd{\omega} = 2c\epsilon_0 \int_{0}^{+\infty} \abs{\tilde g(\omega)}^2 \dd{\omega}~.
\end{equation}
这相当于把波包看作是许多不同频率简谐波的叠加, 总能量面密度是每个简谐波的能量面密度叠加。 所以能量面密度的频率分布,即\textbf{单位频率的能量面密度}为\footnote{原子单位: $s(\omega) = c \abs{\tilde g(\omega)}^2/(2\pi)$}
\begin{equation}\label{eq_WpEng_3}
s(\omega) = 2c\epsilon_0 \abs{\tilde g(\omega)}^2~.
\end{equation}
考虑到光子能量为 $E = \omega\hbar$, 光子能量分布为
\begin{equation}\label{eq_WpEng_2}
s(E) = \frac{2c\epsilon_0}{\hbar} \abs{\tilde g\qty(\frac{E}{\hbar})}^2~.
\end{equation}

\subsubsection{用矢势表示}
\pentry{库仑规范(电动力学)\nref{nod_Cgauge}}{nod_c215}
在库仑规范下, 矢势为 $A(t)$, 对于波包有(\autoref{eq_Cgauge_2})
\begin{equation}
g(t) = -\dv{A(t)}{t}~.
\end{equation}
由傅里叶变换的求导公式(\autoref{eq_FTExp_3})得
\begin{equation}\label{eq_WpEng_4}
\tilde g(\omega) = -\I \omega \tilde A(\omega)~.
\end{equation}
代入\autoref{eq_WpEng_3} 得
\begin{equation}\label{eq_WpEng_5}
s(\omega) = 2c\epsilon_0 \omega^2 \abs{\tilde A(\omega)}^2~,
\end{equation}
若使用\enref{高斯单位制}{GaussU}, 有
\begin{equation}
s(\omega) = \frac{\omega^2}{2\pi c} \abs{\tilde A(\omega)}^2 \qquad (\text{高斯单位制})~.
\end{equation}
