% 堆放排列组合
% keys 排列组合|堆放排列组合

\pentry{隔板法(排列组合)\upref{BarCom}, 范德蒙恒等式\upref{ChExpn}}

如果有 $n$ ($n=1,2\dots$)个不加区分的小球, 有 $N$ 个有编号的盒子, 每个盒子能装的个数不限(可为空), 那么把所有小球都放到盒子里有几种不同方法?

现在把所有的情况根据非空盒子的个数分类. 非空盒子个数可能为1个( $n$ 个小球都在里面), 2个, 一直到 $\min\qty{n,N}$ 个(若 $n\leqslant N$ 则每个小球都在不同的盒子). 如果用 $i$ 个盒子装小球, 那么首先从 $N$ 个盒子里面选择 $i$ 个会有 $C_n^i$ 种情况. 然后要考虑这 $i$ 个有编号盒子装 $n$ 个小球(不为空)又有几种情况. 用隔板法\upref{BarCom}得到共有 $C_{n-1}^{i-1}$ 种情况. 所以一个 $i$ 对应 $C_N^i C_{n-1}^{i-1}$ 种情况.

最后把所有不同 $i$ 的情况数加在一起, 得出所有情况的总数为
\begin{equation}
\sum_{i = 1}^{\min\qty{n,N}} C_N^i C_{n-1}^{i-1}
\end{equation}
又由于 $C_a^b = a!/[(a-b)!b!] = C_a^{a-b}$, 上式可变为
\begin{equation}
\sum_{i=1}^{\min\qty{n,N}}  C_N^i C_{n-1}^{n-i}
\end{equation}
又由范德蒙恒等式\upref{ChExpn}, 上式变为
\begin{equation}
\sum_{i=1}^n C_N^i C_{n-1}^{n-i} = C_{N+n-1}^n = \frac{(N+n-1)!}{n!(N - 1)!}
\end{equation}
这就是最后的答案.
