% 魏尔施特拉斯逼近定理
% keys Stone–Weierstrass|一致逼近|三角级数|泰勒级数|傅里叶级数
\begin{issues}
\issueDraft
\end{issues}

\pentry{泰勒级数(简明微积分)\upref{Taylor}, 傅里叶级数(三角)\upref{FSTri}}

\footnote{参考 Wikipedia \href{https://en.wikipedia.org/wiki/Stone–Weierstrass theorem}{相关页面}.}泰勒级数中, 只有无穷阶可导函数才能用泰勒公式展开成多项式, 但事实上多项式还可以展开更多函数.

\begin{theorem}{魏尔施特拉斯近似定理(Weierstrass approximation theorem)}
闭区间上的连续函数可用多项式级数一致逼近. 具体来说就是若 $f(x)$ 为闭区间 $[a, b]$ 的连续实函数, 那么对于任意给定的 $\epsilon$, 都存在多项式 $p(x)$, 使得 $\abs{f(x) - p(x)} < \epsilon$ 在该区间成立.
\end{theorem}


\begin{theorem}
斯通—魏尔施特拉斯逼近定理有两个
\begin{enumerate}
\item 
\item 闭区间上周期为 $2\pi$ 的连续函数可用三角函数级数一致逼近.
\end{enumerate}
第一逼近定理可以推广至 $\mathbb {R}^{n}$ 上的有界闭集.
\end{theorem}

要求多项式系数, 可以先求三角傅里叶级数(\autoref{FSTri_eq1}~\upref{FSTri})
\begin{equation}
f(x) = \frac{a_0}{2} + \sum_{n = 1}^\infty a_n \cos (\frac{n\pi}{l}x) + \sum_{n = 1}^\infty b_n \sin (\frac{n\pi}{l}x)
\end{equation}
然后使用泰勒公式展开三角函数(\autoref{Taylor_eq10}~\upref{Taylor})得
\begin{equation}
f(x) = \sum_{n=0}^\infty c_n x^n
\end{equation}
\begin{equation}
c_0 = \sum_{n=0}^\infty a_n~,
\quad
c_2 = -\frac{1}{2!}\frac{\pi^2}{l^2} \sum_{n=0}^\infty n^2 a_n~, \quad \dots
\end{equation}
\begin{equation}
c_1 = \frac{\pi}{l} \sum_{n=0}^\infty n b_n~,
\quad
c_3 = -\frac{1}{3!}\frac{\pi^3}{l^3} \sum_{n=0}^\infty n^3 b_n~, \quad \dots
\end{equation}
注意以上求和需要检查是否收敛.

有限项三角级数在无穷远处总是周期的, 而有限项幂级数展开在无穷远处总是发散的.

另一种方法可以用多项式插值(未完成), 需要解方程组, 计算量可能更大. 两种方法都会出现龙格现象(未完成).
