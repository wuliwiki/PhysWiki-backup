% 有界变差
% keys 实分析|数学分析|黎曼-斯蒂尔杰斯积分
% license Usr
% type Wiki

有界变差(Bounded Variation)函数的总变差是有限的。这一概念是后面学习黎曼-斯蒂尔杰斯积分的关键基础。有界变差是描述实数轴上函数变化幅度的一种度量方式。

考虑一个实值函数$f(x)$,在闭区间$a\leq x\leq b$上定义并有限。我们将这个区间分成一些小区间,形成一个分割
\begin{align} 
\Gamma=\{x_{0},\,x_{1},\,\ldots,\,x_{m}\} ~.
\end{align},也就是$\Gamma$是点$x_{i}$的集合,满足$x_{0}=a$,$x_{m}=b$,且$x_{i-1}<x_{i}$。

对于每个分割$\Gamma$,我们计算一种和$S_{\Gamma}$,表示相邻点$f(x_{i})$和$f(x_{i-1})$的绝对差的总和。也就是,
\begin{align} 
S_{\Gamma}=S_{\Gamma}[f;a,b]=\sum_{i=1}^{m}|f(x_{i})-f(x_{i-1})|~.
\end{align}
函数$f$在$[a,b]$上的变差定义为
\begin{align} 
V=V[f;a,b]=\sup_{\Gamma}S_{\Gamma}~, 
\end{align}
其中$\sup$取遍$[a,b]$的所有分割$\Gamma$。由于$0\leq S_{\Gamma}<+\infty$,我们有$0\leq V\leq+\infty$。如果$V$有限,那么$f$在$[a,b]$上的变差有界;如果$V$是无穷大,那么$f$在$[a,b]$上的变差无界。

下面列举几个简单有界变差函数的例子:

\textbf{例子 1}:假设$f$在$[a,b]$上单调。那么,显然,每个$S_{\tau}$都等于$|f(b)-f(a)|$,因此$V=|f(b)-f(a)|$。

\textsl{证明}:设${x_{i}:1\leq i\leq n}$是$[a,b]$的一个分割。考虑

\begin{align} \sum_{i=1}^{n}|f(x_{i})-f(x_{i-1})|=\sum_{i=1}^{n}\left(f(x_{i})-f(x_{i-1})\right)=f(b)-f(a)~. \end{align}

由于这个和的抵消性质,它对$[a,b]$的任何分割都是相同的。因此我们可以看到$V(f,[a,b])=f(b)-f(a)<\infty$。因此$f$在$[a,b]$上是有界变差的。

类似地,如果$f$在$[a,b]$上是递减的,则$V(f,[a,b])=f(a)-f(b)$。

\textbf{例子 2}:假设$f$的图形可以分为有限数量的单调弧段;即假设$[a,b]=\bigcup_{i=1}^{k}[a_{i}a_{i+1}]$,并且$f$在每个$[a_{i}a_{i+1}]$上是单调的。那么$V=\sum_{i=1}^{k}|f(a_{i+1})-f(a_{i})|$。

\textbf{例子 3}:设$f$是狄利克雷函数,定义为$f(x)=1$对于有理数$x$,$f(x)=0$对于无理数$x$。那么,显然,对于任何区间$[a,b]$,$V[a,b]=+\infty$。

\textsl{证明}:设$n\in\mathbb{Z}$,$n>0$。设$[a,b]$是$\mathbb{R}$中的闭区间。我们构造一个分割$P={x_{0},x_{1},...,x_{n+2}}$,使得$V(f,[a,b])\geq\sum_{i=1}^{n+2}|f(x_{i})-f(x_{i-1})|>n$,具体步骤如下。根据定义,$x_{0}=a$。由下面定理:
\begin{theorem}{在任意两个不同的实数之间,存在一个有理数和一个无理数。}
\end{theorem}
可取$x_{1}$为介于$a$和$b$之间的一个无理数。然后取$x_{2}$为介于$x_{1}$和$b$之间的一个有理数。以此类推,取$x_{2i+1}$为介于$x_{2i}$和$b$之间的一个无理数,取$x_{3i}$为介于$x_{2i-1}$和$b$之间的一个有理数。最后,$x_{n+2}=b$。因此,我们已经创建了一个从$a$开始并且在有理数和无理数之间交替的分割,直到最后以$b$结束。现在考虑和$\sum_{i=1}^{n+2}|f(x_{i})-f(x_{i-1})|$,我们知道它至多等于$f$在$[a,b]$上的变差。因此:
\begin{align} 
V(f,[a,b])&\geq\sum_{i=1}^{n+2}|f(x_{i})-f(x_{i-1})| \\ &\geq\sum_{i=2}^{n+1}|f(x_{i})-f(x_{i-1})| \\ &=|f(x_{2})-f(x_{1})|+\cdots+|f(x_{n+1})-f(x_{n})| \\ &=|1-0|+|0-1|+\cdots+|1-0| \\ &=1+1+1+\cdots+1=n ~. 
\end{align}
因此$V(f,[a,b])$是任意大的,因此$V(f,[a,b])=\infty$。

\textbf{例子 4}:定义在$[a,b]$上的函数$f$被称为在$[a,b]$上满足Lipschitz条件,如果存在常数$C$使得
\begin{align} 
|f(x)-f(y)|\leq C|x-y|, \
\forall x,y\in[a,b]~.
\end{align}
这样的函数显然是有界变差的,$V[f;a,b]\leq C(b-a)$。例如,如果$f$在$[a,b]$上有连续的导数,那么(根据中值定理)$f$在$[a,b]$上满足Lipschitz条件。

\begin{example}{设 $f(x)=x\sin\left(1/x\right)$ 为 $0<x\leq1$ 且 $f(0)=0$。 证明 $f$ 在 $[0,1]$ 上有界且连续,但不是有界变差$V[f;0,1]=+\infty$。}

对于 $0 < x \leq 1$,我们知道 $-1 \leq \sin\left(\frac{1}{x}\right) \leq 1$,因此 $-x \leq x\sin\left(\frac{1}{x}\right) \leq x$,在 $[0,1]$ 上 $f(x)$ 有界。

考虑 $x \neq 0$ 的情况。在这种情况下, $f(x) = x \sin(1/x)$ 是两个连续函数的乘积,其中 $x$ 是一个连续函数,而 $\sin(1/x)$ 也是连续的。因此,它们的乘积 $f(x)$ 在 $x \neq 0$ 的情况下是连续的。在 $x = 0$ 处,有
\begin{equation}
\lim_{{x \to 0}} f(x) = \lim_{{x \to 0}} x \sin\left(\frac{1}{x}\right) = 0~.
\end{equation}

因此,$f(x)$ 在 $[0,1]$ 上是连续的。

注意到分割 $\Gamma = \{x_n\}=\{\frac{1}{n\pi+\pi/2}\}$ 我们有,  
\begin{align} f(x_n)=x_n\sin(1/x_n)=\left\{\begin{matrix}x_n& \ \text{even}\\ -x_n& \ \text{odd}\end{matrix}\right.\qquad\text{for}\ n\geq0 ~.
\end{align}
带入计算变差的公式$2$,可得:
\begin{align}
\sum_{n=1}^m|f(x_n)-f(x_{n-1})|&=\sum_{n=1}^m|(-1)^n(x_n+x_{n-1})|\\
&= \sum_{n=1}^m(x_n+x_{n-1})\\&=x_m+x_0+2\sum_{n=1}^{m-1}x_n\\\geq \sum_{n=1}^{m-1} x_n&=\sum_{n=1}^{m-1} \frac{1}{n\pi+\pi/2}~.
\end{align}
根据极限比较审敛法对比调和级数可得:
\begin{align} 
\sum_{n=1}^{m-1}\frac{1}{n\pi+\pi/2}\to\infty\qquad\mathrm{as}\qquad m\to\infty~. 
\end{align}
\end{example}

\begin{exercise}{设 $f(x)=x\cos\left(\pi/x\right)$ 为 $0<x\leq1$ 且 $f(0)=0$。 证明 $f$ 在 $[0,1]$ 上有界且连续,但不是有界变差$V[f;0,1]=+\infty$。}
\end{exercise}

\begin{theorem}{}
\begin{enumerate}
\item 如果$f$在$[a,b]$上的变差有界,那么$f$在$[a,b]$上有界。
\item 设$f$和$g$在$[a,b]$上的变差有界。那么对于任意实常数$c$,$cf$,$f+g$和$fg$在$[a,b]$上的变差也有界。此外,如果存在$\varepsilon>0$使得对于$[a,b]$中的$x$,$|g(x)|\geq\varepsilon$,那么$f/g$在$[a,b]$上的变差也有界。
\end{enumerate}
\end{theorem}

\begin{example}{证明:}
\begin{enumerate}
\item 假设 $f$ 在 $[a,b]$ 上是有界变差的。对于任意 $x\in[a,b]$,考虑仅由三个点组成的分割 $\left\{a,x,b\right\}$,则:
\begin{align} \left|f(x)-f(a)\right|+\left|f(b)-f(x)\right|\ \leq\ V(f;a,b) \\ \Rightarrow\ \left|f(x)-f(a)\right|\ \leq\ V(f;a,b) \\ \left|f(x)\right|=\left|f(a)+f(x)-f(a)\right| \\ \leq\left|f(a)\right|+\left|f(x)-f(a)\right| \\ \leq\left|f(a)\right|+\left|f(f;a,b)\right|\ <\ \infty \\ \Rightarrow\ f\ \ \ \text{is bounded on}\ [a,b] ~. \end{align}
\item 设 $f$ 和 $g$ 是 $[a,b]$ 上的两个有界变差函数。对于 $[a,b]$ 的任意分割 $P$,我们有:\begin{align} \sum\bigl{|}(f+g)(x_{i})-(f+g)(x_{i-1})\bigr{|}&=\sum\bigl{|}\bigl{\{}\,f(x_{i})+g(x_{i})\bigr{\}}-\bigl{\{}f(x_{i-1})+g(x_{i-1})\bigr{\}}\bigr{|} \\ &=\sum\bigl{|}f(x_{i})-f(x_{i-1})+g(x_{i})-g(x_{i-1})\bigr{|} \\ &\leq\sum\bigl{|}f(x_{i})-f(x_{i-1})\bigr{|}+\sum\bigl{|}g(x_{i})-g(x_{i-1})\bigr{|} \\ &\leq V(f;a,b)+V(g;a,b) \\ \Rightarrow V(f+g;a,b)\ &\leq\ V(f;a,b)+V(g;a,b) ~.\end{align}

这表明函数 $f+g$ 是有界变差的。由于 $f+f+\cdots$ 是有界变差的,对于任何实常数 $c$,$cf$(即 $\sum^c f$)也是有界变差的。

\begin{align} &\sum\bigl{|}\bigl{(}f\!g\bigr{)}(x_{i})-\bigl{(}f\!g\bigr{)}(x_{i-1})\bigr{|} \\ &=\sum\bigl{|}f(x_{i})g(x_{i})-f(x_{i-1})g(x_{i-1})\bigr{|} \\ &=\sum\bigl{|}f(x_{i})g(x_{i})-f(x_{i})g(x_{i-1})+f(x_{i})g(x_{i-1})-f(x_{i-1})g(x_{i-1})\bigr{|} \\ &=\sum\bigl{|}f(x_{i})\bigl{\{}g(x_{i})-g(x_{i-1})\bigr{\}}+g(x_{i-1})\bigl{\{}f(x_{i})-f(x_{i-1})\bigr{\}}\bigr{|}\\&\leq\sum\left|f(x_{i})\right||g(x_{i})-g(x_{i-1})\left|+\sum\left|g(x_{i-1})\right|\right|f(x_{i})-f(x_{i-1})\left|\right. \\ &\leq k\sum\left|g(x_{i})-g(x_{i-1})\right|+k\sum\left|f(x_{i})-f(x_{i-1})\right| \\ &\leq k.V(g)+k\left.V(f)\right. ~.\end{align}
\end{enumerate}
\end{example}
\begin{theorem}{}
\begin{enumerate}
\item 如果$[a^{\prime},b^{\prime}]$是$[a,b]$的子区间,则$V[a^{\prime},b^{\prime}]\leq V[a,b]$;即,随着区间的增加,变差也增加。
\item 如果$a<c<b$,则$V[a,b]=V[a,c]+V[c,b]$;即,相邻区间上的变差是可加的。
\end{enumerate}
\end{theorem}
开始证明之前,注意到如果$\Gamma$是$\Gamma$的细分,即$\Gamma$包含了$\Gamma$的所有分割点以及一些额外的点,那么$S_{\Gamma}\leq S_{\bar{\Gamma}}$。这是在$\bar{\Gamma}$包含$\Gamma$的所有点加上一个额外点的情况下,由三角不等式得出的。

\begin{example}{证明:}

\end{example}

\begin{example}{如果 $[a^{\prime},b^{\prime}]$ 是 $[a,b]$ 的子区间,则证明 $P[a^{\prime},b^{\prime}]\leq P[a,b]$ 和 $N[a^{\prime},b^{\prime}]\leq N[a,b]$。}

\end{example}

\begin{theorem}{}
如果 $P$、$N$ 或 $V$ 中的任何一个有限,则三者皆有限。此外,我们有,

\begin{align}
 P\,+\,N\,=\,V,\qquad P\,-\,N\,=\,f(b)\,-\,f(a)~.
\end{align}

或者等价地,

\begin{align}
 P\,=\,\frac{1}{2}[V\,+\,f(b)\,-\,f(a)],\qquad N\,=\,\frac{1}{2}[V\,-\,f(b)\,+\,f(a)]~.
\end{align}
\end{theorem}

\begin{corollary}{约当Jordan定理}
如果一个函数$f$在闭区间$[a,b]$上具有有界变差,那么它可以被写成$[a,b]$上两个有界增函数之差。反之亦然。
\end{corollary}

这个定理指出了一个函数有界变差的一个等价条件,即函数可以表示为两个有界增函数之差。

\textbf{\textsl{证明}}: 假设 $f=f_{1},-f_{2}$,其中 $f_{1}$ 和 $f_{2}$ 在 $[a,b]$ 上有界且递增。那么 $f_{1}$ 和 $f_{2}$ 在 $[a,b]$ 上是有界变差的,因此根据定理2(ii),$f$ 也是如此。

反之,假设 $f$ 在 $[a,b]$ 上是有界变差的。根据定理3(i),$f$ 在每个区间 $[a,x]$,$a\leq x\leq b$ 上都是有界变差的。设 $P(x)$ 和 $N(x)$ 在 $[a,b]$ 上是有界且递增的。此外,对于应用于 $[a,x]$ 的定理 4,当 $a\leq x\leq b$ 时,$f(x)=[P(x)+f(a)]-N(x)$。由于 $P(x)$ 是有界且递增的,因此 $P(x)+f(a)$ 也是如此,推论随之而来。

如果上面证明对于你来说难懂,我建议你阅读下面简单巧妙的证明:

假设 $f\in V[a,b]$,并且令 $v(x)=V(f;a,x)$,$x\in(a,b]$,$v(a)=0$。显然,$f(x)=v(x)-[v(x)-f(x)]$。我们将证明 $v(x)$ 和 $v(x)-f(x)$ 是递增的。对于任意的 $x_{1}<x_{2}$,我们有


\begin{align}
 v(x_{2})-v(x_{1})=V(f;x_1,x_2)\geq0\Leftrightarrow v(x_{2})\geq v(x_{1}) ~.
\end{align}

因此 $v(x)$ 是递增函数。

\begin{align}
f(x_{2})-f(x_{1})\leq|f(x_{2})-f(x_{1})|\leq V(f;x_1,x_2) (f)=v(x_{2})-v(x_{1}) \\ \Leftrightarrow v(x_{1})-f(x_{1})\leq v(x_{2})-f(x_{2}) ~.
\end{align}
因此 $v(x)-f(x)$ 是递增函数。

需要注意的是,由于递增函数的负值是递减的,$f$ 是有界变差的 当且仅当 它是一个有界递增函数和一个有界递减函数的和。

\begin{example}{将 $f(x)=\cos^{2}x$,$0\leq x\leq2\pi$ 表示为两个递增函数之差。}

假设$f(x) = v(x) - [v(x)-f(x)]$, 那么问题在于确定 $v(x)$为递增函数。将 $[0,2\pi]$ 分成四个子区间 $I_{1}=[0,\frac{\pi}{2}]$,$I_{2}=[\frac{\pi}{2},\pi]$,$I_{3}=[\pi,\frac{3\pi}{2}]$ 和 $I_{4}=[\frac{3\pi}{2},2\pi]$。

函数 $f(x)$ 在 $I_{1}$ 和 $I_{3}$ 上从 $0$ 减小到 $1$,在 $I_{2}$ 和 $I_{4}$ 上从 $0$ 增加到 $1$,因此 $f$ 在任何这些子区间上的总变差都是 $1$。要确定 $V_{0}^{x}(f)$,我们需要研究 $x$ 在哪个区间内的不同情况。为了说明原理,假设 $x\in I_{3}$,那么,

\begin{align}
V_{0}^{x}(f)&=V_{0}^{\frac{\pi}{2}}(f)+V_{\frac{\pi}{2}}^{\pi}(f)+|f(x)-f(\pi)| \\ &=1+1+|\cos^{2}x-1|=3-\cos^{2}x ~.
\end{align}


\begin{align}
 v(x)=\left\{\begin{array}{ll}1-\cos^2x&0\leq x\leq\frac{\pi}{2}\\ 1+\cos^2x&\frac{\pi}{2}\leq x\leq\pi\\ 3-\cos^2x&\pi\leq x\leq\frac{3\pi}{2}\\ 3+\cos^2x&\frac{3\pi}{2}\leq x\leq2\pi\end{array}\right. \ ~. 
\end{align}


\end{example}