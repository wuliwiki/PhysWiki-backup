% 尼古拉·哥白尼
% license CCBYSA3
% type Wiki

(本文根据 CC-BY-SA 协议转载自原搜狗科学百科对英文维基百科的翻译)

尼古拉·哥白尼(/koʊˈpɜːrnɪkəs, kə-/;[1][2][3] 波兰语:Mikoaj Kopernik; 德语:Nikolaus Kopernikus;Niklas Koppernigk;1473年2月19日-1543年5月24日)是文艺复兴时期的数学家和天文学家,他构想出一个宇宙模型,将太阳而不是地球置于宇宙的中心,很可能独立于萨摩斯的阿里斯塔克斯,他在大约十八世纪前构想出了这样一个模型。[4]

在1543年他逝世之前,哥白尼模型才在他的书De revolutionibus orbium coelestium (《天体运行论》)中发布,这是科学史上的一件大事,引发了哥白尼革命,为科学革命做出了开创性的贡献。[5]

哥白尼在皇家普鲁士出生和逝世,该地区自1466年以来一直是波兰王国的一部分。他精通多种语言,博学多才,获得了教会法博士学位,同时也是数学家、天文学家、医生、古典学者、翻译家、州长、外交官和经济学家。1517年,他推导出了货币数量理论——经济学中的一个关键概念——1519年,他提出了一个后来被称为格雷欣定律的经济原则。

\subsection{生活}
尼古拉·哥白尼于1473年2月19日出生在波兰王国皇家普鲁士省Toruń(托伦)市。[6]他的父亲是克拉科夫的商人,母亲是一个富有的托伦商人的女儿。[7]尼古拉是四个孩子中最小的。他的兄弟Andreas(安德鲁)在 Frombork(弗龙堡)成为了奥古斯丁教徒。[7]他以母亲的名字命名的妹妹芭芭拉,成为了本笃会修女,在生命的最后几年,成为了Chełmno(库尔姆)修道院的院长;她在1517年后去世。[7]他的妹妹卡塔琳娜嫁给了商人和托伦市议会议员巴特尔·格特纳,留下了五个孩子,哥白尼去世前一直照顾他们。[7]哥白尼从未结过婚,也不知道有没有孩子,但至少从1531年到1539年,他与寄宿管家安娜·席林的关系被瓦尔米亚的两位主教视为丑闻,这两位主教多年来一直规劝他与“情妇”断绝关系。[8]

\subsubsection{1.1 父亲的家庭}
\begin{figure}[ht]
\centering
\includegraphics[width=6cm]{./figures/bb8de03d7ae32025.png}
\caption{托伦出生地(ul。Kopernika 15,左边的)。连同17号的房子(正确),它形成了muze um Mikoaja Kopernika。} \label{fig_GBN_1}
\end{figure}

哥白尼的父亲的家庭可以追溯到奈萨(Neiße)附近西里西亚的一个村庄。这个村庄的名字有不同的拼写,有Kopernik, Copernik, Copernic, Kopernic, Coprirnik,今天被拼写为Koperniki。[9]14世纪,他的家庭成员开始移居到其他西里西亚城市,先是到了波兰首都克拉科夫(1367年),后又移居到托伦(1400年)。[9]哥白尼的父亲-年长的尼古拉来自克拉科夫,很可能是简的儿子。[9]

尼古拉是以他父亲的名字命名的,他父亲第一次作为一个富裕的经营铜的商人出现在记录中,他主要在但泽(Gdańsk)出售铜。[10][11]1458年左右,他从克拉科夫搬到托伦。[12]托伦位于维斯瓦河畔,当时卷入了十三年战争,在这场战争中,波兰王国和普鲁士联邦(一个普鲁士城市、贵族和神职人员的联盟)结盟,与条顿骑士团争夺对该地区的控制权。在这场战争中,像但泽,托伦以及尼古拉·哥白尼的家乡这样的汉萨同盟选择支持波兰国王卡西米尔四世·贾吉隆(Casimir IV Jagiellon),他承诺尊重城市传统的巨大独立,而这正是条顿骑士团所挑战的。尼古拉的父亲积极参与当时的政治活动,支持波兰和城市反对条顿骑士团。[13]1454年,他调解了波兰红衣主教兹比格涅夫·奥列尼基和普鲁士城市之间偿还战争贷款的谈判。[9]在托伦第二次和约时期(1466年),条顿骑士团正式放弃了对其西部省份的所有权利主张,其中,直到波兰第一次(1772年)和第二次(1793年)分区之前,皇家普鲁士仍然是波兰王国的直辖地区。

哥白尼的父亲在1461年至1464年间与天文学家的母亲芭芭拉·瓦岑罗德结婚。[9]他大约死于1483年。[7]

\subsubsection{1.2 母亲的家庭}
\begin{figure}[ht]
\centering
\includegraphics[width=6cm]{./figures/8ce494e56395ed0f.png}
\caption{哥白尼的舅舅,卢卡斯·沃特森罗德二世} \label{fig_GBN_2}
\end{figure}
尼古拉的母亲芭芭拉·瓦岑罗德是一位富有的托伦贵族卢卡斯·瓦岑罗德(1462年已故)和市议员卡塔尔津纳(简·佩考的遗孀)的女儿,市议员在其他来源中被称为为Katarzyna Rüdiger gente Modlibóg(1476年已故)。[7]Modlibógs(莫德里堡家族)是波兰的一个显赫家族,自1271年以来在波兰历史上便广为人知。[14]Watzenrode(瓦岑罗德家族)和哥白尼一家一样,来自西里西亚,靠近Świdnica(施韦迪尼察),1360年后定居托伦。[7]他们很快成为最富有和最有影响力的贵族家庭之一。通过瓦岑罗德斯广泛的婚姻家庭关系,哥白尼与Toruń (托伦), Gdańsk (但泽) 和 Elbląg(埃尔宾)的富裕家庭,以及普鲁士著名的波兰贵族家庭:恰普斯基、泽扬斯基、科诺帕基斯和科西莱克基斯建立了联系。[7]卢卡斯和凯瑟琳有三个孩子:小卢卡斯·瓦岑罗德(1447-1512),他后来成为瓦尔米亚的主教和哥白尼的资助人;芭芭拉是这位天文学家的母亲(1495年后去世);克里斯蒂娜(去世于1502年之前),于1459年与托伦商人兼市长蒂德曼·冯·艾伦结婚。[7]

老卢卡斯·瓦岑罗德是一位富有的商人,1439年至1462年间担任司法法庭庭长,他是条顿骑士团的坚决反对者。[7]1453年,他是托伦的代表团成员,在Grudziądz (格鲁琼兹)会议上策划起义反对他们。[7]在随后的十三年战争中(1454-1466),他通过提供大量的货币补贴来积极支持普鲁士城市的战争努力(他后来只要回了其中的一部分),并在托伦和但泽进行政治活动,亲自在Łasin(塞辛)和 Malbork (马尔堡)参加战斗。[7]他去世于1462年。[7]

小卢卡斯·瓦岑罗德是这位天文学家的舅舅和资助人,他在克拉科夫大学(现为贾吉洛尼亚大学)以及科隆大学和博洛尼亚大学接受教育。他是条顿骑士团的死对头,条顿骑士团的大师曾称他为“魔鬼的化身”。1489年,瓦岑罗德被选为瓦尔米亚(Ermeland, Ermland)的主教,反对国王卡西米尔四世偏爱自己的儿子。国王曾希望自己的儿子担任这一席位。[15]结果,瓦岑罗德与国王争吵不休,直到卡西米尔四世在三年后去世。[16]随后瓦岑罗德与三位相继的波兰君主建立了密切的关系:约翰·艾伯特、亚历山大·贾吉隆和西吉斯蒙德一世。他是每一位统治者的朋友和关键顾问,他的影响极大地加强了瓦尔米亚和波兰本土之间的联系。[17]瓦岑罗德被认为是瓦尔米亚最有权力的人,他的财富、人脉和影响力为哥白尼的教育以及他在弗龙堡大教堂的教士生涯提供了保障。[15]

\subsubsection{1.3 语言}
\begin{figure}[ht]
\centering
\includegraphics[width=6cm]{./figures/1ba6d388047a2af0.png}
\caption{哥白尼的德文信到普鲁士公爵阿尔伯特,为…提供医疗建议乔治·冯·昆海姆(1541)} \label{fig_GBN_3}
\end{figure}
据说,哥白尼能流利地说拉丁语、德语和波兰语;他还会说希腊语和意大利语,并懂一些希伯来语。哥白尼现存的大部分著作是用拉丁语写成的,拉丁语是他在欧洲学术界运用一生的语言。

关于德语是哥白尼的母语的论点是,他出生在一个以讲德语为主的城市,1496年在博洛尼亚学习教会法时,他加入了日耳曼民族(Natio Germanorum)——一个学生组织,根据其1497年的章程,该组织向所有母语为德语的王国和州的学生开放。[18]然而,根据法国哲学家亚历山大·柯瓦雷(Alexandre Koyré)的说法,哥白尼加入日耳曼民族这一行为本身并不意味着哥白尼认为自己是德国人,因为来自普鲁士和西里西亚的学生通常都是这样分类的,这种分类带有某些特权,使得它成为说德语的学生的自然选择,而不管他们的种族或自我认同如何。[18][19]

\subsubsection{1.4 名字}
Kopernik,Copernik,Koppernigk这些姓氏是从1350年开始在克拉科夫以各种拼法记录的,显然是给奈萨公国的Kopernik村的人的(1845年以前是Kopernik,Copernik,Copernik和Koppirnik)。尼古拉·哥白尼的曾祖父被记录为于1386年在克拉科夫获得公民身份。地名Kopernik(现代Koperniki)与波兰语中的dill (koper)和德语中的copper (Kupfer)有不同的联系。虽然后缀-nik(或复数-niki)表示斯拉夫语和波兰语的代理名词。

这一时期常见的是,地名和姓氏的拼写差别很大。哥白尼“对正字法相当漠不关心”。[20]在他的童年时期,大约1480年,他父亲的名字(也就是未来天文学家的名字)在托伦被记录为Niclas Koppernigk。[21]在克拉科夫,他拉丁文的署名为Nicolaus Nicolai de Torunia(尼古拉,托伦尼古拉的儿子)。1496年在博洛尼亚,他在日耳曼Matricula Nobilissimi 大学注册,也就是Natio Germanica Bononiae 的Annales Clarissimae Nacionis Germanorum,相当于Dominus Nicolaus Kopperlingk de Thorn – IX grosseti。[22]在帕多瓦,他的署名变成“Nicolaus Copernik”,随后又改为“Coppernicus”。[20]这位天文学家因此把他的名字拉丁化为Coppernicus,通常用两个“p”(在研究的31个文献中有23个记载如此),[23]但后来在生活中他用了一个“p”。在《天体运行论》的标题页上,雷蒂库斯把这个名字(以名词属格或以所有格的形式)公布为“尼古拉·哥白尼”。

\subsubsection{1.5 教育}
\textbf{在波兰}

父亲去世后,小尼古拉的舅舅小卢卡斯·瓦岑罗德(1447-1512年)带着他长大,让他接受教育,发展事业。[7]瓦岑罗德与波兰著名的知识分子保持着联系,他也是意大利著名的人文主义者和克拉科夫朝臣Filippo Buonaccorsi的朋友。[24]哥白尼童年和教育的早期的原始文件未能得到保存。[7]哥白尼传记作者推测瓦岑罗德首先把年轻的哥白尼送到托伦的圣约翰学校,他本人也曾在这里任教。[7]后来,根据阿米蒂奇的说法,这个男孩在托伦维斯瓦河上游的波兰弗沃茨大教堂学校上学,这里的学生都为考入克拉科夫大学做准备,该所大学也是瓦岑罗德在波兰首都的母校。[25]
\begin{figure}[ht]
\centering
\includegraphics[width=10cm]{./figures/420309e082a8c780.png}
\caption{马尤斯学院在克拉科夫大学哥白尼的波兰文母校} \label{fig_GBN_4}
\end{figure}
在1491-1492年冬季学期,哥白尼以“Nicolaus Nicolai de Thuronia”的名字和他的兄弟安德鲁一起被克拉科夫大学(现在的贾吉洛尼亚大学)录取。[7]哥白尼在克拉科夫天文数学学院的全盛时期开始了他在艺术系的研究(从1491年秋季开始,大概到1495年夏季或秋季),为他后来的数学成就奠定了基础。[7]根据后来一个可靠的说法(Jan Brożek),哥白尼是阿尔伯特·布鲁日斯基的学生,后者当时(从1491年起)是亚里士多德的哲学教授,但在大学之外私下教授天文学;哥白尼开始熟悉布鲁日斯基对乔治·冯·费尔巴哈的《行星理论》的广泛阅读的评论,几乎可以肯定地说,他参加了毕斯库派的伯纳德和萨莫图伊的沃伊切赫·克里帕的讲座,也可能参加了约格沃的扬、沃罗乔的米夏(布雷斯洛)、普涅维的沃伊切赫和奥尔库斯的马尔钦·比利卡的其他天文学讲座。[26]

哥白尼在克拉科夫的研究使他在大学教授的数学天文学(算术、几何、几何光学、宇宙学、理论和计算天文学)中奠定了充分的基础,并对亚里士多德的哲学和自然科学著作(De coelo, 《形而上学》)和阿威洛依(它在将来塑造哥白尼的理论中扮演着重要角色)有了很好的了解,激发了他的学习兴趣,并使他熟悉了人文文化。[15]哥白尼通过独立阅读他在克拉科夫求学时期获得的书籍(欧几里德、哈利·阿本拉格尔、Alfonsine表、约翰尼斯·雷乔蒙塔努斯的Tabulae directionum)拓宽了他从大学讲堂获得的知识;或许,他最早的科学笔记也可以追溯到这一时期,现在部分保存在乌普萨拉大学(Uppsala University)。[15]哥白尼在克拉科夫开始收集大量天文学方面的藏书;这些书后来在16世纪50年代的大洪水期间被瑞典人作为战利品带走,现在存放在乌普萨拉大学图书馆。[27]

哥白尼在克拉科夫的四年对他批判能力的发展中起到了重要作用,他也在这期间开始了对天文学两个“官方”体系中逻辑矛盾的分析——亚里士多德的同心球理论和托勒密偏心圆和本轮理论——对这两个体系的超越和抛弃将是哥白尼自己创立宇宙结构学说的第一步。[15]

大概是在1495年秋天,哥白尼离开克拉科夫去了舅舅瓦岑罗德的教堂,他没有获得学位,瓦岑罗德在1489年被提升为瓦尔米亚的主教王子,不久之后(1495年11月之前),他想让外甥哥白尼接替瓦尔米亚教士职位,前任教士Jan Czanow于1495年8月25日去世而空出这一职位。出于不清楚的原因——可能是因为教会一部分人的反对,他们将此事上诉到罗马——哥白尼的安置被推迟,瓦岑罗德倾向于派他的两个外甥去意大利学习教会法,似乎是为了促进他们的教会事业,从而也加强了他自己在瓦尔米亚教会的影响力。[15]
\begin{figure}[ht]
\centering
\includegraphics[width=6cm]{./figures/8f5348b611e4c308.png}
\caption{圣十字架和圣巴塞洛缪学院教堂在wrocaw} \label{fig_GBN_5}
\end{figure}
哥白尼于1496年年中离开瓦尔米亚——可能是随该教会会长Jerzy Pranghe的随员一同前往意大利——在秋天,可能是10月,哥白尼抵达博洛尼亚,几个月后(1497年1月6日之后),他加入了博洛尼亚大学法律学生的日耳曼民族组织,其中包括来自西里西亚、普鲁士和波美拉尼亚的年轻波兰人以及其他国家的学生。[15]

\textbf{在意大利}

1497年10月20日,哥白尼通过代理人正式继承了两年前授予他的瓦尔米亚教士职位。除此之外,根据1503年1月10日在帕多瓦的一份文件,他将在沃罗斯瓦夫(当时在波希米亚王国)的圣十字学院教堂和圣巴塞洛缪担任一个闲职。尽管哥白尼在1508年11月29日被教皇特许授予以获得更多的圣俸,但是在他的教会生涯中,他不仅没有在教会获得更多的俸禄和更高的地位,而且还在1538年放弃了沃罗斯瓦夫的神职。尚不清楚他是否曾被任命为牧师。[28] 爱德华·罗森声称他不是。[29][30]哥白尼确实接受了一些小任命,这也足以让他胜任教会教士职位。[15] 天主教百科全书认为他很有可能被任命为圣职,因为在1537年,他是瓦尔米亚需要任命的主教席位的四个候选人之一。[31]

哥白尼在1496年秋至1501年春在博洛尼亚待了三年,他似乎不太热衷于研究教会法(他在1503年第二次回到意大利,紧接着他获得了等了七年的的法学博士学位),而更热衷于研究人文学科——可能参加了菲利普·贝罗拉多(Filippo Beroaldo)、安东尼奥·乌尔西奥(Antonio Urceo,又名Codro)、乔瓦尼·加尔松尼(Giovanni Garzoni)和亚历山德罗·阿奇里尼(Alessandro Achillini)的讲座——并研究天文学。他遇到了著名天文学家多梅尼科·玛丽亚·诺瓦拉·达费拉拉( Domenico Maria Novara da Ferrara),成为他的弟子和助手。[15]哥白尼通过阅读乔治·冯·费尔巴哈和约翰尼斯·雷乔蒙塔努斯(威尼斯,1496)的《天文学大成的缩影》(托勒密天文学大成的缩影)发展了新的思想。他通过1497年3月九号在博洛尼亚对金牛座中最亮的恒星毕宿五被月亮遮挡的观测,验证了托勒密月球运动理论中某些特性。人文主义者哥白尼通过仔细阅读希腊和拉丁作家的著述(毕达哥拉斯、萨摩斯的阿里斯塔克斯、克莱门德斯、西塞罗、老普林尼、普卢塔克、菲洛劳斯、赫拉克里德斯、埃克芬托斯、柏拉图),尤其在帕多瓦时期,收集关于古代天文学、宇宙哲学和历法系统的零星历史信息,为他日益增长的疑惑寻求证实。[32]
\begin{figure}[ht]
\centering
\includegraphics[width=6cm]{./figures/0bcff94e19a10aad.png}
\caption{加列拉65号,博洛尼亚,梅尼科·玛丽亚·诺瓦拉的住所。} \label{fig_GBN_6}
\end{figure}
哥白尼在罗马渡过了大赦年1500年,那年春天他和他的兄弟安德鲁来到罗马,无疑是为了在罗马教皇法院当学徒。然而,在这里,他也继续着始于博洛尼亚的天文工作,例如,观察了1500年11月5日至6日晚上的月食。根据雷蒂库斯后来的叙述,哥白尼可能在私下里是以天文学教授的身份向“许多学生和科学界的大师”公开讲授对于当代天文学的数学解决方案的评判,当然这并非是在罗马的萨皮恩扎进行的。[33]

1501年年中,哥白尼在返回瓦尔米亚的途中无疑在博洛尼亚有过短暂停留。在7月28日从教会获得为期两年的学习医学的假期后(因为“他将来可能是有用的医学顾问或者教会的高级教士”),在夏末或秋季,他再次回到意大利,可能由他的兄弟安德鲁和Bernhard Sculteti教士陪同。这一次,他在帕多瓦大学学习,那里以医学学习而闻名,除了在1503年5月至6月短暂访问了费拉拉以通过考试并获得教会法博士学位之外,他从1501年秋季到1503年夏季一直留在帕多瓦。[33]

哥白尼很可能在帕多瓦大学的顶尖教授——巴托洛米奥·达·蒙塔加纳纳、吉罗拉摩·法兰卡斯特罗、加布里埃尔·泽比、亚历山德罗·贝内代蒂——的指导下学习医学,并阅读他当时获得的医学论文,这些论文由瓦莱斯库斯·德·塔兰塔、扬·梅苏、雨果·塞恩西斯、扬·凯瑟姆、阿诺德·德·维拉诺瓦和米歇尔·萨沃纳罗拉撰写,这些论文将形成他后来医学藏书的雏形。[33]

哥白尼必须研究的科目之一是占星术,因为它被认为是医学教育的重要组成部分。[34]然而,与文艺复兴时期的大多数著名天文学家不同,他似乎从未实践过占星术,也从未表达过对占星术的兴趣。[35]

如同在博洛尼亚一样,哥白尼并不局限于他的官方研究。很可能是在帕多瓦的时候,他对希腊文化产生了兴趣。他借助希多罗斯·加沙的语法(1495年)和杰·布·克里斯托纽斯的字典(1499年)熟悉希腊语言和文化,他开始拓展对古典著作的研究,先是从博洛尼亚开始,研究巴西利乌斯·贝萨里昂、洛伦佐·瓦拉等人的著作。似乎也有证据表明,正是在他在帕多瓦逗留期间,一个基于地球运动的世界新体系的想法最终具体化了。[33]随着哥白尼回家的时间临近,1503年春天,他前往费拉拉,1503年5月31日,他在那里通过了规定的考试,被授予教会法博士学位(Nicolaus Copernich de Prusia, Jure Canonico ... et doctoratus)。[36]毫无疑问,就在不久之后(最迟在1503年秋天),他离开意大利回到瓦尔米亚。[33]
\begin{figure}[ht]
\centering
\includegraphics[width=10cm]{./figures/13504257486ffa5c.png}
\caption{“梅尼科·玛丽亚·诺瓦拉的住所就在这儿,他是过去博洛尼亚研究院的教授,波兰数学家兼天文学家尼古拉·哥白尼(NICOLAUS COPERNICUS)在1497-1500年与他的老师进行了杰出的天体观测。于哥白尼诞辰的第五百年放置于这座城市的博洛尼亚大学,博洛尼亚研究院科学院的波兰科学院中。1473 [–] 1973”} \label{fig_GBN_7}
\end{figure}

\subsubsection{1.6 行星观测}
哥白尼对水星进行了三次观测,误差为-3、-15和-1弧分。他对火星进行了一次观测,误差为-24弧分。这四个数据是通过观测火星得到的,误差分别为2弧分、20弧分、77弧分和137弧分。对木星进行了四次观测,误差分别为32、51、-11和25弧分。他对木星进行了四次观测,误差分别为31弧分、20弧分、23弧分和-4弧分。[37]

\subsubsection{1.7 工作}
\begin{figure}[ht]
\centering
\includegraphics[width=10cm]{./figures/44fd8ee43b4d6c25.png}
\caption{天文学家哥白尼,或与上帝的对话,1873年,作者马泰伊科。背景:Frombork大教堂。} \label{fig_GBN_8}
\end{figure}
30岁的哥白尼完成了他在意大利的所有学业后,回到了瓦尔米亚,在那里他将度过余生的40年,除了到克拉科夫和附近的普鲁士城市做过短暂的旅行:Toruń(托伦),Gdańsk(但泽), Elbląg (埃尔宾), Grudziądz (格鲁琼兹), Malbork (马尔堡), Königsberg (克鲁维茨)。[33]

瓦尔米亚的主教辖区享有高度自治,拥有自己的国会(议会)和货币单位(与皇家普鲁士的其他地方一样)以及国库。[38]

哥白尼从1503年到1510年(或者直到他叔叔在1512年3月29日去世)是他叔叔的秘书和医生,住在利兹巴克(海尔斯堡)的主教城堡里,在那里他开始研究他的日心说。他以官方身份参与了他叔叔几乎所有的政治、教会和行政经济职责。从1504年初开始,哥白尼陪同瓦岑罗德参加了在马尔堡和埃尔宾举行的普鲁士皇家国会会议。根据Dobrzycki 和Hajdukiewicz的记载,他“参与了...复杂外交活动的所有重要活动,这些是有野心的政客和政治家在为了有敌意的条顿骑士团和忠于波兰王室之间,维护普鲁士和瓦尔米亚的特殊权利而做的努力”。[33]
\begin{figure}[ht]
\centering
\includegraphics[width=6cm]{./figures/c833c12f0d3cf7d2.png}
\caption{哥白尼对Theophylact Simocattas书信。封面秀盾形纹章关于(从顶部顺时针方向) 波兰立陶宛和克拉科夫。} \label{fig_GBN_11}
\end{figure}
1504-12年间,哥白尼作为他叔叔的随从进行了多次旅行——1504年,去托伦和格但斯克,在波兰国王亚历山大·贾吉隆的见证下参加普鲁士皇家委员会的一次会议;参加在马尔堡(1506年)、埃尔宾 (1507年)和什图姆(1512年)举行的普鲁士国会会议;他可能参加了波兹南会议(1510年)和波兰国王西吉斯蒙德一世在克拉科夫的加冕典礼(1507年)。瓦岑罗德的行程表明哥白尼可能在1509年春天参加了克拉科夫议会。[33]

也许是在后来克拉科夫的活动中,哥白尼将他的一部7世纪拜占庭历史学家西奥菲拉特·西蒙卡特(Theophylact Simocatta)的诗集的希腊文翻译成拉丁文,提交给简·哈雷出版社印刷,这本诗集有85首被称为书信或者信的短诗,据说是在希腊故事中的不同人物之间传递的。它由三部分内容组成——“道德篇”,就人们应该如何生活提供建议;“田园篇”,讲的是牧羊人的生活片段;“爱情篇”,包括很多爱情诗在里面。在每一个话题中这三部分内容都会交替出现。哥白尼把希腊的诗句翻译成了拉丁散文,现在他出版了他的版本Theophilacti scolastici Simocati epistolae morales, rurales et amatoriae interpretatione latina,并将这本书献给他的叔叔,感谢他从他那里得到的一切好处。在这本译作中,在希腊文学是否应该复兴的问题上,哥白尼宣称自己站在人文主义者一边。[39]哥白尼的第一部诗歌作品是一首希腊短诗,很可能是在访问克拉科夫时为约翰内斯·丹蒂斯丘斯就芭芭拉·萨波利亚于1512年与波兰国王老齐格蒙特一世的婚礼而作。[40]

1514年前的某个时候,哥白尼写下了他的日心说的最初大纲,这个大纲只有后来的抄本才有,书名(可能是从抄本上得来的)是Nicolai Copernici de hypothesibus motuum coelestium a se constitutis commentariolus——通常被称为《短论》。这是对世界的日心说机理的简明理论描述,没有数学工具计算,在几何构造的一些重要细节上不同于《天体运行论》;但是它已经是基于关于地球三重运动的相同假设。哥白尼有意识地将《短论》作为他计划出版的书的第一稿,并不打算印刷发行。他只向他最亲密的熟人提供了很少的手稿副本,其中似乎包括几个克拉科夫天文学家,他们曾于1515-30年间一起合作观察日食。第谷·布拉赫将在他自己的著作《新编天文学初阶》中加入《短论》的片段,该论文于1602年在布拉格出版,其依据是他从波希米亚物理学家和天文学家塔德·瓦什·哈耶克那里收到的手稿,后者是雷蒂库斯的朋友。《短论》直到1878年才首次完整出版。[40]
\begin{figure}[ht]
\centering
\includegraphics[width=6cm]{./figures/9138a1f7965db02a.png}
\caption{哥白尼塔在Frombork,他居住和工作的地方;二战后重建} \label{fig_GBN_12}
\end{figure}
1510年或1512年,哥白尼搬到了弗龙堡,这是波罗的海沿岸维斯瓦泻湖西北的一个城镇。1512年4月,他在那里作为瓦尔米业的主教王子参加了洛沙伊宁的法比安的竞选。直到1512年6月初,教会才给哥白尼一个“外部教廷”——大教堂山防御墙外的一所房子。1514年,他购买了弗龙堡要塞城墙内西北处的一座塔。尽管1520年1月条顿骑士团对弗龙堡的突袭摧毁了教会的建筑,哥白尼的天文仪器也可能在突袭中被摧毁,但他会将这两处住所维持到生命的最后。哥白尼在1513年至1516年进行了天文观测,大概是从他的外部教廷进行的;1522-1543年,在一座无明“小塔”(turricula)上,他使用了模仿古代仪器的原始仪器——四分仪、三棱镜、浑天仪。哥白尼在弗龙堡进行了一半以上的他的60多次有记载的天文观测。[40]

哥白尼在弗龙堡永久定居下来,在那里他一直居住到去世,这期间只有在1516-19年和1520-21年被打断,他在瓦尔米亚教会为自己建立了经济和行政中心,这也是瓦尔米亚政治生活的两个主要中心之一。在瓦尔米亚困难的、政治复杂的形势下,对外受到条顿骑士团侵略(条顿军团的攻击;1519-21年的波兰-条顿战争;艾伯特吞并瓦尔米亚的计划),内部受制于强大的分离主义者的压力(瓦尔米亚主教王子的选择;货币改革),他与部分教会的成员提出了一个与波兰王室严格合作的方案,并在其所有的公共活动中表明(保卫他的国家反对骑士团的征服计划;将货币体系与波兰王室统一的提议;支持波兰在瓦尔米亚统治的教会管理中的利益)他是波兰-立陶宛共和国的公民。瓦岑罗德主教舅舅去世后不久,他参加了第二次皮奥特劳·特里布斯基条约(1512年12月7日)的签署,该条约掌控着瓦尔米亚主教的任命,尽管一部分教会成员反对,但他宣布与波兰国王进行忠诚的合作。[40]
