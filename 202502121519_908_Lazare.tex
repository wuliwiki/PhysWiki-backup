% 拉扎尔·卡诺(综述)
% license CCBYSA3
% type Wiki

本文根据 CC-BY-SA 协议转载翻译自维基百科\href{https://en.wikipedia.org/wiki/Lazare_Carnot#References}{相关文章}。

拉扎尔·尼古拉·玛尔格里特·卡诺(法语:[lazaʁ nikɔla maʁɡəʁit kaʁno];1753年5月13日 – 1823年8月2日)是法国数学家、物理学家、军官、政治家,并在法国大革命期间成为公共安全委员会的主要成员之一。他的军事改革,包括实施全民征兵(levée en masse),在将法国革命军转变为一支有效的战斗力量方面发挥了重要作用。

卡诺于1792年当选为国民公会成员,次年成为公共安全委员会的成员,并在第一次反法同盟战争期间,作为战争部长之一,领导了法国的战争努力。他监督了军队的重组,实施了严格的纪律,并通过强制征兵大幅扩展了法国军队。卡诺被誉为1793年至1794年法国军事复兴的功臣,因而被称为“胜利的组织者”。

随着对蒙塔尼亚派激进政治的日益失望,卡诺与马克西米连·罗伯斯比尔决裂,并在1794年热月9日的政变中参与了后者的推翻及随后的处决。卡诺成为了最初的五名执政府成员之一,但在1797年9月18日的热月政变后被罢黜,随后流亡国外。

在拿破仑崛起之后,卡诺返回法国,并于1800年短暂担任战争部长。作为一名坚定的共和主义者,他在拿破仑加冕为皇帝后选择退出了公共生活。1812年,他重返职场,在拿破仑领导下负责安特卫普的防御,抵御第六次反法同盟的进攻;在百日复辟期间,他担任了拿破仑的内政部长。第二次波旁复辟后,卡诺被流放,并于1823年在普鲁士的马格德堡去世。

除了政治生涯,卡诺还是一位杰出的数学家。他于1803年出版的《位置几何学》被认为是投影几何领域的开创性著作。他还因发明了卡诺墙而闻名,这是一种防御工事系统,在19世纪广泛应用于欧洲大陆。