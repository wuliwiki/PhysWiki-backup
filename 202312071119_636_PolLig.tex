% 偏振光
% license Xiao
% type Tutor

\begin{issues}
\issueDraft
\end{issues}

\pentry{平面简谐波\upref{PWave}}

\subsection{光矢量}
大量事实表明,光波对物质的电场作用远大于磁场作用。所以讨论光的时候,通常只讨论 $\vec{E}$,将其定义为光矢。

\subsection{偏振光}

光是一种电磁波,电磁波是一种横波。所谓横波,即波的传播方向与振动方向垂直的波,例如人在奔跑时,挂在脖子上的围巾的运动就可以类比为一种横波。偏振,顾名思义,是偏好某个方向振动。振动方向对于传播的方向不对称就叫做偏振。在这里,偏振光可以理解为沿着确定方向做振动的波。特别的,只有横波才有偏振,纵波则是不存在偏振的。

根据光的偏振方向的不同,有qiang'gui'lü'd可以分为线偏振、圆偏振和椭圆偏振。线偏振,顾名思义为偏振方向呈直线;圆偏振,依据偏振的旋转方向可分为左旋圆偏振与右旋圆偏振,或简称为左(右)旋圆偏;椭圆偏振可分为左(右)旋椭偏。

\begin{itemize}
\item 线偏振光是指光矢量的振动方向总在同一确定的平面内、而振动的方向在传播过程中为一条直线,故称之为线偏振光,又称平面偏振光。
\item 圆偏振光指光矢量的末端的轨迹在垂直于传播方向的平面上呈圆形。即光矢量不断旋转,其大小不变,但方向随时间有规律地变化。
\item 椭圆偏振光类似,指光矢量的末端的轨迹在垂直于传播方向的平面上呈椭圆形。即光矢量不断旋转,其大小、方向都随时间有规律的变化。
\end{itemize}


下面是几种偏振光的形态。


\subsection{偏振片}
偏振片的工作原理是在光的偏振方向上对光的选择性吸收。理想情况下,线性偏振片通过与透光轴平行的光场分量,吸收与透光轴垂直的光场分量。

