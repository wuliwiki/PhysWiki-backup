% 可微映射的导数
% keys 导数|可微映射
% license Xiao
% type Tutor

\pentry{常微分方程的几何图像\upref{GofODE}}
可微映射(\autoref{def_GofODE_2}~\upref{GofODE})$f:U\rightarrow V$ 将 $\mathbb R^n$ 空间的区域 $U$ 映射到 $\mathbb R^m$ 空间区域 $V$,于是就将 $U$ 上的曲线(\autoref{sub_GofODE_1}~\upref{GofODE})$\varphi$ 映射到 $V$ 上的曲线 $\phi$,可微性意味着这一对应是一一的。而切向量是曲线的等价类(\autoref{def_GofODE_3}~\upref{GofODE}),于是曲线 $\varphi,\phi$ 各自对应一切向量 $\dv{\varphi}{t},\dv{\phi}{t}$。这就是说在可微映射 $f$ 作用下,$U$中的切向量 $\dv{\varphi}{t}$ 和 $V$ 中的切向量 $\dv{\phi}{t}$ 对应,这一对应是一一的,因为若 $\dv{\varphi_1}{t}=\dv{\varphi_2}{t}$,则
\begin{equation}
\
\end{equation}

。 描述由可微映射 $f$ 导致的 $U$ 中的切向量和 $V$ 中的切向量的这一对应关系的映射称为 $f$ 的导数。

