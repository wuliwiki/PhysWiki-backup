% 极小多项式(线性代数)
% keys 最小多项式|minimal polynomial|环|线性代数|域
% license Xiao
% type Tutor


\pentry{线性映射\upref{LinMap}}


极小多项式又称最小多项式,描述了一个元素关于给定环的代数性质。本节是线性代数的一部分,因此我们只讨论线性变换的极小多项式。


\subsection{零化多项式}





\begin{definition}{线性变换的多项式}
给定域$\mathbb{F}$上的线性空间$V$,令$A$是$V$上的\textbf{线性变换}。

定义$A^k$是$A$与自身复合$k$次的结果,即$A^k=\overbrace{A\circ A\circ\cdots\circ A}^{k\text{个}A}$,则可以定义线性变换的多项式:若$f$是$\mathbb{F}$上的多项式,表达为
\begin{equation}
f(x) = \sum_{i=0}^m a_ix^i~, 
\end{equation}
那么$f(A)$就是$V$上的线性变换,表达为
\begin{equation}
f(A) = \sum_{i=0}^m a_ixA^i~. 
\end{equation}
\end{definition}



\begin{definition}{零化多项式}\label{def_MinPol_1}
给定域$\mathbb{F}$上的线性空间$V$,令$A$是$V$上的\textbf{线性变换}。

对于$V$的子空间$W$,若$\mathbb{F}$上的多项式$f$满足$f(A)\bvec{w}=\bvec{0}$对任意$\bvec{w}\in W$成立,则称$f$是$A$在$W$上的\textbf{零化多项式(null polynomial)}。
\end{definition}


由\autoref{def_MinPol_1} 显然可知,$f$是$A$在$V$上的零化多项式,当且仅当$f(A)=0$。


对于任意线性变换,其零化多项式一定存在,如以下定理所说:

\begin{theorem}{Hamilton-Cayley定理}
给定线性空间$V$上的线性变换$A$,$f(\lambda)$为其特征多项式,则$f(A)=0$。
\end{theorem}

\textbf{证明}:

任取$V$的一组基,将线性变换$A$表示为矩阵$\bvec{A}$,恒等变换$I$的矩阵则必为$\bvec{I}$。

令
\begin{equation}
f(\lambda) = \sum_{i=0}^n a_i\lambda^i~, 
\end{equation}
再定义$\lambda\bvec{I}-\bvec{A}$的伴随矩阵为
\begin{equation}
\bvec{B}(\lambda)=\sum_{i=0}^n B_i\lambda^i~, 
\end{equation}
其中各$B_i$是常数矩阵。

由伴随ju'zhen

\textbf{证毕}。






















