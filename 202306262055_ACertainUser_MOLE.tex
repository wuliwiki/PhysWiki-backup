% 物质的量与摩尔(高中)

\subsection{粒子个数与物质的量}
\footnote{本文参考了Wikipedia的“物质的量”,因此本文遵循CC-BY-SA。一些论坛讨论可以参考 https://www.zhihu.com/question/333779957}
如今,在九年义务教育的耳濡目染、循循善诱下,我们已经逐渐地相信了原子的概念:世界万物都是由大量原子组成的\footnote{防杠声明:我知道肯定有人想说光啦、宇宙射线啦不是由原子组成的}。

那么,这个大量是多大量呢?微观世界是我们难以想象的。一个原子的半径大概是$0.1 \Si{nm} = 10^{-10} \Si{m}$左右,也就是说$1 \Si{m^3}$的空间大致可以容纳$10^{30}=1000000000000000000000000000000$个原子!(希望我没有数错零的个数)相比之下,世界人口大概是\textsl{微不足道的}$8\times 10^9=8000000000$。

因此,如果我们总用“个”来描述微观粒子的个数,那么也太不方便了。就像我们用$0.1 \Si{nm}$来形容原子的半径而不是$0.0000000001 \Si{m}$一样,我们也需要找一个合适的“个”的进位单位。

目前,我们选用的“进位单位”是\textsl{摩尔 $\Si{mol}$},并定义一摩尔是$6.02\times 10^{23}$个粒子。也就是说,当我们说有"$1 \Si{mol}$粒子"时,我们的意思是“$6.02\times 10^{23}$个粒子”。使用摩尔表示微粒个数时,这个量被称为\textsl{“物质的量”} \footnote{严谨治学的人非常在意这里庞杂而晦涩的单位问题,并且拒绝认为"$\Si{mol}$"与"个"是同一个东西。不过我们目前不需要想太多,只需要暂且了解"$1 \Si{mol}$粒子就是$6.02\times 10^{23}$个粒子"。}。使用摩尔的概念之后,表述微观粒子的个数就会稍微简单一些。例如,我们就可以说$10^{30}$个粒子就是$1.6\times 10^6 \Si{mol}$粒子。

写成公式的形式,就是
$$n=\frac{N}{N_A}~,$$
或者
$$N=n \cdot N_A~.$$
其中$N$是粒子数(“$6.02\times 10^{23}$个粒子”),$n$是物质的量("$1 \Si{mol}$粒子"),$N_A = 6.02\times 10^{23} \Si{mol^{-1}}$称为阿伏伽德罗常数\footnote{这是一个近似值,目前采用的准确数值可以参考物理学常数\upref{Consts}。}。

注意到"$\Si{mol}$"不仅可以表述原子的数量,还可以表述分子、电子等所有类别的微观粒子的数量。例如,我们可以说"$1\Si{mol}$的氢气分子$H_2$",这就意味着“$6.02\times 10^{23}$个$H_2$分子”。当然,由于一个氢气分子由两个氢原子组成,因此"$1\Si{mol}$的氢气分子$H_2$中包含$2\Si{mol}$的氢原子$H$"。

为什么我们选取$N_A = 6.02\times 10^{23} \Si{mol^{-1}}$而不是更简明的$10^{20} \Si{mol^{-1}}$或$10^{24} \Si{mol^{-1}}$?这关乎一些化学与测量技术的\textsl{历史原因}。在那时,没有人知道如何确定原子的尺度!他们可能以氢原子$H$的质量作为基准(见下)。
\begin{figure}[ht]
\centering
\includegraphics[width=12 cm]{./figures/6088d068acf89eff.png}
\caption{$1.66\times10^{-24} \Si{mol}$只鼹鼠 (Mole)。图源Pixabay。} \label{fig_MOLE_1}
\end{figure}\footnote{$1/(6.02\times10^{23}) \approx 1.66\times10^{-24}$ 哈哈,实际中我们不用$\Si{mol}$来形容宏观事物。}

\subsection{摩尔质量:连接宏观与微观}
假设我们拿到了一个纯净物(只含有一种成分),量出了他的质量$m$与体积$V$,那么我们就能计算他的密度
$$\rho = \frac{m}{V}~.$$。
同样地,如果我们能测量他的质量$m$与物质的量$n$,那么我们就能计算他的\textsl{摩尔质量}
$$M = \frac{m}{n}~.$$
摩尔质量的含义可以理解为每$\Si{mol}$物质的质量,单位一般是$\Si{g/mol}$。

这个定义似乎有点空中楼阁,因为物质的量很难测量。不过,有一个好消息是,如果你翻开元素周期表,那么每一种元素的原子的相对原子质量在数值上就是他的\textsl{摩尔质量},单位是$\Si{g/mol}$。这样,我们就可以根据化学式确定该物质的摩尔质量。

举一些例子,我们知道氢$H$的相对原子质量是$1$,那么$1 \Si{mol}$($=6.02\times 10^{23}$个)氢原子的质量就是$1 \Si{g}$。
类似地,氢气分子$H_2$的相对分子质量是$2$,因此$1 \Si{mol}$($=6.02\times 10^{23}$个)氢气分子的质量就是$2 \Si{g}$。
反过来说,如果我们有$2 \Si{g}$的$H_2$,那么我们就知道他的物质的量是$n=1 \Si{mol}$。

写为公式的形式
$$m = n \cdot M~,$$
或者
$$n = \frac{m}{M}~.$$
其中$m$是物质的质量,单位是$\Si{g}$,$n$就是物质的量,单位是$\Si{mol}$,$M$数值上是相对原子(分子)质量,单位是$\Si{g/mol}$。这样,我们就联系了物质的宏观质量与组成该物质的粒子的物质的量。当然,这个公式只容易运用于成分清晰的物质。

考虑到$n$的定义,我们还可以直接联系物质质量与粒子个数:
$$m = \frac{N}{N_A} \cdot M = N \cdot \frac{M}{N_A}~,$$
$$N = m \cdot \frac{N_A}{M}~.$$
这个推导其实没有本质上的新东西,因为$\Si{mol}$本身就代指了微粒个数。

\begin{example}{一瓶水中的水分子数}
一瓶平凡的瓶装水大概是$500 \Si{mL}$,即含有$500 \Si{g}$水。水的相对分子质量是$16+1+1=18$,那么也就是说每瓶瓶装水大概含有$500/18=27.8 \Si{mol} = 1.67 \times 10^{25}$个水分子。
\end{example}

\begin{exercise}{一瓶水中的电子数}
估算上一瓶水中的电子个数与质量。(只考虑水,不考虑瓶壁)
\end{exercise}

\begin{example}{我身上有多少原子}
根据 \href{https://www.thoughtco.com/elemental-composition-of-human-body-603896}{Marie}(站外链接)的说法,组成人体的主要元素中,$O$的质量分数是$65\%$,$C$是$20\%$,$H$是$10\%$。

假设人的体重是$70 \Si{kg} = 70 000\Si{g}$,那么$O$的物质的量是$70000*0.65/16 = 2844 \Si{mol}$,$C$是$ 1166 \Si{mol}$,$H$是$ 7000 \Si{mol}$。以上一共是$3694 \Si{mol}$,约为 $6.6\times10^{27}$个原子,与该作者在\href{https://www.thoughtco.com/how-many-atoms-are-in-human-body-603872}{另一篇文章}(站外链接)中的估计$7\times10^{27}$近似。

当然,如果你问的是\textsl{分子数},那么问题复杂得多。例如,因为人身体中存在大量的高分子(蛋白质、DNA等),他们的“相对分子质量”没有很好的定义。
\end{example}

\subsection{应用物质的量:理想气体}
说了这么多,物质的量有什么实际运用呢?一个最简单的例子是理想气体问题,理想气体的体积、压强、温度由理想气体状态方程联系:
$$pV=nRT~.$$
其中$n$就是气体的物质的量,代指气体中的粒子个数。因此,理想气体的性质与气体粒子的数量更直接相关,而不是气体粒子的质量\footnote{当然也不能说是无关。事实上,实际气体很少能完美地符合理想气体方程。}。运用摩尔的概念能更好地刻画气体性质。

根据$n$的定义,理想气体状态方程可以写为物理学家更喜欢的“微观”形式:
$$pV=NkT~.$$
其中$N$是气体粒子个数,$k=\frac{R}{N_A}$是Boltzmann常数。

\subsection{应用物质的量:化学反应}
\begin{table}[ht]
\centering
\caption{化学反应}\label{tab_MOLE_1}
\begin{tabular}{|c|c|c|c|c|c|}
\hline
 & $2H_2$ & $+$ & $O_2$ & $\rightarrow$ & $2H_2O$ \\
\hline
质量变化$\Delta m (\Si{g})$ & $-4$ & & $-32$ & & $+36$ \\
\hline
粒子数变化 $\Delta N$ & $-2$ &  & $-1$ & & $+2$ \\
\hline
物质的量变化 $\Delta n (\Si{mol})$ & $-2$ &  & $-1$ & & $+2$ \\
\hline
化学反应进度 $\Delta \xi (\Si{mol})$ & $1$ & & $1$ & & $1$ \\
\hline
\end{tabular}
\end{table}
我们来思考一个最为简单的化学反应之一:氢气与氧气燃烧生成水。

如果以质量计量的话,我们知道,“$4\Si{g}$氢气与$32\Si{g}$氧气反应生成 $36\Si{g}$水...”。这固然直接体现了\textsl{物质守恒},但在实际运用时,往往需要繁重地计算物质的质量。

从微观角度上看,反应中,反应物粒子碰撞、结合,生成了新的生成物粒子。并且,参与反应的粒子的消耗与生成总是成比例的,而这个比例就是各物质的化学计量数(化学计量数就是化学反应式中物质名称前的数):“$2$个氢气分子与$1$个氧气分子反应生成 $2$个水分子...” 当然在这种情况下,你不能再说"$0.5$个氢气分子..."了。

由于物质的量直接和粒子个数成正比,因此物质的量变和粒子数变直接相关。如果我们以物质的量计量化学反应,我们只需要看化学计量数就能快速计算反应前后的物质的量变。这让表述变得更为简洁与直观:“$2\Si{mol}$氢气与$1\Si{mol}$氧气反应生成 $2\Si{mol}$水...”。这是物质守恒与元素守恒在化学中的另一种体现。当然,知道了物质的量变,你也可以灵活地最终将其换算为质量变。

最后,如果说“物质的量的变化总是成比例的,而这个比例就是化学计量数”,那么我们将所有物质的量的变化同时除以这种物质对应的化学计量数\footnote{符号上,计生成物为正,反应物为负},那么对于一个化学方程式,变化量就只剩下单一变量!这就是化学反应进度$\Delta \xi = \frac{\Delta n_i}{\nu_i}$。

化学反应进度的概念表明,一个化学反应只有一个有效的自由度,反应物生成物的增加或减少都只是这个自由度的体现。“当化学反应进度增加$1\Si{mol}$,意味着$2\Si{mol}$氢气与$1\Si{mol}$氧气反应生成 $2\Si{mol}$水...”
