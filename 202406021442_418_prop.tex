% 命题
% keys 命题
% license Usr
% type Tutor

\begin{definition}{真值}
真值有\textbf{真(True,$1$)}与\textbf{假(False,$0$)}两个值组成。
\end{definition}

\begin{definition}{命题}
\textbf{命题}是具有\textbf{确切的真值}(\textbf{唯一的真值})的无歧义的陈述句。命题无需知道真值具体是什么,但是要求真值一定唯一。
\end{definition}
例如,“$x>3$”不是命题,但“对于所有实数 $x$,$x>3$”是命题。有歧义的,例如“这句话是假的”不是命题。

\begin{definition}{命题的类型}
\begin{enumerate}
\item \textbf{矛盾式(永假式)}:命题在各种的可能情况下均为假,均不成立。
\item \textbf{重言式(永真式)}:命题在各种的可能情况下均为真,均成立。
\item \textbf{可满足式}:命题存在至少一种可能的解释使得命题成立。
\end{enumerate}

\end{definition}

\begin{definition}{命题常量}
对于在一个解释范围内,命题的真值是确定的,则称这命题是\textbf{命题常量}。
\end{definition}
\bdefin