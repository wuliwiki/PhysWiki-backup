% Python文件读写与字符串处理
学习处理文件和保存数据可让你的程序使用起来更容易: 用户将能够选择输入什么样的数据, 以及在什么时候输入; 用户使用你的程序做一些工作后, 可将程序关闭, 以后再接着往下做.
\subsection{Python文件方法}
\textbf{\verb|open()| 方法}

Python \verb|open()| 方法用于打开一个文件,并返回文件对象,在对文件进行处理过程都需要使用到这个函数,如果该文件无法被打开,会抛出错误.


\verb|open()| 函数常用形式是接收两个参数:\textbf{文件名}(file)和\textbf{模式}(mode).
\begin{lstlisting}[language=python]
open(file, mode='r')
\end{lstlisting}
此方法有很多其他参数,一般情况下用不到.其中\verb|file| 必需,是文件路径(相对或者绝对路径); \verb|mode| 可选,代表文件打开模式.

注意:使用 open() 方法一定要保证关闭文件对象,即调用 close() 方法.
\subsection{从文件中读取数据}
