% 常系数线性微分方程
% 复数|complex number|微分方程|高阶微分方程|ordinary differential equation|解法|特征方程|欧拉方程

\pentry{指数函数(复数)\upref{CExp},线性微分方程的一般理论\upref{ODEb1}}

\subsection{实数轴上的复值函数}

首先介绍一个很有用的概念,复值函数.

复值函数和复变函数是不一样的.复变函数中的“变”指变量,因此复变函数是复数域到复数域的映射;但复值函数的自变量还是实数,它只是实数域到复数域的映射而已.

由于任何复数都可以写成$a+\I b$的形式,其中$a, b\in\mathbb{R}$,因此一个复值函数可以看成是两个实值函数的组合.

设$f(x), g(x)$是某给定区间上的连续实值函数,那么复值函数$z(x)=f(x)+\I g(x)$的极限被定义为$\lim\limits_{x\to x_0}z(x)=\lim\limits_{x\to x_0}f(x)+\I\lim\limits_{x\to x_0}g(x)$.进一步,$z(x)$的导函数是$z'(x)=f'(x)+\I g'(x)$.

复数的指数由欧拉公式定义:对于实数$a, b$,有
\begin{equation}
\E^{a+\I b}=\E^a(\cos b+\I\sin b)
\end{equation}

如果$K$是一个常数复数,那么$\E^{Kt}$就是一个关于实变量$t$的复值函数.这个函数继承了实值函数$\E^{at}$的很多优良性质.

\begin{theorem}{}
设$K, K_i$是复常数,$z(t), z_i(t)$是区间$[a, b]$上可导的复值函数,那么我们有:
\begin{itemize}
\item $\frac{\dd }{\dd t}(K_1z_1(t)+K_2z_2(t))=K_1z_1'(t)+K_2z_2'(t)$;
\item $\frac{\dd}{\dd t}(z_1(t)z_2(t))=z'_1(t)z_2(t)+z_1(t)z'_2(t)$;
\item $\E^{(K_1+K_2)t}=\E^{K_1t}\E^{K_2t}$;
\item $\frac{\dd }{\dd t}\E^{Kt}=K\E^{Kt}$.
\end{itemize}
\end{theorem}

另外,如果记$\overline{K}$是复数$K$的共轭\footnote{即如果存在实数$a, b$使$K=a+\I b$,那么$\overline{K}=a-\I b$.},那么我们还有
\begin{equation}
\E^{\overline{K}}=\overline{\E^K}
\end{equation}

复值函数也可以作常系数常微分方程的解.考虑到常微分方程的全体解构成一个线性空间,复值函数可以看成是把这个解空间拓展为一个复数域上的线性空间.如果$\varphi(x)$、$\phi(x)$是某个常微分方程的实值解,那么$\I\varphi(x)$、$\varphi(x)+\I \phi(x)$也都是同一个方程的解.


\subsection{常系数齐次线性微分方程}

由字面意义,可得定义:
\begin{definition}{}
形如
\begin{equation}\label{ODEb2_eq1}
\frac{\mathrm{d}^n}{\dd t^n}x(t)+a_1\frac{\mathrm{d}^{n-1}}{\dd t^{n-1}}x(t)+\cdots+a_{n-1}\frac{\mathrm{d}}{\dd t}x(t)+a_nx(t)=0
\end{equation}
的方程,称为\textbf{常系数齐次线性微分方程}.其中各$a_i$都是常数.
\end{definition}

求导运算中最方便的函数是什么?是$\E^{Kt}$,因为其结果总是正比于原来的函数.假设存在复数$\lambda$,使得$x(t)=\E^{\lambda t}$是\autoref{ODEb2_eq1} 的解,那么我们得到
\begin{equation}\label{ODEb2_eq2}
\E^{\lambda t}\qty(\lambda^n+a_1\lambda^{n-1}+\cdots+a_{n-1}\lambda+a_n)=0
\end{equation}

由于$\E^\lambda t$处处不为$0$,\autoref{ODEb2_eq2} 就可以化为
\begin{equation}\label{ODEb2_eq3}
\lambda^n+a_1\lambda^{n-1}+\cdots+a_{n-1}\lambda+a_n=0
\end{equation}

\autoref{ODEb2_eq3} 就被称为\autoref{ODEb2_eq1} 的\textbf{特征方程(eigenfunction)}.

如果从\autoref{ODEb2_eq3} 中解出$\lambda$\footnote{$\lambda$可以是复数,这就是为什么我们要先介绍复值函数的概念.},那么$x(t)=\E^{\lambda t}$就是\autoref{ODEb2_eq1} 的一个解.根据预备知识中的\autoref{ODEb1_the1}~\upref{ODEb1},如果\autoref{ODEb2_eq3} 有$n$个互不相同的解$\lambda_i$,那么$\{\E^{\lambda_i t}\}$就构成了\autoref{ODEb2_eq1} 的一组基解.这样,我们就把解\textbf{常系数齐次线性微分方程}化为解\textbf{代数方程}了.

\begin{example}{}
考虑方程
\begin{equation}
\frac{\mathrm{d}^2}{\dd t}x(t)-\frac{\dd }{\dd t}x(t)-2x(t)=0
\end{equation}
其特征方程为
\begin{equation}
(\lambda-2)(\lambda+1)=0
\end{equation}

因此其通解为

\begin{equation}
x(t)=A\E^{2t}+B\E^{-t}
\end{equation}
其中$A, B$是常数.




\end{example}


有的时候,特征方程存在重根,导致无法解出$n$个不同的$\lambda$,这时候该怎么办呢?我们先观察以下例子.

\begin{example}{}
考虑方程
\begin{equation}
\frac{\mathrm{d}^2}{\dd t}x(t)-2\frac{\dd }{\dd t}x(t)+x(t)=0
\end{equation}

其特征方程为
\begin{equation}
(\lambda-1)^2=0
\end{equation}

只能解出一个二重根$\lambda=1$.按照之前的讨论,我们




\end{example}


























