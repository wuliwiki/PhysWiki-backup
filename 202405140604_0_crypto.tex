% 加密货币简介
% keys 黄金|数字货币|加密货币|安全性|挖矿|矿工|服务器|工作证明|持有量证明|算力|双花
% license Xiao
% type Tutor

本文试图在尽量不涉及计算机专业知识的前提下讲解\textbf{加密货币(cryptocurrency)}(也叫\textbf{数字货币}或\textbf{虚拟货币})的意义和原理。 注意为了方便理解, 一些细节描述并不十分严谨。 本文只从技术上科普加密货币的原理,而这也是区块链技术的原理(中国支持区块链技术的发展)。 本文不构成任何投资建议, 请严格遵守当地法律和规定。

\subsection{黄金}
相对于数字货币,各国政府发行的货币称为\textbf{法定货币}(简称\textbf{法币})。 法币的出现最早在 1694 年的英国, 在此之前的漫长岁月中, 人类基本在使用一些天然的物质充当货币, 如早期的贝壳, 直到黄金。 直到今天, 黄金仍然承担着货币的角色, 各个国家的国库中依然存有大量的黄金作为法币的信用保证。

黄金能作为等价物是大家的共识, 但注意它没有任何政府背书或发行, 且总量基本固定。 那么它作为货币时, 价值体现在哪里呢? 简单来说就体现在它的记账功能。 如果直接以商品交换商品, 效率非常低, 因为很难配对到合适的人。 黄金或者一般等价物的使用, 相当于产生了不可篡改账本, 安全地记录每个人所拥有的资源。

黄金使交易更便捷, 从而对古代经济发展起到了重大的助推作用。为什么不能用纸和笔记账? 因为不安全——陌生人难以辨认真伪, 很可能造假, 也可能被大量增发。 可见要让大家都信任一个中立的记账系统, 它必须具有以下特点
\begin{enumerate}
\item \textbf{安全性}: 不能伪造, 不能被篡改, 不能被发行者大量增发, 不会变质。 黄金在这点上的优势无疑是巨大的, 黄金几乎无法人工合成(只有核反应才能合成极少量), 金矿勘探和开采的难度非常大, 在自然条件下可以稳定存放。 但有一个缺点就是鉴别真伪成本较高, 需要彻底熔化或者使用现代仪器。
\item \textbf{便携}: 由于黄金总量相对较少, 所以用作货币单价很高, 只需携带少量即可满足日常需求。 但这比起当代电子设备还是不够方便。
\item \textbf{转账方便}: 黄金交易在古代的确是算非常方便, 但在比起比起电子转账同样差很多。
\end{enumerate}

综上,黄金的价值就在于它利用其出色的物理特性, 在历史上相当长的时间里充当账本的角色, 极大地促进了人类社会的早期发展。 需要指出的是黄金这个 “账本” 只记录每人的余额而不计录转账历史, 这是又一缺点。

\subsection{数字货币}
在互联网高度发达的今天, 我们是否可以在互联网上建立一个像黄金一样的数字账本呢? 只要把以上三点实现得更好, 它就可能具有比黄金更大的潜在价值。 这看起来容易, 实则非常困难, 要付出巨大的代价。 困难的根源在于,互联网不能储存类似黄金的实物而只能储存数据,而数据是可以完美复制的。 如果用一段数据代表一定价值的数字货币,那么其拥有者就可以把它完美复制任意多份而任何人都无法辨别哪些是复制品,所以这种幼稚的方法注定是行不通的。

我们知道\textbf{互联网上的数据都储存在服务器中},服务器就是互联网上用于提供信息服务的计算机,通常放在专门的机房中昼夜连续运行。例如当你访问某个网站或玩某个网游时,你的电脑接收到的数据就是对应的服务器发送的。 我们的网银, 支付软件, 游戏币, 充值卡等, 本质上是政府或者公司发行的, 可以称为\textbf{中心化账本}。 这些账本的数据保存在政府(或公司)所拥有的服务器中。单从技术上来说,服务器的所有者对数据有绝对的控制权, 可以对其进行任意修改甚至删除。 所以保障你支付软件中余额不被服务器拥有者任意修改的,是法律而不是技术。 另外一旦出现服务器被黑客攻击、公司破产等情况发生, 他们发行的数字代币的信用也难以有保障。

另一方面, 政府为了刺激经济会经常增发货币, 但也有可能导致通货膨胀。 例如美国在 2020-2021 年因新冠疫情大量增发美元使其他国家的美元储备贬值; 又例如发行者有能力冻结或没收账户中的财产, 或者控制它与其他币种之间的兑换。 所以只从以上黄金的第 1 个特点来看, 所有被政府或公司控制的中心化数字记账方式都远不如黄金。 所以即使第 2 和第 3 点做得再好也还是有缺憾。

那么创造数字黄金的难点就在于如何使一个基于互联网的记账系统,使其像黄金一样不受任何组织的控制, 即\textbf{去中心化(decentralize)}。 只有这样才可能使第 1 点的安全性能和黄金媲美。 这就要求储存该账本的服务器不完全受单个机构控制,单个机构无法篡改(或毁灭)该网络账本。 直到中本聪在 2009 年提出区块链的概念并发行了\textbf{比特币(bitcoin)}后, 这种机制才慢慢完善。 比特币网络是一个\textbf{去中心化(decentralized)}的网络, 也就是说比特币账本的数据储存在很多不同的服务器中,且这些服务器的地位都是等价的,没有哪个服务器具有控制其他服务器数据的权限。 任何人或机构都可以购买自己的服务器并接入该网络, 共同完成记账。 账本在每个服务器中都有一个完整的备份, 这包含该账本有史以来每一个账号发生的每一笔交易记录, 根据这些交易记录可以计算得到每个账户的余额。 服务器之间会互相确认账本的一致性, 若有不同则以多数一致为准\footnote{“以多数一致为准” 并不完全准确,但在学习更多概念以前你可以暂时这么理解。实际上比特币是以最长的区块链为准。}。 这样一来,虽然每台服务器都有具体的所有者,虽然所有者可以任意篡改这台服务器的数据,但篡改后的数据并不会得到其他服务器的认可,也就无法生效。

比特币用户可以匿名生成任意多个账户, 无需提供任何身份信息, 每个账户都有一个独特且\textbf{公开}的\textbf{地址}(一长串字母和数字), 以及与之匹配的密码(一长串字母和数字), 称为\textbf{私钥(private key)},私钥只有账户的持有者知道,正常情况下任何其他用户以及任何服务器都不可能知道。 转账的操作和常用支付软件类似, 但从原理上来说,转账者需要生成一段转账请求的数据,其中包含自己的地址,接收转账的地址,转账的金额,以及其他一些信息,并\textbf{用私钥对该转账请求进行签名}(签名本身也是一小段数据)。 签名后该转账请求被广播到多个比特币服务器, 注意广播的数据中并不包含私钥,仅包含转账的明文信息和签名。 注意转账信息本身是不加密的。

当服务器收到广播后,虽然服务器并不知道转账者的私钥, 但是可以轻易地\textbf{验证转账请求中的签名是否和账户地址匹配}。 签名和验证签名的原理涉及到密码学中的\textbf{非对称加密}, 超出了本文范围, 你现在只需要知道几个结论即可:
\begin{enumerate}
\item 只有私钥的拥有者可以给转账请求正确签名
\item 任何人都可以验证该签名的真实性,无需拥有私钥
\item 无法通过签名还原出私钥
\end{enumerate}
这三条定律是比特币安全性的基石,它们由密码学保障,且被广泛用于金融、军事等各种重要领域。 如果有人能破解非对称加密,那么全世界大多数网络安全措施将同样对他无效。

一笔转账只有取得网络中大部分服务器的\textbf{共识(concensus)}, 确认一致后才能生效。 由于任何人都可以搭建服务器进行记账, 所以所有的\textbf{转账记录都是公开的}, 那么用户隐私如何保障? 隐私的关键在于地址的匿名性:虽然公开的账本可以查到每个地址的所有转账信息,但却无法直接知道谁拥有该地址。 任何人都可以生成任意多个地址, 甚至每笔交易都可以使用一个不同的地址, 所以用户的隐私仍然可以得到一定程度保障。 更多细节我们将在 “\enref{使用数字货币钱包}{CryWal}” 中具体介绍。

%\addTODO{举例! 画 4 台电脑, 四个人互相连接。 若有一些人转账, 向所有人广播, 一起记账并对账。 若其中有一个人不一致, 那么其他三个不采用他的账本。 但若不诚实的人另外买了 5 台电脑, 那么别人就以为他是对的。}

\subsection{安全性的代价}
既然任何人都可以参与到比特币网络中, 那么将其恶意破坏也会变得比中心化网络更容易。 对于中心化服务器, 控制者只需要给服务器设置一个密码就可以防止被黑客入侵, 而如果有人要恶意破坏比特币网络, 他只需要购买大量服务器, 在网络中的数量达到多数就可以通过所谓的 “共识” 来做假账, 例如花一笔比特币然后事后抹掉记录再花一次(即\textbf{双花, double spending})\footnote{但他仍然不能把其他任意账户中的钱转给自己, 因为他没有其他账户的密钥,无法生成正确签名。}。 所以为了提高账本的安全性, 我们必须使得 “买下大部分服务器” 变得异常艰难。 即大大提高每台服务器所需的成本。 比特币网络的做法就是把记账权和服务器的\textbf{算力}(计算能力)挂钩, 每台服务器必须完成一些很难(但没有意义)的数学题才能获取更多的记账权, 这些题的答案很难被算出, 但算出后很容易验证正确性。 理想状况下是整个网络的总算力极高, 且服务器的所有权非常分散, 使得少部分人或机构几乎不可能取得该网络的控制权从而对数据进行恶意篡改。 这些无意义的计算最终会消耗大量资源, 但这却是维持去中心化账本的安全性的必要代价。 在 2016 年, 据估计比特币网络的总算力已经达到世界前 500 的超级计算机集群的总和。 这使得比特币比其他的货币更为安全。

在另一些加密货币中, 为了不过多消能源和硬件, 也有采用把服务器所有者的加密货币余额作为比重获取记账权。 比特币的方法叫做\textbf{工作证明(proof of power)}, 后者叫做\textbf{持有量证明(proof of stake)}。 另一种较新的方式是\textbf{储存空间和时间证明(proof of space and time)}, 也就是所谓的硬盘挖矿。 但无论如何, 为了使网络具有安全性, 网络中的服务器必须以某种珍贵的资源作为记账权的分配标准, 且这种资源需要容易证明且无法造假。

\subsection{挖矿}
既然比特币服务器需要那么多资源, 为什么还有人愿意搭建呢? 为了补偿这些维护比特币网络安全性的服务器所有者, 比特币协议规定每十分钟左右就会对成功解出数学题并获得记账权的账户进行奖励, 所以人们就形象地把提供服务器算力并获取奖励的行为称为\textbf{挖矿(mining)}, 挖矿的人称为\textbf{矿工(miner)\footnote{miner 也可以指挖矿所用的软件。}}。 而所有的比特币都是通过挖矿产生的。 比特币协议规定大约每 4 年挖矿的收益就会减半, 从而使其总量存在上限(2100 万枚)。 除了挖矿获得奖励外, 矿工还会获得一些转账手续费, 手续费在用户转账时从支付的一方转到矿工的地址中。 当所有比特币都被挖完后, 理想情况下此时比特币的价格已经昂贵到足以让转账手续费支持所有矿工的开支, 这样该系统就能自给自足地一直运行下去。

\addTODO{协议或代码如何更新? 硬分叉是什么? 既然代码开源,人人都可以发行新币,为什么它还有价值?}
