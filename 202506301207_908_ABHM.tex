% 罗伯特·奥本海默(综述)
% license CCBYSA3
% type Wiki

本文根据 CC-BY-SA 协议转载翻译自维基百科\href{https://en.wikipedia.org/wiki/J._Robert_Oppenheimer}{相关文章}。

\begin{figure}[ht]
\centering
\includegraphics[width=6cm]{./figures/0ac14d0320e1b5df.png}
\caption{} \label{fig_ABHM_1}
\end{figure}
J·罗伯特·奥本海默(出生名朱利叶斯·罗伯特·奥本海默,/ˈɒpənhaɪmər/,1904年4月22日-1967年2月18日),是一位美国理论物理学家,曾在第二次世界大战期间担任曼哈顿计划洛斯阿拉莫斯实验室的主任。他因在监督第一批核武器的研发中所扮演的角色而常被称为“原子弹之父”。

奥本海默出生于纽约市,1925年在哈佛大学获得化学学位,1927年在德国哥廷根大学师从马克斯·玻恩获得物理学博士学位。在其他机构从事研究后,他加入加利福尼亚大学伯克利分校物理系,并于1936年成为正教授。奥本海默在量子力学和核物理领域作出了重要贡献,包括提出用于分子波函数的玻恩–奥本海默近似;在正电子理论、量子电动力学和量子场论方面的工作;以及核聚变中的奥本海默–菲利普斯过程。他与学生们还在天体物理领域做出了重要贡献,包括宇宙射线簇射理论,以及中子星和黑洞理论。

1942年,奥本海默被招募参与曼哈顿计划,并于1943年被任命为该计划位于新墨西哥州的洛斯阿拉莫斯实验室主任,负责研发第一批核武器。他的领导能力和科学专长对计划的成功起到了关键作用。1945年7月16日,他出席了代号“三位一体”的首次原子弹试爆。1945年8月,这些核武器在广岛和长崎的原子弹轰炸中被用于对日本作战,这也是迄今为止核武器在战争中唯一一次被使用。

1947年,奥本海默被任命为新泽西州普林斯顿高等研究院院长,并出任新成立的美国原子能委员会(AEC)总顾问委员会主席。他主张对核能和核武器实行国际管控,以避免与苏联陷入军备竞赛,后来又出于部分道德原因反对氢弹的研发。在第二次红色恐慌期间,他的这些立场,加上他过去与美国共产党有过的联系,导致1954年美国原子能委员会对其进行安全听证,并最终撤销了他的安全许可。此后,他继续从事物理学领域的讲学、写作和研究工作,并于1963年因其对理论物理的贡献获得恩里科·费米奖。2022年12月16日,美国能源部长詹妮弗·格兰霍姆撤销了1954年的决定,称当时的决定是“有缺陷的程序”的结果,并确认奥本海默一直是忠诚的。
\subsection{早年生活}
\subsubsection{童年与教育}
奥本海默于1904年4月22日出生在纽约市的一个不虔诚犹太家庭,出生名朱利叶斯·罗伯特·奥本海默\(^\text{[note 1]}\)。母亲埃拉(娘家姓弗里德曼)是一位画家,父亲朱利叶斯·塞利格曼·奥本海默是一位成功的纺织品进口商。\(^\text{[5][6]}\)罗伯特有一个弟弟弗兰克,后来也成为物理学家。\(^\text{[7]}\)他的父亲出生在普鲁士王国黑森-拿骚省仍属于哈瑙时,1888年青少年时期只身前往美国,身无分文,没有高等教育,甚至不会英语。他被一家纺织公司雇用,并在十年内成为公司高管,最终积累了财富。\(^\text{[8]}\)1912年,家族搬到纽约哈德逊高地西88街附近的河滨大道的一套公寓,那一带以豪华的府邸和联排别墅闻名。\(^\text{[6]}\)他们的艺术收藏包括巴勃罗·毕加索、爱德华·维亚尔和文森特·梵高的作品。\(^\text{[9]}\)

奥本海默最初在阿尔奎因预备学校接受教育。1911年,他进入由费利克斯·阿德勒创办的伦理文化学会学校,\(^\text{[10]}\)该校以伦理运动为基础进行教育,其校训是“行为重于信仰”。奥本海默的父亲多年是该学会成员,并担任董事会成员。\(^\text{[11]}\)奥本海默是一名兴趣广泛的学生,热衷于英语和法语文学,特别喜欢矿物学。\(^\text{[12]}\)他在一年内完成了三、四年级课程,并跳过了八年级的一半。\(^\text{[10]}\)他还向著名法国长笛演奏家乔治·巴雷尔私下学习音乐。在学业最后一年,奥本海默开始对化学产生兴趣。\(^\text{[13]}\)1921年毕业,但在捷克斯洛伐克的家族度假期间,他在雅希莫夫探矿时感染了结肠炎,因此耽误了一年继续深造。他在新墨西哥州康复,并在那里爱上了骑马和美国西南部地区。\(^\text{[14]}\)

1922年,18岁的奥本海默进入哈佛学院。他主修化学;哈佛还要求学习历史、文学、哲学或数学。为了弥补因疾病耽误的时间,他每学期修六门课程,而非通常的四门。他被接纳为本科荣誉学会Phi Beta Kappa成员,并因独立学习被授予物理学研究生资格,使他能够跳过基础课程直接修读高级课程。他因珀西·布里奇曼教授的热力学课程而被实验物理学吸引。奥本海默仅用三年时间,于1925年以优等生身份从哈佛获得文学士学位毕业。\(^\text{[15]}\)
\subsubsection{在欧洲的求学经历}
1924年,奥本海默被剑桥大学基督学院录取后,曾致信欧内斯特·卢瑟福,请求允许他在卡文迪许实验室工作,尽管布里奇曼在推荐信中指出,奥本海默在实验室的笨拙表明,相较于实验物理,他更适合从事理论物理研究。卢瑟福对此并不感兴趣,但奥本海默仍然前往剑桥求学;\(^\text{[16]}\)最终,J.J. 汤姆森同意接收他,但条件是他必须完成基础实验课程。\(^\text{[17]}\)

奥本海默在剑桥期间非常不开心,他曾写信给一位朋友说:“我现在过得相当糟糕。实验室的工作无聊透顶,而我做得又极差,以至于完全感觉不到自己学到了任何东西。”\(^\text{[18]}\)他与导师帕特里克·布莱基特(未来的诺贝尔奖得主)关系紧张。根据奥本海默的朋友弗朗西斯·弗格森的说法,奥本海默曾坦白说他曾在布莱基特的办公桌上放过一个涂了毒药的苹果,后来奥本海默的父母说服校方不将他开除。虽然并没有关于投毒事件或留校察看的官方记录,但奥本海默曾定期在伦敦哈雷街接受精神科医生的治疗。\(^\text{[19][20][21][22]}\)

奥本海默身材高瘦,是个抽烟成瘾的人,\(^\text{[23]}\)在专注时常常忘记进食。许多朋友都说他有自我毁灭的倾向。弗格森曾试图通过讲述自己向女友弗朗西丝·基利求婚的故事,分散奥本海默明显的抑郁情绪,但奥本海默突然跳向弗格森,试图掐死他。奥本海默一生都受到抑郁症的困扰,\(^\text{[24][25]}\)曾对弟弟说:“我需要物理,甚于朋友。”\(^\text{[26]}\)

1926年,奥本海默离开剑桥,前往哥廷根大学师从马克斯·玻恩学习;当时哥廷根是全球理论物理学的中心之一。奥本海默在此结识了后来取得巨大成就的朋友,包括维尔纳·海森堡、帕斯卡尔·约尔当、沃尔夫冈·泡利、保罗·狄拉克、恩里科·费米和爱德华·泰勒。他在讨论中非常热情,以至于有时会主导整个讨论。\(^\text{[27]}\)玛丽亚·格佩特曾向玻恩递交了一份由她和其他人签署的请愿书,威胁如果玻恩不让奥本海默安静下来,他们将抵制这门课。玻恩把请愿书放在桌上,让奥本海默看见,却一句话都没说,这一举动收到了预期效果。\(^\text{[28]}\)

1927年3月,年仅23岁的奥本海默在玻恩的指导下获得了哲学博士学位。\(^\text{[29][30]}\)据说在口试结束后,主持口试的詹姆斯·弗兰克教授说道:“我很高兴终于结束了,他差点开始反过来考我。”\(^\text{[31]}\)奥本海默在欧洲期间发表了十余篇论文,包括在量子力学这一新兴领域中的许多重要贡献。他与玻恩共同发表了一篇著名论文,提出了玻恩–奥本海默近似,将分子数学处理中核的运动与电子运动分离,使得在计算时可以忽略核的运动以简化计算。这篇论文至今仍是他引用次数最多的学术成果。\(^\text{[32]}\)
