% 群论
\pentry{逻辑量词, 整数\upref{intger}}



\subsection{基本概念}

\begin{definition}{二元运算}\label{Group_def2}
给定一个非空集合$A$,取$A$中任意两个元素$a, b$(可能是同一个元素). 如果有一条规则使得两个元素可以组合,指向另一个元素 $c$(不一定属于 $A$), 则称这条规则为一个\textbf{二元运算(binary operation)}. 如果 $c\in A$, 那么我们说这个二元运算是\textbf{封闭的(closed)},有的地方也会称作“\textbf{闭合的}”.
\end{definition}

运算的符号可以任意决定,如果使用$\cdot$作为运算符,那么定义中的情况就可以简单写为$a· b=c$. 注意,这里的 $\cdot$ 只是表示某一个运算,不一定是我们通常的乘法或点乘运算.

和二元运算类似,我们也可以称更多元素间相组合的规则为\textbf{多元运算}.

\begin{definition}{群}\label{Group_def1}
一个群$(G, \cdot)$是在集合$G$上赋予了一个运算$\cdot$的结构,该运算满足以下要求:\\

\begin{enumerate}
\item \textbf{闭合性}:$∀x, y∈ G, x· y∈ G$,即任意$G$中元素$x$,$y$满足$x· y$仍是$G$中元素
\item \textbf{结合性}:$∀ x, y, z\in G, x·(y· z)=(x· y)· z$
\item \textbf{单位元}存在性:$∃ e\in G, ∀ x∈ G, e· x=x· e=x$
\item \textbf{逆元}存在性:$∀ x∈ G, ∃ y∈ G, x· y=y· x=e$.通常我们会把这样的$y$称作$x$的逆元,并记为$x^{-1}$
\end{enumerate}
\end{definition}
严格来说,这样的一个群应该表示为$(G,\cdot)$,而$G$表示的是没有赋予运算的集合.但是为了方便讨论,我们通常也会直接用定义群的这个集合来称呼这个群,比如简单地把上述定义的群叫做\textbf{群$G$.}

实际上,我们可以用更为弱化的公理系统来定义群,比如第 4 条只要求存在左逆元, 即只要求$∀ x∈ G, ∃ y∈ G, y· x=e$. 在这种情况下我们仍然可以证明左逆元都是右逆元\footnote{证明: 对任意 $a$, 假设其左右逆元分别为 $b$ 和 $c$, 则 $bac$ }.但是为了方便理解,我们用了以上对称的公理系统.

\subsection{群的例子}

\begin{exercise}{二元群}\label{Group_exe1}
定义一个只含有两个元素的集合,记为$\{0, 1\}$.在这个集合上定义运算 “$\cdot$”,由于只有四种运算方式,所以可以通过列举出每一个运算的结果来定义这个运算:
\begin{equation}\label{Group_eq1}
0\cdot0=0 \qquad 0\cdot1=1 \qquad 1\cdot0=1 \qquad 1\cdot1=0
\end{equation}
\begin{itemize}
\item 请用一个 $2\times2$ 的表格表示运算规则
\item 请根据\autoref{Group_def1}验证这个二元集合配上运算 $\cdot$ 构成一个群
\item 尝试改变\autoref{Group_eq1} 的规则使其同样构成一个群
\end{itemize}
\end{exercise}

在以上例子中,尽管$0\cdot1=1\cdot0$,我依然把它们分别单独写了出来,这是因为群的定义不要求\textbf{交换律}成立,也就是说, 群运算允许 $x\cdot y\neq y\cdot x$.群元素选为$0$和$1$没有特殊原因,只是代表这是群里两个不同的元素而已,任何由两个元素构成的群我们都看作同一个. 运算满足交换律的群被称为\textbf{阿贝尔群(abelian group)}, 否则称为\textbf{非阿贝尔群(non-abelian group)}. 习惯上,我们把阿贝尔群的运算叫做加法,记为“$+$”,而把非阿贝尔群的运算叫做乘法,记为“$\cdot$”,甚至简化为没有符号,比如$ab\not= ba$.不过,即使是非阿贝尔群中也可能存在两个元素$a$和$b$,使得$ab=ba$;这时我们说$a$和$b$交换(commute).

一般地,由于在朴素集合论中我们最多只讨论了集合的基数问题,集合中的元素具体如何命名是没有约束的,因此在集合论意义下元素数目相同的集合都看作同一个.比如说,我们认为$\{\text{猪},3, K\}$和$\{1,2,3\}$是同一个集合.而现在在集合上定义了一个群运算以后所得到的群,即使构成它们的集合相同,群也可能由群运算的不同而产生不同的结构,从而被看作是不同的群.

\begin{example}{$n$ 元循环群}\label{Group_ex2}
取一个由$n$个元素组成的集合$G$,由于集合元素命名的任意性,不妨把$G$记为$\{0, 1, \cdots n-1\}$,定义运算为模$n$的加法,即在一个有$n$个整点的钟表上的加法(见“整数\upref{intger}”).那么这个运算构成$G$上的一个群运算,所构成的群$G$称为 \textbf{$n$元循环群(n-element cyclic group)}.
\end{example}

\begin{example}{$n$ 元置换群}\label{Group_ex3}
首先给定一个$n$元集合,记作$K=\{1,2, \cdots, n\}$,并将$K$中的元素按现有的顺序编号. 把$K$看作是$n$个桶中分别装了1个写着编号的球,初始状态下球的编号和桶的编号一致.我们可以把球从桶里面拿出来并进行任意的置换,保持每个桶里还是只有一个球,但是球的编号不一定和桶的编号一致了.每一个置换可以详细描述为“把1号桶的球和2号桶的球交换”,“把1号桶的球放入3号桶,3号桶的放入4号桶,4号桶的放入1号桶”等等.

我们用全体“置换”动作来作为元素,构成一个集合,称作$n$个元素的置换集合($n$元置换集),记为$S_n$.\textbf{从原始状态}进行任意置换,所得到的结果状态和置换是一一对应的,所以我们也可以用“从原始状态进行置换$f$所得的结果”来表示置换$f$本身.

置换之间可以定义一个运算“$\circ$”,被称为置换间的复合,它是这样定义的:如果$f$和$g$是两个置换,那么$g\circ f$就是先进行$f$置换,再进行$g$置换.注意先后次序是从左到右进行的.

我们也可以这样来理解一个置换:原始状态下,$n$号桶中的小球为$n$.进行一次$f$置换后,$n$号桶中的小球就变成了$f(n)$,再进行一次$g$置换,那么$n$号桶中现在装的小球就变为$g(f(x))$.这个过程也可以看成是进行了一次$g\circ f$运算,让$n$号桶中的小球变成$g\circ f(n)$.

现在我们有了一个由置换组成的集合以及置换之间的运算,把该集合和运算代入群的定义,会发现四个条件都被满足了.因此,$(K, \circ)$是一个群.注意,绝大多数$S_n$是不交换的.

当$K$共有$n$个元素时,习惯上我们把这个群称为$S_n$.
\end{example}


\begin{example}{对称群}\label{Group_ex4}
对称性的意思,是在某种条件下保持不变的性质.比如说,平面上的一个正方形,绕着几何中心旋转角度为$\pi$(即角度制下的$180^\circ$)的时候和没有旋转是一样的,那么我们就说正方形是关于角度为$\pi$的旋转对称的.所有能够使得这个正方形不变的平面变换,比如特定角度的旋转、关于特定轴的翻转,配上变换间的复合运算(即先进行一个变换,再进行另一个),构成了一个群,称为这个正方形的\textbf{对称群(symmetry group)}.

\end{example}

\begin{example}{$n$阶方阵群}\label{Group_ex5}
给定域$F$上,全体$n\times n$可逆矩阵构成的集合,配上矩阵乘法就构成了一个非阿贝尔群.%需要定向到“域”论词条.

这样的群被简记为$GL(n, F)$.当不至于混淆时,这里的$F$一般会指全体实数或者全体复数,这时也会把该群简记为$GL(n)$. 


\end{example}