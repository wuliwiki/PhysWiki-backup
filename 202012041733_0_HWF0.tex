% 氢原子基态的波函数
% 薛定谔方程|束缚态|基态|概率分布

\pentry{波函数简介}%未完成链接

由于波函数的统计诠释,统计在量子力学中经常碰到,所以这里举一个例子让你熟悉一下统计的一些常见计算.

氢原子是唯一有解析解的物理实例,因为它结构简单,只有一个核外电子.由于核外电子质量又远小于原子核的质量,忽略核的运动,且不计万有引力.

氢原子基态的波函数为
\begin{equation}
\psi(\bvec r) = A\E^{-r/a}
\end{equation}
其中 $a_0 = 4\pi\varepsilon_0 \hbar ^2/(m_e e^2)$ 是量子理论中一个重要的常数,\textbf{玻尔半径}.由于这是个球对称函数,所以氢原子的波函数通常在球坐标中表示,即表示成三个球坐标的函数 $\psi (\bvec r) = \psi (r, \theta, \phi)$. 其模长平方同样表示粒子在某点出现的概率密度.由于氢原子基态的波函数是球对称的,所以只是 $r$ 的函数.

\subsection{归一化}
  
概率密度必须归一化,也就是说,在所有地方找到电子的概率之和为必为 $1$. 所以可以用归一化来确定波函数前面的系数 $A$. 把概率密度对整个空间体积分
\begin{equation}
1 = \int \abs{\psi(\bvec r)} ^2 \dd{V}  = \int_0^{+\infty} (A^2 \E^{-2r/a})(4\pi r^2\dd{r})  = A^2 \pi a^3
\end{equation}
所以 $A = 1/\sqrt{\pi a^3}$, 
\begin{equation}
\psi (\bvec r) = \frac{1}{\sqrt{\pi a^3}} \E^{-r/a}
\end{equation}

\subsection{位置的平均值}

 根据连续概率分布中平均值%未完成链接
(或数学期望)的定义
\begin{equation}
\ev{\bvec r} = \int \bvec r \abs{\psi (\bvec r)} ^2 \dd{V} =  \bvec 0
\end{equation}
积分显然为零,因为波函数关于中心呈球对称分布,各个方向的 $\bvec r$ 互相抵消了.所以如果对足够多个处于基态的氢原子测量电子的位置,并求平均位置(矢量), 一定会在原子核处.

\subsection{电子离原子核距离的平均值}

如果在上题中,求平均值的不是位置矢量,而是位置的大小,那么结果显然是大于零的.
\begin{equation}
\ev{r} = \int r \abs{\psi (\bvec r)}^2 \dd{V} = \int r (A^2 \E^{-2r/a})(4\pi r^2 \dd{r})= \frac{3}{8} a^4 A = \frac{3}{2}a
\end{equation}
注意这比玻尔半径要大.

\subsection{电子最可能出现的位置}

 一个位置的波函数模长平方越大,电子越有可能出现在这个位置.所以现在要求的是概率密度出现最大值的位置.
 
根据指数函数的性质,最大值 $\abs{\psi (\bvec r)}_{\max}^2 = \qty( \E^{-0/a}/\sqrt{\pi a^3})^2 = 1/(\pi a^3)$, 最大值位置为 $\bvec r = \bvec 0$.

\subsection{电子与原子核最有可能的距离}
若定义径向概率密度为
\begin{equation}
f(r) = \lim_{\Delta r \to 0} \frac{\int_r^{\Delta r} {\abs{\psi (\bvec r)}}^2 (4\pi r^2 \dd{r}) }{\Delta r} = 4\pi r^2 \abs{\psi(\bvec r)}^2
\end{equation}
要求最可能出现的半径 $f(r)_{\max}$, 可以对其求导(见“导数与函数极值\upref{DerMax}”). 即 $\dv*{f(r)}{r} = 0$ 即 $8 r \E^{-2r/a}/a^3 - (4/a^3)(2/a)r^2 \E^{-2r/a} = 0$, 解得 $a = r$. 
 
这个重要结论说明,\textbf{玻尔半径就是氢原子基态中电子与原子核最有可能的距离}.
