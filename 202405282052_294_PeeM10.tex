% 2010 年考研数学试题(数学一)
% keys 考研|数学一
% license Copy
% type Tutor

\textbf{声明}:“该内容来源于网络公开资料,不保证真实性,如有侵权请联系管理员”

\subsection{选择题}
\begin{enumerate}
\item 极限  $\displaystyle \lim_{n\to\infty}[\frac{x^2}{(x-a)+(x+b)}]^x$=$(\quad )$\\
(A) $1$\\
(B) $e$\\
(C) $e^(a-b)$\\
(D) $e^(b-a)$
\item  设函数 $z=z(x,y)$ 由方程 $\displaystyle F(\frac{y}{x},\frac{z}{x})$ 确定,其中 $F$ 为可微函数,且 $F'_2 \neq0$ ,则  $\displaystyle x \pdv{z}{x}+y\pdv{z}{y}$=$(\quad )$
(A)  $x$
(B)  $z$
(C) $-x$
(D)  $-z$
\item 设 $m,n$ 均是正整数,则反常积分 $\displaystyle \int_0^1 \frac{\sqrt[m]{\ln^2(1-x)}}{\sqrt[n]{x}}$ 的收敛性 $(\quad )$\\
(A) 仅与 $m$ 的取值有关\\
(B) 仅与 $n$ 的取值有关\\
(C)  与 $m,n$ 的取值都有关\\
(D)  与 $m,n$ 的取值都无关
\item $\displaystyle \lim_{n\to\infty} \sum_{i=1}^n \sum_{j=1}^n \frac{n}{(n+i)(n^2+j^2)}$= $(\quad )$\\
(A) $\displaystyle \int_{0}^{1}\dd{x}\int_{0}^{x} \frac{1}{(1+x)(1+y^2)}$\\
(B)$\displaystyle \int_{0}^{1}\dd{x}\int_{0}^{x} \frac{1}{(1+x)(1+y)}$\\
(C)$\displaystyle \int_{0}^{1}\dd{x}\int_{0}^{1} \frac{1}{(1+x)(1+y)}$\\
(D)$\displaystyle \int_{0}^{1}\dd{x}\int_{0}^{1} \frac{1}{(1+x)(1+y^2)}$
\item 设 $\mat A$ 为 $m$x$n$ 矩阵,$\mat B$  为 $n$x$m$  矩阵,$\mat E$  为  $m$ 阶单位矩阵,若 $\mat {AB=E}$ ,则$(\quad )$\\
(A) 秩 $r(\mat A)=m$ ,秩  $r(\mat  B)=m$ \\
(B) 秩 $r(\mat A)=m$ ,秩  $r( \mat B)=n$ \\
(C) 秩 $r(\mat A)=n$ ,秩  $r( \mat B)=m$ \\
(D) 秩 $r(\mat A)=n$ ,秩  $r(\mat B)=n$ \\
\item 设 $\mat A$ 为4阶实对称矩阵,且 $\mat {A^2+A=O}$  。若 $\mat A$ 的秩为3,则 $\mat A$  相似于 $(\quad )$\\
(A) $\pmat{1& & &  \\ &1& & &\\ & &1&\\& & &1}$\\
(B) $\pmat{1& & &  \\ &1& & &\\ & &-1&\\& & &0}$\\
(C) $\pmat{1& & &  \\ &-1& & &\\ & &-1&\\& & &0}$\\
(D) $\pmat{-1& & & \\ &-1& & &\\ & &-1& \\& & &0}$
\item  设随机变量 $X$ 的分布函数 $F(x)=\leftgroup{&0,& x<0,\\ &\frac{1}{2},& 0\le x<1\\ &1-e^{-x},& x \ge 1}$   ,则$P\{X=1\}$=$(\quad )$\\
(A)  $0$\\
(B) $\frac{1}{2}$\\
(C)  $\frac{1}{2}-e^{-1}$\\
(D) $1-e^{-1}$
\item  设 $f_1(x)$ 为标准正态分布的概率密度,$f_2(x)$   为 $[-1,3]$ 上均匀分布的概率密度,若 $f(x)=\leftgroup{af_1(x),\quad x\le0\\bf_2(x),\quad x>0}\quad $ $(a>0,b>0)$ 为概率密度,则 $a,b$应满足$(\quad )$\\
(A) $2a+3b=4$\\
(B)   $3a+2b=4$\\
(C)  $a+b=1$\\
(D)   $a+b=2$
\end{enumerate}
\subsection{填空题}
\begin{enumerate}
\item 设 $\leftgroup{& x=e^{-t}\\& y=\int_0^1 \ln(1+u^2)\dd{u}}$   ,则 $\displaystyle \eval{\dv[2]{y}{x}}_{t=0}$=$(\quad )$
\item $\int_0^{\pi^2}\sqrt{x}\cos \sqrt{x}\dd{x}$=$(\quad )$
\item  已知曲线 $L$ 的方程为 $y=1-\abs{x} (x \in [-1,1])$,起点是 $(-1,0)$ ,终点为 $(1,0)$ ,则曲线积分 $\displaystyle \int_L xy\dd{x}+x^2\dd{y}$=$(\quad )$
\item 设 $\Omega=\{(x,y,z)|x^2+y^2\le z \le1\}$ ,则 $\Omega$ 的形心的竖坐标 $\bar z$= $(\quad )$
\item 设 $\alpha_1=(1,2,-1,0)\Tr ,\alpha_2=(1,1,0,2)\Tr ,\alpha_3=(2,1,1,a)\Tr $  。若由 $\alpha_1,\alpha_2,\alpha_3$  生成的向量空间的维数为2,则  $a$=$(\quad )$
\item  设随机变量 $X$  的概率分布为 $P\{X=k\}=\frac{C}{K!},k=0,1,2\dots$,则 $E(X^2)$=$(\quad )$
\end{enumerate}
\subsection{解答题}
\begin{enumerate}
\item 求微分方程 $y''-3y'+2y=2xe^x$ 的通解。
\item 求函数 $\displaystyle f(x)=\int_{1}^{x^2}(x^2-t)e^{-t^2}\dd{t}$ 的单调区间与极值。
\item (1) 比较 $\displaystyle \int_0^1 \abs{\ln t}[\ln (1+t)]^n\dd{t} $ 与  $\displaystyle \int_0^1 t^n \abs{\ln t}\dd{t} (n=1,2,\dots)$ 的大小,说明理由;\\
(2)记 $\displaystyle u_n= \int_0^1 \abs{\ln t}[\ln (1+t)]^n\dd{t} (n=1,2,\dots)$ ,求极限  $\displaystyle \lim_{n\to\infty} u_n$
\item  求幂级数 $\displaystyle \sum_{n=1}^\infty \frac{(-1)^{n-1}}{2n-1}x^{2n}$ 的收敛域及和函数。
\item 设 $P$ 为椭球面 $S$ :$x^2+y^2+z^2-yz=1$   上的动点,若 $S$ 在点$ P$ 处的切平面与 $xOy$ 面垂直,求点$P$的轨迹$C$,并计算曲面积分 $\displaystyle I=\iint_\Sigma \frac{(x+\sqrt{3})\abs{y-2z}}{\sqrt{4+y^2+z^2-4yz}}\dd{S}$  ,其中 $\Sigma$ 是椭球面  $S$位于曲线$ C$ 上方的部分。
\item 设 $\mat A=\pmat{\lambda &1&1\\0&\lambda-1&0\\1&1&\lambda},\mat b=\pmat{a\\1\\1}$  。已知线性方程组 $\mat {Ax=b}$ 存在2个不同的解。\\
(1)求 $\lambda,a$;\\
(2)求方程组 $\mat {Ax=b}$ 的通解。
\item 已知二次型 $f(x_1,x_2,x_3)=\mat {x \Tr Ax}$  在正交变换 $\mat {x=Qy}$ 下的标准型为 $y_1^2+y_2^2$  ,且 $Q$ 的第三列为 $\displaystyle (\frac{\sqrt{2}}{2},0,\frac{\sqrt{2}}{2})\Tr$。\\
(1)求矩阵  $\mat A$;\\
(2)证明$\mat{ A+E}$  为正定矩阵,其中 $\mat E$ 为3阶单位矩阵。
\item 设二维随机变量 $(X,Y)$ 的概率密度为 $f(x,y)=A$
\end{enumerate}