% AdS/真实世界
% keys AdS
% license Usr
% type Wiki

AdS/CFT起源于弦论。但是这些年AdS/CFT的应用已经不仅仅局限在理论粒子物理领域里面了。AdS/CFT已经被用于分析我们的真实世界。比如说,AdS/CFT可以用于QCD,核物理,非平衡物理和凝聚态物理。事实上,AdS/CFT的论文在arXiv上的所有的物理板块都可以找到踪影。

比方说,强相互作用力的本质可以用QCD来很好的理解。但是微扰论几乎是失效的,因为强相互作用力一般来说也是强耦合的。但是如果我们利用好AdS/CFT对偶,我们就可以利用AdS时空的引力理论去研究强耦合的QCD理论。

一个例子是所谓的夸克-胶子等离子体(QGP)。按照QCD的说法看,质子,中子的基本自由度并不是质子,中子,而是可以被分为更小的单元,比如说夸克,胶子这样的东西。夸克和胶子一般来说会被囚禁在质子和中子中。如果温度足够高,夸克和胶子就会解禁闭,形成夸克胶子等离子体。根据夸克胶子等离子体的实验,夸克胶子等离子体就像流体一样,剪切粘度非常小。这说明夸克饺子等离子体是一个强耦合系统,非常难以分析。但是幸运的是,用AdS/CFT中AdS那边的黑洞预言的CFT这边的夸克胶子等离子体的粘度非常解决夸克胶子等离子体粘度的实验测量值。这让AdS/CFT对于弦理论物理学家之外的物理学家也有了很强的兴趣。

那么黑洞为什么会有粘度呢?如果我们把石头扔进水里,水