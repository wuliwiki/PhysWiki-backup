% R-矩阵法(量子力学)

\subsection{一维薛定谔方程}
一个算符是否为厄米算符与边界条件有关. 例如
\begin{equation}
H = -\frac{1}{2m}\dv[2]{x} + V(r)
\end{equation}
要证明厄米性, 用分部积分法得
\begin{equation}\label{Rmat_eq1}
\int_{-\infty}^{+\infty} uHv\dd{x} - \int_{-\infty}^{+\infty} vHu\dd{x}
= \eval{-\frac{1}{2m}[uv' - u'v]}_{-\infty}^{+\infty}
\end{equation}
由于我们假设波函数在无穷远处消失, 则该式为零, 说明 $H$ 是厄米的. 但如果在有限区间 $[0, a]$ 中, 则波函数的边界条件必须满足 $\eval{[uv' - u'v]}_{0}^{a} = 0$ 才能保证厄米性.

但若边界条件不符合该要求, 为了在 $[0,a]$ 内构造一组离散的正交归一基底, 我们可以拼凑一个厄米算符. 为了方便起见, 我们要求 $u(0) = v(0) = 0$. 把\autoref{Rmat_eq1} 移项并修改积分区间得
\begin{equation}
\qty[\int_{0}^{a} uHv\dd{x} + \frac{1}{2m}u(a)v'(a)] - \qty[\int_{0}^{a} vHu\dd{x} + \frac{1}{2m}u'(a)v(a)]
= 0
\end{equation}
而又可以通过狄拉克 $\delta$ 函数\upref{Delta} 表示为
\begin{equation}
\int_{0}^{a} u\qty[H + \frac{\delta(x-a)}{2m}\dv{x}] v\dd{x} -
\int_{0}^{a} v\qty[H + \frac{\delta(x-a)}{2m}\dv{x}] u\dd{x} = 0
\end{equation}
所以无论波函数在 $x=a$ 端的边界条件如何, 方括号中的算符都是厄米的. 令布洛赫算符(Bloch operator) 为
\begin{equation}
\mathscr L = \delta(x-a)\dv{x}
\end{equation}
那么修正后的哈密顿算符(厄米算符)就是 $H + \mathscr L/(2m)$. 于是本征方程为
\begin{equation}\label{Rmat_eq2}
\qty(H + \frac{\mathscr L}{2m})\chi_i = \frac{k_i^2}{2m} \chi_i
\end{equation}
本征值为 ${k_i^2}/{2m}$. 于是 $\chi_i$ 就构成一组 $[0, a]$ 内的正交归一基底, 满足 $\int_0^a \chi_i \chi_j \dd{x} = \delta_{ij}$. 由于\autoref{Rmat_eq2} 中
