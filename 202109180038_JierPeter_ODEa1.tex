% 一阶常微分方程解法:变量可分离方程
% 常微分方程|ordinary differential equation|ODE|初等解法|变量分离法

形如$\frac{\dd y}{\dd x}=f(x)g(y)$的微分方程,称作\textbf{变量可分离}方程.这类方程是最容易解的.

对方程两边进行移项,得到$\frac{1}{g(y)}\dd y=f(x)\dd x$,这样就把变量分离开了.两边同时求积分,得到等式
\begin{equation}\label{ODEa1_eq1}
\int\frac{1}{g(y)}\dd y=\int f(x)\dd x+C
\end{equation}
其中$C$是积分常数.

可见,\autoref{ODEa1_eq1} 的左边是$y$的函数,右边是$x$的函数,只要能把这两个积分写出来,那么微分方程也就解出来了.

\begin{example}{}
考虑方程$\frac{\dd y}{\dd x}=xy$.

分离变量后有
\begin{equation}
\int\frac{1}{y}\dd y=\int x\dd x+C
\end{equation}

算出积分后可得
\begin{equation}
\ln\abs{y}=\frac{1}{2}x^2+C
\end{equation}

即
\begin{equation}
\abs{y}=\E^{\frac{1}{2}x^2}\cdot\E^C
\end{equation}

将$\E^C$重新记为一个新的常数$K$,那么该方程的通解最终就可以表示为
\begin{equation}
y=\pm K\E^{\frac{1}{2}x^2}
\end{equation}
\end{example}

\begin{exercise}{}
求解方程$\frac{\dd y}{\dd x}=\frac{x}{y}$.答案是$\abs{x}=\abs{y}$,也可以写成$x^2=y^2$.
\end{exercise}

\begin{exercise}{}
求解方程$\frac{\dd v}{\dd t}=\frac{t^2}{v+b}$.答案是$3v^2+6bv=2t^3$,也可以写成$t=(\frac{3}{2}v^2+3bv)^{1/3}$.
\end{exercise}

\subsection{可化为变量可分离形式的方程}

\begin{theorem}{}
形如$\frac{\dd y}{\dd x}=g(\frac{y}{x})$的方程,可以化为变量可分离的形式.
\end{theorem}

\textbf{证明}:

定义一个新的变量$u=\frac{y}{x}$,于是$y=ux$,于是$\dd y=u\dd x+x\dd u$.

于是原方程可写为
\begin{equation}
u+x\frac{\dd u}{\dd x}=g(u)
\end{equation}
也就是
\begin{equation}
\frac{\dd u}{\dd x}=\frac{g(u)-u}{x}
\end{equation}

这是一个关于$x$和$u$的变量可分离方程.

\textbf{证毕}.

\begin{example}{}

\end{example}


















