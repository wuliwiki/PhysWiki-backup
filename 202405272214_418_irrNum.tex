% 无理数(数论)
% keys 无理数
% license Usr
% type Tutor

\begin{definition}{有理数与无理数}
能表示为互素的整数 $a$ 与 $b\neq 0$ 的比值 $a/b$ 的数称为\textbf{有理数}。

非有理数,也就是不能表示为互素整数 $a$ 与 $b\neq 0$ 的比值 $a/b$ 的数称为\textbf{无理数}。
\end{definition}

\begin{theorem}{}
$\sqrt{2}$ 无理。
\end{theorem}
\textbf{证明}:假设若 $\sqrt 2$ 有理,则对于 $\sqrt 2 = a/b$,即方程 $a^2=2b^2$ 有整数解,且 $(a, b) = 1$,而 $b | a^2$,故对于任意 $b$ 的素因子 $p$