% 单位分割
% partition of unity|bump function|光滑函数|光滑延拓|流形|支撑集|support set|support|supp

\pentry{流形\upref{Manif}}

单位分割是数学分析和现代微分几何中的重要理论工具.我们首先直接给出单位分割的概念.

\begin{definition}{单位分割}
给定\textbf{拓扑空间}$X$.如果存在一组函数 $\rho_\alpha: X \to [0, 1]$,使得对于任意 $x\in X$,有:
\begin{enumerate}
\item 只有有限多个 $\rho_\alpha(x)$ 不为零;
\item 全体不为零的 $\rho_\alpha(x)$ 之和为 $1$.
\end{enumerate}
则称这组函数是 $X$ 上的一个\textbf{单位分割(partition of unity)}
\end{definition}

“单位分割”这一术语不难理解:“单位”就是指 $1$ 这个数字,“分割”就是把它拆分开.单位分割中的所有函数加起来,就得到一个 $X$ 上恒等于 $1$ 的函数.

定义中强调“全体不为零的 $\rho_\alpha(x)$ 之和”是出于严谨考虑,因为我们并没有定义“不可数多个数字之和”这一运算,而单位分割完全可以包含任意多个函数.当然,第一条限制了只有有限多个函数值不为零,我们确实可以定义全体函数值之和就是全体非零函数值之和,但那还是需要“定义”这一过程的,不如直接按以上定义里的表述来了.

\begin{definition}{一般集合的支撑集}
给定\textbf{集合}$S$ 和其上一个函数 $f:S\to\mathbb{R}$,记 $\opn{supp} f = \{s\in S| f(s)\not=0\}$,称其为函数 $f$ 的\textbf{支撑集(support)},也可以简称\textbf{支集}.
\end{definition}

\begin{definition}{拓扑空间的支撑集}
给定\textbf{拓扑空间}$X$ 和其上一个函数 $f:X\to\mathbb{R}$,则 $f$ 的\textbf{支撑集}为 $\{s\in S| f(s)\not=0\}$ 的\textbf{闭包(closure)}.
\end{definition}

注意两种语境(一般集合还是赋予了拓扑的集合)下支撑集概念的区别.定义支撑集的概念,是为了方便引出“从属”的概念.这是一个单位分割(作为函数的集合)和覆盖(作为开集的集合)之间的关系.

\begin{definition}{从属}
给定\textbf{拓扑空间}$X$.如果 $\{\rho_\alpha\}$ 是 $X$ 上的一个单位分割,$\{U_\alpha\}$ 是 $X$ 的一个开覆盖,且有 $\forall \alpha, \opn{supp}\rho_\alpha\subseteq U_\alpha$,那么我们说 $\{\rho_\alpha\}$ 是\textbf{从属于(subordinate to)}$\{U_\alpha\}$ 的.
\end{definition}

\subsection{突起函数(隆起函数)}

\begin{definition}{突起函数}
在拓扑空间 $X$ 上,如果对于点 $x\in X$ 的某个邻域 $U$,存在一个函数 $f:X\to\mathbb{R}^+\cup\{0\}$,使得存在 $x$ 的邻域 $O\subseteq U$,有:
\begin{enumerate}
\item $f(O)=\{1\}$;
\item $\opn{supp} f \subseteq U$.
\end{enumerate}
则称 $f$ 是 $x$ 上 $U$ 内的一个\textbf{突起函数(bump function)}.

称 $f^{-1}(1)$ 为 $f$ 的\textbf{平台区}.
\end{definition}

如果给定一个\textbf{流形}上的非负光滑函数 $h$,它在某区域上恒为 $0$,我们是可以利用它来构造一个突起函数的.我们用实数轴举一个例子来抛砖引玉,任意欧几里得空间中的例子可以通过对各个坐标分别进行相同程序得到,任意流形上的例子可以通过选择适当的图来将其局部变为欧几里得空间来重复以上程序.

\begin{example}{}\label{ParUni_ex1}
我们使用的原料是 $h:\mathbb{R}\to\mathbb{R}^+\cup\{0\}$,其中当 $t\leq 0$ 时 $h(t)=0$,当 $t>0$ 时 $h(t)=e^{-1/t}$\footnote{没错,又是这个非常好用的光滑函数.}.

第一步,构造阶梯函数:
\begin{equation}
g(t)=\frac{f(t)}{f(t)+f(1-t)}
\end{equation}

第二步,用阶梯函数变形:
\begin{equation}
f(t)=g(t)g(3-t)
\end{equation}

这样,我们就得到了一个平台区为 $(1, 2)$、支撑集为 $[0, 3]$ 的突起函数了.
\end{example}

\begin{exercise}{}
画出\autoref{ParUni_ex1} 中 $f(t)$ 的图像.
\end{exercise}

\addTODO{讨论bump function和单位分割存在性的关系.}




