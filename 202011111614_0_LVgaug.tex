% 长度规范和速度规范
% keys 长度规范|速度规范|波函数|规范变换|薛定谔方程|麦克斯韦方程组

\begin{issues}
\issueDraft
\end{issues}

\pentry{电磁场中的单粒子薛定谔方程\upref{QMEM}}

我们已知库伦规范下, 电磁场中带电粒子的哈密顿量为(\autoref{QMEM_eq4}~\upref{QMEM})
\begin{equation}
H = -\frac{1}{2m} \laplacian + \I \frac{q}{m} \bvec A \vdot \Nabla + \frac{q^2}{2m} \bvec A^2 + q\varphi
\end{equation}

\subsection{速度规范}
对库仑规范使用规范变换
\begin{equation}\label{LVgaug_eq3}
\Psi(\bvec r, t) = \exp(\I q\chi)\Psi^V(\bvec r, t)
\end{equation}
\begin{equation}\label{LVgaug_eq4}
\chi = \frac{q}{2m} \int^t \bvec A^2(t') \dd{t'}
\end{equation}
就可以将 $\bvec A^2$ 项消去
\begin{equation}
\I \pdv{t} \Psi^V = \qty(H_0 + \frac{1}{m} \bvec A \vdot \bvec p) \Psi^V
\end{equation}
这种规范叫做\textbf{速度规范}.

\subsection{长度规范}

\begin{equation}
\Psi^L(\bvec r, t) =  \exp(\I \chi) \Psi(\bvec r, t)
\end{equation}
\begin{equation}
\chi = \bvec A \vdot \bvec r
\end{equation}
得薛定谔方程为
\begin{equation}
\I \pdv{t} \Psi^L = [H_0 + \bvec{\mathcal{E}} \vdot \bvec r] \Psi^L
\end{equation}
这种规范叫做\textbf{长度规范}.
