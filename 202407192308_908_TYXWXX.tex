% 太阳系外行星
% license CCBYSA3
% type Wiki

(本文根据 CC-BY-SA 协议转载自原搜狗科学百科对英文维基百科的翻译)

太阳系外行星,[1]又称系外行星,是位于太阳系之外的行星。太阳系外行星的第一个证据是在1917年发现的,但是并没有被验证。 第一次对太阳系外行星的科学探测是在1988年,这颗行星在2013年被确认为系外行星。第一次确定的探测发生在1992年。截止至2019年12月,总共有在2971个星系中的3976个行星被确认,其中653个星系有多于一个的行星。

探测系外行星的方法有许多。目前,通过过境测光法和多普勒光谱学法发现的行星最多,但这些方法有明显的观测偏差,更偏向于探测恒星附近的行星。因此,85\%探测到的外行星位于潮汐锁定区内。[2]在一些情况下,一颗恒星附近可以观察到多个行星。大约每五个类太阳恒星 中,有一个是“地球大小的”可居住区的行星。[3]假设银河系中有两千亿颗恒星,那么就有一百一十亿颗潜在的可居住的地球大小的行星,如果计算中包括围绕众多红矮星运行的行星,这个数字将达到四百亿。[4]

已知质量最小的行星是德拉格尔(也称为PSR B1257+12 A或PSR B1257+12 b),它的质量大约是月球质量的两倍。美国宇航局太阳系外行星档案中最大的行星在公元前2562年被记载,[5][6]它的质量大约是木星质量的30倍,尽管根据某些行星的定义(基于氘的核聚变[7]),它太大所以不可能是一颗行星,而是一颗褐矮星。有些行星离它们的恒星非常近,以至于它们只需要几个小时就能绕轨道运行,还有一些行星离它们非常远,以至于它们需要几千年才能绕轨道运行。有些距离恒星太远的行星,很难判断它们是否还受引力束缚。迄今为止,几乎所有被探测到的行星都在银河系内。尽管如此,有证据表明,银河系外的行星,即银河系以外更远的星系中的系外行星,可能存在。[8][9]最近的外行星是比邻星b,距地球4.2光年(1.3秒差距),它围绕着离太阳最近的恒星比邻星运行。[10]

太阳系外行星的发现增强了人们对寻找外星生命的兴趣。人们对运行在恒星的可居住区周围的行星特别感兴趣,因为这些区域可能存在液态水,而液态水是地球表面生命存在的先决条件。行星的可居住性研究还参考了其他一系列因素,来决定行星是否适合容纳生命。[11]

除了系外行星,宇宙中还存在不围绕任何恒星运行的流浪行星。这些行星往往被认为是一个独立的类别,尤其如果它们是气态巨行星,在这种情况下,它们通常被视为亚棕矮星,如WISE 0855-0714。[12]银河系中的流浪行星可能有数十亿颗,甚至更多。[13][14]

\subsection{ 命名系统}
\begin{figure}[ht]
\centering
\includegraphics[width=6cm]{./figures/8f5d7050f19a2955.png}
\caption{系外行星 HIP 65426b 是第一个在恒星 HIP 65426附近被发现的行星.[1]} \label{fig_TYXWXX_1}
\end{figure}
《指定系外行星公约》是国际天文学联盟(IAU)采用的命名多星系统的延伸。对于围绕一颗恒星运行的系外行星来说,这个名称通常是通过取其母星的名字,或者更常见的是,加上一个小写字母来形成的。[15]星系中发现的第一颗行星被命名为“$b$”(母星被认为是“$a$”),之后的行星被命名为后续字母。如果同一系统中的几颗行星同时被发现,离恒星最近的一颗会得到下一个字母,然后是按轨道大小顺序来排列的其他行星。临时的IAU认可的标准存在,以适应环双星的命名。少数系外行星有IAU认可的专有名称。除此之外还有其他命名系统。

\subsection{探测历史}
几个世纪以来,科学家、哲学家和科幻作家都怀疑太阳系外行星的存在,[16]但是没有办法探测到它们,或者知道它们的频率,也无法了解到它们与太阳系的行星有多相似。天文学家否决了十九世纪提出的各种探测主张。早在1917年就发现了太阳系外行星(范·马嫩2号)的第一个证据,但是并没有被验证。[17]在1988年,太阳系外行星疑似首次被科学家探测到。不久之后,首个探测在1992年被确认,同时发现了几颗围绕脉冲星PSR B1257+12运行的地球质量的行星。[17]围绕主序列恒星运行的系外行星的首次确认是在1995年,当时发现了一颗巨大的行星,围绕在51 Pegasi星的四天轨道附近。一些系外行星已经通过望远镜直接成像,但是绝大多数行星是被间接地探测到的,例如利用过境测光法和径向速度法。在2018年2月,研究人员利用钱德拉x光天文台,结合名为微透镜的行星探测技术,发现了遥远星系中行星存在的证据,称“其中一些系外行星相对而言和月球一样小,而另一些则和木星一样大。与地球不同,大多数系外行星与恒星没有紧密的联系,所以它们实际上是在太空中漫游或者在恒星之间松散地轨道运行。我们可以估计这个遥远的星系中的行星数量超过一万亿颗。[18]
\subsubsection{2.1 早期推测}