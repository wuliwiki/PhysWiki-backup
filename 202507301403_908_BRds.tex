% 布尔代数(综述)
% license CCBYNCSA3
% type Wiki

本文根据 CC-BY-SA 协议转载翻译自维基百科\href{https://en.wikipedia.org/wiki/Boolean_algebra_(structure)}{相关文章}。

在抽象代数中,布尔代数或布尔格是一个带补的分配格。这种代数结构刻画了集合运算和逻辑运算的基本性质。布尔代数可以被看作是幂集代数或集合域的推广,或者其元素可以被看作是广义真值。它也是德摩根代数和克莱尼代数的一个特殊情形。

每一个布尔代数都可以产生一个布尔环,反之亦然,其中环的乘法对应于合取或交运算∧,环的加法对应于异或或对称差(不是析取 ∨)。然而,布尔环理论在两个运算之间本质上是不对称的,而布尔代数的公理和定理则通过对偶性原理体现出理论的对称性。[1]
\subsection{历史}
\begin{figure}[ht]
\centering
\includegraphics[width=6cm]{./figures/679baac505e439e9.png}
\caption{} \label{fig_BRds_1}
\end{figure}