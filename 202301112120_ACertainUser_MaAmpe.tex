% 有磁介质时的安培环路定律
% 磁感应强度|磁场|环路定理|磁场强度|磁化强度|磁介质|相对磁导率

\pentry{安培环路定律\upref{AmpLaw},磁化强度\upref{MaInte}}

当电流的磁场中有磁介质存在时, 空间任一点的磁感应强度 $\mathbf B $ 等于导线中传导电流所激发的磁场与磁介质磁化后磁化电流所激发的附加磁场的矢量和,这时安培环路定律应写成
\begin{equation} \label{MaAmpe_eq1}
\oint \mathbf{B} \cdot \mathrm{d} \mathbf{l}=\mu_{0}\left(\sum I+I_{s}\right)
\end{equation}

等式右边的两项电流是穿过以回路为边界的任一曲面的总电流,即传导电流 $\sum I$ 和磁化电流 $I_s$ 的代数和。一般说来, $I $ 是可以测量的,可认为它是已知的;而 $I_s$ 不能事先给定,也无法直接测量,它依赖于介质磁化的情况,而介质的磁化情况又依赖于磁介质中的磁感应强度 $\mathbf B$,因此直接求解方程\autoref{MaAmpe_eq1} 很复杂。为了解决这一困难,我们利用关系\autoref{MaInte_eq1}~\upref{MaInte}将\autoref{MaAmpe_eq1} 改写成
\begin{equation}
\oint \mathbf{B} \cdot \mathrm{d} \mathbf{l}=\mu_{0}\left(\sum I+\oint \mathbf{M} \cdot \mathrm{d} \mathbf{l}\right)
\end{equation}
或
\begin{equation} 
\oint\left(\frac{\mathbf{B}}{\mu_{0}}-\mathbf{M}\right) \cdot \mathrm{d} \mathbf{l}=\sum I
\end{equation}
然后引进一个新的物理量,称为\textbf{磁场强度(magnetic intensity)},用符号 $\mathbf H$ 表示,定义为
\begin{equation} \label{MaAmpe_eq2}
\mathbf{H}=\frac{\mathbf{B}}{\mathbf{\mu}_{0}}-\mathbf{M}
\end{equation}
这样,便有下列简单的形式: 
\begin{equation} \label{MaAmpe_eq3}
\oint \mathbf{H} \cdot \mathrm{d} \mathbf{l}=\sum I
\end{equation}

\autoref{MaAmpe_eq3} 称为有磁介质时的安培环路定律。它表明 $\mathbf H $ 矢量的环流只和传导电流 $I $ 有关,而在形式上与磁介质的磁性无关。因此引入磁场强度 $\mathbf H $ 这个物理量以后,在磁场分布具有高度对称性时,能够使我们比较方便地处理有磁介质时的磁场问题,安培环路定律和静磁场的另一普遍规律一磁场中的高斯定律一起,是处理静磁场问题的基本定理。

在国际单位制中,$\mathbf  H $ 的单位是 $\rm A/m$。

\autoref{MaAmpe_eq2} 表示了磁介质中任一点处磁感应强度 $\mathbf B$、磁场强度 $\mathbf H$ 和磁化强度 $\mathbf M $ 之间的普遍关系,不论磁介质是否均匀,甚至对铁磁性物质都能适用。但是,磁化强度 $\mathbf M $ 不仅和磁介质的性质有关,也和磁介质所在处的磁场有关。我们定义
\begin{equation} \label{MaAmpe_eq4}
\chi_{m}=\frac{M}{H}
\end{equation}
为磁介质的\textbf{磁化率(magnetic susceptibility)},是一与磁介质的性质有关的物理量。因为 $\mathbf M $ 和 $\mathbf H $ 所用的单位相同,所以磁化率 $\chi_m$, 是单位为 $1 $ 的量。如果磁介质是均匀的,则 $\chi_m$ 是常量;如果磁介质是不均匀的,则 $\chi_m$ 是空间位置的函数。对于顺磁质,$\chi_m>0$,磁化强度 $\mathbf M $ 和磁场强度 $\mathbf H $ 的方向相同;对于抗磁质,$\chi_m<0$,磁化强度 $\mathbf M $ 和磁场强度 $\mathbf H $ 的方向相反。\autoref{MaAmpe_eq4} 又可写为:
\begin{equation} \label{MaAmpe_eq5}
\mathbf M = \chi_m\mathbf H
\end{equation}
将其代入\autoref{MaAmpe_eq2} 中可解得
\begin{equation}
\mathbf{B}=\mathbf{\mu}_{0} \mathbf{H}+\mu_{0} \mathbf{M}=\mu_{0}\left(1+\chi_{{m}}\right) \mathbf{H}
\end{equation}

通常称
\begin{equation}
\mu_{{r}}=1+\chi_{m}
\end{equation}
为该介质的\textbf{相对磁导率(relative magnetic permeability)}。于是\autoref{MaAmpe_eq5} 就可以写为:
\begin{equation}
\mathbf B=\mu_{0} \mu_{\mathrm{r}} \mathbf H\overset{\triangle}=\mu \mathbf H
\end{equation}
其中 $\mu = \mu_{0} \mu_{\mathrm{r}} $ 成为磁介质的\textbf{磁导率(magnetic permeability)}。对于真空中的磁场来说,由于 $\mathbf M=0$,所以“真空”的 $\mu_r=1,\chi_m=0$。真空中各点处的磁场强度 $\mathbf H $ 等于该点磁感应强度 $B $ 的 $1/\mu_0$ 倍,即 $\mathbf H = \mathbf B / \mu_0$。