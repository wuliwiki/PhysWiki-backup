% 尼尔斯·阿贝尔(综述)
% license CCBYSA3
% type Wiki

本文根据 CC-BY-SA 协议转载翻译自维基百科 \href{https://en.wikipedia.org/wiki/Niels_Henrik_Abel}{相关文章}。

尼尔斯·亨利克·阿贝尔(Niels Henrik Abel,/ˈɑːbəl/ AH-bəl,挪威语发音:[ˌnɪls ˈhɛ̀nːɾɪk ˈɑ̀ːbl̩],1802年8月5日-1829年4月6日)是挪威著名数学家,在多个数学领域做出了开创性的贡献。[1] 他最著名的成果是第一个完整地证明了一般五次方程无法用根式求解的定理。这个问题曾是当时最重要的未解难题之一,悬而未决长达250多年。[2] 此外,他还是椭圆函数领域的创新者,并发现了阿贝尔函数。他是在贫困中取得这些发现的,26岁时因肺结核早逝。

他的大部分研究成果都集中在六七年间完成。[3] 法国数学家查尔斯·埃尔米特曾评价说:“阿贝尔留给数学家的东西,足够他们研究五百年。”[3][4] 另一位法国数学家阿德里安-玛丽·勒让德则说:“这个年轻的挪威人真是个天才!”[5]
