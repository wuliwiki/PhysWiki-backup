% 双摆的数值计算(Matlab)

\begin{issues}
\issueDraft
\end{issues}

\pentry{双摆和三摆\upref{Pendu3}}

\begin{figure}[ht]
\centering
\includegraphics[width=10cm]{./figures/DbPend_1.png}
\caption{运行结果, 蓝色曲线显示双摆末端的轨迹.} \label{DbPend_fig1}
\end{figure}

\begin{lstlisting}[language=matlab]
% 双摆运动
function double_pendulum
% === 参数设置 ===
m1 = 1; l1 = 1; % 摆 1
m2 = 1; l2 = 1; % 摆 2
th10 = pi; th20 = pi; % 初始角度
w10 = -4; w20 = 6; % 初始角速度
g = 9.8; % 重力加速度
tmin = 0; tmax = 10; Nt = 200; % 时间网格
Nsmooth = 4; % 轨迹平滑(每两帧之间轨迹的线段数)
% ==============
close all;
Y0 = [th10; th20; w10; w20];
options = odeset('RelTol',1e-6);
[T, Y] = ode45(@(t,Y)odefun(t, Y, m1, m2, l1, l2, g), ...
        [tmin,tmax], Y0, options);
Th1 = Y(:,1); Th2 = Y(:,2); W1 = Y(:,3); W2 = Y(:,4);
Ek = 0.5*m1*(l1*W1).^2 + 0.5*m2*((l1*W1.*cos(Th1)+l2*W2.*cos(Th2)).^2 ...
    + (l1*W1.*sin(Th1) + l2*W2.*sin(Th2)).^2);
V = -m1*g*l1*cos(Th1) - m2*g*(l1*cos(Th1) + l2*cos(Th2));
figure; plot(T, Ek + V); xlabel time; title 'total energy';
t = linspace(tmin, tmax, Nt);
tt = linspace(tmin, tmax, Nsmooth*Nt-3);
th1 = interp1(T, Th1, t, 'spline');
thth1 = interp1(T, Th1, tt, 'spline');
th2 = interp1(T, Th2, t, 'spline');
thth2 = interp1(T, Th2, tt, 'spline');
x1 = l1*sin(th1); y1 = -l1*cos(th1);
xx1 = l1*sin(thth1); yy1 = -l1*cos(thth1);
x2 = x1 + l2*sin(th2); y2 = y1 - l2*cos(th2);
xx2 = xx1 + l2*sin(thth2); yy2 = yy1 - l2*cos(thth2);
figure;
for it = 1:Nt
    clf;
    plot([0, x1(it)], [0, y1(it)], 'k'); hold on;
    scatter(x1(it), y1(it), 'k');
    plot([x1(it), x2(it)], [y1(it), y2(it)], 'k');
    scatter(x2(it), y2(it), 'k');
    plot(xx2(1:Nsmooth*it-3), yy2(1:Nsmooth*it-3));
    axis equal; axis([-1,1,-1,1]*(l1+l2)*1.1);
    title(['t = ' num2str(t(it), '%.2f')]);
    xlabel x; ylabel y;
    saveas(gcf, [num2str(it) '.png']);
end
end

function dY = odefun(~, Y, m1, m2, l1, l2, g)
th1 = Y(1); th2 = Y(2); w1 = Y(3); w2 = Y(4);
alpha = th2 - th1;
dw1 = (m2*l2*sin(alpha)*w2^2 - (m1+m2)*g*sin(th1) ...
    + m2*(l1*sin(alpha)*w1^2+g*sin(th2))*cos(alpha))...
    /(m1*l1+m2*l1*sin(alpha)^2);
dw2 = (((m1+m2)*g*sin(th1)-m2*l2*sin(alpha)*w2^2)*cos(alpha)...
    - (m1+m2)*(l1*sin(alpha)*w1^2+g*sin(th2)))...
    /(m1*l2 + m2*l2*sin(alpha)^2);
dY = [w1; w2; dw1; dw2];
end
\end{lstlisting}
