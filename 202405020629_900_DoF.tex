% 自由度
% keys 自由度|独立变量|约束
% license Xiao
% type Tutor

\pentry{柱坐标系\nref{nod_Cylin}, 球坐标系\nref{nod_Sph}}{nod_6a11}

我们已知描述一个质点在三维空间中的位置需要至少三个变量, 无论使用直角坐标, 柱坐标或球坐标。 我们说这个质点有三个\textbf{自由度(degrees of freedom)}。 同理, 若空间中有 $N$ 个质点可以自由移动, 我们就说这些质点组成的系统有 $3N$ 个自由度。

需要注意的是, 以上讨论的粒子都是自由的, 即没有\textbf{约束}。 若给系统施加约束, 自由度就会相应减少。 例如, 若用长度为 $R$ 的细棒将一个质点和坐标原点相连, 则质点只能够在半径为 $R$ 的球面上运动, 这时我们在直角坐标系中施加了约束条件
\begin{equation}\label{eq_DoF_1}
x^2 + y^2 + z^2 = R^2~.
\end{equation}
当我们确定了 $x, y, z$ 中的任意两个变量后, 第三个变量就可以根据上式确定, 所以这时质点就只有两个自由度。 我们把选取的两个自由坐标叫做\textbf{独立(independent)变量}, 剩下的一个坐标就是\textbf{非独立} 的\footnote{注意这个例子中第三个变量可能会有正负两个解, 但我们仍然认为它是非独立的。}。 另一种更直观的方法是把这个质点的位置用球坐标 $(r, \theta, \phi)$ 表示, 约束条件变为
\begin{equation}\label{eq_DoF_2}
r = R~,
\end{equation}
我们马上就得到两个独立的变量 $(\theta, \phi)$ 用于完全确定质点的位置。 我们把\autoref{eq_DoF_1} 或\autoref{eq_DoF_2} 这样的等式称为一个约束条件, 若给自由度为 $N$ 的系统加上 $M$ 个约束条件, 则系统的自由度减少为 $N - M$。

另一个例子是考虑两个质点, 它们之间由长度为 $R$ 细棒相连, 那么原来的 $6$ 个自由度的系统在一个约束条件 $(x_2 - x_1)^2 + (y_2 - y_1)^2 + (z_2 - z_1)^2 = R^2$ 下变为 $5$ 个自由度。 要具体选取五个独立变量, 我们可以用三个独立变量先确定第一个质点的位置, 然后再用剩下两个独立变量加上约束条件确定第二个质点的位置。 我们还可以再继续加约束条件, 例如限制第一个质点只能在 $z$ 轴上运动, 这事实上是两个约束条件 $x_1 = 0$ 和 $y_1 = 0$, 使两个质点的自由度最终减少为 3 个。 我们可以取一个独立变量描述第一个质点在 $z$ 轴上的坐标, 另外两个独立变量描述第二个质点相对于第一个质点的方向。

% \addTODO{轮子的平面运动:3 个自由度; 轮子在地上滚: 2 个自由度; 两个齿轮分别转动: 2  个; 咬合在一起: 1 个;}
