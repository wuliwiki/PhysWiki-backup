% 洛伦兹变换
% 洛伦兹变换|变换矩阵|惯性系

\pentry{时间的变换\upref{SRtime}}
\subsection{洛伦兹变换}

在时间的变换与钟慢效应\upref{SRtime}一节中,我们自然而然地推导出了一维空间中的洛伦兹变换:

\begin{equation}\label{SRLrtz_eq1}
\leftgroup{
&x_2 = \frac{x_1 - vt_1}{\sqrt{1 - v^2/c^2}}\\
&t_2 = \frac{t_1 - vx_1/c^2}{\sqrt{1 - v^2/c^2}}
}
\end{equation}

当然,由于没有哪个惯性系更特殊,考虑到 $K_1$ 相对 $K_2$ 的速度是 $-v$,我们有:

\begin{equation}\label{SRLrtz_eq2}
\leftgroup{
&x_1 = \frac{x_2 + vt_2}{\sqrt{1 - v^2/c^2}}\\
&t_1 = \frac{t_2 + vx_2/c^2}{\sqrt{1 - v^2/c^2}}
}
\end{equation}

当然了,这一逆变换也是可以手动从\autoref{SRLrtz_eq1} 中解出来的.

\begin{example}{垂直方向上的洛伦兹变换}\label{SRLrtz_ex1}

依然考虑火车模型.现在在火车和铁轨的原点都树一根杆子,从铁轨系看来两根杆子高度相同.请说明为什么从火车系看来两根杆子高度依然相同.(提示:竖直的杆子和水平的杆子本质区别是什么?利用同时性的相对性说明一下.你可能需要考虑“两个事件的同时性不是绝对的”以及“同一个事件在任何参考系都是同一个事件”.)

\end{example}

\autoref{SRLrtz_ex1} 的结果说明,尺缩效应和同时性的相对性只发生在火车运动的方向上.这就是说,垂直于火车的坐标轴不会因为火车的运动而发生变化.这就引出了完整版的三维空间中——或者说四维时空中的洛伦兹变换:

\begin{equation}\label{SRLrtz_eq3}
\leftgroup{
&x' = \frac{x - vt}{\sqrt{1 - v^2/c^2}}\\
&y'= y\\
&z' = z\\
&t' = \frac{t - vx/c^2}{\sqrt{1 - v^2/c^2}}
}
\qquad
\leftgroup{
&x = \frac{x' + vt'}{\sqrt{1 - v^2/c^2}}\\
&y = y'\\
&z = z'\\
&t = \frac{t' + vx'/c^2}{\sqrt{1 - v^2/c^2}}
}
\end{equation}

这里我们应用了多数资料中使用的字母习惯,避免读者造成混淆.在上式中,$x_1=x$,$t_1=t$,$x_2=x'$,$t_2=t'$.

\subsection{矩阵表示}

洛伦兹变换也可以用矩阵表示为:
\begin{equation}\label{SRLrtz_eq4}
L=
\left[\begin{matrix}
\frac{1}{\sqrt{1-v^2/c^2}}& -\frac{v/c^2}{\sqrt{1-v^2/c^2}}& 0& 0\\
-\frac{v}{\sqrt{1-v^2/c^2}}& \frac{1}{\sqrt{1-v^2/c^2}}& 0& 0\\
0&0&1&0\\
0&0&0&1
\end{matrix}\right]
\end{equation}

将 $L$ 称为\textbf{洛伦兹矩阵}或\textbf{洛伦兹变换矩阵}.

如果一个事件在 $K_1$ 和 $K_2$ 中的坐标分别是 $\left(\begin{matrix}   t\\x\\y\\z    \end{matrix}\right)$ 和 $\left(\begin{matrix}   t'\\x'\\y'\\z'    \end{matrix}\right)$,那么有

\begin{equation}
\left(\begin{matrix}   t'\\x'\\y'\\z'    \end{matrix}\right)
=
L
\left(\begin{matrix}   t\\x\\y\\z    \end{matrix}\right)
\end{equation}

为了使洛伦兹变换矩阵更具有对称性,我们用 $ct$ 代替 $t$,这样 $ct,x,y,z$ 的量纲就是相同的了.\autoref{SRLrtz_eq1} 变成了以下形式:

\begin{equation}
\leftgroup{
&x_2 = \frac{x_1 - vct_1/c}{\sqrt{1 - v^2/c^2}}\\
&ct_2 = \frac{ct_1 - vx_1/c}{\sqrt{1 - v^2/c^2}}
}
\end{equation}
此时变换矩阵为(设 $\gamma=1/\sqrt{1-v^2/c^2},\beta=v/c$)
\begin{equation}\label{SRLrtz_eq5}
\begin{aligned}
L&=
\left[\begin{matrix}
\frac{1}{\sqrt{1-v^2/c^2}}& -\frac{v/c}{\sqrt{1-v^2/c^2}}& 0& 0\\
-\frac{v/c}{\sqrt{1-v^2/c^2}}& \frac{1}{\sqrt{1-v^2/c^2}}& 0& 0\\
0&0&1&0\\
0&0&0&1
\end{matrix}\right]\\
&=
\left[\begin{matrix}
\gamma& -\gamma\beta& 0& 0\\
-\gamma\beta& \gamma& 0& 0\\
0&0&1&0\\
0&0&0&1
\end{matrix}\right]
\end{aligned}
\end{equation}

如果将时空坐标写成\textbf{四矢量}:$x^{\mu}=(ct,x,y,z)$,洛伦兹变换就可以写成协变形式(为了避免指标混乱,下面记 $(ct',x',y',z')$ 为 $S'$ 系中的时空坐标,$(ct,x,y,z)$ 为 $S$ 系中的时空坐标):
\begin{equation}
x'^{\mu}={L^\mu}_\nu x^\nu
\end{equation}
其中 ${L^\mu}_\nu$ 对应矩阵\autoref{SRLrtz_eq5} 中 $\mu$ 行 $\nu$ 列的元素.
\subsection{任意方向的洛伦兹变换}
设参考系 $S'$ 以速度 $\bvec v$ 相对于参考系 $S$ 运动.
那么 $S$ 参考系中时空坐标 $(t,\bvec r)$ 在 $S'$ 参考系中的坐标为 $(t',\bvec r')$.可以将 $\bvec r$ 和 $\bvec r'$ 平行于 $\bvec v$ 的分量和垂直于 $\bvec v$ 的分量分别考虑,根据洛伦兹变换,有
\begin{equation}
\begin{aligned}
&\bvec r'_{\perp}=\bvec r_{\perp}\\
&\bvec r'_{\parallel}=\gamma(\bvec r_{\parallel}-\bvec v t)\\
&ct'=\gamma(ct-\beta r_{\parallel})
\end{aligned}
\end{equation}

可以用点乘等矢量运算表示出 $\bvec r_\parallel$ 和 $\bvec r_\perp$:
\begin{equation}
\bvec r_{\parallel}=\frac{(\bvec r\cdot \bvec v)\bvec v}{v^2},\bvec r_{\perp}=\bvec r-\frac{(\bvec r\cdot \bvec v)\bvec v}{v^2}
\end{equation}

因此洛伦兹变换公式可以改写成
\begin{equation}
\begin{aligned}
&\bvec r'=\bvec r-\gamma\bvec v t+(\gamma-1)\frac{(\bvec r\cdot \bvec v)\bvec v}{v^2}\\
&ct'=\gamma ct-\gamma\frac{\bvec r\cdot\bvec v}{c}
\end{aligned}
\end{equation}
相应的洛伦兹矩阵为(以 $(ct,\bvec r)$ 为时空坐标)
\begin{equation}\label{SRLrtz_eq6}
\begin{aligned}
{L^\mu}_\nu&=
\left[\begin{matrix}
\gamma& \gamma v_1/c & \gamma v_2/c& \gamma v_3/c\\
\gamma v_1/c&1+(\gamma-1)v_1v_1/v^2& (\gamma-1)v_2v_1/v^2&(\gamma-1)v_3v_1/v^2\\
\gamma v_2/c &(\gamma-1)v_1v_2/v^2&1+(\gamma-1)v_2v_2/v^2&(\gamma-1)v_3v_2/v^2)\\
\gamma v_3/c&(\gamma-1)v_1v_3/v^2)&(\gamma-1)v_3v_2/v^2&1+(\gamma-1)v_3v_3/v^2
\end{matrix}\right]\\
&=\left[\begin{matrix}
\gamma & \gamma v_j/c \\
\gamma v_i/c & \delta_{ij}+(\gamma-1)v_iv_j/v^2
\end{matrix}
\right]
\end{aligned}
\end{equation}

对于两个四矢量 $p^\mu,q^\mu$,可以构造洛伦兹不变量 $p^\mu q_\mu=\eta_{\mu,\nu}p^\mu q^\nu$,代表的意义是 $p$ 和 $q$ 的“点乘”,只不过是闵可夫斯基空间度规下的点乘.以 $x^\mu$ 为例,$\sqrt{-x^\mu x_\mu}=\sqrt{c^2t^2-x^2-y^2-z^2}$ 是洛伦兹不变量,其物理意义是原点到 $x^\mu$ 的时空距离.由于 $\eta_{\mu\nu}p^\mu q^\nu$ 在洛伦兹变换下是保持不变的,所以有
\begin{equation}
\begin{aligned}
\eta_{\mu\nu}p'^\mu q'^\nu&=\eta_{\mu\nu}({L^\mu}_\rho p^\rho) ({L^\nu}_\sigma q^\sigma)\\
&=\eta_{\mu\nu}{L^\mu}_\rho {L^\nu}_\sigma p^\rho q^\sigma
\\
&=\eta_{\rho\sigma} p^\rho q^\sigma
\end{aligned}
\end{equation}
上式对任意的四矢量 $p^\mu,q^\mu$ 都成立,这体现了洛伦兹变换的一个性质:
\begin{equation}\label{SRLrtz_eq7}
\eta_{\mu\nu}{L^\mu}_\rho {L^\nu}_\sigma=\eta_{\rho\sigma}
\end{equation}
即洛伦兹变换是保度规的.

我们上面讨论的洛伦兹变换是不含转动的,仅仅涉及到 boost \footnote{一些参考书将其译为“推促”,表示一个参考系以速度 $\bvec v$ 相对于另一参考系运动.}.可以对洛伦兹变换进行推广.一个\textbf{普通的}洛伦兹变换需要满足的唯一要求是 \autoref{SRLrtz_eq7}.这样的变换除了 boost 以外,还包括三维空间的转动(当然也包括单位元).所有这样的变换构成一个群,我们把它称为\textbf{洛伦兹群}\upref{qed1}.
