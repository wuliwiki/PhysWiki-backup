% 量子力学中的基本算符
% 动量算符|角动量算符|哈密顿算符|哈密顿量|能量算符|生成元
\addTODO{加入目录.}

\pentry{经典力学,量子力学的基本原理(量子力学)\upref{QMPrcp}}

本文中,$\hbar=c=1$.

在\textbf{量子力学的基本原理(量子力学)}的\autoref{QMPrcp_ex1}~\upref{QMPrcp} 和\autoref{QMPrcp_ex2} 中,我们不加证明地给出了动量、能量(哈密顿)、角动量算符在给定表象下的形式,相当于进行了定义.本文将讨论如何从经典力学中导出这几个算符的定义.

\subsection{无穷小算符}

考虑无穷小算符总是有益的,因为微分线性近似,而线性的东西很简单.

如果要求一个算符在时间趋于$0$的时候趋于恒等算符,即连续性,那么对于\textbf{无穷小}自变量$\varepsilon$后,这个算符总可以写为
\begin{equation}\label{OprQM_eq1}
U(\varepsilon) = 1-\I G\varepsilon
\end{equation}
我们称$G$是$U$的生成元.

当$G$是一个厄米算符时,有
\begin{equation}
U^{\dagger}(\varepsilon)U(\varepsilon) = (1-\I G\varepsilon)(1+\I G\varepsilon) = 1+o(\varepsilon)
\end{equation}
其中$o(\varepsilon)$表示比$\varepsilon$更高阶的无穷小.

因此,$G$是厄米算符$\implies$ $U(\varepsilon)$是\textbf{幺正}算符.

量子态归一化要求可观测量是幺正的,因此可观测量的生成元应该是厄米算符.



\subsection{位置算符}

位置算符的本征矢都是位置精确给定的态,本征值即对应的位置,因此位置算符是$x_0$或$\bvec{x}_0$,即空间坐标.

\textbf{位置表象}下,位置算符的表示为
\begin{equation}
\leftgroup{
    \bra{x_0}x_0 &= x_0\qquad\text{一维情况}\\
    \bra{\bvec{x}_0}\bvec{x}_0&=\bvec{x}_0\qquad\text{三维情况}
    }
\end{equation}

% 处于$x_0$或$\pmat{x_0, y_0, z_0}$的位置本征矢,在\textbf{位置表象下}的波函数为$\delta(x-x_0)$或$\delta(x-x_0)\delta(y-y_0)\delta(z-z_0)$.

\addTODO{动量表象下的位置算符由傅里叶变换给出.}





\subsection{动量算符}

\pentry{平移算符\upref{tranOp}}

由经典力学,动量是平移生成元,\autoref{OprQM_eq1} 中的$G$应是动量算符.

显然,\textbf{动量表象}下的动量算符就是$p$或$\bvec{p}$,这和位置算符的情况一致.

又由\autoref{tranOp_eq1}~\upref{tranOp},可知\textbf{位置表象}下一维无穷小平移算符为
\begin{equation}
\bra{x}P(\dd x) = \exp(-\dd x \cdot \partial_x) = 1-\dd x\partial_x
\end{equation}

代回\autoref{OprQM_eq1} ,注意$P(\dd x)$相当于$U(\varepsilon)$,即可得到一维动量算符:
\begin{equation}
p  = -\I\partial_x
\end{equation}

三维情况类似可得
\begin{equation}
\uvec{p} = -\I\nabla
\end{equation}




\subsection{哈密顿算符}

哈密顿算符$H$即能量算符,较为特殊,取决于我们希望将量子力学应用在什么背景下.

对于经典量子力学,我们考虑的背景是经典力学,因此用经典的能量关系
\begin{equation}
E=\frac{p^2}{2m}+V
\end{equation}
来定义哈密顿算子,此时有
\begin{equation}
\bra{x}H = \leftgroup{
    &\frac{-\partial_x^2}{2m}+V\qquad\text{一维情况}\\
    &\frac{-\nabla^2}{2m}+V\qquad\text{三维情况}
}
\end{equation}

如果使用狭义相对论的能量关系
\begin{equation}
E^2 = p^2+m^2
\end{equation}
那么可以得到哈密顿算子的平方
\begin{equation}\label{OprQM_eq2}
\bra{x}H^2 = \leftgroup{
    -\partial_x^2&+m^2\qquad\text{一维情况}\\
    -\nabla^2&+m^2\qquad\text{三维情况}
}
\end{equation}
由此可以把薛定谔方程调整为\textbf{Klein-Gordon 方程}\upref{KGeq}.

Klein-Gordon方程是平方形式,并不协变,且有负能量和负概率密度的问题\footnote{一说负概率密度可以解释为粒子的荷,如电荷.},因此狄拉克(Dirac)想办法把\autoref{OprQM_eq2} 取了平方根,得到\textbf{狄拉克方程}\upref{qed4}.



\subsection{角动量算符}

角动量算符和经典力学有所不同.经典力学中的角动量对应的是量子力学中的\textbf{空间角动量}算符,是空间转动的生成元.但量子力学中还有\textbf{自旋角动量}的概念,是自旋态空间中的转动的生成元.自旋为$1/2$的系统使用一个二维态空间,自旋为$1$的系统使用一个三维态空间,诸如此类.

设




















