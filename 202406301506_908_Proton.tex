% 质子
% license CCBYSA3
% type Wiki

(本文根据 CC-BY-SA 协议转载自原搜狗科学百科对英文维基百科的翻译)

\textbf{质子}属于亚原子粒子,符号为p或者$p^+$,携带为一个基本电荷,其质量略小于中子。质子和中子的质量约为一个原子质量,它们都被称为“核子”。

每个原子的原子核内均存在一个或多个质子,它们是核子的必要组成部分。原子核中质子的数量被称为原子序数(由符号Z表示)。因为每种元素的质子数量不一样,因此每种元素都有自己独特的原子序数。

1917年,欧内斯特·卢瑟福做实验发现,使用$\alpha$粒子撞击氮原子核,可以提取氢原子核。卢瑟福因此推断,氢原子核是氮原子核与所有更重的原子核的基础材料。由于这重要结果,卢瑟福被公认为质子的发现者。

在粒子物理学的现代标准模型里,质子是由两个上夸克与一个下夸克组成的强子。夸克的静质量只贡献出大约1\%质子质量,剩余的质子质量主要源自于夸克的动能与捆绑夸克的胶子场的能量。

因为质子是由三个夸克组成,质子可视为基本粒子,质子具有物理尺寸,但这尺寸并不能完美良好定义,由于质子的表面很模糊,因为这表面是由作用力的影响来定义,而这作用力不会突然终止。质子的半径(更仔细地说,电荷半径)大约为0.84到0.87飞米即$0.84\times10^{-15}$到$0.87\times10^{-15}$ m。

在足够低的温度下,自由质子将和电子结合。然而,这种结合并没有改变质子的性质。当一个质子快速移动通过物质时,它会和电子以及原子核相互作用,进而速度变慢,直至被原子的电子云俘获。因此质子化原子将被产生,它是氢的化合物。当真空中存在自由电子时,足够慢的质子可以和单个自由电子结合,成为中性的氢原子,其在化学上是自由基。在足够低的能量下,这种“自由氢原子”倾向于与许多其他类型的原子发生化学反应。当自由氢原子相互反应时,它们形成中性氢分子$(H_2)$。因此氢分子是星际空间中分子云最常见的分子成分。

\subsection{物理性质}
质子是自旋为-½的费米子,由三个价夸克组成,[1]因此是一种重子(强子的子类型)。两个上夸克和一个下夸克由强力和胶子共同作用而结合在一起.[2]现代物理学观点认为质子由价夸克(上、上、下)、胶子和短暂的海夸克对组成。质子具有近似指数衰减的正电荷分布,均方差半径约为0.8飞米。[3]

质子和中子都是核子,它们可以通过核力结合在一起形成原子核。氢原子最常见的同位素(化学符号为“H”)是一个孤质子。氢原子具有两个重氢同位素,分别为氘和氚:其中氘的原子核包含一个质子和一个中子,氚的原子核包含一个质子和两个中子。其他类型的原子核由两个或多个质子以及不同数量的中子组成。

\subsection{历史}
类氢粒子作为其他原子组成部分的概念是经过长时间发展起来的。早在1815年,基于早期原子量的数值,威廉·普鲁特认为所有原子均由氢原子组成(他称之为“原生系”)。当类氢粒子的原子量被更精确测量时,他的解释被推翻。[4]

1886年,欧根·戈尔德斯坦发现极隧射线(也称为阳极射线)并证实它们是从气体中产生的带正电荷的粒子(离子)。然而,由于来自不同气体的粒子具有不同的荷质比(e/m),因此它们不像约瑟夫·汤姆孙发现的电子一样用单个粒子来表示。威廉·维恩于1898年确认氢离子是电离气体中荷质比最高的粒子。[5]
\begin{figure}[ht]
\centering
\includegraphics[width=6cm]{./figures/4aab6b70ce560c2f.png}
\caption{欧内斯特·卢瑟福在第一届索尔维会议,1911} \label{fig_Proton_4}
\end{figure}
在欧内斯特·卢瑟福于1911年发现原子核后,安东尼乌斯·范登布罗伊克提出周期表中每个元素的位置(其原子序数)等于其核子电荷数。亨利·莫塞莱在1913年使用X射线光谱通过实验证实了这一点。
\begin{figure}[ht]
\centering
\includegraphics[width=6cm]{./figures/1fc65365905b6605.png}
\caption{在异丙醇中检测到质子 威尔逊云雾室} \label{fig_Proton_5}
\end{figure}
1917年(实验于1919年被报道),卢瑟福证明了氢核存在于其他原子核中,这一结果通常被描述为质子的发现。[6]卢瑟福早期认识到氢核是α粒子和氮气相互作用的辐射产物,并可以通过它们在空气中独特的穿透特征以及在闪烁探测器中的表现来进行识别。卢瑟福注意到,当α粒子被射入空气(主要是氮气)时,闪烁探测器显示了典型氢核的特征。通过追踪空气中氮的反应,卢瑟福发现当α粒子进入纯氮气时,效果更明显。卢瑟福确定这种氢只能来自氮气,因此氮气必须含有氢核。一个氮核被α粒子撞击后产生了氧核。这是首次被报道的核反应,$^{14}N + \alpha \to ^{17}O + p$ (这种反应后来在1925年的威尔逊云雾室被直接观察到)。

受普鲁特假设的影响,卢瑟福认为氢是结构最简单质量最轻的元素,即氢是所有元素的组成部分。由于发现氢核存在于其他所有原子核中并怀疑氢核只包含一种粒子,卢瑟福因此给氢核一个特殊的名称, 即质子,在希腊语中意为“第一”。然而,卢瑟福也想到了这个词曾被普鲁特所使用的。在1920年8月24日举行的英国科学促进会上,[7]欧里佛·洛兹要求卢瑟福为正氢核取一个新名字,以避免与中性氢原子混淆。卢瑟福使用质子和普劳顿(源自普鲁特)来进行区分的建议被会议接受。 因此氢核被命名为“质子”,取自普鲁特的词“protyle”。[8]科学文献中首次使用“质子”一词是在1920年。[9]

最近的研究表明,雷暴可以产生能量高达几十兆电子伏的质子。[10][11]

质子通常用于加速器质子疗法或者各种各样的粒子物理学实验,最有力的例子是大型强子对撞机。

在2017年7月的一篇论文中,研究人员测量了质子的质量1.007276466583+15−29 atomic mass units(括号中的值分别是统计和系统不确定性),比来自CODATA 2014的测量值低三个标准差。[12][13]

\subsection{稳定性}
自由质子(不与核子或电子结合的质子)是一种稳定的粒子,尚未被观察到自发分解为其他粒子。自由质子在许多情况下都是天然存在的。在这些情况下,能量或温度足够高,可以将它们与电子分开。它们对电子有一定的亲和力。自由质子存在于温度过高以至于不能与电子结合的等离子体中。高能和高速的自由质子占宇宙射线的90\%。在一些罕见类型的放射性衰变中,自由质子可以直接从原子核发射。质子(连同电子和反中微子)也可在不稳定自由中子的放射性衰变中产生。

由于未观察到自由质子的自发衰变,根据标准模型,质子被认为是稳定的粒子。然而,一些粒子物理学的大统一理论 (GUTs)预测,质子衰变的寿命应该在$10^{31}$年至$10^{36}$年间。实验研究已经确定了质子平均寿命的下限以及可能的衰变产物。[14][15][16]

日本的超级神冈探测器通过实验测得,质子衰变成反μ子和中性介子的平均寿命下限为$6.6\times10^{33}$年,衰变为正电子和中性π介子的平均寿命下限为$8.2\times10^{33}$年。[17]另一个实验来自于加拿大的萨德伯里微中子观测站。 这个实验用来检测任何衰变后的产物,通过从氧16质子衰变后的残余核子中检测到γ射线,测得质子寿命的下限为$2.1\times10^{29}$年。[18]

然而,质子也可通过电子俘获转化为中子(称为逆$\beta$衰变)。对于自由质子,这个过程不是自发产生的,而是在有能量提供时才会发生。等式是:
$$p^+ + e^- \to n +\nu_e~$$
这个过程是可逆的;中子可以通过$\beta$衰变转换为质子。$\beta$衰变是放射性衰变的一种常见形式。事实上,自由中子以这种方式衰变,平均寿命约为15分钟。

\subsection{夸克和质子的质量}
在现代核力理论的量子色动力学中,狭义相对论解释了质子和中子的大部分质量。质子的质量大约是它内部所有夸克静止质量之和的80-100倍,同时胶子的静止质量为零。相比在量子色动力学真空中所有夸克的静止能量,质子内同一区域中的夸克和胶子的额外能量几乎占了质子总重量的99\%。因此,质子的静止质量是运动夸克和胶子系统的不变质量。在这样的系统中,无质量粒子的能量作为系统静止质量的一部分也同样能被测量。

两个术语用于指代组成质子的夸克质量:净夸克质量指夸克本身的质量,而组夸克质量指净夸克质量加上其周围胶子粒子场的质量。[19] [20]这两个质量通常具备不同的值。如上所述,质子的大部分质量来自将夸克结合在一起的胶子,而不是夸克本身。虽然胶子没有质量,但它们拥有能量,即量子色动力学结合能,对质子的总质量贡献很大(参见狭义相对论中的质量)。质子的质量约为$938 MeV/c^2$,其中三个价夸克中的静止质量只贡献了大约$9.4 MeV/c^2$,其余的大部分能量来自于胶子的量子色动力学结合能 。[21] [22]

质子的夸克模型波函数组成为
\begin{equation}
\left|p_{\uparrow}\right\rangle = \frac{1}{\sqrt{18}} \left[ 2 \left| u_{\uparrow} d_{\downarrow} u_{\uparrow} \right\\rangle + 2 \left| u_{\uparrow} u_{\uparrow} d_{\downarrow} \right\rangle + 2 \left| d_{\downarrow} u_{\uparrow} u_{\uparrow} \right\rangle - \left| u_{\uparrow} u_{\uparrow} d_{\downarrow} \right\rangle - \left| u_{\uparrow} d_{\downarrow} u_{\uparrow} \right\rangle - \left| u_{\downarrow} d_{\uparrow} u_{\uparrow} \right\rangle - \left| d_{\downarrow} u_{\uparrow} u_{\uparrow} \right\rangle - \left| d_
{\downarrow} u_{\downarrow} u_{\uparrow} \right\rangle - \left| u_
{\uparrow} u_{\downarrow} d_{\downarrow} \right\rangle \right]~
\end{equation}













