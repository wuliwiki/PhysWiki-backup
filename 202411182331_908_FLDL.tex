% 法拉第电磁感应定律(综述)
% license CCBYSA3
% type Wiki

本文根据 CC-BY-SA 协议转载翻译自维基百科\href{https://en.wikipedia.org/wiki/Faraday\%27s_law_of_induction}{相关文章}。

\begin{figure}[ht]
\centering
\includegraphics[width=8cm]{./figures/0ad787ae5c515839.png}
\caption{法拉第的实验展示了线圈之间的感应现象:右侧的液体电池提供电流,该电流流经小线圈(A),从而产生一个磁场。当线圈静止时,大线圈中没有感应电流产生。但当小线圈在大线圈中移动或移出时(B),大线圈中的磁通量发生变化,从而在大线圈中感应出电流,这一电流通过检流计(G)检测到。[1]} \label{fig_FLDL_1}
\end{figure}
法拉第电磁感应定律(简称法拉第定律)是电磁学中的一条定律,用于预测磁场如何与电路相互作用以产生电动势(emf)。这种现象被称为电磁感应,是变压器、电感器以及许多类型的电动机、发电机和螺线管的基本工作原理。[2][3]

麦克斯韦-法拉第方程(作为麦克斯韦方程组之一)描述了一个事实,即空间变化的(也可能是时间变化的,具体取决于磁场随时间的变化情况)电场总是伴随着时间变化的磁场,而法拉第定律则表明,当通过由导电回路包围的表面的磁通量随时间变化时,导电回路中会产生电动势(即单位电荷沿回路运动一圈时电磁作用所做的功)。

法拉第定律被发现后,其一个方面(变压器电动势)被表述为麦克斯韦-法拉第方程。法拉第定律的方程可以通过麦克斯韦-法拉第方程(描述变压器电动势)和洛伦兹力(描述运动电动势)推导而来。麦克斯韦-法拉第方程的积分形式仅描述变压器电动势,而法拉第定律的方程同时描述变压器电动势和运动电动势。
\subsection{历史}  
电磁感应现象分别由迈克尔·法拉第于1831年和约瑟夫·亨利于1832年独立发现。[4] 法拉第是第一个发表其实验结果的人。[5][6]
\begin{figure}[ht]
\centering
\includegraphics[width=8cm]{./figures/40394efd87bbeabe.png}
\caption{法拉第1831年的演示[7]} \label{fig_FLDL_2}
\end{figure}
法拉第在1831年8月29日的笔记[8]中描述了一项电磁感应的实验演示(见图)[9],他将两根导线缠绕在铁环的两侧(类似于现代的环形变压器)。他对电磁铁新发现的特性进行了评估,并推测,当一侧的导线开始流过电流时,一种波动会通过铁环传播,并在另一侧引起某种电效应。确实,当他将左侧的导线连接或断开电池时,右侧导线连接的电流计的指针显示了瞬时电流(他称之为“电波”)[10]: 182–183 。这种感应是由于电池连接或断开时产生的磁通量变化导致的。[7] 他的笔记还记录到,电池侧导线的圈数越少,电流计指针的扰动就越大。[8]

在两个月内,法拉第发现了几种其他形式的电磁感应。例如,他观察到,当他快速将一根条形磁铁插入或拉出线圈时,会产生瞬时电流;同时,他通过在条形磁铁附近旋转一个带有滑动电接触的铜盘(即“法拉第圆盘”),生成了一个稳定的直流电流(DC)。[10]: 191–195 
\begin{figure}[ht]
\centering
\includegraphics[width=8cm]{./figures/11efdbf01103c931.png}
\caption{法拉第圆盘,第一台电动发电机,一种同极发电机。} \label{fig_FLDL_3}
\end{figure}
迈克尔·法拉第用他称为“力线”的概念解释了电磁感应。然而,当时的科学家普遍拒绝了他的理论观点,主要是因为这些观点没有用数学形式表述。[10]: 510 唯一的例外是詹姆斯·克拉克·麦克斯韦,他在1861-1862年以法拉第的思想为基础,建立了他的定量电磁理论。[10]: 510 [11][12] 在麦克斯韦的论文中,电磁感应的时间变化部分以微分方程的形式表达,奥利弗·亥维赛德将其称为“法拉第定律”,尽管它与法拉第最初的定律版本不同,并未描述运动电动势(motional emf)。亥维赛德的版本(见下文麦克斯韦-法拉第方程)是如今被称为“麦克斯韦方程组”的方程形式之一。

**楞次定律**由埃米尔·楞次(Emil Lenz)于1834年提出,[13] 它描述了“通过回路的磁通量”,并给出了由电磁感应产生的感应电动势和电流的方向(详见下文示例的扩展说明)。

根据阿尔伯特·爱因斯坦的说法,他的狭义相对论理论的大部分基础和发现来源于法拉第1834年提出的电磁感应定律。[14][15]