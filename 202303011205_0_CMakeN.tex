% CMake 笔记

\begin{issues}
\issueDraft
\end{issues}

\pentry{Makefile 笔记\upref{Make}}

参考\href{https://cmake.org/cmake/help/latest/guide/tutorial/index.html}{官方教程}。
S
\subsection{常识}
\begin{itemize}
\item 在 Windows 下可直接使用 GUI, linux 命令行中使用 \verb`ccmake` 可以有 TUI。 否则就用 \verb`cmake`
\item 在 \verb|CMakeLists.txt| 的路径下, \verb`cmake .` 生成 \verb|Makefile|, 然后 \verb|make -j12| 多线程编译。 \verb|make VERBOSE=1 -j12| 输出编译命令。 cmake 会在当前路径生成 \verb|Makefile| 和一些临时文件和文件夹, 清理麻烦, 所以还是建议输出到子文件夹。
\item \verb|CMakeLists.txt| 中, 注释如 \verb|#[[一些注释]]| 或者 \verb|#一些注释|
\item 如果不想改变源码路径, 就用在别的文件夹 \verb|cmake 源码路径| 即可。
\item \verb|cmake -L 源码路径| 列出编译选项, \verb|-LH| 列出选项以及帮助说明。 \verb|-LAH| 列出所有选项(会多出很多 \verb|CMAKE_| 开头的, 不是作者提供的选项), 用 \verb|cmake -D 选项1=值1 -D 选项2=值2 源码路径| 来设置选项(\verb|-D| 后面可以没有空格)。
\item 要指定路径, 用 \verb|cmake -S CMakeList的路径 -B 生成Makefile的路径|, 然后再到 Makefile 的路径 \verb|make -j12| 即可。 如果要清理 cmake 生成的文件以及所有编译出来的文件, 直接把第二个路径清空即可。
\item 当然, 也不是什么平台都可以用 \verb|make|, 所以也可以用 \verb|cmake --build 生成Makefile的路径 [--target 可执行文件] -- -j 12| 其中 \verb|--| 后面的选项会传给 \verb|make| 或者别的具体用于编译的程序。
\item Cmake 的基本语法是 \verb`命令(arg1 arg2 ...)` 其中 \verb|命令| 不区分大小写, \verb`arg` 用空格隔开也可以换行, 如果 \verb|arg| 本身有空格, 用双引号即可
\item \verb`message(STATUS "...")` 可以输出到 stdout。 如果 \verb`message(STATUS ${变量})` 中的变量是 list, 那么中间不会有空格, 应该用 \verb`message(STATUS "${变量}")`, 输出中每个元素会用分号隔开。 \verb`message(STATUS ${变量1} ${变量2} ...)` 相当于把若干变量合成 list。
\item \verb|message(FATAL_ERROR "字符串")| 会输出错误信息并立即退出。
\item 和 bash 一样, 变量都是字符串或者字符串列表, 使用变量格式为 \verb`${变量}`。 注意不能用 \verb|$变量|。
\item 当字符串中没有空格时, 两边是否加双引号是等效的。 被引号引起来的就是单个字符串而不是字符串列表。
\item \verb`set` 可以对变量赋值, 例如 \verb`set(变量, 123)`
\item 变量是区分大小写的。
\item \verb`set` 也可以把许多变量变成一个 list, 如 \verb`set(Foo aaa bbb ccc)` (等效于 \verb|set(Foo "aaa" "bbb" "ccc")|) 将 \verb`Foo` 变成 list
\item 把 list 作为 \verb|命令()| 的参数相当于多个 \verb|arg|。 例如 \verb`命令(${Foo})` 相当于 \verb`命令(aaa bbb ccc)`。
\item \verb|set(MY_LIST "one" "two")| 相当于 \verb|set(MY_LIST "one;two")|。 后者是 list 的内部的表示方法。
\item 所以 \verb`命令("${Foo}")` 相当于 \verb`命令("aaa;bbb;ccc")` (双引号中的是一个字符串不是 list)
\item linux 环境变量和 windows 注册表变量可以直接作为变量使用, 格式如 \verb`$EVN{变量}`。 例子: \verb|message(STATUS $ENV{PWD})|。 注意 cmake 看到的环境变量未必和 bash 中的相同, 可以用 \verb`cmake -E environment` 检查。
\item \verb|separate_arguments(变量)| 可以把一个含有若干空格的字符串拆分成 list (即把空格替换为分号)。
\end{itemize}

\subsection{常用命令}
\begin{itemize}
\item (可以学习一下 Eigen 的 CMakeList.txt)
\item \verb`cmake_minimum_required(VERSION x.x)` (必须有)
\item \verb`project(project_name)` 指定项目名称(必须有)
\item \verb|set(CMAKE_CXX_COMPILER "/usr/bin/g++")| 设置 compiler
\item \verb|set(CMAKE_CXX_FLAGS "-std=c++11")| 设置 flag, 包括 c++ 标准。
\item 用标准的 \verb|set(CMAKE_CXX_STANDARD 11)| 会对 g++ 自动使用 \verb|-std=gnu11| 而不是 \verb|-std=c++11|, 一些情况会出错。
\item 添加 debug 模式的 flag 如 \verb|set(CMAKE_CXX_FLAGS_DEBUG "${CMAKE_CXX_FLAGS_DEBUG} -fsanitize=address")| 或者下面的 \verb`list(APPEND ...)`
\item \verb`set(变量 str1 [str2] ...)` 对变量赋值(若有多个值就生成 list)
\item \verb`list(APPEND 变量 str1 [str2] ...)` 在变量(list)后面添加元素
\item \verb|list(REMOVE_ITEM 变量 str1 ...)| 从 \verb|变量| 的 list 中移除(如果不存在也不报错)
\item \verb|set(变量 "${变量} str1 str2")| 等效于 \verb`list(APPEND ...)`。
\item 使用一个不存在的变量相当于空字符串。
\item \verb`file(GLOB 变量 "folder/*.cpp" "fname2.cpp")` 可以列出所有 \verb|"folder/*.txt"| 文件, 以及 \verb|fname2.cpp| (如果存在) 赋值给 \verb|变量|。 注意文件名包括绝对路径。 match 的结果也会存到 \verb|CMAKE_MATCH_编号|, 编号从 1 开始。
\item \verb|file(READ 文件名 变量)| 读取文件内容到变量。
\item \verb|string(REGEX MATCH "...(...)..." 变量 字符串变量|。 在\verb|字符串变量|中匹配正则表达式, 并把 \verb|()| 中的匹配结果赋值给 \verb|变量|。
\item \verb|string(REPLACE "老词" "新词" 变量 "字符串")| 做字符串替换。
\item \verb|add_definitions(-D FOO -D BAR)| 给编译器添加宏定义
\item \verb|add_compile_options(-Wall -fopenmp)| 给编译器添加选项(注意不能用 \verb|-D|)
\item \verb`add_executable(exe_name source_name)` 前者是可执行文件名, 后者是所有需要 link 的 \verb|c/c++| 文件名,不需要头文件。 cmake 会自动分析代码得到哪个 cpp 调用哪个头文件, 如果头文件改变了, 只有调用它的 cpp 会重新编译。
\item \verb`configure_file(file_in file_out)` 将文件中 \verb`@变量@` 替换为变量 \verb`变量` 的字符串。
\item \verb`include_directories(dir1 [dir2] ...)` 添加头文件的搜索路径
\item \verb`add_library(exe_name source_name)` 单独编译一个 library (在 library 路径的 \verb|CMakeLists.txt| 中使用这个命令而不是 \verb`add_executable` 命令)
\item \verb|link_directories(路径)| 相当于编译器的 \verb`-L` 选项, 添加静态或动态 library 的搜索路径。
\item \verb|set(CMAKE_INSTALL_RPATH 路径)| 相当于设置 \verb|rpath| (到底是 \verb|RUNPATH| 还是 \verb|RPATH|?)
\item \verb`add_subdirectory(dir1 [dir2] ...)` 执行子路径中的 \verb|CMakeLists.txt|
\item \verb`target_link_libraries(exe_name lib1 [lib2] ...)` link 阶段链接 library, 相当于 \verb|-l lib1 -l lib2|
\item \verb`option(opt_name description default)` 定义一个 option 开关(会在 cmake 的 GUI 中显示开关)以及默认值。 \verb`opt_name` 可以在 \verb|CMakeLists.txt| 中的 \verb`if` 语句中使用例如 \verb`if(opt_name) ... endif(opt_name)`。 \verb`default` 可以是 \verb`ON` 或 \verb`OFF`。
\item 注意更新 \verb|option| 的默认值后, 需要清空 cmake 的临时文件才可以生效。
\item 如果需要字符串类型的 option, 可以用 cached variable: \verb|set(变量 "默认值" CACHE STRING "描述")|。 如果之前 \verb|变量| 已经被赋值(例如命令行 \verb|cmake -D 变量=...|), 则值不会被覆盖。 这种变量叫做 \textbf{cached variable}。
\item 以上两种方式产生的变量都可以在 \verb|cmake -LH .| 的帮助中看到说明。
\item 在模板文件 \verb`*.in` 中, 当 \verb`opt_name` 为 \verb`ON` 时, 使用 \verb|configure_file| 命令后 \verb`#cmakedefine opt_name` 会被替换为 \verb`#define opt_name`。 如果想要给宏定义一个值, 用 \verb`#cmakedefine USE_FEATURE_B @USE_FEATURE_B@`
\item 【cmake 3.16 的新功能】 \verb|target_precompile_headers(可执行文件 PRIVATE foo.h bar.h)| 可以编译头文件, 注意貌似不会自动包括 .h 包括的文件。 如果头文件(或者依赖的文件)改了, 那么将会自动重新编译头文件。
\item \verb|execute_process(COMMAND 命令 参数1 参数2 ...)| 可以执行命令, 其中命令和参数中都可以用 \verb|${变量}|。 其他功能(包括命令返回的内容, 文件输入输出), 详见\href{https://cmake.org/cmake/help/latest/command/execute_process.html}{文档}。
\item \verb|include(文件或者模块)|, 详见\href{https://cmake.org/cmake/help/latest/command/include.html}{文档}。 模块可以是 \verb|*.cmake| 的文件, 也可以是 cmake 的内建模块如 \verb|CheckCXXCompilerFlag|
\item \verb|macro(宏名 [参数1 参数2]) 一些命令 endmacro()| 定义宏, 调用如 \verb|ei_add_cxx_compiler_flag(参数1 参数2)|
\item \verb|check_cxx_source_compiles(代码变量 输出变量)|\href{https://cmake.org/cmake/help/latest/module/CheckCXXSourceCompiles.html}{(文档)} 可以判断某个源码是否可以编译成功。 在调用前, 可以用 \verb|CMAKE_REQUIRED_LIBRARIES| 变量设置所需的库, \verb|CMAKE_REQUIRED_FLAGS| 指定额外的编译器选项, 等。
\item \verb|check_cxx_compiler_flag(编译器选项 输出变量)| 可以检查某个编译器选项是否可用。 \verb|输出变量| 是 True 或 False。
\item 循环如 \verb|foreach(变量 IN 列表) ... endforeach()| 或者 \verb|foreach(变量 IN 变量1 变量2 ...) ... endforeach()| 其中 \verb|IN| 在 cmake2.4 以后可以省略。 \href{https://cmake.org/cmake/help/latest/command/foreach.html}{文档}。
\item \verb|find_program(变量 程序名 [路径1 路径2 ...])| 可以在指定路径或者 \verb|$PATH| 中寻找某个可执行文件, 并把绝对路径写入\verb|变量|。 如果没有找到, 会赋空值。
\item \verb|add_custom_command(OUTPUT output_files COMMAND 命令... [ARGS arguments...] [WORKING_DIRECTORY dir])| 相当于 Makefile 中的一个 rule, 指定依赖关系, 以及使用的命令。
\item \verb|add_custom_target(target [COMMAND 命令...] [ARGS arguments...] [DEPENDS depends...] [BYPRODUCTS byproducts...] [WORKING_DIRECTORY dir] [SOURCES sources...])| 会在生成的 Makefile 中添加一个 target。
\end{itemize}

\subsection{依赖}
\begin{itemize}
\item \verb|find_package(包名)|
\item \verb|包名| 会在 \verb|CMAKE_MODULE_PATH| 的路径里面搜索名为 \verb|Find包名.cmake| 的文件。 该文件通常会设置一些变量, 可以在 \verb|find_package| 结束后访问。
\end{itemize}


\subsection{内置变量}
\begin{itemize}
\item \verb|PROJECT_NAME| 是 \verb|project()| 设置的项目名称
\item \verb`PROJECT_SOURCE_DIR` 是源文件的根路径, 就是传给 \verb`ccmake` 的路径
\item \verb`PROJECT_BINARY_DIR` 是 Cmake 的输出路径, 临时文件, Makefile 等都在这个路径。 这就是运行 \verb`ccmake` 的路径
\item \verb|CMAKE_CURRENT_LIST_DIR| 当前在处理的 \verb|CMakeLists.txt| 的路径。
\item \verb|CMAKE_SOURCE_DIR| 和 \verb|CMAKE_BINARY_DIR| 和上面两个有什么区别?
\item \verb|CMAKE_BUILD_TYPE| 可以被用户设置为 \verb|Release|, \verb|Debug| 等。 例如 \verb|cmake -DCMAKE_BUILD_TYPE=Debug| 用 debug 模式编译
\item \verb|WIN32| 可以判断是否在 windows 上(包括 32 和 64 位)
\item \verb|MSVC| 判断是否用 Visual C++ 编译器
\end{itemize}

\subsection{判断}
参考\href{https://cmake.org/cmake/help/latest/command/if.html}{这里}。
\begin{lstlisting}[language=bash]
if(1 或 ON 或 YES 或 TRUE 或 Y)
  ……
elseif(0 或 OFF 或 NO 或 FALSE 或 N 或 IGNORE 或 NOTFOUND)
  这里永远不会执行
elseif(条件 AND (条件 OR 条件) OR NOT 条件)
  ……
elseif(("bar" IN_LIST 变量) OR (file1 IS_NEWER_THAN file2))
  ……
elseif(变量)
  当变量有定义且不是 false constant
if(ENV{变量})
  这里永远不会执行
elseif (DEFINED <name>|CACHE{<name>}|ENV{<name>})
  若定义了变量
elseif (变量1 STREQUAL 变量2)
  比较字符串
elseif (IS_DIRECTORY 某路径)
  判断某路径是否存在
else()
  ……
endif()
\end{lstlisting}

\subsection{CMake 与 Visual Studio}
\begin{itemize}
\item 链接 \verb`*.lib`, 只需要在 \verb`add_executable()` 命令前面插入(必须在之前) \verb`abc.lib` 的路径 \verb`link_directories(path/to/lib)` 然后再 \verb`add_executable()` 之后插入 \verb`target_link_libraries(exe_name abs)` 即可
\item \verb`if(MSVC)...end(MSVC)` 可以专门给 Visual Studio 执行一些命令
\item 生成的 sln 文件中, 除了有 target 的工程, 还会有另外两个工程: \verb`ZERO_CHECK` 会重新运行/更新 \verb`CMakeLists.txt`, \verb`ALL_BUILD` 会编译所有工程。 如果看着不爽的话也可以把这两个 project 删掉。 如果直接 run without debug 的话, 会提示 \verb`ALL_BUILD` 不能 run, 所以要 run 只能右键某个 project 然后 run (所以还是把两个多余的 project 删掉好些)。
\end{itemize}

\subsection{ccmake 使用}
\begin{itemize}
\item \verb`sudo apt install cmake-curses-gui` 安装
\item 在需要生成 binary 的路径打开 \verb`ccmake`, 这个路径可以是源文件路径
\item 如果 binary 路径不是源文件路径, 那么指定路径即可, 例如 \verb`ccmake ../`
\item \verb`h` 帮助, \verb`q` 退出, \verb`c` configure, \verb`G` 生成 Makefile
\end{itemize}
