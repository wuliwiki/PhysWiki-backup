% 全连接网络
% keys 全连接
% license Xiao
% type Tutor


\addTODO{本词条处于草稿阶段。}

\pentry{神经网络\upref{NN}}

\textbf{全连接网络}(Fully-connected neural network, FCNN)是由一系列全连接层组成的深度神经网络,是深度学习中的基本架构。全连接层的特点是相邻两层的任意两个神经元之间均有连接。可以说全连接网络是多层感知机的加深版本。

特别是,我们探索了完全连接的架构是能够学习任何函数的“通用逼近器”的概念。这个概念提供了对全连接架构的普遍性的解释,但也带来了我们在一定深度讨论的许多注意事项。

虽然结构不可知论使得全连接网络具有非常广泛的适用性,但此类网络的性能往往比针对问题空间结构调整的专用网络要弱。

\begin{figure}[ht]
\centering
\includegraphics[width=6cm]{./figures/e2d45dcbed13526d.png}
\caption{全连接层 [1]} \label{fig_FCNN_1}
\end{figure}
\autoref{fig_FCNN_1} 表示的是一个全连接层。

假设一个全连接层有$m$个输入神经元,分别记为:$x_1$, $x_2$, ..., $x_m$,以及$n$个输出神经元,分别记为$y_1$, $y_2$, ..., $y_n$, 从第$i$个输入神经元到第$j$个输出神经元的连接的权值为$w_{i,j}$。那么,第$j$个输出神经元输出的值为:
\begin{equation}
y_j=g(w_{1,j}x_1+w_{2,j}x_2+...+w_{m,j}x_m)~
\end{equation}



\begin{figure}[ht]
\centering
\includegraphics[width=8cm]{./figures/6205d3a76aa14e17.png}
\caption{全连接网络 [1]} \label{fig_FCNN_2}
\end{figure}
\autoref{fig_FCNN_2} 表示的是一个全连接网络,其结构是由全连接层堆叠而成。

全连接网络可以用于处理一维数据,也可以处理二维甚至高维数据。比如,在图像处理中就有不少应用。一张图像可以视为一个二维数据。假设一组图像数据。



\textbf{参考文献}
\begin{enumerate}
\item https://www.oreilly.com/library/view/tensorflow-for-deep/9781491980446/ch04.html
\end{enumerate}