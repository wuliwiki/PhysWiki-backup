% 一阶常系数线性微分方程组
% keys 常微分方程|ODE|ordinary differential equation|方程组|线性变换|矩阵|矩阵指数

\pentry{矩阵指数\upref{MatExp},常系数线性微分方程\upref{ODEb2}}

\subsection{齐次方程}

一阶常系数\textbf{齐次}线性微分方程组形如
\begin{equation}\label{ODEb3_eq1}
\leftgroup{
    \frac{\dd}{\dd t}x_1&=a_{11}x_1+a_{12}x_2+\cdots+a_{1n}x_n\\
    \frac{\dd}{\dd t}x_2&=a_{21}x_1+a_{22}x_2+\cdots+a_{2n}x_n\\
    &\vdots\\
    \frac{\dd}{\dd t}x_n&=a_{n1}x_1+a_{n2}x_2+\cdots+a_{nn}x_n
}
\end{equation}
其中各$x_i$是关于$t$的未知函数,各$a_{ij}$是已知常数.我们要研究的是如何解出这个方程组中的各未知函数.

别看这个方程有那么多变量$x_i(t)$,实际上我们可以把它们放到一起,构成一个$n$维向量$\bvec{x}(t)=(x_1(t), x_2(t), \cdots, x_n(t))$,这样就可以理解为还是只有一个自变量,只不过自变量从标量变成向量了.

以上述向量理解的方式来看,\autoref{ODEb3_eq1} 右边部分就是一个线性变换$\mat{M}\bvec{x}(t)$,其中
\begin{equation}\label{ODEb3_eq4}
\mat{M}=\pmat{
    &a_{11} &a_{12} &\cdots &a_{1n}\\
    &a_{21} &a_{22} &\cdots &a_{2n}\\
    &\vdots &\vdots &\ddots &\vdots\\
    &a_{n1} &a_{n2} &\cdots &a_{nn}
    }
\end{equation}
是已知常数矩阵.

这样,我们就还可以把\autoref{ODEb3_eq1} 写成
\begin{equation}\label{ODEb3_eq3}
\frac{\dd}{\dd t}\bvec{x}=\mat{M}\bvec{x}
\end{equation}
的形式,看起来和一元方程
\begin{equation}\label{ODEb3_eq2}
\frac{\dd}{\dd t}x=ax
\end{equation}
非常像.

\autoref{ODEb3_eq2} 的通解是$x=C\E^{at}$,其中$C$为常数.事实上,\autoref{ODEb3_eq3} 的通解也可以类似地用\textbf{矩阵指数}\upref{MatExp}来表示:
\begin{equation}\label{ODEb3_eq5}
\bvec{X}=\E^{\mat Mt}
\end{equation}

\begin{definition}{基解矩阵}
如果$n\times n$矩阵$\mat X$的每一列作为一个向量,都是\autoref{ODEb3_eq1} 或者说\autoref{ODEb3_eq3} 的线性无关解\footnote{于是有了$n$个线性无关解,根据方程组的存在与唯一性定理,任何一个解都可以用基解矩阵里的列向量线性组合得出.},那么称$\mat X$是 \autoref{ODEb3_eq1} 或者说\autoref{ODEb3_eq4} 的\textbf{基解矩阵}.
\end{definition}

\begin{theorem}{}
\autoref{ODEb3_eq5} 是\autoref{ODEb3_eq1} 或者说\autoref{ODEb3_eq4} 的基解矩阵.
\end{theorem}

证明思路很简单,根据\autoref{MatExp_the1}~\upref{MatExp},直接得到$\frac{\dd}{\dd t}\mat{X}=\mat M\mat X$,又由于求导是对每个矩阵元独立进行的,因此把$\mat X$看成是列向量排列而成的行矩阵,则每个列向量都是一个解.另外,由于$\mat X|_{t=0}=\mat E$,其中$\mat E$为单位矩阵,故$\opn{tr}\mat X=n\not=0$,故根据\autoref{MatExp_the2}~\upref{MatExp},$\abs{\mat{X}}$的行列式不为零,故各解向量线性无关.这样,线性无关的一组解就构成了基解矩阵.


基解矩阵的地位和基本解组一样,都能用来线性组合出所有的解.我们看一个实例:

\begin{example}{}
考虑方程组
\begin{equation}\label{ODEb3_eq6}
\leftgroup{
    \frac{\dd}{\dd t}x_1&=x_1+x_2\\
    \frac{\dd}{\dd t}x_2&=x_2
}
\end{equation}
且已知初值
\begin{equation}\label{ODEb3_eq8}
\leftgroup{
    x_1(0)&=3\\
    x_1(0)&=2
}
\end{equation}
如何求解此方程呢?

首先我们要求出其基解矩阵:
\begin{equation}
\begin{aligned}
\mat X=\E^{\pmat{1&1\\0&1}t}&=\sum_{n=0}^\infty\frac{1}{n!}\pmat{1&n\\0&1}t^n\\
&=\sum_{n=0}^\infty\frac{1}{n!}\pmat{t^n&nt^n\\0&t^n}\\
&=\pmat{\E^t&t\E^t\\0&\E^t}
\end{aligned}
\end{equation}

容易验证,
\begin{equation}
x_1=\E^t, x_2=0
\end{equation}
和
\begin{equation}
x_1=t\E^t, x_2=\E^t
\end{equation}
都是\autoref{ODEb3_eq6} 的解,且它们彼此在任何区间上都线性无关,因此$\mat X$确实是\autoref{ODEb3_eq6} 的基解矩阵.

现在设
\begin{equation}\label{ODEb3_eq7}
\leftgroup{
x_1&=A\E^t+Bt\E^t\\x_2&=B\E^t
}
\end{equation}
其中$A, B$是待定常数.

将\autoref{ODEb3_eq7} 代入初值条件\autoref{ODEb3_eq8} ,解出$A, B$,最终得到满足初值条件的特解
\begin{equation}
\leftgroup{
    x_1&=3\E^t+2t\E^t\\
    x_2&=2\E^t
}
\end{equation}


\end{example}




















