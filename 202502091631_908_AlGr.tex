% 亚历山大·格罗滕迪克(综述)
% license CCBYSA3
% type Wiki

本文根据 CC-BY-SA 协议转载翻译自维基百科\href{https://en.wikipedia.org/wiki/Alexander_Grothendieck#Mathematical_work}{相关文章}。

\begin{figure}[ht]
\centering
\includegraphics[width=6cm]{./figures/a1640137f4bbaf24.png}
\caption{1970年,亚历山大·格罗滕迪克在蒙特利尔。} \label{fig_AlGr_1}
\end{figure}
亚历山大·格罗滕迪克(后来的法语名为亚历克斯·格罗滕迪克,发音:/ˈɡroʊtəndiːk/;德语发音:[ˌalɛˈksandɐ ˈɡʁoːtn̩ˌdiːk] ⓘ;法语发音:[ɡʁɔtɛndik]),1928年3月28日出生,2014年11月13日去世,是一位出生于德国的法国数学家,他在现代代数几何的创立中成为了主要人物。他的研究拓展了该领域的范围,并将交换代数、同调代数、层理论和范畴理论等元素融入了其基础中,而他所谓的“相对”视角则在纯数学的许多领域带来了革命性的进展。许多人认为他是二十世纪最伟大的数学家。

格罗滕迪克于1949年开始了他富有成效且公开的数学家生涯。1958年,他被任命为高等科学研究院(IHÉS)的研究教授,并一直在那里工作,直到1970年,由于个人和政治信念,他因与军事资金的争执而离开。1966年,他因在代数几何、同调代数和K理论方面的突破而获得了菲尔兹奖。他后来成为蒙彼利埃大学的教授,并在继续进行相关数学研究的同时,逐渐退出了数学界,投身于政治和宗教事务(最初是佛教,后来转向更为天主教的基督教观点)。1991年,他搬到了位于比利牛斯山脉的法国小村庄拉塞尔,在那里他过上了隐居生活,仍然致力于数学及其哲学和宗教思想,直至2014年去世。
\subsection{生平}  
\subsubsection{家庭与童年}  
格罗滕迪克出生于柏林,父母为无政府主义者。他的父亲亚历山大·“萨沙”·沙皮罗(也叫亚历山大·塔纳罗夫)有哈西德犹太血统,曾在俄罗斯被囚禁,后于1922年移居德国;他的母亲约翰娜·“汉卡”·格罗滕迪克来自汉堡的一个新教德国家庭,并且是一名记者。[a] 两位父母在青少年时期都脱离了他们的早期背景。[16] 在格罗滕迪克出生时,他的母亲与记者约翰内斯·拉达茨结婚,最初,他的出生名字被记录为“亚历山大·拉达茨”。该婚姻在1929年解除了,沙皮罗承认了自己的父亲身份,但并未与汉卡·格罗滕迪克结婚。[16] 格罗滕迪克有一位母亲那边的兄弟姐妹——同父异母的妹妹麦迪。

格罗滕迪克与父母一起生活在柏林,直到1933年底,父亲为了躲避纳粹主义而搬到巴黎,母亲随之而后。格罗滕迪克被交由威廉·海多恩照料,海多恩是一位路德教牧师和汉堡的教师。[17][18] 据温弗里德·沙尔劳称,在此期间,格罗滕迪克的父母作为非战斗辅助人员参与了西班牙内战。[19][20] 然而,也有其他人表示沙皮罗曾在无政府主义民兵中作战。[21]
\subsubsection{第二次世界大战}  
1939年5月,格罗滕迪克被从汉堡送上前往法国的火车。不久后,他的父亲被关押在勒维尔内(Le Vernet)集中营。[22] 他和母亲随后在1940年至1942年间作为“危险的外国人”被关押在不同的集中营。[23] 第一个营地是里厄克罗斯营地(Rieucros Camp),在那里,他的母亲感染了结核病,这种疾病最终导致她在1957年去世。在那里,格罗滕迪克设法上了当地的学校——孟德尔学校(Mendel)。有一次,他设法从营地逃脱,打算刺杀希特勒。[22] 后来,他的母亲汉卡被转移到居尔斯集中营,直到第二次世界大战结束。[22] 格罗滕迪克被允许与母亲分开生活。[24]

在勒尚邦-sur-Lignon村,他在当地的寄宿家庭或旅馆中得到了庇护和隐藏,尽管有时他必须在纳粹突袭期间躲进树林里,几天没有食物和水也能活下来。[22][24]

他的父亲在维希政府的反犹法令下被逮捕,并被送到德朗西集中营,随后由维希政府交给德国人,被送往奥斯维辛集中营,在1942年被杀害。[8][25]

在勒尚邦,格罗滕迪克就读于塞文学院(Collège Cévenol,现在被称为Le Collège-Lycée Cévenol International),这是一所由当地新教和平主义者和反战活动家于1938年创办的独特中学。许多在勒尚邦藏匿的难民儿童都在塞文学院就读,正是在这所学校里,格罗滕迪克显然第一次对数学产生了浓厚的兴趣。[26]

1990年,因冒着生命危险拯救犹太人,整个村庄被认定为“国际义人”(Righteous Among the Nations)。
\subsubsection{学业与接触研究数学}  
战争结束后,年轻的格罗滕迪克在法国学习数学,最初在蒙彼利埃大学,起初他的表现不佳,甚至在天文学等课程上不及格。[27] 他开始独立学习,重新发现了勒贝格测度。在那里经过三年的越来越独立的学习后,他于1948年继续前往巴黎深造。[17]

最初,格罗滕迪克参加了亨利·卡尔坦(Henri Cartan)在巴黎高等师范学校(École Normale Supérieure)的研讨会,但由于缺乏必要的背景,他无法跟上这个高水平研讨会的进度。在卡尔坦和安德烈·韦伊(André Weil)的建议下,他转到南锡大学(University of Nancy),那里有两位领先的专家在研究格罗滕迪克感兴趣的领域——拓扑向量空间:让·迪厄多内(Jean Dieudonné)和洛朗·施瓦茨(Laurent Schwartz)。后者刚刚获得了菲尔兹奖。迪厄多内和施瓦茨向这位新来的学生展示了他们最新的论文《空间(F)与(LF)中的对偶性》(La dualité dans les espaces (F) et (LF));论文末尾列出了14个开放问题,涉及局部凸空间。[28] 格罗滕迪克引入了新的数学方法,使他能够在几个月内解决所有这些问题。[29][30][31][32][33][34][35]

在南锡,他在这两位教授的指导下写了他的博士论文,内容是关于泛函分析,从1950年到1953年。[36] 在此期间,他成为了拓扑向量空间理论的领先专家。[37] 1953年,他移居巴西圣保罗大学,凭借难民护照(Nansen护照),因为他拒绝加入法国国籍(因为那样会要求他服兵役,而这与他的信念相悖)。他在圣保罗呆到1954年底(除了1953年10月到1954年3月期间在法国的长时间访问)。他在巴西期间发表的工作仍然集中在拓扑向量空间理论;正是在那里,他完成了关于这一主题的最后一项重要工作——关于巴拿赫空间的“度量”理论。

格罗滕迪克于1955年初搬到了堪萨斯州的劳伦斯,并在那里将他以前的研究主题放在一边,开始从事代数拓扑、同调代数,并逐渐转向代数几何。[38][39] 正是在劳伦斯,格罗滕迪克发展了他的阿贝尔范畴理论,并基于此重新构建了层的同调学,说法最终导致了具有深远影响的《东北论文》("Tôhoku paper")。[40]

1957年,他受到奥斯卡·扎里斯基(Oscar Zariski)的邀请,前往哈佛大学访问,但因为拒绝签署承诺不参与推翻美国政府的声明,这个邀请未能成行。扎里斯基警告他说,这样的拒绝可能会让他入狱。然而,入狱的前景并没有让他担心,只要他能够接触到书籍。[41]

回顾格罗滕迪克在南锡时期的情况,与当时从巴黎高等师范学校(École Normale Supérieure)训练出来的学生(如皮埃尔·萨缪尔、罗杰·戈德曼、勒内·托姆、雅克·迪克斯米尔、让·塞尔夫、伊冯娜·布吕哈、让-皮埃尔·塞尔和伯纳德·马尔格朗日)相比,莱拉·施内普斯(Leila Schneps)说:

“他对这一群人及其教授几乎完全陌生,来自如此贫困和混乱的背景,相较于他们,刚开始时他的知识如此贫乏,然而他那种闪电般的升迁至突然成名的过程是如此不可思议;在数学史上,这种现象是独一无二的。”[42]

他在1953年关于拓扑向量空间的早期工作已成功应用于物理学和计算机科学,最终在量子物理学中形成了格罗滕迪克不等式与爱因斯坦—波多尔斯基—罗森悖论之间的关系。[43]
\subsubsection{IHÉS 时代}  
1958年,格罗滕迪克被聘任到高等科学研究院(Institut des hautes études scientifiques,简称IHÉS),这是一所由私人资助的新研究机构,实际上是为了让让·迪厄多内和格罗滕迪克能够工作而创建的。[3] 格罗滕迪克通过在那里举办一系列密集且高产的研讨会吸引了大量关注(这些研讨会实际上是一些工作小组,汇聚了法国及其他年轻一代最有才华的数学家,进行基础性工作)。[17] 格罗滕迪克几乎停止了通过传统的学术期刊发表论文的方式。然而,他仍然能够在大约十年的时间里在数学领域发挥主导作用,培养出强大的数学流派。[44]

在这段时间内,他的学生包括米歇尔·德马兹尔(Michel Demazure,研究SGA3中的群方案)、莫妮克·哈基姆(Monique Hakim,研究相对方案和分类拓扑)、吕克·伊吕斯(Luc Illusie,研究余切复合体)、米歇尔·雷诺(Michel Raynaud)、米歇尔·雷诺(Michele Raynaud)、让-路易·费尔迪耶(Jean-Louis Verdier,导出范畴理论的共同创立者)和皮埃尔·德林热(Pierre Deligne)。在SGA项目中的合作伙伴还包括迈克尔·阿尔廷(Michael Artin,研究étale同调)、尼克·卡茨(Nick Katz,研究单调性理论和Lefschetz铅笔)。让·吉罗(Jean Giraud)在这里也研究了非阿贝尔同调的扭束理论扩展。许多其他数学家,如大卫·穆姆福德(David Mumford)、罗宾·哈特肖恩(Robin Hartshorne)、巴里·马祖尔(Barry Mazur)和C.P. 拉马努贾姆(C.P. Ramanujam)等,也参与了这些工作。
\subsubsection{“黄金时代”}  
在IHÉS的“黄金时代”期间,亚历山大·格罗滕迪克的工作奠定了代数几何、数论、拓扑学、范畴理论和复分析等多个领域的统一主题。[36] 他在代数几何方面的首次(IHÉS之前的)发现是格罗滕迪克–赫尔茨布鲁赫–黎曼–罗赫定理,这是对赫尔茨布鲁赫–黎曼–罗赫定理的代数推广;在这个背景下,他还引入了K理论。随后,按照他在1958年国际数学家大会上的讲话中所概述的计划,他引入了方案理论,并在其《代数几何元素》(Éléments de géométrie algébrique,简称EGA)中详细展开,提供了更为灵活和一般化的代数几何基础,自那时以来,该基础已被该领域采纳。[17] 他接着引入了方案的étale同调理论,提供了证明魏尔猜想的关键工具,并补充了结晶同调和代数德拉姆同调理论。与这些同调理论紧密相关,他提出了拓扑理论的拓扑范畴(topos theory)作为拓扑学的一种推广(在范畴逻辑中也具有相关性)。他还通过范畴化的伽罗瓦理论,提供了方案基本群的代数定义,从而诞生了如今闻名的étale基本群,随后他猜测其进一步推广的存在性,这一推广现已被称为基本群方案(fundamental group scheme)。作为他一致性对偶理论的框架,他还引入了导出范畴,该范畴后来由费尔迪耶(Verdier)进一步发展。[45]

他在这些及其他课题上的研究成果,既在《EGA》一书中发布,也以较为粗糙的形式发表在他在IHÉS主办的《代数几何研讨会讲义》(Séminaire de géométrie algébrique,简称SGA)中。[17]
\subsubsection{政治激进主义}  
格罗滕迪克的政治观点激进且和平主义。他坚决反对美国干预越南战争和苏联的军事扩张。为了抗议越南战争,他在河内周围的森林中讲授范畴理论,而当时河内正遭受轰炸。[46] 1966年,他拒绝出席在莫斯科举行的国际数学家大会(ICM),在那里他原本将获得菲尔兹奖。[7] 约在1970年,格罗滕迪克因发现IHÉS部分资金来自军事而从科学界退休。[47] 几年后,他作为蒙彼利埃大学的教授重返学术界。

尽管军事资金问题或许是格罗滕迪克离开IHÉS的最明显原因,但认识他的人表示,导致这一断裂的原因更为深刻。皮埃尔·卡尔捷(Pierre Cartier),IHÉS的长期访客,在IHÉS四十周年纪念专刊中写了一篇关于格罗滕迪克的文章。[48] 在那篇文章中,卡尔捷指出,作为一位反军事的无政府主义者的儿子,以及在被剥夺权利的环境中成长的人,格罗滕迪克始终对贫困和被压迫者抱有深切的同情。卡尔捷形容,格罗滕迪克开始觉得布尔河(Bures-sur-Yvette)像是“一个金笼子”(une cage dorée)。当格罗滕迪克在IHÉS时,反对越南战争的情绪愈发激烈,卡尔捷认为这也加剧了格罗滕迪克对成为科学界权威人物的厌恶。[3] 此外,在IHÉS待了几年后,格罗滕迪克似乎开始寻找新的智识兴趣。到1960年代末,他开始对数学以外的科学领域产生兴趣。1964年加入IHÉS的物理学家大卫·鲁埃尔(David Ruelle)表示,格罗滕迪克曾几次找他谈论物理学。[b] 生物学比物理学更吸引格罗滕迪克,他还组织了一些关于生物学话题的研讨会。[48]