% 偏微分方程的分类与特征线
% keys 特征线|PDE|偏微分方程
% license Usr
% type Tutor

\begin{issues}
\issueMissDepend
\issueTODO
\issueDraft
\end{issues}

特征线法又被称为达朗贝尔法和行波法,一般适用于解一阶 PDE,对二阶 PDE 的分类也颇有帮助。

\subsection{一维一阶 PDE}

% todo

很多人都对偏微分方程为什么用圆锥曲线来分类有疑惑,一个偏微分方程为什么会跟平面上曲线的分类有关呢?对 PDE(偏微分方程)进行分类又对解其有何帮助?下面来讨论这些问题。

\subsection{二阶线性 PDE 的分类}
在一般的数理方程或介绍 PDE 的书籍中,会把二阶线性 PDE 作为着重点来讲并一般将之分为三类,分别是
\begin{enumerate}
\item 椭圆类(elliptic PDE),例如泊松方程 $\laplacian u = f(x,y,z)$;
\item 抛物线类(parabolic PDE),例如热传导方程 $\frac{\partial T}{\partial t} = k \laplacian T$;
\item 双曲类(hyperbolic PDE),例如波动方程 $\laplacian u = \frac{1}{c^2} \frac{\partial^2 u}{\partial t^2}$。
\end{enumerate}
