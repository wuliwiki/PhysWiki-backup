% 电介质摘要

假设我们知道物质是由带正、负电荷的粒子组成的。在外加电场下,物质中的正、负电荷将对外加电场做出响应。在金属中,存在大量自由的电荷,可以在整个物质内重新分布;而在其余介质中,电荷之间的束缚很强,因此电荷只能在小范围内重新分布。

\begin{itemize}
\item 在外加电场下,介质中产生大量电偶极子\upref{Dielec},这种过程称为介质的极化。这使介质具有宏观可感的电偶密度$\bvec P = \lim_{\Delta V \to 0} \frac{\sum \bvec p_i}{\Delta V}$ \upref{ElecPo}
\item 电偶极子导致了极化电荷 $\bvec \rho_P = - \div \bvec P$。\upref{ElePAP}
\item 极化电荷产生了额外的极化场。
\end{itemize}
