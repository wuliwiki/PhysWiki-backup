% 开放系统互联基本参考模型
% 网络 模型 五层 OSI

\textbf{开放系统互联基本参考模型}(Open System Interconnection Reference Model, OSI/RM)是由国际标准化组织提出的一种试图使得各种计算机在世界范围内互联成网的标准框架。该模型的目标是使得全球计算机能够遵循同一个协议,互相连接并交换数据。

OSI模型具有七个逻辑层次。从下到上依次为:物理层、数据链路层、网络层、传输层、会话层、表示层、应用层。逻辑层次的意思是对于网络上所传输的数据流的封装的层次,而并非是真实物理存在的实体。
\begin{figure}[ht]
\centering
\includegraphics[width=5cm]{./figures/fe3db6a185508986.png}
\caption{OSI七层模型} \label{fig_OSIRM_1}
\end{figure}

\subsection{物理层(Physical layer)}

物理层协议其实就是最底层的通讯协议。该层上传输的是数据的二进制位,即比特(bit)。物理层要做的是用适当的高、低电压来表示0和1。传输信号所用的通讯线路,例如网线、电缆、光缆、微波信道等不属于物理层,也有人将其称为第0层。

\subsection{数据链路层(Data link layer)}

数据链路层常简称为链路层。数据在网络上传输的基本单位是包含两个相邻节点主机的链路。发送方的数据链路层的基本任务就是将上层网络层交付的数据报\textbf{封装成帧}(framing),然后在两个相邻节点之间传输\textbf{帧}(frame)。每个帧包含必要的控制信息(如同步信息、地址信息、差错控制等)。接收方根据接受到的帧的控制信息,能够知道一个帧从哪个比特开始和到哪个比特结束。接收方取出帧内部封装的网络数据包之后,交给上面的网络层。

\subsection{网络层(Network layer)}
网络层负责为分组交换网上的不同主机提供通信服务。在发送数据时,网络层把传输层产生的报文段或用户数据报封装成\textbf{分组}或包进行传送。用IP协议,因此分组也叫做\textbf{IP数据报},或简称为\textbf{数据报}。

\subsection{运输层(Transport layer)}

运输层的任务就是负责向两台主机中进程之间的通信提供通用的数据传输服务。网络层能够保证的是数据在网络上的两个主机中传输,而一个主机中往往同时运行着多个进程,真正的通讯是发生在两台主机的两个进程中,因此在两台主机的进程之间的通用数据传送须要由运输层负责。所谓通用数据传输,意思是这种传输服务不是专门提供给某一个进程,而是可以复用和分用的。实际应用中,须要采用端口的方式来实现。

\subsection{会话层(Session layer)}

会话层负责在数据传输中设置和维护计算机网络中两台计算机之间的通信连接。

\subsection{表示层(Presentation layer)}

表示层把数据转换为能与接收者的系统格式兼容并适合传输的格式。


\subsection{应用层(Aapplication layer)}

应用层是体系结构中的最高层。应用层的任务是通过应用进程间的交互来完成特定网络应用。应用层协议定义的是应用进程间通信和交互的规则。这里的进程就是指主机中正在运行的程序。对于不同的网络应用需要有不同的应用层协议。在互联网中的应用层协议很多,如域名系统DNS,支持万维网应用的HTTP协议,支持电子邮件的SMTP协议,等等。我们把应用层交互的数据单元称为\textbf{报文}(message)。




\subsubsection{参考文献}
\begin{enumerate}
\item 谢希仁. 计算机网络(第7版). 电子工业出版社
\end{enumerate}