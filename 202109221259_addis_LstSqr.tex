% 最小二乘法
% 多元微积分|二元函数|直线拟合|多项式拟合|简谐波拟合

\pentry{二元函数的极值\upref{F2Exm}}

\subsection{直线拟合}
在许多情况下, 我们需要将一组散点数据 $x_i, y_i \ \ (i = 1\dots N)$ 拟合成特定形式的函数曲线. 其中最常见的情况之一是拟合一条直线, 形式为 $y = ax + b$. 例如给出(图1)%图未完成
, 如何确定直线方程的两个最佳系数 $a$ 和 $b$ 呢?

我们可以通过方差来计算拟合的误差, 方差越小则说明拟合得越好.
\begin{equation}\label{LstSqr_eq1}
S_2 = \sum_{i = 1}^N (a x_i + b - y_i)^2
\end{equation}
对于给定的 $N$ 个 $x_i, y_i$, 方差是 $a$ 和 $b$ 的二元函数. 我们只需找到这个二元函数的最小值点即可. 由于函数处处光滑, 最小值点必定满足 $\pdv*{S_2}{a} = \pdv*{S_2}{b} = 0$. 将\autoref{LstSqr_eq1} 代入, 得到一个线性方程组.
\begin{equation}
\leftgroup{
&\qty(\sum_i x_i^2) a + \qty(\sum_i x_i) b = \sum_i x_i y_i \\
&\qty(\sum_i x_i) a + N b = \sum_i y_i}
\end{equation}
我们来计算系数行列式, 如果能证明行列式恒大于零, 则方程组必有唯一解.% 未完成, 线性方程组的解
若令 $\bar x$ 为 $N$ 个 $x_i$ 的平均值, 考虑到 $x_i$ 互不相等, 容易证明
\begin{equation}
N\sum_i x_i^2 - \qty(\sum_i x_i)^2 = N\sum_i (x_i - \bar x)^2 > 0
\end{equation}
证毕. 由于方差恒大于零, 可知其必定存在最小值, 所以系数的唯一解必定是方差函数的极小值点. 接下来解线性方程组即可得到系数 $a, b$. 将解出的 $a,b$ 重新代入\autoref{LstSqr_eq1} 可以得出最小方差的值, 用于判断拟合结果的好坏.

\subsection{多项式拟合}
以二次多项式 $c_2 x^2 + c_1 x + c_0$ 为例, 方差为
\begin{equation}
S_2(c_2, c_1, c_0) = \sum_{i = 1}^N (c_2 x_i^2 + c_1 x_i + c_0 - y_i)^2
\end{equation}
分别令方差对 $c_2, c_1, c_0$ 的偏导为零, 得线性方程组
\begin{equation}\label{LstSqr_eq2}
\leftgroup{
&\qty(\sum_i x_i^4) c_2 + \qty(\sum_i x_i^3) c_1 + \qty(\sum_i x_i^2) c_0 & &= \sum_i y_i x_i^2 \\
&\qty(\sum_i x_i^3) c_2 + \qty(\sum_i x_i^2) c_1 + \qty(\sum_i x_i) c_0 & &= \sum_i y_i x_i \\
&\qty(\sum_i x_i^2) c_2 + \qty(\sum_i x_i) c_1 + \quad N c_0 & &= \sum_i y_i}
\end{equation}
观察系数矩阵可以看出它是一个方阵, 每条斜线上 $x_i$ 的指数相等, 且相邻斜线上 $x_i$ 的指数依次递减. 按照此规律容易写出 $N$ 次多项式拟合的方程组. Matlab 代码见 “最小二乘法数值拟合(多项式)\upref{LSpoly}”.

上式存在的一个问题是, 各个 $c_i$ 偏导为零, 就一定会是全局最小值吗? 我们暂时不会证明一般情况下的系数行列式必定为零, 那么会不会解出的是局部最小值,最大值,甚至鞍点呢? 首先要明确的是函数 $S_2$ 是一个二次函数, 处处光滑, 所以全局最小值必定满足各阶偏导为零, 也就是\autoref{LstSqr_eq2}. 再来考虑有多个解的情况, 根据线性方程组解的结构\upref{LinEq}, 若线性方程组存在多于一个解, 那么这些解必定不可能是分立的而是会构成一个连续的平面(不包含零点), 在这个平面上, 由于各阶偏导处处为零, $S_2$ 必定处处相等, 所以只可能所有解都是全局最小值.

此外, 根据魏尔施特拉斯逼近定理\upref{Weiers}, 闭区间上的连续函数都可以使用多项式拟合, 随着阶数增加, 多项式会一致收敛于要拟合的函数.

\subsection{简谐波拟合}
若要拟合 $A \cos(x + \varphi_0) + C$ 形式的函数, 可以先利用两角和公式把函数化为 $c_1 \cos x + c_2 \sin x + c_3$ 的等效形式(因为前者并不是待定系数的线性组合, 得到的方程组也不是线性方程组), 方差公式为
\begin{equation}
S_2 = \sum_i (c_1 \cos x_i + c_2 \sin x_i + c_3 - y_i)^2
\end{equation}
要求极值, 分别令方差对 $c_1, c_2, c_3$ 的偏导为零, 得线性方程组
\begin{equation}
\leftgroup{
&\qty(\sum_i \cos^2 x_i) c_1 + \qty(\sum_i \sin x_i \cos x_i) c_2 + \qty(\sum_i \cos x_i) c_3 & &= \sum_i y_i \cos x_i\\
&\qty(\sum_i \sin x_i \cos x_i) c_1 + \qty(\sum_i \sin^2 x_i) c_2 + \qty(\sum_i \sin x_i) c_3 & &= \sum_i y_i \sin x_i\\
&\qty(\sum_i \cos x_i) c_1 \quad + \quad \qty(\sum_i \sin x_i) c_2 \quad + \quad N c_3 & &= \sum_i y_i\\
}
\end{equation}
实际应用中, 该方程组同样几乎都有唯一解.


