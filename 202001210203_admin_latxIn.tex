% LaTeX 简介(除公式)

考虑到许多网友只用过 LaTeX 编辑公式而没有写过正文, 这里来结合本编辑器做一个简要的介绍.

LaTeX 是一种所见非所得的排版语言, 即用户编辑的是代码, 需要经过编译过程才能获得最终的显示效果. 小时物理百科的 PDF 编译使用 TeXlive2019, 而在线编辑器则是我们自行开发的.

一个简单完整的 LaTeX 文档如下
\begin{lstlisting} % 未完成:高亮
\documentclass{article}

\begin{document}
\title{My Title}
\author{My Name}
\maketitle

\section{Introduction}
Some introduction.

\begin{equation}
a^2 + b^2 = c^2
\end{equation}

\subsection{Subtitle}
Subsection text.

\end{document}
\end{lstlisting}

编译后效果如
\begin{figure}[ht]
\centering
\includegraphics[width=13cm]{./figures/LatxIn1.png}
\caption{排版效果} \label{latxIn_fig1}
\end{figure}

\subsection{环境}
一个完整的 LaTeX 文档是由许多环境构成的, 环境的格式如下
\begin{lstlisting}
\begin{环境名}[可选设置]
...
...
\end{环境名}
\end{lstlisting}
其中 \lstinline|[可选设置]| 不一定会出现. 在一个完整的 LaTeX 文档中, 最大的环境是 \lstinline|document| 环境, 文档的所有内容(包括其他环境)都在 \lstinline|document| 环境中. 