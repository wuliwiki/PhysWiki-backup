% 树状数组
% 树状数组|数据结构|C++

\subsection{基本原理}

若想实现一下两种操作:
\begin{enumerate}
\item 求一个区间内所有元素的和;
\item 修改某个元素的值。
\end{enumerate}

看到求一段序列的和很容易想到前缀和算法,单次查询的时间复杂度为 $\mathcal{O}(1)$,但是修改某个元素的值会影响前缀和数组,最坏为 $\mathcal{O}(n)$。若用普通数组,求一段数的和为 $\mathcal{O}(n)$,修改某个数为 $\mathcal{O}(1)$。若有 $m$ 次询问,两种做法的全局最坏时间复杂度都为 $\mathcal{O}(n \times m)$。树状数组这两种的操作的时间复杂度即不太慢又不太快,单次查询和修改时间复杂度都为 $\mathcal{O}(\log_2 n)$。

树状数组的基本思想来源于二进制拆分优化。对于一个正整数 $x$,它的二进制表示为 $a_{k - 1}, a_{k - 2}, \cdots , a_1, a_0$。可以将 $x$ 用二进制为 $1$ 的位表示出来,$x = 2^{i_1} + 2^{i_2} + \cdots + 2^{i_{k - 1}} + 2^{i_k}$。

其中 $i_1 > i_2 > \cdots > i_k$,可以将 $x$ 划分为 $\mathcal{O}(\left\lceil \log_2 x \right\rceil)$ 个区间。

\begin{enumerate}
\item 长度为 $2^{i_k}$ 的区间 $[x - 2^{i_k} + 1 , x]$;
\item 长度为 $2^{i_{k - 1}}$ 的区间 $[x - 2^{i_k} - 2^{i_{k - 1}} + 1, x - 2^{i_k}]$;
\item 长度为 $2^{i_{k - 2}}$ 的区间 $[x - 2^{i_k} - 2^{i_{k - 1}} -2^{i_{k - 2}} + 1, x - 2^{i_k} - 2^{i_{k - 1}}]$; \\
$\cdots$
\item 长度为 $2^{i_{1}}$ 的区间 $[x - 2^{i_k} - 2^{i_{k - 1}} -2^{i_{k - 2}} - \cdots -2^{i_1} + 1, x - 2^{i_k} - 2^{i_{k - 1}} - \cdots - 2^{i_2}]$。
\end{enumerate}

例如 $x = 7$,可以表示为 $2^2+2^1+2^0$,区间 $[1, 7]$ 可以分解成 $[1, 4]$、$[5, 6]$、$[7, 7]$ 三个区间。长度分别为 $2^2$、$2^1$、$2^0$。将这三个区间分别用二进制表示出来 $[1, 4] = [(1, 100)_2]$、$[5, 6] = [(101, 110)_2]$、$[7, 7] = [(111, 111)]$。可以发现每个区间的长度就是每个区间的右端点\textbf{二进制表示下最后一位 $1$ 及其后边的所有的 $0$。}就拿 $[5, 6]$ 这个区间举例,二进制表示下右端点为 $(110)_2$,最后一位 $1$ 及后面的所有的 $0$ 就是 $(10)_2 = (2)_{10}$,其区间长度正好为 $2$。

进而引出了 $\tt lowbit$ 操作。

$\tt lowbit$ 操作就是求一个数二进制表示下最后一位 $1$ 及其后边的所有的 $0$ 的数值。

\begin{lstlisting}[language=cpp]
int lowbit(x)
{
    return x & -x;
}
\end{lstlisting}

拿 $(20)_{10}$ 来举例,二进制表示下为 $(10100)_2$,最后一位 $1$ 及其后边的所有的 $0$ 就是 $(100)_2$,转化为十进制后就是 $4$,所以若调用 \verb|lowbit(20)|,则会返回 $4$。