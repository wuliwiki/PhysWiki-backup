% Sylow定理
% 西罗定理|西罗子群|Sylow子群|群论|有限群

\pentry{群作用\upref{Group3}}



拉格朗日定理(\autoref{coset_the2}~\upref{coset})揭示了子群阶数的特点.可惜的是,其逆命题“如果$n\mid \abs{G}$则$G$总有阶数为$n$的子群”是不成立的.比如说,交错群$A_4$就没有$6$阶的子群——你可以动手验证这一点.

但是,挪威数学家Peter Ludwig Sylow于1872年发表的文章\footnote{L. Sylow, \href{https://eudml.org/doc/156588}{Théorèmes sur les groupes de substitutions}, Mathematische Annalen 5 (1872), 584–594. 对此文的英文翻译见 \href{http://www.maths.qmul.ac.uk/~raw/pubs_files/Sylow.pdf}{Theorems on groups of substitutions}.}指出,在$n$是素数或者素数的幂时,拉格朗日定理逆命题是成立的.同时他还发现了所谓的Sylow子群全都是彼此共轭的.

举个例子:考虑阶数为$300$的群$G$,对$300$进行素因子分解得$300=2^2\cdot 3^2\cdot 5^2$,那么阶数为$2^2$的子群总存在,且它们彼此共轭.不过,$2$阶子群虽存在,却不能总保证所有$2$阶子群彼此共轭.


\begin{definition}{Sylow子群}\label{Sylow_def1}
给定群$G$和它的子群$H$.如果$\abs{H}=p^k$,其中$p$是素数,且$p^{k+1}\not\mid\abs{G}$,那么称$H$是$G$的一个$p$\textbf{-Sylow 子群},或\textbf{Sylow-}$p$\textbf{子群},或直接统称为\textbf{Sylow子群}.

群$G$的全体Sylow $p$-子群构成的集合,记为$\opn{Syl}_p(G)$.
\end{definition}



\begin{exercise}{}
考虑循环群$C_{12}$,求它的所有Sylow子群.
\end{exercise}

\begin{example}{}
考虑置换群$S_4$.则
\begin{equation}
C=\{1, \pmat{1&2}\pmat{3&4}, \pmat{1&3}\pmat{2&4}, \pmat{1&4}\pmat{2&3}, \pmat{1&2}, \pmat{3&4}, \pmat{1&3&2&4}, \pmat{1&4&2&3}\}
\end{equation}
是它的一个Sylow-$2$子群.

$C$是$\pmat{1&2}\pmat{3&4}$的中心,因此我们可以类似构造出$S_4$的剩下两个Sylow-$2$子群,分别是$\pmat{1&3}\pmat{2&4}$和$\pmat{1&4}\pmat{2&3}$的中心.这三个Sylow-$2$子群的交集是$V_4=\{1, \pmat{1&2}\pmat{3&4}, \pmat{1&3}\pmat{2&4}, \pmat{1&4}\pmat{2&3}\}$.

如果把一个正方形的四个顶点顺时针依次编号为$1, 3, 2, 4$,则不难看出$C$实际上就是正方形的对称群$D_4$.类似地,$S_4$的每个Sylow-$2$子群都同构于$D_4$.

\end{example}

Sylow定理通常被拆分成三个部分来表述.

\begin{theorem}{Sylow第一定理}
取\textbf{有限群}$G$.如果素数$p\mid\abs{G}$,那么$G$一定有一个Sylow $p$-子群$H_p$.
\end{theorem}

\textbf{证明}:

令$\abs{G}=p^km$,其中$p\not\mid m$.

先证明当$G$是阿贝尔群时,$p$阶子群存在.

任取$g\in G$,如果$g$的阶(\autoref{coset_def1}~\upref{coset})$\opn{ord}g$是$p$的倍数$np$,那么$g^n$生成的循环群就是$p$阶的.否则,设$\opn{ord}g=r$,取$g$生成的子群$H_g$,它是$G$的正规子群(因为$G$阿贝尔),于是$\abs{G/H_g}=p^km/r$.在$G/H_g$中再任挑一个元素,如果该元素的阶\textbf{不是}$p$,同样求出其循环群后用$\abs{G/H_g}$去除掉这个循环群,得到商群.易证有限步后,总能得到一个阶数为$p$\textbf{的倍数}的元素\footnote{注意,这里说的\textbf{阶数}逻辑上有跳步.比如说,在$G/H_g$中找到了一个$h$,其阶数为$sp$,那这只能说明$h^{sp}\in H_g$.不过由于$h^{sp}$本身也是有限阶的,因此$h$在$G$中的阶数也确实是$p$的倍数.}.于是这个元素的一个幂生成的循环群是$p$阶的.

下设$G$是任意的有限群,$Z(G)$是它的中心.设定理对于阶数小于$p^km$的群都成立,然后分类进行归纳讨论.

如果$p\mid \abs{Z(G)}$,那么由于$Z(G)$是阿贝尔群,故存在$G$的$p$阶子群$A$.由于$A\subseteq Z(G)$,故$A\vartriangleleft G$.于是得到商群$G/A$,其阶数为$p^{k-1}m$.由归纳假设,$G/A$有一个$p^{k-1}$阶Sylow子群,因此这个子群是$G$的$p^k$阶子群,从而是$G$的Sylow $p$-子群.

如果$p\not\mid \abs{Z}(G)$,那么根据共轭类等式(The Class Equation\autoref{Group3_the4}~\upref{Group3}),
\begin{equation}
\abs{G} = \abs{Z(G)} + \sum_{C\in O, \abs{C}>1} \abs{C}
\end{equation}
其中$O$是$G$在伴随作用下全体轨道构成的集合.

由于$p\mid \abs{G}$,$p\not\mid \abs{Z(G)}$,故存在$\abs{C}$使得$p\not\mid \abs{C}$,即存在$x\in G$使得$p\not\mid \abs{C_x}$,故$\abs{C_x}\mid m$.由\autoref{Group3_cor3}~\upref{Group3},得$\abs{C_G(x)}=p^km/\abs{C_x}=p^ks<\abs{G}$\footnote{$C_G(x)$即$x$的中心化子,或者说$x$在伴随作用下的迷向子群.},其中$s>1$.因此由归纳假设,$C_G(x)$有一个$p^k$阶子群$A$,从而$G$有个Sylow $p$-子群$A$.

\textbf{证毕}.





\begin{theorem}{Sylow第二定理}\label{Sylow_the1}
取\textbf{有限群}$G$,它的所有Sylow $p$-子群彼此共轭.
\end{theorem}

\textbf{证明}:

令$\abs{G}=p^km$,其中$p\not\mid m$.

任取$G$的两个Sylow $p$-子群$A$和$B$,记$G/B$是$B$在$G$中的左陪集构成的集合\footnote{不是商群,因为不能保证$B\vartriangleleft G$.}.考虑群$A$作用在集合$G/B$上,方式是\textbf{左伴随作用}.

据\autoref{Group3_the5}~\upref{Group3},
\begin{equation}
\abs{G/B} = \abs{\opn{Fix}_A(G/B)} \mod p
\end{equation}

由Sylow子群的定义,知$\abs{G/B}=m$.故$m\abs{\opn{Fix}_A(G/B)} \mod p$,意味着$\abs{\opn{Fix}_A(G/B)}\neq 0$,即\textbf{存在不动点}$gB\in G/B$.

因此$agB=gB$对所有$a\in A$成立,即$ag\in gB$,即$A\subseteq gBg^{-1}$.又因为$\abs{A}=p=\abs{B}=\abs{gBg^{-1}}$,故$A=gBg^{-1}$.


\textbf{证毕}.




\begin{theorem}{Sylow第三定理}\label{Sylow_the2}
取\textbf{有限群}$G$,设$n_p$是它的所有Sylow $p$-子群的数目.令$\abs{G}=p^km$,其中$p\not\mid m$.

那么$n_p\equiv 1\mod p$,或者说$p\mid n_p-1$;且$n_p\mid m$.
\end{theorem}

\textbf{证明}:

如果$\opn{Syl}_p(G)$只有一个元素,即$n_p=1$,那定理自然成立.下设$n_p>1$.


证明第一个性质$n_p\equiv 1\mod p$:

任取$P\in\opn{Syl}_p(G)$,令$P$在$\opn{Syl}_p(G)$上有\textbf{共轭}作用.这个作用的合理性由\autoref{Sylow_the1} 保证.

由于$n_p>1$,故可取$Q\in \opn{Syl}_p(G)-\{P\}$.由\autoref{Sylow_the1} ,存在$g\in G$使得$Q=gPg^{-1}$.

$P\neq Q$且$\abs{P}=\abs{Q}=p^k$ $\implies$存在$h\in P-Q$使得$ghg^{-1}\in Q-P$.于是$g^{-1}(ghg^{-1})g=h\in P$,这意味着$g^{-1}Qg\neq Q$.因此,$\opn{Fix}_P(\opn{Syl}_p(G))$只有$P$这一个元素,从而$\abs{\opn{Fix}_P(\opn{Syl}_p(G))}=1$.

由\autoref{Group3_the5}~\upref{Group3},注意$\abs{P}=p^k$,
\begin{equation}\label{Sylow_eq1}
\begin{aligned}
n_p=\abs{\opn{Syl}_p(G)}\equiv\abs{\opn{Fix}_P(\opn{Syl}_p(G))}\mod p\equiv 1\mod p
\end{aligned}
\end{equation}


证明第二个性质$n_p\mid m$:

考虑整个$G$按\textbf{共轭}作用,作用在$\opn{Syl}_p(G)$上.由\autoref{Sylow_the1} ,这是一个可递作用,因此只有\textbf{一个}轨道,因此任意元素的轨道中元素数量是$n_p$.由\autoref{Group3_cor2}~\upref{Group3},$n_p$等于$\opn{Syl}_p(G)$中任意元素的迷向子群\textbf{指数}.又由\textbf{拉格朗日定理}(\autoref{coset_the2}~\upref{coset})可知子群指数整除群的阶,即$n_p\mid p^km$.

由\autoref{Sylow_eq1} ,$n_p\not \mid p^k$,从而$n_p\mid m$.


\textbf{证毕}.

\autoref{Sylow_the2} 中第二个性质的证明过程中也蕴含了这一关系:$n_p=[G:\opn{Syl}_p(G)]$(子群的指数,见\autoref{coset_def2}~\upref{coset}).


\begin{theorem}{}
取\textbf{有限群}$G$,令$P$是$G$的一个Sylow $p$-子群,$N(P)$是$P$的正规化子(\autoref{NormSG_def2}~\upref{NormSG}).

则对$G$的子群$H\supseteq N(P)$,必有$N(H)=H$.
\end{theorem}

\textbf{证明}:

显然$H\subseteq N(H)$.下证$N(H)\subseteq H$.

由于$P\subseteq N(P)\subseteq H$,由定义显然可知,$P$也是$H$的一个Sylow $p$-子群.

任取$x\in N(H)$,则$xPx^{-1}\subseteq H$也是$H$的一个Sylow $p$-子群.于是由\autoref{Sylow_the1} ,存在$h\in H$使得$xPx^{-1}=hPh^{-1}$.

于是$h^{-1}xPx^{-1}h=P$,这意味着$h^{-1}x\in N(P)\subseteq H$.故$x\in H$,即$N(H)\subseteq H$.

\textbf{证毕}.














