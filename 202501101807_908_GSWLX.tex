% 干涉 (物理学)(综述)
% license CCBYSA3
% type Wiki

本文根据 CC-BY-SA 协议转载翻译自维基百科\href{https://en.wikipedia.org/wiki/Wave_interference}{相关文章}。

对于无线电通信中的干扰,请参阅《干扰(通信)》。
“干涉图样”在此重定向。有关莫尔条纹,请参阅《莫尔条纹》。对于医学术语,请参阅《干涉图样(肌电图)》
\begin{figure}[ht]
\centering
\includegraphics[width=10cm]{./figures/1379a86de5dd6781.png}
\caption{两波的干涉。相位相同:两个较低的波组合(左侧面板), resulting in a wave of added amplitude(建设性干涉)。相位相反:(这里是180度),两个较低的波组合(右侧面板), resulting in a wave of zero amplitude(破坏性干涉)。} \label{fig_GSWLX_1}
\end{figure}
在物理学中,干涉是指两种相干波通过考虑它们的相位差,将它们的强度或位移相加的现象。 如果两波处于相位相同或相反的状态,所产生的波可能具有更大的强度(建设性干涉)或较小的振幅(破坏性干涉)。干涉效应可以在所有类型的波中观察到,例如光波、无线电波、声波、表面水波、引力波或物质波,以及扬声器中的电波。
\begin{figure}[ht]
\centering
\includegraphics[width=10cm]{./figures/f6e23b7411ed2318.png}
\caption{湖面上的干涉水波} \label{fig_GSWLX_2}
\end{figure}
\subsection{词源}  
“干涉”一词源自拉丁语单词 \textbf{inter},意思是“之间”,以及 \textbf{fere},意思是“撞击或打击”,该词在波的叠加上下文中由托马斯·杨于1801年首次使用。[1][2][3]
\subsection{机制}
\begin{figure}[ht]
\centering
\includegraphics[width=10cm]{./figures/17e646f8dccde141.png}
\caption{在二维空间中,右行(绿色)波和左行(蓝色)波的干涉,最终形成(红色)波。} \label{fig_GSWLX_3}
\end{figure}
波的叠加原理指出,当两个或更多相同类型的传播波在同一点相遇时,该点的合成振幅等于各个波的振幅的矢量和。[4] 如果一个波的波峰与另一个同频率波的波峰在同一点相遇,那么振幅就是各个振幅的总和——这就是建设性干涉。如果一个波的波峰与另一个波的波谷相遇,那么振幅等于各个振幅的差——这就是破坏性干涉。在理想介质中(如水、空气几乎是理想介质),能量始终是守恒的,在破坏性干涉的点,波的振幅互相抵消,能量会重新分布到其他区域。例如,当两颗小石子掉进池塘时,会观察到一定的波纹图案,但最终波动会继续传播,只有当波到达岸边时,能量才会从介质中被吸收。
\begin{figure}[ht]
\centering
\includegraphics[width=8cm]{./figures/047b21019d382954.png}
\caption{来自两个点源的波的干涉。} \label{fig_GSWLX_4}
\end{figure}
建设性干涉发生在波之间的相位差是π的偶数倍(180°)时,而破坏性干涉发生在相位差是π的奇数倍时。如果相位差介于这两个极端之间,那么合成波的位移幅度将在最小值和最大值之间。

例如,考虑当两颗相同的石子分别从不同地点投入静止的水池时发生的情况。每颗石子都会从掉落点向外传播出一个圆形波。当两波重叠时,特定点的净位移是各个波的位移之和。在某些点上,这些波会处于同相位,产生最大的位移。在其他地方,波会处于反相位,这些点的净位移为零。因此,水面上的一些部分将保持静止——如上图和右图所示,静止的蓝绿色线条从中心辐射出去。
\begin{figure}[ht]
\centering
\includegraphics[width=10cm]{./figures/5a62fc4dc06a85f2.png}
\caption{这是在白光下拍摄的1.5厘米 x 1厘米肥皂膜区域的照片。膜的厚度和观察几何角度的变化决定了哪些颜色会发生建设性或破坏性干涉。小气泡显著影响周围膜的厚度。} \label{fig_GSWLX_5}
\end{figure}
光的干涉是一种独特的现象,因为我们无法像水波那样直接观察到电磁场的叠加。电磁场的叠加是一个假设的现象,且必须用来解释为什么两束光线能够穿透彼此并继续沿各自的路径传播。光干涉的典型例子包括著名的双缝实验、激光散斑、抗反射涂层和干涉仪。

除了经典的波动模型来理解光学干涉外,量子物质波也展示了干涉现象。
\subsubsection{实值波函数} 
上述可以通过推导两个波的和公式在一维中展示。沿x轴向右传播的正弦波的振幅方程为:  
\[
W_1(x,t) = A \cos(kx - \omega t)~
\]
其中,\(A\) 是峰值振幅,  
\(k = \frac{2\pi}{\lambda}\) 是波数,  
\(\omega = 2\pi f\) 是波的角频率。  
假设第二个频率和振幅相同,但相位不同的波也向右传播:  
\[
W_2(x,t) = A \cos(kx - \omega t + \varphi)~
\]
其中,\(\varphi\) 是两个波之间的相位差(以弧度表示)。这两个波将会叠加并相加:  
\[
W_1 + W_2 = A \left[ \cos(kx - \omega t) + \cos(kx - \omega t + \varphi) \right]~
\]
使用两个余弦和的三角恒等式:  
\[
\cos a + \cos b = 2 \cos \left( \frac{a - b}{2} \right) \cos \left( \frac{a + b}{2} \right)~
\]
可以将其写为:  
\[
W_1 + W_2 = 2A \cos \left( \frac{\varphi}{2} \right) \cos \left( kx - \omega t + \frac{\varphi}{2} \right)~
\]

这表示一个以原始频率传播的波,向右传播,像它的分量一样,其振幅与 \(\cos(\varphi/2)\) 成正比。

建设性干涉:如果相位差是π的偶倍数:\(\varphi = \ldots, -4\pi, -2\pi, 0, 2\pi, 4\pi, \ldots\)则\(|\cos(\varphi / 2)| = 1\)因此,两个波的和是一个振幅是原来两倍的波:  
\[
W_1 + W_2 = 2A \cos(kx - \omega t)~
\]

破坏性干涉:如果相位差是π的奇倍数:\(\varphi = \ldots, -3\pi, -\pi, \pi, 3\pi, 5\pi, \ldots
\]
则
\[
\cos(\varphi / 2) = 0
\]
因此,两个波的和为零:  
\[
W_1 + W_2 = 0~
\]