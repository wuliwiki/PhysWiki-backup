% 东南大学 2012 年 考研 量子力学
% license Usr
% type Note

\textbf{声明}:“该内容来源于网络公开资料,不保证真实性,如有侵权请联系管理员”

\textbf{1.(15 分)}以下叙述是否正确:(1)电子的自旋态空间是3维的:(2)全同玻色子体系的波函数具有交换反对称性;(3)时间反演对称性导致能量守恒;(4)三维各向同性谐振子的所有能级均是非简并的;(5)处于中心力场中的无自旋单粒子的角动量一定是守恒量。

\textbf{2.(15 分)}质量为 $m$ 的粒子处于以下势阱中:
\[
V(x) = m \omega^2 x^2/2, \quad (x > 0);~
\]
\[
V(x) = \infty, \quad (x < 0)~
\]
试求能量本征值。

\textbf{3.(15 分)}质量为 $m$ 的粒子以能量 $E > 0$ 从左入射,碰到势 $V(x) = \gamma \delta(x) (\gamma > 0)$。

\begin{enumerate}
    \item 试用公式
    \[
    j(x,t) = -(i\hbar/2m)(\psi^* \partial \psi/\partial x - \psi \partial \psi^*/\partial x)~
    \]
    求入射几率流密度 $j_i$,反射几率流密度 $j_r$,透射几率流密度 $j_t$ 的表达式;
    
    \item 试证明波函数 $\psi$ 满足
    \[
    \psi'(0^+) - \psi'(0^-) = (2m\gamma/\hbar^2)\psi(0);~
    \]
    
    \item 求透射系数 $t$。
\end{enumerate}

\textbf{4.(15分)}试利用测不准关系估算:(1)一维谐振子的基态能:(2)氢原子的基态能。


\textbf{5.(15分)}粒子的轨道角动量算符定义为:
\[
\hat{L}_x = \hat{y}\hat{p}_z - \hat{z}\hat{p}_y, \quad \hat{L}_y = \hat{z}\hat{p}_x - \hat{x}\hat{p}_z, \quad \hat{L}_z = \hat{x}\hat{p}_y - \hat{y}\hat{p}_x~
\]
试利用基本对易关系 $[\hat{x}_\alpha, \hat{p}_\beta] = i\hbar \delta_{\alpha\beta}$ 求对易式:
\[
[\hat{l}_x, \hat{y}] \quad [\hat{l}_x, \hat{p}_y],\quad[\hat{l}_x, \hat{l}_y]~
\]

\textbf{6.(15分)}二维各向同性谐振子的哈密顿算符为:
\[
\hat{H} = -\frac{\hbar^2}{2m} \left( \frac{\partial^2}{\partial x^2} + \frac{\partial^2}{\partial y^2} \right) + \frac{1}{2} m \omega^2 (x^2 + y^2)~
\]
试求能量本征值及简并度。

\textbf{7.(15分)}一质量为 $m$,空间位置固定的电子处于沿 $x$ 方向的均匀磁场 $B$ 中,其哈密顿算符(不计轨道运动)为
\[
\hat{H} =(eB/mc)\hat{S}_x,~
\]
其中 $\hat{S}$ 为电子自旋角动量算符。已知 $t=0$ 时电子的自旋态为 $S_s$ 的本征态,相应的本征值为 $\hbar / 2$,试求:

\begin{enumerate}
    \item $t(>0)$ 时刻该电子的自旋态。
    \item $t(>0)$ 时刻 $S_s$ 的平均值。
\end{enumerate}

提示:泡利矩阵为
\[
\sigma_x = \begin{pmatrix}
0 & 1 \\
1 & 0
\end{pmatrix}, \quad
\sigma_y = \begin{pmatrix}
0 & -i \\
i & 0
\end{pmatrix}, \quad
\sigma_z = \begin{pmatrix}
1 & 0 \\
0 & -1
\end{pmatrix}.~
\]

\textbf{8.(15分)}设锂金属原子的价电子的哈密顿算符为 
\[
\hat{H}_0 = (\hat{p}^2/2\mu) + V(r),~
\]
守恒量完全集合 $\{\hat{H}_0, \hat{L}^2, \hat{L}_z\}$ 的共同本征态为 $\lvert n,l m \rangle$,能量本征值为 $E_{n_rl}$。

\begin{enumerate}
    \item 若沿 $z$ 方向外加磁场 $B$,则价电子的哈密顿算符变为 
    \[
    \hat{H} = \hat{H}_0 + \omega \hat{L}_z \quad (\omega_L = eB/2\muc),~
    \]
    试求相应的能量本征态和能量本征值。
    
    \item 求存在强磁场 $B$ 时,原子从 $p$ 态 ($l = 1$) 跃迁到 $s$ 态 ($l = 0$) 的光谱线频率(跃迁过程中 $n_l$ 不变)。
\end{enumerate}