% 相对论速度变换
% 相对论|光速|洛伦兹变换
\pentry{斜坐标表示洛伦兹变换\upref{SROb}}
沿用铁轨系$K_1$和火车系$K_2$的设定,令$K_2$相对$K_1$以速度$v$运动,以直角坐标系表示$K_1$,相应地用斜坐标系表示$K_2$.如果一个点在$K_1$中以速度$u$运动,那么它在直角坐标系中扫过一条直线,称为它的\textbf{世界线};这在斜坐标系中也算一条直线.因此,这个点在$K_2$中也是以匀速运动的,记为$u'$.我们希望计算出$u'$相对于$u$的关系.


\subsection{几何解法}

如图所示,$K_1$系的坐标为$x$和$t$,$K_2$系的坐标为$x'$和$t'$,由题设有$\tan{\theta}=v$.图中蓝线表示所讨论点的运动轨迹,因此$\tan{\varphi}=u$.我们的目标是计算出蓝线在$K_2$中的斜率,也就是$\frac{a}{b}$.


\begin{figure}[ht]
\centering
\includegraphics[width=5cm]{./figures/RelVel_1.pdf}
\caption{几何解相对论速度变换的示意图.} \label{RelVel_fig1}
\end{figure}

由正弦定理,$\frac{a}{b}=\frac{\sin{\alpha}}{\sin{\beta}}$.其中$\alpha=\varphi-\theta$,$\beta=\frac{\pi}{2}-\theta-\varphi$.则

\begin{equation}
\begin{aligned}

\frac{\sin{\alpha}}{\sin{\beta}}\\
&=\frac{\sin{(\varphi-\theta)}}{\sin{(\frac{\pi}{2}-\theta-\varphi)}}\\
&=\frac{\sin{(\varphi-\theta)}}{\cos{(\varphi+\theta)}}\\
&=\tan(\varphi-\theta)\cdot\frac{\cos(\varphi-\theta)}{\cos(\varphi+\theta)}\\
&=\tan(\varphi-\theta)\cdot\frac{\cos\varphi\cos\theta+\sin\varphi\sin\theta}{\cos\varphi\cos\theta-\sin\varphi\sin\theta}\\
&=\tan(\varphi-\theta)\cdot\frac{1+\tan\varphi\tan\theta}{1-\tan\varphi\tan\theta}\\
&=

\end{aligned}
\end{equation}








