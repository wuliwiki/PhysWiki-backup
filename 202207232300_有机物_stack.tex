% 栈
% 栈|数据结构|C++

栈是一种“先进后出”的数据结构.栈只有一端可以进出元素,这一段被称为“栈顶”,另一端被称为“栈底”.往栈中插入元素被称为”进栈“,往栈中删除元素被称为“出栈”.

栈是计算机实现递归和基本结构.

C++ 的 STL 已经帮助我们实现好了栈,一般情况我们可以直接使用 STL 库里的栈.

栈的常用操作:
\begin{enumerate}
\item 向栈顶插入一个数 $x$;
\item 从栈顶弹出一个数;
\item 判断栈是否为空;
\item 查询栈顶元素.
\end{enumerate}

C++ STL
\begin{lstlisting}[language=cpp]
stack<int> stk;
stk.push(x);
stk.pop();
if (stk.empty()) // 为空则为 true
cout << stk.top() << endl;
\end{lstlisting}

但我们在这里详细的讲一下如何使用数组模拟栈.

定义一个数组用于模拟栈,再定义一个变量表示栈顶,初始化为 $-1$