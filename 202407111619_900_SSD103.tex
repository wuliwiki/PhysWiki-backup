% 首都师范大学 2003 年硕士入学物理考试试题(一)
% keys 首都师范大学|考研|物理|2003年
% license Copy
% type Tutor

\textbf{声明}:“该内容来源于网络公开资料,不保证真实性,如有侵权请联系管理员”


\begin{enumerate}
\item 傲微幅振动的单摆悬吊在电梯的天花板上。求电梯加速上升和匀速上升两种情况下的单摆振动周期差。
\item 小球自一斜面上运动到底部,下落高度为h。分两种情况考虑:\\
(1)如果将小球视为不受摩擦力的质点;\\
(2)将小球视为无滑滚动的刚体,求在这两种情况下的小球质心位于底部时的速度差。
\item 孤立系中有两个质量任意的质点,它们的初始速度不在一直线上,它们相互间的作用力满足万有引力定律。说明系统满足什么守恒定律,以及两个质点的运动情况。
\item 一均匀带电细棒弯成半径为R的半园环,棒上总电量为+Q。求:圆心o处的电场强
度 $\vec{E}$ 和电位U。\begin{figure}[ht]
\centering
\includegraphics[width=8cm]{./figures/df40f4f0232383a7.png}
\caption{} \label{fig_SSD103_1}
\end{figure}
\item 一个正点电荷Q放在一个内半径为$R_1$,外半径为$R_2$的电介质球壳中心(如图所示),电介质的相对介电常数为$\varepsilon_r$,求:\\
(1)$r<R_1,R_1<r<R_2,r<R_2$,各空间的电位移矢量$\vec D$、电场强度矢量$\vec{E}$、极化强度矢量 $\vec P$ 分布。\\
(2)$r<R_1$空间的点位分布及 $r=R_2$ 面上的极化电荷密度 $\sigma$'。\\
(3)储存在 $R_1<r<R_2$ ,空间的静电场能量。
\begin{figure}[ht]
\centering
\includegraphics[width=6cm]{./figures/39280c599d24c023.png}
\caption{} \label{fig_SSD103_2}
\end{figure}
\item 一无限长载流导线弯成图示形状,电流为I。$\frac{2}{3}$圆弧的半径为 $R$,圆心在o点。求:该载流导线在圆心处产生的磁感应强度 $\vec B$。
\begin{figure}[ht]
\centering
\includegraphics[width=8cm]{./figures/60eed05276edf8d8.png}
\caption{} \label{fig_SSD103_3}
\end{figure}
\item 一根长直导线载有电流I,另有两根平行放置的导体棒,一端接有电阻R,另一端有可滑动的导体ab ,当ab以匀速 $\vec V$向右滑动时(如图)求:\\
(1)感应电动势的大小及回路中的电流的方向。\\
(2)作用在 ab 上的磁场力。
\begin{figure}[ht]
\centering
\includegraphics[width=8cm]{./figures/976e2695e25477b5.png}
\caption{} \label{fig_SSD103_5}
\end{figure}
\item 在每个括号内填写√或x。\\
1.启发就是要通过提出问题来组织教学(),它适合于用讲授法组织教学而不适用于研究性学习或自主性学习的方式组织教学();\\
2.复习的目的并不只限应试(),而考试的目的则就是是检查与选拔();\\
3.概念教学、规律教学、习题教学是中学物理教学的三个支柱( ),所以我们必须重视习题教学。培养了学生的物理解题思维能力,也就是培养了学生的物理思维能力()。
\item 1.物理教材的单元教材分析要分析哪些基本内容?\\
2.除了培养实验操作技能和能力之外,物理实验教学还有哪些教学功能?\\
\item 试从知识与能力的概念区别和掌握知识与培养能力在过程上的不同,分析在物理课堂教学组织中如何落实对学生思维能力的培养。
\item 试分析如何在物理课堂教学中进行科学教育。
\end{enumerate}
