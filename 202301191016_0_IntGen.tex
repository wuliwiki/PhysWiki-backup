% 一般积分
% 一般积分|常微分方程
\pentry{基本知识(常微分方程)\upref{ODEPr}}
\footnote{参考斯米尔诺夫《高等数学》第2卷第1分册}下面将介绍常微分方程的一般积分,它是常微分方程解的一般形式。物理学中“运动积分”(或“运动方程的积分”)(运动积分\upref{motint})和“循环积分”中的“积分”之所以叫“积分”的原因,可以在本词条得到解答。

“积分”的概念源于微积分,“积分”和“导数”彼此对应:“你”是“我”的导数,“我”就是“你”的积分,反之亦然。微分方程是关于未知函数导数的方程,其主要目的是要求得该未知函数,满足微分方程的未知函数称为该方程的解,所以微分方程的解当然就是微分方程中对应的导数的积分。由于这样的理由,人们也把微分方程的解叫作微分方程的\textbf{积分}。
\subsection{$n$ 阶常微分方程的一般积分}
$n$ 阶常微分方程一般形式为(\autoref{ODEPr_eq6}~\upref{ODEPr}):
\begin{equation}\label{IntGen_eq1}
F \qty(x,y,y',\cdots,y^{(n)}) =0
\end{equation}
或写成(\autoref{GO2SOD_eq2}~\upref{GO2SOD})
\begin{equation}\label{IntGen_eq2}
y^{(n)}=f(x,y,y',\cdots,y^{(n-1)})
\end{equation}
自变量 $x$ 的任何函数,若满足方程\autoref{IntGen_eq1} 或\autoref{IntGen_eq2} ,就叫作这方程的\textbf{解};而微分方程的求解问题也叫作求微分方程的\textbf{积分问题}。

对于 $n$ 阶微分方程\autoref{IntGen_eq1} 或\autoref{IntGen_eq2} ,有\textbf{存在与唯一定理},它可以叙述为:若函数 $f(x,y,y',\cdots,y^{(n-1)})$ 是 $(x,y,y',\cdots,y^{(n-1)})$ 的单值函数,且在 $(x_0,y_0,y_0',\cdots,y_0^{(n-1)})$ 的一邻域内,$f$ 连续且有对 $y,y',\cdots,y^{(n-1)}$ 的一阶连续偏微商,则满足初始条件
\begin{equation}\label{IntGen_eq5}
y|_{x=x_0}=y_0,y'|_{x=x_0}=y_0',\cdots,y^{(n-1)}|_{x=x_0}=y_0^{(n-1)}
\end{equation}
的解存在且唯一。

改变初始条件中的常数 $y_0,y_0',\cdots,y_0^{(n-1)}$ ,就可以得到微分方程\autoref{IntGen_eq1} 或\autoref{IntGen_eq2} 的无穷多个解。这就是说,微分方程的解依赖于 $n$ 个任意常数。一般来说,这 $n$ 个任意常数不一定以初始条件的形式在解中出现,而以一般的形式出现\footnote{关于这一点,可以这样理解:当 $n$ 个任易常数确定时,解是唯一的,所以 $n$ 阶微分方程的解具有 $n$ 个自由度,故解应以任意形式依赖于这 $n$ 个任意常数。而以初始条件出现的 $n$ 个数可以看成表示这 $n$ 个任意常数的参数 $C_1,\cdots,C_n$ 的函数,当 $C_1,\cdots,C_n$ 确定时,就确定了对应的初始值}:
\begin{equation}\label{IntGen_eq3}
y=\varphi(x,C_1,\cdots,C_n)
\end{equation}
\begin{definition}{一般积分}
微分方程\autoref{IntGen_eq2} 的这样的含有 $n$ 个任意常数的解\autoref{IntGen_eq3} ,叫作方程\autoref{IntGen_eq2} 的\textbf{一般积分}。
\end{definition}
一般积分\autoref{IntGen_eq3} 也可以写成隐示式
\begin{equation}\label{IntGen_eq4}
\psi(x,y,C_1,\cdots,C_n)=0
\end{equation}
给常数 $C_1,\cdots,C_n$ 以确定的值,就得到方程的一个\textbf{特殊解}。

由方程\autoref{IntGen_eq3} 或\autoref{IntGen_eq4} 对 $x$ 求导,直到 $n-1$ 阶,再用 $x=x_0$ 及初始条件\autoref{IntGen_eq5} 代入,就得到 $n$ 个方程。由这 $n$ 个方程,就可以确定出对应初始条件\autoref{IntGen_eq5} 的一般积分\autoref{IntGen_eq3} 中的任意常数 $C_1,\cdots,C_n$。
\subsection{常微分方程组的一般积分}
根据\autoref{GO2SOD_the2}~\upref{GO2SOD},一般的常微分方程组都等价于一阶的常微分方程组(标准常微分方程组\autoref{BscAlg_def1}~\upref{BscAlg})。这就是说,我们只需要讨论下面 $n$ 个一阶微分方程的方程组
\begin{equation}\label{IntGen_eq6}
y_i'=f_i(x,y_1,\cdots,y_n),\quad i=1,\cdots,n
\end{equation}
像一个方程的情形一样,也有\textbf{存在与唯一定理},其可叙述为:若函数
\begin{equation}
f_i(x,y_1,\cdots,y_n),\quad i=1,\cdots,n
\end{equation}
在 $(x_0,y_1^{(0)},y_2^{(0)},\cdots,y_n^{(0)})$ 的一邻域内连续且有对 $y,y_1,\cdots,y_n$ 的一阶连续偏微商,则满足初始条件
\begin{equation}\label{IntGen_eq8}
y_1|_{x=x_0}=y_1{(0)},y_2|_{x=x_0}=y_2^{(0)},\cdots,y_n|_{x=x_0}=y_n^{(0)}
\end{equation}
的解 $y_i=\omega_i(x)$ 存在且唯一。

和一个方程时一样的理由,方程组\autoref{IntGen_eq6} 的解依赖于 $n$ 个任意常数,这些常数以一般的形式出现在解中:
\begin{equation}\label{IntGen_eq7}
y_i=\psi_i(x,C_1,\cdots,C_n),\quad i=1,\cdots,n
\end{equation}
\begin{definition}{微分方程组的一般积分}
微分方程组\autoref{IntGen_eq6} 的这样的含有 $n$ 个任意常数的解\autoref{IntGen_eq7} ,叫作方程组\autoref{IntGen_eq6} 的\textbf{一般积分}。
\end{definition}
给常数 $C_1,\cdots,C_n$ 以确定的值,就得到方程组\autoref{IntGen_eq6} 的\textbf{特殊解}。若代入初始条件于 \autoref{IntGen_eq7} ,就可由 $n$ 个方程组
\begin{equation}
y_i^{(0)}=\psi_i(x_0,C_1,\cdots,C_n),\quad i=1,\cdots,n
\end{equation}
求出满足初始条件\autoref{IntGen_eq8} 的任意常数的值,再把这些值代入\autoref{IntGen_eq7} ,就得到满足初始条件\autoref{IntGen_eq8} 的解。

由\autoref{IntGen_eq7} ,可以得到微分方程组一般解的下面形状的公式:
\begin{equation}\label{IntGen_eq9}
\varphi_i(x,y_1,\cdots,y_n)=C_i,\quad i=1,\cdots,n
\end{equation}

于是,为了从微分方程组的积分求出方程组的一般积分,需要求出方程组的 $n$ 个这样的积分,才能由等式\autoref{IntGen_eq9} 解出解出 $y_1,\cdots,y_n$ 来。
\begin{definition}{微分方程组的积分}\label{IntGen_def1}
\autoref{IntGen_eq9} 中的每一个都叫作微分方程组\autoref{IntGen_eq6} 的一个\textbf{积分}。
\end{definition}
\textbf{上面的定义正是物理中的“运动积分”、“循环积分”的“积分”之涵义}。

\textbf{有时}我们说方程组的积分,不是指等式\autoref{IntGen_eq9} ,而是指函数 $\varphi_i(x,y_1,\cdots,y_n)$。就是说,若把方程组的任何解代入函数 $\varphi(x,y_1,\cdots,y_n)$ 中,$\varphi$ 成为常数,则函数 $\varphi$ 叫作这方程组的\textbf{积分}。这里 $\varphi$ 当然不是常数,因为解的初始条件是随意的,这个常数当然也是随意的。显然,方程组的一些积分的任意函数 $F(\varphi_1,\cdots,\varphi_k)$ 也是方程组的积分,这只是由\autoref{IntGen_eq9} 推得的,并非什么新结果。
