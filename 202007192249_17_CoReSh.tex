% 宇宙学红移与哈勃定律

\pentry{FRW 度规}
由于物体在宇宙中传播的过程中,宇宙也在加速膨胀,所以为了准确的测量物体在宇宙传播过程中的物理量,我们需要引进宇宙学红移的概念.

\subsection{光子的红移}
从量子力学的描述中,光的波长可以定义为$\lambda=\frac{h}{p}$.当光在$t_1$时刻发射以波长$\lambda_1$发射,于$t_0$时刻被接收所观测到的波长为
\begin{equation}
\lambda_0=\frac{a(t_0}{a(t_1)}\lambda_1,
\end{equation}
其中$a(t)$为$t$时刻的尺度因子.因为宇宙在加速膨胀$a(t_0)>a(t_1)$, 所以容易得$\lambda_0>\lambda_1$.

\subsection{红移因子}
为了计算方便,我们可以通过定义从星系发出的,经过一段时间到达地球后被观测的光的红移为\textbf{红移因子}(redshift parametre)
\begin{equation}
z=\frac{\lambda_0-\lambda_1}{\lambda_1},
\end{equation}
显然我们可以推出
\begin{equation}
1+z=\frac{a(t_0)}{a(t)}=\frac{1}{a(t_1)}.
\end{equation}
一般地我们设现在的宇宙尺度因子$a(t_0)=1$.

\subsection{哈勃定律(Hubble Law)}
我们把$t_1$时刻的宇宙尺度因子$a(t_1)$以现在的时刻$t_0$为原点作泰勒展开,可得
\begin{equation}
a(t_1)=a(t_0)(1+(t-t_0)H_0+\cdots)
\end{equation}
