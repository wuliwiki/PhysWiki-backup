% 阿德里安-马里·勒让德(综述)
% license CCBYSA3
% type Wiki

本文根据 CC-BY-SA 协议转载翻译自维基百科\href{https://en.wikipedia.org/wiki/Adrien-Marie_Legendre}{相关文章}。

\begin{figure}[ht]
\centering
\includegraphics[width=6cm]{./figures/ce46ca8eff284986.png}
\caption{} \label{fig_adla_1}
\end{figure}
阿德里安-马里·勒让德(Adrien-Marie Legendre,/ləˈʒɑːndər, -ˈʒɑːnd/;法语发音:[adʁiɛ̃ maʁi ləʒɑ̃dʁ];1752年9月18日-1833年1月9日)是一位法国数学家,对数学做出了众多贡献。以他命名的重要概念包括勒让德多项式和勒让德变换。他还因在最小二乘法上的贡献而著称,尽管卡尔·弗里德里希·高斯在他之前已发现这一方法,勒让德却是第一个正式发表该方法的人。
\subsection{生平}
阿德里安-马里·勒让德于1752年9月18日出生在巴黎的一个富裕家庭。他在巴黎马扎兰学院(Collège Mazarin)接受教育,并于1770年在物理与数学方面完成了论文答辩。1775年至1780年间,他在巴黎的军事学院(École Militaire)任教,1795年起又在国立高等师范学院(École Normale)任教。同时,他还隶属于法国经度局(Bureau des Longitudes)。1782年,柏林科学院因其关于阻力介质中抛体运动的论文授予勒让德奖项,这篇论文也使他引起了拉格朗日的注意。[5]

法国科学院于1783年任命勒让德为副成员,1785年成为正式院士。1789年,他被选为英国皇家学会会士(Fellow of the Royal Society)。[6]

他参与了英法联合测量项目(Anglo-French Survey, 1784–1790),该项目旨在通过三角测量法精确计算巴黎天文台与格林尼治天文台之间的距离。为此,他于1787年与多米尼克·卡西尼伯爵(Dominique, comte de Cassini)和皮埃尔·梅尚(Pierre Méchain)一同前往多佛和伦敦。他们三人还拜访了天王星的发现者威廉·赫歇尔(William Herschel)。
