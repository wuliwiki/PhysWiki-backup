% Landau-Ginzburg 理论
% keys 相变|超导|平均场
% license Usr
% type Wiki

\begin{issues}
\issueTODO
\end{issues}

\subsection{背景介绍}
Ginzburg 谈到,最初他想要解决在超导体中出现的温差电现象。当超导体出现温度梯度时,导体中会出现特殊的超导电流,并且涌现出电磁场。当时的伦敦方程仅仅能解释其中的部分现象,而且伦敦方程预测普通金属与超导金属的界面能是负的,因此当时急需一个非电磁场起源的界面能来解决温差电现象。到1950年,Landau-Ginzburg 理论解决了这一问题。\footnote{Ginzburg, Vitaly L. "Nobel Lecture: On superconductivity and superfluidity (what I have and have not managed to do) as well as on the “physical minimum” at the beginning of the XXI century." Reviews of Modern Physics 76.3 (2004): 981.}
\subsection{Landau-Ginzburg 理论}
基于二级相变的理论,朗道提出在超导转变温度附近,超导体的自由能可以用一个复的序参量$\psi = |\psi(r)|e^{i\phi(r)}$来描述,序参量的模平方$|\psi(r)|^2$ 可以解释为超导电子密度,自由能密度可以写为\footnote{wikipedia: Ginzburg-Landau theory}
\begin{equation}
f_s = f_n + \alpha(T) |\psi|^2 + \frac{\beta(T)}{2} |\psi|^4 + \frac{1}{2m^*} \left| \left(i\hbar \nabla - e^* A \right)\psi \right|^2 + \frac{B^2}{8\pi}~.
\end{equation}
$f_s$代表体系进入超导态后的自由能,而$f_n$代表体系处于正常金属态时的自由能。在超导温度附近,序参量$\psi$出现由0到非0的转变,考虑一个最简单的情形,首先我们令$B=A = 0$,即考虑一个无外加电磁场的体系,$f_n$ 看作一个常数,不妨设$f_n=0$,并且$\psi$在空间中是均匀的,即$\nabla \psi = 0$. 此时的自由能可以写为
\begin{equation}
f_s = f_n + \alpha(T) |\psi|^2 + \frac{\beta(T)}{2} |\psi|^4 ~.
\end{equation}
对自由能关于序参量求偏导$\partial f_s /\partial |\psi|^2 = 0$,可以得到自由能极点
\begin{align}
&|\psi_{\infty}|^2 = \frac{-\alpha}{\beta},|\psi_0|^2 = 0;\\
&f_s(|\psi_{\infty}|) = -\frac{\alpha^2}{\beta}, f_s(|\psi_0|) = 0 ~.
\end{align}
为保证稳定性,即$|\psi|\rightarrow +\infty$时不会出现$f_s \rightarrow -\infty$, 我们需要令$\beta(T)$恒大于0. 容易看到,当$\alpha>0$时, $|\psi_\infty|^2 <0$ 这一极点不会出现,自由能随序参量增大而单调增加,因此在取到$|\psi|=0$时自由能最低,无超导电子。而当$\alpha<0$时,存在一个极小值点,$|\psi| = |\psi_{\infty}| = \sqrt{-\alpha/\beta}$,此时体系自由能为$-\alpha^2/\beta$,即出现超导转变。Landau-Ginzburg指出,在相变温度附近$\alpha(T) = \alpha_0 (T - T_C)$,此处$\alpha_0 > 0$,$T_c$为超导相变的临界温度。此时体系从临界温度之上变为临界温度之下即对应$\alpha(T)$由正到负。Landau-Ginzburg理论可以将各种实验客观测量用系数$\alpha$、$\beta$表示出来,并预测各个物理量之间的关系。
\subsection{有效质量与有效电荷}
Landau-Ginzburg理论的提出在超导微观理论BCS理论之前,因此这套唯象理论的参数含义并不是个显而易见的事。根据各参数与实验的关系,有效质量不出现在可观测量中。这里的有效电荷的选取是一个有趣且深刻的问题。Ginzburg一开始认为$e^*$作为某些激发的有效电荷,应该跟电子电荷不同,甚至有可能依赖于材料的具体参数,比如压强、温度等等,然而Landau不觉得$e^*$应该与电子电荷$e$有任何的不同。因为电子电荷出现在诸如Landau参数$\kappa$(),透射深度$\delta_0$以及临界磁场
\subsection{透射深度}
Landau-Ginzburg理论解释了超导电流的参数依赖问题,在伦敦方程中,计算得到的透射深度据超导唯象解释——伦敦方程 \autoref{sub_edy34_1}~\upref{edy34},其中的透射深度包含参数$n_s$,即超导电子密度,在Landau-Ginzburg理论中,根据$n_s = |\psi|^2 = -\alpha/\beta$,得到透射深度为
\begin{aligned}
\lambda = \sqrt{\frac{m^* c^2 \beta}{4\pi (e^*)^2 |\alpha|}}.
\end{aligned}
\subsection{临界电流}
\subsection{界面能}
To do
