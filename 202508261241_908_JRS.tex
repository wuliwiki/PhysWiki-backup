% 卷绕数(综述)
% license CCBYSA3
% type Wiki

本文根据 CC-BY-SA 协议转载翻译自维基百科\href{https://en.wikipedia.org/wiki/Winding_number}{相关文章}。

\begin{figure}[ht]
\centering
\includegraphics[width=6cm]{./figures/801932f041ea8a31.png}
\caption{这条曲线相对于点 $p$ 的绕数是2。} \label{fig_JRS_1}
\end{figure}
在数学中,闭合曲线相对于平面中某一点的绕数或绕线指数是一个整数,表示该曲线绕该点逆时针方向环绕的总次数,也就是曲线的“转数”。对于某些非闭合的平面曲线,其绕数可能是非整数。绕数依赖于曲线的方向:如果曲线沿顺时针方向绕点运动,则绕数为负数。

绕数是代数拓扑中的基础研究对象,并且在向量分析、复分析、几何拓扑、微分几何以及物理学(例如弦理论)中都扮演着重要角色。
\subsection{直观描述}
沿着红色曲线运动的一个物体,会绕位于原点的人逆时针转两圈。
假设我们有一条位于 $xy$ 平面上的闭合、有方向的曲线。我们可以把这条曲线想象成某个物体的运动轨迹,而曲线的方向表示物体运动的方向。这样,这条曲线的绕数就等于该物体绕原点逆时针转的总圈数。

在计算总圈数时,逆时针的运动计为正数,而顺时针的运动计为负数。例如,如果一个物体先绕原点逆时针转了四圈,然后又绕原点顺时针转了一圈,那么这条曲线的总绕数就是3。

按照这个规则,一条完全没有绕过原点的曲线,其绕数为0;而一条绕原点顺时针运动的曲线,其绕数为负数。因此,曲线的绕数可以是任意整数。下面的图示展示了绕数从 −2到3的不同曲线。
\begin{figure}[ht]
\centering
\includegraphics[width=10cm]{./figures/f02fc89a7a934eb5.png}
\caption{} \label{fig_JRS_2}
\end{figure}
\subsection{形式化定义}
设$\gamma: [0,1] \to \mathbb{C} \setminus \{a\}$是一条平面上去掉点 $a$ 后的连续闭合路径。曲线$\gamma$绕点$a$的绕数定义为整数:
$$
\mathrm{wind}(\gamma, a) = s(1) - s(0),~
$$
其中 $(\rho, s)$ 是该路径用极坐标表示的形式,即通过如下覆盖映射 $p$ 提升得到的路径:
$$
p: \mathbb{R}_{>0} \times \mathbb{R} \to \mathbb{C} \setminus \{a\} : (\rho_0, s_0) \mapsto a + \rho_0 e^{i 2\pi s_0}.~
$$
绕数的良好定义性源于提升路径的存在性和唯一性(给定覆盖空间的起点),并且因为映射 $p$ 的所有纤维都具有以下形式:$\rho_0 \times (s_0 + \mathbb{Z})$,所以上述表达式不依赖于起点的具体选择。由于路径是闭合的,最终得到的绕数是一个整数。
\subsection{替代表述}
在数学的不同分支中,绕数常常有不同的定义方式。下面列出的所有定义都与前面给出的定义是等价的:
\subsubsection{亚历山大编号法}
1865 年,奥古斯特·费迪南德·莫比乌斯,首次提出了一种用简单组合规则来定义绕数的方法;1928 年,小詹姆斯·沃德尔·亚历山大又独立地提出了同样的规则。根据该方法,任意一条曲线都会将平面分割成若干个连通区域,其中有一个区域是无界的。这一定义具有以下性质:在同一个区域内的任意两点,其对应的绕数是相同的。无界区域内(即平面外延处任意点)的绕数为0。相邻两个区域的绕数之差正好是1;其中绕数较大的区域出现在曲线运动方向的左侧。
