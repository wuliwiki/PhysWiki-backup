% 二体问题(分析力学)
% keys 二体问题|开普勒问题|卢瑟福散射
% license Xiao
% type Tutor

\pentry{运动积分\nref{nod_motint},中心力场问题\nref{nod_CenFrc}}{nod_7e90}

\subsection{两体问题的运动方程}
两体问题研究的对象是两个可以看成质点的物体,质量分别为 $m_1,m_2$,位矢分别为 $\bvec r_1,\bvec r_2$。它们之间的相互作用势是 $V(r)=V(|\bvec r_1-\bvec r_2|)$,也就是说只和两者的距离有关。\enref{开普勒问题}{CelBd}、\enref{卢瑟福散射}{RuthSc} 都属于两体问题。

在这篇文章中我们将用分析力学的方法来解决两体问题。设 $\bvec r_c$ 为质心位置,设 $\bvec r$ 为它们的相对位置,那么有
\begin{equation}
\leftgroup{
\bvec r_c &=\frac{m_1\bvec r_1+ m_2\bvec r_2}{m_1+m_2}~,\\
\bvec r\  &=\bvec r_2-\bvec r_1~.
}\end{equation}
那么体系的动能为质心动能加上系统相对于质心参考系的动能。
\begin{equation}
\begin{aligned}
T&=\frac{1}{2}m_1 \dot{\bvec r}_1^2+\frac{1}{2}m_2 \dot{\bvec r}_2^2\\
&=\frac{1}{2}(m_1+m_2)\dot{\bvec r}_c^2+\frac{1}{2}\frac{m_1m_2}{m_1+m_2}\dot{\bvec r}^2\\
&=\frac{1}{2}M \dot{\bvec r}_c^2+\frac{1}{2}\mu \dot{\bvec r}^2
\end{aligned}~.
\end{equation}
其中 $M$ 为体系的总质量,$\mu$ 称作约化质量:
\begin{equation}
\left\{
\begin{aligned}
&M=m_1+m_2\\
&\mu=\frac{m_1m_2}{m_1+m_2}~.
\end{aligned}
\right.
\end{equation}
在无外界影响的情况下,$\dot{\bvec r}_c=0$(这是因为系统的动量守恒)。系统的势能为 $V(r)$。因此拉格朗日量为 $L=T(\dot{\bvec r})-V(r)$。两个物体一定在同一个平面内作运动,设 $\bvec r$ 在该平面内转过的角度为 $\phi$,设 $|\bvec r|=\rho$。我们取广义坐标 $\rho,\phi$。那么 $T=\frac{1}{2}\mu \dot{\bvec r}^2=\frac{1}{2}\mu \dot \rho^2+\frac{1}{2}\mu \rho^2\dot\phi^2$。体系的拉格朗日量和哈密顿量为
\begin{equation}
\begin{aligned}
&L=\frac{1}{2}\mu \dot \rho^2+\frac{1}{2}\mu \rho^2\dot\phi^2-V(\rho)~,\\
&H=T+V=\frac{1}{2}\mu \dot \rho^2+\frac{1}{2}\mu \rho^2\dot\phi^2+V(\rho)~.
\end{aligned}
\end{equation}
因此可以列出拉格朗日方程:
\begin{equation}\label{eq_twoobj_1}
\begin{aligned}
&\frac{\dd }{\dd t}\frac{\partial L}{\partial \dot\rho}=\frac{\partial L}{\partial \rho}\\&\Rightarrow \mu\ddot\rho-\mu\rho\dot\phi^2+\frac{\dd V}{\dd \rho}=0~,
\end{aligned}
\end{equation}

\begin{equation}\label{eq_twoobj_2}
\begin{aligned}
&\frac{\dd }{\dd t}\frac{\partial L}{\partial \dot\phi}=\frac{\partial L}{\partial \phi}\\&
\Rightarrow \frac{\dd p_\phi}{\dd t}=\frac{\dd }{\dd t}(\mu \rho^2\dot\phi)=0
\\&
\Rightarrow p_\phi=\frac{\partial L}{\partial \dot \phi}=\mu\rho^2\dot\phi ={\rm const}=J ~.
\end{aligned}
\end{equation}
\autoref{eq_twoobj_1} \autoref{eq_twoobj_2} 联立就可以得到两体问题的运动方程(两个方程和两个初始条件,就可以求解两个广义坐标的变化)。或者我们也可以简化方程组,将其中一个替换为能量守恒方程,联立得:
\begin{equation}\label{eq_twoobj_3}
\begin{aligned}
&\left\{
\begin{aligned}
&\frac{1}{2}\mu \qty(\dot\rho^2+\rho^2\dot\phi^2)+V(\rho)=E\\
&\mu\rho^2\dot\phi=J
\end{aligned}
\right.\\
&\Rightarrow \frac{1}{2}\mu \dot\rho^2=E-V(\rho)-\frac{J^2}{2\mu\rho^2}=E-V_{\rm{eff}}(\rho)~.
\end{aligned}
\end{equation}
$V_{\rm{eff}}=V(\rho)+\frac{J^2}{2\mu\rho^2}$ 为有效势能。从有效势能与能量 $E$ 的大小关系,可以判断体系处于束缚态还是散射态。
\begin{figure}[ht]
\centering
\includegraphics[width=7cm]{./figures/4698070f49580e36.png}
\caption{有效势能(图中蓝线)的函数图像} \label{fig_twoobj_1}
\end{figure}
以 \autoref{fig_twoobj_1} 为例,对于相互作用势为 $-k/r$ 的两体系统,有效势能 $V_{\rm{eff}}(r)$ 的函数图象如蓝线所示。如果系统的能量 $E$ 恰好等于有效势能极小值(蓝线的最低处),那么系统将作圆周运动;如果 $E<0$,体系处于束缚态,轨道形状为椭圆;如果 $E=0$,那么体系处于散射态,轨道形状为抛物线;如果 $E>0$,那么体系处于散射态,轨道形状为双曲线。轨道形状是二次曲线的原因,与相互作用势 $V=-k/r$ 有关。

从上面的运动方程\autoref{eq_twoobj_3} 可以求解出 $\rho$ 关于 $t$ 的函数;然而具体要解出轨道形状,还需要知道 $\phi$ 关于 $t$ 的函数,这就又需要借助角动量守恒方程。为了快速求出轨道形状,我们通常令 $u=1/\rho$ 对上面的\autoref{eq_twoobj_3} 进行化简。注意到 $\dot u=-\dot \rho/\rho^2=\mu\dot \phi\dot \rho/J$,所以 $\frac{\dd u}{\dd \phi}=\dot u/\dot \phi=\mu\dot \rho/J$。\autoref{eq_twoobj_3} 式就可以改写为 $u$ 与 $\phi$ 的方程:
\begin{equation}\label{eq_twoobj_4}
\begin{aligned}
&\frac{J^2}{2\mu} \qty(\frac{\dd u}{\dd \phi})^2=E-V(1/u)-\frac{J^2}{2\mu} u^2~,\\
&\qty(\frac{\dd u}{\dd \phi})^2=\frac{2\mu}{J^2}\qty(E-V(1/u)))-u^2~.
\end{aligned}
\end{equation}
或将上式两侧对 $\phi$ 求导,得到二阶运动方程:
\begin{equation}\label{eq_twoobj_5}
\begin{aligned}
\frac{{\dd}^2 u}{\dd \phi^2}\frac{\dd u}{\dd \phi}+\frac{\dd u}{\dd \phi}u&=\frac{\mu}{J^2}\frac{-\dd V\qty(1/u)}{\dd (1/u)}\frac{\dd (1/u)}{\dd u}\frac{\dd u}{\dd \phi}
\\
\Rightarrow 
\frac{{\dd}^2 u}{\dd \phi^2}+u&=-\frac{\mu}{J^2u^2}F(1/u)~.
\end{aligned}
\end{equation}
\autoref{eq_twoobj_4} \autoref{eq_twoobj_5} 被称为\textbf{比耐公式}(Binet 公式)。关于中心力场运动问题的比耐公式的推导,还可以参考 \upref{Binet}。

\subsection{开普勒问题}
在开普勒问题中,相互作用势为 $V(\rho)=-Gm_1m_2/\rho=-k/\rho$。那么 \autoref{eq_twoobj_4} 变为
\begin{equation}
\begin{aligned}
\left|\frac{\dd u}{\dd \phi}\right|&=\sqrt{-u^2+\frac{2\mu k}{J^2}u+\frac{2\mu E}{J^2}}\\
&=\sqrt{-\qty(u-\frac{\mu k}{J^2})^2+\frac{2\mu E}{J^2}+\frac{\mu^2 k^2}{J^4}}~.
\end{aligned}
\end{equation}
该一阶偏微分方程的解的形式为
\begin{equation}
u-\frac{\mu k}{J^2}=\alpha\cos(\phi-\beta)~,
\end{equation}
可以解得
\begin{equation}
\begin{aligned}
\alpha=\sqrt{\frac{2\mu E}{J^2}+\frac{\mu^2 k^2}{J^4}}~.\\
\end{aligned}
\end{equation}
这样就求得了 $u$ 关于 $\phi$ 的表达式。最后将 $u$ 用 $\rho=1/u$ 表示,得到
\begin{equation}
\rho=\frac{p}{1+e\cos(\phi-\beta)}~,
\end{equation}
其中 $e$ 为轨道的偏心率(或者称离心率)。$p,e$ 由下式给出:
\begin{equation}
\begin{aligned}
&p=\frac{J^2}{\mu k}\\
&e=\frac{J^2}{\mu k}\alpha=\sqrt{1+\frac{2J^2E}{\mu k^2}}~,
\end{aligned}
\end{equation}

开普勒问题的轨道运动方程还可以用\enref{龙格—楞次矢量}求解}{Keple1}。

\subsection{卢瑟福散射问题}
卢瑟福通过用 $\alpha$ 粒子轰击金箔,否定了汤姆孙的葡萄干面包模型。卢瑟福惊讶地发现,每 20000 个粒子中,有 1 个 $\alpha$ 粒子会被反弹回去,这是汤姆孙的理论无法解释的。根据卢瑟福的设想,原子内部应该有一个体积很小的区域聚集了所有的正电荷,而负电荷则围绕着它在转,这使得 $\alpha$ 粒子轰击金箔这个两体问题的相互作用可以近似为库仑相互作用。卢瑟福因此提出原子的行星轨道模型,为原子结构的研究作出巨大贡献,于1908年获得诺贝尔化学奖。

在卢瑟福散射问题中,设粒子 1 质量 $m_1$,带电荷 $q_1$;粒子 2 质量 $m_2$,带电荷 $q_2$。两个粒子都是带正电荷的粒子,那么它们之间就有排斥力,相互作用势为
\begin{equation}
V(\rho)=\frac{q_1q_2}{4\pi\epsilon_0\rho}=-\frac{k}{\rho}~.
\end{equation}
相互作用势与 $\rho$ 成反比,因此还可以用求解开普勒问题的方法计算。不同的是这里 $k$ 为负数,因此
\begin{equation}
\begin{aligned}
&p=-\frac{J^2}{\mu |k|}\\
&e=-\sqrt{1+\frac{2J^2E}{\mu k^2}}~,
\end{aligned}
\end{equation}
\begin{equation}
\rho=\frac{p}{1+e\cos(\phi-\beta)}=\frac{J^2/\mu |k|}{\sqrt{1+2J^2E/\mu k^2}\cos(\phi-\beta)-1}~.
\end{equation}
为使分母 $>0$,$\phi$ 有一定取值范围:
\begin{equation}
\cos(\phi-\beta)>\frac{1}{|e|}
\Rightarrow \phi \in (\phi_1,\phi_2)~.
\end{equation}
粒子从无穷远飞来,到与另一粒子距离极小时开始返回,再飞回无穷远。轨道的形状为双曲线。\textbf{散射角}(粒子的偏转角度)$\theta$ 满足为
\begin{equation}
\cot\frac{\theta}{2}=\tan\frac{|\phi_1-\phi_2|}{2} =\sqrt{e^2-1}=\sqrt{\frac{2J^2E}{\mu k^2}}~.
\end{equation}

(在随粒子 1 平动的参考系中)设粒子 2 从无穷远的距离以 $\bvec v_0$ 的速度靠近粒子 1,开始运动所在直线与粒子 2 的距离为 $b$ (称为\textbf{碰撞距离}),那么角动量为 $J=\mu bv_0$,能量为 $E=\mu v_0^2/2$。因此可以得到散射角 $\theta$ 与 $b,v_0$ 的关系式:
\begin{equation}\label{eq_twoobj_6}
\cot \frac{\theta}{2}=\frac{\mu bv_0^2}{k}~.
\end{equation}
如果发射大量相同速度粒子 2,探测被粒子 1 “弹”回的各个方向上的粒子数——不同的碰撞距离将导致不同的散射角。微分散射截面的信息往往反应粒子间相互作用的信息,以帮助人们对粒子的内部结构进行猜测,或对已有的猜想进行实验验证。根据微分散射截面的定义\enref{(散射}{Scater}),$\dd \Omega=2\pi \sin\theta \dd \theta$,$\dd \sigma=2\pi b\dd b$,有
\begin{equation}
\frac{\dd \sigma}{\dd \Omega}=\frac{b}{\sin \theta}\left|\frac{\dd b}{\dd \theta}\right|~.
\end{equation}
对\autoref{eq_twoobj_6} 两边微分,得到 $\dd b$ 和 $\dd \theta$ 的关系:
\begin{equation}
-\frac{\dd \theta/2}{\sin^2(\theta/2)}=\mu v_0^2\dd b/k\Rightarrow \left|\frac{\dd b}{\dd \theta}\right|=\frac{|k|}{2\mu v_0^2\sin^2(\theta/2)}~,
\end{equation}
那么
\begin{equation}
\begin{aligned}
\frac{\dd \sigma}{\dd \Omega}&=\frac{\cot(\theta/2)|k|/(\mu v_0^2)}{\sin\theta}\cdot \frac{|k|}{2\mu v_0^2\sin^2(\theta/2)}=\frac{k^2}{4\mu ^2v_0^4\sin^4(\theta/2)}
\\
&=\frac{k^2}{16E^2\sin^4(\theta/2)}~,
\end{aligned}
\end{equation}
其中 $k=-k_eq_1q_2=-q_1q_2/(4\pi\epsilon_0)$(这里 $k_e$ 是静电常数)。该公式与\autoref{eq_RuthSc_1}~\upref{RuthSc}是一致的。
