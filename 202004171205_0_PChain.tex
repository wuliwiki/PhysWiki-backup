% 复合函数的偏导 链式法则
% 多元微积分|导数|偏导数|全微分|复合函数求导|链式法则

\pentry{全微分\upref{TDiff}, 复合函数求导\upref{ChainR}}

若已知二元函数 $z = f(u,v)$,$z$ 是 $u, v$ 的函数,但若 $u$ 和 $v$ 都又是 $x$ 和 $y$ 的函数,则 $z$ 最终是 $x$ 和 $y$ 的函数,即
\begin{equation}
z(x,y) = f[u(x,y),v(x,y)]
\end{equation}
那如何求 $z$ 对 $x$ 和 $y$ 的偏微分呢?我们先来看全微分关系.首先
\begin{equation}
\dd{z} = \pdv{f}{u} \dd{u} + \pdv{f}{v} \dd{v}
\end{equation}
而 $u$ 和 $v$ 的微小变化又都是由 $x$ 和 $y$ 的微小变化引起的
\begin{equation}
\dd{u} = \pdv{u}{x} \dd{x} + \pdv{u}{y} \dd{y}
\quad
\dd{v} = \pdv{v}{x} \dd{x} + \pdv{v}{y} \dd{y}
\end{equation}
所以
\begin{equation}\ali{
\dd{z} &= \pdv{f}{u} \qty( \pdv{u}{x} \dd{x} + \pdv{u}{y} \dd{y} ) + \pdv{f}{v} \qty( \pdv{v}{x}\dd{x} + \pdv{v}{y} \dd{y} ) \\
   &= \qty( \pdv{f}{u}\pdv{u}{x} + \pdv{f}{v}\pdv{v}{x} )\dd{x} + \qty( \pdv{f}{u}\pdv{u}{y} + \pdv{f}{v}\pdv{v}{y} ) \dd{y}
}\end{equation}
这就是 $z$ 关于 $x$ 和 $y$ 的全微分关系.根据定义
\begin{equation}
\pdv{f}{x} = \pdv{f}{u}\pdv{u}{x} + \pdv{f}{v}\pdv{v}{x}
\end{equation}
\begin{equation}
\pdv{f}{y} = \pdv{f}{u}\pdv{u}{y} + \pdv{f}{v}\pdv{v}{y}
\end{equation}
这也叫偏导的\textbf{链式法则}.

\subsection{通用函数名}
物理中常常会出现一种容易混淆的情况\footnote{“通用函数名”是我的叫法},就是当一个因变量可以有几套自变量(例如上面的 $z(u,v)$ 和 $z(x,y)$)时,通常直接用因变量($z$)作为函数名而另外不定义函数名($f$).然而 $z(u,v)$ 与 $z(x,y)$ 中的 $z$ 并不是同一个函数.以下举例说明

\begin{example}{}\label{PChain_ex1}
在二维直角坐标系中,定义势能函数为
\begin{equation}\label{PChain_eq7}
V=f(x,y)=x^2+y^2+2x
\end{equation}
而若用极坐标描述该势能, 则函数变为
\begin{equation}\label{PChain_eq8}
V = g(r,\theta) = f(r\cos \theta , r\sin \theta ) = r^2 + 2r\cos \theta
\end{equation}
但许多物理书为了表述方便并不用 $f$ 和 $g$ 区分两个不同的函数, 而是使用 $V(x,y)$ 表示\autoref{PChain_eq7} 和 $V(r,\theta)$ 表示\autoref{PChain_eq8}.这样后者就有可能被误解为
\begin{equation}
V(r,\theta) = r^2+\theta^2+2r \quad \text{(错)}
\end{equation}
这就需要从语境中判断是否使用了通用函数名.

使用通用函数名时,要注意判断偏导数使用的是哪一套变量,例如 $\pdv*{V}{x}$ 默认使用 $V(x,y)$ 求偏导, $\pdv*{V}{r}$ 默认使用 $V(r,\theta)$ 求偏导.一种更复杂的情况如 $(\pdv*{V}{x})_\theta$. 按照定义\footnote{见偏导数\upref{ParDer}中的\autoref{ParDer_eq1}},应该是仅用 $x$ 和 $\theta$ 表示 $V$,然后求偏导.考虑极坐标的定义,$\theta$ 不变意味着 $y$ 与 $x$ 成正比即 $y=x\tan\theta$,代入\autoref{PChain_eq7} 得
\begin{equation}
V(x,\theta)=x^2(1+\tan^2 \theta) + 2x
\end{equation}
现在再对 $x$ 求偏导即可(略).
\end{example}
