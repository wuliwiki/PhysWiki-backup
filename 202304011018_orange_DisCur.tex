% 广义安培环路定律
% 位移电流|麦克斯韦修正|麦克斯韦方程组|安培环路定理|旋度|散度

\begin{issues}
\issueDraft
\end{issues}

\pentry{安培环路定律\upref{AmpLaw},法拉第电磁感应定律\upref{FaraEB}}

在法拉第电磁感应定律\upref{FaraEB}中,我们已经知道了“变化的磁场会产生电场”;那么反过来,“变化的电场也会产生磁场”吗?充满着简洁与对称美的电动力学给出了肯定的答案。

\begin{figure}[ht]
\centering
\includegraphics[width=8cm]{./figures/DisCur_1.pdf}
\caption{“变化的电场产生磁场”示意图} \label{DisCur_fig1}
\end{figure}

为了体现“变化的电场也会产生磁场”,我们必须在静电学的安培环路定律\upref{AmpLaw}补充一项,使之由
\begin{equation}
\oint \bvec B \vdot \dd{\bvec r} = \mu_0 I~,
\end{equation}
变为\textbf{广义安培环路定律}或\textbf{麦克斯韦—安培公式}:
\begin{equation}
\oint \bvec B \vdot \dd{\bvec r} = \mu_0 I + \mu_0 \epsilon_0 \int \pdv{\bvec E}{t} \vdot \dd{\bvec a}
\end{equation}

根据斯托克斯定理\upref{Stokes},还可将其写为微分形式
\begin{equation}\label{DisCur_eq2}
\curl \bvec B = \mu_0 \bvec j + \mu_0\epsilon_0 \pdv{\bvec E}{t}
\end{equation}

\subsection{“位移电流”}
由于一些历史原因,可定义\textbf{位移电流(displacement current)}为:
\begin{equation}
\bvec j_d = \epsilon_0 \pdv{\bvec E}{t}
\end{equation}

这使得\autoref{DisCur_eq2} 还可写作
\begin{equation}
\curl \bvec B = \mu_0 (\bvec j + \bvec j_d)
\end{equation}

注意,“位移电流”并不是真正的电流,也不涉及电荷的运动。因此,笔者个人认为“位移电流”这个名词仅仅具有历史意义。

\subsection{广义安培环路定律与电荷守恒}
\pentry{电荷守恒、电流连续性方程\upref{ChgCsv}}
那么,我们为什么可以这么推广安培环路定律呢?我们可以从一些思想实验
%可以补充电容充电的经典实验
,或者电荷守恒中得到启发与思路。以下我们简要说明电荷守恒如何启发我们得到广义安培环路定律。

注意,广义安培环路定律事实上已是电动力学的基本假设之一(他的成立是学科的公理,而不是由其他的定理得到), 所以以下的说明并不是真正的推导。

静电学中的安培环路定律\upref{AmpLaw}为
\begin{equation}\label{DisCur_eq1}
\curl\bvec B = \mu_0\bvec j
\end{equation}
这要求等式右边的矢量场必须是一个无散场。 在静电学问题中 $\bvec j$ 的确是无散场($\pdv*{\rho}{t} = 0$), 然而若要拓展到一般情况, 我们需要给\autoref{DisCur_eq1} 右边加上一个修正项, 使等式右边的散度恒为零。

由电荷连续性方程
\begin{equation}
\div \bvec j + \pdv{\rho}{t} = 0
\end{equation}
使用电场高斯定律
\begin{equation}
\div \qty(\bvec j + \epsilon_0\pdv{\bvec E}{t}) = 0
\end{equation}
可见括号中恒为无散场。 所以不妨猜测
\begin{equation}
\curl\bvec B = \mu_0\bvec j + \epsilon_0\mu_0\pdv{\bvec E}{t}
\end{equation}
这就是说, 和法拉第电磁感应\upref{FaraEB}相似, 变化的电场也会产生磁场, 并且和电流产生的磁场叠加得到总磁场。 该式称为\textbf{广义安培环路定律}或者\textbf{麦克斯韦—安培公式}。
