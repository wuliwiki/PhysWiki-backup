% 天津大学 2011 年考研量子力学答案
% keys 考研|天津大学|量子力学|2011|答案
% license Copy
% type Tutor

\begin{issues}
\issueDraft
\end{issues}

\subsection{ }
\begin{enumerate}
\item 根据题意有:
\begin{equation}
\begin{aligned}
Ax\psi_n(x)&=\frac{A}{\alpha}\qty[\sqrt{\frac{n}{2}}\psi_{n-1}(x)+\sqrt{\frac{n+1}{2}}\psi_{n+1}(x)] \\
&=\frac{A}{\alpha}\sqrt{\frac{n}{2}}\psi_{n-1}(x)+\frac{A}{\alpha}\sqrt{\frac{n+1}{2}}\psi_{n+1}(x)~.
\end{aligned}
\end{equation}
则归一化系数$A$有:
\begin{equation}
\qty(\frac{A}{\alpha}\sqrt{\frac{n}{2}}+\frac{A}{\alpha}\sqrt{\frac{n+1}{2}}=1)\Rightarrow A=\sqrt{\frac{2\alpha^2}{2n+1}}~.
\end{equation}
\end{enumerate}
\subsection{ }
\begin{enumerate}
\item $\hat A=(\hat F+\hat F^\dagger),\hat B=\mathrm i(\hat F-\hat F^\dagger)$
\item 首先以$\hat L_z$的特征向量为基,设特征向量为$e_n$,我们有
\begin{equation}
\begin{aligned}
\hat L_z e_n&=ne_n\\
L_x&=L_x\pm\mathcal i L_y\\
\end{aligned}
\end{equation}

\item 
\end{enumerate}
\subsection{ }
\subsection{ }
\subsection{ }
