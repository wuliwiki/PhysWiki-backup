% 宇宙学红移
% keys FRW 度规|红移|哈勃常数
% license Usr
% type Tutor
\pentry{Friedmann-Robertson-Walker (FRW) 度规\nref{nod_FRW}}{nod_7dd5}
\begin{issues}
\issueMissDepend
\end{issues}

\pentry{FRW 度规 \nref{nod_FRW}}{nod_6c42}
由于物体在宇宙中传播的过程中,宇宙也在加速膨胀,所以为了准确的测量物体在宇宙传播过程中的物理量,我们需要引进宇宙学红移的概念。

\subsection{光子的红移}
假设位于共动坐标$r_1$处的光源在$t_1$时刻发出光子,原点处的我们在$t_0$时刻收到,那么根据光子走过的世界线为$0$这一事实,并结合FRW度规,我们有
\begin{equation}
s=\int^{t_0}_{t_1}\frac{c}{a(t')}\dd t'=\int_0^{r_1}\frac{1}{\sqrt{1-k(r/R)^2}}\dd r~.
\end{equation}


从量子力学的描述中,光子的波长可以定义为 $\lambda=\frac{h}{p}$。当光在 $t_1$ 时刻发射以波长 $\lambda_1$ 发射,于 $t_0$ 时刻被接收所观测到的波长为
\begin{equation}\label{eq_CoReSh_1}
\lambda_0=\frac{a(t_0)}{a(t_1)}\lambda_1~,
\end{equation}
其中 $a(t)$ 为 $t$ 时刻的宇宙尺度因子。因为宇宙在加速膨胀 $a(t_0)>a(t_1)$, 所以容易得 $\lambda_0>\lambda_1$。
那么假设光源在$\dd t_1$后又发出光子,而我们在$\dd t_0$后检测到。由于共动距离不变,所以我们有
\subsection{红移因子}
为了计算方便,我们可以通过定义从星系发出的,经过一段时间到达地球后被观测的光的红移为\textbf{红移因子(redshift parametre)}
\addTODO{补充及引用 “光的多普勒效应” 文章}
\begin{equation}
z=\frac{\lambda_0-\lambda_1}{\lambda_1}~,
\end{equation}
显然从\autoref{eq_CoReSh_1} 我们可以推出
\begin{equation}
1+z=\frac{a(t_0)}{a(t)}=\frac{1}{a(t_1)}. \label{eq_CoReSh_2}~,
\end{equation}
一般地我们设现在的宇宙尺度因子 $a(t_0)=1$.

\subsection{哈勃常数}
我们把 $t_1$ 时刻的宇宙尺度因子 $a(t_1)$ 以现在的时刻 $t_0$ 为原点作泰勒展开,可得
\begin{equation}
a(t_1)=a(t_0)(1+(t-t_0)H_0+\cdots)~.
\end{equation}

从\autoref{eq_CoReSh_2} 我们可以看出 $z=H_0(t_0-t_1)$,这里我们定义了\textbf{哈勃常数(Hubble Constant)} $H_0$. 显然我们可以发现红移参量与光走过的距离 $d=c(t_0-t_1)$ 成正比
\begin{equation}
z\simeq\frac{H_0d}{c}~.
\end{equation}

哈勃常数常常被定义为以下数值
\begin{equation}
H_0 \equiv100 h {\rm kms}^{-1}{\rm Mpc}^{-1}~.
\end{equation}
其中 $h$ 测得的数值为\footnote{12Planck 2013 Results – Cosmological Parameters [arXiv:1303.5076]}
\begin{equation}
h\sim 0.67 \pm 0.01~.
\end{equation}
