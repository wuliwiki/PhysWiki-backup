% 厦门大学 2009 年 考研 量子力学
% license Usr
% type Note

\textbf{声明}:“该内容来源于网络公开资料,不保证真实性,如有侵权请联系管理员”

\subsection{(25分)简述题(每小题5分)}
(1)简要说明以下科学家对量子力学的主要贡献:普朗克(Pianck),爱因斯坦(Einstein),玻尔(Bohr),德布罗意(deBroglie),薛定谔(Schrödinger),海森堡(Heisenberg)。

(2)什么是算符?若力学量在经典力学中有对应的量,则在直角坐标系下通过什么对应方式可以改造为量子力学中的力学量算符?

(3)一组力学量算符具有共同完备本征函数组的充要条件是什么?什么是力学量完全集?

(4)氢原子的能量本征值全是分立的吗?若不全是,说明什么情况是分
立的,什么情况下是连续的。

(5)什么是弹性散射?描述散射问题的主要物理量是什么?
\subsection{(25 分)}
\[
\text{二、} \quad (25 \text{ 分}) \quad \text{设质量为} \ m \ \text{的粒子在一维无限深势阱} \ (0 \leq x \leq a) \ \text{中运动,} \ t = 0 \ \text{时刻的波函数为}~
\]

\[
\psi(x, 0) = \sqrt{\frac{8}{5a}} \left( 1 + \cos\frac{\pi x}{a} \right) \sin\frac{\pi x}{a}~
\]

试求:

(1) 体系能量的期望值;

(2) $t$ 时刻的波函数 $\psi(x, t)$。

\[
\text{[提示]} \quad \psi(x, t) = e^{- \frac{i}{\hbar}^{\hat H t}} \psi(x, 0) \quad \text{, 其中} \ \hat{H} \ \text{为体系的哈密顿量。}~
\]
\subsection{(25 分)}
设质量为 $\mu$ 的粒子处于势场 $V(\vec{r})$ 中,哈密顿量为
    \[    H = \frac{p^2}{2\mu} + V(\vec{r})~ \]
    \begin{enumerate}
        \item 对于定态,证明位力(virial)定理: $2\langle \overline T \rangle = \langle \vec{r} \cdot \nabla V(\vec{r}) \rangle$
        \item 若势场为中心力场 $V(r) = -\frac{k}{r^s}$ ($k>0$),证明存在束缚态的条件为 $0 < s < 2$。
    \end{enumerate}
\subsection{(25 分)}
\[    e^{-\frac{i}{\hbar}\theta L_y} L_z e^{\frac{i}{\hbar}\theta L_y} = L_x \sin\theta + L_z \cos\theta, ~\]
    其中 $L_x$, $L_y$, $L_z$ 为角动量算符的三个直角坐标分量。
\subsection{(25 分)}

\subsection{(25 分)}
