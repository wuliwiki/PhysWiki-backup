% 三角函数(高中)
% keys 高中|三角函数
% license Usr
% type Tutor
\pentry{弧度制与任意角\nref{nod_HsAngl},几何与解析几何初步\nref{nod_HsGeBa},函数\nref{nod_functi}}{nod_1829}
\begin{issues}
\issueDraft
\end{issues}

在初中阶段,\textbf{三角函数(trigonometric functions)}通常是在直角三角形的背景下\aref{引入}{eq_HsGeBa_1}的。在学习时,相信读者尚未学习“函数”这一概念,自然仅仅将其作为名称接受,并未意识到它与数学上的函数有何关联。而现在,在了解了函数表示的是输入与输出之间的确定关系的基础上,回顾初中的学习内容,可以发现:无论直角三角形的边长如何变化,只要其中一个锐角固定,三边之间的任意两者的比例始终不变。这一现象正是函数关系的体现——每个角都对应着唯一的比例,这也解释了三角函数名称的由来。

三角函数是一个历史悠久的数学主题,它的独特性在于,它不仅直接关联于几何图形,同时也符合函数的数学定义。这种双重属性使其从一开始就展现出复杂性。然而,复杂性往往伴随着强大的应用能力——事实上,几乎所有的周期函数都可以用三角函数表示。这一特性催生了一门重要的数学分支——\textbf{\enref{傅里叶分析}{FSTri}(Fourier analysis)},它构成了现代电子信息技术、信号处理等领域的基础,建立了时间与频率之间的数学联系。

在高中阶段,三角函数主要考察的内容包括:

\begin{itemize}
\item \enref{三角恒等变换}{HsAnTf},主要涉及三角函数的代数运算,如基本恒等式、倍角与半角公式等。
\item \enref{解三角形}{CosThe},基本仍是三角函数在几何中的应用,包括正弦定理和余弦定理的运用。
\item \enref{函数视角下的三角函数}{HsTFFv},主要考察三角函数的函数性质,如周期性、单调性、对称性、导数以及与其他函数的复合等内容。
\end{itemize}

这三个部分分别体现了三角函数在代数、几何和函数分析中的不同侧重点。而本文的核心目标正是将初中阶段熟悉的直角三角形中的三角函数推广至任意角,为上面提到的三个应用方向奠定基础。

\subsection{推广三角函数至任意角}

最初,三角函数(如 $\sin$、$\cos$ 等)被用于衡量三角形的边长关系,但这种定义仅适用于 $0^\circ$ 到 $90^\circ$ 之间的锐角,因为直角三角形的定义要求其中一角固定为 $90^\circ$,从而限制了角的取值范围。随着角度的推广至任意角,为了突破这一限制,使三角函数适用于任意角。三角函数的定义也需要从直角三角形推广到符合任意角的更一般的形式,以涵盖所有角度。

\begin{figure}[ht]
\centering
\includegraphics[width=10cm]{./figures/fc460d9041b1fc1b.png}
\caption{任意角示意图} \label{fig_HsTrFu_5}
\end{figure}

如\autoref{fig_HsTrFu_5} 所示,既然\aref{任意角}{def_HsAngl_1}可以在坐标系中由角 $\alpha$ 与圆 $O$ 的交点 $P(u,v)$ 确定,自然可以想到,三角函数的定义也应与圆 $O$ 和点 $P(u,v)$ 相关。此外,由于相差 $2\pi$ 或 $360^\circ$ 的任意角具有相同的终边,显然这些角的三角函数值应该相同。换句话说,所有基本三角函数都应是周期函数,并且 $T=2\pi$ 是它们的一个周期。

接下来的问题是如何在单位圆中构造包含角 $\alpha$ 的直角三角形。这样做的好处在于,既能利用原本基于直角三角形的三角函数定义来推导新的定义,又能保证新定义在锐角情况下与原定义的结果一致,从而实现平滑地扩展。

在坐标系中,一个能把已知条件都利用上的直角三角形构造方式是过点 $P$ 向 $x$ 轴作垂线。此时,斜边即圆的半径 $r$,角 $\alpha$ 的对边等于点 $P$ 的纵坐标 $v$,邻边等于其横坐标 $u$。由于这三个量对于任意角 $\alpha$ 都是确定的,因此可以自然地推广初中学过的三角函数定义:

\begin{gather}
\sin\alpha = \frac{v}{r}~.\\
\cos\alpha = \frac{u}{r}~.\\
\tan\alpha = \frac{v}{u}~.
\end{gather}

\subsection{三角函数}

在几何问题中,三角函数通常以角度为单位表示,但从函数的角度来看,为了使三角函数更自然地融入实数函数体系,通常采用弧度制,将角的大小作为自变量。引入弧度制不仅简化了数学表达,还优化了三角函数的性质。例如,在弧度制下,$\sin x$ 在 $x=0$ 处的导数为 $1$,这一特性使得微积分中的运算更加简洁和直观。

此外,如前文在讨论任意角和弧度制时所述,三角函数的研究也通常放在单位圆中。这是因为单位圆的半径恒定为 $r=1$,使得三角函数的定义中不再显式地依赖 $r$(而正切 $\tan\alpha$ 的表达式本身就不涉及 $r$)。本质上,三角函数仅取决于角的大小,而与斜边长度无关,因此在单位圆中定义三角函数更加清晰,也有利于后续的函数分析和计算。

\begin{definition}{三角函数}\label{def_HsTrFu_1}
设自变量 $x$ 为 $OP$ 与 $x$ 轴的夹角,角 $x$ 与单位圆的交点为 $P(u,v)$,则有:
\begin{itemize}
\item \textbf{正弦函数(sine)}
\begin{equation}
\displaystyle\sin x = v~.
\end{equation}
\item \textbf{余弦函数(cosine)}
\begin{equation}
\displaystyle\cos x = u~.
\end{equation}
\item \textbf{正切函数(tangent)}
\begin{equation}
\displaystyle\tan x = \frac{v}{u}~.
\end{equation}
\end{itemize}
\end{definition}

这里的三角函数在定义方式上,与之前学习的幂函数、指数函数和对数函数有所不同。尽管它仍然是一个运算,但这一运算并非直接的代数运算,而是涉及几何图形、长度等概念。在高中阶段,这种区别并不会带来太大影响,但在高等数学中,这样的定义可能会引发某些循环论证的问题,因此在更深入的数学学习中,三角函数的定义会进一步调整。然而,无论如何调整,这些改进仍然是基于最初直角三角形中的三角函数概念,只是为了逻辑上的严密性进行扩展。

事实上,许多数学概念的发展都遵循类似的过程。最初的概念往往是直观且朴素的,但为了扩大其适用范围或增强其严谨性,数学家会对其重新定义。这时,理解概念的关键在于两点:一是掌握最初的直观概念,二是理解重新定义时所采用的思路。任何数学概念的提出都不是凭空产生的,而是基于已有的知识体系逐步演化而来的。

值得注意的是,由于点 $P$ 位于单位圆上,其坐标 $u$ 和 $v$ 可能为负值,因此三角函数的取值也可能为负。例如,第二象限角的余弦值为负,第三象限角的正弦值为负,这超出了原本直角三角形中三角函数仅表示边长比的概念。毕竟,在直角三角形中,边长始终是非负的。然而,这种推广方式仍然是合理的。一方面,在锐角范围内,它与初中阶段的三角函数定义完全一致;另一方面,它揭示了三角函数更本质的数学关系,即不仅仅是边长的比例,而是结合坐标方向的数值表达。实际上,原定义中没有负值的情况只是直角三角形模型下的特殊表现,而推广至任意角后,负值的引入是自然且必要的,使得三角函数的定义更加普遍和完备。

除上述介绍的三种基本三角函数外,还有另外三种三角函数,在高中阶段通常不涉及,包括\footnote{实际上,历史上还存在两个已经被弃用的三角函数概念:

\textbf{正矢(versine)}\begin{equation}
\mathrm{versin }\alpha=\displaystyle{r-y\over r}~.
\end{equation}

\textbf{余矢(vercosine)}
\begin{equation}
\mathrm{covers }\alpha=\displaystyle{r-x\over r}~.
\end{equation}
为方便读者理解,并没有把上面的表达式放在单位圆中。在\autoref{fig_HsTrFu_3} 中,锐角的正矢和余矢分别对应于点 $P$ 在 $y$ 轴的投影与 $(0,1)$ 之间的部分,以及在 $x$ 轴的投影与 $(1,0)$ 之间的部分。早期的三角函数表中曾经包含这两个函数,最初的目的在于避免因正弦或余弦值过小而引入计算误差。然而,随着计算机的发展,这两个函数由于与其他三角函数的关系较弱,逐渐被弃用。}:
\begin{definition}{*三角函数}\label{def_HsTrFu_2}
设自变量 $x$ 为 $OP$ 与 $x$ 轴的夹角,角 $x$ 与单位圆的交点为 $P(u,v)$,则有:
\begin{itemize}
\item \textbf{余切函数(cotangent)}
\begin{equation}
\displaystyle\cot x= \frac{u}{v}~.
\end{equation}
\item \textbf{正割函数(secant)}
\begin{equation}
\displaystyle\sec x = \frac{1}{u}~.
\end{equation}
\item \textbf{余割函数(cosecant)}
\begin{equation}
\displaystyle\csc x = \frac{1}{v}~.
\end{equation}
\end{itemize}
\end{definition}

显然,它们分别是 $\tan x, \cos x, \sin x$ 的倒数。需要注意的是,尽管余割的英文是 cosecant,但由于 $\cos$ 已经被用来表示余弦(cosine),因此采用另一种方式来表示余割:取单词 cosecant 中的第 1、3、5 个字母,形成 $\csc$ 这一记号。

可以看到,这六个三角函数的定义都与圆密切相关,它们既可以看作是角 $x$ 的变化,也可以看作是点 $P$ 在单位圆上的运动。因此,三角函数也被称为\textbf{圆函数(circular functions)}。在初中和高中阶段,三角函数的背景始终限定在某个直角三角形中,而在扩展后的函数视角下中,可以这样理解:三角函数出现的场景必然可以找到某个点在圆上运动,这个观念在大学阶段会有非常重要的作用。

\subsection{三角函数的几何含义}

之前介绍的三角函数定义较为抽象,为了更清晰地理解六个三角函数的几何含义,以下内容将从几何角度分析锐角情况下任意角的三角函数定义,并探讨这些函数名称的由来。

一般,称呼需要研究的角 $\alpha$为\textbf{正角}。在此前的学习中,已经了解两个角互为余角的条件,即它们的角度之和为 $90^\circ$。换句话说,\textbf{余角}是 $90^\circ$ 减去给定正角后剩余的角。

在圆的背景下,弦、切、割的概念可以描述如下\footnote{关于曲线的切线和割线,可参考 \enref{切线与割线}{TanL}。}:
\begin{itemize}
\item 连接圆上两点的线段称为\textbf{弦}\footnote{如果将弦与弧的组合视作一把“弓”,那么弦正对应于弓上的弦,而“矢”的位置恰好就是这把弓上的“箭”。}。
\item 过圆外一点作直线,与圆交于相异的两点时,该直线称为\textbf{割线}。
\item 过圆外一点作直线,与圆相交于唯一一点时,该直线称为\textbf{切线}。
\end{itemize}

这些几何元素可以在坐标系中直观地表示,如 \autoref{fig_HsTrFu_3} 所示。

\begin{figure}[ht]
\centering
\includegraphics[width=14.25cm]{./figures/e9ea4e779f1c67a4.png}
\caption{任意角的三角函数(锐角)} \label{fig_HsTrFu_3}
\end{figure}

从图中可以清楚地看出:
\begin{itemize}
\item $\sin\alpha$ 和 $\cos\alpha$ 是弦的一半,即对应于单位圆内的直角三角形中的直角边。
\item 通过点 $X_0(1,0)$ 和 $Y_0(0,1)$ 作圆的切线,即坐标轴的垂线,与角的终边所在直线相交于 $T$ 和 $C$,得到的线段 $\overline{X_0T}$ 和 $\overline{Y_0C}$,分别对应$\tan\alpha$ 和 $\cot\alpha$ 。这里也可以看作希望从角的终边所在直线上找到一点作圆的切线,且为了利用直角三角形的特性,要求切线是与坐标轴即始边垂直的,从而反推出了圆外的那个点$T,C$。
\item $\sec\alpha$和$\csc\alpha$分别是切线、坐标轴和角所在直线截得的三角形的斜边,也即从$T,C$作通过圆心的割线得到的$\overline{OT}$和$\overline{OC}$。
\end{itemize}

这样,每个正角的弦、切、割都涉及两个三角形:一个是单位圆内的三角形$\triangle OPP_x$;另一个是包含圆外一点的三角形$\triangle OTX_0$。对于余角,几何结构类似,只是对应的三角形有所调整。

当然根据上面的介绍,相信读者自己也可以画出其他情况下对应的几何图形,例如,钝角情况下的图像如图所示。

可以看出,对于正角,其弦和切的符号是由线段是在x轴上侧还是下侧决定的,对于余角则是由在y轴左侧还是右侧决定的。而割线特殊一些,若是在角的终边上则是正的,若是在角的终边的反向延长线上则是负的。对比之前给出的\autoref{def_HsTrFu_1} 和\autoref{def_HsTrFu_2} ,可以看到这样统一的几何结果是自然的。

\subsection{诱导公式}

\subsection{*同角三角函数的基本关系}

根据现在给出的三角函数定义,很容易得到下面三组同角三角函数的恒等关系:
\begin{itemize}
\item 倒数关系:
\begin{equation}\label{eq_HsTrFu_1}
\begin{split}
\tan \alpha \cdot \cot \alpha = 1\\
\sin \alpha \cdot  \csc \alpha = 1\\
\sec \alpha  \cdot \cos \alpha = 1
\end{split}~.
\end{equation}
\item 乘积关系:
\begin{equation}\label{eq_HsTrFu_2}
\begin{split}
\tan \alpha \cdot\cos \alpha= \sin \alpha\\
\sin \alpha \cdot\cot \alpha= \cos \alpha\\
\cos \alpha \cdot\csc \alpha= \cot \alpha\\
\cot \alpha \cdot\sec \alpha= \csc \alpha\\
\csc \alpha \cdot\tan \alpha= \sec \alpha\\
\sec \alpha \cdot\sin \alpha= \tan \alpha\\
\end{split}~.
\end{equation}
\item 平方关系:
\begin{equation}\label{eq_HsTrFu_3}
\begin{split}
\sin ^{2} \alpha + \cos ^{2}\alpha =1\\
\tan  ^{2} \alpha + 1 =\sec ^{2}\alpha\\
1 + \cot ^{2}\alpha =\csc ^{2}\alpha\\
\end{split}~.
\end{equation}
\end{itemize}
上面的公式太多了,不好记怎么办?有人总结了如\autoref{fig_HsTrFu_4} 所示的方法来辅助记忆:
\begin{itemize}
\item 六边形对角线上的函数互为倒数。
\item 六边形顶点上的函数等于相邻两顶点乘积。
\item 三个倒立的三角(黄色标记)上方两顶点的平方和等于下方顶点。
\end{itemize}
\begin{figure}[ht]
\centering
\includegraphics[width=12cm]{./figures/6390d1e662067a9b.png}
\caption{同角三角函数的基本关系} \label{fig_HsTrFu_4}
\end{figure}

在上面的同角三角函数的恒等关系中,可以发现同一个数值往往有多种不同的表达方式。例如,$1$ 既可以表示为 $\sin^2\alpha+\cos^2\alpha$,也可以表示为 $\sec^2\alpha-\tan^2\alpha$。这种性质表明,由三角函数构成的表达式与代数中熟悉的幂表达式不同,其表示形式通常并不唯一。这种灵活性是三角函数的一个重要特点,也是高中阶段经常需要处理的问题之一。

其实,由于高中阶段不涉及$\cot,\sec,\csc$,因此,上面的关系中,常常使用的只有:

\begin{gather}
\sin ^{2} \alpha + \cos ^{2}\alpha =1~.\\
\tan \alpha= \frac{\sin \alpha}{\cos \alpha}~.
\end{gather}

前者是大名鼎鼎的勾股定理,后者则是初中时学习的$\tan\alpha$的定义。这里之所以给出这么多,一则是为了知识的完整,二则是在一些推导过程中,利用恒等关系可以化简避免错误。如果觉得记忆困难,完全可以放弃,提升熟练度一样可以。

