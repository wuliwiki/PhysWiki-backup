% 冯·诺伊曼-博内斯-哥德尔集合论(综述)
% license CCBYSA3
% type Wiki

本文根据 CC-BY-SA 协议转载翻译自维基百科\href{https://en.wikipedia.org/wiki/Von_Neumann\%E2\%80\%93Bernays\%E2\%80\%93G\%C3\%B6del_set_theory}{相关文章}。

在数学基础中,冯·诺依曼–伯奈斯–哥德尔集合论(NBG)是一种公理化集合论,是泽梅洛–弗兰克尔–选择公理集合论(ZFC)的保守扩展。NBG 引入了“类”的概念,类是由公式定义的集合,其量词仅对集合进行量化。NBG 可以定义比集合更大的类,例如所有集合的类和所有序数的类。摩尔斯–凯利集合论(MK)允许通过量词对类进行量化的公式来定义类。NBG 是有限公理化的,而 ZFC 和 MK 则不是。

NBG 的一个关键定理是类存在定理,它声明,对于每个量词仅对集合进行量化的公式,都存在一个类,该类包含满足该公式的集合。这个类是通过用类逐步构造公式来构建的。由于所有集合论公式都是由两种原子公式(成员关系和相等性)和有限多的逻辑符号构成,因此只需要有限多的公理来构建满足这些公式的类。这就是为什么 NBG 是有限公理化的原因。类还用于其他构造、处理集合论悖论,并用于表述全局选择公理,该公理比 ZFC 的选择公理要强。

约翰·冯·诺依曼在 1925 年将类引入集合论。他的理论的原始概念是函数和参数。利用这些概念,他定义了类和集合。\(^\text{[1]}\)保罗·伯奈斯通过将类和集合作为原始概念重新表述了冯·诺依曼的理论。\(^\text{[2]}\)库尔特·哥德尔简化了伯奈斯的理论,用于他对选择公理和广义连续统假设相对一致性的证明。\(^\text{[3]}\)
\subsection{集合论中的类} 
\subsubsection{类的用途} 
在 NBG 中,类有几个用途:
\begin{itemize}
\item 它们产生了集合论的有限公理化。\(^\text{[4]}\)  
\item 它们用于表述“非常强的选择公理”\(^\text{[5]}\)——即全局选择公理:存在一个定义在所有非空集合类上的全局选择函数\( G \),使得对于每个非空集合\( x \),都有 \( G(x) \in x \)。  
   这比 ZFC 的选择公理要强:对于每个非空集合的集合\( s \),存在一个选择函数 \( f \),定义在\( s \)上,使得对于所有\( x \in s \),都有\( f(x) \in x \)。  
\item 通过认识到某些类不能是集合,集合论悖论得以解决。例如,假设所有序数的类 \( \text{Ord} \)是一个集合。那么\( \text{Ord} \)是一个按\( \in \)良序的传递集合。所以,根据定义,\( \text{Ord} \)是一个序数。因此,\( \text{Ord} \in \text{Ord} \),这与\( \in \)是\( \text{Ord} \)的良序性相矛盾。因此,\( \text{Ord} \)不是一个集合。不能是集合的类称为适当类;\( \text{Ord} \)是一个适当类。\(^\text{[6]}\)
\item 适当类在构造中很有用。在他证明全局选择公理和广义连续统假设的相对一致性时,哥德尔使用适当类来构建构造宇宙。他在所有序数的类上构造了一个函数,对于每个序数,通过对先前构造的集合应用集合构建操作来构造一个构造集。构造宇宙就是这个函数的像。\(^\text{[7]}\)
\end{itemize}
\subsubsection{公理模式与类存在定理}  
一旦类被添加到 ZFC 的语言中,就可以轻松地将 ZFC 转换为一个包含类的集合论。首先,添加类理解的公理模式。这个公理模式声明:对于每个仅对集合进行量化的公式\( \phi(x_1, \ldots, x_n) \),存在一个类\( A \),由满足该公式的\( n \)-元组组成——即:
\[
\forall x_1 \cdots \forall x_n \left[ (x_1, \ldots, x_n) \in A \iff \phi(x_1, \ldots, x_n) \right].~
\]
然后,替换公理模式被替换为一个使用类的单一公理。最后,ZFC 的外延公理被修改以处理类:如果两个类有相同的元素,则它们是相同的。ZFC 的其他公理没有被修改。\(^\text{[8]}\)

这个理论不是有限公理化的。ZFC 的替换公理模式已被一个单一公理所替代,但类理解的公理模式被引入。

为了产生一个有限公理化的理论,首先将类理解的公理模式替换为有限多个类存在公理。然后,这些公理被用来证明类存在定理,该定理暗示公理模式的每个实例。\(^\text{[8]}\)这个定理的证明只需要七个类存在公理,这些公理用于将公式的构造转换为满足该公式的类的构造。
\subsection{NBG 的公理化}  
\subsubsection{类和集合}  
NBG 有两种类型的对象:类和集合。直观上,每个集合也是一个类。公理化这一点有两种方式。[需要非主要来源] 伯奈斯使用了多种排序逻辑,包含两种排序:类和集合。\(^\text{[2]}\)哥德尔通过引入原始谓词避免了排序问题:\( \mathfrak{Cls}(A) \)表示 “A 是一个类”,  
\( \mathfrak{M}(A) \)表示 “A 是一个集合”(在德语中,“集合”是“ Menge”)。他还引入了公理,声明每个集合都是一个类,并且如果类\( A \)是某个类的成员,则\( A \)是一个集合。\(^\text{[9]}\)使用谓词是消除排序的标准方式。埃利奥特·门德尔森修改了哥德尔的方法,使得一切都是类,并将集合谓词\( M(A) \)定义为\( \exists C (A \in C) \)。\(^\text{[10]}\)这种修改消除了哥德尔的类谓词和他的两个公理。

伯奈斯的两排序方法可能一开始看起来更自然,但它创造了一个更复杂的理论。\(^\text{[b]}\)在伯奈斯的理论中,每个集合有两种表示方式:一种作为集合,另一种作为类。此外,有两个成员关系:第一个,表示为“∈”,用于两个集合之间;第二个,表示为“η”,用于集合与类之间。\(^\text{[2]}\) 这种冗余是多种排序逻辑所需要的,因为不同排序的变量作用于论域的不同子域。

这两种方法之间的差异不会影响可证明的内容,但会影响如何编写命题。在哥德尔的方法中,\( A \in C \)(其中\( A \)和\( C \)是类)是一个有效的命题。在伯奈斯的方法中,这个命题没有意义。然而,如果\( A \)是一个集合,就有一个等效的命题:定义“集合\( a \)表示类\( A \)”如果它们有相同的成员集合——即,\(\forall x (x \in a \iff x \eta A)\)命题\( a \eta C \)(其中集合\( a \)表示类\( A \))等价于哥德尔的\( A \in C \)。\(^\text{[2]}\)

本文采用的方法是哥德尔加上门德尔森的修改。这意味着 NBG 是一个基于一阶谓词逻辑的公理化系统,具有相等性,其唯一的原始概念是类和成员关系。
\subsubsection{外延性公理和配对公理的定义和公理} 
集合是至少属于一个类的类:\( A \) 是一个集合当且仅当\( \exists C (A \in C) \)。不是集合的类称为适当类:\( A \)是一个适当类当且仅当\( \forall C (A \notin C) \)。\(^\text{[12]}\)因此,每个类要么是一个集合,要么是一个适当类,且没有类既是集合又是适当类。

哥德尔引入了一个约定,即大写字母变量作用于类,而小写字母变量作用于集合。\(^\text{[9]}\)哥德尔还使用以大写字母开头的名称来表示特定的类,包括定义在所有集合类上的函数和关系。本文采用了哥德尔的约定。这使得我们可以写作:
\[
\exists x \, \phi(x)~
\]
代替  
\[
\exists x \left( \exists C (x \in C) \land \phi(x) \right)~
\]
\[
\forall x \, \phi(x)~
\]
代替  
\[
\forall x \left( \exists C (x \in C) \implies \phi(x) \right)~
\]
以下公理和定义是证明类存在定理所需的。

\textbf{外延性公理} 
如果两个类有相同的元素,则它们是相同的。

\[
\forall A \, \forall B \, \left[ \forall x \, (x \in A \iff x \in B) \implies A = B \right]^\text{[13]}~
\] 
该公理将 ZFC 的外延性公理推广到类。

\textbf{配对公理} 
如果\( x \)和\( y \)是集合,那么存在一个集合\( p \),它的唯一成员是\( x \)和\( y \)。

\[
\forall x \, \forall y \, \exists p \, \forall z \, \left[ z \in p \iff (z = x \lor z = y) \right]^\text{[14]}~
\]
与 ZFC 中一样,外延性公理暗示了集合\( p \)的唯一性,这使我们能够引入符号\( \{x, y\} \)。

\textbf{有序对通过以下方式定义:}
\[
(x, y) = \{\{x\}, \{x, y\}\}~
\]
元组通过有序对递归定义:
\[
(x_1) = x_1,~
\]
对于 \( n > 1 \):
\[
(x_1, \ldots, x_{n-1}, x_n) = ((x_1, \ldots, x_{n-1}), x_n).^\text{[c]}~
\]  
\subsubsection{类存在公理和正则性公理}  
类存在公理将用于证明类存在定理:对于每个仅对集合进行量化的包含\( n \)个自由集合变量的公式,都存在一个满足该公式的\( n \)-元组类。以下示例从两个类(函数)开始,并构建一个复合函数。这个例子展示了证明类存在定理所需的技术,这些技术最终引出了需要的类存在公理。

\textbf{示例 1}:如果类\( F \)和\( G \)是函数,那么复合函数\( G \circ F \)由以下公式定义:
\[
\exists t \left[ (x,t) \in F \land (t,y) \in G \right].~
\]
由于该公式有两个自由集合变量\( x \)和\( y \),类存在定理构造了有序对的类:
\[
G \circ F = \{ (x,y) : \exists t \left[ (x,t) \in F \land (t,y) \in G \right] \}.~
\]
由于该公式是通过使用合取\( \land \)和存在量化\( \exists \)从更简单的公式构造的,因此需要类操作,这些操作将表示简单公式的类结合起来,并生成表示含有\( \land \)和\( \exists \)的公式的类。为了生成表示含有\( \land \)的公式的类,可以使用交集,因为\( x \in A \cap B \iff x \in A \land x \in B \)。

为了生成表示含有\( \exists \)的公式的类,可以使用域,因为\( x \in \text{Dom}(A) \iff \exists t \left[ (x,t) \in A \right] \)。

在进行交集之前,必须给\( F \)和\( G \)中的元组添加一个额外的组件,以使它们有相同的变量。将变量\( y \)添加到\( F \)的元组中,将变量\( x \)添加到\( G \)的元组中:
\[
F' = \{ (x,t,y) : (x,t) \in F \} \quad \text{和} \quad G' = \{ (t,y,x) : (t,y) \in G \}.~
\]
在\( F' \)的定义中,变量\( y \)不受\( (x,t) \in F \)语句的限制,因此\( y \)在所有集合的类\( V \)中取值。类似地,在\( G' \)的定义中,变量\( x \)在\( V \)中取值。因此,需要一个公理来将一个额外的组件(其值范围在\( V \)中)添加到给定类的元组中。

接下来,变量按照相同的顺序排列,以准备进行交集操作:
\[
F'' = \{ (x, y, t) : (x, t) \in F \}~
\]
和
\[
G'' = \{ (x, y, t) : (t, y) \in G \}.~
\]
从\( F' \)到\( F'' \)和从\( G' \)到\( G'' \)需要两次不同的排列,因此需要支持元组组件排列的公理。

\( F'' \) 和 \( G'' \) 的交集处理了合取 \( \land \):
\[
F'' \cap G'' = \{ (x, y, t) : (x, t) \in F \land (t, y) \in G \}.~
\]
由于\( (x, y, t) \)被定义为\( ((x, y), t) \),对\( F'' \cap G'' \)取域处理了\( \exists t \)并生成了复合函数:
\[
G \circ F = \text{Dom}(F'' \cap G'') = \{ (x, y) : \exists t ((x, t) \in F \land (t, y) \in G) \}.~
\]
因此,需要交集和域的公理。

类存在公理分为两组:一组处理语言原语,另一组处理元组。第一组有四个公理,第二组有三个公理。\(^\text{[d]}\)