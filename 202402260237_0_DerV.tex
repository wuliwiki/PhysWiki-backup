% 一元矢量函数的导数
% keys 多元微积分|几何矢量|矢量函数|导数|偏导数
% license Xiao
% type Tutor

\begin{issues}
\issueOther{考虑增加多元向量值函数的偏导数}
\end{issues}


\pentry{几何矢量的运算\nref{nod_GVecOp}, 导数\nref{nod_Der}, 矢量值函数, 切线与割线\nref{nod_TanL}}{nod_ca9e}% 链接未完成

\subsection{矢量的导数}

若几何矢量 $\bvec v$ 只是一个标量 $t$ 的函数,记为 $\bvec v(t)$, 则 $\bvec v$ 对 $t$ 的导数可记为以下的一种
\begin{equation}
\dv{\bvec v}{t}~, \qquad \dv{t}\bvec v~, \qquad \dot{\bvec v}~.
\end{equation}
其定义为(类比\autoref{eq_Der_2}~\upref{Der})
\begin{equation}\label{eq_DerV_1}
\dv{\bvec v}{t} = \lim_{\Delta t \to 0} \frac{\bvec v ( t + \Delta t) - \bvec v(t)}{\Delta t}~,
\end{equation}
唯一与实函数($f:\mathbb R \to \mathbb R$)\upref{functi}的导数不同的是, 这里的减法是\textbf{矢量相减},结果还是矢量。除以 $\Delta t$ 相当于矢量的数乘\upref{GVec} $1/\Delta t$, 结果也是矢量。 所以 $\dv*{\bvec v}{t}$ 也是一个矢量关于标量 $t$ 的函数。

从直角坐标的角度来看, $N$ 维矢量可以用 $N$ 个实数表示, 而两个矢量相减则是它们的各个坐标分别相减, 易得\autoref{eq_DerV_1} 得到的矢量导函数的各个分量等于原矢量函数的各个分量分别求导(详见\autoref{eq_DerV_2} )。

例: 速度和加速度(矢量)\upref{VnA}, 匀速圆周运动的速度\upref{CMVD} 和加速度\upref{CMAD}。

\subsection{几何意义}
矢量函数 $\bvec r(t)$ 可以看成一条含参曲线,也就是把矢量的起点固定在坐标原点,改变 $t$ 时,矢量终点画出的曲线。 那么该曲线在任意 $\bvec r(t)$ 处如果存在切线\upref{TanL}, 那切线的方向就是它的导函数 $\dot{\bvec r}(t)$ 的方向。 要证明,使用\autoref{eq_DerV_1} 即可,

从物理上这是容易理解的,若 $t$ 是时间, $\bvec r$ 是一点的位置矢量\upref{Disp},那么 $\dot{\bvec r}(t)$ 就是这点的速度矢量,速度矢量总是沿运动轨迹的切线方向。

\subsection{矢量的求导法则}
与标量函数一样,由定义不难证明矢量函数求导也是\textbf{线性算符}( $c_i$ 为常数)\footnote{以下法则虽然以导数为例, 但对偏导也同样适用。}
\begin{equation}
\dv{t}[c_1 \bvec v_1(t) + c_2 \bvec v_2(t) + \dots] = c_1\dv{\bvec v_1}{t} + c_2\dv{\bvec v_2}{t} \dots~
\end{equation}

直角坐标中,矢量函数可以看做三个分量上的标量函数且矢量基底不变,所以由上式可得矢量求导就是对每个标量函数求导。
\begin{equation}\label{eq_DerV_2}
\dv{\bvec v}{t} = \dv{t}[v_x(t)\uvec x] + \dv{t}[v_y(t)\uvec y] + \dv{t}[v_z(t)\uvec z]
= \dot v_x(t)\uvec x + \dot v_y(t)\uvec y + \dot v_z(t)\uvec z~.
\end{equation}
要特别注意该式成立的条件是三个基底不随 $t$ 改变,这在其他坐标系中并不成立, 例如“极坐标中单位矢量的偏导\upref{DPol1}”。

例: 匀速圆周运动的速度\upref{CMVD} 和加速度\upref{CMAD}(求导法)。

矢量数乘,内积或叉乘的求导在形式上都与标量函数的情况类似。
\begin{equation}\label{eq_DerV_3}
\dv{t}[f(t)\bvec v(t)] = \dv{f}{t}\bvec v + f\dv{\bvec v}{t}~,
\end{equation}
\begin{equation}\label{eq_DerV_5}
\dv{t}[\bvec u(t)\vdot\bvec v(t)] = \dv{\bvec u}{t}\vdot\bvec v + \bvec u\vdot\dv{\bvec v}{t}~,
\end{equation}
\begin{equation}\label{eq_DerV_8}
\dv{t}[\bvec u(t) \cross \bvec v(t)] = \dv{\bvec u}{t} \cross \bvec v + \bvec u \cross \dv{\bvec v}{t}~.
\end{equation}
由定义出发,不难证明以上三式,这里以\autoref{eq_DerV_5} 为例进行证明。 根据内积定义以及标量函数的求导法则\upref{DerRul}有
\begin{equation}
\ali{
\dv{t} (\bvec u \vdot \bvec v) &= \dv{t} (u_x v_x + u_y v_y + u_z v_z)\\
&= \qty(\dv{u_x}{t} v_x + u_x \dv{v_x}{t} ) + \qty(\dv{u_y}{t} v_y + u_y \dv{v_y}{t}) + \qty(\dv{u_z}{t} v_z   + u_z \dv{v_z}{t} ) \\
&= \qty(\dv{u_x}{t} v_x + \dv{u_y}{t} v_y + \dv{u_z}{t} v_z ) + \qty(u_x \dv{v_x}{t} + u_y \dv{v_y}{t} + u_z \dv{v_z}{t} ) \\
&= \dv{\bvec u}{t}\vdot \bvec v + \bvec u\vdot \dv{\bvec v}{t}~.
}\end{equation}

应用举例: 动量定理\upref{PLaw},角动量定理(单个质点)\upref{AMLaw1}。

\subsection{矢量的高阶导数}
与标量函数的高阶导数类似, 对某个矢量连续求 $N$ 次导数, 就得到该函数的 $N$ 阶导数。 上面在求圆周运动的加速度时, 事实上我们已经计算了位置矢量的导数(速度)的导数, 即位置矢量关于时间的二阶导数。
