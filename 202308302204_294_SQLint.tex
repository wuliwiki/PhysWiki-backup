% SQL 入门(SQLite 为例)
% license Xiao
% type Tutor

\subsection{SQL简单介绍}
SQL数据库一般用于存储程序的各种数据。对于简单的程序来说,可能一个简单的文本文件或表格就足够了。但是如果数据量比较大,且他们之间联系比较多的时候,我们进行数据查询的时候就会很复杂。

简单来说,SQL数据库和办公软件的表格(Excel)有很多相似之处。在一个数据库中,可以创建许多个表格。表格的每一列可以有不同的数据类型,比如整数,小数,文本等。你可以使用SQL专有的命令,对表格中的数据进行插入,修改,删除,查询等操作。

\subsection{SQLite使用教程}
SQL有许多不同的软件,例如MySQL,ProstgreSQL,SQLite等,为了简单起见,我们以SQLite软件作为入门演示。

\begin{enumerate}
\item 可以在\href{https://sqlitebrowser.org/dl/}{官方网站下载页面}下载,选择你需要的操作系统的安装包。安装完成如下图:


\begin{figure}[ht]
\centering
\includegraphics[width=12cm]{./figures/0efb331e4729b6f7.png}
\caption{SQLite主页} \label{fig_SQLint_1}
\end{figure}

\item 根据需求创建数据库(相当于表格),我们以学生-课程数据库为例。

\begin{figure}[ht]
\centering
\includegraphics[width=12cm]{./figures/81626c61a8ee6c7d.png}
\caption{点击创建数据库} \label{fig_SQLint_2}
\end{figure}

\begin{figure}[ht]
\centering
\includegraphics[width=12cm]{./figures/013db632fb5ccad6.png}
\caption{保存在合适的位置} \label{fig_SQLint_3}
\end{figure}


\begin{figure}[ht]
\centering
\includegraphics[width=12cm]{./figures/15dfd0190d5b237c.png}
\caption{新增表} \label{fig_SQLint_5}
\end{figure}

\begin{itemize}
\item 填入表的名称(示例的表名为student)
\item 点击“增加”可增加字段(如图4),一个字段就是就相当于表格的一列,同理也可以进行删除。
\item 类型可以按需选择 INTEGER(整数类型),TEXT(文本类型),BLOB(二进制数据类型),BLOB(浮点类型),NUMERIC(用于高精度计算)。
\\例如:学号(Sno)属于整数类型  名字(Sname)属于文本类型。
\item 一个表格有一个或多个主键,它用于唯一标识表中的特定记录。例如Sno,学生的学号是可以唯一标识一位学生的,所以学号是一个学生的主键。简单来说就是,知道了学生的学号,就能找到对应的学生。如果只知道一个学生的性别Ssex,是不能找到的,这也就是为什么性别不能作为主键的原因。
\item 图4下方的语句属于SQL语句,基础SQL语句参考SQL 入门语法\upref{SQLgrm}。基础好的也可以直接用SQL语句建表。
\end{itemize}
\item 建表成功后如图所示:
% 插入”建表成功“图片

\item 表建立完成以后,我们需要网表里插入一些数据,具体操作如下:
% 插入“插入值”图片
这里我插入了四条数据:
% 插入“写入值”图片
% “四条数据”图片
当然这里也可以使用SQL语句进行写入值:
% “sql语句写入值”
% “运行sql语句”
\item 作为示例,我建立了三个表格
\\1.学生表(Student):学号,姓名,性别,年龄。
\\2.课程表(Course)课程号,ke
\\3.成绩表(SC)
\end{enumerate}





