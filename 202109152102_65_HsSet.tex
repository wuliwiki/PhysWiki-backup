% 集合(高中)
% keys 高中|集合
\subsection{概述}
集合语言是现代数学的基本语言,这种语言可以简洁、准确的表达数学内容.

\subsection{定义}
一般地,指定的某些对象的全体称为\textbf{集合}.集合常用大写字母 $A,B,C,D,\cdots$ 标记.集合中的每个对象叫作这个集合的\textbf{元素}.常用小写字母 $a,b,c,d,\cdots$ 表示集合中的元素.

若 $a$ 在集合 $A$ 中,就说 $a$ \textbf{属于}集合 $A$ ,记作 $a \in A$.若 $a$ 不在集合 $A$ 中,就说 $a$ \textbf{不属于}集合 $A$,记作 $a\notin A$.

数的集合简称\textbf{数集}.\\
自然数组成的集合简称\textbf{自然数集},记作 $N$;\\
正整数组成的集合简称\textbf{正整数集},记作 $N^{*}$ 或 $N^{+}$;\\
整数组成的集合简称\textbf{整数集},记作 $Z$;\\
有理数组成的集合简称\textbf{有理数集},记作 $Q$;\\
实数组成的集合简称\textbf{实数集},记作 $R$.

\subsection{表示}
\textbf{列举法}是把集合中的元素一一列举出来写在大括号内的方法.
符号表示为 ${,\cdots,}$,如 ${x_1,x_2, \cdots ,x_n$.