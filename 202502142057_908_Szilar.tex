% 利奥·西拉德(综述)
% license CCBYSA3
% type Wiki

本文根据 CC-BY-SA 协议转载翻译自维基百科\href{https://en.wikipedia.org/wiki/Leo_Szilard}{相关文章}。

\begin{figure}[ht]
\centering
\includegraphics[width=6cm]{./figures/1949e65970ed96d0.png}
\caption{西拉德,约1960年} \label{fig_Szilar_1}
\end{figure}
莱奥·西拉德(Leo Szilard,/ˈsɪlɑːrd/;匈牙利语:Szilárd Leó [ˈsilaːrd ˈlɛoː];原名Leó Spitz;1898年2月11日 – 1964年5月30日)是一位匈牙利出生的物理学家、生物学家和发明家,他在核物理学和生物学领域做出了许多重要发现。他在1933年构思了核链式反应,并于1936年为此申请了专利。1939年底,他为阿尔伯特·爱因斯坦签署的信件起草了内容,这封信促成了曼哈顿计划的启动,最终制造出了原子弹。随后在1944年,他撰写了《西拉德请愿书》,要求杜鲁门总统在没有将原子弹投放到平民身上的情况下展示其威力。根据吉尔吉·马克斯的说法,他是被称为“火星人”的匈牙利科学家之一。[1]

西拉德最初在布达佩斯的约瑟夫皇帝技术大学(Palatine Joseph Technical University)学习,但他的工程学学业在第一次世界大战期间因服役于奥匈帝国军队而中断。他于1919年离开匈牙利前往德国,进入柏林查理滕堡的技术大学(现为柏林工业大学)就读,但他对工程学感到厌倦,转学到弗里德里希·威廉大学,开始学习物理学。他的博士论文是关于麦克斯韦妖的,这是热力学和统计物理哲学中的一个长期难题。西拉德是第一个认识到热力学与信息理论之间联系的著名科学家。

西拉德为电子显微镜(1928年)、回旋加速器(1929年)提出了最早的专利申请并发布了相关的首批论文,还参与了线性加速器(1928年)在德国的发展。1926年到1930年间,他与爱因斯坦合作开发了爱因斯坦冰箱。

1933年阿道夫·希特勒成为德国总理后,西拉德敦促他的家人和朋友趁还可以时逃离欧洲。他移居英国,并帮助成立了学术援助委员会,这是一个致力于帮助难民学者找到新工作的组织。在英国期间,西拉德与托马斯·A·查尔默斯一起发现了一种同位素分离的方法,这就是著名的西拉德-查尔默斯效应。

预见到欧洲将爆发另一场战争,西拉德于1938年移居美国,在那里他与恩里科·费米和沃尔特·津恩合作研究如何制造核链式反应。1942年12月2日,他亲眼见证了这一目标在芝加哥堆-1中实现。他在芝加哥大学的曼哈顿计划冶金实验室工作,参与了核反应堆设计的相关工作,并担任首席物理学家。他起草了《西拉德请愿书》,主张以非致命方式展示原子弹,但临时委员会最终决定将其用于军事打击。

他与恩里科·费米一起,于1944年申请了核反应堆的专利。他公开警告可能会发展出盐化热核弹,这是一种新型的核武器,可能会灭绝人类。

他在生物科学方面的发明、发现和贡献同样重要;其中包括反馈抑制的发现和化学反应池的发明。根据西奥多·帕克和菲利普·I·马库斯的说法,西拉德提供了至关重要的建议,使得人类细胞的最早克隆成为现实。

1960年,他被诊断为膀胱癌,并接受了他自己设计的钴-60治疗。他帮助创立了索尔克生物研究所,并成为该研究所的常驻学者。西拉德于1962年创立了“适宜生存世界委员会”,旨在向国会、白宫和美国公众传递关于核武器的“理性之声”。他于1964年因心脏病发作在睡梦中去世。
\subsection{早年生活}  
他于1898年2月11日出生在匈牙利王国布达佩斯,原名Leó Spitz。他的父母是中产阶级犹太人,父亲Lajos(Louis)Spitz是一名土木工程师,母亲Tekla Vidor将Leó抚养在佩斯的Városligeti Fasor街。[2] 他有两个弟妹,哥哥Béla出生于1900年,妹妹Rózsi出生于1901年。1900年10月4日,家族将姓氏从德语的“Spitz”改为匈牙利语的“Szilárd”,这个名字在匈牙利语中意味着“坚固”。[3] 尽管有宗教背景,西拉德后来成为了不可知论者。[4][5] 从1908年到1916年,他就读于布达佩斯第六区的Föreáliskola(高中)。他从小对物理学表现出浓厚兴趣,并且在数学方面很有天赋,1916年,他获得了厄尔托奖,这是一个国家级的数学奖项。[6][7]
\begin{figure}[ht]
\centering
\includegraphics[width=6cm]{./figures/63e9374f9b39cc50.png}
\caption{西拉德,约1915年[8]} \label{fig_Szilar_2}
\end{figure}
在第一次世界大战席卷欧洲之际,西拉德于1916年1月22日接到征兵通知,被征召入伍至第5堡垒团,但他仍能够继续他的学业。他于1916年9月以工程学学生身份进入约瑟夫皇帝技术大学。次年,他加入了奥匈帝国军队的第4山地炮兵团,但很快被派往布达佩斯接受军官候补培训。1918年5月,他重新加入了自己的团,但在9月,尚未被送往前线时,他感染了西班牙流感,因而被送回家进行住院治疗。[9] 后来他得知他的团几乎在战斗中被全灭,因此这次生病可能救了他的命。[10] 他于1918年11月在停战后荣誉退役。[11]

1919年1月,西拉德重新开始了他的工程学学习,但此时匈牙利的政治局势动荡,贝拉·库恩领导的匈牙利苏维埃共和国崛起。西拉德和他的哥哥贝拉成立了自己的政治团体——匈牙利社会主义学生联合会,平台基于西拉德提出的税制改革方案。他坚信社会主义是匈牙利战后问题的解决方案,但并非库恩的匈牙利社会主义党,因为该党与苏联有着密切联系。[12] 当库恩政府动荡时,兄弟俩正式将宗教信仰从“以色列人”改为“加尔文派”,但当他们试图重新注册进入现在的布达佩斯技术大学时,由于他们是犹太人,被民族主义学生阻止了。[13]
\subsubsection{在柏林的时光}  
西拉德深信在匈牙利没有未来,因此他于1919年12月25日经奥地利前往柏林,并进入柏林查理滕堡的技术大学(Technische Hochschule)。不久后,他的哥哥贝拉也加入了他。[14] 西拉德对工程学感到厌倦,他的注意力转向了物理学。但技术大学并没有开设物理学课程,因此他转学到弗里德里希·威廉大学,参加了阿尔伯特·爱因斯坦、马克斯·普朗克、瓦尔特·能斯特、詹姆斯·弗兰克和马克斯·冯·劳厄等教授的讲座。[15] 他还结识了其他匈牙利学生尤金·维格纳、约翰·冯·诺依曼和丹尼斯·加博。[16]

西拉德的博士论文《Über die thermodynamischen Schwankungserscheinungen》(《热力学波动现象》)获得了爱因斯坦的赞扬,并在1922年获得最高荣誉。该论文涉及热力学和统计物理哲学中的一个长期难题,即麦克斯韦妖,这是物理学家詹姆斯·克拉克·麦克斯韦提出的一个思想实验。这个问题曾被认为是无法解决的,但在处理这个问题时,西拉德认识到热力学与信息理论之间的联系。[17][18] 1924年,西拉德被任命为冯·劳厄在理论物理研究所的助理。1927年,他完成了晋升资格,并成为物理学的私人讲师(Privatdozent)。在他的晋升资格讲座中,他提交了关于麦克斯韦妖的第二篇论文《Über die Entropieverminderung in einem thermodynamischen System bei Eingriffen intelligenter Wesen》(《通过智能生物干预热力学系统中的熵减少》),这篇论文实际上是在第一篇之后不久写成的。它介绍了现在被称为西拉德发动机的思想实验,并在试图理解麦克斯韦妖的历史中变得重要。该论文还首次引入了负熵与信息之间的关系式。这项工作确立了西拉德在信息理论中的基础性地位;然而,他直到1929年才发表此论文,并选择不再进一步深入该主题。[19][20] 信息学通过诺伯特·维纳和克劳德·E·香农的工作,后来在1940年代和1950年代将这一概念发展成了一个通用理论——尽管在信息学会议期间,约翰·冯·诺依曼指出,西拉德在他对维纳的《信息学》一书的评论中首先将信息与熵等同起来。[21][22]

在柏林的期间,西拉德从事了众多技术发明。1928年,他提交了线性加速器的专利申请,尽管他并不知道古斯塔夫·伊辛(Gustav Ising)在1924年的期刊文章和罗尔夫·维德鲁(Rolf Widerøe)早期的工作设备。[23][24] 1929年,他还申请了回旋加速器的专利。[25] 他还是第一个构思出电子显微镜概念的人,[26] 并于1928年提交了最早的电子显微镜专利。[27] 在1926年到1930年间,他与爱因斯坦合作开发了爱因斯坦冰箱,这个冰箱的特点是没有任何运动部件。[28] 他并没有制造所有这些设备,也没有在科学期刊上发表这些想法,因此这些发明的功劳通常归其他人。由于这一原因,西拉德未能获得诺贝尔奖,但厄尼斯特·劳伦斯因回旋加速器于1939年获得了诺贝尔奖,恩斯特·鲁斯卡则因电子显微镜于1986年获得了诺贝尔奖。[27]
\begin{figure}[ht]
\centering
\includegraphics[width=6cm]{./figures/ad791bda177339fe.png}
\caption{来自费米–西拉德“中子反应堆”专利的图像} \label{fig_Szilar_3}
\end{figure}
西拉德于1930年获得了德国国籍,但他已经对欧洲的政治局势感到不安。[29] 当阿道夫·希特勒于1933年1月30日成为德国总理时,西拉德敦促他的家人和朋友趁还可以时逃离欧洲。[20] 他移居英国,并将自己在苏黎世银行存款的1,595英镑(今天相当于143,000英镑)转移到伦敦的银行。他住在酒店,住宿和餐饮的费用大约是每周5.5英镑。[30] 为了帮助那些不太幸运的人,他帮助创立了学术援助委员会,这是一个致力于帮助难民学者找到新工作的组织,并说服皇家学会在伯灵顿大厦为其提供住宿。他还寻求哈拉尔德·博尔、G·H·哈迪、阿奇博尔德·希尔和弗雷德里克·G·多南等学者的帮助。到1939年第二次世界大战爆发时,该组织已帮助为超过2,500名难民学者找到了安置的地方。[31]
\subsubsection{核物理学}  
1933年9月12日早晨,西拉德在《泰晤士报》上读到了一篇总结拉瑟福德勋爵演讲的文章,在这篇演讲中,拉瑟福德拒绝了将原子能用于实际用途的可行性。演讲中特别提到他最近的学生约翰·科克罗夫特和欧内斯特·沃尔顿于1932年进行的工作,他们通过粒子加速器轰击锂,成功将其“裂变”为\(\alpha\)粒子。[32] 拉瑟福德接着说道:

“在这些过程中,我们可能获得比质子所提供的更多的能量,但从平均来看,我们不能指望通过这种方式获得能量。这是一种非常低效的能量生产方式,任何试图通过原子转化来寻找能源的人都是在说胡话。但这个课题从科学的角度来看是有趣的,因为它为我们提供了关于原子的新见解。”[33]

西拉德对拉瑟福德的否定意见感到非常恼火,便在同一天构思出了核链式反应的想法(类似于化学链式反应),并利用了最近发现的中子。这一想法并未使用核裂变机制,因为核裂变尚未被发现,但西拉德意识到,如果中子能够启动任何形式的能量产生核反应,例如在锂中发生的反应,并且这些中子可以通过相同的反应自行产生,那么由于反应是自我维持的,能量可能以极少的输入获得。[34] 他想对当时已知的92种元素进行系统的调查,以找到一种能允许链式反应的元素,预计成本为8000美元,但由于缺乏资金,他未能实施这一计划。[35]

西拉德于1934年6月为中子引发的核链式反应概念申请了专利,并于1936年3月获得了专利。[36] 根据《专利与设计法案》(1907年,英国)第30条,[37] 西拉德能够将专利转让给英国海军部,以确保其保密,他也这样做了。[38] 因此,他的专利直到1949年才被公开,[36] 当时《专利与设计法案》(1907年,英国)中的相关条款被《专利与设计法案》(1949年7月,英国)废除了。[39] 理查德·罗兹描述了西拉德灵感迸发的时刻:

“在伦敦,南安普敦路穿过拉塞尔广场,位于大英博物馆对面,西拉德在一个灰色的经济大萧条早晨烦躁不安地等待着交通信号灯的变化。前一晚下过小雨;1933年9月12日,星期二清晨,天气凉爽、潮湿且阴沉。午后不久,细雨将再次降临。西拉德后来讲述这段故事时,从未提到那天早晨的目的地。他可能根本没有目的地;他常常步行思考。无论如何,另一个目的地介入了。交通信号灯变为绿色,西拉德走出了人行道。当他穿过街道时,时间在他面前裂开,他看到了通向未来的道路,死亡进入了世界,我们所有的苦难,[41] 未来的形态。[42]”

在构思核链式反应之前,1932年西拉德曾阅读了H.G.威尔斯的《自由的世界》,这本书描述了威尔斯所称的“原子弹”持续爆炸的情形;西拉德在他的回忆录中写道,这本书给他留下了“非常深刻的印象”。[43] 当西拉德将他的专利转让给海军部,以防止消息被更广泛的科学界注意到时,他写道:“知道这[链式反应]意味着什么——我知道,因为我读过H.G.威尔斯的书——我不希望这个专利公开。”[43]

1934年初,西拉德开始在伦敦的圣巴特洛缪医院工作。在医院工作人员中,他与年轻的物理学家托马斯·A·查尔默斯合作,开始研究放射性同位素在医学中的应用。已知通过中子轰击元素可以产生元素的更重同位素,或更重的元素,这一现象被称为费米效应,得名于其发现者——意大利物理学家恩里科·费米。当他们用氡-铍源产生的中子轰击碘乙烯时,他们发现较重的放射性碘同位素从化合物中分离出来。由此,他们发现了一种同位素分离的方法。这种方法后来被称为西拉德–查尔默斯效应,并广泛应用于医学同位素的制备。[44][45][46] 他还曾尝试通过X射线轰击铍,创造核链式反应,但未能成功。[47][48]
\subsection{曼哈顿计划}  
\subsubsection{哥伦比亚大学}  
西拉德于1937年9月拜访了贝拉、罗丝和她的丈夫罗兰(洛兰)·德特雷,他们住在瑞士。经过一场暴风雨后,他和他的兄弟姐妹们在一次未能成功的尝试中度过了一个下午,试图制造一个原型的可折叠雨伞。此次访问的一个原因是他已经决定移居美国,因为他认为欧洲再次发生战争是不可避免且迫在眉睫的。他于1938年1月2日乘坐RMS Franconia号客轮抵达纽约。[49] 在接下来的几个月里,他辗转于不同的地方,与莫里斯·戈德哈伯在伊利诺伊大学厄本那-香槟分校进行研究,随后又在芝加哥大学、密歇根大学和罗切斯特大学进行研究,在那里他进行了铟实验,但再次未能启动链式反应。[50]
\begin{figure}[ht]
\centering
\includegraphics[width=6cm]{./figures/f69ea5b92825fc15.png}
\caption{关于恩里科·费米和莱奥·西拉德的军情报告} \label{fig_Szilar_4}
\end{figure}
1938年初,西拉德在纽约市的国王皇冠酒店定居,这个地方“成为了他余生大部分时间的避风港”,因为那里靠近哥伦比亚大学,他在那里进行研究,虽然没有正式的职称或职位。[51] 他遇到了约翰·R·丹宁,丹宁邀请他在1939年1月的一个下午研讨会上谈论他的研究。[50] 同月,尼尔斯·玻尔带来了一个消息,告诉纽约,德国凯瑟·威廉化学研究所的化学家奥托·哈恩和弗里茨·斯特拉斯曼在1938年12月19日意外地观察到核裂变。哈恩和斯特拉斯曼对他们观察结果的误解后来得到了理论上的修正,并由莉泽·梅特纳和奥托·弗里施解释。梅特纳早在1933年就已知道西拉德的理论,并在重新进行实验后确认西拉德的理论一直是正确的。[51] 当西拉德在普林斯顿大学访问维格纳时得知这一消息时,他立刻意识到铀可能是能够维持链式反应的元素。[52]

由于无法说服费米相信这一点,西拉德决定独自行动。他从哥伦比亚大学物理系主任乔治·B·佩格拉姆那里获得了使用实验室三个月的许可。为了资助实验,他向发明家本杰明·利博维茨借了2,000美元。他给牛津大学的弗雷德里克·林德曼发了电报,请他发送一个铍圆柱体。他说服沃尔特·津成为他的合作者,并雇佣了谢苗·克雷韦尔来研究制造纯铀和石墨的工艺。[53]

西拉德和津在哥伦比亚大学普平大楼的七楼进行了一次简单的实验,使用氡-铍源用中子轰击铀。最初,示波器没有任何反应,但随后津意识到示波器没有插好电源。接通电源后,他们发现自然铀中存在显著的中子增殖,证明了链式反应是可能的。[54] 西拉德后来描述了这一事件:“我们打开开关,看到闪光。我们看了一会儿,然后关闭了一切,回家了。”[55] 然而,他意识到这一发现的意义和后果。“那天晚上,我几乎没有怀疑,世界正走向灾难。”[56]

虽然他们已经证明了铀的裂变产生的中子多于它消耗的中子,但这仍然不是链式反应。西拉德说服费米和赫伯特·L·安德森尝试一个更大的实验,使用500磅(230千克)的铀。为了最大化裂变的机会,他们需要一个中子减速剂来减慢中子的速度。氢被认为是已知的减速剂,因此他们使用了水。实验结果令人失望。显然,氢减慢了中子的速度,但也吸收了它们,导致用于链式反应的中子减少。西拉德随后建议费米使用碳,采用石墨形式。他认为需要大约50吨(49长吨;55短吨)(50.8公吨)的石墨和5吨(4.9长吨;5.5短吨)的铀。作为备用计划,西拉德还考虑了哪里可以找到几吨重水;氘不像普通氢那样吸收中子,但在作为减速剂时具有类似的效果。如此大量的材料需要大量的资金。[57]

西拉德起草了一封机密信件给总统富兰克林·D·罗斯福,解释了核武器的可能性,警告了德国的核武器项目,并鼓励发展一个能够促成核武器制造的计划。在维格纳和爱德华·泰勒的帮助下,他在1939年8月联系了他的老朋友和合作伙伴爱因斯坦,并说服他签署了这封信,借用他的名声为提案背书。[58] 爱因斯坦–西拉德信促成了美国政府对核裂变的研究,并最终导致了曼哈顿计划的成立。罗斯福将信件交给了他的助手,埃德温·M·“帕”·沃森准将,并指示道:“帕,这是需要采取行动的!”[59]

一个铀委员会在莱曼·J·布里格斯的领导下成立,布里格斯是科学家和国家标准局的局长。委员会的第一次会议于1939年10月21日举行,西拉德、泰勒和维格纳出席了会议,他们说服了陆军和海军为西拉德提供6,000美元,用于购买实验用品,特别是更多的石墨。[60] 1940年,一份关于费米和西拉德的陆军情报报告在美国尚未加入第二次世界大战时准备好,其中对二者表示了保留意见。虽然报告中对西拉德有一些事实错误,但它正确地记录了他对德国将赢得战争的严峻预测。[61]

费米和西拉德与国家碳公司(National Carbon Company)的赫伯特·G·麦克弗森和V·C·哈米斯特会面,该公司生产石墨,西拉德在那里做出了另一个重要发现。他询问了石墨中的杂质,并从麦克弗森那里了解到,石墨通常含有硼,一种中子吸收剂。随后,他要求生产不含硼的石墨。[62] 如果他没有这样做,他们可能会像德国的核研究人员那样得出结论,认为石墨不适合作为中子减速剂。[64] 与德国研究人员一样,费米和西拉德仍然认为,制造原子弹需要巨量的铀,因此他们专注于制造可控的链式反应。[65] 费米确定一个裂变的铀原子平均产生1.73个中子。这已经足够,但需要精心设计以最小化损失。[66] 西拉德设计了各种核反应堆的方案。“如果铀计划仅仅依靠想法来进行,”维格纳后来说道,“那除了莱奥·西拉德,没有人能做到。”[65]
\subsubsection{冶金实验室}
\begin{figure}[ht]
\centering
\includegraphics[width=6cm]{./figures/6edb489f3db5509d.png}
\caption{冶金实验室的科学家们,西拉德位于右二,穿着实验室外套。} \label{fig_Szilar_5}
\end{figure}
在1941年12月6日的会议上,国家防务研究委员会决定全力推进原子弹的生产。这一决定因次日日本袭击珍珠港,使美国卷入第二次世界大战而变得更加紧迫。1942年1月,罗斯福正式批准了这一决定。来自芝加哥大学的阿瑟·H·科普顿被任命为研究与开发负责人。尽管西拉德反对,科普顿将所有从事反应堆和钚研究的小组集中到了芝加哥大学的冶金实验室。科普顿制定了一个雄心勃勃的计划:在1943年1月之前实现链式反应,在1944年1月之前开始在核反应堆中生产钚,并在1945年1月之前制造出原子弹。[67]

1942年1月,西拉德作为研究员加入了芝加哥的冶金实验室,后来成为首席物理学家。[67] 阿尔文·温伯格指出,西拉德是项目中的“搅局者”,总是提出令人尴尬的问题。[68] 西拉德提供了重要的见解。尽管铀-238在慢速减速中子作用下不容易发生裂变,但它仍可能与裂变产生的快速中子发生裂变。这一效应虽小,但至关重要。[69] 西拉德提出了一些改进铀封装过程的建议,[70] 并与大卫·古林斯基和埃德·克鲁茨合作,研究了一种从铀盐中回收铀的方法。[71]

当时一个令人头痛的问题是生产反应堆应该如何冷却。采取保守的观点,认为每个中子都必须被保存,最初大多数人认为应该使用氦气冷却,因为氦气吸收的中子非常少。西拉德认为,如果这是一个问题,那么液态铋会是更好的选择。他监督了使用液态铋的实验,但实际操作中的困难太大,最终无法克服。最后,维格纳提出的使用普通水作为冷却剂的计划获得了胜利。[68] 当冷却剂问题变得越来越激烈时,科普顿和曼哈顿计划的负责人,莱斯利·R·格罗夫斯少将,准备解除西拉德的职务,因为他仍是德国公民,但战争部长亨利·L·斯廷森拒绝这样做。[72] 因此,西拉德于1942年12月2日亲自见证了第一次人工自持核链式反应的实现,这是在斯塔格场的看台下进行的第一座核反应堆中发生的,并与费米握手。[73]

西拉德开始采购高质量的石墨和铀,这些是进行大规模链式反应实验所必需的材料。芝加哥大学这次演示和技术突破的成功,部分归功于西拉德的新原子理论、他的铀晶格设计,以及通过与石墨供应商的合作,识别并减轻了一个关键的石墨杂质(硼)的问题。[74]

西拉德于1943年3月成为美国国籍。[75] 在1940年11月他正式加入该项目之前,军方曾为他的发明提供25,000美元,他拒绝了。[76] 他与费米共同拥有核反应堆的专利。[77] 最终,他将自己的专利卖给了政府,以报销他的费用,大约为15,416美元,加上标准的1美元费用。[78] 他继续与费米和维格纳一起从事核反应堆设计,并被认为是“增殖反应堆”一词的创造者。[79]

怀着对人类生命和政治自由的持久热情,西拉德希望美国政府不要使用核武器,而是通过核武器的威胁迫使德国和日本投降。他也担心核武器的长期影响,预测美国使用核武器会引发与苏联的核军备竞赛。他起草了《西拉德请愿书》,主张核弹应该先向敌人展示,只有在敌人拒绝投降时才使用。临时委员会则选择了对城市使用原子弹,尽管西拉德和其他科学家表示反对。[80] 事后,他游说修改1946年的《原子能法》,将核能置于民间控制之下。[81]
\subsection{战争之后}
\begin{figure}[ht]
\centering
\includegraphics[width=6cm]{./figures/95611657315b9dda.png}
\caption{西拉德和诺曼·希尔贝里在芝加哥大学CP-1反应堆遗址合影,战后几年。这座建筑于1957年被拆除。} \label{fig_Szilar_6}
\end{figure}
1946年,西拉德在芝加哥大学获得了一个研究教授职位,使他能够从事生物学和社会科学的研究。他与曾在战争期间在冶金实验室工作的化学家阿龙·诺维克合作。两人认为生物学是一个尚未像物理学那样被充分探索的领域,准备迎接科学突破。事实上,生物学是西拉德在1933年就开始研究的领域,在他投入到核链反应的研究之前。[81] 这对组合取得了显著进展。他们发明了化学恒温器(chemostat),一种用于调节生物反应器中微生物生长速度的设备,[82][83] 并开发了测量细菌生长速度的方法。他们发现了反馈抑制,这是在生长和代谢等过程中一个重要的因素。[84] 西拉德还为西奥多·帕克和菲利普·I·马库斯提供了至关重要的建议,帮助他们在1955年成功克隆了人类细胞。[85]
\subsubsection{个人生活}  
在与后来的妻子特鲁德·“翠德”·魏斯的关系之前,莱奥·西拉德的生活伴侣是幼儿园教师和歌剧女歌手格尔达·菲利普斯博恩(Gerda Philipsborn)。她曾在柏林为难民儿童的庇护所工作,并于1932年移居印度,继续从事这一工作。[86][87] 西拉德于1951年10月13日在纽约与医生特鲁德·魏斯结婚,举行了民事仪式。他们自1929年起认识,并且此后经常通信和互相访问。魏斯于1950年4月在科罗拉多大学担任教职,西拉德也开始在她位于丹佛的家中住上好几个星期,而在此之前他们从未一起度过超过几天的时间。当时,美国社会对未婚异性同居持反感态度,而在她的一名学生发现他们同居后,西拉德开始担心她可能会失去工作。他们的关系一直是异地恋,并且他们保持着婚姻的低调。许多朋友感到震惊,因为他们一直认为西拉德是一个天生的单身汉。[89][90]
\subsubsection{著作}  
1949年,西拉德写了一篇短篇小说《我作为战犯的审判》,在其中他想象自己因美国在与苏联的战争中失败而被指控犯有反人类罪。[91] 他公开警告可能会发展出盐化热核弹,并在1950年2月26日的芝加哥大学圆桌广播节目中解释说,一枚足够大的热核弹,如果使用特定但常见的材料,可能会消灭人类。[92] 他的评论,以及汉斯·贝特、哈里森·布朗和弗雷德里克·赛茨(参与该节目的另外三位科学家)的评论,遭到原子能委员会前主席大卫·利连塔尔的攻击。批评和西拉德的回应被公开发表。[92] 《时代》杂志将西拉德比作“小鸡小鸟”,而原子能委员会则驳回了他的观点,但科学家们讨论是否具有可行性;《原子科学家公报》委托詹姆斯·R·阿诺德进行了一项研究,他得出结论认为是可行的。[95] 物理学家W·H·克拉克认为,50兆吨的钴炸弹理论上确实具有产生足够持久辐射的潜力,可能成为末日武器,[96] 但他认为,即使如此,“也有足够的人可以找到避难所,等到辐射消退,再出来重新开始。”[94]

1961年,他提出了“矿坑城市”的概念,这是相互确保毁灭的早期例子。[97][98]

Szilard出版了一本短篇小说集《海豚的声音》(1961年),在其中他探讨了冷战带来的道德和伦理问题,以及他自己在原子武器开发中的角色。书中同名故事描述了一个位于中欧的国际生物学研究实验室。1962年,他与Victor F. Weisskopf、James Watson和John Kendrew会面后,这个设想变成了现实。当欧洲分子生物学实验室成立时,图书馆被命名为Szilard图书馆,图书馆的印章上也刻有海豚图案。他获得的其他荣誉包括1959年的“和平原子奖”和1960年的“年度人文学者”。1970年,月球远端的一个陨石坑以他的名字命名。1974年成立的Leo Szilard讲座奖由美国物理学会颁发,以表彰他。
\subsubsection{癌症诊断和治疗}  
1960年,Szilard被诊断为膀胱癌。他在纽约纪念斯隆-凯特琳癌症中心接受了钴-60治疗,这种治疗方案给予了他很高的控制权。1962年,他进行了第二轮治疗,增加了剂量。较高的剂量发挥了作用,他的癌症再也没有复发。
\subsubsection{最后的岁月}
\begin{figure}[ht]
\centering
\includegraphics[width=6cm]{./figures/a912a1d82642a32f.png}
\caption{“Salk Institute” 翻译为 “索尔克研究所”。} \label{fig_Szilar_7}
\end{figure}
Szilard在他的最后几年担任位于加利福尼亚州圣地亚哥拉荷亚社区的索尔克生物学研究所的研究员,该所他曾帮助创建。[106] 1962年,Szilard创立了“可居住世界委员会”,旨在向国会、白宫和美国公众传达关于核武器的“理性之声”。[107] 1963年7月,他被任命为该委员会的非驻地研究员,并于1964年4月1日成为驻地研究员,此前他在2月搬到圣地亚哥。[108] 他和Trude住在位于Hotel del Charro的一个小屋里。1964年5月30日,他在睡梦中因心脏病发作去世;当Trude醒来时,无法将他复活。[109][110][111] 他的遗体被火化。[112]

他的档案保存在加利福尼亚大学圣地亚哥分校的图书馆。[108] 2014年2月,图书馆宣布收到来自国家历史出版与档案委员会的资助,用于将其从1938年到1998年的档案数字化。[113]
\subsection{专利}  
\begin{itemize}
\item GB 630726 — 关于化学元素转变的改进 — L. Szilard,申请于1934年6月28日,批准于1936年3月30日  
\item 美国专利2,708,656 — 中子反应堆 — E. Fermi,L. Szilard,申请于1944年12月19日,批准于1955年5月17日  
\item 美国专利1,781,541 — 爱因斯坦冰箱 — 与阿尔伯特·爱因斯坦共同开发,申请于1926年,批准于1930年11月11日
\end{itemize}
\subsection{荣誉与纪念}  
\begin{itemize}
\item 原子和平奖,1959年  
\item 阿尔伯特·爱因斯坦奖,1960年  
\item 美国人文主义协会年度人文主义者,1960年  
\item 位于月球背面的Szilard陨石坑,命名于1970年  
\item 莱奥·Szilard讲座奖,1974年起  
\item 小行星38442 Szilárd,1999年发现[114]  
\item Leószilárdite矿物,2016年报道
\end{itemize}
\subsection{在媒体中}  
Szilard由Máté Haumann在2023年克里斯托弗·诺兰执导的电影《奥本海默》中饰演。[115] Szilard还是一部名为《原子》的音乐剧的主题人物。[116]
\subsection{参见}  
\begin{itemize}
\item 火星人(科学家)  
\item Szilard点
\end{itemize}
\subsection{注释}  
\begin{enumerate}
\item Marx, György. "A marslakók legendája"。原文档案保存于2022年4月9日,2020年4月7日查阅。  
\item Lanouette & Silard 1992,第10-13页。  
\item Lanouette & Silard 1992,第13-15页。  
\item Lanouette & Silard 1992,第167页。  
\item Byers, Nina. "Fermi and Szilard"。2015年5月23日查阅。  
\item Frank 2008,第244-246页。  
\item Blumesberger, Doppelhofer & Mauthe 2002,第1355页。  
\item Szilard, Leo(1979年2月)。“他版本的事实”。《原子科学家公报》。35 (2): 37-40. Bibcode:1979BuAtS..35b..37S. doi:10.1080/00963402.1979.11458587. ISSN 0096-3402。2023年5月29日查阅。  
\item Lanouette & Silard 1992,第36-41页。  
\item Bess 1993,第44页。
\item Lanouette & Silard 1992,第42页。  
\item Lanouette & Silard 1992,第44-46页。  
\item Lanouette & Silard 1992,第44-49页。  
\item Lanouette & Silard 1992,第49-52页。  
\item Lanouette & Silard 1992,第56-58页。  
\item Hargittai 2006,第44页。  
\item Szilard, Leo(1925年12月1日)。“关于将现象学热力学扩展到波动现象”。《物理学杂志》(德文)。32 (1): 753-788. Bibcode:1925ZPhy...32..753S. doi:10.1007/BF01331713. ISSN 0044-3328. S2CID 121162622。
\item Lanouette & Silard 1992,第60-61页。  
\item Szilard, Leo(1929)。“关于智能生物干预下热力学系统中的熵减少”。《物理学杂志》(德文)。53 (11-12): 840-856. Bibcode:1929ZPhy...53..840S. doi:10.1007/BF01341281. ISSN 0044-3328. S2CID 122038206。可在Aurellen.org在线获取英文版。  
\item Lanouette & Silard 1992,第62-65页。  
\item 冯·诺依曼, 约翰(1949)。“诺伯特·维纳《控制论》书评”。《今日物理》. 2: 33-34。  
\item Kline, Ronald(2015)。《控制论的时刻:或我们为什么称我们的时代为信息时代》。约翰·霍普金斯大学出版社。  
\item Telegdi, V. L.(2000)。“Szilard作为发明家:加速器及更多”。《今日物理》. 53 (10): 25-28. Bibcode:2000PhT....53j..25T. doi:10.1063/1.1325189。
\item Calaprice & Lipscombe 2005,第110页。  
\item Lanouette & Silard 1992,第101-102页。  
\item Lanouette & Silard 1992,第83-85页。  
\item Dannen, Gene(1998年2月9日)。“发明家Leo Szilard:幻灯片展示”。2015年5月24日查阅。  
\item 美国专利1,781,541  
\item Fraser 2012,第71页。  
\item Rhodes 1986,第26页。  
\item Lanouette & Silard 1992,第119-122页。  
\item Lanouette & Silard 1992,第131-132页。  
\item Rhodes 1986,第27页。  
\item Lanouette & Silard 1992,第133-135页。
\item Rotblat, J.(1973年3月)。“Leo Szilard的十六面貌”。《自然》. 242 (5392): 67-68. Bibcode:1973Natur.242...67R. doi:10.1038/242067a0. ISSN 1476-4687. S2CID 4163940.  
\item GB专利630726,Leo Szilard,“关于化学元素转变的改进”,发表于1949年9月28日,批准于1936年3月30日  
\item 《专利与设计法》(1907年,英国)第30条:“将某些发明的专利转交给战争部或海军部”。国家档案馆,代表英国政府。2020年1月7日查阅 - 通过legislation.gov.uk网站。  
\item Rhodes 1986,第224-225页。  
\item 《专利与设计法》(1907年,英国)第1条的变动。国家档案馆,代表英国政府。2020年1月7日查阅 - 通过legislation.gov.uk网站。第1-46条被《专利法1949》(第87号)和《注册设计法1949》(第88号)废除,第48条,附表2。  
\item “大约1980年,Szilard在伦敦街角构思链式反应的地点”。Leo Szilard档案。MSS 32。加州大学圣地亚哥分校图书馆,特藏与档案馆。
\item 来自约翰·弥尔顿(1667年)《失乐园》,第一卷,第3节的引用  
\item Rhodes 1986,第13页。  
\item 《H.G. 威尔斯与科学想象力》。《弗吉尼亚季刊》评论。2022年8月6日查阅。  
\item Szilard, L.; Chalmers, T. A.(1934年9月22日)。“通过费米效应从轰击同位素中化学分离放射性元素”。《自然》。134 (3386): 462. Bibcode:1934Natur.134..462S. doi:10.1038/134462b0. ISSN 0028-0836. S2CID 4129460。  
\item Szilard, L.; Chalmers, T. A.(1934年9月29日)。“通过伽玛射线释放的中子探测:一种诱发放射性的全新技术”。《自然》。134 (3387): 494–495. Bibcode:1934Natur.134..494S. doi:10.1038/134494b0. ISSN 0028-0836. S2CID 4111335。
\item Lanouette & Silard 1992,第145-148页。  
\item Lanouette & Silard 1992,第148页。  
\item Brasch, A.; Lange, F.; Waly, A.; Banks, T. E.; Chalmers, T. A.; Szilard, Leo; Hopwood, F. L.(1934年12月8日)。“通过X射线从铍中释放中子:利用电子管诱发放射性”。《自然》。134 (3397): 880. Bibcode:1934Natur.134..880B. doi:10.1038/134880a0. ISSN 0028-0836. S2CID 4106665。  
\item Lanouette & Silard 1992,第166-167页。  
\item Lanouette & Silard 1992,第172-173页。  
\item Lanouette, William(1992年12月)。“Szilard的思想,Fermi的物理学”。《原子科学家公报》。48 (10): 18. Bibcode:1992BuAtS..48j..16L. doi:10.1080/00963402.1992.11460136 – 通过Taylor & Francis。
\item Lanouette & Silard 1992,第178-179页。  
\item Lanouette & Silard 1992,第182-183页。  
\item Lanouette & Silard 1992,第186-187页。  
\item Rhodes 1986,第291页。  
\item Rhodes 1986,第292页。  
\item Lanouette & Silard 1992,第194-195页。  
\item 原子遗产基金会。“爱因斯坦致富兰克林·D·罗斯福的信”。原始版本存档于2022年6月17日。2007年5月26日查阅。  
\item 原子遗产基金会。“爸,这需要采取行动!”。原始版本存档于2012年10月29日。2007年5月26日查阅。
\item Hewlett & Anderson 1962,第19-21页。  
\item Lanouette & Silard 1992,第223-224页。  
\item Weinberg 1994b。  
\item Lanouette & Silard 1992,第222页。  
\item Bethe, Hans A.(2000年3月27日)。“德国铀计划”。《今日物理》. 53 (7): 34-36. Bibcode:2000PhT....53g..34B. doi:10.1063/1.1292473。  
\item Lanouette & Silard 1992,第227页。  
\item Hewlett & Anderson 1962,第28页。  
\item Lanouette & Silard 1992,第227-231页。  
\item Weinberg 1994a,第22-23页。  
\item Weinberg 1994a,第17页。  
\item Weinberg 1994a,第36页。  
\item Lanouette & Silard 1992,第234-235页。  
\item Lanouette & Silard 1992,第238-242页。  
\item Lanouette & Silard 1992,第243-245页。
\end{enumerate}