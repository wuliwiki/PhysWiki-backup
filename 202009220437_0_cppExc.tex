% C++ 异常处理

在初学编程时, 遇到错误我们往往就直接用 \verb|exit()| 终止程序. 但有时候我们不希望程序终止, 而是希望程序自行对错误进行一定的处理. C 语言的常见办法是把函数的返回值(通常是整型)作为错误代码, \verb|0| 代表成功, 其他值对应不同类型的错误. 然而函数调用是重重嵌套的, 将错误代码层层传递是一件很麻烦的事情, 另外, 把错误处理和函数调用的语法分离开也可以使代码的结构更明显.

在 C++ 中有专门的异常处理机制, 一般使用 \verb|throw|(抛出某种类型的错误), \verb|try|(检测某段代码的运行) 和 \verb|catch|(处理某种类型的错误) 三个关键词完成. 来看一个简单的例子.

\begin{lstlisting}[language=cpp]
#include <iostream>
#include <string>
using namespace std;

struct err_info { string where, what;};

void fun2()
{
	err_info e;
	e.where = "fun2()"; e.what = "something wrong!";
	throw e;
}

void fun1() { fun2(); }

int main () {
	try { fun1(); }
	catch (err_info e) {
		cout << "where: " << e.where << endl;
		cout << "what: " << e.what << endl; 
	}
}
\end{lstlisting}
程序中 \verb|main()| 调用 \verb|fun1()|, \verb|fun1()| 接着调用 \verb|fun2()|, 而 \verb|fun2()| 必然会出现一个异常, 抛出了一个类型为 \verb|err_info| 的对象. 这时无论 \verb|fun2()| 是否运行完成
