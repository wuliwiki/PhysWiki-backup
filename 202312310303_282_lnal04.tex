% 线性映射的核与象
% license Xiao
% type Tutor

\begin{issues}
\issueDraft
\end{issues}

线性映射是线性空间之间的同态映射,因此我们可以研究其核与象。
\begin{definition}{}
设$V,W$为域$\mathbb F$上的线性空间,$f:V\rightarrow W$为线性映射。

记$ker\,f=\{\boldsymbol x\in V|f(\boldsymbol x)=\boldsymbol 0\}$,称作线性映射$f$的核(kernel)。记$Im\,f=\{f(\boldsymbol x)|\boldsymbol x\in V\}$,称作线性映射$f$的象(Image)
\end{definition}
\begin{exercise}{}
$f,V,W$的定义同上。验证核与象分别是$V$及$W$的子空间。
\end{exercise}
关于核与象,有两个好用的结论。
\begin{itemize}
\item 核$ker\,f=\{\boldsymbol 0\}\Longleftrightarrow f$是单射。
\item 若象$Im\,f=W\Longleftrightarrow f$是满射
\end{itemize}
在此只证明第一个结论。

proof.
先验证充分条件。反证该映射并非单射,及至少存在两个向量映射到同一个向量,设为$\boldsymbol{x,y}$,那么我们有
\begin{equation}
f(\boldsymbol{x}-\boldsymbol{y})=f(\boldsymbol x)-f(\boldsymbol y)=\boldsymbol 0~,
\end{equation}
由于核只有向量$0$,因此$\boldsymbol {x}=\boldsymbol{y}$

再验证必要条件。假设存在一个非$0$向量映射到$0$,即$f(a^i\boldsymbol x_i)=0$,则$-f(a^i\boldsymbol x_i)=f(-a^i\boldsymbol x_i)=0$,与假设矛盾,证毕。
可见,第一条结论能成立多亏了该同态映射是线性的,这也是线性空间的一个好处。

线性空间的向量构成加法群,因而也有同态结构定理:
\begin{theorem}{}
设$V,W$是域$\mathbb F$上的线性空间。$f:V\rightarrow W$为线性映射。则有:
\begin{equation}
V/ker \,f\cong Im\,f~,
\end{equation}
\end{theorem}
Proof.
由于线性空间的同构只需要维度相同,所以我们只需要构建基之间的映射即可。

设$\{\boldsymbol e_i\}$为$ker\,f$上的一组基,扩充为$V$上全空间的基:$\{\boldsymbol e_i\}\cup \{\boldsymbol \theta_i\}$。由于核中元素都被映射为0.只要证明象的维度与$|\{\boldsymbol \theta_i\}|$一致即可。由于$f(a^i\boldsymbol e_i+b^i\boldsymbol \theta_i)=b^if(\boldsymbol \theta_i)$,而$f(\boldsymbol \theta_i)$是线性无关的,不然$span\{\boldsymbol \theta_i\}$就会有$ker\,f$的元素,与假设矛盾。证毕。
该证明同时也引出了以下定理:
\begin{lemma}{}
对于线性空间$V$和其上的线性映射$f$,我们有
$$\mathrm{dim}V=\mathrm{dim}\,ker\,f+\mathrm {dim}\,Im\,f~,$$
\end{lemma}
利用该定理,我们可以证明一条关于秩的定理:
\begin{theorem}{}
给定两个$n$阶方阵$A$和$B$,若$AB=0$,我们有
\begin{equation}
rank\,A+rank\,B\le n~,
\end{equation}
\end{theorem}
proof.

设$A,B$对应线性空间$V$的线性变换为$f_A,f_B$,该定理又可理解为$Im\,f_A+Im\,f_B\le n$。

$AB=0$意味着$Im\,f_B\le ker\,f_A$,由定理2得:$ker\,f_A=n-Im\,f_A$,移项证毕。
