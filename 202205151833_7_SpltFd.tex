% 分裂域
% keys splitting field|正规扩张|regular extension


\pentry{域的扩张\upref{FldExp}}

本节我们要介绍一个在代数中非常基础且重要的概念:分裂域.简单来说,分裂域就是在一个域中添加某个多项式的全体根所得到的扩域.

\begin{definition}{分裂域}
给定域$\mathbb{F}$及其上一个多项式$f(x)$.若存在扩域$\mathbb{K}/\mathbb{F}$,使得$f(x)$在$\mathbb{K}$上可以分解为$f(x)=\prod_{i=1}^n (x-a_i)$,且$\mathbb{K}=\mathbb{F}(a_1, a_2, \cdots, a_n)$,则称$\mathbb{K}$是$f(x)\in \mathbb{F}[x]$上的\textbf{分裂域(splitting field)}.
\end{definition}

定义看起来有些绕口,先说$f$在$\mathbb{K}$中可以分解,也就是说每一个根都存在,再说$\mathbb{K}$可以看成用这些根对$\mathbb{F}$进行扩域的结果.这么定义是因为我们要先确定元素$a_i$都存在,而为此就需要先确定$\mathbb{K}$存在.但是定义中只说了“若$\mathbb{K}$”存在,这个假设到底成立与否呢?答案是肯定的.

\begin{theorem}{分裂域的存在性}
给定域$\mathbb{F}$及其上一个多项式$f(x)$,则$f(x)\in \mathbb{F}[x]$上的分裂域存在.
\end{theorem}

\textbf{证明}:

当$\opn{deg}f=1$时,定理自然成立,此时$f\in\mathbb{F}[x]$的分裂域就是其本身.

首先在环$\mathbb{F}[x]$上对元素$f(x)$进行因式分解\footnote{也就是画出它的一棵\textbf{真因子树}\upref{FctTre}.},得到其不可约因子.任选其中一个不可约因子$h(x)$,如果$\opn{deg} h = 1$,则跳过本段接下来的步骤.构造环$R_1=\mathbb{F}(x)/<h(x)>=\mathbb{F}(a_1)$,再取其分式域$\mathbb{F}_1$,则$\mathbb{F}_1$就是$\mathbb{F}$的单扩张$\mathbb{F}(a_1)$.

由\textbf{多项式环}\upref{RPlynm}的\autoref{RPlynm_the1}~\upref{RPlynm},$(x-a_1)|h(x)$,因此在$\mathbb{F}_1$上可以分解出$h_1(x)=h(x)/(x-a_1)$.如果$\opn{deg}h_1 = 1$,则跳过本段接下来的步骤.对$h_1(x)$进行相同的操作:构造环$R_2=\mathbb{F}(x)/<h_1(x)>=\mathbb{F}(a_2)$,再取其分式域$\mathbb{F}_2=\mathbb{F}_1(a_2)=\mathbb{F}(a_1, a_2)$.

以此类推,直到$h(x)$在$\mathbb{F}_{k_1}$上分解为一阶多项式之积.

接下来,取$f$在$\mathbb{F}_{k_1}$上的不可约因子$g(x)$,如果$\opn{deg} g = 1$,则跳过本段接下来的步骤.执行相同的扩域操作,直到得到$\mathbb{F}_{k_1+k_2}$,使得$g$在$\mathbb{F}_{k_1+k_2}$上分解为一阶多项式之积.

以此类推,最终可以得到$\mathbb{F}_k$,使得$f$在$\mathbb{F}_k$上可以分解为一阶多项式之积.则$\mathbb{F}_k$就是$f\in\mathbb{F}[x]$的分裂域.

\textbf{证毕}.


\begin{example}{分裂域的一个例子}


在有理数域$\mathbb{Q}$上有多项式$f(x)=(x^2-2)^2(x^2-3)(x^2-6)(x^2+1)$,其在$\mathbb{Q}$上有五个不可约因子:$(x^2-2), (x^2-2), (x^2-3), (x^2-6), (x^2+1)$.

考虑因子$(x^2-2)$,得到扩域$\mathbb{Q}(\sqrt{2})$.在$\mathbb{Q}(\sqrt{2})$有分解:
\begin{equation}
f(x)=(x+\sqrt{2})^2(x-\sqrt{2})^2(x^2-3)(x^2-6)(x^2+1)
\end{equation}

取其不可约因子$x^2-3$,得到扩域$\mathbb{Q}(\sqrt{2}, \sqrt{3})$.

在$\mathbb{Q}(\sqrt{2}, \sqrt{3})$上,$f$有分解:

\begin{equation}
\begin{aligned}
f(x)=&(x+\sqrt{2})^2(x-\sqrt{2})^2(x+\sqrt{3})(x-\sqrt{3})\times\\
&(x+\sqrt{2}\sqrt{3})(x-\sqrt{2}\sqrt{3})(x^2+1)
\end{aligned}
\end{equation}

取其不可约因子$x^2+1$,得到扩域$\mathbb{Q}(\sqrt{2}, \sqrt{3}, \I)$.

则$\mathbb{Q}(\sqrt{2}, \sqrt{3}, \I)=\{a+A\I+(b+B\I)\sqrt{2}+(c+C\I)\sqrt{3}+(d+D\I)\sqrt{6}|a, A, b, B, c, C, d, D\in\mahtbb{Q}\}$就是$f\in\mathbb{Q}[x]$的分裂域.

\end{example}

\begin{corollary}{}
设$\mathbb{K}$是$f(x)\in \mathbb{F}[x]$上的分裂域,则$[\mathbb{K}:\mathbb{F}]\leq \opn{deg}f$.
\end{corollary}























