% Galois 扩张
% 伽罗华扩域|伽罗瓦扩域|代数方程|根式解|古典数学难题|Galois理论基本定理|Fundamental Theorem of Galois Theory

\pentry{正规扩张\upref{NomEx},群作用\upref{Group3}}


完成了对正规扩张和可分扩张的讨论,我们引入极为核心的Galois扩域。在上述讨论中,我们时常涉及域自同构,也看到了域自同构和多项式的根之间对应的关系。对域自同构的结构的研究,将把我们引向著名的古典数学难题“多项式方程的根式解”。



\subsection{Galois扩张的基本性质}

\begin{definition}{Galois扩张}

若域扩张$\mathbb{K}/\mathbb{F}$是\textbf{正规}且\textbf{可分}的,那么称之为一个\textbf{伽罗华扩张(Galois extension)},或译作\textbf{伽罗瓦扩张}。

此时称$\mathbb{K}$\textbf{在}$\mathbb{F}$\textbf{上是Galois的}($\mathbb{K}$ \textbf{is Galois over} $\mathbb{F}$)。

\end{definition}

由于特征为零的域都是完美域,因此对这类域,正规扩张都是Galois扩张。

\begin{example}{}
$\mathbb{Q}(2^{1/3})/\mathbb{Q}$不是Galois扩域,因为它不正规,不包含$\opn{irr}(2^{1/3}, \mathbb{Q})(x)=x^3-2$的两个复数根。
\end{example}

\begin{theorem}{}
有限域都是其素域的Galois扩域。
\end{theorem}

\textbf{证明}:

有限域是其素域的有限扩张,而有限扩张都是代数扩张(\autoref{cor_FldExp_1}~\upref{FldExp})。

由于有限域都是完美域(\autoref{cor_SprbEx_4}~\upref{SprbEx}),故$\mathbb{Z}_p$的代数扩张都是可分扩张。

参照\textbf{有限域}\upref{FntFld}的讨论可知,有限域都是其素域的分裂域,从而是正规扩张。

\textbf{证毕}。


据正规扩张和可分扩张的知识,我们容易得到Galois扩张的几条性质:


\begin{theorem}{}\label{the_GExt_6}
设$\mathbb{K}/\mathbb{M}/\mathbb{F}$是域扩张链,且$\mathbb{K}/\mathbb{F}$是Galois扩张,那么$\mathbb{K}/\mathbb{M}$也是Galois扩张。
\end{theorem}

证明用可分扩张的继承性\autoref{lem_SprbEx_3}~\upref{SprbEx}和正规扩张的继承性\autoref{the_NomEx_6}~\upref{NomEx}得到。

\begin{theorem}{}\label{the_GExt_5}
设$\mathbb{K}/\mathbb{M}/\mathbb{F}$是域扩张链,其中$\mathbb{K}/\mathbb{F}$是Galois扩张,$\mathbb{M}/\mathbb{F}$是\textbf{正规}扩张,那么$\mathbb{M}/\mathbb{F}$是Galois扩张。
\end{theorem}

\textbf{证明}:

由\autoref{def_SprbEx_4}~\upref{SprbEx}易得,$\mathbb{K}/\mathbb{F}$是可分扩张且$\mathbb{M}\subseteq\mathbb{K}$,可推得$\mathbb{M}/\mathbb{F}$也是可分扩张。

\textbf{证明}。

\begin{theorem}{}\label{the_GExt_1}
如果$\mathbb{K}/\mathbb{F}$是Galois扩域,且域合成$\mathbb{EK}$存在,那么$\mathbb{EK}/\mathbb{EF}$是Galois扩域。
\end{theorem}

\textbf{证明}:

据\autoref{the_NomEx_7}~\upref{NomEx},$\mathbb{EK}/\mathbb{EF}$是正规扩域。

考虑到$\mathbb{K}$的元素全都是$\mathbb{F}$的可分元素,从而是$\mathbb{EF}$的可分元素,而$\mathbb{EK}=\mathbb{EF}(\mathbb{K})$,可知$\mathbb{EK}/\mathbb{EF}$是可分扩张。

\textbf{证毕}。



\begin{theorem}{}
设$\mathbb{K}/\mathbb{F}$和$\mathbb{E}/\mathbb{F}$都是Galois扩域,且域合成$\mathbb{EK}$存在,则$\mathbb{EK}/\mathbb{F}$是Galois扩域。
\end{theorem}

\textbf{证明}:

已知\autoref{the_GExt_1} 成立。

由$\mathbb{EK}/\mathbb{EF}$是正规扩张,及$\mathbb{F}[x]\subseteq\mathbb{EF}[x]$,知$\mathbb{EK}/\mathbb{F}$是正规扩张。

由\textbf{可分元素的封闭性}\autoref{cor_SprbE2_4}~\upref{SprbE2}知$\mathbb{EK}/\mathbb{F}$是可分扩张。

\textbf{证毕}。




\begin{theorem}{}
域$\mathbb{K}$的Galois扩域之交,还是它的Galois扩域
\end{theorem}

\textbf{证明}:

由正规扩张和可分扩张相交还是正规扩张和可分扩张,得证。

\textbf{证毕}。



\subsection{Galois群}

回顾\autoref{ex_Group_6}~\upref{Group},集合间的全体双射配合复合运算能构成群。既然域的自同构也是双射,我们也可以研究这些自同构构成的群。不过,相比于一般的域扩张,我们重点关注性质最良好的Galois扩张的情况。

\begin{definition}{Galois群}\label{def_GExt_2}
给定Galois域扩张$\mathbb{K}/\mathbb{F}$,称$\mathbb{K}$上全体保$\mathbb{F}$自同构构成的群为该扩张的\textbf{Galois 群},记为$\opn{Gal}(\mathbb{K}/\mathbb{F})$,或$\opn{Gal}(\mathbb{K}:\mathbb{F})$。
\end{definition}


随便举一个具体的例子:复数域$\mathbb{C}$之于实数域$\mathbb{R}$是一个Galois扩域:正规性来自$\mathbb{C}=\overline{\mathbb{R}}$的事实,可分性是由于$\mathbb{R}$是完美域。那么$\opn{Gal}(\mathbb{C}/\mathbb{R})$是哪个群?或者最基础的问题,这个群有几个元素?

单回答这个问题也许不难,不过我们可以直接得出一般的Galois群元素数量规则:

\begin{theorem}{}\label{the_GExt_2}
如果$\mathbb{K}/\mathbb{F}$是Galois扩域,那么
\begin{equation}
\abs{\opn{Gal}(\mathbb{K}/\mathbb{F})} = [\mathbb{K}:\mathbb{F}]
\end{equation}
\end{theorem}

\textbf{证明}:

当$[\mathbb{K}:\mathbb{F}]$有限时,正规扩张等价于分裂域,且由于可分扩张,因此适用\autoref{the_FldExp_4}~\upref{FldExp}的等号情况,得证。

当$[\mathbb{K}:\mathbb{F}]$无限时,任取$n$个根在$\mathbb{K}-\mathbb{F}$中的多项式$f_i\in\mathbb{F}[x]$,得到$\prod f_i\in\mathbb{F}[x]$的分裂域$\mathbb{F}_1$,则$[\mathbb{K}:\mathbb{F}]>[\mathbb{F}_1:\mathbb{F}]>n$。$\mathbb{F}_1\mathbb{F}$适用有限情况,其Galois群的元素数量等于其扩张次数,而由于$[\mathbb{K}:\mathbb{F}]$无限,还能再取根在$\mathbb{K}-\mathbb{F}_1$中的多项式$g_i\in\mathbb{F}[x]\subseteq \mathbb{F}_1[x]$,构成更大的分裂域,得到更多自同构,因此必有
\begin{equation}
\opn{Gal}(\mathbb{K}/\mathbb{F})>\opn{Gal}(\mathbb{F}_1/\mathbb{F})=[\mathbb{F}_1:\mathbb{F}]>n
\end{equation}
由$n$的任意性,则可知$\opn{Gal}(\mathbb{K}/\mathbb{F})=\infty$。

\textbf{证毕}。



由于$\mathbb{C}$是$x^2+1\in\mathbb{R}[x]$的分裂域,故易证扩张次数为$2$,结合\autoref{the_GExt_2} 就能确定$\opn{Gal}(\mathbb{C}/\mathbb{R})$只有两个元素。显然,除了恒等映射以外,求共轭映射也是一个保$\mathbb{R}$自同构,那这就已经找全了。



\subsubsection{不变子域与Galois群}

给定Galois子域,总能唯一确定一个Galois群,其定义见\autoref{def_GExt_2} 。于是,我们得到了从Galois子域集合到Galois群的一个映射。

给定域的自同构群,总能唯一确定一个不变子域,其定义见\autoref{def_GExt_1} 。于是,我们得到了从自同构集合到Galois子域集合的一个映射。

现在的问题是,上述这两个集合映射是不是双射?即Galois子域和Galois群之间有没有一一对应关系?直觉上好像是的,但我们依然需要严谨的讨论来确认。



\begin{definition}{不变子域}\label{def_GExt_1}
给定域$\mathbb{F}$,设$G$是$\mathbb{F}$的全体自同构群。

取$G$的子群$H$,则集合$\{a\in\mathbb{F}\mid \sigma(a)=a, \forall \sigma\in H\}$构成一个域,称为$\mathbb{F}$的$H$ \textbf{不变子域(fixed field of }$H$\textbf{)},或译作$H$ \textbf{固定子域},记为$\opn{Inv}_\mathbb{F}(H)$或$\opn{Fix}_\mathbb{F}(H)$。
\end{definition}

如果取$H=\opn{Gal}(\mathbb{C}/\mathbb{R})$,那么$\opn{Fix}_{\mathbb{C}}(H)=\mathbb{R}$。同样地,我们也可以直接给出更一般的情况:



\begin{theorem}{(Artin)}\label{the_GExt_3}
给定域$\mathbb{K}$,$G$是它的\textbf{全体自同构}群的\textbf{有限}子群。

则$\mathbb{K}/\opn{Fix}_\mathbb{K}(G)$是Galois扩张,且$\opn{Gal}(\mathbb{K}/\opn{Fix}_\mathbb{K}(G))=G$。
\end{theorem}

\textbf{证明}:

\textbf{任取}$\alpha\in\mathbb{K}$。由于$G$是有限群,故其轨道是有限的,不妨记为$G\alpha=\{\alpha_i\}_{i=1}^n$,其中$\alpha_1=\alpha$,各$\alpha_i$彼此不等。

构造多项式$f_\alpha(x)=\prod_{i=1}^n(x-\alpha_i)$。对于\textbf{任意}$\sigma\in G$,都有$\sigma G=G$,因此$\sigma(\{\alpha_i\})=\{\alpha_i\}$,即$\sigma$是$\{\alpha_i\}$的一个置换。于是,$f_\alpha$的各系数都在$\sigma$下不变,即都是$\opn{Fix}_\mathbb{K}(G)$的元素。因此,$f_\alpha$是$\alpha$在$\opn{Fix}_\mathbb{K}(G)$上的零化多项式。

注意$\alpha$的任意性。因此$f$的最小多项式的根全都在$\mathbb{K}$中,故$\mathbb{K}/\opn{Fix}_\mathbb{K}(G)$是正规扩张。由于各$\alpha_i$不相等,故$\alpha$是$\opn{Fix}_\mathbb{K}(G)$的可分元,故$\mathbb{K}/\opn{Fix}_\mathbb{K}(G)$是可分扩张。由此\textbf{得证}$\mathbb{K}/\opn{Fix}_\mathbb{K}(G)$是Galois扩张。

由不变子域的定义,显然$G\subseteq\opn{Gal}(\mathbb{K}/\opn{Fix}_\mathbb{K}(G))$。

由于任意元素$\alpha\in\mathbb{K}$的轨道中元素数量不会超过$\abs{G}$,$f_\alpha$的次数就是$\alpha$轨道中的元素数量,以及$f_\alpha$是$\alpha$的零化多项式,可知\textbf{任意}$\alpha$的最小多项式次数不会超过$\abs{G}$。因此,应用\autoref{cor_PrmtEl_3}~\upref{PrmtEl},可知$[\mathbb{K}:\opn{Fix}_\mathbb{K}(G)]\leq \abs{G}$。

由\autoref{the_GExt_2} ,可知$\abs{\opn{Gal}(\mathbb{K}/\opn{Fix}_\mathbb{K}(G))}=[\mathbb{K}:\opn{Fix}_\mathbb{K}(G)]\leq\abs{G}$,从而$\opn{Gal}(\mathbb{K}/\opn{Fix}_\mathbb{K}(G))\subseteq G$。

于是\textbf{得证}$G = \opn{Gal}(\mathbb{K}/\opn{Fix}_\mathbb{K}(G))$。

\textbf{证毕}。


\autoref{the_GExt_3} 暗示了自同构群和其不变子域的一一对应关系,即“给定\textbf{有限}自同构群$G$,则能被$G$保持不变的元素,就只能被$G$保持不变”——但是只针对$G$有限的情况。这也体现在以下性质中,注意此处不再需要\textbf{有限}性,以及描述反过来了:



\begin{theorem}{}\label{the_GExt_4}
设$\mathbb{K}/\mathbb{F}$是一个Galois扩张。则$\opn{Gal}(\mathbb{K}/\mathbb{F})$的不变子域就是$\mathbb{F}$。
\end{theorem}

\textbf{证明}:

记$\opn{Gal}(\mathbb{K}/\mathbb{F})=G$,其不变子域为$\mathbb{J}=\opn{Fix}_\mathbb{K}(G)$。

由Galois群的定义,$\mathbb{F}\subseteq\mathbb{J}$。

任取$\alpha\in\mathbb{K}-\mathbb{F}$,记$f_\alpha=\opn{Irr}(\alpha, \mathbb{F})$。则$f_\alpha$的次数大于1,且是可分多项式,故$\alpha$有关于$\mathbb{F}$的共轭元$\beta\neq \alpha$,且正规性保证了$\beta\in\mathbb{K}$中。

因此由\autoref{the_FldExp_4}~\upref{FldExp}第2条可知,存在$\sigma\in\opn{Gal}(\mathbb{K}/\mathbb{F})$使得$\sigma(\alpha)=\beta$。于是$\alpha\not\in \mathbb{J}$。因此$\mathbb{J}\subseteq\mathbb{F}$。

综上,$\mathbb{J}=\mathbb{F}$。

\textbf{证毕}。


\autoref{the_GExt_4} 可以简述为:“给定子域$\mathbb{F}$,全体保$\mathbb{F}$不变的同构,只能保$\mathbb{F}$不变”。




\subsubsection{总结}

我们归纳一下前面说到的两个定理。

\autoref{the_GExt_3} 是说“给定\textbf{有限}自同构群$G$,则$G$的不变子域的Galois群就是$G$”。

\autoref{the_GExt_4} 是说“给定子域$\mathbb{F}$,则$\mathbb{F}$的Galois群的不变子域就是$\mathbb{F}$”。


这两句话,都没有完整指出Galois群和不变子域的一一对应关系。如果取无限自同构真子群$G$,而它的不变子域的Galois群严格大于$G$,这不违反以上两句话。








\subsection{Galois理论基本定理}\label{sub_GExt_1}

这一小节中,我们深入讨论上一小节引出的“自同构群与不变子域的对应关系”。


\begin{lemma}{}\label{lem_GExt_1}
给定Galois扩张$\mathbb{K}/\mathbb{F}$,取它的三个中间域$\mathbb{M}_1, \mathbb{M}_2, \mathbb{M}_3$,并令$\opn{Gal}(\mathbb{K}/\mathbb{M}_i)=H_i$。则下列命题成立:

1。$\mathbb{M}_1\subseteq \mathbb{M}_2 \iff H_1\supseteq H_2$;

2。$\mathbb{M}_1=\mathbb{M}_2\mathbb{M}_3 \iff H_1=H_2\cap H_3$。
\end{lemma}

\textbf{证明}:

1。

$\mathbb{M}_1\subseteq\mathbb{M}_2 \iff$保$\mathbb{M}_2$不变的同构也必保$\mathbb{M}_1$不变$\iff H_1\supseteq H_2$。

2。

$\mathbb{M}_1=\mathbb{M}_2\mathbb{M}_3\implies \mathbb{M}_1\supseteq\mathbb{M}_2\cup\mathbb{M}_3\implies H_1\subseteq H_2\cap H_3$。

$\mathbb{M}_1=\mathbb{M}_2\mathbb{M}_3\implies$ 保$\mathbb{M}_2$和$\mathbb{M}_3$都不变的同构,必保$\mathbb{M}_1$不变$\implies H_2\cap H_3\subseteq H_1$。

综合这两条逻辑链,得$\mathbb{M}_1=\mathbb{M}_2\mathbb{M}_3 \implies H_1=H_2\cap H_3$。

$H_1=H_2\cap H_3\iff$“自同构保$\mathbb{M}_1$不变当且仅当它保$\mathbb{M}_2$和$\mathbb{M}_3$都不变”$\implies \mathbb{M}_1\subseteq\mathbb{M}_2\mathbb{M}_3$\footnote{最后这个箭头是因为,如果存在$\alpha\in\mathbb{M}_1-\mathbb{M}_2\mathbb{M}_3$,那么由可分性,$\alpha$必有关于$\mathbb{M}_2\mathbb{M}_3$的共轭,那么存在一个保$\mathbb{M}_2\mathbb{M}_3$的同构,将$\alpha$映射到它的共轭上,从而该同构不保$\mathbb{M}_1$不变,即$H_1\subsetneq H_2\cap H_3$。因此,$\alpha$不存在。}。

$H_1=H_2\cap H_3\iff$“自同构保$\mathbb{M}_1$不变当且仅当它保$\mathbb{M}_2$和$\mathbb{M}_3$都不变”$\implies \mathbb{M}_1\supseteq\mathbb{M}_2\mathbb{M}_3$\footnote{道理和前一条类似,不能存在$\alpha\in\mathbb{M}_2\mathbb{M}_3-\mathbb{M}_1$,否则可以构造出保$\mathbb{M}_1$但不保$\mathbb{M}_2\mathbb{M}_3$的同构,使得$H_1\supsetneq H_2\cap H_3$。}。

综合这两条逻辑链,得$H_1=H_2\cap H_3 \implies \mathbb{M}_1=\mathbb{M}_2\mathbb{M}_3$。


\textbf{证毕}。









\begin{theorem}{}
给定\textbf{有限}Galois扩张$\mathbb{K}/\mathbb{F}$,取它的两个中间域$\mathbb{M}_1, \mathbb{M}_2$,并令$\opn{Gal}(\mathbb{K}/\mathbb{M}_i)=H_i$。则下列命题成立:

1。$\opn{Gal}(\mathbb{K}/\mathbb{M}_1\cap\mathbb{M}_2)=<H_1, H_2>$\footnote{$<H_1, H_2>$即由$H_1\cup H_2$生成的$\opn{Gal}(\mathbb{K}/\mathbb{F})$的子群。}。

2。$\mathbb{M}_1$和$\mathbb{M}_2$关于$\mathbb{F}$共轭$\iff H_1$和$H_2$在$\opn{Gal}(\mathbb{K}/\mathbb{F})$中共轭。

\end{theorem}

\textbf{证明}:

1。

由于$H_1$和$H_2$中的同构都能保$\mathbb{M}_1$和$\mathbb{M}_2$不变,故$\opn{Gal}(\mathbb{K}/\mathbb{M}_1\cap\mathbb{M}_2)\supseteq<H_1, H_2>$。

由于是有限扩张及可分扩张,故据\autoref{cor_PrmtEl_2}~\upref{PrmtEl},$\mathbb{M}_i=\mathbb{M}_1\cap\mathbb{M}_2(\alpha_i)$。显然,$\alpha_2\not\in\mathbb{M}_1$。

设$f\in\opn{Gal}(\mathbb{K}/\mathbb{M}_1\cap\mathbb{M}_2)$,再任取$h\in\opn{Gal}(\mathbb{K}/\mathbb{M}_1)$,则据\autoref{the_FldExp_5}~\upref{FldExp},$h$可开拓为$\mathbb{K}\to\mathbb{K}$的保$\mathbb{M}_1$自同构,其中$h(\alpha_2)=f(\alpha_2)$。于是$h^{-1}\circ f\in\opn{Gal}(\mathbb{K}/\mathbb{M}_2)$。因此,$f$必定是$\opn{Gal}(\mathbb{K}/\mathbb{M}_i)$中元素相乘的结果。

2。

$\implies$:

由题设,存在$\sigma\in\opn{Gal}(\mathbb{K}/\mathbb{F})$,使得$\sigma \mathbb{M}_1=\mathbb{M}_2$。

任取$f\in H_1$,则易证$\sigma f\sigma^{-1}\in H_2$。因此$\sigma H_1\sigma^{-1}=H_2$\footnote{看起来,$\sigma f\sigma^{-1}\in H_2$比$\sigma H_1\sigma^{-1}=H_2$更强,实际上在Galois理论范围内这二者是等价的(?),原因正是Galois群和不变子域的对应性。}。

$\impliedby$:

由题设,存在$\sigma\in\opn{Gal}(\mathbb{K}/\mathbb{F})$,使得$\sigma H_1\sigma^{-1}=H_2$。

任取$a\in\mathbb{M}_{1}$,则$H_2 \sigma a=\sigma H_1\sigma^{-1}\sigma a=\sigma H_1 a=\sigma a$。因此,$\sigma a\in\opn{Fix}_\mathbb{K}(H_2)$,或者说$\sigma\mathbb{M}_1\subseteq\opn{Fix}_\mathbb{K}(H_2)$。由\autoref{the_GExt_4} ,$\opn{Fix}_\mathbb{K}(H_2)=\mathbb{M}_2$。

故$\sigma\mathbb{M}_1\subseteq\mathbb{M}_2$。

对偶地,可证得$\mathbb{M}_1\supseteq\sigma^{-1}\mathbb{M}_2$。

因此$\sigma\mathbb{M}_1=\mathbb{M}_2$。




\textbf{证毕}。




接下来的性质,和群同态基本定理非常相似。

\begin{theorem}{}\label{the_GExt_8}
设$\mathbb{K}/\mathbb{F}$是\textbf{Galois扩张}(不要求有限),且存在中间域$\mathbb{M}$。

则$\mathbb{M}/\mathbb{F}$是正规扩张$\iff$ $\opn{Gal}(\mathbb{K/\mathbb{M}})\vartriangleleft\opn{Gal}(\mathbb{K/\mathbb{F}})$,且$\opn{Gal}(\mathbb{M}/\mathbb{F})\cong \opn{Gal}(\mathbb{K}/\mathbb{F})/\opn{Gal}(\mathbb{K}/\mathbb{M})$。
\end{theorem}

\textbf{证明}:

$\mathbb{M}/\mathbb{F}$是正规扩张$\iff$对于任意$\sigma\in\opn{Gal}(\mathbb{K}/\mathbb{F})$,都有$\sigma\mathbb{M}=\mathbb{M}$ $\iff$ 对于任意$f\in\opn{Gal}(\mathbb{K}/\mathbb{M})$,都有$\sigma f \sigma^{-1}\in\opn{Gal}(\mathbb{K}/\mathbb{M})$\footnote{因为任取$a\in\mathbb{M}$,都有$\sigma f \sigma^{-1} a=a$。} $\iff\opn{Gal}(\mathbb{K/\mathbb{M}})\vartriangleleft\opn{Gal}(\mathbb{K/\mathbb{F}})$。

利用映射的限制,构造$\opn{Gal}(\mathbb{K}/\mathbb{F})\to\opn{Gal}(\mathbb{M}/\mathbb{F})$的映射,记为$\varphi$,其定义为:$\varphi(\sigma) = \sigma\mid_{\mathbb{M}}$。显然,由于是限制映射,$\varphi$是一个群同态。

取$\sigma\in\opn{Gal}(\mathbb{K}/\mathbb{F})$,则$\varphi(\sigma)=\opn{id}\mid_{\mathbb{M}}$当且仅当$\sigma$保$\mathbb{M}$不变,即$\sigma\in\opn{Gal}(\mathbb{K}/\mathbb{M})$。因此,$\opn{ker}\varphi=\opn{Gal}(\mathbb{K}/\mathbb{M})$。

于是得证$\opn{Gal}(\mathbb{M}/\mathbb{F})\cong \opn{Gal}(\mathbb{K}/\mathbb{F})/\opn{Gal}(\mathbb{K}/\mathbb{M})$。


\textbf{证毕}。


注意一点:由\autoref{the_GExt_5} ,$\mathbb{M}/\mathbb{F}$是\textbf{正规}扩张$\iff$ $\mathbb{M}/\mathbb{F}$是\textbf{Galois}扩张。



\begin{theorem}{}\label{the_GExt_9}
设$\mathbb{K}/\mathbb{F}$是\textbf{有限}Galois扩张,且存在中间域$\mathbb{M}$。

则$\abs{\opn{Gal}(\mathbb{K}/\mathbb{F})}/\abs{\opn{Gal}(\mathbb{K}/\mathbb{M})}=[\mathbb{M}:\mathbb{F}]$。
\end{theorem}

\textbf{证明}:

由\autoref{cor_PrmtEl_2}~\upref{PrmtEl},有限Galois扩张都是单代数扩张。由\autoref{the_GExt_6} ,$\mathbb{K}/\mathbb{M}$也是有限Galois扩张。

设$\mathbb{K}=\mathbb{F}(\alpha)$,则$\mathbb{K}=\mathbb{M}(\alpha)$。记$f=\opn{Irr}(\alpha, \mathbb{F})$,$h=\opn{Irr}(\alpha, \mathbb{M})$。

由\autoref{the_SpltFd_1}~\upref{SpltFd},$\abs{\opn{Gal}(\mathbb{K}/\mathbb{F})}=\deg f$,$\abs{\opn{Gal}(\mathbb{K}/\mathbb{F})}=\deg h$。

由\autoref{the_FldExp_1}~\upref{FldExp},$[\mathbb{K}:\mathbb{F}]=\deg f$,$[\mathbb{K}:\mathbb{F}]=\deg h$。又由\autoref{the_FldExp_3}~\upref{FldExp},$[\mathbb{M}:\mathbb{F}]=[\mathbb{K}/\mathbb{F}]/[\mathbb{K}/\mathbb{M}]$。

综上,
\begin{equation}
\abs{\opn{Gal}(\mathbb{K}/\mathbb{F})}/\abs{\opn{Gal}(\mathbb{K}/\mathbb{M})}=\deg f/\deg h=[\mathbb{M}:\mathbb{F}]
\end{equation}

\textbf{证毕}。


\autoref{the_GExt_9} 实际上可以去掉“有限”的要求,参见\autoref{lem_GExInf_1}~\upref{GExInf}。






\begin{theorem}{}\label{the_GExt_7}
设$\mathbb{K}/\mathbb{F}$是\textbf{有限}Galois扩张,$\mathbb{E}/\mathbb{F}$是域扩张,且域合成$\mathbb{KE}$存在。

则$\mathbb{KE}/\mathbb{E}$和$\mathbb{K}/\mathbb{K}\cap\mathbb{E}$都是有限Galois扩张,且$\opn{Gal}(\mathbb{KE}/\mathbb{E})\cong\opn{Gal}(\mathbb{K}/\mathbb{K}\cap\mathbb{E})$。
\end{theorem}

\textbf{证明}:

由于$\mathbb{E}\mathbb{F}=\mathbb{E}$,据\autoref{the_GExt_1} ,知$\mathbb{KE}/\mathbb{E}$是Galois扩域。设$\{\alpha_i\}_{i=1}^n$是$\mathbb{K}$作为$\mathbb{F}$上线性空间的基,即$\mathbb{K}$中元素都形如$\sum c_i\alpha_i$,其中$c_i\in\mathbb{F}$;那么$\mathbb{KE}$中元素都形如$\sum e_ic_i\alpha_i$,其中$e_i\in\mathbb{E}$;且有$e_ic_i\in\mathbb{E}$。于是,$\{\alpha_i\}_{i=1}^n$也是$\mathbb{K}$作为$\mathbb{E}$上线性空间的基,从而得证有限性。

$\mathbb{K}\cap\mathbb{E}$是$\mathbb{K}/\mathbb{F}$的中间域,故由\autoref{the_GExt_6} 知$\mathbb{K}/\mathbb{K}\cap\mathbb{E}$是Galois扩张,由\autoref{the_FldExp_3}~\upref{FldExp}知有限性。

下证$\opn{Gal}(\mathbb{KE}/\mathbb{E})\cong\opn{Gal}(\mathbb{K}/\mathbb{K}\cap\mathbb{E})$。

注意$\mathbb{KE}\cap \mathbb{K}=\mathbb{K}$。构造\textbf{群同态}$\varphi:\opn{Gal}(\mathbb{KE}/\mathbb{E})\to\opn{Gal}(\mathbb{K}/\mathbb{K}\cap\mathbb{E})$,定义为$\varphi (\sigma) = \sigma\mid_{\mathbb{K}}$。显然这是一个\textbf{满同态}。

如果存在$\sigma_1, \sigma_2\in\opn{Gal}(\mathbb{KE}/\mathbb{E})$,使得$\varphi(\sigma_1)=\varphi(\sigma_2)$,则$\sigma_1\sigma_2^{-1}$保$\mathbb{K}$不变。又因为$\sigma_1$和$\sigma_2$都保$\mathbb{E}$不变,故$\sigma_1\sigma_2^{-1}$保$\mathbb{KE}$不变。由此可知,$\sigma_1=\sigma_2$。因此$\varphi$是\textbf{单同态}。

综上,$\varphi$是一个群同构。

\textbf{证毕}。











\begin{theorem}{Galois理论基本定理(Fundamental Theorem of Galois Theory)}\label{the_GExt_10}

设$\mathbb{K}/\mathbb{F}$是一个\textbf{有限}的Galois扩张,则以下命题成立:

1。设存在中间域$\mathbb{M}$,则$\mathbb{K}/\mathbb{M}$也是有限Galois扩张,且$\mathbb{M}=\opn{Fix}_\mathbb{K}(\opn{Gal}(\mathbb{K}/\mathbb{M}))$。

2。设$H$是$\opn{Gal}(\mathbb{K}/\mathbb{F})$的子群,则$\mathbb{M}=\opn{Fix}_\mathbb{K}(H)$是$\mathbb{K}/\mathbb{F}$的中间域,且$\opn{Gal}(\mathbb{K}/\mathbb{M})=H$。

% 上述性质将$\opn{Gal}(\mathbb{K}/\mathbb{F})$的\textbf{子群}和$\mathbb{K}/\mathbb{F}$的\textbf{中间域}一一对应起来了。

\end{theorem}

\textbf{证明}:

1。

据\autoref{the_GExt_7} ,取$\mathbb{M}=\mathbb{E}$即得$\mathbb{K}/\mathbb{M}$是有限Galois扩张。又由\autoref{the_GExt_4} ,得$\opn{Gal}(\mathbb{K}/\mathbb{M})$的不变子域就是$\mathbb{M}$。

2。

显然,$H$保$\mathbb{F}$不变,所以$H$的不变子域包含$\mathbb{F}$,从而$\opn{Fix}_\mathbb{K}(H)$是$\mathbb{K}/\mathbb{F}$的中间域。

由\autoref{cor_PrmtEl_2}~\upref{PrmtEl},存在$\alpha\in\mathbb{K}$使得$\mathbb{K}=\mathbb{F}(\alpha)$。又由\autoref{the_SpltFd_2}~\upref{SpltFd}和\autoref{the_SpltFd_1}~\upref{SpltFd}知,$\opn{Gal}(\mathbb{K}/\mathbb{F})$本身就是有限群,因此适用Artin\autoref{the_GExt_3} ,从而得证。

\textbf{证毕}。


















