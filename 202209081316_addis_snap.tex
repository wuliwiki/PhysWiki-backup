% Linux 包管理笔记(apt, dpkg, snap)

\begin{issues}
\issueDraft
\end{issues}

\subsection{Apt}
\begin{itemize}
\item \verb|apt-show-versions 包1 包2| 可以查看已安装包的版本
\item \verb|apt show 包| 显示包的详细信息, 包括依赖
\item \verb|apt-cache rdepends 包| 显示它目前被谁依赖
\item \verb|apt remove 包| 不会清除它依赖的包, 但是会清除依赖它的包!
\item \verb|apt autoremove| 清除没有被依赖的, 被自动安装的包, 比如 \ver|apt remove| 了一个包以后, 就用这个清干净
\item 当 apt 出现 conflict 的时候, \verb|aptitude| 通常能提供更详细的信息
\item apt and aptitude and synaptic all use the same \verb|dpkg| backend, so they will not conflict.
\item aptitude will provide multiple solutions to conflict, just type yes or no to choose one of them
\item 列出所有手动安装的包 \verb`comm -23 <(apt-mark showmanual | sort -u) <(gzip -dc /var/log/installer/initial-status.gz | sed -n 's/^Package: //p' | sort -u)`
\item 或者 \verb`comm -23 <(aptitude search '~i !~M' -F '%p' | sed "s/ *$//" | sort -u) <(gzip -dc /var/log/installer/initial-status.gz | sed -n 's/^Package: //p' | sort -u)` 参考\href{https://askubuntu.com/questions/2389/how-to-list-manually-installed-packages}{这里}.
\item \verb|apt --installed list| 列出所有安装的包
\end{itemize}

\subsection{dpkg}
\begin{itemize}
\item \verb`dpkg -l | grep -i 关键字` 可以查找已安装的某个包
\item \verb|sudo dpkg -i package_file.deb| to install a deb file
\item \verb|sudo apt remove package_name| to remove
\end{itemize}

\subsection{Snap}
\begin{itemize}
\item 参考\href{https://www.howtogeek.com/660193/how-to-work-with-snap-packages-on-linux/}{这篇介绍}: snap 基本就是一些微小的 container, 使用虚拟硬盘以解决不同的包使用同一个 dependency 的不同版本.
\item \verb|snap find 关键词| 搜索包
\item 或者到 snap 官网, 查找软件, 然后按 install 就会得到安装命令, 例如 \verb|snap install code --classic| 安装 vscode
\item 如果不加 \verb|--classic|, 会出现如下错误 \verb|error: This revision of snap "code" was published using classic confinement and thus may perform arbitrary system changes outside of the security sandbox that snaps are usually confined to, which may put your system at risk. If you understand and want to proceed repeat the command including --classic.|
\item \verb|snap version| 查看版本
\item \verb|snap remove| 删除包
\item \verb|snap run 软件名| 运行某个软件, 也可以直接 \verb|软件名| 运行(在目录 \verb|/usr/snap/bin| 下)
\end{itemize}
