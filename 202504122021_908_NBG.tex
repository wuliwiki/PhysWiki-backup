% 冯·诺伊曼-博内斯-哥德尔集合论(综述)
% license CCBYSA3
% type Wiki

本文根据 CC-BY-SA 协议转载翻译自维基百科\href{https://en.wikipedia.org/wiki/Von_Neumann\%E2\%80\%93Bernays\%E2\%80\%93G\%C3\%B6del_set_theory}{相关文章}。

在数学基础中,冯·诺依曼–伯奈斯–哥德尔集合论(NBG)是一种公理化集合论,是泽梅洛–弗兰克尔–选择公理集合论(ZFC)的保守扩展。NBG 引入了“类”的概念,类是由公式定义的集合,其量词仅对集合进行量化。NBG 可以定义比集合更大的类,例如所有集合的类和所有序数的类。摩尔斯–凯利集合论(MK)允许通过量词对类进行量化的公式来定义类。NBG 是有限公理化的,而 ZFC 和 MK 则不是。

NBG 的一个关键定理是类存在定理,它声明,对于每个量词仅对集合进行量化的公式,都存在一个类,该类包含满足该公式的集合。这个类是通过用类逐步构造公式来构建的。由于所有集合论公式都是由两种原子公式(成员关系和相等性)和有限多的逻辑符号构成,因此只需要有限多的公理来构建满足这些公式的类。这就是为什么 NBG 是有限公理化的原因。类还用于其他构造、处理集合论悖论,并用于表述全局选择公理,该公理比 ZFC 的选择公理要强。

约翰·冯·诺依曼在 1925 年将类引入集合论。他的理论的原始概念是函数和参数。利用这些概念,他定义了类和集合。\(^\text{[1]}\)保罗·伯奈斯通过将类和集合作为原始概念重新表述了冯·诺依曼的理论。\(^\text{[2]}\)库尔特·哥德尔简化了伯奈斯的理论,用于他对选择公理和广义连续统假设相对一致性的证明。\(^\text{[3]}\)
\subsection{集合论中的类} 
\subsubsection{类的用途} 
在 NBG 中,类有几个用途:
\begin{itemize}
\item 它们产生了集合论的有限公理化。\(^\text{[4]}\)  
\item 它们用于表述“非常强的选择公理”\(^\text{[5]}\)——即全局选择公理:存在一个定义在所有非空集合类上的全局选择函数\( G \),使得对于每个非空集合\( x \),都有 \( G(x) \in x \)。  
   这比 ZFC 的选择公理要强:对于每个非空集合的集合\( s \),存在一个选择函数 \( f \),定义在\( s \)上,使得对于所有\( x \in s \),都有\( f(x) \in x \)。  
\item 通过认识到某些类不能是集合,集合论悖论得以解决。例如,假设所有序数的类 \( \text{Ord} \)是一个集合。那么\( \text{Ord} \)是一个按\( \in \)良序的传递集合。所以,根据定义,\( \text{Ord} \)是一个序数。因此,\( \text{Ord} \in \text{Ord} \),这与\( \in \)是\( \text{Ord} \)的良序性相矛盾。因此,\( \text{Ord} \)不是一个集合。不能是集合的类称为适当类;\( \text{Ord} \)是一个适当类。\(^\text{[6]}\)
\item 适当类在构造中很有用。在他证明全局选择公理和广义连续统假设的相对一致性时,哥德尔使用适当类来构建构造宇宙。他在所有序数的类上构造了一个函数,对于每个序数,通过对先前构造的集合应用集合构建操作来构造一个构造集。构造宇宙就是这个函数的像。\(^\text{[7]}\)
\end{itemize}
\subsubsection{公理模式与类存在定理}  
一旦类被添加到 ZFC 的语言中,就可以轻松地将 ZFC 转换为一个包含类的集合论。首先,添加类理解的公理模式。这个公理模式声明:对于每个仅对集合进行量化的公式\( \phi(x_1, \ldots, x_n) \),存在一个类\( A \),由满足该公式的\( n \)-元组组成——即:
\[
\forall x_1 \cdots \forall x_n \left[ (x_1, \ldots, x_n) \in A \iff \phi(x_1, \ldots, x_n) \right].~
\]
然后,替换公理模式被替换为一个使用类的单一公理。最后,ZFC 的外延公理被修改以处理类:如果两个类有相同的元素,则它们是相同的。ZFC 的其他公理没有被修改。\(^\text{[8]}\)

这个理论不是有限公理化的。ZFC 的替换公理模式已被一个单一公理所替代,但类理解的公理模式被引入。

为了产生一个有限公理化的理论,首先将类理解的公理模式替换为有限多个类存在公理。然后,这些公理被用来证明类存在定理,该定理暗示公理模式的每个实例。\(^\text{[8]}\)这个定理的证明只需要七个类存在公理,这些公理用于将公式的构造转换为满足该公式的类的构造。
\subsection{NBG 的公理化}  
\subsubsection{类和集合}  
NBG 有两种类型的对象:类和集合。直观上,每个集合也是一个类。公理化这一点有两种方式。[需要非主要来源] 伯奈斯使用了多种排序逻辑,包含两种排序:类和集合。\(^\text{[2]}\)哥德尔通过引入原始谓词避免了排序问题:\( \mathfrak{Cls}(A) \)表示 “A 是一个类”,  
\( \mathfrak{M}(A) \)表示 “A 是一个集合”(在德语中,“集合”是“ Menge”)。他还引入了公理,声明每个集合都是一个类,并且如果类\( A \)是某个类的成员,则\( A \)是一个集合。\(^\text{[9]}\)使用谓词是消除排序的标准方式。埃利奥特·门德尔森修改了哥德尔的方法,使得一切都是类,并将集合谓词\( M(A) \)定义为\( \exists C (A \in C) \)。\(^\text{[10]}\)这种修改消除了哥德尔的类谓词和他的两个公理。

伯奈斯的两排序方法可能一开始看起来更自然,但它创造了一个更复杂的理论。\(^\text{[b]}\)在伯奈斯的理论中,每个集合有两种表示方式:一种作为集合,另一种作为类。此外,有两个成员关系:第一个,表示为“∈”,用于两个集合之间;第二个,表示为“η”,用于集合与类之间。\(^\text{[2]}\) 这种冗余是多种排序逻辑所需要的,因为不同排序的变量作用于论域的不同子域。

这两种方法之间的差异不会影响可证明的内容,但会影响如何编写命题。在哥德尔的方法中,\( A \in C \)(其中\( A \)和\( C \)是类)是一个有效的命题。在伯奈斯的方法中,这个命题没有意义。然而,如果\( A \)是一个集合,就有一个等效的命题:定义“集合\( a \)表示类\( A \)”如果它们有相同的成员集合——即,\(\forall x (x \in a \iff x \eta A)\)命题\( a \eta C \)(其中集合\( a \)表示类\( A \))等价于哥德尔的\( A \in C \)。\(^\text{[2]}\)

本文采用的方法是哥德尔加上门德尔森的修改。这意味着 NBG 是一个基于一阶谓词逻辑的公理化系统,具有相等性,其唯一的原始概念是类和成员关系。
\subsubsection{外延性公理和配对公理的定义和公理} 
集合是至少属于一个类的类:\( A \) 是一个集合当且仅当\( \exists C (A \in C) \)。不是集合的类称为适当类:\( A \)是一个适当类当且仅当\( \forall C (A \notin C) \)。\(^\text{[12]}\)因此,每个类要么是一个集合,要么是一个适当类,且没有类既是集合又是适当类。

哥德尔引入了一个约定,即大写字母变量作用于类,而小写字母变量作用于集合。\(^\text{[9]}\)哥德尔还使用以大写字母开头的名称来表示特定的类,包括定义在所有集合类上的函数和关系。本文采用了哥德尔的约定。这使得我们可以写作:
\[
\exists x \, \phi(x)~
\]
代替  
\[
\exists x \left( \exists C (x \in C) \land \phi(x) \right)~
\]
\[
\forall x \, \phi(x)~
\]
代替  
\[
\forall x \left( \exists C (x \in C) \implies \phi(x) \right)~
\]
以下公理和定义是证明类存在定理所需的。

\textbf{外延性公理} 
如果两个类有相同的元素,则它们是相同的。

\[
\forall A \, \forall B \, \left[ \forall x \, (x \in A \iff x \in B) \implies A = B \right]^\text{[13]}~
\] 
该公理将 ZFC 的外延性公理推广到类。

\textbf{配对公理} 
如果\( x \)和\( y \)是集合,那么存在一个集合\( p \),它的唯一成员是\( x \)和\( y \)。

\[
\forall x \, \forall y \, \exists p \, \forall z \, \left[ z \in p \iff (z = x \lor z = y) \right]^\text{[14]}~
\]
与 ZFC 中一样,外延性公理暗示了集合\( p \)的唯一性,这使我们能够引入符号\( \{x, y\} \)。

\textbf{有序对通过以下方式定义:}
\[
(x, y) = \{\{x\}, \{x, y\}\}~
\]
元组通过有序对递归定义:
\[
(x_1) = x_1,~
\]
对于 \( n > 1 \):
\[
(x_1, \ldots, x_{n-1}, x_n) = ((x_1, \ldots, x_{n-1}), x_n).^\text{[c]}~
\]  
\subsubsection{类存在公理和正则性公理}  
类存在公理将用于证明类存在定理:对于每个仅对集合进行量化的包含\( n \)个自由集合变量的公式,都存在一个满足该公式的\( n \)-元组类。以下示例从两个类(函数)开始,并构建一个复合函数。这个例子展示了证明类存在定理所需的技术,这些技术最终引出了需要的类存在公理。

\textbf{示例 1}:如果类\( F \)和\( G \)是函数,那么复合函数\( G \circ F \)由以下公式定义:
\[
\exists t \left[ (x,t) \in F \land (t,y) \in G \right].~
\]
由于该公式有两个自由集合变量\( x \)和\( y \),类存在定理构造了有序对的类:
\[
G \circ F = \{ (x,y) : \exists t \left[ (x,t) \in F \land (t,y) \in G \right] \}.~
\]
由于该公式是通过使用合取\( \land \)和存在量化\( \exists \)从更简单的公式构造的,因此需要类操作,这些操作将表示简单公式的类结合起来,并生成表示含有\( \land \)和\( \exists \)的公式的类。为了生成表示含有\( \land \)的公式的类,可以使用交集,因为\( x \in A \cap B \iff x \in A \land x \in B \)。

为了生成表示含有\( \exists \)的公式的类,可以使用域,因为\( x \in \text{Dom}(A) \iff \exists t \left[ (x,t) \in A \right] \)。

在进行交集之前,必须给\( F \)和\( G \)中的元组添加一个额外的组件,以使它们有相同的变量。将变量\( y \)添加到\( F \)的元组中,将变量\( x \)添加到\( G \)的元组中:
\[
F' = \{ (x,t,y) : (x,t) \in F \} \quad \text{和} \quad G' = \{ (t,y,x) : (t,y) \in G \}.~
\]
在\( F' \)的定义中,变量\( y \)不受\( (x,t) \in F \)语句的限制,因此\( y \)在所有集合的类\( V \)中取值。类似地,在\( G' \)的定义中,变量\( x \)在\( V \)中取值。因此,需要一个公理来将一个额外的组件(其值范围在\( V \)中)添加到给定类的元组中。

接下来,变量按照相同的顺序排列,以准备进行交集操作:
\[
F'' = \{ (x, y, t) : (x, t) \in F \}~
\]
和
\[
G'' = \{ (x, y, t) : (t, y) \in G \}.~
\]
从\( F' \)到\( F'' \)和从\( G' \)到\( G'' \)需要两次不同的排列,因此需要支持元组组件排列的公理。

\( F'' \) 和 \( G'' \) 的交集处理了合取 \( \land \):
\[
F'' \cap G'' = \{ (x, y, t) : (x, t) \in F \land (t, y) \in G \}.~
\]
由于\( (x, y, t) \)被定义为\( ((x, y), t) \),对\( F'' \cap G'' \)取域处理了\( \exists t \)并生成了复合函数:
\[
G \circ F = \text{Dom}(F'' \cap G'') = \{ (x, y) : \exists t ((x, t) \in F \land (t, y) \in G) \}.~
\]
因此,需要交集和域的公理。

类存在公理分为两组:一组处理语言原语,另一组处理元组。第一组有四个公理,第二组有三个公理。\(^\text{[d]}\)

\textbf{处理语言原语的公理:}

\textbf{成员关系} 
存在一个类\( E \),其中包含所有有序对,其第一个组件是第二个组件的成员。
\[
\exists E \, \forall x \, \forall y \, \left[ (x, y) \in E \iff x \in y \right]^\text{[18]}~
\]

\textbf{交集(合取)} 
对于任意两个类\( A \)和\( B \),存在一个类\( C \),它恰好包含属于\( A \)和\( B \)的所有集合。
\[
\forall A \, \forall B \, \exists C \, \forall x \, \left[ x \in C \iff (x \in A \land x \in B) \right]^\text{[19]}~
\]
\textbf{补集(否定)}  
对于任何类\( A \),存在一个类\( B \),它恰好包含不属于\( A \)的所有集合。
\[
\forall A \, \exists B \, \forall x \, \left[ x \in B \iff \neg (x \in A) \right]^\text{[20]}~
\]
\textbf{域(存在量化)} 
对于任何类\( A \),存在一个类\( B \),它恰好包含\( A \)的有序对的第一个组件。
\[
\forall A \, \exists B \, \forall x \, \left[ x \in B \iff \exists y \, ((x, y) \in A) \right]^\text{[21]}~
\]
根据外延性公理,交集公理中的类 \( C \) 和补集与域公理中的类 \( B \) 是唯一的。它们分别表示为:\( A \cap B \),\( \complement A \),和 \( \text{Dom}(A) \)\(^\text{[e]}\)。

前三个公理暗示了空类和所有集合的类的存在:成员关系公理暗示存在一个类 \( E \)。  交集和补集公理暗示了存在\( E \cap \complement E \),即空集。根据外延性公理,这个类是唯一的,它表示为\( \emptyset \)。\( \emptyset \)的补集是所有集合的类\( V \),根据外延性公理,\( V \) 也是唯一的。集合谓词\( M(A) \),最初定义为\( \exists C (A \in C) \),现在重新定义为\( A \in V \),以避免对类进行量化。

\textbf{处理元组的公理:}

\textbf{由\( V \) 生成的乘积} 
对于任何类\( A \),存在一个类\( B \),其元素是有序对,其中第一个组件属于\( A \)。
\[
\forall A \, \exists B \, \forall u \, [u \in B \iff \exists x \, \exists y \, (u = (x, y) \land x \in A)]^\text{[23]}~
\]
\textbf{圆形排列} 
对于任何类\( A \),存在一个类\( B \),其 3 元组是通过对\( A \)的 3 元组应用圆形排列\( (y, z, x) \mapsto (x, y, z) \)获得的。
\[
\forall A \, \exists B \, \forall x \, \forall y \, \forall z \, [(x, y, z) \in B \iff (y, z, x) \in A]^\text{[24]}~
\]
\textbf{交换}  
对于任何类\( A \),存在一个类\( B \),其 3 元组是通过交换\( A \)的 3 元组中的最后两个组件得到的。
\[
\forall A \, \exists B \, \forall x \, \forall y \, \forall z \, [(x, y, z) \in B \iff (x, z, y) \in A]^\text{[25]}~
\]
根据外延性公理,由\( V \)生成的乘积公理暗示了唯一类的存在,记作\( A \times V \)。  
这个公理用于定义所有\( n \)-元组的类\( V^n \):\( V^1 = V \)和\( V^{n+1} = V^n \times V \)。如果 \( A \) 是一个类,外延性公理暗示 \( A \cap V^n \) 是包含\( A \)的所有\( n \)-元组的唯一类。例如,成员关系公理产生一个类 \( E \),它可能包含非有序对的元素,而交集\( E \cap V^2 \)只包含\( E \)的有序对。

\textbf{圆形排列和交换公理}并不暗示唯一类的存在,因为它们仅指定类\( B \)的 3 元组。通过指定这些 3 元组,这些公理还指定了\( n \)-元组,适用于\( n \geq 4 \),因为:
\[
(x_1, \ldots, x_{n-2}, x_{n-1}, x_n) = ((x_1, \ldots, x_{n-2}), x_{n-1}, x_n)~
\]
处理元组的公理和域公理暗示了以下引理,这在类存在定理的证明中得到了应用。

\textbf{元组引理}
\begin{enumerate}
\item \(\forall A \, \exists B_1 \, \forall x \, \forall y \, \forall z \, [(z, x, y) \in B_1 \iff (x, y) \in A]\)
\item \(\forall A \, \exists B_2 \, \forall x \, \forall y \, \forall z \, [(x, z, y) \in B_2 \iff (x, y) \in A]\)
\item \(\forall A \, \exists B_3 \, \forall x \, \forall y \, \forall z \, [(x, y, z) \in B_3 \iff (x, y) \in A]\)
\item \(\forall A \, \exists B_4 \, \forall x \, \forall y \, \forall z \, [(y, x) \in B_4 \iff (x, y) \in A]\)
\end{enumerate}

\textbf{证明}
\begin{itemize}
\item 类\( B_3 \): 对\( A \)应用由\( V \)生成的乘积,得到类\( B_3 \)。
\item 类\( B_2 \): 对\( B_3 \)应用交换,得到类\( B_2 \)。
\item 类\( B_1 \): 对\( B_3 \)应用圆形排列,得到类\( B_1 \)。
\item 类\( B_4 \): 对\( B_2 \)应用圆形排列,然后应用域操作,得到类\( B_4 \)。
\end{itemize}
还需要一个公理来证明类存在定理:正则性公理:由于空类的存在已经被证明,通常给出的该公理的表述如下:\(^\text{[f]}\)

\textbf{正则性公理:}每个非空集合至少有一个与其没有共同元素的元素。  
\[
\forall a \, [a \neq \emptyset \implies \exists u (u \in a \land u \cap a = \emptyset)]~
\]
这个公理意味着集合不能属于它自身:假设\( x \in x \),且令\( a = \{x\} \)。那么\( x \cap a \neq \emptyset \),因为\( x \in x \cap a \)。这与正则性公理矛盾,因为\( x \) 是 \( a \) 中唯一的元素。因此,\( x \notin x \)。正则性公理还禁止集合的无限递降成员序列:  
\[
\cdots \in x_{n+1} \in x_{n} \in \cdots \in x_1 \in x_0~
\]
哥德尔在他的 1940 年专著中陈述了类的正则性,而不是集合的正则性,该专著基于 1938 年的讲座。\(^\text{[26]}\)在 1939 年,他证明了集合的正则性意味着类的正则性。\(^\text{[27]}\)
\subsubsection{类存在定理}
\textbf{类存在定理}——设\( \phi(x_1, \dots, x_n, Y_1, \dots, Y_m) \)是一个仅对集合进行量化并且没有其他自由变量的公式(除了\( x_1, \dots, x_n, Y_1, \dots, Y_m \) 外,可能并不包括这些所有变量)。那么对于所有的\( Y_1, \dots, Y_m \),存在一个唯一的类\( A \)由\( n \)-元组组成,使得:
\[
\forall x_1 \cdots \, \forall x_n \, [(x_1, \dots, x_n) \in A \iff \phi(x_1, \dots, x_n, Y_1, \dots, Y_m)]~
\]
类\( A \)被表示为:
\[
\{(x_1, \dots, x_n) : \phi(x_1, \dots, x_n, Y_1, \dots, Y_m)\}^\text{[g]}~
\]  
定理的证明将分两步进行:
\begin{enumerate}
\item 转换规则用于将给定的公式\( \phi \)转换为等价的公式,从而简化证明的归纳部分。例如,转换后的公式中只有逻辑符号\( \neg \)、\( \land \)和\( \exists \),因此归纳仅处理这三种情况的逻辑符号。
\item 类存在定理通过归纳法证明对转换后的公式。根据转换后的公式的结构,使用类存在公理来生成满足该公式的唯一类\( n \)-元组。
\end{enumerate}
\textbf{转换规则:}在以下的规则 1 和规则 2 中,\( \Delta \)和\( \Gamma \) 表示集合或类变量。这两个规则消除了所有在\( \in \)前和所有等式中的类变量的出现。每次应用规则 1 或规则 2 到子公式时,选择\( i \)使得\( z_i \)与当前公式中的其他变量不同。三个规则会重复应用,直到没有子公式可以再应用它们为止。这样会得到一个仅由\( \neg \)、\( \land \)、\( \exists \)、\( \in \)、集合变量和类变量\( Y_k \)组成的公式,其中\( Y_k \)在\( \in \)前不会出现。
\begin{enumerate}
\item \(\displaystyle Y_{k}\in \Gamma\)被转换为\(\exists z_{i} (z_{i} = Y_{k} \land z_{i} \in \Gamma)\). 
\item \textbf{外延性}被用来将\(\Delta = \Gamma\)转换为\(\forall z_{i} (z_{i} \in \Delta \iff z_{i} \in \Gamma)\).
\item \textbf{逻辑恒等式}被用来将包含 \( \lor \)、\( \implies \)、\( \iff \) 和 \( \forall \) 的子公式转换为仅使用 \( \neg \)、\( \land \) 和 \( \exists \) 的子公式。\\
\textbf{变换规则:有界变量。} 考虑示例1中的复合函数公式,并将其自由集合变量替换为\(x_1\)和\(x_2\):\(\exists t \left[ (x_1, t) \in F \land (t, x_2) \in G \right]\). 归纳证明将移除\( \exists t \),产生公式\((x_1, t) \in F \land (t, x_2) \in G\).然而,由于类存在定理是针对下标变量陈述的,因此该公式并不符合归纳假设所期望的形式。这个问题通过将变量\( t \)替换为\( x_3 \)来解决。对于嵌套量词中的有界变量,通过每个连续的量词增加下标来处理。这导致了规则4,在应用其他规则后必须应用该规则,因为规则1和规则2会产生量化变量。
\item 如果公式中没有自由集合变量,除了\(x_1, \dots, x_n\) 之外,则嵌套在 \(q\) 个量词中的有界变量将被替换为\(x_{n+q}\)。这些变量具有量词嵌套深度 \(q\)。
\end{enumerate}
\textbf{示例 2:}规则 4 应用于公式\( \phi(x_1) \),该公式定义了由以下形式的所有集合组成的类:\(\{\emptyset, \{\emptyset, \dots\}, \dots\}\).即,包含至少\( \emptyset \)和一个包含\( \emptyset \)的集合的集合——例如,\(\{\emptyset, \{\emptyset, a, b, c\}, d, e\}\)
其中 \( a, b, c, d, e \) 是集合。  
\[
\phi(x_1) = \exists u \left[ u \in x_1 \land \neg \exists v (v \in u) \right] \land \exists w \left(w \in x_1 \land \exists y \left[ (y \in w \land \neg \exists z (z \in y)) \right] \right)~
\]  
\[
\phi_r(x_1) = \exists x_2 \left[ x_2 \in x_1 \land \neg \exists x_3 (x_3 \in x_2) \right] \land \exists x_2 \left(x_2 \in x_1 \land \exists x_3 \left[ (x_3 \in x_2 \land \neg \exists x_4 (x_4 \in x_3)) \right] \right)~
\]
由于\( x_1 \)是唯一的自由变量,\( n = 1 \)。量化变量\( x_3 \)在公式\( x_3 \in x_2 \) 中出现两次,且嵌套深度为 2。它的下标是 3,因为 \( n + q = 1 + 2 = 3 \)。如果两个量化范围在相同的嵌套深度上,它们要么是相同的,要么是互不相交的。两个\( x_3 \)的出现位于不相交的量化范围内,因此它们互不干扰。

\textbf{类存在定理的证明。} 证明从应用变换规则到给定公式开始,以产生一个变换后的公式。由于这个公式与给定公式等价,因此通过对变换后的公式证明类存在定理来完成证明。

\textbf{展开引理}——设 \( 1 \leq i < j \leq n \),并且设 \( P \) 是一个类,包含所有满足 \( R(x_i, x_j) \) 的有序对 \( (x_i, x_j) \),即\(P \supseteq \{(x_i, x_j) : R(x_i, x_j)\}\).那么,\( P \)可以扩展成唯一的类\( Q \),该类包含所有满足\( R(x_i, x_j) \)的\( n \)-元组。即,\(Q = \{(x_1, \ldots, x_n) : R(x_i, x_j)\}\).

\textbf{证明:}
\begin{enumerate}
\item 如果 \( i = 1 \),令 \( P_1 = P \)。\\
否则,如果 \( i > 1 \),则在 \( x_i \) 前面加上组件:应用元组引理的命题1到 \( P \) 上,令 \( z = (x_1, \dots, x_{i-1}) \)。这产生一个类 \( P_1 \),它包含所有满足 \( R(x_i, x_j) \) 的 \( (i+1) \)-元组 \( ((x_1, \dots, x_{i-1}), x_i, x_j) = (x_1, \dots, x_{i-1}, x_i, x_j) \).\\
\item 如果 \( j = i + 1 \),则令 \( P_2 = P_1 \).\\
否则,如果 \( j > i + 1 \),则在 \( x_i \) 和 \( x_j \) 之间加上组件:通过应用元组引理的命题2,逐一添加组件 \( x_{i+1}, \dots, x_{j-1} \)。这产生一个类 \( P_2 \),它包含所有满足 \( R(x_i, x_j) \) 的 \( j \)-元组 \( (((\cdots ((x_1, \dots, x_i), x_{i+1}), \cdots ), x_{j-1}), x_j) = (x_1, \dots, x_j) \).\\
\item 如果 \( j = n \),则令 \( P_3 = P_2 \)。\\
否则,如果 \( j < n \),则在 \( x_j \) 后面加上组件:通过应用元组引理的命题3,逐一添加组件 \( x_{j+1}, \dots, x_n \)这产生一个类 \( P_3 \),它包含所有满足 \( R(x_i, x_j) \) 的 \( n \)-元组 \( (((\cdots ((x_1, \dots, x_j), x_{j+1}), \cdots ), x_n) = (x_1, \dots, x_n) \)。
\item 令 \( Q = P_3 \cap V^n \).根据外延性,\( Q \) 是唯一的满足 \( R(x_i, x_j) \) 的 \( n \)-元组类。
\end{enumerate}
\textbf{类存在定理(变换公式版)}——设\(\phi(x_1, \dots, x_n, Y_1, \dots, Y_m)\)是一个公式,满足以下条件:
\begin{enumerate}
\item 公式中除了 \(x_1, \dots, x_n, Y_1, \dots, Y_m\) 之外没有自由变量;
\item 公式中只包含符号 \(\in\), \(\neg\), \(\land\), \(\exists\),集合变量,以及类变量 \(Y_k\)(其中 \(Y_k\) 不出现在 \(\in\) 之前);
\item 公式中仅对集合变量 \(x_{n+q}\) 进行量化,其中 \(q\) 是该变量的量化嵌套深度。
\end{enumerate}
那么,对于所有的\(Y_1, \dots, Y_m\),存在一个唯一的类 \(A\),它包含所有满足以下条件的\(n\)-元组:
\[
\forall x_1 \cdots \forall x_n [(x_1, \dots, x_n) \in A \iff \phi(x_1, \dots, x_n, Y_1, \dots, Y_m)]~
\]
这表示,类\(A\)是由所有满足公式\(\phi(x_1, \dots, x_n, Y_1, \dots, Y_m)\)的\(n\)-元组组成。

证明:基础步骤:

\(\phi\) 具有 0 个逻辑符号。定理的假设意味着 \(\phi\) 是一个原子公式,形式为 \(x_i \in x_j\) 或 \(x_i \in Y_k\)。

情况 1:如果 \(\phi\) 是 \(x_i \in x_j\),我们构建类 \(E_{i,j,n} = \{(x_1, \dots, x_n) : x_i \in x_j\}\),这是满足 \(x_i \in x_j\) 的唯一 \(n\)-元组类。

情况 a:\(\phi\) 是 \(x_i \in x_j\),其中 \(i < j\)。成员公理产生一个类 \(P\),包含所有满足 \(x_i \in x_j\) 的有序对 \((x_i, x_j)\)。应用扩展引理于 \(P\) 得到 \(E_{i,j,n} = \{(x_1, \dots, x_n) : x_i \in x_j\}\)。

情况 b:\(\phi\) 是 \(x_i \in x_j\),其中 \(i > j\)。成员公理产生一个类 \(P\),包含所有满足 \(x_i \in x_j\) 的有序对 \((x_i, x_j)\)。应用元组引理的第 4 条于 \(P\) 得到 \(P'\),包含所有满足 \(x_i \in x_j\) 的有序对 \((x_j, x_i)\)。应用扩展引理于 \(P'\) 得到 \(E_{i,j,n} = \{(x_1, \dots, x_n) : x_i \in x_j\}\)。

情况 c:\(\phi\) 是 \(x_i \in x_j\),其中 \(i = j\)。由于根据正则性公理此公式为假,因此没有 \(n\)-元组满足它,所以 \(E_{i,j,n} = \emptyset\)。

情况 2:如果 \(\phi\) 是 \(x_i \in Y_k\),我们构建类 \(E_{i,Y_k,n} = \{(x_1, \dots, x_n) : x_i \in Y_k\}\),这是满足 \(x_i \in Y_k\) 的唯一 \(n\)-元组类。

情况 a:\(\phi\) 是\(x_i \in Y_k\),其中\(i < n\)。应用乘积公理于\(Y_k\),生成类\(P = Y_k \times V = \{(x_i, x_{i+1}) : x_i \in Y_k\}\)。应用扩展引理于\(P\),得到\(E_{i,Y_k,n} = \{(x_1, \dots, x_n) : x_i \in Y_k\}\)。

情况 b:\(\phi\) 是 \(x_i \in Y_k\),其中 \(i = n > 1\)。应用乘积公理于 \(Y_k\),生成类 \(P = Y_k \times V = \{(x_i, x_{i-1}) : x_i \in Y_k\}\)。应用元组引理的第 4 条于 \(P\),得到 \(P' = V \times Y_k = \{(x_{i-1}, x_i) : x_i \in Y_k\}\)。应用扩展引理于 \(P'\),得到 \(E_{i,Y_k,n} = \{(x_1, \dots, x_n) : x_i \in Y_k\}\)。

情况 c:\(\phi\) 是 \(x_i \in Y_k\),其中 \(i = n = 1\)。则 \(E_{i,Y_k,n} = Y_k\)。

归纳步骤:\(\phi\)有\(k\)个逻辑符号,其中\(k > 0\)。假设归纳假设成立,即定理对所有具有少于\(k\)个逻辑符号的公式\(\psi\)都成立。现在,我们证明定理对于具有\(k\)个逻辑符号的公式\(\phi\)也成立。在这个证明中,类变量列表\(Y_1, \dots, Y_m\) 简写为 \(\vec{Y}\),因此公式—例如\(\phi(x_1, \dots, x_n, Y_1, \dots, Y_m)\)—可以写作\(\phi(x_1, \dots, x_n, Y_{\rightarrow})\)。