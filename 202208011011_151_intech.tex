% 不定积分的常用技巧
% 不定积分

点赞,能有效调动劳动者积极性,所以,请点个赞吧!
1.分项积分法
\int \left[f(x)+g(x)-h(x)\right]\,\mathbb{d}x=\int f(x)\,\mathbb{d}x+\int g(x)\,\mathbb{d}x-\int h(x)\,\mathbb{d}x 
多项式的积分等于各个单项式的积分之和
分母为多项式,可将其化为简单分式再积分
例  \int \frac{\,\mathbb{d}x}{x^2-a^2} , x^2-a^2=(x+a)(x-a) 
设 \frac{1}{x^2-a^2}=\frac{A}{x-a}+\frac{B}{x+a}\Rightarrow A(x+a)+B(x-a)=1 
由掩盖法, A=\frac{1}{2a},B=-\frac{1}{2a} ,所以 \frac{1}{x^2-a^2}=\frac{1}{2a}\left(\frac{1}{x-a}-\frac{1}{x+a} \right) 
\begin{eqnarray}\label{eq} LHS&=&\frac{1}{2a}\left(\int\frac{\mathbb{d}x}{x-a}-\int\frac{\mathbb{d}x}{x+a}\right) \\&=&\frac{1}{2a}\left(\ln|x-a|-\ln|x+a|\right)+C \\&=&\frac{1}{2a}\ln\left|\frac{x-a}{x+a}\right|+C   \end{eqnarray} 
更一般地,让我们求 \int\frac{mx+n}{x^2+px+q}\,\mathbb{d}x 
对分母配方: x^2+px+q=\left(x+\frac{p}{2}\right)^2+q-\frac{p^2}{4} 
令 t=x+\frac{p}{2} ,于是 x=t-\frac{p}{2},\mathbb{d}x=\mathbb{d}t ,令 q-\frac{p^2}{4}=\pm a^2 
令 A=m,B=n-\frac{1}{2}mp ,则 mx+n=At+B 
则 LHS=\int\frac{At+B}{t^2\pm a^2}\,\mathbb{d}t =A\int\frac{t\mathbb{d}t}{t^2\pm a^2}+B\int\frac{\mathbb{d}t}{t^2\pm a^2} 
A\int\frac{t\mathbb{d}t}{t^2\pm a^2}=\frac{A}{2}\int\frac{\mathbb{d}\left(t^2\pm a^2\right)}{t^2\pm a^2}=\frac{A}{2}\ln\left|t^2\pm a^2\right|+C 
当 q>\frac{p^2}{4} 时:
B\int\frac{\mathbb{d}t}{t^2+ a^2}=\frac{B}{a}\arctan{\frac{t}{a}}+C  
\begin{eqnarray}\label{eq1} LHS&=&\frac{A}{2}\ln\left|t^2+ a^2\right|+\frac{B}{a}\arctan{\frac{t}{a}}+C \\&=&\frac{m}{2}\ln\left|x^2+px+q\right|+\frac{2n-mp}{\sqrt{4q-p^2}}\arctan{\frac{2x+p}{\sqrt{4q-p^2}}}+C \end{eqnarray} 
当 q<\frac{p^2}{4} 时:
B\int\frac{\mathbb{d}t}{t^2- a^2}=\frac{B}{2a}\ln\left|\frac{t-a}{t+a}\right|+C 
\begin{eqnarray} LHS&=&\frac{A}{2}\ln\left|t^2- a^2\right|+\frac{B}{2a}\ln\left|\frac{t-a}{t+a}\right|+C \\&=&\frac{m}{2}\ln\left|x^2+px+q\right|+\frac{2n-mp}{2\sqrt{p^2-4q}}\ln\left|\frac{x+2p-\sqrt{p^2-4q}}{x+2p+\sqrt{p^2-4q}}\right|+C \end{eqnarray} 
对于被积函数的分母为二次函数,分子小于二次的情况,普遍可采用上面的公式
对于被积函数的分母大于二次的情况,需用待定系数法,将被积函数分解为简单分式之和.待定系数法有以下两种类型:
方程两端同次幂的系数相同
“掩盖”
若被积函数的分式中分子幂次高于分母幂次,应用长除法,使其变成既约真分式,长除法的大致步骤如下:
用分子的最高次项除以分母的最高次项,得到首商,写在横线上对应位置;
分母乘首商,写在分子前两项之下,同类项对齐;
分子对应项减去刚才的乘积,得到第一余式写在下面,将分子下一项抄下来;
把第一余式当作被除式,重复上述操作,直至余式次数低于除式次数.
例  \frac{x^3-12x^2-42}{x-3}=x^2-9x-27+\frac{-123}{x-3} 
2.分部积分法
\int u\,\mathbb{d}v=uv-\int v\,\mathbb{d}u 
莱布尼兹公式
\int uv^{(n+1)}\,\mathbb{d}x=uv^{(n)}-u'v^{(n-1)}+...+(-1)^nu^{(n)}v+(-1)^{n+1}\int u^{(n+1)}v\mathbb{d}x 
当被积函数的因式之一是多项式时,运用莱布尼兹公式特别方便
例  \int \left(2x^3+3x^2+4x+5\right)e^x\mathbb{d}x 
令 u=2x^3+3x^2+4x+5,\mathbb{d}v=e^x\,\mathbb{d}x,v=e^x 
则 u'=6x^2+6x+4,u''=12x+6,u'''=12 ,v'=e^x,v''=e^x,v'''=e^x 
\begin{eqnarray} LHS&=&(2x^3+3x^2+4x+5)e^x-(6x^2+6x+4)e^x+(12x+6)e^x-12e^x+C \\&=&(2x^3-3x^2+10x-5)e^x+C \end{eqnarray} 
应用 \int P(x)e^ax\,\mathbb{d}x ,其中 P(x) 为 x 的多项式
令 v^{(n+1)}=e^{ax} ,则
v^{(n)}=\frac{e^{ax}}{a},v^{(n-1)}=\frac{e^{ax}}{a^2},v^{(n-2)}=\frac{e^{ax}}{a^3}...\\ 
设 \deg (P)=n ,则:
\int P(x)e^{ax}\,\mathbb{d}x=e^{ax}\left(\frac{P}{a}-\frac{P'}{a^2}+\frac{P''}{a^3}-...\right)
同理, \int P(x)\sin bx\,\mathbb{d}x=\sin bx\left(\frac{P'}{b^2}-\frac{P'''}{b^4}+...\right)-\cos bx\left(\frac{P}{b}-\frac{P''}{b^3}+...\right)+C 
同理, \int P(x)\cos bx\,\mathbb{d}x=\sin bx\left(\frac{P}{b}-\frac{P''}{b^3}+...\right)+\cos bx\left(\frac{P'}{b^2}-\frac{P'''}{b^4}+...\right)+C
把 1 看作被积函数的因式之一,可以帮助积分
应用 1 \int \ln x\,\mathbb{d}x=\int 1\cdot\ln x\,\mathbb{d}x
令 u=\ln x\,,dv=dx\Rightarrow du=\frac{1}{x},v=x 
则 \int \ln x\,\mathbb{d}x=x\ln x-\int x\cdot\frac{\mathbb{d}x}x=x\ln x-x+C 
应用 2 \int\arctan x\,\mathbb{d}x=\int 1\cdot\arctan x\,\mathbb{d}x 
令 u=\arctan x\,,dv=dx\Rightarrow du=\frac{\mathbb{d}x}{1+x^2},v=x 
则 \int\arctan x\,\mathbb{d}x=x\cdot \arctan x-\int\frac{x\,\mathbb{d}x}{1+x^2}=x\cdot \arctan x-\frac{1}{2}\ln\left(1+x^2\right)+C 
研究\int e^{ax}\cos{bx}\,\mathbb{d}x,\int e^{ax}\sin{bx}\,\mathbb{d}x 
令 u=\cos bx\,或 \,u=\sin bx\,,dv=e^{ax}\,\mathbb{d}x 
则 \mathbb{d}u=-b\sin bx \,\mathbb{d}x\,或\,b\cos bx\,\mathbb{d}x\,,v=\frac{e^{ax}}{a} ,于是
\left\{ \begin{align} \int e^{ax}\cos bx\,\mathbb{d}x &=\frac{1}{a}e^{ax}\cos bx+\frac{b}{a}\int e^{ax}\sin bx\,\mathbb{d}x \\\int e^{ax}\sin bx\,\mathbb{d}x&=\frac{1}{a}e^{ax}\sin bx-\frac{b}{a}\int e^{ax}\cos bx\,\mathbb{d}x \end{align} \right. 
解得: \left\{ \begin{align} \int e^{ax}\cos bx\,\mathbb{d}x&=e^{ax}\frac{a\cos bx+b\sin bx}{a^2+b^2}+C\\ \int e^{ax}\sin bx\,\mathbb{d}x&=e^{ax}\frac{a\sin bx-b\cos bx}{a^2+b^2}+C \end{align} \right. 
研究 \int x^k (\ln x)^n\,\mathbb{d}x 
首先求 \int x^k\ln x\,\mathbb{d}x 
令 u=\ln x\,,dv=x^k\,\mathbb{d}x\Rightarrow\mathbb{d}u=\frac{\mathbb{d}x}{x},v=\frac{x^{k+1}}{k+1} 
则 LHS=\frac{x^{k+1}\ln x}{k+1}-\int\frac{x^k\,\mathbb{d}x}{k+1}=\frac{x^{k+1}\ln x}{k+1}-\frac{x^{k+1}}{(k+1)^2}+C 
然后,在 \int x^k (\ln x)^n\,\mathbb{d}x 中:令 u=(\ln x)^n\,,\mathbb{d}v=x^k\,\mathbb{d}x\Rightarrow\,\mathbb{d}u=\frac{n(\ln x)^{n-1}\mathbb{d}x}{x}\,,v=\frac{x^{k+1}}{k+1} 
所以 \int x^k(\ln x)^n\,\mathbb{d}x=\frac{x^{k+1}}{k+1}(\ln x)^n-\frac{n}{k+1}\int x^k(\ln x)^{n-1}\,\mathbb{d}x 
此即原积分式的递推公式
研究 \int x^n e^{ax}\cos bx\,\mathbb{d}x , \int x^n e^{ax}\sin bx\,\mathbb{d}x
令 u=x^n\,,\mathbb{d}v=e^{ax}\cos{bx}\,\mathbb{d}x\,或\,\mathbb{d}v=e^{ax}\sin{bx}\,\mathbb{d}x 
则 \mathbb{d}u=nx^{n-1}\mathbb{d}x\,,v=e^{ax}\frac{a\cos bx+b\sin bx}{a^2+b^2}\,或\,v=e^{ax}\frac{a\sin bx-b\cos bx}{a^2+b^2} 
\int x^n e^{ax}\cos bx\,\mathbb{d}x=x^n e^{ax}\frac{{a\cos bx+b\sin bx}}{a^2+b^2}-\frac{n}{a^2+b^2}\left(a\int x^{n-1}e^{ax}\cos bx\,\mathbb{d}x+b\int x^{n-1}e^{ax}\sin bx\,\mathbb{d}x\right)+C 
\int x^n e^{ax}\sin bx\,\mathbb{d}x=x^n e^{ax}\frac{{a\sin bx-b\cos bx}}{a^2+b^2}-\frac{n}{a^2+b^2}\left(a\int x^{n-1}e^{ax}\sin bx\,\mathbb{d}x-b\int x^{n-1}e^{ax}\cos bx\,\mathbb{d}x\right)+C 
3.换元积分法
3.1 常用换元
\int f(ax+b)\,\mathbb{d}x=\frac{1}{a}\int f(ax+b)\,\mathbb{d}\left(ax+b\right) 
\int x^nf(x^{n+1})\,\mathbb{d}x=\frac{1}{n+1}\int f(x^{n+1})\,\mathbb{d}(x^{n+1}) 
\int \frac{f(\ln x)\,\mathbb{d}x}{x}=\int f(\ln x)\,\mathbb{d}(\ln x) 
\int e^xf(e^x)\,\mathbb{d}x=\int f(e^x)\,\mathbb{d}(e^x) 
\int \frac{f(\sqrt{x})\,\mathbb{d}x}{\sqrt{x}}=2\int f(\sqrt{x})\,\mathbb{d}(\sqrt{x}) 
\int \frac{1}{x^n}f\left(\frac{1}{x^n}\right)\,\mathbb{d}x=(1-n)\int f\left(\frac{1}{x^{n-1}}\right)\,\mathbb{d}\left(\frac{1}{x^{n-1}}\right) 
\int f(\sin x)\cos x\,\mathbb{d}x=\int f(\sin x)\,\mathbb{d}(\sin x) 
\int f(\cos x)\sin x\,\mathbb{d}x=-\int f(\cos x)\,\mathbb{d}(\cos x) 
\int f(\tan x)\sec^2x\,\mathbb{d}x=\int f(\tan x)\,\mathbb{d}(\tan x) 
\sqrt{ax+b} 中,令 \sqrt{ax+b}=u 
3.2 三角换元
a^2-x^2 中,令 x=a\sin\theta 
x^2+a^2 中,令 x=a\tan \theta 
x^2-a^2 中,令 x=a\sec\theta 
例 1 \int \sqrt{a^2-x^2}\,\mathbb{d}x 
令 x=a\sin\theta\,,\mathbb{d}x=a\cos\theta\,\mathbb{d}\theta\,,\sqrt{a^2-x^2}=a\cos\theta 
\begin{eqnarray} LHS&=&\int a\cos\theta\cdot a\cos\theta\,\mathbb{d}\theta\\ &=&a^2\int\cos^2\theta\,\mathbb{d}\theta\\ &=&a^2\int\frac{\cos2\theta+1}{2}\,\mathbb{d}\theta\\ &=&a^2\left(\frac{\sin2\theta}{4}+\frac{\theta}{2}\right)+C\\ &=&\frac{1}{2}a^2\left(\sin\theta\cos\theta+\theta\right)+C\\ &=&\frac{1}{2}\left(x\sqrt{a^2-x^2}+a^2\arcsin\frac{x}{a}\right)+C     \end{eqnarray} 
例 2 \int \sqrt{x^2-a^2}\,\mathbb{d}x 
令 x=a\sec\theta\,,\mathbb{d}x=a\sec\theta\tan\theta\,\mathbb{d}\theta\,,\sqrt{x^2-a^2}=a\tan\theta 
\begin{eqnarray} LHS&=&\int a\tan\theta\cdot a\sec\theta\tan\theta\,\mathbb{d}\theta\\ &=&a^2\int \tan^2\theta \sec\theta\,\mathbb{d}\theta\\ &=&a^2\int (\sec^2\theta-1)\sec\theta\,\mathbb{d}\theta\\ &=&a^2\left(\int\sec^3\theta\,\mathbb{d}\theta-\int\sec\theta\,\mathbb{d}\theta\right)\\ &=&a^2\left(\frac{1}{2}\sec\theta\tan\theta+\frac{1}2\ln\left|\tan\left(\frac{\theta}{2}+\frac{\pi}{4}\right)\right|-\ln\left|\tan\left(\frac{\theta}{2}+\frac{\pi}{4}\right)\right|\right)+C\\ &=&\frac{1}{2}\left(x\sqrt{x^2-a^2}-a^2\ln|x+\sqrt{x^2-a^2}|\right)+C \end{eqnarray} 
例 3 \int\sqrt{x^2+a^2}\,\mathbb{d}x 
令 x=a\tan \theta\,,\mathbb{d}x=a\sec^2\theta\,\mathbb{d}\theta\,,\sqrt{x^2+a^2}=a\sec\theta 
\begin{eqnarray} LHS&=&a^2\int \sec^3\theta\,\mathbb{d}\theta\\ &=&\frac{a^2}{2}\tan\theta\sec\theta+\frac{a^2}{2}\ln\left|\tan\left(\frac{\theta}{2}+\frac{\pi}{4}\right)\right|+C\\ &=&\frac{1}{2}\left(x\sqrt{x^2+a^2}+a^2\ln\left|x+\sqrt{x^2+a^2}\right|\right)+C \end{eqnarray} 
3.3 万能替换法
若被积函数由 \sin x 或 \cos x 组成,可令 t=\tan\frac{x}{2}\,,x=2\arctan t\,,\mathbb{d}x=\frac{2}{1+t^2}\,\mathbb{d}t ,则:
\sin\frac{x}{2}=\frac{t}{\sqrt{1+t^2}}\,,\cos \frac{x}{2}=\frac{1}{\sqrt{1+t^2}} \\ 
使用二倍角公式:
\sin x=2\sin\frac{x}{2}\cos\frac{x}{2}=\frac{2t}{1+t^2}\\ \cos x=2\cos^2\frac{x}{2}-1=\frac{1-t^2}{1+t^2} 
于是,被积函数就化为 t 的有理函数
3.4 欧拉换元法
对形如 \int G(x,\sqrt{ax^2+bx+c})\,\mathbb{d}x 
第一类替换(要求 a>0 ):
令 \sqrt{ax^2+bx+c}=t-\sqrt{a}x 
两边平方并消去二次项得: bx+c=t^2-2\sqrt{a}tx 
所以 x=\frac{t^2-c}{2\sqrt{a}t+b}\,,\sqrt{ax^2+bx+c}=\frac{\sqrt{a}t^2+bt+c\sqrt{a}}{2\sqrt{a}t+b}\,,\mathbb{d}x=2\frac{\sqrt{a}t^2+bt+c\sqrt{a}}{(2\sqrt{a}t+b)^2}\,\mathbb{d}t 
第二类替换(要求 c>0 ):
令 \sqrt{ax^2+bx+c}=xt+\sqrt{c} 
两边平方,消去 c 得: ax+b=t^2x+2\sqrt{c}t 
所以 :
x=\frac{2\sqrt{c}t-b}{a-t^2}\,,\sqrt{ax^2+bx+c}=\frac{2\sqrt{c}t-b}{a-t^2}\, 
t+\sqrt{c}=\frac{\sqrt{c}t^2-bt+a\sqrt{c}}{a-t^2}\,,\mathbb{d}x=2\frac{\sqrt{c}t^2-bt+a\sqrt{c}}{(a-t^2)^2}\mathbb{d}t 
第三类替换:若 ax^2+bx+c 有相异实根 \lambda 和 \mu 
则 ax^2+bx+c=a(x-\lambda)(x-\mu) 
令 \sqrt{ax^2+bx+c}=t(x-\lambda) 
两边平方得,约去 x-\lambda 得: t^2x-ax=-a\mu+\lambda t^2 
于是 x=\frac{\lambda t^2-a\mu}{t^2-a^2}\,,t^2=\frac{a(x-\mu)}{x-\lambda} 
于是: \sqrt{ax^2+bx+c}=\frac{a(\lambda-\mu)t}{t^2-a}\,,\mathbb{d}x=\frac{2a(\lambda-\mu)t}{(t^2-a)^2}\,\mathbb{d}t 
4.倍角法
诸如 \sin^m x\,,\cos^mx\,,\sin^px\cos^qx 的函数的不定积分可以用倍角法,把高次三角函数的积分化为一次倍角三角函数的积分.
在 \mathbb{C} 上,我们有棣莫弗定理: (\cos x+i\sin x)^n=\cos nx+i\sin nx ,令
\cos x+i\sin x=y\\ 
则 \cos x-i\sin x=\frac{1}{\cos x+i\sin x}=\frac{1}{y}\\ 
于是 \cos x=\frac{1}{2}\left(y+\frac{1}{y}\right)\,,\sin x=\frac{1}{2i}\left(y-\frac{1}{y}\right)\\ 
以及 \cos nx=\frac{1}{2}\left(y^n+\frac{1}{y^n}\right)\,,\sin nx=\frac{1}{2i}\left(y^n-\frac{1}{y^n}\right)\\ 
求 \int \sin^{2n}x\,\mathbb{d}x 
\begin{eqnarray} 2^{2n}i^{2n}\sin^{2n}x&=&\left(y-\frac{1}{y}\right)^{2n}\\ &=&\sum_{k=0}^{2n}(-1)^k\binom{2n}{k}y^{2n-2k}\\ &=&\sum_{k=0}^{n-1}(-1)^k\binom{2n}{k}\left(y^{2n-2k}+\frac{1}{y^{2n-2k}}\right)+\binom{2n}{n}\\ &=&2\sum_{k=0}^{n-1}\left[(-1)^k\binom{2n}{k}{\cos2(n-k)x}\right]+\binom{2n}{n} \end{eqnarray} 
故 \sin^{2n}x=\frac{1}{2^{2n}}\frac{1}{i^{2n}}\left[2\sum_{k=0}^{n-1}(-1)^k\binom{2n}{k}{\cos2(n-k)x}+\binom{2n}{n}\right] 
\begin{eqnarray} \int \sin^{2n}x\,\mathbb{d}x&=&\int (-1)^n\frac{1}{2^{2n}}\left[2\sum_{k=0}^{n-1}(-1)^k\binom{2n}{k}{\cos2(n-k)x}+\binom{2n}{n}\right]\mathbb{d}x\\ &=&(-1)^n\frac{1}{2^{2n}}\left[2\sum_{k=0}^{n-1}\left[(-1)^k\binom{2n}{k}\frac{\sin2(n-k)x}{2n-2k}\right]+\binom{2n}{n}x\right]+C \end{eqnarray} 
求 \int \cos^{2n}x\,\mathbb{d}x 
\begin{eqnarray} 2^{2n}\cos^{2n}x&=&\left(y+\frac{1}{y}\right)^{2n}\\ &=&\sum_{k=0}^{2n}\binom{2n}{k}y^{2n-2k}\\ &=&\sum_{k=0}^{n-1}\binom{2n}{k}\left(y^{2n-2k}+\frac{1}{y^{2n-2k}}\right)+\binom{2n}{n}\\ &=&2\sum_{k=0}^{n-1}\left[\binom{2n}{k}\cos2(n-k)x\right]+\binom{2n}{n} \end{eqnarray}
故 \cos^{2n}x=\frac{1}{2^{2n}}\left[2\sum_{k=0}^{n-1}\binom{2n}{k}{\cos2(n-k)x}+\binom{2n}{n}\right]
\begin{eqnarray} \int \cos^{2n}x\,\mathbb{d}x&=&\int \frac{1}{2^{2n}}\left[2\sum_{k=0}^{n-1}\left[\binom{2n}{k}{\cos2(n-k)x}\right]+\binom{2n}{n}\right]\mathbb{d}x\\ &=&\frac{1}{2^{2n}}\left[2\sum_{k=0}^{n-1}\binom{2n}{k}\left[\frac{\sin2(n-k)x}{2n-2k}\right]+\binom{2n}{n}x\right]+C \end{eqnarray} 
求 \int \sin^{2n+1}x\,\mathbb{d}x 
\begin{eqnarray} \int \sin^{2n+1}x\,\mathbb{d}x&=&\int \sin^{2n} x \sin x\,\mathbb{d}x\\ &=&-\int\left(1-\cos^2 x\right)^n\,\mathbb{d}(\cos x) \end{eqnarray} 
而 (1-\cos^2x)^n=\sum_{k=0}^{n}\left[\binom{n}{k}(-1)^k\cos^{2k}x\right] 
\begin{eqnarray} LHS&=&-\int\left(1-\cos^2 x\right)^n\,\mathbb{d}(\cos x)\\ &=&-\int\sum_{k=0}^{n}\left[\binom{n}{k}(-1)^k\cos^{2k}x\right]\,\mathbb{d}(\cos x)\\ &=&-\sum_{k=0}^{n}\left[\binom{n}{k}(-1)^k\frac{\cos^{2k+1}x}{2k+1}\right]+C  \end{eqnarray} 
同理, \int \cos^{2n+1}x\,\mathbb{d}x=\sum_{k=0}^n\frac{(-1)^k\sin^{2k+1}x}{2k+1}+C 
4.倍角法的应用
求 \int \sin^p x\cos^q x\,\mathbb{d}x 
1.若 \,p\, 或 \,q\, 为正奇数,此处设 q=2n+1 
\begin{eqnarray} LHS&=&\int\sin^p x(1-\cos^2 x)^n\,\mathbb{d}\sin x\\ &=&\int\sum_{k=0}^n(-1)^k\binom{n}{k}\sin^{p+2k}x\,\mathbb{d}\sin x\\ &=&\sum_{k=0}^n(-1)^k\binom{n}{k}\frac{\sin^{p+2k+1}x}{p+2k+1}+C \end{eqnarray} 
2.若 \,p+q\, 为负偶数,此处设 p+q=-2n 
\begin{eqnarray} LHS&=&\int \sin^px\cos^q x\,\mathbb{d}x\\ &=&\int \tan^px\cos^{p+q}x\,\mathbb{d}x\\ &=&\int\tan^px\sec^{2n}x\,\mathbb{d}x\\ &=&\int\tan^px(1+\tan^2x)^{n-1}\mathbb{d}\tan x\\ &=&\int\sum_{k=0}^{n}\binom{n-1}{k}\tan^{p+2k}x\,\mathbb{d}\tan x\\ &=&\sum_{k=0}^{n-1}\binom{n-1}{k}\frac{\tan^{p+2k+1}x}{p+2k+1}+C \end{eqnarray} 
5.含三角有理函数的积分
求 \int \frac{\,\mathbb{d}x}{\cos x}
我们知道 \frac{\mathbb{d}}{\mathbb{d}x}\left(\sec x+\tan x\right)=(\sec^2x+\tan x\sec x)\,\mathbb{d}x
令 u=\sec x+\tan x\,,\mathbb{d}u=u\sec x\,\mathbb{d}x
\begin{eqnarray} LHS&=&\int\frac{\mathbb{d}u}u\\ &=&\ln|u|+C\\ &=&\ln|\sec x+\tan x|+C \end{eqnarray}
同理可求 \int \frac{\mathbb{d}x}{\sin x}=-\ln|\csc x+\cot x|+C
求\int\frac{\mathbb{d}x}{a+b\cos x}
a>b 时:
\begin{eqnarray} LHS&=&\int\frac{\mathbb{d}x}{a\left(\cos^2\frac{x}{2}+\sin^2\frac{x}{2}\right)+b\left(\cos^2\frac{x}{2}-\sin^2\frac{x}{2}\right)}\\ &=&\int\frac{\mathbb{d}x}{(a+b)\cos^2\frac{x}{2}+(a-b)\sin^2\frac{x}{2}}\\ &=&\int\frac{\mathbb{d}x}{(a-b)\cos^2\frac{x}{2}\left(\frac{a+b}{a-b}+\tan^2\frac{x}{2}\right)}\\ &=&\frac{2}{a-b}\int\frac{\mathbb{d}\tan\frac{x}2}{\left(\frac{a+b}{a-b}\right)+\tan^2\frac{x}{2}}\\ &=&\frac{2}{\sqrt{a^2-b^2}}\arctan\left(\sqrt{\frac{a-b}{a+b}}\tan\frac{x}2\right)+C \end{eqnarray}
a<b 时:
\begin{eqnarray} LHS&=&\int\frac{\mathbb{d}x}{a\left(\cos^2\frac{x}{2}+\sin^2\frac{x}{2}\right)+b\left(\cos^2\frac{x}{2}-\sin^2\frac{x}{2}\right)}\\ &=&\int\frac{\mathbb{d}x}{(b+a)\cos^2\frac{x}{2}-(b-a)\sin^2\frac{x}{2}}\\ &=&\int\frac{\mathbb{d}x}{(b-a)\cos^2\frac{x}{2}\left(\frac{b+a}{b-a}-\tan^2\frac{x}{2}\right)}\\ &=&\frac{2}{b-a}\int\frac{\mathbb{d}\tan\frac{x}2}{\left(\frac{b+a}{b-a}\right)-\tan^2\frac{x}{2}}\\ &=&\frac{1}{\sqrt{a^2-b^2}}\ln\frac{\sqrt{b+a}+\sqrt{b-a}\tan\frac{x}2}{\sqrt{b+a}-\sqrt{b-a}\tan\frac{x}2}+C \end{eqnarray}
求 \int\frac{\mathbb{d}x}{a+b\sin x}
令 x=y+\frac{\pi}2 ,立刻可得:
a>b 时: LHS=\frac{2}{\sqrt{a^2-b^2}}\arctan\left[\sqrt{\frac{a-b}{a+b}}\tan\left(\frac{x}{2}-\frac{\pi}4\right)\right]+C
a<b 时: LHS=\frac{1}{\sqrt{a^2-b^2}}\ln\frac{\sqrt{b+a}+\sqrt{b-a}\tan\left(\frac{x}2-\frac{\pi}{4}\right)}{\sqrt{b+a}-\sqrt{b-a}\tan\left(\frac{x}2-\frac{\pi}{4}\right)}+C
求 \int\frac{\mathbb{d}x}{a\sin x+b\cos x}
由万能角公式: a\sin x+b\cos x=\sqrt{a^2+b^2}\sin(x+\alpha)
\begin{eqnarray} LHS&=&\frac{1}{\sqrt{a^2+b^2}}\int\frac{\mathbb{d}x}{\sin(x+\alpha)}\\ &=&\frac{1}{\sqrt{a^2+b^2}}\int\frac{\mathbb{d}(x+\alpha)}{\sin(x+\alpha)}\\ &=&-\ln|\csc (x+\alpha)+\cot (x+\alpha)|+C \end{eqnarray}
其中 \alpha=\arctan\frac{b}{a}
\int\frac{\mathbb{d}x}{a+b\cos x+c\sin x}
由万能角公式: b\cos x+c\sin x=\sqrt{b^2+c^2}\cos(x-\alpha)
\begin{eqnarray} LHS&=&\int\frac{\mathbb{d}x}{a+\sqrt{b^2+c^2}\cos(x-\alpha)}\\ &=&\int\frac{\mathbb{d}(x-\alpha)}{a+\sqrt{b^2+c^2}\cos(x-\alpha)} \end{eqnarray}
再应用上面的结论:
a^2>b+c^2 时:
LHS=\frac{2}{\sqrt{a^2-(b^2+c^2)}}\arctan\left(\sqrt{\frac{a-\sqrt{b^2+c^2}}{a+\sqrt{b^2+c^2}}}\tan\frac{x-\alpha}2\right)+C\\
a^2<b+c^2 时:
LHS=\frac{1}{\sqrt{a^2-(b^2+c^2)}}\ln\frac{\sqrt{\sqrt{b^2+c^2}+a}+\sqrt{\sqrt{b^2+c^2}-a}\tan\frac{x}2}{\sqrt{\sqrt{b^2+c^2}+a}-\sqrt{\sqrt{b^2+c^2}-a}\tan\frac{x}2}+C\\ 
其中 \alpha=\arctan\frac{c}{b}
6.三角函数的幂的积分
求 \int \sec^nx\,\mathbb{d}x
若 n 为偶数:
\begin{eqnarray} \int \sec^{2k}x\,\mathbb{d}x&=&\int \sec^{2k-2}x\sec^2x\,\mathbb{d}x\\ &=&\int(1+\tan^2x)^{k-1}\,\mathbb{d}\tan x\\ &=&\int\sum_{i=0}^{k-1}\binom{k-1}{i}\tan^{2i}x\,\mathbb{d}\tan x\\ &=&\sum_{i=0}^{k-1}\binom{k-1}{i}\frac{\tan^{2i+1}x}{2i+1}+C \end{eqnarray}
若 n 为奇数:
\begin{eqnarray} \frac{\mathbb{d}}{\mathbb{d}x}\left(\tan x\sec^{n-2}x\right)&=&\tan x\cdot(n-2)\sec^{n-2}x\tan x+\sec^2x\cdot\sec^{n-2}x\\ &=&(n-2)\sec^{n-2}(\sec^2-1)+\sec^nx\\ &=&(n-1)\sec^nx-(n-2)\sec^{n-2}x   \end{eqnarray}\\
故 \sec^{n}x=\frac{1}{n-1}\frac{\mathbb{d}}{\mathbb{d}x}\left(\tan x\sec^{n-2}x\right)+\frac{n-2}{n-1}\sec^{n-2}x
所以 \int\sec^nx\,\mathbb{d}x=\frac{1}{n-1}\tan x\sec^{n-2}x+\frac{n-2}{n-1}\int \sec^{n-2}x\,\mathbb{d}x
此即 n 为奇数时 \int\sec^nx\,\mathbb{d}x 的递推公式
求 \int \csc^{n}x\,\mathbb{d}x
若 n 为偶数:同理可得 \int \csc^{2k}x\,\mathbb{d}x=-\sum_{i=0}^{k-1}\binom{k-1}{i}\frac{\cot^{2i+1}x}{2i+1}+C
7.微分积分法
求 I_n=\int \frac{\mathrm{d}x}{(a+b\cos x+c\sin x)^n}
设 P=\frac{-b\sin x+c\cos x}{(a+b\cos x+c\sin x)^{n-1}} ,则:
\begin{eqnarray} \frac{\mathrm{d}P}{\mathrm{d}x}&=&\frac{(-b\cos x-c\sin x)(a+b\cos x+c\sin x)^{n-1}-(-b\sin x+c\cos x)^2(n-1)(a+b\cos x+c\sin x)^{n-2}}{(a+b\cos x+c\sin x)^{2n-2}}\\ &=&\frac{(-b\cos x-c\sin x)(a+b\cos x+c\sin x)-(n-1)(-b\cos x+c\cos x)^2}{(a+b\cos x+c\sin x)^n}\\ &=&\frac{(n-1)(a^2-b^2-c^2)}{(a+b\cos x+c\sin x)^n}-\frac{a(2n-3)}{(a+b\cos x+c\sin x)^{n-1}}+\frac{n-1}{(a+b\cos x+c\sin x)^{n-2}} \end{eqnarray}\\
\begin{eqnarray} P&=&\int \frac{\mathrm{d}P}{\mathrm{d}x}\,\mathrm{d}x\\ &=&(n-1)(a^2-b^2-c^2)\int\frac{\mathrm{d}x}{(a+b\cos x+c\sin x)^{n}}-a(2n-3)\int\frac{\mathrm{d}x}{(a+b\cos x+c\sin x)^{n-1}}+(n-2)\int\frac{\mathrm{d}x}{(a+b\cos x+c\sin x)^{n-2}} \end{eqnarray}
故: I_n=\frac{1}{(n-1)(a^2-b^2-c^2)}\left[\frac{-b\sin x+\cos x}{(a+b\cos x+c\sin x)^{n-1}}+a(2n-3)I_{n-1}-(n-2)I_{n-2}\right]
此即 I_n 的递推公式,且 I_0=x , I_1 我们前面已经求过
求 J_n=\int\frac{\sin^m x}{(a+b\cos x)^n}\,\mathrm{d}x
首先 J_0=\int\sin^m x\,\mathrm{d}x ,这个积分我们已经求过
然后求 J_1=\int\frac{\sin^m x}{a+b\cos x}\,\mathrm{d}x
当 m=2k+1 时,令 u=a+b\cos x
则 \cos x=\frac{u-a}{b}\,,\sin x=\sqrt{1-\left(\frac{u-a}{b}\right)^2}\,,\mathrm{d}u=-b\sin x\,\mathrm{d}x
\begin{eqnarray} J_1&=&\int\frac{\sin^{2k+1} x}{a+b\cos x}\,\mathrm{d}x\\ &=&-\frac{1}{b}\int\frac{\sin^{2k}x(-b\sin x\,\mathrm{d}x)}{a+b\cos x}\\ &=&-\frac{1}{b}\int\left[1-\left(\frac{u-a}{b}\right)^2\right]^k\frac{\mathrm{d}u}{u} \end{eqnarray}
展开,逐项积分即可
当 m=2k 时,则:
J_1=\int\frac{\sin^{2k} x}{a+b\cos x}\,\mathrm{d}x=\int\frac{(1-\cos^2k)^k}{a+b\cos x}\,\mathrm{d}x
展开,长除,逐项积分即可
接下来,令 P=\frac{\sin^{m+1}x}{(a+b\cos x)^{n-1}}
\begin{eqnarray} \frac{\mathrm{d}P}{\mathrm{d}x}&=&\frac{(m+1)\sin^mx\cos x(a+b\cos x)^{n-1}-\sin^{m+1}x(n-1)(a+b\cos x)^{n-2}(-b\sin x)}{(a+b\cos x)^{2n-2}}\\ &=&\frac{\sin^m x}{(a+b\cos x)^n}[(m+1)\cos x(a+b\cos x)+(n-1)b(1-\cos^2x)]\\ &=&\frac{\sin^m x}{(a+b\cos x)^n}[(n-1)b+(m+1)a\cos x+(m-n+2)b\cos^2x]\\ &=&\frac{\sin^m x}{(a+b\cos x)^n}\left[(n-1)(b^2-a^2)+\frac{(2n-m+3)a}{b}(a+b\cos x)+\frac{m-n+2}{b}(a+b\cos x)^2\right] \end{eqnarray}
\begin{eqnarray} P&=&\int\frac{\mathrm{d}P}{\mathrm{d}x}\\ &=&\frac{(n-1)(b^2-a^2)}{b}\int\frac{\sin^mx}{(a+b\cos x)^n}\,\mathrm{d}x+\frac{(2n-m-3)a}{b}\int\frac{\sin^mx}{(a+b\cos x)^{n-1}}\,\mathrm{d}x+\frac{m-n+2}{b}\int\frac{\sin^mx}{(a+b\cos x)^{n-2}}\,\mathrm{d}x\\  \end{eqnarray}
故:
J_n=\frac{1}{(n-1)(b^2-a^2)}\left[\frac{b\sin^{m+1}x}{(a+b\cos x)^{n-1}}+a(m-2n+3)J_{n-1}+(n-m-2)J_{n-2}\right]
此即 J_n 的递推公式
求 K_n=\int \frac{\cos^mx}{(a+b\cos x)^n}\,\mathrm{d}x
K_0=\int \cos^mx\,\mathrm{d}x ,这个积分我们前面已经求过
然后求 K_1=\int \frac{\cos^mx}{a+b\cos x}\,\mathrm{d}x ,同样的方法可以求出
接着,用同样的方法可知:
K_n=\frac{1}{(n-1)(a^2-b^2)}\left[\frac{\cos^{m+1}x}{(a+b\sin x)^{n-1}}+a(2n-m-3)K_{n-1}+(m-n+2)K_{n-2}\right]