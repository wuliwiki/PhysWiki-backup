% 行列式与体积
% keys 行列式|体积|二阶|三阶|平行四边形|平行六面体|立方体|行变换

\begin{issues}
\issueDraft
\end{issues}

\pentry{行列式的性质\upref{DetPro}}

\subsection{向量组的体积}

在 “行列式\upref{Deter}” 中我们看到了二阶和三阶行列式从几何上分别对应平行四边形的面积(即二维体积)和平行六面体的体积. 我们现在来证明 $N > 0$ 维空间的情况.

\begin{theorem}{行列式就是向量组的体积}
在标准正交基下,所有向量都可以用列矩阵来表示.在$N$维空间中,选出$N$个向量,将它们的列矩阵排成一行,得到一个$N\times N$方阵$M$.那么以这组向量为边长的平行多面体的“体积”,就是$M$的行列式$\abs{M}$.
\end{theorem}


我们先来看最简单的例子: 一个对角线\footnote{行列式或矩阵的对角线特指所有行标和列标相同的元素}元素都为 1, 其他元素为零的行列式
\begin{equation}
\begin{vmatrix}
1 & & &\\
  & 1 & &\\
  &  & \ddots &\\
  & & & 1
\end{vmatrix} = 1
\end{equation}
这代表 $N$ 维空间中边长都是 1 的立方体的体积.

进一步,
\begin{equation}
\begin{vmatrix}
a_1 & & &\\
  & a_2 & &\\
  &  & \ddots &\\
  & & & a_N
\end{vmatrix} = a_1a_2\cdots a_N
\end{equation}
是$N$维空间中各边长为$a_1, a_2, \cdots, a_N$的“长方体”的体积.

各边长两两垂直的多面体体积很好算,那么不垂直的情况怎么办呢?回忆小学时学习的平行四边形体积的计算方法,即“平移法”,如\autoref{DetVol_fig1} 所示.
\begin{figure}[ht]
\centering
\includegraphics[width=12cm]{./figures/DetVol_1.pdf}
\caption{二维多面体的体积计算示意图.左图是由向量$\bvec{v}$和$\bvec{u}$为边的长方形,右边则是以向量$\bvec{v}$和$\bvec{u}+\bvec{v}/3$为边的平行四边形.二者体积相等.} \label{DetVol_fig1}
\end{figure}

\autoref{DetVol_fig1} 中二者体积(面积)相等,揭示了一个简单而重要的事实:如果一组向量里,其中一个向量$\bvec{u}$进行如下变换:在给定组中,取和$\bvec{u}$不共线的另一个向量$\bvec{v}$,再任取一个实数$a$,将$\bvec{u}$变换为$\bvec{u}+a\bvec{v}$,那么以变换后的向量组为边的平行多面体,其体积和变换前是一样的\footnote{这本质上是因为体积的定义依然是“底乘高”的方式,比如底边长乘以高得面积,底面积乘以高得三维体积.进行上述变换时,高和底边长都没有任何改变,因此体积不变.}.

体积的这一不变性,和行列式的一种不变性紧密联系:\autoref{DetPro_the5}~\upref{DetPro}.该定理中进行的列变换,就等价于上一段所说的向量组的变换.反复利用这个列变换,我们总能把一个任意矩阵化为\textbf{对角矩阵}\footnote{只有主对角线上元素不一定为零,其它元素都必为零的矩阵.},而这个变换每一步都不改变矩阵的行列式.同时,这个列变换相当于上述的“平移法”,每一步都不改变所得到的多面体的体积.

这样一来,任意矩阵所代表的立方体都可以变形为一个长方体,同时不改变其体积;在这个变形过程中,矩阵的行列式都不变.加上我们前面已经证明了,对于立方体(对角矩阵),行列式就等于其体积,因此可以推论,对于任何平行多面体,其体积都等于边长向量构成的矩阵的行列式.

\begin{example}{}\label{DetVol_ex1}
在三维空间中,给定一个平行四面体,其三边长对应的向量分别是$\pmat{1&2&3}\Tr$、$\pmat{2&2&1}\Tr$和$\pmat{1&7&3}\Tr$,则这个平行多面体的体积就是
\begin{equation}
\begin{vmatrix}
1&2&1\\
2&2&7\\
3&1&3
\end{vmatrix} = 25
\end{equation}
\end{example}

\begin{exercise}{}
用平移法,将\autoref{DetVol_ex1} 中的平行多面体变形为立方体,同时保持其面积不变.用这样的方法,计算其面积,验证是否等于$25$.

你的变形过程,相当于
\end{exercise}






% $N$ 维空间中的任意平行体都可以由该立方体经过两个操作(包括任意次序多次操作)得到. 一个是将某条边长乘以常数 $\lambda$ (可以为负), 第二个是将某条边所在的矢量乘以常数 $\lambda$ 然后加到另一条边上. 前者使体积乘以 $\abs{\lambda}$, 后者保持体积不变.

% 如果我们把行列式的每一行(或每一列, 下同)对应到平行体的每条边, 那么根据\autoref{DetPro_the3}~\upref{DetPro} 和\autoref{DetPro_the5}~\upref{DetPro}, 前者同样使行列式的值乘以常数 $\lambda$, 后者同样保持行列式的值不变. 从立方体开始所以经过一系列变换后, 行列式的绝对值同样表示对应平行体的体积.
