% 北京大学 2017 年 考研 量子力学
% license Usr
% type Note

\textbf{声明}:“该内容来源于网络公开资料,不保证真实性,如有侵权请联系管理员”

\subsection{一}
1.宽度为 $2L$ 的无限深势阱,范围为 $-L < x < L$,求能量本征态和相应的本征值。

2.已知 $t = 0$ 时处于基态,势阱宽度突然变为 $4L$,范围为 $-2L < x < 2L$,求随时间变化的波函数表达式 $\varphi(t)$,求处于变化后体系本征态能量的概率,求体系的能量平均值 $\overline{E(t)}$。
\subsection{二}
一个二维谐振子,哈密顿量为
$$\hat{H} = \frac{p_x^2}{2m} + \frac{1}{2} m \omega_x^2 x^2 + \frac{p_y^2}{2m} + \frac{1}{2} m \omega_y^2 y^2~$$
