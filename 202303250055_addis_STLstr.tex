% C++ 标准库常用容器

\begin{issues}
\issueDraft
\end{issues}

\subsection{pair}
\begin{itemize}
\item 例如 \verb|pair<string, int> p("abc", 123);|, \verb|p.first|, \verb|p.second|
\item 当且仅当 \verb|p.first, p.second| 都相等, \verb|==| 才会相等。
\item 比较大小时, 仅需要 \verb|<| 对两种类型都有定义。 若 \verb|p.first < q.first| 则 \verb|p < q|。 若 \verb|p.first == q.first|(根据 \verb|<| 来确定, 不需要使用 \verb|==|), 且 \verb|p.second < q.second|, 也有 \verb|p < q|。
\end{itemize}

\subsection{vector}
\begin{itemize}
\item constructor: \verb|v(N)| (default val), \verb|v(N,val)|, \verb|v({1,2,3})|
\item ini list 赋值: \verb|v = {1,2,3}|, (\verb|vector<vector<int>> vv;|)\verb|vv = {{1,2,3},{2,3},{4,5,6}}|。
\item 成员函数 \verb|insert(iter, val)| 在指定位置插入一个元素。 \verb|insert(iter, n, val)| 插入多个元素。 \verb|insert(iter, iter2_begin, iter2_end)| 第二个 vector 的第一个元素会插入到第一个 vector 的 \verb|iter| 位置。 \verb|insert(iter, initializer_list)| 插入 \verb|initializer_list|
\item 成员函数 \verb|erase(iter)| 删掉一个元素, \verb|erase(iter_beg, iter_end)| 删掉一段元素。
\item \verb|.resize()| 不会改变原来元素的值, 新的值会 value-initialized。
\item \verb|vector<> v(N)| 也是 value-initialized。
\end{itemize}

\subsection{unordered\_map}
\begin{itemize}
\item \href{https://cplusplus.com/reference/unordered_map/unordered_map/}{unordered\_map}
\item \verb|unordered_map<key类型, val类型>|
\item \verb|umap.count(key)| 如果存在返回 1, 否则返回 0。
\item \verb|unordered_map<>::value_type| 是一个 \verb|pair<>(key, value)|
\item \verb|key| 如果不存在, 调用 \verb|umap[key]| 会生成一个新的 \verb|pair|, 其 value 是默认值 (例如 \verb|int| 初始化为 \verb|0|), 并返回该 value 的左值。 如果 \verb|key| 存在, \verb|umap[key]| 也返回左值。
\item \verb|unordered_map| 会先用 \verb|std::hash| 函数查找 \verb|key|, 如果有 hash collilsion 也没关系, 会进一步对比区分。
\item 对每个 element 循环用 \verb|for (auto &e : umap)|, 每个 \verb|e| 是一个 \verb|pair<>|。
\item \verb|umap.insert(pair<>)| 可以插入一项, 但如果 \verb|key| 已经存在, 则不作为。
\item 如果要用 \verb|pair<>(key1,key2)| 作为 key, 定义以下类函数 \verb|hash_pair|, 并声明 \verb|unordered_map<key类型, val类型, hash_pair>|
\begin{lstlisting}[language=cpp]
template<class T> // from boost library
inline void hash_combine(size_t &seed, const T &v) {
    seed ^= hash<T>{}(v) + 0x9e3779b9 + (seed << 6) + (seed >> 2);
}

struct hash_pair { // similar to std::hash, for pair<>
    template<class T, class T1>
    size_t operator()(const pair<T,T1> &a) const {
        size_t h = 0;
        hash_combine(h, a.first);
        hash_combine(h, a.second);
        return h;
    }
};
\end{lstlisting}
\item \verb|unordered_set<>::iterator| 获取 iterator 类型, 和 \verb|umap.begin()| 相同。 \verb|unordered_set<>::const_iterator| 获取 iterator to const 类型, 和 \verb|umap.cbegin()| 相同。
\item \verb|iterator| 只支持 \verb|++| (forward iterator)。
\item \verb|umap.erase(key)| 或者 \verb|umap.erase(iterator)| 删除一个 pair, 若 \verb|key| 不存在则不作为, 但 \verb|iterator| 必须要存在(\verb|umap.end()| 就不行)。
\end{itemize}

\subsection{map}
\begin{itemize}
\item \href{https://cplusplus.com/reference/map/map/}{map}
\item \verb|map<key类型, val类型, 比较函数(可选), allocator(可选)>|
\item 不需要 hash 函数, 会按照 key 自动排序, key 需要可以比较大小。 不允许有重复的 key。
\item 具有双向 iterator
\end{itemize}

\subsection{unordered\_set}
\begin{itemize}
\item \href{https://cplusplus.com/reference/unordered_set/unordered_set/}{unordered\_set}: \verb|unordered_set<key类型, hash类型(默认 std::hash), ...>|
\item 和 \verb|unordered_map| 类似, 不过只有 key 没有 value
\item 成员函数: \verb|operator=, empty, max_size, begin, end, find, count, emplace, insert, erase|
\item \verb|erase| 不存在的元素不会报错。 \verb|insert| 重复的元素也不会报错。 \verb|count| 只可能返回 \verb|0, 1|
\end{itemize}

\subsection{set}
\begin{itemize}
\item 和 \verb|unordered_set| 类似, 只是会自动根据元素的大小排序, 越小的 iterator 元素值越小。
\end{itemize}

\subsection{stack}
\begin{itemize}
\item \href{https://cplusplus.com/reference/stack/stack/}{stack}: \verb|stack <class T, class Container = deque<T>>|
\item 同样是 container adaptor, 成员函数: \verb|empty, size, top, push, emplace, pop, swap|
\item \verb|Container| 至少支持的成员函数: \verb|empty, size, back, push_back, pop_back|
\item 不支持随机访问。
\end{itemize}

\subsection{queue}
\begin{itemize}
\item \href{https://cplusplus.com/reference/queue/queue/}{queue}: \verb|queue<class T, class Container = deque<T>>| 像排队一样, 后面进, 前面出。 不支持随机访问。 如果要 print, 可以复制一个, 然后边 print 边 pop。 事实上, \verb|deque| 是可以 iterate 以及随机访问的。
\item 成员函数: \verb|empty, size, front, back, push, emplace, pop, swap|, swap 交换两个 queue 的内容: \verb|p.swap(q);|, 相当于 \verb|std::swap(p, q)|。 queue 本身并不实现这些功能, 只是通过调用 \verb|Container| 的成员函数来实现(container adaptor)。
\item 其中 \verb|Container| 类型至少应该支持 \verb|empty, size, front, back, push_back, pop_front|, 上一条的功能都是通过调用这些实现的。
\end{itemize}

\subsection{deque}
\begin{itemize}
\item \href{https://cplusplus.com/reference/deque/deque/}{double ended queue}: \verb|deque<class T, class Alloc = allocator<T> >|
\item 直接用 \verb|deque| 比 \verb|stack| 和 \verb|queue| 功能都要更多。
\item 成员函数(基本是 \verb|vector| 的拓展版): \verb|begin, end, operator[], size, max_size, resize, empty, front, back, push_front, pop_back, pop_front, pop_back, insert, erase, swap, clear, emplace, emplace_front, emplace_back|
\end{itemize}

\subsection{priority\_queue}
\begin{itemize}
\item \href{https://cplusplus.com/reference/queue/priority_queue/}{priority\_queue} 总是先处理(\verb|top|, \verb|pop|)最大的值, 但 iterate 时并不是完全排序的。 所以比(ordered) \verb|set| 可能更快。
\item \verb|priority_queue<数值类型, 容器(默认 vector), Compare(默认 less)>|
\item 要想先 pop 最小值 \verb|priority_queue<T, vector<T>, std::greater<T>>|, 或者定义成 \verb|template<typename T> using priority_queue2 = std::priority_queue<T, vector<T>, std::greater<T>>;|
\item 是一个 container adaptor, 底层默认用 \verb|vector| 储存数据。
\item 成员函数 \verb|empty|, \verb|size|, \verb|top|, \verb|push|, \verb|emplace|, \verb|pop|, \verb|swap|
\end{itemize}


\subsection{forward\_list}
\begin{itemize}
\item \href{https://cplusplus.com/reference/forward_list/forward_list/}{单链表}
\item 成员函数: \verb|sort()|(升序排序), \verb|merge(list2)| (合并两个升序的 list, 结果仍然是升序), \verb|insert_after(iter, val)|,  \verb|insert_after(iter, iter2_beg, iter2_end)|, \verb|erase_after(iter)|, \verb|erase_after(iter_beg, iter_end)|。
\end{itemize}

\subsection{list}
\begin{itemize}
\item \href{https://cplusplus.com/reference/list/list/}{双链表}
\item 支持几乎和单链表一样的操作以及更多。
\end{itemize}
