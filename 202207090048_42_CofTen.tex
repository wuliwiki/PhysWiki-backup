% 张量的坐标
% keys 张量坐标|坐标转换关系

\pentry{张量积\upref{TsrPrd}}
进行张量分析往往需要选择空间的基底,并用坐标去刻画张量.本节将证明张量所在空间的基底可由基矢之间的张量积\upref{TsrPrd}表示.
\subsection{记法约定}
在矢量空间 $V$ 和 $V^*$ 中选择相互对偶的基底(\autoref{DualSp_sub1}~\upref{DualSp})
\begin{equation}
V=\langle e_1,\cdots ,e_n\rangle,\quad V^*=\langle e^1,\cdots,e^n\rangle
\end{equation}
这里,按照惯例,空间 $V$ 中的基底指标排列在下方,$V^*$ 中则在上方.而在对应的坐标中,指标的排列则是对立的,即若 $x\in V,f\in V^*$ ,则 $x=\sum_{i}x^i e_i,f=\sum_{i}f_ie^i$.

我们知道, $V$ 和 $V^{**}$ 之间存在自然同构,这使得可以把 $\varphi\in V^{**}$ 和某个矢量 $x_{\varphi}\in V$ 等同起来.即 $x_{\varphi}(f)$ 等同于 $\varphi(f)$,其中 $f\in V^*$:
\begin{equation}
x_{\varphi}(f)\equiv\varphi(f)=f(x_{\varphi})
\end{equation}
上面 $\varphi(f)=f(x_{\varphi})$ 就是 $V$ 与 $V^{**}$ 之间建立的自然同构.

为了表示这种等同,可记
\begin{equation}
f(x)=(f,x)
\end{equation}
该记法按时这是一个内积,但来自于不同的空间,并且对于每个变量都是线性的.当 $f$ 固定时,这是 $V$ 上的一个线性函数,而当 $x$ 固定时,就是 $V^{*}$ 上的一个线性函数.
\begin{definition}{张量的坐标}
设 $T$ 是一个 $(p,q)$ 型的张量,则
\begin{equation}
T^{j_1,\cdots,j_q}_{i_1,\cdots,i_p}:=T(e_{i_1},\cdots,e_{i_p},e^{j_1},\cdots,e^{j_q})
\end{equation}
称为张量 $T$ 在基底 ${e_1,\cdots,e_n}$ 下的\textbf{坐标}(或\textbf{分量}).
\end{definition}

在多重线性映射\upref{MulMap} 词条中,已经说明了所以的 $p-$ 线性映射集合构成一个矢量空间,而张量 $(p,q)$ 型的张量只不过是在 $V_1=\cdots=V_p=V,\;V_1=\cdots=V_q,\;U=\mathbb F$ 时的 $(p+q)-$ 线性映射.所以所有的 $(p,q)$ 型张量构成的集合构成一个矢量空间.
\begin{definition}{$\mathbb T^q_p(V)$}
所有 $V$ 上 $(p,q)$ 型张量构成的集合构成一个矢量空间,记作 $\mathbb T^q_p(V)$.
\end{definition} 
\subsection{矢量空间 $\mathbb T^q_p(V)$ 的基}
\begin{theorem}{}
若 $T^{j_1,\cdots,j_q}_{i_1,\cdots,i_p}$ 是张量 $T$ 在基底 $\{e_1,\cdots,e_n\}$ 的坐标.而 $T_1$ 是如下构建的张量:
\begin{equation}\label{CofTen_eq2}
T_1=\sum_{i,j}T^{j_1,\cdots,j_q}_{i_1,\cdots,i_p}e^{i_1}\otimes\cdots\otimes e^{i_p}\otimes e_{j_1}\otimes\cdots\otimes e_{j_q}
\end{equation}
则 $T=T_1$.
\end{theorem}
\textbf{证明:}首先,张量由其坐标完全确定,这是因为若张量的坐标 $T^{j_1,\cdots,j_q}_{i_1,\cdots,i_p}$ 确定了,则对
\begin{equation}
\begin{aligned}
\forall x_1=\sum_{i_1}a^{i_1}e_{i_1},\quad\cdots,\quad x_p=\sum_{i_p}b^{i_p}e_{i_p}\\
\forall u^1=\sum_{j_1}c_{j_1}e^{j_1},\quad\cdots,\quad u^p=\sum_{j_p}d_{j_p}e^{j_p}
\end{aligned}
\end{equation}
有 
\begin{equation}\label{CofTen_eq1}
T(x_1,\cdots,x_p,u^1,\cdots,u^q)=\sum_{i,j}T^{j_1,\cdots,j_q}_{i_1,\cdots,i_p}a^{i_1}\cdots b^{i_p}c_{j_1}\cdots d_{j_p}
\end{equation}
而若两张量坐标相同,那么上式表明它们作用于任意相同的矢量和对偶空间的矢量构成的矢量组得到的结果相同,即坐标相同的两张量相等.现在只需要说明 $T,T_1$ 的坐标相同即可.

由\autoref{CofTen_eq2} 和张量积定义(\autoref{TsrPrd_def1}~\upref{TsrPrd}),得:
\begin{equation}
T_1(e_{i_1},\cdots,e_{i_p},e^{j_1},\cdots,e^{j_q})=T^{j_1,\cdots,j_q}_{i_1,\cdots,i_p}
\end{equation}

\textbf{证毕!}
 \begin{theorem}{}
 设 $\{e_1,\cdots,e_n\}$ 是空间 $V$ 上一个基底,$\{e^1,\cdots,e^n\}$ 是 $V^*$ 上的对偶基底.那么
\begin{equation}
e^{i_1}\otimes\cdots\otimes e^{i_p}\otimes e_{j_1}\otimes\cdots\otimes e_{j_q}
\end{equation}
构成 $\mathbb T_p^q(V)$ 上的一个基底. 
 \end{theorem}