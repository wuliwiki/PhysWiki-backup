% 加速度的参考系变换
% 加速度|科里奥利加速度|相对加速度

\pentry{速度的参考系变换\upref{Vtrans}}

\subsection{无相对转动}
若两个参考系之间只有平移没有转动, 两个参考系中任意两个固定点之间的加速度都是相等的(类比\autoref{Vtrans_eq1}~\upref{Vtrans}). 令某时刻点 $P$ 相对于 $S$ 系和 $S'$ 系的加速度分别为 $\bvec a_S$ 和 $\bvec a_{S'}$, 再令两坐标系之间的加速度为 $\bvec a_r$, 那么有
\begin{equation}\label{AccTra_eq1}
\bvec a_S = \bvec a_{S'} + \bvec a_r
\end{equation}
同样地, 如果要将该式写成分量的形式, 三个矢量必须使用同一坐标系(见\autoref{Vtrans_ex2}~\upref{Vtrans}).

\begin{example}{}\label{AccTra_ex1}
如\autoref{AccTra_fig1} 一个圆盘绕着圆形轨道的外侧无摩擦滚动, 圆盘中心的相对于轨道中心的角速度为 $\omega$, 求圆盘边缘上一点在任意时刻的加速度.
\begin{figure}[ht]
\centering
\includegraphics[width=4cm]{./figures/AccTra_1.pdf}
\caption{例 1 图示} \label{AccTra_fig1}
\end{figure}

解: 我们令 $S'$ 系的原点为点 $A$, 且 $S'$ 系无相对转动, $\bvec R$ 从 $O$ 指向接触点, $\bvec r$ 从 $A$ 指向边缘上一点 $P$. 那么在 $S'$ 系中, $P$ 的角速度等于 “相对于接触点的角速度” 加 “接触点的角速度”
\begin{equation}
\omega_1 = \frac{\omega R}{r} + \omega
\end{equation}
向心加速度为
\begin{equation}
\bvec a_{S'} = -\omega_1^2 \bvec r
\end{equation}

圆盘中心 $A$ 点对于圆形轨道中心 $O$ 点的加速度, 也就是 $S'$ 相对于 $S$ 的加速度为
\begin{equation}
\bvec a_r = -\omega^2 (R + r)\uvec R
\end{equation}
代入\autoref{AccTra_eq1} 得
\begin{equation}
\bvec a_S = \bvec a_{S'} + \bvec a_r = -\qty(\frac{R}{r} + 1)^2 \omega^2\bvec r - \qty(1 + \frac{r}{R})\omega^2 \bvec R
\end{equation}
\addTODO{当 $\bvec r$ 和 $\bvec R$ 同方向时……, 反方向时……}

% 当 $\bvec r$ 和 $\bvec R$ 同方向时,总加速度为 $\bvec a_S = ( a_{S'} + a_{r})\uvec R$.

% 当 $\bvec r$ 和 $\bvec R$ 反方向时,总加速度为 $\bvec a_S = (-a_{S'} + a_{r})\uvec R$.

另一种方法是假设 $S'$ 以 $A$ 为原点且相对 $S$ 系以 $\omega$ 逆时针旋转, 此时必须考虑下文的科里奥利加速度, 见\autoref{AccTra_exe1}.
\end{example}

\subsection{有相对转动}
类比\autoref{Vtrans_eq2}~\upref{Vtrans}, 若两参考系之间有可能存在转动, 牵连加速度 $\bvec a_{r}$ 的定义会变得比牵连速度 $\bvec v_r$ 更微妙, 因为牵连速度与参考系的选取无关, 而牵连加速度却有关! 我们举例解释.

\begin{example}{牵连加速度}
令 $S'$ 系原点沿着 $S$ 系的单位矢量 $\uvec x$ 匀速运动, 且相对 $S$ 系以恒定角速度矢量 $\bvec \omega = 2\uvec y$ 转动. 令 $t = 0$ 时两系完全重合. 我们来讨论此刻原点处的牵连速度.

显然, 两参考系 $t = 0$ 时刻的固定点就是各自的原点 $O$ 和 $O'$. $O'$ 延 $S$ 系的 $x$ 轴匀速运动, 所以 $S$ 系的观察者会认为牵连加速度为零. 然而在 $S'$ 系的观察者看来, $O$ 始终在做速度不为零的曲线运动, 所以 $t = 0$ 时牵连加速度不为零.
\end{example}

我们在 $S$ 系中讨论问题. 定义 $t$ 时刻点 $P$ 在 $S'$ 系中的固定点相对于 $S$ 系的加速度为 $\bvec a_{r}$. 那么可以证明(证明见下文)
\begin{equation}\label{AccTra_eq2}
\bvec a_S = \bvec a_{S'} + \bvec a_{r} + 2 \bvec \omega \cross \bvec v_{S'}
\end{equation}
这比\autoref{AccTra_eq1} 多出了一项, 其中 $\bvec \omega$ 是 $S'$ 系相对于 $S$ 系的瞬时角速度. 最后一项被称为\textbf{科里奥利加速度(Coriolis Acceleration)}
\begin{equation}\label{AccTra_eq4}
\bvec a_c = 2 \bvec \omega \cross \bvec v_{S'}
\end{equation}

若我们把 $S'$ 相对于 $S$ 的运动分解为原点的平移加绕原点的转动, 那么牵连加速度 $\bvec a_r$ 也可以分解为\textbf{平移加速度}和\textbf{旋转加速度}, 而旋转加速度又可以分为\textbf{向心加速度}(\autoref{CMAD_eq1}~\upref{CMAD})和\textbf{角加速度项}.
\begin{equation}\label{AccTra_eq5}
\bvec a_{r} = \bvec a_{O'} + \dv{t}(\bvec\omega\cross\bvec r_{S'})
= \bvec a_{O'} + \bvec\omega\cross(\bvec\omega\cross\bvec r_{S'}) + \dot{\bvec\omega} \cross\bvec r_{S'}
\end{equation}
其中平移加速度 $\bvec a_{O'}$ 的定义为 $S'$ 系原点在 $S$ 系中的加速度.

一种常见的特殊情况是, 当两坐标系原点重合, 且相对匀速转动时, 有
\begin{equation}
\bvec a_S = \bvec a_{S'} + \bvec\omega\cross(\bvec\omega\cross\bvec r_{S'}) + 2 \bvec \omega \cross \bvec v_{S'}
\end{equation}
最后两项分别是向心加速度以及科里奥利加速度.

\begin{exercise}{}\label{AccTra_exe1}
\begin{enumerate}
\item 请使用\autoref{Vtrans_ex1}~\upref{Vtrans} 的情景验证\autoref{AccTra_eq2}.
\item 使用旋转参考系计算\autoref{AccTra_ex1}, 是否能得到相同的结果?(假设 $S'$ 系关于 $S$ 系逆时针以 $\omega$ 旋转).
\end{enumerate}
\end{exercise}

\begin{exercise}{}
试证明\autoref{AccTra_eq5}.
\end{exercise}

\subsubsection{证明(旋转矩阵)}
\pentry{旋转矩阵的导数\upref{RotDer}}
我们在 $S$ 系中以坐标的形式证明\autoref{AccTra_eq2}, 如无声明式中的矢量都看作是 $S$ 系中的三个坐标. 令点 $P$ 在两系中的坐标分别为 $\bvec r_S(t) = (x\ y\ z)\Tr$ 和 $\bvec r_{S'}(t) = (x'\ y'\ z')\Tr$, 且坐标变换可以用一个旋转矩阵 $\mat R(t)$ 和一个平移矢量 $\bvec d(t)$ 表示为
\begin{equation}
\bvec r_S = \mat R \bvec{r}_{S'} + \bvec d
\end{equation}
两边关于时间求导得\footnote{用符号上方一点表示时间导数, 两点表示时间二阶导数.}
\begin{equation}
\dot{\bvec r}_S = \dot{\mat R} \bvec{r}_{S'} + \mat R \dot{\bvec r}_{S'}+ \dot{\bvec d}
\end{equation}
再求导并整理得
\begin{equation}\label{AccTra_eq3}
\ddot{\bvec r}_S = \mat R \ddot{\bvec r}_{S'} + (\ddot{\mat R} \bvec r_{S'} + \ddot{\bvec d}) + 2 \dot{\mat R} \dot{\bvec r}_{S'}
\end{equation}
下面我们只需证明这三项分别对应\autoref{AccTra_eq2} 的各项即可.

在 $S$ 系中, 显然有 $\bvec a_S = \ddot{\bvec r}_S$. $P$ 在 $S'$ 系中的加速度为 $\ddot{\bvec r}_{S'}$, 乘以旋转矩阵就变换到 $S$ 系中, 所以 $\bvec a_{S'} = \mat R \ddot{\bvec r}_{S'}$.

若 $S'$ 系中的固定点 $\bvec r_{S'}$ 不随时间变化, 则\autoref{AccTra_eq1} 求二阶导数得 $S'$ 系中固定点相对于 $S$ 系中固定点的加速度(在 $S$ 系中的坐标)
\begin{equation}
\bvec a_r = \ddot{\mat R} \bvec r_{S'} + \ddot{\bvec d}
\end{equation}

再来看\autoref{AccTra_eq3} 最后一项, 将\autoref{RotDer_eq4}~\upref{RotDer} 代入, 得
\begin{equation}
2\dot{\mat R} \dot{\bvec r}_{S'} = 2\mat\Omega (\mat R \dot{\bvec r}_{S'}) = 2\bvec \omega \cross \bvec v_{S'}
\end{equation}
这就是\autoref{AccTra_eq2} 的最后一项. 证毕.
