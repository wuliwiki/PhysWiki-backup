% Bash 编程笔记
% license Xiao
% type Note

\begin{issues}
\issueDraft
\end{issues}

% 学习可以参考 v2r 的自动安装脚本

\subsection{基础}
\begin{itemize}
\item 一个\href{https://tldp.org/LDP/abs/html/why-shell.html}{教程}
\item 一个比较笨的调试方法是, 把一些要调试的命令用 \verb`echo "..."` 先输出来看看, 而不是真的执行。 看看变量展开是否和预期的一致(就像 make 一样)。
\item Linux 系统中, shell 一般是指 \textbf{Bash (Bourne Again SHell)}。
\item shell script 就是拓展名为 \verb`.sh` 的文本文件, 也可以没有。
\item 执行 shell script 相当于执行可执行程序, 可以用 \verb`./name.sh 参数1 参数2` 也可以用 \verb`source /path/name.sh 参数1 参数2` (\verb`source` 也可以简写成 \verb`.`。 \verb`sh` shell 中只能用后者。), 前者在一个新的 shell process 中运行然后退出(环境变量之类的改动在退出会消失), 后者在当前的 shell process 运行, 相当于直接把文件中的内容复制粘贴到命令行执行.
\item 用 \verb`;` 隔开两个命令, 第一个失败了第二个仍然会运行, 但 \verb`&&` 隔开则不会(这个其实是逻辑与算符)。
\item \textbf{所有变量都是字符串}
\item \verb`&&` 后面可以直接换行, 不需要加 \verb`\`
\item shell script 后面可以输入若干 arguments, 还可以用 \verb`>` 把 shell 中的 stdout 导入文件.
\item 文件第一行可以用 \textbf{shebang} \verb`#!` 指定运行使用的 shell, 如 \verb`#!/bin/bash`
\item 用 \verb`exit` 结束 script, \verb`exit 0` 声明运行成功
\item 用 \verb`#` 注释
\item 赋值用 \verb`var=xxx`, 等号两边不能有空格。 \verb`$var` 或者 \verb`${var}` 获取它的值。例如 \verb`abc$var` 或者 \verb`abc${var}def`。 后者的花括号不能省略。
\item \verb`${var:0:5}` 可以获取前 5 个字符。 \verb`${var/旧文字/新文字}` 可以对变量的字符串进行替换
\item \verb`${var:-"默认字符串"}` 如果 \verb`var` 未定义或为空,则输出 \verb`默认字符串`。
\item 如果变量 \verb`var` 以 \verb`字符串` 结束, \verb`${var%字符串}` 移除最后的 \verb`字符串`
\item 如果 \verb`var` 以 \verb`字符串` 开始, \verb`${var#字符串}` 移除最前面的 \verb`字符串`
\item 要把变量的内容作为另一个变量的名字展开, 如 \verb`varname=myvar`, 那么 \verb`${!varname}` 相当于 \verb`${myvar}`。 如果 \verb`varname` 是第一个参数, 那么就用 \verb`${!1}`
\item \verb`$(命令)` 或者 \verb|`命令`| 相当于把 \verb`命令` 的输出展开到当前位置, 如 \verb|echo `pwd -P`|。
\item 赋值给变量时, 双引号中的多个空格会变为一个: \verb`bins="123    234"`
\item 脚本中用 \verb`exec 文件名.sh` 执行另一个脚本。
\item 如果要用脚本输出 stderr, 在命令后面加 \verb`1>&2`, 例如 \verb`echo bar 1>&2`。
\item 同理, \verb`./文件名.sh > 123.log` 只会把 stdout 导入文件, 而不包含 stderr。 还要在后面加上 \verb`2>&1`。
\item 要在一个 script 中插入另一个, 用 \verb`"path/to/other_script.sh"` (待验证)
\item 在其他 shell 中, 要使用 bash 执行单个命令(而不是进入 bash) 用 \verb`/bin/bash -c "命令"`
\end{itemize}

\subsubsection{关于引号}
\begin{itemize}
\item \verb`命令 "abc def"` 或者 \verb`命令 'abc def'` 中, \verb`abc def` 是一个参数, 作为字符串传给 \verb`main()` 函数的 \verb`argv[1]`, 不包含引号。 如果不加引号, \verb`abc` 和 \verb`def` 就分别是 \verb`argv[1]` 和 \verb`argv[2]`。
\item \verb`"..."` 中的 \verb`$变量` 会展开, 但是 \verb`'...'` 中不会
\item \verb`'...'` 就是\textbf{绝对的 string literal},里面不会发生任何转义
\item 要在 \verb`'...'` 中 escape 单引号,不能直接做到,可以用 \verb`'it'\''s here.'` 其中三部分分别是 \verb`'it'`, \verb`\'`, \verb`'s here.'`。
\item \verb`echo 'abc\'def'gh'` 会输出 \verb`abc\defgh`
\item 单引号和双引号中间都可以换行。
\item 如果参数中出现了引号想要传给 \verb`main()`, 那么就 escape: \verb`\"` 或者 \verb`\'`
\item 不在引号内的 \verb`\` 可以用 \verb`\\` 来 escape
\item 最外层引号内部的所有其他符号都是 raw string, 不需要 escape。
\item 双引号中可以用 \verb`$var`、\verb`${var}` 等展开。
\end{itemize}

\subsection{pipe}
\begin{itemize}
\item \verb`命令 | 命令2` 使用 pipe 把一个命令在 stdout 的输出作为另一个命令的 stdin 输入。 特别注意 \verb`命令2` 是在 subshell 执行的。 如果在该 subshell 执行多个命令, 可以用 \verb`命令 | { 命令2 ; 命令3 }` 例如 \verb`echo 123 | { read line; echo "The line is: $line"; }` 又例如 \verb`printf "123\n234\n" | while read line; do echo "The line is: $line"; done` 中 \verb`while` 可以看成一个命令。
\end{itemize}


\subsection{判断}
\begin{lstlisting}[language=bash]
echo 123;
if [ $? -ne 0 ]; then # 判断 exit status
...
fi

# 判断路径是否存在(感叹号后面一定要有空格)
if ! [ -d "$repo" ]; then
    echo "directory not found!" 1>&2
fi

# 判断文件是否存在且不是文件夹或者特殊文件
if ! [ -f "$fname" ]; then
    echo "file not found!" 1>&2
fi

# 判断相等
if [ $i -eq 3 ]
    ...
elif [ $i -ne 3 ]
    ...
fi
\end{lstlisting}
\begin{itemize}
\item 一种等效于 \verb`if` 的简写如 \verb`[ -e 文件 ] && mv 文件 新文件`。 例子: \verb`for file in file1 file2 file3; do [ -e "$file" ] && mv "$file" destination_directory done`
\item 类似于 \verb`-f, -d`, 还有 \verb`-z`(判断字符串是否为空), \verb`-n`(判断字符串是否为非空), 例如 \verb`if [ -n $(git remote | grep github) ]`。  \verb`-e` 判断是否是文件(包括文件夹和特殊文件), \verb`-h` 判断是否为 symlink, \verb`file1 -nt file2` (newer than), \verb`file1 -ot file2` (older than), \verb`-r, -w, -x` 是否是文件且具有读写执行权限。
\item \verb`if [ string1 = string2 ]` 判断字符串是否相等, \verb`string1 != string2` 判断是否不相等。
\item 中括号中:\verb`-z` 检查后面的内容是否为空.
\item \verb`-n` 检查是否为非空
\item \verb`-v` 检查变量是否有定义(定义为空字符也算)
\end{itemize}

\subsubsection{双方括号判断}

双方括号不属于 POSIX 标准而是 bash 和 zsh 等特有的。 语法更为现代
\begin{itemize}
\item \verb`if [[ $a == $b ]]; then` 判断相等, 另有 \verb`!=`, \verb`<`, \verb`>` 等。 注意 \verb`<=` 注意比较的不是数值大小而是字符串排序大小。
\item \verb`if [[ "字符串" =~ 正则表达式 ]]` 可以判断字符串是否匹配正则表达式。 其中正则表达式的格式和 \verb`grep -E` 中的是一样的。 例如 \verb`if [[ "$var" =~ ^-?[0-9]+$ ]]` 判断变量是否表示整数。
\begin{lstlisting}[language=bash]
# 判断文件是否为空
if [ -s 文件名 ]; then
    # The file is not-empty.
else
    # The file is empty.
fi

# 比较数字大小
if (( a > b )); then
    ...
fi
\end{lstlisting}
\end{itemize}

\subsection{循环}
\begin{lstlisting}[language=bash]
#!/bin/bash
for ((i = 1; i < 5; ++i))
# for i in {1,2,3,4} # 等效
# for i in 1 2 3 4 # 等效
do
	printf "\nfile${i}.txt\n"
done
\end{lstlisting}
如果在 \verb`printf` 语句后面加上 \verb`&& break`, 当前者失败时就会跳出循环。

另外 \verb`continue` 也可以跳出循环

在当前目录中所有子目录循环
\begin{lstlisting}[language=bash]
for d in */ ; do
    d=${d%/} # 删除目录名最后的斜杠
    echo "$d" # 打印所有子目录名
done
\end{lstlisting}
文件循环
\begin{lstlisting}[language=bash]
for file in *; do
    if [ -f "$file" ]; then
        echo "$file"
    fi
done
\end{lstlisting}

\subsection{数组}
\begin{lstlisting}[language=bash]
#!/bin/bash
arr=(1 3 hello 13)
for ((i=0;i<4;++i))
do
    printf "arr[${i}] = ${arr[i]}\n"
done
\end{lstlisting}


\subsection{输入}
例子
\begin{lstlisting}[language=bash]
#!/bin/bash
# 确保文件最后有两个空格
# use `find . -type f -name "*.matt" -exec ./convert_matt.sh {} \;`
# to convert subfolder
file=${1}
printf "${file}... "
c2=$(tail -c 2 ${file})
if [ "${c2}" == "  " ]
then
	printf "already in new format\n"
else
	printf "  " >> ${file}
	printf "done\n"
fi
\end{lstlisting}

\subsection{特殊变量}
\begin{itemize}
\item \verb`$0` 当前 shell script 的文件名, 例如调用脚本的命令是 \verb`some/../path/test.sh`, 那就返回这串字符。 如果用 \verb`source some/path/test.sh` 或者直接在命令行打 \verb`$0`, 那么就返回 \verb`bash`。
\item \verb`$@` 返回所有 shell arguments
\item \verb`$?` 上一个命令的 \textbf{exit status}, \verb`0` 代表成功, 否则失败
\item \verb`readlink -f "$0"` 获取脚本绝对路径(包括脚本文件名)(不可以直接在命令行使用, 或者 \verb`source 脚本.sh` 时使用)
\item \verb`dirname $(readlink -f "$0")` 获取脚本所在绝对目录(不包括脚本文件名)
\end{itemize}

\subsection{函数}

\begin{itemize}
\item 函数的定义为
\begin{lstlisting}[language=bash]
函数名() {
	...
}
\end{lstlisting}
\item 需要 argument 的话, 就用 \verb`$1, $2` 或者 \verb`${1}, ${2}`, 例如
\begin{lstlisting}[language=bash]
greeting() {
  echo "Hello $1"
}

greeting "Joe"
\end{lstlisting}
\end{itemize}

\verb`$0` 是调用脚本的命令(也就是 \verb`argv[0]`)。 \verb`$#` 可以查看 args 的总个数

一个用于确保执行成功的函数:
\begin{lstlisting}[language=bash]
check_failure() {
    if [ $1 -ne 0 ]; then
        echo "Error! Exiting..."
        exit 1
    fi
}

cmd1 arg1 && cmd2 arg2
check_failure $?
\end{lstlisting}

\subsection{获取系统信息}
\begin{itemize}
\item \verb`lsb_release -rs` 可以获取 ubuntu 的系统版本号(例如 \verb`18.04`)
\item \verb`getconf _NPROCESSORS_ONLN` 可以查看系统有多少逻辑 cpu
\end{itemize}
