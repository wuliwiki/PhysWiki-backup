% 杨-米尔斯理论(综述)
% license CCBYSA3
% type Wiki

本文根据 CC-BY-SA 协议转载翻译自维基百科\href{https://en.wikipedia.org/wiki/Yang\%E2\%80\%93Mills_theory}{相关文章}。

杨-米尔斯理论是由杨振宁和罗伯特·米尔斯于1953年提出的一种量子场论,用于描述核结合力,也广泛用于描述类似的理论。杨-米尔斯理论是一种基于特殊酉群 SU(n) 或更一般的紧李群的规范理论。杨-米尔斯理论试图利用这些非阿贝尔李群来描述基本粒子的行为,是电磁力和弱力的统一(即 U(1) × SU(2))以及量子色动力学(强力理论,基于 SU(3))的核心。因此,它构成了粒子物理学标准模型理解的基础。
\subsection{历史与定性描述}  
\subsubsection{电动力学中的规范理论}  
所有已知的基本相互作用都可以通过规范理论来描述,但这一点的确立花费了几十年的时间。[2] 赫尔曼·外尔的开创性工作始于1915年,当时他的同事艾米·诺特证明了每一个守恒的物理量都有一个匹配的对称性,并最终在1928年出版了他的著作,将对称性几何理论(群论)应用于量子力学。[3]: 194  外尔将诺特定理中相关的对称性命名为“规范对称性”,类比于铁路轨距中的距离标准化。

1922年,厄尔温·薛定谔在工作于薛定谔方程之前的三年,连接了外尔的群概念与电子电荷。薛定谔展示了群 U(1) 会在电磁场中产生一个相位变化 \( e^{i\theta} \),该相位变化与电荷守恒相匹配。[3]: 198  随着量子电动力学在1930年代和1940年代的发展,U(1) 群变换在其中起到了核心作用。许多物理学家认为,必定存在一种与核子动力学类似的理论,特别是杨振宁对这种可能性十分痴迷。