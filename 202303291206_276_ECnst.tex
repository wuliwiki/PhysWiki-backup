% 机械能守恒(单个质点)
% keys 机械能守恒|动能|势能|动能定理

\pentry{动能定理\upref{KELaw1}, 势能\upref{V}}
若质点只受不随时间变化的保守力作用\footnote{即势能函数不随时间变化}, 那么物体在某段时间内从 $A$ 点移动到 $B$ 点, 力场对物体做功能等于初末势能函数之差
\begin{equation}
W_{AB} = V(\bvec r_A) - V(\bvec r_B)~.
\end{equation}
而根据动能定理, 力场对质点做功等于质点的末动能减初动能
\begin{equation}
W_{AB} = E_{kB} - E_{kA} = \frac12 m v_B^2 - \frac12 m v_A^2~.
\end{equation}
结合以上两式,得
\begin{equation}
E_{kA} + V(\bvec r_A) = E_{kB} + V(\bvec r_B)~,
\end{equation}
我们现在定义质点在某个时刻的动能加势能为\textbf{机械能}。 上式就是单个质点机械能守恒的表达式。
