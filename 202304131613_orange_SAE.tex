% 单电子原子模型

\begin{issues}
\issueDraft
\end{issues}

\textbf{单电子原子模型(single active electron SAE)}氦原子 SAE 势能, 是指通过给对氢原子薛定谔方程修改势能项(等效势能), 用于模拟其他原子的行为, 尤其是单电子电离。

例如 Tong-Lin 模型中, 氦原子的等效势能为
\begin{equation}
V(r) = -\frac{Z_c + a_1 \E^{-a_2 r} + a_3 r \E^{-a_4 r} + a_5 \E^{-a_6 r}}{r}~.
\end{equation}
对氦原子, 有
\begin{equation}
Z_c = 1, \ a_1 = 1.231,\ a_2 = 0.662,\ a_3 = -1.325,\ a_4 = 1.236,\ a_5 = -0.231,\ a_6 = 0.480
\end{equation}
满足
\begin{align}
&\lim_{r\to +\infty} \frac{V(r)}{-Z_c/r} = \lim_{r\to +\infty} \frac{V(r)}{-1/r} = 1\\
&\lim_{r\to 0} \frac{V(r)}{-(Z_c + a_1 + a_5)/r} = \lim_{r\to 0} \frac{V(r)}{-2/r} = 1
\end{align}

这个势能的基态 $E_0 = -24.6$eV 就是氦原子单电离 threshold。 但激发态 $E_1 = -4.35$eV, $E_2 = -1.70$eV 并不正确。

而我现在需要氦原子 shake-up 的 SAE 势能。

\begin{lstlisting}[language=matlab]
V = @(r) (-1 -1.231*exp(-0.662*r) + 1.325*r .*exp(-1.236*r) ...
    + 0.231*exp(-0.480*r)) ./ r;
\end{lstlisting}

\subsubsection{shake-up}
若要模拟激发态 shake-up 电离, 一种较为粗暴的做法是直接把 $V$ 乘以一个常数 $2.3632$, 使得势阱更深, 电离能更大, 基态为 $E_0 = 65.4$。 这么做的问题是 $r$ 的两端极限都不物理。 做了 streaking 以后 $n=1,2$ delay 是 $18\Si{as}$。

另一种方式是, 就是重新令 $a_3 = 2.07415$。 这样极限依然正确。做了 streaking 以后 $n=1,2$ delay 是 ?
