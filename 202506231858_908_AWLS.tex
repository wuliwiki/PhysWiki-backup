% 埃瓦里斯特·伽罗瓦(综述)
% license CCBYSA3
% type Wiki

本文根据 CC-BY-SA 协议转载翻译自维基百科 \href{https://en.wikipedia.org/wiki/\%C3\%89variste_Galois}{相关文章}。

埃瓦里斯特·伽罗瓦(Évariste Galois,/ɡælˈwɑː/;法语发音:[evaʁist ɡalwa];1811年10月25日-1832年5月31日)是一位法国数学家和政治活动家。他在少年时期就成功找出了一个多项式是否可用根式求解的充要条件,从而解决了一个困扰数学界达350年的难题。他的工作奠定了伽罗瓦理论和群论的基础——这两个领域后来成为抽象代数的主要分支。

伽罗瓦是一位坚定的共和主义者,积极参与了围绕1830年法国革命的政治动荡。由于他的政治活动,他多次被捕,并服刑数月。出狱不久,他因某些至今仍不明的原因参加了一场决斗,并因伤重身亡。
\subsection{生平}
\subsubsection{早年生活}
伽罗瓦于1811年10月25日出生于尼古拉-加布里埃尔·伽罗瓦和阿德莱德-玛丽(Adélaïde-Marie,娘家姓德芒特,Demante)夫妇之家。[2][4] 他的父亲是一位共和主义者,是布尔拉雷讷自由党派的领袖。1814年路易十八复辟后,他的父亲成为该村的市长。[2] 他的母亲是一位法学家的女儿,精通拉丁语和古典文学,并负责伽罗瓦前十二年的教育。

1823年10月,伽罗瓦进入路易大帝中学,他的老师路易·保罗·埃米尔·理查德识别出了他的非凡才华。[5] 14岁时,他开始对数学产生浓厚兴趣。[5]

伽罗瓦找到了一本阿德里安-玛丽·勒让德(Adrien-Marie Legendre)的《几何原本》(Éléments de Géométrie),据说他“像读小说一样”阅读这本书,并在第一次阅读时就掌握了其内容。15岁时,他已经在阅读约瑟夫-路易·拉格朗日(Joseph-Louis Lagrange)的原始论文,如《代数方程解法的思考》(Réflexions sur la résolution algébrique des équations),这很可能激发了他后来在方程理论方面的研究,\[6] 以及《函数运算讲义》(Leçons sur le calcul des fonctions)——这是专为专业数学家撰写的著作。然而,他在课堂上的表现并不出色,老师还指责他摆出一副天才的架势。\[4]
