% 蒙特卡罗方法(综述)
% license CCBYSA3
% type Wiki

本文根据 CC-BY-SA 协议转载翻译自维基百科\href{https://en.wikipedia.org/wiki/Monte_Carlo_method}{相关文章}。

蒙特卡罗方法,又称蒙特卡罗实验,是一类依赖重复随机抽样以获得数值解的广义计算算法。其基本思想是利用随机性来解决一些本质上可能是确定性的问题。该名称源自摩纳哥的蒙特卡罗赌场,该方法的主要发展者、数学家斯坦尼斯瓦夫·乌拉姆受到他叔叔赌博习惯的启发而命名。

蒙特卡罗方法主要用于三类问题:优化问题,数值积分,从概率分布中抽样生成样本。此外,它还可以用于对输入存在高度不确定性的现象进行建模,例如评估核电站故障的风险。蒙特卡罗方法通常通过计算机模拟实现,能为那些无法解析求解或过于复杂的问题提供近似解。

蒙特卡罗方法广泛应用于多个科学与工程领域,包括:物理、化学、生物学、统计学、人工智能、金融、密码学,也被应用于社会科学,如:社会学、心理学与政治学。它被认为是20世纪最重要且最具影响力的思想之一,并推动了众多科学与技术突破的实现。

不过,蒙特卡罗方法也存在一些局限性和挑战,例如:精度与计算成本之间的权衡,维度灾难,随机数生成器的可靠性,结果的验证与确认。
\subsection{概述}
蒙特卡罗方法虽然形式多样,但通常遵循以下基本步骤:
\begin{enumerate}
\item 定义可能输入的范围(即问题的定义域);
\item 从该范围内的概率分布中随机生成输入样本;
\item 对每个输入进行确定性的计算,得到输出结果;
\item 汇总(聚合)所有结果,以得到最终的估计或解。
\end{enumerate}