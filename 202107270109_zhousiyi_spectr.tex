% 标量场的谱
% 标量场|粒子谱|场论

\pentry{标量场的量子化\upref{quanti}}
前面一节我们讲到如何对标量场进行量子化.这一节我们来看看如何写出标量场理论的谱.

现在我们来定义真空态.
\begin{definition}{真空态$|0\rangle$}
对于所有的$\mathbf p$,都有$a_{\mathbf p}|0\rangle=0$.如果我们不考虑真空能的话.这个态的能量为$E=0$.
\end{definition}
所有其它的态都可以通过在真空态$|0\rangle$前面加上产生算符来构建.一般来说,态$a^\dagger_{\mathbf p}a^\dagger_{\mathbf q}|0\rangle$是哈密顿量$H$的能量为$\omega_{\mathbf p}+\omega_{\mathbf q}+\cdots$的本征态.

由经典场论基础词条中的\autoref{classi_eq2}~\upref{classi}我们可以写出总动量算符
\begin{equation}
\mathbf P = -\int d^3 x \pi(\mathbf x) \nabla \phi(\mathbf x) = \int \frac{d^3p}{(2\pi)^3} \mathbf p\, a^\dagger_{\mathbf p} a_{\mathbf p}~. 
\end{equation}
算符$a_{\mathbf p}^\dagger$能够产生动量为$\mathbf p$,能量为$\omega_{\mathbf p} = \sqrt{|\mathbf p|^2+m^2}$的态.我们把这些激发态叫做粒子.

现在我们来看粒子的统计性质.我们来考虑一个二粒子态$a^\dagger_{\mathbf p}a^\dagger_{\mathbf q}| 0 \rangle$.因为$a^\dagger_{\mathbf p}$和$a^\dagger_{\mathbf q}$是对易的,这个态和$a^\dagger_{\mathbf q}a^\dagger_{\mathbf p}| 0 \rangle$是等价的.一个模式$\mathbf p$可以激发出任意多的粒子数.由此我们得出结论:克莱因-戈登粒子遵守玻色-爱因斯坦统计.

现在我们来定归一化条件$\langle 0 | 0 \rangle = 1$.单粒子态$|\mathbf p\rangle \propto a^\dagger_{\mathbf p} | 0 \rangle$的归一化条件$\langle \mathbf p| \mathbf q\rangle = (2\pi)^3\delta^{(3)}(\mathbf p - \mathbf q)$不是洛仑兹不变的.



