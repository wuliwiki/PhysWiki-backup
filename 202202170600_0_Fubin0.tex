% 重积分的换序、Fubini 定理(简明微积分)

\pentry{重积分\upref{IntN}}

\footnote{参考 Wikipedia 相关页面\href{https://en.wikipedia.org/wiki/Fubini's_theorem}{相关页面}.}以矩形区域的二重定积分为例, 什么时候可以交换积分的顺序呢? 即
\begin{equation}
\int_{y_1}^{y_2}\int_{x_1}^{x_2} f(x, y) \dd{x} \dd{y} = \int_{x_1}^{x_2}\int_{y_1}^{y_2} f(x, y) \dd{y}\dd{x}
\end{equation}
乍看之下, 二重积分是曲面下面的体积, 无论先算哪个都不会变. 但我们来看一个例子

\begin{example}{}
\begin{equation}\label{Fubin0_eq1}
f(x, y) = \frac{x - y}{(x+y)^3}
\end{equation}
\begin{figure}[ht]
\centering
\includegraphics[width=8cm]{./figures/Fubin0_1.png}
\caption{\autoref{Fubin0_eq1} 函数曲面图} \label{Fubin0_fig1}
\end{figure}
该函数在 $(0,0)$ 处有一个奇点, 即没有定义. 但是 $x \in (0, 1]$, $y \in (0, 1]$ 的重积分仍然收敛
\begin{equation}
\int_0^1 \int_0^1 f(x,y) \dd{x} \dd{y} = -\frac{1}{2}
\qquad
\int_0^1 \int_0^1 f(x,y) \dd{y} \dd{x} = \frac{1}{2}
\end{equation}
注意交换顺序以后发现结果竟然不同!

观察\autoref{Fubin0_fig1} 会发现, 函数曲面关于积分区域的正方形的对角线 $x=y$ 反对称, 且在 $x$ 轴和 $y$ 轴靠近原点的地方分别出现了无穷大和无穷小. 所以

\addTODO{函数关于 $y = x$ 反对称, 两个三角形, 积分分别为无穷大和无穷小, 延着斜线积分 $\dd{(x+y)}\dd{(x-y)}$ 等于零.}
\end{example}
