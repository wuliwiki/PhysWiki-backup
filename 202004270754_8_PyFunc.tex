% python函数
函数是组织好的,可重复使用的,用来实现相关功能的代码段.它能提高代码的重复利用率.Python提供了许多\textbf{内建函数},比如 \verb|print()|.我们也可以自己创建函数,这被叫做\textbf{自定义函数}. 需要在程序中多次执行同一项任务时, 你无需反复编写完成该任务的代码, 而只需调用执行该任务的函数, 让Python运行其中的代码. 你将发现, 通过使用函数,程序的编写、阅读、测试和修复都将更容易.

\subsection{函数的定义与调用}
我们通过一个简单的例子开始介绍:
\begin{lstlisting}[language=python]
def  func1():
    # 定义一个函数,函数名为func1
     print('hello python')
func1()
\end{lstlisting}
这个示例演示了最简单的函数结构. 第一行的代码行使用关键字\verb|def| 来告诉Python我们要定义一个函数.然后向Python指出了函数名,还可能在括号内指出函数为完成其任务需要什么样的信息. 在这里, 函数名为\verb|func1|, 它不需要任何信息就能完成其工作, 因此括号是空的(括号必不可少);最后,定义以\textbf{冒号}结尾.
紧跟在def greet_user(): 后面的所有\textbf{缩进行}构成了函数体. 第二行处的文本是\textbf{函数注释},以\verb|#|开始,通常用来描述了函数是做什么的. 文档字符串用三引号括起, Python使用它们来生成有关程序中函数的文档.
代码行print("Hello!") (见❸) 是函数体内的唯一一行代码, greet_user() 只做一项工作: 打印Hello! .
要使用这个函数, 可调用它. 函数调用 让Python执行函数的代码. 要调用 函数, 可依次指定函数名以及用括号括起的必要信息, 如❹处所示. 由于这个函数不需要任何信息, 因
此调用它时只需输入greet_user() 即可. 和预期的一样, 它打印Hello! :