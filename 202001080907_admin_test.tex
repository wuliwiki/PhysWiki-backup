% 角动量守恒
\begin{enumerate}
\item \footnote{Von Herrn J. Schur, Zur Theorie der vertauschbaren Matrizen, Journal f\"ur die Reine und Angewandte Mathematik, {\bf 130} (1905), 66-76.}设$\mathbb{F}^{n\times n}$的子空间$V$中任意两个方阵乘积可交换.证明:$\dim(V)\le 1+\dfrac{n^2}{4}$.//

对$n$归纳证明:若$A_1,\cdots,A_k\in V$线性无关,则$k\le 1+\frac{n^2}{4}$.由于$A_1,\cdots,A_k$在$\mathbb{F}$的扩域上也是线性无关的,故不妨设$\mathbb{F}$是代数封闭域.当$n=1$时,结论显然成立.下设$n\ge 2$.同\S5.2习题9,存在可逆方阵$P$使得$P^{-1}A_iP$都是上三角.设$P^{-1}A_iP=\begin{pmatrix}B_i&* \\ {\bf 0}&*\end{pmatrix}$,$\{B_1,\cdots,B_k\}$生成$\mathbb{F}^{(n-1)\times(n-1)}$的$s$维子空间,则存在$\alpha_j\in\mathbb{F}^{n\times 1}$使得$\left\{\begin{pmatrix}O&\alpha_j\end{pmatrix}\mid 1\le j\le k-s\right\}\subset V$线性无关.同理,设$P^{-1}A_iP=\begin{pmatrix}*&* \\ {\bf 0}&C_i\end{pmatrix}$,$\{C_1,\cdots,C_k\}$生成$\mathbb{F}^{(n-1)\times(n-1)}$的$t$维子空间,则存在$\beta_j\in\mathbb{F}^{1\times n}$使得$\left\{\begin{pmatrix}\beta_j \\ O\end{pmatrix}\mid 1\le j\le k-t\right\}\subset V$线性无关.注意到$\beta_j\alpha_i=0$,$\forall i,j$.由$(k-s)+(k-t)\le n$和归纳假设,$k\le\frac{s+t+n}{2}\le\left\lfloor 1+\frac{(n-1)^2}{4}+\frac{n}{2}\right\rfloor=\left\lfloor 1+\frac{n^2}{4}\right\rfloor$.

\item 设$\mathcal{A}\in L(V)$满足$\mathcal{A}^m=\mathcal{O}$,其中$m$是给定的正整数.$\mathcal{A}$称为{\bf 幂零变换}.证明:存在$V$的子空间$U$使得$V=\bigoplus\limits_{i=1}^m\mathcal{A}^{i-1}(U)$.

对$m$使用数学归纳法.当$m=1$时,结论显然成立.假设结论对$m-1$成立.设$B$是$A$在$\im A$上的限制映射,则$B^{m-1}=O$.因此,存在$\im A$的子空间$U_1$使得$\im A=\bigoplus\limits_{i=1}^{m-1}A^{i-1}(U_1)$.设$\{A(\alpha_i)\mid i\in I\}$是$U_1$的基,$U_0=\Span(\{\alpha_i\mid i\in I\})$,则$A(U_0)=U_1$,$U_0\cap\im A=\{{\bf 0}\}$,并且$V=\left(\bigoplus\limits_{i=1}^{m-1}A^{i-1}(U_0)\right)+\Ker A$.注意到$A^{m-1}(U_0)=A^{m-2}(U_1)\subset\Ker A$.设$\Ker A$的子空间$W_0$使得$V=\left(\bigoplus\limits_{i=1}^mA^{i-1}(U_0)\right)\oplus W_0$,则$U=U_0\oplus W_0$使得$V=\bigoplus\limits_{i=1}^mA^{i-1}(U)$.
\end{enumerate}

