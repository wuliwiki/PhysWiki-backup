% 李群的指数映射
% license Usr
% type Tutor

%预备知识需要添加.
\pentry{李群的李代数\nref{nod_LieGA},流\nref{nod_flow}}{nod_016b}
(本文默认左不变切场都是切空间的切向量经左平移映射延拓得到)
\begin{definition}{}
设$\mathfrak g$是李群$G$的李代数。定义指数映射$\exp :\mathfrak g\to G$,对于任意$X_e\in \mathfrak g$ 有$\exp (X_e)=c_X(1)$。其中$c_X$正是左不变切场$X$的积分流。
\end{definition}
在物理上,我们经常要用到矩阵李群。可以证明,对于矩阵李群,指数映射恰为矩阵的指数函数,可说是名副其实了。

\begin{theorem}{指数映射的性质}
\begin{enumerate}
\item 对于任意$X_e\in \mathfrak g$,$X$的开始于$e$的积分曲线为$c_X(t)\equiv\theta_t(e)=\exp (tX_e)$。
\item 对于任意$X_e\in \mathfrak g$,左不变切场$X$开始于$g$的积分曲线为$g\exp (tX_e)$。
\item 对于任意$x,t\in\mathbb R,X_e\mathfrak g$,指数映射是$\mathbb R\to G$的群同态,满足$\exp ((s+t)X_e)=(\exp sX_e)(\exp tX_e)$。

\item 指数映射是光滑的。
\item 指数映射在$t=0$处的切映射是单位映射。
\item 对于一般线性群$GL(n,\mathbb R)$,有
\begin{equation}
\exp  A=\sum_{k=0}^\infty\frac{A^k}{k!},\quad \forall A\in\mathfrak{gl}(n,\mathbb{R})~.
\end{equation}
\end{enumerate}
\end{theorem}

\textbf{证明:}

(1)设任意$t\in \mathbb R$,$s$是可变参量,$c_X(s)$为$X$的开始于$e$的积分曲线$\theta_t(e)$。定义$\tau(s)=c_X(ts)$。则
\begin{equation}
\dv{c_X(ts)}{s}=\tau'(s)=tc'_X(ts)=tX_{\tau (s)}~.
\end{equation}
因为$\tau(0)=e$并且每一处的切向量场都是$tX$,由极大积分曲线的唯一性得$\tau(s)=c_{tX}(s)=c_X(ts)$。代入$s=1$得$c_{tX}(1)=\exp(tX_e)=c_X(t)$。

(2)由(1)知,我们需要证明对于任意$X_e\in \mathfrak g$,左不变切场$X$开始于$g$的积分曲线为$gc(t)$。
因为
\begin{equation}
\begin{aligned}
\dv{(gc(t))}{t}&=\dv{L_gc(t)}{t}\\
&=L_{g*}\dv{c(t)}{t}\\
&=X_{gc(t)}~.
\end{aligned}
\end{equation}

因此$gc(t)$确实是$X$开始于$g$的积分曲线。


(3)由题意知,我们需要证明$c(s+t)=c(s)c(t)$。依然设$s$为变化参量,$t$为任意固定的常数,$c(t)$为$X$开始于$e$的积分曲线,问题转化为需证等式两端是相同的积分曲线。
因为
\begin{equation}
\dv{c(s+t)}{s}=X_{c(s+t)}~,
\end{equation}
所以$c(s+t)$为$X$开始于$c(s)$的积分曲线。由上题知,$c(s)c(t)$亦是如此。由极大积分曲线的唯一性知这是同一条积分曲线,所以得证。

(4)

(5)

(6)只要证明$c(t)=\sum^{\infty}_{k=0}(tA)^k/(k!)$确实是左不变切场开始于$e$的积分曲线即可。易见$c(0)=I=e$