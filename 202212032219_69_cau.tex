% 因果律
% 因果律|标量场|场论

我们现在仍然考虑海森堡绘景.考虑一个粒子从 $y$ 传到 $x$,幅度是 $\langle 0 |\phi(x)\phi(y)| 0 \rangle$.我们把这个量叫做 $D(x-y)$.因为 $\phi$ 可以展开为 $a$ 和 $a^\dagger$ 的和.所以这个 $D(x-y)$ 总共有四项,但是只有 $\langle 0 | a_{\mathbf p} a^\dagger_{\mathbf q} | 0 \rangle = (2\pi)^3 \delta^{(3)}(\mathbf p - \mathbf q)$ 能够保留下来.所以我们可以得出
\begin{equation}
D(x-y)=\langle 0|\phi(x) \phi(y)| 0\rangle=\int \frac{d^{3} p}{(2 \pi)^{3}} \frac{1}{2 E_{\mathbf{p}}} e^{-i p \cdot(x-y)}~.
\end{equation}
首先我们来看看时空间隔是类时的情况.$x^0 - y^0 = t$, $\mathbf x - \mathbf y = 0$. 我们可以推出
\begin{equation}
\begin{aligned}
D(x-y) &=\frac{4 \pi}{(2 \pi)^{3}} \int_{0}^{\infty} d p \frac{p^{2}}{2 \sqrt{p^{2}+m^{2}}} e^{-i \sqrt{p^{2}+m^{2}} t} \\
&=\frac{1}{4 \pi^{2}} \int_{m}^{\infty} d E \sqrt{E^{2}-m^{2}} e^{-i E t} \\
& \sim e^{-i m t} .
\end{aligned}
\end{equation}
接着我们可以考虑时空间隔是类空的情况 $x^0 - y^0 = 0$, $\mathbf x - \mathbf y = \mathbf r$.
\begin{equation}\label{cau_eq1}
\begin{aligned}
D(x-y) &=\int \frac{d^{3} p}{(2 \pi)^{3}} \frac{1}{2 E_{\mathbf{p}}} e^{i \mathbf{p} \cdot \mathbf{r}} \\
&=\frac{2 \pi}{(2 \pi)^{3}} \int_{0}^{\infty} d p \frac{p^{2}}{2 E_{\mathbf{p}}} \frac{e^{i p r}-e^{-i p r}}{i p r} \\
&=\frac{-i}{2(2 \pi)^{2} r} \int_{-\infty}^{\infty} d p \frac{p e^{i p r}}{\sqrt{p^{2}+m^{2}}}
\end{aligned}
\end{equation}
\begin{figure}[ht]
\centering
\includegraphics[width=14.25cm]{./figures/cau_1.png}
\caption{积分围道的变化.} \label{cau_fig1}
\end{figure}
\autoref{cau_eq1} 的branch cut如\autoref{cau_fig1} 所示,两个branch point在 $im$ 和 $-im$ 处.定义 $\rho = -ip$,我们得到
\begin{equation}
\frac{1}{4 \pi^{2} r} \int_{m}^{\infty} d \rho \frac{\rho e^{-\rho r}}{\sqrt{\rho^{2}-m^{2}}} \underset{r \rightarrow \infty}{\sim} e^{-m r}
\end{equation}
这个式子说明,在光锥外面,传播幅度是指数压低并且是非零的.

那当我们看一个理论是否满足因果律的时候,我们不应该看粒子能不能在类空间隔传播,而是应该看我们在一个点进行的测量是否会影响另一个与这个点是类空间隔的点.那我们可以考虑最简单的对易子 $[\phi(x),\phi(y)]$.如果这个对易子是零的话,一个测量就不能影响到另一个测量.我们可以做一下这样的计算
\begin{equation}
\begin{aligned}
&[\phi(x), \phi(y)]=\int \frac{d^{3} p}{(2 \pi)^{3}} \frac{1}{\sqrt{2 E_{\mathbf{p}}}} \int \frac{d^{3} q}{(2 \pi)^{3}} \frac{1}{\sqrt{2 E_{\mathbf{q}}}} \\
&\times\left[\left(a_{\mathbf{p}} e^{-i p \cdot x}+a_{\mathbf{p}}^{\dagger} e^{i p \cdot x}\right),\left(a_{\mathbf{q}} e^{-i q \cdot y}+a_{\mathbf{q}}^{\dagger} e^{i q \cdot y}\right)\right] \\
=& \int \frac{d^{3} p}{(2 \pi)^{3}} \frac{1}{2 E_{\mathbf{p}}}\left(e^{-i p \cdot(x-y)}-e^{i p \cdot(x-y)}\right) \\
=& D(x-y)-D(y-x) .
\end{aligned}
\end{equation}
当 $(x-y)^2<0$ 时,我们可以对第二项进行一个洛仑兹变换,把 $(x-y)\rightarrow -(x-y)$. 这样的话第一项和第二项就抵消了,得0.如果 $(x-y)^2>0$,我们找不到连续的洛仑兹变换使得 $(x-y)\rightarrow - (x-y)$.这时候如果 $\mathbf x - \mathbf y = 0$,幅度大约是 $(e^{-imt}-e^{imt})$.这样的讨论还可以延伸到复标量场.所以因果律要求,每一个粒子都有与其相对应的反粒子.反粒子与正粒子质量相同,量子数相反.



