% 热力学第一定律
% keys 热力学第一定律|能量守恒|做功|传热|内能

\begin{issues}
\issueDraft
\end{issues}

\pentry{压力体积图\upref{PVgraf}, 理想气体内能\upref{IdgEng}}

\textbf{热力学第一定律}是\textbf{能量守恒}在热力学中的形式. \textbf{热量}是指当热学系统出现温度差时引起的能量转移. 热力学第一定律表明, 外部对系统传递的热量 $Q$ 等于系统对外做功 $W$ 加上系统的\textbf{内能} $E$ 增加:
\begin{equation}\label{Th1Law_eq1}
\Delta Q = W + \Delta E
\end{equation}
系统的内能有时也用 $U$ 表示. 热力学第一定律的另一种表述是:\textbf{第一类永动机}是不可能造成的.
\addTODO{什么是第一类永动机?链接到永动机词条}

虽然功和热都是能量的转移, 但热是通过微观粒子的无规则相互作用传递的, 而功\upref{Fwork}在热力学中特指宏观的作用力和宏观位移产生的. 对于容器中的气体, 把系统的压强记为 $P$, 体积记为 $V$, 对外做功可以写成积分形式:
\begin{equation}
W = \int_{V_1}^{V_2} P \dd{V}
\end{equation}
热力学第一定律写成微分形式是
\begin{equation}\label{Th1Law_eq2}
\dd Q = \dd W + \dd E = P\dd V + \dd E
\end{equation}
注意 $Q$ 和 $W$ 和系统变化的过程本身有关: 也就是说即使系统的初末状态确定, 中间的过程不一样也会导致它们的不同. 这样的量被称为\textbf{过程量}\upref{StaPro}. $E$ 前面用的是全微分符号,是因为内能 $E$ 之和系统的状态有关, 被称为\textbf{状态量}\upref{StaPro}. 所以 $\Delta E$ 也只与系统的初始和最终的状态有关, 与中间的过程无关.

\addTODO{写一些理想气体的例题, 例如 PV 图种, 计算两点间延着不同轨迹的热量}


对理想气体\upref{Igas}, 令分子自由度为 $i$, 有
\begin{equation}
E = \frac{i}{2}n RT
\end{equation}
