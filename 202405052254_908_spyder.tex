% Spyder 笔记
% license Xiao
% type Note

\begin{issues}
\issueDraft
\end{issues}

\begin{itemize}
\item \href{https://docs.spyder-ide.org/current/}{文档}
\item Spyder 是专为科学和工程计算以及数据分析设计的 IDE。
\item 在 \enref{conda}{CondaN} 的 anaconda-navigator 中可以一键安装 spyder
\item editor 中 \verb|#%%| 可以用于分割 cell, 每个 cell 和 \enref{jupyter notebook}{jupyNb} 中的一样, 可以独立在 IPython 命令行中执行。
\end{itemize}

\subsection{待整理}
\begin{itemize}
\item clear() 可以清空 IPython console
\item 可以选择代码的一部分然后单击右键执行 (快捷键F9)
\item 选中代码,用tab键可以缩进, shift + tab 可以删除缩进, cmd+1 可以标注或取消标注
\item edit 中的文件都是通过 console 和对应的 kernal 运行的,必须要选中一个 console 才可以用 F5 运行脚本
\item 尽量使用ipython,不使用convenience import (但从目前来看,ipython不知道为什么不能3D画图, 只好用 python)
\item ipython中,%reset可以清空namespace
\item 写代码要遵守PEP8规则
\item 设置PEP8 提示
  Go to Preferences -> Editor -> Code Introspection/Analysis and select the tickbox next to Style analysis (PEP8)
\item convenience import是在打开console以后自动运行一段脚本,设置如下(一般建议关闭)
   Preferences -> Console -> Advanced Settings -> PYTHONSTARTUP replacement
   关闭IPython中的同样设置
   Preferences -> IPython console -> Graphics 然后关闭 Activate Support
\item 打开Ipython的符号模式
   Preferences -> IPython console -> Advanced Settings -> Use symbolic math
![](Spyder1.png)

\item F5 运行当前文件
\item 可以用 \verb|#%%| 来隔离cell! 然后分块执行. cell就是jupyter里面的cell. 执行快捷键是 shift+<return>
\item Alt+上下箭头可以把选中的内容上下移动
\item 把任何 Object 作为输入放到help()里面都会得到帮助
\end{itemize}
