% 三维欧几里得空间中的曲线
% keys 曲线|微分几何|曲率|curvature|扭率|第二曲率|torsion

三维欧几里得空间是我们最为熟悉的空间,其性质也非常好,易于研究.本节讨论的是三维空间中的曲线的性质,引入曲率、挠率等概念.

\subsection{一般欧几里得空间中曲线的概念}

\begin{definition}{曲线}
令$I$是实数轴$\mathbb{R}$上的一个区间,则称\textbf{连续函数}$f:I\to \mathbb{R}^n$为$\mathbb{R}^n$中的一条\textbf{曲线(curve)},按拓扑学的习惯也称\textbf{道路(path)}.此处连续是指函数的$n$个分量都是$\mathbb{R}\to\mathbb{R}$的连续函数.
\end{definition}

我们可以任意取定一个坐标系,把向量值函数$f$分为$n$个标量值函数,简称为$f$的分量,由此来理解连续的含义.你可能自然会想确认,$f$的分量的连续性,和取定坐标系的方式是否有关?答案是无关的.这是因为我们可以用另一种方式来理解此处的“连续”,那就是取集合$I$与$\mathbb{R}^3$,配上通常的拓扑——即$I$取$\mathbb{R}$的子拓扑,$\mathbb{R}^3$取$\mathbb{R}$的乘积拓扑——所得到的拓扑空间,那么$f$就是拓扑空间之间的映射,其连续性取决于拓扑意义上的连续性,和具体坐标系的选择就无关了.

要强调的一点是,即便两条曲线的轨迹一样,它们也不一定是同一条曲线.比如说,取$f, g:\mathbb{R}\to\mathbb{R}^2$这两个函数,定义为$f(t)=\pmat{\cos t\\\sin t}$和$g(t)=\pmat{\cos 2t\\\sin 2t}$,那么尽管它们的轨迹都是平面上的单位圆,但由于两条曲线的“速度”不一样,我们依然把它们认为是不同的曲线.

\begin{definition}{可微曲线}
如果曲线$f:I\to\mathbb{R}^n$对于任意分量都是可微的,那么称$f$是一个\textbf{可微曲线(differentiable curve)}或\textbf{可微道路(differentiable path)},有时也记为$C^1$的曲线.
\end{definition}

同样地,$f$本身的可微性,和具体坐标系的选择有关系吗?由于“微分”这一概念是实数空间特有的,我们没法像前面一样直接用拓扑的概念绕过坐标系的选择;但答案是一样的,\textbf{无关}.证明这一点的思路也可以应用到之前对一般曲线的讨论上去:考虑变换坐标系后各点的变换,会发现变换后的分量就是过渡矩阵乘以原先的函数列矩阵,也就是说,变换坐标后新的函数分量,是旧分量的某种线性组合.这样一来,变换前可微当且仅当变换后也可微,从而可知和变换无关.

\begin{definition}{光滑曲线}
如果曲线$f:I\to\mathbb{R}^n$对于任意分量都是光滑的,那么称$f$是一个\textbf{光滑曲线(smooth curve)}或\textbf{光滑道路(smooth path)},有时也记为$C^\infty$的曲线.
\end{definition}

和上述坐标变换的讨论类似,可证明光滑曲线的光滑性不依赖于坐标系的选择.

\begin{definition}{正则曲线}

\end{definition}

\subsection{Frenet向量组}

Frenet向量组是\textbf{三维}欧几里得空间$\mathbb{R}^3$中特有的概念,是刻画\textbf{可微曲线}性质的优良工具.

Frenet向量组中的第一个向量,是\textbf{单位切向量}.关于$t$对曲线$f(t)$求导就能得到切向量,如果把$t$看作时间,那么切向量$\frac{\dd}{\dd t}f(t)$也可以看成是曲线上一点的切速度.单位切向量就是方向与切速度相同,但长度为$1$的向量.

为了得到单位切向量,我们可以用切速度除以切速度的大小来获得:$\frac{\dd}{\dd t}f(t)/\abs{\frac{\dd}{\dd t}f(t)}$.但更简单的方法是,令切速度的大小恒为$1$,或者说只讨论切速度大小恒为$1$的\textbf{可微曲线}.当切速度恒为$1$的时候,曲线走过的弧长就恒等于经过的时间,于是我们说这时是用\textbf{弧长}来作曲线的参数.弧长参数通常用字母$s$表示.

\begin{definition}{单位切向量}\label{Curv3D_def1}
使用弧长参数来描述可微曲线$f(s)$,记$T(s)=\frac{\dd}{\dd s}f(s)$,称为曲线在参数为$s$处的\textbf{单位切向量(unit tangent vector)}.
\end{definition}

Frenet向量组中的第二个向量,是\textbf{单位主法向量}.对于处处可微的曲线,切向量总是存在的,但是主法向量不一定存在.主法向量的定义如下:

\begin{definition}{主法向量}\label{Curv3D_def2}
使用弧长参数来描述可微曲线$f(s)$,其单位切向量为$T(s)$.则称$\kappa(s) N(s)=\frac{\dd}{\dd s}T(s)$为曲线在$s$处的\textbf{主法向量(normal vector)}.其中$N(s)$是单位\textbf{向量},$\kappa$是一个\textbf{正数},称作曲线在这一点处的\textbf{曲率(curvature)}.
\end{definition}

由于使用弧长参数以后,单位切向量的大小不变,因此$N(s)$恒与$T(s)$垂直.$N(s)$就是那第二个成员,单位主法向量.

Frenet向量组中的最后一个成员,是\textbf{单位副法向量},是用前两个向量叉乘而来的.

\begin{definition}{副法向量}
条件设定如\autoref{Curv3D_def1} 和\autoref{Curv3D_def2} ,则称$B(s)=T(s)\times N(s)$
\end{definition}






