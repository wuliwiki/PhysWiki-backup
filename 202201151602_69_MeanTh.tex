% 微分中值定理
% keys 拉格朗日中值定理|柯西中值定理|罗尔中值定理|mean value theorem

\pentry{导数\upref{Der}}

\subsection{一点准备}

\begin{definition}{函数极值}
考虑实函数 $f(x)$.如果\textbf{存在}一个实数轴上的开集 $O$,且有 $x_0\in O$,使得对于任意的 $x\in O$,都有 $f(x_0)\geq f(x)$,则称 $f(x_0)$ 是 $f$ 在 $O$ 上的一个\textbf{极大值(maximum)};如果 $x_0$ 满足的条件改为对于任意的 $x\in O$,都有 $f(x_0)\leq f(x)$,则称 $f(x_0)$ 是 $f$ 在 $O$ 上的一个\textbf{极小值(minimum)}.

极大值和极小值统称为\textbf{极值(extremum)}.

如果 $f(x_0)$ 是一个极大值,那么称 $x_0$ 是一个\textbf{极大值点(maximum point)};相应地,极小值对应的自变量 $x_0$ 是一个\textbf{极小值点(minimum point)}.极大值点和极小值点统称\textbf{极值点(extremum point)}.
\end{definition}


简单来说,极大值的意思就是,取包含极大值点的足够小的范围,那么范围内的所有函数值都小于等于极大值.极小值则反过来,范围内的函数值都大于等于它.

我们要求“\textbf{存在}一个开集 $O$”,实际上就是在说存在一个范围.

\begin{example}{}\label{MeanTh_ex1}
考虑实函数 $f(x)=x^3-x$,如\autoref{MeanTh_fig1} 所示.

\begin{figure}[ht]
\centering
\includegraphics[width=8cm]{./figures/MeanTh_1.pdf}
\caption{$f(x)=x^3-x$ 的函数图像.} \label{MeanTh_fig1}
\end{figure}

在 $x=\frac{\sqrt{3}}{3}$ 处,$f$ 取极小值,但显然不是最小值.

在 $x=-\frac{\sqrt{3}}{3}$ 处,$f$ 取极大值,也显然不是最大值.


\end{example}


\begin{theorem}{Fermat定理}\label{MeanTh_the1}
考虑实函数 $f(x)$.如果 $x_0$ 是 $f$ 的一个极值点,且 $f(x)$ 在 $x_0$ 处可导,那么 $f'(x_0)=0$.
\end{theorem}

\textbf{证明}:

假设 $f(x)$ 在 $x_0$ 处可导且取极大值,并\textbf{反设}$f'(x_0)>0$\footnote{取极小值和/或反设 $f'(x_0)<0$ 的情况可以类比,在此不赘述.反设就是指“反过来假设定理不成立”.}.

那么由于可导,$f(x)$ 在 $x_0$ 处的右极限存在且等于导数,即右极限大于零.这样一来,取任意正向接近 $x_0$ 的数列 $\{a_n\}$,则对于任意正整数 $N$,必然总有编号大于 $N$ 的 $a_n$ 使得 $f(a_n)>f(x_0)$\footnote{否则,如果存在一个正整数 $N$ 使得所有编号大于 $N$ 的 $a_n$ 都小于等于 $f(x_0)$,则这些 $a_n$ 计算出的割线斜率 $\frac{f(a_n)-f(x_0)}{a_n-x_0}$ 就小于等于零了,取极限以后,可得 $f'(x_0)\leq 0$,而这和我们反设的“$f'(x_0)<0$”相矛盾.}.又由于 $\lim\limits_{n\to\infty}a_n=0$,故得,对于任意包含 $x_0$ 的开集(范围)$O$(对应编号 $N$),总有 $f(a_n)>f(x_0)$,于是 $f(x_0)$ 就不是极大值了.

结论和假设矛盾,故反设部分不成立.将以上讨论推广到 $f(x_0)$ 取极小值和/或反设 $f'(x_0)<0$ 的情况后,可得最终结论:$f'(x_0)$ 必为零.

\textbf{证毕}.

\autoref{MeanTh_the1} 就可以用来快速计算出\autoref{MeanTh_ex1} 里的两个极值点的位置,即导数为零的地方.


\subsection{三个中值定理}

%缺例图

\begin{definition}{Rolle中值定理}
设 $f(x)$ 在区间 $[a, b]$ 上连续,在 $(a, b)$ 内可导,且 $f(a)=f(b)$,那么存在一个 $x_0\in(a, b)$,使得 $f'(x_0)=0$.
\end{definition}

罗尔微分中值定理可以利用\textbf{费马定理}\autoref{MeanTh_the1} 证明,即对最大值或最小值点处的导数进行分析.如果将函数 $f(x)$ 叠加上一个一次函数 $kx+b$,即满足 $f(b)-f(a)=k(b-a)$,那么就一定存在$a< \xi <b$使得 $f'(\xi)=k$.于是可以得到以下定理

\begin{theorem}{拉格朗日微分中值定理}
若函数 $f(x)$ 在 $[a,b]$ 上连续,在 $(a,b)$ 内可导,则一定存在 $a< \xi <b$,使得
\begin{equation}
f'(\xi)=\frac{f(b)-f(a)}{b-a}
\end{equation}
\end{theorem}

\begin{exercise}{}
证明勒让德(Legendre)多项式\footnote{这个著名的函数出现在球谐函数中,在学电动力学和量子力学时将会多次用到.}

\begin{equation}
P_n(x)=\frac{1}{2^n n!}\frac{\dd{} ^n}{\dd x^n}[(x^2-1)^n]
\end{equation}

在 $(-1,1)$ 内有 $n$ 个互异的实根.

\textbf{提示:}考察多项式函数 $f(x)=(x^2-1)^n$,它在 $x=\pm 1$ 处分别有 $n$ 重根.如果对它求一次导,由罗尔微分中值定理,$f'(x)$ 在 $(-1,1)$ 内至少有一个根,在 $x=\pm 1$ 处分别有 $n-1$ 重根.根据多项式函数因式分解的性质,$f'(x)$ 在 $(-1,1)$ 上只有一个根.再对它求导,…….以此类推,$f^{(n)}(x)$ 在 $(-1,1)$ 内恰好有 $n$ 个互异的根,在 $x=\pm 1$ 不再是 $f^{(n)}(x)$ 的根.
\end{exercise}
由上例可见,罗尔微分中值定理可以用来分析特殊函数的零点情况.有时我们还会通过构造辅助函数来分析零点.例如若连续函数 $f(x)$ 有两个零点 $x_1,x_2$,那么对任意 $a\in \mathbb{R}$,一定存在 $x_1<\xi<x_2$ 满足 $f'(\xi)-af(\xi)=0$.这可以通过构造函数 $g(x)=f(x)e^{-ax}$ 再利用罗尔微分中值定理即可.

\begin{exercise}{}
$\alpha$ 是大于 $0$ 的一个常数.函数 $f(x)=x^\alpha$ 在 $[0,\infty)$ 上是否一致连续(对 $\alpha$ 分类讨论)?

\textbf{提示:}回顾一致连续的定义,对任意 $\epsilon>0$,总存在 $\delta>0$,使得任取 $|x-y|<\delta(0\le x<y)$,都有 $|f(x)-f(y)|<\epsilon$.可以利用微分中值定理,总是存在 $x< \xi <y$,使得 $|f(x)-f(y)|=f'(\xi)|x-y|<\delta f'(\xi)$ ($f(x)$ 是单调递增函数).这样一来我们就将一致连续性与导数联系了起来.要注意的是,当 $\alpha<1$ 时 $f(x)$ 在 $0$ 附近导数趋于无穷大,所以要对 $0$ 附近的邻域单独拎出来讨论(例如取区间 $[0,2]$,闭区间上的连续函数一定一致连续).当 $\alpha\le1$ 时,邻域以外的部分 $f'(\xi)$ 将有上界,因此函数 $f(x)$ 一致连续.当 $\alpha>1$ 时,$f'(x)$ 随 $x$ 的增加而单调增加,趋于正无穷,于是有 $|f(x+\delta)-f(x)|=\delta f'(\xi) \ge\delta f'(x)$,因此 $f(x)$ 不一直连续.
\end{exercise}

\begin{theorem}{柯西微分中值定理}
若函数 $f(x)$ 和 $g(x)$ 在 $[a,b]$ 上连续,在 $(a,b)$ 内可导,而且 $g'(x)\neq 0$,则在 $(a,b)$ 内至少存在一点 $\xi$,使得
\begin{equation}{}
\frac{f'(\xi)}{g'(\xi)}=\frac{f(b)-f(a)}{g(b)-g(a)}
\end{equation}
\end{theorem}
\begin{exercise}{}
函数 $f(x)$ 在 $(0,\infty)$ 上连续,证明对任意 $0 < a < b$,总是存在 $a < \xi < b, a< \psi <b$,使得
\begin{equation}
\frac{2f'(\xi)}{\xi^2}(a^2+ab+b^2)=\frac{3f'(\psi)}{\psi}(a+b)
\end{equation}
\textbf{提示:}上式经过变形可以写成
\begin{equation}
\frac{f'(\xi)}{3\xi^2}(a^3-b^3)=\frac{f'(\psi)}{2\psi}(a^2-b^2)
\end{equation}
这提示我们利用柯西微分中值定理,总是存在 $a<\xi,\psi<b$,使得
\begin{equation}
\begin{aligned}
\frac{f(a)-f(b)}{a^3-b^3}=\frac{f'(\xi)}{3\xi^2}\\
\frac{f(a)-f(b)}{a^2-b^2}=\frac{f'(\psi)}{2\psi}
\end{aligned}
\end{equation}
\end{exercise}