% 2017 年计算机学科专业基础综合全国联考卷
% keys 2017 年计算机 全国 联考卷


\subsection{一、单项选择题}
1~40小题,每小题2分,共80分.下列每题给出的四个选项中,只有一个选项符合题目要求.

1.下列函数的时间复杂度是 \\
\begin{lstlisting}[language=cpp]
int func(int n)
{
    int i=0,sum=0;
    while(sum<n)
        sum+=++i;
    return i;
}
\end{lstlisting}
A. $O(log n)$ $\quad$ B.$O(n^{l/2})$  $\quad$  C.$O(n)$  $\quad$  D.$O(nlogn)$

2.下列关于栈的叙述中,错误的是 \\
I.采用非递归方式重写递归程序时必须使用栈 \\
II.函数调用时,系统要用栈保存必要的信息 \\
III.只要确定了入栈次序,即可确定出栈次序 \\
Ⅳ,栈是一种受限的线性表,允许在其两端进行操作 \\
A.仅I  $\quad$  B.仅I、II、III \\
C.仅I、Ⅲ、Ⅳ  $\quad$  D.仅II、III、Ⅳ

3.适用于压缩存储稀疏矩阵的两种存储结构是 \\
A.三元组表和十字链表 $\quad$ B.三元组表和邻接矩阵 \\
C.十字链表和二叉链表 $\quad$ D.邻接矩阵和十字链表

4.要使一棵非空二叉树的先序序列与中序序列相同,其所有非叶结点须满足的条件是 \\
A.只有左子树  $\quad$  B.只有右子树 \\
C.结点的度均为1 $\quad$ D.结点的度均为2

5.己知一棵二叉树的树形如下图所示,其后序序列为e,a,c,b.d,g,f,树中与结点a同层的结点是 \\
\begin{figure}[ht]
\centering
\includegraphics[width=5cm]{./figures/CSN17_1.png}
\caption{第5题图} \label{CSN17_fig1}
\end{figure}
A.C $\quad$ B.d $\quad$ C.f $\quad$ D.g

6.己知字符集{a,b,c,d,e,f,g,h},若各字符的哈夫曼编码依次是0100, 10, 0000, 0101, 001, 011, 11, 0001,则编码序列010001 100100101 1 1 10101的译码结果是 \\
A. acgabfh  $\quad$  B. adbagbb  \\
C. afbeagd  $\quad$  D. afeefgd

7.己知无向图G含有16条边,其中度为4的顶点个数为3,度为3的顶点个数为4,其他顶点的度均小于3.图G所含的顶点个数至少是 \\
A.10  $\quad$  B.11  $\quad$  C.13  $\quad$  D.15

8.下列二叉树中,可能成为折半查找判定树f不含外部结点1的是 \\
\begin{figure}[ht]
\centering
\includegraphics[width=14.25cm]{./figures/CSN17_2.png}
\caption{第8题图} \label{CSN17_fig2}
\end{figure}

9.下列应用中,适合使用B+树的是 \\
A.编译器中的词法分析 $\quad$ B.关系数据库系统中的索引 \\
C.网络中的路由表快速查找  $\quad$  D.操作系统的磁盘空闲块管理

10.在内部排序时,若选择了归并排序而没有选择插入排序,则可能的理由是 \\
I.归并排序的程序代码更短 \\
II.归并排序的占用空间更少 \\
III.归并排序的运行效率更高 \\
A.仅II  $\quad$  B.仅III  $\quad$  C.仅I、II  $\quad$  D.仅I、III

11.下列排序方法中,若将顺序存储更换为链式存储,则算法的时间效率会降低的是 \\
I.插入排序    II.选择排序  Ⅲ,起泡排序 \\
Ⅳ.希尔排序    V.堆排序 \\
A.仅I、II $\quad$ B.仅II、III  $\quad$  C.仅Ⅲ、Ⅳ  $\quad$  D.仅Ⅳ、V

12.假定计算机Ml和M2具有相同的指令集体系结构(ISA),主频分别为1.5 GHz和1.2 GHz.在Ml和M2上运行某基准程序P,平均CPI分别为2和1,则程序P在Ml和M2上运行时问的比值是 \\
A. 0.4  $\quad$  B. 0.625  $\quad$  C. 1.6  $\quad$  D. 2.5

13.某计算机主存按字节编址,由4个64Mx8位的DRAM芯片采用交叉编址方式构成,并与宽度为32位的存储器总线相连,主存每次最多读写32位数据.若double型变量x的主存地址为804001AH,则读取x需要的存储周期数是 \\
A.1  $\quad$  B.2  $\quad$  C.3  $\quad$  D.4

14.某C语言程序段如下: \\
\begin{lstlisting}[language=cpp]
for(i=0; i<=9; i++)
{
    temp=1;
    for(j=0; j<=i; j++)
        temp*=a[j];
    sum+=temp;
}
\end{lstlisting}
下列关于数组a的访问局部性的描述中,正确的是 \\
A.时间局部性和空间局部性皆有 \\
B.无时间局部性,有空间局部性 \\
C.有时间局部性,无空间局部性 \\
D.时间局部性和空间局部性皆无

15.下列寻址方式中,最适合按下标顺序访问一维数组元素的是 \\
A.相对寻址 $\quad$ B.寄存器寻址 $\quad$ C.直接寻址 $\quad$ D.变址寻址

16.某计算机按字节编址,指令字长固定且只有两种指令格式,其中三地址指令29条,二地址指令107条,每个地址字段为6位,则指令字长至少应该是 \\
A. 24位  $\quad$  B. 26位  $\quad$  C. 28位  $\quad$  D. 32位

17.下列关于超标量流水线特性的叙述中,正确的是 \\
I.能缩短流水线功能段的处理时间 \\
II.能在一个时钟周期内同时发射多条指令 \\
III.能结合动态调度技术提高指令执行并行性 \\
A.仅II  $\quad$  B.仅I、III  $\quad$  C.仅Ⅱ、Ⅲ  $\quad$  D.  I、II和III

18.下列关于主存储器(MM)和控制存储器(CS)的叙述中,错误的是 \\
A. MM在CPU外,CS在CPU内 \\
B. MM按地址访问,CS按内容访问 \\
C. MM存储指令和数据,CS存储微指令 \\
D. MM用RAM和ROM实现,CS用ROM实现

19.下列关于指令流水线数据通路的叙述中,错误的是 \\
A.包含生成控制信号的控制部件 \\
B.包含算术逻辑运算部件(ALU) \\
C.包含通用寄存器组和取指部件 \\
D.白组合逻辑电路和时序逻辑电路组合而成

20.下列关于多总线结构的叙述中,错误的是 \\
A.靠近CPU的总线速度较快 \\
B.存储器总线可支持突发传送方式 \\
C.总线之间须通过桥接器相连 \\
D.  PC I - Expressx16采用并行传输方式

21. 110指令实现的数据传送通常发生在 \\
A. I/O设备和I/O端口之间  $\quad$  B.通用寄存器和I/O设备之间 \\
C. I/O端口和I/O端口之间  $\quad$  D.通用寄存器和I/O端口之间

22.下列关于多重中断系统的叙述中,错误的是 \\
A.在一条指令执行结束时响应中断 \\
B.中断处理期间CPU处于关中断状态 \\
C.中断请求的产生与当前指令的执行无关 \\
D.CPU通过采样中断请求信号检测中断请求 \\

23.假设4个作业到达系统的时刻和运行时间如下表所示.
\begin{table}[ht]
\centering
\caption{第23题图}\label{CSN17_tab1}
\begin{tabular}{|c|c|c|}
\hline
作业 & 到达时刻 & 运行时间 \\
\hline
J1 & 0 & 3 \\
\hline
J2 & 1 & 3 \\
\hline
J3 & 1 & 2 \\
\hline
J4 & 3 & 1 \\
\hline
\end{tabular}
\end{table}
系统在t=2时开始作业调度.若分别采用先来先服务和短作业优先调度算法,则选中的作业分别是 \\
A. J2. J3  $\quad$  B. Jl、J4  $\quad$  C. J2. J4  $\quad$  D. Jl. J3

24.执行系统调用的过程包括如下主要操作: \\
①返回用户态  $\quad$  ②执行陷入(trap)指令 \\
⑧传递系统调用参数  $\quad$ ④执行相应的服务程序 \\
正确的执行顺序是 \\
A.②一③一①一④ $\quad$ B.②一④一⑧j① \\
C.③一②一④一① $\quad$ D.⑧一④一②一①

25.某计算机按字节编址,其动态分区内存管理采用最佳适应算法,每次分配和回收内存后都对空闲分区链重新排序.当前空闲分区信息如下表所示. \\
\begin{table}[ht]
\centering
\caption{第25题图}\label{CSN17_tab2}
\begin{tabular}{|c|c|c|c|c|}
\hline
分区起始地址 & 20K & 500K & 1000K & 200K \\
\hline
分区大小 & 40KB & 80KB & 100KB & 200KB \\
\hline
\end{tabular}
\end{table}
回收起始地址为60 K、大小为140 KB的分区后,系统中空闲分区的数量、空闲分区链第一个分区的起始地址和大小分别是 \\
A.3、20 K. 380 KB  $\quad$  B.3、500 K.  80 KB \\
C,4、20 K.  180 KB  $\quad$  D.4、500 K.  80 KB

26.某文件系统的簇和磁盘扇区大小分别为1 KB和512 B.若一个文件的大小为1026 B,则系统分配给该文件的磁盘空间大小是 \\
A. 1026B  $\quad$  B.  1536 B  $\quad$  C. 1538B  $\quad$  D.  2048 B

27.T列有关基于时间片的进程调度的叙述中,错误的是 \\
A.时间片越短,进程切换的次数越多,系统开销也越大 \\
B.当前进程的时间片用完后,该进程状态由执行态变为阻塞态 \\
C.时钟中断发生后,系统会修改当前进程在时间片内的剩余时间 \\
D.影响时间片大小的主要因素包括响应时间、系统开销和进程数量等 \\

28.与单道程序系统相比,多道程序系统的优点是 \\
I. CPU利用率高  $\quad$  II.系统开销小 \\
III.系统吞吐量大  $\quad$  Ⅳ.I/O设备利用率高 \\
A.仅I、III    B.仅I、Ⅳ \\
C.仅II、Ⅲ    D.仅I、Ⅲ、Ⅳ

29.下列选项中,磁盘逻辑格式化程序所做的工作是 \\
I.对磁盘进行分区 \\
II.建立文件系统的根目录 \\
III.确定磁盘扇区校验码所占位数 \\
Ⅳ.对保存空闲磁盘块信息的数据结构进行初始化
A.仅II  $\quad$  B.仅II、Ⅳ \\
C.仅III、Ⅳ  $\quad$  D.仅I、II、Ⅳ

30.某文件系统中,针对每个文件,用户类别分为4类:安全管理员、文件主、文件主的伙伴、其他用户;访问权限分为5种:完全控制、执行、修改、读取、写入.若文件控制块中用二进制位串表示文件权限,为表示不同类别用户对一个文件的访问权限,则描述文件权限的位数至少应为 \\
A.5  $\quad$  B.9  $\quad$  C. 12  $\quad$  D. 20

31.若文件fl的硬链接为f2,两个进程分别打开fl和f2,获得对应的文件描述符为fdl和fd2,则下列叙述中,正确的是 \\
I. fl和f2的读写指针位置保持相同 \\
II. fl和f2共享同一个内存索引结点 \\
Ⅲ.fdl和fd2分别指向各自的用户打开文件表中的一项 \\
A.仅Ⅲ    B.仅II、III    C.仅I、II    D.  I、II和III

32.系统将数据从磁盘读到内存的过程包括以下操作: \\
①DMA控制器发出中断请求 \\
②初始化DMA控制器并启动磁盘 \\
③从磁盘传输一块数据到内存缓冲区 \\
④执行“DMA结束”中断服务程序 \\
正确的执行顺序是 \\
A.③$\rightarrow$①$\rightarrow$②$\rightarrow$④  $\quad$  B.②$\rightarrow$③$\rightarrow$①$\rightarrow$④ \\
C.②$\rightarrow$①$\rightarrow$③$\rightarrow$④  $\quad$  D.①$\rightarrow$②$\rightarrow$④$\rightarrow$③

33.假设OSI参考模型的应用层欲发送400 B的数据(无拆分1,除物理层和应用层之外,其他备层在封装PDU时均引入20 B的额外开销,则应用层数据传输效率约为 \\
A.80\%  $\quad$  B.83\%  $\quad$  C.8726 $\quad$ D. 9126

34.若信道在无噪声情况下的极限数据传输速率不小于信噪比为30 dB条件下的极限数据传输速率,则信号状态数至少是 \\
A.4 $\quad$  B.8 $\quad$ C. 16 $\quad$ D.32

35.在下图所示的网络中,若主机H发送一个封装访问Internet的IP分组的IEEE 802.11数据帧F,则帧F的地址1、地址2和地址3分别是 \\
\begin{figure}[ht]
\centering
\includegraphics[width=14.25cm]{./figures/CSN17_3.png}
\caption{第35题图} \label{CSN17_fig3}
\end{figure}
A. 00-12-34-56-78-9a, 00-12-34-56-78-9b, 00-12-34-56-78-9c
B. 00-12-34-56-78-9b, 00-12-34-56-78-9a, 00-12-34-56-78-9c
C. 00-12-34-56-78-9b, 00-12-34-56-78-9c, 00-12-34-56-78-9a
D. 00-12-34-56-78-9a, 00-12-34-56-78-9c, 00-12-34-56-78-9b

36.下列IP地址中,只能作为lP分组的源IP地址但不能作为目的IP地址的是 \\
A. 0.0.0.0  $\quad$  B. 127.0.0.1 \\
C. 200.10.10.3  $\quad$  D. 255.255.255.255

