% 光的电磁波性质
% license Usr
% type Tutor

\pentry{麦克斯韦方程组\nref{nod_MWEq}}{nod_1dce}

\subsection{回顾}

光的电磁理论认为,光是一种电磁波。电磁场的普遍规律可以总结为麦克斯韦方程组,积分形式的麦克斯韦方程组为:

\begin{equation}
\begin{aligned}
& \displaystyle \iint\!\!\!\!\!\!\!\!\!\!\subset\!\supset \bvec D \cdot \dd{\bvec \sigma} = Q ~, \\
& \displaystyle \iint\!\!\!\!\!\!\!\!\!\!\subset\!\supset \bvec B \cdot \dd{\bvec \sigma} = 0 ~, \\
& \oint \bvec E \cdot \dd{\bvec l} = - \iint \frac{\partial \bvec B}{\partial t} \dd{\bvec \sigma} ~, \\
& \oint \bvec H \cdot \dd{\bvec l} = \bvec I + \iint \frac{\partial \bvec D}{\partial t} \dd{\bvec \sigma} ~.
\end{aligned}
\end{equation}

方程组中,$\bvec D$、$\bvec E$、$\bvec B$ 和 $\bvec H$ 分别为电位移矢量、电场强度、磁感应强度和磁场强度,对 $\dd{\bvec \sigma}$ 和 $\dd{\bvec l}$ 的积分分别表示对电磁场中任一闭合曲面和闭合回路上的积分。$Q$ 为闭合曲面包含的总电量,$I$ 为闭合回路包围的传导电流\footnote{想象一根闭合绳子捆着若干电线,这些电线都通有电流,你可以把传导电流简单理解为这些电流的总和}。方程组第1式为高斯定理,第2式表示磁场无源,第3式为法拉第电磁感应定律,第4式描述在电磁场中两种电流产生了磁场。

另有麦克斯韦方程组的微分形式:

\begin{equation}
\begin{aligned}
& \div \bvec D = \rho ~, \\
& \div \bvec B = 0 ~, \\
& \curl \bvec E = -\frac{\partial \bvec B}{\partial t} ~, \\
& \curl \bvec H = \bvec j + \frac{\partial \bvec D}{\partial t} ~.
\end{aligned}
\end{equation}

$\rho$ 为电荷体密度,$\bvec j$ 为传导电流密度。

实际上,在方程组中,只有 $\bvec E$ 和 $\bvec B$ 是用于描述场的“真正”物理量,而 $\bvec D$ 和 $\bvec H$ 只是两个辅助量,$\bvec E$ 和 $\bvec D$、$\bvec B$ 和 $\bvec H$ 有着紧密联系,这一联系而电磁场所在的物质性质有关。

对于各向同性线性物质,我们有:

\begin{equation}
\begin{aligned}
& \bvec D = \varepsilon \bvec E ~, \\
& \bvec B = \mu \bvec H ~.
\end{aligned}
\end{equation}

$\varepsilon$ 和 $\mu$ 是两个标量,分别称为介电常数和磁导率。

对于各向异性物质,$\varepsilon$ 和 $\mu$ 不再是标量,而是 