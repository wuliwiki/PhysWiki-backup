% 圆锥曲线的统一定义(高中)
% keys 准线|第二定义|焦点|圆锥曲线|焦点-准线定义
% license Xiao
% type Tutor

\begin{issues}
\issueDraft
\end{issues}

\pentry{圆锥曲线与圆锥\nref{nod_ConSec}}{nod_55cd}

古希腊时期,人们通过截取圆锥面来研究圆、椭圆、抛物线和双曲线等曲线。这种方法虽然直观,并赋予了这些曲线共同的几何起源,但在具体研究时,仍将它们视为彼此独立的四类对象,分别分析各自的性质。它们的代数表达不同,图像形状各异,彼此之间缺乏统一的联系。另外,在高中教材中,常常在介绍椭圆的长轴、短轴和焦点等概念时,直接空降一个“离心率”的概念,并简单地解释为“衡量椭圆扁平程度”的参数。但细心的读者可能会注意到,抛物线和双曲线也有各自的“离心率”,这自然会引发疑问:为什么所有圆锥曲线都有离心率?这个量到底为何被这样定义?不同曲线的离心率之间又有怎样的联系?

实际上,在阿波罗尼乌斯的时代,“离心率”这一概念尚未系统建立。随着解析几何的建立,数学家们才逐渐发现:这些看似不同的曲线,在引入一个定点和一条定直线后,竟然可以通过一个简洁而优雅的定义统一起来。这一定义不仅在解析几何中揭示了圆锥曲线的本质,还在射影几何等更深层次的研究中带来了意想不到的收获。而离心率也是在圆锥曲线的统一定义提出之后,才逐渐发展成为刻画这些曲线的重要参数。

可惜的是,这一部分内容在现行高中阶段的课程中已被完全删除。为带给读者更完整的视角,本文将从统一定义出发,系统探讨圆锥曲线的几何构造及其背后的深层联系。

\subsection{圆锥曲线的焦点-准线定义}

利用准线与焦点得到的。提供了一个统一的视角来看待

\textbf{圆锥曲线的焦点-准线定义(Focus-Directrix Definition of Conic Sections)}。

\begin{definition}{圆锥曲线的焦点-准线定义}\label{def_HsCsFD_1}
在平面上,所有到一个定点的距离与到一条定直线的距离的比值是一个固定常数的点的轨迹,称为\textbf{圆锥曲线(conic section)}。其中,定点称为圆锥曲线的\textbf{焦点(focus)},定直线称为圆锥曲线的\textbf{准线(directrix)},二者互相对应,对应的焦点与准线的距离称作\textbf{焦准距(focal parameter)},通常记作$p$。比值称作圆锥曲线的\textbf{离心率(eccentricity)},通常记作$e$ 。特别地:
\begin{itemize}
\item 当 $e = 0$ 时,轨迹称为\textbf{圆(circle)}\footnote{这一点会在\aref{后文}{sub_HsCsFD_1}提供说明。}。
\item 当 $0 < e < 1$ 时,轨迹称为\textbf{椭圆(ellipse)}。
\item 当 $e = 1$ 时,轨迹称为\textbf{抛物线(parabola)}。
\item 当 $e > 1$ 时,轨迹称为\textbf{双曲线(hyperbola)}。
\end{itemize}
\end{definition}

\begin{figure}[ht]
\centering
\includegraphics[width=11cm]{./figures/52670f52be70ae3b.pdf}
\caption{$p = 1$时,不同离心率 $e$ 的圆锥曲线} \label{fig_Cone_2}
\end{figure}

显然,定点到定直线的垂线为圆锥曲线的对称轴。


\subsection{性质}

离心率表示“扁平程度”:
$$ e = \frac{c}{a} = \sqrt{1 - \frac{b^2}{a^2}} \in [0, 1) ~.$$
椭圆越接近 1 越扁。

\subsection{定义等价性}

\subsubsection{椭圆}

由直角坐标方程可知对称性,可在椭圆的两边做两条准线,令椭圆上任意一点到两焦点的距离分别为 $r_1$ 和 $r_2$,到两准线的距离分别为 $d_1$ 和 $d_2$,则有
\begin{equation}
e = \frac{r_1}{d_1} = \frac{r_2}{d_2} = \frac{r_1 + r_2}{d_1 + d_2}~,
\end{equation}
所以
\begin{equation}
r_1 + r_2 = e(d_1+d_2) = 2e(c + h) = 2\frac{c}{a} \qty( c + \frac{b^2}{c} ) = 2a~,
\end{equation}
证毕。
\subsubsection{双曲线}
双曲线的另一种定义是, 曲线上任意一点到两个焦点距离之差等于 $2a$。 这里证明前两种定义满足该性质。 由对称性, 我们不妨只考虑右支上的某点, 令其到右焦点和右准线的距离分别为 $r_1$ 和 $d_1$, 到左焦点和左准线的距离分别为 $r_2$ 和 $d_2$。 由离心率的定义, 有
\begin{equation}
e = \frac{r_1}{d_1} = \frac{r_2}{d_2} = \frac{r_2 - r_1}{d_2 - d_1}~,
\end{equation}
由于两准线之间的距离恒为 $2a^2/c$, 上式变为
\begin{equation}
r_2 - r_1 = e(d_2 - d_1) = 2a~,
\end{equation}
证毕。

\subsection{*射影几何视角下的圆锥曲线}\label{sub_HsCsFD_1}


\addTODO{$e = 0$ 为什么是圆?}注意根据定义,圆的准线为无穷远, 所以只能使用中, 圆的半径为无穷小。


射影几何中的视角使我们能够用一种统一且优雅的方式看待圆锥曲线。但在射影几何中,这些差异被看作是坐标选择与观察角度所导致的表象变化,它们在更本质的层面上是一类对象的不同表现:它们都是圆锥曲线。圆锥曲线不是三类不同的曲线,而是一个统一的几何实体的三种视角。它让我们跳出了直观图形的束缚,从结构上理解几何对象之间的联系,也为代数几何、复几何乃至更高维的几何打下了坚实的基础。

从射影几何的角度看,圆锥曲线定义为一个圆锥面与一个平面相交的轨迹。这个定义在欧几里得空间中也成立,但射影几何更进一步地指出:在射影平面中,所有非退化的圆锥曲线都是射影等价的。这意味着我们可以通过一个合适的射影变换(即坐标的线性变换加上归一化),将任意一个圆锥曲线变为另一个圆锥曲线——比如将一个椭圆变为一个双曲线或抛物线。

换句话说:
\begin{itemize}
\item 椭圆是在射影平面中与无穷远直线没有实交点的圆锥曲线;
\item 双曲线是在射影平面中与无穷远直线有两个实交点的圆锥曲线;
\item 抛物线是恰好与无穷远直线有一个交点的极限情形。
\end{itemize}

这种分类在射影几何中失去了意义,因为无穷远直线被作为与其他直线同等地位来处理,不再是“例外的部分”。因此,抛物线、椭圆和双曲线不再是本质不同的几何对象,而只是一个对象的不同投影或表示。

此外,射影几何还强调了极点与极线的对偶性,并引入了极线极点变换的工具来研究圆锥曲线的性质,使得很多命题具有了对称且优美的形式。例如:对于一个给定的圆锥曲线,任意一点都有与之对应的一条极线,反之亦然。这种对偶关系在欧氏几何中并不自然存在。

明白了!你希望先以“焦点-准线”的统一定义为出发点,引出为什么引入射影几何的视角可以进一步理解这种统一性。我们将不再提“圆锥体的截面”,而是从代数和几何定义的统一出发,逐步引导学生理解射影几何的优势。

下面是按你的思路调整后的教材正文草稿:

⸻

焦点-准线统一视角下的圆锥曲线:通往射影几何的一扇门

我们已经分别学习过三种圆锥曲线:椭圆、抛物线和双曲线。

它们的定义看起来各不相同,方程也各有特点。但有没有一种方法,能用一句话同时描述这三种曲线呢?答案是:有。

这一节课,我们将首先从“焦点-准线”的角度,找到一个统一的定义;接着,我们将引入一种叫做“射影几何”的新视角,让我们对圆锥曲线之间的联系有更深入的理解。

⸻

一、统一的定义:到准线距离与到焦点距离的比值

我们先回顾一下三种曲线的定义:
	•	椭圆:到两个焦点的距离之和是定值
	•	抛物线:到焦点和到准线的距离相等
	•	双曲线:到两个焦点的距离之差是定值

看上去它们的定义很不一样。但我们可以只保留一个焦点和一条准线,来统一定义这三种曲线:

圆锥曲线是平面上满足“到某个固定点(焦点)的距离与到某条固定直线(准线)的距离的比值是一个常数 $e$”的点的集合。

这个比值 $e$ 叫做离心率(eccentricity),不同的值决定了曲线的形状:
	•	$0<e<1$ 时,是椭圆;
	•	$e=1$ 时,是抛物线;
	•	$e>1$ 时,是双曲线。

这就是我们寻找的统一定义。只需要一个公式,就可以包含我们之前学的三种曲线。是不是很简洁?

但还有一个问题:

为什么 $e=1$ 是一个“分界线”?

离心率为什么只分成这三类,而不是连续变化出更多种曲线?

要理解这个问题,我们需要换一个“看待图形”的方式,也就是今天要介绍的新视角:射影几何。

⸻

二、普通几何的局限:为什么我们看不到统一?

在我们熟悉的平面几何中,有一些“默认”的限制,比如:
	•	平行线不会相交;
	•	直线是无限延伸的;
	•	点只能表示有限的位置。

这些看起来都很自然,但正是这些“限制”,让我们无法从一个更高的角度去看清圆锥曲线之间的关系。

尤其是当我们在统一定义中使用了“准线”这个概念时,我们会发现一个问题:

准线是直线,而焦点是点,它们的地位并不对等。

比如,在抛物线中,焦点和准线之间的距离决定了曲线的开口程度;但在椭圆和双曲线中,焦点之间的关系常常比准线更显眼。这种“不对等”让我们难以一眼看出统一性。

要解决这个问题,我们需要让“点”和“直线”变得对等、互换,这正是射影几何擅长的。

⸻

三、射影几何:添加“无穷远”来重新看世界

射影几何的出发点是:我们不要再区分平行与相交,也不要忽略“无穷远处”的点。

我们在现实中早就见过类似的情形:
	•	平行的铁路轨道,在远方看起来会相交;
	•	街道两旁的建筑,在画中会汇聚到“消失点”;
	•	摄影师知道,透视图中的所有平行线,最终都会“汇聚”。

这些现象的数学表达方式,就是射影几何中的一个核心思想:

所有直线在射影几何中都相交——平行线也会在“无穷远点”相交。

于是我们扩展了平面,引入了一个“无穷远直线”,把所有方向的“无穷远点”放在这条线上。

更惊人的是:

在射影几何中,点和直线可以互换、对称对待;也就是说,直线也可以看成是“由点组成”的,点也可以像直线一样进行变换。

⸻

四、重新看焦点和准线:变换下的对等性

回到我们的统一定义:

到焦点距离与到准线距离的比值等于 $e$

焦点是一个点,准线是一个直线,它们是不一样的。但在射影几何中,我们可以把直线看成是“一个方向上的点的集合”,特别是在加入了“无穷远点”之后,直线也可以被看作是特殊的“点”。

这就让焦点和准线,在某种意义上变得“对等”。

更重要的是:

在射影几何中,通过变换,我们可以把一个圆锥曲线变换成另一种类型的圆锥曲线,只要它们满足相同的基本结构。

举个例子:
	•	一个椭圆,通过一个适当的“射影变换”,可以变成一个抛物线;
	•	抛物线也可以变成双曲线;
	•	这些变换不会改变圆锥曲线的“本质”,只改变它在我们眼中的“样子”。

这就说明,射影几何的世界中,圆锥曲线是一个统一的整体,而不是三种各自孤立的图形。

⸻

五、结语:统一不是结束,而是开始

通过这节课的学习,我们从“到焦点和准线的距离比”这个统一定义出发,进入到了一个新的几何世界——射影几何。在这个世界中,曲线之间的关系变得清晰、自然,而且不再受到原来空间结构的限制。

我们学数学,不只是为了掌握解题方法,更是为了建立更高层次的理解力。射影几何给我们展示了:
	•	同一个对象可以从不同的角度理解;
	•	表面看起来不同的东西,背后可能有统一的结构;
	•	有时,必须打破一些“习惯的规则”,才能看到更完整的图景。

未来你会在更高层的数学中看到更多这样的“统一视角”——它们不仅改变你对数学的看法,也可能改变你看待世界的方式。

⸻

如果你觉得合适,我可以继续帮你补上图示说明、引导性问题、小结框和章节标题编号格式,也可以按章节调整段落长度、难度分层等。需要我继续吗?