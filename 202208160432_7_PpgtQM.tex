% 传播子(波动力学)
% 路径积分|狄拉克符号|薛定谔方程|时间演化算符

\pentry{时间演化算符(量子力学)\upref{TOprt}}

本节采用自然单位制,$\hbar=1$.


\subsection{传播子的概念}


考虑一个$t=0$时刻的初始量子态,记为$\ket{s}$.设有哈密顿算符$H$,则时间演化算符是$\E^{-\I Ht}$.设$H$的本征态为$\ket{a}$,对应本征值为$E_a$.

用$H$\textbf{的本征态}来展开$\ket{s}$,得到$\ket{s}=\sum_a\ket{a}\braket{a}{s}$,则有
\begin{equation}\label{PpgtQM_eq1}
\bra{\bvec{x}}\E^{-\I Ht}\ket{s} = \sum_a\bra{\bvec{x}}\E^{-\I Ht}\ket{a}\braket{a}{s} = \sum_a\braket{\bvec{x}}{a}\braket{a}{s}\E^{-\I E_at}
\end{equation}

用\textbf{位置本征态}$\ket{\bvec{x}}$来展开$\ket{s}$,得到$\ket{s}=\int\mathrm{d}^3 x\ket{\bvec{x}}\braket{\bvec{x}}{s}$,则有
\begin{equation}\label{PpgtQM_eq2}
\braket{a}{s} = \int\mathrm{d}^3x\braket{a}{\bvec{x}}\braket{\bvec{x}}{s}
\end{equation}

将\autoref{PpgtQM_eq2} 代入\autoref{PpgtQM_eq1} ,设$\ket{\bvec{y}}$是位置本征态,$\psi(\bvec{y}, t)$是$\ket{s}$在时刻$t$的波函数,则有
\begin{equation}\label{PpgtQM_eq3}
\ali{
    \psi(\bvec{y}, t)&=\bra{\bvec{y}}\E^{-\I Ht}\ket{s}\\
    &=\sum_a\braket{\bvec{y}}{a}\braket{a}{s}\E^{-\I E_at}\\
    &=\sum_a\braket{\bvec{y}}{a}\E^{-\I E_at}\int\mathrm{d}^3x\braket{a}{\bvec{x}}\braket{\bvec{x}}{s}\\
    &=\int\mathrm{d}^3x\sum_a\braket{\bvec{y}}{a}\E^{-\I E_at}\braket{a}{\bvec{x}}\braket{\bvec{x}}{s}
}
\end{equation}

则定义$K(\bvec{y}, \bvec{x}, t)$为$\sum_a\braket{\bvec{y}}{a}\E^{-\I E_at}\braket{a}{\bvec{x}}$.根据\autoref{PpgtQM_eq3} ,这意味着
\begin{equation}
\psi(\bvec{y}, t) = \int\mathrm{d}^3x K(\bvec{y}, \bvec{x}, t)\psi(\bvec{x}, 0)
\end{equation}
称$K$为\textbf{传播子}.

注意,传播子也可以写成
\begin{equation}
\ali{
K(\bvec{y}, \bvec{x}, t) &= \bra{\bvec{y}}\qty(\sum_a\ket{a}\E^{-\I E_at}\bra{a})\ket{\bvec{x}} 
\\&= \bra{\bvec{y}}\qty(\E^{-\I Ht}\sum_a\ket{a}\bra{a})\ket{\bvec{x}}\\
&= \bra{\bvec{y}}\E^{-\I Ht}\ket{\bvec{x}}
}
\end{equation}

实际计算中,$\E^{-\I Ht}\ket{\bvec{x}}$可能难以算出,因此也常用


\subsection{传播子的性质}




对传播子$K$关于时间求偏导,得到:
\begin{equation}
\ali{
    \I\partial_t K &= \bra{\bvec{y}}\qty(\I\partial_t\E^{-\I Ht})\ket{\bvec{x}}\\
    &= \bra{\bvec{y}}\qty(H\E^{-\I Ht})\ket{\bvec{x}}\\
    &= HK
}
\end{equation}
即$K$\textbf{满足薛定谔方程}.

另外,注意到
\begin{equation}
\ali{
    K(\bvec{y}, \bvec{x}, 0) &= \braket{\bvec{y}}{\bvec{x}} = \delta^3(\bvec{y}-\bvec{x})
}
\end{equation}
即$K(\bvec{y}, \bvec{x}, 0)$可以看成是$t=0$时位置本征态$\ket{\bvec{x}}$的波函数$\psi_{\bvec{x}}(\bvec{y}, 0)$.

由于量子态的演化遵循薛定谔方程,因此综上所述,$K(\bvec{y}, \bvec{x}, t)$可以视为一个自然演化的波函数$\psi_{\bvec{x}}(\bvec{y}, t)$,其初态为$\ket{\bvec{x}}$.



\begin{example}{一维自由粒子}

考虑一个\textbf{一维自由}粒子,显然$[H, p]=0$.具体地,$p=-\I\hbar\partial_x$,$H=\frac{p^2}{2m}=\frac{\hbar^2}{2m}\partial_x^2$.

设$\ket{p_a}$是$p$与$H$的共同本征态,其中下标$a$取值范围为整个实数集,且$p\ket{p_a}=a$,这还意味着$H\ket{p_a}=a^2/2m$.

于是能计算出传播子
\begin{equation}
\ali{
    K(\bvec{y}, \bvec{x}, t) &= \bra{\bvec{y}}\qty(\E^{-\I Ht}\int\ket{p_a}\bra{p_a}\dd a)\ket{\bvec{x}}\\
    &= \bra{\bvec{y}}\qty(\int\E^{-\I a^2t/2m}\ket{p_a}\bra{p_a}\dd a)\ket{\bvec{x}}\\
}
\end{equation}

\end{example}













