% Teaching_hhh+lzh
% teaching
\subsection{Exercise_1}
\begin{itemize}
\item 已知椭圆$\frac{x^2}{a^2}+\frac{y^2}{b^2}=1$
其中(a>b>0)的左右焦点分别为$F_1$,$F_2$.若椭圆C经过点(0,$\sqrt{3}$),离心率为$\frac{1}{2}$,直线L过点$F_2$与椭圆C相交于A,B两点.\\
(1)求椭圆C的方程;\\
(2)若N为$\Delta F_1AF_2$的内心,求$\Delta F_1NF_2$和$\Delta F_1AF_2$的比值;\\
(3)设点A,$F_2$,B在直线x=4上的投影依次为D,G,E,连接AE,BD.试问当直线L的倾斜角改变时,直线AE与BD是否相交于定点T?若是,则证明;不是则说明理由.\\
\\
\item \textbf{Answers:}\\
%(1)$\frac{x^2}{4}+\frac{y^2}{3}=1$\\

\end{itemize}
\\ \\ \\ \\ \\ \\ \\ \\ \\ \\ \\ \\ \\ \\ \\ \\
\subsection{Exercise_2}
\begin{itemize}
\item 已知抛物线$C_1$:$x^2=2py$的焦点在抛物线$C_2$:$y=\frac{1}{2}x^2+\frac{1}{4}$上.\\
(1)求抛物线$C_1$的方程.\\
(2)过抛物线$C_1$上的动点P作抛物线$C_2$的两条切线:PM和PN,其中切点为M和N,若PM和PN的斜率乘积为m,且m$\in$[$\frac{3}{2}$,$\frac{7}{2}$],求|OP|的取值范围.\\
\end{itemize}
\\ \\ \\ \\ \\ \\ \\ \\ \\ \\ \\ \\ \\ \\ \\ \\
\subsection{Exercise_3}
\begin{itemize}
\item 已知抛物线$x^2=2py$(p>0)上一点R:(m,2)到它的准线的距离为3,若点A,B,C分别在抛物线上,且点A、C在y轴右侧,点B在y轴左侧,$\Delta ABC$的重心G在y轴上,直线AB交y轴于点M且满足3|AM|<2|BM|,直线BC交y轴于点N,记$\Delta ABC$,$\Delta AMG$,$\Delta CNG$的面积分别为$S_1$,$S_2$,$S_3$,\\
(1)求p的值及抛物线的准线方程;\\
(2)求$\frac{S_1}{S_2+S_3}$的取值范围.\\
\end{itemize}
\\ \\ \\ \\ \\ \\ \\ \\ \\ \\ \\ \\ \\ \\ \\ \\
\subsection{Exercise_4}
\begin{itemize}
\item 已知椭圆的中心坐标为原点O,焦点在x轴上,斜率为1且过椭圆的右焦点的直线交椭圆于A,B两点,$\overrightarrow{OA}$+$\overrightarrow{OB}$与a=(3,-1)共线,设M为椭圆上任意一点,且$\overrightarrow{OM}$=$\lambda \overrightarrow{OA}+\mu \overrightarrow{OB}$($\lambda,\mu \in R$)求证:$\color{Red}\lambda^2+\mu^2$为定值.\\
\end{itemize}
\\ \\ \\ \\ \\ \\ \\ \\ \\ \\ \\ \\ \\ \\ \\ \\

\subsection{Exercise_5}
\begin{itemize}
\item 已知椭圆的方程为$\frac{x^2}{2}$+$\frac{y^2}{1}$=1,F为左焦点,若过P:(-2,0)的直线与椭圆相交于不同的两点M,N,求$\triangle MNF$面积的最大值.\\
\end{itemize}
\\ \\ \\ \\ \\ \\ \\ \\ \\ \\ \\ \\ \\ \\ \\ \\
\subsection{Exercise_6}
\begin{itemize}
\item 已知椭圆$\frac{x^2}{9}+\frac{y^2}{4}=1$与x轴交于A,B两点,过椭圆上一点P:($x_0,y_0$)(P不与A,B点重合)的切线L的方程为$\frac{x_0\,x}{9}+\frac{y_0\,y}{4}=1$过点A,B且垂直于x轴的垂线分别与L交于C,D两点,设BC,AD交于点Q,则Q点的轨迹方程为?\\
\end{itemize}
\\ \\ \\ \\ \\ \\ \\ \\ \\ \\ \\ \\ \\ \\ \\ \\
\subsubsection{Exercise_7}
\begin{itemize}
\item 椭圆$C_1$:$\frac{x^2}{a^2}+\frac{y^2}{b^2}=1$(a>b>0)的离心率为$\frac{\sqrt{2}}{2}$,x轴被直线$C_2$:$y=x^2-b$截得的线段长度等于$C_1$的短轴长,$C_2$与y轴的交点为M,过坐标原点O的直线L与$C_2$相交于A,B直线MA,MB分别与$C_1$相交于点D,E.\\
(1)求$C_1$,$C_2$的方程.\\
(2)求证:MA$\bot$MB;\\
(3)记$\triangle MAB$,$\triangle MDE$的面积分别为$S_1$,$S_2$,若$\frac{S_1}{S_2}=\lambda$,求
\end{itemize}
