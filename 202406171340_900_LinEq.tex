% 线性方程组的解
% keys 线性方程组|矢量空间|秩|矩阵
% license Xiao
% type Tutor

\begin{issues}
\issueDraft
\end{issues}
% Giacomo:TODO: 拆解成《线性方程组的几何含义》和《线性方程组解的个数》

\pentry{线性方程组与增广矩阵\nref{nod_LinEq2},秩—零化度定理(矩阵)\nref{nod_rnMat}}{nod_1edb}

线性方程组
\begin{equation}
\begin{cases}
A_{11} x_1 + A_{12} x_2 + \cdots + A_{1N} x_N &= b_1 \\
A_{21} x_1 + A_{22} x_2 + \cdots + A_{2N} x_N &= b_2 \\
&\vdots \\
A_{M1} x_1 + A_{M2} x_2 + \cdots + A_{MN} x_N &= b_M
\end{cases} ~
\end{equation}
可以记为
\begin{equation}\label{eq_LinEq_1}
\mat A \bvec x = \bvec b~,
\end{equation}
或者增广矩阵 $\pmat{\mat A \mid \bvec b}$;其中 $\mat A$ 是 $M \times N$ 的矩阵, $\bvec x$ 是 $N$ 维列矢量, $\bvec b$ 是 $M$ 维列矢量, $\mat A \bvec x$ 表示矩阵与列矢量相乘(\autoref{eq_Mat_4})。 $\mat A$ 和 $\bvec b$ 是已知的, $\bvec x$ 是未知的, 被称为方程组的\textbf{解(solution)}。

线性方程组可以有零个解(无解)、一个解(唯一解)或无数解;但不可能只有 $2,3,4,\dots$ 个解。


\subsection{线性方程组的几何含义}\label{sub_LinEq_2}
可以分别从行与列的角度来理解线性方程组的几何含义。\footnote{本文参考了Gilbert Strang的《线性代数》课程,J. Leon 的Linear Algebra with Applications,以及李永乐等的线代考研课程}更深刻、数学的表示可以见\enref{线性方程组的仿射解释}{AS2LF}。

\subsubsection{行视角}
% $$
% \ali{\mat A \bvec x=\bvec b &\Longleftrightarrow 
% \begin{pmatrix}
% A_{11} & A_{12} & A_{13} & \cdots \\
% A_{11} & A_{12} & A_{13} & \cdots \\
% A_{11} & A_{12} & A_{13} & \cdots \\
% \vdots
% \end{pmatrix}
% \begin{pmatrix}
% x_{1} \\
% x_{2} \\
% x_{3} \\
% \vdots
% \end{pmatrix}
% =
% \begin{pmatrix}
% b_{1} \\
% b_{2} \\
% b_{3} \\
% \vdots
% \end{pmatrix} \\
% &\Longleftrightarrow
% \begin{cases}
% b_1 &= A_{11} x_1 + A_{12} x_2 + A_{13} x_3 + \cdots \\
% b_2 &= A_{21} x_1 + A_{22} x_2 + A_{23} x_3 + \cdots \\
% b_3 &= A_{31} x_1 + A_{32} x_2 + A_{33} x_3 + \cdots \\
% &\vdots
% \end{cases}}
% ~
% $$

线性方程组
\begin{equation}
\begin{cases}
A_{11} x_1 + A_{12} x_2 + \cdots + A_{1N} x_N &= b_1 \\
A_{21} x_1 + A_{22} x_2 + \cdots + A_{2N} x_N &= b_2 \\
&\vdots \\
A_{M1} x_1 + A_{M2} x_2 + \cdots + A_{MN} x_N &= b_M
\end{cases} ~
\end{equation}
的每一行方程的解集是 $N$ 维空间中的 $N-1$ 维子空间,称为\textbf{超平面}(比如二维空间中的超平面就是直线,三维空间中的超平面是通常的平面等),方程组的解就是这些超平面的交集。

\begin{example}{}
例如,求解
% $$
% \begin{pmatrix}
% 2&-1 \\
% -1&2
% \end{pmatrix}
% \begin{pmatrix}
% x_1 \\
% x_2
% \end{pmatrix}
% =
% \begin{pmatrix}
% 0 \\
% 3
% \end{pmatrix}~.
% $$
% 即
$$
\begin{cases}
2x_1-x_2=0 \\
-x_1+2_2=3~. \\
\end{cases}
$$
\begin{figure}[ht]
\centering
\includegraphics[width=8cm]{./figures/ed701f2472333a9c.pdf}
\caption{行视角下的线性方程组。解可以理解为直线的交点。仿自Strang的《线性代数》} \label{fig_LinEq_2}
\end{figure}
\end{example}
从这个角度可以很直观的理解“无解”、“无数解”。很显然,一组直线不能仅有两个交点,所以线性方程组也不可能只有两个解。
\begin{figure}[ht]
\centering
\includegraphics[width=12cm]{./figures/d89a22b5cc50341b.pdf}
\caption{“无解(没有交点)”、“唯一解(有唯一交点)”、“无数解(有无数交点)”} \label{fig_LinEq_3}
\end{figure}

另一方面,从用\enref{分块矩阵}{BlkMat}的视角,我们可以将$\mat A$划分为 $M$ 个 $N$ 维行向量$\bvec \alpha_1, \bvec \alpha_2, \cdots, \bvec \alpha_M$,
$$
\mat A \bvec x=\bvec b \Leftrightarrow 
\begin{pmatrix}
\bvec \alpha_1 \\
\bvec \alpha_2 \\
\vdots \\
\bvec \alpha_M
\end{pmatrix}
\bvec x
=
\begin{pmatrix}
b_1 \\
b_2 \\
\vdots \\
b_M
\end{pmatrix}
\Leftrightarrow 
\begin{cases}
b_1 &= \bvec \alpha_1 \bvec x \\
b_2 &= \bvec \alpha_2 \bvec x \\
&\vdots \\
b_M &= \bvec \alpha_M \bvec x
\end{cases}~
$$
这样表述更加简洁。

\subsubsection{列视角}
运用\enref{分块矩阵}{BlkMat}的视角,将$\mat A$划分为 $N$ 个 $M$ 维列向量$\bvec \alpha_1, \bvec \alpha_2, \cdots \bvec \alpha_N$,
$$
\ali{
\mat A \bvec x = \bvec b &\Leftrightarrow 
\begin{pmatrix}
\bvec \alpha_1 & \bvec \alpha_2 & \cdots & \bvec \alpha_N
\end{pmatrix}
\begin{pmatrix}
x_{1} \\
x_{2} \\
\vdots \\
x_{N}
\end{pmatrix} = \bvec b \\
&\Leftrightarrow 
\bvec \alpha_1 x_1 + \bvec \alpha_2 x_2 + \cdots + \bvec \alpha_N x_N = \bvec b
}~
$$
此时,$\bvec b$可以看作是一系列$\bvec \alpha_i$的线性组合,而解$\bvec x$是各个列向量的“系数”。方程组无解的含义即为“$\bvec b$不能由$\bvec \alpha_i$线性组合得到”,无数解的含义即为“有无数种方法线性组合$\bvec \alpha_i$以得到$\bvec b$”,这暗示了这一系列$\bvec \alpha_i$中存在线性相关的项。

\begin{example}{}
还是例如求解
$$
\pmat{ 2 & -1 \\ -1 & 2 } \pmat{ x_1 \\ x_2 } = \pmat{ 0 \\ 3 }~,
$$
即
$$
\pmat{ 2 \\ -1 } x_1 + \pmat{ -1 \\ 2 } x_2 = \pmat{ 0 \\ 3 }~.
$$
\begin{figure}[ht]
\centering
\includegraphics[width=8cm]{./figures/a340417f57880d30.pdf}
\caption{列视角下的线性方程组。解可理解为各列向量的“系数”。仿自Strang的《线性代数》} \label{fig_LinEq_4}
\end{figure}

可见任意向量都能由 $\pmat{ 2 \\ -1 }$ 和 $\pmat{ -1 \\ 2 }$ 的线性组合得到,包括了 $\pmat{ 0 \\ 3 }$。
\end{example}
% Giacomo: 第一种角度似乎非常粗浅,可以在《线性方程组(高中)》中讲?
看起来,在线性方程组中,矩阵$\mat A$的行向量与列向量存在一种微妙的关联。

\subsection{判断线性方程组解的个数}
\begin{figure}[ht]
\centering
\includegraphics[width=14.25cm]{./figures/0cb8501eb2baf4ae.pdf}
\caption{$Ax=b$ 的解} \label{fig_LinEq_1}
\end{figure}

% \addTODO{需要补充证明}%我不会证明QAQ
\addTODO{需要引用《秩—零化度定理(矩阵)》} % Giacomo


关于线性方程组的解,我们有如下定理。记m为$\mat A$的行数,n为$\mat A$的列数,r为$\mat A$的\enref{矩阵的秩}{MatRnk}。可参考\autoref{fig_LinEq_1} 的分类。

\begin{theorem}{$\mat A \bvec x=\bvec b$解的唯一性}
若 $n-r=0$,若解存在,则解唯一。

若 $n-r>0$,若解存在,则解不唯一。
\end{theorem}
n-r事实上是 $\mat A$ 的零空间的基个数 $\opn{dim}(\opn{Nul}(\mat A))=n-r$。

\begin{theorem}{$\mat A \bvec x=\bvec b$解的存在性}
若 $m-r=0$,则解一定存在。

若 $m-r>0$,则解可能不存在。
\end{theorem}
m-r事实上是$\mat A$的左零空间的基个数$dim(Nul(\mat A^T))=m-r$。

\begin{theorem}{$\mat A \bvec x=\bvec b$ 解的存在性 2}
若$\bvec b$是$\mat A$的列向量的线性组合($\bvec b \in Col(\mat A)$),则$\mat A \bvec x=\bvec b$一定存在解;否则无解。
\end{theorem}
由列视角(\autoref{sub_LinEq_2})看,这是显然的。

\begin{corollary}{}
设 $\overline{\mat A} = [\mat A | \bvec b] $ 为A的增广矩阵。

$\mat A \bvec x=\bvec b$无解:$\opn{rank}(\overline{\mat A})=\opn{rank}(\mat A)+1$

唯一解:$\opn{rank}(\overline{\mat A})=\opn{rank}(\mat A)=n$

无数解:$\opn{rank}(\overline{\mat A})=\opn{rank}(\mat A)<n$
\end{corollary}
考虑到秩的含义,结合上述定理,也容易理解该推论。

\begin{theorem}{$\mat A \bvec x=\bvec 0$解的唯一性}
若 $n-r=0$,则只有一个解 $\bvec x = \bvec 0$.

若 $n-r>0$,则解不唯一,且存在(n-r)个线性无关的解。
\end{theorem}
这(n-r)个解即为零空间的(n-r)个基。

\begin{theorem}{$\mat A \bvec x=0$解的存在性}
一定存在平凡解 $\bvec x=0$
\end{theorem}

\begin{theorem}{解的结构}
$\mat A \bvec x=\bvec b$的通解等于$\mat A \bvec x=\bvec b$的一个特解加上$\mat A \bvec x=0$的各个线性无关的解的线性组合,即$\bvec x = \bvec x_p +k_1\bvec x_{n1}+k_2\bvec x_{n2}+\cdots$

因此,通常的解题套路是先求解$\mat A \bvec x=\bvec b$的一个特解,再求解$\mat A \bvec x=0$的各个线性无关的解。

(若n-r=0,则$\mat A \bvec x=0$仅有解$\bvec x_n=0$,$\mat A \bvec x=\bvec b$自然只有一个解$\bvec x = \bvec x_p$)
\end{theorem}

更深入的探讨详见下文:

从\enref{矢量空间}{LSpace}的角度来看, $\bvec x$ 是一个 $N$ 维矢量空间(以下称为 $X$ 空间)中一个矢量关于某组基底的坐标, $\bvec b$ 是一个 $M$ 维矢量空间(以下称为 $Y$ 空间)中一个矢量关于某组基底的坐标。 矩阵 $\mat A$ 可以将 $X$ 空间中的任意矢量映射到 $Y$ 后的坐标。

我们知道 $\mat A$ 的第 $i$ 列代表的矢量就是 $X$ 空间中的第 $i$ 个基底映射到 $Y$ 空间的对应矢量。 我们把 $A$ 的 $N$ 列对应的 $N$ 个矢量记为 $\{\bvec \alpha_i\}$。 先来看一个定理

\subsection{满秩方阵}
我们知道矩阵的秩 $R$ 等于线性无关的\enref{行数或列数}{MatRnk},下面来根据秩来分类讨论方程组的解空间结构。 最简单的情况是 $\mat A$ 为满秩, 即 $R = M = N$。 这时由于 $\{\bvec \alpha_i\}$ 两两线性无关, 它们可以作为 $Y$ 空间的一组基底, 与 $X$ 空间的基底一一对应。 那么这个映射既是单射又是满射。%\addTODO{链接}
对于 $Y$ 空间的任意矢量 $\bvec b$, $X$ 空间都存在唯一的解 $\bvec x$。 特殊地,当 $\bvec b = \bvec 0$ 时(即方程是\textbf{齐次}的),唯一解就是 $\bvec x = \bvec 0$。

\subsection{$R = M < N$}\label{sub_LinEq_1}

当 $\mat A$ 的秩等于 $M$ 且小于 $N$ 时, 映射变为从 $N$ 维空间到更小的 $M$ 维空间。 即非单射: 虽然任意的 $\bvec x$ 仍然映射到唯一的 $\bvec b$, 但任意的 $\bvec b$ 却对应无穷多个 $\bvec x$。 

% \addTODO{举例:三维矢量投影到二维矢量}

\addTODO{引用矩阵版本的零空间}
当方程是齐次的时候, 零空间$X_0$ 是 $N- M$ 维的(为什么?)。 这种情况下,我们希望能解出零空间的 $N - M$ 个基底,使得这组基底的任意线性组合都是齐次方程的解。

对于非齐次方程, 我们可以先求对应的齐次方程组的零空间的一组基底,再求出非齐次方程的任意一个解(\textbf{特解}), 那么非齐次方程组的\textbf{解集}(所有解的集合)就等于零空间中的所有矢量与特解相加。 注意非齐次方程的解集并不构成一个矢量空间, 因为它不包含零矢量($\bvec x = \bvec 0$ 总是对应 $\bvec b = \bvec 0$, 所以不可能是非齐次方程组的解),解集中若干矢量的线性组合也不一定仍然属于解集。
\addTODO{证明}

\subsection{$R < M$}
当 $R < M$ 时, $\{\bvec \alpha_i\}$ 中只有 $R$ 个线性无关, 它们在 $Y$ 空间中\enref{张成}{VecSpn}一个 $R$ 维子空间 $Y_0$。 如果 $\bvec b$ 在 $Y_0$ 中(可以通过 $\bvec b$ 是否与 $\{\bvec \alpha_i\}$ 线性无关来判断), 方程组就存在解, 如果落在子空间外, 方程组就无解。

\footnote{另见 “线性变换与矩阵的代数关系\upref{linmat}” 的 \autoref{the_linmat_2}。}
