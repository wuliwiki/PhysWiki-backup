% Matlab 符号计算和变精度计算(笔记)

\pentry{Matlab 的程序调试及其他功能\upref{MatOtr}}

Matlab 的符号计算需要符号计算工具箱, 取决于你的证书类型, 可能需要额外购买. 另外需要提醒的是, 虽然 Matlab 和 Python 都有符号计算功能, 但在符号计算领域 Mathemtica\upref{Mma} 才更为主流, 其默认界面也更适合符号计算.

\subsection{符号变量和符号表达式}
\begin{itemize}
\item Matlab 中用于储存符号计算表达式的变量类型为 \verb|sym|. 可以用 \verb|syms 变量1 变量2 ...| 声明变量类型为 \verb|sym|. 例如 \verb|syms x y z;|. Matlab 的大部分自带算符和函数支持 \verb|sym| 类型的变量, 例如 \verb|x^2| 就是 \verb|sym| 类型的表达式 $x^2$, $x$ 并不是一个数值而是符号. 若此时令 \verb|expr = x^2|, 那么用 \verb|class(expr)| 可以验证 \verb|expr| 的类型也是 \verb|sym|.

\item 除了使用 \verb|syms| 一次声明几个符号变量, 也可以使用 \verb|sym(字符串)|. 例如 \verb|syms x; expr = x^2;| 得到的 \verb|expr| 和 \verb|expr = sym('x')^2;| 得到的 \verb|expr| 是等效的.

\item  对表达式求导如 \verb|diff(sym('x')^2)|. 若要求偏导, \verb|diff(sym('x')^2 * sym('y')^3)| 默认对 \verb|x| 求偏导, 结果是 $2x y^3$. 也可以声明对 \verb|y| 求偏导, 如 \verb|diff(sym('x')^2 * sym('y')^3), sym('y')|, 或者更简洁地, \verb|diff(sym('x')^2 * sym('y')^3), 'y'|. 这时因为, 如果函数的一个参数需要符号表达式, 但如果输入时使用了字符串或者数值, 那么 Matlab 就会自动将其用 \verb|sym()| 函数进行转换. 建议总是声明对哪个变量求偏导.

\item  要把一个数字作为符号, 可以使用例如 \verb|sym('2')| 或 \verb|sym('7/3')|, 但这仅限于分式, 不允许注入 \verb|sym('sqrt(2)')| 这样的用法.

\item 另一种方法是使用 \verb|sym(数值)|. 例如 \verb|sqrt(sym(2))| 的结果是表达式 $\sqrt 2$, 而不同于数值计算中的 \verb|sqrt(2) = 1.414...|. 由于符号计算是精确无误差的, 无理数 $\sqrt{2}$ 并不会被自动转换为小数形式. 另一个例子, \verb|d = 3.1; sqrt(sym(d))| 得到的是表达式 $961/100$. \verb|sym(数值)| 会自动猜测 \verb|数值| 所代表的根式, 例如 \verb|sym(0.866025403784439)| 的结果是表达式 $\sqrt{3}/2$, 又例如 \verb|sym(pi)| 的结果是圆周率 $\pi$. 注意在新版的 Matlab 中, \verb|sym('pi')| 将得到名为 \verb|pi| 的普通变量, 而不是圆周率.

\item \item 一个 \verb|sym| 类型和一个 double 类型进行 \verb|+,-,*,/,^| 等运算时, \verb|double| 类型的数会自动被 \verb|sym()| 函数转换为 \verb|sym| 类型. 例如 \verb|sym(1)/3| 和 \verb|sym(1)/sym(3)| 是等效的. 这样可以让输入很简单.
\end{itemize}

\subsubsection{符号替换}
\begin{itemize}
\item \verb|subs(符号表达式,符号变量, 新表达式)| 可以把表达式中的所有 \verb|符号变量| 替换为 \verb|新表达式|. 例如 \verb|subs(sin(sym('x')), 'x', 'y')| 相当于 \verb|subs(sin(sym('x')), sym('x'), sym('y'))|, 结果是 $\sin y$. 又例如 \verb|subs(sin(sym('x')), 'x', pi/4)| 的结果是 $\sqrt 2/2$.
\item 如果要对一个符号表达式求数值近似, 那么用 \verb|eval()|, 例如 \verb|eval(sqrt(sym(2)))| 结果是 \verb|1.414...|, 是一个双精度数值, 误差就是双精度类型的相对误差, 约为 $2\e{-16}$.
\end{itemize}

\subsection{变精度计算}
注意变精度计算功能往往和符号计算一同使用, 但该功能本身却可以独立使用. 变精度计算本质上还是数值浮点计算, 计算过程存在数值误差. 只不过我们可以规定每个变量的有效数字为任意多, 而不是统一使用双精度类型的 15-16 位有效数字.

\begin{itemize}
\item 相比于 \verb|eval()| 返回双精度结果, \verb|vpa()| 可以计算符号表达式的任意位有效数字结果, 例如 \verb|vpa(sqrt(sym('2')), 50)| 计算 $\sqrt{2}$ 的 50 位有效数字, 并能保证四舍五入后最后一位有效数字正确. 返回的值是一个 \verb|'sym'| 类型的对象而不是 \verb|'double'| 类型.

\item 虽然 \verb|vpa| 返回的类型也是 \verb|'sym'|, 但它是有误差的. 我们姑且把这种有误差的 \verb|sym| 数字称为\textbf{变精度浮点数}. 我们可以把变精度浮点数当作 \verb|double| 类型的拓展. \verb|double| 在十进制下只有约 15-16 位有效数字, 而变精度浮点数的有效数字位数可以任意指定.

\item 要检查一个 \verb|sym| 类型的对象 \ver|x| 是否有误差, 只需要在命令行中把它显示出来(可以使用 \verb|disp()| 函数, 也可以直接运行 \verb|x| 不加分号). 如果显示中出现了小数点, 那么他就是变精度浮点数.

\item 除了 \verb|vpa()| 函数会输出变精度浮点数, 另一种方法如 \verb|sym('12.3')| 或者 \verb|sym('1.23e1')|, 运行后显示结果为 \verb|12.3|, 存在小数点, 所以是变精度浮点数. 相比之下, \verb|sym('123/10')| 结果显示为 $123/10$ 该数无论参与任何计算都是绝对精确的.

\item 和双精度变量一样, 若把变精度浮点数进行运算, 则其本身的误差会在运算中传递, 且在计算中会产生新的截断误差. 举例: \verb|1 + sym('1e-40') - 1| 的结果显示为 \verb|0.0|, 而 \verb|1 + sym(10)^(-40) - 1| 的结果显示 \verb|1/10000....|(40 个 0).

\item 变精度浮点数的运算的默认位数可以用 \verb|digits(整数)| 设置, 如果不设置, 则默认是 32 位. 也可以使用不含自变量的 \verb|digits()| 查看当前设置的位数.

\item 无论参与变精度运算的数有多少个有效数字, 运算结果取当前的默认位数. 例如设置 \verb|digits(32)| 后, \verb|vpa(sym(pi)/2, 50) + vpa(sym(pi)/2, 100)| 的结果仍然是 32 位有效数字.

\item 若要把默认位数改为 50 位, 则运行 \verb|digits(50)|, 再运行 \verb|1 + sym('1e-40') - 1| 就会得到 \verb|0.000...000999...99938892...|, 这就比默认的 32 位有效数字精确多了, 但还是存在误差.

\item \verb|digits()| 也可以用于获取当前的默认效数字.

\item 变精度浮点数可以和解析表达式混合使用, 例如 \verb|sym('1.2')*sqrt(2*sym('x'))|, 结果是 $1.2 \sqrt{2x}$, 其中 $1.2$ 是不精确的. 相比之下, \verb|sym(12/10)*sqrt(2*sym('x'))| 得到精确的 $6\sqrt{2x}/5$.
\end{itemize}


\subsection{特殊函数}
由于一些特殊函数若用双精度计算,往往在一些区间产生较大误差. 所以 Matlab 决定只使用任意精度来计算. 以复数域的 $\Gamma$ 函数\upref{Gamma} 为例, 若直接输入 \verb|gamma(1+1i)| 则会出错, 因为 Matlab 中非符号计算版本的 \verb|gamma| 函数只支持 \verb|double| 类型的实数输入. 所以要调用符号计算工具箱提供的 \verb|gamma()|, 就输入一个 \verb|sym| 类型的变量即可. 例如 \verb|gamma(vpa(1+1i))| 返回 \verb|0.49801... - 0.15494...i|, 又例如 \verb|gamma(sym(1+1i))| 返回表达式 \verb|gamma(1 + 1i)|. 这是因为 \verb|vpa(1+1i)| 是变精度浮点数, 存在误差, 而 \verb|sym(1+1i)| 不存在误差. 当然, 我们也可以用例如 \verb|vpa(gamma(sym(1+1i)), 50)| 来求表达式的任意有效数字. 但即使使用 \verb|gamma(vpa(1+1i, 50))| 结果也只有 \verb|digits()| 位有效数字.
