% 2006 年中科院普通物理真题 A 卷(甲)
% keys 中科院|物理|考研
% license Copy
% type Tutor

\begin{enumerate}
\item 一带有电荷为$q$、质量为$m$的小球,悬于一不带电绝缘细丝线一端。线的另一端与一无穷大竖直带电导体平板相连,处于平衡状态时细线与平板成$30$°角,如图1所示。 试问此时带电平板的表面电荷密度 $\sigma$ 为多少?
\begin{figure}[ht]
\centering
\includegraphics[width=6cm]{./figures/74b710ea7da90c3f.png}
\caption{} \label{fig_ZKYP06_1}
\end{figure}
\item 如图2所示,电源电动势  $\varepsilon_1=3V,\varepsilon_2=12V$ ,其内阻均可忽略。 $R_1=8\Omega,R_2=4.4\Omega,R_3=2\Omega$  。求:\\
(1)$K$ 断开时,$A$  点的电势  $V_A$=?\\
(2)$K$合上后,$A$  点的电势又为多少?
\begin{figure}[ht]
\centering
\includegraphics[width=8cm]{./figures/a13c2fce2c72f7ec.png}
\caption{} \label{fig_ZKYP06_2}
\end{figure}
\item 延长线过圆心的两根长直导线与均匀金属圆环相接于  两点,ru't
\end{enumerate}