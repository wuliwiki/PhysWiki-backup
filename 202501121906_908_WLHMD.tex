% 威廉·哈密顿(综述)
% license CCBYSA3
% type Wiki

本文根据 CC-BY-SA 协议转载翻译自维基百科\href{https://en.wikipedia.org/wiki/William_Rowan_Hamilton}{相关文章}。

威廉·罗温·汉密尔顿爵士(1805年8月4日-1865年9月2日)是爱尔兰的数学家、物理学家和天文学家。他曾担任都柏林三一学院的天文学安德鲁斯教授。

汉密尔顿是1827年至1865年间邓辛克天文台的第三任台长。他的职业生涯包括对几何光学、傅里叶分析和四元数的研究,其中四元数使他成为现代线性代数的奠基人之一。他在光学、经典力学和抽象代数方面做出了重要贡献。他的工作是现代理论物理的基础,特别是他对牛顿力学的重新表述。汉密尔顿力学,包括其汉密尔顿函数,现在在电磁学和量子力学中都占据着核心地位。