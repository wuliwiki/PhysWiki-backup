% Klein-Gordon方程
% 量子场论|高等量子力学|克莱因-戈登方程|克莱茵-戈登方程

\addTODO{预备知识需要薛定谔量子力学的相关内容,但现在该部分还未整理好,不宜引用.}

\pentry{自然单位制、普朗克单位制\upref{NatUni},爱因斯坦求和约定\upref{EinSum}}

Klein-Gordon方程(以下简称“K-G方程”)的命名源自两位物理学家Oskar Klein和Walter Gordon,他们于1926年指出该方程能够描述狭义相对论中的电子.虽说实际上,电子这样有自旋的费米子应该用Dirac方程来描述,但K-G方程依然成功地描述了相对论性的无自旋复合粒子.


\subsection{问题的引入}

Schrödinger方程(以下以通译称“薛定谔方程”)在量子力学中的地位,就像牛顿三定律在经典力学中的地位一样,是描述理论结构的“公理”.因此,如果要了解量子力学的局限性,可以从研究薛定谔方程本身入手.

\subsubsection{质能关系问题}

回顾单粒子薛定谔方程的表达(注意这里使用了\textbf{自然单位制}\upref{NatUni}):
\begin{equation}\label{KGeq_eq1}
\qty(-\frac{\nabla^2}{2m}+V)\psi = \I \partial_t \psi
\end{equation}

由于量子力学假设$\hat{p}=-\I\nabla$和$\hat{E}=\I\partial_t$分别是动量、能量算子,故\autoref{KGeq_eq1} 左边体现的是经典力学中的\textbf{哈密顿量}:

\begin{equation}\label{KGeq_eq2}
\begin{aligned}
\hat{H}=-\frac{\nabla^2}{2m}+V &= \frac{\hat{p}^2}{2m}+V \\
&\updownarrow\\
H &= \frac{p^2}{2m}+V
\end{aligned}
\end{equation}

\autoref{KGeq_eq2} 上下两部分含义完全不同\footnote{上面一行各项是算符,它们作为量子态之间线性变换的\textbf{特征值}才是能量、动量等可观测量;下面一行各项就是实数,本身即为能量、动量等可观测量.},但其描述的能量-动量-质量关系是一致的.因此薛定谔方程本质上是经典力学的推广,与经典时空观契合,但与相对论时空观矛盾.

\subsubsection{粒子数守恒问题}

回顾量子力学的概率守恒.取薛定谔方程的复共轭,得

\begin{equation}\label{KGeq_eq3}
\qty(-\frac{\nabla^2}{2m}+V)\psi^* = -\I \partial_t \psi^*
\end{equation}

在\autoref{KGeq_eq1} 上乘以$\psi^*$,再减去\autoref{KGeq_eq3} 乘以$\psi$ ,得

\begin{equation}\label{KGeq_eq4}
\begin{aligned}
\psi^*\qty(-\frac{\nabla^2}{2m}+V)\psi - \psi\qty(-\frac{\nabla^2}{2m}+V)\psi^* &= \I\qty(\psi^*\partial_t\psi + \psi\partial_t\psi^*)\\
-\psi^*\frac{\nabla^2}{2m}\psi + \psi\frac{\nabla^2}{2m}\psi^* &= \I \partial_t\qty(\psi\psi^*)\\
\partial_t\qty(\abs{\psi}^2) + \frac{-\I\nabla}{2m}\qty(\psi^*\nabla\psi-\psi\nabla\psi^*) &= 0\\
\partial_t \rho + \frac{\nabla}{2m} \qty(\psi^*\hat{p}\psi + (\psi^*\hat{p}\psi)^*) &= 0
\end{aligned}
\end{equation}
其中$\rho=\abs{\psi}^2$可以理解为粒子的空间位置分布,即粒子数密度.

对于动量本征态容易验证,\autoref{KGeq_eq4} 相当于
\begin{equation}\label{KGeq_eq5}
\partial_t\rho + \nabla\cdot(\rho v) = 0
\end{equation}
其中$v=p/m=\psi^*\hat{p}\psi/m$是粒子的速度.对于非本征态也有类似的阐释,因为任何量子态都是动量本征态的叠加.

\autoref{KGeq_eq5} 意味着任意空间区域内粒子随时间增加的速率,恰为粒子从外部进入该区域的速率,即整个宇宙中粒子数守恒.于是,薛定谔方程无法描述粒子数变化的现象,如质子和电子结合成中子的过程中,质子和电子的数目减少,中子的数目增多.



\subsection{Klein-Gordon方程}

相对论的成功以及自然界广泛存在的粒子数改变的现象,都表明我们必须改变薛定谔方程的形式,才能扩展量子理论的适用范围.本小节介绍的Klein-Gordon方程即是一个良好的扩展.

\subsubsection{方程的导出}

我们考虑从质能关系切入.相对论中的质能关系为
\begin{equation}
E^2=p^2+m^2
\end{equation}
用它代替经典力学的$E=p^2/2m+V$,代入量子力学的算符假设$\hat{p}=-\I\nabla$和$\hat{E}=\I\partial_t$,得到一个方程:
\begin{equation}\label{KGeq_eq6}
\qty(-\nabla^2+m^2)\psi = -\partial_t^2 \psi
\end{equation}

如果使用抽象指标来表示,取$g_{\mu\nu}=\opn{diag}(-1, 1, 1, 1)$,则\autoref{KGeq_eq6} 也表达为
\begin{equation}\label{KGeq_eq7}
\partial_\mu\partial^\mu \psi = m^2\psi
\end{equation}

如果取$g_{\mu\nu}=\opn{diag}(1, -1, -1, -1)$,则\autoref{KGeq_eq6} 应表达为
\begin{equation}\label{KGeq_eq8}
\partial_\mu\partial^\mu \psi = -m^2\psi
\end{equation}
这仅仅是\textbf{号差}选择的习惯问题.

\autoref{KGeq_eq6} 及其抽象指标表达式\autoref{KGeq_eq7} 和\autoref{KGeq_eq8} 被称为\textbf{Klein-Gordon方程}.

%方程的解,概率密度问题,四元数形式,以及wiki上讨论过的其它问题


\subsubsection{连续性方程}

类似处理薛定谔方程的方法,我们给K-G方程左乘一个$\psi^*$,再取结果的复共轭,相减:
\begin{equation}\label{KGeq_eq9}
\begin{aligned}
\psi^*\qty(-\nabla^2+m^2)\psi-\psi\qty(-\nabla^2+m^2)\psi^* &= -\psi^*\partial_t^2 \psi+\psi\partial_t^2 \psi^*\\
-\psi^*\nabla^2\psi+\psi\nabla^2\psi^* &= -\psi^*\partial_t^2\psi+\psi\partial_t^2\psi^*\\
\nabla\cdot\qty(\psi^*\nabla\psi-\psi\nabla\psi^*) &= \partial_t\qty(\psi^*\partial_t\psi-\psi\partial_t\psi^*)\\
\partial_\mu\qty(\psi^*\partial^\mu\psi-\psi\partial^\mu\psi^*) & =0
\end{aligned}
\end{equation}

\autoref{KGeq_eq9} 的最后一步与$g_{\mu\nu}$的号差选择无关.

记$\rho=\frac{\I}{2m}\qty(\psi^*\partial_t\psi-\psi\partial_t\psi^*)$和$\bvec{j}=-\frac{\I}{2m}\qty(\psi^*\nabla\psi-\psi\nabla\psi^*)$,则\autoref{KGeq_eq9} 还可以写成
\begin{equation}
\partial_t\rho+\nabla\cdot\bvec{j} = 0
\end{equation}
这意味着$\bvec{j}$是$\rho$的流.

如果把$\rho$诠释为粒子数的密度显然是不妥当的,因为$\rho$可以取负值.



















