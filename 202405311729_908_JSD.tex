% Jensen-Shannon 散度
% keys Jensen-Shannon Divergence|JS散度
% license Xiao
% type Tutor

\pentry{KL 散度(相对熵)\nref{nod_KLD}}{nod_ee4f}

\textbf{JS 散度}(Jensen-Shannon Divergence,缩写 JSD)是基于 \enref{KL 散度(相对熵)}{KLD}的一种统计学度量,能够衡量两个概率分布之间的差异程度。

设概率空间上有两个概率分布$P$和$Q$,$M=\frac{1}{2}(P+Q)$,为$P$和$Q$的平均,则,$P$和$Q$的$JS$散度定义为
\begin{equation}
JSD(P||Q)=\frac{1}{2}D_{KL}(P||M)+\frac{1}{2}D_{KL}(Q||M)~,
\end{equation}
其中,$D_{KL}$表示$KL$散度。



\textbf{参考文献}
\begin{enumerate}
\item B. Fuglede and F. Topsoe, “Jensen-Shannon divergence and Hilbert space embedding,” in International Symposium onInformation Theory, 2004. ISIT 2004. Proceedings., 2004, p. 31–.
\end{enumerate}