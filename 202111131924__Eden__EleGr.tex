% 格林函数与静电边值问题
% 格林函数|静电学|边值问题|泊松方程

\pentry{静电势的泊松方程\upref{EPoiEQ}}

静电学问题中有一类非常常见的边值问题,例如已知接地导体的空腔中有电荷分布,求空腔内的电势;求带电导体在空间中产生的电场等.使用格林定理可以帮助我们高效率地计算边值问题.

设真空中的电荷分布为 $\rho(\bvec r)$,则空间中的静电势满足泊松方程
\begin{equation}\label{EleGr_eq1}
\nabla^2 \phi = -\frac{\rho}{\epsilon_0}
\end{equation}

定义函数 $\psi(\bvec r,\bvec r')$:
\begin{equation}
\psi(\bvec r,\bvec r')=\frac{1}{|\bvec r-\bvec r'|}
\end{equation}
它则满足以下泊松方程(下式中 $\nabla'^2$ 代表以 $r'$ 为自变量的拉普拉斯算子)
\begin{equation}\label{EleGr_eq2}
\nabla'^2 \psi(\bvec r,\bvec r')=-4\pi\delta(\bvec r-\bvec r')
\end{equation}

\begin{theorem}{格林定理}
设 $\phi,\psi$ 为区域 $V$ 上的标量函数,则通常有
\begin{equation}\label{EleGr_eq3}
\int_V (\phi \nabla^2 \psi -\psi\nabla^2\phi)\dd V=\int_{\partial V}(\phi \nabla \psi-\psi\nabla\phi)\dd {\bvec S} 
\end{equation}
\end{theorem}
\textbf{证明:} 由高斯定理,$\int_V \nabla\cdot \bvec F\dd V=\int_{\partial V}\bvec F \dd {\bvec S}$,
\begin{equation}
\begin{aligned}
\int_{\partial V}&(\phi\nabla\psi-\psi\nabla\phi)\dd{\bvec S}=\int_V \nabla \cdot (\phi\nabla\psi-\psi\nabla\phi)\dd V\\
&=\int_V((\nabla\phi)\cdot(\nabla\psi)+\phi\nabla^2\psi-(\nabla\psi)\cdot(\nabla\phi)-\psi\nabla^2\phi)\dd V\\
&=\int_V(\phi\nabla^2\psi-\psi\nabla^2\phi)\dd V
\end{aligned}
\end{equation}

下面我们用格林定理来解决静电边值问题.下面的体积分以及面积分都是以 $r'$ 为自变量作积分.将 \autoref{EleGr_eq1} \autoref{EleGr_eq2} 中的 $\phi,\psi$ 代入 \autoref{EleGr_eq3}  中,等式左边为
\begin{equation}
\begin{aligned}
&\int_V(\phi(\bvec r')\nabla'^2\psi(\bvec r,\bvec r')-\psi(\bvec r,\bvec r')\nabla'^2\phi(\bvec r'))\dd V
\\
&=\int_V(-\phi(\bvec r')4\pi \delta(\bvec r-\bvec r')+\psi(\bvec r,\bvec r')\frac{1}{\epsilon_0}\rho(\bvec r'))\dd V\\
&=-4\pi\phi(\bvec r)+\frac{1}{\epsilon_0}\int_V\frac{\rho(\bvec r')}{|\bvec r-\bvec r'|}\dd V
\end{aligned}
\end{equation}
等式右边为
\begin{equation}
\begin{aligned}
&\int_{\partial V}(\phi(\bvec r')\nabla'\psi(\bvec r')-\psi(\bvec r')\nabla'\phi(\bvec r'))\cdot \dd {\bvec S}\\
&=\int_{\partial V}\qty(\phi(\bvec r')\nabla'\frac{1}{|\bvec r-\bvec r'|}-\frac{1}{|\bvec r-\bvec r'|}\nabla'\phi(\bvec r'))\dd {\bvec S}
\end{aligned}
\end{equation}
因此有
\begin{equation}
\phi(\bvec r)=\frac{1}{4\pi\epsilon_0}\int_V\frac{\rho(\bvec r')}{|\bvec r-\bvec r'|}\dd V+\frac{1}{4\pi}\int_{\partial V}\qty(\frac{1}{|\bvec r-\bvec r'|}\nabla'\phi(\bvec r')-\phi(\bvec r')\nabla'\frac{1}{|\bvec r-\bvec r'|})\dd {\bvec S}
\end{equation}
上式中第一项描述了区域内电荷密度对电势的贡献,第二项描述了区域边界上的面电荷密度对区域内电势的贡献.

不难看出,为了得到 $V$ 内部的 $\phi(\bvec r)$,我们需要同时代入 $\phi$ 和 $\partial \phi/\partial n$ 在边界上的取值.但一般情况下我们只知道其中之一,如果同时给定 $\phi$ 和 $\partial \phi/\partial n$ 在边界上的取值会因过约束而导致矛盾.根据静电势的唯一性定理(在词条\upref{EPoiEQ}中有定理的描述和证明),只要知道 $\phi$ 或 $\partial \phi/\partial n$ 其中之一在边界上的取值,就可以得出唯一的电势分布.下面我们分别对它们进行讨论.

\subsection{Dirichlet边界条件下的Green函数}

我们需要找到一个新的函数来替换之前的 $\psi$.定义 $G_D(\bvec r,\bvec r')$ 满足以下条件:
\begin{equation}\label{EleGr_eq6}
\begin{cases}
\nabla'^2G_D(\bvec r,\bvec r')=-4\pi\delta(\bvec r-\bvec r')\\
G_D(\bvec r,\bvec r')|_{\partial V}=0
\end{cases}
\end{equation}

我们称 $G_D(\bvec r,\bvec r')$ 为格林函数.用 $G_D$ 代替 $\psi$ 代入格林定理,可以得到
\begin{equation}\label{EleGr_eq4}
\phi(\bvec r)=\frac{1}{4\pi\epsilon_0}\int_V \rho(\bvec r')G_D(\bvec r,\bvec r')\dd V-\frac{1}{4\pi}\int_{\partial V} \phi(\bvec r')\nabla'G_D(\bvec r,\bvec r')\dd {\bvec S}
\end{equation}

对于某个特定区域形状($\partial V$ 的形状),我们可以预先求出它的格林函数 $G_D(\bvec r,\bvec r')$.在求解具体问题时,给定任意的电荷分布 $\rho(\bvec r')$ 以及 Dirichlet 边界条件 $\phi(\bvec r')|_{\partial V}$ ,都可以用\autoref{EleGr_eq4} 求出电势分布.这就是格林函数方法的高效之处\footnote{大多数情况下手算仍是极其复杂的,但用计算机可以方便求解.}.

大部分情况下手动解出格林函数是极其困难的,但对于一些特殊的边界,有很漂亮的格林函数.例如下面这个例子.

\begin{example}{无限大导体平面的 Dirichlet Green 函数}
设三维空间中 $z=0$ 下方是导体,$z=0$ 上方是真空.在 $(0,0,a)(a>0)$ 处有一个电荷量为 $+q$ 的电荷.求 $z=0$ 上方空间的电势分布.\footnote{这个问题也可以用电像法来分析.}

区域 $V$ 为 $z=0$ 上方空间.导体给出了边界条件 $\phi(\bvec r')|_{\partial V}=0$.用 Dirichlet 格林函数方法来求解.设 $\bvec r_+=(x,y,z),\bvec r_-=(x,y,-z)$,容易构造出以下函数
\begin{equation}
G_D(\bvec r_+,\bvec r')=\frac{1}{|\bvec r_+-\bvec r'|}-\frac{1}{|\bvec r_--\bvec r'|}
\end{equation}
它在区域 $V$ 内满足 Dirichlet 格林函数的两个条件.然后就可以利用\autoref{EleGr_eq4} 求解.设 $\bvec r_1=(0,0,a),\bvec r_2-=(0,0,-a)$,得
\begin{equation}
\phi(\bvec r)=\frac{1}{4\pi \epsilon_0}\frac{q}{|\bvec r-\bvec r_1|}-\frac{1}{4\pi \epsilon_0}\frac{q}{|\bvec r-\bvec r_2|}
\end{equation}

\end{example}
\begin{example}{球形边界的 Dirichlet Green 函数}
设空间内有两个半径为 $a$ 的金属半球壳,将它们放置成一个球体,球心位于原点.上半球($z>0$)球面的电势为 $V$,下半球球面电势为 $-V$.求球内电势分布,并写出电势 $\phi(0,0,z)$ 的解析表达式.

区域 $V$ 为以 $(0,0,0)$ 为圆心,$a$ 为半径的球.可以构造 Dirichlet 格林函数:
\begin{equation}
G_D(\bvec r,\bvec r')=\frac{1}{|\bvec r-\bvec r'|}-\frac{a/r'}{|\bvec r-(a^2/r'^2)\bvec r'|}
\end{equation}

容易知道 $\nabla'^2G_D(\bvec r,\bvec r')=-4\pi\delta(\bvec r-\bvec r')$.再考虑该函数在边界上的取值.$\bvec r',(a^2/r'^2)\bvec r'$ 同向,对从原点指向球的任意向量 $\bvec r^*$,设它和 $\bvec r'$ 的夹角为 $\theta$,那么可以用余弦定理化简:
\begin{equation}
\begin{aligned}
&\frac{1}{|\bvec r^*-\bvec r'|}-\frac{a/r'}{|\bvec r^*-(a^2/r'^2)\bvec r'|}
\\&=\frac{1}{\sqrt{a^2+r'^2-2ar'\cos\theta}}-\frac{a/r'}{\sqrt{a^2+a^4/r'^2-2a^3/r'\cos\theta}}
\\&=\frac{1}{\sqrt{a^2+r'^2-2ar'\cos\theta}}-\frac{1}{\sqrt{a^2+r'^2-2ar'\cos\theta}}
\\&=0
\end{aligned}
\end{equation}
于是该函数完全满足\autoref{EleGr_eq6} 的两个条件.

下面来解决原问题.利用格林函数方法,可以写出以下公式
\begin{equation}
\begin{aligned}
\phi(\bvec r)&=-\frac{V}{4\pi}\int_{\text{上半球壳}}\nabla'G_D(\bvec r,\bvec r')\dd{\bvec S}+\frac{V}{4}\int_{\text{下半球壳}}\nabla'G_D(\bvec r,\bvec r')\dd {\bvec S}\\
&=-\frac{Va^2}{4\pi}\int_{0}^{\pi/2}\dd \theta' \int_{0}^{2\pi}\sin\theta' \dd \phi'\qty(\frac{\partial }{\partial n'}\frac{1}{|\bvec r-\bvec r'|})\\
&+\frac{Va^2}{4\pi}\int_{\pi/2}^{\pi}\dd \theta' \int_{0}^{2\pi}\sin\theta' \dd \phi'\qty(\frac{\partial }{\partial n'}\frac{1}{|\bvec r-\bvec r'|})
\\
&=\frac{Va^2}{4\pi}\int_0^{\pi/2}\dd \theta'\int_0^{2\pi}\sin\theta'\dd \phi'\qty( \frac{r'-a\cos\gamma}{(a^2+r'^2-2ar'\cos\gamma)^{3/2}} )
\\& +
\frac{Va^2}{4\pi}\int_{\pi}^{\pi/2}\dd \theta'\int_0^{2\pi}\sin\theta'\dd \phi'\qty( \frac{r'-a\cos\gamma}{(a^2+r'^2-2ar'\cos\gamma)^{3/2}} )
\end{aligned}
\end{equation}
上式中 $\gamma$ 为 $\bvec r'$ 与 $\bvec r$ 的夹角.可以用中心对称关系将两个积分合并为一个积分(在中心对称的变换下,$\gamma$ 变为 $\pi-\gamma$,$\cos\gamma$ 反号),

\end{example}

\subsection{Neumann 边界条件下的 Green 函数}
定义 $G_N(\bvec r,\bvec r')$ 满足以下条件:
\begin{equation}\label{EleGr_eq5}
\begin{cases}
\nabla'^2G_N(\bvec r,\bvec r')=-4\pi\delta(\bvec r-\bvec r')\\
\left.\frac{\partial G_N(\bvec r,\bvec r')}{\partial n}\right|_{\partial V}=const
\end{cases}
\end{equation}
第二个条件之所以写 $const$ 而不写 $0$,是因为这将违反高斯定律.我们可以根据高斯定律计算这个 $const$ 的值:
\begin{equation}
\begin{aligned}
&\int_V \nabla'^2 G_N(\bvec r,\bvec r')\dd V=\int_{\partial V}\frac{\partial G_N(\bvec r,\bvec r')}{\partial n}\dd S=\int_{\partial V}const \dd S=const\cdot A\\
&\int_V \nabla'^2 G_N(\bvec r,\bvec r')\dd V=-4\pi\\
&\Rightarrow const=\frac{-4\pi}{A}
\end{aligned}
\end{equation}
其中 $A$ 为 $\partial V$ 的表面积.所以可以把\autoref{EleGr_eq5} 改写为
\begin{equation}
\begin{cases}
\nabla'^2G_N(\bvec r,\bvec r')=-4\pi\delta(\bvec r-\bvec r')\\
\left.\frac{\partial G_N(\bvec r,\bvec r')}{\partial n}\right|_{\partial V}=-\frac{4\pi}{A}
\end{cases}
\end{equation}
代入格林定理后有
\begin{equation}
\phi(\bvec r)=\frac{1}{4\pi\epsilon_0}\int_V \rho(\bvec r')G_N(\bvec r,\bvec r')\dd V+\frac{1}{4\pi}\int_{\partial V} G_N(r,r')\nabla'\phi(\bvec r')\dd {\bvec S}+\frac{1}{A}\int_{\partial V}\phi(\bvec r')
\end{equation}

注意,上式的最后一项只是一个常数,而静电势函数在 Neumann 边界问题中是无关紧要的(任意加减一个常数都仍然满足泊松方程和边界条件).所以可以直接将解写为
\begin{equation}
\phi(\bvec r)=\frac{1}{4\pi\epsilon_0}\int_V \rho(\bvec r')G_N(\bvec r,\bvec r')\dd V+\frac{1}{4\pi}\int_{\partial V} G_N(r,r')\nabla'\phi(\bvec r')\dd {\bvec S}
\end{equation}
