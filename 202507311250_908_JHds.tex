% 结合代数(综述)
% license CCBYSA3
% type Wiki

本文根据 CC-BY-SA 协议转载翻译自维基百科\href{https://en.wikipedia.org/wiki/Associative_algebra}{相关文章}。

在数学中,交换环(通常是一个域) $K$ 上的结合代数 $A$ 是一个环 $A$,并且带有一个从 $K$ 到 $A$ 的中心(center)的环同态。因此,它是一种代数结构,包含加法、乘法和数量乘法(即由 $K$ 中元素通过环同态的像所定义的乘法)。加法和乘法运算共同赋予 $A$ 环的结构;加法和数量乘法运算共同赋予 $A$ $K$-模或向量空间的结构。在本文中,我们也使用 $K$-代数 一词来指代 $K$ 上的结合代数。

$K$-代数的一个标准例子是定义在交换环 $K$ 上的方阵环,采用通常的矩阵乘法。

一个交换代数是乘法交换的结合代数,或者等价地,是同时也是交换环的结合代数。

在本文中,假设结合代数都有一个乘法单位元,记作 $1$;为强调这一点,有时称为有单位结合代数。在数学的一些领域中不做这一假设,这类结构称为非有单位结合代数。我们也假设所有的环都是有单位的,并且所有的环同态都是保单位元的。

每个环都是它的中心上的结合代数,也是整数环 $\mathbb{Z}$ 上的结合代数。
\subsection{定义}
设 $R$ 是一个交换环(因此 $R$ 也可以是一个域)。一个结合的 $R$-代数 $A$(或更简单地称为 $R$-代数 $A$)是一个环 $A$,并且同时是一个 $R$-模,满足环加法与模加法是同一个运算,并且数量乘法满足
$$
r \cdot (xy) = (r \cdot x)y = x(r \cdot y)~
$$
对所有 $r \in R$ 和代数中的 $x, y$ 成立。(这个定义意味着代数作为一个环是有单位的,因为假设环必须有乘法单位元。)

等价地,一个结合代数 $A$ 是一个环,并且带有一个从 $R$ 到 $A$ 的中心的环同态。如果 $f$ 是这样的同态,则数量乘法为$(r, x) \mapsto f(r) x$(此处乘法是环乘法);如果给定了数量乘法,则该环同态由$r \mapsto r \cdot 1_A$给出。(另见下文“由环同态导出”一节。)每个环都是一个结合的 $\mathbb{Z}$-代数,其中 $\mathbb{Z}$ 表示整数环。

一个交换代数是一个乘法交换的结合代数,或者等价地,是一个同时也是交换环的结合代数。
