% 集合(公理化)
% 集合|公理|公理系统
\begin{issues}
\issueTODO
\end{issues}

\pentry{集合\upref{Set}}
\subsection{产生原因}
通常来说,我们都采取朴素集合论的观点,认为集合是一个最基本的数学概念,不需要严格的定义.但是,\textbf{罗素悖论(antinomy of Russell)}使得这一观念受到挑战.罗素悖论可以叙述为:已知对于任何一个集合,都能判定自己是否属于这个集合.所以,我们可以将不属于自己的集合收集起来构成一个集合.因此就应该存在一个集合$\mathcal{A}=\{A|A\notin A\}$,其中$A$是一个集合.但是,如果认为$\mathcal{A}\in\mathcal{A}$,那么根据定义,$\mathcal{A}\notin\mathcal{A}$;反过来,如果认为$\mathcal{A}\notin\mathcal{A}$,根据定义又有$\mathcal{A}\in\mathcal{A}$.这就构成一个悖论.

罗素悖论的存在使人们认识到,朴素集合论并不像他们想象中的那么严谨.为了解决罗素悖论,人们利用公理对集合进行一系列限制,从而使得在新的公理系统中无法构造出罗素悖论中的集合.

\subsection{ZF公理系统}


\subsection{选择公理和ZFC公理系统}

\subsection{连续统假设}



