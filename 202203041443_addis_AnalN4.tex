% Rudin 实分析与复分析笔记

\begin{itemize}
\item (a) 对每一个复数 $z, e^{z} \neq 0$. (b) $\exp$ 的导数是它自己. (c) $\exp$ 限制在实轴上是单调增加的正函数, 且当 $x \rightarrow \infty$ 时, $\mathrm{e}^{x} \rightarrow \infty$; 当 $x \rightarrow-\infty$ 时, $\mathrm{e}^{x} \rightarrow 0$.
(d) 存在一个正数 $\pi$ 使得 $\mathrm{e}^{\pi / 2}=\mathrm{i}$, 并使得 $\mathrm{e}^{z}=1$ 当且仅当 $z /(2 \pi \mathrm{i})$ 是整数.(e) $\exp$ 是周期函数,其周期是 $2 \pi \mathrm{i}$. (f) 映射 $t \rightarrow \mathrm{e}^{\mathrm{i}}$ 将实轴映到单位圆上. (g)若 $w$ 是复数且 $w \neq 0$, 则存在某个 $z$ 使 $w=\mathrm{e}^{z}$.

\item 1.2 (a) 集 $X$ 的子集族 $\tau$ 称为 $X$ 上的一个拓扑, 若 $\tau$ 具有如下三个性质:(i) $\varnothing \in \tau$ 及 $X \in \tau$. (ii) 若 $V_{1} \in \tau, i=1, \cdots, n$, 则 $V_{1} \cap V_{2} \cap \cdots \cap V_{n} \in \tau$. (iii) 若 $\left\{V_{a}\right\}$ 是由 $\tau$ 的元素构成的集族 (有限、可数或不可数), 则 $\bigcup_{a} V_{a} \in \tau$. (b) 若 $\tau$ 是 $X$ 上的拓扑, 则称 $X$ 为一个拓扑空间, 且 $\tau$ 的元素称为 $X$ 的开集. (c) 若 $X$ 和 $Y$ 为拓扑空间, 且 $f$ 是 $X$ 到 $Y$ 内的映射, 而对 $Y$ 的每一个开集 $V, f^{-1}(V)$ 是 $X$ 的开集, 则称 $f$ 为连续的.

\item 1.3 (a) 集 $X$ 的子集族 $\mathfrak{M}$ 称为 $X$ 的一个 $\sigma$-代数, 若 $\mathfrak{M}$ 具有如下性质: (i) $X \in \mathfrak{M}$. (ii) 若 $A \in \mathfrak{M}$, 则 $A^{c} \in \mathfrak{M}$, 其中 $A^{c}$ 是 $A$ 关于 $X$ 的余集. (iii) 若 $A=\bigcup_{n=1}^{\infty} A_{n}$ 且 $A_{n} \in \mathfrak{M}, n=1,2,3, \cdots$, 则 $A \in \mathfrak{M}$. (c) 若 $X$ 是可测空间, $Y$ 是拓扑空间, $f$ 是 $X$ 到 $Y$ 内的映射,而对 $Y$ 的每一个开集 $V$, $f^{-1}(V)$ 是 $X$ 的可测集, 则 $f$ 称为可测的.

\item 1.5 设 $X$ 和 $Y$ 是拓扑空间, $f$ 是 $X$ 到 $Y$ 内的映射. 当且仅当 $f$ 在 $X$ 的每一点连续时, 映射 $f$ 是连续的.

\item 1.7 设 $Y$ 和 $Z$ 为拓扑空间, 且 $g: Y \rightarrow Z$ 是连续的. (a) 若 $X$ 是拓扑空间, $f: X \rightarrow Y$ 是连续的, 且 $h=g \circ f$, 则 $h: X \rightarrow Z$ 是连续的. (b) 若 $X$ 是可测空间, $f: X \rightarrow Y$ 是可测的, 且 $h=g \circ f$, 则 $h: X \rightarrow Z$ 是可测的. 简言之, 连续函数的连续函数是连续的; 可测函数的连续函数是可测的.

\item 1.10 若 $\mathscr{F}$ 为 $X$ 的任意子集族, 则在 $X$ 内存在一个最小的 $\sigma$-代数 $\mathfrak{M}^{*}$, 使得 $M \subset M^{*}$. $\mathfrak{M}^{*}$ 有时称为由 $\mathscr{F}$ 生成的 $\sigma$-代数.

\item 1.30 我们定义 $L^{1}(\mu)$ 是所有使得 $\int_{X}|f| \mathrm{d} \mu<\infty$ 的、 $X$ 上的复可测函数 $f$ 的集族.
\end{itemize}
