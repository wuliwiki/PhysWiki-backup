% 华中师范大学 2012 年 考研 量子力学
% license Usr
% type Note

\textbf{声明}:“该内容来源于网络公开资料,不保证真实性,如有侵权请联系管理员”

\subsection{选择题(共 18分,每小题3分)}

1.在给定的状态 $\psi(x,t)$ 中 “测量” 粒子的坐标$\underline{\hspace{2cm}}$\\
    A、测量” 使得 $\psi(x,t)$ 不再按照薛定谔方程演化\\
    B、测量” 使得微观粒子不在任何位置\\
    C、“测量” 使得微观粒子的坐标越精确动量就越精确\\
    D、“测量” 使得 $\psi(x,t)$ 突然和不连续的坍塌
   
2.坐标对时间导数的算符是$\underline{\hspace{2cm}}$\\
    A、坐标算符 \\
    B、动量算符 \\
    C、速度算符 \\
    D、角动量算符\\

    
\item  设质量为 $m$ 粒子的两个本征函数分别是 $\psi_1(x) = c_1e^{-ax^2/2}$,$\psi_2(x) = c_2(x^2+b)e^{-ax^2}$,则粒子这两状态的能级间隔为(\\ \\ \\ \\ )
    \begin{enumerate}
    \item $-\frac{\hbar^2}{mb}$
    \item $-\frac{\hbar^2}{(mb)^2}$
    \item $-\frac{\hbar}{mb}$
    \item $\frac{\hbar^2}{(mb)^2}$
    \end{enumerate}

\item  对于任意的 $\mathbf{a}$,若 $(\mathbf{a}|\mathbf{b}) = (\mathbf{a}|\mathbf{c})$,则(\\ \\ \\ \\ )
    \begin{enumerate}
    \item $\mathbf{b} \ne \mathbf{c}$
    \item $\mathbf{b} = \mathbf{c}$
    \item $(\mathbf{a}| \mathbf{b}) = (\mathbf{a}| \mathbf{c})$
    \item $(\mathbf{a}| = \mathbf{b})$
    \end{enumerate}

\item  在一维情况下,若 $U(x) \text{连续}, U(\pm \infty) = 0$ 且 $U(x) < 0$,则该体系(\\ \\ \\ \\ )
    \begin{enumerate}
    \item 两个束缚态
    \item 无束缚态
    \item 一个束缚态
    \item 至少存在一个束缚态
    \end{enumerate}
    
\item  微观体系存在任意态 $|\mathbf{n}\rangle$ 中,能量的平均值 $\bar{E}$(\\ \\ \\ \\ )
    \begin{enumerate}
    \item 体系的基态能量 $E$
    \item 没有确定值
    \item $\geq$ 体系的基态能量 $E$
    \item = 体系的基态能量 $E$
    \end{enumerate}

\end{enumerate}
