% python符号计算
% keys 符号计算|python

\subsection{什么是符号计算?}
符号计算又称计算机代数,通俗地说就是用计算机推导数学公式,如对表达式进行因式分解、化简、微分、积分、解代数方程、求解常微分方程等.在SciPy 数值微分与积分\upref{SciPy}部分我们已经介绍了如何利用\verb|python|实现相关数值计算,这里面我们将进一步介绍符号计算在\verb|python|中的实现.那么数值计算与符号计算有什么区别与联系呢?个人认为:首先在数值计算过程中,所有出现的\verb|变量|或者\verb|参数|在使用之前必须给定具体取值,并且计算结果大多数是近似的;相反的是,在符号计算过程中,变量可以预先不给定取值,计算结果是准确的,解析的.不太准确的表述为:符号计算就是对表达式进行的操作;数值计算是对数据进行的操作.

\subsection{\verb|scipy|库}
在\verb|python|中,专门进行符号计算的库是\verb|sympy|(symbol python的简写).利用这个库可以进行符号表达式的加减乘除等四则运算、符号化简、求导、积分、极限、解方程(组)、解微分方程(组)等等.下面我们将进行逐一介绍.

\subsubsection{符号变量的定义}


