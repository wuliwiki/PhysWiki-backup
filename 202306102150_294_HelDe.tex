% 亥姆霍兹分解 2
% 亥姆霍兹分解|光滑矢量场|标量函数

\begin{issues}
\issueDraft
\end{issues}

\pentry{矢量分析总结\upref{VecAnl}}

在矢量分析中, 三维空间中的\textbf{亥姆霍兹分解(Helmholtz decomposition)} 可在给定边界条件的情况下, 将一个矢量场唯一地分解为无旋场和无散场的和。 它在流体力学和电磁学中皆有应用。

\subsection{全空间的亥姆霍兹分解(傅里叶形式)}
设 $F(x)$ 是三维空间 $\mathbb{R}^3$ 上的光滑矢量场, 满足
$$
\int_{\mathbb{R}^3}|F(x)|^2dx<\infty~.
$$
则它的傅里叶变换 $\hat F$ 是良好定义的。 由此可定义一个标量函数
$$
\begin{aligned}
\Phi(x)
&=\int_{\mathbb{R}^3}\frac{i\xi\cdot\hat F(\xi)}{|\xi|^2}e^{i\xi\cdot x}d\xi\\
&=\frac{1}{4\pi}\int_{\mathbb{R}^3}\frac{\nabla_y\cdot F(y)}{|x-y|}dy\\
&=-\frac{1}{4\pi}\int_{\mathbb{R}^3}\frac{(x-y)\cdot F(y)}{|x-y|}dy~.
\end{aligned}
$$
和一个矢量场
$$
\begin{aligned}
A(x)
&=\int_{\mathbb{R}^3}\frac{i\xi\times\hat F(\xi)}{|\xi|^2}e^{i\xi\cdot x}d\xi\\
&=\frac{1}{4\pi}\int_{\mathbb{R}^3}\frac{\nabla_y\times F(y)}{|x-y|}dy\\
&=-\frac{1}{4\pi}\int_{\mathbb{R}^3}\frac{(x-y)\times F(y)}{|x-y|}dy~.
\end{aligned}
$$
则有
$$
F=-\nabla\Phi+\nabla\times A~.
$$
这就是矢量场 $F$ 的亥姆霍兹分解。

\subsection{区域的亥姆霍兹分解(强形式)}

\subsection{区域的亥姆霍兹分解(弱形式)}
