% 斯特藩—玻尔兹曼定律
% keys 斯特藩|玻尔兹曼|黑体辐射|zeta 函数|辐射功率

\textbf{斯特藩—玻尔兹曼定律(Stefan-Boltzmann law)}指的是, 黑体单位面积的辐射功率与温度的 4 次方成正比
\begin{equation}\label{SteBol_eq2}
\dv{P}{S} = \sigma T^4
\end{equation}
其中 $\sigma$ 是\textbf{斯特藩—玻尔兹曼常数(Stefan-Boltzmann constant)}
\begin{equation}
\sigma = \frac{2\pi^5k_B^4}{15c^2h^3} = 5.670374419\dots \e{-8} \Si{Wm^{-2}K^{-4}}
\end{equation}
其中 $k_B$ 是玻尔兹曼常数, $c$ 是真空中的光速, $h$ 是普朗克常数, 详见 “物理学常数\upref{Consts}”.

\subsection{从黑体辐射定律推导}
\pentry{黑体辐射定律\upref{BBdLaw}}

我们考虑黑体表面一个小平面, 以它的法向量(指向真空)为极轴建立球坐标系. 单位面积单位频率单位立体角的功率见(\autoref{BBdLaw_eq2}~\upref{BBdLaw})
\begin{equation}\label{SteBol_eq1}
I(\nu) = \frac{2h}{c^2} \frac{\nu^3}{\E^{h\nu/(k_B T)} - 1}
\end{equation}
我们只需要对频率和上半球面的立体角做积分即可. 但是注意在不垂直黑体表面的方向, 功率需要乘以 $\cos\theta$(表面积在该方向的投影是 $\cos\theta$).
\begin{equation}
\dv{P}{S} = \int_0^\infty I(\nu) \dd{\nu} \int_0^{2\pi} \dd{\phi} \int_0^{\pi/2}\cos\theta\sin\theta\dd{\theta}
= \pi \int_0^\infty I(\nu) \dd{\nu}
\end{equation}
使用换元积分, 令 $u = h\nu/(k_BT)$ 得
\begin{equation}
\dv{P}{S} = \frac{2\pi h}{c^2} \qty(\frac{k_B T}{h})^4 \int_0^\infty \frac{u^3}{\E^u - 1}\dd{u}
\end{equation}
其中积分的结果可以用黎曼 $\zeta$ 函数% 链接未完成
表示为 $6\zeta(4) = \pi^4/15$. 代入即可得\autoref{SteBol_eq2} .% 未完成: 没有验证
