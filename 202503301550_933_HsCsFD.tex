% 圆锥曲线的统一定义(高中)
% keys 准线|第二定义|焦点|圆锥曲线|焦点-准线定义
% license Usr
% type Tutor

\begin{issues}
\issueDraft
\end{issues}

\pentry{圆锥曲线与圆锥\nref{nod_ConSec}}{nod_55cd}

古希腊时期,人们通过截取圆锥面来研究圆、椭圆、抛物线和双曲线等曲线。尽管这种方式直观,而且给予这些曲线同样的来源,但在研究各类曲线时,仍是分别对待,研究各自的性质。随着坐标系的引入,数学家们逐渐发现,这些看似不同的曲线,其实可以在引入一条直线后,通过一个简洁而优雅的定义统一描述。

这一定义不仅在代数和解析几何中揭示了圆锥曲线的本质,也在射影几何等更高层次的研究中带来了意想不到的收获。可惜的是,这一部分内容在高中阶段已经完全删除,为了带给读者全新的体验,以及开阔视野,本文将围绕这一统一视角,介绍圆锥曲线的几何构造与深层结构。

这一定义不仅在解析几何中揭示了圆锥曲线的本质,也在射影几何等更高层次的研究中带来了意想不到的收获。可惜的是,这部分内容在高中阶段已被完全删除。为带给读者更完整的视角与更新的体验,本文将从这一统一定义出发,探索圆锥曲线的几何构造与背后的深层结构。


\subsection{圆锥曲线的焦点-准线定义}

利用准线与焦点得到的。提供了一个统一的视角来看待

\textbf{圆锥曲线的焦点-准线定义(Focus-Directrix Definition of Conic Sections)}。

\begin{definition}{圆锥曲线的焦点-准线定义}
平面上到一个定点与到一条定直线的距离之比为定值的点构成的图像称为\textbf{圆锥曲线(conic section)}。

其中,定点称为圆锥曲线的\textbf{焦点(focus)},定直线称为圆锥曲线的\textbf{准线(directrix)},二者互相对应。比值称作圆锥曲线的\textbf{离心率(eccentricity)},通常记为$e$ 。特别地:
\begin{itemize}
\item 当 $e = 0$ 时,轨迹称为\textbf{圆(circle)}。
\item 当 $0 < e < 1$ 时,轨迹称为\textbf{椭圆(ellipse)}。
\item 当 $e = 1$ 时,轨迹称为\textbf{抛物线(parabola)}。
\item 当 $e > 1$ 时,轨迹称为\textbf{双曲线(hyperbola)}。
\end{itemize}
\end{definition}

显然,定点到定直线的垂线为圆锥曲线的对称轴。

\subsection{定义等价性}