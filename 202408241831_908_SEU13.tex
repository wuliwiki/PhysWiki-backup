% 东南大学 2013 年 考研 量子力学
% license Usr
% type Note

\textbf{声明}:“该内容来源于网络公开资料,不保证真实性,如有侵权请联系管理员”

\textbf{1.(15 分)}以下叙述是否正确:

(1) 若厄密算符 $\hat{A}$ 与 $\hat{B}$ 对易,则它们必有共同本征态;

(2) 仅当体系处在定态时,守恒量的平均值才不随时间变化;

(3) 一维谐振子的能量本征态既有束缚态,也有散射态;

(4) 厄密算符的本征值必为正数;

(5) 空间平移对称性导致动量守恒。

\textbf{2.(15 分)}质量为 $m$ 的粒子处于 $\delta$ 势阱中,$V(x) = -\gamma \delta(x)$, $(\gamma > 0)$。

(1) 试根据 Schrödinger 方程证明 $x=0$ 处波函数的跃变条件为
$$\psi'(0^+) - \psi'(0^-) = -\left(2m\gamma/\hbar^2\right)\psi(0);~$$

(2) 试求束缚态能级和相应的归一化能量本征函数。

\textbf{3.(15 分)}一质量为 $m$ 的粒子以能量 $E$ 从左往右入射,受到以下势场的散射
$$V(x) =\begin{cases} V_0, & (x < 0) \\\\0, & (x > 0) \end{cases}\quad (V_0 > 0)~$$

在以下两种情况下计算反射系数和透射系数:
(1) $E > V_0$;
(2) $0 < E < V_0$。

提示:一维几率流密度公式为 
$$j(x) = \left(\hbar/i2m\right)\left(\psi^* \partial \psi/\partial x - \psi \partial \psi^*/\partial x\right)~$$

\textbf{4.(15 分)}试证 Bloch 函数
$$\psi_k(r) = \exp(ik \cdot r)\phi_k(r), \quad \phi_k(r) = \phi_k(r + a),~$$
是平移算符 $\hat{D}(a) = \exp\left(-ia \cdot \hat{p}/\hbar\right)$ 的本征态,相应的本征值为 $\exp\left(-ik \cdot a\right)$。

\textbf{5.(15 分)}设体系的 2 个粒子可处于 3 个单粒子态 $\phi_i(q), \phi_j(q), \phi_k(q)$。在以下三种情况下求体系可能的量子态数目:
(1) 不同粒子;
(2) 全同 Bose 粒子;
(3) 全同 Fermi 粒子。