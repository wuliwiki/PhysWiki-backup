% 北京师范大学 2016 年考研普通物理
% keys 北师大|考研|普通物理

1.一质点沿 $x$ 轴运动, 加速度与位置的关系为: $a=3+4 x$, 在 $x=0$ 处, 质点速度为 3 .试求 质点在 $x=5$ 处的速度.

2.一质点从位失为 $\bvec{r}(0)=4 \bvec{j}$ 的位置以初速度 $\bvec{v}(0)=4 \bvec{j}$ 开始运动, 其加速度与时间的关系 为 $\bvec{a}=3 t \bvec{i}-2 \bvec{j} \circ$ 求:\\
(1) 经过多长时间, 质点到达 $x$ 轴;\\
(2) 到达 $x$ 轴的位置

3.某人以 $2.5 \mathrm{m / s}$ 的速度向正西方向跑时,感到风来自正北.若速递增加一倍, 则感到风从正 西北方向吹来.求风速及风向.

4.如\autoref{BNU16_fig1} 所示,在光滑的水平面上放置一质量为 $m_{0}$ 的憵形物,其光滑斜面上放置一个质量为 $m$ 的物体, 劈形物体的倾角为 $\theta$, 求劈形物体的对 $m$ 的支撑力 $\bvec{F_{N}}$, 憵形物的加速度 $\bvec{a}, m$ 相对于 $m_{0}$ 的加速度 $\bvec{a^{\prime}}$
\begin{figure}[ht]
\centering
\includegraphics[width=4cm]{./figures/BNU16_1.pdf}
\caption{第4题图示} \label{BNU16_fig1}
\end{figure}
5.一质量为 $m$, 长为 $L$ 的细杆的底端通过铰链固定在桌面上, 其静止地从垂直位置向下运动.\\
(1) 画出示意图;\\
(2) 当细杆运动到与竖直方向夹角为 $\theta$ 时的角速度;\\
(3) 杆刚与桌面接触时,铰链对杆的支撑力.

6.估算压强 $P=10^{5} \mathrm{~Pa}$, 绝对温度 $T=10^{3} K$ 的氢气分子方均根速率 $\sqrt{v^{2}}$ 和单位体积内的分子个数 $n$ .

7.如\autoref{BNU16_fig2} 所示,绝热容器中装有体积为 $V_{1}$, 绝对温度为 $T_{1}$ 的 1 摩尔单原子分子理想气体和体积为 $V_{2}$, 绝对温度 为 $T_{2}$ 的 1 摩尔单原子分子理想气体.两部分气体用绝热壁隔开,忽略其体积.现把绝热壁去掉, 使两边混合而达到平衡态.求平衡态的绝对温度T和容器内气体在混合过程中发生的总熵变.
\begin{figure}[ht]
\centering
\includegraphics[width=4cm]{./figures/BNU16_2.pdf}
\caption{第7题图示} \label{BNU16_fig2}
\end{figure}
8.1摩尔单原子理想气体作两个绝热过程和两个等容过程构成的循环, 如\autoref{BNU16_fig3} 所示,推导该循环的效率 $\eta$ .
\begin{figure}[ht]
\centering
\includegraphics[width=3cm]{./figures/BNU16_3.pdf}
\caption{第8题图示} \label{BNU16_fig3}
\end{figure}
9.如\autoref{BNU16_fig4} 所示,一圆柱形线圈由50匝表面绝缘的细导线绕成, 圆面积 $S=4.0 \mathrm{~cm}^{2}$, 放在另一个半径为 $R=20 \mathrm{~cm}$ 的大圆形线圈中心, 两者同轴, 大圆线圈有 100 匝表面绝缘的导线绕成.\\
(1) 求两个线圈的互感系数 $M$;\\
(2) 当大线圈导线中电流每秒减少 $100\mathrm{A}$ 时,求小线圈中的感应电动势.
\begin{figure}[ht]
\centering
\includegraphics[width=3cm]{./figures/BNU16_4.pdf}
\caption{第9题图示} \label{BNU16_fig4}
\end{figure}

10.如\autoref{BNU16_fig6} 所示,长直导线和与其同轴的金属圆筒构成圆柱电容器, 其间充满相对介电常量为 $\varepsilon_{r}$ 的均匀电介质,设导线半径为 $R_{1}$, 圆筒的内半径为 $R_{2}$, 沿导线单位长度上的自由电荷为 $\lambda_{0}$, 求:\\
(1) 电介质中的电场强度 $\vec{E}$, 点位移 $\vec{D}$ 和极化强度 $\vec{P}$;\\
(2) 电介质内外表面的极化电荷面密度 $\sigma^{\prime}$;\\
(3) 长度为 $1 m$ 的圆柱电容器的电容 $C$ .
\begin{figure}[ht]
\centering
\includegraphics[width=5cm]{./figures/BNU16_6.pdf}
\caption{第10题图示} \label{BNU16_fig6}
\end{figure}

11.如\autoref{BNU16_fig5} 所示,一根很长的同轴电缆由半径为 $R_{1}$ 和 $R_{2}$ 的同轴圆柱壳组成, 其间充满磁导率为 $\mu$ 的均匀磁介质,内外导体均匀流过方向相反的电流 $J$, 求:
(1) 内导体外的 $\bvec{B}$ 和磁化强度 $\bvec{M}$ 的表达式;\\
(2) 磁介质内外表面的磁化电流 $I_{\text {内 }}^{\prime}, I_{\text {外 }}^{\prime}$;\\
(3) 长为 $l$ 的一段电绱内外导体之间磁场中储存的能量.
\begin{figure}[ht]
\centering
\includegraphics[width=5cm]{./figures/BNU16_5.pdf}
\caption{第11题图示} \label{BNU16_fig5}
\end{figure}
12.\autoref{BNU16_fig7} 中, $\varepsilon=20 \mathrm{~V}, R=10 \Omega, C=1.0\mathrm{\mu F}$, 设开关S接通的瞬间为 $t=0 s$, 求 $t=10^{-4} s$ 时电 容器极板间的位移电流 $I_{s}(t)$ .
\begin{figure}[ht]
\centering
\includegraphics[width=3.6cm]{./figures/BNU16_7.pdf}
\caption{第12题图示} \label{BNU16_fig7}
\end{figure}