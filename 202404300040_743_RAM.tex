% 随机存储器
% keys 随机存储器|随机存取存储器|内存
% license Xiao
% type Tutor

\begin{issues}
\issueDraft
\end{issues}

\subsection{随机存储器——RAM}

随机存储器(Random Access Memory, RAM)是计算机的主要存储介质之一,具有可读写的特性。其“随机存取”(Random Access)的能力意味着数据的读取或写入速度是均匀的,不依赖于数据在存储器中的物理位置。这种特性区别于例如磁带存储设备的顺序访问方式,后者的访问速度会因数据的位置不同而有很大差异。
\subsubsection{主要功能和应用}
RAM的主要功能是临时存储计算机运行中的程序和数据。它是执行中程序的“工作区”,存储着程序执行的指令和处理的数据。当您开启一个应用程序比如文字处理软件或者一个游戏时,程序的代码和所需的数据就被加载到RAM中,因为相比硬盘、SSD或其他形式的永久存储,RAM的数据访问速度更快,延迟更低。

由于其易失性的特点,RAM不适合用于长期数据存储。一旦电源断开,存储在RAM中的信息就会丢失。
\subsubsection{类型}

RAM主要分为两种类型:静态随机存取存储器(SRAM)和动态随机存取存储器(DRAM)。

静态随机存取存储器(SRAM)

SRAM保持数据的方式依靠持续供电的稳定性。它使用六个晶体管构成一个存储单元,不需要像DRAM那样定期刷新。SRAM的速度比DRAM快,但造价更高,功耗更低,因此常用于处理器的缓存。

动态随机存取存储器(DRAM)

DRAM存储每位数据依赖于电容和晶体管。电容会随时间自然放电,因此需要周期性的电荷刷新来保持信息。虽然这种设计使得DRAM在速度上不及SRAM,但其成本较低,密度更高,适合用作计算机的主内存。
应用领域
RAM是现代计算技术中不可或缺的组成部分,广泛应用于各种设备中,从个人计算机到服务器和移动设备等。其快速的数据处理能力对于实时运算和高效数据处理至关重要,尤其是在需要处理大量数据的服务器和大型计算机系统中。

通过不断的技术发展,RAM已经在速度、容量和功耗方面取得了显著进步,使得现代计算机能够支持高性能的应用程序和复杂的操作系统。

\subsection{静态随机访问存储器——SRAM}

静态随机访问存储器(Static random-access memory,SRAM)其中的Static意味着:只要保持通电,存储的数据就可以恒常保持。

具体地,一个SRAM基本单元有0 和 1两个电平稳定状态。SRAM使用六个晶体管存储单元的状态存储一个数据位,通常使用六个MOSFETs,这种RAM的生产成本更高,但通常比DRAM更快且动态功耗更低。

SRAM基本单元由两个CMOS反相器组成。两个反相器的输入、输出交叉连接,即第一个反相器的输出连接第二个反相器的输入,第二个反相器的输出连接第一个反相器的输入。这就能实现两个反相器的输出状态的锁定、保存,即储存了1个比特的状态。

\begin{figure}[ht]
\centering
\includegraphics[width=6cm]{./figures/87070ace2adbda7b.png}
\caption{六管SRAM存储单元示意图} \label{fig_RAM_3}
\end{figure}

\subsubsection{特性}


但是,SRAM用了太多MOS管,而且总有两个MOS管饱和导通,占空间,功耗大。而且地址线多,导致不能做太大。

SRAM往往集成在芯片内部,一般的使用场景有:
作为x86等微处理器的缓存(如L1、L2、L3)
作为寄存器(寄存器堆)
用于FPGA与CPLD

\subsection{动态随机访问存储器——DRAM}


通常我们俗称的"内存"一般是指计算机中的DRAM。具体的说,内存条通常使用DDR4 DRAM,其容量一般为8G或16G,通常,我们需要插两根来组成双通道内存,即构成16G或32G的内存(memory)。

严格来讲,“内存”应该是所有易失性存储器的统称;相对应的,\textbf{非易失性存储器}统称为“外存”。



% \subsection{NVRAM}

% 我们上述所说的内存介质都需要通电来维持数据,这些存储器都属于易失性存储器。然而NVRAM()


参考文献:
\begin{enumerate}
\item 唐朔飞。 计算机组成原理[M]. 高等教育出版社。 2008
\end{enumerate}
