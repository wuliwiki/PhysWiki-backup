% 等式与不等式(高中)
% keys 方程|不等式|代数基本定理
% license Xiao
% type Tutor

\begin{issues}
\issueDraft
\end{issues}


相等和不等关系是从小学阶段就开始接触的基础概念,但由此延伸出的方程、不等式、恒等式、方程组、解等概念,许多人往往会感到不知所谓。人教版初中教材中给出的“方程”定义是:“含有未知数的等式称作方程(equation)。”给“未知数”的定义则是“方程的求解目标”,这看上去是一种令人迷惑的循环定义,进而造成有些人困惑形如$x=3$的等式是否也是一个方程。到高中阶段,在高中教材中依然没有系统性的澄清这些概念。

因此,很多学生在阅读题目时,对解题任务的理解感到模糊,不清楚解方程、联立方程究竟意味着什么,这种认识上的模糊甚至延续到大学阶段,影响对更复杂概念的掌握和后续学习的进展,很多研究者在使用这些术语时也很混乱。本文旨在解决上面提到的问题。

\subsection{关于等式和不等式的基础概念}

下面会先介绍一些基础概念。这些概念的数量很多且前后勾连,且有不少与教材上语焉不详的定义存在出入。这里不要求完整记忆,只需要认真理解并清楚自己脑海中习惯的表达与下面概念的对应即可。

表示数学运算的符号,如加、减、乘、除以及各种函数等,被称为\textbf{运算符(operator)}\footnote{这个概念,在数学领域深入研究后,称为\textbf{算子},有其专属的定义,会在\enref{泛函分析}{FnalNt}中学习。}。用于表示两个数学元素之间关系的符号称为\textbf{关系符(relational symbol)}。例如,“$>$”、“$<$”、“$\leq$”、“$\geq$”、“$\neq$”这些符号称为\textbf{不等号(inequality symbols)},而“$=$”称为\textbf{等号(equality symbol)}。

此外,还有其他类型的关系符号。例如,在集合论中,有表示包含关系的符号“$\subset$”和表示元素属于某个集合的符号“$\in$”;在几何中,有表示平行关系的符号“$\parallel$”和垂直关系的符号“$\perp$”;在逻辑学中,等价符号“$\equiv$”表示等价关系,蕴含符号“$\Rightarrow$”表示蕴含关系。

表示数学运算的符号,如加、减、乘、除以及各种函数等,称作\textbf{运算符}。表示两个数学元素关系的符号,称作\textbf{关系符},比如$>,<,\leq,\geq,\neq$称作\textbf{不等号},$=$称作\textbf{等号}。其他的关系符还有集合中的包含、属于,几何中的平行、垂直,逻辑中的等价、蕴含等。这些关系符在不同数学领域中都扮演着重要角色,用来描述各种对象间的关系。

由数字、变量以及运算符组成的数学符号串,称作\textbf{数学表达式(mathematic expression)},或直接简称\textbf{表达式(expression,或式子)}。表达式可以看作是对某个值或状态的描述,关键点在于它不需要求解,它只是对某种数学数量的表示。对表达式可以进行简化、合并同类项,或者在特定情况下给出某个变量的值进行替代计算。初中时学过的\textbf{代数式(algebraic expression)}是表达式的子集,它特指用运算符号把数或表示数的字母连起来的式子,即将运算限定在了加、减、乘、除以及有理数次的乘方、开方。

在介绍下面的概念前,不得不先介绍一个英语单词“equation”,这个词在中文中通常会翻译做等式或方程,下面为避免混淆,统一称为等式。用等号连接的两个表达式构成的描述相等关系的式子,称作\textbf{等式(equation)}。若自变量取某些值时,给定的两个函数值相等,则称函数的自变量为\textbf{未知数(unknown)},而使等式成立的自变量的值为等式的\textbf{解(solution)}。所有的解构成的集合称为\textbf{解集}。根据未知数允许的取值范围$M$和解集$S$的交集$C=M\cap S$的不同,等式可分为矛盾式和条件等式、恒等式:
\begin{itemize}
\item 如果$C=\varnothing$,则称原等式为\textbf{矛盾式},或称\textbf{方程无解}。
\item 如果$C\neq\varnothing$且$C\subseteq M$,即等式只在某些特定的变量取值下成立,则称原等式为\textbf{条件等式(conditional equation)},一般高中范围内研究的“方程”都是指狭义上的条件等式。
\item 如果$C=M$,则称原等式为$M$上的\textbf{恒等式(identity)},通常定义某个量或关系时使用的等式都是\enref{恒等式}{HsIden}。
\end{itemize}

可以这样认为,方程的解就是条件等式成立的条件。

% 什么是等式和不等式






用不等号连接两个表达式,用于描述不等关系,称作不等式。%举例。不等式与等式一样

如何理解$x=3$?

\subsection{条件方程与不等式方程}


\subsection{方程组}

\subsection{解与解集}

方程和不等式只在解或解集中成立。

方程的解可以分为两大类:解析解和数值解。如果方程的解可以通过有限次的常见运算(如加、减、乘、除等)得到,这种解称为\textbf{解析解(Analytical Solution)}。这时,解的表达式可以用代数形式清晰地表示出来。有些复杂的方程很难找到解析解,甚至解析解根本不存在。在这种情况下,可以使用数值分析方法,如二分法、牛顿法等,通过迭代和近似计算来求解方程。此时得到的解称为\textbf{数值解(Numerical Solution)}。数值解通常通过计算机来计算,能够为复杂问题提供高精度的近似解。

总的来说,解析解是精确的,但不总是存在;数值解是近似的,却总是能提供实用的近似结果。在高中阶段,一般只涉及解析解,但存在大量的方程无法获得解析解,或难以获得解析解。

\subsubsection{有理不等式的解集}

穿针法

\subsubsection{解与零点}

\begin{definition}{代数学基本定理}
任何一个 $n$ 次多项式函数在复数域上都有 $n$ 个零点(重数计入)。
\end{definition}
这意味着在复数范围内,可以找到所有多项式方程的解。



