% 量类和单位
% 量|同类量|量类|单位
\subsection{量类}
现象、物体或物质可定性区别并能定量测量的属性称为\textbf{物理量}(简称\textbf{量}).量的具体意义指大小、轻重、长短等概念,并不是所有的量都可以相互比较,比如表示长短的量和表示大小的量不能相互比较,但表示同一具体意义的量之间可以相互比较.

我们把可以相互比较的量称为\textbf{同类量},比较的结果是一个数.以粗体字母表示量,如 $\boldsymbol{A}$,细体字母表示数,如 $A$.若量 $\boldsymbol{A_1}$ 和量 $\boldsymbol{A_2}$ 可相互比较,则称量 $\boldsymbol{A_1}$ 和 $\boldsymbol{A_2}$ \textbf{同类}.对于所有量构成的集合 $\mathcal{M}$,其上的同类关系显然为一等价关系, 由该等价关系确定的等价类\upref{Relat}称为\textbf{量类}.对任一量 $\boldsymbol{Q}$,与它同类的所有量构成的集合称为 $\boldsymbol{Q}$ 的量类,记作 $\tilde{\boldsymbol{Q}}$.显然 $\boldsymbol{Q}$ 是$\tilde{\boldsymbol{Q}}$ 的元素,可记作 $\boldsymbol{Q}\in \tilde{\boldsymbol{Q}}$,每个 $\boldsymbol{Q}$ 可称为 $\tilde{\boldsymbol{Q}}$ 的\textbf{量值}.例如,所有长度量的集合称为\textbf{长度量类(记作$\tilde{\boldsymbol{l}}$)}.类似的,还有\textbf{质量量类}$\tilde{\boldsymbol{m}}$、\textbf{时间量类}$\tilde{\boldsymbol{t}}$、\textbf{速度量类}$\tilde{\boldsymbol{v}}$等.

设 $\boldsymbol{Q_1}$ 和 $\boldsymbol{Q_2}$ 是同类量,将 $\boldsymbol{Q_2}$ 和 $\boldsymbol{Q_1}$ 进行比较得数为 $Q$,也称用 $\boldsymbol{Q_1}$ 测量 $\boldsymbol{Q_2}$ 得数 $Q$,记作
\begin{equation}\label{QCU_eq1}
\boldsymbol{Q_2}=Q\boldsymbol{Q_1}
\end{equation}
等式\autoref{QCU_eq1} 称为 \textbf{同类量等式}.
\begin{definition}{}
若 $\boldsymbol{Q_1}$和 $\boldsymbol{Q_2}$ 满足同类量等式\autoref{QCU_eq1} ,则称
\begin{equation}\label{QCU_eq4}
\frac{\boldsymbol{Q_2}}{\boldsymbol{Q_1}}=Q
\end{equation}
为同类量 $\boldsymbol{Q_2}$ 与 $\boldsymbol{Q_1}$ 的\textbf{商}.
\end{definition}
\subsection{单位}
在量类 $\tilde{\boldsymbol{Q}}$ 中任选一个非零元素(即该元素测量其它非零元素得数不为0) $\hat{\boldsymbol{Q}}$ 测量其它元素,就可把每一个元素映射到一个实数,这个 $\hat{\boldsymbol{Q}}$ 称为\textbf{单位}.

设用 $\hat{\boldsymbol{Q}}$ 测 $\boldsymbol{Q}$ 得数 $Q$,即
\begin{equation}\label{QCU_eq2}
\boldsymbol{Q}=Q\hat{\boldsymbol{Q}}
\end{equation}
如果改用另一单位 $\hat{\boldsymbol{Q'}}$ 测 $\boldsymbol{Q}$,得数为 $Q'$,即
\begin{equation}\label{QCU_eq3}
\boldsymbol{Q}=Q'\hat{\boldsymbol{Q'}}
\end{equation}
为由\autoref{QCU_eq2} 和\autoref{QCU_eq3} 推得 $\hat{\boldsymbol{Q}}$ 和 $\hat{\boldsymbol{Q'}}$ 的关系,需要用到如下的测量公理.

\textbf{测量公理:}对同类量 $\boldsymbol{Q_1}$、$\boldsymbol{Q_2}$、$\boldsymbol{Q_3}$,若用 $\boldsymbol{Q_1}$ 测量 $\boldsymbol{Q_2}$ 得数为 $\alpha$,用 $\boldsymbol{Q_2}$ 测量 $\boldsymbol{Q_3}$ 得数为 $\beta$,则用 $\boldsymbol{Q_1}$ 测量 $\boldsymbol{Q_3}$ 得数为 $\alpha\beta$.即若
\begin{equation}\label{QCU_eq13}
\boldsymbol{Q_2}=\alpha\boldsymbol{Q_1},\boldsymbol{Q_3}=\beta\boldsymbol{Q_2}\quad\Rightarrow \quad\boldsymbol{Q_3}=(\alpha\beta)\boldsymbol{Q_1}
\end{equation}

\begin{theorem}{}\label{QCU_the2}
$\alpha \boldsymbol{Q}$是一个与 $\boldsymbol{Q}$ 同类的量,并且
\begin{equation}\label{QCU_eq14}
\alpha\qty(\beta \boldsymbol{Q})=\qty(\alpha\beta)\boldsymbol Q
\end{equation}
\end{theorem}
\textbf{证明:}因为 $\alpha\boldsymbol{Q}=\alpha\boldsymbol{Q}$,即用 $\boldsymbol{Q}$ 测量 $\alpha\boldsymbol{Q}$ 得数 $\alpha$ ,这表明 $\alpha\boldsymbol{Q}$ 与 $\boldsymbol{Q}$ 同类.

因为
\begin{equation}
\alpha(\beta\boldsymbol{Q})=\alpha(\beta\boldsymbol{Q}),\beta\boldsymbol{Q}=\beta\boldsymbol{Q}
\end{equation}
由测量公理\autoref{QCU_eq13} ,即得\autoref{QCU_eq14} ,定理得证.

有了测量公理,便可讨论 $\hat{\boldsymbol{Q}}$ 和 $\hat{\boldsymbol{Q'}}$ 的关系.由商的定义\autoref{QCU_eq4} ,可知 $\hat{\boldsymbol{Q}}$ 测量 $\hat{\boldsymbol{Q'}}$ 得数为 $\frac{\hat{\boldsymbol{Q'}}}{\hat{\boldsymbol{Q}}}$ ,又$\hat{\boldsymbol{Q'}}$ 测 $\boldsymbol{Q}$得数 $Q'$,所以由测量公理, $\hat{\boldsymbol{Q}}$ 测 $\boldsymbol{Q}$ 得数 $Q'\frac{\hat{\boldsymbol{Q'}}}{\hat{\boldsymbol{Q}}}$,又由\autoref{QCU_eq2} ,得
\begin{equation}
Q'\frac{\hat{\boldsymbol{Q'}}}{\hat{\boldsymbol{Q}}}=Q
\end{equation}
即
\begin{equation}\label{QCU_eq8}
\frac{\hat{\boldsymbol{Q'}}}{\hat{\boldsymbol{Q}}}=\frac{Q}{Q'}
\end{equation}

\autoref{QCU_the2} 和\autoref{QCU_eq8} 是由测量公理得出的,结合二者,我们得到下面的定理.
\begin{theorem}{}\label{QCU_the3}
$\boldsymbol{Q}$ 与任选单位无关.即任意 $\hat{\boldsymbol{Q}},\hat{\boldsymbol{Q'}}\in\tilde{\boldsymbol{Q}}$,且 $\boldsymbol{Q}=Q\hat{\boldsymbol{Q}},\boldsymbol{Q}=Q'\hat{\boldsymbol{Q'}}$,则
\begin{equation}\label{QCU_eq15}
Q\hat{\boldsymbol{Q}}=Q'\hat{\boldsymbol{Q'}}
\end{equation}
\end{theorem}
\textbf{证明:}由商的定义\autoref{QCU_eq4} ,\autoref{QCU_eq8} 可写成
\begin{equation}
\hat{\boldsymbol{Q'}}=\frac{Q}{Q'}\hat{\boldsymbol{Q}}
\end{equation}
两边作用一个 $Q'$ ,并由\autoref{QCU_the2} ,即得\autoref{QCU_eq15} .定理得证.

同样的,我们可以得到下面的定理.
\begin{theorem}{}\label{QCU_the1}
对量类 $\tilde{\boldsymbol{Q}}$ 的两个元素 $\boldsymbol{Q_1}$ 和 $\boldsymbol{Q_2}$,若
\begin{equation}\label{QCU_eq5}
\boldsymbol{Q_1}=Q_1\hat{\boldsymbol{Q}},\quad \boldsymbol{Q_2}=Q_2\hat{\boldsymbol{Q}}
\end{equation}
则
\begin{equation}\label{QCU_eq7}
\frac{\boldsymbol{Q_2}}{\boldsymbol{Q_1}}=\frac{Q_2}{Q_1}
\end{equation}
\end{theorem}
\textbf{证明:}由商的定义\autoref{QCU_eq4} ,$\boldsymbol{Q_1}$ 测量 $\boldsymbol{Q_2}$ 得数为 $\frac{\boldsymbol{Q_2}}{\boldsymbol{Q_1}}$,由\autoref{QCU_eq5} 第一式, $\hat{\boldsymbol{Q}}$ 测量 $\boldsymbol{Q_1}$ 得数为 $Q_1$ ,由测量公理,$\hat{\boldsymbol{Q}}$ 测量 $\boldsymbol{Q_2}$ 得数为 $Q_1\frac{\boldsymbol{Q_2}}{\boldsymbol{Q_1}}$,又由\autoref{QCU_eq5} 第二式,便有
\begin{equation}\label{QCU_eq6}
Q_1\frac{\boldsymbol{Q_2}}{\boldsymbol{Q_1}}=Q_2
\end{equation}
\autoref{QCU_eq6} 即是\autoref{QCU_eq7} 等价形式,定理得证.

当 $Q_2>Q_1$ 时,我们说量 $\boldsymbol{Q_2}$ 大于量 $\boldsymbol{Q_1}$,并记作 $\boldsymbol{Q_2}>\boldsymbol{Q_1}$.如此,\autoref{QCU_eq8} 就可表达为:用不同单位测量同一量时,单位越大得数越小.
\begin{example}{}\label{QCU_ex1}
若 $\boldsymbol{Q_1}$ 、$\boldsymbol{Q_2}$ 同类,试证明以下式子
\begin{equation}\label{QCU_eq11}
\frac{Q_2\boldsymbol{Q_2}}{Q_1\boldsymbol{Q_1}}=\frac{Q_2}{Q_1}\frac{\boldsymbol{Q_2}}{\boldsymbol{Q_1}}
\end{equation}
\textbf{证明:} 显然,用量 $\boldsymbol{Q_1}$ 测量 $Q_1\boldsymbol{Q_1}$ 得数为 $Q_1$,用量 $\boldsymbol{Q_2}$ 测量 $Q_2\boldsymbol{Q_2}$ 得数为 $Q_2$,即
\begin{equation}\label{QCU_eq9}
\begin{aligned}
Q_1\boldsymbol{Q_1}=Q_1\boldsymbol{Q_1}\\
Q_2\boldsymbol{Q_2}=Q_2\boldsymbol{Q_2}
\end{aligned}
\end{equation}


而用量 $\boldsymbol{Q_1}$ 测量 $\boldsymbol{Q_2}$ 得数为 $\frac{\boldsymbol{Q_2}}{\boldsymbol{Q_1}}$.所以,由测量公理,下式成立
\begin{equation}\label{QCU_eq10}
Q_2\boldsymbol{Q_2}=\qty(Q_2\frac{\boldsymbol{Q_2}}{\boldsymbol{Q_1}})\boldsymbol{Q_1}
\end{equation}
比较\autoref{QCU_eq9} 第一式和\autoref{QCU_eq10} ,由\autoref{QCU_the1} 得:
\begin{equation}
\frac{Q_2\boldsymbol{Q_2}}{Q_1\boldsymbol{Q_1}}=\frac{Q_2\frac{\boldsymbol{Q_2}}{\boldsymbol{Q_1}}}{Q_1}=\frac{Q_2}{Q_1}\frac{\boldsymbol{Q_2}}{\boldsymbol{Q_1}}
\end{equation}
得证.
\end{example}
\begin{example}{}\label{QCU_ex2}
若 $\boldsymbol{Q_1}$、$\boldsymbol{Q_2}$ 同类,试证明
\begin{equation}\label{QCU_eq12}
\frac{\boldsymbol{Q_2}}{\boldsymbol{{Q_1}}}=\frac{1}{\frac{\boldsymbol{Q_1}}{\boldsymbol{Q_2}}}
\end{equation}
\textbf{证明:}由\autoref{QCU_ex1} 中\autoref{QCU_eq9} 第一式和\autoref{QCU_eq10} ,根据\autoref{QCU_the1} 有
\begin{equation}
\frac{Q_1\boldsymbol{Q_1}}{Q_2\boldsymbol{Q_2}}=\frac{Q_1\boldsymbol{Q_1}}{\qty(Q_2\frac{\boldsymbol{Q_2}}{\boldsymbol{Q_1}})\boldsymbol{Q_1}}=\frac{Q_1}{Q_2\frac{\boldsymbol{Q_2}}{\boldsymbol{Q_1}}}
\end{equation}
由\autoref{QCU_eq11} ,上式可化为
\begin{equation}
\frac{Q_1}{Q_2}\frac{\boldsymbol{Q_1}}{\boldsymbol{Q_2}}=\frac{Q_1}{Q_2\frac{\boldsymbol{Q_2}}{\boldsymbol{Q_1}}}
\end{equation}
上式两边可约去数$Q_1/Q_2$,便得待证\autoref{QCU_eq12} .
\end{example}
\autoref{QCU_eq12} 说明,$\boldsymbol{Q_1}$ 测 $\boldsymbol{Q_2}$ 得数与$\boldsymbol{Q_2}$ 测 $\boldsymbol{Q_1}$ 得数互为倒数,于是得下面定理.
\begin{theorem}{}
对量类 $\tilde{\boldsymbol{Q}}$ 的两个元素 $\boldsymbol{Q_1}$ 和 $\boldsymbol{Q_2}$,若 $\boldsymbol{Q_1}$ 测 $\boldsymbol{Q_2}$ 得数为 $Q$,则$\boldsymbol{Q_2}$ 测 $\boldsymbol{Q_1}$ 得数为 $1/Q$,即
\begin{equation}
\boldsymbol{Q_2}=Q\boldsymbol{Q_1}\quad\Rightarrow\quad\boldsymbol{Q_1}=\frac{1}{Q}\boldsymbol{Q_2}
\end{equation}
\end{theorem}