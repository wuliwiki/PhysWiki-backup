% 乔赛亚·威拉德·吉布斯(综述)
% license CCBYSA3
% type Wiki

本文根据 CC-BY-SA 协议转载翻译自维基百科\href{https://en.wikipedia.org/wiki/Josiah_Willard_Gibbs}{相关文章}。

约西亚·威拉德·吉布斯(Josiah Willard Gibbs,/ɡɪbz/,1839年2月11日-1903年4月28日)是一位美国机械工程师和科学家,在物理学、化学和数学领域做出了基础性的理论贡献。他关于热力学应用的研究在将物理化学转变为一门严谨的演绎科学方面起到了关键作用。吉布斯与詹姆斯·克拉克·麦克斯韦和路德维希·玻尔兹曼一起创立了统计力学(该术语由他提出),将热力学定律解释为由大量粒子组成的物理系统可能状态集合的统计特性所导致的结果。吉布斯还研究了麦克斯韦方程在物理光学问题中的应用。作为数学家,他独立于英国科学家奥利弗·赫维赛德(后者在同一时期进行了类似的工作)创建了现代向量分析,并在傅里叶分析理论中描述了“吉布斯现象”。

1863年,耶鲁大学授予吉布斯美国首个工程学博士学位。在欧洲度过三年后,吉布斯将余生的职业生涯都奉献给了耶鲁大学,自1871年起担任数学物理学教授,直到1903年去世。他在相对孤立的环境中工作,成为美国最早获得国际声誉的理论科学家之一,曾被阿尔伯特·爱因斯坦称为“美国历史上最伟大的头脑”。

1901年,吉布斯因其在数学物理方面的贡献,获得了当时国际科学界最高荣誉——由伦敦皇家学会颁发的科普利奖章。

评论家和传记作家们曾指出,吉布斯宁静而孤独的新英格兰生活方式与其思想在国际上的巨大影响之间形成了鲜明对比。尽管他的研究几乎完全是理论性的,但随着20世纪上半叶工业化学的发展,吉布斯成果的实际价值逐渐显现。正如罗伯特·A·密立根所言,在纯科学领域,吉布斯“对于统计力学和热力学的贡献,就如拉普拉斯之于天体力学,麦克斯韦之于电动力学——他几乎将这个领域构建成一个完整的理论体系”。
\subsection{传记}
\subsubsection{家庭背景}
\begin{figure}[ht]
\centering
\includegraphics[width=6cm]{./figures/0453fffcb9e69ba5.png}
\caption{青年时期的威拉德·吉布斯} \label{fig_QSY_1}
\end{figure}
吉布斯出生于康涅狄格州纽黑文。他出身于一个古老的“洋基”家族,自17世纪以来,该家族不断涌现出杰出的美国牧师和学者。他是家中五个孩子中排行第四的孩子,也是父亲约西亚·威拉德·吉布斯与母亲玛丽·安娜(娘家姓范·克里夫,Mary Anna,née Van Cleve)唯一的儿子。在父系方面,他是塞缪尔·威拉德的后代,后者于1701年至1707年间曾担任哈佛学院代理校长。在母系方面,他的一位祖先是乔纳森·迪金森牧师,新泽西学院(后来的普林斯顿大学)首任校长。
“约西亚·威拉德”这个名字在吉布斯家族中代代相传,他与父亲及其他一些家族成员都使用这个名字。它源自他的一位祖先——18世纪曾任马萨诸塞湾省国务秘书的约西亚·威拉德。他的祖母默西·普雷斯科特·吉布斯(Mercy (Prescott) Gibbs)是丽贝卡·米诺特·普雷斯科特·舍曼的妹妹,而后者是美国开国元勋罗杰·舍曼的妻子。因此,吉布斯是舍曼家族的第二代近亲,也与后来涉及“阿米斯塔德号案件”的罗杰·舍曼·鲍德温是堂表亲。

吉布斯的父亲在家庭和学术界通常被称为“约西亚”,而他本人则被称为“威拉德”(Willard)。约西亚·吉布斯是一位语言学家和神学家,自1824年起担任耶鲁大学神学院的圣经文学教授,直到1861年去世。他如今最广为人知的事迹,是作为废奴主义者,在“阿米斯塔德号”事件中为非洲船员找到口译员,使他们能在审判中作证,讲述自己反抗被贩卖为奴的经历。
\subsubsection{教育经历}
威拉德·吉布斯在霍普金斯学校接受教育,并于1854年15岁时进入耶鲁学院。在耶鲁,吉布斯因数学和拉丁语方面的优异成绩而获得奖项,并于1858年以班级前列的成绩毕业。他随后留在耶鲁,成为谢菲尔德科学学院的研究生。19岁时,也就是他刚从本科毕业不久,吉布斯被选入康涅狄格艺术与科学学院,这是一个由耶鲁大学教师为主组成的学术机构。这一时期留下的文献资料相对较少,因此很难精确还原吉布斯早期职业生涯的细节。据传记作者推测,吉布斯在耶鲁大学及康涅狄格学院的主要导师与支持者,很可能是天文学家兼数学家休伯特·安森·牛顿,他是当时研究流星领域的权威,也一直是吉布斯的终生朋友和知己。1861年吉布斯父亲去世后,他继承了一笔足以维持经济独立的遗产。

年轻时期的吉布斯长期受到反复发作的肺部疾病困扰,医生担心他可能容易感染肺结核——他的母亲便因这种病去世。他还患有散光,而当时眼科医生对这种病的治疗尚不熟悉,因此吉布斯不得不自行诊断,并亲自打磨适合自己的眼镜镜片。虽然在后来,他只在阅读或从事近距离工作时才佩戴眼镜,但他体质虚弱以及视力不佳,很可能是他在1861至1865年的南北战争期间没有主动参军的原因。他也未被征召入伍,而是一直留在耶鲁大学度过了整个战时期间。

1863年,吉布斯获得了美国授予的首个工程学博士(PhD)学位,其论文题为《论直齿轮中齿的形状》,他在其中运用几何方法研究齿轮的最优设计。耶鲁大学在1861年成为美国首所提供博士学位的大学,而吉布斯的博士学位是美国在所有学科中授予的第五个博士学位。
