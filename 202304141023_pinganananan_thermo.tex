% 热学(高中)

\begin{issues}
\issueTODO
\end{issues}

% 分子动理论|气体等x定律|固体液体|热力学定律

%\pentry{机械运动基础\upref{HSPM01}, 相互作用\upref{HSPM02}}
%\pentry{词条示例\upref{Sample}}
\subsection{分子动理论}
\subsubsection{对于分子的直观认识}
什么是分子|分子的特征(大小/数量)
当不需要区分分子、原子或者离子在化学变化中所起的作用不同时,而仅仅研究物体的热运动性质以及规律,可以将组成物体的微粒统称为分子。

我们不能直接观察分子,而必须借助高分辨率的显微镜,比如说扫描隧道显微镜\footnote{扫描隧道显微镜是一种可以个探测物质表面结构的仪器。其工作原理是利用探针和物质表面的相互作用来获得物质表面结构的图像信息,分辨尺度为原子尺度。}
\subsubsection{分子热运动}
扩散|布朗运动|热运动
\subsubsection{分子间作用力}
分子间空隙|分子间作用力
\subsubsection{分子动理论}
基本认识……

\subsection{……}
