% 黎曼积分
% 积分|黎曼积分|微积分

我们来定义区间 $[a, b]$ 的实函数 $f(x)$ 的黎曼积分. 令有序实数列
\begin{equation}
\mathcal P = \qty{a = x_0 < x_1 < \dots < x_n = b}
\end{equation}
令
\begin{equation}
m_k = \inf_{x_{k-1} \le x \le x_k} f(x) \qquad M_k = \sup_{x_{k-1} \le x \le x_k} f(x) \quad (1 \le k \le n)
\end{equation}
对应的下和上黎曼求和为
\begin{equation}
\underline I(f, \mathcal P) = \sum_{k = 1}^n m_k\Delta k \qquad \bar I(f, \mathcal P) = \sum_{k = 1}^n M_k \Delta_k
\end{equation}
其中 $\Delta_k = x_k - x_{k-1}$, $1\le k\le n$. 下和上黎曼积分的定义为
\begin{equation}
\underline I(f) = \sup_{\mathcal P} \underline I(f, \mathcal P) \qquad \bar I(f) = \inf_{\mathcal P} \bar I(f, \mathcal P)
\end{equation}
显然, $\underline I(f) \le \bar I(f)$. $f$ 叫做\textbf{黎曼可积(Riemann integrable)} 当且仅当上下黎曼积分相等
\begin{equation}
I(f) = \underline I(f) = \bar I(f)
\end{equation}
我们以下用 $I(f)$ 表示黎曼积分, $\int$ 符号表示勒贝格积分.

\begin{theorem}{}
任何连续函数 $f \in C[a, b]$ 都是黎曼可积的.
\end{theorem}
