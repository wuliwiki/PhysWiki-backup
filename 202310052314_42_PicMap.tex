% 皮卡映射
% keys 皮卡映射|逐次皮卡近似
% license Xiao
% type Tutor

\begin{issues}
\issueDraft
\end{issues}

\pentry{定积分\upref{DInt},映射\upref{map}}
通过曲线 $\varphi$ 构造出一新曲线 $f(\varphi)$ ,使得新曲线上每一点 $f(\varphi(t))$ 的切线平行于曲线 $\varphi$ 上点 $\varphi(t)$ 处给定的向量。具体来说,设 $U$ 是 $\mathbb R^n$ 的一区域,其上每一点 $x$ 都定义了一个向量 $v(x,t)$ (或称为 $V$ 上定义了向量场 $v$)。给定 $V$ 中的曲线 $\varphi:I\rightarrow V$($I$ 为 $t$ 轴上一区间),那么向量场 $v$ 在曲线上每一点 $\varphi(t)$ 对应的向量为 $v(\varphi(t),t)$。于是新曲线 $f(\varphi)$ 是这样的曲线,其在 $t$ 时的点 $f(\varphi(t))$ 的切向量满足 $\dv{}{t}(f\circ \varphi)\big|_{t}=v(\varphi(t),t)$\footnote{当然,平行说明还有个系数,但是这里特指系数为1}。描述这样的曲线 $\varphi$ 到新曲线 $f(\varphi)$ 的映射称为\textbf{皮卡映射(Picard 映射)}。
\subsection{皮卡映射}