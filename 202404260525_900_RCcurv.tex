% 电容—电阻电路充放电曲线
% keys 电容|电阻|微分方程
% license Xiao
% type Tutor

%\begin{issues}
%\issueDraft
%\end{issues}

\pentry{电容\nref{nod_Cpctor}, 一阶线性微分方程\nref{nod_ODE1}}{nod_e6fb}
\begin{figure}[ht]
\centering
\includegraphics[width=5cm]{./figures/86cc5272c9104b0f.pdf}
\caption{电容电阻串联} \label{fig_RCcurv_1}
\end{figure}
回路中有直流电源 $U$ , 电阻 $R$ 和电容 $C$。 当开关拨向1时,接通电源,电容器充电;当开关拨向2时,断开电源,电容器放电。对于充放电过程有如下方程
\begin{equation}
IR+U_C=
\begin{cases}
U& \quad(\text{充电})\\
0& \quad(\text{放电})~.
\end{cases}
\end{equation}
设电容器极板带电量为 $Q$ ,由电流的定义\autoref{eq_I_1}~\upref{I}和电容的定义\autoref{eq_Cpctor_2}~\upref{Cpctor}
\begin{equation}
I = \dv{Q}{t}=C\dv{U_C}{t}~,
\end{equation}
代入得
\begin{equation}
RC\dv{U_C}{t}+U_C =
\begin{cases}
U& \quad  (\text{充电})\\
0& \quad  (\text{放电})~.
\end{cases}
\end{equation}
这是一个一阶线性常微分方程\upref{ODE1}。 初始条件为
\begin{equation}
U_C=\begin{cases}
0&\quad (\text{充电})\\
U&\quad (\text{放电})~.
\end{cases}
\end{equation}
 解得
\begin{equation}\label{eq_RCcurv_5}
U_C(t) = 
\begin{cases}
U\qty(1 - \E^{-t/(RC)})&\quad  (\text{充电})\\
U\E^{-t/(RC)}          &\quad  (\text{放电})~.
\end{cases}
\end{equation}
可以看到当 $t \to \infty$ 时,对于充电过程:$U_C = U$ ;而对放电过程:$U_C = 0$。

记 $\tau =RC$ 为该电路的时间常数。由\autoref{eq_RCcurv_5} 可知,
\begin{equation}\label{eq_RCcurv_6}
U(\tau) = 
\begin{cases}
U(1-e^{-1})&(\text{充电})\\
U/e        &(\text{放电})~.
\end{cases}
\end{equation}
利用\autoref{eq_RCcurv_6} 即可在实验中确定 $\tau$,进而根据 $R$ 确定 $C$,反之亦然。


作出 $RC$ 电路的充放电曲线如\autoref{fig_RCcurv_2} 
\begin{figure}[ht]
\centering
\includegraphics[width=8cm]{./figures/69df8cbcb63d5955.pdf}
\caption{充电过程(左)和放电过程(右)} \label{fig_RCcurv_2}
\end{figure}

