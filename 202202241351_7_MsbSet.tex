% 可测集合
% keys 实变函数|Lebesgue积分

\pentry{集合的测度(实变函数)\upref{SetMet}}
\autoref{SetMet_def1} 

在\textbf{集合的测度(实变函数)}\upref{SetMet}中我们提到,Lebesgue积分是对值域作分划.但是值域作分划以后,有可能出现\autoref{SetMet_ex1}~\upref{SetMet}那样病态的两个集合,它们互不相交,但其并集的测度小于各自测度之和.这就好比对定义域作分划的Riemann积分中,两根柱子的总面积小于单个柱子的面积之和,非常不符合直觉,而且会导致无法建立自洽的理论.

因此,讨论Lebesgue积分时,我们要对集合的性质作出限制,只讨论所谓的\textbf{可测集合(measurable set)},以及进而导出的\textbf{可测函数(measurable function)}.

\begin{definition}{可测集}\label{MsbSet_def1}

设$E$为$\mathbb{R}^n$的子集,满足:对于$\mathbb{R}^n$的任意子集$A$,都有
\begin{equation}
\opn{m^*}(A\cap E)+\opn{m^*}(A\cap E^C)=\opn{m^*}A
\end{equation}
那么称$E$为\textbf{Lebesgue可测集}\footnote{$E^C$表示$E$的补集.},简称\textbf{可测集}.此时记$\opn{m^*}E=\opn{m}E$.

不是可测集的集合,就称为\textbf{不可测集}.

\end{definition}

这个定义初看很绕口,我们在这儿多加些解释.可测集$E$可以看成一种分割,它可以把任何集合$A$都分成两部分,使得分出来的两部分的测度之和就等于$A$的测度.像\autoref{SetMet_ex1}~\upref{SetMet}中最后构造的两个集合$\bigcup_{i=1}^k S_i$和$S_{k+1}$,它们都是\textbf{不可测集},只需要取$A=\bigcup^{k+1}_{i=1} S_i$再比较\autoref{MsbSet_def1} 即可证明这一点.

看起来可测集的定义要求很严格,对$A$毫无约束,那么可测集存在吗?答案是肯定的,最简单的例子就是测度为零的集合,又叫零测集.

\begin{example}{可测集的例子:零测集}

如果$\opn{m^*}E=0$,那么$E$是可测的.

任取$A\in\mathbb{R}^n$,那么由外测度的单调性,$\opn{m^*}(A\cap E)=0$.

又由外测度的次可加性,
\begin{equation}\label{MsbSet_eq1}
\begin{aligned}
\opn{m^*}(A)&=\opn{m^*}([A\cap E]\cup[A\cap E^C])\\
&\leq \opn{m^*}(A\cap E)+\opn{m^*}(A\cap E^C)\\
&=\opn{m^*}(A\cap E^C)
\end{aligned}
\end{equation}

再由单调性知,\autoref{MsbSet_eq1} 中的不等号应为等号.由此得证.

\end{example}

零测集是可测的,这很显然,但我们毕竟是要研究积分,总要讨论非零的测度.那么有没有非零测的集合也是可测集呢?答案也是肯定的.

\begin{example}{可测集的例子:区间}\label{MsbSet_ex1}

任意区间$I$都是可测的.

任取$A\in\mathbb{R}^n$,不失一般性\footnote{这里不失一般性是因为,就算$A$中有点在$I$的边界上,由于边界是零测集,故从$A$中挖去这些点后,其外测度依然不变.}地,设$A$中没有任何点在区间$I$的边界上.那么我们总可以从覆盖$A$的开集中挖去$I$的边界点,剩下的依然是覆盖$A$的开集.这样的开集总是被$I$分成不相交的两部分,这两部分也都是开集.由开集\textbf{体积}的完全可加性即可最终证出$\opn{m^*}(A\cap I)+\opn{m^*}(A\cap I^C)=\opn{m^*}A$.

\end{example}

更进一步,我们还可以通过已知的可测集来构造新的可测集,这由以下定理保证:

\begin{theorem}{可测集的性质}

\begin{enumerate}
\item $E$可测,则$E^C$可测;
\item $E_1$和$E_2$可测,则$E_1\cup E_2$和$E_1\cap E_2$可测;
\item 如果$\{E_i\}_{i=1}^{\infty}$是一列可测集合,且\textbf{两两不相交},那么对于任意$A\in\mathbb{R}^n$,都有:
\begin{equation}\label{MsbSet_eq4}
\opn{m^*}(A\cap \bigcup_{i=1}^{\infty}E_i)=\sum_{i=1}^{\infty}\opn{m^*}(A\cap E_i)
\end{equation}
\item 如果$\{E_i\}_{i=1}^{\infty}$是一列可测集合,那么$\bigcap_{i=1}^{\infty}E_i$和$\bigcup_{i=1}^{\infty}E_i$也都是可测的.
\end{enumerate}

\end{theorem}

\textbf{证明}:


\textbf{1.} 由\autoref{MsbSet_def1} 中$E$和$E^C$地位的对称性直接可得.注意$(E^C)^C=E$.

\textbf{2.} 任取$A\in\mathbb{R}^n$,并记$A_1=A\cap E_1$,$A_2=A\cap E_2-A_1$,$A_0=A-(A_1\cup A_2)$.


由于$E_1$可测,故$\opn{m^*}(A_1\cup A_0)=\opn{m^*}A_1+\opn{m^*}A_0$;类似地,$\opn{m^*}(A_2\cup A_0)=\opn{m^*}A_2+\opn{m^*}A_0$,$\opn{m^*}(A_1\cup A_2)=\opn{m^*}A_1+\opn{m^*}A_2$.

因此
\begin{equation}
\begin{aligned}
\opn{m^*}A&=\opn{m^*}A_1+\opn{m^*}(A_2\cup A_0)\\
&=\opn{m^*}A_1+\opn{m^*}A_2+\opn{m^*}A_0\\
&=\opn{m^*}(A_1\cup A_2)+\opn{m^*}A_0
\end{aligned}
\end{equation}
即
\begin{equation}
\opn{m^*}A=\opn{m^*}(A\cap [E_1\cap E_2])+\opn{m^*}(A\cap [E_1\cap E_2]^C)
\end{equation}
故$E_1\cup E_2$是可测的.

再由集合论的de Morgan公式,可知$E_1\cap E_2=(E_1^C\cup E_2^C)^C$,从而推论出$E_1\cap E_2$也是可测的.

由此还可以推论,有限多个可测集进行交与并运算,结果还是可测集.

\textbf{3.} 为方便计,不妨设$A\subseteq \bigcup_{i=1}^\infty E_i$\footnote{这样就有$A\cap \bigcup_{i=1}^{\infty}E_i=A$,方便书写和理解.},且记$A_i=A\cap E_i$.

由于各$E_i$可测,以及可测集的有限并还是可测集,可推知
\begin{equation}
\opn{m^*}(\bigcup_{i=1}^k A_i)=\sum_{i=1}^k \opn{m^*}A_i
\end{equation}
对任意正整数$k$成立.

设$\opn{m^*}A=c_0$,$\sum_{i=1}^k \opn{m^*}A_i=\opn{m^*}(\bigcup_{i=1}^k A_i)=c_k$,则只需要证明$\lim\limits_{k\to \infty}c_k=c_0$即得证\footnote{因为$\sum_{i=1}^\infty \opn{m^*}A_i=\lim\limits_{k=1}\sum_{i=1}^k\opn{m^*}A_i$.证明$\lim_{k\to\infty}c_k=c_0$相当于证明了$\opn{m^*}(\lim\limits_{k\to\infty}\bigcup^k_{i=1}A_i)=\lim\limits_{k\to\infty}\opn{m^*}(\bigcup^k_{i=1}A_i)$.}.而这由\autoref{SetMet_lem1}~\upref{SetMet}即可直接推得.%由外测度的单调性,可知$\{c_k\}$是单调递增的,而且$c_k\leq c_0$恒成立.那么$\lim\limits_{k\to\infty}c_k$存在且小于等于$c_0$.

\textbf{4.} 由de Morgan公式,$\bigcap_{i=1}^\infty E_i=[\bigcup_{i=1}^\infty E_i^C]^C$,因此只需证明$\bigcup_{i=1}^\infty E_i$是可测集即可.

我们首先证明,当$E_i$两两不交的时候,$\bigcup^\infty_{i=1}E_i$是可测的.

任取$A\in\mathbb{R}^n$,记$A_i=A\cap E_i$,$c_0=\opn{m^*}A$,$A_0=A-\bigcup_{i=1}^\infty E_i$,$c_k=\sum_{i=1}^k\opn{m^*}A_i$,那么$\lim\limits_{k\to \infty}c_k=\opn{m^*}(A\cap[\bigcup_{i=1}^\infty E_k])$.

又由于各$\bigcup_{i=1}^k E_i$都是可测的,故$\sum_{i=0}^k c_i=\opn{m^*}(\bigcup_{i=0}^k A_i)\leq\opn{m^*}(\bigcup_{i=0}^\infty A_i)=\opn{m^*}A$.令$k\to\infty$,则可得
\begin{equation}\label{MsbSet_eq2}
\opn{m^*}(A_0\cup [A-A_0])=\lim_{k\to \infty} \opn{m^*}(\bigcup_{i=0}^k A_i)\leq \opn{m^*} A
\end{equation}

由外测度的次可加性,\autoref{MsbSet_eq2} 的不等号同时也是$\geq$,综合起来应该取等号,即得
\begin{equation}
\opn{m^*}(A_0\cup [A-A_0]) = \opn{m^*} A
\end{equation}
从而得证$\bigcup_{i=1}^\infty E_i$是可测集.




\textbf{证毕}.










