% 氦原子中的对易算符

总哈密顿算
\begin{equation}
H = H_1 + H_2 + V_{12} \qquad H_i = K_i + \frac{L_i^2}{2r^2}
\end{equation}
$K_i$ 是纯径向算符(只和 $r_i$ 有关), 所有的 $L$ 都是纯角向算符(只和角度有关\upref{SphAM}). 不同电子的任意算符可对易, 纯径向算符和纯角向算符可对易.

最复杂的是 $V_{12}$,
\begin{equation}
[V_{12}, L^2] = [V_{12}, M] = 0
\end{equation}
从经典力学的角度来看这是成立的. 数值验证也同样成立: $\mel{L',M'}{\mathcal Y_{l,l}^{0,0}}{L,M} = \delta_{L,L'}\delta_{M,M'}$.
