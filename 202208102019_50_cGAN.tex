% 条件生成对抗网络
% 条件 生成 对抗

\textbf{条件生成对抗网络}(Conditional Generative Adversarial Nets, cGAN)是生成对抗网络的条件版本.可以通过简单地向模型输入数据来构建.

在无条件的生成模型中,对于生成的数据没有模式方面的控制,很有可能造成模式坍塌.而条件生成对抗网络的思想就是通过输入条件数据,来约束模型生成的数据的模式.输入的条件数据可以是类别标签,也可以是训练数据的一部分,又甚至是不同模式的数据.

与原始的生成对抗网络相同,条件生成对抗网络也是玩的是双人最小最大游戏,其目标函数为:
\begin{equation}
\mathop{\min}\limits_G \mathop {\max }\limits_D V(D,G)=E_{x\sim p_{data}(x)}[logD(x|y)]+E_{z\sim p_z(z)}[log(1-D(G(z|y)))]
\end{equation}
其中,$G$表示生成器,$D$表示判别器,$x$表示训练数据,$y$表示输入的条件数据.

条件生成对抗网络基本结构,如图1所示.
\begin{figure}[ht]
\centering
\includegraphics[width=12.5cm]{./figures/cGAN_1.png}
\caption{条件生成网络结构示意图} \label{cGAN_fig1}
\end{figure}

图2表示了用MINST数据集训练的条件生成对抗网络生成数字的部分实验结果.每一行均以对应的数字标签作为输入条件.
\begin{figure}[ht]
\centering
\includegraphics[width=14.25cm]{./figures/cGAN_2.png}
\caption{MINST数据集数字生成部分实验结果} \label{cGAN_fig2}
\end{figure}



\textbf{参考文献:}
\begin{enumerate}
\item M. Mirza and S. Osindero, “Conditional generative adversarial nets,” arXiv preprint arXiv:1411.1784, 2014.
\end{enumerate}