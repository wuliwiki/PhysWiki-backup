% 粒子物理学(综述)
% license CCBYSA3
% type Wiki

本文根据 CC-BY-SA 协议转载翻译自维基百科\href{https://en.wikipedia.org/wiki/Particle_physics}{相关文章}。

粒子物理学或高能物理学是研究构成物质和辐射的基本粒子和力的学科。该领域还研究从基本粒子到质子和中子尺度的粒子组合,而研究质子和中子组合的学科称为核物理学。

宇宙中的基本粒子在标准模型中被分类为费米子(物质粒子)和玻色子(传递力的粒子)。费米子有三代,然而普通物质仅由第一代费米子构成。第一代包括构成质子和中子的上夸克和下夸克,以及电子和电子中微子。已知由玻色子介导的三种基本相互作用是电磁相互作用、弱相互作用和强相互作用。

夸克不能单独存在,而是形成强子。含有奇数个夸克的强子称为重子,而含有偶数个夸克的强子称为介子。两个重子,质子和中子,构成了普通物质的大部分质量。介子是不稳定的,最长寿命的介子也只有几微秒。介子发生在由夸克组成的粒子之间的碰撞后,例如在宇宙射线中快速运动的质子和中子。介子也可以在回旋加速器或其他粒子加速器中产生。