% 可分元素的单扩张是可分扩张
% 可分扩张|分裂域|域同态|域嵌入

\pentry{待定}

\subsection{定理的描述}

本词条是专门用来证明下述\autoref{SprbE2_the1} 的,其自然语言的描述即是本词条的标题.

\begin{theorem}{}\label{SprbE2_the1}
域$\mathbb{F}$上的不可约可分多项式$f(x)$的分裂域$\mathbb{K}$是$\mathbb{F}$的可分扩张.
\end{theorem}

考虑\autoref{SpltFd_cor2}~\upref{SpltFd},我们还可以得到上述\autoref{SprbE2_the1} 的等价描述:

\begin{theorem}{}
如果$a$是域$\mathbb{F}$的可分元素,那么$\mathbb{F}(a)/\mathbb{F}$是可分扩张.
\end{theorem}



\subsection{定理的证明\footnote{该证明的思路取自University of Connecticut的Keith Conrad教授的讲义.}}
%证明思路来源:https://kconrad.math.uconn.edu/blurbs/galoistheory/separable1.pdf

\begin{lemma}{}\label{SprbE2_lem1}
设$\mathbb{L}/\mathbb{K}$是一个域扩张,且$[\mathbb{L}:\mathbb{K}]=n$.设$\sigma:\mathbb{K}\to\mathbb{F}$是一个域同态.

则:

1. $\sigma$开拓而得的域同态$\mathbb{L}\to\mathbb{F}$的数量\textbf{小于等于}$n$.

2. 如果$\mathbb{L}/\mathbb{K}$是\textbf{不可分}扩张,则$\sigma$开拓而得的域同态$\mathbb{L}\to\mathbb{F}$的数量\textbf{小于}$n$.

3. 如果$\mathbb{L}/\mathbb{K}$是\textbf{可分}扩张,则存在扩域$\mathbb{F}'/\mathbb{F}$,使得$\sigma$开拓而得的域同态$\mathbb{L}\to\mathbb{F}'$的数量\textbf{等于}$n$.

\end{lemma}




\subsubsection{\autoref{SprbE2_lem1} 中1. 的证明}

我们用数学归纳法来处理.

显然,当$n=1$时,$\mathbb{L}=\mathbb{K}$,定理成立.


















