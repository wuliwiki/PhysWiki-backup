% C++函数
% C++语言

\begin{issues}
\issueDraft
\end{issues}

\pentry{C++基础\upref{Cpp0}}
函数是组织好的,可重复使用的,用来实现相关功能的代码段.C++ 提供了许多\textbf{内建函数},比如 STL的sort(), cmath的sqrt().我们也可以自己创建函数,这被叫做\textbf{自定义函数}.使用自定义函数,可以让程序变的更简洁.

% 1函数的定义和声明
\subsection{函数的定义和声明}
我们先从几个例子开始
\begin{lstlisting}[language=cpp]
#include <iostream>
using namespace std;

int add1(int a, int b){//声明方式1
    return a+b;
}

int add2(int, int);//声明方式2

int main(){//main函数,每个c++程序运行都是调用main函数,其他自定义函数不能调用它
    int a, b;
    cin>>a>>b;
    cout<<add1(a, b)<<endl;//调用1
    cout<<add2(a, b)<<endl;//调用2
    return 0;
}

int add2(int a, int b){//函数2的代码
    return a+b;
}
\end{lstlisting}
这段代码简单地介绍了C++函数声明方式,现在给出函数声明的伪代码
\begin{lstlisting}[language=cpp]
数据类型 函数名(形参列表){
    函数体
    return 返回值;
}
\end{lstlisting}
备注:数据类型除了int char这些类型外还有一个叫void的类型,就相当于返回一个空类型,这个函数不用写return 返回值;

% 2函数的传值
\subsection{函数的传值}
先来看个例子
\begin{lstlisting}[language=cpp]
#include <iostream>
using namespace std;

void swap(int a, int b){
    int tmp = a;
    a = b;
    b = tmp;
}

int main(){
    int a, b;
    cin >> a >> b;
    swap(a, b);
    cout << a << " " << b;
    return 0;
}
\end{lstlisting}
这个函数的作用是交互两个数,但是由于交换的是在swap函数内部,不对main函数造成影响,所以想要达到原来的目标,a b变量应该改为引用变量
\begin{lstlisting}[language=cpp]
#include <iostream>
using namespace std;

void swap(int &a, int &b){
    int tmp = a;
    a = b;
    b = tmp;
}

int main(){
    int a, b;
    cin >> a >> b;
    swap(a, b);
    cout << a << " " << b;
    return 0;
}
\end{lstlisting}