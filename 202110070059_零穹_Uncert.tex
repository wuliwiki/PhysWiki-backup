% 不确定性原理
% 量子力学|动量|不确定|海森堡

\pentry{平均值\upref{QMavg}, 柯西不等式\upref{CSNeq}, 高斯波包\upref{GausPk}}

\subsection{位置—动量不确定原理}
单个粒子一维运动的波函数 $\psi(x)$ 的位置和动量的标准差为 $\sigma_x$ 和 $\sigma_p$
\begin{equation}\label{Uncert_eq2}
\sigma_x \sigma_p \geqslant \frac{\hbar}{2}
\end{equation}

\begin{example}{无限深势阱的束缚态}\label{Uncert_ex2}
(未完成)证明束缚态满足\autoref{Uncert_eq2}.
\end{example}

\begin{example}{高斯波包}\label{Uncert_ex1}
(未完成)证明高斯波包可以使\autoref{Uncert_eq2} 取等号.

已知高斯波包\autoref{GausPk_eq1}~\upref{GausPk} 形为
\begin{equation}
\psi(x)=A_0e^{-a(x-x_0)^2}e^{\I k_0x}
\end{equation}
我们来求它的动量和位置的不确定度的乘积 $\sigma_x\sigma_p$.
\begin{equation}
\ev{x}=\int_{-\infty}^{+\infty}x\abs{\psi}^2\dd x=\abs{A_0}^2\int_{-\infty}^{+\infty}xe^{-2a(x-x_0)^2}\dd x=0
\end{equation}
上式中,最后积分式为0是因为被积函数为奇函数.

\begin{equation}
\ev{p}=m\dv{\ev{x}}{t}=0
\end{equation}
\begin{equation}
\begin{aligned}
\ev{x^2}&=\int_{-\infty}^{+\infty}x^2\abs{\psi}^2\dd x=\abs{A_0}^2\int_{-\infty}^{+\infty}x^2e^{-2a(x-x_0)^2}\dd x\\
&=\abs{A_0}^2\int_{-\infty}^{+\infty}\qty[(x-x_0)^2+2x_0(x-x_0)+x_0^2]e^{-2a(x-x_0)^2}\dd x\\
&\overset{t=x-x_0}{=}\abs{A_0}^2\int_{-\infty}^{+\infty}\qty[t^2+2x_0t+x_0^2]e^{-2at^2}\dd t\\
&=\abs{A_0}^2\int_{-\infty}^{+\infty}t^2e^{-2at^2}\dd t
=\frac{\abs{A_0}^2}{\sqrt{(2a)^3}}\Gamma\qty(\frac{3}{2})\\
&=\frac{\abs{A_0}^2}{2}\sqrt{\frac{\pi}{(2a)^3}}
\end{aligned}
\end{equation}
\begin{equation}
\begin{aligned}
\ev{p^2}&=-\hbar^2 \int_{-\infty}^{+\infty}\psi^{*}\dv[2]{}{x}\psi\dd x\\
&=-\hbar^2\abs{A_0}^2\int_{-\infty}^{+\infty}\qty[(-2a(x-x_0)+\I k_0)^2-2a]e^{-2a(x-x_0)^2}\dd x\\
&=-\hbar^2\abs{A_0}^2\int_{-\infty}^{+\infty}4a^2(x-x_0)^2e^{-2a(x-x_0)^2}\dd x\\
&=-(2a\hbar\abs{A_0})^2\frac{1}{\sqrt{(2a)^3}}\Gamma\qty(\frac{3}{2})=-\sqrt{\frac{a\pi}{2}}(\hbar\abs{A_0})^2

\end{aligned}
\end{equation}

\end{example}

\subsection{不确定原理的拓展}
任意两个物理量 $A$ 和 $B$ 都满足
\begin{equation}\label{Uncert_eq5}
\sigma_a \sigma_b \geqslant \frac{1}{2\I}\mel{v}{[A, B]}{v}
\end{equation}
\autoref{Uncert_eq2} 可以看成是该式的特例: 令 $A = x, B = p$, 根据正则对易关系
\begin{equation}
[A, B] = [x, p] = \I\hbar
\end{equation}
代入\autoref{Uncert_eq5} 得\autoref{Uncert_eq2}. 

\subsection{证明}
令 $f = (A-a)v$, $g = (B-b)v$, 使用柯西不等式\upref{CSNeq} 有
\begin{equation}\label{Uncert_eq1}
\sigma_a^2 \sigma_b^2 = \braket{f}{f} \braket{g}{g} \geqslant \abs{\braket{f}{g}}^2 = (\Re\braket{f}{g})^2 + (\Im\braket{f}{g})^2
\end{equation}
其中
\begin{equation}
\begin{aligned}
\Im\braket{f}{g} &= \frac{1}{2\I}(\braket{f}{g} - \braket{g}{f})
= \frac{1}{2\I}\mel{v}{[(A-a), (B-b)]}{v}\\
&= \frac{1}{2\I}\mel{v}{[A, B]}{v}
\end{aligned}
\end{equation}
忽略\autoref{Uncert_eq1} 中的 $\Re$ 项(大于等于 0), 得到\autoref{Uncert_eq5}. 容易证明, $\mel{v}{[A, B]}{v}$ 必然是一个纯虚数, 所以\autoref{Uncert_eq5} 右边必为实数.
\addTODO{为什么一定要忽略实数项?为什么一定存在取等号的情况?}
