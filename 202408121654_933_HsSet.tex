% 集合(高中)
% keys 高中|集合
% license Xiao
% type Tutor

\begin{issues}
\issueDraft
\end{issues}

学习初中数学时,你很有可能感受到了,每个部分都有它自己的一套体系和语言(比如,几何证明时的“平行”$\mathbin{\!/\mkern-5mu/\!}$就不会出现在解方程的过程中,而解方程时的“未知数”$x$也不会成为几何证明的要素),每个部分原本就这样各自发展着,从未想过彼此之间有什么可以联系起来的可能。但经过一代代数学家们的不断努力,现代数学展现在了世人面前,它最重要的一个特点就是将整座数学大厦建立在了集合论的基础之上,从此各部分不再是孤立的。同时,集合也能够简洁而准确地表达其他数学内容。这使得,集合语言不但成为了所有数学分支的通用语,也使得不论母语是什么的研究者都可以用相同的语言来展现和理解某个数学思想。可以这么说,不能掌握集合,数学之路就寸步难行。

因此,高中数学理所当然地以“集合”作为一切后面学习的开始,同时,高中阶段也并不要求对集合进行过深的探索。下文会首先介绍集合和元素这两个概念、表示方法以及性质,然后再分别介绍集合与元素的关系以及集合之间的关系。

提醒一下:
\begin{itemize}
\item 这篇文章会涵盖不少与集合相关的新概念(当然也包括集合本身),它和你以往的数学经验不太会有直接的关系,但请不要被这些概念的数量吓倒,它只是名字比较新,事实上却与生活经验联系密切,每个完成高中阶段学习的同学都会认为“集合”是最简单的。
\item 请务必理解这部分的内容,哪怕暂时不熟悉这些新概念的名称,也一定要理解他们的意思。两个很有帮助的理解方向是:为什么需要这个概念?它和其他概念之间的有哪些差异?在文中也会尽可能指明它们来帮助理解。如果学有余力,借助这个机会可以感受一下高中数学思考方式的不同,它是会延伸到高等教育中数学乃至其他学科中。由于这些概念很常用,在使用时再逐步熟悉名称就可以,不必因为名称太多记不住而紧张。
\end{itemize}

\subsection{集合与元素}
\addTODO{配图:有一个人前来卖瓜}
\begin{example}{塑料袋与水果}\label{ex_HsSet_1}
我给了你一个塑料袋,求你帮我把它送给另一个人。塑料袋嘛,就是用来装东西的,里面装了三种水果:苹果、香蕉、橙子。至于多少个,我没说你也没看,反正不轻。

你拎着这些东西到了他那里。他随口问了一声:“这个塑料袋里,装没装苹果?”你打开一看,有,就回答他“装了。”你刚想放下,他又问:“这个塑料袋里,装没装西瓜?”你只能又打开一看,没有,就回答他“没装。”他又要张嘴问你,奈何你的手已经不堪重负,于是你把塑料袋放地上了,跟他说:“东西我都放这里了,都在塑料袋里面我也没动,挺多的你自己看吧。”
\end{example}

数学上,这个塑料袋称为集合,而里面的水果就称为元素。就像刚才的问问题的那个人只关注“塑料袋里有没有某种水果”一样,研究集合时,也只关注一个问题,那就是“这个集合里面有没有某个元素”。至于水果好不好吃、塑料袋会不会破、你的手疼不疼,那都不是集合需要关注的问题。

相信你现在已经对这个塑料袋大概有印象了,请记住塑料袋这个例子,每次遇到跟集合相关的问题时,用塑料袋来理解都蛮好用。现在我们回到数学上来,给出一个集合的概念\footnote{这里的集合概念是朴素的,之后因为引发了一些问题,导致数学家们又创立了新的定义,这个问题直到现在还莫衷一是。但就如一开始所说,高中阶段对集合的要求并不这么艰深,毕竟这事现在也没说太准。},不过其实跟刚才说的塑料袋是一样的,它跟塑料袋的区别也不过就是用了书面语罢了。

\begin{definition}{集合}\label{def_HsSet_2}
一定范围内,某些能够确定的(well-defined,也称良定义的)、不同的对象(object)构成的整体(collection)称为\textbf{集合}(set),常用大写拉丁字母指代$A,B,C,\cdots$,称为集合$A$、集合$B$、集合$C$等。

构成集合的每个对象称为该集合的\textbf{元素}\footnote{有的同学会好奇“集合可不可以作为元素”,结论是肯定的,但是在高中阶段不涉及。}(element),常用小写拉丁字母指代$a,b,c,\cdots$,称为元素$a$、元素$b$、元素$c$等。
\end{definition}

介绍完概念,一般会开始分析概念的特性,教材上一般会说“根据定义,集合的元素具有三个特性:确定性、互异性、无序性。”这三个特性有一点太唬人了,我们回到刚才送水果的例子里面来看一看:

\begin{itemize}
\item 如果被问到某种水果是否在里面,你只要打开塑料袋看看,能看到就是在里面,不能看到就是不在里面,不存在某样东西,你翻过塑料袋之后回答对方“我不确定它在不在里面”。
\item 每个种类的水果都是与众不同的。不管放了多少个苹果,打开塑料袋,如果里面有苹果,那就是有苹果。至于有几个,在判断水果的种类这个问题上,水果的数量(忽略0个这种情况)不会影响这个问题的答案。
\item 在塑料袋里,水果的位置、顺序是不确定的、不重要的。你不知道自己会先看到什么,但只要这些水果都保持在这个塑料袋里,就可以保证回答这个问题时都是一样的。当然,你可以为了容易观看和记录,在塑料袋里面给水果排队,但不论是按名称排,还是按颜色排,都不会影响这个问题的答案。
\end{itemize}

核心就是记住一句话“集合关注的只有一点:有没有某个元素”,下面是三道练手题:

\begin{example}{是否存在一个集合是由“好看的明星”构成的?}
由于“好看”是一个主观标准,而“明星”是一个不确定的概念,因此,除非给定一个可以判定的“明星”概念,并量化“好看”的标准(即确定一些量来审核颜值),否则根据确定性,在一般的语境下,不存在一个由“好看的明星”构成集合。
\end{example}

\begin{example}{由“1,2,3,3,2,1”构成的集合有几个元素?}
翻译成“塑料袋语言”:一个塑料袋里有两个苹果、两个香蕉、两个橙子,问有几种水果。答:三种。

书面语:由于集合中的元素具有互异性,因此集合中只有“1、2、3”三个元素。
\end{example}

\begin{example}{判断:一个班级的学生按出生日期排列和按身高排列构成不同的集合。}
如果将“班级”看成一个集合,将“学生”看成元素,则由于集合的无序性,不论怎么排序都不影响集合,因此错误。
\end{example}

在实际使用时,“互异性”常成为其它题目的考察背景,在高中阶段较为常见;“确定性”是较为深刻的,影响了集合本质的,因此多出现在数学专业的集合论的探讨中;“无序性”看似与生活中常接触到的“序”(量可以比大小)的概念相冲突,但在数学中,“序”是从“无序”中构建出来的\footnote{如果感兴趣可以参考\autoref{def_CartPr_2} 处有序对的构建思路},高中阶段基本也不涉及。

\subsubsection{集合的基数}

这里有一个“元素个数”的概念,在高中教材中一带而过。但由于很重要,这里提及一下,希望你有个印象。

就像询问塑料袋中水果有多少种一样,如果一个集合是有限集\footnote{事实上,对于无限集也有势的概念,在以后的学习会接触到。},那么定义集合的基数如下:

\begin{definition}{集合的基数}
有限集$M$中的元素个数,称作集合的\textbf{基数}(cardinality,也称作集合的\textbf{势}),用${\rm card}(M)$表示。规定${\rm card}(\varnothing)=0$。
\end{definition}

可以证明,对于两个有限集合$A,B$:
\begin{equation}
{\rm card}(A\cup B)={\rm card}(A)+{\rm card}(B)-{\rm card}(A\cap B).~
\end{equation}

\subsection{集合的表示方法}

定义中提到了可以用字母来表示某个集合或元素,但这样的表示太抽象了。研究时常常不光想要知道集合的名字,还想知道集合的样貌,于是数学家们设计了三种表示方法,它们用各自的手段展现一个集合的样子。

\subsubsection{枚举法}

由于只要确定了集合内的所有元素,集合就能确定下来了,枚举法的做法就是像拍X光片,把集合中所有的元素全部写在大括号内,例如:用$\{1,2,3\}$表示由数字$1,2,3$三个元素构成的集合。

特别地,由于有些集合的元素比较多,全都写出来的话比较麻烦,因此在不产生歧义的前提下,一般可以将大括号内部过多的元素用“$\cdots$”代替。

\begin{example}{用枚举法表示是从$0$到$50$的偶数构成的集合$A$}
$$A=\{0,2, \cdots ,48,50\}.~$$
\end{example}

\subsubsection{描述法}

每次都把集合所有的元素都写出来的确是清晰明了,但无法从本质上体现这个集合的特征。

构建集合时,通常会设计一个判断标准,满足判断标准的元素$x$就会认为在这个集合里,判断标准就称作\textbf{性质},记作$p(x)$\footnote{这里的用法是第一次出现,它表示的是把$p$作用在$x$上,产生一个结果,类比一下就像把笔作用在纸上产生了笔迹,这里先大概理解这个记法,在\enref{函数(高中)}{functi}部分还会有更深入的介绍。},意为$x$满足性质$p$。

描述法就能在表示集合时利用性质,体现特征,例如:
\begin{equation}
A=\begin{Bmatrix} x|p(x) \end{Bmatrix}.~
\end{equation}
“$|$”用于分割元素与性质。理解起来如下表:
\begin{table}[h]
\caption{描述法的符号与读法}\label{tab_HsSet1}
\centering
\begin{tabular}{|c|c|c|c|c|c|}
\hline
符号 & $\{$&$x$ &|& $p(x)$&$\}$ \\
\hline
读法 & 集合&由$x$构成&,$x$是满足& 性质$p(x)$&的所有元素。 \\
\hline
\end{tabular}
\end{table}

\begin{example}{用描述法表示是从$0$到$50$的偶数构成的集合$A$}
$$A=\{x|x\text{是大于等于}0\text{且小于等于}50\text{的偶数}\}=\{x|x=2k,k\text{是大于等于}0\text{且小于等于}25\text{的整数}\}.~$$
\end{example}

\subsubsection{图示法}
上面的两种方法都着眼于集合和元素的关系,在研究集合之间的关系时就不太直观、好用了。这时,一般会采用图示法,也就是用草图来表示集合。

\textbf{维恩图}(Veen diagram,也叫\textbf{文氏图}、\textbf{韦恩图})是一种用圆圈来表示集合的一种草图,在进行集合间关系的分析时,可以通过画阴影、图案等方式来分析关系,非常有效,具体效果可以在\autoref{sub_HsSet_1} 和\enref{集合的基本运算(高中)}{HsSeOp}中感受。
\addTODO{这个图应该改一下,图里面应该是集合名称“A”、“自然数”之类的,圆圈里面不应该写Veen图。}
\begin{figure}[ht]
\centering
\includegraphics[width=10cm]{./figures/e449e54347ae8e24.png}
\caption{Veen图} \label{fig_SufCnd_1}
\end{figure}

另一种表示方法是在数轴上用直线或曲线来表示范围,这种表示只适用于区间的表示方法,具体使用方式参见\autoref{sub_HsSet_2} 的“区间”。

注意:图示法的缺点就是不够严谨,也因为是草图,所以只能作为自己理解的辅助出现在草纸上,或在书本中作为辅助理解的工具,而不能作为理由直接出现在证明、计算过程或试卷上。

\subsection{元素与集合的关系}

经过刚才的接触,相信你已经对元素和集合的关系有了隐约的感觉,下面明确地给出定义。

\begin{definition}{属于与不属于}
若 $a$ 在集合 $A$ 中,称 $a$ \textbf{属于}(belong to)集合 $A$ ,记作:
\begin{equation}
a \in A.~
\end{equation}

若 $a$ 不在集合 $A$ 中,称 $a$ \textbf{不属于}(not belong to)集合 $A$,记作:
\begin{equation}
a\notin A~
\end{equation}
\end{definition}
是的,元素与集合之间就只有这两种关系。用维恩图表示就是这样:
\addTODO{维恩图:元素属于集合,元素不属于集合}

而由于确定性的要求,任意一个元素,要么属于一个集合,要么不属于这个集合,不存在第三种情况,即对所有的元素$a$:
\begin{equation}
a\in A\qquad\text{或者}\qquad a\not\in A.~
\end{equation}

对于\autoref{ex_HsSet_1} 而言对方的问题就是判断“苹果、西瓜和塑料袋的关系”,而你的回答就是“苹果$\in$塑料袋,西瓜$\notin$塑料袋”。

\subsection{特殊的集合}\label{sub_HsSet_2}

尽管下面应该开始研究集合间的关系了,但让我们先暂停一下打个岔,看一看那些未来会经常打交道的特殊集合。

\subsubsection{区间}

什么叫区间呢?形象上的理解就是在数轴上割出来一段,比如从$a$到$b$的一段吧(假设$a<b$)。区间有三种,但它们其实对应的都是同一段内容,他们的长度也都一样,区别就只是能不能取到端点:
\begin{itemize}
\item 两端都能取到,称为\textbf{闭区间},记作$[a,b]$,用集合的写法记为$\{x|a\leq x\leq b\}$;
\item 都取不到,称为\textbf{开区间},记作$(a,b)$,用集合的写法记为$\{x|a< x<b\}$;
\item 一边能取到,一边取不到,称为\textbf{半开半闭区间},记作$[a,b)$或$(a,b]$,用集合的写法记为$\{x|a\leq x< b\}$或$\{x|a< x\leq b\}$;。
\end{itemize}
在书写时取得到的点就用方括号“$[$”或者“$]$”,取不到的点就用圆括号“$($”或者“$)$”。
在数轴上,一般如果将取不到的点特意用空心圆表示,而能取到的就直接涂成一个大黑点。当然,也别太大太用力,把纸涂坏了。比如$[-1,2)$在数轴上表示为:
\addTODO{画个数轴的图,[-1,2)的区间}
由于数轴对应的是实数,所以在提及区间时,一定是在讲某些实数。

\begin{definition}{无穷\footnote{此处以定义给出是方便引用,不做定义理解。}}
\textbf{无穷}记作$\infty$。关于无穷的具体学习会在本科阶段进行,高中涉及到这个概念时不会超出下面三点范围:
\begin{enumerate}
\item 无穷不是个数字
\item 无穷不在实数集里,因此涉及无穷的区间端点都只能取“开”。
\item $-\infty$比所有实数都小,$+\infty$比所有实数都大。
\end{enumerate}
\end{definition}

全体实数构成的集合用区间也可以表示,记作$(-\infty,+\infty)$。如果想表示某个单独的不等关系时,利用区间就可以表示为:
\begin{itemize}
\item 小于b的数,记作$(-\infty,b)$,用集合的写法记为$\{x|x< b\}$;
\item 小于等于b的数,记作$(-\infty,b]$,用集合的写法记为$\{x|x\leq b\}$;
\item 大于a的数,记作$(a,+\infty)$,用集合的写法记为$\{x|x> a\}$;
\item 大于等于a的数,记作$[a,+\infty)$,用集合的写法记为$\{x|x\geq a\}$;
\end{itemize}

\subsubsection{空集}

就像有空塑料袋一样,也有一个“空集”的概念。

\begin{definition}{空集}
如果任何元素都不属于某个集合,则这个集合称为\textbf{空集}(empty set),记作 $\varnothing$,即对所有的元素$a$:
\begin{equation}
a\notin\varnothing.~
\end{equation}
\end{definition}

最开始学习的时候,会很容易把它和$0$联想起来,他们的确存在一些联系\footnote{空集的元素数量是0},但建议你学习时就还是把它当成“空塑料袋”就好了,尤其不要把空集当成$\{0\}$,后者是包含一个元素“$0$”的集合。

\begin{definition}{*有限集和无限集}
含有有限个元素的集合叫\textbf{有限集},含有无限个元素的集合叫\textbf{无限集}\footnote{这里给出是因为教材上有提及,但如前面所说高中不涉及这一部分}。
\end{definition}

因为空集元素个数是0,所以空集也是有限集。


\subsubsection{数集}

由于数字在数学领域有特别的地位,也非常常用,于是数学家们把“只有数字构成的集合”简称\footnote{随着学习的深入,你会越来越感受到这帮人真的是一个字都不愿意多说。}为\textbf{数集}(number set),某些特殊的数集采用特定的记号,如下表所示。

\begin{table}[ht]
\centering
\caption{特殊数集及符号}\label{tab_HsSet2}
\begin{tabular}{|c|c|c|c|c|c|}
\hline
集合名称 &自然数集  &正整数集  & 整数集 & 有理数集& 实数集 \\
\hline
集合记号\footnote{在高等数学领域,这些数集的记号为$\mathbb{N,N^+,Z,Q,R}$,既表示他们的地位特殊,同时这些集合的定义都是广泛明确的,使用这个记号会方便交流。} & ${\rm \mathbf{N}}$ & ${\rm \mathbf{N^+}}$ 或 ${\rm \mathbf{N^*}}$ & ${\rm \mathbf{Z}}$ & ${\rm \mathbf{Q}}$ & ${\rm \mathbf{R}}$ \\
\hline
\end{tabular}
\end{table}


\subsection{集合与集合的关系}\label{sub_HsSet_1}
下面我们来研究集合之间的关系,之前我们说过,一个集合完全由它的元素决定,因此,集合间的关系最终也都反映到元素上。
就像两个装的水果一样的塑料袋,我们这时会说他们“一样”一样,我们先来定义相等:
\begin{definition}{集合相等}
对集合A、B,若他们的元素完全相同,则称他们\textbf{相等}(equal),记作:
\begin{equation}
A=B.~
\end{equation}
\end{definition}

所以,这里就有一个判断集合相等的方式:挨个判断这两个集合里的全部元素是否一样。

假设你有两个塑料袋,一个是小袋子,另一个是大袋子。你先把小袋子装满了各种水果,然后把这个小袋子放进大袋子里。现在,大袋子里面有小袋子和里面的水果。接下来,有两种选择:一种是你不再往大袋子里添加任何东西,另一种是你可以再往大袋子里放一些不同种类的水果。无论你做了哪种选择,因为小袋子里的所有水果种类都已经包含在大袋子里,如果只关注袋子里的水果种类的话,那么我们说小袋子是大袋子的“子集”\footnote{注意要求的前提是不能把塑料袋也当成“一种水果”。但有时的研究考虑的是物品种类,这样塑料袋和各种水果就都算成不同的物品了,这个问题不在当前的讨论范围。如果zhuan ye dian}。这个“子集”有点类似于数量之间“小于等于”的关系。

\begin{definition}{子集}\label{def_HsSet_3}
对两个集合$A,B$,若$A$的所有元素都属于集合$B$,即对所有的元素$a$,只要有$a\in A$,就有$a\in B$,则称集合 $A$ 是集合 $B$ 的\textbf{子集}(subset),或者说集合$B$\textbf{包含}集合$A$、集合$A$\textbf{包含于}集合$B$,记作
\begin{equation}
A \subseteq B\qquad\text{或者}\qquad B \supseteq A~.
\end{equation}
否则,若存在$A$的某个元素不属于集合$B$,即$\exists a\in A,a\notin B$,则称集合 $A$ 不是集合$B$的子集,或者说集合$B$\textbf{不包含}集合$A$、集合$A$\textbf{不包含于}集合$B$,记作
\begin{equation}
A \nsubseteq B\qquad\text{或者}\qquad B \nsupseteq A~.
\end{equation}
规定,空集是任何集合的子集,即对任意一个集合$A$:
\begin{equation}
\varnothing \subseteq A~.
\end{equation}
\end{definition}

用维恩图表示就是这样的
\addTODO{维恩图:B包含A,不包含:AB相交、AB不相交、A包含B}

注意区分子集和属于,两者的对象一个是元素与集合的关系,一个是集合间的关系。根据定义,显然任何一个集合都是它本身的子集,即
\begin{equation}
A \subseteq A~.
\end{equation}

因此,对于任何一个集合 $A$ 都有空集和它自身是它的子集。$A=\varnothing$时,他自身就是空集。
同时,根据子集的定义,也可以得到另一个判定集合相等的方法:

\begin{theorem}{根据子集关系判断集合相等}
如果两个集合互为对方的子集,那么他们相等,即:
\begin{equation}\label{eq_HsSet_1}
A\subseteq B,B\subseteq A\implies A=B.~
\end{equation}
\end{theorem}

有时,想要明确表达两个包含关系的集合没有相等,就像只想研究数量之间“小于”的关系时,会用到真子集的概念。

\begin{definition}{真子集}
对于两个集合,$A$ 与 $B$,如果 $A\subseteq B$ ,并且 $A \ne B$,我们就说集合 $A$ 是集合 $B$ 的\textbf{真子集},或者集合 $A$ \textbf{真包含于}集合 $B$、集合 $B$ \textbf{真包含}集合 $A$,记作:\footnote{这里给出的记法是人教版高中课本上的,在高中阶段请只使用这种写法。事实上,还有$A\subset B,B\supset A$和$A\subsetneq B,B\supsetneq A$两种写法用来表示“$A$是$B$的真子集”,且前一种更常用。}
\begin{equation}
A \subsetneqq B\qquad\text{或者}\qquad B \supsetneqq A~.
\end{equation}
\end{definition}

注意,由于维恩图是草图,在使用维恩图时不易区分“真子集”与“子集”的概念,教材中使用图\addTODO{上面包含的图序号}来表示“真子集”。实际使用时,建议只表示“包含”关系,并在运算时时刻注意是否可以取等的条件,并作标记来防止错误。

\begin{exercise}{用枚举法列出满足$\{1\}\subseteq A\subsetneqq\{1,2,4\}$的所有集合$A$}
对$\{1\}\subseteq A$,由子集的定义,$1\in A$。对$A\subsetneqq\{1,2,4\}$由真子集的定义,$A\neq \{1,2,4\}$,且$2,4$可以属于$A$。因此:

$A$可以是$\{1\},\{1,2\},\{1,4\}$。
\end{exercise}
\subsubsection{全集}

就像\autoref{ex_HsSet_1} 里的“水果”,在具体研究时,通常会划定研究的范围,这个划定的范围含有要研究的全部元素。

\begin{definition}{全集}\label{def_HsSet_1}
研究集合间的关系时,如果要研究的集合全都是某个集合的子集,也即涉及到的要研究的元素全都在这个集合中,则称这个集合为\textbf{全集}(universal set),常用符号 $U$ 表示。
\end{definition}

\addTODO{全集维恩图}

全集是一个相对的概念,是为了规定研究范围而定下来的。尽管它的名字很容易给人一种“包含世间万事万物的集合”的感觉,但请不要把它和引号里的那个集合等同起来\footnote{那个集合是不存在的,或者说满足这个概念的“事物”不是集合。},他们没有任何关系。


\subsection{总结}

终于,从完全陌生开始,给高中的第一块内容构建了一个夯实的基础。概念很多,要记住的符号也很多,下面列出的是这一篇文章,涉及到的知识点,供你自查:

\begin{itemize}
\item 集合、元素的概念
\item 集合的三种表示方法
\item 特殊的集合:空集、全集、数集、区间的概念
\item 集合与元素的关系:属于、不属于的概念
\item 集合之间的关系:相等、子集、真子集的概念
\item 集合相等的判断方法
\end{itemize}
