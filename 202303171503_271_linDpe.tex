% 线性相关性
% keys 线性组合|线性相关|矢量|共线|线性相关组|线性无关组|矩阵

\pentry{几何矢量的运算\upref{GVecOp}}

如果存在至少一组\textbf{不全为零}系数 $c_i$ 使几个矢量的线性组合等于零, 即
\begin{equation}\label{linDpe_eq2}
\sum_i^N c_i \bvec v_i = \bvec 0
\end{equation}
那这些矢量就被称为\textbf{线性相关(linearly dependent)}的。 这是因为对于任何一个 $c_j$ 不为零的项, 矢量 $\bvec v_j$ 都可以表示为其他矢量的线性组合。 只需把上式除以 $c_j$ 即可
\begin{equation}\label{linDpe_eq3}
\bvec v_j = -\sum_{i \ne j}\frac{c_i}{c_j} \bvec v_i
\end{equation}
如果不存在满足\autoref{linDpe_eq2} 的非零系数 $c_i$, 这些矢量就是\textbf{线性无关(linearly independent)}的, 即任何矢量都不可能被其他矢量的线性组合表示。

\begin{example}{ }\label{linDpe_ex1}
我们来看在三维几何矢量空间中, 线性无关有什么几何意义。 若两个矢量 $\bvec v_1$ 和 $\bvec v_2$ 线性相关, 意味着存在不全为零实数 $c_1, c_2$ 使
\begin{equation}
c_1 \bvec v_1 + c_2 \bvec v_2 = \bvec 0
\end{equation}
假设 $c_1$ 不为零, 则 $\bvec v_1 = c_2 \bvec v_2 / c_1$。 这个推导可逆, 所以\textbf{两个几何矢量线性相关当且仅当它们共线}, 或者说两个几何矢量线性无关当且仅当它们不共线。

再来看三个矢量的情况。 类比两个矢量的情况, 则线性相关意味着
\begin{equation}
\bvec v_3 = \frac{c_1}{c_3} \bvec v_1 +  \frac{c_2}{c_3} \bvec v_2
\end{equation}
由几何矢量加法的(几何)定义, 要么这三个矢量都共线, 要么 $\bvec v_3$ 落在 $\bvec v_1$ 和 $\bvec v_2$ 所在的平面上。 该过程的逆过程也成立, 所以\textbf{三个几何矢量线性相关当且仅当它们都共线或者共面}, 或者说三个几何矢量线性无关当且仅当它们不共面且两两不共线。
\end{example}

如果一个矢量集合中的矢量是线性相关的,那么这个集合被称为一个\textbf{线性相关组};反之,若线性无关,则称为一个\textbf{线性无关组}。

如果一组矢量之间线性相关,那么至少有一个矢量是“冗余”的,也就是说,它可以被其它矢量的线性组合表示出来。这样一来,对于线性相关的矢量组,如果用它们的线性组合来表示其它矢量,那么表示方式都不是唯一的。线性无关的矢量组,最重要的性质就是它们的线性组合表达式是唯一的,由此引入了基底和坐标\upref{Gvec2}等概念。

显然, 给定一个非零的线性相关组, 通过逐个移除这些“冗余”的矢量, 我们总可以得到一个线性无关组。

以后我们会看到, 若将 $M$ 个 $N$ 维空间的几何矢量的坐标表示为 $N\times M$ 的矩阵, 可以用这个矩阵的秩\upref{MatRnk} $R$ 来判断其中线性无关矢量的个数。 当且仅当 $R = M \leqslant N$ 时, 这 $N$ 个矢量才是线性无关的。
