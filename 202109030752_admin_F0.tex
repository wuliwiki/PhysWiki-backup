% 零函数

\begin{issues}
\issueDraft
\issueOther{这主要是为了解决简并空间散射态的归一化问题}
\end{issues}

\pentry{狄拉克 delta 函数\upref{Delta}}
我们用函数列严格定义狄拉克 $\delta$ 函数, 那么类似地, 下面我们也定义一类函数, 姑且称为\textbf{零函数}. 它的性质比 $\delta$ 函数更简单
\begin{definition}{零函数}
对任何性质良好的函数\footnote{例如迪利克雷条件.} $f: \mathbb R \to \mathbb C$, 若函数列 $z_n: \mathbb R \to \mathbb C$ ($n = 1, 2, \dots$) 满足
\begin{equation}
\lim_{n\to \infty}\int_{-\infty}^{\infty} z_n(x) f(x) = 0
\end{equation}
那么就把 $z_n(x)$ 称为\textbf{零函数}.
\end{definition}

\begin{example}{}
简谐函数 $\sin(nx + \phi)$, $\exp(\I n x)$ 都是零函数. 更一般地, 若 $f(x)$ 是一个性质良好的函数, $f(x)\sin(nx + \phi)$ 和 $f(x)\exp(\I n x)$ 也都是零函数.
\end{example}

为什么定义零函数呢? 我们可以依次判断量子力学中两个散射态是否
