% 流
% license Usr
% type Tutor
我们知道,对切向量场积分可以得到一族曲线,比如地球的一系列纬线。如果加上初始条件,便可以确定具体是哪一条曲线。因此,流形$M$上的积分曲线实际上有两个含参变量,设其为$\theta(t,p),\forall p\in M$。我们可以固定$p$点为曲线的初始位置,得到一条曲线:$\theta_p(t)$,亦可以固定$t$,得到$\theta_t(p):M\rightarrow M$。
\subsection{全局流}
令$p\in M$为曲线的初始位置,即$\theta(0,p)=p$。若$t\in\mathbb R$,我们称曲线族$\theta(t,p):\mathbb R\times M\rightarrow M$为\textbf{全局流(global flow)}。

可以证明\footnote{translation lemma proved in \textsl{introduction to smooth maniflod}},对于$\theta_p(t)$我们有$\theta_s\theta_p(t)=\theta_{p+s}(t)$。
\subsection{局域流}
\subsection{流的基本定理}
