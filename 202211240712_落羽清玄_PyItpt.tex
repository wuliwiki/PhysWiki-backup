% Python 解释器
% Python|解释器

Linux/Unix的系统上,一般默认的 python 版本为 2.x,我们可以将 python3.x 安装在 /usr/local/python3 目录中.

安装完成后,我们可以将路径 /usr/local/python3/bin 添加到您的 Linux/Unix 操作系统的环境变量中,这样您就可以通过 shell 终端输入下面的命令来启动 Python3 .

\begin{lstlisting}[language=bash]
$ PATH=$PATH:/usr/local/python3/bin/python3    # 设置环境变量
$ python3 --version
Python 3.4.0
\end{lstlisting}

在Window系统下你可以通过以下命令来设置Python的环境变量,假设你的Python安装在 C:\Python34 下:

\begin{lstlisting}[language=bash]
set path=%path%;C:\python34
\end{lstlisting}

\subsubsection{交互式编程}

我们可以在命令提示符中输入"Python"命令来启动Python解释器:

\begin{lstlisting}[language=bash]
$ python3
\end{lstlisting}

执行以上命令后,出现如下窗口信息:

\begin{lstlisting}[language=bash]
$ python3
Python 3.4.0 (default, Apr 11 2014, 13:05:11) 
[GCC 4.8.2] on linux
Type "help", "copyright", "credits" or "license" for more information.
>>> 
\end{lstlisting}

在 python 提示符中输入以下语句,然后按回车键查看运行效果:

\begin{lstlisting}[language=bash]
print ("Hello, Python!");
\end{lstlisting}

以上命令执行结果如下:

\begin{lstlisting}[language=bash]
Hello, Python!
\end{lstlisting}

当键入一个多行结构时,续行是必须的.我们可以看下如下 if 语句:

\begin{lstlisting}[language=bash]
>>> flag = True
>>> if flag :
...     print("flag 条件为 True!")
... 
flag 条件为 True!
\end{lstlisting}

\subsubsection{脚本式编程}

将如下代码拷贝至 hello.py文件中:

\begin{lstlisting}[language=python]
print("Hello, Python!")
\end{lstlisting}

通过以下命令执行该脚本:

\begin{lstlisting}[language=bash]
python3 hello.py
\end{lstlisting}

输出结果为:

\begin{lstlisting}[language=python]
Hello, Python!
\end{lstlisting}

在Linux/Unix系统中,你可以在脚本顶部添加以下命令让Python脚本可以像SHELL脚本一样可直接执行:

\begin{lstlisting}[language=python]
#! /usr/bin/env python3
\end{lstlisting}

然后修改脚本权限,使其有执行权限,命令如下:

\begin{lstlisting}[language=bash]
$ chmod +x hello.py
\end{lstlisting}

执行以下命令:

\begin{lstlisting}[language=bash]
./hello.py
\end{lstlisting}

输出结果为:

\begin{lstlisting}[language=bash]
Hello, Python!
\end{lstlisting}
