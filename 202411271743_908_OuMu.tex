% 欧姆定律(综述)
% license CCBYSA3
% type Wiki

本文根据 CC-BY-SA 协议转载翻译自维基百科\href{https://en.wikipedia.org/wiki/Ohm\%27s_law}{相关文章}。

\begin{figure}[ht]
\centering
\includegraphics[width=6cm]{./figures/d4ea765673f5dcae.png}
\caption{欧姆定律的参数:V、I 和 R} \label{fig_OuMu_1}
\end{figure}
欧姆定律指出,通过导体的电流与其两端的电压成正比。引入比例常数“电阻”后,[1] 可以得到描述这种关系的三个数学公式:[2]  
\[
V = IR \quad \text{或} \quad I = \frac{V}{R} \quad \text{或} \quad R = \frac{V}{I}~
\]
其中,\(I\) 是通过导体的电流,\(V\) 是测量的导体两端的电压,\(R\) 是导体的电阻。更具体地说,欧姆定律指出,这个关系中的 \(R\) 是常数,独立于电流。[3] 如果电阻不是常数,那么之前的公式就不能称为欧姆定律,但它仍然可以作为静态/直流电阻的定义。[4] 欧姆定律是一个经验关系,准确描述了绝大多数电导材料在多个数量级的电流下的导电性。然而,一些材料不遵守欧姆定律,这些材料被称为非欧姆材料。

该定律以德国物理学家 Georg Ohm 的名字命名,他在 1827 年发表的一篇论文中,描述了通过包含不同长度电线的简单电路的施加电压和电流的测量。欧姆用一个比现代形式稍微复杂的方程来解释他的实验结果(参见下文的“历史”部分)。

在物理学中,欧姆定律一词也用于指代该定律的各种推广;例如,在电磁学和材料科学中使用的欧姆定律的矢量形式:
\[
\mathbf{J} = \sigma \mathbf{E},~
\]
其中 \( \mathbf{J} \) 是在某一位置的电流密度,\( \mathbf{E} \) 是该位置的电场,\( \sigma \)(希腊字母西格玛)是一个依赖于材料的参数,称为导电率,定义为电阻率 \( \rho \)(希腊字母罗) 的倒数。这个欧姆定律的重新表述是由古斯塔夫·基尔霍夫(Gustav Kirchhoff)提出的。[5]
\subsection{历史}
\begin{figure}[ht]
\centering
\includegraphics[width=6cm]{./figures/30c68ca69887ad9c.png}
\caption{乔治·欧姆} \label{fig_OuMu_2}
\end{figure}
在1781年,乔治·欧姆的工作之前,亨利·卡文迪许进行了实验,使用了不同直径和长度的莱顿瓶和玻璃管,填充了盐溶液。他通过注意完成电路时身体感受到的电击强度来测量电流。卡文迪许写道,“速度”(电流)与“电气化程度”(电压)成正比。当时,他没有将结果与其他科学家分享,[6] 直到詹姆斯·克拉克·麦克斯韦在1879年发布了这些结果。[7]

弗朗西斯·罗纳兹在1814年使用金叶电计,划分了“强度”(电压)和“量”(电流)在干电堆(一种高电压源)中的关系。他发现,在某些气象条件下,干电堆中两者之间的关系并不是成正比的。[8][9]

欧姆于1825年和1826年进行电阻的研究,并于1827年将其结果出版为《Die galvanische Kette, mathematisch bearbeitet》(《电化学电路的数学分析》)。[10] 他在理论解释中受到了约瑟夫·傅里叶在热传导方面研究的很大启发。在实验中,欧姆最初使用了伏打电堆,但后来改用热电偶,因为热电偶在内阻和恒定电压方面提供了更稳定的电压源。他使用了电流计来测量电流,并知道热电偶接点之间的电压与接点温度成正比。然后,他加入了不同长度、直径和材料的测试导线以完成电路。他发现他的数据可以通过以下方程建模:
\[ x = \frac{a}{b + \ell}~\]
其中 \( x \) 是电流计的读数,\( \ell \) 是测试导体的长度,\( a \) 取决于热电偶接点温度,\( b \) 是整个系统的常数。由此,欧姆得出了他的比例定律并公布了他的结果。
\begin{figure}[ht]
\centering
\includegraphics[width=8cm]{./figures/4f5840d8a6863d90.png}
\caption{内阻模型} \label{fig_OuMu_3}
\end{figure}
用现代符号表示,我们可以写为:
\[
I = \frac{\mathcal{E}}{r + R}~
\]
其中,\(\mathcal{E}\) 是热电偶的开路电动势,\(r\) 是热电偶的内阻,\(R\) 是测试电线的电阻。根据电线的长度,这可以表示为:
\[
I = \frac{\mathcal{E}}{r + \mathcal{R}\ell}~
\]
其中,\(\mathcal{R}\) 是测试电线每单位长度的电阻。这样,欧姆定律的系数为:
\[
a = \frac{\mathcal{E}}{\mathcal{R}}, \quad b = \frac{\mathcal{r}}{\mathcal{R}}.~
\]
\begin{figure}[ht]
\centering
\includegraphics[width=6cm]{./figures/49b2f6464922e3ec.png}
\caption{《欧姆定律》在乔治·欧姆的实验记录本中。} \label{fig_OuMu_4}
\end{figure}
欧姆定律可能是最早的、描述电学物理现象的重要定量定律之一。今天,我们几乎认为它是显而易见的。然而,当欧姆首次发表他的研究成果时,情况并非如此;当时的批评者对他所阐述的内容表现出敌对态度。他们称他的工作是“一张赤裸裸的幻想网”,而且教育部长宣称,“一位传播这种异端思想的教授不配教授科学”。当时德国的主流科学哲学认为,自然界的规律是井然有序的,因而不需要通过实验来理解自然,科学真理可以通过推理得出。再者,欧姆的兄弟马丁,一位数学家,也在与德国教育体制作斗争。这些因素阻碍了欧姆的研究成果的接受,直到1840年代,他的工作才广泛被接受。然而,欧姆在去世之前就因其对科学的贡献获得了认可。

到了1850年代,欧姆定律已被广泛知晓并被认为是经过证实的。诸如“巴洛定律”等替代理论被淘汰,尤其是在电报系统设计的实际应用中,正如塞缪尔·摩尔斯在1855年所讨论的那样。

1897年,J.J.汤姆森发现了电子,并很快意识到电子是电路中携带电流的粒子(电荷载体)。1900年,保罗·德鲁德提出了第一个(经典的)电导模型——德鲁德模型,最终为欧姆定律提供了科学解释。在这个模型中,固体导体由静止的原子晶格(离子)构成,导电电子在其中随机运动。施加电压在导体上形成电场,使电子在电场方向上加速,导致电子的漂移,从而产生电流。然而,电子与原子发生碰撞,导致它们散射并使运动变得随机,从而将动能转化为热能(热能)。通过使用统计分布,可以证明电子的平均漂移速度,进而电流,与电场成正比,并且在较大范围的电压下,电流和电压也呈线性关系。

1920年代量子力学的发展稍微改变了这一模型,但在现代理论中,电子的平均漂移速度仍然可以被证明与电场成正比,从而导出了欧姆定律。1927年,阿诺德·索末菲德将量子费米-狄拉克分布应用于德鲁德模型,得出了自由电子模型。一年后,费利克斯·布洛赫证明了电子在固体晶格中以波动的形式(布洛赫电子)运动,因此,德鲁德模型中假设的电子与晶格原子散射的过程并非主要过程;电子主要与杂质原子和材料中的缺陷发生散射。最终的继任者——现代的固体量子能带理论表明,固体中的电子不能像德鲁德模型中假设的那样吸收任意能量,而是被限制在能带内,能带之间存在禁带,电子不能具有禁带中的能量。能带间隙的大小是物质的特征,与其电阻率密切相关,解释了为什么有些物质是电导体,有些是半导体,而有些则是绝缘体。

尽管电导的旧单位“摩”(电阻单位欧姆的倒数)仍然在使用,但在1971年,新的单位“西门子”被采用,以此来纪念恩斯特·维尔纳·冯·西门子。正式论文中更倾向于使用“西门子”单位。

1920年代,研究发现,实际电阻器中的电流会存在统计波动,这些波动与温度有关,即使电压和电阻保持完全恒定;这种波动现在被称为约翰逊–奈奎斯特噪声,源于电荷的离散性质。这种热效应意味着,当电流和电压的测量时间足够短时,测得的V/I比值会与通过时间平均或集体平均测量电流所得到的R值发生波动;但在普通电阻材料的情况下,欧姆定律仍然适用于平均电流。

欧姆的工作远早于麦克斯韦方程和对交流电路频率依赖效应的任何理解。现代电磁理论和电路理论的发展,在适当的限制条件下并不与欧姆定律相冲突。
\subsection{适用范围} 
欧姆定律是一个经验定律,来自许多实验的归纳,这些实验表明,对于大多数材料,电流大致与电场成正比。它不如麦克斯韦方程那样基本,并且并不总是成立。任何给定的材料在强电场下都会崩溃,一些在电气工程中有用的材料在弱电场下是“非欧姆”的。[15][16]

欧姆定律已经在不同的长度尺度上得到了验证。20世纪初,人们曾认为欧姆定律在原子尺度下会失效,但实验结果并未验证这一预期。到2012年,研究人员已经证明,欧姆定律适用于宽度仅为四个原子、高度为一个原子的硅线。[17]
\subsection{微观起源}
电流密度对施加的电场的依赖本质上是量子力学性质的(参见经典与量子导电性)。基于经典力学的定性描述,欧姆定律可以通过保罗·德鲁德(Paul Drude)于1900年提出的德鲁德模型来推导。[18][19]

德鲁德模型将电子(或其他载流子)视为在材料结构中与离子碰撞的弹球。电子会在电场的作用下沿电场的反方向加速。每次碰撞时,电子都会在随机方向上发生偏转,其速度远大于电场所提供的加速速度。最终的结果是,电子由于碰撞而采取锯齿形路径,但通常会朝着与电场方向相反的方向漂移。

漂移速度决定了电流密度及其与电场的关系,并且与碰撞无关。德鲁德通过公式 \( p = -eE\tau \) 计算了电子的平均漂移速度,其中 \( p \) 是平均动量,\( -e \) 是电子的电荷,\( \tau \) 是平均碰撞时间。由于动量和电流密度都与漂移速度成正比,因此电流密度与施加的电场成正比;这就得出了欧姆定律。