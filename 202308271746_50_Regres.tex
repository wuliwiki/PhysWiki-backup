% 回归
% keys 回归 机器学习
% license Xiao
% type Tutor

\textbf{回归}(Regression),在统计学中,是一种用于估计自变量(机器学习中称特征、属性)和因变量(机器学习中称标签)之间相关关系的分析方法[1]。回归分析的过程是确定最能够代表数据趋势的直线或者曲线[2]。所求得的回归直线或曲线,又可以称为拟合直线或拟合曲线。

回归也是一种机器学习中的基本建模方法。当所需要预测的数据是\textbf{连续型}数值时,该学习任务就是回归任务,须要用到回归方法,所求得的模型可以称为回归模型。这点是回归与\textbf{分类}\upref{Class}的主要区别。分类模型所预测的值是离散型数据。

\begin{table}[ht]
\centering
\caption{请输入表格标题}\label{tab_Regres1}
\begin{tabular}{|c|c|c|c|c|c|c|c|c|c|c|}
\hline
编号 & 性别 & 年龄 & 职业 & 睡眠时间(小时) & BMI指数 & 心率 & 舒张压 & 收缩压 & 每日走路步数 & 睡眠障碍 \\\hline
1 & 男 & 27 & 软件工程师 & 6.1 & 超重 & 77 & 83 & 126 & 4200 & 无 \\
\hline
2 & 男 & 28 & 医生 & 6.2 & 正常 & 75 & 80 & 125 & 10000 & 无 \\
\hline
3 & 女 & 30 & 护士 & 6.4 & 正常 & 78 & 86 & 130 & 4100 & 睡眠暂停 \\
\hline
4 & 男 & 29 & 教师 & 6.3 & 肥胖 & 82 & 90 & 140 & 3500 & 失眠 \\
\hline
\end{tabular}
\end{table}

表1所示的是一个简单的睡眠数据集。


\subsubsection{参考文献:}
\begin{enumerate}
\item https://en.wikipedia.org/wiki/Regression_analysis
\item https://www.britannica.com/topic/regression-statistics
\end{enumerate}