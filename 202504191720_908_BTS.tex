% 巴拿赫-塔斯基定理(综述)
% license CCBYNCSA3
% type Wiki

本文根据 CC-BY-SA 协议转载翻译自维基百科\href{https://en.wikipedia.org/wiki/Banach\%E2\%80\%93Tarski_paradox}{相关文章}。

\begin{figure}[ht]
\centering
\includegraphics[width=10cm]{./figures/a4f2dc0381172744.png}
\caption{} \label{fig_BTS_1}
\end{figure}
巴拿赫–塔尔斯基悖论是集合论几何中的一个定理,其内容如下:给定三维空间中的一个实心球体,存在一种将该球体分解为有限个不相交子集的方式,这些子集可以以不同的方式重新组合,从而得到两个与原球体完全相同的副本。实际上,重新组合过程仅涉及移动和旋转这些部分,而不改变它们的原始形状。然而,这些部分本身并不是传统意义上的“固体”,而是无穷多个点的散布。重构可以通过最少五个部分来实现。\(^\text{[1]}\)

定理的另一种形式表述为:给定任意两个“合理的”固体物体(例如一个小球和一个巨大球),这两个物体的切割部分可以重新组合成对方。这通常被非正式地表述为“一个豌豆可以被切割并重新组合成太阳”,并被称为\textbf{“豌豆和太阳悖论”}。

该定理是一个真实悖论:它与基本的几何直觉相矛盾,但并不是错误的或自相矛盾的。通过将球体分割成部分并通过旋转和平移来移动它们,而没有任何拉伸、弯曲或添加新点,“将球体翻倍”似乎是不可能的,因为所有这些操作从直觉上讲都应该保持体积不变。这样的操作保持体积的直觉并不是数学上荒谬的,甚至它也包含在体积的正式定义中。然而,这里不适用这种直觉,因为在这种情况下,无法定义所考虑子集的体积。重新组合它们会产生一个具有体积的集合,而这个体积恰好与开始时的体积不同。

与大多数几何定理不同,这个结果的数学证明在关键的方面依赖于集合论公理的选择。它可以通过选择公理来证明,该公理允许构造不可测量的集合,即没有普通意义上体积的点集,而其构造需要不可数次的选择。\(^\text{[2]}\)

2005年曾证明,分解中的各个部分可以以一种方式选择,使它们可以连续地移动到适当的位置,而不会相互碰撞。\(^\text{[3]}\)

正如Leroy\(^\text{[4]}\)和Simpson\(^\text{[5]}\)独立证明的那样,Banach–Tarski悖论在使用局部空间而不是拓扑空间时不会违反体积。在这种抽象的设置中,可能存在没有点但仍然非空的子空间。悖论分解的部分在局部空间的意义上相互交叠得很厉害,以至于这些交集中的一些应该被赋予正的质量。允许将这种隐藏的质量考虑在内,局部空间理论使得欧几里得空间的所有子集(甚至所有子局部空间)都能得到令人满意的度量。
\subsection{Banach和Tarski的论文}

在1924年发表的一篇论文中,[6] Stefan Banach和Alfred Tarski给出了这种悖论分解的构造,基于Giuseppe Vitali关于单位区间的早期工作,以及Felix Hausdorff关于球体的悖论分解,讨论了与欧几里得空间中各维度子集分解相关的若干问题。他们证明了以下更一般的命题,即Banach–Tarski悖论的强形式:

给定欧几里得空间中任意两个有界子集A和B(至少三维空间),且它们都有非空的内部,存在将A和B划分为有限个不相交子集的方式,  
A = A₁ ∪ ⋯ ∪ Aₖ,  
B = B₁ ∪ ⋯ ∪ Bₖ (对于某个整数k),使得对于每个整数i(1 ≤ i ≤ k),集合Aᵢ和Bᵢ是全等的。  
现在设A为原始球体,B为两个平移后的原始球体的并集。那么该命题的意思是,原始球体A可以分割成若干个部分,然后通过旋转和平移使得结果是整个集合B,其中包含两个A的副本。

Banach–Tarski悖论的强形式在一维和二维空间中是错误的,但Banach和Tarski证明,如果允许计数无限多个子集,则一个类似的命题仍然成立。一维和二维与三维及更高维度之间的区别,源于三维空间中欧几里得运动群E(n)的更丰富结构。对于n = 1, 2,群是可解的,但对于n ≥ 3,它包含一个具有两个生成元的自由群。John von Neumann研究了使得悖论分解成为可能的等价群的性质,并引入了“可容群”这一概念。他还发现了一种平面上的悖论形式,其中使用保持面积不变的仿射变换代替通常的全等变换。

Tarski证明了,可容群恰恰是那些不存在悖论分解的群。由于Banach–Tarski悖论只需要自由子群,这导致了长期以来的von Neumann猜想,该猜想在1980年被推翻。