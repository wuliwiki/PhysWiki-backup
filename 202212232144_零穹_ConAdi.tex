% 浸渐不变量
% 浸渐不变量|绝热不变量

当系统由一些缓变参数 $\lambda_i$ 确定时,系统在运动过程中保持不变的量称为系统的\textbf{浸渐不变量}。这样的系统可以理解为处于一个外场当中,而参数 $\lambda_i$ 描述了系统所处外场的性质。例如处于三维静电场中的二维平面上的电荷系统,系统受到的场的作用与平面所处的位置有关,那么系统所处电场的性质可以用平面所处的 $z$ 坐标来描述(以平面作为 $xOy$ 平面)。为简单起见,我们假设只有一个参数 $\lambda$。

所谓的“缓变”,是指在一个运动周期 $T$ 内 $\lambda$ 的变化很小,即 
\begin{equation}\label{ConAdi_eq4}
\frac{\lambda(t+T)-\lambda(t)}{\lambda(t)}\rightarrow0
\end{equation}
由于 $\lambda(t+T)-\lambda(t)\approx T\dv{\lambda}{t}$,上式可写为(这里简单假设了 $\lambda>0$)
\begin{equation}
T\dv{\lambda}{t}\ll\lambda
\end{equation}
若 $\lambda$ 为常数,则系统是封闭的且能量守恒;若 $\lambda$ 非常数,则系统不封闭,能量不守恒。
\subsection{浸渐不变量的具体形式}
设 $H(p,q,\lambda)$ 是依赖于参数 $\lambda$ 的系统的哈密顿量。由(链接)(注意哈密顿量就是能量):
\begin{equation}\label{ConAdi_eq1}
\dv{E}{t}=\pdv{H}{t}=\pdv{H}{\lambda}\dv{\lambda}{t}
\end{equation}
上式右端不仅依赖于缓变量 $\lambda$,还依赖于快变量 $p,q$。为了消除快变量的影响,可以按周期取平均(\autoref{ConAdi_fig1} )。
\begin{figure}[ht]
\centering
\includegraphics[width=5cm]{./figures/ConAdi_1.pdf}
\caption{按周期取平均后,快变量平均值为0,故可消除快变量的影响} \label{ConAdi_fig1}
\end{figure}
在周期 $T$ 内取平均时,由于 $\lambda$ 变化缓慢 ($\dot\lambda$ 也看成缓变量),可以将 $\lambda,\dv{\lambda}{t}$ 看成常数,从而 $\dv{\lambda}{t}$ 可移到平均化符号外,于是\autoref{ConAdi_eq1} 取平均后有
\begin{equation}\label{ConAdi_eq3}
\overline{\dv{E}{t}}=\dv{\lambda}{t}\overline{\pdv{H}{\lambda}}
\end{equation}
其中
\begin{equation}\label{ConAdi_eq2}
\overline{\pdv{H}{\lambda}}=\frac{1}{T}\int_\tau^{\tau+T}\pdv{H}{\lambda}\dd t
\end{equation}
根据正则方程(\autoref{HamCan_eq2}~\upref{HamCan})$\dot q=\pdv{H}{p}$,有
\begin{equation}
\dd t=\frac{\dd q}{\partial{H}/\partial{p}}
\end{equation}
于是
\begin{equation}
T=\int_{\tau}^{\tau+T}\dd t=\oint\frac{\dd q}{\partial{H}/\partial{p}}
\end{equation}
上式的符号 $\oint$ 是因为周期运动,$q$ 的变化区域应形成个闭环。于是\autoref{ConAdi_eq2} 成为
\begin{equation}
\overline{\pdv{H}{\lambda}}=\frac{\oint\frac{\partial{H}/\partial{\lambda}}{\partial{H}/\partial{p}}\dd q}
{\oint\frac{\dd q}{\partial{H}/\partial{p}}}
\end{equation}
 
进而\autoref{ConAdi_eq3} 成为
\begin{equation}\label{ConAdi_eq5}
\overline{\dv{E}{t}}=\dv{\lambda}{t}
\frac{\oint\frac{\partial{H}/\partial{\lambda}}{\partial{H}/\partial{p}}\dd q}
{\oint\frac{\dd q}{\partial{H}/\partial{p}}}
\end{equation}

前面说过,在取平均值时 $\lambda$ 应看成常数,即上式中的积分是沿着 $\lambda$ 为常数的运动轨道进行的,即沿积分路径哈密顿量保持常值 $E$。而沿着运动路径,$\dot q$ 是 $q$ 的函数,拉氏量 $L$ 是 $q,\dot q,E,\lambda$ 的函数,所以 $p$ 可看成 $q,E,\lambda$ 的函数 $p(q;E,\lambda)$。故将方程 $H(p,q,\lambda)=E$ 对 $\lambda$ 求导,就有
\begin{equation}
\pdv{H}{\lambda}+\pdv{H}{p}\pdv{p}{\lambda}=0
\end{equation}
即 
\begin{equation}
\frac{\partial{H}/\partial{\lambda}}{\partial{H}/\partial{p}}=-\pdv{p}{\lambda}
\end{equation}
并且
\begin{equation}
\frac{\dd q}{\partial{H}/\partial{p}}=\pdv{p}{E}\dd q
\end{equation}

于是\autoref{ConAdi_eq5} 成为
\begin{equation}
\overline{\dv{E}{t}}=-\dv{\lambda}{t}
\frac{\oint\pdv{p}{\lambda}\dd q}
{\oint\pdv{p}{E}\dd q}
\end{equation}
上式可写为
\begin{equation}\label{ConAdi_eq6}
\oint\qty(\pdv{p}{E}\overline{\dv{E}{t}}+\dv{\lambda}{t}\pdv{p}{\lambda})\dd q=0
\end{equation}
上式相对于
\begin{equation}\label{ConAdi_eq7}
\overline{\dv{I}{t}}=0
\end{equation}
而
\begin{equation}
I=\frac{1}{2\pi}\oint p\dd q
\end{equation}
\begin{example}{}
试证明\autoref{ConAdi_eq6} 与\autoref{ConAdi_eq7} 等价。

\autoref{ConAdi_eq6} 与\autoref{ConAdi_eq7} 等价是因为 $I$ 中的积分变量是 $q$ ,积分完后 $I$ 与 $q$ 无关,而 $p=p(q;E,\lambda)$,所以最后 $I=I(E,\lambda)$,那么(注意 $E,\lambda$ 可看作只是 $t$ 的函数。)
\begin{equation}
\dv{I}{t}=\pdv{I}{E}\dv{E}{t}+\pdv{I}{\lambda}\dv{\lambda}{t}=\frac{1}{2\pi}\oint\qty(\pdv{p}{E}\dv{E}{t}+\dv{\lambda}{t}\pdv{p}{\lambda})\dd q
\end{equation}
因为 $\lambda,\dot\lambda$ 都缓慢变化,所以 $\dot E$ 也缓慢变化,所以在取平均时,可认为 $\dv{E}{t}$ 不变并可用 $\overline{\dv{E}{t}}$ 替代,于是
  \begin{equation}
  \begin{aligned}
\overline{\pdv{I}{E}\dv{E}{t}}&=\frac{1}{T}\int_t^{t+T}\oint\pdv{p}{E}\dv{E}{t}\dd q\dd t\\
&=\oint\pdv{p}{E}\qty(\frac{1}{T}\int_t^{t+T}\dv{E}{t}\dd t)\dd q\\
&=\oint\pdv{p}{E}\overline{\dv{E}{t}}\dd q
\end{aligned}
\end{equation}
\end{example}
 