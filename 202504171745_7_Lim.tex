% 极限
% keys 微积分|极限|数列极限|函数极限|无穷小
% license Xiao
% type Tutor

\pentry{数列的极限(简明微积分)\nref{nod_Lim0}}{nod_6fed}

\subsection{数列的极限}

微积分的核心概念是极限,而极限最基础的情形是数列的极限。数列是离散的,比较容易理解,而所有与极限有关的概念也都可以从数列的极限拓展得到。

先来看一个数列的例子。

\begin{example}{}\label{ex_Lim_1}
我们都知道 $\pi$ 是一个无理数,所以 $\pi$ 的小数部分是无限多的。目前用计算机,已经可以将 $\pi$ 精确地计算到小数点后数亿位。然而在实际应用中,往往只用取前几位小数的近似即可。下面给出一个数列,定义第 $n$ 项是 $\pi$ 的前 $n$ 位小数近似(不考虑四舍五入),即
\begin{equation}
a_0 = 3~,\,\, a_1 = 3.1~,\,\, a_2 = 3.14~,\,\, a_3 = 3.141,\,\dots~
\end{equation}
\end{example}

这个数列显而易见的性质,就是当 $n$ \textbf{趋于无穷}时,$a_n$ 趋(近)于 $\pi$。 无穷通常用符号 $\infty$ 来表示(像“8”横过来写)。我们把这类过程叫做\textbf{极限}。以上这种情况,用极限符号表示,就是
\begin{equation}
\lim_{n \to \infty } {a_n} = \pi ~.
\end{equation}
这里 $\lim$ 是极限(limit)的意思,下方用箭头表示某个量变化的趋势\footnote{$\lim\limits_{n \to \infty }$ 在这里相当于一个“操作”,叫\textbf{算符(operator)}, 它作用在数列 $a_n$ 上,把数列变成一个数,即该数列的极限。}。 算符的 “输出” 就是一个数( $a_n$ 的极限值)。所以不要误以为这条式子是说当 $n = \infty$ 时,$a_n=\pi$ \footnote{有两个理由可以说明这种理解不正确:首先,按定义,每个 $a_n$ 都是有理数,而 $\pi$ 是无理数,所以不应该有任何一个 $a_n=\pi$;其次,$\infty$ 不是一个实数,不存在 $n=\infty$ 的说法。这里的 $n\to\infty$ 只是表示 $n$ 的增大是没有限制的。},而要理解成数列 $a_n$ 经过算符 $\lim\limits_{n \to \infty }$ 的作用以后,得出其极限是 $\pi$。 类比函数 $\sin x = y$,并不是说 $x=y$, 而是说 $x$ 经过正弦函数作用后等于 $y$。 

所以从概念上来说,极限中的“趋于” 和“等于” 是不同的。趋于是数列整体的性质,而不是单个数字的性质。我们可以像这样粗略理解“趋近”:
\begin{itemize}
\item 越来越接近,但不一定相等
\item (在不相等的情况下)只有更近,没有最近
\end{itemize}

对极限来说,第2点成立是非常必要的。但是怎样能说明 “没有最近”呢?可以看出,当 $n$ 越大,$a_n$ 越接近 $\pi$, 它们的 “距离”,可以用 $\abs{a_n - \pi}$ 来表示。也就是说,对任何一个 $a_n$, 如果所对应的距离 $\abs{a_n - \pi } \ne 0$, 总能找到一个更大的数 $m>n$, 使 $\abs{a_m - \pi} < \abs{a_n - \pi}$ (也就是 $a_m$ 比 $a_n$ 更靠近 $\pi$),并且要求 $a_m$ 之后的所有项也都能满足这一条件。只有这样,才能从数学上说明上面两个意思。这就是极限思想的精髓。根据这个思想,下面可以写出数列极限的定义。



\begin{definition}{数列的极限}\label{def_Lim_2}
考虑数列 $\{a_n\}$。若存在一个实数 $A$,使得对于\textbf{任意}给定的\textbf{正实数} $\varepsilon > 0$(无论它有多么小),总存在正整数 $N_\epsilon$, 使得对于所有编号 $n>N_\epsilon$ ,都有 $\abs{a_n - A} < \varepsilon$ ($A$ 为常数) 成立,那么数列 $a_n$ 的极限就是 $A$。

将“数列 $\{a_n\}$ 的极限是 $A$”表示为 $\lim\limits_{n\to\infty}a_n=A$。
\end{definition}




由于极限的精髓是“无论多靠近,都能找到$N$使得$N$之后的项满足所要求的靠近程度”,因此极限的概念不仅限于数字序列,也可以推广到任何能定义\textbf{距离}的集合里。比如说,二维平面上可以依次选出点$P_1, P_2, \cdots$,就得到一个\textbf{点列},点列同样可以趋于平面上某一个点。当然,我们可以把任意集合里的元素都看作点,于是数列也是一种特殊的点列。把\autoref{def_Lim_2} 稍加修改,就能得到极限的一般定义:
\begin{definition}{距离函数}
考虑集合$X$,在$X$上定义一个函数$d:X\times X\to\mathbb{R}^+\cup\{0\}$,满足:
\begin{enumerate}
\item \textbf{对称性}:对于任意$x, y\in X$,都有$d(x, y)=d(y, x)$;
\item \textbf{正定性}:对于任意$x, y\in X$,都有$d(x, y)\geq 0$,且等号仅在$x=y$时成立;
\item \textbf{三角不等式}:对于任意$x, y, z\in X$,都有$d(x, z)\leq d(x, y)+d(y, z)$。
\end{enumerate}
则称$d$是一个\textbf{距离函数},$d(x, y)$是点$x$和点$y$之间的\textbf{距离}。
\end{definition}
\begin{definition}{点列的极限}
考虑集合$X$,在$X$上给定了距离函数$d$。

对于$X$中
\end{definition}





 
\begin{figure}[ht]
\centering
\includegraphics[width=14cm]{./figures/5cc5de9bdea4f6fe.pdf}
\caption{两个数列示意图。实心点表示数列 $a_n=(-1)^{n}$,空心点表示数列 $b_n=(1/2)^n$;横实线表示 $y=1$,横虚线表示 $y=1/2$。由图可见,随着 $n$ 增大,黑色点列虽然总有落在 $1$ 上的点,但也总有落在虚线以外的点;而空心点列则总是落在虚线以外。这样,虚线就像一个天堑,随着 $n$ 增大的时候两个数列都有被这个天堑隔开的点,这时我们就说这两个数列都不趋近于 $1$。不过,空心点数列 $\{b_n\}$ 是趋近于 $0$ 的。} \label{fig_Lim_1}
\end{figure}

% 在命题中,通常把 “任意” 用 “ $\forall$” (for all) 表示,把 “存在” 用 “$\exists $” (there exist(s), there is (are))表示。即“ 对 $\forall \varepsilon>0$, $\exists N$, 当 $n>N$ 时,有 $\abs{a_n - A} < \varepsilon$”。 

由于以上讨论中 $\lim$ 作用的对象是数列,那么箭头右边只能是 $\infty$ (准确来说应该是正无穷 $+\infty$, 但是由于数列的项一般是正的,所以正号省略了)。

把定义套用到上面的\autoref{ex_Lim_1} 中, 如果要求 $\abs{a_n - \pi} < 10^{-3}$ (给定 $\varepsilon  = 10^{-3}$),只要令 $N=3$ (当然也可以令 $N=4, N=5$, 等) 就可以保证第 $N$ 项后面所有的项都满足要求。 一般地如果给定 $\varepsilon  = b\e{-q}  (b > 1)$, 就令 $N = q$, 第 $N$ 项以后的项就满足要求。根据定义,这就意味着 $\lim\limits_{n \to \infty } a_n = \pi$。 

我们来看几个简单的例题,加深一下印象。

\begin{exercise}{}
考虑数列 $a_n=\frac{1}{2^n}$。根据定义,证明 $\lim\limits_{n\to\infty}a_n=0$。
\end{exercise}

\begin{exercise}{}\label{exe_Lim_1}
考虑数列 $a_n=(-1)^n$。这个数列存在极限吗?
\end{exercise}

\autoref{exe_Lim_1} 的数列是不存在极限的,因为它的值在 $\pm 1$ 之间反复横跳,也就是说对于任何实数 $A$,$\abs{a_n-A}$ 都只有最多两个值,而且其中一个肯定非零。这就导致如果我们把 $\epsilon>0$ 取得足够小(小于两个 $\abs{a_n-A}$ 中比较大的那个),那么不管 $N$ 多大,总有 $n>N$ 使得 $\abs{a_n-A}>\epsilon$。结果就是这个数列没有任何极限值。我们把这种情况称为\textbf{发散}。

\begin{definition}{数列的敛散性}\label{def_Lim_4}
如果一个数列 $\{a_n\}$ 不存在极限,就称它是\textbf{发散(divergent)}的。如果 $\{a_n\}$ 存在极限,则称它是\textbf{收敛(convergent)}的。
\end{definition}

\subsection{函数的极限}
实函数 $f(x)$ 可以看成是一种“连续”的数列,只不过把元素编号从离散的 $n$ 改为连续的 $x$。类比数列的极限, 我们也可以定义\textbf{函数在正无穷的极限} $\lim\limits_{x\to +\infty} f(x) = A$。

\begin{definition}{函数趋于正无穷时的极限}\label{def_Lim_1}
考虑实函数 $f(x)$。若存在实数 $A$,使得对于\textbf{任意}$\epsilon>0$,总存在\textbf{正实数}$X_\epsilon$,使得对于所有 $x>X_\epsilon$,都有 $\abs{f(x)-A}<\epsilon$,那么我们说 $A$ 是函数 $f(x)$ 在 $x$ 趋于正无穷时的极限。
\end{definition}

可以看到,\autoref{def_Lim_1} 和\autoref{def_Lim_2} 非常相似,只是简单做了替换。不过,函数并不是简单地把数列的概念拓展到连续的情况。数列的编号只能朝着一个方向增大,但实函数的自变量就自由得多,它可以奔向负无穷,也可以集中到一点 $x_0$。

如何描述“自变量趋于一个给定的实数 $x_0$”呢?我们可以拓展一下“趋于无穷”的概念。函数自变量或者数列编号趋于无穷,就是说我们可以把自变量和数列编号取得越来越“接近无穷”,虽然这种说法并不严谨,但它可以提供一个很好的借鉴:函数自变量趋于给定实数 $x_0$,就是说我们取的自变量 $x$ 使得 $\abs{x-x_0}$ 越来越接近 $0$。

现在问题来了,什么叫“越来越”呢?在讨论数列极限的时候,我们没有在意这个细节,因为我们也只能考虑数列编号增大的情况,而这里的“越来越”也自然表示“随着数列编号的增大”了。但是讨论函数自变量趋于给定实数 $x_0$ 的时候,就有些麻烦了,我们没有一个衡量“时间流逝”的自然标准了。要解决这个问题,最好还是再次把数列给请出来。

下面,我直接给出函数极限的定义,请仔细咀嚼,看看数列是怎么用来准确描述函数极限的。

\begin{definition}{函数的极限}\label{def_Lim_3}
考虑实函数 $f(x)$,并给定一个实数 $x_0$。

取数列 $\{x_n\}$,使得 $\lim\limits_{n\to\infty}=x_0$。这样,$\{f(x_n)\}$ 也构成一个数列。

如果对于\textbf{任意}的满足上述要求的数列 $\{x_n\}$,都有实数 $A$ 使得 $\lim\limits_{n\to\infty}f(x_n)=A$,那么我们说 $f(x)$ 在 $x$ 趋近于 $x_0$ 时的极限为 $A$。

将这个极限表述为 $\lim\limits_{x\to x_0}f(x)=A$。
\end{definition}

注意看\autoref{def_Lim_3} 中是如何从 $\lim\limits_{n\to\infty}=x_0$ 和 $\lim\limits_{n\to\infty}f(x_n)=A$ 过渡到 $\lim\limits_{x\to x_0}f(x)=A$ 的。这里,从离散到连续的桥梁,正是“任意”二字。也就是说,“连续”就是“任意的离散”,比如“连续地接近时满足的条件”就是“任意一种离散地接近时都满足的条件”。

\autoref{def_Lim_3} 同时也是最为完整的函数极限定义,只需要把 $x_0$ 替换为 $\pm\infty$ 即可囊括无穷的情况。

% 与数列不同的是, 对于函数我们还可以定义\textbf{函数在负无穷的极限} $\lim\limits_{x\to -\infty} f(x)$(把以上定义的 $>$ 号改成 $<$ 号即可)。

% 另外可以定义 \textbf{$f(x)$ 在 $x_0$ 处的极限} $A$, 即“ 对 $\forall \varepsilon > 0$, $\exists \delta > 0$, 当 $\abs{x - x_0} < \delta$ 时,有 $\abs{f(x) - A} < \varepsilon$”。 注意 $f(x)$ 不需要在 $x_0$ 处有定义。

\begin{example}{}
求函数在某个值处的极限时, 通常可以直接代入数值计算, 如
\begin{equation}
\lim_{x\to 1} 2x + 1 = 3 ~,\qquad \lim_{x\to 2}\frac{x + 1}{x + 2} = \frac34~.
\end{equation}

当无穷大与常数相加时, 可以忽略常数, 如
\begin{equation}
\lim_{x\to +\infty} \frac{x + 1}{2x + 2} = \lim_{x\to +\infty} \frac{x}{2x} = \frac12~.
\end{equation}
\end{example}

\begin{example}{}\label{ex_Lim_2}
考虑符号函数 $\opn{sgn}(x)$,其定义为:$x>0$ 时,$\opn{sgn}(x)=1$,$\opn{sgn}(-x)=-1$,且有 $\opn{sgn}(0)=0$。也就是说,正数的函数值为 $1$,负数的为 $2$,$0$ 的就是 $0$。

现在考虑 $\opn{sgn}(x)$ 在 $0$ 处的极限值。我们首先要研究那些趋近于 $0$ 的数列。

如果我们取数列 $g_n=\frac{1}{2^n}$,那么 $\opn{sgn}(g_n)$ 中各项都恒为 $1$,因此 $\lim\limits_{n\to\infty}\opn{sgn}(g_n)=1$。

但是如果取数列 $h_n=-\frac{1}{2^n}$,那么 $\lim\limits_{n\to\infty}\opn{sgn}(h_n)=-1$。

如果再取数列 $j_n=0$,那么 $\lim\limits_{n\to\infty}\opn{sgn}(j_n)=0$。

以上三个数列都趋于 $0$,但由它们构造的 $\opn{sgn}(g_n)$、$\opn{sgn}(h_n)$ 和 $\opn{sgn}(j_n)$ 的极限却各不相同。这就意味着 $\opn{sgn}(x)$ 在 $0$ 处\textbf{并没有极限值}。
\end{example}

\begin{definition}{函数的敛散性}
拓展数列的敛散性的\autoref{def_Lim_4}。若函数在一点处有极限值,则称之为\textbf{收敛}的;否则,称之为\textbf{发散}的。
\end{definition}

\subsection{左极限和右极限}

自变量趋于无穷的过程,只有一个方向,要么是不停增大(正无穷),要么是不停减小(负无穷)。但如果自变量是趋近一个实数 $x_0$,那么至少就有两个方向,从大于 $x_0$ 的点开始减小(正向接近),和从小于 $x_0$ 的点开始增大(负向)接近。

由于极限的定义是“怎么接近都可以”,因此若 $f(x)$ 在 $x=x_0$ 处存在极限,无论怎么取接近 $x_0$ 的数列,正向接近也好反向接近也罢,哪怕是一会儿正一会儿负地反复横跳,只要接近,这些数列的极限值都是一样的。

但是有些函数则不然。考虑这个函数:$f(x)$,其中 $x<0$ 时 $f(x)=0$,$x\geq 0$ 时 $f(x)=1$。在 $x=0$ 处,如果\textbf{只考虑}正向接近的数列 $\{x_n\}$,那么计算出来的 $\{f(x_n)\}$ 的极限就是 $1$;但如果\textbf{只考虑}负向接近的数列,那么计算出来的极限是 $0$。按照定义,这意味着 $f(x)$ 在 $x=0$ 处没极限。

但是这种情况,我们说它是有左极限和右极限的。

\begin{definition}{左极限和右极限}
对于函数 $f(x)$,给定实数 $x_0$。

如果取任意\textbf{正}向接近的数列 $\{x_n\}$,所得到的数列 $\{f(x_n)\}$ 的极限都是 $A$,那么称 $A$ 是 $f(x)$ 在 $x=x_0$ 处的\textbf{右极限(right limit)}。

如果取任意\textbf{负}向接近的数列 $\{x_n\}$,所得到的数列 $\{f(x_n)\}$ 的极限都是 $B$,那么称 $B$ 是 $f(x)$ 在 $x=x_0$ 处的\textbf{左极限(left limit)}。
\end{definition}

一个很容易想到的定理是,如果函数在某点的左右极限都存在且相等,那么函数的极限存在且等于左右极限。证明留作思考题,要注意的是,左右极限相等并没有直接说明“左右横跳”式的数列,其极限也等于左右极限。





\subsection{无穷小的阶}\label{sub_Lim_1}
如果令 $x\to 0$, 我们就说 $x$ 是\textbf{无穷小}。 但一些无穷小会更快地趋近于 $0$, 若 $x$ 的某个函数 $\alpha(x)$ 满足
\begin{equation}
\lim_{x\to 0} \frac{\alpha(x)}{x} = 0~,
\end{equation}
那 $\alpha(x)$ 就是 $x$ 的\textbf{高阶无穷小}。 若
\begin{equation}
\lim_{x\to 0} \frac{\alpha(x)}{x^n} \ne 0~,
\end{equation}
则称 $\alpha(x)$ 为 $x$ 的 $n$ 阶无穷小。 例如, $c x^n$ ($c$ 为常数)就是 $x$ 的 \textbf{$n$ 阶无穷小}, 记为 $\order{x^n}$。

在求极限时, 若高阶无穷小与低阶无穷小相加, 通常可以忽略高阶无穷小。 另外由定义不难推出
\begin{equation}
\order{x^n} x^m = \order{x^{n + m}} \qquad (m > -n)~.
\end{equation}

在物理中, 当我们用一个函数 $g(x)$ 来近似另一个函数 $f(x)$ 并记为 $f(x) = g(x) + \order{h^n}$ 时(这里 $x$ 是函数的自变量, $h$ 是函数表达式中一个较小的常数), 就说 $g(x)$ 的误差为 $\order{h^n}$。
