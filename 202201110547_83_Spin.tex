% 自旋角动量
% keys 自旋|角动量|对易关系

\begin{issues}
\issueDraft
\end{issues}

\pentry{轨道角动量\upref{QOrbAM}, 张量积空间\upref{DirPro}}

刚体在经典力学中有两种角动量:1.轨道角动量$(\bvec L = \bvec r\cross\bvec p)$,也就是物体的质心围绕原点运动(公转)所产生的.2.自转角动量$(\bvec S = \bvec I\bvec\omega)$,也就是刚体绕其质心(自转)运动所产生的.量子力学中,除了像氢原子这样由电子围绕原子核运动所产生的由球谐函数所描写的电子轨道角动量之外,电子还有着另一种自旋角动量.不过这一自旋角动量和经典力学中的有着本质的区别,因为这一角动量是和空间无关的,也就是不能被坐标$r,\theta,\phi$所描述,因此我们也称其为基本粒子所固有的内禀角动量$(S)$.
\subsection{斯特恩–格拉赫实验}

在\textbf{斯特恩–格拉赫(Stern-Gerlach)}实验中,一束银原子朝一个方向(我们设这个方向为 $\bvec y$ 轴方向)发射并穿过不均匀的磁场.除了有力矩(类比经典电磁理论中的\upref{EBLoop}),还有另外一个力作用在磁偶极子上(类比经典电磁理论中的\upref{EBTorq}):
\begin{equation}\label{Spin_eq1}
\bvec F=\nabla(\bvec \mu\cdot \bvec B)
\end{equation}
这个力可以用来分离具有特定自旋指向的粒子.假设一束较重的中性原子(例如该实验中的银原子)沿 $z$ 轴方向通过一个非均匀磁场区域——比如说
\begin{equation}\label{Spin_eq2}
\bvec B(x,y,z)=-\alpha x \hat{\bvec x}+(B_0+\alpha z)\hat {\bvec z}
\end{equation}
其中 $B_0$ 是一个较强的均匀场,而 $\alpha$ 描述对均匀性的一个小的偏离(实际上我们只需要 $z$ 方向上一个小的偏离,但不幸的是这违背了 $\nabla\cdot \bvec B=0$,所以必须有 $x$ 分量出现).作用在原子上的力为
\begin{equation}
\bvec F=\gamma \alpha (-S_x \hat{\bvec x}+S_z\hat{\bvec z})
\end{equation}
这里的 $S_x$ 表示原子自旋角动量 $x$ 分量的期待值,$S_z$ 表示自旋角动量 $z$ 分量的期待值.由于电子的磁矩和自旋角动量成正比,$\gamma$ 表示这一比例系数(称为磁旋比\upref{BohMag}.不同于经典理论的 $q/2m$,这里的 $\gamma$ 实际上是 $q/m$\upref{BohMag}).

但由于绕 $\bvec B_0$ 的拉莫尔(Larmor)进动,$S_x$ 快速振荡,且平均值为 $0$.所以净力是沿 $z$ 轴方向的.
\begin{equation}
F_z=\gamma\alpha S_z
\end{equation}

原子束穿过磁场受到偏转后打在检测器屏幕上.经典上,我们预期的结果是一条模糊带($S_z$ 没有量子化).但实验结果却是几个分立的模糊的点(在银原子实验中,屏幕上出现两个点).原子束分成了 $2s+1$ 个分立的束,在 Stern-Gerlach 实验中 $s=1/2$.这是因为在银原子中,原子内层的所有电子都是配对的,它们的轨道和自旋角动量都相互抵消.所以净自旋就是最外层一个非配对电子的自旋:$1/2$.

\addTODO{连续多个Stern-Gerlach装置的顺序实验}

\subsection{自旋角动量算符与泡利矩阵}

自旋是量子力学中的基本粒子特有的性质, 描述粒子的波函数不包含自旋的信息, 自旋处于单独的有限维希尔伯特空间中, 和波函数的空间做张量积以后用于描述粒子的状态.

\begin{enumerate}
\item 自旋角动量三个分量算符 $S_x, S_y, S_z$ 的互相对易关系以及自旋模长平方算符 $S^2$ 的对易关系 %(已经不想写了)
\item 与轨道角动量同理,存在一组本征态 $\ket{s,m}$ 

( $s = 0, 1/2, 1, 3/2\dots$, $m = -s, -s+1\dots ,s-1, s$ 但是每种粒子都有固有的 $s$ ) 满足
\begin{equation}
S^2\ket{s, m} = \hbar^2 s(s+1)\ket{s, m}  \quad \text{和} \quad
S_z\ket{s, m} = \hbar m\ket{s, m}
\end{equation}

\item 存在升降算符 $S_\pm = S_x \pm \I S_y$, 且(根号项是归一化系数)
\begin{equation}
S_\pm \ket{s,m} = \hbar \sqrt{s(s + 1) - m(m \pm 1)} \ket{s, m+1} 
\end{equation}

\item 对于 $s = 1/2$ 的粒子(这也是最为重要的情况,因为它是构成普通物质的粒子(质子、中子和电子)的自旋,以及
所有夸克和所有轻子的自旋.),$S^2$和$S_z$一共有 2 个本征态, 分别是 上自旋态:$\ket{1/2, 1/2}$和下自旋态:$\ket{1/2, -1/2}$. 它们的角动量模长平方都是 $3\hbar^2/4$, 角动量 $z$ 分量都是 $\hbar/2$. 利用这两个基矢量,一个自旋为$1/2$粒子的一般态$\chi$可以表示为一个
两元列矩阵或旋量:
\begin{equation}\label{Spin_eq4}
\chi = \pmat{a\\b}=a\chi_++b\chi_-
\end{equation}
以这两个本征态为基底,令第一个代表上自旋为 $\chi_+ =(1, 0)\Tr$, 第二个代表下自旋为 $\chi_- = (0, 1)\Tr$. 可以得出角动量平方算符的矩阵为
\begin{equation}
\mat S^2 = \frac{3\hbar^2}{4} \pmat{1&0\\0&1} \qquad
\mat S_z = \frac{\hbar}{2} \pmat{1&0\\0&-1}
\end{equation}
根据 $S_+ \chi_- = \hbar \chi_+$ 和 $S_- \chi_+ = \hbar \chi_-$,   得到
\begin{equation}
S_x = \frac{\hbar}{2}\pmat{0&1\\1& 0} \qquad
S_y = \frac{\hbar}{2}\pmat{0&-\I\\ \I& 0}
\end{equation}
然后, 定义泡利矩阵.  
\begin{equation}
\sigma_x = \pmat{0&1\\1& 0} \qquad
\sigma_y = \pmat{0&-\I\\ \I& 0} \qquad
\sigma_z = \pmat{1&0\\ 0&-1}
\end{equation}
其实, 根据对易关系直接就可以得到泡利矩阵.

注意到$S_x,S_y,S_z$都是厄密矩阵,当然它们也都表示可观测量.另外,$S_+$和$S_-$不是厄密的,它们显然也不是可观测量.

如果我们要对一个粒子的态$\chi$\autoref{Spin_eq4} 测量它的$S_z$,那么我们得到$+\hbar/2$的概率为$|a|^2$,得到$-\hbar/2$的概率为$|b|^2$,也就是旋量必须要归一化:$|a|^2+ |b|^2 = 1$.

\begin{equation}
\rm{det}\pmat{-\lambda \quad \hbar/2\\\hbar/2\quad -\lambda}=0\Rightarrow\lambda=\pm\frac{\hbar}{2}
\end{equation}
% 未完成
\end{enumerate}
\begin{exercise}{}
假设电子为一个经典的刚性球体,其经典电子半径$r_0$为:
\begin{equation}\label{Spin_eq3}
r_c=\frac{e^2}{8\pi\sigma_0m_e c^2}
\end{equation}
其中$e$为单位电荷,$\sigma_0$为真空电容率,$m_e$为电子静止质量,$c$为光速.试着结合爱因斯坦质能公式$E=mc^2$并且假设电子质量可归因于其电场能量推导出\autoref{Spin_eq3} .电子的角动量为$1/2\hbar$,求它“赤道”上的一个点的运动速度.这个模型有什么意义?
\end{exercise}