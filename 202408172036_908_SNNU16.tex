% 陕西师范大学 2016 年 考研 量子力学
% license Usr
% type Note

\textbf{声明}:“该内容来源于网络公开资料,不保证真实性,如有侵权请联系管理员”

\subsection{填空题}
\begin{enumerate}
    \item 写出各自力场与力自己量子数。
    \item 德布罗意波公式 $\underline{\hspace{2cm}}$,当$m$的粒子,有能量 $E$,求 $\lambda\underline{\hspace{2cm}}$,当在 $U = 150V$ 加速后,$\lambda = \underline{\hspace{2cm}}$。
    \item 在一维势阱中,基态能 $E_1$,激发态 $E_n = \underline{\hspace{2cm}}$。抛物势阱,基态 $E_1$,$E_n = \underline{\hspace{2cm}}$。
    \item 在希尔伯特空间中,属于不同本征值的两个本征函数 $\underline{\hspace{4cm}}$,这些态构成$\underline{\hspace{4cm}}$。
    \item 算符由一个表象转换到另一个表象时,叫$\underline{\hspace{2cm}}$ 变换,该变换$\underline{\hspace{2cm}}$ 不变。
    \item $\hat x,\hat p_x$的对易兴系 $\underline{\hspace{4cm}}$,不确度关系 $\underline{\hspace{4cm}}$。
    \item 斯莱特行列式的系教 $\underline{\hspace{4cm}}$。
    \item $\hat P$ 的空间表述$\underline{\hspace{2cm}}$。
\end{enumerate}
\subsection{简答题}
 $(A+B)(A-B) = A^2 - B^2$, 证 $\{A, B\} = 0$ 为条件。
\subsection{简答题}
三个金刚体包含中:肽D有几何构态?构造?
\subsection{简答题}
一维振子$\overline{x},\overline{p},\overline{T},\overline{v}$测不准关系。
\subsection{简答题}
$t=0$时刻,氢原子的态为:
\[
\Psi(0) = \frac{1}{\sqrt{3}} \psi_{100} + \frac{1}{\sqrt{10}} \psi_{210} + \frac{1}{\sqrt{21}} \psi_{211} + \frac{1}{\sqrt{21}} \psi_{21-1}~
\]

其中态 $\psi_{nlm}$ 中的 $m$ 为主量子数,$l$ 为角量子数,$m$ 为磁量子数。

\begin{enumerate}
    \item $t=0$时刻,氢原子的能量期望值(基本能为 $E_1$)。
    \item 在$\pm$时刻,氢原子处于 $l=1, m=\pm$态 的几率。
\end{enumerate}