% 华东师范大学 2013 年硕士入学考试物理试题
% keys 华东师范大学|考研|物理|2013年
% license Copy
% type Tutor

\textbf{声明}:“该内容来源于网络公开资料,不保证真实性,如有侵权请联系管理员”

\textbf{考生注意:}
可以使用计算器
\begin{enumerate}
\item 一质点沿x轴作直线运动,其$v(t)$曲线如图所示,$t=0$时,质点位于坐标原点,则$t=4.5s$时,质点在$x$轴上的位置为:\\
(A)5m\\
(B)2m\\
(C)0\\
(D)-2m\\
(E)-5m
\begin{figure}[ht]
\centering
\includegraphics[width=6cm]{./figures/ba93a1544a8f8e30.png}
\caption{} \label{fig_HDSD13_1}
\end{figure}
\item 质点作曲线运动,F表示位置矢量,$\bar v$表示速度,$\bar a$表示加速度,S表示路程,$\alpha$表示切向加速度,下列表达式中\\
(1)$dv/dt=\alpha \qquad$
(2)$dr/dt=v \qquad$
(3)$dS/dt=v\qquad$
(4)$\abs{dv/dt}=\alpha_t$\\
(A)只有(1)是对的\\
(B)只有(2)是对的\\
(C)只有(1)(2)是对的\\
(D)只有(3)是对的\\
(E)只有(2)(3)是对的\\
(F)只有(4)是对的
\item 下列关于牛顿定律的说法哪些是不正确的?\\
(A)牛顿定律只在惯性系中成立,在非惯性系中不成立。\\
(B)牛顿第一定律是牛顿第二定律在$F=0$的情况下的特例。
(C)在非惯性系中使用牛顿第二定律时,需考虑惯性力。
(D)牛顿第一定律指出惯性是物质本身的一种属性。
\item 下列说法中,正确的是:\\
(A)物体沿竖直面上光滑的圆弧形轨道下滑的过程中,轨道对物体的支持力不断增加。\\
(B)从点作圆周运动时,所受的合力一定指向圆心。\\
(C)用水平压力把一个物体压着靠在粗糙的竖直墙面上保持静止,当$F$逐渐增大时,物体所受的静摩擦力$f$也随$F$成正比地增大。\\
(D)一辆汽车从静止出发,在平直公路上加速前进的过程中,如果发动机的功率一定,力大小不变,则汽车的加速度不变。

(A)
(B)
(C)
(D)
\end{enumerate}