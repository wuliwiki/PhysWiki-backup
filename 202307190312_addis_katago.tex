% KataGo 笔记

\begin{issues}
\issueDraft
\end{issues}

\subsubsection{安装 katago}
\begin{itemize}
\item \href{https://github.com/lightvector/KataGo/releases}{下载 katago 主程序},有些 release 里面只有源码,有些有 cuda 版本(如果你有 nvidia gpu 就用这个),有些有 opencl 版本(一般笔记本)。
\item 下载完直接解压即可,无需安装。
\item 还需要下载一个\href{https://katagotraining.org/networks/}{网络文件}。放到同一个目录,无需解压,重命名为 \verb|default_model.bin.gz|
\item 现在 katago 就装好了,但它只是一个命令行程序,可以用 cmd 运行一下 test 能不能正常工作。
\end{itemize}

\subsubsection{安装 sabaki}
\begin{itemize}
\item 下载安装 \href{https://sabaki.yichuanshen.de/}{sabaki}
\item 在 sabaki 的 \verb|File -> Preferences -> Engines| 里面添加 katago,名字 \verb|katago| 或者随便取, 路径例如 \verb|C:\Users\用户\Documents\katago\katago.exe|, 参数填 \verb|gtp|, 初始命令填 \verb|time_settings 0 5 1|
\item 菜单 \verb|Engines -> Show Engine Sidebar -> Attach Engine|。 如果 connection failed, 多半是路径的问题。
\end{itemize}

\subsection{分析}
\begin{itemize}
\item 新建或者打开一个 sgf 文件后,打开 \verb|Engine -> Engine Sidebar|, 按 \verb|attach|。 好了以后, 按 \verb|Engne -> Toggle analysis (F4)| 可以开始分析, \verb|View -> Show Heatmap -> Show Score Lead| 可以开启热图。
\end{itemize}
