% Mathematica 文件操作
% keys Mathematica|IO|文件操作

\begin{issues}
\issueDraft
\issueTODO
\end{issues}

\subsection{常用环境变量}

除了丰富的标准文件操作外,Wolfram 语言的统一符号(\verb`Symbol`)架构,
让我们更容易将算法和高级编程应用于许多文件和系统管理任务.此介绍主要参考官方页面:

\begin{itemize}
\item guide/FileOperations:较齐全的文件系统操作的函数列表
\item tutorial/FilesStreamsAndExternalOperations\#12068
\end{itemize}

一个重要的原则是:不要直接使用\verb`裸字符串`硬编码\verb`文件名/文件路径`, 
这样生成的路径依赖于操作系统细节,应该使用 Mathematica 提供的文件系统接口,对路径进行拼接或分割.
同理,工程的根目录应该在运行时计算而不应该硬编码(见末尾举例).
或者,把脚本放在 MMA 的搜索路径里,如 \verb`$UserBaseDirectory/Applications` 目录下.
如需获得操作系统细节,例如可使用:

\begin{itemize}
\item \verb`$OperatingSystem` : 给出正在运行的操作系统的名称.
\item \verb`$PathnameSeparator` : 存储路径分隔符的字符串,可在构建路径时使用.
例如 \verb|$UserBaseDirectory <> "\abcd\"|. 
Windows 中的默认值是 \verb|\\|, 其他系统是 \verb|/|.
在 Windows 中,\verb`FileNameSplit` 这类函数默认同时允许 \verb|\| 和 \verb|/|.
\end{itemize}

\subsubsection{文件后缀名使用惯例}

\begin{itemize}
\item \verb`.m`  : Wolfram 语言源文件
\item \verb`.nb` : Wolfram 系统笔记本文件
\item \verb`.ma` : Wolfram 系统从第 $3$ 版以前的笔记本文件
\item \verb`.mx` : 输出所有 Wolfram 语言表达式
\item \verb`.exe`: WSTP 可执行程序
\item \verb`.tm` : WSTP 模版文件
\item \verb`.ml` : WSTP 流文件
\end{itemize}

\verb`Get`,\verb`Needs`,\verb`Import`,\verb`Install` 等函数读取本地文件时,默认使用的搜索路径为 \verb`$Path`.
全局变量 \verb`$Path` 被定义为字符串的列表, 其中每个字符串代表一个目录.
每次你要求打开文件时, Wolfram 就依次将这些目录暂时设置为当前工作目录,然后从该目录中尝试寻找你要求的文件.
也就是说,如果两个目录中含有同名的文件,排在前面的目录优先,不过你可以通过提供更详细的上层路径来避免歧义.
在 \verb`$Path` 的典型设置中, 当前目录 \verb`.` 和你的主目录 \verb`~` 被列在第一位.

\subsubsection{预定义的环境变量}

\begin{itemize}
\item \verb`$InitialDirectory` : Wolfram 系统启动时的初始目录.
\item \verb`$HomeDirectory` :  你的主目录, 如果被定义过的话
\item \verb`$BaseDirectory` :  Wolfram 系统要加载的全系统文件的基本目录.
\item \verb`$UserBaseDirectory` :  用于 Wolfram 系统加载的, 用户自定义文件的基本目录
\item \verb`$InstallationDirectory` :  你的 Wolfram 系统安装的最高级别目录
\end{itemize}

Wolfram 系统所使用的绝大多数文件都与操作系统无关. 然而, \verb`mx` 和 \verb`.exe` 文件与系统有关.
对于这些文件, 按照惯例, 捆绑上不同计算机系统版本的名称, 形式如 \verb`name/$SystemID/name`.

\subsection{笔记本界面中的接口}

\begin{itemize}
\item \verb`NotebookFileName[]` : 给出当前笔记本的完整路径.
\item \verb`NotebookDirectory[]`: 笔记本父目录
\end{itemize}

\subsubsection{打开笔记本}

\begin{itemize}
\item \verb`NotebookOpen["name"]`:  打开已经存在的笔记本 \verb`"name"`,返回笔记本对象. \verb`"name"` 可以是绝对路径.
\item \verb`NotebookOpen["name", options]`: 使用指定的选项打开笔记本.
\item 若给出相对路径, \verb`NotebookOpen` 则搜索由前端全局选项 \verb`NotebookPath` 指定的目录.
\item 若设置选项 \verb`Visible->False`,\verb`NotebookOpen` 打开的笔记本将带有此选项,它永远不会显示在屏幕上.
\end{itemize}

\subsubsection{保存和关闭笔记本}

\begin{itemize}
\item \verb`NotebookSave[notebook]`: 保存特定笔记本的当前版本.\verb`notebook` 必须是一个 \verb`NotebookObject`.
\item \verb`NotebookSave[notebook, "file"]`, 如果\verb`"file"`存在, 则不加警告地覆盖它.
\item \verb`NotebookClose[notebook]`: 关闭指定的笔记本对象.
\item \verb`NotebookClose[]`:关闭当前在运行的笔记本.
\end{itemize}

\subsection{操作文件和目录}

\begin{itemize}
\item tutorial/FilesStreamsAndExternalOperations\#12068
\item Manipulating Files and Directories
\end{itemize}

\subsubsection{设置工作目录}

\begin{itemize}
\item \verb`SetDirectory["dir"]`:将当前工作目录设置为 \verb`dir`.
\verb`SetDirectory[]` 等同于 \verb`SetDirectory[$HomeDirectory]`.
\item \verb`ResetDirectory[]`; 将当前工作目录重置为之前的值.
\item \verb`DirectoryStack[]`; 给出当前使用的目录序列/目录栈.其中的目录用绝对路径给出.
每次调用\verb`SetDirectory`会在目录栈中压入元素;每次调用\verb`ResetDirectory`会弹出元素.

\end{itemize}

\subsubsection{常用目录操作}

\begin{itemize}
\item \verb`DirectoryQ`;测试名称是否对应于真实的目录.
\item \verb`DirectoryName["name",n]` : 给出路径的父目录, \verb`n` 代表上升 \verb`n` 次. 
默认情形给出父目录, 可以省略 \verb`n`. 可作用于文件和目录, 但不检查目录是否真实存在于硬盘.
\item \verb`DirectoryName[..., OperatingSystem->"os"]` 用来给出某种操作系统风格的路径, 
选项有 \verb`"Windows"`, \verb`"MacOSX"`, 和 \verb`"Unix"`.
\item \verb`ParentDirectory["dir",n]` :给出路径的父目录, \verb`n` 代表上升 \verb`n` 次, 
只能作用于目录, 并且要求目录真实存在.
\end{itemize}

\subsubsection{查找文件}

\begin{itemize}
\item \verb`FileNames[]`:列出当前目录中的所有文件.
\item \verb`FindFile[name]`:找到指定名称的文件,\verb`Get[name]` 和相关函数使用此函数寻找文件.
\end{itemize}

\subsubsection{文件名与拓展名}

\begin{itemize}
\item \verb`FileNameTake["name"]` : 从 \verb`"name"` 的完整路径中提取出最后的文件名.
\item \verb`FileBaseName["file"]` : 给出文件的 \verb`basename`,也就是不包括“拓展名.
\item \verb`FileExtension["file"]`: 给出文件的“拓展名.
\item \verb`FileNameDepth["name"]`: 给出文件路径的“深度, 文件不必真实存在.
\end{itemize}

\subsubsection{计算绝对路径}

\begin{itemize}
\item \verb`ExpandFileName["name"]`:将 \verb`"name"` 展开为当前系统规范下的绝对路径, 
\verb`"name"` 的解析相对于你当前的目录.它展开通常的目录指定, 如 \verb`.` 和 \verb`..`;
它只是对文件名进行解析,并不实际搜索指定的文件.
\item \verb`AbsoluteFileName["name"]`: 给出 \verb`"name"` 文件的绝对路径. 
与 \verb`ExpandFileName` 的区别是, 它会进入文件系统, 检查文件是否真实存在,
其他和 \verb`ExpandFileName` 类似.
\end{itemize}

\subsubsection{组合路径}

\begin{itemize}
\item \verb`FileNameJoin` : 从“路径”的列表,组合出完整的文件名
\item \verb`FileNameSplit` : 将文件的完整路径分割成列表,相当于逆运算.
\item \verb`FileNameDrop["name",n]` : 去掉文件 \verb`"name"` 路径的前 \verb`n` 个片段. 
如果传入\verb`-n`, 那么去掉从末尾开始的 \verb`n` 个片段.
\item \verb`FileExistsQ["name"]`  : 检查文件, 目录等是否存在.
\item \verb`ContextToFileName["context"]`  : 给出 Mathematica 上下文规范(\verb`Context`)对应的文件名.
\end{itemize}

\subsection{应用举例:计算包的本地根目录}

假设你有一些 MMA 脚本组成的工程,它们的根目录为\verb`root`.
也许你需要在脚本内部得到工程的根目录,以便于将脚本文件移动其他位置,甚至其他操作系统上,并且不影响代码的正常运行.

一个简单的实现是:首先我们建立\verb`锚点`:
在 \verb`root` 中新建 \verb`init.wl` 文件(或者别的名字,但后面要相应更改).并写入以下代码:
\begin{lstlisting}[language=mathematica]
(*定义程序包的根目录*)
$srcRoot=AbsoluteFileName[DirectoryName[
If[$Notebooks,NotebookFileName[],$InputFileName],1]]
\end{lstlisting}
如此,\verb`$srcRoot` 变量将保存项目的根目录位置.
在其他项目文件(\verb`.nb`, \verb`.m`, \verb`.wl`……)中,添加以下代码:
\begin{lstlisting}[language=mathematica]
(*本文件的名称*)
$fileName=If[$Notebooks,NotebookFileName[],$InputFileName];
(*如果在前端执行,就刷新笔记本的标题*)
Once@If[$Notebooks,NotebookWrite[Cells[][[1]],
Cell[Last@FileNameSplit[$fileName],"Title"]]];
(*查找 init.wl, 导入根目录和函数定义*)
Once@Catch@Module[{recurFind,start=1,depMax},
depMax=FileNameDepth[$fileName];(*路径的最大层次*)
(*-------定义递归函数-------*)
recurFind[dep_Integer]:=If[dep<=depMax,
SetDirectory[DirectoryName[$fileName,dep]];
(*如果在当前层能找到 init.wl,就运行它,并把根目录添加到搜索路径*)
If[FileExistsQ["init.wl"],
Get["init.wl"];PrependTo[$Path,$srcRoot];
Throw["The base directory is : "<>$srcRoot];,
(*如果这一层找不到,就上升一层*)
recurFind[dep+1]];
ResetDirectory[];(*重设为之前的目录*),
Throw["I cann't find any init.wl in this project"]];
recurFind[start];
]
(* 记录 master Kernel 的运行模式, 可在并行计算中使用 *)
$inNBook=$Notebooks;echo[DateString[]," <<",$fileName];
\end{lstlisting}
运行之后,此脚本文件名保存在 \verb`$fileName` 中,工程跟目录保存在 \verb`$srcRoot` 中,
由于根目录已添加到 \verb`$Path` 变量中,可以简单地使用文件名调用工程中的其他脚本:
\begin{lstlisting}[language=mathematica]
Get["其他脚本名称.wl"]
\end{lstlisting}
