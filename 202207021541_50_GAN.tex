% 生成对抗网络
% 生成 对抗 无监督 深度学习

\textbf{生成对抗网络}(Generative Adversarial Network, GAN)是基于神经网络结构的生成模型,是深度学习中的一种主流方法.该模型在各种问题场景,比如数据生成、艺术造作、图像修复、图像风格转换、语音合成、文本图像互相转换等中均有十分广泛的应用.

生成对抗网络模型主要包含两个网络结构:一是捕获数据分布的生成模型,也称为生成器(通常用G表示),二是估计来自训练数据(而不是G)样本的概率的判别模型,也称为判别器(通常用D表示).生成器G的训练步骤是要最大化判别器D做出错误判断的概率.这个框架对应于最小最大化两人博弈.在生成器G和判别器D的任意函数的解空间中,存在一个独一无二的解,生成器还原训练数据分布,判别器处处等于1/2.

生成器和判别器玩的双人最小最大游戏,其价值函数为:
%\begin{equation}
min \quad maxV(D,G)=E_{x ~ p_{data}(x)}[logD(x)]+E_{z~p_z(z)}[log(1-D(G(z)))]
%\end{equation}

\begin{equation}
% \[\mathop {\min }\limits_G \mathop {\max }\limits_D V(D,G) = {E_{x\~{p_{data}}(x)}}[\log (D(x)] + {E_{z\~{p_z}(z)}}[\log (1 - D(G(z)))]\]
\end{equation}


训练生成对抗网络时,通常会将生成器和判别器连接成一个网络.判别器参数保持不变,更新生成器参数;生成器参数保持不变,更新判别器参数.生成器和判别器连续不断地交替更新,直至训练结束.
\begin{figure}[ht]
\centering
\includegraphics[width=14.25cm]{./figures/GAN_1.png}
\caption{网络演化过程} \label{GAN_fig1}
\end{figure}





\textbf{参考文献:}
\begin{enumerate}
\item I. Goodfellow et al., “Generative adversarial nets,” in Advances in neural information processing systems, 2014, pp. 2672–2680.
\end{enumerate}