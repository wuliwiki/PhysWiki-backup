% 相变平衡条件
% 相平衡|单元系|多元系
\pentry{热动平衡判据\upref{equcri},吉布斯自由能\upref{GibbsG}}
%需要增加一个介绍相变的词条%
\subsection{单元系相变平衡条件}
单元系的意思是系统只有一种化学组分,但有多个相.例如冰水混合物是一元二相系,水与水蒸气组成的体系是一元二相系.水的三相图($P$-$V$ 图)中,有最主要的三条线区分开三个区域——固态、液态、气态(见\autoref{PhEquv_fig1},图片来源自维基百科\href{https://en.wikipedia.org/wiki/Triple_point}{相关页面}.).
\begin{figure}[ht]
\centering
\includegraphics[width=14cm]{./figures/PhEquv_1.png}
\caption{水的相图}} \label{PhEquv_fig1}
\end{figure}

两相分界线处是两相共存的平衡态.对于单元复相系(多个相共存),满足力学平衡和热学平衡条件(用 $\alpha,\beta$ 表示两个相):
\begin{equation}\label{PhEquv_eq1}
p^{\alpha}=p^{\beta}, 
T^{\alpha}=T^{\beta}
\end{equation}
同一种化学组分的摩尔吉布斯函数是压强和体积的函数\autoref{GibbsG_eq1}~\upref{GibbsG}.所以有化学平衡条件
\begin{equation}\label{PhEquv_eq2}
\mu^{\alpha}=\mu^\beta
\end{equation}
\autoref{PhEquv_eq1} \autoref{PhEquv_eq2} 为相变平衡条件.

我们可以从热动平衡判据来更精细地分析系统的平衡条件(这帮助我们从另一个视角重新看待为什么单元多相平衡系统要满足力学平衡和热学平衡条件).对于单元多相系中的某两相,用 $U^\alpha,V^\alpha,n^\alpha$ 表示 $\alpha$ 相的内能、体积、物质的量,对 $\beta$ 相也类似地定义 $U^\beta,V^\beta,n^\beta$.用熵判据(热动平衡判据\upref{equcri})进行计算:
\begin{equation}
\begin{aligned}
&\delta S^\alpha=\frac{\delta U^\alpha}{T^\alpha}+\frac{P^\alpha\delta V^\alpha}{T^{\alpha}}\\
&\delta S=\delta U^\alpha\left(\frac{1}{T^\alpha}-\frac{1}{T^\beta}\right)+\delta V^\alpha\left(\frac{P^\alpha}{T^\alpha}-\frac{P^\beta}{T^\beta}\right)
-\delta n^\alpha\left(\frac{\mu^\alpha}{T_\alpha}-\frac{\mu^\beta}{T^\beta}\right)
\end{aligned}
\end{equation}

由此我们可以得到三个平衡条件

\begin{equation}
T^\alpha=T^\beta,\ \ \ P^\alpha=P^\beta,\ \ \ \mu^\alpha=\mu^\beta
\end{equation}

其中第三个叫相变平衡条件,两相的\textbf{化学势}\upref{GibbsG}相等.例如,若一个冰水混合物系统处于相平衡,那么在这种温度压强情况下下冰的摩尔化学势一定等于水的摩尔化学势.利用这个原理,可以推出 $p-T$ 图上两相平衡曲线的斜率,得到\textbf{克拉伯龙方程}\upref{Clapey}.

化学势可以联系到态函数:吉布斯函数\upref{GibbsG}.吉布斯函数关于 $p,T$ 以及其他热力学参量的变化关系可以从其全微分公式中看出.通过相平衡条件就可以推导相平衡时各热力学参量之间的关系,这正是推导克拉伯龙方程(饱和蒸气压方程)和爱伦费斯特方程时采用的方法.详细的推导见词条\upref{Clapey}.这里我们举一个新的例子.

\subsection{多元系相变平衡条件}