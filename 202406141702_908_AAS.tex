% 原子吸收光谱
% license CCBYSA3
% type Wiki

(本文根据 CC-BY-SA 协议转载自原搜狗科学百科对英文维基百科的翻译)


\begin{figure}[ht]
\centering
\includegraphics[width=6cm]{./figures/acd93a5153633fe4.png}
\caption{火焰原子吸收光谱仪} \label{fig_AAS_1}
\end{figure}

原子吸收光谱(AAS)和原子发射光谱(AES)是一种利用自由原子对气态光辐射(光)的吸收,定量测定化学元素的光谱分析方法。原子吸收光谱是以自由金属离子对光的吸收为基础的。

在分析化学中,该技术用于确定待分析样品中特定元素(被分析物)的浓度。原子吸收光谱法可用于测定溶液中70多种不同的元素,也可以通过电热蒸发直接测定固体样品中的元素,用于药理学、生物物理学,考古学和毒理学研究。

原子发射光谱学最初被用作分析技术,其基本原理是由德国海德堡大学的教授Robert Wilhelm Bunsen和Gustav Robert Kirchhoff在19世纪下半叶确立的。[1]

原子吸收光谱的现代形式主要是在20世纪50年代由一组澳大利亚化学家发展起来的。他们由澳大利亚墨尔本联邦科学与工业研究组织(CSIRO)化学物理部门的的Alan Walsh爵士领导。[2][3]

原子吸收光谱法在化学的不同领域有许多用途,如临床分析生物液体和组织中的金属,如全血、血浆、尿液、唾液、脑组织、肝脏、头发、肌肉组织、精液,在一些制药过程中,在最终药物产品中残留的微量催化剂,以及分析水中的金属含量。

\subsection{原则}

原子吸收光谱技术利用样品的原子吸收光谱来评估样品中特定分析物的浓度。它需要具有已知分析物含量的标准来建立测量吸光度和分析物浓度之间的关系,因此原子吸收光谱依赖于Beer-Lambert定律。

\subsection{原子吸收光谱仪器}

\begin{figure}[ht]
\centering
\includegraphics[width=6cm]{./figures/95d81ebe3871db4f.png}
\caption{原子吸收光谱仪结构框图} \label{fig_AAS_2}
\end{figure}

为了分析样品的原子成分,必须对样品进行雾化。现在最常用的雾化器是火焰和电热(石墨管)雾化器。用光辐射照射原子,辐射源可以是元素特异性线辐射源,也可以是连续辐射源。然后,辐射通过单色仪,单色仪将特定元素的辐射与辐射源发射的所有其他辐射分离开来,最终由检测器进行测量。

\subsubsection{2.1 雾化器}

原子吸收光谱法中最古老和最常用的雾化器是火焰,主要是温度约为2300℃的空气-乙炔火焰以及温度约为2700℃的一氧化二氮[3] -乙炔火焰。另外,后一种火焰提供的雾化环境还原性更强,非常适合对氧具有高亲和力的分析物。

\begin{figure}[ht]
\centering
\includegraphics[width=6cm]{./figures/7c3bdbef37094987.png}
\caption{使用丙烷操作火焰雾化器的实验室火焰光度计} \label{fig_AAS_3}
\end{figure}

液体或溶液样品通常与火焰雾化器一起使用。样品溶液由气动分析喷雾器吸入,转化为气溶胶,并被引入喷雾室,只有最细的气溶胶液滴(< 10 微米)进入火焰,在喷雾室中与火焰气体混合。这种调节过程导致只有约5%的吸入样品溶液到达火焰,但它也保证了相对较高的干扰自由度。

喷雾室的顶部是一个燃烧器头,它产生横向较长的火焰(通常为5-10 cm)并且宽度只有几毫米。辐射光束穿过火焰的最长轴,可以通过调节火焰气体的流速来产生最高浓度的自由原子。还可以调整燃烧器高度,使辐射光束通过火焰中原子云密度最高的区域,产生最高的灵敏度。

在火焰中发生的过程包括去溶剂化(干燥)阶段,在该阶段溶剂蒸发,干燥的样品纳米颗粒保留,蒸发(转移到气相)阶段,在该阶段固体颗粒转化成气体分子,雾化阶段,气体分子解离成自由原子,然后电离(取决于分析物原子的电离电势和特定火焰中可用的能量),其中原子可以部分转化为气态离子。

如果校准标准和样品中的分析物的相转移程度不同,则这些阶段中的每一个都包括干扰风险。这种情况下我们通常不希望发生电离,因为它减少了可用于测量的原子数量,即灵敏度。

在火焰原子吸收光谱法中,样品被吸出时会产生稳态信号。该技术通常用于$mg L^−1$ 范围内的测定,对于某些元素来说可扩展至μg L−1 。
