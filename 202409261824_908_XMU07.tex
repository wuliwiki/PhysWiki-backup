% 厦门大学 2007 年 考研 量子力学
% license Usr
% type Note

\textbf{声明}:“该内容来源于网络公开资料,不保证真实性,如有侵权请联系管理员”

\subsection{一、(25 分)简述题(每小题5分)}
\begin{enumerate}
        \item 什么是叠加原理?
        \item 什么是基态?写出二维各向同性谐振子基态的能级表达式。
        \item 什么是厄米(Hermite)算符?动量算符是不是厄米算符?
        \item 粒子在中心力场中运动,问:$L_z$ 和 $p_y$ 是守恒量吗?为什么?
        其中 $L_z$ 、$p_y$ 分别为轨道角动量和动量在y方向的分量。
        \item "设力学量算符 $\hat{F}$ 和 $\hat{G}$ 能相互对易,若 $\psi$ 是 $\hat{F}$ 的特征态,则 $\psi$ 也是 $\hat{G}$ 的特征态。" 这句话对吗?试举一例说明。
    \end{enumerate}
\subsection{二、(25 分)}
设质量为m的一维自由粒子初始态为 $\psi(x,t)$ ,证明在足够长时间后
    \[  \psi(x,t) = \left(\frac{m}{i\hbar t}\right)^{1/2} \exp\left(\frac{imx^2}{2\hbar t}\right)\phi\left(\frac{x}{t}\right)   ~\]
    式中 
    \[  \psi(x,0) = \int_{-\infty}^{+\infty} \frac{1}{\sqrt{2\pi}} \tilde{\psi}(k) e^{ikx} dk  ~\]
    是 $\psi(x,t)$ 的傅里叶(Fourier)变换。附:$\lim_{a\to 0} \sqrt{\frac{a}{\pi}} \exp(-iax^2) = \delta(x)$
