% 逻辑门、布尔运算
% license Usr
% type Tutor

\begin{issues}
\issueDraft
\end{issues}

\textbf{德摩根定理(De Morgan's laws)}又称\textbf{对偶律}

\begin{itemize}
\item \textbf{与、或、非(AND, OR, NOT)}是三种最常见的逻辑门
\item 一个\textbf{布尔变量}的值只能是 0 或 1。
\item OR, AND, NOT 运算可以表示为 $A + B$, $A\cdot B$, $\overline A$
\item \textbf{德摩根定律(De Morgan's laws)} $\overline{A+B} = \overline A \cdot \overline B$; $\overline{A \cdot B} = \overline A + \overline B$
\item $A+0=A$,$A\cdot 1=A$ 恒等运算(这就是为什么要用加号和乘号,很多性质都很相似)
\item $A+A=A$,$A\cdot A=A$
\item 交换律 $A+B=B+A$,$A\cdot B=B\cdot A$
\item 分配律 $A\cdot(B+C)=A\cdot B+A\cdot C$, $A+(B\cdot C) = (A+B)\cdot(A+C)$
\item 吸收律 $A+(A\cdot B) = A$, $A\cdot(A+B) = A$
\item 如果两边取 NOT,就可以发现 AND 可以由 OR 和 NOT 构成, OR 可以由 AND 和 NOT 构成。 真正独立的只有两个。
\item 注意 OR 无法用 AND 和 NOT 构成。
\item \textbf{与非门 (NAND,NOT AND)}可以构成任意门(NAND 两个输入相连就是 NOT 门)
\item \textbf{或非门(NOR,NOT OR)}也可以构成任意门。 
\item NAND 闪存的名字就是 NAND 门命名的。
\item CPU 里面所有功能都是逻辑门构成的。 例如做整数加法的电路。
\item 任意个输入变量,经过逻辑运算得到一个输出变量。 如果已知每种输入值对应的输出,那么可以用 NOT 把每种输入值整理成全 1,在用若干 AND 的到 1,并与预期的输出进行与运算。 再把所有结果做 OR 即可。
\end{itemize}
