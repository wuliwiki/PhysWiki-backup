% 动量、动量定理(单个质点)
% 动量|动量定理|质点|牛顿第二定律|合外力

\pentry{牛顿第二定律\upref{New3}}

令质点质量为 $m$,速度为 $\bvec v$,定义其\textbf{动量}为
\begin{equation}
\bvec p = m\bvec v
\end{equation}
注意动量是矢量,与速度(矢量)的方向相同,且取决于坐标系.

现在把动量和速度都看做时间的函数. 等式两边求导,速度对时间的导数等于加速度 $\bvec a$
\begin{equation}
\dv{\bvec p}{t} = m \dv{\bvec v}{t} = m\bvec a
\end{equation}
根据牛顿第二定律,$m\bvec a$ 等于质点所受合外力 $\bvec F$ (注意力和加速度也都是时间的函数),所以
\begin{equation}
\dv{\bvec p}{t} = \bvec F
\end{equation}
这就是\textbf{动量定理},即动量的变化率等于合外力. 在牛顿力学中, 动量定理和牛顿第二定律是完全等效的.

动量定理也可以写成微分形式
\begin{equation}\label{PLaw1_eq4}
\dd{\bvec p} = \bvec F \dd{t}
\end{equation}
也就是在极微小时间内的动量变化等于力乘以这段时间.

现在用定积分\upref{DefInt} 中的微元思想考虑动量从时刻 $t_1$ 到 $t_2$ 的总变化,我们可以把这段时间划分为 $N$ 段微小时间,第 $i$ 段所在的时刻记为 $t_i$,每小段时间内 $\bvec F$ 可认为是恒力 $\bvec F(t_i)$
\begin{equation}
\bvec p(t_2)-\bvec p(t_1) = \sum_{i=1}^{N} \Delta\bvec p_i= \sum_{i=1}^{N} \bvec F(t_i) \Delta t_i
\end{equation}
当 $N\to\infty, \Delta t\to 0$ 时该式可以用定积分(矢量函数)% 未完成
表示\footnote{通常省略以上的推导而直接表达为“\autoref{PLaw1_eq4} 两边定积分得到\autoref{PLaw1_eq6}”}
\begin{equation}\label{PLaw1_eq6}
\bvec p(t_2)-\bvec p(t_1) = \int_{t_1}^{t_2}\bvec F(t) \dd{t}
\end{equation}
这是\textbf{动量定理}的积分形式.特殊地,对于恒力 $\bvec F$,右边的积分等于 $(t_2-t_1)\bvec F$, 上式记为
\begin{equation}\label{PLaw1_eq1}
\Delta \bvec p = \bvec F \Delta t
\end{equation}
\begin{example}{圆周运动中向心力的冲量}
一质量为 $m$ 的质点做半径为 $r$ 速度为 $\bvec{v}$ 的圆周运动,其初始位置如\autoref{PLaw1_fig1} 所示.求它经过四分之一的圆周向心力的冲量.
\begin{figure}[ht]
\centering
\includegraphics[width=4cm]{./figures/PLaw1_1.pdf}
\caption{质量为 $m$ 的质点做半径为 $r$ 速度为 $v$ 的匀速圆周运动} \label{PLaw1_fig1}
\end{figure}
\textbf{解:} 质点初始速度 $\bvec{v_0}$ 竖直向上,经过四分之一圆周后到达圆的顶端,此时速度 $\bvec{v_1}$ 水平向右,如\autoref{PLaw1_fig2} 所示
\begin{figure}[ht]
\centering
\includegraphics[width=2.5cm]{./figures/PLaw1_2.pdf}
\caption{经过四分之一圆周后质点速度示意图} \label{PLaw1_fig2}
\end{figure}
显然,$\Delta\bvec v=\bvec{v_1}-\bvec{v_0}$,其大小为 $\abs{\Delta\bvec v}=\sqrt{2}v$ 由于匀速圆周运动,合外力即是向心力,由动量定理\autoref{PLaw1_eq6}  可知,粒子经过四分之一圆周向心力的冲量 $\bvec I$ 即
\begin{equation}
\bvec I=\Delta\bvec p=m(\bvec{v_1}-\bvec{v_0})
\end{equation}
其大小为 $I=\sqrt{2}mv$,方向如图\autoref{PLaw1_fig2} 所示.

记 $\Delta\bvec v$ 方向为 $\uvec{v}$,则上面结果告诉我们,
\begin{equation}\label{PLaw1_eq3}
\int_{0}^{T/4}\bvec F(t) \dd{t}=\sqrt{2}mv\uvec{v}
\end{equation}
由于匀速圆周运动向心力为
\begin{equation}\label{PLaw1_eq2}
\bvec F=m\frac{v^2}{r}\uvec{r}
\end{equation}
注意 $m\frac{v^2}{r}$ 是常数,\autoref{PLaw1_eq2} 代入\autoref{PLaw1_eq3} ,得
\begin{equation}\label{PLaw1_eq5}
\int_{0}^{T/4}\uvec r \dd{t}=\sqrt{2}\frac{r}{v}\uvec{v}=\frac{\sqrt{2}}{\omega}\uvec{v}
\end{equation}
式中 $\omega={v}/{r}$ 是角速度, 指向 $1/4$ 圆弧的中点.

下面内容为拓展部分(可不看\footnote{非看不懂也不要紧}):
\autoref{PLaw1_eq5} 反映了什么样的物理内容呢?

由质点匀速圆周运动容易知道,在同样的时间内质点径矢方向 $\uvec{r}$ 转过的角度 $\theta$ 是不变量.这表明,若将矢量 $\uvec r(t)$ 头尾相连,则 $\uvec r(0)$ 到 $\uvec r(t)$ 将画出一个正多边形的轮廓,如\autoref{PLaw1_fig3} 
\begin{figure}[ht]
\centering
\includegraphics[width=4cm]{./figures/PLaw1_3.pdf}
\caption{单位径矢 $\uvec r$ 画出的正多边形} \label{PLaw1_fig3}
\end{figure}
而在一个周期内,其将画出一个完整的正多边形(请思考).
那么将此多边形边长乘以 $\dd t$ 倍, 得到的正多边形便是由矢量 $\uvec r(t)\dd t$ 头尾相连画出的正多边形轨迹.而 $\dd t$ 实际上是无穷小量,故这个正多边形事实上成为一个圆,而在一个周期内将画出一个完整的圆.由矢量加法的几何图像可知,那么积分
\begin{equation}\label{PLaw1_eq7}
\int_{0}^{t}\uvec r(t) \dd{t}
\end{equation}
的结果将是从 $\uvec r(0)\dd t$ 起点指向 $\uvec r(t)\dd t$ 的终点的向量.
\begin{figure}[ht]
\centering
\includegraphics[width=5cm]{./figures/PLaw1_4.pdf}
\caption{积分\autoref{PLaw1_eq7} 的几何意义} \label{PLaw1_fig4}
\end{figure}
那么这个矢量微元 $\uvec r(t)\dd t$ 首尾相连构成的圆的半径是多少呢?在四分之一周期内这个圆画了1/4,所以由\autoref{PLaw1_eq5} 可知,该圆半径便是 $\frac{1}{\omega}$.
\end{example}
