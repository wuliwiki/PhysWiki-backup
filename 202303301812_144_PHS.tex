% 相简介(热力学)

\subsection{相}
什么是相?不少教科书定义“相是系统中(宏观)物理与化学性质完全相同的一部分”。这个定义固然严谨,但却不大直观。还有另一种不大严谨的理解是,相是“物质的一种存在形式”。这似乎还是很绕口。

\begin{figure}[ht]
\centering
\includegraphics[width=6cm]{./figures/PHS_1.png}
\caption{水的三相。图源Pixabay} \label{PHS_fig1}
\end{figure}
我们先看一种生活中最常见的物质,水。在常温常压环境下,水是液体;在温度很低时,水就凝固为固体冰;而在温度很高时,水就蒸发为看不见的水蒸气\footnote{火锅上升起的“水蒸气”其实是已经凝固了的小水滴,而不是真正的水蒸气}。

很显然,尽管冰、液态水和水蒸气的成分都是水,但是他们的结构、性质却截然不同。这就引出了“相”的概念:冰、液态水、水蒸气是水的不同存在形式,也就是水的不同相态。由于不同相态下同一物质的性质也大不相同,因此你常常需要写出化学反应中反应物、生成物所处的相态。

\subsection{比较容易理解的相}
\begin{figure}[ht]
\centering
\includegraphics[width=10 cm]{./figures/PHS_2.png}
\caption{物质的固相、液相、气相示意图。翻译自Openstax的Chemistry, CC-BY} \label{PHS_fig2}
\end{figure}
\footnote{\textsl{根据原作者的著名要求,要写上Access for free at openstax.org.}}
不只是水,大部分物质都存在这样的气、液、固三相。例如,"干冰"就是固态的二氧化碳,而“铁水”就是融化的铁。

\subsection{不大容易理解的相}
除了以上说的三态之外,还有其他的相吗?比如说,一种物质可以存在多种不同的固、液相吗?

\begin{figure}[ht]
\centering
\includegraphics[width=8cm]{./figures/PHS_3.png}
\caption{钻石与石墨结构示意图。翻译自Openstax的Chemistry, CC-BY} \label{PHS_fig3}
\end{figure}
答案当然是肯定的,(对于纯物质)这就是高中所说的同素异形。老生常谈的例子是石墨和钻石,他们成分都是C,但是他们的结构完全不同,相应的物理、化学性质也不同。例如,石墨的导电性远远好于钻石。因此,我们说,石墨和钻石是C的两种固相。

\begin{figure}[ht]
\centering
\includegraphics[width=8cm]{./figures/PHS_4.png}
\caption{体心立方 BCC 与 面心立方 FCC 结构示意图} \label{PHS_fig4}
\end{figure}
这绝不是孤例。再比如常见的金属铁,在常温下,铁的稳定晶胞结构是体心立方BCC,通常称为$\alpha-Fe$;而当温度升高至$900 ^\circ C$以上时,铁的稳定晶胞结构将变为面心立方FCC,也称$\gamma-Fe$。$\alpha-Fe$与$\gamma-Fe$分别是铁的两种固相。\footnote{随着温度升高至大约$1400 ^\circ C$以上时,铁的稳定晶胞结构又将变为体心立方BCC,称为$\delta - Fe$。在大约$1530 ^\circ C$以上,铁将熔化。}

\begin{figure}[ht]
\centering
\includegraphics[width=5cm]{./figures/PHS_5.pdf}
\caption{油水分层示意图} \label{PHS_fig5}
\end{figure}
对于混合物,也能举出大量的例子。如果你将油与水倒在同一个杯子中,即使你努力地摇匀,你还是会发现静置后杯中的油与水将逐渐分为两层。这两层就分别是油水混合物的两种液相。

%同理,如果你炒菜时盐加的太多(实在太多!),那么你会发现无论如何搅拌也无法使剩余的盐完全溶解,这里面也有两相:液相(盐的水溶液)和固相(未溶解的盐沉淀)。

\subsection{复相系统与相平衡}
上述的最后例子暗示了一些有趣的东西:在某些条件下,一个系统中的物质可以平衡地以多种不同的形式存在,或者说,一种系统中可以多相共存。比如,在初中我们就知道,零度时冰与水可以平衡共存;此时,系统中就有两个平衡的相。

%这其实有点违反我们的生活直觉:我们观察到水杯里的水总会蒸干,而冰箱里的水总会冻结。似乎系统中物质只能以一种形式存在。

为什么两相可以平衡共存?两相共存的条件是什么?平衡时物质如何在两相中分配?这就是复相热力学将要解决的问题,我们将热力学定律运用至复相系统中并试图解决这一类问题。复相平衡背后的物理学很复杂,但简要地说,支配物质在各相中分配的还是\textsl{熟悉而陌生的熵}与热力学第二定律。

\subsection{相变}
另一方面而言,如果环境条件不足以支持相平衡,那么物质将在相之间转移、直到新的平衡形成,这种转移过程称为相变。例如,将冰箱里的冰拿出到室温环境时,固相中的水将转移至液相,从而发生固液相变。\textsl{说人话,就是冰融化为水了。}
