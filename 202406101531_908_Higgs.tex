% 希格斯粒子(科普)
% license CCBYSA3
% type Wiki

(本文根据 CC-BY-SA 协议转载自原搜狗科学百科对英文维基百科的翻译)

  希格斯玻色子是粒子物理标准模型中的基本粒子,由粒子物理理论中的希格斯场量子激发产生。它是以物理学家彼得·希格斯(Peter Higgs)的名字命名的,Higgs在1964年与其他五位科学家一起提出了这种粒子存在的机制。它的存在于2012年由欧洲核子中心(CERN)的ATLAS实验组和CMS实验组基于大型强子对撞机(LHC)上的对撞实验的联合确认。

2013年12月10日,彼得·希格斯(Peter Higgs)和弗朗索瓦·恩格勒(François Englert)两位物理学家,因为他们的理论预测被授予诺贝尔物理学奖。虽然希格斯的名字已经与这个理论(希格斯机制)联系在一起,但在1960年至1972年间,还有其他的一些研究人员在这个问题的不同方面作出了自己独立的贡献。

主流媒体经常将希格斯玻色子称为“\textbf{上帝粒子}”,源于1993年一本关于这个话题的书[1] ,但是许多物理学家,包括希格斯本人,都认为这个绰号是有些夸大的。

\subsection{介绍}

\subsubsection{1.1标准模型}

标准模型是目前物理学家广泛接受的用来解释基本粒子之间力的性质的理论框架,可以用于理解除了引力(一个独立的理论,广义相对论,用于引力理论)之外的已知宇宙中的几乎所有事物。在这个模型中,自然界的基本力来自于我们宇宙的规范不变性和对称性。力是由被称为规范玻色子的粒子传递的[2][3]。

在标准模型中,希格斯粒子是自旋为零的玻色子,没有电荷,也没有色荷。它也非常不稳定,几乎立即衰变为其它粒子。希格斯场是标量场,有两个中性的和两个带电的分量,它们形成了弱同位旋SU(2)对称性的复二重态。希格斯场的势是“墨西哥帽形”的。这导致场在它的基态任何地方都有非零值(包括其他的“空"空间),结果,在非常高的能量之下,电弱相互作用的弱同位旋对称性的破缺。(技术上非零期望值将拉格朗日量中的汤川耦合项转换为质量项。)当这种情况发生时,希格斯场的三个分量被SU(2)和U(1)规范玻色子(“希格斯机制”)吸收,成为现在有质量的传递弱力W及Z玻色子的纵向分量。其余的电中性分量要么表现为希格斯粒子,要么可以单独与费米子耦合(通过汤川耦合),促使这些粒子也获得质量[4]。