% Julia 第 1 章小结
% license Xiao
% type Tutor

本文授权转载自郝林的 《Julia 编程基础》。 原文链接:\href{https://github.com/hyper0x/JuliaBasics/blob/master/book/ch01.md}{第1章:起步}。

\subsubsection{1.5 小结}

在本章,我们先讲了一下 Julia 语言诞生的初衷。然后,我们对它的一些特性进行了略有侧重的说明。Julia 语言中比较亮眼的特性有:可选的类型标注、多重分派机制、多种并行计算方式、元编程的支持、接近 C 语言的性能,等等。

在简要介绍了 Julia 语言的安装以及\verb|julia|命令的使用之后,我们立即开始了第一个 Julia 程序的编写。在经过一番改进之后,我们的第一个程序就成功实现了一个小功能,即:根据我们执行程序时给定的参数值,打印一句简单的问候。

到这里,作为初识 Julia 语言的读者,我觉得你已经知道的够多了。在下一章,我将会着重介绍 Julia 程序的编程环境。这些也是我们在正式地编写 Julia 程序之前很有必要了解的知识。
