% Minimax 搜索和 Alpha-Beta 剪枝
% keys 极小化极大|minimax|alpha-beta|alpha beta|博弈论
% license Usr
% type Tutor

\subsection{Minimax 搜索(极小化极大搜索)}
在双方博弈类型的搜索中,例如棋类的最优化搜索,如果我们是 AI,我们会希望,走下这一步棋后,我们的最大的失败可能最小。为什么这里是“最大的失败可能”而失败可能不是一个定值呢?因为对方的可能下法不止一种,对于每一种对方的下法,我们也都有一个“失败的可能性”(实际上这里是一个最小值,我们选择下最优的下法让我们失败的可能性最小),我们需要使用这些失败的可能性的总体的最大值来评估这步棋的好坏。

所以可以发现,这是一个使得某个最大值(失败的可能性)最小的算法,所以叫做\textbf{极小化极大搜索(Minimax 搜索)}。

那么实际上搜索进行的是,”你一步我一步“的方法。也就是,循环以下过程:
\begin{enumerate}
\item AI(“我”)先选择最优的方法下棋;
\item 对方(“你”)再选对于你来说最优、也就是对我来说最差的一个方法下棋。
\end{enumerate}
考虑一个经过一些常规剪枝技巧的、搜索树为二叉树的对弈。假若现在应当 AI 下棋,我们向后搜索 $4$ 步棋,绘制出搜索树并且标出叶子节点的一个“估价”(也就是 AI,“我”,的胜率):
