% 切比雪夫多项式(综述)
% license CCBYSA3
% type Wiki

本文根据 CC-BY-SA 协议转载翻译自维基百科\href{https://en.wikipedia.org/wiki/Hermann_von_Helmholtz}{相关文章}。

\begin{figure}[ht]
\centering
\includegraphics[width=6cm]{./figures/aa69d9e30009e97a.png}
\caption{第一类切比雪夫多项式 $T_n$ 的前五项绘图} \label{fig_QBXFdx_1}
\end{figure}
切比雪夫多项式是两组与余弦函数和正弦函数相关的正交多项式,分别记作 $T_n(x)$ 和 $U_n(x)$。它们有多种等价定义方式,其中一种起始于三角函数的表示:

第一类切比雪夫多项式 $T_n$ 定义为:
$$
T_n(\cos \theta) = \cos(n\theta)~
$$
类似地,第二类切比雪夫多项式 $U_n$ 定义为:
$$
U_n(\cos \theta)\sin \theta = \sin((n+1)\theta)~
$$
乍一看,这些表达式是否真的定义了关于 $\cos \theta$ 的多项式并不明显,但可以通过莫阿弗公式(de Moivre’s formula)来证明这一点(见下文)。
\begin{figure}[ht]
\centering
\includegraphics[width=6cm]{./figures/57edd55432eeb41d.png}
\caption{第二类切比雪夫多项式 $U_n$ 的前五个多项式图像} \label{fig_QBXFdx_2}
\end{figure}
切比雪夫多项式 $T_n$ 是在区间 $[-1, 1]$ 上绝对值被限制在 1 以内、且首项系数最大的多项式。它们也是满足许多其他性质的“极值”多项式之一\(^\text{[1]}\)。

1952 年,科尔内利乌斯·兰齐奥斯指出,切比雪夫多项式在线性系统求解的逼近理论中具有重要作用\(^\text{[2]}\);$T_n(x)$ 的根,也称为切比雪夫节点,被用作多项式插值中的匹配点,从而优化插值过程。由此得到的插值多项式能够减小龙格现象的问题,并在最大范数意义下提供接近最佳的函数逼近,这也称为“极小极大”准则。这种逼近直接引出了克伦肖–柯蒂斯求积法的方法。

这些多项式以帕夫努季·切比雪夫的名字命名\(^\text{[3]}\)。使用字母 $T$ 是因为该名字的其他音译方式,如法语的 Tchebycheff、Tchebyshev,或德语的 Tschebyschow。
\subsection{定义}
\subsubsection{递推定义}
第一类切比雪夫多项式 $T_n(x)$ 可由以下递推关系定义:
$$
\begin{aligned}
T_0(x) &= 1, \\
T_1(x) &= x, \\
T_{n+1}(x) &= 2x \, T_n(x) - T_{n-1}(x).
\end{aligned}~
$$
第二类切比雪夫多项式 $U_n(x)$ 可由以下递推关系定义:
$$
\begin{aligned}
U_0(x) &= 1, \\
U_1(x) &= 2x, \\
U_{n+1}(x) &= 2x \, U_n(x) - U_{n-1}(x),
\end{aligned}~
$$
这个递推关系与第一类切比雪夫多项式的定义几乎相同,仅在 $n = 1$ 时的初始值规则上有所不同。
\subsubsection{三角定义}
第一类与第二类切比雪夫多项式可以定义为满足以下关系的唯一多项式:
$$
T_n(\cos \theta) = \cos(n\theta)~
$$
以及
$$
U_n(\cos \theta) = \frac{\sin((n+1)\theta)}{\sin \theta},~
$$
其中 $n = 0, 1, 2, 3, \ldots$。

另一种等价的表述方式是利用复数的幂形式:设复数 $z = a + bi$,其模为1,则有
$$
z^n = T_n(a) + i b U_{n-1}(a)~
$$
当研究三角多项式时,切比雪夫多项式可以采用这种形式定义\(^\text{[4]}\)。

可以通过以下观察说明 $\cos(nx)$ 是 $\cos(x)$ 的 $n$ 次多项式:$\cos(nx)$ 是下式左边的实部,这是德·莫弗公式(de Moivre’s formula):
$$
\cos(n\theta) + i\sin(n\theta) = (\cos \theta + i \sin \theta)^n~
$$
该等式右边的实部是 $\cos(x)$ 与 $\sin(x)$ 的多项式,其中 $\sin(x)$ 的幂次均为偶数,因此可利用恒等式
$$
\cos^2(x) + \sin^2(x) = 1~
$$
将所有 $\sin(x)$ 的幂次项替换为 $\cos(x)$ 的表达式。类似地,$\sin(nx)$ 是该多项式的虚部,虚部中的 $\sin(x)$ 的幂次均为奇数,若提取出一个 $\sin(x)$ 因子,则其余部分就可以转换为一个关于 $\cos(x)$ 的 $n-1$ 次多项式。

对于 $x$ 在区间 $[-1, 1]$ 之外的情况,上述定义意味着:
$$
T_n(x) =
\begin{cases}
\cos(n \arccos x) & \text{如果 } |x| \leq 1, \\
\cosh(n \, \mathrm{arcosh}\, x) & \text{如果 } x \geq 1, \\
(-1)^n \cosh(n \, \mathrm{arcosh}(-x)) & \text{如果 } x \leq -1
\end{cases}~
$$
\subsubsection{可交换多项式定义}
切比雪夫多项式还可以通过以下定理来刻画:\(^\text{[5]}\)

若 $F_n(x)$ 是一族在特征为 0 的域上的一元首一多项式族,满足:$\deg F_n(x) = n$,以及对所有 $m$ 和 $n$ 都有 $F_m(F_n(x)) = F_n(F_m(x))$,

那么,经过一个简单的变量变换之后,要么
$$
F_n(x) = x^n \quad \text{对所有 } n~
$$
要么
$$
F_n(x) = 2 \cdot T_n(x/2) \quad \text{对所有 } n~
$$
\subsubsection{Pell 方程定义}
切比雪夫多项式还可以定义为以下 Pell 方程在环 $R[x]$ \(^\text{[6]}\)中的解:
$$
T_n(x)^2 - (x^2 - 1) U_{n-1}(x)^2 = 1~
$$
因此,它们也可以通过 Pell 方程的标准解法生成,即从基本解出发取幂:
$$
T_n(x) + U_{n-1}(x) \sqrt{x^2 - 1} = \left(x + \sqrt{x^2 - 1}\right)^n~
$$
\subsubsection{生成函数}
切比雪夫多项式 $T_n$ 的普通生成函数为:
$$
\sum_{n=0}^{\infty} T_n(x)\, t^n = \frac{1 - t x}{1 - 2 t x + t^2}~
$$
切比雪夫多项式还有多个其他形式的生成函数。其中,指数生成函数为:
$$
\begin{aligned}
\sum_{n=0}^{\infty} T_n(x)\, \frac{t^n}{n!}
&= \frac{1}{2} \left( \exp\left(t \left(x - \sqrt{x^2 - 1} \right)\right)+ \exp\left(t \left(x + \sqrt{x^2 - 1} \right)\right) \right)\\
&= e^{t x} \cosh\left(t \sqrt{x^2 - 1}\right)\\
\end{aligned}~
$$
在二维势场理论和多极展开中,相关的生成函数为:
$$
\sum_{n=1}^{\infty} T_n(x)\, \frac{t^n}{n}
= \ln \left( \frac{1}{\sqrt{1 - 2 t x + t^2}} \right)~
$$
切比雪夫第二类多项式 $U_n$ 的**普通生成函数**为:
$$
\sum_{n=0}^{\infty} U_n(x)\, t^n = \frac{1}{1 - 2 t x + t^2}~
$$
其指数生成函数为:
$$
\sum_{n=0}^{\infty} U_n(x)\, \frac{t^n}{n!}
= e^{t x} \left( \cosh\left(t \sqrt{x^2 - 1}\right)
+ \frac{x}{\sqrt{x^2 - 1}} \sinh\left(t \sqrt{x^2 - 1}\right) \right)~
$$
\subsection{第一类与第二类切比雪夫多项式之间的关系}
第一类和第二类切比雪夫多项式分别对应于参数为 $P = 2x$、$Q = 1$ 的一对互补的 Lucas 数列 $\tilde{V}_n(P, Q)$ 和 $\tilde{U}_n(P, Q)$:
$$
\begin{aligned}
\tilde{U}_n(2x, 1) &= U_{n-1}(x), \\
\tilde{V}_n(2x, 1) &= 2 T_n(x).
\end{aligned}~
$$
由此可推出它们还满足一组互相关联的递推关系式:[7]
$$
\begin{aligned}
T_{n+1}(x) &= x\, T_n(x) - (1 - x^2)\, U_{n-1}(x), \\
U_{n+1}(x) &= x\, U_n(x) + T_{n+1}(x)
\end{aligned}~
$$
第二个式子可以通过第二类切比雪夫多项式的递推定义重新整理为:
$$
T_n(x) = \frac{1}{2} \left( U_n(x) - U_{n-2}(x) \right)~
$$
利用该公式迭代展开,可以得到以下求和公式:
$$
U_n(x) = 
\begin{cases}
2 \sum\limits_{\substack{j > 0 \\ j \text{ 奇数}}}^{n} T_j(x) & \text{当 } n \text{ 为奇数时}, \\
2 \sum\limits_{\substack{j \geq 0 \\ j \text{ 偶数}}}^{n} T_j(x) - 1 & \text{当 } n \text{ 为偶数时}
\end{cases}~
$$
这表明第二类切比雪夫多项式可以表示为第一类切比雪夫多项式的加权和。

通过使用 $T_n(x)$ 的导数公式替换 $U_n(x)$  和 $U_{n-2}(x)$,可以得到 $T_n(x)$ 导数的递推关系式:
$$
2\,T_n(x) = \frac{1}{n+1} \frac{d}{dx} T_{n+1}(x) - \frac{1}{n-1} \frac{d}{dx} T_{n-1}(x), \quad n = 2, 3, \ldots~
$$
这个关系式被用于切比雪夫谱方法中以求解微分方程。

图兰不等式对于切比雪夫多项式为:[8]
$$
\begin{aligned}
T_n(x)^2 - T_{n-1}(x)\,T_{n+1}(x) &= 1 - x^2 > 0, & \text{当 } -1 < x < 1, \\
U_n(x)^2 - U_{n-1}(x)\,U_{n+1}(x) &= 1 > 0
\end{aligned}~
$$
积分关系式如下所示:[9][10]
$$
\begin{aligned}
\int_{-1}^{1} \frac{T_n(y)}{y - x} \cdot \frac{dy}{\sqrt{1 - y^2}} &= \pi\, U_{n-1}(x), \\
\int_{-1}^{1} \frac{U_{n-1}(y)}{y - x} \cdot \sqrt{1 - y^2}\,dy &= -\pi\, T_n(x),
\end{aligned}~
$$
其中积分以主值积分的方式计算。
\subsection{显式表达式}
使用复数幂定义的切比雪夫多项式,可以推导出以下在任意实数 $x$ 上成立的表达式:\
$$
\begin{aligned}
T_n(x) &= \frac{1}{2} \left( \left(x - \sqrt{x^2 - 1} \right)^n + \left(x + \sqrt{x^2 - 1} \right)^n \right) \\
&= \frac{1}{2} \left( \left(x - \sqrt{x^2 - 1} \right)^n + \left(x - \sqrt{x^2 - 1} \right)^{-n} \right)
\end{aligned}~
$$
这两个公式是等价的,因为
$$
\left(x + \sqrt{x^2 - 1} \right)\left(x - \sqrt{x^2 - 1} \right) = 1~
$$
根据 de Moivre 公式,可以得到切比雪夫多项式的一个按幂次 $x^k$ 展开的显式形式:
$$
T_n(\cos \theta) = \operatorname{Re}(\cos n\theta + i\sin n\theta) = \operatorname{Re}((\cos \theta + i \sin \theta)^n)~
$$
其中 $\operatorname{Re}$ 表示复数的实部。展开这个公式得到:
$$
(\cos \theta + i \sin \theta)^n = \sum_{j=0}^{n} \binom{n}{j} i^j \sin^j \theta \cos^{n-j} \theta~
$$
该表达式的实部来自于偶数下标项。注意到$i^{2j} = (-1)^j$,以及$\sin^{2j} \theta = (1 - \cos^2 \theta)^j$,可以得到如下显式公式:
$$
\cos(n\theta) = \sum_{j = 0}^{\lfloor n/2 \rfloor} \binom{n}{2j} (\cos^2 \theta - 1)^j \cos^{n - 2j} \theta~
$$
这进一步意味着:
$$
T_n(x) = \sum_{j = 0}^{\lfloor n/2 \rfloor} \binom{n}{2j} (x^2 - 1)^j x^{n - 2j}~
$$
这可以表示为一个 ${}_2F_1$ 超几何函数:
$$
\begin{aligned}
T_n(x) &= \sum_{k=0}^{\left\lfloor \frac{n}{2} \right\rfloor}\binom{n}{2k}(x^2 - 1)^k x^{n - 2k}\\
&= x^n \sum_{k=0}^{\left\lfloor \frac{n}{2} \right\rfloor} \binom{n}{2k}(1 - x^{-2})^k\\
&= \frac{n}{2} \sum_{k=0}^{\left\lfloor \frac{n}{2} \right\rfloor} (-1)^k \frac{(n-k-1)!}{k!(n-2k)!}(2x)^{n - 2k} \quad \text{(当 } n > 0 \text{ 时)}\\
&= n \sum_{k=0}^{n} (-2)^k \frac{(n+k-1)!}{(n-k)!(2k)!} (1 - x)^k \quad \text{(当 } n > 0 \text{ 时)}\\
&= {}_2F_1 \left( -n, n; \frac{1}{2}; \frac{1}{2}(1 - x) \right)
\end{aligned}~
$$
带有反函数[11][12]:
$$
x^n = 2^{1 - n} \mathop{\sum_{\substack{j = 0 \\ j \equiv n \bmod 2}}^{n}}' \binom{n}{\tfrac{n - j}{2}} T_j(x)~
$$
其中,求和符号上的撇号(′)表示:若求和中出现 $j = 0$,其对应的项需要减半处理。

一个关于 $T_n$ 的相关表达式,可将其表示为带有二项式系数和 2 的幂次的单项式之和:
$$
T_n(x) = \sum_{m = 0}^{\left\lfloor \frac{n}{2} \right\rfloor} (-1)^m \left( \binom{n - m}{m} + \binom{n - m - 1}{n - 2m} \right) \cdot 2^{n - 2m - 1} \cdot x^{n - 2m}~
$$
同样地,$U_n$ 也可以用超几何函数表示为:
$$
U_n(x) = \frac{(x + \sqrt{x^2 - 1})^{n+1} - (x - \sqrt{x^2 - 1})^{n+1}}{2\sqrt{x^2 - 1}}~
$$
它还可以展开为多种等价形式:

二项式展开形式:
$$
= \sum_{k=0}^{\left\lfloor n/2 \right\rfloor} \binom{n+1}{2k+1} (x^2 - 1)^k x^{n - 2k}~
$$
提取 $x^n$ 的因子:
$$
= x^n \sum_{k=0}^{\left\lfloor n/2 \right\rfloor} \binom{n+1}{2k+1} \left(1 - x^{-2}\right)^k~
$$
用带负上下标的组合数写成:
$$
= \sum_{k=0}^{\left\lfloor n/2 \right\rfloor} \binom{2k - (n + 1)}{k} (2x)^{n - 2k}, \quad \text{当 } n > 0~
$$
用交错符号的组合数写成:
$$
= \sum_{k=0}^{\left\lfloor n/2 \right\rfloor} (-1)^k \binom{n - k}{k} (2x)^{n - 2k}, \quad \text{当 } n > 0~
$$
用阶乘和幂级数形式表示:
$$
= \sum_{k=0}^{n} (-2)^k \cdot \frac{(n + k + 1)!}{(n - k)! (2k + 1)!} (1 - x)^k, \quad \text{当 } n > 0~
$$
最后可化为超几何函数表达式:
$$
= (n + 1) \cdot {}_2F_1\left(-n,\, n + 2;\, \tfrac{3}{2};\, \tfrac{1}{2}(1 - x)\right)~
$$
其中,${}_2F_1$ 是高斯超几何函数。
\subsection{性质}
\subsubsection{对称性}
$$
\begin{aligned}
T_n(-x) &= (-1)^n\, T_n(x), \\
U_n(-x) &= (-1)^n\, U_n(x).
\end{aligned}~
$$
也就是说,奇偶阶的切比雪夫多项式具有相应的对称性:偶数阶的切比雪夫多项式是偶函数,因此只包含 $x$ 的偶次幂;奇数阶的切比雪夫多项式是奇函数,因此只包含 $x$ 的奇次幂。
\subsubsection{零点与极值}
任意一种类型、次数为 $n$ 的切比雪夫多项式在区间 $[-1,1]$ 上都有 $n$ 个不同的单根,这些根称为切比雪夫根。其中,第一类切比雪夫多项式($T_n$)的根有时被称为切比雪夫节点,因为它们常用于多项式插值中的节点选择。

利用三角函数定义以及恒等式:
$$
\cos\left((2k + 1)\frac{\pi}{2}\right) = 0,~
$$
可以得出第一类切比雪夫多项式 $T_n$ 的根为:
$$
x_k = \cos\left(\frac{\pi(k + 1/2)}{n}\right),\quad k = 0, 1, \dots, n - 1.~
$$
类似地,第二类切比雪夫多项式 $U_n$ 的根为:
$$
x_k = \cos\left(\frac{k\pi}{n + 1}\right),\quad k = 1, 2, \dots, n.~
$$
第一类切比雪夫多项式 $T_n$ 在区间 $-1 \leq x \leq 1$ 上的极值点位于:
$$
x_k = \cos\left(\frac{k\pi}{n}\right),\quad k = 0, 1, \dots, n.~
$$
第一类切比雪夫多项式的一个独特性质是:在区间 $-1 \leq x \leq 1$ 上,它的所有极值点的函数值都严格等于 $-1$ 或 $1$。因此,这些多项式仅有两个有限的临界值,这也是Shabat 多项式的定义性特征。

第一类和第二类切比雪夫多项式在端点处的极值分别为:
$$
\begin{aligned}
T_n(1) &= 1 \\
T_n(-1) &= (-1)^n \\
U_n(1) &= n + 1 \\
U_n(-1) &= (-1)^n(n + 1)
\end{aligned}~
$$
对于 $n > 0$,第一类切比雪夫多项式 $T_n(x)$ 在区间 $-1 \leq x \leq 1$ 上有 $n + 1$ 个极值点。这些极值点出现在:$\pm1$,或者形如 $\cos\left(\frac{2\pi k}{d}\right)$ 的位置,其中满足条件:$d > 2$,$d$ 整除 $2n$(即 $d \mid 2n$),$0 < k < d/2$,$k$ 与 $d$ 互质(即 $\gcd(k, d) = 1$)。换句话说,极值点的横坐标来自于单位圆上的余弦值,其角度对应的是以 $2\pi \frac{k}{d}$ 分布的点,且这些 $k$ 是与 $d$ 互质的小于 $d/2$ 的整数。
