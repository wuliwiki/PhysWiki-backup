% 凸集和凸体
% keys 凸集|凸体
% license Usr
% type Tutor
\pentry{向量空间\nref{nod_LSpace}}{nod_573c}
凸性是向量空间理论中许多重要部分的基础概念。它不仅是直观的几何概念,也允许纯粹解析的叙述。

\subsection{图集的引入}
设 $L$ 是一实向量空间,$x_1,x_2$ 是它的两点。那么过 $x_1,x_2$ 的直线方向与矢量 $x_2-x_1$ 平行,因此该直线可表示为
\begin{equation}
x_1+k(x_2-x_1).~
\end{equation}
或写为
\begin{equation}
(1-k)x_1+kx_2.~
\end{equation}
明显的,当 $k\geq0$ 时,矢量 $k(x_2-x_1)$ 的模(长度)随 $k$ 的增大而增大。即 $\{x_1+k(x_2-x_1)|k\geq0\}$ 是以 $x_1$ 为原点的正方向(由 $x_1$ 指向 $x_2$ 的方向)的直线部分;反之,$\{x_1+k(x_2-x_1)|k<0\}$ 是以 $x_1$ 为原点的负方向部分。
