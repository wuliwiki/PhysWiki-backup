% 第1章:起步—— 第2节:安装和启动
% keys 第1章 起步 第2节 安装和启动

本文转载自郝林的 《Julia 编程基础》.

\subsubsection{1.2 安装和启动}

我们可以通过多种方式安装 Julia 编程语言,但我建议你使用二进制安装包文件,因为这样最便捷.你可以从该语言的官方中文网站的\href{https://cn.julialang.org/downloads/}{下载页面}中下载安装包文件.如果你在访问这个页面时发现打不开或者非常慢,那么也可以去中国科学技术大学建立的\href{https://mirrors.zju.edu.cn/julia/releases/}{镜像站点}或者浙江大学建立的\href{https://mirrors.ustc.edu.cn/julia/releases/}{镜像站点}下载相应的文件.

请注意,本书讲解的 Julia 语言的版本是`v1.3.1`,所以你最好也使用这个版本.也就是说,你最好下载名称中包含了“1.3.1”的安装包文件.如果你确实想使用更新的版本(如`v1.4.0`)也没有关系,Julia 语言会在同一个主版本下保证向后兼容性(即新版本兼容旧版本).另外,你下载的文件应该与你所用计算机系统的计算架构和操作系统相对应.例如,若你所用的计算机系统是 64 位的,并且所用的操作系统是 Windows,那么就应该下载名为 julia-1.3.1-win64.exe 的文件.若你所用的操作系统是 macOS,那么就应该下载名为 julia-1.3.1-mac64.dmg 的文件.

在下载完成之后,你就可以双击这个文件并按照提示一步步去安装了.不出意外的话,你轻点几下鼠标即可完成安装.

顺便说一下,Julia 语言的版本一般会以`vX.Y.Z`的形式表示.其中的`X`、`Y`和`Z`都只可能是正整数或者`0`.由此,`X.Y.Z`就组成了 Julia 语言的版本号(注意,不是版本).这样的版本号遵循了 \href{https://semver.org}{Semantic Versioning 规范}.简单来说,`X`代表主版本号(或称大版本号),`Y`代表次版本号(或称小版本号),而`Z`则代表修订版本号.另外,`X.Y`也可以被称为特性版本号.因为它的递进一般都代表着软件在特性上的更新.而最后的`Z`的递进,一般代表着软件缺陷的修复.所以它也可以被叫做 BUG 修复版本号.

言归正传.在安装完成后,你可以找到那个鲜艳的 Julia 三色图标并双击(也可以尝试在命令行中输入`julia`并回车).如果在当前界面中出现了类似于下图的提示内容,那么就说明你安装成功了.

\begin{figure}[ht]
\centering
\includegraphics[width=14.25cm]{./figures/JuC1S2_1.png}
\caption{图 1-1 Julia 的初始界面} \label{JuC1S2_fig1}
\end{figure}

顺便说一下,如果你想退出这个界面,那么同时按下`Ctrl`和`d`就可以了.

在使用`julia`命令的时候,我们可以追加一些启动参数.一些常用的参数有:`-e`、`-E`、`-p`和`-i`,以及`-v`和`-h`.详细说明如下.

- `-e`:用于直接对跟在后面的表达式进行求值.例如,我们可以输入`julia -e 'a = 5 * 8; println(a)'`并回车.这时,`julia`会对单引号内的表达式进行逐一求值.多个表达式之间需要以英文分号`;`分隔.第二个表达式`println(a)`在被求值时会在计算机的标准输出上打印`40`.当所有求值都完成后,`julia`命令会直接退出(返回命令行提示符).
- `-E`:与`-e`的功能很类似.但不同的是,追加该参数的`julia`命令在退出之前还会在标准输出上打印出最后一个表达式的求值结果.上面的第二个表达式`println(a)`的求值结果会是`nothing`,表示没有结果值.
- `-p`:指定用于处理并行任务的工作进程的数量.跟在它后面的值必须是一个大于 0 的整数,或者为`auto`(指代当前计算机的 CPU 逻辑核心数).例如,如果我们输入的命令是`julia -p 5`,那么工作进程的总数就会是`6`.这是因为 REPL 环境本身还会占用一个工作进程.如果不追加参数`-p`,那么 Julia 就不会产生额外的工作进程.
- `-i`:用于以交互模式运行命令.这意味着,命令执行后将进入 REPL(Read–eval–print loop)环境.简单来说,这个 REPL 环境就是一个可以与 Julia 的运行时系统进行即时交互的界面.比如,你在这个环境中输入`println("abc")`并回车,它立马就会回显独占一行的`abc`和一个空行.从字面上我们也可以了解到,该环境会读取你输入的表达式、对读到的表达式进行求值、显示表达式的求值结果,然后再次等待读取.如此循环往复.如果我们在输入`julia`命令的时候没有追加任何源码文件,那么它就会以交互模式运行.
- `-v`:仅用于显示当前的 Julia 的版本.比如:`julia version 1.3.1`.
- `-h`:仅用于显示`julia`命令的具体用法.其中包括了所有可用参数的说明.

到这里,我们已经对 Julia 有了一个初步的认识.要想玩转 Julia,我们首先就应该充分熟悉`julia`命令及其 REPL 环境.不过别担心,我们后面要用到它们的地方还多着呢,你有的是机会熟悉.

原文链接:\href{https://github.com/hyper0x/JuliaBasics/blob/master/book/ch01.md}{第1章:起步}