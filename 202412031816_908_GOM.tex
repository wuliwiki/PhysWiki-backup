% 格奥尔格·欧姆(综述)
% license CCBYSA3
% type Wiki

本文根据 CC-BY-SA 协议转载翻译自维基百科\href{https://en.wikipedia.org/wiki/Georg_Ohm}{相关文章}。


乔治·西蒙·欧姆(Georg Simon Ohm,/oʊm/;德语:[ˈɡeːɔʁk ˈʔoːm];1789年3月16日 – 1854年7月6日)是德国物理学家和数学家。作为一名学校教师,欧姆开始研究由意大利科学家亚历山德罗·伏打发明的新的电化学电池。通过使用他自己制作的设备,欧姆发现导体两端施加的电位差(电压)与产生的电流之间存在直接的正比关系。这个关系被称为欧姆定律,而电阻的国际单位“欧姆”(Ω)也以他的名字命名。