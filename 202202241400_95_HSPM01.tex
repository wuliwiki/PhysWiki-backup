% 匀变速直线运动
% 高中物理|位移|速度|加速度

\subsection{基本概念}
\subsubsection{质点}
某些情况下,物体的大小和形状对研究的问题没有影响或影响可以忽略,而只需突出“物体具有质量”这个要素,我们可以把这个物体简化为一个具有质量的物质点,这样的点称为\textbf{质点}.

\subsubsection{参考系}
判断一个物体是运动还是静止,总要选取一个物体作为标准,被选作标准的物体叫\textbf{参照物}.
为了描述一个物体在空间中位置随时间的变化,我们要在参照物上建立一套\textbf{坐标系},并在同一坐标系的各处都配置同步的时钟,这便组成了一个\textbf{参考系}.

一般地,我们不严格区分“参照物”和“参考系”,而是强调\textbf{坐标系是参考系的数学抽象}.

\subsubsection{时间和时刻}
时刻:某一瞬时,是时间轴上的一点.常见描述为“第$n$秒初(末)”.

时间间隔:两个时刻之间的间隔,是时间轴上的一段.