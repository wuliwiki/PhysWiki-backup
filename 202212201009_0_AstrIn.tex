% 天文学笔记(科普)

\begin{issues}
\issueDraft
\end{issues}

\begin{itemize}
\item 太阳系\textbf{八大行星}: 水星(Mercury)、金星(Venus)、地球(Earth)、火星(Mars)、木星(Jupiter)、土星(Saturn)、天王星(Uranus)、海王星(Neptune)。
\item \textbf{天文单位}: 约等于地球到太阳的距离(光走八分钟)。
\item \textbf{光年}: 光走一年的长度
\item 夜空中的亮点绝大部分都是恒星。 一些例外包括八大行星、坠入大气的陨石(流星)、 人造卫星以及飞行器。
\item \textbf{主小行星带}: 介于火星和木星之间。
\item \textbf{柯伊伯带}: 海王星之外。
\item \textbf{奥尔特星云}: 太阳系的最外层。
\item 离太阳最近的恒星是(), 距离太阳约 4 光年, 所以我们看到的是它四年前的样子。 其他恒星也同理。
\item 天文望远镜一般建在山上, 为了避免大气干扰。
\end{itemize}
