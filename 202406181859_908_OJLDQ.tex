% 欧几里得群
% license CCBYSA3
% type Wiki

(本文根据 CC-BY-SA 协议转载自原搜狗科学百科对英文维基百科的翻译)

在数学中,欧几里德群是欧几里德空间𝔼n的(欧几里德)等距群;也就是说,欧几里德群是保持欧几里德空间中任意两点欧几里德距离不变的变换群(也称为欧几里德变换)。 该群仅取决于空间的维度n,并且通常表示为 $E$(n 或ISO(n)。

欧几里德群E(n)由欧几里德空间𝔼n所有的平移变换、旋转变换、反射变换以及它们的任意有限组合构成。欧几里德群可以被看作是空间自身的对称群,并且它包含该空间的任何图(子集)的对称群。

欧几里德等距可以是直接的或间接的,这取决于它是否保留了图形的旋向性。直接的欧几里德等距形成一个子群——特殊的欧几里德群,其中的元素被称为刚体运动或者欧几里德运动。它们包括平移和旋转的任意组合,但不包括反射。

这些群是最古老和被研究最多的群之一,至少在维度为2和维度为3的情况下是如此一一含蓄地说,这些群早在群的概念发明之前就有了。

\subsection{综述}



\subsubsection{1.1 维度}



\subsubsection{1.2 直接和间接等距}



\subsubsection{1.3 群的拓扑}



\subsubsection{1.4 李结构}



\subsubsection{1.5 和仿射群的关系}



\subsection{详细论述}



\subsubsection{2.1 子群的结构,矩阵和向量表达}

欧几里德群是仿射变换群的一个子群。

它具有平移群$T(n)$和正交群$O(n)$作为子群。$E(n)$的任何元素都是一个平移变换复合一个正交变换(等距的线性部分),以下面这种独特的方式:




\subsubsection{2.2 子群}

$E(n)$的子群类型

\textbf{有限群}

这些变换总有一个不动点。 在3D中,每个点在包含关系下在每个方向上在有限群众有两个极大群:$0_h$ 和$l_h$。群 $l_h$ 在包括下一类别的群中甚至是最大的。

没有任意小平移、旋转或其组合的可数无限群

也就是说,对于每个点,等距下的像集在拓扑上是离散的(例如,对于 1 ≤ m ≤ n 由独立方向上的 m 个平移产生的群,并且很可能是有限点群)。 这包括格。 比这些更一般的例子是离散空间群。

具有任意小平移、旋转或其组合的可数无限群

也就是说,在这种情形,存在某些点在等距变换下像集的不是闭合的。例如,在一维空间中,由1和 √2生成的平移群,在二维空间中,由围绕原点旋转1弧度生成的旋转群。
满足存在某些点在等距变换下像集的不是闭合的不可数群

(例如,在2D中,在一个方向上的所有平移,以及在另一个方向上的平移了有理距离的所有平移)。

满足所有点在等距变换下的像集是闭合的不可数群

例如:

\begin{itemize}
\item 保持原点固定的所有直接等距,或者更一般地说,保持某些点固定的所有直接等距(在3D中被称为旋转群)
\item 保持原点固定的所有等距,或者更一般地说,保持某些点固定的所有等距(正交群)
\item 所有直接等距$E^+(n)$
\item 整个欧几里得群$E^(n)$
\item m维子空间中的上述例子的某一个群与正交$(n-m)$维空间中的离散等距群的组合
\item m维子空间中的上述例子的某一个群与正交$(n-m)$维空间中的上述例子的某一个群的组合
\end{itemize}

3D中的组合示例:

\begin{itemize}
\item 围绕一个固定轴的所有旋转
\item 围绕一个固定轴的所有旋转与通过轴的平面和/或垂直于轴的平面的反射的组合
\item 围绕一个固定轴的所有旋转与沿轴的离散平移或沿轴的所有等距的组合
\item 平面中的离散点群、带状群或壁纸群与垂直方向上的任何对称群的组合
\item 所有围绕某个轴的旋转和沿轴的比例平移的组合的等距; 通常,这与围绕同一轴的k次旋转等距相结合$(k \geq 1)$; 等距变换下点的像集是k次螺旋线;另外,可能存在围绕垂直相交的轴的2次旋转,因此存在这些轴的$k$次螺旋。
\item 对于任何点群:所有等距群,它是点群中的等距和平移的组合; 例如,由在原点反向生成的群:它是所有平移的群和所有点的反向生成的群; 这就是$R^3$的广义二面群,$Dih(R^3)$。
\end{itemize}

\subsubsection{2.3 维数小于等于3的等距的概述}



\subsubsection{2.4 交换等距}

某些等距组合不依赖与顺序:

\begin{itemize}
\item 绕同一轴的两个旋转或者螺旋
\item 相对于平面的反射,以及该平面中的平移,围绕垂直于平面的轴的旋转,或相对于垂直平面的反射
\item 相对于平面的平移反射,以及在该平面中的一个点上的反向平移以及保持该点固定的任何等距
\item 绕轴180°的旋转和在平面中通过该轴的反射
\item 绕轴180°旋转和绕垂直轴180°的旋转(结果是绕垂直两者的轴线旋转180度)
\item 相对于同一平面,绕同一轴的两个旋转反射
\item 相对于同一平面的两个滑动反射
\end{itemize}


\subsubsection{2.5 共轭类}

在给定方向上移动给定距离的平移形成一个共轭类,平移群就是所有给定距离共轭类的并集。

在1D中,所有反射都在同一类中。

在2D中,在任一方向上旋转相同角度的旋转都在同一类中。 具有相同距离平移的滑动反射在同一类中。

在3D中:

\begin{itemize}
\item 关于所有点的反向属于同一类。
\item 相同角度的旋转属于同一类。
\item 如果角度相同且平移距离相同,则围绕轴的旋转与沿该轴的平移的组合在同一类中。
\item 平面中的反射属于同一类
\item 平面中的反射与该平面中相同距离的平移的组合属于同一类。
\item 围绕轴且不等于180°相同角度的旋转与平面中垂直于该轴的反射的组合属于同一类。
\end{itemize}


