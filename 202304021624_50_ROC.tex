% 接收者操作特征曲线
% 接收 操作 特徵曲線 信号检测论 分类

\textbf{接收者操作特征曲线}(Receiver operating characteristic curve, ROC)是一种用于评价二分类器的分类性能的图表。该方法来源于信号检测论,是在第二次世界大战中,由电子信号工程师发明的。在心理学领域中也有广泛应用。

曲线中点的坐标是\textbf{真阳率}(True positive rate, TPR)和\textbf{假阳率}(False positive rate, FPR)。随着分类器阈值的变化,真阳率和假阳率分别随之改变,由此产生一系列的点,然后将相邻两点连接起来,即构成接收者操作特征曲线。

\begin{figure}[ht]
\centering
\includegraphics[width=14.25cm]{./figures/b7989d818df6474e.png}
\caption{ROC曲线示意图} \label{fig_ROC_2}
\end{figure}
上图中蓝色曲线即为ROC曲线。横坐标为假阳率,纵坐标为真阳率。红色对角线表示一个完全随机分类器的ROC曲线。如果一个分类器的ROC曲线大体在红色虚线的上方,则表示性能优于随机分类器。

ROC曲线分析可以帮助我们做模型选择,选择最优模型,抛弃次优模型。曲线下面积(AUC)是ROC曲线最基本的评估指标,顾名思义,表示的是曲线下方,横轴上方的面积。通常把整图的面积定义为1,则AUC的值在$0$和$1$之间。该面积值越大,则表示分类模型效果越好;反之,则越差。