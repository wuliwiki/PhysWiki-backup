% 施图姆—刘维尔理论
% license Xiao
% type Tutor

\begin{issues}
\issueDraft
\end{issues}

\pentry{二阶常系数齐次微分方程\nref{nod_Ode2}}{nod_c960}


% 研究 S-L 定理前,模仿向量、我们先讨论函数有关的一些定义。
% \begin{definition}{函数的内积}
% 对于在 $[a, b]$ 上的函数 $f(x)$、$g(x)$,定义他们的内积:
% \begin{equation}
% (f, g) = \int_a^b f(x) g(x) \dd x ~.
% \end{equation}
% 若积分存在,则内积存在。
% \end{definition}

% \begin{definition}{函数正交}
% 两个函数正交定义为他们的内积为 $0$。
% \end{definition}

\textbf{施图姆—刘维尔定理(Sturm–Liouville theorem)} 简称施—刘定理或 S-L 定理。提供了一种找到一类正交函数集,使得能将一个函数展开为这一类正交函数集所构成的级数的方法。

\begin{definition}{施图姆—刘维尔型方程(S-L 方程)}
微分方程
\begin{equation}
\dv{x}\qty[p(x)\dv{y}{x}] - q(x) y = -\lambda w(x) y~.
\end{equation}
被称为 S-L 方程,限制 $x \in [a, b]$。其中 $w(x)$ 又被称为权函数,又写作 $\rho(x)$。
\addTODO{不同本征值的解正交, 零点的数量, 等}
\end{definition}


研究二阶常微分方程的本征值问题时,对于一般的二阶常微分方程:
$$y'' + a(x) y' +b(x) y$$
 
应用: 定态薛定谔方程(束缚态) 为什么不能应用到散射态?
