% Bohr-Sommerfeld 氢原子模型

\begin{issues}
\issueDraft
\end{issues}

\footnote{参考 Wikipedia \href{https://en.wikipedia.org/wiki/Old_quantum_theory}{相关页面}
.}使用原子单位, 氢原子 $n,l,m$ 束缚态中, 总角动量为 $L = \sqrt{l(l+1)}$, 能量为 $E = -Z^2/(2n^2)$. 但不知为何 Sommerfeld 模型中取 $L = l$. 也就是说当 $m = l$ 时角动量可以刚延 $z$ 方向.

Sommerfeld 模型中量子化条件为($T$ 表示轨道的一个周期)
\begin{equation}
\int_T m\dot r \dd{r} = nh
\end{equation}
即
\begin{equation}\label{BohrEc_eq1}
2\int_{a-c}^{a+c} \sqrt{2m\qty(E - \frac{k}{r} - \frac{L^2}{2mr^2})} \dd{r} = kh
\end{equation}
若取 $L = l$ 可证明
\begin{equation}\label{BohrEc_eq2}
n = l + k
\end{equation}
开普勒问题
\begin{equation}
e = \sqrt{1 + \frac{2EL^2}{mk^2}}
\end{equation}
\begin{equation}
a = \frac{k}{2\abs{E}}
\end{equation}

\subsubsection{Matlab 数值积分}
以下程序用于数值验证\autoref{BohrEc_eq2}. 注意原子单位中 $h = 2\pi$.
\begin{lstlisting}[language=matlab]
n = 2; l = 1;
Z = 1; k = -Z;
E = -Z^2/(2*n^2);
L = l;
me = 1;
a = abs(k)/(2*abs(E)); b = L/sqrt(2*me*abs(E));
c = sqrt(a^2 - b^2);
f = @(r)real(sqrt(2*me*(E - k./r - L^2./(2*me*r.^2))));
r = linspace(a-c, a+c, 500); figure; plot(r, f(r));
I = 2*integral(f, a - c, a + c);
disp(I/(2*pi))
\end{lstlisting}
