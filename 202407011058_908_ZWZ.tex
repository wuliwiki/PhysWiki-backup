% 中微子
% license CCBYSA3
% type Wiki

(本文根据 CC-BY-SA 协议转载自原搜狗科学百科对英文维基百科的翻译)

中微子(/nuːˈtriːnoʊ/或/njuːˈtriːnoʊ/)(由希腊字母ν表示)是费米子(一种具有半整数自旋的基本粒子),它仅通过弱力和引力参与相互作用。[1][2]中微子之所以如此命名,是因为它是电中性的,并且因为它的静止质量非常小,以致于人们长期以来认为它为零。中微子的质量比其他已知的基本粒子小得多。弱力的范围非常短,引力相互作用非常弱,中微子作为轻子不参与强相互作用。因此,中微子通常能畅通无阻地穿过普通物质,无法被探测到。[1][2]

弱相互作用产生三味轻子类型之一的中微子:电子中微子($\nu_e$)、μ子中微子($\nu_\mu$)、或$\tau$中微子($\nu_\tau$)并伴随相应的带电轻子。虽然中微子长期以来被认为是无质量的,但现在已经知道有三个不同微小值的离散中微子质量,但它们并不与这三种味道唯一对应。一个带有特定味道的中微子是所有三种质量状态的特定量子叠加。因此,中微子在飞行中在不同的味道之间振荡。例如,在衰变反应中产生的电子中微子可能在远处的探测器中作为$\mu$子或$\tau$中微子相互作用。尽管截至2016年,只有三个质量值的平均差是已知的,但宇宙学观察表明,三个质量的总和必须小于电子质量的百万分之一。

对于每一个中微子,也存在一个相应的反粒子,称为反中微子,它也有半整数自旋,没有电荷。它们与中微子的区别在于轻子数和手性符号相反。为了使总的轻子数守恒,在核$\beta$衰变中,电子中微子只与正电子(反电子)或电子反中微子一起出现,电子反中微子与电子或电子中微子一起出现。

中微子是由各种放射性衰变产生的,包括原子核或强子的$\beta$衰变、核反应,如发生在恒星核心或核反应堆、核弹或粒子加速器中的核反应、超新星爆发期间、中子星自旋期间以及加速粒子束或宇宙射线撞击原子时。地球附近的大多数中微子来自太阳的核反应。在地球附近,垂直于太阳方向,每平方厘米每秒大约有650亿($6.5\times10^{10}$)太阳中微子穿过。[3]

为了研究中微子,可以用核反应堆和粒子加速器人工制造中微子。有大量涉及中微子的研究活动,目标包括确定三个中微子质量值,测量轻子区的CP破坏程度(导致轻子发生);并寻找粒子物理标准模型之外的物理证据,如无中微子双$\beta$衰变,这将是轻子数守恒破坏的证据。中微子也可以用于地球内部的断层摄影。[4][5]

\subsection{历史}
\subsubsection{1.1 泡利的提议}
中微子是沃尔夫冈泡利在1930年首次提出的,用来解释$\beta$衰变如何使能量、动量和角动量(自旋)守恒。与尼尔斯·玻尔相反,他提出了守恒定律的统计版本来解释在$\beta$衰变中观察到的连续能谱,泡利假设了一个未被发现的粒子,他称之为“中子”,使用与命名质子和电子相同的-on结尾来命名。他认为新粒子是在β衰变过程中与电子或β粒子一起从原子核中发射出来的。[6]

詹姆斯·查德威克在1932年发现了一种质量更大的中性核粒子,并将其命名为中子,留下了两种同名的粒子。早些时候(1930年),泡利用“中子”一词既指在$\beta$衰变中保存能量的中性粒子,也指原子核中假定的中性粒子;起初,他并不认为这两种中性粒子彼此不同。[6]中微子一词是通过恩利克·费米引入科学词汇的,他在1932年7月巴黎的一次会议上和1933年10月的苏威会议上使用了中微子,泡利也在那次会议上使用了中微子。这个名字(意大利语中相当于“小中性粒子”)是由爱德华多·阿马尔迪(Edoardo Amaldi)在罗马via Panisperna物理研究所与费米(Fermi)交谈时开玩笑地创造出来的,目的是为了将这个轻中性粒子与查德威克的重中子区分开来。[7]

在费米的β衰变理论中,查德威克的大中性粒子可以衰变为质子、电子和较小的中性粒子(现在称为电子反中微子):
$$n^0 \to p^+ +e^- +\nu_e~$$
费米的论文写于1934年,将泡利中微子与保罗·狄拉克正电子和维尔纳海森堡中子质子模型统一起来,为未来的实验工作提供了坚实的理论基础。《自然》杂志拒绝了费米的论文,称该理论“离现实太远”。他将论文提交给一家意大利杂志,该杂志接受了论文,但在早期对他的理论普遍缺乏兴趣,导致他转向实验物理学。[8]

到1934年,有实验证据反对玻尔关于能量守恒对$\beta$衰变无效的观点:在那一年的索尔维会议上,报道了对β粒子(电子)能谱的测量,表明每种$\beta$衰变的电子能量都有严格的限制。如果能量守恒是无效的,这种限制是不可预期的,在这种情况下,在至少几个衰变中,统计上任意数量的能量都是有可能的。1934年首次测量到的β衰变谱的自然解释是,只有有限的(和守恒的)能量可用,一个新粒子有时会吸收有限能量的不同部分,剩下的留给β粒子。泡利利用这个机会公开强调仍然未被发现的“中微子”一定是一个真实的粒子。

\subsubsection{1.2 直接检测}
\begin{figure}[ht]
\centering
\includegraphics[width=6cm]{./figures/474e24bba6d42f22.png}
\caption{克莱德·科温进行中微子实验大约1956} \label{fig_ZWZ_1}
\end{figure}
1942年,王淦昌首次提出使用β俘获来实验检测中微子。[9]在1956年7月20日的《科学》杂志上,克莱德·考恩、弗雷德里克·莱因斯、哈里森、克鲁斯和麦奎尔发表了他们已经探测到中微子的证据,[10][11]这一结果在近40年后获得了1995年诺贝尔奖。[12]

在这个现在被称为考恩-雷恩中微子实验的实验中,通过在核反应堆中的β衰变产生的反中微子与质子反应产生中子和正电子:
$$\nu_e+p^+\to n^0+e^+~$$
正电子很快找到一个电子,然后它们互相湮灭。产生的两种伽马射线($\gamma$)是可探测的。中子可以通过被适当的原子核俘获并释放出伽马射线来探测。这巧合的两个事件——正电子湮没和中子俘获——给出了反中微子相互作用的独特标志。

1965年2月,包括弗雷德尔·塞尔肖普在内的一个团体在南非的金矿中发现了自然界中第一个中微子。[13]实验是在博克堡附近ERPM矿3公里深处的一个专门准备的小室中进行的。主楼的一块匾纪念了这一发现。实验还实现了原始中微子天文学,并研究了中微子物理和弱相互作用问题。[14]

\subsubsection{1.3 中微子味道}
Cowan和Reines发现的反中微子是电子中微子的反粒子。

1962年,利昂·M·莱德曼、梅尔文·施瓦茨和杰克·斯坦伯格通过首先探测$\mu$子中微子(已经用neutretto这个名字做了假设)的相互作用,[15]证明了不止一种中微子的存在,这使他们获得了1988年诺贝尔物理学奖。

当第三种轻子$\tau$于1975年在斯坦福线性加速器中心被发现时,它也有一个相关的中微子(τ中微子)。第三种中微子类型的第一个证据来自对τ衰变中丢失能量和动量的观察,$\tau$衰变类似于导致发现电子中微子的β衰变。费米实验室的DONUT合作组于2000年宣布首次探测到τ中微子相互作用;从大型正负电子对撞机的理论一致性和实验数据已经推断出它的存在。[16]

\subsubsection{1.4 太阳中微子问题}
在20世纪60年代,现在著名的豪斯休实验首次测量了来自太阳核心的电子中微子通量,发现了一个介于标准太阳模型预测的三分之一到二分之一之间的数值。这种差异,后来被称为太阳中微子问题,在大约三十年的时间里一直没有得到解决,尽管对实验和太阳模型可能存在的问题进行了研究,但没有发现任何问题。最终人们意识到两者都是正确的,而且它们之间的差异是由于中微子比以前假设的要复杂。据推测,这三个中微子的质量非零且略有不同,因此在飞往地球的过程中会振荡成不可察觉的味道。这一假设被一系列新的实验所研究,从而开启了一个新的主要研究领域,这个领域还在继续。中微子振荡现象的最终证实导致了两项诺贝尔奖,一项是小雷蒙德·戴维斯,他构思并领导了Homestake实验,另一项是阿特·麦克唐纳,他领导了SNO实验,该实验可以探测所有中微子的味道,并且没有发现任何缺陷。

\subsubsection{1.5 振荡}
Bruno Pontecorvo在1957年首次提出了一种研究中微子振荡的实用方法,这种方法与kaon振荡相似;在随后的10年里,他发展了真空震荡的数学形式和现代公式。1985年斯坦尼斯拉夫·米哈耶夫和阿列克谢·斯米尔诺夫(扩展了林肯·沃尔夫斯坦1978年的工作)指出,当中微子在物质中传播时,味道振荡可以被改变。这种所谓的米基耶夫-斯米尔诺夫-沃尔夫斯坦效应(MSW effect)很重要,因为许多太阳聚变产生的中微子在到达地球探测器的途中会穿过太阳核心的致密物质(基本上所有的太阳聚变都发生在这里)。

从1998年开始,实验开始表明太阳和大气中微子会改变味(见超级卡米康德和萨德伯里中微子天文台)。这解决了太阳中微子问题:太阳中产生的电子中微子部分变成了实验无法检测到的其他味道。

尽管单个实验,如太阳中微子实验,与中微子味道转换的非振荡机制相一致,但中微子实验暗示中微子振荡的存在。在这方面特别相关的是反应堆实验KamLAND和加速器实验,如MINOS。KamLAND实验确实将振荡确定为太阳电子中微子中涉及的中微子味道转换机制。类似地,MINOS证实了大气中微子的振荡,并给出了更好地确定了质量平方劈裂。[17] 日本的梶田隆章和加拿大的亚瑟·麦克唐纳获得了2015年诺贝尔物理学奖,因为他们在理论和实验上的里程碑式发现,中微子可以改变味道。

\subsubsection{1.6 宇宙中微子}
小雷蒙德·戴维斯和小柴昌俊共同获得了2002年诺贝尔物理学奖。两人都在太阳中微子探测方面进行了开创性的工作,小柴昌俊的工作还促使了对附近大麦哲伦星云中SN 1987A超新星中微子的首次实时观测。这些努力标志着中微子天文学的开始。[18]

SN 1987A代表了超新星中微子的唯一被证实的探测。然而,在我们的星系中,许多恒星已经变成超新星,留下了一个理论上的扩散超新星中微子背景。同样的概念延伸到整个宇宙,给出宇宙中微子背景,或遗迹中微子。

\subsection{ 性质和反应}
中微子具有半整数自旋(½ h),因此是费米子。中微子也是轻子,已经被观察到只通过弱力相互作用,尽管假设它们也在引力作用下相互作用。

\subsubsection{2.1 味道、质量及其混合}
弱相互作用产生三种轻子类型之一的中微子:电子中微子($\nu_e$)、$\mu$子中微子($\nu_\mu$)或$\tau$中微子($\nu_\tau$),分别与相应的电子、$\mu$子和带电轻子相关联。[19]

虽然中微子长期以来被认为是无质量的,但现在我们知道也有三个离散的中微子质量,但它们并不唯一对应于这三种味道。尽管截至2016年,只有这三个质量值的平方差是已知的,[19]但实验表明,这些质量很小。根据宇宙学测量,已经计算出三个中微子质量的总和必须小于电子质量的百万分之一。[19][19]

更正式地说,中微子味本征态不同于中微子质量本征态(简称为1,2,3)。截至2016年,还不知道这三个中哪一个最重。与带电轻子的质量等级相似,质量2比质量3轻的构型通常被称为“正常等级”,而在“反向等级”中,情况正好相反。几个主要的实验努力正在进行,以帮助确定哪个是正确的。[19]

一个中微子以一种特定的味道本征态产生,它是所有三种质量本征态的相关特定量子叠加。这是可能的,因为由于不确定性原理,这三个质量相差很小,以至于在任何实际飞行路径中都无法通过实验加以区分。已经发现,在产生的纯味状态中,每种质量状态的比例强烈依赖于该味。味和质量本征态之间的关系编码在PMNS矩阵中。实验已经确定了矩阵元素的值。[19]

中微子存在质量,这使得中微子有一个微小的磁矩成为可能,在这种情况下中微子也可以参与电磁相互作用;还没有发现这种相互作用。[20]









































