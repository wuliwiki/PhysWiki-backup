% 2011年计算机学科专业基础综合全国联考卷
% 2011 | 计算机学科专业基础综合 | 全国联考卷

\subsection{一、单项选择题}
1-40小题,每小题2分,共80分,下列每小题给出的四个选项中,只有一项符合题目要求的.请在答题卡上将所选项的字母涂黑.

1. 设 n 是描述问题规模的非负整数,下面程序片段的时间复杂度是  \\
x=2;  \\
while(x<n/2)  \\
$\quad$ x=2*x;  \\
A. $O(log_2n)$ $\quad$ B. $O(n)$ $\quad$ C. $O(nlog_2n)$ $\quad$ D. $O(n^2)$

2. 元素$a$,$b$,$c$,$d$,$e$依次进入初始为空的栈中,若元素进栈后可停留、可出栈,直到所有元素都出栈,则在所有可能的出栈序列中,以元素$d$开头的序列个数是  \\
A. 3  $\quad$  B. 4  $\quad$  C. 5  $\quad$  D. 6

3. 已知循环队列存储在一维数组$A[0...n-1]$中,且队列非空时front和rear分别指向队头元素和队尾元素.若初始时队列为空,且要求第1个进入队列的元素存储在$A[0]$处,则初始时front和rear的值分别是 \\
A. $0$,$0$  $\quad$ B. $0$,$n-1$ $\quad$ C. $n-1$,$0$ $\quad$ D. $n-1$,$n-1$ 

4. 若一棵完全二叉树有768个结点,则该二叉树中叶结点的个数是 \\
A. 257 $\quad$ B. 258 $\quad$ C. 384 $\quad$ D. 385


\subsection{二、综合应用题}
41~47小题,共70分.请将答案写在答题纸指定位置上.

41.(8分)已知有$6$个顶点(顶点编号为为$0$~$5$)的有向带权图$G$,其邻接矩阵$A$为上三角矩阵,按行为主序




\subsection{参考答案}
\subsection{一、单项选择题}
1. 解答:A.程序中,执行频率最高的语句为“$x=2*x$”.设该语句执行了$t$次,则$2t+1=n/2$, 故$t=log2(n/2)-1=log2n-2= O(log2n)$.

2. 解答:B.出栈顺序必为d_c_b_a_,e的顺序不定,在任意一个“_”上都有可能.

3. 解答:B.插入元素时,front不变,rear+1.而插入第一个元素之后,队尾要指向尾元素,显然,rear初始应该为$n-1$,front为$0$.

4. 解答:C.叶结点数为$n$,则度为2的结点数为$n-1$,度为1的结点数为0或1,本题中为1(总结点数为偶数),故而即$2n=768$.