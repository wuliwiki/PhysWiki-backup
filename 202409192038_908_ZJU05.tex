% 浙江大学 2005 年 考研 量子力学
% license Usr
% type Note

\textbf{声明}:“该内容来源于网络公开资料,不保证真实性,如有侵权请联系管理员”

\subsection{第一题:简答题(28分)}
\begin{enumerate}
    \item 写出测不准关系;
    \item 写出泡利矩阵;
    \item 对于 $\hat{H} = \frac{\hat{p}^2}{2m} + \alpha \hat{L}_z$,($\alpha$ 为常数),下列力学量中哪些是守恒量?
    \[    \hat{H}, \hat{p}_x, \hat{p}_y, \hat{p}_z, \hat{L}_x, \hat{L}_y, \hat{L}_z, \hat{L}^2 ~\]
    \item 能级的简并度指的是什么?
\end{enumerate}
\subsection{第二题:(21分)}
\begin{enumerate}
    \item 电子在三维均匀磁场中运动,$\mathbf{B} = (0, 0, B)$,试写出描述该系统的哈密顿量;
    \item 现在有三种能级 $E_n^I \propto \frac{1}{n^2}$,$E_n^{II} \propto n^2$,$E_n^{III} \propto n$,请分别指出他们对应的是哪些系统;
    \item 放射性指的是某些原子核中的更小粒子有一定的概率逃逸出来。你认为这与什么量子效应有关?
\end{enumerate}
\subsection{第三题:(只需选做(A)、(B)中一题)(20分)}
已知氢原子的基态波函数为
\[\psi(r, \theta, \varphi) = \frac{1}{\sqrt{\pi a_0^3}} e^{-r/a_0},~\]
求:

(A) 势能的平均值 \(V(r) = -\frac{e^2}{r}\);

(B) 动能的平均值。
\subsection{第四题:(21分)}
考虑一维阶梯势 \(V(x)\):
\[V(x) = \begin{cases} U_0, & x > 0 \\\\0, & x < 0\end{cases}~\]
若能量 \(E\) 的粒子 \( (E > U_0) \) 从左边入射,试求该阶梯的反射系数和透射系数。
\subsection{第五题:(20分)}
