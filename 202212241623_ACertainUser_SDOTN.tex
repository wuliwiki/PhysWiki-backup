% 滑动还是翻转

当我们推动一个物体时,这个物体有时会在平面上滑动,又有时会发生翻转。那么,物体是滑动还是翻转,又和什么因素有关系呢?

\subsection{滑动}
我们先假定物体发生滑动。那么当物体发生滑动(前的瞬间),推力恰好等于静摩擦力。使物体滑移所需要的推力为:

\begin{equation}
\begin{aligned}
F&=f\\
&=\mu G\\
&=\mu mg
\end{aligned}
\end{equation}

\subsection{滚动}
我们先假定物体发生翻转。那么当物体发生翻转的瞬间,推力矩恰好等于重力矩。使物体翻转所需要的推力为:
\begin{equation}
\begin{aligned}
Fh&=mg \frac{d}{2}\\
F&=mg \frac{d}{2h}\\
\end{aligned}
\end{equation}

\subsection{滑动还是翻转}
可见,“滑动还是翻转”问题的关键在于,哪种情况所需的推力较小。

更进一步地说,结果取决于物体与表面的静摩擦系数与物体的几何因素(推力的高度)。如果静摩擦系数较小,那么物体就发生滑动;否则,物体就发生翻转。
\begin{equation}
\min(\mu mg, mg \frac{d}{2h})\Rightarrow
\min(\mu, \frac{d}{2h})
\end{equation}
这符合我们的日常直觉:高摩擦力表面的物体更容易翻转(你一定在铺着橡胶垫的餐桌上推动过水杯),而扁平的物体更容易滑动。
