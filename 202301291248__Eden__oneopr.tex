% 单体算符
% 多体系统|单体算符|二次量子化

\pentry{二次量子化\upref{SecQua}}

现在我们来考察多体系统量子力学的单体算符。具体而言,我们希望考察多体系统中单粒子相关的物理量,由于是可观测的物理量,我们研究的单体算符是 Fock 空间 $\mathcal{F}$ 上的幺正算符。

回忆一下 Fock 空间的定义\autoref{SecQua_eq6}~\upref{SecQua},我们知道它可以根据粒子数不同划分为多个子 Hilbert 空间的直和:
\begin{equation}
\mathcal{F}=\mathcal{H}_1\oplus \mathcal{H}_2\oplus \cdots
\end{equation}
根据我们对内积的定义,对于两个态矢量 $v_i\in \mathcal{H}_i, v_j\in \mathcal{H}_j$,当 $i\neq j$ 时 $v_i,v_j$ 的内积一定是 $0$。而我们又希望单体算符 $A^{(1)}$ 所对应的物理量由 $\bra{v}A^{(1)} \ket{v}$ 给出,这实际上告诉我们 $A^{(1)}$ 作用于 $v_i\in \mathcal{H}_i$ 后所得到的态矢量 $A^{(1)} \ket{v_i}$ 也一定属于 $\mathcal{H}_i$。因此当我们限制在每个子空间 $\mathcal{H}_i$ 上以后, $A^{(1)}$ 仍然是幺正算符。这不仅对单体算符成立,对于两体算符和 $n$ 体算符这都是成立的。

在我们考察单体算符的形式定义之前,我们先来看一个最简单的例子:\textbf{粒子数算符}。
\subsection{粒子数算符}
\subsection{单体算符的形式定义}