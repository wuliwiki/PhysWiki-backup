% 单源最短路径
% 单源最短路径|最短路|算法|C++

单元最短路径问题,为给定一张有向图 $G = (V, E)$,$V$ 是点集,$E$ 是边集,$|V|= n$,$|E| = m$,求给定的源点(起点)$s \in V$ 到每个结点 $v \in V$ 的最短路径.$(x, y, w)$ 表示有一条从结点 $x$ 指向结点 $y$ 的有向边,边权为 $w$.

\subsection{Dijkstra 算法}

\verb|dist[i]| 表示从起点 $s$ 到结点 $i$ 的\textbf{实际}最短路径的长度(这条路径的权值之和).

$\delta(u)$ 表示从起点 $s$ 到结点 $u$ 的估计最短路径长度.任意时刻都存在 \verb|dist[u]| $\geq \delta(u)$.

Dijkstra 算法是一种求解没有负权边的图中的单源最短路问题.将所有结点划分为两个集合,$S$ 集合存储当前已经确定了最短路的结点,$T$ 集合存储当前还未确定最短路的结点.

具体做法是:
\begin{enumerate}
\item 初始化所有点的 \verb|dist| 距离为正无穷,起点的距离为 $0$(\verb|dist[S] = 0|).
\item 每次从 $T$ 集合中选出一条 \verb|dist| 值最小的结点 $t$,并把 $t$ 结点加入 $S$ 集合中.
\item 用 $t$ 更新其他结点.
\item 重复 $2 \sim 3$ 步骤,直到所有点都被加入 $S$ 集合.
\end{enumerate}

朴素 Dijkstra 算法的时间复杂度为 $\mathcal{O}(n^2)$,使用二叉堆可以使操作 $2$ 的时间复杂度从 $\mathcal{O}(n^2)$ 的时间复杂度优化到 $\mathcal{O}(\log_2 n)$.每更新一条边 $(x, y)$,就把 $y$ 这个结点和 \verb|dist[y]| 值插入到二叉堆中.每次找最小值直接取堆顶即可.每次取堆顶时判断堆顶是不是已经被访问过了,如果被访问过了,直接忽略这次操作,否则会重复更新,导致影响时间复杂度.所以堆优化版 Dijkstra 的时间复杂度为 $\mathcal{O}(m \log_2 n)$.

\textbf{Dijkstra 算法准确性证明:}
