% 磁场的高斯定律
% keys 磁场|磁感应强度|散度|高斯定律
% license Xiao
% type Tutor

\begin{issues}
\issueDraft
\end{issues}

\pentry{电场的高斯定律\nref{nod_EGauss}}{nod_8164}

\footnote{本文参考 Wikipedia \href{https://en.wikipedia.org/wiki/Gauss's_law_for_magnetism}{相关页面}。}\textbf{磁场的高斯定律(Gauss's law for magnetism)}是麦克斯韦方程组\upref{MWEq}中的一条方程,它描述了空间中磁场的散度为零这一性质,也就是说如果用磁感线的疏密表达磁场的强度,那么空间中任意一条磁感线都不会有起点和终点(这与电场不同,从一个正电荷出发可以有多条电场线延伸至远处)。

\subsection{磁场高斯定律}
磁场的高斯定律: 对于任意磁场 $\bvec B(\bvec r)$ 和任意闭合曲面, 曲面上的磁通量\upref{BFlux}为零。
\begin{equation}
\oint \bvec B(\bvec r) \vdot \dd{\bvec s} = 0~,
\end{equation}
从电场线的角度看,这意味着如果统计穿过任意的闭合曲面的磁感线数量,那么从内部穿到外面和从外面穿到内部的磁感线是相等的。

也就是说空间任意一点的磁场散度为零。 适用高斯定理\upref{Divgnc}可以写成微分形式:
\begin{equation}
\div\bvec B = 0~.
\end{equation}
上式表达的意思就是磁场的散度为零。如果将空间中某一点的磁场矢量看作是流体在该点的流速,那么磁场是无源无汇的。

接下来我们试着验证一下这一结论是否和毕奥—萨伐尔定律是一致的,也就是说我们能否直接从毕奥—萨伐尔定律\upref{BioSav}所给出的磁场 $\bvec B(r)$ 的表达式推出磁场散度为零的共识。首先我们考虑静磁场下,电流是恒定的,因此电流密度 $\bvec j$ 不会在某一个点聚集或者散开,根据电流守恒方程 $\partial \rho/\partial t + \bvec \nabla\cdot \bvec j=0$ 因此有:
\begin{equation}
\div \bvec j = 0  ~.
\end{equation}
结合毕奥—萨伐尔:
\begin{equation}
\bvec B(\bvec r) = \frac{\mu_0}{4\pi} \int \frac{\bvec j(\bvec r') \cross (\bvec r - \bvec r')}{\abs{\bvec r - \bvec r'}^3} \dd{V'}~.
\end{equation}
利用矢量乘法的规则可得:
\begin{equation}
\div(\bvec j \cross \frac{(\bvec r - \bvec r')}{\abs{\bvec r - \bvec r'}^3})=\frac{(\bvec r - \bvec r')}{\abs{\bvec r - \bvec r'}^3}\cdot(\curl \bvec j)-\bvec j\cdot(\curl \frac{(\bvec r - \bvec r')}{\abs{\bvec r - \bvec r'}^3})~.
\end{equation}
由于 $\curl \frac{(\bvec r - \bvec r')}{\abs{\bvec r - \bvec r'}^3} = 0$:
\begin{equation}
\div \bvec B = 0  ~.
\end{equation}

注意磁场高斯定律适用于经典电动力学的任何情况, 而毕奥—萨伐尔定律只适用于静态(电流密度不随时间发生变化)的情况。

磁场的高斯定律实际上是电场的高斯定律\upref{EGauss}在磁学中的对应,它反映了自然界没有孤立的磁单极(或者我们还没找到)。形象地看,任意一条磁感线都不会起始或终止于空间中的某一点,它要么是闭合的回路,要么从无穷远来延伸到无穷远去。正因为磁场的这条性质,我们可以将磁感应强度 $\bvec B$ 写成某个矢量场 $\bvec A$ 的旋度,其中 $\bvec A$ 称为矢量势(矢势)\upref{BvecA}。
