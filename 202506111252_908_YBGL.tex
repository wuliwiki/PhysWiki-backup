% 约翰·彼得·古斯塔夫·勒热纳·狄利克雷(综述)
% license CCBYSA3
% type Wiki

本文根据 CC-BY-SA 协议转载翻译自维基百科 \href{https://en.wikipedia.org/wiki/Peter_Gustav_Lejeune_Dirichlet}{相关文章}。

约翰·彼得·古斯塔夫·勒让·狄利克雷(Johann Peter Gustav Lejeune Dirichlet,/ˌdɪərɪˈkleɪ/\(^\text{[1]}\),[德语发音:[ləˈʒœn diʁiˈkleː]\(^\text{[2]}\);1805年2月13日-1859年5月5日)是德国数学家。在数论中,他证明了费马大定理的一些特殊情形,并创立了解析数论。在分析学中,他推进了傅里叶级数理论的发展,并且是最早给出函数现代形式定义的数学家之一。在数学物理中,他研究了势理论、边值问题、热扩散和流体力学。

尽管他的姓氏是“勒让·狄利克雷”,但在引用以他命名的成果时,人们通常只使用“狄利克雷”这一名字。
\subsection{生平}
\subsubsection{早年生活(1805–1822)}
古斯塔夫·勒让·狄利克雷于1805年2月13日出生在迪伦(Düren),这是位于莱茵河左岸的一座小镇,当时属于法兰西第一帝国,1815年维也纳会议后归属普鲁士。他的父亲约翰·阿诺德·勒让·狄利克雷(Johann Arnold Lejeune Dirichlet)是一名邮政局长、商人及市议会议员。他的祖父则从比利时列日东北约5公里处的小村庄里舍莱特(Richelette,或更可能是 Richelle [fr])迁居至迪伦,因此家族姓氏“勒让·狄利克雷”(Lejeune Dirichlet,法语意为“来自里舍莱特的年轻人”)由此而来。\(^\text{[3]}\)

尽管家境并不富裕,而且狄利克雷在七个孩子中排行最小,父母仍然支持他的教育。他们先让他就读于一所小学,之后又转入私立学校,希望他日后能成为一名商人。然而年幼的狄利克雷在12岁之前就表现出了对数学的强烈兴趣,最终说服父母让他继续深造。1817年,他被送往波恩文理中学(Gymnasium Bonn [de])就读,受到家庭熟识的彼得·约瑟夫·埃尔韦尼希(Peter Joseph Elvenich,一位学生)的照顾。1820年,狄利克雷转入科隆耶稣会文理中学,在那里他在乔治·欧姆的指导下,进一步拓宽了数学知识。次年他离开了文理中学,仅获得了一张结业证书,因为他无法流利使用拉丁语,未能取得正式的中学毕业文凭。\(^\text{[3]}\)
\subsubsection{巴黎求学时期(1822–1826)}
狄利克雷再次说服了父母,为他的数学学习提供进一步的经济支持,尽管父母希望他从事法律事业。当时德国几乎没有学习高等数学的机会,唯一的例外是哥廷根大学的高斯,但高斯名义上是天文学教授,而且并不喜欢教学。因此,狄利克雷在1822年5月前往巴黎求学。在那里,他在法兰西公学院和巴黎大学听课,师从包括阿谢特在内的多位数学家,同时私下学习高斯的《算术研究》——这本书成为他终生随身携带的重要著作。
1823年,经人推荐,狄利克雷受雇于马克西米连·福伊将军,担任其子女的私人德语教师。借此工作所得的报酬,狄利克雷终于实现了经济独立,不再需要依赖父母的资助。\(^\text{[4]}\)

他的第一项原创研究是对费马大定理在 n= 5 情况下的部分证明,这项成果使他立刻声名鹊起——这是自费马本人完成 n= 4 情况的证明、以及欧拉完成 n= 3 情况的证明之后,该定理上的首次新进展。当时的审稿人之一勒让德很快完成了对 n= 5 情况的完整证明;不久后,狄利克雷也独立完成了自己的证明,并在几年后完成了对 n= 14 情况的完整证明。\(^\text{[5]}\)1825年6月,他获准在法国科学院就 n= 5 情况的部分证明进行讲演——这对于一位年仅20岁、尚无学位的学生来说,是极其罕见的成就。\(^\text{[3]}\)这次在科学院的讲座也使狄利克雷与傅里叶和泊松建立了联系,激发了他对理论物理学的兴趣,尤其是傅里叶的热传导解析理论。
\subsubsection{返回普鲁士,布雷斯劳时期(1825–1828)}
1825年11月,福伊将军去世,狄利克雷在法国也未能找到有报酬的职位,不得不返回普鲁士。傅里叶和泊松将他推荐给了亚历山大·冯·洪堡。当时洪堡受召加入普鲁士国王腓特烈·威廉三世的宫廷,正计划将柏林打造成科学研究中心。他立即向狄利克雷伸出援手,分别向普鲁士政府和普鲁士科学院写信推荐。同时,洪堡还为他争取到了高斯的推荐信。高斯在阅读了狄利克雷关于费马定理的论文后,给予了极高评价,称“狄利克雷展现出极其出色的天赋”。\(^\text{[6]}\)

在洪堡和高斯的支持下,狄利克雷获得了布雷斯劳大学的教职。不过,由于他尚未完成博士论文,便将自己关于费马定理的论文提交给波恩大学作为博士论文。然而,他缺乏拉丁语流利表达能力,无法进行论文要求的拉丁语公开答辩。经过多方讨论,波恩大学最终决定破例,于1827年2月授予他名誉博士学位。同时,教育部长也特批他豁免了进行博士资格答辩时所要求的拉丁语答辩。狄利克雷顺利获得博士资格认证,并于1827–1828学年在布雷斯劳以私人讲师身份授课。\(^\text{[3]}\)

在布雷斯劳任教期间,狄利克雷继续从事数论研究,尤其是在双二次互反律方面做出了重要贡献——当时这是高斯研究的核心课题之一。洪堡借助狄利克雷这一系列的新成果(此时他也得到了天文学家贝塞尔的热烈赞誉),成功为他争取到转往柏林任教的机会。由于狄利克雷当时年纪尚轻(仅23岁),洪堡只为他在普鲁士军事学院争取到一个试用职位,同时名义上仍挂靠布雷斯劳大学。这一试用期被延长了三年,直到1831年该职位最终转为正式编制。
\subsubsection{与蕾贝卡·门德尔松的婚姻}
\begin{figure}[ht]
\centering
\includegraphics[width=6cm]{./figures/bc15bcc214ed1c54.png}
\caption{狄利克雷于1832年与蕾贝卡·门德尔松结婚。他们育有两个孩子:瓦尔特(生于1833年)和弗洛拉(生于1845年)。威廉·亨塞尔于1823年所绘画作。} \label{fig_YBGL_1}
\end{figure}
狄利克雷搬到柏林后,洪堡将他介绍进了银行家亚伯拉罕·门德尔松·巴托尔迪(Abraham Mendelssohn Bartholdy)及其家族主办的著名沙龙。这座宅邸是柏林艺术家与科学家每周的聚会场所,其中包括亚伯拉罕的子女费利克斯·门德尔松(Felix Mendelssohn)与范妮·门德尔松(Fanny Mendelssohn)——两位杰出的音乐家——以及范妮的丈夫、画家威廉·亨塞尔(Wilhelm Hensel)。狄利克雷对亚伯拉罕的女儿蕾贝卡表现出浓厚兴趣,并于1832年与她结婚。

蕾贝卡·亨丽埃特·勒让·狄利克雷(婚前名:蕾贝卡·门德尔松,1811年4月11日—1858年12月1日)是摩西·门德尔松的孙女,同时是费利克斯·门德尔松与范妮·门德尔松最小的妹妹。\(^\text{[7][8]}\) 蕾贝卡出生于汉堡。\(^\text{[9]}\)1816年,她的父母为她安排了洗礼,自此起她改名为蕾贝卡·亨丽埃特·门德尔松·巴托尔迪。\(^\text{[10]}\)她活跃于父母亚伯拉罕与莉娅·门德尔松的著名沙龙中,在德国思想界的高度活跃时期,与众多重要音乐家、艺术家和科学家交往甚密。1829年,她在门德尔松家举办的费利克斯·门德尔松轻歌剧《异乡归来》首演中饰演了一个小角色。她后来曾写道:

“我的哥哥姐姐们偷走了我作为艺术家的名声。在任何其他家庭,我都会被视为一名出色的音乐家,甚至可能领导一个乐团。而在费利克斯与范妮面前,我无法获得任何认可。”\(^\text{[11]}\)

1832年,她与狄利克雷结婚,狄利克雷正是通过洪堡介绍认识了门德尔松家族。\(^\text{[12]}\) 次年(1833年),他们的长子瓦尔特出生。1858年,她在哥廷根去世。
\subsubsection{柏林时期(1826–1855)}
狄利克雷一到柏林便申请在柏林大学授课,教育部长批准了他的调动,并于1831年将他任命到哲学系任教。根据规定,学院要求他重新完成一次资格认证。虽然狄利克雷按要求撰写了资格论文,但他将必须用拉丁语进行的公开讲座一拖就是20年,直到1851年才完成。由于这一正式要求长期未完成,他在学院的职位一直不具备完全的职权,薪资也受到限制,被迫继续兼任军事学院的教职。1832年,年仅27岁的狄利克雷成为普鲁士科学院的成员,是当时最年轻的院士。\(^\text{[3]}\)

狄利克雷以讲解清晰而在学生中有很好的声誉,并且热爱教学,尤其是因为他在大学中讲授的往往是与他研究密切相关的高级课程:数论(他是德国第一位开设数论课程的教授)、分析学和数理物理学。他指导了多位重要德国数学家的博士论文,包括戈特霍尔德·艾森斯坦、利奥波德·克罗内克、鲁道夫·利普希茨和卡尔·威廉·博尔哈特,同时在许多其他科学家的数学教育中也发挥了重要影响,如克里斯托费尔、威廉·韦伯、海涅、冯·塞德尔和温加藤等。在军事学院,狄利克雷设法将微积分纳入课程体系,提升了该校的科学教育水平。然而,他也逐渐感到,在军事学院和大学的双重教学负担,正在限制他投入科研的时间。\(^\text{[3]}\)

在柏林期间,狄利克雷与其他数学家保持联系。1829年,在一次旅行中,他遇到了卡尔·雅可比,当时雅可比是哥尼斯堡大学的数学教授。几年间,他们持续会面并就研究问题进行通信,最终成为了亲密的朋友。1839年,在一次访问巴黎时,狄利克雷遇到了约瑟夫·刘维尔,两位数学家很快成为朋友,保持联系,并在几年后互访并带着家人一起聚会。1839年,雅可比将厄恩斯特·库默的论文寄给狄利克雷,库默当时是一位学校教师。狄利克雷意识到库默的潜力,便帮助他成功当选为柏林科学院院士,并在1842年为他争取到布雷斯劳大学的全职教授职位。1840年,库默娶了奥蒂莉·门德尔松,她是狄利克雷妻子瑞贝卡的堂姐。

1843年,当雅可比生病时,狄利克雷前往哥尼斯堡帮助他,随后还为他争取到了普鲁士国王腓特烈·威廉四世的私人医生的帮助。当医生建议雅可比去意大利休养时,狄利克雷与家人一同陪同雅可比前往。路上,他们由路德维希·施拉夫利陪同,施拉夫利作为翻译同时也对数学产生了浓厚兴趣,狄利克雷和雅可比在旅途中给他讲授数学,施拉夫利后来也成为了重要的数学家。\(^\text{[3]}\)狄利克雷一家将意大利的逗留延长至1845年,他们的女儿弗洛拉就是在意大利出生的。1844年,雅可比作为皇室退休人员搬到柏林,他们的友谊变得更加亲密。1846年,当海德堡大学试图招募狄利克雷时,雅可比向洪堡提供了必要的支持,使狄利克雷的薪水翻倍,从而使他留在柏林;然而,即使这样,狄利克雷依然没有拿到全职教授的薪水,且无法离开军事学院。\(^\text{[13]}\)

持有自由主义观点的狄利克雷与家人支持1848年的革命;他甚至用步枪守卫普鲁士王子的宫殿。在革命失败后,军事学院暂时关闭,导致他失去了大量收入。学院重开后,环境变得对他更加敌对,因为他所教授的军官被要求忠诚于已成立的政府。一些没有支持革命的媒体把他与雅可比以及其他自由派教授列为“红色阵营的成员”。\(^\text{[3]}\)

1849年,狄利克雷与他的朋友雅可比一同参与了高斯博士学位的纪念活动。
\subsubsection{哥廷根时期(1855–1859)}
尽管狄利克雷在学术上的造诣深厚,也获得了诸多荣誉,而且到了1851年,他终于完成了全职教授所需的所有正式要求,但提高他在大学的薪资问题仍然被一拖再拖,他依旧无法脱离军事学院的工作。1855年,高斯去世后,哥廷根大学决定邀请狄利克雷接任其职位。鉴于他在柏林所面临的诸多困难,狄利克雷决定接受邀请,并立即携家人迁往哥廷根。库默(Kummer)则被任命接替他在柏林的数学教授职位。\(^\text{[4]}\)

狄利克雷非常享受在哥廷根的时光,较轻的教学任务使他有更多时间投入科研工作,并与新一代研究者建立了密切联系,尤其是理查德·戴德金和伯恩哈德·黎曼。他到哥廷根后,还设法为黎曼争取到一份小额年薪,使其能留在教师队伍中。尽管戴德金、黎曼、莫里茨·康托尔(Moritz Cantor)和阿尔弗雷德·恩内珀(Alfred Enneper)都已获得博士学位,但他们仍前来听狄利克雷授课,向他学习。戴德金自认为在数学教育上存在不足,称能跟随狄利克雷学习使他“仿佛重获新生”。\(^\text{[3]}\)他后来整理并出版了狄利克雷关于数论的讲义与其他成果,命名为《数论讲义》。

1858年夏天,在蒙特勒旅行期间,狄利克雷突发心脏病。1859年5月5日,他在哥廷根去世,距其妻瑞贝卡去世仅数月之隔。\(^\text{[4]}\)狄利克雷的大脑如今保存在哥廷根大学生理学系,和高斯的大脑一起存放。。柏林科学院在1860年为他举行了正式的追悼会,由库默致悼词,随后委托克罗内克与拉撒路·福克斯编辑并出版了他的全集。
