% 约翰·麦卡锡(综述)
% license CCBYSA3
% type Wiki

本文根据 CC-BY-SA 协议转载翻译自维基百科\href{https://en.wikipedia.org/wiki/Maxwell\%27s_equations}{相关文章}。

\begin{figure}[ht]
\centering
\includegraphics[width=6cm]{./figures/b71f34967eb60ef0.png}
\caption{麦卡锡在2006年的一次会议上} \label{fig_YHMKX_1}
\end{figure}
约翰·麦卡锡(John McCarthy,1927年9月4日-2011年10月24日)是一位美国计算机科学家和认知科学家。他是人工智能学科的创始人之一。[1] 他共同撰写了提出“人工智能”(Artificial Intelligence, AI)这一术语的文献,开发了编程语言家族Lisp,深刻影响了ALGOL语言的设计,普及了分时共享技术,并发明了垃圾回收机制。

麦卡锡的大部分职业生涯都在斯坦福大学度过。[2] 他因对人工智能领域的贡献获得了诸多奖项和荣誉,例如1971年的图灵奖、美国国家科学奖章和京都奖。
\subsection{早年生活与教育}
约翰·麦卡锡于1927年9月4日出生在马萨诸塞州波士顿,他的父亲是一位爱尔兰移民,母亲是立陶宛犹太移民。[4] 他的父母分别是约翰·帕特里克·麦卡锡(John Patrick McCarthy)和艾达·格拉特·麦卡锡(Ida Glatt McCarthy)。大萧条期间,家庭多次搬迁,直到父亲在加利福尼亚州洛杉矶找到一份为服装工会(Amalgamated Clothing Workers)担任组织者的工作。他的父亲来自爱尔兰凯里郡的一个小渔村克罗曼(Cromane)。[5] 他的母亲于1957年去世。[6]

麦卡锡的父母在1930年代是共产党的积极成员,他们鼓励孩子学习和批判性思考。在进入高中之前,麦卡锡通过阅读一本名为《十万个为什么》(*100,000 Whys*)的俄文儿童科普书籍,对科学产生了兴趣。[7] 他精通俄语,并在多次访问苏联期间结交了俄罗斯科学家,但在访问东欧集团后与其拉开了距离,并成为一名保守派共和党人。[8]

麦卡锡提早两年从贝尔蒙特高中毕业,[9] 并于1944年被加州理工学院(Caltech)录取。

他展现出数学的早期天赋;在青少年时期,他通过自学加州理工学院使用的大学数学教材掌握了大学数学知识。因此,他进入加州理工学院后得以跳过前两年的数学课程。[10] 他因未参加体育课程而被加州理工学院停学,[11] 后来服役于美国陆军,之后重新被录取,并于1948年获得数学学士学位(BS)。[12]

麦卡锡在加州理工学院听过约翰·冯·诺依曼的讲座,这对他的未来事业产生了重要启发。

麦卡锡在加州理工完成了研究生学习后,前往普林斯顿大学,并于1951年在唐纳德·C·斯宾塞(Donald C. Spencer)的指导下完成了题为《投影算子与偏微分方程》(*Projection operators and partial differential equations*)的博士论文,获得了数学博士学位。[13]
\subsection{学术生涯}

在普林斯顿大学和斯坦福大学短期任职后,麦卡锡于1955年成为达特茅斯学院的助理教授。

一年后,他于1956年秋天转至麻省理工学院(MIT)担任研究员。在麻省理工学院的最后几年,他已经被学生们亲切地称为“约翰叔叔”。[14]

1962年,麦卡锡成为斯坦福大学的正教授,并一直任职至2000年退休。

麦卡锡倡导数学方法,如λ演算,并为人工智能中的常识推理设计了逻辑系统。
\subsection{计算机科学中的贡献}
\begin{figure}[ht]
\centering
\includegraphics[width=6cm]{./figures/3046a488575648f4.png}
\caption{2008年的麦卡锡} \label{fig_YHMKX_2}
\end{figure}
约翰·麦卡锡是人工智能的“创始之父”之一,与艾伦·图灵、马文·明斯基、艾伦·纽厄尔和赫伯特·西蒙齐名。麦卡锡与明斯基、纳撒尼尔·罗切斯特和克劳德·香农在1956年夏天为著名的达特茅斯会议撰写了一份提案,首次提出了“人工智能”这一术语。这次会议标志着人工智能作为一个独立领域的起点。[9][15](明斯基后来于1959年加入麦卡锡在麻省理工学院的团队。)

1958年,他提出了建议接收器(advice taker),这一概念启发了后来关于问答系统和逻辑编程的研究。

在20世纪50年代末,麦卡锡发现原始递归函数可以扩展为对符号表达式进行计算,由此发明了Lisp编程语言。[16] 他发表的关于函数式编程的开创性论文引入了从λ演算语法中借鉴而来的λ符号,这成为后来的编程语言(如Scheme)语义的基础。Lisp在1960年发表后,很快成为人工智能应用的首选编程语言。

1958年,麦卡锡参与了美国计算机协会(ACM)的一个临时语言委员会,该委员会后来参与了ALGOL 60的设计。1959年8月,他提出了递归和条件表达式的使用,这成为ALGOL的一部分。[17] 随后,他加入了国际信息处理联合会(IFIP)的算法语言和演算工作组(Working Group 2.1),参与ALGOL 60和ALGOL 68的标准制定和维护。[19]

大约在1959年,他发明了所谓的垃圾回收(garbage collection)方法,这是一种自动内存管理技术,用于解决Lisp中的内存问题。[20][21]

在麻省理工学院任职期间,他推动了MAC项目的创建;在斯坦福大学工作时,他帮助建立了斯坦福人工智能实验室,该实验室多年来一直是MAC项目的友好竞争对手。

麦卡锡在开发早期三大分时系统(兼容分时系统、BBN分时系统和达特茅斯分时系统)方面发挥了重要作用。他的同事莱斯特·厄内斯特对《洛杉矶时报》说:

“如果不是因为约翰启动了分时系统的发展,互联网可能不会如此迅速地出现。我们不断为分时系统发明新名字,比如服务器……现在我们称之为云计算,但它仍然只是分时系统的延续。约翰是这一切的开端。”[9]  

——艾琳·吴(Elaine Woo)

1961年,他或许是第一个公开提出实用计算(utility computing)概念的人。在麻省理工学院百年校庆的演讲中,他设想分时技术可能带来一种未来,计算能力甚至具体应用程序可以通过像水或电一样的公共事业模式出售。[22][23] 这种“计算机公用事业”的想法在20世纪60年代末非常流行,但在90年代中期逐渐淡出。然而,自2000年以来,这一概念以新的形式重新出现(如应用服务提供商、网格计算和云计算)。

1966年,麦卡锡和他的斯坦福团队编写了一款计算机程序,与苏联团队进行了一系列国际象棋比赛;麦卡锡团队输掉了两局,平局两局(见*科托克-麦卡锡比赛)。

从1978年至1986年,麦卡锡开发了非单调推理的圈定法(circumscription method)。

1982年,他似乎首次提出了太空喷泉(space fountain)的概念。这是一种延伸至太空的塔,通过由地球发射的颗粒流产生的向外力保持垂直。货物可以通过类似传送带的装置沿颗粒流向上运输。[24]