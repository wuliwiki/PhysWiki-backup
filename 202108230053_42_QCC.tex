% 量类的延拓
% keys 量类|延拓|矢量空间
\pentry{量类和单位\upref{QCU}}
在物理中,不少量类(这里指用单位测量量类中的量所得的数值)在通常情况下或不能取负值,或不连续,或只在某一范围取值.比如质量、面积、体积、密度、电容等不会取负值;电荷取值具有量子化性质;引力常量 $\boldsymbol{G}$ 只有一个等等.为建立一套完整的量纲理论,我们需要对量类进行延拓.在此之前,我们先给出一些理由,以便理解.
\subsection{量类延拓的理由}
某些量类对于一些数值是“没有物理意义”的,比如质量非负,然而,“没有物理意义”的说法其实是非常含混的.比如对速率量类,在相对论里其数值不能超过光速,但若限于牛顿力学,任何速率都是允许的.再如温度,我们知道绝对零度不可达到,但总应将绝对零度视作温度量类 $\tilde{\boldsymbol{T}}$ 的元素,否则“绝对零度不可达到”将意义不明.绝对零度下的温度虽然“没有物理意义”,但不妨纳入温度量类 $\tilde{\boldsymbol{T}}$ 这个集合中,只需将其理解成在具体的问题中不出现而已.现在,“温度量类 $\tilde{\boldsymbol{T}}$” 指由开尔文(作为单位)的任意实数倍组成的集合.同样,对于量子化的量类,只需将其取值的性质改为可以连续,那些“没有物理意义”的值只需理解为在具体问题中不会出现或观察不到.更如,对于引力常量量类 $\tilde{\boldsymbol{G}}$ ,人们过去一直认为其是常量,但现在,越来越多的人相信,在宇宙演化的历史长河中,引力常量 $\boldsymbol{G}$ 是在非常缓慢地改变着的(由狄拉克于1937年率先提出).这一“常量不常”的现象对其它若干物理常量也适用.

当然,对量类进行延拓的最直接理由是建立其对应的数学结构,以便进行各种运算.
\subsection{量类的最大延拓}
\begin{definition}{}
量类 $\tilde{\boldsymbol{Q}}$ 的最大延拓是指:以$\tilde{\boldsymbol{Q}}$ 的任一单位测量延拓后的量类的所有元素,所得实数取遍整个实数集 $\mathbb{R}$.
\end{definition}
今后不加说明,“量类”一词(及其符号$\tilde{\boldsymbol{Q}}$)都指最大延拓的集合,它包含反映正、负状态的两个子集.
\begin{definition}{}
设 $\tilde{\boldsymbol{Q}}$ 是最大延拓的量类,则
\begin{enumerate}
\item $\tilde{\boldsymbol{Q}}_{\text{正}}$ 是$\tilde{\boldsymbol{Q}}$ 的子集,以任一单位测 $\tilde{\boldsymbol{Q}}_{\text{正}}$ 的所有元素的得数能取遍开区间 $(0,\infty)$;
\item $\tilde{\boldsymbol{Q}}_{\text{负}}$ 是$\tilde{\boldsymbol{Q}}$ 的子集,以任一单位测 $\tilde{\boldsymbol{Q}}_{\text{负}}$ 的所有元素的得数能取遍开区间 $(-\infty,0)$
\end{enumerate}
称 $\tilde{\boldsymbol{Q}}_{\text{正}}$ 和$\tilde{\boldsymbol{Q}}_{\text{负}}$ 分别为量类 $\tilde{\boldsymbol{Q}}$ 的\textbf{正半轴} 和 \textbf{负半轴}.
\end{definition}
\subsection{量类是1维矢量空间}
