% 反射
% 反射(什么是反射)(反射定律)|各种反射镜(平面镜、球面镜、非球面镜)
% 前置知识:惠更斯原理
\begin{issues}
\issueDraft
\issueMissDepend
\end{issues}

\subsection{反射}
反射指的是波阵面从一个介质进入到另一个介质之中时,在界面传播方向发生改变,并返回到原来介质的现象。与透镜相比,反射镜成像系统可以在宽频内工作而不产生色差,特别是紫外区和红外区。

过去,在对于光的反射中,常用在玻璃后面镀银来实现反射镜的制作,这称之为第二表面镜。这时候,反射面位于银上,此时前方的玻璃用来保护银面不受损害。在对于反射波阵面更精确的调控中,则将具有高反射率的膜镀于基片前面,以减少反射损失并提高反射像的质量。近来,在高抛光的衬底上真空蒸发镀铝已经成为高质量反射镜的公认标准工艺。
%反射定律
\begin{figure}[ht]
\centering
\includegraphics[width=5cm]{./figures/8b0c7d011c8855db.png}
\caption{请添加图片描述} \label{fig_RefMir_2}
\end{figure}

除了光波以外,反射现象在对于声波、水波等多种波中,也发挥了重要的作用。
\subsection{多种反射镜}

%利用反射定律来看各种镜子