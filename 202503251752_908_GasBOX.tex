% 盒中气体(综述)
% license CCBYSA3
% type Wiki

本文根据 CC-BY-SA 协议转载翻译自维基百科\href{https://en.wikipedia.org/wiki/Gas_in_a_box}{相关文章}。

在量子力学中,量子粒子在盒子中的结果可以用来研究量子理想气体在盒子中的平衡情况,这个盒子包含了大量的分子,这些分子除了瞬时热化碰撞外彼此不相互作用。这个简单的模型不仅可以用来描述经典理想气体,还可以用来描述各种量子理想气体,例如理想的重费米气体、理想的重玻色气体以及黑体辐射(光子气体),光子气体可以视为无质量的玻色气体,其中通常假设热化是通过光子与平衡质量之间的相互作用来促进的。

利用麦克斯韦-玻尔兹曼统计、玻色-爱因斯坦统计或费米-狄拉克统计的结果,并考虑非常大的盒子的极限,使用托马斯-费米近似(以恩里科·费米和刘埃文·托马斯命名)来表示能级的简并度为微分,并将态的求和转化为积分。这使得通过分配函数或大分配函数可以计算气体的热力学性质。这些结果将应用于有质量和无质量的粒子。更完整的计算将留待单独的文章中,但在本文中将给出一些简单的例子。
\subsection{托马斯-费米近似用于态的简并度}  
对于盒子中的有质量和无质量粒子,粒子的态由一组量子数 [\(n_x, n_y, n_z\)] 枚举。动量的大小由以下公式给出:
\[
p = \frac{h}{2L} \sqrt{n_x^2 + n_y^2 + n_z^2} \quad \quad n_x, n_y, n_z = 1, 2, 3, \ldots~
\]
这里,\(h\)是普朗克常数,\(L\)是盒子的一边的长度。粒子的每一个可能的状态可以看作是一个三维网格上的点,该网格由正整数构成。原点到任何点的距离将是:
\[
n = \sqrt{n_x^2 + n_y^2 + n_z^2} = \frac{2Lp}{h}~
\]
假设每一组量子数指定f个状态,其中f是粒子的内部自由度的数量,可以通过碰撞改变。例如,旋转量子数为1/2的粒子有\(f = 2\),分别对应两个自旋状态。对于大的n值,从上述方程得出的动量大小小于或等于p的状态数大致为:
\[
g = \left( \frac{f}{8} \right) \frac{4}{3} \pi n^3 = \frac{4 \pi f}{3} \left( \frac{Lp}{h} \right)^3~
\]
这就是\(f\)乘以半径为\(n\)的球体的体积除以八,因为只考虑具有正\(n_i\)的八分之一区域。使用连续近似,动量大小介于\(p\)和\(p+dp\)之间的状态数为:
\[
dg = \frac{\pi}{2} f n^2 \, dn = \frac{4 \pi f V}{h^3} p^2 \, dp~
\]
其中\(V=L^3\)是盒子的体积。注意,在使用这个连续近似(也称为Thomas−Fermi近似)时,失去了对低能态的表征能力,包括自旋量子数\(n_i=1\)的基态。对于大多数情况,这不会是一个问题,但在考虑玻色−爱因斯坦凝聚时,气体中有很大一部分处于或接近基态,此时处理低能态的能力变得非常重要。