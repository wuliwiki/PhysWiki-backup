% 圆周运动的向心力
% keys 向心力|圆周运动

\pentry{圆周运动的加速度\upref{CMAD}, 牛顿运动定律\upref{New3}}

在惯性参考系中, 质点的运动符合牛顿第二定律 $\bvec F = m\bvec a$. 所以要使一个质点做半径为 $R$ 的圆周运动, 那么我们就可以把任意时刻圆周运动的加速度矢量(\autoref{CMAD_eq4}~\upref{CMAD})乘以质量 $m$ 得到它该时刻受到的合力. 我们把这个力叫做向心力
\begin{equation}
\bvec F_c = -m \omega^2 \bvec r
\end{equation}

\subsection{有趣的推导}
为了更好地理解为何向心力会产生圆周运动, 我们可以假设一个小球在运动的过程中被反弹若干次, 使得它的轨道称为闭合的正 $N$ 边形. 每次撞击
