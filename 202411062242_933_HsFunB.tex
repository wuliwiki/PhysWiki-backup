% 函数回顾(高中)
% keys 初中|函数|正比例|反比例|二次
% license Xiao
% type Tutor

\begin{issues}
\issueDraft
\end{issues}

在初中阶段,函数的概念初步展示了变量之间的关系。初中接触的函数主要包括正比例函数、反比例函数和一次函数、二次函数。接下来会在介绍实数和坐标系的概念后,逐一回顾每种函数的特性和相关概念。

\subsection{实数与坐标系}

\textbf{实数(real number)}包含了常见的整数、分数、小数和无理数(如π和√2)。它们既可以是正数,也可以是负数,还包括零。实数的意义在于它们可以代表连续的量,帮助精确地描述事物的大小、位置等属性。无论是温度、距离,还是速度,实数都为这些变量的表达提供了基础。

下面会从运算的视角来重新审视学过的数字。

最简单也最熟悉的数字,莫过于\textbf{自然数(natural number)}。自然数是每个人开对数字最开始的概念,自然数指的是从$0$开始\footnote{也有领域认为数字从$1$开始。}一个接一个地排列的数。这个排列的过程,也称作\textbf{递增(increment)}或\textbf{后继(successor)}。在递增的基础上,人们抽象出了加法运算,又从加法抽象出了乘法运算。而自然数对于加法和乘法都是\textbf{封闭的(closed)},也就是说任意两个自然数的和或积都是自然数。

随着对数字的进一步需求,会遇到表示“少于零”的情况,比如温度计上的零度以下温度,银行账户的负债等。这就引入了负数的概念。此时,减法也可视作与后者相反数的和。\textbf{整数(integer)}包括所有自然数和负数,也使得它不仅对加法和乘法封闭,对减法也封闭。这是一次数的扩充。

整数虽然很有用,但当需要表示更精确的数量,比如半个苹果或三分之一米时,仅用整数就不够了。这时,引入了分数。此时,除法也可视作与后者倒数的积。当然,定义了$0$没有倒数\footnote{$0$有倒数也不是不可以,但是引入之后造成的麻烦会很大,因此,数学领域放弃了这个设定。}。\textbf{有理数(rational number)}\footnote{有理数的翻译本身是you}包括所有整数和分数。而这时,有理数也可以统一表示成两个整数之比(如  \frac{a}{b} ,其中 b \neq 0 )。

4. 无理数的出现

仔细观察会发现,有理数虽然可以表示很多分数,但仍然有些数量无法用它们表示。比如,正方形对角线的长度和圆的直径与周长之比,都无法用分数精确地表达。这样的数字称为“无理数”。无理数和有理数不同,不能写成两个整数的比,而是无限不循环小数。经典的例子有 \sqrt{2} 、 \pi 、 e 等。

5. 实数:有理数与无理数的融合

将有理数与无理数集合到一起,形成了一个更广泛的数字集合,称为“实数”。实数的范围包括所有可以在数轴上找到的位置,既包括分数、整数,也包括像 \sqrt{2} 、 \pi 这样不可分割的数字。



直线上的点与实数一一对应,称作数轴。

坐标系的引入:在数学中,理解空间位置很重要。平面直角坐标系就是在二维空间中定位的工具。这个系统包含两条互相垂直的数轴,横轴叫做x轴,竖轴叫做y轴。通过 x 和 y 的组合,可以确定平面上任意一点的位置。这个坐标系统不仅仅是位置的标记,更是表示函数图像的基础。坐标系使得计算结果之外,还能够“看见”数字之间的关系。
象限
正方向
原点

\subsection{正比例函数}

\textbf{正比例函数(proportional function)}

对应的是一条过原点的直线,

斜率

正比例函数是一种最简单的线性关系。

定义与基本形式:正比例函数的表达式为  y = kx ,其中 k 是一个常数,称它为“斜率”。它表示的是每单位的 x 增加,会带来 y 的变化。当 k 为正时,y 随着 x 增加而增加;当 k 为负时,y 随着 x 增加而减少。

斜率的作用:斜率 k 决定了这条直线的倾斜程度。k 越大,直线越陡峭;k 越小,直线越平缓。这就好比在山坡上行走,坡度越大,行走的难度也越大。

图像特点:正比例函数的图像是一条经过原点的直线。这是因为当 x 为 0 时,y 也必然为 0,所以图像一定会穿过原点。无论 k 的值如何变化,正比例函数的图像总是一条直线,代表着一种直接、简单的关系。

\subsection{反比例函数}

、\textbf{反比例函数(inversely proportional function)}
对应的是两条双曲线。你还学过如何根据反比例函数的表达式,通过已知的点来求解函数的值。

中心对称性

矩形面积相同

反比例函数描述了一种“此增彼减”的关系。

定义与表达式:反比例函数的标准形式为  y = \frac{k}{x} ,其中 k 是常数。当 x 增加时,y 减少;当 x 减少时,y 增加。这意味着 x 和 y 之间存在一种反向变化的关系。

图像特点:反比例函数的图像是一条双曲线,分别位于坐标轴的两侧,并且关于原点对称。这种对称性表明,x 值变号时,y 值也会相应地变号。

矩形面积不变性:反比例函数的一个特殊性质是,对于任意点 (x, y),x 和 y 的乘积总是等于常数 k。这类似于矩形的面积总是等于长乘以宽,即使长宽发生变化,只要乘积不变,面积就保持不变。

\subsection{一次函数}

截距、\textbf{一次函数(linear function)}
一次函数是最常见的线性关系。

定义与一般形式:一次函数的表达式是  y = mx + b ,其中 m 是斜率,决定了直线的倾斜度;b 是截距,表示直线与 y 轴的交点。这条直线不一定经过原点,b 决定了它在 y 轴上截取的位置。

图像特点:一次函数的图像是一条直线。m 控制了直线的倾斜程度,b 则决定了直线在 y 轴上的起点。因此,一次函数可以表示更灵活的线性关系。

\subsection{二次函数}

二次函数(quadratic function)的图像可能与  x  轴有两个交点,并且具有一个对称轴和一个最低点。

轴对称性:图像上对称点到对称轴的距离相等,且连线与对称轴垂直。

关于二次函数和一元二次方程的关系参见\enref{因式分解与一元二次方程}{quasol}。

二次函数描述了一种曲线关系,常用于表示自然界中的抛物运动。

定义与标准形式:二次函数的标准形式为  y = ax^2 + bx + c ,其中 a, b, c 是常数。a 的正负决定了抛物线的开口方向:当 a 为正,开口向上;当 a 为负,开口向下。

图像特点:二次函数的图像是一条抛物线,具有明显的对称性。抛物线的对称轴可以帮助找到对称的两个点以及最高点或最低点。其顶点是抛物线的最高点(当 a < 0)或最低点(当 a > 0)。

二次函数与一元二次方程的关系:二次函数图像与 x 轴的交点是方程的解,即求得的 x 值可以使得 y 为零。这在很多实际问题中帮助找到最优值或临界点。
