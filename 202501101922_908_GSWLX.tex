% 干涉 (物理学)(综述)
% license CCBYSA3
% type Wiki

本文根据 CC-BY-SA 协议转载翻译自维基百科\href{https://en.wikipedia.org/wiki/Wave_interference}{相关文章}。

对于无线电通信中的干扰,请参阅《干扰(通信)》。
“干涉图样”在此重定向。有关莫尔条纹,请参阅《莫尔条纹》。对于医学术语,请参阅《干涉图样(肌电图)》
\begin{figure}[ht]
\centering
\includegraphics[width=10cm]{./figures/1379a86de5dd6781.png}
\caption{两波的干涉。相位相同:两个较低的波组合(左侧面板), resulting in a wave of added amplitude(建设性干涉)。相位相反:(这里是180度),两个较低的波组合(右侧面板), resulting in a wave of zero amplitude(破坏性干涉)。} \label{fig_GSWLX_1}
\end{figure}
在物理学中,干涉是指两种相干波通过考虑它们的相位差,将它们的强度或位移相加的现象。 如果两波处于相位相同或相反的状态,所产生的波可能具有更大的强度(建设性干涉)或较小的振幅(破坏性干涉)。干涉效应可以在所有类型的波中观察到,例如光波、无线电波、声波、表面水波、引力波或物质波,以及扬声器中的电波。
\begin{figure}[ht]
\centering
\includegraphics[width=10cm]{./figures/f6e23b7411ed2318.png}
\caption{湖面上的干涉水波} \label{fig_GSWLX_2}
\end{figure}
\subsection{词源}  
“干涉”一词源自拉丁语单词 \textbf{inter},意思是“之间”,以及 \textbf{fere},意思是“撞击或打击”,该词在波的叠加上下文中由托马斯·杨于1801年首次使用。[1][2][3]
\subsection{机制}
\begin{figure}[ht]
\centering
\includegraphics[width=10cm]{./figures/17e646f8dccde141.png}
\caption{在二维空间中,右行(绿色)波和左行(蓝色)波的干涉,最终形成(红色)波。} \label{fig_GSWLX_3}
\end{figure}
波的叠加原理指出,当两个或更多相同类型的传播波在同一点相遇时,该点的合成振幅等于各个波的振幅的矢量和。[4] 如果一个波的波峰与另一个同频率波的波峰在同一点相遇,那么振幅就是各个振幅的总和——这就是建设性干涉。如果一个波的波峰与另一个波的波谷相遇,那么振幅等于各个振幅的差——这就是破坏性干涉。在理想介质中(如水、空气几乎是理想介质),能量始终是守恒的,在破坏性干涉的点,波的振幅互相抵消,能量会重新分布到其他区域。例如,当两颗小石子掉进池塘时,会观察到一定的波纹图案,但最终波动会继续传播,只有当波到达岸边时,能量才会从介质中被吸收。
\begin{figure}[ht]
\centering
\includegraphics[width=8cm]{./figures/047b21019d382954.png}
\caption{来自两个点源的波的干涉。} \label{fig_GSWLX_4}
\end{figure}
建设性干涉发生在波之间的相位差是π的偶数倍(180°)时,而破坏性干涉发生在相位差是π的奇数倍时。如果相位差介于这两个极端之间,那么合成波的位移幅度将在最小值和最大值之间。

例如,考虑当两颗相同的石子分别从不同地点投入静止的水池时发生的情况。每颗石子都会从掉落点向外传播出一个圆形波。当两波重叠时,特定点的净位移是各个波的位移之和。在某些点上,这些波会处于同相位,产生最大的位移。在其他地方,波会处于反相位,这些点的净位移为零。因此,水面上的一些部分将保持静止——如上图和右图所示,静止的蓝绿色线条从中心辐射出去。
\begin{figure}[ht]
\centering
\includegraphics[width=10cm]{./figures/5a62fc4dc06a85f2.png}
\caption{这是在白光下拍摄的1.5厘米 x 1厘米肥皂膜区域的照片。膜的厚度和观察几何角度的变化决定了哪些颜色会发生建设性或破坏性干涉。小气泡显著影响周围膜的厚度。} \label{fig_GSWLX_5}
\end{figure}
光的干涉是一种独特的现象,因为我们无法像水波那样直接观察到电磁场的叠加。电磁场的叠加是一个假设的现象,且必须用来解释为什么两束光线能够穿透彼此并继续沿各自的路径传播。光干涉的典型例子包括著名的双缝实验、激光散斑、抗反射涂层和干涉仪。

除了经典的波动模型来理解光学干涉外,量子物质波也展示了干涉现象。
\subsubsection{实值波函数} 
上述可以通过推导两个波的和公式在一维中展示。沿x轴向右传播的正弦波的振幅方程为:  
\[
W_1(x,t) = A \cos(kx - \omega t)~
\]
其中,\(A\) 是峰值振幅,  
\(k = \frac{2\pi}{\lambda}\) 是波数,  
\(\omega = 2\pi f\) 是波的角频率。  
假设第二个频率和振幅相同,但相位不同的波也向右传播:  
\[
W_2(x,t) = A \cos(kx - \omega t + \varphi)~
\]
其中,\(\varphi\) 是两个波之间的相位差(以弧度表示)。这两个波将会叠加并相加:  
\[
W_1 + W_2 = A \left[ \cos(kx - \omega t) + \cos(kx - \omega t + \varphi) \right]~
\]
使用两个余弦和的三角恒等式:  
\[
\cos a + \cos b = 2 \cos \left( \frac{a - b}{2} \right) \cos \left( \frac{a + b}{2} \right)~
\]
可以将其写为:  
\[
W_1 + W_2 = 2A \cos \left( \frac{\varphi}{2} \right) \cos \left( kx - \omega t + \frac{\varphi}{2} \right)~
\]

这表示一个以原始频率传播的波,向右传播,像它的分量一样,其振幅与 \(\cos(\varphi/2)\) 成正比。
\begin{itemize}
\item 建设性干涉:如果相位差是π的偶倍数:\(\varphi = \ldots, -4\pi, -2\pi, 0, 2\pi, 4\pi, \ldots\)则\(|\cos(\varphi / 2)| = 1\)因此,两个波的和是一个振幅是原来两倍的波:  
\[
W_1 + W_2 = 2A \cos(kx - \omega t)~
\]
\item 破坏性干涉:如果相位差是π的奇倍数:\(\varphi = \ldots, -3\pi, -\pi, \pi, 3\pi, 5\pi, \ldots\)则\(\cos(\varphi / 2) = 0\)
因此,两个波的和为零:  
\[
W_1 + W_2 = 0~
\]
\end{itemize}
\subsubsection{在两列平面波之间}
\begin{figure}[ht]
\centering
\includegraphics[width=10cm]{./figures/7167c3fcfc048f4a.png}
\caption{两平面波干涉的几何排列} \label{fig_GSWLX_6}
\end{figure}
一种简单的干涉图样是通过两列同频率的平面波相交得到的。一个波沿水平传播,另一个波则沿与第一个波成角度 θ 向下传播。假设在点 B 处这两个波是同相的,那么沿 x 轴的相位变化是这样的:在点 A 处的相位差为
\[
\Delta \varphi = \frac{2\pi x \sin \theta}{\lambda}.~
\]
可以看出,当
\[
\frac{x \sin \theta}{\lambda} = 0, \pm 1, \pm 2, \ldots~
\]
时,两波是同相的;而当
\[
\frac{x \sin \theta}{\lambda} = \pm \frac{1}{2}, \pm \frac{3}{2}, \ldots~
\]
时,它们的相位差为半个周期。

当两波同相时发生建设性干涉,当它们相位差为半个周期时发生破坏性干涉。于是,干涉条纹图样就会形成,其中最大值的间隔为
\[
d_f = \frac{\lambda}{\sin \theta},~
\]
df 被称为条纹间距。条纹间距随着波长的增加而增大,并随着角度 θ 的减小而增大。

这些条纹出现在两个波重叠的地方,且条纹间距在整个区域内是均匀的。
\begin{figure}[ht]
\centering
\includegraphics[width=8cm]{./figures/ab6f63e2067b447a.png}
\caption{重叠平面波中的干涉条纹} \label{fig_GSWLX_7}
\end{figure}
\subsubsection{两个球面波之间}
\begin{figure}[ht]
\centering
\includegraphics[width=8cm]{./figures/bf60f6eeed4a2300.png}
\caption{两个点光源之间的光学干涉,这些光源具有不同的波长和源间距离。} \label{fig_GSWLX_8}
\end{figure}
点光源产生球面波。如果来自两个点光源的光重叠,干涉图案描绘了两波之间相位差在空间中的变化方式。这取决于波长和点光源之间的距离。右侧的图示展示了两个球面波之间的干涉。波长从上到下增加,光源之间的距离从左到右增加。

当观察平面足够远时,干涉条纹将呈现出一系列几乎直线的形式,因为此时波几乎是平面的。

\subsubsection{多束光}

当多个波叠加时,只要它们之间的相位差在观察时间内保持不变,就会发生干涉。

有时,期望多个相同频率和振幅的波相加得到零(即发生破坏性干涉,相互抵消)。这正是例如三相电力和衍射光栅的原理。在这两种情况下,结果是通过相位的均匀间距实现的。

很容易看到,如果一组波具有相同的振幅,并且它们的相位以相等的角度间隔分布,那么它们将相互抵消。使用相位量,每个波可以表示为 \( Ae^{i\varphi_n} \),其中 \( N \) 是从 \( n = 0 \) 到 \( n = N-1 \) 的波数,并且相位差 \( \varphi_n - \varphi_{n-1} = \frac{2\pi}{N} \)。

要证明
\[
\sum_{n=0}^{N-1} A e^{i\varphi_n} = 0~
\]
只需假设相反情况,然后将两边同时乘以 \( e^{i \frac{2\pi}{N}} \)。

Fabry–Pérot 干涉仪使用多个反射之间的干涉。

衍射光栅可以被认为是一个多束干涉仪;因为它产生的峰值是由衍射光栅中每个单元所透过的光的干涉所产生的;有关干涉与衍射的更多讨论,请参见干涉与衍射。
\subsubsection{多束光}
当多个波叠加时,如果它们之间的相位差在观察时间内保持不变,就会发生干涉。

有时需要使多个频率和振幅相同的波相加为零(即发生破坏性干涉,互相抵消)。例如,三相电力和光栅的原理就是基于这个概念。在这两种情况下,结果是通过相位的均匀间隔实现的。

很容易看出,如果一组波的振幅相同,并且它们的相位在角度上均匀分布,那么这些波会互相抵消。使用相量表示法,每个波可以表示为:\(\displaystyle Ae^{i\varphi _{n}}\)
对于从n=0到n=N-1的N个波,其中\(\displaystyle \varphi _{n}-\varphi _{n-1}={\frac {2\pi }{N}}\)
要证明:
\[
\displaystyle \sum_{n=0}^{N-1}Ae^{i\varphi _{n}}=0~
\]
只需要假设相反的情况,然后将两边都乘以\(\displaystyle e^{i{\frac {2\pi }{N}}}.\)

法布里–珀罗干涉仪利用多个反射之间的干涉。

衍射光栅可以看作是一个多束光干涉仪,因为它产生的峰值是通过光栅中每个单元传输的光之间的干涉产生的;有关干涉与衍射的更多讨论,参见干涉与衍射。
\subsection{复数值波函数}

机械波和重力波可以直接观察到,它们是实值波函数;而光波和物质波不能直接观察到,它们是复值波函数。实值波与复值波干涉之间的一些区别包括:\\
a.干涉涉及不同类型的数学函数:经典波是表示从平衡位置偏移的实值函数;而光波或量子波函数是复值函数。经典波在任何点上可以是正的或负的;而量子概率函数是非负的。\\
b.相互干涉的波类型不同:任何两个不同的实波在同一介质中都会发生干涉;复波必须是相干的才能干涉。实际上,这意味着波必须来自同一源并具有相似的频率。\\
c.实波干涉通过简单地将两波的偏移量(或振幅)相加来得到;而复波干涉,我们需要测量波函数的模方。\\
\subsubsection{光波干涉}
\begin{figure}[ht]
\centering
\includegraphics[width=8cm]{./figures/4f52fbc3894abb96.png}
\caption{通过光学平板在反射表面上产生干涉条纹。来自单色光源的光线穿过玻璃并从平板的底面和支撑表面反射。表面之间的微小间隙意味着两束反射光的路径长度不同。此外,从底板反射的光线会经历 180° 的相位反转。因此,在路径差为 λ/2 的奇数倍的位置(a),波会相互增强;在路径差为 λ/2 的偶数倍的位置(b),波会相互抵消。由于表面之间的间隙在不同位置略有变化,因此可以看到一系列交替的明暗带,即干涉条纹。} \label{fig_GSWLX_9}
\end{figure}
由于光波的频率(大约为 \(10^{14}\) Hz)太高,当前的探测器无法检测光电场的变化,因此只能观察到光学干涉图案的强度。给定点处的光强度与波的平均振幅的平方成正比。可以用数学公式表示如下:

在点 \(r\) 处,两波的位移为:
\[
U_1(\mathbf{r}, t) = A_1(\mathbf{r}) e^{i[\varphi_1(\mathbf{r}) - \omega t]}~
\]
\[
U_2(\mathbf{r}, t) = A_2(\mathbf{r}) e^{i[\varphi_2(\mathbf{r}) - \omega t]}~
\]
其中,\(A\) 代表位移的幅度,\(\varphi\) 代表相位,\(\omega\) 代表角频率。

两个波的合成波的位移为:
\[
U(\mathbf{r}, t) = A_1(\mathbf{r}) e^{i[\varphi_1(\mathbf{r}) - \omega t]} + A_2(\mathbf{r}) e^{i[\varphi_2(\mathbf{r}) - \omega t]}~
\]
点 \(r\) 处的光强度由下式给出:
\[
I(\mathbf{r}) = \int U(\mathbf{r}, t) U^*(\mathbf{r}, t)\, dt \propto A_1^2(\mathbf{r}) + A_2^2(\mathbf{r}) + 2A_1(\mathbf{r}) A_2(\mathbf{r}) \cos[\varphi_1(\mathbf{r}) - \varphi_2(\mathbf{r})]~
\]
这可以表示为个别波的强度之和:
\[
I(\mathbf{r}) = I_1(\mathbf{r}) + I_2(\mathbf{r}) + 2 \sqrt{I_1(\mathbf{r}) I_2(\mathbf{r})} \cos[\varphi_1(\mathbf{r}) - \varphi_2(\mathbf{r})]~
\]
因此,干涉图案展示了两波之间的相位差,最大值发生在相位差为 \(2\pi\) 的倍数时。如果两束光的强度相等,则最大值是单独光束强度的四倍,最小值的强度为零。

从经典角度来看,为了产生干涉条纹,两波必须具有相同的极化状态,因为不同极化的波无法互相抵消或叠加在一起。相反,当不同极化的波叠加时,会产生不同极化状态的波。

从量子力学的角度来看,保罗·狄拉克(Paul Dirac)和理查德·费曼(Richard Feynman)的理论提供了一种更现代的理解方式。狄拉克证明了每个光子的量子或光子都是独立作用的,他著名地说过:“每个光子都与自己干涉。”理查德·费曼则展示了通过评估路径积分,其中考虑了所有可能路径,许多高概率路径将浮现出来。例如,在薄膜中,如果膜的厚度不是光波长的整数倍,则光量子无法穿透,只有反射是可能的。

\textbf{光源要求}

上述讨论假设相互干涉的波是单色的,即具有单一频率——这要求它们在时间上是无限的。然而,这既不实际也不是必须的。两束频率在一定时间内固定的相同波,会在它们重叠的期间产生干涉图样。两束相同的波,若由有限时间内的窄频谱波组成(但短于它们的相干时间),会产生一系列间隔略有不同的干涉条纹,只要这些间隔的变化比平均条纹间隔小得多,那么在这两束波重叠的时间内,依然会观察到干涉条纹。

常规光源发出的光波具有不同的频率,并且从光源的不同点在不同时间发射。如果光被分成两束波再重新合并,每一束单独的光波可能会与它的另一半产生干涉图样,但这些单独产生的干涉图样具有不同的相位和间距,通常无法观察到整体的干涉图样。然而,单一元素光源,如钠灯或汞灯,具有较窄的发射频谱。当这些光源经过空间和颜色过滤后,再分成两束波,它们可以叠加产生干涉条纹。所有在激光发明之前的干涉仪都使用这种光源,并取得了广泛的成功应用。

激光束通常更接近单色光源,因此使用激光生成干涉条纹要简单得多。由于激光束可以轻松观察到干涉条纹,这有时也会带来问题,因为杂散反射可能产生虚假的干涉条纹,从而导致错误。

通常,干涉测量中使用单束激光束,尽管也有观察到使用两个独立激光的干涉现象,只要它们的频率足够匹配,以满足相位要求。这也曾被观察到在两个不相干的激光光源之间的宽场干涉中。

也可以使用白光观察干涉条纹。白光干涉图样可以被视为由一组“谱”组成,每个谱的条纹间距略有不同。如果所有的条纹图样在中心处相位一致,则随着波长的减小,条纹的大小会增大,合成强度将显示三到四个不同颜色的条纹。杨在他关于双缝干涉的讨论中优雅地描述了这一点。由于白光条纹只有在两束波从光源传播相等的距离时才会产生,因此它们在干涉仪中非常有用,因为它们能够标识零路径差的条纹。

\textbf{光学装置}

为了产生干涉条纹,光源发出的光必须被分成两束波,然后再将它们重新合并。传统上,干涉仪通常被分为振幅分割系统和波面分割系统。

在振幅分割系统中,使用分束器将光分成两束沿不同方向传播的光波,然后将这两束光波叠加以产生干涉图样。迈克耳孙干涉仪和马赫-曾德干涉仪就是振幅分割系统的例子。

在波面分割系统中,光波在空间上被分割——例如杨氏双缝干涉仪和劳埃德镜。

干涉现象也可以在日常现象中看到,比如虹彩和结构色。例如,肥皂泡中的颜色是由光在薄肥皂膜的前后表面反射后发生干涉所产生的。根据膜的厚度,不同的颜色会发生建设性和破坏性的干涉。
\begin{figure}[ht]
\centering
\includegraphics[width=14.25cm]{./figures/09a495fd8f2dd9bb.png}
\caption{} \label{fig_GSWLX_10}
\end{figure}
\subsubsection{量子干涉}
量子干涉——物质的波动行为——与光学干涉相似。假设  \(\Psi(x,t)\)是描述量子力学对象的薛定谔方程的波函数解。则在位置 \(x\) 处观察到该物体的概率为 \(P(x) = |\Psi(x,t)|^2 = \Psi^*(x,t) \Psi(x,t)\)其中 * 表示复共轭。量子干涉关注当波函数表示为两个项的和或线性叠加时,关于该概率的问题:  
\[
\Psi(x,t) = \Psi_A(x,t) + \Psi_B(x,t)~
\]
则概率为  
\[
P(x) = |\Psi(x,t)|^2 = |\Psi_A(x,t)|^2 + |\Psi_B(x,t)|^2 + (\Psi_A^*(x,t) \Psi_B(x,t) + \Psi_A(x,t) \Psi_B^*(x,t))~
\]

通常,\(\Psi_A(x,t)\) 和 \(\Psi_B(x,t)\) 对应于不同的情境A和B。当是这种情况时,方程  \(\Psi(x,t) = \Psi_A(x,t) + \Psi_B(x,t)\)表示该物体可以处于情境A或情境B。上述方程可以解释为:物体在位置 \(x\) 处的概率是物体处于情境A时在位置 \(x\) 处找到物体的概率,加上物体处于情境B时在位置 \(x\) 处找到物体的概率,再加上一个额外的项。这个额外的项,即量子干涉项,是 \(\Psi_A^*(x,t) \Psi_B(x,t) + \Psi_A(x,t) \Psi_B^*(x,t)\)如上方程所示。与经典波的情况一样,量子干涉项可以对 \(|\Psi_A(x,t)|^2 + |\Psi_B(x,t)|^2\)进行加法(建设性干涉)或减法(破坏性干涉),取决于量子干涉项是正值还是负值。如果在所有 \(x\) 处该项都不存在,那么就没有与情境A和B相关的量子力学干涉。

量子干涉最著名的例子是双缝实验。在这个实验中,来自电子、原子或分子的物质波接近带有两个缝隙的屏障。一个缝隙成为 \(\Psi_A(x,t)\),另一个缝隙成为 \(\Psi_B(x,t)\)。干涉图样出现在远侧,由适合物质波粒子的探测器观察到。这个图样与光学双缝干涉图样相匹配。
\subsection{应用}
\subsubsection{拍音(Beat)}
在声学中,拍音是指两个频率略有不同的声音之间的干涉图案,表现为音量的周期性变化,其变化速率等于两个频率的差。

在能够产生持续音调的调音乐器中,拍音可以很容易地被识别。当两个音调调至同音时,会产生一种特殊效果:当两个音调音高接近但不完全相同,频率的差异会产生拍音。音量像颤音一样变化,因为声音交替进行建设性干涉和破坏性干涉。当两个音调逐渐接近同音时,拍音的速度变慢,甚至可能变得不可察觉。当两个音调相距较远时,拍音的频率开始接近人类音高感知的范围,拍音听起来就像一个音符,并且会产生组合音。这个组合音也可以被称为“缺失基音”,因为任意两个音调的拍频等于它们隐含的基频。
\subsubsection{干涉测量学}
干涉测量学在物理学的发展中发挥了重要作用,并且在物理和工程测量中具有广泛的应用。它对物理学的影响以及其应用涵盖了各种类型的波。

\textbf{光学干涉测量学}

托马斯·杨(Thomas Young)于1803年通过双缝干涉仪实验演示了干涉条纹,当两个小孔被来自另一个小孔的光照亮时,后者又由阳光照射。杨能够通过条纹的间距估算出光谱中不同颜色的波长。该实验在光的波动理论得到普遍接受方面发挥了重要作用。在量子力学中,该实验被认为展示了光和其他量子粒子的波动性与粒子性的不可分离性(波粒二象性)。理查德·费曼(Richard Feynman)常常说,整个量子力学的理论都可以通过仔细思考这一单一实验的含义来得出。

迈克耳孙-莫雷实验的结果通常被认为是对光以太理论的有力反证,并支持相对论的理论。

干涉测量学已被用于定义和校准长度标准。当米被定义为铂铱合金条上两个标记之间的距离时,迈克耳孙和贝努瓦(Benoît)使用干涉测量法测量了新标准中红色镉线的波长,并证明它可以作为长度标准使用。60年后,1960年,国际单位制(SI)中的米被定义为等于氪-86原子在真空中橙红色发射线的1,650,763.73个波长。这一定义在1983年被替代,米被定义为光在真空中在特定时间间隔内所经过的距离。干涉测量学仍然是建立长度测量校准链的基础。

干涉测量学还用于滑块(美国称为量块)的校准和坐标测量机的校准,也被用于光学组件的测试。
\textbf{天文干涉测量学}
\begin{figure}[ht]
\centering
\includegraphics[width=8cm]{./figures/2f0f00369d7d71a1.png}
\caption{非常大阵列(Very Large Array,VLA)是一个由许多较小望远镜组成的干涉阵列,类似于许多大型射电望远镜。} \label{fig_GSWLX_11}
\end{figure}
1946年,天文干涉测量技术被开发出来。天文射电干涉仪通常由抛物面天线阵列或二维全向天线阵列组成。阵列中的所有望远镜通常彼此相隔较远,并通过同轴电缆、波导、光纤或其他类型的传输线连接在一起。干涉测量学增加了收集的总信号,但其主要目的是通过一个称为孔径合成(Aperture synthesis)的过程大幅提高分辨率。该技术通过将来自不同望远镜的信号波进行叠加(干涉)来工作,原理是:相位相同的波会相互叠加,而相位相反的波会互相抵消。这创造了一个组合的望远镜,其分辨率相当于一个单一的天线,且其直径等于阵列中最远天线之间的间距(但灵敏度不同)。

\textbf{声学干涉测量学}

声学干涉仪是一种用于测量气体或液体中声波物理特性的仪器,如速度、波长、吸收或阻抗。一个振动的晶体产生超声波,这些波被辐射到介质中。波遇到平行于晶体放置的反射器后反射回源,并进行测量。
\subsection{另见}
\begin{itemize}
\item 主动噪声控制
\item 音频拍频(Beat,声学)
\item 相干性(物理学)
\item 衍射
\item 海丁格条纹(Haidinger fringes)
干涉光刻(Interference lithography)
- 干涉可见度(Interference visibility)
- 干涉仪(Interferometer)
- 劳埃德镜(Lloyd's Mirror)
- 莫尔条纹(Moiré pattern)
- 多路径干扰(Multipath interference)
- 牛顿环(Newton's rings)
- 光程(Optical path length)
- 薄膜干涉(Thin-film interference)
- 雷利粗糙度准则(Rayleigh roughness criterion)
- 上升衰落(Upfade)
\end{itemize}