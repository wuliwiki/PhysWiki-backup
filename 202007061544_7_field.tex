% 域
% 实数|域|加|乘
%概念介绍基本完成
\pentry{环\upref{Ring}}

\subsection{体和域的概念}

\begin{definition}{体}
给定一个集合$H$,如果这个集合中定义了两个运算,加法“+”和乘法“\times”,并且$H$对于加法构成一个阿贝尔群,而$H-\{0\}$构成群($0$为$H$加法群的单位元),并且乘法对加法满足分配律,即对于任何$a, b, c\in H$,满足$a\times(b+c)=a\times b+a\times c$,那么我们称$(H, +, \times)$构成一个\textbf{体(skew field)},或称\textbf{可除环(division ring)}、\textbf{除环}.
\end{definition}

简单来说,体就是能进行加减乘除的一个集合,其中加法是可交换的,乘法却不一定.由于乘法不一定交换,这就使得除法运算相对复杂,但我们在此不过多展开.

\begin{example}{四元数体}
四元数\upref{Quat}词条中所定义的全体四元数构成的集合,配上所定义的加法和乘法,构成一个体,称为\textbf{四元数体}.
\end{example}

\begin{example}{矩阵体}
某个域上的全体$n$阶可逆矩阵,配上矩阵加法和乘法,构成一个体.
\end{example}

类比子群和子环的定义,我们也可以定义子体.

\begin{definition}{子体}
体$H$的子集$S$如果满足其对已有的加法和乘法仍然构成体,那么称$S$是$H$的一个\textbf{子体}.
\end{definition}
子体的概念引出了以下关键概念.
\begin{definition}{素体与素域}
一个体的子体之交显然还是一个子体,因此每个体都存在唯一的非平凡不可约子体,称为这个体的\textbf{素体(prime field)}或\textbf{素域}.
\end{definition}

素体又被称作素域的原因是,素体必然是$\mathbb{Z}_p$或$\mathbb{Q}$,其中$p$是素数.这两种体都是域.

那么我们刚才讨论中的域是什么呢?

\begin{definition}{域}
给定一个体$\mathbb{F}$,如果$\mathbb{F}$的乘法满足交换律,那么称其为一个\textbf{域(field)}.
\end{definition}

一个域的子体总是乘法可交换的,因此也都称为域的\textbf{子域(sub-field)}.

数学界主流将域视作乘法可交换的体,因此当谈到域时,总是认为乘法可交换.少数数学家会把我们以上定义的体称为域,而将我们定义的域称为交换域,但这并不是主流,因此本书使用以上定义.

\begin{example}{数域}
包含整数集的域,称为数域.最重要的数域有三个,有理数域$\mathbb{Q}$,实数域$\mathbb{R}$和复数域$\mathbb{C}$,其中$\mathbb{Q}$是最小的数域,也就是说任何数域都包含它;实数域是有理数域的完备化,意味着有理数域中的收敛数列都收敛于某个实数;复数域是最大的数域,也就是说任何数域都是复数域的子域.

注意,$\mathbb{Z}_p$并不是$\mathbb{Z}$的子域,因为在$\mathbb{Z}_p$中,$(p-1)+1=0$,而这在$\mathbb{Z}$中是不可能的.
\end{example}

素域的概念对于描述任意的域是关键,以至于我们用素域定义了一个概念,称作域的特征:

\begin{definition}{域的特征}
给定域$\mathbb{F}$,如果它所包含的素域是$\mathbb{Z}_p$,那么称$\mathbb{F}$的\textbf{特征(character)}是$p$;如果它的素域是$\mathbb{Q}$,那么称它的特征是$0$.
\end{definition}

由定义可见,特征的值取素数或者$0$,这个值在很大程度上决定了域的代数性质.

