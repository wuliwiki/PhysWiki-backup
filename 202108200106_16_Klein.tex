% 克莱因-戈登传播子
% keys 克莱因|戈登|传播子

我们现在来看$[\phi(x),\phi(y)]$这个量.因为这个量是一个c数,所以我们有
\begin{equation}
[\phi(x),\phi(y)] = \langle 0 | [\phi(x),\phi(y)] | 0 \rangle
\end{equation}
我们可以用四维形式来表示它.我们首先假设$x^0>y^0$.
\begin{equation}\label{Klein_eq1}
\begin{aligned}
& \langle 0|[\phi(x), \phi(y)]| 0\rangle\\
=&\int \frac{d^{3} p}{(2 \pi)^{3}} \frac{1}{2 E_{\mathbf{p}}}\left(e^{-i p \cdot(x-y)}-e^{i p \cdot(x-y)}\right) \\ 
=& \int \frac{d^{3} p}{(2 \pi)^{3}}\left\{\left.\frac{1}{2 E_{\mathbf{p}}} e^{-i p \cdot(x-y)}\right|_{p^{0}=E_{\mathbf{p}}}+\left.\frac{1}{-2 E_{\mathbf{p}}} e^{-i p \cdot(x-y)}\right|_{p^{0}=-E_{\mathbf{p}}}\right\} \\
=& \int \frac{d^{3} p}{(2 \pi)^{3}} \int \frac{d p^{0}}{2 \pi i} \frac{-1}{p^{2}-m^{2}} e^{-i p \cdot(x-y)} 
\end{aligned}
\end{equation}
\begin{figure}[ht]
\centering
\includegraphics[width=14cm]{./figures/Klein_1.png}
\caption{在最后一步中,$p^0$的积分围道如图所示.} \label{Klein_fig1}
\end{figure}
对于$x^0>y^0$我们从下面闭合围道,包围两个极点.对于$x^0<y^0$我们从上面闭合围道,结果是0.因此\autoref{Klein_eq1} 的的最后一行以及闭合围道的办法可以用下式表示
\begin{equation}
D_R(x-y)\equiv \theta(x^0-y^0)\langle 0 | [\phi(x),\phi(y)]|0 \rangle
\end{equation}
可以证明,$D_R(x-y)$满足
\begin{equation}
(\partial^2+m^2)D_R(x-y) = -i \delta^{(4)}(x-y)
\end{equation}
$D_R(x-y)$在$x^0<y^0$时为零,所以被称为\textbf{延迟格林函数}.对延迟格林函数进行傅里叶变换可得.
\begin{equation}
D_R(x-y) = \int \frac{d^4p}{(2\pi)^4} e^{-ip(x-y)} \tilde D_R (p)
\end{equation}
其中动量空间的格林函数$\tilde D_R(p)$满足下式
\begin{equation}\label{Klein_eq2}
(-p^2+m^2) \tilde D_R(p) = -i
\end{equation}
由\autoref{Klein_eq2} 可推出
\begin{equation}
D_R (x-y) = \int \frac{d^4p}{(2\pi)^4} \frac{i}{p^2-m^2} e^{-ip\cdot(x-y)}
\end{equation}
积分围道如图所示.这种积分围道被称为Feynman prescription.
