% Baker-Hausdorff 公式
% Baker|Hausdorff|定理
\pentry{组合\upref{combin}}
\begin{issues}
\issueOther{添加量子力学的例子}
\end{issues}


\footnote{参考樱井量子力学\cite{Sakurai}.}Baker-Hausdorff 公式是一个相当有用的公式.在数学上,它可用于给出李群-李代数对应的深层结果的相对简单的证明;在量子力学中,它可实现系统哈密顿量在薛定谔绘景和海森堡绘景\upref{HsbPic}的转换,并在微扰论中也有诸多应用.本节将给出该公式的一个证明和由它导出的一些重要的结果.

\textbf{Baker-Hausdorff公式}是指
\begin{equation}\label{BAHA_eq1}
\E ^{A}B\E^{-A}=\sum_{n=0}^{\infty}\frac{1}{n!}[A^{(n)},B]
\end{equation}
其中,
\begin{equation}\label{BAHA_eq5}
A^{(n)}\equiv\underbrace{[A,[A,\cdots,[A}_{n\text{个}},B]\cdots]
=\sum_{m=0}^{n}(-1)^{n-m}C_{n}^{m}A^mBA^{n-m}
\end{equation}

\subsection{证明}
在证明Baker-Hausdorff公式\autoref{BAHA_eq1} 之前,我们先证明\autoref{BAHA_eq5} .\\

\textbf{\autoref{BAHA_eq5} 的证明:}我们用数学归纳法来证明.
$A^{(0)},A^{(1)}$ 显然成立:
\begin{equation}
\begin{aligned}
A^{(0)}&=B=\sum_{m=0}^{0}(-1)^{0-m}C_{0}^{m}A^mBA^{0-m}\\
A^{(1)}=&[A,B]=AB-BA=\sum_{m=0}^{1}(-1)^{1-m}C_{1}^{m}A^mBA^{1-m}
\end{aligned}
\end{equation}

假设对 $n=k-1$ 时\autoref{BAHA_eq5} 成立,则
\begin{equation}
\begin{aligned}
A^{(k)}&=[A,A^{(k-1)}]=AA_{k-1}-A_{k-1}A\\
&=A\sum_{m=0}^{k-1}(-1)^{k-1-m}C_{k-1}^{m}A^mBA^{k-1-m}-\qty(\sum_{m=0}^{k-1}(-1)^{k-1-m}C_{k-1}^{m}A^mBA^{k-1-m})A\\
&=\sum_{m=0}^{k-1}(-1)^{k-1-m}C_{k-1}^{m}A^{m+1}BA^{k-1-m}-\sum_{m=0}^{k-1}(-1)^{k-1-m}C_{k-1}^{m}A^mBA^{k-m}\\
&=A^{k}B+\sum_{m=0}^{k-2}(-1)^{k-1-m}\qty(C_{k-1}^{m}+C_{k-1}^{m+1})A^{m+1}BA^{k-1-m}-(-1)^{k-1}BA^k\\
&=A^{k}B+\sum_{m=0}^{k-2}(-1)^{k-1-m}C_{k}^{m+1}A^{m+1}BA^{k-1-m}-(-1)^{k-1}BA^k\\
&=A^{k}B+\sum_{m=1}^{k-1}(-1)^{k-m}C_{k}^{m}A^{m}BA^{k-m}+(-1)^{k}BA^k\\
&=\sum_{m=0}^{k}(-1)^{k-m}C_{k}^{m}A^{m}BA^{k-m}
\end{aligned}
\end{equation}

由数学归纳法原理,\autoref{BAHA_eq5} 得证.上面证明中 $C_n^m$ 为组合数\upref{combin}.

现在,我们将用两种方法证明Baker-Hausdorff公式.
\subsubsection{纯数学证明}
\footnote{郝柏林.统计微扰论的生成泛函.1978年统计物理讨论会综述报告}先来证明下面得引理.
\begin{lemma}{}\label{BAHA_lem2}
\begin{equation}\label{BAHA_eq2}
A^nB=\sum_{m=0}^{n}C_{n}^mA^{(m)}A^{n-m}
\end{equation}
\end{lemma}
\textbf{证明:}这里同样用数学归纳法来证明.当 $n=1$ 时,\autoref{BAHA_eq2} 显然成立:
\begin{equation}
AB=BA+[A,B]
\end{equation}
假设 $n=k-1$ 时\autoref{BAHA_eq2} 成立,则
\begin{equation}\label{BAHA_eq3}
\begin{aligned}
A^{k}B=A\sum_{m=0}^{k-1}C_{k-1}^mA^{(m)}A^{k-1-m}=\sum_{m=0}^{k-1}C_{k-1}^mAA^{(m)}A^{k-1-m}
\end{aligned}
\end{equation}

因为
\begin{equation}
A^{(m+1)}=[A,A^{(m)}]\Rightarrow AA^{(m)}=A^{(m)}A+A^{(m+1)}
\end{equation}
所以\autoref{BAHA_eq3} 可改写为
\begin{equation}
\begin{aligned}
A^k B&=\sum_{m=0}^{k-1}C_{k-1}^mA^{(m)}A^{k-m}+\sum_{m=0}^{k-1}C_{k-1}^mA^{(m+1)}A^{k-1-m}\\
&=A^{(0)}A^{k}+\sum_{m=1}^{k-1}\qty[C_{k-1}^m+C_{k-1}^{m-1}]A^{(m)}A^{k-m}+A^{(k)}\\
&=A^{(0)}A^{k}+\sum_{m=1}^{k-1}C_{k}^mA^{(m)}A^{k-m}+A^{(k)}\\
&=\sum_{m=0}^{k}C_{k}^mA^{(m)}A^{k-m}
\end{aligned}
\end{equation}
由数学归纳法原理,证得\autoref{BAHA_lem2} 

现在,Baker-Hausdorff公式就呼之欲出了!

\textbf{Baker-Hausdorff公式的证明:}
由\autoref{BAHA_lem2} 
\begin{equation}\label{BAHA_eq4}
\E^A B=\sum_{n=0}^{\infty}\frac{A^nB}{n!}=\sum_{n=0}^{\infty}\frac{1}{n!}\sum_{i=0}^{n}C_{n}^iA^{(i)}A^{n-i}
\end{equation}

将上式求和符号交换顺序,注意指标要求满足 $i\leq n$,可将\autoref{BAHA_eq4} 改写为
\begin{equation}\label{BAHA_eq6}
\begin{aligned}
\E^A B&=\sum_{i=0}^{\infty}\sum_{n=i}^{\infty}\frac{1}{n!}C_{n}^iA^{(i)}A^{n-i}\\
&=\sum_{i=0}^{\infty}\frac{1}{i!}A^{(i)}\sum_{n=i}^{\infty}\frac{1}{(n-i)!}A^{n-i}\\
&=\sum_{i=0}^{\infty}\frac{1}{i!}A^{(i)}\E^A
\end{aligned}
\end{equation}
\autoref{BAHA_eq6} 两边作用于 $\E^{-A}$ ,证得\autoref{BAHA_eq1} 

\subsubsection{较物理的证明}
\footnote{何勇.用算符矩阵表示方法简洁推导 Baker-Hausdorff 公式[J].大学物理,2015,34(1):30-31}现在,我们给出一种量子力学语言的证明方式.选择 $A$ 表象为工作空间,其基右矢集合为 $\qty{\ket{a_n}}$,且 $A\ket{a_n}=a_n\ket{a_n}$.为方便起见,仅考虑离散态的情况.完备性条件为
\begin{equation}
\sum_{n}\ket{a_n}\bra{a_n}=1
\end{equation}
由\autoref{BAHA_eq5} ,关于 $A$ 算符的矩阵表达式为:
\begin{equation}\label{BAHA_eq7}
\begin{aligned}
A^{(n)}&=\sum_{i,j,k,l}\sum_{m=0}^{n}(-1)^{n-m}C_{n}^{m}\ketbra{a_i}{a_i}A^m\ketbra{a_k}{a_k}B\ketbra{a_l}{a_l}A^{n-m}\ketbra{a_j}{a_j}\\
&=\sum_{i,j}\sum_{m=0}^{n}(-1)^{n-m}C_n^m{a_i}^m{a_j}^{n-m}\bra{a_i}B\ket{a_j}\ketbra{a_i}{a_j}\\
&=\sum_{i,j}\sum_{m=0}^{n}(-1)^{n-m}C_n^m{a_i}^m{a_j}^{n-m}B^i_{\; j}\ketbra{a_i}{a_j}\\
&=\sum_{i,j}(a_i-a_j)^nB^i_{\; j}\ketbra{a_i}{a_j}
\end{aligned}
\end{equation}
$\E^{A}B\E^{-A}$ 的矩阵表达式为
\begin{equation}\label{BAHA_eq8}
\begin{aligned}
\E^{A}B\E^{-A}&=\sum_{i,j,k,l}\ketbra{a_i}{a_i}\E^{A}\ketbra{a_k}{a_k}B\ketbra{a_l}{a_l}\E^{-A}\ketbra{a_j}{a_j}\\
&=\sum_{i,j}\E^{a_i-a_j}B^{i}_{\;j}\ketbra{a_i}{a_j}\\
&=\sum_{i,j}\sum_{n=0}^{\infty}\frac{1}{n!}(a_i-a_j)^nB^{i}_{\;j}\ketbra{a_i}{a_j}\\
&=\sum_{n=0}^{\infty}\frac{1}{n!}\sum_{i,j}(a_i-a_j)^nB^{i}_{\;j}\ketbra{a_i}{a_j}
\end{aligned}
\end{equation}
对比\autoref{BAHA_eq7} 和\autoref{BAHA_eq8} ,得
\begin{equation}\label{BAHA_eq1}
\E ^{A}B\E^{-A}=\sum_{n=0}^{\infty}\frac{1}{n!}[A^{(n)},B]
\end{equation}
Baker-Hausdorff公式\autoref{BAHA_eq1} 得证.

\subsection{Baker-Campbell-Hausdorff公式}
\pentry{矩阵指数\upref{MatExp}}
\textbf{Baker-Campbell-Hausdorff公式}是下列方程关于 $Z$ 的解
\begin{equation}
\E^{A}\E^{B}=\E^{Z}
\end{equation}
我们将用Baker-Hausdorff公式导出Baker-Campbell-Hausdorff公式的一些性质.

如果算子 $A$ 与 $B$ 不对易,算子函数 $\E^{A+B}$ 不能简单地分解为  $\E^{A}$ 和 $\E^{B}$ 乘积.只能引入另一个算子 $K(\tau)$ 和 $K'(\tau)$( $A,B$ 与 $\tau$ 无关) ,写成
\begin{equation}\label{BAHA_eq9}
\E^{\tau(A+B)}=\E^{\tau B}K(\tau)\E^{\tau A}=\E^{\tau A}K'(\tau)\E^{\tau B}
\end{equation}
\autoref{BAHA_eq9} 中交换 $A,B$.可由 $K$ 导出 $K'$,故只需讨论 $K(\tau)$.下面我们来求 $K(\tau)$ 的具体形式.

\autoref{BAHA_eq9} 对 $\tau$ 微分,并利用\autoref{MatExp_eq3}~\upref{MatExp},可得关于算子 $K$ 的微分方程:
\begin{equation}\label{BAHA_eq10}
\begin{aligned}
\E^{\tau(A+B)}A+\E^{\tau B}K(\tau)\E^{\tau A}B&=B\E^{\tau B}K(\tau)\E^{\tau A}+\E^{\tau B}\pdv{K}{\tau}\E^{\tau A}+\E^{\tau(A+B)}A\\
&\Downarrow\\
\pdv{}{\tau}K&=K\E^{\tau A}B\E^{-\tau A}-BK
\end{aligned}
\end{equation}

将\autoref{BAHA_eq1} 带入\autoref{BAHA_eq10} :
\begin{equation}\label{BAHA_eq11}
\pdv{}{\tau}K=\sum_{n=0}^{\infty}\frac{\tau^n}{n!}KA^{(n)}-BK
\end{equation}
由\autoref{BAHA_eq9} ,其初值条件为:
\begin{equation}
K(0)=1
\end{equation}
\autoref{BAHA_eq11} 的具体求解依赖于 $A,B$ 的性质.

\begin{example}{}
已知 $A^{(1)}=[A,B]$ 是个数,$A^{(n)}=0(n\geq 2)$.求 $\E^{\tau(A+B)}$

由\autoref{BAHA_eq11} :
\begin{equation}
\pdv{}{\tau}K=KB+K\tau[A,B]-BK
\end{equation}
要求 $K(\tau)$ 是个数,则上式变为:
\begin{equation}
\pdv{}{\tau}K=K\tau[A,B]
\end{equation}
解得:
\begin{equation}
K=\E^{\frac{1}{2}[A,B]\tau^2}
\end{equation}
将 $K$ 中 $A,B$ 交换,得到 
\begin{equation}
K'=\E^{\frac{1}{2}[B,A]\tau^2}
\end{equation}

故
\begin{equation}\label{BAHA_eq12}
\E^{\tau(A+B)}=\E^{\tau B}\E^{\tau A}\E^{\frac{1}{2}[A,B]\tau^2}=\E^{\frac{1}{2}[B,A]\tau^2}\E^{\tau A}\E^{\tau B}
\end{equation}
\autoref{BAHA_eq12} 还可写为
\begin{equation}
\begin{aligned}
\E^{2\tau A+2\tau B}&=\E^{\tau(A+B+A+B)}=\E^{\tau (A+B)}\E^{\tau (A+B)}\E^{\frac{1}{2}\qty[\tau(A+B),\tau(A+B)]\tau^2}\\
&=\E^{\tau(A+B)}\E^{\tau(A+B)}\\
\E^{2\tau A+2\tau B}&=\E^{\tau(A+B)}\E^{\tau(A+B)}\\
&=\E^{\tau B}\E^{\tau A}\E^{\frac{1}{2}[A,B]\tau^2}\E^{\frac{1}{2}[B,A]\tau^2}\E^{\tau A}\E^{\tau B}\\
&=\E^{\tau B}\E^{2\tau A}\E^{\tau B}
\end{aligned}
\end{equation}
\end{example}
