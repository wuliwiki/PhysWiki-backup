% 高斯波包
% keys 高斯分布|波包|光学|量子力学

\begin{issues}
\issueDraft
\end{issues}

\pentry{波包\upref{WvPck}, 高斯分布\upref{GausPD}}

\begin{figure}[ht]
\centering
\includegraphics[width=14.25cm]{./figures/GausPk_1.pdf}
\caption{高斯波包(\autoref{GausPk_eq1} ),蓝色为实部,红色为虚部, $x_0 = 0$, $A_0 = 1$, $a = 1/20$, $k_0 = 5$.} \label{GausPk_fig1}
\end{figure}

\footnote{参考 Wikipedia \href{https://en.wikipedia.org/wiki/Wave_packet}{相关页面}.}\textbf{高斯波包(Gaussian wave packet)}是指轮廓为高斯分布的波包, 在光学和量子力学中有重要应用. 高斯波包用复函数表示为($A_0$ 为复数)
\begin{equation}\label{GausPk_eq1}
f(x) = A_0 \E^{-a(x-x_0)^2}\E^{\I k_0 (x-x_0)}
\end{equation}

\subsection{频谱}
\pentry{傅里叶变换(指数)\upref{FTExp}}
要求\autoref{GausPk_eq1} 的傅里叶变换 $g(k)$, 由\autoref{FTExp_ex1}~\upref{FTExp}以及傅里叶变换性质\autoref{FTExp_eq4}~\upref{FTExp}和\autoref{FTExp_eq7}~\upref{FTExp}得
\begin{equation}
g(k) = A_0\sqrt{\frac{\pi}{a}} \exp[-\frac{(k+k_0)^2}{4a}] \E^{-\I x_0 k}
\end{equation}
