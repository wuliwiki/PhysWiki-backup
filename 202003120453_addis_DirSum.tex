% 直和
% 矢量空间|子空间|直和空间|直和

\pentry{子空间\upref{SubSpc}}
\begin{definition}{直和}
令矢量空间 $V_1$ 和 $V_2$ 为 $V$ 的两个子空间, 且任意 $\ket{v} \in V$ 都能表示为
\begin{equation}
\ket{v} = c_1 \ket{v_1} + c_2 \ket{v_2}
\qquad
\ket{v_1} \in V_1,\ \ket{v_2} \in V_2
\end{equation}
那么空间 $V$ 就是 $V_1$ 和 $V_2$ 的\textbf{直和空间}, 用\textbf{直和}运算记为
\begin{equation}
V = V_1 \oplus V_2
\end{equation}
\end{definition}

直和空间的所有矢量可以分为三类, 分别是 $V_1$ 中的矢量, $V_2$ 的矢量, 以及只能表示为 $V_1$ 和 $V_2$ 中矢量的线性组合的矢量.

\subsection{直和空间的基底}
从基底的角度来看, 若 $V_1$ 和 $V_2$ 中分别有一组基底 $\ket{\alpha_i}$ $(i = 1, \dots, N_1)$ 和 $\ket{\beta_i}$ $(i = 1, \dots, N_2)$, 那么直和空间中的任意矢量可以表示为(系数可以部分或全部为零)
\begin{equation}
\ket{v} = \sum_i a_i \ket{\alpha_i} + \sum_j b_j \ket{\beta_j}
\end{equation}
也就是说直和空间就是两个子空间的基底合并后张成的空间. 注意合并后的 $N_1 + N_2$ 个矢量之间不一定线性无关, 所以不一定能构成直和空间中的一组基底. 虽然某个 $\ket{\alpha_i}$ 不能表示为其他 $\ket{\alpha_i}$ 的线性组合, 但却可能可以表示为一些 $\ket{\beta_i}$ 的线性组合, $\ket{\beta_i}$ 亦然.

如果把这些 “多余” 的矢量全部剔除, 使任意 $\ket{\alpha_i}$ 不能用 $\ket{\beta_i}$ 的线性组合表示, 且任意 $\ket{\beta_i}$ 不能用 $\ket{\alpha_i}$ 的线性组合表示, 那么我们就得到了直和空间中的一组基底, 其维数小于或等于 $N_1 + N_2$, 且大于或等于 $N_1$ 和 $N_2$.

\begin{example}{}\label{DirSum_ex1}
若三维空间中有两个不共线的几何矢量 $\ket{v_1}, \ket{v_2}$, 它们张成一个平面, 或二维子空间. 另有一个矢量 $\ket{v_3}$, 独自张成一条直线, 即一维空间. 若 $\ket{v_3}$ 落在 $\ket{v_1}, \ket{v_2}$ 张成的平面内, 则三个矢量的所有线性组合仍然在该平面内, 所以直和空间仍然是该平面. 但如果 $\ket{v_3}$ 落在平面外, 则三个矢量将会张成整个三维空间, 所以直和就是三维空间.
\end{example}

\begin{example}{}\label{DirSum_ex2}
若三维空间中有两个不共线的几何矢量 $\ket{v_1}, \ket{v_2}$, 它们张成一个平面(二维子空间) $V_{12}$. 另有 $V_{12}$ 平面外的两个不共线且与的几何矢量 $\ket{v_3}$ 和 $\ket{v_4}$, 张成另一个二维子空间 $V_{34}$. 然而四个矢量中只有三个是线性无关的(容易证明任意三个都线性无关), 所以 $V = V_{12} \oplus V_{34}$ 是三维而不是四维几何矢量空间, 四个矢量中的任意三个都可以作为 $V$ 空间的基底.
\end{example}
