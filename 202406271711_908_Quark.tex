% 夸克
% license CCBYSA3
% type Wiki

(本文根据 CC-BY-SA 协议转载自原搜狗科学百科对英文维基百科的翻译)

\textbf{夸克}是一种基本粒子,同时也是构成物质的基本成分。夸克相互结合,形成一种复合粒子,叫强子,其中最稳定的是质子和中子,它们是原子核的组成部分。[1]基于一种叫”夸克禁闭“的现象,夸克永远不会被直接观察到或孤立地发现。它们只能在强子中找到,包括重子(如质子和中子)和介子。[2][3]因此,人们对夸克的大部分认识来自对强子的观察。

夸克有各种各样的内禀性质,包括电荷、质量、色荷和自旋。在粒子物理学标准模型中,夸克是唯一一种经历所有四个基本互相作用的基本粒子,基本相互作用有时也会被称为“基本力”(电磁力,万有引力,强相互作用力,和弱相互作用)。夸克同时也是唯一已知的基本电荷非整数的粒子。

夸克有六种"味",分别是上,下,粲, 奇,底和顶。在所有夸克中,上夸克和下夸克具有最低的质量。较重的夸克通过粒子衰变过程迅速转变成上或下夸克。粒子衰变是一个从较高质量状态转变成较低质量状态的过程。因此,上夸克和下夸克通常是稳定的,所以在宇宙中最常见,而奇夸克、粲夸克、底夸克和顶夸克只能在高能碰撞中产生(例如宇宙线的和粒子加速器)。夸克每一种味都有一种对应的反粒子,称为\textbf{反夸克},它与夸克的不同之处,在于它的一些特性跟夸克大小一样但正负不同。

夸克模型分别由物理学家默里·盖尔·曼和乔治·茨威格在1964年独立提出的。引入夸克这一概念,是为了能更好地整理各种强子,而当时并没有什么能证实夸克存在的物理证据,直到1968年SLAC开发出深度非弹性散射实验为止。夸克的六种味已经全部被加速器实验所观测到;而于1995年在费米实验室被观测到的顶夸克,是最后发现的一种。

\subsection{分类}
\begin{figure}[ht]
\centering
\includegraphics[width=14.25cm]{./figures/9a82ca6b63e8f20f.png}
\caption{标准模型中的六个夸克粒子(以紫色显示)。前三列中的每一列都形成了一个重要的世代。} \label{fig_Quark_1}
\end{figure}

标准模型是描述目前已知所有基本粒子的理论框架。这个模型包含了夸克的六个“味”, 分别味 上 (u),下(d),粲(s),奇(c),底(b),以及顶(t)中。[4] 夸克的反粒子由对应夸克符号上的条表示,例如u表示上反夸克。与一般的反物质一样,反夸克和其相应的夸克有相同的质量,平均寿命,自旋,但是电荷相反。[4]

夸克自旋为-1⁄2,根据自旋统计定理,它们是费米子,并服从泡利不相容原理,即相同的量子态不可以被多个费米子占据。这与玻色子(具有整数自旋的粒子)不同,即相同的量子态可以被多个玻色子占据。[5]与轻子不同,夸克拥有色荷,因此它们之间有强相互作用。这种吸引力使得夸克结合在一起,形成了复合粒子,称为强子。决定强子量子数的夸克被称为价夸克;除此之外,在强子内不定数量的虚“”夸克、反夸克和胶子并不影响其量子数。[6]强子有两个家族,它们分别是重子和介子,其中重子有三个价夸克,介子有一个价夸克和一个反夸克。[7]最常见的重子是质子和中子,它们是原子核的组成部分。目前发现很多强子(参见重子列表和介子列表),它们通常通过其夸克含量以及其性质来区分。拥有较多价夸克的“奇异”强子,例如从夸克模型预测的四夸克(qqqq)和五夸克(qqqqq),直到21世纪早期才被发现。[8][9][10][11]

基本费米子分为三代,每个都包含两个轻子和两个夸克。第一代包括上夸克和下夸克,第二代包括奇夸克和粲夸克,以及第三代包括底夸克和顶夸克。所有对第四代夸克和其他费米子的研究都失败了,[12][13]有强有力的间接证据表明不超过三代。[14][15][16]较高代的粒子通常质量较大,稳定性较低。它们通过弱相互作用衰变转化为下一代粒子。自然界中最常见的只有第一代(上下)夸克。较重的夸克只能在高能碰撞中产生(例如宇宙射线),然后迅速衰变。然而,人们认为夸克的出现开始于大爆炸时期,当时宇宙处于非常热和稠密的阶段。目前对重夸克的研究多是在人工创造的条件下进行的,例如粒子加速器。[17]

夸克具有电荷、质量、色荷和味这四个属性,是唯一已知的参与当代物理学所有四个基本相互作用理论的基本粒子。这四个基本相互作用力分别是,电磁相互作用力,万有引力,强相互作用力,和弱相互作用。除了在能量的极限(普朗克能量)和距离标度(普朗克距离)的条件下,涉及到粒子相互作用的粒子太弱了。目前,标准模型重不考虑万用引力作用,主要是因为没有成功的涉及万有引力的量子力学理论存在。

引力太弱,除了在能量的极限(普朗克能量)和距离标度(普朗克距离)中。然而,既然没有成功重力量子理论引力不是标准模型所描述的。

请参见下面的来查看夸克六种“味”的性质。

\subsection{历史}
\begin{figure}[ht]
\centering
\includegraphics[width=6cm]{./figures/62ccfb331751f703.png}
\caption{默里·盖尔·曼} \label{fig_Quark_2}
\end{figure}

\begin{figure}[ht]
\centering
\includegraphics[width=6cm]{./figures/5a8a3037822de78c.png}
\caption{乔治·茨威格} \label{fig_Quark_3}
\end{figure}

\begin{figure}[ht]
\centering
\includegraphics[width=6cm]{./figures/c4a65218fb231c65.png}
\caption{导致发现的事件的照片Σ++c重子,1974年在布鲁克黑文国家实验室} \label{fig_Quark_4}
\end{figure}






