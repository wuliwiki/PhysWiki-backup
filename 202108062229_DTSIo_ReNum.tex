% 实数

\subsection{从有理数到实数}

我们知道, 有理数集$\mathbb{Q}$是对四则运算封闭的最小的数系. 从正整数开始, 为了使得任意两个整数都能相减, 我们引入了零和负整数, 从而得到了整数集$\mathbb{Z}$; 而为了使得任意两个整数的除法都有意义 (当然, 要剔去除数为零的情形), 我们又引入了形如$m/n$的数, 从而得到了有理数集$\mathbb{Q}$. 有理数的英文 rational number 即来源于ratio (比例) 一词. 小学算术已经告诉我们, 有理数的和, 差, 积, 商都是有理数 (仍然需要假定除数不等于零), 而且对于任何有理数$r$都有$r+0=r$, $r\cdot1=r$. 用近代代数学的语言, 这表示有理数集构成了一个\textbf{域 (field)}.

但是我们也知道, 并非所有来自实际问题的度量对象都能用有理数来表示. 例如, 假若承认勾股定理 (在古希腊, 发现并证明它的是毕达哥拉斯), 那么直角边长为1的等腰直角三角形的斜边长$c$满足$c^2=2$. 毕达哥拉斯的门徒西帕索斯发现, 这个奇特的数$c$不能表示为两个整数的比. 西帕索斯的发现打击了毕达哥拉斯学派的信条"万物皆 (有理) 数", 因而被试图维护教义的门徒们杀害.

\begin{exercise}{$\sqrt{2}$是无理数}
利用数论中的素因子分解定理 (每个正整数都可以唯一分解成它的素因子乘积; 这件事并不是显然的), 证明不存在整数$m,n$使得$m^2=2n^2$. 更一般地, 如果$p$是素数, 那么不存在整数$m,n,k>1$使得$m^k=pn^k$.
\end{exercise}

然而