% 海因茨·霍普夫(综述)
% license CCBYSA3
% type Wiki

本文根据 CC-BY-SA 协议转载翻译自维基百科 \href{https://en.wikipedia.org/wiki/Heinz_Hopf}{相关文章}。

海因茨·霍普(Heinz Hopf,1894年11月19日-1971年6月3日)是德国数学家,研究领域包括动力系统、拓扑学和几何学。
\subsection{早期生活与教育}
霍普出生于德国帝国的格雷布申(现波兰弗罗茨瓦夫的格拉比辛),父亲是威廉·霍普,母亲是伊丽莎白(原姓基尔赫纳)。他的父亲出生为犹太人,霍普出生一年后父亲皈依了新教;母亲来自一个新教家庭。

霍普于1901年至1904年就读于卡尔·米特尔豪斯高年级男子学校,随后进入布雷斯劳的科尼格·威廉中学。他从小便显示出数学天赋。1913年,他进入上西里西亚的弗里德里希·威廉大学,聆听了恩斯特·施泰尼茨、阿道夫·克内塞尔、马克斯·德恩、厄尔哈德·施密特和鲁道夫·斯图尔姆的讲座。第一次世界大战爆发后,霍普积极参军,曾两次受伤,并于1918年获得铁十字勋章(一级)。

战争结束后,霍普继续在海德堡(1919/20冬季和1920年夏季)和柏林(从1920/21年冬季开始)继续他的数学教育。他在路德维希·比伯巴赫的指导下学习,并于1925年获得博士学位。
