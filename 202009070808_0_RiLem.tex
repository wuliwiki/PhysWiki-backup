% 里斯引理 (赋范空间)
% keys Riesz|范数|里斯引力|垂直|紧的

\begin{issues}
\issueNeedCite
\end{issues}

\pentry{范数\upref{NormV}}

\textbf{里斯引理(Riesz's lemma)} 是赋范空间理论中的一个基本引理. 它的表述如下:

\begin{lemma}{里斯引理 (Riesz's lemma)}
设 $X$ 是赋范空间, $\|\cdot\|$是它的范数, $M$是它的真闭子空间. 则任给$\alpha\in(0,1)$, 都存在$x$使得$\|x\|=1$, 且$\text{dist}(x,M)\geq\alpha$.
\end{lemma}

这个引理有鲜明的直观意义: 任给赋范空间中的“平面”, 总有一个单位向量与这平面“几乎垂直”.

\subsection{证明}
证明是直接的构造. 任取$y\in X\setminus M$, 命$\delta=\text{dist}(y,M)$. 由于$M$是闭的, 故$\delta>0$. 根据距离的定义, 任给$\varepsilon>0$, 皆存在$z_0\in M$使得
$$
\delta\leq \|y-z_0\|<\delta+\varepsilon.
$$
命$x=(y-z_0)/\|y-z_0\|$, 则$\|x\|=1$, 而且对于任何$z\in M$皆有
$$
\|x-z\|
=\left\|\frac{y-z_0}{\|y-z_0\|}-z\right\|
=\left\|\frac{y-(z_0+z\|y-z_0\|)}{\|y-z_0\|}\right\|.
$$
由于$z_0+z\|y-z_0\|\in M$, 所以上式右边大于
$$
\frac{\delta}{\delta+\varepsilon}.
$$
于是对于给定的$0<\alpha<1$, 只要取$\varepsilon$足够小即可使得
$$
\text{dist}(x,M)
=\inf_{z\in M}\|x-z\|
>\frac{\delta}{\delta+\varepsilon}
>\alpha.
$$

\subsection{应用举例}
如果$X$是无穷维的赋范空间, 那么存在递增的子空间族
$$
M_1\subset M_2\subset M_3\subset...,
$$
使得$\text{dim}(M_k)=k$. 有限维的子空间都是闭子空间, 所以根据里斯引理, 存在$x_k\in M_k\setminus M_{k-1}$使得
$$
\|x_k\|=1,\,\text{dist}(x_k,M_k)>\frac{1}{2}.
$$
这样一来, 恒有$\|x_m-x_n\|>1/2$成立, 于是$X$中闭单位球面上的序列$\{x_n\}$没有收敛子序列. 这表示\textbf{无穷维赋范空间中的闭单位球不可能是紧的.}