% 球套定理
% keys 完备性|球套
% license Usr
% type Tutor
\pentry{柯西序列、完备度量空间\nref{nod_cauchy}}{nod_55e7}
在分析学中,所谓区间套定理(\autoref{the_RCompl_3})被广泛的应用。在度量空间理论中,本节所谓的球套定理也其中类似的作用。

首先需要明确度量空间 $(X,d)$ 的\textbf{球}是指:以某点 $x_0\in X$ 为中心,正实数 $r\in\mathbb R^+$ 为半径的集合 
\begin{equation}
B_{r}(x_0):=B(x_0,r):=\{x|d(x,x_0)\leq(or <)r,x\in X\}.~
\end{equation}
若上式中是 $\leq$ 则称为\textbf{闭球},若是 $<$ 则称为\textbf{开球}。
\begin{definition}{球套}\label{def_SNesT_1}
设 $(X,d)$ 是度量空间,序列 $\{B_n\}$ ($B_n:=B(x_n,r_n)$)中的每一个 $B_i$ 都是 $X$ 中的球。若 $B_{i+1}\subset B_{i},i=1,2,\ldots$,则称 $\{B_n\}$ 为 $X$ 上的\textbf{球套}。若球套中每一球都是闭球,则球套称为\textbf{闭球套};若每一球都是开球,则称为\textbf{开球套}。
\end{definition}

\begin{theorem}{球套定理}\label{the_SNesT_1}
度量空间 $(X,d)$ 是完备的充要条件是:$X$ 中半径趋于0的任一闭球套的有非空的交。即若 $\{B_n\}$ 是 $X$ 闭球套,且 $\lim\limits_{n\rightarrow\infty}r_n=0$,那么 $\bigcap\limits_n B_n\neq\emptyset$。
\end{theorem}

\textbf{证明:}

\textbf{必要性:}设 $(X,d)$ 是完备的,并设 $\{B_n\}$ 是其上的任一闭球套,其中 $B_n$ 的球心为 $x_n$,半径为 $r_n$。则序列 $\{x_n\}$ 是柯西序列。事实上,当 $m>n$ 时,$d(x_m,x_n)<r_n$(根据球套的定义),而 $\lim\limits_{n\rightarrow\infty}r_n=0$。这就是说,任一 $\epsilon>0$,存在 $N$,只要 $n\geq N$,就有 $d(x_m,x_n)<\epsilon$,因此 $\{x_n\}$ 是柯西序列。

由于完备性, $\{x_n\}$ 的极限存在,设 $x=\lim\limits_{n\rightarrow\infty}x_n$。$B_n$ 显然包含所有的点 $x_i,i\geq n$,于是 $x$ 的任一邻域都包含有 $B_n$ 的点,因而 $x\in B_n,n=1,2,\ldots$,进而
$x\in\bigcap\limits_n B_n$。即闭球套有非空的交。

\textbf{充分性:}设 $\{x_n\}$ 是柯西序列。若能以该序列构造(通过 $\{x_n\}$ 的某一子序列)半径趋于0的闭球套,则由定理假设知该闭球套具有公共点。由于该公共点的每一邻域都包含从某一项 $n$ 开始的(子序列的)所有点,那么它就是(子序列的)极限点。从而 $\{x_n\}$ 收敛到该极限点。我们构造如下:

在 $\{x_n\}$ 中选取这样的点 $x_{n_1}$ 作为半径为1的闭球 $B_1$ 的中心,使得一切 $n\geq n_1$,$d(x_n,x_{x_1})<1/2$(根据柯西序列\autoref{def_cauchy_3})。然后在 $\{x_n\}$ 中选取 $x_{n_2}$ 作为半径为1/2的闭球 $B_2$ 的中心,使得 $n_2>n_1$,且对一切 $n\geq n_2$,$d(x_n,x_{x_2})<1/2^2$。以此类推,我们总是选取满足 $n_{k+1}>n_k$,且对一切 $n\geq n_{k+1}$,$d(x_n,x_{n_{k+1}})<1/2^{k+1}$ 的 $x_{n_{k+1}}$ 作为半径为 $1/2^k$ 的球 $B_{k+1}$ 的中心。由于对 $k=1,\ldots,$ 成立
\begin{equation}
\begin{aligned}
d(z,x_{n_k})&\leq d(z,x_{n_{k+1}})+d(x_{n_k},x_{n_{k+1}})\\
&<1/2^{k+1}+1/2^k\\
&=3/2^{k+1}\\
&<1/2^{k-1}.
\end{aligned}~
\end{equation}
所以上面构造的闭球序列是闭球套。因此,它们有非空的交,从而就是序列 $\{x_{n_k}\}$ 的极限点。从而就是柯西序列 $\{x_n\}$ 的极限。

\textbf{证毕!}









