% 共轭与共轭类
% 共轭|共轭类|共轭元素|等价关系|conjugate

\begin{issues}
\issueDraft
\end{issues}
\begin{definition}{共轭}
若有群元$d,f\in G$,且$\exists g\in G$使得$gdg^{-1}=f$,则称群元$d$与群元$f$共轭。记作$d$~$f$。
\end{definition}

共轭关系是一个等价关系,满足自反率、对称率和传递率:\\
自反率:$g=ggg^{-1}$,则$g$~$g$。 \\
对称率:若$d$~$f$,则$\exists g\in G$使得$gdg^{-1}=f$,那么$d=g^{-1}fg=
g^{-1}f(g^{-1})^{-1}$,则$f$~$d$ \\
传递率:若$d$~$f$,$f$~$h$,则$\exists g_1,g_2\in G$,使$h=g_2fg_2^{-1}=
g_2g_1dg_1^{-1}g_2^{-1}$ $=g_2g_1d(g_2g_1)^{-1}$,则有$d$~$h$。

这一概念与矩阵中的相似矩阵类似。

\begin{definition}{共轭类}
群中所有相互共轭的元素的集合称为群的一个共轭类。
\end{definition}

\begin{example}{求$D3$群的共轭类}
首先列出$D3$群的乘法表:
\begin{table}[ht]
\centering
\caption{$D3$群乘法表}\label{gpcon_tab1}
\begin{tabular}{|c|c|c|c|c|c|c|}
\hline
        $D3$ & $~e~$ & $~d~$ & $~f~$ & $~a~$ & $~b~$ & $~c~$ \\ \hline
        $e$ & $e$ & $d$ & $f$ & $a$ & $b$ & $c$ \\ \hline
        $d$ & $d$ & $f$ & $e$ & $b$ & $c$ & $a$ \\ \hline
        $f$ & $f$ & $e$ & $d$ & $c$ & $a$ & $b$ \\ \hline
        $a$ & $a$ & $c$ & $b$ & $e$ & $f$ & $d$ \\ \hline
        $b$ & $b$ & $a$ & $c$ & $d$ & $e$ & $f$ \\ \hline
        $c$ & $c$ & $b$ & $a$ & $f$ & $d$ & $e$ \\ \hline
\end{tabular}
\end{table}

从图中可以看出有以下关系:$a^{-1}da=f$,$d^{-1}ad=b$,$d^{-1}bd=c$














\end{example}