% 正交子空间
% 正交|子空间|内积|矢量空间|高等代数

\begin{issues}
\issueTODO
\end{issues}

\pentry{内积\upref{InerPd}, 直和\upref{DirSum}, 张成空间\upref{VecSpn}}

\begin{definition}{正交子空间}
一个内积空间 $V$ 中, 如果两个子空间 $V_1$ 和 $V_2$ 任意各选一个矢量 ${v_1}$ 和 ${v_2}$, 它们的内积都有 $\ev{v_1, v_2} = 0$, 那么我们就说者两个子空间是\textbf{正交}的.
\end{definition}

构造正交子空间的一种简单的方法是, 在 $V$ 中找到两组矢量 $x_1, \dots, x_m$ 和 $y_1, \dots, y_m$, 确保对任意 $x_i$ 和 $y_j$ 正交, 那么 $x_1, \dots, x_m$ 张成\upref{VecSpn}的子空间必定和 $y_1, \dots, y_m$ 张成的子空间正交.

\begin{theorem}{}
从基底的角度来看, 两个空间正交的充分必要条件是: 如果从两空间各选一组基底 ${\alpha_i}$ $(i = 1, \dots, N_1)$ 和 ${\beta_i}$ $(i = 1, \dots, N_2)$, 有对任意 $i, j$ 都有 $\ev{\alpha_i, \beta_j} = 0$.
\end{theorem}

\begin{exercise}{}
证明两个正交子空间中, 只有零矢量是共同矢量.
\end{exercise}

\subsection{正交子空间的直和}

若两个正交子空间的维数分别为 $N_1$ 和 $N_2$, 它们之和等于母空间的维数 $N$, 那么就说它们是\textbf{互补(complementary)}的. 若分别在这两个空间中取一组基底, 那么将他们合并起来就得到了母空间中的一组基底.

特殊地, 如果 $V_1$ 和 $V_2$ 的两组基底合并后仍然正交归一, 那么合并后就得到了直和空间 $V = V_1 \oplus V_2$ 中的一组正交归一基底. 但注意直和空间中的任意一组正交归一基底未必可以划分为 $V_1$, $V_2$ 空间中的两组基底.

\begin{example}{}
三维几何矢量空间中, 建立直角坐标系, 那么 $\uvec x$ 和 $\uvec y$ 张成的二维矢量空间(平面)与 $\uvec z$ 张成的一维矢量空间(直线)正交.
\end{example}

\begin{example}{}
虽然 $xy$ 平面和 $xz$ 平面是两个垂直的平面, 但它们并不是两个正交子空间. 例如矢量 $\uvec x$ 是两个平面共同的矢量, 但 $\uvec x$ 和它本身不正交.
\end{example}

\subsection{正交补}
我们在 “直和\upref{DirSum}” 中已经定义了补空间的概念, 现在来定义一种特殊的补空间.
\begin{definition}{正交补空间}\label{OrthSp_def1}
在 $V$ 空间中, 若 $V_1$ 和 $V_2$ 正交且 $V = V_1 \oplus V_2$, 那么 $V_1$ 和 $V_2$ 互为对方的\textbf{正交补空间(Orthogonal complement)}, 简称\textbf{正交补}.
\end{definition}

\begin{theorem}{}\label{OrthSp_the1}
从基底的角度来看, 两个空间正交的充分必要条件是: 如果从两空间各选一组基底 ${\alpha_i}$ $(i = 1, \dots, N_1)$ 和 ${\beta_i}$ $(i = 1, \dots, N_2)$, 那么 $\{\alpha_1, \dots, \alpha_{N_1}, \beta_1, \dots, \beta_{N_2}\}$ 就是 $X$ 的一组正交归一基底.
\end{theorem}

\addTODO{正交补是唯一的}
\addTODO{如何求正交补?}
