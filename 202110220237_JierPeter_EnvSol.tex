% 包络和奇解
% 常微分方程

\subsection{包络}

\begin{definition}{包络}

在$x-y$平面上,定义一族曲线$\{\Phi_c\}$,其中$c$是一个连续参数.每条曲线$\Phi_c$的表达式为$f(x, y, c)=0$.

如果存在一条曲线$\Phi$,它本身不是任何一条$\Phi_c$,但是在$\Phi$的每一点处都有一条$\Phi_c$和它相切,那么我们就称$\Phi$是曲线族$\{\Phi_c\}$的\textbf{包络(envolope)}.

\end{definition}

\begin{example}{包络的例子}
考虑曲线族$\{\Phi_c\}$,其中$\Phi_c$的表达式为
\begin{equation}
x^2+cx-y=0
\end{equation}



\end{example}













