% 例: 有限维方阵

\pentry{
矩阵的本征方程\upref{MatEig}
有界算子的谱\upref{BddSpe}
有界算子的预解式\upref{BddRsv}
谱投影\upref{SpePrj}
}

如果将之前提到的泛函分析概念应用于有限维方阵, 则更容易看出它们的意义, 加深直观理解. 

这里一直设$A$是$n\times n$复矩阵. 我们把它视为$\mathbb{C}^n$到自己的一个线性算子. 如词条 有界算子的谱\upref{BddSpe} 所说, 谱集$\sigma(A)$恰为$A$的特征值的集合, 而谱半径$r(A)$当然就是特征值的最大模. 

现在设$\sigma(A)=\{\lambda_1,...,\lambda_m\}$(重数大于1的特征值算作一个谱点). 我们可以把预解式$R(z;A)=(z-A)^{-1}$视为矩阵值亚纯函数, 而显然$\sigma(A)$就是它的极点集. 我们更可以借助矩阵的若尔当分解而写出$R(z;A)$的明显表达式. 

首先来回忆若尔当标准型分解定理:

\begin{theorem}{若尔当分解}
\begin{enumerate}
\item 每个特征值$\lambda_k$都对应了$A$的一个不变子空间$V^{(k)}$, 全空间恰等于诸$V^{(k)}$的直和. $V_k$的维数$n_k$恰等于$\lambda_k$的代数重数 (即$\lambda_k$作为特征多项式$\det(z-A)$根的重数).
\item 设$\lambda_k$的几何重数 (即线性无关的特征向量的个数) 是$m_k$. 则不变子空间$V^{(k)}$中存在一组线性无关的向量$e^{(k)}_1,...,e^{(k)}_{m_k}$, 使得子空间
$$
V^{(k)}_l:=\{e^{(k)}_l,(A-\lambda_k)e^{(k)}_l,...(A-\lambda_k)^{m_k}e^{(k)}_l\}
$$
彼此交集为零, 且有直和$V^{(k)}=\bigoplus_{l=1}^{m_k}V^{(k)}_l$. 每个$V^{(k)}_l$(称为一个循环子空间) 恰含有一个对应于$\lambda_k$的线性无关的特征向量.
\end{enumerate}
\end{theorem}
