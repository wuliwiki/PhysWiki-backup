% 2016 年计算机学科专业基础综合全国联考卷
% keys 2016 计算机 考研 全国联考

\subsection{一、单项选择题}
1~40小题,每小题2分,共80分.下列每题给出的四个选项中.只有一个选项符合试题要求.

1.已知表头元素为c的单链表在内存中的存储状态如下表所示.
\begin{table}[ht]
\centering
\caption{第1题表}\label{CSN16_tab1}
\begin{tabular}{|c|c|c|}
\hline
地址 & 元素 & 链接地址 \\
\hline
1000H & a & 1010H \\
\hline
1004H & b & 100CH \\
\hline
1008H & c & 1000H \\
\hline
100CH & d & NULL \\
\hline
1010H & e & 1004H \\
\hline
1014H &   &  \\
\hline
\end{tabular}
\end{table}
现将f存放于1014H处并插入到单链表中,若f在逻辑上位于a和e之间,则a,e,f的“链接地址”依次是 \\
A.1010H,1014H,1004H $\qquad$ B.1010H,1004H,1014H \\
C.1014H,1010H,1004H $\qquad$ D.1014H,1004H,1010H

2.已知一个带有表头结点的双向循环链表L,结点结构为 \\
\begin{table}[ht]
\centering
\caption{第2题表}\label{CSN16_tab2}
\begin{tabular}{|c|c|c|}
\hline
prev & data & next \\
\hline
\end{tabular}
\end{table}
其中,prev和next分别是指向其直接前驱和直接后继结点的指针.现要删除指针p所指的结点,正确的语句序列是 \\
A. p->next->prev=p->prev; p->prev->next=p->prev; free (p); \\
B. p->next->prev=p->next; p->prey-> next=p->next; free (p); \\
C. p->next->prev=p->next; p->prev->next=p->prev; free (p); \\
D. p-> next-> prey=p->prey; p->prev->next=p->next; free (p);

3.设有如下图所示的火车车轨,入口到出口之间有n条轨道,列车的行进方向均为从左至右,列车可驶入任意一条轨道.现有编号为1~9的9列列车,驶入的次序依次是8,4,2,5,3,9,1,6,7.若期望驶出的次序依次为1~9,则n至少是 \\
\begin{figure}[ht]
\centering
\includegraphics[width=14.25cm]{./figures/CSN16_1.png}
\caption{第3题图} \label{CSN16_fig1}
\end{figure}
A. 2 $\qquad$ B.3 $\qquad$ C.4 $\qquad$ D.5

4.有一个$100$阶的三对角矩阵$M$,其元素$m_{i,j}$($1$≤$i$≤$100$,$1$≤$j$≤$100$)按行优先次序压缩存入下标从0开始的一维数组$Ⅳ$中.元素$m_{30}$,$30$在$N$中的下标是 \\
A.86 $\qquad$ B.87 $\qquad$ C.88 $\qquad$ D.89