% 巨正则系综(综述)
% license CCBYSA3
% type Wiki

本文根据 CC-BY-SA 协议转载翻译自维基百科\href{https://en.wikipedia.org/wiki/Grand_canonical_ensemble}{相关文章}。

在统计力学中,巨正则系综(grand canonical ensemble,亦称宏观正则系综)是一种用于描述与热库处于热力学平衡(热平衡与化学平衡)的粒子力学系统可能状态的统计系综\(^\text{[1]}\)。该系综中的系统具有开放性特征——系统可与热库交换能量和粒子,因此系统的不同微观态可能具有不同的总能量与总粒子数。系统的体积、形状及其他外部坐标在所有可能微观态中保持恒定。

巨正则系综的热力学变量为化学势(符号:\(\mu\))与绝对温度(符号:\(T\))。该系综的特性还受体积(符号:\(V\))等力学变量的影响,这些变量决定了系统内部状态的性质。因此,该系综常被称为\(\mu VT\)系综,因为这三个物理量均为该系综的固定参数。
\subsection{基础概念}
在统计力学中,巨正则系综的基础概念可表述如下:该系综对每个微观态赋予的概率\(P\)由以下指数形式给出:
\[
P = e^{(\Omega + \mu N - E)/(kT)},~
\]
式中,\(N\)表示微观态中的粒子数,\(E\)为微观态的总能量,\(k\)为玻尔兹曼常数。  

物理量Ω称为巨势,对特定系综而言为常数。当选择不同的\(\mu,V,T\)参数时,概率分布和\(\Omega\)值将发生改变。巨势Ω具有双重作用:其一作为概率分布的归一化因子(所有微观态的概率之和必须为1);其二可通过函数\(\Omega(\mu,V,T)\)直接计算许多重要的系综平均值。  

当系统允许多种粒子数变化时,概率表达式推广为:  
\[
P = e^{(\Omega + \mu_1N_1 + \mu_2N_2 + \ldots + \mu_sN_s - E)/(kT)},~
\]  
其中,\(\mu_1\)表示第一种粒子的化学势,\(N_1\)为对应微观态中该种粒子数,依此类推(\(s\)为不同粒子种类总数)。需注意粒子数的精确定义(参见下文关于粒子数守恒的注释)。  

部分学者将巨正则系综的分布称为广义玻尔兹曼分布\(^\text{[2]}\)。

巨正则系综特别适用于描述以下系统:系统的几何形状固定,但因与热库接触(例如导体中的电子对应电接地,空腔中的光子对应吸光表面),其能量和粒子数可发生显著涨落。该系综为严格推导无相互作用量子粒子系统的费米-狄拉克统计或玻色-爱因斯坦统计提供了自然框架(参见下文示例)。 

\textbf{公式注记} 

该概念的另一表述形式将概率写为:\(P = \frac{1}{\mathcal{Z}} e^{(\mu N - E)/(kT)},\)  
其中通过巨配分函数\(\mathcal{Z} = e^{-\Omega/(kT)}\)替代了巨势Ω。本文中以巨势为基础的公式,均可通过简单的数学变换,利用巨配分函数重新表述。  
\subsection{适用性}  
巨正则系综是描述与一个热库保持热平衡和化学平衡的孤立系统可能状态的系综(推导过程类似于热浴推导正常正则系综的方法,可以在Reif\(^\text{[3]}\)中找到)。巨正则系综适用于任何巨小的系统,无论是小型还是巨型;只需要假设与其接触的热库要巨得多(即取宏观极限)。

系统被孤立的条件是必要的,以确保其具有明确的热力学量和演化。\(^\text{[1]}\)然而,在实际应用中,通常希望使用巨正则系综来描述与热库直接接触的系统,因为正是这种接触保证了系统的平衡。在这些情况下,使用巨正则系综通常是通过以下方式之一来合理化的:1)假设接触是微弱的,或者 2)将一部分热库的连接纳入到分析中的系统中,以便正确地模拟连接对感兴趣区域的影响。另一种方法是使用理论方法来建模连接的影响,从而得到一个开放的统计系综。

另一个出现巨正则系综的情况是考虑一个巨型的热力学系统(即“与自身处于平衡状态的系统”)。即使系统的精确条件实际上不允许能量或粒子数的变化,巨正则系综仍然可以用来简化某些热力学性质的计算。原因在于,一旦系统非常巨,各种热力学系综(微正则系综、正则系综)在某些方面会变得与巨正则系综等效。\(^\text{[注 1]}\)当然,对于小系统,即使在均值上,不同的系综也不再是等效的。因此,当应用于粒子数固定的小系统(如原子核)时,巨正则系综可能会非常不准确。\(^\text{[4]}\)
\subsection{性质}  
\begin{itemize}
\item 唯一性:对于给定温度和化学势的系统,大正则系综是唯一确定的,并且不依赖于任意选择,如坐标系的选择(经典力学)或基底的选择(量子力学)。\(^\text{[1]}\) 大正则系综是唯一一个在常温、常体积和常化学势下重现基本热力学关系的系综。\(^\text{[5]}\)  
\item 统计平衡(稳态):尽管底层系统在不断运动,大正则系综本身并不会随时间演化。实际上,系综仅仅是系统守恒量(能量和粒子数)的函数。\(^\text{[1]}\)  
\item 与其他系统的热平衡和化学平衡:两个系统,各自由大正则系综描述,且具有相等的温度和化学势,若接触并达到热平衡和化学平衡[note 2],它们将保持不变,且最终合并的系统将由同样的温度和化学势的大正则系综描述。\(^\text{[1]}\) 
\item 最大熵:对于给定的机械参数(固定\(V\)),大正则系综中对对数概率的平均值 \( -\langle \log P \rangle \)(也称为“熵”)是对于任何具有相同 \( \langle E \rangle \)、 \( \langle N_1 \rangle \) 等的系综(即概率分布\(P\))所能达到的最大值。\(^\text{[1]}\) 
\item 最小大势能:对于给定的机械参数(固定\(V\))以及给定的\(T,\mu_1,\cdots,\mu_s\)的值,大正则系综的平均值\({\displaystyle \left\langle E + kT\log P - \mu_{1}N_{1} - \ldots \mu_{s}N_{s} \right\rangle}\)是任何系综中可能的最小值。\(^\text{[1]}\)
\end{itemize}

