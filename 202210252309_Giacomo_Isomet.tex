% 等距映射与保形映射
% isometry|conformal map|等距映射|保形映射

\subsection{等距变换}
等距变换,或称等度量变换,是指在变换前后不改变切向量的长度的变换——更一般地,不改变任何两个切向量的内积.

\begin{definition}{等距变换}
对于两个正则曲面 $S$ 和 $R$,如果存在同胚映射 $\varphi:S\to R$,使得对于任意 $p\in S$ 和任意两个 $w_1, w_2\in T_pS$,都有 $w_1\cdot w_2=\dd\varphi_p(w_1)\cdot\dd\varphi_p(w_2)$,那么我们说 $\varphi$ 是一个\textbf{等距变换(isometry)},称 $S$ 和 $R$ 是\textbf{等度量的(isometric)}.
\end{definition}

\begin{definition}{局部等距变换}
对于两个正则曲面 $S$ 和 $R$,如果存在同胚映射 $\varphi:S\to R$,使得在某一个 $p\in S$ 处有一个邻域,在这个邻域里 $\varphi$ 是等距映射,则称 $\varphi$ 是在 $p$ 处的\textbf{局部等距变换(local isometry)}.如果存在 $\varphi$ 在每一个点上都是局部等距变换,则称 $S$ 和 $R$ 是\textbf{局部等距的(locally isometric)}.
\end{definition}

局部等距的例子可以依靠以下定理来寻找.

\begin{theorem}{}
对于两个正则曲面 $S$ 和 $R$,给定两个局部坐标系 $\bvec{x}:U\to S$ 和 $\bvec{y}:V\to R$,使得两个坐标系下计算的 $E, G, F$\footnote{见\autoref{FForm_def1}~\upref{FForm}.}相等,则映射 $\varphi=\bvec{y}\circ\bvec{x}:\bvec{x}(U)\to R$ 是一个局部等距变换.
\end{theorem}

\subsection{保形映射}

等距变换是保形变换的一个特例.等距变换过于简单,因此我们引入范围更广一些的保形映射,比起前者有更多有趣的结构.

\begin{definition}{保形映射}
对于两个正则曲面 $S$ 和 $R$,如果存在同胚映射 $\varphi:S\to R$,使得对于任意 $p\in S$ 和任意两个 $w_1, w_2\in T_pS$,都有\begin{equation}
\dd\varphi_p(w_1)\cdot\dd\varphi_p(w_2)=\lambda^2(p)w_1\cdot w_2
\end{equation}
其中 $\lambda$ 是 $S$ 上的一个\textbf{处处不为零}的可微函数,则称 $\varphi$ 是一个\textbf{保形映射(conformal map)},也叫\textbf{共形映射};$S$ 和 $R$ 就被称为是\textbf{共形(conformal)}的.
\end{definition}

\begin{definition}{局部保形映射}
对于两个正则曲面 $S$ 和 $R$,如果存在同胚映射 $\varphi:S\to R$,使得在某一个 $p\in S$ 处有一个邻域,在这个邻域里 $\varphi$ 是保形映射,则称 $\varphi$ 是在 $p$ 处的\textbf{局部共形映射(local conformal map)}.如果存在 $\varphi$ 在每一个点上都是局部保形变换,则称 $S$ 和 $R$ 是\textbf{局部共形的(locally conformal)}.
\end{definition}

类似等距变换,我们也有以下定理.

\begin{theorem}{}
对于两个正则曲面 $S$ 和 $R$,给定两个局部坐标系 $\bvec{x}:U\to S$ 和 $\bvec{y}:V\to R$,使得两个坐标系下计算的 $E, G, F$ 只差一个因子 $\lambda^2(p)$,而 $\lambda(p)$ 是 $S$ 上处处不为零的可微函数,则映射 $\varphi=\bvec{y}\circ\bvec{x}:\bvec{x}(U)\to R$ 是一个局部等距变换.
\end{theorem}

由正则曲面的定义,我们还容易得到:

\begin{theorem}{}
任意两个正则曲面必然是局部共形的.
\end{theorem}



