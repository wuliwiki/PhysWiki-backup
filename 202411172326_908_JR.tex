% 詹姆斯·焦耳(综述)
% license CCBYSA3
% type Wiki

本文根据 CC-BY-SA 协议转载翻译自维基百科\href{https://en.wikipedia.org/wiki/James_Prescott_Joule}{相关文章}。

詹姆斯·普雷斯科特·焦耳 (James Prescott Joule) FRS FRSE (/dʒuːl/;[1][2][a] 1818年12月24日—1889年10月11日) 是一位英国物理学家、数学家和酿酒师,出生于兰开夏郡的索尔福德。焦耳研究了热的本质,并发现了热与机械功之间的关系。这一发现促成了能量守恒定律的提出,进而导致了热力学第一定律的发展。国际单位制中能量的导出单位“焦耳”(joule)即以他的名字命名。

他与开尔文勋爵(Lord Kelvin)合作,开发了一个绝对热力学温标,后被称为开尔文温标(Kelvin scale)。焦耳还观察到了磁致伸缩现象,并发现了电阻中的电流与其散热之间的关系,即焦耳第一定律(Joule's first law)。他关于能量转化的实验成果首次发表于1843年。

\subsection{早年生活} 
詹姆斯·焦耳出生于1818年,其父本杰明·焦耳(Benjamin Joule,1784–1858)是一位富有的酿酒师,母亲是爱丽丝·普雷斯科特(Alice Prescott)。他出生在索尔福德的新贝利街(New Bailey Street)。[3] 焦耳年轻时接受了著名科学家约翰·道尔顿(John Dalton)的指导,并深受化学家威廉·亨利(William Henry)以及曼彻斯特工程师彼得·尤尔特(Peter Ewart)和伊顿·霍奇金森(Eaton Hodgkinson)的影响。他对电学充满了兴趣,他和兄弟曾通过相互施加电击以及给家里的仆人施加电击进行实验。[4]  

成年后,焦耳接管了家族的酿酒业务,科学对他来说只是一个严肃的爱好。大约在1840年,他开始研究用新发明的电动机取代酿酒厂的蒸汽机的可行性。他关于这一主题的首批科学论文发表于威廉·斯特金(William Sturgeon)的《电学年鉴》(*Annals of Electricity*)。焦耳还是伦敦电学会(London Electrical Society)的成员,该学会由斯特金等人创立。[需要引用]  

部分出于商人对经济效益量化的需求,部分出于科学上的好奇心,焦耳开始研究哪种原动机效率更高。1841年,他发现了“焦耳第一定律”:“\textbf{任何伏打电流的正确作用所产生的热量,与该电流强度的平方乘以其所经历的导电阻力成正比。}”[5] 他进一步认识到,在蒸汽机中燃烧一磅煤比在电池中消耗一磅昂贵的锌更为经济。焦耳用一个通用标准来衡量不同方法的输出——将一磅重的物体提升一英尺的能力,即“英尺-磅”标准。[需要引用]  

然而,焦耳的兴趣逐渐从单纯的财务问题转向探讨从某一给定能源中可以提取多少功,这使他开始思考能量的可转化性。1843年,他发表了实验结果,显示他在1841年量化的加热效应是由于导体内的热量生成,而非热量从设备的其他部分转移而来。这一发现直接挑战了热质理论,该理论认为热量既不能被创造也不能被销毁。自1783年安托万·拉瓦锡(Antoine Lavoisier)提出以来,热质理论一直在热学领域占据主导地位。拉瓦锡的声望以及萨迪·卡诺(Sadi Carnot)自1824年以来在热机理论中的成功实践,确保了热质理论的影响力。然而,年轻的焦耳既不属于学术界,也不属于工程专业,他的研究因此面临重重困难。热质理论的支持者通常会指出珀尔帖–塞贝克效应的对称性,以此声称热量和电流在一定程度上是可以通过可逆过程相互转化的。[需要引用]  