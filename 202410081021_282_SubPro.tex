% 进程和子进程笔记(Linux/C++)
% license Usr
% type Note

\subsection{exec 系列命令}
如果不改变环境变量也可以用 \verb`execv()`,完整的例子如
\begin{lstlisting}[language=cpp]
#include <unistd.h>    // For execv()
#include <sys/wait.h>  // For wait()
#include <iostream>    // For std::cout
#include <cstdio>      // For perror()

int main() {
    // Define the command and its arguments
    char *args[] = {(char *)"/bin/ls",
        (char *)"-l", (char *)NULL};

    // Fork a new process
    pid_t pid = fork();

    if (pid == -1) {
        // Fork failed
        perror("fork failed");
        return 1;
    } else if (pid == 0) {
        std::cout << "running ls command" << std::endl;
        // Replace the child process with a new program (ls -l)
        execv("/bin/ls", args);
        // If execv() returns, an error occurred
        perror("execv failed");
        return 1;
    } else {
        // Parent process
        int status;
        // Wait for the child to complete. pid is now of the child process
        waitpid(pid, &status, 0);
        std::cout << "Child process finished with status: "
            << status << std::endl;
    }
}
\end{lstlisting}
该程若运行成功会把 \verb`ls` 的输出打到 stdout。

\begin{itemize}
\item C/C++ 中最万能的子进程函数是 \verb`int execve(const char *path, char *const argv[], char *const envp[]);`
\begin{lstlisting}[language=cpp]
char *args[] = {"/bin/ls", "-l", NULL};
char *env[] = {"HOME=/usr/home", "LOGNAME=home", NULL};
execve("/bin/ls", args, env);
\item \verb`getpid()` 可以获取当前进程的 pid
\item 子进程中用 \verb`getppid()` 获取父进程的 pid
\item 对子进程和父进程, \verb`pid_t pid = fork()` 分别返回 0 和子进程的 pid。
\end{lstlisting}
\end{itemize}

\subsection{pipe 管道编程}
\begin{itemize}
\item \verb`popen()` 可以读或写入管道,但不能又读又写
\end{itemize}

写管道
\begin{lstlisting}[language=cpp]
#include <cstdio>
#include <cstdlib>
int main() {
    FILE* fp = popen("wc -w", "w");
    if (fp == nullptr) {
        perror("popen failed");
        exit(EXIT_FAILURE);
    }
    const char* input = "Hello, how many words are here?\n";
    fprintf(fp, "%s", input);
    pclose(fp);
    return 0;
}
\end{lstlisting}
读管道
\begin{lstlisting}[language=cpp]
FILE* fp = popen("ls -l", "r");
if (fp) {
    char buffer[128];
    while (fgets(buffer, sizeof(buffer), fp) != nullptr) {
        printf("%s", buffer);
    }
    pclose(fp);
}
\end{lstlisting}

如果要同时读写管道,需要用 \verb`pipe()` 和 \verb`dup2()`
\begin{itemize}
\item \verb`int pipe(int pipefd[2]);` 可以创建一个管道,使一个进程可以从 \verb`pipefd[1]` 写入,另一个进程从 \verb`pipefd[0]` 读取。
\item \verb`dup2(源fd,目标fd)` (fd 是 file descriptor),把 \verb`目标fd` 打开的文件(如果有)先关掉,然后将其指向 \verb`源fd` 的文件。 成功后, 两个 fd 就指向同一个内核中的 file description,包括错误状态,文件中的当前位置全部都共享。 
\end{itemize}

\begin{lstlisting}[language=cpp]
#include <unistd.h>
#include <sys/wait.h>
#include <stdio.h>
#include <stdlib.h>
#include <string.h>

int main() {
    int pipe_stdin[2];  // Pipe for child's stdin
    int pipe_stdout[2]; // Pipe for child's stdout
    pid_t pid;

    // Create pipes
    if (pipe(pipe_stdin) == -1 || pipe(pipe_stdout) == -1) {
        perror("pipe failed");
        exit(EXIT_FAILURE);
    }

    // Fork a child process
    pid = fork();
    if (pid == -1) {
        perror("fork failed");
        exit(EXIT_FAILURE);
    }

    if (pid == 0) {
        // Child process

        // Redirect child's stdin to pipe_stdin[0]
        dup2(pipe_stdin[0], STDIN_FILENO);
        close(pipe_stdin[0]);
        close(pipe_stdin[1]);  // Close unused write end

        // Redirect child's stdout to pipe_stdout[1]
        dup2(pipe_stdout[1], STDOUT_FILENO);
        close(pipe_stdout[0]); // Close unused read end
        close(pipe_stdout[1]);

        // Execute the command
        // Replace 'cat' with your command
        execl("/bin/sh", "sh", "-c", "cat", NULL);

        // If execl fails
        perror("execl failed");
        exit(EXIT_FAILURE);
    } else {
        // Parent process
        close(pipe_stdin[0]);  // Close unused read end of stdin pipe
        close(pipe_stdout[1]); // Close unused write end of stdout pipe

        // Write to the child's stdin
        const char* input = "Hello, child process!\n";
        write(pipe_stdin[1], input, strlen(input));

        // Close stdin write end after writing
        close(pipe_stdin[1]);

        // Read the child's stdout
        char buffer[128];
        ssize_t count;
        while ((count = read(pipe_stdout[0],
            buffer, sizeof(buffer) - 1)) > 0) {
            buffer[count] = '\0';  // Null-terminate the buffer
            printf("Child output: %s", buffer);
        }

        close(pipe_stdout[0]); // Close stdout read end

        // Wait for the child to terminate
        wait(NULL);
    }

    return 0;
}
\end{lstlisting}
