% 威廉·哈密顿(综述)
% license CCBYSA3
% type Wiki

本文根据 CC-BY-SA 协议转载翻译自维基百科\href{https://en.wikipedia.org/wiki/William_Rowan_Hamilton}{相关文章}。

威廉·罗温·汉密尔顿爵士(1805年8月4日-1865年9月2日)是爱尔兰的数学家、物理学家和天文学家。他曾担任都柏林三一学院的天文学安德鲁斯教授。

汉密尔顿是1827年至1865年间邓辛克天文台的第三任台长。他的职业生涯包括对几何光学、傅里叶分析和四元数的研究,其中四元数使他成为现代线性代数的奠基人之一。他在光学、经典力学和抽象代数方面做出了重要贡献。他的工作是现代理论物理的基础,特别是他对牛顿力学的重新表述。汉密尔顿力学,包括其汉密尔顿函数,现在在电磁学和量子力学中都占据着核心地位。

\subsection{早年生活}  
汉密尔顿是莎拉·哈顿(1780年–1817年)和阿奇博尔德·汉密尔顿(1778年–1819年)的第九个孩子,他们住在都柏林的多米尼克街29号,后改为36号。汉密尔顿的父亲来自都柏林,曾担任律师。三岁时,汉密尔顿被送去与他的叔叔詹姆斯·汉密尔顿一起生活,詹姆斯·汉密尔顿是三一学院的毕业生,在梅斯郡的特里姆经营一所学校。

汉密尔顿从小就展现出天赋。叔叔观察到,汉密尔顿从小就表现出惊人的语言学习能力——这一说法曾被一些历史学家质疑,他们认为汉密尔顿对语言的理解仅仅是基础性的。在七岁时,他已开始学习希伯来语,13岁之前,在叔叔的教导下,他已掌握了12种语言:古典和现代欧洲语言、波斯语、阿拉伯语、印度斯坦语、梵语、马拉地语和马来语。汉密尔顿早期教育中对语言的重视,归因于他父亲希望他能为英国东印度公司工作。

作为一名熟练的心算高手,年轻的汉密尔顿能够将一些计算得出许多小数位数。1813年9月,美国心算神童泽拉·科尔本在都柏林展示。科尔本9岁,比汉密尔顿大一岁。两人参加了一场心算比赛,科尔本显然获胜。

面对失败,汉密尔顿减少了学习语言的时间,转而专注于数学。10岁时,他偶然发现了一本拉丁语版的《欧几里得几何原本》;12岁时,他学习了牛顿的《算术通论》。到16岁时,他已经读完了《自然哲学的数学原理》的大部分内容,并学习了一些关于解析几何和微积分的最新著作。
\subsection{学生时期}
1822年中期,哈密尔顿开始系统地学习拉普拉斯的《天体力学》。在此期间,他发现了《天体力学》中的一个逻辑错误,这一观察使哈密尔顿认识到约翰·布林克利(当时是爱尔兰皇家天文学家)的重要性。1822年11月和12月,他完成了自己的前三篇原创数学论文。在第一次访问邓辛克天文台时,他将其中两篇展示给了布林克利,布林克利要求他进一步发展这些论文。哈密尔顿遵从了要求,并于1823年初将修改后的版本提交,布林克利认可了这个修改版。

1823年7月,哈密尔顿通过考试进入都柏林三一学院,17岁时开始了他的大学生涯。他的导师是查尔斯·博伊顿,一位家族朋友,博伊顿向他推荐了巴黎高等师范学院数学小组的现代数学著作。约翰·布林克利评论这位早熟的哈密尔顿:“这个年轻人,我不是说将来会是,而是现在就是他这个时代的第一位数学家。”

学院为哈密尔顿颁发了希腊语和物理学两项顶级奖项(最高成绩)。在所有科目和考试中,他都是第一。他的目标是通过竞争考试赢得三一学院的奖学金,但未能如愿。1826年,布林克利被任命为克洛因的主教后,哈密尔顿在1827年被任命为布林克利离职后留下的两个空缺职位:安德鲁斯天文学教授和爱尔兰皇家天文学家。尽管大学生涯被这样缩短,但他分别获得了经典学科和数学学位(1827年获得学士学位,1837年获得硕士学位)。
\subsubsection{个人生活与诗歌} 
1824年,哈密尔顿在爱德沃斯镇通过理查德·巴特勒牧师(他叔叔詹姆斯·哈密尔顿的助理)认识了小说家玛丽亚·爱德沃斯。与此同时,他的叔叔还将他介绍给了位于梅斯郡的迪士尼家族。迪士尼家的儿子们就读于三一学院,哈密尔顿与他们成为了朋友。在夏丘,他遇到了迪士尼家的妹妹凯瑟琳·迪士尼。

哈密尔顿对凯瑟琳·迪士尼产生了感情,但她的家庭不赞成这段关系,凯瑟琳被迫嫁给了威廉·巴洛牧师(她姐姐丈夫的弟弟)。婚礼于1825年举行。哈密尔顿在1826年写了一首长诗《热情者》来表达他对她的感情。二十多年后,在1847年,他向约翰·赫歇尔透露,在这段时期他可能成为了一位诗人。

1825年,哈密尔顿遇到了阿拉贝拉·劳伦斯,莎拉·劳伦斯的妹妹。莎拉是他诗歌的主要通信对象和直言不讳的批评者。他通过玛丽亚·爱德沃斯的社交圈与阿拉贝拉认识。
\subsection{在邓辛克天文台}
哈密尔顿现在是爱尔兰皇家天文学家,他定居在邓辛克天文台,并在那里度过了余生。[8] 从1827年到1865年他一直住在那里。[18] 在邓辛克天文台的早期,哈密尔顿相当规律地观察天体;[19] 后来他将常规的观察工作交给了助手查尔斯·汤普森。[20][21] 哈密尔顿的姐妹们也支持天文台的工作。[3]

哈密尔顿的天文学入门讲座广受赞誉;除了学生,讲座还吸引了学者、诗人和女性。[22] 费利西亚·赫曼斯在听过他的讲座后写下了她的诗《孤独学生的祈祷》。[23]
\subsubsection{个人生活、旅行和诗歌访问 } 
亨密尔顿在1827年邀请四个姐妹来到天文台并与他一起生活,他们一直照料家庭,直到他在1833年结婚。四个姐妹包括伊丽莎·玛丽·亨密尔顿(1807–1851),她是位诗人。[3] 在1827年,亨密尔顿写信给他的妹妹格雷丝,提到“部分”劳伦斯姐妹在都柏林与他的妹妹伊丽莎见过面。[24][25]

亨密尔顿刚被任命为天文台的皇家天文学家,他便与亚历山大·尼莫(Alexander Nimmo)一起开始了在爱尔兰和英格兰的旅行,尼莫指导他学习经纬度。[26] 其中一次旅行是到位于利物浦附近Gateacre的莎拉·劳伦斯的学校,在那里,亨密尔顿有机会评估计算员诺克斯先生。[27] 他们在同年9月访问了威廉·华兹华斯位于瑞达尔山的住所,当时作家凯撒·奥特威也在场。[28][29]: 410  访问结束后,亨密尔顿向华兹华斯送去了许多诗歌,成为了他的“诗歌弟子”[30]
\begin{figure}[ht]
\centering
\includegraphics[width=6cm]{./figures/27024658bf75c9cc.png}
\caption{精神计算者诺克斯先生,1827年石版画} \label{fig_WLHMD_1}
\end{figure}
当华兹华斯在1829年夏天访问都柏林时,和约翰·马歇尔及其家人一起,他住在了亨密尔顿的天文台Dunsink。[29]: 411  在1831年与尼莫的第二次英格兰旅行中,亨密尔顿在伯明翰与尼莫分道扬镳,前往利物浦地区拜访母亲那边的劳伦斯姐妹及其家族。之后,他们在湖区重新会合,攀登了赫尔维林山并与华兹华斯共进茶。亨密尔顿随后通过爱丁堡和格拉斯哥返回都柏林。[15][31]

1832年,亨密尔顿访问了位于海格特的塞缪尔·泰勒·柯尔律治,在此期间,莎拉·劳伦斯给他的一封意外介绍信帮助他顺利拜访了柯尔律治。此外,他还与阿拉贝拉一起拜访了威廉·罗斯科的家族,罗斯科于1831年去世。[32][33]

亨密尔顿是位虔诚的基督徒,被描述为“圣经的爱好者,正统且忠实的国教成员”,并且有着“对启示宗教真理的深刻信仰”。[34][35][36]
\subsection{家庭}
在就读于三一学院期间,汉密尔顿向朋友的妹妹求婚,但她的拒绝使年轻的汉密尔顿陷入抑郁和病痛,甚至一度想自杀。[37] 1831年,他再次向奥布里·德·维尔的妹妹艾伦·德·维尔求婚,结果她也拒绝了。[37] 汉密尔顿最终于1833年与海伦·玛丽·贝利结婚,[37] 她是一个乡村牧师的女儿,两人育有三个孩子:威廉·埃德温·汉密尔顿(1834年出生),阿奇博尔德·亨利(1835年出生)和海伦·伊丽莎白(1840年出生)。[38] 汉密尔顿的婚姻生活证明是困难且不幸福的,因为贝利性格虔诚、害羞、胆怯,并且常年患病。[37]
\subsection{死亡}
汉密尔顿在去世前仍保持良好的精神状态,并继续完成他在过去六年中一直从事的《四元数要素》的编写工作。他于1865年9月2日因痛风发作而去世。[39] 他葬于都柏林的圣杰罗姆公墓。
\subsection{物理学}
汉密尔顿对经典力学和光学做出了杰出贡献。

他在1823年向约翰·布林克利(John Brinkley)提交的一篇早期论文中作出了他的第一次发现,布林克利在1824年将其以“焦线”这一标题呈交给了爱尔兰皇家学院。论文像往常一样被提交给了一个委员会,该委员会建议在出版前对其进行进一步的开发和简化。在1825至1828年间,这篇论文得到了扩展,并成为了一个更清晰的、关于一种新方法的阐述。[6] 在此期间,汉密尔顿开始欣赏光学的性质和重要性。[40]

1827年,汉密尔顿提出了一个单一函数的理论,这个函数如今被称为汉密尔顿主函数,它将力学和光学理论结合在一起。这个理论有助于建立数学物理中光波理论的基础。他在1832年首次预测其存在,并在他《光线系统的第三补编》中做了介绍。

爱尔兰皇家学院的论文最终被命名为《光线系统的理论》(1827年4月23日),第一部分于1828年在《爱尔兰皇家学院学报》上印刷。第二和第三部分的重要内容出现在三卷本补编(第一部分的补编)中,这些补编同样刊登在该学报上,此外,还有两篇论文《动力学中的一般方法》,它们分别于1834年和1835年在《哲学学报》上发表。在这些论文中,汉密尔顿发展了他的核心原理“变动作用”。

这项工作的一个结果是对透明双轴晶体(即单斜晶体、正交晶体或三斜晶体)的预测。[41] 一束光线以一定角度进入这种晶体时,将以一个空心光锥的形式射出。这个发现被称为圆锥折射。[6] 汉密尔顿从奥古斯丁-让·弗雷涅尔(Augustin-Jean Fresnel)引入的波面几何学中找到了这一现象的依据,波面有奇异点。[42] 对这一现象的基本数学解释是,波面不是凸体的边界。对这一现象的更深入理解直到20世纪中期的微局部分析才得以实现。[43]

从光学到动力学的转变,是通过“变动作用”方法的应用实现的,这一转变发生在1827年,并向皇家学会报告,关于该主题的两篇论文分别在1834年和1835年发表于《哲学学报》。
\subsubsection{工作背景与重要性}
哈密尔顿力学是处理运动方程的强大新技术。哈密尔顿的进展扩展了可以解决的机械问题的范畴。他的“变动作用量”原理基于变分法,属于最小作用量原理的广义问题类群,最早由皮埃尔·路易·莫泊图、欧拉、约瑟夫·路易·拉格朗日等人研究。哈密尔顿的分析揭示了比之前理解的更深的数学结构,尤其是动量与位置之间的对称性。发现现在称为拉格朗日方程和拉格朗日方程的功劳也应归功于哈密尔顿。

拉格朗日力学和哈密尔顿方法在物理学中对连续经典系统和量子力学系统的研究中都证明了其重要性:这些技术在电磁学、量子力学、相对论和量子场论中得到了应用。在《爱尔兰生物学词典》中,David Spearman 写道:

他为经典力学设计的公式同样适用于量子理论,并促进了量子理论的发展。哈密尔顿的形式主义并未过时;新思想仍然发现它是描述和发展物理学各个领域最自然的媒介,现在被普遍称为哈密尔顿量的函数,几乎是任何物理领域计算的起点。

包括刘维尔、雅可比、达尔布克、庞加莱、科尔莫哥洛夫、普里戈金和阿诺德在内的许多科学家在力学、微分方程和辛几何学等方面扩展了哈密尔顿的工作。
\subsection{数学} 
哈密尔顿的数学研究似乎是在没有合作的情况下进行并发展到完全成熟的,他的著作不属于任何特定的学派。大学当局在选举他担任天文学教授时,原本希望他能够自由地度过时间,尽其所能地推动科学发展,没有任何限制。
\subsubsection{四元数}
\begin{figure}[ht]
\centering
\includegraphics[width=6cm]{./figures/55a86bafd0624f8d.png}
\caption{都柏林布鲁姆桥上的四元数纪念碑} \label{fig_WLHMD_2}
\end{figure}
汉密尔顿在1843年发现了四元数的代数。[5]: 210 在许多相关的先前研究中,1840年本杰明·奥林德·罗德里格斯(Benjamin Olinde Rodrigues)达到了一个几乎等同于四元数发现的结果,只是名字不同。[47]

汉密尔顿在寻找将复数(可以看作是二维阿根图中的点)扩展到更高空间维度的方法。在研究四维空间时,他创造了四元数代数。根据汉密尔顿的说法,1843年10月16日,他和妻子一起沿都柏林的皇家运河散步时,突然想到了这个方程的解决方案:
\[i^2 = j^2 = k^2 = ijk = -1 ~\]
他随后用小刀将这个方程刻在了附近的布鲁姆桥(Broom Bridge,汉密尔顿称其为Brougham Bridge)的桥侧。[5]: 210 

四元数的引入意味着放弃了交换律,这对于当时来说是一个极为激进的步骤。在这种几何代数的原型中,汉密尔顿还引入了向量代数中的叉积和点积,四元数的乘积即为叉积减去点积的标量部分。汉密尔顿还将四元数描述为一个有序的四元素实数的倍数,并将第一个元素称为“标量”部分,剩余的三个元素称为“向量”部分。他创造了“张量”和“标量”这两个新词,并且是第一个在现代意义上使用“向量”一词的人。[48]
\subsubsection{其他数学工作} 
汉密尔顿还研究了方程理论中的五次方程解法,考察了尼尔斯·亨里克·阿贝尔(Niels Henrik Abel)、乔治·杰里德(George Jerrard)等人在各自研究中得出的结果。他有一篇关于傅里叶分析中波动函数的论文,并且发明了霍多图(hodograph)。他在某些物理重要的微分方程解法,尤其是通过数值近似的研究方面,也进行了一些调查,但只有部分成果在《哲学杂志》上不定期地发表。[6]

汉密尔顿还引入了“爱可赛游戏”(icosian game)或称汉密尔顿的谜题。它基于图论中的“哈密尔顿路径”概念。[3]
\subsection{出版物} 
\begin{itemize}
\item Hamilton, Sir W.R. (1853), 《四元数讲座》, 都柏林: Hodges 和 Smith  
\item Hamilton, Sir W.R., Hamilton, W.E.(编辑)(1866), 《四元数元素》, 伦敦: Longmans, Green, & Co.  
\item Hamilton, W.R. (1833), 《天文学入门讲座》, 《都柏林大学评论与季刊》第一卷, 都柏林三一学院  
\item 有关Hamilton的数学论文,见 David R. Wilkins, 《Sir William Rowan Hamilton (1805–1865): 数学论文》 
\end{itemize} 

Hamilton介绍了四元数和双四元数作为分析方法,后者通过引入复数系数扩展到八维。在1853年他整理的工作中,《四元数讲座》一书“形成了1848年及随后的讲座课程的内容,这些课程在都柏林三一学院的讲堂上讲授”。Hamilton自信地宣称,四元数将被发现作为研究工具具有强大的影响力。

当他去世时,Hamilton正在撰写四元数科学的最终陈述。他的儿子William Edwin Hamilton于1866年出版了《四元数元素》一书,这本厚重的762页的著作。由于副本不足,Charles Jasper Joly在第二版中进行了准备,并将此书分为两卷,第一卷于1899年出版,第二卷于1901年出版。第二版的主题索引和脚注改善了《元素》的可访问性。
\subsubsection{荣誉与奖项}  
Hamilton曾两次获得皇家爱尔兰学院的Cunningham奖章。[49]第一次获奖是在1834年,因其在圆锥折射方面的研究,次年他还因此获得了皇家学会的皇家奖章。[50]他在1848年再次获得此奖。

1835年,作为当年在都柏林召开的英国学会会议的秘书,Hamilton被总督封为爵士。随后的荣誉接踵而至,其中包括他在1837年当选为皇家爱尔兰学院的院长,以及被任命为圣彼得堡科学院的通讯院士这一罕见的荣誉。此后,1864年,刚成立的美国国家科学院选举了首批外籍院士,并决定将Hamilton的名字列在名单首位。[51]
\subsection{遗产}
\begin{figure}[ht]
\centering
\includegraphics[width=6cm]{./figures/92bb788c63168b9b.png}
\caption{庆祝哈密顿诞辰200周年的爱尔兰纪念币} \label{fig_WLHMD_3}
\end{figure}
与四元数发现相关的Broom桥下方的纪念牌匾,于1958年11月13日由埃蒙·德·瓦莱拉(Éamon de Valera)揭幕。[52][53]自1989年以来,爱尔兰国立大学梅努斯分校(Maynooth University)每年组织一次名为“哈密顿之行”(Hamilton Walk)的朝圣活动,数学家们从邓辛克天文台(Dunsink Observatory)步行到桥上,尽管桥上已无雕刻的痕迹,但一块石质纪念牌匾仍然纪念着这一发现。[54]

哈密顿研究所是梅努斯大学的应用数学研究所,皇家爱尔兰学院每年举办公开的哈密顿讲座,讲座曾邀请过穆雷·盖尔曼(Murray Gell-Mann)、弗兰克·威尔切克(Frank Wilczek)、安德鲁·怀尔斯(Andrew Wiles)和蒂莫西·高尔斯(Timothy Gowers)等人演讲。2005年是哈密顿诞辰200周年,爱尔兰政府将其定为“哈密顿年”,以庆祝爱尔兰科学。都柏林三一学院通过启动哈密顿数学研究所来标志这一年。[55]

1943年,爱尔兰发行了两枚纪念邮票,标志着四元数公布100周年。[56]2005年,爱尔兰中央银行发行了10欧元的纪念银质证明币,纪念哈密顿诞辰200周年。
\subsubsection{纪念活动}
\begin{itemize}
\item 哈密顿方程是经典力学的一种表述方式。  
\item 力学中许多其他的概念和对象,如哈密顿原理、哈密顿主函数、哈密顿–雅可比方程、凯莱–哈密顿定理,都是以哈密顿的名字命名的。  
\item 哈密顿量是物理学中既可以指代一个函数(经典力学中的哈密顿量),也可以指代一个算符(量子力学中的哈密顿算符),而在图论中,它有另一种含义。  
\item 四元数代数通常用 H 或黑板粗体 \(\mathbb{H}\) 来表示,以此向哈密顿致敬。  
\item 都柏林三一学院的哈密顿大楼也是以他的名字命名的。
\end{itemize}
\subsection{在文学中} 
一些现代数学家认为,哈密顿在四元数方面的工作曾被查尔斯·卢特维奇·道奇森在《爱丽丝梦游仙境》中进行讽刺。特别是疯狂帽子的茶会被认为是对四元数的愚蠢表现,并且暗示需要回归到欧几里得几何学。[58] 但是,2022年9月,提出了一些证据来反驳这一观点,表明这一说法似乎基于对四元数及其历史的误解。[59]
\subsection{家庭} 
哈密顿于1833年娶了海伦·贝利,海伦是尼讷镇(位于蒂珀雷里县)牧师亨利·贝利的女儿,同时也是天文台附近邻居的姐妹。[60][15]: 108  他们有三个孩子:记者威廉·埃德温·哈密顿(1834年出生)、阿奇博尔德·亨利(1835年出生)和海伦·伊丽莎·阿米莉亚(1840年出生)。[61] 海伦在其母亲去世前的1837年,曾长期与寡母一起住在贝利农场(尼讷)。此外,她也曾多次离开邓辛克,与姐妹们一起度过大部分时间,特别是在1840到1842年期间。[62] 据报道,哈密顿的婚姻生活相当困难。[5]: 209  在19世纪40年代初的困境时期,他的妹妹悉尼管理着他的家务;当海伦回来后,哈密顿在经历了一段抑郁期后感到更为愉快。[15]: 125, 126
\subsection{另见} 
\begin{itemize}
\item 天文学家列表  
\item 以威廉·罗恩·哈密顿命名的事物列表  
\item 理论物理学
\end{itemize}