% 使用数字货币钱包

\subsection{安全事项}
\begin{itemize}
\item 建议使用不易中病毒的操作系统生成钱包(如 iOS, Mac, Linux 等), 如果你使用 windows, 确保你的系统和软件都是正版的, 没有录屏软件, 并先进行杀毒.
\item 强烈建议使用系统自带的浏览器和输入法(在 iOS 上, 即使使用第三方输入法, 也不要打开全部权限)进行任何数字货币有关的操作.
\end{itemize}

\subsection{比特币钱包种类}
我们接下来主要以比特币为例讲解, 其他数字货币也大同小异. 比特币最原始的钱包形式就是一个地址和一个密钥. 一对真实有效的例子如
\begin{lstlisting}[language=bash]
地址: 15Vq6G2x7pZkGnLZBgqCCaruG5UZCtn8dr
密钥: Kzr3NAC9z5sE71MWrzdqoYz4wfmYhu6zainJHgQRBW6hcYZ5Xh25
\end{lstlisting}
一个用于生成地址和密钥的常用网站为 \href{https://www.bitaddress.org/}{bitaddress.org}. 需要注意的是地址和密钥是拥有比特币的唯一凭证, 任何持有两者的人就等同于持有该地址中的所有比特币. 比特币一旦被盗几乎不可能追回.

以下大致列出钱包的类型.
\begin{itemize}
\item 纸钱包: 顾名思义, 把地址和密钥写在纸上, 或者打印出来(通常包含二维码), 并保存到安全的地方(如保险柜)就叫纸钱包. 纸钱包的缺点是使用起来不太方便, 以及纸容易被烧毁, 泡烂, 褪色等.
\item 钢钱包: 为了克服纸钱包的脆弱, 市面上也可以买到钢钱包, 可以在上面以打点等方式记录信息. 钢钱包是保存比特币最安全稳定的方式.
\item 硬件钱包: 硬件钱包通常带有 USB 接口, 地址和密钥信息储存在钱包中, 需要转账时将其插入电脑即可.
\item 手机离线钱包: 把地址和密钥
\end{itemize}



https://www.bitaddress.org/


钱包密钥等敏感信息在生成以后, 第一时间将其转移到离线设备上(例如写在纸上或者存入硬件钱包).

纸钱包

