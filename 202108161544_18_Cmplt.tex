% 完备公理

\pentry{实数\upref{ReNum}}

从这里开始, 我们将不再区分有理数和由有理数确定的戴德金分割. 

通过戴德金分割, 我们得以填补有理数集中的"空隙", 从而得到了实数集. 纯粹从四则运算的角度, 还不能看出实数与有理数之间有何区别. 然而如果引入一些超出四则运算范围的操作, 即可看出实数与有理数的决定性区别: 实数集是\textbf{完备 (complete)} 的, 或者模糊地说, 实数集里"没有空隙".

\subsection{完备公理}

怎么刻画"不存在缝隙"这样的直观特性呢?

仍旧以数$\sqrt{2}$为例. 我们记得这个数的下类是$L=\{l\leq0\}\cup\{l>0:l^2<2\}$, 上类是$R=\{r>0:r^2>2\}$. $L$和$R$的并集就是有理数集$\mathbb{Q}$, 而由于 2 并非任何有理数的平方, 因此$L$和$R$之间"空无一物", 分割$L|R$并不是由一个实际存在的分点确定的. 但实数集却与此不同: 在区间$(-\infty,\sqrt{2})$和$(\sqrt{2},+\infty)$之间的确存在着一个分点$\sqrt{2}$. 

一般来说, 如果像定义戴德金分割那样对实数集$\mathbb{R}$进行操作, 那么得不到任何新的对象: \textbf{每一个这样的分割都必定是由一个分点确定的.} 实际上, 如果将实数集$\mathbb{R}$分成不相交的两部分$A\cup B$, 满足

\begin{enumerate}
\item 如果$a\in A$, 那么任何小于$a$的实数$a'$都属于$A$.
\item 如果$b\in B$, 那么任何大于$b$的实数$b'$都属于$B$.
\item 如果$a\in A$, $b\in B$, 那么必有$a\leq b$.
\item $A$不包含最大的元素;
\end{enumerate}

那么$L=A\cap\mathbb{Q}$满足戴德金分割下类的定义, 从而$R=B\cap\mathbb{Q}$自动满足分割上类的定义, 于是$L|R$自动成为一个戴德金分割, 它自然就确定了一个实数$x$. 这个实数$x$自然就是"分割"$A\cup B$的分点, 它满足如下的性质:

\textbf{对于$a\in A$, $b\in B$, 总有$a\leq x\leq B$. 而且, 实际上更有$A=(-\infty,x)$, $B=[x,+\infty)$.}

\begin{exercise}{}
证明这一点. 提示: 如果存在$a\in A$使得$a>x$, 那么开区间$(x,a)$中存在有理数$q$.
\end{exercise}

仿照这个思路, 我们便能证明如下的定理:

\begin{theorem}{实数的完备性}
设$A,B$是实数集的子集, 使得对于任意的$a\in A$, $b\in B$都有$a\leq b$. 那么存在实数$x$使得对于任意的$a\in A$, $b\in B$都有$a\leq x\leq b$.
\end{theorem}

上面的这个性质"两个集合中间一定有一个元素"被称作\textbf{"完备公理" (the axiom of completeness)}.

\begin{exercise}{}
证明实数的完备性, 即定理 1. 提示: 考虑集合$A$中所有元素的下类的并集, 它仍然还是一个下类.
\end{exercise}

\subsection{实数的公理刻画}

在得到了完备公理后, 我们便可以给出实数集合的公理刻画了. 

我们称一个带有二元运算$+$, $\cdot$和序关系$<$的集合$\mathfrak{R}$为一个实数模型, 如果上述运算和序关系满足下列一组公理:

\subsubsection{有序阿贝尔群公理}
\begin{itemize}
\item 
\end{itemize}

\begin{theorem}{实数的唯一实现}
任意两个满足实数公理的实数模型都是同构的.
\end{theorem}