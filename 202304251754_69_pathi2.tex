% 路径积分与关联函数(量子力学)
% 路径积分|关联函数|量子力学

\pentry{路径积分(量子力学)\upref{PIntQM}}
在量子力学或量子场论中,N 点关联函数 $\bra{\Omega}\hat x(t_1)\cdots \hat x(t_n)\ket{\Omega}$ (其中 $\ket{\Omega}$ 为理论的真空,即哈密顿量 $H$ 的基态。)给出了描述理论的一切信息。所以我们经常从关联函数出发来研究我们的理论(无论是量子力学还是场论)。比如我们可以通过计算关联函数来得到系统基态的能量、激发态的能谱、粒子散射的振幅等等。在这一词条中我们将展现如何利用路径积分公式来计算理论的关联函数。
\subsection{关联函数的路径积分公式}
\addTODO{从薛定谔绘景到海森堡绘景}

\begin{theorem}{}
$\bra{x_f,t_f} T[\hat x(t_1)\cdots \hat x(t_n)]\ket{x_i,t_i}$ 可以用路径积分公式表达为
\begin{equation}
\begin{aligned}
        \bra{x_f,t_f} T[\hat x(t_1)\cdots \hat x(t_n)]\ket{x_i,t_i}=
        \mathcal{N}\int \mathcal{D}[x] x(t_1)\cdots x(t_n) \exp(i S[x,\dot x])~.
\end{aligned}
\end{equation}
\end{theorem}
我们需要知道如何从一个给定的理论得到它的 N 点编时关联函数
 $\bra{x_f,t_f} T[\hat x(t_1)\cdots \hat x(t_n)]\ket{x_i,t_i}$ 。
这里初态和末态以及 $\hat x(t_1)\cdots \hat x(t_n)$ 中的时间指标代表它们是海森堡绘景下的态矢量和算符。因此作绘景变换后可以得到(不妨设 $t_1>t_2>\cdots t_n$)
\begin{align}
\bra{x_f} e^{-iH (t_f-t_1)} \hat x e^{-iH(t_1-t_2)}\cdots \hat x e^{-iH(t_N-t_i)}\ket{x_i}
\end{align}
\subsection{Gell-Mann-Low 定理}
\addTODO{证明}