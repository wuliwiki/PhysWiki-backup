% 亚伯拉罕·棣莫弗(综述)
% license CCBYSA3
% type Wiki

本文根据 CC-BY-SA 协议转载翻译自维基百科\href{https://en.wikipedia.org/wiki/Abraham_de_Moivre}{相关文章}。

\begin{figure}[ht]
\centering
\includegraphics[width=6cm]{./figures/a68a19e0be08cf7f.png}
\caption{约瑟夫·海默绘《德·莫弗像》,1736年} \label{fig_YBLH_1}
\end{figure}
亚伯拉罕·德·莫弗尔爵士(Abraham de Moivre FRS,法语发音:[abʁaam də mwavʁ],1667年5月26日-1754年11月27日)是一位法国数学家,以“德·莫弗公式”闻名——这是一个将复数与三角函数联系起来的公式。他还因在正态分布和概率论方面的工作而享有盛誉。

由于法国对胡格诺派的新教徒进行宗教迫害,特别是在1685年《枫丹白露敕令》颁布后达到高潮,他年少时迁居英格兰。他是艾萨克·牛顿、埃德蒙·哈雷和詹姆斯·斯特林的朋友。在英格兰的胡格诺派流亡者中,他也是编辑和翻译家皮埃尔·德·梅佐的同僚。

德·莫弗尔撰写了一本关于概率论的著作《机会论》,据说广受赌徒欢迎。他最早发现了比奈公式,这是一个将黄金分割数 φ 的 n 次幂与第 n 个斐波那契数联系起来的闭式表达式。他同样是最早提出中心极限定理的人之一,该定理是现代概率论的基石。
\subsection{生平}
\begin{figure}[ht]
\centering
\includegraphics[width=6cm]{./figures/832799eab7c34141.png}
\caption{《机遇论》,1756年} \label{fig_YBLH_2}
\end{figure}
\subsubsection{早年经历}
亚伯拉罕·德·莫弗于1667年5月26日出生在香槟地区的维特里勒弗朗索瓦。他的父亲丹尼尔·德·莫弗是一名外科医生,非常重视教育。尽管德·莫弗一家是新教徒,他最初却在当地的天主教基督兄弟会学校就读,这在当时宗教紧张的法国是相当少见的宽容表现。在他11岁时,父母将他送往色当新教学院就读。在那里他学习了四年希腊文,老师是雅克·迪朗代尔。这所新教学院建于1579年,由亨利-罗贝尔·德·拉·马克的遗孀弗朗索瓦丝·德·波旁倡议创办。

1682年,色当新教学院被当局关闭,德·莫弗转而前往索米尔学习逻辑学,为期两年。尽管数学并不在他的正式课程中,德·莫弗却自学了多部数学著作,包括法国奥拉托利会神父及数学家让·普雷斯特的《数学基础》,以及荷兰物理学家、数学家、天文学家和发明家克里斯蒂安·惠更撰写的赌博概率小册子《掷骰游戏中的推理》。1684年,德·莫弗搬到巴黎学习物理学,并首次接受系统的数学训练,跟随雅克·奥扎南进行私人学习。

1685年,法国国王路易十四颁布《枫丹白露敕令》,废除了给予法国新教徒广泛权利的《南特敕令》。这一法令严重压制新教信仰,禁止新教礼拜,强制要求所有儿童由天主教神父施洗。德·莫弗被送往圣马丹修道院附属学校,这是当局用来对新教儿童进行天主教灌输的场所。

目前尚不清楚德·莫弗何时离开圣马丹修道院并迁往英格兰。修道院的记录显示他在1688年离开,而德·莫弗和他的兄弟则于1687年8月28日以胡格诺派身份加入伦敦的萨伏伊教会。
\subsubsection{中年时期}
抵达伦敦时,德·莫弗已经是一位具有扎实数学功底的人,熟悉当时许多标准教材。\(^\text{[1]}\)为了谋生,他成为一名数学私人教师,在伦敦各地拜访学生授课,或在咖啡馆里讲授数学。他在一次拜访德文郡伯爵时看到牛顿的新书《自然哲学的数学原理》,便继续深入研究数学。翻阅该书后,他意识到这本书远比他以前读过的任何书都要深奥,于是下定决心要通读并理解它。然而,由于他每天要在伦敦各地奔波授课,没有太多时间静心学习,他便将书页撕下放入口袋,利用课间穿梭的空隙阅读。

据一个可能是传说的故事称,牛顿晚年时常把前来请教数学问题的人转交给德·莫弗,并说:“这些问题他比我更懂。”\(^\text{[2]}\)

到1692年,德·莫弗已与埃德蒙·哈雷成为朋友,不久后也与牛顿结识。1695年,哈雷将德·莫弗根据研究《原理》中“流数”而写的第一篇数学论文提交给皇家学会,并于同年发表在《哲学汇刊》上。不久之后,他又将牛顿著名的二项式定理推广为多项式定理。该方法于1697年被皇家学会知悉,并于当年11月30日将德·莫弗选为会员。

成为会员后,在哈雷的鼓励下,德·莫弗开始将注意力转向天文学。1705年,他凭直觉发现:“任一行星的向心力与其到中心的距离成正比,与回转体直径与垂直于切线的立方乘积成反比。”换言之,若某行星 M 绕焦点 F 作椭圆轨道运动,在某点 P,PM 是椭圆切线,且 FPM 为直角(即 FP 为切线的垂线),则 P 点处的向心力与 FM / (R·(FP)³) 成正比,其中 R 是 M 点处的曲率半径。瑞士数学家约翰·伯努利于1710年证明了这一公式。

尽管学术上取得诸多成果,德·莫弗始终未能获得大学数学教授职位,这使他不得不长期依赖费时的家教工作维生,而这对他造成的负担远大于同时代许多数学家。这种境况部分源于当时英国人对法国出身者的偏见。\(^\text{[3][4][5]}\)

1697年11月,他被选为皇家学会会员\(^\text{[1]}\),1712年被任命为一个由该学会组建的委员会成员,负责评议牛顿和莱布尼茨在微积分发明上的优先权争议。该委员会成员还包括阿布斯诺特、希尔、哈雷、琼斯、马金、伯内特、罗巴茨、博内、阿斯顿和泰勒等人。关于这场争议的详细内容,可参见“莱布尼茨与牛顿的微积分争议”词条。

德·莫弗终生贫困。据说他是圣马丁巷与克兰本街交汇处“老斯劳特咖啡馆”的常客,在那里靠下棋赚取微薄收入。
\subsubsection{晚年时期}
德·莫弗直到1754年去世前一直在研究概率与数学领域。去世后,还有几篇他的论文陆续发表。随着年龄增长,他变得愈发迟钝,睡眠时间也越来越长。人们普遍流传这样一个说法:德·莫弗注意到自己每晚比前一晚多睡15分钟,并据此精确计算出自己死亡的日期——当他的睡眠时间累计达到24小时的那天,即1754年11月27日。恰巧,他的确在那一天于伦敦去世,遗体最初安葬于圣马丁教堂,但后来被迁葬。

不过,这一“预言自己死亡日期”的说法并没有在当时留下任何文献记录,因此真实性受到质疑。\(^\text{[7]}\)
\subsection{概率论}
德·莫弗在解析几何和概率论的发展中起到了开创性作用,他在前人的基础上,尤其是克里斯蒂安·惠更斯和伯努利家族成员的工作之上,进一步扩展了这些理论。他撰写了概率论的第二本教科书《机遇论:一种计算游戏中事件概率的方法》。(关于博弈的第一本著作是吉罗拉莫·卡尔达诺在16世纪60年代撰写的《掷骰之书》,但直到1663年才出版。)

这本书共有四个版本:1711年拉丁文版,以及1718年、1738年和1756年的英文版。在后续版本中,德·莫弗加入了他1733年未发表的一项成果,即用我们今天称为正态分布或高斯函数的形式,对二项分布进行近似的首次陈述\(^\text{[8]}\)。这是第一次提出用分布的变异性(即标准差)为单位来计算某个误差大小的概率的方法,也是首次识别“可能误差”的计算。这些理论他还应用于赌博问题与年金计算表。

在概率论中,常见表达式之一是阶乘 $n!$,但在没有计算器的年代,计算较大 $n$ 的阶乘非常耗时。德·莫弗于1733年提出了一个估算阶乘的公式:$n! \approx cn^{(n + 1/2)} e^{-n}$他得出了常数 $c$ 的近似表达式,而后由詹姆斯·斯特林(James Stirling)指出该常数为 $\sqrt{2\pi}$\(^\text{[9]}\)。

德·莫弗还发表了题为《生命年金》的文章,首次揭示了随年龄变化的死亡率呈正态分布。基于此,他提出了一个简便公式,可用于估算基于个人年龄的年金支付收入。这类似于现代保险公司使用的公式类型。
\subsubsection{关于泊松分布的优先权问题}
一些关于泊松分布的结果最早由德·莫弗在《运气衡量,或关于由偶然事件决定的游戏中事件概率的研究》)中首次提出,并发表于《皇家学会哲学汇刊》第219页\(^\text{[10]}\)。因此,有些学者主张应将泊松分布命名为“德·莫弗分布”\(^\text{[11][12]}\)。
\subsection{德·莫弗公式}
1707 年,德·莫弗推导出一个方程式,从中可以得到如下公式:
$$
\cos x = \tfrac{1}{2}(\cos(nx) + i\sin(nx))^{1/n} + \tfrac{1}{2}(\cos(nx) - i\sin(nx))^{1/n}~
$$
他证明了该公式对所有正整数 $n$ 都成立\(^\text{[13][14]}\)。1722 年,他提出了另一组方程式,从中可以推导出更为人熟知的德·莫弗公式形式:
$$
(\cos x + i \sin x)^n = \cos(nx) + i \sin(nx)
^\text{[15][16]}~
$$
1749 年,欧拉使用他自己的欧拉公式对该公式进行了推广,证明了它对于任意实数 \( n \) 成立,这使得证明过程相当简洁\(^\text{[17]}\)。这条公式非常重要,因为它将复数与三角函数联系了起来。此外,该公式还可用于从 \(\cos x\) 和 \(\sin x\) 推导出 \(\cos(nx)\) 和 \(\sin(nx)\) 的有用表达式。

