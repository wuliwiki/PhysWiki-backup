% 不定积分
% 微积分|积分|不定积分

\pentry{基本初等函数的导数\upref{FunDer}}

一个实数函数 $f(x)$ 的\textbf{不定积分(indefinite integral)}是另一个函数 $F(x)$, 叫做 $f(x)$ 的\textbf{原函数(primitive function)}. 不定积分用符号表示为
\begin{equation}\label{Int_eq1}
F(x) = \int f(x) \dd{x} 
\end{equation}
不定积分被定义为\textbf{求导的逆运算}.即若能找到 $F(x)$ 使其导数为 $f(x)$, 那么 $F(x)$ 就是  $f(x)$ 的一个原函数.
\begin{equation}\label{Int_eq2}
F'(x) = f(x)
\end{equation}

给出一个 $f(x)$, 可以找到许多不同的原函数, 且这些原函数都只相差一个常数. 也就是说, 给 $f(x)$ 的任意一个原函数加上一个常数 $C$, 就可以得到 $f(x)$ 的另一个原函数. $C$ 叫做\textbf{积分常数}.

证明: 由于常数导数\upref{Der}%未完成:求导词条里面有讲吗?
为 $0$, 给原函数加上常数后\autoref{Int_eq2} 仍然成立
\begin{equation}\label{Int_eq3}
\dv{x} [F(x) + C] = f(x)
\end{equation}
我们可以从几何上来理解该式: 将函数曲线 $y = F(x)$ 整体在 $y$ 方向平移并不影响某个 $x$ 坐标处函数曲线的斜率.

\subsection{不定积分的基本性质}%未完成

由于求导是线性运算,% 求导词条里面有讲吗?
不定积分也是线性运算.即若干函数的线性组合的积分等于分别对这些函数积分再线性组合.令 $a_n$ 为常数,有
\begin{equation}\label{Int_eq4}
\int [a_1 f_1(x) + a_2 f_2(x)\dots] \dd{x}  = a_1 \int f_1(x) \dd{x} + a_2 \int f_2(x) \dd{x} \dots
\end{equation}

\subsection{不定积分计算方法}
与求导不同,计算不定积分没有特定的步骤,这里介绍几种方法
\begin{enumerate}
\item 最简单直接的方法是把已知的各种常见函数的导数写成积分的形式,例如已知 $\sin x$ 的导数是 $\cos x$, $\cos x$ 的积分就是 $\sin x$ 加任意常数.
\item 换元积分法\upref{IntCV}, 包括第一类换元法和第二类换元法.

\item 分部积分法\upref{IntBP}

\item 查表法.许多高等数学教材(包括本书)都会给出一个积分表.当然,在信息技术发达的今天这种方法几乎已经被计算软件和网站取代.

\item 计算软件和网站.常见的符号计算软件有 Mathematica %未完成:引用
,Maple 等,数学网站有 Wolfram Alpha %未完成:引用
等(建议先把积分技巧练熟再使用这些方法).其中 Wolfram Alpha 对许多积分还会给出详细的计算步骤.
 \end{enumerate}

对于一些常用积分,一般要求能熟记或快速推出.见积分表\upref{ITable} 中的\textbf{常用积分}部分.%未完成:真的有这个部分吗?



