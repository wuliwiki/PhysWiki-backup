% 原子结构和波粒二象性(高中)
% keys 必修三能量子|普朗克黑体辐射|光电效应|原子核式结构|玻尔模型|粒子波动性|量子力学初步
% license Xiao
% type Tutor
% maybe前半部分可以叫黑体辐射和光电效应

\begin{issues}
\issueTODO
\end{issues}

%\pentry{功和机械能\nref{nod_HSPM07},分子动力学\nref{nod_thermo} }{nod_5ffc}

\subsection{普朗克黑体辐射理论}
\subsubsection{黑体辐射}
在了解什么是黑体辐射之前,必须先明确黑体的定义。\textbf{黑体}指的是一种可以完全吸收入射的各种波长的电磁波而不将其反射的物体。我们常认为一个带有小孔的空腔可以作为一个黑体,因为如果有电磁波从小孔入射,在其中不管发生了多少次的反射和折射,都很难从空腔再度射出,因此满足对于黑体的定义。

一个常见的误区是,黑体虽然不能\textsl{反射},却仍然可以向外\textsl{辐射}电磁波,这是由于黑体在吸收能量的时候具有一定温度。这种辐射称之为\textbf{黑体辐射}。更进一步的,对于一般材料的热辐射,不仅仅和所选材料的温度有关,还和材料的种类以及表面状态相关,但是在黑体的情况下,黑体辐射的电磁波强度按照波长的分布仅仅和黑体的温度有关。

\subsubsection{黑体辐射的实验规律}
利用现代化设备,可以测出黑体辐射电磁波的强度按波长分布情况。实验表明,随着温度的升高,各个波长的辐射强度都有增加,这是符合我们直观认识的;与此同时,辐射强度的极大值会向着波长较短的方向进行移动。

这种现象该如何从微观上去进行解释呢?我们知道,物体存在着不停运动的带电微粒,每个带电微粒的振动都会产生变换的电磁场,这是物体电磁辐射的来源,而微粒的运动又和热密切相关,因此即可把温度和电磁辐射相互关联起来。德国物理学家维恩和英国物理学家瑞利分别提出了辐射强度按波长分布的理论公式。维恩公式在短波区和实验非常接近,但是在长波区域偏差较大;瑞利的结果在长波区和实验接近,但是在短波区会预测处无限大的辐射强度,称之为\textbf{紫外灾难}。显然和实验结果是不符合的。

那有没有换一种和全部实验结果相符合的理论公式呢?德国物理学家普朗克刚提出了普朗克公式解决了这一问题,普朗克公式可写成如下形式:
$$u(\lambda)=\dfrac{8\pi hc}{\lambda^5}\cdot\dfrac{1}{e^{hc/\lambda k_BT}-1}~.$$其中,$u(\lambda)$是能流密度,$h$是普朗克常量,$k_B$是玻尔兹曼常数。

\subsubsection{能量子}
从普朗克公式即可看出,普朗克引入了一个在经典物理中从未涉及过的常量$h$到对于黑体辐射的解释中,那么,对应于这一个新常量,是否有一些新的物理特性呢?答案是肯定的,如果想要得到这个公式,就需要先假定组成黑体的振动带电微粒所具有的能量并不是连续的,而是某一个最小能量单元$\epsilon_0$的整数倍。这个最小的能量单元称之为\textbf{能量子},表达式可以写作$$\epsilon_0=h\nu ~,$$其中$\nu$是带电微粒的振动频率,也即带电微粒吸收或者辐射电磁波的频率。另外,该公式也将能量子和我们新引入的常量也联系在了一起,$h=6.62607015\times 10^{-34}\Si{J\cdot s}$,是普朗克常量。

这个假设和我们的日常经验十分不同,在日常观察中,以动能为例,如果想让一个确定的物体具有更大的动能,则需要稍微增加一点它的速度,而速度是连续的,因此该物体所具有的动能也应该是连续的。但是这些认知在宏观世界才成立,等物理学深入到微观的量子世界,根据普朗克的假设,只有微观粒子的能量是分立的,也即所谓\textbf{量子化}的,才能解释微观世界的诸多现象。能量是否连续,也是微观和宏观世界物理规律最重要的区别之一。从观测的角度来说,在宏观世界中,即使每一个组成宏观物体的微观微粒能量都是分立的,但是由于数目极其巨大,每一个能量子的微粒比较于宏观的能量又极其微小,因此在观测中仍然会表现为连续的能量。

% 补充图片 黑体辐射的实验规律 P68
\subsection{光电效应}
\subsubsection{光电效应的实验规律}
%能量是否是分立的并非是宏观和量子之间的唯一区别。
\textbf{光电效应}指的是,光照射在金属的表面上使得金属表面的电子受激从金属上逸出的现象,这些被激发出来的电子称之为\textbf{光电子}。其中,如果入射光的频率太小,那么不会有任何的光电子产生,此时所对应的入射光频率是\textbf{截止频率},和金属本身有关;另外,如果入射光保持不变,改变发射极板和接收极板之间的电压,则会出现一个\textbf{饱和电流},即在电压加到一定值时,再次增加电压并不能增大电流,此时可以认为从发射极板中被激发出来的电子完全被接收极板所吸收,如果增加入射的光强,则饱和电流也会随之增大;第三,如果在发射极板和接收极板之间增加反向电压,即让激发出来具有一定初速度的电子在两极板之间减速,可以得到一个\textbf{截止电压},即恰好使得光电流减小到$0$时候的反向电压,同一种金属在受到同一个入射光频率的照射下,对应的截止电压是一致的,和入射光的强度无关;最后,光电效应是一个瞬时现象,只要入射光的频率高于截止频率,则照射在金属上时会立刻产生光电流,而没有一个时间积累的现象。

我们试图从经典电磁波的知识中来解释光电现象。想象一块碱金属,电子由于受到库伦作用力被束缚在原子核的周围,如果想要电子从金属中逸出,则需要外界作用在其上让电子具有某些能量,从而逃脱原子核的束缚。在光电实验中,电子可以吸收入射光的能量从而脱离金属。但我们进一步从光的电磁理论来分析这一实验现象,理论上越强的光和越长的积累时间对应更大的能量输入,理应对应更多的光电子、更高的截止电压;另外,截止频率和光电效应的瞬时性并非是必须的。这显然和我们观察到的实验现象有了巨大的冲突,那是否意味着我们之前做出的某些理论假设是有所不正确的呢?
\subsubsection{光电效应理论}
仿照之前普朗克的修正方式,如果我们不再认为电子吸收到光的能量来源于光电磁辐射的特性,也即并非存在一个能量积累,而是类似于小球之间的碰撞,由带能量的光子撞击碱金属中的电子,并传递自身能量,使得电子受激辐射。从这个观点来看,之前理论和实验不符的现象就可以得到解释。
\begin{exercise}{解释光电效应}
请试着采用普朗克能量子的观念来解释光电效应的实验现象。理解为什么实验中存在截止频率和瞬时性,以及为什么是入射光的频率而非强度决定更高的截止电压。
\end{exercise}
此时我们对光电效应的实验现象做出了定性的解释,定量的解释由爱因斯坦在普朗克量子假说的基础上进一步假定光子能量分立所得到。爱因斯坦认为,入射的光子能量也是分立的,其能量大小等于$h\nu$,其中$h$是普朗克常量,$\nu$为入射光的频率。这样,我们可以进行定量计算\footnote{之后的学习会提到利用此假设推导黑体辐射定律的方式。}。

当光子照射金属板时,光子的能量完全被某个电子吸收,在电子逃逸的情况下,光子的能量转化为电子的动能以及逃离金属所需要的能量。如果电子在金属的表面,此时脱离金属所需要做的功最小,叫做\textbf{逸出功}。不同金属的逸出功大小并不相同,这是显而易见的,因为不同金属原子核对于电子的束缚能力也是不一样的。

我们将上述物理过程写成等式的形式$$h\nu=E_k+W_0~,$$ $W_0$是逸出功,可以发现更高频率的光对应更大的初始动能,由于$E_k=\dfrac{1}{2}m_ev_C^2$,电子的最大初速度也会更大,因此会对应更高的截止频率。而如果光的频率小到让$h\nu<W_0$时,可以发现此时电子的初动能小于零,电子不会被激发,临界值$\nu_c=\dfrac{W_0}{h}$就是光电效应的截止频率。

\begin{example}{计算截止电压}
在光电效应的实验中,我们更容易测量到截止电压。请通过我们先前学到的知识推导截止电压$U_c$和入射光频率$\nu$以及逸出功$W_0$之间的关系。

解:我们刚才学到,$h\nu=E_k+W_0$,而截止电压为恰好没有光电流的电压,即满足$E_k=eU_c$,联立消去$E_k$,有
$$U_c=\dfrac{h}{c}\nu-\dfrac{W_0}{e}~.$$
可以发现截止电压和入射光频率呈线性关系。美国科学家密立根通过测量截止电压和入射光频率的实验,确定光电效应中的普朗克常量$h$和黑体辐射的普朗克常量相同,以此验证了爱因斯坦光电效应理论的正确性。
\end{example}
\subsubsection{光的波粒二象性}
我们之前对光的认知更偏向于认为光是一种电磁波,即光的波动说,但是从之前对光电效应的讨论中,我们可以发现,利用光的电磁理论并不能很好地解释实验现象,除非假设光是由一个个光子这些粒子所组成的。那么,光到底是一种波还是一种粒子呢?在物理学史上,光的波动学说和粒子学说开展了长期的争论,双方各有支持自己学说的实验证据。

举例说明,体现光的波动性的一个经典例子就是杨氏双缝干涉实验:
%% 放图啊
入射光穿过平面上的两条狭缝,在光屏上会形成干涉条纹,明纹为两波峰相遇、干涉相长,暗纹为波谷相遇,干涉相消。
而体现光粒子性的实验有我们刚刚提到的光电效应,以及以下介绍的康普顿效应。

康普顿在研究石墨对X射线的散射实验中发现,散射出来的X射线中包含波长大于入射波长的部分,这一现象被称之为\textbf{康普顿效应}。如果从经典的角度理解,散射波源于入射电磁波所引起的物质内部带电微粒的受迫振动,而这应该是一个固有频率,也即入射电磁波的频率。但是,如果假设入射的电磁波由包含动量和能量的光子组成,其动量为$$p=\dfrac{h}{\lambda}~,$$则光子可以传递一部分动量给电子,使得光的动量和能量减小,出射时波长更长。

最终,人们意识到,光\textbf{既}具有波动性,也\textbf{也}具有粒子性,这也称之为光的\textbf{波粒二象性}。对这些微粒奇特物理性质的认识促进科学家们建立了量子力学。


%\subsection{原子核式结构}
% 原子
% \subsection{波尔模型}
\subsection{粒子波动性}
\subsubsection{德布罗意波}
光具有波粒二象性,那是否还有其他的物质具有相同的性质呢?对于光的深入研究告诉我们,除了广为人知的波动性以外,对光粒子性的研究是被忽视了的。而我们熟知的实物粒子,如电子、质子、中子,常年以来都被科学家们关注其作为粒子的性质;日常接触的宏观物体,直观来看显然是表现为粒子性的。既然如此,这些实物粒子是否具有波动性成为了一个可供探索的科学问题。

结合先前对于能量子、光电效应物理表达式的学习,我们尝试将实物粒子的波动性写成类似的形式,即粒子的能量$\epsilon$和其对应波的频率$\nu$满足如下关系:
$$\nu=\dfrac{\epsilon}{h}~,$$这个式子等价于$$\lambda=\dfrac{h}{p}~,$$即反映粒子的波长$\lambda$和粒子动量$p$之间的关系。这一假设最先由法国物理学家德布罗意在二十世纪初提出,这种与实物粒子相联系的波被称之为德布罗意波,也叫物质波。

\subsubsection{德布罗意波的实验验证}
实物粒子也具有波动性这一假设需要得到实验上的验证,当我们考虑光的波动性的时候,经典的实验有光的干涉、衍射现象,同理,如果在对与实物粒子的实验中也能观察到干涉衍射现象,则可以证明实物粒子也具有波动性。

考虑公式$\lambda=\dfrac{h}{p}$,可知在速度一定时,质量越小的粒子会对应更大的波长,其波动性也就更为突出,因此,我们将目光放在质量较小的电子身上。1927年,物理学家戴维孙与汤姆孙进行了电子衍射实验,得到了类似光的衍射实验的衍射条纹,这证明了电子是具有波动性的,也为实物粒子的波动性提供了实验证据。更多的实物粒子波动性在后续的实验验证中被发现。

\begin{exercise}{估算德布罗意波长}
假设奔跑的运动员体重为$80\mathrm{kg}$,速度为$10\mathrm{m/s}$,请计算他此时对应的德布罗意波长是多少?并说明物质波在实验上较晚被认识到的原因。
\end{exercise}

物质波的实验证实告诉我们,实物粒子和光一样,也具有波粒二象性。
\subsection{量子力学初步}
建立 应用
从本章开始的黑体辐射,到由此引入的能量量子化,再到光的波粒二象性,以及最后发现德布罗意波,论证实物粒子也具有波粒二象性,乃至氢原子光谱热学初步(高中)\nref{nod_therHS}
% 改成atomHS 本文改成qutmHS
等。这一连串的物理进展,都和我们在经典物理中学习到的图像有所区别。
%% 画图时间
% 
%% 错别字纠正
% 
% 或者感觉应该把粒子波动性和量子力学初步放在这一章里,主要就是量子力学的内容
% 然后原子结构(核式结构和波尔模型)独立出来,先说整体的原子结构,再说后续的原子核放射性,和书本下一章做结合
% 开篇就是原子的结构 然后核式结构 然后电子的运动规律(光谱) 然后原子核的聚变裂变反应
