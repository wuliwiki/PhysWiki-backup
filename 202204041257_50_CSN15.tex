% 2015 年计算机学科专业基础综合全国联考卷
% 2015 计算机 考研 真题 全国卷

\subsection{一、单项选择题}
\textbf{1~40小题,每小题2分,共80分.下列每题给出的四个选项中.只有一个选项符合题目要求.}

1.已知程序如下: \\
\begin{lstlisting}[language=cpp]
int S(int n)
{
    return(n<=0)?0:s(n-1)+n;
}

void main()
{ 
    cout<<S(1);
}
\end{lstlisting}
程序运行时使用栈来保存调用过程的信息,自栈底到栈顶保存的信息依次对应的是 \\
A.main( )→S(1)→S(0) $\quad$ B.S(0)→S(1)→main( ) \\
C.main( )→S(0)→S(1) $\quad$ D.S(1)→S(0)→main( )

2.先序序列为a,b,c,d的不同二叉树的个数是 \\
A.13 $\quad$ B.14 $\quad$ C.15 $\quad$ D.16

3.下列选项给出的是从根分别到达两个叶结点路径上的权值序列,能属于同一棵哈夫曼树的是 \\
A.24,10,5和24,10,7 $\quad$ B.24,10,5和24,12,7 \\
C.24,10,10和24,14,11 $\quad$ D.24,10,5和24,14,6

4.现有一棵无重复关键字的平衡二叉树(AVL树),对其进行中序遍历可得到一个降序序列.下列关于该平衡二叉树的叙述中,正确的是 \\
A.根结点的度一定为2 $\quad$ B.树中最小元素一定是叶结点 \\
C.最后插入的元素一定是叶结点 $\quad$ D.树中最大元素一定无左子树

5.设有向图G=(V,E),顶点集V={v0,v1,v2,v3},边集E:{<v0,v1>,<v0,v2>,<v0,v3>,<v1,v3>}.若从顶点v0.开始对图进行深度优先遍历,则可能得到的不同遍历序列个数是 \\
A.2 $\quad$ B.3 $\quad$ C.4 $\quad$ D.5

6.求下面带权图的最小(代价)生成树时,可能是克鲁斯卡尔(Kruskal)算法第2次选中但不.是普里姆(Prim)算法(从v4开始)第2次选中的边是
\begin{figure}[ht]
\centering
\includegraphics[width=10cm]{./figures/CSN15_1.png}
\caption{第6题图} \label{CSN15_fig1}
\end{figure}
A.(v1,v3) $\quad$ B.(v1,v4) $\quad$ C.(v2,v3) $\quad$ D.(v3,v4)

7.下列选项中\textbf{不能}构成折半查找中关键字比较序列的是 \\
A.500,200,450,180 $\quad$ B.500,450,200,180 \\
C.180,500,200,450 $\quad$ D.180,200,500,450

8.已知字符串s为“abaabaabacacaabaabcc”,模式串t为“abaabc5’.采用KMP算法进行匹配,第一次出现“失配”(s[i]≠t[j])时,i=j=5,则下次开始匹配时,i和j的值分别是 \\
A.i=1,j=0 $\quad$ B.i=5,j=0 \\
C.i=5,j=2 $\quad$ D.i=6,j=2

9.下列排序算法中,元素的移动次数与关键字的初始排列次序无关的是 \\
A.直接插入排序 $\quad$ B.起泡排序 \\
C.基数排序 $\quad$ D.快速排序

10.已知小根堆为8,15,10,21,34,16,12,删除关键字8之后需重建堆,在此过程中,关键字之间的比较次数是 \\
A.1 $\quad$ B.2 $\quad$ C.3 $\quad$ D.4

11.希尔排序的组内排序采用的是 \\
A.直接插入排序 $\quad$ B.折半插入排序 \\
C.快速排序 $\quad$ D.归并排序

12.计算机硬件能够直接执行的是 \\
Ⅰ.机器语言程序 $\quad$ Ⅱ.汇编语言程序 $\quad$ Ⅲ.硬件描述语言程序 \\
A.仅Ⅰ $\quad$ B.仅Ⅰ、Ⅱ \\
C.仅Ⅰ、Ⅲ $\quad$ D.Ⅰ、Ⅱ、Ⅲ

13.由3个“1”和5个“0”组成的8位二进制补码,能表示的最小整数是 \\
A.-126 $\quad$ B.-125 \\
C.-32 $\quad$ D.-3

14.下列有关浮点数加减运算的叙述中,正确的是 \\
Ⅰ.对阶操作不会引起阶码上溢或下溢 \\
Ⅱ.右规和尾数舍入都可能引起阶码上溢 \\
Ⅲ.左规时可能引起阶码下溢 \\
Ⅳ.尾数溢出时结果不一定溢出 \\
A.仅Ⅱ、Ⅲ $\quad$ B.仅Ⅰ、Ⅱ、Ⅳ \\
C.仅Ⅰ、Ⅲ、Ⅳ $\quad$ D.Ⅰ、Ⅱ、Ⅲ、Ⅳ

15.假定主存地址为32位,按字节编址,主存和Cache之间采用直接映射方式,主存块大小为4个字,每字32位,采用回写(Write Back)方式,则能存放4K字数据的Cache的总容量的位数至少是
A.146K B.147K
C.148K D.158K
16.假定编译器将赋值语句“x=x+3;”转换为指令“add xaddr,3”,其中,xaddr是x对应的存储单元地址.若执行该指令的计算机采用页式虚拟存储管理方式,并配有相应的TLB,且Cache使用直写(Write Through)方式,则完成该指令功能需要访问主存的次数至少是
A.0 B.1
C.2 D.3

