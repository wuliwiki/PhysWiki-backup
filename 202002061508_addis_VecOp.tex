% 矢量算符

\pentry{矢量内积\upref{Dot}, 叉乘\upref{Cross}, 偏微分算符\upref{ParOp}}

我们先区分两种函数, 第一种是普通的多元函数 $f(x, y, z)$, 也叫\textbf{标量函数}, 即自变量 $x, y, z$ 是实数, 因变量也是实数\footnote{一些情况下也可以是复数}. 另一种是\textbf{矢量函数}, 一般用粗体加以区分(手写的时候在上方加箭头), 如 $\bvec f(x, y, z)$, 即因变量是一个 3 维列矢量\footnote{一些情况下也可以是 $N = 1, 2, \dots$ 维}

定义三维的\textbf{矢量算符}为 (也叫 nabla 或 del 算符)
\begin{equation}
\Nabla = \uvec x \pdv{x} + \uvec y \pdv{y} + \uvec z \pdv{z}
\end{equation}
$\Nabla$ 作用在标量函数(因变量是实数或复数) $f(x, y, z)$ 上的结果称为函数的\textbf{梯度}, 是一个矢量函数
\begin{equation}
\Nabla f(x, y, z) = \qty(\uvec x \pdv{x} + \uvec y \pdv{y} + \uvec z \pdv{z}) f = \uvec x \pdv{f}{x} + \uvec y \pdv{f}{y} + \uvec z \pdv{f}{z}
\end{equation}
这可以类比矢量与标量的乘法.


$\Nabla$ 与矢量函数(因变量是矢量) $\bvec f(x, y, z)$ 的作用有两种, 第一是 “点乘”, 结果称为函数的\textbf{散度}, 是一个标量函数
\begin{equation}
\begin{aligned}
\Nabla \vdot \bvec f(x, y, z) &= \qty(\uvec x \pdv{x} + \uvec y \pdv{y} + \uvec z \pdv{z}) \vdot \qty(\bvec x f_x + \bvec y f_y + \bvec z f_z)\\
&= \pdv{f}{x} + \pdv{f}{y} + \pdv{f}{z}
\end{aligned}
\end{equation}
另一种是 “叉乘”, 得到函数的


(未完成: 例如梯度散度旋度算符, 例如量子力学角动量算符, 要区分对矢量函数作用还是对标量函数作用)
