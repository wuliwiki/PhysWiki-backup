% 马尔可夫链蒙特卡洛
% keys 马尔可夫链|蒙特卡洛方法|计算统计学
% license Usr
% type Wiki

\textbf{马尔可夫链蒙特卡洛}(Markov Chain Monte Carlo,简称MCMC)是一种用于从概率分布中抽样的统计方法。这种方法结合了马尔可夫链和蒙特卡洛模拟的思想,被广泛应用于贝叶斯统计、统计物理学、机器学习等领域。

\subsection{动机}
\subsubsection{贝叶斯推断 Bayesian Inference}

贝叶斯推断的主要目标是通过结合先验分布和观测数据,利用贝叶斯定理推导参数的后验分布。在观测到数据\(D\)后,更新我们对参数\(\theta\)的信念。这一过程通过贝叶斯定理实现,其中先验概率分布 \(P(\theta)\) 表达了我们在看到数据之前对参数的信念,似然函数 \(P(D|\theta)\) 表达了在给定参数情况下观测到实际数据的可能性,而后验概率分布 \(P(\theta|D)\) 则是我们在看到数据后对参数的更新信念。该定理建立了先验概率 \(P(\theta)\)、似然函数 \(P(D|\theta)\) 和得到的后验概率 \(P(\theta|D)\) 之间的关系。

\begin{equation}
P(\theta|D) = \frac{P(D|\theta) \times P(\theta)}{P(D)} = \frac{P(D|\theta) \times P(\theta)}{\int P(D|\hat{\theta}) \times P(\hat{\theta}) \, d\hat{\theta}}~.
\end{equation}

然而,在计算分母中的证据 \(P(D)\) 时可能具有挑战性,特别是在涉及高维空间或复杂的多峰模型的情况下。这种计算需要在整个参数空间上进行积分,这可能非常耗费资源。因此,当面临没有解析解或共轭先验不可用的情况时,使用蒙特卡洛积分等数值积分方法变得至关重要。

\subsubsection{蒙特卡罗积分 Monte Carlo Integration}

蒙特卡罗积分是一种通过随机抽样来估计复杂积分的数值近似方法。均值估计法包括通过随机抽样在定义域内生成大量均匀分布的随机点。随后,它计算这些随机点对应的函数值的平均值,并乘以定义域的总体积,以得到函数积分的估计值。对于利用均值估计方法计算 \(I = \int_a^b h(x) \,dx\),其中 \(X \sim \text{D}(a,b)\),
\begin{align} 
E [h(X)] =& \  \frac{1}{b-a} \int_a^b h(x) \ dx = \frac{I}{b-a}~.
\end{align}

考虑到 \(\{ X_i\} \ \text{i.i.d.}  \sim \text{D}(a,b), \ i=1,\ldots,N \),使得 \(Y_i = h(X_i)\) 表示一组独立同分布的随机变量,大数定律使我们能够推断:
\begin{align} 
\frac{1}{N} \sum_{i=1}^N h(X_i) & \to E [h(X)] = \frac{I}{b-a}, \ (N \to \infty)\\
\tilde I &= \frac{b-a}{N} \sum_{i=1}^N h(X_i)~.
\end{align}