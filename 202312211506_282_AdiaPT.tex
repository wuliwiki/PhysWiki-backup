% 含时微扰理论(绝热)
% license Usr
% type Tutor

\pentry{一阶不含时微扰理论(量子力学)\upref{TIPT}, 绝热近似(量子力学)\upref{AdiaQM}}

许多时候我们会把不含时微扰理论和绝热近似理论和一起使用, 拓展出yi'ge。

接下来考虑一个应用问题: 含时薛定谔方程中如果初始波函数处于任意一个态 $\Psi(0)$, 然后非常缓慢地随时间增加含时微扰 $\lambda(t) H'$ 的强度直到 $\lambda(t)=1$, 这个过程中 $\Psi(t)$ 会如何随时间变化?列出含时薛定谔方程就是(令 $\lambda(0)=0$)
\begin{equation}
[H + \lambda(t)H']\Psi(t) = \I\hbar\dot \Psi(t)~.
\end{equation}


\textbf{绝热近似(adiabatic approximation)}\upref{AdiaQM}告诉我们, 如果 $\Psi(0)$ 是一个非简并本征态(或好量子态) $\psi_n$, 那么该过程中 $\Psi(t)$ 将始终保持 $H+\lambda(t) H'$ 的本征态(好量子态) $\psi_n(\lambda)$,但需要乘以一个额外的相位因子 $\E^{\I\alpha_n(t)}$。 而 $\psi_n(\lambda)$ 又可以进一步通过不含时微扰进行计算。

由于含时薛定谔方程是线性的,如果 $\Psi(0)$ 是任意波函数,那么只需把它表示为 $\psi_n(0)$ 的线性组合,那么就有
\begin{equation}
\Psi(t) = \sum_n c_n \psi_n(t) \E^{\I\alpha_n(t)}~.
\end{equation}


这是可以的(参考\cite{GriffQ})。 那么既然好量子态是完备的, 含时薛定谔方程又是线性的, 上面的猜测的确成立。

是否先分解为好量子态的线性组合, 然后再对它们分别用不含时微扰理论修正?

也就是说,如果初始时波函数处于某个简并空间的非好本征态, 那么在含时薛定谔方程中给 $H$ 慢慢加上 $H'$ 后,波函数将被分割到若干个不同能级的简并空间中,测量其能量将可能随机得到对应的多个本征值中之一。 但在这个过程中每个好本征态会累积额外的相位。
