% 沃尔夫冈·泡利(综述)
% license CCBYSA3
% type Wiki

本文根据 CC-BY-SA 协议转载翻译自维基百科\href{https://en.wikipedia.org/wiki/Wolfgang_Pauli}{相关文章}。

\begin{figure}[ht]
\centering
\includegraphics[width=6cm]{./figures/7d0315d83d1af784.png}
\caption{泡利于1945年} \label{fig_Pauli2_1}
\end{figure}
沃尔夫冈·恩斯特·泡利(Wolfgang Ernst Pauli,/ˈpɔːli/,[5] 德语:[ˈvɔlfɡaŋ ˈpaʊli];1900年4月25日—1958年12月15日)是一位奥地利物理学家,量子力学的先驱。1945年,在阿尔伯特·爱因斯坦的提名下,[6] 泡利因其“通过发现一条新的自然法则——泡利不相容原理(Pauli Exclusion Principle)所做出的决定性贡献”而获得诺贝尔物理学奖。这一发现涉及**自旋理论**,该理论是物质结构理论的基础。为了保持**β衰变**中的**能量守恒**,他提出了一种质量极小、电中性的粒子,其后被**恩里科·费米(Enrico Fermi)**命名为**中微子(neutrino)**。该粒子最终于1956年被探测到。