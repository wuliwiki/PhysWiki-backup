% 埃尔文·薛定谔
% license CCBYSA3
% type Wiki

(本文根据 CC-BY-SA 协议转载自原搜狗科学百科对英文维基百科的翻译)


欧文·鲁道夫·约瑟夫·亚历山大·薛定谔(UK: /ˈʃrɜːdɪŋər/, US: /ˈʃroʊ-, ˈʃreɪ-/;[1] 德语:[ˈɛɐ̯viːn ˈʃʁøːdɪŋɐ];1887年8月12日至1961年1月4日),有时写作 Erwin Schrodinger 或者 Erwin Schroedinger,是一位获得诺贝尔奖的奥地利物理学家,他在量子理论领域取得了许多基础性的研究成果:薛定谔方程提供了一种计算系统波函数及其随时间动态变化的方法。

此外,他还是物理领域的很多著作的作者:统计力学和热力学、电介质物理学、颜色理论、电动力学、广义相对论和宇宙学,他多次尝试构建统一场论。在他的书中生命是什么 薛定谔从物理学的角度看待生命现象,解决了遗传学的问题。他非常重视科学的哲学方面、古代和东方的哲学概念、伦理和宗教。 他还写了哲学和理论生物学的著作。他也以他的“薛定谔猫”思想的实验而闻名。[2]

\subsection{传记}
\subsubsection{1.1 早年}
1887年8月12日,薛定谔出生于 Erdberg [de]奥地利维也纳,父亲鲁道夫·薛定谔 Rudolf Schrödinger (德语)[3] (cerecloth 生产者、植物学家)[4]母亲乔治·埃米莉亚·布伦达·薛定谔(内·鲍尔)(父亲是亚历山大· 保尔Alexander Bauer (德语),[5][6][7] 化学教授,维也纳科技大学)。他是他们唯一的孩子。

他的母亲有一半奥地利血统,一半英国血统;他的父亲是天主教徒,母亲是路德教徒。尽管他是在一个宗教家庭中长大的路德教徒,但他自称是无神论者。[8] 然而,他对东方宗教和泛神论有浓厚的兴趣,他在著作中使用了宗教象征主义。[9] 他也相信他的科学工作是一种接近神性的方法,尽管是在隐喻的意义上。[10]

他还能在校外学习英语,因为他的外祖母是英国人。[11] 1906年至1910年间,在弗朗茨·艾克纳(1849-1926)和弗里德里希·哈泽内尔(1874-1915)的指导下,薛定谔在维也纳学习。他还与卡尔·威廉·弗里德里希·弗里茨·科尔劳施进行了实验性工作。

1911年,薛定谔成为艾克纳的助手。早年,薛定谔深受亚瑟·叔本华的影响。由于他对叔本华作品的广泛阅读,他一生都对色彩理论和哲学有浓厚的兴趣。在他的演讲“思想和物质”中,他说“在空间和时间上延伸的世界只是我们的代表。”这重复的是叔本华主要著作中的第一句话。

\subsubsection{1.2 中年}
\begin{figure}[ht]
\centering
\includegraphics[width=6cm]{./figures/3420d552020e95e7.png}
\caption{年轻时的科学家欧文·薛定谔} \label{fig_Erwin_1}
\end{figure}
1914年,埃尔温·薛定谔实现了适应训练(venia legendi)中。1914年至1918年间,他作为奥地利要塞炮兵(戈里齐亚、杜伊诺、西斯蒂安娜、普罗赛柯、维也纳)的委任军官参加了战争工作。1920年,在耶拿,他成为马克斯·维恩的助理,1920年9月,在斯图加特他获得了ao教授(ausserordentlicher教授)的职位,大致相当于Reader(英国)或副教授(美国)。1921年,在布雷斯劳(现在的波兰弗罗茨瓦夫),他成为了o教授(奥德利彻教授,即正教授)。

1921年,他搬到了苏黎世大学。1927年,他在柏林弗里德里希·威廉大学接替马克斯·普朗克的职位。1934年,薛定谔决定离开德国;他不喜欢纳粹的反犹太主义。他成为牛津大学玛格达林学院的研究员。他到达后不久,就和保罗·狄拉克一起获得了诺贝尔奖。他在牛津的职位晋升进展不顺利;他和两个女人住在一起这样非常规的家庭安排,[12] 不能被接受。1934年,薛定谔在普林斯顿大学讲课;他没有接受那里提供的永久的职位。同样,他想和妻子和情妇建房子的愿望也落空了。[13] 他本来有希望在爱丁堡大学找到一份工作,但签证延迟了,最后他于1936年在奥地利的卡尔·弗朗岑斯格拉茨大学找到了一份工作。他还接受了印度阿拉哈巴德大学物理系的主席职位。[14]

1935年,在这些任期问题发生的过程中,在与阿尔伯特·爱因斯坦广泛联系后,他提出了现在所谓的薛定谔猫思想的实验。

\subsubsection{1.3 晚年}
\begin{figure}[ht]
\centering
\includegraphics[width=10cm]{./figures/dba9f32935bff71d.png}
\caption{1942年都柏林高等研究院的薛定谔(右前排第2排)和瓦莱拉(左起第4排)} \label{fig_Erwin_2}
\end{figure}
1938年,继德奥合并之后,由于1933年逃离德国和他众所周知的对纳粹主义的反对,薛定谔遇到了问题。[15] 他发表声明放弃了对纳粹主义的反对意见(后来他后悔这么做,并向爱因斯坦解释了原因)。[16] 然而,这并没有完全使新的政权满意,并且由于政治上的不可靠性,卡尔·弗朗岑斯格拉茨大学解雇了他。他受到骚扰,并收到不离开德国的警告,但他和妻子逃到了意大利。从那里,他接受了牛津和根特大学的客座教授职位。[16][15]

同年,他收到了爱尔兰数学家埃蒙·德·瓦莱拉的个人邀请,邀请他居住在爱尔兰,并同意帮助他在都柏林建立一个高级研究所。[17] 他搬到都柏林的克隆塔尔夫,1940年成为理论物理学院的院长,并在那里呆了17年。他于1948年成为归化的爱尔兰公民,但保留了奥地利公民身份。他还写了大约50本关于各种主题的出版物,包括他对统一场论的探索。

1944年,他写道 《生命是什么?》其中讨论了负熵和具有生物体遗传密码的复杂分子的概念。根据詹姆斯·沃森的回忆录, 《DNA,生命的秘密》,薛定谔的书给了沃森研究基因的灵感,这使得1953年DNA双螺旋结构的成功发现。同样,在弗朗西斯·克里克的自传书《 多么疯狂的追求》中,描述了他如何受到薛定谔关于遗传信息如何储存在分子中的推测的影响。

直到1955年退休前,薛定谔一直呆在都柏林。他一生都对印度教的吠檀多哲学感兴趣,这影响了他在《 生命是什么?》结尾的推测——关于个体意识只是遍及宇宙的单一意识的一种表现的可能性。[18] 手稿“伽利略未发表的对话片段”是为The King's Hospital第155版Blue Coat所写的[19] ,从那时起不久,都柏林的The King's Hospital寄宿学校重新出现了[20] 这是为了庆祝他离开都柏林,就任维也纳大学物理系主任。

1956年,他回到维也纳。在世界能源大会期间的一次重要演讲中,他拒绝谈论核能,因为他对此持怀疑态度,而是做了一次哲学演讲。在此期间,薛定谔偏离了主流量子力学对波粒二象性的定义,独自推动了波的概念,引起了很多争议。

\subsubsection{1.4 个人生活}
\begin{figure}[ht]
\centering
\includegraphics[width=6cm]{./figures/26528637892c1afe.png}
\caption{Annemarie和Erwin Schrödinger的墓地; 在铭牌上方刻有Schrödinger的量子力学波动方程:} \label{fig_Erwin_3}
\end{figure}
1920年4月6日,薛定谔与安娜玛丽(Anny)贝尔泰尔结婚。[21] 薛定谔患有肺结核,在20世纪20年代曾几次住在阿罗萨的疗养院。正是在那里,他建立了波动方程。[22] 如上所述,薛定谔的个人生活非常不传统。当他于1938年移民爱尔兰时,他为自己、妻子以及另一位女性希尔德·马奇夫人获得了签证。马奇是薛定谔一位奥地利同事的妻子,1934年,他和她生了一个女儿。[22] 薛定谔亲自给埃蒙·德·瓦莱拉的爱尔兰总理写信,为马奇夫人申请签证。1939年10月 ménage à trois 正式在都柏林定居。[22] 在居住在爱尔兰期间,薛定谔又和两个不同的女人生了两个女儿。[22]

1961年1月4日,薛定谔在维也纳死于肺结核,享年73岁。[23] 他离开后Anny成了寡妇,薛定谔被安葬在奥地利的阿尔卑巴赫一个天主教公墓。虽然他不是天主教徒,但负责墓地的牧师在得知薛定谔是宗座科学院的一员后,允许安葬。[24] 他的妻子Anny(生于1896年12月3日)于1965年10月3日去世。薛定谔的孙子特里·鲁道夫教授继承了量子物理学家的衣钵,在伦敦帝国学院任教。[25][26]

\subsection{科学活动}
\subsubsection{2.1 早期职业生涯}
早年,薛定谔在电气工程、空气电力学和空气放射学领域进行了实验,但他通常和他从前的老师弗朗兹·艾克纳一起工作。他还研究振动理论、布朗运动理论和数理统计。1912年,应 《电与磁手册》编辑的邀请,薛定谔写了一篇题为 《Dieelectrism》的文章。同年,薛定谔对放射性物质的可能传播高度分布作出了理论估计,这是解释观测到的大气放射性所必需的。1913年8月,他在塞哈姆进行了几次实验,证实了他和维克托·弗朗兹·赫斯的理论估计。由于这项工作,薛定谔获得了1920年的奥地利科学院 Haitinger 奖。[27] 这位年轻的研究人员在1914年进行的其他实验研究是用来验证气泡中毛细压力的公式,以及研究金属表面伽马射线下落时出现的软β射线的性质。最后一部著作是他和朋友弗里茨·科尔劳施一起完成的。1919年,薛定谔完成了他最后一次相干光物理实验,随后专注于理论研究。

\subsubsection{2.2 量子力学}
\textbf{新量子理论}

在职业生涯的最初几年,薛定谔开始熟悉量子理论的思想,这些思想是在马克斯·普朗克、阿尔伯特·爱因斯坦、尼尔斯·玻尔、阿诺·索末菲和其他人的著作中发展起来的。这些知识帮助他解决了理论物理中的一些问题,但是当时的奥地利科学家还没有准备好放弃传统的经典物理方法。

直到20世纪20年代初,在他与索末菲和沃尔夫冈·泡利相识并移居德国之后,薛定谔关于原子理论和光谱理论的第一批出版物才开始出现。1921年1月,薛定谔完成了他关于这个主题的第一篇文章,关于玻尔-索末菲效应的框架——电子相互作用对碱金属光谱的某些特征。他特别感兴趣的是量子理论中相对论性考虑的引入。1922年秋,他使用数学家赫尔曼·外尔(1885-1955)发展的方法,从几何角度分析了原子中的电子轨道。这项工作表明量子轨道与某些几何性质相关联,是预测波动力学某些特征的重要步骤。同年早些时候,他基于光量子的假设以及对能量和动量的考虑,创建了谱线的相对论性多普勒效应的薛定谔方程。他喜欢他的老师艾克纳关于守恒定律的统计性质的想法,所以他热情地接受了玻尔、克拉默和斯莱特的文章,这些文章提出了在单个原子过程中(例如,在辐射发射过程中)违反这些定律的可能性。尽管汉斯·盖革和瓦尔特·博特的实验很快对此产生了怀疑,但把能源作为统计概念的想法对薛定谔的一生来说具有极大的吸引力,他在一些报告和出版物中对此进行了讨论。[28]

\textbf{波动力学的创造}

1926年1月,薛定谔在物理学年鉴发表文章《Quantisierung als Eigenwertproblem》(量化作为特征值问题)[29] ,这篇文章主要讲的是波动力学,并展示了现在所谓的薛定谔方程。在这篇文章中,他给出了与时间无关系统的波动方程的“推导”,并证明了它给出了正确的类氢原子能量特征值。这篇论文被普遍认为是二十世纪最重要的成就之一,并在量子力学的大部分领域,甚至在所有物理和化学领域引发了一场革命。第二篇论文仅在四周后提交,解决了量子谐振子、刚性转子和双原子分子问题,并给出了新的薛定谔方程推导。五月份发表的第三篇论文展示了他的方法与海森堡的方法的等价性,并给出了斯塔克效应的处理方法。本系列的第四篇论文展示了如何处理系统随时间变化的问题,如散射问题。在这篇文章中,他介绍了波动方程的一个复数解,以防止出现四阶和六阶微分方程。(这可以说是量子力学从实数转换到复数的时刻。)当他为了降低微分方程的阶而引入复数时,神奇的事情发生了,整个波动力学都在他的脚下。(他最终把阶数减少到一。)[30] 这些论文是他的主要成就,同时被当时的物理界马上承认其重大意义。

薛定谔对量子理论的含义并不完全满意。他写了关于量子力学的概率解释,说:“我不喜欢它,很抱歉我曾经和它有任何关系。”(只是为了嘲笑量子力学的哥本哈根诠释,他设计了一个著名的思想实验,叫做“薛定谔猫悖论”。)[31]

\textbf{在统一场论上的工作}

继他在量子力学方面的工作之后,薛定谔投入了大量的精力在统一场论上——在广义相对论的基本框架内将引力、电磁力和核力结合起来,并与阿尔伯特·爱因斯坦在工作进行了广泛的交流。[32] 1947年,在Royal Irish Academy上的谈话中,他宣布了一个结果——“仿射场论”[33] 但这一声明被爱因斯坦批评为“初步的”,并未能得到理想的统一理论。[32] 在统一场论上的尝试失败后,薛定谔放弃了统一场论的工作,转向其他研究。

\subsubsection{2.3 色彩学}
薛定谔对心理学特别感兴趣,尤其是颜色感知和色度学(德语: Farbenmetrik)中。他花了几年时间研究这些问题,并在这个领域发表了一系列论文:
\begin{itemize}
\item “颜料理论”, 物理学年鉴,(4),62,(1920),603–22(最高亮度颜料理论)
\item “在塔格森的法本米特理论中, 物理学年鉴,(4),63,(1920),397-456;481–520(日光视觉颜色测量理论概述)
\item “法本米特克”, Zeitschrift für Physik,1,(1920),459–66(颜色测量)。
\item “理论上讲, 数学-自然科学, 134,471,(论四色理论与三色理论的关系)。
\item “能源的莱雷·冯·德·斯特兰登”, 物理和气象学家第2卷,第1部分(1926)(色差阈值)。
\end{itemize}

他在颜色感知心理学方面的工作跟随着牛顿、麦克斯韦和冯·亥姆霍兹的步伐。其中一些论文已被翻译成英文,可在以下网址找到: 色彩科学的来源艾德。大卫·碎石,麻省理工学院出版社(1970)。

\subsection{宝贵财富}
今天,由薛定谔猫提出的哲学问题仍在争论,并且仍然是他在大众科学领域最持久的遗产,而在更高的技术层面上,薛定谔方程是他最持久的遗产。迄今为止,薛定谔被称为“量子力学之父”。月球背面的大陨石坑“薛定谔”就是以他的名字命名的。1993年埃尔温·薛定谔国际数学物理研究所在维也纳成立。

1983-1997年奥地利1000先令钞票(第二高面额)设计的主要特征是薛定谔的肖像。

在爱尔兰利默里克的利默里克大学,一座建筑以他的名字命名,[34] 柏林阿德勒肖夫的“埃尔温·薛定谔Zentrum”也是如此。[35]

2013年,Google用涂鸦庆祝了薛定谔126岁生日。[36][37]