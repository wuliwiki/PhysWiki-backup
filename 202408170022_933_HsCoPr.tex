% 复合命题(高中)
% keys 命题|高中|且|或|非|条件|量词|逆否|复合
% license Usr
% type Tutor

\begin{issues}
\issueDraft
\end{issues}

\pentry{命题与推理\nref{nod_HsLogi},集合的基本运算\nref{nod_HsSeOp}}{nod_acc5}

在高中数学中,逻辑是不可忽视的基础工具。尽管教材中逻辑内容的比重有所减少,但掌握逻辑对理解和应用数学知识仍然至关重要。逻辑不仅帮助我们严谨地推导数学定理,还在日常生活中增强了我们分析和解决问题的能力。

当你已经掌握了简单命题的基础后,学习复合命题就成为了你进一步深入逻辑世界的关键一步。顾名思义,\textbf{复合命题}(compound proposition)就是由原子命题通过各种方式(逻辑连接词组合、量词限定变量和条件连接等方式)复合形成的命题。如果用搭积木来比喻的话,我们已知的命题就像一块块积木,而命题连结词就是这些积木的摆放方式,我们可以把不同的积木(命题)摆在一起(连结起来),得到我们想要的状况(命题)。通过学习复合命题,你将学会如何将简单的逻辑拼接成更复杂的推理过程。无论是在解决数学难题时,还是在面对生活中的复杂决策时,理解和应用复合命题都会帮助你更加清晰地思考,增强你的推理能力,为未来的学术挑战和实际应用打下坚实的基础。

\subsection{逻辑连接词}

逻辑连接词用于将简单命题组合成更复杂的复合命题,也可以组合使用,表达更复杂的命题。常见的逻辑连接词包括“且”、“或”和“非”。在接下来的部分中,我们将详细介绍这些逻辑连接词的定义、真值表和应用实例。

请注意,尽管这部分内容已经在当前的高中教材中删除,但事实上总在出现,对下面概念的模糊会造成解题时的困惑。

\subsubsection{且}

在学习数学的过程中,你应该遇见过以下的情况:
\begin{itemize}
\item 两个或多个方程需要同时成立,这时会把他们放在一起称为“方程组”,求解方程组的方法称为“消元”。
\item 证明某个四边形是矩形,就需要满足以下两个条件同时成立:四边形的对角线相等、四个角都是直角。
\item 在集合论中,$A \cap B$表示集合$A$和$B$的交集,其中的元素必须既属于$A$,又属于$B$。
\end{itemize}

在上面的例子中,一个方程组中方程的关系、用来描述某个事物的多个条件的关系,交集的元素和原始集合的关系,就是“且”。“消元”依据的也就是“且”这个前提,毕竟只有不论在哪个方程里,同一个变量取的都是同一个值,才能做代换。
\begin{definition}{且}\label{def_HsCoPr_1}
若逻辑连接词连接的两个命题$A,B$只有同时成立,整个复合命题才为真,则这个逻辑连接词称为\textbf{且}(and,也称为\textbf{而且}、\textbf{同}、\textbf{合取}),记作\footnote{符号不做硬性要求,下同。}:
\begin{equation}
A\land B~.
\end{equation}
,读作“$A$且$B$”或者“(命题)$A,B$的合取”(the conjunction of propositions A and B)。
\end{definition}

\textbf{真值表}是一种用来系统地列出逻辑命题在各种情况下的真值的工具。为了方便刚接触命题概念的同学理解,本文的真值表都给出了两个值,事实上写哪个都可以。通过真值表,我们可以清楚地看到,真值表前面两列是各个命题的真值,最后一列是复合命题在被连接的命题取不同真值时的结果。下面是“且”的真值表。

\begin{table}[ht]
\centering
\caption{“且”真值表}\label{tab_HsCoPr1}
\begin{tabular}{|c|c|c|}
\hline
$A$ & $B$ & $A \land B$<br>($A$且$B$) \\
\hline
真(1) & 真(1) & 真(1) \\
\hline
真(1) & 假(0) & 假(0)\\
\hline
假(0) & 真(1) & 假(0)\\
\hline
假(0) & 假(0) & 假(0) \\
\hline
\end{tabular}
\end{table}

可以看出,只有当$A$和$B$都是真的时候$A\land B$($A$且$B$)才是真,否则只要$A$和$B$里面有任何一个是假(当然也包括两个全是假),$A\land B$就是假。如果把“真”换成$1$,把“假”换成$0$,则上面的真值表看起来就像是乘法运算,在复杂的逻辑表示时,也经常会像乘法一样,在不引起歧义的情况下,将$\land$省略,记作$AB$。

\subsubsection{或}

在学习数学的过程中,你也应该曾遇见过以下几种情况:
\begin{itemize}
\item 一元二次方程$(x-1)(x-2)=0$有两个解:$x=1$,$x=2$。这两个解无论代入哪个,原方程都成立。
\item 如果一个实数$x\neq0$,则要么$x<0$,要么$x>0$。
\item 在集合论中,$A \cup B$表示集合$A$和$B$的并集,其中的元素不是在$A$中,就是在$B$中(当然也可以同时在两者之中)。
\end{itemize}
在上面的例子中,一元二次方程解的关系、不等式之间的关系,并集的元素和原始集合的关系,就是“或”。
\begin{definition}{或}\label{def_HsCoPr_2}
若逻辑连接词连接的两个命题$A,B$只要有一个成立,整个复合命题就为真,则称这个逻辑连接词为\textbf{或}(or,也称为\textbf{或者}、\textbf{析取}),记作\footnote{这里提供一个记忆方法,开口向上的像一个将两侧全部装入的容器,开口向下的则像将两侧只各自存在的内容滑出去。这两个符号的开口和与他们关系紧密的“交集”和“并集”一样。}:
\begin{equation}
A\lor B~.
\end{equation}
,读作“$A$或$B$”或者“(命题)$A,B$的析取”(the disjunction of propositions A and B)。
\end{definition}

这里需要注意,我们日常生活中使用的“或者”,例如“你或者他”这样的表达,通常意味着二者之一,而不包括两者都选的情况。在逻辑学中,这种情况称为“异或”\footnote{\textbf{异或}(Exclusive or 或XOR),记作$A\oplus B$},不过这在高中阶段通常不会详细涉及,此处只是用于扩展视野和解答疑惑。

比如上面的第二个例子中“要么$x<0$,要么$x>0$”,这里的“或”其实指的是$x$的实际结果要么在一个区间内,要么在另一个区间内,这种情况在实际中可以理解为“异或”,因为$x$不可能同时在两个互斥的区间中。但在数学中的“或”通常指的是满足任一条件时就可以(即所谓的“包含或”),且在这个例子中,由于$x < 0$和$x > 0$不能同时成立,或者说即使同时成立的情况不影响最终结果,这里的“或”与“异或”是等价的,因此我们通常直接将其归为“或”(认为最终的解集是两个区间的并集)。

在高中数学中,当我们谈到“或”时,指的就是只要满足多个条件中的任意一个(而不是只能一个),结果就成立,“或”的真值表如下:

\begin{table}[ht]
\centering
\caption{“或”真值表}\label{tab_HsCoPr2}
\begin{tabular}{|c|c|c|}
\hline
$A$ & $B$ & $A \lor B$<br>($A$或$B$) \\
\hline
真(1) & 真(1) & 真(1) \\
\hline
真(1) & 假(0) & 真(1) \\
\hline
假(0) & 真(1) & 真(1) \\
\hline
假(0) & 假(0) & 假(0) \\
\hline
\end{tabular}
\end{table}

可以看出,只要$A$或$B$有一个是真的时候$A\lor B$($A$或$B$)就是真,否则也就是两个全是假的时候,$A\land B$才是假。由于真值只有0和1,因此若认为“$1+1=1$”的话,“或”看上去就像加法一样。而就像代数中对最简结果的要求是最简多项式一样,一般逻辑表达式的化简结果是由“或”连接的。

\subsubsection{非}

在我们的日常生活中,“否定”是一种常见的思维方式,它可以帮助我们明确和区分不同的情况。无论是在做决定还是在表达意见时,我们常常会用到“非”的概念,例如:
\begin{itemize}
\item 当我们说“今天不是周末”时,我们实际上是在否定“今天是周末”这个命题,明确了今天是工作日。
\item 在考试中,如果有人说:“我没有不及格。”这句话实际上是对“我不及格”这个命题的否定,意味着“我及格了”。
\item 朋友问你:“你明天上午有空吗?”你回答:“我明天上午没有空。”这里的“没”体现在对“有空”这个命题的否定,明确表示你有其他安排。
\end{itemize}

上面的“不”、“没有”、“没”都是用来表示命题的否定,就是逻辑上的“非”。

\begin{definition}{非}
若一个逻辑连接词修饰命题$A$得到的命题,真值总与原命题$A$相反,则这个逻辑连接词称为\textbf{非}(not,也称为\textbf{否定}),记作:
\begin{equation}
\lnot A\qquad\text{或者}\qquad\bar{A}~.
\end{equation}
,读作“非$A$”或“(命题)$A$的否定”(the negation of proposition A)。
\end{definition}

与之前接触的两个符号不同,“非”只作用在一个命题上,不论这个命题是简单命题,还是复合命题。“非” 的真值表如下:

\begin{table}[ht]
\centering
\caption{“非”真值表}\label{tab_HsCoPr3}
\begin{tabular}{|c|c|}
\hline
$A$ & $\lnot A$<br>(非$A$) \\
\hline
真(1) & 假(0) \\
\hline
假(0) & 真(1) \\
\hline
\end{tabular}
\end{table}

“非”作用在修饰的整个命题上,这件事在简单命题上很容易理解,但是作用在复合命题上时,很容易产生混淆,在“量词命题”和“条件命题”中需要提起精神注意。

\subsubsection{逻辑连接词的性质}

就像\aref{交集与并集的性质}{tab_HsSeOp1},逻辑连接词也有类似的性质。注意,表中的$A,B$代表命题,而等号代表着,不论命题真值为何,只要左右条件相同,等号左右的两个命题永远具有相同的真值。表中的$T,F$代表着真值为真、假的命题。下表不要求推导,也不要求记住,此处给出是为了方便查阅和对照集合中的:空集、全集、交集、并集、补集等概念进行理解。

\begin{table}[ht]
\centering
\caption{“且”与“或”的性质}\label{tab_HsCoPr4}
\begin{tabular}{|c|c|c|c|}
\hline
 & 且$\land$ & 或$\lor$ & 备注 \\
\hline
1 & $A\land B = B\land A$ & $A\lor B = B\lor A$ & 交换律(Commutative Law) \\
\hline
2 & $ A \land (B \land C) = (A \land B) \land C$  &$ A \lor (B \lor C) = (A \lor B) \lor C$ & 结合律(Associative Law) \\
\hline
3 & $ A \land (B \lor C) = (A \land B) \lor (A \land C) $  & $ A \lor (B \land C) = (A \lor B) \land (A \lor C) $ & 分配律(Distributive Law) \\
\hline
4 & $ A \lor (A \land B) = A $  &$ A \land (A \lor B) = A $ & 吸收律 (Absorption Law)\\
\hline
5 & $A\land A = A$ & $A\lor A = A$ & 幂等律(Idempotent laws)\\
\hline
6 & $ A \land (\lnot A) = F $ &$ A \lor (\lnot A) = T $  & 排中律(Laws of the excluded middle) \\
\hline
7 & $A\land F = F$ & $A\lor F = A$ & 与假命题(F)的关系 \\
\hline
8 & $ A \land T = A $  &$ A \lor T = T $ & 与真命题(T)的关系 \\
\hline
9 & $ \lnot(A \land B) = (\lnot A) \lor (\lnot B) $  &$ \lnot(A \lor B) = (\lnot A) \land (\lnot B)  $ & 德摩根定律 \\
\hline
\end{tabular}
\end{table}


\subsection{量词命题}

描述关于某些变量的通用性质或存在性质的命题,称为\textbf{量词命题}(quantified propositions)。原本开放命题中真值根据变量取值确定,量词命题使用\textbf{量词}(quantifiers)来限定开放命题中的变量,使得量词命题具有明确的真值。以“$x$是一个偶数”为例,如果用“所有”来限定$x$,即“所有的自然数$x$都是偶数”,是假命题,而用“存在”来限定$x$,即“存在一个自然数$x$是偶数”,则是真命题。

注意,如“任意$m>n$”等单独用量词来限定变量的语句是没有意义的,量词需要在命题中使用。

\subsubsection{全称量词}

在陈述中表达所述事物的全体的含义时,使用的量词称为\textbf{全称量词}(universal quantifier),记作$\forall$,可以读作“任意”、“所有”、“每一个”等。形如“集合$M$中所有的元素$x$,都满足性质$P(x)$”的命题称为\textbf{全称量词命题}或称其\textbf{恒成立}(universal proposition),记作:


\begin{equation}
\forall x\in M,P(x)~.
\end{equation}

\subsubsection{存在量词}

在陈述中表达所述事物的个体、部分或特例的含义时,使用的量词称为\textbf{存在量词}(existential quantifier,也称作\textbf{特称量词},Particular Quantifier),记作$\exists$,可以读作“存在”、“有”、“至少有一个”等。形如“集合$M$中存在某个元素$x$满足性质$P(x)$”的命题称为\textbf{存在量词命题},记作:

\begin{equation}
\exists x\in M,P(x)~.
\end{equation}

\subsection{条件命题}

\textbf{条件命题}(conditional proposition或\textbf{蕴含命题},implication)用于表达两个命题之间的条件关系,通常形如“若 $P$ ,则 $Q$ ” ,由一个前件(antecedent)$P$和一个后件(consequent)$Q$组成,通过“如果……那么……”(if…then…)的形式来连接。条件命题在计算机领域广泛使用。

“若 $P$ ,则 $Q$ ”表示如果命题 $P$ 为真,那么命题 $Q$ 也为真。条件命题通常用符号$P\Rightarrow Q$或$P\rightarrow Q$表示,也读作“$P$蕴含$Q$”\footnote{事实上,用逻辑连接词表示的蕴含的定义是$\lnot P\lor Q$,关于蕴含还有很多知识,可以参考}。

\subsubsection{充分与必要}

如果这个条件命题成立,则$P$称为$Q$的充分条件,反过来看则$Q$称为$P$的必要条件。
当且仅当
%移动自充分必要条件,并加以调整
若由命题 $A$ 能推导出命题 $B$, 则 $A$ 是 $B$ 的充分条件, $B$ 是 $A$ 的必要条件。如何理解这个定义呢?下面举两个例子。

\begin{example}{}
命题 $A$:四边形 $ABCD$ 是一个正方形。

命题 $B$:四边形 $ABCD$ 的四条边相等。

首先我们考虑 $A$ 对 $B$ 的关系。显然,由 $A$ 可以推出 $B$, 说明 $A$ 中有充分的信息能得到 $B$, 所以叫做 $B$ 的\textbf{充分条件}。 $A$ 中包括得到 $B$ 所必要的信息,还\textbf{可能}包括一些其他信息,例如由命题 $A$ 可以得出四边形任意两条临边垂直。 这些多出来的信息并不一定是得到 $B$ 所必须的,因为还有许多其他的四边形四条边相等但并不是正方形。

那如何判断 $A$ 中有没有多余的信息呢?我们可以反过来试图用 $B$ 推导命题 $A$, 若原则上得不出 $A$ (而不是因为我们逻辑水平不够),则证明 $A$ 中有多余的条件。这时我们说 $A$ 不是 $B$ 的\textbf{必要条件},因为 $A$ 中的一些信息是多余的,也就是没有必要的。综上, $A$ 是 $B$ 的\textbf{充分非必要条件}。

现在我们从 $B$ 的角度考虑。虽然由条件 $B$ 不能推导出条件 $A$, 但是 $B$ 是 $A$ 中信息的一部分, $B$ 必须要成立才有可能使 $A$ 成立,也就是说如果 $B$ 不成立 $A$ 就不可能成立(四条边不全相等的四边形一定不是正方形)。所以说 $B$ 是 $A$ 的必要条件。另外,由 $B$ 中的少量信息不能得到 $A$, 所以 $B$ 不是 $A$ 的充分条件。 综上, $B$ 是 $A$ 的\textbf{必要非充分条件}。
\end{example}


\begin{example}{}
命题 $A$ :三角形 $X$ 的其中两内个角分别为 $90^\circ$ 和 $45^\circ$。

命题 $B$ :三角形 $X$ 有两个 $45^\circ$ 的内角。

利用三角形三个内角和为 $180^\circ$ 的事实,可以从 $A$ 推出 $B$, 说明 $A$ 是 $B$ 的充分条件, $B$ 是 $A$ 的必要条件。但也可以从 $B$ 推出 $A$, 说明 $B$ 是 $A$ 的充分条件, $A$ 是 $B$ 的必要条件。所以 $A$ 和 $B$ 既是彼此的充分条件也是彼此的必要条件。所以我们说 $A$ 和 $B$ \textbf{互为充分必要条件}。若 $A$ 是 $B$ 的充分必要条件, $B$ 一定也是 $A$ 的充分必要条件。因为两种表述都意味着 $A$,  $B$ 命题\textbf{等效},所提供的信息都是一样的,两者都没有任何多余的或者缺失的信息。
\end{example}

需要注意的是:
\begin{enumerate}
\item 充分/必要条件是两个命题之间的关系,若只说一个命题是充分/必要条件没有意义。
\item 讨论充分/必要条件需要在一定的前提下进行。以上两个例子中的前提如: 我们讨论的是欧几里得几何中的平面四边形和三角形。 当然,我们也可以把这个前提直接写在每个命题中。
\item 在证明 $A$ 是 $B$ 的充分必要条件时,需要分别证明 $A$ (相对于 $B$)的充分性和必要性。充分性需要由 $A$ 证明 $B$, 必要性需要由 $B$ 证明 $A$。 
\item 在证明 $A$ 是 $B$ 的充分非必要条件时,除了需要证明 $A$ 的充分性,还需非必要性,即 $B$ 不能推出 $A$。 只要我们可以举出一个 $B$ 成立 $A$ 不成立的反例,就立刻证明了不可能由 $B$ 推出 $A$。 
\end{enumerate}

简要说明:

有命题 $A$ 和 $B$ 
\begin{enumerate}
\item $A$ 推出 $B$ , $B$ 推不出 $A$ ,则为充分不必要,如图:
\begin{figure}[ht]
\centering
\includegraphics[width=5cm]{./figures/813479566ad8bd27.png}
\caption{充分不必要} \label{fig_SufCnd_2}
\end{figure}
口诀:有之必然,无之未必不然
\item $A$ 推出 $B$ , $B$ 推出 $A$ ,则为充要,如图:\begin{figure}[ht]
\centering
\includegraphics[width=5cm]{./figures/977aec7c48daaeea.png}
\caption{充要} \label{fig_SufCnd_3}
\end{figure}
\item $A$ 推不出 $B$ , $B$ 推不出 $A$ ,则既不充分也不必要,如图:\begin{figure}[ht]
\centering
\includegraphics[width=5cm]{./figures/72797bcad5cfcfdc.png}
\caption{既不充分也不必要} \label{fig_SufCnd_4}
\end{figure}
\item $A$ 推不出 $B$ , $B$ 推出 $A$ ,则必要不充分,如图:\begin{figure}[ht]
\centering
\includegraphics[width=5cm]{./figures/b8c97ea69e5f63a9.png}
\caption{必要不充分} \label{fig_SufCnd_5}
\end{figure}
口诀:有之未必然,无之必不然
\end {enumerate}

%以上移动自充分必要条件,并加以调整




\subsubsection{逆命题与否命题}

原命题(Original Proposition):  $P \Rightarrow Q $

逆命题 (Converse):  $Q \Rightarrow P$ 

否命题 (Inverse):  $\neg P \Rightarrow \neg Q $


\begin{example}{区分命题的否定和否命题}
命题的否定和否命题造成很多人的困惑,无法分清。其实,二者只是在中文翻译上,都使用了“否”字,造成了理解上的混淆,事实上看二者的英文名称就不易混淆了,下面给出二者的对比。

\textbf{否命题}

只有条件命题才有否命题。得到否命题的过程需要分别对原命题的条件和结论两个命题进行否定。否命题与原命题的真值无关:
\begin{itemize}
\item 否命题与原命题的真值可能相同,如:原命题是“如果一个整数是偶数,那么它能被2整除。”,否命题是“如果一个整数不是偶数,那么它不能被2整除。”,原命题与否命题都是真命题;
\item 否命题与原命题的真值可能不同,如:原命题是“如果一个物体是金属的,那么它可以导电。”,否命题是“如果一个物体不是金属的,那么它不能导电。”,原命题是真命题,否命题是假命题。
\end{itemize}

\textbf{命题的否定}

所有命题都有否定。获得命题的否定时,需要对整个命题进行否定。命题的否定的真值一定与原本的命题完全相反。
针对条件命题“若p,则q”它的否定是“p且非q”\footnote{也就是$P\Rightarrow Q$的否定是$P\land\neg Q$}
比如:“如果一个整数是偶数,那么它能被2整除。”的否定是“一个整数是偶数且不能被2整除。”

\end{example}

\subsubsection{逆否命题}

逆否命题 (Contrapositive):  $\neg Q \Rightarrow \neg P $
原命题与逆否命题相等,而逆命题与否命题互为逆否命题,因此二者也相等。


\subsection{总结}

下面这道习题能够帮助你检查自己是否完全掌握了本文的内容。
\begin{exercise}{求“任意平面四边形,若这个四边形是矩形,则这个四边形的对角线长度相等且有一个内角是直角。”的否定、否命题、逆命题、逆否命题,并判断真值。}
题目给出的命题是真命题。
\begin{itemize}
\item 否定:存在一个平面四边形,这个四边形是矩形,且这个四边形的对角线长度不相等或所有内角都不是直角。(假命题)
\item 否命题:任意平面四边形,若这个四边形不是矩形,则这个四边形的对角线长度不相等或所有内角都不是直角。(真命题)
\item 逆命题:任意平面四边形,若这个四边形的对角线长度相等且有一个内角是直角,则这个四边形是矩形。(真命题)
\item 逆否命题:任意平面四边形,若这个四边形的对角线长度不相等或所有内角都不是直角,则这个四边形不是矩形。(假命题)
\end{itemize}
\end{exercise}