% 时间的计量 2

\pentry{时间的计量\upref{TimeCa}}}

UT1 is the principal form of Universal Time. While conceptually it is mean solar time at 0° longitude, precise measurements of the Sun are difficult. Hence, it is computed from determining the positions of distant quasars using long baseline interferometry, laser ranging of the Moon and artificial satellites, as well as the determination of GPS satellite orbits. UT1 is the same everywhere on Earth, and is proportional to the rotation angle of the Earth with respect to distant quasars, specifically, the International Celestial Reference Frame (ICRF), neglecting some small adjustments. The observations allow the determination of a measure of the Earth's angle with respect to the ICRF, called the Earth Rotation Angle (ERA, which serves as a modern replacement for Greenwich Mean Sidereal Time). UT1 is required to follow the relationship
 radians
where [14]

\begin{equation}
ERA = 2\pi(0.7790572732640 + 1.00273781191135448 T_u)
\end{equation}
其中 $T_u = \text{UT1 儒略日} - 2451545.0$.
