% 电离辐射
% license CCBYSA3
% type Wiki

(本文根据 CC-BY-SA 协议转载自原搜狗科学百科对英文维基百科的翻译)

\textbf{致电离辐射(电离辐射)}是指能量足够高而能使原子或分子中的电子解离、也就是使他们电离的辐射。电离辐射通常包括高能亚原子粒子和离子、高速运动的原子(通常大于光速的1\%),以及高能电磁波。

$\gamma$射线、$X$射线,以及紫外线中的高能部分属于电离辐射,紫外线中低能部分以及所有紫外线以下的所有频谱,包括可见光(包括几乎所有类型的激光)、红外线、微波、无线电波则属于非电离辐射。因为不同的分子和原子在具有不同的电离能,紫外线中电离辐射和非电离辐射之间没有明确的边界。一般习惯将边界置于10eV 和33eV之间。

典型的致电离亚原子粒子来自放射性衰变,包括α粒子、β粒子和中子。几乎所有放射性衰变产物都是致电离的,因为放射性衰变的能量通常远远高于电离所需的能量。其他自然产生的亚原子电离粒子有μ子、介子、正电子以及宇宙射线与地球大气层相互作用后产生的产生的次级宇宙射线中的其他粒子。[1][2]宇宙射线是由恒星和某些天体事件产生的,例如超新星爆炸。宇宙射线也可能在地球上产生放射性同位素(例如碳14),其会发生衰变并产生电离辐射。宇宙射线和的衰变放射性的同位素的衰变是地球上自然电离辐射的主要来源,被称为背景辐射。电离辐射也可以通过以下方式人工产生:$X$射线管、粒子加速器以及任何产生放射性同位素人工行为。

电离辐射不能直接被人类感觉到,因此必须使用辐射检测仪器(如盖革计数器)来指示并测量它们。一些高强度的电离辐射可以与物质相互作用发出可见光,如契伦科夫辐射和辐射发光现象。电离辐射可用于各种领域,如医学、核电、研究、制造、建筑和许多其他领域,但如果不采取适当的屏蔽措施,则会对健康造成危害。暴露于电离辐射会对活体组织造成损伤,并可能导致辐射灼伤、细胞损伤、放射病、癌症和死亡。

\subsection{类型}
\begin{figure}[ht]
\centering
\includegraphics[width=6cm]{./figures/42d0cf06dd16f8f8.png}
\caption{阿尔法(α)辐射由快速移动的氦-4(4He)细胞核并被一张纸挡住。Beta(β)由电子组成的辐射被铝板阻挡。伽马(γ)由高能光子组成的辐射在穿透致密材料时最终被吸收。中子(n)辐射由被轻元素阻挡的自由中子组成,如氢,它减缓和/或俘获它们。未示出:银河宇宙射线,由高能带电核组成,例如质子、氦核和被称为 HZE 离子的高电荷核。} \label{fig_DLFS_1}
\end{figure}
\begin{figure}[ht]
\centering
\includegraphics[width=6cm]{./figures/9e60e1ff6d93f21f.png}
\caption{云室是电离辐射可视化的几种方法之一。在粒子物理学的早期,它们主要用于研究,但今天仍然是一种重要的教学工具。} \label{fig_DLFS_2}
\end{figure}
电离辐射根据产生电离效应的粒子或电磁波的性质进行分类。它们有不同的电离机制,可以分为直接电离和间接电离。

\subsubsection{1.1 直接电离辐射}
任何带电的有质量粒子只要具有足够的动能都可以通过库仑力作用直接电离原子,这些粒子包括电子、$\mu$介子、带电的$\pi$介子、质子和失去电子的高能带电原子核。当在相对论性速度下移动时,这些粒子有足够的动能产生电离,但相对论性速度不是必需的。例如,典型的α粒子是致电离的,其速度约5\% 光速,而33 eV(足以产生电离)的电子速度约为1\% 光速。

要最先发现的两种致电离粒子源被赋予了的特殊名称:从原子核中发射的氦核被称为$\alpha$粒子,而通常(但不总是)在相对论性速度发射的电子被称为$\beta$粒子。

天然宇宙射线主要由相对论性质子组成,也包括较重的原子核,如氦离子和其他高能重离子。在大气中,这种粒子通常被空气分子阻止,产生短寿期的带电$\pi$介子,它们很快衰变为$\mu$子,$\mu$子是到达地面(并在一定程度上穿透地面)的一种主要宇宙射线辐射。$\pi$介子也可以在粒子加速器中大量产生。

\textbf{$\alpha$粒子}

$\alpha$粒子由两个质子和两个中子组成,与氦核相同。$\alpha$粒子辐射通常在$\alpha$衰变过程中产生,也可能以其他方式产生。$\alpha$粒子是以希腊字母表中的第一个字母命名。$\alpha$粒子的符号是$\alpha$或$\alpha^{2+}$。因为它们与氦核相同,它们有时也被写作$He^{2+}$或者$^{4}$
$2^{He^2+}$表示氦离子带有+2电荷(缺少两个电子)。如果离子从环境中获得电子,$\alpha$粒子可以写成正常的(电中性的)氦原子$^{4}$
$2^{He^2+}$。

$\alpha$粒子是一种电离能力极强的粒子辐射。当它们由放射性$\alpha$衰变产生时,它们的穿透深度很浅,只能穿透几厘米的空气,无法穿透皮肤死层。三元裂变产生的$\alpha$粒子能量是其三倍,并且在空气中穿透得更远。氦核构成宇宙射线的10-12\%,通常也比核衰变过程产生$\alpha$粒子的能量高得多,因此当能够穿过人体和密度较高的屏蔽层。然而,这种类型的辐射能够被地球大气层大大减弱,其相当于一个大约10米深水构成的防辐射罩。[3]

\textbf{$\beta$粒子}

$\beta$粒子是由某些类型的放射性核素(如钾-40)发射的高能、高速电子或正电子。$\beta$粒子产生过程被称为$\beta$衰变。它们由希腊文字母$\beta$表示。$\beta$衰变有两种形式,$\beta^-$和$\beta^+$,分别代表电子和正电子。[4]

对于具有$\beta$放射性的东西,可以用盖革计数器或其他辐射探测器检测到。当接近$\beta$发射体时,探测器将指示放射性急剧增加。当探测器探头被屏蔽以阻挡$\beta$射线时,放射性指示将显著减少。

高能$\beta$粒子在穿过物质时会通过“韧致辐射”效应产生$X$射线或“次级电子”($\delta$射线)。这两种情况都会造成间接电离效应。

当屏蔽$\beta$发射体时,轫致辐射现象需要考虑,因为$\beta$粒子与屏蔽材料的相互作用产生轫致辐射。这种效应在高原子序数的材料中更显著,因此低原子序数的材料被用于屏蔽$\beta$源。

\textbf{正电子和其他类型的反物质}

\textbf{正电子}或者\textbf{反电子}是电子的反粒子或反物质。当低能正电子与低能电子碰撞时,发生湮灭现象产生成两个或更多的$Y$光子。

正电子可由具有$\beta^+$放射性的核衰变(通过弱相互作用)产生,也可由高能光子的电子对效应产生。正电子是医用电离辐射的常用射线源,可用于正电子发射断层扫描中(PETCT)。

由于正电子是带正电荷的粒子,它们也可以通过库仑相互作用直接电离原子。

\textbf{带电原子核}

带电原子核主要来自银河宇宙射线和太阳粒子事件,除了$\alpha$粒子(带电氦核)外在地球上没有自然来源。然而,高能质子、氦核和高能重离子可以被相对较薄的屏蔽层、衣服或皮肤阻止。然而,由此产生的相互作用将产生次级辐射,并导致级联生物效应。例如,如果只有一个组织原子被高能质子撞击并移位,碰撞将在体内引起进一步的相互作用。这被称为“线性能量传输”(LET),它利用了粒子的弹性散射过程。

线性能量传输可以直观理解为为一个台球以动量守恒的方式撞击另一个台球,将两个球都带走,第一个球的能量不相等地分配给两个球。当带电原子核撞击空间中相对运动缓慢的物体原子核时,线性能量传输发生并且通过碰撞产生中子、$\alpha$粒子、低能质子和其他原子核将,这些都会对组织的总吸收剂量做出贡献。[5]

\subsubsection{1.2 间接电离辐射}
间接电离辐射是电中性的,因此不会与物质发生强烈的相互作用。大部分电离效应是由次级电离引起的。

间接电离辐射的一个例子是中子辐射。

\textbf{光子辐射}

\begin{figure}[ht]
\centering
\includegraphics[width=10cm]{./figures/5b806c515e5da559.png}
\caption{不同类型的电磁辐射} \label{fig_DLFS_3}
\end{figure}
即使光子是电中性的,它们也可以通过光电效应和康普顿效应直接电离原子。这些相互作用中的任何一种都会导致原子中的电子以相对论性速度被弹出,将该电子转化为$\beta$粒子(次级$\beta$粒子),从而电离许多其他原子。由于大多数受影响的原子被次级$\beta$粒子直接电离,光子被称为间接电离辐射。[6]

如果光子辐射是由核内的核反应、亚原子粒子衰变或放射性衰变产生的,则称为$\gamma$射线。如果它在原子核外产生,则被称为$X$射线。光子这个通用术语被用来描述这两者。[7][8][9]
\begin{figure}[ht]
\centering
\includegraphics[width=10cm]{./figures/6f9a30288dd018a0.png}
\caption{γ射线对铅的总吸收系数(原子序数82),横坐标为γ射线能量,反映了三种效应的贡献。光电效应在低能时占主导地位。高于5兆电子伏时,粒子对效应开始占据主导地位。} \label{fig_DLFS_4}
\end{figure}
$X$射线的能量通常低于$\gamma$射线,传统上习惯以波长10-11m或能量100keV的光子作为二者分界。[10]这个阈值的确定是由于旧X射线管的能力局限性和对同质异能跃迁缺乏了解导致的。现代的技术和发现导致了$X$射线和$\gamma$射线能量区域相重叠。在许多领域,它们在功能上是相同的,不同之处仅在于辐射的来源。然而,在天文学中,辐射源通常无法可靠地确定,旧的能量区分标准被保留下来,X射线被定义为大约120 eV至120 keV之间,$\gamma$射线是任何能量高于100至120 keV的射线,无论来源如何。天文学绝大多数”伽马射线天文事件“都是已经确认不起源于核放射性过程,而更接近与产生天文学X射线的过程,除了那些是由更高能电子驱动的事件。

光电效应是有机材料与低于100 keV光子相互作用的主要机制,也是最初典型的$X$射线管的原理。当能量超过100 keV时,光子主要通过康普顿散射效应产生电离,随着能量进一步提高到超过5 MeV,则通过粒子对效应产生间接电离。随后相互作用过程中会先后发生的两次康普顿散射。在两个散射事件中,$\gamma$射线将能量传递给电子,并且以不同的方向和更低的能量继续前进。

\textbf{低能光子的能量边界确定}

电离能最低的元素为铯,其店里能为3.89 eV。然而,美国联邦通信委员会将能量大于10 eV(相当于波长124纳米的远紫外线)的光子定义为电离辐射。[11]这大致相当于氧的第一电离能氧和氢的电离能,都约为14 eV。[12]在环保局的一些参考文献中,引用了典型水分子的电离能(33 eV)[13]作为电离辐射生物阈值:该值代表所谓的w值,这是国际辐射单位与测量委员会(ICRU)对“气体中每形成的一对离子消耗的平均能量”的通俗称呼,[14]其包含了电离能以及其他过程损失的能量,如激发。[15]33 eV对应38纳米波长的电磁辐射,接近传统认定的极紫外线和X射线之间分界点:约10 nm波长,125eV。因此,$X$射线总是致电离辐射,但只有极紫外辐射满足所有电离辐射的定义。

如上所述,电离辐射对细胞的生物效应有点类似于具有更广能谱分布的分子损伤辐射,后者电离辐射重叠并包含更大范围的能量的电磁辐射,在某些系统(如叶片中的光合系统)能量下限可达到紫外线区域甚至进入可见光区域。虽然DNA总是容易受到电离辐射的损害,但辐射能量足以激发某些分子键形成嘧啶二聚体,DNA分子也会收到损害,该能量可能接近但小于电离能。一个很好的例子能量在3.1 eV (400 nm)附近的紫外线,由于胶原蛋白中的光反应(在紫外线$B$波段区域内)可以导致DNA中的损伤(例如嘧啶二聚体),从而使未受保护的皮肤被晒伤。这样虽然分子中的电子的激发不足以导致电离,但会产生类似的非热效应,使得中低能紫外线能够对生物组织造成损伤。可见光和最接近可见光能量的紫外线$A$波段已经被证明某种程度上会导致皮肤中形成活性氧,这些电子激发的分子可以造成间接损害,尽管它们不会造成晒伤(红斑)。[16]同电离辐射损伤一样,皮肤中的所有这些效应都超出了简单热效应的范畴。

\textbf{中子}

\begin{figure}[ht]
\centering
\includegraphics[width=10cm]{./figures/73f47c4c7ce4d65d.png}
\caption{辐射相互作用:伽马射线由波浪线表示,带电粒子和中子由直线表示。小圆圈显示电离发生的位置。} \label{fig_DLFS_5}
\end{figure}
中子没有电荷,因此通常不会在单一过程或者与物质相互作用中产生直接电离。然而,快中子能够通过线性能量传输与氢中的质子相互作用,使目标区域中物质的原子核发生散射,导致氢原子的直接电离。因此当中子撞击氢核时,会产生质子辐射(快质子)。这些质子本身是致电离辐射,因为它们具有高能量并且带电荷,能够与物质中的电子相互作用。

中子撞击氢以外其他原子核时,发生线性能量传输时转移的能量相对较少。但对于许多被中子撞击的原子核,也会发生非弹性散射。弹性或非弹性散射的发生取决于中子的速度,是快中子还是热中子亦或介于两者之间,同时还取决于它所撞击的原子核及其中子截面。

在非弹性散射中,中子很容易通过中子俘获过程被原子核吸收而产生中子活化。中子以这种方式与大多数类型物质的相互作用通常会产生放射性核。例如,自然界中含量丰富的氧-16 核经历中子活化后,通过质子发射快速衰变形成氮-16 ,进而迅速衰变为氧-16,并发射出高能β射线。这个过程可以写成:

$16^O (n,p)16^N$(快中子能量大于11MeV时可被俘获)

$16^N \to 16^O + \beta^-\text{(衰变t_{1/2} = 7.13秒)}$

高能$\beta^-$射线进一步与其他原子核相互作用,通过切割致辐射发射高能X射线

$16^O (n,p)16^N$反应是压水堆冷却水产生X射线的主要来源,并对水冷核反应堆运行时的辐射有巨大贡献。

为了最好地屏蔽中子,应使用富含氢的碳氢化合物。

在裂变材料中,次级中子可能产生核链式反应,导致裂变物产生大量电离。

在原子核内,自由中子是不稳定的,平均寿命为14分42秒。自由中子通过发射电子和反电子中微子而衰变为质子,这个过程被称为$\beta$衰变。

在右图中,中子与目标材料的原子碰撞,然后变为反冲原子而产生电离效应。在中子路径的末端,中子被原子核(n, $\gamma$)反应俘获,发射一个中子俘获光子。这样足够的能量可以被称为电离辐射。

\subsection{物理效应}
\begin{figure}[ht]
\centering
\includegraphics[width=10cm]{./figures/29cc0bd1138a432d.png}
\caption{电离空气在来自回旋加速器的粒子电离辐射束周围发出蓝色光} \label{fig_DLFS_6}
\end{figure}

\subsubsection{2.1 核效应}
分子电离可导致辐射分解(化学键断裂),并形成高活性的自由基。即使在最初的辐射停止后,这些自由基也可能与邻近的物质发生化学反应。(例如,由空气电离形成的臭氧导致聚合物的臭氧裂解)。电离辐射还可以通过提供反应所需的活化能来加速现有的化学反应,如聚合和腐蚀。光学材料在电离辐射的作用下变暗。

空气中的高强度电离辐射可以产生可见的电离空气辉光,呈蓝紫色。在临界事故,核爆炸后的蘑茹云周围、或受损核反应堆内部如切尔诺贝利灾难都可以观察到这种辉光。

单原子流体,例如熔融的钠,不存在化学键和晶格结构,因此它们对电离辐射的化学效应免疫。具有非常负的生成焓的简单双原子化合物,如氟化氢,在电离后会迅速自发重整。

\subsubsection{2.3 电效应}
物质的电离会暂时增加它们的电导率,有可能会使电流提高到具有破坏性的水平。这对于电子设备中使用的半导体微电子器件是一个需要特别注意的危险,增大的电流会引入操作误差,甚至永久损坏器件。在高辐射环境,例如核工业和大气外层(外空间)下使用的设备,应具有抗辐照能力,需要通过设计、材料选择和制造工艺来实现这一点。

太空中的质子辐射也可能导致数字电路中的单粒子翻转现象。

电离辐射的电效应在充气辐射探测器中得到利用,例如盖革计数管及电离室。

\subsection{健康影响}
一般来说,电离辐射对生物是有害甚至致命的,但是一些类型的电离辐射具有医学应用,如治疗癌症和甲状腺毒症的放射疗法。

暴露于电离辐射对健康的不利影响主要可分为两大类:
\begin{itemize}
\item 确定性效应(组织损伤反应),很大程度上是由于收到高剂量辐射灼伤后的细胞死亡或功能失常
\item 随机效应,包括癌症和可遗传效应,涉及暴露个体中由于体细胞的突变导致的癌症或其后代中由于生殖细胞突变导致的可遗传疾病。[18]
\end{itemize}
最常见的影响是随机诱发癌症,暴露后潜伏期为几年或几十年。例如,电离辐射是慢性粒细胞白血病的唯一原因。[19]发生这种情况的机制是众所周知的,但是预测风险水平的定量模型仍然有争议。最广泛接受的模型假设电离辐射导致的癌症发病率随有效辐射剂量以每西弗5.5\%的速度增加。如果这个线性模型是正确的,那么自然的背景辐射是对公众健康最危险的辐射源,其次是医学成像。电离辐射的其他随机效应有致畸、认知下降和心脏病。

\subsection{测量学}
下表以国际单位制和非国际单位制显示辐射和剂量。附图为国际放射防护委员会给出的剂量率之间的关系。
\begin{figure}[ht]
\centering
\includegraphics[width=14.25cm]{./figures/13eb0c575f89c4e6.png}
\caption\label{fig_DLFS_7}
\end{figure}

\begin{table}[ht]
\centering
\caption{放射性和检测到的电离辐射之间关系}\label{DLFS}
\begin{tabular}{|c|c|c|c|c}
\hline
\textbf{量} & \textbf{探测器} & \textbf{CGS单位} & \textbf{国际单位制} & \textbf{其他单位} \\
\hline
衰变速度 &  & 居里 & 贝克 &  \\
\hline
粒子通量 & 盖革计数器、正比计数器、闪烁体 & 个厘米$^{2}$ · 秒 & 个米$^{2}$ · 秒 & 每分钟计数,粒子数每厘米$^{2}$每秒\\
\hline
能量通量 & 热释光剂量计,胶片卡剂量计 & 兆电子伏(MeV)厘米$^{2}$ & 焦耳米$^{2}$ &  \\
\hline
辐射能量 & 正比计数器 & 电子伏特 & 焦耳 & \\
\hline
线性能量传输 & 导出量 & 兆电子伏(MeV)厘米 & 焦耳米 & keV微米 \\
\hline
比释动能 & 电离室、半导体探测器、石英纤维剂量计、卡尼沉降计 & esu(静电单位)厘米$^{3}$ & 戈瑞(gray) & 伦琴\\
\hline
吸收剂量 & 热量计 & 拉德(rad) & 戈瑞(gray) & 伦琴当量\\
\hline
等效剂量 & 导出量 & 雷姆(rem) & 西弗(sievert) & \\
\hline
有效剂量 & 导出量 & 雷姆(rem) & 西弗(sievert) & 背景辐射当量时间\\
\hline
承诺剂量 & 导出量 & 雷姆(rem) & 西弗(sievert) & 香蕉等效剂量\\
\hline
\end{tabular}
\end{table}

\subsection{应用}
电离辐射有许多工业、军事和医疗用途。随着时间推移,必须不断地平衡其有用性与危害性。例如,曾经鞋店的店员用x光检查孩子的鞋号,但当人们更好地了解电离辐射的风险时,这种做法就停止了。[20]

中子辐射对核反应堆和核武器至关重要。$X$射线、$\gamma$射线、$\beta$射线和正电子辐射的穿透力用于医学成像、无损检测和各种工业仪表。放射性示踪物用于医疗和工业应用,以及生物和辐射化学。$\alpha$射线用于静电消除器和烟雾探测器。电离辐射的杀菌效果可用于清洗医疗器械、食品辐照和昆虫无菌技术。对碳-14 的测量可以用来确定古代生物(例如有数千年历史的木材)遗骸的年代。

\subsection{ 辐射源}
电离辐射是由核反应、核衰变、甚高温或电磁场中带电粒子的加速产生的。自然来源包括太阳、闪电和超新星爆炸。人工源包括核反应堆、粒子加速器和 $X$射线管。

下表是联合国原子辐射效应科学委员会(UNSCEAR)人类受到的辐射照射类型的分类。

\begin{table}[ht]
\centering
\caption{辐射的类型}\label{DLFS_1}
\begin{tabular}{|c|c|c}
\hline
\textbf{公众照射} &  &  \\
\hline
 & 	一般事件 & 宇宙射线\\
\hline
 &  一般事件 & 地面辐射\\
\hline
自然辐射源 & 增强辐射源 & 	金属矿床开采和熔炼\\
\hline
 &  & 磷酸盐工业\\
\hline
 &  & 采煤和煤炭发电\\
\hline
 &  & 石油和天然气钻井\\
\hline
 &  & 稀土和二氧化钛工业\\
\hline
 &  & 锆和陶瓷工业\\
\hline
 &  & 镭和钍的应用\\
\hline
 &  & 其他暴露情况\\
\hline
人工辐射源 & 和平目的 &	核电生产 \\
\hline
 &  & 核材料和放射性材料的运输\\
\hline
 &  & 核能以外的应用\\
\hline
 & 军事目的 & 核试验\\
\hline
 & & 环境中的残留物。放射性沉降物\\
\hline
历史状况 & & \\
\hline
事故暴露 & & \\
\hline
\textbf{职业辐射} & & \\
\hline
自然辐射源 &  & 空勤人员和航空人员的宇宙射线照射\\
\hline
\hline
\hline
 
\hline
\end{tabular}
\end{table}