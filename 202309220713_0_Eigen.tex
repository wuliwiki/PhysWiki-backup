% Eigen (C++ 线性代数库)笔记
% license Xiao
% type Note

\subsection{安装}
Eigen 的\href{http://eigen.tuxfamily.org/index.php?title=Main_Page}{主页}。
\begin{enumerate}
\item 首先\href{http://eigen.tuxfamily.org/index.php?title=Main_Page#Documentation}{下载} Eigen, 不需要任何安装。
\item 把解压后的文件夹加入 include 路径。
\item 在使用 Eigen 的代码中加入 \verb|#include <Eigen/Dense>| 等头文件即可。
\end{enumerate}

\subsection{基础}
\begin{itemize}
\item 与 Matlab 的\href{https://eigen.tuxfamily.org/dox/AsciiQuickReference.txt}{语法对照表}。
\item \href{https://eigen.tuxfamily.org/dox/AsciiQuickReference.txt}{ascii 快速参考}
\item 所有的 Eigen 名字都有 namespace \verb|Eigen|
\end{itemize}

\begin{lstlisting}[language=cpp]
// 矩阵
// 第一个维度固定在 3, 第二个维度可以动态变化
Matrix<double, 3, Dynamic> A;
// 两个动态维度, Aligned 可以把起始地址 padding 到一个 2^n 的整数倍的地址
// 以满足 SIMD 优化需求,默认开启。 RowMajor 是行主序矩阵
Matrix<double, Dynamic, Dynamic, Aligned | RowMajor> A 
typedef Matrix<int, 4, 4> Matrix4i;
typedef Matrix<double, 4, 4> Matrix4d;
typedef Matrix<double, Dynamic, Dynamic> MatrixXd;
typedef Matrix<double, Dynamic, 1> VectorXd;
typedef Matrix<double, 1, Dynamic> RowVectorXd;

declaration & initialization
MatrixXd a(10,15); // No initialization
a.setRandom(m,n); 随机赋值
Matrix<>::setZero(); 把矩阵赋值为 0;
Matrix<>::setZero(m,n); 把矩阵赋值为 m*n 的 0 矩阵;
\end{lstlisting}

以下函数用法类似
\verb|setOnes(); setConstant(); setIdentity(); setRandom(); setLinSpaced();|
注意其中 \verb|setConstant()| 输入 1 个变量时与 \verb|fill()| 功能相同, 输入 3 个变量时最后一个为常数。 \verb|setLinSpaced(size, val1, val2)| 只能对 Vector 或者 RowVector 使用。 如果想赋值给 \verb|MatrixXd|, 可以用 \verb|MatrixXd a = VectorXd::LinSpaced(3, 3, 4)|; 对于整型, \verb|setLinSpaced| 不保证最后一个值等于 val2, 而是保证间隔相等。 这时候只能先生成 Xd, 然后 .cast<int>.
* 逗号赋值都是 RowMaor 的, 无论 a 是什么 major.
a << 1,2,3,4;

\subsection{operations}
获取信息
\begin{itemize}
\item \verb|Matrix<>::size()| 相当于 Matlab 的 numel(), 另外, rows() 和 cols() 分别是行数和列数。
\item \verb|Matrix<>::data()| 可以获得 Matrix 数据的指针, 用于直接读写矩阵数据。 注意 rowwise 需要专门声明。 也可以用 \verb|&| 来获取矩阵元的地址。
\item 矩阵转置用 \verb|Matrix<>::transpose()| 复数矩阵共轭用 conjugate(), 共轭转置用 adjoint()
\item 矩阵元求和, \verb|Matrix<>::sum(); Matrix<>::colwise().sum(); Matrix<>::rowwise().sum()|
\item 点乘和叉乘如 \verb|v.dot(w), v.cross(w)|
类似的有 \verb|mean(), prod(), all(), any(), maxCoeff(), minCoeff()|
获取子矩阵
对 Vector, 有 \verb|Matrix<>::head(n), tail(n), segment(n)|. 对 Matrix, 有 \verb|block(i,j,rows,cols)|, 有 \verb|row(i), col(j), topRows(n), middleRows(i, rows), bottomRows(n), leftCols(n), middleCols(j, cols), rightCols(n), topLeftCorner(rows,cols), bottomRightCorner(rows,cols)| 等。
\end{itemize}


\subsubsection{逐个矩阵元的运算}
\begin{itemize}
\item \verb|Matrix<>::cwiseAbs()|
\item 类似的有 \verb|cwiseInverse(); cwiseMax(); cwiseMin(); cwiseSign(); cwiseSqrt(); Matrix<>::array()| 用于把 Matrix 转换为 array, 逐个元素运算, 或加减一个常数。 如 \verb|Matrix<>::array().square()|
\item 类似的有 \verb|round(); pow(); sqrt();| 注意 array 可以直接赋值给 matrix.
\item 复制 vector 给矩阵的每一列 \verb|mat.colwise() = v;| 类似地, \verb|+=, -=| 等也可以使用。
\end{itemize}


\subsubsection{Aliasing}
如果矩阵同时出现在等号两边, 就有可能出现 Aliasing. 例如 \verb|MatrixXi mat(3,3);  mat.bottomRightCorner(2,2) = mat.topLeftCorner(2,2);| 这时, 要在等号右边的加上 \verb|.eval()|. 特殊地, 对于转置等, 可以直接用 \verb|mat = mat.transposeInPlace()|. 类似地有 \verb|adjointInPlace()|.

\subsubsection{SVD (支持 complex)}
要用 SVD, 要 \verb|#include<Eigen/SVD>|, 例程:
\begin{lstlisting}[language=cpp]
MatrixXd  A(3,3), U, V, X;
A << 1, 2, 3, 4, 5, 6, 7, 8, 9;
BDCSVD<MatrixXd> svd(A, ComputeThinU | ComputeThinV);
U = svd.matrixU(); V = svd.matrixV(); X = svd.singularValues();
\end{lstlisting}

\subsection{外部接口}
这里介绍如何直接操作 \verb|Matrix<>| 底层的数据, 以及如何将内存中已有的数据直接给 Eigen 使用而无需复制。

\subsubsection{访问底层数据}
\begin{itemize}
\item \verb|Matrix<>::data()| 可以返回第一个矩阵元的指针。 然后就可以当做矩阵来用了。
\item 查看了一下源码, \verb|Matrix<>::innerStride()| 一律返回 1, \verb|outerStride()| 返回 \verb|innerSize()|, 而 \verb|innerSize| 对于 c-major 是 \verb|rows()|, 对于 r-major 是 \verb|cols()|. 这就说明矩阵的内存显然是连续的(无论怎么优化), 可以放心使用。
\item 要注意 \verb|Matrix<> |到底是 col-major 还是 row-major, 默认是 col-major. 不推荐用 \verb|#define EIGEN_DEFAULT_TO_ROW_MAJOR| 将默认改为 row-major. 建议用例如 \verb`typedef Matrix<double, Dynamic, Dynamic, Aligned | RowMajor> RMatrixXd.`
\end{itemize}

\subsubsection{使用内存数据}
\begin{itemize}
\item 创建 \verb|Map<>| 类即可, 所有接受 \verb|Matrix<>| 的函数同样也接受 \verb|Map<>| (除了自己写的函数)
\item \verb|Map<>| 有三个模板参数, 但是只需要用一个, 即 \verb|Map<Matrix<>>|.
\item 创建 Map object 如 \verb|Map<MatrixXf> a(duuble* pa, rows,cols);|
\end{itemize}

\subsection{其他标量类型}
Eigen 使用模板变成的巨大优势就是可以使用自定义标量类型。 代码已经默认支持 \verb|float,double,long double| 以及对应的复数。 要让矩阵元支持其他类型(如 g++ 的 \verb|__float128|,甚至 GMP 的任意精度类型), 只需要告诉 Eigen 新增标量的特性即可:
\begin{lstlisting}[language=cpp]
namespace Eigen {
template<> struct NumTraits<Qdoub> : GenericNumTraits<Qdoub>
{
    typedef Qdoub Real;
    typedef Qdoub NonInteger;
    typedef Qdoub Nested;
    static inline Real epsilon() { return FLT128_EPSILON; }
    static inline Real dummy_precision() { return 0; }
    static inline int digits10() { return FLT128_DIG; }
    enum {
        IsInteger = 0,
        IsSigned = 1,
        IsComplex = 0,
        RequireInitialization = 0,
        ReadCost = 2,
        AddCost = 2,
        MulCost = 6
    };
};

template<> struct NumTraits<Qcomp> : GenericNumTraits<Qcomp>
{
    typedef Qdoub Real;
    typedef Qdoub NonInteger;
    typedef Qdoub Nested;
    static inline Real epsilon() { return FLT128_EPSILON; }
    static inline Real dummy_precision() { return 0; }
    static inline int digits10() { return FLT128_DIG; }
    enum {
        IsInteger = 0,
        IsSigned = 0,
        IsComplex = 1,
        RequireInitialization = 0,
        ReadCost = 4,
        AddCost = 4,
        MulCost = 12
    };
};
typedef Matrix<Qdoub, Dynamic, Dynamic> MatrixXq;
typedef Matrix<Qcomp, Dynamic, Dynamic> MatrixXqc;
typedef Matrix<Qdoub, Dynamic, 1> VectorXq;
typedef Matrix<Qcomp, Dynamic, 1> VectorXqc;
} // namespace Eigen
\end{lstlisting}


\subsection{稀疏矩阵(Sparse Matrix)}

参考:\href{https://eigen.tuxfamily.org/dox/group__TutorialSparse.html}{稀疏矩阵文档主页}, \href{https://eigen.tuxfamily.org/dox/classEigen_1_1SparseMatrix.html}{SparseMatrix 类文档},\href{https://eigen.tuxfamily.org/dox/group__TopicSparseSystems.html}{解稀疏线性方程文档}。

\verb|Eigen::SparseMatrix<Scalar_, Options_, StorageIndex_>| 模板中,\verb|Scalar_| 是矩阵元的类型, \verb|Options_| 只能是 \verb|Eigen::ColMajor| 或 \verb|Eigen::RowMajor|,默认是 \verb|ColMajor|。 \verb|StorageIndex_| 是内部数组的 index 的类型,必须是有符号整数,默认 \verb|int|。 注意这不是成员函数中行标和列表的 index 类型(\verb|Eigen::Index|)。

除非特殊说明,我们下面都以 \verb|RowMajor| 为例。

\subsubsection{内部结构}
\begin{itemize}
\item Eigen 中 \verb|RowMajor| 的稀疏矩阵类似 CSR(\autoref{sub_SprMat_3}~\upref{SprMat}),但允许储存矩阵元的数组 \verb|Values| 中有一些预留的空位。
\item \verb|RowMajor| 的 \textbf{inner dimension} 是行的方向, \textbf{outer dimmension} 是列。 ColMajor 则相反。
\item \verb|InnerIndices| 是非零矩阵元的列标, 大小元素都和 \verb|Values| 一一对应,空位处的值无意义。
\item \verb|OuterStarts|: \verb|Values[OuterStarts[i]]| 是第 \verb|i| 行的第一个非零元。如果该行为零,那就和下一个非零元的 \verb|OuterStarts| 相同。 \verb|OuterStarts| 的尺寸比列数多 1,最后一个元素是 \verb|Values| 的长度。
\item \verb|InnerNNZs| 是每一列非零元素的个数(不包含空位), 长度和列数相同。
\end{itemize}

\subsubsection{常用成员函数}
\begin{itemize}
\item \verb|Scalar coeff(Index row, Index col) const|
\item \verb|Scalar &coeffRef(Index row, Index col)|
\item \verb|Index cols() const| 列数
\item \verb|Index rows() const| 行数
\item \verb|void conservativeResize(Index rows, Index cols)| 改变矩阵尺寸,保留原来的值。
\item \verb|StorageIndex *innerIndexPtr()| 返回 \verb|InnerIndices| 数组的指针
\item \verb|StorageIndex *innerNonZeroPtr()| 返回 \verb|InnerNNZs| 数组的指针
\item \verb|Scalar &insert(Index row, Index col)| 返回一个引用,用于插入一个值。该矩阵元之前不能存在值
\item \verb|bool isCompressed() const| 是否是 CSR 格式
\item \verb|void makeCompressed()| 变为 CSR 格式
\item \verb|sqeeze()| 类似 \verb|makeCompressed()|, 但仍然会保留少量空位。
\item \verb|Index nonZeros() const| 非零元的总个数
\item \verb|StorageIndex *outerIndexPtr()| 返回 \verb|OuterStarts| 数组的指针

\item \verb|void setFromTriplets(const InputIterators &begin, const InputIterators &end)| 从 \verb|vector<Eigen::Triplet<Scalar>>| 给稀疏矩阵赋值(不一定是 \verb|vector|)。
\item \verb|Scalar *valuePtr()| 返回 \verb|Values| 数组的指针
\end{itemize}

\subsubsection{解线性方程组}
\addTODO{见 \verb|SLISC/tests/eigen.cpp|}
\begin{itemize}
\item \verb`ConjugateGradient<SparseMatrix<Comp>, Lower|Upper> solver(H);`
\item \verb|solver.make(H)|
\item 要验证精度,可以用 \verb|(A*x - b).norm()/b.norm()|。
\end{itemize}
