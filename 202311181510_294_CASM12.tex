% 中国科学院 2012 年考研数学(甲)
% keys 中科院|2012|数学
% license Xiao
% type Tutor

\begin{issues}
\issueDraft
\end{issues}


\subsection{选择题}
1.函数$f(x)=xcosx^2$, 正确结论是()\\
(A).在($-\infty~,+\infty$)内有界\\
(B).当$x\to\infty$时$f(x)$为无穷大\\
(C)在($-\infty~,+\infty$)内无界\\
(D).当$x\to\infty$时$f(x)$极限存在\\

2.函数$f(x)$在($-\infty~,+\infty$)上是连续函数,且$0<m<f(x)<M<\infty$。则
$ \frac{1}{m} \int_{-m}^{m}(f(t)-M \dd{t})$的最大取值区间是()\\
(A).$(-M-m,m-M) \quad$( B).$ (2m-2M,0)$\\ (C).$(m-M,0)\qquad \qquad $(D).$(0,M+m)$

3.微分方程$y y''-(y')^2=0$的一个特解是()\\
(A).$y=xe^x$ $\quad$ (B).$y=x\ln x$ $\quad$ (C).$y=\ln x$  $\quad$ (D).$y=e^x$

4.已知$n,m$是正整数,且$n<m$,如果
\begin{equation}
A=\int_{0}^{1} x^m(1-x)^n \dd{x},B=\int_{0}^{1}x^n(1-x)^{m+1} \dd{x}~,
\end{equation}则下面结论正确的一个是()\\
(A).$A>B$ $\quad$ (B).$A=B$ $\quad$ (C).$A<B$ $\quad$ (D).$A,B$的大小关系不确定

5.函数$f(x)=e^x-x^2-4x-3$在其定义域内零点的个数是()\\
(A).1 $\quad$(B).2 $\quad$(C).3 $\quad$ (D).多于3

6.若函数$f(x)=\leftgroup{
    &e^x(sinx+cosx),&x\geqslant0\\
    &abx^2+ax+2a+b,&x<0
    }$
    的导数在$(-\infty,\infty)$上连续,则()\\
    (A).$a=2,b=-1$   $\quad$  (B).$a=2,b=-3$   \\
    (C).$a=1,b=-3$   $\quad$   (A).$a=1,b=-1$   

7.若幂级数\begin{equation}
\sum_{n=1}^\infty a_n(x-1)^n ~
\end{equation}在x=4处条件收敛,则级数
\begin{equation}
\sum_{n=1}^\infty (-1)^n(1+2^n)a_n~
\end{equation}\\
$\displaystyle \sum_{n=1}^\infty (-1)^n(1+2^n)a_n $
(A).条件收敛 $\quad$ (B).发散 $\quad$ (C)绝对收敛$\quad$(D).不能确定

8.设$S$为螺旋面$x=ucosv,y=usinv,z=v$的一部分,$0\leqslant u \leqslant \sqrt{15},0 \leqslant v \leqslant \pi,$则$\int \int_{S} \sqrt{x^2+y^2} \dd{S}$的值为()\\
(A).17 $\pi \quad$ (B).19$ \pi \quad$ (C).21$ \pi \quad$  (D).23$ \pi \quad$

9.$\displaystyle \lim_{\substack{x \to 0}} (\frac{x}{sinx})^{\frac{1}{1-cosx}}$的值为()\\
(A).$e^\frac{1}{3}\quad$     (B).$e^-\frac{1}{3}\quad$    (C).$e^\frac{1}{2}\quad$      (D).$e^-\frac{1}{2}\quad$

10.一平面过点$M(1,1,-1)$且与直线$L:\frac{x}{2}=\frac{y+1}{1}=\frac{z-3}{-1}$垂直,则该水平面与平面$x-2y-z+1=0$的交线的方向数是()\\
(A).$(-5,1,3)\quad$  (B).$(1,-3,5)\quad$  (C).$(1,-5,3)\quad$   (D).$(3,-1,5)\quad$

\subsection{应用题}
1.证明极限$\displaystyle \lim_{\substack {n \to \infty}}(\frac{1}{n}+\frac{1}{n+1}+ \dots +\frac{1}{3n})$存在,并求出极限值。

2.求微分方程$yy''-3y'+2y=e^x(2x+1)$的通解。

3.计算$\int \int_{D}(x|y|+xy)\dd{x}\dd{y}$,其中D是由抛物线$5y=x^2-6$和抛物线$y^2=x$围成的闭区域。

4.将函数$f(x)=|x-1| (0\leqslant x \leqslant \pi)$展开成正弦级数。

5.设函数$f(x)=\int_{x}^{1}e^{-t^2}\dd{t}$,求$\int_{0}^{1}x^2f(x)\dd{x}$的值.

6.计算曲线积分$\oint \frac{xc\dd{y}-y\dd{x}}{x^2+2y^2}$,其中$L$是由直线$x+y=1$和半圆周$x^2+y^2=1,x \leqslant 0$所围成的闭曲线,方向为逆时针方向。

7.设函数$f(x)$连续,且$f^2(x) \leqslant {|x|}^3$,记$F(x)=\int_{0}^{1}f(xt)\dd{t}$,求$F'(x)$,并讨论$F'(x)$的连续性。

8.函数$f(x)$在$[a,b]$上连续,在$(a,b)$内可导,$0<a<b$。证明存在\ke,使得