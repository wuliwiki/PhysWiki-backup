% 约翰·蓝道尔(综述)
% license CCBYSA3
% type Wiki

本文根据 CC-BY-SA 协议转载翻译自维基百科 \href{https://en.wikipedia.org/wiki/John_Randall_(physicist)}{相关文章}。

\begin{figure}[ht]
\centering
\includegraphics[width=6cm]{./figures/fc0ea7384a703f1d.png}
\caption{} \label{fig_YHldr_1}
\end{figure}
约翰·特顿·兰德尔爵士(Sir John Turton Randall,FRS FRSE,\(^\text{[2]}\)1905年3月23日-1984年6月16日)是一位英国物理学家和生物物理学家,以对腔体磁控管的重大改进而闻名。腔体磁控管是厘米波雷达的重要组成部分,也是盟军在第二次世界大战中取得胜利的关键技术之一,同时也是微波炉的核心部件。\(^\text{[3][4]}\)

兰德尔与哈里·布特合作,制造出一种能够以10厘米波长发射微波脉冲的电子管。\(^\text{[3]}\)对于他们发明的重要性,不列颠哥伦比亚大学维多利亚分校军事史教授大卫·齐默曼指出:“磁控管依然是所有类型短波无线电信号的关键电子管。它不仅通过使我们能够开发机载雷达系统改变了战争进程,它仍然是今天微波炉核心的关键技术。腔体磁控管的发明改变了世界。”\(^\text{[3]}\)

兰德尔还曾领导伦敦国王学院的团队研究DNA的结构。他的副手莫里斯·威尔金斯与剑桥大学卡文迪许实验室的詹姆斯·沃森和弗朗西斯·克里克共同因DNA结构的解析获得了1962年诺贝尔生理学或医学奖。他的其他团队成员还包括罗莎琳德·富兰克林、雷蒙德·戈斯林、亚历克斯·斯托克斯和赫伯特·威尔逊,他们都参与了DNA结构研究。
\subsection{教育与早年生活}
约翰·兰德尔于1905年3月23日出生在兰开夏郡纽顿利威洛斯,是悉尼·兰德尔(Sidney Randall,苗圃及种子商人)与其妻汉娜·考利(Hannah Cawley,约翰·特顿[John Turton]之女,后者是当地煤矿经理)唯一的儿子,也是三名孩子中的长子。\(^\text{[2]}\)他在阿什顿因梅克菲尔德(文法学校和曼彻斯特维多利亚大学接受教育,1925年获得物理学一级荣誉学位和毕业奖学金,1926年获得理学硕士学位。\(^\text{[2]}\)

1928年,他与多丽丝·达克沃斯结婚。
\subsection{职业与研究}
1926年至1937年间,兰德尔在通用电气公司位于温布利的实验室从事研究工作,他在为放电灯研发发光粉方面发挥了重要作用。 他还积极研究这些发光机制。\(^\text{[2]}\)

到1937年,他已被公认为英国该领域的领先研究者,并获得了伯明翰大学英国皇家学会奖学金,在马克·奥利芬特领导的物理系与莫里斯·威尔金斯合作,研究磷光的电子陷阱理论。\(^\text{[5][6][7][8]}\)
\subsubsection{磁控管}
