% 开放系统互联基本参考模型
% 网络 模型 五层 OSI

\textbf{开放系统互联基本参考模型}(Open System Interconnection Reference Model, OSI/RM)是由国际标准化组织提出的一种试图使得各种计算机在世界范围内互联成网的标准框架。该模型的目标是使得全球计算机能够遵循同一个协议,互相连接并交换数据。

OSI模型具有七个逻辑层次。从下到上依次为:物理层、数据链路层、网络层、传输层、会话层、表示层、应用层。逻辑层次的意思是对于网络上所传输的数据流的封装的层次,而并非是真实物理存在的实体。
\begin{figure}[ht]
\centering
\includegraphics[width=5cm]{./figures/fe3db6a185508986.png}
\caption{OSI七层模型} \label{fig_OSIRM_1}
\end{figure}

\subsection{物理层(physical layer)}

物理层协议其实就是最底层的通讯协议。该层上传输的是数据的二进制位,即比特(bit)。物理层要做的是用适当的高、低电压来表示0和1。传输信号所用的通讯线路,例如网线、电缆、光缆、微波信道等不属于物理层,也有人将其称为第0层。

\subsection{数据链路层(data link layer)}

数据链路层常简称为链路层。数据在网络上传输的基本单位是包含两个相邻节点主机的链路。发送方的数据链路层的基本任务就是将上层网络层交付的数据报\textbf{封装成帧}(framing),然后在两个相邻节点之间传输\textbf{帧}(frame)。每个帧包含必要的控制信息(如同步信息、地址信息、差错控制等)。接收方根据接受到的帧的控制信息,能够知道一个帧从哪个比特开始和到哪个比特结束。接收方取出帧内部封装的网络数据包之后,交给上面的网络层。

\subsection{网络层(network layer)}
网络层负责为分组交换网上的不同主机提供通信服务。在发送数据时,网络层把传输层产生的报文段或用户数据报封装成\textbf{分组}或包进行传送。用IP协议,因此分组也叫做\textbf{IP数据报},或简称为\textbf{数据报}。



\textbf{参考文献:}
\begin{enumerate}
\item 谢希仁. 计算机网络(第7版). 电子工业出版社
\end{enumerate}