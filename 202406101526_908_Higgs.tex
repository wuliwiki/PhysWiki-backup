% 希格斯粒子(科普)
% license CCBYSA3
% type Wiki

(本文根据 CC-BY-SA 协议转载自原搜狗科学百科对英文维基百科的翻译)

\section{希格斯粒子}

希格斯玻色子是粒子物理标准模型中的基本粒子,由粒子物理理论中的希格斯场量子激发产生。它是以物理学家彼得·希格斯(Peter Higgs)的名字命名的,Higgs在1964年与其他五位科学家一起提出了这种粒子存在的机制。它的存在于2012年由欧洲核子中心(CERN)的ATLAS实验组和CMS实验组基于大型强子对撞机(LHC)上的对撞实验的联合确认。

2013年12月10日,彼得·希格斯(Peter Higgs)和弗朗索瓦·恩格勒(François Englert)两位物理学家,因为他们的理论预测被授予诺贝尔物理学奖。虽然希格斯的名字已经与这个理论(希格斯机制)联系在一起,但在1960年至1972年间,还有其他的一些研究人员在这个问题的不同方面作出了自己独立的贡献。

主流媒体经常将希格斯玻色子称为“\textbf{上帝粒子}”,源于1993年一本关于这个话题的书[1] ,但是许多物理学家,包括希格斯本人,都认为这个绰号是有些夸大的。
