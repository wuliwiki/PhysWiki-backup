% 尼古拉·特斯拉(综述)
% license CCBYSA3
% type Wiki

本文根据 CC-BY-SA 协议转载翻译自维基百科\href{https://en.wikipedia.org/wiki/Nikola_Tesla}{相关文章}。


\begin{figure}[ht]
\centering
\includegraphics[width=6cm]{./figures/4da9af8e4f34bc2b.png}
\caption{} \label{fig_Tesla_1}
\end{figure}
尼古拉·特斯拉(/ˈnɪkələ ˈtɛslə/;塞尔维亚西里尔字母:Никола Тесла,[nǐkola têsla];1856年7月10日 – 1943年1月7日)是塞尔维亚裔美国工程师、未来学家和发明家。他以对现代交流电(AC)电力供应系统设计的贡献而闻名。

特斯拉出生并成长于奥斯曼帝国,在1870年代,他首先学习了工程学和物理学,但并未获得学位。随后,他在1880年代初期,在电话通信和大陆爱迪生公司(Continental Edison)新兴的电力行业中积累了实践经验。1884年,他移民到美国,并成为美国公民。他在纽约市的爱迪生机械厂工作了短暂时间后,便开始独立创业。在合作伙伴的帮助下,为了融资和推广自己的创意,特斯拉在纽约设立了实验室和公司,开发各种电气和机械设备。他的交流电感应电动机和相关的多相交流电专利,于1888年获得了西屋电气公司的许可,这使他赚得了可观的财富,并成为该公司最终推广的多相电系统的基石。

为了开发可以申请专利并商业化的发明,特斯拉进行了多种实验,包括机械振荡器/发电机、电气放电管和早期的X射线成像。他还制造了一艘无线控制的船,是最早展出的一批之一。特斯拉作为发明家广为人知,并在他的实验室向名人和富有的赞助人展示自己的成就,他的公共讲座也因其表演性质而备受关注。整个1890年代,特斯拉在纽约和科罗拉多斯普林斯进行高电压、高频率的电力实验,追求无线照明和全球无线电力传输的构想。1893年,他宣布了使用自己设备进行无线通信的可能性。特斯拉试图将这些想法付诸实践,通过未完成的沃登克利夫塔项目,这是一个跨洲的无线通信和电力传输塔,但在资金耗尽之前他未能完成该项目。

在沃登克利夫塔之后,特斯拉在1910年代和1920年代进行了一系列发明实验,取得了不同程度的成功。由于花费了大部分的钱,特斯拉在一系列纽约酒店中居住,并留下了未付的账单。他于1943年1月在纽约市去世。特斯拉的工作在他去世后逐渐被遗忘,直到1960年,国际计量大会将国际单位制(SI)中磁通密度的单位命名为“特斯拉”,以此向他致敬。自1990年代以来,特斯拉的公众兴趣重新兴起。

\begin{figure}[ht]
\centering
\includegraphics[width=6cm]{./figures/4e958945d0c8af76.png}
\caption{} \label{fig_Tesla_2}
\end{figure}