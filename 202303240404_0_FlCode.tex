% 库仑函数程序(Matlab 和 Mathematica)

\pentry{库仑函数\upref{CulmF}}

\addTODO{$G_l$ 没有定义。}

\subsection{库仑函数}
在 Mathematica 中定义库仑函数\footnote{注意较新版本的 Mathematica 中已经定义了该函数, 再次定义会报错。} $F_l(\eta, \rho)$ 为(\autoref{CulmF_eq1}~\upref{CulmF}) % 已验证
\begin{lstlisting}[language=mma]
CoulombF[l_, η_, ρ_] := 2^l E^(-π η/2) \
    Abs[Gamma[l + 1 + I η]]/(2 l + 1)! ρ^(l + 1) E^(-I ρ) \
    Hypergeometric1F1[l + 1 - I η, 2 l + 2, 2 I ρ]
\end{lstlisting}
或者使用\autoref{CulmF_eq2}~\upref{CulmF} % 已验证
\begin{lstlisting}[language=mma]
CoulombF[l_, η_, ρ_] := (-I/2)^(l + 1) \
   2^l E^(-π η/2) Abs[Gamma[l + 1 + I η]]/(2 l + 1)! \
   WhittakerM[I η, l + 1/2, 2 I ρ]
\end{lstlisting}

在 Matlab 中定义(\autoref{CulmF_eq2}~\upref{CulmF})为, 需要使用符号计算工具箱 % 已验证
\begin{lstlisting}[language=matlab, caption=coulombF\_sym.m]
% 第一类库仑函数 F_l(eta, rho)
% 已使用 Mathematica 验证
% l, eta 是标量, rho 支持数组
function ret = coulombF_sym(l, eta, rho)
syms L Eta Rho;
F = (-0.5i)^(L+1)*2^L*exp(-sym(pi)*Eta/2)....
    *abs(gamma(L+1+1i*Eta))/factorial(2*L+1)....
    *whittakerM(1i*Eta, L+1/2, 2i*Rho);

F = subs(F, L, vpa(l,17));
F = subs(F, Eta, vpa(eta,17));
ret = real(double(subs(F, Rho, vpa(rho,17))));
end
\end{lstlisting}

\subsection{库仑平面波程序}
以下 Maltab 程序用于计算\autoref{CulmWf_eq7}~\upref{CulmWf}。
\begin{lstlisting}[language=matlab, caption=coulomb\_plane.m]
% 直角坐标系的库仑平面波
% Sign = 1 球面波出射, Sign = -1 球面波入射
% 'numel(k) = 3'
% 归一化: delta(\vec k - \vec k'),
%     无穷远平面波振幅 1/(2*pi)^1.5
function Psi = coulomb_plane(k,X,Y,Z,ZZ,Sign)
if (ZZ < 0)
    error('ZZ (nuclear charge) must be positive!');
end
Sign = sign(Sign);
k0 = norm(k);
eta = -ZZ/k0;
k_dot_r = k(1)*X + k(2)*Y + k(3)*Z;
kr = k0*sqrt(X.^2+Y.^2+Z.^2);
C = 1/(2*pi)^1.5 * double(gamma(1+Sign*1i*vpa(eta))) * exp(-pi*eta*0.5);
Psi = C * exp(1i*k_dot_r) .* hypergeom(-Sign*1i*eta, 1, ....
    Sign*1i*kr - 1i*k_dot_r);
end
\end{lstlisting}

一个并行版的程序(参考 Matlab 并行运算笔记\upref{MatPar}):
\begin{lstlisting}[language=matlab, caption=coulomb\_plane\_par.m]
% 直角坐标系的库仑平面波 (coulomb_plane 的并行版, 需要 parallel toolbox)
% 6 和 12 线程的机器上, 12 worker 的效率大概是单线程的 5.32 倍
% parfor 划分成列是划分成元素的 3 倍, 因为 hypergeom 输入矩阵效率最高

% Sign = 1 球面波出射, Sign = -1 球面波入射
% 'numel(k) = 3'
% 归一化: delta(\vec k - \vec k'),
%     无穷远平面波振幅 1/(2*pi)^1.5
function Psi = coulomb_plane_par(k,X,Y,Z,ZZ,Sign)
if (ZZ < 0)
    error('ZZ (nuclear charge) must be positive!');
end
Sign = sign(Sign);
k0 = norm(k);
eta = -ZZ/k0;
k_dot_r = k(1)*X + k(2)*Y + k(3)*Z;
kr = k0*sqrt(X.^2+Y.^2+Z.^2);
C = 1/(2*pi)^1.5 * double(gamma(1+Sign*1i*vpa(eta))) * exp(-pi*eta*0.5);
Psi = zeros(size(X)); Ncol = size(Psi, 2);
parfor i = 1:Ncol
    Psi(:,i) = C * exp(1i*k_dot_r(:,i)) .* hypergeom(-Sign*1i*eta, 1, ....
        Sign*1i*kr(:,i) - 1i*k_dot_r(:,i));
end
end
\end{lstlisting}

对应的 Mathematica 程序 % 已和 Matlab 数值验证
\begin{lstlisting}[language=mma]
CoulombPlane[k_, r_, Z_, Sgn_] := 
 Module[{k0, η, kdotr, kr, Coef, S},
  k0 = Norm[k]; η = -Z/k0; S = Sign[Sgn];
  kdotr = Dot[k, r]; kr = k0 Norm[r];
  Coef = 1/(2 π)^(3/2) Gamma[1 + S I η] Exp[-π η/2];
  Psi = Coef Exp[I kdotr] Hypergeometric1F1[-S I η, 1, 
     S I kr - I kdotr]]
\end{lstlisting}
