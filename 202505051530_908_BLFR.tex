% 布拉利-福尔蒂悖论(综述)
% license CCBYSA3
% type Wiki

本文根据 CC-BY-SA 协议转载翻译自维基百科\href{https://en.wikipedia.org/wiki/Burali-Forti_paradox}{相关文章}。

在集合论中,布拉利-福尔提悖论展示了构造“所有序数的集合”会导致矛盾,因此证明了在允许构造该集合的系统中存在反论证。该悖论以切萨雷·布拉利-福尔提命名,他在1897年发表了一篇论文,证明了一个定理,这个定理在他未曾意识到的情况下,与乔治·康托尔先前证明的结果相矛盾。伯特兰·罗素随后注意到了这一矛盾,并在1903年出版的《数学原理》一书中提到了这一点,称这一矛盾是由布拉利-福尔提的论文启发的,因此该悖论以布拉利-福尔提的名字命名。
\subsection{用冯·诺依曼序数表述}
我们将通过反证法来证明这一点。

令$\Omega$是由所有序数构成的集合。
\begin{enumerate}
\item $\Omega$是传递的,因为对于$\Omega$的每一个元素$x$ (它是一个序数,可以是任何序数)和$x$ 的每一个元素$y$ (即根据冯·诺依曼序数的定义,对于每个序数$y < x$),我们有y是Ω的一个元素,因为根据该序数构造的定义,任何序数只包含序数。
\item $\Omega$是按成员关系良序的,因为它的所有元素也都按此关系良序。
\item 因此,根据步骤2和3,我们知道Ω是一个序数类,并且根据步骤1,它也是一个序数,因为所有作为集合的序数类也是序数。
\item 这意味着$\Omega$是$\Omega$的一个元素。
\item 根据冯·诺依曼序数的定义,$\Omega< \Omega$等同于$\Omega$是$\Omega$的一个元素。这个结论由步骤5证明。
\item 但是,没有任何一个序数类小于它自己,包括Ω,因为根据步骤4($\Omega$是一个序数类),即$\Omega \not\prec \Omega$。
\end{enumerate}
我们从$\Omega$作为集合的前提推导出了两个矛盾的命题($\Omega< \Omega$ 和$\Omega \not\prec \Omega$),因此我们证明了$\Omega$不是一个集合。
\subsection{更一般地说}
上述悖论版本是时代错误的,因为它假设了约翰·冯·诺伊曼对序数的定义,根据该定义,每个序数是所有前序序数的集合,而这个定义在布拉利-福尔蒂提出这个悖论时尚未被知道。以下是一个假设较少的描述:假设我们以一种未指定的方式将每个良序排列与一个称为其顺序类型的对象关联(顺序类型就是序数)。这些顺序类型(序数)本身自然地是良序的,这种良序排列必须具有顺序类型$\Omega$。很容易证明,在朴素集合论中(并且在ZFC中仍然成立,但在新基础中不成立),所有小于固定α的序数的顺序类型是$\alpha$本身。因此,所有小于$\Omega$的序数的顺序类型就是$\Omega$本身。但这意味着$\Omega$作为序数的一个适当初始段的顺序类型,严格小于所有序数的顺序类型,而后者根据定义是$\Omega$本身。这就产生了矛盾。

如果我们使用冯·诺伊曼的定义,其中每个序数被认定为所有前序序数的集合,那么悖论是不可避免的:所有小于固定$\alpha$的序数的顺序类型是$\alpha$本身这一问题必须成立。冯·诺伊曼序数的集合,就像罗素悖论中的集合一样,在任何具有经典逻辑的集合论中都不能是一个集合。但是,在新基础理论中,顺序类型的集合(定义为良序排列在相似性下的等价类)实际上是一个集合,并且悖论被避免了,因为小于$\Omega$的序数的顺序类型实际上并不是$\Omega$。
\subsection{悖论的解决方案}
现代的形式集合论公理,如ZF和ZFC,通过不允许使用“所有具有性质P的集合”这样的术语来构造集合,从而规避了这一反逻辑现象,这在朴素集合论中是可能的,也在戈特洛布·弗雷格的公理体系中是可能的——特别是《算术基本法则》中的基本法则V。奎因的系统《新基础》(NF)采用了不同的解决方法。罗塞尔(1942年)证明,在奎因系统《数学逻辑》(ML)的原始版本中,即《新基础》的扩展,可以推导出布拉利-福尔蒂悖论,证明该系统是矛盾的。奎因在罗塞尔发现后对ML进行的修订不再存在这个缺陷,事实上,随后由王浩证明了其与NF是等一致的。
\subsection{另见}
\begin{itemize}
\item 绝对无限
\end{itemize}
\subsection{参考文献}
\begin{itemize}
\item 布拉利-福尔蒂,切萨雷(1897),“关于超限数的一个问题”,《巴勒莫数学学会会报》,11: 154–164,doi:10.1007/BF03015911,S2CID 121527917
\item 欧文·科皮(1958),“布拉利-福尔蒂悖论”,《科学哲学》,25(4): 281–286,doi:10.1086/287617
\item 格雷戈里·H·摩尔;亚历杭德罗·加尔西亚迭戈(1981),“布拉利-福尔蒂悖论:对其起源的重新评估”,《数学历史》,8(3): 319–350,doi:10.1016/0315-0860(81)90070-7
\item 巴克利·罗塞尔(1942),“布拉利-福尔蒂悖论”,《符号逻辑学杂志》,7(1): 1–17,doi:10.2307/2267550,JSTOR 2267550,MR 0006327,S2CID 13389728
\end{itemize}
\subsection{外部链接}
\begin{itemize}
\item 斯坦福哲学百科全书:“悖论与当代逻辑”——安德里亚·坎蒂尼著
\end{itemize}
