% 南京大学 2013 年考研普通物理
% 南大|南京大学|普物|普通物理
\subsection{力学}
1. 人造地球卫星的轨道如果贴近地面,卫星绕地球运动的周期为 $T$,当卫星的轨道贴近 $X$ 星球表面时,卫星绕 $X$ 星球运动的周期 $T' \approx T$,这说明 $X$ 星球何种属性与地球相近?

2. 质量为 $m$ 的物体,放在光滑水平面上.物体的左右两侧分别与劲度系数为 $k$ 的弹簧相连,左侧弹簧末端固定在墙上.现在右侧弹簧末端施加拉力 $F$,并移动一段距离 $l$ 后停止,就拉力移动速度非常缓慢和非常迅速两种情况,求出拉力所做的功.
\subsection{热学}
1. 两瓶同种气体,他们的体积、压强相同,但温度不同,试说明\\
(1) 单位体积内的分子数是否相同\\
(2) 哪瓶气体的分子平均速率大.

2. 有一高100米的拦水大坝,坝顶和坝底的水有 $10^\circ\mathrm{C}$ 温差.试比较在理想情况下,利用温差发电和落差发电分别从1克水中利用的能量,说明何种方式比较有利.实际生活中一般采取何种方式?
\subsection{电磁学}
1. 写出麦克斯韦方程组,并说明物理意义.

2. 假设一无限长的密绕螺线管载有电流 $I$,单位长度的匝数为 $n$,不考虑漏磁\\
(1) 求螺线管内外的磁场分布;\\
(2) 螺线管上通或不通电流,对在螺线管外运动的电子而言是否有区别?为什么?
\subsection{光学}
1. 可将光入射到一个有250条/毫米的投射性ping'mian
