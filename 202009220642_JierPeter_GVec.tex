% 几何矢量
% 线性代数|几何矢量|矢量|单位矢量|标量|平行四边形法则|坐标

\begin{issues}
\issueTODO
\issueOther{本词条需要重新创作和整合,融入章节逻辑体系.}
\end{issues}

\pentry{集合\upref{Set}, 充分必要条件\upref{SufCnd}}

我们来回顾高中学的\textbf{几何矢量}, 本文中常简称为“矢量”. 粗略地说,矢量是空间中的一些有长度有方向的箭头. 我们对它的位置不感兴趣, 所有长度和方向相同的矢量都视为同一矢量. 本书中矢量用正黑体表示, 如 $\bvec a$. 在手写时, 可以在字母上方加箭头表示, 如 $\overrightarrow{a}$. 特殊地, 如果一个矢量的长度等于 1, 那么它就是一个\textbf{单位矢量}, 本书中在矢量上面加上 “\^{}” 符号表示单位矢量, 如 $\uvec a$. 为了与矢量区分, 我们把单个的实数或复数称为\textbf{标量}.

\subsection{几何矢量}

\textbf{矢量(vector)}又被翻译为\textbf{向量}.这个概念最初来自直观的几何矢量.早在民国时期,学者们引入vector的概念时就将它译成了汉语. 由于当时的物理学家和数学家没有太多交集, 物理学家将它翻译为向量, 而数学家翻译为矢量. 90 年代时, 国家名词委员会商议确定一个统一的 vector 译名, 但却无法轻易割舍两个译名中的任何一个, 因为它们都非常信达雅地表明了vector 的含义:向量即有方向的量, 矢量即像箭矢一样的量. 大概是出自物理学家和数学家的互相尊重, 最后确定的方案是双方互换译名, 从此物理学界称矢量, 数学界称向量\footnote{以下加粗部分为力学家朱照宣教授的回忆.\textbf{在20世纪90年代初,国家名词委为此(vector)召开会议,想协调双方,由主任钱三强亲自主持. 我曾戏称这是个 “一字会”. 当时的情况是, 学科有分支, 术语有派生, 犹如家族有后裔. 祖宗互相谦让, 但子孙繁多, 已无法协调. 钱先生在会上没有说倾向于哪方面的话. 矢量、向量的分歧,一直维持到今. 力学这学科, 和数学、物理同样有 “亲”, 力学中 vector 用什么? 当年我在 “一字会” 后还有情绪, 埋怨钱先生作为领导“不表态”. 过了好些年,才懂得这类事,最多只能因势利导, 不能靠行政命令或专家拍板. 事实上, 台湾物理界至今用的是还 “向量”.}}. \textbf{本书中不区分两个译名的使用}.

本词条中讨论的\textbf{几何矢量}, 也是经典物理学中最长见的一类矢量, 是一种具有\textbf{长度}和\textbf{方向}的量,因此可以画成箭头来表示.生活中这样具有长度和方向的量十分常见: 速度有大小有方向,因此可以表示为箭头,箭头的长度代表速度的大小;加速度\upref{VnA}也有大小, 有方向; 我从一个地方运动到另一个地方,那么从起点到终点可以画一根箭头, 这箭头就是位移矢量\upref{Disp}. 不止在生活中, 一切领域里具有方向和大小概念的量都被称为几何矢量. 

为什么要强调是“几何”矢量呢?在数学中,矢量的含义要比几何矢量更广泛,也就是说,几何矢量虽然是数学家所研究的矢量的一种, 但不是唯一的一种, 广义来说任何矢量空间\upref{LSpace}中的元素都叫做矢量. 以后会看到, 本文介绍的矢量是一个 “实数域上的赋范线性空间” 中的元素.这个名字听起来很吓人,但它其实只是三个概念的结合:实数域,范数和线性空间.实数域即我们所熟悉的实数,它作为一个域使得我们可以用它来构建线性空间,而范数则是线性空间上一种表达向量的“绝对值”的函数.这些概念我们都会在对应词条中深入讨论.

\subsubsection{零矢量}
特别地,长度为零的矢量称为\textbf{零矢量(zero vector)}, 不作方向区分(可以认为它与任何矢量平行). 零矢量依然是矢量, 要注意和数字零进行区分.

需要注意的是,为了避免引入过多的要素从而导致概念过于复杂,我们目前讨论的几何矢量只有两个本质属性,长度和方向.将任一矢量进行任意平移后, 它仍然是同一个矢量.


当你熟悉了基本的微积分和线性代数后,可以进入微分几何的学习.在微分几何中,我们讨论一种叫“切向量”的对象,这就是一种和所在位置有关的抽象的向量.当然,本节作为线性代数的第一节课,到切向量还有不少东西要掌握,因此只需要知道有这么回事就足够了.

当然,如果矢量平移并不改变矢量本身,我们完全可以把所有几何矢量的起点都挪到一起,以这个公共起点为\textbf{原点},那么空间中\textbf{矢量的终点所在点}就和\textbf{矢量本身}一一对应.这可能会带来一个问题:我们知道,杠杆原理中力的作用效果和力的作用点相关,而力又是矢量,这似乎和“平移不变”矛盾.实际上,在描述杠杆原理等规律时,我们使用两个矢量,一个表示力,一个则是力的作用点的位置矢量,位置矢量的原点根据问题需要来选取.也就是说,描述杠杆原理中力的作用本来就需要两个点的坐标,你可以看成需要的是“力的起点和终点”,也可以按照平移不变原则来看成需要的是“公共起点已给定时,作用点的位置和力的终点的位置”.

当把所有矢量的起点都挪到原点后,我们就可以用终点的几何坐标来描述几何矢量.


\subsection{坐标和维度}

从现在开始,如无特别说明,我们默认几何矢量起点在原点.这样,我们就可以把一个矢量理解为空间中的一个点,即它的终点.本节仅从直觉上引入坐标的概念,基本上是符合多数人的几何直觉的.要更准确地描述相关概念,请参考词条.\addTODO{请援引相关词条.}

我们知道,实数轴可以看成一个一维的几何空间.如果以数字$0$所在的点为原点,那么我们也可以把这根轴本身看成一个几何矢量的集合,每个数字所在的点都是一个几何矢量.如\autoref{GVec_fig1} 所示,点$P$表示一个长度为$3.14$的矢量,而点$Q$表示一个长度为$6$的矢量.这样一来,每个实数$x$都可以表示一个矢量,其长度为$\abs{x}$;当$x>0$时,对应的矢量指向正方向,当$x<0$时指向负方向,而对于$x=0$,它对应的是零矢量,而零矢量没有方向之分.\textbf{任何方向的零矢量都是同一个矢量}.

\begin{figure}[ht]
\centering
\includegraphics[width=10cm]{./figures/GVec_1.pdf}
\caption{实数轴上的几何矢量.} \label{GVec_fig1}
\end{figure}

实数轴上的矢量只有一个方向可以选择,虽然有正负之分,但都沿着一条线.因此只需要用一个实数就可以唯一地表示一个矢量.这样的实数,就是对应矢量在实数轴上的一个坐标,该实数的绝对值就是矢量的“长度”.


这里所说的矢量的长度和日常经验可能有所不同.当我们讨论一个长度为$1\Si{m}$的位移矢量时,我们可以有不同的坐标来描述这一个矢量,或者说,可以把$1\Si{m}$对应到不同的实数.如果我们用每格长$1\Si{m}$的坐标轴去度量,那么这个矢量的“长度”就是$1$;如果用每格长$1\Si{mm}$的坐标轴去度量,那么其“长度”就是$1000$.因此我们所说的长度并不具有绝对的意义,而只有相对的意义\footnote{严格来说,这是因为对于同一个矢量空间,我们可以赋予不同的范数/内积\upref{InerPd}.}:同一个矢量的长度在不同坐标表示下可能不一样,但是无论在什么坐标表示下,两条“共线”的矢量的长度之比都是一样的.


实数轴上的矢量没有方向区分(最多是正负之分),因此不用考虑方向,方便我们专心讨论长度问题.实数轴被称为一个\textbf{一维}的几何矢量空间,因为所有矢量只需要\textbf{一个数字}就能\textbf{唯一}确定.

是否存在需要更多数字才能确定一个矢量的空间呢?当然有,二维平面就是这样空间,其上的矢量就需要两个数字来描述.这个时候矢量就有方向之分了,如\autoref{GVec_fig2} 中矢量$P$的方向“北偏东$45^\circ$”和$Q$的方向“正东”就有所区别了.

\begin{figure}[ht]
\centering
\includegraphics[width=6cm]{./figures/GVec_2.pdf}
\caption{二维平面上的几何矢量.} \label{GVec_fig2}
\end{figure}

因为矢量是“具有长度和方向”的量,我们可以用一个数字来表示矢量的长度,一个数字来表示矢量的方向(从给定轴算起逆时针旋转的角度,即辐角),这就是极坐标\upref{Polar}的表示方法.当然,我们更熟悉的是用直角坐标系来表示,如\autoref{GVec_fig2} 所示,如果矢量$P$和$Q$的长度都是$1$,但$P$的辐角为$\pi/4$,$Q$的辐角为$0$,那么直角坐标系下就可以把$P$表示为$(\sqrt{2}/2, \sqrt{2}/2)$,把$Q$表示为$(1, 0)$.

同理,三维空间中的矢量需要用三个数字来确定.

\begin{figure}[ht]
\centering
\includegraphics[width=7cm]{./figures/GVec_3.pdf}
\caption{三维空间里的几何矢量.} \label{GVec_fig3}
\end{figure}

矢量的坐标表示,不一定直接体现了矢量的两个要素:“方向”和“长度”,但是一定蕴含了这两个要素.比如说,\autoref{GVec_fig3} 中矢量$P$的长度并没有直接出现在其坐标$(a_x, a_y, a_z)$中,但是可以用坐标计算出来:$\sqrt{a_x^2+a_y^2+a_z^2}$.

我们容易想象出 $1$ 到 $3$ 维的几何矢量空间以及它们的两种运算, 但却很难想象更高维的情况.作为本章的主要目的,当你理解了本章的所有甚至只是部分词条后,应当可以熟悉该如何讨论任意维度的空间.讨论的核心在于,$n$维空间中的每个矢量都需要$n$个数字来确定,我们可以“定义”一个直角坐标系,把$3$维直角坐标系中的性质推广开来.

\subsubsection{拓展:斜坐标系}

当笛卡尔首次提出“坐标”的概念时,他并没有指定必须是“直角坐标”.实际上,他认为任何两根不平行的直线都可以用来刻画二维平面上的坐标,具体方式详见\textbf{斜坐标系}\upref{ObSys}.不只是二维平面,任意维度的空间中都可以使用斜坐标系来描述空间中的点.

使用直角坐标系的好处是高度对称,因为当你选定相互垂直的$x$轴和$y$轴以后,$y$轴的两边到$x$轴的角度都是$\pi/2$;另外,几何学中的勾股定理依赖于直角;虽然我们也可以用余弦定理来描述任意三角形的边长关系,但比起勾股定理要麻烦得多,因为表达式中多了一项.使用直角坐标系能让讨论简洁很多.

你可能注意到了,“垂直”本身是一个经验上的概念,似乎没有任何理论上的方法可以不借助量角器等工具判断一个角到底是不是直角.也就是说,“垂直”是一个没有定义的概念.我们将会在\textbf{矢量内积}\upref{Dot}中讨论该如何定义垂直.
\addTODO{矢量内积词条尚未整理好以配合本词条.}



\subsection{参考书推荐}
对于几何矢量在高中数学以及几何学中的用途,张景中院士的《绕来绕去的向量法》一书的内容通俗易懂且内容丰富, 单墫的《向量与立体几何(数学奥林匹克命题人讲座)》也是一本实用的小册子. 感兴趣的读者可用这两本读物作参考. 另外,数学界也借用物理中“质心”的概念,发展出了 “质点几何学” 分支, 其本质仍然和向量几何一模一样, 只不过换了个观点看问题,实质上是仿射几何的一种优化版表述; 在上述张景中院士的书中提到了质点几何学, 莫绍揆教授也出版了专门讨论此话题的《质点几何学》,但并不建议读者特意花太多精力学习\footnote{如果你好奇的话,只需要知道向量法中的\textbf{定比分点公式}就是连接质点几何和向量几何的桥梁,就可以把几何问题在这两个观点之间互相转化了.}.
