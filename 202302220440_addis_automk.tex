% Automake 笔记

\begin{issues}
\issueDraft
\end{issues}

\pentry{Makefile 笔记\upref{Make}}

\subsection{使用}
经典的三部使用
\begin{lstlisting}[language=bash]
./configure [一些选项]
make [-j4]
make check # (可选)跑一些测试
make install
\end{lstlisting}
其中的选共享可以通过 \verb|.configure --help| 查看。 例如 \verb|--prefix=绝对安装路径|, 编译后 \verb|make install| 后头文件库文件等都会放到这个路径。 当然还会有一些包作者定义的选项, 例如如果依赖于第三方的包且没有安装在默认路径,会要求提供安装路径(即使输入相对路径,编译出来 so 的 rpath 也会是绝对路径)。

如果依赖别的包, 选项一般是 \verb|--with-包名=绝对路径|。 如果用了没用,可能还需要设置环境变量 \verb|LIBRARY_PATH|, 相当于 \verb|g++| 的 \verb|-L| 选项。 如果要跑 \verb|make check|, 最好也添加相同的路径到 \verb|LD_LIBRARY_PATH|, 否则测试时链接不上会 fail。

注意许多库直接把头文件库文件安装到标准目录而不是子文件夹。 为了避免冲突, 以及区分哪些文件是哪些库安装的, 强烈建议自定义安装路径。 另外如果没有管理员权限, 也只能安装到自定义路径。

\begin{itemize}
\item 在安装目录中, \verb|lib/pkgconfig/*.pc| 文件中会给出一些 configure 时的选项。 以及编译新程序需要的 \verb|-I| 命令和 \verb|-l| 命令。
\end{itemize}

若要 build 一个 debug 版本, 参考\href{https://stackoverflow.com/questions/4553735/gnu-autotools-debug-release-targets}{这里}的高赞。 例如 arb\upref{ArbLib} 中(注意 \verb|./config --help| 中只有 \verb|CFLAGS| 没有 \verb|CPPFLAGS|), 用
\begin{lstlisting}[language=bash]
./configure --prefix=/安装路径 \
     CFLAGS=-DDEBUG CFLAGS="-g3 -O0"
make [-j4]
make install
\end{lstlisting}
即可。 现在在编译主程序后用 gdb 调试就可以进入到 arb 的内部函数中调试。

\subsection{开发者}
\begin{itemize}
\item 一般来说建议学 Cmake\upref{CMakeN} 而不是 autotool。 Makefile 的基础还是要学的。 但如果要维护或者修改一些老项目就需要了。
\item 这里有一个\href{https://devmanual.gentoo.org/general-concepts/autotools/index.html}{教程}。
\item 参考\href{https://www.gnu.org/software/automake/manual/html_node/Autotools-Introduction.html}{官方教程}。
\item 发展历史参考\href{https://www.zhihu.com/question/22644913/answer/141475420}{知乎这篇文章}。
\begin{figure}[ht]
\centering
\includegraphics[width=14.25cm]{./figures/automk_1.png}
\caption{autotools 的结构(来自\href{https://www.zhihu.com/question/22644913/answer/141475420}{知乎})} \label{automk_fig1}
\end{figure}
\item automake 是 autotools 的一部分, 也叫做 \textbf{GNU Build System}
\item autotools 主要包含 autoconf(用于生成 \verb|configure| 脚本),automake(用于生成 \verb|Makefile.in|),libtool(用于制作 shared library)
\end{itemize}

