% 匀晶相图
\footnote{本文参考了Callister的Material Science and Engineering An Introduction,刘智恩的《材料科学基础》与Gaskell的Introduction to the Thermodynamics of Materials}

\subsection{匀晶转变}
\pentry{相图(未完成)}

\begin{figure}[ht]
\centering
\includegraphics[width=10cm]{./figures/ISOMOR_1.pdf}
\caption{典型的匀晶系合金相图,以$Ni-Cu$系统为例。本图仅作示例,未按实际比例绘制。固相区的相变已略去。数据来源:唐仁政,二元合金相图} \label{ISOMOR_fig1}
\end{figure}

\begin{figure}[ht]
\centering
\includegraphics[width=10cm]{./figures/ISOMOR_3.pdf}
\caption{$\omega_{Ni}=50\%$合金的冷却} \label{ISOMOR_fig3}
\end{figure}

%匀晶转变(\autoref{ISOMOR_fig1} 中b-d段):由单一液相直接生成单一固相 $L\rightarrow\alpha$。以下以Ni-Cu合金的平衡冷却为例,介绍匀晶转变的特点。

想象我们逐步冷却$\omega_{Ni}=50\%$的合金:
\begin{itemize}
\item 在$A$点处,系统中只有液体$L$
\item 在$B$点处(液相线以下,固相线以上),单相固体$\alpha$开始析出。$\alpha$是$Ni$与$Cu$的固溶体。此时$\alpha$相中$Ni$的浓度为$60\%$。这种从单一液相中直接形成单一固相的过程称为匀晶转变。
\item 随着温度降低,固相$\alpha$继续生成、液体$L$继续减少。固、液相的比例可由杠杆定律计算。%未完成
\item 在$C$点处,固相$\alpha$中$Ni$的浓度已经由$60\%$降低至$55\%$。可见,匀晶转变期间,固相的成分比例发生改变。
\item 在$D$点处(固相线以下),液体已经完全凝固。现在系统中不再有液相,而只有单一的固相。

\end{itemize}

总结一下匀晶转变的特点:
\begin{itemize}
\item 匀晶转变不是恒成分转变。当α相刚开始生成时,α相中高熔点组分的浓度高于液相;随着温度降低、α相不断生成,该组分的浓度逐渐降低。例如,Ni熔点高于Cu,因此新生成的相α中Ni的浓度更高。Ni、Cu经由扩散过程进出固、液相。当平衡冷却时(冷却速度足够慢),原子有充分的时间扩散以达到热力学平衡。

\begin{figure}[ht]
\centering
\includegraphics[width=7cm]{./figures/ISOMOR_2.pdf}
\caption{两相中Ni浓度的变化示意图} \label{ISOMOR_fig2}
\end{figure}

\item 匀晶转变不是恒温转变。相转变时,系统的自由度f=2-2+1=1,因此相转变时,温度可以在一定范围内变化。
\item 匀晶转变后生成单一固相,因此只有两组分(至少在一定范围内)能互溶时,匀晶转变才可能发生。
\end{itemize}

全程的相转变:$L\to\alpha$

\subsection{热力学}
\pentry{化学势(未完成),相变平衡条件\upref{PhEquv}}
%好久没摸这块了,希望公式没打错(
在匀晶转变的过程中,在固相与液相中的Cu,Ni的化学势分别相等。假定二者均满足理想条件,由化学势相同,所以得
\begin{align}
&\mu_{Ni,l,mixed}=\mu_{Ni,s,mixed}\\
&\mu_{Cu,l,mixed}=\mu_{Cu,s,mixed}\\
\end{align}
根据化学势的相关结论,对于Ni,有
$$
\mu_{Ni,l}^*+RT \ln x_{Ni,l}=\mu_{Ni,s}^*+RT \ln x_{Ni,s}=\mu_{Ni,l}^*+\Delta G_{Ni, l\rightarrow s} + RT \ln x_{Ni,s}
$$
即
$$
RT \ln x_{Ni,l}=\Delta G_{Ni, l\rightarrow s} + RT \ln x_{Ni,s}
$$
即
\begin{equation}
x_{Ni,l}=x_{Ni,s}e^{\frac{\Delta G_{Ni, l\rightarrow s}}{RT}}
\end{equation}
同理,对于Cu,有
\begin{equation}
x_{Cu,l}=x_{Cu,s}e^{\frac{\Delta G_{Cu, l\rightarrow s}}{RT}}
\end{equation}
再根据
\begin{align}
&x_{Cu,l}+x_{Ni,l}=1\\
&x_{Cu,s}+x_{Ni,s}=1\\
\end{align}
原则上可解方程。
