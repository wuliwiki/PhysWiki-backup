% 刚体的主轴
% 转动惯量|刚体|惯性张量|主轴|本征矢

\begin{issues}
\issueDraft
\end{issues}

\pentry{惯性张量\upref{ITensr}, 对称矩阵的本征问题\upref{SymEig}}

刚体转动惯量和角速度的关系为
\begin{equation}
\bvec L = \mat I \bvec \omega
\end{equation}
我们下面来讨论什么情况下 $\bvec L$ 和 $\bvec \omega$ 会共线.

我们知道惯性张量是一个 $3\times 3$ 的对称矩阵, 即 $I_{ij} = I_{ji}$. 而对称矩阵必定具有三个相互垂直的本征矢量 $\bvec \omega_i$, 满足
\begin{equation}
\bvec I \bvec \omega_i = I_i \bvec \omega_i \qquad (i = 1,2,3)
\end{equation}
注意把 $\bvec \omega_i$ 乘以任意常数仍然满足上式. 也就是说, 如果刚体绕着三个主轴旋转, 那么它的转动惯量与角速度共线.

\begin{example}{长方体的主轴}
在\autoref{ITensr_ex1}~\upref{ITensr}中, 我们知道长方体的惯性张量为
\begin{equation}
\mat I = \frac{1}{12} M
\pmat{
   b^2 + c^2 & 0 & 0\\
   0 & a^2 + c^2 & 0\\
   0 & 0 & a^2 + b^2
}
\end{equation}
求本征方程就相当于把矩阵对角化, 但这已经是一个对角化矩阵, 所以本征矢就是三个单位矢量 $(1,0,0)$, $(0,1,0)$ 和 $(0,0,1)$, 即三个主轴分别沿长方体的三条边, 主轴对应的转动惯量分别为
\begin{equation}
\begin{aligned}
&I_1 = \frac{1}{12}M(b^2+c^2)\\
&I_2 = \frac{1}{12}M(a^2+c^2)\\
&I_3 = \frac{1}{12}M(a^2+b^2)
\end{aligned}
\end{equation}
\end{example}
