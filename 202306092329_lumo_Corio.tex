% 科里奥利力
% 惯性系|惯性力|非惯性系|旋转参考系|离心力|科里奥利力

\pentry{离心力\upref{Centri}}

\textbf{科里奥利力(Coriolis Force)}是匀速旋转的参考系中由质点运动产生的惯性力。
\begin{equation}
\bvec F_c = 2m \bvec v_{S'} \cross \bvec \omega~.
\end{equation}
其中 $\bvec v_{S'}$ 是质点相对于旋转参考系 $S'$ 的瞬时速度, $\bvec\omega$ 是旋转系相对于某惯性系 $S$ 转动的角速度矢量\upref{CMVD}。
式中的乘法是叉乘\upref{Cross}。
在匀速转动参考系(属于非惯性系)中,若质点保持相对静止,则惯性力只有离心力。然而当质点与转动参考系有相对速度时,惯性力中还会增加一个与速度垂直的力,这就是科里奥利力。地理中的地转偏向力就是科里奥利力,可用上式计算(见“地球表面的科里奥利力\upref{ErthCf}”)。

由叉乘的定义可得, 科里奥利力与速度矢量始终保持垂直, 所以科里奥利力不会对质点做功。

\subsection{推导}
\pentry{连续叉乘的化简\upref{TriCro}, 圆周运动的速度\upref{CMVD}, 加速度的坐标变换\upref{AccTra}}
我们可以直接根据惯性力的定义(\autoref{eq_Iner_1}~\upref{Iner}) 和加速度的坐标变换(\autoref{eq_AccTra_4}~\upref{AccTra}) 得到任意非惯性系 $S'$ 中质点的总惯性力($S$ 为任意惯性系) 为
\begin{equation}
\bvec F_c = m( \bvec a_{S'} - \bvec a_{S} ) = -m\bvec a_{r} + 2 m \bvec v_{S'} \cross  \bvec \omega~.
\end{equation}
其中第一项包含平移惯性力和转动惯性力, 转动惯性力又可划分为离心力以及角加速度产生的惯性力(见\autoref{eq_Corio_1}), 但与质点相对 $S'$ 的速度无关, 所以只将科里奥利力定义为第二项。

\subsection{另一种推导}\label{sub_Corio_1}
\addTODO{移动到 “加速度的参考系变换\upref{AccTra}” 中}
\footnote{参考 \cite{Goldstein} 相关章节。}类比\autoref{eq_Vtran2_2}~\upref{Vtran2}, 若 $S'$ 系与 $S$ 系原点始终重合, 且相对 $S'$ 相对 $S$ 系以角速度 $\bvec\omega$ 旋转, 对任意一个随时间变化的矢量(假设一阶导数存在), 我们把它在 $S$ 和 $S'$ 系中的时间导数分别记为 $(\dot{\bvec A})_{S}$ 和 $(\dot{\bvec A})_{S'}$, 则有
\begin{equation}\label{eq_Corio_4}
(\dot{\bvec A})_{S} = (\dot{\bvec A})_{S'} + \bvec\omega\cross\bvec A~.
\end{equation}
最后一项参考\autoref{eq_CMVD_5}~\upref{CMVD}。 注意该式中的矢量为几何矢量\upref{GVec} 而不是坐标列矢量, 若要将该式记为坐标形式, 应该使用同一坐标系\upref{Vtrans}。

我们先将 $\bvec A$ 替换为质点的位矢 $\bvec r$, 得参考系中质点的速度关系为(即\autoref{eq_Vtrans_1}~\upref{Vtrans})
\begin{equation}\label{eq_Corio_5}
\bvec v_{S} = \bvec v_{S'} + \bvec\omega\cross\bvec r~.
\end{equation}
两边在 $S$ 系中对时间求导得
\begin{equation}\label{eq_Corio_6}
\bvec a_{S} = (\dot{\bvec v}_{S'})_{S} + \bvec\omega\cross\bvec v_{S} + \dot{\bvec\omega} \cross\bvec r~.
\end{equation}
注意 $S'$ 系中的加速度 $\bvec a_{S'}$ 并不是 $(\dot{\bvec v}_{S'})_{S}$, 而是 $(\dot{\bvec v}_{S'})_{S'}$。 令\autoref{eq_Corio_4} 中的 $\bvec A = \bvec v_{S'}$, 得
\begin{equation}\label{eq_Corio_7}
(\dot{\bvec v}_{S'})_{S} = \bvec a_{S'} + \bvec\omega\cross\bvec v_{S'}~.
\end{equation}
将\autoref{eq_Corio_5} 和\autoref{eq_Corio_7} 代入\autoref{eq_Corio_6}, 得(参考 “连续叉乘的化简\upref{TriCro}”)
\begin{equation}
\bvec a_{S} = \bvec a_{S'} + 2\bvec\omega\cross\bvec v_{S'} + \bvec\omega\cross(\bvec\omega\cross\bvec r) + \dot{\bvec\omega} \cross\bvec r~.
\end{equation}
所以旋转参考系中的总惯性力(\autoref{eq_Iner_1}~\upref{Iner})为
\begin{equation}\label{eq_Corio_1}
\bvec f = m(\bvec a_{S'} - \bvec a_{S}) = 2m\bvec v_{S'}\cross \bvec\omega  -m\bvec\omega\cross(\bvec\omega\cross\bvec r) - m \dot{\bvec\omega} \cross\bvec r~.
\end{equation}
其中第一项被称为科里奥利力(唯一一项与 $\bvec v_{S'}$ 有关的), 第二项为离心力(\autoref{eq_Centri_5}~\upref{Centri}), 第三项为角加速度产生的惯性力。
