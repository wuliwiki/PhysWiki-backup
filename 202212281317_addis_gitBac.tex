% 用 git 和类似工具备份文件夹

\begin{issues}
\issueDraft
\end{issues}

\pentry{Git 笔记\upref{Git}}

以下 bash 程序可以把若干个文件夹中的每个都用 git 备份到 \verb|备份目录|。 运行完后, 每个 \verb|文件夹*| 会保存为 \verb|备份目录/文件夹*.git|。 这个其实就是 git 仓库中的 \verb|.git| 文件夹(注意并不是 bare repo, 不可以用于上游)。

要注意的是, Git (尤其是 Windows 上)处理大的二进制文件的速度较慢。 笔者在 Windows 上在两个 HDD 硬盘之间用该脚本备份, 写入速度平均 12MiB/s。 但笔者认为 Git 的丰富功能和灵活性、以及广泛的普及可以弥补这一不足。 另见 bup\upref{bupBac}(待研究)、git-annex\upref{gitanx}(最新版本依赖于 symlink) 以及 git-lfs(没有本地repo)。 更多讨论见文末。

\begin{lstlisting}[language=bash, git-backup.sh]
repos="文件夹1 文件夹2 ..."
dir="备份目录"

for repo in $repos
do
  printf "\n\n\n===============================\n"
  echo "$repo"
  printf "===============================\n\n\n"
  if ! [ -d "$repo" ]; then
    echo "error: directory not found!"
    continue
  fi
  gitdir="${dir}${repo}.git"
  dirflag="--work-tree=$repo --git-dir=$gitdir"
  echo "mkdir $gitdir ..."
  mkdir -p "$gitdir" &&
  echo "git init..." &&
  git $dirflag init &&
  echo "git add -A ..." &&
  git $dirflag add -A --verbose &&
  echo "git commit..."
  # 其他有用的命令:
  # git $dirflag commit -m 'update'
  # git $dirflag status
done
\end{lstlisting}

\subsection{其他工具}
事实上, 如果要备份一些不常改动的大文件(例如视频、电影), 要防止 bitrot, 只需要生成一个哈希表: \verb`find . -type f -exec sha1sum {} \; | sort > sha1sum.txt`。 然后把文件夹复制粘贴到备份文件夹即可。 要更新, 可以用 rsync(同样具有对比 hash 的功能)也可以自己写一个 bash 脚本或者简单的程序做这件事情。 要检查 bitrot, 就再运行一次输出到 \verb|sha1sum2.txt|, 然后用 \verb|git diff --no-index| 对比两个文件即可。

最后, 至于网络备份, 国内的主流网盘基本都支持秒传功能。 所谓秒传就是通过检测文件 hash 和大小等来避免重复上传任何人已经传过的文件。 所以如果同样的文件夹用网盘第二次备份, 那么所有文件都将秒传。 一些网盘(如某度)还支持删除重复文件的功能, 这样就完美了。 不过为了防和谐, 
