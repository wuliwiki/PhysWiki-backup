% 伽利略·伽利莱(综述)
% license CCBYSA3
% type Wiki

本文根据 CC-BY-SA 协议转载翻译自维基百科\href{https://en.wikipedia.org/wiki/Galileo_Galilei}{相关文章}。

\begin{figure}[ht]
\centering
\includegraphics[width=8cm]{./figures/f16d6423773a5687.png}
\caption{约1640年的肖像画} \label{fig_JLL_1}
\end{figure}
伽利略·迪·文琴佐·博纳尤蒂·德·伽利略(\textbf{Galileo di Vincenzo Bonaiuti de' Galilei},1564年2月15日-1642年1月8日),通常简称为伽利略·伽利莱(\textbf{Galileo Galilei},/ˌɡælɪˈleɪoʊ ˌɡælɪˈleɪ/,美国英语中也读作 /ˌɡælɪˈliːoʊ -/;意大利语:[ɡaliˈlɛːo ɡaliˈlɛːi]),或单名称为伽利略(\textbf{Galileo}),是一位佛罗伦萨的天文学家、物理学家和工程师,有时被描述为“博学多才之人”。他出生于当时属于佛罗伦萨公国的比萨市。伽利略被誉为观测天文学之父、现代经典物理学之父、科学方法之父以及现代科学之父。

伽利略研究了速度与加速度、重力与自由落体、相对性原理、惯性、抛体运动等,并在应用科学和技术领域开展了工作,描述了摆的特性和“静水天平”。他是文艺复兴早期温度计(即热测量仪)的开发者之一,还发明了多种军用罗盘。通过他改进的望远镜,伽利略观测到了银河的恒星、金星的位相、木星的四大卫星、土星的光环、月球的陨石坑以及太阳黑子。他还制作了一种早期显微镜。

伽利略对哥白尼日心说的支持遭到了天主教会内部和一些天文学家的反对。1615年,罗马宗教裁判所对这一问题进行了调查,得出结论认为他的观点与当时普遍接受的《圣经》解释相矛盾。[9][10][11]

后来,伽利略在《两大世界体系的对话》(1632年)中为自己的观点辩护,但书中似乎对教皇乌尔班八世进行了攻击和嘲讽,这使伽利略疏远了教皇及之前一直大力支持他的耶稣会士。[9] 他因此受到宗教裁判所的审判,被认定为“严重可疑的异端”,并被迫公开认错。伽利略此后被软禁在家中度过余生。[12][13] 在此期间,他撰写了《两种新科学》(1638年),主要探讨了运动学和材料强度问题。[14]
\subsection{早年生活与家庭}  
伽利略于1564年2月15日出生在比萨(当时属于佛罗伦萨公国),是文琴佐·伽利莱和朱莉亚·阿曼纳蒂的长子。他的父亲文琴佐是著名的鲁特琴演奏家、作曲家和音乐理论家,他的母亲朱莉亚是当地一位显赫商人的女儿。两人于1562年结婚,当时文琴佐42岁,而朱莉亚24岁。伽利略自己也成为了一名出色的鲁特琴演奏家,并可能从父亲那里早早学会了对权威的怀疑态度。[15][16]  

伽利略的五个兄弟姐妹中,有三个在婴儿期幸存下来。他最小的弟弟米开朗基罗(或称米开拉尼奥洛)也成为了一名鲁特琴演奏家和作曲家,但这一生都给伽利略带来了经济负担。[17] 米开朗基罗未能履行父亲承诺的嫁妆分担责任,这导致伽利略的妹夫们试图通过法律途径追讨欠款。此外,米开朗基罗有时还需要向伽利略借钱,以支持他的音乐事业和旅行。这些经济压力可能促使伽利略早年就产生了发明一些能够带来额外收入的装置的想法。[18]  

伽利略八岁时,家人搬到了佛罗伦萨,但他被留在比萨,由穆齐奥·泰达尔迪照顾了两年。十岁时,他离开比萨,与家人团聚在佛罗伦萨,并开始接受雅科波·博尔吉尼的指导。[15] 从1575年至1578年,他在佛罗伦萨东南约30公里的瓦隆布罗萨修道院接受教育,特别是在逻辑方面的学习。[19][20]
\subsubsection{名字}  
伽利略通常只用他的名字来称呼自己。在当时的意大利,姓氏并非必需,而他的名字“伽利略”(Galileo)与他有时使用的家族姓氏“伽利莱”(Galilei)源于同一祖先。无论是他的名字还是姓氏,最终都可追溯到他的祖先伽利略·博纳尤蒂(Galileo Bonaiuti),一位15世纪佛罗伦萨的重要医生、教授和政治家。[21] 伽利略·博纳尤蒂被安葬在佛罗伦萨的圣十字大教堂,这也是大约200年后伽利略·伽利莱的安葬地。[22]  

当伽利略使用多于一个名字时,他有时称自己为“伽利略·伽利莱·林切奥”(Galileo Galilei Linceo),以表示他是林切伊学院(Accademia dei Lincei)的一员。该学院是教皇国成立的一所精英科学组织。在16世纪中期的托斯卡纳,长子通常以父母的姓氏命名为名字。[23] 因此,伽利略·伽利莱的名字未必是专门为了纪念他的祖先伽利略·博纳尤蒂。意大利男性名“伽利略”(Galileo)及其派生的姓氏“伽利莱”(Galilei)来源于拉丁语“Galilaeus”,意为“加利利人”。[24][21]  

伽利略名字和姓氏的圣经渊源后来成为一个著名双关语的主题。[25] 1614年,在伽利略事件期间,伽利略的一位反对者、多米尼加会士托马索·卡奇尼(Tommaso Caccini)发表了一篇颇具争议且影响深远的布道词。他在布道中引用《使徒行传》1:11说道:“加利利人哪,你们为什么站着望天呢?”(可能含有对伽利略的讽刺)。[citation needed]  
\subsubsection{子女}  
\begin{figure}[ht]
\centering
\includegraphics[width=6cm]{./figures/79d02887461111a7.png}
\caption{据认为是伽利略长女维尔吉尼亚的肖像,她对父亲特别忠诚。} \label{fig_JLL_2}
\end{figure}
尽管伽利略是一位虔诚的天主教徒,[26] 他与玛丽娜·甘巴(Marina Gamba)未婚生育了三个孩子。他们育有两个女儿:维尔吉尼亚(Virginia,生于1600年)和利维娅(Livia,生于1601年),以及一个儿子:文琴佐(Vincenzo,生于1606年)。[27]  

由于孩子们是非婚生的,伽利略认为两个女儿难以嫁人,且无法承担高昂的经济支持或嫁妆费用。这些费用可能会使伽利略重蹈帮助两位姐妹嫁人的经济困境。[28] 因此,她们唯一合适的选择就是修道生活。两位女儿都被阿尔切特里的圣马太修道院接纳,并在此度过了余生。[29]  

维尔吉尼亚进入修道院后改名为玛利亚·切莱斯特(Maria Celeste)。她于1634年4月2日去世,并与伽利略合葬在佛罗伦萨的圣十字大教堂。利维娅进入修道院后改名为阿尔坎杰拉修女(Sister Arcangela),大部分时间身体都不好。伽利略的儿子文琴佐后来被确认为合法继承人,并与塞斯蒂莉亚·博基内里(Sestilia Bocchineri)结婚。[30]  
\subsection{职业生涯与首次科学贡献}
伽利略年轻时曾认真考虑成为一名神职人员,但在父亲的敦促下,他于1580年进入比萨大学学习医学。[31] 他受到了佛罗伦萨的吉罗拉莫·博罗(Girolamo Borro)和弗朗切斯科·博纳米奇(Francesco Buonamici)讲座的影响。[20] 1581年,在学习医学时,他注意到一个吊灯在空气流动的作用下以大小不同的弧度摆动。他发现,通过与自己的心跳比较,无论吊灯摆动的幅度多大,摆动一次所需的时间似乎相同。回到家后,他用两根等长的摆进行了实验,一个摆幅较大,另一个摆幅较小,发现它们的摆动时间相同。直到近一百年后,克里斯蒂安·惠更斯(Christiaan Huygens)的研究才将摆的等时性用于制造精准的计时器。[32]  

在此之前,伽利略被刻意远离数学,因为医生的收入高于数学家。然而,在偶然听了一场几何学讲座后,他说服了不情愿的父亲,让他改学数学和自然哲学,而不是医学。[32] 他发明了一种热测量仪(thermoscope),这是温度计的前身。1586年,他出版了一本小册子,介绍他设计的静水天平,这一发明使他首次引起学术界的关注。此外,伽利略还学习了绘画艺术(\textbf{disegno}),1588年在佛罗伦萨的美术学院(\textbf{Accademia delle Arti del Disegno})担任透视法与明暗法的讲师。同年,他应佛罗伦萨学院邀请发表了两场讲座,题为《但丁地狱的形状、位置与大小》,试图提出关于但丁地狱的严格宇宙模型。[33] 受城市艺术传统和文艺复兴艺术家作品的启发,伽利略形成了美学思维。在美术学院任教期间,他与佛罗伦萨画家奇戈利(Cigoli)建立了终生的友谊。[34][35]  

1589年,伽利略被任命为比萨大学的数学教授。1591年,他的父亲去世,他承担起照顾弟弟米开朗基罗的责任。1592年,他转到帕多瓦大学,教授几何学、力学和天文学,直至1610年。[36] 在此期间,伽利略在基础科学(如运动学和天文学)以及应用科学(如材料强度和望远镜的开创性研究)方面取得了重要发现。他的兴趣广泛,还研究了占星术,这在当时是一门与数学、天文学和医学紧密相关的学科。[37][38]
\subsubsection{天文学} 
\textbf{开普勒超新星}  

第谷·布拉赫(Tycho Brahe)及其他人曾观测过1572年的超新星。1605年1月15日,奥塔维奥·布伦佐尼(Ottavio Brenzoni)写信给伽利略,提到了1572年的超新星以及1601年较暗的新星。伽利略在1604年观测并讨论了开普勒超新星。由于这些新星没有显示出可检测的日视差,伽利略得出结论认为它们是遥远的恒星,因此推翻了亚里士多德关于天穹不变性的观点。[39]

\textbf{折射望远镜}
\begin{figure}[ht]
\centering
\includegraphics[width=8cm]{./figures/534f6baa94d041ab.png}
\caption{伽利略的“cannocchiali”望远镜,展于佛罗伦萨伽利略博物馆} \label{fig_JLL_3}
\end{figure}
或许仅仅基于对荷兰人汉斯·李普希(Hans Lippershey)在1608年尝试申请专利的首个实用望远镜的描述,[40] 伽利略在次年制作了一台放大倍率约为3倍的望远镜。他随后改进了设计,制造出放大倍率高达约30倍的版本。[41] 使用伽利略望远镜,观察者可以看到放大且正立的地面影像——这就是通常所说的地面望远镜或单筒望远镜。他还用它来观测天空;一段时间内,他是少数能够制作适合天文观测望远镜的人之一。1609年8月25日,他向威尼斯人展示了一台早期望远镜,放大倍率约为8倍或9倍。  

伽利略的望远镜也成为他的一个盈利副业,他将望远镜卖给商人,后者发现它在海上导航和贸易中都非常有用。伽利略于1610年3月出版了一本简短的论文,题为《星际信使》(**Sidereus Nuncius**),记录了他最初的天文望远镜观测成果。[42]

\textbf{月球}
\begin{figure}[ht]
\centering
\includegraphics[width=6cm]{./figures/7cfb4912b973d48d.png}
\caption{《星际信使》中月球的插图,1610年在威尼斯出版} \label{fig_JLL_4}
\end{figure}
1609年11月30日,伽利略将他的望远镜对准了月球。[43] 虽然他并非第一个通过望远镜观察月球的人(英国数学家托马斯·哈里奥特(Thomas Harriot)早在四个月前就进行了观察,但他仅记录到“奇怪的斑点”),[44] 但伽利略是第一个推测月球表面不规则阴影是由月球上的山脉和陨石坑遮挡光线所致的人。在他的研究中,他还绘制了月球地形图,并估算了山脉的高度。月球并非如亚里士多德所宣称的那样是一颗透明而完美的球体,也不像但丁所描述的那样是“第一颗行星”,一颗“升入天穹的永恒明珠”。  

伽利略有时被认为在1632年发现了月球纬度上的天平动,[45] 尽管托马斯·哈里奥特或威廉·吉尔伯特(William Gilbert)可能早已做出这一发现。[46]  

伽利略的朋友、画家奇戈利(Cigoli)在他的某幅画作中包含了一幅逼真的月球描绘,这可能是基于他通过自己望远镜的观察所得。[34]

\textbf{木星的卫星} 

1610年1月7日,伽利略通过望远镜观测到木星附近有三个他当时描述为“极其微小、完全不可见的小恒星”,它们位于木星附近,并与木星在一条直线上。[47] 随后几晚的观测显示,这些“恒星”相对于木星的位置在不断变化,这种现象无法用它们是真正固定的恒星来解释。1月10日,伽利略注意到其中一个“恒星”消失了,他推测这是因为它被木星遮挡了。在接下来的几天里,他得出结论,这些天体正在围绕木星运行:他发现了木星四大卫星中的三个。[48]  

伽利略在1月13日发现了第四颗卫星。他将这四颗卫星命名为“美第奇之星”(Medicean stars),以此向未来的资助人托斯卡纳大公科西莫二世·德·美第奇(Cosimo II de' Medici)及其三位兄弟致敬。[49] 不过,后来天文学家为了纪念伽利略,将这些卫星改名为“伽利略卫星”。这些卫星实际上也在1610年1月8日被西蒙·马里乌斯(Simon Marius)独立发现,并在1614年出版的《木星的世界》(\textbf{Mundus Iovialis})中被马里乌斯命名为\textbf{木卫一(Io)}、\textbf{木卫二(Europa)}、\textbf{木卫三(Ganymede)} 和 \textbf{木卫四(Callisto)}。[50]
\begin{figure}[ht]
\centering
\includegraphics[width=8cm]{./figures/9e52a2a199660cbe.png}
\caption{1684年绘制的法国地图,展示了早期地图的轮廓(浅色线条)与使用木星卫星作为精确时间参考进行的新测量结果(较粗线条)之间的对比。} \label{fig_JLL_5}
\end{figure}
伽利略对木星卫星的观测在天文学界引发了争议:一颗行星有更小的天体围绕其运行,这与亚里士多德宇宙学的原则不符,后者认为所有天体都应围绕地球运转。[51][52] 起初,许多天文学家和哲学家拒绝相信伽利略能够发现这样的现象。[53][54] 更加复杂的是,其他天文学家很难验证伽利略的观测结果。当伽利略在博洛尼亚展示望远镜时,与会者难以看到木星的卫星。其中一人,马丁·霍尔基(Martin Horky),注意到通过望远镜观察某些固定恒星(如室女座的角宿一)时,它们会显得是双星。他认为这证明了望远镜在观察天体时具有误导性,从而对木星卫星的存在表示怀疑。[55][56]  

然而,罗马的克里斯托弗·克拉维乌斯(Christopher Clavius)天文台确认了伽利略的观测结果。尽管对如何解释这些现象仍存疑问,但当伽利略次年访问罗马时,他受到了英雄般的欢迎。[57] 在接下来的18个月里,伽利略持续观测这些卫星,到1611年年中,他得出了这些卫星轨道周期的极其准确的估计值——这是约翰内斯·开普勒(Johannes Kepler)曾认为不可能完成的壮举。[58][59]  

伽利略还看到了其发现的实际用途。在海上确定船只的东西位置需要船上的时钟与本初子午线的时钟同步。解决这个经度问题对于航行安全至关重要,因此西班牙和后来荷兰都设立了巨额奖励来鼓励这一问题的解决。由于伽利略发现的卫星的掩食现象相对频繁且时间可以被非常准确地预测,它们可以用来校准船上的时钟。伽利略申请了这些奖励。然而,从船上观测这些卫星过于困难,但这种方法被用于陆地测量,包括重新绘制法国的地图。[60]: 15–16 [61]

\textbf{金星的相位}
\begin{figure}[ht]
\centering
\includegraphics[width=14.25cm]{./figures/5532f8a9bd158eb0.png}
\caption{1610年,伽利略·伽利莱通过望远镜观察到金星展现出相位变化,尽管它在地球的天空中始终靠近太阳(第一张图)。这一发现证明了金星围绕太阳运行,而不是围绕地球运行,这与哥白尼的日心模型的预测相符,并否定了当时普遍接受的地心模型(第二张图)。} \label{fig_JLL_6}
\end{figure}
从1610年9月起,伽利略观察到金星呈现出一整套类似月球的相位变化。尼古拉·哥白尼(Nicolaus Copernicus)提出的日心模型预测,金星应该会展现所有相位,因为金星绕太阳的轨道会使其被照亮的半球在金星位于太阳的对侧时面对地球,而在金星位于太阳的地球侧时背向地球。在托勒密的地心模型中,任何行星的轨道都不可能与承载太阳的球壳相交。因此,传统上金星的轨道被完全置于太阳的近侧,在这种情况下金星只能呈现新月和朔相。也可以将其轨道完全置于太阳的远侧,这样金星只能呈现盈凸相和满相。

伽利略通过望远镜观察到金星的弦月相、盈凸相和满相后,托勒密的地心模型变得无法成立。在17世纪早期,由于这一发现,大多数天文学家转而支持某种形式的地-日心混合行星模型,[62][63] 如第谷模型(Tychonic model)、卡佩拉模型(Capellan model)和扩展卡佩拉模型(Extended Capellan model),这些模型有的包含每日自转的地球,有的则没有。这些模型解释了金星的相位变化,同时避免了完全日心模型关于恒星视差的“驳斥”。因此,伽利略对金星相位的发现是从完全地心模型向通过地-日心混合模型过渡到完全日心模型的两阶段转变中最具实证意义的贡献之一。[citation needed]

\textbf{土星与海王星}  

1610年,伽利略观察到土星,他最初误将土星的光环视为行星,[64] 认为它是一个由三个天体组成的系统。当他后来再次观测时,发现土星的光环正对地球,导致他以为两个“天体”消失了。1616年,他再次观察到光环的重新出现,这使他更加困惑。[65]  

1612年,伽利略观察到海王星。在他的笔记中,这颗行星被记录为许多微弱、不引人注目的恒星之一。他没有意识到这是一颗行星,但注意到它相对于背景恒星的运动,然后失去了对它的进一步跟踪。[66]  

\textbf{太阳黑子}  

伽利略通过肉眼和望远镜研究了太阳黑子。[67] 它们的存在再次挑战了正统亚里士多德天体物理学中“天体不变且完美”的理论。1612年至1613年,弗朗切斯科·西兹(Francesco Sizzi)及其他人观察到太阳黑子的轨迹出现明显的年度变化,[68] 这一现象成为反对托勒密体系和第谷地日混合体系的强有力证据。[c] 关于太阳黑子的发现优先权以及其解释的争议,使伽利略与耶稣会士克里斯托夫·谢纳(Christoph Scheiner)之间展开了一场长期且激烈的争论。谢纳曾将他的发现告知马克·维尔瑟(Mark Welser),后者向伽利略询问他的意见。两人都未意识到约翰内斯·法布里修斯(Johannes Fabricius)早期的观察和关于太阳黑子的发表。[72]  

\textbf{银河系与恒星}  
伽利略观察到银河系,发现其并非人们之前认为的模糊云气,而是由无数密集的恒星组成,使其从地球上看似为云状。他还定位了许多肉眼无法看到的遥远恒星,并在1617年观察了大熊座中的双星——北斗七星的 Mizar。[73]  

在《星际信使》中,伽利略报告说恒星在望远镜中仅显示为光点,外观几乎没有改变,而行星则在望远镜中呈现为圆盘。但随后在《论太阳黑子的书信》中,他报告说望远镜揭示了恒星和行星的形状都“非常圆”。从那时起,他继续报告望远镜显示了恒星的圆形,并测量了恒星的视直径,大约为几角秒。[74][75] 他还设计了一种不用望远镜测量恒星视直径的方法。在《两大世界体系的对话》中,他描述了这一方法:用一根细绳挡在恒星的视线中,测量能够完全遮挡恒星的最大距离。通过测量这个距离和绳子的宽度,他可以计算恒星在观测点的夹角。[76][77][78]  

在《对话》中,伽利略报告说他测得一等星的视直径不超过5角秒,而六等星的视直径约为5/6角秒。和他同时代的大多数天文学家一样,伽利略没有意识到他测量的恒星视直径是虚假的,主要由衍射和大气扰动引起,并不代表恒星的真实大小。然而,伽利略的数值比第谷等人此前对最亮恒星视直径的估计要小得多,这使他能够反驳反对哥白尼理论的论点,例如第谷提出的“恒星必须极其巨大以至于其周年视差不可探测”的说法。[79][80][81] 西蒙·马里乌斯(Simon Marius)、乔瓦尼·巴蒂斯塔·里乔利(Giovanni Battista Riccioli)和马丁努斯·霍滕修斯(Martinus Hortensius)等天文学家也对恒星进行了类似测量,马里乌斯和里乔利得出结论,恒星的视直径虽然较小,但仍不足以完全回答第谷的论点。[82][83]
\subsubsection{潮汐理论}
\begin{figure}[ht]
\centering
\includegraphics[width=6cm]{./figures/72e99e7fa5604283.png}
\caption{伽利略·伽利莱,由弗朗切斯科·波尔恰绘制的肖像} \label{fig_JLL_7}
\end{figure}
\textbf{红衣主教贝拉明于1615年写道,哥白尼体系无法成立,除非有“真正的物理证明表明太阳不是绕地球运转,而是地球绕太阳运转”。}[84] 伽利略认为,他关于潮汐的理论能够提供这种证据。[85] 这一理论对他而言如此重要,以至于他最初打算将《两大世界体系的对话》命名为《海潮涨落的对话》。[86] 不过,宗教裁判所下令将标题中的潮汐参考移除。[citation needed]

对于伽利略而言,潮汐是由于地球表面某一点随着地球的自转和绕太阳的公转而加速或减速时,海水在海洋中来回晃动所引起的。他在1616年首次向红衣主教奥尔西尼(Cardinal Orsini)提交了他的潮汐理论。[87] 他的理论首次揭示了海洋盆地形状对潮汐大小和时间的重要性。例如,他正确解释了为什么亚得里亚海中部的潮汐几乎可以忽略不计,而两端的潮汐却更为显著。然而,作为关于潮汐成因的一般理论,他的观点是失败的。[citation needed]

如果他的理论是正确的,每天只会有一次涨潮。然而,伽利略和他的同时代人知道这一点不足,因为威尼斯每天有两次涨潮,间隔约12小时。伽利略将这一异常归因于几个次要原因,包括海洋的形状、深度以及其他因素。[88][89] 阿尔伯特·爱因斯坦后来表示,伽利略发展了他的“引人入胜的论点”,并因对地球运动的物理证明的渴望而不加批判地接受了它们。[90]  

伽利略还否定了从古代以及其同时代的约翰内斯·开普勒提出的月球引力导致潮汐的观点。[91] 他同样对开普勒提出的行星椭圆轨道理论不感兴趣。[92][93] 伽利略继续支持他的潮汐理论,并认为这是地球运动的最终证明。[94]
\subsubsection{关于彗星的争论与《观察者》} 
1619年,伽利略与耶稣会罗马学院(Collegio Romano)数学教授奥拉齐奥·格拉西(Father Orazio Grassi)神父陷入了一场争论。这场争论最初围绕彗星的本质展开,但到伽利略在1623年发表《观察者》(\textbf{Il Saggiatore})——这一争论的最后一击时,已演变为一场关于科学本质的更广泛争议。书籍的标题页称伽利略为哲学家和托斯卡纳大公的“首席数学家”(\textbf{Matematico Primario})。[95]  

由于《观察者》包含了大量伽利略关于科学实践的观点,这本书被视为他的科学宣言。[96][97] 1619年初,格拉西神父匿名发表了一本小册子,标题为《1618年三颗彗星的天文争议》(\textbf{An Astronomical Disputation on the Three Comets of the Year 1618}),[98] 讨论了前一年11月末出现的一颗彗星的本质。格拉西认为,这颗彗星是一个炽热的天体,以恒定的距离沿大圆的一段轨道运动,[99][100] 由于它在天空中的移动速度比月亮慢,因此它的距离必定比月亮更远。[citation needed]  

格拉西的观点和结论在一篇名为《论彗星的对话》(\textbf{Discourse on Comets})的文章中受到了批评。[101] 该文章以伽利略的弟子、佛罗伦萨律师马里奥·圭杜奇(Mario Guiducci)的名义发表,但主要由伽利略撰写。[102] 伽利略和圭杜奇并未对彗星的本质提出明确的理论,[103][104] 他们提出的一些假说后来被证明是错误的。(当时,第谷·布拉赫已经提出了对彗星研究的正确方法。)在文章的开篇,伽利略和圭杜奇对耶稣会士克里斯托夫·谢纳(Christoph Scheiner)进行了毫无必要的侮辱,[105][106][107] 并在文中多次贬低罗马学院的教授们。[105] 耶稣会士对此感到冒犯,[105][104] 格拉西很快以“洛塔里奥·萨西奥·西根萨诺”(Lothario Sarsio Sigensano)的笔名发表了一篇回应文章《天文学与哲学的天平》(\textbf{The Astronomical and Philosophical Balance}),[108][109] 假称这是他的一位学生所作。[citation needed]  

《观察者》是伽利略对《天文学的天平》的强烈回应。[101] 它被广泛认为是争论文学的杰作,[110][111] 在书中,“萨西”的论点被毫不留情地嘲讽和批评。[112] 这本书广受赞誉,特别令新任教皇乌尔班八世感到满意,因为伽利略将其献给了他。[113] 在过去十年中,巴贝里尼(即后来的乌尔班八世)曾在罗马支持伽利略及其所在的林切伊学院(Lincean Academy)。[114]  

伽利略与格拉西的争论永久地疏远了许多耶稣会士,[115] 伽利略和他的朋友们确信是耶稣会士促成了他后来受到的谴责,[116] 尽管对此的确凿证据并不充分。[117][118]
\subsubsection{关于日心说的争议}
\begin{figure}[ht]
\centering
\includegraphics[width=6cm]{./figures/55ff24265526737a.png}
\caption{克里斯蒂亚诺·班蒂1857年的画作《伽利略面对罗马宗教裁判所》} \label{fig_JLL_8}
\end{figure}
在伽利略与教会发生冲突的时期,多数受过教育的人都支持亚里士多德的地心说,认为地球是宇宙的中心和所有天体的轨道中心,或者支持第谷·布拉赫的混合体系,该体系将地心说和日心说结合起来。[119][120] 对日心说以及伽利略相关著作的反对意见结合了宗教和科学的观点。宗教反对主要来自《圣经》中暗示地球静止不动的经文。[d] 科学反对则来自第谷,他认为如果日心说是正确的,那么应该可以观察到周年恒星视差,但当时并未发现这种现象。[e] 阿里斯塔克斯和哥白尼曾正确地推测,由于恒星距离遥远,视差可以忽略不计。然而,第谷反驳道,既然恒星看起来具有可测量的角大小,那么如果它们如此遥远,它们的大小必须远远超过太阳,甚至大于地球轨道。[123] 直到后来,天文学家才意识到恒星的视亮度是由一种光学现象——爱里斑(Airy disk)——引起的,恒星的视亮度与其亮度有关,而非其真实物理大小(参见\textbf{Magnitude#History})。[123]  

伽利略根据1609年的天文观测为日心说辩护。1613年12月,佛罗伦萨的大公夫人克里斯蒂娜(Christina)以《圣经》为依据,对伽利略的朋友和追随者贝内代托·卡斯特利(Benedetto Castelli)提出了关于地球运动的反对意见。[f] 受此事件的启发,伽利略写了一封信给卡斯特利,论证日心说实际上并不违反《圣经》文本,并指出《圣经》的权威在于信仰和道德,而非科学。这封信未被发表,但广泛传播。[124] 两年后,伽利略写了一封给克里斯蒂娜的信,将他先前用八页阐述的论点扩展至四十页。[125]  

到1615年,伽利略关于日心说的著作被尼科洛·洛里尼神父(Father Niccolò Lorini)提交给了罗马宗教裁判所,洛里尼声称伽利略及其追随者试图重新解释《圣经》,[d] 这被视为违反特伦特会议的行为,并被认为带有危险的基督教新教倾向。[126] 洛里尼特别引用了伽利略写给卡斯特利的信。[127] 伽利略前往罗马为自己及其观点辩护。1616年初,弗朗切斯科·因戈利(Francesco Ingoli)发起了与伽利略的争论,向伽利略递交了一篇驳斥哥白尼体系的论文。伽利略后来表示,他认为这篇论文在随后的反日心说行动中起了重要作用。[128] 因戈利可能受宗教裁判所委托撰写关于此争议的专家意见,其论文成为宗教裁判所采取行动的依据。[129]  

论文集中提出了18个针对日心说的物理和数学论据,主要借鉴了第谷·布拉赫的观点,尤其是关于日心说需要恒星远大于太阳的假设。[g] 此外,该论文还包括四个神学论据,但因戈利建议伽利略关注物理和数学论据,他并未提及伽利略的圣经观点。[131]

1616年2月,宗教裁判所的委员会宣布日心说在哲学上是“愚蠢而荒谬的”,在神学上则是“正式的异端”,因为它在许多方面明确地与《圣经》的字面意义相矛盾。宗教裁判所认为,关于地球运动的观点“在哲学上得到了相同的判断,并且……在神学真理方面,至少是信仰上的错误”。[132] 教皇保罗五世指示红衣主教贝拉明将这一裁定通知伽利略,并命令他放弃日心说。2月26日,伽利略被召到贝拉明的住所,接到命令:“完全放弃……太阳静止于世界中心而地球运动的观点,并从此不得以任何方式持有、教授或辩护该观点,无论是口头还是书面形式。”[133] 教皇的禁书目录会还下令禁止阅读哥白尼的《天体运行论》和其他日心说著作,直至进行修改。[133]  

在接下来的十年里,伽利略远离了这一争议。1623年,红衣主教马费奥·巴贝里尼当选为教皇乌尔班八世,这鼓励了伽利略重新开始撰写关于这一主题的书。巴贝里尼是伽利略的朋友和崇拜者,并曾反对1616年对伽利略的警告。伽利略最终完成的著作《两大世界体系的对话》于1632年出版,获得了宗教裁判所的正式批准和教皇的许可。[134]
\begin{figure}[ht]
\centering
\includegraphics[width=6cm]{./figures/d9b6990f5fee1170.png}
\caption{尤斯图斯·苏斯特曼斯1635年绘制的伽利略肖像} \label{fig_JLL_9}
\end{figure}
早先,教皇乌尔班八世曾亲自要求伽利略在书中列出支持和反对日心说的论点,并注意不要宣扬日心说。然而,无论是无意还是故意,《两大世界体系的对话》中支持亚里士多德地心说的角色辛普利乔(Simplicio)经常在争论中暴露自己的错误,有时显得像个傻瓜。尽管伽利略在书的前言中声明,该角色的名字来源于一位著名的亚里士多德哲学家(拉丁文为*Simplicius*,意大利语为“Simplicio”),但在意大利语中,“Simplicio”也带有“愚蠢的人”或“简单之人”的含义。[135][136] 这种对辛普利乔的描写使《两大世界体系的对话》看起来像一本宣扬日心说的书:对亚里士多德地心说的攻击和对哥白尼理论的辩护。[citation needed]  

大多数历史学家认为,伽利略并非出于恶意,他对书引发的反应感到措手不及。[h] 然而,教皇并未轻视书中可能包含的对他的公众嘲弄,也无法接受明显对哥白尼体系的支持。

伽利略疏远了他最重要和最有权势的支持者之一——教皇,并于1632年9月被召往罗马为自己的著作辩护。[140] 他最终于1633年2月抵达罗马,并被带到宗教裁判官文琴佐·马库拉尼(Vincenzo Maculani)面前接受指控。在整个审判过程中,伽利略坚决主张自己自1616年以来一直忠实地履行了不持有任何被禁止观点的承诺,并最初否认他曾为这些观点辩护。然而,他最终被劝说承认,虽然这并非他的本意,但《两大世界体系的对话》的读者确实可能会得出这是为哥白尼理论辩护的印象。鉴于伽利略的否认缺乏可信性,即他在1616年后从未持有哥白尼观点,也从未打算在《对话》中为其辩护,他在1633年7月的最终审问中被威胁,如不说出真相将遭受酷刑,但他尽管受到威胁,仍坚持否认。[141][142][143]  

宗教裁判所的判决于6月22日宣布,主要包括三项内容:  
\begin{itemize}
\item 伽利略被认定“严重涉嫌异端”(尽管他从未被正式指控为异端,因此免除了身体惩罚),[144] 具体指他持有以下观点:太阳静止于宇宙中心;地球不是宇宙中心并且在运动;以及在某观点被宣告与《圣经》相悖后仍可以认为其是可能的并为之辩护。他被要求“放弃、诅咒并厌恶”这些观点。[145][146][147][148]  
\item 他被判处在宗教裁判所的意愿下接受正式监禁。[149] 次日,这一判决被改为软禁,他在余生中一直处于这种状态。[150]  
\item 他的问题作品《对话》被禁;此外,未在审判中宣布的一项行动是禁止出版他所有的作品,包括他未来可能撰写的任何著作。[151][152]
\end{itemize}
\begin{figure}[ht]
\centering
\includegraphics[width=6cm]{./figures/0613373b2536bed9.png}
\caption{原本归于穆里略创作的肖像画,描绘了伽利略凝视着监狱墙上刻下的文字“E pur si muove”(“然而它在移动”,此图中无法辨认)。然而,关于这幅画的归属及其相关叙述后来遭到了质疑。} \label{fig_JLL_10}
\end{figure}
根据流行的传说,在伽利略放弃其“地球绕太阳运动”理论之后,他据说低声说出了反抗性的短语“然而它在移动”(\textbf{E pur si muove})。据称,一幅1640年代的画作,由西班牙画家巴托洛梅·埃斯特班·穆里略(Bartolomé Esteban Murillo)或其画派的一位艺术家创作,其中隐藏的文字直到1911年修复时才显现,描绘了伽利略在监禁中凝视着地牢墙上写着“E pur si muove”的画面。最早已知关于这一传说的文字记载出现在他去世一个世纪之后。基于这幅画作,斯蒂尔曼·德雷克(Stillman Drake)写道:“现在毫无疑问,这句著名的话在伽利略去世之前就已被归于他了。”[153] 然而,天体物理学家马里奥·利维奥(Mario Livio)的深入调查表明,这幅画作很可能是1837年由佛兰芒画家罗曼-尤金·范·马尔代亨(Roman-Eugene Van Maldeghem)创作的一幅画的复制品。[154]  

在伽利略的一段软禁期间,他与友好的锡耶纳大主教阿斯卡尼奥·皮科洛米尼(Ascanio Piccolomini)共处,随后于1634年获准返回佛罗伦萨附近阿尔切特里的别墅,在那里他在软禁中度过了部分时光。伽利略被要求每周诵读《七首忏悔诗篇》(\textbf{Seven Penitential Psalms})一次,持续三年。然而,他的女儿玛利亚·切莱斯特(Maria Celeste)获得教会许可后,主动替他完成了这一任务。[155]  

伽利略在软禁期间完成了他最优秀的著作之一《两种新科学》(\textbf{Two New Sciences})。在这部作品中,他总结了四十年前所做的关于两门学科(现称运动学和材料强度学)的研究。为了避开审查制度,该书在荷兰出版。这部著作受到了阿尔伯特·爱因斯坦的高度赞扬。[156] 因此,伽利略常被称为“现代物理学之父”。1638年,伽利略完全失明,并患上了严重的疝气和失眠症,因此获准前往佛罗伦萨寻求医疗建议。[14]  

达娃·索贝尔(Dava Sobel)认为,在伽利略1633年因异端审判和判决之前,教皇乌尔班八世因宫廷阴谋和国家问题而分心,开始担忧受到迫害或生命威胁。在这种背景下,伽利略的问题被宫廷内部人士和伽利略的敌人提到教皇面前。因被指责在捍卫教会方面表现软弱,乌尔班出于愤怒和恐惧而对伽利略作出了反应。[157] 马里奥·利维奥(Mario Livio)则将伽利略及其发现置于现代科学和社会背景之下。他特别指出,伽利略事件在某种程度上与现代的“科学否认”现象有相似之处。[158]
\subsection{逝世}
\begin{figure}[ht]
\centering
\includegraphics[width=8cm]{./figures/a54b4eda264b6d70.png}
\caption{伽利略之墓,位于佛罗伦萨圣十字大教堂} \label{fig_JLL_11}
\end{figure}
伽利略在去世前一直接待访客,直到1642年1月8日因发烧和心悸去世,享年77岁。[14][159] 托斯卡纳大公费迪南多二世希望将他葬在佛罗伦萨圣十字大教堂的主教堂内,安放在他父亲和其他祖先的墓旁,并为他建造一座大理石陵墓以示敬意。[160][161]  

然而,由于教皇乌尔班八世及其侄子红衣主教弗朗切斯科·巴贝里尼(Cardinal Francesco Barberini)的反对,这一计划被取消。[160][161][162] 原因是伽利略曾因“严重涉嫌异端”而被天主教会定罪。[163] 他被安葬在大教堂南侧耳堂通向圣器室的一条走廊尽头,靠近初学者小礼拜堂的一个小房间内。[160][164]  

1737年,在为纪念伽利略建造的纪念碑竣工后,他的遗体被重新迁葬至大教堂的主堂。[165][166] 在迁葬过程中,他的遗骸中被取走了三根手指和一颗牙齿。[167] 其中一根手指目前陈列于意大利佛罗伦萨的伽利略博物馆(Museo Galileo)。[168]
\begin{figure}[ht]
\centering
\includegraphics[width=6cm]{./figures/a3be0c2db4eca121.png}
\caption{伽利略右手的中指} \label{fig_JLL_12}
\end{figure}
\subsection{科学贡献}  
“这些以及其他数量不少、同样值得了解的事实,我已经成功证明;更重要的是,我认为,通过我的工作,仅作为这门广阔而卓越科学的开端,为那些比我更敏锐的头脑开启了探索其深远角落的途径和方法。”

—— 伽利略·伽利莱,《两种新科学》
\subsubsection{科学方法} 
伽利略通过实验与数学的创新结合,对运动科学做出了原创性贡献。[169] 当时更典型的科学研究是威廉·吉尔伯特(William Gilbert)关于磁学和电学的定性研究。伽利略的父亲文琴佐·伽利莱(Vincenzo Galilei)是一位鲁特琴演奏家和音乐理论家,他通过实验可能确立了物理学中最古老的非线性关系之一:对于拉紧的弦,其音高与张力的平方根成比例。[170] 这些观察属于毕达哥拉斯音乐传统的框架,这一传统对乐器制造者来说十分熟知,其中包括一个事实:将琴弦按整数比例分割会产生和谐音阶。因此,数学早已在一定程度上与音乐和物理科学相关,而年轻的伽利略能看到自己父亲的观察拓展了这一传统。[171]  

伽利略是最早明确指出自然规律具有数学性质的现代思想家之一。在《观察者》(*The Assayer*)中,他写道:“哲学写在这本伟大的书中,即宇宙……它用数学的语言写成,其字符是三角形、圆形及其他几何图形;……”[172] 他的数学分析是后期经院自然哲学家传统的进一步发展,这一传统是伽利略在学习哲学时掌握的。[173] 他的工作标志着科学最终从哲学和宗教中分离的又一步,这一分离是人类思想史上的重大进展。他经常愿意根据观察结果调整自己的观点。  

为了进行实验,伽利略必须建立长度和时间的标准,以便在不同时间和实验室中进行的测量可以以可重复的方式进行比较。这为使用归纳推理验证数学定律提供了可靠的基础。[citation needed] 伽利略对数学、理论物理学和实验物理学之间的适当关系展现出了现代理解。他理解抛物线既可以通过圆锥曲线的视角理解,也可以通过纵坐标(y)随横坐标(x)的平方变化的关系来理解。伽利略进一步断言,在没有空气阻力或其他干扰的情况下,抛物线是均匀加速物体的理想弹道轨迹。他承认这一理论的适用性有限,并从理论上指出,若投射物的轨迹规模可与地球相当,则不可能是抛物线。[174][175][176] 然而,他仍然坚持认为,在其时代火炮射程范围内,投射物轨迹与抛物线的偏差非常小。[174][177][178]
\subsubsection{天文学}
伽利略通过他的折射望远镜,于1609年末观察到月球表面并不平滑。[34] 次年初,他观测到了木星的四大卫星。[49] 同年晚些时候,他观察到金星的相位变化——这为日心说提供了证明,同时还观察了土星,但当时他误以为土星的光环是另外两颗行星。[64] 1612年,他观测到海王星并注意到其运动,但并未识别其为一颗行星。[66]  

伽利略研究了太阳黑子,[67] 银河系,并对恒星进行了多项观察,包括如何在没有望远镜的情况下测量恒星的视直径。[76][77][78]  

1619年,他创造了“极光”(\textbf{Aurora Borealis})这一术语,源自罗马的黎明女神(Aurora)和希腊语中北风的名称,用来描述当太阳风粒子激发磁层时在北方和南方天空中出现的光。[179]
\subsubsection{工程学}
\begin{figure}[ht]
\centering
\includegraphics[width=8cm]{./figures/d0e19a1ff2e7e18f.png}
\caption{鲁本斯1602-1606年的画作《在曼图亚友人圈中的自画像》。伽利略是左侧第三人。画中远处描绘了北极光的景象。} \label{fig_JLL_13}
\end{figure}
伽利略在现代所谓的工程学领域做出了许多贡献,与纯粹的物理学有所区别。在1595年至1598年间,伽利略设计并改进了一种几何和军用罗盘,适用于炮手和测量员的使用。这种罗盘是在尼科洛·塔塔利亚(Niccolò Tartaglia)和圭多巴尔多·德尔·蒙特(Guidobaldo del Monte)早期设计的仪器基础上进一步发展而成。对于炮手来说,它不仅提供了一种更安全且更精确的抬高火炮的方法,还能快速计算不同大小和材料的炮弹所需的火药量。作为几何工具,这种罗盘可以用来构造任意正多边形,计算任意多边形或圆扇形的面积,以及进行多种其他计算。  

在伽利略的指导下,仪器制造商马尔切安东尼奥·马佐莱尼(Marc'Antonio Mazzoleni)生产了100多件这种罗盘。伽利略将其以50里拉的价格出售,并附带一本由他撰写的使用手册,同时还开设了一门关于使用罗盘的课程,收费为120里拉。[180]
\begin{figure}[ht]
\centering
\includegraphics[width=8cm]{./figures/6ef345fc18f73198.png}
\caption{伽利略的几何和军用罗盘,据认为由他的专属仪器制造师马尔切安东尼奥·马佐莱尼于约1604年制作。} \label{fig_JLL_14}
\end{figure}
1593年,伽利略制作了一种温度计,通过气泡中空气的膨胀和收缩推动连接管中的水流动。[citation needed]  

1609年,伽利略与英国人托马斯·哈里奥特(Thomas Harriot)及其他人一起,成为最早使用折射望远镜观察恒星、行星或卫星的人之一。“望远镜”(\textbf{telescope})这一名称是希腊数学家乔瓦尼·德米西亚尼(Giovanni Demisiani)为伽利略的仪器创造的。[181][182] 这一命名是在1611年费德里科·切西亲王(Prince Federico Cesi)为接纳伽利略加入其林切伊学院(\textbf{Accademia dei Lincei})而举办的一场宴会上提出的。[183]  

1610年,伽利略利用望远镜近距离放大观察昆虫的部分结构。[184][185] 到1624年,他已使用复合显微镜。同年5月,他将其中一台仪器赠送给佐伦红衣主教(Cardinal Zollern),由其转赠给巴伐利亚公爵;[186] 同年9月,他又将另一台赠予切西亲王。[187] 一年后,林切伊学院的另一位成员乔瓦尼·法伯(Giovanni Faber)以希腊词 \textbf{μικρόν}(micron,意为“小”)和 \textbf{σκοπεῖν}(skopein,意为“观察”)为伽利略的发明命名为“显微镜”(\textbf{microscope}),以对应“望远镜”的命名方式。[188][189]  

1625年,利用伽利略显微镜制作的昆虫插图被出版,这似乎是复合显微镜使用的首次清晰记录。[187]  
\begin{figure}[ht]
\centering
\includegraphics[width=6cm]{./figures/ebdf158b052cdf11.png}
\caption{已知最早的摆钟设计,由伽利略·伽利莱构思。} \label{fig_JLL_15}
\end{figure}
1612年,伽利略在确定木星卫星的轨道周期后提出,如果能够充分准确地掌握它们的轨道信息,可以将其位置作为通用时钟,从而实现经度的测定。在他余生中,他时不时地研究这个问题,但实践中存在许多困难。该方法首次由乔瓦尼·多梅尼科·卡西尼(Giovanni Domenico Cassini)于1681年成功应用,并随后广泛用于大型土地测量。例如,这一方法被用于绘制法国地图,后来又由泽布伦·派克(Zebulon Pike)在1806年绘制美国中西部地图时采用。在海上导航中,由于精密的望远镜观测更加困难,经度问题最终通过约翰·哈里森(John Harrison)发明的便携式海洋计时器得以解决。[190]  

在晚年完全失明时,伽利略设计了一种用于摆钟的擒纵机构(称为“伽利略擒纵器”),尽管直到17世纪50年代克里斯蒂安·惠更斯(Christiaan Huygens)制造出第一个完全可操作的摆钟后,这一设计才被付诸实施。[citation needed]  

伽利略曾多次受邀就工程方案提供建议,以缓解河流洪水问题。1630年,马里奥·圭杜奇(Mario Guiducci)可能在确保伽利略被咨询比森齐奥河(Bisenzio River)附近开辟新河道的计划中发挥了重要作用。[191]  

普通滚珠轴承的问题在于滚珠会彼此摩擦,从而产生额外的摩擦力。通过将每个滚珠装入一个笼子中可以减少这种摩擦。这种“笼式”滚珠轴承最早由伽利略在17世纪描述。[192]
\subsubsection{物理学}
\begin{figure}[ht]
\centering
\includegraphics[width=8cm]{./figures/b94a075b33164854.png}
\caption{《伽利略与维维亚尼》,作者提托·莱西,1892年} \label{fig_JLL_16}
\end{figure}
伽利略在物体运动方面的理论和实验研究,与开普勒和勒内·笛卡尔的独立研究一起,为艾萨克·牛顿发展经典力学奠定了基础。

\textbf{摆}  

伽利略进行了多次与摆相关的实验。人们普遍认为(得益于文琴佐·维维亚尼的传记)这些实验始于他在比萨大教堂观察青铜吊灯的摆动,并用自己的脉搏作为计时器。伽利略对摆的最早记录兴趣见于他死后出版的笔记《论运动》(*On Motion*),[193] 但后来的实验被描述在他的《两种新科学》中。伽利略声称一个简单的摆是等时的,即无论摆动幅度如何,摆动的时间总是相同。实际上,这只是近似成立,[194] 正如克里斯蒂安·惠更斯(Christiaan Huygens)后来发现的那样。此外,伽利略还发现,摆的周期的平方与摆长成正比。
