% 【导航】文件管理
% license Usr
% type Map

\begin{issues}
\issueDraft
\end{issues}

\subsection{简介}
一个通常的数据储存设备无论它的原理是什么,从概念上都可以看作是一根很长的纸带,划分成许多小格,每个格子只有两种状态 0 或 1,可以多次改变。每个格子都具有一个非负整数编号,且按编号按顺序递增。 对纸带的每次读写都可以通过编号指定格子。这就是著名的图灵机中的储存模型。

这个模型看起来很简单,但在实际中我们通常对储存设备有一些性能上的要求:
\begin{enumerate}
\item 纸带足够长,足够容纳要保存的数据
\item 读写速度快
\item 写入的数据一段时间内不易损坏
\item 如果发生数据损坏,能检测出来而不是在读取时给出错误的数据
\end{enumerate}

事实上,如果你去购买一块普通硬盘并在你电脑上进行默认的格式化,它在第 3 和第 4 点上的表现可能会比你以为的要差的多。 因为无论是传统的机械硬盘(HDD)还是更小更快的固态硬盘(SSD),它都可能会出随机出现少量的数据损坏(bitrot)。当然大部分时候你并不会察觉到(例如你不会注意到你的电影中有一个画面改变了一个像素)。但如果你比较倒霉,在较为关键的数据中出现了损坏,有可能导致你的电脑无法正常开机。

更糟糕的是,当一块硬盘用了几年后,或者写入了一定量的数据后,它有可能直接整体发生故障导致无法进行任何读写。
% \addTODO{需要一个硬盘常识的文章}

要解决这些问题,使用更好的硬盘是远远不够的,无论是个人还是企业,一般都需要从软件层面上让数据储存变得更可靠。

\subsection{文件系统}
一些额外的功能:
\begin{itemize}
\item 可以查看纸带在之前某个时间点的数据
\end{itemize}

计算机文件简介\upref{ComFil}

【推荐】ZFS 文件系统(Zettabyte File System)笔记\upref{ZFS}

Linux 分区和文件系统操作笔记(Gparted, fdisk, resize2fs, grub, Clonezilla)\upref{fdisk}

创建网络文件夹(NFS、SMB、sshfs)笔记\upref{NFS}


\subsection{数据管理软件}
计算机文件备份基础(附 python 多版本增量备份脚本)\upref{SimBac}

如何给文件加密(含 python 加密脚本)\upref{encryp}

用网盘增量备份文件\upref{PanBak}

Git 笔记\upref{Git}

用 Git 备份文件夹\upref{gitBac}

Git-LFS 笔记\upref{gitLFn}
