% 点集拓扑学(综述)
% license CCBYSA3
% type Wiki

本文根据 CC-BY-SA 协议转载翻译自维基百科\href{https://en.wikipedia.org/wiki/General_topology}{相关文章}。

\begin{figure}[ht]
\centering
\includegraphics[width=10cm]{./figures/16b10920f52e4ba5.png}
\caption{} \label{fig_DJTP_1}
\end{figure}
在数学中,一般拓扑(或称点集拓扑)是拓扑学的一个分支,主要研究拓扑学中使用的基本集合论定义和构造。它是大多数其他拓扑学分支的基础,包括微分拓扑、几何拓扑和代数拓扑。

点集拓扑中的基本概念是连续性、紧致性和连通性:
\begin{itemize}
\item 连续函数直观上是将相邻的点映射到相邻的点。
\item 紧致集是指可以被有限多个任意小的集合覆盖的集合。
\item 连通集是指不能被分成两个彼此远离的部分的集合。
\end{itemize}
“附近”、“任意小”和“远离”这些术语都可以通过使用开集的概念来精确定义。如果我们改变“开集”的定义,就会改变连续函数、紧致集和连通集的定义。对于“开集”的每一种定义选择,都称为一种拓扑。具有拓扑的集合称为拓扑空间。

度量空间是拓扑空间中的一个重要类别,在这些空间中,可以为集合中的点对定义一个实数的非负距离,也称为度量。具有度量简化了许多证明,且许多最常见的拓扑空间都是度量空间。
\subsection{历史}
一般拓扑学起源于多个领域,其中最重要的包括:
\begin{itemize}
\item 对实数线子集的详细研究(曾被称为点集的拓扑;这种用法现已过时)
\item 流形概念的引入
\item 在功能分析的早期阶段对度量空间,特别是赋范线性空间的研究。
\end{itemize}
一般拓扑学在大约1940年形成了现在的形式。可以说,它几乎囊括了连续性直觉的所有内容,以一种技术上足够的形式,能够应用于数学的任何领域。
\subsection{集合上的拓扑}
设 $X$ 是一个集合,$\tau$ 是 $X$ 的子集族。如果 $\tau$ 满足以下条件,则称 $\tau$ 是 $X$ 上的拓扑:\(^\text{[1][2]}\)
\begin{enumerate}
\item 空集和 $X$ 本身是 $\tau$ 的元素
\item $\tau$ 的元素的任意并集是 $\tau$ 的元素
\item $\tau$ 的有限多个元素的任意交集是 $\tau$ 的元素
\end{enumerate}
如果 $\tau$ 是 $X$ 上的拓扑,则对偶 $(X, \tau)$ 称为拓扑空间。可以使用符号 $X_\tau$ 来表示 $X$ 上带有特定拓扑 $\tau$ 的集合。

$\tau$ 的成员称为 $X$ 中的开集。如果 $X$ 的一个子集的补集在 $\tau$ 中(即其补集是开集),则称该子集是闭集。$X$ 的一个子集可以是开集、闭集、两者(闭开集)或都不是。空集和 $X$ 本身总是既是闭集又是开集。
\subsubsection{拓扑的基}
一个拓扑空间 $X$ 及其拓扑 $T$ 的基 $B$ 是 $T$ 中开集的一个集合,使得 $T$ 中的每个开集都可以写成 $B$ 中元素的并集。\(^\text{[3][4]}\)我们说基 $B$ 生成了拓扑 $T$。基是有用的,因为拓扑的许多性质可以简化为关于生成该拓扑的基的陈述——并且许多拓扑可以通过定义生成它们的基来最容易地定义。
\subsubsection{子空间与商空间}
每个拓扑空间的子集都可以赋予子空间拓扑,其中开集是大空间的开集与子集的交集。对于任何索引族的拓扑空间,可以赋予积空间积拓扑,该拓扑由在投影映射下,各个因子的开集的逆像生成。例如,在有限积中,积拓扑的基由所有开集的积组成。在无限积中,还需要额外的要求:在一个基本开集中的所有投影中,除了有限个投影之外,其他的投影都是整个空间。

商空间定义如下:如果 $X$ 是一个拓扑空间,$Y$ 是一个集合,并且 $f : X \to Y$ 是一个满射函数,则 $Y$ 上的商拓扑是那些在 $f$ 下具有开逆像的 $Y$ 的子集。换句话说,商拓扑是使得 $f$ 连续的最细拓扑。商拓扑的一个常见例子是当在拓扑空间 $X$ 上定义一个等价关系时。此时,映射 $f$ 是到等价类集合的自然投影。
\subsubsection{拓扑空间的例子}
给定的集合可以具有多种不同的拓扑。如果给一个集合赋予不同的拓扑,它就被视为一个不同的拓扑空间。

\textbf{离散拓扑与平凡拓扑}

任何集合都可以赋予离散拓扑,在该拓扑中,所有子集都是开集。在这种拓扑中,唯一收敛的序列或网是那些最终恒定的序列或网。此外,任何集合也可以赋予平凡拓扑(也叫不可分拓扑),在这种拓扑中,只有空集和整个空间是开集。在这种拓扑中的每个序列和网都收敛到空间的每一点。这个例子表明,在一般的拓扑空间中,序列的极限不一定是唯一的。然而,通常拓扑空间必须是豪斯多夫空间,在这种空间中极限点是唯一的。

\textbf{余有限拓扑与余可数拓扑}

任何集合都可以赋予余有限拓扑,在这种拓扑中,开集是空集和补集是有限集的集合。这是任何无限集合上的最小的 $T_1$ 拓扑。

任何集合也可以赋予余可数拓扑,在这种拓扑中,如果一个集合是空集或者它的补集是可数的,则该集合被定义为开集。当集合是不可数时,这种拓扑在许多情况下作为反例。

\textbf{实数和复数的拓扑}

在 $\mathbf{R}$(实数集)上有许多方法可以定义拓扑。标准的实数拓扑是由开区间生成的。所有开区间的集合构成了该拓扑的基(或基底),这意味着每个开集都是从基中某些集合的并集。特别地,这意味着一个集合是开集,如果在该集合中的每个点周围都存在一个非零半径的开区间。更一般地,欧几里得空间 $\mathbf{R}^n$ 也可以赋予拓扑。在 $\mathbf{R}^n$ 的通常拓扑中,基本开集是开球。类似地,复数集 $\mathbf{C}$ 和 $\mathbf{C}^n$ 也有一个标准拓扑,其中基本开集是开球。

实数线也可以赋予下限拓扑。在这里,基本开集是半开区间 $[a, b)$。这种拓扑在实数集 $\mathbf{R}$ 上比上面定义的欧几里得拓扑要细致;如果一个序列在这种拓扑中收敛到一个点,当且仅当它在欧几里得拓扑中从上方收敛到该点。这个例子表明,一个集合可以有许多不同的拓扑定义。

\textbf{度量拓扑}

每个度量空间都可以赋予度量拓扑,其中基本开集是由度量定义的开球。这是任何赋范向量空间上的标准拓扑。在有限维向量空间中,对于所有范数,这个拓扑是相同的。

\textbf{更多例子}

\begin{itemize}
\item 在任何给定的有限集合上都存在众多拓扑。这类空间被称为有限拓扑空间。有限空间有时用于提供关于拓扑空间的一般猜想的例子或反例。
\item 每个流形都有一个自然的拓扑,因为它是局部欧几里得的。类似地,每个单纯形和每个单纯复形从$\mathbf{R}^n$中继承一个自然的拓扑。
\item 扎里斯基拓扑在一个环或代数簇的谱上按代数方式定义。在$\mathbf{R}^n$或$\mathbf{C}^n$上,扎里斯基拓扑的闭集是多项式方程组的解集。
\item 线性图具有一种自然的拓扑,它概括了图形中顶点和边的许多几何方面。
\item 在泛函分析中,许多线性算子集合被赋予了通过指定某个特定函数序列何时收敛到零函数来定义的拓扑。
\item 任何局部域都有一个原生拓扑,并且可以扩展到该域上的向量空间。
\item 谢尔宾斯基空间是最简单的非离散拓扑空间。它与计算理论和语义学有重要的关系。
\item 如果 $\Gamma$ 是一个序数,则集合 $\Gamma = [0, \Gamma)$ 可以赋予由区间 $(a, b)$,[0, b) )和 $(a, \Gamma)$ 生成的顺序拓扑,其中 $a$ 和 $b$ 是 $\Gamma$ 的元素。
\end{itemize}
\subsection{连续函数}
连续性是通过邻域来表示的:如果且仅如果对于任何 $f(x)$ 的邻域 $V$,存在 $x$ 的一个邻域 $U$,使得 $f(U) \subseteq V$,那么函数 $f$ 在点 $x \in X$ 处是连续的。直观地说,连续性意味着无论 $V$ 多么"小",总会存在一个包含 $x$ 的邻域 $U$,使得 $f$ 在 $U$ 上的像包含在 $V$ 内,并且 $f$ 的像包含 $f(x)$。这与以下条件是等价的:$Y$ 中的开(闭)集的原像在 $X$ 中是开(闭)集。在度量空间中,这一定义等价于在分析中常用的 $\varepsilon$-$\delta$ 定义。

一个极端的例子:如果集合$X$赋予离散拓扑,那么到任何拓扑空间$T$的所有函数
$$
f \colon X \rightarrow T~
$$
都是连续的。另一方面,如果$X$赋予不可离散拓扑,并且空间$T$ 至少是$T_0$空间,那么唯一的连续函数是常函数。反过来,任何值域是不可离散的函数都是连续的。
\subsubsection{替代定义}
存在几种等价的拓扑结构定义,因此也有几种等价的方式来定义连续函数。

\textbf{邻域定义}

基于逆像的定义通常不容易直接使用。以下准则通过邻域来表达连续性:如果函数 $f$ 在某点 $x \in X$ 连续,当且仅当对于 $f(x)$ 的任意邻域 $V$,都存在 $x$ 的邻域 $U$,使得 $f(U) \subseteq V$。直观上,连续性意味着无论 $V$ 变得多么“小”,总有一个包含 $x$ 的邻域 $U$,使得其映射落在 $V$ 内。

如果 $X$ 和 $Y$ 是度量空间,那么考虑以 $x$ 和 $f(x)$ 为中心的开球的邻域系统,而不是所有邻域,这与在度量空间中的 δ-ε 连续性定义是等价的。然而,在一般的拓扑空间中,并没有邻近或距离的概念。

然而需要注意的是,如果目标空间是豪斯多夫空间,仍然有 $f$ 在 $a$ 处连续当且仅当当 $x$ 趋近于 $a$ 时,$f(x)$ 的极限是 $f(a)$ 这一点成立。在孤立点处,每个函数都是连续的。

\textbf{序列和网}

在多个上下文中,空间的拓扑通常通过极限点来方便地指定。在许多情况下,这是通过指定一个点何时是序列的极限来实现的,但对于某些在某种意义上过于大的空间,也指定一个点何时是由一个有向集索引的更一般的点集的极限,这些点集被称为网。\(^\text{[5]}\)一个函数只有在它将序列的极限映射到序列的极限时才是连续的。在前一种情况下,极限的保持也是足够的;在后一种情况下,一个函数可能保持所有序列的极限,但仍然不是连续的,保持网的极限是必要且充分的条件。

详细来说,一个函数 $f: X \rightarrow Y$ 如果每当 $X$ 中的序列 $(x_n)$ 收敛于极限 $x$ 时,序列 $(f(x_n))$ 收敛于 $f(x)$,则称该函数是序列连续的。\(^\text{[6]}\)因此,序列连续的函数“保持序列的极限”。每个连续函数都是序列连续的。如果 $X$ 是一个首数可数空间并且满足可数选择公理,那么反之亦然:任何保持序列极限的函数都是连续的。特别地,如果 $X$ 是度量空间,序列连续性和连续性是等价的。对于非首数可数空间,序列连续性可能严格弱于连续性。(对于这两种性质等价的空间称为序列空间。)这促使我们在一般拓扑空间中考虑网而非序列。连续函数保持网的极限,事实上,这一特性表征了连续函数。

\textbf{闭包算子定义}

与指定拓扑空间的开子集不同,拓扑也可以通过闭包算子(记作 cl)来确定,闭包算子将任意子集 $A \subseteq X$ 映射到其闭包,或者通过内核算子(记作 int)来确定,内核算子将任意子集 $A$ 映射到其内核。在这种情况下,一个函数
$$
f: (X, \mathrm{cl}) \to (X', \mathrm{cl'})~
$$
在拓扑空间之间是连续的,当且仅当对于所有的子集 $A \subseteq X$,有
$$
f(\mathrm{cl}(A)) \subseteq \mathrm{cl'}(f(A)).~
$$
也就是说,给定 $X$ 中任意一个元素 $x$ 属于某个子集 $A$ 的闭包,$f(x)$ 属于 $f(A)$ 的闭包。这等价于以下要求,对于 $X'$ 中所有的子集 $A'$,有
$$
f^{-1}(\mathrm{cl'}(A')) \supseteq \mathrm{cl}(f^{-1}(A')).~
$$
此外,
$$
f: (X, \mathrm{int}) \to (X', \mathrm{int'})~
$$
是连续的,当且仅当对于 $X$ 中的任何子集 $A$,有
$$
f^{-1}(\mathrm{int'}(A)) \subseteq \mathrm{int}(f^{-1}(A)).~
$$
\subsubsection{性质}
如果 $f: X \to Y$ 和 $g: Y \to Z$ 是连续的,那么它们的复合函数 $g \circ f: X \to Z$ 也是连续的。如果 $f: X \to Y$ 是连续的,并且
\begin{itemize}
\item $X$ 是紧致的,那么 $f(X)$ 是紧致的。
\item $X$ 是连通的,那么 $f(X)$ 是连通的。
\item $X$ 是路径连通的,那么 $f(X)$ 是路径连通的。
\item $X$ 是 Lindelöf 的,那么 $f(X)$ 是 Lindelöf 的。
\item $X$ 是可分的,那么 $f(X)$ 是可分的。
\end{itemize}
对于固定集合 $X$,可能的拓扑是部分有序的:如果拓扑 $\tau_1$ 比另一个拓扑 $\tau_2$ 更粗(记作 $\tau_1 \subseteq \tau_2$),那么当且仅当每个在 $\tau_1$ 下是开的子集也是在 $\tau_2$ 下开的子集时,拓扑 $\tau_1$ 就是比 $\tau_2$ 更粗的。然后,恒等映射
$$
\text{id}_X: (X, \tau_2) \to (X, \tau_1)~
$$
当且仅当 $\tau_1 \subseteq \tau_2$ 时是连续的(参见拓扑的比较)。更一般地,连续函数
$$
(X, \tau_X) \to (Y, \tau_Y)~
$$
如果将拓扑 $\tau_Y$ 替换为更粗的拓扑,和/或将拓扑 $\tau_X$ 替换为更细的拓扑,仍然保持连续性。
\subsubsection{同胚映射}
与连续映射的概念对称的是开映射,其特征是开集的像也是开集。事实上,如果一个开映射 $f$ 存在逆函数,那么该逆函数是连续的;如果一个连续映射 $g$ 存在逆函数,那么该逆函数是开映射。对于两个拓扑空间之间的一个双射映射 $f$,其逆映射 $f^{-1}$ 不一定是连续的。一个双射连续映射,如果其逆映射也是连续的,则称该映射为同胚映射。

如果一个连续双射的定义域是紧致空间,而它的值域是 Hausdorff 空间,那么该映射是同胚映射。
\subsubsection{通过连续函数定义拓扑}
给定一个函数
$$
f \colon X \rightarrow S,~
$$
其中 $X$ 是一个拓扑空间,$S$ 是一个集合(没有指定拓扑),则 $S$ 上的最终拓扑定义为:令 $S$ 的开集为那些子集 $A \subseteq S$,使得 $f^{-1}(A)$ 在 $X$ 中是开集。如果 $S$ 已经有一个现有拓扑,那么 $f$ 对应于该拓扑是连续的,当且仅当现有拓扑比 $S$ 上的最终拓扑更粗。由此,最终拓扑可以被描述为使得 $f$ 连续的最细拓扑。如果 $f$ 是满射,则这个拓扑与通过 $f$ 定义的等价关系下的商拓扑在规范上是等同的。

对偶地,对于从集合 $S$ 到拓扑空间 $X$ 的函数 $f$,$S$ 上的初始拓扑的基是由那些形如 $f^{-1}(U)$ 的开集构成,其中 $U$ 在 $X$ 中是开集。如果 $S$ 已经有一个现有拓扑,则 $f$ 对应于该拓扑是连续的,当且仅当现有拓扑比 $S$ 上的初始拓扑更细。由此,初始拓扑可以被描述为使得 $f$ 连续的最粗拓扑。如果 $f$ 是单射,则这个拓扑与 $S$ 在 $X$ 中作为子集所拥有的子空间拓扑在规范上是等同的。

集合 $S$ 上的拓扑由所有连续函数$S \rightarrow X$到所有拓扑空间 $X$ 的类唯一确定。对偶地,可以对映射$X \rightarrow S$应用类似的概念。
\subsection{紧致集}
正式地,拓扑空间 $X$ 被称为紧致的,如果其每一个开覆盖都有一个有限子覆盖。否则,它被称为非紧致。明确地,这意味着对于每一个任意的集合
$$
\{ U_{\alpha} \}_{\alpha \in A}~
$$
的 $X$ 的开子集,满足
$$
X = \bigcup_{\alpha \in A} U_{\alpha},~
$$
存在 $A$ 的一个有限子集 $J$,使得
$$
X = \bigcup_{i \in J} U_i.~
$$
一些数学分支,如代数几何,通常受到法国布尔巴基学派的影响,使用术语“拟紧致”来表示一般概念,并将“紧致”这一术语保留给既是豪斯多夫空间又是拟紧致的拓扑空间。紧致集有时被称为紧致体(compactum),复数形式为紧致体(compacta)。

实数集 $R$ 中每个有限长度的闭区间都是紧致的。更进一步地,在 $R^n$ 中,一个集合当且仅当它是闭的且有界时才是紧致的。(见海涅-博雷尔定理)

每个紧致空间的连续映像都是紧致的。

豪斯多夫空间的紧致子集是闭集。

每个从紧致空间到豪斯多夫空间的连续双射必然是同胚。

每个紧致度量空间中点的序列都有一个收敛子序列。

每个紧致的有限维流形都可以嵌入某个欧几里得空间 $R^n$。
\subsection{连通集}
一个拓扑空间 $X$ 如果是两个不相交的非空开集的并集,则称其为不连通的。否则,$X$ 被称为连通的。一个拓扑空间的子集,如果在其子空间拓扑下是连通的,则称该子集是连通的。一些作者将空集(具有唯一拓扑的空集)排除在连通空间之外,但本文不遵循这种做法。

对于一个拓扑空间 $X$,以下条件是等价的:
\begin{enumerate}
\item $X$ 是连通的。
\item $X$ 不能被分割为两个不相交的非空闭集。
\item $X$ 中唯一既是开集又是闭集(即连通集)的子集是 $X$ 和空集。
\item $X$ 中唯一边界为空的子集是 $X$ 和空集。
\item $X$ 不能表示为两个非空分离集的并集。
\item 从 $X$ 到 $\{0,1\}$(这个空间赋予离散拓扑)的唯一连续函数是常数函数。
\item 实数集 $R$ 中的每个区间都是连通的。
\end{enumerate}
连通空间的连续映像是连通的。
\subsubsection{连通分支}
非空拓扑空间的最大连通子集(按包含关系排序)称为该空间的连通分支。任何拓扑空间$X$的分支构成$X$的一个划分:它们是互不相交的、非空的,并且它们的并集就是整个空间。每个分支都是原空间的闭子集。因此,在它们的数量有限的情况下,每个分支也是开子集。然而,如果它们的数量是无限的,则不一定如此;例如,有理数集的连通分支是单点集,而这些单点集并不是开集。

设$\Gamma_x$为拓扑空间 $X$ 中点 $x$ 的连通分支,且 $\Gamma_x'$ 为包含 $x$ 的所有开闭集的交集(称为 $x$ 的准分支)。那么,$\Gamma_x \subset \Gamma_x'$当且仅当 $X$ 是紧Hausdorff空间或局部连通时,二者相等。
\subsubsection{非连通空间}
一个空间,如果它的所有分支都是单点集,则称其为完全非连通的。与此性质相关,空间 $X$ 被称为完全分离的,如果对于 $X$ 中的任何两个不同元素 $x$ 和 $y$,存在不相交的开邻域 $U$ 和 $V$,使得 $X$ 是 $U$ 和 $V$ 的并集。显然,任何完全分离的空间都是完全非连通的,但反之则不成立。例如,取两个有理数集 $\mathbf{Q}$ 的副本,并将它们在除零外的每个点上识别起来。由此得到的空间,采用商拓扑,是完全非连通的。然而,考虑零的两个副本,可以发现该空间并不是完全分离的。事实上,它甚至不是Hausdorff空间,且完全分离的条件严格强于Hausdorff空间的条件。
\subsubsection{路径连通集}
\begin{figure}[ht]
\centering
\includegraphics[width=8cm]{./figures/8dd00b0ccfcb65a6.png}
\caption{这个 $\mathbf{R}^2$ 的子空间是路径连通的,因为可以在空间中的任何两个点之间画出一条路径。} \label{fig_DJTP_2}
\end{figure}
路径从点 $x$ 到点 $y$,在拓扑空间 $X$ 中,从点 $x$ 到点 $y$ 的路径是一个从单位区间 $[0,1]$ 到 $X$ 的连续函数 $f$,其中 $f(0) = x$ 且 $f(1) = y$。
$X$ 的路径分量是 $X$ 下的一个等价类,等价关系使得如果从 $x$ 到 $y$ 存在路径,则 $x$ 与 $y$ 是等价的。
如果 $X$ 中的点之间存在一条路径将任意两点连接起来,则称 $X$ 是路径连通的(或称为路径可连接的或 0-连通的);即,$X$ 至多有一个路径分量。许多作者会排除空集的情况。

路径连通与连通性,每一个路径连通的空间都是连通的。然而,反过来并不总是成立:并不是所有连通的空间都是路径连通的。例如,扩展的长线 $L^*$ 和拓扑学正弦曲线就是连通但不路径连通的空间。

然而,实数线 $R$ 的子集当且仅当是路径连通时才是连通的;这些子集是 $R$ 的区间。此外,$R^n$ 或 $C^n$ 的开子集当且仅当是路径连通时才是连通的。另外,对于有限拓扑空间,连通性和路径连通性是相同的。
