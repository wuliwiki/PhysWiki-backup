% Hubbard模型
% Hubbard model|hubbard model|Hubbard模型|hubbard模型

在本节中,通过对电子多体系统之间的相互作用进行近似计算,仅考虑其中的库伦相互作用来得到Hubbard模型的哈密顿量。

选择原子轨道基$|j\rangle =c_j^\dagger |0\rangle$,写出系统的哈密顿量:

\begin{equation}
H=\sum\limits_{i,j,\sigma}\langle i |H_0| j \rangle c_{i,\sigma}^\dagger c_{j,\sigma}^~+\sum\limits_{l,m,k,n,\sigma,\sigma'}\langle l,m|V|k,n\rangle c_{l,\sigma}^\dagger  c_{m,\sigma'}^\dagger c_{n,\sigma'}^~c_{k,\sigma}^~~.
\end{equation}

上式中的$H_0$代表动能部分的哈密顿量,其对应的$\sum\limits_{i,j,\sigma}\langle i |H_0| j \rangle c_{i,\sigma}^\dagger c_{j,\sigma}^~$部分为跃迁项。

上式中后一部分代表相互作用项,由于库仑相互作用并不影响自旋,所以电子被相互作用散射后自旋不变,这一点在哈密顿量中的体现则为发生散射后原本自旋为$\sigma$的粒子自旋还为$\sigma$,自旋为$
\sigma'$的粒子散射后自旋还为$\sigma'$。也就是说上式中后一项中$l$和$k$对应的是同一粒子的两个态,而$m$和$n$对应的是同一个粒子的两个态。

下面分别对ha'mi'du