% 彼得·诺尔(综述)
% license CCBYSA3
% type Wiki

本文根据 CC-BY-SA 协议转载翻译自维基百科\href{https://en.wikipedia.org/wiki/Peter_Naur}{相关文章}。

\begin{figure}[ht]
\centering
\includegraphics[width=6cm]{./figures/2ce8177aaad1664e.png}
\caption{2008年的诺尔} \label{fig_BDNR_1}
\end{figure}
彼得·诺尔(Peter Naur,1928年10月25日—2016年1月3日)是丹麦计算机科学的先驱,并于2005年获得图灵奖。他最为人知的贡献是与约翰·巴克斯(John Backus)一起,提出了用于描述大多数编程语言语法的巴克斯-诺尔范式(BNF)符号。此外,他还参与了编程语言ALGOL 60的创建。
\subsection{传记}
诺尔最初的职业是天文学家,并于1957年获得哲学博士(Ph.D.)学位,但他与计算机的接触使他改变了职业方向。从1959年到1969年,他在丹麦计算机公司Regnecentralen工作,同时还在尼尔斯·玻尔研究所和丹麦技术大学讲授课程。从1969年到1998年,诺尔在哥本哈根大学担任计算机科学教授。

他是国际信息处理联合会(IFIP)工作组2.1——算法语言与计算(Algorithmic Languages and Calculi)的成员,该工作组负责指定、支持并维护ALGOL 60和ALGOL 68语言。1960年至1993年间,他还是《BIT数值数学》期刊的编辑委员会成员,该期刊专注于数值分析。

诺尔的主要研究领域包括计算机程序和算法的设计、结构与性能。他在软件工程和软件架构方面也作出了开创性的贡献。在他的著作《计算:一种人类活动》(1992年)中,汇集了他对计算机科学的贡献,他拒绝了将编程视为数学分支的形式主义编程学派。他不喜欢将“巴科斯–诺尔范式”与他联系在一起(这是唐纳德·克努斯给出的说法),并表示他更愿意将其称为“巴科斯正常形式”。

诺尔与计算机科学家Christiane Floyd结婚。

诺尔不喜欢“计算机科学”这一术语,建议将其称为“数据学”或“数据科学”。“数据学”这一术语在丹麦和瑞典被称为“datalogi”,而“数据科学”现在通常用于数据分析,包括统计学和数据库。

自1960年代中期以来,计算机科学在丹麦一直以彼得·诺尔的“数据学”术语进行研究,数据学即数据过程的科学。从Regnecentralen和哥本哈根大学开始,哥本哈根计算机科学传统通过与应用和其他知识领域的紧密联系,发展出了自己独特的特点。这一传统在教育领域尤为明显。全面的项目活动是课程的一部分,通过实际经验,理论得以与现实解决方案结合。诺尔早期就认识到计算机科学教育面临的独特挑战。他的创新成果在其他大学也得到了证明。哥本哈根大学的计算机科学培训与诺尔的研究观点紧密相关,尤其是在计算机科学教育的形成方面。

在晚年,诺尔直言不讳地批评整体科学追求:诺尔可能与经验主义学派相契合,认为不应追求事物之间更深层次的联系,而应关注可观察到的事实。他从这个观点出发,批评了某些哲学和心理学的学派。他还提出了一种名为“突触状态理论”的人类思维理论。

诺尔因在定义编程语言ALGOL 60方面的工作而获得了2005年计算机协会(ACM)A.M. 图灵奖。特别是,他在编辑具有影响力的《ALGOL 60算法语言报告》中的角色,以及该报告开创性的使用BNF的贡献,得到了认可。诺尔是唯一获得图灵奖的丹麦人。

诺尔于2016年1月3日因病去世。他在Gentofte的故居现由社会学家Claire Maxwell所有。
\subsection{参考书目}
以下数字参考了E. Sveinsdottir和E. Frøkjær出版的参考书目。[citation needed] 诺尔发表了大量关于天文学、计算机科学、社会问题、古典音乐、心理学和教育的文章和章节。
\begin{itemize}
\item 66. 《小行星51号Nemausa与基准的赤纬系统》,博士论文,1957年  
\item 95. (编辑)Backus, J. W.; Wegstein, J. H.; van Wijngaarden, A.; Woodger, M.; Bauer, F. L.; Green, J.; Katz, C.; McCarthy, J.; Perlis, A. J.; Rutishauser, H.; Samelson, K.; Vauquois, B.(1960年5月)。“ALGOL \item 60算法语言报告”。《通讯ACM》 3(5):299–314。doi:10.1145/367236.367262. S2CID 278290。以及其他几本期刊。  
\item 128. (编辑)Backus, J. W.; Wegstein, J. H.; van Wijngaarden, A.; Woodger, M.; Naur, P.; Bauer, F. L.; Green, J.; Katz, C.; McCarthy, J.; Perlis, A. J.; Rutishauser, H.; Samelson, K.; Vauquois, B.(1963年1月)。“ALGOL 60算法语言修订报告”。《通讯ACM》 6(1):1–17。doi:10.1145/366193.366201. S2CID 7853511。  
\item 144. “跳转语句与良好的Algol风格”。《BIT》 3(3):204–208,1963年。doi:10.1007/BF01939987。ISSN 0006-3835。  
\item 212. —; Randell, Brian; Buxton, J.N.(1976年)[1969]。《软件工程会议,1968年10月7日至11日,德国Garmisch》。ISBN 978-0884053347。OCLC 610836679。  
\item 213. —; Gram, C.; Hald, J.; Hansen, H. B.; Wessel, A.(1969年)。《Datamatik - 学生文献》。  
\item 247, 249. (与B. Pedersen合著)《数学4课程书》,2卷,哥本哈根大学,1971年,第2版1972年。  
\item 264. 《计算方法简明概述》,397页,《学生文献》,1974年。  
\item 274. 《数据学2 1975/76》,102页,哥本哈根大学,1975年,新版1976年。  
\item 333. —(1992年)。《计算:一种人类活动》。ACM Press。ISBN 978-0201580693。  
\item 347. —(1995年)。《知识与逻辑与规则的神秘:包括知识与行动中的真实陈述*人类知识活动的计算建模*一致描述作为学术与科学的核心》。Springer。ISBN 978-0-7923-3680-8。  
\item 363. 诺尔,彼得(1999年)。《反哲学词典:思维 - 语言性 - 科学性》。Naur.com出版。ISBN 87-987221-0-7;2001年英文版,ISBN 87-987221-1-5。  
\item 382. 诺尔,彼得(2002年)。《心理学的科学重构》。Naur.com。ISBN 978-87-987221-2-0。  
\item —(2007年1月)。“计算与人类思维”。《通讯ACM》 50(1):85–94。doi:10.1145/1188913.1188922。  
\item Daylight, E.G.; 诺尔,彼得(2011年)。《软件工程中的多元化:图灵奖得主彼得·诺尔解释》。孤独学者。ISBN 978-94-91386-00-8。
\end{itemize}
\subsection{另见}
\begin{itemize}
\item 计算机科学先驱名单
\end{itemize}
\subsection{参考文献}
\begin{enumerate}
\item "彼得·诺尔去世,享年87岁"。原文已存档于2016年1月4日。于2016年1月4日查阅。
\item Jeuring, Johan; Meertens, Lambert; Guttmann, Walter (2016年8月17日)。“IFIP工作组2.1简介”。Foswiki。原文已存档于2021年3月8日。于2020年9月2日查阅。
\item Swierstra, Doaitse; Gibbons, Jeremy; Meertens, Lambert (2011年3月2日)。“ScopeEtc: IFIP21: Foswiki”。Foswiki。原文已存档于2018年9月2日。于2020年9月2日查阅。
\item Fröberg, Carl Erik。“BIT - 一本与计算机相关的北欧期刊”。于2013年7月31日查阅。
\item Naur, Peter (1985)。“彼得·诺尔,《编程作为理论构建》”(PDF)。计算机科学:计算机、数据与信息科学学院。威斯康星大学麦迪逊分校。于2020年9月2日查阅。
\item Sveinsdottir, Edda; Frøkjær, Erik (1988)。“数据学 - 哥本哈根计算机科学传统”。《BIT》 28(3):450–472。doi:10.1007/BF01941128. S2CID 9672754。
\item "诺尔,突触-状态理论的精神生活"(PDF)。2004年。原文已存档于2011年9月27日。于2011年6月15日查阅。
\item "软件先驱彼得·诺尔获ACM图灵奖"。2006年2月。原文已存档于2007年6月9日。
\item Devantier, Nicolai (2016年1月4日)。“世界著名的丹麦IT专家彼得·诺尔去世——Computerworld”。《Computerworld》 (丹麦语)。于2016年1月4日查阅。
\end{enumerate}
\subsection{外部链接}
\begin{itemize}
\item 个人网站,包含详细的参考书目,原文已存档于2015年7月28日(Wayback Machine)
\item 2006年UIST会议演讲
\item 2005年图灵奖获奖者彼得·诺尔的ACM资料,作者:Edgar G. Daylight
\end{itemize}