% 斯蒂芬·霍金(综述)
% license CCBYSA3
% type Wiki

本文根据 CC-BY-SA 协议转载翻译自维基百科\href{https://en.wikipedia.org/wiki/Stephen_Hawking}{相关文章}。

\begin{figure}[ht]
\centering
\includegraphics[width=6cm]{./figures/d175407efe80fdaf.png}
\caption{霍金,大约1980年} \label{fig_HJ_1}
\end{figure}
斯蒂芬·威廉·霍金(Stephen William Hawking,1942年1月8日—2018年3月14日)是英国的理论物理学家、宇宙学家和作家,曾担任剑桥大学理论宇宙学研究中心的研究主任。[6][17][18] 从1979年到2009年,他是剑桥大学的卢卡斯数学教授,这一职位被广泛认为是世界上最具声望的学术职务之一。[19]

霍金出生于牛津,来自一个医学世家。1959年10月,17岁的他开始在牛津大学大学学院学习,并获得了物理学一等荣誉学位。1962年10月,他开始在剑桥大学三一学院攻读研究生,并于1966年3月获得应用数学和理论物理学博士学位,专业方向为广义相对论和宇宙学。1963年,霍金在21岁时被诊断为一种早期发病、进展缓慢的运动神经元病,这种病症在几十年中逐渐使他瘫痪。[20][21] 失去语言能力后,他通过语音生成设备进行交流,最初使用手持开关,后来通过单个面部肌肉来控制设备。[22]

霍金的科学成就包括与罗杰·彭罗斯(Roger Penrose)合作研究广义相对论框架下的引力奇点定理,以及理论预测黑洞会发射辐射,这一现象通常被称为霍金辐射。最初,霍金辐射的预测颇具争议。但到了1970年代末,随着进一步研究的发表,这一发现被广泛接受,成为理论物理学中的重大突破。霍金是第一个提出将广义相对论与量子力学结合来解释宇宙学的理论的人。他是多世界解释的积极支持者。[23][24] 他还提出了微型黑洞的概念。[25]

霍金通过几部畅销的科普作品取得了商业成功,他在书中讨论了自己的理论和宇宙学问题。他的著作《时间简史》曾连续237周登上《星期日泰晤士报》畅销书榜,创下纪录。霍金是英国皇家学会会员、教宗科学院终身会员,并获得了美国总统自由勋章,这是美国的最高平民荣誉奖。2002年,霍金在BBC的“100位最伟大的英国人”评选中排名第25位。霍金于2018年去世,享年76岁,诊断为运动神经元病后,他活过了50多年。
\subsection{早年生活} 
\subsubsection{家庭}  
霍金于1942年1月8日出生在牛津,父母是弗兰克·霍金和伊莎贝尔·艾琳·霍金(娘家姓沃克)。霍金的母亲出生在苏格兰格拉斯哥的一个医生家庭。霍金的父亲,来自约克郡的富有曾祖父,在20世纪初的大农业萧条中因购买农田过度投资而破产。霍金的曾祖母通过在家中开设学校,成功将家庭从经济困境中解救出来。尽管家庭经济条件拮据,霍金的父母都进入了牛津大学,弗兰克学习医学,伊莎贝尔学习哲学、政治学和经济学。伊莎贝尔曾在一家医学研究所担任秘书,而弗兰克是一名医学研究员。霍金有两个妹妹,菲利帕和玛丽,以及一个收养的哥哥,爱德华·弗兰克·大卫(1955年–2003年)。

1950年,当霍金的父亲成为国家医学研究所寄生虫学部的负责人时,霍金一家搬到了赫特福德郡的圣奥尔本斯。在圣奥尔本斯,霍金一家被认为是非常聪明且有些古怪;全家人常常在一起吃饭时各自安静地看书。他们生活朴素,住在一栋大而凌乱、维护不善的房子里,出行则开着一辆改装过的伦敦出租车。在霍金的父亲因工作常常前往非洲时,霍金一家曾在马约卡岛度过四个月,探访霍金母亲的朋友贝里尔和她的丈夫,诗人罗伯特·格雷夫斯。
\subsubsection{小学和中学时期}  
霍金在伦敦的拜伦之家学校开始了他的学业。他后来指责该校的“进步教育方法”使他在学校时未能学会阅读。在圣奥尔本斯,年仅8岁的霍金曾在圣奥尔本斯女子中学就读了几个月。当时,年轻男孩可以进入某些楼宇上课。

霍金就读了两所私立学校,首先是拉德莱特学校,然后从1952年9月起,进入了圣奥尔本斯学校。霍金早早通过了11+考试,并提前一年进入圣奥尔本斯学校。霍金一家对教育非常重视。霍金的父亲希望他能进入威斯敏斯特学校,但13岁的霍金在考试当天生病,因此未能参加奖学金考试。由于没有奖学金的经济资助,霍金的家庭负担不起学费,因此他继续留在圣奥尔本斯学校。这个决定有一个积极的后果——霍金得以与一群喜欢玩桌游、制造烟花、制作模型飞机和船只的朋友们保持密切联系,还一起长时间讨论基督教和超感知知觉等话题。从1958年开始,在数学老师迪克兰·塔塔的帮助下,他们用时钟零件、旧电话交换机和其他回收零件一起造了一台计算机。

虽然在学校里他被称为“爱因斯坦”,但霍金最初并不是学术上非常成功。随着时间的推移,他开始展现出相当强的科学天赋,并在塔塔的启发下,决定在大学学习数学。霍金的父亲建议他学习医学,担心数学专业毕业生的工作机会较少。他还希望儿子能进入自己母校——牛津大学。由于当时牛津大学无法开设数学专业,霍金决定学习物理和化学。尽管校长建议他等到明年再考,霍金还是在1959年3月参加了考试并获得了奖学金。
\subsubsection{本科时期}
霍金于1959年10月,在17岁时开始了在牛津大学大学学院的大学教育。在前18个月里,他感到无聊且孤单——他认为学术工作“轻而易举”。他的物理学导师罗伯特·伯曼后来表示,“他只需要知道某件事是可以做到的,他就能做到,而不需要看别人是怎么做的。”在他的第二和第三学年里,情况发生了变化。根据伯曼的说法,霍金开始更加努力“融入集体”,并逐渐发展成一个受欢迎、活泼且机智的大学成员,喜欢古典音乐和科幻小说。这一转变的一部分原因是他决定加入大学的划船俱乐部——大学学院划船俱乐部,并成为一名舵手。在当时的划船教练看来,霍金培养了一个“冒险者”的形象,他经常让划船队在充满风险的航道上划行,导致船只损坏。霍金估计,在牛津大学的三年里,他大约学习了1,000个小时。这些不太令人印象深刻的学习习惯使得他参加期末考试时面临挑战,他决定只回答那些需要理论物理知识的问题,而不回答那些需要记忆事实的问题。获得一等学位是他计划继续在剑桥大学攻读宇宙学研究生的条件。

考试前一天晚上,霍金焦虑不安,导致睡眠不好,考试结果处于一等和二等荣誉学位的边界,这使得他需要接受牛津大学考官的口试(viva)。霍金担心自己被认为是一个懒散且难以相处的学生。因此,在口试中当被问到自己的计划时,他回答道:“如果你给我一等学位,我就去剑桥。如果我得到二等学位,我就留在牛津,所以我希望你们会给我一等学位。”他比自己认为的更受人尊敬;正如伯曼所评论的,考官“足够聪明,意识到他们正在与一个比大多数人都聪明的人交谈”。在获得物理学一等学位后,并与朋友一起完成了去伊朗的旅行,霍金于1962年10月开始了在剑桥大学三一学院的研究生学习。
\subsubsection{研究生时期}
霍金的第一年博士生涯非常困难。最初,他感到失望,因为他被分配给了现代宇宙学的奠基人之一丹尼斯·威廉·赛阿玛作为导师,而不是著名的天文学家弗雷德·霍伊尔。霍金还发现,他在数学方面的训练不足以应对广义相对论和宇宙学的研究工作。在被诊断出患有运动神经元病后,霍金陷入了抑郁症——尽管医生建议他继续学业,但他觉得这样做没有什么意义。他的病情进展比医生预期的要慢。尽管霍金无法独立行走,且他的言语几乎无法理解,但最初医生预测他只剩下两年寿命的诊断并未成真。在赛阿玛的鼓励下,霍金重新投入到他的研究中。

霍金在学术界开始以聪明和大胆著称,当他在1964年6月的一次讲座上公开挑战霍伊尔及其学生贾扬特·纳尔利卡的工作时,他的声誉得到了进一步的提升。

当霍金开始他的博士研究时,物理学界关于宇宙创生的主流理论——大爆炸理论与稳态理论——存在着激烈的争论。在罗杰·彭罗斯关于黑洞中心时空奇点定理的启发下,霍金将这种思维方式应用到整个宇宙上;并在1965年完成了以此为主题的博士论文。霍金的论文于1966年获得批准。此外,还有其他积极的发展:霍金获得了剑桥大学冈维尔与凯厄斯学院的研究奖学金;他于1966年3月获得应用数学与理论物理博士学位,专业为广义相对论和宇宙学;他的论文《奇点与时空几何》与彭罗斯的论文并列获得当年著名的亚当斯奖。
\subsubsection{职业生涯}  
\subsubsection{1966–1975}  
在与彭罗斯的合作中,霍金扩展了他博士论文中首次探讨的奇点定理的概念。这不仅包括奇点的存在,还包括宇宙可能从一个奇点开始的理论。他们的联合论文在1968年的引力研究基金会竞赛中获得了第二名。在1970年,他们发表了一个证明,表明如果宇宙遵循广义相对论,并符合亚历山大·弗里德曼提出的任何物理宇宙学模型,那么宇宙一定是从一个奇点开始的。1969年,霍金接受了一个特别创设的“科学卓越奖学金”,以便继续留在凯厄斯学院。

1970年,霍金提出了后来被称为黑洞动力学第二定律的理论,即黑洞的事件视界永远不会变小。与詹姆斯·M·巴丁和布兰登·卡特一起,他提出了黑洞力学的四条定律,并将其与热力学进行了类比。令霍金恼火的是,约翰·惠勒的研究生雅各布·贝肯斯坦进一步推进并最终正确地将热力学概念应用于黑洞,并提出了黑洞的熵理论。

在1970年代初,霍金与卡特、维尔纳·以色列和大卫·C·罗宾逊的合作强有力地支持了惠勒的“无毛定理”,该定理指出,不论黑洞的原始物质是什么,黑洞都可以完全由质量、电荷和旋转的属性来描述。他的论文《黑洞》在1971年1月赢得了引力研究基金会奖。霍金的第一本书《时空的大尺度结构》,由他与乔治·埃利斯共同编写,于1973年出版。

从1973年开始,霍金开始研究量子引力和量子力学。受到访问莫斯科并与雅科夫·博里索维奇·泽尔多维奇和阿列克谢·斯塔罗宾斯基的讨论启发,霍金的这一研究工作表明,根据不确定性原理,旋转黑洞会发射粒子。令霍金恼火的是,他多次检查的计算结果与他的第二定律相矛盾,这一定律声称黑洞永远不会变小,而支持了贝肯斯坦关于黑洞熵的推理。

从1974年开始,霍金的研究结果表明,黑洞会发射辐射,今天被称为霍金辐射,这种辐射可能会持续,直到黑洞耗尽能量并蒸发。最初,霍金辐射存在争议。但到1970年代末,随着进一步研究成果的发布,这一发现被广泛接受,成为理论物理学的一个重大突破。霍金于1974年当选为皇家学会会员(FRS),成为在霍金辐射宣布后最年轻的科学家之一。

1974年,霍金被任命为加州理工学院(Caltech)的谢尔曼·费尔柴尔德杰出访问教授。他与该校的朋友、教职工基普·索恩合作,并与他就天鹅座X-1是否为黑洞进行了一场科学赌注。这个赌注是一项“保险政策”,以防黑洞不存在的理论为真。霍金承认他在1990年输掉了这场赌局,这是他与索恩和其他人进行的多个赌局中的第一个。自那次访问以来,霍金几乎每年都与加州理工学院保持联系,每年都在那里待上一个月左右。
\subsubsection{1975–1990}
霍金于1975年回到剑桥,担任引力物理学讲师,职位比之前更加学术化。1970年代中后期,黑洞和研究黑洞的物理学家受到了公众越来越多的关注。霍金经常接受报纸和电视的采访。他的工作也得到了越来越多的学术认可。1975年,他获得了爱丁顿奖章和庇护十一世金奖,1976年则获得了丹尼·海因曼奖、麦克斯韦奖和休斯奖。1977年,他被任命为引力物理学教授。次年,他获得了阿尔伯特·爱因斯坦奖章和牛津大学的荣誉博士学位。

1979年,霍金被选为剑桥大学卢卡斯数学教授。此职位的首次讲座题为《理论物理的终结是否即将来临?》,并提出了N = 8超引力作为解决物理学家研究的许多突出问题的领先理论。他的晋升恰逢健康危机,他因此不得不勉强接受一些家庭护理服务。与此同时,他的物理学研究方法也发生了变化,变得更加直觉和推测,而不再坚持数学证明。他曾告诉基普·索恩:“我宁愿是对的,也不一定要严格。”
1981年,他提出黑洞中的信息在黑洞蒸发时是不可逆丧失的。这一信息悖论违反了量子力学的基本原则,引发了多年的辩论,包括与伦纳德·萨斯金德和杰拉德·霍夫特之间的“黑洞战争”。
\begin{figure}[ht]
\centering
\includegraphics[width=6cm]{./figures/91894cfe613aa7f4.png}
\caption{霍金在1980年代参加的旧金山ALS大会} \label{fig_HJ_2}
\end{figure}
宇宙膨胀理论提出,在大爆炸之后,宇宙最初迅速膨胀,随后才进入较慢的膨胀阶段。该理论由艾伦·古斯(Alan Guth)提出,并由安德烈·林德(Andrei Linde)进一步发展。1981年10月,霍金与加里·吉本斯(Gary Gibbons)在莫斯科的一次会议后,组织了1982年夏季在剑桥大学举行的为期三周的“极早期宇宙”诺菲尔德研讨会,会议主要集中于宇宙膨胀理论的研究。霍金还开始了关于宇宙起源的新一线量子理论研究。1981年,在梵蒂冈的一次会议上,他提出了一项研究,暗示宇宙可能没有边界——即没有起始和终结。

霍金随后与吉姆·哈特尔(Jim Hartle)合作,发展了这一研究,并在1983年共同发表了一种模型,称为哈特尔-霍金状态。该模型提出,在普朗克时期之前,宇宙在时空中没有边界;在大爆炸之前,时间并不存在,因此宇宙的起始概念是没有意义的。经典大爆炸模型中的初始奇点被一个类似北极的区域取代。正如北极是所有指向北方的线汇聚并结束的地方,那里并没有实际的“边界”。最初,这一无边界假设预测了一个封闭的宇宙,这也对上帝存在的命题产生了影响。霍金解释道:“如果宇宙没有边界而是自成一体……那么上帝就不会有自由选择宇宙如何开始的权力。”

霍金并没有排除创造者的存在,在《时间简史》一书中,他问道:“统一理论是否如此具有吸引力,以至于它能带来自身的存在?”他还说:“如果我们发现了完整的理论,那将是人类理性的终极胜利——因为那时我们将知道上帝的心意”;在他早期的工作中,霍金是以比喻的方式谈论上帝的。在同一本书中,他还指出,上帝的存在并不是解释宇宙起源所必需的。与尼尔·图罗克的进一步讨论使霍金意识到,上帝的存在与开放宇宙的理论也是相容的。

在霍金关于时间箭头的进一步研究中,1985年他发表了一篇论文,提出如果无边界命题正确,那么当宇宙停止膨胀并最终坍缩时,时间将倒流。这一理论在唐·佩奇(Don Page)的论文和雷蒙德·拉弗拉姆(Raymond Laflamme)的独立计算之后,霍金撤回了这一概念。

荣誉继续颁发:1981年,他获得了美国富兰克林奖章;1982年新年荣誉中,他被任命为大英帝国指挥官勋章(CBE)获得者。这些奖项并未显著改变霍金的经济状况,出于支付孩子教育费用和家庭开销的需要,他在1982年决定撰写一本通俗的宇宙学书籍,以便让大众理解。霍金没有选择学术出版社,而是与大众市场出版社班坦书籍签订了合同,并为书籍获得了大额预付款。《时间简史》一书的初稿于1984年完成。

霍金使用语音生成设备发出的第一个信息是请求助手帮助他完成《时间简史》的写作。班坦的编辑彼得·古萨尔迪(Peter Guzzardi)推动霍金用非技术性语言清晰地阐述他的思想,这个过程让霍金感到越来越恼火,需要进行多次修订。该书于1988年4月在美国出版,6月在英国出版,迅速成为畅销书,并在两国的畅销书榜单上停留了几个月。该书已被翻译成多种语言,截至2009年,销量估计达到900万本。

媒体的关注非常强烈,新闻周刊封面和一档电视特别节目都称霍金为“宇宙的主人”。这份成功为霍金带来了可观的经济回报,但也让他面临了名人身份的挑战。霍金广泛地旅行以推广他的工作,并享受深夜派对的乐趣。由于难以拒绝邀请和来访者,他的工作时间和对学生的关注受到限制。一些同事对霍金所获得的关注感到不满,认为这是因为他的残疾。

霍金获得了更多的学术认可,包括五个荣誉学位、1985年的皇家天文学会金奖、1987年的保罗·狄拉克奖章,以及与彭罗斯共同获得的1988年沃尔夫奖。在1989年的生日荣誉中,他被任命为荣誉伴侣勋章(CH)成员。据报道,他在1990年代末拒绝了骑士封号,原因是反对英国的科学资助政策。
\subsubsection{1990–2000}
\begin{figure}[ht]
\centering
\includegraphics[width=6cm]{./figures/5a5fafc32d5b0975.png}
\caption{霍金与弦理论学者大卫·格罗斯和爱德华·维滕一起出席了2001年1月在印度塔塔研究所(TIFR)举行的弦论会议。} \label{fig_HJ_3}
\end{figure}
霍金继续从事物理学研究:1993年,他与加里·吉本斯共同编辑了一本关于欧几里得量子引力的书,并出版了自己关于黑洞和大爆炸的文章集。[168] 1994年,在剑桥的牛顿研究所,霍金与彭罗斯共同讲授了六场讲座,后来这些讲座于1996年出版,名为《时空的本质与时间》。[169] 1997年,他承认与加州理工学院的基普·索恩和约翰·普雷斯基尔在1991年进行的科学公开赌注失败。霍金曾打赌,彭罗斯提出的“宇宙审查猜想”——即“裸奇点”不能出现在视界之外——是正确的。[170]

在发现自己可能过早认输之后,霍金提出了一个新的、更精细的赌注。这个赌注明确指出,这样的奇点将出现,且没有额外条件。[171] 同年,霍金、索恩和普雷斯基尔进行了另一个赌注,这次是关于黑洞信息悖论。[172][173] 索恩和霍金认为,由于广义相对论使黑洞无法辐射并失去信息,因此霍金辐射所携带的质量能量和信息必须是“新的”,而不是来自黑洞事件视界内的内容。由于这与微因果律下的量子力学相矛盾,因此量子力学理论需要重新编写。普雷斯基尔则持相反意见,他认为,既然量子力学表明黑洞发出的信息与较早时坠入黑洞的信息有关,那么广义相对论中给出的黑洞概念必定需要某种方式的修正。[174]

霍金还保持着他的公众形象,包括将科学普及到更广泛的观众。1992年,基于《时间简史》的电影版由埃罗尔·莫里斯执导、史蒂文·斯皮尔伯格制作首映。霍金原本希望电影偏重科学内容而非传记,但最终被说服改变了主意。虽然这部电影获得了评论界的好评,但并未广泛上映。[175] 1993年,霍金出版了一本集合了随笔、访谈和演讲的流行读物《黑洞与婴儿宇宙及其他随笔》[176],而1997年,则推出了六集的电视系列《斯蒂芬·霍金的宇宙》和一本配套书。在这次创作中,霍金坚持完全聚焦于科学。[177][178]
\subsubsection{2000–2018}
\begin{figure}[ht]
\centering
\includegraphics[width=6cm]{./figures/b096806f8ffd7756.png}
\caption{霍金在2006年5月5日,法国巴黎的法国国家图书馆(Bibliothèque nationale de France)为天文学与粒子实验室的揭幕仪式上演讲,并发布了他的一部作品《上帝创造了整数》在法国的版本。} \label{fig_HJ_4}
\end{figure}
霍金继续为大众撰写作品,2001年出版了《宇宙简史》(The Universe in a Nutshell),并于2005年与伦纳德·穆洛迪诺夫共同编写了《时间简史的简明版》(A Briefer History of Time),以更新他早期的著作,旨在使其更易为广泛读者所理解。2006年,他还出版了《上帝创造了整数》(God Created the Integers)一书。[180] 从2006年开始,霍金与CERN的托马斯·赫尔托格(Thomas Hertog)和吉姆·哈特尔(Jim Hartle)合作,发展了一种自上而下的宇宙学理论,认为宇宙并非从唯一的初始状态开始,而是从多种不同的初始状态开始。因此,基于单一初始状态来预测宇宙当前状态的理论是不可行的。[181] 自上而下的宇宙学认为,当前“选择”了从多种可能的历史中形成的过去。该理论为解决精细调节问题提供了可能的解释。[182][183]

霍金继续广泛旅行,包括前往智利、复活节岛、南非、西班牙(2008年获得丰塞卡奖)、[184][185] 加拿大,[186] 以及多次前往美国。[187] 由于与残疾相关的实际原因,霍金逐渐开始乘坐私人飞机旅行,到2011年,私人飞机成为了他唯一的国际旅行方式。[188]

到2003年,物理学界的共识逐渐形成,认为霍金关于黑洞信息丧失的观点是错误的。[189] 在2004年都柏林的讲座中,他承认了1997年与普雷斯基尔的赌约,但描述了自己对于信息悖论问题的解决方案,这一方案涉及到黑洞可能具有多重拓扑的可能性。[190][174] 在2005年发表的相关论文中,他提出通过考察宇宙的所有替代历史,信息悖论得到了解释,黑洞中信息丧失的情况被没有信息丧失的宇宙所抵消。[173][191] 2014年1月,他称黑洞中信息丧失是他“最大的错误”。[192]

在另一场长期的科学争论中,霍金曾坚决主张并下注,认为永远不会找到希格斯玻色子。[193] 该粒子作为希格斯场理论的一部分由彼得·希格斯(Peter Higgs)于1964年提出。霍金和希格斯于2002年和2008年就此问题展开了激烈的公开辩论,希格斯批评霍金的工作,并抱怨霍金的“名人身份使他获得了他人没有的即时信誉”。[194] 该粒子于2012年7月在CERN的大型强子对撞机建成后被发现。霍金迅速承认自己输了赌注[195][196],并表示希格斯应该获得诺贝尔物理学奖,[197] 他最终于2013年获得该奖项。[198]

2007年,霍金与他的女儿露西(Lucy)共同出版了《乔治的宇宙秘密钥匙》(George's Secret Key to the Universe),这本儿童书籍旨在以易于理解的方式解释理论物理,并且其中的角色与霍金家庭成员相似。[199] 此书的续集分别于2009年、2011年、2014年和2016年出版。[200]

2002年,在一次全英国的投票中,BBC将霍金列入“百大伟大英国人”名单。[201] 他获得了皇家学会的科普利奖章(2006年),[202] 美国总统自由勋章(2009年),[203] 以及俄罗斯特殊基础物理学奖(2013年)。[204]
\begin{figure}[ht]
\centering
\includegraphics[width=6cm]{./figures/2b1ee2f886292f03.png}
\caption{霍金在2008年9月揭幕科尔普斯时钟(Corpus Clock)时的照片。} \label{fig_HJ_5}
\end{figure}
许多建筑物以霍金的名字命名,包括位于萨尔瓦多圣萨尔瓦多的霍金科学博物馆(Stephen W. Hawking Science Museum)、剑桥的霍金大楼(Stephen Hawking Building)以及加拿大的佩里米特研究所霍金中心(Stephen Hawking Centre)[205][206][207]。恰如其分的是,鉴于霍金与时间的关系,他于2008年9月在剑桥的科尔普斯基督学院揭幕了机械“时间吞噬者”科尔普斯时钟(Corpus Clock)[208][209]。

在他的职业生涯中,霍金指导了39名成功的博士生[1]。其中有一位博士生未能顺利完成博士学位[1][需要更好的来源]。根据剑桥大学的规定,霍金于2009年退休,结束了他的卢卡斯数学教授职务[122][210]。尽管曾有提议认为他可能会因对基础科学研究公共资金削减的不满而离开英国[211],霍金仍继续在剑桥大学应用数学与理论物理系担任研究主任[212]。

2009年6月28日,作为对他1992年提出的“时光旅行到过去几乎不可能”的假设的一个幽默验证,霍金举办了一场开放的派对,配有开胃小菜和冰镇香槟,但他只在派对结束后才公开宣传该事件,以确保只有时间旅行者会知道并前来参加;如预期的那样,没人出席派对[213]。

\begin{figure}[ht]
\centering
\includegraphics[width=6cm]{./figures/6e36d49259884d6c.png}
\caption{2015年8月24日,霍金在斯德哥尔摩水滨会议中心举办公开讲座。} \label{fig_HJ_6}
\end{figure}
2015年7月20日,霍金帮助发起了突破计划(Breakthrough Initiatives),这是一个寻找外星生命的努力。[214] 霍金制作了《霍金:新地球探险》(Stephen Hawking: Expedition New Earth),一部关于太空殖民的纪录片,作为《明日世界》2017年的一集播出。[215][216]

2015年8月,霍金表示,当某物进入黑洞时,并非所有信息都丧失,根据他的理论,可能存在从黑洞中恢复信息的可能性。[217] 2017年7月,霍金获得了伦敦帝国学院的荣誉博士学位。[218]

霍金的最后一篇论文——《从永恒膨胀中平滑退出?》(A smooth exit from eternal inflation?)——于2018年4月27日后发布在《高能物理学杂志》上。[219][220]
\subsection{个人生活}
\subsubsection{婚姻}
霍金在1962年于一个派对上遇见了未来的妻子简·怀尔德。次年,霍金被诊断出患有运动神经元病。1964年10月,这对情侣订婚,意识到由于霍金的生命预期较短以及身体上的限制,未来会面临很多挑战。[121][221] 霍金后来表示,订婚让他“有了活下去的理由”。[222] 他们于1965年7月14日在他们共同的故乡圣奥尔本斯结婚。[82]

这对夫妇定居在剑桥,住在霍金能够步行到应用数学与理论物理系(DAMTP)的地方。在婚后的头几年,简每周住在伦敦,完成她在西菲尔德学院的学位课程。他们多次前往美国参加会议和与物理相关的访问。简开始攻读西菲尔德学院的博士课程,研究中世纪西班牙诗歌(并于1981年完成)。这对夫妻有三个孩子:罗伯特,1967年5月出生,[223][224]露西,1970年11月出生,[225]以及蒂莫西,1979年4月出生。[117]

霍金很少讨论自己的疾病和身体挑战——甚至在他们交往期间,也不与简谈论此事。[226] 由于霍金的残疾,家务和家庭责任完全由妻子承担,这使得他有更多时间思考物理学。[227] 当霍金在1974年被任命为加州理工学院的客座教授时,简提议让一名研究生或博士后与他们同住并帮助照顾霍金。霍金同意了,伯纳德·卡尔成为了第一个承担这个角色的学生,和他们一起生活。[228][229] 这家人度过了一个通常很愉快且富有启发性的一年。

1975年,霍金回到剑桥,搬入新家并开始新工作,担任讲师。霍金与在加州理工学院建立了深厚友谊的唐·佩奇一起工作,佩奇成了住在霍金家的研究生助理。在佩奇和一名秘书的帮助下,简的家庭责任有所减少,她可以回到自己的博士论文和新爱好——唱歌上。[231]

1977年12月左右,简在教堂合唱团唱歌时认识了风琴师乔纳森·赫利尔·琼斯。赫利尔·琼斯与霍金一家人建立了亲密关系,到1980年代中期,他和简之间发展出了浪漫的感情。[120][232][233] 据简回忆,霍金接受了这种情况,并表示“只要我继续爱他,他不会反对”。[120][234][235] 简和赫利尔·琼斯决定不破坏家庭关系,他们的关系保持了长时间的柏拉图式友谊。[236]

到1980年代,霍金的婚姻已经维持了多年紧张。简感到被照顾霍金所需的护士和助手们的干预所压倒。[237] 霍金的名人身份对同事和家人来说是一种挑战,而过着全球童话般的生活对这对夫妻来说也是一种沉重负担。[238][182] 霍金的宗教观念与简强烈的基督教信仰存在差异,也加剧了夫妻间的紧张关系。[182][239][240] 1985年,霍金进行气管切开手术后,需要一名全职护士,护理工作分为三班进行。1980年代末期,霍金与其中一名护士伊莲娜·梅森变得亲密,令一些同事、护理人员和家人感到不安,他们对梅森的强势个性和保护性过强感到困扰。[241] 1990年2月,霍金告诉简,他要与梅森分开并离开了家庭。[144] 1995年,霍金与简离婚后,于9月与梅森结婚,[144][243] 并表示:“太棒了——我娶了我爱的人。”[244]

1999年,简·霍金出版了回忆录《音乐与星辰的舞动》,描述了她与霍金的婚姻以及其破裂。书中的内容引起了媒体的轰动,但如同他处理个人生活的惯例一样,霍金对此未做公开评论,只是表示自己不阅读关于自己的传记。[245] 在第二次婚姻后,霍金的家人感到被排除在外,与他的关系疏远。[240] 在2000年代初期的约五年里,霍金的家人和工作人员越来越担心他遭受身体虐待。[246] 警方进行了调查,但由于霍金拒绝提出投诉,调查被关闭。[247]

2006年,霍金与梅森悄悄离婚,[248][249] 随后霍金与简、孩子们和孙子们恢复了更亲密的关系。[182][249] 回顾这一段更加幸福的时光,简的书籍经过修订,重新命名为《走向无尽:我与霍金的生活》,并于2007年出版,[247] 并于2014年被改编成电影《万物理论》上映。[250]
\subsubsection{残疾}
霍金患有一种罕见的早发、进展缓慢的运动神经元病(MND;也称为肌萎缩侧索硬化症(ALS)或卢·盖里希病),这是一种致命的神经退行性疾病,影响大脑和脊髓中的运动神经元,逐渐使他在数十年内瘫痪。[21]

霍金在牛津的最后一年曾经历过日益加剧的笨拙感,包括在楼梯上摔倒和划船时的困难。[251][252] 问题逐渐加重,他的言语开始变得有些含糊。家人在他回家过圣诞节时注意到了这些变化,并开始进行医学检查。[253][254] 1963年,当霍金21岁时,医生诊断他患上了运动神经元病。当时,医生给出的预期寿命为两年。[255][256]

到1960年代末,霍金的身体能力逐渐下降:他开始使用拐杖,且无法定期讲课。[257] 随着他逐渐失去书写能力,他发展出了补偿性的视觉方法,包括将方程式以几何图形的方式来理解。[258][259] 物理学家沃纳·以色列后来将霍金的成就比作莫扎特在头脑中作曲一整部交响曲。[260][261] 霍金极度独立,不愿接受帮助或为自己的残疾做任何妥协。他希望被视为“首先是一个科学家,其次是一个科普作家,并且,在所有重要的方面,像下一个人一样,是一个拥有相同欲望、驱动力、梦想和抱负的普通人”。[262] 他的妻子简后来指出:“有人会称之为决心,也有人说是固执。我有时两者都称过。”[263] 霍金在1960年代末经过多次劝说后才接受使用轮椅,[264] 但最终他因驾驶轮椅时的野性驾驶风格而声名狼藉。[265] 霍金是一个受欢迎且机智的同事,但他的疾病,以及他直率的名声,让他与一些人保持了距离。[263]

霍金开始使用轮椅时,使用的是标准的电动轮椅模型。这些轮椅中,最早的一个存世例子由BEC Mobility制造,并在2018年11月通过佳士得拍卖,以296,750英镑的价格售出。[266] 霍金一直使用这种类型的轮椅直到1990年代初,当时他的手部能力逐渐丧失,无法再操作轮椅。此后,霍金使用了各种不同的轮椅,包括2007年的DragonMobility Dragon升降电动轮椅,照片中霍金出席美国宇航局50周年庆典时所用的就是这种轮椅;[267] 2014年的Permobil C350;以及2016年的Permobil F3。[268]

霍金的言语能力逐渐退化,到1970年代末,他的言语只有家人和最亲密的朋友能够理解。为了与他人沟通,熟悉他的人会将他的言语翻译成可理解的语音。[269] 在与大学就谁支付他进出工作场所所需的坡道问题发生争执后,霍金和他的妻子开始为剑桥的残疾人士争取更好的无障碍设施和支持,[270][271] 包括大学的适应性学生宿舍。[272] 总体而言,霍金对自己作为残疾权利倡导者的角色感到矛盾:虽然他希望帮助他人,但他也希望将自己与疾病及其挑战区分开。[273] 他在这一领域的缺乏参与曾遭到一些批评。[274]

1985年中,霍金在访问位于法国和瑞士边界的欧洲核子研究中心(CERN)时感染了肺炎,而在他的病情下,这次感染是致命的;他病重到甚至被问及是否要终止生命支持。简拒绝了这一提议,但结果是必须进行气管切开手术,这要求24小时护理,并导致他失去了剩余的言语能力。[275][276] 英国国民健康服务系统准备为他支付养老院费用,但简坚决决定让他留在家中。护理费用由一家美国基金会资助。[277][278] 为了提供全天候支持,雇佣了三班轮流工作的护士。其中一位护士是伊莲娜·梅森,她后来成为霍金的第二任妻子。[279]

为了进行沟通,霍金最初通过抬眉选择拼字卡上的字母,[280] 但1986年,他收到了来自Words Plus公司首席执行官沃尔特·沃尔托斯的“平衡器”计算机程序,这个程序是沃尔托斯为帮助他的岳母开发的,她同样患有ALS,失去了说话和写作的能力。[281] 通过这种方法,霍金现在只需按一个开关,就可以从一个大约2,500到3,000个单词的库中选择短语、单词或字母。[282][283] 该程序最初运行在桌面计算机上。伊莲娜·梅森的丈夫大卫是计算机工程师,他将一台小型计算机改装并固定在霍金的轮椅上。[284]

霍金通过不再需要他人解读其言语,评论道:“现在我能比失去声音之前更好地沟通了。”[285] 他使用的语音有美国口音,但不再生产这种声音。[286][287] 尽管后来有其他声音选项,霍金坚持使用这个原始语音,表示他更喜欢这个声音并与其产生了认同感。[288] 起初,霍金通过手部按开关来激活语音生成,每分钟可以说出多达15个字。[152] 讲座通常预先准备,并以短节段的形式发送到语音合成器中进行播放。[286]

随着时间推移,霍金逐渐失去了手部的使用能力,并在2005年开始通过面部肌肉的动作来控制他的沟通设备,[289][290][291] 每分钟大约可以打出一个单词。[290] 随着这一能力的下降,他面临着发展为“锁定综合症”的风险,于是霍金与英特尔公司合作,研究能够将他的脑波或面部表情转化为开关激活信号的系统。在多个原型未能如预期那样工作后,他们最终选择了伦敦初创公司SwiftKey开发的自适应词预测系统,这个系统与他原有的技术相似。霍金适应新系统的过程较为顺利,且该系统经过进一步开发,输入了大量霍金的论文和其他书面材料,使用了类似于智能手机键盘的预测软件。[182][281][291][292]

到2009年,他已经无法独立驾驶轮椅,但为他的轮椅新增的打字机制的开发者正在尝试使用霍金的下巴动作来控制轮椅的移动。这个方法 proved to be difficult, as霍金无法移动颈部,试验表明,尽管他确实能够控制轮椅,但移动是间歇性的、跳跃的。[281][293] 在他生命的最后阶段,霍金经历了越来越严重的呼吸困难,常常需要使用呼吸机,并且需要频繁住院。[182]
\subsubsection{残疾外展活动}
从1990年代开始,霍金接受了作为残疾人榜样的角色,进行讲座并参与筹款活动。[294] 到21世纪之交,他与其他11位人道主义者一起签署了《千年残疾宪章》,呼吁各国政府防止残疾并保护残疾人的权利。[295][296] 1999年,霍金获得了美国物理学会的朱利叶斯·埃德加·利连费尔德奖。[297]

2012年8月,霍金为2012年伦敦残奥会开幕式中的“启蒙”环节进行解说。[298] 2013年,霍金亲自出演的传记纪录片《霍金》上映。[299] 2013年9月,他表达了支持为末期病人合法化安乐死的观点。[300] 2014年8月,霍金接受了冰桶挑战,以提高公众对ALS/MND的认识并为研究筹集资金。由于他在2013年患有肺炎,医生建议不要让冰水浇在他身上,但他的孩子们志愿代替他接受挑战。[301]
\subsubsection{太空旅行计划}
2006年底,霍金在接受BBC采访时透露,他最大的未实现愿望之一就是去太空旅行。[302] 听到这一点后,理查德·布兰森向他提供了由维珍银河公司免费提供的太空旅行,霍金立即接受了这个提议。除了个人的雄心壮志,他还希望通过此举激发公众对太空飞行的兴趣,并展示残疾人也能有巨大潜力。[303] 2007年4月26日,霍金乘坐一架特别改装的波音727-200喷气机,随Zero-G公司在佛罗里达海岸进行飞行,以体验失重感。[304] 起初担心这些飞行动作会给他带来不适的担忧被证明是错误的,飞行时间也延长至八个抛物线轨迹。[302] 这次飞行被描述为一次成功的测试,旨在看看他是否能承受太空飞行中涉及的重力加速度。[305] 当时,霍金的太空之旅预计最早在2009年进行,但商业太空航班在他去世前并未开始。[306]
\subsection{去世}
霍金于2018年3月14日在剑桥的家中去世,享年76岁。[307][308][309] 他的家人表示他“安详去世”。[310][311] 他在科学、娱乐、政治及其他领域的人物中得到了悼念。[312][313][314][315] 剑桥的贡维尔与凯斯学院(Gonville and Caius College)降半旗致敬,学生和访客签署了悼念册。[316][317][318] 在2018年平昌冬季残奥会闭幕式上,国际残奥委员会主席安德鲁·帕森斯(Andrew Parsons)在闭幕辞中向霍金致敬。[319]

他的私人葬礼于2018年3月31日举行,地点是剑桥的圣玛丽大教堂(Great St Mary's Church)。[320][321] 葬礼的宾客包括《万物理论》中的演员埃迪·雷德梅恩(Eddie Redmayne)和费莉西蒂·琼斯(Felicity Jones)、皇后乐队吉他手兼天体物理学家布赖恩·梅(Brian May)、模特莉莉·科尔(Lily Cole)。[322][323] 此外,曾在《霍金传》中饰演霍金的演员本尼迪克特·康伯巴奇(Benedict Cumberbatch)、宇航员蒂姆·皮克(Tim Peake)、皇家天文学家马丁·里斯(Martin Rees)以及物理学家基普·索恩(Kip Thorne)也为葬礼朗读了致辞。[324] 虽然霍金是无神论者,但葬礼依然按照传统的英国圣公会仪式举行。[325][326] 火化后,霍金的骨灰于2018年6月15日安放在西敏寺的中殿,与艾萨克·牛顿爵士和查尔斯·达尔文的墓相邻。[16][322][327][328]