% 惯性约束核聚变
% license CCBYSA3
% type Wiki

(本文根据 CC-BY-SA 协议转载自原搜狗科学百科对英文维基百科的翻译)

\textbf{惯性约束核聚变(ICF)}是一种聚变能试图发起的研究核聚变通过加热和压缩燃料靶进行的反应,通常为通常含有以下物质的混合物的颗粒形式氘和氚。典型的燃料球大约有针头那么大,大约有10个毫克燃料。

为了压缩和加热燃料,能量使用高能激光束、电子或离子传递到靶的外层,尽管出于各种原因,几乎所有ICF装置截至2015年使用激光。受热的外层向外爆炸,对目标的其余部分产生反作用力,向内加速,压缩目标。该过程旨在产生穿过目标向内传播的冲击波。一组足够强大的冲击波可以压缩和加热中心的燃料,以至于发生聚变反应。

ICF是聚变能量研究的两个主要分支之一,另一个是磁约束聚变。当ICF在20世纪70年代初首次提出时,它似乎是一种实用的发电方法,并且该领域蓬勃发展。20世纪70年代和80年代的实验表明,这些装置的效率比预期的低得多,达到点火并不容易。在20世纪80年代和90年代,为了理解高强度激光和等离子体的复杂相互作用,进行了许多实验。这些导致了新机器的设计,更大,最终达到点火能量。

最大的可操作ICF实验是美国的国家点火装置(NIF),它是利用早期实验数十年的经验设计的。然而,像那些早期的实验一样,NIF没有达到点火,并且截至2015年,正在产生大约1⁄3所需的能量水平。[1]

\subsection{描述}
\subsubsection{1.1 基本融合}
\begin{figure}[ht]
\centering
\includegraphics[width=6cm]{./figures/2941b793a05597d5.png}
\caption{间接驱动激光器ICF使用黑腔室其内表面从任一侧用激光束锥照射,以用平滑的高强度X射线在内部浸泡融合微胶囊。可以看到最高能量的X射线通过环空器泄漏,这里用橙色/红色表示。} \label{fig_GXYS_1}
\end{figure}
聚变反应结合了较轻的原子,例如氢一起形成更大的。通常这些反应发生在如此高的温度下离子ed,他们的电子被高温剥去;因此,核聚变通常被描述为“原子核”,而不是“原子”。

原子核带正电,因此由于静电力相互排斥。克服这种排斥需要大量的能量,这被称为库仑障壁或者聚变势垒能量。一般来说,导致较轻的原子核融合需要较少的能量,因为它们的电荷较少,因此势垒能量较低,当它们融合时,会释放更多的能量。随着原子核质量的增加,有一个点是反应不再释放净能量——克服能量屏障所需的能量大于最终聚变反应释放的能量。

从能源角度来看,最好的燃料是氘和氚的一对一混合;两者都是氢的重同位素。D-T(氘和氚)混合物的屏障很低,因为它的中子和质子比率很高。原子核中中性中子的存在有助于通过核力将它们拉在一起,而带正电荷的质子的存在通过静电力将原子核推开。氚是任何稳定或中度不稳定核素中中子与质子比率最高的核素之一——两个中子和一个质子。添加质子或移除中子会增加能量屏障。

在标准条件下的D-T混合物不进行融合;在核力将原子核拉在一起形成稳定的集合之前,必须将它们压在一起。即使在太阳炎热、密集的中心,平均质子在融合之前也会存在数十亿年。[2]对于实际的聚变发电系统,必须通过将燃料加热到数千万度,和/或压缩到巨大的压力来大幅提高燃料利用率。任何特定燃料熔化所需的温度和压力称为劳森准则。自20世纪50年代第一颗氢弹诞生以来,这些条件就已经为人所知。在地球上遇到劳森判据是极其困难的,这也解释了为什么聚变研究花了许多年才达到目前的高技术水平。[3]
\subsubsection{1.2 ICF作用机制}
在氢弹中,聚变燃料用单独的裂变炸弹压缩和加热(参见Teller-Ulam设计)中。多种机制将裂变“初级”爆炸的能量转移到聚变燃料中。一个主要的机制是,初级粒子发出的x射线闪光被捕获在炸弹的工程外壳内,导致外壳和炸弹之间的体积充满x射线“气体”。这些x射线均匀地照射到聚变区的外部,即“次级”,迅速加热它,直到它向外爆炸。这种向外的吹出导致次级线圈的其余部分向内压缩,直到达到聚变反应开始的温度和密度。

裂变炸弹的要求使得这种方法不适合发电。这种引爆器不仅生产成本过高,而且制造这种炸弹的最小尺寸也是可以的,大致由所使用的钚燃料的临界质量来定义。一般来说,建造产量小于约1千吨的核装置似乎很困难,而聚变二次堆会增加产量。这使得从爆炸中提取能量成为一个困难的工程问题; PACER 项目研究了工程问题的解决方案,但也证明了成本在经济上是不可行的。

PACER的参与者之一,约翰·努科尔斯(John nuckols)开始探索随着次级粒子的尺寸减小,启动聚变反应所需的初级粒子的尺寸发生了什么变化。他发现,当次级粒子达到毫克大小时,激发它所需的能量落入兆焦范围。这远远低于原子弹的需求,因为原子弹的主爆炸源在万亿焦耳射程内,相当于大约6盎司的TNT炸药。

这在经济上是不可行的,这种设备的成本将超过它所生产的电能的价值。然而,有许多其他设备可能能够重复传递这种能量水平。这导致了使用一种设备将能量“束”到聚变燃料上,确保机械分离的想法。到20世纪60年代中期,激光器似乎将发展到可以获得所需能量水平的程度。

通常ICF系统使用单个激光器驾驶员其光束被分成许多光束,这些光束随后被分别放大一万亿倍或更多。这些被许多镜子送入反应室(称为目标室),镜子的位置是为了均匀地照射整个表面上的目标。驾驶员施加的热量导致目标的外层爆炸,就像氢弹燃料缸的外层被裂变装置的x光照射时一样。

材料从表面爆炸导致内部剩余的材料被巨大的力向内驱动,最终坍缩成一个微小的近球形球。在现代惯性约束聚变装置中,产生的燃料混合物的密度高达铅密度的100倍,约为1000克/厘米3。这个密度不足以独自创造任何有用的聚变速率。然而,在燃料崩溃的过程中,冲击波也形成并以高速进入燃料的中心。当他们遇到从燃料中心的另一侧进入的对手时,那个点的密度会提高很多。

给定正确的条件,在被冲击波高度压缩的区域中的融合率可以释放出大量高能的α粒子α。由于周围燃料的高密度,它们在被“热化”之前只移动很短的距离,将能量作为热量损失给燃料。这种额外的能量将在加热的燃料中引起额外的聚变反应,释放出更多的高能粒子。这个过程从中心向外扩展,导致一种自我维持的烧伤,称为点火。
\begin{figure}[ht]
\centering
\includegraphics[width=14.25cm]{./figures/f5ecab750c2b7c82.png}
\caption{激光惯性约束聚变阶段示意图。蓝色箭头代表辐射;橙色被吹走了;紫色是向内传输的热能。 Laser beams or laser-produced X-rays rapidly heat the surface of the fusion target, forming a surrounding plasma envelope. Fuel is compressed by the rocket-like blowoff of the hot surface material. During the final part of the capsule implosion, the fuel core reaches 20 times the density of lead and ignites at 100,000,000 ˚C. Thermonuclear burn spreads rapidly through the compressed fuel, yielding many times the input energy.} \label{fig_GXYS_2}
\end{figure}
\subsubsection{1.3 成功的问题}
自20世纪70年代早期实验以来,提高惯性约束聚变性能的主要问题是向靶输送能量、控制内爆燃料的对称性、防止燃料过早加热(在达到最大密度之前)、通过流体动力学不稳定性防止热燃料和冷燃料过早混合以及在压缩燃料中心形成“紧密”冲击波会聚。

为了将冲击波聚焦在目标的中心,必须使目标具有极高的精度和球形,其表面(内部和外部)的像差不超过几微米。同样,激光束的瞄准必须非常精确,并且光束必须同时到达目标上的所有点。然而,光束定时是一个相对简单的问题,并且通过在光束的光路中使用延迟线来实现皮秒级的定时精度来解决。困扰内爆靶实现高对称性和高温度/密度的另一个主要问题是所谓的“束-束”不平衡和束各向异性。这些问题分别是,由一个光束传递的能量可能高于或低于撞击目标的其他光束,以及光束直径撞击目标内的“热点”,这导致目标表面上的不均匀压缩,从而形成瑞利-泰勒不稳定性[4]在燃料中过早混合,并在最大压缩时降低加热效率。由于冲击波的形成,里克特迈耶-梅什科夫不稳定性也在此过程中形成。
\begin{figure}[ht]
\centering
\includegraphics[width=10cm]{./figures/5d114c9228a96cb3.png}
\caption{惯性约束聚变靶是一种填充泡沫的圆柱形靶,带有机械扰动,被新星激光压缩。这张照片拍摄于1995年。形象秀目标的压缩,以及瑞利-泰勒不稳定性的增长。[4]} \label{fig_GXYS_3}
\end{figure}
在过去二十年的研究中,通过使用各种光束平滑技术和光束能量诊断来平衡光束到光束的能量,所有这些问题都在不同程度上得到了缓解;然而,RT的不稳定性仍然是一个主要问题。多年来,目标设计也有了巨大的改进。现代的低温氢冰靶倾向于在用低功率 IR 激光照射时在塑料球的内部冻结一薄层氘,以使其内表面光滑,同时用配备有摄像机的显微镜对其进行监控,从而允许密切监控该层,确保其“光滑”。[5]充满氘氚(D-T)混合物的低温靶是“自平滑的”,因为放射性氚同位素衰变产生少量热量。这通常被称为“ beta -分层”。[6]
\begin{figure}[ht]
\centering
\includegraphics[width=6cm]{./figures/479154e2cb1cf81f.png}
\caption{镀金的国家点火装置 (NIF)环空器模型。} \label{fig_GXYS_4}
\end{figure}
某些目标被一个小金属圆柱体包围,用激光束照射,而不是目标本身,这种方法被称为“间接传动“。[7]在这种方法中,激光聚焦在圆柱体的内侧,将其加热到超热等离子体,该等离子体主要以 X射线辐射。然后,来自该等离子体的X射线被目标表面吸收,以与直接被激光击中相同的方式内爆。目标对热x射线的吸收比直接吸收激光更有效,然而环空器或者“燃烧室”本身也需要相当多的能量来加热,因此显著降低了激光到目标能量传递的整体效率。因此,即使在今天,它们也是一个有争议的特征;同样多的人”直接激励“设计不使用它们。大多数情况下,间接驱动环空器目标用于模拟热核试验,因为其中的聚变燃料也主要被X射线辐射内爆。
\begin{figure}[ht]
\centering
\includegraphics[width=6cm]{./figures/1db60d73ec8c6a27.png}
\caption{惯性约束融合燃料微胶囊(有时称为“微球”),其尺寸可用于NIF,可填充氘和氚气体或DT冰。胶囊可以插入环空器(如上)并在间接传动模式或直接用激光能量照射直接驱动配置。以前激光系统中使用的微胶囊明显更小,这是因为早期激光能够传递到目标的辐射功率较小。} \label{fig_GXYS_5}
\end{figure}
正在探索各种ICF驱动程序。自20世纪70年代以来,激光已经有了显著的进步,从少数几个激光器增加了能量和功率焦耳s和千瓦至兆焦耳(参见NIF激光)和数百太瓦,主要使用频率加倍或三倍的光从钕玻璃放大器。

重离子束对于商业化生产特别有趣,因为它们易于创建、控制和聚焦。另一方面,很难实现有效内爆目标所需的非常高的能量密度,大多数离子束系统需要使用围绕目标的环空器来平滑辐射,进一步降低了离子束能量与内爆目标能量耦合的整体效率。

\subsection{ 历史}
\subsubsection{2.1 第一个概念}

\textbf{在美国}

惯性约束聚变的历史可以追溯到1957年在日内瓦举行的“原子促进和平”会议。这是美国和俄罗斯超级大国之间由联合国主办的大型国际会议。在活动中涉及的许多主题中,有人考虑过使用氢弹来加热充满水的地下洞穴。由此产生的蒸汽将被用于驱动传统的发电机,从而提供电能。[8]

这次会议促成了“犁铧行动”的努力,并于1961年以这个名字命名。作为犁铧的一部分,研究了三个主要概念;在 PACER 项目下产生能量,使用大型核爆炸进行挖掘,并作为天然气工业的一种核压裂。PACER于1961年12月直接测试,当时3 kt Gnome 项目设备被安放在新墨西哥州的层状盐中。尽管有各种理论和尝试来阻止它,放射性蒸汽还是从离测试点有一段距离的钻杆中释放出来。作为PACER的一部分,进一步的研究导致了许多工程空洞取代天然空洞,但在此期间,整个犁铧的努力从糟糕变得更糟,特别是在1962年释放出大量沉降物的轿车失败后。尽管如此,PACER继续获得一些资金,直到1975年,一项第三方研究表明,PACER的电力成本将相当于燃料成本是其十倍以上的常规核电站。[9]

“原子促进和平”会议的另一个结果是促使约翰·努科尔斯开始考虑炸弹聚变方面发生的情况。当裂变炸弹爆炸时,它会释放X射线,从而使聚变侧内爆。这个“次级”被缩小到非常小的尺寸。他最早的工作是研究在仍然有很大的“增益”来提供净能量输出的情况下,聚变炸弹可以制造得多小。这项工作表明,在非常小的尺寸下,大约毫克,点燃它只需要很少的能量,远小于裂变的“初级”。[8]实际上,他提议使用悬浮在金属壳中心的一滴D-T燃料建造微型全聚变炸药,今天称之为环空器。外壳提供了与氢弹炸弹外壳相同的效果,将x射线捕获在里面,使它们照射燃料。主要区别在于x射线不会由外壳内的初级辐射源提供,而是由某种外部设备从外部加热外壳,直到它在x射线区域发光(参见热辐射)。能量将由当时未被识别的脉冲电源输送,他用炸弹术语称之为“主电源”。[10]

这种方案的主要优点是高密度下聚变过程的效率。根据劳森准则,在环境压力下将D-T燃料加热到收支平衡所需的能量可能是将其压缩到能够提供相同聚变速率的压力所需能量的100倍。因此,理论上,ICF方法在增益方面会显著更有效。[10]这可以通过考虑燃料缓慢加热的传统情况下的能量损失来理解,例如磁聚变能量;环境中的能量损失率是基于燃料与周围环境之间的温差,随着燃料的加热,温差会持续增加。在惯性约束聚变的情况下,整个环空器充满了高温辐射,限制了损失。[11]

\textbf{在德国}

大约在同一时间(1956年),核聚变先驱卡尔·冯·魏茨泽克在德国马克斯·普朗克研究所组织了一次会议。在这次会议上,弗里德沃德·温特尔贝格提出了通过高能炸药驱动的会聚冲击波实现热核微爆炸的非裂变点火。[12]前东德的一份解密报告进一步提到了温特尔贝格在德国进行的核爆炸研究斯塔西(Staatsicherheitsdienst)。[13]

1964年,温特尔贝格提出,点燃可以通过加速到1000的强烈微粒束来实现 km/s。[14]1968年,他提议使用马克思发生器产生的强烈电子束和离子束来达到同样的目的。[15]这种方案的优点是,带电粒子束的产生不仅比激光束的产生更便宜,而且由于强自磁场束场,还可以捕获带电聚变反应产物,从而大大降低了束点燃圆柱形靶的压缩要求。

\textbf{在苏联}

1967年,古尔根·阿斯卡莱恩研究员发表了一篇文章,提出在聚变中使用聚焦激光束氘化锂或氘。[16]
\subsubsection{2.2 早期研究}
到20世纪50年代末,劳伦斯利佛摩国家实验室大学的努科勒及其合作者对惯性约束聚变概念进行了多次计算机模拟。1960年初,这产生了1 密壳内的氘-氚燃料。模拟表明,输入环空器的5 MJ功率将产生50 MJ的聚变输出,增益为10。当时激光尚未发明,人们考虑了各种可能的驱动器,包括脉冲功率机、带电粒子加速器、等离子枪和超高速粒子枪。[17]

在这一年中,取得了两项重要的理论进展。新的模拟考虑了脉冲中传递能量的时间,称为“脉冲整形”,从而导致更好的内爆。此外,外壳变得更大更薄,形成了与几乎实心的球相反的薄外壳。这两个变化极大地提高了内爆的效率,从而大大降低了压缩它所需的能量。使用这些改进,计算出需要大约1 MJ的驱动器,五倍的改进。在接下来的两年里,提出了其他几个理论进展,特别是雷·基德美国发展了一种没有环空器的内爆系统,即所谓的“直接驱动”方法斯特灵高露洁罗恩·扎巴斯基研究的是只有1μg D-T燃料的非常小的系统。[18]

1960年激光的引入休斯研究实验室在加利福尼亚出现了一个完美的驱动机制。从1962年开始,利弗莫尔的导演小约翰·福斯特爱德华·泰勒开始了针对ICF方法的小规模激光研究。即使在这个早期阶段,ICF系统对于武器研究的适用性也是众所周知的,这也是它能够获得资金的主要原因。[19]在接下来的十年中,LLNL为基础激光等离子体相互作用研究制造了几个小型实验设备。
\subsubsection{2.3 发展开始了}
1967年基普·西格尔利用出售早期公司电导创股份的收益创办了KMS工业公司,电导创是全息术。20世纪70年代初,他成立了KMS聚变开始发展基于激光的惯性约束聚变系统。[20]这一发展引起了包括LLNL在内的武器实验室的强烈反对,他们提出了各种理由,认为不应允许KMS公开发展ICF。这种反对意见是通过原子能委员会传递的,该委员会要求为他们自己的努力提供资金。除了背景噪音之外,还有一个谣言,称苏联正在进行一项激进的ICF计划,即新的更高功率的CO2玻璃激光器、电子束驱动器概念和 20世纪70年代的能源危机为许多能源项目增加了动力。[19]

1972年,努科勒斯在自然介绍惯性约束聚变,并建议试验台系统可以与kJ范围内的驱动器产生聚变,以及与MJ驱动器产生高增益系统。[21][22]

尽管资源有限,业务问题众多,KMS聚变还是于1974年5月1日成功地展示了从ICF过程进行的聚变。[23]然而,在这一成功之后不久,西格尔就去世了,大约一年后,KMS融合结束了,公司按照西格尔的人寿保险政策经营。[20]这时,几个武器实验室和大学已经开始了他们自己的项目,特别是固体激光器s(Nd:玻璃激光器在LLNL和罗彻斯特大学,和氟化氪 受激准分子激光器s系统位于洛斯阿拉莫斯和海军研究实验室。

虽然KMS的成功导致了一项重大的发展努力,但随后的进展过去是,现在仍然是,聚变研究普遍存在的似乎难以解决的问题所阻碍的。
\subsubsection{2.4 高能惯性约束聚变}
高能ICF实验(每次数百焦耳和更大的实验)始于20世纪70年代初,当时首次设计了所需能量和功率的激光器。这是在成功设计磁约束聚变系统之后的一段时间,大约是在70年代早期引入的特别成功的托卡马克设计的时候。尽管如此,在20世纪70年代中后期多次能源危机刺激下,聚变研究获得了大量资金,性能迅速提高,惯性设计很快达到了最佳磁系统相同的“低于盈亏平衡”条件。

特别是LLNL,资金非常充足,并启动了一个主要的激光聚变发展计划。他们的贾纳斯激光器于1974年开始运行,并验证了使用钕玻璃激光器产生非常高功率器件的方法。在长光程激光器和独眼激光器中探索了聚焦问题,这导致了更大的阿格斯激光器。这些都不是为了成为实用的ICF设备,但每一种都将现有技术发展到了有信心基本方法是有效的程度。当时,人们认为制造一个更大的独眼巨人类型的设备可以压缩和加热ICF目标,导致在“短期”内点火。这是一个基于利用所谓的“爆炸推进器”型燃料胶囊的实验中看到的聚变产量外推的误解。在70年代末80年代初,随着各种等离子体不稳定性和激光-等离子体能量耦合损耗模式逐渐被理解,对实现点火所需的靶上激光能量的估计几乎每年翻一番。认识到简单的爆炸推进器靶设计和很少的千焦(kJ)激光辐射强度永远不会扩展到高增益聚变产量,导致了将紫外激光能量增加到100 kJ水平的努力,以及先进烧蚀器和低温DT冰靶设计的生产。
\subsubsection{2.5 湿婆和新星}
ICF驱动程序设计的最早的严肃和大规模尝试之一是\textbf{湿婆激光器},一个20束掺钕玻璃激光系统,建立于1978年开始运行的LLNL。Shiva是一种“概念验证”设计,旨在证明聚变燃料胶囊的压缩程度是氢的液体密度的许多倍。在这种情况下,湿婆成功地将其颗粒压缩到氘液体密度的100倍。然而,由于激光与热电子的强耦合,致密等离子体(离子)的过早加热是有问题的,并且熔化产率低。Shiva未能有效加热压缩的等离子体,这表明使用光学倍频器作为解决方案,在351处将来自激光器的红外光频率提高三倍,进入紫外线 nm。1980年在激光能量学实验室发现的新发现的高效频率三倍高强度激光的方案使得这种目标照射方法能够在24束ω激光器和 NOVETT激光器中进行实验,随后是能量是Shiva的10倍的 Nova激光器设计,第一个设计的具体目标是达到点火条件。

Nova也未能实现点火的目标,这一次是因为成丝导致其光束中激光强度的严重变化(以及光束之间的强度差异),这导致目标处辐射平滑度的巨大不均匀性和不对称内爆。早期开创的技术无法解决这些新问题。但这一失败再次导致了对内爆过程的更好理解,前进的道路似乎再次明确,即增加照射的均匀性,通过光束平滑技术减少激光束中的热点,以减少目标上的瑞利-泰勒不稳定性印迹,并将目标上的激光能量至少增加一个数量级。融合研究的资金在80年代受到严重限制,但Nova仍然成功地为下一代机器收集了足够的信息。
\subsubsection{2.6 国家点火装置}
由此产生的设计,现在被称为国家点火装置,于1997年开始在LLNL建造。NIF的主要目标是作为所谓的核管理计划的旗舰实验装置,支持LLNLs传统的炸弹制造角色。于2009年3月完成,[24]NIF现在已经用全部192束激光进行了实验,包括创造激光能量传输新记录的实验。[25][26]首次可信的点火尝试最初计划在2010年进行,[来源请求]但截至2012年9月30日,点火尚未实现。[27]截至2013年10月7日,该设施被认为已经实现了聚变商业化的一个重要里程碑,即燃料胶囊首次释放出比应用于它的更多的能量。[28]这离满足劳森标准还有很长的路要走,但这是向前迈出的一大步。[29]