% 凸集的可分离性
% keys 凸集|分离性
% license Usr
% type Tutor

\pentry{凸集和凸体\nref{nod_ConSet},泛函与线性泛函\nref{nod_Funal}}{nod_959a}

线性空间中集 $M,N$ 的分离性是指存在一个\enref{线性泛函}{Funal} $f$,使得泛函在这两个集上的值由一个常数区分开。凸集的分离性是指线性空间中两个凸集,若其中之一的\enref{核}{ConSet}非空且不与另一集相交,则必存在非零线性泛函将这两个凸集分离。

\begin{definition}{分离性}
设 $M,N$ 是实线性空间的两个子集,$f:L\rightarrow\mathbb R$ 是线性泛函,若存在常数 $C$ 使得任意 
\begin{equation}
f(x)\left\{\begin{aligned}\leq C,\quad x\in N\\
\geq C,\quad x\in M
\end{aligned}\right.~
\end{equation}
则称 $f$ \textbf{分离}集 $M,N$。
即指
\begin{equation}
\inf_{x\in M}f(x)\geq\sup_{x\in N}f(x).~
\end{equation}
当不等式中的等号不成立时,称为\textbf{严格分离}。
\end{definition}
\subsection{性质}
\begin{theorem}{}
设 $f$ 是线性空间 $L$ 上的线性泛函,$M,N\subset L$,则以下3个命题等价:
\begin{enumerate}
\item $f$ 分离 $M,N$;
\item $f$ 分离 $M-N$ 与 $\{0\}$;
\item 任意 $x\in L$,$f$ 分离 $M-x$ 和 $N-x$。
\end{enumerate}
\end{theorem}
\textbf{证明:}
这由下式直接看出:
\begin{equation}
\begin{aligned}
f(M)\geq C&\geq f(N)\\
&\Updownarrow \\
f(M-N)&\geq0\\
&\Updownarrow \\
f(M-x-(N-x))&=f(M-x)-f(N-x)\geq0\\
&\Updownarrow\\
f(M-x)&\geq f(N-x).
\end{aligned}~ 
\end{equation}


\textbf{证毕!}


\begin{theorem}{}
设 $M,N$ 是实线性空间 $L$ 中的\enref{凸集}{ConSet},且 $\ker M\neq\emptyset,\ker M\cap N=\emptyset$,则存在分离 $M,N$ 的线性泛函。
\end{theorem}

\textbf{证明:}


\textbf{证毕!}


