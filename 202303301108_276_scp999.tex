% 流体的控制方程
% 流体|流体力学

\begin{issues}
\issueDraft
\end{issues}

\pentry{流体运动的描述方法\upref{fluid1}}

流体力学所遵循的仍是古典力学的规律和方程。由此,在计算流体力学(CFD)中,我们仍可对一个小的控制体进行分析,得出以下基本的\textbf{控制方程}。

\subsection{质量守恒方程}
对于密度为$\rho$的控制体,由于其内部质量守恒,有

$\frac{\partial \rho }{\partial t}+\bigtriangledown \cdot (\rho \overrightarrow{u})=0~.$

若为不可压缩流体,则上式变为

$\bigtriangledown \cdot \overrightarrow{u} =0$

这意味着它是一个无源场。

若流体为流动状态,则有

$\frac{\partial \rho u}{\partial x}+\frac{\partial \rho v}{\partial y}+\frac{\partial \rho w}{\partial z}=0$

其中$u,v,w$分别控制体在为$x,y,z$方向上的速率。


\subsection{动量守恒方程(N-S方程)}
由动量守恒,我们知道:

单位时间内控制体内的动量增量=单位时间内流入控制体内的流体动量+外界的力
