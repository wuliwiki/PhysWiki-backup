% 量子比特
% license Xiao
% type Tutor

\pentry{量子力学基本原理\upref{QMPrcp}}

在经典信息学中,一个比特(bit)代表着一个取值为0或者1的随机变量。比如一个电容器的状态可以离散表示为一个比特。当电容器处于高电平的时候,我们将其状态记为1,否则则记为0。在基于经典物理的信息论中,我们认为0和1这两种状态是可以被准确无误地区分开的。

在量子信息处理中,量子比特(qubit)是比特这个概念的量子对应。它描述了一个由$\ket{0},\ket{1}$表示的二能级量子系统的状态。我们仍然希望两种“量子状态”是可以被准确无误地区分开的,这自然要求着$\braket{0}{1}=0$。也就是说,$\ket{0},\ket{1}$张成了一个二维希尔伯特空间。

和经典比特不同的是,量子比特可以处于两种状态的叠加态上。也就是说,一个一般的量子比特可以处在状态
\begin{equation}
\ket{\psi}=a\ket{0}+b\ket{1}~,\quad a,b\in\mathbb{C}~,\quad |a|^2+|b|^2=1~.
\end{equation}
在$\{\ket{0},\ket{1}\}$基的测量下,有$|a|^2$的概率得到状态0,有$|b|^2$的概率得到状态1。

我们目前讨论的全部都是纯态,这相当于是在说,系统的信息已经完全被掌握了(在经典物理中,这代表着系统处于相空间中的一个点)。但是即使我们掌握了系统的全部信息,伴随我们的测量仍然会出现随机性。这也是量子力学和经典世界很不一样的地方。



\subsection{物理实现}

有很多种不同的物理系统都可以实现量子比特。最简单的例子是自旋-1/2系统,它自然带有一个二维希尔伯特空间。不那么平凡的例子是光子的偏振自由度。虽然光子是自旋-1的系统,但是因为其没有静止质量,纵波方向的自由度被禁止。因此其偏振只能有两个取值,因此也可以作为量子比特的载体。此外,即使系统有着多于两个能级,比如很多类型的原子和介观量子电路,只要我们只考虑其中的两个能级,并保证有办法确定系统具体处在什么能级上,那么也可以将其作为量子比特的载体。

本部分不会花过多篇幅来讨论实际的物理载体,但是需要强调的是,量子信息科学并不是建立在抽象数学上的空中楼阁,而是有着扎实的物理根基。

\subsection{布洛赫(Bloch)球表示}

\subsub

我们来数一数一个量子比特有多少自由的(实数)参数。$a,b$各有两个参数,$|a|^2+|b|^2=1$构成一个约束条件,看起来有三个约束,但是由于量子态的全局相位可以忽略,因此又会少一个自由度。因此描述一个量子比特只需要两个实数参数就够了。

根据归一化条件,一个一般的量子比特的描述也可以用
\begin{equation}
\ket{\psi}=e^{i\alpha}\left(\cos\frac{\theta}{2}\ket{0}+\sin\frac{\theta}{2}e^{i\phi}\ket{1}\right)~
\end{equation}
来表示。

由于全局相位$\alpha$并不重要,因此我们总可以使用$\theta\in[0,\pi]$和$\phi\in[0,2\pi]$来表示一个任意的量子态。

可以看到在Bloch球表示当中,$\theta$和$\phi$刚好就是Bloch球上的球坐标。也就是说,任意一个量子比特所处的态都位于Bloch球面上。

\begin{figure}[ht]
\centering
\includegraphics[width=5cm]{./figures/d9ce8ed28f14c8db.png}
\caption{量子比特的Bloch球表示} \label{fig_Qubit}
\end{figure}

容易验证,当Bloch球上的两个量子态正交时,它们在球面上的点的连线会过球心。这样的点对被称为对径点对。这些点对中最重要的有三对,它们分别对应着$z,x,y$三个坐标轴与球面的交点:
\begin{equation}
\ket{0},\ket{1};\quad \ket{\pm}=\frac{1}{\sqrt{2}}(\ket{0}\pm\ket{1});\quad \ket{\pm i}=\frac{1}{\sqrt{2}}(\ket{0}\pm i\ket{1})~.
\end{equation}

另一个重要的性质是,Bloch球面上方向$\hat{n}=(\sin\theta\cos\phi,\sin\theta\sin\phi,\cos\theta)$所对应的量子态$\cos\frac{\theta}{2}\ket{0}+\sin\frac{\theta}{2}e^{i\phi}\ket{1}$,刚好是矩阵$\hat{n}\cdot\vec{\sigma}$本征值为$+1$的本征态。而对径点$\sin\frac{\theta}{2}\ket{0}-\cos\frac{\theta}{2}e^{i\phi}\ket{1}$则是$\hat{n}\cdot\vec{\sigma}$本征值为$-1$的本征态。这一结论将会在单比特量子门中有重要应用。

我们在这里先不讨论混态的情况,并将其放到量子系综以及量子态层析的章节。

\subsection{高维量子比特}

如果不加说明,量子比特都指代的是包含两个能级的量子系统的状态,不过在一些特殊情况下,我们也可以考虑更加高维的量子比特。为了做到这点只需要将希尔伯特空间的维度推广到$d$维。

当$d=3$时,我们称此时的量子比特为qutrit,在$d$为任意大于等于2的正数时,我们称此时的量子比特为qudit。

\subsection{多个量子比特}

\pentry{张量积\upref{DirPro}}

单个量子比特构成了量子信息的基本单元。在更加一般的量子信息处理任务中,我们往往需要处理$n$个不同的量子比特。复合系统的原理告诉我们,如果系统由两个子系统$A,B$所构成,那么这个系统的希尔伯特空间由子系统希尔伯特空间的张量积$\mathcal{H}_{AB}=\mathcal{H}_A\otimes\mathcal{H}_B$描述。这保证了我们可以合理地描述多个量子比特的量子态。

如果系统含有$n$个量子比特,并且记录第$k$个量子比特对应的系统的基矢为$\{\ket{0}_k,\ket{1}_k\}$,那么\footnote{这里用到了一个隐含假设,那就是每个量子比特对应的系统都是可编号、可区分的。这并不与量子力学中的全同性假设矛盾。比如在定域系统等体系中,我们仍然可以对不同的量子模式进行区分。}$$
|\psi\rangle=\sum_{j_1, j_2, \ldots, j_n \in\{0,1\}} \alpha_{j_1 j_2 \ldots j_n}\left|j_1\right\rangle_1 \otimes\left|j_2\right\rangle_2 \otimes \cdots \otimes\left|j_n\right\rangle_n~,
$$其中$\alpha_{j_1j_2\ldots j_n}$是复数而且满足归一化条件$\sum_{j_1j_2\ldots j_n}|a_{j_1j_2\ldots j_n}|^2=1$。

在不引起歧义的情况下,我们往往会将多比特空间的基矢量简记为$$\left|j_1\right\rangle_1 \otimes\left|j_2\right\rangle_2 \otimes \cdots \otimes\left|j_n\right\rangle_n\to \ket{j_1j_2\ldots j_n}~. $$

张量积的一个重要特性是:$\mathcal{H}_{AB}$中的态$\ket{\psi}_{AB}$不一定能够写作$\mathcal{H}_A$和$\mathcal{H}_B$中的态的张量积。如果一个两比特纯态满足这个性质,那么我们称这个态是两比特纠缠态。
\begin{exercise}{}
证明两比特量子态$$\ket{\Phi^{\pm}}=\frac{1}{\sqrt{2}}(\ket{00}\pm\ket{11})~,\quad \ket{\Psi^{\pm}}=\frac{1}{\sqrt{2}}(\ket{01}\pm\ket{10})~$$
都是两比特纠缠态。
\end{exercise}

纠缠在信息的角度代表着不平凡的量子关联,在后面的章节中我们将会详细地讨论这件事情。
