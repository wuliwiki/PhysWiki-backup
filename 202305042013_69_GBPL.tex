% 吉布斯相律
% 相律 相平衡 Gibbs

\footnote{本文参考自朱文涛《简明物理化学》}

\begin{theorem}{吉布斯相律}
在平衡的多组分多相系统中
\begin{equation}
f=C-P+n~.
\end{equation}
其中:

f: 自由度
C: 独立组分数
P: 相数
n: 约束条件
\end{theorem}
吉布斯相律是热平衡条件\upref{TherEq}的推论之一。一个证明可见文末。

\subsubsection{独立组分数}
\begin{definition}{独立 组分 数}
系统的独立组分数由以下公式定义:
\begin{equation}
C=S-R-R'~.
\end{equation}
其中:

C: 独立组分数
S: 物质种类数
R: 化学反应数
R': 浓度关系数
\end{definition}

可以认为,独立组分数是被更严格定义的物质种类数。

例如,在一杯纯净水中,S=1 (H2O), R=0, R'=0,因此C=1.

也有人认为,H2O会自发电离,生成H+与OH-,所以物质种类S=3 (H2O, H+, OH-);但是在这种情况下,系统中还有化学反应平衡 $H_2O\rightarrow H^++OH^-$, R=1 与物质守恒 c(H+)=c(OH-), R'=1。最终仍有 C=S-R-R'=1。

\subsubsection{自由度}
自由度f的含义为 可以在一定范围内独立变动而不引起系统相变系统的变量的个数。自由度必须大等于0,否则这样的系统是热力学不稳定的。例如常温常压下,略微独立地改变一杯纯净水的温度、压力,都不会导致水发生相变。

\subsubsection{约束条件}
%原书没有指明该变量的名称%
n与系统所受约束条件有关,在无外场的情况下,一般取2(温度、压力均可变)或1(仅温度可变)。例如,一个恒压系统的n=1.

\begin{example}{水沸腾时温度不变}
众所周知,1大气压下水沸腾时水温总为100℃。试用相律说明。

此时系统中C=1(水),P=2(气相与液相),n=1(恒压系统),因此f=0,系统不再有可变的变量,系统的温度必须是一个定值,即此时水的沸点100℃。

若不假定压力条件,则n=2,f=1,这意味着系统有一个自由度。或许你已经猜到,这意味着水的沸点随压力变化。
\end{example}

\begin{example}{铁的三相共存}
\begin{figure}[ht]
\centering
\includegraphics[width=14cm]{./figures/61d231eeeb86f579.png}
\caption{铁碳相图} \label{fig_GBPL_1}
\end{figure}
\footnote{该图片来自网络}

如图,为什么727℃时,铁碳合金的三相共存区是一条直线?

此时系统中C=2(铁与碳),P=3($\alpha, \gamma, Fe_3C$),n=1(恒压系统),因此f=0,系统不再有可变的变量,系统的温度必须是一个定值,即三相平衡温度727℃。系统温度略高或略低于此都会导致三相不再能稳定共存,而发生相变。

此结论可以推广至所有的恒压二元平衡相图。二元相图中,所有三相共存区均为水平直线段。
\end{example}

\subsubsection{证明}
1.总变量数

每一相的性质取决于温度、压力与其中各物质的含量 (临时约定$x_{a,b}$的含义为相a中物质b的浓度):

$X_1=f(T_1, p_1,x_{1,1},x_{1,2},x_{1,3}...,x_{1,S})~,$

$X_2=f(T_2,p_2,x_{2,1},x_{2,2},x_{2,3}...,x_{2,S})~.$

。。。

对于一个P相S物质的系统,共有P(S+2)个变量

2. 变量间的独立关系式子数

当系统平衡时,系统的变量应满足如下关系:

\begin{itemize}
\item 各相温度相同,共有(P-1)个等式
\begin{equation}
T_1=T_2=T_3=...
\end{equation}
\item 各相压力相同,共有(P-1)个等式
\begin{equation}
p_1=p_2=p_3=...
\end{equation}
\item 各相中物质的浓度和为1, 共有P个等式
\begin{equation}
\sum x_{1,i} = 1, \sum x_{2,i} = 1, ...
\end{equation}
\item 各物质在各相中的化学势相同,共有S(P-1)个等式
\begin{equation}
\mu_{1,1}=\mu_{2,1}=\mu_{3,1}=..., \mu_{1,2}=\mu_{2,2}=\mu_{3,2}..., ...
\end{equation}
\item 化学反应与浓度限制条件,共$(R+R')$个等式
\end{itemize}

3.系统自由度

系统自由度=总变量数-变量之间的独立关系式子数

因此,系统自由度 
\begin{equation}
f = P(S+2) - 2(P-1) - P - S(P-1)-R-R'=S-R-R'+P+2=C-P+2~.
\end{equation}

%有人会推广吗
可推广至一般的情况,即
\begin{equation}
f = C-P+n~.
\end{equation}
