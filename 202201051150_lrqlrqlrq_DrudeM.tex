% 德鲁德模型

汤姆逊(J.J.Thomsom)在1987年发现了电子,这对物质结构的物理理论产生了直接且深刻的影响.显然的也就引出了金属导电性和其内部自由电子的存在息息相关.

在汤姆逊的发现三年之后,德鲁德(Drude)较为成功的借鉴了理想气体动力学理论的思想和假设,并且将其运用在对金属的研究上.

德鲁德的自由电子气体模型简单地将金属看作是由\textbf{价电子(Valence electron)}所构成的基本均匀的电子气体.

\subsection{假设}
\begin{enumerate}
\item \textbf{独立电子近似(Independent electron approximation)}:电子之间不会相遇,不存在任何相互作用.
\item \textbf{自由电子近似}:1.电子和粒子之间不会相遇;2.电子在每次碰撞前后都是沿着直线运动;
\item \textbf{Jellium 近似}:正电荷,也就是原子核被假定均匀分布在空间中;电子密度在空间中也是一个均匀的量.由于正电荷的均匀分布,因此其对电子施加的电场为零.
\item 类似于理想气体的碰撞:电子会“忘记”碰撞前的速度,也就是电子碰撞前后的速度互不相关.电子碰撞之后的速度由能量(温度)决定:
\begin{equation}
\frac{3}{2}kT = \frac{1}{2}mv^2
\end{equation}

\end{enumerate}