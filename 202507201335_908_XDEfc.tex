% 薛定谔方程(综述)
% license CCBYSA3
% type Wiki

本文根据 CC-BY-SA 协议转载翻译自维基百科\href{https://en.wikipedia.org/wiki/Schr\%C3\%B6dinger_equation}{相关文章}。

薛定谔方程是一个偏微分方程,用于描述非相对论量子力学体系的波函数的演化过程。\(^\text{[1]: 1–2 }\) 它的发现是量子力学发展史上的一个重要里程碑。该方程以奥地利物理学家埃尔温·薛定谔的名字命名。他于1925年提出该方程,并于1926年发表,从而奠定了其后获得1933年诺贝尔物理学奖的工作基础。\(^\text{[2][3]}\)

在概念上,薛定谔方程是量子力学中对应于经典力学中牛顿第二定律的表达。给定一组已知的初始条件,牛顿第二定律可以用数学方式预测一个物理系统随时间演化的轨迹。薛定谔方程则给出了波函数随时间的演化规律,而波函数是对一个孤立物理系统的量子力学描述。该方程是薛定谔在路易·德布罗意提出“所有物质都具有伴随的物质波”这一假设的基础上提出的。薛定谔方程成功预测了与实验观测一致的原子束缚态。\(^\text{[4]: II:268 }\)

薛定谔方程并不是研究量子力学系统和进行预测的唯一方法。量子力学的其他表述方式还包括维尔纳·海森堡提出的矩阵力学,以及主要由理查德·费曼发展的路径积分表述。在比较这些方法时,使用薛定谔方程的方式有时被称为“波动力学”。

薛定谔提出的方程是非相对论性的,因为它在时间上是一阶导数,而在空间上是二阶导数,因此空间与时间在方程中并不对等。保罗·狄拉克将狭义相对论与量子力学结合成一个统一的表述形式,在非相对论极限下会简化为薛定谔方程。这就是狄拉克方程,它在空间和时间上都只包含一阶导数。

另一个偏微分方程,即克莱因–戈尔登方程,虽然是一个相对论性的波动方程,但在描述概率密度时出现了问题:概率密度可能为负值,这在物理上是不可接受的。狄拉克通过对克莱因–戈尔登算符进行所谓的“开平方”处理,引入了狄拉克矩阵,从而解决了这一问题。

在现代物理的语境中,克莱因–戈尔登方程用于描述无自旋粒子,而狄拉克方程则用于描述自旋为1/2的粒子。
\subsection{定义}
\subsubsection{预备知识}
在物理或化学的入门课程中,通常会以一种仅需掌握基础微积分(特别是关于空间与时间的导数)的概念和符号就能理解的方式来介绍薛定谔方程。薛定谔方程的一个特例,是针对一维空间中单个非相对论粒子的位置空间形式,其表达如下:
$$
i\hbar \frac{\partial}{\partial t}\Psi(x,t) = \left[ -\frac{\hbar^2}{2m} \frac{\partial^2}{\partial x^2} + V(x,t) \right] \Psi(x,t)~
$$
在这个方程中,$\Psi(x, t)$ 是波函数,即为每个时刻 $t$ 下的每个位置 $x$ 分配一个复数值的函数;$m$ 是粒子的质量;$V(x, t)$ 是势能函数,用来表示粒子所处环境中的势场\(^\text{[5]: 74 }\);$i$ 是虚数单位;$\hbar$ 是约化普朗克常数,其单位为作用量(能量乘以时间)\(^\text{[5]: 10 }\)。
\begin{figure}[ht]
\centering
\includegraphics[width=8cm]{./figures/ca5fd69c61da5de2.png}
\caption{满足非相对论自由薛定谔方程(即 $V = 0$)的波函数的复数图像。更多细节参见“波包”。} \label{fig_XDEfc_1}
\end{figure}
在超越上述简单情形的更广义框架中,保罗·狄拉克\(^\text{[6]}\)、大卫·希尔伯特\(^\text{[7]}\)、约翰·冯·诺依曼\(^\text{[8]}\)和赫尔曼·外尔\(^\text{[9]}\)等人所发展出的量子力学数学表述,规定一个量子力学系统的状态是一个向量 $|\psi\rangle$,它属于一个可分的复希尔伯特空间$\mathcal{H}$。该向量被假定在希尔伯特空间的内积下是归一化的,即用狄拉克记号表示,它满足$\langle \psi | \psi \rangle = 1$这个希尔伯特空间的具体形式取决于所研究的系统。例如:用于描述位置和动量的希尔伯特空间是平方可积函数空间 $L^2$;用于描述单个质子的自旋的希尔伯特空间则是二维复向量空间 $\mathbb{C}^2$,配有通常的内积形式\(^\text{[5]: 322 }\)。

感兴趣的物理量 —— 例如位置、动量、能量、自旋 —— 由可观测量来表示,而可观测量是作用于希尔伯特空间上的自伴算符。一个波函数可以是某个可观测量的本征矢,这种情况下称其为该可观测量的本征态,与之对应的本征值表示该本征态下该物理量所取的值。更一般地,一个量子态通常是多个本征态的线性叠加,这被称为量子叠加态。当对某个可观测量进行测量时,结果将是该算符的某个本征值,其出现的概率由玻恩规则给出:在最简单的情形中,如果本征值 $\lambda$ 是非简并的(即仅对应一个本征态),那么概率由$|\langle \lambda | \psi \rangle |^2$给出,其中 $|\lambda\rangle$ 是与本征值 $\lambda$ 对应的本征矢。更一般的情形中,如果本征值是简并的(即对应多个本征态),那么概率由$\langle \psi | P_{\lambda} | \psi \rangle$给出,其中 $P_{\lambda}$ 是投影到对应本征子空间(本征空间)上的投影算符。\(^\text{[note 1]}\)

例如,一个动量本征态将是一个无限延伸的完全单色波,它不是平方可积函数;同样,一个位置本征态将是一个狄拉克δ分布,它也不是平方可积函数,甚至严格来说并不属于函数范畴。因此,它们都不属于粒子的希尔伯特空间。物理学家有时将这些超出希尔伯特空间的本征态视为“广义本征矢”。这类状态用于计算上的便利,但不表示实际的物理态\(^\text{[10][11]: 100–105}\) 。因此,如上文中所使用的位置空间波函数 $\Psi(x, t)$ 可以写成时间相关态矢 $|\Psi(t)\rangle$ 与非物理但便于计算的“位置本征态” $|x\rangle$ 的内积形式:
$$
\Psi(x, t) = \langle x | \Psi(t) \rangle~
$$
\subsubsection{含时薛定谔方程}
\begin{figure}[ht]
\centering
\includegraphics[width=8cm]{./figures/c2d7ce6537b1413d.png}
\caption{这三行中的每一行都是满足含时薛定谔方程的简谐振子波函数。左侧:波函数的实部(蓝色)和虚部(红色)。右侧:在给定位置处找到粒子的概率分布。前两行是定态的例子,它们对应于驻波。最下方一行是一个非常态(非定态)的例子。} \label{fig_XDEfc_2}
\end{figure}
薛定谔方程的形式取决于具体的物理情境。其中最一般的形式是含时薛定谔方程,它描述了一个随时间演化的系统\(^\text{[12]: 143}\) :

\textbf{含时薛定谔方程(一般形式)}
$$
i\hbar \frac{d}{dt}|\Psi(t)\rangle = \hat{H}|\Psi(t)\rangle~
$$
其中,$t$ 是时间,$|\Psi(t)\rangle$ 是量子系统的态矢($\Psi$ 为希腊字母 psi),$\hat{H}$ 是一个可观测量,称为哈密顿算符。

“薛定谔方程”一词既可以指这一\textbf{一般形式},也可以特指其非相对论版本。这个一般形式的方程非常通用,广泛应用于整个量子力学领域,包括狄拉克方程和量子场论等情形,通过代入不同形式的哈密顿量来适配不同系统。而特定的非相对论版本是一种近似形式,在许多实际情况下能给出精确的结果,但其适用范围有限(参见相对论量子力学和相对论量子场论)。

在应用薛定谔方程时,首先需写出系统的哈密顿量,即考虑构成该系统的粒子的动能与势能,然后将其代入薛定谔方程中。由此得到的偏微分方程通过求解波函数来获得,而波函数包含了关于该系统的全部信息。在实际操作中,通常取波函数的绝对值的平方作为概率密度函数来使用\(^\text{[5]: 78 }\)。例如,对于一个在位置空间中的波函数 $\Psi(x, t)$,我们有:
$$
\Pr(x, t) = |\Psi(x, t)|^2~
$$
\subsubsection{定态薛定谔方程}
前述的含时薛定谔方程预测波函数可以形成驻波,称为定态。这些态具有特别重要的意义,因为研究它们可以简化对任意初始态的含时薛定谔方程的求解过程。定态还可以通过一个更简洁的形式来描述,即定态薛定谔方程(含时项被省略的版本)。

\textbf{定态薛定谔方程(一般形式)}
$$
\hat{H}|\Psi\rangle = E|\Psi\rangle~
$$
其中,$E$ 是系统的能量\(^\text{[5]: 134  }\)。该形式仅在哈密顿量本身不显含时间的情况下使用。然而,即使在这种情况下,整个波函数仍然依赖于时间,这一点将在后文关于线性叠加的部分中进一步解释。从线性代数的角度来看,这就是一个本征值方程,因此波函数是哈密顿算符的本征函数,其对应的本征值为 $E$。
\subsection{性质}
\subsubsection{线性性}
薛定谔方程是一个线性微分方程,这意味着如果两个态矢量$|\psi_1\rangle$ 和 $|\psi_2\rangle$ 是薛定谔方程的解,那么任意线性组合
$$
|\psi\rangle = a|\psi_1\rangle + b|\psi_2\rangle~
$$
也是它的解,其中 $a$ 和 $b$ 是任意复数\(^\text{[13]: 25   }\)。此外,这种求和还可以扩展为任意多个态矢量的线性组合。

这一性质允许量子态的叠加态仍然是薛定谔方程的解。更一般地说,一个薛定谔方程的通解可以通过对一组基态进行加权求和得到。常见的一种选择是使用能量本征态作为基底,这些本征态是定态薛定谔方程的解。在这种基底下,一个含时的态矢量 $|\Psi(t)\rangle$ 可以表示为以下线性组合:
$$
|\Psi(t)\rangle = \sum_n A_n e^{-iE_n t/\hbar} |\psi_{E_n}\rangle~
$$
其中,$A_n$ 是复数系数,$|\psi_{E_n}\rangle$ 是定态薛定谔方程的解,满足:
$$
\hat{H}|\psi_{E_n}\rangle = E_n |\psi_{E_n}\rangle~
$$
\subsubsection{幺正性}
在哈密顿算符 $\hat{H}$ 为常量的情况下,薛定谔方程的解为\(^\text{[12]}\):
$$
|\Psi(t)\rangle = e^{-i\hat{H}t/\hbar} |\Psi(0)\rangle~
$$
其中,
$$
\hat{U}(t) = e^{-i\hat{H}t/\hbar}~
$$
被称为时间演化算符,它是一个幺正算符,即它保持希尔伯特空间中任意两个向量之间的内积不变【13】。幺正性是薛定谔方程所描述的时间演化的一个基本特性。

如果初始态为 $|\Psi(0)\rangle$,那么任意时刻 $t$ 的态将由幺正算符 $\hat{U}(t)$ 给出:
$$
|\Psi(t)\rangle = \hat{U}(t) |\Psi(0)\rangle~
$$
反过来,假设 $\hat{U}(t)$ 是一族由参数 $t$ 标定的连续幺正算符,我们可以在不失一般性的前提下选取参数化方式,使得:$\hat{U}(0)$ 是单位算符;对任意 $N > 0$,都有 $\hat{U}(t/N)^N = \hat{U}(t)$。

那么,$\hat{U}(t)$ 必然满足如下形式:
$$
\hat{U}(t) = e^{-i\hat{G}t}~
$$
其中 $\hat{G}$ 是某个自伴算符,称为这族幺正算符的生成元。哈密顿量正是这种生成元之一(在自然单位制中,$\hbar$ 被设为1,因此该因子可忽略)。

为了验证生成元是厄米的,可以考虑:
$$
\hat{U}(\delta t) \approx \hat{U}(0) - i\hat{G} \delta t~
$$
那么有:
$$
\hat{U}(\delta t)^\dagger \hat{U}(\delta t) \approx (\hat{U}(0)^\dagger + i\hat{G}^\dagger \delta t)(\hat{U}(0) - i\hat{G} \delta t) = I + i\delta t(\hat{G}^\dagger - \hat{G}) + O(\delta t^2)~
$$
因此,只有当 $\hat{G}^\dagger = \hat{G}$ 时,即 $\hat{G}$ 是厄米算符,$\hat{U}(t)$ 才能保持幺正性(至少在一阶近似下)\(^\text{[15]}\)。
\subsubsection{基变换}
薛定谔方程通常是以位置的函数形式来呈现的,但从矢量-算符的角度看,它在希尔伯特空间中任何一个完备的 ket 基底下都可以有有效的表述。如前所述,为了计算的方便,也可以使用超出物理希尔伯特空间之外的“基底”。这一点可以通过对非相对论、无自旋粒子的位置空间与动量空间薛定谔方程的比较来说明\(^\text{[11]: 182}\) 。对于这类粒子,其希尔伯特空间是三维欧几里得空间上的复数平方可积函数空间,其哈密顿算符由一个关于动量算符的二次动能项和一个势能项组成:
$$
i\hbar \frac{d}{dt}|\Psi(t)\rangle = \left( \frac{1}{2m} \hat{p}^2 + \hat{V} \right) |\Psi(t)\rangle~
$$
记 $\mathbf{r}$ 为三维位置矢量,$\mathbf{p}$ 为三维动量矢量,则位置空间中的薛定谔方程为:
$$
i\hbar \frac{\partial}{\partial t} \Psi(\mathbf{r}, t) = -\frac{\hbar^2}{2m} \nabla^2 \Psi(\mathbf{r}, t) + V(\mathbf{r}) \Psi(\mathbf{r}, t)~
$$
而其动量空间对应形式涉及波函数和势能的傅里叶变换:
$$
i\hbar \frac{\partial}{\partial t} \tilde{\Psi}(\mathbf{p}, t) = \frac{\mathbf{p}^2}{2m} \tilde{\Psi}(\mathbf{p}, t) + (2\pi\hbar)^{-3/2} \int d^3\mathbf{p}' \, \tilde{V}(\mathbf{p} - \mathbf{p}') \tilde{\Psi}(\mathbf{p}', t)~
$$
其中,
$\Psi(\mathbf{r}, t)$ 和 $\tilde{\Psi}(\mathbf{p}, t)$ 是由态矢量 $|\Psi(t)\rangle$ 派生出来的:
$$
\Psi(\mathbf{r}, t) = \langle \mathbf{r} | \Psi(t) \rangle~
$$
$$
\tilde{\Psi}(\mathbf{p}, t) = \langle \mathbf{p} | \Psi(t) \rangle~
$$
注意,这里的 $|\mathbf{r}\rangle$ 和 $|\mathbf{p}\rangle$ 并不属于希尔伯特空间本身,但它们与希尔伯特空间中所有元素之间的内积都是良定义的。

当将三维空间限制为一维时,位置空间中的薛定谔方程就简化为前文所给出的薛定谔方程的第一种形式。量子力学中位置与动量之间的关系在一维情况下尤为直观。在规范量子化中,经典变量 $x$ 和 $p$ 被提升为自伴算符 $\hat{x}$ 和 $\hat{p}$,它们满足如下的基本对易关系:
$$
[\hat{x}, \hat{p}] = i\hbar~
$$
这意味着\(^\text{[11]: 190 }\):
$$
\langle x | \hat{p} | \Psi \rangle = -i\hbar \frac{d}{dx} \Psi(x)~
$$
也就是说,在位置表象中,动量算符 $\hat{p}$ 的作用是:$\hat{p} = -i\hbar \frac{d}{dx}$从而,$\hat{p}^2$ 对应的是二阶导数;而在三维中,这个二阶导数就变成了拉普拉斯算符$\nabla^2$。

基本对易关系还意味着:位置算符与动量算符是傅里叶共轭的。因此,原本在位置变量下定义的函数可以通过傅里叶变换转换为动量变量下的函数\(^\text{[5]: 103–104 }\)。在固体物理中,薛定谔方程常用动量表述形式写出,因为布洛赫定理保证了周期性晶格势能只会将 $\tilde{\Psi}(p)$ 与 $\tilde{\Psi}(p + \hbar K)$ 耦合在一起,这里 $K$ 是离散的倒格矢。这使得在布里渊区中的每个点上,可以相互独立地求解动量空间中的薛定谔方程\(^\text{[16]: 138}\) 。
\subsubsection{概率流密度}
薛定谔方程与局域概率守恒是一致的\(^\text{[11]: 238 }\)。它还确保一个归一化的波函数在时间演化后仍然保持归一化。在矩阵力学中,这意味着时间演化算符是一个幺正算符\(^\text{[17]}\) 。相比之下,例如克莱因–戈尔登方程,尽管可以通过重新定义波函数的内积使其对时间不变,但波函数模平方的总体体积积分却不一定是时间不变的\(^\text{[18]}\) 。

在非相对论量子力学中,概率的连续性方程表示为:
$$
\frac{\partial}{\partial t} \rho(\mathbf{r}, t) + \nabla \cdot \mathbf{j} = 0~
$$
其中:
$$
\mathbf{j} = \frac{1}{2m} \left( \Psi^* \hat{\mathbf{p}} \Psi - \Psi \hat{\mathbf{p}} \Psi^* \right)
= -\frac{i\hbar}{2m} \left( \psi^* \nabla \psi - \psi \nabla \psi^* \right)
= \frac{\hbar}{m} \operatorname{Im}(\psi^* \nabla \psi)~
$$
这里的 $\mathbf{j}$ 被称为概率流密度或概率通量(即单位面积上的概率流量)。

如果将波函数表示为:
$$
\psi(\mathbf{x}, t) = \sqrt{\rho(\mathbf{x}, t)} \exp\left( \frac{i S(\mathbf{x}, t)}{\hbar} \right)~
$$
其中 $S(\mathbf{x}, t)$ 是实函数,表示波函数的复相位,那么概率流密度可以写为:
$$
\mathbf{j} = \frac{\rho \nabla S}{m}~
$$
因此,波函数相位的空间变化被认为描述了波函数的概率流。尽管 $\frac{\nabla S}{m}$ 表面上类似于“速度”,但它并不代表粒子在某一点的速度,因为位置与速度的同时测量违背了不确定性原理\(^\text{[17]}\) 。
\subsubsection{变量分离法}
如果哈密顿量不显含时间,薛定谔方程可以写成如下形式:
$$
i\hbar \frac{\partial}{\partial t} \Psi(\mathbf{r}, t) = \left[-\frac{\hbar^2}{2m} \nabla^2 + V(\mathbf{r})\right] \Psi(\mathbf{r}, t)~
$$
左边的算符只与时间有关,右边的算符只与空间有关。用变量分离法求解这个方程的思路是:设解为一个空间部分与时间部分的乘积形式\(^\text{[19]}\):
$$
\Psi(\mathbf{r}, t) = \psi(\mathbf{r}) \, \tau(t)~
$$
其中,$\psi(\mathbf{r})$ 是仅依赖于粒子空间坐标的函数,$\tau(t)$ 是仅依赖于时间的函数。将该形式代入薛定谔方程左侧可以发现,时间部分 $\tau(t)$ 只是一个相位因子:
$$
\Psi(\mathbf{r}, t) = \psi(\mathbf{r}) \, e^{-iEt/\hbar}~
$$
这类解被称为定态解,因为其随时间变化的部分是纯相位因子,在通过玻恩规则计算概率密度时会被抵消,不影响可观测量\(^\text{[[12]: 143ff ]}\)。

波函数的空间部分满足如下的时间无关方程\(^\text{[20]}\):
$$
\nabla^2 \psi(\mathbf{r}) + \frac{2m}{\hbar^2} [E - V(\mathbf{r})] \psi(\mathbf{r}) = 0~
$$
其中能量 $E$ 也出现在之前时间部分的相位因子中。

这个形式可以推广到任意粒子数和任意维度的系统(只要势能不含时间):时间无关薛定谔方程的驻波解对应的是具有确定能量的状态,而不是能量叠加的概率分布。在物理学中,这些驻波称为“定态”或“能量本征态”;在化学中,它们被称为“原子轨道”或“分子轨道”。能量本征态构成一个基底:任意波函数都可以表示为对离散能量态的和、或对连续能量态的积分,更一般地说,是对某个测度的积分。这体现了谱定理:在有限维的状态空间中,这对应于厄米矩阵本征矢的完备性。

变量分离法对于时间无关薛定谔方程同样非常有用。例如,根据问题的对称性,可以在笛卡尔坐标系下将变量分离,如:
$$
\psi(\mathbf{r}) = \psi_x(x) \psi_y(y) \psi_z(z)~
$$
或者在球坐标系下将变量分离为径向与角向部分:
$$
\psi(\mathbf{r}) = \psi_r(r) \psi_\theta(\theta) \psi_\phi(\phi)~
$$
