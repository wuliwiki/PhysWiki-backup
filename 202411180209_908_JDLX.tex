% 经典力学(综述)
% license CCBYSA3
% type Wiki

本文根据 CC-BY-SA 协议转载翻译自维基百科\href{https://en.wikipedia.org/wiki/Classical_mechanics}{相关文章}。

\begin{figure}[ht]
\centering
\includegraphics[width=8cm]{./figures/c0ac5b53deabd6dc.png}
\caption{显示卫星绕地球轨道运动的示意图,其中展示了垂直的速度和加速度(力)矢量,通过经典力学的解释进行表示。} \label{fig_JDLX_1}
\end{figure}
经典力学是一种物理理论,用于描述物体的运动,例如抛射物、机械部件、航天器、行星、恒星和星系等。经典力学的发展涉及物理学方法和哲学的重大变革。[1] “经典”这一限定词将这种力学类型与20世纪初物理学革命之后发展起来的物理学区分开来,这些现代物理理论揭示了经典力学的局限性。[2]

经典力学最早的形式通常被称为牛顿力学。它基于17世纪以艾萨克·牛顿爵士为代表的奠基性工作的物理概念,以及牛顿、戈特弗里德·威廉·莱布尼茨、莱昂哈德·欧拉等人发明的数学方法,用来描述物体在力的作用下的运动。后来,基于能量的方法由欧拉、约瑟夫-路易·拉格朗日、威廉·罗恩·哈密顿等人发展,最终形成了分析力学(包括拉格朗日力学和哈密顿力学)。这些进步主要发生在18世纪和19世纪,超越了早期的工作;经过一定的修正,它们被应用于现代物理学的各个领域。

如果一个遵循经典力学定律的物体的当前状态已知,就可以确定它将来的运动方式以及过去的运动轨迹。然而,混沌理论表明,经典力学对长期预测并不可靠。在研究非极端质量且速度未接近光速的物体时,经典力学能够提供精确的结果。当研究的对象接近原子直径大小时,就需要使用量子力学;而描述接近光速的速度时,则需要使用狭义相对论。当物体质量极大时,广义相对论则变得适用。一些现代文献将相对论力学归入经典物理学范畴,认为它是该领域最成熟和精确的形式。
\subsection{分支}  
\subsubsection{传统划分}  
经典力学传统上分为三个主要分支。\textbf{静力学}是经典力学的一个分支,研究作用在不加速的物理系统上的力和力矩的分析,这些系统与其环境处于平衡状态。[3] \textbf{运动学}描述点、物体以及物体系统(物体群)的运动,而不考虑引起这些运动的力。[4][5][3] 作为一个研究领域,运动学常被称为“运动的几何学”,有时甚至被视为数学的一个分支。[6][7][8] \textbf{动力学}不仅仅描述物体的行为,还考虑解释这些行为的力。一些作者(例如Taylor (2005)[9] 和 Greenwood (1997)[10])将狭义相对论包含在经典动力学中。
\subsubsection{力与能量}  
另一种划分方法基于数学形式主义的选择。经典力学可以用多种不同的数学方式来表述。这些不同形式的物理内容是相同的,但它们提供了不同的洞察力,并有助于进行不同类型的计算。虽然“牛顿力学”一词有时被用作非相对论经典物理的同义词,但它也可以特指一种基于牛顿运动定律的特定形式主义。在这种意义上,牛顿力学强调力作为矢量量的重要性。[11]  

与之相对的是,\textbf{解析力学}利用描述系统整体运动的标量属性——通常是动能和势能。运动方程通过某种关于标量变化的基本原理从标量量推导出来。解析力学的两个主要分支是:  \textbf{拉格朗日力学},它使用广义坐标和对应的广义速度,在配置空间中进行分析;\textbf{哈密顿力学},它使用坐标和对应的广义动量,在相空间中进行分析。  

这两种表述通过对广义坐标、速度和动量进行\textbf{勒让德变换}是等价的;因此,两者包含描述系统动力学所需的相同信息。此外,还有其他表述方法,例如\textbf{哈密顿-雅可比理论}、\textbf{鲁斯力学}和\textbf{阿佩尔运动方程}。

所有描述粒子和场运动的方程,无论采用何种形式主义,都可以从一种广泛适用的结果——\textbf{最小作用量原理}推导出来。一个重要的结果是\textbf{诺特定理},它将守恒定律与它们所关联的对称性联系起来。
\subsubsection{按应用领域划分}  
另一种划分方式是根据应用领域:  

\begin{itemize}
\item  天体力学:涉及恒星、行星和其他天体的运动。  
\item  连续介质力学:用于描述作为连续体的材料,例如固体和流体(包括液体和气体)。  
\item  相对论力学:包括狭义和广义相对论,用于研究速度接近光速的物体的运动。  
\item  统计力学:为将单个原子和分子的微观性质与材料的宏观或整体热力学性质联系起来提供了理论框架。
\end{itemize}
\subsection{物体及其运动的描述}
\begin{figure}[ht]
\centering
\includegraphics[width=6cm]{./figures/196e37c82e19a9ea.png}
\caption{抛体运动的分析是经典力学的一部分。} \label{fig_JDLX_2}
\end{figure}
为了简化,经典力学通常将现实世界中的物体建模为点粒子,即大小可忽略不计的物体。点粒子的运动由少量参数决定:其位置、质量以及作用在其上的力。经典力学还描述了非点状扩展物体的更复杂运动。欧拉定律在这一领域对牛顿定律进行了扩展。角动量的概念依赖于与描述一维运动相同的微积分方法。火箭方程将物体动量变化率的概念扩展到包含物体“损失质量”的影响。(这些推广和扩展是通过将刚体分解为一组点粒子并基于牛顿定律推导出来的。)

实际上,经典力学能够描述的物体总是具有非零的大小。(非常小的粒子,如电子,其行为更准确地由量子力学描述。)具有非零大小的物体因额外的自由度而表现出比假想的点粒子更复杂的行为。例如,一个棒球在运动的同时可以旋转。然而,通过将此类物体视为由大量协同作用的点粒子组成的复合物体,可以使用点粒子的结果来研究它们。复合物体的质心表现得像一个点粒子。

经典力学假设物质和能量具有明确且可知的属性,例如在空间中的位置和速度。非相对论力学还假设力是瞬时作用的(参见\textbf{远距作用})。
\subsubsection{运动学}  
点粒子的位置是相对于一个坐标系定义的,该坐标系以空间中任意固定的参考点作为原点 \( O \)。一个简单的坐标系可以通过一个标记为 \( \mathbf{r} \) 的矢量描述粒子 \( P \) 的位置,该矢量由一个箭头表示,从原点 \( O \) 指向点 \( P \)。一般来说,点粒子不需要相对于 \( O \) 静止。在 \( P \) 相对于 \( O \) 运动的情况下,\( \mathbf{r} \) 被定义为时间 \( t \) 的函数。在爱因斯坦相对论之前(称为伽利略相对论),时间被认为是绝对的,即,对于任意一对事件,所有观察者所观测到的时间间隔都是相同的。[12]  

除了依赖于绝对时间,经典力学还假设空间的结构遵循欧几里得几何。[13]

\textbf{速度与速率}  
 
速度(即位移随时间变化的速率)定义为位置对时间的导数:  
\[
\mathbf{v} = \frac{\mathrm{d} \mathbf{r}}{\mathrm{d} t}~
\]  
在经典力学中,速度可以直接相加或相减。例如,一辆汽车以每小时60公里向东行驶,并超过另一辆以每小时50公里向东行驶的汽车,则较慢的汽车会感知较快的汽车以 \( 60 - 50 = 10 \) km/h 的速度向东运动。然而,从较快汽车的角度来看,较慢的汽车以10 km/h向西移动,通常表示为 \(-10 \) km/h,其中符号负号表示方向相反。  

作为矢量量,速度需要用矢量分析方法进行处理。

数学上,如果将之前讨论中的第一个物体的速度表示为矢量 \(\mathbf{u} = u \mathbf{d}\),第二个物体的速度表示为矢量 \(\mathbf{v} = v \mathbf{e}\),其中 \(u\) 是第一个物体的速率,\(v\) 是第二个物体的速率,\(\mathbf{d}\) 和 \(\mathbf{e}\) 分别是两个物体运动方向的单位矢量,则第二个物体观察到的第一个物体的速度为:
\[
\mathbf{u}' = \mathbf{u} - \mathbf{v}~
\]
类似地,第一个物体观察到的第二个物体的速度为:
\[
\mathbf{v}' = \mathbf{v} - \mathbf{u}~
\]
当两个物体沿同一方向运动时,该公式可以简化为:
\[
\mathbf{u}' = (u - v) \mathbf{d}~
\]
或者,忽略方向,仅以速率表示其差值:
\[
u' = u - v~
\]
\textbf{加速度}  
 
加速度,即速度的变化率,定义为速度对时间的导数(位置对时间的二阶导数):  
\[
\mathbf{a} = \frac{\mathrm{d} \mathbf{v}}{\mathrm{d} t} = \frac{\mathrm{d}^2 \mathbf{r}}{\mathrm{d} t^2}~
\]  
加速度表示速度随时间的变化。速度的变化可以表现为大小的变化、方向的变化,或两者同时变化。  

有时,速度大小的减小被称为\textbf{减速}(deceleration),但通常来说,任何随时间变化的速度,包括减速在内,统称为加速度。

\textbf{参考系}  

虽然可以相对于任何状态运动的观察者描述粒子的位置、速度和加速度,经典力学假设存在一个特殊的参考系族,在其中自然的力学定律呈现出相对简单的形式。这些特殊的参考系被称为\textbf{惯性参考系}。惯性参考系是一个理想化的参考系,在其中,如果一个物体所受的合力为零,它将以恒定的速度运动;即,要么保持静止,要么沿直线做匀速运动。在惯性参考系中,牛顿运动定律 \( F = ma \) 是有效的。[14]: 185   

\textbf{非惯性参考系}相对于某个惯性参考系加速运动。例如,一个相对于惯性参考系旋转的参考系并不是惯性参考系。[14] 从惯性参考系观察,非惯性参考系中的粒子似乎以不能通过现有力场解释的方式运动,因此方程中似乎出现了额外的力,这些力仅仅是相对加速度的结果。这些力被称为\textbf{虚拟力}、\textbf{惯性力}或\textbf{伪力}。  

考虑两个参考系 \( S \) 和 \( S' \)。对于每个参考系中的观察者,一个事件在参考系 \( S \) 中的时空坐标为 \( (x, y, z, t) \),在 \( S' \) 中的时空坐标为 \( (x', y', z', t') \)。假设在所有参考系中时间的测量方式相同,并且在 \( t = 0 \) 时满足 \( x = x' \),那么在参考系 \( S \) 和 \( S' \)(两者以相对速度 \( u \) 沿 \( x \) 方向运动)中观察到的同一事件的时空坐标关系为:  
\[
\begin{aligned}
x' &= x - tu, \\
y' &= y, \\
z' &= z, \\
t' &= t.
\end{aligned}~
\]  
这一组公式定义了一种称为\textbf{伽利略变换}的群变换(非正式称为伽利略变换)。该群是狭义相对论中\textbf{庞加莱群}的一个极限情况,当速度 \( u \) 远小于光速 \( c \) 时适用。  

这些变换带来了以下结果:  
\begin{itemize}
\item \( v' = v - u \)(从参考系 \( S' \) 的角度看,粒子的速度 \( v' \) 比从参考系 \( S \) 的角度看其速度 \( v \) 慢了 \( u \))  
\item \( a' = a \)(粒子的加速度在任何惯性参考系中都是相同的)  
\item \( F' = F \)(粒子所受的力在任何惯性参考系中都是相同的)  
\item 在经典力学中,光速不是常数;而在相对论力学中光速被赋予的特殊地位,在经典力学中没有对应关系。 
\end{itemize} 
对于某些问题,使用旋转坐标系(参考系)会更加方便。这样可以选择保留到一个惯性参考系的映射,或者额外引入\textbf{虚拟离心力}和\textbf{科里奥利力}。
\subsection{牛顿力学}  
在物理学中,力是指引起物体速度变化(即加速)的任何作用。力源于某种场,例如静电场(由静电荷产生)、电磁场(由运动电荷产生)或引力场(由质量产生)等。  

牛顿是第一个用数学表达力与动量关系的人。一些物理学家将牛顿第二定律视为力和质量的定义,而另一些人将其看作一个基本假设,即自然法则。[15] 无论哪种解释,其数学表达都是相同的,历史上称为“牛顿第二定律”:  
\[
\mathbf{F} = \frac{\mathrm{d} \mathbf{p}}{\mathrm{d} t} = \frac{\mathrm{d} (m\mathbf{v})}{\mathrm{d} t}~
\] 
其中,\( m\mathbf{v} \) 被称为(标准)动量。因此,作用在粒子上的合力等于粒子动量随时间的变化率。由于加速度的定义为 \( \mathbf{a} = \frac{\mathrm{d} \mathbf{v}}{\mathrm{d} t} \),第二定律可以写成更简化和常见的形式:  
\[
\mathbf{F} = m\mathbf{a}~
\]  
只要作用在粒子上的力是已知的,牛顿第二定律就足以描述粒子的运动。一旦可以获得每个作用力的独立关系,就可以将它们代入牛顿第二定律,得到一个常微分方程,这个方程被称为运动方程。  

\textbf{示例}:假设摩擦力是作用在粒子上的唯一力,并且摩擦力可以表示为粒子速度的函数,例如:  
\[
\mathbf{F}_{\mathrm{R}} = -\lambda \mathbf{v}~
\]  
其中 \( \lambda \) 是一个正常数,负号表明力的方向与速度的方向相反。此时,运动方程为:  
\[
-\lambda \mathbf{v} = m\mathbf{a} = m\frac{\mathrm{d} \mathbf{v}}{\mathrm{d} t}~
\]  
将其积分可以得到:  
\[
\mathbf{v} = \mathbf{v}_0 e^{-\lambda t / m}~
\]  
其中,\( \mathbf{v}_0 \) 是初速度。这意味着,随着时间的推移,粒子的速度以指数形式衰减为零。在这种情况下,等效的观点是粒子的动能被摩擦力吸收,并根据能量守恒定律转化为热能,从而导致粒子减速。通过进一步积分,还可以得到粒子的位置 \( \mathbf{r} \) 随时间的变化关系。  

\textbf{重要的力}包括引力和用于描述电磁现象的洛伦兹力。此外,牛顿第三定律有时可以用来推导作用在粒子上的力:如果已知粒子 \( A \) 对另一个粒子 \( B \) 施加了力 \( \mathbf{F} \),则 \( B \) 必须对 \( A \) 施加大小相等、方向相反的反作用力 \( -\mathbf{F} \)。牛顿第三定律的强形式要求 \( \mathbf{F} \) 和 \( -\mathbf{F} \) 沿 \( A \) 和 \( B \) 的连线作用,而弱形式则没有这样的要求。弱形式的牛顿第三定律的例子通常可以在磁力中找到。[需要澄清]  
\subsubsection{功与能量} 
如果对粒子施加一个恒定力 \( \mathbf{F} \),使其产生位移 \( \Delta \mathbf{r} \)(注释 1),则力所做的功定义为力矢量与位移矢量的标量积:  
\[
W = \mathbf{F} \cdot \Delta \mathbf{r}~
\]  
更一般地说,如果粒子从位置 \( \mathbf{r}_1 \) 沿路径 \( C \) 移动到位置 \( \mathbf{r}_2 \),而力随位置变化,则作用在粒子上的功由线积分给出:  
\[
W = \int_{C} \mathbf{F}(\mathbf{r}) \cdot \mathrm{d} \mathbf{r}~
\]  
如果从 \( \mathbf{r}_1 \) 到 \( \mathbf{r}_2 \) 的功与路径无关,则称该力为\textbf{保守力}。重力是保守力,理想弹簧的弹力(由胡克定律描述)也是保守力,而摩擦力是非保守力。  

粒子质量为 \( m \),速度为 \( v \) 时,其动能 \( E_k \) 为:  
\[
E_k = \frac{1}{2} m v^2~
\]  
对于由多个粒子组成的扩展物体,其总动能为所有粒子动能的总和。  

\textbf{功-能定理}指出,对于质量 \( m \) 不变的粒子,从位置 \( \mathbf{r}_1 \) 移动到 \( \mathbf{r}_2 \) 所做的总功 \( W \) 等于粒子动能的变化 \( \Delta E_k \):  
\[
W = \Delta E_k = E_{k2} - E_{k1} = \frac{1}{2} m \left(v_2^2 - v_1^2\right)~
\]  
\textbf{保守力}可以表示为一个标量函数(称为\textbf{势能})的梯度,记作 \( E_p \):  
\[
\mathbf{F} = -\mathbf{\nabla} E_p~
\]  
如果作用在粒子上的所有力都是保守力,且 \( E_p \) 为总势能(定义为作用力通过重新排列物体间位置所做的功),则通过对每个力的势能求和可以得到:  
\[
\mathbf{F} \cdot \Delta \mathbf{r} = -\mathbf{\nabla} E_p \cdot \Delta \mathbf{r} = -\Delta E_p~
\]  
势能的减少等于动能的增加:  
\[
-\Delta E_p = \Delta E_k \implies \Delta (E_k + E_p) = 0~
\]  
这个结果被称为\textbf{能量守恒定律},表明总能量:  
\[
\sum E = E_k + E_p~
\]  
在时间上是恒定的。这一定律非常有用,因为许多常见的力是保守力。
\subsection{拉格朗日力学} 
拉格朗日力学是一种基于\textbf{驻定作用原理}(也称为最\textbf{小作用原理})的经典力学表述。这一理论由意大利-法国数学家和天文学家约瑟夫-路易·拉格朗日(Joseph-Louis Lagrange)于1760年在其向都灵科学院的报告中首次提出,[16] 并在1788年的重要著作《分析力学》(\textbf{Mécanique analytique})中得到总结。  

拉格朗日力学将一个力学系统描述为一个由\textbf{配置空间} \( M \) 和其中的一个\textbf{平滑函数} \( L \)(称为拉格朗日量)组成的对 \( (M, L) \)。对于许多系统,拉格朗日量可以表示为:  
\[
L = T - V~
\]  
其中,\( T \) 是系统的动能,\( V \) 是系统的势能。  

\textbf{驻定作用原理}要求系统从拉格朗日量 \( L \) 派生出的\textbf{作用量泛函}在系统的时间演化过程中保持在一个驻定点(即最大值、最小值或鞍点)。这一约束条件允许通过拉格朗日方程计算系统的运动方程。[17]