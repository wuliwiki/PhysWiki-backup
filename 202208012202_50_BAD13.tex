% 北京航空航天大学2013年数据结构与C语言程序设计
% keys 北京航空航天大学 2013 数据结构 C语言程序设计 计算机 考研


考生注意:所有答题务必书写在考场提供的答题纸上,写在本试题单上的答题一律无效(本题单不参与阅卷).

\subsection{一、单项选择题}
(本题20分,每小题各2分)

1. 对于K度为n的线性表,建立其对应的单链表的时问复杂度为____. \\
A. $O(1)$ $\quad$ B $O(log_2n)$ $\quad$ C. $O(n)$ $\quad$ D. $O(n^2)$

2. 一般情况下.在一个双向链表中插一个新的链结点,____. \\
A.需要器改4个指针域自的指针 $\quad$ B.需要修改3个指针域自的指针 \\
C.需要修改2个指针域自的指针 $\quad$ D.只需要修改1个指针域自的指针

3. 假设用单十字母表示中糍表达式中的十运算数(或棒运算对象).井利用堆栈产生中缀表达式对应的后辍表达式.对于中辍表达式$A+B*(C/D-E)$,当从左至右扫描到运算数$E$时,堆栈中的运算符依次是____.(注:不包含表达式的分界符) \\
A. $+*/-$  $\quad$  B. $+*(/-$ $\quad$  C. $+*-$ $\quad$ D. $+*(-$

4. 若某二叉排序树的前序遍历序列为50,20,40,30,80,60,70.则后序遍历序列____. \\
A.30.40 20.50.70.60.80 $\quad$ B.30.40.20.W.60.80.50. \\
C.70.60.80.50.30.40.20 $\quad$ D.70.60.90.30.40.20.50.

5. 分别以6、3、8、12、5、7对应叶结点的权值构造的哈夫曼(Huffman)树的深度____. \\
A.6  $\quad$ B.5 $\quad$ C.4 $\quad$ D.3

6. 下列关于图的叙述中,\textbf{错误}的是 \\
A. 根据图的定义,图中至少有一个顶点 \\
B. 根据图的定义,图中至少有一个个顶点和一条边(弧) \\
c. 具有$n$个顶点的无向图最多有$n\times(n-1)/2$条边 \\
D. 具有$n$个顶点的有向图最多有$n\times(n-1)$条边(弧) \\

7. 若在有向图$O$的拓扑序列中,顶点$v_i$,在顶点$v_j$之前,则下列$4$种情形中不可出现的是____. \\
A. $G$中有弧$<v_i,v_j>$ \\
B. $G$中没有弧$<v_i,v_j>$ \\
C. $G$中有一条从顶点$v_i$到顶点$v_j$的路径 \\
D. $G$中有一条从顶点$v_j$到顶点$v_i$的路径.

8下列关于啬拄操作的叙述中,错误的是____, \\
    A在顺序表中查找元素可以采用顺序查拽法,也可H采用折半查找法
    B在链表中查找结点只能采用顺序查找法,不能采用折半查找琏
    c -般情况T,顺序查找法不如折半查找法的时间效率高
    D折半查找的过程可Uffi一棵#Z为“判定村”的二叉树来描述.
    9在-棵m阶B-树中.除根结点Z外的任何分支结点包含关键字的个数至少
E____.
  A. mri-i:    B. mr2    C rm/21-1:    D n吡1.
