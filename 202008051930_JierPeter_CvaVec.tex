% 协变和逆变

\pentry{正交归一基底\upref{OrNrB},爱因斯坦求和约定\upref{EinSum}}

在有内积的概念之前,线性空间的结构高度对称,任意线性无关的向量组都可以当一组基,而且各种基之间没有本质区别.但是有了内积的概念之后,一些特殊的基比其它的基更方便表达线性空间,这就是\textbf{正交归一基底},有时也称\textbf{标准正交基}或\textbf{单位正交基}.

但有的时候,更方便描述物理现象基也许不是标准正交基,这样一来就不得不牺牲分量描述的简洁性.比如说,如果取二维内积空间的一组基$\{\bvec{e}_1, \bvec{e}_2\}$,它不是标准正交的,那么任意向量$\bvec{v}=a\bvec{e}_1+b\bvec{e}_2$的分量就并不是$\bvec{v}\cdot\bvec{e}_1$和$\bvec{v}\cdot\bvec{e}_2$了.这个时候算出任意向量的分量麻烦了许多,因此我们引入新的方法来简化这一过程.

\begin{definition}{协变和逆变基向量}\label{CvaVec_def1}
设线性空间$V$有一组基$\{\bvec{e}_i\}_{i=1}^n$,称其为一组\textbf{协变基(covariance base)},它对应一组\textbf{逆变基(contravariance base)}$\{\bvec{e}^i\}_{i=1}^n$,其中$\bvec{e}_i\cdot\bvec{e}^j=\delta_{ij}$.协变基中的向量称为\textbf{协变基向量},逆变基中的向量称为\textbf{逆变基向量}.
\end{definition}

如果一个向量$\bvec{v}=x^i\bvec{e}_i=y_i\bvec{e}^i$\footnote{注意此处使用了爱因斯坦求和约定},那么有$x^i=\bvec{v}\cdot\bvec{e^i}$,$y_i=\bvec{v}\cdot\bvec{e_i}$.注意各式子中上下标的使用.

这样一来,协变和逆变基配合使用,任意向量$\bvec{v}$的分量就容易计算了.特别地,当协变基是标准正交基的时候,两个基是一样的.

\subsection{协变和逆变的过渡矩阵}

\pentry{过渡矩阵\upref{TransM}}

协变基和逆变基分别是两个不同的基,因此我们可以计算它们之间的过渡矩阵.由于逆变基的定义使用了内积,因此要考虑引入一个标准正交基.

设线性空间$V$中有一组标准正交基$\{\bvec{e}_i\}^n_{i=1}$,令$\{\bvec{r}_i\}^n_{i=1}$和$\{\bvec{r}^i\}^n_{i=1}$分别是一组协变基和逆变基.设$\{\bvec{r}_i\}^n_{i=1}$在标准正交基中的坐标矩阵\footnote{将$\bvec{r}_i$写成列向量,作为坐标矩阵的第$i$列.}是$\bvec{M}$,而$\{\bvec{r}^i\}^n_{i=1}$的坐标矩阵是$\bvec{N}=\bvec{MT}$,其中$\bvec{T}$是从协变基到逆变基的过渡矩阵,那么由\autoref{CvaVec_def1} 知,$\bvec{N}\Tr\bvec{M}=\bvec{E}$,其中$\bvec{E}$为$n$阶单位方阵.




