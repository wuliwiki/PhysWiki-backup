% 浙江理工大学 2011 年数据结构
% 2011年浙江理工大学991数据结构考研真题

\subsection{一、单选题}
在每小题的四个备选答案中选出一个正确答案,每小题3分,共45分.

1. 若线性表最常用的操作是存取第$i$个元素及其前趋的值,则采用(  )存储方式节省时间. \\
A.单链表 $\qquad$ B.双链表 $\qquad$ C.单循环链表 $\qquad$ D.顺序表

2. 设输入序列为$1$、$2$、$3$、$4$,则借助栈所得到的输出序例不可能是(  ) \\
A.1、2、3、4 \\
B.4、1、2、3 \\
C.1、3、4、2 \\
D.4、3、2、1

3. 常对数组进行的两种基本操作是(  ). \\
A.建立与删除 \\
B.插入与修改 \\
C.查找与修改 \\
D.查找与插入

4. 数组$Q[n]$用来表示个循环队列,$f$为当前队列头元素的前一位置, $r$为队尾元素的位置,假定队列中元素的个数小于$n$ ,计算队列中元素的公式为(  ) \\
A. r-f \\
B. (n+f-r)\%n \\
c. n+r-f \\
D. (n+r-f)\%n

5.广义表((a, b, C, d))的表尾是(  ) \\
A. a \\
B.( ) \\
C.(a,b,c, d) \\
D.((a,b,c,d))

6.实现任意二叉树的后序遍历的非递归算法而不使用栈结构,最佳方案是二叉树采用(    )存储结构. \\
A.三叉链表 \\
B.广义表存储结构 \\
C.二叉链表 \\
D.顺序表存储结构

7.在线索化二叉树中, $P$所指的结点没有左子树的充要条件是(    ). \\
A. P->left==null \\
B. P->ltag=1 \\
C. P->ltag==1且P->left==null \\
D. 以上都不对

8.稀疏矩阵一般的压缩存储方法有两种,即(    ). \\
A.二维数组和三维数组 \\
B.三元组和散列 \\
C.三元组和十字链表 \\
D.散列和十字链表

9.有$n$个结点的有向图的边数最多有(    ). \\
A. n \\
B. n(n-1) \\
C. n (n-1)/2 \\
D. 2n

10.带权有向图$G$用邻接矩阵$A$存储,则顶点$i$的入度等于$A$中(    ) \\
A.第$i$行非无穷元素之和 \\
B.第$i$列非无穷元素之和 \\
C.第$i$行非零且非无穷元素个数 \\
D.第$i$列非零且非无穷元素个数

11.采用邻接表存储的图的深度优先遍历算法类似于二叉树的_____. \\
A.先序遍历 \\
B.中序遍历 \\
C.后序遍历 \\
D.按层遍历

12.链表适用于()查找. \\
A.顺序 \\
B.二分法 \\
C.顺序,也能二分法 \\
D.随机

13.有一个长度为12的有序表,按二分查找法对该表进行查找,在表内各元素等概率情况下查找成功所需的平均比较次数为 \\
A.35/12 \\
B.37/12 \\
C.39/12 \\
D.43/12

14.快速排序在下列()情况 下最易发挥其长处. \\
A.被排序的数据中含有多个相同排序码 \\
B.被排序的数据已基本有序 \\
C.被排序的数据完全无序 \\
D.被排序的数据中的最大值和最小值相差悬殊

15.若一组记录的排序码为(46, 79, 56, 38, 40, 84),则利用堆排序的方法建立的初始堆为____. \\
A.79,46,56,38,40,84 \\
B.84,79,56,38,40,46 \\
C.84,79,56,46,40,38 \\
D.84,56,79,40,46,38

\subsection{二、填空题}
(每空3分,共30分.)

1.设$n$为正整数,求以下程序段中以记号@的语句的频度是 \\
\begin{lstlisting}[language=cpp]
k=0;
for(i=1;<=m;i++){
    for(j=1;j<=n;j++)
        @k++;
}
\end{lstlisting}

2.在一个单链表中的$P$所指结点之前插入一个$S$所指结点时,可执行如下操作:  \\
\begin{lstlisting}[language=cpp]
S->next = P->next;
P->next = S;
T = P->data;
P->data = (  ①  );
S->data = (  ②  );
\end{lstlisting}

3.在单链表中,要删除某一指定的结点(该结点不为首元结点),必须找到该结点的(    )结点.

4.在具有$n$个单元的循环队列中,队满时共有(    )个元素.

5.三维数组$a[4][5][6]$(下标从$0$开始计, $a$有$4\times5\times6$个元素) ,每个元素的长度是$2$ ,则$a[2][3][4]$的地址是(    ). (设$a[0][0][0]$的地址是$1000$,数据以行为主序方式存储).

6.$5$层完全二叉树至少有(    )个结点.

7.在一个有向图中,所有顶点的入度之和等于所有顶点的出度之和的(    )倍.

8.在各种查找方法中,平均查找长度与结点个数$n$无关的查找方法是(    ).

9.设要将序列(Q,H,C,Y,P,A,M,S,R,D,F,X)中的关键码按字母序的升序重新排列,则冒泡排序一趟扫描的结果是(    ).

\subsection{三、简答题}
(3题,共45分.)

1. (6分)说明线性表、栈与队列的异同点.

2. (19分)设二叉树BTree的存储结构如下:
\begin{table}[ht]
\centering
\caption{第三2题表}\label{ZSDS11_tab1}
\begin{tabular}{|c|c|c|c|c|c|c|c|c|c|c|}
\hline
 & 1 & 2 & 3 & 4 & 5 & 6 & 7 & 8 & 9 & 10 \\
\hline
left & 0 & 0 & 2 & 3 & 7 & 5 & 8 & 0 & 10 & 1 \\
\hline
data & j & h & f & d & b & a & c & e & g & i \\
\hline
right & 0 & 0 & 9 & 4 & 0 & 0 & 0 & 0 & 0 & 0 \\
\hline
\end{tabular}
\end{table}
其中,BTree为树根结点指针, left、right 分别为结点的左、右孩子指针域,在这里使用结点编号作为指针域值, 0表示指针域
值为空; data为结点的数据域.请完成如下问题:
(5)画出二叉树BTree的逻辑结构;
(6)写出按先序、中序和后序遍历二叉树BTree所得到的结点序列;
(7)画出二叉树BTree的后线索化树.


