% 神经网络
% keys 神经网络 人工神经网络
% license Xiao
% type Tutor

\textbf{神经网络}(Neural network, NN),准确地说,是\textbf{人工神经网络}(Artificial neural network, ANN),在机器学习领域中是指“由具有适应性的简单单元组成的广泛并行互联的网络,其组织能够模拟生物神经系统对真实世界物体所做出的交互反应”[1]。

神经网络是机器学习中广泛使用的一种基本方法。该方法具有较好的曲线拟合能力,能够从数据中学习离散型、连续型或者向量型函数。

\subsection{动机}

神经网络最初是受到生物神经系统结构的启发,而提出的机器学习模型。生物的神经系统,比如人脑,从结构上讲,是由大量的基本单位——神经元(neural)通过各种复杂的互相连接而构成。从功能的角度,irj

\subsection{基本结构}





\subsubsection{参考文献}
\begin{enumerate}
\item T. Kohonen, “An introduction to neural computing,” Neural Networks, vol. 1, no. 1, pp. 3–16, 1988.
\end{enumerate}