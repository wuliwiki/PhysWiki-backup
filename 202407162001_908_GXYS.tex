% 惯性约束核聚变
% license CCBYSA3
% type Wiki

(本文根据 CC-BY-SA 协议转载自原搜狗科学百科对英文维基百科的翻译)

\textbf{惯性约束核聚变(ICF)}是一种聚变能试图发起的研究核聚变通过加热和压缩燃料靶进行的反应,通常为通常含有以下物质的混合物的颗粒形式氘和氚。典型的燃料球大约有针头那么大,大约有10个毫克燃料。

为了压缩和加热燃料,能量使用高能激光束、电子或离子传递到靶的外层,尽管出于各种原因,几乎所有ICF装置截至2015年使用激光。受热的外层向外爆炸,对目标的其余部分产生反作用力,向内加速,压缩目标。该过程旨在产生穿过目标向内传播的冲击波。一组足够强大的冲击波可以压缩和加热中心的燃料,以至于发生聚变反应。

ICF是聚变能量研究的两个主要分支之一,另一个是磁约束聚变。当ICF在20世纪70年代初首次提出时,它似乎是一种实用的发电方法,并且该领域蓬勃发展。20世纪70年代和80年代的实验表明,这些装置的效率比预期的低得多,达到点火并不容易。在20世纪80年代和90年代,为了理解高强度激光和等离子体的复杂相互作用,进行了许多实验。这些导致了新机器的设计,更大,最终达到点火能量。

最大的可操作ICF实验是美国的国家点火装置(NIF),它是利用早期实验数十年的经验设计的。然而,像那些早期的实验一样,NIF没有达到点火,并且截至2015年,正在产生大约1⁄3所需的能量水平。[1]

\subsection{描述}
\subsubsection{1.1 基本融合}
\begin{figure}[ht]
\centering
\includegraphics[width=6cm]{./figures/2941b793a05597d5.png}
\caption{间接驱动激光器ICF使用黑腔室其内表面从任一侧用激光束锥照射,以用平滑的高强度X射线在内部浸泡融合微胶囊。可以看到最高能量的X射线通过环空器泄漏,这里用橙色/红色表示。} \label{fig_GXYS_1}
\end{figure}
聚变反应结合了较轻的原子,例如氢一起形成更大的。通常这些反应发生在如此高的温度下离子ed,他们的电子被高温剥去;因此,核聚变通常被描述为“原子核”,而不是“原子”。

原子核带正电,因此由于静电力相互排斥。克服这种排斥需要大量的能量,这被称为库仑障壁或者聚变势垒能量。一般来说,导致较轻的原子核融合需要较少的能量,因为它们的电荷较少,因此势垒能量较低,当它们融合时,会释放更多的能量。随着原子核质量的增加,有一个点是反应不再释放净能量——克服能量屏障所需的能量大于最终聚变反应释放的能量。

从能源角度来看,最好的燃料是氘和氚的一对一混合;两者都是氢的重同位素。D-T(氘和氚)混合物的屏障很低,因为它的中子和质子比率很高。原子核中中性中子的存在有助于通过核力将它们拉在一起,而带正电荷的质子的存在通过静电力将原子核推开。氚是任何稳定或中度不稳定核素中中子与质子比率最高的核素之一——两个中子和一个质子。添加质子或移除中子会增加能量屏障。

在标准条件下的D-T混合物不进行融合;在核力将原子核拉在一起形成稳定的集合之前,必须将它们压在一起。即使在太阳炎热、密集的中心,平均质子在融合之前也会存在数十亿年。[2]对于实际的聚变发电系统,必须通过将燃料加热到数千万度,和/或压缩到巨大的压力来大幅提高燃料利用率。任何特定燃料熔化所需的温度和压力称为劳森准则。自20世纪50年代第一颗氢弹诞生以来,这些条件就已经为人所知。在地球上遇到劳森判据是极其困难的,这也解释了为什么聚变研究花了许多年才达到目前的高技术水平。[3]
\subsubsection{1.2 ICF作用机制}
在氢弹中,聚变燃料用单独的裂变炸弹压缩和加热(参见Teller-Ulam设计)中。多种机制将裂变“初级”爆炸的能量转移到聚变燃料中。一个主要的机制是,初级粒子发出的x射线闪光被捕获在炸弹的工程外壳内,导致外壳和炸弹之间的体积充满x射线“气体”。这些x射线均匀地照射到聚变区的外部,即“次级”,迅速加热它,直到它向外爆炸。这种向外的吹出导致次级线圈的其余部分向内压缩,直到达到聚变反应开始的温度和密度。

裂变炸弹的要求使得这种方法不适合发电。这种引爆器不仅生产成本过高,而且制造这种炸弹的最小尺寸也是可以的,大致由所使用的钚燃料的临界质量来定义。一般来说,建造产量小于约1千吨的核装置似乎很困难,而聚变二次堆会增加产量。这使得从爆炸中提取能量成为一个困难的工程问题; PACER 项目研究了工程问题的解决方案,但也证明了成本在经济上是不可行的。

PACER的参与者之一,约翰·努科尔斯(John nuckols)开始探索随着次级粒子的尺寸减小,启动聚变反应所需的初级粒子的尺寸发生了什么变化。他发现,当次级粒子达到毫克大小时,激发它所需的能量落入兆焦范围。这远远低于原子弹的需求,因为原子弹的主爆炸源在万亿焦耳射程内,相当于大约6盎司的TNT炸药。

这在经济上是不可行的,这种设备的成本将超过它所生产的电能的价值。然而,有许多其他设备可能能够重复传递这种能量水平。这导致了使用一种设备将能量“束”到聚变燃料上,确保机械分离的想法。到20世纪60年代中期,激光器似乎将发展到可以获得所需能量水平的程度。

通常ICF系统使用单个激光器驾驶员其光束被分成许多光束,这些光束随后被分别放大一万亿倍或更多。这些被许多镜子送入反应室(称为目标室),镜子的位置是为了均匀地照射整个表面上的目标。驾驶员施加的热量导致目标的外层爆炸,就像氢弹燃料缸的外层被裂变装置的x光照射时一样。

材料从表面爆炸导致内部剩余的材料被巨大的力向内驱动,最终坍缩成一个微小的近球形球。在现代惯性约束聚变装置中,产生的燃料混合物的密度高达铅密度的100倍,约为1000克/厘米3。这个密度不足以独自创造任何有用的聚变速率。然而,在燃料崩溃的过程中,冲击波也形成并以高速进入燃料的中心。当他们遇到从燃料中心的另一侧进入的对手时,那个点的密度会提高很多。

给定正确的条件,在被冲击波高度压缩的区域中的融合率可以释放出大量高能的α粒子α。由于周围燃料的高密度,它们在被“热化”之前只移动很短的距离,将能量作为热量损失给燃料。这种额外的能量将在加热的燃料中引起额外的聚变反应,释放出更多的高能粒子。这个过程从中心向外扩展,导致一种自我维持的烧伤,称为点火。
\begin{figure}[ht]
\centering
\includegraphics[width=14.25cm]{./figures/f5ecab750c2b7c82.png}
\caption{激光惯性约束聚变阶段示意图。蓝色箭头代表辐射;橙色被吹走了;紫色是向内传输的热能。 Laser beams or laser-produced X-rays rapidly heat the surface of the fusion target, forming a surrounding plasma envelope. Fuel is compressed by the rocket-like blowoff of the hot surface material. During the final part of the capsule implosion, the fuel core reaches 20 times the density of lead and ignites at 100,000,000 ˚C. Thermonuclear burn spreads rapidly through the compressed fuel, yielding many times the input energy.} \label{fig_GXYS_2}
\end{figure}
\subsubsection{1.3 成功的问题}
自20世纪70年代早期实验以来,提高惯性约束聚变性能的主要问题是向靶输送能量、控制内爆燃料的对称性、防止燃料过早加热(在达到最大密度之前)、通过流体动力学不稳定性防止热燃料和冷燃料过早混合以及在压缩燃料中心形成“紧密”冲击波会聚。

为了将冲击波聚焦在目标的中心,必须使目标具有极高的精度和球形,其表面(内部和外部)的像差不超过几微米。同样,激光束的瞄准必须非常精确,并且光束必须同时到达目标上的所有点。然而,光束定时是一个相对简单的问题,并且通过在光束的光路中使用延迟线来实现皮秒级的定时精度来解决。困扰内爆靶实现高对称性和高温度/密度的另一个主要问题是所谓的“束-束”不平衡和束各向异性。这些问题分别是,由一个光束传递的能量可能高于或低于撞击目标的其他光束,以及光束直径撞击目标内的“热点”,这导致目标表面上的不均匀压缩,从而形成瑞利-泰勒不稳定性[4]在燃料中过早混合,并在最大压缩时降低加热效率。由于冲击波的形成,里克特迈耶-梅什科夫不稳定性也在此过程中形成。
\begin{figure}[ht]
\centering
\includegraphics[width=10cm]{./figures/5d114c9228a96cb3.png}
\caption{惯性约束聚变靶是一种填充泡沫的圆柱形靶,带有机械扰动,被新星激光压缩。这张照片拍摄于1995年。形象秀目标的压缩,以及瑞利-泰勒不稳定性的增长。[4]} \label{fig_GXYS_3}
\end{figure}
在过去二十年的研究中,通过使用各种光束平滑技术和光束能量诊断来平衡光束到光束的能量,所有这些问题都在不同程度上得到了缓解;然而,RT的不稳定性仍然是一个主要问题。多年来,目标设计也有了巨大的改进。现代的低温氢冰靶倾向于在用低功率 IR 激光照射时在塑料球的内部冻结一薄层氘,以使其内表面光滑,同时用配备有摄像机的显微镜对其进行监控,从而允许密切监控该层,确保其“光滑”。[5]充满氘氚(D-T)混合物的低温靶是“自平滑的”,因为放射性氚同位素衰变产生少量热量。这通常被称为“ beta -分层”。[6]
\begin{figure}[ht]
\centering
\includegraphics[width=6cm]{./figures/479154e2cb1cf81f.png}
\caption{镀金的国家点火装置 (NIF)环空器模型。} \label{fig_GXYS_4}
\end{figure}
某些目标被一个小金属圆柱体包围,用激光束照射,而不是目标本身,这种方法被称为“间接传动“。[7]在这种方法中,激光聚焦在圆柱体的内侧,将其加热到超热等离子体,该等离子体主要以 X射线辐射。然后,来自该等离子体的X射线被目标表面吸收,以与直接被激光击中相同的方式内爆。目标对热x射线的吸收比直接吸收激光更有效,然而环空器或者“燃烧室”本身也需要相当多的能量来加热,因此显著降低了激光到目标能量传递的整体效率。因此,即使在今天,它们也是一个有争议的特征;同样多的人”直接激励“设计不使用它们。大多数情况下,间接驱动环空器目标用于模拟热核试验,因为其中的聚变燃料也主要被X射线辐射内爆。
\begin{figure}[ht]
\centering
\includegraphics[width=6cm]{./figures/1db60d73ec8c6a27.png}
\caption{惯性约束融合燃料微胶囊(有时称为“微球”),其尺寸可用于NIF,可填充氘和氚气体或DT冰。胶囊可以插入环空器(如上)并在间接传动模式或直接用激光能量照射直接驱动配置。以前激光系统中使用的微胶囊明显更小,这是因为早期激光能够传递到目标的辐射功率较小。} \label{fig_GXYS_5}
\end{figure}
正在探索各种ICF驱动程序。自20世纪70年代以来,激光已经有了显著的进步,从少数几个激光器增加了能量和功率焦耳s和千瓦至兆焦耳(参见NIF激光)和数百太瓦,主要使用频率加倍或三倍的光从钕玻璃放大器。

重离子束对于商业化生产特别有趣,因为它们易于创建、控制和聚焦。另一方面,很难实现有效内爆目标所需的非常高的能量密度,大多数离子束系统需要使用围绕目标的环空器来平滑辐射,进一步降低了离子束能量与内爆目标能量耦合的整体效率。

\subsection{ 历史}
\subsubsection{2.1 第一个概念}

\textbf{在美国}

惯性约束聚变的历史可以追溯到1957年在日内瓦举行的“原子促进和平”会议。这是美国和俄罗斯超级大国之间由联合国主办的大型国际会议。在活动中涉及的许多主题中,有人考虑过使用氢弹来加热充满水的地下洞穴。由此产生的蒸汽将被用于驱动传统的发电机,从而提供电能。[8]

这次会议促成了“犁铧行动”的努力,并于1961年以这个名字命名。作为犁铧的一部分,研究了三个主要概念;在 PACER 项目下产生能量,使用大型核爆炸进行挖掘,并作为天然气工业的一种核压裂。PACER于1961年12月直接测试,当时3 kt Gnome 项目设备被安放在新墨西哥州的层状盐中。尽管有各种理论和尝试来阻止它,放射性蒸汽还是从离测试点有一段距离的钻杆中释放出来。作为PACER的一部分,进一步的研究导致了许多工程空洞取代天然空洞,但在此期间,整个犁铧的努力从糟糕变得更糟,特别是在1962年释放出大量沉降物的轿车失败后。尽管如此,PACER继续获得一些资金,直到1975年,一项第三方研究表明,PACER的电力成本将相当于燃料成本是其十倍以上的常规核电站。[9]

“原子促进和平”会议的另一个结果是促使约翰·努科尔斯开始考虑炸弹聚变方面发生的情况。当裂变炸弹爆炸时,它会释放X射线,从而使聚变侧内爆。这个“次级”被缩小到非常小的尺寸。他最早的工作是研究在仍然有很大的“增益”来提供净能量输出的情况下,聚变炸弹可以制造得多小。这项工作表明,在非常小的尺寸下,大约毫克,点燃它只需要很少的能量,远小于裂变的“初级”。[8]实际上,他提议使用悬浮在金属壳中心的一滴D-T燃料建造微型全聚变炸药,今天称之为环空器。外壳提供了与氢弹炸弹外壳相同的效果,将x射线捕获在里面,使它们照射燃料。主要区别在于x射线不会由外壳内的初级辐射源提供,而是由某种外部设备从外部加热外壳,直到它在x射线区域发光(参见热辐射)。能量将由当时未被识别的脉冲电源输送,他用炸弹术语称之为“主电源”。[10]

这种方案的主要优点是高密度下聚变过程的效率。根据劳森准则,在环境压力下将D-T燃料加热到收支平衡所需的能量可能是将其压缩到能够提供相同聚变速率的压力所需能量的100倍。因此,理论上,ICF方法在增益方面会显著更有效。[10]这可以通过考虑燃料缓慢加热的传统情况下的能量损失来理解,例如磁聚变能量;环境中的能量损失率是基于燃料与周围环境之间的温差,随着燃料的加热,温差会持续增加。在惯性约束聚变的情况下,整个环空器充满了高温辐射,限制了损失。[11]

\textbf{在德国}

大约在同一时间(1956年),核聚变先驱卡尔·冯·魏茨泽克在德国马克斯·普朗克研究所组织了一次会议。在这次会议上,弗里德沃德·温特尔贝格提出了通过高能炸药驱动的会聚冲击波实现热核微爆炸的非裂变点火。[12]前东德的一份解密报告进一步提到了温特尔贝格在德国进行的核爆炸研究斯塔西(Staatsicherheitsdienst)。[13]

1964年,温特尔贝格提出,点燃可以通过加速到1000的强烈微粒束来实现 km/s。[14]1968年,他提议使用马克思发生器产生的强烈电子束和离子束来达到同样的目的。[15]这种方案的优点是,带电粒子束的产生不仅比激光束的产生更便宜,而且由于强自磁场束场,还可以捕获带电聚变反应产物,从而大大降低了束点燃圆柱形靶的压缩要求。

\textbf{在苏联}

1967年,古尔根·阿斯卡莱恩研究员发表了一篇文章,提出在聚变中使用聚焦激光束氘化锂或氘。[16]
\subsubsection{2.2 早期研究}
到20世纪50年代末,劳伦斯利佛摩国家实验室大学的努科勒及其合作者对惯性约束聚变概念进行了多次计算机模拟。1960年初,这产生了1 密壳内的氘-氚燃料。模拟表明,输入环空器的5 MJ功率将产生50 MJ的聚变输出,增益为10。当时激光尚未发明,人们考虑了各种可能的驱动器,包括脉冲功率机、带电粒子加速器、等离子枪和超高速粒子枪。[17]

在这一年中,取得了两项重要的理论进展。新的模拟考虑了脉冲中传递能量的时间,称为“脉冲整形”,从而导致更好的内爆。此外,外壳变得更大更薄,形成了与几乎实心的球相反的薄外壳。这两个变化极大地提高了内爆的效率,从而大大降低了压缩它所需的能量。使用这些改进,计算出需要大约1 MJ的驱动器,五倍的改进。在接下来的两年里,提出了其他几个理论进展,特别是雷·基德美国发展了一种没有环空器的内爆系统,即所谓的“直接驱动”方法斯特灵高露洁罗恩·扎巴斯基研究的是只有1μg D-T燃料的非常小的系统。[18]

1960年激光的引入休斯研究实验室在加利福尼亚出现了一个完美的驱动机制。从1962年开始,利弗莫尔的导演小约翰·福斯特爱德华·泰勒开始了针对ICF方法的小规模激光研究。即使在这个早期阶段,ICF系统对于武器研究的适用性也是众所周知的,这也是它能够获得资金的主要原因。[19]在接下来的十年中,LLNL为基础激光等离子体相互作用研究制造了几个小型实验设备。
\subsubsection{2.3 发展开始了}
1967年基普·西格尔利用出售早期公司电导创股份的收益创办了KMS工业公司,电导创是全息术。20世纪70年代初,他成立了KMS聚变开始发展基于激光的惯性约束聚变系统。[20]这一发展引起了包括LLNL在内的武器实验室的强烈反对,他们提出了各种理由,认为不应允许KMS公开发展ICF。这种反对意见是通过原子能委员会传递的,该委员会要求为他们自己的努力提供资金。除了背景噪音之外,还有一个谣言,称苏联正在进行一项激进的ICF计划,即新的更高功率的CO2玻璃激光器、电子束驱动器概念和 20世纪70年代的能源危机为许多能源项目增加了动力。[19]

1972年,努科勒斯在自然介绍惯性约束聚变,并建议试验台系统可以与kJ范围内的驱动器产生聚变,以及与MJ驱动器产生高增益系统。[21][22]

尽管资源有限,业务问题众多,KMS聚变还是于1974年5月1日成功地展示了从ICF过程进行的聚变。[23]然而,在这一成功之后不久,西格尔就去世了,大约一年后,KMS融合结束了,公司按照西格尔的人寿保险政策经营。[20]这时,几个武器实验室和大学已经开始了他们自己的项目,特别是固体激光器s(Nd:玻璃激光器在LLNL和罗彻斯特大学,和氟化氪 受激准分子激光器s系统位于洛斯阿拉莫斯和海军研究实验室。

虽然KMS的成功导致了一项重大的发展努力,但随后的进展过去是,现在仍然是,聚变研究普遍存在的似乎难以解决的问题所阻碍的。
\subsubsection{2.4 高能惯性约束聚变}
高能ICF实验(每次数百焦耳和更大的实验)始于20世纪70年代初,当时首次设计了所需能量和功率的激光器。这是在成功设计磁约束聚变系统之后的一段时间,大约是在70年代早期引入的特别成功的托卡马克设计的时候。尽管如此,在20世纪70年代中后期多次能源危机刺激下,聚变研究获得了大量资金,性能迅速提高,惯性设计很快达到了最佳磁系统相同的“低于盈亏平衡”条件。

特别是LLNL,资金非常充足,并启动了一个主要的激光聚变发展计划。他们的贾纳斯激光器于1974年开始运行,并验证了使用钕玻璃激光器产生非常高功率器件的方法。在长光程激光器和独眼激光器中探索了聚焦问题,这导致了更大的阿格斯激光器。这些都不是为了成为实用的ICF设备,但每一种都将现有技术发展到了有信心基本方法是有效的程度。当时,人们认为制造一个更大的独眼巨人类型的设备可以压缩和加热ICF目标,导致在“短期”内点火。这是一个基于利用所谓的“爆炸推进器”型燃料胶囊的实验中看到的聚变产量外推的误解。在70年代末80年代初,随着各种等离子体不稳定性和激光-等离子体能量耦合损耗模式逐渐被理解,对实现点火所需的靶上激光能量的估计几乎每年翻一番。认识到简单的爆炸推进器靶设计和很少的千焦(kJ)激光辐射强度永远不会扩展到高增益聚变产量,导致了将紫外激光能量增加到100 kJ水平的努力,以及先进烧蚀器和低温DT冰靶设计的生产。
\subsubsection{2.5 湿婆和新星}
ICF驱动程序设计的最早的严肃和大规模尝试之一是\textbf{湿婆激光器},一个20束掺钕玻璃激光系统,建立于1978年开始运行的LLNL。Shiva是一种“概念验证”设计,旨在证明聚变燃料胶囊的压缩程度是氢的液体密度的许多倍。在这种情况下,湿婆成功地将其颗粒压缩到氘液体密度的100倍。然而,由于激光与热电子的强耦合,致密等离子体(离子)的过早加热是有问题的,并且熔化产率低。Shiva未能有效加热压缩的等离子体,这表明使用光学倍频器作为解决方案,在351处将来自激光器的红外光频率提高三倍,进入紫外线 nm。1980年在激光能量学实验室发现的新发现的高效频率三倍高强度激光的方案使得这种目标照射方法能够在24束ω激光器和 NOVETT激光器中进行实验,随后是能量是Shiva的10倍的 Nova激光器设计,第一个设计的具体目标是达到点火条件。

Nova也未能实现点火的目标,这一次是因为成丝导致其光束中激光强度的严重变化(以及光束之间的强度差异),这导致目标处辐射平滑度的巨大不均匀性和不对称内爆。早期开创的技术无法解决这些新问题。但这一失败再次导致了对内爆过程的更好理解,前进的道路似乎再次明确,即增加照射的均匀性,通过光束平滑技术减少激光束中的热点,以减少目标上的瑞利-泰勒不稳定性印迹,并将目标上的激光能量至少增加一个数量级。融合研究的资金在80年代受到严重限制,但Nova仍然成功地为下一代机器收集了足够的信息。
\subsubsection{2.6 国家点火装置}
由此产生的设计,现在被称为国家点火装置,于1997年开始在LLNL建造。NIF的主要目标是作为所谓的核管理计划的旗舰实验装置,支持LLNLs传统的炸弹制造角色。于2009年3月完成,[24]NIF现在已经用全部192束激光进行了实验,包括创造激光能量传输新记录的实验。[25][26]首次可信的点火尝试最初计划在2010年进行,[来源请求]但截至2012年9月30日,点火尚未实现。[27]截至2013年10月7日,该设施被认为已经实现了聚变商业化的一个重要里程碑,即燃料胶囊首次释放出比应用于它的更多的能量。[28]这离满足劳森标准还有很长的路要走,但这是向前迈出的一大步。[29]
\subsubsection{2.7 快速点火}
最近的发展是“快速点火”的概念,它可以提供一种在压缩后直接加热高密度燃料的方法,从而使内爆的加热和压缩阶段分离。在这种方法中,首先使用驱动激光系统“正常”压缩目标,然后当内爆达到最大密度(在驻点或“爆炸时间”)时,第二个超短脉冲超高功率拍瓦 (PW)激光传输聚焦在核心一侧的单个脉冲,急剧加热它,并有望开始聚变点火。两种类型的快速点火是“等离子体穿透”方法和“壳锥”方法。在第一种方法中,千兆瓦激光器被简单地期望直接穿过内爆胶囊的外部等离子体,并撞击和加热致密芯,而在壳锥方法中,胶囊被安装在小的高z(高原子序数)锥体的端部,使得锥体的尖端突出到胶囊的芯中。在第二种方法中,当胶囊内爆时,拍瓦可以清楚地看到高密度核,并且不需要浪费穿过“日冕”等离子体的能量钻孔;然而,视锥的存在对内爆过程有着重要的影响。目前正在进行几个项目来探索快速点火方法,包括罗彻斯特大学的欧米茄激光器升级、日本的GEKKO XII 设备以及一个全新的5亿英镑的设施,称为艾泰,计划在欧盟建造。如果成功,快速点火方法可以显著降低输送到目标的总能量;而NIF使用2 MJ,艾泰的司机是200 kJ和加热器70 kJ,然而预测的聚变增益甚至高于NIF。
\subsubsection{2.8 其他项目}
法国项目“激光mgajoule ”于2002年完成了第一条实验线,并于2014年完成了第一次目标拍摄。[30]截至2016年,该机器大约完成了75%。

使用完全不同的方法是z捏设备。$Z-pinch$使用大量电流,这些电流被转换成包含极细导线的圆柱体。电线蒸发形成导电的大电流等离子体;产生的周向磁场挤压等离子体缸,使其内爆,从而产生可用于驱动燃料胶囊内爆的高功率x射线脉冲。这种方法的挑战包括相对较低的驱动温度,导致内爆速度缓慢和潜在的巨大不稳定性增长,以及高能x射线引起的预热。[31][32]

最近,温特尔贝格提出用千兆伏超级马克思发生器点燃氘微爆炸,这是一个由多达100台普通马克思发电机驱动的马克思发生器。[33]

\subsection{作为能源}
自20世纪70年代末开始研究使用惯性约束聚变建造的实际发电厂,当时惯性约束聚变实验开始上升到更高的功率;他们被称为\textbf{惯性聚变能量},或\textbf{食品展览会}植物。这些装置将向反应室输送连续的靶流,通常几秒钟,并从内爆和聚变中捕获产生的热和中子辐射,以驱动传统的汽轮机。
\subsubsection{3.1 技术挑战}
综合实地演练在达到点火所需条件方面面临持续的技术挑战。但是,即使这些都要解决,也有大量的实际问题似乎同样难以克服。激光驱动系统最初被认为能够产生商业上有用的能量。然而,随着20世纪70年代和80年代对达到点火所需能量的估计急剧增加,这些希望被放弃了。考虑到激光放大过程的低效率(约1-1.5\%)和发电损失(蒸汽驱动涡轮系统的效率通常约为35\%),聚变增益必须在350左右,才能在能量上达到平衡。这些成果似乎不可能产生,ICF的工作主要转向武器研究。[来源请求]

随着最近快速点火和类似方法的引入,情况发生了巨大变化。在这种方法中,在第一个实验设备艾泰中预测增益为100。给定大约100的增益和大约1\%的激光效率,艾泰产生大约相同量的融合创造它需要电能。通过使用更新的设计,用激光二极管代替闪光灯,激光效率似乎也有可能提高一个数量级,激光二极管被调谐以在被强烈吸收的频率范围内产生它们的大部分能量。最初的实验设备提供大约10\%的效率,并且建议在一些额外的开发中20\%是真正的可能性。

像NIF这样的“经典”设备大约有330台 MJ的电力用于产生驱动光束,产生大约20的预期产量 MJ,最大可信产率为45 MJ。在反应堆中使用相同种类的数字,结合快速点火和新型激光器,可以显著提高性能。艾泰大约需要270英镑 kJ的激光能量,所以假设第一代二极管激光驱动器在10\%,反应器将需要大约3 电功率MJ。预计这将产生约30个 聚变能MJ。[34]即使很差的电能转换似乎也能提供真实世界的功率输出,产量和激光效率的逐步提高似乎也能提供商业上有用的输出。
\subsubsection{3.2 实际问题}
ICF系统在从其反应中产生有用能量时面临一些与磁系统相同的二次功率提取问题。主要考虑的问题之一是如何在不干扰目标和驱动光束的情况下成功地从反应室中移除热量。另一个严重的问题是聚变反应中释放的大量中子与植物反应,导致它们本身变得强烈放射性,以及机械弱化金属。由传统金属如钢建造的聚变工厂寿命相当短,必须经常更换核心安全壳。

处理这两个问题的一个当前概念,如\textbf{HYLIFE-II}基线设计,就是使用氟锂铍,熔化的混合物氟化物的盐锂和铍它既保护反应室不受中子影响又带走热量。氟锂铍然后被传入一个热交换器在那里它加热涡轮机中使用的水。[35]裂变锂核产生的氚也可以被提取出来,以关闭电厂的热核燃料循环,这是永久运行的必要条件,因为氚不是天然存在的,必须制造。另一个概念,\textbf{Sombrero},使用由碳纤维强化聚合物建造的反应室,该反应室具有非常低的中子截面。冷却由熔融陶瓷提供,选择这种陶瓷是因为它能够阻止中子进一步移动,同时也是一种有效的传热介质。[36]
\begin{figure}[ht]
\centering
\includegraphics[width=6cm]{./figures/2e63c5f62df374de.png}
\caption{新星的惯性约束聚变内爆,创造了密度和温度极高的“微型太阳”条件,甚至可以与我们的太阳核心相媲美。} \label{fig_GXYS_6}
\end{figure}
\subsubsection{3.3 经济可行性}
即使这些技术进步解决了综合实地演练中的大量问题,另一个不利于综合实地演练的因素是燃料成本。即使努科勒斯正在开发他最早关于这个想法的详细计算,他的同事们也指出:如果一台IFE机器产生50 MJ的聚变能量,人们可能会预计一次发射可能会产生10 MJ的输出功率。转换成更知名的单位,这相当于2.8千瓦时的电力。当时电网电力批发价格约为0.3美分/千瓦时,这意味着这次枪击事件的货币价值可能是1美分。在其间的50年里,即使通货膨胀率和2012年的通货膨胀率一样,电力价格也保持不变加拿大安大略省大约是2.8美分/千瓦时[37]

因此,为了使综合实地演练工厂在经济上可行,以2012年美元计算,燃料喷射的成本必须远低于10美分。当这一反对意见首次被提出时,努科尔建议使用从滴管状设备喷射到环空器中的液滴。[38]鉴于对更高的目标均匀性的需求不断增加,这种方法似乎不实用,因为即使是内部烧蚀体和燃料本身目前的成本也比这高几个数量级。此外,努克尔斯的解决方案是将燃料滴入固定的环空器中,然后在连续循环中重复使用,但在当前的能量水平下,环空器每次发射都会被摧毁。

直接驱动系统避免了环空器的使用,因此在燃料方面可能更便宜。然而,这些系统仍然需要烧蚀器件,精度和几何因素甚至更重要。它们也远不如间接驱动系统发达,并且在内爆物理学方面面临更多的技术问题。目前,对于直接驱动系统是否真的会降低运行成本还没有很强的共识。

\subsection{预计发展}
这种项目的各个阶段如下,惯性约束聚变发展的顺序遵循大致相同的大纲:

燃烧演示
一些聚变能量释放的可再现性成就(不一定是> 1的品质因子)。
高增益演示
具有足够能量增益的反应器可行性的实验证明。
工业示范
验证各种技术选项,以及定义商业反应器所需的全部数据。
商业演示
证明反应堆长时间工作的能力,同时尊重安全、责任和成本的所有要求。
目前,根据现有数据,[38]惯性约束聚变实验没有超过第一阶段,尽管Nova和其他科学家已经多次证明了这一领域的操作。短期内,一些新系统预计将进入第二阶段。

要进行真正的工业示范,还需要做更多的工作。特别是,激光系统需要能够在高工作频率下运行,可能每秒一至十次。本文中提到的大多数激光系统即使每天运行一次也有困难。艾泰部分预算也致力于这方面的研究。因为二极管激光器将电能转换成效率更高的激光,所以运行温度也更低,这反过来又允许它们以更高的频率工作。艾泰目前正在研究以1:1 MJ运行的设备 赫兹,或者10赫兹时100千焦 赫兹。

R&D继续在欧盟和日本进行惯性核聚变能源研究。高功率激光能量研究(艾泰)设施是一个拟议的实验聚变装置,正在进行初步设计,以便在欧盟继续发展激光驱动惯性约束方法。艾泰是第一个专门设计来研究快速点火产生核聚变的方法。使用比传统设计小得多的激光器,但产生大约相同大小的聚变功率输出,将提供高得多的能量Q建筑成本降低了约十倍。自21世纪初艾泰设计以来的理论研究对快速点火产生了怀疑,但一种被称为冲击点火已经被提议解决其中一些问题。[39][40][41]日本开发了KOYO-F聚变反应堆设计和激光惯性聚变试验实验反应堆。[42][43][44]2017年4月,彭博新闻社报道说迈克·卡西迪,前谷歌副总裁兼Loon项目随着谷歌[x],启动了清洁能源初创公司阿波罗聚变,以开发混合聚变-裂变反应堆技术。[45][46]

\subsection{核武器计划}
惯性约束聚变实验中遇到的非常热和致密的条件类似于热核武器中产生的条件,并应用于核武器计划。例如,惯性约束聚变实验可以用来帮助确定弹头性能如何随着老化而下降,或者作为设计新武器计划的一部分。保留核武器计划方面的知识和企业专长是追求ICF的另一个动机。[47][48]美国对NIF的资助来源于“核武器储备管理”计划,该计划的目标也是相应的。[49]有人认为ICF研究的某些方面可能违反了《全面禁试条约》或者《核不扩散条约》。[50]从长远来看,尽管存在巨大的技术障碍,ICF的研究可能会创造出“纯融合武器”。[51]

\subsection{中子源}
惯性约束聚变有可能产生比散裂多几个数量级的中子。中子能够比x光更有效地定位分子中的氢原子、解析原子分子热运动和研究光子的集体激发。中子散射分子结构研究可以解决与蛋白质折叠、跨膜扩散、质子转移机制、分子马达动力学等相关的问题。通过将热中子调制成慢中子束。[52]ICF产生的中子与裂变材料结合,可以潜在地用于混合核聚变设计中来产生电能。

\subsection{参考文献}
[1]
^Blackburn, T. G.; Ridgers, C. P.; Kirk, J. G.; Bell, A. R. (7 January 2014). "Quantum Radiation Reaction in Laser–Electron-Beam Collisions". Physical Review Letters. 112 (1): 055001. arXiv:1503.01009. Bibcode:2014PhRvL.112a5001B. doi:10.1103/PhysRevLett.112.015001. PMID 24483905..

[2]
^"FusEdWeb | Fusion Education". Fusedweb.llnl.gov. Archived from the original on 2013-05-10. Retrieved 2013-10-11..

[3]
^Hoffman, Mark (2013-03-23). "What Is The Lawson Criteria, Or How to Make Fusion Power Viable". Scienceworldreport.com. Retrieved 2014-08-23..

[4]
^Hayes, A. C.; Jungman, G.; Solem, J. C.; Bradley, P. A.; Rundberg, R. S. (2006). "Prompt beta spectroscopy as a diagnostic for mix in ignited NIF capsules". Modern Physics Letters A. 21 (13): 1029. arXiv:physics/0408057. Bibcode:2006MPLA...21.1029H. doi:10.1142/S0217732306020317..

[5]
^惯性约束聚变计划活动,2002年4月。 Archived 5月 11, 2009 at the Wayback Machine.

[6]
^惯性约束聚变计划活动,2006年3月。 Archived 5月 11, 2009 at the Wayback Machine.

[7]
^Lindl, John; Hammel, Bruce (2004), "Recent Advances in Indirect Drive ICF Target Physics", 20th IAEA Fusion Energy Conference (PDF), Lawrence Livermore National Laboratory, retrieved August 23, 2014.
[8]
^Nuckolls 1998, p. 1..
[9]
^长时间的和平核爆炸“”,原子科学家公报,1976年10月,第24-25页。.
[10]
^Nuckolls 1998, p. 2..
[11]
^Nuckolls 1998, p. 3..
[12]
^斯图加特大学图书馆档案,康沃特7号,霍克博士的遗产,1956年冯·魏茨泽克,在哥廷根开会.
[13]
^前东德民主共和国的史塔西报告,MfS-年度股东大会,作者:“德国民主共和党员联合会”,柏林中央档案馆,1987年.
[14]
^F.温特尔贝格,自然科学基金会。19a,231 (1964).
[15]
^F.温特尔贝格,物理学版本174,212 (1968).
[16]
^Gurgen Askaryan (1967). Новые физические эффекты [New Physical Effects]. Nauka i Zhizn (in Russian). 11: 105.CS1 maint: Unrecognized language (link).
[17]
^Nuckolls 1998, p. 4..
[18]
^Nuckolls 1998, pp. 4-5..
[19]
^Nuckolls 1998, p. 6..
[20]
^肖恩·约翰斯顿,《》采访拉里·西伯特博士“,美国物理联合会,2004年9月4日.
[21]
^Nuckolls, John; Wood, Lowell; Thiessen, Albert; Zimmerman, George (1972), "Laser Compression of Matter to Super-High Densities: Thermonuclear (CTR) Applications" (PDF), Nature, 239 (5368): 139–142, Bibcode:1972Natur.239..139N, doi:10.1038/239139a0, retrieved August 23, 2014.
[22]
^Lindl, J.D. (1993), "The Edward Teller medal lecture: The evolution toward Indirect Drive and two decades of progress toward ICF ignition and burn", International workshop on laser interaction and related plasma phenomena (PDF), Department of Energy (DOE)'s Office of Scientific and Technical Information (OSTI), retrieved August 23, 2014.
[23]
^Wyatt, Philip (December 2009). "The Back Page". Aps.org. Retrieved 2014-08-23..
[24]
^Hirschfeld, Bob (March 31, 2009). "DOE announces completion of world's largest laser". Publicaffairs.llnl.gov. Archived from the original on May 27, 2010. Retrieved 2014-08-23..
[25]
^Jason Palmer (2010-01-28). "Laser fusion test results raise energy hopes". BBC News. Retrieved 2010-01-28..
[26]
^"Initial NIF experiments meet requirements for fusion ignition". Lawrence Livermore National Laboratory. 2010-01-28. Archived from the original on 2010-05-27. Retrieved 2010-01-28..
[27]
^William J. Broad. "So Far Unfruitful, Fusion Project Faces a Frugal Congress"..
[28]
^Philip Ball (12 February 2014). "Laser fusion experiment extracts net energy from fuel". Nature. pp. 12–27. doi:10.1038/nature.2014.14710. Retrieved 2014-02-13..
[29]
^"Nuclear fusion milestone passed at US lab". BBC News. 7 October 2013. Retrieved 8 October 2013. fusion reaction exceeded the amount of energy being absorbed by the fuel.
[30]
^https://web.archive.org/web/20221028221608/http://www-lmj.cea.fr/fr/lmj/index.htm.
[31]
^Z箍缩发电厂聚变能脉冲功率驱动系统 Archived 1月 17, 2009 at the Wayback Machine.
[32]
^Grabovskii, E. V. (2002). Fast Z - Pinch Study in Russia and Related Problems. DENSE Z-PINCHES: 5th International Conference on Dense Z-Pinches. AIP Conference Proceedings. 651. pp. 3–8. Bibcode:2002AIPC..651....3G. doi:10.1063/1.1531270..
[33]
^Winterberg, Friedwardt (2008-12-01). "Ignition of a deuterium micro-detonation with a gigavolt super marx generator". arXiv:0812.0394 [physics.gen-ph]..
[34]
^Dunne, Mike (2006), "HiPER: a laser fusion facility for Europe", Fast Ignition Workshop (PDF), Central Laser Facility, Rutherford Appleton Laboratory, retrieved August 23, 2014.
[35]
^Olson, Craig; Tabak, Max; Dahlburg, Jill; Olson, Rick; Payne, Steve; Sethian, John; Barnard, John; Spielman, Rick; Schultz, Ken; Peterson, Robert; Peterson, Per; Meier, Wayne; Perkins, John (1999), "Inertial Fusion Concepts Working Group, Final Reports of the Subgroups", 1999 Fusion Summer Study (PDF), Columbia University, retrieved August 23, 2014.
[36]
^Sviatoslavsky, I.N.; Sawan, M.E.; Peterson, R.R.; Kulcinski, G.L.; MacFarlane, J.J.; Wittenberg, L.J.; Mogahed, E.A.; Rutledge, S.C.; Ghose, S.; Bourque, R. (1991), "SOMBRERO - A Solid Breeder Moving Bed KrF Laser Driven IFE Power Reactor", 14th IEEE/NPSS Symposium on Fusion Engineering (PDF), Fusion Technology Institute, University of Wisconsin, retrieved August 23, 2014.
[37]
^"IESO Power Data". Ieso.ca. Archived from the original on 2014-10-02. Retrieved 2014-08-23..
[38]
^Nuckolls 1998, p. 5..
[39]
^Perkins, L. J.; Betti, R.; LaFortune, K. N.; Williams, W. H. (2009). "Shock Ignition: A New Approach to High Gain Inertial Confinement Fusion on the National Ignition Facility" (PDF). Physical Review Letters. 103 (4): 045004. Bibcode:2009PhRvL.103d5004P. doi:10.1103/PhysRevLett.103.045004..
[40]
^HiPER Project Team (1 December 2013). HiPER Preparatory Phase Completion Report (PDF). Retrieved 1 May 2017..
[41]
^Ribeyre, X.; Schurtz, G.; Lafon, M.; Galera, S.; Weber, S. (2009). "Shock ignition: an alternative scheme for HiPER". Plasma Physics and Controlled Fusion (in 英语). 51 (1): 015013. Bibcode:2009PPCF...51a5013R. doi:10.1088/0741-3335/51/1/015013. ISSN 0741-3335..
[42]
^Norimatsu, Takayoshi; Kozaki, Yasuji; Shiraga, Hiroshi; Fujita, Hisanori; Okano, Kunihiko; Azech, Hiroshi (2013). "Laser Fusion Experimental Reactor LIFT Based on Fast Ignition and the Issue". CLEO: 2013 (2013), Paper ATh4O.3 (in 英语). Optical Society of America: ATh4O.3. doi:10.1364/CLEO_AT.2013.ATh4O.3. ISBN 978-1-55752-972-5..
[43]
^Norimatsu, T.; Kawanaka, J.; Miyanaga, M.; Azechi, H. (2007). "Conceptual Design of Fast Ignition Power Plant KOYO-F Driven by Cooled Yb:YAG Ceramic Laser". Fusion Science and Technology. 52 (4): 893–900. doi:10.13182/fst52-893..
[44]
^Norimatsu, T. (2006). "Fast ignition Laser Fusion Reactor KOYO-F - Summary from design committee of FI laser fusion reactor" (PDF). US-Japan workshop on Power Plant Studies and related Advanced Technologies with EU participation (24-25 January 2006, San Diego, CA)..
[45]
^Stone, Brad (3 April 2017). "Former Google Vice President Starts a Company Promising Clean and Safe Nuclear Energy". Bloomberg.com. Retrieved 2017-05-01..
[46]
^Thompson, Avery (3 April 2017). "Can a Googler's Fusion Startup Kickstart Nuclear Power?". Popular Mechanics (in 英语). Retrieved 2017-05-01..
[47]
^理查德·加温《今日军备控制》,1997年.
[48]
^"Science". Lasers.llnl.gov. Retrieved 2014-08-24..
[49]
^"Stockpile Stewardship". Lasers.llnl.gov. Retrieved 2014-08-24..
[50]
^Makhijani, Arjun; Zerriffi, Hisham (1998-07-15). "Dangerous Thermonuclear Quest". Ieer.org. Retrieved 2014-08-23..
[51]
^琼斯和冯·希佩尔,《科学与全球安全》,1998年,第7卷,p129-150。 Archived 3月 9, 2008 at the Wayback Machine.
[52]
^Taylor, Andrew; Dunne, M; Bennington, S; Ansell, S; Gardner, I; Norreys, P; Broome, T; Findlay, D; Nelmes, R (February 2007). "A Route to the Brightest Possible Neutron Source?". Science. 315 (5815): 1092–1095. Bibcode:2007Sci...315.1092T. doi:10.1126/science.1127185. PMID 17322053..