% 分类
% keys 分类 分类器
% license Xiao
% type Wiki

\pentry{数据\upref{datast}}

\textbf{分类}(Classfication)是一种把实例或者对象划归到特定的类别中的操作。在机器学习中,分类指的是从给定特征来预测离散型输出值的学习任务。预测输出值就称为类别。与之对应的,预测连续型变量的操作是\textbf{回归}。

机器学习中的分类是通过建立一个从输入空间$\bvec X$到输出空间$\bvec Y$(有限个离散值的集合)的映射来实现的。该映射被称为分类模型或者\textbf{分类器}(Classifier)。学习算法的任务就是从训练数据中学习相应的规律,从而建立分类器。该学习过程就是分类器的训练过程。常用的分类器有很多,比如决策树、逻辑回归、支持向量机、神经网络等。

最常见的分类任务是二分类(Binary classification),即类别只有两类的分类任务。通常,称两类中的一类为正类(Positive class),另一类为反类或负类(Negative class)。显然,此时,输出空间$\bvec Y$中元素的个数为$2$,可以表示为$Y=\{+1, -1\}$或者$Y=\{0, 1\}$。也有不少文献使用“$+$”表示正例,"$-$"表示反例。如果,一个分类任务的待预测类别数是三个或者三个以上的话,就称为多分类(Muti-class classification)任务[1]。


\begin{table}[ht]
\centering
\caption{睡眠数据集}\label{tab_Class1}
\begin{tabular}{|c|c|c|c|c|c|c|c|c|c|c|}
\hline
编号 & 性别 & 年龄 & 职业 & 睡眠时间(小时) & BMI指数 & 心率 & 舒张压 & 收缩压 & 每日走路步数 & 睡眠障碍 \\\hline
1 & 男 & 27 & 软件工程师 & 6.1 & 超重 & 77 & 83 & 126 & 4200 & 无 \\
\hline
2 & 男 & 28 & 医生 & 6.2 & 正常 & 75 & 80 & 125 & 10000 & 无 \\
\hline
3 & 女 & 30 & 护士 & 6.4 & 正常 & 78 & 86 & 130 & 4100 & 睡眠暂停 \\
\hline
4 & 男 & 29 & 教师 & 6.3 & 肥胖 & 82 & 90 & 140 & 3500 & 失眠 \\
\hline
\end{tabular}
\end{table}



\textbf{参考文献:}
\begin{enumerate}
\item 周志华. 机器学习[M]. 北京:清华大学出版社. 2016, 3
\end{enumerate}