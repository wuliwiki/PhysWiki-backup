% 陕西师范大学 2012 年 考研 量子力学
% license Usr
% type Note

\textbf{声明}:“该内容来源于网络公开资料,不保证真实性,如有侵权请联系管理员”

\subsection{选择正确答案:(每小题3分,共36分)}
\begin{enumerate}
\item  具有一个价电子的原子,其哈密顿算符 $\hat{H} = \frac{\hbar^2}{2m} \nabla^2 + V_{\text{eff}}(r) + g(r) \mathbf{\hat L} \cdot \mathbf{\hat S}$. 下面三组力学量中给出该体系的守恒量的是_______。
    \begin{itemize}
        \item (a) $\hat{H}, \hat{L}^2, L_z, (\hat{S}^2), \hat{S_z}$;
        \item (b) $\hat{H}, \hat{L}^2, (\hat{S}^2), \hat{J_z}\hat{S_z}$;
        \item (c) $\hat{H}, \hat{L}^2, (\hat{S}^2), \hat{J^2},\hat{J_z}$。
    \end{itemize}
\item  若某一原子的能级表示为 $3^2P_{1/2}$,则该能级的简并度是_______。
 \begin{itemize}
        \item (a) 3;
        \item (b) 2;
        \item (c) 6。
    \end{itemize}
    \item 按照量子力学理论,下列表述中正确的是_______。
    \begin{itemize}
        \item (a) 体系的力学量用波函数表示;
        \item (b) 体系的状态用厄米算符表示;
        \item (c) 体系的状态用波函数描述,力学量用厄米算符表示。
    \end{itemize}
     \item 有三个可能的对易关系\\
     1.$[\hat{x}, \hat{p}_y] = i \hbar$\\
     2. $[\hat{p}_x, \hat{p}_y] = i \hbar \hat{p}_z$\\
     3.$[\hat{L}_z, \hat{L}^2] = 2[\hat{L}_z, \hat{L}]$\\
     下面的三种表述中正确的是_______。
     \begin{itemize}
        \item (a) 关系 (1) 正确,(2) 和 (3) 错误。
        \item (b) (1) 和 (2) 正确,(3) 错误。
        \item (c) (1), (2) 和 (3) 都错误。
    \end{itemize}
    \item 由辐射的吸收和发射理论知,偶极跃迁的选择定则可表述为\[
l' - l = \pm1, \quad \Delta m = m' - m = 0, \pm1.~\]
根据这一条件,以下选项中正确的是_______。
     \begin{itemize}
        \item (a) 条件满足时跃迁是允许的。
        \item (b) 条件不满足是跃迁是禁戒的。
        \item (c) 条件满足时原子将发射或吸收一个电子。
    \end{itemize}
    \item 根据量子力学理论,若角动量量子数 $l$ 表示,则一刚体转子的角动量的可能取值正比于_______。
         \begin{itemize}
        \item (a) $l(l+1)^{1/2}\hbar$;
        \item (b) $l(l+1)\hbar^2$;
        \item (c) $l\hbar$。
    \end{itemize}
    \item 扫描隧道电子显微镜(STEM)的工作原理是基于下述哪一效应或性质_______。
         \begin{itemize}
        \item (a) 电子的波动性质。
        \item (b) 电子的波-粒二象性。
        \item (c) 电子的隧穿效应。
    \end{itemize}
    \item 对于一个处在能量本征态 $\psi_n(r)$ 的体系 $\left( \hat{H} \psi_n(r) = E_n \psi_n(r) \right)$ 进行能量测量的结果是_______。
    \begin{itemize}
    \item (a) 结果不确定。
    \item (b) 按照一定几率分布的两个以上本征值。
    \item (c) 本征值 $E_n$,相应几率为 1。
\end{itemize}
    \item 能量为 10 eV 微观粒子,下面数字中最接近该粒子物质波波长的是_______。 
    \begin{itemize}
    \item (a) $10^{-3}$ 米;
    \item (b) $10^{-6}$ 米;
    \item (c) $10^{-10}$ 米。
\end{itemize}
    \item $\hat{S}$ 为电子的自旋算符,下面表述中不正确的是_______。  
    \begin{itemize}
    \item (a) $\hat{S}_x \hat{S}_y - \hat{S}_y \hat{S}_x = 0$;
    \item (b) $\hat{S}_x \hat{S}_y + \hat{S}_y \hat{S}_z = 0$;
    \item (c) $\hat{S}_x^2 = \hat{S}_y^2 = \hat{S}_z^2$。
\end{itemize}
    \item 根据测不准原理,下述陈述中正确的是_______。 
    \begin{itemize}
    \item (a) 位置和动量不能够同时被精确测量。
    \item (b) 位置和动量能够同时被精确测量。
    \item (c) 任何条件下位置或动量均不能被精确测量。
\end{itemize}
    \item $\psi(\mathbf{r}, t)$ 表示一体系的状态。下述表述正确的是_______。
    \begin{itemize}
    \item (a) $\psi(\mathbf{r}, t)$ 描述的状态与相同 $\psi(\mathbf{r}, t)$。
    \item (b) $|\psi(\mathbf{r}, t)|^2$ 决定体系位置的几率分布。
    \item (c) $\psi(\mathbf{r}, t)$ 表达与 $\psi(\mathbf{r}, t)$ 完全相同的状态。
\end{itemize}
\end{enumerate}
\subsection{求解 (25分)}
一微观粒子所在势场可表示为
\[
V(x) = 
\begin{cases} 
0, & 0 \leq x \leq a \\
\infty, & x < 0 , x > a
\end{cases}~
\]
(1)通过求解薛定谔方程求给出该粒子能量本征值和本征函数系。(15分)\\
(2)若粒子位于基态,求在区域 $\left\{x = \frac{a}{4}, x = \frac{3a}{4}\right\}$ 中找到粒子的几率。(10分)
\subsection{证明(15 分)}
\begin{enumerate}
    \item 当 $\hat{A}$ 和 $\hat{B}$ 对易时,证明 $ (\hat{A} + \hat{B})(\hat{A} - \hat{B}) = \hat{A}^2 - \hat{B}^2$。(5 分)
    
    \item 应用坐标和动量的对易关系证明 $[\hat{L}_z, x] = i \hbar y$。(注:公式中的 $h$ 代表 $\hbar$。)(10 分)
\end{enumerate}
\subsection{求解 (20 分)}
若厄米算符 $\hat{F}$ 的本征方程为 $\hat{F} \varphi_i(x) = \lambda_i \varphi_i(x)$,则 $\hat{F}$ 表象指的是以本征函数集 $\{\varphi_i(x)\}$, $i=1,2,\dots,n$ 为基矢量所张开的空间。假设 $\hat{F}$ 的本征值谱是分立的,在 $\hat{F}$ 表象中求解以下问题:
\begin{enumerate}
    \item 任意一个状态 $|\psi\rangle$ 的表示。(8 分)
    \item 任意一个可观测力学量 $\hat{A}$ 的表示,并指出在什么条件下可得到对角矩阵表示。(12 分)
\end{enumerate}
\subsection{求解 (15 分)}
一具有二能级结构的原子的哈密顿算符为 $\hat H_0$,相应的能级分别为 $E_{01}$ 和 $E_{02}$。若该体系受到微小扰动 $\hat H'$,且该扰动在 $\hat H_0$ 表象中的矩阵元分别为 $H'_{12} = H'_{21} = a$ 和 $H'_{11} = H'_{22} = b$,其中 $a$ 和 $b$ 为实数。求体系能级的二级修正,并给出二级近似能量。
\subsection{求解 (14 分)}
在含有三个无相互作用的全同波色子的体系中,每一波色子有三个可能状态 $\varphi_{k1}$、$\varphi_{k2}$ 和 $\varphi_{k3}$。分别列出在下述两种条件下体系的所有可能状态:
\begin{enumerate}
    \item $\varphi_{k1}$、$\varphi_{k2}$ 和 $\varphi_{k3}$ 三个态各有一个粒子占据 (7 分)
    \item $\varphi_{k1}$,$\varphi_{k2}$态有一个粒子占据,$\varphi_{k3}$ 有两个粒子占据 (7 分)
\end{enumerate}
\subsection{求解 (25 分)}
一类氢原子所处状态为
\[
\psi = C R_{43}(r) \begin{pmatrix}
\sqrt{5} Y_3^1(\theta, \varphi) \\
\sqrt{2} Y_3^2(\theta, \varphi)
\end{pmatrix}~
\]

\begin{enumerate}
    \item 求归一化常数 $C$。(5 分)
    \item 检验态 $\psi$ 是否为总角动量算符 $\hat{J}^2$ 的本征态。若是,求相应的本征值;若不是,简述理由。(10 分)
    \item 计算自旋 z 分量 $\hat{S}_z$ 的平均值。(5 分)
    \item 求总角动量算符 $\hat{J}^2$ 的矩阵表达式。(5 分)
\end{enumerate}