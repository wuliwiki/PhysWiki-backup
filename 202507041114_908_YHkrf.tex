% 约翰·考克饶夫(综述)
% license CCBYSA3
% type Wiki

本文根据 CC-BY-SA 协议转载翻译自维基百科\href{https://en.wikipedia.org/wiki/John_Cockcroft}{相关文章}。

\begin{figure}[ht]
\centering
\includegraphics[width=6cm]{./figures/ed431b9628bc8c4d.png}
\caption{1961年的考克饶夫} \label{fig_YHkrf_1}
\end{figure}
约翰·道格拉斯·考克饶夫爵士(Sir John Douglas Cockcroft,1897年5月27日-1967年9月18日)是英国核物理学家,因与欧内斯特·沃尔顿共同实现原子核裂变而获1951年诺贝尔物理学奖,这一成就对核能的发展起到了关键作用。

在第一次世界大战期间,考克饶夫曾在西线担任皇家野战炮兵服役。战后,他在曼彻斯特市立工艺学院学习电气工程,同时在大都会维克斯特拉福德园区担任学徒,并成为该公司研究部门的一员。随后,他获得奖学金进入剑桥大学圣约翰学院,并于1924年6月参加三一试,成为Wrangler(剑桥数学优等生)。欧内斯特·卢瑟福接纳考克饶夫在卡文迪许实验室攻读研究生,考克饶夫于1928年在卢瑟福的指导下完成博士学位。在沃尔顿和马克·奥利芬特的合作下,他建造了后来被称为考克饶夫–沃尔顿发生器的装置。考克饶夫和沃尔顿利用这一装置首次实现了对原子核的人造裂变,这一壮举被大众称为“劈开原子”。

在第二次世界大战期间,考克饶夫担任英国供应部科研助理主任,负责雷达相关工作。他还是处理弗里施–佩尔斯备忘录(该备忘录计算出原子弹在技术上可行)相关问题的委员会成员,并参与了随后成立的MAUD委员会。1940年,作为提泽德代表团的一员,他将英国技术与美国同行共享。战争后期,提泽德代表团成果以SCR-584雷达和近炸引信的形式返回英国,并被用于协助击落V-1飞弹。

1944年5月,他出任蒙特利尔实验室主任,负责监督ZEEP和NRX反应堆的开发,以及乔克河实验室的创建。

战后,考克饶夫出任哈韦尔原子能研究机构(AERE)主任,1947年8月15日,低功率、石墨慢化的GLEEP反应堆在哈韦尔启动,成为西欧首座投入运行的核反应堆。随后在1948年又建成了英国实验堆0号(BEPO)。哈韦尔参与了温斯凯尔反应堆和化学分离工厂的设计。在他的领导下,哈韦尔还参与了前沿聚变研究,包括ZETA计划。他坚持要求在温斯凯尔反应堆的排气烟囱上安装过滤器,这一做法曾被讥讽为“考克饶夫的愚行”,但在1957年温斯凯尔火灾导致其中一座反应堆堆芯燃烧并释放放射性物质后,这一措施证明了其重要性。

1959年至1967年,他出任剑桥大学丘吉尔学院首任院长。1961年至1965年,他还担任堪培拉澳大利亚国立大学校监。
\subsection{早年经历}
约翰·道格拉斯·考克饶夫,也被称作“Johnny W.”,于1897年5月27日出生在英格兰约克郡西区托德莫登,是纺织厂主约翰·阿瑟·考克饶夫和妻子安妮·莫德(娘家姓菲尔登,Annie Maude née Fielden)的长子。他有四个弟弟:埃里克、菲利普、基思和莱昂内尔。1901年至1908年,他在沃尔斯登的英格兰教会学校接受早期教育,1908年至1909年就读于托德莫登小学,1909年至1914年就读于托德莫登中学,在校期间,他参加了足球和板球运动。在这所学校就读的女生中,有他未来的妻子尤尼斯·伊丽莎白·克拉布特里。1914年,他获得了约克郡西区的郡优秀奖学金,进入曼彻斯特维多利亚大学学习数学。

1914年8月,第一次世界大战爆发。考克饶夫于1915年6月完成在曼彻斯特的第一学年。他加入了校内的军官训练团,但并不希望成为军官。在暑假期间,他在威尔士金梅尔军营的基督教青年会食堂工作。1915年11月24日,他参军入伍。1916年3月29日,他加入了皇家野战炮兵第59训练旅,在此接受通信兵训练。随后,他被分配到西线战场第20(轻型)师所属的第92野战炮兵旅B炮兵连服役。

考克饶夫曾参与了向兴登堡防线的推进战役和第三次伊普尔战役。他申请转任军官并获批准。1918年2月,他被派往布莱顿学习炮兵知识,1918年4月前往北安普敦郡威登贝克的候补军官学校,接受野战炮兵军官培训。1918年10月17日,他被任命为皇家野战炮兵中尉。

战争结束后,考克饶夫于1919年1月从军队退役。他选择不返回曼彻斯特维多利亚大学,而是在曼彻斯特市立工艺学院学习电气工程。由于他已在曼彻斯特维多利亚大学完成一年学业,因此获准跳过课程的第一年。他于1920年6月获得理学士学位。该校电气工程教授迈尔斯·沃克(Miles Walker)说服他在大都会维克斯公司进行学徒训练。他获得了英国1851年博览会皇家委员会颁发的“1851年博览会奖学金”,并于1922年6月提交了硕士论文《交流电的谐波分析》。

随后,沃克建议考克饶夫申请剑桥大学圣约翰学院(沃克的母校)的奖学金。考克饶夫申请成功,获得了30英镑的奖学金和20英镑的助学金(发放给经济条件有限的本科生)。大都会维克斯同意向他提供50英镑,条件是他完成学业后返回公司任职。沃克和考克饶夫的一位姑妈帮助他凑齐了总计316英镑的学费。作为其他大学的毕业生,他获准跳过三一试第一年的课程。他于1924年6月参加了三一试考试,取得了B*等级并成为Wrangler(剑桥数学优等生),并获得了学士学位。

1925年8月26日,考克饶夫在托德莫登桥街联合卫理公会教堂与伊丽莎白·克拉布特里结婚。他们共育有六个孩子,第一个孩子是男孩,名为蒂莫西,不幸在婴儿期夭折。此后,他们育有四个女儿:琼·多萝西娅(Joan Dorothea,昵称Thea)、乔斯林、伊丽莎白·菲尔登、凯瑟琳·海伦娜,以及另一位儿子克里斯托弗·休·约翰。
\subsection{核研究}
\begin{figure}[ht]
\centering
\includegraphics[width=8cm]{./figures/966e76c528980952.png}
\caption{约翰·考克饶夫从两岁起直到28岁所居住的西约克郡沃尔斯登的住宅} \label{fig_YHkrf_2}
\end{figure}
在大都会维克斯公司研究主管和迈尔斯·沃克的推荐下,欧内斯特·卢瑟福同意接收考克饶夫到卡文迪许实验室担任研究生。1924年,考克饶夫以剑桥圣约翰学院基金奖学金和国家奖学金的资助身份,正式注册为博士研究生。在卢瑟福的指导下,他撰写了博士论文《分子流在表面凝结时出现的现象》,并发表在《皇家学会会刊》上。1928年9月6日,他获得了博士学位。在此期间,他曾担任俄罗斯物理学家彼得·卡皮察的助手,协助其进行极低温下磁场物理的研究工作,并帮助设计和建造了液化氦设备。

1919年,卢瑟福利用衰变镭原子释放的α粒子成功实现了氮原子的裂变。这项实验及后续实验为探索原子核结构提供了线索。为了进一步研究这一领域,卢瑟福需要一种能够以足够高的速度克服原子核电荷排斥力的人造粒子加速手段,这为卡文迪许实验室开辟了一条新的研究方向。他将这一课题分配给考克饶夫、托马斯·阿利博恩和欧内斯特·沃尔顿研究。他们随后建造了后来被称为“考克饶夫–沃尔顿加速器”的装置。马克·奥利芬特为他们设计了质子源。

一个关键时刻是考克饶夫阅读了乔治·伽莫夫关于量子隧穿效应的论文后,意识到由于这一现象,所需的加速电压比最初设想的要低得多。实际上,他计算出只需能量为30万电子伏特的质子即可穿透硼原子核。随后,考克饶夫和沃尔顿花了两年时间继续改进他们的加速器。卢瑟福从剑桥大学为他们申请到一笔1000英镑的经费,用于购买变压器和其他所需设备。

1928年11月5日,考克饶夫当选为剑桥大学圣约翰学院院士。1932年3月,他和沃尔顿开始操作他们的加速器,用高能质子轰击锂和铍。他们原本预期会观测到此前法国科学家报道过的伽马射线,但并未发现。1932年2月,詹姆斯·查德威克证明此前观测到的实际上是中子。随后,考克饶夫和沃尔顿转而寻找α粒子。1932年4月14日,沃尔顿用质子轰击锂靶时,注意到可能出现了α粒子。考克饶夫和随后赶来的卢瑟福确认了这一发现。当晚,考克饶夫和沃尔顿在卢瑟福家中撰写了致《自然》杂志的简报,宣布了他们的实验结果,即首次实现了对原子核的人造裂变,其反应式可描述如下:
$$
_3^7\text{Li} + p \rightarrow 2\,_2^4\text{He} + 17.2\,\text{MeV}~
$$
这一壮举被大众称作“劈开原子”。凭借这一成就,考克饶夫和沃尔顿于1938年获得休斯奖章,1951年获得诺贝尔物理学奖。他们随后继续利用质子、氘核和α粒子实现对碳、氮和氧的裂变,并证明他们已成功制备放射性同位素,包括碳-11和氮-13。
\begin{figure}[ht]
\centering
\includegraphics[width=8cm]{./figures/b8bbad429132038f.png}
\caption{考克饶夫–沃尔顿倍压电路} \label{fig_YHkrf_3}
\end{figure}
1929年,考克饶夫被任命为剑桥大学圣约翰学院机械科学导师。1931年被任命为物理导师,1933年成为初级财务主管,负责学院建筑物的维护,当时许多建筑因年久失修而状况不佳。学院的大门楼因死亡钟甲虫造成的破坏需要部分拆除修复,考克饶夫还监督了电力线路的重新布线工作。1935年,卢瑟福在卡皮察返回苏联后,任命他为蒙德实验室研究主任。他负责监督安装新的低温设备,并指导低温领域的研究工作。1936年,他当选为皇家学会院士;1939年当选为剑桥大学自然哲学杰克逊讲座教授,自1939年10月1日起生效。

考克饶夫和沃尔顿清楚他们的加速器存在局限性。在美国,欧内斯特·劳伦斯研发出一种更先进的设计,即回旋加速器。尽管卡文迪许实验室使用的加速器性能不如美国的先进设备,但凭借巧妙的实验物理,依然保持了领先。然而,考克饶夫敦促卢瑟福为卡文迪许实验室购置回旋加速器。卢瑟福因价格昂贵而犹豫不决,但奥斯汀勋爵捐赠的25万英镑,使得基于劳伦斯设计的36英寸(910毫米)回旋加速器得以建造,并修建了用于安置该设备的新楼翼。考克饶夫监督了这项工作。回旋加速器于1938年10月投入运行,新楼翼于1940年竣工。奥利芬特认为36英寸的回旋加速器规模不够大,于是在伯明翰大学开始建造更大型的60英寸回旋加速器。然而,1939年欧洲爆发第二次世界大战导致建设延迟,并且该设备在战后建成时已接近过时。
\subsection{第二次世界大战}
\begin{figure}[ht]
\centering
\includegraphics[width=8cm]{./figures/5becb6c17b46182e.png}
\caption{} \label{fig_YHkrf_4}
\end{figure}
第二次世界大战爆发时,考克饶夫出任英国供应部科研助理主任,负责雷达相关工作。1938年,亨利·提泽德爵士(Sir Henry Tizard)向考克饶夫展示了“链家”(Chain Home)系统,即由英国皇家空军(RAF)建设的一系列沿海预警雷达站,用于探测和追踪飞机。此后,他帮助调配科学家力量,使该系统全面投入运行。

1940年,他成为科研与技术发展咨询委员会成员。同年4月,他成为空战科学研究委员会成员,该委员会成立旨在处理“弗里施–佩尔斯备忘录”提出的相关问题(该备忘录计算出制造原子弹在技术上可行)。该委员会于1940年6月被“MAUD委员会”取代,考克饶夫也成为其成员之一。该委员会领导了英国早期开创性的原子能研究工作。1940年8月,考克饶夫作为“提泽德代表团”成员前往美国。由于英国虽然研发出许多新技术,但缺乏充足的工业能力加以充分利用,因此决定将这些技术与尚未参战的美国分享。提泽德代表团提供的信息包含了战时最重大的科学进展。共享的技术包括雷达技术,尤其是伯明翰大学奥利芬特团队设计的大幅改进的空腔磁控管(美国历史学家詹姆斯·巴克斯特三世称其为“有史以来运往美国最有价值的货物”),近炸引信设计、弗兰克·惠特尔喷气发动机的细节,以及描述原子弹可行性的“弗里施–佩尔斯备忘录”。除这些最重要的技术外,还包括火箭、增压器、瞄准具和潜艇探测设备等设计图纸。1940年12月,他返回英国。

回国后不久,考克饶夫被任命为汉普郡克赖斯特彻奇空防研究发展机构(ADRDE)主任。他的重点工作是利用雷达击落敌机。当时,GL Mk. III雷达被开发用于目标追踪和预测射击,但到1942年,美国开发用于同一目的的SCR-584雷达已可供使用,考克饶夫建议通过租借法案引进该设备。他主动购入SCR-584进行测试,并于1943年10月在谢佩岛进行的试验中,明确证明SCR-584性能更优。这使他在供应部内非常不得人心,但他掌握的情报显示,德国正计划部署V-1飞弹。1944年1月1日,罗纳德·维克斯中将紧急致电华盛顿,请求提供134套SCR-584雷达设备。
\begin{figure}[ht]
\centering
\includegraphics[width=6cm]{./figures/dc0d38e6ebff2f36.png}
\caption{} \label{fig_YHkrf_5}
\end{figure}