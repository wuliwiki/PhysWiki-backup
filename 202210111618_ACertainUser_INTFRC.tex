% 刚体的内力

%本文的“内力”指的不是在武侠类题材中、一种通过修炼内功,从自身气血或者外界灵气等转化而来的特殊能量.
\pentry{简单刚体系统的静力学分析\upref{RGDFA}}
\footnote{本文参考了张娟的《理论力学》}

在材料力学中,我们关心一个刚体内部的受力情况.本文简要探讨平面杆件某截面处内力的分析方法.

\begin{example}{强度理论}
强度理论认为,材料失效的原因是由于材料内部某处的内力超过了材料所能承受的极限.因此,分析材料内力对于设计或选用可靠的材料起重要作用.
\end{example}

\subsection{截面法 Section Method}
分析材料内力的常用方法是“截面法”,该方法非常的通俗易懂.

0. 先确定作用在该杆件上的外力\upref{RGDFA}.

1. 在需要计算内力的截面处,假想切割杆件,将其“一分为二”.
\begin{figure}[ht]
\centering
\includegraphics[width=8cm]{./figures/INTFRC_1.png}
\caption{假想切割刚体} \label{INTFRC_fig1}
\end{figure}

2. 选取左半段(或右半段)杆件.此时杆件截面处的内力变成了“外力”.类似于钉子模型\upref{RGDFA},内力的效果可看作一个平行于杆件拉力$F_T$、一个垂直于杆件的剪切力$F_S$与一个力偶$M$.
\begin{figure}[ht]
\centering
\includegraphics[width=8cm]{./figures/INTFRC_2.png}
\caption{画出杆件截面处的内力} \label{INTFRC_fig2}
\end{figure}

3. 根据力的平衡条件\upref{RGDFA},即可解出内力.

\subsection{内力分布图}
我们可以做出沿杆件方向,各截面处内力的分布情况.这便于看出内力的分布规律、最大最小值等.
\begin{figure}[ht]
\centering
\includegraphics[width=8cm]{./figures/INTFRC_3.png}
\caption{杆件内剪切力分布图示意图.蓝色为作用于刚体上的外力} \label{INTFRC_fig3}
\end{figure}

可以通过截面法分析各处的内力、并绘制内力分布图.

在外力分布规则的情况下,有一些辅助技巧可以帮助快速分析内力.

以下简要介绍内剪切力的绘制技巧,以分析左半段截面上的内剪切力为例.注意剪切力是有方向的,此处约定向上为正.
\begin{itemize}
\item 内剪切力从零开始,从左往右依次判断.
\item 截面每经过一个向上的外剪切力,内剪切力在数值上\textbf{减去}该外剪切力的绝对值大小.
\item 截面每经过一个向下的外剪切力,内剪切力在数值上加上该外剪切力的绝对值大小.
\item 若截面经过不受外剪切力的区域,内剪切力保持不变.
\item 如果计算正确,在最右端处,内剪切力应该为$0$(最右端处无外剪切力),或最右端的外jie qi.\footnote{\autoref{INTFRC_fig3} 中因为最右端还存在一个外剪切力,因此可以理解为在这个外剪切力的右端无限小处的内剪切力为$0$.或者说,在这种情况下,最右端处的内剪切力数值上等于该外剪切力.}
\end{itemize}
%截面每经过一个外剪切力,内剪切力在数值上再\textbf{减去}该外剪切力.即遇到向上的外剪切力(+F),内剪切力在向下方向上增大F,即在数值上减去F;反之,遇到向下的外剪切力(-F),内剪切力在向上方向上增大F,即在数值上加上F;若截面经过没有外剪切力的区域,内剪切力保持不变.
