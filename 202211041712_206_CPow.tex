% 幂函数(复数)
% keys 复数|幂函数|反函数

\begin{issues}
\issueDraft
\issueTODO
\end{issues}


\pentry{复变函数\upref{Cplx}}

\subsection{复数幂函数}

\subsubsection{定义}
任意给定$\alpha\in \mathbb{C}$,对于复变量$z\ne 0$,定义$z$的$\alpha$次幂函数为
\begin{align}
w&=z^\alpha\notag\\
&=\exp\{\alpha \ln z\}
\end{align}
\subsubsection{分解}
上面那个式子有点抽象,我们可以把它分解开来研究
\begin{align}
\ln z &= \ln |z|+i\phi(z)\notag\\
\alpha&=\alpha_R+i\alpha_I\notag\\
\end{align}

\begin{align}
z^\alpha&=\exp\{(\alpha_R+i\alpha_I)[\ln |z|+i\phi(z)]\notag\}\\
&=\exp\{[\alpha_R \ln|z|-\alpha_I\phi (z)]+i[\alpha_I\ln|z|+a_R\phi(z)]\}\notag\\
&=|z|^{\alpha_R}\E^{-\alpha_I\phi(z)}\cdot \E^{i[\alpha_I\ln(z)+\alpha_R\phi(z)]}
\end{align}

\begin{align}
\therefore |z^\alpha|&=|z|^{\alpha_R}\E^{-\alpha_I\phi(z)}\notag\\
\arg z^\alpha&=\alpha_I\ln(z)+\alpha_R\phi(z)\notag
\end{align}

\[\text{其中}\phi(z)=\arg z+2k\pi,k\in\mathbb Z\]

\subsubsection{分析}
幂函数特点:\\
\begin{itemize}
\item $z^\alpha$ 的模长和幅角都分别与 $z$ 和 $\alpha$ 有关\\
\item $z^\alpha$可能是单值的、有限多值或无限多值的函数,取决于$\alpha$
\end{itemize}

下面讨论不同的$\alpha$下幂函数的单值或多值性\\

\[
(1)~~\alpha_R=n,\alpha_I=0~~(n\in\mathbb Z)
\]
\begin{align}
|z^\alpha|&=|z|^{\alpha_R}\E^{-\alpha_I\phi(z)}\notag\\
&=|z|^{n}\notag\\
\arg z^\alpha&=\alpha_I\ln(z)+\alpha_R\phi(z)\notag\\
&=n(\arg z+2k\pi)\notag\\
\because e^{i2kn\pi}&=1\notag\\
\therefore z^\alpha&=|z|^{n}\cdot \E^{in\arg z}
\end{align}
\[\text{此时幂函数为单值函数}\]
\[
(2)~~\alpha_R=\frac 1 n,\alpha_I=0~~(n\in\mathbb Z)
\]

\begin{align}
|z^\alpha|&=|z|^{\alpha_R}\E^{-\alpha_I\phi(z)}\notag\\
&=|z|^{\frac 1 n}\notag\\
\arg z^\alpha&=\alpha_I\ln(z)+\alpha_R\phi(z)\notag\\
&=\frac 1 n(\arg z+2k\pi)\notag\\
\therefore z^\alpha&=|z|^{\frac 1 n}\cdot \E^{i(\frac {\arg z} n+2\frac k n \pi)}
\end{align}

 \\一般情况下, 这是一个比较复杂的函数, 含有不同的分支(因为 $\phi(z)$ 可以加整数个 $2\pi$).% 未完成: 分支是什么?
当且仅当 $a$ 为整数时才不会出现分支. 在数值计算中, 分支切割线出现在 $\phi(z) = \pm\pi$ 处, 这是因为数值计算通常取 $\phi(z)\in(-\pi, \pi]$.