% 样本均值与方差
% license Usr
% type Tutor

在实际场景中,我们经常遇到需要测量某一个量的分布情况的场景。例如,如果我们想要知道上海市所有人的身高分布情况,进行一次像人口普查一样大规模的调研显然是不现实的(上海人口已经来到接近2500万人)。此时更实际的做法是在虹桥火车站或者外滩附近随机采访100个人,统计他们的身高情况。如果将所有上海人的身高作为\textbf{总体},我们随机调查得到的数据就是一个\textbf{样本}。而我们想要知道,样本数据可以在多大的程度上反映全体数据的分布情况。

假设总体数据满足分布 $X\sim(\mu, \tau^2)$,我们从中抽取的样本 $\{X_i \}_{i=1}^n$。对于样本数据,可以计算其均值:\begin{equation}
\bar X=\frac{\sum_{i=1}^n X_i}{n}~
\end{equation}

我们当然希望样本均值可以无偏地反映总体均值 $\bar X=\mu$,但显然这是不可能的。事实上,由于 $\bar X$ 是样本的统计量,其本身也满足一定的分布。

\begin{theorem}{}
若总体满足分布 $X\sim (\mu, \tau^2)$,则其样本 $\{X_i \}_{i=1}^n$ 均值 $\bar X$ 满足分布 $\bar X\sim (\mu, \frac{\tau^2}{n})$
\end{theorem}
\textbf{证明:}

期望:
\begin{equation}
E(\bar X) =E(\frac{\sum_{i=1}^n X_i}{n}) 
        =\frac 1 n \sum_{i=1}^n E(X_i)=\mu~
\end{equation}

方差:
\begin{\begin{aligned}
D(\bar X) &= E()
\end{aligned}}

