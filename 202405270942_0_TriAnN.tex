% 三角定位笔记
% license Usr
% type Note

\begin{issues}
\issueDraft
\end{issues}

问题描述:若观测者在可以测量已知点之间的角度,已知观测点的 2D 或 3D 坐标,如何确定观测者的坐标?

\begin{itemize}
\item 如果只有三个点,可以用 “解三棱锥 1 (Matlab)\upref{Pmd1}” 的方法。 但符合的点不止一个。
\item 如果有多个点,还是用优化方法。目标函数取观测者的测量角度(因为测量误差也是角度)。
\item 但是该问题中,会不会存在局部最优解呢? 是否应该使用全局最优?
\end{itemize}

问题描述:若把上一个问题中的测量角度改为测量观测者和已知点之间的距离,又该如何?(GPS 定位问题)
