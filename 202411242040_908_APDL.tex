% 安培力定律(综述)
% license CCBYSA3
% type Wiki

本文根据 CC-BY-SA 协议转载翻译自维基百科\href{https://en.wikipedia.org/wiki/Amp\%C3\%A8re\%27s_force_law}{相关文章}。

在静磁学中,两根载流导线之间的吸引或排斥力(见下方第一幅图)通常被称为安培力定律。这种力的物理来源是每根导线根据毕奥-萨伐尔定律产生磁场,而另一根导线则根据洛伦兹力定律受到磁力作用。

\subsection{公式}
\subsubsection{特殊情况:两条直平行导线}
安培力定律中最为人熟知且最简单的例子(在2019年5月20日之前[1])用来定义电流的国际单位制(SI)单位\textbf{安培}。该定律表述,两条直平行导线之间的每单位长度的磁力为:
\[
\frac{F_m}{L} = 2k_{\text{A}} \frac{I_1 I_2}{r},~
\]
其中: \( k_{\text{A}} \) 是由毕奥-萨伐尔定律定义的磁力常数;\( \frac{F_m}{L} \) 是单位长度上的总磁力(在较短导线上,较长导线被近似为相对于较短导线无限长); \( r \) 是两导线之间的距离;\( I_1 \) 和 \( I_2 \) 是两导线中传输的直流电流。

这种公式在以下情况下是良好的近似:如果一根导线的长度远大于另一根,可以将较长的导线近似为无限长;
如果两根导线之间的距离相对于导线的长度较小(使得无限长导线近似成立),但同时相对于导线的直径又较大(使得导线可以被近似为无限细的线)。\( k_{\text{A}} \) 的值取决于所选的单位系统,而 \( k_{\text{A}} \) 的值决定了电流单位的大小。

在国际单位制(SI)中[2][3]:
\[
k_{\text{A}} = \frac{\mu_0}{4\pi},~
\]
其中:\( \mu_0 \) 是磁常数(在国际单位中,称为真空磁导率)。

SI单位中,磁常数的值为:
\[
\mu_0 = 1.25663706212(19) \times 10^{-6} \, \text{H/m}.~
\]
\subsubsection{一般情况}

针对任意几何形状的磁力的一般公式基于迭代线积分,并将毕奥-萨伐尔定律和洛伦兹力结合在一个公式中,具体如下:[4][5][6]
\[
\mathbf{F}_{12} = \frac{\mu_0}{4\pi} \int_{L_1} \int_{L_2} \frac{I_1 d\boldsymbol{\ell}_1 \times \left( I_2 d\boldsymbol{\ell}_2 \times \hat{\mathbf{r}}_{21} \right)}{|r|^2},~
\]
其中:
\begin{itemize}
\item \( \mathbf{F}_{12} \) 是由导线 2 对导线 1 施加的总磁力(通常以牛顿为单位测量);
\item \( I_1 \) 和 \( I_2 \) 分别是通过导线 1 和导线 2 的电流(通常以安培为单位测量);
\item 双重线积分表示导线 2 上的每个微元对导线 1 上的每个微元产生的磁力的总和;
\item \( d\boldsymbol{\ell}_1 \) 和 \( d\boldsymbol{\ell}_2 \) 是与导线 1 和导线 2 对应的微小向量(通常以米为单位),关于线积分的详细定义可参考相关资料;
\item \( \hat{\mathbf{r}}_{21} \) 是从导线 2 上的微元指向导线 1 上微元的单位向量,|r|** 是这两个微元之间的距离;
\item 符号 × 表示矢量叉乘;
\item 电流 \( I_n \) 的正负号取决于其与 \( d\boldsymbol{\ell}_n \) 的方向关系(例如,如果 \( d\boldsymbol{\ell}_1 \) 指向传统电流方向,则 \( I_1 > 0 \))。
\end{itemize}

若需确定材料介质中导线之间的磁力,则需要将磁常数 \( \mu_0 \) 替换为介质的实际磁导率。