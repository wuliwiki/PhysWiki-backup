% 格罗斯–皮塔耶夫斯基方程(综述)
% license CCBYSA3
% type Wiki

本文根据 CC-BY-SA 协议转载翻译自维基百科\href{https://en.wikipedia.org/wiki/Gross\%E2\%80\%93Pitaevskii_equation}{相关文章}。

格罗斯–皮塔耶夫斯基方程(Gross–Pitaevskii equation,简称 GPE,以尤金·P·格罗斯和列夫·彼得罗维奇·皮塔耶夫斯基命名)在哈特里–福克近似与伪势相互作用模型的基础上,用来描述由相同玻色子组成的量子体系的基态。

玻色–爱因斯坦凝聚是指一类玻色子气体,其所有粒子处于同一个量子态中,因此可以由相同的波函数描述。一个自由量子粒子可由单粒子薛定谔方程描述。对于现实气体中粒子间的相互作用,则需使用多体薛定谔方程加以考虑。在哈特里–福克近似中,整个由 $N$ 个玻色子组成的体系的总波函数 $\Psi$ 被认为是单粒子波函数 $\psi$ 的乘积:
$$
\Psi(\mathbf{r}_1, \mathbf{r}_2, \dots, \mathbf{r}_N) = \psi(\mathbf{r}_1)\psi(\mathbf{r}_2) \dots \psi(\mathbf{r}_N)~
$$

