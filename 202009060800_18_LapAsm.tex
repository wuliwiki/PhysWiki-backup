% 拉普拉斯方法

\pentry{渐近展开\upref{Asympt}, 高斯积分\upref{GsInt}, Gamma函数\upref{Gamma}}

在分析数学中, 拉普拉斯方法是一种计算含参数积分的渐近展开式的办法. 所考察的积分一般具有如下形式:
\begin{equation}\label{LapAsm_eq1}
I(\lambda)=\int_a^b \phi(x)\E^{\lambda f(x)}dx,
\end{equation}
其中$\lambda$是正的实参数. 要考察当$\lambda\to+\infty$时积分的行为. 拉普拉斯方法背后的想法很简单: 如果函数$f$在某点处达到极大值, 那么当$\lambda$很大时, 只有极大值点附近的贡献才比较可观, 其余部分相比起来都要小得多.

\subsection{Watson 引理}
拉普拉斯方法基于 Watson 引理. 它本身也是很有用的.

\begin{lemma}{Watson 引理}
设$\alpha>0$, 函数$\phi(x)$在区间$[0,b]$上连续, 而且$x\to0$时有渐近展开
$$
\phi(x)\simeq c_0+c_1x+c_2x^2+...
$$
则当$\lambda\to+\infty$时, 含参数的积分
$$
I(\lambda)=\int_0^b \phi(x)\E^{-\lambda x^\alpha}dx
$$
有渐近展开式
$$
sdf
$$
\end{lemma}

