% 相似变换和相似矩阵
% 相似变换|相似矩阵|酉矩阵|对角化|本征值

\pentry{酉矩阵\upref{UniMat}}

\begin{definition}{相似变换}
令 $\mat U$ 为 $N$ 维酉矩阵. $N$ 维方阵 $\mat A$ 的\textbf{相似变换}为
\begin{equation}
\mat B = \mat U \mat A \mat U\Her
\end{equation}
\end{definition}

注意酉矩阵满足 $\mat U\Her = \mat U^{-1}$.

由于酉矩阵乘以酉矩阵还是酉矩阵, 多次相似变换可以看作一次相似变换.

\subsection{对角化}
\pentry{厄米矩阵的本征问题\upref{HerEig}}
若相似变换可以使矩阵变为对角矩阵, 我们把这个过程称为对角化.

一个 $N$ 维矩阵可以被对角化当且仅当它是(实)对称矩阵或厄米矩阵.
\begin{equation}\label{MatSim_eq1}
\mat U\Her \mat A \mat U = \mat \Lambda
\end{equation}

对角化后, 对角矩阵 $\mat \Lambda$ 的对角元就是矩阵 $\mat A$ 的本征值 $\lambda_i$, $\mat U$ 的第 $i$ 列矢量就是 $\lambda_i$ 对应的本征矢. 所以我们时常把 “对角化” 作为 “解矩阵的本征方程” 的同义词.

(未完成)

\begin{example}{由本征值和本征矢求矩阵}
已知本征方程
\begin{equation}
\mat A \bvec v = \lambda \bvec v
\end{equation}
的 $N$ 个本征值和本征矢为 $\lambda_i$ 和 $\bvec v_i$, 求矩阵 $\bvec A$.

把\autoref{MatSim_eq1} 两边分别左乘 $\mat U$, 右乘 $\mat U\Her$, 得
\begin{equation}
\mat A = \mat U \mat \Lambda \mat U\Her
\end{equation}
\end{example}
