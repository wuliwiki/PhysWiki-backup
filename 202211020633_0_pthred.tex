% Pthreads 笔记

\begin{issues}
\issueDraft
\end{issues}

参考\href{https://www.geeksforgeeks.org/multithreading-c-2/}{这里}. 其中第二个例子(编译时使用 \verb|-lpthread| 选项):
\begin{lstlisting}[language=cpp]
#include <stdio.h>
#include <stdlib.h>
#include <unistd.h>
#include <pthread.h>

int g = 0;

void *myThreadFun(void *vargp)
{
	int *myid = (int *)vargp;
	static int s = 0;
	++s; ++g;
	printf("Thread ID: %d, Static: %d, Global: %d\n", *myid, ++s, ++g);
}

int main()
{
	int i;
	pthread_t tid;
	for (i = 0; i < 3; i++)
		pthread_create(&tid, NULL, myThreadFun, (void *)&tid);
	pthread_exit(NULL);
	return 0;
}
\end{lstlisting}

\begin{itemize}
\item \verb|pthread_create| 第一个参数会赋值给 \verb|tid|, 第二个参数是输入一些 attributes, 要默认就用 \verb|NULL|, 第三个参数是函数指针, 第四个是函数参数. 函数调用在 \verb|tid| 被赋值以后.
\item 其他的功能和 \verb|std::thread| 大同小异.
\end{itemize}

\addTODO{\lstinline|pthread_exit()| 是什么? \lstinline|joint()| 怎么实现? mutex 怎么实现? 见\href{https://www.geeksforgeeks.org/thread-functions-in-c-c/}{这里}, 以及\href{https://www.geeksforgeeks.org/mutex-lock-for-linux-thread-synchronization/}{这里}. 笔者的一个小程序 Exec 见\href{https://github.com/MacroUniverse/Exec/tree/master}{这里}.}
