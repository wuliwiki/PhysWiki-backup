% 泡利不相容原理(综述)
% license CCBYSA3
% type Wiki

本文根据 CC-BY-SA 协议转载翻译自维基百科\href{https://en.wikipedia.org/wiki/Pauli_exclusion_principle}{相关文章}。

\begin{figure}[ht]
\centering
\includegraphics[width=6cm]{./figures/5713325387bd5c08.png}
\caption{沃尔夫冈·泡利在1929年哥本哈根的一次讲座中。[1] 沃尔夫冈·泡利提出了泡利不相容原理。} \label{fig_PLBXR_1}
\end{figure}

在量子力学中,泡利不相容原理(德语:Pauli-Ausschlussprinzip)指出,在遵循量子力学定律的系统中,两个或多个具有半整数自旋(即费米子)的相同粒子不能同时占据相同的量子态。奥地利物理学家沃尔夫冈·泡利于1925年首先为电子提出了这一原理,随后在1940年通过自旋-统计定理将其推广至所有费米子。

对于原子中的电子,不可违背原理可以表述如下:在一个多电子原子中,不可能有两个电子的所有四个量子数取值都相同。这四个量子数分别是:主量子数 \( n \),角量子数 \( \ell \),磁量子数 \( m_\ell \),以及自旋量子数 \( m_s \)。例如,如果两个电子处于同一个轨道中,那么它们的 \( n \)、\( \ell \) 和 \( m_\ell \) 值相等。在这种情况下,它们的自旋量子数 \( m_s \) 值必须不同。由于自旋量子数 \( m_s \) 仅能取 \( +1/2 \) 或 \( -1/2\),因此其中一个电子必须具有 \( m_s = +1/2 \),另一个必须具有 \( m_s = -1/2 \)。

具有整数自旋的粒子(玻色子)不受泡利不相容原理的约束。任意数量的相同玻色子可以占据同一量子态,例如激光产生的光子或玻色-爱因斯坦凝聚态中的原子。  

更严格的表述是:在交换两个相同粒子的情况下,总(多粒子)波函数对于费米子是反对称的,而对于玻色子是对称的。这意味着,如果交换两个相同粒子的空间和自旋坐标,则总波函数对费米子会改变符号,而对玻色子不会改变符号。

因此,假设两个费米子处于相同的状态——例如,在同一原子的同一轨道上且具有相同的自旋——那么交换它们后系统不会发生任何变化,总波函数也应保持不变。然而,对于费米子来说,总波函数必须在交换两个粒子时改变符号,而同时又保持不变的唯一可能性是该波函数在所有地方都为零,这意味着这种状态不可能存在。这种推理对玻色子不适用,因为对于玻色子而言,交换粒子时波函数的符号不会改变。
\subsection{概述}  
泡利不相容原理描述了所有费米子(具有半整数自旋的粒子)的行为,而玻色子(具有整数自旋的粒子)遵循其他原则。费米子包括夸克、电子和中微子等基本粒子。此外,重子(如质子和中子,它们由三个夸克组成)以及某些原子(如氦-3)也是费米子,因此同样受泡利不相容原理的约束。  

原子的整体自旋可以不同,这决定了它们是费米子还是玻色子。例如,氦-3的自旋为\(1/2 \),因此是费米子,而氦-4的自旋为 0,因此是玻色子。泡利不相容原理支撑了日常物质的许多特性,从其大尺度稳定性到原子的化学行为。

半整数自旋意味着费米子的固有角动量值是\( \hbar = h / 2\pi \)(约化普朗克常数)的半整数倍(1/2、3/2、5/2 等)。在量子力学理论中,费米子由反对称态描述。相比之下,具有整数自旋的粒子(玻色子)具有对称的波函数,并且可以占据相同的量子态。玻色子包括光子、负责超导现象的库珀对,以及\(W\)和\(Z\)玻色子。费米子得名于服从费米–狄拉克统计分布,而玻色子则得名于玻色–爱因斯坦统计分布。
\subsection{历史}  
在20世纪初,人们发现具有偶数个电子的原子和分子在化学上比具有奇数个电子的原子和分子更稳定。例如,在1916年吉尔伯特·N·路易斯发表的文章《原子与分子》中,他提出了六条化学行为的基本假设,其中第三条指出,原子倾向于在任何给定的电子层中保持偶数个电子,特别是八个电子。他假设这些电子通常对称地排列在一个立方体的八个角上。1919年,化学家欧文·朗缪尔提出,如果原子中的电子以某种方式连接或聚集在一起,那么元素周期表就可以得到解释。当时,人们认为电子以一定的电子层(壳层)围绕原子核分布。1922年,尼尔斯·玻尔更新了他的原子模型,假设某些特定数量的电子(例如2、8和18)对应于稳定的“闭合壳层”。

泡利试图寻找这些数字的解释,这些数字最初只是基于实验观察的经验规律。与此同时,他也在尝试解释原子光谱学中的\textbf{塞曼效应}以及铁磁性的实验结果。他在1924年埃德蒙·C·斯托纳的一篇论文中发现了一个关键线索。斯托纳指出,在碱金属光谱的外磁场中,对于给定的主量子数\( n \),所有简并能级分裂后,单个电子的能级数目恰好等于相同\( n \)值下稀有气体(惰性气体)闭合电子层中的电子数目。这一发现使泡利意识到,若以四个量子数来定义电子态,那么复杂的闭合电子层电子数目可以归结为每个状态只能容纳一个电子的简单规则。为此,他引入了一个新的二值量子数,后来由萨缪尔·古兹密特和乔治·乌伦贝克确定为\textbf{电子自旋}。
\subsection{量子态对称性与泡利不相容原理的关系}  
在他的诺贝尔演讲中,泡利阐明了量子态对称性对不相容原理的重要性:  

在不同类别的对称性中,最重要的(而且对于两个粒子来说也是唯一的)是对称类和反对称类。在对称类中,当交换两个粒子的空间和自旋坐标时,波函数的值保持不变;而在反对称类中,这种交换会使波函数的符号发生改变……反对称类是泡利不相容原理的正确且普遍的波动力学表述。


具有单值多粒子波函数的泡利不相容原理等价于要求波函数对粒子交换具有反对称性。如果 \( |x\rangle \) 和 \( |y\rangle \) 代表描述单粒子系统的希尔伯特空间的基矢量,那么它们的**张量积**产生描述两粒子系统的希尔伯特空间的基矢量:\(|x, y\rangle = |x\rangle \otimes |y\rangle\)任何两粒子状态都可以表示为这些基矢量的叠加(即求和):  
\[
|\psi\rangle = \sum_{x,y} A(x,y) |x,y\rangle~
\]
其中,每个\(A(x, y)\)是一个复数标量系数。若波函数对交换运算反对称,则满足:\(A(x, y) = -A(y, x)\)这意味着当 \( x = y \) 时,必有 \( A(x, y) = 0 \),即泡利不相容原理。这一结论在任意基底下都成立,因为局部的基底变换不会改变反对称矩阵的反对称性。

反过来,如果在任意基底中,对角元素 \( A(x, x) \) 都为零,那么波函数分量  
\[
A(x, y) = \langle \psi | x, y \rangle = \langle \psi | ( |x\rangle \otimes |y\rangle )~
\]
必然是反对称的。为了证明这一点,考虑矩阵元  
\[
\langle \psi | ( (|x\rangle + |y\rangle ) \otimes (|x\rangle + |y\rangle ) )~
\]
这是零,因为两个粒子不可能同时处于叠加态 \( |x\rangle + |y\rangle \),因此该状态的概率为零。然而,这一表达式等于:  
\[
\langle \psi | x, x \rangle + \langle \psi | x, y \rangle + \langle \psi | y, x \rangle + \langle \psi | y, y \rangle.~
\]
其中,第一项 \( \langle \psi | x, x \rangle \) 和最后一项 \( \langle \psi | y, y \rangle \) 是对角元素,根据假设它们等于零。因此整个和也等于零。因此,波函数矩阵元满足:  
\[
\langle \psi | x, y \rangle + \langle \psi | y, x \rangle = 0~
\]
即  
\[
A(x, y) = -A(y, x)~
\]
对于\( n > 2 \)颗粒子的系统,多粒子基态变为\( n \)个单粒子基态的张量积,波函数的系数\(A(x_1, x_2, \dots, x_n)\)由\( n \) 个单粒子态确定。反对称性条件规定,当任意两个粒子态交换时,波函数系数必须翻转符号,即\(A(\dots, x_i, \dots, x_j, \dots) = -A(\dots, x_j, \dots, x_i, \dots), \quad \forall i \neq j\).泡利不相容原理是这一条件的直接推论:如果对于某些 \( i \neq j \),有\(x_i = x_j\),则必然有\(A(\dots, x_i, \dots, x_j, \dots) = 0\).这表明,在费米子系统中,所有\( n \)个粒子都不能处于相同的量子态。
\subsubsection{高级量子理论} 
根据自旋-统计定理,整数自旋的粒子占据对称量子态,而半整数自旋的粒子占据反对称量子态。此外,量子力学的基本原则只允许粒子的自旋取整数或半整数值。在相对论量子场论中,泡利不相容原理可以通过在虚时间下对半整数自旋粒子应用旋转算符推导出来。

在一维系统中,玻色子和费米子都可能服从泡利不相容原理。具有无限强\(\delta\)函数斥力的一维玻色气体等效于自由费米气体。其原因在于,在一维系统中,粒子的交换必须经过彼此,而如果斥力无限强,这种交换无法发生,从而导致玻色子表现出类似费米子的行为。该模型由量子非线性薛定谔方程描述。在动量空间中,对于具有有限斥力的玻色气体(\(\delta\) 函数相互作用),泡利不相容原理仍然成立。[9] 这一原理同样适用于一维相互作用自旋系统、哈伯德模型,以及其他可以通过\textbf{Bethe Ansatz}求解的模型。在这些模型中,基态等效于费米球。
\subsection{应用}  
\subsubsection{原子} 
泡利不相容原理有助于解释许多物理现象。其中一个特别重要的后果是复杂的电子层结构及原子如何共享电子,从而解释了元素的多样性及其化学结合方式。在电中性原子中,束缚电子的数目与原子核中的质子数相等。由于电子是费米子,它们无法占据与其他电子相同的量子态,因此在原子内部,电子必须“堆叠”起来,即在相同电子轨道上时具有不同的自旋,如下面所述。

例如,中性氦原子(He)具有两个束缚电子,这两个电子都可以占据最低能量(1s)态,但必须具有相反的自旋。由于自旋是电子量子态的一部分,因此这两个电子处于不同的量子态,不违反泡利不相容原理。然而,自旋只能取两个不同的值(本征值)。在锂原子(Li)中,存在三个束缚电子,但由于1s轨道最多只能容纳两个电子,第三个电子无法再占据 1s 轨道,必须进入能量更高的轨道。最低可用的能级是2s轨道,因此锂原子的基态电子排布为\(1s^22s\)。 同样,随着元素原子序数的增加,电子必须填充能量逐渐升高的电子层。元素的**化学性质主要取决于最外层电子数;尽管不同元素可能占据不同数量的电子层,但如果最外层电子数相同,它们的化学性质会相似,这便构成了元素周期表的基础。[10]: 214–218

为了检验氦原子是否遵守泡利不相容原理,戈登·德雷克[11]进行了精确计算,研究违反该原理的假想态,这些态被称为帕若尼克态。随后,K. Deilamian 等人[12] 利用原子束光谱仪搜索了德雷克计算出的帕若尼克态 \( 1s2s \,^1S_0 \)。然而,实验未能找到该状态,并显示其统计权重的上限为\( 5 \times 10^{-6} \)。(泡利不相容原理意味着该状态的权重应为零。)

