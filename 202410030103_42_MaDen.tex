% 度量空间的稠密性
% keys 稠密|可分空间
% license Usr
% type Tutor

\pentry{度量空间的闭包}{nod_8856}
在序集中根据偏序关系有稠密性的概念。比如有理数集是实数集的稠密子集的是指任意满足 $a<b$ 的两个实数 $a,b$,恒有有理数 $c$ 存在,使得 $a\leq c\leq b$。更一般的,偏序集 $B$ 是偏序集 $A$ 的稠密子集是指,任意 $A$ 中满足 $a<b$ 的两元素 $a,b$,恒有 $c\in B$,使得 $a\leq c\leq b$(见稠密性与完备性\upref{OrdCom})。也就是说:$B$ 是 $A$ 的稠密子集,相当于 $A$ 中的两元素“之间”都有 $B$ 的两元素将它们分隔。

同样,在度量空间中,根据“距离关系”也有稠密性的概念。类比序集中的稠密性概念,我们可以猜测:度量空间 $B$ 在 $A$ 中是稠密子集,是指在 $A$ 的任意两不同的元素之间恒有 $B$ 的元素将它们分隔。即,任意 $a\neq b$ 的两元素 $a,b\in A$,恒有 $c\in B$ 存在,使得 $d(a,c)+d(c,b)=d(a,b)$。若将 $a,b$ “分半”,取它们的“中间点”,则中间点和 $a,b$ 之间又有 $B$ 的元素将它们分隔。如此继续分下去,就会发现,任意 $A$ 的元素的任一邻域都必然存在 $B$ 的元素。即 $A$ 的每一点都是 $B$ 接触点。这就是说 $A\subset [B]$(闭包的概念)。

\begin{definition}{稠密子集}
设 $A,B$ 是度量空间 $X$ 的两个子集,且 $B\subset A$。若 $A\subset [B]$,则称 $B$ 是 $A$ 的\textbf{稠密子集}。若 $[A]$
\end{definition}