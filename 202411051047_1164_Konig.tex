% 质点系的动能、柯尼希定理
% keys 质点系|动能|柯尼希定理|动量|质心
% license Xiao
% type Tutor

\pentry{质点系的动量\nref{nod_SysMom}, 质心参考系\nref{nod_CMfram}}{nod_4709}

某参考系中,\textbf{质点系的动能}定义为每个质点的动能之和
\begin{equation}\label{eq_Konig_2}
E_k = \frac12 \sum_i m_i v_{i}^2 ~.
\end{equation}
其中 $m_i$ 是第 $i$ 个质点的质量, $v_i$ 是其速度的大小。

\subsection{柯尼希定理}

\begin{theorem}{柯尼希定理(König's theorem)}
某参考系 $S$ 中,质点系的动能(\autoref{eq_Konig_2})等于 $S$ 中质点系\textbf{质心的动能}加上质点系在\textbf{\enref{质心系}{CMfram} $S_c$ 中的动能},即
\begin{equation}\label{eq_Konig_1}
E_k = \frac12 Mv_c^2 + \frac12 \sum_i m_i v_{ci}^2 ~.
\end{equation}
其中 $M=\sum_i m_i$ 是所质点系的总质量, $v_c$ 是质心系相对于当前参考系的运动速度的大小, $m_i$ 是第 $i$ 个质点的质量, $v_{ci}$ 是第 $i$ 个质点在质心系中的速度。

所谓 “质心的动能”, 就是 $S$ 中质心处质量为 $M$ 的质点的动能, 即 $Mv_c^2/2$。
\end{theorem}
注意根据速度的叠加原理, 任意质点的三个速度矢量满足关系(\autoref{eq_Vtrans_1})
\begin{equation}\label{eq_Konig_3}
\bvec v_i = \bvec v_c + \bvec v_{ci}~.
\end{equation}

\begin{example}{圆环滚动的动能}
一个圆环在水平地面上延直线无摩擦不打滑地滚动, 其半径为 $R$, 质量为 $M$, 角速度为 $\omega$, 求地面参考系中圆环的动能。

解: 把圆环看成很多个小块, 每块看作一个质点 $m_i$, 每个质点相对于圆心旋转的线速度大小都是 $v_{ci} = \omega R$, 所以不打滑时地面相对于圆心平移的移动速度大小, 等于圆心相对于地面平移的速度大小, 同样是 $v_c = \omega R$, 代入\autoref{eq_Konig_1} 得动能为
\begin{equation}
E_k = \frac12 M v_c^2 + \frac12 \sum_i m_i v_{ci}^2 = \frac12 M\omega^2 R^2 + \frac12 \sum_i m_i\omega^2 R^2 = M\omega^2 R^2~.
\end{equation}
\end{example}

\subsection{证明}
在 $S$ 系中,根据\autoref{eq_Konig_2} 和\autoref{eq_Konig_3} 有
\begin{equation}
\ali{
E_k &= \frac12 \sum_i m_i \bvec v_{i}^2
= \frac12 \sum_i m_i (\bvec v_c + \bvec v_{ci} )^2 \\
 &= \frac12 \sum_i m_i \bvec v_{c}^2 + \frac12 \sum_i m_i \bvec v_{ci}^2 + \sum_i m_i \bvec v_c \vdot \bvec v_{ci}~.
}\end{equation}
对比\autoref{eq_Konig_1} 可知,现在只需证明 $\sum\limits_i m_i \bvec v_c \vdot \bvec v_{ci} = 0$ 即可。考虑到
\begin{equation}
\sum_i m_i \bvec v_c \vdot \bvec v_{ci}  = \bvec v_c \vdot \sum_i m_i \bvec v_{ci}~.
\end{equation}
而质心系中的质点系动量为零(\autoref{eq_CM_8}), 所以
\begin{equation}
\sum_i m_i \bvec v_{ci} = \bvec 0~.
\end{equation}
证毕。

