% 复旦大学 1997 量子真题
% license Usr
% type Note

\textbf{声明}:“该内容来源于网络公开资料,不保证真实性,如有侵权请联系管理员”

\begin{enumerate}
    \item 设体系处于 $\psi = C_1 \psi_1 + C_2 \psi_2$ 态 ($\psi_1$ 和 $\psi_2$ 正交归一化,即 $\langle \psi_1 | \psi_1 \rangle = \langle \psi_2 | \psi_2 \rangle = 1$ ),求
    
        \item $|C_1|^2$ 的可能测值及相应几率;(5分)
        \item $|C_2|^2$ 的可能测值及相应几率;(5分)
        \item $|C_1|^2 + |C_2|^2$ 的可能测值及相应几率。(10分)
    \end{enumerate}

    
    对于氢原子基态,计算 $\Delta X \cdot \Delta P_x$。(20分)
    \item 在 $\psi$ 态中,求 $Y$ 的表征值。(20分)
    \item 电子自旋计算。略去原子核,假定有如下 $\vec{B} = B_0 \vec{e}_z$,在 $t=0$ 时刻,电子的旋向沿 $y$ 轴取极值为 $\vec{B}_{ex} = B_1 \sin \omega t \vec{e}_x$。试问一级金智能解决方法:对 $t>t_1$,到磁旋方向的变化。
