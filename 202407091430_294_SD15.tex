% 苏州大学 2015 年硕士物理考试试题
% keys 苏州大学|考研|物理|2015年
% license Copy
% type Tutor

\begin{enumerate}
\item 如图1所示,质量为$m$半径为$R$的匀质圆盘,可以绕通过其中心且垂直于盘面的水平固定轴自由转动。今有一质量同为$m$的橡皮泥沿着通过圆盘中心并与盘轴垂直的水平方向飞来,在盘边缘处与静止的圆盘发生碰撞,并粘在边缘处。试求:\\
(1)该体系对于轴的转动惯量$I$;\\
(2)当橡皮泥与盘心的连线转到与水平线的夹角为$\theta$时,该系统的角加速度$\beta$;\\
(3)当橡皮泥与盘心的连线转到竖直时,该系统的角速度$\omega$;
\begin{figure}[ht]
\centering
\includegraphics[width=8cm]{./figures/dd459755f9a9c866.png}
\caption{} \label{fig_SD15_3}
\end{figure}
\item 质量为$0.5kg$的质点与一轻弹簧在一光滑的水平面上组成简谐振动系统,若周期为$2s$,振幅为$0.06m$,计时开始时($t=0$),质点恰好处在$x=0.03m$,并向$x$轴正向运动,试求:\\
(1)该质点的振动方程;\\
(2)初始位置处该系统的振动动能及振动势能。
\item 图2(a)、(b)分别表示某平面简谐波$t=0$时刻的波形图及$ P$点的振动图线,其它数据均在图中示出。试求:\\
(1)图中 $P$点的振动方程(余弦形式);\\
(2)此平面波的波动方程(余弦形式)。
\begin{figure}[ht]
\centering
\includegraphics[width=10cm]{./figures/c5605b9444a7c524.png}
\caption{} \label{fig_SD15_2}
\end{figure}
\item 在平直公路上行驶的某汽车,在刚刚关闭发动机时(作为计时起点)的速度为$ v_0$,之后仅在与速率成正比(比例系数为$a$)的阻力作用下行驶,直到停止。试求:\\
(1)关闭发动机后,$t$时刻汽车的速率$v(t)$;\\
(2)关闭发动机后,该车最多能够行驶的距离。
\item 总电量为$Q$的均匀带电细棒,弯成半径为$R$的半圆环,如图3所示,
试求:\\
(1)圆心处的电场强度;\\
(2)圆心处的电势(选无穷远为参考点);\\
(3)置于环心处试探电荷$q_0$所具有的电势能。
\begin{figure}[ht]
\centering
\includegraphics[width=6cm]{./figures/015c83aa9e946607.png}
\caption{} \label{fig_SD15_1}
\end{figure}
\item 电路参数如图4所示,试计算各条支路中的电流。
\begin{figure}[ht]
\centering
\includegraphics[width=6cm]{./figures/463686a850a27d87.png}
\caption{} \label{fig_SD15_4}
\end{figure}
\item 如图5所示两个很长的导线沿圆环的直径(半径为 a)方向与导体环相连,此环上、下半环的电阻率之比为1:2,截面积相同。若直导线中电流为$I$,试求:\\
(1)流经上、下半环各自的电流;\\
(2)环心处的磁感应强度$ B$的大小及方向。
\begin{figure}[ht]
\centering
\includegraphics[width=8cm]{./figures/b7aa6d92337f5529.png}
\caption{} \label{fig_SD15_5}
\end{figure}
\item 载有电流为$I$的长直导线旁平行且共面地置一单匝矩形平面线圈,线圈高度为$L$。当线圈以速率$v$向右平动,并达到图6示位置时。试求:\\
(1)线圈中的感应电动势大小和方向;\\
(2)导线与线圈之间的互感系数。
\item 白光(波长范围$400nm~760nm$)垂直照射在厚度为 $4nm $的肥皂薄膜(折射率为$1.33$)上,若在反射光中观察肥皂膜,哪些波长的光被加强?
\end{enumerate}