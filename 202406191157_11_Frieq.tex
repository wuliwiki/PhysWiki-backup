% 宇宙学的基本方程
% license Usr
% type Tutor

\subsection{弗里德曼方程和加速度方程}
\textbf{弗里德曼方程(Friedmann Equation)}是爱因斯坦方程的时间部分,即:
\begin{equation}G_{00}+\Lambda g_{00}=8\pi GT_{00}\quad\Rightarrow\quad\left(\frac{\dot{a}}{a}\right)^2=H^2=\frac{8\pi G}{3}\rho+\frac{\Lambda}{3}-\frac{k}{a^2}~,\end{equation}
可见,如果宇宙常数项足够大,$\dot a$将永不为0,宇宙将一直膨胀。而如果宇宙常数项很小,或者为负值,宇宙则有可能在将来经历停止膨胀,然后收缩的命运(比之如今的宇宙热历史理论,该方程是时间反演不变的。)\footnote{有一个理论是宇宙常数并非“常数”,而是缓慢变化的,称这种可能性为quintessence。}

对上式进行求导,便得到\textbf{加速度方程(acceleration equation)}:

\begin{equation}
\Rightarrow\quad\frac{\ddot{a}}{a}-\frac{\Lambda}{3}=-\frac{4\pi G}{3}(\rho+3P)~,
\end{equation}
又称作\textbf{雷乔杜里方程(Raychaudhuri equation)}。丛该方程上看,$\rho,P$的作用是使膨胀减速,可以理解为引力作用;宇宙常数项则能促进宇宙膨胀。

\subsection{连续性方程}
由爱因斯坦方程一节可知,四动量守恒方程在弯曲流形的拓展为
\begin{equation}
\nabla_{\mu}T^{\mu\nu}=0~.
\end{equation}
将能动张量写为$(1,1)$型张量,再用Christoffel 符号表示其协变微分,则有

\begin{equation}\nabla_{\mu}T^{\mu}_\nu=\partial_{\mu} T_{\nu}^\mu+\Gamma_{\alpha\mu}^\mu T_{\nu}^\alpha-\Gamma_{\nu\mu}^\alpha T_{\alpha}^\mu=0~.
\end{equation}
代入具体联络的具体数值:
\begin{equation}\Gamma_{00}^0=0\quad;\quad\Gamma_{01}^1=\Gamma_{02}^2=\Gamma_{03}^3=\frac{\dot{a}}{a}~,\end{equation}
便得到\textbf{连续性方程(continuity function)}——
\begin{equation}

\end{equation}