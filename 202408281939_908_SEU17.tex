% 东南大学 2017 年 考研 量子力学
% license Usr
% type Note

\textbf{声明}:“该内容来源于网络公开资料,不保证真实性,如有侵权请联系管理员”

\subsection{1}
\subsection{(共 30分,每小题3分)选样题}
\begin{enumerate}
    \item $\{x_i, p_j\}$ 的共同本征函数为 
(a) $\delta(x-a)\delta(y-b)\quad  (b) \delta(x-a)\delta(y-b) \quad (c) \delta(x-a)\delta(p-b) \quad (d) \delta(p-a)\delta(p-b)$
    \item 无自旋单粒子在 $xy$ 平面内运动,力学量完全集可选为 
(a) $\{p_x, p_y\} \quad (b) \{r_x, r_y\}\quad (c) \{r_x, r_y, p_x, p_y\}\quad (d) \{L_z, p_x, p_y\}$
    \item $A$ 和 $\beta$ 均为线性算符,$[\alpha, \beta]$ 可为 
(a) $A\beta + \beta A\quad (b) A\beta\quad (c) \beta A + \beta^2\quad (d) A\beta - \beta A$
    \item 设 $\sigma = |\psi\rangle \langle\phi|$,则 $\sigma$ 可为\\ 
(a)$|\phi\rangle \langle\psi| \quad (b) -\langle\phi|\psi\rangle \quad (c) \langle\phi|\psi\rangle\quad d) |\psi\rangle \langle\phi|$
     \item 设 \\( S \\) 为电子自旋向上态的最小投影数,视为:
\end{enumerate}

