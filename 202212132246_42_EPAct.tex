% 端点可变的作用量
% keys 作用量的偏导数|作用量的全微分
\pentry{作用量原理\upref{HamPrn}}
本词条将证明几个个关于作用量的重要公式,其中两个是重要的,即作用量对时间的偏导数等于负的能量(哈密顿量),对坐标的偏导数等于动量对应分量,或说作用量的梯度等于动量.其数学形式为
\begin{equation}\label{EPAct_eq1}
\begin{aligned}
\pdv{S}{t}&=-H\\
\pdv{S}{q^i}&=p_i
\end{aligned}
\end{equation}
其中 $t$ 是作用量积分里的积分上限(末时刻) $t_2$,$q^i$ 是对应末时刻 $t$ 的广义坐标.

并且
\begin{equation}\label{EPAct_eq2}
\dd S=\sum_i p_i^{(2)}\dd {q^i}^{(2)}-H^{(2)}\dd t^{(2)}-\sum_i p_i^{(1)}\dd {q^i}^{(1)}+H^{(1)}\dd t^{(1)}
\end{equation}
这里 $(1),(2)$ 分别代表起点和终点对应值.

既然\autoref{EPAct_eq1} 是关于终点的偏微分,\autoref{EPAct_eq2} 是关于两端点的微分,这就是说这里的作用量实际上是端点可变的作用量.

\subsection{证明:}
这里的公式事实上和变分学的端点可变问题\upref{EPQue} 中的一样,那里有更严格的证明,只需明确物理意义即可.然而,我们这里给出较之更适合物理人的证明,以避免深入了解变分学.

