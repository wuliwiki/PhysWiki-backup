% C++ 标准库笔记

\begin{issues}
\issueDraft
\end{issues}

\subsection{string}
\begin{itemize}
\item 要 \verb`#include <string>`, \verb`use std::string`.
\item constructors: \verb`string s1;`, \verb`string s2{s1};`, \verb`string s3{"something"};`, \verb`string s4(10, 'c');`
\item \verb`getline(cin, str);` 从命令行读取一行字到 \verb`string str` 中.
\item \verb`string::empty()` 判断是否是空.
\item \verb`string::size()` 返回字符数.
\item \verb`str[n]` 可以直接读取或赋值某个字符.
\item \verb`str1 + str2` 可以连接两个字符串.
\item \verb`str1 + "something"` 可以连接 string 和 literal, 但不能是 \verb`wchar_t`.
\item \verb`str1 == str2`, \verb`str1 != str2` 可以比较字符串是否相同.
\end{itemize}

\subsection{vector}
\begin{itemize}
\item \verb`v.size()` 检查大小
\item \verb`v.empty()` 检查是否为空
\item range based for loop \verb`for (auto &i : v)`
\item iterator 类似指针, \verb`auto a = v.begin()`, 生成第一个元素的 iterator, \verb`v.end()` 生成最后一个元素后的一个 iterator. \verb`*a` 获取 a 指向的元素, \verb`*(a + n)` 获取 a 后面的第 n 个元素. \verb`==` 只有在 iterator 指向同一个 vector 的同一个元素时才成立.
\end{itemize}

\subsection{ctime}
\begin{itemize}
\item \verb`clock()` 用于测量 CPU 时, 而不是真正的时间. 如果在 ubuntu 下用 OpenMP, 时间将是所有 CPU 的累加.
\begin{lstlisting}[language=cpp]
clock_t start, stop;
start = clock();
...
stop = clock();
\end{lstlisting}
\item \verb`time()` 用于测量从 1970 年某时刻起流逝的秒数, 但仅限于整数秒.
若要精确测量物理时间, 用 chrono, 这个头文件用起来要复杂得多, 见我在 nr3plus.h 中写的 tic(), toc() 函数.
\end{itemize}

\subsection{algorithm}
\begin{itemize}
\item 提供 \verb`max()`, \verb`min()`, \verb`swap()` 交换两个变量值.
\end{itemize}


\subsection{stdlib}
\begin{itemize}
\item \verb`int system("命令")` 在调用程序的环境(如 linux 的 bash,windows 的 cmd)中执行命令,例如获得文件列表 \verb`ls`, 当前目录 \verb`pwd` 等. 但这些命令返回的值不能直接获得,需要 redirect 到文件再读取文件.
\item \verb`int remove("文件名")` 可以删除文件.
\end{itemize}
