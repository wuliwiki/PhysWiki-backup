% CLion 笔记
% license Xiao
% type Note

\begin{issues}
\issueDraft
\end{issues}

本文假设我们在 linux 桌面环境(或 Windows 的 WSL)下使用 CLion。

\subsection{常识}
\begin{itemize}
\item 可以直接通过 CMakeList.txt 创建工程, 工程所有的信息都在 .idea 文件夹里面
\item 设置里面 keymap 可以把快捷键直接设置成和 Visual Studio 相同
\item 用 gdb 调试时可以直接使用 gdb 命令行
\end{itemize}

\subsection{安装}
\begin{itemize}
\item \href{https://www.jetbrains.com/help/clion/installation-guide.html}{安装文档}, 推荐用 Standalone installation
\item 直接\href{https://www.jetbrains.com/clion/download/#section=linux}{下载}安装包 \verb`CLion-2022.2.tar.gz` 解压 \verb`sudo tar -xvzf CLion-*.tar.gz -C /opt/` (WSL2 和 Ubuntu 20/22 亲测成功)
\item Ubuntu 上安装也可以直接用 \verb`sudo snap install clion --classic`
\item  启动用 \verb`sh /opt/clion-*/bin/clion.sh`, 可以在 \verb`~/.bashrc` 中设置 alias, 需要桌面快捷方式在菜单栏的 \verb`Tools -> Create Desktop Entry` 设置
\item 运行用 \verb`/snap/clion/203/bin/clion.sh`, 可以在 bashrc 中设置 alias
\item 不要用软件自带的更新功能,applying patch 非常缓慢,直接卸载,下载最新版本安装要快得多。
\end{itemize}


\subsection{其他}
\begin{itemize}
\item \verb`.idea/workspace.xml` 文件挺重要的, 可以从 \verb`.idea/.gitignore` 文件中注释掉
\item Windows 中可以选 WSL 环境编译, 直接用 g++, clang++, 和 gdb
\item 可以 MobaXterm 远程 GUI(亲测)。
\item 比起远程 GUI,更好的办法是在本地装 GUI 然后在开始界面选择 Remote Development 的 SSH。 只要能连 ssh 就可以远程调试。


\item CLion 可以不设置任何工程, 仅仅用于 debug 可执行文件(可以在其他任何地方编译), 见\href{https://www.jetbrains.com/help/clion/debug-arbitrary-executable.html}{官方文档}。
\end{itemize}

\subsection{调试已经编译好的可执行文件}\label{sub_CLionN_1}
\begin{itemize}
\item 创建自定义 Target: \verb`File > Settings > Build, Execution, Deployment > Custom Build Targets`
\item 按 \verb`+` 按钮
\item 命名为 "Debug Existing Executable",保存关闭
\item 打开 \verb`Run > Edit Configurations`
\item 按 \verb`+`,选择 "Custom Build Application"
\item Target 中选择 Debug Existing Executable
\item Executable 选择要 debug 的可执行文件
\item Program arguments: 命令行选项
\item Working directory: 工作目录
\item Environment variables: 环境变量
\end{itemize}

\subsection{简单的 cmake 文件}
如果你只用 \verb`Makefile` 不想迁移到 \verb`cmake` 也没关系,你只需要写一个很简单的 wrapper,就可以用它作为 CLion 的工程文件。 一个简单的例子:
% cmake
\begin{lstlisting}[language=none]
# just a wrapper for Makefile for CLion
cmake_minimum_required(VERSION 3.10)
project(FEDVR-TDSE)

# ==== wrapper of Makefile ===
add_custom_target(my_targ1
  COMMAND make -f make/all.mak opt_debug=true main1.x
  WORKING_DIRECTORY "${CMAKE_CURRENT_SOURCE_DIR}"
)

add_custom_target(my_targ2
  COMMAND make -f make/all.mak opt_debug=true main2.x
  WORKING_DIRECTORY "${CMAKE_CURRENT_SOURCE_DIR}"
)


# ==== dummy settings to make CLion happy ===
set(CMAKE_CXX_STANDARD ${opt_std})
set(CMAKE_CXX_STANDARD_REQUIRED ON)
set(CMAKE_CXX_EXTENSIONS OFF)

file(GLOB ALL_HEADERS
    "${PROJECT_SOURCE_DIR}/headers/*.h")

message(STATUS "${ALL_HEADERS}")

add_executable(dummy1
    ${PROJECT_SOURCE_DIR}/main1.cpp ${ALL_HEADERS})

add_executable(dummy2
    ${PROJECT_SOURCE_DIR}/main2.cpp ${ALL_HEADERS})
\end{lstlisting}

\subsubsection{cmake 环境}
\begin{itemize}
\item \verb`Settings > Build,Execution,Development > Toolchains` 选择编译环境, 如 WSL, Mingw, Docker, SSH 等。 会检测 CMake 的版本以及 GDB 的版本。
\item \textbf{SSH} 设置中甚至可以自动读取 \verb`~/.ssh/config` 中的 host 设置, 但是第一次编译时可能要稍微花点时间把源码传到 remote host (目录如 \verb`/tmp/tmp.TH5zt7WrwC`)。 另外仍然不会运行 \verb`~/.bashrc`, 需要在 CLion 中手动添加环境变量。
\item \textbf{MinGW} 设置时选择自己安装的 MSYS2: 路径不要填 exe 文件,而是 exe 同目录的 mingw64 文件夹,如 \verb`C:\...\msys64-20240712\mingw64`
\item 下面的 CMake 面板里面的齿轮按钮可以设置 \verb`cmake`。 也可以用 \verb`Settings > Build,Execution,Development > CMake` 进入。 包括 Build options \verb`-- -j 16`, 运行 \verb`cmake` 时的环境变量(例如 \verb`CPATH, LIBRARY_PATH`)。
\item \textbf{Remote Development} 在 File 菜单中选这个可以用 SSH 直接连到 Linux 服务器, 在服务器的文件系统中打开项目并调试。 第一次使用会自动在服务器下载安装 backend, 在本地安装 client。
\item Linux/WSL 环境下 CLion 通过 \verb`sh` 运行 CMake, 而且应该是新开的 session,所以还是不要试图通过脚本设置环境变量了
\end{itemize}

\subsubsection{运行环境}
\begin{itemize}
\item 上面菜单栏的 Select Run Debug Configuration 可以设置运行时的 working dir, arguments, input file, 环境变量(例如 \verb`LD_LIBRARY_PATH`)。
\item 注意如果你在 \verb`~/.bashrc` 中设置的环境变量,如果用菜单栏打开 CLion,并不会生效。 所以此时在命令行中启动 CLion 即可省略手动设置。
\item 为了方便在命令行启动 clion, 可以在 \verb`~/.bashrc` 中添加 \verb`alias clion="sh /opt/clion-*/bin/clion.sh"`。
\end{itemize}

\subsection{设置}
\begin{itemize}
\item 粘贴的时候会自动 indent 有时候比较难受, 在 \verb`Editor > General > Smart Keys > Reformat on paste` 里面设置。
\end{itemize}
