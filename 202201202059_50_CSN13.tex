% 2013 年计算机学科专业基础综合全国联考卷
% keys 计算机 考研 全国卷

\subsection{一、单项选择题}
1~40小题,每题2分,共80分.下列每题给出的四个选项中,只有一个选项符合要求.

1. 已知两个长度分别为m和n的升序链表,若将它们合并为一个长度$m+n$的降序链表,则最坏情况下的时间复杂度是: \\
A. $O(n)$ $\qquad$ B.$O(m \times n)$ $\qquad$ C.$O(min(m,n))$ $\qquad$ D.$O(max(m,n))$

2. 一个栈的入栈序列为$1,2,3,,n$,其出栈序列是$p_1, p_2, p_3, ...$.若$p_2=3$,则$p_3$可能取值的个数是: \\
A. $n-3$  $\qquad$  B.$n-2$ $\qquad$ C.$n-1$ $\qquad$ D.无法确定

3.若将关键字1,2,3,4,5,6,7依次插入到初始为空的平衡二叉树$T$中,则$T$中平衡因子为0的分支结点的个数是 \\
   A. 0 $\qquad$ B. 1 $\qquad$ C. 2 $\qquad$ D. 3

4.已知三叉树$T$中6个叶结点的权分别是2,3,4,5,6,7,$T$的带权(外部)路径长度最小是 \\
   A. 27 $\qquad$ B. 46 $\qquad$ C. 54 $\qquad$ D. 56 

5.若$X$是后序线索二叉树中的叶结点,且$X$存在左兄弟结点$Y$,则$X$的右线索指向的是 \\
A.$X$的父结点 \\
B. 以$Y$为根的子树的最左下结点 \\
C.$X$的左兄弟结点$Y$ \\
D.以$Y$为根的子树的最右下结点

6.在任意一棵非空二叉排序树$T_1$中,删除某结点$v$之后形成二叉排序树$T_2$,再将$v$插入$T_2$形成二叉排序树$T_3$.下列关于$T_1$与$T_3$的叙述中,正确的是  \\
I.若$v$是$T_1$的叶结点,则$T_1$与$T_3$不同 \\
II.	若$v$是$T_1$的叶结点,则$T_1$与$T_3$相同 \\
III.若$v$不是$T_1$的叶结点,则$T_1$与$T_3$不同 \\
IV.	若$v$不是$T_1$的叶结点,则$T_1$与$T_3$相同 \\
A. 仅 I、III $\qquad$ B. 仅 I、IV $\qquad$ C. 仅 II、III $\qquad$ D. 仅 II、IV

7.设图的邻接矩阵$A$如下所示.各顶点的度依次是 \\
\begin{equation}
A=\begin{bmatrix}
0 & 1 & 0 & 1 \\
0 & 0 & 1 & 1 \\
0 & 1 & 0 & 0 \\
1 & 0 & 0 & 0
\end{bmatrix}
\end{equation}
A. 1,2,1,2 $\qquad$ B. 2,2,1,1 $\qquad$ C. 3,4,2,3 $\qquad$ D. 4,4,2,2

8.若对如下无向图进行遍历,则下列选项中,\textbf{不}是广度优先遍历序列的是 \\
\begin{figure}[ht]
\centering
\includegraphics[width=12.5cm]{./figures/CSN13_2.png}
\caption{第8题图} \label{CSN13_fig2}
\end{figure}
A. h,c,a,b,d,e,g,f $\qquad$ B. e,a,f,g,b,h,c,d \\
C. d,b,c,a,h,e,f,g $\qquad$ D. a,b,c,d,h,e,f,g 

9.下列AOE网表示一项包含8个活动的工程.通过同时加快若干活动的进度可以缩短整个工程的工期.下列选项中,加快其进度就可以缩短工程工期的是 \\
\begin{figure}[ht]
\centering
\includegraphics[width=12.5cm]{./figures/CSN13_1.png}
\caption{第9题图} \label{CSN13_fig1}
\end{figure}
A.c和e  $\qquad$ B.d和e $\qquad$ C.f和d $\qquad$ D.f和h

10.	在一株高度为2的5阶B树中,所含关键字的个数最少是 \\
A.5                   B. 7                C. 8                D. 14 

11.	对给定的关键字序列110,119,007,911,114,120,122进行基数排序,则第2趟分配收集后得到的关键字序列是 \\
A. 007,110,119,114,911,120,122 $\qquad$ B. 007,110,119,114,911,122,120 \\
C. 007,110,911,114,119,120,122 $\qquad$ D. 110,120,911,122,114,007,119

12.	某计算机主频为1.2GHz,其指令分为4类,它们在基准程序中所占比例及CPI如下表所示. \\
 \begin{table}[ht]
 \centering
 \caption{第12题表}\label{CSN13_tab1}
 \begin{tabular}{|c|c|c|}
 \hline
 指令类型 & 所占比例 & $CPI$ \\
 \hline
 $A$ & $50\%$ & $2$ \\
 \hline
 $B$ & $20\%$ & $3$ \\
 \hline
 $C$ & $10\%$ & $4$ \\
 \hline
 $D$ & $20\%$ & $5$ \\
 \hline
 \end{tabular}
 \end{table}
该机的MIPS数是  \\
A.100 $\qquad$ B.200 $\qquad$ C.400 $\qquad$ D.600

13.	某数采用IEEE 754单精度浮点数格式表示为C640 0000H,则该数的值是 \\ 
A. $-1.5\times2^{13}$ $\qquad$ B.$-1.5\times2^{12}$ $\qquad$ C. -0.5\times213 $\qquad$ D.$-0.5\times2^{12}$

14.	某字长为8位的计算机中,已知整型变量x、y的机器数分别为[x]补=1 1110100,[y]补=1 \\
0110000.若整型变量z=2*x+y/2,则z的机器数为 \\
    A. 1 1000000         B. 0 0100100         C. 1 0101010          D. 溢出 \\

15.	用海明码对长度为 8 位的数据进行检/纠错时,若能纠正一位错.则校验位数至少为 \\ 
A. 2                 B. 3                 C. 4                 D. 5 

