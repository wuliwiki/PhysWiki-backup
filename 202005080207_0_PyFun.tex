% Python 的函数

\subsection{函数}
Python 中的函数与数学中的函数不完全一样, 函数可以有若干个输入变量和输出变量(也可以没有). 下面我们定义一个简单的函数来计算长方形的面积
\begin{lstlisting}[language=python]
def f(a, b, c):
    volumn = a*b*c
    return volumn
\end{lstlisting}
这段代码用到了两个 Python 的\textbf{关键字(keyword)} \verb|def| 和 \verb|return|. 关键字是指在程序中有特殊含义的单词, 不能作为变量名和函数名的名称. 其中 \verb|def| 用于定义函数, \verb|f| 是函数名, \verb|a|, \verb|b| 和 \verb|c| 分别是函数的\textbf{输入变量(argument)}. 冒号以后是\textbf{函数体}, 可以有若干行命令. 注意这些命令前面必须有\textbf{缩进(indentation)}.  在以上代码中, 函数体的第一行计算面积, 第二行将用关键字 \verb|return| 将计算的结果作为输出并退出函数.

现在我们可以使用这个函数, 使用方法和 \verb|sin|, \verb|sqrt| 等数学函数一样, 只是不同输入变量要用逗号隔开.
\begin{lstlisting}[language=python]
V = f(1.2, 3.4, 6)
print(V)
\end{lstlisting}
