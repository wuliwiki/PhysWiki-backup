% 动态场问题(电动力学)(摘要)

\begin{issues}
\issueDraft
\issueOther{需要check}
\end{issues}

\pentry{静电场与静磁场(摘要)\upref{estfid}}

\subsection{电磁感应}
在理论与现实的诘难下\footnote{理论的诘难是,如果你对静磁旋度方程两边取旋度$\curl \bvec B = \mu_0 \bvec j$,你会发现 $\div (\curl \bvec B) = \mu_0 \div \bvec j$,根据矢量运算法则,这意味着 $\div \bvec j = 0$,与电荷守恒方程 $\div \bvec j +\pdv{q}{t} = 0$矛盾;现实的诘难是,人们发现在变化磁场中的线圈中会产生电流,而这是无法靠洛伦兹力公式解释的:磁场不能驱动静止的电荷,也不能对其做功。},法拉第与麦克斯韦等人前仆后继地提出了电磁感应定律,即耳熟能详的“磁生电、电生磁”。电磁感应定律引入了两个含时项,如下表所示:

\begin{figure}[ht]
\centering
\includegraphics[width=10 cm]{./figures/e9265eb3665eb1d0.pdf}
\caption{不大严谨的电磁感应示意图} \label{fig_dynfld_1}
\end{figure}
%我不大确定产生的感应场具体是不是这样的,但我觉得至少这样很直观...

\begin{table}[ht]
\centering
\caption{电场与磁场的旋度(无源)}\label{tab_dynfld1}
\begin{tabular}{|c|c|}
\hline
电场 $\bvec E$ & 磁场 $\bvec B$ \\
\hline
$$\oint \bvec E \cdot \dd {\bvec l} = -\int \pdv{\bvec B}{t} \cdot \dd {\bvec A}~$$
$$\curl \bvec E = -\pdv{\bvec B}{t}~$$
法拉第定律:变化的磁场感应出电场;楞次定律:$-\pdv{\bvec B}{t}$前的负号,代表一种阻碍作用 \upref{FaraEB}
 & 
 $$ \oint \bvec B \cdot \dd {\bvec l} = \mu_0 \epsilon_0 \int \pdv{\bvec E}{t} \cdot \dd {\bvec A}~$$
 $$\curl \bvec B = \mu_0 \epsilon_0 \pdv{\bvec E}{t}~$$ 
 安培-麦克斯韦定律:变化的电场感应出磁场 \upref{DisCur}
 \\
\hline
\end{tabular}
\end{table}

这两个含时项的内涵远比他表面上看起来的深远,所有对于变化电磁场的理解,从光学到电路到无线通信,都离不开这两个含时项。可以说,这两个含时项直接翻开了人类历史的新篇章,\textsl{以及电动力学的后大半本书。}

\subsection{Maxwell方程}
结合电磁感应定律与原有的静场方程,我们就得到了如雷贯耳的Maxwell方程组 \upref{MWEq}:
\begin{table}[ht]
\centering
\caption{Maxwell 方程组}\label{tab_dynfld2}
\begin{tabular}{|c|c|c|}
\hline
&电场 $\bvec E$ & 磁场 $\bvec B$ \\
\hline
散度方程 & 
$$\div \bvec E = \frac{\rho}{\epsilon_0}~$$ 
&
$$\div \bvec B = 0~$$\\
\hline
旋度方程 & 
$$ \curl \bvec E = -\pdv{\bvec B}{t} ~$$
 &
$$\curl \bvec B = \mu_0 \bvec j + \mu_0 \epsilon_0 \pdv{\bvec E}{t}~$$ \\
\hline 
\end{tabular}
\end{table}

Maxwell方程组的这四个方程是电动力学的根基。

现在,在Maxwell方程组的加持下,原则上我们不仅能处理静场问题,还能处理各种动态问题,包括那些场源也在随时间变化的问题,\textsl{正所谓电“动”力学}(电磁感应定律似乎没有直接提及场源的变化如何导致电磁场的变化,但Maxwell方程组确实能处理这个问题)。

\subsection{势}
\begin{table}[ht]
\centering
\caption{电势与磁势}\label{tab_dynfld3}
\begin{tabular}{|c|c|c|}
\hline
* & 电场 $\bvec E$ & 磁场 $\bvec B$ \\
\hline
势与场 \upref{EMPot} & $$\bvec E = -\grad \varphi - \pdv{\bvec A}{t}~$$ & $$\bvec B = \curl \bvec A~$$ \\
\hline
势的任意性,“规范” \upref{Gauge} ;$\lambda$是同一标量函数 & $$\varphi += - \pdv{\lambda}{t}~$$ 
& $$\bvec A += \grad \lambda~$$ $\div \bvec A$的取值可被控制 \\
\hline
势的方程;达朗贝尔方程 \upref{LoGaug} & $$\laplacian \varphi - \mu_0\epsilon_0 \pdv[2]{\varphi}{t} = -\frac{\rho}{\epsilon_0}~$$ (洛伦兹规范)& $$\laplacian \bvec A - \mu_0\epsilon_0 \pdv[2]{\bvec A}{t} = -\mu_0 \bvec j~.$$ (洛伦兹规范)\\
\hline
场源导致的势;推迟势 \upref{RetPt0}
& $$\varphi(\bvec r, t) = \frac{1}{4\pi\epsilon_0} \int \frac{\rho(\bvec r', t-\frac{R}{c})}{R} \dd V'~.$$ (洛伦兹规范) $c$指光速
& $$\bvec A(\bvec r, t) = \frac{\mu_0}{4\pi} \int \frac{\bvec j(\bvec r', t-\frac{R}{c})}{R}\dd{V'}~.$$ (洛伦兹规范)\\
\hline
\end{tabular}
\end{table}
洛伦兹规范 \upref{LoGaug}:取 $\div \bvec A = -\mu_0 \epsilon_0 \pdv{\varphi}{t}~$。

\subsubsection{势}
在动态问题中,$\bvec E$不再是无旋场,因此不能按照套路定义标势;但是,$\bvec B$仍然是无散的,因此仍可定义磁矢势 $\bvec B = \curl \bvec A~$。将磁矢势的方程带入$ \curl \bvec E = -\pdv{\bvec B}{t}$,发现$ \curl \bvec E = -\pdv{\bvec \curl \bvec A}{t} = - \curl \pdv{\bvec A}{t} \Rightarrow \curl(\bvec E + \pdv{\bvec A}{t}) = 0$,即 $(\bvec E + \pdv{\bvec A}{t})$ 这个奇怪的组合作为整体是无旋的,自然可为其定义一个标势 $\bvec E + \pdv{\bvec A}{t} = -\grad \varphi$,即有$\bvec E  = -\grad \varphi - \pdv{\bvec A}{t} $。

由此,我们推广了势的含义,从而得到了动态问题中的势 \upref{EMPot}。

\subsubsection{势的规范变换;洛伦兹规范}
同静场问题一样,势的取值依旧具有任意性,只是此时电势与磁势得\textsl{此消彼长}:在磁势加上 $\grad \lambda$的同时,电势得加上 $ - \pdv{\lambda}{t}$,才能再得到相同的场。\upref{Gauge}

同样地,我们可以做规范变换;不过与静场问题不同,我们一般不选取 $\div \bvec A =0$ (库伦规范 \upref{Cgauge}),反而选取$\div \bvec A = -\mu_0 \epsilon_0 \pdv{\varphi}{t}~$(洛伦兹规范 \upref{LoGaug})。尽管洛伦兹规范形式上看起来更复杂,但它使势的达朗贝尔方程高度对称、简明(见下)。

\subsubsection{达朗贝尔方程;推迟势}
将势的表达形式代回Maxwell方程组、选取洛伦兹规范,并运用一些数学技巧,就能得到达朗贝尔方程组 \upref{LoGaug}。达朗贝尔方程组之于势,犹如Maxwell方程组之于场。或许你已经注意到,达朗贝尔方程形式上是一个波动方程。

根据达朗贝尔方程,我们可以解出势对动态场源的响应的方程(你可以根据格林函数法从达朗贝尔方程推导 \footnote{比如参考周磊教授的《电动力学讲义》},或者将其带回达朗贝尔方程检验 \cite{GriffE},不过两种方法都十分艰难)。现在,无论是势还是源都是含时的:这无可厚非,因为动态问题中场源是变化的,而势为了响应动态变化的场源,自然也必须是动态的。

看起来,动态问题中的势只不过是比静场问题中的势形式上多了一个含时项。但问题远非如此简单,正如显式写出的那样,场源与场点所含的“时”是不同的:场点的“时”是$t$,而场源的“时”是 $(t-\frac{R}{c})$ (如果你忘记了,$R = \abs{\bvec r-\bvec r'}$是场源到场点的距离),二者相差一个$(-\frac{R}{c})$的延时。这是为什么?这就涉及到近代物理中一个不可回避的重要观点:信息传递不是及时的,而是需要时间的,并且传递速度是光速$c$。

举个例子,$t=t_0$时刻的场点,能够感受到$R$距离之外场点$t=t_0$时的及时信息吗?不能,因为信息传递是需要时间的、$t=t_0$时刻场源的信息还没被发送到场点。场点能知道的,只是$\frac{R}{c}$前,即$t=t_0-\frac{R}{c}$时刻场源的信息。这就引出了推迟势 \upref{RetPt0} 的概念。有时定义推迟时刻 $t_r = t - \frac{R}{c}$。

要记住,$t_r = t-\frac{R}{c} = t-\frac{\abs{\bvec r-\bvec r'}}{c}$是含$R$与$\bvec r'$的。对于同一个场点,各个场源的$t_r = t - \frac{R}{c}$ 仍是不同的,因为场点到各个场源的距离 $R$是不同的。



\subsubsection{运动电荷的电磁场}
最后,我们能不能像处理静场问题中的点电荷一样,根据这些知识处理一个运动点电荷的电磁场?答案当然是可以,但是由于晦涩难懂的推迟势问题,即使是最简单的匀速运动电荷的电场也异常繁杂,此处不再\textsl{Ctrl+V}。具体请参考 李纳-维谢尔势\upref{LWP} 与 带电粒子的辐射\upref{chgrad}。

尽管动态问题中的势看起来只比静场问题中的势多了一个含时项,但如果你试图对场的方程做同样操作,你将得不到正确的场,正确的形式可参考 杰斐缅柯方程 \cite{GriffE}。

\subsection{电磁场的“物质性”}
在上文中,我们知道了信息在电磁场中传递是需要时间的。这似乎让我们察觉到电磁场如同一种“信使”、具有某种物质含义。接下来,我们会论证电磁场自己的能量、动量以及角动量,这让你更充分地感受到电磁场的“物质性”。

\subsubsection{能量与能量流} 
\upref{EBS}
%有很多种方法论证电磁场具有能量,并且电荷系统中所谓“系统的电势能”其实就是暗藏在电场的能量。在这里,我们提供一种从洛伦兹力出发的途径。
考虑一个小区域内广义洛伦兹力对电荷做功,这将改变电荷的机械动能:
$$
\begin{aligned}
\pdv{u_{mech}}{t} &= \bvec f \cdot \bvec v = \rho (\bvec E + \bvec v \times \bvec B)\cdot \bvec v = \rho \bvec E \cdot \bvec v = \bvec E \cdot \bvec j \\
 & = \bvec E \cdot (\frac{1}{\mu_0} \curl \bvec B - \epsilon_0 \pdv{\bvec E}{t})\\
 & = \frac{1}{\mu_0} \bvec E \cdot (\curl \bvec B) - \epsilon_0 \bvec E \pdv{\bvec E}{t}\\
 & = \frac{1}{\mu_0} (-\bvec B \pdv{\bvec B}{t} - \div(\bvec E \times \bvec B))- \epsilon_0 \bvec E \pdv{\bvec E}{t}\\
 & = - (\frac{1}{2\mu_0} \pdv{B^2}{t} + \frac{\varepsilon_0}{2} \pdv{E^2}{t}) - \frac{1}{\mu_0} \div(\bvec E \times \bvec B)\\
 & = - \pdv{}{t} (\frac{1}{2\mu_0} B^2 + \frac{\varepsilon_0}{2} E^2) - \frac{1}{\mu_0} \div(\bvec E \times \bvec B)
\end{aligned}
~.
$$

注意到 $(\frac{1}{2\mu_0} B^2 + \frac{\varepsilon_0}{2} E^2) = u_{emf}$就是我们以前知道的电磁场能量密度。电荷系统中所谓“系统的电势能”其实就是暗藏在电场的能量。

将上式重新整理,我们得到
$$
- \frac{1}{\mu_0} \div(\bvec E \times \bvec B) = \pdv{u_{mech}}{t} + \pdv{u_{emf}}{t}~,
$$
类比电荷守恒方程$-\div \bvec j = \pdv{\rho}{t}$,其中的物理意义已经很显然了,这个方程告诉我们流入区域的$\frac{1}{\mu_0} \bvec E \times \bvec B$ (负号代表流入) 将转换为粒子的机械动能与电磁场的能量,那么$ \frac{1}{\mu_0} \bvec E \times \bvec B$应该是某种与电磁场能量有关的流。由此,我们将其定义为Poynting矢量$\bvec S$,代表电磁场能流密度。
$$
\bvec S = \frac{1}{\mu_0} \bvec E \times \bvec B~.
$$

\subsubsection{动量与动量流}
具体可参考 电磁场的动量守恒、动量流密度张量\upref{EBP}。

\subsubsection{角动量}
具体可参考 \cite{GriffE}。
