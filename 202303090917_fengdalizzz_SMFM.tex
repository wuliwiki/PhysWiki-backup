% 索末菲模型
% 固体物理|自由电子气模型

虽然德鲁特模型在电子的声、光和热导率方面解释得比较好,但其在解释电子比热时却遇到了困难。

根据经典的能均分定理,金属电子气中每个电子的平均内能为$3k_BT/2$,对热容的贡献为$3k_B/2$,然而实验中的电子热容几乎测不到,只有德鲁特模型给出的1\%左右,而且与温度密切相关。
\begin{figure}[ht]
\centering
\includegraphics[width=6cm]{./figures/SMFM_1.png}
\caption{常见金属热容随温度的变化} \label{SMFM_fig1}
\end{figure}
解释这个现象需要用到索菲亚模型。
\subsection{基本假设}
\begin{enumerate}
\item \textbf{独立电子近似(Independent electron approximation)}:电子之间不会相遇,不存在任何相互作用。
\item \textbf{自由电子近似}:相对电子而言,晶体中的离子的运动忽略不计,同时也忽略电子与离子的库伦力作用。
\item \textbf{不碰撞假设}:电子以物质波的形式存在,并且不会与离子碰撞。
\item \textbf{泡利不相容原理}:电子是费米子,满足泡利不相容原理。
\end{enumerate}
\subsection{动量空间、费米能和态密度}
根据上面的假设,我们可以写出电子满足的薛定谔方程:
\begin{equation}
\frac{-\hbar^2}{2m}\lambda^2\psi=E\psi
\end{equation}
