% Linux 分区和文件系统操作笔记(Gparted, fdisk, resize2fs, grub, Clonezilla)

\begin{issues}
\issueDraft
\end{issues}

\subsection{文件操作}
\begin{itemize}
\item \verb|sudo dd| 是一个很危险的命令, 它直接操作硬盘上的任意位置的数据, 无论是否被挂载. \verb|sudo dd if=/dev/random of=/dev/sda| 会把整个硬盘包括分区方式(MBR,GPT)等所有信息抹掉
\item MBR (Master Boot Record) 和 GPT (GUID Partition Table) 是唯一两种硬盘的分区方式(partition schemes)
\item MBR 的 grub 设置文件在 \verb|/boot/grub/grub.cfg| , GPT 的在 \verb|/boot/efi/EFI/ubuntu/grub.cfg|
\item 要备份 MBR,用 \verb|dd if=/dev/sda of=/data/disk.img bs=446 count=1| , 还原同理, 但还原最好只用 440 字节
\item 用 dd 把一个硬盘的分区克隆的另一个硬盘的分区后(两个分区大小必须完全相同), 再 clone MBR 用 \verb|dd if=/dev/sda of=/dev/sdc bs=440 count=1|
\item 要擦除 MBR 中的 grub, 首先备份! 然后用 \verb|sudo dd if=/dev/zero of=/dev/sdx bs=440 count=1|
\item 不同的分区当然可以有不同的文件系统
\item \verb|Gparted| 是一个 GUI 硬盘分区工具, 在 Ubuntu live CD 中自带
\item \verb|GParted| 使用 \verb|e2image -ra -p /dev/sda1 /dev/sdb1| 来复制分区, 没有数据的部分不会复制,Arch Wiki 中的 “disk cloning” 词条列出了很多工具
\item grub 可以装在和系统不同的盘上
\item \verb|sda, sdb| 等编号是不稳定的, 可能重启等操作之后会变
\item \verb|fdisk -l| 查看所有挂载的硬盘
\item \verb|lsblk| 也可以查看
\item \verb|resize2fs -p /dev/sd? ???K| 可以改变 ext4 文件系统的大小, Gparted 用的就是这个命令. 这个命令需要很长时间.
\item \verb|mklabel| 修改分区的 label
\item \verb|mkfs.ext4 /dev/sdx1| 或者 \verb|mke4fs -t ext4 /dev/sdb1| 把某个分区格式化为 \verb|ext4|
\item Clonezilla 的分区备份对 GPT 和 MBR 都是通用的, 如果发生任何问题, 用 live cd 重装系统,确保能正常启动, 然后还原对应的分区就行,无需任何额外设置.
\item 之前 SMART 提示 threshold exceeded,但并没有看到那一项有问题,现在重新写 GPT,安装,clonezilla 还原, 又可以正常用了.
\item  正常的 ext4 系统分区用 Gparted 缩小以后, 突然就不能启动了!貌似并不是 grub 的问题,内核已经加载然后弄出一堆错误.
\item 有一个工具叫做 boot-repair 可能可以用来修复 bootloader 的问题
\item 有一个设置 Grub 的 GUI 软件叫做 Grub Customizer
\item \verb|lilo| 是一个可以安装 generic windows boot loader on MBR 的命令
\item \verb|sda, sdb| 等在 grub2 命令行里面的 ls 名字叫做 \verb|hd1, hd2...|
\item \verb|extFAT| 在 Ubuntu 中并不怎么支持 resize 和移动, 还是用 windows 的傲梅比较好
\end{itemize}

\subsubsection{/etc/fstab}
\addTODO{ }

\subsection{Bootloader}
\begin{itemize}
\item 开机的时候,BIOS/EUFI会先启动bootloader然后再由bootloader加载系统(也就是按F2或Del以后进入的那个界面选择的其实是bootloader)
\item windows的bootloader自动加载windows,而grub会搜索每个硬盘和分区,并列出所有选项
\item 在GPT中,bootloader总是会在EFI分区,而在MBR中,bootloader在masterrecord中
\item \verb|sudo grub-install /dev/sd#| 可以在某个硬盘中安装 grub (注意如果硬盘只有一个分区的话, 分区名和硬盘名相同, 安装可能就会出错)
\item \verb|df| 只会显示已挂载的硬盘, 而 \verb|fdisk| 或者 \verb|disk, gparted| 软件才会显示所有连接的硬盘
\item 要安装 ubuntu live 镜像到 disk, 用 \verb|sudo dd bs=4M if=/path/to/ISOfile of=/dev/sda status=progress oflag=sync| 注意 \verb|sda| 后面不能有数字! 这样 \verb|dd| 会把整个硬盘克隆成 iso 的内容而无视之前的任何 partition table 和分区. 此时 u 盘和光盘完全等效,都是只读的(亲测成功).
\end{itemize}

\begin{itemize}
\item grub 在菜单中按 c 进入命令行模式, 然后 ls 可以看见有哪些硬盘和分区, 例如 (hd0,msdos1), 然后继续 ls (hd0,msdos1)/ 按 tab,就会提示该目录下有什么
\item grub 在菜单中按 e 就可以编辑临时的 grub config 文件,填入 ls 获得的信息即可
\item grub 的菜单是由 \verb|/boot/grub/grub.cfg| 或者 \verb|grub.config| 定义的
\item 如何克隆 MBR 硬盘上安装的 Ubuntu?
\item 如果克隆 MBR 硬盘的分区, 那么 grub 并不会一起克隆过去, 所以还需要 \verb|sudo grub-install ...| 在新硬盘上面安装 grub (如果说什么不推荐用啥不稳定的,就在后面加 --force). 然后再把原来的 grub.cfg 拷贝到新硬盘就好了(不确定新安装 grub 会不会把 cfg 重置).
\item 注意克隆分区以后, 目标分区的 uuid 会和原来的 uuid 一模一样,如果拔掉原来的硬盘没什么问题, 但如果要一起插的话还是需要改一下
\item \verb|blkid| 可以查看所有分区的 uuid
\item uuid 需要 umount 才可以修改, 随机生成 uuid 用 \verb|tune2fs -U random /dev/sdb?| , disk 中会看到变化, 如果 blkid 看见 uuid 没有变化, 就 mount 再 umount 即可刷新
\item \verb|fdisk| 也可以改变 uuid, \verb|sudo fdisk /dev/sd?| 然后按 x 再按 i 即可
\item 理论上 GPT 是可以在 BIOS 主板上 boot 的(实际上对大部分 linux 可以), 见\href{https://superuser.com/questions/1337344/is-it-possible-to-boot-linux-from-a-gpt-disk-on-a-bios-system}{这里}.
\end{itemize}
