% 仿射法在解析几何中的应用
% 仿射变换 圆锥曲线
\subsection{引例:2022年全国乙卷解析几何大题}

已知椭圆 \(E\) 的中心为坐标原点,对称轴为 \(x\) 轴、\(y\) 轴,且过 \(A(0,-2)\),\(B(\frac{3}{2},-1)\) 两点.

(1)求 \(E\) 的方程;

(2)设过点 \(P(1,-2)\) 的直线交 \(E\) 于 \(M,N\) 两点,过 \(M\) 且平行于 \(x\) 轴的直线与线段 \(AB\) 交于点 \(T\) .点 \(H\) 满足 \(\overrightarrow{MT}=\overrightarrow{TH}\) ,证明:直线 \(HN\) 过定点.

\textbf{解:}(1) \(E:\frac{x^2}{3}+\frac{y^2}{4}=1\)
\begin{figure}[ht]
\centering
\includegraphics[width=14.25cm]{./figures/affine_13.png}
\caption{请添加图片描述} \label{affine_fig13}
\end{figure}
(2)置 \(x'=\frac{x}{\sqrt{3}},y'=\frac{y}{2}\)

则 \(E\) 化为单位圆 \(x'^2+y'^2=1\) , 且 \(\text{A}'(0,-1)\,,\text{B}'\left(\frac{\sqrt{3}}{2},-\frac{1}{2}\right)\,,\text{P}'\left(\frac{\sqrt{3}}{3},-1\right)\)

容易发现 \(\triangle \text{OAB}\) 为等边三角形
我们猜想 \(\text{HN}\) 恒过 \(\text{A}\),故只需证 \(\text{H},\text{N},\text{A}\) 三点共线,只需证:
$$\frac{\text{MN}}{\text{PN}}=\frac{\text{HM}}{\text{AP}}$$

而我们知道 \(\frac{\text{QM}}{\text{PQ}}=\frac{\text{TM}}{\text{AP}}\) 

即:
$$1-\frac{\text{PM}}{\text{PQ}}=\frac{\text{TM}}{\text{AP}}$$

两式相除,得:
$$\frac{1-\frac{\text{MP}}{\text{PQ}}}{\frac{\text{MN}}{\text{PN}}}=\frac{1}{2}$$

即:
$$\frac{\text{PN}}{\text{MN}}-\frac{\text{PM}\cdot\text{PN}}{\text{PQ}\cdot\text{MN}}=\frac{1}{2}$$

由切割线定理:
$${\text{PM}}\cdot{\text{PN}}=\text{AP}^2=\frac{1}3$$

又:
$$\text{PQ}=\text{TP}\cos\angle\text{OPN}=\frac{\sqrt{3}}{6}\cos\angle\text{OPN}$$

故 \(\text{PN}-\frac{2\sqrt{3}}{3}\cos\angle\text{OPN}=\frac{1}{2}\text{MN}\) ,即:
$$\text{PN}-\text{OP}\cos\angle\text{OPN}=\frac{1}{2}\text{MN}$$

此式显然
\subsection{仿射法的定义及性质}
\begin{definition}{仿射变换}
设椭圆\,\(\frac{x^2}{a^2}+\frac{y^2}{b^2}=1\),其中\,\(a>b>0\),置变换:
$$x'=\frac{x}{a},y'=\frac{y}{b}$$
则椭圆化为单位圆\,\(C:x'^2+y'^2=1\)
\end{definition}
届时,我们可以就可以抛开繁琐的代数计算,运用几何性质解决问题.此前,我们先介绍仿射变换的几个性质.
\begin{lemma}{}
变换后,平面内任意一条直线的斜率变为原来的\,\(\frac{a}{b}\)
\end{lemma}
\begin{lemma}{}
变换后,平面上任意区域的面积变为原来的\,\(\frac1{ab}\)
\end{lemma}
\begin{lemma}{}
变换后,线段中点依然是线段中点;关于水平线、铅垂线、坐标原点对称的元素依然对称;平面区域的重心保持不变
\end{lemma}
\begin{lemma}{}
变换前后,平行关系保持不变
\end{lemma}
\begin{lemma}{}
变换前后,平行线段的长度比保持不变
\end{lemma}
\subsection{仿射法导出的常用结论}
\begin{corollary}{}
设椭圆 \(\frac{x^2}{a^2}+\frac{y^2}{b^2}\),直线 \(l\) 交椭圆于点 \(A\) 和\,\(B\),点 \(P(x_0,y_0)\) 为线段 \(AB\) 的中点,求直线斜率
\begin{figure}[ht]
\centering
\includegraphics[width=14.25cm]{./figures/affine_1.png}
\caption{请添加图片描述} \label{affine_fig1}
\end{figure}

\textbf{解:}作变换 \(x'=\frac{x}{a},y'=\frac{y}{b}\) ,则椭圆化为单位圆 \(C:x'^2+y'^2=1\),且 \(P'\left(\frac{x_0}{a},\frac{y_0}{b}\right)\)

故 \(k_{OP'}=\frac{ay_0}{bx_0}\)

由性质3,\(P'\) 为 \(A'B'\) 的中点

在圆中,由垂径定理,\(OP'\bot A'B'\)

而 \(k_{A'B'}\cdot k_{OP'}=-1\) ,得 \(k_{A'B'}=-\frac{bx_0}{ay_0}\)

由上述性质 1, \(k_l=\frac{b}{a}\cdot k_{A'B'}=-\frac{b^2x_0}{a^2y_0}\)
\end{corollary}
\begin{corollary}{}
设椭圆   \(\Gamma:\frac{x^2}{a^2}+\frac{y^2}{b^2}=1(a>b>0)\),直线 \(l\) 切椭圆于点 \(P(x_0,y_0)\),求直线斜率 

\textbf{解:}作变换 \(x'=\frac{x}{a},y'=\frac{y}{b}\) ,则椭圆化为单位圆 \(C:x'^2+y'^2=1\),\(P'\left(\frac{x_0}{a},\frac{y_0}{b}\right)\) 

故  \(k_{OP'}=\frac{ay_0}{bx_0}\) 

在圆中,由切线定理, \(OP'\bot l'\) 

而  \(k_{l'}\cdot k_{OP'}=-1\) ,得 \(k_{l'}=-\frac{bx_0}{ay_0}\) 

由上述性质 1, \(k_l=\frac{b}{a}\cdot k_{l'}=-\frac{b^2x_0}{a^2y_0}\) 

\end{corollary}

\begin{corollary}{}
设椭圆 \(\Gamma:\frac{x^2}{a^2}+\frac{y^2}{b^2}=1(a>b>0)\), \(A\) 、\(B\) 和 \(M\) 为椭圆上的点,点 \(A\) 和 \(B\) 关于原点对称,求证: \(k_{MA}\cdot k_{MB}\) 为一定值
​
\begin{figure}[ht]
\centering
\includegraphics[width=14.25cm]{./figures/affine_2.png}
\caption{请添加图片描述} \label{affine_fig2}
\end{figure}

\textbf{证明:}作变换 \(x'=\frac{x}{a},y'=\frac{y}{b}\) ,则椭圆化为单位圆 \(C:x'^2+y'^2=1\)

则 \(A'B'\) 为圆 \(C\) 的直径,所以 \(M'A'\bot M'B'\) ,\(k_{M'A'}\cdot k_{M'B'}=-1\)

由性质1, \(k_{MA}\cdot k_{MB}=-1\cdot \frac{b}{a}\cdot \frac{b}{a}=-\frac{b^2}{a^2}\)

\end{corollary}

\begin{corollary}{}
设椭圆 \(\Gamma:\frac{x^2}{a^2}+\frac{y^2}{b^2}=1(a>b>0)\), \(A\) 和 \(B\) 为椭圆上的点,点 \(M\) 为 \(AB\) 的中点,求证: \(k_{AB}\cdot k_{OM}\) 为一定值

\begin{figure}[ht]
\centering
\includegraphics[width=14.25cm]{./figures/affine_3.png}
\caption{} \label{affine_fig3}
\end{figure}

\textbf{证明:}作变换 \(x'=\frac{x}{a},y'=\frac{y}{b}\) ,则椭圆化为单位圆 \(C:x'^2+y'^2=1\)

由性质3, \(M'\) 为 \(A'B'\) 的中点,由垂径定理, \(O'M'\bot A'B'\) , \(k_{OM'}\cdot k_{A'B'}=-1\)

由性质1, \(k_{OM}\cdot k_{AB}=-1\cdot \frac{b}{a}\cdot \frac{b}{a}=-\frac{b^2}{a^2}\)

\end{corollary} 

\begin{corollary}{等角定理}
设椭圆 \(\Gamma:\frac{x^2}{a^2}+\frac{y^2}{b^2}=1(a>b>0)\), \(A\) 和 \(B\) 为椭圆上的点,直线 \(AB\) 交 \(x\) 轴于点 \(P(x_0,0)\) ,在 \(x\) 轴上求一点 \(G\) ,使 \(x\) 轴平分 \(\angle AGB\) 

\begin{figure}[ht]
\centering
\includegraphics[width=14.25cm]{./figures/affine_4.png}
\caption{请添加图片描述} \label{affine_fig4}
\end{figure}
\textbf{解:}作变换 \(x'=\frac{x}{a},y'=\frac{y}{b}\) ,则椭圆化为单位圆 \(C:x'^2+y'^2=1\), \(P'\left(\frac{x_0}{a},0 \right)\) 

由性质3, \(\angle A'G'B'\) 依然被 \(x\) 轴平分,记 \(B'G'\) 交圆 \(O\) 于另一点 \(D'\)

易见 \(\angle A'OG'=\frac{1}{2}\cdot\angle A'OD'=\angle A'B'D'\) 

又 \(\angle A'P'O=\angle B'P'G'\)

所以 \(\angle OA'P'=\angle OG'B'\)

又 \(\angle A'G'O=\angle B'G'O\)

故 \(\angle OA'B'=\angle A'G'O\)

又 \(\angle A'OG'=\angle A'OG'\)

故 \(\triangle A'OG'\sim \triangle P'OA'\)

进而 \(\frac{OP'}{OA'}=\frac{OA'}{OG'}\)

故 \(OP'\cdot OG'=OA'^2=1\) 

即 \(G'\left(\frac{a}{x_0},0\right)\)

由性质1, \(G\left( \frac{a^2}{x_0},0\right)\) 

\end{corollary}

\begin{corollary}{}
过椭圆 \(\Gamma:\frac{x^2}{a^2}+\frac{y^2}{b^2}=1\) 上任一点 \(A(x_0,y_0)\) 作两条倾斜角互补的直线交椭圆于点 \(P\),\(Q\) ,求证: \(k_{PQ}\) 为一定值
\begin{figure}[ht]
\centering
\includegraphics[width=14.25cm]{./figures/affine_5.png}
\caption{请添加图片描述} \label{affine_fig5}
\end{figure}
\textbf{证明:}作变换 \(x'=\frac{x}{a},y'=\frac{y}{b}\) ,则椭圆化为单位圆 \(C:x'^2+y'^2=1\) , \(A\left(\frac{x_0}{a},\frac{y_0}{b}\right)\) 

由性质3, \(A'P'\),\(A'Q'\) 仍然关于铅垂线对称,故 \(\angle P'A'B'=\angle OA'B'\)

同弧所对的圆周角等于圆心角的一半,故 \(\angle P'OB'=\angle Q'OB'\)

又 \(OP'=OQ'\) ,等腰三角形三线合一,所以 \(OB'\bot P'Q' \), \(k_{P'Q'}\cdot k_{OB'}=-1\)

又因为 \(k_{OA'}=-k_{OB'}\) ,所以 \(k_{OA'}\cdot k_{P'Q'}=1\)

而 \(k_{OA'}=\frac{a\cdot y_0}{b\cdot x_0}\) ,因此 \(k_{P'Q'}=\frac{b\cdot x_0}{a\cdot y_0}\)

由性质1, \(k_{PQ}=\frac{b^2\cdot x_0}{a^2 \cdot y_0}\) 
\end{corollary}
\begin{corollary}{}
求椭圆 \(\frac{x^2}{a^2}+\frac{y^2}{b^2}=1\) 内接三角形的最大面积和外切三角形的最小面积
\begin{figure}[ht]
\centering
\includegraphics[width=14.25cm]{./figures/affine_6.png}
\caption{请添加图片描述} \label{affine_fig6}
\end{figure}
\textbf{解:}作变换 \(x'=\frac{x}{a},y'=\frac{y}{b}\) ,则椭圆化为单位圆 \(C:x'^2+y'^2=1\)

由琴生不等式:
$$S'_{\text{ins}}\leq \frac{3}{2}\cdot \sin\frac{\pi}{3}=\frac{3\sqrt{3}}{4}, S'_{\text{ext}}\geq 3\cot\frac{\pi}{6}=3\sqrt{3}$$ 

由性质2: 
$$S_{\text{ins}\max}=\frac{3\sqrt{3}}{4}\cdot ab, S_{\text{ext}\max}=3\sqrt{3}\cdot ab$$ 

\end{corollary}
\begin{corollary}{}
求以 \(O\) 为重心的椭圆 \(\frac{x^2}{a^2}+\frac{y^2}{b^2}=1\) 的内接三角形 \(\triangle ABC\) 的面积
\begin{figure}[ht]
\centering
\includegraphics[width=14.25cm]{./figures/affine_7.png}
\caption{请添加图片描述} \label{affine_fig7}
\end{figure}
\textbf{解:}作变换 \(x'=\frac{x}{a},y'=\frac{y}{b}\) ,则椭圆化为单位圆 \(C:x'^2+y'^2=1\)

由性质3, \(O\) 仍为 \(\triangle A'B'C'\) 的重心,故 \(\vec{OA'}=\vec{OB'}=\vec{OC'}\)  , \(\triangle A'B'C'\) 为等边三角形

故 \(S_{\triangle A'B'C'}=\frac{3\sqrt{3}}{4}\) ,由性质3, \(S_{\triangle ABC}=\frac{3\sqrt{3}}{4}\cdot ab\) 
\end{corollary}
\begin{corollary}{软解定理}
已知椭圆 \(\Gamma:\frac{x^2}{a^2}+\frac{y^2}{b^2}=1\),过 \(T(t,0)\) 的直线 \(l\) 交 \(\Gamma\) 于 \(A\),\(B\) 两点,且 \(\frac{AT}{BT}=p\) ,求 \(l\) 斜率

\textbf{解:}作变换 \(x'=\frac{x}{a},y'=\frac{y}{b}\) ,则椭圆化为单位圆 \(C:x'^2+y'^2=1\)

由性质5,仍有 \(\frac{A'T'}{B'T'}=p\) ,记 \(m=A'T',n=B'T'\) ,于是 \(\frac{m}{n}=p\) 

由切割线定理, \(A'T'\cdot B'T'=MT'\cdot NT'\) ,其中 \(M\),\(N\) 为单位圆与 \(x\) 轴的两交点

即 \(m\cdot n=(1+t')(1-t')=1-t'^2\) 

接着,
$$(m+n)^2=m^2+n^2+2mn=\frac{mn}{\frac{m}{n}}+\frac{mn}{\frac{n}{m}}+2mn=\left(p+\frac{1}{p}+2\right)(1-t'^2)$$

由垂径定理
$$1-\frac{k'^2t'^2}{k'^2+1}=\frac{1}{4}\left(p+\frac{1}{p}+2\right)(1-t'^2)$$ 

整理,由性质1得:
$$\frac{1}{1+(\frac{b}{ak})^2}=1-\frac{1}{4}\left(p+\frac{1}{p}+2\right)\left(1-\frac{t}{a}^2\right)$$ 

\end{corollary}

\begin{corollary}{}
设椭圆 \(\Gamma:\frac{x^2}{a^2}+\frac{y^2}{b^2}=1\) ,\(C\) 为第三象限内椭圆上的一点, \(A\),\(B\) 分别是椭圆的上顶点、右顶点,直线 \(OA\) 与 \(y\) 轴交于点 \(M\) ,直线 \(OB\) 与 \(x\) 轴交于点 \(N\) ,求证:四边形 \(ABNM\) 的面积为定值\begin{figure}[ht]
\centering
\includegraphics[width=14.25cm]{./figures/affine_8.png}
\caption{请添加图片描述} \label{affine_fig8}
\end{figure}

\textbf{证明:}作变换 \(x'=\frac{x}{a},y'=\frac{y}{b}\) ,则椭圆化为单位圆 \(O:x'^2+y'^2=1\)

于是 \(OM'=\tan{\beta},ON'=\tan{\alpha},\alpha+\beta=\frac{\pi}{4}\) 

于是
$$\begin{eqnarray} S'&=&\frac{1}{2}\cdot B'M'\cdot A'N'\\ &=&\frac{1}{2}(\tan\alpha+1)(\tan\beta+1)\\ &=&\frac{1}{2}\cdot\left[\tan\alpha\tan\beta+(\tan\alpha+\tan\beta)+1\right]\\ &=&\frac{1}{2}\cdot[\tan\alpha\tan\beta+1-\tan\alpha\tan\beta+1]\\ &=&1 \end{eqnarray}$$

由性质2, \(S=ab\) 
\end{corollary}
\begin{corollary}{蒙日圆}
求椭圆 \(\Gamma:\frac{x^2}{a^2}+\frac{y^2}{b^2}=1\) 两垂直切线的交点的轨迹方程

\textbf{解:}置 \(x'=\frac{x}{a}\,,y'=\frac{y}b\) ,则椭圆化为单位圆 \(x'^2+y'^2=1\) 

如图,设直线 \(\text{AB}\),\(\text{AC}\) 的斜率分别为 \(k_1\),\(k_2\) ,由仿射性质, \(k_1k_2=-\frac{a^2}{b^2}\) 

设 \(\text{A}(x_0,y_0)\) ,于是: 
$$\text{AB}:y=k_1(x-x_0)+y_0\text{BC}:y=k_2(x-x_0)+y_0 $$

原点到切线的距离为圆的半径,即:\(\frac{|k_{1,2}x_0-y_0|}{\sqrt{k_{1,2}^2+1}}=1\)

化简得:
$$(x_0^2-1)k_{1,2}^2-2x_0y_0k_{1,2}+y_0^2-1=0$$

由韦达定理:
$$\frac{y_0^2-1}{x_0^2-1}=-\frac{a^2}{b^2}$$

即:
$$a^2x_0^2+b^2y_0^2=a^2+b^2$$

回代,得:
$$\text{M}:x^2+y^2=a^2+b^2$$
\end{corollary}
\begin{corollary}{}
设过椭圆 \(\Gamma:\frac{x^2}{a^2}+\frac{y^2}{b^2}=1\) 外一点 \(P(x_0,y_0)\) 作椭圆的一条切线 \(l\) 切椭圆于 \(M\) ,连结 \(PM\),求 \(|PM|\)
\begin{figure}[ht]
\centering
\includegraphics[width=14.25cm]{./figures/affine_9.png}
\caption{请添加图片描述} \label{affine_fig9}
\end{figure}
\textbf{解:}过 \(O\) 作直线 \(l'\) 交椭圆于两点, 靠近点 \(P\) 一侧的点为 \(A\) ,设 \(|PM|=\lambda|OA|\) ,我们只需要求出 \(\lambda\) 以及 \(|OA|\) 

不妨令 \(x'=\frac{x}{a},y'=\frac{y}{b}\) ,则椭圆化为圆 \(C:x'^2+y'^2=1\)

变换前后,平行线段比例不变,故 \(\lambda=\frac{|P'M'|}{|OA'|}=\sqrt{\frac{m^2}{a^2}+\frac{n^2}{b^2}-1}\) 

至于 \(|OA|\) ,可以求出切线斜率,与椭圆联立,这里不再赘述
\end{corollary}

\subsection{杂例}
这里列举一些利用仿射法解决解析几何问题的例子,其中蕴含着该方法的常用技巧
\begin{example}{}
已知点 \(P\left(t,\frac{1}{2}\right)\) 在椭圆 \(\Gamma:\frac{x^2}{2}+y^2=1\) 内,过点 \(P\) 的直线 \(l\) 与椭圆 \(\Gamma\) 相交于 \(A\) 和 \(B\) 两点,且点 \(P\) 是线段 AB 的中点, O 为坐标原点.求 \(S_{\triangle OAB}\) 的最大值

\textbf{解:}令 \(x'=\frac{x}{\sqrt{2}},y'=y\) ,则 \(C:x'^2+y'^2=1\) ,\(P'\left(\frac{\sqrt{2}}{2}\cdot t,\frac{1}{2}\right)\)

\(OP'=\sqrt{\left(\frac{\sqrt{2}}{2}\cdot t\right)^2+{\frac{1}{2}}^2}=\sqrt{\frac{t^2}{2}+\frac{1}{4}}\)

由性质3, \(P'\) 为 \(A'B'\) 的中点

由垂径定理, \(OP'\bot A'B'\) 
故 \(A'B'=2\sqrt{1-OP'^2}=2\sqrt{\frac{3}{4}-\frac{t^2}{2}}\)

\(S'=\frac{1}{2}\cdot OP' \cdot A'B'=\sqrt{\left(\frac{t^2}{2}+\frac{1}{4}\right)\left(\frac{3}{4}-\frac{t^2}{2}\right)},t\in[-1,1]\)

\(S'_{max}=\frac{1}{2}\)

由性质2, \(S_{max}=\frac{\sqrt{2}}{2}\) 
\end{example}

\begin{example}{}
已知椭圆 \(\Gamma:\frac{x^2}{b^2}+\frac{y^2}{b^2}=1(a>b>0)\) 的左焦点为 \(F(-2,0)\) ,离心率为 \(\frac{\sqrt{6}}{3}\).

(1)求椭圆 \(\Gamma\) 的标准方程;

(2)设 \(T\) 为直线 \(x=-3\) 上一点,过点 \(F\) 作 \(TF\) 的垂线交椭圆于 \(P,Q\) 两点,当四边形 \(OPTQ\) 是平行四边形时,求四边形 \(OPTQ\) 的面积.

\textbf{解:}(1)椭圆的标准方程为 \(\frac{x^2}{6}+\frac{y^2}{2}=1\)
\begin{figure}[ht]
\centering
\includegraphics[width=14.25cm]{./figures/affine_14.png}
\caption{请添加图片描述} \label{affine_fig14}
\end{figure}
(2)令 \(x'=\frac{x}{\sqrt{6}},y'=\frac{y}{\sqrt{2}}\) ,则 \(F'\left(-\frac{2}{\sqrt{6}},0\right) , l':x'=-\frac{3}{\sqrt{6}}\)

由性质4, \(OP'T'Q'\) 仍为平行四边形,又 \(OP'=OQ'\) , \(OP'T'Q'\) 为菱形

故 \(P'Q'\bot OT'\)

记  \(l'\bot x\) 轴于 \(H’\) 

因为 \(\angle T'H'O=\angle F'BO,\angle T'OH'=\angle T'OH'\)

所以 \(\triangle T'OH'\sim\triangle F'BO\)

故 \(\frac{OF'}{OT'}=\frac{OB}{OH'}\)

得 \(OB=\frac{\sqrt{2}}{2}\)

由垂径定理, \(P'Q'=2P'B=2\sqrt{1-OB^2}=\sqrt{2}\)

则 \(S_{菱形OP'T'Q'}=\sqrt{2}\cdot\frac{\sqrt{2}}{2}=1\)

由性质2, \(S_{四边形OPTQ}=1\cdot \sqrt{6}\cdot\sqrt{2}=2\sqrt{3}\) 
\end{example}

\begin{example}{}
已知椭圆 \(\Gamma:\frac{x^2}{2}+y^2=1\) ,过左焦点 \(F\) 的直线 \(l\) 交椭圆于 \(P,Q\) 两点, \(M\) 为 \(PQ\) 的中点, \(O\) 为坐标原点. 若 \(\triangle FMO\) 是以 \(OF\) 为底边的等腰三角形,求直线 \(l\)  的方程
\begin{figure}[ht]
\centering
\includegraphics[width=12.45cm]{./figures/affine_10.png}
\caption{请添加图片描述} \label{affine_fig10}
\end{figure}
\textbf{解:}令 \(x'=\frac{x}{\sqrt{2}},y'=y\),则 \(F'(-\frac{1}{\sqrt{2}},0)\)

由性质3, \(M'F',OM'\) 仍旧关于铅垂线对称

在圆中,由垂径定理, \(OM'\bot P'Q'\)

故 \(\triangle OM'F'\) 为等腰直角三角形, \(OF'=\sqrt{2} OM'\)

设 \(l':y'=k'\left(x'+\frac{1}{\sqrt{2}}\right)\) ,于是 \(\frac{1}{\sqrt{2}}=\sqrt{2}\cdot \frac{k'}{\sqrt{2(k'^2+1)}}\)

解得 \(k'=1\) ,由性质1, \(k=\frac{\sqrt{2}}{2}\)

故 \(l:y=\frac{\sqrt{2}}{2}(x+1)\) 
\end{example}

\begin{example}{2017全国I理}
已知 \(C:\frac{x^2}4+y^2=1,P_2(0,1)\),设直线 \(l\) 不经过 \(P_2\) 点且与 \(C\) 相交于 \(A,B\) 两点,若直线 \(P_2A\) 与直线 \(P_2B\) 的斜率之和为 \(-1\),求证:\(l\) 过定点
\begin{figure}[ht]
\centering
\includegraphics[width=12.45cm]{./figures/affine_11.png}
\caption{请添加图片描述} \label{affine_fig11}
\end{figure}
\textbf{解:}作变换 \(x'=\frac{x}{2},y'=y\) ,则椭圆化为单位圆 \(x'^2+y'^2=1\)

记如图所示的 \(\alpha,\beta\) ,则 \(A'(\cos2\alpha,\sin2\alpha),B'(\cos2\beta,\cos2\beta)\)

\(k_1’=\cot{\alpha},k_2'=\cot{\beta}\) ,由性质1, \(k_1'+k_2'=-2\) ,即 \(\cot{\alpha}+\cot{\beta}=-2\)

移项,得 \(\cot\alpha+\cot \beta+2=0\)

进而 \((\cot\alpha+\cot \beta+2)(\cot \alpha+\cot\beta)=0\)

展开,得 \(\cot^2\alpha+2\cot\alpha=\cot^2\beta+2\cot\beta\)

所以 \(\cot^2\alpha+2\cot\alpha+\frac{1}{2}=\cot^2\beta+2\cot\beta+\frac{1}{2}\)

整理,得 \(-\frac{(\tan\alpha+1)^2}{2\tan^2\alpha}=-\frac{(\tan\beta+1)^2}{2\tan^2\beta}\)

得 
$$\frac{\frac{2\tan\alpha}{1+\tan^2\alpha}+1}{\frac{1-\tan^2\alpha}{1+\tan^2\alpha}-1}=\frac{\frac{2\tan\beta}{1+\tan^2\beta}+1}{\frac{1-\tan^2\beta}{1+\tan^2\beta}-1}$$

由万能公式, \(\frac{\sin2\alpha+1}{\cos2\alpha-1}=\frac{\sin2\beta+1}{\cos2\beta-1}\)

所以 \(l’\) 恒过定点 \((1,-1)\)

所以 \(l\) 恒过定点 \((2,-1)\) 
\end{example}
\begin{example}{}
 已知椭圆 \(\Gamma:\frac{x^2}{9}+\frac{y^2}{3}=1\) ,过点 \(P(0,2)\) 作两直线分别交 \(\Gamma\) 于 \(A,B\) 和 \(C,D\) , \(AB,CD\) 的中点分别为 \(M,N\) ,且 \(k_{AB}\cdot k_{CD}=-2\) ,证明: \(MN\) 过定点
\begin{figure}[ht]
\centering
\includegraphics[width=14.25cm]{./figures/affine_12.png}
\caption{请添加图片描述} \label{affine_fig12}
\end{figure}
\textbf{证明:}作变换 \(x'=\frac{x}{3},y'=\frac{y}{\sqrt{3}}\) ,则椭圆化为单位圆 \(x'^2+y'^2=1\), \(P'\left(0,\frac{2}{\sqrt{3}}\right)\)

由性质1, \(k_{A'B'}\cdot k_{C'D'}=-6\)

由性质3, \(M',N'\) 仍为 \(P'A',P'B'\) 的中点

由垂径定理, \(OM'\bot P'A',ON'\bot P'B'\)

所以 \(O,N',M',P'\) 四点共圆,则 
$$\frac{P'G'}{M'G'}=\frac{N'G'}{OG'}=\frac{P'N'}{OM'}$$
$$\frac{P'G'}{N'G'}=\frac{M'G'}{OG'}=\frac{P'M'}{ON'}$$

所以 \(\frac{P'G'^2}{OG'^2}=\left(\frac{P'N'\cdot P'M'}{OM'\cdot ON'}\right)^2=\left(k_{A'B'}\cdot k_{C'D'}\right)^2=36\) 

故 \(\frac{P'G'}{OG'}=6\) 

又 \(OG'+P'G'=OP'=\frac{2\sqrt{3}}{3}\) 

故 \(OG'=\frac{2\sqrt{3}}{3}\cdot\frac{1}{7}\,,OG=\frac{2}{7}\) 

即 \(MN\) 过定点 \(\left(0,\frac{2}{7}\right)\) 
\end{example}
\subsection{习题}
\begin{exercise}{2019全国II理}
已知 \(C:\frac{x^2}4+\frac{y^2}2=1\),过坐标原点的直线交 \(C\) 于 \(P,Q\) 两点,\(P\) 在第一象限,\(PE\bot x\) 轴,垂足为 \(E\),联结 \(QE\) 并延长交 \(C\) 于 \(G\).求证:\(\triangle PQG\) 是直角三角形
\end{exercise}
\begin{exercise}{2020全国I理}
已知椭圆 \(\Gamma:\frac{x^2}{9}+y^2=1\) , \(A,B\) 是椭圆的左、右顶点,过直线 \(l:x=6\) 上一点 P 作 \(PA,PB\) 分别交椭圆于 \(C,D\) ,证明: \(CD\) 过定点
\end{exercise}

\begin{exercise}{}
 \(P,Q\) 是椭圆 \(\Gamma:\frac{x^2}{8}+\frac{y^2}{2}=1\) 上两动点, \(\angle PAQ\) 的角平分线恒垂直于 \(x\) 轴,试判断直线 \(PQ\) 的斜率是否为定值
\end{exercise}