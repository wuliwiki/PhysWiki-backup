% 迭代加深
% 迭代加深|DFS|算法

\pentry{深度优先搜索(DFS)
\upref{DFS}}

迭代加深是用于优化 DFS 的,DFS 是沿着一条路径一直往下走,一条路走到黑,直到没有路可走了才回溯。但如果存在一课搜索树,树的深度很大,但答案却在深度很浅的右半部分子树中,DFS 会搜很多无用的结点。

如下图,答案在深度很浅的红色部分,普通的 DFS 会先遍历蓝色部分,从而浪费大量的时间。

\begin{figure}[ht]
\centering
\includegraphics[width=10cm]{./figures/ID_1.png}
\caption{搜索树} \label{ID_fig1}
\end{figure}

迭代加深的基本思想是:加一个搜索深度的限制,每次只从不超过深度限制的部分进行搜索,如果在限制搜索内无法搜索到答案,那么就将限制扩大一倍。迭代加深看起来很像 BFS,但两者之间还是有区别的,BFS 每次只是扩展当前结点相邻的一层,但迭代加深的本质还是 DFS,还是会沿着一条路径搜索。BFS 的空间复杂度是指数级别的,但 DFS 是和深度呈正比的。假设当前的深度限制为 $k$,每次深度限制增加的时候,虽然会重复搜索 $1 \sim k - 1$ 层的结点,但重复的部分比起来总比深度很深、搜索结点呈指数级增长的普通 DFS 要好。

\textbf{例题:\href{http://poj.org/problem?id=2248}{加成序列}。}

数据范围 $n$ 最大为 $100$,搜索树的深度最大为 $100$,但加成序列的增长规模却很大,所以本题就是搜索深度很深,但答案深度很浅的情况,典型的迭代加深的使用场景,所以就可以使用迭代加深。

搜索顺序为:依次枚举序列中的每个位置可以填什么数。

剪枝优化:
\begin{itemize}
\item 剪枝一(优化搜索顺序):可以从大到小枚举每个位置填的数,就可以更快的触发剪枝条件。
\item 剪枝二(排除等效冗余):因为要找到 $k = x_i + x_j$,假设当前的 $k = 5$,序列中存在 $(1, 4)$ 和 $(2, 3)$ 这两种情况,所以可能会重复计算同一个数,所以可以开一个数组用于判断一个数有没有别加过。
\item 剪枝三(最优化剪枝):如果当前由两个数计算出来的和大于 $n$ 了,直接返回;如果当前计算出来的数小于前一个数,那么也直接返回,因为要保证序列严格单调递增。
\end{itemize}

\textbf{C++ 代码:}

\begin{lstlisting}[language=cpp]
int n, path[110];

bool dfs(int u, int depth)
{
    if (u > depth) return false;    // 不能超过深度限制
    if (path[u - 1] == n) return true;  // 最后一个数如果等于 n 了

    bool st[N] = {0};
    // 倒着枚举,优化搜索顺序
    for (int i = u - 1; i >= 0; i -- )
        for (int j = i; j >= 0; j -- )
        {
            int s = path[i] + path[j];
            if (s > n || s <= path[u - 1] || st[s]) continue;

            st[s] = true;
            path[u] = s;
            if (dfs(u + 1, depth)) return true;
        }

    return false;
}

int main()
{
    path[0] = 1;    // 第一个数一定是 1
    while (cin >> n, n)
    {
        int depth = 1;
        while (!dfs(1, depth)) depth ++ ;   // 无法找到答案,搜索深度限制就加一
        for (int i = 0; i < depth; i ++ ) cout << path[i] << ' ';
        cout << endl;
    }
    return 0;
}
\end{lstlisting}