% 绝热近似(量子力学)
% license Usr
% type Tutor

\begin{issues}
\issueTODO
\end{issues}

\pentry{薛定谔方程(单粒子一维)\upref{TDSE11},量子简谐振子(升降算符法)\upref{QSHOop}}

\footnote{参考 Griffiths\cite{GriffE} 的章节: The Adiabatic Approximation、 Shankar\cite{Shankar} 的 Chap18-P478、 Wikipedia \href{https://en.wikipedia.org/wiki/Adiabatic_theorem}{相关页面}。}量子力学中,\textbf{绝热近似(adiabatic approximation)}说的大概是: 若系统初始时处于某个离散非简并的本征态,那么当哈密顿量随时间缓慢改变时(改变的特征时间远大于本征态的), 那改变过程中波函数将仍然处于同一个本征态,但整体相位会发生某种改变。下面先给出定量结论,证明留到文末。

令含时薛定谔方程为(\autoref{eq_TDSE11_6}~\upref{TDSE11})
\begin{equation}
H(t)\Psi(t) = \I\hbar\dot\Psi(t)~.
\end{equation}
当系统不存在简并时, 绝热近似下含时薛定谔方程的通解可以表示为($C_n$ 为常数,由初始波函数决定)
\begin{equation}\label{eq_AdiaQM_2}
\Psi(t) \approx \sum_n C_n \psi_n(t) \E^{\I\gamma_n(t)}\E^{\I\theta_n(t)}~.
\end{equation}
其中 $\psi_n(t)$ 是 $H(t)$ 一组正交归一本征态,任意时刻都满足不含时薛定谔方程(时间看作数学参数)
\begin{equation}\label{eq_AdiaQM_3}
H(t)\psi_n(t) = E_n\psi_n(t)~.
\end{equation}
和正交归一化
\begin{equation}\label{eq_AdiaQM_6}
\braket*{\psi_m(t)}{\psi_n(t)} = \delta_{m,n}~.
\end{equation}
另外为了方便且不失一般性本文规定 $\psi_n(t)$ 始终是实值函数(否则有可能出现一个随时间变化的整体相位让事情更复杂)。

两个相位分别为(都是实值函数)
\begin{equation}
\theta_n(t) = -\frac{1}{\hbar} \int_0^t E_n(t')\dd{t'}~.
\end{equation}
\begin{equation}
\gamma_m(t) = \I\int_0^t \braket*{\psi_m(t')}{\dot\psi_m(t')}\dd{t'}~,
\end{equation}

\begin{example}{}
\begin{enumerate}
\item 当无限深势阱\upref{ISW}缓慢变长。
\item 量子简谐振子(升降算符法)\upref{QSHOop}的劲度系数 $k$ 缓慢变化。
\end{enumerate}
\end{example}

容易看出若 $H(t)$ 不随时间变化时,通解就回到了熟悉的通解(\autoref{eq_TDSE11_5}~\upref{TDSE11})
\begin{equation}
\Psi(t) = \sum_n C_n \psi_n \E^{-\I E_n t/\hbar}~.
\end{equation}

\autoref{eq_AdiaQM_2} 中 $C_n$ 为常数是一个很有力的结论。它告诉我们若开始时波函数处于某个(非简并)本征态,那么它将始终(近似)处于该本征态。

该理论在对分子的计算中有广泛的应用,且有一个响亮的名字,叫\textbf{波恩—奥本海默近似(Born–Oppenheimer approximation)}。 这是因为在分子运动中,原子核的运动速度通常要比电子慢得多,使绝热近似效果较好。

同为含时近似理论,绝热近似和含时微扰理论\upref{TDPTc}有什么区别呢? 前者不要求 $H(t)$ 缓慢变化,例如用激光波包对原子光电离时,电场随时间的周期变化往往并不算慢。 那可以使用绝热近似的情况是否可以使用含时微扰理论呢? 理论上可以,但计算比较麻烦,因为含时微扰使用初始的本征态展开任意时刻的波函数。

\subsection{能级分裂}
\pentry{一阶不含时微扰理论(量子力学)\upref{TIPT}}
若考虑的时间段内,只有初始的一瞬间存在简并, 那么可以认为这个瞬间波函数几乎不发生变化(毕竟 $H(t)$ 是缓慢变化),令 $\psi_n(0)$ 取好量子态\upref{TIPT},并假设系统始终是非简并的即可。
\begin{example}{}
给氢原子的任意束缚态 $\psi_{n,l,m}$ 缓慢施加外电场或磁场(参考 “类氢原子斯塔克效应(微扰)\upref{HStark}”,以及“塞曼效应\upref{ZemEff}”)。注意 $\psi_{n,l,m}$ 并不是好本征态,需要先做投影。
\addTODO{推导}
\end{example}

\subsection{推导}
若哈密顿量不随时间改变,
\begin{equation}
\Psi_n(t) = \psi_n \E^{-\I E_n t}~.
\end{equation}
若随时间改变, 本征态和本征值都变为时间的函数 $\psi_n(t)$ 和 $E_n(t)$。 但仍然正交归一。 此时的含时波函数仍然可以用它们展开
\begin{equation}
\Psi(t) = \sum_n c_n(t) \psi_n(t) \E^{\I \theta_n(t)}~,
\end{equation}
其中
\begin{equation}
\theta_n(t) = -\frac{1}{\hbar} \int_0^t E_n(t')\dd{t'}~.
\end{equation}
代入含时薛定谔方程
\begin{equation}\label{eq_AdiaQM_1}
H(t)\Psi(t) = \I \dot \Psi(t)~,
\end{equation}
得
\begin{equation}\label{eq_AdiaQM_5}
\dot c_m(t) = -\sum_n c_n \braket*{\psi_m}{\dot\psi_n}\E^{\I(\theta_n-\theta_m)}~.
\end{equation}
另外对\autoref{eq_AdiaQM_3} 求时间偏导得
\begin{equation}\label{eq_AdiaQM_4}
\mel*{\psi_m}{\dot H}{\psi_n} = (E_n-E_m)\braket*{\psi_m}{\dot\psi_n} + \delta_{m,n}E_n~.
\end{equation}
对\autoref{eq_AdiaQM_6} 求导可以证明矩阵 $\braket*{\psi_m}{\dot\psi_n}$ 是一个反对称矩阵,即满足
\begin{equation}\label{eq_AdiaQM_7}
\braket*{\psi_m}{\dot\psi_n} = -\braket*{\psi_n}{\dot\psi_m}~.
\end{equation}
注意对角元为零。 也就是说矩阵 $\I\braket*{\psi_m}{\dot\psi_n}$ 是对角元为零的厄米矩阵。
\autoref{eq_AdiaQM_4} 代入\autoref{eq_AdiaQM_5} 得
\begin{equation}\label{eq_AdiaQM_8}
\dot c_m(t) = - \sum_{n}^{E_n\approx E_m} c_n \braket*{\psi_m}{\dot\psi_n}\E^{\I(\theta_n-\theta_m)}
- \sum_{n}^{E_n\ne E_m} c_n \frac{\mel*{\psi_m}{\dot H}{\psi_n}}{E_n-E_m}\E^{\I(\theta_n-\theta_m)}~.
\end{equation}
其中第一个求和中的 $n$ 满足是对所有在 $t$ 取值范围内可能使 $E_n(t)$ 和 $E_m(t)$ 相等或非常接近;而所有剩下的 $n$ 都放到第二项的求和中。 现在还没有使用任何近似。 绝热近似就在于假设 $\dot H$ 非常小,从而忽略该式中第二个求和。 注意当 $E_n(t)$ 和 $E_m(t)$ 非常接近时该近似可能会失效, 所以我们把它保留到第一个求和中不予忽略。 下文我们将详细讨论这点。

\subsection{非简并情况}
若在考虑的时间区间内, $H(t)$ 始终没有发生简并, 那么\autoref{eq_AdiaQM_8} 的第一个求和就只有 $n=m$ 一项且为零
\begin{equation}
\dot c_m(t) = - \sum_{n}^{n\ne m} c_n \frac{\mel*{\psi_m}{\dot H}{\psi_n}}{E_n-E_m} \E^{\I(\theta_n-\theta_m)}~.
\end{equation}
现在还没有使用任何近似。忽略第二项,得到一维齐次亥姆霍兹方程\upref{HmhzEq},解得
\begin{equation}
c_m(t) = c_m(0)\E^{\I\gamma_m(t)}~,
\end{equation}
其中
\begin{equation}
\gamma_m(t) = \I \int_0^t \braket*{\psi_m(t')}{\dot\psi_m(t')}\dd{t'}~.
\end{equation}
这就得到了\autoref{eq_AdiaQM_2}。根据\autoref{eq_AdiaQM_7},$\gamma_m(t)$ 恒为实数。

\subsection{简并情况}
忽略\autoref{eq_AdiaQM_8} 的第二项,有
\begin{equation}
\dot c_m(t) = \sum_{n}^{E_n(t)=E_m(t)} c_n \braket*{\psi_m}{\dot\psi_n}\E^{\I(\theta_n-\theta_m)}~.
\end{equation}
这是齐次亥姆霍兹方程组,若把厄米矩阵 $\I\braket*{\psi_m}{\dot\psi_n}\E^{\I(\theta_n-\theta_m)}$ 记为矩阵 $\mat A$,再令 $\mat A$ 不包含在求和中的矩阵元为零,那么 $\mat A$ 就是一个块对角矩阵,每个对角块代表一个本征子空间(或者若干个可能在某时刻本征值相同的本征子空间张成的空间),不同子空间之间不存在耦合。所有 $c_m(t)$ 记为列向量 $\bvec c(t)$,上式可以表示为矩阵乘法
\begin{equation}
\dot{\bvec c}(t) = \mat A(t) \bvec c(t)~.
\end{equation}
其通解为(引用未完成)
\begin{equation}
\bvec c(t) = \mat U(t)\bvec c(0)~.
\end{equation}
其中
\begin{equation}
\mat U(t) = \hat{\mathcal{T}}\exp[-\I \int_0^t \mat A(t') \dd{t'}]~
\end{equation}
是一个块对角的酉矩阵\upref{UniMat},也就是每个对角块都分别是一个酉矩阵。这类似于薛定谔方程的演化子(链接未完成),且每个子空间独立演化,概率保持不变。

\addTODO{有没有可能即使 $E_m=E_n$,$\braket*{\psi_m}{\dot\psi_n}$ 也恒为零呢?什么时候?例如缓慢增加氢原子的核电荷时 $\psi_{n,l,m}$ 之间会耦合嘛? stark 的好本征态之间呢?}

\subsubsection{能级分裂与合并}
\autoref{eq_AdiaQM_8} 的两个求和中,至于哪些项放在第二个求和进而被忽略,可能取决于时刻 $t$。 这就是说矩阵 $\mat A(t)$ 的对角块可能有些时候会发生拆分或合并(多个对角块发生耦合后合并为一个)。 但若这种合并的持续时间只有很短乃至一瞬间,那我们可以认为合并前后波函数不发生变化,也就是假设合并不存在。

\addTODO{或许可以搞个矩阵来数值验证一下,例如氢原子的 stark 效应是否真的可以这么搞}

\addTODO{啥时候讲 avoided crossing 啊……}
