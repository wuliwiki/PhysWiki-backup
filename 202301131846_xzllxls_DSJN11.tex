% 暨南大学 2011 年信息科学技术学院830数据结构考研真题
% 暨南大学 2011 年信息科学技术学院830数据结构考研真题

\subsection{一.选择题(每题2分,共30分)}
1.算法分析的目的是( )。 \\
A.找出数据结构的合理性 \\
B.研究算法中的输入和输出关系 \\
C.分析算法的效率以求改进 \\
D.分析算法的易读性和文档性

2.下列函数中渐近时间复杂度最小的是( )。 \\
A. T1(n) =1ogan+5000n \\
B. T2 (n) =n"-8000n \\
C. T3 (n) =n'+5000n \\
D. T4(n) =2nlogn-1000n

3.线性表的动态链表存储结构与顺序存储结构相比,优点是( )。 \\
A.所有的操作算法实现简单 \\
B.便于随机存取 \\
C.便于插入与删除 \\
D.便于节省存储器空间

4.若进栈序列为1,2,3, 4,5, 6,且进栈和出栈可以穿插进行,则可能出现的出栈序列为( )。 \\
A. 3,2,6,1,4,5 \\
B. 5,6,4,2,3,1 \\
C.5,1,2,3,4,6 \\
D. 3,4,2, 1,6, 5

5.顺序存储的线性表的第一-个元素的存储地址是100,每个元素的长度为4,则第4个元素的存储地址是( )。 \\
A.108 \\
B.112 \\
C.116 \\
D.120

6.在任意一一棵二叉树的先序序列和后序序列中,各叶子之间的相对次序关系( )。 \\
A.不一定相同 \\
B.互为逆序 \\
C.都不相同 \\
D.都相同

7.高度为5的二叉树至多有结点数为()。 \\
A. 63 \\
B. 32 \\
C. 31 \\
D. 64

8.图的邻接矩阵表示法适用于表示( )。 \\
A.无向图 \\
B.有向图 \\
C.稠密图 \\
D.稀疏图

9.在一个单链表中,若p所指的结点不是最后一个结点,在p之后插入s所指的结点,则执行( )。 \\
A. s->next=p; p->next=s \\
B. p->next=s; s->next=p \\
C. p=s; s->next=p->next \\
D. s->next=p->next; p->next=s

10.若在线性表中采用折半查找法查找元素,该线性表应该是( )。 \\
A.元素按值有序 \\
B.采用顺序存储结构 \\
C.元素按值有序且采用顺序存储结构 \\
D.元素按值有序且采用链式存储结构

11.已知一棵二叉树结点的先序序列为ABDGCFK, 中序序列为DGBAFCK, 则结点的后序序列为(). \\
A. GDBFKCA \\
B. DGBFKCA \\
C. KFCABDG \\
D. CAFKGDB

12.对于元素是整数(占2个字节)的n行n列对称矩阵A,采用以行序为主的压缩存储方式存储到一维数组s[n* (n+1)/2]中(下三角),若A[1][1]的起始地址是400,问元素A[8][5]的存储地址是( 5心). \\
A. 432 \\
B. 563 \\
C. 484 \\
D. 464

13.在所有排序方法中,关键字的比较次数与记录的初始排列无关的是()。 \\
A. Shell排序 \\
B.冒泡排序 \\
C. 直接插入排序 \\
D.直接选择排序

14.具有6个顶点的无向图至少应有( ) 条边才能确保是一个连通图。 \\
A. 5 \\
B. 6 \\
C. 7 \\
D. 8

15. 如果T2是由树T1转换而来的二叉树,那T1中结点的先序就是T2中结点的( )。 \\
A. 先序 \\
B. 中序 \\
C. 后序 \\
D. 层次序

\subsection{二.填空题(每题2分,共20分)}
1.在数据结构中, 数据的逻辑结构分()和().

2.若对关键字序列(12, 18, 4,3, 6, 13,2, 9, 19, 8)进行快速排序(以第-一个元素为支点),则第一趟排序得到的结果为().

3.堆排序采用了( )作为其数据结构,如果希望第一次就能找出最小关键字记录,就建立( )堆。

4.二叉树中度为0的结点数为30,度为1的结点数为30,总结点数为(  ).

5.向栈中压入元素的操作是先()后().

6.在()的情况下,链队列的出队操作需要修改尾指针。

7.所谓连通图G的生成树,是G的包含其全部n个顶点的一一个 极小连通子图。它必定包含且:包含G的()条边.

8.对于一个有向图,若一个顶点的度为k1, 出度为k2, 则对应邻接表中该顶点单链表中的边节点数为().

9.设GetHead(p)为求广义表p的表头函数,GetTail (p)为求广义表p的表尾函数。其中()是函数符号,运算GetTail (GetHead(a, b), (c, d, e)))的结果是().

10.对n个结点进行快速排序,最大比较次数是().

\subsection{三,判断题(每题1分,共10分,正确的选t,错误的选f)}
1.一个广义表的表尾总是一个广义表。( )

2.顺序表用一维数组作为存储结构,因此顺序表是一-维数组。( )

3.双循环链表中, 任一结点的前驱指针均为不空。( )

4.存储图的邻接矩阵中, 邻接矩阵的大小不但与图的顶点个数有关,而且与图的边数也有关。()

5.当从一个最小堆中删除一个元素时,需要把堆尾元素填补到堆顶位置,然后再按条件把它逐层向下调整,直到调整到合适位置为止。(  )

6.栈和队列都是顺序存取的线性表,但它们对存取位置的限制不同。( )

7.一个无序的元素序列可以通过构造-棵二叉排序树而变成-一个有序的元素序列。(t)

8.一棵m阶B+树中每个结点最多有m个关键码,最少有2个关键码。( )

9.拓扑排序是一种内部排序的算法。()

10.空串与空格相同。()

