% 合同变换
% license Xiao
% type Tutor


给定域$\mathbb F$上的线性空间$V$,$f$为对称双线性函数:$V\times V\to \mathbb F$,也就是二次型$q(v),v\in V$。再给定$V$上的一组基$\{\boldsymbol e_i\}$,则二次型可以表示为:
\begin{equation}
\eta_{ij}=f(\boldsymbol {\mathrm e}_i,\boldsymbol {\mathrm e}_j)~,
\end{equation}
对于$V$的另一组基$\{\boldsymbol \theta_i\}$,二次型可以表示为$\Theta_{ij}=f(\boldsymbol {\mathrm \theta}_{i},\boldsymbol {\mathrm \theta}_{j})$。设过渡矩阵为$T^i_j,\boldsymbol {\theta}_i=T^j_i\boldsymbol e_j$。那么我们可以看到二次型在不同基下的表示是如何通过过渡矩阵联系在一起的:
\begin{equation}
\begin{aligned}
\Theta_{ij}&=f(\boldsymbol {\mathrm \theta}_{i},\boldsymbol {\mathrm \theta}_{j})\\
&=f(T^r_i\boldsymbol e_r,T^s_j\boldsymbol e_s)\\
&=T^r_i \eta_{rs}T^s_j~,
\end{aligned}
\end{equation}
把上式两端的二次型张量写作(1,1)型矩阵,则有:
$$\Theta^i_j=T^i_r\eta^r_sT^s_j~,$$
写成更清楚的矩阵形式,就是我们的合同变换:$\Theta=T^T\eta T$。
\begin{definition}{矩阵合同变换}
设A,B为n阶矩阵,若存在可逆矩阵C,使得
\begin{equation}
C^TAC=B~,
\end{equation}
则称矩阵A与B合同,记作$A\simeq B$
\end{definition}
\begin{exercise}{}
证明合同关系是等价关系。即满足
\begin{enumerate}
\item 反身性  $A\simeq A$
\item 对称性  若$A\simeq B$,则$B\simeq A$
\item 传递性  若$A\simeq B,\quad B\simeq C$,则$A\simeq C$
\end{enumerate}
\end{exercise}
从定义式我们可以发现,合同变换不改变矩阵的秩。yo