% 热传导定律
% 热传导

\subsection{傅里叶传导定律}
定义热流密度 $\bvec h$:单位时间单位面积的热流量.傅里叶热传导定律说的是

\begin{equation}
\bvec h = -\kappa \bvec \nabla T
\end{equation}

其中 $\kappa$ 为系统的热导率.这意味着热量会沿温度梯度的方向传导.

现在简化系统模型,设各层温度不均匀,在同一个高度上各处温度相等.那么傅里叶热传导定律可以简化为
\begin{equation}
H=\frac{\Delta Q}{\Delta t}=-\kappa \frac{\dd T}{\dd z}S
\end{equation}

由于单位体积内需要吸收 $c\rho \Delta T$ 的热量才能升高 $\Delta T$ 的温度,所以可以得到积分关系式
\begin{equation}
\int_V c\rho \frac{\partial T}{\partial t} \dd V = \int_V \frac{\partial U}{\partial t} \dd V= \oint_S \kappa \bvec \nabla T \cdot \bvec \dd S
\end{equation}

由此可以得到微分表达式 $c\rho \frac{\partial T}{\partial t} = \kappa \nabla^2 T$,即
\begin{equation}
\frac{\partial T}{\partial t}=k\nabla^2 T
\end{equation}

\subsection{几个例子}

\subsection{微观解释}