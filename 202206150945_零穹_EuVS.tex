% 欧几里得矢量空间
% 欧几里得矢量空间

\pentry{正定二次型\upref{DeQua}}

\begin{definition}{欧几里得矢量空间}\label{EuVS_def1}
定义在实数域 $\mathbb R$ 上的矢量空间 $V$ ,若其带有一个对称的双线性型 $(x,y)\mapsto(x|y)$(\autoref{MulMap_def2}~\upref{MulMap}),且对应的二次型 $ x\mapsto(x| x)$ 是正定的,则称空间 $V$ 是一个\textbf{欧几里得矢量空间}.
\end{definition}
通常,对称的双线性型 $(*|*)$ 在 $,y$ 处的值称为它们的\textbf{纯量积}.我们用符号 $(*|*)$ 代替通常的 $f(x,y)$.这里我们不用 $(x,y)$ 和 $\langle x,y\rangle$ 代替纯量乘积,出于这样的考虑:已经有简单的矢量对 $(x,y)$ ,它是笛卡尔积 $V\times V$ 的元素(\autoref{Set_eq1}~\upref{Set});而 $\langle x,y\rangle $ 又是矢量 $x,y$ 生成的子空间(\autoref{VecSpn_def1}~\upref{VecSpn}).
再一次把纯量乘积的性质列出来:
\begin{enumerate}
\item \textbf{对称性:}$(x|y)=(y|x)$;
\item \textbf{线性:}$(\alpha x+\beta y| z)=\alpha( x| z)+\beta( y| z)$;
\item \textbf{正定性:}$( x| x)>0,\;\forall x\neq0(( 0| x)=0)$.
\end{enumerate}
\begin{example}{}\label{EuVS_ex1}
次数 $\leq n-1$ 的多项式\upref{OnePol} (其域为实数域 $\mathbb R$)对通常的加法和数乘构成一个矢量空间 $V=P_n$.对任意两个矢量(多项式)$f,g\in P_n$ ,数 
\begin{equation}\label{EuVS_eq1}
(f|g)=\int_a^b f(x)g(x)\dd x
\end{equation}
给出 $P_n$ 上向量间的纯量乘积.此纯量乘积是用\autoref{EuVS_eq1} 在连续函数(区间 $[a,b]$ 上)的无穷维矢量空间 $C(a,b)$ 上给出的.相应的无穷维欧几里得空间则表示为 $C_2(a,b)$.
\end{example}
\begin{example}{}
欧几里得矢量空间 $V$ 的任一子空间 $U$ 本身也是欧几里得矢量空间,因为 $V$ 中纯量乘积在 $U$ 中的限制定义了双线性函数 $U\times U\rightarrow\mathbb R$,且保持纯量乘积的性质.特别的,域 $\mathbb R$ 本身可看成是个1维的实的矢量空间.
\end{example}
\begin{definition}{长度}\label{EuVS_def2}
在欧几里得矢量空间 $V$ 中,称非负实数
\begin{equation}
\abs{\abs{ v}}=\sqrt{( v| v)}
\end{equation}
是任意矢量 $v\in V$ 的\textbf{长度}或 \textbf{模}.长度为1的矢量称为\textbf{标准的}.
\end{definition}
因为 $(v|v)\neq0$ ,所以任意矢量 $v$ 的长度都是完全确定的.

容易验证下面几个性质:
\begin{enumerate}
\item $v\neq0\Rightarrow \abs{\abs{v}}>0$.
\item $\abs{\abs{\lambda v}}=\abs{\lambda}\cdot\abs{\abs{ v}}$
\end{enumerate}


\begin{example}{矢量的标准化}\label{EuVS_ex2}
任意矢量 $v$ 乘以它的长度的倒数$\frac{1}{\abs{\abs{ v}}}$便可将其标准化,即
\begin{equation}
\abs{\abs{\frac{1}{\abs{\abs{v}}}v}}=1
\end{equation}
\end{example}
\begin{theorem}{柯西-布尼亚科夫斯基不等式}\label{EuVS_the1}
欧几里得向量空间中,对任意矢量 $x,y\in V$,成立不等式
\begin{equation}\label{EuVS_eq4}
\abs{(x|y)}\leq\abs{\abs{x}}\,\abs{\abs{\bvec y}}
\end{equation}
且等号仅在 $y=\lambda_0 x,\;\lambda_0\in\mathbb R$ (即矢量共线)时成立.
\end{theorem}
\textbf{证明:}由纯量乘积的性质
\begin{equation}\label{EuVS_eq2}
\begin{aligned}
&(\lambda x-y|\lambda x- y)\\
&\underset{\text{线性}}{=}\lambda^2( x|x)-\lambda (x|y)-\lambda (y|x)+(y|y)\\
&\underset{\text{对称性}}{=}\lambda^2(x|x)-2\lambda (x|y)+(y|y)\\
&\underset{\text{正定性}}{\geq}0
\end{aligned}
\end{equation}
\autoref{EuVS_eq2} 最后一不等式可看成关于 $\lambda$ 的二次三项式,其判别式满足
\begin{equation}\label{EuVS_eq3}
(2(x|y))^2-4(x| x)(y|y)\leq0
\end{equation}
即
\begin{equation}
\begin{aligned}
(x|y)^2&\leq(x|x)(y|y)\\
&\Downarrow\\
\abs{(x| y)}&\leq\abs{\abs{x}}\,\abs{\abs{y}}
\end{aligned}
\end{equation}
当等号成立时,\autoref{EuVS_eq3} 应取等,即二次三项式仅有一根 $\lambda_0$.对照\autoref{EuVS_eq2} 有
\begin{equation}
(\lambda_0 x- y|\lambda_0 x-y)=0
\end{equation}
即 $ y=\lambda_0 x$.

\textbf{证毕!}
\begin{corollary}{三角不等式}\label{EuVS_cor1}
矢量 $ x, y$ 与 $ x+ y$ 的长度满足不等式
\begin{equation}
\abs{\abs{ x\pm y}}\leq\abs{\abs{ x}}+\abs{\abs{ y}}
\end{equation}
\end{corollary}
\textbf{证明:}由\autoref{EuVS_the1} 
\begin{equation}
\begin{aligned}
\abs{\abs{ x\pm y}}^2&=\abs{\abs{ x}}^2+\abs{\abs{ y}}^2\pm2( x| y)\leq\abs{\abs{ x}}^2+\abs{\abs{\bvec y}}^2+2( x| y)\\
&\leq \abs{\abs{ x}}^2+\abs{\abs{ y}}^2+2\abs{\abs{ x}}\cdot\abs{\abs{ y}}=(\abs{\abs{ x}}+\abs{\abs{ y}})^2
\end{aligned}
\end{equation}

\begin{example}{}
在空间 $C_2(a,b)$(\autoref{EuVS_ex1} ) 上,\autoref{EuVS_eq4} 就变为
\begin{equation}
\abs{\int_a^b f(x)g(x)\dd x}\leq\sqrt{\int_a^b f^2(x)\dd x}\cdot\sqrt{\int_a^b g^2(x)\dd x}
\end{equation}
\end{example}
\autoref{EuVS_the1} 意味着
\begin{equation}
-1\leq\frac{( x| y)}{\abs{\abs{ x}}\,\abs{\abs{y}}}\leq1
\end{equation}
也就是说,存在一个角 $\varphi$ ,使得
\begin{equation}
\cos\varphi=\frac{( x| y)}{\abs{\abs{ x}}\,\abs{\abs{ y}}}
\end{equation}
\begin{definition}{夹角}\label{EuVS_def3}
欧几里得矢量空间中,由
\begin{equation}
\cos\varphi=\frac{( x| y)}{\abs{\abs{ x}}\,\abs{\abs{ y}}}
\end{equation}
确定的角 $\varphi$ 称之为矢量 $x$ 和 $y$ 的\textbf{夹角}. 
\end{definition}
\begin{definition}{正交}
如果矢量的夹角为 $\pi/2$,亦即 $( x| y)=0$,则称它们是\textbf{正交的},记作 $ x\perp y$.
\end{definition}
\begin{theorem}{毕达哥拉斯定理}
\begin{equation}
x\perp y\Rightarrow\abs{\abs{ x+ y}}^2=\abs{\abs{ x}}^2+\abs{\abs{ y}}^2
\end{equation}
\end{theorem}
\begin{exercise}{}
试证明对两两正交的矢量 $v_1,\cdots, v_n$,成立
\begin{equation}
\abs{\abs{ v_1+\cdots+ v_n}}^2=\abs{\abs{ v_1}}^2+\cdots+\abs{\abs{ v_n}}^2
\end{equation}
\end{exercise}
\begin{example}{}
与给定矢量 $v$ 正交的所有矢量的集合是一个子空间,该空间称为 $v$ 的\textbf{正交补}.事实上,若
\begin{equation}
 x\perp v,\quad y\perp v
\end{equation}
则
\begin{equation}
(\alpha x+\beta y| v)=\alpha( x| v)+\beta( y| v)=0,\quad \forall\alpha,\beta\in\mathbb R
\end{equation}
\end{example}
若矢量 $v$ 与子空间 $U$ 任意矢量都正交,则说 $v$ 正交于子空间 $U$,记作 $v\perp U$.
\begin{definition}{正交补}
设 $U$ 是 $V$ 的子空间,则集合
\begin{equation}
\{ x| x\perp U\wedge x\in V\}
\end{equation}
也是一个子空间,称为 $U$ 的\textbf{正交补},记作 $U^{\perp}$ .
\end{definition}
\begin{definition}{正交基底}
称欧几里得空间 $V$ 的基底 $(e_1,\cdots,e_n)$ 是正交的,如果
\begin{equation}
( e_i| e_j)=0,\;i\neq j=1,\dots,n.
\end{equation}
若此外还有 $( e_i| e_i)=1,\;i=1,\cdots ,n$ ,则称此基底为\textbf{标准正交基底}.
\end{definition}

和所有的正定型一样,欧几里得矢量空间必有标准正交基底(因为对称双线性型必有规范基底\autoref{GuaOQu_the1}~\upref{GuaOQu},而正定的双线性型是非退化的\autoref{DeQua_def1}~\upref{DeQua},而每一基都可标准化\autoref{EuVS_ex2} ).
\begin{exercise}{}
在标准正交基底下,矢量 $x$ 在基矢量 $e_i$ 上的坐标 $x_i$为
\begin{equation}
(\bvec x| e_i)=x_i
\end{equation}
\end{exercise}
\begin{definition}{投影}
称纯量乘积 $( x| e)$ 是矢量 $x$ 在直线 $\langle e\rangle_{\mathbb R}$ 上的\textbf{投影},其中 $ e$ 是个长度为1的矢量.  
\end{definition}
如此,矢量 $ x$ 在标准正交基底 $( e_1,\cdots, e_n)$ 下的坐标与 $x$ 在坐标轴 $\langle e_i\rangle_\mathbb{R}$ 上的投影一致. 