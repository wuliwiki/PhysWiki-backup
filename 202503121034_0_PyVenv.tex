% Python 虚拟环境 venv 笔记
% license Xiao
% type Note

\pentry{Python 简介与安装\nref{nod_Python}}{nod_702f}

\verb`venv` 是最常用的 python 虚拟环境管理工具(Python 3.3+ 内置),其他还有 \verb`conda, virtualenv, pipenv`。

\subsection{使用}
\begin{itemize}
\item 要在当前路径创建新环境, 用 \verb`python3 -m venv 环境名`(\verb`-m` 选项是把一个 python 包作为一个程序运), 然后会出现一个名为 \verb`环境名` 的文件夹。
\item 在 Ubuntu22.04 下, 该命令会实测失败并提示安装 \verb`sudo apt install python3.10-venv`, 装好再试即可。
\item 运行 \verb`source 环境名/bin/activate` 进入刚创建的虚拟环境。 此时 bash 的行首会标示当前的环境名(\verb`(环境名) $`)。
\item 此时 \verb`pip3 install ...` 就可以在当前环境安装一些包。
\item \verb`pip3 list` 会发现当前几乎没有安装的包。 也可以安装新包。
\item \verb`deactivate` 退出虚拟环境。 再用 \verb`pip3 list` 就会看到系统默认的包。
\item 删掉 \verb`环境名` 文件夹即可删掉虚拟环境。
\item \verb`pip freeze > 依赖.txt` 可以保存当前环境的所有包的版本
\item \verb`pip install -r 依赖.txt` 可以恢复指定版本的包
\end{itemize}

\subsection{原理}
\begin{itemize}
\item venv 完全是通过修改环境变量实现的, 注意并不会像 \verb`chroot` 一样改变路径。
\item \verb`activate` 是个 bash 脚本,设置如下环境变量 \verb`VIRTUAL_ENV`(\verb`环境名` 的绝对路径), \verb`VIRTUAL_ENV_PROMPT`, \verb`PS1`(bash 提示符前标明环境名), \verb`PATH`(添加 \verb`环境名/bin` 到最开头)。
\item \verb`环境名/bin` 文件夹除了 activate 脚本外,还会有一些名为 \verb`python*` 的符号链接, 链接到应该是用于创建 venv 的 python 可执行文件。 另外还会有一些名为 \verb`pip*` 的内容一样的脚本文件。
\item \verb`环境名/lib` 文件夹里面应该是标准库和 pip 安装的库。
\item \verb`pyvenv.cfg` 文件里面有一些 venv 配置信息。
\end{itemize}

\subsection{回收}

\subsubsection{【这是 virtualenv 不是 venv】安装}
\begin{itemize}
\item \verb`virtualenv` 比 \verb`venv` 要老,但工作原理类似。
\item 安装: \verb`pip3 install virtualenv`
\item 在 ubuntu 上还需要安装 \verb`apt install python版本号-venv` 其中版本号可以用 \verb`python3 --version` 查看。
\end{itemize}
