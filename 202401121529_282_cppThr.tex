% C++ 多线程笔记(std::thread)
% license Xiao
% type Note

\begin{itemize}
\item 在 Linux 或其他 POSIX 系统上 \verb|std::thread| 底层通常使用 pthread\upref{pthred}。
\item \verb|#include <thread>|, \verb|#include <mutex>|
\item \verb|std::thread th(函数, arg1, arg2, ...)|; 创建一个线程, 调用\verb|函数|(可以是函数指针, 函数对象, lambda), \verb|arg| 是\verb|函数|的变量。
\item \verb|th.join()| 可以让主程序等待某个线程自己退出。
\item 声明一个全局变量 \verb|std::mutex m| 相当于 openmp 的 atomic 操作, 可以避免多个线程操作同一数据。
\item \verb|m.lock()| 给 mutex 上锁, 如果已经被别人上锁就暂停该线程并等待解锁。 \verb|m.try_lock()| 试图上锁, 如果已经被别人上锁就返回 \verb|false|。
\item \verb|m.unlock()| 解锁,以便别的线程上锁。
\item 为了避免在 \verb|m.lock()| 和 \verb|m.unlock()| 之间发生 throw(这样就无法自动解锁), 通常不直接调用他们, 而是用 \verb|std::lock_guard<std::mutex> guard(m)|。 constructor 相当于 lock, destructor 相当于 unlock。
\item \verb`unique_lock<>` 类似于 \verb`lock_guard` 但更灵活。
\item \verb|std::this_thread::sleep_for(std::chrono::seconds(2));| 可以让某个线程暂停
\item \verb`sleep_for` 不是 busy wait 而是像 \verb`cin` 等待时一样运行 cpu 执行别的东西,也就是让系统接管,直到某个事件发生然后再继续。
\item 相比之下, \textbf{自旋锁}就是指一个线程一直用一个空循环不断检查锁的状态直到解锁。
\end{itemize}

例程(编译时要给 linker 加上 \verb|-l pthread|):
\begin{lstlisting}[language=cpp]
#include <thread>
#include <mutex>
#include <chrono>
using namespace std;

mutex mut;

// A dummy function
void myfun(int id, int *i)
{
    if (id == 1)
        this_thread::sleep_for(chrono::milliseconds(100));
    lock_guard<mutex> guard(mut);
    printf("id = %d, i = %d\n", id, *i);
    int j = *i+1;
    if (id == 2)
        this_thread::sleep_for(chrono::milliseconds(100));
    *i = j;
}

int main()
{
    int i = 0;
    thread th1(myfun, 1, &i);
    thread th2(myfun, 2, &i);
    myfun(0, &i);
    th1.join();
    th2.join();
    return 0;
}
\end{lstlisting}

另外也可以把 \verb|mut| 声明为 \verb|myfun()| 中的一个 \verb|static| 变量。

另外注意 \verb|mutex| 不光适用于 \verb|std::thread|, 对任何其他 threading 库例如 \verb|pthread| 或者 OpenMP\upref{OpenMP} 都有效。

\addTODO{互斥锁、条件锁、读写锁以及自旋锁分别什么时候用?}
% 参考 https://www.zhihu.com/question/66733477

\subsubsection{thread\_local 变量}
\begin{itemize}
\item 在全局变量, 以及函数和类中的 static 变量的声明前面加 \verb|thread_local|,可以让其在每个线程中保持自己独立的 copy, 使函数变得 thread safe。 例如上面的 \verb|myfun| 中如果声明了 \verb|thread_local static vector<int> a;|, 那么每个线程第一次调用该函数时就会生成一个独立的变量 \verb|a|, 若一个线程多次调用该函数, 那么每次调用时 \verb|a| 都会有上次调用结束前的值。
\end{itemize}

\subsection{std::condition\_variable}
\verb`condition_variable` 用于 block 一个或者多个线程,直到某个另外的线程发送一个信号。

\begin{lstlisting}[language=cpp]
#include <iostream>
#include <thread>
#include <mutex>
#include <condition_variable>

// 把 condition_variable 需要的三个东西封装起来更方便使用
class ConditionVar {
private:
    std::mutex mtx;
    std::condition_variable cv;
    bool ready = false;

public:
    // 暂停一些线程
    void wait() {
        std::unique_lock<std::mutex> lock(mtx); // 上锁
        cv.wait(lock, [this]{ return ready; }); // 在线程休眠以前解锁
    }
    // 唤醒所有线程
    void notify_all() {
        std::lock_guard<std::mutex> lock(mtx);
        ready = true;
        cv.notify_all();
    }
    // 随机唤醒一个线程
    void notify_one() {
        std::lock_guard<std::mutex> lock(mtx);
        ready = true;
        cv.notify_one();
    }
};

ConditionVar cv;

void wait(int id) {
	printf("Thread %d started.\n", id);
    cv.wait();
	printf("Thread %d is resumed.\n", id);
}

void resume() {
    std::this_thread::sleep_for(std::chrono::seconds(1)); // 模拟一些工作
    cv.notify_all(); // 全部唤醒
    printf("resume signal sent!\n");
}

int main() {
    std::thread threads[3];
    
    for (int i = 0; i < 2; ++i)
        threads[i] = std::thread(wait, i);

    threads[2] = std::thread(resume);

    for (auto &th : threads)
        th.join();
}
\end{lstlisting}
运行结果
\begin{lstlisting}[language=cpp]
Thread 0 started.
Thread 1 started.
resume signal sent!
Thread 0 is resumed.
Thread 1 is resumed.
\end{lstlisting}
