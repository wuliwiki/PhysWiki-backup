% 氢原子的跃迁偶极子矩阵元(束缚态到连续态)
% keys 氢原子|偶极子|跃迁|微扰
% license Xiao
% type Tutor

\begin{issues}
\issueDraft
\end{issues}

\pentry{跃迁概率(含时微扰)\nref{nod_HionCr}}{nod_11be}

本文使用\enref{原子单位制}{AU}。

\begin{figure}[ht]
\centering
\includegraphics[width=10cm]{./figures/a540cf946ddeffb6.pdf}
\caption{$\abs{\mel{k_z}{z}{1,0,0}}$, 实线、虚线:平面波近似, 点: 无近似(代码: \verb`plot_hydrogen_trans_dipole.m`)} \label{fig_HyIon2_1}
\end{figure}

\subsection{长度规范(平面波近似)}
归一化的平面波和归一化的氢原子基态为(用平面波近似末态库仑函数)
\begin{equation}
\ket{\bvec k} = (2\pi)^{-3/2} \E^{\I \bvec k \vdot \bvec r}~,
\qquad \ket{0} = \frac{1}{\sqrt{\pi}} \E^{-r}~.
\end{equation}
长度规范下的跃迁偶极子, 可以在极坐标系中积分(令 $\uvec k$ 方向为极轴, $\uvec r$ 与其夹角为 $\theta$)
\begin{equation}
\mel{\bvec k}{\bvec r}{0}
=  \frac{\uvec k}{(2\pi)^{3/2}\sqrt{\pi}} \int_0^{+\infty} \int_0^\pi \E^{-r} \E^{-\I k r \cos\theta} r \cos\theta \cdot 2\pi r^2 \sin\theta \dd{\theta} \dd{r}~
\end{equation}
换元, 令 $u = \cos\theta$, 得\footnote{最后一步可通过 Wolfram Alpha 或 \enref{Mathematica}{Mma} 获得。}
\begin{equation}\label{eq_HyIon2_1}\ali{% 已检查多次
\mel{\bvec k}{\bvec r}{0} &= \frac{1}{\sqrt 2 \pi} \uvec k \int_0^{+\infty} r^3 \E^{-r} \int_{-1}^1 \E^{-\I k r u} u  \dd{u} \cdot \dd{r}\\
&=  \I\frac{\sqrt2 \uvec k}{k\pi}  \int_0^{+\infty} r^2 \E^{-r} \qty[\cos(kr) - \frac{1}{kr}\sin(kr)] \dd{r}\\
&= -\I \frac{8 \sqrt2}{\pi} \frac{\bvec k}{(k^2+1)^3}~,
}\end{equation}
注意这是一个纯虚数。 Matlab 代码如下:

\begin{lstlisting}[language=matlab, caption=hydrogen\_trans\_dipole\_plane\_approx\_z.m]
% hydrogen transition dipole, approximate Coulomb plane wave with plane wave
% since middle and right parts are symmetric,
%     assuming \bvec k in z+ direction
% <\bvec k|\bvec r|n,0,0> = [0, 0, <\bvec k|z|n,0,0>]
% output numerical integration and analytical result (eq_HyIon2_1)
function [dipole_z, dipole_analy_z] = ...
    hydrogen_trans_dipole_plane_approx_z(kz, Z, n)
% === params ===
rmax = 20; Nr = 200; Nth = 100;
% ==============
k = [0, 0, kz];
r = linspace(0, rmax, Nr); dr = rmax / (Nr-1);
th = linspace(0, pi, Nth); dth = pi / (Nth-1);
ph = 0;
[R, Th] = ndgrid(r, th);
zz = R .* cos(Th);
Psi_n = hydrogen_Psi(Z, n, 0, 0, r', th, ph);
% <\bvec k|\bvec r|0>
Psi_k = 1/(2*pi)^(3/2) * exp(1i*kz.*zz);
dipole_z = sum(sum(conj(Psi_k).*zz.*Psi_n .*R.^2.*sin(Th))) ...
    * dr * dth * 2*pi;
% eq_HyIon2_1
if n == 1
    dipole_analy_z = -1i*(8*sqrt(2))/pi * kz / (dot(k,k) + 1)^3;
end
end
\end{lstlisting}


\subsection{速度规范}
注意一阶微扰理论中的初态和末态波函数都是无微扰(无外场)情况下的, 与规范无关。 要计算 $\mel{\bvec k}{\grad}{0}$, 先看积分
% Merzbacher 19.87 上面一条
\begin{equation}
\int \exp(-\I \bvec k \vdot \bvec r) \exp(-r) \dd[3]{r} = \frac{8\pi }{(k^2 + 1)^2}~.
\end{equation}
使用算符 $\grad$ 的反厄米性得
\begin{equation}
\begin{aligned}
&\int \exp(-\I \bvec k \vdot \bvec r) \grad \exp(-r) \dd[3]{r}
= -\int [\grad \exp(\I \bvec k \vdot \bvec r)]^* \exp(-r) \dd[3]{r}\\
&= \I \bvec k \int \exp(-\I \bvec k \vdot \bvec r) \exp(-r) \dd[3]{r}
= \I \frac{8 \pi  \bvec k}{(k^2 + 1)^2}~,
\end{aligned}
\end{equation}
乘以归一化系数得
\begin{equation}\label{eq_HyIon2_2}
\mel{\bvec k}{\grad}{0} = \I\frac{2\sqrt{2}}{\pi}\frac{\bvec k}{(k^2 + 1)^2}~.
\end{equation}
该式代入\autoref{eq_SIcros_8} ($q = -1$)得微分截面为
\begin{equation}
% dipole approximation 下与第二项 Merzbacher 19.87 相同, 但他的据说与实验吻合, 相差应该不大。
\pdv{\sigma}{\Omega} = \frac{32}{mc\omega} \frac{k(\bvec k \vdot \uvec e)^2}{(k^2 + 1)^4}
= \frac{64}{mc} \frac{k(\bvec k \vdot \uvec e)^2}{(k^2 + 1)^5}~.
\end{equation}
对于质子数为 $Z$ 类氢原子有
\begin{equation}
\pdv{\sigma}{\Omega} = \frac{32 Z^5}{mc\omega} \frac{k(\bvec k \vdot \uvec e)^2}{(k^2 + Z^2)^4}~.
\end{equation}

\subsection{两种规范对比}
如果 $\ket{\bvec k}$ 是库仑函数(能量本征态)应该有(\autoref{eq_DipEle_3})
\begin{equation}
\mel{\bvec k}{\bvec r}{0} = -\frac{\mel{\bvec k}{\grad}{0}}{m\omega_{k0}}~.
\end{equation}
其中 $\omega_{k0} = k^2/2 + 1/2$, 但实际上\autoref{eq_HyIon2_1} 和\autoref{eq_HyIon2_2} 满足
\begin{equation}
\mel{\bvec k}{\bvec r}{0} = -2\frac{\mel{\bvec k}{\grad}{0}}{m\omega_{k0}}~.
\end{equation}
这说明在使用平面波近似库仑函数时, 长度规范的 transition amplitude 恰好是速度规范的 2 倍, 截面是四倍(待求证)。

教材中推导微分截面一般使用速度规范, 因为速度规范的结果与实验吻合更好。

\subsection{使用库仑平面波}
\subsubsection{长度规范}

理论上若把上面的平面波换成\enref{库仑平面波}{CulmWf}(库仑势能中的精确散射态), 那么理论上用不同的规范结果是一样的。先看球面波投影(类比\autoref{eq_HDipM_1}):
\begin{equation}\label{eq_HyIon2_3}\ali{
&\quad \mel{C_{l',m'}(k)}{r\cos\theta}{\psi_{n,l,m}}\\
&= \delta_{m,m'}(\delta_{l+1,l'}\mathcal C_{l,m} + \delta_{l,l'+1}\mathcal{C}_{l',m'})
\sqrt{\frac{2}{\pi}}\int_0^\infty F_{l'}(\eta, kr)(\uvec r) R_{n,l}(r)r^2\dd{r}~.
}\end{equation}
库仑平面波投影(\autoref{eq_CulmWf_9}):
\begin{equation}\label{eq_HyIon2_4}
\mel*{\psi_{\bvec k}^{(-)}}{z}{\psi_{n,l,m}} = \sum_{l'} \frac{\I^{-l'}}{k} \E^{\I\sigma_{l'}} Y_{l',m}(\uvec k) \int \mel{C_{l',m}(k)}{z}{n,l,m}~.
\end{equation}
其中 $\eta = -Z/k$。 对 $l=0$ 有

笔者没有见过该积分的解析解, 长度规范的 Matlab 数值积分代码如下:
% 已验证:氢原子 |1,0,0> |2,0,0> polarizability
\begin{lstlisting}[language=matlab, caption=hydrogen\_trans\_dipole\_sph.m]
% <C_{l1,m1}(k1)|z|n2,l2,m2>
% eq_HyIon2_3
function ret = hydrogen_trans_dipole_sph(Z, k1, l1, m1, ...
                                          n2, l2, m2, r_max)
if m1 ~= m2 || abs(l1 - l2) ~= 1
    ret = 0; return;
end
eta = -Z/k1;
f = @(r) coulombF_sym(l1, eta, k1*r) .* hydrogen_Rnl(Z, n2, l2, r) .* r.^2;
I_r = sqrt(2/pi)*integral(f, 0, r_max);
l_ = min(l1, l2);
I_ang = sqrt(((l_+1)^2-m2^2)/((2*l_+1)*(2*l_+3)));
ret = I_r * I_ang;
end
\end{lstlisting}

\begin{lstlisting}[language=matlab, caption=hydrogen\_trans\_dipole\_plane.m]
% <\bvec k|z|n,l,m>
% eq_HyIon2_4
% 未验证
function ret = hydrogen_trans_dipole_plane(...
    Z, k1, th1, ph1, n2, l2, m2, r_max)
if n2 <= 0 || l2 < 0 || l2 >= n2 || abs(m2) > l2
    error('illegal n2,l2,m2');
end
ret = 0;
for l1 = [l2-1, l2+1]
    m1 = m2;
    if l1 < 0 || abs(m1) > l1
        continue;
    end
    mel_sph = hydrogen_trans_dipole_sph(...
        Z, k1, l1, m1, n2, l2, m2, r_max);
    ret = ret + (1i^(-l1))/k1 * exp(1i*coulomb_sigma(l1, -Z/k1))...
        * SphHarm(l1,m1,th1,ph1) * mel_sph;
end
end
\end{lstlisting}

\subsubsection{速度规范}
\addTODO{速度规范数值积分,验证和长度规范一样}

\subsubsection{画图代码}
\begin{lstlisting}[language=matlab,caption=plot\_hydrogen\_trans\_dipole.m]
% plot hydrogen transition dipole
% compare different methods

% == params ==
Z = 1;
n = 1; l = 0; m = 0;
kmin = 0.01; kmax = 3; Nk = 51;
r_max = 100;
% ============

kz = linspace(kmin, kmax, Nk);
dipole_z0_appr = zeros(1, Nk);
dipole_z0_ana = dipole_z0;
dipole_z = dipole_z0;

% plane wave approx
for i = 1:Nk
    [dipole_z0_appr(i), dipole_z0_ana(i)]...
        = hydrogen_trans_dipole_plane_approx_z(kz(i), Z, n);
end

% Coulomb wave
parfor i = 1:Nk
    disp(i);
    dipole_z(i) = hydrogen_trans_dipole_plane(...
        Z, kz(i), 0, 0, n, l, m, r_max);
end

figure; plot(kz, abs(dipole_z0_appr)); hold on;
plot(kz, abs(dipole_z0_ana), '--');
plot(kz, abs(dipole_z), '.-k');

legend({'plane wave approx', ...
    'plane wave approx (analytical)', 'exact (Coulomb plane wave)'});
grid on; axis([0, 3, 0, 2.5]);

xlabel 'k [au]'; ylabel 'abs dipole';
\end{lstlisting}
