% 网球拍定理(科普)
% keys 网球拍定理|旋转|转动惯量|Tennis racket theorem|贾尼别科夫效应|Dzhanibekov Effect
% license Usr
% type Art

\pentry{角动量(科普)\upref{AngMo}}



\subsection{现象描述}


\textbf{中间轴定理},又称\textbf{网球拍定理}或\textbf{贾尼别科夫效应(Dzhanibekov Effect)},收录于于法国数学家、物理学家潘索(Louis Poinsot)于1834年出版的\textsl{Théorie Nouvelle de la Rotation des Corps}(《转动物体的新理论》),并由苏联航天员贾尼别科夫(Vladimir Dzhanibekov)在1985年观测到实验证据。


称之为“网球拍”定理,是因为抛掷网球拍的时候能观察到相应的现象,但我们在家里用一本书或者一部手机就能完成这个实验。以手机为例,将手机正常握持,然后抛向空中(请在安全环境下做这个实验,如在床上,以防手机摔坏),使得手机绕如\autoref{fig_Dzhani_2} 所示的轴在空中旋转一周后落回手中。绝大多数情况下,落回手中的手机不再是正面朝上,而是背面朝上。再抛掷一次,手机旋转一周后落回手中,又会变成正面朝上。


%\begin{figure}[ht]
%\centering
%\includegraphics[width=7cm]{./figures/edb6af44a3d38d05.pdf}
%\caption{网球拍定理示意图。图中手机是正面朝上。将手机抛向空中,使得手机绕图中虚线轴旋转一周后落回手中,则通常手机变成背面朝上。} \label{fig_Dzhani_1}
%\end{figure}



\begin{figure}[ht]
\centering
\includegraphics[width=12cm]{./figures/28ef0f94d7aeadab.pdf}
\caption{网球拍定理示意图。图中手机是正面朝上。将手机抛向空中,使得手机绕图中\textbf{左边}的虚线轴旋转一周后落回手中,则通常手机变成背面朝上。} \label{fig_Dzhani_2}
\end{figure}

仔细观察抛掷过程会发现,扔手机的时候手给手机施加的力不对称,使得手机在空中也会绕如






















