% 范德瓦尔斯气体
% 范式方程|范德瓦尔斯气体

\begin{issues}
\issueDraft
\end{issues}

范德瓦尔斯方程是理想气体向真实气体的推广,架起了微观图像与宏观测量之间的桥梁.

范德瓦尔斯对理想气体作了两点修正:1、真实气体占据一定体积;2、真实气体间有分子间作用势(Lennard-Jones 势是一个很好的近似).范德瓦尔斯方程展现出惊人的威力——从它的图像上可以看出气液相变线,可以找到临界点…… 1910年诺贝尔物理学奖授予范德瓦尔斯,以表彰他为气体和液体状态方程所作的工作.

\subsection{范式方程}
范德瓦尔斯方程可以写为
\begin{equation}
\left(p+\frac{a}{V_m^2}\right)(V_m-b)=RT
\end{equation}

$b$ 是因为真实气体分子总占据一定体积而做的修正.$a$ 是考虑分子间作用力(主要是吸引力)而做的修正.这里只给出\textbf{不严谨的分析计算},但给出的结果是与统计物理计算结果是一致的.

设气体分子的有效直径为 $d$,分子原本能达到的空间体积为 $V_0=\frac{1}{6}\pi d^3$,当考虑它与另一分子的碰撞时,它所能达到的空间体积减少了 $\frac{4}{3}\pi d^3$.$1\rm mol$ 气体含有 $N_A$ 个气体分子,从一个粒子的角度看,它面对 $N_A-1$ 个排斥球,而每个排斥球只有一面可能对它产生排斥,体积只能算一半:
\begin{equation}
b=\frac{1}{2}(N_A-1)(\frac{4}{3}\pi d^3)=4N_A V_0
\end{equation}

由于分子间作用力(主要是吸引力),碰撞容受到朝向容器内的吸引力而动量减小:所以要引入内压强 $\Delta p$.内压力正比于单位时间内碰撞器壁的粒子数,又正比于粒子数密度(影响吸引力的大小),所以 $\Delta p$ 正比于 $\frac{1}{V_m^2}$,所以设这个修正量为 $a/V_m^2$,$a$ 与相互作用势有关.设分子间作用力在 $r>d$ 时为林纳德琼斯势,当 $r\le d$ 时为钢球势(势能趋向于无穷大).当 $r>d$ 时,有 $\phi(r)=-\epsilon_0(d/r)^6$,这样经过简单的积分可以证明

\begin{equation}
a=4V_0\epsilon_0 N_A^2
\end{equation}

\addTODO{用集团展开推导范式方程的词条}
\subsection{范德瓦尔斯等温线}
范德瓦尔斯方程描述的系统的等温线如下图:
\begin{figure}[ht]
\centering
\includegraphics[width=8cm]{./figures/Vand_1.png}
\caption{范德瓦尔斯气体等温线} \label{Vand_fig1}
\end{figure}

范德瓦尔斯方程的等价形式是
\begin{equation}
V_m^3-V_m^2(b+\frac{RT}{p})+V_m\frac{a}{p}-\frac{ab}{p}=0
\end{equation}

\begin{figure}[ht]
\centering
\includegraphics[width=8cm]{./figures/Vand_2.png}
\caption{范德瓦尔斯的等温过程 MADBK}} \label{Vand_fig2}
\end{figure}


这意味着在 $p-V$ 图上,一个压强 $p$ 可以对应 $3$ 个 $V$.以\autoref{Vand_fig2} 中的黑色等温曲线为例,当系统等温压缩从 $K$ 向 $B$ 过渡时,水蒸气的压强逐渐增大,但仍是气体. 
从范德瓦尔斯等温线可以观察出若干特点:有气液相变,有\textbf{临界点},并且可以通过临界点的测量得到 $a,b$.图中的红色曲线正是临界温度下的等温线.所谓临界温度,指的是在这一温度下,


\addTODO{临界点压强与温度,相变曲线的计算,麦克斯韦等面积法则}
\subsection{范德瓦尔斯气体的热力学量}
\addTODO{熵,等温热容,等压热容,……}