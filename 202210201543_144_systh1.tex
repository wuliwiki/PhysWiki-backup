% 质心与质心系的物理量

%一篇总结性质的文章?
\subsection{质心的物理量}
在质心、质心系\upref{CM}中,我们已经定义了某参考系中质心的位置
$$\bvec r_c =\frac{1}{M}\sum_i m_i \bvec r_i$$
基于此,我们可以继续定义该参考系中质心的机械力学物理量.

质量
$$M=\sum m_i$$

速度
$$v_c = \dv{\bvec r_c}{t}$$

动量
$$\bvec P_c=M v_c$$

角动量
$$\bvec L_c = \bvec r_c \times \bvec P_c $$

动能
$$\bvec E_{k,c} = \frac{1}{2} M v_c^2$$

\subsection{质心与质心系的物理量}
那么,质心的物理量又如何与质点系的物理量联系起来呢?对于动量、角动量、动能,我们有某参考系中系统的物理量=该参考系中质心的物理量+质心系中各质点关于质心的物理量之和

动量
$$\bvec P = \bvec P_c$$
动量是一个特例,因为质心参考系中系统的动量之和为零,因此系统的总动量直接等于某参考系中质点的动量.

角动量
$$\bvec L = \bvec L_c + \bvec L'_c$$
某参考系中系统的角动量=该参考系中质心的角动量+质心系中各质点关于质心的角动量之和

动能
$$E_k = E_{k,c} + E_k'$$