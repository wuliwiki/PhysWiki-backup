% 小时的个人目录
% keys 线性代数|数学物理|数学分析|拓扑|量子力学
% license Xiao
% type Tutor

\subsection{第一篇论文}
氦原子数值解 TDSE 笔记\upref{HeTDSE}, 氦原子波函数数值分析\upref{HeAnal}, 光电离时间延迟:从一维波包到氦原子\upref{HeAna2}。

\subsection{线性代数}
投影算符\upref{projOp}, 矩阵的迹\upref{trace}

\subsection{数学物理}
连带勒让德函数\upref{AsLgdr}, 球谐函数\upref{SphHar}, 广义球谐函数\upref{GenYlm}, 平面波的球谐展开\upref{Pl2Ylm}, 库仑势能的球谐展开\upref{PChYlm}, 库仑函数\upref{CulmF}, 库仑波函数\upref{CulmWf}, Wigner 3j 符号\upref{ThreeJ}, Wigner 6j 符号\upref{SixJ}, Wigner 9j 符号\upref{NineJ}, 张量积空间\upref{DirPro}

\subsection{数学分析}
黎曼积分与勒贝格积分\upref{Rieman}, 数学分析笔记\upref{AnalNt}, 泛函分析笔记1\upref{FnalNt}, 泛函分析笔记2\upref{FnalN2}, 泛函分析笔记3\upref{FnalN3}, 泛函分析笔记4\upref{FnalN4}, 泛函分析笔记5\upref{FnalN5}

\subsection{拓扑学}
拓扑空间\upref{Topol}, 流形\upref{Manif}

\subsection{力学}
拉普拉斯—龙格—楞次矢量\upref{LRLvec}, 哈密顿正则方程\upref{HamCan}, 分析力学笔记\upref{ClsMec}

\subsection{电磁}
电多极展开\upref{EMulPo}, 拉格朗日电磁势\upref{EMLagP}

\subsection{量子力学}
全同粒子\upref{IdPar}, 含时微扰理论\upref{TDPT}, 量子散射的延迟\upref{tDelay}, 多通道散射\upref{MulSct}, 电磁场中的单粒子薛定谔方程\upref{QMEM}, 长度规范和速度规范\upref{LVgaug}, 加速度规范\upref{AccGau}, Volkov 波函数\upref{Volkov}, Keldysh 参数\upref{keldis}, 密度矩阵\upref{denMat}

\subsection{计算物理}
氢原子薛定谔方程数值解\upref{HyTDSE}, 氢原子球坐标数值解 TDSE\upref{HTDSE}, 氢原子电离截面\upref{HionCr},  单电子原子模型\upref{SAE}

\subsection{计算机}
\subsubsection{Linux}
\entry{SSH 笔记}{SSH},\entry{Linux 的二进制兼容问题}{LinBin},\entry{Linux 命令笔记}{LinNt},\entry{Vim 笔记}{Vim},\entry{Linux 基础笔记}{Linux},\entry{Ubuntu 笔记}{Ubuntu},\entry{WSL 笔记}{WSLnt},\entry{FTP/SFTP 笔记}{SFTP},\entry{Bash 编程笔记}{Bash},\entry{chroot 笔记}{chroot},\entry{service、systemd、systemctl 笔记1}{systmd},% \entry{AppImage 笔记}{AppImg}
\entry{Nginx 笔记}{nginxN}
\entry{搭建 Linux 局域网(笔记)}{LinLAN}
\entry{Linux 包管理笔记(apt, dpkg, snap)}{snap}
\entry{Linux 分区和文件系统操作笔记(Gparted, fdisk, resize2fs, grub, Clonezilla)}{fdisk}
% \entry{Linux 硬盘 RAID 阵列笔记}{RAIDnt}
\entry{ZFS (Zettabyte File System)笔记}{ZFS}
\entry{Linux 键盘设置(XKB)笔记}{XKB}
\entry{Mingw-w64 和 MSYS2 笔记}{Mingw}
\entry{pacman 笔记}{pacman}
% \entry{X11 笔记}{x11}
% \entry{zlib 笔记}{zlib}
% \entry{CentOS 笔记}{CentOS}
\entry{Logical Volume Management(LVM)笔记}{LVMnt}
\entry{iSH 笔记}{iSH}
\entry{Linux 的 System Call 笔记}{SysCal}
\entry{Linux 进程的信号(笔记)}{LinSig}
% \entry{在 Linux 中创建内存盘(RAM disk)}{RAMdsk}
% \entry{Surface Pro 7+ 上 Ubuntu 22.04(笔记)}{SP7ubu}
\entry{FFmpeg 笔记}{ffmpeg}