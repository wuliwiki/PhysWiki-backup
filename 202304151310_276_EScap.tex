% 晶格热容的爱因斯坦理论
% keys 晶格振动|热容|玻尔兹曼分布

\pentry{玻尔兹曼分布(统计力学)\upref{MBsta},热容\upref{ThCapa},一维单原子链晶格\upref{onatom}}

经典理论中,常将固体视作 $N$ 个原子组成的体系,每个原子在平衡位置作经典的简谐微振动,共 $3N$ 个自由度,可看成 $3N$ 个振子,每一个自由度上有振动过程中的动能和势能,即
\begin{equation}
\epsilon=\frac{1}{2m}p^2+\frac{m\omega^2}{2}q^2~.
\end{equation}
经典理论中用能量均分定理讨论了固体热容,则固体的内能为
\begin{equation}
U=2\cdot 3N\cdot \frac{kT}{2}=3NkT
\end{equation}
所得结果在高温和室温范围内与实验结果符合,但是在低温附近却与实验结果不符。爱因斯坦首先用量子理论分析了固体热容问题,成功解释了固体热容随温度下降而下降的实验事实。

\subsection{固体热容的爱因斯坦理论}
爱因斯坦假设每个振子的频率都相同,振子都在它们各自的平衡位置附近作振动,互相之间可以分辨\footnote{这样一来,状态数不需要除以 $N!$。},用 $\omega$ 表示圆频率,振子的能级为\footnote{至于谐振子的能级为什么是分立的,可以参考量子力学的词条\upref{QSHOop}。在统计力学中常常要用到这些量子力学的基本假设。}
\begin{equation}
\epsilon_n=\hbar \omega\left(n+\frac{1}{2}\right)
\end{equation}

由于振子之间是可以分辨的(不需考虑全同粒子假设),所以系统遵从玻尔兹曼分布(\autoref{eq_MBsta_5}~\upref{MBsta})。配分函数为
\begin{equation}
Z_1=\sum_{n=0}^\infty e^{-\beta \hbar \omega(n+1/2)}=\frac{e^{-\beta\hbar \omega/2}}{1-e^{-\beta\hbar\omega}}
\end{equation}
注意一共有 $3N$ 个振子,所以固体的内能为
\begin{equation}\label{eq_EScap_1}
\begin{aligned}
E&=3N\sum_{n=0}^\infty \hbar\omega(n+1/2)e^{-\alpha-\beta\hbar\omega(n+1/2)}\\
&=3N\frac{\partial }{\partial \beta}\ln Z_1\\
&=3N\frac{\hbar \omega}{2}+\frac{3N\hbar \omega}{e^{\beta\hbar\omega}-1}
\end{aligned}
\end{equation}

式\autoref{eq_EScap_1} 的第一项 $3N \hbar\omega/2$ 为零点能量,第二项为温度为 $T$ 时振子的热激发能量。由此可根据\autoref{eq_ThCapa_1}~\upref{ThCapa}求得定容热容:
\begin{equation}
C_V=\left(\frac{\partial U}{\partial T}\right)_V=3Nk\left(\frac{\hbar \omega}{kT}\right)^2\frac{e^{\hbar \omega/kT}}{(e^{\hbar \omega/kT}-1)^2}
\end{equation}

引入爱因斯坦特征温度 $\theta_E=\hbar\omega/k$,化简上式得
\begin{equation}\label{eq_EScap_2}
C_V=3Nk\left(\frac{\theta}{T}\right)^2\frac{e^{\frac{\theta}{T}}}{(e^{\frac{\theta}{T}}-1)^2}
\end{equation}

这就是爱因斯坦的固体热容公式。可以验证在高温区满足 $C_V=3Nk$,在低温区内热容随温度减小。

\subsection{理论的局限性}
爱因斯坦模型采用了与经典理论大不相同的量子假设,并对晶格振动采用了很简单的假设,得到了与实验非常符合的固体热容公式。这是早期量子论最重要的成果之一,极大地推动了量子力学的发展,并为固体物理的研究指明了一条道路。理论能反映出 $C_V$ 随温度趋向低温时下降的基本趋势。但是在低温范围,爱因斯坦理论值下降得很陡,与实验不相符。

在爱因斯坦模型的思想的基础上,德拜提出了新的“声子气体”模型\upref{Debye},它的计算结果在低温下能更好地拟合实验得到的晶格热容量曲线。