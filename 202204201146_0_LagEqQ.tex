% 拉格朗日方程(额外广义力)

\begin{issues}
\issueDraft
\end{issues}

\pentry{欧拉—拉格朗日方程\upref{Lagrng}}

在之前的讨论中, 在列拉格朗日方程时, 所有的系统外力都可以纳入势能项 $V$ 中. 然而许多情况中还可能出现种种非保守力, 例如摩擦力等. 这时就需要引入\textbf{广义力(generalized force)}的概念. 若系统的广义坐标为 $q_i$, 则每个广义坐标对应的广义力定义为
\begin{equation}
Q_i = \sum_j \bvec F_j \vdot \pdv{\bvec r_j}{q_i} \qquad (i=1,\dots,N)
\end{equation}
其中 $\bvec F_j$ 是除约束力的系统外力, 作用点为 $\bvec r_j$.

若非约束力 $\bvec F_j$ 不包含于势能项 $V$, 那么拉格朗日方程(\autoref{Lagrng_eq1}~\upref{Lagrng})可以拓展为
\begin{equation}\label{LagEqQ_eq1}
\dv{t} \pdv{L}{\dot q_i} = \pdv{L}{q_i} + Q_i
\qquad (i=1,\dots,N)
\end{equation}
证明见 “拉格朗日方程的证明、达朗贝尔定理\upref{dAlbt}”.

事实上, 在证明中会发现 $\pdv*{L}{q_i} = \pdv*{T}{q_i} + \pdv*{V}{q_i}$ 的第二项同样也是广义力
\begin{equation}
\pdv{V}{q_i} = Q_i^{(V)} = \sum_j \bvec F_j^{(V)} \vdot \pdv{\bvec r_j}{q_i}
\end{equation}
所以\autoref{LagEqQ_eq1} 也可以完全抛开势能的概念写成
\begin{equation}
\dv{t} \pdv{L}{\dot q_i} = \pdv{T}{q_i} + Q_i^{(tot)}
\qquad (i=1,\dots,N)
\end{equation}
其中 $Q_i^{(tot)} = Q_i^{(V)} + Q_i$.

\addTODO{加入许多例题, 例如弹簧振子摩擦力}
