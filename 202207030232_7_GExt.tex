% Galois扩域
% keys 伽罗华扩域|伽罗瓦扩域|代数方程|根式解|古典数学难题|Galois理论基本定理|Fundamental Theorem of Galois Theory

\pentry{正规扩张\upref{NomEx},群作用\upref{Group3}}

完成了对正规扩张和可分扩张的讨论,我们引入极为核心的Galois扩域.在上述讨论中,我们时常涉及域自同构,也看到了域自同构和多项式的根之间对应的关系.对域自同构的结构的研究,将把我们引向著名的古典数学难题“多项式方程的根式解”.



\subsection{Galois扩张的基本性质}

\begin{definition}{Galois扩张}

若域扩张$\mathbb{K}/\mathbb{F}$是\textbf{正规}且\textbf{可分}的,那么称之为一个\textbf{伽罗华扩张(Galois extension)},或译作\textbf{伽罗瓦扩张}.

此时称$\mathbb{K}$\textbf{在}$\mathbb{F}$\textbf{上是Galois的}($\mathbb{K}$ \textbf{is Galois over} $\mathbb{F}$).

\end{definition}

由于特征为零的域都是完美域,因此对这类域,正规扩张都是Galois扩张.

\begin{example}{}
$\mathbb{Q}(2^{1/3})/\mathbb{Q}$不是Galois扩域,因为它不正规,不包含$\opn{irr}(2^{1/3}, \mathbb{Q})(x)=x^3-2$的两个复数根.
\end{example}

\begin{theorem}{}
有限域都是其素域的Galois扩域.
\end{theorem}

\textbf{证明}:

有限域是其素域的有限扩张,而有限扩张都是代数扩张(\autoref{FldExp_cor1}~\upref{FldExp}).

由于有限域都是完美域(\autoref{SprbEx_cor4}~\upref{SprbEx}),故$\mathbb{Z}_p$的代数扩张都是可分扩张.

参照\textbf{有限域}\upref{FntFld}的讨论可知,有限域都是其素域的分裂域,从而是正规扩张.

\textbf{证毕}.


据正规扩张和可分扩张的知识,我们容易得到Galois扩张的几条性质:


\begin{theorem}{}
设$\mathbb{K}/\mathbb{M}/\mathbb{F}$是域扩张链,且$\mathbb{K}/\mathbb{F}$是Galois扩张,那么$\mathbb{K}/\mathbb{M}$也是Galois扩张.
\end{theorem}

证明用可分扩张的继承性\autoref{SprbEx_lem3}~\upref{SprbEx}和正规扩张的继承性\autoref{NomEx_the6}~\upref{NomEx}得到.

\begin{theorem}{}
设$\mathbb{K}/\mathbb{M}/\mathbb{F}$是域扩张链,其中$\mathbb{K}/\mathbb{F}$是Galois扩张,$\mathbb{M}/\mathbb{F}$是\textbf{正规}扩张,那么$\mathbb{M}/\mathbb{F}$是Galois扩张.
\end{theorem}

\textbf{证明}:

由\autoref{SprbEx_def4}~\upref{SprbEx}易得,$\mathbb{K}/\mathbb{F}$是可分扩张且$\mathbb{M}\subseteq\mathbb{K}$,可推得$\mathbb{M}/\mathbb{F}$也是可分扩张.

\textbf{证明}.

\begin{theorem}{}\label{GExt_the1}
如果$\mathbb{K}/\mathbb{F}$是Galois扩域,且域合成$\mathbb{EK}$存在,那么$\mathbb{EK}/\mathbb{EF}$是Galois扩域.
\end{theorem}

\textbf{证明}:

据\autoref{NomEx_the7}~\upref{NomEx},$\mathbb{EK}/\mathbb{EF}$是正规扩域.

考虑到$\mathbb{K}$的元素全都是$\mathbb{F}$的可分元素,从而是$\mathbb{EF}$的可分元素,而$\mathbb{EK}=\mathbb{EF}(\mathbb{K})$,可知$\mathbb{EK}/\mathbb{EF}$是可分扩张.

\textbf{证毕}.



\begin{theorem}{}
设$\mathbb{K}/\mathbb{F}$和$\mathbb{E}/\mathbb{F}$都是Galois扩域,且域合成$\mathbb{EK}$存在,则$\mathbb{EK}/\mathbb{F}$是Galois扩域.
\end{theorem}

\textbf{证明}:

已知\autoref{GExt_the1} 成立.

由$\mathbb{EK}/\mathbb{EF}$是正规扩张,及$\mathbb{F}[x]\subseteq\mathbb{EF}[x]$,知$\mathbb{EK}/\mathbb{F}$是正规扩张.

由\textbf{可分元素的封闭性}\autoref{SprbE2_cor4}~\upref{SprbE2}知$\mathbb{EK}/\mathbb{F}$是可分扩张.

\textbf{证毕}.




\begin{theorem}{}
域$\mathbb{K}$的Galois扩域之交,还是它的Galois扩域
\end{theorem}

\textbf{证明}:

由正规扩张和可分扩张相交还是正规扩张和可分扩张,得证.

\textbf{证毕}.



\subsection{Galois群}

回顾\autoref{Group_ex6}~\upref{Group},集合间的全体双射配合复合运算能构成群.既然域的自同构也是双射,我们也可以研究这些自同构构成的群.不过,相比于一般的域扩张,我们重点关注性质最良好的Galois扩张的情况.

\begin{definition}{Galois群}

给定Galois域扩张$\mathbb{K}/\mathbb{F}$,称$\mathbb{K}$上全体保$\mathbb{F}$自同构构成的群为该扩张的\textbf{Galois 群},记为$\opn{Gal}(\mathbb{K}/\mathbb{F})$,或$\opn{Gal}(\mathbb{K}:\mathbb{F})$.

\end{definition}


随便举一个具体的例子:复数域$\mathbb{C}$之于实数域$\mathbb{R}$是一个Galois扩域:正规性来自$\mathbb{C}=\overline{\mathbb{R}}$的事实,可分性是由于$\mathbb{R}$是完美域.那么$\opn{Gal}(\mathbb{C}/\mathbb{R})$是哪个群?或者最基础的问题,这个群有几个元素?

单回答这个问题也许不难,不过我们可以直接得出一般的Galois群元素数量规则:

\begin{theorem}{}\label{GExt_the2}
如果$\mathbb{K}/\mathbb{F}$是Galois扩域,那么
\begin{equation}
\abs{\opn{Gal}(\mathbb{K}/\mathbb{F})} = [\mathbb{K}:\mathbb{F}]
\end{equation}
\end{theorem}

\textbf{证明}:

当$[\mathbb{K}:\mathbb{F}]$有限时,正规扩张等价于分裂域,且由于可分扩张,因此适用\autoref{FldExp_the4}~\upref{FldExp}的等号情况,得证.

当$[\mathbb{K}:\mathbb{F}]$无限时,任取$n$个根在$\mathbb{K}-\mathbb{F}$中的多项式$f_i\in\mathbb{F}[x]$,得到$\prod f_i\in\mathbb{F}[x]$的分裂域$\mathbb{F}_1$,则$[\mathbb{K}:\mathbb{F}]>[\mathbb{F}_1:\mathbb{F}]>n$.$\mathbb{F}_1\mathbb{F}$适用有限情况,其Galois群的元素数量等于其扩张次数,而由于$[\mathbb{K}:\mathbb{F}]$无限,还能再取根在$\mathbb{K}-\mathbb{F}_1$中的多项式$g_i\in\mathbb{F}[x]\subseteq \mathbb{F}_1[x]$,构成更大的分裂域,得到更多自同构,因此必有
\begin{equation}
\opn{Gal}(\mathbb{K}/\mathbb{F})>\opn{Gal}(\mathbb{F}_1/\mathbb{F})=[\mathbb{F}_1:\mathbb{F}]>n
\end{equation}
由$n$的任意性,则可知$\opn{Gal}(\mathbb{K}/\mathbb{F})=\infty$.

\textbf{证毕}.



由于$\mathbb{C}$是$x^2+1\in\mathbb{R}[x]$的分裂域,故易证扩张次数为$2$,结合\autoref{GExt_the2} 就能确定$\opn{Gal}(\mathbb{C}/\mathbb{R})$只有两个元素.显然,除了恒等映射以外,求共轭映射也是一个保$\mathbb{R}$自同构,那这就已经找全了.




\begin{definition}{不变子域}
给定域$\mathbb{F}$,设$G$是$\mathbb{F}$的全体自同构群.

取$G$的子群$H$,则集合$\{a\in\mathbb{F}\mid \sigma(a)=a, \forall \sigma\in H\}$构成一个域,称为$\mathbb{F}$的$H$ \textbf{不变子域(fixed field of }$H$\textbf{)},或译作$H$ \textbf{固定子域},记为$\opn{Inv}_\mathbb{F}(G)$或$\opn{Fix}_\mathbb{F}(G)$.
\end{definition}

如果取$H=\opn{Gal}(\mathbb{C}/\mathbb{R})$,那么$\opn{Fix}_{\mathbb{C}}(H)=\mathbb{R}$.同样地,我们也可以直接给出更一般的情况:



\begin{theorem}{(Artin)}\label{GExt_the3}
给定域$\mathbb{F}$,$G$是它的\textbf{全体自同构}群的\textbf{有限}子群.

则$\mathbb{F}/\opn{Fix}_\mathbb{F}(G)$是Galois扩张,且$\opn{Gal}(\mathbb{F}/\opn{Fix}_\mathbb{F}(G))=G$.
\end{theorem}

\textbf{证明}:

\textbf{任取}$\alpha\in\mathbb{F}$.由于$G$是有限群,故其轨道是有限的,不妨记为$G\alpha=\{\alpha_i\}_{i=1}^n$,其中$\alpha_1=\alpha$,各$\alpha_i$彼此不等.

构造多项式$f_\alpha(x)=\prod_{i=1}^n(x-\alpha_i)$.对于\textbf{任意}$\sigma\in G$,都有$\sigma G=G$,因此$\sigma(\{\alpha_i\})=\{\alpha_i\}$,即$\sigma$是$\{\alpha_i\}$的一个置换.于是,$f_\alpha$的各系数都在$\sigma$下不变,即都是$\opn{Fix}_\mathbb{F}(G)$的元素.因此,$f_\alpha$是$\alpha$在$\opn{Fix}_\mathbb{F}(G)$上的零化多项式.

注意$\alpha$的任意性.因此$f$的最小多项式的根全都在$\mathbb{F}$中,故$\mathbb{F}/\opn{Fix}_\mathbb{F}(G)$是正规扩张.由于各$\alpha_i$不相等,故$\alpha$是$\opn{Fix}_\mathbb{F}(G)$的可分元,故$\mathbb{F}/\opn{Fix}_\mathbb{F}(G)$是可分扩张.由此\textbf{得证}$\mathbb{F}/\opn{Fix}_\mathbb{F}(G)$是Galois扩张.

由不变子域的定义,显然$G\subseteq\opn{Gal}(\mathbb{F}/\opn{Fix}_\mathbb{F}(G))$.

由于任意元素$\alpha\in\mathbb{F}$的轨道中元素数量不会超过$\abs{G}$,$f_\alpha$的次数就是$\alpha$轨道中的元素数量,以及$f_\alpha$是$\alpha$的零化多项式,可知\textbf{任意}$\alpha$的最小多项式次数不会超过$\abs{G}$.因此,应用\autoref{PrmtEl_cor3}~\upref{PrmtEl},可知$[\mathbb{F}:\opn{Fix}_\mathbb{F}(G)]\leq \abs{G}$.

由\autoref{GExt_the2} ,可知$\abs{\opn{Gal}(\mathbb{F}/\opn{Fix}_\mathbb{F}(G))}=[\mathbb{F}:\opn{Fix}_\mathbb{F}(G)]\leq\abs{G}$,从而$\opn{Gal}(\mathbb{F}/\opn{Fix}_\mathbb{F}(G))\subseteq G$.

于是\textbf{得证}$G = \opn{Gal}(\mathbb{F}/\opn{Fix}_\mathbb{F}(G))$.

\textbf{证毕}.


\autoref{GExt_the3} 揭示了自同构群和其不变子域的一一对应关系,这也体现在以下性质中,注意此处不再需要\textbf{有限}性:



\begin{theorem}{}
设$\mathbb{K}/\mathbb{F}$是一个Galois扩张.则$\opn{Gal}(\mathbb{K}/\mathbb{F})$的不变子域就是$\mathbb{F}$.
\end{theorem}

\textbf{证明}:

记$\opn{Gal}(\mathbb{K}/\mathbb{F})=G$,其不变子域为$\mathbb{J}=\opn{Fix}_\mathbb{K}(G)$.

由Galois群的定义,$\mathbb{F}\subseteq\mathbb{J}$.

任取$\alpha\in\mathbb{K}-\mathbb{F}$,记$f_\alpha=\opn{Irr}(\alpha, \mathbb{F})$.则$f_\alpha$的次数大于1,且是可分多项式,故$\alpha$有关于$\mathbb{F}$的共轭元$\beta\neq \alpha$,且正规性保证了$\beta\in\mathbb{K}$中.

因此由\autoref{FldExp_the4}~\upref{FldExp}第2条可知,存在$\sigma\in\opn{Gal}(\mathbb{K}/\mathbb{F})$使得$\sigma(\alpha)=\beta$.于是$\alpha\not\in \mathbb{J}$.因此$\mathbb{J}\subseteq\mathbb{F}$.

综上,$\mathbb{J}=\mathbb{F}$.

\textbf{证毕}.



\subsection{Galois理论基本定理}

这一小节中,我们深入讨论上一小节引出的“自同构群与不变子域的对应关系”.

\begin{theorem}{}
给定Galois扩张$\mathbb{K}/\mathbb{F}$,取它的三个中间域$\mathbb{M}_1, \mathbb{M}_2, \mathbb{M}_3$,并令$\opn{Gal}(\mathbb{K}/\mathbb{M}_i)=H_i$.则下列命题成立:

1.$\mathbb{M}_1\subseteq \mathbb{M}_2 \iff H_1\supseteq H_2$;

2.$\mathbb{M}_1=\mathbb{M}_2\mathbb{M}_3 \iff H_1=H_2\cap H_3$;

3.$\mathbb{M}_1=\mathbb{M}_2\cap\mathbb{M}_3 \iff H_1$是包含$H_2\cap H_3$的最小的群.

4.$\mathbb{M}_1$和$\mathbb{M}_2$为共轭域,当且仅当$H_1$和$H_2$在$\opn{Gal}(\mathbb{K}/\mathbb{M})$中共轭.

\end{theorem}

\textbf{证明}:

1.

$\mathbb{M}_1\subseteq\mathbb{M}_2 \iff$保$\mathbb{M}_2$不变的同构也必保$\mathbb{M}_1$不变$\iff H_1\supseteq H_2$.

2.

$\mathbb{M}_1=\mathbb{M}_2\mathbb{M}_3\implies \mathbb{M}_1\supseteq\mathbb{M}_2\cup\mathbb{M}_3\implies H_1\subseteq H_2\cap H_3$.

$\mathbb{M}_1=\mathbb{M}_2\mathbb{M}_3\implies$ 保$\mathbb{M}_2$和$\mathbb{M}_3$都不变的同构,必保$\mathbb{M}_1$不变$\implies H_2\cap H_3\subseteq H_1$.

综合这两条逻辑链,得$\mathbb{M}_1=\mathbb{M}_2\mathbb{M}_3 \implies H_1=H_2\cap H_3$.

$H_1=H_2\cap H_3\iff$“自同构保$\mathbb{M}_1$不变当且仅当它保$\mathbb{M}_2$和$\mathbb{M}_3$都不变”

\textbf{证毕}.














\begin{theorem}{Galois理论基本定理(Fundamental Theorem of Galois Theory)}

设$\mathbb{K}/\mathbb{F}$是一个\textbf{有限}的Galois扩张,则以下命题成立:

1.设存在中间域$\mathbb{M}$,则$\mathbb{K}/\mathbb{M}$也是有限Galois扩张,且$\mathbb{M}=\opn{Fix}_\mathbb{K}(\opn{Gal}(\mathbb{K}/\mathbb{M}))$.

2.设$H$是$\opn{Gal}(\mathbb{K}/\mathbb{F})$的子群,则$\mathbb{J}=\opn{Fix}_\mathbb{K}(H)$是$\mathbb{K}/\mathbb{F}$的中间域,且$H=\opn{Gal}(\mathbb{K}/\mathbb{M})$.

上述性质将$\opn{Gal}(\mathbb{K}/\mathbb{F})$的\textbf{子群}和$\mathbb{K}/\mathbb{F}$的\textbf{中间域}一一对应起来了.

\end{theorem}

\textbf{证明}:



\textbf{证毕}.


















