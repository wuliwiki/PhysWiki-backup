% Euler 图
% keys Euler图
% license Usr
% type Tutor

\pentry{链、路、圈、回\nref{nod_PatCyc}}{nod_0858}
\cite{graph2}在\enref{链、路、圈、回}{PatCyc}中我们已经介绍了Euler迹和Euler回,现在我们介绍Euler图的概念。Euler图是为了纪念图论创始人Euler的,其是通过推广Euler解决Konigsberg七桥问题中抽象出的图的一类图。七桥问题大概是说:一个人可否从家出发经过所有的七座桥一次且仅一次并返回家。若把桥看成边,连接桥的两边陆地看成点,经过桥必须先经过桥两边的陆地,而要经过所有的桥,相当于要经过所有的点和所有的边,且仅经过一次然后最后返回家相当于:从某个代表家所在陆地的点出发,行走通过所有的边一次且经过所有的点最终回到起点。若按点边点排列,这相当于边不同的一个序列,即\aref{迹}{def_PatCyc_2},并且端点相同,即还是\aref{回}{def_PatCyc_2}。这就是我们定义Euler迹和Euler回的历史渊源。Euler图就是有Euler回的图。

\begin{definition}{Euler图}
设 $G$ 是个图,若 $G$ 有 $Eluer$ 回,则称 $G$ 是\textbf{Euler图}。
\end{definition}



\begin{theorem}{}
$G$ 是 Euler 图,等价于 $G$ 是连通的平衡图(\autoref{def_DGraph_2})。
\end{theorem}





\begin{corollary}{}
$G$ 是Euler 图,等价于 $G$ 不含\enref{奇度点}{DGraph}。
\end{corollary}


\begin{corollary}{}
非平凡图有Euler迹,等价于$G$ 连通且最多有两个奇度点。
\end{corollary}







