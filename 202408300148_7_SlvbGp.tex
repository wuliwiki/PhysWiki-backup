% 可解群
% keys 可解群|solvable group|导出列|换位子|换位子群|derived series|commutator|commutator group|正规序列|次正规序列|normal series|subnormal series|合成序列|composite series
% license Usr
% type Tutor


\pentry{换位子群\nref{nod_CmtGrp}}{nod_3aa9}


可解群在Galois理论中起到关键作用,用于判断代数方程的根式可解性,由此而得名。由于我们现在尚未深入Galois理论,就不讨论何谓“可解”,而仅仅从群结构的角度研究这种群的性质。


\begin{definition}{可解群}\label{def_SlvbGp_1}
给定群$G$,定义$G^{(0)}=G$且对于任意正整数$k$,$G^{(k)}=[G^{(k-1)}, G^{(k-1)}]$。

称
\begin{equation}\label{eq_SlvbGp_1}
G^{(0)}\rhd G^{(1)}\rhd G^{(2)}\rhd \cdots~
\end{equation}
为$G$的\textbf{导出列(derived series)}。

若导出列在有限步内终结于$\{e\}$,或者等价地说,存在非负整数$n$使得$G^{(n)}=\{e\}$,则称$G$\textbf{可解(solvable)}。

\end{definition}


\autoref{def_SlvbGp_1} 的\autoref{eq_SlvbGp_1} 直接使用了正规子群符号$\rhd$,这是由\autoref{the_CmtGrp_2} 保证的。



\autoref{def_SlvbGp_1} 的基础是换位子群的概念,但我们也可以仅用正规子群的概念来定义可解群:



\begin{definition}{(次)正规序列}

给定群$G$,称序列
\begin{equation}
G=G_1\rhd G_2\rhd\cdots G_n=\{e\}~
\end{equation}
为$G$的一个\textbf{次正规序列(subnormal series)}。若该序列还满足$\forall k\in \mathbb{Z}\cap\{1, n\}, G_k\lhd G$,则称之为一个\textbf{正规序列(normal series)}。

\end{definition}



\begin{theorem}{可解群的另一定义}

给定群$G$,则$G$是可解群,当且仅当存在$G$的正规序列$G=G_1\rhd G_2\rhd\cdots G_n=\{e\}$,且对于任意正整数$k$,$G_k/G_{k+1}$都是阿贝尔群。

\end{theorem}


\textbf{证明}:

\textbf{必要性}:

设$G$可解,即存在非负整数$n$使得$G^{(n)}=\{e\}$。令$G_k=G^{(k-1)}$,则对于任意正整数$k$,$G_{k+1}\subseteq [G_k, G_k]$,于是由\autoref{the_CmtGrp_1} 可知,$G_k/G_{k+1}$都是\textbf{阿贝尔群}。又由\autoref{the_CmtGrp_2},可知各$G_k\lhd G$,因此
\begin{equation}
G=G_1\rhd G_2\rhd\cdots\rhd G_{n+1}=\{e\}~
\end{equation}
是一个\textbf{正规序列}。

\textbf{充分性}:

设存在$G$的正规序列$G=G_1\rhd G_2\rhd\cdots G_n=\{e\}$,且对于任意正整数$k$,$G_k/G_{k+1}$都是阿贝尔群。

由\autoref{the_CmtGrp_1} 可知,$G_{k+1}\supseteq [G_k, G_k]$对任意正整数$k$成立。考虑到$G_2\supseteq[G_1, G_1]=G^{(1)}$,从而$G_3\supseteq[G_2, G_2]\supseteq [G^{(1)}, G^{(1)}]=G^{(2)}$;以此类推,$G_{k}\supseteq G^{(k-1)}$对任意正整数$k$成立。

因此,$G_n=\{e\}\implies G^{(n-1)}=\{e\}$。

\textbf{证毕}。




\subsection{可解群的结构}



































