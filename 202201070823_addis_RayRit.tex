% Rayleigh-Ritz 方法

\begin{issues}
\issueDraft
\end{issues}

\pentry{厄米矩阵的本征问题\upref{HerEig}}

Rayleigh-Ritz 方法是一种近似计算本征值问题的方法. 若要求 $N\times N$ 厄米矩阵\upref{HerMat} $\mat A$ 的本征值和本征矢, 先把算符投影到由 $n < N$ 个正交归一基底 $\mat v_i$ 张成的子空间中. 若把每个列矢量 $\mat v_i$ 矩阵 $\mat V$, 那么子空间的投影矩阵就是
\begin{equation}
\mat B = \mat V\Her \mat A \mat V
\end{equation}
也就是说, 每个矩阵元为
\begin{equation}
B_{i,j} = \bvec v_i\Her \mat A \bvec v_j
\end{equation}

令 $\mat A, \mat B$ 的本征方程分别为
\begin{equation}
\mat A \bvec x_i = \lambda_i \bvec x_i
\end{equation}
\begin{equation}
\mat B \bvec y_i = \mu_i \bvec y_i
\end{equation}
令 $\tilde{\bvec x}_i = V \bvec y_i$. 于是 $(\mu_i, \tilde{\bvec x}_i)$ 就是近似本征值和本征矢, 叫做 Ritz pair.

只有当 $\qty{\bvec v_i}$ 恰好张成 $\mat A$ 的某几个本征矢的子空间时, 才会得到精确解.

\textbf{瑞利商(Rayleigh quation)}:
\begin{equation}
\rho(\bvec v) = \frac{\bvec v\Her \mat A \bvec v}{\bvec v\Her \bvec v}
\end{equation}
这相当于量子力学中的平均值, 而根据变分原理, 任何 $\rho(\tilde {\bvec x}_i)$ 都大于最小的本征值(基态能量).
