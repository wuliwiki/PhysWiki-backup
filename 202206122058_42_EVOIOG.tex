% 欧几里得矢量空间的正交化、同构及正交群
% keys 欧氏矢量空间|正交化|同构|正交矩阵|正交群

\pentry{欧几里得矢量空间\upref{EuVS}}
对欧几里得矢量空间 $V$(\autoref{EuVS_def1}~\upref{EuVS}),其上任一基底都可正交标准化,而空间 $V$ 上的对称双线性型 $(*|*)$ 实际上给出了 $V$ 中的度量性质,即长度(\autoref{EuVS_def2}~\upref{EuVS})和夹角(\autoref{EuVS_def3}~\upref{EuVS}),这意味着就度量性质来说,欧几里得矢量空间 $V$ 和 $\mathbb{R}^n$ 没有差别.由于标准正交基底的重要作用,往往要研究不同标准正交基底之间的转换关系,即两基底对应的过渡矩阵(或转换矩阵),与两标准正交基底对应的过渡矩阵称为正交矩阵,它们将构成一个群,称之正交群.
\subsection{正交化过程}
\begin{theorem}{标准正交化过程}
设 $\bvec e_1,\cdots,\bvec e_m$ 是 $n$ 维欧几里得矢量空间 $V$ 中的一组 $m$ 个线性无关的矢量.那么,存在一个标准正交矢量组 $\bvec e'_1,\cdots,\bvec e'_m$ ,使得线性包络(或称张成空间\autoref{VecSpn_def1}~\upref{VecSpn})
\begin{equation}
L_i=\langle\bvec e_1,\cdots,\bvec e_i\rangle
\end{equation}
和
\begin{equation}
L'_i=\langle\bvec e'_1,\cdots,\bvec e'_i\rangle
\end{equation}
当 $i=1,\cdots,m$ 时都重合,$m\leq n$.
\end{theorem}
\textbf{证明:}令 $\bvec e'_1=\lambda\bvec e_1,\;\lambda=\norm{\bvec e_1}^{-1}$,则 $L_1=\langle\bvec e_1\rangle=\langle \bvec e'_1\rangle=L'_1$ 显然成立,即 $m=1$ 的情形.

设当 $1\leq k<m$ 时,已构造出所需的矢量组 $\bvec e'_1,\cdots,\bvec e'_k$,使得 $L_1=L'_i,\;i=1,\cdots,k$.我们来找出 $\bvec e'_{k+1}$.

首先,矢量 $\bvec e_{k+1}$ 不包含在 $L_k=L'_k$ 中(否则,$\bvec e_{k+1}$ 可用 $\bvec e_1,\cdots,\bvec e_k$ 线性表示).令
\begin{equation}
\bvec v=\bvec e_{k+1}-\sum_{i=1}^{k}\lambda_i\bvec e'_i
\end{equation}
而 $\lambda_1,\cdots,\lambda_k$ 是任意纯量(显然 $\bvec v\neq\bvec 0$).于是
\begin{equation}
L_{k+1}=\langle\bvec e_1,\cdots,\bvec e_k,\bvec e_{k+1}\rangle=\langle\bvec e_1,\cdots,\bvec e_k,\bvec v\rangle
\end{equation}
如果 $\bvec v\perp L'_k$,那么便找到了 $\bvec e'_{k+1}=\frac{\bvec v}{\norm{\bvec v}}$ .而做到这一点的充要条件是,对任意 $j=1,\cdots,k$,成立
\begin{equation}
\begin{aligned}
&0=\qty(\bvec v|\bvec e'_j)=\qty(\bvec e_{k+1}|\bvec e'_j)-\qty(\sum_{i=1}^k\lambda_i\bvec e'_i|\bvec e'_j)\\
&=\qty(\bvec e_{k+1}|\bvec e'_j)-\sum_{i=1}^k\lambda_i(\bvec e'_i|\bvec e'_j)=\qty(\bvec e_{k+1}|\bvec e'_j)-\lambda_j
\end{aligned}
\end{equation}
于是,只需取 $\lambda_j=\qty(\bvec e_{k+1}|\bvec e'_j)$ 即可.