% 斯坦尼斯瓦夫·乌拉姆(综述)
% license CCBYSA3
% type Wiki

本文根据 CC-BY-SA 协议转载翻译自维基百科\href{https://en.wikipedia.org/wiki/Stanis\%C5\%82aw_Ulam}{相关文章}。
\begin{figure}[ht]
\centering
\includegraphics[width=6cm]{./figures/15377c1b183c5f57.png}
\caption{乌拉姆在洛斯阿拉莫斯} \label{fig_Ulam_1}
\end{figure}
斯坦尼斯瓦夫·马尔钦·乌拉姆(波兰语:[sta'ɲiswaf 'mart͡ɕin 'ulam];1909年4月13日 – 1984年5月13日)是波兰数学家、核物理学家和计算机科学家。他参与了曼哈顿计划,提出了特勒–乌拉姆热核武器设计,发现了细胞自动机的概念,发明了蒙特卡罗计算方法,并提出了核脉冲推进技术。在纯数学和应用数学领域,他证明了多个定理并提出了若干猜想。

乌拉姆出生于奥匈帝国利沃夫的一个富裕的波兰犹太家庭;他在利沃夫工艺大学学习数学,并于1933年在卡齐米日·库拉特科夫斯基(Kazimierz Kuratowski)和弗沃季米日·斯托热克(Włodzimierz Stożek)的指导下获得博士学位。[1] 1935年,乌拉姆在华沙遇到了约翰·冯·诺依曼,后者邀请他到新泽西州普林斯顿的高等研究院待几个月。从1936年到1939年,他每年夏天都回波兰,学年则在马萨诸塞州剑桥的哈佛大学度过,在那里他致力于建立关于遍历理论的重要成果。1939年8月20日,他和17岁的弟弟亚当·乌拉姆一起最后一次乘船前往美国。1940年,他成为威斯康星大学麦迪逊分校的助理教授,并于1941年成为美国公民。

1943年10月,乌拉姆收到了汉斯·贝特的邀请,加入位于新墨西哥州洛斯阿拉莫斯的曼哈顿计划秘密实验室。在那里,他负责进行流体动力学计算,以预测爆炸透镜在内爆型武器中的行为。他被分配到爱德华·泰勒的团队,在泰勒和恩里科·费米的指导下,参与了泰勒的“超级”炸弹项目。战后,他离开洛斯阿拉莫斯,成为南加州大学的副教授,但在1946年回到洛斯阿拉莫斯,继续从事热核武器的研究。在一群女性“计算员”的帮助下,他发现泰勒的“超级”设计不可行。1951年1月,乌拉姆和泰勒共同提出了泰勒–乌拉姆设计,这一设计成为所有热核武器的基础。

乌拉姆考虑了火箭核推进的问题,这一问题由“罗孚计划”(Project Rover)进行研究。他提出了一种替代“罗孚计划”核热火箭的方法,即利用小规模的核爆炸进行推进,这一方案后来成为了“猎鹰计划”(Project Orion)。与费米、约翰·帕斯塔(John Pasta)和玛丽·青果(Mary Tsingou)一起,乌拉姆研究了著名的费米–帕斯塔–乌拉姆–青果问题(Fermi–Pasta–Ulam–Tsingou problem),这一问题成为了非线性科学领域的启发来源。他可能最为人知的是意识到,电子计算机使得将统计方法应用于没有已知解的函数变得可行。随着计算机的发展,蒙特卡罗方法(Monte Carlo method)已成为解决许多问题的常见且标准的方法。
\subsection{波兰}  
乌拉姆于1909年4月13日出生在加利西亚的莱姆堡(Lemberg)。当时,加利西亚属于奥匈帝国的加利西亚和洛多梅里亚王国,波兰人称其为奥地利分治区。1918年,莱姆堡成为新恢复的波兰第二共和国的一部分,并重新取回了其波兰名字——利沃夫(Lwów)。

乌拉姆家族是一个富裕的波兰犹太家庭,从事银行业、工业和其他专业工作。乌拉姆的直系家庭“生活富足,但并不算非常富有”。他的父亲,约瑟夫·乌拉姆(Józef Ulam),出生在利沃夫,是一名律师;母亲安娜(Anna, née Auerbach)出生于斯特里(Stryj)。他的叔叔米哈乌·乌拉姆(Michał Ulam)是一名建筑师、建筑承包商和木材工业家。从1916年到1918年,约瑟夫的家庭曾暂时居住在维也纳。返回后,利沃夫成为波兰–乌克兰战争的中心,期间该市遭遇了乌克兰的围困。
\begin{figure}[ht]
\centering
\includegraphics[width=6cm]{./figures/d11840c23803e51e.png}
\caption{位于乌克兰利沃夫的苏格兰咖啡馆大楼现在是Szkocka餐厅和酒吧的所在地(该餐厅以原苏格兰咖啡馆命名)。} \label{fig_Ulam_2}
\end{figure}
1919年,乌拉姆进入了利沃夫第七中学,并于1927年毕业。[10] 随后,他在利沃夫理工学院学习数学。在卡兹米日·库拉托夫斯基的指导下,他于1932年获得文学硕士学位,并于1933年获得科学博士学位。[9][11] 在1929年,年仅20岁的乌拉姆在《数学基础》杂志上发表了他的第一篇论文《关于集合的函数》。[11] 从1931年到1935年,他前往并在维尔纽斯(今立陶宛首都)、维也纳、苏黎世、巴黎和英国剑桥学习,在那里他结识了G·H·哈迪和苏布拉马尼扬·钱德拉塞卡。[12]

乌拉姆与斯坦尼斯瓦夫·马祖尔、马克·卡茨、弗沃季米日·斯托热克、卡兹米日·库拉托夫斯基等人一起,是利沃夫数学学派的成员。该学派的创始人是胡戈·施泰因豪斯和斯特凡·巴纳赫,他们是雅努什·卡兹米日大学的教授。这些数学家常常在苏格兰咖啡馆聚会,讨论他们的问题,这些问题被收录在《苏格兰书》中,这是由巴纳赫的妻子提供的一本厚重的笔记本。乌拉姆是这本书的主要贡献者之一。在1935年至1941年间记录的193个问题中,他作为唯一作者贡献了40个问题,又与巴纳赫和马祖尔一起合作贡献了11个问题,并与其他人共同贡献了15个问题。1957年,他从施泰因豪斯那里收到了这本幸存下来的书,并将其翻译成了英语。[13] 1981年,乌拉姆的朋友R·丹尼尔·莫尔丁发布了扩展版和注释版。[14]
\subsection{移居美国}  
1935年,乌拉姆在华沙遇见的约翰·冯·诺依曼邀请他前往新泽西州普林斯顿的高等研究院待几个月。那年12月,乌拉姆启程前往美国。在普林斯顿,他参加了讲座和研讨会,听到了奥斯瓦尔德·维布伦、詹姆斯·亚历山大和阿尔伯特·爱因斯坦的演讲。在冯·诺依曼家的一次茶话会上,他遇到了G·D·伯克霍夫,伯克霍夫建议他申请哈佛大学学者协会的职位。[9] 根据伯克霍夫的建议,乌拉姆从1936年到1939年夏季在波兰度过,学年则在马萨诸塞州剑桥的哈佛大学度过,并与约翰·C·奥克斯托比合作,研究了遍历理论的相关成果。这些成果于1941年发表在《数学年刊》上。[10][15] 1938年,乌拉姆的母亲安娜·汉娜·乌拉姆(原名奥尔巴赫)因癌症去世。

