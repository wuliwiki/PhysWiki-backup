% 集合的基本运算(高中)
% keys 集合|基本运算|运算
% license Usr
% type Tutor
\pentry{集合\nref{nod_HsSet}}{nod_120e}

\begin{issues}
\issueDraft
\end{issues}
相信此刻的你,已经基本了解了集合的基础知识。就像小学接触数字那样,认识了十以内的数字,就要开始接触他们的加减乘除了,“加减乘除”被称为运算,本文介绍的就是集合的基本运算。

\subsection{三种基础运算}

\subsubsection{交集}

生活中,当我们感慨与一个人没有联系,彼此渐行渐远,互不来往,逐渐消失在对方的生活里时,常常会说,“我与他没有交集了”。而当心里有很多感情同时涌现出来的时候,会说自己“百感交集”。尽管这些词创立的时候与下面要讲的这个运算没有直接的关系,但他们在语义上表达的内容差不多,前者是指两者没有任何相同的内容共存,后者是指同时有很多感情共存。

\begin{definition}{交集}
一般地,由既属于集合 $A$ 又属于集合 $B$ 的所有元素组成的集合叫做 $A$ 与 $B$ 的\textbf{交集(intersection set)},记作 $A \cap B$(读作“A交B”),即
\begin{equation}
A\cap B = \begin{Bmatrix} x|x\in A \wedge x\in B \end{Bmatrix}~.
\end{equation}
\end{definition}

根据交集定义,可得
\begin{equation}
\begin{aligned}
&A\cap B = B\cap A~, \\ 
&A\cap B \subseteq A ~,\\
&A\cap B \subseteq B~, \\
&A\cap A = A ~,\\
&A\cap \varnothing = \varnothing~.
\end{aligned}
\end{equation}
\subsubsection{并集}
\begin{definition}{并集}
由属于集合 $A$ 或属于集合 $B$ 的所有元素组成的集合,叫作 $A$ 与 $B$ 的\textbf{并集(union set)},记作 $A\cup B$(读作“A并B”),即
\begin{equation}
A\cup B = \begin{Bmatrix}x|x\in A \vee x\in B\end{Bmatrix}~.
\end{equation}
\end{definition}

根据并集的定义,可得
\begin{equation}
\begin{aligned}
&A\cup B = B\cup A~, \\
&A\subseteq A\cup B ~,\\
&A\cup A = A ~,\\
&A\cup \varnothing = A~.
\end{aligned}
\end{equation}
\subsubsection{补集}
\begin{definition}{补集}
设 $U$ 是全集,$A$ 是 $U$ 的一个子集(即$A\subseteq U$),则由 $U$ 中所有不属于 $A$ 的元素组成的集合,叫作 $U$ 中子集 $A$ 的\textbf{补集}\footnote{也称为\textbf{余集}}(complementary set),记作$\complement_UA$,即
\begin{equation}
\complement_UA = \begin{Bmatrix}x|x\in U \wedge \notin A\end{Bmatrix}~.
\end{equation}
\end{definition}

由补集定义可得,
\begin{equation}
\begin{aligned}
&A\cup (\complement_UA) = U~, \\
&A\cap (\complement_UA) = \varnothing~.
\end{aligned}
\end{equation}
\subsection{复合运算}

\subsection{*德摩根定律}


\begin{equation}
\complement_U(\bigcup_{i\in I} A_i)=\bigcap_{i\in I} \complement_UA_i.~
\end{equation}

\begin{equation}
\complement_U(\bigcap_{i\in I} A_i)=\bigcup_{i\in I} {\complement_UA_i}.~
\end{equation}


\subsection{*集合运算与逻辑运算的关系}


\addTODO{下面的内容移动至本科内容中}

\subsection{*差集}

\subsection{幂集}