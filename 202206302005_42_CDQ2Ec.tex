% 保距群(欧氏空间)
% keys 运动|保距变换|保距群

\begin{issues}
\issueDraft
\end{issues}

\pentry{欧几里得空间\upref{EucSp}}
在欧几里得空间 $\mathbb E$ 中,保距群是保持欧几里得空间中点与点之间距离不变的映射构成的集合,保持点距离不变的映射称为\textbf{保距映射},又称\textbf{运动}.欧氏空间中的映射是个运动,当且仅当它是个线性部分为正交线性算子的仿射映射(\autoref{AfSp_def2}~\upref{AfSp}).需要说明的是,这里所说的欧氏空间 $\mathbb A$ 中的映射指的是 $\mathbb E$ 到 $\mathbb E$ 上的映射.下面将一一说明.
\subsection{偶氏空间中的运动}
先构建起基本材料
\begin{definition}{运动(保距映射)}
设 $(\mathbb E,V,\rho)$ 是欧几里点空间,变换 $f:\mathbb E\rightarrow\mathbb E$ 称为\textbf{运动}(或\textbf{保距映射}),如果它保持点与点之间距离不变,即 $\forall \dot p,\dot q\in\mathbb E$ ,都有
\begin{equation}
\rho(f(\dot p),f(\dot q))=\rho(\dot p,\dot q)
\end{equation}
\end{definition}
下面证明运动必是个线性部分为正交线性算子的仿射自同构(\autoref{AfSp_def2}~\upref{AfSp}).在此之前,先证明一个引理
\begin{lemma}{}
任意仿射变换 $f:\mathbb E\rightarrow\mathbb E$ ,设其线性部分为 $\mathcal F$,则必有
\begin{equation}
f=t_a g
\end{equation}
其中 $t_a$ 是矢量 $a=\overrightarrow{of(o)}$ 对应的平移,$g$ 是一个在给定点 $\dot o$ 处保持不动的仿射变换.

现在,若将固定点改为 $\dot o'$ ,那么需取 $a'=a+(\mathcal F-\mathcal E)\overrightarrow{oo'}$ 来代替 $a$.
\end{lemma}

\textbf{证明:} 设 $V$ 是与空间 $\mathbb E$ 伴随的矢量空间,则 $\forall v\in V$ ,有
\begin{equation}\label{CDQ2Ec_eq1}
f(\dot o+v)=f(\dot o)+\mathcal F v=\dot o+\overrightarrow{of(o)}+\mathcal F v
\end{equation}
若令
\begin{equation}\label{CDQ2Ec_eq2}
g(\dot o+v)=\dot o+\mathcal F v,\quad a=\overrightarrow{of(o)}
\end{equation}
则\autoref{CDQ2Ec_eq1} 可写为
\begin{equation}
f(\dot o+v)=g(\dot o+v)+a=t_a g(\dot o+v)
\end{equation}
由于 $v$ 的任意性,故 $f=t_ag$.由\autoref{CDQ2Ec_eq2},$g$ 显然对 $\dot o$ 不变.

当用固定点取 $\dot o'$,显然 $\overrightarrow{of(o)}$ 需用 $\overrightarrow{o'f(o')}$ 替换,而
\begin{equation}
\begin{aligned}
f(\dot o')=f(\dot o)+\mathcal F \overrightarrow{oo'}\quad\Rightarrow \quad \dot o'+\overrightarrow{o'f(o')}&=\dot o+\overrightarrow{of(o)}+\mathcal F \overrightarrow{oo'}\\
&\Downarrow\\
\overrightarrow{o'f(o')}&=\overrightarrow{of(o)}+(\mathcal F-\mathcal E)\overrightarrow{oo'}
\end{aligned}
\end{equation}
注意 $a=\overrightarrow{of(o)}$,并令 $a'=o'f(o')$ ,即得 $a'=a+(\mathcal F-\mathcal E)\overrightarrow{oo'}$.

\textbf{证毕!}


\begin{theorem}{运动必是仿射自同构}
变换 $f:\mathbb E\rightarrow\mathbb E$ 是个运动,当且仅当, $f$ 是个线性部分为 $U$ 上的正交线性算子(\autoref{LiOper_sub4}~\upref{LiOper})的仿射变换.
\end{theorem}
\textbf{证明:}先来证明定理中是较为显然的一方面,即由后推出前:设仿射变换 $f$ 线性部分的正交线性算子为 $\mathcal F$ ,则 $\forall \dot p,\dot q\in\mathbb E$,有
\begin{equation}
\begin{aligned}
f(\dot p+\overrightarrow{pq})&=f(\dot p)+\mathcal F\overrightarrow{pq}\\
&\Downarrow\\
\rho(f(\dot p),f(\dot q))&=\rho(f(\dot p),f(\dot p)+\mathcal F\overrightarrow{pq})\\
&=\norm{\mathcal F\overrightarrow{pq}}=\norm{\overrightarrow{pq}}=\rho(\dot p,\dot q)
\end{aligned}
\end{equation}=
