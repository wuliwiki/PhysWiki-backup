% 浙江大学 2001 年 考研 量子力学
% license Usr
% type Note

\textbf{声明}:“该内容来源于网络公开资料,不保证真实性,如有侵权请联系管理员”

\subsection{第一题: (15 分)}
(1) 试确定,在 3K 温度下,空腔辐射的最大能量密度所对应的光子的波长 $\lambda_m$ 是多少?

(2) 此时,光子的对应能量是多少?

(3) 光电效应中,如何测定某金属板的逸出功 $A$?
\subsection{第二题:(20分)}
设氢原子处于状态:$$\psi(r, \theta, \phi) = \frac{1}{2} R_{21}(r) Y_{10}(\theta, \phi) - \frac{\sqrt{3}}{2} R_{31}(r) Y_{11}(\theta, \phi)~$$

1. 问测量氢原子的能量,所得的可能值及相应的几率为多少?

2. 问测量氢原子的角动量平方 \( \hat{L}^2 \),所得的可能值及相应的几率为多少?

3. 问测量氢原子的角动量分量 \( \hat{L}_z \),所得的可能值及相应的几率为多少?
\subsection{第三题:(20分)}
1. 一质量为 \( m \) 的粒子于势场 \( V(x) \) 中运动,
$$
V(x) = 
\begin{cases} 
\infty, & x<0 \\
0, & 0 \leq |x| \leq a \\
\infty, & x > a 
\end{cases}~
$$
求该粒子的能级及对应的本征波函数?

2. 若一质量为 \( m \) 的粒子与势场
$$
V(x) = 
\begin{cases} 
V_0 > 0, & |x| \geq a \\
0, & |x| < a 
\end{cases}~
$$
中运动,求束缚态能级 \( E \) 所满足的方程。

3. 若一质量为 \( m \) 的粒子于三维势场 \( V(r) \) 中运动,
$$
V(r) = 
\begin{cases} 
-V_0, & 0 \leq r \leq a \\
0, & r > a
\end{cases},
V_0 > a~
$$
则若欲得二个束缚态,其势能值 \( V_0 \) 至少应为多少?

