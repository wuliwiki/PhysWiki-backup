% 相对论(综述)
% license CCBYSA3
% type Wiki

(本文根据 CC-BY-SA 协议转载自原搜狗科学百科对英文维基百科的翻译)


相对论通常包含阿尔伯特·爱因斯坦(Albert Einstein)的两个相互关联的理论:狭义相对论和广义相对论。[1]狭义相对论适用于基本粒子及其相互作用,描述了除引力以外的所有物理现象。广义相对论解释了引力定律及其与其他自然力的关系,[2]适用于宇宙学和天体物理学领域,包括天文学。[3]

该理论在20世纪改变了理论物理学和天文学,取代了主要由艾萨克·牛顿(Isaac Newton)创立的有200年历史的力学理论。[3][4][5]它引入了一些概念,包括作为时间和空间统一实体的时空、同时性的相对性、运动学和引力时间膨胀、以及长度收缩。在物理学领域,相对论改进了基本粒子及其基本相互作用的科学,同时迎来了核时代。借助相对论,宇宙学和天体物理学预测了非同寻常的天文现象,如中子星、黑洞和引力波。[3][4][5]

\begin{figure}[ht]
\centering
\includegraphics[width=6cm]{./figures/1aba0c5b7b223588.png}
\caption{广义相对论中时空曲率三维类比的二维投影} \label{fig_XDL_1}
\end{figure}

\subsection{发展和认可}

1905年,阿尔伯特·爱因斯坦在阿尔伯特·迈克耳孙、亨德里克·洛伦兹、儒勒·昂利·庞加莱等人的理论成果和实证结果的基础上,发表了狭义相对论。马克斯·普朗克、赫尔曼·闵可夫斯基等人做了后续研究。

爱因斯坦在1907年至1915年间发展了广义相对论,并在1915年后得到许多其他人的贡献。广义相对论的最终形式发表于1916年。[3]

“相对论”一词是基于普朗克于1906年使用的“相对理论”(德语:Relativtheorie)这一表达,他强调了相对论是如何运用相对论原理的。在同一篇论文的讨论部分,阿尔弗雷德·布切勒首次使用了“相对论”这个表述(德语:Relativitätstheorie)。[6][7]

截止20世纪20年代,物理界理解并接受了狭义相对论。[8]它迅速成为理论家和实验家在原子物理学,核物理学和量子力学等新领域的重要和必要的工具。

相比之下,广义相对论似乎没有那么有用,只是对牛顿重力理论的预测做了一些微小的修正。[3]它似乎没有什么被实验论证的潜力,因为它的大部分理论都是基于天文的范围。它的数学论看似很复杂,只有少数人能完全理解。约在1960年,广义相对论成为物理学和天文学的核心。应用于广义相对论的新数学技术简化了计算,使其概念更容易可视化。随着天文现象的发现,如类星体(1963年),3-开尔文微波背景辐射(1965年),脉冲星(1967年)和第一批黑洞候选者(1981年),[3]该理论解释了它们的属性,并对它们进行的测量进一步证实了这一理论。

\subsection{狭义相对论}

狭义相对论是关于时空结构的理论。这是在爱因斯坦1905年的论文《论动体的电动力学》中提出的,其他许多物理学家对狭义相对论也做出了贡献。狭义相对论基于两个在经典力学中相互矛盾的假设:

\begin{enumerate}
\item 对于所有相对于彼此匀速运动的观察者来说,物理定律是相同的(相对论原理)。
\item 无论观察者的相对运动或光源的运动如何,真空中的光速对他们来说都是一样的。
\end{enumerate}

由此产生的理论比经典力学更好地应对实验。例如,假设2解释了迈克耳孙-莫雷实验的结果。此外,该理论有许多令人惊讶和出乎意料的结果。其中一些是:

\begin{itemize}
\item 相对同时:如果两个观察者处于相对运动状态,则对于一个观察者同时发生的两个事件对于另一个观察者可能不是同时发生的。
\item 时间膨胀:移动的时钟比观察者的“静止”时钟走得更慢。
\item 长度收缩:在相对于观察者移动的方向上,物体被测量的长度缩短。
\item 最大速度是有限的:没有任何物理物体、信息或场线能比真空中的光速更快。
\item 重力的作用只能以光速在空间中传播,不能比光速更快或即刻传播。
\item 质能等价:$E = mc^2$ 能量和质量是等价的,可以相互转化。
\item 相对质量,是被一些研究者使用的概念。[9]
\end{itemize}

狭义相对论的定义特征是用洛伦兹变换代替经典力学中的伽利略变换。

\subsection{广义相对论}

广义相对论是爱因斯坦在1907-1915年间发展起来的引力理论。广义相对论的发展始于等效原理,等效原理指加速运动和静止在引力场中的状态在物理上是相同的,例如,当站在地球表面时的状态。其理论的结果阐明自由落体是惯性运动:自由落体中的物体正在下落,因为当没有力施加在物体上时,物体是这样运动的,而非像经典力学所述,物体的下落是由于重力。广义相对论与经典力学和狭义相对论并不相容,因为在经典力学和狭义相对论中,惯性运动的物体不能相对于彼此加速,但自由落体的物体会加速。为了解决这个难题,爱因斯坦首先提出时空是弯曲的。1915年,他设计了爱因斯坦场方程,将时空的曲率与质量、能量和动量联系起来。

广义相对论的一些结果是:

\begin{itemize}
\item 引力时间膨胀:在更深的重力井中,时钟运行得更慢。[10]
\item 进动:在牛顿的引力理论中,轨道以一种意想不到的方式进动。(这已经在水星轨道和双星系统中被观察到。)
\item 光线偏转:光线在重力场的存在下弯曲。
\item 参考系拖拽:旋转的质量“拖拽着”它们周围的时空。
\item 空间的度量膨胀:宇宙正在膨胀,它的远处以比光速更快的速度远离我们。
\end{itemize}

严格意义上,广义相对论是一种引力理论,其定义特征是使用爱因斯坦场方程。场方程的解是度规张量,它定义了时空的拓扑结构和物体如何惯性运动。

\subsection{实验证据}

爱因斯坦指出相对论属于一类“原理理论”。因此,它采用的分析方法意味着该理论的原理不是基于假设,而是基于经验发现。通过观察自然过程,我们了解它们的一般特征,设计数学模型来描述我们所观察到的,并通过分析手段,我们推导出必须满足的必要条件。对不同事件的测量必须满足这些条件,并符合理论的结论。[2]

\subsubsection{4.1 狭义相对论的实验验证}

相对论是一种可证伪的理论:它做出的预测可以通过实验来检验。就狭义相对论而言,包括相对论原理、光速的恒定性和时间膨胀。[11] 自爱因斯坦于1905年发表论文以来,狭义相对论的预测已经在许多测试中被证实,但在1881年至1938年间进行的三次实验对验证狭义相对论至关重要。这些是迈克耳孙-莫雷实验、肯尼迪-桑代克实验和艾夫斯-史迪威实验。爱因斯坦在1905年从第一性原理推导出洛伦兹变换,但这三个实验允许从实验证据中推导出该变换。

麦克斯韦方程组——经典电磁学的基础——将光描述为以特征速度运动的波。现代观点认为光不需要传播媒介,但麦克斯韦和他的同时代人相信光波在媒介中传播,类似于声音在空气中传播,涟漪在池塘表面传播。这种假想的介质被称为发光以太,相对于“固定恒星”静止,地球通过它运动。菲涅尔的部分以太拖曳假说排除了一阶效应$(v/c)$的测量,尽管二阶效应的观测$( v^2/c^2)$在原则上是可能的,麦克斯韦认为它们太小,用当时的技术无法探测到。[12][13]

\begin{figure}[ht]
\centering
\includegraphics[width=6cm]{./figures/99a78b85b5e3324c.png}
\caption{迈克耳孙-莫雷实验图解} \label{fig_XDL_2}
\end{figure}

迈克耳孙-莫雷实验被设计用来探测“以太风”的二阶效应——以太相对于地球的运动。迈克耳孙设计了一种名为迈克耳孙干涉仪的仪器来实现这一点。该仪器非常精确,足以检测预期的效果,但当1881年进行第一次实验时,他得到了一个零值的结果,[14]1887年的结果又一次得到零值。[15]尽管未能探测到以太风令人失望,但结果被科学界接受了。[13]为了挽救以太范式,菲茨杰拉德和洛伦兹分别创造了一个特里假设:物质的长度随着它们在以太中的运动而变化。[16]这是菲茨杰拉德-洛伦兹收缩的起源,他们的假说没有理论基础。对迈克耳孙-莫雷实验零结果的解释是,光的往返旅行时间是各向同性的(与方向无关),但仅靠结果是不足以推翻以太理论或验证狭义相对论的预测。[17][18]

虽然迈克耳孙-莫雷实验表明光速是各向同性的,但它没有说明在不同惯性系中光速的大小是如何变化的(如果有的话)。肯尼迪-桑代克实验就是为此而设计的,并于1932年由罗伊·肯尼迪和爱德华·桑戴克首次进行。[19]他们得到了一个零值的结果,并得出结论:“没有效果……除非太阳系在太空中的速度不超过地球轨道速度的一半。[18][20]这种可能性被认为太巧合了,无法提供一个可接受的解释,所以从他们的实验的零值结果可以得出结论,在所有惯性参考系中,光的往返时间都是相同的。[17][18]

\begin{figure}[ht]
\centering
\includegraphics[width=6cm]{./figures/1ee3f9b8a098947b.png}
\caption{肯尼迪-桑代克实验显示了干涉条纹} \label{fig_XDL_3}
\end{figure}

艾夫斯-史迪威实验是由赫伯特·艾夫斯和G.R.史迪威在1938年首先进行的,[21]在1941年更准确。[22]它被设计用来测试横向多普勒效应——来自移动光源的光在垂直于其速度方向上的红移——这是爱因斯坦在1905年预测的。策略是将观测到的多普勒频移与经典理论预测的进行比较,并寻找洛伦兹因子校正。这样的校正被观察到,由此得出结论,根据狭义相对论,移动原子钟的频率是改变的。[17][18]

那些经典的实验已经被重复了许多次,而且精度越来越高。其他实验包括,例如,相对论性能量和动量在高速下的增加,时间膨胀的实验测试,以及对洛伦兹违逆的现代研究。

\subsubsection{4.2 广义相对论的实验验证}

广义相对论也已经被多次证实,经典的实验是水星轨道的近日点进动,太阳对光的偏转,以及光的重力红移。其他测试证实了等效原理和框架拖动。

\subsection{现代应用}

相对性效应远不仅仅是理论上的兴趣,而是重要的实际工程问题。基于卫星的测量需要考虑相对论效应,因为根据相对论,每个卫星相对于地球上的用户都在运动,因此处于不同的参照系中。全球化的定位系统,如全球定位卫星、全球导航卫星系统和即将问世的伽利略实验,必须考虑到所有的相对论效应,比如地球引力场的后果,才能精确工作。[23]在高精度时间测量中也是如此。[24]如果忽略相对论性的考虑,从电子显微镜到粒子加速器的仪器都不会工作。

\subsection{参考文献}

\begin{enumerate}
\item Einstein A. (1916), Relativity: The Special and General Theory (Translation 1920), New York: H. Holt and Company.
\item Einstein, Albert (November 28, 1919). "Time, Space, and Gravitation" . The Times..
\item Will, Clifford M (2010). "Relativity". Grolier Multimedia Encyclopedia. Retrieved 2010-08-01..
\item Will, Clifford M (2010). "Space-Time Continuum". Grolier Multimedia Encyclopedia. Retrieved 2010-08-01..
\item Will, Clifford M (2010). "Fitzgerald–Lorentz contraction". Grolier Multimedia Encyclopedia. Retrieved 2010-08-01..
\item Planck, Max (1906), "Die Kaufmannschen Messungen der Ablenkbarkeit der β-Strahlen in ihrer Bedeutung für die Dynamik der Elektronen (The Measurements of Kaufmann on the Deflectability of β-Rays in their Importance for the Dynamics of the Electrons)" , Physikalische Zeitschrift, 7: 753–761.
\item Miller, Arthur I. (1981), Albert Einstein's special theory of relativity. Emergence (1905) and early interpretation (1905–1911), Reading: Addison–Wesley, ISBN 978-0-201-04679-3.
\item Hey, Anthony J.G.; Walters, Patrick (2003). The New Quantum Universe (illustrated, revised ed.). Cambridge University Press. p. 227. Bibcode:2003nqu..book.....H. ISBN 978-0-521-56457-1..
\item Greene, Brian. "The Theory of Relativity, Then and Now". Retrieved 2015-09-26..
\item Feynman, Richard Phillips; Morínigo, Fernando B.; Wagner, William; Pines, David; Hatfield, Brian (2002). Feynman Lectures on Gravitation. West view Press. p. 68. ISBN 978-0-8133-4038-8., Lecture 5.
\item Roberts, T; Schleif, S; Dlugosz, JM (ed.) (2007). "What is the experimental basis of Special Relativity?". Usenet Physics FAQ. University of California, Riverside. Retrieved 2010-10-31.CS1 maint: Extra text: authors list (link).
\item Maxwell, James Clerk (1880), "On a Possible Mode of Detecting a Motion of the Solar System through the Luminiferous Ether" , Nature, 21 (535): 314–315, Bibcode:1880Natur..21S.314., doi:10.1038/021314c0.
\item Pais, Abraham (1982). "Subtle is the Lord ...": The Science and the Life of Albert Einstein (1st ed.). Oxford: Oxford Univ. Press. pp. 111–113. ISBN 978-0-19-280672-7..
\item Michelson, Albert A. (1881). "The Relative Motion of the Earth and the Luminiferous Ether" . American Journal of Science. 22 (128): 120–129. Bibcode:1881AmJS...22..120M. doi:10.2475/ajs.s3-22.128.120..
\item Michelson, Albert A. & Morley, Edward W. (1887). "On the Relative Motion of the Earth and the Luminiferous Ether" . American Journal of Science. 34 (203): 333–345. Bibcode:1887AmJS...34..333M. doi:10.2475/ajs.s3-34.203.333.CS1 maint: Multiple names: authors list (link).
\item Pais, Abraham (1982). "Subtle is the Lord ...": The Science and the Life of Albert Einstein (1st ed.). Oxford: Oxford Univ. Press. p. 122. ISBN 978-0-19-280672-7..
\item Robertson, H.P. (July 1949). "Postulate versus Observation in the Special Theory of Relativity" (PDF). Reviews of Modern Physics. 21 (3): 378–382. Bibcode:1949RvMP...21..378R. doi:10.1103/RevModPhys.21.378..
\item Taylor, Edwin F.; John Archibald Wheeler (1992). Spacetime physics: Introduction to Special Relativity (2nd ed.). New York: W.H. Freeman. pp. 84–88. ISBN 978-0-7167-2327-1..
\item Kennedy, R.J.; Thorndike, E.M. (1932). "Experimental Establishment of the Relativity of Time". Physical Review. 42 (3): 400–418. Bibcode:1932PhRv...42..400K. doi:10.1103/PhysRev.42.400..
\item Robertson, H.P. (July 1949). "Postulate versus Observation in the Special Theory of Relativity" (PDF). Reviews of Modern Physics. 21 (3): 381. Bibcode:1949RvMP...21..378R. doi:10.1103/revmodphys.21.378..
\item Ives, H.E.; Stilwell, G.R. (1938). "An experimental study of the rate of a moving atomic clock". Journal of the Optical Society of America. 28 (7): 215. Bibcode:1938JOSA...28..215I. doi:10.1364/JOSA.28.000215..
\item Ives, H.E.; Stilwell, G.R. (1941). "An experimental study of the rate of a moving atomic clock. II". Journal of the Optical Society of America. 31 (5): 369. Bibcode:1941JOSA...31..369I. doi:10.1364/JOSA.31.000369..
\item "Archived copy" (PDF). Archived from the original (PDF) on 2015-11-05. Retrieved 2015-12-09.CS1 maint: Archived copy as title (link).
\item Francis, S.; B. Ramsey; S. Stein; Leitner, J.; Moreau, J.M.; Burns, R.; Nelson, R.A.; Bartholomew, T.R.; Gifford, A. (2002). "Timekeeping and Time Dissemination in a Distributed Space-Based Clock Ensemble" (PDF). Proceedings 34th Annual Precise Time and Time Interval (PTTI) Systems and Applications Meeting: 201–214. Archived from the original (PDF) on 17 February 2013. Retrieved 14 April 2013..
\end{enumerate}
