% 斜坐标系

\pentry{极坐标系\upref{Polar},柱坐标系\upref{Cylin},球坐标系\upref{Sph}}

斜坐标系是通常的直角坐标系的延伸.以二维空间中为例,任取两根相互不平行的坐标轴$x$和$y$,都能对平面上任意一点$P$进行定位,方法是认为坐标轴上已经给定了标尺,从$P$处画一条平行于$y$轴的直线,它与$x$轴相交处的标尺数值即为$P$的$x$坐标;类似地,画一条平行于$x$轴的直线,它与$y$轴相交处的标尺数值即为$P$的$y$坐标.

\begin{figure}[ht]
\centering
\includegraphics[width=8cm]{./figures/ObSys_1.pdf}
\caption{一个斜坐标系的示意图.图中$x$、$y$两轴的夹角是$72.9^\circ$,两轴的标尺已经给出.在这个坐标系中,$P$的坐标是$<2.5, 4>$.} \label{ObSys_fig1}
\end{figure}

斜坐标系的标尺不一定是均匀的.事实上,只要标尺数值沿着坐标轴的方向一直递增就可以.

\subsection{与直角坐标的转换}

为方便计,我们只考虑标尺均匀的斜坐标系.

给定直角坐标系$xOy$,由相互垂直的$x$轴和$y$轴构成,两轴交于原点$O$.另给定斜坐标系$x'Oy'$,由$x'$轴和$y'$轴构成,两轴分别和对应的直角坐标轴成角度$\theta_x$和$\theta_y$,也相交于原点$O$.

