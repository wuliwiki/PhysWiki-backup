% 线性映射的张量积
% keys 对称积|反对称积|对称幂|反对称幂
% license Xiao
% type Wiki

\begin{issues}
\issueDraft
\issueOther{可以对照张量的对称化和交错化\upref{SIofTe}进行阅读}
\end{issues}
\pentry{向量空间的张量积\upref{TPofSp}}

\subsection{线性映射的张量积}

两个线性映射 $f: V \to W$,$g: V' \to W'$ 之间可以定义它们的张量积\footnote{更严格的写法应该是 $\sum_i v_i \otimes v_i' \mapsto \sum_i f(v_i) \otimes g(v_i')$,不过本文的所有映射都是线性映射,所以只需要定义一组基的线性变换即可。}
\begin{equation}
\begin{aligned}
f \otimes g: V \otimes V' &\to W \otimes W', \\
v \otimes v' &\mapsto f(v) \otimes g(v')~.
\end{aligned}
\end{equation}
此时 $f \otimes g \in L(V \otimes V', W \otimes W')$。

从另一个角度来说,全体 $V$ 到 $W$ 的线性映射的集合 $L(V, W)$ 是一个向量空间(参考\autoref{sub_LinMap_1}~\upref{LinMap}),因此可以定义两个线性映射空间之间的张量积 $L(V, W) \otimes L(V', W')$;用这种方法我们也可以定义线性映射的张量积,此时 $f \otimes g \in L(V, W) \otimes L(V', W')$。这两种定义并不完全等价。

\begin{theorem}{}
线性映射空间的张量积是张量积的线性映射空间的子集,即\footnote{用更严谨的说法是,它们之间存在一个(典范的)线性映射}
\begin{equation}
L(V, W) \otimes L(V', W') \subseteq L(V \otimes V', W \otimes W')~.
\end{equation}
如果 $V, W$ 是有限维度向量空间,那么上述包含关系相等。
\end{theorem}

\addTODO{给出反例,并证明有限维度时它们等价}



\subsubsection{用矩阵表示张量积}

\addTODO{Kronecker product}

\subsection{线性映射的对称/反对称幂}

\pentry{向量空间的对称/反对称幂\upref{vecSAS}}