% 无穷 Galois 扩张与Krull定理
% 域论|域扩张|Galois群|伽罗华|伽罗瓦|Krull拓扑|Krull定理|无限Galois扩张|伽罗华扩张

\pentry{Galois扩张\upref{GExt},拓扑空间\upref{Topol}}

\addTODO{尚未完成}

\textbf{Galois扩张}\upref{GExt}中除了Galois扩张和Galois群的基本性质,剩下的重点内容全是\textbf{有限}Galois扩张的情况,见\autoref{GExt_sub1}~\upref{GExt}.作为提醒,再总结一次:有限Galois扩张都是单代数扩张,且为分裂域.

本节介绍的是无限Galois扩张中的性质,将有限扩张的\textbf{Galois理论基本定理}(\autoref{GExt_the10}~\upref{GExt})进行拓展,得到\textbf{Krull}定理.Krull的工作亮点,在于给Galois群赋予了一个拓扑结构.


为了得到Krull拓扑,我们要先观察Galois扩域的一些性质.注意,接下来我们不再限定为有限扩张了.


由\autoref{GExt_the6}~\upref{GExt},$\mathbb{K}/\mathbb{M}$是Galois扩张,因此$\opn{Gal}(\mathbb{K}/\mathbb{M})$存在.有了这一点,我们就可以讨论下面两条引理:

\begin{lemma}{}\label{GExInf_lem2}
设$\mathbb{K}/\mathbb{F}$是Galois扩张,且存在中间域$\mathbb{M}$.

则$\mathbb{M}/\mathbb{F}$是Galois扩张 $\iff$ $\mathbb{M}/\mathbb{F}$是正规扩张 $\iff$ $\opn{Gal}(\mathbb{K}/\mathbb{M})\triangleleft \opn{Gal}(\mathbb{K}/\mathbb{F})$ .
\end{lemma}

\autoref{GExInf_lem2} 实际上就是\autoref{GExt_the8}~\upref{GExt},因此证明参见该定理.


\begin{lemma}{}\label{GExInf_lem1}
设$\mathbb{K}/\mathbb{F}$是Galois扩张,且存在中间域$\mathbb{M}$.则$[\opn{Gal}(\mathbb{K}/\mathbb{F}):\opn{Gal}(\mathbb{K}/\mathbb{M})]=[\mathbb{M}:\mathbb{F}]$.


\end{lemma}


\textbf{证明}:

取$f, g\in\opn{Gal}(\mathbb{K}/\mathbb{F})$,则$f$和$g$模$\opn{Gal}(\mathbb{K}/\mathbb{M})$同余\textbf{当且仅当}$f^{-1}g\in\opn{Gal}(\mathbb{K}/\mathbb{M})$,或者说$f$和$g$限制在$\mathbb{M}$上是相同的.

因此,$\opn{Gal}(\mathbb{K}/\mathbb{M})$的每个左陪集对应一个$\mathbb{M}/\mathbb{F}$的映射.

\textbf{证毕}.


这条引理很像\autoref{GExt_the9}~\upref{GExt},只不过不再要求是有限扩张了.这显得\autoref{GExt_the9}~\upref{GExt}似乎没有存在的必要,然而我们依然将它保留了,体现“有限Galois扩张就是单扩张”的思路.




\begin{theorem}{}
设$\mathbb{K}/\mathbb{F}$是Galois扩域,$\mathcal{M}=\{\opn{Gal}(\mathbb{K}/\mathbb{M})\mid \mathbb{M}/\mathbb{F}\text{是有限扩张}\}$.则有:

1.$\forall H\in\mathcal{M}$,$[\opn{Gal}(\mathbb{K}/\mathbb{F}): H]$\footnote{即群指数,子群$H$在$\opn{Gal}(\mathbb{E}/\mathbb{K})$中陪集的数量.}是有限的;

2.$\bigcap_{H\in\mathcal{M}}=\{e\}$.这里$e$是群的单位元,即$\mathbb{K}$上的恒等映射$\opn{id}_{\mathbb{K}}$.

3.$\forall H_1, H_2\in\mathcal{M}$,有$H_1\cap H_2\in\mathcal{M}$;

4.$\forall H\in\mathcal{M}$,$\exists N\triangleleft\opn{Gal}(\mathbb{K}/\mathbb{F})$,且$\{e\}\neq N\subseteq H$;

\end{theorem}

\textbf{证明}:

1.

由\autoref{GExInf_lem1} 直接可得.

2.

任取$\sigma\in\opn{Gal}(\mathbb{K}/\mathbb{F})$,只要$\sigma\not=\opn{id}_{\mathbb{K}}$,那么就存在$\alpha\in\mathbb{K}-\mathbb{F}$使得$\sigma(\alpha)\neq\alpha$.取$\mathbb{M}=\mathbb{F}(\alpha)$,则$\sigma\not\in\opn{Gal}(\mathbb{K}/\mathbb{F})$.故$\sigma\not\in \bigcap_{H\in\mathcal{M}}$.

3.

由\autoref{GExt_lem1}~\upref{GExt},$H_1\cap H_2$是$\mathbb{M}_1\mathbb{M}_2$的Galois群.

有限域的合成可以看成$\mathbb{M}_1$用$\mathbb{M}_2$的元素反复进行有限次单扩张的结果.由于是Galois扩张,故这些单扩张全都是代数扩张,从而是有限扩张,从而$\mathbb{M}_1\mathbb{M}_2/\mathbb{F}$是有限扩张.

4.

首先讲一下证明思路:由于$N$也是$\opn{Gal}(\mathbb{K}/\mathbb{F})$的子群,因此如果$N$存在,其必有不变子域.由

\textbf{证毕}.





\subsection{Krull 定理}\label{GExInf_sub1}


\subsubsection{Krull拓扑}

考虑任意集合$X$和$Y$,令$M\subseteq X^Y$.任取$f\in M$,以及$X$的\textbf{有限}子集$S$,令
\begin{equation}
V(f, S) = \{g\in M\mid g(s)=f(s), \forall s\in S\}
\end{equation}
即$V(f, S)$是全体属于$M$且限制在$S$上与$f$相等的映射的集合.

任取$h\in V(f, S)\cap V(g, T)$.由于$h$是在$S$上和$f$相等、在$T$上和$g$相等,

则有$V(f, S)\cap V(g, T) = V(h, S\cup T)$





































