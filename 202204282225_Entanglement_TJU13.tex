% 天津大学 2013 年考研量子力学
% 考研|天津大学|量子力学|2013

\subsection{ }
\begin{enumerate}
\item 假设氢原子处于状态$\Psi (r,\theta,\varphi)=\frac{1}{\sqrt{3}}\Psi_{310}-\frac{\sqrt{2}}{\sqrt{3}}\Psi_{210}$,写出氢原子主量子数,角动量的平方及角动量第三分量的平均值.
\item 举例说明电子和光子具有波粒二象性.
\item 写出$\Psi = A \sin(kx)\cdot\cos(2kx)$的动量平方的平均值.
\end{enumerate}
\subsection{ }
\begin{enumerate}
\item 质量为$m$,频率为$\omega$的谐振子初始时刻处于状态$Ax^2\Psi_n(x)$.\\
(1)求归一化系数$A$.\\
(2)任何时刻谐振子波函数及坐标的平均值.
\item 对于质量为$m$,频率为$\omega$的各向同性谐振子,写出相应波函数及能级.
\end{enumerate}
\subsection{ }
质量为$m$的粒子被限制在无限大平行板之间,设板间距为$d$,求体系能级和波函数.
\subsection{ }
质量为$m$的粒子处于势场$V(x)=\leftgroup{&\infty \quad x<0,x>a\\ &\alpha x \quad 0\le x \le \frac{a}{2}\\ &0 \quad \frac{a}{2}\le x \le a} $