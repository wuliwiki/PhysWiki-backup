% 乔治·斯托克斯(综述)
% license CCBYSA3
% type Wiki

本文根据 CC-BY-SA 协议转载翻译自维基百科\href{https://en.wikipedia.org/wiki/Sir_George_Stokes,_1st_Baronet}{相关文章}。


乔治·加布里埃尔·斯托克斯爵士,第一代从男爵(/stoʊks/;1819年8月13日-1903年2月1日),是爱尔兰数学家和物理学家。斯托克斯出生于爱尔兰斯莱戈郡,在剑桥大学度过了整个职业生涯,并在1849年至1903年去世期间担任卢卡斯数学教授长达54年,是该职位任期最长的持有者。

作为物理学家,斯托克斯在流体力学领域作出了开创性的贡献,包括纳维-斯托克斯方程;在物理光学方面,他的研究涵盖偏振和荧光等现象。作为数学家,他普及了矢量微积分中的“斯托克斯定理”,并对渐近展开理论作出了贡献。斯托克斯与菲利克斯·霍普-塞勒一道,首次揭示了血红蛋白的携氧功能,并展示了血红蛋白溶液通气后所产生的颜色变化。

1889年,斯托克斯被英国君主封为从男爵。1893年,他因“在物理科学领域的研究与发现”获得当时全球最负盛名的科学奖项——皇家学会的科普利奖章。他曾于1887年至1892年在英国下议院担任剑桥大学选区的议员,隶属保守党。斯托克斯还于1885年至1890年担任皇家学会会长,并曾短暂出任剑桥大学彭布罗克学院院长。由于他的大量通信往来以及担任皇家学会秘书期间的工作,他被称为维多利亚时代科学的大门守卫者,其贡献远远超越了他发表的论文本身\(^\text{[1]}\)。
