% 稠密性与完备性
% keys 序|稠密|完备

\begin{issues}
\issueDraft
\issueTODO
\end{issues}

我们从最常见的两个全序集开始,即$(\mathbb{Q};\leq),(\mathbb{R};\leq)$.我们知道它们可以表示在数轴上,如下图.

\addTODO{缺少图.}

直观上,有理数集密集在数轴上,实数集则填满了整个数轴.在数学上,我们分别称之为稠密性和完备性.

\subsection{(序)稠密性}

如何严格定义稠密性呢?不妨以自然数和有理数为例.

我们说自然数集是离散的,因为两个相邻的自然数之间不能有其他自然数,所以是有“空隙”的.比如不存在一个自然数既大于3又小于4.从数轴上看,自然数集是一系列零散的点.

那么对于有理数呢?有理数集应该是稠密的,这符合我们的直观.当我们想进行类似于自然数的讨论时,一个问题就出现了:不存在两个相邻的有理数.比如对于0和1,$\frac{1}{2}$就在它们之间,因此0和1不相邻.实际上,对任何两个有理数都是如此.仔细一想,这不就反映了稠密的本质吗?

因此,我们有如下的定义.

\begin{definition}{(序)稠密性}
如果偏序集$(A;\leq)$满足对任意$a,b\in A$,存在$c\in A$使$a\leq c\leq b$,则称$(A;\leq)$是(序)稠密的.

如果偏序集$(A;\leq)$的一个非空子集$B$满足对任意$a,b\in A$,且$a\neq b$,存在$c\in B$使$a\leq c\leq b$,则称$B$在$A$中稠密,或称$B$是$A$的稠密子集.
\end{definition}

\begin{example}{}
$(\mathbb{Q};\leq),(\mathbb{R};\leq)$是稠密的,且$\mathbb{Q}$是$\mathbb{R}$的一个稠密子集.
\end{example}

注意对于$\mathbb{Q}$和$\mathbb{R}$而言,当$a\leq b$时总有$a\leq\frac{a+b}{2}\leq b$.

此外由实数的阿基米德性可推出$\mathbb{Q}$在$\mathbb{R}$中稠密.\footnote{实数的阿基米德性是指对任意正实数$a,b$,总存在正整数$n$使$na>b$.}

\subsection{(序)完备性}

\subsection{完备性的拓扑视角}

\pentry{连续性\upref{Topo3} 紧致性\upref{Topo2}}