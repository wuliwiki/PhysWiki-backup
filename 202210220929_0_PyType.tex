% Python 的变量
% keys Python|变量类型|整数|浮点|字符

\begin{issues}
\issueDraft
\end{issues}

\pentry{Python 简介\upref{Python}}

Python 基本变量类型有: \verb|bool|(布尔型), \verb|int|(整型)(长度不限), \verb|float|(浮点型)(双精度浮点), \verb|complex| (复数, 如 \verb|2+3j|), \verb|str|(字符串). 注意 python 本身没有定义单精度浮点数, 但在 \verb|numpy| 库中有 \verb|float32| 类型. 我们可以用 \verb|type()| 函数查看某个变量的类型. 例如执行
\begin{lstlisting}[language=python]
n = 123; x = 3.14; print(type(n)); print(type(x))
\end{lstlisting}
结果为
\begin{lstlisting}
<class 'int'>
<class 'float'>;
\end{lstlisting}
判断变量是否为某个类型
\begin{lstlisting}[language=python]
type(i) == int # true
\end{lstlisting}

用 \verb|is| 判断两个变量是否是同一个对象, 例如 \verb|a = [1,2,3]; b = [1,2,3];|, 那么 \verb|a == b| 返回 \verb|True| 而 \verb|a is b| 返回 \verb|False|. 此时改变 \verb|a| 的元素, \verb|b| 不会改变. 但若令 \verb|b = a;|, 那么 \verb|a, b| 会同时改变, 此时 \verb|a is b| 和 \verb|a == b| 都返回 \verb|True|.

\subsection{整数}
与一些编译语言不同, Python 的整数类型(integer)没有长度限制(除超出了内存大小). 例如
\begin{lstlisting}[language=python]
print(12345678901234567890123456789 + 1)
\end{lstlisting}
的结果为 \verb|12345678901234567890123456790|.

默认情况下整数用十进制表示, 如果需要输入 \textbf{2 进制(binary)}, 可以在前面加 \verb|0b| 或 \verb|0B|, 例如 \verb|0b1001| 表示 10 进制的 \verb|9|. 类似地, \verb|0o| 或 \verb|0O| 开头表示 \textbf{8 进制(octal)}; \verb|0x| 或 \verb|0X| 开头的表示 \textbf{16 进制(hexadecimal)}, 16 进制中的 10 到 15 分别用大写或小写字母 \verb|a| 到 \verb|f| 表示. 例如 \verb|0xff| 表示十进制的 \verb|255|. 不同进制的整数同样没有长度限制.

\subsection{类型转换}
转换格式为 \verb|类型(变量)|. 例如 \verb|int('123')| 会把字符串 \verb|'123'| 变为整数 \verb|123|.
\begin{lstlisting}[language=python]
"'我们在这里完整列举一下'"
a = "小时百科"  #str 字符串
b = 2 #int 整型
c = 5.2 #float 浮点型
d = True #bool 布尔型
e = int(c)#返回值5.0,字符串中的数字可以转换,文字不可以
e = str(b)#返回值2,此时的2可以用来进行字符串的加减,如:
f = e + "00"
print(f)#返回值200
e = int (d)
print (e)#返回值1
e = float(d)
print (e)#返回值1.0
e = bool(a)
print (e)#返回值True
e = bool(b)
print (e)#返回值True
e = bool(c)
print (e)#返回值True
e = bool(0)
print (e)#返回值False
\end{lstlisting}
\subsection{字符串}
raw string: \verb|r'foo\nbar'| 其中 \verb|\n| 会被当成两个字符.

\subsection{变量和对象}
和 C 语言类似的语言不同, Python 中的变量和对象(object)是分开考虑的. 对象可以理解为内存上的一小段某类型的数据, 而变量是指向对象的一个个名称. 例如两个变量可以指向同一个对象, 要判断变量 \verb|a| 和 \verb|b| 是否指向同一个对象, 用 \verb|a is b|, 返回一个 \verb|bool|.

\subsection{变量的范围}
\begin{itemize}
\item \verb|dir()| will give you the list of in scope variables:
\item \verb|globals()| will give you a dictionary of global variables
\item \verb|locals()| will give you a dictionary of local variables
\end{itemize}
