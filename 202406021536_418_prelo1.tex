% 个体词与谓词(数理逻辑)
% keys 个体词|谓词
% license Usr
% type Tutor

\pentry{原子命题\nref{nod_propco}}{nod_d33c}

\subsection{定义}
\begin{definition}{个体词}
在原子命题中,可以独立存在的客体(句子中的主语、宾语等),称为个体词,可以是\textbf{个体常量}或\textbf{个体变量}。
\begin{itemize}
\item \textbf{个体常量}表示特定的或具体的个体,是不变的,一般用带或不带下标的小写字母 $a, b, c, \dots, a_1, a_2, \dots$ 表示。
\item \textbf{个体变量}表示泛指的或抽象的、不定的个体,是变化的,一般用带或不带下标的小写字母 $x, y, z, \dots, x_1, x_2, \dots$ 表示。
\end{itemize}

\end{definition}
\begin{definition}{论域}
所有个体词的取值范围称为\textbf{论域}(或\textbf{取值域}),常用字母 $D$ 表示。
\end{definition}


\begin{definition}{谓词}
用以刻画客体的属性或心智(类似于一元函数),或多个客体之间的关系(类似于多元函数)的词称为\textbf{谓词}。
\end{definition}

\subsection{例子}
个体词与谓词的定义比较抽象,下面借助几个例子来理解。

\begin{example}{}
用 $P(x)$ 表示 $x$ 是北方人、$Q(x)$ 表示 $x$ 怕冷,$c$ 表示李华,符号化下面这个句子:

除非李华是北方人,否则李华一定怕冷。
\end{example}
首先表示为命题逻辑的形式,先用 $p$ 表示“李华怕冷”,用 $q$ 表示“李华是北方人”,原来的句子就可以化作
\begin{equation}
\neg p \to  q ~.
\end{equation}
显然 $p$ 就是 $Q(c)$,而 $q$ 就是 $P(c)$,所以这句子就可以化为
\begin{equation}
\neg Q(c) \to  P(c) ~.
\end{equation}
