% 超几何函数(综述)
% license CCBYSA3
% type Wiki

本文根据 CC-BY-SA 协议转载翻译自维基百科\href{https://en.wikipedia.org/wiki/Hypergeometric_function}{相关文章}。

\begin{figure}[ht]
\centering
\includegraphics[width=6cm]{./figures/d7c2f81e451fe639.png}
\caption{超几何函数 ${}_2F_1(a, b; c; z)$ 在复平面上从 $-2 - 2i$ 到 $2 + 2i$ 的图像,其中参数取值为 $a = 2$、$b = 3$、$c = 4$,图像的颜色由 Mathematica 13.1 的函数 ComplexPlot3D 生成。} \label{fig_CJHhs_1}
\end{figure}
在数学中,高斯或普通超几何函数 ${}_2F_1(a,b;c;z)$ 是由超几何级数表示的特殊函数,它包含了许多其他特殊函数作为特例或极限情形。它是一个二阶线性常微分方程(ODE)的解。任何具有三个正规奇点的二阶线性常微分方程都可以转化为该方程。

关于超几何函数所涉及的成千上万条恒等式的系统整理,可以参见 Erdélyi 等人(1953年)和 Olde Daalhuis(2010年)的参考著作。至今尚无已知的体系可以组织所有这些恒等式;事实上,也没有已知的算法可以生成所有恒等式。目前已知的算法各自能生成不同系列的恒等式。发现恒等式的算法理论仍是一个活跃的研究课题。
\subsection{历史}
“超几何级数”这一术语最早由约翰·沃利斯在其1655年的著作《无穷算术》中提出。

超几何级数曾被莱昂哈德·欧拉研究过,但最早对其进行系统全面研究的是卡尔·弗里德里希·高斯,时间是在1813年。

19世纪的研究包括恩斯特·库默尔(Ernst Kummer,1836年)的工作,以及伯恩哈德·黎曼(Bernhard Riemann,1857年)通过超几何函数所满足的微分方程对其进行的基本刻画。

黎曼证明,对于 ${}_2F_1(z)$ 的二阶微分方程,在复平面上可以通过其在黎曼球面上的三个正规奇点来进行刻画。

而当超几何方程的解为代数函数的情形,则由赫尔曼·施瓦茨确定下来,这就是著名的“施瓦茨表”。
