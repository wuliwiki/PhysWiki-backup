% 因果结构
% 相对论|时空|广义相对论|因果性|因果|曲线|黎曼度量|伪黎曼度量

\pentry{连通性\upref{Topo3},黎曼度量与伪黎曼度量\upref{RiMetr}}



在研究时空的理论中,因果结构是一个基本问题.根据David B. Malament于1977年发表的论文\textsl{The class of continuous timelike curves determines the topology of spacetime}\footnote{Malament, Journal of Mathematical Physics 18:7, 1399-1404. },在一个给定的时空中,确定了弱因果性就可以完全确定该时空的微分结构.本词条旨在对因果结构相关的部分概念进行收集和解释,方便阅读相关论文中.

\subsection{概念}

\subsubsection{时空}

一个\textbf{时空(spacetime)}是指一个无界的、连通的四维伪黎曼流形$(M, g)$,其中$g$是一个洛伦兹度规.该度规在任何观察者的参考系中可以表示为对角矩阵$\opn{diag}(1, -1, -1, -1)$.

时间流逝方向分为\textbf{未来导向}和\textbf{过去导向}的.这个分类是指时空中\textbf{类时}切向量的等价划分.如果两个类时切向量的内积为正,那么我们把它们归入同一类,由此把所有类时向量分成两类,称其中一类为未来导向的,另一类为过去导向的.未来和过去的选择是随机的,因为在目前的研究中两个等价类是完全等同的,尚未找到区分它们的细致结构——然而实际的宇宙中似乎有明显的时间导向,因此寻找描述两个等价类根本差异的结构也是数学物理学中一个令人感兴趣的话题.

\subsubsection{曲线}

一条\textbf{曲线(curve)}是指一个从实数轴上的任意开区间$I$到拓扑空间的连续映射,在时空理论中这个拓扑空间自然是指时空.如果曲线是一个光滑映射,那么也被称为一条\textbf{光滑曲线(smooth curve)}.另外,此术语等同于一些拓扑学家口中的\textbf{道路(path)}.

一条\textbf{类时曲线(timelike curve)}是指一条切向量处处类时的曲线.

一条\textbf{类空曲线(spacelike curve)}是指一条切向量处处类空的曲线.

一条\textbf{类光曲线(lightlike curve)}或\textbf{零曲线(null curve)}是指一条切向量处处类光的曲线.

一条\textbf{因果曲线(causal curve)}或\textbf{非类空曲线(non-spacelike curve)}是指一条类时或类光的曲线,它以“因果”命名是因为它完全可以成为一个粒子的世界线,因而体现了粒子随着时间运动的因果顺序.


\subsubsection{因果结构}

对于时空中的两个点$x, y\in M$,我们定义它们之间的不同关系如下:

\begin{itemize}
\item $y$是$x$的\textbf{类时未来},如果存在一条\textbf{未来导向}的\textbf{类时}曲线自$x$出发,并经过$y$.这个关系常记为$x\ll y$.
\item $y$是$x$的\textbf{严格因果未来},如果存在一条\textbf{未来导向}的\textbf{类时或类光}曲线自$x$出发,并经过$y$.这个关系常记为$x< y$.
\item $y$是$x$的\textbf{因果未来},如果$y$是$x$的严格因果未来或者$x=y$.这个关系常记为$x\leq y$或者$x\prec y$.
\end{itemize}

对于时空中的两个集合$A$与$O$,如果




