% 重心
% 重力|质心|质心系|力矩

\begin{issues}
\issueDraft
\end{issues}

\pentry{质心\upref{CM}, 力矩\upref{Torque}}

我们来将\autoref{CM_ex1}~\upref{CM} 拓展到一般情况

我们先来定义重心: 一个刚体在均匀重力场中的\textbf{重心(center of gravity)}, 就是它所受的重力关于重心产生的合力矩为零的点.

可以证明, 质心就是重心.

\subsection{力矩}
\addTODO{刚体在计算力矩时可以看成重力都集中在质心一点 (参考刚体摆例题). 由此可得刚体通过某点被悬挂时, 重心必然在该点或其正下方.}
