% 平衡位置和圈
% keys 平衡位置|圈|闭轨线|常微分方程的解
% license Xiao
% type Tutor

\pentry{自治系统解的特点\upref{AuSy}}
本节介绍自治系统(\autoref{def_ODEPr_2}~\upref{ODEPr})解的分类。概括来说,自治系统的解(或轨线\upref{AuSy})有三种情形:
\begin{enumerate}
\item 定点解,即轨线成为个不依赖自变量 $x$ 的点,称为\textbf{平衡位置};
\item 周期解,即轨线自身相交,称为\textbf{闭轨线}或\textbf{圈(环)};
\item 自身不相交的轨线。
\end{enumerate}
\begin{theorem}{}\label{the_EqliP_1}
设 $y^i=\varphi_i(x)$ 是自治系统的解,其最大存在区间为 $(m_1,m_2)$,并且存在 $m_1<x_1\neq x_2<m_2$,满足
\begin{equation}\label{eq_EqliP_1}
\varphi_i(x_1)=\varphi_i(x_2)~.
\end{equation}
则 $m_1=-\infty,m_2=+\infty$,且\textbf{要么}:对一切值 $x$ 成立 $\varphi_i(x)=a^i$,其中 $a^i$ 是与 $x$ 无关的常数,这时轨线成为解的定义空间中的点。此时解本身称为\textbf{平衡位置}。\\
\textbf{要么}:存在正数 $T$,使得对任意 $t$ 成立
\begin{equation}\label{eq_EqliP_2}
\varphi_i(x+T)=\varphi_i(x)~.
\end{equation}
且当 $\abs{\tau_1-\tau_2}<T$ 时,至少对一个 $i$,成立 $\varphi_i(\tau_1)\neq\varphi_i(\tau_2)$。此时解称为\textbf{周期的},$T$ 称为\textbf{最小正周期}。此时轨线称为\textbf{闭轨线}或\textbf{圈(环)}。
在证明定理前,先定义解的周期概念。
\end{theorem}
\begin{definition}{周期}
满足\autoref{eq_EqliP_2} 的 $T$ 称为解 $y^i=\varphi_i(x)$ 的周期。
\end{definition}
$T$之所以是最小的正周期,是因为否则的化,将会存在 $T_1<T$,使得 $\varphi_i(x+T_1)=\varphi_i(x)$,那么对任意 $\tau_2=\tau_1+T_1$ ,成立 $\varphi_i(\tau_1)=\varphi_i(\tau_2)$,然而此时 $\abs{\tau_1-\tau_2}=T_1<T$,结果与 $\varphi_i(\tau_1)\neq\varphi_i(\tau_2)$矛盾。所得矛盾证明了定理中的 $T$ 是最小正周期。

\begin{theorem}{周期集的性质}
设 $F$ 是解的所有周期构成的集合,则 $F$ 是闭的,且其配合加法构成一群。
\end{theorem}
\textbf{证明:}将 $x$ 换成 $x-T$, 就有 $\varphi_i(x)=\varphi_i(x-T)$,于是 $-T$ 也是周期。设 $T_1,T_2\in F$,则
\begin{equation}
\varphi_i((x+T_1)+T_2)=\varphi_i(x+T_1)=\varphi_i(x)~.
\end{equation}
因此此 $T_1+T_2$ 是周期。于是 $F$ 在加法下构成群。

闭集是集的任意收敛点都在集中的集合(\autoref{def_realat_1}~\upref{realat})。所以要证 $F$ 闭,只需对任意收敛于 $T_0$ 的周期序列 $T_1,\cdots, T_m,\cdots$ ,证明
\begin{equation}
\varphi_i(x+T_0)=\varphi_i(x)~
\end{equation}
即可。由于
\begin{equation}
\varphi_i(x+T_m)=\varphi_i(x)~
\end{equation}
恒成立,并且 $\varphi_i(x)$ 连续,所以
\begin{equation}
\lim_{m\rightarrow\infty}\varphi_i(x+T_i)=\lim_{m\rightarrow\infty}\varphi_i(x)~.
\end{equation}
即 $\varphi_i(x+T_0)=\varphi_i(x)$。

\textbf{证毕!}

\subsection{\autoref{the_EqliP_1} 的证明}

\textbf{证明:}由\autoref{eq_EqliP_1} ,并根据\autoref{the_AuSy_2}~\upref{AuSy}
\begin{equation}\label{eq_EqliP_3}
\varphi_i(x)=\varphi_i(x+x_1-x_2)~.
\end{equation}
由此,区间 $(m_1-x_1+x_2,m_2-x_1+x_2)$ 应合于 $(m_1,m_2)$,于是 $m_1=-\infty,m_2=+\infty$。

\autoref{eq_EqliP_3} 也表明 $x_1-x_2$ 是解的周期。若周期集 $F$ 中无最小正数,也即对任意 $\epsilon$, 有正周期 $c<\epsilon$,由 $F$ 是加法群, 所以 $mc$ ($m$ 是整数)是周期。任意实数 $c_0$,若选取 
\begin{equation}
(c_0-\epsilon)/c<c_0/c-<m<(c_0+\epsilon)/c
\end{equation}



\textbf{证毕!}












