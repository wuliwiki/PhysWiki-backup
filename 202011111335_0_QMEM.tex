% 电磁场中的单粒子薛定谔方程
% keys 电磁场|薛定谔方程|哈密顿算符|标势|矢势|广义动量

\begin{issues}
\issueOther{与词条 EMTDSE 重复, 需要合并}
\end{issues}

\pentry{点电荷的拉格朗日和哈密顿量\upref{EMLagP}, 电磁场标势和矢势\upref{EMPot}, 量子力学基本假设\upref{QMPos}, 原子单位制\upref{AU}}

本文使用原子单位. 电动力学中,电磁场中单个粒子的哈密顿量为
\begin{equation}\label{QMEM_eq1}
H = \frac{1}{2m} (\bvec p - q\bvec A)^2 + q\varphi
\end{equation}
其中 $\varphi$ 和 $\bvec A$ 分别是电磁场的标势和矢势,都是位置和时间的函数. $\bvec p$ 是广义动量,
\begin{equation}
\bvec p = m \bvec v + q\bvec A
\end{equation}
这个公式适用于任意规范. 将 ${\bvec p} = -\I\hbar\grad$, 代入得量子化的哈密顿算符为
\begin{equation}\label{QMEM_eq2}
\ali{
H &= \frac{\bvec p^2}{2m} - \frac{q}{2m} (\bvec A \vdot \bvec p + \bvec p \vdot \bvec A)
+ \frac{q^2}{2m} \bvec A^2 + q \varphi\\
&= -\frac{1}{2m} \laplacian + \I \frac{q}{2m} (\bvec A \vdot \Nabla + \Nabla \vdot \bvec A) + \frac{q^2}{2m} \bvec A^2 + q\varphi
}\end{equation}
注意其中 $\bvec p = -\I\Nabla$ 代表的是\textbf{广义动量}而不是 $m\bvec v$, 算符 $\Nabla \vdot \bvec A$ 是指先把波函数乘以矢势再取散度而不是直接对 $\bvec A$ 取散度(想想量子力学中算符相乘的定义).

=====================

如果我们对波函数也进行一个相位变换, 这个方程在标势和矢势的规范变换下形式不变

\begin{equation}
\Psi = \Psi' \exp(\I q\chi)
\end{equation}
其中 $\chi(\bvec r, t)$ 是一个任意可导函数. 将以上三式代入薛定谔方程, 只需要把不带撇的变量替换为带撇的变量.

常见的规范如长度规范和速度规范\upref{LVgaug}.

========================

如果对电磁场进行规范变换(\autoref{Gauge_eq3}~\upref{Gauge})
\begin{equation}
\bvec A = \bvec A' + \grad \chi
\qquad
\varphi = \varphi' - \pdv{\chi}{t}
\end{equation}

导致波函数发生相位变化
\begin{equation}\label{QMEM_eq3}
\Psi'(\bvec r, t) = \exp[\I q\chi(\bvec r, t)] \Psi(\bvec r, t)
\end{equation}
其中 $\chi$ 是\autoref{Gauge_eq3}~\upref{Gauge} 中的任意标量函数 $\lambda$.

在库仑规范下,\autoref{QMEM_eq2} 变为
\begin{equation}\label{QMEM_eq4}
H = -\frac{1}{2m} \laplacian + \I \frac{q}{m} \bvec A \vdot \Nabla + \frac{q^2}{2m} \bvec A^2 + q\varphi
\end{equation}
这是因为
\begin{equation}
\div (\bvec A \Psi) = (\div \bvec A) \Psi + \bvec A \vdot (\grad \Psi) = \bvec A \vdot (\grad \Psi)
\end{equation}

