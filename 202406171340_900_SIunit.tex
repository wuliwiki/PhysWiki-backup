% 国际单位制
% license Xiao
% type Tutor

% 物理常数|国际单位|测量
% 未完成: 只是写了基本单位的定义, 还要提及 “\enref{物理单位前缀}{UniPre}”

\footnote{参考 Wikipedia \href{https://en.wikipedia.org/wiki/International_System_of_Units}{相关页面}。}\textbf{国际单位制(SI Units)}是一套国际通用的单位标准。

本文采用 2019 年 5 月 20 日开始生效的新国际单位标准, 该标准中的数值也被称为 2018 CODATA 推荐值。 CODATA 是科学技术数据委员会(Committee on Data for Science and Technology)的简称。 在新标准中, 所有基本单位都可以通过 7 个预定义的物理常数来测定(\autoref{tab_Consts_1})。 以下的数值除了有特殊说明, 都是精确值(无限位小数用省略号表示), 不存在误差。

国际单位中有 7 个基本单位, 分别是秒(s), 米(m), 千克(kg),安培(A),开尔文(K),摩尔(mol), 坎德拉(cd)。 其他单位都是导出单位, 可以用基本单位表示。

\subsubsection{时间:秒(s)}
一秒等于铯(Cs)原子 133 基态的超精细能级之间的跃迁辐射的电磁波周期的 $9,192,631,770$ 倍。 

说明: 我们知道原子中的电子具有不同的能级, 当电子从一个能级跃迁到一个更低的能级时, 会放出一个光子。 光子的频率为 $\nu  = \varepsilon /h$,   其中 $\varepsilon $ 是光子的能量, $h$ 为普朗克常数。

\subsubsection{长度:米(m)}
真空中, 光在 $1/299792458$ 秒内传播的距离。

说明: 由于真空中的光速是物质和信息能传播的最快速度(见\enref{狭义相对论}{SpeRel}相关内容), 且在任何参考系中都相同, 所以可以作为一个精确的标准。 结合秒的定义, 就可以通过实验得到一米的长度。 根据米的定义, 一秒中光可以在真空中传播
\begin{equation}
c = 299792458 \Si{m/s}~.
\end{equation}

\subsubsection{质量:千克(kg)}
千克的定义需要使得普朗克常数精确等于 $h = 6.62607015\times10^{-34}\Si{Js}$。

说明: 2019 年 5 月开始, 千克根据普朗克常数定义(见\enref{量子力学}{QMIntr}相关内容)。 这个定义可以类比“米”的定义(使光速精确地等于 $299792459\Si{m/s}$): $h$ 的单位 $\Si{Js}$ 也可以表示为 $\Si{kg\cdot m^2/s}$, 我们已经定义了 “米” 和 “秒”, 所以通过测量普朗克常数, 我们就可以定义千克。

历史上, 千克最初在 1795 年被定义为一升水的质量, 但在实际操作中会遇到许多困难使结果不太精确。 1799 年使用国际公斤原器的质量来定义, 并复制若干份分别存放, 但经过长时间后被发现和复制品存在细微误差。

\subsubsection{力:牛顿(N)}
等效于 $\Si{kg \cdot m s^{-2}}$。 等于使 $1\Si{kg}$ 物体获得 $1\Si{m/s^2}$ 加速度的力。

说明: 该定义符合牛顿第二定律(\autoref{eq_New3_1})。

\subsubsection{压强:帕斯卡(Pa)}
使 $1\Si{m^2}$ 面积受力为 $1\Si{N}$ 的压强。

\subsubsection{能量:焦耳(J)}
等效于 $\Si{kg\cdot ms^{-2}}$。 $1\Si{N}$ 的恒力将受力物体沿力的方向移动 $1\Si{m}$, 做功为 $1\Si{J}$。

\subsubsection{电荷:库仑(C)}
定义每个电子的电荷精确等于 $1.602176634\e{-19} \Si{C}$。 $\Si{C} = \Si{A\cdot s}$。

\subsubsection{电势/电压:伏特(V)}
等效于 $\Si{J/C}$ 或者 $\Si{kg\cdot m^2 s^{-3} A^{-1}}$。 伏特的定义使得 $1\Si{C}$ 的电荷增加 $1\Si{V}$ 电势, 需要 $1\Si{J}$ 的能量。

\subsubsection{电场强度}
单位是 $\Si{V/m}$ 或 $\Si{N/C}$ 或 $\Si{m\cdot kg \cdot s^{-3} A^{-1}}$。

\subsubsection{电容:法拉(F)}
等效于 $C/V$ 或者 $\Si{s^4 m^{-2} kg^{-1} A^2}$。 法拉是电容量的单位, 一个 $1\Si{F}$ 的电容器两端施加 $1\Si{V}$ 电压, 可以储存 $1\Si{C}$ 净电荷。

\subsubsection{电容率}
单位 $\Si{F/m}$ 或者 $\Si{s^4 m^{-3} kg^{-1} A^2}$。

\subsubsection{电流:安培(A)}
流过导体某一横截面的电荷量与所用时间之比。等效于 $\Si{C \cdot s^{-1}}$。

\subsubsection{磁场:特斯拉(T)}
等效于 $\Si{kg\cdot A^{-1}s^{-2}}$。 可以由洛伦兹力(\autoref{eq_Lorenz_1})或安培力(\autoref{eq_FAmp_1})来定义。

\subsubsection{电感:亨利(H)}
等效于 $\Si{ s^{-2}m^2 kg \cdot A^{-2}}$。

用感生电动势来定义电感单位亨利:电流以每秒 $1$ 安培的节奏变化($1 \Si{A} \cdot \Si{s}^{-1}$),如果在电感上产生的感应电动势的电压是 $1 \Si{V}$,这种电感就是 $1 \Si{H}$。

\subsubsection{磁导率}
单位 $\Si{H/m}$ 或者 $\Si{s^{-2} m \cdot kg \cdot A^{-2}}$。

磁导率是某种磁介质中磁感应强度 $B$ 与磁场强度 $H$ 之比,一般用符号 $\mu$ 表示,即 $\mu=B/H$。

\subsubsection{温度:开尔文(K)}
开尔文温度的定义应该使得玻尔兹曼常数精确等于
\begin{equation}
k_B = 1.3806505\e{-23} \Si{J/K}~.
\end{equation}
例如, 理想气体中分子平均动能为 $\ev{E_k} = 3 k_B T/2$, 理论上 我们可以根据测其动能来定义温度。
