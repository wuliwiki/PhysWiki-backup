% 引力波
% license CCBYSA3
% type Wiki

(本文根据 CC-BY-SA 协议转载自原搜狗科学百科对英文维基百科的翻译)


引力波是时空曲率中的扰动,由加速的质量所产生,并以光速从源头向外传播。它们是由Henri Poincaré在1905年提出的[1],后来在1916年被爱因斯坦根据他的广义相对论所预测。[2][3][4] 引力波以引力辐射的形式传输能量,这是一种类似于电磁辐射的辐射能。[5] 作为经典力学的一部分,牛顿的万有引力定律并没有规定它们(引力波)的存在,因为该定律是基于物理相互作用瞬时(以无限大速度)传播的假设,这展示了经典物理学方法无法解释相对论现象的一个例子。

引力波天文学是观测天文学的一个分支,它使用引力波来收集关于可探测引力波来源的观测数据,例如由白矮星、中子星和黑洞组成的双星系统,和超新星等事件,以及大爆炸后不久的早期宇宙的形成。

1993年,Russell A. Hulse和Joseph H. Taylor, Jr.因为发现和观测赫尔斯-泰勒双星而获得诺贝尔物理学奖,这是引力波存在的第一个间接证据。[6]

2016年2月11日,激光干涉引力波天文台和处女座干涉仪科学合作组织宣布他们首次直接观测到引力波。这一观察发生在五个月前,即2015年9月14日,在这次观测中使用了增进LIGO探测器。这次事件的引力波起源于一对合并的黑洞。 在第一次探测到引力波的消息宣布之后,激光干涉引力波天文台的仪器又探测到了两次确认的引力波事件以及一次潜在的引力波事件。[7][8] 2017年8月,两个激光干涉引力波天文台的仪器和处女座干涉仪观测到第四次来自合并黑洞的引力波,[9] 和来自双星合并的第五次引力波。 另外几个引力波探测器正在计划或建造中。[10]

2017年,Rainer Weiss、Kip Thorne和Barry Barish因他们在引力波直接探测中所作的贡献而获得了诺贝尔物理学奖。

\begin{figure}[ht]
\centering
\includegraphics[width=6cm]{./figures/b96d7130530cad7b.png}
\caption{对两个黑洞的碰撞的模拟。除了形成深重力阱并合并成一个更大的黑洞外,当黑洞相互旋转时引力波将向外传播。} \label{fig_YLB_1}
\end{figure}

\subsection{介绍}

在爱因斯坦的广义相对论中,引力被视为时空曲率导致的一种现象。这种曲率是由质量的存在所引起的。一般来说,在一个给定体积的空间中包含的质量越大,其边界处的时空曲率就越大。[11] 当有质量的物体在时空中移动时,曲率会发生改变,反映出这些物体位置的变化。在某些情况下,加速的物体会引发曲率的变化,并以波的形式向外以光速传播。这些传播现象被称为引力波。

\begin{figure}[ht]
\centering
\includegraphics[width=6cm]{./figures/7857d71a3c37254e.png}
\caption{线性极化引力波} \label{fig_YLB_2}
\end{figure}

当引力波经过观察者时,观察者会发现时空被应变的影响扭曲了。物体之间的距离随着波的传播有节奏地增加和减少,频率等于波的频率。尽管这些自由物体从未受到不平衡的力的作用,这种情况还是会发生。这种效应的大小与其跟引力波源的距离成反比。[12] 由于它们的质量在它们彼此靠近的轨道上有非常大的加速度,互相螺旋靠近合并的双中子星被预测为引力波的强大来源。然而,由于我们与这些辐射源之间巨大的天文距离,在地球上测量到的影响预计非常小,应变不到$1/10^{20}$。科学家已经用越来越灵敏的探测器证明了这些波的存在。最灵敏的探测器完成了由LIGO和VIRGO天文台提供的$1/{5 \times 10^{22}}$截至2012年)的灵敏度测量任务。[13] 欧洲空间局目前正在开发一个名为激光干涉空间天线的空基天文台。

引力波可以穿透电磁波无法穿透的空间区域。它们能被用于观察黑洞合并以及遥远宇宙中可能存在的其他奇异物体的合并。这样的系统不能用更传统的方法来观察,比如光学望远镜或射电望远镜,因此引力波天文学为宇宙的运作提供了新的洞察。特别地,引力波可能是宇宙学家感兴趣的,因为它们提供了观察非常早期宇宙的可能的方法。这在传统天文学中是不可能的,因为在复合之前,宇宙对电磁辐射是不透明的。[14] 引力波的精确测量也将使科学家能够更彻底地测试广义相对论。

原则上,引力波可以以任何频率存在。然而,非常低频的引力波是不可能被检测到的,而且也没有非常高频的可探测引力波的可靠来源。Stephen Hawking和Werner Israel列出了引力波的不同频段,这些频段可能被探测到,范围从$ 10^{-7}$ 赫兹一直到$ 10^{11}$ 赫兹。

\subsection{历史}

\begin{figure}[ht]
\centering
\includegraphics[width=6cm]{./figures/47ac1b9d531d543c.png}
\caption{原初引力波被假设为来自于宇宙暴胀,后者是紧接着大爆炸的一种超光速膨胀。(2014)[1][2][3]} \label{fig_YLB_3}
\end{figure}

1893年,Oliver Heaviside用重力和电的平方反比定律之间的类比讨论了引力波的可能性。[15] 1905年,Henri Poincaré根据洛伦兹变换的要求,提出了从物体发出并以光速传播的引力波,[16] 并建议,类似于加速电荷产生电磁波,相对论性引力场论中的加速质量应产生引力波。[17][18] 当爱因斯坦在1915年发表他的广义相对论时,他对Poincaré的想法持怀疑态度,因为该理论暗示不存在“引力偶极子”。尽管如此,他仍然考虑了这个想法,并根据各种近似得出结论:事实上,引力波必须有三种类型(赫尔曼·外尔称之为纵向-纵向、横向-纵向和横向-横向)。[18]

然而,爱因斯坦所做的近似导致许多人(包括爱因斯坦本人)怀疑这个结果。1922年,Arthur Eddington证明了爱因斯坦的两种波是他所使用的坐标系的赝像,可以通过选择合适的坐标使得其任何速度传播,这导致爱丁顿开玩笑说它们“以思想的速度传播”。[19] 这也使得人们对第三种引力波类型(横向-横向)的物理实在性提出了质疑,Eddington证明了这种类型的引力波总是以光速传播,跟坐标系的选择无关。1936年,爱因斯坦和Nathan Rosen向《物理评论》提交了一篇论文,在这篇论文中他们声称引力波不可能存在于完整的广义相对论中,因为场方程的任何这样的解都有奇点。该杂志将他们的手稿送交Howard P. Robertson审阅,他匿名报告说,爱因斯坦和Rosen所讨论的奇点仅仅是源于他们所使用的圆柱坐标的无害的坐标奇点。不熟悉同行评审的概念的爱因斯坦愤怒地撤回了手稿,再也没有在《物理评论》上发表文章。尽管如此,他的助手Leopold Infeld(曾与Robertson联系过)还是让爱因斯坦相信了这一批评是正确的。这篇论文后来被用相反的结论重写了,并在其他地方发表。[18][19]

1956年,Felix Pirani通过用明显可见的黎曼曲率张量来重新表述引力波,纠正了使用各种坐标系造成的混乱。当时这项工作被大多数人忽视,因为科学共同体关注的是一个不同的问题:引力波是否能传输能量。这个问题是由Richard Feynman在1957年教堂山的第一次“GR”会议上提出的一个思想实验所解决的。简而言之,Feynman那个被称为“粘珠论证”的论证指出,如果一个人拿着一根带有珠子的棒,那么经过的引力波的作用将是沿着棒移动珠子;摩擦会产生热量,这就意味着经过的波做功了。不久之后,曾经的引力波怀疑者Hermann Bondi发表了“粘珠论证”的详细版本。[18]

教堂山会议后,Joseph Weber开始设计和建造第一个引力波探测器,现在它被称为韦伯棒。1969年,Weber声称已经探测到第一个引力波,到1970年,他已经“探测”到有规律地来自银心的信号;然而,探测的频率很快引起了人们对他的观测的有效性的怀疑,因为根据Weber的观测结果所推断的银河系能量损失率会在比推断的银河系年龄短得多的时间尺度上耗尽我们银河系的能量。到20世纪70年代中期,当其他研究组在世界各地建立自己的韦伯棒并进行反复实验却未能找到任何信号时,人们对Weber的观测结果的怀疑更强烈了。到了70年代后期,普遍的共识是Weber的结果是站不住脚的。[18]

在同一时期,引力波存在的第一个间接证据被发现了。1974年,Russell Alan Hulse和Joseph Hooton Taylor, Jr.发现了第一颗双星,这一发现为他们赢得了1993年的诺贝尔物理学奖。在接下来的十年里,脉冲星计时观测显示赫尔斯-泰勒脉冲星的轨道周期逐渐衰减,这与广义相对论所预测的引力辐射所带来的能量和角动量损失相吻合。[20][18]

尽管Weber的结果不可信,但这种引力波的间接探测激发了进一步的研究。一些团体继续改进Weber最初的概念,而另一些团体则使用激光干涉仪来探测引力波。使用激光干涉仪探测引力波的想法似乎是由不同的人独立提出的,包括1962年的M. E. Gertsenshtein和V. I. Pustovoit,[21] 以及1966年的Vladimir B. Braginskiĭ。Robert L. Forward和Rainer Weiss在20世纪70年代开发了第一批原型。[22][23] 在接下来的几十年里,越来越敏感的仪器被建造了出来,最终我们有了GEO600、激光干涉引力波天文台和处女座干涉仪。[18]

经过多年的零结果,改进后的探测器于2015年投入使用——激光干涉引力波天文台于2015年9月14日首次直接探测到引力波。据推断,这个被称为GW150914的信号起源于两个质量分别为 36+4−4M⊙ 和 29+4−4 M⊙的黑洞的合并,最终得到一个质量为 62+4−4 M⊙的黑洞。这表明引力波信号携带了大约3个太阳质量的能量,这大约是$ 5\times10^{47}$焦耳。[24][24][24]

一年前,当BICEP2声称他们在宇宙微波背景中探测到引力波的印记时,激光干涉引力波天文台似乎被击败了。然而,BICEP2后来被迫撤回了他们的结果。[24][25]

2017年,Rainer Weiss、Kip Thorne和Barry Barish因在引力波探测中的所作出的贡献而获得了诺贝尔物理学奖。[26][26][26]

\subsection{引力波经过时的影响}

\begin{figure}[ht]
\centering
\includegraphics[width=6cm]{./figures/a8f0ad2b79f7a7f9.png}
\caption \label{fig_YLB_4}
\end{figure}

引力波不断通过地球;然而,即使是最强的引力波也只有很小的影响,它们的波源通常距我们很远。例如,由GW150914事件中灾难性的最终合并所发出的引力波在经过十亿光年后到达地球,作为时空中的涟漪,它使得一个4公里长的激光干涉引力波天文台臂的长度改变了一个质子宽度的千分之一,这按比例相当于使我们到太阳系外最近的恒星的距离改变了一根头发的宽度。[26] 即使是极端引力波,它们的这种微小效应也只能被最精密的探测器在地球上探察到。

通过想象一个完全平坦的时空区域,在其中一组静止的测试粒子位于一个平面上(例如计算机屏幕的表面),我们可以看到一个极其夸张的引力波通过的效果。当引力波沿着垂直于粒子所在平面的直线穿过粒子时(即沿着观察者进入屏幕的视线),粒子将跟随时空扭曲以“十字形”的方式发生振荡,如动画所示。被测试粒子包围的区域面积不变,并且没有沿着引力波传播方向的运动。[来源请求]

出于讨论的目的,动画中描绘的振荡被夸大了。在现实中,引力波的振幅非常小(正如线性引力所描述的)。然而,它们有助于说明与在圆形轨道上运动的一对质量所产生的引力波相关的振荡。在这种情况下,引力波的振幅是恒定的,但是它的偏振面以两倍于轨道速度的速度变化或旋转,所以时变引力波的大小(或称“周期性时空应变”)如动画所示表现出变化。[27] 如果质量的轨道是椭圆形的,那么根据爱因斯坦的四极子公式,引力波的振幅也会随着时间而变化。[28]

与其他波一样,引力波由如下这些特征所描述:

\begin{itemize}
\item 
\end{itemize}


\subsection{来源}

\subsubsection{4.1 双星系统}

\textbf{致密双星系统}

\subsubsection{4.2 黑洞双星}

\subsubsection{4.3 超新星}

\subsubsection{4.4 自转的中子星}

\subsubsection{4.5 暴胀}

\subsection{性质和行为}

\subsubsection{5.1 能量、动量和角动量}

\subsubsection{5.2 红移}

\subsubsection{5.3 量子引力、波粒方面和引力子}

\subsubsection{5.4 对研究早期宇宙的意义}

\subsubsection{5.5 确定运动方向}

\subsection{引力波天文学}

\subsection{探测}

\subsubsection{7.1 间接探测}

\subsubsection{7.2 困难}

\subsubsection{7.3 地面探测器}

\textbf{共振天线}

\textbf{干涉仪}

\textbf{Einstein@Home}

\subsubsection{7.4 天基干涉仪}

\subsubsection{7.5 使用脉冲星计时阵列}

\subsubsection{7.6 原初引力波}

\subsubsection{7.7 激光干涉引力波天文台和处女座干涉仪的观测}

\subsection{8 在小说中}

\subsection{参考文献}