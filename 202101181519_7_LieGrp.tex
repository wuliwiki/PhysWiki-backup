% 李群
% keys Lie

\pentry{拓扑群\upref{TopGrp},流形\upref{Manif}}

\subsection{李群的概念}

当拓扑群的拓扑部分构成光滑流形时,所得到的拓扑群被称为李群.

\begin{definition}{实李群}
给定一个实光滑流形$G$,若在$G$上定义了一个群运算“$\cdot$”,且满足流形空间之间的映射$f:G\times G\rightarrow G$是一个光滑映射,其中$f(g_1, g_2)=g_1\cdot g_2^{-1}$,那么称$G$是一个\textbf{李群(Lie group)}.
\end{definition}

\begin{example}{实李群的例子}
\begin{enumerate}
\item \textbf{拓扑群}\upref{TopGrp}词条中\autoref{TopGrp_ex1}~\upref{TopGrp}所举的两个例子,实数轴$\mathbb{R}$和单位圆$S^1$都是实流行,都构成实李群.
\item \textbf{拓扑群}\upref{TopGrp}词条中\autoref{TopGrp_ex2}~\upref{TopGrp}所举的例子,一般线性群,是一个实李群.
\item 一般线性群$\opn{GL}(k, \mathbb{R})$的子群,正交群$\opn{O}(k, \mathbb{R})$,以及特殊正交群$\opn{O}(k, \mathbb{R})$\footnote{正交群指的是全体保度量线性变换构成的群,等价于全体行列式的绝对值是$1$的矩阵构成的乘法群.特殊正交群是正交群的子群,☞是由其中行列式为$+1$的矩阵构成的.}都是实李群.
\end{enumerate}
\end{example}

为了加深对李群的印象,我们再举出一个重要的反例:

\begin{definition}{环面\footnote{参见维基百科https://en.wikipedia.org/wiki/Lie_group.}}
记$\mathbb{T}=S^1\times S^1$为李群$S^1$的积李群,即其群乘法为$S^1$的群乘法的积,流形为$S^1$的流形的积.取$\mathbb{T}^2$的子群$H=\{(\E^{\theta\I}, \E^{a_0\theta\I})|\theta\in\mathbb{R}\}$,其中$a_0$是一个无理数.

按以上方式定义的群$H$是一个拓扑群,但不是流形,因而\textbf{不是李群}.这是因为无理数$a_0$导致$H$的图像是环面上密绕的一条线,在任何一个点的任何一个开邻域里都有无数条不连通的线段,因此任何点处都无法找到同构于欧几里得空间的\textbf{图}.

不过,如果
\end{definition}






\subsection{和李代数的联系}



