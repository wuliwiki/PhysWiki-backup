% 狄拉克场
% 狄拉克|场论|费米子

\begin{definition}{相对论性不变}
如果$\phi$是一个场或者是多个场,$\mathcal D$是微分算符.那么,我们说\textbf{$\mathcal D \phi = 0$是相对论不变的},就是说如果$\phi(x)$满足这个方程,我们再\textbf{对参考系}进行转动或者boost这样的操作,换到别的参考系,则变换后的场,在新的参考系下,满足同样的方程.我们也可以考虑\textbf{物理上}对所有的粒子或者场进行转动或者boost这样的操作,这时候方程$\mathcal\phi = 0$仍然保持不变.这种物理上对场进行操作的办法叫做\textbf{主动}的办法.
\end{definition}
用拉式量写出的场论让洛仑兹不变的讨论变得非常容易.如果一个理论的运动方程是从洛仑兹标量的拉式量推导出来的,那么这个理论的运动方程一定是自动洛仑兹不变的.

考虑如下的洛仑兹变换
\begin{equation}
x^\mu \rightarrow x'^\mu = \Lambda^\mu{}_\nu x^\nu ~. 
\end{equation}
在这边变换下,$\phi$的变换为
\begin{equation}\label{Dirac_eq1}
\phi(x)\rightarrow \phi'(x)=\phi(\Lambda^{-1}x)~.
\end{equation}
这个变换让克莱因-戈登场的拉式量保持不变.
\begin{equation}
\mathcal L(x)\rightarrow \mathcal L(\Lambda^{-1}x)
\end{equation}
运动方程同样是保持不变的
\begin{equation}
(\partial^2+m^2)\phi'(x) = 0 ~.
\end{equation}
\autoref{Dirac_eq1} 这样的变换规则是对于$\phi$这样的场最简单的变换规则.这是对于只有一个分量的场的唯一的一种可能.但是对于有多个分量的场来说,变换规则会更为复杂一些.我们以矢量场的变换来做个例子.

在三维旋转下,矢量场的变换规则为
\begin{equation}
V^i(x) \rightarrow R^{ij} V^j (R^{-1}x)~.
\end{equation}
洛仑兹变换下,矢量场的变换规则为
\begin{equation}
V^\mu(x) \rightarrow \Lambda^\mu{}_{\nu}V^\nu(\Lambda^{-1}x)~.
\end{equation}
任意阶的张量可以通过从矢量增加更多的指标来得到.每增加一个指标,我们就在变换规则前面多加一个$\Lambda$.用这些矢量场和张量场,我们可以写出一些洛仑兹不变的方程,比如说麦克斯韦方程
\begin{equation}
\partial^\mu F_{\mu\nu} = 0~,\quad \partial^2 A_\nu -\partial_\nu \partial^\mu A_\mu = 0~.  
\end{equation}
这些麦克斯韦方程可以从拉式量直接推出
\begin{equation}
\mathcal L_{\rm Maxwell} = - \frac{1}{4} (F_{\mu\nu})^2 = - \frac{1}{4} (\partial_\mu A_\nu - \partial_\nu A_\mu)^2~. 
\end{equation}
我们的目标是写出洛仑兹不变的方程.那么如何找到这些洛仑兹不变的方程呢?我们可以首先来研究一下场的变换规则.然后写出洛仑兹不变的拉式量就不难了.

首先我们可以考虑线性的变换.如果$\Phi_a$是一个$n$个组分的场.那么洛伦兹变换的规则可以由一个$n\times n$的矩阵$M(\Lambda)$来给定
\begin{equation}
\Phi_a (x) \rightarrow M_{ab} (\Lambda) \Phi_b(\Lambda^{-1} x)~.
\end{equation}
我们把指标去掉,写成下面这个更简单的形式
\begin{equation}
\Phi\rightarrow M(\Lambda)\Phi~.
\end{equation}
现在我们来考虑两个连续的$M(\Lambda)$作用在场$\Phi$上面.两个洛仑兹变换分别记为$\Lambda$和$\Lambda'$.最终的结果是一个新的洛伦兹变换$\Lambda''$.这也就是说,洛伦兹变换组成了一个\textbf{群}.对于$\Lambda''=\Lambda'\Lambda$来说,我们有
\begin{equation}
\Phi\rightarrow M(\Lambda')M(\Lambda)\Phi = M(\Lambda'')\Phi~.
\end{equation}
矩阵$M$和变换$\Lambda$的乘法关系必须是一一对应的.用数学语言来说,我们就说矩阵$M$是洛伦兹群的$n$-维表示.

那么如何找到洛伦兹群的有限维的矩阵表示呢?

回答这个问题之前,我们来回顾一下三维空间里的转动群.这个群具有任何维度的表示.维度跟自旋的关系为$n=2s+1$.最重要的非平凡表示是二维的表示,对应于自旋$1/2$.这个表示的矩阵是$2\times 2$的行列式为1的幺正矩阵,表达式如下
\begin{equation}
U = e^{-i\theta^i\sigma^i/2}~.
\end{equation} 
其中$\theta^i$是三个任意的参数,$\sigma^i$是泡利$\sigma$矩阵.

对于任意的连续群,无限接近于恒等变换的变换定义了一个矢量空间,称作一个群的李代数.这个矢量空间的基矢量,被称作这个群的生成元,或者说是这个群的李代数.对于转动群来说,角动量算符$J^i$的生成元满足如下的对易关系
\begin{equation}\label{Dirac_eq2}
[J^i,J^j] = i \epsilon^{ijk} J^k ~.
\end{equation}

有限的转动操作可以通过对这些操作取指数得到,在量子力学里面,算符
\begin{equation}
R = \exp [-i \theta^i J^i]~.
\end{equation}
给定了对于轴$\hat\theta$转动$|\theta|$角的算符.算符$J^i$的对易关系决定了这些转动操作的乘法规则.因此,满足对易关系\autoref{Dirac_eq2} 的一系列矩阵,通过指数化的办法,生成了转动群的一个表示.在之前的一个例子里面,角动量算符的表示
\begin{equation}
J^i \rightarrow \frac{\sigma^i}{2}~.
\end{equation}
生成了转动群的表示.一般来说,我们可以通过找一个群的生成元的矩阵表示然后对这些无穷小的变换进行指数化来找到一个连续群的矩阵表示.

接下来我们考虑洛伦兹变换群的生成元的对易关系.角动量算符的表达式如下
\begin{equation}
J^{ij} = -i (x^i\nabla^j- x^j\nabla^i)~.
\end{equation}
定义$J^3 = J^{12}$.我们可以把它推广到四维的洛仑兹变换的生成元
\begin{equation}
J^{\mu\nu} = i (x^\mu \partial^\nu - x^\nu \partial^\nu)~.
\end{equation}
这个微分算符的对易关系为
\begin{equation}\label{Dirac_eq4}
[J^{\mu\nu},J^{\rho\sigma}] = i (g^{\nu\rho}J^{\mu\sigma} - g^{\mu\rho}J^{\nu\sigma}-g^{\nu\sigma}J^{\mu\rho}+g^{\mu\sigma}J^{\nu\rho})~.
\end{equation}
现在我们考虑如下的$4\times 4$矩阵
\begin{equation}
(\mathcal J^{\mu\nu})_{\alpha\beta} = i (\delta^\mu{}_\alpha \delta^\nu{}_\beta - \delta^\mu{}_\beta \delta^\nu{}_\alpha)~.
\end{equation}
我们可以把无穷小变换按照下面的办法参数化
\begin{equation}
V^\alpha \rightarrow (\delta^\alpha{}_\beta - \frac{i}{2}\omega_{\mu\nu}(\mathcal J^{\mu\nu})^\alpha{}_\beta) V^\beta ~.
\end{equation}
其中$V$是一个四矢量,$\omega_{\mu\nu}$是一个反对称的张量,给出了无穷小的角.例如我们可以取定$\omega_{12}=-\omega_{21} = \theta$其他的分量都置成零.我们有
\begin{equation}
V \rightarrow \begin{pmatrix}
1 & 0 & 0 & 0 \\
0 & 1 & -\theta & 0 \\
0 & \theta & 1 & 0 \\
0 & 0 & 0 & 1
\end{pmatrix}
V
\end{equation}
取$\omega_{01} = -\omega_{10}=\beta$可以得出
\begin{equation}
V \rightarrow \begin{pmatrix}
1 & \beta & 0 & 0 \\
\beta & 1 & 0 & 0 \\
0 & 0 & 1 & 0 \\
0 & 0 & 0 & 1
\end{pmatrix}
V
\end{equation}
这是一个$x$-方向上的无限小boost.

\subsubsection{狄拉克方程}
接下来我们来找对应于自旋$1/2$的洛伦兹群的表示.首先考虑一系列$n\times n$的矩阵$\gamma^\mu$.这些矩阵满足如下的反对易关系
\begin{equation}\label{Dirac_eq3}
\{\gamma^\mu,\gamma^\nu\}\equiv\gamma^\mu\gamma^\nu+\gamma^\nu\gamma^\mu = 2 g^{\mu\nu} \times \boldsymbol{1}_{n\times n}~.
\end{equation}
我们可以马上写下洛伦兹代数的$n$-维表示
\begin{equation}
S^{\mu\nu} = \frac{i}{4}[\gamma^\mu,\gamma^\nu]~.
\end{equation}
重复使用\autoref{Dirac_eq3} 我们可以验证这些矩阵满足\autoref{Dirac_eq4} 这样的反对易关系.这个计算在任意的维度都成立.在洛伦兹或者欧式度规也都成立.特别地,它在三维欧氏空间




