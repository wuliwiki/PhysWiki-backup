% Mathematica文件操作
% Mathematica|IO|文件操作

\begin{issues}
\issueDraft
\issueTODO
\end{issues}

\subsection{常用环境变量}

除了丰富的标准文件操作外,Wolfram 语言的统一符号架构,
让我们更容易将算法和高级编程应用于许多文件和系统管理任务.此介绍主要参考官方页面:

\begin{itemize}
\item guide/FileOperations, 比较全的文件系统操作的函数列表
\item tutorial/FilesStreamsAndExternalOperations#12068
\end{itemize}

关键是:不要直接使用字符串函数硬编码 文件名/文件路径, 
这样的到的路径依赖于操作系统, 应该使用 Mathematica 提供的文件系统接口.

\begin{lstlisting}[language=mathematica]
$OperatingSystem : 给出正在运行的操作系统的名称.
$PathnameSeparator : 字符串,在构建路径名的时候使用. Windows 的默认值是 "\\", 其他系统是 "/".
在 Windows 中, 像 FileNameSplit 这样的函数默认同时允许 \ 和 /.
\end{lstlisting}

文件名使用惯例:

\begin{lstlisting}[language=mathematica]
name.m  : Wolfram 语言源文件
name.nb : Wolfram 系统笔记本文件
name.ma : Wolfram 系统从第3版以前的笔记本文件
name.mx : 输出所有 Wolfram 语言表达式
name.exe: WSTP 可执行程序
name.tm : WSTP 模版文件
name.ml : WSTP 流文件
\end{lstlisting}


\verb`$Path`: 默认的目录列表, 用于搜索输入文件的相关目录. 
一般来说, 全局变量 \verb`$Path` 被定义为一个字符串的列表, 每个字符串代表一个目录.
每次你要求打开文件时, Wolfram 就暂时将这些目录中的依次变成你的当前工作目录,然后从该目录中尝试找到你要求的文件.
在 \verb`$Path` 的典型设置中, 当前目录 \verb`.` 和你的主目录 \verb`~` 被列在第一位.

\begin{lstlisting}[language=mathematica]
+ DirectoryName["name",n] : 给出路径的父目录, n 代表上升 n 次. 
默认情形给出父目录, 不用写 n. 可作用于文件和目录, 不检查目录是否真实存在.
+ DirectoryName[..., OperatingSystem->"os"] 用来给出某种操作系统风格的路径, 
选项有 "Windows", "MacOSX", 和 "Unix".
+ ParentDirectory["dir",n] :给出路径的父目录, n代表上升 n 次, 
只能作用于目录, 且要求目录真实存在.
\end{lstlisting}

\begin{lstlisting}[language=mathematica]
+ $InitialDirectory :  是 Wolfram 系统启动时的初始目录.
+ $HomeDirectory :  你的主目录, 如果被定义过的话
+ $BaseDirectory :  是 Wolfram 系统要加载的全系统文件的基本目录.
+ $UserBaseDirectory :  用于 Wolfram 系统加载的用户特定文件的基本目录
+ $InstallationDirectory :  你的 Wolfram 系统安装的最高级别目录
\end{lstlisting}

Wolfram 系统所使用的绝大多数文件都与操作系统无关. 然而, \verb`mx` 和 \verb`.exe` 文件与系统有关.
对于这些文件, 按照惯例, 对不同计算机系统版本的名称进行捆绑, 形式如 \verb`name/$SystemID/name`.

\subsection{笔记本}

\begin{itemize}
\item \verb`NotebookFileName[]` : 给出当前笔记本的完整路径.
\item \verb`NotebookDirectory[]`: 笔记本父目录
\end{itemize}


\begin{itemize}
\item \verb`NotebookOpen["name"]`:  打开已经存在的笔记本"name", 返回笔记本对象. "name"可以是绝对路径.
\item \verb`NotebookOpen["name",options]`: 使用指定的选项打开笔记本.
\item \verb`NotebookOpen[File["path"]]` 和 \verb`NotebookOpen[URL["url"]]` 也被支持.
\item \verb`NotebookOpen` 通常会导致一个新的笔记本窗口在你的屏幕上被打开.
如果 \verb`NotebookOpen` 打开指定的文件失败, 则返回 \verb`$Failed`.
若给出相对路径, \verb`NotebookOpen` 搜索由前端的全局选项 \verb`NotebookPath` 指定的目录
若使用选项 \verb`Visible->False` 设置, \verb`NotebookOpen` 将打开带有此选项的笔记本,它永远不会显示在屏幕上.
\item \verb`NotebookOpen` 将当前 \verb`selection` 初始化设置在笔记本的第一行单元之前.

\end{itemize}



NotebookSave[notebook]: 保存特定笔记本的当前版本.
    notebook必须是一个NotebookObject.
    NotebookSave[notebook] 将笔记本保存在一个文件中, 文件名由笔记本对象 notebook 给出.
    NotebookSave 写入对应的 Wolfram 语言表达式, 以及 Wolfram 语言注释, 以便于前端再次读入笔记本.
    NotebookSave[notebook, "file"], 如果"file"存在, 则不加警告地覆盖它.
    NotebookSave[notebook,File["file"]]也被支持.
    如果给定选项 Interactive->True, 前端将提示用户为笔记本选择一个文件名.

NotebookClose[notebook]: 关闭指定的笔记本对象.
NotebookClose[] : 关闭当前在运行的笔记本.
    NotebookClose将使笔记本从你的屏幕上消失, 并将使所有引用该笔记本的笔记本对象失效.
    如果给定了选项设置Interactive->True, 前端将提示用户是否关闭笔记本而不保存.

\subsection{操作文件和目录}

\begin{itemize}
\item tutorial/FilesStreamsAndExternalOperations#12068
\item Manipulating Files and Directories
\end{itemize}

\begin{itemize}
\item ExpandFileName["name"] : 将"name"文件展开成当前系统规范的绝对路径, 给出相对于你当前目录的名称.
它展开通常的目录指定, 如 \verb`.` 和 \verb`..`;它只是对文件名进行操作:它并不实际搜索指定的文件.
它支持 \verb`ExpandFileName[File["name"]]`, 以及 \verb`ExpandFileName[URL["file:///path"]]`, 
后者将基于文件的 \verb`URL` 转换为绝对文件名.
\item \verb`AbsoluteFileName["name"]`: 给出 \verb`"name"` 文件的绝对路径. 
与 \verb`ExpandFileName的区别是, 它会进入文件系统, 检查文件是否真实存在.
同样相对于你当前目录的名称, 可以处理目录指定, 如., ..和 ~.
它也支持 AbsoluteFileName[File["name"]].
\end{itemize}

\begin{itemize}
\item FileNameTake["name"] : 从"name"的完整路径中提取出最后的文件名.
\item FileBaseName["file"]: 给出文件的 basename, 也就是不包括拓展名.
\item FileExtension["file"]  : 给出文件的拓展名.
\item FileNameDepth["name"] : 给出文件路径的深度, 文件不必真实存在.
\end{itemize}



\begin{itemize}
\item FileNameJoin : 从路径列表中组合出完整的文件名
\item FileNameSplit : 将文件的完整路径分割开
\item FileNameDrop["name",n] : 去掉文件"name"路径的前n个片段. 如果是-n, 那么去掉从末尾开始的n个.
\item FileExistsQ["name"]  : 检查文件, 目录等等是否存在.
\item ContextToFileName["context"]  : 给出 Mathematica 上下文规范对应的文件名.
\end{itemize}



