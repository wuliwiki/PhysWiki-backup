% 叉乘的矩阵形式
% 叉乘|矢量积|矩阵|反对称矩阵

\pentry{矢量叉乘\upref{Cross}}
对任意矢量 $\bvec a$ 和 $\bvec b$, 令
\begin{equation}\label{CrosMt_eq1}
\bvec c = \bvec a \cross \bvec b
\end{equation}
该运算可以看作列矢量 $\bvec b$ 到列矢量 $\bvec c$ 的线性变换. 我们知道线性变换可以用矩阵表示, 所以必存在矩阵 $\mat A$, 满足
\begin{equation}\label{CrosMt_eq3}
\bvec c = \mat A \bvec b
\end{equation}
令 $\bvec a$ 的坐标为 $(a_x, a_y, a_z)$, 根据叉乘的分量表达式(\autoref{Cross_eq8}~\upref{Cross}), 易得变换矩阵为
\begin{equation}\label{CrosMt_eq2}
\mat A = \pmat{
0 & -a_z & a_y\\
a_z & 0 & -a_x\\
-a_y & a_x & 0
}
\end{equation}
这是一个\textbf{反对称矩阵}, 即 $A_{ij} = -A_{ji}$.

同理, \autoref{CrosMt_eq1} 也可以看作是 $\bvec a$ 到 $\bvec c$ 的线性变换
\begin{equation}
\bvec c = \mat B \bvec a
\end{equation}
其中
\begin{equation}
\mat B = \pmat{
0 & b_z & -b_y\\
-b_z & 0 & b_x\\
b_y & -b_x & 0
}
\end{equation}
这恰好与\autoref{CrosMt_eq2} 符号相反.
