% 本构关系;广义胡克定律(弹性力学)

\begin{issues}
\issueDraft
\end{issues}
%工程量似乎有点巨大。。。先占位
\footnote{本文参考了冯西桥的《弹性力学》。}
\pentry{杨氏模量、泊松比、剪切模量、广义胡克定律的基本形式\upref{YoungM}}

\textsl{小学二年级的我们}就已经知道"力使物体变形"。之前我们分别探讨了应力\upref{STRESS}与应变\upref{Strain},现在我们要找到二者间的关系。这就是材料的本构关系。

以下是各向同性的线弹性材料的本构关系。本构关系一共包括$6$个独立的方程以联系$6$个独立的应力-应变分量。为了构建本构关系,需要知道材料的两个力学属性,例如杨氏模量$E$与泊松比$\nu$. 举个例子,铁的杨氏模量约为$E = 200 \Si{GPa} = 2 \times 10^{11} \Si{Pa}$\footnote{防杠声明:铁的模量具体数值和成分、加工工艺、工作条件甚至是加工批次都有关,因此不能一概而论;此外,实际的材料往往是各向异性的。},泊松比约为$\nu=0.3$(这也是常见金属的泊松比)。

\subsection{应力-应变本构关系}
\begin{equation}\label{ELACST_eq1}
\sigma_{ij}=\frac{E}{1+\nu}\varepsilon_{ij}+\delta_{ij}\frac{\nu E}{(1+\nu)(1-2\nu)}\sum_k\varepsilon_{kk}\qquad i,j=1,2,3
\end{equation}
或
\begin{equation}\label{ELACST_eq2}
\varepsilon_{ij}=\frac{1+\nu}{E}\sigma_{ij}-\delta_{ij}\frac{\nu}{E}\sum_k\sigma_{kk} \qquad i,j=1,2,3
\end{equation}

其中$\delta_{ij} = \left \{
\begin{aligned}
1 &\qquad i = j\\
0 &\qquad i \ne j\\
\end{aligned}
\right.
$

\subsection{Lame常数}
若定义Lame常数
\begin{equation}
\begin{aligned}
G&=\frac{E}{2(1+\nu)}\\
\lambda &= \frac{\nu E}{(1+\nu)(1-2\nu)}\\
\end{aligned}
\end{equation}
则\autoref{ELACST_eq1} , \autoref{ELACST_eq2} 分别化为
\begin{equation}
\sigma_{ij}=2G\varepsilon_{ij}+\delta_{ij}\lambda\sum_k\varepsilon_{kk}\qquad i,j=1,2,3
\end{equation}
与
\begin{equation}
\varepsilon_{ij}=\frac{1}{2G}\sigma_{ij}-\delta_{ij}\frac{\nu}{E}\sum_k\sigma_{kk} \qquad i,j=1,2,3
\end{equation}

\subsection{应力-位移本构关系}
如果往本构关系 \autoref{ELACST_eq1} 中代入位移几何方程\upref{Strain},那么本构关系化为
\begin{equation}
\sigma_{ij}=\frac{E}{1+\nu}\left(\pdv{u_i}{x_j}+\pdv{u_j}{x_i}\right)
+\delta_{ij}\frac{\nu E}{(1+\nu)(1-2\nu)}\sum_k\pdv{u_k}{x_k}\qquad i,j=1,2,3
\end{equation}

\subsection{简单的推导}
以$\varepsilon_11$为例。根据单向拉伸的广义胡克定律\upref{YoungM}
