% 库仑势垒
% license CCBYSA3
% type Wiki

(本文根据 CC-BY-SA 协议转载自原搜狗科学百科对英文维基百科的翻译)

\textbf{库伦势垒}是由于静电相互作用而产生的能量势垒,两个原子核需要克服静电相互作用才能靠得足够近以进行核反应。\textbf{库伦势垒}以库仑定律命名,库仑定律又以物理学家查尔斯-奥古斯丁·德·库仑(Charles-Augustin de Coulomb)命名。

\subsection{能量势垒}
这个能垒由静电势能给出:
$$U_{\text{coul}} = k \frac{q_1 q_2}{r} = \frac{1}{4\pi\epsilon_0} \frac{q_1 q_2}{r}~$$

其中

$k\text{是库仑常数}= 8.9876\times10^9 N m^2 C^{-2}$;

$\epsilon_0$ 是自由空间的介电常数;

$q_1, q_2$是相互作用粒子的电荷;

$r$ 是相互作用半径。

正值表示排斥力,所以相互作用的粒子越靠近能量水平越高。负势能表示束缚态(由于吸引力)。

库仑势垒随着碰撞原子核的原子序数(即质子数)而增加:
\begin{equation}
U_{\text{coul}} = \frac{k Z_1 Z_2 e^2}{r}~
\end{equation}

其中 $e$ 是基本电荷, $1.602 \\ 176 \\ 53 \\times 10^{-19}C$, $Z_i $是相应的原子序数。

为了克服这个障碍,原子核必须以高速碰撞,从而它们的动能驱使它们足够靠近,以便发生强相互作用,并将它们结合在一起。

根据气体动力学理论,气体的温度不过是气体中粒子平均动能的量度。对于经典的理想气体,气体粒子的速度分布由麦克斯韦-玻尔兹曼分布给出。根据这种分布,可以确定速度足以克服库仑势垒的粒子比例。

实际上,由于量子力学隧穿效应,克服库仑势垒所需的温度比预期的要低,伽莫夫已经证实了这一点。考虑隧穿效用穿越障碍物和速度分布,会得到发生核聚变的限制范围,称为Gamow窗口。

詹姆斯·查德威克在1932年因观察到不存在库伦势垒,而发现了中子。[1][2]

\subsection{参考文献}
[1]
^Chadwick, James (1932). "Possible existence of a neutron". Nature. 129 (3252): 312. Bibcode:1932Natur.129Q.312C. doi:10.1038/129312a0..

[2]
^Chadwick, James (1932). "The existence of a neutron". Proceedings of the Royal Society of London A: Mathematical, Physical and Engineering Sciences. 136 (830): 692–708. Bibcode:1932RSPSA.136..692C. doi:10.1098/rspa.1932.0112..