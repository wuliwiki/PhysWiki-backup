% 辛流形
% 哈密顿流|Hamiltonian flow|Symplectic Manifold|辛形式|哈密顿正则方程|理论力学|分析力学|经典力学|classical mechanics|流形


\subsection{引入的动机}

辛流形是经典力学研究中自然出现的几何结构.

要描述质点组成的系统所处的状态,只需要知道每个质点的速度和位置,或者动量和位置.自由质点组成的系统,其所有可能的状态构成的\textbf{相空间(phase space)}是一个欧几里得空间.具体地,$M$维空间中的$N$个自由质点构成的系统,其相空间是$M^N$维\textbf{欧几里得空间}.

但是如果存在约束,那么系统的相空间一般来说不是欧几里得空间.比如说,一个$3$维空间中的自由质点,其相空间就是$\mathbb{R}^3\times \mathbb{R}^3$,即描述其位置需要一个$\mathbb{R}^3$、描述动量又需要一个$\mathbb{R}^3$,合起来就是一个$6$维欧几里得空间$\mathbb{R}^6$.但如果此质点被固定在轻质杆的一端,而杆的另一端在某惯性系中固定不动,那么其相空间就是$S^2\times \mathbb{R}^2$,或者更准确地,$T^* S^2$.$S^2$描述的是质点的位置,这是一个\textbf{流形};而$\mathbb{R}^2$则是$S^2$的\textbf{余切空间}.

一般地,系统的相空间是一个$2N$维流形,此流形是一个$N$维流形上的\textbf{余切丛}.

有了描述状态的相空间,我们还需要描述系统的相点运动规律,那就是牛顿三定律,我们写为哈密顿正则方程的形式:

\begin{equation}\label{SplcMn_eq1}
\left\{
\begin{aligned}
\frac{\dd q_i}{\dd t} &= \frac{\partial H}{\partial p_i}\\
\frac{\dd p_i}{\dd t} &= -\frac{\partial H}{\partial q_i}
\end{aligned}
\right. 
\end{equation}

由\autoref{SplcMn_eq1} 可见,决定系统状态随时间变化的是哈密顿函数,它是定义在相空间上的一个函数.以相空间上各点为起点,系统状态随时间的变化构成一族曲线,相空间上每一点处都由过这点的曲线定义了一个切向量.综上,哈密顿函数定义了一个相空间上的切向量场,称为该函数决定的\textbf{哈密顿向量场(Hamiltonian flow)}.那么函数如何与一个切向量场联系起来呢?

给定流形上的可微函数$f$,我们总能找到与之关联的一个余切向量场$\dd f$,定义为:对于任意切向量场$X$,都有$Xf = (X, \dd f)$,这里$(X, \dd f)$表示切场与余切场之间的作用.一般的流形是无法把函数和切场联系起来的,本质上是因为对偶空间之间没有自然同构.

要建立对偶空间之间的自然同构,只需要给这对对偶空间定义一个度量即可.在线性空间$V$上定义度量$g$后,就可以把$V^*$中的$\phi$对应到$V$中的$\bvec{v}$,方式是:$\phi(\bvec{u})=g(\bvec{v}, \bvec{u})$对任意$\bvec{u}\in V$成立.这个自然同构,就是所谓的\textbf{指标升降}.用指标表示法,主空间$V$中的向量表示为$v^i$和$u^i$,度量就表示为$g_{ij}$,于是内积表示为$g(\bvec{v}, \bvec{u})=g_{ij}v^iu^j$,所以用内积定义的$v^i$的对应,就是$v_i=v^jg_{ij}$,于是$v_iu^i=g(\bvec{v}, \bvec{u})$,所以$v_i$就是上面说的对偶向量$\phi$.





























