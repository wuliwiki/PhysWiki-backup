% 北京大学 2012 年 考研 普通物理
% license Usr
% type Note

\textbf{声明}:“该内容来源于网络公开资料,不保证真实性,如有侵权请联系管理员”

\subsection{选择题}
\begin{enumerate}
\item 有一组描述电磁场运动的著名方程,叫做()
A.Gibbs方程B.XX方程CXXX方程DMaxwe11方程
\item 有一组由状态函数的微分导出的重要关系,叫做()
A.Boltzmann关系B.XX关系CXXX关系DMaxwe1l关系3.有一个数学很好的英国人,擅长计算土星环,叫做()
A.FaradayB.XXC.XXXD.Maxwell(好吧,这道题是我脑补的)
\item 电偶极辐射与频率的几次方成正比?(原题是选择的形式,下同)
\item 类空间隔的事件的时序能否颠倒?类时间隔的事件的时序能否颠倒?
\item 匀速运动的带电粒子是否有对外辐射?
\end{enumerate}
\subsection{大题:}

1.轴沿Z向的均匀磁化圆柱,磁化强度为常量,指向X方向。求柱内外各点的磁标势,H,B.

2.z轴上分布有均匀电荷,线密度已知。电荷以速度v相对S系向Z轴正向运动。
(1)在随电荷一起运动的S系中,求出空间各点的E,B.

(2)(E,B的变换关系已给出)求S系中,空间各点的EB.

(3)分别在S和S`系中,在某点放置一磁针,它会不会偏转?造成这种差别的原因是什
么?

3.已知正负电子湮灭生成光子:e(-)+e(+)=2v

(1)反应平衡时,三者的化学势有怎样的关系?光子化学势为零,正负电子的化学势又有怎样的关系?

(2)已知止负电子数相等,求二者的化学势。

(3)求三者对体系热容的贡献。

4.长为乚的一维空间中,分布有N个刚性“分子”,每个分子长为a(就是一个长为a的线段)。分子不可交换,从左到右依次编为1,2,…,N。

L>>Na。相邻分子间有相互作用势U(x1-x2),当x1-x2>a时U=0;当x1-x2<a 时U-〉正无穷。x1,x2分别是相邻两分子的左端点的坐标。

(1)(第二维里系数B2的计算公式已给出)求$B2$.

(2)严格地计算位形积分$Q_n$($Qn$的积分形式已给出)。据此求出状态方程,验证(1)的结论

(3)(配分函数已给出)求体系的熵,证明它是一个广延量。

