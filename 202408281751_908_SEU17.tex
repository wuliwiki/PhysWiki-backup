% 东南大学 2017 年 考研 量子力学
% license Usr
% type Note

\textbf{声明}:“该内容来源于网络公开资料,不保证真实性,如有侵权请联系管理员”

\subsection{1}
\subsection{(共 30分,每小题3分)选样题}
\begin{enumerate}
    \item[(1)] $\{x_i, p_j\}$ 的共同本征函数为 
    \begin{itemize}
        \item[(a)] $\delta(x-a)\delta(y-b)$
        \item[(b)] $\delta(x-a)\delta(y-b)$
        \item[(c)] $\delta(x-a)\delta(p-b)$
        \item[(d)] $\delta(p-a)\delta(p-b)$
    \end{itemize}
    
    \item[(2)] 无自旋单粒子在 $xy$ 平面内运动,力学量完全集可选为 
    \begin{itemize}
        \item[(a)] $\{p_x, p_y\}$
        \item[(b)] $\{r_x, r_y\}$
        \item[(c)] $\{r_x, r_y, p_x, p_y\}$
        \item[(d)] $\{L_z, p_x, p_y\}$
    \end{itemize}

    \item[(3)] $A$ 和 $\beta$ 均为线性算符,$[\alpha, \beta]$ 可为 
    \begin{itemize}
        \item[(a)] $A\beta + \beta A$
        \item[(b)] $A\beta$
        \item[(c)] $\beta A + \beta^2$
        \item[(d)] $A\beta - \beta A$
    \end{itemize}

    \item[(4)] 设 $\sigma = |\psi\rangle \langle\phi|$,则 $\sigma$ 可为 
    \begin{itemize}
        \item[(a)] $|\phi\rangle \langle\psi|$
        \item[(b)] $-\langle\phi|\psi\rangle$
        \item[(c)] $\langle\phi|\psi\rangle$
        \item[(d)] $|\psi\rangle \langle\phi|$
    \end{itemize}
\end{enumerate}
