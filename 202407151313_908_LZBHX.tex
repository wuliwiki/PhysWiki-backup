% 量子不和谐
% license CCBYSA3
% type Wiki

(本文根据 CC-BY-SA 协议转载自原搜狗科学百科对英文维基百科的翻译)

在 量子信息论中, \textbf{量子不和谐} 是对一个量子系统中的两个子系统之间非经典相关性的度量 。它包括了由于 量子物理学 效应而产生的关联性,但不一定涉及 量子纠缠。

量子不和谐的概念是由Hardord OLLivier, Wojciech H. Zurek[1][2] 亨德森和 Vlatko Vedral各自提出的。[3] 奥利维尔(OLLivier)和祖里克(Zurek)也将其作为 定量 相关性的量化指标。[2] 从这两个研究小组的工作中可以看出,量子关联可以以某种混合 可分离状态下存在;[4] 换句话说,单独的可分离性并不意味着没有量子关联。因此,量子不和谐的概念超越了早先在纠缠态和可分离(非纠缠)量子态之间所做的区分。

\subsection{定义和数学关系}
\begin{figure}[ht]
\centering
\includegraphics[width=8cm]{./figures/c1e7741235199510.png}
\caption{单个H(X),H(Y),相关的H(X,Y)及具有互信息I(X,Y)的一对相关子系统X,Y的条件熵} \label{fig_LZBHX_1}
\end{figure}
在数学术语中,量子不和谐是由 量子互信息定义的。更具体地说,量子不和谐是两个表达式之间的差异,在 经典极限中,每个表达式都表达 交互信息。这两个表达式是:
\begin{equation}
\begin{aligned}
I(A; B) &= H(A) + H(B) - H(A, B) \\
J(A; B) &= H(A) - H(A \mid B) \\
J(A; B) &= H(A) - H(A \mid B)
\end{aligned}~
\end{equation}
