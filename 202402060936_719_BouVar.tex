% 有界变差
% keys 实分析|数学分析|黎曼-斯蒂尔杰斯积分
% license Usr
% type Wiki

有界变差(Bounded Variation)函数的总变差是有限的。这一概念是后面学习黎曼-斯蒂尔杰斯积分的关键基础。有界变差是描述实数轴上函数变化幅度的一种度量方式。

考虑一个实值函数$f(x)$,在闭区间$a\leq x\leq b$上定义并有限。我们将这个区间分成一些小区间,形成一个分割
\begin{align} 
\Gamma=\{x_{0},\,x_{1},\,\ldots,\,x_{m}\} ~.
\end{align},也就是$\Gamma$是点$x_{i}$的集合,满足$x_{0}=a$,$x_{m}=b$,且$x_{i-1}<x_{i}$。

对于每个分割$\Gamma$,我们计算一种和$S_{\Gamma}$,表示相邻点$f(x_{i})$和$f(x_{i-1})$的绝对差的总和。也就是,
\begin{align} 
S_{\Gamma}=S_{\Gamma}[f;a,b]=\sum_{i=1}^{m}|f(x_{i})-f(x_{i-1})|~.
\end{align}
函数$f$在$[a,b]$上的变差定义为
\begin{align} 
V=V[f;a,b]=\sup_{\Gamma}S_{\Gamma}~, 
\end{align}
其中$\sup$取遍$[a,b]$的所有分割$\Gamma$。由于$0\leq S_{\Gamma}<+\infty$,我们有$0\leq V\leq+\infty$。如果$V$有限,那么$f$在$[a,b]$上的变差有界;如果$V$是无穷大,那么$f$在$[a,b]$上的变差无界。

下面列举几个简单有界变差函数的例子:

\textbf{例子 1}:假设$f$在$[a,b]$上单调。那么,显然,每个$S_{\tau}$都等于$|f(b)-f(a)|$,因此$V=|f(b)-f(a)|$。

\textsl{证明}:设${x_{i}:1\leq i\leq n}$是$[a,b]$的一个分割。考虑

\begin{align} \sum_{i=1}^{n}|f(x_{i})-f(x_{i-1})|=\sum_{i=1}^{n}\left(f(x_{i})-f(x_{i-1})\right)=f(b)-f(a)~. \end{align}

由于这个和的抵消性质,它对$[a,b]$的任何分割都是相同的。因此我们可以看到$V(f,[a,b])=f(b)-f(a)<\infty$。因此$f$在$[a,b]$上是有界变差的。

类似地,如果$f$在$[a,b]$上是递减的,则$V(f,[a,b])=f(a)-f(b)$。

\textbf{例子 2}:假设$f$的图形可以分为有限数量的单调弧段;即假设$[a,b]=\bigcup_{i=1}^{k}[a_{i}a_{i+1}]$,并且$f$在每个$[a_{i}a_{i+1}]$上是单调的。那么$V=\sum_{i=1}^{k}|f(a_{i+1})-f(a_{i})|$。

\textbf{例子 3}:设$f$是狄利克雷函数,定义为$f(x)=1$对于有理数$x$,$f(x)=0$对于无理数$x$。那么,显然,对于任何区间$[a,b]$,$V[a,b]=+\infty$。

\textbf{例子 6}:定义在$[a,b]$上的函数$f$被称为在$[a,b]$上满足Lipschitz条件,如果存在常数$C$使得
\begin{align} 
|f(x)-f(y)|\leq C|x-y|, \
\forall x,y\in[a,b]~.
\end{align}
这样的函数显然是有界变差的,$V[f;a,b]\leq C(b-a)$。例如,如果$f$在$[a,b]$上有连续的导数,那么(根据中值定理)$f$在$[a,b]$上满足Lipschitz条件。

\begin{example}{设 $f(x)=x\sin\left(1/x\right)$ 为 $0<x\leq1$ 且 $f(0)=0$。 证明 $f$ 在 $[0,1]$ 上有界且连续,但不是有界变差$V[f;0,1]=+\infty$。}

对于 $0 < x \leq 1$,我们知道 $-1 \leq \sin\left(\frac{1}{x}\right) \leq 1$,因此 $-x \leq x\sin\left(\frac{1}{x}\right) \leq x$,在 $[0,1]$ 上 $f(x)$ 有界。

考虑 $x \neq 0$ 的情况。在这种情况下, $f(x) = x \sin(1/x)$ 是两个连续函数的乘积,其中 $x$ 是一个连续函数,而 $\sin(1/x)$ 也是连续的。因此,它们的乘积 $f(x)$ 在 $x \neq 0$ 的情况下是连续的。在 $x = 0$ 处,有
\begin{equation}
\lim_{{x \to 0}} f(x) = \lim_{{x \to 0}} x \sin\left(\frac{1}{x}\right) = 0~.
\end{equation}

因此,$f(x)$ 在 $[0,1]$ 上是连续的。

注意到分割 $\Gamma = \{x_n\}=\{\frac{1}{n\pi+\pi/2}\}$ 我们有,  
\begin{align} f(x_n)=x_n\sin(1/x_n)=\left\{\begin{matrix}x_n& \ \text{even}\\ -x_n& \ \text{odd}\end{matrix}\right.\qquad\text{for}\ n\geq0 ~.
\end{align}
带入计算变差的公式$2$,可得:
\begin{align}
\sum_{n=1}^m|f(x_n)-f(x_{n-1})|&=\sum_{n=1}^m|(-1)^n(x_n+x_{n-1})|\\
&= \sum_{n=1}^m(x_n+x_{n-1})\\&=x_m+x_0+2\sum_{n=1}^{m-1}x_n\\\geq \sum_{n=1}^{m-1} x_n&=\sum_{n=1}^{m-1} \frac{1}{n\pi+\pi/2}~.
\end{align}
根据极限比较审敛法对比调和级数可得:
\begin{align} 
\sum_{n=1}^{m-1}\frac{1}{n\pi+\pi/2}\to\infty\qquad\mathrm{as}\qquad m\to\infty~. 
\end{align}
\end{example}

\begin{exercise}{设 $f(x)=x\cos\left(\pi/x\right)$ 为 $0<x\leq1$ 且 $f(0)=0$。 证明 $f$ 在 $[0,1]$ 上有界且连续,但不是有界变差$V[f;0,1]=+\infty$。}
\end{exercise}

\begin{theorem}{}
\begin{enumerate}
\item 如果$f$在$[a,b]$上的变差有界,那么$f$在$[a,b]$上有界。
\item 设$f$和$g$在$[a,b]$上的变差有界。那么对于任意实常数$c$,$cf$,$f+g$和$fg$在$[a,b]$上的变差也有界。此外,如果存在$\varepsilon>0$使得对于$[a,b]$中的$x$,$|g(x)|\geq\varepsilon$,那么$f/g$在$[a,b]$上的变差也有界。
\end{enumerate}
\end{theorem}

\begin{example}{证明:}

\end{example}
\begin{theorem}{}
\begin{enumerate}
\item 如果$[a^{\prime},b^{\prime}]$是$[a,b]$的子区间,则$V[a^{\prime},b^{\prime}]\leq V[a,b]$;即,随着区间的增加,变差也增加。
\item 如果$a<c<b$,则$V[a,b]=V[a,c]+V[c,b]$;即,相邻区间上的变差是可加的。
\end{enumerate}
\end{theorem}
开始证明之前,注意到如果$\Gamma$是$\Gamma$的细分,即$\Gamma$包含了$\Gamma$的所有分割点以及一些额外的点,那么$S_{\Gamma}\leq S_{\bar{\Gamma}}$。这是在$\bar{\Gamma}$包含$\Gamma$的所有点加上一个额外点的情况下,由三角不等式得出的。

\begin{example}{证明:}

\end{example}



