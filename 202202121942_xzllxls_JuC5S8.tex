% Julia 第 5 章小结
% 第 5 章小结

本文授权转载自郝林的 《Julia 编程基础》. 原文链接:\href{https://github.com/hyper0x/JuliaBasics/blob/master/book/ch05.md}{第 5 章 数值与运算}.


\subsection{5.8 小结}

在本章,我们主要探讨了 Julia 中的数值及其运算方式.

这些数值的具体类型共有 19 种.常用的有,布尔类型、有符号整数类型、无符号整数类型和浮点数类型.另外还有复数类型、有理数类型和无理数类型.我们重点讨论了整数类型和浮点数类型,其中还涉及到两种可以代表任意精度数值的类型.

对于整数,我们需要注意无符号整数的表示形式和整数的溢出行为.即使我们在无符号整数字面量的最左侧添加了负号,它也会表示为一个正整数.这与我们的直觉是不同的.而整数的溢出行为,取决于整数类型的宽度是否小于当前计算机系统的字宽.

对于浮点数,Julia 拥有 3 种不同精度的常规类型.我们在表示其值的时候可以用一些方式加以区分.我们需要注意那些浮点数中的特殊值,并记住它们在运算过程的作用和影响.

我们还讨论了针对这些数值的数学运算方式,介绍了数学运算符、位运算符、更新运算符、比较操作符,以及这些操作符的优先级和结合性.我们应该重点关注其中会影响到运算的表达和正确性的那些内容.

此外,我们也阐释了数学运算的一些细节.这涉及到 Julia 的类型提升系统.有了它,我们才能将不同类型的值放在同一个运算表达式中.这个系统以及其中的默认规则在数学运算的过程中起到了很重要的作用.并且,它还允许我们对已有的规则进行修改,或对现有的规则进行扩充.

最后,为了方便你进一步探索,我还简单地罗列了一些有用的数学函数.虽然并不完全,但这些函数都是我们在编程时最常用到的.

Julia 中的数值类型确实有不少.但如果依照它们的命名规律(如宽度的大小、有无符号等),我们还是很容易记住它们的.我们应该按需取材,使用恰当类型的数值来存储各种数据.这方面通常需要考虑便捷性、存储空间、程序性能、传输效率等等因素.