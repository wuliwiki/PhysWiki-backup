% 指数函数(高中)
% keys 指数|指数函数|自然常数
% license Usr
% type Tutor

\pentry{函数\nref{nod_functi},函数的性质\nref{nod_HsFunC},幂运算与幂函数\nref{nod_power}}{nod_d767}

\begin{issues}
\issueDraft
\end{issues}

在日常生活中,我们常常遇到指数函数的应用。想象一下,细菌每过一小时就分裂一次,数量翻倍。刚开始可能只是几百个细菌,但很快,数量会成千上万地增长,这正是指数增长的典型例子。通过指数函数,你能理解这种现象,并学会如何计算这些快速变化的量。再比如,银行里的利息也是按照一定的比例增长,时间越长,利息越多,当时间长到一定程度,大部分的帐目都将是利息,它将远超过本金,这也是典型的指数增长现象。就像雪球越滚越大,指数函数也能迅速变得非常大。

这种增长速度与自身数量直接相关的性质就指数增长,它的速度比幂函数快得多。在一个环境比较理想的情况下,大部分事物的增长都会遵循这个规律,直到环境变化或达到环境的承载能力,才会改变。指数函数的这个性质使它在很多实际问题中应用广泛。比如,地球人口在几十年内翻倍之类的现象背后都有指数函数在起作用。通过学习指数函数,你不仅能解决这些问题,还能更好地理解世界中的很多快速变化现象。

\subsection{指数函数}

回看幂运算的\aref{定义}{def_power_1},如果将底数作为参数,指数作为自变量的函数就称为指数函数,指数函数的名称指的就是自变量的在指数位置上,注意不要与幂函数相混淆。

\begin{definition}{指数函数}
形如
\begin{equation}
f(x) = a^x~.
\end{equation}
的函数称作\textbf{指数函数(exponential function)},其中 $a\in\mathbb R^+$。
\end{definition}

这里之所以要求参数$a\in\mathbb R^+$,是因为负数的实数幂次在非常多点上是未定义的,造成函数的定义域不连续,难以研究\footnote{类比狄利克雷函数可知,这个函数无法画出图像}。0的情况是平凡的,它是直线$y=1,x\neq0$,它在后面的研究中会作为一个参考基准。

\subsection{指数函数的性质}

有了幂函数的经验,同样,下面根据$a(a>0)$的性质讨论指数函数的性质。

\subsubsection{定义域和值域}

根据指数函数参数的限定,$x$可以取任意实数。此时根据幂运算的法则,$f(x)>0$恒成立。

根据$0^a=1$可知,函数恒过定点$(0,1)$。

\subsubsection{单调性}

对于$f(x)=a^x$,任取定义域上的两点$x_1>x_2$,则平均变化率为

\begin{equation}\label{eq_HsExpF_1}
\frac{f(x_1)-f(x_2)}{x_1-x_2}=\frac{a^{x_2}(a^{x_1-x_2}-1)}{(x_1-x_2)}~.
\end{equation}

由于\autoref{eq_HsExpF_1} 的分母和$a^{x_2}$为正,因此讨论$a^{x_1-x_2}$和$1$的关系。由于$x_1>x_2$,$x_1-x_2>0$。为了讨论方便,取$1^{x_1-x_2}$。由于幂函数在参数为正时在第一象限内是增函数,因此$a>1$时,$a^{x_1-x_2}>1^{x_1-x_2}$,$0<a<1$时,$a^{x_1-x_2}<1^{x_1-x_2}$。

综上,$a>0$时,\autoref{eq_HsExpF_1} 的值大于$0$,函数在定义域上是递增的;反之$0<a<1$时,函数则是递减的。

\subsubsection{不同的$a$的函数图象的关系}



\subsection{指数爆炸}

指数爆炸意味着一个函数值随自变量呈指数级别的快速增长,它的显著特征是初期增速缓慢,但随后会急剧加速。指数函数的增长速度非常快,对于初等函数而言,当参数 x 足够大时,指数函数的增长速度是最快的。具体来说,若参数 $a > 1$,在第一象限内($x > 0$)的典型函数增长速度从慢到快通常满足以下顺序:

\begin{equation}
 a < \log_a{x} <x^a < a^x~.
\end{equation}

式子中,常数 $a$ 是一个固定值,不随 $x$ 改变,或者说不增加。对数函数 $\log_a{x}$ 的值在 $x$越大时,仍在增加,但增速会越来越慢,仅略大于不增。幂函数 $x^a$ 和指数函数 $a^x$的增长速度都会随着 $x$ 增加,$x^a$ 增速逐渐加快,但$x^a$比指数函数 $a^x$ 慢,一般认为相较于指数函数,幂函数是线性或近似线性的。指数增长会呈现“爆炸式”的加速,远超其他初等函数。

\subsection{柯西函数方程}

事实上,指数函数就是满足柯西函数方程$f(x+y)=f(x)f(y)$的一个解。


\subsection{自然常数$e$}

最后要先介绍一个特殊的常数$\E \approx 2.71828$。它和早已在小学时就接触过的$\pi$有许多相似的性像。

他们都是无理数,这意味着它们不能表示为两个整数的比值。它们的小数部分是无限且不循环的,也就是说,在任何整数进制中它们都永远不会终止或重复。

他们也都是超越数,意思是它们不能作为任何有理系数多项式方程的解。换句话说,它们不能通过根式表示。这比无理数的要求更加严格。$e$由查尔斯·埃尔米特(Charles Hermite)在1873年证明,$\pi$由费迪南德·冯·林德曼(Ferdinand von Lindemann)在1882年证明。

二者都可以用无穷展开的方式来表示,下面给出两个常见的展开方式\footnote{关于求和符号可以参考\enref{求和符号(高中)}{SumSym},关于阶乘可以参考\enref{阶乘(高中)。}{factor}}:
\begin{equation}
\pi=4\sum_{n=0}^\infty\frac{(-1)^i}{2i+i}~.
\end{equation}
\begin{equation}
e=\sum_{n=0}^\infty\frac{1}{i!}~.
\end{equation}

$e$的定义有很多种方式,极限定义是被广为了解的,但它有些抽象。本文将另给出一个定义,这个定义比较简单,但在了解微分运算之后,才能体会这个定义的简洁。
假设某个初始值进行增长,随着增长次数变得无限多且增长的频率越来越频繁,最终增长的极限值是 e。

\begin{equation}
e = \lim_{n \to \infty} \left( 1 + \frac{1}{n} \right)^n~.
\end{equation}


微积分中的自然对数函数:$e$ 是使得
\begin{equation}
\frac{d}{dx} f(x) = f(x)~.
\end{equation}
成立的指数函数的底数,意味着以 $e$ 为底的指数函数是唯一的保持自身斜率不变的增长函数。