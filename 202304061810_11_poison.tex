% 泊松括号
% 哈密顿量|泊松括号|守恒量

\begin{issues}
\issueDraft
\end{issues}

\pentry{哈密顿正则方程\upref{HamCan}}
在理论力学里,泊松括号的引入能更加简洁地表明运动积分需要满足的条件。所谓运动积分,可以简单理解为在某一动力学系统中,不随时间改变的常数。动力学系统总满足二阶微分方程,所以我们总可以找到这样的运动积分。设f(p,q,t)为粒子关于动量、坐标和时间的函数,且为运动积分,则根据定义我们有:


给定函数 $u(q, p, t)$ 和 $v(q, p, t)$, 定义泊松括号为
\begin{equation}
\pb{u}{v} = \sum_i \pdv{u}{q_i}\pdv{v}{p_i} - \pdv{v}{q_i}\pdv{u}{p_i}~,
\end{equation}
容易证明泊松括号满足
\begin{equation}
\pb{v}{u} = -\pb{u}{v}~.
\end{equation}

\subsection{泊松括号与守恒量}
对任意不显含时的物理量 $\omega (q,p)$ 都有
\begin{equation}\label{eq_poison_1}
\dot \omega  = \sum_i \qty(\pdv{\omega}{q_i} \dot q_i + \pdv{\omega}{p_i} \dot p_i)
= \sum_i \qty(\pdv{\omega}{q_i} \pdv{H}{p_i} - \pdv{H}{q_i} \pdv{\omega}{p_i})
= \pb{\omega}{H} ~.
\end{equation}
所以若泊松括号恒等于零, 则该物理量守恒。

同理, 当 $\omega (q,p,t)$ 显含时间时有
\begin{equation}
\dot \omega  =  \pb{\omega}{H}  + \pdv{\omega}{t}~,
\end{equation}

量子力学中的对易算符对应泊松括号。 该式对应量子力学中的算符平均值演化方程。 % 链接未完成
