% 缺少暗物质的星系
% keys 暗物质
% license Usr
% type Tutor

2018年,van Dokkum等人声称,超扩散星系NGC1052-DF2(靠近巨大的椭圆星系NGC1052)包含很少或没有暗物质。其恒星质量估计为$2 \times 10^8 M_\odot$,而其总质量用一群不寻常的球状星团作为追踪器,估计在不到2倍。这对于这种类型的星系来说是一个非常令人惊讶的值,它们通常具有几百的质光比。这些结果及其不确定性已经经过仔细审查(例如,距离是否真的是约20 Mpc;如果星系更近,那么其推断出的恒星质量将显著下降,质光比将增加)。2019年,该团队在同一个系统中发现了NGC1052-DF4,具有相同的属性。如果这些结果得到确认,它们的解释以及对暗物质范式的影响已经被讨论。这些星系可以被解释为仅仅是统计异常,或者是由于剥夺了它们的暗物质内容的特殊情况(例如,被一个巨大的邻近星系潮汐剥离,或者类似小子弹团簇事件)。无论如何,公平地说,它们的存在对标准的小星系形成理论构成了挑战。一些数值模拟在它们的结果中没有发现这种星系的证据,而其他模拟则表明,具有适当特征的小暗物质匮乏星系确实形成了,这要归功于与巨大宿主的近距离接触。此外,这些结果将对使用修改重力而不是暗物质来解释旋转曲线的理论施加压力,因为在这些理论中,人们应该总是在足够大的距离上,在普通物质的存在下,总是检测到重力的修改。