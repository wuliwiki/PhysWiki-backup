% 同时性的相对性

\pentry{狭义相对论的基本假设\upref{SpeRel}}

\subsection{事件}

在和时空相关的理论中,当我们描述一件事的时候,我们并不关心这件事具体是什么,只关心它发生在何时何地.因此为了将来的讨论,我们首先需要定义“事件”的概念.

一个\textbf{事件(event)}是指在时空坐标系中的一个点.事件所发生的时间、地点,就是事件作为一个点的坐标.

\subsection{对事件的观测}

狭义相对论的核心是光.在任何参考系中,光的表现都应该一样,不仅仅是光速不变,其它方面如能量密度的变化也应该是一致的.

给定一个惯性参考系$K$.如果在$K$中某一时刻,某个点向四周均匀发射一道球面光,那么这个光球的半径是恒定以光速膨胀的,而光球所含的总能量是不会变的(能量守恒定律).由于这道光是均匀的,这就使得光球上任意一点的光能密度和半径的平方成反比.如果我们知道了发射的总光能,那么接收到光的人就可以据此计算出光源距离自己有多远;由于光速恒定不变,还可以用这个距离反推这道光是什么时候发出来的.

综上所述,如果一个事件发生时均匀向四周发射一道球面光,总光能是给定的,那么宇宙中一切接收到了这个球面光的观察者都能反推出(在他们各自的参考系中)这一事件所发生的时间、地点.

在接下来的思想实验中,我们就可以假设每个事件发生时都伴随着这样一道光的辐射,方便观察者确定事件的时空坐标.

\subsection{同时性的相对性}



