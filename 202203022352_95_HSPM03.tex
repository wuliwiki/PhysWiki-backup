% 牛顿运动定律
% keys 牛顿|牛顿运动定律|惯性|加速度

\begin{issues}
\issueDraft
\issueTODO
\end{issues}

\pentry{机械运动基础\upref{HSPM01}, 相互作用\upref{HSPM02}}

\subsection{牛顿第一定律}
\subsubsection{内容}
一切物体总保持匀速直线运动状态或静止状态,除非作用在它上面的力迫使它改变这种状态.

\subsubsection{惯性}
物体保持原来匀速直线运动状态或静止状态的性质叫做惯性,是物体本身的固有属性.质量是物体惯性大小的唯一量度,物体的质量越大,其运动状态越难改变,惯性越大;质量越小,其运动状态越容易改变,惯性越小.

\subsubsection{对牛顿第一定律的理解}
牛顿第一定律揭示了一切物体都具有保持原来匀速直线运动状态或静止状态的性质,即一切物体都具有惯性,所以牛顿第一定律又叫\textbf{惯性定律}.

牛顿第一定律定性地揭示了运动和力的关系,说明力不是维持物体运动状态的原因,而是改变物体运动状态的原因.

牛顿第一定律是牛顿在总结前人观念的基础上得出的,是在理想实验的基础上加以科学推理和抽象得到的.

牛顿第一定律无法由实验直接验证,它所描述的是一种不受外力的理想状态.

当物体所受合力为0的时候,其效果跟不受外力时一致,但不能把“合力为0”说成“不受外力”.

\subsection{牛顿第二定律}

\subsection{牛顿第三定律}