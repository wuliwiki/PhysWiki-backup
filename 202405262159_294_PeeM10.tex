% 2010 年考研数学试题(数学一)
% keys 考研|数学一
% license Copy
% type Tutor

\subsection{选择题}
\begin{enumerate}
\item 极限  $\displaystyle \lim_{n\to\infty}[\frac{x^2}{(x-a)+(x+b)}]^x$=$(\quad )$\\
(A) $1$\\
(B) $e$\\
(C) $e^(a-b)$\\
(D) $e^(b-a)$
\item  设函数 $z=z(x,y)$ 由方程 $\displaystyle F(\frac{y}{x},\frac{z}{x})$ 确定,其中 $F$ 为可微函数,且 $F'_2 \neq0$ ,则  $\displaystyle x \pdv{z}{x}+y\pdv{z}{y}$=$(\quad )$
(A)  $x$
(B)  $z$
(C) $-x$
(D)  $-z$
\item 设 $m,n$ 均是正整数,则反常积分 $\displaystyle \int_0^1 \frac{\sqrt[m]{\ln^2(1-x)}}{\sqrt[n]{x}}$ 的收敛性 $(\quad )$\\
(A) 仅与 $m$ 的取值有关\\
(B) 仅与 $n$ 的取值有关\\
(C)  与 $m,n$ 的取值都有关\\
(D)  与 $m,n$ 的取值都无关
\item $\displaystyle \lim_{n\to\infty} \sum_{i=1}^n \sum_{j=1}^n \frac{n}{(n+i)(n^2+j^2)}$= $(\quad )$\\
(A) $\displaystyle \int_{0}^{1}\dd{x}\int_{0}^{x} \frac{1}{(1+x)(1+y^2)}$\\
(B)$\displaystyle \int_{0}^{1}\dd{x}\int_{0}^{x} \frac{1}{(1+x)(1+y)}$\\
(C)$\displaystyle \int_{0}^{1}\dd{x}\int_{0}^{1} \frac{1}{(1+x)(1+y)}$\\
(D)$\displaystyle \int_{0}^{1}\dd{x}\int_{0}^{1} \frac{1}{(1+x)(1+y^2)}$
\item 设 $\mat A$ 为4阶实对称矩阵,且 $\mat {A^2+A=O}$  。若 $\mat A$ 的秩为3,则 $\mat A$  相似于 $(\quad )$\\
(A) $\pmat{1& & &  \\ &1& & &\\ & &1&\\& & &1}$\\
(B) $\pmat{1& & &  \\ &1& & &\\ & &-1&\\& & &0}$\\
(C) $\pmat{1& & &  \\ &-1& & &\\ & &-1&\\& & &0}$\\
(D) $\pmat{-1& & & \\ &-1& & &\\ & &-1& \\& & &0}$
\item  设随机变量 $X$ 的分布函数 $F(x)=\leftgroup{&0,& x<0,\\ &\frac{1}{2},& 0\le x<1\\ &1-e^{-x},& x \ge 1}$   ,则$P\{X=1\}$=$(\quad )$\\
(A)  $0$\\
(B) $\frac{1}{2}$\\
(C)  $\frac{1}{2}-e^{-1}$\\
(D) $1-e^{-1}$
\item  
(A) 
(B)
(C)
(D)

\end{enumerate}