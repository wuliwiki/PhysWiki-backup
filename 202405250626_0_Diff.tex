% 一元函数的微分(简明微积分)
% keys 微积分|导数|极限|微分|微分近似
% license Xiao
% type Tutor

\pentry{基本初等函数的导数\nref{nod_FunDer}}{nod_b8ec}
我们回顾导数的定义,令 $y = f(x)$,
\begin{equation}\label{eq_Diff_4}
f'(x) = \dv{y}{x} = \lim_{\Delta x\to 0} \frac{\Delta y}{\Delta x}~.
\end{equation}
这告诉我们, 当 $\Delta x$ 越小, 以下近似就越精确
\begin{equation}\label{eq_Diff_2}
\Delta y \approx f'(x) \Delta x~.
\end{equation}
这说明,当 $\Delta x$ 很小的时候, 对应的 $\Delta y$ 和 $\Delta x$ 成正比。

形式上,为了表示 $\Delta x\to 0$ 的过程中上式越来越精确地成立,我们可以写成
\begin{equation}\label{eq_Diff_1}
\dd{y} = f'(x)\dd{x}~.
\end{equation}
并把它叫做函数 $f(x)$ 的\textbf{微分}。 要特别注意该式只是\autoref{eq_Diff_4} 的另一种方便的\textbf{书写形式}。而不可以理解为等号左边的无穷小等于右边的无穷小。 因为无穷小并不是具体的实数,无法比较是否精确相等。

\subsection{微分近似}\label{sub_Diff_1}
% 类似\autoref{eq_Diff_1} 的微分关系式默认取极限 $\dd{x} \to 0$ 才能写成等号, 但只要在一定范围 $\Delta x$ 内导函数 $f'(x)$ 的变化非常小, 就可以将函数值的变化量 $\Delta y = f(x+\Delta x)-f(x)$ 近似为(\autoref{fig_Diff_1})

在实际应用中, 即使 $\Delta x$ 只是一个较小的具体的数而不是无穷小,我们也往往可以适用\autoref{eq_Diff_2} 来估计 $\Delta y$。
\begin{equation}\label{eq_Diff_3}
\Delta y \approx f'(x) \Delta x~.
\end{equation}
注意在该近似式中不能使用微分符号 $\mathrm{d}$ 或等号。

\begin{figure}[ht]
\centering
\includegraphics[width=4.5cm]{./figures/c8827920d12da2be.pdf}
\caption{微分近似用函数曲线的切线增量 $f'(x)\Delta x$ 来近似函数增量 $\Delta y$, 另见\autoref{fig_Der_1}~\upref{Der}。} \label{fig_Diff_1}
\end{figure}
注意\autoref{eq_Diff_3} 中的近似成立的条件是, 在 $\Delta x$ 所在的区间内, $f'(x)$ 的变化足够小,或者说 $f(x)$ 在该区间足够的\textbf{线性}(即足够接近直线)。

\begin{example}{测量误差}\label{ex_Diff_1}
若测得立方体的边长为 $a$, 测量边长的最大可能误差为 $\sigma_a$, 假设 $\sigma_a \ll a$, 估计立方体体积的最大误差 $\sigma_V$。

解: 立方体的体积与边长的关系为 $V(a)=a^3$, 根据微分近似,有
\begin{equation}
\sigma_V \approx V'(a) \sigma_a = 3a^2 \sigma_x~.
\end{equation}
\end{example}

\begin{example}{细圆环的面积和薄球壳的体积}\label{ex_Diff_2}
\begin{figure}[ht]
\centering
\includegraphics[width=10cm]{./figures/4fa92b61a3e93d0b.pdf}
\caption{细圆环的面积} \label{fig_Diff_2}
\end{figure}

1. 圆的面积关于其半径的函数为 $A(r) = \pi r^2$, 对该式进行微分得 $\dd{A} = 2\pi r\dd{r}$。 注意到 $2\pi r$ 为 $r$ 对应的周长, 所以微分近似告诉我们, 半径为 $r$, 宽度为 $\Delta r \ll r$ 的圆环的面积约等于该圆环的周长乘以圆环的宽度。

2. 球的体积关于其半径的函数为 $V(r) = 4\pi r^3/3$, 求微分得 $\dd{V} = 4\pi r^2 \dd{r}$。 注意到 $4\pi r^2$ 为 $r$ 对应的球表面积, 所以微分近似告诉我们, 半径为 $r$, 厚度为 $\Delta r \ll r$ 的球壳的体积等于该球壳的表面积乘以球壳厚度。

注意这两个近似成立的前提都是在 $\Delta r$ 的范围内, $A'(r)$ 或 $V'(r)$ 变化得足够慢。 也就是需要满足
\begin{equation}
\Delta r \ll r~,
\end{equation}
即厚度远小于半径。 你可以给 $r$ 和 $\Delta r$ 取不同的具体数值,验证以上近似的误差有多大。
\end{example}

\begin{example}{失败的例子}
作为一个失败的例子,我们对 $y = x^2$ 以及 $x=0$ 使用\autoref{eq_Diff_3}。 由于 $f'(x) = 2x$, 有 $f'(0) = 0$,也就是说无论 $\Delta x$ 多小都会估计出 $\Delta y = 0$。

但若 $\Delta x = x - 0$,则精确来说 $\Delta y = x^2 - 0^2 = \Delta x^2$。 所以该近似的相对误差达到了 $-100\%$。
\end{example}
