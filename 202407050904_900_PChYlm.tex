% 库仑势能的球谐展开
% keys 球坐标|球谐函数|拉普拉斯方程|库仑势能|勒让德多项式
% license Xiao
% type Tutor

\begin{issues}
\issueDraft
\end{issues}

\pentry{球坐标系中的拉普拉斯方程\nref{nod_SphLap}, 库仑定律\nref{nod_ClbFrc}}{nod_3449}

库仑势能的球谐展开为
\begin{equation}
\frac{1}{\abs{\bvec r - \bvec r'}} = 4\pi \sum_{l=0}^{\infty} \frac{1}{2l+1} \frac{r_<^l}{r_>^{l+1}} \sum_{m = -l}^l Y_{l,m}^*(\uvec r') Y_{l,m}(\uvec r)~.
\end{equation}
根据\autoref{eq_SphHar_10}, 令 $\alpha$ 为 $\uvec r, \uvec r'$ 的夹角, 上式也可以记为
\begin{equation}\label{eq_PChYlm_1}
\frac{1}{\abs{\bvec r - \bvec r'}} = \sum_{l=0}^{\infty} \frac{r_<^l}{r_>^{l+1}} P_l(\cos\alpha) = \sum_{l=0}^{\infty} \sqrt{\frac{4\pi}{2l+1}} \frac{r_<^l}{r_>^{l+1}} Y_{l,0}(\alpha, 0)~.
\end{equation}

\begin{figure}[ht]
\centering
\includegraphics[width=14.2cm]{./figures/9728a21d1a4cf09b.png}
\caption{左: $1/\abs{\bvec r - \uvec z}$ 使用\autoref{eq_PChYlm_1} 展开, 右: 左图延 $z$ 轴的截线 $1/\abs{z - 1}$。 注意左图中 $r = 1$ 的圆周上有肉眼可见的不收敛, 增大 $l$ 的个数即可。} \label{fig_PChYlm_1}
\end{figure}

\subsection{证明}
和 “\enref{平面波的球谐展开}{Pl2Ylm}” 中的证明类似, 我们既可以直接积分得到径向函数, 根据\autoref{eq_PChYlm_1} 还可以假设 $\bvec r'$ 在 $z$ 轴。 也可以在球坐标中解偏微分方程。

除 $\bvec r'$ 点外, $f(\bvec r) = \frac{1}{\abs{\bvec r - \bvec r'}}$ 符合拉普拉斯方程。\enref{球坐标系中拉普拉斯方程}{SphLap}的通解为(\autoref{eq_SphLap_6})
\begin{equation}
f(\bvec r) = f(r, \theta, \phi) = \sum_{l = 0}^\infty \sum_{m = -l}^l \qty(C_{l,m} r^l + \frac{C'_{l,m}}{r^{l+1}})P_l^m(\cos\theta)\E^{\I m\phi}~,
\end{equation}
把空间划分为 $r < r'$ 和 $r > r'$ 两个区域。 $r = 0$ 时函数需要有定义, 所以 $r < r'$ 区间 $C'_{l,m} = 0$。 另外要求 $r \to \infty$ 时 $f(\bvec r) \to 0$, 所以当 $r > r'$, $C_{l,m} = 0$。

接下来对比系数即可求出 $C_{l,m}$ 和 $C_{l,m}'$。
\addTODO{具体计算}

\subsection{附: 画图代码}
\autoref{fig_PChYlm_1} 的 Matlab 代码如下, 其中 \verb`surfSph` 函数见 “\enref{Matlab 球坐标中的分布图}{MatPol}”。 \verb`SphHarm` 函数见 “\enref{球谐函数数值计算}{YlmNum}”。
\begin{lstlisting}[language=matlab, caption=Ylm\_Coul\_exp.m]
% expansion of 1/|r-r'| into Y_lm
% r' in z axis
lmax = 20;
r0 = 1;
Nr = 300; Nth = 300;
r = linspace(1e-5, 2.5, Nr);
mark = r <= r0;
th = linspace(0, 2*pi, Nth); ph = zeros(size(th));
data = zeros(Nr, Nth);
for l = 0:lmax
    % eq_PChYlm_1
    data(mark, :) = data(mark, :) + ...
        sqrt(4*pi/(2*l+1)) * (r(mark).').^l ./ r0^(l+1) ...
        .* SphHarm(l, 0, th, ph);
    data(~mark, :) = data(~mark, :) + ...
        sqrt(4*pi/(2*l+1)) * r0^l ./ (r(~mark).').^(l+1) ...
        .* SphHarm(l, 0, th, ph);
end

[R, Th] = ndgrid(r, th); Ph = zeros(size(Th));
figure;
subplot(1, 2, 1);
surfSph(R, Th, Ph, data);
caxis([0, 5]);
title(['l <= ' num2str(l)]);
colorbar; colormap jet;

subplot(1, 2, 2);
plot(r, data(:,1));
hold on; plot(r, 1./abs(r-r0), '.');
legend('球谐展开', '1/|z-z''|');
axis([0, r(end), 0, 5]);
xlabel z;
title(['l <= ' num2str(l)]);
\end{lstlisting}
