% 量子简谐振子(级数法)
% 量子力学|微分方程|能级|厄密多项式|Hermite|薛定谔方程

\begin{issues}
\issueDraft
\end{issues}

\pentry{简谐振子(升降算符)\upref{QSHOop}}

量子力学中, 简谐振子的定态薛定谔方程为
\begin{equation}\label{QSHOxn_eq2}
-\frac{\hbar^2}{2m} \dv[2]{\psi}{x} + \frac12 m \omega^2 x^2\psi  = E\psi
\end{equation}
能级为
\begin{equation}
E_n = \qty(\frac12 + n)\hbar \omega 
\end{equation}
归一化的束缚态波函数为
\begin{equation}\label{QSHOxn_eq1}
\psi_n (x) = \frac{1}{\sqrt{2^n n!}} \qty(\frac{\alpha^2}{\pi })^{1/4} H_n(u) \E^{-u^2/2}
\end{equation}
其中
\begin{equation}
\alpha \equiv \sqrt{\frac{m\omega}{\hbar }}, \qquad
u \equiv \alpha x
\end{equation}
$H_n(u)$ 是 Hermite 多项式\upref{HermiP}.

\begin{figure}[ht]
\centering
\includegraphics[width=14cm]{./figures/QSHOxn_1.pdf}
\caption{简谐振子的前几个束缚态波函数(\autoref{QSHOxn_eq1}, $\alpha = 1$)} \label{QSHOxn_fig1}
\end{figure}

前4个波函数分别为(注意函数的奇偶性与角标的奇偶性相同)
\begin{equation}\ali{
\psi_0(x) &= \qty(\frac{\alpha^2}{\pi})^{1/4} \E^{-u^2/2}, &
\psi_2(x) &= \qty(\frac{\alpha^2}{\pi })^{1/4} \frac{1}{\sqrt 2 } (2u^2 - 1)\E^{-u^2/2},\\
\psi_1(x) &= \qty(\frac{\alpha ^2}{\pi })^{1/4} \sqrt2u \E^{-u^2/2}, \quad &
\psi_3(x) &= \qty(\frac{\alpha ^2}{\pi })^{1/4} \frac{1}{\sqrt 3} u(2u^2 - 3)\E^{-u^2/2}.
}\end{equation}

Matlab 代码如(使用原子单位制\upref{AU})
\begin{lstlisting}[language=matlab, caption=psi\_SHO.m]
function psi = psi_SHO(m, w, n, x)
alpha = sqrt(m*w); u = alpha * x;
psi = 1/sqrt(2^n*factorial(n)) *...
    (alpha^2/pi)^0.25 * hermiteH(n, u) .* exp(-u.^2/2);
end
\end{lstlisting}

\subsection{推导(生成函数法)}

% \addTODO{可以直接用级数解也可以用升降算符.}

% \subsubsection{升降算符法}


为了方便,记$\varepsilon \equiv 2mE/\hbar^2$,再使用$\hbar=1$的单位制,于是\autoref{QSHOxn_eq2} 化为
\begin{equation}\label{QSHOxn_eq7}
\frac{\mathrm{d}^2}{\dd y^2}\psi(x) + (\varepsilon-m^2\omega^2x^2)\psi(x)=0
\end{equation}

如果记$\psi(x)=h(x)\exp(-m\omega x^2/2)$\footnote{指数上的符号是为了让波函数在无穷远处趋于0.},那么代入
\begin{equation}

\end{equation}


\subsubsection{生成函数}

考虑生成函数$g(x, t)\equiv \exp(-t^2+2tx)$,用它来定义一系列Hermite多项式$H_n$:
\begin{equation}\label{QSHOxn_eq3}
g(x, t) = \exp(-t^2+2tx) = \sum_{n=0}^\infty H_n(x)\frac{t^n}{n!}
\end{equation}
即
\begin{equation}
\frac{\partial^n}{\partial t^n}g(x, t)\mid_{t=0} = H_n(x)
\end{equation}
容易得
\begin{equation}\label{QSHOxn_eq6}
H_0(x) = g(x, 0) = 1
\end{equation}

对\autoref{QSHOxn_eq3} 求关于$x$的偏导数得:
\begin{equation}\label{QSHOxn_eq4}
\ali{
    \sum_{n=0}^\infty H_n(x)\frac{2t^{n+1}}{n!}&=2t\exp(-t^2+2tx)\\
    &=\frac{\partial}{\partial x}\exp(-t^2+2tx)\\
    &=\frac{\partial}{\partial x}\sum_{n=0}^\infty H_n(x)\frac{t^n}{n!}\\
    &=\sum_{n=0}^\infty H'_n(x)\frac{t^n}{n!}
}
\end{equation}
比较\autoref{QSHOxn_eq4} 两端后得
\begin{equation}\label{QSHOxn_eq5}
H'_n(x) = 2nH_{n-1}(x)
\end{equation}

由\autoref{QSHOxn_eq5} 和\autoref{QSHOxn_eq6} ,再考虑到$H_n(x)$的常数项是$H_n(0)=\frac{\partial^n}{\partial t^n}g(0, t)\mid_{t=0}=\frac{\partial^n}{\partial t^n}\exp(-t^2)\mid_{t=0}$,可以推知所有Hermite多项式:
\begin{equation}
H_0(x)=1\implies H'_1(x)=2\implies H_1(x)=2x
\end{equation}
\begin{equation}
H_1(x)=2x\implies H'_2(x)=8x\implies H_2(x)=4x^2-2
\end{equation}
\begin{equation}
H_2(x)=4x^2-2\implies H'_3(x)=24x^2-12\implies H_3(x)=8x^3-12x
\end{equation}
以此类推.


\subsubsection{简谐振子波函数}





