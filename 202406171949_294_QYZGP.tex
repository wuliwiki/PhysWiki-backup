% 氢原子光谱
% license CCBYSA3
% type Wiki
(本文根据 CC-BY-SA 协议转载自原搜狗科学百科对英文维基百科的翻译)

氢原子光谱指的是氢原子内之电子在不同能级跃迁时所发射或吸收不同波长、能量之光子而得到的光谱。氢原子光谱为不连续的线光谱,自无线电波、微波、红外光、可见光、到紫外光区段都有可能有其谱线。根据电子跃迁的后所处的能阶,可将光谱分为不同的线系。理论上有无穷个线系,前6个常用线系以发现者的名字命名。

\subsection{物理学}
氢原子由围绕其原子核运行的电子组成。电子和质子之间的电磁力使电子形成一组量子态,每个量子态都有对应的能量。氢原子的玻尔模型将这些状态视为围绕原子核的独特轨道。如图所示,每个能量状态或轨道由一个整数n表示。玻尔模型后来被量子力学所取代,在量子力学中,电子占据原子轨道而不是轨道,但氢原子的允许能级仍然与早期理论中的相同。

当电子从高能态跃迁或跳跃到低能态时,就会发生光谱发射。为了区分这两种状态,通常将低能态指定为n’,将高能态指定为n。发射光子的能量对应于这两种状态之间的能量差。由于每种状态的能量是固定的,它们之间的能量差也是固定的,因此跃迁总是会产生具有相同能量的光子。

谱线根据n’被分成系列。使用每个系列中的希腊字母,从系列的最长波长/最低频率开始按顺序命名。例如,2 → 1线被称为“Lyman-alpha”(Ly-α),而7 → 3线被称为“Paschen-delta”(Pa-δ)。

也有不属于这些系列的氢发射谱线,例如波长为21厘米的谱线。这些发射谱线对应于及其罕见的原子现象,如超精细跃迁。[1]由于相对论修正,精细结构也导致单个谱线以两条或多条紧密组合的细谱线形式出现。[2]

在量子力学理论中,原子发射的离散光谱是建立在薛定谔方程的基础上的,该方程主要研究类氢原子的能谱,而与时间相关的等效海森堡方程则较适用于研究由外部电磁波[3]驱动的原子。

在原子吸收或发射光子的过程中,守恒定律适用于整个孤立系统,如原子加上光子。因此,在光子吸收或发射的过程中电子的运动总是伴随着原子核的运动,由于原子核的质量是有限的,类氢原子的能谱总是依赖于原子核的质量。由于氢原子是只有一个质子的原子核,氢原子的光谱能量仅取决于原子核(例如在库仑场中):事实上,一个质子的质量大约是电子质量的5000倍,在零阶近似时电子质量可以忽略。[3]

\subsection{里德伯公式}
玻尔模型中能级之间的能量差,以及发射/吸收光子的波长,由里德堡公式给出:[4]