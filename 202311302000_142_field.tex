% 环和域
% keys 实数|域|加|乘
% license Xiao
% type Tutor
% 概念介绍基本完成

\begin{issues}
\issueDraft
\issueOther{与词条环\upref{Ring}重复,建议修改}
\issueOther{素域,除环(体)应当单独设计单独的词条,建议修改}
\end{issues}


\pentry{环\upref{Ring}}

在群\upref{Group}的基础上,我们可以定义更复杂的对象,环和域。简单来说,环和域各自有两个运算,通常称为加法和乘法,其中加法必须构成一个阿贝尔群。由于多了一个运算,我们还需要考虑两个运算之间的复合关系,因此还额外引入了一个性质,即乘法对加法的分配性。

\subsection{比群多了一个运算: 环和域}






\begin{definition}{环}
一个\textbf{环(ring)}是一个集合$R$及其上两个运算,\textbf{加法}$+$和\textbf{乘法}$\times$,构成的三元组$(R, +, \times)$,并且满足以下公理:
\begin{enumerate}
    \item $(R, +)$构成\textbf{阿贝尔群}:
    \begin{itemize}
    \item 加法$+$在集合$R$上是\textbf{封闭}的。
    \item 加法$+$具有\textbf{结合性}。
    \item 加法$+$在集合$R$上存在\textbf{单位元},记为$0$,常称为“零元”。
    \item 对于任意$R$中元素$a$,存在其\textbf{加法逆元}$-a$,使得$a+(-a)=0$。
    \item 加法$+$是\textbf{交换的}。
    \end{itemize}
    \item $(R, \times)$构成\textbf{幺半群}:
    \begin{itemize}
    \item 乘法$\times$是\textbf{封闭}的。
    \item 乘法$\times$具有\textbf{结合性}。
    \item 乘法$\times$存在\textbf{单位元},记为$1$,常称为“幺元”\footnote{这是有争议的。有的数学家从形式上定义,不要求环有幺元,而把有幺元的特别称呼为“幺环”;但是有的数学家认为,我们很少研究不含幺元的环,所以不如直接定义环必须有幺元,简化讨论。本书使用后一种传统,但是有时也会为了强调而使用“幺环”的术语,也就是说我们会将“环”和“幺环”视作等价的术语来使用。英语中将我们此处定义的环称为ring,而去掉乘法单位元之后所得结构称为rng。}。
    \end{itemize}
    \item \textbf{乘法}对\textbf{加法}满足左右\textbf{分配律}:对于任意$x, y, z\in R$,有$x\times(y+z)=x\times y+x\times z$(左分配律),且$(y+z)\times x=y\times x+z\times x$(右分配律)\footnote{分左右两种分配律,是因为环的乘法不一定交换。}。
\end{enumerate}

不至于混淆时,为方便计,可将“环$(R, +, \times)$”简称为“环$R$”,并将乘法符号省略,即将$a\times b$写为$ab$。

如果乘法$\times$还具有\textbf{交换性},则称$(R, +, \times)$为一个\textbf{交换环(commutative ring)}。
\end{definition}


环配合其加法构成一个\textbf{阿贝尔群},而乘法只能构成一个\textbf{半群}——乘法不是有幺元吗,为什么不能是幺半群?因为有$0$这个元素在,任何元素乘以$0$还是$0$,就像小学就学过的实数乘法一样。这是由\textbf{分配律}导致的,我们把它写为以下定理:



\begin{theorem}{}
设$R$是一个环,$0$是其加法单位元。则对于任意$r\in R$,有$r\times 0=0\times r=0$。
\end{theorem}
\textbf{证明}:

$r\times 0=r\times (r-r)=r\times r-r\times r=0$。

\textbf{证毕}。

域是一种特殊的环,定义如下:

\begin{definition}{}\label{def_field_4}
一个\textbf{域(field)}是一个集合$\mathbb{F}$及其上两个二元运算,\textbf{加法}$+$和\textbf{乘法}$\times$,构成的三元组$(\mathbb{F}, +, \times)$,满足以下公理:

\begin{enumerate}
    \item $(\mathbb{F}, +)$构成\textbf{阿贝尔群}:
    \begin{itemize}
    \item 加法在$\mathbb{F}$上$+$是\textbf{封闭}的。
    \item 加法$+$具有\textbf{结合性}。
    \item 加法$+$在$\mathbb{F}$上存在\textbf{单位元},记为$0$,常称为“零元”。
    \item 对于任意$\mathbb{F}^*$中元素$a$,存在其\textbf{加法逆元}$-a$,使得$a+(-a)=0$。
    \item 加法$+$是\textbf{交换的}。
    \end{itemize}
    \item $(\mathbb{F}^*, \times)$构成\textbf{阿贝尔群},其中$\mathbb{F}^*=\mathbb{F}-\{0\}$:
    \begin{itemize}
    \item 乘法在$\mathbb{F}^*$上$\times$是\textbf{封闭}的。
    \item 乘法$\times$具有\textbf{结合性}。
    \item 乘法在$\mathbb{F}^*$上存在\textbf{单位元},记为$1$,常称为“幺元”。
    \item 对于任意$\mathbb{F}^*$中元素$a$,存在其\textbf{乘法逆元}$a^{-1}$,使得$a\times a^{-1}=1$。
    \end{itemize}
    \item \textbf{乘法}对\textbf{加法}满足\textbf{分配律}:对于任意$x, y, z\in R$,有$x\times(y+z)=x\times y+x\times z$(左分配律),且$(y+z)\times x=y\times x+z\times x$(右分配律)\footnote{分左右两种分配律,是因为环的乘法不一定交换。}。
    \item 
\end{enumerate}

不至于混淆时,为方便计,可将“域$(\mathbb{F}, +, \times)$”简称为“域$\mathbb{F}$”,并将乘法符号省略,即将$a\times b$写为$ab$。
\end{definition}


比较下来,域的定义只比环的定义多了第10和第11条,一个要求乘法可逆(即可以做除法),一个要求乘法交换;其它部分则完全相同。两个定义中,第1到第5条定义了加法的性质,使得环或域在加法下构成交换群;第5条定义了两个运算间的关系,即分配律;剩下的则定义了乘法的性质。可以看到,域比环更具体,环比域更抽象。当然了,群最抽象,环和域都是群的具体例子。

一般来说,为了方便,我们通常会省去乘法符号,把 $r\times s$ 写为 $rs$。元素 $r$ 的加法逆元记为 $-r$,这样就可以把 $r+(-s)$ 记为 $r-s$。如果元素 $r$ 有乘法逆元,那么我们把它的乘法逆元记为 $r^{-1}$,于是就有 $rr^{-1}=r^{-1}r=1$,像我们在群论初步中看到 $xx^{-1}=x^{-1}x=e$ 一样。

要注意的是,环的乘法只有在去掉加法单位元 $0$ 的时候才能构成幺半群或者群,这是因为任何元素乘以 $0$ 还是 $0$,就像小学就学过的实数乘法一样。这是由\textbf{分配律}导致的,我们把它写为以下定理:

\begin{theorem}{}
设 $R$ 是一个环,$0$ 是其加法单位元。则对于任意 $r\in R$,有 $r\times 0=0\times r=0$。
\end{theorem}
\textbf{证明}:

$r\times 0=r\times (r-r)=r\times r-r\times r=0$。

\textbf{证毕}。



我们可以简单地理解为,\textbf{环}是“能进行\textbf{加减乘}运算的集合”,其中\textbf{乘法还不一定交换};\textbf{域}则是能“进行\textbf{加减乘除}运算的集合”,而且\textbf{加法和乘法都可以交换}。

最常见的环是\textbf{整数环}。全体整数构成的集合,配备通常的加法、乘法运算后,构成一个环,并且还是交换环。实函数也可以用来构成环:如果 $f, g$ 都是从实数到实数的映射,那么对于任意实数 $x$,定义 $(f+g)(x)=f(x)+g(x)$ 以及 $(f\times g)(x)=f(x)g(x)$,这样,\textbf{全体实函数的集合}配备如此定义的 $+$ 和 $\times$ 运算,就构成一个交换环。

非交换环的例子也可以用函数构造出来,方法是把以上定义的函数环中的\textbf{乘法}替换为\textbf{复合}运算:$(f\circ g)(x)=f(g(x))$。这样,由于映射的复合一般不交换\footnote{比如说,对于函数 $f=x+1$ 和函数 $g=x^2$,我们有 $f\circ g=x^2+1$,但 $g\circ f=(x+1)^2=x^2+1+2x\not=f\circ g$。},于是 $(\{\text{全体实函数}\}, +, \circ)$ 就是一个非交换环。除此之外,我们将来会遇到的\textbf{矩阵}等对象也能构成非交换环,但是矩阵乘法的非交换性本质上相当于映射复合的非交换性,因为矩阵可以用来表示线性变换,此时矩阵的乘法对应的是线性变换的复合。不用担心,我们会在线性代数章节里再讨论这些。

最常见的域就是有理数域和实数域。在有理数集合和实数集合上配备通常的加法、乘法运算后,都构成域,称作有理数域和实数域。

从定义很容易看出域一定是环,但上面所举的环的例子中,都有乘法不可逆的元素,因此它们都只是环,不是域。比如说,整数环里 $2$ 这个元素,就不存在整数的乘法逆元;实函数环里 $f(x)=x$ 也不存在乘法逆元(此处取交换的环的那个例子的定义,用实数的乘法导出函数的乘法),因为它有零点 $f(0)=0$,导致不存在实函数 $g$ 使得 $(f\times g)(x)$ 恒为 $1$。




\subsection{体和素域的概念}

\begin{definition}{体}\label{def_field_1}
给定一个集合 $H$,如果这个集合中定义了两个运算,加法 “+” 和乘法 “$\times$”,并且 $H$ 对于加法构成一个阿贝尔群,而 $H-\{0\}$ 构成群($0$ 为 $H$ 加法群的单位元),并且乘法对加法满足分配律,即对于任何 $a, b, c\in H$,满足 $a\times(b+c)=a\times b+a\times c$,那么我们称 $(H, +, \times)$ 构成一个\textbf{体(skew field)},或称\textbf{可除环(division ring)}、\textbf{除环}。
\end{definition}

像在环论中省略乘法的符号一样,我们也常常把体中的“$\times$”符号省略,比如说,将分配律表示为 $a(b+c)=ab+ac$。

简单来说,体就是能进行加减乘除的一个集合,其中加法是可交换的,乘法却不一定。由于乘法不一定交换,这就使得除法运算相对复杂,但我们在此不过多展开。

\begin{example}{四元数体}
四元数\upref{Quat}词条中所定义的全体四元数构成的集合,配上所定义的加法和乘法,构成一个体,称为\textbf{四元数体}。
\end{example}

\begin{example}{矩阵体}
某个域上的全体 $n$ 阶可逆矩阵,配上矩阵加法和乘法,构成一个体。
\end{example}

类比子群和子环的定义,我们也可以定义子体。

\begin{definition}{子体}
体 $H$ 的子集 $S$ 如果满足其对已有的加法和乘法仍然构成体,那么称 $S$ 是 $H$ 的一个\textbf{子体}。
\end{definition}
子体的概念引出了以下关键概念。
\begin{definition}{素体与素域}
一个体的子体之交显然还是一个子体,因此每个体都存在唯一的非平凡不可约子体,称为这个体的\textbf{素体(prime field)}或\textbf{素域}。
\end{definition}

素体又被称作素域的原因是,素体必然是 $\mathbb{Z}_p$ 或 $\mathbb{Q}$,其中 $p$ 是素数。这两种体都是域。关于这一点,可参考下面一小节“\textbf{素域的附加讨论}”。

% 那么我们刚才讨论中的域是什么呢?

% \begin{definition}{域}\label{def_field_2}
% 给定一个体 $\mathbb{F}$,如果 $\mathbb{F}$ 的乘法满足交换律,那么称其为一个\textbf{域(field)}。
% \end{definition}

% 一个域的子体总是乘法可交换的,因此也都称为域的\textbf{子域(sub-field)}。

数学界主流将域视作乘法可交换的体,因此当谈到域时,总是认为乘法可交换。少数数学家会把我们以上定义的体称为域,而将我们定义的域称为交换域,但这并不是主流,也不是小时百科所使用的定义。

\begin{example}{数域}
复数域的任何子域,被统称为\textbf{数域}。最重要的数域有三个,有理数域 $\mathbb{Q}$,实数域 $\mathbb{R}$ 和复数域 $\mathbb{C}$,其中 $\mathbb{Q}$ 是最小的数域,也就是说任何数域都包含它;实数域是有理数域的完备化,意味着有理数域中的收敛数列都收敛于某个实数;复数域是最大的数域,也就是说任何数域都是复数域的子域。

注意,$\mathbb{Z}_p$ 并不是 $\mathbb{Z}$ 的子域,因为在 $\mathbb{Z}_p$ 中,$(p-1)+1=0$,而这在 $\mathbb{Z}$ 中是不可能的。
\end{example}

素域的概念对于描述任意的域是关键,以至于我们用素域定义了一个概念,称作域的特征:

\begin{definition}{域的特征}\label{def_field_2}
给定域 $\mathbb{F}$,如果它所包含的素域是 $\mathbb{Z}_p$,那么称 $\mathbb{F}$ 的\textbf{特征(character)}是 $p$;如果它的素域是 $\mathbb{Q}$,那么称它的特征是 $0$。

域$\mathbb{F}$的特征记为$\opn{ch}\mathbb{F}$或者$\opn{char}\mathbb{R}$。
\end{definition}

由定义可见,特征的值取素数或者 $0$,这个值在很大程度上决定了域的代数性质。


\subsection{素域的附加讨论}

\footnote{本节选自小时百科系列教材《代数学》的“初窥抽象代数-环和域-素域”小节}素域的可能形式极为简单。回忆第3.2节\footnote{原书“整数环”一节。}中的讨论,任何幺环必然包括一个子环$\mathbb{Z}_n$,其中$\mathbb{Z}_0=\mathbb{Z}$。域和体当然都是幺环,所以也有相同的性质。由于体比域更抽象,我们接下来就设$\mathbb{F}$为一个体来进行讨论。

如果$\mathbb{F}$中包含的$\mathbb{Z}_n$是$\mathbb{Z}$,那么由于$\mathbb{F}$的每个元素都有乘法逆元,我们就可以把在其中把$\mathbb{Z}$的分式域构造出来,即$\mathbb{Q}$。而有理数域$\mathbb{Q}$就是一个素域。

为什么呢?假设$\mathbb{K}$是$\mathbb{Q}$的一个子域,那么它必须包含$1$这个元素。用$1$不断地和自身相加,我们又必须得到$\mathbb{Z}$本身——不可能得到其它的$\mathbb{Z}_n$,因为那样就违反了$\mathbb{Q}$的运算规律了。接下来,由于$\mathbb{K}$是一个域,因此由乘法和加法的逆运算可知,它必须包含$\mathbb{Z}$和其分式域,结果就是$\mathbb{K}\supseteq\mathbb{Q}$。但又因为按设定,$\mathbb{K}\subseteq\mathbb{Q}$,所以最终就有$\mathbb{K}=\mathbb{Q}$。因此$\mathbb{Q}$的子域只能是它自己,故知$\mathbb{Q}$是一个素域。

如果$\mathbb{F}$中包含的$\mathbb{Z}_n$中$n\neq 0$,那么$n$必然是一个素数。为什么呢?假设存在正整数$a, b$使得$ab=n$,那么在$\mathbb{Z}_n$中就有$ab=0$,这违反了域的乘法规则。

现在,假设$\mathbb{F}$包含一个$\mathbb{Z}_p$,其中$p$为素数。而这个$\mathbb{Z}_p$本身就是一个域。由于域同时也是加法群,而子群的元素数量必须整除原来的群的元素数量,因此含有$p$个元素的$\mathbb{Z}_p$不可能有非平凡\footnote{“非平凡”此处指除了$\{0\}$和其本身之外的情况。}子群,进而不可能有子域。因此$\mathbb{Z}_p$就是一个素域。

结合以上讨论,我们知道了$\mathbb{Q}$和$\mathbb{Z}_p$都是素域,其中$p$是素数。而任何体中,由于加法和乘法的封闭性,必然包含$\mathbb{Q}$或$\mathbb{Z}_p$中的一个。由此还可顺便得出,素体都是素域,这是因为我们上述讨论中都没用到乘法的交换性。












