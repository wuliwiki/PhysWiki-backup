% 勒贝格控制收敛定理
% license Usr
% type Tutor

\begin{issues}
\issueDraft
\end{issues}

\pentry{Lebesgue 积分\nref{nod_Lebes1}}{nod_3b68}

\textbf{勒贝格控制收敛定理(Lebesgue dominated control theorem, DCT)}是测度论和实分析中的一个强大结果,它提供了在一序列函数的积分的极限等于极限函数的积分的条件。这里是该定理的正式陈述:

\begin{theorem}{勒贝格控制收敛定理}
设 $(X,M,\mu)$ 是一个测度空间,让 $\{f_n\}$ 是可测函数的序列 $f_n: X\to\mathbb R$ 或 $f_n:X\to\mathbb C$。 它们几乎处处逐点收敛于函数 $f: X\to\mathbb R$ 或 $f:X\to\mathbb C$。 如果存在一个非负的可积函数 $g:X\to[0,\infty)$, 对所有的 $n$ 和几乎所有的 $x\in X$ 有 $\abs{f_n(x)}\leq g(x)$ 那么 $f$ 是可积的,并且
\begin{equation}
\lim_{n\to\infty}\int f_n(x)\dd{\mu} = \int f(x)\dd{\mu}~.
\end{equation}
\end{theorem}

定理的关键组成部分:
