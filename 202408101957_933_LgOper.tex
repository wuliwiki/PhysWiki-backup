% 大型运算符
% keys 运算符|求和|求积|极限|积分|并集|交集|最值
% license Xiao
% type Wiki

\begin{issues}
\issueDraft
\end{issues}

这篇文章会介绍大型运算符,他们往往代表了一个过程。

\subsection{累计运算过程}
\subsubsection{离散}
累计运算过程一般包括:
\begin{itemize}
\item 析取符号:$\bigvee$
\item 合取符号:$\bigwedge$
\item 并集符号:$\bigcup$
\item 交集符号:$\bigcap$
\item 求和符号:$\sum$
\item 求积符号:$\prod$
\item 上积符号:$\coprod$
\end{itemize}
\subsubsection{连续}
积分符号:$\int$


\subsubsection{记法}

\subsubsection{*无穷情况}
\addTODO{预备知识或者上文没有出现极限符号,另外不要在本文出现微积分内容,这不是综述类百科,标题显示为高中内容}
有时会出现,下限位置为$-\infty$,上限位置为$+\infty$等情况,比如级数等情况。这种情况并非实指去到对应点,而是指取相应极限,即:

\begin{equation}
\sum_{i=-\infty}^n a_i:= \lim_{a\to-\infty}\sum_{i=a}^n a_i.~
\end{equation}

\begin{equation}
\sum_{i=0}^{+\infty} a_i:= \lim_{a\to+\infty}\sum_{i=0}^a a_i.~
\end{equation}

特殊地:
\begin{equation}
\sum_{i=-\infty}^{+\infty} a_i:= \lim_{a\to+\infty\atop b\to-\infty}\sum_{i=b}^a a_i.~
\end{equation}

\begin{example}{设等比级数通项$a_n=a_1q^{n-1}$,用求和符号表示等比级数和}
$$
\sum_{i=1}^{+\infty} a_i=
\leftgroup{
{a_1\over 1-q},\qquad |q|<1\\  
\text{发散},\qquad |q|\geq1
} ~.
$$
\end{example}


\subsection{极限运算过程}

$\lim$

\subsection{极值运算过程}

\begin{itemize}
\item 上确界:$\sup$
\item 下确界:$\inf$
\item 最大值:$\max$
\item 最小值:$\min$
\end{itemize}
