% 电介质的简单模型

\pentry{电偶极子\upref{eleDpl}}

简单的连续电荷模型: 假设\textbf{电介质(dielectric)}是一种完全不导电的物质, 其中有连续均匀分布的正负电荷分布, 电荷密度分布为 $\rho(\bvec r)$. 若电介质中处处无电场, 那么它呈电中性; 若某点有电场, 该点处负电荷不动, 正电荷向电场方向移动 $\bvec d$. 那么我们就把介质中某一点的\textbf{极化密度(polarization density)}定义为
\begin{equation}
\bvec P = \rho \bvec d
\end{equation}
若取一个体积元 $\Delta V$, 把连续电荷分布看成是由许多点电荷构成的, 令其中的正点电荷为 $q_i$, 每个正电荷对应一个负电荷 $-q_i$, 则
\begin{equation}
\bvec P = \sum_i \frac{q_i\bvec d}{V} = \frac{1}{V} \sum_i \bvec p_i
\end{equation}
所以极化密度也可以理解为电偶极矩的体密度.

\subsection{极化强度与电场}
若电解质中某点处电场为 $\bvec E$, 如果该介质是线性的, 该点处的极化强度就与
\begin{equation}
\bvec P = \chi \epsilon_0 \bvec E
\end{equation}
