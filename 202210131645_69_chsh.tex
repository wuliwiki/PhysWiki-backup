% Bell 不等式与 CHSH 不等式
% EPR佯谬|定域隐变量理论|Bell 不等式|CHSH 不等式

\pentry{EPR 佯谬与定域隐变量理论\upref{EPR}}

1964 年,Bell 给出了基于定域隐变量假定得出 Bell 不等式,从而把 EPR 佯谬所引发的一些思辨和哲学性的问题定量化和具体化,得到一个能够通过实验验证的判据.在这之后,1969 年 John Clauser,Michael Horne,Abner Shimony 和 Richard Holt 四人提出了 CHSH 不等式,使得 Bell 检验的实行更容易. 由 CHSH 不等式也可以导出 Bell 不等式,从而得到 Bell 定理,即\textbf{局域隐变量理论}\footnote{爱因斯坦等人于 EPR 佯谬的论文中提出的一个不同于正统量子力学的理论,基于局域实在论的假设,并假设一个完备的理论需要隐变量去刻画.}不能再现量子力学纠缠的某些结果.因此下面将直接给出 CHSH 不等式的证明.

\subsection{定域隐变量理论}
我们仍然以 Bohm 表述下的 Bell 实验为例,假设 Alice 通过用测量方式 $x$ 得到粒子 $A$ 沿 $\bvec n_x$ 方向的自旋.Bob 通过测量方式 $y$ 得到粒子 $B$ 沿 $\bvec n_y$ 方向的自旋.假设 Alice 得到的测量结果为 $a$,Bob 得到的测量结果为 $b$.

\begin{figure}[ht]
\centering
\includegraphics[width=12cm]{./figures/chsh_1.png}
\caption{Bell 实验} \label{chsh_fig1}
\end{figure}

为了为了简化讨论,假设测量结果取值为 $\{-1,1\}$,因此相应的物理量测量算符可以表达为 $\bvec \sigma \cdot \bvec n$(注意到自旋 $\bvec S=\bvec \sigma/2$ 在各个方向上的本征值为 $\pm 1/2$).

\begin{equation}
\begin{aligned}
&P(a|x)=\int \dd \lambda q(\lambda)P(a|x,\lambda),\\
&P(b|y)=\int \dd \lambda q(\lambda)P(b|y,\lambda),\\
&P(ab|xy)=\int \dd \lambda q(\lambda)P(ab|xy,\lambda)
\end{aligned}
\end{equation}

\begin{equation}
P(ab|xy,\lambda)=P(a|x,\lambda)P(b|y,\lambda)
\end{equation}
