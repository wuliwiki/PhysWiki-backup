% 曲线的长度
% keys 曲线|积分|长度|导数|勾股定理
% license Xiao
% type Tutor

\pentry{定积分\nref{nod_DefInt}}{nod_bbe6}

\begin{figure}[ht]
\centering
\includegraphics[width=10cm]{./figures/578602b039ddd5e6.pdf}
\caption{使用一系列直线段近似曲线} \label{fig_CurLen_3}
\end{figure}

\begin{figure}[ht]
\centering
\includegraphics[width=10cm]{./figures/9ddac280c9cc773a.pdf}
\caption{在一点附近用一直线段近似曲线} \label{fig_CurLen_1}
\end{figure}

直角坐标系中, 曲线通常用函数 $y(x)$ 描述。 在计算曲线长度时,我们使用一系列直线段来“近似”曲线。具体地说,如\autoref{fig_CurLen_1} 所示,我们在曲线上某点 $(x, y)$ 附近用一直线段近似曲线。根据勾股定理, 直线段的长度为
\begin{equation}
\dd{l} = \sqrt{\dd{x}^2 + \dd{y}^2} = \sqrt{1 + \dot y^2} \dd{x}~.
\end{equation}
两边积分得
\begin{equation}\label{eq_CurLen_1}
l = \int_{x_1}^{x_2} \sqrt{1 + \dot y^2} \dd{x}~.
\end{equation}
这就是曲线在区间 $[x_1, x_2]$ 的长度。

\begin{example}{抛物线}
对抛物线 $y=\frac{x^2}{2p}$, 在区间 $[0,x]$ 对应的弧长为
\begin{equation}
\begin{aligned}
l& = \frac{1}{p}\int_{0}^{x}\sqrt{x^2+p^2}\dd{x}\\
&=\frac{1}{p}\bigg[\frac{1}{2}x\sqrt{x^2+p^2}+\frac{p^2}{2}\ln(x+\sqrt{x^2+p^2})\bigg]\Bigg\lvert_{0}^{x}\\
&=\frac{x}{2p}\sqrt{x^2+p^2}+\frac{p}{2}\ln\frac{x+\sqrt{x^2+p^2}}{p}~.
\end{aligned}
\end{equation}
\end{example}

你一定很想问,为什么我们不能像计算面积时一样,用无限小矩形的边长来近似曲线的长度呢?这其实是一个\textsl{小学二年级}的、古老的脑筋急转弯问题。你总能把每个矩形的各边都平移到下方与左方。最终你会发现,如\autoref{fig_CurLen_2} 所示,不论你如何细致地划分矩形,你得到的“长度”都不是曲线的长度。
\begin{figure}[ht]
\centering
\includegraphics[width=10cm]{./figures/f535004b5a016af2.pdf}
\caption{平移矩形的边} \label{fig_CurLen_2}
\end{figure}

\subsection{含参曲线}
若平面上的曲线可以用参数 $t$ 表示为 $x(t), y(t)$(三维情况同理), 那么 $t \in [t_1, t_2]$ 对应的一段曲线长度为
\begin{equation}\label{eq_CurLen_2}
l = \int_{t_1}^{t_2} \sqrt{\dot x^2 + \dot y^2} \dd{t}~,
\end{equation}
推导和上面的过程类似。

\begin{example}{椭圆的弧长}\label{ex_CurLen_1}
椭圆的参数方程为(\autoref{eq_Elips3_1}~\upref{Elips3})
\begin{equation}
\leftgroup{
&x(t) = a\cos t\\
&y(t) = b\sin t
} \qquad
(a > b > 0)~.
\end{equation}
由于椭圆的离心率为 $e = \sqrt{1 - b^2/a^2}$, 椭圆上 $t \in [0, \varphi]$ 对应的圆弧长度等于
\begin{equation}
\begin{aligned}
l &= \int_0^{\varphi} \sqrt{a^2\sin^2 t + b^2 \cos^2 t} \dd{t}
= b\int_0^\varphi \sqrt{1 - e^2\sin^2 t} \dd{t}\\
&= b E(\varphi, e)~.
\end{aligned}
\end{equation}
其中函数 $E(x)$ 是第二类不完整椭圆积分(\autoref{eq_EliInt_3}~\upref{EliInt})。

椭圆的周长 $c$ 为 $t \in [0, \pi/2]$ 对应弧长的 4 倍, 即 $c = 4bE(\pi/2, e)$。
\end{example}
