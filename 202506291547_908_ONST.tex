% 欧内斯特·劳伦斯(综述)
% license CCBYSA3
% type Wiki

本文根据 CC-BY-SA 协议转载翻译自维基百科\href{https://en.wikipedia.org/wiki/Ernest_Lawrence}{相关文章}。

欧内斯特·奥兰多·劳伦斯(Ernest Orlando Lawrence,1901年8月8日 – 1958年8月27日)是美国的加速器物理学家,因发明回旋加速器而获得1939年诺贝尔物理学奖。他因在曼哈顿计划中进行铀同位素分离的工作而闻名,也因创办了劳伦斯伯克利国家实验室和劳伦斯利物浦国家实验室而著名。

劳伦斯毕业于南达科他大学和明尼苏达大学,1925年在耶鲁大学获得物理学博士学位。1928年,他被聘为加利福尼亚大学伯克利分校的物理学副教授,并在两年后成为该校最年轻的正教授。在某个晚上,他在图书馆里看到了一个产生高能粒子的加速器示意图,这引起了他的兴趣。他思考如何将其做得更加紧凑,最终想出了一个思路,即将加速室设计成一个圆形,置于电磁铁的两极之间。由此诞生了第一台回旋加速器。

劳伦斯随后建造了一系列越来越大、成本越来越高的回旋加速器。他的辐射实验室在1936年成为加利福尼亚大学的一个官方部门,劳伦斯担任其主任。除了回旋加速器在物理学中的应用,劳伦斯还支持其在放射性同位素医学应用研究中的使用。在第二次世界大战期间,劳伦斯在辐射实验室开发了电磁同位素分离技术。该技术使用了被称为“卡鲁特朗”的装置,这是一种结合了标准实验室质谱仪和回旋加速器的混合型设备。在田纳西州橡树岭建造了一个巨大的电磁分离工厂,后来被称为Y-12。这个过程效率低下,但它成功地实现了目标。

战后,劳伦斯广泛倡导政府资助大型科学项目,并且是“大科学”理念的有力支持者,这一理念要求大型机器和大量资金。劳伦斯强烈支持爱德华·泰勒争取建立第二个核武器实验室的运动,这个实验室最终设在加利福尼亚州利物浦。劳伦斯去世后,加利福尼亚大学的理事会将劳伦斯利物浦国家实验室和劳伦斯伯克利国家实验室以他的名字命名。化学元素103号被命名为劳伦素,以表彰他在1961年伯克利发现这一元素。
\subsection{早年生活}
欧内斯特·奥兰多·劳伦斯于1901年8月8日出生在南达科他州的坎顿。他的父母卡尔·古斯塔夫斯(1871–1954)和冈达·雷吉娜(原姓雅各布森)(1874–1959)均为挪威移民的后代,他们在坎顿的一所高中教授课程,劳伦斯的父亲还担任该校的校长。他有一个弟弟约翰·H·劳伦斯,后来成为一名医生,并在核医学领域开创了先河。在成长过程中,他最好的朋友是梅尔·图夫,这位朋友后来也成为了一位杰出的物理学家。

劳伦斯就读于坎顿和皮埃尔的公立学校,然后进入明尼苏达州北菲尔德的圣奥拉夫学院,但在一年后转学到南达科他州的南达科他大学。他于1922年获得化学学士学位,1923年在明尼苏达大学获得物理学硕士(M.A.)学位,导师是威廉·弗朗西斯·格雷·斯旺。作为硕士论文,劳伦斯设计了一种实验装置,通过磁场旋转一个椭球体。

劳伦斯跟随斯旺先后进入芝加哥大学和耶鲁大学,最终在耶鲁大学完成了他的物理学博士(PhD)学位,获得国家研究员奖学金,并于1925年撰写了关于钾蒸气光电效应的博士论文。他被选为Sigma Xi会员,并在斯旺的推荐下,获得了国家研究委员会的奖学金。与当时通常前往欧洲不同,他选择留在耶鲁大学,继续作为研究员与斯旺合作。

在与弗吉尼亚大学的杰西·比姆斯合作后,劳伦斯继续研究光电效应。他们证明了光电子在光子撞击光电表面后的$2 \times10^{-9}$秒内就会出现——这接近当时的测量极限。通过快速开关光源来缩短发射时间,使得发射的能量谱变得更宽,符合维尔纳·海森堡的不确定性原理。
\subsection{早期职业生涯}
1926年和1927年,劳伦斯分别收到来自华盛顿大学和加利福尼亚大学的助理教授职位邀请,年薪为3,500美元(相当于2024年的63,400美元)。耶鲁大学迅速匹配了这个职位,但年薪为3,000美元。劳伦斯选择留在更具声望的耶鲁大学,但由于他从未担任过讲师,部分同事对此职位表示不满,许多人认为这并未能弥补他南达科他移民背景的不足。

1928年,劳伦斯被聘为加利福尼亚大学的物理学副教授,两年后成为正教授,成为该校最年轻的教授。基于弗雷德里克和伊莲娜·居里于1934年发表的关于人工放射性的研究成果,劳伦斯通过在他的实验室中用高能质子轰击碳-13元素,发现了氮-13同位素。他和他的团队,包括马丁·卡门和塞缪尔·鲁本,在用高能质子轰击石墨时意外发现了碳-14同位素。罗伯特·戈登·斯普劳尔在劳伦斯成为教授的第二天成为了加利福尼亚大学的校长,他是博希米亚俱乐部的成员,并在1932年赞助了劳伦斯加入该俱乐部。通过这个俱乐部,劳伦斯结识了威廉·亨利·克罗克、埃德温·保利和约翰·弗朗西斯·内兰。这些有影响力的人物帮助他为他的核粒子研究筹集资金。人们对粒子物理学的医学应用充满了巨大的期望,这也促成了劳伦斯能够获得早期研究资金的大部分来源。

在耶鲁大学时,劳伦斯遇到了玛丽·金伯利(莫莉)·布鲁默,她是耶鲁大学医学院院长乔治·布鲁默的四个女儿中的长女。他们于1926年初次见面,并于1931年订婚,1932年5月14日,在康涅狄格州纽黑文的三一教堂举行了婚礼。他们育有六个孩子:埃里克、玛格丽特、玛丽、罗伯特、芭芭拉和苏珊。劳伦斯将他的儿子罗伯特命名为以纪念他的亲密朋友、理论物理学家罗伯特·奥本海默,后者在伯克利与他关系深厚。1941年,莫莉的妹妹埃尔西与埃德温·麦克米兰结婚,麦克米兰后来与格伦·T·希博格一起获得了1951年诺贝尔化学奖。
\subsection{回旋加速器的开发}
\subsubsection{发明}
使劳伦斯声名鹊起的发明,起初仅是一个草图,画在一张纸餐巾上。1929年某个晚上,劳伦斯在图书馆里翻阅一篇罗尔夫·维德尔厄的期刊文章时,被其中的一幅图示吸引。这幅图描绘了一种通过一系列小的“推动”来产生高能粒子的装置。图中的装置按直线排列,使用越来越长的电极。当时,物理学家们开始探索原子核。1919年,新西兰物理学家欧内斯特·拉塞福德将阿尔法粒子射入氮原子,成功地将一些核中的质子撞击出来。但是,原子核带有正电荷,会排斥其他带正电的原子核,并且它们通过物理学家刚开始理解的力紧密结合在一起。要打破它们、使它们解体,需要更高的能量,达到数百万伏特的级别。
\begin{figure}[ht]
\centering
\includegraphics[width=10cm]{./figures/faed33bdaaa7ac28.png}
\caption{劳伦斯1934年专利中的回旋加速器操作示意图} \label{fig_ONST_1}
\end{figure}
劳伦斯意识到,这种粒子加速器很快就会变得过长且难以操作,不适合他的大学实验室。在思考如何使加速器更紧凑时,劳伦斯决定将一个圆形加速室放置在电磁铁的两极之间。磁场会将带电的质子保持在螺旋路径上,同时它们在两个连接到交变电压的半圆形电极之间加速。经过大约一百圈,质子将以高能粒子束的形式撞击靶标。劳伦斯兴奋地告诉同事们,他发现了一种方法,可以在不使用任何高电压的情况下获得非常高能量的粒子。他最初与尼尔斯·埃德尔夫森合作。他们的第一个回旋加速器是用黄铜、金属丝和密封蜡制成的,直径只有四英寸(10厘米)——可以用一只手拿着,成本大约为25美元(相当于2024年的600美元)。

为了发展这一想法,劳伦斯需要有能力的研究生来进行工作。埃德尔夫森于1930年9月离开,去担任助理教授,劳伦斯用大卫·H·斯洛恩和M·斯坦利·李文斯顿替代了他,分别让他们着手开发维德尔厄的加速器和埃德尔夫森的回旋加速器。两人都有自己的资金支持。这两种设计都证明是可行的,到1931年5月,斯洛恩的线性加速器已能够将离子加速到1 MeV。李文斯顿面临更大的技术挑战,但当他在1931年1月2日对他的11英寸回旋加速器施加1,800伏特电压时,他成功让80,000电子伏的质子旋转起来。一周后,他使用3,000伏特电压达到了1.22 MeV,这远远足够支撑他关于回旋加速器构建的博士论文。
\subsubsection{发展}
\begin{figure}[ht]
\centering
\includegraphics[width=8cm]{./figures/f1028083d22e9a26.png}
\caption{1940年在伯克利举行的会议,讨论计划中的184英寸(4.67米)回旋加速器(见黑板上的示意图):劳伦斯、阿瑟·康普顿、范尼瓦尔·布什、詹姆斯·B·科南特、卡尔·T·康普顿和阿尔弗雷德·李·卢米斯。} \label{fig_ONST_2}
\end{figure}
在接收到成功的初步信号后,劳伦斯开始规划一台更大、更强的机器,这成了一个反复出现的模式。1932年初,劳伦斯和李文斯顿设计了一台27英寸(69厘米)的回旋加速器。800美元的11英寸回旋加速器所用的磁铁重达2吨,但劳伦斯在帕洛阿尔托的一个废品堆场里发现了一个原本用于第一次世界大战时支撑跨大西洋无线电连接的巨大80吨磁铁,用于27英寸回旋加速器。在这个回旋加速器中,劳伦斯拥有了一种强大的科学仪器,但这并未转化为科学发现。1932年4月,约翰·科克罗夫特和欧内斯特·沃尔顿在英国剑桥大学的卡文迪许实验室宣布,他们用质子轰击锂并成功将其转变为氦。所需的能量非常低——完全在11英寸回旋加速器的能力范围内。得知此事后,劳伦斯向伯克利发了一封电报,要求验证科克罗夫特和沃尔顿的实验结果。该团队直到9月才完成验证,主要由于缺乏足够的探测设备。

尽管重要的发现继续与劳伦斯的辐射实验室擦肩而过,主要是因为实验室专注于回旋加速器的发展而非其科学用途,但通过他日益增大的机器,劳伦斯能够为高能物理实验提供至关重要的设备。围绕这一设备,他建立了世界上最顶尖的核物理学研究实验室,这个实验室在1930年代成为了核物理新领域的领先研究中心。他于1934年获得了回旋加速器的专利,并将其转让给了研究公司,这是一个资助劳伦斯早期大部分工作的私人基金会。

1936年2月,哈佛大学校长詹姆斯·B·科南特向劳伦斯和奥本海默提出了有吸引力的邀请。加利福尼亚大学校长罗伯特·戈登·斯普劳尔通过改善条件作出回应。1936年7月1日,辐射实验室成为加利福尼亚大学的一个官方部门,劳伦斯正式被任命为其主任,配有全职副主任,且大学同意每年拨出20,000美元用于实验室的研究活动(相当于2023年的350,000美元)。劳伦斯采用了一个简单的商业模式:“他用物理系的研究生和初级教职员工、愿意为任何薪水工作的刚获得博士学位的人、以及能无偿服务的奖学金获得者和富有的客人来为实验室配备人员。”
\subsubsection{反响}
利用新建的27英寸回旋加速器,伯克利的团队发现,他们用新发现的氘轰击的每种元素都会释放能量,而且释放的能量在相同的范围内。因此,他们推测存在一种新的、迄今未知的粒子,可能是源源不断的能量来源。纽约时报的威廉·劳伦斯将劳伦斯描述为“科学界的新奇迹工作者”。应科克罗夫特的邀请,劳伦斯参加了1933年在比利时召开的索尔维大会。这是全球顶级物理学家的定期聚会,几乎所有与会者都来自欧洲,但偶尔像罗伯特·A·米利肯或阿瑟·康普顿这样的杰出美国科学家会被邀请参加。劳伦斯被邀请在会上展示回旋加速器的研究成果。

劳伦斯关于无尽能源的主张在索尔维大会上得到了截然不同的反响。他遭遇了来自卡文迪许实验室的詹姆斯·查德威克的强烈怀疑。查德威克是1932年发现中子并因此获得1935年诺贝尔奖的物理学家。查德威克用一种带有轻蔑语气的英国口音对劳伦斯说,他认为劳伦斯的团队所观察到的现象是他们的仪器遭到污染。
\begin{figure}[ht]
\centering
\includegraphics[width=8cm]{./figures/ac6850fa5a08f73e.png}
\caption{1939年完成后的60英寸(1.52米)回旋加速器。图中展示了其开发和使用的关键人物,站立从左到右依次为:唐纳德·库克西、戴尔·R·科尔森、欧内斯特·劳伦斯、罗伯特·L·桑顿、约翰·巴克斯和温菲尔德·索尔兹伯里。背景中是路易斯·阿尔瓦雷兹和埃德温·麦克米兰。} \label{fig_ONST_3}
\end{figure}
当劳伦斯回到伯克利时,他动员团队仔细检查结果,收集足够的证据以说服查德威克。与此同时,在卡文迪许实验室,拉塞福德和马克·奥利凡特发现,氘与氘发生聚变形成氦-3,这正是回旋加速器团队所观察到的现象。查德威克不仅正确地指出他们所观察到的是仪器污染,此外,他们还忽视了另一个重要发现——核聚变的现象。劳伦斯的回应是继续推动更大回旋加速器的建设。27英寸回旋加速器于1937年6月被一台37英寸回旋加速器取代,随后在1939年5月,37英寸回旋加速器又被60英寸回旋加速器取代。该加速器用于轰击铁,并于6月首次产生了放射性同位素。

由于为医学用途筹集资金,特别是癌症治疗,通常比为核物理研究筹集资金更容易,劳伦斯鼓励将回旋加速器用于医学研究。劳伦斯与他的弟弟约翰和加利福尼亚大学生理学系的以色列·莱昂·查伊科夫合作,支持使用放射性同位素进行治疗用途的研究。磷-32在回旋加速器中易于产生,约翰利用它治愈了一位患有多血症的女性,这是一种血液疾病。约翰还在1938年使用37英寸回旋加速器产生的磷-32,对患有白血病的小鼠进行测试。他发现,放射性磷集中在快速生长的癌细胞中。这一发现随后导致了对人类患者的临床试验。1948年对这种治疗方法的评估显示,在某些情况下确实出现了缓解。劳伦斯还希望中子能够用于医学治疗。第一位癌症患者于11月20日接受了来自60英寸回旋加速器的中子治疗。查伊科夫进行了一项实验,使用放射性同位素作为放射性示踪剂,以探索生化反应的机制。
\begin{figure}[ht]
\centering
\includegraphics[width=8cm]{./figures/71ea0980a3ef4c69.png}
\caption{加利福尼亚大学辐射实验室的工作人员在60英寸回旋加速器的磁铁框架内,1938年;图中展示了诺贝尔奖得主欧内斯特·劳伦斯、埃德温·麦克米兰和路易斯·阿尔瓦雷兹,以及J·罗伯特·奥本海默和罗伯特·R·威尔逊。} \label{fig_ONST_4}
\end{figure}
劳伦斯于1939年11月因“发明和发展回旋加速器以及利用其获得的成果,特别是在人工放射性元素方面的成就”而获得诺贝尔物理学奖。他是伯克利大学的第一位诺贝尔奖得主,也是南达科他州的第一位诺贝尔奖得主,同时也是在公立大学工作期间获得此荣誉的第一人。由于第二次世界大战,诺贝尔颁奖典礼于1940年2月29日在加利福尼亚伯克利的惠勒大厅举行。劳伦斯从瑞典驻旧金山总领事卡尔·E·瓦勒斯特德手中接过了奖章。罗伯特·W·伍德写信给劳伦斯,并预见性地写道:“当你为铀的灾难性爆炸奠定基础时……我相信诺贝尔先生一定会赞同。”

1940年3月,阿瑟·康普顿、范尼瓦尔·布什、詹姆斯·B·科南特、卡尔·T·康普顿和阿尔弗雷德·李·卢米斯前往伯克利,讨论劳伦斯关于建设一台184英寸回旋加速器的提议,该加速器将配备一台估计重达4,500吨的磁铁,预计成本为265万美元(相当于2023年的4,500万美元)。洛克菲勒基金会为该项目提供了115万美元的启动资金。
\subsection{第二次世界大战与曼哈顿计划}
\subsubsection{辐射实验室}
在欧洲爆发第二次世界大战后,劳伦斯被卷入了军事项目。他帮助为麻省理工学院辐射实验室招募人员,该实验室的美国物理学家开发了由马克·奥利凡特团队在英国发明的腔体磁控管。新实验室的名字故意模仿了劳伦斯在伯克利的实验室,以出于安全考虑。他还参与了为水下声学实验室招募人员的工作,以开发侦测德国潜艇的技术。与此同时,伯克利的回旋加速器工作继续进行。1940年12月,格伦·T·希博格和埃米利奥·塞格雷使用60英寸(150厘米)回旋加速器,用氘轰击铀-238,产生了一种新元素——海王星-238,该元素通过β衰变形成钚-238。其同位素钚-239能够发生核裂变,这为制造原子弹提供了另一种途径。

劳伦斯向塞格雷提供了一份研究助理的工作——对于发现了一个元素的人来说,这个职位相对较低,薪水为每月300美元,期限为六个月。然而,当劳伦斯得知塞格雷因法律原因被困在加利福尼亚时,他将塞格雷的工资进一步减少至每月116美元。当加利福尼亚大学的理事会因塞格雷是外国国籍而试图终止其工作时,劳伦斯设法通过将塞格雷聘为由洛克菲勒基金会资助的兼职讲师来保留他。当塞格雷的博士生吴健雄(中国国籍)和肯尼斯·罗斯·麦肯齐(加拿大国籍)毕业时,劳伦斯也为他们做出了类似的安排,以便他们能继续留在实验室工作。
\begin{figure}[ht]
\centering
\includegraphics[width=8cm]{./figures/63fb59c2c0a9347d.png}
\caption{卡鲁特朗中铀同位素分离的示意图} \label{fig_ONST_5}
\end{figure}
1941年9月,奥利凡特在伯克利与劳伦斯和奥本海默会面,他们向他展示了新建的184英寸(4.7米)回旋加速器的选址。奥利凡特反过来批评美国人未能落实英国MAUD委员会的建议,该委员会提议开发原子弹项目。劳伦斯已经考虑过分离裂变同位素铀-235与铀-238的问题,这一过程今天被称为铀浓缩。分离铀同位素非常困难,因为这两种同位素具有几乎相同的化学性质,只能通过它们微小的质量差异逐渐分离。使用质谱仪分离同位素是奥利凡特在1934年通过对锂的研究开创的技术。

劳伦斯开始将他旧的37英寸回旋加速器改造成一个巨大的质谱仪。根据他的建议,曼哈顿计划的负责人莱斯利·R·格罗夫斯准将军任命奥本海默为新墨西哥州洛斯阿拉莫斯实验室的负责人。与此同时,辐射实验室开发了电磁铀浓缩过程,洛斯阿拉莫斯实验室则设计并建造了原子弹。与辐射实验室一样,洛斯阿拉莫斯实验室由加利福尼亚大学管理。

电磁同位素分离使用了名为“卡鲁特朗”的装置,它是质谱仪和回旋加速器这两种实验室仪器的混合体。这个名字来源于“加利福尼亚大学回旋加速器”(California university cyclotrons)。1943年11月,劳伦斯的伯克利团队增加了29名英国科学家,其中包括奥利凡特。

在电磁过程中,磁场根据质量偏转带电粒子。这个过程既不科学优雅,也不具有工业效率。与气体扩散工厂或核反应堆相比,电磁分离工厂需要消耗更多稀缺材料,操作需要更多劳动力,建设成本也更高。然而,仍然批准了这一过程,因为它基于成熟的技术,因此代表了较小的风险。而且,它可以分阶段建造,并能迅速达到工业化生产能力。
\subsubsection{橡树岭}
位于田纳西州橡树岭的电磁分离工厂的设计和建设责任被分配给了Stone & Webster公司。卡鲁特朗使用了14,700吨银,由密尔沃基的Allis-Chalmers公司制造,并运送到橡树岭。设计要求包括五个第一阶段处理单元,称为Alpha赛道,以及两个最终处理单元,称为Beta赛道。1943年9月,格罗夫斯批准建设另外四个赛道,称为Alpha II。当工厂按计划于1943年10月启动进行测试时,由于磁铁的强大力量,14吨的真空罐发生了对准偏差,必须将其更加牢固地固定。更严重的问题出现在磁线圈开始短路时。12月,格罗夫斯命令打开一个磁铁,发现里面满是铁锈。随后,格罗夫斯命令拆除赛道,并将磁铁送回工厂进行清洁。现场建立了一个酸洗厂,用于清洁管道和配件。
\begin{figure}[ht]
\centering
\includegraphics[width=8cm]{./figures/41daf05b7acb4726.png}
\caption{大规模电磁铁Alpha I赛道,用于田纳西州橡树岭Y-12工厂的铀浓缩,约1944–45年。劳伦斯开发的卡鲁特朗位于环形结构周围。} \label{fig_ONST_6}
\end{figure}
田纳西东曼公司被聘请来管理Y-12。Y-12最初将铀-235的浓度提高到13\%到15\%之间,并在1944年3月将最初几百克的铀运送到洛斯阿拉莫斯实验室。最终产品的铀含量是铀原料的五千八百二十五分之一。其余的铀则被分散到过程中的设备上。经过艰苦的回收努力,到了1945年1月,铀-235的产量达到了铀原料的10\%。到2月,Alpha赛道开始接收来自新的S-50热扩散厂的轻度浓缩(1.4\%)铀。下个月,它开始接收来自K-25气体扩散厂的增强型(5\%)铀。到1945年4月,K-25开始生产足够浓缩的铀,直接供给Beta赛道使用。

1945年7月16日,劳伦斯与查德威克和查尔斯·A·托马斯一起观察了第一颗原子弹的三位一体核试验。没有人比劳伦斯对其成功更加兴奋。如何将这种现在已具备功能的武器用于日本成为科学家们讨论的问题。虽然奥本海默主张不向日本领导人展示新武器的威力,劳伦斯则坚信进行展示是明智的。当铀弹在没有预警的情况下被用于广岛的原子弹轰炸时,劳伦斯为自己的成就感到无比自豪。

劳伦斯希望曼哈顿计划能开发出改进版的卡鲁特朗,并建设Alpha III赛道,但这些被认为是经济上不可行的。Alpha赛道在1945年9月关闭。尽管它们的表现达到了历史最佳水平,但仍无法与K-25和新建的K-27竞争,后者在1946年1月开始运行。12月,Y-12工厂关闭,田纳西东曼公司的员工从8600人减少到1500人,每月节省了200万美元。辐射实验室的员工数量也从1945年5月的1086人下降到年底的424人。
\subsection{战后职业生涯}
\subsubsection{大科学}
战后,劳伦斯广泛倡导政府资助大型科学项目。他是“大科学”的有力支持者,倡导大机器和大资金的需求。1946年,他向曼哈顿计划申请了超过200万美元的资金,用于辐射实验室的研究(相当于2023年的2400万美元)。格罗夫斯批准了这笔资金,但削减了多个项目,包括希博格关于在人口密集的伯克利建立“热”辐射实验室的提议,以及约翰·劳伦斯关于生产医学同位素的提议,因为现在可以通过核反应堆更好地满足这一需求。一个障碍是加利福尼亚大学,该校急于解除其战时的军事责任。劳伦斯和格罗夫斯成功说服斯普劳尔接受合同延期。1946年,曼哈顿计划在加利福尼亚大学的物理学研究上的投入是大学自身支出的七倍。

184英寸回旋加速器是用曼哈顿计划的战时资金完成的。它融合了埃德·麦克米兰的新思路,并以同步回旋加速器的形式完成。它于1946年11月13日开始运行。自1935年以来,劳伦斯首次积极参与实验,与尤金·加德纳合作,尝试利用同步加速器创造最近发现的π介子,但未能成功。然后,塞萨尔·拉特斯使用他们创造的装置,在1948年发现了负π介子。

国家实验室的管理责任于1947年1月1日移交给新成立的原子能委员会(AEC)。同年,劳伦斯请求为他的项目拨款1,500万美元(相当于2023年的1.61亿美元),这些项目包括一个新的线性加速器和一个新的吉电子伏同步加速器,后来被称为贝伐特龙。加利福尼亚大学管理洛斯阿拉莫斯实验室的合同将于1948年7月1日到期,部分董事会成员希望将大学从管理加利福尼亚州以外的实验室的责任中解脱出来。经过一些谈判,大学同意将合同延长四年,并任命诺里斯·布拉德伯里为教授,他在1945年10月取代奥本海默成为实验室的主任。不久之后,劳伦斯收到了他所请求的所有资金。
\begin{figure}[ht]
\centering
\includegraphics[width=6cm]{./figures/6b1720ca2a84577f.png}
\caption{} \label{fig_ONST_7}
\end{figure}
尽管劳伦斯投票支持富兰克林·罗斯福,他实际上是一个共和党人,他强烈反对奥本海默在战前尝试使辐射实验室的工人加入工会,认为这属于“左倾活动”。劳伦斯认为政治活动是浪费时间,应该将时间更好地用于科学研究,他更倾向于将政治活动排除在辐射实验室之外。在冷战后期,加利福尼亚大学的冷酷氛围中,劳伦斯认为众议院反美活动委员会的行动是合法的,并不认为这些行动反映了涉及学术自由或人权的系统性问题。他保护实验室中的个人,但更关心实验室的声誉。他被迫为辐射实验室的成员辩护,例如罗伯特·瑟伯,他们曾被大学的人员安全委员会调查。在几个案例中,他为实验室人员出具了证明信。然而,劳伦斯禁止罗伯特·奥本海默的弟弟弗兰克进入辐射实验室,这损害了他与罗伯特的关系。加利福尼亚大学的忠诚宣誓运动也导致了一些教职工的离开。罗伯特·奥本海默的安全许可被撤销时,劳伦斯因病拒绝参加听证会,但在他缺席的情况下,呈交了一份他批评奥本海默的文字记录。劳伦斯在建立一个富有创意和合作精神的实验室方面的成功,被政治紧张局势所带来的不满和不信任所削弱。
