% 金属的变形(科普)

%\textsl{Iron and steel will bend and break, bend and break, bend and break; Iron and steel will bend and break, my fair lady. ---童谣 London Bridge is Falling Down}
\pentry{金属材料结构(科普)\upref{MetInt}}
\subsection{变形}
\footnote{本文参考了刘智恩的《材料科学基础》与Callister的Material Science and Engineering An Introduction. 部分文本、图片来自网络。本文适用于CC-BY-SA。}
正如我们拉伸一根弹簧,弹簧会变形一样;当我们拉伸一根金属棒时,金属棒也会变形。只不过由于金属棒的“弹性系数”很大,以正常人的手劲一般拉不出看得见的变形。

\begin{example}{}
\begin{figure}[ht]
\centering
\includegraphics[width=12cm]{./figures/b948a8df309da225.png}
\caption{框架结构} \label{fig_MetDfm_1}
\end{figure}
与弹簧类似,金属结构提供的支持力也源自金属的细微变形。。。只要在安全的范围内。
\end{example}

根据变形的性质,变形一般分为两类:弹性变形与塑性变形。顾名思义,弹性变形后,撤去外力后金属的形状能恢复原样;而塑性变形后,即使撤去外力,金属的形状也不能恢复。塑性变形只在外力大到超过一定境界时才发生。

弹性变形与塑性变形不是非此即彼,而可以相辅相成。一次变形可能既包括弹性变形也包括塑性变形。

\begin{figure}[ht]
\centering
\includegraphics[width=6cm]{./figures/9f26e797ac1e2548.pdf}
\caption{弹性变形与塑性变形示意图} \label{fig_MetDfm_2}
\end{figure}

那么,为什么会有两种类型的变形呢?这就涉及到变形的微观原理了。大体而言,弹性变形时原子间的“键”被拉伸,但原子并没有运动到新的位置。因此撤去外力后原子可以回到原位,体现为形状恢复原样;
\begin{figure}[ht]
\centering
\includegraphics[width=8cm]{./figures/cf8c48390344cd90.pdf}
\caption{弹性变形示意图} \label{fig_MetDfm_11}
\end{figure}
而塑性变形后,原子间原本的键已经被破坏、原子运动到了新的位置,并形成了新的键。因此撤去外力后原子不能回到原位,体现为形状永久改变。
\begin{figure}[ht]
\centering
\includegraphics[width=10cm]{./figures/cfe7193b51ece88a.pdf}
\caption{塑性变形示意图} \label{fig_MetDfm_12}
\end{figure}

\subsection{弹性变形机制}
\begin{figure}[ht]
\centering
\includegraphics[width=8cm]{./figures/2584baf079a6bee0.pdf}
\caption{原子间作用力是吸引力与斥力的组合} \label{fig_MetDfm_5}
\end{figure}
原子\footnote{我们此处假设物体是直接由原子组成的。实际物体还可以由分子、离子组成,但这并不影响此处的定性结论}依靠原子间作用力(或者说,键\footnote{根据强度与具体机制等,键可以再细化为化学键(金属键、共价键、离子键)与分子间作用力等})连接在一起。原子间作用力的具体机制非常复杂,但简化的来说,原子间作用力可以理解为一组电荷交互作用,包括电子间的排斥力与原子核对电子的吸引力等。

\begin{figure}[ht]
\centering
\includegraphics[width=10cm]{./figures/642ca7b519d09efc.pdf}
\caption{原子间作用力示意图} \label{fig_MetDfm_4}
\end{figure}

这对吸引力与排斥力导致了一种微妙的平衡。当物体不受外力时,吸引力与斥力最终将相同并让合力为零,并使原子保持一定的平衡间距。而当外力试图拉开原子时,原子间距的增大使吸引力大于斥力,合力倾向于吸引被拉开的原子;同样的,当外力试图压缩原子时,合力倾向于弹开被压缩的原子。这就是为什么物体体现出抵抗外力作用的“弹性”。

这顺带解释了固体能保持一定形态,而液体、气体却相对松散的原因:固体中粒子间的作用力远强于液体、气体。

\subsection{塑性变形机制}

\subsubsection{理想与现实的矛盾}
现在,我们来更具体地思考一下,塑性变形时金属经历了什么。假设你有一块完整的晶体,现在你要施加外力使其塑性变形。看起来,为了使原子运动到新位置,你得大力出奇迹、破坏一整面原子间的键。
\begin{figure}[ht]
\centering
\includegraphics[width=14cm]{./figures/6a396aa1edc5803a.pdf}
\caption{塑性变形...?} \label{fig_MetDfm_13}
\end{figure}
毫无疑问,这需要非常大的能量(\textsl{与手劲})!可事实上,现实中的金属强度远低于此(大概是按照这种理论计算的1/100至1/1000)。怎么回事呢?

\subsubsection{位错的运动}
或许你还记得位错\upref{MetInt}的概念。位错理论的提出正是为了解释金属的塑性变形。假定金属中存在一个位错,这时,奇迹发生了:由于位错的存在,现在上下部分相对运动时,只需断一列的键而不是一面的键!这大大降低了原子运动的难度。
\begin{figure}[ht]
\centering
\includegraphics[width=14cm]{./figures/820effe48d9d7b4a.pdf}
\caption{位错的运动} \label{fig_MetDfm_3}
\end{figure}

在\href{https://m.sohu.com/a/350972084_120056486/}{这里}可以看到一张动图。(站外链接)

这有点像蠕动的毛毛虫,或者地板上鼓起包的地毯。直接推动铺平了的地毯很难,但推动地毯上鼓起的凸包就容易地多。

这就是\textsl{滑移}机制,金属塑性形变的主要机制之一。当然,这并不是故事的全部,而且本文只探讨了最理想的情况,例如假定金属中的原子排列方向是完全一致的,即整个金属中只有一个硕大的晶粒。

\subsubsection{位错运动的方向:滑移系}
那么,位错能够在任意的面或方向上运动吗?答案显然又是。。。否定的。\textbf{一般而言,位错只在它偏好的运动方向与运动面上滑移。}这里的“面”不是专指具体的某一个面,而指的是一系列法向量特定的相互平行的面。

\begin{theorem}{滑移系,滑移面,滑移方向}
一般而言,滑移在特定的面上,沿着特定的方向进行。

这些特定的面与方向分别称为滑移面与滑移方向;一种滑移面与其上的一种滑移方向称为一个滑移系。

\end{theorem}
\begin{figure}[ht]
\centering
\includegraphics[width=6cm]{./figures/2a1fe432bb72e0b5.pdf}
\caption{一种可能的滑移系示意图,切面为滑移面,黑色线为滑移方向} \label{fig_MetDfm_6}
\end{figure}

那么,位错偏好哪一些滑移面或滑移方向呢?一般而言,位错偏好的\textbf{滑移面是晶体的密排面,而滑移方向是晶体的密排方向}(密排面可以理解为晶体中原子堆积得最密的面)。可见,可能的滑移系与晶胞的种类有关\footnote{滑移系还与温度有关,有些额外的滑移系将在高温下被激活。因此高温时,晶体的变形能力更好。}。由于一种晶体中密排面、方向不止一种,因此滑移系也不止一种。

\begin{example}{钛强了}
\begin{figure}[ht]
\centering
\includegraphics[width=4cm]{./figures/995962f2a3ac37da.png}
\caption{密排六方晶胞示意图,\href{https://www.geogebra.org/m/xrzejabt}{一个可交互的模型}(站外链接)} \label{fig_MetDfm_10}
\end{figure}
为什么常听说金属钛的强度高?一部分因为钛是密排六方(HCP)结构,而密排六方的滑移系很少,因而位错难以运动。
\end{example}

\begin{figure}[ht]
\centering
\includegraphics[width=8cm]{./figures/5860ee6afe27c5fa.pdf}
\caption{金属的宏观塑性变形} \label{fig_MetDfm_7}
\end{figure}
实际上也可以观察到,单晶金属变形时,往往沿着特定方向形成台阶状结构。这些台阶状结构就是一系列位错滑移后留下的小台阶之和。这类实验证据支持了滑移理论。


\subsubsection{位错运动的条件:分切应力足够大}

\begin{figure}[ht]
\centering
\includegraphics[width=8cm]{./figures/d8a46490c622f4b8.pdf}
\caption{只有特定方向的分切应力(深红色)是有效的驱动力。} \label{fig_MetDfm_14}
\end{figure}

即使位错大幅降低了滑移的难度,也得有足够大的力才能驱动位错运动。那么,全部外力都能驱动位错运动吗?遗憾的是,由于位错只在滑移系上运动,因此只有外力在\textbf{位错滑移方向与滑移面上的分力}才能驱动位错运动。\autoref{fig_MetDfm_14} 简要地表述了这个现象。

\begin{figure}[ht]
\centering
\includegraphics[width=8cm]{./figures/ed3493c0e1bff0f1.pdf}
\caption{进一步的分析} \label{fig_MetDfm_8}
\end{figure}

为了更进一步地解决这个问题,我们建立一个更复杂的模型。如\autoref{fig_MetDfm_8} 所示,我们假设滑移面是红色面,滑移方向是黑色线。那么,外力F在滑移方向上的分切应力为
\begin{equation}
\tau=\frac{F \cos \lambda}{A/{\cos \phi}}=\frac{F}{A}{\cos \lambda}{\cos \phi}
\end{equation}

$\lambda$与$\phi$之和不一定为$90^\circ$,他们是相对独立的变量。至于为什么力要除以面积,这是使用了在材料科学中更实用、准确的概念,应力。你可以不大严谨地权且理解为“压强”。\textsl{霍金曾经说过,文章每多一个公式,就少一半读者,但我没霍金那么厉害,我只能迫不得已地在这里放一个公式(}

只有当这个有效的切应力大于某一最小值时,位错才会开始滑移。这个最小值被称为\textbf{临界分切应力$\tau_k$}。
$$
\tau>\tau_k\Rightarrow\text{位错开始滑移}
$$
% \begin{definition}{临界分切应力}
% 临界分切应力指晶体恰好开始滑移时,滑移方向上的分切应力。
% \end{definition}
换而言之,若要达到临界切分应力$\tau_k$,那么外力至少为 $F_s=\frac{\tau_k A}{\cos \lambda \cos \phi} $ 。
%若写为$\sigma_s=\frac{F_s}{A}$为材料的(屈服)强度或弹性限度\footnote{咬文嚼字地说,有一些细微的不同},指材料在塑性变形前能承受的最大应力。

有时也定义$(\cos \lambda \cos \phi)$被称为取向因子。取向因子与外力的作用方向、晶体中原子的排列方向、位错运动的滑移系等有关。可见,外力的最小大小与取向因子紧密关联,这强烈明示了晶体材料的\textbf{各向异性},即沿不同的方向,晶体的性质不同。\footnote{由于现实材料多为多晶材料,原子排列方式不完全一致,因此各向异性性质被削弱。}

\begin{figure}[ht]
\centering
\includegraphics[width=6cm]{./figures/2351d461053c4eda.pdf}
\caption{晶体的各向异性示意图。由于两个物件内原子的排列方向不同,因此他们在塑性变形前所能承受的最大外力也不同。} \label{fig_MetDfm_9}
\end{figure}
