% 拓扑向量空间
% keys 拓扑向量空间|局部凸拓扑向量空间|局部凸空间
% license Xiao
% type Wiki
\pentry{拓扑空间\nref{nod_Topol},向量空间\nref{nod_LSpace}}{nod_3053}

\subsection{拓扑向量空间}

\begin{definition}{拓扑向量空间}
一个 $\mathbb{F}$-拓扑向量空间 $V$ 是一个实(或复)向量空间同时也是一个拓扑空间,使得加法 $+: V \times V \to V$ 和数乘 $\cdot: \mathbb{F} \times V \to V$ 都是连续函数。

在常见的的语境下我们要求拓扑向量空间是Hasdorff空间(\enref{见分离性}{Topo5}),此时拓扑向量空间是一个\enref{拓扑(加法)群}{TopGrp}。
\end{definition}

\begin{example}{}
\enref{赋范空间}{NormV}、\enref{内积空间}{InerPd}都是拓扑向量空间。

事实上:赋范空间和内积空间都可以通过定义度量成为度量空间,而度量空间是拓扑空间,对赋范空间(范数为 $\norm{\cdot}$),自然的度量定义为 $d(x,y):=\norm{x-y}$,对内积空间(内积为 $(\cdot,\cdot)$)则为 $d(x,y):=\sqrt{(x-y,x-y)}$。因此它们都是拓扑空间。由于它们本身就是在线性空间上定义的,因而都是线性空间。所以只需验证加法和数乘的连续性。

既然它们的拓扑由度量空间来保证,那么可以用度量定义的拓扑来验证。任意 $z_0=x_0+y_0$,设 $O_{z_0}$ 是 $z_0$ 邻域,于是其包含某一开球  $B(z_0,\epsilon)$, 选取 $x_0$ 的邻域 $B(x_0,\epsilon/2)$ 和 $y_0$ 的邻域 $B(y_0,\epsilon/2)$,于是任意 $x\in B(x_0,\epsilon/2),y\in B(y_0,\epsilon/2)$,对赋范空间和内积空间分别成立 
\begin{equation}
\begin{aligned}
d(x+y,z_0)=&\norm{x+y-(x_0+y_0)}\\
\leq& \norm{x-x_0}+\norm{y-y_0}=d(x,x_0)+d(y,y_0)\leq \epsilon;\\
d(x+y,z_0)=&\sqrt{(x+y-z_0,x+y-z_0)}\\
=&\sqrt{(x-x_0|x-x_0)+(y-y_0|y-y_0)+2(x-x_0,y-y_0)}\\
\leq&\sqrt{(x-x_0|x-x_0)+(y-y_0|y-y_0)+2\sqrt{(x-x_0,x-x_0)}\sqrt{(y-y_0,y-y_0)}}\\
=& \sqrt{(x-x_0,x-x_0)}+\sqrt{(y-y_0,y-y_0)};\\
=&d(x,x_0)+d(y,y_0)\leq \epsilon;\\
\end{aligned}~
\end{equation}
\end{example}
即对赋范空间和n

\addTODO{有限维度实向量空间有唯一确定的(Hausdorff)拓扑结构}
% Giacomo:可以参考 https://kconrad.math.uconn.edu/blurbs/topology/finite-dim-TVS.pdf

\subsection{局部凸拓扑向量空间/局部凸空间}

\addTODO{待续}
% Giacomo:希望有专家来续写这部分。

% \begin{definition}{局部凸拓扑向量空间/局部凸空间}
% 一个拓扑向量空间 $V$ 被称为\textbf{局部凸(locally convex)}的,如果它满足

% \end{definition}


