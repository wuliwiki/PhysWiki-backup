% 哈尔滨工业大学 2004 年硕士物理考试试题
% keys 哈尔滨工业大学|考研|物理|2004年
% license Copy
% type Tutor


\textbf{声明}:“该内容来源于网络公开资料,不保证真实性,如有侵权请联系管理员”
\begin{enumerate}
\item 如图所示,透镜的中心厚度为$\bar{OO'}=20cm$,折射率为$n=1.5$,前后表面的曲率半径分别是$20cm$和$40cm$,透镜的后表面镀铝反射膜,在前表面左方距 $O$点 $20cm$ 远的$Q$点放置一个高度为 $1mm $的小物$y$,在傍轴条件下求透镜最后成像的位置、高度、大小、虚实和放缩情况?
\begin{figure}[ht]
\centering
\includegraphics[width=8cm]{./figures/f44e6b31ffc567ba.png}
\caption{} \label{fig_HGD04_2}
\end{figure}
\item 一束在$x-z$平面、沿与$z$轴成如图的$\theta$角方向传播的波长为$\lambda$的平面波和一束点波源$Q$位于$Z$轴上、且与坐标原点$o$相距$a$的发散球面波发生干涉,在傍轴条件下求$z=0$平面上干涉条纹的形状方程及间距公式,并用文字说明干涉条纹的位置和形状。
\begin{figure}[ht]
\centering
\includegraphics[width=8cm]{./figures/f9f3d28dd3f1326e.png}
\caption{} \label{fig_HGD04_1}
\end{figure}
\item 杨氏干涉装置中的$S$点光源发出波长为$\lambda=6000A$°的单色光,间距为$d=0.4mm$的双缝$S_1$和$S_2$对称分布于光轴两侧,衍射屏与观察屏的距离为$D=100cm$,一个焦距为$f=10cm$的薄透镜$L$置于衍射屏和观察屏之间,若薄透镜与衍射屏的距离分别为(1)$A=8cm$和(2)$A=10cm$,在傍轴条件下分别求观察屏$\Sigma$上这两种情况的干涉条纹的形状和间距?
\begin{figure}[ht]
\centering
\includegraphics[width=10cm]{./figures/d5243ec28980c5d0.png}
\caption{} \label{fig_HGD04_3}
\end{figure}
\item 如图所示,衍射屏上有四条平行透光狭缝,缝宽都是$a$,fe间不透明部分的宽度分别是a,2a,“,求单色平行光正入射到该衍射屏上时的夫琅和费衍射强度分布?
\end{enumerate}