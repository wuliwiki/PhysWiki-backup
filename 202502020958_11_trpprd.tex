% 三重积
% keys 向量三重积|标量三重积|triple product
% license Usr
% type Tutor

\pentry{几何向量的点乘\nref{nod_Dot},几何向量的叉乘\nref{nod_Cross}}{nod_744a}




在\textbf{三维欧几里得空间}中,有一类常用的特殊运算,称为\textbf{三重积(triple product)},又称\textbf{混合积},就是选三个向量进行点乘或叉乘。三重积分为两类:标量三重积的结果是标量,向量三重积的结果是向量。




\subsection{标量三重积}


取三个向量$\bvec{a}, \bvec{b}, \bvec{c}$,定义它们的\textbf{标量三重积}为$\bvec{a}\cdot\qty(\bvec{b}\times\bvec{c})$。

$\bvec{b}\times\bvec{c}$是向量$\bvec{b}$和向量$\bvec{c}$的叉乘,结果是向量;于是$\bvec{a}\cdot\qty(\bvec{b}\times\bvec{c})$是向量$\bvec{a}$和向量$\bvec{b}\times\bvec{c}$的点乘,结果是标量。

如果将三个向量分别展开为直角坐标,如下:
\begin{equation}
\bvec{a}\sim\pmat{a_1\\a_2\\a_3}, \bvec{b}\sim\pmat{b_1\\b_2\\b_3}, \bvec{c}\sim\pmat{c_1\\c_2\\c_3}, ~
\end{equation}
那么根据叉乘和点乘的计算公式可得
\begin{equation}
\begin{aligned}
\bvec{a}\cdot\qty(\bvec{b}\times\bvec{c}) \sim{}& \pmat{a_1\\a_2\\a_3}\cdot\qty(\pmat{b_1\\b_2\\b_3}\times \pmat{c_1\\c_2\\c_3})\\
={}& \pmat{a_1\\a_2\\a_3}\cdot\pmat{b_2c_3-b_3c_2\\b_3c_1-b_1c_3\\b_1c_2-b_2c_1}\\
={}& \pmat{a_1b_2c_3-a_1b_3c_2\\a_2b_3c_1-a_2b_1c_3\\a_3b_1c_2-a_3b_2c_1}. 
\end{aligned}~
\end{equation}

标量三重积有直观的几何意义。考虑一个平行六面体,取其一个顶点作为原点,过此顶点的三条边作为三个向量,如\autoref{fig_trpprd_1} 所示。




\begin{figure}[ht]
\centering
\includegraphics[width=8cm]{./figures/a3cc16a5159b6752.pdf}
\caption{标量三重积示意图。图中$c_\parallel$是向量$\vec{c}$垂直于$\vec{a}, \vec{b}$平面的分量。} \label{fig_trpprd_1}
\end{figure}

$\color{NavyBlue}\vec{a}\color{Black}\times\color{Red}\vec{b}\color{Black}$是垂直于$\color{NavyBlue}\vec{a}\color{Black}, \color{Red}\vec{b}\color{Black}$平面的向量,其长度恰为以$\color{NavyBlue}\vec{a}\color{Black}, \color{Red}\vec{b}\color{Black}$为边的平行四边形的面积。$\abs{\qty(\color{NavyBlue}\vec{a}\color{Black}\times\color{Red}\vec{b}\color{Black})\cdot\color{OliveGreen}\vec{c}\color{Black}}$等于$\color{NavyBlue}\vec{a}\color{Black}\times\color{Red}\vec{b}\color{Black}$的长度乘以$\color{OliveGreen}\vec{c}\color{Black}$的长度,即图中平行六面体的底面积乘以高,即其体积。

考虑到三条边的对称性(没有哪条边更特殊),结合内积的交换性,可知三重积$\bvec{a}\cdot\qty(\bvec{b}\times\bvec{c})$的几何意义,就是以三个向量为边的平行六面体的体积或其$-1$倍。





\subsection{向量三重积}


取三个向量$\bvec{a}, \bvec{b}, \bvec{c}$,定义它们的\textbf{向量三重积}为$\bvec{a}\times\qty(\bvec{b}\times\bvec{c})$。

$\bvec{b}\times\bvec{c}$是向量叉乘向量,结果是向量;三重积$\bvec{a}\times\qty(\bvec{b}\times\bvec{c})$同样是向量叉乘向量,结果依然是向量。



向量三重积不具有交换性和结合性,也就是说$\bvec{a}\times\qty(\bvec{b}\times\bvec{c})\neq \qty(\bvec{a}\times\bvec{b})\times\bvec{c}$;不过它遵守一种叫“\textbf{Jacobi结合性}”的性质,即:
\begin{equation}
\bvec{a}\times\qty(\bvec{b}\times\bvec{c})+\bvec{b}\times\qty(\bvec{c}\times\bvec{a})+\bvec{c}\times\qty(\bvec{a}\times\bvec{b})=\bvec{0}. ~
\end{equation}





从几何上也能理解为什么向量三重积不具有结合性。如\autoref{fig_trpprd_2} 所示,$\color{Red}\vec{b}\color{Black}\times\color{NavyBlue}\vec{c}\color{Black}$是垂直于$\color{Red}\vec{b}\color{Black}, \color{NavyBlue}\vec{c}\color{Black}$平面的向量,它跟$\color{OliveGreen}\vec{a}\color{Black}$做叉乘后又垂直于$\color{Red}\vec{b}\color{Black}\times\color{NavyBlue}\vec{c}\color{Black}$自身了,因此最终结果还是在$\color{Red}\vec{b}\color{Black}, \color{NavyBlue}\vec{c}\color{Black}$平面上。同理,$\qty(\color{OliveGreen}\vec{a}\color{Black}\times\color{Red}\vec{b}\color{Black})\times\color{NavyBlue}\vec{c}\color{Black}$也必然在$\color{OliveGreen}\vec{a}\color{Black}, \color{Red}\vec{b}\color{Black}$平面上。




\begin{figure}[ht]
\centering
\includegraphics[width=12cm]{./figures/23207c1117f04b94.pdf}
\caption{向量三重积示意图。} \label{fig_trpprd_2}
\end{figure}



具体来说,
\begin{equation}\label{eq_trpprd_1}
\color{OliveGreen}\vec{a}\color{Black}\times\qty(\color{Red}\vec{b}\color{Black}\times\color{NavyBlue}\vec{c}\color{Black}) = \color{Red}\vec{b}\color{Black}\qty(\color{OliveGreen}\vec{a}\color{Black}\cdot\color{NavyBlue}\vec{c}\color{Black})-\color{NavyBlue}\vec{c}\color{Black}\qty(\color{OliveGreen}\vec{a}\color{Black}\cdot\color{Red}\vec{b}\color{Black})~. 
\end{equation}



\subsubsection{\autoref{eq_trpprd_1} 的几何证明}

\autoref{eq_trpprd_1} 可以用叉乘和点乘的计算公式证明,也可以用下列几何方法证明。

由于$\color{OliveGreen}\vec{a}\color{Black}\times\qty(\color{Red}\vec{b}\color{Black}\times\color{NavyBlue}\vec{c}\color{Black})$在$\color{Red}\vec{b}\color{Black}, \color{NavyBlue}\vec{c}\color{Black}$平面上,因此可以设
\begin{equation}
\color{OliveGreen}\vec{a}\color{Black}\times\qty(\color{Red}\vec{b}\color{Black}\times\color{NavyBlue}\vec{c}\color{Black}) = x\color{Red}\vec{b}\color{Black}+y\color{NavyBlue}\vec{c}\color{Black}~. 
\end{equation}

于是
\begin{equation}
\begin{aligned}
0 ={}& \color{OliveGreen}\vec{a}\color{Black}\cdot\qty(\color{OliveGreen}\vec{a}\color{Black}\times\qty(\color{Red}\vec{b}\color{Black}\times\color{NavyBlue}\vec{c}\color{Black}))\\
={}& \color{OliveGreen}\vec{a}\color{Black}\cdot\qty(x\color{Red}\vec{b}\color{Black}+y\color{NavyBlue}\vec{c}\color{Black})\\
={}& x\color{OliveGreen}\vec{a}\color{Black}\cdot\color{Red}\vec{b}\color{Black}+y\color{OliveGreen}\vec{a}\color{Black}\cdot\color{NavyBlue}\vec{c}\color{Black}. 
\end{aligned}~
\end{equation}
因此
\begin{equation}
\frac{x}{y} = -\frac{\color{OliveGreen}\vec{a}\color{Black}\cdot\color{NavyBlue}\vec{c}\color{Black}}{\color{OliveGreen}\vec{a}\color{Black}\cdot\color{Red}\vec{b}\color{Black}}~, 
\end{equation}
或者说
\begin{equation}
x = \gamma\color{OliveGreen}\vec{a}\color{Black}\cdot\color{NavyBlue}\vec{c}\color{Black},\quad y=-\gamma \color{OliveGreen}\vec{a}\color{Black}\cdot\color{Red}\vec{b}\color{Black}~. 
\end{equation}

这个$\gamma$对任何向量三重积都是一样的。这是因为\textbf{线性性},比如如果
\begin{equation}
\left\{
\begin{aligned}
\bvec{a}\times\qty(\bvec{b}_1\times\bvec{c}) ={}& \gamma_1\bvec{b}_1(\bvec{a}\cdot\bvec{c}) - \gamma_1\bvec{c}(\bvec{a}\cdot\bvec{b}), \\
\bvec{a}\times\qty(\bvec{b}_2\times\bvec{c}) ={}& \gamma_2\bvec{b}_2(\bvec{a}\cdot\bvec{c}) - \gamma_2\bvec{c}(\bvec{a}\cdot\bvec{b}), 
\end{aligned}
\right. ~
\end{equation}
那么
\begin{equation}
\bvec{a}\times\qty((\bvec{b}_1+\bvec{b}_2)\times\bvec{c}) = (\gamma_1\bvec{b}_1+\gamma_2\bvec{b}_2)(\bvec{a}\cdot\bvec{c}) - (\gamma_1\bvec{c}+\gamma_2\bvec{c})(\bvec{a}\cdot\bvec{b}). 
~\end{equation}
可是$\gamma_1\bvec{b}_1+\gamma_2\bvec{b}_2$应该和$\bvec{b}_1+\bvec{b}_2$共线,因此必须有$\gamma_1=\gamma_2$。

既然$\gamma$对所有三重积都一样,我们就可以选择比较特殊的三重积来计算出$\gamma$。比如说,选择\textbf{相互正交}的$\color{OliveGreen}\vec{a}\color{Black}$和$\color{Red}\vec{b}\color{Black}$,于是
\begin{equation}
\begin{aligned}
\gamma\abs{\color{OliveGreen}\vec{a}\color{Black}}^2\abs{\color{Red}\vec{b}\color{Black}}\\
={}& \gamma\color{OliveGreen}\vec{a}\color{Black}(\color{OliveGreen}\vec{a}\color{Black}\cdot\color{Red}\times\vec{b}\color{Black})\\
={}& \gamma\color{OliveGreen}\vec{a}\color{Black}(\color{OliveGreen}\vec{a}\color{Black}\cdot\color{Red}\vec{b}\color{Black})-\gamma\color{Red}\vec{b}\color{Black}(\color{OliveGreen}\vec{a}\color{Black}\cdot \color{OliveGreen}\vec{a}\color{Black})\\
={}& \color{OliveGreen}\vec{a}\color{Black}\times\qty(\color{OliveGreen}\vec{a}\color{Black}\times\color{Red}\vec{b}\color{Black})\\
={}& \abs{\color{OliveGreen}\vec{a}\color{Black}}^2\abs{\color{Red}\vec{b}\color{Black}}. 
\end{aligned}
~\end{equation}
由此可知,$\gamma=1$。

综上即可得证\autoref{eq_trpprd_1} 。







































