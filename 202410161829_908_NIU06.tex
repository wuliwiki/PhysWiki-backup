% 南京理工大学 2006 年 研究生入学考试试题 普通物理(B)
% license Usr
% type Note

\textbf{声明}:“该内容来源于网络公开资料,不保证真实性,如有侵权请联系管理员”

\subsection{填空题(26分,每空2分)}
\begin{enumerate}
        \item y质点作平面运动,运动方程为 $\vec{r} = (3t + 1) \hat{i} + (t^2 - 1) \hat{j}$,则t时刻质点的速度为$\underline{\hspace{2cm}}$,加速度为$\underline{\hspace{2cm}}$。
        
        \item 质量 $m$ 均匀分布的链条,总长为 $L$,有长度b伸在桌外。若自静止释放,则链条全部脱离光滑桌面的速度为$\underline{\hspace{3cm}}$,加速度为$\underline{\hspace{3cm}}$,若以桌面为势能零点,此时的势能为$\underline{\hspace{3cm}}$。
        \begin{figure}[ht]
        \centering
        \includegraphics[width=6cm]{./figures/4a1e4323cfe6e162.png}
        \caption{} \label{fig_NIU06_1}
        \end{figure}

        \item 如图,曲线I表示 $27^\circ C$ 的氧气分子的Maxwell速度分布,则图示中 $v_1 = \underline{\hspace{3cm}}$(6)。设曲线II也表示氧气分子某一温度下的Maxwell速度分布,且 $v_2 = 600 , m/s$,则曲线II对应的理论体温标 $T_2 = \underline{\hspace{3cm}}$。
        \begin{figure}[ht]
        \centering
        \includegraphics[width=6cm]{./figures/4263a7e1e4f0d198.png}
        \caption{} \label{fig_NIU06_2}
        \end{figure}

        \item 一理想卡诺循环一次卡诺循环对外做功1000J,卡诺循环高温热源温度 $T_1 = 500K$,低温热源温度 $T_2 = 300K$,则在一次循环中,在高温热源处吸热 $Q_1 = \underline{\hspace{3cm}}$,在低温热源处放热 $Q_2 = \underline{\hspace{3cm}}$。
        
        \item 弹簧振子的振动周期为T,现将弹簧截去一半,则新弹簧质子的振动周期为 $\underline{\hspace{3cm}}$,再将两半弹簧并联使用,则其振动周期变为 $\underline{\hspace{3cm}}$。
        
        \item 一平面简谐波方程(波函数)为 $y_1 = A \cos 2 \pi \left( \frac{t}{T} - \frac{x}{\lambda} \right)$,在 $x = \frac{\lambda}{2}$ 处有一反射壁,若平面波从空气传到反射壁而反射,反射时振幅不变,已知空气为波介质,则反射波波动方程为 $\underline{\hspace{3cm}}$,波节点的位置为 $\underline{\hspace{3cm}}$。
    \end{enumerate}
    \subsection{填空题(30分,每空2分)}
    \begin{enumerate}
    \item 半径为 $R$,带电量为 $Q$ 的金属导体球,其外表面处的电场强度大小为 $\underline{\hspace{2cm}}$,导体球内部或距球心 $r (< R)$ 处的电势为 $\underline{\hspace{2cm}}$。导体球外距球心 $r$ 处的电势为$ \underline{\hspace{3cm}}$。
    \item 一半径为 $R$ 的无限长薄圆管,平行轴向有一宽度为 $a$ ( $a \\ll R$ ) 的无限长细线,如图所示。管壁上均匀地通有稳恒电流,设管壁圆周上单位长度电流为 $I$,方向垂直纸面向上,则轴 $O$ 处的磁感应强度大小 $B_0 =$ $\underline{\hspace{2cm}}$,方向为 $\underline{\hspace{2cm}}$;
    \begin{figure}[ht]
    \centering
    \includegraphics[width=6cm]{./figures/2bd1e63cfabe151f.png}
    \caption{} \label{fig_NIU06_3}
    \end{figure}
    \item 在两个偏振化方向正交的偏振片之间插入第三个偏振片,当最后透过的光强为入射的自然光强的 $\frac{1}{8}$ 时,第三个偏振片偏振化方向与第一个偏振片偏振化方向的夹角为 $\underline{\hspace{2cm}}$,当最后透射光强为零时,第三个偏振片偏振化方向与第一个偏振片偏振化方向的夹角为 $\underline{\hspace{2cm}}$。
    \item 用氢-氖激光器发出的波长为 $632.8\text{nm}$ 的单色光做牛顿环实验,测得第 $k$ 级暗环的半径为 $5.625\text{mm}$,第 $k+5$ 级暗环的半径为 $7.956\text{mm}$,则所用平凸透镜的曲率半径 $R =$ $\underline{\hspace{2cm}}$,$k$ 的级数为 $\underline{\hspace{2cm}}$。
    \item 静系中测得一棒的长度为$l$,其质量线密度为$p$,若此棒沿其长度方向以速度$v$运动,其线密度为 $\underline{\hspace{2cm}}$,若此棒沿与其长度方向垂直的方向以速度$v$运动,其线密度为 $\underline{\hspace{2cm}}$
    \item 基态氢原子中的电子吸收一个能量为 $13.0ey$ 的光子后所能到达的最高能态为  $\underline{\hspace{2cm}}$,从该能态直接跃迁回基态辐射的光子频率为 $\underline{\hspace{2cm}}$
    \item 爱因斯坦光电效应方程为$\underline{\hspace{2cm}}$,某金属的电子逸出功为.4.2电子伏特,要从金属表面释放出电子,照射光的波长必须满足$\underline{\hspace{2cm}}$
\end{enumerate}

\subsection{(12分)}
一长为$L$质量为$m$的均细棒,一端可绕水平光滑轴$O$在竖直平面内转动。当细棒静止在竖直位置时,有质量为 $m_0$,速度为$v_0$的子弹,水平射入其下端而不复出。此后棒摆到水平位置后重又下落。求子弹射入棒前的速度$v_0$”。
\begin{figure}[ht]
\centering
\includegraphics[width=6cm]{./figures/9b003e0d7de0af93.png}
\caption{} \label{fig_NIU06_4}
\end{figure}
\subsection{(10分)}
劲度系数为$k$的轻弹,上端接一水平的轻平台,下端固定于地面。当质量为$m$的人站于平台上,弹簧压缩了$x_0$,并由此位置开始向下运动作为初始时刻,设系统振动的振幅为 $A$,且向下为正,求振动方程。
\subsection{(12分)}
如图为一循环过程的下$T-V$图线。该循环的工质为$v mol$的理想气体,其$C_r,m$和$r$均已知且为常量。已知$a$点的温度为$T_1$,体积为$V_1,b$点的体积为$V_2,ca$ 为绝热过程。求循环的效率。
\begin{figure}[ht]
\centering
\includegraphics[width=6cm]{./figures/0af64b6f0b5b981b.png}
\caption{} \label{fig_NIU06_5}
\end{figure}
\subsection{(12分)}
长为$L$的圆柱形电容器内外半径分别为$R_1$,和$R_2$,,其间充满相对介电常数为$\varepsilon$,的均匀介质,电容器带电量为 $Q$。求:

(1)两极间的电势差:(2)电容器的电容;(3)系统储存的电场能:

\subsection{(12分)}
一个半径$R=0.18m$的半圆形闭合线圈,载有电流$I=10A$,放在$B-10T$的均匀外磁场中,磁场方向与线圈平面平行,如图所示。求:

(1)线圈磁矩的大小与方向;

(2)线圈所受磁力矩的大小与方向;

(3)在磁力矩作用下,线圈平面绕纸面内过0点的竖直轴转过$\pi/2$时,磁力矩作的功(设$I$在旋转过程中不变)。
\begin{figure}[ht]
\centering
\includegraphics[width=6cm]{./figures/5c2cf0a45f68d533.png}
\caption{} \label{fig_NIU06_6}
\end{figure}
\subsection{(12 分)}如图所示,在长直导线旁有一与之共面的矩形线框,其中 $AB$ 在线框上以速度 $v$ 沿平行导线方向运动,线框两边与导线平行,距离分别为 $a$ 和 $b,t=0$ 时 $AB$ 边离线框起点为 $x_0$,长直导线中通以变化电流 $I = I_0 \cos{\omega t}$,求线框中的感应电动势。
\begin{figure}[ht]
\centering
\includegraphics[width=6cm]{./figures/490e04683dcfcb32.png}
\caption{} \label{fig_NIU06_7}
\end{figure}

\subsection{(12 分)}波长 $\lambda = 600 nm$ 的单色光垂直入射在一光栅上,第二级明线出现 在 $\sin{\theta_2} = 0.3$ 处,第四级为第一个缺级。求: ① 光栅常数; ② 屏上可观察到的明级数及其级数。

\subsection{(12 分)}动能为 2 eV 的电子,从无穷远处向着静止的氢原子运动,最后被质子所束缚形成基态的氢原子,求 (1)在此过程中放出的光波的波长; (2)电子绕质子运动的动能;(3)电子的德布罗意波长。

\textbf{附常用物理常数:}
\begin{itemize}
    \item 电子静止质量 $m_0 = 9.1 \times 10^{-31} \\, \mathrm{kg}$
    \item 电子电量 $e = 1.6 \times 10^{-19} \\, \mathrm{C}$
    \item 普朗克常数 $h = 6.63 \times 10^{-34} \\, \mathrm{J \cdot s}$
    \item 真空中光速 $c = 3 \times 10^8 \\, \mathrm{m/s}$
    \item 玻尔兹曼常数 $k = 1.38 \times 10^{-23} \\, \mathrm{J/K}$
    \item 普适气体恒量 $R = 8.31 \\, \mathrm{J/(mol \cdot K)}$
    \item 引力常量 $G = 6.67 \times 10^{-11} \\, \mathrm{N \cdot m^2 / kg^2}$
    \item 真空电容率 $\varepsilon_0 = 8.85 \times 10^{-12} \\, \mathrm{C^2 / (N \cdot m^2)}$
\end{itemize}
