% Hartree-Fock 方法
% Hartree|Fock|量子力学|薛定谔方程|多体|自旋

Hartree Fock 方法的精髓是假设多粒子波函数 $\Psi(\bvec r_1, \bvec r_2\dots \bvec r_N)$ 等于每个单粒子波函数(包括自旋)的乘积 $u_1(q_1) u_2(q_2)\dots u_N(q_N)$ (Hartree 函数), 其中不同的单粒子波函数要求正交归一. 然后找到最优的单粒子波函数 $u_1, u_2,\dots u_N$ 使总哈密顿的平均值最小. 所以该方法属于变分法. 得到的能级大于精确能级.

注意 Hartree 函数并不满足全同粒子的对易关系, 既不是对称也不是反对称. 对于全同费米子, 方法是令多粒子函数为单粒子函数(包括自旋)的 Slater 行列式.
\begin{equation}
\Psi(q_1, q_2\dots) = \frac{1}{\sqrt {N!}} \vmat{
u_1(q_1) & u_2(q_1) & \cdots  \\ 
u_1(q_2) & u_2(q_2) & \cdots  \\ 
\vdots & \vdots & \ddots  
} \end{equation}
变分法的拉格朗日乘数函数为
\begin{equation}
L = \bra{\Psi}\Q H\ket{\Psi} - \sum_{ij} \varepsilon_{ij} \braket{u_i}{u_j}
\end{equation}
为了化简第二项, 令原基底为另一组基底的幺正变换
\begin{equation}
\pmat{u_1\\ {u_2} \\ \vdots }
= \Q U\pmat{ u'_1 \\ u'_2\\  \vdots}
\end{equation}
可以证明第一项不变. 因为
\begin{equation}\ali{
\Psi(q_1, q_2\dots) & = \frac{1}{\sqrt {N!}}
\vmat{
u_1(q_1) & u_2(q_1) & \cdots  \\ 
u_1(q_2) & u_2(q_2) & \cdots  \\ 
\vdots & \vdots & \ddots}
= \frac{1}{\sqrt {N!}}
\vmat{\mat U
\pmat{u'_1(q_1) & u'_1(q_2) & \cdots  \\ 
u'_2(q_1) & u'_2(q_2) & \cdots  \\ 
\vdots & \vdots & \ddots}}\\
& = \frac{\abs{\mat U}}{\sqrt {N!}}
\vmat{
u'_1(q_1) & u'_2(q_1) & \cdots  \\ 
u'_1(q_2) & u'_2(q_2) & \cdots  \\ 
\vdots & \vdots & \ddots}
= \frac{\E^{\I \theta }}{\sqrt {N!}}
\vmat{
u'_1(q_1) & u'_2(q_1) & \cdots  \\ 
u'_1(q_2) & u'_2(q_2) & \cdots  \\ 
\vdots & \vdots & \ddots}
}\end{equation}
注意这里用到了 $\abs{\mat A\mat B} = \abs{\mat A}\abs{\mat B}$ 和 $\abs{\mat U}= {\E^{\I \theta}}$.  这是说, 平均能量关于 $u$ 的公式不受幺正变换的影响. 对第二项,
\begin{equation}
\sum_{ij}  \varepsilon_{ij} \braket{u_i}{u_j}
= \int (u'_1{}^* ,u'_2{}^* \dots) \mat U\Her \varepsilon \mat U
\pmat{ u_1' \\ u_2' \\ \vdots} \dd{q}
\end{equation}

若我们选择幺正变换, 使得 ${\mat U\Her }\mat \varepsilon \mat U = {E_i}{\delta_{ij}}$, %未完成: 这里表达不规范
即把矩阵 $\mat \varepsilon$ 对角化(下文可知 $\mat\varepsilon$ 为对称矩阵), 得
\begin{equation}
\sum_{ij} \varepsilon_{ij} \braket{u_i}{u_j}
= \sum_i {E_i}  \braket{u_i}{u_j}
\end{equation}
现在把所有的撇号省略, 拉格朗日函数为
\begin{equation}
L =\bra{\Psi}\Q H \ket{\Psi}  - \sum_i E_i \braket{u_i}{u_i}
\end{equation}
即约束条件只需要归一化, 正交会自动完成. 现在来化简第一项. 首先把总波函数中的行列式记为求和的形式
\begin{equation}
\Psi  = \frac{1}{\sqrt {N!}} \sum_{N!} (-1)^p \Q P{\Psi_H}  \equiv \sqrt{N!} \Q A \Psi_H
\end{equation}

其中 $\Psi_H = u_1(q_1) u_2(q_2)\dots$ 是 Hartree 函数,  $\Q P$ 是置换算符, 相当于做 $p$ 次双粒子置换( $p$ 是逆序数), 行列式展开后共有 $N!$ 项.  $\Q A$ 为反对称化算符, 由于 $\Q H$ 和 $\Q A$ 存在一组共同本征矢, $\comm*{\Q H}{\Q A} = 0$,  另外可以证明, $\Q A^2 = \Q A$ (意义是反对称化只需要一次)(可先证明 $N = 2,3$,  高阶行列式同理).
\begin{equation}
\begin{aligned}
\bra{\Psi}\Q H\ket{\Psi}  &= N!\bra{\Psi_H} \Q A\Q H\Q A \ket{\Psi_H} = N!\bra{\Psi_H}\Q H \Q A^2\ket{\Psi_H} \\
&= N!\bra{\Psi_H} \Q H\Q A\ket{\Psi_H}   = \sum_{N!} (-1)^p \bra{\Psi_H}\Q H\Q P\ket{\Psi_H}\\ 
\end{aligned}
\end{equation}
我们现在考虑多电子原子(离子)问题
\begin{equation}
\Q H = \sum_i \Q h_i + \sum_{i < j} \frac{1}{r_{ij}} \qquad
\Q h_i =  - \frac12 \laplacian + \frac{Z}{r_i}
\end{equation}
\begin{equation}
\begin{aligned}
\bra{\Psi}\sum_i \Q h_i \ket{\Psi} & = \sum_i \sum_{N!} ( - 1)^p \bra{\Psi_H}{\Q h_i}\Q P \ket{\Psi_H} = \sum_i \bra{\Psi_H} \Q h_i \ket{\Psi_H}\\
&= \sum_i \bra{u_i} \Q h_i \ket{u_i}
\end{aligned}
\end{equation}
 
这是因为只有当 $\Q P$ 为 $1$ 时(行列式的对角项, 逆序数 $p = 0$ )积分才不为零. 同理, $\Q H$ 剩下的部分为
\begin{equation}
\bra{\Psi}\sum_{i < j} \frac{1}{r_{ij}} \ket{\Psi}  = \sum_{i < j} \sum_{N!} {(-1)^p}\bra{\Psi_H}\frac{1}{r_{ij}} \Q P \ket{\Psi_H}
\end{equation}
现在 $P = 1$ 或者 $P = P_{ij}$ ( $p=1$ )时积分都可能不为零. 所以
\begin{equation}
\begin{aligned}
\bra{\Psi}\sum_{i < j} \frac{1}{r_{ij}} \ket{\Psi} & = \sum_{i < j} \bra{\Psi_H}\frac{1}{r_{ij}}(1 - P_{ij}) \ket{\Psi_H} \\
&= \sum_{i < j} \bra{u_i u_j}\frac{1}{r_{ij}}\ket{u_i u_j}  - \sum_{i < j} \bra{u_i u_j} \frac{1}{r_{ij}} \ket{u_j u_i}\\
& \equiv \sum_{i < j} J_{ij} - \sum_{i < j} K_{ij}\\ 
\end{aligned}
\end{equation}
注意这里 $\ket{u_i u_j}\Her$ 记为 $\bra{u_i u_j}$ 而不是 $\bra{u_j u_i}$.  另外易证 $J_{ij} = J_{ji}$, $K_{ij} = K_{ji}$ (交换积分变量即可) 所以
\begin{equation}
L = \sum_i \bra{u_i} \Q h_i \ket{u_i}  + \sum_{i< j} \bra{u_i u_j}\frac{1}{r_{ij}} \bra{u_i u_j}  - \sum_{i < j} \bra{u_i u_j} \frac{1}{r_{ij}} \ket{u_j u_i}   - \sum_i E_i \braket{u_i}{u_j}
\end{equation}
现在, 类似于变分法中的过程, 把任意一个 $\bra{u_k}$ 变为 $\bra{u_k + \delta u_k}$,  减去上式, 令为 $0$,得
 \begin{equation}
\bra{\delta u_k}{h_k}\ket{u_k}  + \sum_j^{(j \ne k)} \bra{\delta u_k u_j} \frac{1}{r_{jk}} \ket{u_k u_j}  - \sum_j^{(j \ne k)} \bra{\delta u_k u_j}\frac{1}{r_{ij}}\ket{u_j u_k} - E_k \braket{\delta u_k}{u_k}  = 0
\end{equation}
即
 \begin{equation}
\bra{\delta u_k} \qty[h_k \ket{u_k}  + \sum_j^{(j \ne k)} \bra{u_j} \frac{1}{r_{jk}} \ket{u_j} \ket{u_k}  - \sum_j^{(j \ne k)} \bra{u_j}\frac{1}{r_{ij}}\ket{u_k}\ket{u_j}  - E_k\ket{u_k}] = 0
\end{equation}
最后, 由于 $\bra{\delta u_k}$ 可以取任意微小函数, 与之相乘的ket必须为零
\begin{equation}
h_k \ket{u_k}  + \qty[\sum_j^{(j \ne k)} \bra{u_j}\frac{1}{r_{jk}} \ket{u_j}] \ket{u_k}  - \sum_j^{(j \ne k)} \qty[\bra{u_j}\frac{1}{r_{ij}} \ket{u_k}] \ket{u_j} = E_k\ket{u_k} 
\end{equation}
这是所谓的, 非线性耦合微分积分本征方程组.

注意虽然现在 trial 波函数满足全同费米子的反对称, 一般却不是总自旋角动量的本征函数(其实也不是总轨道角动量的本征函数, 除非把无穷个不同的 $\Psi_H$ 求和). 为了实现这点, 可以先指定总自旋角动量 $\ket{S,M}$,  然后通过对行列式线性组合构建自旋部分为 $\ket{S,M} $ 的 trial 波函数.

\begin{example}{氦原子}
对于 He 原子的 singlet 自旋态 $(\uparrow\downarrow  -  \downarrow\uparrow )/\sqrt 2 $,  可以用一个行列式构建trial波函数. 由于自旋为反对称, 轨道波函数必须为对称, 对于基态, 这意味着两个轨道波函数相同
\begin{equation}
\Psi  = \frac{1}{\sqrt 2 }\vmat{
\phi(\bvec r_1)\uparrow & \phi(\bvec r_1)\downarrow \\ 
\phi(\bvec r_2) \uparrow & \phi(\bvec r_2) \downarrow}
= \phi(\bvec r_1) \phi(\bvec r_2) (\uparrow\downarrow  -  \downarrow\uparrow )/\sqrt 2 
\end{equation}
把 $u_1 = \phi\uparrow$, $u_2 = \phi\downarrow$ 代入本征方程组得单个轨道波函数的本征方程
\begin{equation}
\Q h\ket{\phi}  + \bra{\phi}\frac{1}{r_{12}} \ket{\phi} \ket{\phi}  = E \ket{\phi} 
\end{equation}
当然也可以直接把 $\Psi  = \phi(\bvec r_1)\phi(\bvec r_2)(\uparrow\downarrow  -  \downarrow\uparrow)/\sqrt 2 $ 代入能量平均值公式, 用分母 $\braket{\phi}{\phi}$ 归一化或者用拉格朗日乘数法做, 得到的方程与上式一样
\begin{equation}\ali{
L &= 2\bra{\phi \phi} \Q h \ket{\phi \phi }  + \bra{\phi \phi }\frac{1}{r_{12}}  \ket{\phi \phi }  - \lambda [\braket{\phi}{\phi} - 1]  \\
&= 2\bra{\phi}\Q h \ket{\phi}  + \bra{\phi \phi }\frac{1}{r_{12}}  \ket{\phi \phi }  - \lambda [\braket{\phi}{\phi} - 1]
}\end{equation}
注意拉格朗日乘数法中的优化函数可以使用限制条件化简. 令增量为 0
\begin{equation}
2\bra{\delta \phi} h \ket{\phi}  + 2\bra{\delta \phi \phi} \frac{1}{r_{12}} \ket{\phi \phi }  - \lambda \braket{\delta\phi}{\phi} = 0
\end{equation}
其中使用了 $\delta \bra{\phi \phi} = \bra{\delta \phi\phi} + \bra{\phi \delta \phi}$. 
\begin{equation}
\Q h \ket{\phi}  + \bra{\phi}\frac{1}{r_{12}} \ket{\phi} \ket{\phi}  = \frac{\lambda }{2}  \ket{\phi} 
\end{equation}
\end{example}

对于其他的自旋态, 往往需要行列式的线性组合才能构造总自旋本征态, 这个比较复杂, 先来看另一种方法. 我们可以直接指定总自旋态, 如果要求轨道波函数反对称, 可把 $u_i(q_i) = \phi_i(\bvec r_i)$ 直接代入本征方程组即可, 而当要求轨道波函数对称时, 可以把以上的反对称化算符 $\Q A$ 中的 $(-1)^p$ 去掉改成对称化算符 $\Q B$ 
\begin{equation}
\Q B \ket{\Psi_H}  = \frac{1}{N!} \sum_{N!} \Q P\ket{\Psi_H}
\end{equation}
$\Q B$ 同样满足 $\comm*{\Q H}{\Q B} = 0$,  $\Q B^2 = \Q B$.  以上推导全部有效(所有负号改成正号即可), 新的本征方程组变为
\begin{equation}
h_k \ket{u_k} + \qty[\sum_j^{(j \ne k)} \bra{u_j} \frac{1}{r_{jk}} \ket{u_j}] \ket{u_k} + \sum_j^{(j \ne k)} \qty[\bra{u_j} \frac{1}{r_{ij}} \ket{u_k}] \ket{u_j} = E_k \ket{u_k}
\end{equation}

\begin{example}{氦原子}
考虑 $He$ 的 $1s2s$,  $^1S$ 态, 即自旋为 singlet $\ket{0,0}  = (\uparrow\downarrow  -  \uparrow\downarrow)/\sqrt 2 $.  两个空间波函数不相同. 直接把 $u_1 = \phi_{1s}$, $u_2 = \phi_{2s}$ 代入上面的对称本征方程组, 得
\begin{equation}
\Q h_1 \ket{\phi_{1s}} + \bra{\phi_{2s}} \frac{1}{r_{12}} \ket{\phi_{2s}} \ket{\phi_{1s}} + \bra{\phi_{2s}} \frac{1}{r_{12}} \ket{\phi_{1s}} \ket{\phi_{2s}} = E_1\ket{\phi_{1s}}
\end{equation}
\begin{equation}
\Q h_2 \ket{\phi_{2s}} + \bra{\phi_{1s}} \frac{1}{r_{12}} \ket{\phi_{1s}} \ket{\phi_{2s}} + \bra{\phi_{1s}} \frac{1}{r_{12}} \ket{\phi_{2s}} \ket{\phi_{1s}} = E_2 \ket{\phi_{2s}} 
\end{equation}

如果直接用 $\ket{\Psi} = (\phi_1 \phi_2 + \phi_2\phi_1)/\sqrt 2$,  结果相同.
\end{example}
 
