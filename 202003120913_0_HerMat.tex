% 厄米矩阵
% keys 矩阵|线性代数|复共轭|转置|厄米共轭

\pentry{对称矩阵}

\subsection{厄米共轭}
我们把矩阵元可以取复数的矩阵叫做\textbf{复数矩阵}. % 未完成: 这句话是不是应该在前面提一下.
与矩阵转置(\autoref{Mat_eq3}\upref{Mat})类似, 要对一个复数矩阵做\textbf{厄米共轭}, 就先将其做转置, 再对每个矩阵元取复共轭. 矩阵 $\mat A$ 的厄米共轭记为 $\mat A\Her$, 其第 $i$ 行第 $j$ 列的矩阵元为
\begin{equation}
A\Her_{i,j} = A_{j,i}^*
\end{equation}
注意当矩阵元都是实数时, 厄米共轭就是转置.


\subsection{厄米共轭的基本性质}
任意常数乘以厄米矩阵再取共轭, 等于该常数的复共轭乘以矩阵的厄米共轭
\begin{equation}\label{HerMat_eq1}
(c \mat A)\Her = c^* \mat A\Her
\end{equation}

类比转置,% 引用未完成
矩阵相乘的厄米共轭等于分别做厄米共轭, 逆序排列, 再相乘
\begin{equation}\label{HerMat_eq2}
(\mat A \mat B \dots \mat C)\Her  = \mat C\Her \dots \mat B\Her \mat A\Her
\end{equation}

\begin{exercise}{证明}
请根据相关定义证明\autoref{HerMat_eq1} 和\autoref{HerMat_eq2}.
\end{exercise}

\subsection{厄米矩阵}
若 $\mat A$ 的厄米共轭是其本身, 即
\begin{equation}
\mat A\Her = \mat A
\end{equation}
那么我们称其为\textbf{厄米矩阵}. 厄米矩阵关于对角线对称的任意两个矩阵元互为复共轭
\begin{equation}
A_{i,j} = A_{j,i}^*
\end{equation}
特殊地, 对角线上的矩阵元的复共轭等于本身 ($A_{i,i} = A_{i,i}^*$), 所以厄米矩阵对角线上的矩阵元都是实数.

% 举例未完成
