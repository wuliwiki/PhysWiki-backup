% Sperner定理
% 组合数学|Sperner定理

先给出一些记号:
\begin{itemize}
\item 集合{1,2,$\cdots$,n}简记为$[n]$
\item 集合$X$的所有子集构成的集族记为$2^{X}$
\end{itemize}

\begin{definition}{独立族}
设$\mathcal{F}\subset 2^{[n]}$是$[n]$的一个子集族.我们说$\mathcal{F}$是\textbf{独立的}如果$\forall A,B\in \mathcal{F}$我们都有$A\not\subset B$和$B\not\subset A$(即$\mathcal{F}$中的任意两个集合互不包含).
\end{definition}

下面我们给出Sperner定理:
\begin{theorem}{Sperner's Theorem}
对于任意的独立子集族$a$,我们有:$|\mathcal{F}|\leq \pmat{n\\\lfloor \frac{n}{2}\rfloor}$

\end{theorem}
为给出定理1的证明,我们先引入\textbf{链}的定义:
\begin{definition}{链}
一个$[n]$的子集构成的\textbf{链}指一列子集使得:\\
$A_1\subset A_2\subset A_3\subset \cdots \subsetA_k$
\end{definition}

\textbf{定理1}的证明概要:\\
\begin{enumerate}
\item 
\end{enumerate}

