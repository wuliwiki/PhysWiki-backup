% 素理想与极大理想
% keys 素理想|极大理想|环|环论|理想|素数

\pentry{整环\upref{Domain}}

本节介绍两类极为重要的理想:素理想和极大理想.





\begin{definition}{素理想}
给定\textbf{交换环}$R$,如果$P$是它的一个真理想,并且满足“对于任意$a, b\in R$,如果$ab\in P$,则必有$a\in P$或$b\in P$”,则称$P$是$R$的一个\textbf{素理想(prime ideal)}.
\end{definition}

注意,素理想的定义不一定非得是整环,也可以是含有零因子的交换环.比较一下它和素元素的定义,形式上是完全一样的.

\begin{definition}{极大理想}
给定\textbf{交换环}$R$,如果$M$是它的一个真理想,并且满足“对于任意$R$的理想$I$,如果$M\subsetneq I$,那么$I=R$”,则称$M$是$R$的一个\textbf{极大理想(maximal ideal)}.
\end{definition}

极大理想的定义很直观,任何比极大理想大的理想都只能是$R$本身.

素理想的一大作用,是把零因子收集了起来,使得商环能成为一个整环.同时,素理想和素元素之间高度类似.我们可以把这些话写成如下紧凑的定理:

\begin{theorem}{}
给定交换环$R$和它的一个理想$P$,则以下条件等价:
\begin{enumerate}
\item $P$是$R$的素理想.
\item 如果$R$的两个理想$I_1$、$I_2$满足“若$I_1I_2\in P$,则$I_1\in P$或$I_2\in P$”.
\item $R/P$是一个整环.
\end{enumerate}
\end{theorem}

\textbf{证明}:



\textbf{证毕}.




