% 埃尔文·薛定谔
% license CCBYSA3
% type Wiki

(本文根据 CC-BY-SA 协议转载自原搜狗科学百科对英文维基百科的翻译)


欧文·鲁道夫·约瑟夫·亚历山大·薛定谔(UK: /ˈʃrɜːdɪŋər/, US: /ˈʃroʊ-, ˈʃreɪ-/;[1] 德语:[ˈɛɐ̯viːn ˈʃʁøːdɪŋɐ];1887年8月12日至1961年1月4日),有时写作 Erwin Schrodinger 或者 Erwin Schroedinger,是一位获得诺贝尔奖的奥地利物理学家,他在量子理论领域取得了许多基础性的研究成果:薛定谔方程提供了一种计算系统波函数及其随时间动态变化的方法。

此外,他还是物理领域的很多著作的作者:统计力学和热力学、电介质物理学、颜色理论、电动力学、广义相对论和宇宙学,他多次尝试构建统一场论。在他的书中生命是什么 薛定谔从物理学的角度看待生命现象,解决了遗传学的问题。他非常重视科学的哲学方面、古代和东方的哲学概念、伦理和宗教。 他还写了哲学和理论生物学的著作。他也以他的“薛定谔猫”思想的实验而闻名。[2]

\subsection{传记}
\subsubsection{1.1 早年}
1887年8月12日,薛定谔出生于 Erdberg [de]奥地利维也纳,父亲鲁道夫·薛定谔 Rudolf Schrödinger (德语)[3] (cerecloth 生产者、植物学家)[4]母亲乔治·埃米莉亚·布伦达·薛定谔(内·鲍尔)(父亲是亚历山大· 保尔Alexander Bauer (德语),[5][6][7] 化学教授,维也纳科技大学)。他是他们唯一的孩子。

他的母亲有一半奥地利血统,一半英国血统;他的父亲是天主教徒,母亲是路德教徒。尽管他是在一个宗教家庭中长大的路德教徒,但他自称是无神论者。[8] 然而,他对东方宗教和泛神论有浓厚的兴趣,他在著作中使用了宗教象征主义。[9] 他也相信他的科学工作是一种接近神性的方法,尽管是在隐喻的意义上。[10]

他还能在校外学习英语,因为他的外祖母是英国人。[11] 1906年至1910年间,在弗朗茨·艾克纳(1849-1926)和弗里德里希·哈泽内尔(1874-1915)的指导下,薛定谔在维也纳学习。他还与卡尔·威廉·弗里德里希·弗里茨·科尔劳施进行了实验性工作。

1911年,薛定谔成为艾克纳的助手。早年,薛定谔深受亚瑟·叔本华的影响。由于他对叔本华作品的广泛阅读,他一生都对色彩理论和哲学有浓厚的兴趣。在他的演讲“思想和物质”中,他说“在空间和时间上延伸的世界只是我们的代表。”这重复的是叔本华主要著作中的第一句话。

\subsubsection{1.2 中年}
\begin{figure}[ht]
\centering
\includegraphics[width=6cm]{./figures/3420d552020e95e7.png}
\caption{年轻时的科学家欧文·薛定谔} \label{fig_Erwin_1}
\end{figure}
1914年,埃尔温·薛定谔实现了适应训练(venia legendi)中。1914年至1918年间,他作为奥地利要塞炮兵(戈里齐亚、杜伊诺、西斯蒂安娜、普罗赛柯、维也纳)的委任军官参加了战争工作。1920年,在耶拿,他成为马克斯·维恩的助理,1920年9月,在斯图加特他获得了ao教授(ausserordentlicher教授)的职位,大致相当于Reader(英国)或副教授(美国)。1921年,在布雷斯劳(现在的波兰弗罗茨瓦夫),他成为了o教授(奥德利彻教授,即正教授)。

1921年,他搬到了苏黎世大学。1927年,他在柏林弗里德里希·威廉大学接替马克斯·普朗克的职位。1934年,薛定谔决定离开德国;他不喜欢纳粹的反犹太主义。他成为牛津大学玛格达林学院的研究员。他到达后不久,就和保罗·狄拉克一起获得了诺贝尔奖。他在牛津的职位晋升进展不顺利;他和两个女人住在一起这样非常规的家庭安排,[12] 不能被接受。1934年,薛定谔在普林斯顿大学讲课;他没有接受那里提供的永久的职位。同样,他想和妻子和情妇建房子的愿望也落空了。[13] 他本来有希望在爱丁堡大学找到一份工作,但签证延迟了,最后他于1936年在奥地利的卡尔·弗朗岑斯格拉茨大学找到了一份工作。他还接受了印度阿拉哈巴德大学物理系的主席职位。[14]

1935年,在这些任期问题发生的过程中,在与阿尔伯特·爱因斯坦广泛联系后,他提出了现在所谓的薛定谔猫思想的实验。

\subsubsection{1.3 晚年}
\begin{figure}[ht]
\centering
\includegraphics[width=10cm]{./figures/dba9f32935bff71d.png}
\caption{1942年都柏林高等研究院的薛定谔(右前排第2排)和瓦莱拉(左起第4排)} \label{fig_Erwin_2}
\end{figure}
1938年,继德奥合并之后,由于1933年逃离德国和他众所周知的对纳粹主义的反对,薛定谔遇到了问题。[15] 他发表声明放弃了对纳粹主义的反对意见(后来他后悔这么做,并向爱因斯坦解释了原因)。[16] 然而,这并没有完全使新的政权满意,并且由于政治上的不可靠性,卡尔·弗朗岑斯格拉茨大学解雇了他。他受到骚扰,并收到不离开德国的警告,但他和妻子逃到了意大利。从那里,他接受了牛津和根特大学的客座教授职位。[16][15]

同年,他收到了爱尔兰数学家埃蒙·德·瓦莱拉的个人邀请,邀请他居住在爱尔兰,并同意帮助他在都柏林建立一个高级研究所。[17] 他搬到都柏林的克隆塔尔夫,1940年成为理论物理学院的院长,并在那里呆了17年。他于1948年成为归化的爱尔兰公民,但保留了奥地利公民身份。他还写了大约50本关于各种主题的出版物,包括他对统一场论的探索。

1944年,他写道 《生命是什么?》其中讨论了负熵和具有生物体遗传密码的复杂分子的概念。根据詹姆斯·沃森的回忆录, 《DNA,生命的秘密》,薛定谔的书给了沃森研究基因的灵感,这使得1953年DNA双螺旋结构的成功发现。同样,在弗朗西斯·克里克的自传书《 多么疯狂的追求》中,描述了他如何受到薛定谔关于遗传信息如何储存在分子中的推测的影响。

直到1955年退休前,薛定谔一直呆在都柏林。他一生都对印度教的吠檀多哲学感兴趣,这影响了他在《 生命是什么?》结尾的推测——关于个体意识只是遍及宇宙的单一意识的一种表现的可能性。[18] 手稿“伽利略未发表的对话片段”是为The King's Hospital第155版Blue Coat所写的[19] ,从那时起不久,都柏林的The King's Hospital寄宿学校重新出现了[20] 这是为了庆祝他离开都柏林,就任维也纳大学物理系主任。

1956年,他回到维也纳。在世界能源大会期间的一次重要演讲中,他拒绝谈论核能,因为他对此持怀疑态度,而是做了一次哲学演讲。在此期间,薛定谔偏离了主流量子力学对波粒二象性的定义,独自推动了波的概念,引起了很多争议。

\subsubsection{1.4 个人生活}
\begin{figure}[ht]
\centering
\includegraphics[width=6cm]{./figures/26528637892c1afe.png}
\caption{请添加图片标题} \label{fig_Erwin_3}
\end{figure}
