% 等效原理
% license CCBYSA3
% type Wiki

(本文根据 CC-BY-SA 协议转载自原搜狗科学百科对英文维基百科的翻译)

在广义相对论中,等效原理是引力和惯性质量的等效,阿尔伯特·爱因斯坦的解释是,当站在一个大质量物体(如地球)上时,局部经历的引力“力”与观察者在非惯性(加速)参照系中经历的拟力相同。

\subsection{爱因斯坦关于惯性质量和引力质量相等的陈述}

稍加思考就会发现惯性质量和重力质量相等的定律相当于引力场赋予物体的加速度与物体的性质无关的论断。对于牛顿在重力场中的运动方程,完整地写出来是:
(Inertial mass)   (Acceleration)   (Intensity of the gravitational field)   (Gravitational mass).

只有当惯性质量和重力质量在数值上相等时,加速度才与物体的性质无关。

\subsection{引力理论的发展}

\begin{figure}[ht]
\centering
\includegraphics[width=6cm]{./figures/3a1ba1b6506f55e2.png}
\caption{在1971年的阿波罗15号任务中,宇航员大卫·斯科特证明了伽利略是对的:月球上所有受重力作用的物体的加速度都是一样的,即使是锤子和羽毛。} \label{fig_DXYL_1}
\end{figure}

类似等效原理的东西出现在17世纪初,当时 伽利略 表达 实验盟友 加速 的 测试质量 由于 引力 与数量无关 块 正在加速。

开普勒利用伽利略的发现,通过准确描述如果月球在其轨道上停止并向地球坠落会发生什么,展示了等效原理的知识。这可以在不知道重力是否或以何种方式随距离减小的情况下推导出来,但需要假设重力和惯性相等。

如果两块石头被放置在世界上的任何一个靠近彼此的地方,并且放在第三个同源体的影响范围之外,这些石头就会像两个磁针一样,在中间点聚集在一起,每块石头以与另一块石头的相对质量成比例的空间相互靠近。如果月球和地球没有被它们的万有引力力或其他类似力保持在它们的轨道上,地球将会以它们距离的54分之一登上月球,月球将会通过另外46分之一落向地球,它们将会在那里相遇,然而,前提是假设两者的物质密度相同。

——— Kepler,“新星天文学”,1609年
1/54的比值是开普勒根据月球和地球的直径对它们质量比的估计。他陈述的准确性可以通过牛顿惯性定律$F=ma$和伽利略重力观测即距离$D=1/2at^2$来推断,将质量与加速度设置为相等是等效原理。$D=(1/2) at^2$ 。 注意到每个质量的碰撞时间是相同的,开普勒的陈述是:$D_moon/Dearth = M_Earth/M_moon$ ,既不知道碰撞时间,也不知道重力加速度是否是距离的函数。

牛顿的引力理论简化并形式化了伽利略和开普勒的思想,他认识到开普勒的“万有引力力或其他超越重力和惯性的等效力”是不需要的,从开普勒的行星定律就可推导出重力是如何随距离减小的。

阿尔伯特·爱因斯坦(Albert Einstein)在1907年恰当地引入了等效原理,当时他观察到物体以1g的速度向地球中心加速(g = 9.81 m/s2是地球表面重力加速度的标准参考值)相当于惯性运动物体的加速度,在自由空间的火箭上将观察到该物体以1g的速度加速。爱因斯坦这样说:

我们……假设重力场和参考系统的相应加速度完全物理等价。

——— Einstein,1907年
也就是说,在地球表面就相当于在一艘由引擎加速的宇宙飞船里(远离任何重力源)。地球表面上加速度相等的方向或矢量是“向上”的,或者直接与行星的中心相对,而宇宙飞船中的加速度矢量直接与推进器喷射的质量相反。根据这个原理,爱因斯坦推断自由落体是惯性运动。自由落体中的物体不会经历向下加速(例如朝向地球或其他巨大物体),而是失重和无加速。在惯性系中,参考物体(以及光子或光)遵循牛顿第一定律,以匀速直线运动。类似地,在弯曲的时空中,惯性粒子或光脉冲的世界线会尽可能的笔直(在空间和时间上)。[1] 这样的世界线被称为测地线,从惯性系的角度来看它是一条直线。这就是为什么加速度计在自由落体时没有记录任何加速度;因为没有任何加速度。

举个例子:一个惯性物体沿着测地线在太空中运动,可以被捕获到一个绕着巨大引力质量的轨道中,而不需要经历加速度。这个可能性是存在的,因为时空在接近一个大的引力质量的地方是完全弯曲的。在这种情况下,测地线围绕质心向内弯曲,自由浮动(失重)惯性体将简单地跟随这些弯曲的测地线进入椭圆轨道。机载加速度计永远不会记录任何加速度。

相比之下,在牛顿力学中,重力被认为是一种力。这种力把有质量的物体拉向任何大质量物体的中心。在地球表面,重力被地球表面的机械(物理)阻力抵消。所以在牛顿物理学中,一个人在一个(非旋转的)大质量物体的表面上休息是在惯性参照系中。这些考虑暗示了爱因斯坦在1911年精确阐述的关于等效原理的以下推论:

每当观察者观察到作用在所有物体上的力与每个物体的惯性质量成正比时,该观察者就处于加速参照系中。

爱因斯坦也提到了两个参照系,K和K’。K是一个均匀的引力场,而K’没有引力场,但可以被均匀加速,使得两个参照系中的物体受到相同的力:

如果我们假设系统K和K’在物理上完全相等,也就是说,如果我们假设系统K是在一个没有引力场的空间里并且被均匀加速的,我们就能对这个经验定律做出非常令人满意的解释。这种精确的物理等价的假设使得我们不可能讨论参照系的绝对加速度,就像相对论禁止我们谈论系统的绝对速度一样;这使得引力场中所有物体的平等坠落看起来似乎是理所当然的事。

——— Einstein,1911年
这一观察是最终形成广义相对论的开始。爱因斯坦建议把它提升到一般原理的地位,他在构建相对论时称之为“等效原理”:

只要我们把自己限制在牛顿力学占主导地位的纯机械过程中,我们就能确定系统K和K’的等价性。但是我们的这种观点不会有任何更深层次的意义,除非系统K和K’在所有物理过程中是等价的,也就是说,除非关于K与K’的自然法则完全一致。通过假设这是真的,我们得出了一个有很大的启发性意义的原则。因为通过对相对于具有均匀加速度的参考系统发生的过程的理论考虑,我们获得了关于均匀重力场中过程的重要的信息。

——— Einstein,1911年
爱因斯坦将等效原理与狭义相对论结合起来(假设),预测了在引力势中时钟以不同的速率运行,在他提出弯曲时空的概念之前,甚至预测光线会在引力场中弯曲。

所以爱因斯坦所描述的原始等效原理得出结论,自由落体和惯性运动在物理上是等效的。等价原则的这种形式可以表述如下:在一个无窗的房间里,观察者无法区分是在地球表面上还是在有1g加速的深空宇宙飞船里。严格来说这并不是完全正确的,因为巨大的物体会产生潮汐效应(由引力场的强度和方向的变化引起),而这种效应在深空加速宇宙飞船中是不存在的。因此,房间应该足够小,小到可以忽略潮汐效应。

虽然等效原理指导了广义相对论的发展,但它不是相对论的基本原理,而是该理论几何性质的简单结果。在广义相对论中,自由落体遵循时空测地线,而我们所感知的重力是我们无法遵循时空测地线的结果,因为物质的机械阻力阻止了我们这样做。

由于爱因斯坦发展了广义相对论,就有必要形成一个框架体系来检验这一理论与其他可能的与狭义相对论相兼容的引力理论。罗伯特·迪克提出了两个新的原则,作为他测试广义相对论理论的一部分,即所谓的爱因斯坦等效原理和强等效原理,每个原理都以弱等效原理为出发点。它们只是在是否适用于重力实验方面有所不同。

另一个需要澄清的是,等效原理假设恒定加速度为1g,而不考虑1g的产生机理。如果我们的确要考虑它的机理,那么我们必须假设前面提到的无窗房间有一个固定的质量。在1g加速意味着被施加了一个恒定的力,该力= m*g,其中m是无窗房间及其内装物(包括观察者)的质量。现在,如果观察者在房间内跳跃,那么自由地躺在地板上的物体将会瞬间减轻重量,因为由于观察者为了跳跃而向后推地板,加速度将会瞬间降低。当观察者在空中时,物体会增加重量,因为由此产生的无窗房间减少的质量会导致更大的加速度;当观察者落地并再次推地板时,物体会再次减轻种量;最终,物体会恢复到最初的重量。为了使这些所有的效应与我们在一颗有1g重力的行星上测量的效果相等,我们必须假设无窗房间的质量与该行星相同。此外,没有窗户的房间不能引起自身重力,否则情况会进一步改变。如果我们希望实验或多或少精确地证明1g重力和1g加速度的等效性,这些显然都是技术性且实用的。
\subsection{现代用法}

\subsection{挑战}

\subsection{说明}

\subsection{实验}

\subsection{笔记}


\subsection{参考文献}