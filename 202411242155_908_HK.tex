% 罗伯特·胡克(综述)
% license CCBYSA3
% type Wiki

本文根据 CC-BY-SA 协议转载翻译自维基百科\href{https://en.wikipedia.org/wiki/Robert_Hooke}{相关文章}。

罗伯特·胡克 (Robert Hooke) FRS (/hʊk/;1635年7月18日-1703年3月3日)[4][a] 是一位英国博学者,活跃于物理学(“自然哲学”)、天文学、地质学、气象学和建筑学领域。[5] 他被认为是最早在1665年利用其设计的复合显微镜研究微观生物的科学家之一。[6][7] 胡克年轻时是一位贫困的科学研究者,后来成为他那个时代最重要的科学家之一。[8] 在1666年的伦敦大火之后,胡克以测量员和建筑师的身份,通过完成超过一半的地产界线测绘工作以及协助城市的快速重建,获得了财富和声誉。[9][8] 在他去世后的几个世纪中,胡克经常受到作家的贬低,但在20世纪末,他的名誉得以恢复,并被誉为“英格兰的达·芬奇”。[10]

胡克是皇家学会的院士,从1662年起担任其首任实验策展人。[9] 从1665年至1703年,他还担任格雷沙姆学院的几何学教授。[11] 胡克的科学生涯始于担任物理科学家罗伯特·波义耳(Robert Boyle)的助手。胡克制作了用于波义耳气体定律实验的真空泵,并亲自进行了实验。[12] 1664年,胡克观测到火星和木星的自转。[11] 胡克在1665年出版的著作《显微图谱》(*Micrographia*)中首次提出了“细胞”(cell)一词,这本书激发了显微研究的热潮。[13][14] 在光学领域的研究中——特别是对光折射的研究——胡克提出了光的波动理论。[15] 他是第一个提出以下假说的人:物质因热膨胀的原因,[16] 空气由不断运动的小颗粒组成,并由此产生压力,[17] 以及热是一种能量的概念。[18]

在物理学中,胡克推测重力遵循反平方定律,并且可以说是第一个提出行星运动中这种关系假设的人。[19][20] 这一原理后来被艾萨克·牛顿(Isaac Newton)进一步发展并形式化为牛顿的万有引力定律。[21] 对这一见解的优先权争议促成了胡克与牛顿之间的竞争。在地质学和古生物学中,胡克创立了“水陆球”理论,[22] 因而质疑了《圣经》中关于地球年龄的观点;他还提出了物种灭绝的假说,并认为山丘和山脉是由地质过程抬升而成的。[23] 通过识别已灭绝物种的化石,胡克预示了生物进化论的诞生。[22][24]
\subsection{生平与作品}
\subsubsection{早年生活}  
有关胡克早年生活的大部分信息来自他于1696年开始撰写但未完成的自传;理查德·沃勒 (Richard Waller) 在1705年出版的《罗伯特·胡克博士遗著》序言中提到了这部自传。[25][b] 沃勒的作品,以及约翰·沃德 (John Ward) 的《格雷沙姆教授的生平》[27] 和约翰·奥布里的《简短生平记》[28],构成了胡克生平最重要的同时期传记资料。

胡克于1635年出生在怀特岛的弗雷什沃特 (Freshwater),父母是塞西莉·贾尔斯 (Cecily Gyles) 和英国国教牧师约翰·胡克 (John Hooke),后者是弗雷什沃特全圣教堂的助理牧师。[29] 罗伯特是四个兄弟姐妹中最小的,比其他人小七岁(两个男孩和两个女孩);他身体虚弱,起初并不被认为能活下来。[30][31] 虽然他的父亲教授了他一些英语、(拉丁)语法和神学,但罗伯特的教育大体上被忽视了。[32] 自己摸索着成长的胡克制作了许多机械玩具;看到一个拆开的黄铜时钟后,他用木头制作了一个复制品,并“能正常运转”。[32]

胡克的父亲于1648年10月去世,遗嘱中留给罗伯特40英镑(外加祖母留给他的10英镑)。[33][c] 13岁时,他带着这笔钱去了伦敦,成为著名画家彼得·莱利 (Peter Lely) 的学徒。[35] 胡克还接受了肖像画家塞缪尔·库珀 (Samuel Cowper) 的“绘画指导”,[34] 但“油画颜料的气味不适合他的体质,导致他经常头痛”,于是他成为威斯敏斯特学校校长理查德·布斯比 (Richard Busby) 的学生。[37] 胡克很快掌握了拉丁语、希腊语和欧几里得《几何原本》;[11] 他还学会了弹奏管风琴[38],并开始了对力学的终生研究。[11] 胡克后来在为罗伯特·波义耳的作品和自己《显微图谱》配图时,展示了他精湛的绘画技艺。[39]

1653年,胡克进入牛津大学基督教会学院,他以风琴师和唱诗班成员的身份获得了免费学费和住宿,并通过担任勤务生获得了基本收入,[40][d] 尽管他直到1658年才正式注册。[40] 1662年,胡克获得了文学硕士学位。[38]

在牛津学习期间,胡克还受雇为托马斯·威利斯博士的助理——威利斯是一位医生、化学家和牛津哲学俱乐部的成员。[42][e] 哲学俱乐部由沃达姆学院院长约翰·威尔金斯创立,他领导了这个重要的科学家团体,这个团体后来成为皇家学会的核心。[44] 1659年,胡克向俱乐部描述了一些比空气重的飞行方法的要素,但他得出结论认为人类的肌肉力量不足以实现这一目标。[45] 通过俱乐部,胡克认识了赛思·沃德(大学的萨维利安天文学教授),并为沃德开发了一种机制,以改进用于天文时间测量的摆钟的规律性。[46] 胡克将他在牛津的日子描述为他终生科学热情的基础。[47] 他在这里结识的朋友,尤其是克里斯托弗·雷恩,在他整个职业生涯中都对他至关重要。威利斯还将胡克介绍给罗伯特·波义耳,俱乐部试图吸引波义耳来牛津工作。[48]

1655年,波义耳搬到牛津,胡克名义上成为他的助理,但实际上是他的共同实验者。[48] 波义耳一直在研究气体压力;尽管亚里士多德的格言“自然厌恶真空”广为流传,但真空可能存在的想法刚刚开始被讨论。胡克为波义耳的实验开发了一种空气泵,而不是使用拉尔夫·格雷特雷克斯的泵,胡克认为后者“过于粗糙,无法完成任何重要任务”。[49] 胡克的设备促成了随后归因于波义耳的著名定律的发展;[50][f] 胡克具有特别敏锐的观察力,并且是一个熟练的数学家,这些特质波义耳并不具备。胡克教授波义耳学习了欧几里得的《几何原本》和笛卡尔的《哲学原理》;[9] 他们还通过实验认识到火是一种化学反应,而不是亚里士多德所说的自然界的基本元素。[52]