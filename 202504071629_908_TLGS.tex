% 泰勒公式(综述)
% license CCBYSA3
% type Wiki

本文根据 CC-BY-SA 协议转载翻译自维基百科\href{https://en.wikipedia.org/wiki/Taylor\%27s_theorem}{相关文章}。

\begin{figure}[ht]
\centering
\includegraphics[width=10cm]{./figures/a2db9ae0e729b15e.png}
\caption{指数函数 \( y = e^x \)(红色)及其在原点附近的四阶泰勒多项式(绿色虚线)。} \label{fig_TLGS_1}
\end{figure}
在微积分中,泰勒定理给出了一个\( k \)次可导函数在某个给定点的近似,通过一个\( k \)次多项式,称为\( k \)阶泰勒多项式。对于一个光滑函数,泰勒多项式是该函数泰勒级数在 \( k \)阶的截断。一级泰勒多项式是该函数的线性近似,二级泰勒多项式通常称为二次近似\(^\text{[1]}\)。泰勒定理有多个版本,其中一些版本给出了函数通过其泰勒多项式近似的误差的明确估计。

泰勒定理以数学家布鲁克·泰勒命名,他在1715年提出了该定理的一个版本,\(^\text{[2]}\)尽管早在1671年,詹姆斯·格雷戈里就已提到过该结果的早期版本\(^\text{[3]}\)。

泰勒定理在初级微积分课程中教授,是数学分析中的一个核心基本工具。它提供了简单的算术公式,用于准确计算许多超越函数的值,如指数函数和三角函数。它是解析函数研究的起点,并在数学的各个领域、数值分析以及数学物理中具有基础性意义。泰勒定理也可以推广到多变量和向量值函数。它为一些开创性的早期计算机提供了数学基础:查尔斯·巴贝奇的差分机通过数值积分其泰勒级数的前七项来计算正弦、余弦、对数和其他超越函数。
\subsection{动机}
\begin{figure}[ht]
\centering
\includegraphics[width=10cm]{./figures/4b527c76614bf0b7.png}
\caption{函数 \( f(x) = e^x \)(蓝色)及其在 \( a = 0 \) 处的线性近似 \( P_1(x) = 1 + x \)(红色)。} \label{fig_TLGS_2}
\end{figure}
如果实值函数\( f(x) \)在点\( x = a \)处可导,那么它在该点附近有一个线性近似。也就是说,存在一个函数\( h_1(x) \),使得
\[
f(x) = f(a) + f'(a)(x - a) + h_1(x)(x - a), \quad \lim_{x \to a} h_1(x) = 0.~
\]
这里,
\[
P_1(x) = f(a) + f'(a)(x - a)~
\]
是\( f(x) \)在\( x \) 接近点\( a \)时的线性近似,其图像\( y = P_1(x) \)是\( y = f(x) \)在\( x = a \)处的切线。近似的误差是:\(R_1(x) = f(x) - P_1(x) = h_1(x)(x - a)\).

当\( x \)趋近于\( a \)时,这个误差比\( (x - a) \)收敛得更快,这使得\( f(x) \approx P_1(x) \)成为一个有用的近似。
\begin{figure}[ht]
\centering
\includegraphics[width=10cm]{./figures/5a6062276bf4fb22.png}
\caption{函数 \( f(x) = e^x \)(蓝色)及其在 \( a = 0 \) 处的二次近似 \( P_2(x) = 1 + x + \frac{x^2}{2} \)(红色)。注意到近似的改进。} \label{fig_TLGS_3}
\end{figure}
为了更好地近似\( f(x) \),我们可以拟合一个二次多项式,而不是线性函数:
\[
P_2(x) = f(a) + f'(a)(x - a) + \frac{f''(a)}{2}(x - a)^2.~
\]
这个多项式不仅匹配了\( f(x) \)在\( x = a \)处的一阶导数,而且匹配了二阶导数,正如通过微分可以看出的一样。

泰勒定理确保了在\( x = a \)的足够小邻域内,二次近似比线性近似更准确。具体来说,
\[
f(x) = P_2(x) + h_2(x)(x - a)^2, \quad \lim_{x \to a} h_2(x) = 0.~
\]
这里,近似的误差是
\[
R_2(x) = f(x) - P_2(x) = h_2(x)(x - a)^2,~
\]
考虑到\( h_2 \)的极限行为,随着\( x \)趋近于\( a \),这个误差比\( (x - a)^2 \)收敛得更快。

类似地,如果我们使用更高阶的多项式来近似 \( f \),我们可能得到更好的近似,因为这样我们可以在选定的基点匹配更多的导数。

一般来说,通过一个\( k \)次多项式近似函数时,误差会比\( (x - a)^k \)收敛得更快,随着\( x \)趋近于\( a \)。然而,也存在一些函数,即使是无限可导的函数,对于这些函数,增加近似多项式的阶数并不会提高近似的准确度:我们说这样的函数在\( x = a \)处不具备解析性:它不能仅通过该点的导数来(局部)确定。
\begin{figure}[ht]
\centering
\includegraphics[width=10cm]{./figures/49ba2aed0eecf75b.png}
\caption{函数 \( f(x) = \frac{1}{1 + x^2} \)(蓝色)通过其泰勒多项式 \( P_k \)(红色和绿色)的近似,其中 \( k = 1, \ldots, 16 \),以 \( x = 0 \)(红色)和 \( x = 1 \)(绿色)为中心。近似在区间 \( (-1, 1) \) 和 \( (1 - \sqrt{2}, 1 + \sqrt{2}) \) 之外没有任何改进。} \label{fig_TLGS_4}
\end{figure}
泰勒定理具有渐近性质:它仅告诉我们,使用\( k \)阶泰勒多项式\( P_k \)近似时,误差 \( R_k \)相对于任何非零的 \( k \) 次多项式,随着\( x \to a \)时趋向零的速度更快。它并没有告诉我们在扩展中心的任何具体邻域内误差有多大,但为此目的,存在针对余项的明确公式(如下所示),这些公式在对\( f \)进行某些额外规则性假设时有效。这些增强版的泰勒定理通常会在扩展中心的一个小邻域内提供对近似误差的统一估计,但这些估计不一定适用于过大的邻域,即使函数\( f \)是解析的。在这种情况下,可能需要选择多个具有不同扩展中心的泰勒多项式,才能对原始函数进行可靠的泰勒近似(参见图4的动画)。

我们可以通过余项来使用几种方法:
\begin{enumerate}
\item 估计误差:对于一个度数为\( k \)的多项式\( P_k(x) \),在给定的区间\( (a - r, a + r) \)上近似\( f(x) \)时,估计误差。(给定区间和度数,我们可以找到误差。)
\item 找到最小的度数\( k \),使得多项式\( P_k(x) \)在给定区间 \( (a - r, a + r) \) 上能够将\( f(x) \)近似到给定的误差容限内。(给定区间和误差容限,我们可以找到度数。)
\item 找到最大的区间\( (a - r, a + r) \),使得\( P_k(x) \)在该区间内将\( f(x) \)近似到给定的误差容限内。(给定度数和误差容限,我们可以找到区间。)
\end{enumerate}
\subsection{一元实变量的泰勒定理} 
\subsubsection{定理的陈述}  
泰勒定理最基本版本的精确陈述如下:.

泰勒定理\(^\text{[4][5][6]}\)— 设\( k \geq 1 \)是整数,且函数\( f : R\to R\)在点\(a \in R\)处\( k \)次可导。则存在一个函数\( h_k : R\to R\),使得
\[
f(x) = \sum_{i=0}^{k} \frac{f^{(i)}(a)}{i!} (x - a)^i + h_k(x)(x - a)^k,~
\]
且
\[
\lim_{x \to a} h_k(x) = 0.~
\]
这称为余项的佩阿诺形式。

在泰勒定理中出现的多项式是函数\( f \)在点\( a \)处的\( k \)阶泰勒多项式
\[
P_k(x) = f(a) + f'(a)(x - a) + \frac{f''(a)}{2!}(x - a)^2 + \cdots + \frac{f^{(k)}(a)}{k!}(x - a)^k~
\]
泰勒多项式是唯一的“渐近最佳拟合”多项式,意味着如果存在一个函数\( h_k : R \to R\) 和一个\( k \)阶多项式\( p \),使得
\[
f(x) = p(x) + h_k(x)(x - a)^k, \quad \lim_{x \to a} h_k(x) = 0,~
\]
则\( p = P_k \)。泰勒定理描述了余项的渐近行为
\[
R_k(x) = f(x) - P_k(x),~
\]
这是使用泰勒多项式近似\( f \)时的近似误差。使用小\( o \)符号,泰勒定理中的表述为
\[
R_k(x) = o(|x - a|^k), \quad x \to a.~
\]
\subsubsection{余项的明确公式}  
在对函数\( f \)进行更强的规则性假设下,存在几个精确的泰勒多项式余项\( R_k \)的公式,最常见的公式如下。

余项的均值形式—设\( f : R \to R \)在\( a \)和\( x \)之间的开区间上是\( k + 1 \)次可导,并且\( f^{(k)} \)在\( a \)和\( x \)之间的闭区间上是连续的。\(^\text{[7]}\)那么,
\[
R_k(x) = \frac{f^{(k+1)}(\xi_L)}{(k+1)!}(x - a)^{k+1}~
\]
其中\( \xi_L \)是介于\( a \)和\( x \)之间的某个实数。这是余项的拉格朗日形式\(^\text{[8]}\)。

类似地,
\[
R_k(x) = \frac{f^{(k+1)}(\xi_C)}{k!}(x - \xi_C)^k(x - a)~
\]
其中\( \xi_C \)是介于\( a \)和\( x \)之间的某个实数。这是余项的柯西形式\(^\text{[9]}\)。

两者都可以被视为以下结果的特例:考虑\( p > 0 \)
\[
R_k(x) = \frac{f^{(k+1)}(\xi_S)}{k!}(x - \xi_S)^{k+1-p} \frac{(x - a)^p}{p}~
\]
其中\( \xi_S \)是介于\( a \)和\( x \)之间的某个实数。这是余项的施洛米尔赫形式(有时称为施洛米尔赫-罗什形式)。当选择\( p = k + 1 \)时,得到的是拉格朗日形式,而当选择\( p = 1 \)时,得到的是柯西形式。

泰勒定理的这些改进通常通过均值定理来证明,因此得名。此外,注意到当\( k = 0 \) 时,这实际上就是均值定理。还可以找到其他类似的表达式。例如,如果\( G(t) \)在闭区间上连续,并且在开区间\( (a, x) \)上可导且导数不为零,那么
\[
R_k(x) = \frac{f^{(k+1)}(\xi)}{k!}(x - \xi)^k \frac{G(x) - G(a)}{G'(\xi)}~
\]
其中\( \xi \)是介于\( a \)和\( x \)之间的某个数。这一版本涵盖了余项的拉格朗日形式和柯西形式作为特例,并且使用柯西均值定理证明。通过取\( G(t) = (x - t)^{k+1} \)可以得到拉格朗日形式,而通过取\( G(t) = t - a \)可以得到柯西形式。

余项的积分形式比之前的形式更为复杂,要求理解勒贝格积分理论才能实现完全的通用性。然而,它也适用于黎曼积分的意义,只要\( f \)的\( (k+1) \)阶导数在闭区间\( [a, x] \)上是连续的。

余项的积分形式\(^\text{[10]}\)—设\( f^{(k)} \)在闭区间\( [a, x] \)上是绝对连续的。那么
\[
R_k(x) = \int_a^x \frac{f^{(k+1)}(t)}{k!} (x - t)^k \, dt.~
\]
由于\( f^{(k)} \)在闭区间\( [a, x] \)上是绝对连续的,函数\( f^{(k+1)} \)作为\( L^1 \)-函数存在,且该结果可以通过使用微积分基本定理和分部积分法进行形式化计算来证明。
\subsubsection{余项的估计}  
在实践中,通常需要估计泰勒近似中的余项,而不是得到其精确的公式。假设函数 \( f \) 在包含 \( a \) 的区间 \( I \) 上是 \( (k + 1) \) 次连续可导的。假设存在实数常数 \( q \) 和 \( Q \),使得
\[
q \leq f^{(k+1)}(x) \leq Q~
\]
在整个区间 \( I \) 上成立。那么,余项满足不等式[11]
\[
q \frac{(x - a)^{k+1}}{(k + 1)!} \leq R_k(x) \leq Q \frac{(x - a)^{k+1}}{(k + 1)!}~
\]
当 \( x > a \) 时,且对于 \( x < a \) 也有类似的估计。这是拉格朗日形式余项的一个简单结果。特别地,如果
\[
|f^{(k+1)}(x)| \leq M~
\]
在区间 \( I = (a - r, a + r) \) 上成立,其中 \( r > 0 \),那么
\[
|R_k(x)| \leq M \frac{|x - a|^{k+1}}{(k+1)!} \leq M \frac{r^{k+1}}{(k+1)!}~
\]
对于所有 \( x \in (a - r, a + r) \) 都成立。第二个不等式称为一致估计,因为它对区间 \( (a - r, a + r) \) 上的所有 \( x \) 都成立。
\subsubsection{示例}
\begin{figure}[ht]
\centering
\includegraphics[width=10cm]{./figures/cbea249384dc362d.png}
\caption{函数 \( e^x \)(蓝色)通过其泰勒多项式 \( P_k \)(红色)进行近似,其中 \( k = 1, \ldots, 7 \),并且泰勒多项式是以 \( x = 0 \) 为中心的。} \label{fig_TLGS_5}
\end{figure}
假设我们希望在区间\( [-1, 1] \)上找到函数\( f(x) = e^x \)的近似值,同时确保近似误差不超过\( 10^{-5} \)。在这个示例中,我们假设我们只知道指数函数的以下性质:
\[
e^0 = 1, \quad \frac{d}{dx} e^x = e^x, \quad e^x > 0, \quad x \in \mathbb{R}.~
\]
从这些性质可以得出,\( f^{(k)}(x) = e^x \)对于所有\( k \)都成立,特别地,\( f^{(k)}(0) = 1 \)。因此,函数 \( f \)在\( 0 \)处的\( k \)阶泰勒多项式及其拉格朗日形式的余项给出如下:
\[
P_k(x) = 1 + x + \frac{x^2}{2!} + \cdots + \frac{x^k}{k!},~
\]
\[
R_k(x) = \frac{e^\xi}{(k+1)!} x^{k+1},~
\]
其中\( \xi \)是介于\( 0 \)和\( x \)之间的某个数。由于\( e^x \)是递增的(根据公式★),我们可以直接使用\( e^x \leq 1 \)对于\( x \in [-1, 0] \)来估计区间\( [-1, 0] \)上的余项。为了得到区间\( [0, 1] \)上余项的上界,我们使用性质\( e^\xi < e^x \)对于\( 0 < \xi < x \),从而估计:
\[
e^x = 1 + x + \frac{e^\xi}{2} x^2 < 1 + x + \frac{e^x}{2} x^2, \quad 0 < x \leq 1,~
\]
使用二阶泰勒展开。然后我们求解\( e^x \),得出:
\[
e^x \leq \frac{1 + x}{1 - \frac{x^2}{2}} = 2 \frac{1 + x}{2 - x^2} \leq 4, \quad 0 \leq x \leq 1,~
\]
通过最大化分子并最小化分母。结合这些关于\( e^x \)的估计,我们可以看到:
\[
|R_k(x)| \leq \frac{4 |x|^{k+1}}{(k+1)!} \leq \frac{4}{(k+1)!}, \quad -1 \leq x \leq 1,~
\]
因此,当
\[
\frac{4}{(k+1)!} < 10^{-5} \quad \Longleftrightarrow \quad 4 \cdot 10^5 < (k+1)! \quad \Longleftrightarrow \quad k \geq 9,~
\]
(参见阶乘或手动计算值 \( 9! = 362880 \) 和 \( 10! = 3628800 \))。作为结论,泰勒定理得到如下近似:
\[
e^x = 1 + x + \frac{x^2}{2!} + \cdots + \frac{x^9}{9!} + R_9(x), \quad |R_9(x)| < 10^{-5}, \quad -1 \leq x \leq 1.~
\]
例如,这个近似提供了一个小数表达式:\(e \approx 2.71828\),精确到五个小数位。
\subsection{与解析性的关系}  
\subsubsection{实值解析函数的泰勒展开}  
设 \( I \subset \mathbb{R} \) 为开区间。根据定义,函数 \( f : I \to \mathbb{R} \) 是实值解析的,如果它可以由收敛的幂级数在局部定义。这意味着,对于每个 \( a \in I \),存在某个 \( r > 0 \) 和一组系数 \( c_k \in \mathbb{R} \),使得 \( (a - r, a + r) \subset I \) 并且

\[
f(x) = \sum_{k=0}^{\infty} c_k (x - a)^k = c_0 + c_1 (x - a) + c_2 (x - a)^2 + \cdots, \quad |x - a| < r.
\]

一般来说,幂级数的收敛半径可以通过柯西–哈达玛公式来计算:

\[
\frac{1}{R} = \limsup_{k \to \infty} |c_k|^{\frac{1}{k}}.
\]

这个结果是通过与几何级数的比较得到的,使用相同的方法可以证明,如果基于 \( a \) 的幂级数对于某个 \( b \in \mathbb{R} \) 收敛,则它必须在闭区间 \( [a - r_b, a + r_b] \) 上均匀收敛,其中 \( r_b = |b - a| \)。这里只考虑幂级数的收敛性,并且可能 \( (a - R, a + R) \) 会超出函数 \( f \) 的定义域 \( I \)。