% QCD 费曼规则
% license Usr
% type Tutor


\begin{issues}
\issueMissDepend
\issueAbstract
\end{issues}

QCD 主要考虑夸克内线、胶子(传播子)内线、鬼粒子内线;夸克-胶子相互作用顶点、三胶子顶点、四胶子顶点与胶子-鬼粒子顶点。鬼粒子仅在内线中出现,不会有外线鬼粒子。$f_{abc}$ 为结构常数。

\subsection{内线}
\subsubsection{夸克}
\begin{figure}[ht]
\centering
\includegraphics[width=6cm]{./figures/802e961bc222fa42.png}
\caption{夸克内线} \label{fig_qcdfey_1}
\end{figure}
\begin{equation}
\mathrm i \delta^{mn} \frac{1}{ p\not ~- m} = \mathrm i S^{mn}_F(p)~.
\end{equation}
其中分子处的 $m$ 非指标而是质量。

\subsubsection{胶子}
\begin{figure}[ht]
\centering
\includegraphics[width=6cm]{./figures/72568c83b4b0f177.png}
\caption{胶子内线} \label{fig_qcdfey_2}
\end{figure}
\begin{equation}
\mathrm i D_{F, \mu\nu}^{ab}(p) = -\mathrm i \delta^{ab} \frac{1}{p^2} \left[ g_{\mu\nu} - \left(1-\frac{1}{\xi}\right) \frac{p_\mu p_\nu}{p^2} \right] ~.
\end{equation}
其中 $\xi$ 是规范选择项。

\subsubsection{鬼粒子}
\begin{figure}[ht]
\centering
\includegraphics[width=6cm]{./figures/0222c47fda024c63.png}
\caption{鬼粒子内线} \label{fig_qcdfey_3}
\end{figure}
\begin{equation}
\mathrm i \widetilde{\Delta}_F(q) = \mathrm i \delta^{ab} \frac{1}{q^2} ~.
\end{equation}

\subsection{外线}
外线没有鬼粒子,胶子的规则类似于 Q
