% 线段树
% 线段树|数据结构|C++|高级数据结构

线段树(Segment tree)是一种二叉树形的数据结构,用以存储区间或线段,并且可以在 $O(\log N)$ 的时间复杂度查询区间最大值、最小值、总和等属性.

\textbf{线段树的存储:}

线段树除了最后一层节点外是一棵满二叉树,因此可以用堆\upref{heap}的存储方式来存储线段树.
具体来说就是开一个一维数组,根节点的编号为 $1$,编号为 $x$ 的结点的左子节点的编号为 $x \times 2$,右子节点的编号为:$x \times 2 + 1$,父节点的编号为 $\left\lfloor\dfrac{x}{2}\right\rfloor$.

因此我们可以用一个结构体来存储线段树,线段树除了最后一层结点外是一棵满二叉树,除了最后一层结点外的结点个数为:$N + \dfrac{N}{2} + \dfrac{N}{4} + \cdots + 2 + 1 = 2N - 1$,最后一层的结点个数最坏情况下是 $2N$ 个结点,所以数组大小应不小于 $4N$ 才能保持不越界.

\begin{figure}[ht]
\centering
\includegraphics[width=14.25cm]{./figures/STree_1.png}
\caption{二叉树视角} \label{STree_fig1}
\end{figure}

\begin{figure}[ht]
\centering
\includegraphics[width=14.25cm]{./figures/STree_2.png}
\caption{区间视角} \label{STree_fig2}
\end{figure}


可以看出,线段树的每个结点都代表一个区间,叶结点的区间长度都为 $1$,对于每个区间结点 $[l, r]$,左子结点为 $[l, mid]$,右子结点为 $[mid + 1, r]$,$mid = \left\lfloor\dfrac{l+r}{2}\right\rfloor$.

\textbf{线段树的建树($\text{build}$)操作}:

一般来说,线段树每个结点上存储了很多信息,具体存什么信息得根据具体情况判断,这里以存储区间最大值为例,我们用递归来建树,每个叶结点 $[i, i]$ 保存 $a_i$ 的值,每次递归左子节点和右子节点,最后根据子节点的信息更新当前结点的信息,这一操作称为 $\text{pushup}$ 操作.

\begin{lstlisting}[language=cpp]
struct Node {
    int l, r, v;  // v 代表区间最大值
}tr[4 * N];

void build(int u, int l, int r) 
{
    tr[u] = {l, r};
    
    if (l == r) // 叶节点
    {
        tr[u] = {l, r, a[l]};  // 也可只写 tr[u].v = a[l];
        return;
    }
    
    int mid = l + r >> 1;
    build(u << 1, l, mid);          // 左子节点[l, mid],编号为:u << 1
    build(u << 1 | 1, mid + 1, r);  // 右子节点[mid + 1, r],编号为:u << 1 | 1
    pushup(u);
}

build(1, 1, n);   // 调用建树
\end{lstlisting}

\textbf{线段树的 $\text{pushup}$ 操作:}

线段树可以很容易的把左右两个子结点的信息上传到当前结点,所以在记录每个结点 $i$ 的最大值就可以用左子节点 $\mathtt{i<<1}$ 的最大值和右子节点 $\mathtt{i<<1|1}$ 的最大值两者取一个最大值就是当前结点 $i$ 的最大值.

\begin{lstlisting}[language=cpp]
void pushup(int u)
{
    tr[u].v = max(tr[u << 1].v, tr[u << 1 | 1].v);
}
\end{lstlisting}