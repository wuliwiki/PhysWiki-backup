% 海因里希 赫兹
% license CCBYSA3
% type Wiki

(本文根据 CC-BY-SA 协议转载自原搜狗科学百科对英文维基百科的翻译)

\textbf{海因里希·鲁道夫·赫兹}(/hɜːrts/;德语:[ˈhaɪ̯nʁɪç ˈhɛɐ̯ts];[1][2]1857年2月22日 –1894年1月1日)是一个德国物理学家,他最先令人信服地证明了由詹姆斯·克拉克·麦克斯韦的电磁学方程所预测的电磁波的存在。为了纪念他,频率的单位(每秒周数)被命名为“赫兹”。[3]

\subsection{生平}
海因里希·鲁道夫·赫兹于1857年出生在汉堡的一个富有而有教养的汉萨家庭,汉堡当时是德意志邦联的一个主权国家。他的父亲是古斯塔夫·费迪南·赫兹。[4]他的母亲是安娜·伊丽莎白·普费弗科恩。

在汉堡的学校Gelehrtenschule des Johanneums学习的时候,赫兹就显示出了在科学和语言学习方面的天赋,并学习了阿拉伯语和梵文。他在德国城市德累斯顿、慕尼黑和柏林在古斯塔夫·基尔霍夫和赫尔曼·亥姆霍兹的指导下学习科学和工程。1880年,赫兹从柏林大学获得博士学位,在接下来的三年里,他一直在亥姆霍兹手下做博士后研究并担任他的助手。1883年,赫兹在基尔大学担任理论物理学讲师。1885年,赫兹成为卡尔斯鲁厄大学的正教授。

1886年,赫兹与卡尔斯鲁厄大学的几何学讲师马克斯·多尔博士的女儿伊丽莎白·多尔结婚。他们有两个女儿:生于1887年10月20日的约翰娜以及生于1891年1月14日的玛蒂尔德,后者后来成为一位著名的生物学家。在此期间,赫兹对电磁波进行了里程碑式的研究。

赫兹于1889年4月3日开始在波恩物理研究所担任物理学教授和研究所主任,并在这个职位上一直工作到去世。在此期间,他致力于理论力学的研究,他的研究成果发表在了名为《Die Prinzipien der Mechanik in neuem Zusammenhange dargestellt》的著作中(即《新形式下的力学原理》),这本书出版于他死后的1894年。

\subsubsection{1.1 逝世}
1892年,赫兹在一轮严重的偏头痛后被诊断出感染,并接受手术。他最终于1894年因肉芽肿伴多血管炎在德国波恩去世,葬于汉堡的奥尔斯多夫公墓,享年36岁。[5][6][7][8]

赫兹的妻子伊丽莎白·赫兹(1864-1941)没有再婚。赫兹留下了两个女儿,乔安娜(1887-1967)和玛蒂尔德(1891-1975)。赫兹的女儿们从未结婚,因此他也没有后代。[9]

\subsection{科学工作}
\subsubsection{2.1 电磁波}
\begin{figure}[ht]
\centering
\includegraphics[width=10cm]{./figures/324973487fa4c6ed.png}
\caption{赫兹的1887年用于产生和检测无线电波的装置:火花发射器(左)由偶极天线和一个接收器(右)组成,该偶极天线具有由Ruhmkorff线圈(T)的高压脉冲供电的火花隙(S),接收器则由环形天线和火花间隙组成。} \label{fig_Hertz_1}
\end{figure}
1864年,苏格兰数学物理学家詹姆斯·克拉克·麦克斯韦提出了一个全面的电磁学理论,现在被称为麦克斯韦方程组。麦克斯韦的理论预测耦合的电场和磁场可以以“电磁波”的形式在空间中传播。麦克斯韦提出光包括短波长的电磁波,但没有人能够证明这一点,也没有人能够产生或探测到其他波长的电磁波。

当赫兹在1879年做研究的时候,亥姆霍兹建议赫兹将他的博士论文方向定为对麦克斯韦理论的检验。亥姆霍兹在那一年也将普鲁士科学院的“柏林奖”问题设定为通过实验证明绝缘体极化和去极化的一个由麦克斯韦理论所预测的电磁效应。[10][11]亥姆霍兹确信赫兹是最有可能获胜的候选人。[11]赫兹没有看出任何方法来建造一个仪器以进行实验测试,他认为这太困难了,于是转而研究电磁感应。赫兹在基尔期间确实对麦克斯韦方程组进行了分析,表明它们确实比当时流行的“超距作用”理论更有效。[12]
\begin{figure}[ht]
\centering
\includegraphics[width=6cm]{./figures/d8723858e49c1754.png}
\caption{赫兹的无线电波接收器中的一个:一个圈状天线,并带有可调节的微米火花间隙 (底部).[1]} \label{fig_Hertz_2}
\end{figure}
赫兹在卡尔斯鲁厄获得教授职位后,于1886年秋季用一对里斯螺旋进行了实验,当时他注意到将莱顿瓶放电到一个线圈会在另一个线圈中产生火花。赫兹有了一个关于如何建造仪器的想法,现在有了一个方法来解决1879年那个证明麦克斯韦理论的“柏林奖”问题(尽管实际的奖已经于1882年因为没有人能领取而过期)。[13][14]他用了一个Ruhmkorff线圈驱动火花隙和一米长线对作为发射器。电容球位于末端,用于电路谐振调节。他的接收器是一个简单的半波偶极天线,其末端之间有一个微米火花间隙。这个实验产生并接收了现在所谓的在非常高频率范围内的无线电波。
\begin{figure}[ht]
\centering
\includegraphics[width=14.25cm]{./figures/2c4fa40df05afd00.png}
\caption{赫兹的第一个无线电发射器:一个偶极谐振器,由一对一米的铜线组成,它们之间有7.5毫米的火花间隙,最后是30厘米的锌球。[1]当感应线圈在两侧之间施加高电压时,穿过火花间隙的火花在导线中产生射频电流的驻波,并辐射无线电波。波的频率大约为50 MHz,大约是现代电视发射机中使用的频率。} \label{fig_Hertz_3}
\end{figure}
1886年至1889年间,赫兹进行了一系列实验,证明他观察到的效应是麦克斯韦所预测的电磁波的结果。从1887年11月的论文《绝缘体中电扰动产生的电磁效应》开始,赫兹向柏林科学院的亥姆霍兹发送了一系列论文,包括1888年那篇显示横向自由空间电磁波以有限的速度传播一段距离的论文。[14][15]在赫兹使用的设备中,电场和磁场以横波的形式从电线辐射出去。赫兹将振荡器放置在距离锌反射板约12米的位置以产生驻波。每个波约4米长。使用环形探测器,他记录下了波的振幅和分量方向是如何变化的。赫兹测量了麦克斯韦波,并证明了这些波的速度等于光速。赫兹也测量了波的电场强度、极化和反射。这些实验证明光和这些波都是服从麦克斯韦方程组的电磁辐射。
\begin{figure}[ht]
\centering
\includegraphics[width=10cm]{./figures/9675a86697cec035.png}
\caption{赫兹的定向火花发射器(中心),一个半波偶极天线,由两个13厘米黄铜棒制成,中心有火花间隙(特写左侧),由Ruhmkorff线圈驱动,位于1.2 m x 2 m的圆柱形金属抛物面反射器的焦线上。[8] 它辐射出一束66厘米的波,频率约为450 MHz。接收器(右)是类似抛物线偶极天线,具有微米火花间隙.} \label{fig_Hertz_4}
\end{figure}
\begin{figure}[ht]
\centering
\includegraphics[width=6cm]{./figures/bb88f518466efa24.png}
\caption{请添加图片标题} \label{fig_Hertz_5}
\end{figure}
