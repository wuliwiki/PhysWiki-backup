% 中山大学 2006 年研究生入学物理考试试题
% keys 中山大学|考研|物理|2006年
% license Copy
% type Tutor

\textbf{声明}:“该内容来源于网络公开资料,不保证真实性,如有侵权请联系管理员”

\subsection{简答题}
\begin{enumerate}
\item 一个$a$粒子与$Au$核散射,在实验室坐标系下,若:\\
(1)$Au$ 核视为不动,哪些力学量守恒?\\
(2)考虑$Au$核的质量,哪些力学最守恒?
\item 简述麦克斯韦方程的实验依据与假设。
\item 泡利不相容原理的内容,并由此说明He原子基态(1sls 电子组态)不存在$^3S_1$态。
\end{enumerate}
\subsection{计算题}
\begin{enumerate}
\item 质量为$m$,长为$L$的均质细杆$OA$,能绕通过其$O$端的竖直轴在水平面内自由地旋转。在$A$端作用一个水平力该力垂直于初始位置的细杆,大小和方向不变,当杆从初始位置转过的角度$\theta$时求其角速度。
\item 、已知$ N_2$(可视为理想气体)进行了由绝热、等压和等容三条曲线组成的热机循环(正循环)。其在绝热过程中气体的体积变化有如下两种情况:\\
(1)变为原来的$n$倍;\\
(2)变为原来的$1/n$倍。在P-V图画出这两种循环过程,并求出相应效率。
\item 边长为$a$的正方形线圈与载流长直导线处于同一平面内,正方形近直导线的一边与直导线的距离为$b$。当导线通过的电流以速度$\dv{I}{t}$ 随时间变化时,求线圈中的感应电动势。
\item 用每$1mm$有$500$条栅线的光栅,观察钠黄光($\lambda=5900A$)的谱线。求:\\
(1)当光线垂直入射时,最多能看到几级谱线?若缝宽为$0.001mm$,第几级谱线缺级?
(2)当光线斜入射时(入射角为 $30$°),情况又如何?
\item 质量为$m$质点,在谐振子势$U=\frac{1}{2}kx^2$中运动,且受一阻尼力$f=-\gamma v$作用。已知质点从非平衡位置回到平衡点后刚好静止不动。求:\\
(1)$\gamma$取什么范围?\\
(2)$\gamma$取何值时,质点能最快回到平衡点?
\end{enumerate}
\subsection{试验题}
当波长为$\lambda$的光照射在某材料表面时,出射的电子能量为$E$。实验测得数据如下表所示。根据实验数据和爱因斯坦光电方程($E=hv-\phi$)确定普朗克常数$h$和该材料的逸出功$\phi$。
