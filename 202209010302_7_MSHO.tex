% 多自由度简谐振子(经典力学)
% 简谐振子|mode|模

\pentry{简谐振子(经典力学)\upref{SHO}}

\subsection{二自由度简谐振子}

考虑轻质弹簧与两个光滑滑块构成的体系,如\autoref{MSHO_fig1} 所示.

\begin{figure}[ht]
\centering
\includegraphics[width=10cm]{./figures/MSHO_1.pdf}
\caption{两个滑块和两根弹簧构成的简谐振子.} \label{MSHO_fig1}
\end{figure}

和\textbf{简谐振子(经典力学)}\upref{SHO}的情况一样,弹簧的原长不重要,因此图中没给出.我们用$x_1$和$x_2$分别表示滑块$m_1$和$m_2$的位置,其中$x_1$表示弹簧$k_1$的长度减去其原长,$x_2$表示弹簧$k_2$的长度减去其原长.显然,当$x_1=x_2=0$时两根弹簧都处于原长.

则这个体系的运动方程为
\begin{equation}
\leftgroup{
    m_1\ddot{x}_1 &= -k_1x_1+k_2x_2\\
    m_2\ddot{x}_1 &= -k_2x_2
}
\end{equation}

这个体系也是也是一个简谐振子.由于它需要两个参数$x_1, x_2$来刻画,因此也称作\textbf{二自由度}的简谐振子.

当然,我们也可以在\autoref{MSHO_fig1} 的右边加一堵墙,再加一根弹簧将$m_2$和右边的墙连接起来,得到的同样是二自由度简谐振子.


\begin{figure}[ht]
\centering
\includegraphics[width=8cm]{./figures/MSHO_2.pdf}
\caption{另一种二自由度简谐振子的模型.图中方块是固定的墙面,只有粉色的圆球有质量,其它构件都是轻质的.红色和蓝色分别表示只能沿着水平或垂直方向移动的滑杆,粉色圆球被固定在这两个滑杆中.两个滑杆分别连接到图示的两根弹簧上.} \label{MSHO_fig2}
\end{figure}

\autoref{MSHO_fig2} 所示的模型也是一个二自由度简谐振子,两个参数分别是圆球的横位移和纵位移.

一般地,二自由度简谐振子的运动方程写为
\begin{equation}\label{MSHO_eq1}
\leftgroup{
    \ddot{x}_1 &= k_{11}x_1+k_{12}x_2\\
    \ddot{x}_2 &= k_{21}x_1+k_{22}x_2
}
\end{equation}


\subsubsection{二自由度简谐振子的方程解法}

类比\textbf{简谐振子(经典力学)}\upref{SHO}中的解法,我们可以猜想















