% 唯一析因环

% 唯一分解环|唯一分解整环|唯一析因|唯一分解|因式分解|因子分解|UFD|整环

\pentry{真因子树\upref{FctTre}}

唯一析因环,又称唯一分解整环,顾名思义,就是每个元素都有唯一的不可约因式分解.这个定义用真因子树的语言来描述颇为方便.

\begin{definition}{唯一析因环}
对于整环 $R$,如果它具有\textbf{有限析因性}和\textbf{唯一析因性}\footnote{见\textbf{真因子树}词条的\autoref{FctTre_def1}~\upref{FctTre}.},那么称其为一个\textbf{唯一析因环(unique factorization domain)},常简称UFD.
\end{definition}

“唯一析因”在有的地方也称为“唯一分解”.

唯一析因环的好处显而易见,每个元素都可以唯一对应一种素元素分解,比如每个非平凡整数都可以表示为素数的乘积,且这种乘积是唯一的.

\begin{example}{正面例子}
\begin{itemize}
\item 整数环 $\mathbb{Z}$ 是唯一析因环.
\item 域 $\mathbb{F}$ 上的多项式环 $\mathbb{F}[x]$ 是唯一析因环.
\end{itemize}
\end{example}

唯一析因环的概念脱胎于一个经典的错误.柯西等人曾以为自己证明了\textbf{费马大定理},而事实上他们的证明依赖了一个直觉上成立的假设,用现代语言来说就是“所有的环都是唯一析因环”.然而很可惜,存在不唯一析因的环,这也成了此类证明中的命门.

\begin{example}{反面例子}
我们给出一些不是唯一析因环的例子.
\begin{itemize}
\item 对于素数 $p>1$,环 $\mathbb{Z}[\sqrt{-p}]=\{a+b\I\sqrt{p}|a, b\in\mathbb{Z}\}$ 存在有多种因式分解方式的元素.比如说,在环 $\mathbb{Z}[2]$ 上,$6$ 可以分解为 $2\times (1+\I\sqrt{2})\times(1-\I\sqrt{2})$,也可以分解为 $(2+\I\sqrt{2})\times(2-\I\sqrt{2})$.
\item 复数域上的全体\textbf{整函数}构成的环不是UFD,因为存在可以无限析因的元素\footnote{本例取自维基百科\href{https://en.wikipedia.org/wiki/Unique_factorization_domain}{相关页面}.}:$\sin{\pi z=\pi z\prod\limits_{i=1}^\infty(1-\frac{z^2}{i^2})}$.
\end{itemize}
\end{example}

\addTODO{关于整函数的概念,缺少“全纯函数”词条作为预备知识.}
