% 中科院 2012 年考研普通物理
% keys 中科院|2012|普通物理
\subsection{选择题}
1. 如\autoref{CAS12_fig1} 所示, 两个固定小球质量分别是 $m_{1}$ 和 $m_{2}$, 在它们的连线上总可以找到一
点 $p$, 使得质量为 $m$ 的质点在该点所受到的万有引力的合力为零, 则质点在 $p$
点的万有引力势能\\
(A) 与无穷远处的万有引力势能相等;\\
(B) 与该质点在两小球连线间其它各点处相比势能最大;\\
(C) 与该质点在两小球连线间其它各点处相比势能最小;\\
(D) 无法判定.
\begin{figure}[ht]
\centering
\includegraphics[width=3.5cm]{./figures/CAS12_1.pdf}
\caption{选择题1图示} \label{CAS12_fig1}
\end{figure}
2.两个全同的均质小球 ${A}$ 和 ${B}$ 都放置在光滑水平面上, 球 ${A}$ 静止.在某一时刻球 ${B}$ 与 ${A}$ 发生完全弹性斜碰撞(即碰撞时 ${A}$、 ${~B}$ 的质心连线方向与球 ${B}$ 的速度方向不同),则碰撞后两球的速度方向\\
(A) 相同;$\quad$
(B) 夹角为锐角;$\quad$
(C) 相垂直;$\quad$
(D) 夹角为钝角.

3. 一质点同时参与相互垂直的两个谐振动, 且振动的频率相等.下列说法错误的是\\
(A) 若两振动的初相位相同, 则质点轨迹为直线段;\\
(B) 若两振动的初相位相差 $\pi / 4$, 且振幅相等, 则质点轨迹为椭圆;\\
(C) 若两振动的初相位相差 $\pi / 2$, 且振幅不相等, 则质点轨迹为䧎圆;\\
(D) 若两振动的初相位相差 $\pi$, 且振幅相等, 则质点轨迹为圆.

4. 一个电量为 $q$ 、质量为 $m$ 的带电粒子在匀强磁场中作半径为 $r$ 的圆周运动.如果运动的频率是 $f$, 则磁感应强度大小为\\
(A) $\frac{4 \pi m f}{q}$;$\quad$
(B) $\frac{3 \pi m f}{q}$;$\quad$
(C) $\frac{2 \pi m f}{q}$;$\quad$
(D) $\frac{\pi m f r}{q}$ .

5. 无限长直导线均匀带电, 电荷线密度为 $\lambda$ .距直导线距离 $r$ 处的电场强度大小为\\
(A) $\frac{\lambda}{4 \pi \varepsilon_{0} r^{2}}$;$\quad$
(B) $\frac{\lambda}{4 \pi \varepsilon_{0} r}$;$\quad$
(C) $\frac{\lambda}{2 \pi \varepsilon_{0} r^{2}}$;$\quad$
(D) $\frac{\lambda}{2 \pi \varepsilon_{0} r}$ .

6. 在国际单位制中, 磁通量的量纲为\\
(A) $M L^{2} T^{-2} I^{-1}$;$\quad$
(B) $M L^{2} T I^{-1}$;$\quad$
(C) $M L^{2} T^{-2} I$;$\quad$
(D) $M L T^{-2} I^{-1}$ .

7. 关于平衡态下理想气体,以下哪个说法是错误的?\\
(A) 分子大小比分子间的平均距离小得多, 分子的大小可以忽略不计;\\
(B) 除碰撞瞬间外, 分子之间以及分子与容器壁之间都没有相互作用力;\\
(C) 各个分子的速度大小相同;\\
(D) 分子向各个方向运动的几率均等.

8. 动能相同的电子与质子的德布罗意波长哪个较长?\\
(A) 电子;$\quad$
(B) 质子;$\quad$
(C) 一样长;$\quad$
(D) 不能确定

\subsection{简答题}

1. 荡秋千时,为什么人可以越荡越高, 而固定在秋千上的物体却越荡越低?试 分析其原因并简述之.

2. 试写出真空中麦克斯韦方程组的积分形式, 并简述位移电流的含义.

3. 什么是牛顿环?它的特点是什么?
\subsection{解答题}
1.两根相同的均质杆 $A B$ 和 $B C$, 质量均为 $m$, 长均为 $l, A$ 端被光 滑铰链到一个固定点, 两杆始终在竖直平面内运动. $C$ 点有外力使得两杆保持静止, $A $、$ C$ 在同一水平线上, $\angle A B C=90^{\circ}$ .某时刻撤去该力,\\
(1) 若两杆在 $\mathrm{B}$ 点固结在一起, 求初始瞬间两杆的角加速率;\\
(2) 若两杆在 $\mathrm{B}$ 点光滑铰接在一起, 求初始瞬间两杆的角加速率.
\begin{figure}[ht]
\centering
\includegraphics[width=4cm]{./figures/CAS12_2.pdf}
\caption{解答题1图示} \label{CAS12_fig2}
\end{figure}

2. 如图所示, 一弹性系数为 $k$ 、原长为 $l$ 的水平轻质弹簧一端固定 在墙上, 另一端连接一个质量为 $M$ 的滑块 $\mathrm{A}$ .在外力作用下, 弹簧被压缩了 距离 $d$, 另有一个质量为 $m$ 的滑块 $B$ 紧靠 $A$ 放置, 系统保持静止.在 $t=0$ 时 刻,突然撤去外力,忽略系统摩擦, 试求:\\
(1) 什么时候 $B$ 与 $A$ 脱离?\\
(2) 脱离后,$A$ 的位移随时间的变化关系(设平衡位置为零点)?
\begin{figure}[ht]
\centering
\includegraphics[width=3cm]{./figures/CAS12_3.pdf}
\caption{解答题2图示} \label{CAS12_fig3}
\end{figure}
3. 一同心球形电容器, 内导体球半径为 $a$, 外导体半径为 $b$, 中间 充满不均匀的电介质, 介电常数为 $\varepsilon=\varepsilon_{0} /(1+k r)$, 其中 $\varepsilon_{0}$ 为真空介电常数, $k$ 为 常数, $r$ 是距球心的距离.若内导体球带电量为 $Q$, 外球接地, 试求:\\
(1) 当 $a<r< b$ 时, 电位移矢量 $\vec{D}(r)$; \\
(2) 电容器的电容 $C$; \\
(3) 当 $a<r< b$ 时, 极化电荷密度 $\rho(r)$; \\
(4) 在 $r=a$ 和 $r=b$ 处的极化电荷面密度.

4.一个均匀带电的圆环, 半径为 $R$, 总电量为 $Q$, 圆环绕通过圆心
垂直于环面的轴匀速转动, 角速度为 $\omega$ .求:\\
(1) 圆环中心处的磁感应强度;\\
(2) 轴线上离圆心 $a$ 处的磁感应强度.

5. 一个均匀带电的圆环, 半径为 $R$, 总电量为 $Q$, 圆环绕通过圆心 垂直于环面的轴匀速转动, 角速度为 $\omega$ .求:\\
(1) 圆环中心处的磁感应强度;\\
(2) 轴线上离圆心 $a$ 处的磁感应强度.

6.波长为 $600 \mathrm{~nm}$ 的单色平行光垂直通过直径为 $3.0 \mathrm{~cm}$ 、焦距为 $50 \mathrm{~cm}$ 的薄凸透镜,求透镜像方 Airy 斑的直径.