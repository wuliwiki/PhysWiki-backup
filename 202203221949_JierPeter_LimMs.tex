% 依测度收敛
% 实变函数|测度|极限|收敛

\pentry{可测函数\upref{MsbFun}}

我们已经知道,函数有“逐点收敛”和“一致收敛”等不同的收敛方式.逐点收敛的概念最简单,但是性质不太好;一致收敛的性质就非常好.

现在我们有了外测度的概念,就可以构造一种新的收敛方式.

\begin{definition}{依测度收敛}

设$E\subseteq \mathbb{R}^n$是可测集,$f$和$f_i$都是其上的\textbf{几乎处处有限的}可测函数,其中$i$取遍全体正整数.如果对于任意固定的$\epsilon>0$,有
\begin{equation}
\lim\limits_{i\to\infty}\opn{m}\{x\in E|\abs{f_i(x)-f(x)}\geq\epsilon\}=0
\end{equation}
则称函数列$\{f_i\}$\textbf{依测度收敛}到函数$f$,记为$f_i\overset{\opn{m}}\to f$(于$E$).

\end{definition}

简单来说,依测度收敛就是指任取一个精度范围$\epsilon$,$f_i(x)$偏离$f(x)$超过精度范围的$x$称为“不听话”的点,那么随着$i$增大,不听话的点构成的集合的外测度趋于零.

类似地,一致收敛可以简单解释为:任取一个精度范围$\epsilon$,$f_i(x)$偏离$f(x)$超过精度范围的$x$称为“不听话”的点,那么随着$i$增大,不听话的点构成的集合趋于空集.

发现没?简单说法里,一致收敛和依测度收敛的简单解释,只有最后一句有差别.从这个差别可以容易看出,一致收敛能推出依测度收敛.更进一步,几乎处处一致收敛也能推出依测度收敛.















