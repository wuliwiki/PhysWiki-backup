% 波的能量

\pentry{一维波动方程\upref{WEq1D}}

当机械波传播到介质中的某处时,该处原来不动的质点开始振动,因而具有\textbf{动能},同时该处的介质也将产生形变,因而也具有\textbf{势能}.

波动传播时,介质由近及远地振动着,由此可见,能量是向外传播出去的.这是波动的重要特征.

\begin{figure}[ht]
\centering
\includegraphics[width=5cm]{./figures/WaEner_1.png}
\caption{一小段线元} \label{WaEner_fig1}
\end{figure}

如\autoref{WaEner_fig1}所示,在弦线上在$x$处取线元$\Delta x$,设弦线的线密度(单位长度的质量)为$\rho_l$,其质量为$\rho_l\Delta x$.
当弦线中有平面简谐波传播时,设波函数为
\begin{equation}
y=A \cos \left[\omega\left(t-\frac{x}{u}\right)+\phi_{0}\right]
\end{equation}
线元的动能为
\begin{equation}
\Delta E_{\mathrm{k}}=\frac{1}{2} \rho_{l} \Delta x\left(\frac{\partial y}{\partial t}\right)^{2}
\end{equation}
