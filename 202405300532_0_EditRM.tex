% 编辑器使用说明
% keys 小时百科|在线编辑器|latex
% license CCBYSA3
% type Tutor

欢迎使用小时百科/云笔记编辑器。

该编辑器同时用于\href{https://wuli.wiki/editor/}{编辑小时百科}以及\href{https://wuli.wiki/note/}{小时云笔记}两个板块。 它们的功能几乎相同, 但百科有合作编辑功能而云笔记暂时没有。另外百科对文章的格式会有更多限制。
\begin{figure}[ht]
\centering
\includegraphics[width=13cm]{./figures/19c6cc6482ff004d.png}
\caption{编辑器截图(\href{https://wuli.wiki/apps/editor.gif}{查看 GIF 动画})} \label{fig_editor_3}
\end{figure}

\subsection{如果您已经会 LaTeX}
本编辑器只支持 LaTeX 的公式\footnote{公式渲染和知乎等网站一样使用\href{https://www.mathjax.org/}{MathJax}。 MathJax。} 以及其他一些简单的 LaTeX 命令(见菜单栏), 所以\textbf{如果您直接把别处的完整 LaTeX 代码复制进来几乎肯定会遇到严重的问题!编辑器仅支持我们定义的模板,不要尝试定义或修改任何模板设置!}

整个百科(或一个用户的整个云笔记,下同)是一个 \verb|document| 环境, 主文件是 \verb|main.tex|(点击右上角的“打开文章”按钮,第一个就是), 其他所有 \verb|文章id.tex| 文件都将作为一个 \verb|\section| 插入 \verb|main.tex| 中。 这需要手动在 \verb|main.tex| 中使用我们定义的 \verb|\entry{文章名}{文章id}| 命令, 相当于在该位置定义 \verb|\section{文章名}\label{文章id}| 然后把 \verb|文章id.tex| 文件的内容直接插入。 所以本质上,整个百科只有 \verb|main.tex| 一个文件。 相比于完整的 LaTeX 编译器,我们的网页编辑器进行了一些优化,例如可以单独编译一篇文章而不是整个 \verb`main.tex`(通常需要很长时间)。

以下大部分不是 LaTeX 的教程而是编辑器的教程, 所以同样建议您看一看。 可能\textbf{大大提高编辑效率}的功能有: 引用按钮(自动添加 \verb|\label| 和 \verb|\autoref| 来引用公式图表等), 自动补全和符号替换(可以在设置里面自定义), 以及各种快捷键(\autoref{tab_editor_1} )。

\textbf{编辑器零基础视频教程}(\href{https://www.bilibili.com/video/av87698355/}{Biblibili}, \href{https://zhuanlan.zhihu.com/p/105869878}{知乎}, \href{https://www.youtube.com/watch?v=AN2tXNanD9U&t=1s}{YouTube})。
% \item \addTODO{提高编辑效率 (未完成): 使用代码补全, 符号显示, 自动引用, 实时预览, 符号显示。}
% \item \addTODO{其他功能(未完成): 编译 pdf, 转载到知乎等。}

\subsection{注册和申请}
要使用编辑器,您需要进行\href{https://wuli.wiki/forum}{注册}, 登录成功后, 访问 \href{https://wuli.wiki/editor/}{wuli.wiki/editor} 即可(云笔记则访问 \href{https://wuli.wiki/note/}{wuli.wiki/note})。
\begin{itemize}
\item \textbf{编辑百科:} 如果您只是进行一些细微的修改如错别字和笔误,则无需申请成为作者。 但如果您需要进行更多创作或修改, 则需要在\href{https://wuli.wiki}{网站主页}申请成为作者。 您的任何修改都需要我们在后台审核后方可生效(也可能不采用),这个过程一般需要少于一天。
\item \textbf{云笔记:}云笔记功能目前对所有注册用户开放,无需申请。 云笔记目前已经进入稳定阶段,数据会永久保留,定期备份,请放心使用。 注意我们正在进行频繁更新,可能偶尔会出现无法编辑,但不会丢失数据,一般一天内会恢复。
\end{itemize}

\subsection{创作流程}
\begin{itemize}
\item 进入编辑器初始页面后, 点\textbf{新建文章}(红色加号)按钮可以新建文章(一个 \verb`tex` 文件), 点\textbf{打开文章}(红色文件夹)按钮可以编辑已有文章。
\item 申请成为百科作者后默认拥有新建和编辑普通文章的权限, 但却不能直接发布文章, 而是需要拥有发布权限的用户审核通过后发布。 没有发布的文章可以在 \href{http://wuli.wiki/changed/}{wuli.wiki/changed/} 中看到, 单无法在 \href{http://wuli.wiki/changed/}{wuli.wiki/online/} 或者 \href{http://wuli.wiki/book/}{wuli.wiki/book/} 中看到。
\item 若一篇文章被某个作者编辑, 则该文章被该作者占用。 其他用户无法修改该文章,只能以只读模式打开,除非有 “占用管理” 权限。 当文章被取消占用或被发布时,才会取消占用。
\item 目前,管理员会每天对所有新建和被修改的文章进行粗略的审核,并对其进行发布(即使处于草稿阶段)。这主要是为了防止误删、恶意编辑和违规内容。更深入的讨论一般在创作群中。
\item 新建的文章不会自动出现在目录中,需要手动编辑 \verb`main.tex` 以修改目录。
\item 目录文件 \verb{main.tex`和参考文献列表 \verb`bibliography.tex` 都会每 6 分钟会自动取消占用。
\end{itemize}

\subsection{编辑器简介}

\begin{itemize}
\item 【新】新增了直接粘贴图片功能, 例如先使用操作系统自带的截图, 再在代码窗口按 \verb|Ctrl+V|, 可以自动上传剪切板中的图片并创建图片环境。
% \item 【新】新增了实时预览功能, 即改变代码时预览自动更新。 该功能仍然有一些小 bug 待修复, 如果遇到问题可以在设置面板中关闭。 在非实时预览模式下, 编辑过程中用快捷键 \verb|Ctrl+S| 可以保存并刷新预览(建议经常刷新, 便于定位错误)。 也可以用工具栏的保存图标。
\item 【语言】本编辑器使用 LaTeX 语言的一个子集, 一个简单的 LaTeX 介绍见 \href{https://wuli.wiki/online/latxIn.html}{LaTeX 结构简介}。 除公式外,绝大部分支持的命令都可以通过工具栏插入, 所有支持的命令见\enref{小时百科文章示例}{Sample}。
\item 【自定义命令】我们在模板中用 \verb|\newcommand{}{}| 加入了一些自定义命令, 但不会覆盖原有的 LaTeX 命令。 若希望加入新的自定义命令, 请与管理员协商, 也可以使用下文的 “自动补全” 功能作为代替。
\end{itemize}

\subsection{公式}
\begin{itemize}
\item 公式环境支持大部分 LaTeX 命令, 严格来说是所有 \href{https://www.mathjax.org/}{MathJax} 支持的命令。
\item 一个简单的公式编辑器见\href{https://www.codecogs.com/latex/eqneditor.php}{这里} (不建议使用, 建议练习手动输入)。
\item 一个简单的 TeX/LaTeX 公式入门教程见\href{https://chaoli.club/index.php/211}{超理论坛}。
\item 本编辑器额外支持支持部分 \href{http://mirrors.ibiblio.org/CTAN/macros/latex/contrib/physics/physics.pdf}{Physics 宏包}中的命令,以及百科模板中自定义的快捷命令(见\enref{文章示例}{Sample})。
\item 行内公式插入到两个美元符号之间, 如 \verb|$a^2+b^2=c^2$| 显示为 $a^2 + b^2 = c^2$。
\item 独立公式只能用 \verb|equation| 环境(推荐), \verb|align| 环境或者 \verb|gather| 环境。 \verb|equation| 环境可以通过工具栏的公式图标插入, 也可以打 \verb|\beq| 然后按 Tab 键或者回车插入, 如
\begin{equation}\label{eq_editor_1}
a^2 + b^2 = c^2~,
\end{equation}
\item 公式中所有常用的和自定义的命令见\enref{文章示例}{Sample}。
\item 为增加代码可读性, 公式中一些命令会显示为对应的符号(如希腊字母, 求和符号, 不等号等), 注意这不会影响源码(复制时得到的也是命令而不是符号)。 设置面板(齿轮图标)可以选择关闭该功能。
\item 工具栏中的 “内部引用” 按钮可以引用同一页面的公式并生成链接(图片表格等同理), 如 “\autoref{eq_editor_1}”。 “外部引用” 按钮可以引用其他文章的公式或图表。
\end{itemize}

\subsection{LaTeX 结构}

整个百科(或用户笔记)是 LaTeX 的一个 \verb|\document| 环境, 目录中每个 “部分” 是一个 \verb|\part|, 每个 “章” 是一个 \verb|\chapter|, 每篇文章是一个 \verb|\section|, 文章中蓝色的小标题是 \verb|\subsection|, 黑色的小标题是 \verb|\subsubsection|。 编辑器打开的一篇文章的文件就是一个 section 的内容(不需要 \verb|\section| 命令)。 用 TeXlive 编译 pdf 的时候所有文章文件都会通过 \verb|\input{xxx.tex}| 插入到主文件 \verb`main.tex` 中。

网页版的百科文章目录由 \verb|main.tex| 文件生成, 所以必须把新建的文章在这里插入并保存才能更新目录。 否则虽然页面可以访问但却不会出现在目录中。

每篇文章文件(后缀名为 tex)都有一个独一无二的文件名(即 \verb`文章id`),限制为小于等于 6 个字母或数字,不区分大小写。可以将通过将光标停留在编辑器中的 tab 上查看或者通过地址栏的 url 查看。

\begin{figure}[ht]
\centering
\includegraphics[width=4cm]{./figures/89cb63348bbde05a.png}
\caption{查看文件名(\verb`文章id`)} \label{fig_editor_2}
\end{figure}

每篇文章的 label 与文件名相同, 转换后输出的网页文件(html)也有相同的文件名, 可以在浏览器的地址栏中看到。例如本文的 LaTeX 文件是 \verb|EditRM.tex|, label 是 \verb|EditRM|, 网址(旧版界面)为 \href{https://wuli.wiki/online/EditRM.html}{wuli.wiki/online/EditRM.html}, 新界面的网址为 \href{https://wuli.wiki/EditRM}{wuli.wiki/EditRM}。

\subsection{菜单栏}
\begin{itemize}
\item 将光标停留在任意按钮上都会出现提示说明按钮的名称。 要新建文章, 点击红色的加号按钮, 根据提示新建即可。 要打开已有文章, 点击最右边的打开, 搜索需要的文章即可。
\item \textbf{加粗}、\textbf{斜体}、\textbf{大标题}、\textbf{小标题}按钮:可以在您选中一段文字后按下,给这段文字添加相应格式。 也可以按下以后再填写需要的文字。
\item \textbf{历史版本}按钮(红色的时钟):编辑器会将有改动的文章每隔 5 分钟备份一次, 可以用查看历史版本,可以勾选其中两个进行对比,也可以点击其中一个进行恢复。
\item \textbf{内部引用}按钮(实心双引号)可以引用同一文章的公式, 图表, 例题,子节等。 \textbf{外部引用}按钮(空心双引号)可以引用其他文章的各种环境。 被引用对象没有 label 时会自动插入 label。
\item \textbf{窗口定位按钮}如果要在网页预览和 LaTex 代码之间跳转到对应位置, 可以通过搜索关键词实现。 例如在预览窗口复制一段文字, 在编辑窗口搜索就可以跳转到对应内容。(更新:现在可以直接选中后用 \verb`Alt` + \verb`F` 或者定位按钮)
\item \textbf{下载源文件}(蓝色云朵箭头按钮)仅限在\href{http://wuli.wiki/note/}{小时云笔记}中使用。 点击菜单栏的 “下载” 按钮可以下载所有 tex 文件以及图片, 然后用 TeXlive 的 XeLaTeX 编译 \verb|main.tex| 即可(需要编译 2 到 4 次,取决于 pdf 的页数), 推荐使用 TeXlive 2019 或者更新版本。 TeXlive 的使用详见\enref{安装使用 TeXlive}{TeXliv}”。 在 Linux 环境中也可以直接用 \verb|make| 命令编译(会自动编译足够的次数)。
\item \textbf{导出 Markdown 文件(转载到知乎)}按钮:普通用户仅限在\href{http://wuli.wiki/note/}{小时云笔记}中使用(百科编辑器中只有管理员可用), 点击 “导出 Markdown 文件” 按钮生成与知乎兼容的 md 文件(矢量图和表格等暂不兼容), 在知乎的回答或文章编辑器中直接导入该文件即可。
\item \textbf{全文搜索}按钮(地球和放大镜) 用于搜索百科所有文章中的 LaTeX 代码。 输入关键词即可, 默认支持\enref{正则表达式}{regex}, 区分大小写。 该功能是通过命令行的 \verb|git grep| 命令实现的, 具体命令为 \verb|git grep --no-index 用户输入|, 详细功能参考\href{https://git-scm.com/docs/git-grep}{官方文档}。
\end{itemize}

\subsubsection{预备知识}
\begin{itemize}
\item \textbf{预备知识按钮}用于插入预备知识列表 \verb`\pentry{知识点1\nref{节点id1},知识点2\nref{节点id2}}{节点id}`。
\item 一个预备知识列表定义了\href{https://wuli.wiki/tree/}{知识树}中的一个节点(简单说就是知识点),它的范围是它到下一个预备知识列表之间(如果没有下一个就到文末)的全部内容。 该节点的 ID 就是 \verb`节点id`。
\item 每篇文章无论是否存在预备知识列表,都会有一个\textbf{默认的节点} \verb`nod_XXX` 其中 \verb`XXX` 是该文章的 \verb`文章id`。
\item 同一篇文章的每个节点默认依赖于上一个节点,默认节点依赖最后一个节点。
\item 如果学习某节点前\textbf{必须}先弄懂一些\textbf{其他文章}中的节点,就在 \verb`\pentry{}{}` 的第一个花括号中把它们列出来,并使用 \verb`\nref{}` 引用它们的 ID。
\item \verb`\nref{}` 命令可以通过 “外部引用” 按钮中的 “节点” 按钮来插入。 在提示 “输入序号” 时,如果不输入,则引用一篇文章的默认节点(即依赖于整篇文章)。
\end{itemize}

\subsection{编辑器说明}
\begin{table}[ht]
\centering
\caption{Windows 快捷键}\label{tab_editor_1}
\begin{tabular}{|c|c|c|c|}
\hline
保存文章 & \verb|Ctrl| + \verb|S| & 打开文章 & \verb|Ctrl| + \verb|O| \\
\hline
新建文章 & \verb|Ctrl| + \verb|Alt| + \verb|N| & 关闭文章 & \verb|Ctrl| + \verb|Alt| + \verb|W| \\
\hline
查找文本 & \verb|Ctrl| + \verb|F| & 替换文本 & \verb|Ctrl| + \verb|H| \\
\hline
增大字号 & \verb|Shift| + \verb|Alt| + \verb|+| & 减小字号 & \verb|Shift| + \verb|Alt| + \verb|-| \\
\hline
显示编辑器选项 & \verb|Ctrl| + \verb|Q| & 跳转到某行 & \verb|Ctrl| + \verb|G| \\
\hline
撤销 & \verb`Ctrl` + \verb`Z` & 重做 & \verb`Ctrl` + \verb`Y` \\
\hline
向左缩进 & \verb`Tab` 或 \verb|Ctrl| + \verb|[| & 向右缩进 & \verb`Tab` 或 \verb|Ctrl| + \verb|]| \\
\hline
关闭不保存 & \verb|Shift| + \verb|点击关闭| & 注释选中的行 & \verb`Ctrl` + \verb`K` 松开再按 \verb`C` \\
\hline
\end{tabular}
\end{table}

\begin{table}[ht]
\centering
\caption{Mac 快捷键}\label{tab_editor_2}
\begin{tabular}{|c|c|c|c|}
\hline
保存文章 & \verb|Cmd+S| & 打开文章 & \verb|Cmd+O| \\
\hline
新建文章 & \verb|Cmd+Opt+N| + \verb|Alt| + \verb|N| & 关闭文章 & \verb|Ctrl+Opt+W| \\
\hline
查找文本 & \verb|Cmd+F| & 替换文本 & \verb|Cmd+H| \\
\hline
增大字号 &  & 减小字号 & \\
\hline
显示编辑器选项 & \verb|Ctrl+Q| & 跳转到某行 & \verb|Ctrl+G| \\
\hline
向左缩进 & \verb|Cmd+ [| & 向右缩进 & \verb|Cmd+]| \\
\hline
关闭不保存 & \verb|Shift+点击关闭| &  &  \\
\hline
\end{tabular}
\end{table}

\begin{itemize}
\item 【新】快捷功能: 若剪切板有图片(例如截图以后), 在编辑器中直接用 \verb|Ctrl+V| 就可以上传该图片。
\item 正文中请使用中文标点, 编辑器会自动把空心句号替换为全角实心句号(如果您在使用笔记功能,可以选择在设置中关闭这个功)。
\item 双击文章的 tab 也可以关闭文章。
\item 任何时候打出反斜杠会自动提示可以自动补全的命令, 用上下键选择, 用 Tab 键或回车确认。 候选词未必是从最左边开始匹配, 例如打 \verb|\tbf| 按 tab 就会得到 \verb|\textbf{}|。
\item 如果自动补全带括号, 例如 \verb|\frac{}{}|, 补全后光标会自动进入第一个大括号, 再次按 Tab 光标会跳到第二个括号, 再按 Tab 光标会跳到第二个大括号外。
\item 打 \verb|\beq| 按 Tab 会自动出现 \verb|\begin{equation}...\end{equation}|, 其他环境也同理(\verb`itemize` 环境用 \verb`\bit`, \verb`enumerate` 环境用 \verb`benu` 等)。
\item 用光标选中一串字符后按下加粗按钮, 这串字符会自动插入 \verb|\textbf{}| 中。 同样, 选中字符串后输入 \verb|(| 等括号, 这个字符串会自动插入 \verb|()| 中。 表示行内公式的 \verb|$$| 也支持该操作。 用 “对齐” 按钮添加 \verb|aligned| 环境同理。
\item 编辑器在生成网页时会将编辑器中的 LaTex 代码转换为通用的 LaTex 代码(网页中右键点击公式获得), 即不需要自定义命令和额外宏包, 可以在任何支持 LaTeX 公式的环境使用(如知乎的公式编辑器)。
\item 编辑器支持一键转载到知乎(有少量不兼容, 例如表格和代码), 使用菜单栏的 “导出 Markdown 文件” 按钮, 然后将导出的文件上传到知乎即可。 该功能只能在编辑个人笔记时使用, 未经允许请勿转载百科内容。
\item 若要搜索整个百科的 LaTex 源码, 用\href{https://github.com/MacroUniverse/PhysWiki-log/tree/master/contents}{这个页面}, 若要搜索所有作者的编辑历史, 用\href{https://github.com/MacroUniverse/PhysWiki-backup}{这个页面}。
\end{itemize}

\subsection{编辑器设置}
编辑器中的 “设置” 按钮(齿轮图标)可以添加 “自动补全” 规则。 “自动补全” 如上文所描述的, 在输入 LaTeX 命令的过程中, 候选框会显示可以补全的命令, 用上下键选择命令, 然后用 tab 键或回车键补全。 补全规则的格式说明可以点击设置面板中的帮助按钮获得。

【已停用】 “符号替换” 功能是指, 在 LaTeX 公式中输入一些命令时, 编辑器会自动将其显示为对应的符号, 例如 \verb|\alpha| 显示为 \lstinline|α|, \verb|\sum| 显示为 \lstinline|∑| 等。 这样做是为了增加源码的可读性, 注意这只是一种视觉效果, 不会影响源码本身。 设置面板中的也可以添加 “符号替换” 规则或者将其关闭。

\addTODO{详细介绍设置面板的其他按钮}
\addTODO{tab 可以拖动}
