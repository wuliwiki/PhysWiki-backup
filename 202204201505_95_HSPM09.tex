% 机械振动(高中)
% keys 机械振动|弹簧振子|简谐运动|单摆|共振

\begin{issues}
\issueDraft
\issueTODO
\end{issues}

物体或物体的一部分在某个位置附近所做的往复运动叫做\textbf{机械振动},简称\textbf{振动}.

\subsection{简谐运动}

由弹簧和小球组成的系统,叫做\textbf{弹簧振子},其中的小球叫做\textbf{振子}.弹簧振子是一种理想模型,研究其运动时,小球被视为质点,并忽略弹簧的质量以及运动过程中的阻力.

\begin{figure}[ht]
\centering
\includegraphics[width=5cm]{./figures/HSPM09_1.png}
\caption{弹簧振子} \label{HSPM09_fig1}
\end{figure}

\autoref{HSPM09_fig1} 为安置在光滑水平面的弹簧振子,弹簧的一端被固定.弹簧处于自然状态时,振子静止,所受合力为零,此时振子所处的位置叫\textbf{平衡位置}.

当沿水平方向拉动(或推动)振子使其偏离平衡位置并释放,振子将在平衡位置的两侧做往复运动(振动).

振动过程中,振子在竖直方向上所受合力为零,在水平方向上受到弹簧弹力$\bvec F$的作用.弹力$\bvec F$的方向与偏离平衡位置的位移方向相反,总是指向平衡位置,其作用是将振子拉回平衡位置,这个力$\bvec F$叫做\textbf{回复力}.根据胡克定律可知:
\begin{equation}
\bvec F=-k\bvec x
\end{equation}
