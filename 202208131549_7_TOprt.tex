% 时间演化算符(量子力学)
% 薛定谔方程

\pentry{量子力学的基本原理(量子力学)\upref{QMPrcp}}


本节介绍时间演化算符,以此为切入点,引入量子态的演化方程,即薛定谔方程.


“我们应当记住的首要之点是:时间在量子力学中只是一个参量而不是一个算符.特别地,时间不是前一章所说的可观测量.像谈论位置算符一样谈论时间算符是无意义的.”——樱井纯,J. 拿波里塔诺,《现代量子力学》,2.1节.

量子力学中,时间不是一个算符,意味着量子力学认为时间是独立存在的,即采用经典时空观.


\subsection{时间演化算符}

\begin{definition}{时间演化算符}\label{TOprt_def1}
设一个物理系统在时间$t$时的态矢量为$\ket{s, t}$,而$t_0<t$是一个初始时间,那么定义
\begin{equation}\label{TOprt_eq1}
\mathcal{U}(t, t_0)\ket{s, t_0}=\ket{s, t}
\end{equation}
其中$\mathcal{U}(t, t_0)$称为从$t_0$到$t$的\textbf{时间演化算符(time evolution operator)}.
\end{definition}

从\autoref{TOprt_def1} 可以看出来,我们只需要研究清楚时间演化算符的性质,就能从一个初始态算出之后任意时间的态.

那么时间演化算符应该具有什么样的性质呢?


首先,时间演化算符必须满足\textbf{结合性},即
\begin{equation}\label{TOprt_eq2}
\mathcal{U}(t_2, t_1)\mathcal{U}(t_1, t_0) = \mathcal{U}(t_2, t_0)
\end{equation}
这样才能确定唯一的演化结果$\ket{s, t}$.

有了结合性,我们可以省略掉初始时间,而把$\mathcal{U}(t, t_0)$简记为$\mathcal{U}(t-t_0)$,即把自变量由“初始时间和结束时间”替换为“演化所用时间”.同样,也可以把量子态$\ket{s, t_0}$简记为$\ket{s}$.此时$\ket{s, t_0+t}=\mathcal{U}(t)\ket{s}$.

接着,我们希望量子态随时间\textbf{连续地}变化,因此有
\begin{equation}\label{TOprt_eq3}
\lim_{t\to 0}\mathcal{U}(t) = \mathcal{U}(0) = 1
\end{equation}
其中$1$是恒等变换.

最后,我们希望一个量子态归一化以后,在演化过程中始终保持归一化.也就是说,$\bra{s}H^\dagger H\ket{s}=\braket{s}{s}=1$对任意态$\ket{s}$成立,即
\begin{equation}\label{TOprt_eq4}
H^\dagger H=1
\end{equation}
满足\autoref{TOprt_eq1} 的方程被称为\textbf{幺正(unitary)}的.



有了三条规则,\autoref{TOprt_eq2} ,\autoref{TOprt_eq3} 和\autoref{TOprt_eq4} ,就可以推导时间演化算符的具体形式了.


\subsubsection{无穷小时间演化算符}

我们首先考虑演化用时趋于$0$时,时间演化算符的极限.这是因为微分的思想即线性近似的思想,而线性的情形是最好处理的.为了方便,我们将不使用极限语言,而是用“无穷小”的术语,这并不失严谨性.

记$\dd t$是一段无穷小时间,则由连续性\autoref{TOprt_eq3} ,可知应设
\begin{equation}\label{TOprt_eq5}
\mathcal{U}(\dd t) = 1+\Omega \dd t
\end{equation}
其中$\Omega$是某个确定的算符.

\autoref{TOprt_eq5} 的形式天然满足结合性:
\begin{equation}
\begin{aligned}
\mathcal{U}(t_2)\mathcal{U}(t_1)&=(1+\Omega \dd t_2)(1+\Omega \dd t_1)\\
&=1+\Omega(\dd t_2+\dd t_1)\\
&=\mathcal{U}(t_2+t_1)
\end{aligned}
\end{equation}

接下来要确定$\Omega$的形式.根据幺正性\autoref{TOprt_eq4} ,我们有
\begin{equation}
(1+\Omega^\dagger \dd t)(1+\Omega \dd t)=1
\end{equation}
展开后,剔除高阶无穷小项$\dd t^2$,可得到
\begin{equation}
\Omega^\dagger + \Omega = 0
\end{equation}
因此,$\Omega$是一个\textbf{反厄米算符}(\autoref{QMPrcp_def19}~\upref{QMPrcp}).


现在,借用经典力学中“哈密顿量是时间演化生成元”的概念,令$\Omega$为哈密顿算符$H$的某个倍数.考虑到哈密顿算符是可观测量(能量),应为\textbf{厄米算符},再考虑到量纲,故可以设
\begin{equation}\label{TOprt_eq6}
\Omega=-\frac{\I}{\hbar}H
\end{equation}

这里我们直接给出了调整量纲的常量$\hbar$.为什么是$\hbar$,而不是$h$或别的什么同量纲量呢?这个问题在接下来的运动方程中就能得到解答.



\subsubsection{一般的时间演化算符}

从\autoref{TOprt_eq5} 和\autoref{TOprt_eq6} ,我们已经得到了无穷小极限下时间演化算符的线性近似:
\begin{equation}
\mathcal{U}(\dd t)=1-\frac{\I}{\hbar}H \dd t
\end{equation}

由结合性,可得
\begin{equation}
\mathcal{U}(t+\dd t) = \mathcal{U}(t)\mathcal{U}(\dd t) = \mathcal{U}(t)(1-\frac{\I}{\hbar}H \dd t)
\end{equation}

因此有\footnote{$\partial\mathcal{U}(t)/\partial t=(\mathcal{U}(t+\dd t)-\mathcal{U}(t))/\dd t$.}
\begin{equation}
\I\hbar\frac{\partial}{\partial t}\mathcal{U}(t) = H\mathcal{U}(t)
\end{equation}
















