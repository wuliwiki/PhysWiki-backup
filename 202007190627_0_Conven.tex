% 本书符号与规范
% 符号|规范|单位制

这里列出本书中一些可能会产生歧义的符号以及规范, 并给出对应词条的链接.

\subsection{数学}
\begin{itemize}
\item 较长的整数可以使用逗号对每三位进行分割, 如 $2,456,789$.
\item 集合\upref{Set} 中 $\subseteq$ 和 $\subset$ 表示子集, $\subsetneq$ 表示真子集, 但应该尽量避免使用 $\subset$.
\item 映射\upref{map} 的分类使用\autoref{map_def1}.
\item 使用粗体与正体表示几何矢量\upref{GVec}, 如 $\bvec v$, 也可以用于表示列矢量或行矢量以及矩阵, 矩阵一般用大写, 如 $\mat A$.
\item 在粗体与正体矢量上方加 hat 表示单位矢量, 如 $\uvec x$.
\item 也可以用狄拉克符号表示任意矢量空间的矢量, 如 $\ket{v}$, 对偶矢量如 $\bra{v}$, 内积如 $\braket{u}{v}$.
\item 如无声明词条默认使用国际单位制\upref{Consts}, 若使用其他单位制, 要在每个词条开头用脚注声明 “本词条使用 xxx 单位制”. 例如原子单位\upref{AU}, 厘米—克—秒\upref{CGS}, 或高斯单位制\upref{GaussU}.
\end{itemize}

\subsection{电动力学}
\begin{itemize}
\item 使用 $\epsilon$ 而不是 $\varepsilon$ 表示电介质常量.
\end{itemize}

\subsection{相对论}
\begin{itemize}
\item 用 $\gamma$ 表示 $1/\sqrt{1 - v^2/c^2}$.
\end{itemize}

\subsection{统计力学}
\begin{itemize}
\item $\Xi$ 表示配分函数.
\item $k_B$ 表示玻尔兹曼常数.
\item 如果温度使用能量单位, 意思是该温度(开尔文温标)乘以玻尔兹曼常数 $k_B T$.
\end{itemize}
