% 极端曲线
% 高维|极端曲线|极值|必要条件

\pentry{绝对极值与相对极值(变分学)\upref{AbPol},欧拉方程(变分学)\upref{ElueEV}}

从欧拉方程\upref{ElueEV} 一节中,可以知道极值曲线是使二维空间中泛函 
\begin{equation}
J(y)=\int_a^b F(x,y,y')\dd x
\end{equation}
取极值的曲线,它满足欧拉方程.当推广到高维空间时,我们称使 $n$ 维空间中泛函 
\begin{equation}\label{ExtCur_eq1}
J(y_1,\cdots, y_n)=\int_{a_0}^{a_1}F(x;y_1,\cdots,y_n;y_1',\cdots,y_n') 
\end{equation}
取极值的曲线 
\begin{equation}
y_1=y_1(x);\cdots;y_n=y_n(x)
\end{equation}
中的每一个为\textbf{极端曲线}.式中,$a_0,a_1$ 分别是 $n$ 维空间中可取曲线的起止点的参数值. 
在高维空间中,泛函\autoref{ExtCur_eq1} 取极值的必要条件满足如下定理
\begin{theorem}{}
如果曲线 
\begin{equation}
y_1=y_1(x),y\cdots,y_n=y_n(x)
\end{equation}
属于可取曲线 $C_1$ 类,并且给出积分 
\begin{equation}
J(y_1,\cdots, y_n)=\int_{a_0}^{a_1}F(x;y_1,\cdots,y_n;y_1',\cdots,y_n') 
\end{equation}
 的极值.其中,$F$ 和它对所有变数的二阶以内的偏微商都是连续的.那么函数 $y_1=y_1(x),y\cdots,y_n=y_n(x)$ 满足微分方程组
\begin{equation}
F_{y_1}-\dv{}{x}F_{y_1'}=0\cdots, F_{y_n}-\dv{}{x}F_{y_n'}=0
\end{equation}

\end{theorem}
