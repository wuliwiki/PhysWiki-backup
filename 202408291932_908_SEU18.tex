% 东南大学 2018 年 考研 量子力学
% license Usr
% type Note

\textbf{声明}:“该内容来源于网络公开资料,不保证真实性,如有侵权请联系管理员”

\subsection{(共30分,时小题:分)列断題(以下叙述是否正确)}
\\(1)坐标表象中的波函数可以完全描述一个量子态,但动量空间的波函数不能完全描述一个量子态;\\
(2)若体系具有空间反演对称性,则其能量本征态一定是偶宇称态;\\
(3)对任何力学体系,只要将经典哈密顿函数中的所有动力学变量换位相应的力学量算符,就可以得到正确描述该体系量子力学效应的哈密顿算算符;\\
(4)自由粒子的宇称,能量,动量,角动量均为守恒量;\\
(5)除特殊情况外,幺正算符一般不是力学量算符;\\
(6)对于磁场中的带电粒子,薛定谔方程和所有可观测量均具有规范不变性;\\
(7)若一个体系的量子态空间未付为3,则所有可能的量子态数目为3;
(8)任何时空变换对称性均对应于一个守恒量;\\
(9)某些双原子分子的振动光谱线强度随频率的分布表现出强弱交替现象,这与全同粒子系统量子态的交换对称性或反对称性有关。\\
(10)碱金属原子光谱的双线结构与电子的自旋-轨道耦合效应有关。\\
\subsection{(共30分,每小题3分)选择题:}

1. 以下关于Pauli算符的等式,哪个是错误的:\\
   (A) $\hat{\sigma}_z^2 = 1$; \\
   (B) $\hat{\sigma}_x \hat{\sigma}_y = i\hat{\sigma}_z$; \\
   (C) $\hat{\sigma}_x \hat{\sigma}_y = -\hat{\sigma}_y \hat{\sigma}_x$; \\
   (D) $\hat{\sigma}_x \hat{\sigma}_y = \hat{\sigma}_y \hat{\sigma}_x$\\

2.设$\hat{a}^\dagger$为谐振子的上升算符,能量本征态为$|n\rangle (n=0,1,\cdots)$,则$\hat{a}^\dagger \ket{n}$等于\\
   (A) $0$;\\ 
   (B) $|n\rangle$; \\
   (C) $\sqrt{n+1}|n+1\rangle$; \\
   (D) $\sqrt{n-1}|n-1\rangle$\\

3. 设体系的基态能为$E_0$,在某个量子态下的能量平均值为$\bar{H} $,则必有\\
   (A) $\bar{H} \geq E_0$; \\
   (B) $\bar{H} \leq E_0$; \\
   (C) $\bar{H} = E_0$; \\
   (D) $\bar{H}< E_0$\\

4. 设$\langle \psi|A|\phi\rangle$,则$\langle \phi|B|\psi \rangle$的表达式为\\
   (A) $B\langle \phi|A^\dagger \psi\rangle$; \\
   (B) $A^\dagger \langle \phi|B^\dagger \psi\rangle$; \\
   (C) $\langle \phi|B|A^\dagger \psi \rangle$; \\
   (D) $\langle \phi|B|A \psi \rangle$\\

5.设 $|\psi\rangle = \hat{A}\hat{B}|\phi\rangle$,则 $\langle \psi |$ 的表达式为\\
   (A) $\hat{B} \hat{A}\ket{\phi}$;; \\
   (B) $ \hat{A} \hat{B} \ket{\phi}$; \\
   (C) $\langle \phi | \hat{B}^{\dagger} \hat{A}^{\dagger}$; \\
   (D)  $\langle \phi | \hat{B}^{\dagger} \hat{A}^{\dagger}$\\