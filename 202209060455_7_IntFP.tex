% 场与粒子的相互作用
% keys 场论|相互作用|interaction|电磁场

\pentry{自由粒子拉格朗日函数(狭义相对论)\upref{FPLSR}}

本文中$c=1$,闵可夫斯基空间度规为$g_{\mu\nu}\opn{diag}(1, -1, -1, -1)$.

本文的目的是简单介绍描述场与粒子相互作用的拉格朗日方法之\textbf{思路},启发自Suskkind的\textsl{Special Relativity and Classical Field Theory}\cite{SusskindClassicalFields}.特别要强调,我们并不是把粒子和场分开讨论,得到两种作用量或者拉格朗日函数;我们研究的是粒子和场构成的整体,只有对这个整体的\textbf{一个}作用量.

\subsection{场对粒子的作用}

粒子的运动轨迹由拉格朗日函数决定,因此要体现场对粒子的作用,就需要在粒子的拉格朗日函数里有场的出现.

自由粒子的拉格朗日函数\upref{FPLSR}为$L(t, x^\mu, \dot{x}^\mu ) = -m\sqrt{1-\dot{x}^\mu \dot{x}^\nu g_{\mu\nu}}$.如何添加一个“相互作用”的项呢?注意到和粒子有关的部分是$\sqrt{1-\dot{x}^\mu \dot{x}^\nu g_{\mu\nu}}$,我们带着这部分,简单地把标量场$\phi$加进去试试:
\begin{equation}\label{IntFP_eq1}
\mathcal{L}(t, x^\mu, \dot{x}^\mu ) = -(m+g\phi)\sqrt{1-\dot{x}^\mu \dot{x}^\nu g_{\mu\nu}}
\end{equation}
其中$g$是一个常数,常称为\textbf{耦合常数(coupling constant)},用来表征场对粒子的作用强度.

把\autoref{IntFP_eq1} 代入粒子的欧拉-拉格朗日方程,得到粒子的运动方程
\begin{equation}\label{IntFP_eq2}
\frac{\dd }{\dd t} \qty[-\qty(m+g\phi)\frac{-\dot{x}^\rho}{\sqrt{1-\dot{x}^\mu \dot{x}^\nu g_{\mu\nu}}}] = -g\frac{\partial \phi}{\partial x^\rho}\sqrt{1-\dot{x}^\mu \dot{x}^\nu g_{\mu\nu}}
\end{equation}

如果取\textbf{非相对论}和\textbf{弱场}极限,即$\dot{x}^\rho\ll c$且$g\phi$和$\dot{x}^\mu \dot{x}^\nu g_{\mu\nu}$相当,那么\autoref{IntFP_eq1} 化为
\begin{equation}\label{IntFP_eq3}
\mathcal{L}(t, x^\mu, \dot{x}^\mu ) = -m-g\phi+\frac{1}{2}mv^2
\end{equation}
其中$v^2=\dot{x}^\mu \dot{x}^\nu g_{\mu\nu}$.

于是\autoref{IntFP_eq2} 化为、或者说\autoref{IntFP_eq3} 的欧拉-拉格朗日方程为
\begin{equation}\label{IntFP_eq4}
\frac{\dd}{\dd t} m\dot{x}^\rho = -g\frac{\dd \phi}{\dd x^k}
\end{equation}
这正是粒子在标量势下的运动方程.

如果$\phi$是引力势场,$g=m$是物体的质量,那\autoref{IntFP_eq4} 描述的就是低速、弱场近似下(牛顿理论)粒子在引力场中的运动,故可以说物体的质量是引力场的耦合常数.

\begin{example}{电磁场}

正文中讨论的是最简单的标量势的情况,现在我们尝试讨论一个矢量场.



\end{example}






\subsection{粒子对场的作用}
\pentry{狄拉克 delta 导函数\upref{delta2}}

由于我们是把粒子和场作为一个整体,不可割裂,因此粒子拉格朗日函数中的相互作用项也应该是场的相互作用项.

以上一小节讨论的\autoref{IntFP_eq1} 为例,我们希望$-g\phi\sqrt{1-\dot{x}^\mu \dot{x}^\nu g_{\mu\nu}}$也是影响场演化的相互作用项.问题是,粒子的作用量是对时间的积分:
\begin{equation}
\mathcal{S}_{\text{粒子相互作用项}} = \int -g\phi\sqrt{1-\dot{x}^\mu \dot{x}^\nu g_{\mu\nu}}\dd t
\end{equation}
因为粒子有给定轨迹.但场的作用量是对一片时空区域积分,怎么办呢?

首先要再次把问题简化成低速、弱场的情况,这样粒子的拉格朗日函数相互作用项就是
\begin{equation}
\mathcal{L}_{\text{粒子相互作用项}} = -g\phi
\end{equation}

只需要添加狄拉克函数,就可以把粒子的作用量从对时间的积分改写为对时空区域的积分:
\begin{equation}
\ali{
\mathcal{S}_{\text{粒子相互作用项}} &= \int -g\phi \dd t \\
&= \int -g\delta^3(x^1(t), x^2(t), x^3(t))\phi \dd x^4
}
\end{equation}
其中$\delta^3$是三维狄拉克函数,在任意参考系中,取任意时间断面,它的中心点就在粒子的轨迹上.

场的拉格朗日函数自由项取\autoref{CFa1_ex1}~\upref{CFa1}中的$\frac{1}{2}\partial_\mu\phi\partial_\nu\phi g^{\mu\nu}$(加了个系数$1/2$,下面会看到用处何在).加上相互作用项,凑出场的拉格朗日函数:
\begin{equation}
\mathcal{L}_{\text{场}} = \frac{1}{2}\partial_\mu\phi\partial_\nu\phi g^{\mu\nu} - g\delta^3(x^1(t), x^2(t), x^3(t))\phi
\end{equation}

现在代入场的欧拉-拉格朗日方程\autoref{CFa1_eq5}~\upref{CFa1}看看:
\begin{equation}
(\partial_\mu\partial_\nu)\phi g^{\mu\nu} = -g\delta^3(x^1(t), x^2(t), x^3(t))
\end{equation}
或者写成更明显的形式:
\begin{equation}\label{IntFP_eq5}
(\partial_0^2-\partial_1^2-\partial_2^2-\partial_3^2)\phi = -g\delta^3(x^1(t), x^2(t), x^3(t))
\end{equation}

考虑一个极为简单的特殊情况:粒子和场都保持静止,不随时间变化,此时有$\partial_0^2\phi = 0$.假设粒子总待在点$(x, y, z)$,即$\delta^3(x^1(t), x^2(t), x^3(t))=\delta^3(x, y, z)$.

于是\autoref{IntFP_eq5} 变成
\begin{equation}
(\partial_1^2+\partial_2^2+\partial_3^2)\phi = g\delta^3(x, y, z)
\end{equation}
这正是泊松方程.也就是说,这里的场$\phi$可以解释为电势,$(x, y, z)$是电荷静止的位置,ou'he$g$













