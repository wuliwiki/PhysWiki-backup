% 用 Matlab 制作 gif 动画

这里给出一个 Matlab 代码 \verb|im2gif.m|, 通过选中文件夹中的多个图片来生成 gif 动画. 运行程序后, 会弹出文本框提示用户选择多张图片. 图片会按照文件名排序. 选择以后, 为了减小 gif 的大小, 每张图片在读取之后将会把颜色划分为指定的个数 \verb|Ncolor|, 选中图片越多, 颜色数越多, gif 的文件就越大. 然后程序以指定的文件名生成 gif 文件, gif 的每一帧间隔为 \verb|dt| 秒.

\begin{lstlisting}[language=matlab, caption=im2gif.m]
function im2gif()
% === 参数 =====
fname = 'animation.gif'; % 文件名
Ncolor = 64; % 颜色数(最大 256)
dt = 0.1; % 间隔时间(秒)
% ================
[filename,path] = uigetfile('*.*', 'multiselect', 'on');
cd(path);
I = imread(filename{1});
[X,cmap] = rgb2ind(I,Ncolor,'nodither');
imwrite(X, cmap, fname, 'gif', 'Loopcount', inf, 'DelayTime', dt);
figure;
for ii = 2:numel(filename)
    if strcmp(filename{ii}, fname)
        continue;
    end
    I = imread(filename{ii});
    [X,cmap] = rgb2ind(I,Ncolor,'nodither');
    imshow(X, cmap); drawnow;
    imwrite(X, cmap, fname, 'gif', 'WriteMode', ....
            'append', 'DelayTime', dt);
end
end
\end{lstlisting}

要用 Matlab 制作一个完整的动画(例如这个演示\href{https://wuli.wiki/apps/Lim0.html}{数列极限的动画}), 可以先写一个程序画好每一帧后用 \verb|saveas()| 自动保存为 \verb|1.jpg|, \verb|2.jpg|, 等:
\begin{lstlisting}[language=matlab]
% 数列极限的动画演示
close all;
A = 1;
n = 1:40;
an = (-1).^(n+6) ./(n+6).^1.2 + A;
figure;
set(gcf, 'units','pixels','outerposition',[200 200 800 400]);
epsilon = linspace(0.06, 0.03, 60);
plot(n, an);
axis([0, n(end), -0.1+A, 0.1+A]);
xlabel n; ylabel a_n;
set(gca, 'FontSize', 14);
hold on;
k = 0;
for eps = epsilon
    k = k + 1;
    hf = fill([0, 1, 1, 0]*n(end), [-1, -1, 1, 1]*eps+A, ....
        [233,30,99]/255, 'FaceAlpha', 0.1, 'EdgeColor', 'none');
    m1 = abs(an - A) > eps;
    m2 = ~m1;
    hs1 = scatter(n(m1), an(m1), 'marker', 'o', 'SizeData', 30,....
        'MarkerEdgeColor', 'none', 'MarkerFaceColor', 'b');
    hs2 = scatter(n(m2), an(m2), 'marker', 'o', 'SizeData', 30,....
        'MarkerEdgeColor', 'none', 'MarkerFaceColor', 'r');
    ht = text(find(m2,1), eps + A + 0.02, ....
        ['n > ' num2str(find(m2,1)-1)], 'FontSize', 14);
    ht2 = text(28, eps + A + 0.02, ....
        ['|a_n - ' num2str(A) '| < ' num2str(eps, '%4.3f')], 'FontSize', 14);
    pause(0.16);
    saveas(gcf, [num2str(k) '.jpg']);
    delete(hf); delete(hs1); delete(hs2);
    delete(hs3); delete(ht); delete(ht2);
end
\end{lstlisting}
