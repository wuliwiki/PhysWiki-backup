% 南京大学 2012 年考研普通物理
% 南大|南京大学|普物|普通物理
\addTODO{画图}
\subsection{力学}
1. 在赤道无风带有一因无风而停止的船.船长决定采用将 $m=200\opn{Kg}$ 的锚升到 $s=20\opn m$ 的桅杆顶端的权宜之计使船运动,船其余部分的质量为 $M=100 \opn{Kg}$ .求船的运动速度的大小与方向.

2. 质量为 $m$ 的子弹以速度 $v_0$ 射入质量为 $M$,半径为 $R$ 的圆盘边缘并留在该处.$v_0$ 的方向与射入处的半径方向垂直,试就以下两种情况\\
(1) 盘心处装有一与圆盘垂直的光滑固定轴;\\
(2) 圆盘是自由的;\\
求子弹入射圆盘系统总动能之比.

3. 劲度系数为 $k$ 的弹簧两端连接质量为 $m_1$、$m_2$ 的两物体,系统置于光滑水平面上,求系统作相对振动时的圆频率.
\subsection{热学}
1. 一个定压热容量为 $C_p$(常数),温度为 $T_i$ 的物体,与温度为 $T_f$ 的热源在定压条件下接触而达到平衡,求总熵变.

2. 对于某理想气体,已知
\begin{equation}
S=\frac n2 (\sigma+5R\ln{\frac Un}+2R\ln{\frac Vn})
\end{equation}
其中 $n$ 为摩尔数,$U$ 为气体内能,$V$ 为体积,$\sigma$ 是常数\\
(1) 求定压和定容比热;\\
(2) 有一间年久漏风的屋子,起初屋子的温度与室外平衡为 $0^\circ\mathrm{C}$,生炉子3小时后,室内温度$20^\circ\mathrm{C}$.假设室内空气满足上述方程,比较室内空气能量密度在上述两个温度下的大小.

3. 有一顺磁系统由 $N$ 个磁偶极子组成,每个磁偶极子的磁矩大小为 $\mu$,系统处于均匀外场 $\bf H$ 中并处在温度为 $T$ 时的平衡态,利用经典统计求出系统的感应磁矩,并讨论高温和低温两种极限情况.
\subsection{电磁学}
1. 半径为 $a$,$b$ ($a<b$)的同心导体球之间充有相对介电常数 $\varepsilon_r=1/(1+kr)$ 的非均匀电介质,其中 $k$ 为常数,$r$ 为径向坐标,内表面有电荷 $Q$,外表面接地,计算\\
(1) $ a<r<b $ 的 $D$\\
(2) 电容 $C$\\
(3) $ a<r<b $ 的 $\rho_{\text{体}}$\\
(4) $ r=a $ 和 $b$ 的$\rho_{\text{面}}$

2. 长度为 $L$ 的铜棒以距端点 $x$ 处为支点,并以角速度 $\omega$ 围绕支点垂直于铜棒的轴旋转.匀强磁场 $\bf B$ 平行于转轴,求棒两端的电势差.
\subsection{光学}
1. 借助宽度为 $D$ 的单缝衍射公式.估算用一台物镜口径为 $0.5 \opn m$ 的望远镜可分辨的月球上最小火山口的直径.假设光的波长为 $5000\mathring{\opn A}$,地球到月球的距离为 $3.8\times 10^8 \opn m$.

2. 四个理想的偏振片堆叠起来,每片的通光轴相对于前一片都顺时针旋转 $30^\circ$.非偏振光入射,穿过偏振片堆后,光强为?