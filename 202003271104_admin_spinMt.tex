% 自旋矩阵

\pentry{自旋矩阵\upref{spinMt}, 简谐振子升降算符归一化\upref{QSHOnr}}

我们已知 $1/2$ 自旋的矩阵可以用 $\hbar/2$ 乘以泡利矩阵得到. 以下我们试图计算任意 $n/2$ 自旋粒子的三个自旋矩阵 $\mat S_x$, $\mat S_y$ 和 $\mat S_z$.

以下使用 $\mat S_z$ 的本征态 $\ket{l, m}$ 作为基底.

基本思路是先求出升降算符 $S_\pm = S_x \pm \I S_y$ 的矩阵. 再把它们分别相加和相减得到 $\mat S_x$ 和 $\mat S_y$.

已知 
\begin{equation}
S_- \ket{l,m} = \hbar \sqrt{l(l + 1) - m(m - 1)} \ket{l,m}
\end{equation}
同理可证
\begin{equation}
S_+ \ket{l,m} = \hbar\sqrt{l(l + 1) - m(m + 1)} \ket{l,m}
\end{equation} 