% 欧多克索斯(综述)
% license CCBYSA3
% type Wiki

本文根据 CC-BY-SA 协议转载翻译自维基百科 \href{https://en.wikipedia.org/wiki/Eudoxus_of_Cnidus}{相关文章}。

欧多克斯(Eudoxus of Cnidus,/ˈjuːdəksəs/;古希腊语:Εὔδοξος ὁ Κνίδιος,Eúdoxos ho Knídios;约公元前390年—约公元前340年)是一位古希腊的天文学家、数学家、医生和立法者。\(^\text{[1]}\)他是阿尔喀塔斯和柏拉图的学生。他的原始著作均已佚失,但有些片段保存在希帕克斯的《阿拉托斯与欧多克斯〈天象〉注释》中。\(^\text{[2]}\)比提尼亚的西奥多修斯所著的《球面论》可能是基于欧多克斯的某部著作。
\subsection{生平}
欧多克斯是埃斯喀涅斯之子,出生并去世于克尼多斯(Cnidus,又作 Knidos),这是位于安纳托利亚西南海岸的一座城市。\(^\text{[3]}\)欧多克斯的出生与去世年份并不确切,但第欧根尼·拉尔修记载了一些传记细节,并提到阿波罗多洛斯说他在第103届奥林匹亚会期(公元前368年至前365年)达到了事业巅峰,并声称他在53岁时去世。19世纪的数学史学家据此推断其生卒年为公元前408年至前355年,\(^\text{[4]}\)但20世纪的学者认为这些日期相互矛盾,更倾向于其出生于约公元前390年。\(^\text{[5]}\)“欧多克斯”这个名字意为“受尊敬的”或“声誉良好的”(希腊语:εὔδοξος,源自 eu “好”与 doxa “意见、信念、名声”,相当于拉丁语中的 Benedictus)。

据第欧根尼·拉尔修援引卡利马科斯的《书目》所述,欧多克斯曾在大希腊地区塔兰图姆的阿尔喀塔斯门下学习数学,并向西西里人腓利斯顿学习医学。23岁时,他与医生忒俄墨冬一同前往雅典学习哲学,忒俄墨冬是他的资助者,也可能是他的情人。\(^\text{[6]}\)在雅典他只停留了两个月,住在比雷埃夫斯,每天往返七英里(约11公里)步行去听智者们的讲座。之后他返回克尼多斯。朋友们后来集资送他前往埃及赫利奥波利斯,在那里他学习了16个月的天文学与数学。自埃及北上,他又前往马尔马拉海南岸的基齐库斯,即古称的普罗庞提斯地区。随后他南下前往摩索拉斯之宫廷。在旅途中,他也收了一批自己的学生。

大约在公元前368年,欧多克斯带着他的学生返回雅典。有些资料称,在约公元前367年柏拉图前往叙拉古期间,他曾担任柏拉图学园的学监,并曾教授过亚里士多德。他最终回到故乡克尼多斯,在城邦议会任职。在克尼多斯期间,他建造了一个天文观测所,并继续撰写著作,讲授神学、天文学和气象学。他育有一子,名为阿里斯塔戈拉斯,另有三位女儿,分别名为阿克提斯、菲尔提斯和德尔菲斯。

在数学天文学方面,欧多克斯因提出同心球体理论而闻名,这一理论是对行星运动的早期解释之一。诗人阿拉托斯也称赞他曾制造过一个天球仪。\(^\text{[7]}\)

他在比例理论上的工作展现出对无理数和线性连续统的洞察:该理论能严谨地处理连续量,而不仅仅局限于整数或有理数。当16世纪塔塔利亚等人重新发现这一理论时,它成为自然科学中定量研究的基础,并启发了理查德·戴德金对实数理论的研究。\(^\text{[8]}\)

火星和月球上都有以他命名的陨石坑。此外,一条代数曲线——欧多克斯双曲线也以他命名。
\subsection{数学}
欧多克斯被一些人视为古典希腊最伟大的数学家,整个古代时期仅次于阿基米德。\(^\text{[9]}\)他很可能是欧几里得《几何原本》第五卷的大部分内容的主要来源。\(^\text{[10]}\)欧多克斯严谨地发展了安提丰的“穷竭法”——这是积分学的前身,在后一个世纪中也被阿基米德巧妙地加以运用。在应用这一方法时,欧多克斯证明了一些重要的数学命题,例如:圆的面积之比等于其半径的平方之比,球的体积之比等于其半径的立方之比,锥体的体积等于与其底面积和高相同的柱体体积的三分之一,而金字塔的体积也是相应棱柱体体积的三分之一。\(^\text{[11]}\)

欧多克斯引入了“未量化的数学量”这一概念,用以描述并操作诸如线段、角度、面积和体积等连续的几何实体,从而避免了对无理数的直接使用。在此过程中,他逆转了毕达哥拉斯学派对“数”与“算术”的强调,转而以几何概念作为严谨数学的基础。毕达哥拉斯学派的一些成员,例如欧多克斯的老师阿尔喀塔斯,曾认为只有算术才能为数学证明提供基础。出于对不可通约量(即无法用整数比例表示的量)进行理解与运算的需要,欧多克斯建立了可能是历史上第一个基于显式公理体系的数学演绎结构。欧多克斯的这一思想转向,引发了数学领域长达两千年的分裂。结合当时希腊知识界普遍对实际应用问题缺乏兴趣的态度,也导致了对算术和代数技术发展的显著退却。\(^\text{[11]}\)

毕达哥拉斯学派发现,正方形的对角线与其边长之间不存在共同的度量单位;这就是著名的发现——√2 无法表示为两个整数之比。这个发现预示着在整数和有理分数之外,还存在不可通约的量,但同时也动摇了几何中“度量”和“计算”这一整体理念的根基。例如,欧几里得在《几何原本》第一卷第47命题中(即毕达哥拉斯定理),采用了面积相加的方法给出了一个复杂的证明,而直到第六卷第31命题,他才使用相似三角形的方式给出一个更简明的证明,这后一种方法依赖于线段比值的概念。

古希腊数学家并不像我们今天那样使用数值和代数方程来进行计算;他们是通过“比例关系”来表达几何量之间的联系。在他们的观念中,两个量的“比”并不是一个数值,而是这两个量之间的一种原始关系。

欧多克斯被认为确立了“两个比相等”的定义,这正是欧几里得《几何原本》第五卷的核心内容。

在欧几里得《几何原本》第五卷的定义5中写道:

“当第一与第二之比与第三与第四之比满足如下条件时,这四个量被称为成同一比例关系:若任意取第一与第三的任意倍数,第二与第四的任意倍数,则在相同的对应顺序下,前者倍数大于、等于或小于后者倍数时,所取的另一组倍数也分别同样大于、等于或小于。”

用现代符号表达,可以更清晰地阐明这一定义。设有四个量 $a, b, c, d$,分别考虑前两个量之比 $a/b$ 与后两个量之比 $c/d$。要说这两个比值相等,即:$a/b = c/d$这可以通过如下条件来定义:

对于任意两个正整数 $m$ 和 $n$,分别构造前组的等倍数 $m \cdot a$ 和 $n \cdot b$,以及后组的等倍数 $m \cdot c$ 和 $n \cdot d$。若满足:若 $m \cdot a > n \cdot b$,则必有 $m \cdot c > n \cdot d$;若 $m \cdot a = n \cdot b$,则必有 $m \cdot c = n \cdot d$;若 $m \cdot a < n \cdot b$,则必有 $m \cdot c < n \cdot d$。则称 $a/b = c/d$。这个定义是欧多克斯为了解决无理量与比值问题而建立的严谨比例理论基础,允许在不使用具体数值的前提下,对几何量之间的比例进行逻辑推理。

这意味着,当且仅当那些大于$a/b$ 的比值 $n/m$ 与那些大于 $c/d$ 的比值完全一致,并且对于“等于”和“小于”的情况也同样一致时,才可以说:$a/b = c/d$

这种理解方式可以与戴德金分割相比较:戴德金通过某个实数两侧的有理数集合(大于它、等于它、小于它)来定义一个实数。

欧多克斯的定义依赖于对以下类似量的比较:$m \cdot a$ 与 $n \cdot b$,$m \cdot c$ 与 $n \cdot d$而不依赖于这些量是否可以用某个共同的单位来度量。

这一定义的复杂性,反映出其中深刻的概念性与方法论上的革新。欧多克斯的比例定义使用了“对于所有……”这样的全称量词,从而在逻辑上掌握了“无限”和“无穷小”的概念,这一点与现代极限和连续性的ε-δ定义极为相似。

而《几何原本》第五卷的定义4,即阿基米德性质,则是由阿基米德归功于欧多克斯的发明。\(^\text{[12]}\)
\subsection{天文学}
在古希腊,天文学是数学的一个分支;天文学家的目标是建立几何模型,以模拟天体运动的表观现象。因此,将欧多克斯的天文学工作划分为一个独立的领域,其实是一种现代的分类便利。

一些欧多克斯的天文学著作名称虽已流传下来,但原文大多已佚失,这些著作包括:
\begin{itemize}
\item 《太阳的消失》:可能是关于日食的讨论;
\item 《八年周期》:关于一个由月亮、太阳与金星构成的历法八年周期;
\item 《天象》与《镜象》:关于球面天文学,可能基于他在埃及和克尼多斯的观测;
\item 《论速度》:关于行星运动的著作。
\end{itemize}
我们对《天象》的内容相对了解较多,因为欧多克斯的这部散文体著作成为后人阿拉托斯同名诗作的基础。天文学家希帕克斯在其对阿拉托斯诗的注释中曾引用过欧多克斯的原文。
\subsubsection{欧多克斯的行星模型}
我们可以通过亚里士多德的《形而上学》第十二卷第8章,以及6世纪的奇里乞亚哲学家辛普利修斯对亚里士多德另一部著作《论天》的注释,对欧多克斯《论速度》一书的内容有一个大致了解。根据辛普利修斯所记载的一个故事,柏拉图曾向希腊天文学家提出一个问题:“通过假设哪些匀速且有序的运动,可以解释行星的表观运动?”[13] 柏拉图设想,那些看似混乱游移的行星运动,其实可以通过围绕球形地球的多个匀速圆周运动的组合来解释——这一设想在公元前4世纪似乎还是一个新颖的思想。

在多数现代对欧多克斯模型的重建中,月球被分配了三个球体:
\begin{itemize}
\item 最外层的球体以24小时为周期向西转动,用以解释月亮的东升西落;
\item 第二层的球体以一个月为周期向东转动,用以解释月亮在黄道带中的月度移动;
\item 第三层的球体同样以一个月为周期完成旋转,但其轴线略有倾斜,用以解释月亮在纬度方向上的偏移(即偏离黄道的运动)以及交点的移动。
\end{itemize}
太阳也被分配了三个球体:

其中第二层球体的运动周期是一年而非一个月;第三层球体的存在说明欧多克斯误以为太阳也存在纬度方向上的运动(即认为太阳也会偏离黄道),这一点被认为是他的一处错误认知。

