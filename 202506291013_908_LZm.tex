% 莉泽·迈特纳(综述)
% license CCBYSA3
% type Wiki

本文根据 CC-BY-SA 协议转载翻译自维基百科\href{https://en.wikipedia.org/wiki/Lise_Meitner}{相关文章}。

\begin{figure}[ht]
\centering
\includegraphics[width=6cm]{./figures/03b50d02f58c4e7f.png}
\caption{} \label{fig_LZm_1}
\end{figure}
伊莉莎·“莉泽”·迈特纳(Elise "Lise" Meitner,/ˈmaɪtnər/,德语:[ˈliːzə ˈmaɪtnɐ] ⓘ,1878年11月7日-1968年10月27日)是一位奥地利-瑞典核物理学家,在核裂变的发现中发挥了关键作用。

1906年完成博士研究后,迈特纳成为维也纳大学第二位获得物理学博士学位的女性。她的科研生涯大部分时间在柏林度过,在那里她曾任教于凯撒·威廉化学研究所,担任物理学教授及系主任。她是德国第一位晋升为物理学正教授的女性。由于纳粹德国反犹太的纽伦堡法案,她在1935年失去了职位,而1938年的奥地利并入德意志第三帝国导致她失去奥地利国籍。1938年7月13日至14日,在德克·科斯特的帮助下,她逃往荷兰。之后她在斯德哥尔摩生活多年,并于1949年成为瑞典公民,但在1950年代迁往英国,与家人团聚。

1938年年中,凯撒·威廉化学研究所的化学家奥托·哈恩和弗里茨·施特拉斯曼证明,通过中子轰击铀可以形成钡的同位素。哈恩将他们的发现告知了迈特纳,1938年12月底,迈特纳与她的侄子、同为物理学家的奥托·罗伯特·弗里施一起,通过对哈恩和施特拉斯曼实验数据的正确解释,阐明了这一过程的物理机制。1939年1月13日,弗里施重复了哈恩和施特拉斯曼观察到的过程。在1939年2月的《自然》杂志上,迈特纳和弗里施发表报告,将这一过程命名为“裂变”。核裂变的发现推动了二战期间核反应堆和原子弹的发展。

迈特纳并未与长期合作伙伴奥托·哈恩共享1944年因核裂变而授予的诺贝尔化学奖,许多科学家和记者称她未获奖“不公”。根据诺贝尔奖档案,1924年至1948年间,她曾19次被提名诺贝尔化学奖,1937年至1967年间曾30次被提名诺贝尔物理学奖。尽管未获得诺贝尔奖,迈特纳在1962年仍受邀出席林岛诺贝尔奖获得者大会。她还获得了许多其他荣誉,包括1997年以她命名的元素109号“镆”。爱因斯坦曾称赞迈特纳为“德国的居里夫人”。
