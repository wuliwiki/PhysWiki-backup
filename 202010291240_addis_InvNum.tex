% 逆序数
% 逆序数|排列

\begin{issues}
\issueDraft
\end{issues}

\pentry{排列\upref{permut}}

\footnote{参考 Wikipedia \href{https://en.wikipedia.org/wiki/Inversion_(discrete_mathematics)}{相关页面}.}我们把集合 $\qty{1,\dots,N}$ 的第 $n$ 种排列记为 $p_n$, 该排列的元素按照顺序分别记为 $p_{n,1}, \dots, p_{n,N}$. 对于任意 $i < j$, 如果满足 $p_{n,i} > p_{n,j}$ 我们就把 $i, j$ 或者 $p_{n,i}, p_{n,j}$ 称为排列 $p_n$ 的一个\textbf{逆序对(inversion)}. 一个排列中所有逆序对的个数就叫\textbf{逆序数(inversion number)}.

在行列式和 Levi-Civita 符号中, 我们只对逆序数的奇偶性感兴趣. 可以证明交换排列中任意两个数, 逆序数奇偶性改变. 

\addTODO{证明}
