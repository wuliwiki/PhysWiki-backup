% 科里奥利力
% keys 惯性系|惯性力|非惯性系|旋转参考系|离心力|科里奥利力

\pentry{离心力\upref{Centri},平面旋转矩阵\upref{Rot2D}}%未完成

\textbf{科里奥利力(Coriolis Force)}是旋转的参考系中由质点运动产生的惯性力.
\begin{equation}
\bvec F_c = 2m \bvec v_{S'} \cross \bvec \omega
\end{equation}
其中 $m$ 是质点的质量, $\bvec v_{S'}$ 是质点相对于旋转参考系 $S'$ 的瞬时速度, $\bvec\omega$ 是旋转系相对于某惯性系 $S$ 转动的角速度矢量.%未完成: 考虑使用脚注或链接
式中的乘法是叉乘\upref{Cross}.
在匀速转动参考系(属于非惯性系)中,若质点保持相对静止,则惯性力只有离心力.然而当质点与转动参考系有相对速度时,惯性力中还会增加一项与速度垂直的力,这就是科里奥利力.地理中的地转偏向力就是科里奥利力,可用上式计算(见“地球表面的科里奥利力\upref{ErthCf}”).

\subsection{推导(矢量法)}
\pentry{连续叉乘的化简\upref{TriCro}, 速度的参考系变换\upref{Vtrans}}

我们先令质点的位矢关于时间变化的函数为 $\bvec r(t)$, 某时刻相对于 $S$ 和 $S'$ 系的速度分别为 $\bvec v_{S}$ 和 $\bvec v_{S'}$, 根据\autoref{Vtrans_ex1}\upref{Vtrans} 中的结论, 任意时刻都有
\begin{equation}\label{Corio_eq5}
\bvec v_{S} = \bvec v_{S'} + \bvec\omega\cross\bvec r
\end{equation}
两边在 $S$ 系中对时间求导得(我们先假设 $\bvec \omega$ 是常矢量\footnote{如果角速度 $\bvec \omega$ 随时间变化, 那么惯性力中会多出一项, 但这并不属于科里奥利力, 所以科里奥利力的表达式并不要求角速度不变.})
\begin{equation}\label{Corio_eq6}
\bvec a_{S} = \qty(\dv{\bvec v_{S'}}{t})_{S} + \bvec\omega\cross\bvec v_{S}
\end{equation}
注意 $S'$ 系中的加速度 $\bvec a_{S'}$ 并不是上式右边第一项, 而是 $(\dv*{\bvec v_{S'}}{t})_{S'}$. 令\autoref{Corio_eq4} 中的 $\bvec A = \bvec v_{S'}$, 得
\begin{equation}\label{Corio_eq7}
\qty(\dv{\bvec v_{S'}}{t})_{S} = \bvec a_{S'} + \bvec\omega\cross\bvec v_{S'}
\end{equation}
将\autoref{Corio_eq5} 和\autoref{Corio_eq7} 代入\autoref{Corio_eq6}, 得
\begin{equation}
\bvec a_{S} = \bvec a_{S'} + 2\bvec\omega\cross\bvec v_{S'} + \bvec\omega\cross(\bvec\omega\cross\bvec r)
\end{equation}
所以旋转参考系中的总惯性力(\autoref{Iner_eq1}\upref{Iner})为
\begin{equation}
\bvec f = m(\bvec a_{S'} - \bvec a_{S}) = -2m\bvec\omega\cross\bvec v_{S'} - m\bvec\omega\cross(\bvec\omega\cross\bvec r)
\end{equation}
其中第二项为离心力(\autoref{Centri_eq5}\upref{Centri}), 而第一项被称为科里奥利力.

% 未完成:需要引用矩阵相乘的求导法则
\subsection{推导(旋转矩阵法)}
设空间中存在一个惯性系 $S$ 和一个非惯性系 $S'$ 相对于惯性系 $S$ 绕 $z$ 轴以角速度 $\omega$ 逆时针匀速旋转(右手定则\upref{RHRul}). 由于 $z$ 轴和 $c$ 轴始终重合( $z=c$), 只需要考虑 $x,y$ 坐标和 $a,b$ 坐标之间的关系即可.

令平面旋转矩阵为% 未完成:链接
\begin{equation}
\mat R(\theta) \equiv \begin{pmatrix}
\cos \theta & - \sin \theta \\
\sin \theta & \cos \theta
\end{pmatrix}
\end{equation}
其意义是把坐标逆时针旋转角 $\theta$. 两坐标系之间的坐标变换为
\begin{equation}
\pmat{x\\y}_{S} = \mat R(\omega t) \pmat{a\\b}_{S'}
\qquad
\pmat{a\\b}_{S'} = \mat R(-\omega t) \pmat{x\\y}_{S}
\end{equation}
为了得到质点在惯性系中的加速度,对上面左式的 $(x,y)\Tr$ 求二阶时间导数得\footnote{某个量上方加一点表示对时间的一阶导数,两点表示对时间的二阶导数.} $S$ 系中的加速度(以 $\uvec x, \uvec y$ 为基底)
\begin{equation}\label{Corio_eq1}
\bvec a_{S} = \pmat{\ddot x \\ \ddot y}_{S} = 
\ddot{\mat R}(\omega t) \pmat{a\\b} + 2\dot{\mat R} (\omega t) \pmat{\dot a \\ \dot b} + \mat R(\omega t)\pmat{\ddot a \\ \ddot b}
\end{equation}
其中\footnote{\autoref{Corio_eq2} 和\autoref{Corio_eq3} 相当于用矩阵推导了匀速圆周运动的速度和加速度公式\upref{CMVD}\upref{CMAD}.}
\begin{equation}\label{Corio_eq2}
\dot{\mat R}(\omega t) = \omega \begin{pmatrix}
\cos(\omega t + \pi /2) &  - \sin(\omega t + \pi /2)\\
\sin(\omega t + \pi /2) & \cos(\omega t + \pi /2)
\end{pmatrix}
= \omega \mat R(\omega t + \pi /2)
\end{equation}
\begin{equation}\label{Corio_eq3}
\ddot{\mat R} (\omega t)  =  - \omega ^2 \mat R (\omega t)
\end{equation}
 代入\autoref{Corio_eq1} 得
\begin{equation}
\bvec a_{S} =
- \omega ^2 \mat R(\omega t)\pmat{a\\b} + 2\omega \mat R(\omega t + \pi /2)\pmat{\dot a \\ \dot b} + \mat R(\omega t)\pmat{\ddot a \\ \ddot b}
\end{equation}
上式中的每一项都是以 $\uvec x, \uvec y, \uvec z$ 为基底的坐标.所有坐标乘以 $\mat R(-\omega t)$, 得到以 $\uvec a, \uvec b, \uvec c$ 为基底的坐标
\begin{equation}
\bvec a_{S} =
- \omega^2 \pmat{a\\b}_{S'} + 2\omega \mat R(\pi /2)\pmat{\dot a\\ \dot b}_{S'} + \pmat{\ddot a\\ \ddot b}_{S'}
\end{equation}
所以旋转参考系中的总惯性力(\autoref{Iner_eq1}\upref{Iner})为(以 $\uvec a, \uvec b, \uvec c$ 为基底)
\begin{equation}\label{Corio_eq10}
\bvec f = m(\bvec a_{S'} - \bvec a_{S})
=  m \omega ^2 \pmat{a\\b}_{S'} - 2m\omega \mat R(\pi /2)\pmat{\dot a\\ \dot b}_{S'}
\end{equation}
其中第一项是已知的离心力\autoref{Centri_eq3}\upref{Centri}, 我们将第二项定义为科里奥利力 $\bvec F_c$. 科里奥利力可以用叉乘记为
\begin{equation}
\bvec F_c = 2m \bvec v_{S'} \cross \bvec \omega
\end{equation}
其中 $\bvec\omega$ 是 $S'$ 系旋转的角速度矢量, $\bvec v_{S'}$ 是质点相对于 $S'$ 系的速度.最后, 我们可以写出\autoref{Corio_eq10} 的矢量形式
\begin{equation}
\bvec f = m \omega ^2 \bvec r + 2m \bvec v_{S'} \cross \bvec \omega 
\end{equation}
 




