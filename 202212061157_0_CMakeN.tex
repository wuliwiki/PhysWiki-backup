% CMake 笔记

\begin{issues}
\issueDraft
\end{issues}

\pentry{Makefile 笔记\upref{Make}}

参考\href{https://cmake.org/cmake/help/latest/guide/tutorial/index.html}{官方教程}.

\subsection{常识}
\begin{itemize}
\item 在 windows 下可直接使用 GUI, linux 命令行中使用 \verb`ccmake` 可以有 TUI. 否则就用 \verb`cmake`
\item 在 \verb|CMakeLists.txt| 的路径下, \verb`cmake .` 生成 Makefile, 然后 \verb|make -j12| 多线程编译. \verb|make VERBOSE=1 -j12| 输出编译命令. cmake 会在当前路径生成 \verb|Makefile| 和一些临时文件和文件夹, 清理麻烦, 所以还是建议输出到子文件夹.
\item 如果不想改变源码路径, 就用在别的文件夹 \verb|cmake 源码路径| 即可.
\item \verb|cmake -L 源码路径| 列出编译选项, \verb|-LH| 列出选项以及帮助说明. \verb|-LAH| 列出所有选项(会多出很多 \verb|CMAKE_| 开头的, 不是作者提供的选项), 用 \verb|cmake -D 选项1=值1 -D 选项2=值2 源码路径| 来设置选项(\verb|-D| 后面可以没有空格.
\item 要指定路径, 用 \verb|cmake -S CMakeList的路径 -B 生成Makefile的路径|, 然后再到 Makefile 的路径 \verb|make -j12| 即可. 如果要清理 cmake 生成的文件以及所有编译出来的文件, 直接把第二个路径清空即可.
\item 当然, 也不是什么平台都可以用 \verb|make|, 所以也可以用 \verb|cmake --build 生成Makefile的路径 [--target 可执行文件] -- -j 12| 其中 \verb|--| 后面的选项会传给 \verb|make| 或者别的具体用于编译的程序.
\item Cmake 的基本语法是 \verb`command(arg1 arg2 ...)` 其中 \verb|command| 不区分大小写, \verb`arg` 用空格隔开, 如果 \verb|arg| 本身有空格, 用双引号即可
\item \verb`message(STATUS "...")` 可以输出到 stdout. 如果 \verb`message(STATUS ${var})` 中的变量是 list, 那么中间不会有空格, 应该用 \verb`message(STATUS "${var}")`, 输出中每个元素会用分号隔开.
\item 两个 \verb`arg` 之间可以换行
\item 变量可以是 string 或者 list of strings, 使用变量格式为 \verb`${VAR}`
\item \verb`set` 可以对变量赋值, 例如 \verb`set(var, 1)`
\item \verb`set` 也可以把许多变量变成一个 list, 如 \verb`set(Foo a b c)` 将 \verb`Foo` 变成包含 \verb`a b c` 的 list, 当 \verb`${Foo}` 被使用时相当于 \verb`a b c`, 例如 \verb`command(${Foo})` 相当于 \verb`command(a b c)`. 另外 \verb`command("${Foo}")` 相当于 \verb`command("a b c")` (双引号中的是一个变量)
\item \verb|set(MY_LIST "one;two")| 相当于 \verb|set(MY_LIST "one" "two")|. 前者是内部的表示方法.
\item linux 环境变量和 windows 注册表变量可以直接作为变量使用, 格式如 \verb`$EVN{VAR}`. 例子: \verb|message(STATUS $ENV{PWD})|. 注意 cmake 看到的环境变量未必和 bash 中的相同, 可以用 \verb`cmake -E environment` 检查.
\item \verb|separate_arguments(VAR)| 可以把一个含有若干空格的字符串拆分成 list (即把空格替换为分号).
\end{itemize}

\subsection{常用命令}
\begin{itemize}
\item \verb`cmake_minimum_required(VERSION x.x)`
\item \verb`project(project_name)` 指定项目名称
\item \verb|set(CMAKE_CXX_COMPILER "/usr/bin/g++")| 设置 compiler
\item \verb|set(CMAKE_CXX_FLAGS "-std=c++11")| 设置 flag, 包括 c++ 标准.
\item 用标准的 \verb|set(CMAKE_CXX_STANDARD 11)| 会对 g++ 自动使用 \verb|-std=gnu11| 而不是 \verb|-std=c++11|, 一些情况会出错.
\item \verb|cmake -DCMAKE_BUILD_TYPE=Debug| 用 debug 模式编译
\item 添加 debug 模式的 flag 如 \verb|set(CMAKE_CXX_FLAGS_DEBUG "${CMAKE_CXX_FLAGS_DEBUG} -fsanitize=address")|
\item \verb`set(var str1 [str2] ...)` 对变量赋值(若有多个值就生成 list)
\item \verb`list(APPEND var str1 [str2] ...)` 在变量(list)后面添加元素
\item \verb|list(REMOVE_ITEM var str1 ...)| 从 \verb|var| 的 list 中移除(如果不存在也不报错)
\item \verb|set(var "${var} str1 str2")| 等效于上一条
\item 使用一个不存在的变量相当于空字符串.
\item \verb`file(GLOB var "folder/*.cpp" "fname2.cpp")` 可以列出所有 \verb|"folder/*.txt"| 文件, 以及 \verb|fname2.cpp| (如果存在) 赋值给 \verb|var|. 注意文件名包括绝对路径.
\item \verb|add_definitions(-D FOO -D BAR)| 给编译器添加宏定义
\item \verb|add_compile_options(-Wall -fopenmp)| 给编译器添加选项(注意不能用 \verb|-D|)
\item \verb`add_executable(exe_name source_name)` 前者是可执行文件名, 后者是所有需要 link 的 \verb|c/c++| 文件名,不需要头文件. cmake 会自动分析代码得到哪个 cpp 调用哪个头文件, 如果头文件改变了, 只有调用它的 cpp 会重新编译.
\item \verb`configure_file(file_in file_out)` 将文件中 \verb`@var@` 替换为变量 \verb`var` 的字符串.
\item \verb`include_directories(dir1 [dir2] ...)` 添加头文件的搜索路径
\item \verb`add_library(exe_name source_name)` 单独编译一个 library (在 library 路径的 \verb|CMakeLists.txt| 中使用这个命令而不是 \verb`add_executable` 命令)
\item \verb|link_directories(路径)| 相当于编译器的 \verb`-L` 选项, 添加静态或动态 library 的搜索路径.
\item \verb|set(CMAKE_INSTALL_RPATH 路径)| 相当于设置 \verb|rpath| (到底是 \verb|RUNPATH| 还是 \verb|RPATH|?)
\item \verb`add_subdirectory(dir1 [dir2] ...)` 执行子路径中的 \verb|CMakeLists.txt|
\item \verb`target_link_libraries(exe_name lib1 [lib2] ...)` link 阶段链接 library, 相当于 \verb|-l lib1 -l lib2|
\item \verb`option(opt_name description default)` 定义一个 option 开关(会在 gui 中显示开关)以及默认值. \verb`opt_name` 可以在 \verb|CMakeLists.txt| 中的 \verb`if` 语句中使用例如 \verb`if(opt_name) ... endif(opt_name)`. \verb`default` 可以是 \verb`ON` 或 \verb`OFF`
\item 在 configure 的模板文件中, 当 \verb`opt_name` 为 \verb`ON` 时 \verb`#cmakedefine opt_name` 会被替换为 \verb`#define opt_name`
\item 【cmake 3.16 的新功能】 \verb|target_precompile_headers(可执行文件 PRIVATE foo.h bar.h)| 可以编译头文件, 注意貌似不会自动包括 .h 包括的文件. 如果头文件(或者依赖的文件)改了, 那么将会自动重新编译头文件.
\end{itemize}

\subsection{内置变量}
\begin{itemize}
\item \verb`PROJECT_SOURCE_DIR` 是源文件的根路径, 就是传给 \verb`ccmake` 的路径
\item \verb`PROJECT_BINARY_DIR` 是 Cmake 的输出路径, 临时文件, Makefile 等都在这个路径. 这就是运行 \verb`ccmake` 的路径
\end{itemize}

\subsection{判断}
参考\href{https://cmake.org/cmake/help/latest/command/if.html}{这里}.
\begin{lstlisting}[language=bash]
if(<condition> AND (<condition> OR <condition>))
  <commands>
elseif(("bar" IN_LIST var) OR (file1 IS_NEWER_THAN file2))
  <commands>
elseif (var)
  当 var 不是 false constant
elseif(DEFINED <name>|CACHE{<name>}|ENV{<name>})
  若定义了变量
else()
  <commands>
endif()
\end{lstlisting}

\subsection{CMake 与 Visual Studio}
\begin{itemize}
\item 链接 \verb`*.lib`, 只需要在 \verb`add_executable()` 命令前面插入(必须在之前) \verb`abc.lib` 的路径 \verb`link_directories(path/to/lib)` 然后再 \verb`add_executable()` 之后插入 \verb`target_link_libraries(exe_name abs)` 即可
\item \verb`if(MSVC)...end(MSVC)` 可以专门给 Visual Studio 执行一些命令
\item 生成的 sln 文件中, 除了有 target 的工程, 还会有另外两个工程: \verb`ZERO_CHECK` 会重新运行/更新 \verb`CMakeLists.txt`, \verb`ALL_BUILD` 会编译所有工程. 如果看着不爽的话也可以把这两个 project 删掉. 如果直接 run without debug 的话, 会提示 \verb`ALL_BUILD` 不能 run, 所以要 run 只能右键某个 project 然后 run (所以还是把两个多余的 project 删掉好些).
\end{itemize}

\subsection{ccmake 使用}
\begin{itemize}
\item \verb`sudo apt install cmake-curses-gui` 安装
\item 在需要生成 binary 的路径打开 \verb`ccmake`, 这个路径可以是源文件路径
\item 如果 binary 路径不是源文件路径, 那么指定路径即可, 例如 \verb`ccmake ../`
\item \verb`h` 帮助, \verb`q` 退出, \verb`c` configure, \verb`G` 生成 Makefile
\end{itemize}
