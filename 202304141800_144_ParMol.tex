% 偏摩尔量

\begin{issues}
\issueDraft
\end{issues}

%正在重写中...

\pentry{相简介(热力学)\upref{PHS}}
\footnote{本文参考了朱文涛的《简明物理化学》与Schroeder的《热物理学导论》}

\subsection{多元多相系统中的状态量}
\pentry{态函数\upref{statef}}
\begin{figure}[ht]
\centering
\includegraphics[width=8cm]{./figures/ae7e6997f377b0b8.pdf}
\caption{多元多相系统} \label{fig_ParMol_1}
\end{figure}

在多元多相系统中,我们需要扩展状态量与“状态公理”的含义。

在多元多相系统中,由于不同相的热力学性质不同,因此往往需要分别写出每一个相的状态量。\textsl{此次的上标含义是相,而不是指数!}
$$U^\alpha, U^\beta,...$$

同时,由于物质可在各相间流动、每一相中的物质含量可以变化,因此相状态量也与各物质的物质的量有关。
$$U^\alpha = U^\alpha (p^\alpha, T^\alpha, n_1^\alpha,n_2^\alpha,...)$$

\subsection{单组份系统:摩尔量}
我们先来思考一种最简单的单组份系统,即系统中只包括一种物质。以下我们先讨论一个相中的摩尔量,因此省略相的角标。

我们先以内能为例:
$$U = U (p, T, n)$$
由于内能是广延量,因此内能应当与相中物质的量成正比。比如,如果物质的量增多到原来的$4$倍,那么内能也将变为原来的$4$倍。
$$U(p, T, 4 n) = 4 U (p, T, n)$$
这是广延量的一个非常好的性质:根据齐次方程欧拉定理,我们可以将其化为以下形式:
$$U(p, T, n) = n \cdot u (p, T)$$
可见, $u (p, T)$ 不再与系统的规模相关,而成为了某种强度量。我们将其定义为\textbf{摩尔内能},意味着每摩尔物质所具有的内能。

同理,对于其他广延量,我们可以定义相应的摩尔量:
$$
\begin{aligned}
S(p, T, n) &= n \cdot s (p, T)\\
V(p, T, n) &= n \cdot v (p, T)\\
G(p, T, n) &= n \cdot \mu (p, T)\\
&...\\
\end{aligned}
$$
Gibbs自由能的摩尔量写法上比较特别,一般写为$\mu$,也称为化学势。以后你会知道为什么人们\textsl{对Gibbs的摩尔量搞特殊}。对于纯物质的摩尔量,可加$*$上标与混合物质中的区分(见下),例如$v^*, \mu^*$。

\subsubsection{摩尔量的热力学基本关系}
\pentry{热力学关系式\upref{MWRel}}

我们先写出化学势(摩尔Gibbs自由能)的定义
$$G = n \cdot \mu$$
对其求微分
$$\dd G = n \cdot \dd \mu + \cdot \mu \dd n$$
假设系统中的物质数不发生变动,即$\dd n = 0$,那么
$$\dd G = n \cdot \dd \mu$$
同时,根据热力学基本关系式\upref{MWRel} :
$$\dd G = -S \dd T + V \dd P$$
因此
$$-S \dd T + V \dd P = n \cdot \dd \mu$$
或者
$$
\dd \mu = -s \dd T + v \dd P
$$
这就是化学势的微分等式。可见,摩尔量的微分与热力学关系式 中的相同,只是把广延量换为了相应的摩尔量。

将其带回上式,得多元系统中的热力学基本关系式:
$$\dd G = -S \dd T + V \dd P + \mu \dd n$$

可见,只是补充了额外的一项化学势与物质的量。可以证明其余的热力学关系式也可以这么修正,不过证明过程比较繁琐:
$$
\begin{aligned}
&\dd U = T \dd S - P \dd V + \mu \dd N\\
&\dd H = T \dd S + V \dd P + \mu \dd N\\
&\dd A = -S \dd T - P \dd V + \mu \dd N\\
\end{aligned}
$$

\subsubsection{不同环境下的摩尔量}
\pentry{盖斯定律与设计路径\upref{Hess}}
假如我们知道了 某一状态下物质的摩尔量$\mu(T_0,p_0)$,怎么计算另一状态下的摩尔量 $\mu(T_0,p_1)$?答案还是设计路径\upref{Hess} :
$$\mu(T_0,p_1) = \mu(T_0,p_0) + \int_{p_0}^{p_1} \left( \pdv{\mu}{p} \right)_{T_0} \dd p$$
根据上述的热力学基本关系,$\left( \pdv{\mu}{p} \right)_{T_0} = v$
因此,$$\mu(T_0,p_1) = \mu(T_0,p_0) + \int_{p_0}^{p_1} v \dd p$$

许多热力学实验量都是在特定环境(例如,$T = 298 \Si{K}, p = 101325\Si{Pa}$)下测量的,因此你需要使用这些公式来换算。

\begin{exercise}{}
写出 $\mu(T_1,p_1)$ 的表达式。
\end{exercise}

\subsection{多组分系统:偏摩尔量}
我们还是以内能为例:
$$U = U (p, T, n_1, n_2, ...)$$
如法炮制,根据状态量的广延性质,我们还有
$$ U (p, T, 4n_1, 4n_2, ...) = 4U (p, T, n_1, n_2, ...)$$
注意这里的“广延性质”与单组份的情况不大一样:只有当所有组分的含量都番4倍,总的内能才会上升到原来的4倍。

因此我们同样根据欧拉定理,可以定义
$$ U (p, T, n_1, n_2, ...) = \sum_i n_i u_i(p, T, x_1, x_2, ...) $$
其中,每一个 $u_i$被称为该组元的\textsl{偏摩尔内能},该公式也称偏摩尔量的集合公式。

特别要注意的是,偏摩尔量有一些相当独特的性质:
\begin{itemize}
\item 尽管偏摩尔量与相的规模无关,但是\textbf{与相中各物质的比例}有关。当系统中物质比例改变时,偏摩尔量也可能改变。
\item 偏摩尔量与纯物质的摩尔量一般\textbf{不相同}。
\item 尽管集合公式是一个漂亮的累加,但是偏摩尔量\textbf{不适合}理解为“每摩尔物质具有的热力学量”。例如,在某些溶液中,溶质的溶解反而降低了溶液的总体积。此时,溶质的偏摩尔体积$v_B$的值是负数。显然,物质的体积不能是负数。
\end{itemize}

\begin{example}{}
例如,某一条件$(p,T)$下系统中具有$4 \Si{mol} A$物质与$2 \Si{mol} B$物质,那么系统的内能
$$U = 4 u_{A} +  2 u_B $$
如果系统中具有$8 \Si{mol} A$物质与$4 \Si{mol} B$物质,那么
$$U = 8 u_A +  4 u_B$$
但是,如果系统中具有$8 \Si{mol} A$物质与$2 \Si{mol} B$物质,由于物质的比例不再是2:1,因此偏摩尔量改变。
$$U \ne 8 u_A +  2 u_B$$
\end{example}

同理,我们可以定义其他广延量的相应的偏摩尔量
$$
\begin{aligned}
V &= \sum_i  n_i \cdot v_i \\
G &= \sum_i  n_i \cdot \mu_i \\
&...\\
\end{aligned}
$$

\subsubsection{偏摩尔量的微分定义}
我们还是以内能为例:
$$U = U (p, T, n_1, n_2, ...)$$
对其求全微分
\begin{equation}\label{eq_ParMol_1}
dU = \left(\pdv{U}{p}\right)_{T,n_1,n_2,...}dp + \left(\pdv{U}{T}\right)_{p,n_1,n_2,...}dT+\left(\pdv{U}{n_1}\right)_{p,T,n_2,...}dn_1+\left(\pdv{U}{n_2}\right)_{p,T,n_1,n_3,...}dn_2+...
\end{equation}
\textsl{不难看出},偏摩尔量其实就是 %需要补充证明
\begin{equation}
{u_i} = \left(\pdv{U}{n_i}\right)_{p,T,n_j, j \neq i} 
\end{equation}
这告诉我们偏摩尔量的另一层含义:$u_i \dd n_i$ 是某一状态下保持其余条件不变,再往系统中加入少量物质i,系统内能的变化。

\subsubsection{Gibbs-Duhem 公式}
\begin{equation}
\sum n_B \dd Z_B = 0
\end{equation}

推导:对集合公式两端求导,$dZ=\sum n_B \dd Z_B + \sum {Z_B}  \dd n_B$,即得证。

特别地,对于二元混合物,
\begin{equation}
n_1 \dd {Z_1} = - n_2 \dd {Z_2}
\end{equation}
或
\begin{equation}
\dd {Z_1} = -\frac{n_2}{n_1} \dd {Z_2}
\end{equation}

