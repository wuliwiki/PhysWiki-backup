% 浙江大学 2008 年 考研 量子力学
% license Usr
% type Note

\textbf{声明}:“该内容来源于网络公开资料,不保证真实性,如有侵权请联系管理员”

\subsection{第一题:简答题(30分)}
(1) 从正则对易关系 $[x_i, \hat{p}_j] = i\hbar \delta_{ij}$ 推出角动量算符的对易关系;

(2) 用测不准关系估算氢原子的基态能量;

(3) 什么是量子跃迁?什么是选择定则?线偏振光和圆偏振光照射下的选择定则有什么区别?

(4) 什么是塞曼效应?什么是斯塔克效应?

(5) 什么是受激辐射?什么是光电效应?

\subsection{第二题:(25分)}
设电子以给定的能量 $E = \frac{\hbar^2 k^2}{2m}$ 自左入射,遇到一个方势阱
\[V(x) = \begin{cases} 0 & x < 0, x > a \\\\- V_0 & 0 \leq x \leq a\end{cases}~\]

(a)求反射系数和透射系数:

(b)给出发生共振隧穿的条件;

(c)考虑到电子有自旋(自旋向下或向上),你能否借用上面的结果,设计一个量子调控装置,使反射回来的只有自旋向上的电子而没有自旋向下的电子?
\subsection{第三题:(20分)}
于下列中心势场:

\begin{equation}
    (a) \quad V(r) = a \delta (r) \quad \quad (b) \quad V(r) = b e^{-ar} \quad \quad (c) \quad V(r) = c e^{-a r^2}~
\end{equation}

(从三种势中选做一个即可!),用玻恩近似化计算散射截面 $\sigma (\theta)$。

\subsection{第四题:(25 分)}
当前物理前沿的一个重要领域是自旋霍尔效应,其中有一类为二维电子气型系统。该系统的哈密顿量为

\begin{equation}
    \hat{H} = \frac{1}{2m} \left( \hat{p}_x^2 + \hat{p}_y^2 \right) + \alpha \left( \hat{p}_x \hat{\sigma}_y - \hat{p}_y \hat{\sigma}_x \right)~
\end{equation}

其中 $\alpha$ 是一个系数,$\hat{\sigma}_x$、$\hat{\sigma}_y$ 代表泡利矩阵。试从该系统的薛定谔方程

\begin{equation}
    i \hbar \frac{\partial \psi}{\partial t} = H \psi~
\end{equation}

出发,导出连续性方程,并给出相应的几率密度和几率流密度的表达式。
\subsection{第五题:(25 分)}
许多物理问题可以化成两能级系统,如
\[\hat{H} = \hat{H}_0 + \hat{H}' = \begin{pmatrix} A + a & b \\\\ b & B +a \end{pmatrix},~\]
其中 \(a\), \(b\) 为实数,并且 \(a\) 远小于 \(A - B\),

\begin{enumerate}
    \item (a) 试求能级的精确值;
    \item (b) 再用微扰公式写出能级(到二级近似),并比较两种结果。
\end{enumerate}
\subsection{第六题:(25分)}

当前冷原子物理研究非常活跃。在实验中,粒子常常是被束缚在谐振子势中,因此其哈密顿量为

\[\hat{H_0} = \frac{\hat{p}^2}{2m} + \frac{1}{2}m\omega^2r^2~\]

假如粒子间有相互作用 \(\hat{H}' = J\hat{S_1} \cdot \hat{S_2}\),其中 \(\hat{S_1}\)、\(\hat{S_2}\) 分别代表粒子1和粒子2的自旋,参数 \(J > 0\)。

(a) 如果把两个自旋 \(\frac{1}{2}\) 的全同粒子放在上述势中,试写出基态能量和基态波函数;

(b) 如果把两个自旋1的全同粒子放在上述势中,试写出基态能量和基态波函数(注意:参数在不同范围内,情况会不同)。

