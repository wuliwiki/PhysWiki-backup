% 向量空间上的范数
% license Usr
% type Tutor

% 移动自 NormV,原作者 addis; DTSIo

\textbf{范数(norm)}可以看作几何向量\upref{GVec}的模长在一般向量空间上的拓展。
\begin{definition}{}\label{def_NormV_1}
设 $X$ 是实数或复数域上的向量空间。 $X$ 上的范数是满足如下条件的非负函数 $\|\cdot\|$:
\begin{enumerate}
\item $\norm{x} \geqslant 0$ (正定)
\item $\norm{x} = 0$ 当且仅当 $x = 0~.$
\item $\|\lambda x\| = |\lambda|\|x\|~.$
\item $\|x_1+x_2\| \leqslant \|x_1\|+ \|x_2\|$ (三角不等式)
\end{enumerate}
如果一个向量空间中定义了范数, 我们就把它称为\textbf{赋范空间(normed space)}。
\end{definition}