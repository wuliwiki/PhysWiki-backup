% 费米黄金法则

\pentry{几种含时微扰\upref{TDPEx}}

方形脉冲(\autoref{TDPEx_eq2}~\upref{TDPEx})
\begin{equation}
\abs{c_i(t)}^2 = \frac{\abs{W_{fi}}^2}{\hbar^2} \Delta t^2 \sinc^2[\omega_{fi}\Delta t/2]
\end{equation}

方形脉冲中的简谐微扰(\autoref{TDPEx_eq1}~\upref{TDPEx})
\begin{equation}
\abs{c_i(t)}^2 = \frac{\abs{W_{fi}}^2}{4\hbar^2} \Delta t^2 \{\sinc^2[(\omega_{fi}-\omega)\Delta t/2] + \sinc^2[(\omega_{fi}+\omega)\Delta t/2]\}
\end{equation}
当 $\Delta t$ 变大的时候, $\sinc^2$ 函数趋近于 $\delta$ 函数(\autoref{sinc_eq1}~\upref{sinc}).
\begin{equation}
\sinc^2[(\omega_{fi}\mp\omega)\Delta t/2] \to \frac{2\pi\hbar}{\Delta t}\delta(E_f - E_i \mp \omega\hbar)
\end{equation}
若 $E_{f0} = E_i \pm \omega\hbar$ 附近的态能量密度为 $\rho(E_{f0})$, 那么总跃迁概率约等于
\begin{equation}
\Delta P = \rho(E_{f0})\int_{E_{f0}-\epsilon}^{E_{f0}+\epsilon}\abs{c_i(t)}^2 \dd{E_f}
= \frac{\pi}{2\hbar} \abs{W_{fi}}^2\rho(E_{f0})\Delta t
\end{equation}
