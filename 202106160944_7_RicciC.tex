% Ricci曲率
% keys 里奇曲率|曲率|测地线|广义相对论|引力|爱因斯坦场方程|relativity|gravity|Einstein Field Equation|geodesic|curvature|differential geometry|manifold|流形|联络|connection

\pentry{仿射联络(切丛)\upref{affcon}}

本节中默认$(M, \nabla)$是一个配备了仿射联络$\nabla$的流形$M$.

\subsection{再论曲率映射}

我们在这里重新誊写一遍\textbf{仿射联络(切丛)}\upref{affcon}中定义的\textbf{曲率}映射.

\begin{definition}{曲率}
定义$R:\mathfrak{X}(M)\times\mathfrak{X}(M)\to\opn{End}(\mathfrak{X}(M))$为:对于任意$X, Y\in\mathfrak{X}(M)$,有$R(X, Y)=\nabla_X\nabla_Y-\nabla_Y\nabla_X-\nabla_{[X, Y]}$,称该映射为$(M, \nabla)$的\textbf{曲率(curvature)}.
\end{definition}

整体来看,曲率就是把两个光滑向量场映射为一个光滑向量场.一个自然的问题就是,这个映射,是$C^{\infty}$线性的吗?答案是肯定的,这一点由以下定理描述.

\begin{theorem}{曲率的线性性}
令$f, g, h$为$M$上的光滑函数,$X, Y, Z$为$M$上的光滑向量场.则我们有:$R(fX, gY)hZ=fgh\cdot R(X, Y)Z$.
\end{theorem}

\textbf{证明}:

由于$\nabla_{fX}=f\nabla_X$,且$[fX, gY]=fg[X, Y]+f(Xg)Y-g(Yf)X$

故有
\begin{equation}
\begin{aligned}
&\nabla_{fX}\nabla_{gY}-\nabla_{gY}\nabla_{fX}-\nabla_{[fX, gY]}\\=&f\nabla_X(g\nabla_Y)-g\nabla_Y(f\nabla_X)-fg\nabla_{[X, Y]}-\nabla_{f(Xg)Y-g(Yf)X}\\
\\=&fg\nabla_X\nabla_Y+f(Xg)\nabla_Y-gf\nabla_Y\nabla_X-g(Yf)\nabla_X\\&-fg\nabla_{[X, Y]}-f(Xg)\nabla_Y+g(Yf)\nabla_X\\
=&fg\nabla_X\nabla_Y-gf\nabla_Y\nabla_X-fg\nabla_{[X, Y]}\\
=&fg(\nabla_X\nabla_Y-\nabla_Y\nabla_X-\nabla_{[X, Y]})
\end{aligned}
\end{equation}

\textbf{证毕}.

























