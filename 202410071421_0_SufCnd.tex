% 充分必要条件
% keys 充分条件|必要条件|命题
% license Usr
% type Tutor

% 未完成: 引用韦恩图文章, 
\pentry{集合(高中)\nref{nod_HsSet}}{nod_af30}

\begin{example}{}
命题 $A$ :三角形 $X$ 的其中两内个角分别为 $90^\circ$ 和 $45^\circ$。

命题 $B$ :三角形 $X$ 有两个 $45^\circ$ 的内角。

利用三角形三个内角和为 $180^\circ$ 的事实,可以从 $A$ 推出 $B$, 说明 $A$ 是 $B$ 的充分条件, $B$ 是 $A$ 的必要条件。但也可以从 $B$ 推出 $A$, 说明 $B$ 是 $A$ 的充分条件, $A$ 是 $B$ 的必要条件。所以 $A$ 和 $B$ 既是彼此的充分条件也是彼此的必要条件。所以我们说 $A$ 和 $B$ \textbf{互为充分必要条件}。若 $A$ 是 $B$ 的充分必要条件, $B$ 一定也是 $A$ 的充分必要条件。因为两种表述都意味着 $A$,  $B$ 命题\textbf{等效},所提供的信息都是一样的,两者都没有任何多余的或者缺失的信息。
\end{example}

需要注意的是:
\begin{enumerate}
\item 充分/必要条件是两个命题之间的关系,说一个孤立命题是充分/必要条件没有意义。
\item 讨论充分/必要条件需要在一定的前提下进行。以上两个例子中的前提如: 我们讨论的是欧几里得几何中的平面四边形和三角形。 当然,我们也可以把这个前提直接写在每个命题中。
\item 在证明 $A$ 是 $B$ 的充分必要条件时,需要分别证明 $A$ (相对于 $B$)的充分性和必要性。充分性需要由 $A$ 证明 $B$, 必要性需要由 $B$ 证明 $A$。 
\item 在证明 $A$ 是 $B$ 的充分非必要条件时,除了需要证明 $A$ 的充分性,还需非必要性,即 $B$ 不能推出 $A$。 只要我们可以举出一个 $B$ 成立 $A$ 不成立的反例,就立刻证明了不可能由 $B$ 推出 $A$。 
\end{enumerate}

\subsection{用韦恩图理解}

简要说明:

有命题 $A$ 和 $B$ 
\begin{enumerate}
\item $A$ 推出 $B$ , $B$ 推不出 $A$ ,则为充分不必要,如图:
\begin{figure}[ht]
\centering
\includegraphics[width=8cm]{./figures/813479566ad8bd27.png}
\caption{充分不必要} \label{fig_SufCnd_2}
\end{figure}
口诀:有之必然,无之未必不然
\item $A$ 推出 $B$ , $B$ 推出 $A$ ,则为充要,如图:\begin{figure}[ht]
\centering
\includegraphics[width=10cm]{./figures/977aec7c48daaeea.png}
\caption{充要} \label{fig_SufCnd_3}
\end{figure}
\item $A$ 推不出 $B$ , $B$ 推不出 $A$ ,则既不充分也不必要,如图:\begin{figure}[ht]
\centering
\includegraphics[width=12cm]{./figures/72797bcad5cfcfdc.png}
\caption{既不充分也不必要} \label{fig_SufCnd_4}
\end{figure}
\item $A$ 推不出 $B$ , $B$ 推出 $A$ ,则必要不充分,如图:\begin{figure}[ht]
\centering
\includegraphics[width=9cm]{./figures/b8c97ea69e5f63a9.png}
\caption{必要不充分} \label{fig_SufCnd_5}
\end{figure}
口诀:有之未必然,无之必不然
\end {enumerate}
