% 绝对收敛与条件收敛

\pentry{数项级数\upref{Series}}

乍看起来, 一个数项级数$\sum_{n=1}^\infty a_n$的收敛性似乎没有什么特别之处: 它只是涉及到部分和的极限而已. 然而, 鉴于有限项的求和允许任意换序, 我们自然要问: 对于这个极限过程, 还能够这样随意换序吗? 由此就生发出了许多微妙的问题.

\subsection{从一个例子开始}

考虑级数
$$
\sum_{n=1}^\infty\frac{(-1)^{n+1}}{n}
=1-\frac{1}{2}+\frac{1}{3}-\frac{1}{4}+...
$$
它的各项绝对值组成的级数是发散的调和级数, 但这个级数本身却是收敛的. 实际上, 它的第$N$个部分和$S_N$可以计算如下: 当$N$是奇数时,
$$
S_N=\left(1-\frac{1}{2}\right)+\left(\frac{1}{3}-\frac{1}{4}\right)+...+\left(\frac{1}{N-2}-\frac{1}{N-1}\right)+\frac{1}{N},
$$
而当$N$是偶数时,
$$
S_N=\left(1-\frac{1}{2}\right)+\left(\frac{1}{3}-\frac{1}{4}\right)+...+\left(\frac{1}{N-1}-\frac{1}{N}\right).
$$
由于
$$
0<\frac{1}{N-1}-\frac{1}{N}<\frac{1}{N^2},
$$
可见对于$M>N$总有
$$
|S_M-S_N|<\sum_{k=N}^M\frac{1}{k^2}.
$$
根据柯西判据, 部分和序列$\{S_N\}$是有极限的, 所以原来的级数收敛. 以后将会算出这个极限等于$\ln2=0.6931...$