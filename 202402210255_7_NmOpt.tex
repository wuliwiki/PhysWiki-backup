% 正规算子
% keys 酉变换|厄米算子|酉空间|厄米变换|厄密变换|厄密算子|hermitian|normal operator|hermitian operator
% license Xiao
% type Tutor



\pentry{伴随映射\nref{nod_AdjMap}}{nod_?}


“算子”一词是“线性算子”的简称,在本文中指代“线性变换”或者“线性映射”。

本文讨论的是酉空间上的情况,但多数结论可以应用到内积空间或者更一般的配备了非退化双线性形式的空间中。


\subsection{概念}


\begin{definition}{正规算子}

给定酉空间$U$上的线性变换$A$,令$A^\dagger$是其\textbf{伴随变换}。若$AA^\dagger = A^\dagger A$,则称$A$是$U$上的\textbf{正规算子(normal operator)}。

如果$A=A^\dagger$,则称$A$是一个\textbf{厄米算子(hermitian operator)}。如果$AA^\dagger = E$,其中$E$是单位方阵,则称$A$是一个\textbf{酉算子(unitary operator)}。

\end{definition}


显然,厄米算子和酉算子都是正规算子。厄米算子的性质非常好,而一般的算子总可以分解为厄米算子的组合,这就为研究算子带来了便利:



\begin{lemma}{}
任取酉空间$U$上的线性变换$A$,则存在$U$上\textbf{唯一}的厄米算子$H_1$和$H_2$,使得
\begin{equation}
A = H_1 + \I H_2~. 
\end{equation}
\end{lemma}


\textbf{证明}:

对于任意的$A$,定义
\begin{equation}
\begin{cases}
H_1 =\frac{1}{2}\qty(A+A^\dagger), \\
H_2 =\frac{1}{2\I}\qty(A-A^\dagger)~. 
\end{cases}
\end{equation}

显然$H_1+\I H_2=A$,因此\textbf{存在性}得证。

反过来,假设$A = H_1 + \I H_2$,则$A^\dagger = H_1 - \I H_2$,从而$H_1$和$H_2$能由$A$\textbf{唯一}决定。

\textbf{证毕}。




正规算子的定义依赖于酉空间的内积,因为伴随变换的定义依赖于内积或者一个非退化双线性形式。因此,讨论正规算子的同时可以讨论向量的正交性,而正规算子实际上就是“可以被正交对角化的线性变换”。我们接下来就一步步推导出这一点。





\subsection{正规算子的对角化}



\begin{lemma}{算子交换则有公共特征向量}\label{lem_NmOpt_1}

给定$n$维酉空间$U$上的线性变换$A$和$B$,若$AB=BA$,则存在$\bvec{u}\in U$,它是$A$和$B$的公共特征向量。

\end{lemma}

\addTODO{可能需要引用“复矩阵都可以上三角化”的定理。}

\textbf{证明}:

有限维酉空间上任何算子都至少有一个特征向量,这是\textbf{代数学基本定理}\upref{BscAlg}保证的。

取$A$的特征向量$\bvec{x}$,使得$A\bvec{x}=\lambda\bvec{x}$。

构造子空间$W=\opn{Span}\{\bvec{x}, B\bvec{x}, B^2\bvec{x}, \cdots, B^{n-1}\bvec{x}\}$,则显然有
\begin{equation}
AB^k\bvec{x} = B^kA\bvec{x} = \lambda B^k\bvec{x}~
\end{equation}
对任意正整数$k$都成立,故$W$中的向量都是$A$的特征向量,特征值都是$\lambda$。

又因为按构造可知$W$是$B$的不变子空间,故$W$上必有$B$的特征向量$\bvec{u}$,此即为所求。

\textbf{证毕}。



由\autoref{lem_NmOpt_1} 显然可知如下推论:

\begin{corollary}{算子交换则有公共特征向量}

给定$n$维酉空间$U$上的有限多个线性变换$A_1, A_2, \cdots, A_k$,则存在它们的公共特征向量。

\end{corollary}



利用\autoref{lem_NmOpt_1} ,还可以得到正规算子的对角化:


\begin{theorem}{正规算子必能正交对角化}\label{the_NmOpt_1}
给定$n$维酉空间$U$上的线性变换$A$,若$AA^\dagger = A^\dagger A$,则存在一组标准正交基$\{\bvec{e}_i\}_{i=1}^n$,使得$A$的矩阵是对角矩阵。
\end{theorem}

\textbf{证明}:

据\autoref{lem_NmOpt_1} ,可以取$A$和$A^\dagger$的一个模长为$1$的公共特征向量$\bvec{e}_1$,将其拓展为一组标准正交基,则在这组基下$A$和$A^\dagger$的矩阵的第一列第一行不为零,第一列其它行都为零。又因为标准正交基下$A$和$A^\dagger$的矩阵互为“转置后取共轭”,因此可知两个矩阵的第一行第一列不为零,第一行其它列都为零。

这意味着,可以通过以酉矩阵为过渡矩阵的相似变换,将$A$和$A^\dagger$的矩阵同时化为第一行和第一列已经对角化的情况,于是接下来可以看除了第一行和第一列的部分,同样利用不改变第一行和第一列的酉矩阵将第二行和第二列同时对角化。以此类推,最终能用一系列酉相似变换将$A$和$A^\dagger$的矩阵同时对角化,即得证。

\textbf{证毕}。


\autoref{the_NmOpt_1} 反过来也成立:


\begin{theorem}{能正交对角化的就是正规算子}

给定$n$维酉空间$U$上的线性变换$A$,若在某组标准正交基下$A$是对角矩阵,则$A$是正规算子。

\end{theorem}

\textbf{证明}:

由于标准正交基下$A^\dagger$的矩阵就是$A$的矩阵做转置和取共轭后的结果,因此$A^\dagger$的矩阵也是对角矩阵。显然,对角矩阵的乘法总是可交换的,因此$AA^\dagger = A^\dagger A$。

\textbf{证毕}。


综上,我们即得到正规算子的等价定义:


\begin{corollary}{}
有限维酉空间上的正规算子
\end{corollary}















