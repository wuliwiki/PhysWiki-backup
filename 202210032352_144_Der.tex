% 导数(简明微积分)
% 微积分|导数|导函数|切线|极限

% 未完成: 举几个例子,说明常数导数为零, 直线导数为定值等!
\pentry{函数\upref{functi},切线与割线\upref{TanL},函数的极限(简明微积分)\upref{FunLim}}
\footnote{本文参考了\cite{同济高},\cite{Thomas}},
\footnote{除非特别声明,我们暂且假定探讨的函数在定义域内处处可导,\textsl{就和大多数物理学家和工程师所默许的一样}}
\subsection{一点处的导数}
\subsubsection{几何含义}
一个一元函数 $y = f(x)$ 在直角坐标系中表示为一条曲线. 在切线与割线\upref{TanL}中,我们已经初步了解了什么是切线与割线.

% \begin{figure}[ht]
% \centering
% \includegraphics[width=14cm]{./figures/Der_2.png}
% \caption{B趋向A,割线趋向切线} \label{Der_fig2}
% \end{figure}
\begin{figure}[ht]
\centering
\includegraphics[width=14cm]{./figures/Der_3.png}
\caption{B趋向A,割线趋向切线} \label{Der_fig2}
\end{figure}

我们先写出割线的直线方程.由于割线就是一条经过A,B两点的直线,根据高中数学知识,很容易得到
\begin{equation}
y-y_A=\frac{y_B-y_A}{x_B-x_A}(x-x_A)=\frac{f(x_B)-f(x_A)}{x_B-x_A}(x-x_A)
\end{equation}

现在,如\autoref{Der_fig2} 所示,我们固定A点不动,让B点趋近于A点(即令$x_B\rightarrow x_A$),这使割线趋向于切线.观察割线的方程,除了斜率项$k=\frac{f(x_B)-f(x_A)}{x_B-x_A}$,其余项均与$x_B$无关.当$x_B\rightarrow x_A$时,割线(此时被称为切线)的斜率即可用极限\upref{FunLim}来定义
\begin{equation}
k=\lim_{x_B\to x_A}\frac{f(x_B)-f(x_A)}{x_B-x_A}
\end{equation}

我们把这个切线的斜率定义为$x=x_A$处$f(x)$的导数.因此,我们说函数一点处的导数值就等于这点处切线的斜率.

\begin{figure}[ht]
\centering
\includegraphics[width=5cm]{./figures/Der_1.pdf}
\caption{点 $A$ 的切线.$f'(A)=k=\tan \theta$}
\end{figure}

更一般的,有
\begin{definition}{函数一点处的导数}
\begin{equation}
f'(x_0)=\lim_{x\to x_0}\frac{f(x)-f(x_0)}{x-x_0}
\end{equation}
也可以写为
\begin{equation}\label{Der_eq2}
f'(x_0)=\lim_{\Delta x\to0}\frac{f(x_0+\Delta x)-f(x_0)}{\Delta x}
\end{equation}
\end{definition}

% 若切线存在,该切线与 $x$ 轴的夹角 $\theta$ 的正切值,就叫点 $A$ 的导数.当函数在 $A$ 点递增时,可能的取值为 $\theta \in (0,\pi/2)$, 即 $\tan \theta  \in (0, + \infty)$. 递减时,取 $\theta  \in (-\pi/2,0)$, 即 $\tan \theta \in (-\infty ,0)$. 当切线水平时,$\theta  = \tan \theta  = 0$. 
\subsubsection{“物理”含义}
仔细观察$x=x_0$附近切线的形状与$f(x)$的形状,你很容易看出这两者是几乎一样的\footnote{这个结论可不是我瞎说的,数学上是能给出严格的证明},这启发我们在一个小区域内用切线来近似原函数.

\begin{figure}[ht]
\centering
\includegraphics[width=14cm]{./figures/Der_2.pdf}
\caption{将切点放大,会发现切线和曲线在切点附近 “重合”}\label{Der_fig1}
\end{figure}

此时,若想计算$x$轻微增加后,函数值$y$的增量,就没有必要复杂地计算
$$\Delta y = f(x_0+\Delta x) - f(x_0)$$
,而只需要计算
\begin{equation}
\Delta y = f'(x_0) \Delta x
\end{equation}
这就引入了微分的概念\upref{Diff}.更重要的是,这启发了我们导数的另一层含义:导函数确定了$x=x_0$处,$y$关于$x$变化的“敏感度”,即$x$轻微变化时,$y$会做出多大的响应.例如,一个绝对值大的导数值意味着$y$会因为$x$的轻微变化而剧烈变化.

理解这一含义的最直白例子或许是速度\upref{VnA1},速度被定义为物体位置关于时间的导数$v=\dv{r}{t}$.即使只靠直觉,我们也能理解\textsl{“速度快”就是指“一瞬间他就从我眼前飞过去了”},这就是说当时间轻微增加时,物体的位置大幅变化.

%若函数曲线在某一点附近是光滑的,那么在这点附近取一小段,当这一段取得足够小,可以近似认为它是线段且与切线重合(如下图). 以这条线段为斜边,作一直角三角形,令其底边长为 $\dd{x}$ (在微积分中,通常把非常小的一段 $\Delta x$ 记为 $\dd{x}$,  $\dd{x}$ 是一不能分割的整体符号,而不是两个量相乘),竖直边的边长为 $\dd{y}$ (当函数递增时, $\dd{y}$ 取正值,反之取负值).根据上面导数的定义,$\dv*{y}{x} = \tan \theta $ 就是函数的导数.所以导数通常表示为 $\dv*{y}{x}$, 导数的倒数则为 $\dv*{x}{y}$. 

%由上面的讨论可得,当 $x$ 增加一小段 $\Delta x$ 时,$y$ 轴的增量约为 $\Delta y \approx f'(x)\Delta x$,且当 $\Delta x$ 越小,这条式子就越精确成立, 记为 $\dd{y} = f'(x) \dd{x}$.这个关系就叫函数的微分.
\subsubsection{单侧导数}
%需要补充一张图片,但好像有bug,上传新图片会覆盖掉旧的图片
类似于单侧极限\upref{FunLim},我们也可以引入单侧导数的概念.如果B点从右侧趋近A点,但始终不运动到A点的左侧,那么此时切线的斜率即为该点处函数的右导数值,可以记为$f'_+(x)$.用极限的语言可以写为:

\begin{definition}{单侧导数}
右导数:
\begin{equation}
f'_+(x_0) = \lim_{x\to x_0^+} \frac{f(x)-f(x_0)}{x-x_0}
\end{equation}
同理,可以定义左导数:
\begin{equation}
f'_-(x_0) = \lim_{x\to x_0^-} \frac{f(x)-f(x_0)}{x-x_0}
\end{equation}
\end{definition}

类似于极限,我们也有
\begin{theorem}{}
函数在某点可导的充分必要条件是它左右导数都存在并相等.
$$f'(x_0)=A\Longleftrightarrow f'_+(x_0)=f'_-(x_0)=A$$

也就是说,若左(或右)导数不存在,或者左、右导数存在但不相等,那此处的导数就不存在.
\end{theorem}

\begin{example}{}
如图,在棱角处,虽然函数连续,甚至左、右导数均存在,但他们的大小不相同,因此在棱角处该函数不可导.
\begin{figure}[ht]
\centering
\includegraphics[width=5cm]{./figures/Der_3.pdf}
\caption{棱角处不可导}
\end{figure}

\end{example}
\subsection{导函数}
在上文中,我们定义了一点处函数的导数.原则上我们可以任意选取函数定义域中的一点,然后用根据上文“导数的定义”找到该点处的导数值.这也就是说,对于函数定义域中的任意一点,都有一个导数值与之对应,这符合函数\upref{functi}的定义,也就是说我们可以在这二者间定义一个新函数,这就是$f(x)$的“导函数”;在不引起混淆的情况下往往简称为“导数”.
%若函数曲线在 $x$ 的某一开区间内的每一点都可导, 则这个区间上每一个 $x$ 对应一个导数.将其写成关于 $x$ 的函数 $g(x)$,  $g(x)$  就是该区间上的 \textbf{导函数}. 

字面上看,导函数的定义与一点处函数的导数完全类似.
\begin{definition}{导函数}
\begin{equation}
f'(x)=\lim_{\Delta x\to0}\frac{f(x+\Delta x)-f(x)}{\Delta x}
\end{equation}
\end{definition}

通常将导函数记为以下的一种%(后3种记号的来源见下文)
\begin{equation}\label{Der_eq1}
f'(x),\quad [f(x)]',\quad \dv{y}{x},\quad \dv{f}{x},\quad \dv{x}f(x)
\end{equation}
在物理中, 还可以在物理量上方加一点表示对时间求导(注意仅限于对时间求导), 例如 $\dot f(t) = \dv*{f(t)}{t}$.

\begin{example}{}
计算$f(x)=2x+3$的导数.

根据定义,
$$
\ali{
f'(x)&=\lim_{\Delta x\to0}\frac{f(x+\Delta x)-f(x)}{\Delta x}\\
&=\lim_{\Delta x\to0}\frac{[2(x+\Delta x)+3]-(2x+3)}{\Delta x}\\
&=\lim_{\Delta x\to0}\frac{2\Delta x}{\Delta x}\\
&=2
}
$$
事实上,所有直线方程的导函数都是常函数,且数值上等于自己的斜率.当然,实际上很少直接使用定义计算导数,有一些技巧可以简化求导过程(见本节其他文章).
\end{example}

% 若切线不存在(例如折线的棱角处,但也有其他更复杂的情况), 我们说点 $A$ 不可导. 如果某区间内是 “光滑” 的, 那么的该区间内处处可导.

% \begin{figure}[ht]
% \centering
% \includegraphics[width=5cm]{./figures/Der_3.pdf}
% \caption{棱角处不可导}
% \end{figure}

% \subsection{导数的严谨理解}
% 导数的代数理解就是: 一个量关于另一个量的变化率. 例如质点直线运动时,速度的大小就是其路程对时间的导数.把这种描述用极限\upref{Lim}表达出来就是
% \begin{equation}\label{Der_eq2}
% f'(x) = \lim_{\Delta x \to 0} \frac{f(x + \Delta x) - f(x)}{\Delta x}
% \end{equation}
% 在图3的右图中,$\Delta x$ 的始末位置并不非常重要,既可以从 $x$ 取到 $x + \Delta x$, 也可以从 $x - \Delta x$  取到 $x$ 等等( 因为当 $\Delta x$ 非常小的时候,$x$ 附近的曲线基本处处跟切线重合,它们的斜率都是一样的). 所以导数的定义也有其他类似的形式

% \begin{equation}
% f'(x) = \lim_{\Delta x \to 0} \frac{f(x) - f(x - \Delta x)}{\Delta x} = \lim_{\Delta x \to 0} \frac{f(x + \Delta x) - f(x - \Delta x)}{2\Delta x}
% \end{equation}
% 虽然上面用到了诸如“近似”等词,但根据定义,极限都是精确的.

% 例子: 速度 加速度(一维)\upref{VnA1}.







