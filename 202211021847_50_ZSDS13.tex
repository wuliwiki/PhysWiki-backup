% 浙江理工大学 2013 年数据结构
% 浙江理工大学 2013 年数据结构

\subsection{一、 单选题(在每小题的四个备选答案中选出一一个正确答案.每小题2分,共20分.)}

1. 链表不具备的特点是 \\
A.可随机访问任一结点 \\
B .插入删除不需要移动元素 \\
C .不必事先估计存储空间 \\
D .所需空间与其长度成正比

2. 设线性表有n个元素,以下算法中,___在顺序表 上实现比在链表上实现效率更高. \\
A.交换第0个元素与第1个元素的值 \\
B .顺序输出这n个元素的值 \\
C .输出第i(Osisn-1)个元素值 \\
D .输出与给定值x相等的元素在线性表中的序号

3. 设输入序列为a、b、C、d ,则借助栈所得到的输出序列不可能是___. \\
A.a、b、c、d \\
B.d、c、b、a \\
C.a、c、d、b \\
D.d、a、b、c

4. 为解决计算机主机与打印机之间的速度不匹配问题,通常设计一个打印数据缓冲区, 主机将要输出的数据依次写入到该缓冲区,而打印机则依次从该缓冲区中取出数据.该缓冲区的逻辑结构应该是 \\
A.栈 $\qquad$ B.队列 $\qquad$ C.树 $\qquad$ D.图

5. 设哈夫曼树中的叶子结点总数为m ,若用二叉链表作为存储结构,则该哈夫曼树中总共有___个空指针域. \\
A.2m $\qquad$ B.4m $\qquad$ C . 2m+1 $\qquad$ D.2m-1

6. 二叉树若用顺序存储结构表示,则下列四种运算中____最容易实现. \\
A.先序遍历二叉树 $\qquad$ B.层次遍历二叉树 \\
C.中序遍历二叉树 $\qquad$ D.后序遍历二又树

7. 以下关于有向图的说法正确的是_ \\
A .强连通图是任何顶点到其他所有顶点都有边 \\
B .完全有向图一定是强连通图 \\
C .有向图中某顶点的入度等于出度 \\
D .有向图边集的子集和顶点集的子集可构成原有向图的子图

8. 若一个有向图中的顶点不能排成一一个拓扑结构序列,则可断定该有向图 \\
A.含有多个出度为0的顶点 \\
B.是个强连通图 \\
C.含有多个入度为0的顶点 \\
D.含有顶点数目大于1的强连通分量

9. 顺序查找法适合于存储结构为的线性表. \\
A.哈希存储 \\
B.压缩存储 \\
C.顺序存储或链式存储 \\
D.索引存储

10. 在所有排序方法中,关键字比较的次数与记录地初始排列次序无关的是_ \\
A.shell排序 \\
B.冒泡排序 \\
C.直接插入排序 \\
D.简单选择排序

\subsection{二、填空题(每空2分,共30分.)}

1. 下面程序段的时间复杂度是
\begin{lstlisting}[language=cpp]
for (i=0; i<n; i++)
  for (j=0; j<m; j++)
    A[i][j]=0;
\end{lstlisting}

2. 向一个不带头节点的栈指针为Ist的链式栈中插入一个*s所指节点时,则执行(   )和(    ).

3. 在二叉链表中判断某指针p所指结点为叶子结点的条件是按_遍历一棵二叉排序树所得到的结点访问序列是一一个有序序列.

5. 广义表A=((a,b,c,d),( ))的表尾是(    ).

6. 有一个10阶对称矩阵A ,采用压缩存储方式(以行序为主存储,且A[0][0]=1) ,则A[8][5]的地址是(    )

7. 高度为h(>=0)的二叉树,至少有(    )个结点,最多有(    )个结点.

8. 普里姆(PRIM)算法更适合于求边(    )的网的最小生成树.

9. 在无向图G的邻接矩阵A中,若A[i][j]等于1 ,则A[j][i]等于(    ).

10. 在对一组记录(54, 38, 96, 23, 15, 72, 60, 45 , 83)进行直接插入排序时,当把第7个记录60插入到有序表时,为寻找插入位置需比较(    )次.

11. 若一组记录的排序码为( 46, 79, 56, 38, 40, 84) ,则利用堆排序的方法建立的初始堆为().

12. 有一个长度为10的有序表,按折半查找法对该表进行查找,在表内各元素等概率情况下查找成功所需的平均比较次数为().

13. 在一棵平衡的二叉树中,每个节点的平衡因子B的取值范围是().

\subsection{三、判断题(每小题2分,共20分.)}

1. 对于数据结构,相同的逻辑结构,对应的存储结构也必相同.( )

2. 哈夫曼树中没有度数为1的结点.( )

3. 线性表中的所有元素都有一个前驱元素和后继元素.()

4. 除了删除和插入操作外,数组的主要操作还有存取、修改、检索和排序.()

5. 链表的每一个结点都恰好包含一个指针. ( )

6. 无向图的邻接矩阵一定是对称矩阵,且有向图的邻接矩阵一定是非对称矩阵. ( )

7. 若有一个结点是某二叉树子树的中序遍历序列中的最后一个结点,则它必是该子树的前序遍历序列中的最后一个结点.( )

8. 冒泡排序在初始关键字序列为逆序的情况下执行的交换次数最多. ( )

9. 满二叉树一定是完全二叉树,完詮二叉树不一定是满二叉树.( )

10. 快速排序是排序算法中平均性能最好的一种排序.( )

\subsection{四、应用题(共50分.)}

1. (14分)已知一棵二叉树如右图所示: \\
(1)中序全线索化二叉树; \\
(2)写出对该二叉树进行先序遍历和后序遍历的结果; \\
(3)试画出其相应的树.

2. (12分)已知某有向图的邻接矩阵为:
\begin{table}[ht]
\centering
\caption{第四2题图:邻接矩阵}\label{ZSDS13_tab1}
\begin{tabular}{|c|c|c|c|c|c|c|c|c|c|c|}
\hline
 & V1 & V2 & V3 & V4 & V5 & V6 & V7 & V8 & V9 & V10 \\
\hline
V1 & 0 & 1 & 1 & 1 & 0 & 0 & 0 & 0 & 0 & 0 \\
\hline
V2 & 0 & 0 & 0 & 1 & 1 & 0 & 0 & 0 & 0 & 0 \\
\hline
V3 & * & * & * & * & * & * & * & * & * & * \\
\hline
* & * & * & * & * & * & * & * & * & * & * \\
\hline
* & * & * & * & * & * & * & * & * & * & * \\
\hline
* & * & * & * & * & * & * & * & * & * & * \\
\hline
* & * & * & * & * & * & * & * & * & * & * \\
\hline
* & * & * & * & * & * & * & * & * & * & * \\
\hline
* & * & * & * & * & * & * & * & * & * & * \\
\hline
* & * & * & * & * & * & * & * & * & * & * \\
\hline
\end{tabular}
\end{table}
(1)画出此图的对应邻接表,要求边结点按照序号从大到小排序; \\
(2)写出以(1)为存储结构的、顶点V1为出发点的深度优先遍历次序; \\
(3)写出以(1)为存储结构的、顶点V1为出发点的广度优先遍历次序.

