% Makefile 笔记
% make|makefile|gnu|编译

% 摘自我的 GitHub/Notes/Programming/Fortran/makefile_笔记2.md

\pentry{Makefile 简介\upref{Make0}}

\begin{itemize}
\item 要系统地学习 make, 参考资料是 \href{https://www.gnu.org/software/make/manual/}{GNU Make 文档}, 提供网页版和 pdf 下载。
\item make 程序广义上是声明如何更新一系列文件, 以及他们的 dependencies。 当某个文件改变后, 任何依赖于它的文件都需要通过指定的规则 (rule) 来更新
\item \verb|Makefile| 的语法虽然类似 bash\upref{Bash}, 但并不是 bash。 例如给变量赋值时, bash 不允许等号两边有空格, 但 Makefile 可以。
\item 使用 \verb|make| 命令是可以选择使用多线程, 例如 \verb|make -j4| 使用 4 线程。 也可以 \verb|make -j`getconf _NPROCESSORS_ONLN`| 自动设置为 cpu 的数量。
\item 要指定 Makefile 文件, 用 \verb|make -f 文件|
\item 要显示详细的推导信息, 用 \verb|make --debug=v|
\item 突然发现, rule 的 “:” 前面不可以有空格 ?
\item rule 的第一行用于声明什么文件取决于什么文件, 剩下的行声明用什么命令去更新。
\item \verb|:| 后面的文件会按照顺序更新, 而且只会更新一次, 若以如果更新第二个文件的时候第一个文件被删掉也不会出错。
\item \verb|:| 后面可以有 phony target, 比如说 \verb|make clean| 中的 \verb|clean|
\item \verb|$(var)| 和 \verb|${var}| 是完全等效的, 在双引号内也可以替换。 在 rule 的命令中也可以替换。
\item 赋值有两种, \verb|var1 = $(...)| 可以延迟展开, 也就是说使用每次 \verb|$(var1)| 时会重新求一次 \verb|$(...)|。 如果用 \verb|var1 := $(...)|, 那么会马上对此时的 \verb|$(...)| 展开, 何时使用 \verb|$(var1)| 都不受其他因素影响。 如果等号右边没有变量, 那么两种赋值等效。
\item 如果要在 rule 的命令中使用 bash 的变量而不是 Makefile 的变量, 要把 \verb|$| escape 成 \verb|$$|。 如 \verb|echo $${var}|, 此时传给 bash 的命令就是 \verb|echo ${var}|。
\item 判断句如下, 注意第二行前面是空格而不是 tab。
\begin{lstlisting}[language=makefile]
ifeq ($(var1), true)
    var2 = false
else ifneq($(var2), abc)
    ...
endif
\end{lstlisting}
\item Makefile 中 \verb|$(info 显示 一些 文字)| 会在 make 的时候显示文字到命令行。
\item 要想列出 \verb|Makefile| 包含的所有 target, 直接在命令行用 \verb|make|, 空格, 然后按两次 tab 自动补全即可。
\item \verb|rm| 命令后面记得加 \verb|-f| 选项, 否则如果文件不存在就会出错导致 make 就会失败
\item 可以根据所有 rule 中的 target 和 dependency 画一个树状图, 如果某个点的文件不存在, 那么就会先运行生成它的 rule, 如果图中的任何一点更新了, 从这点到顶点的所有文件都要更新。 具体的规则是, 如果一个 target 的任何一个 dependency 比它要新, 那么 target 就要重新编译。
\item “goal :” 可以声明 make 的终极目标, 如
\begin{lstlisting}[language=makefile]
goal: file1 file2 ...
	command1
	command2
\end{lstlisting}
\item 注意 Makefile 中空格和 tab 是区分的! command 前面必须是 tab。
\item 如果没有 goal 的话, default goal 就是第一个开头不为 “.” 的 target。 剩下的 rule 的顺序应该可以随意
任何一个被依赖的文件改变了, 或者它们依赖的文件改变了, 就会执行 command。 make 并不知道 command 的含义。 只是把它传给 shell 来执行。
\item implicit rules 大概就是可以仅声明 “A: B”, 由自定义的命令或者 make 默认的命令来生成 “A”。 implicit rule 中的 dependency 是至少有这些 dependency, 而不是只能有这些 dependency。
\item “MAKEFLAGS = -r” 大概是用来取消默认 implicit rules, 包括所有后缀名识别
\item 老的 suffix rule 和 implicit rules 的功能差不多, 现在已经过时了, 应该用 implicit rule。 一个例子如
\begin{lstlisting}[language=makefile]
.f90.o:
	gfortran -c $<
\end{lstlisting}
其中 \verb|$<| 是 auto variable 中的一个 (见 10.5.3 Automatic Variables), 在执行的时候被替换成 \verb|:| 右边的第一个 dependency。 现在如果有
\begin{lstlisting}[language=makefile]
file1.o: file1.f90 file2.o file3.o
\end{lstlisting}
那么应该会执行 \verb|gfortran -c file1.f90|。 另外, 如果 “file2.o” 或 “file3.o” 被更新了, 这条命令应该也会再执行一次。
\item \verb|$^| 列出所有的 prerequisites (“:” 右边的内容)
\item \verb|$(shell ...)| 可以执行 shell 命令并把命令行输出替换到当前未知, 如 \verb|$(shell echo *.f90)| 可以在当前位置列出所有 “.f90” 文件。 如果命令太长可以用 \verb|\| 换行。 注意输出中的换行符会替换为空格。
\item 如果只是想在非 recipe 中执行某个 shell 命令并丢弃输出, 那么用 \verb|没用的变量 := $(shell 命令)|。  如果要把结果也输出到命令行, 再用 \verb|$(info 变量)| 即可。 赋值一定要有, 要不然就相当于把输出的内容直接插入到 Makefile 中造成语法错误。 另外赋值一定要用 \verb|:=| 千万不能用 \verb|=|, 前者立即展开 \verb|$(...)|, 后者只有在 \verb|没用的变量| 被使用时才展开。
\item \verb|$@| 大概就是 target file (如果 “:” 左边只有一个文件的话)
\item 如果在 rule 的命令中用 \verb|source 脚本| 会出错, 因为默认 shell 是 \verb|/bin/sh|。 可以用 \verb|. 脚本| 等效替代。 也可以用 \verb|SHELL := /bin/bash|。
\item 在 rule 的命令中用 \verb|: 要显示的内容| 可以在命令行显示内容。 其中 \verb|: 命令| 不做任何事情。
\item \verb|make VAR1=... VAR2=...| 可以设置参数(等号两边不能有空格!), 相当于在 \verb|Makefile| 里面使用 \verb|VAR1=... VAR2=...|, 如果 \verb|Makefile| 里面已经设置了这些参数(默认值), 那么则会覆盖。 
\item 在 rule 的命令前面加一个 \verb|@| 就可以不在 std 输出命令的内容。 例如 \verb|@printf "一些信息"| 或者 \verb|@echo '一些信息'|
\item \verb|Makefile| 的 recipe 中可以再次调用 \verb|make 目标|, 这样可以保证目标按顺序运行, 或者 \verb|cd ... && make 目标| 这样可以更改当前目录。
\item 第一次进入 Makefile 时, \verb|MAKELEVEL| 的值是 0。 如果再次使用 \verb|make| 命令, \verb|MAKELEVEL| 会递增。
\item rule 也可以在条件中定义, 例如
\begin{lstlisting}[language=makefile]
ifeq ($(var), val)
目标1:
	命令1
else
目标1:
	命令2
endif
\end{lstlisting}
\item \verb|$(error 一些 错误 信息)| 放在 \verb|命令2| 里面可以提示错误。
\item recipe 中如果一个命令太长, 可以用 \verb|\| 换行, 缩进可以用多个 tab。
\item 如果一个 recipe 的\textbf{最后一个} exit code 是非 0, 那么 make 将终止。 linux 的 \verb|false| 命令会主动返回非 0。
\end{itemize}

\subsubsection{implicit rules 的变量}
\begin{itemize}
\item 参考官方\href{https://www.gnu.org/software/make/manual/html_node/Implicit-Variables.html}{文档}。
\item makefile 的 Implicit Rules, 会使用一些预定义的变量, 可以用 \verb|--no-builtin-variables| 关闭。
\item 这些变量可以在 makefile 中定义, 可以作为 make 的参数, 也可以作为环境变量
\item 在使用 automake\upref{automk} 和 cmake\upref{CMakeN} 等工具时, 也可以用这些变量来指定额外的编译选项(当他们生成的是 Makefile 时)。
\item \verb|CC| 是 C 编译器, \verb|CXX| 是 C++ 编译器
\item \verb|CFLAGS| 是 C 编译器的选项, \verb|CXXFLAGS| 是 C++ 编译器的选项, \verb|FFLAGS| 是 Fortran 编译器选项。
\end{itemize}

\subsection{字符串处理}
\begin{itemize}
\item \verb|$(subst 旧词, 新词, 字符串)| 字符串替换
\item \verb|$(addsuffix 后缀, 列表)| 把列表中每一个元素后面都加上 \verb|后缀|
\item \verb|$(addprefix 后缀, 列表)| 同理
\item \verb|$(notdir 列表)| 把列表中的 \verb|目录/文件| 变为 \verb|文件|。
\item \verb|$(sort 列表)| 可以排序并移除列表中重复的元素
\item \verb|$(filter-out 列表, 列表1)| 把 \verb|列表1| 中的元素从 \verb|列表| 中移除。
\end{itemize}


\subsection{其他}
\begin{itemize}
\item \verb|include 文件| 可以把某个文件的内容插入当前地方, 如果文件不存在则会警告, 且如果该文件是一个 target, 每次用 \verb|make| 会先检查是否需要更新。 若不想要警告, 也不想更新, 可以用 \verb|-include 文件|。
\item \verb|g++| 的 \verb|-MM| 选项可以生成某个 cpp 文件或者 h 文件的所有依赖(包括依赖的依赖)。 \verb|-MM -nostdinc++| 则可以在依赖中去掉标准库中的头文件。 用 \verb|-include xxx| 还可以在依赖中加上 \verb|xxx|。
\item \verb|g++| 的 \verb|-M| 选项也一样, 但会生成多条依赖关系, 每个依赖关系只包含直接依赖。
\item 可以分多次指定一个文件的依赖, 如 \verb|file: file1 file2|, 然后又 \verb|file: file3 file4|。 但是 recipe 只能有一个, 可以写在任意一个依赖关系下面, 也可以另外添加一个 \verb|file:| 然后写在下面。
\end{itemize}
