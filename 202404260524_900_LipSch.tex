% Lippmann-Schwinger 方程
% license Xiao
% type Tutor

\pentry{薛定谔方程(单粒子多维)\nref{nod_QMndim}, 格林函数\nref{nod_GreenF}, 亥姆霍兹方程的格林函数}{nod_2074}

\begin{equation}
H = H_0 + V, \qquad H_0 = \frac{\bvec p^2}{2m}~.
\end{equation}
令 $H_0\ket{\phi} = E\ket{\phi}$, $(H_0+V)\ket{\psi}=E\ket{\psi}$。 那么形式上就有
\begin{equation}
\ket{\psi} = \frac{1}{E-H_0} V\ket{\psi} + \ket{\phi}~.
\end{equation}
但 $1/(E-H_0)$ 是奇异的, 要解决这个问题, 可以把它变得稍微复数一些($\varepsilon$ 是无穷小)
\begin{equation}
\ket*{\psi^{(\pm)}} = \ket{\phi} + \frac{1}{E-H_0\pm \I\varepsilon} V \ket*{\psi^{(\pm)}}~,
\end{equation}
这就是 \textbf{Lippmann-Schwinger 方程}。 放到位置表象中就是
\begin{equation}\label{eq_LipSch_1}
\braket*{\bvec x}{\psi^{(\pm)}} = \braket{\bvec x}{\phi} + \int \dd[3]{x'} \mel{\bvec x}{\frac{1}{E-H_0\pm \I\varepsilon}}{\bvec x'} \mel{\bvec x'}{V}{\psi^{(\pm)}}~,
\end{equation}
这是一个积分方程。 如果 $\ket{\phi}$ 是平面波。
\begin{equation}\label{eq_LipSch_2}
G_\pm (\bvec x, \bvec x') = \frac{1}{2m} \mel{\bvec x}{\frac{1}{E-H_0\pm \I\varepsilon}}{\bvec x'}~,
\end{equation}
可以证明(见下文)
\begin{equation}\label{eq_LipSch_3}
G_\pm (\bvec x, \bvec x') = -\frac{1}{4\pi} \frac{\E^{\pm\I k\abs{\bvec x-\bvec x'}}}{\abs{\bvec x-\bvec x'}}~,
\end{equation}
其中 $k = \sqrt{2mE}$。 这就是亥姆霍兹方程的格林函数
\begin{equation}
(\laplacian +k^2)G_\pm (\bvec x, \bvec x') = \delta(\bvec x-\bvec x')~.
\end{equation}

经过一番推导, \autoref{eq_LipSch_1} 变为
\begin{equation}
\braket*{\bvec x}{\psi^{(\pm)}} = \braket{\bvec x}{\phi} - 2m\int \dd[3]{x'} \frac{\E^{\pm\I k\abs{\bvec x-\bvec x'}}}{4\pi\abs{\bvec x-\bvec x'}} V(\bvec x')\braket*{\bvec x'}{\psi^{(\pm)}}~,
\end{equation}
然后计算可以发现平面波散射的边界条件为
\begin{equation}
\braket*{\bvec x}{\psi^{(\pm)}} \overset{r\to\infty}{\longrightarrow} \frac{1}{(2\pi)^{3/2}} \qty[\E^{\I \bvec k\vdot \bvec x} + f(\bvec k', \bvec k)\frac{\E^{\I kr}}{r}]~,
\end{equation}
其中
\begin{equation}
f(\bvec k', \bvec k) = -4\pi^2 m \mel*{\bvec k'}{V}{\psi^{(+)}}~.
\end{equation}

\subsubsection{证明}
现在来证明\autoref{eq_LipSch_3}。
\begin{equation}\label{eq_LipSch_4}
\begin{aligned}
&\frac{1}{2m} \mel{\bvec x}{\frac{1}{E-H_0\pm \I\varepsilon}}{\bvec x'}\\
&= \frac{1}{2m} \int \dd[3]{k'} \int \dd[3]{k''} \braket{\bvec x}{\bvec k'}\mel{\bvec k'}{\frac{1}{E-\bvec p^2/(2m)\pm \I\varepsilon}}{\bvec k''} \braket{\bvec k''}{\bvec x'}~.
\end{aligned}
\end{equation}
易知
\begin{equation}
\frac{1}{E-\bvec p^2/(2m)\pm \I\varepsilon}\ket{\bvec k''} = \frac{\ket{\bvec k''}}{E- k^2/(2m)\pm \I\varepsilon}~,
\end{equation}
所以
\begin{equation}
\mel{\bvec k'}{\frac{1}{E-H_0\pm \I\varepsilon}}{\bvec k''} = \frac{\delta(\bvec k'-\bvec k'')}{E-k'^2/2m\pm\I\varepsilon}~.
\end{equation}
另外
\begin{equation}
\braket{\bvec x}{\bvec k'} = \frac{\E^{\I \bvec k'\vdot \bvec x}}{(2\pi)^{3/2}}
\qquad
\braket{\bvec k''}{\bvec x'} = \frac{\E^{-\I \bvec k''\vdot \bvec x'}}{(2\pi)^{3/2}}~,
\end{equation}
于是\autoref{eq_LipSch_4} 右边变为
\begin{equation}
\frac{1}{(2\pi)^32m}\int\dd[3]{k'}  \frac{\E^{\I \bvec k'\vdot (\bvec x-\bvec x')}}{E-k'^2/2m\pm\I\varepsilon}~.
\end{equation}
在球坐标中积分, 使用留数定理\footnote{然而我也不知道具体怎么操作……}, 就得到\autoref{eq_LipSch_3}。 结果与 $\varepsilon$ 无关。
