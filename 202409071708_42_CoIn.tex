% 协变性和不变性
% keys 协变性|不变性|物理
% license Usr
% type Tutor

协变性和不变性是相对论中会遇到的术语,本节给出它们具体的定义。
\subsection{概念的引入}
物理定律经常被表达为一个矢量等于另一个矢量,例如,Newton定律 
\begin{equation}
m\bvec a=\bvec F.~
\end{equation}
若换一参考系(重选基底),它和原参考系由旋转变换相联系。设 $R(\theta)$ 是这两参考系下矢量的变换矩阵,即若 $\bvec x$ 是旧参考系下表达的矢量,则新参考系下表达的该矢量 $\bvec x'$ 和旧参考系下的 $\bvec x$ 关系为 $\bvec x'=R(\theta)\bvec x$。将 $R(\theta)$ 作用于Newton定律,就有
\begin{equation}
mR(\theta)\bvec a=R(\theta)\bvec F.~
\end{equation}
由于加速度是矢量,因此新参考系下的加速度为 $\bvec a'=R(\theta)\bvec a$。假设 $\bvec F$ 像一个矢量一样变换(虽然这里对力使用了矢量相同的符号,但由于力),那么新坐标下 $\bvec F'=R(\theta)\bvec F$。于是在新坐标系下,就有
\begin{equation}
m\bvec a'=\bvec F'.~
\end{equation}
即两坐标系下










