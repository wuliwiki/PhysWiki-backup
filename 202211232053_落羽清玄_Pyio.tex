% Python 输入和输出
% Python|输入|输出|print|input

\begin{issues}
\issueTODO
\issueDraft
\end{issues}

\pentry{Python 基本数据类型与转换\upref{PyData}}

\begin{itemize}
\item Python两种输出值的方式: 表达式语句和 print() 函数.
\item 第三种方式是使用文件对象的 write() 方法,标准输出文件可以用 sys.stdout 引用.
\item 如果你希望输出的形式更加多样,可以使用 str.format() 函数来格式化输出值.
\item 如果你希望将输出的值转成字符串,可以使用 repr() 或 str() 函数来实现.
\end{itemize}

\begin{itemize}
\item str(): 函数返回一个用户易读的表达形式
\item repr(): 产生一个解释器易读的表达形式.
\end{itemize}

注意:在python2.x版本中,\verb|print| 是一个关键字,输出内容不需要加括号, python 3.x 中它是一个函数, 可以介绍任意多个用逗号隔开的变量, 他们甚至可以是不同类型的. 例如
\begin{lstlisting}[language=python]
print volumn  #2.x
print(volumn) # 3.x
\end{lstlisting}
