% 中国科学院 2018 年考研普通物理
% 中国科学院|考研|普通物理

\subsection{选择题}
\begin{enumerate}
\item 单位质点在一个势能为 $V(r) = -\frac{a}{r}e^{-m_{o}r}+\frac{k}{2}r^{2}$ 的保守力场中运动,其中 $r$ 是质点离中心的径向距离,$(a,m_{0},k)$ 是势能参数.那么在该保守立场中,单位质点所受到的力为\\
A. $-\frac{ae^{-m_{0}r}}{r^{2}}-\frac{am_{0}e^{-m_{0}r}}{r}-kr$\\
B. $-\frac{ae^{-m_{0}r}}{r^{2}}-\frac{am_{0}e^{-m_{0}r}}{r}+kr$\\
C. $-kr$\\
D. $-\frac{ae^{-m_{0}r}}{r^{2}}-\frac{am_{0}e^{-m_{0}r}}{r}$\\

A. \\
B. \\
C. \\
D. \\

\item 光滑水平桌面上静放一根长为 $L$ 的均质细杆,其质量为 $M$ .已知 $P$ 点离杆的右端点距离为 $\frac{L}{4}$ ,如图所示.以 $P$ 点为轴心,转轴垂直于杆,则杆的转动惯量为\\
A. $\frac{1}{12}ML^{2} \quad$
B. $\frac{5}{24}ML^{2} \quad$
C. $\frac{7}{48}ML^{2} \quad$
D. $\frac{1}{24}ML^{2} \quad$

\item 有一弹性介质,其介质密度记为$\rho$.在该介质中,有一种机械振动,其振东量表示为$\xi = \cos{[\omega(t-\frac{x}{u})]}$.下面有关介质中能量密度的说法正确的是\\
A. 能量密度只有动能项,其值为$\frac{\rho}{2}\omega^{2}\sin^{2}{[\omega(t-\frac{x}{u})]}$\\
B. 能量密度只有势能项,其值为$\frac{\rho}{2}\omega^{2}\cos^{2}{[\omega(t-\frac{x}{u})]}$\\
C. 能量密度包含动能项和势能项,其值为$\rho\omega^{2}\sin^{2}{[\omega(t-\frac{x}{u})]}$\\
D. 能量密度包含动能项和势能项,其值为$\rho\omega^{2}\cos^{2}{[\omega(t-\frac{x}{u})]}$\\

\item 体积固定的容器内有一定量的气体,若温度升高,则分子的平均碰撞频率$\underline{~~~~~~~~~~}$,分子的平均自由程$\underline{~~~~~~~~~~}$.\\
A. 不变,减小 $\quad$
B. 不变,不变 $\quad$
C. 增大,减小 $\quad$
D. 增大,不变 $\quad$

\item 如图所示,一个构成闭合电路的导电细线圈被扭成“8”字型(交叉处不接触),线圈的1部分比2部分面积大.将其放置在随时间变强的均匀磁场中,磁场方向与线圈所在的纸平面垂直并指向纸外,则线圈的2部分中电流是的方向为\\
A. 顺时针 $\quad$
B. 逆时针 $\quad$
C. 无电流 $\quad$
D. 不能确定 $\quad$

\item 真空中,一半径为$R$的球体均匀带电.在球内挖去一半径为$r$的小球,且使球体其余部分带电分布不变.设小球空腔球心处为$O$点.如果以无穷远处为电势零点,则在小球被挖去前后,关于$O$点的电场强度和电势变化,下列说法正确的是\\
A. 电场强度和电势都不发生改变 \\
B. 电场强度和电势都发生改变 \\
C. 电场强度不发生改变,电势发生改变 \\
D. 电场强度发生改变,电势不发生改变 \\
\item 钠原子处于基态时将束缚在最内层的一个电子移动到无穷远处的能量,与下面哪个能量量级最接近?\\
A. $\quad$
B. 逆时针 $\quad$
C. 无电流 $\quad$
D. 不能确定 $\quad$
\end{enumerate}