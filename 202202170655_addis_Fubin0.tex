% 重积分的换序、Fubini 定理(简明微积分)

\pentry{重积分\upref{IntN}}

\footnote{参考 Wikipedia 相关页面\href{https://en.wikipedia.org/wiki/Fubini's_theorem}{相关页面}.}以矩形区域的二重定积分为例, 什么时候可以交换积分的顺序呢? 即
\begin{equation}
\int_{y_1}^{y_2}\int_{x_1}^{x_2} f(x, y) \dd{x} \dd{y} = \int_{x_1}^{x_2}\int_{y_1}^{y_2} f(x, y) \dd{y}\dd{x}
\end{equation}
乍看之下, 二重积分是曲面下面的体积, 无论先算哪个都不会变. 但我们来看一个例子

\begin{example}{}\label{Fubin0_ex1}
\begin{equation}\label{Fubin0_eq1}
f(x, y) = \frac{x - y}{(x+y)^3}
\end{equation}
\begin{figure}[ht]
\centering
\includegraphics[width=8cm]{./figures/Fubin0_1.png}
\caption{\autoref{Fubin0_eq1} 函数曲面图} \label{Fubin0_fig1}
\end{figure}
该函数在 $(0,0)$ 处有一个奇点, 即没有定义. 但是 $x \in (0, 1]$, $y \in (0, 1]$ 的重积分仍然收敛
\begin{equation}
\int_0^1 \int_0^1 f(x,y) \dd{x} \dd{y} = -\frac{1}{2}
\qquad
\int_0^1 \int_0^1 f(x,y) \dd{y} \dd{x} = \frac{1}{2}
\end{equation}
注意交换顺序以后发现结果竟然不同!

观察\autoref{Fubin0_fig1} 会发现, 函数曲面关于积分区域的正方形的对角线 $x=y$ 反对称, 且在 $x$ 轴和 $y$ 轴靠近原点的地方分别出现了无穷大和无穷小. 所以 “曲面下的体积” 是否能取有限值(收敛), 取决于无穷大和无穷小是否能在积分过程中抵消.

为了更好理解, 我们不妨延对正方形的两个对角线方向分别进行积分, 即令 $u = (x+y)/\sqrt{2}$, $v=(x-y)/\sqrt{2}$. 那么令
\begin{equation}
g(u,v) = f(x(u,v), y(u,v)) = \frac{v}{2u^3}
\end{equation}
该函数表示同一个形状的曲面, 我们可以把这看作是把原来的直角坐标系绕 $z$ 轴旋转\upref{Rot2D}了 45°. 所以如果先对 $v$ 积分再对 $u$ 积分, 根据函数曲面的反对称性可以断定对 $v$ 的积分为恒零(对于 $u\ne 0$), 那么再把 $u$ 从大到小积分, 那么结果也还是零. 所以现在我们得到了第三个积分结果——零. 而如果先对 $u$ 积分再对 $v$ 积分, 那么这就相当于我们先计算 $x=y$ 分割出的一个三角形的体积(负无穷), 再计算另一个三角形的体积(正无穷). 这就类似于做定积分 $\int_{-\infty}^{+\infty}1/v\dd{v}$, 你可以认为它不收敛, 但在某种定义下, 其结果也可以是零(链接未完成: 主值积分).

所以现在我们看到曲面下方的体积完全取决于我们采用哪两个方向积分. 如果积分的过程中刚好可以把正无穷和负无穷实时抵消, 那么积分结果就是有限值, 且值的大小取决于具体抵消的方式. 如果上面的坐标旋转使用其他角度, 还会得到许多其他不同的值.
\end{example}

\begin{theorem}{}
对多重黎曼积分, 如果被积函数是绝对值可积的, 那么积分结果与积分变量的次序无关.
\end{theorem}

套用在\autoref{Fubin0_ex1} 中, 我们需要看看 $\abs{x-y}/(x+y)^3$ 的积分是否是有限的. 容易验证积分结果是 $+\infty$, 所以并不能保证换序积分结果不变. 过程留做习题.
