% 宇称算符
% keys 宇称|奇宇称|偶宇称|厄米算符

\pentry{多元函数积分和宇称\upref{IntPry}}

对多元函数 $f(\bvec x) = f(x_1, \dots, x_N)$.  定义宇称算符 $\Pi$ 如下
\begin{equation}
\Pi f(\bvec x) = f(-\bvec x)
\end{equation}
若函数内积\upref{InerPd}定义为(星号表示复共轭)
\begin{equation}
\braket{f}{g} = \int f^*(\bvec x) g(\bvec x) \dd[N]{x}
\end{equation}
容易证明宇称算符是一个厄米算符.% 链接未完成
对本征方程
\begin{equation}
\Pi f(\bvec x) = \lambda f(\bvec x)
\end{equation}
容易证明\footnote{分别令 $\bvec x$ 为某对称的两点 $\bvec x_1$ 和 $-\bvec x_1$, 本征方程要求 $f(-\bvec x_1) = \lambda f(\bvec x_1)$ 且 $f(\bvec x_1) = \lambda f(-\bvec x_1)$, 所以 $\lambda^2 = 1$, $\lambda = \pm 1$.}宇称算符的本征值 $\lambda$ 只可能等于 $1$ 或 $-1$.
