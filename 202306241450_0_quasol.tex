% 二次方程求根公式

\subsection{求根公式}
解决二次方程最常规的方法是运用求根公式。
对于二次方程
$$ax^2+bx+c=0 \qquad a\neq 0~,$$
定义判别式
$$\Delta = b^2-4ac~,$$
那么方程的两个根分别为
$$
\begin{aligned}
x_1&=\frac{-b+\sqrt{\Delta}}{2a}~,\\
x_2&=\frac{-b-\sqrt{\Delta}}{2a}~.
\end{aligned}
$$

可见,当
$$
\begin{aligned}
\Delta &> 0 \Rightarrow \text{方程有两个不同的根}\\
\Delta &= 0 \Rightarrow \text{方程有两个相同的根}\\
\Delta &< 0 \Rightarrow \text{方程无实数根(但有两个共轭的复数根)}\\
\end{aligned}~
$$

\subsubsection{几何含义}
假设$f(x)=ax^2+bx+c$,那么求解方程$ax^2+bx+c=0$即化为寻找函数$f(x)$的所有零点$f(x)=0$。
\begin{figure}[ht]
\centering
\includegraphics[width=14cm]{./figures/652b2f4efe76656f.pdf}
\caption{$f(x)$示意图。从左到右为$\Delta > 0, \Delta = 0, \Delta < 0$} \label{fig_quasol_1}
\end{figure}
可见,$x=-\frac{b}{2a}$ 为函数$f(x)$的对称轴,两个零点(如果存在)关于该轴对称。

\subsection{配方法}
对于一些特定的问题,可以将方程配方并求解,有时这比直接使用求根公式更为简便。

例如,可以将方程配方为如下形式:
$$(x-a)(x-b)=0\Rightarrow x_1=a, x_2=b~.$$
\begin{figure}[ht]
\centering
\includegraphics[width=5cm]{./figures/fe527c666ebfd775.pdf}
\caption{$f(x)=(x-a)(x-b)$示意图} \label{fig_quasol_2}
\end{figure}

或者
$$(x-a)^2=b\Rightarrow x_1=\sqrt{b}+a, x_2=-\sqrt{b}+a~.$$