% 格奥尔格·欧姆(综述)
% license CCBYSA3
% type Wiki

本文根据 CC-BY-SA 协议转载翻译自维基百科\href{https://en.wikipedia.org/wiki/Georg_Ohm}{相关文章}。

\begin{figure}[ht]
\centering
\includegraphics[width=6cm]{./figures/33a59a3a2faf0fbf.png}
\caption{乔治·西蒙·欧姆  1789年3月16日  埃尔朗根,勃兰登堡-拜罗伊特(现为德国)} \label{fig_GOM_1}
\end{figure}

乔治·西蒙·欧姆(Georg Simon Ohm,/oʊm/;德语:[ˈɡeːɔʁk ˈʔoːm];1789年3月16日 – 1854年7月6日)是德国物理学家和数学家。作为一名学校教师,欧姆开始研究由意大利科学家亚历山德罗·伏打发明的新的电化学电池。通过使用他自己制作的设备,欧姆发现导体两端施加的电位差(电压)与产生的电流之间存在直接的正比关系。这个关系被称为欧姆定律,而电阻的国际单位“欧姆”(Ω)也以他的名字命名。

\subsubsection{早年生活}  
乔治·西蒙·欧姆出生于一个新教家庭,地点是埃尔朗根,勃兰登堡-拜罗伊特(当时属于神圣罗马帝国)。他是锁匠约翰·沃尔夫冈·欧姆和埃尔朗根裁缝的女儿玛丽亚·伊丽莎白·贝克的儿子。虽然他的父母并未接受正式教育,但欧姆的父亲是一位受人尊敬的人,他通过自学达到了较高的学识水平,并能够通过自己的教导为儿子们提供优良的教育。[4] 在家族的七个孩子中,只有三人活到了成年:乔治·西蒙、他的弟弟马丁(后来成为一位著名数学家)和他的妹妹伊丽莎白·巴巴拉。他的母亲在他十岁时去世。

从小,乔治和马丁就由父亲亲自教育,父亲将他们培养到较高的数学、物理、化学和哲学水平。乔治·西蒙从十一岁到十五岁就读于埃尔朗根中学,但在学校的科学训练上并未得到很多指导,这与他和马丁在父亲那里接受的启发性教育形成了鲜明对比。这一特点使欧姆家族与伯努利家族相似,正如埃尔朗根大学的教授卡尔·克里斯蒂安·冯·朗斯多夫所指出的。