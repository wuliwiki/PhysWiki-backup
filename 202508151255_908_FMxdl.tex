% 费马小定理(综述)
% license CCBYNCSA3
% type Wiki

本文根据 CC-BY-SA 协议转载翻译自维基百科\href{https://en.wikipedia.org/wiki/Fermat\%27s_little_theorem}{相关文章}。

在数论中,费马小定理指出:如果 $p$ 是一个质数,那么对于任意整数 $a$,数 $a^p - a$ 都是 $p$ 的整数倍。用模运算的记号表示,就是:
$$
a^p \equiv a \pmod{p}~
$$
例如,当 $a = 2$、$p = 7$ 时,有 $2^7 = 128$,而 $128 - 2 = 126 = 7 \times 18$,正好是 7 的倍数。

如果 $a$ 不能被 $p$ 整除,也就是说 $a$ 与 $p$ 互素,那么费马小定理等价于以下陈述:$a^{p-1} - 1$ 是 $p$ 的整数倍,或者用符号表示为:
$$
a^{p-1} \equiv 1 \pmod{p}~
$$
例如,当 $a = 2$、$p = 7$ 时,有 $2^6 = 64$,而 $64 - 1 = 63 = 7 \times 9$,也是 7 的倍数。

费马小定理是费马素性检验的理论基础,也是初等数论中的一个基本定理。该定理得名于皮埃尔·德·费马,他在 1640 年提出了这一结论。之所以称为“小定理”,是为了将其与费马大定理区分开来。
\subsection{历史}
皮埃尔·德·费马(Pierre de Fermat)最早是在一封写于 **1640 年 10 月 18 日的信中向他的朋友兼知己弗雷尼克尔·德·贝西(Frénicle de Bessy)陈述了这一定理。他的表述等价于如下内容:

如果 $p$ 是质数,而 $a$ 是任意一个不能被 $p$ 整除的整数,那么 $a^{p-1} - 1$ 一定能被 $p$ 整除。

费马的原始陈述是:

Tout nombre premier mesure infailliblement une des puissances −1 de quelque progression que ce soit, et l'exposant de la dite puissance est sous-multiple du nombre premier donné −1; et, après qu'on a trouvé la première puissance qui satisfait à la question, toutes celles dont les exposants sont multiples de l'exposant de la première satisfont tout de même à la question.

这段话翻译成中文,并加入一些注释和公式以便理解,可以表述为:

每一个质数 $[p]$ 必定整除某一个“某个等比级数 $[a, a^2, a^3, \dots]$”中的“某个幂减一”的结果(即存在某个 $t$,使得 $a^t - 1$ 可被 $p$ 整除),且这个幂的指数 $t$ 是 $p - 1$ 的一个约数。找到第一个满足条件的幂之后,所有指数是这个指数 $t$ 的倍数的幂,同样也满足这一性质。

也就是说,如果某个 $t$ 满足 $a^t \equiv 1 \pmod{p}$,那么所有 $t$ 的倍数也都满足 $a^{kt} \equiv 1 \pmod{p}$。

费马并没有讨论当 $a$ 是 $p$ 的倍数时的情形,也并没有给出他的证明,只是写道:

Et cette proposition est généralement vraie en toutes progressions et en tous nombres premiers; de quoi je vous envoierois la démonstration, si je n'appréhendois d'être trop long.

翻译为中文是:

“这个命题在所有的数列和所有的质数中都是普遍成立的;我本可以把它的证明寄给你,只是担心写得太长了。”\(^\text{[5]}\)

欧拉在 1736 年发表的论文中首次给出了费马小定理的正式证明。这篇论文题为 《Theorematum Quorundam ad Numeros Primos Spectantium Demonstratio》(英文译名为:“关于素数的一些定理的证明”),发表于圣彼得堡科学院的期刊中\(^\text{[6][7]}\)。但实际上,莱布尼茨早在 1683 年之前的一份未发表手稿中,就几乎给出了相同的证明\(^\text{[3]}\)。

“费马小定理”这一术语最早可能出现在1913 年库尔特·亨泽尔所著的《Zahlentheorie》(数论)一书中\(^\text{[8]}\):

Für jede endliche Gruppe besteht nun ein Fundamentalsatz, welcher der kleine Fermatsche Satz genannt zu werden pflegt, weil ein ganz spezieller Teil desselben zuerst von Fermat bewiesen worden ist.

翻译为中文:

“在每一个有限群中都存在一个基本定理,这通常被称为‘费马小定理’,因为其中的一个非常特殊的部分最早是由费马证明的。”

在英语中较早使用该术语的例子出现在 A.A. 阿尔伯特于 1937 年出版的《现代高等代数》一书中,第 206 页提到:

“所谓的‘小’费马定理”\(^\text{[9]}\)。
\subsubsection{后续发展历史}
一些数学家曾独立提出了一个相关猜想(有时被误称为“中国猜想”):当且仅当 $p$ 是质数时,有 $2^p \equiv 2 \pmod{p}$。实际上,“如果”部分是正确的,它是费马小定理的一个特例;然而,“只有当”这部分是错误的。

例如:
$$
2^{341} \equiv 2 \pmod{341}~
$$
但 $341 = 11 \times 31$ 是合数,它是 以 2 为底的伪素数。详见下文。
\subsection{证明}
费马小定理有多种不同的证明方法。其中一种常见的做法是将它作为欧拉定理的推论来证明。
\subsection{推广}
欧拉定理是费马小定理的推广形式:对于任意模数 $n$ 和任意与 $n$ 互素的整数 $a$,都有
$$
a^{\varphi(n)} \equiv 1 \pmod{n}~
$$
其中 $\varphi(n)$ 表示欧拉函数,即从 1 到 $n$ 中与 $n$ 互素的整数的个数。费马小定理实际上就是欧拉定理的一个特例,因为如果 $n$ 是质数,则 $\varphi(n) = n - 1$。

欧拉定理的一个推论是:对于任意正整数 $n$,如果整数 $a$ 与 $n$ 互素,那么对于任意整数 $x$ 和 $y$,有
$$
x \equiv y \pmod{\varphi(n)} \quad \Rightarrow \quad a^x \equiv a^y \pmod{n}.~
$$
这可以由欧拉定理直接推出:如果
$$
x \equiv y \pmod{\varphi(n)},~
$$
那么存在整数 $k$ 使 $x = y + k\varphi(n)$,于是
$$
a^x = a^{\,y + \varphi(n)k} = a^y \cdot \left(a^{\varphi(n)}\right)^k \equiv a^y \cdot 1^k \equiv a^y \pmod{n}.~
$$
如果 $n$ 是质数,这也是费马小定理的一个推论。这一性质在模运算中应用广泛,因为它允许将指数很大的模幂运算化简为指数小于 $n$ 的情形,从而大大简化计算。
