% 赫尔曼·外尔(综述)
% license CCBYSA3
% type Wiki

本文根据 CC-BY-SA 协议转载翻译自维基百科\href{https://en.wikipedia.org/wiki/Hermann_Weyl#Weyl_equation}{相关文章}。

\begin{figure}[ht]
\centering
\includegraphics[width=6cm]{./figures/b6bfe45a42d757ab.png}
\caption{赫尔曼·克劳斯·雨果·外尔  1885年11月9日  德国帝国,埃尔姆斯霍恩} \label{fig_WR_1}
\end{figure}
赫尔曼·克劳斯·雨果·外尔(Hermann Klaus Hugo Weyl,ForMemRS,德语发音:[vaɪl];1885年11月9日-1955年12月8日)是一位德国数学家、理论物理学家、逻辑学家和哲学家。尽管他的大部分职业生涯是在瑞士苏黎世和美国新泽西州普林斯顿度过的,他依然被认为是哥廷根大学数学传统的一部分,该传统由卡尔·弗里德里希·高斯、大卫·希尔伯特和赫尔曼·闵可夫斯基代表。

外尔的研究在理论物理学和纯数学领域(例如数论)具有重要意义。他是20世纪最具影响力的数学家之一,也是早期普林斯顿高等研究院的重要成员。[4][5]

外尔在空间、时间、物质、哲学、逻辑、对称性以及数学史等领域作出了非凡的贡献。他是最早设想将广义相对论与电磁学定律结合起来的人之一。弗里曼·戴森(Freeman Dyson)曾写道,外尔是唯一一个可以与19世纪“最后的伟大通才数学家”亨利·庞加莱和大卫·希尔伯特相提并论的人。迈克尔·阿蒂亚(Michael Atiyah)特别指出,每当他研究一个数学主题时,总会发现外尔早已涉足其中。[7]

\subsection{传记}
赫尔曼·外尔出生于德国汉堡附近的小镇埃尔姆斯霍恩,曾就读于阿尔托纳的克里斯蒂亚内乌姆文理中学(Gymnasium Christianeum)。[8] 他的父亲路德维希·外尔(Ludwig Weyl)是一名银行家,母亲安娜·外尔(Anna Weyl,娘家姓Dieck)则来自一个富裕家庭。[9]

1904年至1908年间,外尔在哥廷根大学和慕尼黑大学学习数学和物理。他在哥廷根大学获得博士学位,导师是他非常敬仰的大卫·希尔伯特。

1913年9月,外尔在哥廷根与弗里德里克·贝尔塔·海伦·约瑟夫(Friederike Bertha Helene Joseph,1893年3月30日-1948年9月5日)结婚,她的昵称是“海拉”(Hella)。海伦是布鲁诺·约瑟夫博士(Dr. Bruno Joseph,1861年12月13日-1934年6月10日)的女儿,后者是一名医生,在德国里布尼茨-达姆加滕(Ribnitz-Damgarten)担任卫生官职(Sanitätsrat)。海伦是一位哲学家(现象学家埃德蒙·胡塞尔的弟子),也是西班牙文学作品的翻译家,尤其将西班牙哲学家何塞·奥尔特加·伊·加塞特的作品译成德文和英文。[12] 正是通过海伦与胡塞尔的密切联系,赫尔曼得以熟悉并深受胡塞尔思想的影响。

赫尔曼与海伦有两个儿子:弗里茨·约阿希姆·外尔(Fritz Joachim Weyl,1915年2月19日-1977年7月20日)和迈克尔·外尔(Michael Weyl,1917年9月15日-2011年3月19日),两人均出生于瑞士苏黎世。[13] 海伦于1948年9月5日在新泽西州普林斯顿去世。同年9月9日,普林斯顿为她举行了纪念仪式,致辞者包括她的儿子弗里茨·约阿希姆·外尔,以及数学家奥斯瓦尔德·维布伦(Oswald Veblen)和理查德·柯朗特(Richard Courant)。[14]

1950年,赫尔曼与雕塑家埃伦·贝尔(Ellen Bär,娘家姓Lohnstein,1902年4月17日-1988年7月14日)结婚。埃伦是苏黎世教授理查德·约瑟夫·贝尔(Richard Josef Bär,1892年9月11日-1940年12月15日)的遗孀。[15][16]

在哥廷根担任教职数年后,外尔于1913年前往苏黎世,担任苏黎世联邦理工学院(ETH Zürich)的数学教授[17],在那里他与阿尔伯特·爱因斯坦成为同事。当时爱因斯坦正在完善广义相对论的细节。爱因斯坦对外尔产生了深远的影响,使他对数学物理产生了浓厚的兴趣。1921年,外尔结识了理论物理学家埃尔温·薛定谔,后者当时是苏黎世大学的教授。他们后来成为亲密的朋友。外尔曾与薛定谔的妻子安妮玛丽(Annemarie,昵称Anny,娘家姓Bertel)发生过某种形式的无子女恋情,而同时安妮玛丽正在抚养薛定谔与他人所生的非婚生女儿露丝·乔吉·埃里卡·马奇(Ruth Georgie Erica March),露丝于1934年出生在英国牛津。[18][19]

外尔在1928年国际数学家大会(ICM)上作为主旨发言人于意大利博洛尼亚发表演讲,[20] 并在1936年于奥斯陆召开的国际数学家大会上担任特邀发言人。他于1928年被选为美国物理学会会士,[21] 1929年成为美国艺术与科学院院士,[22] 1935年加入美国哲学会,[23] 1940年成为美国国家科学院院士。[24] 在1928-1929学年,他作为访问教授在普林斯顿大学任教,[25] 并与霍华德·P·罗伯逊(Howard P. Robertson)合作撰写了一篇题为《关于无限小几何基础中群论问题》的论文。[26]

1930年,外尔离开苏黎世,接替大卫·希尔伯特在哥廷根的职位。然而,随着1933年纳粹上台,他因妻子是犹太人而离开德国。他曾被新成立的美国新泽西州普林斯顿高等研究院邀请担任教职,但因不愿离开祖国而拒绝。随着德国政治局势恶化,他改变了主意,接受了再次提供的职位,并在高等研究院工作至1951年退休。他与第二任妻子埃伦(Ellen)共同度过了在普林斯顿和苏黎世的时光。1955年12月8日,外尔在苏黎世因心脏病发作去世。

外尔于1955年12月12日在苏黎世火化。[27] 他的骨灰一直保存在私人手中,直到1999年,之后被安葬在普林斯顿公墓的一个户外骨灰存放墙中。[28] 外尔的儿子迈克尔·外尔(1917–2011)的遗体被安葬在同一骨灰存放墙中,与外尔的骨灰相邻。

外尔是一位泛神论者。[29]
\subsection{贡献}
\subsubsection{特征值的分布 } 
\begin{figure}[ht]
\centering
\includegraphics[width=6cm]{./figures/9d4106519803831f.png}
\caption{赫尔曼·魏尔(左)和恩斯特·佩施尔(右)} \label{fig_WR_2}
\end{figure}
1911年,魏尔发表了《关于特征值的渐近分布》(*Über die asymptotische Verteilung der Eigenwerte*),在其中他证明了拉普拉斯算子在紧致域中的特征值遵循所谓的魏尔定律的分布。1912年,他提出了一种基于变分原理的新证明方法。魏尔多次回到这一主题,研究了弹性系统并提出了魏尔猜想。这些研究开启了现代分析的一个重要领域——特征值的渐近分布。
\subsubsection{流形与物理的几何基础}
1913年,魏尔发表了《黎曼曲面的概念》(*Die Idee der Riemannschen Fläche*),对黎曼曲面进行了统一的研究。在这部著作中,魏尔利用点集拓扑的方法,使黎曼曲面理论更加严谨,这种方法后来成为流形研究的模型。他还吸收了L.E.J.布劳威尔早期的拓扑学研究成果,用于这一目的。  

作为哥廷根学派的重要人物,魏尔从一开始就完全掌握了爱因斯坦的研究工作。他在1918年出版的《空间、时间、物质》(*Raum, Zeit, Materie*)中跟踪了相对论物理的发展,该书在1922年出版了第四版。同样在1918年,他引入了规范(*gauge*)的概念,并首次提出了现在称为规范理论的例子。魏尔的规范理论是试图将电磁场和引力场建模为时空几何性质的一次不成功的尝试。在黎曼几何中,魏尔张量对于理解共形几何的本质具有重要意义。  

魏尔在物理学中的总体方法基于埃德蒙·胡塞尔的现象学哲学,特别是胡塞尔1913年的著作《纯粹现象学与现象学哲学的观念:第一卷:纯粹现象学的一般导论》(*Ideen zu einer reinen Phänomenologie und phänomenologischen Philosophie. Erstes Buch: Allgemeine Einführung in die reine Phänomenologie*)。胡塞尔强烈回应了戈特洛布·弗雷格对其第一部关于算术哲学的批评,并探讨了数学和其他结构的意义,这些结构被弗雷格区分为经验参照之外的内容。
\subsubsection{拓扑群、李群与表示理论}
从1923年到1938年,魏尔发展了紧致群的理论,采用矩阵表示的方法。在紧致李群的情况下,他证明了一个基本的特征公式。  

这些成果奠定了理解量子力学对称结构的基础,他将其建立在群论的框架上,其中包括旋量的研究。这与约翰·冯·诺依曼在很大程度上推动的量子力学数学公式化一起,形成了自1930年左右以来广为人知的处理方式。非紧致群及其表示,尤其是海森堡群,也在这一具体背景下得到了简化,他在1927年的魏尔量子化中提出了现存最好的经典物理与量子物理之间的桥梁。从那时起,得益于魏尔的阐述,李群和李代数不仅成为纯数学的主流领域,也成为理论物理的重要组成部分。  

他的著作《经典群》(*The Classical Groups*)重新审视了不变量理论。书中探讨了对称群、一般线性群、正交群和辛群,以及它们的不变量和表示理论的相关成果。
\subsubsection{调和分析与解析数论} 
魏尔还展示了如何在丢番图逼近中使用指数和,并提出了模 1 均匀分布的判据,这是解析数论中的一个基础性步骤。这一研究成果被应用于黎曼 ζ 函数以及加法数论,并被许多后续学者进一步发展。
\subsubsection{数学基础}
在《连续统》(*The Continuum*)中,魏尔通过采用伯特兰·罗素分级类型理论的低级别,发展了预测分析的逻辑。他在不使用选择公理、排中律的证明以及避免乔治·康托尔的无限集的情况下,成功构建了大部分经典微积分。在这一时期,魏尔借鉴了德国浪漫主义主观唯心主义哲学家费希特的激进构造主义思想。  

在发表《连续统》后不久,魏尔短暂地完全转向布劳威尔的直觉主义。在《连续统》中,构造性的点被视为离散实体。然而,魏尔希望构建一种不是点的集合的连续统。他撰写了一篇引发争议的文章,为自己和布劳威尔宣布了一场“革命”。[30] 这篇文章在传播直觉主义观点方面比布劳威尔的原始著作本身影响更大。  

1918年2月9日,在苏黎世的一次数学家聚会上,乔治·波利亚和魏尔打了一个赌,讨论数学未来的发展方向。魏尔预测,在接下来的20年中,数学家们会意识到诸如实数、集合和可数性的概念完全模糊不清,而且关于实数的上确界性质的真假问题,就像问黑格尔关于自然哲学基本断言的真实性一样毫无意义。[31] 他认为,任何对这些问题的回答都无法验证,与经验无关,因此是无意义的。  

然而,在几年内,魏尔认为布劳威尔的直觉主义对数学施加的限制过于严格,这正是批评者一直以来的观点。《危机》一文曾引起魏尔的形式主义导师希尔伯特的不安,但到了1920年代后期,魏尔部分地将自己的立场与希尔伯特的立场调和起来。  

到1928年左右,魏尔似乎决定,数学直觉主义与他对胡塞尔现象学哲学的热情不相容,尽管他早先曾认为两者可以共存。在他生命的最后几十年,魏尔强调数学是一种“符号建构”,并转向了更接近于希尔伯特和恩斯特·卡西尔的立场。然而,魏尔很少提到卡西尔,仅在一些简短的文章和段落中表达了这一观点。  

到1949年,魏尔对直觉主义的最终价值深感失望。他写道:  
“在布劳威尔的体系中,数学获得了最高的直观清晰性。他成功地以一种自然的方式发展了分析的起步阶段,并且比以往更加紧密地保持了与直观的联系。然而,不可否认的是,在向更高、更一般的理论推进时,经典逻辑简单法则的不可适用性最终导致了几乎难以忍受的笨拙局面。数学家痛苦地看着自己以为是用混凝土砖块建造的宏伟建筑大部分在他眼前化为云雾。”  

正如约翰·L·贝尔所说:  
“我认为非常遗憾的是,魏尔没有看到20世纪70年代平滑无穷小分析的出现,这是一种数学框架,在其中实现了他对真实连续统的愿景,这种连续统不是由离散元素‘合成’而成。尽管平滑无穷小分析的基础逻辑是直觉主义的——即排中律通常不能被肯定——但在这一框架下发展的数学避免了魏尔所提到的‘难以忍受的笨拙’。”  
\subsubsection{魏尔方程} 
1929年,魏尔提出了一种方程,被称为魏尔方程,用于取代狄拉克方程。该方程描述了无质量费米子。一个普通的狄拉克费米子可以分解为两个魏尔费米子,或者由两个魏尔费米子组成。中微子曾一度被认为是魏尔费米子,但现已证明它们具有质量。魏尔费米子在电子学应用中备受关注。2015年,行为类似于魏尔费米子的准粒子在一种称为魏尔半金属的晶体中被发现,这是一种拓扑材料。[32][33][34]
\subsubsection{哲学}
魏尔自青年时期便对哲学产生兴趣,他阅读了伊曼努尔·康德的《纯粹理性批判》,其中将空间和时间视为知识的先验概念(尽管后来他不喜欢康德与欧几里得几何过于紧密的联系)。从1912年起,他深受埃德蒙·胡塞尔及其现象学的影响,这种影响也体现在他著作《空间、时间、物质》的某些段落中。1927年,他为《哲学手册》撰写的文章《数学与自然科学哲学》由奥尔登堡出版社出版,后来被单独出版并修订为一本书。  

在试图重建魏尔哲学的起源并将其整合到哲学主流中时,诺曼·谢罗卡(Norman Sieroka)指出,魏尔与其苏黎世的哲学同事弗里茨·梅迪库斯(Fritz Medicus)之间进行了深入且长期的讨论。梅迪库斯是约翰·戈特利布·费希特的专家。根据费希特的《科学学说》(\textbf{Wissenschaftslehre})和哲学,“存在”源于“绝对自我”与其物质邻域(\textbf{Umgebung})的相互作用。这一思想对魏尔产生了深远影响,并反映在他对拓扑学的邻域概念(连续统)的运用中,以及在他对广义相对论的构想中,同时还受到魏尔直接著作中提到的胡塞尔现象学的影响。  

根据谢罗卡的观点,魏尔还从戈特弗里德·威廉·莱布尼茨的物质理论(如单子论)和德国唯心主义(如费希特的辩证法)中汲取了影响,这些思想融入了魏尔对物理学中物质概念的哲学解释中,尤其是在量子理论和广义相对论的背景下。此外,这些影响还体现在数学哲学中,即符号与其环境在数学理论结构中的相互作用(如布劳威尔影响下关于形式主义与直觉主义的辩论)。他将数学内部关于直觉主义和形式主义的辩论理解为胡塞尔现象学和费希特构造主义之间的对话。  

在量子力学发展之前的1920年代,受到量子理论统计特性的启发(这一特性当时正变得日益清晰),魏尔从场论的物质描述转向了一种包含空间环境的“主动物质”(\textbf{agens})理论。此前,他曾使用微分几何方法描述广义相对论及其自身的扩展,这些研究催生了现代规范场理论的概念。然而,在量子理论的影响下,他逐渐放弃了这种“几何场论”。  

根据谢罗卡,费希特和恩斯特·卡西尔(Ernst Cassirer)也是魏尔晚期哲学的重要影响(科学作为“符号建构”)。魏尔与马丁·海德格尔的互动较少为人所知。尽管魏尔不同意海德格尔关于死亡的观点,但其“邻域”(\textbf{Umgebung})的概念受到了海德格尔存在主义的影响。
\subsection{名言}  
\begin{itemize}
\item 数学的终极基础和终极意义的问题仍然是开放的;我们不知道它将朝哪个方向找到最终的解决方案,甚至不知道是否可以期待一个最终的客观答案。“数学化”可能是人类的一种创造性活动,就像语言或音乐一样,具有原初的独创性,其历史性决定难以完全客观地理性化。  
——《全集》(\textbf{Gesammelte Abhandlungen}),引用自《美国哲学会年鉴》(\textbf{Year Book – The American Philosophical Society}),1943年,第392页  
\item 在这些日子里,拓扑的天使和抽象代数的恶魔正在为每个数学领域的灵魂而战。  
——魏尔(1939b,第500页)  
\item 每当你处理一个赋有结构的实体 \( S \) 时,试着确定它的自同构群,即那些逐一保持所有结构关系不变的变换群。通过这种方式,你可以期望深入了解 \( S \) 的构成。  
——《对称性》(\textbf{Symmetry}),普林斯顿大学出版社,第144页;1952年  
\item 超越个别科学所得的知识,仍然存在理解的任务。尽管哲学的观点从一个体系摇摆到另一个体系,我们仍无法放弃它,否则我们将把知识变成毫无意义的混乱。  
——《空间-时间-物质》(\textbf{Space-Time-Matter}),第4版(1922),英文译本,Dover出版社(1952年),第10页;魏尔加粗部分的重点。
\end{itemize}  
\subsection{参考文献}
\begin{itemize}
\item 1911. 《关于特征值的渐近分布》(Über die asymptotische Verteilung der Eigenwerte),《哥廷根皇家科学协会通讯》(Nachrichten der Königlichen Gesellschaft der Wissenschaften zu Göttingen),第110–117页(1911年)。  
\item 1913. 《黎曼曲面的概念》(Die Idee der Riemannschen Fläche),第2版,1955年。《黎曼曲面的概念》(The Concept of a Riemann Surface),艾迪生-韦斯利出版社(Addison–Wesley)。  
\item 1918. 《连续统》(Das Kontinuum),译文1987年,《连续统:对分析基础的批判性研究》(The Continuum: A Critical Examination of the Foundation of Analysis),ISBN 0-486-67982-9。  
\item 1918. 《空间、时间、物质》(Raum, Zeit, Materie),5个版本至1922年版,由于尔根·埃勒斯(Jürgen Ehlers)注释,1980年。英译第4版,亨利·布罗斯(Henry Brose)译,1922年,《空间、时间、物质》(Space-Time-Matter),梅休恩出版社(Methuen),1952年由Dover出版社重印,ISBN 0-486-60267-2。  
\item 1923. 《空间问题的数学分析》(Mathematische Analyse des Raumproblems)。  
\item 1924. 《什么是物质?》(Was ist Materie?)  
\item 1925. (1988年由K. Chandrasekharan编辑出版)《黎曼的几何思想》(Riemann's Geometrische Idee)。  
\item 1927. 《数学与自然科学哲学》(Philosophie der Mathematik und Naturwissenschaft),第2版1949年。英译本《数学与自然科学哲学》(Philosophy of Mathematics and Natural Science),普林斯顿出版社(Princeton),ISBN 0689702078。2009年由弗兰克·维尔切克(Frank Wilczek)撰写新序,普林斯顿大学出版社,ISBN 978-0-691-14120-6。  
\item 1928. 《群论与量子力学》(Gruppentheorie und Quantenmechanik),由H. P. Robertson翻译为《群论与量子力学的理论》(The Theory of Groups and Quantum Mechanics),1931年,1950年由Dover出版社重印,ISBN 0-486-60269-9。  

\end{itemize}