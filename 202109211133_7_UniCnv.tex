% 一致收敛
% keys 数学分析|一致收敛|极限

\begin{issues}
\issueDraft
\end{issues}

\addTODO{需添加预备知识,收敛或逐点收敛.}

函数列$f_n$逐点收敛到$f$,只要求对于任意固定的$x_0$,数列$\{f_n(x_0)\}$收敛到$f(x_0)$即可.接下来介绍的\textbf{一致收敛}则是一个更强的要求.

一个函数列 $f_n(x)$ 一致收敛到 $f(x)$ 的定义是: 当对于任意 $\epsilon > 0$, 存在 $N$, 当 $n \geqslant N$ 时对任意 $x$ 都有
\begin{equation}
\abs{f_n(x) - f(x)} < \epsilon
\end{equation}
或者记为
\begin{equation}
\lim_{n\to\infty} \qty(\max \abs{f_n(x) - f(x)}) = 0
\end{equation}

一致收敛是比(逐点)收敛更强的条件.

\begin{example}{逐点收敛但不一致收敛的例子}\label{UniCnv_ex1}
在区间$(0, 1)$上考虑函数列$f_n(x)=x^n$和函数$f(x)=0$,则容易验证$f_n$逐点收敛到$f$.

但是,对于任意$\epsilon>0$和任意$f_n$,总可以找到$x\in(\epsilon^{1/n}, 1)$,使得$f_n(x)>\epsilon$.因此按照定义,$f_n$\textbf{不}一致收敛到$f$.

\end{example}

\begin{example}{逐点收敛但不一致收敛的例子}\label{UniCnv_ex2}
在区间$[0, 1]$上考虑函数列$f_n(x)=\sin(\pi\cdot x^n)$和函数$f(x)=0$,则容易验证$f_n$逐点收敛到$f$.

但是,对于任意$\epsilon>0$和任意$f_n$,总可以找到$x=\frac{1}{2^n}$,使得$f_n(x)=1>\epsilon$.因此按照定义,$f_n$\textbf{不}一致收敛到$f$.

\end{example}

逐点收敛只考虑,是不是每个点都收敛.而一致收敛的威力在于,有一个统一的进度,每个点收敛的进度不得比这更慢.\autoref{UniCnv_ex1} 和\autoref{UniCnv_ex2} 里举出的反例就是如此,不管你怎么取$\epsilon$作为限定,总存在跟不上进度的点,导致这种限定不是\textbf{一致}的.

一致收敛还可以从“函数之间的距离”角度来考虑.











