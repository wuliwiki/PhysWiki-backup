% 欧姆定律(综述)
% license CCBYSA3
% type Wiki

本文根据 CC-BY-SA 协议转载翻译自维基百科\href{https://en.wikipedia.org/wiki/Ohm\%27s_law}{相关文章}。

欧姆定律指出,通过导体的电流与其两端的电压成正比。引入比例常数“电阻”后,[1] 可以得到描述这种关系的三个数学公式:[2]  
\[
V = IR \quad \text{或} \quad I = \frac{V}{R} \quad \text{或} \quad R = \frac{V}{I}~
\]
其中,\(I\) 是通过导体的电流,\(V\) 是测量的导体两端的电压,\(R\) 是导体的电阻。更具体地说,欧姆定律指出,这个关系中的 \(R\) 是常数,独立于电流。[3] 如果电阻不是常数,那么之前的公式就不能称为欧姆定律,但它仍然可以作为静态/直流电阻的定义。[4] 欧姆定律是一个经验关系,准确描述了绝大多数电导材料在多个数量级的电流下的导电性。然而,一些材料不遵守欧姆定律,这些材料被称为非欧姆材料。
