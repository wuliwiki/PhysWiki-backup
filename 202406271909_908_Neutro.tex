% 中子
% license CCBYSA3
% type Wiki

(本文根据 CC-BY-SA 协议转载自原搜狗科学百科对英文维基百科的翻译)

\textbf{中子}是亚原子粒子,符号为$n$或者n⁰,净电荷为零且质量略大于质子。质子和中子构成了原子的核子。因为质子和中子在原子核内的行为相似,每个质子和中子的质量大约为1原子质量单位,它们统称为核子。[1]它们的性质和相互作用由核物理描述。

原子核的化学和核性质由质子数以及中子数决定,分别称为原子序数和中子数。原子质量数是核子的总数。比如,碳原子序数为6,并且常见的碳-12有6个中子,而罕见的碳-13有7个中子。有些元素在自然界中只有一种稳定同位素,例如氟。有些元素则有许多稳定的同位素,比如锡有十种稳定同位素。

在原子核内,质子和中子通过核力结合在一起。除了单质子氢原子之外,原子核的稳定需要中子。中子在核裂变和聚变中大量产生。它们通过裂变、聚变和中子俘获过程参与在恒星内化学元素的核合成。

中子对核能生产至关重要。在詹姆斯·查德威克1932年发现中子后的十年里,中子被用来诱导许多不同类型的核嬗变。随着1938年核子裂变的发现,[2]人们很快认识到,裂变事件产生的每一个中子都可能引起进一步的裂变事件,这种级联反应被称为核链式反应。这些发现导致了第一个自持的核反应堆 ( 芝加哥一号堆,1942年)和第一个核武器 ( 三一,1945年)的产生。

自由中子虽然不是电离原子,但会导致致电离辐射。因此,当剂量较大时,它们具备生物性危害。地球上存在有小的自然的自由中子,主要来源于宇宙线射线浴以及地壳内自发裂变元素产生的辐射。用于辐射和中子散射实验的自由中子来自于专用的中子源,如中子发生器、研究堆和散裂源。

\subsection{描述}
原子核由许多质子形成,$Z$(原子序数)和中子数,$N$(中子数),由核力结合在一起。原子序数定义原子的化学性质,中子数决定同位素或核素。[3]同位素和核素这两个术语经常被当作同义词使用,但是它们分别指化学性质和核性质。严格来说,同位素是两种或多种质子数相同的核素;具有相同中子数的核素被称为同中子异荷素。原子质量数,符号$A$,等于$Z+N$。具有相同原子质量数的核素称为同量异位素。氢原子最常见的同位素(带有化学符号${}^{1}h)$),其核子是一个单独的质子。重氢同位素的原子核$(D\text{或}{}^{2}H)$和氚$(T\text{或}{}^{3}H)$分别包含一个和两个中子。其他所有类型的原子核都由两个或多个质子和不同数量的中子组成。化学元素铅中最常见的核素,${}^{208}Pb)$有82个质子和126个中子。核素表包括所有已知的核素。尽管中子不是化学元素,但它包含在这个表中。[3]

自由中子的质量为939,565,$413.3 eV /c^{2}$,或$1.674927471\times10^{-27}$ kg,或1.00866491588 u。中子的均方半径约为$0.8\times10^{-15}$ m,或0.8 fm ,[4]这是自旋为1的费米子。[5]质子带正电荷,直接受电场影响,而中子不带电荷,故不受电场影响。然而,中子具有磁矩,因此受磁场影响。中子的磁矩是负的,其取向与其自旋方向相反。[6]

自由中子是不稳定的,其衰变产物为质子,电子和反中微子,平均寿命不到15分钟(881.5±1.5 s)。这种被称为β衰变的放射性衰变是可能的,因为中子的质量略大于质子且自由质子是稳定的。然而,结合在原子核中的中子或质子可以是稳定的,也可以是不稳定的,这取决于核素。在$\beta$衰变,中子衰变为质子,反之亦然,受弱力支配,它需要发射或吸收电子和中微子,或它们的反粒子。
\begin{figure}[ht]
\centering
\includegraphics[width=6cm]{./figures/a27c6c16b4c60c50.png}
\caption{铀-235吸收中子引起的核裂变。重核素分裂成更轻的成分和额外的中子。} \label{fig_Neutro_1}
\end{figure}
质子和中子在核子核力的影响下表现几乎相同。同位旋的概念视质子和中子为同一粒子的两个量子态,用于模拟核子在强力和弱子下的相互作用。由于核力在短距离内的强度,核子的结合能比原子中束缚电子的电磁能大7个数量级。因此核反应(例如核子裂变)产生的能量密度是化学反应的1000多万倍。根据质能公式,核结合能降低了原子核的质量。因此,核子内部电磁排斥所产生的能量是核能的主要来源,这也使得核反应堆和核子炸弹成为可能。在核裂变中,中子被重核素吸收(例如,铀-235)导致核素变得不稳定,并分裂成轻核素和额外的中子。带正电荷的轻核素然后排斥,释放电磁势能。

中子被归类为强子,因为它是由夸克组成的复合粒子。同时它也被分类为重子,因为它由三个价夸克组成[7]。中子的大小及其磁矩表明中子是一种复合粒子,而不是基本粒子。一个中子包含两个带电荷$-\frac{1}{3}e$的下夸克和一个带电荷$+\frac{2}{3}e$的上夸克。

像质子一样,中子的夸克通过由胶子做介质的强力结合在一起。[8]核力源于更基本的强力的次要影响。

\subsection{发现}
中子及其性质的发现是20世纪上半页原子物理学非凡发展的核心,最终导致了1945年的原子弹。在1911年的卢瑟福模型中,原子由一个小的带正电的原子核及其周围大得多的带负电荷的电子云组成。1920年,卢瑟福提出,原子核由带正电的质子和不带电的粒子组成,被认为是以某种方式束缚的质子和电子。[9]电子被认为在原子核内,因为已知的β辐射可以从原子核发射出电子。[9]卢瑟福称这些不带电荷的粒子为中子,由拉丁语词根表示中立(中性)和希腊语后缀-打开(亚原子粒子名称中使用的后缀,即电子和质子)组成。[10][11]然而,在1899年的文献中就发现了对这个词中子的引用。[12]

在整个20世纪20年代,物理学家假设原子核是由质子和“核电子”组成[13][14]。但是这个假设有个明显的问题,那就是很难调和原子核的质子-电子模型与量子力学的海森堡不确定关系。[15]由奥斯卡·克莱因于1928年发现的[16][17]克莱因悖论,从量子力学层面上进一步反对电子限制在原子核内的假设。[16]观察到的原子和分子的自旋跟质子-电子假设预期的合自旋不一致。质子和电子的自旋均为 ½ ħ。相同元素的同位素(具有相同质子数的元素)具有整数自旋或分数自旋,例如中子自旋必须是分数。然而,没有可能让电子和质子的合自旋(应该结合形成中子)来获得中子的分数自旋。

1931年,瓦尔特·博特和赫伯特·贝克尔发现,来自钋的α粒子辐射到铍、硼或锂后,就会产生具备异常穿透性的辐射。该辐射不受电场的影响,所以两人都认为是伽马辐射。[18][19]次年,巴黎的伊雷娜·约里奥-居里和弗雷德里克·约里奥-居里表明,这种“伽马”辐射落在石蜡或任何其他含氢的化合物上,它会射出能量非常高的质子。[20]卢瑟福和詹姆斯·查德威克在卡文迪许实验室在剑桥被伽马射线的解释所说服。[21]查德威克很快进行了一系列实验,表明新的辐射由不带电荷的粒子组成,这些粒子的质量大约与质子相同。[22][22][23]这些粒子是中子。查德威克因这一发现获得了1935年的诺贝尔物理学奖。

维尔纳·海森堡和其他人很快发展了由质子和中子组成的原子核模型[24][25][26]。[27][28]质子-中子模型解释了核自旋的难题。$\beta$辐射的起源可以解释为恩利克·费密1934年由$\beta$衰变过程中子通过衰变变成质子,一个电子和一个(尚未发现的)中微子。[29]1935年,查德威克和他的博士生莫里斯·戈德哈伯报告了中子质量的第一次精确测量。[30][31]

1934年,费米用中子轰击了较重的元素,以在高原子序数元素中诱导放射性。1938年,费米获得了诺贝尔物理学奖“因为他证明了中子辐照产生的新放射性元素的存在,也因为他发现了慢中子引起的核反应“。[32]1938年奥托·哈恩,莉泽·迈特纳和弗里茨·施特拉斯曼发现核子裂变,或中子轰击导致铀核分馏成轻元素。[33][34][35]1945年,哈恩获得了1944年的诺贝尔化学奖 "因为他发现了重原子核的裂变。"[36][37][38]核裂变的发现导致了第二次世界大战结束时核能和原子弹的发展。
\begin{figure}[ht]
\centering
\includegraphics[width=10cm]{./figures/72376c3d96f99be8.png}
\caption{氢、氦、锂和氖原子中原子核和电子能级的模型。实际上,原子的直径大约是原子核直径的10万倍。} \label{fig_Neutro_2}
\end{figure}

\subsection{$\beta$衰变与原子核的稳定性}
\begin{figure}[ht]
\centering
\includegraphics[width=6cm]{./figures/3b2f9e698de6f000.png}
\caption{中子经由中间重 W玻色子β衰变为质子、电子和电子反中微子的费曼图} \label{fig_Neutro_3}
\end{figure}
在粒子物理学的标准模型下,中子保持重子数守恒的的唯一可能衰变模式是中子的一个夸克通过弱相互作用变性。中子的一个下夸克通过发射一个 W玻色子衰变为更轻的上夸克。通过这个过程,中子衰变为质子(包含一个下夸克和两个上夸克),一个电子和一个电子反中微子。此即为标准的β衰变的过程。

因为质子之间的电磁排斥远大于使得它们吸引的核相互作用,中子是任何包含一个以上质子的原子核的必要组成部分(参见双质子和中子质子比)。[39]中子通过核力与质子在原子核中相互结合,有效地调节质子之间的排斥力并使得原子核稳定。








