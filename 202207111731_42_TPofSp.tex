% 空间的张量积
% 空间的张量积

\begin{issues}
\issueDraft
\end{issues}

\pentry{张量积\upref{TsrPrd}}
矢量空间的张量积在微分几何,群表示论,数学物理中有各式各样的作用.
\begin{definition}{空间的张量积}
设 $V_1,V_2,W$ 是域 $\mathbb F$ 上的矢量空间,$\sigma:V_1\times V_2\rightarrow W$ 是由 $V_1,V_2$ 是个双线性映射.若对域 $\mathbb F$ 上任意矢量空间 $U$ ,任一双线性映射 $\varphi:V_1\times V_2\rightarrow U$, 都存在唯一的线性变换 $\psi:W\rightarrow U$ 使
\begin{equation}
\varphi=\psi\sigma
\end{equation}
那么 $W$ 就称为 $V_1$ 与 $V_2$ 的\textbf{张量积}.
\end{definition}
\begin{lemma}{}
设 $V_1,V_2$ 是域 $\mathbb F$ 上的矢量空间,$V_1$ 的维数是 $n$,$e_1,\cdots,e_n$ 是 $V_1$ 的一组基.对 $V_1$ 中任意 $n$ 个元素 $\alpha_1,\cdots,\alpha_n$ ,存在唯一的线性映射 $f:V_1\rightarrow V_2$ 使
\begin{equation}
f(e_i)=\alpha_i,\quad i=1,\cdots,n
\end{equation}
\end{lemma}
\textbf{证明:}若有另一线性映射 $f_1$,使得
\begin{equation}
f_1(e_i)=\alpha_i,\quad i=1,\cdots ,n
\end{equation}
那么对任意 $v=\sum_\limits{i}v^ie_i\in V_1$,有
\begin{equation}
f_1(v)=\sum_{i}v^if_1(e_i)=\sum_{i}v^i\alpha_i=\sum_{i}v^if(e_i)=f(v)
\end{equation}
即 $f_1=f$.

\textbf{证毕!}
\begin{lemma}{}\label{TPofSp_lem1}
设 $V_1,V_2$ 分别是 $n$ 维和 $m$ 维的矢量空间, $\{\mu^i\}$,$\{\nu^i\}$ 分别是 $V_1^*,V_2^*$ 的基,那么
\begin{equation}\label{TPofSp_eq1}
\mu^i\otimes\nu^j,\quad i=1,\cdots,n,\quad j=1,\cdots,m
\end{equation}
 是 $\mathcal L(V_1,V_2;\mathbb F)$ \upref{MulMap}的一组基.
\end{lemma}
\textbf{证明:}先证明 \autoref{TPofSp_eq1} 之间线性无关:若
\begin{equation}
\sum_{i,j}\lambda_{ij}\mu^i\otimes\nu^j=0
\end{equation}
右边的 $0$ 是矢量空间 $\mathcal L(V_1,V_2;\mathbb F)$
中的零元(注意,由张量积定义 $\mu^i\otimes\nu^j\in \mathcal L(V_1,V_2;\mathbb F)$ ).于是
\begin{equation}
\lambda_{kl}=\qty(\sum_{i,j}\lambda_{ij}\mu^i\otimes\nu^j)(\mu_k,\nu_l)=0(\mu_k,\nu_l)=0
\end{equation}
其中,最后的 $0$ 是域 $\mathbb F$ 的零元,$\{\mu_k\},\{\nu_l\}$ 分别是 $V_1,V_2$ 的与 $\{\mu^i\}$,$\{\nu^i\}$ 对偶的基.所以得到\autoref{TPofSp_eq1} 的线性无关性.

又 $\mathcal L(V_1,V_2;\mathbb F)$ 的维数为 $nm$,由基的定义(\autoref{VecSpn_def2}~\upref{VecSpn}),所以\autoref{TPofSp_eq1} 就是它的一组基.

\textbf{证毕!}

由\autoref{TPofSp_lem1} ,容易证得下面的推论
\begin{corollary}{}\label{TPofSp_cor1}
映射
\begin{equation}
\sigma(f,g):=f\otimes g,\quad f\in V_1^*,g\in V_2^*
\end{equation}
是空间 $\mathcal L(V_1,V_2;\mathbb F)$ 中的一个矢量,即 $V_1^*,V_2^*$ 上的一双线性型.
\end{corollary}
注意:上述的引理和推论,将矢量空间和其对偶空间互调仍然成立,因为彼此是对方的对偶空间(对偶空间的对称性).

\begin{theorem}{}
设 $V_1,V_2$ 是域 $\mathbb F$ 上有限维矢量空间,则矢量空间 $V_1,V_2$ 的张量积存在,且在同构的意义下是唯一的.
\end{theorem}
\textbf{证明:}由\autoref{TPofSp_cor1} ,映射 
\begin{equation}
\sigma:V_1\times V_2\rightarrow\mathcal L(V_1^*,V_2^*;\mathbb F) 
\end{equation}
是  $\mathcal L(V_1^*,V_2^*;\mathbb F)$ 中的矢量.

而由 \autoref{TPofSp_lem1} ,若 $\{\mu_i\},\{\nu_i\}$ 分别是 $V_1,V_2$ 的基,$n=\dim V_1,m=\dim V_2$ .那么
\begin{equation}
\sigma(\mu_i,\nu_j)=\mu_i\otimes \nu_j,\quad i=1,\cdots,n,\quad j=1,\cdots,m
\end{equation}
是 $\mathcal L(V_1^*,V_2^*;\mathbb F)$ 的一组基.

现在证明,$\mathcal{V_1^*,V_2^*}$ 就是 $V_1,V_2$ 的张量积.设 $U$ 是域 $\mathbb F$ 上任一线性空间,
\begin{equation}
\varphi:V_1\times V_2\ot
\end{equation}

