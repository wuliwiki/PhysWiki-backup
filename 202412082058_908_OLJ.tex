% 欧拉角(综述)
% license CCBYSA3
% type Wiki

本文根据 CC-BY-SA 协议转载翻译自维基百科\href{https://en.wikipedia.org/wiki/Euler_angles}{相关文章}。
\begin{figure}[ht]
\centering
\includegraphics[width=8cm]{./figures/f7361f2b5f088c22.png}
\caption{} \label{fig_OLJ_1}
\end{figure}
欧拉角是由莱昂哈德·欧拉提出的三个角度,用于描述刚体相对于固定坐标系的方向。[1]

它们也可以表示物理学中运动参考系的方向,或三维线性代数中一般基的方向。

经典欧拉角通常采用倾斜角度的方式,其中零度表示垂直方向。后来,由彼得·古思里·泰特(Peter Guthrie Tait)和乔治·H·布赖恩(George H. Bryan)提出了替代形式,主要用于航空学和工程学中,其中零度表示水平位置。
\subsection{链式旋转等价性}
欧拉角可以通过元素几何或旋转组合(即链式旋转)来定义。几何定义表明,三个元素旋转(绕坐标系的轴旋转)总是足够将物体定向到任何目标参考系。

这三个元素旋转可以是外在旋转(绕原始坐标系xyz轴旋转,假设坐标系保持静止),也可以是内在旋转(绕旋转坐标系XYZ轴旋转,该坐标系与运动体固连,在每次元素旋转后,物体相对于外部参考系的方向会发生变化)。

在下面的各节中,带有撇号标记的轴(例如,z″)表示元素旋转后的新轴。

欧拉角通常用 α、β、γ 或 ψ、θ、φ 来表示。不同的作者可能会使用不同的旋转轴集来定义欧拉角,或者使用不同的名称来表示相同的角度。因此,任何涉及欧拉角的讨论都应该首先明确它们的定义。

在不考虑使用两种不同约定来定义旋转轴(内在或外在)的情况下,旋转轴有十二种可能的旋转顺序,可以分为两组:

\begin{itemize}
\item \textbf{正确的欧拉角}(\(z-x-z,x-y-x,y-z-y,z-y-z,x-z-x,y-x-y\))
\item \textbf{泰特-布赖恩角}(x-y-z,y-z-x,z-x-y,x-z-y,z-y-x,y-x-z)。
\end{itemize}

泰特-布赖恩角也被称为卡尔丹角、航海角、航向、仰角和倾斜角,或偏航、俯仰和滚转角。有时,这两类旋转顺序都被称为“欧拉角”。在这种情况下,第一组旋转顺序被称为正确的或经典的欧拉角。