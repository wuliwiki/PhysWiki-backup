% 金属的变形(科普)

\subsection{变形}
正如我们拉伸一根弹簧,弹簧会变形一样;当我们拉伸一根金属棒时,金属棒也会变形.只不过由于金属棒的“弹性系数”很大,以正常人的手劲一般拉不出看得见的变形.

\begin{example}{}
\begin{figure}[ht]
\centering
\includegraphics[width=12cm]{./figures/MetDfm_1.png}
\caption{框架结构}} \label{MetDfm_fig1}
\end{figure}
与弹簧类似,金属结构提供的支持力也源自金属的细微变形...只要在安全的范围内.
\end{example}

根据变形的性质,变形一般分为两类:弹性变形与塑形变形.顾名思义,弹性变形后,撤去外力后金属的形状能恢复原样;而塑形变形后,即使撤去外力,金属的形状也不能恢复.塑形形变只在外力大到超过一定境界时才发生.

弹性变形与塑性变形不是非此即彼,而可以相辅相成.一次变形可能既包括弹性形变也包括塑形形变.

\begin{figure}[ht]
\centering
\includegraphics[width=10cm]{./figures/MetDfm_2.png}
\caption{弹性变形与塑性变形示意图} \label{MetDfm_fig2}
\end{figure}

那么,为什么会有两种类型的变形呢?这就涉及到变形的微观原理了.大体而言,弹性变形时原子间的“键”被拉伸,但原子并没有运动到新的位置,因而撤去外力后原子可以回到原位,体现为形状恢复原样;

而塑形变形后,原子间原本的键已经被破坏、原子运动到了新的位置,并形成了新的键.因此,塑性变形后金属的形状发生永久改变.

\subsection{塑性变形机制}
\pentry{金属材料结构(科普)\upref{MetInt}}
接下来,我们更细致地探讨一下塑性变形.此处简要探讨\textsl{滑移},这是塑性形变的主要机制之一.本文以单晶体为例,即金属中只有一个硕大的晶粒,原子的排列方向都一致.

\subsubsection{位错的运动}
或许你还记得位错\upref{MetInt}的概念.金属的滑移变形与位错有着密不可分的关系;位错理论的提出正是为了解释金属的塑性变形.

假设你有一块完整的晶体,现在你要施加外力使其塑性变形.看起来,为了使原子运动到新位置,你得先大力出奇迹、破坏一整面的键.毫无疑问这需要非常大的能量(\textsl{与手劲})!

可事实上,现实中的金属强度远低于此(大概是按照这种理论计算的1/100至1/1000).怎么回事呢?这时,假定金属中有一个位错;当位错存在时,奇迹发生了:由于位错的存在,现在上下部分相对运动时,只需断一列的键而不是一面的键.这大大降低了原子运动的难度.当位错运动离开晶体时,就会在晶体边缘留下一个台阶.
\begin{figure}[ht]
\centering
\includegraphics[width=14cm]{./figures/MetDfm_3.png}
\caption{位错的运动1} \label{MetDfm_fig3}
\end{figure}

这有点像蠕动的毛毛虫,或者地板上鼓起包的地毯.
\begin{figure}[ht]
\centering
\includegraphics[width=14cm]{./figures/MetDfm_5.png}
\caption{位错的运动2}} \label{MetDfm_fig5}
\end{figure}

\begin{figure}[ht]
\centering
\includegraphics[width=8cm]{./figures/MetDfm_4.png}
\caption{位错的运动3} \label{MetDfm_fig4}
\end{figure}

在\href{https://5b0988e595225.cdn.sohucs.com/q_70,c_zoom,w_640/images/20191101/b54fbd26dc14497d8607965e6b395c96.gif}{这里}可以看到一张动图,\href{https://m.sohu.com/a/350972084_120056486/}{原文}(站外链接)

看起来,位错大大降低了滑移的难度!事实上,位错运动是滑移机制的主要因素.

\subsubsection{位错运动的方向:滑移系}
那么,位错能够在任意的面或方向上如此运动吗?答案显然又是...否定的.\textbf{一般而言,位错有偏好的运动方向与运动面,位错也只在这些方向与面上运动}.

\begin{theorem}{滑移系,滑移面,滑移方向}
滑移沿着特定的面与方向进行.

这些特定的面与方向分别称为滑移面与滑移方向;一个滑移面与其上的滑移方向称为一个滑移系.

\end{theorem}
\begin{figure}[ht]
\centering
\includegraphics[width=6cm]{./figures/MetDfm_6.png}
\caption{一种可能的滑移系示意图,切面为滑移面,黑色线为滑移方向} \label{MetDfm_fig6}
\end{figure}

这些滑移面一般是晶体的密排面,而滑移方向一般是密排方向.(密排面可以理解为晶体中原子堆积得最密的面)因此,滑移系与晶胞的种类有关.

这也是为什么金属变形时\footnote{需要教科书式的单晶体才能观察到如此教科书式的结论}往往沿着特定方向形成台阶状结构.这些台阶状结构(图中所示“滑移带”)就是一系列位错滑移后留下的小台阶之和.
\begin{figure}[ht]
\centering
\includegraphics[width=14cm]{./figures/MetDfm_7.png}
\caption{金属的宏观塑性变形} \label{MetDfm_fig7}
\end{figure}

\subsubsection{位错运动的条件:临界分切应力}
即使位错大幅降低了滑移的难度,也得有足够大的外力才能驱动位错运动;更准确的说,是\textbf{位错滑移方向上的力}得足够大.满足这一条件的最小分力被称为临界分切应力.

\begin{definition}{临界分切应力}
临界分切应力指晶体恰好开始滑移时,滑移方向上的分切应力.
\end{definition}

\begin{figure}[ht]
\centering
\includegraphics[width=12cm]{./figures/MetDfm_8.png}
\caption{外力的分解.注意只有滑移方向上的分力是“有效”的} \label{MetDfm_fig8}
\end{figure}
\footnote{$\lambda$与$\phi$之和不一定为$90^\circ$,他们是相对独立的变量:)}
如图所示,外力F在滑移方向上的分力\footnote{在材料科学中,往往更关注\textsl{应力},即“一点处的力的大小”,类似于压强的概念,因此要除以力作用面的面积.正应力(垂直于截面)常记为$\sigma=\dv{F}{A}$,切应力(平行于截面)记为$\tau=\dv{F}{A}$.对于均匀拉伸,可以认为拉力在截面是均匀分布的,因此$\sigma=\frac{F}{A}$}为

$$\tau=\frac{F \cos \lambda}{A/{\cos \phi}}=\frac{F}{A}{\cos \lambda}{\cos \phi}$$
若要达到临界切分应力为$\tau_k$,那么外力至少为 $F_k=\frac{\tau_k A}{\cos \lambda \cos \phi} $ .

其中,$(\cos \lambda \cos \phi)$被称为取向因子,与外力的作用方向、晶体中原子的排列方式、位错运动的滑移系等有关.可见,外力的最小大小与取向因子紧密关联,这强烈明示了晶体材料的\textbf{各向异性}.\footnote{由于现实材料多为多晶材料,各向异性性质被削弱.}

\begin{figure}[ht]
\centering
\includegraphics[width=4cm]{./figures/MetDfm_9.png}
\caption{晶体的各向异性示意图} \label{MetDfm_fig9}
\end{figure}
