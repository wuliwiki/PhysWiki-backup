% 笛卡尔积
% keys 直积|有序数对|笛卡尔积
% license Usr
% type Tutor

在刚刚接触到乘法计算时,作为一个运算结果,“积”是作为与“乘法”相关的概念被引入的。后来,随着对向量的学习的深入,内积和外积逐渐也成为了熟悉的概念,二者分别与点乘($\cdot$)和叉乘($\times$)相对应。或许,“卷积”和“张量积”等概念也偶尔会出现在你的视野中。他们往往是与一个逐渐抽象的“乘法”相对应,说他逐渐抽象,是因为他与我们熟知的数的乘法的样子和计算方法相去甚远。而还称呼它是乘法,是因为某种程度上,它保留了乘法的一些特性。

下面会涉及到一点点的物理知识:在物理上,常常会有通过“乘法”来定义一个新的物理量,比如:功是力与位移的乘积,力矩是力与力臂的乘积,电路中功率是电流与电压的乘积(先忽略这个乘积具体的形式)等。这个新的物理量与原有的两个物理量之间都存在关系。而“加法”往往是在一个概念内部量的多少的计算,基本是不涉及其他概念的。

数学上有一个概念叫作\textbf{直积}(direct product),一般使用它来组合两个同类的已知对象,来定义新对象,比如:集合、群、模、拓扑空间等。而作用在两个集合上的直积便称为\textbf{笛卡尔积}(Cartesian product)。下面会先直接

\subsection{有序对}

在刚开始接触集合这个概念的时候,一定会知道,集合有这样一个性质——“集合内的元素的具有无序性”。而现实生活中,“序”这个概念又是必不可少的。有序对(ordered pair)的出现就是为了能够有概念来表示两个元素顺序。因此,定义时就希望这个概念能够满足两个性质:

\begin{enumerate}
\item 唯一性:每一个有序对是唯一定义的。
\item 顺序性:有序对中的元素顺序是固定的,即$(a, b) \neq (b, a)$,除非$a = b$。
\end{enumerate}

这样,这个“序”的概念才能够稳定。

\begin{definition}{有序对}
有序对(a, b)是一种用于表示两个元素关系且满足唯一性和顺序性的数据结构。
\end{definition}



最常见的定义方法是通过库拉托夫斯基对(Kuratoswki pair):

$(a, b) = \{\{a\}, \{a, b\}\}$





有序数对在很多数学领域中有广泛的应用:

- **关系和函数**:一个二元关系 \(R\) 可以看作是有序对的集合。例如,函数 \(f : A \to B\) 可以看作是满足 \(f(a) = b\) 的有序对 \((a, b)\) 的集合。
- **笛卡尔积**:集合 \(A\) 和 \(B\) 的笛卡尔积 \(A \times B\) 是所有有序对 \((a, b)\) 的集合,其中 \(a \in A\) 且 \(b \in B\)。
- **拓扑学**:在拓扑学中,点对的定义和处理也是通过有序对来实现的。

\subsection{笛卡尔积}

\textbf{笛卡尔积}是一个集合领域的概念,经常通过对。

