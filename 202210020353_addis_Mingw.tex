% Mingw-w64 和 MSYS2 笔记

\begin{itemize}
\item Mingw 仅支持 32-bit 程序, 现在一般用 \href{https://en.wikipedia.org/wiki/Mingw-w64}{Mingw-w64}, 既支持 32 也支持 64-bit
\item Mingw-W64 \href{https://www.mingw-w64.org/}{官网}.
\item 一个\href{https://www3.ntu.edu.sg/home/ehchua/programming/howto/Cygwin_HowTo.html}{教程}.
\item \href{https://www.msys2.org/}{MSYS2} 是一个 windows 上的 bash shell 调用 MinGW-W64 以及 CygWin.
\item 双击 \verb|mingw64.exe| 即可打开 mingw 命令行. \verb|mingw64.exe| 所在的目录就是命令行的根目录.
\item 可以检查 \verb|g++| 的版本.
\item 用 \verb|g++| 编译程序以后, 会出现 \verb|a.exe| 而不是 \verb|a.out|. 这个执行文件是可以双击执行的, 但是运行完会马上退出. 可以在程序最后用 \verb|getchar()|. 也可以打开一个 cmd 命令行然后运行 \verb|a.exe|. 当然在 \verb|mingw64| 的命令行也可以执行. 另外执行时 \verb|.exe| 拓展名可以省略.
\item 不能使用 apt, 一切库都要自己手动编译. 或者用 MinGW instellation 安装 32-bit 版本.
\item \verb|c| 盘的目录为 \verb|/c|
\item 如果创建 symlink 会直接复制而不是真的 link
\item \verb|MinGW instellation manager| 是一个 GUI 界面可以下载安装编译好的 mingw32 程序. 例如安装 \verb|mingw32-make|, 安装以后就和 \verb|make| 的用法一样.
\item \verb|g++| 中定义的宏有 \verb|__GNUC__|, \verb|__MINGW32__|, \verb|__MINGW64__|
\item 编译动态链接库\upref{gppLib}的方法和 linux 中一样, 同样支持用 \verb|ldd| 查看依赖关系
\begin{lstlisting}[language=bash]
g++ -shared -o lib0.dll lib0.o
g++ -shared -o lib1.dll lib1.o -L . -l 0
g++ -c main.cpp
g++ -o main.x main.o -l1 -L./ -Wl,-rpath,./
\end{lstlisting}
\end{itemize}

\subsection{MSYS2 笔记}
\begin{itemize}
\item \href{https://www.msys2.org/}{官网}, \href{https://www.msys2.org/docs/what-is-msys2/}{msys 和 WSL, cygwin 等的对比}
\item MSYS 让 linux 开发者可以开发在 Windows 上运行的软件. WSL 制作的软件不能不安装 WSL 直接在 Windows 上运行.
\item MSYS2 使用 Pacman\upref{pacman} 进行包管理, 和 Arch linux 一样.
\item 所有可用的包在\href{https://packages.msys2.org/queue}{这里}查找. 名字一般和 \verb|apt| 命令中的不一样.
\item 根目录在 \verb|C:\msys64\|, 和 mingw 一样可以用 \verb|/c| 访问 C 盘.
\item 用 \verb|g++| 编译时, 宏 \verb|__MSYS__| 有定义, 而 \verb|__MINGW32__| 和 \verb|__MINGW64__| 没定义.
\item \verb|pacman -S base-devel|
\end{itemize}
