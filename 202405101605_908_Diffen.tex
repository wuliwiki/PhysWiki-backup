% 有限差分
% keys 差分公式|导数|差分
% license Xiao
% type Tutor
\pentry{高阶导数(简明微积分)\nref{nod_HigDer}}{nod_14f8}
在实际当中,由于计算机的内存总是有限的,不可能存储一个连续的数据(因为实数的稠密性告诉我们:任意两实数之间有无穷多的数),甚至不可能存储一个小数点后有无限位数的实数。所以在计算机进行求导计算时自变量 $x$ 的变化 $\Delta x$ 不可能取到无限小,即微分 $\dd f({x})$ 不可能实现,转而只能取一有限大小的 $\Delta x$,与微分相对应,此时 $\Delta f(x)$ 称为\textbf{差分}。
\begin{definition}{差分}
设函数 $f(x)$ 定义在某区间 $\mathcal{X}$ 上,给定自变量 $x$ 任一固定增量 $\Delta x$。称
\begin{equation}
\Delta f(x)=f(x+\Delta x)-f(x)~,
\end{equation}
为函数 $f(x)$ 的\textbf{一阶差分}。显然一阶差分也是 $x$ 的函数(因为 $\Delta x$ 固定),一阶差分的差分称为\textbf{二阶差分}。\textbf{$n$ 阶差分}可归纳定义为
\begin{equation}
\Delta^n f(x)=\Delta[\Delta^{n-1}f(x)]~.
\end{equation}
\end{definition}
\begin{theorem}{差分公式}\label{the_Diffen_1}
$n$ 阶差分 $\Delta^n f(x)$ 具有如下公式
\begin{equation}\label{eq_Diffen_1}
\Delta^nf(x)=\sum_{i=0}^n(-1)^iC_n^if(x+(n-i)\Delta x)~.
\end{equation}
其中,$C_n^i$ 为\enref{组合数}{HsCb}。
\end{theorem}
\autoref{eq_Diffen_1} 表明,$n$ 阶差分可直接用函数 $f(x)$ 本身在等距分点
\begin{equation}
x,x+\Delta x,\cdots,x+n\Delta x~
\end{equation}
表示出。
\begin{theorem}{差分与导数关系}\label{the_Diffen_2}
设函数 $f(x)$ 在闭区间 $[x_0,x_0+n\Delta x]$ 上有直到 $n-1$ 阶连续导数,且至少在开区间 $(x_0,x_0+n\Delta x)$ 上有直到 $n$ 阶有限导数 $f^{(n)}(x)$ 。于是成立下面公式
\begin{equation}\label{eq_Diffen_2}
\Delta^nf(x_0)=f^{(n)}(\xi_n)\Delta x^n ,\quad x_0<\xi_n<x_0+n\Delta x~.
\end{equation}
\end{theorem}
若在点 $x_0$ 处导数 $f^{(n)}(x)$ 存在且连续,则让 $\Delta x\rightarrow0$ (此时 $\xi_n\rightarrow x_0$),得
\begin{equation}
f^{(n)}(x_0)=\lim_{\Delta x\rightarrow0}\frac{\Delta^n f(x_0)}{\Delta x^n}~,
\end{equation}
 这个公式给出了用一次极限步骤求得 $n$ 阶导数的可能性。
\subsection{证明}
\subsubsection{\autoref{the_Diffen_1} 的证明}
当 $n=1,2$ 时,定理显然成立:
\begin{equation}
\begin{aligned}
\Delta f(x)=&f(x+\Delta x)-f(x)\\
=&\sum_{i=0}^1(-1)^iC_1^if(x+(1-i)\Delta x)~,\\
\Delta^2 f(x)=&\Delta (\Delta f(x))=\Delta f(x+\Delta x)-\Delta f(x)\\
=&f(x+2\Delta x)-2f(x+\Delta x)+f(x)\\
=&\sum_{i=0}^2(-1)^iC_2^if(x+(2-i)\Delta x)~.
\end{aligned}
\end{equation}
假设 $n=k$ 时,\autoref{eq_Diffen_1} 成立,即
\begin{equation}
\Delta^kf(x)=\sum_{i=0}^k(-1)^iC_k^if(x+(k-i)\Delta x)~.
\end{equation}
那么
\begin{equation}
\begin{aligned}
\Delta^{k+1} f(x)=&\Delta(\Delta^k f(x))=\Delta\qty(\sum_{i=0}^k(-1)^iC_k^if(x+(k-i)\Delta x))\\
=&\sum_{i=0}^k(-1)^iC_k^i\Delta f(x+(k-i)\Delta x)\\
=&\sum_{i=0}^k(-1)^iC_k^i\qty[f(x+(k+1-i)\Delta x)-f(x+(k-i)\Delta x)]\\
=&f(x+(k+1)\Delta x)+\sum_{i=1}^{k}(-1)^iC_k^if(x+(k+1-i)\Delta x)\\
&+\sum_{i=0}^{k-1}(-1)^{i+1}C_k^i f(x+(k-i)\Delta x)+(-1)^{k+1}f(x)\\
=&f(x+(k+1)\Delta x)+\sum_{i=1}^{k}(-1)^{i}(C_k^{i}+C_k^{i-1}) f(x+(k+1-i)\Delta x)+(-1)^{k+1}f(x)\\
=&f(x+(k+1)\Delta x)+\sum_{i=1}^{k}(-1)^{i}C_{k+1}^{i} f(x+(k+1-i)\Delta x)+(-1)^{k+1}f(x)\\
=&\sum_{i=0}^{k+1}(-1)^iC_{k+1}^if(x+(k+1-i)\Delta x)~.
\end{aligned}
\end{equation}
第三个等式用到了 $\Delta x$ 固定这一事实,$C_k^i+C_k^{i-1}=C_{k+1}^i$ 见\autoref{eq_HsCb_5}~\upref{HsCb}。

\textbf{证毕!}
\subsubsection{\autoref{the_Diffen_2} 证明}
当 $n=1$ 时,\autoref{eq_Diffen_2} 便是拉格朗日公式\autoref{eq_MeanTh_2}~\upref{MeanTh}。

假设 $n$ 为 $n-1$ 时,\autoref{eq_Diffen_2} 成立。则
\begin{equation}
\Delta^n f(x_0)=\Delta(\Delta^{n-1}f(x_0))=\qty[f^{n-1}(\xi_{n-1}+\Delta x)-f^{n-1}(\xi_{n-1})]\Delta x^{n-1}~.
\end{equation}
其中,$x_0<\xi_{n-1}<x_0+(n-1)\Delta x$。上式右边应用拉格朗日公式\autoref{eq_MeanTh_2}~\upref{MeanTh},便得\autoref{eq_Diffen_2} ,且
\begin{equation}
x_0<\xi_{n-1}<\xi_n<\xi_{n-1}+\Delta x<x_0+n\Delta x~.
\end{equation}
\textbf{证毕!}
