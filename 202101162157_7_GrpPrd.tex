% 群的直积和直和
% keys 群|直积|笛卡尔积|直和|半直和
\pentry{正规子群\upref{NormSG}}

群的直积,是在群作为集合的笛卡尔积上,由群运算自然导出的一个群.

\begin{definition}{两个群的直积}
给定群$G$和$H$,群运算的符号省略.在集合$G\times H$上定义运算:对于任意$(g_i, h_i)\in G\times H$,有$(g_1, h_1)(g_2, h_2)=(g_1g_2, h_1h_2)$.集合$G\times H$配合以上定义的运算,构成一个群,称为群$G$和$H$的\textbf{直积(direct product)}.
\end{definition}

容易看出,两个群直积的单位元是$(e, e)$——注意这里的两个$e$分属不同的群,通常是不同的元素.

这个定义分割开了参与直积的不同群的运算,因此可以很方便地直接推广到任意多个群的直积:

\begin{definition}{任意多个群的直积}
给定任意多个群$\{H_i\}$,在这些群作为集合的笛卡尔积上,各分量运算分别进行运算,且遵循各自所属群的运算规则.该笛卡尔积配合该运算规则构成一个群,称为这些群的\textbf{直积(direct product)},记为$\otimes_iH_i$.
\end{definition}

在群论中还有一个和直积很类似的概念,常使人混淆,这就是群的\textbf{直和}.直和实际上是直积的一个特例:任意给定群,都可以使用这些群来构造直积,但是直和指的是已经给定了一个群,使用它的特定子群来生成它.

群的直积可以推广为以下概念:

\begin{definition}{半直积}
给定群$G$,如果有$G$的一个\textbf{正规子群}$N$和一个\textbf{子群}$H$,使得$G=NH$\footnote{就是说,集合$G=\{nh|n\in N, h\in H\}$.},并且$N\cap H=\{e\}$,那么我们称$G$是$N$和$H$的\textbf{半直积(semi-direct product)}.
\end{definition}

可以注意到,直积是半直积的一种,只要把$\{(g, e)\}$和$\{g\}$等同、把$\{(e, h)\}$和$\{h\}$等同即可.这样,尽管本节中直积是用“运算的笛卡尔积”来定义的,而半直积是用“已有的群运算”来定义的,这两个在特定情况下是等价的.

半直积不一定是直积,这是因为定义中我们只要求参与运算的两个群中的一个为正规子群,而如果两个群$G$和$H$进行直积,那么容易证明它们俩都是群$G\times H$的正规子群.从这也可以看出来为什么半直积的定义要先给出$G$,而不是像直积的定义一样直接用两个群的乘积得到$G\times H$.

事实上,我们也可以用以上定义半直积的语言来描述直积:给定群$G$,如果有$G$的两个正规子群$H$和$N$,满足$H\cap N=\{e\}$,并且$G=NH$,那么称$G$是$N$和$H$的直积,记为$G=N\times H$.

最后,还有一个常见的术语:

\begin{definition}{群的直和}
当$G$和$H$都是交换群时,我们也称$G\times H$为这两个群的\textbf{直和(direct sum)},此时也可以把它表示为$G+H$.另外,任意个交换群的直积$\otimes_iH_i$也可以表示为$$
\end{definition}











