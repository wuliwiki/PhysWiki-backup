% 群的扩张
% keys 扩张|group extension|正合序列|exact sequence|短正合序列|short exact sequence
% license Usr
% type Tutor


\pentry{直积和半直积(群)\nref{nod_GrpPrd}}{nod_4306}

如果说求商群的过程是在“模糊化”一个群,那么直积可以理解为“精细化”一个群。

具体地,考虑群直积$G\times H$,则$\{e\}\times H$是其一个正规子群,于是求商群$G\times H/\{e\}\times H$的过程就是把第一分量相同的元素$(g, h_1)$和$(g, h_2)$都视为同一个元素,即模糊了它们之间的差别。反之,已知群$G$和$H$,则求直积的过程可以理解为把每个元素$g\in G$细化为集合$\{(g, h)\mid h\in H\}$,从一个点变成更多点。


那么求商群和求直积是不是互为逆运算呢?很可惜,并不是。



\begin{example}{}
考虑循环群$\mathbb{Z}/4\mathbb{Z}$和Klein群$K_4=\mathbb{Z}/2\mathbb{Z}\times \mathbb{Z}/2\mathbb{Z}$。二者都含$4$个元素,都有正规子群$\mathbb{Z}/2\mathbb{Z}$,但$\mathbb{Z}/4\mathbb{Z}\neq \mathbb{Z}/2\mathbb{Z}\times \mathbb{Z}/2\mathbb{Z}$,故不能循环群$\mathbb{Z}/4\mathbb{Z}$写成它的商群和正规子群的直积。
\end{example}




如何正确表达求商群的逆运算呢?如果说求商群$G/N$是把正规子群$N$的每个左陪集都看成一个元素,抹去其运算细节,那么反过来,把$G/N$中的每个元素都扩张为一个群,就能得到$G$,我们称这个过程为\textbf{群的扩张}。



\begin{definition}{群的扩张}
给定群$K$和$N$,如果存在一个群$G$和群同态$f:G\to K$,使得$\opn{ker}f\cong N$,则称$G$为群$$
\end{definition}




































