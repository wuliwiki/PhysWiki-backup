% 证明矩阵行秩等于列秩
% 线性无关|列秩|行秩|零空间|解空间|正交归一|基底

\pentry{矩阵与矢量空间\upref{MatLS}, 正交子空间\upref{OrthSp}}

我们从矢量空间的角度\upref{MatLS} 证明, 假设矩阵 $\mat A$ 的尺寸为 $N_Y \times N_X$, 对应的线性算符为 $A: X\to Y$. $X, Y$ 空间的维数分别为 $N_x, N_y$. $\mat A$ 的厄米共轭\upref{HerMat}记为\footnote{如果矩阵是实数的, 那么厄米共轭就是转置 $\mat A\Tr$.} $\mat A\Her$, 对应算符记为 $A\Her$.

我们前面讲过 $X$ 空间中满足 $A \ket{x} = \bvec 0$ 的所有 $\ket{x}$ 构成的空间为 $A$ 的零空间(链接未完成), 记为 $X_0$. $Y$ 空间中满足 $A\Her |y\rangle = 0$ 的子空间为 $Y$ 的零空间, 记为 $Y_0$. 定义整个 $X$ 空间通过 $A$ 映射后得到 $Y$ 的另一个子空间为 $Y_1$ 空间, 即 $A(X) = Y_1$, 同理定义 $A\Her(Y) = X_1$. % 未完成: 矩阵与矢量空间中添加空间的映射及对应符号

可以证明(见下文) $X_0$ 与 $X_1$ 互为正交补; $Y_0$ 与 $Y_1$ 互为正交补(\autoref{OrthSp_def1}~\upref{OrthSp}), 且
\begin{equation}\label{RCrank_eq1}
X_0 \oplus X_1 = X
\qquad
Y_0 \oplus Y_1 = Y
\end{equation}
于是我们可以在 $X$ 和 $Y$ 空间中分别找到一套正交归一基底, 使得这些基底分为两组, 分别张成两个正交子空间. 计算 $A(X)$ 时, $X_0$ 中的基底映射后还是零, 所以其中的基底可以去掉, 只剩下 $X_1$ 中的基底做映射, $A\Her(Y)$ 也同理:
\begin{equation}
\begin{aligned}
&A(X_1) = A(X) = Y_1\\
&A\Her(Y_1) = A\Her(Y) = X_1
\end{aligned}
\end{equation}
线性映射后的空间维度总是小于等于映射前的维度, 所以如果两个方向都存在映射, 维度只能相等, 即都是一一映射. 所以 $X_1$ 和 $Y_1$ 的维度相同, 即 $A$ 和 $A\Her$ 线性无关的列数相同\footnote{矩阵线性无关的列数等于值域的维度(链接未完成).}, 即 $A$ 线性无关的行数和列数相同, 即行秩等于列秩. 证毕.

\subsubsection{补充证明}
以下证明 $Y_0$ 和 $Y_1$ 正交以及\autoref{RCrank_eq1} 中的第二条. 第一条同理可得.

我们先在 $Y_1$ 中找到一套 $N_{Y1}$ 个正交归一基底 $\ket{y_{1i}}$, 再在 $Y$ 空间中找到剩下 $N_Y - N_{Y1}$ 个正交归一基底 $\ket{y_i}$. 对任意 $\ket{x} \in X$, 有 $A \ket{x} \in Y_1$, 所以
\begin{equation}
\mel{y_i}{A}{x} = 0
\end{equation}
对两边做厄米共轭得% 未完成: 狄拉克符号里面需要介绍
\begin{equation}
\mel{x}{A\Her}{y_i} = 0
\end{equation}
对任意 $\ket{x}$ 都成立, 所以
\begin{equation}
A\Her\ket{y_i} = 0
\end{equation}
所以 $\ket{y_i} \in Y_0$. 然而对于任意 $\ket{y_1} \in Y_1$, 必存在一些矢量 $\ket{x}$ 使
\begin{equation}
\mel{y_1}{A}{x} \ne 0
\iff
\mel{x}{A\Her}{y_1} \ne 0
\end{equation}
所以
\begin{equation}
A\Her\ket{y_1} \ne 0
\end{equation}
即 $\ket{y_1} \notin Y_0$. 所以 $\ket{y_i}$ 就是 $Y_0$ 的(完备)正交归一基底, 且与 $\ket{y_{1i}}$ 组成 $Y$ 的完备正交归一基底. 证毕.
