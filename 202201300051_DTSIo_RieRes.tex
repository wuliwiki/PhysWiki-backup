% 黎曼重排定理

\pentry{绝对收敛与条件收敛\upref{Convg}}

相比于绝对收敛级数, 条件收敛级数究竟是哪里有问题呢? 回忆 绝对收敛与条件收敛\upref{Convg}中给出的例子
$$
\sum_{n=1}^\infty\frac{(-1)^{n+1}}{n}
=1-\frac{1}{2}+\frac{1}{3}-\frac{1}{4}+...
$$
这个级数的和等于$\ln2$, 但我们已经通过一种重排技巧将它的和缩减到了本来的和的一半. 这个性质不是偶然的. 条件收敛级数的收敛完全是因为相邻项之间的正负抵消, 而将这件事精确化的正是如下的

\begin{theorem}{黎曼重排定理}
设$\sum_{n=1}^\infty a_n$是条件收敛但不绝对收敛的级数, 一般项都是实数. 则对于任何一个实数$A$, 都存在正整数集$\mathbb{N}$的一个重排$\sigma$, 使得
$$
\sum_{n=1}^\infty a_{\sigma(n)}=A.
$$
\end{theorem}