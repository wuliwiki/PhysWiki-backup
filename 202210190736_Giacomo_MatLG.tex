% 矩阵李群
% 矩阵李群|李群

\begin{issues}
\issueDraft
\end{issues}

\pentry{一般线性群\upref{GL},域上的代数\upref{AlgFie}}

% 我希望把这个词条写成不需要微分几何前置的样子,参见GTM222
\subsection{矩阵李群}

$M_n(\mathbb{C})$ 是全体 $n \times n$ 复矩阵的集合.作为一个向量空间,它同构于 $\mathbb{C}^{n^2}$.作为一个有限维度实向量空间\footnote{复向量空间当然也是实向量空间},它也是一个拓扑空间,而且和 $\mathbb{C}^{n^2}$ 是同胚的.

$M_n(\mathbb{C})$ 和 $\mathbb{C}^{n^2}$ 不同的地方在于,它本身构成一个 $\mathbb{C}$-代数,即矩阵的乘法.

\addTODO{考虑将这部分移动到《一般线性群》里}

全体 $n \times n$ 可逆矩阵的集合 $\opn{GL}(n, \mathbb{C}) \subseteq M_n(\mathbb{C})$ 构成一个群,同时也是拓扑空间 $M_n(\mathbb{C})$ 的一个开集合(因此是个子流形).

\begin{definition}{矩阵李群}
对于群 $\opn{GL}(n, \mathbb{C})$ 的子群 $G$,$G$ 被称为一个\textbf{矩阵李群}如果它是 $\opn{GL}(n, \mathbb{C})$ 的一个闭子集. Definition 1.4 \cite{GTM222} 
\end{definition}

对于一个矩阵李群 $G$,我们有
$$
G \subseteq \opn{GL}(n, \mathbb{C}) \subseteq M_n(\mathbb{C})
$$
$G$ 在 $\opn{GL}(n, \mathbb{C})$ 是闭的,但在 $M_n(\mathbb{C})$ 不一定.

\subsection{例子}

\begin{example}{}
$$\opn{SL}(n, \mathbb{C}) \subseteq \opn{GL}(n, \mathbb{C})$$
$$\opn{GL}(n, \mathbb{R}) \subseteq \opn{GL}(n, \mathbb{C})$$
$$\opn{SL}(n, \mathbb{R}) \subseteq \opn{GL}(n, \mathbb{C})$$
\end{example}

\subsubsection{幺正群和正交群}
\addTODO{定义}
\subsubsection{广义正交群}
\addTODO{定义}

\addTODO{定义:洛伦兹群}

\addTODO{定理:在复数下只有一种正交群}

\subsection{矩阵李群是李群}
\pentry{李群\upref{LieGrp}}

\begin{theorem}{}
矩阵李群是 $\opn{GL}(n, \mathbb{C})$ 的子李群
\end{theorem}

\begin{exercise}{}
证明它.
\end{exercise}