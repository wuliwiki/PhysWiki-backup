% 电磁脉冲
% license CCBYSA3
% type Wiki

(本文根据 CC-BY-SA 协议转载自原搜狗科学百科对英文维基百科的翻译)

电磁脉冲(Electromagnetic Pulse,EMP),有时也称为瞬变电磁干扰,是电磁能量的短脉冲。这种脉冲的来源可以是自然发生的,也可以是人为的,并且根据来源的不同,可以是辐射场、电场、磁场或传导电流。

电磁脉冲干扰通常会破坏或损坏电子设备,在较高的能量水平下,强电磁脉冲(如雷击)会损坏建筑物和飞机机体的物理结构。电磁脉冲效应的管理是电磁兼容( electromagnetic compatibility,EMC)工程的一个重要分支。

释放高能电磁脉冲的破坏性武器已经被开发出来。

\subsection{一般特性}
电磁脉冲是电磁能量的短脉冲。它的短持续时间意味着它将在一个频率范围内传播。电磁脉冲的典型特征是:
\begin{itemize}
\item 能量类型(辐射、电、磁或传导)。
\item 存在的频率范围或频谱。
\item 脉冲波形:形状、持续时间和振幅。
\end{itemize}
最后两种,频谱和脉冲波形通过傅里叶变换相互关联,可以被视为描述同一脉冲的两种不同方式。
\subsubsection{1.1 能量类型}
电磁脉冲能量可以凭借如下四种形式传递:
\begin{itemize}
\item 电场
\item 磁场
\item 电磁辐射
\item 电传导
\end{itemize}
由于麦克斯韦方程,任何一种形式的电磁能脉冲总是伴随着其他形式电磁能量,然而在典型的脉冲中,一种形式将占主导地位。

一般来说,只有辐射传递方式可以用于长距离传递,其他辐射仅适用于短距离传递。但也有一些例外,比如太阳磁耀斑。
\subsubsection{1.2 频率范围}
电磁能量脉冲通常包括从直流DC(零赫兹)到某个上限的许多频率,这种频率范围取决于辐射源。电磁脉冲的频率范围,有时被称为“直流至日光”,不包括由光学(红外、可见、紫外)和电离(x光和伽马射线)组成的最高频率范围。

某些类型的电磁脉冲事件会留下光学痕迹,如闪电和火花,但这些都是流经空气的电流的副作用,并不属于电磁脉冲本身。
\subsubsection{1.3 脉冲波形}
脉冲波形描述了其瞬时振幅(场强或电流)随时间的变化情况。真实的脉冲往往非常复杂,所以经常使用简化模型。这种模型通常用图表或数学方程来描述。
\begin{figure}[ht]
\centering
\includegraphics[width=14.25cm]{./figures/06d6169b006b519c.png}
\caption{} \label{fig_DCMC_1}
\end{figure}
大多数电磁脉冲都有一个非常尖锐的前沿,迅速积累到最大水平。经典模型是一条双指数曲线,它急剧上升,迅速达到峰值,然后以很慢的速度进行衰减。然而,来自受控开关电路的脉冲通常近似于矩形或“正方形”脉冲的形式。

电磁脉冲事件通常会在周围环境或材料中引发相应的信号。强耦合通常发生在相对较窄的频带上,导致特征阻尼正弦波。在视觉上,在双指数曲线的较长寿命包络内,可以直观的看到高频正弦波的增长和衰减。由于耦合模式的传输特性,阻尼正弦波通常比原始脉冲具有低得多的能量和更窄的频带宽度。实际上,电磁脉冲测试设备经常直接注入这些阻尼正弦波,而不是试图重现高能威胁脉冲。

在脉冲序列中,例如在数字时钟电路中,波形以规则的间隔重复。单个完整的脉冲周期足以描述这样一个有规律的、重复的序列。

\subsection{类型}
电磁脉冲产生于辐射源发出短时间能量脉冲的地方。尽管EMP通常在周围环境中激发相对窄带的阻尼正弦波响应,但能量本质上通常是宽带的。有些类型是作为重复和规则的脉冲序列产生的。

不同类型的电磁脉冲:有来自自然、人为和武器效应的电磁脉冲。

自然电磁脉冲事件的类型包括:
\begin{itemize}
\item 雷电电磁脉冲(LEMP)。放电通常是最初的大电流,至少是百万安培,然后是一系列能量下降的脉冲。
\item 静电放电(ESD),由于两个带电物体非常接近甚至接触。
\item 流星电磁脉冲。由流星体与宇宙飞船碰撞或穿过地球大气层的流星体爆炸破裂而产生的电磁能量的释放。[1][2]
\item 日冕物质抛射(CME)。等离子体爆发和伴随的磁场,从日冕中喷出并释放到太阳风中。有时被称为太阳电磁脉冲。[3]
\end{itemize}

(民用)人为电磁脉冲事件的类型包括:
\begin{itemize}
\item 电路的开关动作,无论是隔离的还是重复的(如脉冲序列)。
\item 当电枢旋转时,内部电触点接通和断开连接时,电动机可以产生一系列脉冲。
\item 汽油发动机点火系统可以在火花塞通电或点火时产生一系列脉冲。
\item 数字电子电路的连续开关动作。
\item 电力线路。这些电压可能高达几千伏,足以损坏保护措施不足的电子设备。
\end{itemize}

军用电磁脉冲的类型包括:
\begin{itemize}
\item 核爆炸产生的核电磁脉冲(NEMP)。另一种变体是高空核电磁脉冲(HEMP),是由于粒子与地球大气和磁场的相互作用所产生二次脉冲。
\item 无核电磁脉冲(NNEMP)武器。
\end{itemize}
\subsubsection{2.1 闪电}
闪电是不寻常的,因为它通常有一个初级的低能量“先导”放电,然后积累到主脉冲,这个过程期间可能会产生几次较小的爆发。[4][5]
\subsubsection{2.2 静电放电(ESD)}
静电放电事件的特点是电压很高,但电流很小,有时会产生可见火花。尽管从技术上讲闪电是一种非常大的静电放电事件,但静电放电被视为一种局部的小现象。静电放电也可以是人为的,例如从范德格拉夫发电机(Van de Graaff generator)收到的冲击。

静电放电事件除了给人们带来不愉快的冲击之外,还会通过注入高压脉冲损坏电子电路。这种静电放电事件还会产生火花,进而引发火灾或燃料蒸汽爆炸。为此,在给飞机加油或将任何燃油蒸气暴露于空气之前,燃油喷嘴首先会连接到飞机上,以安全地排放任何静电。
\subsubsection{2.3 开关脉冲}
电路的开关动作会造成电流的急剧变化。这种剧烈的变化是电磁脉冲的一种形式。

简单的电源包括感应负载,如继电器、螺线管和电动机中的电刷触点。通常情况下,它们向任何存在的电连接发送脉冲并辐射能量脉冲。振幅通常很小,信号可以被视为“噪声”或“干扰”。电路的“关断或“打开”会导致电流的突然变化。这反过来会导致开路触点上的电场出现大脉冲,从而导致电弧放电和损坏。通常有必要结合设计特征来限制这种影响。

电子管或阀门、晶体管和二极管等电子设备也可以很快打开和关闭,同样会造成类似的问题。一次性脉冲可能是由固态开关和其他偶尔使用的器件引起的。然而,现代计算机中数以百万计的晶体管可能会在高于1 GHz的频率上重复切换,从而造成看似连续的干扰。
\subsubsection{2.4 核电磁脉冲(NEMP)}
核电磁脉冲是核爆炸产生的电磁辐射的突变脉冲。由此产生的快速变化的电场和磁场可能与电气/电子系统耦合,从而产生破坏性的电流和电压浪涌。[6]

发射的强伽马射线也能电离周围的空气,当空气原子先失去电子,然后又重新获得电子时,产生二次电磁脉冲。

NEMP武器的设计是为了最大限度地发挥电磁脉冲的破坏机制,有些武器能够在大范围内摧毁敏感的电子设备。

高空电磁脉冲(HEMP)武器是一种NEMP弹头,设计用于在地球表面上方引爆。爆炸将伽马射线释放到平流层中部,作为二次电离效应,产生的高能自由电子与地球磁场相互作用,产生的电磁脉冲比低海拔较稠密空气中通常产生的电磁脉冲强得多。
\subsubsection{2.5 无核电磁脉冲(NNEMP)}
无核电磁脉冲(NNEMP)是一种不使用核技术的武器产生的电磁脉冲。可以实现这一目标的器件包括放入单回路天线的一个大的低电感电容器组,、微波发生器和爆炸磁量压缩发生器。为了获得最佳耦合到目标所需的脉冲频率特性,在脉冲源和天线之间增加了波形整形电路或微波发生器。虚阴极振荡器是一种真空管,特别适合高能脉冲的微波转换。[7]

NNEMP发电机可以作为炸弹、巡航导弹(如CHAMP导弹)和无人驾驶飞机的有效载荷携带,减少机械、热和电离辐射效应,但不会产生核武器带来的后果。

NNEMP武器的射程远小于核电磁脉冲。几乎所有用作武器的NNEMP装置都需要化学炸药作为其初始能源,产生的能量只有同等重量的核爆炸装置的10-6(百万分之一)。[8] 来自NNEMP武器的电磁脉冲必须来自武器内部,而核武器产生电磁脉冲作为次要效应。[9] 这些事实限制了NNEMP武器的射程,但其具有更好的目标识别特性。小型电子炸弹的效果已被证明足以应对某些恐怖主义或军事行动。这种操作的例子包括摧毁对许多地面车辆和飞机的操作起着至关重要作用的电子控制系统。[10]

用于产生无核电磁脉冲的爆炸磁通量压缩发生器的概念早在1951年就由苏联的安德烈·萨哈罗夫(Andrei Sakharov)提出,[11] 但在其他国家出现类似的想法之前,各国一直对无核电磁脉冲进行保密。
\subsubsection{2.6 电磁成形}
电磁脉冲产生的巨大力量可以用来塑造或形成物体,从而作为其制造过程的一部分。

\subsection{影响}
轻微的电磁脉冲事件,尤其是脉冲序列,会导致低水平的电噪声或干扰,从而影响敏感设备的运行。例如,二十世纪中叶的一个常见问题是由汽油发动机点火系统发出的干扰,导致了收音机噼啪作响以及电视机在屏幕上显示条纹。为了让汽车制造商安装干扰抑制器来减少电磁脉冲对电子设备的干扰,则出台了相关的法律。

在高电压水平下,电磁脉冲会引发火花,例如给汽油发动机车辆加油时产生的静电放电。由于这种火花会引起燃料-空气爆炸,因此必须对其采取预防措施。[12]

一个巨大而充满能量的电磁脉冲可以在受损单元中感应出高电流和电压,这会导致设备功能的中断或被永久损坏。

强大的电磁脉冲还会直接影响磁性材料,破坏存储在磁带和计算机硬盘等介质上的数据。硬盘通常由重金属外壳保护。一些信息技术资产处置服务提供商和计算机回收商使用受控电磁脉冲来擦除这种磁介质。[13]

雷击等非常大的电磁脉冲事件也能够通过热效应或电流产生的非常大的磁场破坏效应,直接损坏树木、建筑物和飞机等物体。间接影响可能是由加热引起的电气火灾。因此,大多数工程结构和系统需要某种形式的防雷设计。

高能电磁脉冲的破坏性影响促使了电磁脉冲武器的引入,从具有小作用半径的战术导弹到针对大范围内发挥电磁脉冲效应的核弹。

\subsection{控制}
\begin{figure}[ht]
\centering
\includegraphics[width=6cm]{./figures/cd17d5c01d3165aa.png}
\caption{电磁脉冲模拟器HAGII-C测试波音E-4飞机。} \label{fig_DCMC_2}
\end{figure}
与任何电磁干扰一样,电磁脉冲的威胁(自然的和人为的)也会受到控制措施的控制。

因此,大多数控制措施侧重于设备对电磁脉冲效应的敏感性,并加固或保护设备免受伤害。除武器外的人为来源也受到控制措施的制约,以限制脉冲能量的发射。

在电磁脉冲和其他射频威胁存在的情况下,确保设备正确运行的原则被称为电磁兼容性。
