% 集合
\pentry{公理系统\upref{axioms}}
\subsection{集合}

对于物理学习而言,集合论没必要从公理角度来严格理解,所以在此不会给出用于划定集合论讨论范围的公理系统,而是朴素集合论的解释,即比较接近自然语言的表达方式.

\textbf{集合(set)}是由\textbf{元素(element)}组成的.任何事物和概念都可以成为元素,任何不同的元素都可以放在一起,构成一个集合.可以说,如果我们划定一个讨论的范围,那么这个范围就是一个集合,范围涉及到的事物和概念就是这个集合当中的元素.公理系统的作用,也就是在所有 % 未完成

表达一个集合的方式有多种,最简单的方式是列出所有集合中的元素.在数学中规定的语法规范是用大括号“$\{\}$”来列举集合中的一切元素,以逗号隔开彼此.比如,$\{\text{猪}, \text{牛}, \text{狗}, \text{羊}, \text{猫}\}$是了一个具有五个元素的集合,$\{1,2,3,4,\dots\}$则是全体正整数的集合.第二个例子并没有显然地列举出所有正整数,只是用省略号表达了这个意思;也就是说,表达一个集合的方式并没有死板的规定,只要能让读者理解就可以了.

另一种常见的表达集合的方式是确定一个规则,语法规范是“$\{x|x \text{需要满足的条件}\}$”.比如全体正整数的集合,也可以写为$\{x|\text{$x$ 是一个正整数}\}$.如果有多个条件,也可以列在一起,比如全体正整数的集合:$\{x|x \text{是一个正数,且 $x$ 是一个整数}\}$.特别地,如果某条规则是“$x$属于某集合”,我们通常会将这个条件写到单竖线的前面,如全体正整数的集合:$\{x\in\mathbb{Z}|\text{$x$ 是一个正数}\}$. 这里,$\in$是一个简写的符号,$A\in B$等于说“$A$是$B$的元素”.

如果集合$A$的元素都是集合$B$的元素,那么称$A$是$B$的\textbf{子集(subset)}.一切集合都是自身的子集.如果$A$是$B$的子集但又和$B$不同,也就是说$A$没有包含$B$的所有元素,那么称$A$是$B$的真子集.

有一个特殊的集合,它不含有任何元素,被称为\textbf{空集(empty set)},记作 $\varnothing$. $\varnothing$是一切集合的子集.

\subsection{属于和包含}
为了简化表达,数学家把集合论中常用的动词表示成简略的形式.

$A\in B$或者$B\ni A$等价$B$于“$A$是$B$的元素”,表达“属于”关系. 

$A\subset B$ 或者 $B\supset A$等价于“$A$是$B$的子集”,表达“包含”关系. 

注意区分这两个情况,前一个情况中$A$是$B$的元素,后一个情况中$A$是$B$的子集.另外,集合本身也可以是别的集合的元素,元素的概念没有限定,任何事物和概念都可以成为元素,包括集合.

其它形式的子集符号,如$\subseteq$,$\supseteq$等,在不同的文献中可能有不同的含义,所以一般没有特殊说明时不会使用.


\subsection{集合运算}

集合间可以互相操作,生成新的集合,这种操作被称为集合间的\textbf{运算(operation)}.

$\cap$表示两个集合的交,意思是将两个集合中共有的元素提取出来,组成一个新的集合.比如说,$\mathbb{N^+}$表示全体自然数的集合,$\mathbb{R^+}$表示全体正实数的集合,$\mathbb{Z}$表示全体整数的集合,那么显然我们可以有$\mathbb{N^+}=\mathbb{R^+}\cap\mathbb{Z}$. 多个集合$A_i$的交集,可以写为$A_0\cap A_1\cap A_2\cap A_3⋯$,也可以用一个大号的交集符号简记为$\bigcap A_i$,表示“所有形式为$A_i$的集合的交集”.

类似地,将两个集合中都有的元素提取出来,组成一个新的集合的操作,被称为集合的并,用符号$\cup$ 和$\bigcup$表示.注意,如果两个元素中有相同元素,那么在并集中这个元素只出现一次.这是因为我们关心的是每个元素是否出现在集合中,计算集合元素数量时也不会重复计算同一个元素.这是一个并集的例子:$\{$猪,牛,狗,羊,猫$\}\cup\mathbb{N^+}$=$\{\text{猪,狗,猫,牛,羊}, 1,2,3,4,⋯\}$. 注意,列举时元素的顺序也不影响集合的本质.

对于集合 $A$ 和 $B$, $A\backslash B$ 或者 $A-B$ 表示他们的\textbf{差集}. 差集所包含的元素是 $A$ 中全体元素中减去 $B$ 中元素, 如果 $B$ 还含有 $A$ 中所没有的元素, 那么这部分元素可以忽略掉.例如,如果令$A=\{0,1,2,3\}$, $B=\{2,3,4\}$, 那么 $A-B=\{0,1\}$.

对于集合$A$和$B$,$A\times B$ 表示集合间的\textbf{笛卡尔积(Cartesian product)}, 得到一个新的集合. $A\times B$ 中的元素表示为 $(a,b)$,其中$a\in A$, $b\in B$.用集合论的术语表达就是
\begin{equation}
A\times B=\{(a,b)|a\in A, b\in B\}
\end{equation}
例如,如果令$A=\{0,1,2,3\}, B=\{2,3,4\}$,那么
\begin{equation}
A\times B=\{ (0,2),(0,3),(0,4),(1,2),(1,3),(1,4),(2,2),(2,3),(2,4),(3,2),(3,3),(3,4) \}
\end{equation}
可以看到,集合$A$有4个元素,集合$B$有3个元素,而集合$A\times B$有3$\times$4=12个元素.
