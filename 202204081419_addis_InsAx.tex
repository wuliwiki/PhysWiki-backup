% 刚体的瞬时转轴
% 转轴|刚体|陀螺|角速度

\pentry{刚体\upref{RigBd}}

当刚体绕某个固定点做任意转动时, 我们在每个时刻仍然能找到一个经过该点的\textbf{瞬时转轴}以及延转轴的瞬时角速度(矢量) $\bvec \omega$.

某时刻瞬时转轴的定义是: 该时刻刚体在内任意落在转轴上的点速度为零. 事实上, 我们只需要在某时刻找到刚体上瞬时速度为 0 的任意两个不同点, 就可以过这两点作出瞬时转轴, 并确保该直线上的所有点瞬时速度都为 0.

\begin{example}{进动陀螺的瞬时转轴}\label{InsAx_ex2}
在陀螺进动的例子(\autoref{AMLaw_ex2}~\upref{AMLaw}) 中, 我们可能会认为陀螺的瞬时转轴就是陀螺的轴. 但陀螺的轴时时刻刻都在运动, 除了与地面接触的点外, 任意一点的瞬时速度都不为 0. 所以此时陀螺的轴并不是瞬时转轴.

注意陀螺的自转和进动的方向相同(都是顺时针或逆时针), 我们可以知道真正的瞬时转轴同样经过与地面的接触点, 但会在陀螺轴的上方. 我们只需要找到陀螺圆盘表面上瞬时速度为 0 的一点即可求出瞬时转轴的倾角.
\end{example}
