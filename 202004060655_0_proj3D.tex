% 三维投影
% 立体几何|投影|透视|画图

% 参考 https://en.wikipedia.org/wiki/3D_projection

当我们在平面上画三维物体时, 我们需要某种投影算法把物体上的每个点对应到平面上的一点. 以下介绍两种常用的方法, 一种是\textbf{平行投影(parallel projection)}, 另一种是\textbf{透视投影(perspective projection)}.

% 首先来比较两个图(图未完成: 左图是长方体的平行投影, 右图是长方体的透视投影)
% 参考 https://construct3.ideas.aha.io/ideas/C3-I-754

\subsection{平行投影}
故名思意, 平行投影是指在空间中指定一个方向(如图未完成), 将三维物体上的每一点沿着该方向投影到与该方向垂直的平面上. 工程制图中的正视图, 侧视图等都属于平行投影. 这种投影的特点是, 空间中的任意两条平行线的投影仍然是平行线.

\subsection{透视投影}
当人眼或相机观察一个三维物体时, 使用的是透视投影.

(用一张图介绍原理)

我们在平面后方取一个固定点称为\textbf{焦点}, 
