% 反射变换(高等代数)
% license Usr
% type Tutor
反射变换的概念来源于平面上的轴反射。根据初中几何学里就已接触的轴反射定义,我们可以用矩阵表示该二维空间中的线性变换。
\begin{equation}
\mat M=\pmat{0&1\\-1&0}~.
\end{equation}
可以验证,$\mat M\bvec e_x=-\bvec e_x,\mat M \bvec e_y=\bvec e_y$。因此这确实是一个保持向量在平行轴方向的分量不变,垂直轴的方向反向的轴反射。

任意图像在轴反射后形状不变,也就是说,这是一个保距变换。同理,我们也可以对任意向量作关于平面的反射,或者更延伸一些,在$\mathbb R^n$中作关于超平面$\mathbb R^{n-1}$的反射。(下文将超平面简称为平面)
\begin{definition}{}
设$S$是$n$维向量空间$\mathbb R^n$中的平面,$\bvec n$是其\textbf{单位法向量},对于任意$\bvec x\in \mathbb R^n$,定义其\textbf{反射(reflection)}$R:\mathbb R^n\rightarrow \mathbb R^n$为
\begin{equation}
R(\bvec x)=\bvec x-2(\bvec x,\bvec n)\bvec n~.
\end{equation}
\end{definition}
可以验证,$R$是保距变换。


\begin{theorem}{}
保距变换一定是反射变换的复合。
\end{theorem}
\textbf{证明:}

在$n=1$的时候,保距变换要么是恒等变换,要么是反射变换,因为只有这两种变换不改变基向量的范数。因此,定理自然成立。

假设定理对$1<n<k$成立,则$\mathbb R^{k-1}$上的保距变换可以写为至多$k-1$个反射变换的复合。 

假设$f$是$\mathbb R^k$中的保距变换,且不是恒等变换。设$f(\bvec x)=\bvec y,\bvec z=\bvec y-\bvec x$,$S_z$为以$\bvec z$为法向量的超平面,并用$R_z$表示关于该超平面的反射变换。可以验证:$R_z\circ f(\bvec x)=\bvec x$