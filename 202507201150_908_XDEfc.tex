% 薛定谔方程(综述)
% license CCBYSA3
% type Wiki

本文根据 CC-BY-SA 协议转载翻译自维基百科\href{https://en.wikipedia.org/wiki/Schr\%C3\%B6dinger_equation}{相关文章}。

薛定谔方程是一个偏微分方程,用于描述非相对论量子力学体系的波函数的演化过程。\(^\text{[1]: 1–2 }\) 它的发现是量子力学发展史上的一个重要里程碑。该方程以奥地利物理学家埃尔温·薛定谔的名字命名。他于1925年提出该方程,并于1926年发表,从而奠定了其后获得1933年诺贝尔物理学奖的工作基础。\(^\text{[2][3]}\)

在概念上,薛定谔方程是量子力学中对应于经典力学中牛顿第二定律的表达。给定一组已知的初始条件,牛顿第二定律可以用数学方式预测一个物理系统随时间演化的轨迹。薛定谔方程则给出了波函数随时间的演化规律,而波函数是对一个孤立物理系统的量子力学描述。该方程是薛定谔在路易·德布罗意提出“所有物质都具有伴随的物质波”这一假设的基础上提出的。薛定谔方程成功预测了与实验观测一致的原子束缚态。\(^\text{[4]: II:268 }\)

薛定谔方程并不是研究量子力学系统和进行预测的唯一方法。量子力学的其他表述方式还包括维尔纳·海森堡提出的矩阵力学,以及主要由理查德·费曼发展的路径积分表述。在比较这些方法时,使用薛定谔方程的方式有时被称为“波动力学”。
