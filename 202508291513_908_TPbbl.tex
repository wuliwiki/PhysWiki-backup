% 拓扑不变量(综述)
% license CCBYNCSA3
% type Wiki

本文根据 CC-BY-SA 协议转载翻译自维基百科\href{https://en.wikipedia.org/wiki/Topological_property}{相关文章}

在拓扑学及其相关数学领域中,拓扑性质或拓扑不变量是指在同胚下保持不变的拓扑空间的某种性质。换句话说,拓扑性质是一个在同胚映射下封闭的拓扑空间的真类。也就是说,如果某个空间 $X$ 具有某一性质,那么所有与 $X$ 同胚的空间也都具有这一性质。通俗地说,拓扑性质就是能够用开集来描述的空间性质。

在拓扑学中,一个常见的问题是判断两个拓扑空间是否同胚。为了证明两个空间不是同胚,只需找到一个它们不共有的拓扑性质即可。
\subsection{ }
一个性质 $P$ 可以具备以下特征:
\begin{itemize}
\item \textbf{遗传性},若对任意拓扑空间 $(X, \mathcal{T})$ 及其子集 $S \subseteq X$,其子空间 $\bigl(S, \mathcal{T}|_S\bigr)$ 也具有性质 $P$。
\item \textbf{弱遗传性},若对任意拓扑空间 $(X, \mathcal{T})$ 及其闭子集 $S \subseteq X$,其子空间 $\bigl(S, \mathcal{T}|_S\bigr)$ 也具有性质 $P$。
\end{itemize}
\subsection{常见的拓扑性质}
\subsubsection{基数函数}
\begin{itemize}
\item 空间 $X$ 的基数 $|X|$:指空间 $X$ 中点的数量。
\item 空间 $X$ 的拓扑基数$|\tau(X)|$:指该空间拓扑(即所有开集的集合)的基数。
\item 权$w(X)$:指空间 $X$ 的拓扑基的最小基数。
\item 稠密度$d(X)$:指空间 $X$ 的一个稠密子集的最小基数,即闭包等于 $X$ 的子集所含的最少点数。
\end{itemize}