% QED的重整化理论(单圈修正)
% 重整化|单圈修正|量子电动力学

\pentry{QED的费曼规则\upref{qedfey}}

裸的拉氏量为
\begin{equation}
\begin{aligned}
\mathcal{L}=
\bar{\psi}_0 (i\not\partial -m_0)\psi_0 -\frac{1}{4}(F_0^{\mu\nu})^2 - e_0\bar\psi_0 \gamma_\mu\psi_0 A_0^\mu
\end{aligned}
\end{equation}
为了消除圈图的紫外发散,采取一定的正规化方案(例如我们最常使用的维数正规化方案),然后再调整裸参数使得可观测量的计算结果与实验相符,即满足一定的重整化条件。或者我们采取 OS 方案或 $\overline{MS}$ 方案,不同的重整化方案实际上是对场量进行了平移和缩放。设从裸的拉氏量出发,得到的电子和光子的两点编时格林函数为
\begin{equation}
\begin{aligned}
&G^{(2)}(p^2)=\int\dd[4]{x} e^{ipx} \bra{\Omega}\psi_0(x)\psi_0(0)\ket{\Omega}=\frac{iZ_2}{\not p-m+i\epsilon} + \cdots\\
&D_{\mu\nu}^{(2)}(p^2)=\int\dd[4]{x} e^{ipx} \bra{\Omega}A_{0\mu}(x)A_{0\nu}(0)\ket{\Omega}=\frac{iZ_3}{\not p-m+i\epsilon}+\cdots
\end{aligned}
\end{equation}
因此我们可以定义新的场量 $\psi_0=\sqrt{Z_2} \psi,A_{0\mu} = \sqrt{Z_3} A_\mu$,得到新的拉氏量
\begin{equation}
\mathcal{L} = Z_2\bar\psi (i\not\partial - m_0)\psi - \frac{1}{4}Z_3 (F^{\mu\nu})^2-e_0Z_2Z_3^{1/2} \bar\psi \gamma^\mu\psi A_\mu
\end{equation}
再将 $m_0,e_0$ 吸收进新的参数中,定义:
\begin{equation}
Z_m = Z_2 m_0 / m,\quad Z_1 = e_0Z_2Z_3^{1/2} / e
\end{equation}
其中 $e,m$ 为物理电荷量(在大的空间尺度上测得的量)和物理质量。利用这些新的重整化参数 $Z_1,Z_2,Z_3,Z_m$,我们写出重整化的拉氏量。
\begin{equation}
\begin{aligned}
\mathcal{L} = \bar\psi (iZ_2\not\partial - Z_m m)\psi -\frac{1}{4}Z_3F^{\mu\nu}F_{\mu\nu}-Z_1 e\bar\psi \gamma^\mu \psi A_\mu
\end{aligned}
\end{equation}
提取出其中自由场的部分后,微扰的拉氏量就是
\begin{equation}
\mathcal{L}_1 = - Z_1 e \bar\psi \gamma^\mu\psi A_\mu + \mathcal{L}_\text{CT},\quad \mathcal{L}_\text{CT} = \bar\psi (i\delta_2 \not\partial - \delta_m m)\psi - \frac{1}{4}\delta_3 F^{\mu\nu}F_{\mu\nu}
\end{equation}
由于重整化不改变理论的对称性,我们要求重整化的拉氏量仍然满足 $U(1)$ 规范对称性,那么重整化参数要满足 $Z_1=Z_2$。这里的证明实际上是不那么严格的,之后会利用 Ward 等式给出一个严格的证明。
\subsection{OS 重整化方案}
\subsubsection{重整化条件}
在 OS 重整化方案下,我们要求电子的正规传播子(两点编时格林函数的傅里叶变换)在物理质量处的留数为 $1$,光子的正规传播子在 $p^2=0$ 处留数为 $1$。除此以外,我们还要求电子光子顶点 $-ie\Gamma^\mu(p'-p)$ 在 $q=p'-p$ 趋于 $0$ 时的极限为 $-ie\gamma^\mu$。这四个重整化条件可以用于确定四个重整化参数。
\begin{equation}
\begin{aligned}
\Sigma(\not p=m) &= 0;\\
\left.\frac{\dd }{\dd p}\Sigma(\not p)\right|_{\not p=m}=0
\end{aligned}
\end{equation}