% QED的重整化理论-顶点函数的单圈修正
% keys QED的费曼规则|顶点函数|形状因子

\pentry{QED的重整化理论-电子自能和光子自能的单圈修正\upref{qedlop}}

类似电子自能和光子自能的单圈修正\upref{qedlop}中的讨论,我们先在 OS 重整化方案下进行讨论,用维数正规化处理发散积分。我们常常将 $i\mathcal{M}(e^-(p)\gamma(q)\rightarrow e^-(p'))=\bar u(p') [-ie\Gamma^\mu(p,p')]u(p) \epsilon_\mu(q)$ 称作 QED 的正规顶点,它对应于两条在壳电子外线、一条在壳光子外线的所有截肢费曼图的贡献之和。将 $[-ie\Gamma^\mu(p,p')]$ 称为 QED 的顶点函数,它仍保持着四矢
量的洛伦兹结构,且根据重整化条件,我们希望
\begin{equation}
-ie\Gamma^\mu(p,p')|_{q=0} = -ie\gamma^\mu
\end{equation}
让我们来先分析 $\Gamma^\mu$ 的洛伦兹结构。由于它服从洛伦兹四矢量的变换规则,且是 $4\times 4$ 矩阵,它一定具有以下的形式
\begin{equation}
\begin{aligned}
\Gamma
\end{aligned}
\end{equation}
