% 深度优先搜索算法
% license CCBYSA3
% type Wiki

(本文根据 CC-BY-SA 协议转载自原搜狗科学百科对英文维基百科的翻译)

\textbf{深度优先搜索(DFS)}是用于遍历或搜索树或图数据结构的算法。该算法从根节点开始(在图的情况下选择一些任意节点作为根节点),并在回溯之前尽可能沿着每个分支进行探索。

法国数学家查尔斯·皮埃尔·特雷毛[1]为解决迷宫的策略,在19世纪研究了深度优先搜索。[2][3]

\subsection{属性}
DFS的时间和空间分析因其应用领域而异。在理论计算机科学中,DFS通常用于遍历整个图,并且需要时间为$O(|V| + |E|)$,[4]且图的大小是线性的。在这些应用中,它也使用空间$O(|V|)$,最坏的情况是存储当前搜索路径上的顶点堆栈以及一组已经访问过的顶点。因此,在此设置中,时间和空间界限与广度优先搜索相同,选择使用这两种算法中的哪一种较少取决于它们的复杂度,而更多地取决于这两种算法产生的顶点排序的不同属性。

适用于与特定领域相关的DFS应用,例如在人工智能中搜索解决方案或者网络爬虫,要遍历的图通常要么太大而不能完整访问,要么是无限的(DFS可能会遭受不终止)。在这种情况下,搜索仅在有限深度的情况下进行;由于资源有限,例如内存或磁盘空间,通常不使用数据结构来跟踪所有先前访问过的顶点的集合。当在有限的深度中搜索时,时间在扩展顶点和边的数量方面仍然是线性的(尽管由于一些顶点可能被搜索不止一次,而其他顶点根本不被搜索,导致这个数量与整个图的大小不同),但是DFS的这种变体的空间复杂度仅与深度限制成比例,因此远小于使用广度优先搜索搜索到相同深度所需的空间。对于这样的应用,DFS也更适合选择看起来很可能的分支的启发式方法。当事先不知道合适的深度极限时,迭代深化深度优先搜索以递增的限制顺序重复应用DFS。在人工智能的分析模式下,一个分子因子如果深度大于1,迭代加深会使运行时间增加一个常数因子,在这种情况下,由于每层节点数的几何增长,正确的深度限制是已知的。

DFS也可以用于收集图节点的样本。然而,与不完全的 BFS 相似,不完全的DFS对高度的节点有偏差。

\subsection{例子}
对于下图:
\begin{figure}[ht]
\centering
\includegraphics[width=6cm]{./figures/500a8c3379641f49.png}
\caption\label{fig_SDYXSS_1}
\end{figure}
深度优先搜索从A开始,假设图中的左边缘在右边缘之前已经被选择,并且假设搜索记住以前访问过的节点,并且不会重复搜索(因为这是一个小图),将按以下顺序访问节点: $A,B,D,F,E,C,g$。在这样的边缘遍历搜索下建立Trémaux树,这是一种图论中有重要应用的结构。执行相同的搜索而不记住先前访问的节点会导致访问节点的顺序为$A,B,D,F,E,A,B,D,F,E,$等。永远陷入$A,B,D,F,E,$循环,且永远不会达到$C$或$G$。

迭代深化是避免这种无限循环的一种技术,并将到达所有节点。

\subsection{深度优先搜索的输出}
\begin{figure}[ht]
\centering
\includegraphics[width=8cm]{./figures/e22f0a18cb4996a8.png}
\caption{由生成树定义的四种类型的边} \label{fig_SDYXSS_2}
\end{figure}
图的深度优先搜索的一种方便的描述是用搜索过程中到达的顶点的生成树来表示的。基于这个生成树,原始图的边可以分为三类:\textbf{前边缘},它从树的一个节点指向它的一个后代节点,\textbf{后边缘},它从一个节点指向它先前的节点之一,\textbf{交叉边缘},不使用前边缘和后边缘。有时树边缘属于生成树本身,\textbf{树边缘}的边与前向边分开分类。如果原始图是无向图,那么它的所有边都是树边缘或后边缘。
\subsubsection{3.1 DFS排序}