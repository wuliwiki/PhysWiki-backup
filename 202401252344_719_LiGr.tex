% 线性回归
% keys 回归|线性回归|最小二乘
% license Xiao
% type Tutor

\pentry{函数(高中)\upref{functi},几何矢量的内积\upref{Dot},回归\upref{Regres}}

\textbf{线性回归}(Linear regression)是一种利用线性函数对自变量(特征)和因变量之间的关系进行建模的方法。线性回归是机器学习中一种广泛使用的基本回归算法。含有有多个特征的线性回归称为多元线性回归。

假设有$n$个特征(自变量)$x_1$,$x_2$,...,$x_n$,一个输出变量$y$,线性回归的一般形式表示如下:
\begin{equation}
y=f(\bvec x)=w_1x_1+w_2x_2+...+w_nx_n+b~.
\end{equation}
其中,系数$w_1$,$w_2$,...,$w_n$为特征的权重,$b$为偏置。上式也可以写成向量的形式:
\begin{equation}
y=f(\bvec x)=\bvec w^T\bvec x+b~.
\end{equation}
其中,$\bvec x = [x_1,x_2,...,x_n]$,$\bvec w = [w_1,w_2,...,w_n]$.

\textbf{模型参数:} 在线性回归中,模型的参数包括权重向量$\mathbf{w}$和偏置$b$。这些参数通过学习算法进行调整,以使模型尽可能准确地拟合训练数据。

\textbf{训练过程:} 线性回归的训练过程旨在找到最佳的权重和偏置,以最小化预测值与实际值之间的差距。常用的方法包括最小二乘法和梯度下降法。

一个一元线性回归的示意图如下:
\begin{figure}[ht]
\centering
\includegraphics[width=14.25cm]{./figures/9fc43706c178cfe4.png}
\caption{示意图} \label{fig_LiGr_1}
\end{figure}
图中,蓝色表示数据点,红色直线表示最终求得的线性回归结果。

线性回归适合处理的是数值型数据。但也可以处理标签型数据,此时,须要先对标签数据做连续化预处理。

例如,性别,有两个可能的取值:男、女。为了将性别作为特征送给线性回归使用,可以设男=0,女=1.又如,苹果大小,这个特征,本来有三个标签取值:大、中和小。我们可以令小=1、中=2、大=3,然后再作为特征以便回归使用。

值得注意的是,不是所有标签型数据的特征都可以按照上述方法连续化。考察前面两个例子,其特征的取值是有偏序关系的,因而可以直接按照序关系取数值做连续化。然而,一个没有序关系的特征,就不能直接按照上述方法处理。举个例子,颜色,可能取得的值为:红、绿和蓝。那么,我们就不好直接令红=1,绿=2,蓝=3.因为它们之间没有既定的顺序,如果我们直接赋值$1$、$2$、$3$,就会人为引入一个顺序关系,从而可能会带来问题。此时,可以采用的解决办法是,采用一个三维向量来表示三个特征的取值。设$x$、$y$、$z$分别表示是否为红、绿、蓝。若是,则为1,否则为0.,那么,红、绿、蓝三种颜色可以分别表示为$[1,0,0]$、$[0,1,0]$和$[0,0,1]$。这样做的好处是,避免人为引入一个顺序关系。


\subsection{实战操作}

前面介绍了线性回归的基本概念和理论。接下来,介绍一下在实际应用时,如何对数据做线性回归分析。实际上,有很多现成的软件包能够提供了线性回归功能。我们分别做一些介绍。

\subsubsection{WEKA}

%% 编写计划:新建一个词条,简要介绍一下WEKA软件。然后把那个词条作为本词条的预备知识。

打开软件,点击启动界面右侧的Explorer大按钮,打开数据处理窗口。

第1步,打开待处理数据集。点击Open file...,在弹出的对话框中,选择数据集即可。这里选择cpu.afff数据集。
\begin{figure}[ht]
\centering
\includegraphics[width=12cm]{./figures/08d1acc6381bcf42.png}
\caption{打开数据集} \label{fig_LiGr_5}
\end{figure}

第2步,准备选择建模算法。(1)点击Classify;(2)点击Choose按钮,选择算法。如图2所示。注意,软件里面并没有Regression这一选项卡。分类和回归算法都在Classify这一选项卡中。
\begin{figure}[ht]
\centering
\includegraphics[width=12cm]{./figures/2edea0417e229a1f.png}
\caption{选择算法} \label{fig_LiGr_2}
\end{figure}

第3步,选择线性回归算法。在弹出的算法选择窗口里中,依次点击选择:classifiers->functions->LinearRegression。最后点击右下方Choose按钮确认。流程如图3所示。
\begin{figure}[ht]
\centering
\includegraphics[width=12cm]{./figures/b675ff69b082aad1.png}
\caption{选择线性回归算法} \label{fig_LiGr_3}
\end{figure}

第4步,执行操作。点击Start按钮即可。
\begin{figure}[ht]
\centering
\includegraphics[width=10cm]{./figures/7649016a6333b5e9.png}
\caption{执行操作} \label{fig_LiGr_4}
\end{figure}

软件立即用线性回归方法进行建模。稍后,右侧会显示分析结果。线性回归方程为:$class=0.0491 \times MYCT+0.0152 \times MMIN+0.0056 \times MMAX+0.6298 \times CACH+1.4599 \times CHMAX-56.075$.如图6所示,图中红框标注出的就是回归直线方程。
\begin{figure}[ht]
\centering
\includegraphics[width=14cm]{./figures/d677927d0dc4894f.png}
\caption{分析结果} \label{fig_LiGr_6}
\end{figure}
输出结果中还显示了数据集的简单统计信息和模型误差、标准差等信息。