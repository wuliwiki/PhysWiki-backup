% 相控阵
% license CCBYSA3
% type Wiki

(本文根据 CC-BY-SA 协议转载自原搜狗科学百科对英文维基百科的翻译)

\begin{figure}[ht]
\centering
\includegraphics[width=10cm]{./figures/67b1f779e41710d7.png}
\caption{相控阵的工作原理动态图它由一个由发射机(TX)供电的天线单元阵列(A)组成。每个天线的馈电电流通过由计算机(C)控制的移相器(φ)提供。移动的红线是每个天线单元发射的无线电波的波阵面示意图。单个波阵面是球形的,但是它们在天线前组合(叠加)形成一束在特定方向传播的平面波。移相器使无线电波在线路上依次延迟,因此每个天线发射波前的时间比它下面的天线晚。这导致产生的平面波与天线轴成θ角。通过改变相移,计算机可以立即改变光束的角度θ。大多数相控阵都有二维天线阵列,而不是上图的线性阵列,波束可以在二维方向上转向。无线电波的传播速度在视觉上减慢了。} \label{fig_XKZ_1}
\end{figure}

在天线理论中,相控阵通常指的是电子扫描阵列,这是一种由计算机控制的天线阵列,它产生的无线电波束可以在无需移动天线的条件下控制其指向不同的方向。[1][2][3][4][5][6][7][8] 在阵列天线中,来自发射机的射频电流以正确的相位关系馈送到各个天线,使得来自各个天线的无线电波相加在一起以增加预定方向的辐射增益,同时抵消以抑制其他方向的辐射。在相控阵中,来自发射机的功率通过被称为移相器的设备馈送到天线,移相器由计算机系统控制,计算机系统可以改变相位,从而将无线电波束导向不同的方向。由于阵列必须由许多小天线(有时几千个)组成才能获得高增益,相控阵主要适用于无线电频谱的高频端、超高频和微波波段,在这些波段中单个天线元件非常小。

相控阵被发明用于军事雷达系统,可以快速扫描雷达波束,以探测飞机和导弹。这些相控阵雷达系统现在得到了广泛的使用,相控阵正在向到民用领域扩展。相控阵原理也用于声学,相控阵声波传感器用于医学超声成像扫描仪(相控阵超声)、油气勘探(反射地震学)和军事声纳系统。[9]

术语“相控阵”在偶尔也用于馈电功率的相位以及天线阵列的辐射方向图固定的非阵列天线。[6][10] 例如,通过馈电产生特定的辐射模式的调幅AM广播无线电天线由多个主辐射器组成,也称为“相控阵”。

\subsection{类型}
无源相控阵或无源电子扫描阵(PESA)是一种相控阵,其中天线元件连接到单个发射机和/或接收机,如顶部动画所示。PESA是最常见的相控阵类型。

有源相控阵或有源电子扫描阵(AESA)是一种相控阵,其中每个天线单元都有自己的发射机/接收机单元,全部由计算机控制。有源阵列是一种更先进的第二代相控阵技术,广泛用于军事领域;与PESA不同,它们可以同时向不同方向辐射多个频率的多束无线电波。

共形天线是一种相控阵,其中各个天线不是布置在平面上,而是安装在曲面上。移相器补偿由于天线元件在表面上的不同位置而导致的不同路径长度的波,允许阵列辐射平面波。共形天线用于飞机和导弹,将天线集成到飞机的弯曲表面以减少空气动力阻力。

\subsection{历史}
\begin{figure}[ht]
\centering
\includegraphics[width=8cm]{./figures/37938c249e3db96d.png}
\caption{上图是费迪南·布劳恩(Ferdinand Braun)1905年的定向天线,采用相控阵原理,由3个单极天线组成一个等边三角形。一个天线馈线的四分之一波长延迟导致阵列以波束形式辐射。延迟可以手动切换到3个馈电中的任何一个,从而将天线波束旋转120°。} \label{fig_XKZ_2}
\end{figure}
\begin{figure}[ht]
\centering
\includegraphics[width=6cm]{./figures/a05705147b19f7da.png}
\caption{美国阿拉斯加PAVE PAWS有源相控阵弹道导弹探测雷达。它于1979年建成,是首批有源相控阵之一。} \label{fig_XKZ_3}
\end{figure}
相控阵传输最初是由诺贝尔奖获得者卡尔·费迪南德·布劳恩(Nobel laureate Karl Ferdinand Braun )在1905年展示的,他展示了无线电波在一个方向上的增强传输。[11][12] 二战期间,诺贝尔奖获得者路易斯·阿尔瓦雷斯(Luis Alvarez )在一个用于“地面控制方法”的快速可控雷达系统中使用相控阵传输,该系统有助于飞机着陆。与此同时,德国的GEMA建造了马穆特(Mammut 1)1号。[13]后来,在剑桥大学开发了几个大型相控阵之后,它被用于射电天文学,并为安东尼·休伊什和马丁·赖尔(Antony Hewish and Martin Ryle)赢得了诺贝尔物理学奖。这种设计也用于雷达,并在干涉测量天线中得到推广。
\begin{figure}[ht]
\centering
\includegraphics[width=6cm]{./figures/0721f1d47aad3ada.png}
\caption{组成平面阵列的一些2677交叉偶极天线单元的特写。这种天线产生的窄“铅笔”波束只有2.2度宽} \label{fig_XKZ_4}
\end{figure}
\begin{figure}[ht]
\centering
\includegraphics[width=6cm]{./figures/c54370eeb5d02cd3.png}
\caption{美国F-22猛禽战斗机机头内的有源相控阵雷达天线。几乎所有战斗机现在都使用相控阵雷达。} \label{fig_XKZ_5}
\end{figure}
2004年,加州理工学院的研究人员展示了第一台集成硅基相控阵接收机(24GHz,8个单元)。[14]随后,加州理工学院团队于2005年演示了一款24GHz的互补金属氧化物半导体(CMOS)相控阵发射机,[15]并于2006年演示了一款集成天线的完全集成的77GHz相控阵收发机。[16][17] 2007年,美国国防部高级研究计划局(DARPA)的研究人员宣布了一种16元相控阵雷达天线,它也与所有相关电路集成在一个硅片上,工作频率为30-50 GHz。[18][18]
\begin{figure}[ht]
\centering
\includegraphics[width=6cm]{./figures/238447fbbe4fbba4.png}
\caption{美国F-22猛禽战斗机机头内的有源相控阵雷达天线。几乎所有战斗机现在都使用相控阵雷达。} \label{fig_XKZ_6}
\end{figure}
BMEWS & PAVE PAWS雷达
单个天线辐射的信号相对幅度以及它们之间的叠加和相消干涉效应决定了阵列的有效辐射方向图。相控阵可以指向固定的辐射方向,或者在方位角或仰角上快速扫描。1957年,加利福尼亚休斯飞机公司(Hughes Aircraft Company, California)首次演示了方位和仰角的同时电子扫描的相控阵天线。[19]
\begin{figure}[ht]
\centering
\includegraphics[width=6cm]{./figures/dea4cc241f728c6a.png}
\caption{第二次世界大战的Mammut相控阵雷达} \label{fig_XKZ_7}
\end{figure}

\subsection{ 应用程序}
\subsubsection{3.1 广播}
在广播工程中,许多调幅广播电台使用相控阵来增强信号强度,从而扩大许可城市的覆盖范围,同时最大限度地减少对其他地区的干扰。由于中波频率下白天和夜间电离层传播的差异,调幅广播电台通常在日出和日落时通过切换提供给天线单元(主辐射器)的相位和功率级别,使天线在白天(地波)和夜晚(天波)辐射模式之间进行改变。对于短波广播,许多电台使用水平偶极子阵列。常见的排列是在4×4阵列中使用16个偶极子。通常这些阵子位于一个线栅反射器前面。相位通常是可切换的,以允许波束在方位角和仰角上转向。

私人无线电爱好者可以使用更适中的相控阵长线天线系统来从远距离接收长波、中波和短波无线电广播。

在甚高频上,相控阵广泛用于调频广播。这极大地增加了天线增益,将发射的射频能量沿着地平线放大,从而极大地增加了电台的广播范围。在这些情况下,从发射器到每个元件的距离是相同的,或者相隔一个(或整数倍)波长。调整阵列相位,使较低的元件延迟(使到它们的距离变长)会导致波束向下倾斜,如果天线在很高的无线电塔上,这种效果将会很明显。

其他相位调整可以在不倾斜主瓣的情况下增加远场向下辐射的能量,为极高的山顶位置创造零填充,或者减少近场的辐射,以防止那些工人或者甚至附近的居民过度暴露于辐射中。后一种效果也可以通过半波间距实现——在全波间距的现有单元中间插入附加单元。这种定相实现了与全波间隔大致相同的水平增益;也就是说,五元全波间隔阵列等于九元或十元半波间隔阵列。

\subsubsection{3.2 雷达}
许多海军的战舰也使用相控阵雷达系统。由于波束可以快速转向,相控阵雷达允许军舰使用一个雷达系统进行水面探测和跟踪(寻找船只)、空中探测和跟踪(寻找飞机和导弹)以及导弹发射引导。很久之前,飞行中的每一枚地对空导弹都需要一个专用的火控雷达,这意味着雷达制导武器只能同时攻击少量目标。相控阵系统可用于在导弹飞行的中期阶段控制导弹。在飞行的最后阶段,连续波火控控制器为目标提供最终的制导。因为雷达波束是电子操纵的,相控阵系统可以迅速引导雷达波束,从而在控制几枚飞行中导弹的同时,并保持对多目标的火控跟踪。

AN/SPY-1相控阵雷达是部署在现代美国巡洋舰和驱逐舰上的宙斯盾作战系统的一部分,“能够同时执行搜索、跟踪和导弹制导功能,拥有跟踪超过100个目标的能力。”[20]同样,服务于法国、俄罗斯和新加坡的泰勒斯·赫刺克勒斯(Thales Herakles)相控阵多功能雷达具有200个目标的跟踪能力,能够在一次扫描中实现自动目标探测、确认和启动跟踪,同时为从船上发射的MBDA Aster导弹提供中途制导。[21]德国海军和荷兰皇家海军开发了有源相控阵雷达系统(APAR)。MIM-104爱国者和其它地面防空系统使用相控阵雷达也有类似的优势。
\begin{figure}[ht]
\centering
\includegraphics[width=6cm]{./figures/ad8d5c95f0703340.png}
\caption{有源相控阵雷达安装在德国海军萨克森级护卫舰(Sachsen-class frigate)F220汉堡的顶部相控阵也应用于海军有源(发射和接收)和无源(仅接收)声纳以及船体和拖曳阵列声纳。} \label{fig_XKZ_8}
\end{figure}

\subsubsection{3.3 空间探测器通信}
信使号是执行对水星空间探测任务的宇宙飞船(2011-2015[22])。这是首次使用相控阵天线进行通信的深空探测任务。辐射元件是圆极化的开槽波导。这种天线使用的是X波段,使用了26种辐射单元,并且可以适度减少单元个数。[23]

\subsubsection{3.4 天气研究用途}
自2003年4月23日以来,国家强风暴实验室一直在俄克拉荷马州诺曼使用美国海军提供的SPY-1A相控阵天线进行天气研究。人们希望这项研究对雷暴和龙卷风产生更好的理解,从而增加对恶劣天气的预警时间和对龙卷风预测概率。目前的项目参与者包括国家强风暴实验室和国家气象服务雷达中心、洛克希德·马丁公司、美国海军、俄克拉何马大学气象学院、电气和计算机工程学院和大气雷达研究中心、俄克拉何马州高等教育董事、联邦航空管理局以及基础商业和工业。该项目包括研发、未来技术转让以及该系统在美国的潜在部署。预计需要10至15年才能完成,初步建设费用约为2500万美元。[24]日本RIKEN高级计算科学研究所(AICS)的一个团队已经开始了使用相控阵雷达和一种新算法进行即时天气预报的实验工作。[25]
\begin{figure}[ht]
\centering
\includegraphics[width=6cm]{./figures/f2e0e4f298a2c36b.png}
\caption{俄克拉荷马州诺曼市(Norman, Oklahoma)国家强风暴实验室AN/SPY-1A雷达装置。封闭的天线罩为天线提供保护。} \label{fig_XKZ_9}
\end{figure}

\subsubsection{3.5 光学}
在电磁波的可见光或红外光谱范围内,可以构建光学相控阵。它们可用于波长多路复用器和滤波器、[26]激光束控制和全息术。合成阵列外差检测是将整个相控阵复用到单个元件光电探测器上的有效方法。光学相控阵发射器中的动态光束形成可用于电子光栅或矢量图像扫描,而无需使用透镜或无透镜投影仪中的机械移动部件。[27]光学相控阵接收器已经被证明能够通过选择性地观察不同的方向来充当无透镜相机。[28][29]

\subsubsection{3.6 宽带卫星互联网收发器}
OneWeb和Starlink是两个低地球轨道卫星,截至2019年仍在建设中。它们旨在为消费者提供宽带互联网连接;两个系统的用户终端均使用相控阵天线。[30][31]

\subsubsection{3.7 射频识别(RFID)}
到2014年,相控阵天线被集成到射频识别系统中,将单个系统的覆盖面积增加100\%,达到76200平方米(820000平方英尺),同时仍然使用传统的无源超高频标签。[32]

\subsubsection{3.8 人机界面(HMI)}
2008年,东京大学信田实验室(University of Tokyo's Shinoda Lab)开发了一种相控阵声波传感器,称为机载超声触觉显示器(AUTD),用于诱导触觉反馈。[33]这个系统被证明能使用户对虚拟全息对象进行交互操作。[34]

\subsection{数学观点和公式}
\begin{figure}[ht]
\centering
\includegraphics[width=6cm]{./figures/99a04f230ebbbf5d.png}
\caption{上图是相控阵的辐射图,其中包含7个相隔四分之一波长的发射器,显示光束切换方向。相邻发射器之间的相移从45度切换到45度} \label{fig_XKZ_10}
\end{figure}
\begin{figure}[ht]
\centering
\includegraphics[width=14.25cm]{./figures/97127c5016271f4d.png}
\caption{极坐标下的相控阵辐射方向图} \label{fig_XKZ_11}
\end{figure}

