% 哥德尔不完备定理(综述)
% license CCBYSA3
% type Wiki

本文根据 CC-BY-SA 协议转载翻译自维基百科\href{https://en.wikipedia.org/wiki/G\%C3\%B6del\%27s_incompleteness_theorems}{相关文章}。

哥德尔的不完全性定理是数学逻辑中的两个定理,涉及形式公理化理论中可证明性的极限。这些结果由库尔特·哥德尔在1931年发布,在数学逻辑和数学哲学中都具有重要意义。这些定理被广泛地,但并非普遍地解释为,证明了希尔伯特寻找一个完整且一致的公理集合来描述所有数学的计划是不可能实现的。

第一个不完全性定理声明,任何一个一致的公理系统,只要其定理可以通过有效程序(即算法)列出,都无法证明关于自然数算术的所有真理。对于任何这样的形式系统,总会存在一些关于自然数的陈述,这些陈述是正确的,但在该系统内无法证明。

第二个不完全性定理,是第一个定理的扩展,表明该系统无法证明自身的一致性。

通过使用对角线论证,哥德尔的不完全性定理是首批关于形式系统局限性的紧密相关定理之一。随后,塔尔斯基提出了真理的形式不可定义性定理,丘奇证明了希尔伯特的判定问题是不可解的,图灵的定理表明不存在可以解决停机问题的算法。
\subsection{形式系统:完整性、一致性和有效公理化} 
不完全性定理适用于那些足够复杂的形式系统,这些系统能够表达自然数的基本算术,并且是一致的且具有有效的公理化。特别是在一阶逻辑的背景下,形式系统也被称为形式理论。一般来说,形式系统是一个推理工具,由一组特定的公理以及符号操作规则(或推理规则)组成,这些规则允许从公理推导出新的定理。一个这样的系统的例子是一阶皮亚诺算术系统,这是一个所有变量都指代自然数的系统。在其他系统中,如集合论,只有一些形式系统中的句子表达关于自然数的陈述。不完全性定理涉及的是这些系统内的形式可证明性,而不是“非正式意义上的可证明性”。

形式系统可能具有几个属性,包括完整性、一致性和有效公理化的存在。不完全性定理表明,包含足够算术内容的系统无法同时具备这三种属性。
\subsubsection{有效公理化}  
如果一个形式系统的定理集合是递归可枚举的,则该系统被称为有效公理化(也称为有效生成)。这意味着存在一个计算机程序,理论上可以枚举该系统的所有定理,而不会列出任何非定理的陈述。有效生成的理论的例子包括皮亚诺算术和泽梅洛–弗兰克尔集合论(ZFC)。\(^\text{[1]}\)

被称为真算术的理论包括在皮亚诺算术语言中关于标准整数的所有真陈述。该理论是一致且完整的,并包含足够的算术内容。然而,它没有递归可枚举的公理集合,因此不满足不完全性定理的假设。
\subsubsection{完整性}  
一组公理是(语法上或否定上)完整的,如果对于公理语言中的任何陈述,该陈述或其否定可以从这些公理中证明出来。\(^\text{[2]}\)这是与哥德尔的第一个不完全性定理相关的概念。它不应与语义完整性混淆,语义完整性意味着该公理集合能证明给定语言中的所有语义重言式。在他的完整性定理中(不应与这里描述的不完全性定理混淆),哥德尔证明了一阶逻辑在语义上是完整的。但它不是语法完整的,因为在一阶逻辑的语言中,有些句子既不能从逻辑的公理中证明,也不能反驳。

在一个数学系统中,像希尔伯特这样的思想家认为,找到这样一种公理化方法,使得可以通过证明其否定来证明或反驳每个数学公式,迟早只是时间问题。

一个形式系统可能是有意设计为语法不完整的,正如逻辑通常那样。或者它可能是不完整的,仅仅是因为并没有发现或包含所有必要的公理。例如,没有平行公设的欧几里得几何是不完整的,因为语言中的一些陈述(例如平行公设本身)不能从其余的公理中证明出来。同样,稠密线性序的理论是不完整的,但在添加一个额外的公理,即在序列中没有端点的公理后,它变得完整。连续统假设是ZFC语言中的一个陈述,在ZFC中无法证明,因此ZFC不是完整的。在这种情况下,没有显而易见的候选公理能够解决这个问题。

一阶皮亚诺算术的理论似乎是一致的。假设这确实是正确的,请注意它有一个无限但递归可枚举的公理集合,并且可以编码足够的算术内容以符合不完全性定理的假设。因此,根据第一个不完全性定理,皮亚诺算术不是完整的。该定理给出了一个算术陈述的明确示例,该陈述在皮亚诺算术中既不能被证明,也不能被反驳。此外,这个陈述在通常模型中为真。此外,没有有效公理化的一致扩展皮亚诺算术能够是完整的。
\subsubsection{一致性}  
一组公理是(简单地说)一致的,如果没有任何陈述既能从公理中证明,又能证明其否定,否则该公理集合是不一致的。也就是说,一个一致的公理系统是没有矛盾的。

皮亚诺算术可以从ZFC中证明其一致性,但不能从自身内部证明。类似地,ZFC不能从自身内部证明其一致性,但ZFC + “存在一个不可接近的基数”证明了ZFC是一致的,因为如果\( \kappa \)是最小的这样的基数,那么\( V_\kappa \) 位于冯·诺伊曼宇宙内部是ZFC的一个模型,而一个理论是一致的,当且仅当它有一个模型。

如果将皮亚诺算术语言中的所有陈述都作为公理,那么这个理论是完整的,具有递归可枚举的公理集合,并且可以描述加法和乘法。然而,它不是一致的。

不一致理论的其他例子来自于集合论中,当假设无限制的理解公理 schema 时所产生的悖论。
\subsubsection{包含算术的系统}
不完全性定理仅适用于能够证明关于自然数的足够多事实的形式系统。一个足够的集合是罗宾逊算术Q的定理集合。一些系统,如皮亚诺算术,可以直接表达关于自然数的陈述。其他系统,如ZFC集合论,能够将关于自然数的陈述翻译成它们的语言。这两种选项都适用于不完全性定理。

给定特征的代数封闭域的理论是完整的、一致的,并且具有一个无限但递归可枚举的公理集合。然而,无法将整数编码到这个理论中,而且该理论不能描述整数的算术。一个类似的例子是实闭域的理论,它本质上等同于塔尔斯基的欧几里得几何公理。所以欧几里得几何本身(在塔尔斯基的表述中)是一个完整、一致、有效公理化的理论的例子。

普雷斯堡算术系统由一组自然数的公理组成,仅包含加法运算(乘法被省略)。普雷斯堡算术是完整的、一致的、递归可枚举的,能够编码加法但不能编码自然数的乘法,这表明对于哥德尔的定理,理论不仅需要编码加法,还需要编码乘法。

丹·威拉德(Dan Willard,2001年)研究了一些弱的算术系统家族,这些系统允许足够的算术作为关系来形式化哥德尔编号,但不够强大以使乘法成为一个函数,因此无法证明第二不完全性定理;也就是说,这些系统是一致的,并且能够证明它们自身的一致性(见自验证理论)。
\subsubsection{相互冲突的目标} 
在选择一组公理时,其中一个目标是能够证明尽可能多的正确结果,而不证明任何错误的结果。例如,我们可以想象一组正确的公理,它们允许我们证明关于自然数的每一个正确的算术命题(Smith 2007,第2页)。在标准的一阶逻辑系统中,一个不一致的公理集合将证明它语言中的每个命题(这有时被称为爆炸原理),因此它自动是完整的。然而,一个既完整又一致的公理集合证明的是一组最大的非矛盾定理。

在前面几节中用皮亚诺算术、ZFC和ZFC + "存在一个不可接近的基数"所展示的模式通常无法打破。这里,ZFC + "存在一个不可接近的基数"无法从自身证明一致性。它也不是完整的,如连续统假设所示,这在ZFC + "存在一个不可接近的基数"中是无法解决的。\(^\text{[3]}\)

第一个不完全性定理表明,在可以表达基本算术的形式系统中,永远不可能创建一个完整且一致的有限公理列表:每次添加一个额外的、一致的陈述作为公理时,仍然存在其他正确的陈述,即使在有了新公理的情况下也无法证明。如果添加的一个公理使得系统完整,那么它的代价就是使系统变得不一致。甚至不可能有一个无限的公理列表是完整、一致且有效公理化的。
\subsection{第一个不完全性定理}  
哥德尔的第一个不完全性定理首次出现在哥德尔1931年论文《关于《数学原理》及相关系统中的形式不可判定命题 I》的“定理 VI”中。定理的假设在随后的时间里被J. 巴克利·罗斯尔(1936年)通过使用罗斯尔技巧改进。结合罗斯尔的改进,得出的定理可以用以下英语表述,其中“形式系统”包括假设系统是有效生成的。

\textbf{第一个不完全性定理:}“任何一致的形式系统\( F \),只要该系统能够进行一定量的基本算术运算,都是不完整的;即在\( F \)的语言中存在一些命题,这些命题在\( F \)中既无法证明也无法反驳。”(Raatikainen 2020)

定理所提到的不可证明命题\( G_F \)通常被称为系统\( F \)的“哥德尔句子”。证明构造了一个特定的哥德尔句子\( G_F \)用于系统\( F \),但在系统语言中有无限多的命题具有相同的特性,例如哥德尔句子与任何逻辑有效命题的合取。

每个有效生成的系统都有自己的哥德尔句子。可以定义一个更大的系统\( F' \),它包含整个\( F \)以及\( G_F \)作为额外的公理。这样不会得到一个完整的系统,因为哥德尔定理也适用于\( F' \),因此 \( F' \)也不能是完整的。在这种情况下,\( G_F \)确实是\( F' \)中的一个定理,因为它是一个公理。由于\( G_F \) 仅说明它在\( F \)中不可证明,因此它在 \( F' \) 中可证明并不产生矛盾。然而,由于不完全性定理适用于\( F' \),会有一个新的哥德尔命题\( G_{F'} \)对应于\( F' \),从而表明\( F' \)也是不完整的。\( G_{F'} \)将与\( G_F \)不同,因为\( G_{F'} \)将指向\( F' \),而不是\( F \)。
\subsubsection{哥德尔句子的语法形式} 
哥德尔句子的设计是间接地自指。该句子声明,当使用特定的步骤序列构造另一个句子时,该构造的句子在系统\( F \)中不可证明。然而,这些步骤的序列是如此安排,以至于构造出的句子实际上就是\( GF \) 本身。通过这种方式,哥德尔句子\( GF \)间接地声明它在\( F \)中不可证明。\(^\text{[4]}\)

为了证明第一个不完全性定理,哥德尔展示了系统内可证明性的概念可以完全用算术函数来表示,这些算术函数作用于系统句子的哥德尔编号。因此,系统不仅可以证明有关数字的某些事实,还可以间接证明关于其自身陈述的事实,前提是它是有效生成的。关于系统内句子的可证明性的问题被表示为关于数字本身的算术性质的问题,如果系统是完整的,这些问题是可以决定的。

因此,尽管哥德尔句子间接地指向系统\( F \)的句子,但当它被作为算术陈述来解读时,哥德尔句子仅直接指向自然数。它断言没有自然数具有某个特定的属性,这个属性由一个原始递归关系给出(Smith 2007,第141页)。因此,哥德尔句子可以用算术语言以简单的语法形式写出。特别地,它可以表示为算术语言中的一个公式,该公式由若干个前导的全称量词后跟无量词体(这些公式属于算术层级中的\( \Pi_1^0 \)级别)。通过MRDP定理,哥德尔句子可以被重写为一个声明,即当将整数代入其变量时,特定的多变量整数系数的多项式永远不会取零值(Franzén 2005,第71页)。
\subsubsection{哥德尔句子的真值}  
第一个不完全性定理表明,适当形式理论\( F \)哥德尔句子\( G_F \)在\( F \)中是不可证明的。因为当它被解释为关于算术的陈述时,这种不可证明性正是该句子(间接)所断言的内容,因此哥德尔句子实际上是真的(Smoryński 1977,第825页;另见 Franzén 2005,第28-33页)。因此,哥德尔句子\( G_F \)通常被称为“真实但不可证明”(Raatikainen 2020)。然而,由于哥德尔句子无法正式指定其预期的解释,句子\( G_F \) 的真值只能通过来自系统外部的元分析来得出。通常,这种元分析可以在一个被称为原始递归算术的弱形式系统中进行,该系统证明了命题\( \text{Con}(F) \rightarrow G_F \),其中\( \text{Con}(F) \)是一个典型的陈述,断言\( F \)的一致性(Smoryński 1977,第840页;Kikuchi & Tanaka 1994,第403页)。

尽管一致理论的哥德尔句子作为关于算术预期解释的陈述是正确的,但哥德尔句子在一些非标准算术模型中是错误的,这是哥德尔完整性定理的一个结果(Franzén 2005,第135页)。该定理表明,当一个句子独立于一个理论时,该理论将具有既包含句子为真的模型,也包含句子为假的模型。如前所述,系统\( F \)的哥德尔句子是一个算术陈述,声称不存在具有某个特定属性的数字。不完全性定理表明,这个断言将独立于系统\( F \),而哥德尔句子的真值来自于没有任何标准自然数具有所讨论的属性这一事实。任何使哥德尔句子为假的模型必须包含某个元素,该元素在该模型中满足该属性。这样的模型必须是“非标准的”——它必须包含一些不对应于任何标准自然数的元素(Raatikainen 2020,Franzén 2005,第135页)。
\subsubsection{与说谎者悖论的关系} 
哥德尔在其论文《关于《数学原理》及相关系统中的形式不可判定命题 I》的引言部分中特别提到了理查德悖论和说谎者悖论,作为他语法不完全性结果的语义类比。说谎者悖论是“这个句子是假的”这句话。对这个说谎者句子的分析表明,它既不能为真(因为如果它为真,如它所断言的那样,它是假的),也不能为假(因为如果它为假,那么它就为真)。哥德尔句子 \( G \)对于系统\( F \)做出了类似于说谎者句子的断言,但将“真”替换为“可证明”:\( G \)说“\( G \)在系统\( F \)中不可证明”。对\( G \)的真值和可证明性的分析是对说谎者句子真值分析的形式化版本。

在哥德尔句子中,不能将“不可证明”替换为“假”,因为谓词“\( Q \)是一个假公式的哥德尔编号”不能表示为算术公式。这个结果被称为塔尔斯基的不可定义性定理,它是由哥德尔在研究不完全性定理的证明时独立发现的,并且塔尔斯基也独立发现了这一定理。
\subsubsection{哥德尔原始结果的扩展} 
与哥德尔1931年论文中陈述的定理相比,许多当代的不完全性定理表述在两个方面更加普遍。这些广义的表述被措辞为适用于更广泛的系统类别,并且被措辞为包含较弱的一致性假设。

哥德尔证明了《数学原理》系统的不完全性,这是一个特定的算术系统,但可以为任何具有一定表达能力的有效系统给出类似的证明。哥德尔在他的论文引言中提到这一事实,但为了具体性将证明限制在一个系统中。在现代的不完全性定理表述中,通常将有效性和表达能力条件作为不完全性定理的假设陈述,因此它不再局限于任何特定的形式系统。在1931年哥德尔发表其结果时,表述这些条件所需的术语尚未发展出来。

哥德尔原始的不完全性定理的陈述和证明要求假设系统不仅是一致的,而且是\( \omega \)-一致的。如果一个系统不是\( \omega \)-不一致的,则它是\( \omega \)-一致的;而如果存在一个谓词\( P \),使得对于每个特定的自然数\( m \),系统证明\( \neg P(m) \),但同时系统也证明存在一个自然数\( n \)使得\( P(n) \)成立,则该系统是\( \omega \)-不一致的。也就是说,系统说存在一个具有属性\( P \)的数字,但否认它具有任何具体的值。系统的\( \omega \)-一致性意味着它的一致性,但一致性并不意味着\( \omega \)-一致性。J. 巴克利·罗斯尔(1936年)通过找到一种证明的变式(罗斯尔技巧)加强了不完全性定理,这种变式只要求系统是一致的,而不是\( \omega \)-一致的。这主要具有技术性意义,因为所有关于算术的真实形式理论(其公理是关于自然数的所有真实命题)都是\( \omega \)-一致的,因此原始表述的哥德尔定理适用于这些理论。现在,假设系统仅一致,而非\( \omega \)-一致的更强版本的不完全性定理通常被称为哥德尔不完全性定理,也叫做哥德尔–罗斯尔定理。
\subsection{第二个不完全性定理}
对于每个包含基本算术的形式系统\( F \),可以规范地定义一个公式\( \text{Cons}(F) \),表达\( F \)的一致性。这个公式表达了“不存在一个自然数,编码了在系统\( F \)中的形式推导,其结论是一个语法上的矛盾”的属性。语法上的矛盾通常被认为是“0=1”,在这种情况下,\( \text{Cons}(F) \)表示“没有自然数编码从\( F \)的公理推导出‘0=1’”。

\textbf{哥德尔的第二个不完全性定理表明},在一般假设下,这个规范的一致性陈述\( \text{Cons}(F) \)在\( F \)中是无法被证明的。该定理首次出现在哥德尔1931年的论文《关于《数学原理》及相关系统中的形式不可判定命题 I》的“定理 XI”中。在以下表述中,“形式化系统”一词还包括假设\( F \)是有效公理化的。这个定理声明,对于任何一个一致的系统\( F \),如果该系统能够进行一定量的基本算术运算,则\( F \)的一致性不能在\( F \)本身中得到证明。\(^\text{[5]}\)这个定理比第一个不完全性定理更强,因为第一个不完全性定理中构造的陈述并没有直接表达系统的一致性。第二个不完全性定理的证明是通过在系统\( F \)本身中形式化第一个不完全性定理的证明得到的。
\subsubsection{表达一致性} 
在第二个不完全性定理中,有一个技术性的微妙之处,涉及如何在\( F \)的语言中将一致性表达为一个公式。表达系统一致性的方法有很多种,并不是所有方法都能得出相同的结果。第二个不完全性定理中的公式\( \text{Cons}(F) \)是一致性的一个特定表达。

一致性声明的其他形式化可能在\( F \)中是不等价的,其中一些甚至可能是可证明的。例如,一阶皮亚诺算术(PA)可以证明“PA的最大一致子集是一致的”。但是,因为PA是一致的,PA的最大一致子集实际上就是PA,因此在这个意义上,PA“证明了它自己是一致的”。但PA无法证明的是,PA的最大一致子集实际上就是整个PA。(这里所说的“PA的最大一致子集”是指在某个特定有效枚举下,PA公理的最大一致初始段。)
\subsubsection{希尔伯特–伯奈斯条件}  
第二个不完全性定理的标准证明假设可证明性谓词\( Prov_{A}(P) \)满足希尔伯特–伯奈斯可证明性条件。设\( \#(P) \)代表公式\( P \)的哥德尔编号,可证明性条件如下:
\begin{enumerate}
\item 如果 \( F \) 证明 \( P \),则 \( F \) 证明 \( Prov_A(\#(P)) \)。
\item \( F \) 证明 1;即 \( F \) 证明 \( Prov_A(\#(P)) \rightarrow Prov_A(\#(Prov_A(\#(P)))) \)。
\item \( F \) 证明 \( Prov_A(\#(P \rightarrow Q)) \land Prov_A(\#(P)) \rightarrow Prov_A(\#(Q)) \)(这是模态推理的类似形式)。
\end{enumerate}
有些系统,如罗宾逊算术,足够强大,能够满足第一个不完全性定理的假设,但并不证明希尔伯特–伯奈斯条件。然而,皮亚诺算术足够强大,能够验证这些条件,所有比皮亚诺算术更强的理论也能满足这些条件。
\subsubsection{一致性证明的含义}  
哥德尔的第二个不完全性定理还意味着,满足上述技术条件的系统\( F_1 \)不能证明任何系统\( F_2 \)的一致性,尤其是\( F_2 \)证明了\( F_1 \)的一致性。这是因为这样的系统\( F_1 \)可以证明,如果\( F_2 \)证明\( F_1 \)的一致性,那么\( F_1 \)实际上是一致的。关于\( F_1 \)一致性的声明具有“对于所有数字\( n \),\( n \)具有可判定的属性,即不是\( F_1 \)中矛盾证明的代码”这样的形式。如果\( F_1 \)确实不一致,那么\( F_2 \)会证明对于某个\( n \),\( n \)是\( F_1 \) 中矛盾的代码。但如果\( F_2 \)也证明\( F_1 \) 是一致的(即不存在这样的\( n \)),那么它本身就会不一致。这一推理可以在\( F_1 \) 中形式化,证明如果\( F_2 \)一致,那么\( F_1 \)也一致。由于根据第二个不完全性定理,\( F_1 \)并不能证明自己的统一性,因此它也不能证明\( F_2 \)的一致性。

这个第二个不完全性定理的推论表明,举例来说,无法通过任何可以在皮亚诺算术(PA)中形式化的有限方法来证明皮亚诺算术的一致性。举例来说,广泛接受的原始递归算术(PRA)系统被认为是有限数学的准确形式化,它在PA中是可证明一致的。因此,PRA不能证明PA的一致性。这个事实通常被认为意味着希尔伯特的计划无法执行,希尔伯特的计划旨在通过给出一个有限的证明来证明“理想”(无穷)数学原则在“真实”(有限)数学命题证明中的一致性,进而证明理想原则的一致性。\(^\text{[6]}\)

如果一个系统\( F \)证明了自己的一致性,这不会提供任何有趣的信息。因为不一致的理论可以证明一切,包括它们的一致性。因此,在\( F \)中进行的一致性证明不会为我们提供关于\( F \)是否一致的任何线索;通过这样的证明,关于\( F \)一致性的任何疑问都不会得到解决。一致性证明的兴趣在于能够在某个在某种意义上比\( F \)本身更少存疑的系统\( F' \)中证明系统\( F \)的一致性,例如,\( F' \)比\( F \)弱。对于许多自然发生的理论\( F \)和\( F' \),例如\( F = \)泽梅洛–弗兰克尔集合论和\( F' = \)原始递归算术,\( F' \)的一致性可以在\( F \)中证明,因此根据第二个不完全性定理的推论,\( F' \)无法证明\( F \)的一致性。

第二个不完全性定理并没有完全排除在不同的公理系统中证明一致性的可能性。例如,格哈德·根岑在一个包含断言序数\( \varepsilon_0 \)是良基的公理的不同系统中证明了皮亚诺算术的一致性;见根岑的一致性证明。根岑的定理促使了证明理论中序数分析的发展。
\subsection{不可判定命题的例子}  
在数学和计算机科学中,"不可判定"一词有两种不同的含义。第一种含义是与哥德尔定理相关的证明理论含义,即一个命题在指定的推理系统中既不能被证明也不能被反驳。第二种含义(在此不讨论)与可计算性理论相关,适用于决策问题,而不是命题。决策问题是一个可数无限集,每个问题都需要一个“是”或“否”的回答。如果没有可计算的函数能正确回答问题集中的每个问题,则称该问题是不可判定的(见不可判定问题)。

由于"不可判定"一词有这两层含义,"独立"一词有时被用来代替"不可判定"以表示"既不能被证明也不能被反驳"的含义。

在特定推理系统中,命题的不可判定性本身并没有解决该命题的真值是否明确定义,或是否可以通过其他方式确定的问题。不可判定性仅意味着所考虑的特定推理系统不能证明命题的真或假。是否存在所谓的“绝对不可判定”命题,其真值永远无法得知或定义不明确,是数学哲学中的一个争议点。

哥德尔和保罗·科恩的联合工作提供了两个具体的不可判定命题(第一层含义的不可判定性):连续统假设在ZFC(集合论的标准公理化)中既不能被证明也不能被反驳,选择公理在ZF(即ZFC公理中去除选择公理后的部分)中既不能被证明也不能被反驳。这些结果并不需要不完全性定理。哥德尔在1940年证明,这两个命题都不能在ZF或ZFC集合论中被反驳。在1960年代,科恩证明了这两个命题也不能从ZF中得到证明,且连续统假设不能从ZFC中得到证明。

谢拉(Shelah,1974年)证明了群论中的怀特海德问题在标准集合论中是不可判定的(第一层含义的不可判定性)。\(^\text{[7]}\)

格雷戈里·蔡廷在算法信息理论中产生了不可判定命题,并在该领域证明了另一个不完全性定理。蔡廷的不完全性定理声明,对于任何能够表示足够算术的系统,存在一个上界\( c \),使得在该系统中无法证明某个特定数字具有大于\( c \)的科尔莫戈洛夫复杂性。虽然哥德尔定理与说谎者悖论相关,蔡廷的结果则与贝瑞悖论相关。
\subsubsection{在更大系统中可证明的不可判定命题}
这些是哥德尔“真实但不可判定”命题的自然数学等价物。它们可以在一个更大的系统中得到证明,该系统通常被接受为有效的推理形式,但在像皮亚诺算术这样的更有限的系统中是不可判定的。

1977年,巴黎和哈林顿证明了巴黎–哈林顿原理,一个无限拉姆齐定理的版本,在(一阶)皮亚诺算术中是不可判定的,但可以在更强大的二阶算术系统中证明。基尔比(Kirby)和巴黎后来证明了古德斯坦定理,一个关于自然数序列的命题,比巴黎–哈林顿原理要简单一些,它也在皮亚诺算术中是不可判定的。

克鲁斯卡尔树定理,该定理在计算机科学中有应用,也是从皮亚诺算术中不可判定的,但可以在集合论中证明。事实上,克鲁斯卡尔树定理(或其有限形式)在一个更强的系统ATR0中也是不可判定的,该系统编码了基于一种叫做预测主义的数学哲学所接受的原则。\(^\text{[8]}\)相关但更一般的图小定理(2003年)对计算复杂性理论有重要影响。