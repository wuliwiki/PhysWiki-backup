% 纯不可分扩张
% keys 可分扩张|域|扩域|代数扩张

\pentry{可分扩张\upref{SprbEx}}
%GTM 242 V.6


我们容易想到并且熟悉的扩域,通常是可分扩张.相对地,凡不是可分扩张的情况,都叫做不可分扩张——也就是说,不可分扩张是可分元、不可分元混杂的扩域.这样一来,代数扩张就分为两类了.似乎不可分扩张只是可分扩张的例外,不像可分扩张有那么好的性质;可相对的,下面定义的纯不可分扩张就很有意思了.

\begin{definition}{纯不可分扩张}

设$\mathbb{K}/\mathbb{F}$是代数扩域.如果$\mathbb{K}-\mathbb{F}$中的元素在$\mathbb{F}$上\textbf{都是不可分的},那么称$\mathbb{K}/\mathbb{F}$是\textbf{纯不可分扩张(purely inseparable extension)},且称$\mathbb{K}$在$\mathbb{F}$上(over $\mathbb{F}$)是\textbf{纯不可分}的.

\end{definition}

任何域$\mathbb{F}$都是自己的纯不可分扩张,因为不存在$\mathbb{F}-\mathbb{F}$的元素,命题前提为假则命题恒真.

一切代数扩张,必是进行一次可分扩张后再进行一次纯不可分扩张的结果.要注意,反过来的“代数扩张必是先进行纯不可分扩张再进行可分扩张的结果”并不总是成立.



\begin{theorem}{}
给定代数扩域$\mathbb{K}/\mathbb{F}$,则集合$\mathbb{S}=\{\alpha\in\mathbb{K}|\alpha\text{在}\mathbb{F}\text{上可分}\}$是一个域.

且$\mathbb{K}/\mathbb{S}$是一个纯不可分扩张.
\end{theorem}

\textbf{证明}:

由\textbf{可分元素的封闭性}(\autoref{SprbE2_cor4}~\upref{SprbE2}),可知$\mathbb{S}$是$\mathbb{F}$的可分闭包与$\mathbb{K}$的交集,显然是个域.

反设$\mathbb{K}-\mathbb{S}$中存在$\mathbb{S}$的可分元素$a$,那么$\mathbb{S}(a)/\mathbb{S}$是可分扩张.由可分扩张的传递性(\autoref{SprbE2_cor1}~\upref{SprbE2}),可知$\mathbb{S}(a)/\mathbb{F}$是可分扩张,这与可分闭包的定义矛盾.因此反设不成立,即$\mathbb{K}-\mathbb{S}$中全是$\mathbb{S}$的不可分元素.

\textbf{证毕}.

由于特征为$0$的域都是\textbf{完美域}(定义见\autoref{SprbEx_the5}~\upref{SprbEx}),不存在不可分元素,因此纯不可分扩张的讨论集中在特征为素数$p$的域上.

考虑域$\mathbb{F}$,$\opn{ch}\mathbb{F}=p$.对于任意$a, b\in\mathbb{F}$和$k\in\mathbb{Z}^+$,都有$(a-b)^k=a^k-b^k$.























