% 共晶、共析相图
共晶转变:由一个液相生成两个固相 $L \rightarrow \alpha+\beta$

共析转变:由一个固相生成两个固相 $\gamma \rightarrow \alpha+\beta$

共晶转变与共析转变具有很多相似之处,因此本文主要介绍共晶转变.

在热力学中,往往只关心转变前后的相变化;但由于动力学因素,往往生成一定的组织结构.本文一并简要讨论.

\subsection{共晶相图}
\begin{figure}[ht]
\centering
\includegraphics[width=12cm]{./figures/EUTECT_1.png}
\caption{典型的共晶系合金相图.注意固体部分不完全是固溶体.} \label{EUTECT_fig1}
\end{figure}
根据具体的成分不同,共晶系合金的转变过程可分为以下几类.

\subsubsection{共晶合金}
\begin{figure}[ht]
\centering
\includegraphics[width=14cm]{./figures/EUTECT_2.png}
\caption{共晶合金} \label{EUTECT_fig2}
\end{figure}

全程的相转变:$L \rightarrow \alpha+\beta$

全程的组织转变:$L \rightarrow (\alpha+\beta)_{eutectic}$

i处,发生共晶转变.
\begin{itemize}
\item 共晶转变:由一个液相生成两个固相,并形成一定组织结构(珠光体) $L \rightarrow (\alpha+\beta)_{eutectic}$
\item 共晶转变是恒成分转变:即共晶转变中,先后结晶部分的成分一致.例如,α相中Sn浓度始终为18.3%
\item 共晶转变是恒温转变: 相转变时,系统自由度f=2-3+1=0,因此相转变温度是定值,在相图上体现为三相区是直线
\end{itemize}

随后发生脱熔转变 $\alpha \rightarrow \beta_{II}, \beta \rightarrow \alpha_{II}$
\begin{itemize}
\item 随温度降低,α相溶解Sn、β相溶解Pb的能力均减弱,Sn将以β相固溶体的形式从α中析出,而Pb将以α相固溶体的形式从β中析出,称为二次相 $\alpha_{II},\beta_{II} $.
\item 脱熔转变生成的$\alpha_{II},\beta_{II} $结构与性质与$\alpha, \beta$完全相同.
\item 此时,αII,βII难以与共晶组织中α,β区分,可以不标出
\end{itemize}
