% 调和场
% 散度|旋度|调和场|拉普拉斯

\begin{issues}
\issueDraft
\end{issues}

\pentry{调和函数(未完成)}

我们把散度和旋度都为零的场称为\textbf{调和场}. 注意这并不是一个常用的数学名词, 笔者只在个别中文教材中见过. 由于旋度为零, 积分与路径无关, 必定可以定义势函数 $u(\bvec r)$ (包含一个任意常数项), 而调和场就是其梯度
\begin{equation}\label{HarmF_eq1}
\bvec f(\bvec r) = \div u
\end{equation}
要保证散度 $\div \bvec f$ 为零, \autoref{HarmF_eq1} 就要求 $u$ 是一个调和函数:
\begin{equation}
\laplacian u = 0
\end{equation}
所以调和场的充分必要条件是它可以表示为一个调和函数的梯度, 当且仅当给调和函数加一个任意常数时, $\bvec f(\bvec r)$ 不会改变.

与调和函数类似, 调和场的一个显著特点是其在无穷远处不为零, 
