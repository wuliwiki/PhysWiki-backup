% 滑块和运动斜面问题
% keys 滑块|斜面|人船模型|加速度|动量守恒

\begin{issues}
\issueTODO
\end{issues}

\pentry{人船模型} % 链接未完成

\begin{figure}[ht]
\centering
\includegraphics[width=10cm]{./figures/blkSlp_1.pdf}
\caption{受力分析} \label{blkSlp_fig1}
\end{figure}

在滑块斜面问题的基础上, 如果我们假设斜面质量为 $M$, 滑块质量为 $m$ ,滑块、斜面、地面三者之间均无摩擦, 那么滑块在斜面上自由下滑时,相对斜面的加速度是多少呢?

令 $x, y$ 为滑块水平方向和竖直方向移动的距离. $X$ 为斜面水平方向移动的距离, $l$ 为滑块相对斜面的位移大小.对滑块与斜面组成得系统而言,在水平方向不受力,动量守恒,质心在水平方向速度 $\bvec{v_{cx}}$ 不变(见人船模型).以系统质心所在竖直方向为 $y$ 轴,地面为 $x$ 轴建立直角坐标系,则有
\begin{equation}
mx+MX=0 \qquad x-X=l\cos\theta
\end{equation}
解得
\begin{equation}\label{blkSlp_eq2}
x = \frac{M}{M + m}l\cos\theta \qquad X = -\frac{m}{M + m}l\cos\theta
\end{equation}
另外竖直方向有
\begin{equation}
y = -l\sin\theta
\end{equation}

以下介绍三种方法, 都可以解得滑块相对斜面的加速度为
\begin{equation}\label{blkSlp_eq1}
a = \ddot l = \frac{g\sin\theta(M+m)}{M + m\sin^2\theta}
\end{equation}

\subsection{受力分析法}
\addTODO{使用高中的方法, 用 “拉格朗日方程法” 中的变量 $x, y, X$ 列方程}

由运动学关系式:
\begin{equation}
\bvec{r}=\bvec{r_{O}}+\bvec{r_{e}}
\end{equation}
其中,$\bvec{r}$ 为物体相对固定坐标系中的位移、$\bvec{r_{O}}$ 为平动坐标系基点$O$的位矢、$\bvec{r_{e}}$ 为物体相对平动坐标系的位移.\\
带入
\begin{equation}
\bvec{r}=(x,y)\qquad \bvec{r_O}=(X,0)\qquad \bvec{r_e}=(l\cos\theta,-l\sin\theta)
\end{equation}
得
\begin{equation}
x=X+l\cos\theta \qquad y=-l\sin\theta
\end{equation}
对物块受力分析,如\autoref{blkSlp_fig1} ,由牛顿第二定律\autoref{New3_eq1}~\upref{New3}
\begin{equation}
m\ddot{\bvec{r}}=m\bvec g+\bvec{N}
\end{equation}
对斜面,其合外力沿 $x$ 负方向
带入 $\bvec g=(0,-g)$
\subsection{非惯性系法}
\pentry{惯性力\upref{Iner}}
这是最简单的方法. 在斜面的参考系, 滑块会受到向右的惯性力 $-m\ddot X$, 所以沿斜面向下使用牛顿第二定律\upref{New3}得
\begin{equation}
-m\ddot X\cos\theta + mg\sin\theta = m\ddot l
\end{equation}
把\autoref{blkSlp_eq2} 代入解得\autoref{blkSlp_eq1}.

\subsection{拉格朗日方程法}
\pentry{拉格朗日方程\upref{Lagrng}}
考虑动量守恒, 这个系统只有一个自由度, 即一个广义坐标 $l$. 拉格朗日量等于
\begin{equation}
\begin{aligned}
L = T - V &= \frac12 m(\dot x^2 + \dot y^2) + \frac12 M \dot X^2 - mgy\\
&= \frac{1}{2}m \qty( \frac{M\cos^2\theta}{M + m}+\sin^2\theta) \dot l^2 + mg\sin\theta \cdot l\\
&=\frac{1}{2}m\frac{M+m\sin^2\theta}{m+M}\dot{l}^2+mg\sin\theta\cdot l
\end{aligned}
\end{equation}
代入拉格朗日方程(\autoref{Lagrng_eq1}~\upref{Lagrng})
\begin{equation}
\dv{t} \pdv{L}{\dot l} = \pdv{L}{l}
\end{equation}
有
\begin{equation}
m\frac{M+m\sin^2\theta}{m+M}\ddot{l}=mg\sin\theta
\end{equation}

解得\autoref{blkSlp_eq1}.
