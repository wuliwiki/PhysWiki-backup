% SLISC 库简介
% C++|矩阵

\begin{issues}
\issueDraft
\end{issues}

相对于 fortran 或者 Matlab 等, 用 C++ 做数值计算的一个缺陷就是语言本身(或标准库)没有矩阵类型(以及高维矩阵类型). 但我们可以用第三方库或者自己写一个. 本书选择后者, 原因是为了教学需要我们需要保持代码的简单易读, 避免复杂的 C++ 语法.

本书中我们将大量使用 \textbf{SLISC} 库, 即 \textbf{Scientific Library in Simple C++}. 代码可以从我们的 \href{https://github.com/MacroUniverse/SLISC0}{GitHub 仓库}下载. 该库的特点是尽量不使用 C++ 的复杂语法(如模板)和复杂的类结构, 使代码便于阅读学习和修改, 同时又保持相对较高的性能. SLISC 使用兼容性高得 C++11 标准.

我们希望把矩阵进行一定程度的封装, 但又几乎不损失性能.

流行的矩阵库, 如 Eigen, 但是代码极其复杂, 错误信息不容易懂, 修改起来非常困难. 例如 \verb|MatrixXd| 矩阵类就有 6 次继承, 一个 2x2 的矩阵在 gdb 里面调试的时候显示出来的变量信息是这样的:
\begin{lstlisting}[language=cpp]
<Eigen::PlainObjectBase<Eigen::Matrix<double,-1,-1,0,-1,-1>>> = 
{<Eigen::MatrixBase<Eigen::Matrix<double,-1,-1,0,-1,-1>>> = 
{<Eigen::DenseBase<Eigen::Matrix<double,-1,-1,0,-1,-1>>> = 
{<Eigen::DenseCoeffsBase<Eigen::Matrix<double,-1,-1,0,-1,-1>, 3>> = 
{<Eigen::DenseCoeffsBase<Eigen::Matrix<double,-1,-1,0,-1,-1>, 1>> = 
{<Eigen::DenseCoeffsBase<Eigen::Matrix<double,-1,-1,0,-1,-1>, 0>> = 
{<Eigen::EigenBase<Eigen::Matrix<double,-1,-1,0,-1,-1>>> =
{<No data fields>}, <No data fields>}, 
<No data fields>}, <No data fields>}, <No data fields>}, <No data fields>},
m_storage = {m_data = 0x855ceb0, m_rows = 2, m_cols = 2}}, <No data fields>
\end{lstlisting}
实际上这里面真正有用的只有最后一行, 显示了矩阵在内存中的地址 \verb|m_data|, 行数 \verb|m_rows| 以及列数 \verb|m_cols|. 这样的库只适合直接拿来用, 不适合读和改, 尤其是对非计算机专业的人来说.


底层用 MKL 来实现, 速度非常快.

所有的函数都尽量使用指针 interface (因为是最兼容的!最笨的办法往往是最灵活的) 然后可以再封装一层更友好的 interface

slicing 会有少量的 overhead. 由用户自己权衡是使用 slicing 还是指针 interface.

\subsection{重新定义类型名称}
在 C++ 标准中, 一些基本类型在内存中的长度在不同计算机上可能会不同. 例如 \verb|long| 有时候和 \verb|int| 一样是 4 字节, 而另一些时候则和 \verb|long long| 一样是 8 字节. 因此, SLISC 仿照 Numerical Recipe 的方式, 定义自己的类型名称:
\begin{itemize}
\item \verb|Char| 是 \verb|char|, 即 1 字节字符或整数
\item \verb|Uchar| 是 \verb|unsigned char|, 即 1 字节无符号整数
\item \verb|Short| 是 2 字节整数
\item \verb|Int| 是 4 字节整数
\item \verb|Llong| 是 8 字节整数
\item \verb|Float| 是 \verb|float|(4 字节)
\item \verb|Doub| 是 \verb|double|(8 字节)
\item \verb|Comp| 是 \verb|std::complex<double>| (16 字节)
\end{itemize}
另外我们还定义一些附加类型, 用于声明函数的参数. 例如 \verb|Doub_I| 是 \verb|const Doub|, 表示函数的输入参数. 又如 \verb|Doub_O| 是 \verb|Doub &|, 表示函数的输出参数. \verb|Doub_IO| 也是 \verb|Doub &|, 但表示既作为输入也作为输出. 以上的标量类型都同理, 在后面加 \verb|_I|, \verb|_O|, \verb|_IO| 分别表示 in, out, in-out.

\begin{itemize}
\item \verb|Char_I| 是字符或者 1 字节整数
\item \verb|Int| 是 4 字节整数
\item \verb|Llong| 是 8 字节整数
\item \verb|Float| 是 \verb|float|(4 字节)
\item \verb|Doub| 是 \verb|double|(8 字节)
\item \verb|Comp| 是 \verb|std::complex<double>| (16 字节)
\end{itemize}

\subsection{SLISC 库编程习惯}
\begin{itemize}
\item 尽量不使用 \verb|unsigned| 类型
\item 为了速度, SLISC 中的函数尽量不会改变容器尺寸,需要实现手动设置
\end{itemize}
