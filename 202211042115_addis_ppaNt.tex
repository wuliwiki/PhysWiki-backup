% Personal Package Archive (PPA) 笔记(Linux)

\begin{issues}
\issueDraft
\end{issues}

\pentry{Linux 包管理笔记(apt, dpkg, snap)\upref{snap}}

\begin{itemize}
\item 参考\href{https://itsfoss.com/ppa-guide/}{这个}.
\item \verb|/etc/apt/sources.list| 文件会列出软件 repo 的网址
\item 添加 ppa: \verb|sudo add-apt-repository ppa:dr-akulavich/lighttable|. 然后就可以 \verb|update| 和 \verb|install| 了.
\item 添加 ppa 并不会改变 \verb|sources.list|, 而是会在 \verb|/etc/apt/sources.list.d| 添加两个文件, 一个 \verb|.list| 文件, 一个 \verb|.list.save|
\item 有一些 deb 安装包同样会添加 sources.list, 这样以后就可以方便更新了.
\end{itemize}
