% 深度学习 CNN 入门 2
% keys 卷积
% license Xiao
% type Wiki


\pentry{卷积\nref{nod_Conv},神经网络\nref{nod_NN},全连接网络\nref{nod_FCNN},Python 导航\nref{nod_PyFi}}{nod_c889}

本文从更基础的部分来解释CNN神经网络,上文是说的神经网络的基本结构,这篇解释的是CNN的底层逻辑。

首先需要了解一下卷积的概念。
定义:在泛函分析中,卷积是通过两个函数$f$和$g$生成第三个函数的一种数学运算,其本质是一种特殊的积分变换,表征函数$f$与$g$经过翻转和平移的重叠部分函数值乘积对重叠长度的积分。
数学表达式:对于连续函数,卷积的数学表达式通常为
\begin{equation}
(f*g)(t)=\int_{-\infty}^{+\infty}f(\tau)g(t-\tau)~.
\label{juanji}
\end{equation}
在CNN中,卷积操作主要用于特征提取。它通过滑动一个称为“卷积核”(或“滤波器”)的小型矩阵窗口在输入数据(如图像)上,进行元素级别的乘法并求和,从而生成新的特征图(Feature Map)。

假设我们有一个简单的3x3的输入矩阵(图像的一个局部区域)和一个2x2的卷积核:
输入矩阵(Input Matrix):
\begin{table}[ht]
\centering
\caption{输入矩阵F}\label{tab_CNN21}
\begin{tabular}{|c|c|c|}
\hline
1 & 0 & 1 \\
\hline
2 & 1 & 0 \\
\hline
0 & 1 & 1 \\
\hline
\end{tabular}
\end{table}
卷积核(Kernel)(只是方便演示,一般卷积核是奇数,方便确定中心):
\begin{table}[ht]
\centering
\caption{卷积核G(Kernel)}\label{tab_CNN22}
\begin{tabular}{|c|c|}
\hline
1 & 0 \\
\hline
0 & 1 \\
\hline
\end{tabular}
\end{table}
卷积操作的过程如下:

1.定位初始位置:首先,将卷积核放置在输入矩阵的左上角(或指定的起始位置)。
进行元素级乘法并求和:将卷积核中的每个元素与输入矩阵中对应位置的元素相乘,然后将所有乘积相加。在本例中,卷积核与输入矩阵左上角2x2区域的乘积之和为 1∗1+0∗2+0∗1+1∗0=1。这里注意对应的下标是不同的,是F(1,1)对应G(2,2),F(1,2)对应G(2,1)...这在前面数学表达式有体现。

2.滑动卷积核:按照指定的步长(Stride)将卷积核向右滑动(通常是1个单位),然后重复步骤2,直到卷积核无法再向右滑动为止。之后,将卷积核回到最左边,向下移动一个步长,继续向右滑动,直到遍历完整个输入矩阵。

3.记录输出:每次卷积操作的结果都会被记录下来,形成新的特征图的一个元素。在本例中,由于步长为1且没有填充(Padding),所以输出特征图的大小会比输入矩阵小(具体取决于卷积核大小、步长和填充方式)。

步长(Stride):卷积核在输入矩阵上滑动的距离。步长越大,输出特征图越小。
填充(Padding):在输入矩阵的边界外添加额外的值(通常是0),以便在卷积过程中保持输入和输出的尺寸相同或按预期变化。
卷积核大小:卷积核的维度决定了每次卷积操作覆盖的输入矩阵的区域大小,也影响了特征图的尺寸和提取的特征类型。
​
最后输出是:\begin{table}[ht]
\centering
\caption{输出矩阵}\label{tab_CNN24}
\begin{tabular}{|c|c|}
\hline
2 & 0 \\
\hline
3 & 2 \\
\hline
\end{tabular}
\end{table}

把这个在PyTorch中写出来,可以使用torch.nn.Conv2d来定义一个二维卷积层,该层可以应用于图像等二维数据上。对于给出的例子(输入矩阵为3x3,卷积核为2x2,步长为1,无填充),可以这样定义卷积层并应用它:


\begin{lstlisting}[language=python]
import torch  
import torch.nn as nn  
  
# 定义一个输入张量,形状为[batch_size, channels, height, width]
#这里假设batch_size=1, channels=1  
# 注意PyTorch中的输入张量通常需要加上一个批次维度和一个通道维度  
input_tensor = torch.tensor([[[[1, 0, 1],  
                               [2, 1, 0],  
                               [0, 1, 1]]]], dtype=torch.float32)  
  
# 定义一个2x2的卷积核
# 这里out_channels=1(输出通道数)
#in_channels=1(输入通道数)
#kernel_size=2(卷积核大小)  
conv_layer = nn.Conv2d(in_channels=1, out_channels=1,
 kernel_size=2, stride=1, padding=0)  
  
# 应用卷积层  
output_tensor = conv_layer(input_tensor)  
  
# 输出结果,注意输出张量也会包含批次维度和通道维度  
print(output_tensor)  
  
# 如果你只想看到卷积后的结果数组,可以这样做:  
print(output_tensor.squeeze().detach().numpy())  
# squeeze()去掉批次和通道维度(如果它们为1)
#detach()是为了从计算图中分离,numpy()转换为numpy数组
\end{lstlisting}

注意,这个函数并没有直接设置卷积核矩阵中的具体元素值,这些值是在训练过程中通过反向传播算法自动学习的。
具体来说,当你创建一个Conv2d层的实例时,PyTorch会为该层随机初始化一组权重(即卷积核中的矩阵元素)。这些权重的初始值通常是小的随机数,可以是均匀分布或正态分布中抽取的。然后,在训练过程中,这些权重会根据损失函数对模型输出的梯度进行更新,以最小化损失函数。

另外,Conv2d层还有一个bias属性,它表示每个输出通道的可选偏置项。与权重类似,偏置项也是随机初始化的,并在训练过程中进行更新。如果你想要设置偏置项的初始值,你可以通过访问bias属性来实现。但是,同样地,在大多数情况下,你应该让PyTorch自动为你初始化偏置项.