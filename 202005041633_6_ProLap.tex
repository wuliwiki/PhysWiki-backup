% 拉普拉斯变换的性质

\pentry{拉普拉斯变换\upref{LapTra}}

前面我们介绍了拉普拉斯变换,那么这个变换到底给我们带来了什么特别的好处呢?我们下面就来看一下它到底具有哪些性质.

\subsection{线性性质}


若$\mathscr L[f_1(t)] = \bar f_1(p),\mathscr L[f_2(t)] = \bar f_2(p)$,则
\begin{equation}
\mathscr L[c_1f_1(t)+c_2f_2(t)] = c_1\bar f_1(p) + c_2\bar f_2(p)
\end{equation}


\begin{equation}
\begin{aligned} \mathscr L[c_{1} f_{1}(t)+c_{2} f_{2}(t)] & = \int_{0}^{\infty}\left[c_{1} f_{1}(t)+c_{2} f_{2}(t)\right] \mathrm{e}^{-\rho t} \mathrm{d} t \\ &=\int_{0}^{\infty} c_{1} f_{1}(t) \mathrm{e}^{-p t} \mathrm{d} t+\int_{0}^{\infty} c_{2} f_{2}(t) \mathrm{e}^{-\rho t} \mathrm{d} t \\ &=c_{1} \bar{f}_{1}(p)+c_{2} \bar{f}_{2}(p) \end{aligned}
\end{equation}

