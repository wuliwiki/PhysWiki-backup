% 理想气体混合的熵变

\pentry{理想气体分压定律\upref{PartiP}, 熵的宏观表达式\upref{MacroS}}

\footnote{本文参考了Schroeder的《热物理学导论》,本文使用 CC BY-SA 3.0 协议。} 在本文中,我们将讨论等温(或绝热,因为理想气体混合不导致内能改变,因此等温与绝热在本问题中没有区别)等压情况下,理想气体混合熵变的问题。

\subsection{同种气体的混合}
\begin{figure}[ht]
\centering
\includegraphics[width=8 cm]{./figures/06054324d830d4c9.pdf}
\caption{同种气体的混合} \label{fig_IGME_1}
\end{figure}
首先,我们先思考一种最简单的混合过程:同种气体的混合。当我们拿开挡板之后会发生什么?你可能觉得\textsl{什么都不会发生}:既然两侧的气体已经是完全一样的,那有和没有挡板有什么区别吗?事实正是如此的简单(伴随着些许的诡异):在这种情况下,同种气体的混合并不会导致熵变。
\begin{equation}
\Delta S = 0~.
\end{equation}


这个问题比想象中的更微妙而深刻一些。这个现象暗示了同种物质的各个微观粒子是完全相同的,而这也是统计力学与量子力学的基本假设之一。在微观世界中,你不能说“这个粒子、那个粒子”,因为他们是完全一样而不可区分的。

\subsection{异种气体的混合}
\begin{figure}[ht]
\centering
\includegraphics[width=8 cm]{./figures/00f50917178c747f.pdf}
\caption{异种气体的混合} \label{fig_IGME_fig2}
\end{figure}

现在我们来看异种气体的混合。先定义摩尔分数$x_A = \frac{n_A}{\sum n_i} = \frac{n_A}{n_A+n_B+...}$

尽管移除挡板后系统的总压强还是$p$,但$A, B$各自的\textsl{分压}\upref{PartiP} 却降低了。例如,$A$气体的分压从混合前的$p$降为$p'=p \cdot x_A$。

先计算$A$气体的熵变。根据熵的计算公式\upref{MacroS},此时
$$
\Delta S_A=-nR\ln p |^{p \cdot x_A}_p=-n_A R \ln x_A~.
$$
同理,$B$气体的熵变是
$$
\Delta S_B=-n_B R \ln x_B~,
$$
因此系统的总熵变
$$
\Delta S = \Delta S_A+\Delta S_B= -n_A R \ln x_A - n_B R \ln x_B~.
$$
写为更为一般的形式
\begin{equation}
\Delta S = -R \sum_i n_i \ln x_i >0~.
\end{equation}
不要被这里的负号迷惑了:由于$x_i<1$,因此$\ln x_i < 0$,那么$\Delta S>0$。可见,混合是一个熵增过程。

还可以从另一个角度理解:根据理想气体方程$pV=nRT$,将熵的表达式中的$p$换成$V$,那么我们有:
$$
\Delta S_A = n_A R \ln \frac{V_2}{V_1}~.
$$
其中$V_1$表示 移除挡板前$A$气体的运动范围(只有容器左侧);$V_2$表示 移除挡板后$A$气体的运动范围(整个容器)。很显然,移除挡板后,$A$气体的运动范围从容器左侧扩大到了整个容器,这就导致$A$气体的混乱程度上升了(或者按统计力学的说法,$A$气体可能的微观态个数增加了),也就是熵上升了。

% \subsection{其他混合过程}
% 尽管上述两个例子非常简单,但他们是理解理想气体混合熵变问题所必不可少的。
%TODO: 举更多例子
