% 氦原子单散射态的数值解

根据论文, 求本征态可以在每个 $L,M$ 空间中单独求.
\begin{equation}
\Psi(\bvec r_1, \bvec r_2) = \frac{1}{r_1 r_2}\sum_{l_1, l_2}  \psi_{l_1, l_2}^{L, M}(r_1, r_2)\mathcal{Y}_{l_1, l_2}^{L, M}(\uvec r_1, \uvec r_2)
\end{equation}
那么接下来使用一组基底来展开径向波函数
\begin{equation}
\psi_{l_1, l_2}^{L, M}(r_1, r_2) R_{n_1,l_1}^{(Z=2)}(r_1)
\end{equation}

满足边界条件
\begin{equation}
\psi_{l'_1,l'_2}^{L, M}(r_1, r_2) \overset{r_2\to\infty}{\longrightarrow} \delta_{l_1,l'_1}\delta_{l_2,l'_2} r_1 R_{n_1,l_1}^{(Z=2)}(r_1)
\sin\qty[k_2 r_2 - \frac{\pi l_2}{2} +\frac{1}{k_2}\ln(2k_2 r_2) + \sigma_{l_2} + \delta_{n_1,l_1,l_2}^{L,M}]~.
\end{equation}
