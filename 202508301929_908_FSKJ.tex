% 仿射空间(综述)
% license CCBYSA3
% type Wiki

本文根据 CC-BY-SA 协议转载翻译自维基百科\href{https://en.wikipedia.org/wiki/Affine_space}{相关文章}

\begin{figure}[ht]
\centering
\includegraphics[width=6cm]{./figures/c472da6323adb22e.png}
\caption{在$\mathbb{R}^3$ 中,蓝色的上平面$P_2$不是一个向量子空间,因为:$\mathbf{0} \notin P_2$;$\mathbf{a} + \mathbf{b} \notin P_2$。因此,$P_2$是一个仿射子空间。它的方向(即与该仿射子空间关联的线性子空间)是绿色的下平面$P_1$,这个平面是一个向量子空间。虽然向量$\mathbf{a}$和$\mathbf{b}$都属于$P_2$,但它们的差向量是一个位移向量,它不属于$P_2$,而是属于向量空间 $P_1$。} \label{fig_FSKJ_1}
\end{figure}
在数学中,仿射空间是一种几何结构,它推广了欧几里得空间的一些性质,但这些性质与距离和角度测量无关,仅保留了平行性和平行线段长度比例等相关特性。仿射空间是仿射几何的基本背景。

与欧几里得空间类似,仿射空间中的基本对象称为点,它们可以看作空间中的“位置”,没有大小或形状,即零维对象。通过任意两点,可以画出一条无限延伸的直线(一维点集);通过任意不共线的三点,可以画出一个平面(二维点集);更一般地,任意 $k+1$ 个处于一般位置的点,可以确定一个$k$ 维平面或仿射子空间。仿射空间的一个显著特征是平行线的概念:同一平面内的两条平行直线永不相交;同一平面内的非平行直线则必定在某一点相交。并且,对于任意一条直线和空间中的任意一点,总能画出一条通过该点且与原直线平行的直线。所有相互平行的直线属于同一个方向的等价类。

与向量空间中的向量不同,仿射空间中没有一个特定的点作为原点*。
在仿射空间中:

* 没有预定义的“点与点相加”或“点与数相乘”的概念。

然而,对于任意一个仿射空间,可以通过**两点之间的差**来构造一个**关联向量空间**。这些差向量被称为:

* **自由向量(free vectors)**
* **位移向量(displacement vectors)**
* **平移向量(translation vectors)**
* 或简称 **平移(translations)**\[1]。

同样,将一个**位移向量加到某个点**上是有意义的,这会生成一个新的点,即从原点沿该向量平移后的点。

虽然点不能被随意相加,但可以取点的**仿射组合(affine combination)**:
即系数和为 1 的加权和,这会产生另一个点。
这些系数定义了经过这些点所在平面的**重心坐标系(barycentric coordinate system)**。

---

**任意向量空间都可以被看作一个仿射空间**,这相当于“忘记零向量的特殊角色”。
在这种情况下:

* 向量空间中的元素既可以看作仿射空间中的点,也可以看作位移向量(平移)。
* 当把零向量看作一个点时,它就被称为**原点**。

将一个固定向量加到向量空间的某个**线性子空间**上,可以得到这个向量空间的一个**仿射子空间**。
我们通常说,这个仿射子空间是通过某个**平移向量**将该线性子空间**从原点平移**得到的。

在有限维情形下:

* 这样的仿射子空间就是某个**非齐次线性系统的解集**;
* 该仿射空间的位移向量则是**对应齐次线性系统的解集**,这是一个**线性子空间**。

相比之下,**线性子空间**总是包含向量空间的原点。

---

**仿射空间的维数**定义为其**平移向量空间的维数**:

* 一维仿射空间称为**仿射直线(affine line)**;
* 二维仿射空间称为**仿射平面(affine plane)**;
* 在 $n$ 维的仿射空间或向量空间中,维数为 $n-1$ 的仿射子空间称为**仿射超平面(affine hyperplane)**。
