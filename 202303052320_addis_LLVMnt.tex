% LLVM 笔记

\begin{issues}
\issueDraft
\end{issues}

\begin{itemize}
\item Ubuntu 安装: \verb|sudo apt install llvm-14|
\end{itemize}
