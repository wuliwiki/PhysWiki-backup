% 理想气体(正则系宗法)
% keys 理想气体|正则系综|统计力学|能级|配分函数
% license Xiao
% type Tutor

\begin{issues}
\issueDraft
\end{issues}
\pentry{热力学量的统计表达式(玻尔兹曼分布)\nref{nod_TheSta}}{nod_a8a9}

\subsection{可区分粒子和不可区分粒子}
对于可区分粒子,从粒子的角度求和,配分函数为(dis $=$ distinguishable)
\begin{equation}\label{eq_IdCE_1}
\ali{
Q_{dis} & = \sum_{i_1 = 0}^\infty  \sum_{i_2 = 0}^\infty \dots\sum_{i_N = 0}^\infty  \E^{-\beta (\varepsilon_{i_1} + \varepsilon_{i_2}\dots)} = \sum_{i_1 = 0}^\infty \E^{-\beta \varepsilon_{i_1}} \sum_{i_2 = 0}^\infty \E^{-\beta\varepsilon_{i_2}}\dots \sum_{i_N = 0}^\infty \E^{-\beta\varepsilon_{i_N}}\\
& = \qty(\sum_{i = 0}^\infty \E^{-\beta\varepsilon_i})^N = Q_1^N~.
}\end{equation}
从能级的角度求和
\begin{equation}\label{eq_IdCE_2}
\ali{
Q_{dis} & = \sum_{\{n_i\}} \frac{N!}{n_0! n_1!\dots} \exp(-\beta \sum_{i = 0}^\infty n_i \varepsilon_i) = \sum_{\{n_i\}} \frac{N!}{n_0! n_1!\dots} \E^{-n_1\varepsilon_1\beta} \E^{-n_2\varepsilon_2\beta}\dots
}~\end{equation}
由\autoref{eq_IdCE_1} 和\autoref{eq_IdCE_2} 物理意义可知, 二者相等。再从能级的角度考虑, 若粒子不可区分(由于这个配分函数是最常用的, 所以不写角标)
\begin{equation}\label{eq_IdCE_3}
Q = \sum_{\{n_i\}} \E^{-n_1\varepsilon_1\beta} \E^{-n_2\varepsilon_2\beta}\dots~
\end{equation}
比较\autoref{eq_IdCE_2},  求和的每项少了一个因子 $N!/(n_0! n_1!\dots)$

理想气体条件:能级占有率极低, 几乎没有两个粒子在同一个能级上, 所以大部分 $n_i = 0$,  $0! = 1$。  个别 $n_i = 1$,  $1! = 1$。
可以近似认为
\begin{equation}
\frac{N!}{n_0! n_1!\dots} \approx N!~,
\end{equation}
所以
\begin{equation}
Q = \frac{1}{N!} Q_{dis} = \frac{1}{N!} Q_1^N~.
\end{equation}
那如何求 $Q_1$ 呢? 

\subsection{对单粒子相空间积分}
注意每个量子态对应的相空间体积为 $h$ 的空间维数次方。
\begin{equation}
Q_1 = \frac{1}{h^3} \int \E^{-\frac{p^2}{2m\cdot kT}} \dd[3]{p} \dd[3]{x}  = \frac{V}{h^3} \int \E^{-\frac{p^2}{2m\cdot kT}} \dd[3]{p} = \frac{V}{\lambda^3}~.
\end{equation}
其中 $\lambda $ 叫做热力学波长, 正比与粒子热运动的德布罗意波
\begin{equation}
\lambda  = \frac{h}{\sqrt{2\pi mkT}}~.
\end{equation}

\subsection{对单粒子能级密度积分}
用单粒子能级密度 $a(\varepsilon)$ 对玻尔兹曼因子积分:
\begin{equation}
a(\varepsilon) = \frac{2\pi V(2m)^{3/2}}{h^3} \varepsilon^{1/2}~,
\end{equation}
\begin{equation}
Q_1 = \sum_{i = 0}^\infty \E^{-\beta \varepsilon_i} = \int_0^\infty a (\varepsilon) \E^{-\beta\varepsilon} \dd{\varepsilon}\\
= \frac{2\pi V(2m)^{3/2}}{h^3} \int_0^\infty \varepsilon^{1/2} \E^{-\beta\varepsilon} \dd{\varepsilon}~.
\end{equation}
对积分换元, 令 $x = \beta\varepsilon$, 
\begin{equation}
\int_0^\infty \varepsilon^{1/2} \E^{-\beta\varepsilon} \dd{\varepsilon} = (kT)^{3/2} \int_0^\infty  x^{1/2} \E^{-x} \dd{x}
= \Gamma (3/2) (kT)^{3/2}
= \frac{\sqrt\pi}{2} (kT)^{3/2}~,
\end{equation}
\begin{equation}
Q_1 = \sum_{i = 0}^\infty \E^{-\beta \varepsilon_i}  = \int_0^\infty  a (\varepsilon) \E^{ - \beta \varepsilon} \dd{\varepsilon}  = \frac{2\pi V (2m)^{3/2}}{h^3} \frac{\sqrt \pi}{2} (kT)^{3/2}  = \frac{V}{\lambda^3}~.
\end{equation}

\subsection{对系统的能级密度积分}
现在我们试图直接求 $Q$,系统的总能级密度为
% 链接未完成
\begin{equation}
g(E) = \dv{\Omega_0}{E}  = \frac{V^N}{N! h^3} \frac{(2\pi m)^{3N/2}}{(3N/2 - 1)!} \E^{3N/2-1}~,
\end{equation}
\begin{equation}
Q = \int_0^\infty  g(E) \E^{-E\beta} \dd{E}  = \frac{V^N (2\pi m)^{3N/2}}{N! h^3 (3N/2 - 1)!}\int_0^\infty \E^{3N/2-1} \E^{-\beta E} \dd{E}~.
\end{equation}
令 $x = \beta E$ 对积分换元,
\begin{equation}\ali{
\int_0^\infty  \E^{3N/2-1} \E^{-\beta E} \dd{E} & = (kT)^{3N/2} \int_0^\infty x^{3N/2-1}\E^{-x} \dd{x}  \\
& = (kT)^{3N/2} \Gamma (3N/2) \\
& = (kT)^{3N/2} (3N/2-1)!
}~\end{equation}
代入上式得
\begin{equation}\ali{
Q & = \frac{V^N (2\pi m)^{3N/2}}{N! h^3 (3N/2-1)!} (kT)^{3N/2}(3N/2 - 1)! \\
& = \frac{V^N (2\pi mkT)^{3N/2}}{N! h^3} = \frac{1}{N!} \qty(\frac{V}{\lambda^3})^N \\
& = \frac{1}{N!} Q_1^N~,
}\end{equation}
与之前的结果都一样。


\subsection{热力学性质}
得到系统的配分函数 $Q$ 以后, 可由用亥姆霍兹自由能得到热力学的性质
\begin{equation}
F =  - kT\ln Q =  - kT(N\ln{Q_1} - N\ln N + N)~,
\end{equation}
\begin{equation}
S = Nk \qty(\ln \frac{V}{N\lambda^3} + \frac52)~,
\end{equation}
\begin{equation}
P =  - \qty(\pdv{F}{V})_{T,N} = \frac{NkT}{V} \qquad \text{(理想气体状态方程)}~,
\end{equation}
\begin{equation}
\mu  = kT\ln \frac{N{\lambda ^3}}{V} = kT\ln \frac{N}{Q_1}~.
\end{equation}
在巨正则系综里, 定义逸度为 $z = {\E^{\mu/(kT)}}$,  则 $N = {zV}/{\lambda ^3} = z{Q_1}$。 

\subsection{分布函数}
若有 $N$ 个粒子组成理想气体, 每个能级平均有多少粒子(由于理想气体的条件是能级占有率 $\ev{n_i} \ll 1$,  但仍然会有分布曲线)

对任何一个粒子来说, 出现在 $\varepsilon_i$ 能级(非简并)的概率是 $\E^{-\beta\varepsilon_i}/Q_1 = \lambda^3 \E^{ - \beta\varepsilon_i}/V$。  那么 $N$ 个没有相互作用的粒子在该能级的平均粒子数就为
\begin{equation}
\ev{n_i} = \frac{N\lambda^3}{V} \E^{-\beta\varepsilon_i}~,
\end{equation}
理想气体的化学能 $\mu = kT\ln N\lambda^3/V$,  即 $\E^{\mu/kT} = N\lambda^3/V$。  代入上式, 得
\begin{equation}
\ev{n_i} = \E^{\beta\mu} \E^{-\beta\varepsilon_i} = \E^{(\mu - \varepsilon_i)/(kT)}~.
\end{equation}
这就是麦克斯韦—玻尔兹曼分布。

