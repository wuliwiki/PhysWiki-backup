% 华东师范大学 1998 年 考研 量子力学
% license Usr
% type Note

\textbf{声明}:“该内容来源于网络公开资料,不保证真实性,如有侵权请联系管理员”

\subsection{一}
一电子在沿 $x-$ 方向的强度为$E$的均匀静电场中,
\begin{enumerate}
\item 写出该一维电子系统的哈密顿算符;
\item 该电子动量算符平均值的时间变化率是多少?将所得结果与经典物理学结果比较.
\end{enumerate}
\subsection{二}
\begin{enumerate}
\item 由经典波动力学的驻波条件,德布罗意波粒二象性假设及有关公式($E=hv,p=h/\lambda$),直接导出宽度为$a$的无限深方势井中一自由粒子能量的可能值;写出相应的波函数形式,
\item 对本例中所包含的基本观点,方法及其结果,您能说些什么?
\end{enumerate}
\subsection{三}
\begin{enumerate}
\item 算符$S_i=\frac{\hbar}{2}\sigma$,〔其中,$\sigma(i=x,y,z)$为泡里矩阵〕有何物理意义?写出它们的矩阵形式及基本对易关系:写出它们的本征值和本征态形式.
\item 设某电子处于某一状态,在该状态中测量其自旋沿 $z-$方向的值恒定为$\frac{\hbar}{2}$问,若在该状态中测量其自旋沿$x-$方向的值,所得结果是什么?简要说明您的思考步序及计算公式.
\end{enumerate}
