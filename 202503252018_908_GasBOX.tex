% 盒中气体(综述)
% license CCBYSA3
% type Wiki

本文根据 CC-BY-SA 协议转载翻译自维基百科\href{https://en.wikipedia.org/wiki/Gas_in_a_box}{相关文章}。

在量子力学中,量子粒子在盒子中的结果可以用来研究量子理想气体在盒子中的平衡情况,这个盒子包含了大量的分子,这些分子除了瞬时热化碰撞外彼此不相互作用。这个简单的模型不仅可以用来描述经典理想气体,还可以用来描述各种量子理想气体,例如理想的重费米气体、理想的重玻色气体以及黑体辐射(光子气体),光子气体可以视为无质量的玻色气体,其中通常假设热化是通过光子与平衡质量之间的相互作用来促进的。

利用麦克斯韦-玻尔兹曼统计、玻色-爱因斯坦统计或费米-狄拉克统计的结果,并考虑非常大的盒子的极限,使用托马斯-费米近似(以恩里科·费米和刘埃文·托马斯命名)来表示能级的简并度为微分,并将态的求和转化为积分。这使得通过分配函数或大分配函数可以计算气体的热力学性质。这些结果将应用于有质量和无质量的粒子。更完整的计算将留待单独的文章中,但在本文中将给出一些简单的例子。
\subsection{托马斯-费米近似用于态的简并度}  
对于盒子中的有质量和无质量粒子,粒子的态由一组量子数 [\(n_x, n_y, n_z\)] 枚举。动量的大小由以下公式给出:
\[
p = \frac{h}{2L} \sqrt{n_x^2 + n_y^2 + n_z^2} \quad \quad n_x, n_y, n_z = 1, 2, 3, \ldots~
\]
这里,\(h\)是普朗克常数,\(L\)是盒子的一边的长度。粒子的每一个可能的状态可以看作是一个三维网格上的点,该网格由正整数构成。原点到任何点的距离将是:
\[
n = \sqrt{n_x^2 + n_y^2 + n_z^2} = \frac{2Lp}{h}~
\]
假设每一组量子数指定f个状态,其中f是粒子的内部自由度的数量,可以通过碰撞改变。例如,旋转量子数为1/2的粒子有\(f = 2\),分别对应两个自旋状态。对于大的n值,从上述方程得出的动量大小小于或等于p的状态数大致为:
\[
g = \left( \frac{f}{8} \right) \frac{4}{3} \pi n^3 = \frac{4 \pi f}{3} \left( \frac{Lp}{h} \right)^3~
\]
这就是\(f\)乘以半径为\(n\)的球体的体积除以八,因为只考虑具有正\(n_i\)的八分之一区域。使用连续近似,动量大小介于\(p\)和\(p+dp\)之间的状态数为:
\[
dg = \frac{\pi}{2} f n^2 \, dn = \frac{4 \pi f V}{h^3} p^2 \, dp~
\]
其中\(V=L^3\)是盒子的体积。注意,在使用这个连续近似(也称为Thomas−Fermi近似)时,失去了对低能态的表征能力,包括自旋量子数\(n_i=1\)的基态。对于大多数情况,这不会是一个问题,但在考虑玻色−爱因斯坦凝聚时,气体中有很大一部分处于或接近基态,此时处理低能态的能力变得非常重要。

不使用任何近似,具有能量\(\varepsilon_i\)的粒子数为:
\[
N_i = \frac{g_i}{\Phi(\varepsilon_i)}~
\]
其中,\(g_i\)是状态i的简并度,Φ(εᵢ)为:
\[
\Phi(\varepsilon_i) =
\begin{cases}
e^{\beta (\varepsilon_i - \mu)}, & \text{对于遵循麦克斯韦-玻尔兹曼统计的粒子} \\
e^{\beta (\varepsilon_i - \mu)} - 1, & \text{对于遵循玻色-爱因斯坦统计的粒子} \\
e^{\beta (\varepsilon_i - \mu)} + 1, & \text{对于遵循费米-狄拉克统计的粒子}
\end{cases}~
\]
其中,\(\beta= 1/K_BT\),\(K_B\)是玻尔兹曼常数,\(T\)是温度,\(\mu\)是化学势。(见麦克斯韦-玻尔兹曼统计、玻色-爱因斯坦统计和费米-狄拉克统计。)

使用托马斯-费米近似,能量在\(E\)和\(E+dE\)之间的粒子数\(dN_E\)为:
\[
dN_E = \frac{dg_E}{\Phi(E)}~
\]
其中,\(dg\)ₑ是能量在\(E\)和\(E+dE\)之间的状态数。
\subsection{能量分布}  
利用本文前面各节中得到的结果,可以确定盒子中气体的一些分布。对于一个粒子系统,变量\( A \)的分布\( P_A \)定义为表达式  
\[
P_A dA = \frac{dN_A}{N} = \frac{dg_A}{N \Phi_A}~
\]
其中,\( P_A dA \)是粒子中变量\( A \)的值位于\( A \)和\( A + dA \)之间的比例。

其中:
\begin{itemize}
\item \( dN_A \),具有变量 \( A \) 在 \( A \) 和 \( A + dA \) 之间的值的粒子数;
\item \( dg_A \),具有变量 \( A \) 在 \( A \) 和 \( A + dA \) 之间的值的状态数;
\item \( \Phi_A^{-1} \),一个状态具有值 \( A \) 的概率被粒子占据;
\item \( N \),总粒子数。
\end{itemize}
由此可得:
\[
\int_A P_A dA = 1~
\]
对于动量分布\( P_p \),具有动量大小在\( p \)和\( p + dp \)之间的粒子所占的比例为:
\[
P_p dp = \frac{Vf}{N} \cdot \frac{4\pi}{h^3 \Phi_p} \cdot p^2 dp~
\]
对于能量分布\( P_E \),具有能量在\( E \)和\( E+dE \)之间的粒子所占的比例为:
\[
P_E dE = P_p \frac{dp}{dE} dE~
\]
对于盒子中的粒子(以及自由粒子),能量\( E \)和动量\( p \)之间的关系对于有质量粒子和无质量粒子是不同的。对于有质量粒子,
\[
E = \frac{p^2}{2m}~
\]
而对于无质量粒子,
\[
E = pc~
\]
其中\( m \)是粒子的质量,\( c \)是光速。根据这些关系,

对于有质量粒子:
\begin{itemize}
\item 
\[
dg_E = \left( \frac{Vf}{\Lambda^3} \right) \frac{2}{\sqrt{\pi}} \beta^{3/2} E^{1/2} dE~
\]
\[
P_E dE = \frac{1}{N} \left( \frac{Vf}{\Lambda^3} \right) \frac{2}{\sqrt{\pi}} \frac{\beta^{3/2} E^{1/2}}{\Phi(E)} dE~
\]
其中\( \Lambda \)是气体的热波长,
\[
\Lambda = \sqrt{\frac{h^2 \beta}{2\pi m}}~
\]
这是一个重要的量,因为当\( \Lambda \)大约等于粒子间的平均距离\( (V/N)^{1/3} \)时,量子效应开始占主导地位,气体不再能被认为是麦克斯韦-玻尔兹曼气体。
\item对于无质量粒子:
\[
dg_E = \left( \frac{Vf}{\Lambda^3} \right) \frac{1}{2} \beta^3 E^2 dE~
\]
\[
P_E dE = \frac{1}{N} \left( \frac{Vf}{\Lambda^3} \right) \frac{1}{2} \frac{\beta^3 E^2}{\Phi(E)} dE~
\]
其中\( \Lambda \)是无质量粒子的热波长,
\[
\Lambda = \frac{ch\beta}{2 \pi^{1/3}}~
\]
\end{itemize}
\subsection{具体示例}  
以下章节给出了某些特定情况下的结果示例。
\subsubsection{有质量的麦克斯韦-玻尔兹曼粒子}  
对于这种情况:
\[
\Phi(E) = e^{\beta (E - \mu)}~
\]
对能量分布函数进行积分并解出\( N \)得到:
\[
N = \left( \frac{Vf}{\Lambda^3} \right) e^{\beta \mu}~
\]
将其代入原始的能量分布函数中,得到:
\[
P_E dE = 2 \sqrt{\frac{\beta^3 E}{\pi}} e^{-\beta E} dE~
\]
这些结果与经典的麦克斯韦-玻尔兹曼分布得到的结果相同。更多的结果可以在关于理想气体的经典部分中找到。
\subsubsection{有质量的玻色–爱因斯坦粒子}  
对于这种情况:
\[
\Phi(E) = \frac{e^{\beta E}}{z} - 1~
\]
其中
\[
z = e^{\beta \mu}~
\]
对能量分布函数进行积分并解出\( N \)得到粒子数:
\[
N = \left( \frac{Vf}{\Lambda^3} \right) \text{Li}_{3/2}(z)~
\]
其中\( \text{Li}_{3/2}(z) \)是多对数函数。多对数项必须始终为正实数,这意味着它的值将在\( z \)从 0 到 1 时,从 0 增加到\( \zeta(3/2) \)。随着温度降至零,\( \Lambda \)将变得越来越大,直到最终\( \Lambda \)达到一个临界值\( \Lambda_c \),此时\( z = 1 \),并且
\[
N = \left( \frac{Vf}{\Lambda_c^3} \right) \zeta(3/2)~
\]
其中\( \zeta(z) \)表示黎曼ζ函数。温度\( \Lambda = \Lambda_c \)时的温度被称为临界温度。对于低于此临界温度的温度,上述粒子数的方程没有解。临界温度是玻色–爱因斯坦凝聚开始形成的温度。问题在于,如上所述,在连续近似中忽略了基态。然而,事实证明,上述粒子数方程很好地表达了处于激发态的玻色子数,因此:
\[
N = \frac{g_0 z}{1 - z} + \left( \frac{Vf}{\Lambda^3} \right) \text{Li}_{3/2}(z)~
\]
其中加上的项是基态中的粒子数。基态能量被忽略了。这个方程将在零温度下成立。更多的结果可以在关于理想玻色气体的文章中找到。
\subsubsection{无质量的玻色–爱因斯坦粒子(例如黑体辐射)}  
对于无质量粒子的情况,必须使用无质量能量分布函数。将此函数转换为频率分布函数是方便的:
\[
P_{\nu} d\nu = \frac{h^3}{N} \left( \frac{Vf}{\Lambda^3} \right) \frac{1}{2} \frac{\beta^3 \nu^2}{e^{(h\nu - \mu) / k_{\rm B} T} - 1} d\nu~
\]
其中 \( \Lambda \) 是无质量粒子的热波长。然后,谱能量密度(单位频率单位体积的能量)为:
\[
U_{\nu} d\nu = \left( \frac{N h \nu}{V} \right) P_{\nu} d\nu = \frac{4 \pi f h \nu^3}{c^3} \frac{1}{e^{(h\nu - \mu) / k_{\rm B} T} - 1} d\nu.~
\]
其他热力学参数可以类比于有质量粒子的情况推导出来。例如,对频率分布函数进行积分并解出\(N\)得到粒子数:
\[
N = \frac{16 \pi V}{c^3 h^3 \beta^3} \, \mathrm{Li}_3 \left( e^{\mu / k_{\rm B} T} \right).~
\]
最常见的无质量玻色气体是黑体中的光子气体。将“盒子”视为黑体腔体时,光子不断被壁面吸收和重新发射。在这种情况下,光子的数量是不守恒的。在玻色–爱因斯坦统计的推导中,当粒子数的限制被去除时,这实际上等同于将化学势(\(\mu\))设置为零。此外,由于光子有两种自旋态,因此\(f\)的值为 2。此时,谱能量密度为:
\[
U_{\nu} d\nu = \frac{8 \pi h \nu^3}{c^3} \frac{1}{e^{h\nu / k_{\rm B} T} - 1} d\nu~
\]
这就是普朗克黑体辐射定律的谱能量密度。注意,如果对无质量麦克斯韦-玻尔兹曼粒子进行此过程,将恢复维恩分布,这对于高温或低密度情况下的普朗克分布是一种近似。