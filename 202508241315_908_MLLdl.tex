% 莫雷拉定理(综述)
% license CCBYSA3
% type Wiki

本文根据 CC-BY-SA 协议转载翻译自维基百科\href{https://en.wikipedia.org/wiki/Morera\%27s_theorem}{相关文章}。

在复分析中,莫雷拉定理(Morera's theorem,以贾钦托·莫雷拉 Giacinto Morera 命名)给出了一个判断函数是否为全纯函数的判据。设 $f$ 是定义在复平面开集 $D$ 上的连续复值函数,如果它对 $D$ 中的每一条分段 $C^1$ 的闭曲线 $\gamma$ 都满足
$$
\oint_{\gamma} f(z)\,dz = 0,~
$$
那么 $f$ 必定在 $D$ 上是全纯函数。这个条件等价于 $f$ 在 $D$ 上存在一个原函数(反导函数)。不过该定理的逆命题一般并不成立:一个全纯函数并不一定在其定义域上都有原函数,除非附加额外条件。如果定义域 $D$ 是单连通的,那么逆命题成立,这正是柯西积分定理的内容,即全纯函数沿闭合曲线的积分为零。

一个典型的反例是函数 $f(z) = 1/z$,它在 $\mathbf{C} - \{0\}$ 上是全纯的。在 $\mathbf{C} - \{0\}$ 中任意一个单连通邻域 $U$ 上,$1/z$ 都有一个原函数,可以写为$L(z) = \ln(r) + i\theta, \quad z = re^{i\theta}$.
由于 $\theta$ 的取值可以相差任意整数倍的 $2\pi$,所以在区域 $U$ 中,只要能连续地选取 $\theta$ 的分支,就可以在该 $U$ 上定义出 $1/z$ 的一个原函数。(无法在包含原点的闭合曲线上连续地定义 $\theta$,正是 $1/z$ 在整个定义域 $\mathbf{C} - \{0\}$ 上没有原函数的根本原因。)此外,由于常数的导数为 0,原函数中加上任意常数仍然是 $1/z$ 的一个原函数。

从某种意义上说,$1/z$ 这个反例具有普遍性:对于任何在其定义域上没有原函数的解析函数,其根本原因都可以追溯到 $1/z$ 在 $\mathbf{C} - \{0\}$ 上没有原函数这一事实。
\begin{figure}[ht]
\centering
\includegraphics[width=6cm]{./figures/3611ee82bcb8dce2.png}
\caption{} \label{fig_MLLdl_1}
\end{figure}
\subsection{证明}