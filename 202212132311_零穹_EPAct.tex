% 端点可变的作用量
% 作用量的偏导数|作用量的全微分
\pentry{作用量原理\upref{HamPrn}}
本词条将证明几个个关于作用量
\begin{equation}
S=\int_{t_1}^{t_2}L\dd t
\end{equation}

的重要公式,其中两个是重要的,即作用量对时间的偏导数等于负的能量(哈密顿量),对坐标的偏导数等于动量对应分量,或说作用量的梯度等于动量.其数学形式为
\begin{equation}\label{EPAct_eq1}
\begin{aligned}
\pdv{S}{t}&=-H\\
\pdv{S}{q^i}&=p_i
\end{aligned}
\end{equation}
其中 $t$ 是作用量积分里的积分上限(末时刻) $t_2$,$q^i$ 是对应末时刻 $t$ 的广义坐标.

并且
\begin{equation}\label{EPAct_eq2}
\dd S=\sum_i p_i^{(2)}\dd {q^i}^{(2)}-H^{(2)}\dd t^{(2)}-\sum_i p_i^{(1)}\dd {q^i}^{(1)}+H^{(1)}\dd t^{(1)}
\end{equation}
这里,上标 $(1),(2)$ 分别代表起点和终点对应值.

既然\autoref{EPAct_eq1} 是关于终点的偏微分,\autoref{EPAct_eq2} 是关于两端点的微分,这就是说这里的作用量实际上是端点可变的作用量.

可能读者已经疑惑了,作用量的自变量不应是个函数么?证明这里的自变量是起止时刻和初末位置了.事实上,我们要找的作用量对应物理系统的演化,那么系统演化的曲线是使作用量取极值的曲线,而在端点和起止时刻确定时系统演化我们认为只有一个,那么作用量就可看成这一极值曲线的两端点和对应起止时刻的函数.
\subsection{证明:}
这里的公式事实上和变分学的端点可变问题\upref{EPQue} 中的一样,那里有更严格的证明,只需明确物理意义即可.然而,我们这里给出较之更适合物理人的证明,以避免深入了解变分学.

\autoref{EPAct_eq1} 两式对应的问题可以这样描述:1.初末位置及初始时刻固定,而末时刻可变作用量如何变化.
2.将 $S$ 看着系统真实演化的量,当初始位置 $q(t_1)=q^{(1)}$ 固定但 $t_2$ 时刻对应位置可变时,$S$ 如何变.

我们先证明情形2对应的等式.注意到
\begin{equation}
\delta S=\left.\pdv{L}{\dot q^i}\delta q^i\right|_{t_1}^{t_2}+\int_{t_1}^{t_2}\qty(\pdv{L}{q}-\dv{}{t}\pdv{L}{\dot q})\delta q\dd t
\end{equation}



