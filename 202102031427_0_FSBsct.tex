% 方势垒散射数值计算

\begin{issues}
\issueDraft
\issueOther{未完成: 需要改进程序, 计算波函数随时间的变化. (给系数加上 $\exp(-\I E t)$ 即可)}
\issueOther{未完成: 用行波来投影, 验证\autoref{SqrPot_eq3}~\upref{SqrPot}}
\end{issues}

\pentry{方势垒的定态波函数\upref{SqrPot}}
已完美还原高斯波包, 参考\href{https://chaoli.club/index.php/4541/}{超理论坛的讨论}.

\begin{figure}[ht]
\centering
\includegraphics[width=10cm]{./figures/FSBsct_1.png}
\caption{运行结果} \label{FSBsct_fig1}
\end{figure}

\begin{lstlisting}[language=matlab, caption=FSBsct.m]
clear; close all;

% === params ====
k0 = 4; a = 1/6;
Nk = 300; kmax = k0 + 12*a; kmin = max(kmax/Nk, k0 - 12*a);
x0 = 0;
Nx = 2000; xmin = x0-6/a; xmax = x0+6/a;
psi = @(x)exp(1i*k0*x).*exp(-(a*(x-x0)).^2);
L = 3; V0 = 1; m = 1;
% ==============

% plot FSB
x = linspace(xmin, xmax, Nx); dx = x(2)-x(1);
figure; plot(x, real(psi(x)));
E = 0.9;
[psi_e,A_e,C_e,D_e] = FSB(x,E,L,m,V0,0);
hold on; plot(x, psi_e);
[psi_o,A_o,C_o,D_o] = FSB(x,E,L,m,V0,1);
plot(x, psi_o);
xlabel x; title '\psi(x) and \psi_k(x)'

% verify orthogonality
E1 = 1.5; E2 = 1.6; k1 = sqrt(2*m*E1); k2 = sqrt(2*m*E2);
[~,A1,C1,D1] = FSB(0,E1,L,m,V0,0);
[~,A2,C2,D2] = FSB(0,E2,L,m,V0,0);
ph1 = atan2(C1,D1); ph2 = atan2(C2,D2);
I = integral(@(x)FSB(x,E1,L,m,V0,0).*FSB(x,E2,L,m,V0,0)...
    -sin(k2*x+ph2).*sin(k1*x+ph1)/pi, 0, L);
I_expect = sin(ph2-ph1)/(k2-k1)/2/pi - sin(ph1+ph2)/(k1+k2)/2/pi;
disp([I, I_expect]);

% plot FFT
k = linspace(kmin, kmax, Nk); dk = k(2)-k(1);
[g1, k1] = FFT(fftresize(psi(x),Nx*2), dx);
g = interp1(k1, g1, k);
figure; plot(k, abs(g), '.');
axis([kmin, kmax, 0, max(abs(g))*1.1]);
xlabel k; title 'FFT of \psi(x)';

A = zeros(1,Nk); B = A;
E = k.^2/(2*m);
A0 = 0; %integral(@(x)psi(x), xmin, xmax);
psi1 = A0*ones(size(x));
for i = 1:Nk
    A(i) = integral(@(x)psi(x).*FSB(x,E(i),L,m,V0,0), ...
            xmin, xmax, 'RelTol', 1e-16);
    B(i) = integral(@(x)psi(x).*FSB(x,E(i),L,m,V0,1), ...
            xmin, xmax, 'RelTol', 1e-16);
    psi1 = psi1 + dk*A(i)*FSB(x,E(i),L,m,V0,0) ...
                + dk*B(i)*FSB(x,E(i),L,m,V0,1);
end
figure; plot(x, real(psi(x)));
hold on; plot(x, real(psi1), '--');
plot([xmin, -L,-L, L, L, xmax], [0, 0, 0.5, 0.5, 0, 0], 'b');
axis([xmin,xmax,-1,1]);
title 'reconstruction of \psi(x)'
\end{lstlisting}
