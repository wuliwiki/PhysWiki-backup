% Linux 命令笔记

\pentry{Linux 基础\upref{Linux}}

\subsection{基础}
\begin{itemize}
\item \href{http://faculty.salina.k-state.edu/tim/unix_sg/index.html}{一个教程}
\item Linux 命令行区分大小写
\item \verb`man 命令`  会给出命令的帮助 (manual)
\item \verb`clear` 清空命令行(老版本中其实是向下滚动一页), \verb|Ctrl+L| 也可以。
\item 几乎任何时候, Ctrl+C (有时候 Ctrl+D, 例如退出 bash) 可以强制终止当前的工作 (注意只用于可以安全终止的任务!)
\item \verb`^` 代表 Ctrl 键
\item Ubuntu 中, \verb|Ctrl++| 就是放大, 记住, 是 + 不是 =.
\item Linux 并没有文件拓展名...
\item 可执行文件: 对 linux 来说缀名没有意义。 可执行文件如果在 system path, 就可以直接打文件名执行命令(一般没有拓展名)。 如果在某个目录, 就打 \verb`目录/文件名` 执行。 如果在当前目录就用 \verb`./文件名` 执行
\item 除了二进制可执行文件, 文本文件也可以被看作可执行文件(与拓展名无关)。 默认用 bash 执行, 但如果有 shebang, 就用 shebang 指定的程序执行
\item shebang 在文件第一行以 \verb`#!` 开头, 如 \verb`#!/bin/bash` (默认) 或者 \verb`#!/usr/bin/python`
\item 命令的短选项可以合并, 例如 \verb`ls -l -h` 可以写成 \verb`ls -lh`. 有些段选项有对应的长选项, 如 \verb`ls -l` 相当于 \verb`ls --human-readable`.
\item \verb`ls -1` 可以将文件名列成一列, \verb`ls -v` 可以按照数字的大小排序(否则是逐位排序, 如 1,10,100,2,10,200), \verb`ls *.md` 可以只列出 md 文件
\item \verb|ls ./**/文件名| 可以任意层子文件夹中的 \verb|文件名|。 如果无效, 需要先设置 \verb|shopt -s globstar|。
\item \verb|stat 文件| 可以显示一些 metadata, 例如大小权限, 最后访问和修改时间等。
\item 要区分命令的 argument 和命令选项的 argument, 后者紧跟在选项后面.
\item 使用 Tab 键可以自动完成命令, 安装 \verb|bash-completion| 软件可以让该功能更完善
\item \verb`Ctrl-D` 可以关闭当前 terminal
\item \verb`$变量` 或 \verb|${变量}| 表示变量的值, 自定义变量可以用 \verb`变量=值`, 注意等号两边不能有空格。
\item \verb`env` 命令可以查看环境变量
\item 在 \verb`/etc/profile` 文件的底部定义变量可以让所有人登录后都获得该变量。如果要只对一个用户创建该变量, 在其 home 目录创建 \verb|.profile| 文件即可。 另外也可以在 \verb|~/.bashrc| 中 \verb|export 变量=值|
\item 每次打开新的 shell,\verb`/etc/bashrc` 和 \verb`~/.bashrc` 文件都会被执行
\item \verb|history| 命令可以输出最近 1000 条输入的命令
\item 用 \verb`!数字` 可以重新输入第 \verb`数字` 条历史命令
\item \verb`df` 用于查看磁盘空间, \verb`-h` 选项可以显示 \verb`Mb`, \verb`Gb` 等, \verb|-T| 选项可以增加显示文件系统类型.
\item \verb`du /some/path` 用于查看指定目录中所有子文件夹的大小, \verb`-a` 选项可以显示文件夹和文件. \verb`--max-depth=N` is used to control the level of subfolders to display, \verb`-h` is for human readable file size (\verb`Mb`, \verb`Gb`, etc)
\item \verb|du -hd 0| 用于查看当前目录的大小(不会跟随 symlink, 但是会包括 mount)。 \verb|-h| 是 human readable, \verb|-d| 是 depth。 \verb|-x| 不会计入被 mount 的文件夹。
\item \verb`ctrl+u` 可以删除光标前面的所有内容
\item \verb`ctrl+c` 可以放弃当前输入直接开始新的一行
\item \verb|which 命令| 可以查看某个命令的文件位置, \verb|whereis 命令| 还可以查看源码、设置文件和文档等的位置(如果有)。
\item 大括号可以展开, 例如 \verb|touch 路径/{文件1,文件2}| 相当于 \verb|touch 路径/文件1 路径/文件2| 也可以用两点如 \verb|touch 路径/文件{1..100}| 可以生成 100 个文件。 如果有多个大括号就做 “张量积”, 例如 \verb|{a,b}/{1,2}| 展开为 \verb|a1 a2 b1 b2|。 如果需要指定步长就 \verb|{1..10..3}| 代表 \verb|1,4,7,10|。
\end{itemize}

\subsection{目录与文件}
\begin{itemize}
\item \verb`pwd` (present working directory)
\item \verb`pwd -P` 可以显示当前的绝对目录, 即不含 \verb`~` 等符号以及 symbolic link
\item \verb`cd` (change directory) \verb`/` (硬盘根目录) \verb`~` (缩写 /home/parallels) (用户的 home 文件夹)
\item 只用 \verb`cd` 相当于 \verb`cd ~`
\item \verb`./` 表示当前目录,常用于执行可执行文件或者 \verb`.sh`
\item 常用目录有 \verb`~/Documents` ,  \verb`~/Desktop`, \verb`/usr/bin` (gfortran 安装在这里)
\item \verb`cd 文件夹名` 只能用于当前目录中的文件夹
\item \verb`cd ..` 返回上层目录
\item \verb`cd -` 返回刚才所在的目录
\item 要 cd 到当前目录的绝对目录, 可以用 \verb|cd `pwd -P`|, 两个 \verb|`| (backtick)内的东西会先展开
\item \verb`rm 文件名` 或者 \verb`rm /目录/文件名` 删除文件  (注意是永久删除!)
\item \verb`mv 文件名 目录` 可以移动文件, \verb`mv 文件名 新文件名` 可以重命名。 注意 \verb|mv| 会覆盖文件, 但是不会覆盖文件夹(只要目标的同名文件夹非空就不行, 即使里面没有同名文件)。 此时可以用 \verb|rsync|。
\item \verb`cp 文件名 目录` 复制文件 \verb`cp 文件名 新文件名` 复制到当前目录且重命名. 复制多个文件用 \verb`cp 文件1 文件2 目录`, 也可以在 \verb|目录| 前面加 \verb|-t|。 注意 \verb|cp| 会覆盖目标文件以及文件夹。
\item \verb|cp| 在只拷贝一个 symlink 是会拷贝它指向的文件而不是 symlink 本身。 \verb|cp -d| 选项可以只拷贝 link。
\item \verb|cp -r 文件夹| 时, 如果文件夹本身不是 symlink, 不会跟随文件夹中包含的 symblink。 除非使用 \verb|-L| 或者 \verb|--dereference|, 那么将不会在目标中创建任何 symlink, 而是全部展开。
\item \verb|cp| 会默认覆盖目标的同名文件, \verb|cp -n| 可以不覆盖且跳过(但不会提示)。 如果要提示且手动逐个选择是否覆盖, 用 \verb|-i| 选项。
\item \verb|-a| (\verb|--archive|)相当于 \verb|-r --no-dereference --preserve=all| all 包括 \verb|mode,ownership,timestamps,context,links,xattr|
\item 要使用外接储存器(U盘, 移动硬盘等), 需要新建一个文件夹, 然后连接外接储存器, 叫做 Mount The Drive. 习惯上, 新建文件夹都放在 \verb`/mnt` 目录下
\item 要连接储存器, 用 \verb`sudo fdisk -l` 显示所有的外接硬盘的路径名
\item \verb`sudo mount <硬盘地址> <新建文件夹路径>`
\item \verb`sudo umount <硬盘地址>` 断开连接
\item \verb`sudo mount -a` 把 \verb|/etc/fstab| 中的设置全部 mount 一次
\item 创建 symbolic link (有点相当于快捷方式), \verb`ln -s /path/to/file /path/to/symlink` 如果目标已经有同名文件或软链存在, 就用 \verb|ln -sfn ...| (如果目标是软链, 不加 \verb|n| 无效)。
\item \verb|readlink 软链| 或者 \verb|ls -l 软链| 都可以查看软链的内容。 注意软链的内容可以是相对或者是绝对路径。
\item \verb|readlink -f 相对路径| 可以把相对路径变为绝对路径(展开所有软链,最后没有 \verb|/|)。 \verb|相对路径|
\item \verb|realpath 路径|
\item \verb|dirname 相对或绝对路径/文件名或文件夹| 返回 \verb|相对或绝对路径| (最后没有 \verb|/|)。
\item \verb`mkdir` 新建文件夹, \verb`rmdir` 删除空文件夹, \verb`rmdir -r` (recursive) 删除所包含的文件夹(其实不用 \verb`rmdir` 用 \verb`rm` 也可以但有点危险), \verb`-P` 选项可以一次创建多层目录
\item \verb|dos2unix 文件1 文件2...| 可以把 Windows 文件中的 CRLF 转换为 Linux 中的 LF 换行。 
\item \verb`dd` 命令可以直接在当前位置生成一个大小为 100 MB 的文件, 内容为随机, 还会显示写入速度
\verb`dd if=/dev/urandom of=./dump.txt bs=1M count=1024` 写 1024Mb 随机文件
\item \verb|dd conv=notrunc if=... of=文件名 bs=.. seek=第几字节 count=..|  从文件的某个字节开始修改文件(覆盖内容)。
\item \verb|dd| 也可以用来复制文件, 可以用来测试传输速度。
\item \verb`locate ???` 命令可以从数据库中寻找计算机上所有文件名包含 \verb`???` 的文件, 但刚创建的文件可能找不到
\item \verb`sudo updatedb` 可以将刚创建的文件加到数据库中, 让 \verb`locate` 可以找到(\verb|-v| 选项显示哪些文件夹被更新)
\item 对于整个系统, \verb`locate` 比 \verb`find` 要快得多, 因为数据库已经是 index 过的 \verb`-c` 选项可以显示匹配结果的个数, \verb`-i` 选项可以不区分大小写
\item \verb|plocate| 是一个比较快的实现, \verb|mlocate| 是一个比较安全的实现, 命令都是 \verb|locate|, \verb|/etc/updatedb.conf| 中可以设置 \verb|updatedb| 的一些选项, 例如忽略挂载到某个目录的硬盘, 忽略 \verb|.git| 文件夹等, 忽略某些文件系统等。
\item 文件批量重命名 \verb|rename 's/老名字/新名字/' *|。 \verb|-n| 选项显示重命名的预览, 不真的重命名
\item \verb|sudo swapoff -a| 不使用 swap file, \verb|sudo swapon -a| 恢复使用。 可能还需要在 \verb|/etc/fstab| 里面注释掉相应的行。
\item 要改变 swap 文件的尺寸: \verb|sudo swapoff /swapfile|, \verb|sudo dd if=/dev/zero of=/swapfile bs=1M count=1024 oflag=append conv=notrunc|, \verb|sudo mkswap /swapfile|, \verb|sudo swapon /swapfile| 一般无需重启, 但如果系统不允许关闭 swap, 也可以考虑用多个 swap 文件。有时候 \verb|/swapfile| 也叫做 \verb|/swap|。
\item 用 \verb|free -h| 可以查看当前 RAM 和 swap 的大小和使用情况。
\item 如果要开机自动使用 swapfile, 用 \verb|/swapfile none swap sw 0 0|
\item \verb|tree| (非系统自带) 可以显示当前目录的树状图, \verb|tree -a| 也显示隐藏文件。
\item \verb|sort| 可以排列 stdin 输入的列。 \verb|-k| 可以按列排序
\item \verb`cmp --silent 文件1 文件2 || echo "files are different"` 可以用于比较两个文件是否相同。
\end{itemize}

\subsubsection{rsync}
\begin{itemize}
\item 可以用于同步和备份文件(后者很广泛), 支持 ssh 进行远程传输, 也支持本地传输。
\item 格式 \verb`rsync -avzh src1 src2 ... dest` 其中 \verb|src1| 等文件或文件夹会自动同步到 \verb|dest| 文件夹中(例如生成 \verb`dest_path/src1` 等)。 如果 \verb|dest| 不存在, 会自动用 mkdir 生成,但只能生成一层。 file, folder 和 \verb|dest_path| 都可以是远程或者本地
\item 例如从远程传到本地: \verb|rsync -azvh 用户名@域名或IP:/路径/* 本地路径/|。 \verb`-a` 是 archive (包括所有子文件夹, 文件的 permission 和 symlink 都会原封不动复制), \verb`-v` 是 verbose (显示详情), \verb`-z` 是 zip (传输过程压缩) \verb`-h` 人可读。
\item 如果在不同文件系统之间, 用了 \verb|sudo| 也还是显示 \verb|Operation not permitted|, 可以把 \verb|-a| 选项改为 \verb|-r|。
\item \verb|–dry-run| 可以看看哪些文件会修改
\item 若要在每个文件传输完成之后从原来文件夹删除, 用 \verb`--remove-source-files`。 注意不会删除任何目录。 可以另外用 \verb|find src -type d -empty -delete| 删除空目录。
\item 同步文件夹时, 用 \verb|源文件夹/| 不会包括 \verb|源文件夹| 本身, 而去掉斜杠就会。
\item 若要删除目标文件夹中, 源文件夹不存在的内容, 用 \verb`--delete`。
\item 常用的同步命令: \verb|rsync -avh --delete 目录1/ 目录2/|。 完成后, 两个目录的内容将一模一样。
\item 如果远程目录有需要 \verb`\` 的字符, 例如 \verb`\(...\)`, 需要用三次, 例如 \verb`\\\(...\\\)`。 本地的 shell 会先把 \verb`\\` 变为 \verb`\`, \verb`\(` 变为 \verb`(`
\item \verb|--checksum| 选项: 如果文件大小和修改时间都相同, 会用 checksum 检查它们是否真的相同。 如果不加, rsync 会直接跳过大小和修改时间都相同的同名文件。
\item 当两个文件只有部分不同时, rsync 会用 delta-transfer algorithm, 只传输不同的部分。 \verb|--whole-file| 可以取消这个功能。
\end{itemize}


\subsubsection{split}
\begin{itemize}
\item 把文件 \verb`file` 分割成 \verb`file-01`, \verb`file-02`, ...
\item \verb`split -b 500000000 -d file  path/to/file-`
\item \verb`-d` 拓展名用数字而不是字母
\item \verb`-b 10` 划分成 10 字节的小块
\item \verb`-l 4` 划分成每个文件 4 行
\item \verb`-n 10` 把文件 (几乎) 等分为 10 份
\item \verb`-a 4` 拓展名位数为 4
\item 如果文件个数超出拓展名位数, 那么位数会自动增加, 且增加后第一位不变以保持正确排序
要缝合分割后的文件, 用 \verb`cat file-* > file`
\end{itemize}


\subsubsection{find}
\begin{itemize}
\item \verb`find /path1/ path2/ -name '<filename>'` 可以搜索该目录下的文件. 加上 \verb`-iname` 选项可以不区分大小写. 也可以用 \verb`'*<filename>*'` 来表示含该文字的文件名. 加 \verb`-user 用户名` 选项用于查找某个用户的文件, 加 \verb`-size +100M` 可以查找大于 100M 的文件。 \verb`-type d` 选项搜索目录, 默认是只搜索文件
\item 如果要执行, 用 \verb`find ... -exec xxxx {} \;` 可以将找到的每个结果分别用程序 \verb`xxxx` 执行, \verb`find ... -exec xxxx {} +` 将找到的结果用 \verb`xxxx` 一次执行(如果数量太多也可能自动分为几次)。 其中 \verb`{}` 可以重复使用。
\item \verb`-execdir ... {} ... \;` 相当于先 \verb`cd` 到每个文件所在目录然后执行命令
\item 删除当前目录以及子目录的空文件夹 \verb`find . -empty -type d -delete` 删除空文件 \verb`find . -empty -type f -delete`
\item \verb|find . -type f -executable| 找所有的可执行文件。
\item Linux 用 \verb`find 文件夹 -type f -exec sha1sum {} \; | sort | sha1sum` 可以对比两个文件夹。 对比包含文件名, 每个文件的路径, 和每个文件(包括隐藏文件)的 hash, 不包括 metadata (例如权限, 最后修改时间等), 不包括空目录, 不包括软链。
\item 如果要对比两个文件夹中所有包含的软链(它们的名字,路径以及内容), 对他们分别用 \verb`find test -type l -exec printf "{} -> " \; -exec readlink {} \; | sort | sha1sum`
\end{itemize}

\subsubsection{系统目录}
\begin{itemize}
\item \verb|/home/用户名| 普通用户的用户文件
\item \verb|/root| 管理员的用户文件
\item \verb|/boot| 系统内核
\item \verb|/etc| 系统设置
\item \verb|/usr| 在用户之间分享的文件
\item \verb|/var| 各种数据文件包括 log
\end{itemize}
\begin{figure}[ht]
\centering
\includegraphics[width=10cm]{./figures/4dbb5e21417b5813.png}
\caption{根目录} \label{fig_LinNt_2}
\end{figure}
详细标准见\href{https://www.pathname.com/fhs/}{这里}。

\subsection{包管理}
\begin{itemize}
\item 详见 “Linux 包管理笔记(apt, dpkg, snap)\upref{snap}”
\item \verb`sudo apt-get install <appname>` 安装程序(99\%的程序都可以这样安装)(需要联网)
\item \verb`sudo apt-get remove <appname>` 卸载程序
\item \verb`sudo apt-get upgrade <appname>` 升级程序
\item \verb`apt-get` (和 \verb`apt-cash`, 如果听说过)中常用的功能都在 \verb`apt` 中有同样的, 所以大部分情况下只需要 \verb`apt` 就可以了, 例如 \verb`apt install`, \verb`apt purge`。 \verb`apt` 对新手更友好, 例如会默认显示进度条, 更新的 package 数等。
\item linux 里面的程序设置什么的基本都储存在对应的 configuration file 里面. 通常这种文件以 \verb|.conf| 结尾
\item \verb`apt show 包` 可以显示某个包的信息
\item 如果安装了例如 \verb`libblas`, \verb`libgsl` 等 library, 用 \verb`dpkg -L 包` 可以查看头文件和库文件(\verb`.a`, \verb`.so`) 的位置。
\item 如果包的名字以 \verb`-dev` 结尾, 通常意味着这个包提供头文件和库文件, 可以让你在写程序时调用
\item 如果包的名字以 \verb`-dbg` 解为, 通常意味着这个包的二进制中有调试信息, 可以在 debugger 里面分析
\item \verb|apt list --installed| 列出所有安装了的包
\item 不同的 ubuntu 版本可以用 apt 安装的包的版本也是有一些区别的。
\item \verb|dpkg| 用于手动(通过安装包文件)安装和卸载指定的包, 不会自动安装或者卸载 dependency, 也不会从网上下载包。 \verb|apt| 会调用 \verb|dpkg|。
\item \verb|aptitude| 是比 \verb|apt| 更高级的包管理器, 还有 GUI 界面。
\item 下载文件用 \verb|wget [-q] 网址 [-O 新文件名]| 其中 \verb|-q| 是安静模式, \verb|-O| 是把下载的文件重命名。
\end{itemize}

\subsection{文本相关}
\begin{itemize}
\item \verb`touch 文件名` 命令可以生成一个空的文本文件, 也可以以更新文件的修改时间。
\item 但是比较著名的文本编辑还是 \verb|vim|
\item \verb`nano` 是一个轻量级命令行文本编辑器, 可以编辑文本文件. 打开文件命令: \verb`nano 文件名`。 但一般还是用 vim。
\item \verb`head` 和 \verb`tail` 命令可以预览一个文件的前几行和后几行
\item \verb`echo <text>` 可以在命令行显示该文字, 也就是把命令的 args 输出到 stdout
\item \verb`echo <text> > <filename>` 可以把文件中的文本替换 \verb`>>` 可以附加在文本最后. 如果用双引号, 甚至还可以换行.
\item \verb`>` 是 output redirect 操作, 类似, \verb`<` 是 input redirection 操作, 可以把文档中的内容输入到命令行.
\item 如果 \verb|>| 或者 \verb|>>| 的对象需要 \verb|sudo| 权限, 那么用 \verb|tee|, 如 \verb`echo 一些文字 | sudo tee [-a] 文件` 其中 \verb|-a| 是 append。
\item linux 命令行有两个主要的 output, stdout 和 stderr, \verb`>` 只用于将 stdout 输出到文件, 如果也要输出 stderr 有三种办法: 1.\verb`<command> > file1.out 2 > file2.out`, 2. \verb`<command> > file.out 2 > &1`, 3. \verb`<command> &> file.out` 其中后两种将 stdout 和 stderr 输出到同一文件.
\item 运行一个程序,指定输入输出文件, 并让 standartd err 显示在 standard out \verb`./program.x < <inputfile> > <outputfile> 2>&1` 
\item \verb`>` 的输出文件如果不存在, 就会新建一个.
\item  \verb`cat 文件` 把文件的内容输出到 stdout
\item  \verb`cat` 把 stdin 的内容输出到 stdout
\item \verb`命令 | less` 可以用键盘滚动输出结果(用于长输出以及没有滚动条的情况), 按 q 键退出
\item \verb`|` 的作用相当于把前面一个命令的 stdout 输出放到后面一个命令的 stdin 输入中.
\item \verb`命令 > <文件名.txt>` 可以把命令行的输出写入文件中 !! 例如: \verb`date > time.txt , cal > calander.txt`
\item \verb`grep <string> < <filename>` 可以在文档中高亮标出指定字符串, 用 \verb|-i| 不区分大小写。
\item \verb`cat <filename> | grep <string>` 可以达到同样的效果
\item \verb`grep -iRl "文字" [目录]` 可以搜索当前目录或者指定目录下所有子文件夹中文件的内容, 并列出文件名。 \verb`i` 选项不区分大小写, \verb`R` 搜索子文件夹, \verb`l` 仅列出文件名。 要列出每一个 match, 用 \verb`grep -iR` 即可
\item \verb|egrep| 是支持正则表达式\upref{regex}的 \verb|grep|, 例如 \verb`lsb_release -a 2>&1 | egrep -o -m 1 [0-9]{2}\.[0-9]{2}` 可以获取 ubuntu 系统的版本号(相当于 \verb|lsb_release -rs|)。 其中 \verb|-m 1| 只匹配第一个结果, \verb|-o| 只输出匹配部分。
\item 用 vim 的时候, 有时要在前面加上 \verb`sudo`, 否则有可能编辑了保存不了(例如一些 configuration file). 也可以要首先用 \verb`chown` 把文件所有权变成自己的.
\item \verb`#` 在文件中起标注的作用 (如果文件中有 linux 代码), 在命令行, \verb`#` 后面的代码同样会被忽略.
\item 要转换文档 encoding,用 \verb|iconv -f [encoding] -t [encoding] -o [newfilename] [filename]|。 微软的 ANSI 格式其实叫做 iso-8859-1, 例如 \verb`iconv -f iso-8859-1 -t utf-8 -o out.txt in.txt`.
\item \verb|tr [opt] SET1 [SET2]| 可以用来把 stdin 中 \verb|SET1| 中的字符分别替换为 \verb|SET2| 中的, 输出到 stdout。 另外 \verb|-s| 选项可以把替换后相邻重复的字符只保留一个。 例如 \verb`echo $PATH | tr : '\n'` 可以分行显示每个路径(\verb|'\n'| 也可以换成 \verb|\\n|)。

\subsubsection{sed}
\item \verb|sed|(stream editor) 命令通常用于批量自动修改文本文件, 而不是用交互的编辑器手动编辑。
\item \verb`sed` 命令可以用来处理文件或 \verb|stdin| 中的文字, 并输出到文件或者 \verb|stdout|. 例如
\verb|sed -e 's/^!\*nq/    /' inputfile.f90 > outputfile.f90|, 其中 \verb`-e` 是将后面的命令作用到 \verb`inputfile.f90` 上. \verb`s/str1/str2/` 是将文本中的 \verb`str1` 替换成 \verb`str2`, \verb`\*` 转义为 \verb`*` (直接用 \verb`*` 就是通配符), \verb`^` 大概是要求 \verb`str1` 出现在行首.
\item \verb|sed -i '...' 文件| 可以直接修改文件而不是输出到 \verb|stdout|。 例如 \verb|sed -i '8d' 文件| 把文件的第 8 行删除。 由于只有一个命令, \verb|-e| 可以省略。
\item 将子目录下所有头文件中的 CRLF 都替换为 LF: \verb`sudo find . -name "*.h" | xargs sudo sed -i -e 's/\r//g'`。 选项 \verb`-i` 直接修改源文件, \verb`-e` 用于指定后面的 \verb`s/旧内容/新内容/g`
\item 上一条中的 \verb|g| 代表 global, 意思是全部替换, 如果没有 \verb|g| 将只替换第一个。
\item \verb`xarg` 用于使用 piping 构建新命令, 即将 \verb|find| 输出到 \verb|stdout| 的东西放到 \verb`sudo sed -i -e 's/\r//g'` 的后面作为参数。
\end{itemize}

\subsection{账户与权限}
\begin{itemize}
\item \verb`sudo` (super user do) 加在命令前可以获得管理员权限, \verb`su` (super user) 可以把以下所有操作改为管理员权限 (不建议常用)。 如果 ubuntu 中不允许 \verb|su|, 那么就用 \verb|sudo -i| 也是一样的。
\item \verb`su 用户名` 可以临时改变登录的身份, \verb|sudo -i| 或者 \verb|su| 可以持续管理员身份。
\item 使用 \verb|sudo| 以后, 命令的执行者就会临时变为 \verb|root|, 一些 config 文件也会使用 \verb|/root/| 文件夹中的而不是 \verb|/home/用户名| 中的, 例如 \verb|/root/.ssh/config|。
\item \verb`sudo` 的密码是用户密码, 而 \verb`su` 的密码是 root 密码, 后者的权限更高
\item \verb`useradd [-m] [-g 用户组] [-U] [-s /bin/bash] 用户名`。 其中 \verb|-m| 给新用户创建 home 目录, \verb|-U| 选项创建和用户名相同的用户组并作为 primary group (不能和 \verb|-g| 选项同时使用), \verb|-g| 用于指定用户的 primary group, \verb|-s| 指定 shell。
\item \verb|adduser 用户名| 可以用互动的方式创建用户。
\item \verb`passwd 用户名` 可以修改密码
\item \verb`userdel 用户名` 删除账户
\item \verb`users` 命令显示目前所有登录的用户, \verb`who` 命令可以显示更详细的信息
\item \verb`/etc/passwd` 文件中记录了系统的所有 user, 格式为 \verb|用户:密码(x代表已加密):用户ID:组ID:用户的可选信息(姓名电话等):HOME文件夹:默认shell|。 通常 1000 以下的用户 ID 是系统用户。
\item \verb|/etc/group| 文件记录了所有用户组的信息, 格式为 \verb|组名:密码:组ID:组员|
\item 一个用户只能有一个 primary group, 就是 \verb|/etc/passwd| 中的。 也可以有多个 secondary group, 就是 \verb|/etc/group| 中的每行最后一个信息。
\item 用户创建的文件和文件夹默认的组就是 primary group。
\item \verb|chgrp| 可以改变文件的组
\item \verb|groupadd 组名| 新建用户组
\item \verb`groupdel 组名` 移除用户组
\item \verb`gpasswd -a 用户名 组名` 可以在组内新增用户
\item \verb`ls -l` 最左边第一个字符是 filetype, 第一列(之后三个字符)是 owner 的权限, 第二列是 group 的权限, 第三列是其他人的权限, \verb`r (read) w (write) x (execute)`。 filetype 可以是 \verb|–|: 常规文件, \verb|d| 目录, \verb|l| symbolic link, 还有一些其他不常见的,详见\href{https://linuxconfig.org/identifying-file-types-in-linux}{这里}。
\item \verb`ls` 选项: \verb`-a` (all)(显示隐藏) \verb`-m` (挤到一起) \verb`-1` 显示为一列 \verb`-v` 按照数字大小排序
\item \verb`chown 用户名 文件` 修改文件的所有者, \verb|-R| 对文件夹 recursive
\item \verb|ls -l| 中第二列是 “link 的数量”, 比较鸡肋, 普通文件和软链是 1, 文件夹是 2。
\item 你自己对某个文件/文件夹是什么权限, 你调用的程序也只有这些权限。
\item \verb`chmod 权限 文件` 修改文件的权限. <权限> 是三位8进制数, 每一位是 x(1),w(2),r(4)之和. 常见的例子: \verb|7| 代表 \verb|rwx|, \verb|6| 代表 \verb|rw|, \verb|4| 代表 \verb|r|
\item \verb`chmod 权限 -R 文件夹` 可以修改文件夹和所有内部子文件夹和子文件的权限。 千万注意不要为了方便把一些系统文件夹甚至 \verb|/home/用户| 文件夹的设置文件设得太开放, 这可能会影响一些秘钥的自动验证, 尤其是 \verb|ssh| 相关的。
\item 注意对于文件夹而言,\verb|x| 权限控制的是 \verb|cd| 到其中的权限, 而 \verb|r| 权限则控制是否可以看到其中有什么文件。 如果只有 \verb|x| 没有 \verb|r| 则代表你可以 \verb|cd| 进去, 但 \verb|ls| 用不了。 如果它的子文件夹也只有 \verb|x| 没有 \verb|r|, 那么你无法知道这个子文件夹存在, 但如果你知道它的名字就可以 \verb|cd| 进去。
\item 如果你对一个文件夹的 \verb|rwx| 都没有, 那么文件夹和其子文件夹内的所有内容你都无法看到也无法进行任何操作, 无论内部的文件和子文件夹有多开放。
\item 如果你对一个文件夹只有 \verb|rw| 没有 \verb|x|, 也只是好了那么一点点, 你可以对文件夹就地重命名(不能移动), 只能看到文件夹内第一层的文件和子文件夹的名字(看不到权限),其他同样什么也做不了。
\item 用户创建的文件的权限默认是 \verb|644|, 文件夹默认 \verb|755|(允许 \verb|x|,所有人可以进入文件夹)。
\item \verb|umask 002| 可以修改新建文件的默认权限为 \verb|664| (666-002=664), 且新建文件夹的默认权限被修改为 \verb|775| (777-002=775)。 如果设 \verb|umask 007|, 那么文件默认权限为 \verb|660|(6-7=0), 文件夹为 \verb|770|。
\item \verb`groups` 或者 \verb`groups <user>` 可以查看用户所在的组
\item \verb`id` 可以查看用户的信息 \verb`uid (user id) gid (group id)`
\item \verb`vim /etc/group` 可以查看记录 group 信息的文件
\item \verb`gpasswd -d <username> <groupname>` 可以在组中移除用户
\item \verb`hostname` 可以显示当前主机的名字. 也可以用环境变量 \verb|HOSTNAME|
\item \verb|sudo hostname 新名字| 可以改变 hostname
\item 其实真正和 windows 的任务管理器相对应的是 \verb`top`, 会显示实时进程. 
\item \verb`K <PID>` 会结束对应编号的进程. 系统的关键进程是杀不掉的.
\item \verb`last -数字` 可以查看最近若干次登录的记录
\item \verb|sudo visudo| 可以安全地编辑管理员权限文件 \verb|/etc/sudoers|(保存时会检查语法错误), 默认使用 nano 编辑器, 要用 vim
 可以 \verb|sudo update-alternatives --config editor| 然后选 vim。
\item 例如在 \verb|visudo| 设置里面 \verb|Defaults  env_reset| 后面加上 \verb|,timestamp_timeout=3600|, 那么可以设置输入密码以后一个小时内不需要重复输入(默认只有几分钟)
\end{itemize}

\subsection{进程}
\begin{itemize}
\item \verb`ps aux` 会显示当前所有\textbf{进程 (process)}, \textbf{PID} 是 process ID, 用于终止进程
\item \verb`ps aux | grep 搜索词` 会搜索当前进程中包含 \verb`搜索词` 的内容
\item 一个很占 cpu 的命令可以用于测试 \verb`dd if=/dev/random of=/dev/null &` 其中 \verb`&` 表示在后台运行,按回车后会输出 PID (进程号)。 连续输入这个进程 4 次, 然后用 \verb`top` 命令打开进程管理, 可以看到它们出现在顶部。
\item top 中的 \verb`NI` 就是 nice value
\item 在 top 界面中用 \verb`r` (renice) 可以调整进程的 nice value (优先级,越小越优先)按 ESC 可以退出
\item \verb|ionice| 可以调整硬盘读写的优先级: \verb|ionice -p <PID>| 查看某个进程的优先级(0 到 7, 从高到低), \verb|ionice -p <PID> -n 优先级| 可以改变优先级。
\item 在 top 界面中用 \verb`k` 可以杀进程,输入进程的 PID,两次回车即可
\item \verb`pstree` 可以看到哪些进程生成了哪些子进程, 例如 shell 里面运行 .sh 文件会生成一个子进程用于运行一个新的 shell
\item \verb`kill -STOP 进程号` 可以暂停程序
\item \verb`kill -CONT` 可以继续暂停的程序
\item 如果一个进程 kill 不掉, 可以试试 \verb|kill -9|
\end{itemize}

\subsection{文件打包压缩}
\subsubsection{tar}
\begin{itemize}
\item \verb`tar -czvf <path/name.tar.gz> <some/path/or/file>` 是常见的压缩命令. 其中 \verb`-c` 是 create, \verb`-v` 是 verbose, \verb`-f` 指定目标文件(一定要紧接文件名), \verb`-z` 使用 gzip 压缩. 如果不需要压缩就用 \verb`-cvf <path/name.tar>`.
\item 要解压, windows 中可以用 好压或 Matlab, linux 中把以上命令中的 \verb`-c` 换成 \verb`-x` (extract)即可, 如 \verb`tar -xzvf 文件.tar.gz -C 目标文件夹` 如果省略 \verb`-C` 选项就解压到当前文件夹.
\item 在使用 \verb|tar| 前, 可以用环境变量 \verb|GZIP=-9| 来指定压缩级别(1 到 9)。
\item 如果要直接压缩文件可以用 \verb|gzip [-9] < 文件 > 文件.gz|, 其中 \verb|-9| 可以增加压缩率。
\end{itemize}

\subsubsection{zip}
\begin{itemize}
\item 要压缩为 zip 文件, 用 \verb|zip -r 文件名.zip files|
\item \verb|unzip archive.zip| 解压
\item 要添加密码, 用 \verb|zip -r -P 密码 文件名.zip files|
\item \verb|-0| 到 \verb|-9| 可以设置压缩等级, 9 得到的文件最小。
\item \verb|zip -r 文件名.zip files -x "*.log"| 可以排除所有 \verb|.log| 后缀名的文件
\end{itemize}

\subsubsection{7zip}
\begin{itemize}
\item 在最后用 \verb`-o/path/to/extract/` 指定输出路径
\item 如果要解压分卷压缩, 直接 \verb`7z x 第一个文件名` 即可
\item ubuntu 上安装用 \verb`sudo apt-get install p7zip-full`

\item 将每个子文件夹中的文件分别压缩,删除原文件
\verb`find . -type f -exec 7z a {}.7z -p密码 -sdel {} \;`
\item 将每个子文件夹中的压缩包分别解压,删除压缩包
\verb`find . -type f -name "*.7z" -execdir 7z x {} -p密码 \; -exec rm {} \;`
\item 注意即使用完全一样的命令, 产生的压缩包也不是完全一样的(看 sha1sum)
\end{itemize}

\subsection{文件加密}
\begin{itemize}
\item \verb|openssl| 可以用来给文件或字符串加密或解密
\item 文件加密: \verb`openssl aes-256-cbc -nosalt -in 文件名 -out 加密文件名 -pass pass:密码`
\item 文件解密: \verb`openssl aes-256-cbc -nosalt -d -in 加密文件名 -out 文件名 -pass pass:密码`
\item 字符串加密: \verb`echo '需要加密的字符串' | openssl enc -base64 -e -aes-256-cbc -nosalt -pass pass:密码`
\item 字符串解密: \verb`echo "需要解密的字符串" | openssl enc -base64 -d -aes-256-cbc -nosalt -pass pass:密码`
\item salt 在加密过程中加入了随机性, 使加密更安全。 如果没有 \verb|-nosalt|, 文件加密得到的结果的 hash 是不稳定的。
\item 这样也可以给文件加密得到稳定 hash: \verb`cat 文件名 | openssl enc -base64 -e -aes-256-cbc -pass pass:密码 > 加密文件名`
\item \verb|openssl| 同样也可以生成密钥对, 然后用密钥给文件加密, 具体问 GPT 吧。
\end{itemize}

\subsection{网络}
\subsubsection{sftp}
\begin{itemize}
\item SFTP 其实是 ssh 的一个子功能, 通过 ssh 协议传输文件, 只要能 ssh 到某个机器, 就可以 sftp
\item 连接方法为 \verb`sftp [usr@][url]` 就和 \verb`ssh [usr@][url]` 一样。 成功以后会出现提示符 \verb`sftp>`, 可以进行下一步操作
\item \verb`sftp>` 环境下同样可以用 \verb`cd`, \verb`ls`, \verb`pwd` 等命令
\item \verb`get /path/to/file` 可以传输某个文件到本地(进入 sftp 时的目录)
\item \verb`get /remote/file /local/file` 可以传输某个文件到本地指定的目录和文件名, 但该目录必须存在
\item \verb`get -r 文件夹` 可以传送整个文件夹
\item 如果要指定端口用 \verb`sftp -oPort=xxx xxx@xxx.xxx.xxx.xxx`
\end{itemize}

\subsubsection{curl}
\begin{itemize}
\item \verb|curl| 是 client URL 的意思, 用于通过 url 从服务器接受或发送信息。
\item \verb|curl "网址" -so 文件| 可以把某个 url 返回的字符串(如 html 和 json)保存到文件中, 其中 \verb|-s| 是 silent, \verb|-o| 是 output file。 这大概相当于在浏览器中输入 \verb|网址| 然后复制 html 源码等。
\item \verb|curl --header "<HeaderName>: 123" www.example.com|。 可以发送一个 header 的 url, 也可以用 \verb|-H|
\item \verb|curl -u 用户名:密码 www.example.com| 使用 Basic Auth 进行用户登录
\end{itemize}

\subsubsection{服务器}
\begin{itemize}
\item \verb`sudo tasksel` 快速安装某一类型的服务器需要所有的软件 (email server , file server, ...). 进入界面以后, 用空格选择想要的项目然按空格确认安装.
\item \verb`sudo /etc/init.d/<aptname> start` 或者 \verb`stop` 或者 \verb`restart` 可以进行对某个软件的 service 进行操作, 而不需要重启 linux 系统.
\end{itemize}

\subsection{环境变量}
\begin{itemize}
\item \verb|echo $VAR| 显示环境变量或任何变量。 也可以用 \verb|printenv VAR|
\item \verb|env| 命令可以显示所有环境变量, 也可以用 \verb|printenv|。
\item \verb|env| 可以先对某个程序修改环境变量, 再运行该程序
\item \verb|export VAR=XXX| 修改环境变量(对子进程也有效)
\item \verb|VAR=XXX| 修改环境变量(对子进程无效)
\item \verb|export VAR=~/opt/bin:$VAR| 可以 append 某个变量(如 PATH)
\item \verb|PWD| 是当前路径, 修改该变量相当于 \verb|cd|
\item \verb|PATH| 是命令搜索路径,多个路径之间用冒号分隔, 只有这里面的目录中的执行文件才不指定路径。 常见路径包括 \verb`/bin`, \verb`/sbin`, \verb`/usr/bin`,\verb`/usr/sbin`, 其中带 \verb`s` 的目录中的文件通常不是给普通用户调用的, 根目录的文件夹中的程序通常是用于启动或修复系统必须的, 所以普通用户用得最多的是 \verb`/usr/bin`
\item 如果想把 \verb|PATH| 每个路径显示成一行, 用 \verb`echo $PATH | tr : \\n`。 \verb|tr| 命令把 \verb|:| 替换为换行符。
\item \verb|HOME| 是用户的 \verb|~| 文件夹
\item \verb`USER` 是当前用户名
\item \verb`LANG` 是当前 local, 推荐使用 \verb`en_US.UTF-8`
\item \verb`BASH` 是当前 bash 程序所在目录
\item \verb|PWD| 是当前目录
\item \verb|LD_LIBRARY_PATH| 是动态链接库的默认搜索地址, 见这里\upref{gppLib}。
\item \verb|LD_RUN_PATH| 编译的链接阶段, 如果没有指定 \verb|rpath|, 就用该变量中的 path 替代\upref{gppLib}。
\item 另外 \verb|g++| 中的 \verb|-L| 选项(make 的时候搜索静态或动态库的路径)可以用环境变量 \verb|LIBRARY_PATH| 设置, \verb|-I| 选项(搜索头文件的路径)可以用环境变量 \verb|CPATH| 设置。
\item 在 cpp 程序中获取环境变量可以用 \verb`<cstdlib>` 中的 \verb`char* getenv(const char* env_var);`
\end{itemize}

\subsection{其他}
\begin{itemize}
\item \verb`alias` 可以给一个长命令设置别名, 例如 \verb|alias abcde='echo hello'|, 仅限于当前 terminal, 可以添加到 \verb|~/.bashrc| 中每次 login 都定义。
\end{itemize}

\begin{figure}[ht]
\centering
\includegraphics[width=14cm]{./figures/20bca4cd283fbfef.png}
\caption{参考 \href{https://www.howtogeek.com/}{How to geek}} \label{fig_LinNt_1}
\end{figure}

\subsection{硬件}
\begin{itemize}
\item \verb|lscpu| 查看 CPU 信息, \verb|free| 查看内存使用情况, \verb|ifconfig| 和 \verb|ip| 查看网卡。
\end{itemize}

\subsection{多媒体}
\begin{itemize}
\item HEIC 文件转换为 \verb|jpg| 或 \verb|png|: 安装 \verb|sudo apt install libheif-examples|。 然后 \verb|find . -iname "*.heic" -exec heif-convert -q 100 {} {}.jpg或png \;|
\end{itemize}
