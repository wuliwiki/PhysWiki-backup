% 戴森级数
% keys 戴森级数|时间演化算符
% license Xiao
% type Tutor

\pentry{时间演化算符(量子力学)\upref{TOprt}}

\subsection{戴森级数}

考虑时间演化算符所满足的微分方程式:

\begin{equation}
i\hbar \frac{d}{dt}\hat U (t,t_0) = \hat H(t) \hat U (t,t_0)~.
\end{equation}

将其转化为积分方程,且考虑$\lim\limits_{t\rightarrow t_0}\hat U(t,t_0)=\hat I$,则有:

\begin{equation}
\hat U (t,t_0) =\hat I - \frac{i}{\hbar}\int^t_{t_0} \hat H(t_1) \hat U (t_1,t_0)dt_1~.
\end{equation}

将$\hat{U}$不断迭代,则有:

\begin{align}
\hat U (t,t_0) &=\hat I - \frac{i}{\hbar}\int^t_{t_0} \hat H(t_1) \hat U (t_1,t_0)dt_1 \\
&=\hat I - \frac{i}{\hbar}\int^t_{t_0} \hat H(t_1) dt_1 - (\frac{i}{\hbar})^2\int^t_{t_0}\int^{t_1}_{t_0} \hat H(t_1)H(t_2) \hat U (t_1,t_0)dt_2 dt_1 \\
&= \cdots \\
&=\hat I +\sum\limits^{\infty}\limits_{n=1}(-\frac{i}{\hbar})^n\int^{t}_{t_0}\int^{t_1}_{t_0}\cdots\int^{t_{n-1}}_{t_0}H(t_1)H(t_2)\cdots H(t_n)dt_n\cdots dt_2dt_1 ~,
\end{align}

上式即被称为戴森级数。

\subsection{时序算符}

考虑算符$\hat T$,其作用为:
$$\hat T\hat H(t_1)\hat H(t_2)=\leftgroup{  
\hat H(t_1)\hat H(t_2) , t_1>t_2&  \\  
\hat H(t_2)\hat H(t_1) , t_1 \leq t_2}~,
$$

显然有$\hat TH(t_1)H(t_2)=\hat TH(t_2)H(t_1)$。





则:

\begin{align}
\int^t_{t_0}\int^{t}_{t_0}\hat T\hat H(t_1)\hat H(t_2)dt_1dt_2 &= 
\int^t_{t_0}\int^{t_2}_{t_0}\hat T\hat H(t_1)\hat H(t_2)dt_1dt_2+\int^t_{t_0}\int^{t_1}_{t_0}\hat T\hat H(t_1)\hat H(t_2)dt_2dt_1 \\ 
&= 2\int^t_{t_0}\int^{t_1}_{t_0}\hat H(t_1)\hat H(t_2)dt_2dt_1~.
\end{align}

同时:

\begin{align}
\int^t_{t_0}\int^{t}_{t_0}\hat T\hat H(t_1)\hat H(t_2)dt_1dt_2 &= \hat T\int^t_{t_0}\int^{t}_{t_0}\hat H(t_1)\hat H(t_2)dt_1dt_2 \\&= \hat T\int^{t}_{t_0}\hat H(t_1)dt_1\int^t_{t_0}\hat H(t_2)dt_2\\&=\hat T\left[\int^t_{t_0}\hat H(t_0)dt_0\right]^2 ~,
\end{align}

此结论不仅局限为两个哈密顿量相乘的项$\hat H$的作用始终为将哈密顿量按其时间综量大小排序。

由此可以将戴森级数写成:


\begin{align}
\hat U (t,t_0) &=\hat I +\sum\limits^{\infty}\limits_{n=1}(-\frac{i}{\hbar})^n\int^{t}_{t_0}\int^{t_1}_{t_0}\cdots\int^{t_{n-1}}_{t_0}H(t_1)H(t_2)\cdots H(t_n)dt_n\cdots dt_2dt_1 \\
&=\hat I +\hat T \sum\limits^{\infty}\limits_{n=1}\frac{1}{n!}(-\frac{i}{\hbar})^n \left[\int^t_{t_0}\hat H(t_0)dt_0\right]^n \\
&=\hat T \exp(-\frac{i}{\hbar}\int^t_{t_0}\hat H(t_0)dt_0)~.
\end{align}
