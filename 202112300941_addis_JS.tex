% JavaScript 入门笔记

\pentry{HTML 基础}
JavaScript 常见于网页中, 一般浏览器都可以运行. 我们以 Chrome 浏览器为例演示 \verb|Hello World| 程序.

\begin{lstlisting}[language=js]
<!DOCTYPE html>
<html>
	<body>
		<div id = a></div>
		<script>
            console.log("hello world 1");
            window.alert("hello world 2");
			document.getElementById("a").innerHTML = "hello world 3";
            document.write("hello world 4");
		</script>
	</body>
</html>
\end{lstlisting}
把这段程序复制到一个文本文件, 并命名为 \verb|test.html|, 用 Chrome 打开即可自动运行. 按 F12 可打开调试窗口和命令行.

\verb|<script>...</script>| 中的 4 行程序就是 JavaScript, 它们这里分别演示了用 4 种不同的方法显示 “hello world”:
\begin{enumerate}
\item 输出到命令行(需要 F12 才能看到)
\item 弹出提示窗
\item 插入到 \verb|<div>...</div>| 元素中
\item 插入到 \verb|<script>| 之前
\end{enumerate}

\subsection{常识}
\begin{itemize}
\item \verb|<body>| 中任意位置可以包含任意多个 \verb|<script>|
\item 用 \verb|<script src="路径/文件名.js 或者 url.js"></script>| 插入代码文件
\item 在第一行插入 \verb|'use strict'| 后使用严格语法. 例如不声明的变量会出错.
\item 命令后的 \verb|;| 不是必须的, 可用于在同一行中分隔两个命令.
\end{itemize}

\subsection{变量}
\begin{itemize}
\item 声明变量 \verb|let a = 1, b = 2| 或者用 \verb|const| 和 \verb|var|
\item \verb|var| 是全局变量, 且可以重新声明, \verb|let| 和 \verb|const| 是局部的
\item \verb|typeof| 获取变量类型
\item 自带类型有 \verb|number| (不区分整数和浮点数,浮点是双精度), \verb|string| (单引号双引号通用, 没有 char)
\item 所有自定义的类型都叫做 \verb|object|
\item 合并字符串用 \verb|字符串 + 字符串|, 也可以 \verb|字符串 + 数字|
\end{itemize}

\subsection{math.js}
\subsubsection{常用}
\begin{itemize}
\item 常数 \verb|pi|, \verb|e|, 
\item 常用函数 \verb|round|, \verb|exp|, \verb|sin|, \verb|cos|, \verb|sqrt|, \verb|log|, \verb|atan2|, \verb|pow|
\end{itemize}

\subsubsection{复数}
\begin{itemize}
\item 在代码前面插入 \verb|<script src="math.js"></script>|, 也可以是 url 如 \verb|https://cdnjs.cloudflare.com/ajax/libs/mathjs/10.0.2/math.js|
\item \verb|let c = math.complex(2, 3)| 生成复数
\item \verb|c.re| 和 \verb|c.im| 获取实部和虚部, 可以赋值 \verb|c.re = 5|. 也可以用 \verb|math.re(c)| 和 \verb|math.im(c)|, 不能赋值.
\item 共轭 \verb|c.conjugate()| 或者 \verb|math.conj(c)|, 绝对值 \verb|math.abs(c)| 或 \verb|c.abs()|, 幅角 \verb|math.arg(c)| 或者 \verb|c.arg()|, 加法 \verb|math.add(a, b)|, 乘法 \verb|math.multiply(a, b)|, 除法 \verb|math.divide(a, b)|, 根号 \verb|math.sqrt(-4)|, 倒数 \verb|a.inverse()|.
\end{itemize}


\begin{lstlisting}
// functions and constants
math.atan2(3, -3) / math.pi          // 0.75
math.derivative('x^2 + x', 'x')      // 2*x+1
math.pow([[-1, 2], [3, 1]], 2)
     // [[7, 0], [0, 7]]

// expressions
math.evaluate('1.2 * (2 + 4.5)')     // 7.8
math.evaluate('12.7 cm to inch')     // 5 inch
math.evaluate('sin(45 deg) ^ 2')     // 0.5
math.evaluate('9 / 3 + 2i')          // 3 + 2i
math.evaluate('det([-1, 2; 3, 1])')  // -7

// chaining
math.chain(3)
    .add(4)
    .multiply(2)
    .done() // 14
\end{lstlisting}