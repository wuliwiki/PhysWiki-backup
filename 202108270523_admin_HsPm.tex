% 排列
% 高中|排列

\subsection{定义}
一般地,从 $n$ 个不同元素中任取 $m$ ($m \leq n$) 个元素,按照一定顺序排列成一列,称为从 $n$ 个不同元素中取出 $m$ 个元素的排列数,用符号 $A_n^m$ 表示.

根据分布乘法计数原理可得排列数公式\begin{equation}\label{HsPm_eq1}
A_n^m = n (n - 1)(n - 2) \cdots (n - m + 1)
\end{equation}

\subsection{变形}
我们对 \autoref{HsPm_eq1} 进行变形,\begin{equation}\label{HsPm_eq2}
\begin{aligned}
A_n^m &= \frac{n(n - 1)(n - 2) \cdots 2 \cdot 1}{(n - m)(n - m - 1) \cdots 2 \cdot 1}\\
&= \frac{n!}{(n - m)!}
\end{aligned}
\end{equation}

\autoref{HsPm_eq2} 为排列数的另一种表达形式