% 范德蒙矩阵、范德蒙行列式
% license Xiao
% type Tutor

\begin{issues}
\issueDraft
\end{issues}

\pentry{秩\upref{MatRnk}}

\footnote{参考 Wikipedia \href{https://en.wikipedia.org/wiki/Vandermonde_matrix}{相关页面}。}\textbf{范德蒙矩阵(Vandermonde matrix)}是一种特殊的行列式和多项式相关。

\begin{definition}{}
范德蒙矩阵是一个 $n\times m$ 的矩阵\upref{Mat}, 定义为
\begin{equation}\label{eq_VandDe_1}
\mat V = 
\pmat{1 & x_1 & x_1^2 & \dots & x_1^{m-1}\\
1 & x_2 & x_2^2 & \dots & x_2^{m-1}\\
1 & x_3 & x_3^2 & \dots & x_3^{m-1}\\
\vdots & \vdots & \vdots & \ddots & \vdots\\
1 & x_n & x_n^2 & \dots & x_n^{m-1}}~.
\end{equation}
若 $\mat{\mat V}$ 是方阵($m = n$), 其行列式\upref{Deter}称为\textbf{范德蒙行列式(Vandermonde determinant)}。

一些文献中也把\autoref{eq_VandDe_1} 中的各列左右翻转, 即按照幂从大到小排列。
\end{definition}

可应用于多项式最小二乘法拟合(\autoref{sub_LstSqr_1}~\upref{LstSqr}) 以及多项式插值。


\subsection{性质}
当 $m \le n$ 时, 矩阵的秩\upref{MatRnk}为 $m$ 当且仅当所有的 $x_i$ 各不相等。

当 $m \ge n$ 时, 矩阵的秩为 $n$ 当且仅当至少 $n$ 个 $x_i$ 各不相等。
\subsubsection{证明}
先证明 $m = n$ 时,范德蒙矩阵满秩,即秩不为0。

求方阵 $\mat V $的行列式:
\begin{equation}
\vmat {\mat V} = \vmat{1 & x_1 & x_1^2 & \dots & x_1^{m-1}\\
1 & x_2 & x_2^2 & \dots & x_2^{m-1}\\
1 & x_3 & x_3^2 & \dots & x_3^{m-1}\\
\vdots & \vdots & \vdots & \ddots & \vdots\\
1 & x_n & x_n^2 & \dots & x_n^{m-1}}~.  
\end{equation}
根据行列式计算法则,

