% 南京航空航天大学 2004 量子真题
% license Usr
% type Note

\textbf{声明}:“该内容来源于网络公开资料,不保证真实性,如有侵权请联系管理员”

\subsection{简答题(每小题12分,共60分)}
1. 已知一维运动粒子波函数为
$$ \Psi(x, t) = e^{i(px - Et)/\hbar} + e^{-i(px + Et)/\hbar}~$$ 
其中 $p$ 为动量,$E$ 为能量。说明:
\begin{enumerate}
    \item 它是不是动量算符 $\hat{p}$ 的本征函数?
    \item 它是不是动量平方算符 $\hat{p}^2$ 的本征函数?~
\end{enumerate}

2. 对一维运动粒子,求算符 $\hat{p} + x$ 的本征函数和本征值。其中 $\hat{p}$ 为动量算符,$x$ 为位置坐标算符。

3. 已知氢原子处在
$$ \psi(r, \theta, \phi) = \frac{1}{\sqrt{\pi a_0^3}} e^{-r/a_0}~$$ 
状态,其中 $a_0$ 为第一玻尔半径。计算其势能 $V(r) = -\frac{1}{4\pi \epsilon_0} \frac{e^2}{r}$ 的平均值。