% Matlab 符号计算和任意精度计算(笔记)

Matlab 的符号计算需要符号计算工具箱, 取决于你的证书类型, 可能需要额外购买. 

\begin{itemize}
\item Matlab 中用于储存符号计算表达式的变量类型为 \verb|sym|. 可以用 \verb|syms 变量1 变量2 ...| 声明变量类型为 \verb|sym|. 例如 \verb|syms x y z;|. Matlab 的大部分自带算符和函数支持 \verb|sym| 类型的变量, 例如 \verb|x^2| 就是 \verb|sym| 类型的表达式 $x^2$, $x$ 并不是一个数值而是符号. 若此时令 \verb|expr = x^2|, 那么用 \verb|class(expr)| 可以验证 \verb|expr| 的类型也是 \verb|sym|.
\item 除了使用 \verb|syms| 一次声明几个符号变量, 也可以使用 \verb|sym(字符串)|. 例如 \verb|syms x; expr = x^2;| 得到的 \verb|expr| 和 \verb|expr = sym('x')^2;| 得到的 \verb|expr| 是等效的.
\item  对表达式求导如 \verb|diff(sym('x')^2)|. 若要求偏导, \verb|diff(sym('x')^2 * sym('y')^3)| 默认对 \verb|x| 求偏导, 结果是 $2x y^3$. 也可以声明对 \verb|y| 求偏导, 如 \verb|diff(sym('x')^2 * sym('y')^3), sym('y')|, 或者更简洁地, \verb|diff(sym('x')^2 * sym('y')^3), 'y'|. 建议总是声明对哪个变量求偏导.
\item  要把一个数字作为符号, 可以使用例如 \verb|sym('2')| 或 \verb|sym('7/3')|, 但这仅限于分式, 不允许注入 \verb|sym('sqrt(2)')| 这样的用法.
\item 另一种方法是使用 \verb|sym(数值)|. 例如 \verb|sqrt(sym(2))| 的结果是表达式 $\sqrt 2$, 而不同于数值计算中的 \verb|sqrt(2) = 1.414...|. 由于符号计算是精确无误差的, 无理数 $\sqrt{2}$ 并不会被自动转换为小数形式. 另一个例子, \verb|d = 3.1; sqrt(sym(d))| 得到的是表达式 $961/100$.
\end{itemize}
