% 两体问题(用分析力学方法求解)
% keys 两体问题|开普勒问题|卢瑟福散射

\pentry{拉格朗日方程\upref{Lagrng}}

两体问题研究的对象是两个可以看成质点的物体,质量分别为 $m_1,m_2$,位矢分别为 $\bvec r_1,\bvec r_2$.它们之间的相互作用势是 $V(r)=V(|\bvec r_1-\bvec r_2|)$,也就是说只和两者的距离有关.开普勒问题\upref{CelBd}、卢瑟福散射\upref{RuthSc} 都属于两体问题.

在这个词条中我们将用分析力学的方法来解决两体问题.设 $\bvec r_c$ 为质心位置,设 $\bvec r$ 为它们的相对位置,那么有
\begin{equation}
\left\{
\begin{aligned}
&\bvec r_c=\frac{m_1\bvec r_1+ m_2\bvec r_2}{m_1+m_2}\\
&\bvec r=\bvec r_2-\bvec r_1
\end{aligned}
\right.
\end{equation}
那么体系的动能为质心动能加上系统相对于质心参考系的动能.
\begin{equation}
\begin{aligned}
T&=\frac{1}{2}m_1 \dot{\bvec r_1}^2+\frac{1}{2}m_2 \dot{\bvec r_2}^2\\
&=\frac{1}{2}(m_1+m_2)\dot{\bvec r_c}^2+\frac{1}{2}\frac{m_1m_2}{m_1+m_2}\dot{\bvec r}^2\\
&=\frac{1}{2}M \dot{\bvec{\bvec r_c}}^2+\frac{1}{2}\mu \dot{\bvec{\bvec r}}^2
\end{aligned}
\end{equation}
其中 $M$ 为体系的总质量,$\mu$ 称作约化质量:
\begin{equation}
\left\{
\begin{aligned}
&M=m_1+m_2\\
&\mu=\frac{m_1m_2}{m_1+m_2}
\end{aligned}
\right.
\end{equation}
在无外界影响的情况下,$\dot{\bvec r_c}=0$(这是因为系统的动量守恒).系统的势能为 $V(r)$.因此拉格朗日量为 $L=T(\dot{\bvec r})-V(r)$.两个物体一定在同一个平面内作运动,设 $\bvec r$ 在该平面内转过的角度为 $\phi$,设 $|\bvec r|=\rho$.我们取广义坐标 $\rho,\phi$.那么 $T=\frac{1}{2}\mu \dot{\bvec r}^2=\frac{1}{2}\mu \dot \rho^2+\frac{1}{2}\mu \rho^2\dot\phi^2$.体系的拉格朗日量和哈密顿量为
\begin{equation}
\begin{aligned}
&L=\frac{1}{2}\mu \dot \rho^2+\frac{1}{2}\mu \rho^2\dot\phi^2-V(\rho)\\
&H=T+V=\frac{1}{2}\mu \dot \rho^2+\frac{1}{2}\mu \rho^2\dot\phi^2+V(\rho)
\end{aligned}
\end{equation}
因此可以列出拉格朗日方程:
\begin{equation}\label{twoobj_eq1}
\begin{aligned}
&\frac{\dd }{\dd t}\frac{\partial L}{\partial \dot\rho}=\frac{\partial L}{\partial \rho}\\&\Rightarrow \mu\ddot\rho-\mu\rho\dot\phi^2+\frac{\dd V}{\dd \rho}=0
\end{aligned}
\end{equation}

\begin{equation}\label{twoobj_eq2}
\begin{aligned}
&\frac{\dd }{\dd t}\frac{\partial L}{\partial \dot\phi}=\frac{\partial L}{\partial \phi}\\&
\Rightarrow \frac{\dd p_\phi}{\dd t}=\frac{\dd }{\dd t}(\mu \rho^2\dot\phi)=0
\\&
\Rightarrow p_\phi=\frac{\partial L}{\partial \dot \phi}=\mu\rho^2\dot\phi ={\rm const}=J 
\end{aligned}
\end{equation}
\autoref{twoobj_eq1} \autoref{twoobj_eq2} 联立就可以得到两体问题的运动方程(两个方程和两个初始条件,就可以求解两个广义坐标的变化).或者我们也可以简化方程组,将其中一个替换为能量守恒方程,联立得:
\begin{equation}
\begin{aligned}
&\left\{
\begin{aligned}
&\frac{1}{2}\mu \qty(\dot\rho^2+\rho^2\dot\phi^2)+V(\rho)=E\\
&\mu\rho^2\dot\phi=J
\end{aligned}
\right.\\
&\Rightarrow \frac{1}{2}\mu \dot\rho^2=E-V(\rho)-\frac{J^2}{2\mu\rho^2}=E-V_{\rm{eff}}(\rho)
\end{aligned}
\end{equation}
$V_{\rm{eff}}=V(\rho)+\frac{J^2}{2\mu\rho^2}$ 为有效势能,其中 $\frac{J^2}{2\mu\rho^2}$ 实际上就是离心势.