% 热力学(综述)
% license CCBYSA3
% type Wiki

本文根据 CC-BY-SA 协议转载翻译自维基百科\href{https://en.wikipedia.org/wiki/Thermodynamics}{相关文章}。

\subsection{热力学}
热力学研究热量、功和温度,以及它们与能量、熵以及物质和辐射的物理性质之间的关系。这些量的行为由热力学四大定律所支配,这些定律通过可测量的宏观物理量进行定量描述,但可以通过统计力学从微观成分的角度加以解释。热力学在科学和工程的广泛领域中起着重要作用。

从历史上看,热力学起源于提高早期蒸汽机效率的需求,尤其是在法国物理学家萨迪·卡诺(1824年)的工作下,他认为发动机的效率是帮助法国赢得拿破仑战争的关键。[1] 苏格兰-爱尔兰物理学家开尔文勋爵是第一个在1854年给出简明定义的人,[2] 他说:“热力学是研究热量与作用于物体接触部分之间的力以及热量与电作用之间关系的学科。” 德国物理学家和数学家鲁道夫·克劳修斯重新阐述了卡诺循环,并为热理论提供了更真实、更可靠的基础。他于1850年发表的论文《论热的动因》[3] 首次提出了热力学第二定律。1865年,他引入了熵的概念。1870年,他引入了适用于热的维里定理。[4]

热力学的最初应用于机械热机,随后迅速扩展到化合物和化学反应的研究。化学热力学研究熵在化学反应过程中的作用本质,为该领域的扩展和知识提供了大部分内容。其他形式的热力学也随之出现。统计热力学或统计力学关注从微观行为出发对粒子集体运动的统计预测。1909年,康斯坦丁·卡拉西奥多里提出了一种纯数学方法的公理化表述,这种描述通常被称为几何热力学。