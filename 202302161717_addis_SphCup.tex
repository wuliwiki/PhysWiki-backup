% Clebsch–Gordan 系数
% 线性代数|角动量加法|基底|归一化|本征值|本征基底|CG系数|选择定则|对称性|正交归一性|球谐函数

\begin{issues}
\issueAbstract
\end{issues}

\pentry{角动量加法\upref{AMAdd}}

\subsection{相位约定}
任何算符的一组归一化本征基底各自乘以一个任意相位因子 $\E^{\I \phi_I}$ 仍然是一组归一化的本征基底(本征值不变)。 所以基底变换矩阵(如 CG 矩阵), 每一行或每一列分别乘以一个相位因子, 仍然表示相同的基底变换。 这些相位怎么取被称为\textbf{相位约定(phase convention)}。 一般的约定\footnote{如 Wikipedia,Griffiths Quandum 的定义}是使\textbf{所有 CG 系数为实数}, 且第一行和第一列(对应最大的 $m_1$ 和最大的 $L$)的矩阵元大于零。

\subsection{解析表达式为}
按照这种相位约定, CG 系数的解析表达式为
\begin{equation}\label{SphCup_eq2}
\ali{
&\quad \braket{l_1, m_1, l_2, m_2}{l_1, l_2, L, M} = \bmat{l_1 & l_2 & L\\ m_1 & m_2 & M}\\
&= \sqrt{\frac{(2L+1)(L+l_1-l_2)!(L-l_1+l_2)!(l_1+l_2-L)!}{(l_1+l_2+L+1)!}}\\
&\times\sqrt{(L+M)!(L-M)!(l_1-m_1)!(l_1+m_1)!(l_2-m_2)!(l_2+m_2)!}\\
&\times\sum_{k = k_{min}}^{k_{max}} \frac{(-1)^k}{k!(l_1+l_2-L-k)!(l_1-m_1-k)!(l_2+m_2-k)!}\\
&\times \frac{1}{(L-l_2+m_1+k)!(L-l_1-m_2+k)!}
}\end{equation}
其中求和的上下限需要保证每个含有 $k$ 的阶乘自变量都大于等于 0, 所以有
\begin{equation}
\ali{
&k_{min} = \max\{0, \ \ l_2 - m_1 - L,\ \  l_1 + m_2 - L\} \\
&k_{max} = \min\{l_1+l_2-L,\ \  l_1 - m_1, \ \ l_2 + m_2\}
}\end{equation}

\subsection{基本选择定则}
从“角动量加法\upref{AMAdd}” 中可以得到
\begin{equation}
\abs{l_1 - l_2} \leqslant L \leqslant l_1 + l_2 \qquad
\end{equation}
从物理上理解, 两个矢量相加得到第三个矢量 $\bvec v_1 + \bvec v_2 = \bvec v$, 则他们的模长 $v_1, v_2, v$ 必须满足三角不等式 $\abs{v_1 - v_2} \leqslant v \leqslant v_1 + v_2$。

我们还可以得到
\begin{equation}\label{SphCup_eq4}
m_1 + m_2 = M
\end{equation}
\begin{equation}\label{SphCup_eq5}
-L \leqslant M \leqslant L \qquad
-l_1 \leqslant m_1 \leqslant l_1 \qquad
-l_2 \leqslant m_2 \leqslant l_2
\end{equation}
当有可能出现半整数时, 还要求\footnote{推导过程中注意所有数都可能取半整数。 若两个数都是半整数或都是整数, 它们相加等于整数, 否则相加等于半整数。 半整数乘以 2 等于奇数, 整数乘以 2 等于偶数。 类比: 若两个数都是奇数或偶数, 它们相加等于偶数, 否则相加等于奇数。}
\begin{equation}
M + L \in \mathbb{N} \qquad
m_1 + l_1 \in \mathbb{N} \qquad
m_2 + l_2 \in \mathbb{N}
\end{equation}
当且仅当一个 CG 系数满足以上选择定则时, 它才会在 CG 表中出现(可能是 0)。

\subsection{对称性}
由于 3j 符号\upref{ThreeJ}具有更简洁的对称性, 我们可以用其推导 CG 系数的对称性\footnote{下式中如果 $L$ 是整数, 那么前面取正负号都行,但如果是半整数, 则只能取负号, 推导时要注意。}
\begin{equation}\label{SphCup_eq7}
\ali{
\bmat{l_1 &l_2 &L\\ m_1 &m_2 &M}
&= (-1)^{l_1+l_2-L}\bmat{l_1 &l_2 &L\\ -m_1 &-m_2 &-M}\\
&= (-1)^{l_1+l_2-L}\bmat{l_2 &l_1 &L\\ m_2 &m_1 &M}\\
&= (-1)^{l_1-m_1}\sqrt{\frac{2L+1}{2l_2+1}} \bmat{l_1 &L &l_2\\ m_1 &-M &-m_2}\\
&= (-1)^{l_2+m_2}\sqrt{\frac{2L+1}{2l_1+1}} \bmat{L &l_2 &l_1\\ -M & m_2 &-m_1}
}\end{equation}

\subsection{对称性选择定则}
符合基本选择定则的 CG 系数也可能为 0。 我们可以用更多的选择定则来找到这些系数。 由 3j 符号\upref{ThreeJ}的选择定则(\autoref{ThreeJ_eq7},\autoref{ThreeJ_eq8}), 当第一行三个数之和为奇数时\footnote{由 CG 系数的对称性推导得到的条件会复杂很多,但与该条件是等价的。}
\begin{equation}
\bmat{l_1 & l_2 & L\\ 0 & 0 & 0}
= \bmat{l & l & L\\ m & m & M}
= \bmat{l_1 & l & l\\ m_1 & m & -m}
= \bmat{l & l_2 & l\\ m & m_2 & -m}
= 0
\end{equation}
注意即使加入了这些选择定则, CG 系数仍然可能为零, 事实上要找到所有的零系数非常困难\footnote{见 T A Heim, J Hinze and A R P Rau, Some classes of `'nontrivial zeroes' of angular momentum addition coefficients, 一个例子是 $\bmat{3 & 3 & 2 \\ -2 & 2 & 0}$, 注意第一行之和是偶数。}。

\subsection{特殊情况}
一些情况下\autoref{SphCup_eq2} 有更简单的表达式
\begin{equation}\label{SphCup_eq9}
\bmat{l_1 & l_2 & 0 \\ m_1 & m_2 & 0} = \frac{(-1)^{l_1-m_1}}{\sqrt{2l_1+1}}
\end{equation}

\subsection{正交归一性}
由于 $\ket{l_1, l_2, L, M}$ 都是正交归一的, 所以有
\begin{equation}\label{SphCup_eq11}
\ali{
&\sum_{m_1, m_2} \braket{l_1', l_2', L', M'}{l_1, m_1, l_2, m_2}\braket{l_1, m_1, l_2, m_2}{l_1, l_2, L, M}\\
&= \delta_{l_1, l_1'} \delta_{l_2, l_2'} \delta_{L, L'} \delta_{M, M'}
}\end{equation}
即
\begin{equation}
\sum_{m_1, m_2} \bmat{l'_1 & l'_2 & L'\\ m_1& m_2 &M'} \bmat{l_1 & l_2 & L\\ m_1& m_2 &M} = \delta_{l_1, l_1'} \delta_{l_2, l_2'} \delta_{L, L'} \delta_{M, M'}
\end{equation}
根据选择定则, 只需要对满足\autoref{SphCup_eq4} 和\autoref{SphCup_eq5} 的 $m_1$ 和 $m_2$ 即可(双重求和变为一次求和)。

\subsection{与球谐函数的关系}
三个球谐函数之积的积分可以表示成两个 CG 系数或 3j 符号\upref{ThreeJ}相乘\footnote{见 Bransden 附录 A4, 以及 Wikipedia 的 3j/CG coefficients 页面}
\begin{equation}\label{SphCup_eq1}
\ali{
&\quad \int Y_{l_1 m_1} (\uvec r) Y_{l_2 m_2} (\uvec r) Y_{l_3 m_3}(\uvec r) \dd{\Omega}\\
&= (-1)^{m_3} \sqrt{\frac{(2l_1+1)(2l_2+1)}{4\pi(2l_3+1)}} \bmat{l_1& l_2& l_3\\ 0 & 0 & 0}\bmat{l_1 & l_2 & l_3\\  m_1 & m_2 & -m_3}\\
&= \sqrt{\frac{(2l_1+1)(2l_2+1)(2l_3+1)}{4\pi}}  \pmat{l_1& l_2& l_3\\ 0 & 0 & 0}\pmat{l_1 & l_2 & l_3\\  m_1 & m_2 & m_3}
}\end{equation}
