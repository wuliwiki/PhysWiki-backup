% IDA-star 算法
% 搜索|算法

\pentry{迭代加深\upref{ID} A-star 算法\upref{Astar}}

IDA* 算法是加了估价函数的迭代加深。

A* 算法是在优先队列 BFS 上加了估价函数,估价函数也当然可以和 DFS 结合,但如果只是和最普通的 DFS 结合在一起很容易出现搜索深度很深,但答案深度很浅的情况,所以可以将迭代加深 DFS 和估价函数结合在一起,就形成了 IDA-star 算法。IDA* 算法的估价函数和 A* 非常类似,都是表示当前状态到目标状态的估计距离,IDA* 当然也要满足估计距离不大于真实距离这个前提。IDA* 使用迭代的框架,如果当前深度加估计距离大于深度限制,则直接回溯。

\textbf{IDA* 的框架:}
\begin{lstlisting}[language=cpp]
int f();   // 估价函数

// depth 当前搜索层数,max_depth 为迭代加深的搜索深度限制
bool dfs(int depth, int max_depth)  
{
    // 如果当前层数 + 估价函数 > 深度限制,则直接回溯
    if (depth + f() > max_depth) return false; 
    if (!f()) return true;  // 一般估价函数为 0 说明找到了答案
    
    /*
    以下为 dfs 内容
    */
    
    return false;   // 找不到答案就回溯
}

int main()
{
    int depth = 0;
    while (!dfs(0, depth)) depth ++ ;   // 迭代加深
    
    return 0;
}
\end{lstlisting}