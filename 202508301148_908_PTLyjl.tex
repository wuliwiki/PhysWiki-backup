% 庞特里亚金类(综述)
% license CCBYSA3
% type Wiki

本文根据 CC-BY-SA 协议转载翻译自维基百科\href{https://en.wikipedia.org/wiki/Pontryagin_class}{相关文章}

在数学中,庞特里亚金类,以列夫·庞特里亚金的名字命名,是实向量丛的一类特征类。庞特里亚金类所在的上同调群的次数总是4的倍数。
\subsection{定义}
给定定义在流形$M$ 上的一个实向量丛$E$,它的第 $k$ 阶庞特里亚金类$p_k(E)$ 定义为:
$$
p_k(E) = p_k(E, \mathbb{Z}) = (-1)^k \, c_{2k}(E \otimes \mathbb{C}) 
\;\; \in \;\; H^{4k}(M, \mathbb{Z})~
$$
其中:
\begin{itemize}
\item $c_{2k}(E \otimes \mathbb{C})$:表示向量丛$E$的复化$E \otimes \mathbb{C} = E \oplus iE$的第 $2k$ 阶陈类。
\item $H^{4k}(M, \mathbb{Z})$:表示流形$M$的次数为 $4k$ 的整系数上同调群。
\end{itemize}
有理庞特里亚金类$p_k(E, \mathbb{Q})$ 则定义为$p_k(E)$在$H^{4k}(M, \mathbb{Q})$(即流形 $M$ 的次数为$4k$的有理系数上同调群)中的像。
\subsection{性质}
总庞特里亚金类,实向量丛$E$的总庞特里亚金类定义为:
$$
p(E) = 1 + p_1(E) + p_2(E) + \cdots \;\;\in\;\; H^*(M, \mathbb{Z}),~
$$
其中 $H^*(M, \mathbb{Z})$ 表示流形 $M$ 的整系数上同调环。

该类在模去 2 阶挠元后,对向量丛的Whitney和(Whitney sum)是乘法的,即:
$$
2p(E \oplus F) = 2p(E) \smile p(F),~
$$
其中 $\smile$ 表示上同调中的杯积,$E$ 和 $F$ 是定义在同一流形 $M$ 上的两个向量丛。

分量形式下的性质,对各阶庞特里亚金类 $p_k$ 而言,有:

一阶:
  $$
  2p_1(E \oplus F) = 2p_1(E) + 2p_1(F),~
  $$
二阶:

  $$
  2p_2(E \oplus F) = 2p_2(E) + 2p_1(E) \smile p_1(F) + 2p_2(F),~
  $$
以此类推。

消失条件并不意味着平凡性,庞特里亚金类和 Stiefel–Whitney 类同时消失,并不意味着该向量丛是平凡丛。例如,在 9 维球面 $S^9$ 上,存在一个唯一的、秩为 10 的非平凡向量丛 $E_{10}$(在向量丛同构意义下)。其黏合函数来自同伦群:
$\pi_8(O(10)) = \mathbb{Z}/2\mathbb{Z}$.在这个例子中:所有庞特里亚金类都消失,因为 $S^9$ 在次数 9 上不存在对应的上同调群。所有 Stiefel–Whitney 类也都消失,特别是第九阶 Stiefel–Whitney 类 $w_9$ 由Wu 公式:$w_9 = w_1 w_8 + Sq^1(w_8)$可知为零。此外,这个向量丛是稳定非平凡的:即使把 $E_{10}$ 与任意秩的平凡丛做 Whitney 和,结果依然是非平凡的【Hatcher 2009, p.76】。

对于一个$2k$维向量丛$E$,有如下关系:
$$
p_k(E) = e(E) \smile e(E),~
$$
其中:$e(E)$ 表示向量丛$E$的欧拉类;$\smile$表示上同调类的杯积。
\subsubsection{庞特里亚金类与曲率}
大约在 1948 年,陈省身与安德烈·韦伊证明,有理庞特里亚金类:
$$
p_k(E, \mathbb{Q}) \;\in\; H^{4k}(M, \mathbb{Q})~
$$
可以表示为依赖于向量丛曲率形式的多项式微分形式。这一结果,即陈–韦伊理论,揭示了代数拓扑与整体微分几何之间的重要联系。

对于定义在$n$维可微流形$M$上、配备了联络的向量丛$E$,其总庞特里亚金类可以表示为:
$$
p = \left[
1
-\frac{{\rm Tr}(\Omega^2)}{8\pi^2}
+\frac{{\rm Tr}(\Omega^2)^2 - 2{\rm Tr}(\Omega^4)}{128\pi^4}
-\frac{{\rm Tr}(\Omega^2)^3 - 6{\rm Tr}(\Omega^2){\rm Tr}(\Omega^4) + 8{\rm Tr}(\Omega^6)}{3072\pi^6}
+\cdots
\right]
\;\in\; H_{dR}^*(M),~
$$
其中:$\Omega$ 表示该向量丛的曲率形式;$H_{dR}^*(M)$ 表示流形$M$的de Rham上同调群。
\subsubsection{流形的庞特里亚金类}
一个光滑流形的庞特里亚金类被定义为其切丛的庞特里亚金类。

1966 年,诺维科夫证明:若两个紧致、定向的光滑流形是同胚的,那么它们在$H^{4k}(M, \mathbf{Q})$中的有理庞特里亚金类$p_k(M, \mathbf{Q})$ 是相同的。此外,如果流形的维数至少为 5,那么在给定同伦类型和庞特里亚金类的情况下,至多只有有限多个不同的光滑流形符合这一条件\(^\text{[1]}\)。
\subsubsection{由陈类导出庞特里亚金类}
复向量丛$\pi: E \to X$的庞特里亚金类可以完全由其陈类决定。这是因为有以下事实:
向量丛的实复化满足:$E \otimes_{\mathbb{R}} \mathbb{C} \;\cong\; E \oplus \bar{E}$,其中$\bar{E}$是$E$的复共轭丛。Whitney 和公式以及复共轭丛陈类的性质:$c_i(\bar{E}) = (-1)^i \, c_i(E)$,$c(E \oplus \bar{E}) = c(E) \cdot c(\bar{E})$.基于这些关系,可以得到公式:
$$
1 - p_1(E) + p_2(E) - \cdots + (-1)^n p_n(E)
= (1 + c_1(E) + \cdots + c_n(E)) \cdot (1 - c_1(E) + c_2(E) - \cdots + (-1)^n c_n(E)).~
$$
在曲线上的复向量丛,对曲线来说,公式化简为:
$$
(1 - c_1(E))(1 + c_1(E)) = 1 + c_1(E)^2.~
$$
因此,在曲线上的所有复向量丛,其庞特里亚金类都是平凡的。一般情况下的展开,观察乘积的前两项:
$$
(1 - c_1(E) + c_2(E) + \ldots + (-1)^n c_n(E))
\cdot
(1 + c_1(E) + c_2(E) + \ldots + c_n(E))
= 1 - c_1(E)^2 + 2c_2(E) + \ldots~
$$
可以得到:$p_1(E) = c_1(E)^2 - 2c_2(E)$.对于线丛,由于维度原因有:$c_2(L) = 0$因此公式进一步简化。
\subsubsection{四次 K3 曲面上的庞特里亚金类}
回顾一下,一个在$\mathbb{CP}^3$(复射影三维空间)中消零集合为光滑子簇的四次多项式,所定义的就是一个K3 曲面。

利用如下的法丛正合列:
$$
0 \;\longrightarrow\; \mathcal{T}_X
\;\longrightarrow\; \mathcal{T}_{\mathbb{CP}^3}|_X
\;\longrightarrow\; \mathcal{O}(4)
\;\longrightarrow\; 0~
$$
我们可以得到:
$$
\begin{aligned}
c(\mathcal{T}_X)
&= \frac{c(\mathcal{T}_{\mathbb{CP}^3}|_X)}{c(\mathcal{O}(4))} \\[6pt]
&= \frac{(1 + [H])^4}{(1 + 4[H])} \\[6pt]
&= (1 + 4[H] + 6[H]^2) \cdot (1 - 4[H] + 16[H]^2) \\[6pt]
&= 1 + 6[H]^2.
\end{aligned}~
$$
由此可知:

第一陈类:$c_1(X) = 0$,第二陈类:$c_2(X) = 6[H]^2$.由于根据Bézout 引理,$[H]^2$对应于四个点,因此第二陈数为:$c_2 = 24$.在这种情况下,由于公式:$p_1(X) = -2c_2(X)$

我们得到:$p_1(X) = -48$.这个数值可以进一步用于计算球面的第三稳定同伦群\(^\text{[3]}\)。
\subsection{庞特里亚金数}
庞特里亚金数是一类光滑流形的拓扑不变量。对于流形$M$,如果其维数不是 4 的倍数,则所有庞特里亚金数都为零。定义:

给定一个光滑的$4n$维流形$M$ 和一组自然数:

$k_1, k_2, \ldots, k_m$满足:$k_1 + k_2 + \cdots + k_m = n$,

则庞特里亚金数$P_{k_1, k_2, \ldots, k_m}$ 定义为:
$$
P_{k_1, k_2, \ldots, k_m}
= p_{k_1} \smile p_{k_2} \smile \cdots \smile p_{k_m}([M]),~
$$
其中:$p_k$:表示流形 $M$ 的第 $k$ 个庞特里亚金类;$[M]$:表示流形 $M$的基本类。
\subsubsection{性质}
\begin{enumerate}
\item 庞特里亚金数是定向边界同胚不变量;结合Stiefel–Whitney 数,它们可以确定一个定向流形的定向边界同胚类。
\item 闭黎曼流形的庞特里亚金数(以及庞特里亚金类)可以通过对黎曼流形曲率张量的某些多项式积分**来计算。
\item 一些拓扑不变量,如符号与$\hat{A}$-属($\hat{A}$-genus),可以通过庞特里亚金数来表示。关于庞特里亚金数线性组合给出流形符号的定理,可参见希尔伯茨符号定理。
\end{enumerate}
\subsection{推广}
对于具有四元数结构的向量丛,还存在四元数庞特里亚金类。
\subsection{另见}
\begin{itemize}
\item 陈–西蒙斯形式
\item 希尔伯茨符号定理
\end{itemize}
\subsection{参考文献}
\begin{enumerate}
\item Novikov, S. P. (1964). "Homotopically equivalent smooth manifolds. I". *Izvestiya Akademii Nauk SSSR. Seriya Matematicheskaya*. 28**: 365–474. MR 0162246.
\item Mclean, Mark. "Pontryagin Classes" (PDF). 2016-11-08 存档 (PDF) \[自出版资料?]
\item "A Survey of Computations of Homotopy Groups of Spheres and Cobordisms" (PDF). p. 16. 2016-01-22 存档 (PDF) 

\end{enumerate}