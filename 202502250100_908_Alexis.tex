% 亚历克西·克洛德·克莱罗(综述)
% license CCBYSA3
% type Wiki

本文根据 CC-BY-SA 协议转载翻译自维基百科\href{https://en.wikipedia.org/wiki/Alexis_Clairaut}{相关文章}。
\begin{figure}[ht]
\centering
\includegraphics[width=6cm]{./figures/d23fa39a2d6e39da.png}
\caption{阿列克西·克劳德·克莱罗} \label{fig_Alexis_1}
\end{figure}
阿列克西·克劳德·克莱鲁(Alexis Claude Clairaut,发音:/ˈklɛəroʊ/;法语:[alɛksi klod klɛʁo];1713年5月13日-1765年5月17日)是法国数学家、天文学家和地球物理学家。他是杰出的牛顿主义者,他的工作帮助确立了艾萨克·牛顿在1687年《自然哲学的数学原理》中所阐述的原则和结果的有效性。克莱鲁是前往拉普兰的远征的关键人物之一,该远征帮助确认了牛顿关于地球形状的理论。在这一背景下,克莱鲁推导出一个数学结果,现称为“克莱鲁定理”。他还解决了引力三体问题,是第一个成功地得出月球轨道近日点进动的令人满意结果的人。在数学上,他还因提出克莱鲁方程和克莱鲁关系而闻名。