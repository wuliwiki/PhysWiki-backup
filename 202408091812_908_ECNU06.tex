% 华东师范大学 2006 年 考研 量子力学
% license Usr
% type Note

\textbf{声明}:“该内容来源于网络公开资料,不保证真实性,如有侵权请联系管理员”

\subsection{(15分)}
两个无简并的厄密算符$\hat{A},\hat{B}$满足:$\hat{A}'=\hat{B}'=l$.$\hat{A}\hat{B}+\hat{B}\hat{A}=0$。求:
\begin{enumerate}
\item 在$A$表象中$\hat{A},\hat{B}$的矩阵表达形式;
\item 由$A$表象到$B$表象的么正变换矩阵$S$.
\end{enumerate}
\subsection{(15分)}
考虑三个自旋为$1/2$的非全同粒子组成的系统,哈密顿量为$$\hat{H} = \frac{A}{\hbar^2} \hat{\mathbf{S}}_1 \cdot \hat{\mathbf{S}}_2 + \frac{B}{\hbar^2} (\hat{\mathbf{S}}_1 \cdot \hat{\mathbf{S}}_3 + \hat{\mathbf{S}}_2 \cdot \hat{\mathbf{S}}_3)~$$ ($A$ 和 $B$为两个变量数)

其中$\hat{\mathbf{S}}_1, \hat{\mathbf{S}}_2, \hat{\mathbf{S}}_3$ 分别为三个粒子的自旋算符。求该系统的能级及能级简并度。