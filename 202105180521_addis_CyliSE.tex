% 柱坐标系中的薛定谔方程
% 薛定谔方程|柱坐标|径向方程|哈密顿算符

\begin{issues}
\issueDraft
\end{issues}

\pentry{球坐标系中的定态薛定谔方程\upref{RadSE}}

\begin{equation}
u(r) = \sqrt r R(r)
\end{equation}
\begin{equation}
H = K_r + \frac{L_z^2}{2m r^2}
\end{equation}
\begin{equation}
K_r R = -\frac{1}{2m} \frac1r \dv{r} \qty(r \dv{R}{r}) =  - \frac{1}{2m} \frac{1}{\sqrt r} \qty(\dv[2]{u}{r} + \frac{u}{4 r^2})
\end{equation}
\begin{equation}
\frac{L_z^2}{2m r^2}\psi  = \frac{1}{2m} \frac{m_z^2}{r^2}\psi 
\end{equation}
所以径向方程为
\begin{equation}
- \frac{1}{2m} \dv[2]{u}{r} + \qty[V(r) + \frac{1}{2m} \qty(\frac{m_z^2}{r^2} - \frac{1}{4 r^2})]u = Eu
\end{equation}
