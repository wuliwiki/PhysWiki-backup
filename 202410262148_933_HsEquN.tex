% 等式与不等式(高中)
% keys 方程|不等式|代数基本定理
% license Xiao
% type Tutor

\begin{issues}
\issueDraft
\end{issues}


相等和不等关系是从小学阶段就开始接触的基础概念,但由此延伸出的方程、不等式、恒等式、方程组、解等概念,许多人往往会感到不知所谓。人教版初中教材中给出的“方程”定义是:“含有未知数的等式称作方程(equation)。”给“未知数”的定义则是“方程的求解目标”,这看上去是一种令人迷惑的循环定义。到高中阶段,在高中教材中依然没有系统性的澄清这些概念。

因此,很多学生在阅读题目时,对解题任务的理解感到模糊,不清楚解方程、联立方程究竟意味着什么,这种认识上的模糊甚至延续到大学阶段,影响对更复杂概念的掌握和后续学习的进展。很多研究者在使用这些术语时也很混乱。本文旨在解决上面提到的问题,下面会先介绍一些基础概念。这些概念并不要求完整记忆,但需要认真理解,才能对什么是方程有一个比较清晰的认知。

\subsection{关于等式和不等式的基础概念}

表示数学运算的符号,如加、减、乘、除以及各种函数等,称作\textbf{运算符}。表示两个数学元素关系的符号,称作\textbf{关系符},比如$>,<,\leq,\geq,\neq$称作\textbf{不等号},$=$称作\textbf{等号}。其他的关系符还有集合中的包含、属于,几何中的平行、垂直,逻辑中的等价、蕴含等。

由数字、变量以及运算符组成的数学符号串,称作\textbf{数学表达式(mathematic expression)},或直接简称\textbf{表达式(expression)}。表达式可以看作是对某个值或状态的描述,关键点在于它不需要求解,它只是对某种数学数量的表示。对表达式可以进行简化、合并同类项,或者在特定情况下给出某个变量的值进行替代计算。初中时学过的\textbf{代数式(algebraic expression)}是表达式的子集,它特指用运算符号把数或表示数的字母连起来的式子,即将运算限定在了加、减、乘、除以及有理数次的乘方、开方。

这里不得不先介绍一个英语单词“equation”,这个词在中文中通常会翻译做等式或方程。

% 什么是等式和不等式
用等号连接的两个表达式,用于描述相等关系,称作等式。用不等号连接两个表达式,用于描述不等关系,称作不等式。%举例。

含有等号的式子叫做等式。 等式可分为矛盾式和条件等式、恒等式。

只在某些特定的变量取值下成立的等式称作\textbf{条件方程(conditional equation)}

不等式与等式一样

如何理解$x=3$?

\subsection{条件方程与不等式方程}

含有未知数的等式


而方程的解是特定值。


方程是对于所涉及变量的某些值成立的数学陈述。
另一方面,恒等式是对于所涉及变量的所有允许值都成立的方程。无论其变量的值如何,恒等式都保持相同的相等性。

广泛使用的方程的定义是“含有未知数的等式”,但这样的定义会使人困惑如$x=3$是否也是一个方程。

方程是包含未知数的表达两个代数表达式相等的数学关系等式。


\subsection{方程组}

\subsection{解与解集}

方程和不等式只在解或解集中成立。

方程的解可以分为两大类:解析解和数值解。如果方程的解可以通过有限次的常见运算(如加、减、乘、除等)得到,这种解称为\textbf{解析解(Analytical Solution)}。这时,解的表达式可以用代数形式清晰地表示出来。有些复杂的方程很难找到解析解,甚至解析解根本不存在。在这种情况下,可以使用数值分析方法,如二分法、牛顿法等,通过迭代和近似计算来求解方程。此时得到的解称为\textbf{数值解(Numerical Solution)}。数值解通常通过计算机来计算,能够为复杂问题提供高精度的近似解。

总的来说,解析解是精确的,但不总是存在;数值解是近似的,却总是能提供实用的近似结果。在高中阶段,一般只涉及解析解,但存在大量的方程无法获得解析解,或难以获得解析解。

\subsubsection{有理不等式的解集}

穿针法

\subsubsection{解与零点}

\begin{definition}{代数学基本定理}
任何一个 $n$ 次多项式函数在复数域上都有 $n$ 个零点(重数计入)。
\end{definition}
这意味着在复数范围内,可以找到所有多项式方程的解。



