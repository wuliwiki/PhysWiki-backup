% 泛函与线性泛函
% keys 泛函|线性泛函
% license Usr
% type Tutor

\pentry{向量空间\nref{nod_LSpace}}{nod_db0a}
泛函是线性(向量)空间中的数值函数。泛函的英文单词为“functional”,后缀“-al”在这里表示“属于,像,相关的”\footnote{见https://www.etymonline.com/cn/word/-al}。因此,“functional” 就指代与函数相关的对象。而“泛”在这里的中文的意思是“泛指”的意思,即比“函数”更广的函数。事实上,“泛函”完美反映了泛函本身的定义。这个“更广”广在泛函的定义空间不再特指数构成的线性空间——数域,而是更一般的线性空间。

从泛函的定义,就能够避免现存的大多数误解:即把泛函理解作函数的函数。事实上这一说法不光理解错误,表述也是错误的。稍微正确一点的表述是函数(线性)空间的函数,然而这只是泛函的特殊情形,更yi'ban


