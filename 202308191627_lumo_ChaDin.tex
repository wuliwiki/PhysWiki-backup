% 求定积分的一些方法
% license Xiao
% type Tutor

\begin{issues}
\issueDraft
\end{issues}

\pentry{牛顿—莱布尼兹公式(简明微积分)\upref{NLeib},不定积分的常用技巧\upref{intech}}

\footnote{本文参考了\cite{同济高},\cite{Thomas}与武忠祥的《考研高数》课程}在实操中,定积分的计算也一般交给计算机完成(运用符号积分或者数值积分);不过,既然考试还喜欢考定积分计算,你就\textsl{不得不}会做点题。

\subsection{微积分基本定理}
原则上说,因为你总能通过微积分基本定理(牛顿—莱布尼兹公式)\upref{NLeib}联系不定积分\upref{Int}(原函数)与定积分,因此你可以先找到相应的原函数再赋值计算。这使得不定积分的所有积分技巧\upref{intech}同样适用于定积分,此处不再复述。

值得注意的是,在运用分部积分法或换元法后,你必须相应的改变积分上下限。例如,$$\int^b_a uv'dx = uv|^b_a-\int^b_a vu'dx~,$$, 
$$\int^b_a f'(g(x))g'(x)dx = \int ^{u(b)}_{u(a)} f'(u)du~.$$


然而,由于定积分有确定的积分上下限,比起不定积分,定积分有着更多样的、也更方便的求解技巧。因此,做定积分时应先化简积分式,而不要直接求原函数(这往往过于复杂)。以下介绍一些求解定积分的方法。注意,这些结论只适用于定积分(不能用他们求解不定积分!),并假定函数在区间上连续、可积。

\subsection{对称性}
运用被积函数的对称性(奇偶性)可以简化计算。
$$
\int ^a_{-a} f(x) = 
\left \{
\begin{aligned}
0&,\text{f是奇函数,f(-x)=-f(x)}\\
2\int ^a_0 f(x)&,\text{f是偶函数,f(-x)=f(x)}\\
\end{aligned}~.
\right.
$$
在实操中,\textbf{运用对称性时,往往需要拆分积分区间或积分函数},从而更好地发现、运用对称性。

\subsection{周期性}
如果T是f的一个周期,那么 $\int ^{a+nT}_{a} f(x)dx= n\int^{T}_0 f(x)dx~.$

\subsection{几何含义}
有些被积函数有着特殊的几何意义,例如$\int ^R_0 \sqrt{R^2-x^2}=\frac{\pi R^2}{4}$。(被积函数是圆的方程,那么积分的几何意义便是圆的面积的一部分。)

\subsection{三角函数特殊公式}
$\int ^\pi_0 x f(\sin(x))dx = \frac{\pi}{2}\int ^\pi_0 f(\sin(x))dx~.$

点火公式:
$
\int ^{\frac{\pi}{2}}_{0} \sin^n(x) dx= 
\left \{
\begin{aligned}
\frac{n-1}{n} \frac{n-3}{n-2}...\frac{1}{2} \frac{\pi}{2},\text{n是偶数}\\
\frac{n-1}{n} \frac{n-3}{n-2}...\frac{2}{3},\text{n是奇数}\\
\end{aligned}~.
\right.
$
