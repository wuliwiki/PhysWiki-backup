% 相变的热力学量变化

\begin{issues}
\issueDraft
\issueOther{感觉质量不大行...}
\end{issues}


\begin{figure}[ht]
\centering
\includegraphics[width=8 cm]{./figures/9f9b0c7adf0109a9.pdf}
\caption{即使压力与温度相同,液态水与水蒸气的各种热力学性质还是不同} \label{fig_PTTVC2_1}
\end{figure}

我们此次先假定系统中只有一种物质。在相简介(热力学)\upref{PHS} 中我们已经知道物质的性质与物质的相态有关。比如说,即使压力与温度相同,液态水与水蒸气的各种热力学性质还是不同。
$$
\begin{aligned}
U_g(p,T) &\ne U_l(p,T)\\
S_g(p,T) &\ne S_l(p,T)\\
H_g(p,T) &\ne H_l(p,T)\\
&...
\end{aligned}
$$
其中$U_g(p,T)$指气态水的内能、$U_l(p,T)$指液态水的内能等。

进一步地,我们可以定义相应的差值:
$$
\begin{aligned}
U_g(p,T) &= U_l(p,T) + \Delta ^ g_l U (p,T)\\
S_g(p,T) &= S_l(p,T) + \Delta ^ g_l S (p,T)\\
H_g(p,T) &= H_l(p,T) + \Delta ^ g_l H (p,T)\\
&...\\
\end{aligned}
$$
其中,$\Delta ^ g_l U (p,T)$就代表了气态水与液态水的内能之差等。从更有物理意义的角度而言,这个差值意味着,等温等压条件下系统发生相变时,系统的内能就必然增加(减少)这么多的量。

在国内教材的符号规范中,$\Delta^g_l$的上标$g$代表相变后的末相气相,下标$l$代表相变前的初始相液相。显然,由于$U$是与温度、压力、物质种类有关的量,因此$\Delta ^ g_l U (p,T)$也是与温度、压力、物质种类有关的量。



实操中更常用的量是等温等压条件下相变的焓变与熵变:
$$
\Delta H, \Delta  S
$$

\begin{example}{相变潜热}
为什么相变焓是一个重要的热力学量?我们回顾焓的含义:在等温等压相变、无非体积功过程中,有
$$
\Delta H = \delta q - \delta w_{others} = \delta q
$$
也就是说,这种情况下系统的焓变直接代表着系统的热效应。因此,相变焓也称为“相变潜热”。
\end{example}

由此可以计算 相变的Gibbs自由能变
$$
\Delta G = \Delta H - T \Delta S
$$
由于状态量的广延性质,更常用的是摩尔热力学量变,即在变化量上除以物质的量。符号上补充$M$下标\footnote{在一些课本中,$M$下标被省略,$\Delta X$始终指相应的摩尔量变。}。例如,摩尔焓变:
$$
\Delta H_M = \Delta H / n
$$
那么
$$
H_g = H_l + n \cdot \Delta^g_l H_M
$$
