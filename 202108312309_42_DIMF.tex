% 量纲式
% 量纲式|物理规律

\begin{issues}
\issueTODO
\end{issues}

\pentry{单位制和量纲\upref{USD}}
在单位制和量纲\upref{USD}这一节中我们提到了量纲式,并且给出了\autoref{USD_def1}~\upref{USD},本节将证明量纲式的通用表达式,并给出另外一种较为更“数学”的定义.本节的定理将给出一个极其重要的结论,它使得我们对物理规律有一个更深刻的认识.
\subsection{定义方程}
在单位制和量纲\upref{USD}这一节中,\autoref{USD_ex1}~\upref{USD} 和\autoref{USD_ex2}~\upref{USD} 提到了定义方程和终极定义方程,我们先给出它们的定义.
\begin{definition}{}
在单位制 $\mathscr{Z}$ 中,定义导出量类 $\tilde{\boldsymbol{C}}$ 单位 $\hat{\boldsymbol{C}}_{\mathscr{Z}}$ 的物理规律对应的方程称为该单位制 $\mathscr{Z}$ 中$\hat{\boldsymbol{C}}_{\mathscr{Z}}$ 的\textbf{定义方程}.
\end{definition}
\begin{definition}{}
在某一单位制 $\mathscr{Z}$ 中,若将导出量类 $\tilde{\boldsymbol{C}}$ 的单位 $\hat{\boldsymbol{C}}_{\mathscr{Z}}$ 的\textbf{定义方程} 中涉及的量类(该导出量类 $\tilde{\boldsymbol{C}}$ 除外)都用基本量类来表示,得到的定义方程称为该单位制 $\mathscr{Z}$ 中$\hat{\boldsymbol{C}}_{\mathscr{Z}}$ 的\textbf{终极定义方程},简称 \textbf{终定方程}.
\end{definition}

\begin{theorem}{}\label{DIMF_the2}
任一单位制 $\mathscr{Z}$ 的任一导出单位 $\hat{\boldsymbol{C}}_{ \mathscr{Z}}$ 的终定方程都是幂单项式,即
\begin{equation}\label{DIMF_eq6}
C=k_{\text{终}}J_1^{\sigma_1}\cdots J_l^{\sigma_l}
\end{equation}
\end{theorem}
\textbf{证明:}记单位制为 $\mathscr{Z}$,为便于陈述,设 $\mathscr{Z}$ 制有3个基本量类——$\tilde{\boldsymbol{l}}$、$\tilde{\boldsymbol{m}}$和$\tilde{\boldsymbol{t}}$.选定基本单位$\hat{\boldsymbol{l}}$、$\hat{\boldsymbol{m}}$和$\hat{\boldsymbol{t}}$ .任一导出量类 $\tilde{\boldsymbol{C}}$ 的导出单位 $\hat{\boldsymbol{C}}$ 的终定方程为
\begin{equation}
C=f(l,m,t)
\end{equation}
设 $\mathscr{Z}$ 是 $\mathscr{Z}$ 的同族制,其基本单位是$\hat{\boldsymbol{l}}'$、$\hat{\boldsymbol{m}}'$和$\hat{\boldsymbol{t}}'$,导出量类 $\tilde{\boldsymbol{C}}$ 的单位是  $\hat{\boldsymbol{C}}'$,又有
\begin{equation}\label{DIMF_eq1}
C'=f(l',m',t')
\end{equation}
$f$ 不加撇是因为同族制有相同的定义方程.令
\begin{equation}
x\equiv \eval{\mathrm{dim}}_{\mathscr{Z,Z'}}\tilde{\boldsymbol{l}},\quad y\equiv \eval{\mathrm{dim}}_{\mathscr{Z,Z'}}\tilde{\boldsymbol{m}},\quad z\equiv \eval{\mathrm{dim}}_{\mathscr{Z,Z'}}\tilde{\boldsymbol{t}}
\end{equation}
则由\autoref{USD_eq9}~\upref{USD},得
\begin{equation}
x=\frac{\hat{\boldsymbol{l}}}{\hat{\boldsymbol{l}}'}=\frac{l'}{l},\quad y=\frac{\hat{\boldsymbol{m}}}{\hat{\boldsymbol{m}}'}=\frac{m'}{m},\quad z=\frac{\hat{\boldsymbol{t}}}{\hat{\boldsymbol{t}}'}=\frac{t'}{t},\quad 
\end{equation}
故 $l'=xl,m'=ym,t'=zt$,代入\autoref{DIMF_eq1} 得
\begin{equation}
C'=f(xl,ym,zt)
\end{equation}
由量纲的定义又知
\begin{equation}
\mathrm{\eval{dim}_{\mathscr{Z,Z'
}}}\tilde{\boldsymbol{C}}=\frac{f(xl,ym,zt)}{f(l,m,t)}
\end{equation}
上式右边可写成 $x,y,z$ 的函数 $g(x,y,z)$(因为 $\mathcal{Z}$ 制固定而 $\mathscr{Z'}$ 任意, 导致变数仅是 $x,y,z$),则
\begin{equation}\label{DIMF_eq2}
f(xl,ym,zt)=g(x,y,z)f(l,m,t)
\end{equation}
上式两边对 $x$ 求导,得
\begin{equation}
lf_1(xl,ym,zt)=g_1(x,y,z)f(l,m,t)
\end{equation}
下标1代表对第一个自变数求偏导.取 $x=y=z=1$,得
\begin{equation}
lf_1(xl,ym,zt)=g_1(1,1,1)f(l,m,t)
\end{equation}
令 $\lambda=g_1(1,1,1)$,则上式可写为
\begin{equation}
l\pdv{f}{l}=\lambda f
\end{equation}
积分便得 
\begin{equation}\label{DIMF_eq3}
f(l,m,t)=\phi(m,t)l^{\lambda}
\end{equation}
其中,$\phi(m,t)$ 是 $m,t$ 的某个函数.

上式带入\autoref{DIMF_eq2} ,便消去了 $l$ 并给出
\begin{equation}
\phi(ym,zt)=g(x,y,z)\phi(m,t)x^{-\lambda}
\end{equation}
将上式对 $y$ 求偏导,得
\begin{equation}
m\phi_1(ym,zt)=g_2(x,y,z)\phi(m,t)x^{-\lambda}
\end{equation}
其中, $g_2(x,y,z)$ 的下标2代表对第二个自变数求导.取 $x=y=z=1$,又得
\begin{equation}
m\phi_1(m,t)=g_2(1,1,1)\phi(m,t)
\end{equation}
令 $\mu=g_2(1,1,1)$,则上式可简记为
\begin{equation}
m\pdv{\phi}{m}=\phi f
\end{equation}
积分得
\begin{equation}\label{DIMF_eq4}
\phi(m,t)=\phi(t)m^{\mu}
\end{equation}
其中 $\phi(t)$ 是 $t$ 的某个函数.类似的还可求得函数
\begin{equation}\label{DIMF_eq5}
\phi(t)=kt^{\tau}
\end{equation}
其中, $\tau=g_3(1,1,1)$,$k$ 是积分常数.
联立\autoref{DIMF_eq3} 、\autoref{DIMF_eq4} 、\autoref{DIMF_eq5} ,最终得
\begin{equation}
f(l,m,t)=kl^{\lambda}m^{\mu}t^{\tau}
\end{equation}
可见, $f(l,m,t)$ 的确是幂单项式.该方法可推广至任一单位制,有兴趣读者不妨试试.

定理得证.

\begin{theorem}{}\label{DIMF_the1}
任一物理量类的量纲式都存在,而且都是幂单项式.特别的,对于导出量类 $\tilde{\boldsymbol{C}}$,若其导出单位的终定方程形为\autoref{DIMF_eq6},则其量纲式为
\begin{equation}\label{DIMF_eq8}
\mathrm{dim}\tilde{\boldsymbol{C}}=\qty(\mathrm{dim}\tilde{\boldsymbol{J}}_1)^{\sigma_1}\cdots\qty(\mathrm{dim}\tilde{\boldsymbol{J}}_l)^{\sigma_l}
\end{equation}
其中,$\sigma_1,\cdots,\sigma_l$ 称为 $\tilde{\boldsymbol{C}}$ 的\textbf{量纲指数}.
\end{theorem}
\textbf{证明:}设 $\mathscr{Z}$ 为任一单位制, 对基本物理量类,\autoref{DIMF_the1} 显然成立.对导出物理量类,\autoref{DIMF_the2} 告诉我们其有终定方程\autoref{DIMF_eq6} .设 $\mathscr{Z'}$ 为与 $\mathscr{Z}$ 同族的另一单位制,以 $C'$ 和 $J_1',\cdots,J_l'$ 代表各有关量在 $\mathscr{Z'}$ 制的数,则
\begin{equation}\label{DIMF_eq7}
C'=k_{\text{终}}J_1'^{\sigma_1}\cdots J_l'^{\sigma_l}
\end{equation}
 式中, $k_{\text{终}}$ 不变是由同族单位制\autoref{USD_def2}~\upref{USD} 中的条件2决定的.\autoref{DIMF_eq6} 和\autoref{DIMF_eq7} 两式相除便有
 \begin{equation}
 \frac{C'}{C}=\qty(\frac{J_1'}{J_1})^{\sigma_1}\cdots\qty(\frac{J_l'}{J_l})^{\sigma_l}
 \end{equation}
 由\autoref{QCU_eq8}~\upref{QCU}
 \begin{equation}
 \frac{\hat{\boldsymbol{C}}}{\hat{\boldsymbol{C}}'}=\frac{C'}{C}, \quad\frac{\hat{\boldsymbol{J}}_i}{\hat{\boldsymbol{J}}_i'}=\frac{J_i'}{J_i}\quad (i=1,\cdots ,l)
 \end{equation}
 所以
 \begin{equation}
 \frac{\hat{\boldsymbol{C}}}{\hat{\boldsymbol{C}}'}=\qty(\frac{\hat{\boldsymbol{J}}_1}{\hat{\boldsymbol{J}}_1'})^{\sigma_1}\cdots\qty(\frac{\hat{\boldsymbol{J}}_l}{\hat{\boldsymbol{J}}_l'})^{\sigma_l}
 \end{equation}
 由\autoref{USD_eq9}~\upref{USD},即得\autoref{DIMF_eq8} .定理得证.

 \autoref{DIMF_the1} 表明导出量类 $\tilde{\boldsymbol{C}}$ 的量纲是基本量类的量纲的 $l$ 元函数( 为基本量类的个数),函数关系由导出量类 $\tilde{\boldsymbol{C}}$ 的量纲式\autoref{DIMF_eq8}  给出
 \begin{definition}{}
 \autoref{DIMF_eq8} 称为量类 $\tilde{\boldsymbol{C}}$ 在单位制族 $\tilde{\mathscr{Z}}$ 中的\textbf{量纲式}.$\mathscr{Z}$ 为\autoref{DIMF_the1} 证明中所选的单位制.
 \end{definition}
 \begin{definition}{}
 量纲指数全部为0的量称为\textbf{无量纲量}或\textbf{量纲恒为1的量}.
 \end{definition}
与\autoref{DIMF_the1} 的证明一样,容易证明下面的定理