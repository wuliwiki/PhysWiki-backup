% Python RoboMaster EP 教程—与机器人建立连接
% keys 机器人|Robomaster
% license Usr
% type Tutor

Robomaster EP 支持3种与计算机连接的方式:WiFi 直连模式,WiFi 组网模式和 USB(RNDIS) 连接模式。

\autoref{sub_PyRM2_1} WIFI直连

\subsection{WIFI直连}\label{sub_PyRM2_1}

Wi-Fi 直连 :通过将机器人设置为直连模式,并连接机器人的 Wi-Fi 热点进行接入(智能中控顶部有热点名称与密码),Wi-Fi 直连模式下,机器人默认 IP 为 192.168.2.1。

开启机器人电源,切换智能中控的连接模式开关至 直连模式,如图1所示:

\begin{figure}[ht]
\centering
\includegraphics[width=4cm]{./figures/bde4a2d47ab0cf95.png}
\caption{直连模式} \label{fig_PyRM2_1}
\end{figure}

输入以下代码,可以测验是否连接

\begin{lstlisting}[language=python]
from robomaster import robot

import robomaster

if __name__ == '__main__':
    # 如果本地IP 自动获取不正确,手动指定本地IP地址
    # robomaster.config.LOCAL_IP_STR = "192.168.2.20"
    
    # 在程序内实例化Robot对象
    ep_robot = robot.Robot()

    # 程序与机器人联网并指定连接方式为AP 直连模式
    ep_robot.initialize(conn_type='ap')

    version = ep_robot.get_version()
    # 输出机器人固件版本号信息
    print("Robot version: {0}".format(version))
    # 关闭机器人
    ep_robot.close()
\end{lstlisting}
