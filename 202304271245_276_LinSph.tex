% 直线和球的交点
% 直线|球|交点|内积|几何矢量|Matlab

\pentry{Matlab 的函数\upref{MatFun}, 一元二次方程}

直线参数方程可以用几何矢量\upref{GVec}表示为 $\bvec P_0 + \lambda \uvec v$, 其中 $\uvec v$ 是单位矢量。 球心为 $\bvec P_c$, 半径为 $R$, 那么直线和球相交的点满足
\begin{equation}
(\bvec P_0 + \lambda \uvec v - \bvec P_c)^2 = R^2
\end{equation}
其中左边的平方表示矢量和自身的内积\upref{Dot}, 即矢量模长的平方。 展开后得到关于 $\lambda$ 的二次方程
\begin{equation}
\lambda^2 + 2\uvec v \vdot (\bvec P_0 - \bvec P_c) \lambda + (\bvec P_0 - \bvec P_c)^2 - R^2 = 0
\end{equation}
直接使用求根公式得
\begin{equation}
\lambda = \frac{-b \pm \sqrt{b^2 - 4c}}{2}
\end{equation}
其中
\begin{equation}
b = 2\bvec v \vdot (\bvec P_0 - \bvec P_c)
\qquad
c = (\bvec P_0 - \bvec P_c)^2 - R^2
\end{equation}

以下给出 Matlab 代码
\addTODO{应该支持多条直线而不是多个球……}
\begin{lstlisting}[language=matlab, caption=LSph2P.m]
% 计算直线和球的焦点
% numel(L)=6, size(P)=[N,3], numel(R)=1
function [P1,P2] = LSph2P(L6, P, R)
L6 = L6(:)';
P0 = L6(1:3); v0 = L6(4:6)/norm(L6(4:6));
P0_P = P0-P;
b = 2*P0_P*v0';
c = dot(P0_P, P0_P, 2) - R^2;
d = b.^2 - 4*c;
if d < 0
    P1 = []; P2 = []; return;
end
d = sqrt(d);
P1 = 0.5*(-b -d).*v0 + P0;
if d == 0
    P2 = []; return;
end
P2 = 0.5*(-b +d).*v0 + P0;
end
\end{lstlisting}
