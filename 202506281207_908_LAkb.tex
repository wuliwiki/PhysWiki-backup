% 利昂·库珀(综述)
% license CCBYSA3
% type Wiki

本文根据 CC-BY-SA 协议转载翻译自维基百科\href{https://en.wikipedia.org/wiki/Leon_Cooper}{相关文章}。

莱昂·N·库珀(Leon N. Cooper,原名Kupchik;1930年2月28日-2024年10月23日)是美国的理论物理学家和神经科学家。他因在超导性方面的工作获得诺贝尔物理学奖。库珀提出了库珀对的概念,并与约翰·巴丁和约翰·罗伯特·施里弗合作,发展了常规超导性的BCS理论\(^\text{[1][2]}\)。在神经科学领域,库珀共同开发了BCM理论,用以解释突触可塑性\(^\text{[3]}\)。
\subsection{传记}
\subsubsection{童年与教育}
莱昂·N·库珀于1930年2月28日出生在纽约市的布朗克斯区[4]。他的中间名字“N.”并没有特定含义,尽管一些来源错误地认为他的中间名是尼尔[4]。

他的父亲欧文·库珀来自白俄罗斯,在1917年俄国革命后移民到美国。他的母亲安娜(Anna,原姓Zola)库珀来自波兰;在莱昂七岁时去世[4]。父亲在再婚后将家族姓氏从库珀改为库珀[4]。

莱昂就读于布朗克斯科学高中,并于1947年毕业[5][6]。随后,他在位于曼哈顿上城的哥伦比亚大学学习,1951年获得文学学士学位[7]。他继续留在哥伦比亚大学攻读研究生,分别于1953年获得文学硕士学位[7],并于1954年获得哲学博士学位[7][8]。他的博士论文研究了缪子原子,导师是罗伯特·瑟伯[9][10]。
\subsubsection{科学生涯}
库珀在普林斯顿的高等研究院做了一年的博士后研究。之后,他在伊利诺伊大学厄本那-香槟分校和俄亥俄州立大学任教,直到1958年加入布朗大学[8]。他在布朗大学度过了他余下的职业生涯。

库珀于1973年创立了布朗大学的大脑与神经系统研究所,并成为该所的首任主任[7]。1974年,他被任命为布朗大学的科学教授,这个职位由托马斯·J·沃森资助[7]。库珀曾在多个机构担任访问研究职位,包括普林斯顿高等研究院和瑞士日内瓦的欧洲核子研究中心(CERN)。

他与同事查尔斯·埃尔鲍姆于1975年创立了科技公司Nestor,该公司旨在寻找人工神经网络的商业应用[11][12]。Nestor与英特尔合作,开发了1994年发布的Ni1000神经网络计算机芯片[13]。
\subsubsection{个人生活}
\begin{figure}[ht]
\centering
\includegraphics[width=6cm]{./figures/1d902c058aeb9f86.png}
\caption{} \label{fig_LAkb_1}
\end{figure}
库珀第一次与玛莎·肯尼迪结婚,他们有两个女儿。[4] 1969年,他与凯·阿拉德结婚。[14] 他于2024年10月23日去世,享年94岁,地点是他位于罗德岛普罗维登斯的家中。[4]
\subsection{研究}
\subsubsection{超导性}
\begin{figure}[ht]
\centering
\includegraphics[width=6cm]{./figures/9c5efef362b3409a.png}
\caption{} \label{fig_LAkb_2}
\end{figure}