% 丹尼尔·伯努利(综述)
% license CCBYSA3
% type Wiki

本文根据 CC-BY-SA 协议转载翻译自维基百科\href{https://en.wikipedia.org/wiki/Daniel_Bernoulli}{相关文章}。

\begin{figure}[ht]
\centering
\includegraphics[width=6cm]{./figures/12ffafd992474bb6.png}
\caption{丹尼尔·伯努利的肖像,约1720-1725年} \label{fig_BNL_1}
\end{figure}
丹尼尔·伯努利 FRS(/bɜːrˈnuːli/ 伯-努-利;瑞士标准德语:[ˈdaːni̯eːl bɛrˈnʊli];1700年2月8日[公历1月29日] – 1782年3月27日)是一位瑞士数学家和物理学家,并且是来自巴塞尔的伯努利家族中的众多杰出数学家之一。他特别以其将数学应用于力学,尤其是流体力学,以及在概率和统计学领域的开创性工作而闻名。[3] 他的名字在伯努利原理中得以纪念,这一原理是能量守恒的一个具体例子,描述了支撑20世纪两项重要技术——化油器和飞机机翼——运作原理的数学机制。[4][5]
\subsection{早年生活}
\begin{figure}[ht]
\centering
\includegraphics[width=6cm]{./figures/5f34cc0a7fd312e1.png}
\caption{《流体动力学》(1738年)封面} \label{fig_BNL_2}
\end{figure}
丹尼尔·伯努利出生于荷兰格罗宁根,来自一个著名的数学家家庭。[6] 伯努利家族最初来自安特卫普,当时属于西属荷兰,但为了逃避西班牙对新教徒的迫害,家族移民至瑞士。伯努利家族在短暂居住法兰克福后,最终定居巴塞尔。

丹尼尔是约翰·伯努利(微积分的早期发展者之一)之子,也是雅各布·伯努利(概率论的早期研究者及数学常数e的发现者)之侄。[6] 他有两个兄弟,尼克劳斯和约翰二世。丹尼尔·伯努利被W. W. 罗丝·鲍尔描述为“年轻伯努利家族中最有才华的成员”。[7]

据说,他与父亲的关系很差。他们都参加了巴黎大学的一场科学竞赛,并且并列第一。约翰因无法忍受被认为与丹尼尔平起平坐的“耻辱”,将丹尼尔逐出了家门。约翰据称曾在他的《水力学》一书中抄袭了丹尼尔的《流体动力学》一书中的关键思想,并将这些想法的时间回溯到《流体动力学》出版之前。[citation needed] 丹尼尔曾尝试与父亲和解,但都未能成功。[8]

在上学时,约翰鼓励丹尼尔学习商业,理由是数学家们的收入非常微薄。丹尼尔最初拒绝了,但后来在父亲的要求下,决定一边学习商业一边学习医学,前提是父亲会私下教授他数学。[8] 丹尼尔在巴塞尔、海德堡和斯特拉斯堡学习医学,并于1721年获得解剖学和植物学博士学位。[9]

他是莱昂哈德·欧拉的同代人和亲密朋友。[10] 1724年,他作为数学教授前往圣彼得堡,但在那里非常不愉快。一次短期的生病[8]、俄国东正教会的审查[11]以及与薪水的争执让他找到了离开圣彼得堡的借口,并于1733年离开。[12] 他回到巴塞尔大学,先后担任医学、形而上学和自然哲学的教授直至去世。[13]

1750年5月,他被选为皇家学会会士。[14]
\subsection{数学工作}
\begin{figure}[ht]
\centering
\includegraphics[width=6cm]{./figures/4533568252099694.png}
\caption{丹尼尔·伯努利} \label{fig_BNL_3}
\end{figure}
他最早的数学著作是《Exercitationes》(数学练习),该书于1724年在戈尔德巴赫的帮助下出版。两年后,他首次指出,常常需要将复合运动分解为平移运动和旋转运动。他的主要著作是《Hydrodynamica》(流体动力学),于1738年出版。这本书与约瑟夫·路易·拉格朗日的《解析力学》相似,按单一原理——能量守恒原理来安排,其中的所有结果都可以视为该原理的推论。随后,他发表了关于潮汐理论的论文,与欧拉和科林·麦克劳林的论文一起,获得了法国科学院的奖项:这三篇论文包含了从艾萨克·牛顿的《自然哲学的数学原理》出版到皮埃尔-西蒙·拉普拉斯的研究期间,关于潮汐问题的所有成果。伯努利还撰写了大量关于各种力学问题的论文,特别是与振动弦相关的问题,并给出了布鲁克·泰勒和让·勒龙·达朗贝尔的解答。[7]
\begin{figure}[ht]
\centering
\includegraphics[width=6cm]{./figures/a5fdc0a9aec6cccf.png}
\caption{1738年版伯努利的《流体动力学》} \label{fig_BNL_4}
\end{figure}
\begin{figure}[ht]
\centering
\includegraphics[width=6cm]{./figures/52fe45a7cfc714d6.png}
\caption{1738年版《流体动力学》第一部分的第一页} \label{fig_BNL_5}
\end{figure}
\subsection{经济学与统计学}  
在他1738年出版的《新风险度量理论概述》(*Specimen theoriae novae de mensura sortis*)一书中,伯努利提出了对圣彼得堡悖论的解决方案,作为风险厌恶、风险溢价和效用经济学理论的基础。伯努利常常注意到,在做出涉及不确定性的决策时,人们并不总是试图最大化可能的货币收益,而是试图最大化“效用”,这一经济学术语包含了个人的满足感和利益。伯努利意识到,对于人类来说,获得的金钱与效用之间存在直接关系,但这种关系随着获得的金钱增多而逐渐减弱。例如,对于年收入为10,000美元的人来说,额外获得100美元的收入将带来比年收入为50,000美元的人更多的效用。

伯努利在1766年对天花的发病率和死亡率数据进行的分析,是最早试图分析涉及删失数据的统计问题之一,他通过这项研究证明了接种疫苗的有效性。
\subsection{物理学}  
在《流体动力学》(*Hydrodynamica*,1738年)中,伯努利为气体的动力学理论奠定了基础,并将这一思想应用于解释博伊尔定律。

他与欧拉合作研究弹性理论,并共同发展了欧拉-伯努利梁方程。伯努利原理在气动学中具有重要应用。

根据列昂·布里卢因(Léon Brillouin)的说法,叠加原理最早由丹尼尔·伯努利于1753年提出:“一个振动系统的整体运动可以表示为其固有振动的叠加。”
\subsection{工作}
\begin{itemize}
\item 《获得1737年皇家科学院双重奖的作品》(*Pieces qui ont remporté le Prix double de l'Academie royale des sciences en 1737*)。巴黎:皇家印刷局,1737年。
\end{itemize}
\subsection{遗产}
\begin{itemize}
\item 2002年,伯努利被列入圣地亚哥航空航天博物馆的国际航空航天名人堂。[21]
\end{itemize}
\subsection{参见}
\begin{figure}[ht]
\centering
\includegraphics[width=6cm]{./figures/37593ef005d28131.png}
\caption{《获得1737年皇家科学院双重奖的作品》(法语)} \label{fig_BNL_6}
\end{figure}
\begin{itemize}
\item 《流体动力学》  
\item 传染病的数学建模
\end{itemize}
\subsection{参考文献}  
\subsubsection{脚注}  
\begin{enumerate}
\item Mangold, Max (1990). 《Duden — 发音词典》. 第3版. 曼海姆/维也纳/苏黎世, Duden出版社.  
\item "Daniel Bernoulli". 知名人物数据库. 访问日期:2019年10月14日.  
\item Anders Hald (2005). 《概率与统计的历史及其在1750年前的应用》. 约翰·威利与儿子出版社. 第6页. ISBN 9780471725176.  
\item Richard W. Johnson (2016). 《流体动力学手册》. CRC出版社. 第2-5页. ISBN 9781439849576.  
\item Dale Anderson; Ian Graham; Brian Williams (2010). 《飞行与运动:飞行的历史与科学》. Routledge出版社. 第143页. ISBN 9781317470427.  
\item Rothbard, Murray. 《丹尼尔·伯努利与数学经济学的创立》存档于2013年7月28日通过Wayback Machine, Mises Institute(摘自《从奥地利视角看经济思想史》)  
\item Rouse Ball, W. W. (2003) [1908]. 《伯努利家族》. 《数学历史简短记》 (第4版). Dover出版社. ISBN 0-486-20630-0.  
\item O'Connor, John J.; Robertson, Edmund F., "Daniel Bernoulli", MacTutor数学历史档案,圣安德鲁斯大学 (1998).  
\item Anderson, John David (1997). 《空气动力学的历史及其对飞行器的影响》. 纽约,NY: 剑桥大学出版社. ISBN 0-521-45435-2.  
\item Calinger, Ronald (1996). "莱昂哈德·欧拉:圣彼得堡的第一段岁月 (1727–1741)" (PDF). 《数学史》. 23 (2): 121–166. doi:10.1006/hmat.1996.0015. 存档于2019年3月28日.
\item Calinger, Ronald (1996). 第127页  
\item Calinger, Ronald (1996),第127-128页  
\item [匿名] (2001) "丹尼尔·伯努利", 《大英百科全书》  
\item "图书馆与档案馆目录"。皇家学会。访问日期:2010年12月13日。[永久失效链接]  
\item 英文翻译见:Bernoulli, D. (1954). "关于风险度量新理论的阐述"(PDF)。《经济计量学》。22 (1): 23-36. doi:10.2307/1909829. JSTOR 1909829. S2CID 9165746. 存档(PDF)于2008年5月13日。  
\item 斯坦福哲学百科全书: "圣彼得堡悖论" 由 R. M. Martin 编写  
Cooter & Ulen (2016),第44-45页。  
\item 再版于 Blower, S; Bernoulli, D (2004). "小儿麻痹症的死亡率与接种的优势的新分析尝试"(PDF)。《医学病毒学评论》。14 (5): 275–288. doi:10.1002/rmv.443. PMID 15334536. S2CID 8169180. 存档(PDF)于2007年9月27日。  
\item Timoshenko, S. P. (1983) [1953]. 《材料强度的历史》。纽约:Dover出版社. ISBN 0-486-61187-6.  
\item Brillouin, L. (1946). 《周期性结构中的波传播:电气滤波器与晶体格子》,麦格劳–希尔,纽约,第2页。  
\item Sprekelmeyer, Linda, 编辑。《我们敬重的人:国际航空航天名人堂》。\item Donning Co. Publishers, 2006. ISBN 978-1-57864-397-4.  
\end{enumerate}
\subsection{参考文献} 
\begin{itemize}
\item (原文基于公共领域的《数学史》罗斯条目)  
\item Chisholm, Hugh, 编者(1911年)。"Bernoulli § V. Daniel Bernoulli"。《大英百科全书》。第3卷(第11版)。剑桥大学出版社,第805页。  
\item Cardwell, D.S.L.(1971年)。《从瓦特到克劳修斯:早期工业时代热力学的兴起》。海内曼出版社:伦敦。ISBN 0-435-54150-1。  
\item Cooter, Robert; Ulen, Thomas(2016年)。《法与经济学》。伯克利法学书籍(第6版)。伯克利:阿迪森-韦斯利出版社。ISBN 978-0-13-254065-0。  
\item Mikhailov, G.K.(2005年)。"Hydrodynamica"。收录于 Grattan-Guinness, Ivor(编)。《西方数学的里程碑著作 1640–1940》。爱思唯尔出版社,第131–142页。ISBN 978-0-08-045744-4。  
\item Pacey, A. J.; Fisher, S. J.(1967年12月)。"Daniel Bernoulli and the vis viva of compressed air"。《英国科学史期刊》。第3卷(第4期):388–392。doi:10.1017/S0007087400002934. S2CID 145513749。  
\item Straub, Hans(1970年)。"Bernoulli, Daniel"。《科学传记词典》。第2卷。纽约:查尔斯·斯克里布内尔公司,第36–46页。ISBN 0-684-10114-9。
\end{itemize}
\subsection{外部链接}
\begin{itemize}
\item "Bernoulli Daniel"。Mathematik.ch。已从原始网站归档,归档日期为2015年10月23日。检索日期:2007年9月7日。
\item Rothbard, Murray. 《丹尼尔·伯努利与数学经济学的创立》,2013年7月28日存档于Wayback Machine,米塞斯研究所(摘自《奥地利视角下的经济学思想史》)。
\item Weisstein, Eric Wolfgang(编)。"Bernoulli, Daniel (1700–1782)"。ScienceWorld。
\item 丹尼尔·伯努利的作品,存档于古腾堡计划。
\item 关于丹尼尔·伯努利的作品或资料,存档于互联网档案馆。
\end{itemize}