% 双缝实验
% license CCBYSA3
% type Wiki

(本文根据 CC-BY-SA 协议转载自原搜狗科学百科对英文维基百科的翻译)

双缝实验(double-slit experiment,或称双狭缝实验)是一种演示光子或电子等等微观物体的波动性与粒子性的实验。

在现代物理学中,双缝实验证明光和物质可以显示经典的波和粒子的特性;此外,它显示了量子力学现象的基本概率性质。这个实验是托马斯·杨在1801年首次用光进行的。1927年,戴维孙和格默证明了电子也有相同的性质,这种性质后来扩展到原子和分子。早在量子力学和波粒二象性概念出现之前,托马斯·杨的光实验就是经典物理学的一部分。他认为这证明了光波理论是正确的,他的实验有时被称为杨氏实验[1] 或杨氏狭缝。

该实验属于一类普通的“双路径”实验,其中一个波被分成两个独立的波,然后合并成一个波。两种波的路径长度的变化会导致相移,从而产生干涉图样。另一个版本是马赫-曾德尔干涉仪,它用镜子分割光束。

在这个实验的基本版本中,相干光源,例如激光束,照亮有两个平行狭缝的平板,穿过狭缝的光在平板后面的屏幕上被观察到。[2] 光的波动性质导致穿过两个狭缝的光波发生干涉,在屏幕上产生亮带和暗带——如果光是由经典粒子组成的话,这种结果是不可能出现的。[2][3] 然而,人们总是发现光在屏幕上的离散点被吸收,作为单个粒子(而不是波)打在屏幕上,通过粒子密度的变化会显示出干涉图案。[4] 此外,包括狭缝处探测器的实验版本发现,每个探测到的光子穿过一个狭缝(就像经典粒子一样),而不是穿过两个狭缝(就像波一样)。[5][6][7][8][9] 然而,这些实验证明,如果检测到粒子穿过哪个狭缝,它们就不会形成干涉图样。这些结果证明了波粒二象性原理。[10][11]

当向双缝发射时,发现其他原子级实体,如电子,表现出相同的行为。 此外,个别离散撞击的探测被观察到具有内禀的概率性,这用经典力学是无法解释的。

这个实验可以用比电子和光子大得多的实体来完成,尽管随着尺寸的增加会变得更加困难。进行双缝实验的最大实体是每个包含810个原子的分子(其总质量超过10,000个原子质量单位)。

双缝实验(及其变体)已成为经典的思想实验,因为它清晰地表达了量子力学的核心难题。因为它证明了观察者预测实验结果能力的根本局限性,理查德·费曼称之为“一种无法用任何经典方式解释的现象,这种现象蕴含着量子力学的核心。事实上,它包含了量子力学中唯一的神秘。

\subsection{概观}
\begin{figure}[ht]
\centering
\includegraphics[width=8cm]{./figures/423775cb916e2722.png}
\caption{相同的双缝配置(缝间0.7 mm);在上图中,一条狭缝是闭合的。在单缝图像中,由于狭缝的非零宽度,形成了衍射图样(主带两侧的微弱点)。在双缝图像中也可以看到衍射图样,但其强度是单缝图像的两倍,并增加了许多较小的干涉条纹。} \label{fig_SFSY_1}
\end{figure}
如果光严格地由普通粒子或经典粒子组成,这些粒子通过狭缝以直线发射,并被允许照射到另一侧的屏幕上,我们将会看到与狭缝的大小和形状相对应的图案。然而,当实际进行这个“单缝实验”时,屏幕上的图案是光扩散的衍射图案。狭缝越小,扩散角度越大。图像的顶部显示了当红色激光照射狭缝时形成的图案的中心部分,如果仔细观察,还有两个微弱的边带。使用更精细的仪器可以看到更多的波段。这可以用衍射来解释,即图案是光波从狭缝干涉的结果。
\begin{figure}[ht]
\centering
\includegraphics[width=8cm]{./figures/0726aeb0a385322b.png}
\caption{粒子波函数的模拟:双缝实验。白色的模糊表示粒子。像素越白,则在如果测量到那个地方找到一个粒子的概率就越大。} \label{fig_SFSY_2}
\end{figure}
如果一个人照亮两个平行的狭缝,来自两个狭缝的光再次干涉。这里的干涉是一个更明显的模式,有一系列交替的亮带和暗带。波段的宽度是照明光频率的一个特性。[12] (见右边底部的照片。)当托马斯·杨(1773-1829)首次证明这一现象时,它表明光由波组成,因为亮度的分布可以用波前的交替加法和减法干涉来解释。[13] 杨的实验在19世纪初进行,在人们接纳光波理论方面发挥了重要作用,击败了艾萨克·牛顿提出的光微粒学说,后者是17和18世纪公认的光传播模型。然而,后来光电效应的发现表明,在不同的情况下,光可以表现得好像是由离散的粒子组成的。这些看似矛盾的发现使得有必要超越经典物理学,将光的量子性质考虑在内。

费曼喜欢说所有的量子力学都可以通过仔细思考这个单一实验的含义而得到。[13] 他还提出(作为思想实验),如果探测器放置在每个狭缝之前,干涉图样就会消失。[14]

恩格尔特-格林伯格对偶关系提供了量子力学背景下双缝干涉数学的详细处理。

泰勒于1909年首次进行了低强度双缝实验,[15] 方法是降低入射光的水平,直到光子发射/吸收事件基本不重叠。直到1961年,图宾根大学的克劳斯·约翰逊用电子束进行了双缝实验,才使用光以外的东西进行实验。[16][17] 1974年,意大利物理学家皮尔·乔治·梅里(Pier Giorgio Merli)、吉安·弗朗哥·米西罗利(Gian Franco Missiroli)和朱利奥·波齐(Giulio Pozzi)用单电子和双棱镜(而不是狭缝)重复了这个实验,表明每一个电子都会像量子理论预测的那样干涉自己。[18][19] 2002年,单电子版本的实验被《物理世界》的读者评为“最美丽的实验”。[20]

2012年,斯特凡诺·弗拉博尼(Stefano Frabboni)及其同事最终按照费曼提出的原始方案,用电子和真实狭缝进行了双缝实验。他们将单个电子发送到纳米技术制造的狭缝(约100纳米宽)上,通过用单电子探测器收集传输的电子,他们可以显示双缝干涉图样的形成。[21]

\subsection{实验者的变化}
\subsubsection{2.1 单粒子干涉}
\begin{figure}[ht]
\centering
\includegraphics[width=6cm]{./figures/c2906c47e7124052.png}
\caption{电子随时间的积聚} \label{fig_SFSY_3}
\end{figure}
单粒子干涉这个实验的一个重要版本涉及单个粒子(或波——为了一致性,这里称它们为粒子)。如预期的那样,通过双缝设备一次发送一个粒子会导致单个粒子出现在屏幕上。然而,值得注意的是,当这些粒子被允许一个接一个地积累起来时,干涉图样就出现了(见相邻的图像)。这证明了波粒二象性,即所有物质都表现出波粒两种性质:粒子在单一位置被测量为单一脉冲,而波描述了在屏幕上特定位置吸收粒子的概率。[22] 这种现象已经被证明发生在光子、电子、原子甚至包括布基球在内的一些分子身上。[23][24][25][26][27] 因此,电子实验为电子、质子、中子、甚至更大的通常被称为粒子的实体都有自己的波动性质甚至波长(与其动量相关)的观点提供了确证。

被探测到的概率是波振幅的平方,可以用经典波计算(见下文)。粒子不会以可预测的顺序到达屏幕,所以知道所有先前的粒子出现在屏幕上的什么位置以及以什么顺序出现并不能说明未来的粒子将在哪里被检测到。[28] 如果波在某一点被抵消,那并不意味着粒子消失了;它会出现在其他地方。自从量子力学诞生以来,一些理论家就一直在寻找方法来整合额外的决定因素或“隐变量”,如果这些因素或“隐变量”变得为人所知,就能够解释每个个体与目标碰撞的位置。[29]

涉及两个或更多叠加粒子的更复杂的系统不适用上述解释。[30]

\subsubsection{2.2 “双向”实验与互补原则}
一个众所周知的思想实验预测,如果粒子探测器位于狭缝处,显示光子穿过哪个狭缝,干涉图案将消失。[31] 这个单向实验说明了光子既可以作为粒子也可以作为波的互补原理,但是不能同时作为粒子和波来观察。[31][32][33]尽管这个思想实验在量子力学历史上很重要(例如,参见关于爱因斯坦版本实验的讨论),但是直到20世纪70年代才提出这个实验在技术上可行的实现。[34](教科书盖丹肯实验的简单的实现是不可能的,因为没有吸收光子就无法探测到光子。)目前,已经进行了多个实验来说明互补性的各个方面。[35]

1987年 [36][37] 进行的一项实验产生的结果表明,在不完全破坏干涉的情况下,可以知道粒子会选择哪条路径。这表明测量的效果对传输中粒子的干涉程度较小,从而仅在一定程度上影响了干涉图样。换句话说,如果人们不坚持确定光子到底穿过了哪个裂缝,人们仍然可以检测到(退化的)干涉图案。[38]

\subsubsection{2.3 延迟选择和量子擦除变体}
\begin{figure}[ht]
\centering
\includegraphics[width=8cm]{./figures/8e77efff9ca84ce5.png}
\caption{惠勒的延迟选择实验图,展示了光子通过狭缝后确定路径的原理。} \label{fig_SFSY_4}
\end{figure}
惠勒的延迟选择实验表明,在粒子穿过狭缝后提取“哪条路径”信息似乎可以追溯性地改变其先前在狭缝的行为。

量子擦除实验证明,通过擦除或以其他方式使“哪条路径”信息永久不可用,可以恢复波的行为。

《科学美国人》的一篇文章给出了量子擦除现象的一个简单的能够在家做的例子。[39]如果在每个狭缝之前设置偏振器,使其轴相互垂直,干涉图案将被消除。偏振器可以被认为是向每个光束引入了哪条路径的信息。在检测器前面引入第三个偏振器,其轴相对于其他偏振器为45°,从而“擦除”该信息,干涉图案则会重新出现。[39] 这也可以通过将光视为经典波,以及使用圆偏振器和单光子来解决。[40] 使用纠缠光子对的偏振器的实现没有经典的解释。[40]

\subsubsection{2.4 弱测量}
在2012年一次广为人知的实验中,研究人员声称已经确定了每个粒子所走的路径,而对粒子产生的干涉模式没有任何负面影响。[41] 为了做到这一点,他们使用了一种设置,使得到达屏幕的粒子不是来自点状源,而是来自具有两个强度最大值的源。然而,像斯文森这样的评论家指出,[42] 实际上在双缝实验的这个变体中形成的弱测量和海森堡测不准原理之间没有冲突。弱测量后选择不允许同时测量每个粒子的位置和动量,而是允许测量到达不同位置的粒子的平均轨迹。换句话说,实验者正在创建一个完整轨迹景观的统计地图。[42]

\subsubsection{2.5 其他变体}
\begin{figure}[ht]
\centering
\includegraphics[width=6cm]{./figures/e26af4767cb89ff5.png}
\caption{实验室双缝配置; 顶部支柱之间的距离约2.5厘米(1英寸)。} \label{fig_SFSY_5}
\end{figure}
1967年,戈尔和曼德尔用两个独立的激光器作为光源证明了两个光源的干涉。[43][44]

1972年的实验表明,在一个双缝系统中,任何时候只有一个缝是开着的,但只要路径不同,探测到的光子可能来自任何一个缝,就可以观察到干涉。实验条件使得系统中的光子密度远小于1。[45][46]

1999年,双缝实验成功地用布基球分子(每个分子包含60个碳原子)进行。[24][47] 布基球足够大(直径约0.7纳米,比质子大近50万倍),可以在电子显微镜下看到。

在2005年,埃利提出了一个薄金属屏幕的光传输的实验和理论研究,该薄金属屏幕被两个亚波长狭缝穿孔,被许多光波长分开。上述实验显示远场双缝图案的总强度作为入射光束波长的函数而减小或增强。[48]

2012年,内布拉斯加大学林肯分校的研究人员用理查德·费曼描述的电子进行了双缝实验,使用了新的仪器来控制双缝的传输和监控单电子探测事件。电子被电子枪发射,穿过一个或两个62纳米宽×4微米高的狭缝。[49]

2013年,双缝实验成功完成,每个分子包含810个原子(总质量超过10,000个原子质量单位)。[50][50]
\begin{figure}[ht]
\centering
\includegraphics[width=8cm]{./figures/7645d105e9f13e17.png}
\caption{等宽(A)和非等宽(B)等离子体狭缝的近场强度分布规律。} \label{fig_SFSY_6}
\end{figure}

\textbf{流体动力先导波类似物}

流体动力学类似物已经被开发出来,可以重建量子力学系统的各个方面,包括通过双缝的单粒子干涉。[50] 一滴硅油沿着液体表面弹跳,通过与其自身波场的共振相互作用自推进。液滴每次反弹都会轻轻搅动液体。与此同时,过去反弹的涟漪影响了它的进程。液滴与其自身波纹的相互作用形成了所谓的先导波,导致它表现出以前被认为是基本粒子特有的行为——包括通常被认为是基本粒子像波一样在空间中传播而没有任何特定位置的行为,直到它们被测量出。[51][52]

通过这个流体动力学先导波系统模拟的行为包括量子单粒子衍射、[53] 量子轨道、轨道能级分裂、自旋和多模态统计。也可以推断不确定关系和不相容原理。视频显示了该系统的各种功能。

然而,包含两个或更多叠加粒子的更复杂的系统不适合如此简单、经典直观的解释。[30] 因此,还没有开发出纠缠的流体动力学模拟。[50] 然而,光学类似物是可能的。[54]

\subsection{经典波动光学公式}
\begin{figure}[ht]
\centering
\includegraphics[width=8cm]{./figures/f12de47e626543d1.png}
\caption{由平面波产生的双缝衍射图样。} \label{fig_SFSY_7}
\end{figure}
许多光的行为可以用经典波动理论来模拟。惠更斯-菲涅耳原理就是这样一个模型;它指出波前上的每个点产生一个次级小波,并且在任何随后的点上的扰动可以通过对该点上各个小波的贡献求和来找到。这种求和需要考虑各个小波的相位和幅度。只能测量光场的强度——它与振幅的平方成正比。

在双缝实验中,两个狭缝被一束激光照射。如果狭缝的宽度足够小(小于激光的波长),狭缝将光衍射成圆柱形波。这两个圆柱形波阵面是叠加的,并且在组合波阵面的任何点上的振幅以及强度都取决于这两个波阵面的幅度和相位。两个波之间的相位差由两个波传播的距离差决定。

$d \sin\theta$