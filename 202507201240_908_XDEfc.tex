% 薛定谔方程(综述)
% license CCBYSA3
% type Wiki

本文根据 CC-BY-SA 协议转载翻译自维基百科\href{https://en.wikipedia.org/wiki/Schr\%C3\%B6dinger_equation}{相关文章}。

薛定谔方程是一个偏微分方程,用于描述非相对论量子力学体系的波函数的演化过程。\(^\text{[1]: 1–2 }\) 它的发现是量子力学发展史上的一个重要里程碑。该方程以奥地利物理学家埃尔温·薛定谔的名字命名。他于1925年提出该方程,并于1926年发表,从而奠定了其后获得1933年诺贝尔物理学奖的工作基础。\(^\text{[2][3]}\)

在概念上,薛定谔方程是量子力学中对应于经典力学中牛顿第二定律的表达。给定一组已知的初始条件,牛顿第二定律可以用数学方式预测一个物理系统随时间演化的轨迹。薛定谔方程则给出了波函数随时间的演化规律,而波函数是对一个孤立物理系统的量子力学描述。该方程是薛定谔在路易·德布罗意提出“所有物质都具有伴随的物质波”这一假设的基础上提出的。薛定谔方程成功预测了与实验观测一致的原子束缚态。\(^\text{[4]: II:268 }\)

薛定谔方程并不是研究量子力学系统和进行预测的唯一方法。量子力学的其他表述方式还包括维尔纳·海森堡提出的矩阵力学,以及主要由理查德·费曼发展的路径积分表述。在比较这些方法时,使用薛定谔方程的方式有时被称为“波动力学”。

薛定谔提出的方程是非相对论性的,因为它在时间上是一阶导数,而在空间上是二阶导数,因此空间与时间在方程中并不对等。保罗·狄拉克将狭义相对论与量子力学结合成一个统一的表述形式,在非相对论极限下会简化为薛定谔方程。这就是狄拉克方程,它在空间和时间上都只包含一阶导数。

另一个偏微分方程,即克莱因–戈尔登方程,虽然是一个相对论性的波动方程,但在描述概率密度时出现了问题:概率密度可能为负值,这在物理上是不可接受的。狄拉克通过对克莱因–戈尔登算符进行所谓的“开平方”处理,引入了狄拉克矩阵,从而解决了这一问题。

在现代物理的语境中,克莱因–戈尔登方程用于描述无自旋粒子,而狄拉克方程则用于描述自旋为1/2的粒子。
\subsection{定义}
\subsubsection{预备知识}
在物理或化学的入门课程中,通常会以一种仅需掌握基础微积分(特别是关于空间与时间的导数)的概念和符号就能理解的方式来介绍薛定谔方程。薛定谔方程的一个特例,是针对一维空间中单个非相对论粒子的位置空间形式,其表达如下:
$$
i\hbar \frac{\partial}{\partial t}\Psi(x,t) = \left[ -\frac{\hbar^2}{2m} \frac{\partial^2}{\partial x^2} + V(x,t) \right] \Psi(x,t)~
$$
在这个方程中,$\Psi(x, t)$ 是波函数,即为每个时刻 $t$ 下的每个位置 $x$ 分配一个复数值的函数;$m$ 是粒子的质量;$V(x, t)$ 是势能函数,用来表示粒子所处环境中的势场\(^\text{[5]: 74 }\);$i$ 是虚数单位;$\hbar$ 是约化普朗克常数,其单位为作用量(能量乘以时间)\(^\text{[5]: 10 }\)。
\begin{figure}[ht]
\centering
\includegraphics[width=8cm]{./figures/ca5fd69c61da5de2.png}
\caption{满足非相对论自由薛定谔方程(即 $V = 0$)的波函数的复数图像。更多细节参见“波包”。} \label{fig_XDEfc_1}
\end{figure}
在超越上述简单情形的更广义框架中,保罗·狄拉克\(^\text{[6]}\)、大卫·希尔伯特\(^\text{[7]}\)、约翰·冯·诺依曼\(^\text{[8]}\)和赫尔曼·外尔\(^\text{[9]}\)等人所发展出的量子力学数学表述,规定一个量子力学系统的状态是一个向量 $|\psi\rangle$,它属于一个可分的复希尔伯特空间$\mathcal{H}$。该向量被假定在希尔伯特空间的内积下是归一化的,即用狄拉克记号表示,它满足$\langle \psi | \psi \rangle = 1$这个希尔伯特空间的具体形式取决于所研究的系统。例如:用于描述位置和动量的希尔伯特空间是平方可积函数空间 $L^2$;用于描述单个质子的自旋的希尔伯特空间则是二维复向量空间 $\mathbb{C}^2$,配有通常的内积形式\(^\text{[5]: 322 }\)。

感兴趣的物理量 —— 例如位置、动量、能量、自旋 —— 由可观测量来表示,而可观测量是作用于希尔伯特空间上的自伴算符。一个波函数可以是某个可观测量的本征矢,这种情况下称其为该可观测量的本征态,与之对应的本征值表示该本征态下该物理量所取的值。更一般地,一个量子态通常是多个本征态的线性叠加,这被称为量子叠加态。当对某个可观测量进行测量时,结果将是该算符的某个本征值,其出现的概率由玻恩规则给出:在最简单的情形中,如果本征值 $\lambda$ 是非简并的(即仅对应一个本征态),那么概率由$|\langle \lambda | \psi \rangle |^2$给出,其中 $|\lambda\rangle$ 是与本征值 $\lambda$ 对应的本征矢。更一般的情形中,如果本征值是简并的(即对应多个本征态),那么概率由$\langle \psi | P_{\lambda} | \psi \rangle$给出,其中 $P_{\lambda}$ 是投影到对应本征子空间(本征空间)上的投影算符。\(^\text{[note 1]}\)

例如,一个动量本征态将是一个无限延伸的完全单色波,它不是平方可积函数;同样,一个位置本征态将是一个狄拉克δ分布,它也不是平方可积函数,甚至严格来说并不属于函数范畴。因此,它们都不属于粒子的希尔伯特空间。物理学家有时将这些超出希尔伯特空间的本征态视为“广义本征矢”。这类状态用于计算上的便利,但不表示实际的物理态\(^\text{[10][11]: 100–105}\) 。因此,如上文中所使用的位置空间波函数 $\Psi(x, t)$ 可以写成时间相关态矢 $|\Psi(t)\rangle$ 与非物理但便于计算的“位置本征态” $|x\rangle$ 的内积形式:
$$
\Psi(x, t) = \langle x | \Psi(t) \rangle~
$$
\subsubsection{含时薛定谔方程}
\begin{figure}[ht]
\centering
\includegraphics[width=8cm]{./figures/c2d7ce6537b1413d.png}
\caption{这三行中的每一行都是满足含时薛定谔方程的简谐振子波函数。左侧:波函数的实部(蓝色)和虚部(红色)。右侧:在给定位置处找到粒子的概率分布。前两行是定态的例子,它们对应于驻波。最下方一行是一个非常态(非定态)的例子。} \label{fig_XDEfc_2}
\end{figure}
薛定谔方程的形式取决于具体的物理情境。其中最一般的形式是含时薛定谔方程,它描述了一个随时间演化的系统\(^\text{[12]: 143}\) :

\textbf{含时薛定谔方程(一般形式)}
$$
i\hbar \frac{d}{dt}|\Psi(t)\rangle = \hat{H}|\Psi(t)\rangle~
$$
其中,$t$ 是时间,$|\Psi(t)\rangle$ 是量子系统的态矢($\Psi$ 为希腊字母 psi),$\hat{H}$ 是一个可观测量,称为哈密顿算符。

“薛定谔方程”一词既可以指这一\textbf{一般形式},也可以特指其非相对论版本。这个一般形式的方程非常通用,广泛应用于整个量子力学领域,包括狄拉克方程和量子场论等情形,通过代入不同形式的哈密顿量来适配不同系统。而特定的非相对论版本是一种近似形式,在许多实际情况下能给出精确的结果,但其适用范围有限(参见相对论量子力学和相对论量子场论)。

在应用薛定谔方程时,首先需写出系统的哈密顿量,即考虑构成该系统的粒子的动能与势能,然后将其代入薛定谔方程中。由此得到的偏微分方程通过求解波函数来获得,而波函数包含了关于该系统的全部信息。在实际操作中,通常取波函数的绝对值的平方作为概率密度函数来使用\(^\text{[5]: 78 }\)。例如,对于一个在位置空间中的波函数 $\Psi(x, t)$,我们有:
$$
\Pr(x, t) = |\Psi(x, t)|^2~
$$
\subsubsection{定态薛定谔方程}
前述的含时薛定谔方程预测波函数可以形成驻波,称为定态。这些态具有特别重要的意义,因为研究它们可以简化对任意初始态的含时薛定谔方程的求解过程。定态还可以通过一个更简洁的形式来描述,即定态薛定谔方程(含时项被省略的版本)。

\textbf{定态薛定谔方程(一般形式)}
$$
\hat{H}|\Psi\rangle = E|\Psi\rangle~
$$
其中,$E$ 是系统的能量\(^\text{[5]: 134  }\)。该形式仅在哈密顿量本身不显含时间的情况下使用。然而,即使在这种情况下,整个波函数仍然依赖于时间,这一点将在后文关于线性叠加的部分中进一步解释。从线性代数的角度来看,这就是一个本征值方程,因此波函数是哈密顿算符的本征函数,其对应的本征值为 $E$。
