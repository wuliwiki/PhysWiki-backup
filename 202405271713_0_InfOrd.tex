% 无穷小、无穷大和阶数(极简微积分)
% license Xiao
% type Tutor

\pentry{\enref{函数的极限(极简微积分)}{FunLim}}{nod_4ff1}

首先,我们要严格区分一个给定的实数和无穷大或无穷小之间的区别。 一个给定的实数是指一个具体的值,例如 $3.2\e{-100}$ 或者 $1.6\e{1000}$。 无论一个确定的实数多么大或多么小,都不能说它是无穷。

无穷小是一个过程,具体来在求函数 $f(x)$ 的极限时,如果
\begin{equation}\label{eq_InfOrd_1}
\lim_{x\to \square} f(x) = 0~,
\end{equation}
$f(x)$ 就可以叫做无穷小。 这里的 $\square$ 代表一个实数或者 $\pm\infty$。如果我们说一个符号或者函数是无穷小,那么就暗含者我们在讨论某个\autoref{eq_InfOrd_1} 这样的极限。

反之,如果有
\be