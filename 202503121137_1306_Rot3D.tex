% 三维旋转矩阵
% keys 线性代数|矩阵|平面旋转矩阵|空间旋转矩阵
% license Xiao
% type Tutor

\begin{issues}
\issueDraft
\end{issues}

\pentry{平面旋转矩阵\nref{nod_Rot2D}, 自由度\nref{nod_DoF}}{nod_f7b7}

类比\enref{平面旋转矩阵}{Rot2D},空间旋转矩阵是三维直角坐标的旋转变换,所以应该是 $3 \cross 3$ 的方阵。不同的是平面旋转变换只有一个自由度 $\theta $, 而空间旋转变换除了转过的角度还需要考虑转轴的方向, 三维空间中的方向有两个自由度, 所有三维旋转矩阵共有 3 个自由度。

若已经知道空间直角坐标系中三个单位正交矢量
\begin{equation}
\uvec x=\pmat{1\\0\\0}~, \quad
\uvec y=\pmat{0\\1\\0}~, \quad
\uvec z=\pmat{0\\0\\1}~.
\end{equation}
经过三维旋转矩阵变换以后变为另外三个正交归一矢量。 仍然以 $\uvec x, \uvec y, \uvec z$ 作为基底, 把他们分别记为
\begin{equation}\label{eq_Rot3D_1}
\pmat{a_{11}\\a_{21}\\a_{31}} ~,\quad \pmat{a_{12}\\a_{22}\\a_{32}} ~,\quad \pmat{a_{13}\\a_{23}\\a_{33}}~.
\end{equation}
类比\enref{平面旋转矩阵}{Rot2D},可以得到旋转矩阵为
\begin{equation}
\mat R_3 = \begin{pmatrix}
{a_{11}}&{a_{12}}&{a_{13}}\\
{a_{21}}&{a_{22}}&{a_{23}}\\
{a_{31}}&{a_{32}}&{a_{33}}
\end{pmatrix}~.\end{equation}
这 9 个矩阵元只有 3 个是独立的, 因为我们有 6 个条件: 每个列矢量模长等于 1(3 个等式), 且两两间正交(3 个等式)。

除了通过三个单位矢量构建旋转矩阵, 我们可以通过由转轴的方向和旋转的角度来计算每个矩阵元, 参考 “\enref{罗德里格旋转公式}{RotA}” 和 “\enref{四元数}{QuatN}”。 另一种常见的方法是使用\enref{欧拉角}{EulerA}。

\begin{example}{分别给出绕 $x,y,z$ 轴旋转的三维矩阵。}

\begin{figure}[ht]
\centering
\includegraphics[width=12cm]{./figures/6c4cd05249048f8f.png}
\caption{旋转矩阵} \label{fig_Rot3D_1}
\end{figure}

出于转轴已经固定为某一条坐标轴,所以仅需要一个参量 $\theta$ 就可以确定旋转后的矩阵状态。

以绕 $x$ 轴旋转的三维矩阵为例,

\end{example}

\addTODO{例题: 分别给出绕 $x,y,z$ 轴旋转的三维矩阵。}

\subsection{被动理解}
\addTODO{参考 “平面旋转变换” 中的讲述}

% 先解释 平面旋转矩阵中的矩阵元是 x*x' 内积得出的, 任何直角坐标的变换都可以用内积完成。
% 顺便可以解释为什么如果所有列矢量正交归一, 所有行矢量也会正交归一。

\subsection{逆矩阵}
\addTODO{如果我们把\autoref{eq_Rot3D_1} 中的三个正交归一基底记为……}

% 未完成 三维的旋转旋转矩阵与二维旋转矩阵具有许多相似的性质。
