% 横波与纵波

\subsection{横波 Transverse Wave}
\begin{figure}[ht]
\centering
\includegraphics[width=8cm]{./figures/LAT_1.pdf}
\caption{横波示意图,图中可见波峰与波谷。背景灰色圆圈是各粒子的平衡位置。} \label{LAT_fig1}
\end{figure}
横波中,各粒子的振动方向垂直于波的传播方向。相邻微元间受剪切力并发生剪切变形。由于空气等介质不能发生剪切变形,因此横波不能在空气中传播。

\subsection{纵波 Longitudinal Wave}
\begin{figure}[ht]
\centering
\includegraphics[width=8cm]{./figures/LAT_2.pdf}
\caption{横波示意图,图中可见疏部与密部。} \label{LAT_fig2}
\end{figure}
纵波中,各粒子的振动方向平行于波的传播方向。相邻微元间受拉力、压力力并发生拉伸、压缩变形。在空气中,“拉伸变形”、“压缩变形”可以理解为气体分子的疏散与聚集。

\subsection{附录:绘制相应图形的octave/matlab代码}
\begin{lstlisting}[language=matlab]
%绘制横波
T=2;
v=5;
t=0;
A=0.5;

w=2*pi/T;
k=w/v;

for t=0:0.1:10
  [x y]=meshgrid(-5:5);
  xo=x;
  yo=y;
  [a b] = size(x);
  for i = 1:a
    for j=1:b
      wave = A*cos(k*x(i,j)-w*t);
      x(i,j)=x(i,j)+wave;
    end
  end

  clf
  hold on
  scatter(xo,yo,'MarkerEdgeColor',[0.9 0.9 0.9]);
  scatter(x,y);
  axis equal
  axis([-6 6 -6 6])
  axis off
  hold off
  drawnow
  pause(0.1)
end
\end{lstlisting}
