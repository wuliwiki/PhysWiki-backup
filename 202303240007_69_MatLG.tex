% 矩阵李群
% 矩阵李群|李群

\begin{issues}
\issueDraft
\end{issues}

\pentry{一般线性群\upref{GL},域上的代数\upref{AlgFie}}

% 我希望把这个词条写成不需要微分几何前置的样子,参见GTM222
\subsection{矩阵李群}

\footnote{本文参考\cite{GTM222}} $M_n(\mathbb{C})$ 是全体 $n \times n$ 复矩阵的集合。全体 $n \times n$ 可逆矩阵的集合 $\opn{GL}(n, \mathbb{C})$ 构成一个群,同时也是拓扑空间 $M_n(\mathbb{C})$ 的一个开集合(因此是个子流形)。

\begin{definition}{矩阵李群}
对于群 $\opn{GL}(n, \mathbb{C})$ 的子群 $G$,$G$ 被称为一个\textbf{矩阵李群}如果它是 $\opn{GL}(n, \mathbb{C})$ 的一个闭子集。 Definition 1.4 \cite{GTM222} 
\end{definition}

对于一个矩阵李群 $G$,我们有
$$
G \subseteq \opn{GL}(n, \mathbb{C}) \subseteq M_n(\mathbb{C})
$$
$G$ 在 $\opn{GL}(n, \mathbb{C})$ 中是闭的,但在 $M_n(\mathbb{C})$ 中不一定。

\subsection{例子}

\begin{example}{一般线性群和特殊线性群}
\begin{equation}
\begin{aligned}
\opn{GL}(n, \mathbb{R}) \subseteq \opn{GL}(n, \mathbb{C})
\end{aligned}
\end{equation}
为 $\mathbb R$ 上的一般线性群,它被定义为 $\opn{GL}(n,\mathbb R)=\{M\in M_n(n,\mathbb{R}): \det M\neq 0 \}$。
可以看出它是 $n^2$ 维的李群。类似地, ${\opn{GL}(n,\mathbb{C})}$ 为 $2n^2$ 维的李群(以下我们讨论的李群的维数都是它们作为实流形的维数)。

\begin{equation}
\begin{aligned}
\opn{SL}(n, \mathbb{C}) \subseteq \opn{GL}(n, \mathbb{C})
\end{aligned}
\end{equation}
为 $\mathbb C$ 上的特殊线性群,它被定义为 $\{M\in M_n(n,\mathbb{C}): \det M=1 \}$,由于加了 $\det M=1$ 的限制,它相当于 $2n^2$ 维流形上的一个 $2n^2-2$ 维的超曲面,它是 $2n^2-2$ 维的李群。

\begin{equation}
\begin{aligned}
\opn{SL}(n, \mathbb{R}) \subseteq \opn{GL}(n, \mathbb{C})
\end{aligned}
\end{equation}
为 $\mathbb R$ 上的特殊线性群,由于加了 $\det M=1$ 的限制,它相当于 $n^2$ 维流形上的一个 $n^2-1$ 维的超曲面,它是 $n^2-1$ 维的李群。
\end{example}

\subsubsection{幺正群和正交群}
我们通常要研究某个 $n$ 维复空间上保内积的线性变换,即 $(u,v)=(Mu,Mv),\forall u,v\in \mathbb{C}_n$。这也就意味着 $u^\dagger v=u^\dagger M^\dagger M v$ 对于任意 $n$ 维复向量 $u,v$ 都成立,这类线性变换也被称作为幺正变换(或酉变换)。因此幺正群被定义为
\begin{equation}
\begin{aligned}
U(n)=\{ M\in M_n(n,\mathbb C): M^\dagger M=I_n \}
\end{aligned}
\end{equation}
可以证明 $U(n)$ 是 $n^2$ 维的李群。

类似地,在实空间上正交群 $O(n)$ 被定义为 
\begin{equation}
\begin{aligned}
O(n)=\{ M\in M_n(n,\mathbb R): M^T M=I_n \}
\end{aligned}
\end{equation}
可以证明 $O(n)$ 是 $\frac{1}{2}n(n+1)$ 维李群。例如 $n=3$ 时 $O(n)$ 是三维李群,它意味着三维空间旋转由三个连续自由度(参量)描述(除了连续自由度以外,还包括空间反演变换,这意味着正交群并非连通的李群,这也促使我们在数学上去严格地定义“连通”的概念……)。
\subsubsection{广义正交群}
\addTODO{定义}
\begin{definition}{洛伦兹群}
洛伦兹群是满足以下条件的 $4\times 4$ 矩阵所构成的群
\begin{equation}
\Lambda^T \begin{pmatrix}
1&&&\\
&-1&&\\
&&-1&\\
&&&-1
\end{pmatrix}\Lambda =\begin{pmatrix}
1&&&\\
&-1&&\\
&&-1&\\
&&&-1
\end{pmatrix}
\end{equation}
或者
\begin{equation}
O(1,3)=\{\Lambda\in M_n(4,\mathbb R):\Lambda^T\eta\Lambda=\eta\}
\end{equation}

\end{definition}
关于洛伦兹群更丰富的性质见洛伦兹群的李代数\upref{lielot}。

\addTODO{定理:在复数下只有一种正交群}

\subsubsection{辛群和紧辛群}
\addTODO{定义}
% 可以专门开个篇章讲

\subsubsection{(广义)欧几里得群}
\addTODO{定义}
\addTODO{定义:伽利略群和庞加莱群}
% 可以专门开个篇章讲

\subsubsection{海森堡群}
\addTODO{定义}
% 如果物理人认为有必要的话

\subsubsection{性质}
\pentry{李群\upref{LieGrp},紧致性\upref{Topo2},道路连通性\upref{Topo4},}

\begin{theorem}{}
矩阵李群是李群 $\opn{GL}(n, \mathbb{C})$ 的子李群
\end{theorem}

\begin{exercise}{}
证明它。
\end{exercise}

\begin{exercise}{}
判断例子中的那些矩阵李群是
\begin{itemize}
\item 紧致的;
\item (道路)连通的;
\item 单连通的。
\end{itemize}
\end{exercise}
\addTODO{单连通的词条}

