% 矩阵与线性映射
% 矩阵|矢量空间|旋转矩阵|映射|线性映射

\pentry{平面旋转矩阵\upref{Rot2D}, 矢量空间\upref{LSpace}}

% 未完成: 添加算符的概念
% 线性算符将矢量空间 $X$ 中的矢量线性映射到 $Y$ 中. 如果 $X$
% 线性算符对应矩阵, 记为 $\hat A$, 如果有一组正交归一基底 $\ket{u_i}$, 那么矩阵元为
% \begin{equation}
% A_{i,j} = \mel{u_i}{\hat A}{u_j}
% \end{equation}

我们已经知道 $M\times N$ 的矩阵\upref{Mat}可以表示一个 $N$ 维列矢量的线性组合, 得到一个 $M$ 维列矢量(\autoref{Mat_eq4}~\upref{Mat}).
\begin{equation}\label{MatLS_eq1}
\bvec y = \mat A \bvec x
\end{equation}
我们可以把这个 $\bvec x$ 看做任意一个 $N$ 维矢量空间 $X$ 中某矢量关于某组基底 $\{\bvec x_i\}$ 的坐标, 而把 $\bvec y$ 看做任意一个 $M$ 维矢量空间(以下称为 $Y$ 空间)中某矢量关于某组基底 $\{\bvec y_i\}$ 的坐标.

这样, 我们就通过矩阵 $\mat A$ 建立了从 $X$ 空间到 $Y$ 空间的一个映射\upref{map}. 即 $X$ 空间的任意矢量 $\bvec x$, 都可以映射到 $Y$ 空间中唯一矢量 $\bvec y$. 注意映射并不要求不同的 $\bvec x$ 映射到不同的 $\bvec y$, 就像函数 $y = f(x)$ 中一个 $x$ 值只能对应一个 $y$, 值, 但反之则不一定.

特殊地, 当矩阵 $\mat A$ 为方阵时, 矩阵 $\mat A$ 可以用于表示 $X$ 空间到自身的\textbf{自映射}, 即 $\bvec x$ 和 $\bvec y$ 都是 $X$ 空间中的矢量, 但 $\{\bvec x_i\}$ 和 $\{\bvec y_i\}$ 仍然可以是 $X$ 空间中两组不同的基底.

由矩阵与列矢量乘法的性质(\autoref{Mat_eq17}~\upref{Mat})可知 $X$ 空间中若干个矢量做任意线性组合然后映射到 $Y$ 空间, 等于这些矢量先分别映射到 $Y$ 空间再做同样的线性组合. 我们把这样的映射叫做\textbf{线性映射(linear map)}\footnote{以后我们不区分“线性映射” 和“线性变换”.}, 矩阵代表的映射都是线性映射.

作为一个简单的例子, 我们来看平面旋转矩阵\upref{Rot2D}
\begin{equation}
\mat R_2 = \begin{pmatrix}
\cos\theta & - \sin\theta\\
\sin\theta &\cos\theta
\end{pmatrix}
\end{equation}
这是一个方阵, 对应二维矢量空间(例如二维几何矢量构成的空间)的自映射. 对于这个矩阵我们有“主动” 和“被动” 两种理解, 前者假设基底不变而矢量旋转, 后者假设矢量不变而基底旋转\footnote{注意“主动” 和“被动” 并不是两种唯一的理解, 例如我们可以选择让基底顺时针旋转 $\theta/2$, 矢量逆时针旋转 $\theta/2$.}. 这个映射中, 映射前后的矢量有一一对应\upref{map}关系.

我们还可能有\textbf{多对一}映射, 即多个矢量映射后可能得到同一个矢量(\autoref{MatLS_ex1}). 来看一个例子.

\begin{example}{投影矩阵}\label{MatLS_ex1}
我们考虑一个投影变换: 将平面上任意几何矢量投影到 $\uvec x + \uvec y$ 方向上得到该方向的矢量. 已知该变换是线性的, 写出变换矩阵(变换前后使用同一组正交归一基底 $\uvec x_1, \uvec x_2$)

解: 与“平面旋转矩阵\upref{Rot2D}” 中的方法同理, 先考虑各基底的投影变换. $\uvec x_1 = (1, 0)\Tr$ 投影后变为 $(1, 1)\Tr /\sqrt2$, $\uvec x_2$ 投影后同样变为 $(1, 1)\Tr /\sqrt2$, 所以投影变换矩阵即使两个列矢量组成的矩阵
\begin{equation}
\mat P = \frac{1}{\sqrt{2}} \pmat{1 && 1\\ 1 && 1}
\end{equation}

注意该变换中虽然每个矢量都映射到同一空间的唯一的矢量, 但不同的矢量有可能映射到同一个矢量. 所以这是一个多对一映射.
\end{example}

\subsection{定义域和值域}
\pentry{子空间\upref{SubSpc}}
\autoref{MatLS_eq1} 表示的线性映射中, \textbf{定义域(domain)}是 $X$ 空间中的任意矢量, 而\textbf{值域(range)}却不一定是整个 $Y$ 空间, 也可能是 $Y$ 的一个子空间. 例如\autoref{MatLS_ex1} 中投影变换的值域就是沿 $\uvec x_1 + \uvec x_2$ 方向的任意矢量(包括零矢量)构成的一维矢量空间, 是二维矢量空间中的一个子空间.

% 那么, 我们应该如何确定值域呢? 为什么值域必定是一个子空间?
% 未完成:
% 值域空间就是自定义空间中任意一组基底线性变换后张成的空间.
% 值域空间的维度只可能小于或等于定义域空间的维度
% 如果值域空间的维度和定义域空间的维度相同, 那么映射就是一一映射.
% 如果值域空间的维度小于定义域空间的维度, 那么映射就是多对一映射, 且值域中的每个矢量都有无穷多个定义域中的矢量与之对应

% 添加空间的映射及对应符号
