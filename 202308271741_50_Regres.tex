% 回归
% keys 回归 机器学习
% license Xiao
% type Tutor

\textbf{回归}(Regression),在统计学中,是一种用于估计自变量(机器学习中称特征、属性)和因变量(机器学习中称标签)之间相关关系的分析方法[1]。回归分析的过程是确定最能够代表数据趋势的直线或者曲线[2]。所求得的回归直线或曲线,又可以称为拟合直线或拟合曲线。

回归也是一种机器学习中的基本建模方法。当所需要预测的数据是连续型数值时,该任务就是回归任务,须要用到回归方法,所求得的模型可以称为回归模型。这点是回归与分类的主要区别。分类模型所预测的值是离散型数值。



参考文献:
\begin{enumerate}
\item https://en.wikipedia.org/wiki/Regression_analysis
\item https://www.britannica.com/topic/regression-statistics
\end{enumerate}