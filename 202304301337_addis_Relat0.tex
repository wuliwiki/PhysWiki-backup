% 狭义相对论(科普)

\begin{issues}
\issueDraft
\end{issues}

\pentry{经典力学和相对论(科普)\upref{CM0}}

本文的参考系都是指惯性系。

狭义相对论有两个基本假设
\begin{enumerate}
\item 惯性系平权
\item 光速不变
\end{enumerate}

注意这仅仅是两个假设, 就像牛顿三定律一样。 不幸的是由于技术限制我们并没有能力创造一个速度足够快的参考系(例如光速的 5\%)在上面直接测量光速以及其他相对论实验, 但我们有大量间接的方法可以实验验证。 光速不变常被民间科学家用于攻击相对论, 关于光速不变假设的历史背景和一些误解见 “光速不变假设的一些误解和历史\upref{SpeRel}”。

我们知道经典力学中速度是可以相加的, 如果一个人在火车上向前射击, 那么地面上看来子弹的速度等于火车的速度加上子弹相对枪口或者火车的速度。 所以如果有人告诉你说子弹的速度无论相对于火车还是相对于地面都一样快, 那你肯定会觉得他在胡说。 而爱因斯坦提出的光速不变假设恰恰就是说无论使用哪个惯性参考系, 任何光相对于该参考系的速度都是一样的, 无论这个光是从哪里发出如何发出的。 可见狭义相对论推翻了经典力学的根基——\textbf{时空观}。

\subsection{绝对时空观}
为了方便讨论时间的概念, 我们不妨认为每个参考系中, 空间中的每一点都挂满了和这个参考系相对静止的时钟, 且这些时钟制作精良, 不受外界因素干扰, 也不取决于特定的原理(水滴、摆锤、电路等等)。 经典力学的绝对时空观认为, 我们可以一劳永逸地一次让这些钟全部彼此校准, 那么它们就会永远保持绝对的同步,使得无论从什么角度来看, 只要两个事件发生时, 它们所在位置的时钟读数相同, 那么这两个事件就是同时发生的, 毫无歧义。

绝对的空间也可以类似理解。 每个参考系中的长度都是一样的, 火车上的一截尺子无论朝向如何, 如果你在某个时刻把它贴近地面, \textbf{同时}把它的两端的位置在地面做一个标记, 那么地面上这两个标记之间的长度测量出来也等于尺子的长度。 注意这个概念需要建立在上面对\textbf{同时}的定义的基础上。 做这两个标记就可以看成两个事件, 如果我们无法在所有参考系都对两个事件是否同时达成一致, 那么我们也无法明确火车上的长度和地面上的长度是否一致。 可见时间和空间的概念往往是纠缠在一起的, 这也是为什么经常把它们统称为\textbf{时空}。

由于绝对时空观在相对论提出以前根深蒂固, 所以在当时的人看来光速不变是荒谬的。 因为速度的定义是绝对的距离除以绝对的时间间隔, 那么在一段绝对的时间 $\Delta t = t_2 - t_1$ 内(上文的钟表读数相减), 光在火车上走过的距离 $s_1$ 地上的人不会有歧义, 而火车在这段时间内同样走过了一段距离 $s_0$, 那么由于地面上的距离可以相加(即使在狭义相对论中, 同一个参考系中的距离也是可以相加的), 所以光相对于地面走过的距离为 $s_2 = s_0 + s_1$, 所以把 $s_1, s_2$ 分别除以时长 $\Delta t$, 得到光相对于两个参考系的速度必定是不同的, 任何物体的运动也都一样, 光没有理由例外。

\subsection{相对时空观}
所以如果一定要假设光速不变, 那么就必须要改变时空观。 我们尽量保守地改变上面关于时空的假设, 不多不少, 直到新的时空观能容纳光速不变这个现象为止。 首先是关于同时性的问题, 在一个参考系中, 我们如何确定两个不同位置的时钟同步呢? 既然我们假设光速不变, 那最直接的方法就是从这两个时钟的中点同时向它们发射一道光, 如果当光到达两个时钟时, 它们的读数相同, 那我们就说它们\textbf{在当前参考系中同时}。 注意我们在使用文字时必须非常谨慎, 因为我们在推翻根植于常识中的认知。 我们还没开始讨论别的参考系中观察到的东西, 所以要强调只是在当前参考系中同时。 另外这个过程中我们还假设了同一个参考系中长度是绝对的, 可以用一把尺子测量任何地方的长度, 否则我们无法确定两个钟的中点在什么位置。

所以在同一个参考系中, 相对论时空观和绝对时空观的区别并不大, 处于不同位置的观察者仍然对两个事件的同时性和空间的长度么有任何争议。 所以要建立相对时空观, 我们需要从具有相对运动的不同参考系的观察者之间如何看待对方入手, 所以还是回到火车的问题。 注意即使在绝对时空观中, 也不存在这两个参考系哪个更优越的问题——地面上的人可以认为自己静止火车在动, 火车上的人也完全可以认为自己静止而火车外的所有物体都在运动(从太阳的参考系看地球也的确如此)。 而在相对论时空观中, 我们已经知道两个参考系内部已经把和自己相对静止的时钟都分别校准了, 我们还没有对不同参考系之间的时钟做出任何比较。

在进一步讨论之前,我们还要把\textbf{事件}这个概念也做一个抽象。 在给定的参考系中, 仍然假设空间中每一点的时钟都无歧义地进行了同步, 每一个位置也可以用若干坐标描述, 那么一个事件就可以抽象为一个空间位置和该位置的时钟的一个读数, 例如使用三维直角坐标就有 $(x, y, z, t)$。 当然本文只讨论一个方向上的运动, 所以我们只需要一个空间坐标, 所以给定参考系中一个事件可以简化为两个坐标 $(x, t)$。 例如要在火车参考系中同时把尺子的两端在地面上做两个记号, 就可以理解为两个事件, 用火车参考系的时空坐标分别描述为 $(x_1', t_0')$ 以及 $(x_2', t_0')$, 那么火车上两个标记的距离显然就是 $L' = x_2' - x_1'$。 为了区分两个参考系, 我们把火车参考系的时空坐标后面都加一撇, 而地面参考系的坐标则不加。

在绝对时空观中, 同一个事件在不同的参考系中会具有不同的空间坐标, 但时间坐标是与参考系无关的, 空间坐标之间的距离也与参考系无关。 为了避免回到绝对时空观, 我们需要假设同一事件在不同参考系中都可能是不同的, 包括两个事件的各个坐标之间的距离也可能不同。

现在回到光速不变的问题中。 在两个参考系中测量光速的问题可以简化成两个事件, 事件 1 是手电筒发光, 用火车参考系的坐标记为 $(x_1', t_1')$, 事件 2 是一段时间后光到达了空间中某个位置, 记为 $(x_2', t_2')$。 那么在火车中测量的光速就是
\begin{equation}
v' = \frac{x_2' - x_1'}{t_2' - t_1'}
\end{equation}
同理, 在地面上观测这两个事件, 可以分别用 $123$
