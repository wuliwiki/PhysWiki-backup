% 幂零群
% keys nilpotent group|正规序列|次正规序列|normal series|subnormal series|可解群
% license Xiao
% type Tutor

\pentry{换位子群\nref{nod_CmtGrp}}{nod_54d6}

\autoref{the_CmtGrp_2} 说明,给定群$G$,则求换位子群的运算在$G$的正规子群集合上封闭,因此我们可以认为“求换位子群”运算是定义在全体正规子群集合上的。

从运算的角度,我们可以定义$G^1=G$,而对任意正整数$k$,$G^{k+1}=[G, G^{k}]$,像是用求换位子群运算对$G$反复乘方,求其幂次。当然,这只是一个类比,求换位子群运算和乘法差别很大。


\begin{definition}{幂零群}
给定群$G$,若存在正整数$k$使得$G^k=\{e\}$,则称$G$是\textbf{幂零的(nilpotent)}。
\end{definition}

一个显而易见的性质是:


\begin{lemma}{}\label{lem_NipGrp_2}
幂零群的子群和商群都幂零。
\end{lemma}

\textbf{证明}:

设$G$是幂零群,$H$是其任意子群,$N$是其任意正规子群。

由于$H\subseteq G$,故$[H, H^{k}]\subseteq [G, H^{k}]$;又因为$H^1\subseteq G^1$,故可归纳得$H^k\subseteq G^k$对任意正整数$k$成立。由此易证$H$幂零。

由于$G$幂零,故存在正整数$n$使得$G^n\subseteq N$。这意味着,任取$g_1, g_2, \cdots, g_n\in G$,则
\begin{equation}
[g_1, [g_2\cdots[g_{n-1}, g_n]\cdots]] \in N~. 
\end{equation}

因此
\begin{equation}
[g_1N, [g_2N\cdots[g_{n-1}N, g_nN]\cdots]] = N~. 
\end{equation}
由此得证$\qty(G/N)^n=N$。

\textbf{证毕}。



\begin{lemma}{}\label{lem_NipGrp_1}
给定群$G$。若群$N\subseteq \opn{C}(G)$,且$G/N$和$N$皆为幂零群,则$G$幂零。
\end{lemma}


\textbf{证明}:

设$\qty(G/N)^n=N$。

则任取$g_1, g_2, \cdots, g_n\in G$,有
\begin{equation}
[g_1, [g_2\cdots[g_{n-1}, g_n]\cdots]] \in N~. 
\end{equation}

由于$N\subseteq\opn{C}(G)$,故对任意$g_0\in G$和$c\in\opn{C}(G)$有$[g_0, c]=e$。因此
\begin{equation}
[g_0, [g_1, [g_2\cdots[g_{n-1}, g_n]\cdots]]] =e~. 
\end{equation}
即$G^{n+1}=e$。

\textbf{证毕}。


幂零群的性质\autoref{lem_NipGrp_1} 和可解群的性质\autoref{lem_SlvbGp_1} 非常相似,不同的是可解群的此性质只要求$N\lhd G$,幂零群则进一步要求$N\subseteq \opn{C}(G)$。注意,$N\subseteq \opn{C}(G)$可推出$N\lhd G$。

这意味着幂零群的结构与其中心有密不可分的关系。


由\autoref{lem_NipGrp_2} ,$G/\opn{C}(G)$也是幂零群,我们可以再计算这个商群的中心$\opn{C}\qty(G/\opn{C}(G))$,然后和$\opn{C}(G)$比较。不过,商群$G/\opn{C}(G)$和原群$G$的元素不一样,无法直接比较,所以我们需要再把商群的元素用自然同态的逆映射对应回原群来讨论,即我们实际上想要比较的是$\pi^{-1}\qty(\opn{C}\qty(G/\opn{C}(G)))$和$\opn{C}(G)$,这两个群可不一定一样大。



\begin{example}{$\pi^{-1}\qty(\opn{C}\qty(G/\opn{C}(G)))$和$\opn{C}(G)$不同的例子}

考虑置换群$\opn{S}_3$。

简单计算验证可得,$\opn{C}\qty(\opn{S}_3)=\{e\}$。

取$\opn{A}_3\lhd \opn{S}_3$,则$\opn{S}_3/\opn{A}_3=\{0, 1\}$,即二元群,其中$0$代表$\opn{A}_3$。令$\pi:\opn{S}_3\to \opn{S}_3/\opn{A}_3$为自然同态。

显然,$\opn{C}\qty(\{0, 1\})=\{0, 1\}$,即$\pi^{-1}\qty(\opn{C}\qty(\opn{S}_3))$

\end{example}
























