% Euler-Maclaurin求和公式

\subsubsection{Euler求和公式}
\begin{theorem}{}
    设函数 $f:X\to Y$ 在区间$[0,+\infty)$上连续可微,则有如下Euler求和公式成立	
        \begin{equation}\autoref{EMSum_eq1} 
            \sum_{k=1}^{n}f(k)=\int_{0}^{n}f(x)\,\dd x
            +\frac{f(n)-f(0)}{2}+\int_{0}^{n}\psi(x)f'(x)\dd x
        \end{equation}
    其中 $\psi(x)=x-\lfloor x \rfloor-1/2=\{x\}-1/2$
\end{theorem}
证明如下:将区间$[0,n]$划分为长度为$1$的小区间$[k-1,k](k=1,2,\cdots,n)$,则
​\[
\int_{k-1}^{k}\psi(x)f'(x)\,\dd x
=\frac{f(k)+f(k-1)}{2}-\int_{k-1}^{k}f(x)\,\dd x
\]
​对$k$从1到$n$求和,于是
​\[
\int_{0}^{n}\psi(x)f'(x)\,\dd x
=\sum_{k=1}^{n}f(k)-\frac{f(n)-f(0)}{2}-\int_{0}^{n}f(x)\,\dd x
\]
这样就证明了Euler求和公式.注意到\autoref{EMSum_eq1} 右端积分比较复杂,一般情况下,仅仅估计其上界.\\
设函数 $f\in{C^1[0,+\infty)}$ ,再设函数
$\varphi(x)=\displaystyle{\int_{0}^{x}\psi(t)\,\dd t}$,
易知该函数是一个以1为周期的函数,且
$-1/8\leqslant\varphi(x)\leqslant 0$,因此
\begin{equation}
    \left|\int_{0}^{n}\psi(x)f'(x)\,\dd x\right|
    =\left|\int_{0}^{n}\varphi(x)D^2(f)\mathrm{d}x\right|
    \leqslant\frac{|f'(n)-f'(0)|}{8}
\end{equation}