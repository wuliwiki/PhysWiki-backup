% 格罗斯–皮塔耶夫斯基方程(综述)
% license CCBYSA3
% type Wiki

本文根据 CC-BY-SA 协议转载翻译自维基百科\href{https://en.wikipedia.org/wiki/Gross\%E2\%80\%93Pitaevskii_equation}{相关文章}。

格罗斯–皮塔耶夫斯基方程(Gross–Pitaevskii equation,简称 GPE,以尤金·P·格罗斯和列夫·彼得罗维奇·皮塔耶夫斯基命名)在哈特里–福克近似与伪势相互作用模型的基础上,用来描述由相同玻色子组成的量子体系的基态。

玻色–爱因斯坦凝聚是指一类玻色子气体,其所有粒子处于同一个量子态中,因此可以由相同的波函数描述。一个自由量子粒子可由单粒子薛定谔方程描述。对于现实气体中粒子间的相互作用,则需使用多体薛定谔方程加以考虑。在哈特里–福克近似中,整个由 $N$ 个玻色子组成的体系的总波函数 $\Psi$ 被认为是单粒子波函数 $\psi$ 的乘积:
$$
\Psi(\mathbf{r}_1, \mathbf{r}_2, \dots, \mathbf{r}_N) = \psi(\mathbf{r}_1)\psi(\mathbf{r}_2) \dots \psi(\mathbf{r}_N)~
$$
其中,$\mathbf{r}_i$ 是第 $i$ 个玻色子的坐标。如果气体中粒子之间的平均间距大于散射长度(即处于所谓的稀薄极限),那么可以用伪势来近似表示方程中真实的相互作用势能。在足够低的温度下,德布罗意波长远大于玻色子之间相互作用的作用范围,[3] 此时的散射过程可以很好地用 s 波散射(即偏波分析中的 $\ell = 0$,也称为硬球势)来近似描述。在这种情况下,该体系的伪势模型哈密顿量可以写为:
$$
H = \sum_{i=1}^{N} \left( -\frac{\hbar^2}{2m} \frac{\partial^2}{\partial \mathbf{r}_i^2} + V(\mathbf{r}_i) \right) + \sum_{i<j} \frac{4\pi \hbar^2 a_s}{m} \delta(\mathbf{r}_i - \mathbf{r}_j)~
$$

