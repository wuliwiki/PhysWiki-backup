% 定向
% 取向

本文翻译自Encyclopedia of Mathematics(数学百科)的\href{https://encyclopediaofmath.org/wiki/Volume_form}{Orientation}\footnote{Volume form. Encyclopedia of Mathematics. URL: http://encyclopediaofmath.org/index.php?title=Volume_form&oldid=32331.}词条.

\subsection{一般概念}

在传统数学中,一个\textbf{定向(orientation)}(或译作\textbf{取向})是指一种坐标系的等价划分,如果两个坐标系\textbf{正相关(positively related)}则是等价的.

对于有限实线性空间$\mathbb{R}^n$,一个坐标系由一组基确定,而两组基等价的条件是\textbf{转移矩阵}\upref{TransM}的行列式为正数.这个等价关系划分出两个等价类.对于复数的情况,即$\mathbb{C}^n$,任取其复基$\{e_1, \cdots, e_n\}$,则能导出实基$\{e_1, \cdots, e_n, \I e_1, \cdots, \I e_n\}$,从而可以将其视为$\mathbb{R}^{2n}$.任意两个复基分别导出的实基就是正相关的(也就是说,复结构定义了$\mathbb{R}^{2n}$上的定向).

在一条线、一个面或者更一般的实\textbf{仿射空间}\upref{AfSp}$E^n$上,一个坐标系由一个点(原点)和一组基给定,坐标系的变换由一个平移(改变原点)和一个基变换给定.坐标系的变换是正的,当且仅当基变换的转移矩阵行列式为正数.(举个例子:基向量的偶置换.)两个坐标系定义的定向相同,当且仅当其中一个可以连续地变为另一个,即存在由参数$t\in[0, 1]$给定的一族坐标系$O_t, e_t$关于$t$是连续的,则$O_0, e_0$到$O_1, e_1$的变换就是连续的.在$n-1$维超平面上的\textbf{反射(reflection)}映射能反转定向,即将一个定向中的坐标系映入另一个定向.

坐标系的等价类也能用不同的\textbf{几何体(geometric figures)}\footnote{译注:geometric figures指任何点、线、面等构成的集合,是几何空间的子集.}来定义.如果一个几何体$X$按照某种规则与一个坐标系关联,那么它的镜像在同一个规则下就与取向不同的另一个坐标系关联,于是$X$(以及给定的那个规则)就定义了一个定向.比如说,在仿射平面$E^2$上,一个给定了方向的圆就定义了一个定向,其中正定向里的代表坐标系就是原点在圆心处、重点在圆上的两个向量,而第一个向量到第二个向量沿着给定方向走的角度最小\footnote{译注:原文比这还绕口.总之,给定的方向就是规定逆时针或者顺时针之类的方向.}.在$E^3$中,可以用一根螺丝来作参考系\footnote{译注:这里原文改成参考系(frame)了,译者也很疑惑.},令第一个基向量沿着螺丝旋进的方向,而第二个和第三个基向量之间的旋转则沿着螺丝旋进时旋转的方向.一个基(或称参考系)也可以用著名的\textbf{右手定则}来定义,即用右手大拇指、食指和中指来确定\footnote{译注:即向量叉乘的记忆法则,食指指向前方,中指向掌心弯折,大拇指翘起,则食指方向叉乘中指方向,所得方向就是大拇指所指方向.}.

如果给定了$E^n$的一个定向,那么每一个半空间$E^n_+$就定义了边界面$E^{n-1}$上的一个定向.比如说,如果$E^n$的定向中后$n-1$个基向量都落在$E^{n-1}$中,而第一个基向量指入$E^n_+$,那么后$n-1$个基向量就定义了$E^{n-1}$上的一个定向.在$E^n$中,也可以用一个$n$维\textbf{单形}\footnote{译注:见\textbf{单纯形与复形}\upref{SimCom}.}($E^2$中的三角形,$E^3$中的三角锥)的顶点顺序来定义,即将原点选为第一个顶点,基向量则是从顶点顺次指向其它顶点的向量.同一组顶点的两个顺序属于统一定向,当且仅当它们之间是偶置换关系.一个给定了顶点顺序(至多差一个偶置换)的单形称为\textbf{定向的(oriented)}.一个$n$维定向单形的每一个$(n-1)$-面$\sigma^{n-1}$都有一个诱导定向:如果第一个顶点不在$\sigma^{n-1}$中,那么剩下的顶点顺序就被定义为$\sigma^{n-1}$的正向.

在一个\textbf{连通}\upref{Topo3}的\textbf{流形}\upref{Manif}$M$上,坐标系以\textbf{图册(atlas)}的形式出现:一组覆盖了$M$的图.如果各图之间的变换都是正的,那么称这些图构成的图册是定向的.对于一个微分流形来说,这意味着任何两个图之间的Jacobi矩阵处处为正.如果存在一个定向的图册,那么称$M$是可定向的.此时,全体定向图册的集合被分成两个等价类,两个图册等价当且仅当在它们俩中各任取一个图,这两图之间的变换都是正的.

















