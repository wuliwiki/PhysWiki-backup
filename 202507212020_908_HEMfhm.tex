% 赫尔曼·冯·亥姆霍兹(综述)
% license CCBYSA3
% type Wiki

本文根据 CC-BY-SA 协议转载翻译自维基百科\href{https://en.wikipedia.org/wiki/Hermann_von_Helmholtz}{相关文章}。

\begin{figure}[ht]
\centering
\includegraphics[width=6cm]{./figures/3999a7bd3b84e395.png}
\caption{} \label{fig_HEMfhm_3}
\end{figure}
赫尔曼·路德维希·费迪南德·冯·亥姆霍兹(Hermann Ludwig Ferdinand von Helmholtz,/ˈhɛlmhoʊlts/;德语:[ˈhɛʁman fɔn ˈhɛlmˌhɔlts];1821年8月31日-1894年9月8日,自1883年起冠以“冯(von)”的贵族头衔)是一位德国物理学家和医生,在多个科学领域作出了重要贡献,尤其以流体动力学稳定性理论而闻名\(^\text{[2]}\)。以他命名的亥姆霍兹协会是德国最大的科研机构联合体\(^\text{[3]}\)。

在生理学和心理学领域,亥姆霍兹以其关于眼睛的数学研究、视觉理论、空间视觉感知的观点、色觉研究、音调感觉与听觉感知理论,以及对感知生理学中经验主义的探讨而著称。在物理学中,他以能量守恒定律、电双层理论、电动力学、化学热力学,以及热力学的力学基础研究而闻名。尽管能量守恒原则的发展也归功于尤利乌斯·冯·迈尔、詹姆斯·焦耳和丹尼尔·伯努利等人,但亥姆霍兹被认为是第一个以最一般形式提出能量守恒原理的人\(^\text{[4]}\)。

作为哲学家,亥姆霍兹以其科学哲学、关于知觉规律与自然规律之间关系的见解、美学科学思想,以及关于科学的文明力量等理念而受到关注。到19世纪末,亥姆霍兹发展出一种广义的康德方法论,包括对知觉空间中可能取向的先验确定,这不仅激发了对康德的新解读\(^\text{[4]}\),也对现代后期的新康德主义哲学运动产生了重要影响\(^\text{[5]}\)。
\subsection{生平}
\subsubsection{早年经历}
亥姆霍兹出生于波茨坦,是当地文理中学校长费迪南德·亥姆霍兹的儿子。父亲曾学习古典语言学和哲学,是出版人兼哲学家伊曼努尔·赫尔曼·费希特(的密友。亥姆霍兹的研究受到约翰·戈特利布·费希特和伊曼努尔·康德哲学思想的影响,他尝试在诸如生理学等经验领域中追溯这些理论的体现。

年轻时,亥姆霍兹对自然科学兴趣浓厚,但父亲希望他学习医学。1842年,亥姆霍兹在柏林的腓特烈-威廉医学外科研究院获得医学博士学位,并在夏里特医院完成为期一年的实习\(^\text{[6]}\)(因为医学专业提供财政资助)。

虽然主要接受的是生理学训练,亥姆霍兹却在许多其他主题上都有著述,从理论物理学到地球年龄的估计,再到太阳系的起源等问题。
\subsubsection{大学任职}
1848年,亥姆霍兹的第一份学术职务是在柏林艺术学院担任解剖学教师\(^\text{[7]}\)。随后,他于1849年在普鲁士的哥尼斯堡大学被任命为生理学副教授。1855年,他接受了波恩大学的全职解剖学与生理学教授职位。然而,他在波恩并不特别满意,三年后调任至巴登的海德堡大学,担任生理学教授。1871年,他接受了最后一个大学职位,在柏林的腓特烈·威廉大学(今柏林洪堡大学)担任物理学教授。
\subsection{研究工作}
\subsubsection{亥姆霍兹}
力学

亥姆霍兹的第一项重要科学成就,是他于1847年撰写的一篇关于能量守恒的论文。这项工作是在其医学研究和哲学背景的语境中完成的。他对能量守恒的研究起初源于对肌肉代谢的探索,试图证明肌肉运动过程中并没有能量的损失,其动机在于说明肌肉的运动不需要任何“生命力”来驱动。这是对当时在德国生理学中占主导地位的自然哲学(Naturphilosophie)和生命力论等投机哲学传统的直接否定。他反对一些生命力论者提出的观点——即“生命力”可以无限地驱动一台机器\(^\text{[4]}\)。

在此前萨迪·卡诺、贝努瓦-保罗·埃米尔·克拉佩龙和詹姆斯·普雷斯科特·焦耳等人的研究基础上,亥姆霍兹提出了一个假设,认为力学、热、光、电和磁都是单一“力”的表现形式——用今天的术语来说,即“能量”。他在其著作《论力的守恒》(Über die Erhaltung der Kraft, 1847)中发表了这一理论\(^\text{[8]}\)。

在19世纪50至60年代,亥姆霍兹与威廉·汤姆森(后来的开尔文勋爵)及威廉·兰金基于前者的出版物,共同推广了“宇宙热寂”这一概念。

在流体动力学方面,亥姆霍兹也作出多项贡献,包括在无粘性流体中提出的“亥姆霍兹涡旋定理”。
\subsubsection{感官生理学}
\begin{figure}[ht]
\centering
\includegraphics[width=6cm]{./figures/287cdf893acb7c66.png}
\caption{亥姆霍兹的多音调汽笛,格拉斯哥亨特博物馆。} \label{fig_HEMfhm_1}
\end{figure}
亥姆霍兹是人类视觉和听觉科学研究的先驱之一。他受到心理物理学的启发,致力于探索可测量的物理刺激与其对应的人类感知之间的关系。例如,通过改变声波的振幅可以使声音听起来更响或更轻,但声音压力振幅的线性增加并不会引起听感上的线性变化。为了使感知到的响度以线性方式变化,声音的物理强度必须以指数方式增加,这一事实如今已广泛应用于电子设备的音量控制之中。亥姆霍兹在实验研究物理能量(物理学)与其感知(心理学)之间关系方面开辟了道路,目标是建立“心理物理定律”。

亥姆霍兹的感官生理学研究为其学生威廉·冯特(Wilhelm Wundt)的工作奠定了基础,冯特被认为是实验心理学的奠基人之一。与亥姆霍兹相比,冯特更明确地将自己的研究描述为经验哲学的一种形式,并将其作为对“心灵”这一独立存在的研究。早年在反对自然哲学时,亥姆霍兹强调唯物主义的重要性,更多地关注“心灵”与身体之间的统一关系\(^\text{[9]}\)。
\begin{figure}[ht]
\centering
\includegraphics[width=6cm]{./figures/801f9de8c71fbaf6.png}
\caption{1848年的亥姆霍兹} \label{fig_HEMfhm_2}
\end{figure}
\subsubsection{眼科光学}
1851年,亥姆霍兹发明了检眼镜,彻底革新了眼科学领域。检眼镜是一种用来检查人眼内部的仪器,这一发明使他一夜成名。那时,亥姆霍兹的研究兴趣日益集中在感官生理学上。他的主要著作《生理光学手册》(德语原名 Handbuch der Physiologischen Optik,英文常译为 Handbook of Physiological Optics 或 Treatise on Physiological Optics,第三卷的英文译本见[此处](https://archive.org/details/physiologicalopt03helmrich)),提出了关于深度知觉、颜色视觉和运动知觉的经验性理论,并在19世纪后半叶成为该领域的基础性参考著作。在1867年出版的第三卷也是最后一卷中,亥姆霍兹阐述了“无意识推理”在知觉中的重要性。该手册首次被翻译成英文是在1924–1925年,由詹姆斯·P·C·索思奥尔代表美国光学学会主编。
他的调节理论(眼睛如何聚焦)在20世纪最后十年之前一直没有受到挑战。

在此后的几十年中,亥姆霍兹不断修订和完善这部手册,多次出版新版。他与持有相反观点的埃瓦尔德·黑林在空间视觉与色觉方面存在激烈争论,这场争论也使得19世纪下半叶的生理学界分裂成两个阵营。
\subsubsection{神经生理学}
1849年,亥姆霍兹在柯尼斯堡工作期间,测量了神经纤维中信号传导的速度。当时大多数人认为神经信号的传播快得无法测量。亥姆霍兹使用了刚解剖出来的青蛙坐骨神经和与之相连的小腿肌肉,并使用检流计作为灵敏的计时装置:他在指针上安装了一面小镜子,用来反射光束到房间对面的刻度尺上,从而极大提高了灵敏度。亥姆霍兹报告称\(^\text{[11][12]}\),神经信号传导速度在 24.6 至 38.4 米/秒之间\(^\text{[10]}\)。
\subsubsection{声学与美学}
\begin{figure}[ht]
\centering
\includegraphics[width=6cm]{./figures/afaa5883003466f4.png}
\caption{冯·亥姆霍兹的最后一张照片,拍摄于他病倒前三天。} \label{fig_HEMfhm_4}
\end{figure}
1863年,亥姆霍兹出版了《音调的感觉》,再次展现了他对感知物理学的浓厚兴趣。这本书对20世纪的音乐学家产生了深远影响。为了识别复杂声音中包含的多个纯正正弦波成分的不同频率或音高,亥姆霍兹发明了亥姆霍兹共鸣器\(^\text{[13]}\)。

亥姆霍兹发现,不同组合的共鸣器可以模拟元音的声音:亚历山大·格拉汉姆·贝尔对此尤其感兴趣,但由于他不会阅读德语,误解了亥姆霍兹的图示,认为亥姆霍兹已通过电线传输多个频率——这将实现电报信号的多路复用。实际上,电能仅用于维持共鸣器的振动。贝尔未能重现他以为亥姆霍兹已完成的实验,但他后来表示,如果当时能读懂德文,他就不会根据“谐波电报”原理发明电话了\(^\text{[14][15][16][17]}\)。

亥姆霍兹于1881年的肖像,由路德维希·克瑙斯绘制。

亚历山大·J·埃利斯翻译的英文版首次出版于1875年(根据1870年第三版德文原作);埃利斯根据1877年第四版德文原作的第二版英译本则于1885年出版;1895年和1912年的第三、第四版英文版为第二版的重印\(^\text{[18]}\)。
\subsubsection{电磁学}
\begin{figure}[ht]
\centering
\includegraphics[width=6cm]{./figures/685163aa0f09d408.png}
\caption{赫尔姆霍兹共鸣器(i)及其仪器设备} \label{fig_HEMfhm_5}
\end{figure}