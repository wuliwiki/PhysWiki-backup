% 布洛赫理论
% 晶格|波函数|薛定谔方程|周期函数|bloch|布洛赫

\begin{issues}
\issueDraft
\end{issues}

\pentry{定态薛定谔方程\upref{SchEq}}

在固体物理中,我们研究原子按一定规则排布而成的理想晶格,研究电子在一个具有晶格周期性的等效势场 $V(\bvec r)$ 中的运动。根据定态薛定谔方程,我们有
\begin{equation}\label{eq_Bloch_4}
H\psi=\qty[-\frac{\hbar^2}{2m}\nabla^2+V(\bvec r)] \psi = E\psi~,
\end{equation}
势场 $V(\bvec r)$ 满足周期性
\begin{equation}
V(\bvec r)=V(\bvec r+\bvec R_n)~,
\end{equation}
其中 $\bvec R_n$ 为任意晶格矢量。

为了研究周期性势场下定态薛定谔方程的解,我们可以从最简单的一维薛定谔方程出发。

\subsection{Bloch theory}
\footnote{参考\cite{GriffQ}}\textbf{布洛赫(Bloch)}理论也叫 Floquet 理论,是由对称性得到的严格结论。它的内容可以表述如下:

\begin{theorem}{Bloch theory I}
周期势场$V(\bvec r) = V(\bvec k + \bvec R)$ 下的本征态波函数满足
\begin{equation}\label{eq_Bloch_1}
\psi(\bvec r + \bvec R) = \E^{\I \bvec r \vdot \bvec R}\psi(\bvec r)~,
\end{equation}
其中 $\bvec k$ 的量纲是$[L^{-1}]$,$\bvec R$是晶格矢量,简称为\textbf{格矢}。后面我们将会看到,$\bvec k$ 具有电子动量的物理意义。 
\end{theorem} 

也就是说,将空间坐标$\bvec r$平移格矢$\bvec R$,平移前后的波函数只相差一个跟$\bvec R$有关的相位因子,不改变波函数本身的实际性质。

$\psi (\bvec r)$称为\textbf{布洛赫函数(Bloch function)},晶体中用布洛赫函数来描述的电子称为\textbf{布洛赫电子},与真空中的自由电子区分。

布洛赫定理的另一种等价描述为:

\begin{theorem}{Bloch theory II}
周期势场中的本征态函数$\psi(\bvec r)$ 可写成一个\textbf{平面波}和\textbf{周期性函数}的乘积:
\begin{equation}\label{eq_Bloch_2}
\psi(\bvec r) = \E^{\I \bvec k \vdot \bvec r} u(\bvec r)
\end{equation}
$u( \bvec r)$ 是布洛赫函数的\textbf{周期部分},周期为$\bvec R$,$\E^{\I \bvec k \vdot \bvec r}$ 是平面波部分。
\end{theorem}

从Bloch定理的等价描述中我们可以初步看到,布洛赫函数是真空中自由电子的平面波在周期性势场的调制的结果。

\begin{exercise}{}
请根据定理证明两种描述的等价性。答案在下面。
\end{exercise}
由\autoref{eq_Bloch_1} 到\autoref{eq_Bloch_2} ,等式两边同乘$\E^{-\I \bvec k \vdot (\bvec R + \bvec r)}$:
\begin{equation}
\E^{-\I \bvec k \vdot (\bvec R + \bvec r)}\psi(\bvec r + \bvec R) = \E^{-\I \bvec k \vdot \bvec r}\psi(\bvec r)~,
\end{equation}
观察左右两式,可以看到明显的周期函数:
\begin{equation}
u(\bvec R + \bvec r) = u(\bvec r),~\quad u(\bvec r) = \E^{-\I \bvec k \vdot \bvec r}\psi(\bvec r)~.
\end{equation}
整理一下,也即
\begin{equation}
\psi(\bvec r) = \E^{\I \bvec k \vdot \bvec r} u(\bvec r)
\end{equation}
证毕。由\autoref{eq_Bloch_2} 到\autoref{eq_Bloch_1} :
\begin{equation}
\psi(\bvec r + \bvec R) = \E^{\I \bvec k \vdot (\bvec R + \bvec r)} u(\bvec r + \bvec R)
= \E^{\I \bvec k \vdot \bvec R } \E^{\I \bvec k \vdot  \bvec r}u(\bvec r )= \E^{\I \bvec k \vdot \bvec R } \psi(\bvec r).
\end{equation}
证毕。

上面我们仅利用了函数的周期性性质。下面用量子力学的语言证明Bloch定理,即周期性空间平移的效果仅是给波函数添加phase shift, 不改变物理实质。

定义平移算符$\bvec T_{\bvec R}$,将$\bvec T_{\bvec R}$作用到任何函数$f(\bvec r)$上的作用是:
\begin{equation}\
\bvec T_{\bvec R} f(\bvec r) = f(\bvec R + \bvec r)
\end{equation}
也即将函数的坐标平移矢量$\bvec R$,这里$\bvec R = n_1 \bvec a_1 + n_2 \bvec a_2 + n_3 \bvec a_3 $ 是任意格矢。我们可以想到,如果对函数 $f(\bvec r + \bvec R)$ 再平移一次空间坐标$\bvec R'$,则应得到$f(\bvec r + \bvec R + \bvec R')$,其效果等价于对 $f(\bvec r)$ 平移坐标$\bvec R + \bvec R'$,也即:
\begin{equation}
\bvec T_{\bvec R'}\bvec T_{\bvec R} f(\bvec r) = \bvec T_{\bvec R + \bvec R'}f(\bvec r )
\end{equation}

由

那么 $[D,H] = 0$。 所以存在能量和 $D$ 的共同本征态, 马上就得到\autoref{eq_Bloch_1} 。

波函数也可以记为
\begin{equation}
\psi(x) = \E^{\I K x} u(x)~,
\end{equation}
其中 $u(x)$ 是一个周期为 $a$ 的函数。 也就是波函数是一个振幅受周期性调制的平面波。

如果我们施加循环边界条件($N$ 是晶体一个方向的原子数, 阿伏伽德罗常数数量级)
\begin{equation}
\psi(x+Na) = \psi(x)~,
\end{equation}
得
\begin{equation}
K = \frac{2\pi}{a} \frac{n}{N} \qquad (n \in \mathbb Z)~,
\end{equation}

一个例子见一维 delta 势能晶格\upref{DelCry}。

\subsection{三维薛定谔方程}

\cite{黄昆}\cite{Bransden}布洛赫(Bloch)波函数定义为
\begin{equation}\label{eq_Bloch_3}
\phi(\bvec r) = \E^{\I \bvec k \vdot \bvec r} u(\bvec r)~,
\end{equation}
其中 $u(\bvec r)$ 具有与晶格同样的周期性。
\begin{equation}
u(\bvec r)=u(\bvec r+\bvec R_n)~,
\end{equation}
$\bvec R_n$ 为晶格矢量。即在任何平移 $\bvec R_n$ 的操作下势场 $V(\bvec r)$ 都是不变的。我们引入描述晶格平移对称性的算符 $T_1,T_2,T_3$,它们的定义是
\begin{equation}
T_\alpha u(\bvec r)= u(\bvec r+\bvec a_\alpha),\alpha=1,2,3~,
\end{equation}
其中 $\bvec a_\alpha,\alpha=1,2,3$ 是晶格的三个基矢。它们是相互对易的,而且容易证明它们和哈密顿算符 $H$ 也相互对易。下面我们要做的就是找出 $H,T_1,T_2,T_3$ 的共同本征态,用以描述晶格中的电子。设
\begin{equation}
\begin{aligned}
&H\psi=E\psi,\\
&T_\alpha \psi = \lambda_\alpha \psi, \alpha=1,2,3~.
\end{aligned}
\end{equation}
\addTODO{需要增加原胞、布拉伐格子、倒格子相关的词条}
设晶格在三个方向上的原胞数量分别为 $N_1,N_2,N_3$,那么可以引入晶格的周期性边界条件:
\begin{equation}
\psi(\bvec r)=\psi(\bvec r+N_\alpha \bvec a_\alpha),\alpha=1,2,3
\end{equation}
可以得出 $\lambda_\alpha$ 具有下列形式
\begin{equation}
\lambda_\alpha=e^{ \dfrac{2\pi i l_\alpha}{N_\alpha}}~,
\end{equation}
其中 $l_\alpha$ 为整数。

如果引入倒格子矢量 $\bvec b_1,\bvec b_2,\bvec b_3$,满足 $\bvec a_i \cdot \bvec b_j = 2\pi \delta_{ij}$,那么
\begin{equation}
\begin{aligned}
&\lambda_\alpha = e^{i \bvec k \cdot \bvec a_\alpha}\\
&\bvec k=\sum_\alpha\frac{l_\alpha}{N_\alpha} \bvec b_\alpha~.
\end{aligned}
\end{equation}
我们可以将 \autoref{eq_Bloch_2} 三维晶格的布洛赫定理写成以下形式
\begin{equation}
\begin{aligned}
\psi\qty(\bvec r+\sum_\alpha m_\alpha\bvec a_\alpha) &= T_1^{m_1} T_2^{m_2} T_3^{m_3} \psi(\bvec r)\\
&=e^{i\bvec k\cdot \qty(\sum_\alpha m_\alpha\bvec a_\alpha)}\psi(\bvec r)~.
\end{aligned}
\end{equation}
其中 $\bvec k$ 称为简约波矢,它对应于平移算符操作本征值的量子数 $l_1,l_2,l_3$。注意到当 $l_\alpha$ 增加 $N_\alpha$ 时 $\lambda_\alpha$ 没有发生变化,所以我们将 $l_\alpha$ 限制在 $0\cdots N_\alpha-1$ 因此得到简约波矢与量子数的一一对应。这相当于把 $\bvec k$ 限制在 $\bvec k$ 空间中由 $\bvec b_1,\bvec b_2,\bvec b_3$ 构成的原胞中,$\bvec k$ 的允许值个数为 $N_1N_2N_3$,也就是晶体的原胞总数。另一种更方便的方式,是将 $\bvec k$ 限制在 $\bvec k$ 空间的维格纳塞茨原胞中,这个区域被称为第一布里渊区。