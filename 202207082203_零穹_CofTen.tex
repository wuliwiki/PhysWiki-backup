% 张量的坐标
% 张量坐标|坐标转换关系

\pentry{张量积\upref{TsrPrd}}
进行张量分析往往需要选择空间的基底,并用坐标去刻画张量.

在矢量空间 $V$ 和 $V^*$ 中选择相互对偶的基底(\autoref{DualSp_sub1}~\upref{DualSp})
\begin{equation}
V=\langle e_1,\cdots ,e_n\rangle,\quad V^*=\langle e^1,\cdots,e^n\rangle
\end{equation}
这里,按照惯例,空间 $V$ 中的基底指标排列在下方,$V^*$ 中则在上方.而在对应的坐标中,指标的排列则是对立的,即若 $x\in V,f\in V^*$ ,则 $x=\sum_{i}x^i e_i,f=\sum_{i}f_ie^i$.

我们知道, $V$ 和 $V^{**}$ 之间存在自然同构,这使得可以把 $\varphi\in V^{**}$ 和某个矢量 $x_{\varphi}\in V$ 等同起来.即 $x_{\varphi}(f)$ 等同于 $\varphi(f)$,其中 $f\in V^*$:
\begin{equation}
x_{\varphi}(f)\equiv\varphi(f)=f(x_{\varphi})
\end{equation}
上面 $\varphi(f)=f(x_{\varphi})$ 就是 $V$ 与 $V^{**}$ 之间建立的自然同构.

为了表示这种等同,可记
\begin{equation}
f(x)=(f,x)
\end{equation}
该记法按时这是一个内积,但来自于不同的空间,并且对于每个变量都是线性的.当 $f$ 固定时,这是 $V$ 上的一个线性函数,而当 $x$ 固定时,就是 $V^{*}$ 上的一个线性函数.
