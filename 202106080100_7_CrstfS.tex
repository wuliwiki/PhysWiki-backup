% Christoffel符号
% keys 克里斯托费尔|克氏符|克氏|Christoffel|测地线|geodesic|广义相对论|relativity

\pentry{仿射联络(流形)\upref{affcon}, 爱因斯坦求和约定\upref{EinSum}}

仿射联络的定义是高度抽象的,并不涉及具体的运算.在物理学中,计算出某个参考系中各切向量的坐标分量是必须的,因为这些坐标分量才是观察者能看到、测量到的直接现象.Christoffel符号就是方便进行运算的一个概念.

本词条中默认$(M, \nabla)$为一个带仿射联络的实流形.

\subsection{联络形式}

对于任意$p\in M$,取$p$的一个邻域$U\subseteq M$,使得存在一组光滑向量场$\{\uvec{e}_i\}$构成$C^{\infty}(U)$上的一组基.这就是说,$U$上的每个光滑向量场都可以表示为$f^i\uvec{e}_i$的形式,其中各$f^i$是$U$上的光滑函数.

对于任意$X\in\mathfrak{X}(U)$,我们知道$\nabla_X\uvec{e}_i$也是一个光滑向量场,因此存在一组\textbf{光滑函数}$W^j_i(X)$,使得$\nabla_x\uvec{e}_i=W^j_i(X)\uvec{e}_j$.

每个光滑函数$W^j_i(X)$都由$X$唯一确定,而且由$\nabla$的性质知,对于任意光滑函数$f$和$g$,光滑向量场$X, Y$,都有$W^j_i(fX+gY)=fW^j_i(X)+gW^j_i(Y)$.也就是说,$W^j_i$本身是$X$的\textbf{线性函数},也就是$U$上的一个$1$-形式.

我们将以上讨论所得出的$W^j_i$称为$\nabla$的\textbf{联络形式(connection form)}.

\subsubsection{联络形式作为坐标分量}

在线性代数中我们知道,一个向量(或者任何非零阶的张量)不能简单地理解为一组坐标数字,因为它的坐标具体取值取决于基的选择.而以上讨论的$W^j_i$虽然不是数字,却也有类似的性质,即“$W^j_i$具体是哪个$1$-形式,取决于选择哪一组$\{\uvec{e}_i\}$作为$\mathfrak{X}(M)$的基”.换句话说,$W^j_i(X)$具体是哪个函数,不仅取决于$X$,也取决于$\{\uvec{e}_i\}$的选择.

因此,尽管$W_i^j$是微分形式,我们也把它看成一种坐标分量,即联络$\nabla$在基$\{\uvec{e}_i\}$下的\textbf{局部}坐标分量.之所以强调“局部”,是因为我们只能保证在$\mathfrak{X}(U)$中能找到一组基,而在整个$\mathfrak{X}(M)$中则不一定存在基\footnote{比如考虑$M=S^2$,即球面,那么球面上任何一个连续向量场总存在零点,因此对于任意\textbf{两个}光滑向量场$\{\uvec{e}_i\}$,在$\uvec{e}_1$的零点$p$处,仅靠$\uvec{e}_2$是无法张成整个切空间$T_pM$的,因此只要一个光滑向量场在$p$点的zhi}.

















