% 光电离时间延迟

\pentry{量子散射的延迟\upref{tDelay}, 含时微扰理论\upref{TDPT}}

本文使用原子单位制\upref{AU}. 例如一个有限深势阱(短程势)中有一个束缚态, 被一个电场波包电离之后, 光电子波包逃出势阱. 那么光电子波包的延迟是多少呢? 我们以下使用一阶含时微扰理论\upref{TDPT} 来分析. 我们假设势阱在坐标原点, 且 $t = 0$ 时电场波包的中心刚好到达原点.

初态和末态能量分别为 $E_0, E$, 令 $\omega = E - E_0$, 不含时的末态记为 $\psi_E(x)$, 并令其为满足边界条件
\begin{equation}
\psi_E(x \to +\infty) \propto \E^{\I kx}
\end{equation}
其中 $A(k)$ 是实函数. 那么一阶微扰系数为
\begin{equation}
c_E(t) = -\I  \mel{\psi_E}{H'}{\psi_0} \tilde f(-\omega)
\end{equation}
电离波包可以由末态展开:
\begin{equation}
\psi(x, t) = \int c_E(t) \psi_E(x)\E^{-\I E t} \dd{E}
\end{equation}
那么根据\autoref{tDelay_eq2}~\upref{tDelay} 延迟为
\begin{equation}
\tau = \pdv{E} \arg [c_E]
\end{equation}
对于没有 chirp 的电场波包, $\tilde f(-\omega)$ 相位恒定(想象一个高斯波包乘以正弦函数的傅里叶变换). 所以随 $E$ 变化的相位只有矩阵元
\begin{equation}\label{HeAna2_eq1}
\tau = \pdv{E} \arg \mel{\psi_E}{H'}{\psi_0}
\end{equation}

\addTODO{如果是长程库仑势能, 延迟就会取决于距离而不收敛. 但动量却收敛, 所以使用 streaking 仍然会获得一个延迟, 但这个延迟和上面的是两码事, 然而 Pazourek 仍然使用\autoref{HeAna2_eq1}, 这里面还有更深的奥妙……}
