% 保罗·狄拉克(综述)
% license CCBYSA3
% type Wiki

本文根据 CC-BY-SA 协议转载翻译自维基百科\href{https://en.wikipedia.org/wiki/Paul_Dirac}{相关文章}。


\begin{figure}[ht]
\centering
\includegraphics[width=6cm]{./figures/887a811793832fb1.png}
\caption{1933年的狄拉克} \label{fig_DLK1_1}
\end{figure}
保罗·阿德里安·莫里斯·狄拉克(Paul Adrien Maurice Dirac,/dɪˈræk/;1902年8月8日 – 1984年10月20日)是英国的数学物理学家和理论物理学家,被认为是量子力学的奠基人之一。[6][7] 狄拉克为量子电动力学和量子场论的基础奠定了基础。[8][9][10][11] 他曾担任剑桥大学卢卡斯数学教授、佛罗里达州立大学物理学教授,并于1933年获得诺贝尔物理学奖。

狄拉克于1921年毕业于布里斯托大学,获得电气工程学的一级荣誉理学士学位,1923年获得数学的一级荣誉文学学士学位。[12] 随后,他于1926年从剑桥大学获得物理学博士学位,撰写了首篇关于量子力学的论文。[13]

狄拉克对量子力学和量子电动力学的早期发展作出了基础性贡献,并创造了后者的术语。[10] 其中,他在1928年提出了狄拉克方程,该方程描述了费米子的行为,并预测了反物质的存在,[14] 这一方程被认为是物理学中最重要的方程之一,[8] 并被一些物理学家视为“现代物理学的真正种子”。[15] 他在1931年撰写了一篇著名论文,[16] 进一步预测了反物质的存在。[17][18][14] 狄拉克与厄尔温·薛定谔共同分享了1933年诺贝尔物理学奖,以表彰他们“发现了原子理论的新生产性形式”。[19] 他是最年轻的诺贝尔理论物理学奖获得者,直到1957年T·D·李获奖。[20] 狄拉克还为广义相对论与量子力学的和解做出了巨大贡献。他的1930年专著《量子力学原理》是量子力学最具影响力的经典著作之一。[21]

狄拉克的贡献不仅限于量子力学。他还为“管道合金”项目作出了贡献,这是英国在二战期间研究和建造原子弹的计划。[22][23] 狄拉克对铀浓缩过程和气体离心机做出了基础性贡献,[24][25][26][23] 他的工作被认为是“可能是离心机技术中最重要的理论成果”。[27] 他还对宇宙学做出了贡献,提出了“大数假说”。[28][29][30][31] 狄拉克还在弦理论诞生之前预见到了弦理论,提出了如狄拉克膜、狄拉克-伯恩-因费尔德作用等工作,这些贡献对现代弦理论和规范理论至关重要。[32][33][34][35]

狄拉克被朋友和同事视为个性独特。在1926年给保罗·埃伦费斯特的信中,阿尔伯特·爱因斯坦写道:“我在费劲地研究狄拉克。这种在天才与疯狂之间的摇摆实在可怕。”在另一封关于康普顿效应的信中,他写道:“我完全不理解狄拉克的细节。”[36] 1987年,阿卜杜斯·萨拉姆宣称:“狄拉克无疑是这个世纪最伟大的物理学家之一……除爱因斯坦外,没有人能在如此短的时间内,对本世纪物理学的发展产生如此决定性的影响。”[37] 1995年,斯蒂芬·霍金表示:“狄拉克比任何人都做得更多,除了爱因斯坦之外,他推动了物理学的发展并改变了我们对宇宙的认识。”[38] 安东尼诺·齐基奇认为狄拉克对现代物理学的影响超过了爱因斯坦,[15] 斯坦利·德塞尔则评论道:“我们都站在狄拉克的肩膀上。”[39] 狄拉克被广泛认为与艾萨克·牛顿、詹姆斯·克拉克·麦克斯韦和爱因斯坦相提并论。[40][41][42]
\subsection{个人生活}  
\subsubsection{早年}
\begin{figure}[ht]
\centering
\includegraphics[width=6cm]{./figures/3f71a2576aff2af0.png}
\caption{克拉拉·埃瓦尔德(Clara Ewald)所绘的保罗·狄拉克肖像(1939年)} \label{fig_DLK1_2}
\end{figure}
保罗·阿德里安·莫里斯·狄拉克(Paul Adrien Maurice Dirac)于1902年8月8日出生在英国布里斯托尔市父母的家中,并在该市的比肖普斯顿区长大。他的父亲,查尔斯·阿德里安·拉迪斯拉斯·狄拉克(Charles Adrien Ladislas Dirac),是来自瑞士圣莫里茨的移民,具有法国血统,在布里斯托尔担任法语教师。他的母亲,弗洛伦斯·汉娜·狄拉克(Florence Hannah Dirac,娘家姓霍尔滕),出生在康沃尔的利斯基尔,是一个康沃尔卫理公会家庭的成员。她是由其父亲命名的,父亲是一位海军船长,在克里米亚战争期间曾与弗洛伦斯·南丁格尔(Florence Nightingale)相识。她在年轻时搬到布里斯托尔,并在那里担任布里斯托尔中央图书馆的图书管理员;尽管如此,她依然认为自己的身份是康沃尔人,而不是英国人。保罗有一个妹妹,比阿特丽斯·伊莎贝尔·玛格丽特(Béatrice Isabelle Marguerite),昵称贝蒂,还有一个哥哥,雷金纳德·查尔斯·费利克斯(Reginald Charles Félix),昵称费利克斯,后者在1925年3月自杀。狄拉克后来回忆道:“我的父母非常痛苦,我从未知道他们如此在意……我从来不知道父母应该如此关心孩子,但从那时起我知道了。” 

查尔斯和孩子们在1919年10月22日正式成为瑞士国籍,之前他们一直是瑞士公民。狄拉克的父亲为人严格且专制,尽管他不赞成体罚。狄拉克与父亲的关系紧张,甚至在父亲去世后,狄拉克写道:“我现在感到更加自由,我是我自己的人。”查尔斯强迫孩子们只能用法语与他交谈,以便他们能够学会这门语言。当狄拉克发现自己无法用法语表达想法时,他选择保持沉默。
\subsubsection{教育背景}  
狄拉克首先在比肖普路小学(Bishop Road Primary School)接受教育,随后进入了全男生的商人冒险家技术学院(Merchant Venturers' Technical College,后来的科瑟姆学校,Cotham School),他的父亲曾在该校担任法语教师。该校隶属于布里斯托大学,校区和师资共享。学校强调技术学科,如砖砌、制鞋和金属加工,以及现代语言。这在当时英国的中等教育仍以古典学科为主的背景下,显得非常与众不同,狄拉克后来对此表示了感激。

狄拉克凭借布里斯托大学的奖学金在该校工程学院学习电气工程,该学院与商人冒险家技术学院共用设施。在完成学位前不久,即1921年,他参加了剑桥大学圣约翰学院的入学考试。他顺利通过并获得了70英镑的奖学金,但这笔钱不足以支付在剑桥的生活和学费。尽管他以一等荣誉学位毕业于电气工程学,但由于战后经济萧条,他未能找到工程师工作。于是,他接受了一个在布里斯托大学攻读数学文学学士学位的机会,并且免去了第一年的课程,因为他已获得工程学学位。在彼得·弗雷泽的影响下,狄拉克称他为最好的数学教师,并对投影几何学产生了浓厚兴趣,开始将其应用于明可夫斯基发展出的相对论的几何版本。

1923年,狄拉克再次以一等荣誉学位毕业,并获得了科学与工业研究部门提供的140英镑奖学金。再加上圣约翰学院提供的70英镑奖学金,这使得他能够在剑桥大学生活。在剑桥,狄拉克继续深入研究广义相对论(这一兴趣源自他在布里斯托大学时期的学习)和新兴的量子物理学领域,师从拉尔夫·福勒(Ralph Fowler)。1925至1928年,他获得了1851年皇家博览会研究奖学金。1926年6月,他完成了第一篇量子力学博士论文,这是世界上首篇提交的量子力学博士论文。随后,他在哥本哈根和哥廷根继续进行研究。1929年春季,他曾在威斯康星大学麦迪逊分校担任访问教授。
\subsubsection{家庭}
\begin{figure}[ht]
\centering
\includegraphics[width=7cm]{./figures/e3efba3af8241e4f.png}
\caption{保罗和曼西·狄拉克(Paul and Manci Dirac)于1963年7月在哥本哈根} \label{fig_DLK1_3}
\end{figure}
1937年,狄拉克与玛吉特·维格纳(Margit Wigner)结婚,玛吉特是物理学家尤金·维格纳(Eugene Wigner)的妹妹,并且是离婚妇女。狄拉克将玛吉特的两个孩子——朱迪丝和加布里埃尔——视如己出抚养。保罗和玛吉特·狄拉克还共同育有两个女儿,玛丽·伊丽莎白和弗洛伦斯·莫妮卡。

玛吉特,外号曼奇(Manci),于1934年从她的故乡匈牙利到新泽西州的普林斯顿探望哥哥,在安纳克斯餐厅的晚餐时遇见了“坐在隔壁桌的孤独男子”。这位韩国物理学家金裕成(Y. S. Kim)曾见过并受狄拉克影响,他在回忆中也提到:“对于物理学界来说,玛吉特照顾我们的尊敬的保罗·A·M·狄拉克是非常幸运的。在1939到1946年间,狄拉克发表了十一篇论文。狄拉克能够保持正常的研究生产力,完全是因为玛吉特负责了其他一切事务。”
\subsection{个性} 
狄拉克以其精确和沉默寡言的性格在同事中闻名。他在剑桥的同事们开玩笑地定义了一个单位——“狄拉克”(dirac),代表每小时说一个单词。[77] 当尼尔斯·玻尔抱怨自己写科学文章时不知道如何结束一个句子时,狄拉克回答道:“我在学校时学到的规则是,永远不要在不知道句子的结尾时开始写它。”[78] 他还批评物理学家J.罗伯特·奥本海默对诗歌的兴趣:“科学的目标是以更简单的方式让复杂的事情变得可以理解;而诗歌的目标是以无法理解的方式表达简单的事情。这两者是不相容的。”[79]  

狄拉克在研究生时期的日记中写道,他专注于自己的研究,只在星期天停下来独自散步。[80]  

一个轶事提到,1929年8月,维尔纳·海森堡和狄拉克乘坐远洋班轮前往日本参加一个会议。两人都二十多岁,未婚,性格截然不同。海森堡是个喜欢调情和跳舞的风流人物,而狄拉克——如其传记作者格雷厄姆·法梅洛(Graham Farmelo)所称,是一位“爱德华时代的怪才”——如果被迫参与任何形式的社交或闲聊,都会感到痛苦。狄拉克问海森堡:“你为什么跳舞?”海森堡回答:“当有漂亮的女孩时,这是一种乐趣。”狄拉克思索了一会儿,然后突然说:“但,海森堡,你怎么提前知道女孩是漂亮的?”[81]  

玛吉特·狄拉克在20世纪60年代告诉乔治·伽莫夫和安东·卡普里,她的丈夫曾对一位来访者说:“允许我介绍一下维格纳的妹妹,她现在是我的妻子。”[82][83]  

另一个关于狄拉克的故事是,他第一次在一次会议上见到年轻的理查德·费曼时,沉默了很久后说道:“我有一个方程。你也有一个吗?”[84]  

在一次会议上发表演讲后,一位同事举手说:“我不理解黑板右上角的那个方程。”经过一段长时间的沉默,主持人问狄拉克是否想回答这个问题,狄拉克回答道:“那不是一个问题,而是一个评论。”[85][86]  

狄拉克还以个人的谦逊著称。他将自己首次写下的关于量子力学算符时间演化的方程称为“海森堡运动方程”。大多数物理学家将描述半整数自旋粒子的统计称为费米-狄拉克统计,而整数自旋粒子的统计称为玻色-爱因斯坦统计。然而,狄拉克晚年在授课时始终坚持称前者为“费米统计”,后者为“玻色统计”,他解释说这是出于“对称性”的原因。[87]
\subsubsection{对宗教的看法}  
海森堡回忆起1927年索尔维会议期间,沃尔夫冈·泡利、海森堡和狄拉克关于爱因斯坦和普朗克宗教观点的讨论。狄拉克的发言是批评宗教的政治目的,当海森堡后来向博尔提及此事时,博尔认为他的解释非常清晰。[88] 海森堡想象狄拉克可能会说:

“我不知道为什么我们要讨论宗教。如果我们诚实——科学家必须诚实——我们必须承认,宗教是一些虚假的主张,没有任何现实基础。上帝的观念是人类想象的产物。很容易理解,为什么原始人会在对自然力量的恐惧和颤抖中将这些力量人格化。可是如今,当我们了解了许多自然过程时,我们不再需要这些解决方案。我无法理解全能的上帝假设到底能帮助我们什么。我看到的是,这种假设导致了诸如上帝为何允许如此多的苦难和不公,贫富之间的剥削,以及所有他本可以阻止的恐怖事件这样的无益问题。如果宗教依然在被传授,那并非因为它的思想仍能说服我们,而是因为某些人想要让下层阶级安静。安静的人比那些喧嚣和不满的人更容易管理。他们也更容易被剥削。宗教是一种鸦片,让一个国家沉浸在美好的幻想中,从而忘记正在对人民施加的种种不公。因此,国家和教会这两大政治力量有着密切的联系。它们都需要那种幻觉——即上帝是仁慈的,奖励所有未反抗不公、默默履行职责、不抱怨的人——无论是在天国还是在地球上。这正是为什么那种诚实的主张——上帝只是人类想象的产物——会被视为最严重的罪孽。”[89]  

海森堡的看法则较为宽容。泡利在一开始有所评论后保持沉默,但当他被要求发表看法时,说:“好吧,我们的朋友狄拉克有一种宗教,而它的指导原则是‘没有上帝,狄拉克是他的先知。’”所有人,包括狄拉克,都哄堂大笑。[90][91]

在晚年,狄拉克在1963年5月的《科学美国人》上发表了一篇提到上帝的文章,他写道:

“似乎自然界的一个基本特征是,基本的物理法则可以用一种非常美丽和强大的数学理论来描述,而要理解这些法则,需要相当高水平的数学能力。你可能会想:为什么自然是以这种方式构建的?我们只能回答:我们目前的知识表明,自然确实是以这种方式构建的。我们只能接受这一点。或许可以这样描述这种情况:上帝是一个非常高阶的数学家,他在构建宇宙时使用了非常先进的数学。我们微弱的数学尝试让我们能够理解宇宙的一部分,而随着我们发展出越来越高级的数学,我们可以期望更好地理解宇宙。”[92]  

1971年,在一次会议上,狄拉克阐述了他对上帝存在的看法。[93] 狄拉克解释说,只有在过去发生了一个极不可能的事件时,上帝的存在才可以被证明:

“或许开始生命是极其困难的。可能开始生命是如此困难,以至于它只发生在所有行星中一次……让我们考虑一下,假设生命在我们有适当物理条件时开始的机会是10^-100。我没有任何逻辑理由提出这个数字,我只是希望你们把它当作一种可能性。在这种情况下……几乎可以肯定生命不会开始。而我觉得,在这种情况下,必须假设上帝的存在才能启动生命。因此,我想要建立这样一个联系:如果物理法则是这样的,开始生命的机会极小,以至于仅凭盲目的机会生命不会开始,那么就必须有一个上帝,而这样的上帝可能会通过后来的量子跃迁来显示他的影响。另一方面,如果生命可以非常容易地开始,并且不需要任何神的影响,那么我会说没有上帝。”[94]  

狄拉克并没有对这一问题作出明确的立场,但他描述了科学上回答上帝问题的可能性。[94]
\subsection{职业生涯}
\begin{figure}[ht]
\centering
\includegraphics[width=8cm]{./figures/7ec80dc660067c65.png}
\caption{1927年在布鲁塞尔召开的索尔维会议,汇聚了世界顶尖物理学家。狄拉克位于中排中央,坐在阿尔伯特·爱因斯坦的后方。} \label{fig_DLK1_4}
\end{figure}
这个方程引入了一个显著的概念,即每个费米子粒子都有一个反粒子,例如正电子是电子的反粒子。这一概念来源于他的方程。他被认为是量子场论的创始人,量子场论是现代所有关于亚原子或“基本”粒子的理论工作的基础,这些工作对于我们理解自然力至关重要,同时他还创建了量子电动力学并首次提出了这一术语。[9][11] 他提出并研究了磁单极子的概念,这是一种尚未在实验证明的物体,旨在为詹姆斯·克拉克·麦克斯韦的电磁学方程带来更高的对称性。狄拉克还创造了“费米子”和“玻色子”这两个术语。[95]  

在他的职业生涯中,狄拉克受到数学美学原则的驱动,[96] 彼得·戈达德曾指出:“狄拉克将数学美学作为选择理论物理研究方向的最终标准。”[97] 狄拉克因其出色的数学才能而广受认同,在大学期间,学术界一致认为狄拉克具备“在数学物理学领域的最高水平的才能”,[98] 埃比尼泽·坎宁安曾表示:“狄拉克是我在数学物理学领域遇到的最具原创性的学生。”[99] 因此,狄拉克以其“惊人的物理直觉和发明新数学以创造新物理的能力”著称。[17] 在他的职业生涯中,狄拉克对数学领域做出了许多重要贡献,包括狄拉克δ函数、狄拉克代数和狄拉克算符。
\begin{figure}[ht]
\centering
\includegraphics[width=8cm]{./figures/6996cf93c5f39064.png}
\caption{狄拉克(前排左起第三),旁边是埃蒙·德·瓦莱拉(前排左起第四),厄尔温·薛定谔(前排右起第二),在1942年都柏林高级研究院。} \label{fig_DLK1_5}
\end{figure}
\subsubsection{量子理论}  
狄拉克在1925年9月末迈出了进入新量子理论的第一步。 Ralph Fowler,他的研究导师,收到了一篇由维尔纳·海森堡(Werner Heisenberg)撰写的探索性论文,该论文基于老的量子理论(博尔和索末菲框架)。海森堡大量依赖博尔的对应原理,但改变了方程式,使它们直接涉及可观测的量,从而导出了量子力学的矩阵形式。福勒将海森堡的论文转交给狄拉克,要求他仔细研究这篇论文。

狄拉克被海森堡建立的一个神秘数学关系所吸引,初看时难以理解。几周后,当狄拉克回到剑桥时,他突然意识到这个数学形式与经典粒子运动的泊松括号(Poisson brackets)具有相同的结构。当时,他对泊松括号的记忆相当模糊,但他发现E. T. 惠特克(E. T. Whittaker)的《粒子与刚体的分析力学》对他有所启发。通过新的理解,狄拉克发展出了一种基于非对易动力学变量的量子理论。这使得他提出了迄今为止最深刻和最重要的量子力学的普遍公式。他通过使用狄拉克括号的创新方法,成功以一种新颖且更具启发性的方式得出了量子化规则。凭借这项工作(发表于1926年),狄拉克获得了剑桥大学的博士学位。这为费米–狄拉克统计学奠定了基础,适用于由许多相同的自旋1/2粒子(即遵循泡利不相容原理的粒子)组成的系统,如固体和液体中的电子,特别在半导体导电领域中具有重要意义。

狄拉克以不太关心量子理论解释问题而著称。事实上,在他为自己所写的纪念文章中,他写道:“量子力学的解释已经有许多作者讨论过,我不想在这里讨论这个问题。我想讨论更基础的东西。”然而,在1964年,他写了一篇关于基于海森堡图像的量子场论解释的短文;他在文中的主要观点是,薛定谔模型并不适用于此目的。
\subsubsection{狄拉克方程} 
1928年,狄拉克在2×2自旋矩阵的基础上提出了狄拉克方程,作为电子波函数的相对论性运动方程。他声称这一发现是独立于沃尔夫冈·泡利在非相对论性自旋系统中的工作而得出的(狄拉克曾对亚伯拉罕·派斯说:“我相信我独立于泡利得到了这些[矩阵],可能泡利也独立于我得到了这些。”)[106]。这一工作使得狄拉克预测了正电子的存在,正电子是电子的反粒子,他用后来被称为“狄拉克海”的概念来解释这一现象。[107] 正电子于1932年被卡尔·安德森观察到。狄拉克方程还为解释量子自旋作为一种相对论现象做出了贡献。

恩里科·费米在1934年提出的β衰变理论中,物质(费米子)在衰变过程中会被创造和摧毁,这一过程促使了对狄拉克方程的重新解释,将其作为描述任何自旋为ħ/2的点粒子的“经典”场方程,并且该方程受到涉及反对易子的量子化条件的约束。通过这种重新解释,1934年由沃尔夫冈·泡利提出,狄拉克场方程被准确地描述为一个量子场方程,适用于所有基本物质粒子——今天的夸克和轻子——这一方程与麦克斯韦方程、杨-米尔斯方程和爱因斯坦场方程一样,是理论物理学中的核心方程。狄拉克被认为是量子电动力学的奠基人,因他首次使用了这一术语。他还在1930年代初提出了真空极化的概念。这项工作为下一代理论物理学家,特别是施温格、费曼、田中信一郎和戴森在量子电动力学公式化中的发展提供了关键性贡献。

狄拉克的《量子力学原理》(The Principles of Quantum Mechanics)于1930年出版,是科学史上的一个里程碑。它迅速成为该学科的标准教科书之一,至今仍在使用。在这本书中,狄拉克将沃尔夫冈·泡利在矩阵力学方面的工作和厄尔温·薛定谔在波动力学方面的工作结合成一个单一的数学形式,利用这一形式将可测量的量与作用在描述物理系统状态的希尔伯特空间向量上的算符关联。该书还引入了狄拉克δ函数。继1939年发表的一篇文章后,狄拉克在该书第三版中也引入了“Bra–Ket”符号,从而使这一符号在今天得到广泛使用。[109][110]
\subsubsection{磁单极子}  
1931年,狄拉克提出,宇宙中存在单一磁单极子就足以解释电荷量子化的问题。[111] 尽管做出了许多尝试并提出了初步的发现,但至今尚未检测到这样的磁单极子。[112](参见磁单极子的搜索)
\subsubsection{引力}  
狄拉克对引力场进行了量子化。[46][113] 他的工作为规范量子引力奠定了基础。[114] 在1959年林道会议的讲座中,狄拉克讨论了为什么引力波具有“物理意义”。[115] 1964年,狄拉克预测引力波将具有明确的能量密度。[113] 狄拉克在1959年的几次讲座中重新引入了“引力子”这一术语,并指出引力场的能量应以量子形式存在。[116][117]
\subsubsection{弦理论}  
狄拉克被认为预见了弦理论,他在1962年的一篇论文中提出了狄拉克膜和狄拉克-博恩-因费尔德作用[118][119],以及其他一些贡献。[32][33] 他还发展了一个包含动力学约束的量子场的一般理论,[120][121][32] 这一理论为今天的规范理论和超弦理论奠定了基础。[32][46][122]
\subsubsection{重要且有影响力的工作} 
在沃尔夫冈·泡利提出泡利不相容原理——即两个电子不能占据相同的量子能级——后不久,恩里科·费米和狄拉克[103] 都意识到这一原理将极大地改变电子系统的统计力学。这项工作成为了费米-狄拉克统计的基础。[123]: 488

狄拉克在1933年写了一篇有影响力的论文,讨论了量子力学中的拉格朗日量。[124] 这篇论文为朱利安·施温格和他的量子作用原理[125]提供了基础,并为理查德·费曼发展全新量子力学方法——路径积分公式——奠定了基础。[113][126]

在1963年的一篇论文中,[127]狄拉克开始研究反德西特空间(AdS)上的场论。[128] 这篇论文包含了结合特殊相对论与量子力学(适用于强子内部夸克)的数学,并为现代量子光学中至关重要的两模压缩态奠定了基础,尽管当时狄拉克并未意识到这一点。[129] 狄拉克早在1930年代就曾研究过AdS,[130]并在1935年发表了一篇相关论文。[131]

1930年,维克多·魏斯科普夫和尤金·维格纳发表了他们著名且现已成为标准的原子和分子物理中自发辐射发射的计算。[132] 值得注意的是,狄拉克在1927年2月曾给尼尔斯·玻尔写信,得出了相同的计算结果,[133]但他并未发表这一结果。[134]

1938年,[135]狄拉克对亚伯拉罕-洛伦兹电子理论中的质量进行了重正化,提出了亚伯拉罕-洛伦兹-狄拉克力,这是相对论-经典电子模型;然而,这一模型的解表明力随着时间的推移呈指数增长。[136]

费米的黄金法则,即用于计算时间相关系统中量子跃迁的公式,由恩里科·费米提出的“黄金法则”,是由狄拉克推导出来的。[137] 狄拉克是最早提出发展时间相关扰动理论的人,早期工作涉及半经典原子与电磁场的相互作用。狄拉克与沃尔夫冈·泡利、约翰·阿奇博尔德·惠勒、理查德·费曼和弗里曼·戴森最终将这一概念发展成现代物理学中不可或缺的工具,用于计算任何物理系统的性质及一系列现象。[138]
\subsubsection{剑桥大学} 
狄拉克从1932年到1969年担任剑桥大学卢卡斯数学教授。他在1934年构想了赫利康涡流同位素分离过程。[139][140] 1937年,他提出了一个基于大数假说的猜想性宇宙模型。在第二次世界大战期间,他开展了关于通过气体离心机浓缩铀的理论工作。[141] 他在1941年引入了分离工作单位(SWU)。[142] 他还参与了“管道合金”计划,这是英国在二战期间研究和制造原子弹的项目。[143][23]

狄拉克的量子电动力学(QED)中包含了具有无限自能的项。为了解决这个问题,发展出了一个被称为重正化的方法,但狄拉克始终未接受这一方法。“我必须说,我对这种情况非常不满意,”他在1975年说道,“因为这种所谓的‘好理论’确实涉及到忽略出现在方程中的无穷大,任意地忽略它们。这简直不是合乎理性的数学。合乎理性的数学应该是在量很小时忽略一个量,而不是因为它是无限大的,你不想要它,所以就忽略它!”[144] 他拒绝接受重正化,使得他的这一领域的工作逐渐偏离主流。汤本永真一郎、施温格和费曼掌握了这一方法,最终创造出了具有前所未有精度的量子电动力学(QED),因此获得了诺贝尔物理学奖的正式认可。[145]

在1950年代,为了寻找更好的量子电动力学,狄拉克发展了基于限制的哈密顿理论[146][147],该理论基于他在1949年加拿大国际数学大会上的讲座。狄拉克还解决了将施温格-汤本永方程转化为薛定谔表示的问题[148],并给出了标量介子场(自旋零的π介子或伪标量介子)、矢量介子场(自旋一的ρ介子)和电磁场(自旋一的无质量玻色子,光子)的显式表达式。

限制系统的哈密顿量是狄拉克的众多杰作之一。[citation needed] 它是哈密顿理论的强大推广,适用于弯曲时空的情况。哈密顿量的方程只涉及六个自由度,这些自由度由每个状态所在表面的点所描述的 \( g_{rs} \) 和 \( p^{rs} \) 给出。\( g_{m0} \)(m = 0, 1, 2, 3)仅通过变量 \( g^{r0} \), \( (-g^{00})^{-1/2} \) 出现在理论中,这些变量在运动方程中作为任意系数出现。对于每个表面点 \( x^{0} = \text{constant} \),有四个约束或弱方程。三个 \( H_{r} \) 形成表面上的四维矢量密度。第四个 \( H_{L} \) 是表面上的三维标量密度,HL ≈ 0;Hr ≈ 0(r = 1, 2, 3)。

在1950年代末,狄拉克将他发展出的哈密顿方法应用于将爱因斯坦的广义相对论转化为哈密顿形式[149][150],并在萨拉姆和德维特的建议下,进一步完成了引力的量子化问题,使其更接近物理学的其他领域。1959年,他在美国物理学会纽约会议上发表了关于“引力场能量”的邀请讲座。[151] 1964年,他出版了《量子力学讲义》(伦敦:学术出版社),讨论了包括弯曲时空量子化在内的非线性动力学系统的约束动力学。他还在1967年国际理论物理中心/国际原子能机构的的的里雅斯特当代物理学研讨会上发表了一篇题为《引力场量子化》的论文。

狄拉克在1963-1964年期间在耶希瓦大学所作的量子场论讲座,后于1966年作为《贝尔弗科学研究生院专著系列》第3号出版。
\subsubsection{佛罗里达州立大学和迈阿密大学}
1969年,因年龄原因(67岁),狄拉克被迫从剑桥大学的职位退休。[152] 在退休之前,他被提供了迈阿密大学(位于佛罗里达州科勒尔盖布尔斯)的访问职位;他接受了这个邀请,并加入了该校新成立的理论研究中心。[153] 1970年9月,他又接受了佛罗里达州立大学(FSU,位于塔拉哈西)的访问教授职位,并将家人搬到了塔拉哈西。1972年,他在FSU接受了全职教授的职位。[96][154]

关于他在塔拉哈西的生活的当代记载描述称,那段时间总体上是愉快的,唯一的不适应是他显然觉得夏季的炎热让人压抑,因此喜欢逃离到剑桥。[155] 他每天步行约一英里到工作地点,喜欢在附近的两个湖泊之一(银湖和失落湖)游泳,而且比在剑桥大学时更加社交,后者他大多在家工作,除非上课或开讲座。在佛罗里达州立大学,他通常会和同事一起吃午餐,然后小睡一会。[156]

在FSU的最后十二年里,狄拉克发表了超过60篇论文,其中包括一本关于广义相对论的简短书籍。[157] 他最后一篇论文(1984年)题为《量子场论的不足》,其中包含了他对量子场论的最终评判:“这些重正化规则与实验结果出奇地一致,甚至过于一致。大多数物理学家因此认为这些规则是正确的。我觉得这并不是一个充分的理由。仅仅因为结果与观察结果一致,并不能证明理论是正确的。”这篇论文以以下的话结束:“我花了许多年寻找一个能引入理论的哈密顿量,但至今未能找到。我将继续努力,尽我所能,而其他人,我希望也能沿着这样的思路继续下去。”[158]

1975年,狄拉克在新南威尔士大学举办了一系列五讲座,随后以《物理学的方向》(1978年)出版。他将这本书的版税捐赠给该大学,用于建立狄拉克讲座系列。为纪念这次讲座,新南威尔士大学还设立了“银狄拉克奖章”,以表彰在理论物理学方面做出贡献的人。[159]
\subsubsection{学生}  
在他众多的学生中[3][160],有霍米·J·巴巴(Homi J. Bhabha)、弗雷德·霍伊尔(Fred Hoyle)、约翰·波尔金霍恩(John Polkinghorne)和弗里曼·戴森(Freeman Dyson)[161]。波尔金霍恩回忆说,狄拉克“曾经被问到他最根本的信念是什么。他走到黑板前,写下了‘自然法则应该以美丽的方程式表达’”[162]。
\subsubsection{荣誉}  
狄拉克与厄尔温·薛定谔共同获得了1933年诺贝尔物理学奖,“以表彰他们在原子理论方面的新开创性成果”[19]。狄拉克还于1939年获得了皇家奖章,并于1952年获得了科普利奖章和马克斯·普朗克奖章。他于1930年当选为皇家学会会员[163][45],并于1938年当选为美国哲学会会员[164],1948年成为美国物理学会荣誉会员,1949年成为美国国家科学院会员[165],1950年当选为美国艺术与科学院院士[166],1971年成为伦敦物理学会荣誉会员。他于1969年获得首届J. 罗伯特·奥本海默纪念奖[167][168]。狄拉克于1973年成为功勋勋章成员,此前他曾拒绝骑士勋章,因为他不希望被人以名字来称呼[81][169]。

在列夫·朗道根据物理学家的生产力和天才所制定的对数级别(从0到5)中,他将狄拉克排在1级,与量子力学的其他奠基人,如维尔纳·海森堡和厄尔温·薛定谔一起[170]。
\subsubsection{去世}
1984年,狄拉克在佛罗里达州塔拉哈西去世,并被埋葬在塔拉哈西的罗斯劳恩公墓[171]。狄拉克童年的家位于布里斯托尔的比绍普斯顿,那里设有一块蓝色纪念牌[172],而附近的狄拉克路则以他与布里斯托尔的联系为名。1991年8月1日,一块纪念石在瑞士圣莫里斯(他父亲家族的故乡)的花园里竖立。1995年11月13日,一块由伯灵顿绿色板岩制成并刻有狄拉克方程的纪念标牌在西敏寺揭幕[171][173]。西敏寺院长爱德华·卡彭特最初拒绝了该纪念碑的申请,认为狄拉克是反基督教的,但最终在五年的时间里,他被说服同意了[174]
\subsection{遗产}
在狄拉克诞辰100周年之际,理查德·达利茨写道:“狄拉克工作的影响和重要性随着岁月的流逝而日益增加,今天的物理学家每天都在使用他所发展出的概念和方程式。”[46]

狄拉克去世后,两个物理学专业组织设立了年度奖项,以纪念他的贡献。英国物理学会(Institute of Physics)为表彰“对理论物理(包括数学和计算物理)的杰出贡献”而颁发狄拉克奖章[175]。第一批获奖者分别是斯蒂芬·霍金(1987年)、约翰·斯图尔特·贝尔(1988年)和罗杰·彭罗斯(1989年)。自1985年起,国际理论物理中心(ICTP)每年在狄拉克的生日(8月8日)颁发狄拉克奖。

佛罗里达州立大学的狄拉克-赫尔曼奖(Dirac-Hellman Award)由布鲁斯·P·赫尔曼于1997年设立,用于奖励该校研究人员在理论物理方面的杰出工作。[177] 佛罗里达州立大学的保罗·A·M·狄拉克科学图书馆(Paul A.M. Dirac Science Library),由曼奇(Manci)于1989年12月开放,命名以纪念他,并且他的文件存放在那里。[179] 图书馆外有一座由加布里埃拉·博洛巴斯(Gabriella Bollobás)雕刻的狄拉克雕像。[180] 位于佛罗里达州塔拉哈西创新公园的国家高磁场实验室所在街道被命名为保罗·狄拉克大道(Paul Dirac Drive)。在他的故乡布里斯托尔之外,英国牛津郡迪德科特(Didcot)也有一条以他命名的街道——狄拉克路(Dirac Place)。[181] 布里斯托尔的狄拉克-希格斯科学中心(Dirac-Higgs Science Centre)也以他的名字命名。[182]

英国广播公司(BBC)将一种视频编码器命名为“Dirac”,以纪念狄拉克。1983年发现的一颗小行星也以狄拉克命名。[183] “分布式研究利用先进计算”(DiRAC)和“Dirac”软件也都以他的名字命名。
\subsection{出版物}
\begin{enumerate}
\item \textbf{《量子力学原理》 (1930)}: 这本书总结了量子力学的思想,使用了现代形式化方法,这些方法在很大程度上是由狄拉克自己发展的。在书的最后部分,他还讨论了电子的相对论理论(即狄拉克方程),这是他开创的工作之一。此书未涉及当时已有的其他量子力学文献。
\item \textbf{《量子力学讲义》 (1966)}: 这本书主要讲解了曲率时空中的量子力学。
\item \textbf{《量子场论讲义》 (1966)}: 这本书使用哈密顿形式主义阐述了量子场论的基础。
\item \textbf{《希尔伯特空间中的自旋量子数》 (1974)}: 这本书基于1969年在迈阿密大学讲授的讲义,讨论了自旋量子数的基本方面,从一个实希尔伯特空间的形式主义开始。狄拉克以富有预言性的话语结束本书:“我们发现在一个仅从费米子变量出发的理论中,若费米子变量的数量是无限的,就会自动出现玻色子变量。必定存在与电子相关的这种玻色子变量……”
\item \textbf{《广义相对论理论》 (1975)}: 这本69页的著作总结了爱因斯坦的广义相对论。
\end{enumerate}
\subsection{参考文献}
\subsubsection{引用文献}
\begin{enumerate}
\item Bhabha, Homi Jehangir (1935). *On cosmic radiation and the creation and annihilation of positrons and electrons* (PhD thesis). University of Cambridge. EThOS uk.bl.ethos.727546.
\item BHarish-Chandra, School of Mathematics and Statistics, University of St Andrews.
\item Paul Dirac at the Mathematics Genealogy Project.
\item DeWitt, C. M., & Rickles, D., eds., *The Role of Gravitation in Physics: Report from the 1957 Chapel Hill Conference* (Berlin: Edition Open Access, 2011), p. 30.
\item Polkinghorne, John Charlton (1955). *Contributions to quantum field theory* (PhD thesis). University of Cambridge. EThOS uk.bl.ethos.727138.
\item Simmons, John (1997). *The Scientific 100: A Ranking of the Most Influential Scientists, Past and Present*. Secaucus, New Jersey: Carol Publishing Group. pp. 104–108. ISBN 978-0806517490.
\item Mukunda, N., *Images of Twentieth Century Physics* (Bangalore: Jawaharlal Nehru Centre for Advanced Scientific Research, 2000), p. 9.
\item Berry, Michael (1 February 1998). "Paul Dirac: the purest soul in physics". *Physics World*. Retrieved 16 October 2023.
\item Duck, Ian; Sudarshan, E.C.G. (1998). "Chapter 6: Dirac's Invention of Quantum Field Theory". *Pauli and the Spin-Statistics Theorem*. World Scientific Publishing. pp. 149–167. ISBN 978-9810231149.
\item "Quantum Field Theory > The History of QFT (Stanford Encyclopedia of Philosophy)". plato.stanford.edu. Retrieved 22 October 2023.
\item Bhaumik, Mani L. (2022). "How Dirac's Seminal Contributions Pave the Way for Comprehending Nature's Deeper Designs". *Quanta*. 8 (1): 88–100. arXiv:2209.03937. doi:10.12743/quanta.v8i1.96. S2CID 212835814.
\item Schmitz, Kenneth S. (2018). *Physical Chemistry: Multidisciplinary Applications in Society*. Elsevier. p. 310. ISBN 978-0-12-800513-2.
\item Gordin, Michael D. (6 February 2017). "Dr. Strange". *American Scientist*. Retrieved 29 May 2024.
\item "Discovering the positron". *timeline.web.cern.ch*. Retrieved 23 October 2023.
\item Zichichi, Antonino (2 March 2000). "Dirac, Einstein and physics". *Physics World*. Retrieved 22 October 2023.
\item Dirac, Paul (1931). "Quantised singularities in the electromagnetic field". *Proceedings of the Royal Society of London*. Series A, Containing Papers of a Mathematical and Physical Character. 133 (821): 60–72. Bibcode:1931RSPSA.133...60D. doi:10.1098/rspa.1931.0130. ISSN 0950-1207.
\item Gottfried, Kurt (2011). "P. A. M. Dirac and the discovery of quantum mechanics". *American Journal of Physics*. 79 (3): 2, 10. arXiv:1006.4610. Bibcode:2011AmJPh..79..261G. doi:10.1119/1.3536639. S2CID 18229595.
\item Kragh, Helge (10 September 2013). "Paul Dirac and The Principles of Quantum Mechanics", *Research and Pedagogy: A History of Quantum Physics through Its Textbooks*, MPRL – Studies, Berlin: Max-Planck-Gesellschaft zur Förderung der Wissenschaften, ISBN 978-3-945561-24-9, retrieved 23 October 2023.
\item "The Nobel Prize in Physics 1933". *The Nobel Foundation*. Retrieved 4 April 2013.
\item Farmelo, Graham (2008). "Paul Dirac: The Mozart of Science". *www.ias.edu*. Retrieved 8 May 2024.
\item Brown, Laurie M. (2006). "Paul A. M. Dirac's Principles of Quantum Mechanics" (PDF). *Physics in Perspective*. 8 (4): 381–407. Bibcode:2006PhP.....8..381B. doi:10.1007/s00016-006-0276-4. S2CID 120303937.
\item Cathcart, Brian (25 May 2006). "Tube Alloys directorate (act. 1941–1945)". *Oxford Dictionary of National Biography* (online ed.). Oxford University Press. doi:10.1093/ref:odnb/93791. Retrieved 25 October 2023. (Subscription or UK public library membership required.)
\item Vrobel, Daniel P. (2020). *Paul Dirac: The Atomic Centrifuge and the Tube Alloys Project* (Master thesis). Florida State University.
\item McKie, Robin (31 January 2009). "Anti-matter and madness". *amp.theguardian.com*. Retrieved 24 October 2023.
\item Kemp, R. Scott (26 June 2009). "Gas Centrifuge Theory and Development: A Review of U.S. Programs". *Science & Global Security*. 17 (1): 1–19. Bibcode:2009S&GS...17....1K. doi:10.1080/08929880802335816. ISSN 0892-9882.
\item Gilinsky, Victor (2010). "Remembrances of Dirac". *Physics Today*. 63 (5): 59. Bibcode:2010PhT....63e..59G. doi:10.1063/1.3431338.
\item Kragh 1990, p. 158
\item Dirac, Paul Adrien Maurice (5 April 1938). "A new basis for cosmology". *Proceedings of the Royal Society of London*. Series A. Mathematical and Physical Sciences. 165 (921): 199–208. Bibcode:1938RSPSA.165..199D. doi:10.1098/rspa.1938.0053. ISSN 0080-4630. S2CID 121069801.
\item Kragh, Helge (2014). "Paul Dirac and the Magic of Large Numbers". *Masters of the Universe: Conversations with Cosmologists of the Past*: 217–237. doi:10.1093/acprof:oso/9780198722892.003.0012. ISBN 978-0-19-872289-2 – via Oxford Academic.
\item Saibal, Ray; Mukhopadhyay, Utpal; Ray, Soham; Bhattacharjee, Arjak (2019). "Dirac's large number hypothesis: A journey from concept to implication". *International Journal of Modern Physics D*. 28 (8): 1930014–1930096. Bibcode:2019IJMPD..2830014R. doi:10.1142/S0218271819300143. S2CID 127899548 – via World Scientific.
\item Dubois, Eve-Aline; Füzfa, André; Lambert, Dominique (2022). "The large numbers hypothesis in cosmology". *The Fifteenth Marcel Grossmann Meeting*. WORLD SCIENTIFIC: 1741–1744. Bibcode:2022mgm..conf.1741D. doi:10.1142/9789811258251_0259. ISBN 978-981-12-5824-4. S2CID 225097737.
\item Kragh 1990, pp. 198, 348
\item Sanyuk, Valerii I.; Sukhanov, Alexander D. (1 September 2003). "Dirac in 20th century physics: a centenary assessment". *Physics-Uspekhi*. 46 (9): 937–956. doi:10.1070/PU2003v046n09ABEH001165. ISSN 1063-7869. S2CID 250754932.
\item Han, Xiaosen (1 April 2016). "The Born–Infeld vortices induced from a generalized Higgs mechanism". *Proceedings of the Royal Society A: Mathematical, Physical and Engineering Sciences*. 472 (2188): 20160012. Bibcode:2016RSPSA.47260012H. doi:10.1098/rspa.2016.0012. ISSN 1364-5021. PMC 4892282. PMID 27274694.
\item "Biographical and Research Highlights". *Department of Applied Mathematics and Theoretical Physics (DAMTP)*. Retrieved 21 June 2024.
\item Kragh 1990, p. 82] "Dirac verstehe ich im Einzelnen überhaupt nicht (Compton-Effekt)"
\item Kursunoglu, Behram N.; Wigner, Eugene P. (1987). *Paul Adrien Maurice Dirac: Reminiscences about a Great Physicist*. Cambridge University Press. p. 262. ISBN 9780521340137.
\item "The Strangest Man: The Hidden Life of Paul Dirac, Quantum Genius". *CERN Courier*. 25 August 2009. Retrieved 22 October 2023.
\item Deser, Stanley (2004). "P.A.M. Dirac and the Development of Modern General Relativity". *International Journal of Modern Physics A*. 19 (supp01): 99–105. Bibcode:2004IJMPA..19S..99D. doi:10.1142/S0217751X04018622. ISSN 0217-751X.
\item Hey, Tony; Walters, Patrick (1987). *The Quantum Universe*. Cambridge University Press. p. 124. ISBN 978-0521267441.
\item Close, Frank (20 May 2009). "Paul Dirac: a physicist of few words". *Nature*. 459 (7245): 326–327. Bibcode:2009Natur.459..326C. doi:10.1038/459326a. ISSN 1476-4687.
\item Kragh 1990, pp. ix, 12.
\item Farmelo 2009, p. 10
\item Farmelo 2009, pp. 18–19
\item Dalitz, R. H.; Peierls, R. (1986). "Paul Adrien Maurice Dirac. 8 August 1902 – 20 October 1984". *Biographical Memoirs of Fellows of the Royal Society*. 32: 137–185. doi:10.1098/rsbm.1986.0006. JSTOR 770111.
\item "Paul Dirac: a genius in the history of physics". *Cern Courier*. 15 August 2002. Retrieved 4 February 2022.
\item Farmelo 2009, pp. 8, 441
\item Farmelo 2009, pp. 8
\item Farmelo 2009, pp. 441
\item Kragh 1990, p. 1
\item Farmelo 2009, pp. 10–11
\item Farmelo 2009, pp. 77–78
\item Farmelo 2009, p. 79
\item Farmelo 2009, p. 34
\item Farmelo 2009, p. 22
\item Mehra 1972, p. 17
\item Kragh 1990, p. 2
\item Farmelo 2009, pp. 13–17
\item Farmelo 2009, pp. 20–21
\item Mehra 1972, p. 18
\item Farmelo 2009, p. 23
\item Farmelo 2009, p. 28
\item Farmelo 2009, pp. 46–47
\item Galison, Peter (2000). "The Suppressed Drawing: Paul Dirac's Hidden Geometry". *Representations* (72): 145–166. doi:10.2307/2902912. ISSN 0734-6018. JSTOR 2902912.
\item Farmelo 2009, p. 53
\item Farmelo 2009, pp. 52–53
\item 1851 Royal Commission Archives
\item Farmelo 2009, p. 101
\item Kursunoglu, Behram N.; Wigner, Eugene Paul, eds. (1990). *Paul Adrien Maurice Dirac: Reminiscences about a Great Physicist*. Cambridge University Press. p. 132. ISBN 0521386888. Retrieved 30 September 2020.
\item "Paul Adrien Maurice Dirac". *University of Wisconsin-Madison*. Retrieved 30 September 2020.
\item Farmelo 2009, p. 284
\item Farmelo 2009, p. 253
\item Farmelo 2009, p. 256
\item Farmelo 2009, p. 288
\item Farmelo 2009, pp. 305, 323
\item Kim, Young Suh (1995). "Wigner's Sisters". Archived from the original on 3 March 2008.
\item Farmelo 2009, p. 89
\item "Paul Adrien Maurice Dirac". *University of St. Andrews*. Retrieved 4 April 2013.
\item Mehra 1972, pp. 17–59
\item Kragh 1990, p. 17
\item McKie, Rob (1 February 2009). "Anti-matter and madness". *The Guardian*. Retrieved 4 April 2013.
\item Gamow 1966, p. 121
\item Capri 2007, p. 148
\item Zee 2010, p. 105
\item Raymo, Chet (17 October 2009). "A quantum leap into oddness". *The Globe and Mail*. (Review of Farmelo's *The Strangest Man*.)
\item Farmelo 2009, pp. 161–162, who attributes the story to Niels Bohr.
\item Mehra, Jagdish; Rechenberg, Helmut (2001). *The Historical Development of Quantum Theory*. Springer Science & Business Media. p. 746. ISBN 9780387951805.
\item Pais, A., *Niels Bohr's Times: In Physics, Philosophy, and Polity* (Oxford: Clarendon Press, 1991), p. 320.
\item Heisenberg 1971, pp. 85–86
\end{enumerate}