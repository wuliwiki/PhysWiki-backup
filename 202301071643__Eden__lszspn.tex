% LSZ约化公式(旋量场)
% LSZ约化公式|量子电动力学|旋量场|S矩阵|费曼图


我们用分析动量空间编时格林函数的极点行为的方法来推导 LSZ 约化公式。将时间分为三个积分区域 $1,2,3$,先在积分区域 $1$ 中(即 $t>T_+$)对 $\psi(x)$ 傅里叶变换(这里 $\psi(x)$ 是海森堡表象下的场算符)。与标量场不同的是,这里我们将 $\int_{3} \dd[4]{x} e^{ipx} u^{s\dagger}(\bvec p)$ 作用于 $\psi(x)$,来将场算符约化为具有特定动量以及自旋的出态。这里的旋量 $u$ 是以物理质量 $m$ 为参数的。
\begin{equation}
\begin{aligned}
&\int_{1} \dd[4]{x} e^{ipx} u^{s\dagger}(\bvec p)\langle \Omega|T[\psi(x)\cdots]|\Omega\rangle\\
=&\int_{T_+}^\infty \dd x^0 e^{ip^0x^0}\int \dd[3]{\bvec x} e^{-i\bvec p\cdot \bvec x} u^{s\dagger}_\alpha(\bvec p)
\sum_{\lambda, s'}\int \frac{\dd[3] \bvec q}{(2\pi)^3} \frac{1}{2E_{\bvec q}(\lambda)}\langle \Omega|\psi_\alpha(x)|\lambda_{\bvec q},s'\rangle \langle \lambda_{\bvec q},s'| T[\cdots]|\Omega\rangle\\
\stackrel{p^0\rightarrow E_{\bvec p}}{\sim}
&
\sum_{s'}\int_{T_+}^\infty \dd x^0 e^{i(p^0-E_{\bvec p}(\lambda))x^0}
\frac{1}{2E_{\bvec p}} u^{s\dagger}_\alpha(\bvec p)\langle \Omega|\psi_\alpha(0)|\bvec 0,s',+\rangle\langle \bvec p,s',+| T[\cdots]|\Omega\rangle\\
=&\sum_{s'}\frac{i\sqrt{Z_2}}{2E_{\bvec p}}\frac{e^{i(p^0-E_{\bvec p}+i\epsilon)T_+)}}{(p^0-E_{\bvec p}+i\epsilon)} u^{s\dagger}(\bvec p)u^{s'}(\bvec p)\langle \bvec p,s',+| T[\cdots]|\Omega\rangle= \frac{i\sqrt{Z_2}}{p^0-E_{\bvec p}+i\epsilon} \langle \bvec p,s,+| T[\cdots]|\Omega\rangle
\end{aligned}
\end{equation}
经过相似的推导,共可以得到四组结果,分别对应将场算符约化为入射和出射的费米子或反费米子。
\begin{equation}
\begin{aligned}
&\int_{ 1} \dd[4]{x} e^{ip'x} \bar{u}^{s'}(\bvec p')\gamma^0\langle \Omega|T[\psi(x)\cdots]|\Omega\rangle
\stackrel{p'^0\rightarrow E_{\bvec p'}}{\sim}
\frac{i\sqrt{Z_2}}{p'^0-E_{\bvec {p'}}+i\epsilon} \langle \bvec p',s',+| T[\cdots]|\Omega\rangle\\
&
\int_{ 1} \dd[4]{y} e^{ik'y} \langle \Omega|T[\bar{\psi}(y)\cdots]|\Omega\rangle \gamma^0 v^{r'}(\bvec k')
\stackrel{k'^0\rightarrow E_{\bvec k'}}{\sim}
\frac{i\sqrt{Z_2}}{{k'}^0-E_{\bvec k'}+i\epsilon} \langle \bvec k',r',-| T[\cdots]|\Omega\rangle\\
&
\int_{ 3} \dd[4]{z} e^{-ipz}
\langle \Omega|T[\cdots\bar{\psi}(z)]| \Omega\rangle\gamma^0  u^{s}(\bvec p)
\stackrel{p^0\rightarrow E_{\bvec p}}{\sim}
\frac{i\sqrt{Z_2}}{p^0-E_{\bvec p}+i\epsilon} \langle \Omega| T[\cdots]|\bvec p,s,+\rangle\\
&
\int_{ 3} \dd[4]{w} e^{-ikw}
\bar{v}^{r}(\bvec k)\gamma^0\langle \Omega|T[\cdots\psi(w)]| \Omega\rangle  
\stackrel{k^0\rightarrow E_{\bvec k}}{\sim}
\frac{i\sqrt{Z_2}}{k^0-E_{\bvec k}+i\epsilon} \langle \Omega| T[\cdots]|\bvec k,r,-\rangle
\end{aligned}
\end{equation}
经过波包调制以后,可以将编时格林函数的场算符划分为两部分,一部分场算符位于无穷远未来,在 $ 1$ 区域,另一部分场算符在 $ 3$ 区域。它们所对应的单粒子态在空间上是间隔很远的,那么就可以被约化为相应的入态和出态。最后得到的入态与出态的内积就是 S-矩阵。最后要注意编时算符作用下交换两个费米子场算符的左右位置时需要改变符号,我们约定多粒子态 $\bra{\bvec p',s',+;\bvec k',r',-;\cdots}=-\bra{\bvec k',r',-;\bvec p',s',+;\cdots}$ 是 $\ket{\bvec p',s',+;\bvec k',r',-;\cdots}$ 的共轭。由上述公式最终可以得到
\begin{equation}
\begin{aligned}
&
\prod_{i'}\int \dd[4]{x_{i'}} e^{ip'_{i'}x_{i'}}
\prod_{j'}\int \dd[4]{y_{j'}} e^{ik'_{j'}y_{j'}}
\prod_{i}\int \dd[4]{z_{i}} e^{-ip_{i}z_{i}}
\prod_{j}\int \dd[4]{w_{j}} e^{-ik_{j}w_{j}}
\\
&
\quad \left[\bar u^{s'_{i'}}(\bvec p'_{i'})\gamma^0\right]_{\alpha} 
\left[{\bar v}^{r_j}(\bvec k_{j})\gamma^0\right]_{\delta}
\bra\Omega T[\psi_\alpha(x_1)\bar\psi_\beta(y_1)\bar\psi_\gamma(z_1)\psi_\delta(w_1)\cdots]\ket \Omega 
\left[\gamma^0 u^{s_i}(\bvec p_{i})\right]_{\gamma} \left[\gamma^0 v^{r'_{j'}}(\bvec k'_{j'})\right]_{\beta}
\\
&
\mapsto 
\left(\prod_{i'}\frac{i\sqrt{Z_2}} {{p'_{i'}}^0-\omega_{\bvec p'_{i'}}+i\epsilon}\right)
\left(\prod_{j'}\frac{i\sqrt{Z_2}}{{k'_{j'}}^0-\omega_{\bvec k'_{j'}}+i\epsilon}\right)
\left(\prod_{i}\frac{i\sqrt{Z_2}}{p_i^0-\omega_{\bvec p_i}+i\epsilon}\right)
\left(\prod_{j}\frac{i\sqrt{Z_2}}{k_j^0-\omega_{\bvec k_j}+i\epsilon}\right)
\\
&
\quad \bra{\bvec k',r',-;\bvec p',s',+;\cdots} S \ket{\bvec p,s,+;\bvec k,r,-;\cdots}
\end{aligned}
\end{equation}
其中 $\mapsto$ 表示取所有的动量在壳,并且仅仅考察其中最为奇异的多极点部分的贡献。$Z_2$ 是旋量场的场强重整化因子。上式还可以等价地写为
\begin{equation}
\begin{aligned}
&
\prod_{i,j,i',j'}
\int \dd[4]{x_{i'}} e^{ip'_{i'}x_{i'}}
\int \dd[4]{y_{j'}} e^{ik'_{j'}y_{j'}}
\int \dd[4]{z_{i}}e^{-ip_{i}z_{i}}
\int \dd[4]{w_{j}}e^{-ik_{j}w_{j}}
\\
&
\quad \bar u^{s'_{i'}}(\bvec p'_{i'})(i\not \partial_{x_{i'}}-m) \bar v^{r_{j}}(\bvec k_{j})(i\not \partial_{w_{j}}-m) 
\bra\Omega T[\psi(x_1)\bar\psi(y_1)\bar\psi(z_1)\psi(w_1)\cdots]\ket \Omega \\
&
\quad (-i\overleftarrow{\not \partial}_{z_i} -m )u^{s_i}(\bvec p_i) (-i\overleftarrow{\not \partial}_{y_{j'}}-m) v^{r'_{j'}}(\bvec k'_{j'})
\\
&
\mapsto 
(i\sqrt{Z_2})^{4+\cdots} \bra{\bvec k',r',-;\bvec p',s',+;\cdots} S \ket{\bvec p,s,+;\bvec k,r,-;\cdots}
\end{aligned}
\end{equation}
可以对上式作分部积分将 $\not\partial$ 算子替换为 $\not p$,来验证上面的两个 LSZ 约化公式的等价性。