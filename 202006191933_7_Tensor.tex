% 张量
\pentry{线性映射,映射\upref{map}}

%线性映射词条需要从抽象角度重新创作;线性映射词条大概率会引用映射词条,到时候要在预备知识中删除映射词条这一项

\subsection{线性函数和向量}
\subsubsection{线性函数}
\begin{definition}{线性函数}
给定域$\mathbb{F}$上的$n$维线性空间$V$,称$f:V\rightarrow \mathbb{F}$为$V$到域$\mathbb{F}$上的一个\textbf{线性函数},如果$f$满足以下线性性:对于任意的$\bvec{v}_1, \bvec{v}_2\in V$和$a_1, a_2\in\mathbb{F}$,都有$a_1f(\bvec{v}_1)+a_2f(\bvec{v}_2)=f(a_1\bvec{v}_1+a_2\bvec{v}_2)$.
\end{definition}

如果把域$\mathbb{F}$本身看成一个一维的线性空间,那么线性函数就是$V$到这个一维空间上的线性映射.因此,我们只需要任取一个$V$的基,研究基向量被$f$映射到哪里,就可以计算出任意的$\bvec{v}\in V$被映射到哪里了.

设$V$有基$\{\bvec{e}_i\}_{i=1}^n$.如果基向量$\bvec{e}_i$被映射为$f(\bvec{e}_i)=m_i\in\mathbb{F}$,那么对于任意的向量$\bvec{v}=a_1\bvec{e}_1+a_2\bvec{e}_2+\cdots+a_n\bvec{e}_n$,都有:
\begin{equation}
\begin{aligned}
f(a_1\bvec{e}_1+a_2\bvec{e}_2+\cdots+a_n\bvec{e}_n)&=f(a_1\bvec{e}_1)+f(a_2\bvec{e}_2)+\cdots+f(a_n\bvec{e}_n)\\&=m_1a_1+m_2a_2+\cdots+m_na_n
\end{aligned}
\end{equation}

$m_1a_1+m_2a_2+\cdots+m_na_n$可以看成是向量$\bvec{m}$点乘$\bvec{v}$所得到的,其中$\bvec{m}=m_1\bvec{e}_1+m_2\bvec{e}_2+\cdots+m_n\bvec{e}_n$.

也就是说,每个线性函数$f$对应一个向量$m$,使得$f(\bvec{v})=\bvec{m}\cdot\bvec{v}$.

\subsubsection{$2-$线性函数}

继续使用上述给定的$n$维线性空间$V$,但是我们现在拿出$2$个$V$来构造映射$f:V\times V\rightarrow\mathbb{F}$.

\begin{definition}{2-线性映射}
称$f:V\times V\rightarrow\mathbb{F}$为一个$2-$线性映射,如果对于任意固定的$\bvec{v}_0$,$f(\bvec{v}_0, \bvec{v})$和$f(\bvec{v}, \bvec{v}_0)$都是$\bvec{v}$的线性函数,那么称$f$是一个$V^2$上的$2-$\textbf{线性函数}或\textbf{双线性函数}.
\end{definition}

双线性函数的另一种表达方法,是
\begin{equation}
\begin{aligned}
f(a_1\bvec{v}_1+a_2\bvec{v}_2, b_1\bvec{u}_1+b_2\bvec{u}_2)&=a_1f(\bvec{v}_1, b_1\bvec{u}_1+b_2\bvec{u}_2)+a_2f(\bvec{v}_2, b_1\bvec{u}_1+b_2\bvec{u}_2)\\&=a_1b_1f(\bvec{v}_1, \bvec{u}_1)+a_2b_1f(\bvec{v}_2, \bvec{u}_1)+a_1b_2f(\bvec{v}_1, \bvec{u}_2)+a_2b_2f(\bvec{v}_2, \bvec{u}_2)
\end{aligned}
\end{equation}

因此,给定了$V$的基$\{\bvec{e}_i\}_{i=1}^n$以后,可以把向量都表示成其坐标构成的列向量,把双线性函数表示成一个矩阵:
\begin{equation}
\begin{aligned}
&\bvec{v}=a_1\bvec{e}_1+a_2\bvec{e}_2\rightarrow c_v=\pmat{a_1\\a_2}\\
&\bvec{u}=b_1\bvec{e}_1+b_2\bvec{e}_2\rightarrow c_u=\pmat{b_1\\b_2}\\
&f\rightarrow \bvec{M}=\pmat{f(\bvec{e}_1, \bvec{e}_1)&f(\bvec{e}_1, \bvec{e}_2)\\f(\bvec{e}_2, \bvec{e}_1)&f(\bvec{e}_2, \bvec{e}_2)}\\
\end{aligned}
\end{equation}

这样,我们就有$f(\bvec{v}, \bvec{u})=c_v^T\bvec{M}c_u$.注意这里$c_v^T$表示$c_v$作为矩阵的转置\footnote{如果使用其它的基,那么向量$\bvec{v}$、$\bvec{u}$的坐标和映射$f$的矩阵会有不同表示,但是计算出来的$c_v^T\bvec{M}c_u$仍然是一致的.特别要注意的是,两个$V$中的基必须选择相同的向量,按相同的顺序排列,否则$f$无法表示成矩阵.}.

\begin{exercise}{}
把列向量看成矩阵,根据矩阵的运算法则,验证$f(\bvec{v}, \bvec{u})=c_v^T\bvec{M}c_u$.
\end{exercise}

\subsubsection{矩阵的运算回顾}

矩阵只是一种运算的表达方式.虽然我们常见的矩阵元素都是实数或者复数,但是只要是可以相加和相乘的元素都可以当作矩阵元素.

\begin{example}{}
矩阵运算的例子.
\begin{itemize}
%
\item 向量可以和数字相乘.如果$\bvec{v}_i$表示向量,$a_i$表示数字,那么$\pmat{\bvec{v}_1&\bvec{v}_2&\bvec{v}_3}$是三个向量排成的矩阵,$\pmat{a_1\\a_2\\a_3}$是三个数字排成的矩阵,按照向量的数乘来进行矩阵乘法,$\pmat{\bvec{v}_1&\bvec{v}_2&\bvec{v}_3}\pmat{a_1\\a_2\\a_3}=(a_1\bvec{v}_1+a_2\bvec{v}_2+a_3\bvec{v}_3)$就是一个向量,$\pmat{a_1\\a_2\\a_3}\pmat{\bvec{v}_1&\bvec{v}_2&\bvec{v}_3}=\pmat{a_1\bvec{v_1}&a_1\bvec{v_2}&a_1\bvec{v_3}\\a_2\bvec{v_1}&a_2\bvec{v_2}&a_2\bvec{v_3}\\a_3\bvec{v_1}&a_3\bvec{v_2}&a_3\bvec{v_3}}$是$9$个向量排成的矩阵.
\item 向量之间可以有点乘.因此$\pmat{\bvec{v}_1&\bvec{v}_2\\\bvec{v}_3&\bvec{v}_4}$
%
\end{itemize}
\end{example}



