% 柱坐标系中的拉普拉斯方程
% keys 柱坐标系|贝塞尔方程|分离变量法|拉普拉斯方程
% license Xiao
% type Tutor

\pentry{分离变量法\upref{SepVar}, 拉普拉斯方程(直角坐标), 柱坐标的拉普拉斯算符\upref{CylNab}}

拉普拉斯方程
\begin{equation}
\laplacian u(\bvec r) = 0~.
\end{equation}
用柱坐标的拉普拉斯算子\upref{CylNab}表示为
\begin{equation}
\frac{1}{r}\pdv{r}\qty(r \pdv{u}{r}) + \frac{1}{r^2} \pdv[2]{u}{\theta} + \pdv[2]{u}{z} = 0~.
\end{equation}

使用分离变量法, 其通解可以记为
\begin{equation}
f(\bvec r) = \sum_{l,m} [A_l J_l(kr) + B_l Y_l(kr)]\E^{\I m \theta} (C_l\E^{lz} + C_l \E^{-lz})~.
\end{equation}
其中 $A_l, B_l, C_l, D_l$ 是待定常数, 所以它们中只有三个自由度。

拉普拉斯在柱坐标中分离变量后,径向方程为贝塞尔方程(\autoref{eq_Bessel_1}~\upref{Bessel})
\begin{equation}
x \dv{x} \qty(x \dv{y}{x}) + (x^2 - m^2)y = 0~,
\end{equation}
其中 $x = lr$。

\subsection{推导}
令 $u(\bvec r) = R(r) \Phi(\theta) Z(z)$, 代入方程得
\begin{equation}
\frac{1}{rR}\pdv{r} \qty(r\pdv{R}{r}) + \frac{1}{r^2 \Phi} \pdv[2]{\Phi}{\theta} + \frac{1}{Z} \pdv[2]{Z}{z} = 0~.
\end{equation}
前两项只是 $r$ 和 $\theta $ 的函数, 第三项只是 $z$ 的函数, 所以它们分别为常数。 令
\begin{equation}\label{eq_CylLap_4}
\frac{1}{Z} \pdv[2]{Z}{z} = l^2~,
\end{equation}
则前两项为
\begin{equation}
\frac{1}{rR} \pdv{r} \qty(r\pdv{R}{r}) + \frac{1}{r^2 \Phi} \pdv[2]{\Phi}{\theta} =  - l^2~.
\end{equation}
为了继续分离 $r$ 和 $\theta$, 两边乘以 $r^2$,   则左边第二项只是关于 $\theta$  的函数, 剩下的部分只是关于 $r$ 的函数。 令
\begin{equation}\label{eq_CylLap_6}
\frac{1}{\Phi } \dv[2]{\Phi}{\theta} = -m^2~.
\end{equation}
则剩下的部分为 $m^2$, 即
\begin{equation}\label{eq_CylLap_7}
r \dv{r} \qty(r\dv{R}{r}) + (l^2 r^2 - m^2)R = 0~.
\end{equation}

令 $x = lr$,   $y(x) = R(r)$ 则
\begin{equation}\label{eq_CylLap_8} 
x \dv{x} \qty(x\dv{y}{x}) + (x^2 - m^2)y = 0~.
\end{equation}
到此为止, 三个变量已经完全分离, 各自的微分方程为\autoref{eq_CylLap_4}, \autoref{eq_CylLap_6},\autoref{eq_CylLap_7}。

$Z(z)$ 的通解为 $C_1\E^{lz} + C_2 \E^{-lz}$,   $\Phi(\theta)$ 的通解为 $\E^{\I m\theta}$。   \autoref{eq_CylLap_7} 的解不能用有限的初等函数表示, \autoref{eq_CylLap_8} 为贝塞尔方程\upref{Bessel}的标准形式, 两个线性无关解是贝塞尔函数 $J_l(x), Y_l(x)$。

需要注意的是, 贝塞尔函数的阶数 $m$ 是角向方程 $(\dv*[2]{\Phi}{\theta})/\Phi = -m^2$ 的参数, 而不是径向方程的参数 $l$。 参数 $l$ 被包含在自变量 $x$ 中。
