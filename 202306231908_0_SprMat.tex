% 稀疏矩阵
% keys 系数矩阵|数据结构|计算物理|数值计算

\textbf{稀疏矩阵(Sparse Matrix)}有不同的储存方式(数据结构), 这里介绍几种常见的\footnote{参考 Wikipedia \href{https://en.wikipedia.org/wiki/Sparse_matrix}{相关页面}。}。 本文的数组索引从 0 开始。

\subsection{Banded}
\textbf{带对角矩阵(Banded Matrix)}只储存矩阵主对角线上下的若干条对角线, 上带宽和下带宽分别指定主对角线上面和下面有几条对角线, 例如三对角矩阵的上带宽和下带宽都是 1。 带内即使有矩阵元为零也必须储存。 这样就可以按照 row major 或者 column major 来储存。 详见 “数据结构:带对角矩阵\upref{BanDmt}”。

\subsection{Coordinate List (COO)}
COO 格式列出非零矩阵元和对应的行标列标。 通常将它们储存为三个数组 \verb|a|, \verb|ia|, \verb|ja|, 顺序任意。 除此之外, 有时还需要储存三个数组的长度 \verb|nnz| (none zero) 以及矩阵的尺寸。

\subsection{Compressed Sparse Row (CSR)}
也叫 Compressed Row Storage (CRS), 这种格式做矩阵与矢量相乘较快。

CRS 格式储存为三个一维数组 \verb|a|, \verb|ia|, \verb|ja|, 想象 COO 格式的矩阵元按照行主序排列后储存, 那么行标将会出现类似 \verb|0,0,1,1,1,1,2,2,3,3,3| 这样的重复。 所以为了提高效率可以把行标矩阵 \verb|ia| 的信息压缩, 令 \verb|ia[i]| 表示第 \verb|i| 行上方所有行的矩阵元个数, \verb|ia| 的长度是矩阵行数加一。 所以 \verb|ia[0]| 恒为零, 且最后一个元素就是非零元的个数 \verb|Nnz|。 第 \verb|i| 行矩阵元从 \verb|a[ia[i]]| 一直到 \verb|a[ia[i+1]-1]|。 注意如果矩阵的第 \verb|i| 行是空的, 那么 \verb|ia[i+1]| 等于 \verb|ia[i]|, \verb|ia[i+1]-1| 反而比 \verb|ia[i]| 要小, 于是循环 \verb|for(n=ia[i]; n<ia[i+1]; ++n)| 不执行。 \verb|ja| 和 COO 一样仍然是对应矩阵元的列标。

\subsection{Compressed Sparse Column (CSC)}
也叫 Compressed Column Storage (CCS), 与 CRS 一样, 只是改为 column major。
