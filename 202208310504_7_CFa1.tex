% 从分析力学到场论
% keys 经典场|狭义相对论|作用量|拉格朗日函数

%在相对论中创建目录:经典场论


\pentry{向量丛和切丛\upref{TanBun},欧拉—拉格朗日方程\upref{Lagrng}}



所谓场论,就是研究场的运动,或者说变化的理论.经典场论所研究的是四维时空中的场,与其说是一种理论,不如说是建立理论的一种范式.我们可以把电磁场解释为场,用场论的方法研究它的变化;也可以把单个的粒子解释为场;诸如此类.场的类型也有很多,标量场,向量场,一般的张量场,都有可能.从经典场论延伸,我们还可以在千奇百怪的时空流形上建立各式各样的张量场来解释物质,但那就是量子场论以及其它内容要讨论的了.本章中,我们主要讨论\textbf{经典场},并且在讨论具体的物理概念前,我们先研究“场”这个数学对象.就像用牛顿三定律研究粒子运动之前,我们要先研究“函数”这个数学对象.

牛顿力学认为粒子是一种无大小的数学对象,并且被赋予了一个描述其特征的标量,称为“质量”.牛顿动力学所研究的,就是“粒子”这种对象运动的规律.最古典的处理方式,是用\textbf{牛顿三定律}来描述这种规律,相当于描述每时每刻粒子的运动状态和其改变;拉格朗日力学则提出了另一种研究范式,即从整体着手,研究哪些粒子运动轨迹是允许的,此时描述规律的方式变成了\textbf{最小作用量原理}.角度虽有不同,但两种处理方式都是描述“粒子”的运动.

现在,我们要从拉格朗日力学的方法出发,讨论如何用场论来处理牛顿力学,也即经典场论思想的由来.


\subsection{一个粒子的运动:从粒子观点到场论观点}

\subsubsection{粒子观点}

考虑一个粒子在三维空间中的运动.粒子的位置由一个三维向量函数$\bvec{r}(t)$描述,它是时间$t$的函数.定义拉格朗日函数
\begin{equation}
\mathcal{L}(t, \bvec{r},\dot{\bvec{r}}) =\frac{1}{2}\qty(\dot{\bvec{r}}(t))^2-V(\bvec{r})
\end{equation}
其中$V$是空间的(标量)函数.

那么就可以得到其所关联的作用量,从给定的时间$a$到$b$:
\begin{equation}
\mathcal{S}=\int ^b_a \mathcal{L} \dd t
\end{equation}
于是,粒子的合法运动就是满足“在任意时间段中的作用量取最小值”的运动.用一点变分的技巧,我们就能得出粒子运动的\textbf{欧拉-拉格朗日方程}:
\begin{equation}\label{CFa1_eq1}
\frac{\dd}{\dd t}\frac{\partial\mathcal{L}}{\partial \dot{\bvec{r}}} = \frac{\partial\mathcal{L}}{\partial \bvec{r}}
\end{equation}

符合给定初值条件\autoref{CFa1_eq1} 的向量值函数$\bvec{r}(t)$,就是粒子在该初值条件下的运动.


\subsubsection{场论观点}

换一种视角来理解这个粒子的运动:粒子在某一时刻的位置,可以理解为时间轴上的\textbf{某一点处粘着的三维向量丛}上的一点;粒子所有时刻的位置构成的轨迹,也可以理解为时间轴上三维向量丛的截面.

再准确些说,四维时空中粒子运动的轨迹,可以理解为一维时空上$\mathbb{R}^3$丛的一个截面.所谓一维时空,就是时间轴,没有空间部分.也就是说,三维的空间部分被理解为一个向量丛了,这样就自然把轨迹的概念化为场的概念了.

现在,我们用场论观点重新叙述一遍粒子的运动规律:

一维时空上有一个向量丛$\mathbb{R}^3$,对于该丛上的每个截面$\phi(t)$,能诱导另一个截面$\dot{\phi}(t)=\frac{\dd}{\dd t}\phi(t)$.

定义拉格朗日函数为
\begin{equation}\label{CFa1_eq2}
\mathcal{L}(t, \phi, \partial_t\phi) = \frac{1}{2}\qty(\partial_t \phi(t))^2 - V(\phi(t))
\end{equation}
它可以理解为给定一个截面$\phi$后,一维时空上的函数.

于是作用量为
\begin{equation}\label{CFa1_eq4}
\mathcal{S}=\int_a^b \mathcal{L}\dd t
\end{equation}
它是给定截面和一维时空上的一个区域后,由拉格朗日函数确定的一个数字.

最小作用量原理用\textbf{欧拉-拉格朗日方程}描述:
\begin{equation}
\frac{\dd}{\dd t}\frac{\partial\mathcal{L}}{\partial\dot{\phi}} = \frac{\partial \mathcal{L}}{\partial\phi}
\end{equation}

场论观点基本上和粒子观点没有区别,只不过用不同的概念解释了相同的东西.四维时空的轨迹变成了一维时空的向量丛截面,相空间上的拉格朗日函数变成了给定截面时一维时空的函数,诸如此类.最重要的是,求作用量时的积分区间,被解释为一维时空上的一个区域.

这是用场论处理单个粒子运动的示例,在这个例子里,场是粒子的运动轨迹.很简单,很直白,但是很有启发意义.我们可以由此出发,推广出场论应该具有的形式,得到新的建立理论的范式.


\subsection{拉格朗日场论}

经典场论所研究的时空,是四维经典时空或者闵可夫斯基时空.我们之前一直强调“一维时空”而不说“时间轴”,就是为了这里的概念自然拓展,从时间轴到四维时空.



考虑$\mathbb{R}^4$上某个\textbf{张量丛的截面},也就是一个\textbf{张量场},记为$\phi$.

时空是一个流形,上面定义了一个度量张量$g^{\mu\nu}$.假设给定了一个拉格朗日函数$\mathcal{L}(x^\mu, \phi, \partial_\mu\phi)$,该如何推广\autoref{CFa1_eq4} ,得到作用量的表达式呢?答案很简单,“积分区域是时空上的给定区域”.
\begin{equation}
\mathcal{S} = \int_{U\subseteq \mathbb{R}^4} \mathcal{L}\dd x^0\dd x^1\dd x^2\dd x^3
\end{equation}

类似地,我们可以用最小作用量原理导出四维时空上场的欧拉-拉格朗日方程:
\begin{equation}\label{CFa1_eq5}
\partial_\mu \frac{\partial\mathcal{L}}{\partial(\partial_\mu \phi)} = \frac{\partial\mathcal{L}}{\partial \phi}
\end{equation}

这就是这个张量场的演化规律.

特别要注意的是,在很多场论材料中,作用量依然是仅对时间求积分,因此这些材料会将我们定义的$\mathcal{L}$称为“拉格朗日密度(Lagrangian density)”,而拉格朗日量则为其在全空间的积分.

\begin{example}{牛顿力学的推广}\label{CFa1_ex1}
我们从最简单的场入手,闵可夫斯基时空上的标量场$\phi$.它可以理解为闵可夫斯基流形上,一个一维向量丛的截面.该流形上的度量张量表示为$g^{\mu\nu}=\opn{diag}(1, -1, -1, -1)$.

我们可以仿照\autoref{CFa1_eq2} ,构造标量场$\phi$的一个拉格朗日函数,它必须是一个标量函数:
\begin{equation}\label{CFa1_eq3}
\mathcal{L}(r, \phi, \partial_0\phi, \partial_1\phi, \partial_2\phi, \partial_3\phi) = \frac{1}{2}(g^{\mu\nu}\partial_\mu\phi \partial_\nu\phi) - V(\phi)
\end{equation}
其中$r$是时空上的点.注意比较\autoref{CFa1_eq2} 和\autoref{CFa1_eq3} ,看看是怎么推广的.

把\autoref{CFa1_eq3} 代入\autoref{CFa1_eq5} ,就可以得到$\phi$在\textbf{这个拉格朗日函数}下的运动方程:

\begin{equation}
(\partial^2_0-\partial^2_1-\partial^2_2-\partial^2_3)\phi = -\frac{\partial V}{\partial \phi}
\end{equation}

这个运动方程正是牛顿第二定律的推广:
\begin{equation}
\partial^2_0 \phi_i = -\frac{\partial V}{\partial \phi_i}
\end{equation}
其中向量场$\phi_i$被解释为粒子在欧几里得时空中的空间位置.

\end{example}



\autoref{CFa1_ex1} 中的$V(\phi)$被称为$\phi$的\textbf{场势(field potential)}.它可以理解为一种“能量密度”,各点的场势取决于这个点上场的取值.场势的具体形式依照情况而定,有时是根据实验数据.



\begin{example}{Klein-Gordon场}

见\textbf{Klein-Gordon方程}的\autoref{KGeq_sub1}~\upref{KGeq}.

\end{example}




\subsection{哈密顿场论}

\pentry{哈密顿正则方程\upref{HamCan}}

哈密顿形式较为特殊.首先,我们回顾一下粒子的情况:

\subsubsection{粒子的哈密顿力学}

粒子的哈密顿函数$H(t, q_r, p_r)$是对拉格朗日函数$\mathcal{L}(t, q_r, \dot{q}_r)$作\textbf{勒让德变换}\upref{TrLgdr},将自变量$\dot{q}_r$都替换为$p_r=\frac{\partial \mathcal{L}}{\partial \dot{q}_r}$得来的:
\begin{equation}
H = \sum_r \dot{q}_rp_r-\mathcal{L}
\end{equation}

粒子的行为由正则方程约束:
\begin{equation}
\leftgroup{
    \dot{q}_r &= \frac{\partial H}{\partial p_r}\\
    \dot{p}_r &= -\frac{\partial H}{\partial q_r}
}
\end{equation}



\subsubsection{场论的哈密顿形式}


现在我们把这一方法推广到场上.然而有一点需要注意,哈密顿量与时间共轭,因此我们得选出一个时间轴,也就是\textbf{选定参考系}来讨论,而不像拉格朗日形式可以脱离参考系讨论.

%参考physics stack exchange 和 Haku的笔记续写



















