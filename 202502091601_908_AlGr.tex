% 亚历山大·格罗滕迪克(综述)
% license CCBYSA3
% type Wiki

本文根据 CC-BY-SA 协议转载翻译自维基百科\href{https://en.wikipedia.org/wiki/Alexander_Grothendieck#Mathematical_work}{相关文章}。

\begin{figure}[ht]
\centering
\includegraphics[width=6cm]{./figures/a1640137f4bbaf24.png}
\caption{1970年,亚历山大·格罗滕迪克在蒙特利尔。} \label{fig_AlGr_1}
\end{figure}
亚历山大·格罗滕迪克(后来的法语名为亚历克斯·格罗滕迪克,发音:/ˈɡroʊtəndiːk/;德语发音:[ˌalɛˈksandɐ ˈɡʁoːtn̩ˌdiːk] ⓘ;法语发音:[ɡʁɔtɛndik]),1928年3月28日出生,2014年11月13日去世,是一位出生于德国的法国数学家,他在现代代数几何的创立中成为了主要人物。他的研究拓展了该领域的范围,并将交换代数、同调代数、层理论和范畴理论等元素融入了其基础中,而他所谓的“相对”视角则在纯数学的许多领域带来了革命性的进展。许多人认为他是二十世纪最伟大的数学家。

格罗滕迪克于1949年开始了他富有成效且公开的数学家生涯。1958年,他被任命为高等科学研究院(IHÉS)的研究教授,并一直在那里工作,直到1970年,由于个人和政治信念,他因与军事资金的争执而离开。1966年,他因在代数几何、同调代数和K理论方面的突破而获得了菲尔兹奖。他后来成为蒙彼利埃大学的教授,并在继续进行相关数学研究的同时,逐渐退出了数学界,投身于政治和宗教事务(最初是佛教,后来转向更为天主教的基督教观点)。1991年,他搬到了位于比利牛斯山脉的法国小村庄拉塞尔,在那里他过上了隐居生活,仍然致力于数学及其哲学和宗教思想,直至2014年去世。