% 永恒的陀螺 (转载)
% license Usr
% type Art


\subsection{你是凭什么想到这个的?}

学物理最困惑我们的是,“你是凭什么想到这个的?”

如果有人能够把物理学家发现的思维过程一步一步给我们展示出来就太好了,这么做的好处,首先是欣赏,欣赏一个大师如何被一个现象吸引、困扰,进而定义问题,做出种种尝试,然后是挫败,接连的挫败,继而是灵感,耐心地尝试,非常接近于成功,然后功亏一窥……

这听起来像是追求异性,上世纪最富天才名声的两位物理学家朗道和费曼就是这么形容的。费曼表示研究物理对他来说就象是性,虽然很少有功利的用途,但又绝对不能缺少。朗道也曾经酸溜溜地表示: “漂亮姑娘都和别人结婚了,现在只能追求一些不太漂亮的姑娘了。”这里漂亮姑娘指的是量子力学。

%(这里我要表达对费曼敬意,因为他在量子力学已经成型的年代,发现了量子力学的路径积分表示和非常直观的图形技术,他比朗道年青,但他确实追到了更迷人的姑娘。)

最好的展现物理思维的场所是讲台,好的讲师都是天才的演员,比如费曼,比如Sidney Coleman。欣赏物理思维的point不是看其如何顺畅地解决问题,相反我们要看的是正在展开思维者是如何掉进他自己挖的坑里,在坑里苦苦挣扎,然后坚强、倔强并且也是聪明地从坑里爬出来。比如杨振宁就曾回忆说他很欣赏他的老师泰勒(Edward Teller)的讲授,泰勒很忙,氢弹之父嘛,他上课不做准备,就是上来现讲,所以常常被挂在讲台上,但对杨振宁来说这正是窥探大师如何思维的绝佳机会。

我们在精心准备好的演讲里,在反复修改的paper里反而不能学习如何思维,用柏拉图的话说这些都属于第二等的知识,它们由规定好的公理、定义出发,剪除无数不成功的路径,顺着已经探索好并修剪过的路径顺势而下,这就类似我们去已经开发好的旅游景点游玩,只是观光,说不上探索。

人思维的倾向可分为两大类,图像的、和语言符号的。前者和人的视觉经验有关,对正常人来说,有超过95\%的信息是通过视觉信息获得的,我们平时看到的山川大地、美形美景都构成了图像思维的基础,或如亚里士多德在《形而上学》开篇中所说:我们总是在看,贪婪地看。

原文是这样的:……在诸感觉中,(人)尤其喜爱视觉……比之于任何事情,我们也更喜欢观看,其理由是,在所有感觉中,视觉最能帮助我们认识事物并揭示事物之间的差别。

就信息的获取来说,人是压倒性地依赖视觉。但人又是社会性的,他们在一起,发生关系,这就必须依赖语言和听觉现象。而要把这些记录下来,超越生命和时代,就需要发明书写的技术,即使用文字和符号来记录。语言/符号思维自然也是重要的思维倾向。

Anne Roe在The Making of a Scientist (New York, 1953)中,曾统计了不同科学领域内学者偏好的思维类型\footnote{摘自普赖斯,《巴比伦以来的科学》}:

\begin{figure}[ht]
\centering
\includegraphics[width=6cm]{./figures/346b8d1d6d0a4928.png}
\caption{请添加图片标题} \label{fig_QMPre2_1}
\end{figure}
这里样本比较小,但已足以说明问题,即物理学家是极其偏向图像思维的,全部实验物理学家(6),和几乎半数的理论物理学家(3/7)都倾向于图像思维。\\


~

比如狄拉克就承认自己非常依赖图像思维,狄拉克的教育不是严格的精英教育,他大学的第一个学位是工程学,作为一名工科生他修习了大量投影几何和工程制图的课,而这些都在牛津或剑桥学生的射程之外,这些教育经历以及他本人供认的对图像思维的依赖应该对他的研究工作有影响。但可惜的是狄拉克是个沉默的人,他不喜欢和别人分享他的科学发现的故事,所以我们无从知道,他的那些伟大发现,比如表象变换、投影算符、空穴(正电子)是如何与栩栩如生地发生在他脑子里的图像关联的,但我们必须承认,就我刚刚提到的这三个例子,我们普通人作为后进往往要借助图像思维才能获得直观的理解。

讽刺的是狄拉克的文风和授课都如水晶般清澈,其经典的《量子力学原理》中没有任何图表。但,我们千万不要被他骗了,他只是在尽力隐藏自己,就像他不愿与人分享自己的工作进展,甚至也不关心其他同行的工作一样。

我们也思考,比如在公交的路上,我就经常一个人陷入沉思,有时觉得idea发展的不错,想通了一些道理,但如果不记录下来,那些道理转瞬就会被我忘掉,而要记录下来,打字或写字的速度显然又跟不上思维的速度。但,胡塞尔\footnote{上世纪数得着的大哲学家,海德格尔的老师,先学物理,然后改数学、心理学,最后聚焦在哲学上,因其写作的风格,是一位极高产的哲学家。}可以,他是个用笔思维的人,他用速记法把他的思维记录下来,这样思维和记录就同步了。

物理学家中玻尔是通过语言思维的典范,他的基本工作方式就是和人聊天,通过聊天了解对方的工作兴趣,同时发展自己的idea。玻尔作为量子力学的早期缔造者,他比海森堡、泡利、狄拉克等稍微年长一些,他是团队(哥本哈根学派)出色的组织者和精神领袖,他和每一个到访的物理学家交谈,通过说话,把他的思路展示出来。玻尔是个喋喋不休的人,想到什么就说什么,甚至会把如何选择词汇的过程大声说出来\footnote{The power of silence. {http://physicsworld.com/cws/article/indepth/2014/apr/03/the-power-of-silence}}。

玻尔在等一个好听众,一个能听懂他说话的人。而海森堡就是这个人(这里真的很微妙,因为我们知道还有《哥本哈根》,也许这两个人真的都太能听懂对方的弦外之音了,都对语言太敏感了)。

在海森堡眼里玻尔用词讲究,精妙措辞的背后有长长的思想在等着他继续挖掘,玻尔的语言暗示着很多哲学反思,但玻尔尚未彻底把它们说清楚。海森堡深深地被这种工作方式打动,他认为他在玻尔这里学会了思维,而在哥廷根他只是学会了计算。