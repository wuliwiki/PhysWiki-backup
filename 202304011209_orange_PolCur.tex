% 极化电流
% 极化电流|电介质|分子电流|磁介质|麦克斯韦方程组

\pentry{磁化强度\upref{MaInte}}

% 未完成

这里推导介质中麦克斯韦方程组的 $\curl \bvec H$ 的那条。

\subsection{磁化电流}
首先证明磁介质产生的磁化电流。 假设磁偶极子都是小线圈组成, 曲面内部净电流为零, 曲面边界只有穿过小线圈才能在曲面上产生净电流。
\begin{equation}
I = I_1 (\pi R^2 \uvec n \vdot \dd{\bvec l}) n = \bvec M \vdot \dd{\bvec l}
\end{equation}~.
即
\begin{equation}
\oint \bvec j_M \vdot \dd{\bvec s} = \oint \bvec M \vdot \dd{\bvec l}
\end{equation}
根据散度定理,
\begin{equation}
\curl \bvec M = \bvec j_M
\end{equation}
证毕。

\subsection{极化电流}
再来看电介质的极化电流。我们已经知道电偶极子会导致极化电荷\upref{ElePAP}:
$$\rho_P=-\div \bvec P$$
因此我们合理地相信,如果某些因素(比如,外电场的变化)改变了电偶极子的密度,那么极化电荷密度也可能变化。
$$\pdv{\rho_P}{t}=-\pdv{\div \bvec P}{t}=-\div \pdv{\bvec P}{t}$$
而根据电荷守恒\upref{ChgCsv},变化的电荷必然对应一个电流
$$\div \bvec j + \pdv{\rho}{t}=0$$
因此
$$
\div \bvec j_P -\div \pdv{\bvec P}{t}=0
$$
即变化的电偶极子产生极化电流。
\begin{equation}
\bvec j_p = \dv{\bvec P}{t}
\end{equation}
当然,就我们的假设\upref{Dielec}而言,介质中的电荷仅仅是在小范围内重新分布,并没有真正地从介质的一端运动到另一端。

\subsection{$\bvec H$的环路定律}
将自由电流、磁化电流、极化电流都带入Maxwell方程的广义安培环路定律\upref{MWEq}
\begin{equation}
\curl {\bvec B} = \mu_0 (\bvec j_f + \bvec j_M + \bvec j_p + \bvec j_E) = \mu_0 \bvec j_f + \mu_0 \curl \bvec M + \mu_0 \pdv{\bvec P}{t} + \mu_0 \epsilon_0 \pdv{\bvec E}{t}
\end{equation}
因此,
\begin{align}
\frac{1}{\mu_0}\curl {\bvec B} &= \bvec j_f + \curl \bvec M + \pdv{\bvec P}{t} + \epsilon_0 \pdv{\bvec E}{t}\\
\frac{1}{\mu_0}\curl {\bvec B} - \curl \bvec M &= \bvec j_f + \pdv{(\epsilon_0\bvec E+\bvec P)}{t}\\
\end{align}
定义\textbf{磁场强度}为
\begin{equation}
\bvec H = \frac{\bvec B}{\mu_0} - \bvec M
\end{equation}
且注意到右侧第二项即为$\bvec D$关于时间的导数。因此
\begin{equation}
\curl \bvec H = \bvec j_f + \pdv{\bvec D}{t}
\end{equation}

\subsection{线性磁介质的假设}
另外, 若假设磁介质为线性,则有
\begin{equation}
\bvec M = \chi_M \bvec H
\qquad
\mu_r = 1 + \chi_M
\qquad
\bvec H = \bvec B/\mu
\qquad
\bvec j_M = \chi_M \bvec j_f
\end{equation}
这就是说, $\chi > 0$ 的磁介质会使电流加强。
