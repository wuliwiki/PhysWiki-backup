% 素理想与极大理想
% 素理想|极大理想|环|环论|理想|素数

\pentry{整环\upref{Domain}}

本节介绍两类极为重要的理想:素理想和极大理想.





\begin{definition}{素理想}
给定\textbf{交换环}$R$,如果$P$是它的一个真理想,并且满足“对于任意$a, b\in R$,如果$ab\in P$,则必有$a\in P$或$b\in P$”,则称$P$是$R$的一个\textbf{素理想(prime ideal)}.
\end{definition}

注意,素理想的定义不一定非得是整环,也可以是含有零因子的交换环.比较一下它和素元素的定义,形式上是完全一样的.

\begin{definition}{极大理想}
给定\textbf{交换环}$R$,如果$M$是它的一个真理想,并且满足“对于任意$R$的理想$I$,如果$M\subsetneq I$,那么$I=R$”,则称$M$是$R$的一个\textbf{极大理想(maximal ideal)}.
\end{definition}

极大理想的定义很直观,任何比极大理想大的理想都只能是$R$本身.

素理想的一大作用,是把零因子收集了起来,使得商环能成为一个整环.同时,素理想和素元素之间高度类似.我们可以把这些话写成如下紧凑的定理:

\begin{theorem}{}\label{Ideals_the1}
给定交换环$R$和它的一个理想$P$,则以下条件等价:
\begin{enumerate}
\item $P$是$R$的素理想.
\item 如果$R$的两个子集$S_1$、$S_2$满足“若$S_1S_2\in P$,则$S_1\in P$或$S_2\in P$”.
\item $R/P$是一个整环.
\end{enumerate}
\end{theorem}

\textbf{证明}:

(1到2):根据素理想的定义和$S_1S_2$的定义,对于任意的$s_1\in S_1$和$s_2\in S_2$,那么$s_1$和$s_2$中必有一个是$P$的元素.如果$S_1\not\subset P$,那么必存在某一个$s_1\in S_1$使得$s_1\not\in P$.这样,任意和这个$s_1$配对的$s_2\in S_2$都必须是$P$的元素了,从而$S_2\subseteq P$.

(2到3):任取$P$的左陪集$S_1$和$S_2$,如果$S_1S_2=P$,那么必有$S_1$和$S_2$的其中之一就是$P$.换句话说,不存在非零(非$P$)的元素相乘等于$0$($P$).从而$R/P$无零因子,加上继承自$R$的交换性,它就是一个整环.

(3到2):而左陪集的运算可以看成任意选取代表元素进行运算后取结果所在的左陪集.对于任意的$a, b\in R$,如果$ab\in P$,那么由于$R/P$是整环,必有$a\in P$或者$b\in P$.这就是素理想的定义.

\textbf{证毕}.

整环的定义已经非常良好了,再进一步,添上“乘法单位元存在性”就能得到域.事实上,这正是极大理想的作用:

\begin{theorem}{}
给定交换环$R$和它的一个理想$M$,则以下两个条件等价:
\begin{enumerate}
\item $M$是$R$的极大理想.
\item $R/M$是一个域.
\end{enumerate}
\end{theorem}

\textbf{证明}:

(1到2):先证明\textbf{极大理想必是素理想}.取$R$的一个子集$S=\{s\in R|\text{存在} r_s\in R-M, \text{使得}r_ss\in M\}$.显然,$M\subseteq S$.取任何$r\in R, s\in S$,$rs$必然在$S$中($s$和$rs$共享一个$r_s$).因此$S$具有吸收律.再考虑到任何$s_1, s_2\in S$,都有$s_1+s_2\in S$,故$S$是$R$的理想.由于$M\subseteq S$是极大理想,可知$S=M$或者$S=R$.由此易得,$R$是素理想.

由于$M$是素理想,由\autoref{Ideals_the1} , $R/M$是整环.

\textbf{证毕}.





