% 光子火箭
% 光子火箭|动能|功率|推力|反物质

\begin{issues}
\issueDraft
\end{issues}

一个光子的能量
\begin{equation}
\hbar\omega = mc^2~.
\end{equation}
一个光子的动量
\begin{equation}
p = mc = \frac{\hbar\omega}{c}~.
\end{equation}
功率和推力的关系
\begin{equation}
F = p \dv{N}{t} = \frac{\hbar\omega}{c}\dv{N}{t} = \frac{P}{c}~,
\end{equation}
所以 $1N$ 的推力需要 $3\e8\Si{W}$ 的功率! 这与光子的能量无关。

如果使用光子火箭, 用反物质湮灭产生的光子推动, 有
\begin{equation}
F = \frac{P}{c} = c\dv{m}{t}~.
\end{equation}
产生 100 吨的推力只需要每秒消耗 3.3 克反物质, 而功率却是惊人的 $3\e{14}\Si{W}$, 可以每秒钟汽化约 $10^5$ 立方米的水(从常温常压到水蒸气)。

\subsection{一般情况}
若使用经典力学, 用于推进的介质喷出速度越大, 同样的推力功率就越大。

功率
\begin{equation}
P = \frac{v^2}{2}\dv{m}{t}~,
\end{equation}
或者
\begin{equation}\label{eq_PhRoc_1}
\dv{m}{t} = \frac{2P}{v^2}~.
\end{equation}
推力
\begin{equation}\label{eq_PhRoc_2}
F = v\dv{m}{t} = \frac{2P}{v}~.
\end{equation}
所以在设计推进器时, \autoref{eq_PhRoc_1} \autoref{eq_PhRoc_2} 会互相制约, 在同样的功率下, 要想推力翻倍, 工质需要消耗原来的 4 倍。

\subsection{火箭加速的完整计算}
在没有引力的环境。 火箭以以相对速度 $u$ 和工质消耗 $\alpha = \dv*{m}{t}$  喷出燃料。
\begin{equation}
F = u\alpha~.
\end{equation}
火箭质量
\begin{equation}
m(t) = M - \alpha t~,
\end{equation}
加速度
\begin{equation}
a(t) = \frac{F}{m(t)} = \frac{u}{M/\alpha - t}~,
\end{equation}
速度
\begin{equation}
v(t) = \int_0^t a(t') \dd{t'} = -u \ln(1 - \alpha t / M)~.
\end{equation}
若火箭质量减去燃料为 $M_0$, 那么时间最大取 $t = (M - M_0)/\alpha$, 最大速度为
\begin{equation}
v_f = u \ln(M / M_0)~,
\end{equation}
或者
\begin{equation}\label{eq_PhRoc_3}
M = M_0\E^{v_f/u}~,
\end{equation}
可见所需燃料随末速度呈指数增长。 注意这与 $\alpha$ 无关。

甚至 $\alpha$ 随时间变化时我们也可以得到\autoref{eq_PhRoc_3} 的结论
\begin{equation}
v = \int_0^t \frac{u\alpha(t')}{m(t')}\dd{t'} = \int_{M_0}^M \frac{u\dd{m}}{m} = u\ln{\frac{M}{m}}~.
\end{equation}
可见 $v$ 是火箭总质量 $m$ 的函数, 无论 $\alpha$ 如何随时间变化。 该式称为\textbf{齐奥尔科夫斯基公式}。
