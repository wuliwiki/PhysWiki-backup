% 泛函分析笔记
% keys 泛函分析|数学分析|空间|Banach 空间|希尔伯特空间

参考书: Applied Functional Analysis application to mathematical physics (Zeidler)
% 另参考 OneNote 和 iMessage 中的笔记

\subsection{Banach 空间}

\begin{itemize}
\item $\mathbb R$ 和 $\mathbb C$ 分别表示实数域和复数域复数域, $\mathbb K$ 表示二者中的一个

\item $\mathbb K^N$ 表示 $N$ 元 \textbf{tuple} $(\xi_1, \xi_2, \dots, \xi_N)$

\item 某区间上的连续函数 $u:[a, b] \to \mathbb{R}$ 可以表示为 $C[a, b]$

\item $\mathbb{K}$ \textbf{上(over $\mathbb K$)}的矢量空间(linear space 就是 vector space) 表示只能以 $\mathbb K$ 的元乘以某个矢量

\item $C[a, b]$ 是无穷维矢量空间

\item 用 $\norm{u} > 0$ 表示\textbf{范数(norm)}

\item 定义了范数的空间就叫\textbf{赋范空间(normed space)}, 满足 (1) $\norm{u} \geqslant 0$, (2) $\norm{u} = 0$ iff $u = 0$, (3) $\norm{\alpha u} = \abs{\alpha} \norm{u}$, (4) $\norm{u + v} \leqslant \norm{u} + \norm{v}$

\item 定义两个矢量之间的 \textbf{distance} 为 $\norm{u - v}$

\item 模可以用于定义极限 $\lim_{n\to\infty} u_n = u$ 为 $\lim_{n\to\infty} \norm{u_n - u} = 0$, 即 $u_n$ 收敛到 $u$

\item 上一条中, (1) $u$ 是唯一的, (2) $u_n$ 是有界的(bounded), (3) $\norm{u_n} \to \norm{u}$, (4) $u_n + v_n \to u + v$, (5) $\alpha_n u_n \to \alpha u$

\item 柯西序列(Cauchy sequence): 对任意 $\varepsilon > 0$, 存在 $N$, 当 $n, m \geqslant N$ 就有 $\norm{u_n - u_m} < \varepsilon$

\item 在赋范空间中, 每个收敛序列都是柯西序列

\item 赋范空间 $X$ 是 Banach 空间当且仅当每个柯西数列都收敛

\item 在 Banach 空间中, 收敛序列都是柯西序列

\item 空间 $X := C[a, b]$ 是实数 Banach 空间, 定义模长为 $\norm{u} := \max_{a \leqslant x \leqslant b} \abs{u(x)}$. $u_n \to u$ 意味着 $\norm{u_n - u} = \max_{a \leqslant x \leqslant b}  \abs{u_n(x) - u(x)} \to 0$. 也就是 $u_n(x)$ 一致收敛到 $u$

\item 如果柯西序列 $u_n$ 的子序列 $u_{n'} \to u$, 那么 $u_n \to u$

\item 若 $\sum_{j=1}^\infty \norm{u_{j+1} - u_j} < \infty$, 那么 $u_n$ 是柯西序列

\item 集合 $U_\varepsilon (u_0) := \{u \in X: \norm{u - u_0} < \varepsilon\}$ 叫做 $u_0$ 的 $\varepsilon$-\textbf{邻域 (neighborhood)}

\item $X$ 的子集 $M$ 是\textbf{开集} 当且仅当对任意 $u \in M$ 都存在属于 $M$ 的邻域

\item $X$ 的子集 $M$ 是\textbf{闭集} 当且仅当每个序列的极限都属于 $M$

\item $X$ 的子集 $M$ 是 closed 当且仅当 $X - M$ 是开的

\item $M$ 和 $Y$ 是集合, $u \in M, v \in Y$ 算符 $A: M \to Y$ 代表映射 $v = Au$, 其中 $M$ 是\textbf{定义域(domain of definition)}, 也记为 $D(A)$. 值域(range) 是 $A(M) := \{v \in Y: v = Au, u \in M\}$, 也记为 $R(A)$

\item $A$ 叫 \textbf{surjective} 当且仅当 $A(M) = Y$, 叫做 \textbf{injective} 当且仅当 $Au = Av$ 意味着 $u = v$, \textbf{bijective} 如果前两者都符合

\item 如果 $A$ 是 bijective, 存在逆算符 $A^{-1}: Y \to M$, 定义为 $A^{-1} v = u$ 当且仅当 $Au = v$

\item 算符也叫函数

\item $A: M \subseteq X \to Y$ 表示 $A: M \to Y$ 且 $M \subseteq X$, 当 $Y = \mathbb K$, 就把 $A$ 叫做\textbf{泛函(functional)}
\end{itemize}

\subsection{1.20 Linear Operators}

\begin{itemize}
\item \textbf{线性算符} $A: L \subseteq X\to Y$ 是线性的当且仅当 $A (\alpha u + \beta v) = \alpha Au + \beta Av$

\item 算符的 \textbf{null space} 为 $N(A) := \qty{u \in X: Au = 0}$

\item 线性算符 $A$ 是连续的, 当且仅当存在 $c > 0$ 使 $\norm{Au} \leqslant c\norm{u}$ 对所有 $u$ 都成立

\item 线性算符是 injective 的当且仅当 $N(A) = {0}$

\item 定义线性连续算符 $A: X \to Y$ 的\textbf{算符范数(operator norm)} 为 $\norm{A} := \sup_{\norm{v} \leqslant 1} \norm{A v}$

\item 当 $X \ne {0}$, 有 $\norm{A} := \sup_{\norm{v} = 1} \norm{A v}$
\end{itemize}

\subsection{1.11 Compactness}
\begin{itemize}
\item 如果赋范空间的集合 $M$ 满足每个序列都有收敛的子序列, 那么 $M$ 就是 \textbf{relatively compact}

\item 如果赋范空间的集合 $M$ 满足每个序列都有收敛的子序列且收敛到 $M$ 中, 那么 $M$ 就是 \textbf{compact}

\item 如果存在 $r \geqslant 0$ 使任意 $u \in M$ 都有 $\norm{u} \leqslant r$, 那么 $M$ 就是有界的

\item $M$ 是 compact 当且仅当它是 relatively compact 且闭合

\item 每个 compact 集都是有界的

\item $\mathbb K^N$ 上的子集若使用 $\norm{u} := \abs{u}_\infty$, 那么它是 relatively compact 当且仅当它是有界的

\item \textbf{Arzela-Ascoli theorem}: 令 $X := C[a, b]$, 且 $\norm{u} := \max_{a\leqslant x\leqslant b}\abs{u(x)}$. 那么若 $M \subseteq X$ 有界且一致连续, 那么 $M$ 就是 relatively compact

\item Weierstrass 定理: 令 $f: M\to \mathbb R$ 为赋范空间中 compact 非空子集 $M$ 上的连续函数, 那么 $f$ 在 $M$ 上存在一个最大值和最小值

\item 令 $X, Y$ 为 $\mathbb K$ 上的赋范空间, 令 $A: M \subseteq X \to Y$ 中 $M$ 为非空 compact, $A$ 为连续算符. 那么 $A$ 是一致连续的.

\item \textbf{finite $\epsilon$ net} 对任意 $\epsilon > 0$, 存在有限多个点 $v_1, \dots, v_J \in M$ 

\item $A: M \subseteq X \to Y$ is call \textbf{compact} iff $A$ is continuous and $A$ transform bound sets into relatively compact sets.

\item operator $(Au)(x) := \int_a^b F(x, y, u(y)) \dd{y}$ is compact, where $-\infty < a < b < \infty$, $u(x)$ is bounded.
\end{itemize}

\subsection{1.13 The Minkowski Functional and Homeomorphisms}
\begin{itemize}
\item Two norms on the normed space $X$ are called \textbf{equivalent} iff there are positive numbers $\alpha$ and $\beta$ such that $\alpha \norm{u} \leqslant \norm{u}_1 \leqslant \beta \norm{u}$ for all $u \in X$

\item Two norms on a finite-dimensional linear space $X$ over $\mathbb K$ are always equivalent.

\item Each finite-dimensional normed space is a Banach space.

\item The points $u_0, \dots, u_N$ are called to be in general position iff $u_1 - u_0, u_2 - u_0,\dots, u_N - u_0$ are linearly independent. This property does not depend on the order.

\item $N$-simplex is $\mathcal S := \opn{co}\qty{u_0, \dots, u_N}$ ($\opn{co}$ is complex hull), where the points are in general position. $0$-simplex is a single point.

\item \textbf{barycenter} of simplex is $b := \sum_{j=0}^N u_j / (N+1)$

\item $k$-face of $\mathcal S$ is the convex hull of $k+1$ distinct vertices of $\mathcal S$

\item $\opn{diam} M := \sup_{u, v\in M} \norm{u - v}$ is the \textbf{diameter} of $M$, and $\opn{dist}(u, M) := \inf_{w \in M} \norm{u - w}$ is the \textbf{distance} of the point $u$ from the point $M$.
\end{itemize}

\subsection{1.14 The Brouwer Fixed-Point Theorem}

\begin{itemize}
\item The continuous operator $A: M \to N$ has a fixed point provided $M$ is a compact, convex, nonempty set in a finite-dimensional normed space over $\mathbb K$.
\end{itemize}

\subsection{1.15 The Schauder Fixed-Point Theorem}
\begin{itemize}
\item The compact operator $A: M \to M$ has a fixed point provided M is a bounded, closed, convex, nonempty subset of a Banach space $X$ over $\mathbb K$.
\end{itemize}

\subsection{1.20 Linear Operators}
\begin{itemize}
\item 有限维矢量空间中的线性算符可以表示为矩阵, 所有这些算符都是连续的

\item 用 $L(X, Y)$ 表示线性连续算符 $A: X \to Y$, $X$ 是 $\mathbb K$ 上的赋范空间, $Y$ 是 $\mathbb K$ 上的 Banach 空间. $L(X, Y)$ 是 $\mathbb K$ 上的 Banach 空间, 范数就是算符的范数
\end{itemize}

\subsection{1.21 The Dual Space}

\begin{itemize}
\item 令 $X$ 为 $\mathbb K$ 上的一个赋范空间, 一个线性的连续算符 $f: X \to \mathbb K$ 称为\textbf{线性连续泛函(linear continuous functional)}

\item 所有 $X$ 上的线性连续泛函叫做 $X$ 的\textbf{对偶空间(dual space)} $X^*$, $X^* = L(X, \mathbb K)$

\item $f \in X^*$ 作用在 $u \in X$ 上可以记为 $\ev{f, u} := f(u)$

\item $f\in X^*$ 的范数为 $\norm{f} := \sup_{\norm{v} \leqslant 1} \abs{f(v)}$, 所以 $\abs{f(u)} \leqslant \norm{f}\norm{u}$

\item 令 $X$ 为 $\mathbb K$ 上的赋范空间, 那么对偶空间 $X^*$ 使用上述范数就是 $\mathbb K$ 上的 Banach 空间.
\end{itemize}

\subsection{1.23 Banach Algebras and Operator Functions}
\begin{itemize}
\item By a \textbf{Banach algebra} $\mathcal B$ over $\mathbb K$ we understand a Banach space over $\mathbb K$, where an additional multiplication $AB$ is defined such that $AB \in \mathcal B$ for all $A, B \in \mathcal B$. More over, for $A, B, C \in \mathcal B$ and $\alpha \in \mathbb K$, $(AB)C = A(BC)$, $A(B+C) = AB + AC$, $(B+C)A = BA + CA$, $\alpha(AB) = (\alpha A)B = A(\alpha B)$, $\norm{AB} \leqslant \norm{A}\norm{B}$. Exist $E \in \mathcal B$ such that $AE = EA$ for all $A \in \mathcal B$ and $\norm{E} = 1$

\item define \textbf{operator function} through $F(A) := \sum_{j=0}^\infty a_j A^j$, and $F(z) = \sum_{j=0}^\infty a_j z^j, z\in \mathbb K$

\item Let $X$ be a Banach space over $\mathbb K$. For each $A \in L(X, X)$ with $\norm{A} < r$, $F(A) \in L(X, X)$
\end{itemize}

\subsection{1.25 Application to the Spectrum}
\begin{itemize}
\item 考虑 $Au = \lambda u, u \in X, \lambda \in \mathbb C$

\item 令 $A\in L(X,X)$, $X$ 是非空的复 Banach 空间.  $\lambda$ 是\textbf{本征值(eigen value)} 当 $u\ne 0$.

\item resolvent set $\rho(A)$: 使得 $(A-\lambda I)^{-1}: X \to X$ 存在且 $\in L(X,X)$. $(A-\lambda I)$ 叫做 resolvent.

\item $\sigma(A) := \mathbb C - \rho(A)$ 叫做 spectrum

\item spectrum 是 $\mathbb C$ 中的 compact 子集且 $\abs{\lambda} \leqslant \norm{A}$ 对所有 $\lambda\in\sigma(A)$ 成立

\item 每个本征值都属于 spectrum

\item resolvent set $\rho(A)$ 是一个开集

\item Banach 空间 $X$ 上的一个算符 $B: X\to X$ 如果值域 $R(B)$ 是闭的且 null space 是有限维的, 那么他就是 \textbf{semi-Fredholm} 的

\item \textbf{essential spectrum} $\sigma_e(A)$ 包括所有使得 $(A - \lambda I)$ 不是 semi-Fredholm 的 $\lambda$
\item $\sigma_e(A) \subseteq \sigma(A)$
\item $\sigma_e(A)$ 就是所有具有无穷简并的本征值的集合
\item 如果 $X$ 是有限维的, 那么 $A$ 的 essential spectrum 是空的 
\end{itemize}

\subsection{1.26 Density and Approximation}
\begin{itemize}
\item $M \subseteq X$ is called \textbf{dense} in $X$ iff $\bar M = X$ where $\bar M$ is the closure of $M$

\item \textbf{countable}, \textbf{at most countable}

\item $X$ is separable iff there is an at most countable dense subset $M \subseteq X$

\item \textbf{Weierstrass approximation theorem}: $X := C[a, b]$, where $-\infty < a < b < \infty$. The set of all polynomials $p(x) := a_0 + a_1 x + \dots$ with real coefficients is dense in $X$

\item $C[a, b]$ is separable.

\item Each finite-dimensional normed space over $\mathbb K$ is separable

\item Let $X$ be a separable normed space over $\mathbb K$. There exist a sequence $\qty{X_n}$ of finite-dimensional linear subspaces $X_n$ of $X$ such that $X_1 \subseteq X_2 \subseteq \dots \subseteq X$ and $\bigcup_{n=1}^\infty X_n = X$

\end{itemize}

\subsection{2.1 Hilbert Spaces}
\begin{itemize}
\item Inner product\upref{InerPd} notation is $(u|v)$, $(u|v) \in K$

\item \textbf{pre-Hilbert space} over $\mathbb K$ is a linear space $X$ over $\mathbb K$ together with an inner product.

\item \textbf{Schwarz inequality}\upref{CSNeq} is the most important inequality in pre-Hilbert and Hilbert spaces.

\item each pre-Hilbert space is also a normed space with the norm $\norm{u} := \sqrt{(u|u)}$

\item A \textbf{Hilbert space} is a pre-Hilbert space that is a Banach space with the norm above
\end{itemize}
