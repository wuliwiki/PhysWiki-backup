% 极限
% keys 微积分|极限|数列极限|函数极限|无穷小

\subsection{数列的极限}
先来看一个数列的例子.

\begin{example}{}
我们都知道 $\pi$ 是一个无理数,所以 $\pi$ 的小数部分是无限多的.目前用计算机,已经可以将 $\pi$ 精确地计算到小数点后数亿位.然而在实际应用中,往往只用取前几位小数的近似即可.下面给出一个数列,定义第 $n$ 项是 $\pi$ 的前 $n$ 位小数近似(不考虑四舍五入),即
\begin{equation}
a_0 = 3,\,\, a_1 = 3.1,\,\, a_2 = 3.14,\,\, a_3 = 3.141,\,\dots.
\end{equation}
\end{example}

这个数列显而易见的性质,就是当 $n$ \textbf{趋于无穷}时,$a_n$ 趋(近)于 $\pi$. 无穷通常用符号 $\infty$ 来表示(像“8”横过来写).我们把这类过程叫做\textbf{极限}.以上这种情况,用极限符号表示,就是
\begin{equation}
\lim_{n \to \infty } {a_n} = \pi 
\end{equation}
这里 $\lim$ 是极限(limit)的意思,下方用箭头表示某个量变化的趋势.$\lim\limits_{n \to \infty }$ 在这里相当于一个“操作”,叫\textbf{算符(operator)}, $a_n$ 是其作用的对象(相当于函数的自变量). 算符的“因变量”就是一个数( $a_n$ 的极限值).所以不要误以为这条式子是说当 $n \to \infty$ 时,$a_n=\pi$ ($a_n$ 是有理数,$\pi$ 是无理数,等式恒不成立),而要理解成 $a_n$ 经过算符 $\lim\limits_{n \to \infty }$ 的作用以后,得出其极限是 $\pi$. 类比函数 $\sin x = y$,并不是说 $x=y$, 而是说 $x$ 经过正弦函数作用后等于 $y$. 

所以从概念上来说,极限中的“趋于” 和“等于” 是不同的.趋于更强调变化的过程.趋于的意思可以粗略理解为
\begin{itemize}
\item 越来越接近,但不一定相等
\item (在不相等的情况下)只有更近,没有最近
\end{itemize}

对极限来说,第2点成立是非常必要的.但是怎样能说明 “没有最近”呢?可以看出,当 $n$ 越大,$a_n$ 越接近 $\pi$, 它们的 “距离”,可以用 $\abs{a_n - \pi}$ 来表示.也就是说,对任何一个 $a_n$, 如果所对应的距离 $\abs{a_n - \pi } \ne 0$, 总能找到一个更大的数 $m>n$, 使 $\abs{a_m - \pi} < \abs{a_n - \pi}$ (更近),并且要求之后的所有项都能满足这一条件.只有这样,才能从数学上说明上面两个意思.这就是极限思想的精髓.根据这个思想,下面可以写出数列极限的定义.这个定义无需硬记,如果理解了上面的描述,就觉得它理所当然了.

% 未完成:数学定义的格式!
\textbf{数列极限的定义}: 对于任意给定的 $\varepsilon$(无论它有多么小),总存在 $N$, 当 $n>N$ 时,就有 $\abs{a_n - A} < \varepsilon$ ($A$为常数) 成立,那么数列 $a_n$ 的极限就是 $A$. 

在命题中,通常把 “任意” 用 “ $\forall$” (any) 表示,把 “存在” 用 “$\exists $” (exist)表示.即“ 对 $\forall \varepsilon$, $\exists N$, 当 $n>N$ 时,有 $\abs{a_n - A} < \varepsilon$”. 

由于以上讨论中 lim 作用的对象是数列,那么箭头右边只能是 $\infty$ (准确来说应该是正无穷 $+\infty$, 但是由于数列的项一般是正的,所以正号省略了).

把定义套用到上面的例一中, 如果要求 $\abs{a_n - \pi} < 10^{-3}$ (给定 $\varepsilon  = 10^{-3}$),只要令 $N=3$ (当然也可以令 $N=4, N=5$, 等) 就可以保证第 $N$ 项后面所有的项都满足要求. 一般地如果给定 $\varepsilon  = b \times 10^{-q}  (b > 1)$, 就令 $N = q$, 第 $N$ 项以后的项就满足要求.这就从定义直接证明了 $\lim\limits_{n \to \infty } a_n = \pi$. 

\subsection{函数的极限}
类比数列的极限, 我们也可以定义\textbf{函数在正无穷的极限} $\lim\limits_{x\to +\infty} f(x) = A$, 即“ 对 $\forall \varepsilon$, $\exists X$, 当 $x>X$ 时,有 $\abs{f(x) - A} < \varepsilon$”.

与数列不同的是, 对于函数我们还可以定义\textbf{函数在负无穷的极限} $\lim\limits_{x\to -\infty} f(x)$(把以上定义的 $>$ 号改成 $<$ 号即可).

另外可以定义\textbf{$f(x)$ 在 $x_0$ 处的极限} $A$, 即“ 对 $\forall \varepsilon$, $\exists \delta$, 当 $\abs{x - x_0} < \delta$ 时,有 $\abs{f(x) - A} < \varepsilon$”.

\begin{example}{}
求函数在某个值处的极限时, 通常可以直接代入数值计算, 如
\begin{equation}
\lim_{x\to 1} 2x + 1 = 3 \qquad \lim_{x\to 2}\frac{x + 1}{x + 2} = \frac34
\end{equation}

当无穷大与常数相加时, 可以忽略常数, 如
\begin{equation}
\lim_{x\to +\infty} \frac{x + 1}{2x + 2} = \lim_{x\to +\infty} \frac{x}{2x} = \frac12
\end{equation}
\end{example}

\subsection{无穷小的阶}
如果令 $x\to 0$, 我们就说 $x$ 是\textbf{无穷小}. 但一些无穷小会更快地趋近于 $0$, 若 $x$ 的某个函数 $\alpha(x)$ 满足
\begin{equation}
\lim_{x\to 0} \frac{\alpha(x)}{x} = 0
\end{equation}
那 $\alpha(x)$ 就是 $x$ 的\textbf{高阶无穷小}. 若
\begin{equation}
\lim_{x\to 0} \frac{\alpha(x)}{x^n} \ne 0
\end{equation}
则称 $\alpha(x)$ 为 $x$ 的 $n$ 阶无穷小. 例如, $c x^n$ ($c$ 为常数)就是 $x$ 的 \textbf{$n$ 阶无穷小}, 记为 $\order{x^n}$.

在求极限时, 若高阶无穷小与低阶无穷小相加, 通常可以忽略高阶无穷小. 另外由定义不难推出
\begin{equation}
\order{x^n} x^m = \order{x^{n + m}} \qquad (m > -n)
\end{equation}

在物理中, 当我们用一个函数 $g(x)$ 来近似另一个函数 $f(x)$ 并记为 $f(x) = g(x) + \order{h^{n + 1}}$ 时(这里 $x$ 是函数的自变量, $h$ 是函数表达式中一个较小的常数), 就说 $g(x)$ 精确到 $\order{h^n}$.



