% 永恒的陀螺 (转载)
% license Usr
% type Art


\subsection{你是凭什么想到这个的?}

学物理最困惑我们的是,“你是凭什么想到这个的?”

如果有人能够把物理学家发现的思维过程一步一步给我们展示出来就太好了,这么做的好处,首先是欣赏,欣赏一个大师如何被一个现象吸引、困扰,进而定义问题,做出种种尝试,然后是挫败,接连的挫败,继而是灵感,耐心地尝试,非常接近于成功,然后功亏一窥……

这听起来像是追求异性,上世纪最富天才名声的两位物理学家朗道和费曼就是这么形容的。费曼表示研究物理对他来说就象是性,虽然很少有功利的用途,但又绝对不能缺少。朗道也曾经酸溜溜地表示: “漂亮姑娘都和别人结婚了,现在只能追求一些不太漂亮的姑娘了。”这里漂亮姑娘指的是量子力学。

%(这里我要表达对费曼敬意,因为他在量子力学已经成型的年代,发现了量子力学的路径积分表示和非常直观的图形技术,他比朗道年青,但他确实追到了更迷人的姑娘。)

最好的展现物理思维的场所是讲台,好的讲师都是天才的演员,比如费曼,比如Sidney Coleman。欣赏物理思维的point不是看其如何顺畅地解决问题,相反我们要看的是正在展开思维者是如何掉进他自己挖的坑里,在坑里苦苦挣扎,然后坚强、倔强并且也是聪明地从坑里爬出来。比如杨振宁就曾回忆说他很欣赏他的老师泰勒(Edward Teller)的讲授,泰勒很忙,氢弹之父嘛,他上课不做准备,就是上来现讲,所以常常被挂在讲台上,但对杨振宁来说这正是窥探大师如何思维的绝佳机会。

我们在精心准备好的演讲里,在反复修改的paper里反而不能学习如何思维,用柏拉图的话说这些都属于第二等的知识,它们由规定好的公理、定义出发,剪除无数不成功的路径,顺着已经探索好并修剪过的路径顺势而下,这就类似我们去已经开发好的旅游景点游玩,只是观光,说不上探索。

人思维的倾向可分为两大类,图像的、和语言符号的。前者和人的视觉经验有关,对正常人来说,有超过95\%的信息是通过视觉信息获得的,我们平时看到的山川大地、美形美景都构成了图像思维的基础,或如亚里士多德在《形而上学》开篇中所说:我们总是在看,贪婪地看。

原文是这样的:……在诸感觉中,(人)尤其喜爱视觉……比之于任何事情,我们也更喜欢观看,其理由是,在所有感觉中,视觉最能帮助我们认识事物并揭示事物之间的差别。

就信息的获取来说,人是压倒性地依赖视觉。但人又是社会性的,他们在一起,发生关系,这就必须依赖语言和听觉现象。而要把这些记录下来,超越生命和时代,就需要发明书写的技术,即使用文字和符号来记录。语言/符号思维自然也是重要的思维倾向。

Anne Roe在The Making of a Scientist (New York, 1953)中,曾统计了不同科学领域内学者偏好的思维类型\footnote{摘自普赖斯,《巴比伦以来的科学》}: