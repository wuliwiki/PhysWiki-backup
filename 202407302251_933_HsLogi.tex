% 数理逻辑(高中)
% keys 逻辑|高中
% license Usr
% type Tutor

\begin{issues}
\issueDraft
\end{issues}
\pentry{集合\nref{nod_HsSet}}{nod_fc7f}

尽管高中数学教材中在逐渐弱化逻辑的概念,但作为数学根基的内容,学习逻辑对于理解高中的内容,不仅是数学学科,对于理科内容甚至是文科内容都具有相当的助益。同时,这一部分内容在本科学习阶段往往也会作为学生已知的部分略讲或跳过。因此,在高中阶段接触和学习一部分逻辑内容是必要的。

\textbf{逻辑}是研究如何进行正确推理和论证的学科。逻辑能帮助人们理解和判断,哪些推理是合理的,哪些推理是错误的,在数学中,逻辑用于证明定理,确保论证的严谨性和准确性。\textbf{数理逻辑}是逻辑学的一个分支,它应用数学的方法研究逻辑。


\subsection{命题}

一个可以明确判断真假的陈述句被称为\textbf{命题}(proposition)。每个命题都有一个确定的\textbf{真值}(truth value),真值用\textbf{布尔值}表示。布尔值可以是“真”(True)或“假”(False),用符号表示的话,“$1$”或者“$T$”代表真,“$0$”或者“$F$”代表假。

如果一个命题里不包含变量,那么这个命题被称为\textbf{陈述命题}(propositional statement)或\textbf{简单命题}(atomic statement)。这种命题直接陈述一个事实,并且它的真假值是固定的。例如,“2 是一个偶数”是一个陈述命题,它总是真。

如果一个命题包含变量,那么这个命题被称为\textbf{开放命题}(open proposition)。这种命题的真假值取决于变量的取值。例如,“x 是一个偶数”就是一个含变量的命题,其中 x 可以取不同的值,因此命题的真假值也会随之变化。如果 x=2,命题为真;如果 x=3,命题为假。



比如,“2 是一个偶数”是一个命题,它的布尔值是真(True);而“3 是一个偶数”也是一个命题,但它的布尔值是假(False)。

\subsubsection{特殊的命题}
定义:
公理:
悖论:

\subsection{逻辑连接词}

\subsubsection{且}

\textbf{且}(and,也称为同)记号为$A\land B$。

\subsubsection{或}

\textbf{或}(or,也称为或者)记号为$A\lor B$。


\subsubsection{非}
\textbf{非}(not,也称为同)记号为$\lnot A$。

\subsection{量词}

\subsubsection{全称量词}

$\forall$

\subsubsection{存在量词}

$\exists$

\subsection{条件}
如果那么 当且仅当
\subsection{演绎与归纳}

\subsubsection{演绎}

\subsubsection{归纳}