% 麦克斯韦关系
% keys 热力学|态函数

\pentry{热力学第一定律\upref{Th1Law}}

\subsection{麦氏关系}

\begin{align}
&\left(\frac{\partial T}{\partial V}\right)_S=-\left(\frac{\partial p}{\partial S}\right)_V\label{MWRel_eq1}
\\
&\left(\frac{\partial T}{\partial p}\right)_S=\left(\frac{\partial V}{\partial S}\right)_p
\\
&\left(\frac{\partial S}{\partial V}\right)_T=\left(\frac{\partial p}{\partial T}\right)_V
\\
&\left(\frac{\partial S}{\partial p}\right)_T=-\left(\frac{\partial V}{\partial S}\right)_V
\end{align}

\subsection{推导}
根据 $\dd U=T\dd S-p\dd V$,可得 $\Big(\partial U/\partial S\Big)_V=T$,$\Big(\partial U/\partial V\Big)_S=-p$.再根据 $\partial^2 U/(\partial V\partial S)=\partial^2 U/(\partial S\partial V)$,就可以推出 \autoref{MWRel_eq1}.

对热力学函数焓 $H=U+pV$,自由能 $F=U-TS$,吉布斯函数 $G=U-TS+pV$ 都可以列出微分表达式,于是类似地可以推出剩余 $3$ 个麦克斯韦关系.

\begin{example}{内能方程}
对于某个热力学系统(例如某种气体),我们想知道在恒温条件下,它的内能随体积的改变会如何变化.这个物理量在实验上不能直接测量(测量系统吸收了多少热量是很困难的),但我们可以期待将它用其他物理量表达出来.

\begin{equation}
\dd U=T\dd S-p\dd V
\end{equation}

其中 $T\dd S$ 也就是说

\end{example}
