% 莫尔斯引理

\begin{issues}
\issueTODO
\issueDraft
\end{issues}

\pentry{巴拿赫不动点定理 \upref{ConMap} 隐函数定理的不动点证明 \upref{IFTFix}}

莫尔斯引理在微分几何学中占有基本地位。 粗略地说, 它表示: 一个光滑函数在其非退化临界点附近的表现, 完全由其在这一点处的二阶导数决定。

\subsection{表述与直观}
\begin{lemma}{莫尔斯引理}
设$n$元光滑函数$f$定义在坐标原点附近, 满足$f'(0)=0$. 如果 Hessian 矩阵
$$
f''(0)=\left(\frac{\partial^2f}{\partial x^i\partial x^j}(0)\right)
$$
是可逆的, 则在坐标原点的某个小邻域上, 存在微分同胚$\varphi(x)$, 使得
$$
(f\circ\varphi)(x)=f(0)+\frac{1}{2}\langle f''(0)x,x\rangle.
$$
\end{lemma}

我们注意到, $f(0)+\frac{1}{2}\langle f''(0)x,x\rangle$正是$f$的泰勒展开截断到二阶的近似表达式. 因此, 莫尔斯引理说明: 在非退化临界点附近, 可以找到一个坐标变换, 使得$f$的表达式在新坐标之下与$f$的二阶近似重合.

\subsection{证明}
设 Hessian 矩阵$f''(0)=(h_{ij})$. 将新坐标$\varphi^{-1}(x)$记作$y$, 从而将方程重写为
\begin{equation}\label{Morse_eq1}
f(x)=f(0)+\frac{1}{2}\sum_{i,j=1}^nh_{ij}y^iy^j.
\end{equation}
我们希望新坐标$y$是二阶接近旧坐标$x$的, 所以可以设
$$
y^k
=x^k+\sum_{i,j=1}^nA^k_{ij}(x)x^ix^j,
$$
其中$A_{ij}(x)$是待求解的光滑函数. 将$f(x)$进行泰勒展开到三阶, 得到
$$
f(x)=f(0)+\frac{1}{2}\sum_{i,j=1}^nh_{ij}x^ix^j+
\sum_{i,j,k=1}^nR_{ijk}(x)x^ix^jx^k
$$
其中最后一项是积分形式的余项给出的, $R_{ijk}(x)$都是已知的光滑函数. 于是, 将$y$代入\autoref{Morse_eq1} 右边, 将$f(x)$的泰勒展开式代入\autoref{Morse_eq1} 左边, \autoref{Morse_eq1} 就化成了
$$
\sum_{i,j,k,l=1}^nh_{kl}A^k_{ij}(x)x^ix^jx^k
=\sum_{i,j,k=1}^nR_{ijk}(x)x^ix^jx^k
-\frac{1}{2}\sum_{i,j,k,l=1}^n\sum_{p,q=1}^nh_{pq}A^p_{ij}(x)A^q_{kl}(x)x^ix^jx^kx^l.
$$
如果对比$x^ix^jx^k$的系数, 会发现只需要求解如下关于$A(x)$的方程:
\begin{equation}\label{Morse_eq2}
\sum_{l=1}^nh_{kl}A^k_{ij}(x)
=R_{ijk}(x)
-\frac{1}{2}\sum_{l=1}^n\sum_{p,q=1}^nh_{pq}A^p_{ij}(x)A^q_{kl}(x)x^l.
\end{equation}
到目前为止还没有用到$(h_{ij})$是非退化矩阵. 现在要用到它了. 将$(h_{ij})$的逆矩阵写为$(c_{ij})$, 同时记$B_{ijk}(x)=\sum_{l=1}^nh_{kl}A^k_{ij}(x)$, 则\autoref{Morse_eq2} 就等价于
$$
B_{ijk}(x)
=R_{ijk}(x)-\frac{1}{2}\sum_{l=1}^n\sum_{p,q=1}^nc_{pq}B_{ijp}(x)B_{klq}(x)x^l.
$$
这已经是关于未知量$[B_{ijk}(x)]_{i,j,k=1}^n$的不动点型方程了, 对于接近$0$的$x$, 映射
$$
B_{ijk}\to R_{ijk}(x)-\frac{1}{2}\sum_{l=1}^n\sum_{p,q=1}^nc_{pq}B_{ijp}B_{klq}x^l
$$
自然是压缩映射, 因此关于$[B_{ijk}(x)]_{i,j,k=1}^n$的方程会有唯一一个接近$[R_{ijk}(x)]_{i,j,k=1}^n$的解. 这就给出了满足条件的新坐标
$$
y^k
=x^k+\sum_{i,j,l=1}^nc^{kl}B_{ijl}(x)x^ix^j.
$$ 