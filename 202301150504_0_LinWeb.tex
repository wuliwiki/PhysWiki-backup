% Linux 网络笔记

\begin{issues}
\issueDraft
\end{issues}

\begin{itemize}
\item \verb`sudo ifconfig` 显示网卡信息, 以及 ip 地址 (inet addr)
\item \verb|ip -4 addr show scope global| 也可以(有什么不同?)
\item 不同的网卡有不同的 ip 地址, 通常有一张网卡是连互联网的, 其他网卡都是局域网, 甚至虚拟网卡(例如虚拟机, docker 等)。 每个局域网的 ip 范围由子网掩码决定。 对这些不同的局域网来说, 当前电脑就是一个\textbf{网关(gateway)}把它们连起来。
\item 可以想象每张网卡上都长了很多\textbf{端口(port)}, 用于给不同的应用收发各种消息。 例如 http 协议的默认端口是 80, https 是 443
\item 要找到公网网卡, 用 \verb`ip route show | grep default` 找默认网卡, 默认网卡几乎肯定是公网网卡。
\item \verb`ifconfig 网卡名 up` 启动网卡, \verb`ifconfig 网卡名 down` 关闭网卡, 可以用这两个命令重启网卡
\item \verb`ifoncig` 用于查看网络相关信息: 其中 \verb`Link encap:Ethernet` 表示在使用 Ethernet, \verb`HWaddr` 是硬件地址, 即 MAC 地址, \verb`inet addr` 本机的 IP 地址, \verb`Bcast` broadcast 地址, \verb`Mask` 掩码, \verb`UP` 表示 Ethernet 的 kernel module 被加载, \verb`BROADCAST` 支持 broadcasting, 从 DHCP 获取 IP 的必要条件, \verb`RUNNING` 准备好传输数据, \verb`mtu` (Maximum Transmission Unit), 就是 packet 的最大尺寸
\item \verb`netplan` 是 ubuntu 18.04 的默认管理网络设置的程序, 比如设置 hdcp, 静态 ip, 掩码, 网关等, 设置完成以后用 \verb`sudo netplan apply` 可以生效。 但有时候还需要重启网卡才能成功。
\item \verb`sudo hpclient` 显示 ip 地址
\item \verb`sudo /etc/init.d/networking restart` 重启网络服务 
\item 要从某个网址下载文件, 只需安装 wget 软件即可, 使用方法如 \verb`wget http://...` 文件会下载到当前文件夹
\item \verb`ping` 可以用来检查是否有网络连接, 比如 \verb`ping google.com` 也可以用来查看某个域名的 ip 地址, 也可以直接使用 ip 地址如 \verb`ping 8.8.8.8` (8.8.8.8 是谷歌的主要 DNS 服务器)
\item \verb`ping 域名` 和 \verb|host 域名| 都可以检查域名的 ip
\item 如果连不上网, 可以参考\href{https://upcloud.com/community/tutorials/troubleshoot-network-connectivity-linux-server/}{这篇文章}的步骤调试
\item 如果想 ping 某个端口, 用 \verb|telnet IP 端口号|
\item 如果想获得某个 ip 的子网掩码, 用 \verb`whois 66.220.156.68 | grep CIDR` 输出如 \verb|CIDR: 66.220.144.0/20|
\item \verb|traceroute 域名或者ip| 可以查看从当前机器出发到某个机器所经过的网关的 ip (有些会隐藏)。 例如局域网里面的某台机器 traceroute 到外网的某个网址, 那么第一条显示的是当前局域网的网关(用于该局域网的网卡)的 ip, 如果该网关仍然没有处于公网, 那么第二条会显示第一条中网关所在的另一个局域网的网关的 ip…… 直到公网 ip, 然后各种互联网上的公网 ip, 最终达到网址的服务器的 ip。 注意每条中只会显示进入某个机器所用网卡的 ip, 不会显示出去时所用网卡的 ip。
\end{itemize}

\subsection{iptables}
\begin{itemize}
\item 参考 \href{https://www.tecmint.com/linux-iptables-firewall-rules-examples-commands/}{25 Useful IPtable Firewall Rules Every Linux Administrator Should Know}。
\item \verb|iptables| 是一个命令行防火墙, 它不是一个 service 而是个命令, 所以不能 turn on/off。
\item 但是可以关闭 ufw 服务: \verb|sudo ufw status|, \verb|sudo ufw disable|, \verb|sudo ufw enable| (其实我需要的只是 disable)
\item table 的种类有 \verb|FILTER|, \verb|NAT|, \verb|MANGLE|, 每种又有不同的 \textbf{chain}, 例如 \verb|INPUT|, \verb|OUTPUT|, \verb|FORWARD|,  \verb|PREROUTING|, \verb|POSTROUTING| 等。
\item 查看所有防火墙规则 \verb|iptables -L -n -v|
\item 查看某个 table 的规则用 \verb|-t|, 如 \verb|iptables -t nat -L -v -n|
\item 屏蔽某个 ip: \verb|iptables -A INPUT -s xxx.xxx.xxx.xxx -j DROP|
\item 屏蔽某个 ip 的 tcp traffic: \verb|iptables -A INPUT -p tcp -s xxx.xxx.xxx.xxx -j DROP|
\item 取消屏蔽 ip: \verb|iptables -D INPUT -s xxx.xxx.xxx.xxx -j DROP|
\item 屏蔽一个 OUTPUT 端口: \verb|iptables -A OUTPUT -p tcp --dport xxx -j DROP|
\item 允许一个 INPUT 端口:  \verb|iptables -A INPUT  -p tcp --dport xxx -j ACCEPT|
\item 允许多个 INPUT 端口 \verb|iptables -A INPUT  -p tcp -m multiport --dports 22,80,443 -j ACCEPT|
\item 允许多个 OUTPUT 端口 \verb|iptables -A OUTPUT -p tcp -m multiport --sports 22,80,443 -j ACCEPT|
\item 屏蔽 facebook: \verb|host facebook.com|, \verb`whois 66.220.156.68 | grep CIDR`, 例如得到 \verb|66.220.144.0/20|, 那么 \verb|iptables -A OUTPUT -p tcp -d 66.220.144.0/20 -j DROP|
\item Port Forwarding: \verb|iptables -t nat -A PREROUTING -i eth0 -p tcp --dport 25 -j REDIRECT --to-port 2525| 就是网卡 eth0, 从端口 25 到端口 2525
\item 屏蔽某个 mac 地址 \verb|iptables -A INPUT -m mac --mac-source 00:00:00:00:00:00 -j DROP|
\end{itemize}

\begin{itemize}
\item 另外参考一个\href{https://www.digitalocean.com/community/tutorials/how-to-forward-ports-through-a-linux-gateway-with-iptables}{具体教程}, 讲解如何设置网关(文章中称为防火墙)的两个网卡间的 port forwarding 可以让外网访问内网的 nginx 服务器。
\end{itemize}

\subsection{NAT}
\begin{itemize}
\item \verb|NAT| 的原理大概就是把不同内网的 ip 地址 + 端口 (socket) 映射到某个(例如学校的)公网 ip 地址和不同的端口, 并在链接建立以后把后者的端口回传给前者的端口
\item 所以两个不同的 NAT 后面的电脑是可以 P2P 连接的, 但是首先要通过一个公网服务器建立连接, 这几乎是常规操作了(猜测 teamviewer 应该就是这个原理)
\end{itemize}
