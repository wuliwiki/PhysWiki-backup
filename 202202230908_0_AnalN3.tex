% 数学分析笔记 3

本文参考 \cite{Rudin}.

\subsection{Chap 9. 多元函数}

\begin{itemize}
\item 9.1 (a) 向量空间 (b) 线性组合; 若 $S \subset R^n$, $E$ 是 $S$ 内元素的所有线性组合的集, 就说 $S$ \textbf{生成} $E$. (c) 线性无关 (d) 维度 (e) 基; 坐标

\item 9.3 基底和线性无关向量的一些定理

\item 9.4 线性变换(线性算子)

\item 9.5 线性算子是 1-1 的当且仅当值域是定义域

\item 9.6 (a) $L(X,Y)$ 代表所有 $X$ 到 $Y$ 的线性变换构成的集. $L(X,X)$ 写成 $L(X)$. 线性变换的线性组合. (b) 线性变换的乘积. (c) 范数 $\norm{A}$ 为所有数 $\abs{A\bvec x}$ 的最小上界, 其中 $x$ 取遍 $R^n$ 中所有 $\abs{\bvec x}\leqslant 1$ 的向量. 对 $x\in R^n$ 有不等式 $\abs{A\bvec x}\leqslant \norm{A}\abs{\bvec x}$

\item 9.7 (a) $A\in L(R^n,R^m)$, 则 $\norm{A}<\infty$ 且 $A$ 是一致连续映射. (b) $\norm{A+B}\leqslant \norm{A}+\norm{B}$, $\norm{cA} = \abs{c}\norm{A}$. 以 $\norm{A-B}$ 作为距离, 那么 $L(R^n,R^m)$ 就是一个度量空间. (c) $\norm{BA} \leqslant \norm{B}\norm{A}$

\item 9.8 设 $\Omega$ 为 $R^n$ 上所有可逆线性算子的集合. (a) 若 $A\in\Omega$, $B\in L(R^n)$, 而且 $\norm{B-A}\cdot\norm{A^{-1}}<1$, 则 $B\in \Omega$. (b) $\Omega$ 是 $L(R^n)$ 的开子集, 映射 $A\to A^{-1}$ 在 $\Omega$ 上是连续的.

\item 9.9 \textbf{矩阵}: $A\bvec x_j = \sum_{i=1}^m a_{ij}\bvec y_i$ ($1\leqslant j\leqslant n$); $\norm{A}\leqslant \qty{\sum_{i,j}a_{ij}^2}^{1/2}$

\item 9.11 设 $E$ 是 $R^n$ 中的开集, $\bvec f: E\to R^m$, $x\in E$. 如果存在把 $R^n$ 映入 $R^m$ 的线性变换 $A$, 使得 $\lim_{\bvec h\to 0} \abs{\bvec f(\bvec x+\bvec h) - \bvec f(\bvec x) - A\bvec h}/{\abs{\bvec h}} = 0$, 就说 $\bvec f$ 在 $\bvec x$ 处\textbf{可微}, 并写成\footnote{$A$ 的矩阵就是雅可比矩阵} $\bvec f'(\bvec x) = A$. (注: $A_{ij} = \pdv*{f_i}{x_j}$)

\item 一元函数的可导和可微等价.

\item 9.13 上面的极限能被写成 $\bvec f(\bvec x+\bvec h) - \bvec f(\bvec x) = \bvec f'(\bvec x)\bvec h + \bvec r(\bvec h)$, 其中余项 $\bvec r(\bvec h)$ 满足 $\lim_{\bvec h\to0} \abs{\bvec r(\bvec h)}/\abs{\bvec h} = 0$.

\item 9.16 \textbf{偏导数}

\item 9.17 若 $\bvec f$ 在点 $x\in E$ 可微, 那么偏导数 $D_jf_i(x)$ 存在, 且 $\bvec f'(\bvec x)\bvec e_j = \sum_{i=1}^m (D_jf_i)(\bvec x)\bvec u_i$ ($1\leqslant j\leqslant n$)

\item 9.18 例: 设 $\gamma$ 是把开区间 $(a,b)\subset R^1$ 映入开集 $E\subset R^n$ 内的可微映射, 即 $\gamma$ 是 $E$ 内的可微曲线. 令 $f$ 为域 $E$ 上的实值可微函数. 于是 $f$ 是从 $E$ 到 $R^1$ 内的可微映射. 定义 $g(t) = f(\gamma(t))$ ($a< t<b$). 于是由链式法则得到 $g'(t) = \bvec f'(\gamma(t))\gamma'(t)$ ($a<t<b$).

\item \textbf{梯度} $(\grad f)(\bvec x) = \sum_{i=1}^n (D_i f)(\bvec x)\bvec e_i$

\item 9.20 设 $\bvec f$ 是开集 $E\subset R^n$ 到 $R^m$ 内的可微映射. 如果 $\bvec f'$ 是把 $E$ 映入 $L(R^n,R^m)$ 的连续映射, 就说 $\bvec f$ 是在 $E$ 内\textbf{连续可微}的. 更明确地, 它要求对每个 $x\in E$ 以及 $\epsilon > 0$, 存在 $\delta >0$, 使当 $y\in E$ 以及 $\abs{\bvec x-\bvec y}<\delta$ 时, $\norm{\bvec f'(\bvec y)-\bvec f'(\bvec x)} < \epsilon$. 我们也说 $\bvec f$ 是 $\mathscr C'$ 映射, 或者 $\bvec f\in \mathscr'(E)$.

\item 9.21 设 $\bvec f$ 把开集 $E\subset R^n$ 映入 $R^m$ 内. 那么当且仅当 $\bvec f$ 的所有偏导数 $D_jf_i$ ($i\leqslant i\leqslant m, 1\leqslant j\leqslant n$)在 $E$ 上都存在并且连续时, $\bvec f\in \mathscr C'(E)$.

\item 9.22 设 $X$ 是度量为 $d$ 的度量空间. 如果 $\varphi:X\to X$, 并且存在 $c<1$, 对一切 $x,y\in X$, 使得 $d(\varphi(x),\varphi(y))\leqslant cd(x,y)$, 那么, 就说 $\varphi$ 是 $X$ 到 $X$ 内的一个\textbf{凝缩函数}.

\item 9.23 如果 $X$ 是完备度量空间, $\varphi$ 是 $X$ 到 $X$ 内的凝缩函数, 那么存在唯一满足 $\varphi(x)=x$ 的 $x\in X$. 也就是说 $\varphi$ 有唯一的不动点.

\item 粗略地说, 反函数定理说的是, 一个连续可微映射 $\bvec f$, 在使线性变换 $\bvec f'(\bvec x)$ 可逆的点 $\bvec x$ 的邻域内是可逆的. (且反函数也是连续可微的)

\item 9.24 设 $\bvec f$ 把开集 $E\subset R^n$ 映入 $R^m$ 内的 $\mathscr C'$ 映射, 对某个 $a\in E$, $\bvec f'(\bvec a)$ 可逆, 且 $\bvec b = \bvec f(\bvec a)$. 那么 (a) 在 $R^n$ 内存在开集 $U$ 和 $V$, 使得 $a\in U, b\in V$ , $\bvec f$ 在 $U$ 上是 1-1 的, 并且 $f(U) = V$; (b) 若 $\bvec g$ 是 $\bvec f$ 的逆, 他在 $V$ 内由 $\bvec g(f(\bvec x)) = x$ ($\bvec x\in U$) 确定, 那么 $\bvec g\in \mathscr C'(V)$.

\item 9.28 \textbf{隐函数定理}: 

\item 9.38 如果 $\bvec f$ 把开集 $E\subset R^n$ 映入 $R^n$ 内, 且 $\bvec f$ 在点 $\bvec x\in E$ 可微, $\bvec f'(\bvec x)$ 的行列式就叫做 $\bvec f$ 在 $\bvec x$ 的\textbf{雅可比行列式(Jacobian)}. 记为 $J_f(\bvec x) = \det f'(\bvec x)$. 如果 $(y_1,\dots,y_n)=\bvec f(x_1,\dots, x_n)$, 我们又把 $J_f(\bvec x)$ 记为 $\pdv*{(y_1,\dots,y_n)}{(x_1,\dots, x_n)}$.
\end{itemize}

\subsection{Chap 10. 微分形式的积分}

\begin{itemize}
\item 10.1 设 $I^k$ 是 $R^k$ 中的 $k$-方格, 它由满足 $a_i\leqslant x_i \leqslant b_i$($i=1,\dots,k$) 的一切 $\bvec x=(x_1,\dots,x_k)$ 组成, $I^j$ 是 $R^j$ 中的 $j$-方格, 它由前 $j$ 个不等式来确定, $f$ 是 $I^k$ 上的连续函数. 令 $f = f_k$, 而用下式定义 $I^{k-1}$ 上的函数 $f_{k-1}$: $f_{k-1}(x_1,\dots,x_{k-1}) = \int_{a_k}^{b_k} f_k(x_1,\dots,x_{k-1},x_k)\dd{x_k}$. $f_k$ 在 $I^k$ 上的一致连续性表明 $f_{k-1}$ 在 $I^{k-1}$ 上连续. 因此, 我们能够重复应用这种手续, 得到 $I^j$ 上的连续函数 $f_j$,…… $k$ 步以后, 就能得到一个数 $f_0$, 我们就把这个数叫做 $f$ 在 $I^k$ 上的积分, 并写成 $\int_{I^k} f(\bvec x)\dd{\bvec x}$ 或者 $\int_{I^k} f$.

\item 10.2 对每个 $f\in \mathscr C(I^k)$, 积分结果与顺序无关.

\item 10.3 $R^k$ 上一个(实或复)函数 $f$ 的\textbf{支集(support)}, 是使 $f(x)\ne 0$ 的一切点集的闭包. 如果 $f$ 是带有紧支集的连续函数, 令 $I^k$ 是含有 $f$ 的支集的任意 $k$-方格, 并定义 $\int_{R^k}f = \int_{I^k}f$. 这样定义的积分显然与 $I^k$ 的选择无关.

\item 10.4 令 $Q^k$ 是由 $R^k$ 中符合 $x_1+\dots+x_k$ ($i=1,\dots,k$) 的一切点 $\bvec x=(x_1,\dots,x_k)$ 组成的 $k$-单形. 

\item 10.5 \textbf{本源映射}: $\bvec G(\bvec x) = \bvec x + [g(\bvec x)-x_m]\bvec e_m$.

\item 10.6 在 $R^n$ 上, 只把标准基的某一对成员交换, 而其他成员不变的线性算子 $B$ 叫做\textbf{对换}. $B$ 也可以看成是交换两个坐标, 而基不变.

\item 10.10 设 $E$ 是 $R^n$ 中的开集. $E$ 中的 \textbf{$k$-曲面($k$-surface)}是从紧集 $D\subset R^k$ 到 $E$ 内的 $\mathscr C'$ 映射 $\Phi$. $D$ 叫做 $\Phi$ 的\textbf{参数域(parameter domain)}. $D$ 中的点记为 $\bvec u = (u_1,\dots, u_k)$.

\item 10.11 设 $E$ 是 $R^n$ 中的开集. $E$ 中的 $k\geqslant 1$ 次\textbf{微分形式(differential form)}(简称为 $E$ 中的 $k$-\textbf{形式})是一个函数, 用符号表示为 $\omega = \sum a_{i_1\dots i_k}(\bvec x)\dd{x_{i_1}} \wedge \dots \wedge \dd{x_{i_k}}$(指标 $i_1,\dots,i_k$ 各自从 $1$ 到 $n$ 独立变化) 的, 它给 $E$ 中的每个 $k$-曲面 $\Phi$ 规定一个数 $\int_\Phi \omega = \int_D \sum a_{i_1,\dots,i_k}(\Phi(\bvec u)) \pdv*{(x_{i_1},\dots,x_{i_k})}{(u_1,\dots,u_k)}\dd{\bvec u}$, 其中 $D$ 是 $\Phi$ 的参数域. 假设 $a_{i_1,\dots,i_k}$ 为 $E$ 上的连续实函数. 如果 $\Phi$ 的分量为 $\phi_1, \dots, \phi_n$, 那么上面的雅可比行列式就由 $(u_1,\dots,u_k)\to (\phi_{i_1}(\bvec u),\dots,\phi_{i_k}(\bvec u))$. 如果函数 $a_{i_1}\dots a_{i_k}$ 都属于 $\mathscr C'$ 或者 $\mathscr C''$, 那么就说 $k$-形式 $\omega$ 属于 $\mathscr C'$ 类 或者 $\mathscr C''$ 类.

\item 10.12 例子 (a) $\omega=y\dd{x}+x\dd{y}$ 那么 $\int_\gamma \omega = \int_0^1 [\gamma_1(t)\gamma'_2(t)+\gamma_2(t)\gamma'_1(t)]\dd{t} = \gamma_1(1)\gamma_2(1)-\gamma_1(0)\gamma_2(0)$ 对闭合曲线为零. $1$-形式的积分时常叫做\textbf{线积分}. (b) $\int_\gamma x\dd{y}$ 是逆时针曲线所围面积, $\int_\gamma y\dd{x}$ 是顺时针曲线所围面积. (c) $R^3$ 上的 $3$-曲面 $\Phi(r,\theta,\varphi)=(r\sin\theta\cos\varphi,r\sin\theta\sin\varphi, r\cos\theta)$. $J_\Phi(r,\theta,\varphi) = r^2\sin\theta$ 因此 $\int_\Phi \dd{x}\wedge\dd{y}\wedge\dd{z} = \int_D J_\Phi = 4\pi/3$.

\item 10.13 $w,w_1,w_2$ 是 $E$ 中的 $k$-形式. 当且仅当对于 $E$ 中的每个 $k$-曲面 $\Phi$, $\omega_1(\Phi)=\omega_2(\Phi)$ 时, 就写成 $\omega_1 =\omega_2$. 特别低, $\omega=0$ 表示对 $E$ 中每个 $k$-曲面 $\Phi$, $\omega(\Phi)=0$. 令 $c\in R$, 那么 $c\omega$ 由 $\int_\Phi c\omega=c\int_\Phi \omega$ 确定, 而 $\omega = \omega_1+\omega_2$ 表示…… $-\omega$ 表示.

\item 考虑 $\omega=a(x)\dd{x_{i_1}}\wedge\dots\wedge \dd{x_{i_k}}$, 令 $\bar\omega$ 是对调两个脚标所得的 $k$-形式. 由(雅可比)行列式的性质, $\bar\omega=-\omega$. 特别地, 如果微分形式的某一项出现了两个相同的下表, 就有 $\omega=0$.

\item 10.14 设 $i_1,\dots,i_k$ 为正整数, $1\leqslant i_1<i_2<\dots<i_k\leqslant n$, 又设 $I$ 是 $k$ 元有序组 ($k$-序组) $\{i_1,\dots,i_k\}$, 那么我们称 $I$ 为递增 $k$-指标, 采用简短的记法 $\dd{x_I}=\dd{x_{i_1}}\wedge\dots\wedge\dd{x_{i_k}}$. 这些形式的 $\dd{x_I}$ 叫做 $R^n$ 中的\textbf{基本 $k$-形式}. 恰好存在 $n!/k!(n-k)!$ 个.

\item 如果微分形式中每个 $k$-元组变为递增 $k$-指标, 那么就得到 $\omega$ 的标准表示 $\omega=\sum_I b_I(\bvec x)\dd{x_I}$, 其中求和遍历一切递增 $k$-指标 $I$.

\item 10.15 唯一性定理

\item 10.16 \textbf{基本 $k$-形式的乘积}: 设 $I=\{i_1,\dots,i_p\}$, $J=\{j_1,\dots,j_q\}$. …… 积是 $R^n$ 中的 $(p+q)$-形式, $\dd{x_I}\wedge \dd{x_J} = \dd{x_{i_1}} \wedge \dots \wedge \dd{x_{i_p}} \wedge \dd{x_{j_1}}\wedge \dots \wedge \dd{x_{j_q}}$. 如果 $I,J$ 有公共元素, 那么结果为零. 如果没有公共元素, 把元素递增排列, 得到的 $(p+q)$-指标写作 $[I,J]$. 那么 $\dd{x_{[I,J]}}$ 是基本 $(p+q)$-形式. ……

\item 10.17 \textbf{乘法}: $\omega,\lambda$ 分别是…… $p$-形式和 $q$-形式, 标准表示为 $\omega=\sum_I b_I(\bvec x)\dd{x_I}$, $\lambda=\sum_J c_J(\bvec x)\dd{x_J}$. 其中 ……. 他们的乘积定义为 $\omega\wedge\lambda = \sum_{I,J}b_I(\bvec x)c_J(\bvec x)\dd{x_I}\dd{x_J}$.

\item 10.18 $\omega$ 是 …… $k$-形式. 定义微分算子 $d$, 它给每个 $\omega$ 联系上一个 $(k+1)$-形式. $E$ 中的 $\mathscr C'$ 类的 $0$-形式, 恰好是实函数 $f\in \mathscr C'(E)$, 定义 $\dd{f}=\sum_{i=1}^n (D_i f)(\bvec x)\dd{x_i}$. 如果 $\omega=\sum b_I(\bvec x)\dd{x_I}$ 是 $k$-形式 $\omega$ 的标准表示, 而对每个递增 $k$-指标 $I$ 来说 $b_I\in \mathscr C'(E)$, 就定义 $\dd{\omega} = \sum_I (db_I)\wedge \dd{x_I}$.

\item 10.26 对向量空间 $X,Y$, 令 $f:X\to Y$. 如果 $\bvec f-\bvec f(\bvec 0)$ 是线性的, 就称 $\bvec f$ 为\textbf{仿射的(affine)}. 也就是说要求存在某个 $A\in L(X,Y)$ 使得 $\bvec f(\bvec x) = \bvec f(0)+A\bvec x$.

\item 定义\textbf{标准单形(standard simplex)} $Q^k$ 为由形如 $\bvec u = \sum_{i=1}^k \alpha_k\bvec e_i$ 使 $\alpha_i\geqslant 0$($i=1,\dots,k$)并且 $\sum \alpha_i\leqslant 1$ 的一切 $\bvec u\in R^k$ 组成的集.

\item \textbf{有向仿射 $k$-单形(oriented affine $k$-simplex)} $\sigma = [\bvec p_0,\bvec p_1,\dots,\bvec p_k]$ 是用 $Q^k$ 作参数域, 由仿射映射 $\sigma(\alpha_1\bvec e_1+\dots+\alpha_k\bvec e_k) = \bvec p_0 + \sum_{i=1}^k \alpha_i(\bvec p_i-\bvec p_0)$. 给出的 $R^n$ 中的 $k$-曲面. 注意……

\item 10.28 开集 $E\subset R^n$ 中的\textbf{仿射 $k$-链} $\Gamma$ 是 $E$ 中有限多个有向仿射 $k$-单形 $\sigma_1,\dots,\sigma_r$ 的集体. 这些 $k$-单形不必各不相同.

\item 设 $\omega$ 是 $E$ 中的 $k$-形式, 定义 $\int_\Gamma \omega = \sum_{i=1}^r \int_{\sigma_i}\omega$.

\item 10.30 …… 复合映射 $\Phi=T\circ\sigma$ 是 $V$ 中的以 $Q^k$ 为参数域的 $k$ 曲面. 称 $\Phi$ 为 $\mathscr C''$ 类的有向 $k$-单形. $V$ 中 $\mathscr C''$ 类的有向 $k$-单形 $\Phi_1,\dots,\Phi_r$ 的有限集 $\Psi$ 叫做 $V$ 中的 $\mathscr C''$ 类的 $k$-链.

\item 10.33 \textbf{Stokes 定理}: 设 $\Psi$ 是开集 $V\subset R^m$ 中的 $\mathscr C''$ 类 $k$-链, $\omega$ 是 $V$ 中的 $\mathscr C'$ 类 $(k-1)$-形式, 那么 $\int_\Psi\dd{\omega}=\int_{\partial \Psi}\omega$.
\end{itemize}


\subsection{Chap 11. Lebesgue 理论}

\begin{itemize}
\item 对集合 $A, B$, 定义 $A-B$ 为 $\{x|x\in A, x\notin B\}$.

\item 11.1 设 $\mathscr R$ 是由集构成的一个\textbf{类(family of sets)}\footnote{集合的集合叫类}, 并且由 $A\in \mathscr R$, $B\in \mathscr R$ 能推出 $A\bigcup B\in \mathscr R$, $A- B\in \mathscr R$. 就说 $\mathscr R$ 是\textbf{环(ring)}.

\item 由于 $A\bigcap B=A-(A-B)$, 必然有 $A\bigcap B\in \mathscr R$.

\item 如果一旦 $A_n\in \mathscr R$, $n=1,2,3,\dots$ 就有 $\bigcup_{n=1}^\infty A_n \in \mathscr R$, 那么 $\mathscr R$ 就叫 \textbf{$\sigma$-环}.

\item 11.2 如果 $\phi$ 能给每个集合 $A\in \mathscr R$ 指派广义实数系内的一个数 $\phi(A)$, 就说它是定义在 $\mathscr R$ 上的\textbf{集函数(set function)}. 如果能从 $A\cap B=0$ 得出 $\phi(A\bigcup B)=\phi(A)+\phi(B)$, 那么 $\phi$ 就是\textbf{可加的(additive)}. 如果能从 $A_i\bigcap A_j = 0$ ($i\ne j$) 得出 $\phi(\bigcup_{n=1}^\infty A_n) = \sum_{n=1}^\infty \phi(A_n)$, $\phi$ 就是\textbf{可数可加的(countably additive)}. 注意这与 $A_n$ 的排列次序无关, 所以重排定理表明等式右边只要收敛就绝对收敛. 如果不收敛, 它的部分和就趋于 $\pm\infty$.

\item 11.4 $R^p$ 中的\textbf{区间(interval)}指的是满足 $a_i\leqslant x_i\leqslant b_i$($i=1,\dots,p$) 的点 $x=(x_1,\dots,x_p)$ 的集合, 或者把任意或全部 $\leqslant$ 改为 $<$ 的集合. 允许任意 $a_i=b_i$. 规定空集属于区间. 若 $A$ 是有限个区间的并, 就说 $A$ 是\textbf{初等集(elementary set)}.

\item 令 $I$ 是区间, 定义 $m(I)=\Pi_{i=1}^p (b_i-a_i)$. 如果 $A=I_1\bigcup \dots \bigcup I_n$, 并且这些区间两两不相交, 就令 $m(A)=m(I_1)+\dots+m(I_n)$. 用 $\mathscr E$ 表示所有初等集的类.

\item (12) $\mathscr E$ 是环, 但不是 $\sigma$-环. (13) 如果 $A\in \mathscr E$, 那么 $A$ 必定是有限个不相交的区间的并. (15) $m$ 在 $\mathscr E$ 上可加. 当 $p=1,2,3$ 时, $m$ 分别是长度,面积和体积.

\item 11.5 设 $\phi$ 是 $\mathscr E$ 上的非负可加的集函数. 如果对于每个 $A\in \mathscr E$ 和 $\varepsilon>0$, 存在闭集 $F\in \mathscr E$ 和开集 $G\in\mathscr E$ 满足 $F\subset A\subset G$, 并且 $\phi(G)-\varepsilon \leqslant \phi(A) \leqslant \phi(F) + \varepsilon$, 就说 $\phi$ 是\textbf{正规的(regular)}.

\item 11.6 集函数 $m$ 是正规的. ……

\item 11.7 设 $\mu$ 在 $\mathscr E$ 上可加, 正规, 非负且有限. $E$ 是 $R^p$ 中的任何集. 考虑由初等开集 $A_n$ 组成的 $E$ 的覆盖 $E\subset \bigcup_{n=1}^\infty A_n$. 定义 $\mu^*(E) = \inf \sum_{n=1}^\infty \mu(A_n)$, 此处的下确界是对 $E$ 的一切初等开集组成的可数覆盖来取的. $\mu^*(E)$ 叫做 $E$ 对应于 $\mu$ 的\textbf{外测度(outer measure)}. “测度” 的拼音是 (ce4 du4).

\item 对于所有的 $E$, $\mu^*(E)\geqslant 0$, 且当 $E_1\subset E_2$ 时, $\mu^*(E_1) \leqslant \mu^*(E_2)$.

\item 11.8 (a) 对每个 $A\in \mathscr E$, $\mu^*(A)=\mu(A)$. 这说明 $\mu^*$ 把 $\mu$ 从初等集拓展到 $R^n$ 上的一般集合 (b) \textbf{次加性 (subadditivity)}:如果 $E\subset R^n$, $E=\bigcup_1^\infty E_n$, 那么 $\mu^*(E)\leqslant \sum_{n=1}^\infty \mu^*(E_n)$.

\item 11.9 对任意 $A,B \subset R^p$, 定义 $S(A,B)=(A-B)\bigcup (B-A)$, $d(A,B)=\mu^*(S(A,B))$. 如果 $\lim_{n\to\infty} d(A,A_n)=0$, 就记 $A_n\to A$. 若有一列初等集 $\{A_n\}$ 满足 $A_n\to A$, 就说 $A$ 是\textbf{有限 $\mu$ 可测的(finitely $\mu$-measurable)}, 记为 $A\in \mathfrak M_F(\mu)$. 若 $A$ 是可数多个有限 $\mu$ 可测集的并, 就说 $A$ 是 \textbf{$\mu$ 可测的($\mu$-measurable)}, 记为 $A\in \mathfrak M(\mu)$. $S(A,B)$ 是所谓的 “\textbf{对称差(symmetric difference)}”. $d(A,B)$ 是一个距离函数.

\item 11.10 $\mathfrak M(\mu)$ 是 $\sigma$-环, $\mu^*$ 在 $\mathfrak M(\mu)$ 上可数可加.

\item …… 就把原来定义在 $\mathscr E$ 上的 $\mu$ 推广成 $\sigma$-环 $\mathfrak M(\mu)$ 上的可数可加集函数了. 这个推广了的集函数叫做一个\textbf{测度(measure)}. $\mu=m$ 的特殊情形叫做 $R^p$ 上的 \textbf{Lebesgue 测度}.

\item 11.12 如果集合 $X$ 存在子集(称为可测集)组成的 $\sigma$-环 $\mathfrak M$, 以及其上的一个非负可数可加集函数 $\mu$ (称为\textbf{测度}), 就说 $X$ 是\textbf{测度空间(measure space)}. 如果还有 $X\in \mathfrak M$, 那么 $X$ 称为\textbf{可测空间(measurable space)}.

\item 概率论中, 事件可以看作是集合, 而事件发生的概率是可加(或可数可加)集函数.

\item 11.13 设 $f$ 是可测空间 $X$ 上的函数, 在扩大实数系内取值. 如果 $\{x|f(x)>a\}$ 对于每个实数 $a$ 可测, 就说函数 $f$ 是\textbf{可测的(measurable)}.

\item 11.17 设 $\{f_n\}$ 是一列可测函数. 当 $x\in X$, 令 $g(x)=\sup f_n(x)$ ($n=1,2,3,\dots$), $h(x)=\limsup_{n\to\infty} f_n(x)$. 那么 $g,h$ 可测.

\item 推论 (a) 假若 $f,g$ 可测, 那么 $\max(f,g)$ 和 $\min(f,g)$ 可测. 如果 $f^+=\max(f,0)$, $f^-=-\min(f,0)$, 那么 $f^+,f^-$ 都可测.

\item …… (b) 可测函数序列的极限函数是可测函数.

\item 11.18 设 $f,g$ 是定义在 $X$ 上的可测实函数, 而 $F$ 是 $R^2$ 上的实连续函数, 令 $h(x)=F(f(x),g(x))$($x\in X$), 那么 $h$ 可测. 特别地, $f+g$ 和 $fg$ 可测.

\item 11.19 设 $s$ 是 $X$ 上的实值函数. 如果 $s$ 的值域是有限集, 就说 $s$ 为\textbf{简单函数(simple function)}. 设 $E\subset X$, 令 $K_E(x)=\leftgroup{&1 \quad  (x\in E)\\ & 0 \quad (x\notin E)}$. $K_E$ 称为 $E$ 的\textbf{特征函数(characteristic function)}.

\item 11.20 设 $f$ 为 $X$ 上的实函数. 那么存在一列简单函数 $\{s_n\}$, 对于每个 $x\in X$ 当 $n\to\infty$ 时 $s_n(x)\to f(x)$. 若 $f$ 可测, 可选 $\{s_n\}$ 为可测函数序列. 若 $f\geqslant 0$, 可以选 $\{s_n\}$ 为单调增序列

\item 设 $s(x) = \sum_{n=1}^n c_i K_{E_i}(x)$ ($x\in X, c_i>0$) 可测, 而且 $E\in \mathfrak M$. 我们定义 $I_E(s) = \sum_{n=1}^n c_i \mu(E\bigcap E_i)$. 如果 $f$ 可测且为非负的, 我们定义 $\int_E f\dd{\mu} = \sup I_E(s)$. 这里 $\sup$ 是对所有满足 $0\leqslant s\leqslant f$ 的简单函数 $s$ 而取的. 这就是 $f$ 关于测度 $\mu$ 在集 $E$ 上的 \textbf{Lebesgue 积分}.

\item 11.22 设 $f$ 为可测函数, 考虑两积分 $\int_E f^\pm \dd{\mu}$. 如果其中至少有一个有限, 我们定义 $\int_E f\dd{\mu} = \int_E f^+\dd{\mu} - \int_E f^-\dd{\mu}$. 若两个积分都有限, 那么左边也有限, 就说 $f$ 在 $E$ 上对 $\mu$ 是 \textbf{Lebesgue 可积}的, 记为 $f\in \mathscr L(\mu)$, 若 $\mu=m$, 就说在 $E$ 上 $f\in \mathscr L$.
\end{itemize}
