% Möbius 函数(数论)
% 莫比乌斯函数|初等数论|莫比乌斯反演公式|Möbius反演公式

\pentry{数论函数\upref{NumFun},二项式定理\upref{BiNor}}

\addTODO{还可继续添加性质和应用等内容}

德国数学家August Ferdinand Möbius于1832年提出Möbius函数的概念,这是数论中一个重要的积性函数。Möbius函数在初等数论和解析数论中随处可见,多以Möbius反演的形式出现。


任何正整数都可以唯一地分解为其质因数的幂的乘积。比如说,$24=2^3\times 3$,$300=2^3\times 3\times 5^2$。一般地,我们把整数的质因数分解记为$\prod_{k=1}^r p_k^{f_k}$,其中各$p_k$是互不相等的素数,各$f_k$都是正整数。Möbius的概念正是建立在正整数质因数分解上的:


\begin{definition}{Möbius函数}\label{def_MbusF_1}
对于任意正整数$n=\prod_{k=1}^r p_k^{f_k}$,其Möbius函数$\mu:\mathbb{Z}^+\to\{-1, 0, 1\}$定义为:
\begin{equation}
\mu(n)=\mu(\prod_{k=1}^r p_k^{f_k})=
\leftgroup{
    &1 &\quad &(\text{如果}n=1)\\
    &(-1)^r  &\quad &(\text{如果}f_k=1\text{恒成立})\\
    &0 &\quad &(\text{如果有一个}f_k>1)~.
}
\end{equation}

\end{definition}

简单来说,一个正整数的质因子中如果有幂次超过$2$的,则它的Möbius函数为$0$;其余情况,则由不同质因子数量的奇偶性决定,奇则为$-1$,偶则为$+1$。


Möbius函数有以下性质:

\begin{theorem}{Möbius的积性}\label{the_MbusF_1}
给定\textbf{互素}的正整数$a, b$,则$\mu(ab)=\mu(a)\mu(b)$。
\end{theorem}

只需要检查$ab, a, b$各自的质因数即可得证\autoref{the_MbusF_1} 。显然,要求互素是因为,不互素时$a, b$会有公共质因数,此时$\mu(ab)=0$。





\begin{theorem}{求和性质}\label{the_MbusF_2}

\begin{equation}\label{eq_MbusF_1}
\sum_{d\mid n}\mu(d)=
\leftgroup{
    &1 \quad (n=1)\\
    &0 \quad (n>1)~.
}
\end{equation}

\end{theorem}

\textbf{证明}:

设$S$是$n$的全体质因子构成的集合($n=1$时$S=\varnothing$)。每个$d$都是$n$的若干质因子求积的结果,而我们只需考虑其中各质因子最多只出现一次的情况。因此,我们所考虑的$d$,和$S$的子集一一对应,即$d$就是这个子集中各素数相乘的结果。\autoref{eq_MbusF_1} 极为遍历所有$S$的子集的求和。

当$n=1$时,易验证\autoref{eq_MbusF_1} 第一行成立。下设$n$至少有一个质因子。

记$S=\{a_i\}_{i=1}^r$。则由$k$个$a_i$相乘而得的$d$,一共有$C^k_r$个,它们均满足$\mu(d)=(-1)^k$。于是
\begin{equation}
\sum_{d\mid n}\mu(d)=\sum_{k=0}^r(-1)^kC^k_r~.
\end{equation}
又由\textbf{二项式定理}\upref{BiNor}可知
\begin{equation}
\sum_{k=0}^r(-1)^kC^k_r=(1-1)^r=0~.
\end{equation}
\textbf{证毕}。

将\autoref{eq_MbusF_1} 作为指数,即得求积的性质:


\begin{corollary}{求积性质}\label{cor_MbusF_1}
任取正实数$a$,则
\begin{equation}
\prod_{d\mid n}a^{\mu(d)}=
\leftgroup{
    &a \quad (n=1)\\
    &1 \quad (n>1)~.
}
\end{equation}
\end{corollary}






\begin{theorem}{Möbius反演公式}\label{the_MbusF_3}
取映射$f:\mathbb{Z}^+\to\mathbb{Z}^+$,令$S(n)=\sum_{d\mid n}f(d)$,$P(n)=\prod_{d\mid n}f(d)$。则有
\begin{equation}
f(n) = \sum_{d\mid n}\mu(d)S\qty(\frac{n}{d})~,
\end{equation}
和
\begin{equation}
f(n) = \prod_{d\mid n}\qty(P(\frac{n}{d}))^{\mu(d)}~.
\end{equation}
\end{theorem}

\textbf{证明}:

由\autoref{the_MbusF_2} ,$\sum_{dk\mid n}\mu(d)\neq 0$当且仅当$k=n$。于是
\begin{equation}
\begin{aligned}
\sum_{d\mid n}\mu(d)S\qty(\frac{n}{d})&=\sum_{d\mid n}\sum_{dk\mid n}\mu(d)f(k)\\
&=\sum_{k\mid n}\sum_{dk\mid n}\mu(d)f(k)\\
&=f(n)~.
\end{aligned}
\end{equation}


由\autoref{cor_MbusF_1} ,$\prod_{dk\mid n}\qty(f(k))^{\mu(d)}\neq 1$当且仅当$k=n$。于是
\begin{equation}
\begin{aligned}
\prod_{d\mid n}\qty(P(\frac{n}{d}))^{\mu(d)}&=\prod_{d\mid n}\qty(\prod_{dk\mid n}f(k))^{\mu(d)}\\
&=\prod_{d\mid n}\prod_{dk\mid n}\qty(f(k))^{\mu(d)}\\
&=\prod_{k\mid n}\prod_{dk\mid n}\qty(f(k))^{\mu(d)}\\
&=f(n)~.
\end{aligned}
\end{equation}

\textbf{证毕}。

















