% 磁多极矩
% 磁多极矩|磁矢势|多级展开|磁矩

\pentry{电多极展开\upref{EMulPo}}
\begin{lemma}{分子电流观点}
对于物质磁性的解释,把每个宏观体积元内的分子看成完全一样的电流环,及具有同样的面积 $a$ 和取向(由面元矢量 $\bvec a$ 表示),环内具有同样的电流 $I$,而磁性由分子电流引发.故而后面\autoref{edy33_eq4} 可以把电流密度的体积分转变为电流的环路积分.
\end{lemma}
\subsection{磁矢势的多级展开}
类似于电多级展开的讨论,我们考虑一个局域在原点附近的电流密度分布 $\bvec J(\bvec x')$ 在远离电流区域的 $\bvec x$ 处所产生的磁矢势(也就是说 $|\bvec x|\gg|\bvec x'|$).电流分布在小区域 $V$ 内.
% 链接未完成
\begin{equation}
\bvec A(\bvec x)=\dfrac {\mu_0}{4\pi} \int_V \dfrac{ \bvec J(\bvec x^{\prime})\dd V^{\prime}}{|\bvec x - \bvec x'|}\label{edy33_eq1}
\end{equation}
为了计算远处任意一点 $\bvec x$ 处的磁矢势,我们可以做泰勒展开.
\begin{align}
\bvec A(\bvec x)&=\dfrac {\mu_0}{4\pi}\int_V \bvec J(\bvec x')\Big[\dfrac{1}{|\bvec x|}-\bvec x'\cdot\nabla \dfrac{1}{|\bvec x|}+\dfrac{1}{2!}\sum_{i,j}x_i^{\prime}x_j^{\prime}\pdv{}{x_i}{x_j}{\dfrac 1 {|\bvec x|}}+\cdots\Big]\dd \bvec V^{\prime}
\label{edy33_eq2_0}
\\
&=\dfrac{\mu_0}{4\pi|\bvec x|}\int_V \bvec J(\bvec x')\dd V'
-\sum_{j}\dfrac{\mu_0 \bvec x_j}{4\pi |\bvec x|^3}\int_V \bvec J(\bvec x')\bvec x'_j + \cdots
\label{edy33_eq2}
\end{align}
\autoref{edy33_eq2} 中,第一项
\begin{equation}
\bvec A^{(0)}(\bvec x)=\dfrac {\mu_0}{4\pi |\bvec x|}\int_V \bvec J(\bvec x')\dd{ V^{\prime}}\label{edy33_eq3}
\end{equation}
由于电流的连续性,% 引用未完成
把电流分为许多闭合的流管,对于每一个流管来说,有
\begin{equation}
\int_V \bvec J(\bvec x^{\prime}) \dd{V^{\prime}}=\oint_L I\dd{l'}=I\oint_L \dd{l'}=0\label{edy33_eq4}
\end{equation}
得\begin{equation}
\bvec A^{(0)}=0\label{edy33_eq5}
\end{equation}
对展开式第二项
\begin{equation}
\bvec A^{(1)}=-\dfrac{\mu_0 }{4\pi}\int_V \bvec J(\bvec x^{\prime})\Big (\bvec x^{\prime}\cdot\nabla\dfrac{1}{|\bvec x|}\Big)\dd{V^{\prime}}\label{edy33_eq6}
\end{equation}
由\autoref{edy33_eq4} 得
\begin{equation}
\bvec A^{(1)}=-\dfrac{\mu_0 I }{4\pi}\int_L (\bvec x^{\prime}\cdot\nabla\dfrac{1}{|\bvec x|}) \dd{l'}=\dfrac{\mu_0 I }{4\pi}\int_L (\bvec x^{\prime}\cdot\dfrac{\bvec x}{|\bvec x|^3})\dd l^{\prime}
=\dfrac{\mu_0 I }{4\pi|\bvec x|^3}\int_L (\bvec x^{\prime}\cdot \bvec x)\dd l^{\prime}
\label{edy33_eq7}
\end{equation}
其中有 $\dd{\bvec x}^{\prime}=\dd l^{\prime}$, 因为 $ \bvec x^{\prime}$ 为闭合流管上的坐标.
全微分绕闭合回路积分为 $0$
\begin{equation}
0=\oint_L \dd{(\bvec x^{\prime}\cdot \bvec x)\bvec x^{\prime}}=\oint_L (\bvec x^{\prime}\cdot\bvec x)\dd l^{\prime}+\oint_l (\dd l^{\prime}\cdot \bvec x)\bvec x^{\prime}\label{edy33_eq8}
\end{equation}
所以可以推出
\begin{equation}
\oint_L (\bvec x^{\prime}\cdot \bvec R)\dd{l^{\prime}}=\dfrac1 2\oint_L (\bvec x^{\prime}\times \dd l^{\prime})\times \bvec R \label{edy33_eq9}
\end{equation}

\begin{equation}
\bvec A^{(1)}=\dfrac{\mu_0}{4\pi R^3}\cdot\dfrac{I}{2}\oint_L (\bvec x^{\prime}\times \dd{l^{\prime}})\times \bvec R=\dfrac{\mu_0}{4\pi}\dfrac{\bvec m \times \bvec R}{R^3}\label{edy33_eq10}
\end{equation}
其中 $\bvec m =\dfrac{1}{2}\oint_L (\bvec x^{\prime}\times \dd{l^{\prime}})$ 被称为磁矩,而对于一个小线圈(分子电流),他所围的面元 $\Delta\bvec S$ 可以表示为
\begin{equation}
\Delta\bvec S=\dfrac 1 2\oint_L \bvec x^{\prime}\times \dd{\bvec l^{\prime}}
\end{equation}
故而,有等式
\begin{equation}
\bvec m=I\Delta \bvec S
\end{equation}
