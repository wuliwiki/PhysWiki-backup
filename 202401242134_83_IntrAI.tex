% 人工智能导论
% keys 人工智能|机器学习|深度学习
% license Usr
% type Tutor

本文旨在为人工智能部分后续的词条建立基本的概念和总览的图景。

\textbf{人工智能}(Artificial Intelligence,简称AI)是一门研究如何使计算机具有智能行为的科学与技术。它涵盖了一系列的技术、方法和应用,旨在\textbf{使计算机系统能够模拟、理解和执行人类智能的各种任务。}人工智能领域的发展已经走过了几个阶段,从最初的符号主义到现代的机器学习和深度学习。

\subsection{符号主义和搜索算法}

人工智能的最初阶段可以追溯到20世纪50年代和60年代,这一时期被称为符号主义时代。研究人员试图通过使用符号和规则来模拟人类智能。这些系统基于专家系统,其中包含了领域专家提供的知识,以解决特定类型的问题。然而,符号主义在处理复杂的、模糊的问题时面临困难,导致了人工智能研究的新方向的产生。

比如,\textbf{基于规则推理}(Rule Base Reasoning,RBR)的方法是一种将专家所掌握的知识和经验转化为规则,通过启发式推理进行推理解的技术。这种方法在解决问题时根据明确的前提条件产生明确的结果,使其推理过程相对清晰。举例来说,对动物的分类规则可以通过IF-THEN语句表示,从而实现对动物种类的判定,如老虎或企鹅。

基于规则的专家系统是早期专家系统的代表,其推理过程相对明确,规则正确时可以得到较为准确的结论。这使得基于规则的专家系统成为一种简单实用、广泛应用的专家系统。尤其对于特定领域的问题,基于规则的方法表现出色,成为解决实际问题的有效手段。

然而,基于规则的专家系统也存在一些缺点。首先,规则的构造高度依赖于专家的经验积累,如果专家的经验不准确,则系统的结果也可能不准确。其次,这类专家系统缺乏自学习能力,更新迭代需要专家经验的不断积累。虽然这两个缺点存在,但从系统开发的角度来看,专家系统是一个持续迭代优化的过程。在实际应用中,不希望专家系统因为自学习而导致结果不可预测的情况,因此依赖专家的经验积累是一个相对可控的方向。

同时期的还有搜索算法,\textbf{搜索算法更侧重于问题空间的系统搜索,而基于规则推理更专注于通过已有规则的逻辑推理解决问题。}搜索算法在当时游戏中的人工智能有着广泛的应用,比如:路径规划、游戏策略、排序等问题。实际上传统的人工智能最初解决的问题大多都是游戏问题。在接下来一些词条介绍传统人工智能算法的细节时,常会涉及人工智能玩棋牌之类益智游戏的算法。

以下是一些常见的搜索算法:
\begin{itemize}
\item 深度优先搜索(Depth-First Search,DFS): 从起始状态开始,沿着一个路径一直探索到底,直到找到目标状态或无法继续。DFS使用堆栈来存储路径信息。
\item 广度优先搜索(Breadth-First Search,BFS): 从起始状态开始,逐层地扩展搜索,先探索离起始状态最近的节点。BFS使用队列来存储待探索的节点。
\item 启发式搜索: 使用启发函数(Heuristic Function)来评估每个状态的“好坏”,并优先探索那些看起来更有可能达到目标的状态。著名的启发式搜索算法包括A*算法。
\item A*算法: 结合了深度优先搜索和广度优先搜索的优势,使用启发函数来估计从当前状态到目标状态的代价,并选择代价最小的路径。A*算法保证找到最短路径。
\end{itemize}

\subsection{机器学习}

随着计算能力的提升和数据的增加,人工智能进入了一个新的阶段。研究者们逐渐转向机器学习,这是一种让计算机通过学习经验来改进性能的方法。机器学习的基本思想是通过训练模型来识别模式,并从数据中提取知识。监督学习、无监督学习和强化学习等不同类型的机器学习算法开始崭露头角。

机器学习中的几个关键概念包括:
\begin{itemize}
\item \textbf{数据表示}:数据需要以计算机可以处理的形式表示,常见的有表格、向量等。良好的数据表示对于构建一个有效的机器学习系统至关重要。其中,线性代数中向量、矩阵、特征值分解等概念,为表达数据与模型,以及设计算法奠定了框架。例如,多数机器学习算法会将特征转换为向量表示,并基于向量空间中的几何关系进行建模。

\item \textbf{模型}:机器学习中的模型是对实际问题的抽象表示。常见的模型有线性回归、决策树、神经网络等。模型定义了输入和输出之间的关系。其中,概率和统计学让我们可以描述不确定性,是处理复杂现实世界建立模型的必要工具。

\item \textbf{学习-优化算法}:优化算法提供了调整模型参数以优化性能的方法。常见的优化算法包括:梯度下降、Adam、RMSprop等,这些算法被广泛应用于神经网络的训练过程。其中广泛用到了微积分与优化理论使我们能够推导学习算法的目标函数,设计求解最优参数的方法。

\item \textbf{损失函数}:损失函数量化了模型对给定数据的性能。通过最小化损失函数,学习算法可以自动优化模型的参数。常见的损失函数包括均方误差、分类交叉熵等。
\end{itemize}

由此我们可以对机器学习的整个过程有了更加清晰的认识,\textbf{(必读)机器学习基本过程}:

机器学习首先我们需要\textbf{收集训练数据},数据分为监督学习和无监督学习,监督学习的数据是带标签的。无监督学习数据是没有标签的。数据的好坏直接决定了训练后模型的好坏,也就是说基本你掌握OpenAI的所有模型算法,但是没有它们优质的数据,你的模型效果也不会很好。

在收集好了数据之后,接下来是\textbf{模型的选择}。机器学习的模型的构建需要对实际问题有着深刻的洞见,以及对模型背后数学理论有着清晰的认知。同时,还要权衡实际推理使用中的资源和响应时间。

当有了数据和模型之后,就可以将数据放到初始化的模型中,在第一轮推理中,你会得到模型在给定数据上的预测结果。这可能是一个类别概率分布、回归值或其他任务相关的输出。例如,在分类任务中,可能得到每个类别的概率分布。

如果你有标签信息,可以使用模型的预测结果与实际标签通过损失函数计算损失。无监督学习也能够从数据本身或者模型之间计算损失。

\textbf{损失函数}的选择也是需要根据实际的问题选择的,不过一般来说,并不需要从零开始构建一个新的损失函数。在有了损失后,我们就可以根据第一轮的损失,通过选择合适的\textbf{优化函数}去更新模型的参数,然后以此进行N轮到学习,直到模型收敛,也就是损失不在下降或者模型的损失可以被接受。

在构建模型、选择损失函数和优化函数时,深刻了解实际问题,并对模型背后的数学理论有清晰的认识是至关重要的。这意味着对问题的思考不仅要深入,而且需要扎实的数学基础。这两者共同构成了区分普通机器学习工程师和专家的重要因素。只有通过深刻理解实际问题的需求,并具备对模型背后数学原理的清晰认识,我们才能更好地构建出有效的模型,选择合适的损失函数和优化函数,从而取得更优越的机器学习结果。这种深度的理解和扎实的数学基础是在复杂问题中做出明智决策的关键。

在学习机器学习时,这与学习物理类似。\textbf{尽管掌握数学原理是必要的基础,但更重要的是花时间去理解和思考如何将现实问题抽象成数学模型。}回顾牛顿和爱因斯坦同时期,可能存在其他数学能力更强的物理学家,但是正是由于他们独特的敏锐洞见力和灵感,使得他们在物理领域取得了突出的成就。因此,机器学习学习的关键在于不仅仅掌握数学基础,还要培养对实际问题的抽象和理解能力。

\subsection{深度学习}

深度学习是机器学习的一个子领域,通过神经网络模拟人脑的结构来处理复杂的任务。深度学习在图像识别、语音识别、自然语言处理等领域取得了巨大成功。深度神经网络的层次结构允许模型逐渐抽象和理解数据的特征,从而提高了学习和泛化能力。深度学习的成功推动了人工智能的飞速发展,成为当前人工智能研究的主要推动力。


深度学习的过程与机器学习相似,唯一的区别在于模型的构建,深度学习中使用神经网络。目前来看,神经网络模型尚未有完善的数学理论,能够更高效深刻地利用这些模型。然而,神经网络模型展现出了惊人的效果。在训练和开发神经网络模型方面,更类似于进行大型实验工程。这意味着我们在深度学习中更侧重于通过实验和调整来改进模型,而不仅仅是基于理论的推导和分析。这种实验性的方法使得深度学习更像是一项不断探索和优化的工程,与传统机器学习在模型开发方面的方式有所不同。尽管数学理论尚未完善,但深度学习的实际效果推动了其在各个领域的广泛应用。

深度学习目前来说类似于1900年的物理学界,各种新奇的现象出现在实验中,一个新的理论呼之欲出。如果你对自己的天赋有信心,认为自己有着天才般敏锐的洞见力,那不妨更加深入的学习神经网络的原理。