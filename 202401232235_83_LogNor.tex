% 对数正态分布
% keys 统计学|正态分布|概率论
% license Usr
% type Wiki

\textbf{对数正态分布} (
Log-normal distribution) 是一种概率分布,它的特点是其对数服从正态分布。对数正态分布常用于描述那些取值范围为正数的随机变量,如金融领域的股票价格、收益率等。它具有右偏(正偏)的特性,即分布的尾部向右延伸。对数正态分布在统计学、金融学和生物学等领域有广泛应用,特别是当研究对象的增长是指数型的时候,对数正态分布常常能够提供较好的拟合。

对数正态分布的概率密度函数(PDF)可以表示为:

\begin{equation}
f(x) = \frac{1}{x} \frac{1}{\sqrt{2 \pi \sigma^2}} \exp \left(-\frac{(\log (x)-\mu)^2}{2 \sigma^2}\right)~.
\end{equation}

其中,\( x > 0 \) 是随机变量,\( \mu \) 是对数正态分布的均值,\( \sigma \) 是标准差,\( \ln \) 表示自然对数。

\begin{example}{计算推导对数正态分布的最大值时 \( x \) 的值}
我们需要求导数并令导数等于零来找到极值点。为了方便导数的计算,首先计算 $\ln(f(x))$,
\begin{equation}
    \ln(f(x)) = -\frac{(\mu-\log (x))^2}{2 \sigma^2}-\log (\sigma x)-\frac{1}{2} \log (2 \pi)~.
\end{equation}
对 $\ln(f(x))$ 求导得到如下结果,
\begin{equation}
\frac{\partial}{\partial x}\left(-\frac{(\mu-\log (x))^2}{2 \sigma^2}-\log (\sigma x)-\frac{1}{2} \log (2 \pi)\right) = -\frac{-\mu+\sigma^2+\log (x)}{\sigma^2 x}~.
\end{equation}
然后解 $\ln(f(x))'=0$ 得到 $x$ 的值,
\begin{equation}
    \exp \left({-\mu+\sigma^2+\log (x)}\right) = 0 \rightarrow x = \E^{\mu}\E^{-\sigma^2}~.
\end{equation}
\end{example}
\begin{example}{对数正态分布的矩(Moments)}
第 $r$ 阶原始矩表示 $X$ 的随机变量的 $r$ 次方的期望值。对于具有概率密度函数 $f(y)$ 的随机变量 $X$,第 $r$ 阶原始矩为,

\begin{equation}
E\left[X^r\right]=\int_{-\infty}^{\infty} \frac{1}{\sqrt{2 \pi \sigma^2}} \exp \left(r \mu+r y-y^2 / 2 \sigma^2\right)dy ~.
\end{equation}

首先,我们计算前述积分表达式的不定积分得到,
\begin{equation}
\int \frac{\exp \left(r \mu+r y-\frac{y^2}{2 \sigma^2}\right)}{\sqrt{2 \pi \sigma^2}} d y=\frac{\sigma e^{\left(r^2 \sigma^2\right) / 2+\mu r} erf\left(\frac{y-r \sigma^2}{\sqrt{2} \sigma}\right)}{2 \sqrt{\sigma^2}}~.
\end{equation}

这里特殊函数 $erf$ 是误差函数。然后,我们代入 $+\infty$ 和 $-\infty$,注意到 $erf(+\infty)=1$ 和 $erf(-\infty)=-1$,所以
\begin{equation}
E\left[X^r\right]=\frac{\sigma e^{\left(r^2 \sigma^2\right) / 2+\mu r}}{2 \sqrt{\sigma^2}}-\left(-\frac{\sigma e^{\left(r^2 \sigma^2\right) / 2+\mu r}}{2 \sqrt{\sigma^2}}\right) = e^{r\mu+\frac{1}{2}\sigma^2}~.
\end{equation}

当 $r=1$ 时,我们有 $E\left[X\right]=e^{\mu+\frac{1}{2}\sigma^2}$,$Var[X] = E\left[X^2\right] - \left(E\left[X\right]\right)^2$ 是
\begin{align}
Var[X] &= \E^{2\mu+2\sigma^2} - \E^{2\mu+\sigma^2} \\
&= \E^{2\mu+\sigma^2}(\E^{2\mu+2\sigma^2-2\mu-\sigma^2}- 1) = \E^{2\mu+\sigma^2}(\E^{\sigma^2}- 1)~.
\end{align}
\end{example}