% 近独立子系
% 热力学系统|近独立子系|无相互作用

\pentry{热力学笔记(科普)\upref{HeatIn}}

\textbf{近独立子系}被定义为:大量粒子组成的热力学系统,且忽略粒子间的力学相互作用,不同粒子间可以看作是近独立的。

近独立子系是热力学的重要研究对象,由于不同粒子间可看作是近独立的,可以通过分析系统的能级来计算系统的配分函数\footnote{例如正则系综法\upref{CEsb}。},可以分析粒子的速率分布函数\footnote{例如麦克斯韦—玻尔兹曼分布\upref{MxwBzm}。},因此系统的各个热力学量就可以方便地用统计力学公式进行计算\footnote{例如玻尔兹曼分布(统计力学)\upref{MBsta}。},可以得到与实验符合得很好的结果。

近独立子系一般有三种分布:玻尔兹曼分布,费米狄拉克分布,玻色爱因斯坦分布,其中玻尔兹曼分布是经典极限情形下的分布。同时,粒子间可以有交换相互作用,考虑到量子系统中玻色子和费米子的性质,系统会呈现出同经典玻尔兹曼分布不同的结果\footnote{例如玻色爱因斯坦凝聚\upref{BEC}、金属中的自由电子气体\upref{mfcgas}。}。

\subsection{麦克斯韦-玻尔兹曼分布}
近独立子系中,子系的分布函数可以由 $\bar a_\lambda \propto g_\lambda e^{-\alpha-\beta\epsilon_\lambda}$ 给出\footnote{玻尔兹曼分布(统计力学)\upref{MBsta}。}。这里要强调的是,这一公式不仅可以适用于经典理想气体,还可以推广到具有不同能级的玻色系统或费米系统中,但这时我们需要以另一种视角去看待系统和子系的能级,且此处的讨论与玻尔兹曼分布(统计力学)\upref{MBsta}中第三节的讨论的方法不同,但我们接下来会看到它们得出的结果是一样的,请读者仔细体会其中的差异。

以玻色子系统为例,能级 $\epsilon_l$ 的简并度为 $g_l$,且同一个能级上可以有多个粒子占据,那么此处我们将它不仅仅视为一个能级,而是视为能量为 $0,\epsilon_l,2\epsilon_l,3\epsilon_l,\cdots, n\epsilon_l,\cdots$ 的无限多种能级,分别代表在 $\epsilon_l$ 能级上系统可能的状态。在这无穷多个能级上考虑玻尔兹曼分布,$\bar{a}_{n,l} \propto n\cdot \exp(-n\alpha-\beta\cdot n\epsilon_l),n=0,1,2,\cdots $。因此这里我们可以定义玻色系统的子系配分函数。要注意的是,正是因为是我们讨论的是近独立子系,子系与子系间相互作用近似忽略,所以才能够单独地讨论子系配分函数以及它的粒子占据数。
\begin{equation}
Z_1=\sum \exp(-\beta\cdot n\epsilon_l)
\end{equation}
对所有可能的 $n$ 求和,再乘以简并度 $g_l$,除以配分函数,就得到了能级 $\epsilon_l$ 所对应的期望粒子数:
\begin{equation}
\begin{aligned}
\bar a_l&=g_l \frac{\sum_n n\cdot \exp(-n\beta\epsilon_l)}{\sum_n \exp(-n\beta\epsilon_l)}=g_l (1-\exp(-\beta\epsilon_l)) \frac{-1}{\epsilon_l}\pdv{\beta} \frac{1}{1-\exp(-\beta\epsilon_l)}\\
&=g_l \frac{1}{\exp(\beta\epsilon_l)-1}
\end{aligned}
\end{equation}