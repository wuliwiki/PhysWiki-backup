% C++ 字符串(笔记)
% license Xiao
% type Note

\begin{issues}
\issueDraft
\end{issues}

\subsection{常识}
\begin{itemize}
\item 任意两个 string literal 出现在一起会自动拼接: \verb`"abc" "def"` 相当于 \verb|"abcdef"|。 中间可以换行。
\item \verb|u8"..."| 指定对字符串使用 UTF-8 编码(一些编译器不加 \verb|u8| 也默认使用), \verb|U"..."| 用于把字符串转换为 UTF-32 编码(\verb|basic_string<char32_t>| 类型)。 这和源码文件的字符编码无关(首先确保你的文本文件使用编译器指定的编码)。
\item 普通字符串中可以用 \verb|\0数字| (8 进制) 或 \verb|\x数字| (16 进制) 来指定某个不好表示的 code point。
\item 裸字符串: \verb|R"(一些字符)"|。 里面的字符不会被 escape。 若要更裸一点, 还可以用 \verb|R"自定义标志(一些字符)自定义标志"|。 中间的 \verb|一些字符| 可以换行。
\item 如果需要高性能, 函数还得是用 \verb|const char *| 输入字符串(会需要手打很多 \verb|.c_str()|)。 或者其次用 \verb|const string&| 加上一个 index 指明起始位置, 或者用 C++17 的 \verb|string_view|。
\end{itemize}

\subsection{字符串和数的转换}

\begin{itemize}
\item \verb|size_t string::find(char或const string&, size_t start)|; 若找不到, 返回 \verb|string::npos|(no position 的缩写), 这和 \verb|numeric_limits<size_t>::max()| 是一样的。 若嫌麻烦也可以 \verb|using string::npos|。
\end{itemize}

\subsubsection{string}
\begin{itemize}
\item 要 \verb`#include <string>`, \verb`use string`.
\item constructors: \verb`string s1;`, \verb`string s2{s1};`, \verb`string s3{"something"};`, \verb`string s4(10, 'c');`
\item \verb`string::empty()` 判断是否是空.
\item \verb`string::size()` 返回字符数.
\item \verb`str[n]` 可以直接读取或赋值某个字符.
\item \verb`str1 + str2` 可以连接两个字符串.
\item \verb`str1 + "something"` 可以连接 string 和 literal, 但不能是 \verb`wchar_t`.
\item \verb`str1 == str2`, \verb`str1 != str2` 可以比较字符串是否相同.
\end{itemize}

\subsubsection{数字转换}
\begin{itemize}
\item 以下字符串转整数的算法: 会忽略前面的空格, 然后可以有一个 \verb|+,-| 号, 然后跟一串数字, 直到遇见非数字。
\item \verb|stoi, stoil, stoill| (C++)(例如 \verb|int stoi(const string &str, size_t *pos = nullptr, int base = 10);|) \verb|pos| 用于输出处理了多少个字符。 但是缺点是如果处理子串, 需要生成一个新的字符串, 进行不必要的复制和内存分配。 如果出错会 \verb|throw|。
\item \verb|strtol, strtoll|(不存在 \verb|strtoi|) (C 语言)例如 \verb|long strtol(const char* str, char** str_end, int base)|, \verb|str_end| 会指向处理完的后一个字符。 如果出错, 会把 \verb|errno| 设为 \verb|ERANGE| (在调用前要手动先设为 \verb|0|,比较麻烦)。
\item \verb|atoi, atol, atoll| (C 语言)(ascii to integer 的缩写) 例如 \verb|int atoi( const char* str );|。 不建议用, 因为如果超出 \verb|int| 范围的行为没有定义, 如果转换不了会返回 0。
\end{itemize}


\subsubsection{UTF-8 编码}
\begin{itemize}
\item \verb|0| 开头的字节都是 ascii 字符。
\item \verb|10| 开头的字节都不是字符的起始字节。
\item \verb|110| 开头的字节是用 2 个字节表示一个字符。
\item \verb|1110| 开头的字节用 3 个字节表示一个字符。
\item \verb|11110| 开头的字节用 4 个字节表示一个字符。
\item 所以搜索字符或者字符串时, 只要关键词是合法的 utf-8 字符串(哪怕只有一个 ascii 字符), 搜到的必然也是合法的, 不可能把多字节表示的字符砍断, \verb|string::find()| 可以正常使用。
\item \verb|string::find_first_of()| 这样的函数就用不了了。 因为单个字符一般也要用字符串来表示。
\item 跳到上一个下一个字符也要用 iterator, 而不是直接用 index (SLISC\upref{SLISC} 提供一个这样的 iterator, 也包含 utfcpp 库)。
\item STL 不提供这样的功能, 常用的第三方库有 \href{https://github.com/nemtrif/utfcpp}{utfcpp (header only)} 以及 \href{https://tzlaine.github.io/text/doc/html/index.html}{Boost.TexT}。
\item 所以如果不追求性能, 还是直接用 \verb|basic_string<char32_t>| 和 \verb|U"字符串"| 最容易, 体验和 python 差不多。 但许多人还是喜欢 utf-8 (甚至微软的 utf-16)。
\end{itemize}


\subsubsection{iostream}
\begin{itemize}
\item \verb`wchart_t` 类型的 literal 例如 \verb`L"this is a string"`
\item \verb`<< endl` 用于换行且 \verb|cout.flush()|(把缓存立即输出)。
\item \verb|cin >>| 会忽略任意多个空格和回车, 以及 \verb|tab| 等空白字符。 读取字符串时,遇到空白字符结束。 如果 parse 错误或者遇到 EOF,则不会赋值,并设置相应的错误 bit。 此时如果 \verb|if(cin)|, 就会转换为 \verb|false|。
\item \verb`<< hex <<` 把后面的数字都变成 16 进制, \verb`<< dec <<` 把后面的数字都变成 10 进制.
\item \verb`cin.getline(char* s, streamsize n, char delim)` 用于读取一个 \verb`char[] name`, 不超过 \verb|n| 个字符(一般设为 \verb|s| 的长度),到 \verb|delim| 之前停止。 然后存在 \verb|s| 里面, 后面加上 \verb`\0`.
\item \verb|std::getline(std::istream& input, std::string& str, char delim);| 功能也类似,只是输出的类型改变。
\item \verb`cout.precision(N);` 可以控制输出的有效数字位数.
\end{itemize}

\subsubsection{iomanip}
\begin{itemize}
\item \verb|<< std::fixed| 可以把后面的浮点数都不适用科学计数法 \verb|<< std::setprecision(3)| 在非科学计数法中规定小数点后的位数。 所以二者一起使用就是定点定位数小数。 如果单独使用 \verb|<< std::setprecision(3)|, 就取 3 为有效数字而不是小数,输出的字符串可能是科学计数法。
\item \verb`setw(n)` 用于把两个数字的间隔控制在 n 个字节之内. (用于列对齐).
\item \verb`setprecision(n)` 用于显示 n 位小数.
\item \verb`setiosflags(ios::left)` 用于左对齐, 另有 \verb`ios::fixed` (非科学计数法) 或 \verb`ios::right` .
\end{itemize}

\subsubsection{cstring}
\begin{itemize}
\item 用于处理 \verb`\0` 结尾的字符串.
\item \verb`strlen()` 函数用于返回字符串长度(不包括 \verb`\0`), 返回类型是 \verb`size_t`. 对于 \verb`wchar_t` 字符串, 用 \verb`wcslen()`.
\item \verb|int strcmp(pstr1, pstr2)| 比较两个字符串,若返回 \verb|0| 则相等, 小于零则不相等且第一个字符串较小, 大于零则较大。 大小的比较使用 lexicographic 顺序,即逐个字节比较值的大小, 第一个不同的字节决定大小。 若 \verb|str1| 是 \verb|str2| 的开头的一小部分,那么 \verb|str1| 较小(因为 \verb|\0| 肯定更小)。
\item \verb`strcpy(pstr1, pstr2)` 把 \verb`pstr2` 指向的字符串拷贝的 \verb`pstr1` 的地址。 如果 \verb|pstr1| 分配的长度不够,会继续往后写产生内存错误。 更安全的版本是 \verb|errno_t strcpy_s(char *dest, rsize_t destsz, const char *src);|。 如果成功, 返回 0, 若失败, \verb|dest| 被设为空字符串。 另一个安全版本是 \verb|char *strncpy(char *destination, const char *source, size_t num);
|。 \verb|num| 是最多复制的字符数,超出该长度的部分会被忽略。
\item \verb|char *strcat(char *dest, const char *src);| 把 \verb|src| 添加到 \verb|dest| 末尾(若没有分配足够的空间则会内存溢出) 返回 \verb|dest|。 一个更安全的函数是 \verb|char *strncat(char *destination, const char *source, size_t num);|。 如果 \verb|source| 的长度大于 \verb|num|, 那么只会赋值 \verb|num| 个。
\end{itemize}

\subsection{【回收内容】MFC 的 CString 类}
\begin{itemize}
\item \verb`CString(TCHAR str)` 可以把 \verb`str` 从 \verb`TCHAR` 转换成 \verb`CString`.
\item \verb`CString Class` 需要 \verb`#include <atlstr.h>`
\item 控制台输出的时候用 \verb`wcout << str.GetString()`.  str 是 CString 的一个 object.
\item \verb`Format` 函数可以把数值转换为 \verb`cstring`. \verb`int num; CString str{}; str.Format(_T("%d"), num);`
\item \verb`CString::GetLength()` 函数可以返回字符个数.
\item \verb`CString::GetAt()` 可以获取某个字符
\item \verb`CString::Left(int count)` 获取左边 count 个字符 CString::Right 同理.
\item \verb`CString::GetMid(int start, int count)` 可以获取 substring
\item \verb`CString::Delete(int index, ind nCount)` 函数可以删除从第 index 到 index + nCount -1 的 nCount 个字符.
\item \verb`CString::Insert(ind index, CString str)` 可以把 str 插入到 index 的位置.
\item CString::TrimRight(TCHAR, str) 可以从右边重复删除 str 字符串. 另见 TrimLeft().
\item \verb`CString CString::Left(int nCount)` 提取前 nCount 个字符.
\item int CString::ReverseFind(TCHAR ch) 可以从后面开始搜索 ch, 返回 ch 的 ind (从 0 开始!).
\item 单个 char 可以总转换为 CString

\item 要在命令行中用 \verb`wcout` 输出中文, 要添加头文件 \verb`<io.h>` 和 \verb`<fcntl.h>`, 然后添加命令 \verb|_setmode(_fileno(stdout), _O_U16TEXT);| 要还原, 添加 \verb|_setmode(_fileno(stdout), _O_TEXT);| 注意只有在两条命令之间可以输出中文, 且不能使用 cout.
\end{itemize}
