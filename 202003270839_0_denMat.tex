% 密度矩阵(量子力学)

\pentry{矩阵的迹\upref{trace}, 投影算符\upref{projOp}}

\footnote{参考 Shankar, Principles of Quantum Mechanics 2ed, 以及 Wikipedia}若一个系综中的 $N$ 个系统中, 有 $n_i$ ($i = 1,2,\dots,k$) 个在状态 $\ket{i}$ (这里假设 $\ket{i}$ 是正交归一的). 那么这个系综可以用\textbf{密度矩阵(density matrix)}(或算符)描述
\begin{equation}
\rho = \sum_i p_i\ket{i}\bra{i}   
\end{equation}
其中 $p_i = n_i/N$ 是随机选一个系统, 处于状态 $\ket{i}$ 的概率. 若所有系统都处于同一个 $\ket{i}$, 那么这个系综就是\textbf{纯的(pure)}, 否则就是\textbf{混合的(mixed)}.

\subsection{非纯态的等效}
纯态 $\ket{\psi}$ 可以唯一地表示为密度矩阵 $\rho = \ket{\psi}\bra{\psi}$, 它的意义也相当明确.

然而对于非纯态, 我们可以使用不同的正交归一的(纯态)基底的概率组合来表示, 测量结果却是一样的. 因为测量并不能区分量子概率和经典概率.

例如有一束电子. 若这束电子是纯态的, 我们就可以通过多次测量得到这个纯态(除了一个总体相位因子). 例如通过测量 $\ket{y+}$ 和 $\ket{y-}$ 的概率, 我们可以确定 $\ket{x+} + c\ket{x-}$ 中的复数 $c$.

但若这束电子的自旋方向是随机的, 我们既可将其等效为随机的一半 $\ket{x+}$ 和 $\ket{x-}$ 构成的, 也可以等效为随机的一半 $\ket{y+}$ 和 $\ket{y-}$ 构成的. 虽然 “实际” 上它们是不一样的, 但任何测量都无法区分这两种情况. 所以密度矩阵可以是 $(\ket{x+}\bra{x+} + \ket{x-}\bra{x-})/2$ 也可以是 $(\ket{y+}\bra{y+} + \ket{y-}\bra{y-})/2$.

\subsection{测量}
对于某个物理量对应的算符 $\Omega$, 它的\textbf{系综平均值(ensemble average)}为
\begin{equation}
\ev{\bar\Omega} = \sum_i p_i \mel{i}{\Omega}{i}
\end{equation}
这个平均值既包含了每个 $\ket{i}$ 的平均, 又包含了对每个系统的平均.

系综平均也可以用迹表示为 $\opn{tr}(\Omega\rho)$. 根据迹的定义,
\begin{equation}
\opn{tr}(\Omega\rho) = \sum_j \mel{j}{\Omega\rho}{j} = \sum_{i,j} p_i\mel{j}{\Omega}{i} \braket{i}{j} = \sum_{i} p_i\mel{i}{\Omega}{i} = \ev{\bar\Omega}
\end{equation}
证毕.

对于纯态, 获得测量值 $\omega$ 的概率可以看作投影算符 $\mathbb P_\omega$ 的平均值(满足 $\mathbb P_\omega^2 = \mathbb P_\omega$)
\begin{equation}
P(\omega) = \abs{\braket{\omega}{\psi}}^2 = \braket{\mathbb P_\omega \psi}{\mathbb P_\omega \psi} = \mel{\psi}{\mathbb P_\omega}{\psi}
\end{equation}
所以对于混合态, 测量值 $\omega$ 的概率为
\begin{equation}
\overline{P(\omega)} = \opn{tr}(\mathbb P_\omega\rho)
\end{equation}

\subsection{密度矩阵的性质}
\begin{itemize}
\item 密度矩阵算符是厄米算符(自伴算符)

\item 密度矩阵的迹为 1
\end{itemize}
