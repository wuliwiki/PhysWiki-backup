% 泡利矩阵
% keys 泡利矩阵|自旋算符
% license Xiao
% type Tutor

\begin{issues}
\issueDraft 
欠缺四元数及泡利矩阵的同构关系,最后全文增加\hat。。
\end{issues}

\pentry{幺正变换\upref{Unitar},爱因斯坦求和约定\upref{EinSum}}

\begin{definition}{泡利矩阵}\label{def_pauliM_1}
泡利矩阵$\sigma_i$为 $2\times 2$ 幺正厄米矩阵,与自旋矩阵的关系为:
\begin{equation}
S_i=\frac{\hbar}{2}\sigma_i~.
\end{equation}
\end{definition}
在$\sigma_z$表象下,泡利矩阵的一般形式为:
\begin{equation}
\sigma_1 = \begin{pmatrix}
0 & 1\\
1 & 0
\end{pmatrix},\quad 
\sigma_2 = \begin{pmatrix}
0 & -i\\
i & 0
\end{pmatrix},\quad 
\sigma_3 = 
\begin{pmatrix}
1 & 0\\
0 & -1
\end{pmatrix}
~.
\end{equation}
有时也定义零号泡利矩阵为单位矩阵:
\begin{equation}
\sigma_0 = I=\pmat{1&0\\0&1}~.
\end{equation}

根据定义,我们可以直接得到泡利矩阵的对易关系:
\begin{equation}
[\sigma_i,\sigma_j]=2\mathrm {i}\epsilon ^{\,\,\, k}_{ij}\sigma_k,\{\sigma_i,\sigma_j\}=2\delta_{ij}I~,
\end{equation}
其中$i,j,k\in\{1,2,3\}$。$\delta_{ij}$是克罗内克函数,$\epsilon ^{\,\,\, k}_{ij}$是列维-奇维塔符号(Levi-Civita symbol),由对易运算的反对称性可知,这是一个关于下标的反对称张量。

\subsection{泡利矩阵的其他性质}
由泡利矩阵的一般形式得:
\begin{equation}
\sigma_i\cdot \sigma_i = \begin{pmatrix}1&0\\0&1\end{pmatrix}~,
\end{equation}
即每个泡利矩阵的平方都是单位矩阵,则 $\sigma_i \cdot  \sigma_i{}^\dagger = I$,所以泡利矩阵也是幺正矩阵。
\begin{exercise}{}
证明在别的角动量分量表象下,我们依然有:$\sigma_i^2=I$
\end{exercise}

由泡利矩阵的对易关系可得:
\begin{theorem}{}
\begin{equation}\label{eq_pauliM_1}
\sigma_i\sigma_j = \delta_{ij}I + \epsilon ^{\,\,\, k}_{ij}\sigma_k~.
\end{equation}
\end{theorem}
Proof.

$i=j$时,显然成立。
$i\neq j$时,利用$\epsilon ^{\,\,\, k}_{ij}=-\epsilon ^{\,\,\, k}_{ji}$及对易关系得证。


\begin{exercise}{}
证明:
$\mathrm{tr}[\sigma_i \sigma_j]= 2\delta_{ij}$。
\end{exercise}
\subsection{泡利矩阵与四元数}
特殊酉群$SU(2)$可以看作实数域上的四维线性空间