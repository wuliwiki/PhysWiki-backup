% 函数极限的性质
% 函数极限|序列极限

\pentry{函数的极限\upref{limfx}}

我们先列举几个函数极限的基本性质,由于它们的几何直观非常明显,这里不予证明.读者可以根据函数极限的定义尝试进行证明,练习用 $\epsilon$-$\delta$ 语言证明函数极限的性质.
\begin{theorem}{函数极限的唯一性}
  若函数 $f(x)$ 在 $x_0$ 处极限存在,则在 $x_0$ 处极限唯一.
\end{theorem}
\begin{theorem}{局部保序性}
设函数 $f(x),g(x)$ 在 $x_0$ 处极限存在,若 $f(x)\le g(x), \forall x\in U_0(x_0,\delta_0)$,那么 $\lim\limits_{x\rightarrow x_0} f(x)\le \lim\limits_{x\rightarrow x_0}g(x)$.
\end{theorem}
\begin{theorem}{局部保号性}
设函数 $f(x)$,若 $\lim\limits_{x\rightarrow x_0}f(x)=A>0$,那么存在 $x_0$ 的一个去心邻域 $U_0(x_0,\delta)$,满足 $\forall x\in U_0(x_0,\delta)$,$f(x)>0$.
\end{theorem}
\begin{theorem}{局部有界性}
设函数 $f(x)$,$\lim\limits_{x\rightarrow x_0}f(x)$ 存在(不为无穷大量),那么存在 $x_0$ 的一个去心邻域  $U_0(x_0,\delta)$,满足 $\exists M>0$,$\forall x\in U_0(x_0,\delta)$,$|f(x)|<M$,即 $f(x)$ 在 $U_0(x_0,\delta)$ 上有界.
\end{theorem}
\subsection{函数极限的四则运算}
设函数 $f(x),g(x)$,分别对于六种自变量的变化情况
\begin{equation}
  x\rightarrow x_0;\ x\rightarrow x_0^+;\ x\rightarrow x_0^{-};\ x\rightarrow \infty;\ x\rightarrow +\infty;\ x\rightarrow -\infty
\end{equation}
  若 $f(x)\rightarrow A,\ g(x)\rightarrow B$,那么可以证明
\begin{equation}
  \begin{aligned}
  &h_1(x)=f(x)+g(x)\rightarrow A+B\\
  &h_2(x)=f(x)-g(x)\rightarrow A-B\\
  &h_3(x)=f(x)\cdot g(x)\rightarrow A\cdot B\ (A\neq 0,B\neq 0)\\
  &h_4(x)=f(x)/ g(x)\rightarrow A/B\ (B\neq 0)
  \end{aligned}
\end{equation}
  若广义极限 $A,B$ 为无穷大量,则可以规定一些特殊的四则运算,例如 $(+\infty)+(+\infty)=+\infty,\ 
  (+\infty)\cdot (+\infty)=+\infty$ 等等.
\begin{exercise}{}
\begin{enumerate}
\item  设函数 $f(x)$,若 $\lim\limits_{x\rightarrow x_0}f(x)=A$,证明:对于任意 $r<A$,存在 $x_0$ 的一个去心邻域 $U_0(x_0,\delta)$,满足 $\forall x\in U_0(x_0,\delta)$,$f(x)>r$.(特别地,当 $r=0$ 时为局部保号性)
\item 求 $\lim\limits_{x\rightarrow +\infty}(x^2+1)/(1-2x^2)$.
\item 设函数 $f(x),g(x)$,若 $f(x)$ 在 $x_0=0$ 处极限为 $0$,而 $h(x)=f(x)/g(x)$ 在 $x_0=0$ 处极限为 $1$,证明 $g(x)$ 在 $x_0=0$ 处极限存在且也为 $0$.
\end{enumerate}

\end{exercise}

\subsection{函数极限的其他性质}
类似于序列极限,函数极限也有夹逼收敛原理:
\begin{theorem}{夹逼收敛原理}

  设函数 $f(x),h(x),g(x)$,

  若 $f(x)\le h(x)\le g(x),\ \forall x\in U_0(x_0,\delta_0)$,且 $\lim\limits_{x\rightarrow x_0}f(x)=\lim\limits_{x\rightarrow x_0} g(x)=A$,那么 $\lim\limits_{x\rightarrow x_0}h(x)=A$.
\end{theorem}

\begin{exercise}{}
思考:函数 $f(x)$ 在 $U(a,\delta_0)$ 上有定义,序列 $\{x_n\}$ 收敛于 $a$,则什么情况下 $\lim\limits_{n\rightarrow \infty}f(x_n)=f(a)$ 成立?
\end{exercise}
设 $f(x)=1/x,\ g(x)=x^2$

  容易证明 $\lim\limits_{x\rightarrow 4}f(x)=1/4,\lim\limits_{x\rightarrow 2}g(x)=4$.

  那么是否 $\lim\limits_{x\rightarrow 2}f(g(x))=f(\lim\limits_{x\rightarrow 2}g(x))=f(4)=1/4$ 呢?经验证是成立的.

  我们自然地就想到,是否对于任何函数 $f(x),g(x)$,都满足 $\lim\limits_{x\rightarrow a}f(g(x))=f(\lim\limits_{x\rightarrow a}g(x))$.然而答案是否定的.这种复合函数的极限运算需要满足一些限制条件,这将在后面的词条中提到.
\begin{exercise}{}
$f(x)=[x],g(x)=1-x^2$,判断 $\lim\limits_{x\rightarrow 0}f(g(x)) $ 与 $f(\lim\limits_{x\rightarrow 0}g(x))$ 是否相等.
\end{exercise}