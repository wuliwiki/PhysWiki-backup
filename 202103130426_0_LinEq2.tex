% 线性方程组解的结构 2

\pentry{线性映射的结构\upref{MatLS2}}

下面我们从线性映射和向量空间的角度理解线性方程组 $\bvec A \bvec x = \bvec b$.


\begin{definition}{线性方程}
对给定的线性映射 $A:X\to Y$ 以及给定 $b \in Y$, 线性方程为
\begin{equation}\label{LinEq2_eq2}
Ax = b
\end{equation}
所有满足该式的 $x \in X$ 的集合 $X_s$ 叫做方程的\textbf{解集}.
\end{definition}

首先注意 $A$ 未必把 $Y$ 中的每个元素都射中, 所有被射中的元素的集合 $Y_1 = A(X) \subseteq Y$ 叫做线性映射的值空间\upref{LinMap}. 所以只有 $b \in Y_1$ 时\autoref{LinEq2_eq2} 有解, 否则无解(解集为空集). 用映射的语言, 解集 $X_s$ 就是求集合 $\qty{b}$ 的逆向\upref{map} $A^{-1}(\qty{b})$.

当\autoref{LinEq2_eq2} 中 $b = 0$ 时方程叫做齐次方程. 根据定义, 齐次方程的解就是映射的零空间.

\begin{theorem}{线性方程}
线性方程的解集可以表示为
\begin{equation}\label{LinEq2_eq1}
X_s = X_0 + x_1
\end{equation}
其中 $x_1$ 为 $X_s$ 中的任意元素,  $X_0$ 为映射的零空间\upref{LinMap}.
\end{theorem}
说明: \autoref{LinEq2_eq1} 表示把 $X_0$ 中的每一个向量与 $x_1$ 相加得到集合 $X_s$. 注意当 $x_1 \ne 0$ 时容易证明解集 $X_s$ 不是一个向量空间(不存在零向量).

证明: 首先证明 $X_0 + x_1$ 中的元素满足 $Ax = b$.

要证明\autoref{MatLS2_the1}~\upref{MatLS2}
