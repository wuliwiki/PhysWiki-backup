% 布洛赫理论
% 晶格|波函数|薛定谔方程|周期函数|bloch|布洛赫

\begin{issues}
\issueDraft
\end{issues}

\pentry{定态薛定谔方程\upref{SchEq}}

在固体物理中,我们研究原子按一定规则排布而成的理想晶格,研究电子在一个具有晶格周期性的等效势场 $V(\bvec r)$ 中的运动.根据定态薛定谔方程,我们有
\begin{equation}
H\psi=\qty[-\frac{\hbar^2}{2m}\nabla^2+V(\bvec r)] \psi = E\psi
\end{equation}
势场 $V(\bvec r)$ 满足周期性
\begin{equation}
V(\bvec r)=V(\bvec r+\bvec R_n)
\end{equation}
其中 $\bvec R_n$ 为任意晶格矢量.

为了研究周期性势场下定态薛定谔方程的解,我们可以从最简单的一维薛定谔方程出发.

\subsection{一维薛定谔方程}
\footnote{参考\cite{GriffQ}}\textbf{布洛赫(Bloch)}理论也叫 Floquet 理论. 一维薛定谔方程中, 如果 $V(x)$ 是一个以 $a$ 为周期的函数, 那么解将满足
\begin{equation}\label{Bloch_eq1}
\psi(x+a) = \E^{\I K a}\psi(x)
\end{equation}
其中 $K$ 是一个常数. 令 $D(a)$ 为平移算符, 向右平移 $a$, 那么 $[D,H] = 0$. 所以存在能量和 $D$ 的共同本征态, 马上就得到\autoref{Bloch_eq1} .

波函数也可以记为
\begin{equation}
\psi(x) = \E^{\I K x} u(x)
\end{equation}
其中 $u(x)$ 是一个周期为 $a$ 的函数. 也就是波函数是一个振幅受周期性调制的平面波.

如果我们施加循环边界条件($N$ 是晶体一个方向的原子数, 阿伏伽德罗常数数量级)
\begin{equation}
\psi(x+Na) = \psi(x)
\end{equation}
得
\begin{equation}
K = \frac{2\pi}{a} \frac{n}{N} \qquad (n \in \mathbb Z)
\end{equation}

一个例子见一维 delta 势能晶格\upref{DelCry}.

\subsection{三维薛定谔方程}

(这参考的是 Brandsen?)布洛赫(Bloch)波函数定义为
\begin{equation}
\phi(\bvec r) = \E^{\I \bvec k \vdot \bvec r} u(\bvec r)
\end{equation}
其中 $u(\bvec r)$ 具有与晶格同样的周期性.
\begin{equation}
u(\bvec r)=u(\bvec r+\bvec R_n)
\end{equation}
$\bvec R_n$ 为晶格矢量.即在任何平移 $\bvec R_n$ 的操作下势场 $V(\bvec r)$ 都是不变的.我们引入描述晶格平移对称性的算符 $T_1,T_2,T_3$,它们的定义是
\begin{equation}
T_\alpha u(\bvec r)= u(\bvec r+\bvec a_\alpha),\alpha=1,2,3
\end{equation}
其中 $\bvec a_\alpha,\alpha=1,2,3$ 是晶格的三个基矢.它们是相互对易的,而且容易证明它们和哈密顿算符 $H$ 也相互对易.下面我们要做的就是找出 $H,T_1,T_2,T_3$ 的共同本征态,用以描述晶格中的电子.设
\begin{equation}
\begin{aligned}
&H\psi=E\psi,\\
&T_\alpha \psi = \lambda_\alpha \psi, \alpha=1,2,3
\end{aligned}
\end{equation}
\addTODO{需要增加原胞、布拉伐格子、倒格子相关的词条}
设晶格在三个方向上的原胞数量分别为 $N_1,N_2,N_3$,那么可以引入晶格的周期性边界条件:
\begin{equation}
\psi(\bvec r)=\psi(\bvec r+N_\alpha \bvec a_\alpha),\alpha=1,2,3
\end{equation}
可以得出 $\lambda_\alpha$ 具有下列形式
\begin{equation}
\lambda_\alpha=e^{ \dfrac{2\pi i l_\alpha}{N_\alpha}}
\end{equation}
其中 $l_\alpha$ 为整数.

如果引入倒格子矢量 $\bvec b_1,\bvec b_2,\bvec b_3$,满足 $\bvec a_i \cdot \bvec b_j = 2\pi \delta_{ij}$,那么
\begin{equation}
\begin{aligned}
&\lambda_\alpha = e^{i \bvec k \cdot \bvec a_\alpha}\\
&\bvec k=\sum_\alpha\frac{l_\alpha}{N_\alpha} \bvec b_\alpha
\end{aligned}
\end{equation}
我们可以将三维晶格的布洛赫定理写成以下形式
\begin{equation}
\begin{aligned}
\psi\qty(\bvec r+\sum_\alpha m_\alpha\bvec a_\alpha) &= T_1^{m_1} T_2^{m_2} T_3^{m_3} \psi(\bvec r)\\
&=e^{i\bvec k\cdot \qty(\sum_\alpha m_\alpha\bvec a_\alpha)}\psi(\bvec r)
\end{aligned}
\end{equation}
其中 $\bvec k$ 称为简约波矢,它对应于平移算符操作本征值的量子数 $l_1,l_2,l_3$.
