% 光电离时间延迟

\begin{issues}
\issueDraft
\end{issues}

\pentry{量子散射的延迟\upref{tDelay}, 含时微扰理论\upref{TDPT}}

本文使用原子单位制\upref{AU}. 例如一个有限深势阱(短程势)中有一个束缚态, 被一个电场波包电离之后, 光电子波包逃出势阱. 那么光电子波包的延迟是多少呢? 我们以下使用一阶含时微扰理论\upref{TDPT} 来分析.

\subsection{一维势阱}
我们假设势阱在坐标原点, 且 $t = 0$ 时电场波包的中心刚好到达原点.

初态和末态能量分别为 $E_0, E$, 令 $\omega = E - E_0$, 不含时的末态记为 $\psi_E(x)$, 并令其为满足边界条件
\begin{equation}\label{HeAna2_eq2}
\psi_E(x \to +\infty) \propto \E^{\I kx}
\end{equation}
其中 $A(k)$ 是实函数. 那么一阶微扰系数为(\autoref{TDPTc_eq1}~\upref{TDPTc})
\begin{equation}
c_E(t) = -\I  \mel{\psi_E}{H'}{\psi_0} \tilde f(-\omega)
\end{equation}
电离波包可以由末态展开:
\begin{equation}
\psi(x, t) = \int c_E(t) \psi_E(x)\E^{-\I E t} \dd{E}
\end{equation}
那么根据\autoref{tDelay_eq4}~\upref{tDelay} 延迟为
\begin{equation}\label{HeAna2_eq3}
\tau = \pdv{E} \arg [c_E]
\end{equation}
对于没有 chirp 的电场波包, $\tilde f(-\omega)$ 相位恒定(想象一个高斯波包乘以正弦函数的傅里叶变换). 所以随 $E$ 变化的相位只有矩阵元, 绝对时间延迟为
\begin{equation}\label{HeAna2_eq1}
\tau = \pdv{E} \arg \mel{\psi_E}{H'}{\psi_0}
\end{equation}
这个延迟是相对与 $t = 0$ 也就是电磁波包到达原点的时间.

\subsection{三维库仑势能光电离延迟}\label{HeAna2_sub1}
还是考虑类氢原子, 库仑球面波记为 $\ket{C_{l,m}(k)}$, 包含球谐函数 $Y_{l,m}(\uvec r)$. 含时微扰($E = k^2/2$)
\begin{equation}\label{HeAna2_eq4}
\ket{\psi(t)} = \sum_{l,m} \int c_{l,m}(k) \ket{C_{l,m}(k)} \E^{-\I E t} \dd{k}
\end{equation}
其中系数为
\begin{equation}
c_{l,m}(k) = -\sqrt{2\pi}\I \mel{C_{l,m}}{z}{i} \tilde f(-\omega)
\end{equation}
相位不随 $k$ 或 $E$ 变化. 注意对于氢原子束缚态作为初始态, 由于选择定则 $\Delta m = 0$, $\Delta l = \pm 1$, 只有最多两个 $c_{l,m}$ 不为零.

投影到库伦平面波为 $\ket{C(\bvec k)} = \sum_{l,m} a_{l,m}^{(-)}(\bvec k) \ket{C_{l,m}}$, 把 $\ket{\psi(t)}$ 投影到 $\ket{C(\bvec k)}$ 上有(\autoref{CulmWf_eq10}~\upref{CulmWf})
\begin{equation}
\braket{C(\bvec k)}{\psi(t)} = \sum_{l,m} a_{l,m}^{(-)*}(\bvec k) c_{l,m}(k) \E^{-\I E t} 
= \sum_{l,m} \frac{\I^{-l}}{k} \E^{\I\sigma_l} Y_{l,m} (\uvec k) c_{l,m}(k) \E^{-\I E t}
\end{equation}

求和中每一项可以看作一个独立的波包, 具有不同的延迟. 所以和\autoref{HeAna2_eq3} 同理, 每一项的延迟仅由库仑相移贡献
\begin{equation}
t_{EWS}^C = \pdv{\sigma_l}{E}
\end{equation}
但是库仑平面波本身还存在不收敛的相移 $-\eta\ln (2kr)$, 这对能量求导以后是一个负的, 会带来一个负延迟, 即提前.

库仑渐进相位为(\autoref{CulmF_eq7}~\upref{CulmF})
\begin{equation}
F_l(k, r) \overset{r\to \infty}{\longrightarrow} \sin\qty[kr - \frac{\pi l}{2} + \frac{Z}{k}\ln(2kr) + \sigma_l(k)] % Z > 0
\end{equation}
\begin{equation}
\sigma_l(k) = \arg[\Gamma(l+1-\I Z/k)] \qquad (Z > 0)
\end{equation}
\begin{figure}[ht]
\centering
\includegraphics[width=8cm]{./figures/HeAna2_1.png}
\caption{库仑相移 $\sigma_l(k)$} \label{HeAna2_fig1}
\end{figure}
其中 $\sigma_l(k)$ 产生的相移与距离无关, 可以单纯看作是非直线轨道带来的延迟(不确定), 并且是 XUV 波包一瞬间带来的. $kr\ln(2kr)$ 只和距离有关, 带来的延迟是库仑力产生的, 而且并不收敛. 那么光电离的总延时就是
\begin{equation}\label{HeAna2_eq5}
\tau = \pdv{E} \arg\qty[\sigma_l(k) + \frac{Z}{k}\ln(2kr)] % Z > 0
\end{equation}
第一项是瞬间的, 而第二项是波包移动过程中缓慢积累且不收敛的.

一定要强调光电离延迟 $\tau$ 和 streaking 延迟是不一样的. 在 Streaking 实验中, 前两项都会直接转换为 streaking 延迟(因为电离瞬间产生). 第三项在上式中不收敛, 但 streaking 延迟却总是收敛的. 配合 IR 的扰动就变为了 $t_{CLC}$ 项.

\subsection{直接投影}\label{HeAna2_sub2}
对于氢原子, 一阶微扰也可以直接投影到库仑平面波(\autoref{CulmWf_eq7}~\upref{CulmWf}, 取减号) $\ket{C(\bvec k)}$ ($E = k^2/2$)
\begin{equation}
\ket{\psi(t_f)} = \int c(\bvec k)\ket{C(\bvec k)} \E^{-\I E t} \dd[3]{k}
\end{equation}
\begin{equation}
c(\bvec k) = -\sqrt{2\pi}\I\mel{C(\bvec k)}{z}{i} \tilde f(-\omega)
\end{equation}
这相对于\autoref{HeAna2_eq4} 不过是变换了一下基底. 根据边界条件\autoref{CulmWf_eq4}~\upref{CulmWf}, $C(\bvec k)$ 本身也自带一个不收敛的相位 $-\eta \ln 2kr$, 而电场的傅里叶变换 $\tilde f(-\omega)$ 相位恒定, 所以对某个方向 $\uvec k$, 有
\begin{equation}
\tau = \pdv{E} \arg[\mel{C(\bvec k)}{z}{i} + \frac{Z}{k}\ln 2kr]
\end{equation}
这和\autoref{HeAna2_eq5} 只是使用的基底不同, 对比后会发现
\begin{equation}
\pdv{E} \arg\mel{C(\bvec k)}{z}{i} = \pdv{E} \arg\qty[\sigma_l(k)]
\end{equation}
注意\autoref{HeAna2_eq5} 和初态无关, 所以这里同样无关. 但实际上 TDSE 的结果是有关系的, 要考虑初态, 只有使用 Pazourek 的 dDLC 修正项.

\subsection{双电子系统}
双电子完全也可以分别使用\autoref{HeAna2_sub1} 和\autoref{HeAna2_sub2} 两种方法. 这是完全一样的. 一阶微扰的末态必须是能量本征态, 氦原子和 $H$ 对易的算符只有 $L^2, L_z$, 所以精确的球面波能量本征态记为 $\ket{E,L,M}$. 它可以分解为
\begin{equation}
\ket{E,L,M} = \frac{1}{r_1r_2}\sum_{l_1,l_2}\psi_{l_1,l_2}^{L, M}(r_1, r_2)\mathcal Y_{l_1,l_2}^{L, M}(\uvec r_1, \uvec r_2)
\end{equation}
理论上要解本征态 $\ket{E,L,M}$, 就用许多无限长的长方形网格, 每张网格是一个 $(l_1,l_2,L)$ 分波, 网格之间被 $1/r_{12}$ 势能耦合. 网格上 $r_1$ 在区间 $(0,a]$, $r_1 = a$ 处边界条件是波函数为 “零”, 另外两条边 $r_1 = r_2 = 0$ 也有波函数为零. $r_2$ 的区间为 $(0,\infty)$, 这样才能保证 $E$ 可以连续取值而不是离散的.

但波函数有 6 维度, 我们只确定了三个量子数 $H,L,M$, 所以还可以规定在 $r_2\to\infty$ 时, 某个 $(l_1,l_2)$ 分波是束缚态和库仑波函数 $Z=1$ 的乘积
\begin{equation}
\psi_{l_1,l_2}^{L, M} \overset{r_2\to\infty}{\longrightarrow} r_1 R_{n_1,l_1}(r_1) F_{l_2}(r_2)
\end{equation}
所有其他的 $(l_1,l_2)$ 分波在 $r_2\to\infty$ 波函数都消失. 这样就可以把这个态记为 $\ket{L,M; l_1,l_2, n_1, k_2}$. 分号用于提醒后面的量子数只是无穷远处的边界条件, 这个态仍然只有 $E,L,M$ 是 well defined. 如果把这些态用 CG 系数线性组合一下, 就可以得到 $\ket{l_1, m_1, l_2,m_2,n_1,k_2}$, 注意此时除了能量, 剩下的量子数都只是在 $r_2\to+\infty$ 处是 well defined.

不考虑 IR, 在 XUV 结束后把波函数投影到精确的 $\ket{E; m_1,m_2,l_1,l_2,k_2}$ 上就可以得到精确的单电离动量谱, 但由于在 $r_2\to\infty$ 处 $\ket{E; m_1,m_2,l_1,l_2,k_2}\approx \ket{n_1,l_1,m_1}\ket{l_2,m_2,k_2}$, 也可以等到波包足够远时直接投影到后者.

所以我们无法控制 $r_2\to \infty$  的渐进相位, 可以记为 $\delta_l(E)$.





和平面波单电离散射态分别记为
\begin{equation}
\ket{n_1,l_1,l_2,L,\bvec k}
\end{equation}
它们可以近似看作是, 但是后面的相位是不一样的.

但氦原子的本征态是 $L,M$ 算符的本征态, 而且从基态的一阶微扰是 $L = 1, M = 0$.所以 XUV 过后的波函数为
\begin{equation}
\Psi(\bvec r_1, \bvec r_2) = \frac{1}{r_1 r_2}\sum_{l_1,l_2} \psi_{l_1,l_2}(r_1, r_2)\mathcal Y_{1,0}^{l_1,l_2}(\uvec r_1, \uvec r_2)
\end{equation}
对于 $L = 1$, 根据三角形法则(\autoref{AMAdd_fig2}~\upref{AMAdd}), 可以支持 $\abs{l_2 - l_1} \le 1$ 除了 $(0,0)$ 的所有分波. 


\subsection{回收的内容}
\addTODO{如果是长程库仑势能, 延迟就会取决于距离而不收敛. 但动量却收敛, 所以使用 streaking 仍然会获得一个延迟, 但这个延迟和上面的是两码事, 然而 Pazourek 仍然使用\autoref{HeAna2_eq1}, 这里面还有更深的奥妙……}

如果\autoref{HeAna2_eq2} 中本来就有额外的取决于能量的相位 $\delta(E)$ 那么同样需要加到\autoref{HeAna2_eq1} 中
\begin{equation}
\tau = \pdv{E} \arg \mel{\psi_E}{H'}{\psi_0} + \pdv{E} \delta(E)
\end{equation}
并且这个相位可能还会取决于距离和其他参数 $\delta(E, x)$. 例如库仑相移中, 这个相位产生的延迟并不随距离收敛.

\addTODO{另一个问题是, 延迟不仅和 He+ 的 $n$ 有关还与 $l$ 有关, 或者说还与 Stark 效应的好量子数有关, 那么 Pazourek 使用的是哪个呢? 还是说取平均呢? 还是要仔细看 Pazo12}
