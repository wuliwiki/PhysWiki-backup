% Floquet 理论
% 薛定谔方程|简谐|周期变化|周期函数|傅里叶级数

\pentry{布洛赫理论\upref{Bloch}, 傅里叶级数(指数)\upref{FSExp}}

\footnote{参考\cite{Bransden}}Floquet 理论在原子的多光子过程中有大量应用。 如果含时哈密顿算符随时间周期 $T$ 变化
\begin{equation}
H(t+T) = H(t)~.
\end{equation}
那么波函数为
\begin{equation}\label{eq_Floque_1}
\psi(x, t) = \exp(-\I Et) F(x, t)
\end{equation}
其中 $F(x, t)$ 同样是时间的周期函数, 周期为 $T$。 所以又可以展开为傅里叶级数
\begin{equation}
F(x, t) = \sum_{n} \E^{-\I n\omega t} F_n(x)
\end{equation}
代入\autoref{eq_Floque_1} 就得到 Floquet-Fourier 展开
\begin{equation}
\psi(x, t) = \E^{-\I Et} \sum_{n} \E^{-\I n\omega t} F_n(x)
\end{equation}
如果我们令
\begin{equation}
H(t) = H_0 + \sum_{n} \E^{-\I n\omega t} H_n
\end{equation}
一起代入含时薛定谔方程得到一组完全不含时的方程
\begin{equation}\label{eq_Floque_2}
\sum_{i=-\infty}^{+\infty} H_{n-i} F_i(x) = (E + n\omega - H_0) F_n(x)
\end{equation}
如果 $H$ 是简谐的
\begin{equation}
H(t) = H_0 + H_+\E^{-\I\omega t} + H_-\E^{+\I\omega t}
\end{equation}
那么\autoref{eq_Floque_2} 化简为
\begin{equation}
H_{+} F_{n-1}(x) + H_{-} F_{n+1}(x) = (E + n\omega - H_0) F_n(x)
\end{equation}
