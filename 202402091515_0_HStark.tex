% 类氢原子斯塔克效应(微扰)
% license Xiao
% type Tutor

\begin{issues}
\issueDraft
\end{issues}

\pentry{不含时微扰理论\nref{nod_TIPT}}{nod_8e0d}

微扰理论($\mathcal{E_z}$ 是 $z$ 方向电场):
\begin{equation}
H' = \mathcal{E_z} z~.
\end{equation}
矩阵元为
\begin{equation}\label{eq_HStark_1}
H'_{l',l} = \mathcal{E_z}\mel{\psi_{n,l',m}}{z}{\psi_{n,l,m}}~.
\end{equation}

但事实上氢原子加上匀强电场后是不存在数学上严格的束缚态的,因为无论电场多弱,在电场反方向的某个距离外,势能都会小于基态能量,使波函数变为散射态。在含时薛定谔方程中,波函数可能会有长时间处于微扰理论给出的 “束缚态”,但这只能算是一种\textbf{亚稳态(metastable state)},仍会有不为零的隧道电离概率。

\begin{example}{氢原子 $n=2$ 的斯塔克效应}
先考虑 $n=2$, $m=0$ 的情况, 这是一个二维希尔伯特子空间,基底为 $\ket{2,0,0}$ 和 $\ket{2,1,0}$。 根据\autoref{tab_HDipM_1}~\upref{HDipM}, \autoref{eq_HStark_1} 为
\begin{equation}
\mat H' = -3\mathcal{E_z}\pmat{0 & 1\\ 1 & 0}~.
\end{equation}
本征值为 $E_{\pm}^1 = \mp 3\mathcal{E_z}$, 好本征态为 $\ket{2\pm 0} = (\ket{200} \pm \ket{210})/{\sqrt 2}$, 也被称为 \textbf{Stark 态}。

\begin{figure}[ht]
\centering
\includegraphics[width=8cm]{./figures/cf2a4c9b0548dbae.png}
\caption{$\ket{2+}$ 的概率密度函数的 $x$-$z$ 切面, 可见电子向下偏移, 电场向上为正, 所以本征能量变小。 $\ket{2-}$ 态是此图上下翻转, 本征能量变大。} \label{fig_HStark_1}
\end{figure}

不要以为\autoref{fig_HStark_1} 是外电场扭曲波函数的结果, $\ket{2\pm}$ 本身就是无电场的氢原子 $n=2$ 本征态。 施加了电场后波函数反而需要进一步修正。

从经典电磁学角度来理解, 电偶极子在电场中的能量(\autoref{eq_eleDP2_1}~\upref{eleDP2})等于 $-d_z \mathcal{E}_z$, 其中 $d_z$ 是 $z$ 方向电偶极子
\begin{equation}
d_z^{(\pm)} = \mel{2\pm}{z}{2\pm} = \pm 3~.
\end{equation}
\end{example}

\addTODO{如果初始时, 波函数处于 $n=2$ 子空间的任意状态, 例如 $\ket{20}$, 那么当逐渐施加电场后, 波函数会如何变化?}

\begin{example}{氢原子的极化率}
若氢原子处于某个好量子态,使用一阶微扰求其极化率(polarizability) $\alpha$,定义为 $\bvec p_z = \alpha \bvec{\mathcal E_z}$,其中 $\bvec p_z$ 为 $z$ 方向的电偶极子, $\bvec{\mathcal E_z}$ 为电场强度。(实验数据参考\href{https://physicspages.com/pdf/Electrodynamics/Polarizability\%20of\%20hydrogen.pdf}{这篇})

该问题中, $H^1 = z$, $\bvec p_z = -\mel{\psi_n}{z}{\psi_n}$。 做波函数的一阶近似 $\psi_n \approx \psi_n^0 + \psi_n^1$, 有
\begin{equation}\ali{
\bvec p_z &\approx -\mel{\psi_n^0 + \psi_n^1}{H^1}{\psi_n^0 + \psi_n^1}\\
&= -\mel{\psi_n^0}{H^1}{\psi_n^0} - 2\Re[\mel{\psi_n^0}{H^1}{\psi_n^1}] - \mel{\psi_n^1}{H^1}{\psi_n^1}~.
}\end{equation}
其中第一项就是 $E_n^1$。 第二项第三项分别是二阶和三阶小量,所以第三项可忽略。 以后会发现第二项的 $\mel{\psi_n^0}{H^1}{\psi_n^1}$ 就是二阶能量修正 $E_n^2$(\autoref{eq_TIPT2_4}~\upref{TIPT2}),所以
\begin{equation}
\bvec p_z \approx -E_n^0 - 2E_n^2~.
\end{equation}
其中 $E_n^0$ 和电场成正比,而(\autoref{eq_TIPT2_2}~\upref{TIPT2})
\begin{equation}
E_n^2 = \sum_{m}^{E_m\ne E_n} \frac{\abs{\mel{\psi_m^0}{H^1}{\psi_n^0}}^2}{E_n^0-E_m^0}~.
\end{equation}
正比于电场平方。

例如计算其氢原子好本征态
\begin{equation}
\ket{2,\pm, 0} = \ket{2,0,0}\pm \ket{2,1,0}~.
\end{equation}
则 $E_n^1 = -3$, 但 $E_n^2$ 却比较难算。
\addTODO{算啊}
\end{example}


\subsection{好本征态列表(z 方向电场)}
\pentry{氢原子的跃迁偶极子矩阵元列表\nref{nod_HDipM}}{nod_c9f2}

注意第 $n$ 能级有 $n^2$ 基底。
\begin{itemize}
\item $n=2$
\begin{equation}
\ket{2,0,0},\ket{2,1,0} \Rightarrow-\frac{3}{Z}\pmat{0 & 1\\ 1 & 0}~.
\end{equation}
\begin{equation}
\mp\frac{3}{Z} \Rightarrow \frac{1}{\sqrt{2}}(\ket{2,0,0}\pm\ket{2,1,0})~.
\end{equation}
\begin{equation}
\frac{0}{Z} \Rightarrow \ket{2,1,\pm 1}~.
\end{equation}


\item $n=3$
\begin{equation}
\ket{3,0,0},\ket{3,1,0},\ket{3,2,0}\Rightarrow
-\frac{3\sqrt{3}}{Z}\pmat{0 & \sqrt{2} & 0\\ \sqrt{2} & 0 & 1\\ 0 & 1 & 0}~.
\end{equation}
\begin{equation}
\mp\frac{9}{Z} \Rightarrow \frac{1}{\sqrt{3}}\ket{3,0,0} \pm \frac{1}{\sqrt{2}}\ket{3,1,0} + \frac{1}{\sqrt{6}}\ket{3,2,0}~.
\end{equation}
\begin{equation}
\frac{0}{Z} \Rightarrow \frac{1}{\sqrt{3}}\ket{3,0,0} - \sqrt{\frac{2}{3}}\ket{3,2,0}~.
\end{equation}
\end{itemize}
\begin{equation}
\ket{3,1,\pm1},\ket{3,2,\pm1} \Rightarrow
-\frac{3\sqrt{3}}{Z}\pmat{0 & 1\\1 & 0}~.
\end{equation}
\begin{equation}
\mp\frac{3\sqrt{3}}{Z} \Rightarrow \frac{1}{\sqrt 2}(\ket{3,0,1}\pm\ket{3,1,1})~.
\end{equation}
