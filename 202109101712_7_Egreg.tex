% 高斯绝妙定理
% Theorema Egregium|高斯曲率|Gaussian curvature|微分几何

\pentry{黎曼联络\upref{RieCon}}

本节我们来讨论高斯绝妙定理.

在最初的古典微分几何研究中,常常需要将流形理解为某个$\mathbb{R}^n$空间中的超平面,进行具体的、复杂的计算,从而得到其性质.比如说,很多二维的流形都可以表示为三维空间中的一个曲面,比如球、平面、双曲面等等;也有的二维流形没法在三维空间中表示,比如Klein瓶,但是在四维空间中一定可以表示的\footnote{题外话:任意$n$维实流形,都可以嵌入到$\mathbb{R}^{2n}$中.}.

高斯第一个发现“曲率”这一\textbf{内蕴}量,并把该发现命名为\textbf{绝妙定理(Gauss's Theorema Egregium)}\footnote{注意“绝妙定理(Theorema Egregium)”是拉丁语.}.连高斯都觉得绝妙的发现到底是什么呢?这就需要解释何为“内蕴”了:它是指,曲率的计算不依赖于具体的嵌入(即不需要关心)









