% 电磁场的能动张量
% 能动张量|诺特定理|电磁场
\pentry{经典场论基础\upref{classi},电磁场的能量守恒、坡印廷矢量\upref{EBS},电磁场的动量守恒、动量流密度张量\upref{EBP}}

我们继续使用自然单位制,令 $\mu_0=\epsilon_0=c=1$ 来简化表达.依照习惯,上下标使用希腊字母如 $\mu, \nu$ 时,取值范围为 $\{0, 1, 2, 3\}$;使用拉丁字母如 $i, j$ 时,取值范围为 $\{1, 2, 3\}$.

\subsection{电动力学的守恒量}
根据经典场论\upref{classi} 中能动张量 ${T^\mu}_\nu$ 的定义
\begin{equation}
T^\mu{}_\nu \equiv -\frac{\partial \mathcal L}{\partial (\partial_\mu \phi)} \partial_\nu \phi + \mathcal L \delta^\mu{}_\nu
\end{equation}
可以写出电磁场的能动张量
\begin{equation}\label{EMtens_eq1}
T^\mu{}_\nu =-\frac{\partial \mathcal L}{\partial (\partial_\mu A_\rho)} \partial_\nu A_\rho + \mathcal L \delta^\mu{}_\nu
\end{equation}
根据诺特定理,能动张量对应着四个守恒流:
\begin{equation}
\partial_\mu T^\mu{}_\nu=0
\end{equation}
其中 $T^0{}_0$ 对应着哈密顿量密度,或者说电磁场的能量密度;$T^0{}_i$ 对应着电磁场的动量密度.我们已经推导过电磁场的能量密度为 $\mathcal H=\frac{1}{8\pi}(\bvec E^2+\bvec B^2)$ \upref{EBS},也推导过电磁场的动量密度为 $\mathcal P=\frac{1}{4\pi} \bvec E\times \bvec B$\upref{EBP}.下面可以通过推导 \autoref{EMtens_eq1} 验证这些结论.

电磁场的拉氏量密度为
\begin{equation}
\begin{aligned}
\mathcal{L}&=-\frac{1}{16\pi}F^{\mu\nu} F_{\mu\nu}\\
&=-\frac{1}{16\pi}(\partial^\mu A^\nu-\partial^\nu A^\mu)(\partial_\mu A_\nu-\partial_\nu A_\mu)
\end{aligned}
\end{equation}
所以
\begin{equation}
\begin{aligned}
T^\mu{}_\nu&=-\frac{\partial \mathcal{L}}{\partial (\partial_\mu A_\rho)}\partial_\nu A_\rho+\mathcal{L}\delta^\mu{}_\nu\\
&=\frac{1}{4\pi}F^{\mu\rho}\partial_\nu A_\rho-\frac{1}{16\pi}\delta^\mu{}_\nu F^{\sigma\lambda}F_{\sigma\lambda}
\end{aligned}
\end{equation}
或者将 $\nu$ 指标上升:
\begin{equation}
T^{\mu\nu}=\frac{1}{4\pi}F^{\mu\rho}\partial^\nu A_\rho - \frac{1}{16\pi} \eta^{\mu\nu}F^{\sigma\lambda}F_{\sigma\lambda}
\end{equation}
然而现在的 $T^{\mu\nu}$ 是不对称的.为了使 $T^{\mu\nu}$ 成为对称张量,需要增添一项 $\partial_\rho \psi^{\mu\nu\rho}$,最终可以将 $T^{\mu\nu}$ 写为
\begin{equation}
\begin{aligned}
T^{\mu\nu}&=\frac{1}{4\pi}F^{\mu\rho}(\partial^\nu A_\rho-\partial_\rho A^\nu) - \frac{1}{16\pi} \eta^{\mu\nu}F^{\sigma\lambda}F_{\sigma\lambda}\\
&=\frac{1}{4\pi}F^{\mu\rho}F^\nu{}_\rho - \frac{1}{16\pi} \eta^{\mu\nu}F^{\sigma\lambda}F_{\sigma\lambda}
\end{aligned}
\end{equation}

现在来计算 $T^{\mu\nu}$ 的每一项.
\begin{equation}
\begin{aligned}
T^{00}&=\frac{1}{4\pi}\bvec E \cdot \bvec E+\frac{1}{16\pi}(2\bvec B^2-2\bvec E^2)\\
&=\frac{1}{8\pi}(\bvec E^2+\bvec B^2)=\mathcal{H}
\end{aligned}
\end{equation}
\begin{equation}
\begin{aligned}
T^{0i}&=\frac{1}{4\pi}E^j F^i{}_j=\frac{1}{4\pi}\epsilon_{ijk}E_jB_k\\
&=\frac{1}{4\pi}\bvec E\times \bvec B=\mathcal{P}_i
\end{aligned}
\end{equation}
这些恰好对应着电磁场的能量与动量密度.
\addTODO{待检查}
\addTODO{其余推导}