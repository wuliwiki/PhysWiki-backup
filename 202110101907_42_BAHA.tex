% Baker-Hausdorff公式
% keys Baker|Hausdorff|定理

\begin{issues}
\issueTODO
\end{issues}


Baker-Hausdorff公式是一个相当有用的公式.在数学上,它可用于给出李群-李代数对应的深层结果的相对简单的证明;在量子力学中,它可实现系统哈密顿量在薛定谔绘景和海森堡绘景的转换,并在微扰论中也有诸多应用.本节将给出该公式的一个证明和由它导出的一些重要的结果.

\textbf{Baker-Hausdorff公式}是指
\begin{equation}
\begin{aligned}
\E ^{A}B\E^{-A}=\sum_{n=0}^{\infty}\frac{1}{n!}A^{(n)}\\
A^{(n)}\equiv\underbrace{[A,[A,\cdots,[A}_{n\text{个}},B]]\cdots]
\end{aligned}
\end{equation}

\subsection{证明}
在证明之前,先来证明下面的引理.
\begin{lemma}{}\label{BAHA_lem1}
\begin{equation}\label{BAHA_eq1}
A^{(n)}\equiv\underbrace{[A,[A,\cdots,[A}_{n\text{个}},B]]=\sum_{m=0}^{n}(-1)^{n-m}C_{n}^{m}A^mBA^{n-m}
\end{equation}
\end{lemma}
\textbf{证明:}我们用数学归纳法来证明.
$A^{(0)},A_1$显然成立:
\begin{equation}
\begin{aligned}
A_0&=B=\sum_{m=0}^{0}(-1)^{0-m}C_{0}^{m}A^mBA^{0-m}\\
A_1=&[A,B]=AB-BA=\sum_{m=0}^{1}(-1)^{1-m}C_{1}^{m}A^mBA^{1-m}
\end{aligned}
\end{equation}

假设对 $n=k-1$ 时\autoref{BAHA_eq1} 成立,则
\begin{equation}
\begin{aligned}
A_k&=[A,A_{k-1}]=AA_{k-1}-A_{k-1}A\\
&=A\sum_{m=0}^{k-1}(-1)^{k-1-m}C_{k-1}^{m}A^mBA^{k-1-m}-\qty(\sum_{m=0}^{k-1}(-1)^{k-1-m}C_{k-1}^{m}A^mBA^{k-1-m})A\\
&=\sum_{m=0}^{k-1}(-1)^{k-1-m}C_{k-1}^{m}A^{m+1}BA^{k-1-m}-\sum_{m=0}^{k-1}(-1)^{k-1-m}C_{k-1}^{m}A^mBA^{k-m}\\
&=A^{k}B+\sum_{m=0}^{k-2}(-1)^{k-1-m}\qty(C_{k-1}^{m}+C_{k-1}^{m+1})A^{m+1}BA^{k-1-m}-(-1)^{k-1}BA^k\\
&=A^{k}B+\sum_{m=0}^{k-2}(-1)^{k-1-m}C_{k}^{m+1}A^{m+1}BA^{k-1-m}-(-1)^{k-1}BA^k\\
&=A^{k}B+\sum_{m=1}^{k-1}(-1)^{k-m}C_{k}^{m}A^{m}BA^{k-m}+(-1)^{k}BA^k\\
&=\sum_{m=0}^{k}(-1)^{k-m}C_{k}^{m}A^{m}BA^{k-m}
\end{aligned}
\end{equation}

由数学归纳法原理,引理得证.现在,我们利用\autoref{BAHA_lem1} 来证明Baker-Hausdorff公式.
\subsubsection{Baker-Hausdorff公式的证明}
由\autoref{BAHA_lem1} 
\begin{equation}
[\E^A,B]=
\end{equation}