% 引力子
% license CCBYSA3
% type Wiki

(本文根据 CC-BY-SA 协议转载自原搜狗科学百科对英文维基百科的翻译)

在量子引力理论中,\textbf{引力子}是假设的关于重力的量子,一个调节重力的基本粒子。由于广义相对论的重归一化是一个突出的数学问题,因此没有引力子的完整量子场论。在弦理论中,引力子是基本弦的无质量状态,被认为是量子引力的一致理论。

如果引力子存在,它应该是无质量的,因为引力的作用范围很长,并且近似以光速传播。引力子一定是spin-2玻色子,因为引力的来源是应力-能量张量,一个二阶张量(与电磁的spin-2光子相比,其来源是四电流一阶张量)。此外,可以证明任何无质量的spin-2场都会产生与引力无法区分的力,因为无质量的spin-2场会像引力相互作用相同的方式耦合到应力-能量张量。这个结果表明,一个无质量的spin-2粒子一定是引力子。[1]

\subsection{理论}
据推测,重力相互作用是由一种尚未发现的基本粒子介导的,这种粒子被称为引力子。另外三个已知的自然力是由基本粒子介导的:光子的电磁作用,胶子的强相互作用以及了W、Z玻色子的弱相互作用。所有这三种力似乎都具有粒子物理学的标准模型。在经典极限条件下,一个成功的引力子理论可以简化为广义相对论,在弱场极限下,它可以简化为牛顿万有引力定律。[2][3][4]

引力子这个术语最初是由苏联物理学家Dmitrii Blokhintsev和F. Gal'perin于1934年创造的。[5]

\subsubsection{1.1 引力子和重正化}
当描述引力子相互作用时,费曼图的经典理论以及诸如one-loop图的半经典修正表现的非常平常。然而,具有至少两个回路的费曼图会导致紫外线发散。因为与量子电动力学和Yang-Mills理论模型不同,量子化的广义相对论不是摄动的,因而不能被微扰地重新归一化。因此,物理学家通过摄动方法计算粒子发射或吸收引力子的概率,得到了无法计算的答案,该理论失去了预测的准确性。这些问题和互补近似框架是证明需要一个比量子化广义相对论更统一的理论来描述在普朗克尺度附近行为的基础。

\subsubsection{1.2 与其他力的比较}
就像其他力的载流子一样,引力在广义相对论中起着决定事件发生的时空的作用。在某些描述中,能量改变了时空本身的“形状”,而重力是这种形状的结果,乍一看似乎很难与粒子之间作用的力相提并论。[6]由于该理论的亚纯不变性不允许将任何特定的时空背景选作“真实”时空背景,因此广义相对论被认为与背景无关。相反,标准模型不是与背景无关的,Minkowski空间作为固定的时空背景享有特殊地位。[7]为了调和这些差异,引入了一个量子引力理论。[8]这个理论是否应该独立于背景是一个悬而未决的问题。这个问题的答案将决定我们对引力在宇宙命运中所起的特定作用的理解。[9]

\subsubsection{纯理论中的引力子}
弦理论预测了引力子的存在及其明确定义的相互作用。扰动弦理论中的引力子是处于非常特殊的低能量振动状态的闭合弦。引力子在弦论中的散射可以根据AdS / CFT的对应关系进行分析,根据保形场论中的相关函数来计算,也可以根据矩阵论来计算。

弦理论中引力子的一个特征是,作为没有端点的闭合弦,它们不会束缚在膜上,可以在它们之间自由移动。如果我们生活在一个膜上(正如膜理论所假设的那样),引力子从膜“泄漏”到高维空间可以解释为什么引力是如此的小,而来自与我们相邻的其他膜的引力子可以为暗物质提供潜在的解释。然而,如果引力子在膜之间完全自由移动,这将过分稀释引力,导致违反牛顿平方反比定律。为了解决这个问题,丽莎·蓝道尔发现了具有自身引力的三膜,阻止引力子的自由漂移,从而导致我们观察到的重力稀释,同时大致保持了牛顿平方反比定律。[10]

艾哈迈德·法拉赫·阿里(Ahmed Farag Ali)和索利亚·达斯(Saurya Das)提出的理论将量子力学校正(使用Bohm轨迹)添加到广义相对论测地线中。如果引力子的质量很小但不为零,则可以解释宇宙常数而无需暗能量,从而解决了体积小的问题。[11]该理论用于解释宇宙常数的微小性,并在2014年Gravity Research Foundation 论文大赛上获得了荣誉奖。[12]此外,该理论还在2015年Gravity Research Foundation 论文竞赛中获得了优秀奖,用于自然解释由于所提出的量子修正而观察到的宇宙的大尺度均匀性和各向同性。[13]

\subsection{能量和波长}
虽然引力子被认为是无质量的,但它们仍然会像其他任何量子粒子一样携带能量。光子能量和胶子能量也由无质量的粒子携带。目前,尚不清楚哪些变量可以确定引力子能量,即单个引力子携带的能量。

或者,如果引力子是大质量的,则对引力波的分析产生了引力子质量的新上限。引力子的康普顿波长至少是$1.6\times,或约1.6 光年,对应于不大于7.7×10−23 eV/c2。[14]波长和质能之间的关系是用普朗克-爱因斯坦关系式计算的,该公式将电磁波长与光子能量联系起来。然而,如果引力子是引力波的量子,那么引力子的波长和相应的粒子能量之间的关系从根本上不同于光子,因为引力子的康普顿波长不等于引力波的波长。相反,下界引力子的康普顿波长大约是GW170104事件的引力波长(约1,700公里)的9×109倍。该报告[14]没有详细说明这一比例的来源。引力子可能不是引力波的量子,或者这两种现象以不同的方式相关联。
\subsubsection{}