% 算符

\pentry{导数\upref{Der}}

\textbf{算符}(也叫\textbf{算子})可以理解为 “函数的函数”, 即一个函数经过算符作用可以得到另一个算符. 例如将一个函数乘以一个数 $\lambda$ 或另一个函数, 又例如对一个函数求或偏导, 又或是依次进行若干个不同的操作.

\begin{example}{}
令算符 $\Q A$ 为
\begin{equation}
\Q A = 2\sin(x)
\end{equation}
那么它作用在函数 $f(x)$ 上就是
\begin{equation}
\Q A f(x) = \sin(x) f(x)
\end{equation}
令
\begin{equation}
\Q B = \dv{x} + 1
\end{equation}
那么
\begin{equation}
\Q A f(x) = \qty(\dv{x} + 1)f(x) = \dv{f(x)}{x} + f(x)
\end{equation}
令
\begin{equation}
\Q C = \Q A \Q B = \sin(x) \qty(\dv{x} + 1)
\end{equation}
那么
\begin{equation}
\Q C f(x) = \Q A [\Q B f(x)] = \sin(x)\dv{f(x)}{x} + \sin(x)f(x)
\end{equation}
令
\begin{equation}
\Q D = {\Q B}^2
\end{equation}
相当于 $\Q B$ 对某函数作用两次, 即
\begin{equation}
\Q D = \qty(\dv{x} + 1)\dv{f(x)}{x} + f(x)
\end{equation}

\end{example}


\begin{equation}
\Q B = \Q A^2 = \qty(\dv{x} + 1)^2 = \pdv[2]{x} + \pdv[2]{y} + 2\pdv{}{x}{y}
\end{equation}
那么
\begin{equation}
\Q B f(x) = \qty(\dv{x} + 1)f(x) = \dv{f(x)}{x} + f(x)
\end{equation}



\begin{equation}

\end{equation}

\begin{equation}
\qty(\pdv{x} + \pdv{y})^2 = 
\end{equation}

% \subsection{线性}
% 若一个算符 $\Q A$ 对两个不同的函数 $f, g$ 和两个不同的常数满足
% \begin{equation}
% \Q A (f + g)
% \end{equation}
% 是\textbf{线性}的
