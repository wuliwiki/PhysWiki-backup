% 初步认识数据库系统
% 数据,数据库,数据库系统

\begin{issues}
\issueTODO
\end{issues}


\subsection{四大基本概念}
\subsubsection{数据–Data:}
数据(Data)是数据库中存储的基本对象\\
\begin{enumerate}
\item 数据的定义\\
描述事物的符号记录\\
\item 数据的种类\\
文本、图形、图像、音频、视频、学生的档案记录等\\
\item 数据的特点\\
数据与其语义是不可分的\\
\end{enumerate}
\subsubsection{数据库–Database}
\begin{enumerate}
\item 数据库的定义\\
数据库(Database,简称DB)是长期储存在计算机内、有组织、可共享的大量数据的集合。\\
\item 数据库的基本特征\\
\begin{itemize}
\item 数据按一定的数据模型组织、描述和储存
\item 可为各种用户共享
\item 冗余度较小
\item 数据独立性较高
\item 易扩展
\end{itemize}
\end{enumerate}

\subsubsection{数据库管理系统–DataBase Management System}
\begin{enumerate}
\item DBMS的概念
DBMS是位于用户与操作系统之间的一层数据管理软件。是基础软件,是一个大型复杂的软件系统

\item DBMS的用途
科学地组织和存储数据、高效地获取和维护数据

\item DBMS的主要功能
\begin{itemize}
\item 数据定义功能\\
提供数据定义语言(DDL)\\
定义数据库中的数据对象\\

\item 数据组织、存储和管理\\
分类组织、存储和管理各种数据\\
确定组织数据的文件结构和存取方式\\
实现数据之间的联系\\
提供多种存取方法提高存取效率\\

\item 数据操纵功能\\
提供数据操纵语言(DML)\\
实现对数据库的基本操作 (查询、插入、删除和修改)\\

\item 数据库的事务管理和运行管理\\
数据库在建立、运行和维护时由DBMS统一管理和控制\\
保证数据的安全性、完整性、多用户对数据的并发使用\\
发生故障后的系统恢复\\

\item 数据库的建立和维护功能(实用程序)\\
数据库初始数据装载转换\\
数据库转储\\
介质故障恢复\\
数据库的重组织\\
性能监视分析等\\

\item 其它功能\\
DBMS与网络中其它软件系统的通信\\
两个DBMS系统的数据转换\\
异构数据库之间的互访和互操作\\
\end{itemize}
\end{enumerate}

\subsubsection{数据库系统–Database System}
\begin{enumerate}
\item 数据库系统(Database System,简称DBS)的概念\\
在计算机系统中引入数据库后的系统构成

\item 数据库系统的构成\\
\begin{itemize}
\item 数据库 Database
\item 数据库管理系统(及其开发工具)Database Management System
\item 应用系统
\item 数据库管理员 Database Administrator
\end{itemize}

\item 数据库系统的特点

\begin{itemize}
\item 数据结构化\\
数据彼此之间产生联系,发生关系\\
数据结构化是数据库系统与文件系统的根本区别\\

\item 数据的共享性高,冗余度低,易扩充\\
数据是面向整体的,可以被多个用户、多个应用程序共享使用\\
减少数据冗余,节约存储空间,避免数据之间的不相容性与不一致性\\

\item 数据独立性高\\
数据独立性包括数据的\textbf{物理独立性}和\textbf{逻辑独立性}\\

物理独立性是指数据在磁盘上的数据库中如何存储是由DBMS管理的,用户程序不需要了解,应用程序要处理的只是数据的逻辑结构。\\

逻辑独立性是指用户的应用程序与数据库的逻辑结构是相互独立的,也就是说,数据的逻辑结构改变了, 用户程序也可以不改变。\\

\item 数据由DBMS统一管理和控制\\

数据库的共享是并发的共享\\
多个用户可以同时存取数据库中的数据\\
DBMS必须提供以下几方面的数据控制功能:\\
\begin{itemize}
\item 数据的安全性保护(security)\\
保护数据,以防止不合法的使用造成的数据的泄密和破坏。\\
\item 数据的完整性检查(integrity)\\
将数据控制在有效的范围内,或保证数据之间满足一定的关系。\\
\item 数据库的并发访问控制(concurrency)\\
对多用户的并发操作加以控制和协调,防止相互干扰而得到错误的结果。\\
\item 数据库的故障恢复(recovery)\\
将数据库从错误状态恢复到某一已知的正确状态。\\
\end{itemize}

\end{itemize}

\end{enumerate}