% 菲涅尔半波带法
% keys half wave band|Fresnel|optics|光学
% license Usr
% type Tutor




菲涅尔半波带法是一种处理连续分布的波源时,简化的计算方法。



\subsection{波的干涉}


设介质或空间中存在一个波源,如果要研究此波源在整个空间中产生的波是怎样随时间变化的,只需要任取空间中的一个点$P$,研究清楚$P$上的波函数如何变化,则由$P$的任意性,相当于已经得到了整个空间中波的分布。

产生波的点,我们称之为\textbf{源点};被选中用于研究的点,我们称之为\textbf{场点}。

当空间中只有一个源点时,场点的波可以理解为源点的波弱化并时间延迟后的结果。设源点为$S$,场点$P$到$S$的距离为$L$,波速为$c$,则在时间$t_0$时$P$点处的波函数,就是时间$t_0-L/c$时$S$点处的波函数乘以一个系数,此系数正比于$1/L^{n/2}$,其中$n$是波的维度\footnote{波的维度为描述波函数所需要的空间维度;这也正是介质的维度。}减一。此系数表达的是能量守恒,波随着传递而减弱\footnote{比如二维波,如水波。波源的振动在时间$t$后传播到整个半径为$ct$的圆上,振动能量均匀分布到这个圆上,因此圆上的能量密度正比于$1/ct$。由于振动所含的能量正比于振幅的平方,因此圆上振幅正比于$1/\qty(ct)^{1/2}$。}。

比如说,若源点的振动为$A\cos \omega t$,那么场点的振动则为$\frac{AC}{L^n/2}\cos \omega(t-L/c)$。

当空间中有多个源点时,任意时刻场点的波即为所有源点在场点单独产生的波函数相加。
























