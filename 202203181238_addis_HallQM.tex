% Hall 量子力学笔记

\begin{issues}
\issueDraft
\end{issues}

\subsection{A.2 Measure Theory}

\begin{itemize}
\item \textbf{度量空间(measure space)} $(X,\Omega,\mu)$. \textbf{可积(integrable)} $\int_X \abs{\psi} \dd{\mu} < \infty$.

\item \textbf{生成(generated)}的 $\sigma$-代数

\item 可测空间 $(X, \mu)$ 的度量 $\mu$ 被称为 \textbf{$\sigma$-有限的($\sigma$-finite)}, 如果 $X$ 可是可数个有限测度集的并.

\item Definition A.5 Suppose $\mu$ and $\nu$ are two $\sigma$-finite measures on a measure space $(X, \Omega)$. Then we say that $\mu$ is \textbf{绝对连续(absolutely continuous)} with respect to $\nu$ if for all $E \in \Omega$, if $\nu(E)=0$ then $\mu(E)=0$. We say that $\mu$ and $\nu$ are \textbf{equivalent} if each measure is absolutely continuous with respect to the other.

\item Theorem A.6 (\textbf{Radon-Nikodym}) Suppose $\mu$ and $\nu$ are two $\sigma$-finite measures on a measure space $(X, \Omega)$ and that $\mu$ is absolutely continuous with respect to $\nu$. Then there exists a non-negative, measurable function $\rho$ on $X$ such that $\mu(E)=\int_{E} \rho d \nu$, for all $E \in \Omega$. The function $\rho$ is called the \textbf{密度(density)} of $\mu$ with respect to $\nu$.

\item A \textbf{countable increasing union} means the union of a sequence $E_j$ of sets
where $E_j$ is contained in $E_{j+1}$ for each $j$,

\item Definition A.7 A collection $\mathcal{M}$ of subsets of a set $X$ is called a \textbf{monotone class} if $\mathcal{M}$ is closed under countable increasing unions and countable decreasing intersections.

\item (私货)Monotone Class: 1. if $A_{1}, A_{2}, \ldots \in M$ and $A_{1} \subseteq A_{2} \subseteq \cdots$ then $\bigcup_{i=1}^{\infty} A_{i} \in M$, and 2. if $B_{1}, B_{2}, \ldots \in M$ and $B_{1} \supseteq B_{2} \supseteq \cdots$ then $\bigcap_{i=1}^{\infty} B_{i} \in M$.

\item Theorem A.8 (\textbf{Monotone Class Lemma}) Suppose $\mathcal{M}$ is a monotone class of subsets of a set $X$ and suppose $\mathcal{M}$ contains an algebra $\mathcal{A}$ of subsets of $X$. Then $\mathcal{M}$ contains the $\sigma$-algebra generated by $\mathcal{A}$.

\item Corollary A.9 Suppose $\mu$ and $\nu$ are two finite measures on a measure space $(X, \Omega)$. Suppose $\mu$ and $\nu$ agree on an algebra $\mathcal{A} \subset \Omega$. Then $\mu$ and $\nu$ agree on the $\sigma$-algebra generated by $\mathcal{A}$.

\item Theorem A.10 Suppose $\mu$ is a measure on the Borel $\sigma$-algebra in a locally compact, separable metric space $X$. Suppose also that $\mu(K)<\infty$ for each compact subset $K$ of $X$. Then the space of continuous functions of compact support on $X$ is dense in $L^{p}(X, \mu)$, for all $p$ with $1 \leq p<\infty$.
\end{itemize}

\subsection{A.3 Elementary Functional Analysis}

\subsubsection{A.3.1 The Stone–Weierstrass Theorem}
\begin{itemize}
\item If $X$ is a compact metric space, let $\mathcal{C}(X ; \mathbb{R})$ and $\mathcal{C}(X ; \mathbb{C})$ denote the space of continuous real- and complex-valued continuous functions, respectively. A subset $\mathcal{A}$ of $\mathcal{C}(X ; \mathbb{F})$ is called an \textbf{algebra} if it is closed under pointwise addition, pointwise multiplication, and multiplication by elements of $\mathbb{F}$, where $\mathbb{F}=\mathbb{R}$ or $\mathbb{C}$. An algebra $\mathcal{A}$ is said to \textbf{separate points} if for any two distinct points $x$ and $y$ in $X$, there exists $f \in \mathcal{A}$ such that $f(x) \neq f(y)$. We use on $\mathcal{C}(X ; \mathbb{F})$ the supremum norm, given by $\|f\|_{\text {sup }}:=\sup _{x \in X}|f(x)|$, and $\mathcal{C}(X, \mathbb{F})$ is complete with respect to the associated distance function, $d(f, g)=\|f-g\|_{\text {sup }}$.

\item Theorem A.11 (\textbf{Stone-Weierstrass}, Real Version) Let $X$ be a compact metric space and let $\mathcal{A}$ be an algebra in $\mathcal{C}(X ; \mathbb{R})$. If $\mathcal{A}$ contains the constant functions and separates points, then $\mathcal{A}$ is dense in $\mathcal{C}(X ; \mathbb{R})$ with respect to the supremum norm.

\item Theorem A.12 (Stone-Weierstrass, Complex Version) Let $X$ be a compact metric space and let $\mathcal{A}$ be an algebra in $\mathcal{C}(X ; \mathbb{C})$. If $\mathcal{A}$ contains the constant functions, separates points, and is closed under complex conjugation, then $\mathcal{A}$ is dense in $\mathcal{C}(X ; \mathbb{C})$ with respect to the supremum norm.
\end{itemize}

\subsubsection{A.3.2 The Fourier Transform}

\begin{itemize}
\item Definition A.13 For any $\psi \in L^{1}\left(\mathbb{R}^{n}\right)$, define the Fourier transform of $\psi$ to be the function $\hat{\psi}$ on $\mathbb{R}^{n}$ given by $\hat{\psi}(\mathbf{k})=(2 \pi)^{-n / 2} \int_{-\infty}^{\infty} e^{-i \mathbf{k} \cdot \mathbf{x}} \psi(\mathbf{x}) d \mathbf{x}$

\item Proposition A.14 For any $\psi \in L^{1}\left(\mathbb{R}^{n}\right)$, the Fourier transform $\hat{\psi}$ of $\psi$ has the following properties: (1) $|\hat{\psi}(\mathbf{k})| \leq(2 \pi)^{-n / 2}\|\psi\|_{L^{1}}$, (2) $\hat{\psi}$ is continuous, and (3) $\hat{\psi}(\mathbf{k})$ tends to zero as $|\mathrm{k}|$ tends to $\infty$.

\item Definition A.15 The \textbf{Schwartz space} $\mathcal{S}\left(\mathbb{R}^{n}\right)$ is the space of all $C^{\infty}$ functions $\psi$ on $\mathbb{R}^{n}$ such that $\lim _{x \rightarrow \pm \infty}\left|\mathbf{x}^{\mathbf{j}} \partial^{\mathbf{k}} \psi(\mathrm{x})\right|=0$ for all $n$-tuples of non-negative integers $\mathbf{j}$ and $\mathbf{k}$. Here if $\mathbf{j}=\left(j_{1}, \ldots, j_{n}\right)$ then $\mathrm{x}^{\mathbf{j}}=x_{1}^{j_{1}} 	\cdots x_{n}^{j_{n}}$ and $\partial^{\mathbf{j}}=\left(\frac{\partial}{\partial x_{1}}\right)^{j_{1}} \cdots\left(\frac{\partial}{\partial x_{n}}\right)^{j_{n}}$. An element of the Schwartz space is called a \textbf{Schwartz function}.

\item Proposition A.16 If $\psi$ belongs to $\mathcal{S}\left(\mathbb{R}^{n}\right)$, then $\hat{\psi}$ also belongs to $\mathcal{S}\left(\mathbb{R}^{n}\right)$.

\item Proposition A.17 If $\psi$ is a Schwartz function, the following properties hold
1. We have $\widehat {\left({\frac{\partial \psi}{\partial x_{j}}}\right)}	 (\mathbf{k})=i k_{j} \hat{\psi}(\mathbf{k})$ 2. The function $\hat{\psi}$ is differentiable at every point and the Fourier transform of the function $x_{j} \psi(x)$ is given by $\widehat{x_{j} \psi}(\mathbf{k})=-i \frac{\partial}{\partial k_{j}} \hat{\psi}(\mathbf{k})$

\item Theorem A.18 The Fourier transform on $\mathcal{S}\left(\mathbb{R}^{n}\right)$ has the following properties.
1. The Fourier transform maps the Schwartz space onto the Schwartz space.
2. For all $\psi \in \mathcal{S}\left(\mathbb{R}^{n}\right)$, the function $\psi$ can be recovered from its Fourier transform by the Fourier inversion formula: $\psi(\mathrm{x})=(2 \pi)^{-n / 2} \int_{-\infty}^{\infty} e^{i \mathbf{k} \cdot \mathbf{x}} \hat{\psi}(k) d \mathbf{k}$
3. For all $\psi \in \mathcal{S}\left(\mathbb{R}^{n}\right)$, we have the \textbf{Plancherel theorem}: $\int_{\mathbb{R}^{n}}|\psi(\mathbf{x})|^{2} d \mathbf{x}=\int_{\mathbb{R}^{n}}|\hat{\psi}(\mathbf{k})|^{2} d \mathbf{k}$

\item (私货) an \textbf{isometry} (or \textbf{congruence}, or \textbf{congruent transformation}) is a distance-preserving transformation between metric spaces

\item Theorem A.19 The Fourier transform extends to an isometric map $\mathcal{F}$ of $L^{2}\left(\mathbb{R}^{n}\right)$ onto $L^{2}\left(\mathbb{R}^{n}\right)$. This map may be computed as $\mathcal{F}(\psi)(\mathbf{k})=(2 \pi)^{-n / 2} \lim _{A \rightarrow \infty} \int_{|\mathbf{x}| \leq A} e^{-i \mathbf{k} \cdot \mathbf{x}} \psi(\mathbf{x}) d \mathbf{x}$, where the limit is in the norm topology of $L^{2}\left(\mathbb{R}^{n}\right)$. The inverse map $\mathcal{F}^{-1}$ may be computed as $\left(\mathcal{F}^{-1} f\right)(\mathrm{x})=(2 \pi)^{-n / 2} \lim _{A \rightarrow \infty} \int_{|\mathbf{x}| \leq A} e^{i \mathbf{k} \cdot \mathbf{x}} f(\mathbf{k}) d \mathbf{k}$

\item (私货) \textbf{Lebesgue's dominated convergence theorem}: Let $\left(f_{n}\right)$ be a sequence of complex-valued measurable functions on a measure space $(S, \Sigma$, $\mu)$. Suppose that the sequence converges pointwise to a function $f$ and is dominated by some integrable function $g$ in the sense that $\left|f_{n}(x)\right| \leq g(x)$ for all numbers $n$ in the index set of the sequence and all points $x \in S$. Then $f$ is integrable (in the Lebesgue sense) and
$\lim _{n \rightarrow \infty} \int_{S}\left|f_{n}-f\right| d \mu=0$ which also implies $\lim _{n \rightarrow \infty} \int_{S} f_{n} d \mu=\int_{S} f d \mu$

\item If $\psi$ belongs to $L^{1}\left(\mathbb{R}^{n}\right) \cap L^{2}\left(\mathbb{R}^{n}\right)$, then by dominated convergence, the above limit coincides with the $L^{1}$ Fourier transform in Definition A.13.

\item Definition A.20 For two measurable functions $\phi$ and $\psi$, define the \textbf{convolution} $\phi * \psi$ of $\phi$ and $\psi$ by the formula $(\phi * \psi)(\mathbf{x})=\int_{\mathbb{R}^{n}} \phi(\mathbf{x}-\mathbf{y}) \psi(\mathbf{y}) d \mathbf{y}$ provided that the integral is absolutely convergent for all $\mathrm{x}$.

\item Proposition A.21 Suppose that $\phi$ and $\psi$ belong to $L^{1}\left(\mathbb{R}^{n}\right) \cap L^{2}\left(\mathbb{R}^{n}\right)$. Then $\phi * \psi$ is defined and belongs to $L^{1}\left(\mathbb{R}^{n}\right) \cap L^{2}\left(\mathbb{R}^{n}\right)$ and we have $(2 \pi)^{-n / 2} \mathcal{F}(\phi * \psi)=\mathcal{F}(\phi) \mathcal{F}(\psi)$. This result is proved by plugging $\phi * \psi$ into the definition of the Fourier transform, writing $e^{-i \mathbf{k} \cdot \mathbf{x}}$ as $e^{-i \mathbf{k} \cdot \mathbf{y}} e^{-i \mathbf{k} \cdot(\mathbf{x}-\mathbf{y})}$, and using Fubini's theorem.

\item Proposition A.22 For all $a>0$ and $b \in \mathbb{C}$, we have $\frac{1}{\sqrt{2 \pi}} \int_{-\infty}^{\infty} e^{-x^{2} /(2 a)} e^{b x} d x=\sqrt{a} e^{a b^{2} / 2}$
\end{itemize}

\subsection{A.3.4 Banach Spaces}
\begin{itemize}
\item Definition A.35 If $V_{1}$ and $V_{2}$ are normed spaces, a linear map $T: V_{1} \rightarrow$ $V_{2}$ is bounded if $\sup _{\psi \in V_{1} \backslash\{0\}} \frac{\|T \psi\|}{\|\psi\|}<\infty$. If $T$ is bounded, then the supremum is called the operator norm of $T$, denoted $\|T\|$.

\item Theorem A.36 (Bounded Linear Transformation Theorem) Let $V_{1}$ be a normed space and $V_{2}$ a Banach space. Suppose $W$ is a dense subspace of $V_{1}$ and $T: W \rightarrow V_{2}$ is a bounded linear map. Then there exists a unique bounded linear map $\tilde{T}: V_{1} \rightarrow V_{2}$ such that $\left.\tilde{T}\right|_{W}=T$. Furthermore, the norm of $\tilde{T}$ equals the norm of $T$.

\item Definition A.37 If $V$ is a normed space over $\mathbb{F}(\mathbb{F}=\mathbb{R}$ or $\mathbb{C})$, then a bounded linear functional on $V$ is a bounded linear map of $V$ into $\mathbb{F}$, where on $\mathbb{F}$ we use the norm given by the absolute value. The collection of all bounded linear functionals, with the norm given by $\sup _{\psi \in V_{1} \backslash\{0\}} \frac{\|T \psi\|}{\|\psi\|}$, is called the dual space to $V$, denoted $V^{*}$.

\item Theorem A.38 If $V$ is a normed vector space, then the following results hold.
1. The dual space $V^{*}$ is a Banach space.
2. For all $\psi \in V$, there exists a nonzero $\xi \in V^{*}$ such that $|\xi(\psi)|=\|\xi\|\|\psi\|$. In particular, if $\xi(\psi)=0$ for all $\xi \in V^{*}$, then $\psi=0$.

\item Theorem A.39 (Closed Graph Theorem) Suppose that $V_{1}$ is a Banach space and $V_{2}$ a normed vector space. For any linear map $T: V_{1} \rightarrow V_{2}$, let $\operatorname{Graph}(T)$ denote the set of pairs $(\psi, T \psi)$ in $V_{1} \times V_{2}$ such that $\psi \in V_{1}$. If the graph of $T$ is a closed subset of $V_{1} \times V_{2}$, then $T$ is bounded.
\end{itemize}

\subsection{A.4 Hilbert Spaces and Operators on Them}

\begin{itemize}
\item Definition A.45 Suppose $\mathbf{H}_{j}$ is a sequence of separable Hilbert spaces. Then the Hilbert space \textbf{direct sum}, denoted $\mathbf{H}:=\bigoplus_{j=1}^{\infty} \mathbf{H}_{j}$ is the space of sequences $\psi=\left(\psi_{1}, \psi_{2}, \psi_{3}, \ldots\right)$ such that $\psi_{n} \in \mathbf{H}_{n}$ and such that $\|\psi\|^{2}:=\sum_{j=1}^{\infty}\left\|\psi_{j}\right\|_{j}^{2}<\infty$. The \textbf{finite direct sum} of the $\mathbf{H}_{j}$ 's is the set of $\psi=\left(\psi_{1}, \psi_{2}, \psi_{3}, \ldots\right)$ such that $\psi_{j}=0$ for all but finitely many values of $j$.

\item Proposition A.49 1. If $V$ is a closed subspace of $\mathbf{H}$, every $\psi \in \mathbf{H}$ can be decomposed uniquely as $\psi=\psi_{1}+\psi_{2}$, with $\psi_{1} \in V$ and $\psi_{2} \in V^{\perp}$. 2. If $V$ is any subspace of $\mathbf{H}$, then $\left(V^{\perp}\right)^{\perp}=\bar{V}$, where $\bar{V}$ is the closure of $V$. In particular, if $V$ is closed, then $\left(V^{\perp}\right)^{\perp}=V$.

\item Definition A.50 A set $\left\{e_{j}\right\}$ of elements of $\mathbf{H}$, where $j$ ranges over an arbitrary index set, is said to be \textbf{orthonormal} if $\left\langle e_{j}, e_{k}\right\rangle=\left\{\begin{array}{cc}
0 & j \neq k \\
1 & j=k
\end{array} .\right.$ An orthonormal set $\left\{e_{j}\right\}$ is an \textbf{orthonormal basis} for $\mathbf{H}$ if the space of finite linear combinations of the $e_{j}$ 's is dense in $\mathbf{H}$.

\item Proposition A.51 Suppose $\left\{e_{j}\right\}$ is an orthonormal basis for H. Then every $\psi$ can be expressed uniquely as a convergent sum $\psi=\sum_{j} a_{j} e_{j}$ where the coefficients are given by $a_{j}=\left\langle e_{j}, \psi\right\rangle$. If $\psi$ is as in above, then $\|\psi\|^{2}=\sum_{j}\left|a_{j}\right|^{2}$. Finally, if $\left\langle a_{j}\right\rangle$ is any sequence such that $\sum_{j}\left|a_{j}\right|^{2}<\infty$, there exists a unique $\psi \in \mathbf{H}$ such that $\left\langle e_{j}, \psi\right\rangle=a_{j}$ for all $j$.
\end{itemize}

\subsubsection{A.4.3 The Riesz Theorem and Adjoints}

\begin{itemize}
\item We let $\mathcal{B}(\mathbf{H})$ denote the space of bounded linear maps of $\mathbf{H}$ to $\mathbf{H}$. It is not hard to show that $\mathcal{B}(\mathbf{H})$ forms a Banach space under the operator norm.

\item Theorem A.52 (\textbf{Riesz Theorem}) If $\xi: \mathrm{H} \rightarrow \mathbb{C}$ is a bounded linear functional, then there exists a unique $\chi \in \mathbf{H}$ such that $\xi(\psi)=\langle\chi, \psi\rangle$ for all $\psi \in \mathbf{H}$. Furthermore, the operator norm of $\xi$ as a linear functional is equal to the norm of $\chi$ as an element of $\mathbf{H}$.

\item Proposition A.53 For any $A \in \mathcal{B}(\mathbf{H})$, there exists a unique linear operator $A^{*}: \mathbf{H} \rightarrow \mathbf{H}$, called the \textbf{adjoint} of $A$, such that $\langle\phi, A \psi\rangle=\left\langle A^{*} \phi, \psi\right\rangle$ for all $\phi, \psi \in \mathbf{H}$. For all $A, B \in \mathcal{B}(\mathbf{H})$ and $\alpha, \beta \in \mathbb{C}$ we have $\left(A^{*}\right)^{*} =A$, $(A B)^{*} =B^{*} A^{*}$, $(\alpha A+\beta B)^{*} =\bar{\alpha} A^{*}+\bar{\beta} B^{*}$, $I^{*} =I$. The operator $A^{*}$ is bounded and $\left\|A^{*}\right\|=\|A\|$.

\item Definition A.54 An operator $A \in \mathcal{B}(\mathbf{H})$ is said to be self-adjoint if $A^{*}=A$ and skew-self-adjoint if $A^{*}=-A$.

\item Definition A.55 An operator $U$ on $\mathbf{H}$ is \textbf{unitary} if $U$ is surjective and preserves inner products, that is, $\langle U \phi, U \psi\rangle=\langle\phi, \psi\rangle$ for all $\phi, \psi \in \mathbf{H}$.
If $U$ is unitary, then $U$ preserves norms $(\|U \psi\|=\|\psi\|$ for all $\psi \in \mathbf{H})$; therefore, $U$ is bounded with $\|U\|=1$. By the polarization identity (Proposition A.59), if $U$ preserves norms, then it also preserves inner products.

\item Proposition A.56 A bounded operator $U$ is unitary if and only if $U^{*}=$ $U^{-1}$, that is, if and only if $U U^{*}=U^{*} U=I$.

\item Proposition A.57 For any closed subspace $V \subset \mathbf{H}$, there is a unique bounded operator $P$ such that $P=I$ on $V$ and $P=0$ on the orthogonal complement $V^{\perp}$. This operator is called the \textbf{orthogonal projection} onto $V$ and it satisfies $P^{2}=P$ and $P^{*}=P$. Conversely, if $P$ is any bounded operator on $\mathrm{H}$ satisfying $P^{2}=P$ and $P^{*}=P$, then $P$ is the orthogonal projection onto a closed subspace $V$, where $V=\operatorname{range}(P)$.
\end{itemize}

\subsubsection{A.4.4 Quadratic Forms}

\begin{itemize}
\item Definition A.58 A \textbf{半双线性形式(sesquilinear form)} on $\mathbf{H}$ is a map $L: \mathbf{H} \times \mathbf{H} \rightarrow \mathbb{C}$ that is conjugate linear in the first factor and linear in the second factor. A sesquilinear form is bounded if there exists a constant $C$ such that $|L(\phi, \psi)| \leq C\|\phi\|\|\psi\|$ for all $\phi, \psi \in \mathbf{H}$.

\item Proposition A.59 If $L$ is a sesquilinear form on $\mathbf{H}, L$ can be recovered from its values on the diagonal (i.e., the value of $L(\psi, \psi)$ for various $\psi$ 's) as follows: $L(\phi, \psi) =\frac{1}{2}[L(\phi+\psi, \phi+\psi)-L(\phi, \phi)-L(\psi, \psi)]
-$ $\frac{i}{2}[L(\phi+i \psi, \phi+i \psi)-L(\phi, \phi)-L(i \psi, i \psi)]$. This formula is known as the \textbf{polarization identity}.

\item Definition A.60 A quadratic form on a Hilbert space $\mathbf{H}$ is a map $Q$ : $\mathbf{H} \rightarrow \mathbb{C}$ with the following properties: (1) $Q(\lambda \psi)=|\lambda|^{2} Q(\psi)$ for all $\psi \in \mathbf{H}$ and $\lambda \in \mathbb{C}$, and (2) the map $L: \mathbf{H} \times \mathbf{H} \rightarrow \mathbb{C}$ defined by $L(\phi, \psi) =\frac{1}{2}[Q(\phi+\psi)-Q(\phi)-Q(\psi)]$ $-\frac{i}{2}[Q(\phi+i \psi)-Q(\phi)-Q(i \psi)]$ is a sesquilinear form. A quadratic form $Q$ is bounded if there exists a constant $C$ such that $|Q(\phi)| \leq C\|\phi\|^{2}$ for all $\phi \in \mathbf{H}$. The smallest such constant $C$ is the norm of $Q$.

\item Proposition A.61 If $Q$ is a quadratic form on $\mathbf{H}$ and $L$ is the associated sesquilinear form, we have the following results.
1. For all $\psi \in \mathbf{H}$, we have $Q(\psi)=L(\psi, \psi)$.
2. If $Q$ is a bounded, then $L$ is bounded.
3. If $Q(\psi)$ belongs to $\mathbb{R}$ for all $\psi \in \mathbf{H}$, then $L$ is conjugate symmetric, that is, $L(\phi, \psi)=\overline{L(\psi, \phi)}$ for all $\phi, \psi \in \mathbf{H}$.

\item Example A.62 If $A$ is a bounded operator on $\mathbf{H}$, one can construct a bounded quadratic form $Q_{A}$ on $\mathbf{H}$ by setting $Q_{A}(\psi)=\langle\psi, A \psi\rangle, \quad \psi \in \mathbf{H}$. The associated sesquilinear form $L_{A}$ is then given by $L_{A}(\phi, \psi)=\langle\phi, A \psi\rangle, \quad \phi, \psi \in \mathbf{H}$.

\item 
\end{itemize}

