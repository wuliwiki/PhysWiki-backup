% 量子力学常见问题(科普)
% license Usr
% type Tutor

\begin{issues}
\issueDraft
\end{issues}

这里的量子力学指的是非量子场论的,也就是只考虑一次量子化。
\addTODO{考虑把以下一些内容放到量子力学的基本原理(科普)\upref{QM0} 中}

\begin{itemize}
\item 问:量子力学是什么?
\item 答:和所有物理理论一样,量子力学是物理学家对自然规律的一种系统猜想,它本质上是一个数学模型,其中部分概念被赋予了可观测的含义。具体而言,量子力学本质上是\textbf{线性代数},物理系统的状态被解释为一个向量(称为\textbf{态矢量}),一个系统所有可能的状态向量构成该系统的态空间,对该系统的观测行为被解释为一个线性变换,物理上常称为线性算子。状态向量可以表示为波函数,波函数相当于状态向量的分量。
\item 问:还有不可观测的概念吗?
\item 答:波函数,或者说状态向量本身,是不可观测的,但是状态向量的模长代表概率密度,因此是可观测的。可观测的量,一定能以某种方式,通过实验得到一个实数,比如经典力学里物体的位置、动量、能量等。概率密度的实验方式,就是多次测量处于某一状态的系统,得到某个实验结果出现的频率。
\item  问:量子是什么?
\item 答:“量子力学” 中的 “量子” 可以理解为 “量子现象”。 量子现象不是某个具体的现象,更不是一个实物粒子。 而是对在经典力学中连续的,但在量子力学中存在\textbf{离散}数值的\textbf{一类现象的统称}。什么是离散? “离散” 是和 “连续” 相对的, 例如我们认为数轴上的实数是可以连续取值的,但数轴上的 1,2,3 这些整数就是一个个离散的值。 例如经典力学认为,电子绕原子核运动的能量可以是连续变化的,也就是取某个区间的任意实数。 但量子力学和实验都告诉我们,它只能取某些特定的,\textbf{离散的}值,这就是一种量子现象。 又例如经典力学认为单一频率的光(也就是电磁波)给电子等粒子传递的能量可以取连续的值,但量子力学和实验都表明它只能传输某个最小能量的整数倍,这也是一种量子现象。 在后者中,这束光中不可分割的最小整体也被称为 “光量子”,但一般直接叫做光子,光子是基本粒子的一种。
\item 问:所有光子的能量都是某个能量的整数倍吗?
\item 答:不是的,一束光中只能包含整数个光子,这是一种量子现象。 但每个光子的能量只和它对应的频率有关,且\textbf{频率可以连续取值},不存在量子化。若一束光只有单一的频率 $\nu$, 那么它只能由能量为 $E = h\nu$ 的整数个光子组成,其中 $h$ 是普朗克常数。$\nu$ 不一定取整数,不可以认为所有光子的能量都是普朗克常数的整数倍。 如果一束光具有一定的频率分布(例如阳光),那么它由不同频率和能量的光子组成。
\item 问:经典物理、量子物理、量子力学、量子场论是什么关系?
\item 答:经典物理是以牛顿三定律、牛顿万有引力定律、麦克斯韦方程组等为基本假设的,计算物质运动和相互作用的理论。一般认为相对论也属于经典力学的一部分。从数学模型上说,经典力学是欧几里得几何学,也可以说是辛几何学;电动力学、相对论是微分几何;量子力学则是线性代数。 经典力学中一切都是确定的,也就是说如果我们知道系统在某个时刻的状态,那么根据经典力学就可以唯一地求解出接下来任意时刻的状态。经典力学中大部分物理量都是取连续值的。经典力学理论在计算小尺度的微观物体(例如原子)时几乎彻底失效,于是量子理论被用于描述微观物体的运动和相互作用规律。\textbf{量子场论}则是涉及到了相对论性量子力学与经典场论的结合,即狭义相对论、量子力学和经典场论三者结合的理论,他用于描述多粒子系统,为描述粒子的产生、湮灭等的过程提供了有效的理论框架支持,为粒子物理标准模型提供了数学基础和理论框架。
\addTODO{未完成}
\item 问:为什么说波尔原子模型是半经典理论?
\item 答:
\item 问:量子力学告诉我们很多事情都是随机的,那量子力学和经典力学冲突么?
\item 答:在不考虑极端情况的前提下(例如类似黑洞的极端天体),在宏观上是不冲突的。任意一个物理量都可以由动量与位置得出,这两个量的期待值符合经典力学。
\item 问:粒子是什么?和量子有什么关系?为什么量子力学名词里面那么多 “子” 字,都是一个意思吗?
\item 答:无论什么语言中,术语的含义往往不能从字面上理解和对比。 \textbf{粒子}的英文是 \textbf{particle},\textbf{量子}的英文是 \textbf{quantum},这两个单词并没有共同的词根在里面, 只是中文翻译中都包含 “子” 而已,二者表示完全不同的意思。上面已经解释了 “量子” 的含义, 我们再来看看 “粒子” 的含义。经典力学中的粒子一般是指一个有质量的,体积很小(相对于考察的尺度来说)的物体,例如一粒沙子。如果我们在计算时忽略它的体积和转动,就是通常所说的\textbf{质点},但它是可能存在内部结构,可以继续分割的。而量子力学讨论的 “粒子” 一般指组成物质的不可分割的\textbf{基本粒子},具有不同的种类。例如组成原子的电子、质子、中子都属于基本粒子。 同样地,这里面一些共同的 “子” 也(部分)是由于翻译造成的。 英文中\textbf{电子(electron)},\textbf{质子(proton)},\textbf{中子(neutron)},\textbf{光子(photon)}也同样存在共同的词根 “on”, 但不是所有基本粒子都这样,例如中微子(nutrino)就没有。
\end{itemize}


============= 下面还的还需要整理 ===========

\begin{itemize}
\item 问:光子是什么?和电子有什么本质不同?
\item 答:

\item 问:波函数就是粒子本身吗?
\item 答:波函数完整描述了微观粒子的所有状态(除了自旋),但并不是粒子本身。就像经典力学中质点的位置和动量决定了它在某个时刻的一切状态,但不能说位置和速度是粒子本身。 不能认为波函数的大小就是粒子大小,更不能认为波函数的概率密度就是粒子的质量密度或者电荷密度。带电粒子之间的相互作用是哈密顿算符中的一项而不是根据概率波的分布而决定的。

\item 问:波函数的概率诠释是不是因为我们的测量技术不够?

\item 答:现有理论也没有说除了波函数以外还需要什么额外的信息,事实上 EPR 告诉我们不存在额外的信息,也就是所谓的隐变量。

\item 问:电子是否在电子云中的不同位置随机跳动?

\item 答:波函数包含了粒子状态的所有信息,在测量电子的位置以前,电子没有具体的位置,更不会在不同的位置之间跳动出现。事实上我们不知道也不关心电子的形状,不可以把它想象成一个小球或者一个质点。

\item 问:原子中电子的速度会超过光速吗?

\item 这种误解可能来源于上一个问题中 “电子在不同位置随机跳动” 产生的 “瞬间移动”。 虽然使用波函数描述的电子没有运动轨迹的概念,但波函数包含了一切可观测量的信息。 也就是它不仅包含位置的概率分布,还可以告诉我们如果测量电子的速度会发生什么。当我们测量电子速度时,同样不会得到一个确切的值,而是一个概率分布,分布曲线也通常是一个 “包” 的形状。 在通常的氢原子中,电子的平均速度约等于 $2\e{6}\Si{m/s}$, 只有光速的 1\% 不到。 但如果把氢原子核换成质子更多的原子核如金元素的原子核(序号 79),那电子的平均速度会提高数十倍,需要在量子力学中考虑相对论效应,但速度仍然不会超过光速。

\item 波粒二象性指什么? 为什么有的时候表现出波动性有的时候表现出粒子性?

\item 出现半个粒子的情况。波函数并不只包含概率信息,而是包含粒子的一切信息,包括动量能量分布和一切可观测物理量的分布。

毕竟他只是几个数字,数字当然不是物质。波函数也是一个道理,函数也同样不是物质本身。

至于除了波函数给出的信息之外,粒子“本身”到底是无大小的点还是小球还是三角形,这目前并没有任何实验可以证明或者证伪,所以讨论它没有意义。

但注意经典力学中描述的基本粒子也是不可分割的质点,所以在量子场论出现以前粒子的不连续性还算不上是量子力学特有的 “量子现象”。量子力学(非场论)并没有明确把二者联系起来。在量子场论中基本粒子是可以通过量子场激发出来的,所以粒子在产生和湮灭中的不可分割性本身也可以看作是一种量子现象。

\addTODO{以下可以考虑专门写一个原子的科普}
\item 问:原子中电子处于高能级不稳定,从而跃迁至低能级,并以光子的形式放出能量。但是,进一步地,
• 原子处于高能级时,为什么会不稳定?“高能级”必然会导致“不稳定”吗?二者有真正的因果关系吗?
• 为什么原子、分子的跃迁,能量通常以光子的形式,而不是其他的形式放出?
• 为什么跃迁会导致系统光子数目的增加,其辐射光子的内在机制又是什么?
——这些问题,高中教科书中并没有提到。这也是我在高中阶段对原子物理最大的困惑。
\item 答:1.对一个不受外界干扰的真空中的孤立的高能级原子来说,自发辐射是导致不稳定的唯一因素,你可以理解为真空并不空而是存在一些“零点扰动”,这扰动导致了自发辐射。这个扰动是量子力学(非场论)所解释不了的。

2.若一个高能级原子受到外部电磁场的作用,它可能存在受激辐射或者激发到更高的能态,这是量子力学(非场论)可以解释的,但在能级改变后,为什么释放或吸收光子,是量子力学(非场论)解释不了的。

3.量子力学(非场论)甚至没有光子的概念,但是有质量的粒子在许多情况下倾向于(而不是严格规定)吸收或释放整数倍的 hv(v 是经典电磁场的频率),这已经使得光子的概念呼之欲出了。

4.为什么是光子:经典物理中光本质是电磁波,而运动的电荷会发出电磁波。量子场论也认为同样的现象是存在的,只不过具体过程和经典物理中大不相同。笔者对量子场论了解不多,但基本粒子要释放能量无非就是释放若干个费米子或者玻色子,不存在宏观系统中放热之类的能量释放。而原子能级跃迁释放的那点能量根本不可能激发出一个有质量的粒子(根据 mc^2 这需要大得多的能量释放),而公认的无质量粒子就只有光子和胶子,后者主要伴随强相互作用出现,能级跃迁只涉及电磁相互作用,所以释放光子就自然而然了。

补充/更正:基本粒子要释放能量不一定要产生新的基本粒子,也可以把能量以动能或者势能的形式传给已有基本粒子。所以原子发生能级跃迁的第三种原因就是和其他原子或者已经存在的其他基本粒子发生碰撞。例如在原子气体中,这种碰撞是大量随机发生的,这既可能导致高能态的原子跃迁到低能态,也可能反过来。所以在一定温度的气体中这会达到一种平衡,使得处于每个能级的原子数量最终几乎保持不变。又例如实验上除了用电磁波轰击原子外,也可以用电子或质子等有质量的粒子轰击原子,用于探索原子(或分子)的各种性质。


\end{itemize}
