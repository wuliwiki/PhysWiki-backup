% 计算复杂性理论(综述)
% license CCBYSA3
% type Wiki

本文根据 CC-BY-SA 协议转载翻译自维基百科\href{https://en.wikipedia.org/wiki/Computational_complexity_theory}{相关文章}。

在理论计算机科学和数学中,计算复杂性理论专注于根据资源使用情况对计算问题进行分类,并探索这些分类之间的关系。计算问题是由计算机解决的任务。一个计算问题可以通过机械地应用数学步骤(如算法)来解决。

如果一个问题的解决需要大量资源,无论使用何种算法,都被视为固有的困难问题。该理论通过引入计算模型来正式化这种直觉,以研究这些问题并量化它们的计算复杂性,即解决问题所需的资源量,如时间和存储空间。还使用其他复杂性度量,如通信量(用于通信复杂性)、电路中的门数(用于电路复杂性)以及处理器数量(用于并行计算)。计算复杂性理论的一个重要作用是确定计算机能够做什么以及不能做什么的实际限制。P与NP问题,作为七大千年奖问题之一,是计算复杂性领域的一部分。

在理论计算机科学中,与计算复杂性紧密相关的领域有算法分析和可计算性理论。算法分析与计算复杂性理论之间的一个关键区别是,前者致力于分析特定算法解决问题所需的资源量,而后者则提出一个更为一般的问题,即所有可能用于解决同一问题的算法。更精确地说,计算复杂性理论试图对能够或不能在适当限制的资源下解决的问题进行分类。反过来,施加对可用资源的限制是计算复杂性与可计算性理论的区别所在:后者理论探讨的是哪些类型的问题原则上可以通过算法解决。
\subsection{计算问题}
\subsubsection{问题实例}  
\begin{figure}[ht]
\centering
\includegraphics[width=8cm]{./figures/7bac2fedcf74bc30.png}
\caption{穿越14个德国城市的旅行推销员之旅} \label{fig_JSFZ_1}
\end{figure}
一个计算问题可以视为一个无限的实例集合,每个实例都有一组(可能为空)的解。计算问题的输入字符串称为问题实例,不应与问题本身混淆。在计算复杂性理论中,问题指的是待解决的抽象问题。与此相对,问题的一个实例是一个相对具体的表述,可以作为决策问题的输入。例如,考虑素数测试问题。实例是一个数字(例如,15),如果该数字是素数,解答是“是”,否则是“否”(在这种情况下,15不是素数,答案是“否”)。换句话说,实例是问题的特定输入,解答是与该输入对应的输出。

为了进一步突出问题和实例之间的区别,考虑旅行商问题的决策版本实例:是否存在一条最多2000公里的路线,经过德国的15个最大城市?对于这个特定问题实例的定量答案,对解决问题的其他实例帮助不大,例如询问一条在米兰所有景点之间,且总长度不超过10公里的环路。因此,复杂性理论关注的是计算问题,而不是特定的问题实例。
\subsubsection{表示问题实例}  
在考虑计算问题时,问题实例通常是一个由字母表组成的字符串。通常,字母表被视为二进制字母表(即{0, 1}集合),因此这些字符串是位字符串。如同实际计算机一样,必须对除位字符串外的数学对象进行适当的编码。例如,整数可以用二进制表示,图形可以通过其邻接矩阵直接编码,或者通过将其邻接表编码为二进制来表示。

尽管一些复杂性理论定理的证明通常假设某种具体的输入编码选择,但讨论通常保持足够抽象,以独立于编码选择。这可以通过确保不同的表示方法可以高效地相互转换来实现。
\subsubsection{决策问题作为形式语言}
决策问题是计算复杂性理论中研究的核心对象之一。决策问题是一种计算问题,其答案是“是”或“否”(也可以是1或0)。决策问题可以被视为一种形式语言,其中语言的成员是那些输出为“是”的实例,而非成员则是那些输出为“否”的实例。目标是借助算法来判断给定的输入字符串是否是所考虑的形式语言的成员。如果决定该问题的算法返回“是”作为答案,则称该算法接受该输入字符串,否则称其拒绝该输入。

一个决策问题的例子是以下问题:输入是一个任意图。问题的内容是判断给定的图是否是连通的。与此决策问题相关的形式语言是所有连通图的集合——为了精确定义这个语言,需要决定如何将图编码为二进制字符串。
\subsubsection{函数问题}
函数问题是一种计算问题,其中每个输入期望一个单一的输出(来自一个总函数),但输出比决策问题更复杂——即输出不仅仅是“是”或“否”。著名的例子包括旅行推销员问题和整数分解问题。

人们可能会认为,函数问题的概念比决策问题的概念要丰富得多。然而,事实并非如此,因为函数问题可以转化为决策问题。例如,两个整数的乘法可以表示为一组三元组 \((a, b, c)\),使得 \(a \times b = c\) 成立。判断给定三元组是否属于该集合,相当于解决两个数相乘的问题。
\subsubsection{衡量实例的大小}
为了衡量解决一个计算问题的难度,可能需要了解解决该问题的最佳算法所需的时间。然而,运行时间通常依赖于实例的大小。特别是,较大的实例需要更多时间来解决。因此,解决一个问题所需的时间(或所需的空间,或任何复杂度度量)是作为实例大小的函数来计算的。输入大小通常以比特为单位衡量。复杂度理论研究算法在输入大小增加时的扩展性。例如,在判断一个图是否连通的问题中,对于一个拥有 \(2n\) 个顶点的图,相比于一个拥有 \(n\) 个顶点的图,解决该问题需要更多的时间吗?

如果输入大小是 \(n\),则所需的时间可以表示为 \(n\) 的函数。由于对于相同大小的不同输入,所需的时间可能不同,因此定义最坏情况下的时间复杂度 \(T(n)\) 为所有大小为 \(n\) 的输入所需时间的最大值。如果 \(T(n)\) 是 \(n\) 的多项式,则称该算法为多项式时间算法。科布汉的论题(Cobham's thesis)认为,如果一个问题能够通过多项式时间算法解决,那么它可以用合理的资源量来解决。