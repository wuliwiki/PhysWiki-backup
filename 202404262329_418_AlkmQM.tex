% 碱金属原子(量子力学)
% keys 碱金属|量子力学
% license Usr
% type Tutor

\pentry{薛定谔方程(单粒子多维)\nref{nod_QMndim},球谐函数 \nref{nod_SphHar}}{nod_060d}

碱金属原子本质是由价电子(最外层电子)来决定性质的。价电子所受到原子实(原子核、内层电子)作用的势函数描述为
\begin{equation}
V(r) = -e^2/r - \lambda a e^2/r^2, \ (0 < \lambda \le 1/8) ~.
\end{equation}
是一个中心势场。则薛定谔方程是可分离变量的,使得波函数可以表示为球谐函数与径向解的乘积的形式——$\psi = Y_{l, m}(\theta, \phi) R(r)$。