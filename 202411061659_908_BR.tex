% 让·勒朗·达朗贝尔(综述)
% license CCBYSA3
% type Wiki

本文根据 CC-BY-SA 协议转载翻译自维基百科\href{https://en.wikipedia.org/wiki/Jean_le_Rond_d\%27Alembert}{相关文章}。

“‘达朗贝尔’重定向至此。其他用法,参见达朗贝尔 (消歧义)。  
不要与德朗布尔混淆。”
\begin{figure}[ht]
\centering
\includegraphics[width=6cm]{./figures/9e2e02aa6a250ffa.png}
\caption{达朗贝尔的粉彩肖像,由莫里斯·昆汀·德·拉图尔创作,1753年。} \label{fig_BR_1}
\end{figure}
让-巴蒂斯特·勒朗·达朗贝尔[a](/ˌdæləmˈbɛər/ DAL-əm-BAIR;[1] 法语:[ʒɑ̃ batist lə ʁɔ̃ dalɑ̃bɛʁ];1717年11月16日-1783年10月29日)是法国数学家、力学家、物理学家、哲学家和音乐理论家。在1759年之前,他与丹尼斯·狄德罗共同担任《百科全书》的编辑。[2] 用于求解波动方程的达朗贝尔公式以他的名字命名。[3][4][5] 波动方程有时也被称为达朗贝尔方程,代数学基本定理在法语中以达朗贝尔命名。
\subsection{早年生活}
达朗贝尔出生于巴黎,是作家克劳丁·盖林·德·坦森和骑士路易-卡缪·德图什(当时任炮兵军官)的私生子。在他出生时,德图什正在国外。出生几天后,他的母亲将他遗弃在巴黎圣让勒朗教堂的台阶上。根据习俗,他以该教堂的守护圣人命名。达朗贝尔被送到一个孤儿院,但他的父亲找到他并将他安置在一位玻璃工的妻子——鲁索夫人家中,达朗贝尔在这里生活了将近50年。[6] 鲁索夫人对他鼓励甚少。当他向她讲述自己的一些发现或所写的东西时,她通常会回复道:

“你永远只会成为一个哲学家——那是什么?不过是一个一生折腾自己、死后才被人谈论的傻瓜罢了。”[7]

德图什暗中资助了让·勒朗的教育,但并不希望他的亲子关系被正式承认。
\subsection{学习与成年生活}

达朗贝尔最初就读于一所私立学校。骑士德图什在1726年去世时,给达朗贝尔留下了每年1200里弗的年金。在德图什家族的影响下,12岁的达朗贝尔进入了詹森派的四国学院(也称为“马扎兰学院”)。在这里,他学习了哲学、法律和艺术,并于1735年获得了艺术学士学位。

在后来的生活中,达朗贝尔轻视他在詹森派那里学到的笛卡尔主义原则,包括“物理推动、先天观念和漩涡论”。詹森派引导达朗贝尔走向教会职业,试图阻止他从事诗歌和数学等方面的兴趣。然而,神学对达朗贝尔来说是“相当空洞的食粮”。他进入了法学院学习了两年,并于1738年被提名为律师。

他对医学和数学也有兴趣。让最初登记的名字是让-巴蒂斯特·达朗贝格(Jean-Baptiste Daremberg),后来他可能出于音韵原因改名为达朗贝尔(d’Alembert)。[8]

后来,为了表彰达朗贝尔的成就,普鲁士的腓特烈大帝曾建议用“达朗贝尔”来命名一颗疑似存在(但实际上不存在)的金星卫星,然而达朗贝尔拒绝了这一荣誉。[9]
\subsection{职业生涯}
\begin{figure}[ht]
\centering
\includegraphics[width=6cm]{./figures/e93464b8ea9b0e1e.png}
\caption{关于流体阻力的新实验} \label{fig_BR_2}
\end{figure}
\textbf{1739年7月,他首次在数学领域做出贡献},在给科学院的一封信中指出了他在《分析学证明》(1708年由夏尔-勒内·雷诺出版)中发现的错误。当时,《分析学证明》是一本标准的数学入门书,达朗贝尔本人曾用它来学习数学的基础。达朗贝尔也精通拉丁文,在他晚年期间致力于塔西佗作品的翻译,得到了包括狄德罗在内的广泛赞誉。
\textbf{1740年},他提交了他的第二篇科学论文,来自流体力学领域的《固体折射的论文》,得到了克莱罗的认可。在这篇论文中,达朗贝尔从理论上解释了折射现象。\\
\textbf{1741年},经过几次失败的尝试后,达朗贝尔被选入科学院。随后他于1746年被选为柏林科学院院士,[10] 并在1748年被选为皇家学会会员。[11]\\
\textbf{1743年},他出版了最著名的著作《动力学论》,在其中发展了他自己的运动定律。[12]\\
\textbf{在1740年代末期《百科全书》组织编纂时},达朗贝尔被聘为与狄德罗共同担任编辑(负责数学和科学部分),一直工作到1757年因一系列危机导致出版暂时中断为止。他为《百科全书》撰写了超过一千篇文章,包括著名的《前言》。达朗贝尔在“怀疑我们认为所见之物在我们之外是否确实存在”时,“放弃了唯物主义的基础”。[13] 通过这种方式,达朗贝尔与唯心主义者贝克莱的观点一致,并预示了康德的先验唯心主义。[需要引用]\\
\textbf{1752年},他描述了如今被称为“达朗贝尔悖论”的现象,即在无粘、不可压缩流体中浸入的物体所受的阻力为零。\\
\textbf{1754年},达朗贝尔被选为法国科学院的成员,并于1772年4月9日成为该院的永久秘书。[14]\\
\textbf{1757年},达朗贝尔在《百科全书》第七卷中的一篇文章中指出,日内瓦的神职人员已从加尔文主义转向纯粹的苏西尼主义,依据是伏尔泰提供的信息。日内瓦的牧师们对此极为愤慨,成立了一个委员会来回应这些指控。在雅各布·韦尔内、让-雅克·卢梭等人的压力下,达朗贝尔最终解释说,他认为不接受罗马教会的人都可以被称为苏西尼主义者,这就是他想表达的全部意思。在回应批评后,他不再参与《百科全书》的编辑工作。[15]\\
\textbf{1781年},他被选为美国艺术与科学学院的外国荣誉会员。[16]
\subsection{音乐理论}
达朗贝尔首次接触音乐理论是在1749年,当时他被要求审查让-菲利普·拉莫提交给科学院的一份《备忘录》。这篇文章是他与狄德罗合作撰写的,后来成为拉莫1750年《和声原理论证》的基础。达朗贝尔在评论中高度评价了拉莫的演绎逻辑,将其视为理想的科学模型。他在拉莫的音乐理论中看到了对自己科学思想的支持:一种具有强演绎综合结构的完整系统方法。

两年后的1752年,达朗贝尔在《根据拉莫先生的原理的理论与实践音乐要素》一书中试图对拉莫的作品进行全面评述。[17] 他强调了拉莫的主要观点,即音乐是一门数学科学,能够从一个单一原理推导出所有音乐实践的要素和规则,以及其中所使用的明确的笛卡尔方法。达朗贝尔帮助推广了这位作曲家的作品,同时宣传了自己的理论。[17] 他声称自己“澄清、发展和简化”了拉莫的原理,并认为单一的“共鸣体”(corps sonore)的概念不足以涵盖音乐的全部。[18] 达朗贝尔主张需要三个原则来生成大调、调式和八度音的识别。然而,由于他并非音乐家,他误解了拉莫思想中的一些细微之处,改变和删除了那些无法完美契合他对音乐理解的概念。