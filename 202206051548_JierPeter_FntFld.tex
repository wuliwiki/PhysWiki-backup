% 有限域
% 素域|域的扩张|伽罗瓦域|伽罗华域|Galois Field

\pentry{分裂域\upref{SpltFd},欧几里得环\upref{EuRing}}

有限域,即元素数量有限的域,有时也称\textbf{伽罗华域(Galois Field)},是一类性质良好的代数结构,在近代编码学、密码学、计算机理论等领域有广泛应用.

\subsection{有限域的基本结构}



设$\mathbb{F}$是一个有限域.

回顾\textbf{环和域}\upref{field}中的讨论,可知$\mathbb{F}$的特征一定是一个素数$p$,其素域相应为$\mathbb{Z}_p$.再由\textbf{域的扩张}\upref{FldExp}讨论可知,$\mathbb{F}$应该是$\mathbb{Z}_p$的有限扩张,假设$[\mathbb{F}:\mathbb{Z}_p]=k$,则$\abs{\mathbb{F}}=p^k$.

由此可知,有限域的阶一定是$p^k$.反过来,$p^k$阶的域一定存在吗?是否唯一呢?

由域的定义,非零元构成乘法群,因此$G=(\mathbb{F}-\{0\}, \times)$是一个\textbf{有限群},且$\abs{G}=p^k-1$.于是,对于任意$a\in G\subseteq\mathbb{F}$,有
\begin{equation}
a^{p^k}=a
\end{equation}
于是,$G$中元素都是多项式$f(x)=x^{p^k}-x$的根.再考虑到$f(0)=0$,因此$\mathbb{F}$中的元素都是$f$的根.

可$\mathbb{F}$中一共就$p^k$个不同的元素,因此\footnote{再注意到,多项式环都是欧几里得环,进而是主理想整环,进而是唯一析因环;而多项式总能被自己的根的一次因式约去.于是,多项式不同根的数目,不会超过自己的次数.}它们就是$f\in\mathbb{Z}_p[x]$的全部根.因此,$\mathbb{F}$是$f\in\mathbb{Z}_p[x]$的\textbf{分裂域}.

由分裂域的存在唯一性,$p^k$阶的域也是存在且唯一的.我们将其记为$\opn{GF}(p^k)$,意为“阶数为$p^k$的伽罗华域”;有时候也记为$\mathbb{F}_{p^k}$.

\begin{example}{}
由上述讨论,$f(x)=x^{p^k}-x$在$\opn{GF}(p^k)$上不同根的数目恰为$p^k$,故显然没有重根.

这一点也符合\autoref{SprbEx_cor1}~\upref{SprbEx}.$\opn{D}f(x)=p^kx^{p^k-1}-1$的最低次项是零次的,而显然$\opn{D}f\not\mid f$,故得$(f, \opn{D}f)=1$,故$f$应无重根.
\end{example}


\begin{theorem}{}\label{FntFld_the1}
有限域的乘法群,必是循环群.
\end{theorem}

\textbf{证明}:

考虑$\mathbb{F}=\opn{GL}(p^k)$,其非零元构成的乘法群记为$G$.

设$G$中各元素的阶,最大的是$m$.则$m\mid \abs{G} = p^k-1$,且$\forall g\in G$都是$f(x)=x^m-1$的根.

$f\in\mathbb{F}[x]$一共$m$个根,而$G$中的$p^k-1$个元素都是其根,加之$m\mid p^k-1$,可知$m=p^k-1$.

因此,阶数为$m=p^k-1=\abs{G}$的那个元素,就是$G$的循环生成元.

\textbf{证毕}.



\begin{corollary}{}\label{FntFld_cor1}
有限域是其素域的单扩域.
\end{corollary}

\textbf{证明}:

沿用\autoref{FntFld_the1} 的设定.

设$g_0$是$G$的循环生成元,则$G=\{g_0^i\mid i\in\mathbb{Z}^+\}$.

因此,$\mathbb{F}=G\cup\{0\}=\mathbb{Z}_p(g_0)$.

\textbf{证毕}.

\begin{definition}{原根}

如果$a$是$\opn{GL}(p^k)$的乘法群的循环生成元,由\autoref{FntFld_cor1} 知$\opn{GL}(p^k)=\mathbb{Z}_p(a)$.称$a$是$\opn{GL}(p^k)$的\textbf{本原元素(primitive element)}或\textbf{原根}.

\end{definition}






























