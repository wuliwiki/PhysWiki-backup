% 流、流密度
% keys 矢量场|电流密度|能流密度|速度
% license Xiao
% type Tutor

\pentry{矢量场\nref{nod_Vfield}}{nod_9847}

\subsection{流,流量}
单位时间流经某个截面的某种物理量叫\textbf{流(current)}。 最常见的例子是电流\upref{I}, 即单位时间流经某截面的电荷量, 用极限\upref{Lim}定义为
\begin{equation}
I = \lim_{\Delta t \to 0} \frac{\Delta q}{\Delta t}~.
\end{equation}
注意流的大小一般与所选择截面的位置, 方向, 面积都有关系。 我们也可以把电荷 $q$ 替换成其他物理量如质量、能量、粒子数, 分别得到质量流、能流、粒子流。

一段时间内流经截面的某种物理量的总量就叫做\textbf{流量}, 流量是时间 $t$ 的函数, 所以由导数\upref{Der} 的定义, 流是流量关于时间的导数。 反之, 流量是流在某段时间的定积分\upref{DefInt}。
\begin{equation}
\Delta q = \int_{t_1}^{t_2} I(t) \dd{t}~.
\end{equation}

\subsection{流密度}
\textbf{流密度(current density)}可以用于描述某时刻流体在的空间流动的速率。 我们以水流为例, 在一条河流或管道中, 某时刻 $t$ 在空间中任意一点 $\bvec r$ 处, 都对应一个水流速度 $\bvec v$, 如果我们在该点放置一个与速度垂直的微小截面(通常叫\textbf{面元}), 令其面积为 $\Delta S$, 在一段微小时间 $\Delta t$ 内流经截面的质量为 $\Delta m$, 那么\textbf{质量流密度(mass current density)}可以用极限定义为
\begin{equation}\label{eq_CrnDen_1}
\bvec j(\bvec r, t) = \uvec n \lim_{\Delta S, \Delta t \to 0} \frac{\Delta m}{\Delta S \Delta t}~,
\end{equation}
其中 $\uvec n$ 表示面元正方向法向量或者面元处流体的速度方向。 流密度是一个关于位置的矢量函数, 即矢量场\upref{Vfield}。 \autoref{eq_CrnDen_1} 中的质量可以替换为不同的物理量, 若替换为能量则称为\textbf{能流密度}, 若是粒子数则称为\textbf{粒子流密度}, 若是电荷量则称为\textbf{电流密度}\upref{Idens}, 等等。 

我们也可以根据密度和速度来定义流密度
\begin{equation}\label{eq_CrnDen_2}
\bvec j(\bvec r, t) = \rho(\bvec r, t) \bvec v(\bvec r, t)~.
\end{equation}
其中密度定义为(质量同样可以替换成其他物理量)
\begin{equation}\label{eq_CrnDen_3}
\rho(\bvec r, t) = \lim_{\Delta V\to 0} \frac{\Delta m}{\Delta V}~.
\end{equation}
\autoref{eq_CrnDen_2} 的定义和\autoref{eq_CrnDen_1} 是等效的, 因为时间 $\Delta t$ 内流经 $\Delta S$ 的体积为 $\Delta V = \Delta S \cdot v  \Delta t$, 代入\autoref{eq_CrnDen_3} 再代入\autoref{eq_CrnDen_2} 就得到了\autoref{eq_CrnDen_1}。

\subsection{流密度与流}
\pentry{曲面积分 通量\nref{nod_SurInt}}{nod_f761}
我们来讨论如何通过流密度来计算流。 通过\autoref{ex_SurInt_1}~\upref{SurInt} 和面积分(通量)的定义(\autoref{eq_SurInt_1}~\upref{SurInt}), 易得流密度作为矢量场在某截面上的通量就是该截面的流:
\begin{equation}
I(t) = \int \bvec j(\bvec r, t) \dd{\bvec s}~.
\end{equation}
