% Matlab 最小二乘法拟合多项式

\pentry{最小二乘法\upref{LstSqr}}

事实上, Matlab 的 \verb|polyfit| 函数已经可以实现多项式拟合, 但为了教学我们重新实现一次.
\begin{figure}[ht]
\centering
\includegraphics[width=8cm]{./figures/LSpoly_1.png}
\caption{15 阶多项式拟合正弦函数(蓝点为数据点, 红线为多项式)} \label{LSpoly_fig1}
\end{figure}

\begin{figure}[ht]
\centering
\includegraphics[width=8cm]{./figures/LSpoly_2.png}
\caption{20 阶多项式拟合锯齿波(蓝线为数据点, 红线为多项式)} \label{LSpoly_fig2}
\end{figure}

以下代码使用\autoref{LstSqr_eq2}~\upref{LstSqr} 解出多项式拟合的最小二乘法系数 $c_i$.
\begin{equation}
p(x) = c_0 + c_1 x + c_2 x^2 + \dots
\end{equation}

\begin{lstlisting}[language=matlab, caption=ls_poly.m]
% 多项式最小二乘法拟合
% p(x) = c(1)x^order + c_2 x^(order-1) + ... c(order+1)
% numel(x) = numel(y)
function c = ls_poly(x, y, order)
Sx = zeros(1, 2*order+1);
Sx(1) = numel(x);
for n=1:2*order
    Sx(n+1) = sum(x.^n);
end

Sxy = zeros(order+1, 1);
for n=0:order
    Sxy(n+1) = sum(y .* x.^n);
end
Sxy = flip(Sxy);

A = zeros(order+1, order+1);
for i = 0:order
    for j = 0:order
        A(order-i+1,order-j+1) = Sx(i+j+1);
    end
end
c = A \ Sxy; % 解线性方程组
end
\end{lstlisting}

使用例子: 拟合正弦函数.
\begin{lstlisting}[language=matlab, caption=ls_poly_demo.m]
% ls_poly_demo
x = linspace(-10, 10, 100);
y = sin(x);
figure; plot(x, y);
c = ls_poly(x, y, 15);
y1 = polyval(c, x); % 求值
hold on; plot(x, y1);
axis([-10, 10, -1, 1]);
\end{lstlisting}
运行结果如\autoref{LSpoly_fig1}.

\begin{lstlisting}[language=matlab, caption=ls_poly_demo2.m]
% ls_poly_demo2
x = linspace(-1, 1, 50);
y = zeros(size(x));
y(x > 0) = -x(x > 0);
y(x > 0.5) = x(x > 0.5)-1;
y(x < 0) = x(x < 0);
y(x < -0.5) = -x(x < -0.5)-1;
figure; plot(x, y, '.');
c = ls_poly(x, y, 20);
y1 = polyval(c, x);
hold on; plot(x, y1);
\end{lstlisting}
