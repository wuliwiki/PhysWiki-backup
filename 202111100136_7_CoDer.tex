% 协变导数
% Covariant Derivative|微分几何|differential geometry|christoffel symbol|克氏符|流形

\pentry{Christoffel符号\upref{CrstfS}}

定义联络时,我们讲联络看成是向量场之间的映射,$\nabla_\bvec{X}\bvec{Y}$中的$\bvec{X}$和$\bvec{Y}$都是光滑向量场.也就是说,我们着眼于场的整体,而没有关注局部的性质,比如某个点或者某个轨迹上的切向量场如何变化.

不过,在\textbf{Christoffel符号}\upref{CrstfS}词条中我们看到,在具体图中计算联络时,只用到了几个定义在欧几里得空间的函数,也就是向量场的坐标值函数和Christoffel符号.欧几里得空间上函数的求导是可以考虑局部的,也就是我们可以计算一个点上函数的导数,而不必像联络的定义那样要考虑整个场的变换.这就意味着我们有可能局部地计算联络.

\subsection{协变导数}

考虑一个带联络的流形$(M, \nabla)$.令$c:I\to M$是一个从区间$[0, 1]$到流形$M$上的连续映射,我们称之为一条\textbf{道路}.沿着$c(t)|_{t\in I}$定义一个光滑切向量场$\bvec{X}$.






















