% 雅可比行列式
% 多元微积分|坐标系变化|全微分|混合积|矩阵|雅可比矩阵|行列式|雅可比行列式

若有坐标系变换
 \begin{equation}
\leftgroup{
x = x(u,v,w)\\ y = y(u,v,w)\\ z = z(u,v,w)
}\end{equation}
根据全微分关系%(链接未完成)
\begin{equation}
\pmat{\dd{x}\\ \dd{y}\\ \dd{z}} =
\pmat{
\pdv*{x}{u} &  \pdv*{x}{v} & \pdv*{x}{w} \\ 
\pdv*{y}{u} & \pdv*{y}{v} & \pdv*{y}{w} \\ 
\pdv*{z}{u} & \pdv*{z}{v} & \pdv*{z}{w} }
\pmat{\dd{u}\\ \dd{v}\\ \dd{w}}
\end{equation}
其中 $\mat J$ 叫做雅可比矩阵.

考虑 $uvw$ 坐标系中的一个体积元 $(u,v,w)$-$(u + \dd{u}, v + \dd{v}, w + \dd{w})$,  一般情况下(不需要是正交曲线坐标系), 体积元为平行六面体, 起点为 $(u,v,w)$  的三条棱对应的矢量分别为
 \begin{equation}
\pmat{\dd{x_1}\\\dd{y_1}\\\dd{z_1}} = 
\mat J\pmat{\dd{u}\\0\\0} = 
\pmat{J_{11}\\J_{21}\\J_{31}} \dd{u}
\end{equation} 
\begin{equation}
\pmat{\dd{x_2}\\\dd{y_2}\\\dd{z_2}} = 
\mat J \pmat{0\\ \dd{v}\\0} = 
\pmat{J_{12}\\J_{22}\\J_{32}} \dd{v}
\end{equation} 
\begin{equation}
\pmat{\dd{x_3}\\\dd{y_3}\\\dd{z_3}} = 
\mat J \pmat{0\\0\\\dd{w}} = 
\pmat{J_{13}\\J_{23}\\J_{33}} \dd{w}
\end{equation} 
由于平行六面体的体积是同一起点三条矢量的混合积%(链接未完成)
\begin{equation}
\dd{V} = \vmat{
\dd{x_1} & \dd{y_1} & \dd{z_1}\\
\dd{x_2} & \dd{y_2} & \dd{z_2}\\
\dd{x_3} & \dd{y_3} & \dd{z_3}}
= \vmat{
\dd{x_1} & \dd{x_2} & \dd{x_3}\\
\dd{y_1} & \dd{y_2} & \dd{y_3}\\
\dd{z_1} & \dd{z_2} & \dd{z_3}}
= \abs{\mat J} \dd{u}\dd{v}\dd{w}
\end{equation}
其中 $\abs{\mat J}$  叫做\textbf{雅可比行列式}.

