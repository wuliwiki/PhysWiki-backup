% 苏州大学 2012 年硕士物理考试试题
% keys 苏州大学|2012年|考研|物理
% license Copy
% type Tutor

\textbf{科目代码:838}

常用物理常数和公式:

真空介电常数$\varepsilon_0=8.85 *10^{-12} \quad C^2/(N.m^2)$

库伦常数 $k=1/(4\pi \varepsilon_0)=9*10^9 \quad N.m^2/c^2$

真空磁导率 $\mu_0=1.26*10^{-6}\quad H/m$

电子电量 $e=+1.6*10^{-19}\quad C$

电子质量 $m_e=9.11*10^{-32}\quad kg=510 \quad keV/c^2$

真空光速 $c=3.00* 10^8 \quad m/s$

普朗克常数$h=6.63x10^{-34}\quad JS$

玻尔效曼常数$k_B =1.38*10^{-23}J/K$

$Hc=1240 eV.nm$

$1 eV=1.6*10^{-19}J$

$F=k Q_1 Q_2/r^2$

$Q=CV$

$C=\varepsilon A/d$

$F=qE$

$V=kQ/r$

$F=ILB\sin \theta$

$F=qvB$

$d\sin \theta =m\lambda $

$P=h/\lambda$

\begin{enumerate}
\item 如下图所示,水平面上两个固定的点电荷,带电量都是$+e$,两者相距$ 2a$,一个$\alpha $粒子(带电量$+2e$)沿者这两个点电荷的中垂线快速穿过,试计算$\alpha$粒子在什么位置时的受力最大?
\begin{figure}[ht]
\centering
\includegraphics[width=6cm]{./figures/07431eed4f8c2041.png}
\caption{} \label{fig_SD12_1}
\end{figure}
\item 如图所示,三个半径相同的均匀导体圆环两两正交,在各交点处彼此连接,每个圆环的电阻为$R$,求 $A$点到$B$点之间的等效电阻$R_{AB}$。
\begin{figure}[ht]
\centering
\includegraphics[width=6cm]{./figures/d227e5891aec7425.png}
\caption{} \label{fig_SD12_2}
\end{figure}
\item 如下图所示,一个内半径为$ a$外半径为$b$的均匀带电绝缘环状薄片,以角速度 $w $绕着过中心O点垂直于环片平面的轴旋转,环片上总带电量为$Q$,求环片中心O点的磁感应强度$B$的大小和方向。
\begin{figure}[ht]
\centering
\includegraphics[width=6cm]{./figures/20fe004321fff1f4.png}
\caption{} \label{fig_SD12_3}
\end{figure}
\item 用长度为$L$的细金属线一端连着一个绝缘球$P$另一端悬挂在O点,构成一个圆锥摆,$P$作水平匀速圆周运动时金属丝与整直线夹角为$\theta$,角速度为$\omega$,如图所示。在空间分布有水平方向的匀强磁场$B$,则金属丝上$P$点与O点之间的最小电势差为多大?最大电势差为多大?
\end{enumerate}