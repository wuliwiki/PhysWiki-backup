% 单纯同调群的计算
% keys 复形|complex|同调|homology|群|群同态|三角剖分

\pentry{单纯剖分(三角剖分)\upref{Traglt}}

对一个拓扑空间进行单纯剖分,也就是把这个空间用一个复形来表示.复形上的单纯同调群,就是这个拓扑空间的单纯同调群.本节讨论一些常见拓扑空间的单纯同调群计算,以及一些有助于计算的定理.

\subsubsection{连通复形上的零维单纯同调群}

考虑复形$K$,将其顶点的集合记为$\{a_i\}_{i=1}^n$,那么$K$的零维链群$C_0(K)$、同时也是零维闭链群$Z_0(K)$\footnote{因为零维的链必然闭链.},就是顶点集合的\textbf{自由生成阿贝尔群}.

如果一条零维链为$x_o=\sum m_ia_i$,其中$m_i\in\mathbb{Z}$,则称$\opn{ind} x_0=\sum m_i$为$x_0$的\textbf{指数(index)}.

用指数定义一个\textbf{群同态}$\epsilon: C_0(K)=Z_0(0)\to\mathbb{Z}$\footnote{这句的意思是,$\epsilon$既是$C_0(K)$上的映射,又是$Z_0(K)$上的映射,因为这两个群是一样的.},其中$\epsilon(x)=\opn{ind} x$.

\begin{lemma}{}
$B_0(K)=\opn{ker}\epsilon$.换句话说,$K$上的一条零维(闭)链,“它是边缘链”$\iff$“它的指数为零”.
\end{lemma}

\textbf{证明}:

$\Rightarrow$:一维链$\sum m_{ij}a_ia_j$的边缘就是$\partial\sum m_{ij}a_ia_j=\sum m_{ij}(a_j-a_i)$,其中$m_{ij}=m{ji}$.易得$\opn{ind}\sum m_{ij}(a_j-a_i)=0$.

一个简单的例子是,$a_0a_1+a_0a_2$的边缘就是$a_1-a_0+a_2-a_0$,其指数就是$1-1+1-1=0$.


$\Leftarrow$:如果一个零维链$\sum m_ia_i$指数为零,那么$\sum a_i=0$.我们可以把$m_ia_i$分为$\abs{m_i}$个不同的$\opn{sgn}(m_i)a_i$\footnote{$\opn{sgn}$是符号函数,比如对于$-2a_i$,就可以分成$2$个$-a_i$.},然后

\textbf{证毕}.













