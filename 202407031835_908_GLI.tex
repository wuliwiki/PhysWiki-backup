% G 力
% license CCBYSA3
% type Wiki

(本文根据 CC-BY-SA 协议转载自原搜狗科学百科对英文维基百科的翻译)

$g$力描述为重力的等效,或者更常见的单位质量力的量度–典型加速度–一种对重量的感知力 ,1克物体的$g$力等于地球上重力加速度的常规值, $g$大约$9.8 m/s^2$[1]。由于重力加速度间接产生重力,任何$g$力都可以被描述为“单位质量的重力”。当一个物体的表面被另一个物体的表面推动而产生$g$力时,这个推动的反作用力对物体质量的每一个单位产生等大反向的$g$力。所涉及的力的类型是通过物体内部的机械应力传递的。重力加速度(除了某些电磁力的影响)是物体自由落体产生加速度的原因。[2][3]

物体受到的$g$力是由作用在物体自由运动上的所有非重力和非电磁力的矢量和决定的。实际上,如上所述,这些是物体表面之间的接触力。这种力对物体造成应力和应变,因为它们必须从物体表面传递。由于这些应变,大的$g$力可能对物体产生损坏。

即使$g$力是用标准重力自由落体加速度的倍数来描述的,但重力单独作用不会产生$g$力。因此,地球表面的标准重力只是间接产生$g$力,这实际是对重力产生抵抗的一种机械力。正是这些机械力在质量上产生了$g$力。例如,一个1g的力作用在位于地球表面的物体上,$g$力是由地面向上施加的机械力引起的,它克服重力阻止物体自由下落。来自地面向上的接触力确保静止在地球表面的物体相对于自由落体而进行加速运动。(自由落体是物体向地球中心自由下落时遵循的路径)。物体内部的应力来自于仅从与地面接触点传递而来的接触力。

物体在引力场中自由运动时有质量而不表现重力或重力较小的一种状态,此时物体以惯性轨迹进行运动而感觉不到重力,这种情况被称为零重力(意思是零$g$力)。这一点可以通过电梯内自由降落到地球中心(在真空中)的“零重力”条件,或者(非常近似地)地球轨道航天器内的“零重力”条件来证明。这些都是没有重量感的坐标加速度(速度变化)的例子。没有重力(零重力)的体验,不管它是如何产生的,都等同于失重。

在没有引力场的情况下,或者在与引力场成直角的方向上,适当的加速度和坐标加速度是相同的,任何坐标加速度都必须由相应的$g$力加速度产生。这里的一个例子是自由空间中的火箭,发动机产生简单的速度变化,并对火箭和乘客产生$g$力。

\subsection{单位和测量}
国际单位制中加速度的计量单位是米/秒2。然而,为了区分自由落体的加速度和简单加速度(速度变化率),单位$g$(或$g$)经常被使用。1g 是地球表面重力产生的每单位质量的力,是标准重力(符号: gn),定义为 9.80665米每秒的平方,[4] 或者等同地 9.80665每千克质量的牛顿力。请注意单位定义不随位置变化—站在月球上的$g$力约为0.165g,即几乎是在地球上的 1/6 。

单位$g$不是国际单位制之一,国际单位制单位用“$g$”表示克。 此外,“$g$”不应与“$G$”混淆,后者是重力常数的标准符号。[5] 这种符号通常用于航空,特别是在特技飞行或军用航空中,用来描述飞行员为了保持清醒而必须克服的增加的力,而不是重力诱发的意识丧失。[6]

$g$力的测量通常使用加速度计来实现。在某些情况下,$g$力可以使用适当校准的天平进行测量。比力是$g$力的另一个名称。

\subsection{加速度和力}
$g$力这个术语在技术上是不正确的,因为它是加速度而不是力的度量。加速度是一个矢量,$g$力加速度(简称“g-forces”)通常表示为标量,正$g$力指向下方(表示向上加速度),负$g$力指向上方。因此,$g$力是加速度的矢量。它是一种加速度,必须由机械力产生,不能由简单的引力产生。仅有重力作用的物体没有$g$力,被称为是失重的。

$g$力,当乘以它们作用的质量时,这个术语在某种与机械相关联的力上具有一定的意义,这个力产生压缩应力和拉伸应力。这种力产生了重力的感觉,但是由于向下方向上正重力的定义,等式带有符号变化,因此重力的方向与$g$力加速度的方向相反:

重力=质量×(-g力)(Weight = mass × (-g-force))
负号的原因是g力在物体上产生的实际力(即测量的重量)与g力的符号方向相反,因为在物理学中,重力不是产生加速度的力,而是与之等大反向的反作用力。如果向上的方向被认为是正的(正常的笛卡尔约定),那幺正g力(指向上方的加速度矢量)在任何质量上产生一个向下的力/重力(一个例子是火箭发射的正加速度,产生向下的重力)。同样,一个负g力是指向下方加速度矢量(y轴上的负方向),这种向下的加速度在一个方向上产生向上重力 (从而将飞行员向上拉出座椅,并迫使血液流向正常方向的飞行员头部)。

如果g力(加速度)垂直向上,由地面(通过时空加速)施加,或者由电梯地板施加给站立的人,那么大部分身体会受到压缩应力,在任何高度,如果乘以面积,该压缩应力就是相关的机械力,该机械力是g力和支撑质量(支撑水平面以上的质量,包括从该水平面以上垂下的手臂)的乘积。与此同时,臂本身也承受拉伸应力,在任何高度,如果乘以面积,也就是相关的机械力,它是g力和悬挂在机械支撑点以下的质量的乘积。机械阻力从与地板或支撑结构的接触点开始扩散,并在未支撑的端部(在从下方支撑的情况下是顶部,例如座位或地板,身体或物体悬挂部分的底部)逐渐减小到零。在压缩力被计算为负张力的情况下,每单位质量(物体各部分之间的变化,使得物体在它们之间的切片具有单位质量)在g力方向上的张力变化率等于g力加上切片上的非重力外力(如果有的话)(在与g力相反的方向上被计为正)。

不管这个g力是由对重力的机械阻力引起的,还是由机械力产生的坐标加速度(速度变化)引起的,或者是由这两者的组合引起的,但对于给定的g力,应力是相同的。因此,对人们来说,无论是否引起坐标加速,所有的机械g力都感觉完全一样。同样,对于物体来说,它们能否承受机械g力而不损坏的问题对于任何类型的g力都是一样的。例如,地球上的向上加速(例如,上升时速度增加,下降时速度降低)感觉就像静止在具有较高表面重力的天体上一样。引力单独作用不会产生任何g力;g力仅由机械推拉产生。对于一个自由的物体(在空间中自由运动的物体),这种g力只有在“惯性”路径(重力的自然效应或质量惯性的自然效应)改变时才会出现。这种改变只能由重力以外的影响引起。

涉及g力的重要案例包括: