% 量子简谐振子(升降算符法)
% keys 简谐振子|本征值|升降算符|能级
% license Xiao
% type Tutor

\begin{issues}
\issueAbstract
\end{issues}

%未完成
%(首先要参考”升降算符”这篇, 根据里面的定义证明a+a-就是H的升降算符, 然后就好办了).

\pentry{简谐振子\nref{nod_SHO}, 升降算符\nref{nod_RLop}}{nod_79f2}

量子简谐振子的能级为
\begin{equation}
E_n = \qty(n+\frac12)\omega \hbar~.
\end{equation}
基态波函数为
\begin{equation}
\psi_0 (x) = \frac{1}{\pi^{1/4} \beta^{1/2}} \E^{-(x/\beta)^2/2}~.
\end{equation}
其中具有长度量纲的常数 $\beta$ 为
\begin{equation}
\beta  = \sqrt{\frac{\hbar}{m\omega}}~.
\end{equation}

以下使用升降算符法解出能级和基态, 总结其中的关键步骤为: 令升降算符为
\begin{equation}
\Q a_\pm = \frac{1}{\sqrt{2m\omega\hbar}} (m\omega x \mp \I p) \quad~,
\end{equation}
满足
\begin{equation}
a_+ \psi_n = \sqrt{n + 1} \psi_{n+1}~,
\quad
a_- \psi_n = \sqrt n \psi_{n-1}~,
\end{equation}
那么哈密顿算符为
\begin{equation}
H = \omega\hbar \qty(\Q n + \frac12)~,
\end{equation}
其中 $\Q n$ 为量子数算符
\begin{equation}
\Q n = a_+ a_ - \qquad~.
\end{equation}

\subsection{详细推导(升降算符法)}

在经典的简谐振子模型中,若质点沿 $x$ 轴方向振动,且在 $x = 0$ 处平衡,则势能函数 $V(x) = k x^2/2$。 由于自由振动的频率为 $\omega = \sqrt{k/m}$, 所以势能可记为
\begin{equation}
V(x) = \frac12 m \omega^2 x^2~.
\end{equation}

我们直接把这个势能代入薛定谔方程,即考虑的哈密顿算符为
\begin{equation}
H = \frac{p^2}{2m} + V = \frac{p^2}{2m} + \frac12 m\omega^2 x^2 = \frac{1}{2m} [p^2 + (m\omega x)^2]~.
\end{equation}
定态薛定谔方程(能量的本征方程)为
\begin{equation}\label{eq_QSHOop_6}
H\psi  = E\psi~.
\end{equation}

通常解这个方程需要使用幂级数%\addTODO{链接}
,但我们可以先利用\enref{升降算符}{RLop}来得到能量的本征值,再求本征函数。


\subsubsection{寻找升降算符}

形式上$H=A^2x^2+B^2p^2$,其中$A^2=m\omega^2/2, B^2=1/2m$。如果$H, x, p$是\textbf{数字},那么有$H = (Ap+\I Bx)(Ap-\I Bx)$;然而实际上它们都是\textbf{算符},所以有:
\begin{equation}\label{eq_QSHOop_1}
(Ax+\I Bp)(Ax-\I Bp) = H-\I AB[x, p] = H+\hbar AB~.
\end{equation}
容易证明,算符$(Ax\pm\I Bp)$的对易关系为:
\begin{equation}\label{eq_QSHOop_2}
[(Ax+\I Bp), (Ax+-\I Bp)] = 2\hbar AB~.
\end{equation}
把\autoref{eq_QSHOop_2} 代回\autoref{eq_QSHOop_1} 易得
\begin{equation}
[H, (Ax\pm\I Bp) = [H+ \hbar AB, (Ax\pm\I Bp)] = \mp 2\hbar AB ((Ax\pm\I Bp))~.
\end{equation}
由\autoref{eq_RLop_1},显然$(Ap\pm \I Bx)$是$H$的升降算符,能把$H$的本征态的本征值改变$\mp 2\hbar AB=\mp \hbar \omega$。

给$(Ax\pm\I Bp)$乘上适当的系数,我们就能得到 $H$ 的升降算符,它们分别可以把本征值升降 $\omega\hbar$ :
\begin{equation}
\Q a_\pm = \frac{1}{\sqrt{2m\omega\hbar}} (m\omega x \mp \I p), 
\qquad
\Delta E = \pm \omega \hbar~.
\end{equation}

根据\autoref{eq_RLop_1}, 对任意一个 $H$ 的本征函数 $\psi_n$, 有
\begin{equation}
H(a_\pm\psi_n) = (E_n\pm\hbar\omega) (a_ \pm \psi_n)~.
\end{equation}
根据\autoref{sub_RLop_1} 给出的“可对角化算符具有升降算符的充要条件”,简谐振子的定态薛定谔方程的解中,哈密顿算符$H$的本征值 $E_n$ 取离散值,且相邻两个能级相差 $\Delta E = \hbar \omega$ 。


由于$H,\hat n$之间为线性关系,推断其有共同的本征态,记为$\psi_n$.则$\hat n$的本征方程
\begin{equation}
\hat n \psi_n=n\psi_n~.
\end{equation}
从而
\begin{equation}
H\psi_n=\hbar\omega(n+\frac12)\psi_n~.
\end{equation}
由于
\begin{equation}
\begin{bmatrix}
N,a_-
\end{bmatrix}=
\begin{bmatrix}
a_+ a_-,a_-
\end{bmatrix}=a_+
\begin{bmatrix}
a_-,a_-
\end{bmatrix}+
\begin{bmatrix}
a_+,a_-
\end{bmatrix}a_-=-a_-~,
\end{equation}
和
\begin{equation}
[N,a_+]=a_+~,
\end{equation}


\addTODO{谐振子有基态的理由不充分。考虑引入粒子数算符来解释?}
类似于\enref{无限深势阱}{ISW},谐振子也应该有一个最低能级 $E_0$ 和对应的 $\psi_0(x)$。 所以 $a_-$ 必然对 $\psi_0$ 无效,即得到的波函数没有物理意义,所以不妨猜测 $a_- \psi_0 = 0$ 
即
\begin{equation}
(m\omega x + \I p)\psi_0 = 0
\quad \Rightarrow \quad
\dv{x} \psi_0 =  - \frac{m\omega x}{\hbar } \psi_0~.
\end{equation}
%(可以根据这个,解出最低能级为 $\frac12 \hbar \omega $, 于是乎,就有了猫猫爬梯梯的图片了).
这是一阶齐次线性微分方程,%通解
通解为
\begin{equation}
\psi_0(x) = \frac{1}{\pi^{1/4}\sqrt\beta} \E^{-(x/\beta)^2/2}~,
\end{equation}
其中 $\beta = \sqrt{\hbar /m\omega}$ 具有长度量纲。不难验证上式是定态薛定谔方程\autoref{eq_QSHOop_6} 的解,本征值为 $E_0=\omega\hbar/2$。



\subsubsection{归一化条件}

本小节的目的是证明 $\psi_{n+1} = a_+\psi_n/\sqrt{n+1}$,  $\psi_{n-1} = a_- \psi_n/\sqrt n$ 。


\begin{equation}\ali{
A^2 & \equiv \bra{\psi_n}a^*_+a_+\ket{\psi_n}\\
& = \int (a_+ \psi_n)^* (a_+ \psi_n) \dd{x}
= \int (a_+^* a_+ \psi_n)^* (\psi_n) \dd{x}\\
&= \int (a_- a_+ \psi_n)^* (\psi_n) \dd{x}~,
}\end{equation}
而 $a_- a_+ \psi_n = H\psi_n/(\omega\hbar) + \psi_n/2 = (n + 1)\psi_n$。 所以
\begin{equation}
A^2 = (n + 1)\int \psi_n^*{\psi_n} \dd{x}  = n + 1\qquad
A = \sqrt{n+1}~.
\end{equation}
所以 $\psi_{n+1} = a_+ \psi_n/\sqrt{n+1}~.$



\subsubsection{推导谐振子的能量本征态}

首先考虑能量基态,记为$\ket{0}$,其在位置表象的波函数为$\psi_0(x)$。记位置算子的本征矢量为$\ket{x_0}$,其波函数为$\delta(x-x_0)$。

注意,如果态$\ket{s}$在位置表象下的波函数为$\psi(x)$,则$\braket{x}{s}=\psi(x)$。

设$a_{\pm}$是升降算符,则
\begin{equation}\label{eq_QSHOop_4}
a_-\ket{0}=0 \implies \bra{x}a_-\ket{0}=0~.
\end{equation}

由于$a_-\frac{1}{\sqrt{2m\omega\hbar}} (m\omega x + \I p)$,故\autoref{eq_QSHOop_4} 可以视为$\ket{0}$满足的微分方程,即$a_-\ket{0}$对应的波函数是$0$:
\begin{equation}\label{eq_QSHOop_3}
(m\omega x + \hbar\frac{\dd}{\dd x}) \psi_0(x) = 0~.
\end{equation}
\autoref{eq_QSHOop_3} 归一化的解为
\begin{equation}
\psi_0(x)=\qty(\frac{1}{\pi^{1/4}\sqrt{x_0}})\exp\qty(-\frac{1}{2}\cdot\frac{m\omega}{\hbar}x^2)~.
\end{equation}

考虑到$a_+\ket{n}=\sqrt{n+1}\ket{n+1}$,同样可以用$\psi_n(x)=\braket{x}{n}$求得
\begin{equation}
\psi_1(x) = \braket{x}{1} = \bra{x}a_+\ket{0}=\qty(\sqrt{\frac{m\omega}{2\hbar}})\qty(x-\frac{m\omega}{\hbar}\frac{\dd}{\dd x})\braket{x}{0}~.
\end{equation}

以此类推,可求得所有能量本征态的波函数,只需要对基态波函数不断使用升算符和归一化系数:%公式链接
\begin{equation}
\psi_n = \frac{a_+}{\sqrt {n!}} \psi_0~.
\end{equation}




%这段应该不需要了
% \subsection{升降算符的证明}

% 根据升降算符的定义, 要证明 $a_\pm$ 是升降算符只需证明对易关系

% \begin{equation}
% \comm{H}{a_\pm} =  \pm \omega\hbar a_\pm
% \end{equation}
% 根据升降算符定义
% \begin{equation}
% \comm{H}{a_\pm} = \frac{1}{2m\sqrt{2m\hbar \omega}} \comm{(m\omega x)^2 + p^2}{m\omega x \mp \I p}
% \end{equation}
% 不难证明
% \begin{equation}
% \comm{\Q A + \Q B}{\Q C + \Q D} = \comm{\Q A}{\Q C} + \comm{\Q A}{\Q D} + \comm{\Q B}{\Q C} + \comm{\Q B}{\Q D}
% \end{equation}
% 和
% \begin{equation}
% \comm{x^2}{x} = \comm{p^2}{p} = 0
% \end{equation}
% 将两式代入后,只剩下两个交叉项
% \begin{equation}
% \comm{H}{a_\pm} = \frac{\omega}{2\sqrt{2m\hbar \omega}} \qty(\mp \I \comm{x^2}{p} + \comm{p^2}{x})
% \end{equation}
% 同样可以证明
% \begin{equation}
% \comm{\Q A\Q B}{\Q C} = \Q A \comm{\Q B}{\Q C} + \comm{\Q A}{\Q C}\Q B
% \end{equation}
% 所以
% \begin{equation}
% \comm{x^2}{p} = x\comm{x}{p} + \comm{x}{p} x = 2\I\hbar  x
% \end{equation}
% \begin{equation}
% \comm{p^2}{x} =  -  p\comm{x}{p} - \comm{x}{p} p =  - 2\I\hbar  p
% \end{equation}
% 代入得
% \begin{equation}
% \comm{H}{a_\pm} = \frac{\pm\omega\hbar}{\sqrt{2m\hbar\omega}} (x \mp \I p) =  \pm \omega\hbar a_ \pm
% \end{equation}
% 证毕。
