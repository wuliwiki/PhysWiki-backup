% 动量(高中)
% 动量|冲量|动量定理|动量守恒定律|碰撞

\begin{issues}
\issueDraft
\issueTODO
\end{issues}

\pentry{牛顿运动定律\upref{HSPM03}}

\subsection{动量}

物体的质量和速度的乘积叫做\textbf{动量},是体现运动物体作用效果的物理量,表达式为:
\begin{equation}
\bvec p = m\bvec v
\end{equation}

动量是一个矢量,单位为千克米每秒($\mathrm{kg\cdot m/s}$),方向与速度方向相同.

质量为$m$的物体,在恒定合外力$\bvec F $的作用下做匀变速直线运动,速度由$\bvec v_1$变为$\bvec v_2$,则其动量变化为:
\begin{equation}
\Delta \bvec p =\bvec p_2 - \bvec p_1 = m\bvec v_2 - m\bvec v_2 = m\Delta \bvec v
\end{equation}

\subsection{冲量}

力和力的作用时间的乘积叫做力的\textbf{冲量},是体现力在其作用时间上积累效果的物理量,表达式为:
\begin{equation}
\bvec I=\bvec F \Delta t
\end{equation}

恒力的冲量,可用上式求解;变力的冲量,其计算可考虑分解为多个恒力作用的阶段、计算力—时间图像上对应图形的面积(力的方向恒定时)、求平均力再代入计算或使用动量定理(\autoref{HSPM08_eq1} )等方法.

\subsection{动量定理}

物体在一个过程中所受力的冲量等于它在这个过程始末的动量变化量.

对于式1,若速度的变化所用时间为$\Delta t$,则动量的变化率为:
\begin{equation}\label{HSPM08_eq1}
\frac{\Delta \bvec p}{\Delta t}=\frac{m\Delta v}{\Delta t}=m\bvec a=\bvec F
\end{equation}

可见动量的变化率等于合外力.

由\autoref{HSPM08_eq1} 可得:
\begin{equation}
\bvec F\Delta t=\Delta \bvec p
\end{equation}

\subsection{动量守恒定律}

\subsection{碰撞}
