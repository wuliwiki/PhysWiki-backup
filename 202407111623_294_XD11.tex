% 厦门大学 2011 年硕士入学物理考试试题
% keys 厦门大学|2001年|考研|物理
% license Copy
% type Tutor

\textbf{声明}:“该内容来源于网络公开资料,不保证真实性,如有侵权请联系管理员”

\begin{enumerate}
\item 跳伞运动员初张伞时(设此初始时刻$t=0$)的速度为$v_0$。已知下降过程中阻力大小与速度平方成正比,比例系数为$a$。若人伞总质量为$m$,求跳伞运动员的速度随时间变化的函数。
\item -匀质细棒长为$ 2L$,质量为 $m$,以与棒长方向相垂直的速度$v_0$在光滑水平面内平动时,与前方一固定的光滑支点O发生完全非弹性碰撞。碰撞点位于棒中心的一方$\frac{1}{2}L$处,如图所示。求棒在碰撞后的瞬时绕O点转动的角速度$\omega$。(已知长度为$l$,质量为$m$的均匀细棒绕通过其端点日与其垂直的轴转动时的转动惯量J=-m/。)
\end{enumerate}