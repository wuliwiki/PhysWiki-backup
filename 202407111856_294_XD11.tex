% 厦门大学 2011 年硕士入学物理考试试题
% keys 厦门大学|2001年|考研|物理
% license Copy
% type Tutor

\textbf{声明}:“该内容来源于网络公开资料,不保证真实性,如有侵权请联系管理员”

\begin{enumerate}
\item 跳伞运动员初张伞时(设此初始时刻$t=0$)的速度为$v_0$。已知下降过程中阻力大小与速度平方成正比,比例系数为$a$。若人伞总质量为$m$,求跳伞运动员的速度随时间变化的函数。
\item -匀质细棒长为$ 2L$,质量为 $m$,以与棒长方向相垂直的速度$v_0$在光滑水平面内平动时,与前方一固定的光滑支点O发生完全非弹性碰撞。碰撞点位于棒中心的一方$\frac{1}{2}L$处,如图所示。求棒在碰撞后的瞬时绕O点转动的角速度$\omega$。(已知长度为$l$,质量为$m$的均匀细棒绕通过其端点且与其垂直的轴转动时的转动惯量$J=\frac{1}{3}ml^2$)
\begin{figure}[ht]
\centering
\includegraphics[width=6cm]{./figures/fc0910d91171e693.png}
\caption{} \label{fig_XD11_1}
\end{figure}
\item 容器中储存有由$16$克氧气和$4$克氦气组成的混合气体,系统的压强为$\rho=1.0*10^5Pa$,温度$T=300K$。已知气体普适常量$R=8.31J/(mo1.K)$,氧气和氦气的分子量分别为32和4,试计算:\\
(1)系统的内能;\\
(2)该混合气体的定容摩尔热容;\\
(3)混合气体分子的方均根速率。
\item 已知大气压强为$l$高的水银柱,水银的密度为$\rho$。一粗细均匀的刚性细管出长度均为$l$的两部分组成,其中一部分沿水平方向,另一部分沿竖直方向,有一段长度为$l/2$的水银柱,开始时,水银柱刚好位于细管水平部分的右半部并处于平衡状态,被封闭的气柱长度为 $l/2$,如图所示。现使气柱中的气体缓慢膨胀,直到水银从细管的开口端全部逸出为止。\\
(1)求整个过程中封闭气柱中的气体压强$\rho$与气柱长度$x$的函数关系;\\
(2)已知封闭气柱中气体的定容摩尔热容$C_{F,m}=\frac{3}{2}R$,其中$R$为气体普适常量,求在整个过程中封闭气柱中的气体与外界交换的热量(忽略水银柱与气之间的热交换 )。
\item 电荷$Q$均匀分布在半径为$R$的球体内,求其电场能量。
\item 粗细均匀的金属导线构成如图线框,中间为正三角形$abc$,电流$I$从无限远处沿垂直于$bc$ 的方向从$a$端流入,从$b$端沿$cb$方向流向无限远处,三角形边长为$l$,求三角形中心点$O$处的磁感应强度。
\begin{figure}[ht]
\centering
\includegraphics[width=6cm]{./figures/2ead566e8812dfde.png}
\caption{} \label{fig_XD11_2}
\end{figure}
\item 在半径为$a$的细长螺线管中,磁场随时间的变化率为$\dv{B}{t}$且$\dv{B}{t}$>0,一直导线驾成等腰梯形闭合回路 $ABCD$,上下底分别长为$a$和$2a$放置如图,求梯形各边以及整个回路的感应电动势。
\begin{figure}[ht]
\centering
\includegraphics[width=6cm]{./figures/5c4ece38ef2eb835.png}
\caption{} \label{fig_XD11_3}
\end{figure}
\end{enumerate}