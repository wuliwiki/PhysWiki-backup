% 子群
% 子群|正规子群|左陪集|同余|商群|拉格朗日定理

\pentry{群\upref{Group}}

\subsection{子群}

\begin{definition}{子群}
给定一个群$(G, \cdot)$,如果集合$G$有一个子集$H$,使得$e\in H$且$H$中的元素在运算$\cdot$下仍然封闭,那么显然\footnote{由于已经知道$(G,\cdot)$构成一个群了,群的四条公理中,结合性、单位元存在性以及逆元存在性都被满足了.}$(H,\cdot)$也构成一个群.称$H$是群$G$的\textbf{子群(subgroup)}.
\end{definition}

虽然群和子群的联系很紧密,但是我们通常还是把它们看作由完全无关的集合所构成的,只不过可以自然地应用已经存在的群运算和子集关系来定义子群的运算.这样,将已有的运算直接用在子集上,有时被称作在子集上\textbf{导出}或\textbf{诱导(induce)}了一个运算,有时也称子集上的运算是\textbf{限制在子集上的运算}.比如在定义里,群$H$的运算实际上被认为是和$G$的运算不一样,严格来说应该记为$\cdot|_H$,意思是“限制在$H$上的$\cdot$”. 但是不至于引起混淆的时候,也可以简单记为“$\cdot$”,并认为是同一个运算.

一般来说,当我们说$H$是一个子群时,强调的是$H$和$G$的关系;但如果我们说$H$是一个群,我们关心的是$H$本身作为群的性质,而没有强调它和其它群的关系.

\begin{example}{整数加群的子群}\label{Group1_ex1}
我们已经知道,全体整数配上通常的加法,能构成一个群(\autoref{Group_ex1}~\upref{Group}),可以简称为“整数加群”.整数加群的集合中包括了全体整数.如果我们只取偶数,会发现偶数之间的加法和加法的逆(减法)的结果还是偶数,也就是说,全体偶数对于整数加法是封闭的,可以构成整数加群的一个子群.这个子群,我们记为$2\mathbb{Z}$.

类似地,如果取某个整数$n$的全体倍数构成集合,这个集合上的加法也是封闭的,构成子群,记为$n\mathbb{Z}$.比如说,$3\mathbb{Z}=\{0, \pm3, \pm6, \pm9, \cdots\}$.特别地,也可以把$\mathbb{Z}$本身看成是$1$的倍数构成的集合.
\end{example}

\begin{theorem}{子群的判别法}\label{Group1_the3}
设有群$G$及其子集$H$,则$H$是$G$的子群当且仅当:对于任意$x, y\in H$,有$x^{-1}y\in H$.
\end{theorem}

充分性证明:
\begin{itemize}
\item 单位元存在性: 我们需要证明群 $G$ 中的单位元属于 $H$. 取$x\in H$, 且 $y = x$, 那么 $e = y^{-1}x \in H$.

\item 逆元存在性:由于 $x$ 是群 $G$ 的元, $x^{-1}$ 必然存在, 但我们还要证明 $x^{-1} \in H$. 取 $x \in H$, $y = e \in H$, 那么 $x^{-1} = x^{-1}y \in H$.

\item 封闭性:取$x, y\in H$,则$x^{-1}\in H$,进而$xy=(x^{-1})^{-1}y\in H$.

\item 结合性自然继承自$G$的结合性.
\end{itemize}

必要性证明: 如果$H$是个群,显然对于任意$x, y\in H$,有$x^{-1}y\in H$.

\textbf{证毕}.
