% 正定二次型
% keys 正定型|雅可比方法
\pentry{实二次型\upref{RQuaF}}
\subsection{正定二次型}
\begin{definition}{实二次型的分类}
非退化的实二次型 $q:V\rightarrow\mathbb R$ 称为\textbf{正定的}(\textbf{负定的}),如果 $q(\bvec x)>0(q(\bvec x)<0)$ 对任意矢量 $\bvec x\neq0$ 都成立.$q$ 称为\textbf{半正定的}(或\textbf{非负定的}),如果 $q(\bvec x)\geq0$ 对所有 $\bvec x\in V$ 成立.最后,$q$ 称为\textbf{不定的},如果它有时取正有时取负.
\end{definition}
由于实二次型均可化为标准型\autoref{RQuaF_the1}~\upref{RQuaF},故实二次型的各种类型对应的标准型如下($n=\mathrm{dim}_\mathbb R \,V$):
\begin{enumerate}
\item 正定型:\begin{equation}\label{DeQua_eq1}
q(\bvec x)=\sum_{i=1}^n x_i^2;
\end{equation}
\item 负定型:\begin{equation}
q(\bvec x)=-\sum_{i=1}^n x_i^2;
\end{equation}
\item 半正定型:\begin{equation}
q(\bvec x)=\sum_{i=1}^r x_i^2,\quad r\leq n;
\end{equation}
\item 不定型:\begin{equation}
q(\bvec x)=\sum_{i=1}^s x_i^2-\sum_{i=s+1}^r x_i^2,\quad0<s< r.
\end{equation}
\end{enumerate}
\begin{definition}{正定双线性型}
与正定二次型相配极的双线性型\upref{QuaFor}称为\textbf{正定的}.
\end{definition}
类似的术语同样可照搬到矩阵上,因为二次型对应一个与之配极的双线性型,双线性型又对应一个矩阵,它们之间这样一一对应的关系使得术语可照搬.
\begin{theorem}{}
矩阵 $F$ 是正定矩阵的充要条件为
\begin{equation}
F=A^TA
\end{equation}
其中,$A$ 是实的非退化矩阵.
\end{theorem}
\textbf{证明:}
\begin{enumerate}
\item \textbf{必要性:}
因为正定矩阵的标准型为单位矩阵 $E$,即在某基底下,正定矩阵 $F$ 化为 $E$,设这两基底对应的过渡矩阵为 $B$ (这显然是个非退化矩阵,因为两基底可相互表示),于是
\begin{equation}
B^TFB=E\quad\Rightarrow\quad F={(B^T)}^{-1}EB^{-1}={(B^{-1})}^TB^{-1}
\end{equation}
令 $A=B^{-1}$,便得 $F=A^TA$.
\item \textbf{充分性:}因为 $F=A^TA=A^TEA$,而 $A$ 非退化,所以
\begin{equation}
{(A^{-1})}^TFA^{-1}=E
\end{equation}
即在过渡矩阵 $A^{-1}$ 下,矩阵 $F$ 化为 $E$,由\autoref{DeQua_eq1} ,可知 $F$ 正定.
\end{enumerate}
\textbf{证毕!}
\subsection{雅可比方法}
\begin{definition}{顺序主子式}
\begin{equation}
\Delta_1=f_{11},\;\cdots,\;\Delta_i=\begin{vmatrix}
f_{11}&\cdots&f_{1k}\\
\vdots&\vdots&\vdots\\
f_{i1}&\cdots&f{ii}
\end{vmatrix},\quad
\cdots
\end{equation}
称为矩阵 $F=(f_{ij})$的\textbf{顺序主子式}.$\Delta_i$ 称为\textbf{ $i$ 阶顺序主子式}.
\end{definition}