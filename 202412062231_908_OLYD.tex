% 欧拉运动定律(综述)
% license CCBYSA3
% type Wiki

本文根据 CC-BY-SA 协议转载翻译自维基百科\href{https://en.wikipedia.org/wiki/Euler\%27s_laws_of_motion}{相关文章}。

在经典力学中,欧拉运动定律是将牛顿针对质点的运动定律扩展到刚体运动的运动方程。这些定律由莱昂哈德·欧拉在艾萨克·牛顿提出其运动定律约50年后提出。

\subsection{概述}  
\subsubsection{欧拉第一定律}  
欧拉第一定律指出,刚体的线动量 \( p \) 的变化率等于作用在刚体上的所有外力 \( F_{\text{ext}} \) 的合力:[2]  
\[
F_{\text{ext}} = \frac{d\mathbf{p}}{dt}~
\]  
构成刚体的粒子之间的内力不会改变刚体的动量,因为它们互相作用并产生相等且相反的力,最终没有净效应。[3]  

刚体的线动量是刚体质量与其质心速度 \( v_{\text{cm}} \) 的乘积。[1][4][5]