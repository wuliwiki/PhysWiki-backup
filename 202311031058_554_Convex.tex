% 凸函数(补)
% keys 凸集合;凸函数
% license Usr
% type Tutor

\begin{definition}{凸集}
对于一个集合$C\subseteq\mathbb{R}^n,\forall\boldsymbol{x}_1,\boldsymbol{x}_2,\theta\in[0,1],$满足
\begin{equation}
\theta\boldsymbol{x}_1+(1-\theta)\boldsymbol{x}_2\in C~
\end{equation}
称$C$是凸集.
\end{definition}
\begin{definition}{凸集}
对于一个集合$C\subseteq\mathbb{R}^n,\forall\boldsymbol{x}_1,\cdot,\boldsymbol{x}_n,\theta_k\in[0,1],k=1,2,\cdots,n$满足
\begin{equation}
\theta_1\boldsymbol{x}_1+\cdots+\theta_n\boldsymbol{x}_n\in C~
\end{equation}
称$C$是凸集.
\end{definition}
\begin{definition}{凸集}
对于一个集合$C\subseteq\mathbb{R}^n,$存在映射$p:\mathbb{R}^n\to\mathbb{R},p(\boldsymbol{x}),\int_Cp(\boldsymbol{x})\mathrm{d}\boldsymbol{x}=1,\forall\boldsymbol{x}\in C,$满足
\begin{equation}
\int_C\boldsymbol{x}p(\boldsymbol{x})\mathrm{d}\boldsymbol{x}\in C~
\end{equation}
称$C$是凸集.
\end{definition}
\begin{definition}{凸函数}
存在映射$f:\mathbb{R}^n\to\mathbb{R},\mathrm{dom} f$是凸集合,如果$\forall\boldsymbol{x_1},\boldsymbol{x_2}\in\mathrm{dom} f, \theta\in[0,1],$满足
\begin{equation}
f(\theta\boldsymbol{x}_1+(1-\theta)\boldsymbol{x}_2)\leqslant\theta f(\boldsymbol{x}_1) + (1-\theta)f(\boldsymbol{x}_2)~
\end{equation}
\end{definition}
称映射$f$是定义在$\mathrm{dom} f$上的凸函数.如果满足
\begin{equation}
f(\theta\boldsymbol{x}_1+(1-\theta)\boldsymbol{x}_2)<\theta f(\boldsymbol{x}_1) + (1-\theta)f(\boldsymbol{x}_2)~
\end{equation}
称映射$f$是定义在$\mathrm{dom} f$上的强凸函数.
\begin{definition}{凹函数}
存在映射$f:\mathbb{R}^n\to\mathbb{R},\mathrm{dom} f$是凸集合,如果$\forall\boldsymbol{x_1},\boldsymbol{x_2}\in\mathrm{dom} f, \theta\in[0,1],$满足
\begin{equation}
f(\theta\boldsymbol{x}_1+(1-\theta)\boldsymbol{x}_2)\geqslant\theta f(\boldsymbol{x}_1) + (1-\theta)f(\boldsymbol{x}_2)~
\end{equation}
\end{definition}
称映射$f$是定义在$\mathrm{dom} f$上的凹函数.如果满足
\begin{equation}
f(\theta\boldsymbol{x}_1+(1-\theta)\boldsymbol{x}_2)>\theta f(\boldsymbol{x}_1) + (1-\theta)f(\boldsymbol{x}_2)~
\end{equation}
称映射$f$是定义在$\mathrm{dom} f$上的强凹函数.
\begin{theorem}{凸函数判定定理1}
$f:\mathbb{R}^n\to\mathbb{R}$是凸函数,当且仅当$g:\mathbb{R}\to\mathbb{R},g(t)=f(\boldsymbol{x}+t\boldsymbol{v}),t=\{t|\boldsymbol{x}+t\boldsymbol{v}\in\mathrm{dom} f,\boldsymbol{v}\in\mathbb{R}^n\}$是凸函数.
\end{theorem}
\begin{theorem}{凸函数判定定理2·一阶条件}
$f:\mathbb{R}^n\to\mathbb{R}$是凸函数,当且仅当$g:\mathbb{R}\to\mathbb{R},g(t)=f(\boldsymbol{x}+t\boldsymbol{v}),t=\{t|\boldsymbol{x}+t\boldsymbol{v}\in\mathrm{dom} f,\boldsymbol{v}\in\mathbb{R}^n\}$是凸函数.
\end{theorem}