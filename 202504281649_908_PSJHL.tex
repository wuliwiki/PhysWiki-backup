% 朴素集合论(综述)
% license CCBYSA3
% type Wiki

本文根据 CC-BY-SA 协议转载翻译自维基百科\href{https://en.wikipedia.org/wiki/Naive_set_theory}{相关文章}。

朴素集合论是用于讨论数学基础的几种集合论之一。\(^\text{[3]}\)与公理化集合论不同,公理化集合论是使用形式逻辑定义的,而朴素集合论则是以自然语言非正式地定义的。它描述了离散数学中常见的数学集合的各个方面(例如维恩图和关于其布尔代数的符号推理),并且足以满足当代数学中集合论概念的日常使用。\(^\text{[4]}\)

集合在数学中具有重要意义;在现代形式化处理方式中,大多数数学对象(如数字、关系、函数等)都是通过集合来定义的。朴素集合论足以应对许多目的,同时也为更为正式的处理方法提供了一个基础。
\subsection{方法}  
在“朴素集合论”的意义上,朴素理论是一种非形式化的理论,即使用自然语言描述集合及其运算的理论。这种理论将集合视为柏拉图式的绝对对象。诸如“和”、“或”、“如果...那么”、“非”、“存在某个”、“对每个”之类的词汇,按照普通数学中的用法处理。出于方便,即使在更高级的数学中,朴素集合论及其形式化方法仍然占主导地位——包括在集合论本身更为正式的设置中。

集合论的最初发展是朴素集合论。它是在19世纪末由乔治·康托尔创立的,作为他研究无限集合的一部分\(^\text{[5]}\),并由戈特洛布·弗雷格在他的《算术基础》一书中进一步发展。

朴素集合论可能指代几个非常不同的概念。它可以指:
\begin{itemize}
\item 公理化集合论的非正式呈现,例如保罗·哈尔莫斯的《朴素集合论》。
\item 乔治·康托尔早期或后期的理论以及其他非正式系统。
\item 明确不一致的理论(无论是否是公理化的),例如戈特洛布·弗雷格的理论\(^\text{[6]}\),该理论导致了罗素悖论,以及朱塞佩·皮亚诺\(^\text{[7]}\)和理查德·德德金德的理论。
\end{itemize}
\subsubsection{悖论}  
假设任何属性都可以用来形成集合,且没有限制,这会导致悖论。一个常见的例子是罗素悖论:不存在一个由“所有不包含自身的集合”组成的集合。因此,朴素集合论的一致系统必须包含对可以用来形成集合的原则的某些限制。
\subsection{康托尔的理论}  
有些人认为乔治·康托尔的集合论实际上并未涉及集合论悖论(参见Frápolli 1991)。确定这一点的一个困难在于康托尔没有提供他系统的公理化。到1899年,康托尔已经意识到从无约束解释他的理论会导致一些悖论,例如康托尔悖论\(^\text{[8]}\)和布拉利-福尔蒂悖论\(^\text{[9]}\),但他并不认为这些悖论会使他的理论失效。\(^\text{[10]}\)实际上,康托尔悖论可以从上述(错误的)假设中推导出来——即任何属性\(P(x)\)都可以用来形成集合——这里的\(P(x)\)可以是“x是一个基数”。弗雷格明确地对一个理论进行了公理化,该理论中可以解释朴素集合论的形式化版本,正是这个形式化理论在伯特兰·罗素提出悖论时所处理的,而不一定是康托尔所想的理论——如前所述,康托尔意识到几个悖论的存在。
\subsubsection{公理化理论}  
公理化集合论是为了回应这些早期理解集合的尝试而发展起来的,目的是精确定义哪些操作是允许的,以及何时允许。
\subsubsection{一致性}  
朴素集合论不一定是不一致的,只要它正确地规定了可以考虑的集合。这可以通过定义来完成,定义是隐含的公理。所有公理可以明确地陈述,就像哈尔莫斯的《朴素集合论》一样,实际上这是对通常的公理化泽梅洛-弗兰克尔集合论的非正式呈现。它被称为“朴素”的原因是其语言和符号是普通非正式数学中的用法,而且它不涉及公理系统的一致性或完备性。

同样,公理化集合论也不一定是一致的:不一定是没有悖论的。根据哥德尔的不完备性定理,一个足够复杂的一阶逻辑系统(包括大多数常见的公理化集合论)不能在理论内部被证明是一致的——即使它实际上是一致的。然而,通常认为常见的公理化系统是一致的;通过它们的公理,它们确实排除了某些悖论,如罗素悖论。基于哥德尔的定理,我们只是不知道——而且永远也无法知道——这些理论或任何一阶集合论中是否没有悖论。

“朴素集合论”这个术语今天仍然在一些文献中使用\(^\text{[11]}\),指的是弗雷格和康托尔研究的集合论,而不是现代公理化集合论的非正式对应物。
\subsubsection{实用性}  
选择公理化方法与其他方法之间的差异在很大程度上是出于方便。在日常数学中,最佳选择可能是非正式地使用公理化集合论。通常,只有在传统要求时,才会提到特定的公理,例如,选择公理在使用时常常被提到。同样,正式的证明只有在特殊情况下才会出现。这种公理化集合论的非正式使用(取决于符号)可以呈现出与下文所述的朴素集合论相同的外观。与严格的形式化方法相比,它在阅读和写作上更为简便(在大多数陈述、证明和讨论中),并且比严格形式化的方法更不容易出错。
\subsection{集合、成员关系和等式}
在朴素集合论中,集合被描述为一个明确定义的对象集合。这些对象被称为集合的元素或成员。对象可以是任何事物:数字、人、其他集合等。例如,4是所有偶整数集合的成员。显然,偶数集合是无限大的;并没有要求集合必须是有限的。
\begin{figure}[ht]
\centering
\includegraphics[width=6cm]{./figures/a0b50c3cf3b9f84d.png}
\caption{乔治·康托尔原始集合定义的段落} \label{fig_PSJHL_1}
\end{figure}
集合的定义可以追溯到乔治·康托尔。他在1915年的文章《超无限集合论的基础》中写道:

"Unter einer 'Menge' verstehen wir jede Zusammenfassung M von bestimmten wohlunterschiedenen Objekten unserer Anschauung oder unseres Denkens (welche die 'Elemente' von M genannt werden) zu einem Ganzen."

— 乔治·康托尔

集合是我们感知或思维中一组确定且彼此不同的对象的整体,这些对象被称为集合的元素。

— 乔治·康托尔
\subsubsection{关于一致性的说明}  
\begin{figure}[ht]
\centering
\includegraphics[width=6cm]{./figures/47e8e4e33d9a79c7.png}
\caption{吉useppe·皮亚诺在《算术原理新方法展示》(Arithmetices principia nova methodo exposita)一书中首次使用符号ϵ} \label{fig_PSJHL_2}
\end{figure}
从这个定义中并不能推导出集合是如何形成的,以及对集合进行哪些操作会再次产生集合。在“明确定义的对象集合”中的“明确定义”一词本身无法保证集合的确切构成和哪些不是集合的一致性和明确性。尝试实现这一点将属于公理化集合论或公理化类论的领域。

在此背景下,非正式表述的集合论问题在于,它们并未源自(也不隐含)任何特定的公理化理论,因此可能会有多个广泛不同的形式化版本,这些版本有不同的集合和不同的规则来形成新集合,而这些版本都符合原始的非正式定义。例如,康托尔的逐字定义允许在什么构成集合方面有相当大的自由。另一方面,不太可能康托尔特别关心包含猫和狗的集合,而只是关心包含纯数学对象的集合。这样的集合类的一个例子可以是冯·诺依曼宇宙。但即使确定了考虑中的集合类,在不引入悖论的情况下,哪些集合形成规则是允许的,仍然不是很清楚。

为了固定下面的讨论,“明确定义”一词应被解释为一种意图,带有隐式或显式的规则(公理或定义),以排除不一致性。目的是将一致性方面常常深奥且复杂的问题与通常更简单的上下文分开讨论。由于哥德尔第二不完备定理,即使是公理化集合论也无法明确排除所有可能的不一致性(悖论),因此这并不会妨碍朴素集合论在下面考虑的简单上下文中的实用性。它只是简化了讨论。除非特别提及,一致性从此被视为理所当然。
\subsubsection{成员关系} 
如果x是集合A的成员,那么也可以说x属于A,或者x在A中。这用x ∈ A表示。符号∈是从希腊字母epsilon("ε")派生而来,由吉useppe·皮亚诺在1889年引入,它是希腊语单词ἐστί(意思是“是”)的第一个字母。符号∉常用来表示x ∉ A,意思是“x不在A中”。
\subsubsection{等式}  
当两个集合A和B具有完全相同的元素时,它们被定义为相等的;也就是说,如果A的每个元素都是B的元素,并且B的每个元素也是A的元素。(参见外延公理。)因此,集合完全由其元素确定;描述是无关紧要的。例如,元素为2、3和5的集合等于所有小于6的素数集合。如果集合A和B相等,用符号A = B表示(如通常所示)。
\subsubsection{空集合}  
空集合,表示为∅,有时表示为{},是一个没有任何成员的集合。因为集合完全由其元素确定,所以只能有一个空集合。(参见空集合公理。)\(^\text{[12]}\) 虽然空集合没有成员,但它可以是其他集合的成员。因此,∅ ≠ {∅},因为前者没有成员,而后者有一个成员。\(^\text{[13]}\)
\subsection{指定集合}  
描述集合的最简单方式是将其元素列在大括号中(这被称为外延地定义集合)。因此,{1, 2}表示一个集合,它的唯一元素是1和2。(参见配对公理。)注意以下几点:

元素的顺序无关紧要;例如,{1, 2} = {2, 1}。
元素的重复(多重性)无关紧要;例如,{1, 2, 2} = {1, 1, 1, 2} = {1, 2}。
(这些是前一节中等式定义的结果。)