% 一阶隐式常微分方程
% keys 隐式方程|ODE|differential euqation

\pentry{一阶常微分方程解法:常数变易法\upref{ODEa2},一阶常微分方程解法:恰当方程\upref{ODEa3}}

我们之前讨论的常微分方程都是显式写出导函数表达式的,即 $\frac{\dd y}{\dd x}=F(x, y)$ 的形式。很多时候,一阶微分方程常被写为 $F(x, y, \frac{\dd y}{\dd x})=0$ 的形式,如果这样的方程可以被改写为显式的形式,那么我们就可以尝试用预备知识中介绍过的方法来解方程;但如果难以改写或者解出来的形式极为复杂,那我们也可以尝试\textbf{换元}的方法。

本节介绍四种一阶隐式方程和它们的换元方法。

\subsection{第一种}

第一个要讨论的是形如
\begin{equation}\label{eq_ODEa4_2}
y=f(t, \frac{\dd y}{\dd t})~
\end{equation}
的方程。这里自变量用的是通常代表时间的 $t$,为的是提示该怎么换元——如果 $y$ 是位移,那 $\dd y/\dd t$ 就是速度,这就是我们要的变换。

令 $v=\frac{\dd y}{\dd t}$,则原方程变为 $y=f(t, v)$。在方程两边同时对 $t$ 求导,得到
\begin{equation}\label{eq_ODEa4_1}
v=\frac{\partial f(t, v)}{\partial t}+\frac{\partial f(t, v)}{\partial v}\frac{\dd v}{\dd t}~.
\end{equation}

\autoref{eq_ODEa4_1} 就是一个关于 $t, v$ 的一阶微分方程,用我们之前讨论过的方法就可以解出,再将解出的 $v$ 代回\autoref{eq_ODEa4_2} 即可得到原方程的通解。

\begin{example}{}\label{ex_ODEa4_2}


考虑方程
\begin{equation}\label{eq_ODEa4_3}
y=\qty(\frac{\dd y}{\dd x})^2+2x\frac{\dd y}{\dd x}~.
\end{equation}
令 $v=\frac{\dd y}{\dd x}$,代入\autoref{eq_ODEa4_3} ,并两端对 $x$ 求导,则\autoref{eq_ODEa4_3} 化为
\begin{equation}
v=2v\frac{\dd v}{\dd x}+2v+2x\frac{\dd v}{\dd x}~,
\end{equation}
整理一下,得
\begin{equation}
v\dd x+(2v+2x)\dd v=0~.
\end{equation}
这不是一个恰当方程\footnote{$\frac{\frac{\partial (2v+2x)}{\partial x}-\frac{\partial v}{\partial v}}{v}=\frac{1}{v}$ 是 $v$ 的函数,因此我们可以为它找到一个积分因子 $f(v)$。},不过我们可以给它添加一个积分因子 $f(v)=\E^{\int 1/v \dd v}=v$,把它变成一个恰当方程
\begin{equation}
v^2\dd x+(2v^2+2xv)\dd v=0~.
\end{equation}

令 $u(x, v)=v^2x+\frac{2}{3}v^3$,那么 $\dd u=v^2\dd x+(2v^2+2xv)\dd v$。

因此,\autoref{eq_ODEa4_3} 的通解为 $u=C$,即
\begin{equation}
\leftgroup{
    \begin{aligned}
    v^2x+\frac{2}{3}v^3&=C\\
    v^2+2xv&=y~.
    \end{aligned}
}
\end{equation}



\end{example}


\begin{example}{}\label{ex_ODEa4_1}
考虑方程
\begin{equation}\label{eq_ODEa4_7}
5\qty(\frac{\dd y}{\dd x})^2+5x\frac{\dd y}{\dd x}+x^2=y~.
\end{equation}
令 $v=\dd y/\dd x$,两边对 $x$ 求导,则原方程化为
\begin{equation}\label{eq_ODEa4_8}
10v\frac{\dd v}{\dd x}+5v+5x\frac{\dd v}{\dd x}+2x=v~,
\end{equation}
整理得
\begin{equation}\label{eq_ODEa4_4}
10v\frac{\dd v}{\dd x}+5x\frac{\dd v}{\dd x}+4v+2x=0~.
\end{equation}

\autoref{eq_ODEa4_4} 还可以进一步整理为
\begin{equation}
(\frac{5}{2}\frac{\dd v}{\dd x}+1)(4v+2x)=0~.
\end{equation}

由 $\frac{5}{2}\frac{\dd v}{\dd x}+1=0$ 得通解
\begin{equation}\label{eq_ODEa4_9}
v=-\frac{2}{5}x+C~.
\end{equation}
代入 $v=\dd y/\dd x$ 和\autoref{eq_ODEa4_7} 就得到原方程的\textbf{第一个通解}
\begin{equation}\label{eq_ODEa4_5}
y=-\frac{1}{5}x^2+C_1x+5C_1^2~,
\end{equation}
其中 $C_1$ 是积分常数。

但这个方程\textbf{还有一个特解}:取 $4v+2x=0$,再代入 $v=\dd y/\dd x$ 和\autoref{eq_ODEa4_7} ,得到
\begin{equation}\label{eq_ODEa4_6}
y=-\frac{1}{4}x^2~,
\end{equation}
因此,\autoref{eq_ODEa4_5} 和\autoref{eq_ODEa4_6} 都是\autoref{eq_ODEa4_7} 的通解。

% 出题的思路草稿
% % $(A\frac{\dd v}{\dd x}+B)(Cv+Dx)=(ACv\frac{\dd v}{\dd x}+ADx\frac{\dd v}{\dd x}+BCv+BDx)$

% % \begin{equation}
% % \leftgroup{
% %     AC=10\\
% %     AD=5\\
% %     BC=4\\
% %     BD=2\\
% % }
% % \end{equation}

% % $\frac{2X}{Y}=\frac{Y-1}{2}$
% 此处 $X, Y$ 是原方程前两个的系数,$2X/Y=AC/AD=BC/BD=(Y-q)/2$

\end{example}

\autoref{ex_ODEa4_1} 较为复杂,我们在这里做一点补充。

整个\autoref{ex_ODEa4_1} 的求解思路,是首先作变量代换,去解\autoref{eq_ODEa4_8} ,其结果就是\autoref{eq_ODEa4_9} 和 $4p+2=0$。但这是\autoref{eq_ODEa4_8} 的解,由于求导会把一些常数项消掉,\autoref{eq_ODEa4_8} 的解会比变量代换前的\autoref{eq_ODEa4_7} 多一些,我们还是得代回\autoref{eq_ODEa4_7} 看看该怎么约束。

实际上在解答过程中,\autoref{eq_ODEa4_5} 是先写为 $y=-\frac{1}{5}x^2+C_1x+C_2$ 的,有两个待定常数。这是\autoref{eq_ODEa4_8} 的解。我们把它代回\autoref{eq_ODEa4_7} ,计算后发现 $C_2=5C_1^2$,因此最终写成了\autoref{eq_ODEa4_5} 的形式。

如果把\autoref{eq_ODEa4_5} 和\autoref{eq_ODEa4_6} 的图像画出来,我们会发现,\autoref{eq_ODEa4_6} 内切于每一条\autoref{eq_ODEa4_5} 
。\autoref{eq_ODEa4_6} 被称为一个\textbf{奇解},其图像也被称为\textbf{包络线包络和奇解\upref{EnvSol}}。



\addTODO{添加展示包络线的gif。}





\subsection{第二种}

第二个是形如
\begin{equation}\label{eq_ODEa4_10}
x=f(y, \frac{\dd y}{\dd x})~
\end{equation}
的方程。

同样令 $v=\frac{\dd y}{\dd x}$,代入\autoref{eq_ODEa4_10} 并两边同时对 $y$ 求导,得到
\begin{equation}
\frac{1}{v}=\frac{\partial f}{\partial y}+\frac{\partial f}{\partial v}\frac{\dd v}{\dd y}~.
\end{equation}

整理一下得
\begin{equation}\label{eq_ODEa4_13}
\frac{\partial f}{\partial v}\dd v +\qty(\frac{\partial f}{\partial y}-\frac{1}{v})\dd y=0~,
\end{equation}

我们就可以尝试用之前的办法来解。

\begin{example}{}
考虑方程
\begin{equation}\label{eq_ODEa4_11}
\qty(\frac{\dd y}{\dd x})^2+x\frac{\dd y}{\dd x}+2y=0~.
\end{equation}

令 $v=\frac{\dd y}{\dd x}$,代入\autoref{eq_ODEa4_11} ,整理得
\begin{equation}\label{eq_ODEa4_12}
x=\frac{-v^2-2y}{v}=-v-\frac{2y}{v}~.
\end{equation}



两边关于 $y$ 求导,整理得(或者直接把\autoref{eq_ODEa4_12} 代入\autoref{eq_ODEa4_13} 得)
\begin{equation}\label{eq_ODEa4_14}
\qty(-1+\frac{2y}{v^2})\dd v+\qty(-\frac{3}{v})\dd y=0~.
\end{equation}

按照在\textbf{一阶常微分方程解法:恰当方程}\upref{ODEa3}中讨论的方法,\autoref{eq_ODEa4_14} 有一个积分因子 $h(v)=v^{1/3}$,从而将\autoref{eq_ODEa4_14} 化为
\begin{equation}
\qty(-v^{1/3}+\frac{2y}{v^{5/3}})\dd v+\qty(-\frac{3}{v^{2/3}})\dd y=0~,
\end{equation}
即
\begin{equation}
\dd (-\frac{3}{4}v^{4/3}-\frac{3y}{v^{2/3}})=0~,
\end{equation}
也即\footnote{注意这一步跳步了,约掉了一个 $3$。}
\begin{equation}
-\frac{1}{4}v^{4/3}-\frac{y}{v^{2/3}}=C~,
\end{equation}

整理得
\begin{equation}\label{eq_ODEa4_15}
y=-\frac{1}{4}v^2-Cv^{2/3}~.
\end{equation}

利用 $v=\frac{\dd y}{\dd x}$,再代回\autoref{eq_ODEa4_11} ,得到
\begin{equation}\label{eq_ODEa4_16}
x=-\frac{1}{2}v+Kv^{-1/3}~,
\end{equation}
其中 $K=2C$ 是常数。

\autoref{eq_ODEa4_15} 和\autoref{eq_ODEa4_16} 就是以 $v$ 为参数的\autoref{eq_ODEa4_11} 的通解。



\end{example}



\subsection{第三种}

形如 $F(x, \frac{\dd y}{\dd x})=0$ 的方程。

同样令 $v=\frac{\dd y}{\dd x}$,我们发现 $F(x, v)=0$ 是 $Oxv$ 平面上的一条曲线。用参数 $t$ 来表示这条曲线:
\begin{equation}\label{eq_ODEa4_17}
\leftgroup{
    x=\varphi(t)\\
    v=\phi(t)~.
}
\end{equation}

那么 $x$ 相当于已经解出来了,接下来只需要考虑 $y$ 怎么用参数 $t$ 表示。

由于 $v\dd x=\dd y$ 恒成立,代入\autoref{eq_ODEa4_17} 后可得
\begin{equation}
\phi(t)\varphi'(t)\dd t=\dd y~,
\end{equation}
即
\begin{equation}\label{eq_ODEa4_18}
y=\int \phi(t)\varphi'(t)\dd t+C~.
\end{equation}

这样,\autoref{eq_ODEa4_17} 和\autoref{eq_ODEa4_18} 结合,就能得到参数形式的解。

\begin{example}{}
考虑方程
\begin{equation}
x^2-\qty(\frac{\dd y}{\dd x})^3+x\frac{\dd y}{\dd x}=0~.
\end{equation}
令 $v=\frac{\dd y}{\dd x}$ 得
\begin{equation}\label{eq_ODEa4_19}
x^2-v^3+xv=0~.
\end{equation}
设 $v=tx$,那么\autoref{eq_ODEa4_19} 化为
\begin{equation}
-t^3x^3+(1-t)x^2=0~,
\end{equation}
即
\begin{equation}\label{eq_ODEa4_20}
x^2\qty(t^3x+t-1)=0~.
\end{equation}

由\autoref{eq_ODEa4_19} 可知 $x$ 不恒为 $0$,因此\autoref{eq_ODEa4_20} 可以用参数 $t$ 解出 $x$:
\begin{equation}\label{eq_ODEa4_21}
x=\frac{1-t}{t^3}~.
\end{equation}

又因为
\begin{equation}
v=tx=\frac{1-t}{t^2}~
\end{equation}
和
\begin{equation}
\frac{\dd x}{\dd t}=\frac{2t-3}{t^4}~,
\end{equation}

故
\begin{equation}\label{eq_ODEa4_22}
y=\int \frac{1-t}{t^2}\cdot\frac{2t-3}{t^4}\dd t=\frac{40t^2-75t+36}{60t^5}+C~.
\end{equation}

这样,\autoref{eq_ODEa4_21} 和\autoref{eq_ODEa4_22} 就构成一组参数解。



\end{example}



\subsection{第四种}

形如 $F(y, \frac{\dd y}{\dd x})=0$ 的方程。

解法类似第三种。令 $v=\frac{\dd y}{\dd x}$,将 $F(y, v)=0$ 写成参数曲线的形式:
\begin{equation}\label{eq_ODEa4_23}
\leftgroup{
    y=\varphi(t)\\
    v=\phi(t)~.
}
\end{equation}

那么由 $\dd x=\frac{\dd y}{v}$,可知
\begin{equation}\label{eq_ODEa4_24}
x=\int \frac{\varphi'(t)}{\phi(t)}\dd t+C~.
\end{equation}

这样,\autoref{eq_ODEa4_23} 和\autoref{eq_ODEa4_24} 就构成一组参数解。

\begin{example}{}
考虑方程
\begin{equation}\label{eq_ODEa4_25}
\qty(\frac{\dd y}{\dd x})^2-y\frac{\dd y}{\dd x}+y^3=0~.
\end{equation}

令 $v=\frac{\dd y}{\dd x}$,\autoref{eq_ODEa4_25} 化为
\begin{equation}\label{eq_ODEa4_26}
v^2-vy+y^3=0~.
\end{equation}

令 $y=tv$,那么\autoref{eq_ODEa4_26} 化为
\begin{equation}\label{eq_ODEa4_27}
t^3v^3+(1-t)v^2=0~.
\end{equation}

由\autoref{eq_ODEa4_26} ,$v$ 不恒为 $0$,因此\autoref{eq_ODEa4_27} 等价于
\begin{equation}
v=\frac{t-1}{t^3}~.
\end{equation}

于是
\begin{equation}\label{eq_ODEa4_28}
y=tv=\frac{t-1}{t^2}~,
\end{equation}

由于 $\dd x=\frac{\dd y}{v}$,故
\begin{equation}\label{eq_ODEa4_29}
\begin{aligned}
x
&=\int \frac{1}{v} \dd v+C\\
&=\int \frac{t^3}{t-1}\cdot\frac{2-t}{t^3}\dd t+C\\
&=\ln\abs{t-1}-t+C~.
\end{aligned}
\end{equation}

\autoref{eq_ODEa4_28} 和\autoref{eq_ODEa4_29} 就构成一组参数解。








\end{example}











