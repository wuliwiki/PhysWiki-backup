% 有磁介质时的安培环路定理

\pentry{安培环路定理\upref{AmpLaw},磁化强度\upref{MaInte}}

当电流的磁场中有磁介质存在时, 空间任一点的磁感应强度$\mathbf B $等于导线中传导电流所激发的磁场与磁介质磁化后磁化电流所激发的附加磁场的矢量和,这时安培环路定理应写成
\begin{equation} \label{MaAmpe_eq1}
\oint \mathbf{B} \cdot \mathrm{d} \mathbf{l}=\mu_{0}\left(\sum I+I_{s}\right)
\end{equation}

等式右边的两项电流是穿过以回路为边界的任一曲面的总电流,即传导电流$\sum I$和磁化电流$I_s$的代数和.一般说来, $I $是可以测量的,可认为它是已知的;而$I_s$不能事先给定,也无法直接测量,它依赖于介质磁化的情况,而介质的磁化情况又依赖于磁介质中的磁感应强度$\mathbf B$,因此直接求解方程\autoref{MaAmpe_eq1}很复杂.为了解决这一困难,我们利用关系式\autoref{MaInte_eq1}\upref{MaInte}将\autoref{MaAmpe_eq1}改写成
\begin{equation}
\oint \mathbf{B} \cdot \mathrm{d} \mathbf{l}=\mu_{0}\left(\sum I+\oint \mathbf{M} \cdot \mathrm{d} \mathbf{l}\right)
\end{equation}
或
\begin{equation}
\oint\left(\frac{\mathbf{B}}{\mu_{0}}-\mathbf{M}\right) \cdot \mathrm{d} \mathbf{l}=\sum I
\end{equation}
然后引进一个新的物理量,称为磁场强度(magnetic intensity),用符号$\mathbf H$表示,定义为
\begin{equation} \label{MaAmpe_eq2}
\mathbf{H}=\frac{\mathbf{B}}{\mathbf{\mu}_{0}}-\mathbf{M}
\end{equation}
这样,便有下列简单的形式:
\begin{equation} \label{MaAmpe_eq3}
\oint \mathbf{H} \cdot \mathrm{d} \mathbf{l}=\sum I
\end{equation}

\autoref{MaAmpe_eq3}称为有磁介质时的安培环路定理.它表明$\mathbf H $矢量的环流只和传导电流$I $有关,而在形式上与磁介质的磁性无关.因此引入磁场强度$\mathbf H $这个物理量以后,在磁场分布具有高度对称性时,能够使我们比较方便地处理有磁介质时的磁场问题,安培环路定理和静磁场的另一普遍规律一磁场中的高斯定理一起,是处理静磁场问题的基本定理.

在国际单位制中,$\mathbf  H $的单位是$\rm A/m$.

\autoref{MaAmpe_eq2}表示了磁介质中任一点处磁感应强度B 、磁场强度H 和磁化强度
M 之间的普遍关系,不论磁介质是否均匀,甚至对铁磁性物质都能适用但是,
磁化强度M 不仅和磁介质的性质有关,也和磁介质所在处的磁场有关我们
定义
M-
= H m x
(8-63)
为磁介质的' (magnetic susceptib 小ty) , 是一与磁介质的性质有关的物理
桢.因为M 和H 所用的单位相同,所以磁化率儿, 是单位为1 的址.如果磁介质
是均匀的,则儿,是常址;如果磁介质是不均匀的,则x.. , 是空间位置的函数对千顺
磁质, X.. >0, 磁化强度M 和磁场强度H 的方向相同;对于抗磁质,凡, <0, 磁化强
度M 和磁场强度H 的方向相反式(8-63) 又可写为
M 于H (8-64)
将其代入式(8-61) 中可解得