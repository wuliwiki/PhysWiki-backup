% 机械振动(高中)
% 机械振动|弹簧振子|简谐运动|单摆|共振

\begin{issues}
\issueDraft
\issueTODO
\end{issues}

物体或物体的一部分在某个位置附近所做的往复运动叫做\textbf{机械振动},简称\textbf{振动}.

\subsection{简谐运动}

由弹簧和小球组成的系统,叫做\textbf{弹簧振子},其中的小球叫做\textbf{振子}.弹簧振子是一种理想模型,研究其运动时,小球被视为质点,并忽略弹簧的质量以及运动过程中的阻力.

\begin{figure}[ht]
\centering
\includegraphics[width=5cm]{./figures/HSPM09_1.png}
\caption{弹簧振子} \label{HSPM09_fig1}
\end{figure}

\autoref{HSPM09_fig1} 为安置在光滑水平面的弹簧振子,弹簧的一端被固定.弹簧处于自然状态时,振子静止,所受合力为零,此时振子所处的位置叫\textbf{平衡位置}.

当沿水平方向拉动(或推动)振子使其偏离平衡位置并释放,振子将在平衡位置的两侧做往复运动(振动).振动过程中,振子在竖直方向上所受合力为零,在水平方向上受到弹簧弹力$\bvec F$的作用.弹力$\bvec F$的方向与偏离平衡位置的位移$\bvec x$的方向相反,总是指向平衡位置,其作用是将振子拉回平衡位置,这个力$\bvec F$叫做\textbf{回复力}.以\autoref{HSPM09_fig1} 向右为正方向,根据胡克定律可知:
\begin{equation}\label{HSPM09_eq1}
F=-kx
\end{equation}
式中的负号表示回复力的方向与振子偏离平衡位置的位移方向相反.

物体在满足\autoref{HSPM09_eq1} 的回复力作用下发生的运动,叫做\textbf{简谐运动}(\textbf{简谐振动}),其位移与时间的关系遵循正弦或余弦函数的规律.简谐运动的位移—时间表达式为:
\begin{equation}\label{HSPM09_eq2}
x=A\cos(\omega t + \varphi_0)
\end{equation}
$A$为振幅,表示振子偏离平衡距离的最大距离;$\omega$为角频率,表示简谐运动物体振动的快慢;$\varphi_0$为初相位,表示$t=0$时,简谐运动物体所处的状态.

由\autoref{HSPM09_eq2} 可知简谐运动是一个周期性往复运动.振子经历$A$→$O$→$A'$→$O$→$A$这样一个完整的振动过程,叫\textbf{全振动}.

\begin{figure}[ht]
\centering
\includegraphics[width=7.2cm]{./figures/HSPM09_2.png}
\caption{弹簧振子运动过程的三个特殊位置} \label{HSPM09_fig2}
\end{figure}

\begin{table}[ht]
\centering
\caption{一次全振动的物理量变化规律}\label{HSPM09_tab1}
\begin{tabular}{|c|c|}
\hline
过程或位置 & 物理量变化情况 \\
\hline
$A$ & 位移向右,达最大值;回复力向左,达最大值;加速度向左,达最大值;振子动能为零;弹簧的势能达最大值 \\
\hline
$A$→$O$ & 位移向右,减小;回复力向左,减小;加速度向左,减小;振子速度向左,增大;振子动能增大;弹簧的势能减小 \\
\hline
$O$ & 位移为零;回复力为零;加速度为零;振子速度向左,达最大值,动能达最大值;弹簧的势能为零 \\
\hline
$O$→$A'$ & 位移向左,增大;回复力向右,增大;振子速度向左,减小;振子动能减小; \\
\hline
* & * \\
\hline
* & * \\
\hline
* & * \\
\hline
\end{tabular}
\end{table}