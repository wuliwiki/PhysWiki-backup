% 博苏克-乌拉姆定理(综述)
% license CCBYSA3
% type Wiki

本文根据 CC-BY-SA 协议转载翻译自维基百科\href{https://en.wikipedia.org/wiki/Borsuk\%E2\%80\%93Ulam_theorem}{相关文章}。

在数学中,博苏克–乌拉姆定理指出:每一个从 $n$ 维球面 $S^n$ 到 $n$ 维欧几里得空间 $\mathbb{R}^n$ 的连续函数,必定存在一对对踵点被映射到同一个点。这里,对踵点指的是位于球面上、从球心看方向完全相反的两点。

形式化表述:若$f: S^n \to \mathbb{R}^n$是一个连续函数,则存在$x \in S^n$使得$f(-x) = f(x)$。

特例说明:当 $ n = 1$:这意味着地球赤道上总存在一对正对着的点,它们的温度相同。这个结论也适用于任何圆周。注意这依赖于温度在空间中连续变化这一假设,而这在现实中不一定成立\(^\text{[1]}\)。

当 $ n = 2 $:可以解释为地球表面在任意时刻总存在一对对踵点,它们的温度和气压完全相同(假设这两个物理量都在空间中连续变化)。

与奇函数等价的其他表述:记$ S^n $为 $n$ 维球面,$ B^n $为 $n$维单位球体:
\begin{itemize}
\item 若  $g: S^n \to \mathbb{R}^n$是一个连续奇函数(即 $g(-x) = -g(x)$),则存在$x \in S^n$使得$g(x) = 0$。
\item 若 $g: B^n \to \mathbb{R}^n$是一个连续函数,且在边界 $S^{n-1}$ 上为奇函数,那么必存在$x \in B^n$使得$g(x) = 0$。
\end{itemize}
\subsection{历史}
据 Matoušek(2003年第25页)记载,博苏克–乌拉姆定理最早的历史性表述出现在 Lyusternik 与 Shnirel'man 于1930年的著作中。第一个正式的证明由卡罗尔·博苏克于1933年给出,他将这一问题的提出归功于斯坦尼斯瓦夫·乌拉姆。自那以后,许多作者都给出了该定理的不同证明方式,这些证明被 Steinlein(1985)汇总整理。
\subsection{等价表述}
以下命题与博苏克–乌拉姆定理是等价的\(^\text{[2]}\):
\subsubsection{关于奇函数的表述}
一个函数 $g$ 被称为奇函数(也称为对极函数或保对极函数),如果对于每一个 $x$,都有
$g(-x) = -g(x)$

博苏克–乌拉姆定理等价于以下两个命题:

1. 每一个连续的奇函数 $g : S^n \to \mathbb{R}^n$ 都有零点。\\
2. 不存在从 $S^n$ 到 $S^{n-1}$ 的连续奇函数。\\

证明:博苏克–乌拉姆定理等价于命题 (1)

($\Longrightarrow$) 如果博苏克–乌拉姆定理成立,那么它当然对奇函数也成立。而对于奇函数 $g$,若有$g(-x) = g(x)$则必须有$g(x) = 0$,因为又有 $g(-x) = -g(x)$。所以每个连续奇函数都有零点。

($\Longleftarrow$) 对于任意连续函数 $f : S^n \to \mathbb{R}^n$,可以构造一个连续奇函数$g(x) = f(x) - f(-x)$如果每个奇函数都有零点,那么 $g$ 存在某个零点 $x$,即$f(x) = f(-x)$,从而得出博苏克–乌拉姆定理成立。

为了证明命题 (1) 与命题 (2) 等价,使用以下连续奇函数:

\begin{itemize}
\item 明显的包含映射$i : S^{n-1} \to \mathbb{R}^n \setminus \{0\}$
\item 辐射投影映射$p : \mathbb{R}^n \setminus \{0\} \to S^{n-1}, \quad x \mapsto \frac{x}{|x|}$
\end{itemize}
证明如下:

((1) ⟹ (2))反设成立:若存在连续奇函数 $f : S^n \to S^{n-1}$,则复合映射 $i \circ f : S^n \to \mathbb{R}^n \setminus \{0\}$ 是一个连续奇函数,这与 (1) 每个连续奇函数都有零点相矛盾(因为值域不含零点)。

((1) ⟸ (2))同样使用反设:若存在连续奇函数 $f : S^n \to \mathbb{R}^n \setminus \{0\}$则 $p \circ f : S^n \to S^{n-1}$ 是连续奇函数,这与 (2) 不存在这样的映射相矛盾。因此,两者等价。
\subsection{证明}
\subsubsection{一维情形}
一维情形可以使用中值定理(IVT)轻松证明。

设 $g(x) = f(x) - f(-x)$ 是一个定义在圆上的实值连续奇函数。随便取一个 $x$。若 $g(x) = 0$,则定理成立。否则,**不妨设** $g(x) > 0$。由于 $g$ 是奇函数,有 $g(-x) = -g(x) < 0$。由中值定理知,存在某点 $y \in S^1$ 使得 $g(y) = 0$,从而有$f(y) = f(-y)$成立,证毕。
\subsubsection{一般情形}
\textbf{代数拓扑证明}

假设存在一个奇连续函数$h : S^n \to S^{n-1}$其中 $n > 2$($n=1$ 的情形已在前面证明,$n=2$ 的情形可用基本覆盖空间理论处理)。由于 $h$ 是奇函数,我们可以考虑将它在对极点作用下传递到商空间中,于是得到一个在实射影空间之间的诱导连续映射:
$h' : \mathbb{RP}^n \to \mathbb{RP}^{n-1}$这个映射在基本群之间诱导同构。根据 Hurewicz 定理,它在系数为 $\mathbb{F}_2$(两个元素的有限域)的上同调环之间诱导如下同态:
$$
H^*(\mathbb{RP}^n; \mathbb{F}_2) = \mathbb{F}_2[a]/(a^{n+1}) \leftarrow H^*(\mathbb{RP}^{n-1}; \mathbb{F}_2) = \mathbb{F}_2[b]/(b^n)~
$$
其中 $b \mapsto a$。但这样一来,左侧的 $b^n = 0$ 被映射为 $a^n \neq 0$,矛盾!

因此,假设不成立,也就是说,不存在这样的奇连续函数 $h$,这就证明了 Borsuk–Ulam 定理。

另外,也可以证明一个更强的结论:任何从 $S^{n-1} \to S^{n-1}$ 的奇映射,其度数为奇数。从这个结论也可以推出 Borsuk–Ulam 定理。

\textbf{组合证明}

Borsuk–Ulam 定理可以由 Tucker 引理推导而来\(^\text{[2][4][5]}\)。

设$g : S^n \to \mathbb{R}^n$是一个连续的奇函数。由于 $g$ 在紧空间上连续,它是一致连续的。因此,对于任意 $\epsilon > 0$,存在 $\delta > 0$,使得当 $S^n$ 上任意两点的距离小于 $\delta$ 时,它们在 $g$ 下的像之间的距离也小于 $\epsilon$。

接下来,对 $S^n$ 进行一个边长至多为 $\delta$ 的**三角剖分**。对每个顶点 $v$ 赋予一个标签 $l(v) \in \{\pm 1, \pm 2, \dots, \pm n\}$,具体规则如下:
\begin{itemize}
\item 标签的绝对值是使 $g(v)$ 中某坐标的绝对值最大的那个坐标索引:$|l(v)| = \arg \max_k |g(v)_k|$
\item 标签的符号为该坐标上的符号:$l(v) = \operatorname{sgn}(g(v)_{|l(v)|}) \cdot |l(v)|$
\end{itemize}
由于 $g$ 是奇函数,即满足 $g(-v) = -g(v)$,所以标签满足:$l(-v) = -l(v)$因此这是一个奇标记,可应用 Tucker 引理。根据该引理,存在一对相邻顶点 $u, v$,使得 $l(u) = -l(v)$。不妨设 $l(u) = 1, l(v) = -1$,这说明:在 $g(u)$ 与 $g(v)$ 中,第 1 个坐标为最大坐标分量且 $g(u)_1 > 0$,$g(v)_1 < 0$由于剖分边长不超过 $\delta$,所以 $g(u)$ 和 $g(v)$ 的欧几里得距离小于等于 $\epsilon$,特别地:$|g(u)_1 - g(v)_1| = |g(u)_1| + |g(v)_1| \leq \epsilon$由于 $g(u)_1 > 0, g(v)_1 < 0$,它们符号相反。于是有$|g(u)_1| \leq \epsilon$而第 1 个坐标是最大坐标分量,所以对所有 $k \in \{1, \dots, n\}$,都有$|g(u)_k| \leq \epsilon$因此整向量的范数满足$|g(u)| \leq c_n \epsilon$其中 $c_n$ 是一个仅依赖于 $n$ 和所选范数的常数。由于上面结论对任意 $\epsilon > 0$ 都成立,且 $S^n$ 是紧空间,必然存在某点 $u$,使得$|g(u)| = 0$即$g(u) = 0$从而证明了 Borsuk–Ulam 定理。
\subsection{推论}
\begin{itemize}
\item 没有任何 $\mathbb{R}^n$的子集与 $S^n$ 同胚:即不存在一个 $\mathbb{R}^n$ 中的子集与 $n$ 维球面 $S^n$ 拥有相同的拓扑结构。
\item 火腿三明治定理:对于 $\mathbb{R}^n$ 中任意的 $n$ 个紧致集合 $A_1, A_2, \dots, A_n$,总能找到一个超平面将每个集合划分为测度相等的两部分。
\end{itemize}
\subsubsection{等价命题}
上文已展示如何从 Tucker 引理 推出 Borsuk–Ulam 定理,反过来也是成立的:也可以从 Borsuk–Ulam 定理推出 Tucker 引理。因此,这两个定理是等价的。存在一类不动点定理,它们通常存在三种彼此等价的表述形式:1. 拓扑代数形式2. 组合数学形式3. 集合覆盖形式这些不同形式可以用完全不同的方法分别证明,但它们之间又可以相互归约。此外,每一列中最下方的命题都可以推出其所在列上方的命题\(^\text{[6]}\)。
\begin{figure}[ht]
\centering
\includegraphics[width=14.25cm]{./figures/0d3ac80e97b572b6.png}
\caption{} \label{fig_BSKwlm_1}
\end{figure}
\subsection{推广}
\begin{itemize}
\item 在最初的 Borsuk–Ulam 定理中,函数 $f$ 的定义域是单位$n$ 维球面(即单位 $n$ 维球体的边界)。更一般地,当 $f$ 的定义域是 $\mathbb{R}^n$ 中任何开、有限、对称并包含原点的子集的边界时,该定理仍然成立(这里“对称”是指:如果 $x$ 在该子集中,则 $-x$ 也在其中)\(^\text{[7]}\)。
\item 更进一步地,设 $M$ 是一个紧致的 $n$ 维黎曼流形,若 $f: M \rightarrow \mathbb{R}^n$ 是连续函数,则对于任意给定的 $\delta > 0$,总存在一对点 $x, y \in M$,使得:$f(x) = f(y)$,且$x$ 与 $y$ 之间存在一条长度为 $\delta$ 的测地线连接\(^\text{[8][9]}\)。
\item 现在考虑一个将每个点映射到其对跖点的函数 $A(x) = -x$。注意:$A(A(x)) = x.$原始的 Borsuk–Ulam 定理声称,存在某个点 $x$,使得:$f(A(x)) = f(x).$更一般地,对于所有满足 $A(A(x)) = x$ 的函数 $A$,该定理都成立\(^\text{[10]}\)。然而,对任意函数 $A$ 并不总是成立\(^\text{[11]}\)。
\end{itemize}
\subsection{参见}
\begin{itemize}
\item 拓扑组合学
\item 项链划分问题
\item 火腿三明治定理
\item 角谷定理
\item 伊姆雷·巴拉尼
\end{itemize}
\subsection{注释}
\begin{enumerate}
\item Jha, Aditya;Campbell, Douglas;Montelle, Clemency;Wilson, Phillip L.(2023年7月30日)。“论连续体谬误:温度是连续函数吗?”《物理学基础》Foundations of Physics,第53卷第4期:第69页。Bibcode: 2023FoPh...53...69J。doi:10.1007/s10701-023-00713-x。hdl:1721.1/152272。ISSN 1572-9516。
\item Prescott, Timothy(2002)。《Borsuk–Ulam 定理的扩展》(本科论文)。哈维·马德学院。CiteSeerX: 10.1.1.124.4120。
\item Joseph J. Rotman,《代数拓扑导论》,1988年,施普林格出版社,ISBN 0-387-96678-1(详见第12章)。
\item Freund, Robert M.;Todd, Michael J.(1982)。“Tucker 组合引理的构造性证明”。《组合理论杂志A辑》,第30卷第3期:321–325。doi:10.1016/0097-3165(81)90027-3。
\item Simmons, Forest W.;Su, Francis Edward(2003)。“通过 Borsuk–Ulam 与 Tucker 定理实现共识对半划分”。《数学社会科学》,第45期:15–25。doi:10.1016/s0165-4896(02)00087-2。hdl:10419/94656。
\item Nyman, Kathryn L.;Su, Francis Edward(2013),“一个直接蕴含 Sperner 引理的 Borsuk–Ulam 等价定理”,《美国数学月刊》,第120卷第4期:346–354,doi:10.4169/amer.math.monthly.120.04.346,JSTOR: 10.4169/amer.math.monthly.120.04.346,MR: 3035127。
\item “Borsuk 不动点定理”,《数学百科全书》,EMS Press,2001年 [1994年版本]。
\item Hopf, H.(1944)。“若干已知映射与覆盖定理的推广”(德语原题:“Eine Verallgemeinerung bekannter Abbildungs-und Überdeckungssätze”)。《葡萄牙数学杂志》。
\item Malyutin, A. V.;Shirokov, I. M.(2023)。“关于 f-邻点的 Hopf 型定理”。《西伯利亚电子数学通报》,第20卷第1期:165–182。
\item Yang, Chung-Tao(1954)。“关于 Borsuk–Ulam、Kakutani–Yamabe–Yujobo 与 Dyson 定理”。《数学年刊》,第60卷第2期:262–282。doi:10.2307/1969632。JSTOR: 1969632。
\item Jens Reinhold;Faisal;Sergei Ivanov。“Borsuk–Ulam 的推广”,发表于 Math Overflow,检索时间:2015年5月18日。
\end{enumerate}
\subsection{参考文献}
\begin{itemize}
\item * Borsuk, Karol(1933)。《关于 n 维欧几里得球体的三个定理》(德文)\[*Drei Sätze über die n-dimensionale euklidische Sphäre*]。发表于 *Fundamenta Mathematicae*,第20卷,第177–190页。doi:10.4064/fm-20-1-177-190。(PDF 已归档,存档时间:2022-10-09)

* Lyusternik, Lazar;Shnirel'man, Lev(1930)。《变分问题中的拓扑方法》(俄文)\[*Topological Methods in Variational Problems*]。莫斯科国立大学数学与力学研究所出版。

* Matoušek, Jiří(2003)。《使用 Borsuk–Ulam 定理》(英文)\[*Using the Borsuk–Ulam Theorem*]。柏林:施普林格出版社(Springer Verlag)。doi:10.1007/978-3-540-76649-0。ISBN: 978-3-540-00362-5。

* Steinlein, H.(1985)。《Borsuk 对极定理及其推广与应用综述》。发表于《非线性分析中的拓扑方法》研讨会(Sém. Math. Supér. Montréal, Sém. Sci. OTAN),第95卷,第166–235页。

* Su, Francis Edward(1997年11月)。《Borsuk–Ulam 蕴含 Brouwer 不动点定理:一种直接构造》(PDF)\[*Borsuk-Ulam Implies Brouwer: A Direct Construction*]。发表于《美国数学月刊》(*The American Mathematical Monthly*),第104卷第9期,第855–859页。CiteSeerX: 10.1.1.142.4935。doi:10.2307/2975293。JSTOR: 2975293。原始 PDF 已于2008-10-13归档。访问时间:2006-04-21。
\end{itemize}

\subsection{外部链接}
\begin{itemize}
\item [谁(还)在乎拓扑学?被偷的项链与 Borsuk–Ulam 定理(YouTube 视频)](https://www.youtube.com/results?search_query=borsuk+ulam+necklace)
\item [Borsuk–Ulam 探索器:一个交互式的 Borsuk–Ulam 定理演示平台](https://www.maa.org/press/periodicals/loci/joma/the-borsuk-ulam-explorer)
\end{itemize}