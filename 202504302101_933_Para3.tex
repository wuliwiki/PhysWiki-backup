% 抛物线(高中)
% keys 极坐标系|直角坐标系|圆锥曲线|抛物线
% license Xiao
% type Tutor

\begin{issues}
\issueDraft
\end{issues}

\pentry{解析几何\nref{nod_JXJH},圆\nref{nod_HsCirc},双曲线\nref{nod_Hypb3},点到直线的距离\nref{nod_P2Line}}{nod_7c17}

不知道读者在初次接触双曲线时,是否产生了一种似曾相识的感觉:它的一支看起来与初中阶段学习过的二次函数图像——抛物线,非常相似。二者都不封闭、有一个开口、略微弯曲、向无限延伸,甚至也拥有一条对称轴。相信一些读者可能早已不禁在心中将双曲线的一支等同于抛物线,认为双曲线不过是“两个抛物线”的组合而已。

这种误解并不罕见,难以否认,抛物线与双曲线在形状上确实有些相像。但从本质上看,二者在几何定义、解析式结构以及性质上都有明显区别。之所以容易混淆,很大程度上与初中学习的重点有关。当时的教学更侧重于二次函数的代数表达与图像之间的关系,例如判断开口方向、对称轴位置、顶点坐标、零点特性等。这些内容有助于建立对抛物线的基本印象,但主要停留在函数视角,对抛物线作为几何图形本身的理解较为有限。

大多数人对抛物线的印象,往往停留在现实生活中物体被抛出后所形成的轨迹。在理想状态下,这类轨迹正是一条抛物线,这也正是“抛物线”名称的来源。然而,随着人们的进一步研究发现,抛物线并不仅仅出现在物理运动中,它还具有独特的几何性质,在许多实际工程中发挥着重要作用。例如,雷达天线的反射面通常设计成抛物面结构,原因就在于抛物线具备一种精确的聚焦特性:来自远处的平行电磁波在抛物面上反射后,会准确地汇聚到焦点;而从焦点出发的信号,也能被反射成方向一致的平行波。这一聚焦能力,使抛物面非常适合实现能量的集中与传输,使抛物线广泛出现于雷达、卫星通信设备、汽车大灯以及太阳能灶等场景中。

\subsection{抛物线的定义}

由于初中阶段已经花了很多力气研究二次函数,并且基本了解了抛物线的图像特征,因此此处不再从函数的角度展开,而是直接进入定义的探讨。

在研究椭圆和双曲线的过程中,曾尝试将圆的定义加以推广。圆可以看作是满足条件 $|O_1P| = |O_2P| = r$ 的点$P$的集合,其中 $O_1$ 与 $O_2$ 是重合的点,二者之间的距离 $d(O_1,O_2)=0$。如果放宽这个限制,即允许 $d(O_1,O_2)$ 不为零,并且打开第一个等号,就可以得到椭圆和双曲线这两种新曲线,这在之前已经探究过了。那么,如果不打开第一个等号,而是改为打开第二个等号,只要求 $|O_1P| = |O_2P|$呢?之前提到过,在这种设定下,点 $P$ 的轨迹是所有关于两个定点等距的点,也就是两点连线的垂直平分线。感觉似乎这样修改之后没有什么可改动的空间了。

不过,不妨换个思路,改变一下条件中的几何元素呢?也就是说,保留其中一个点 $O_1$,将另一个点 $O_2$ 替换为一条固定的直线 $l$,那么问题就变成了:考虑所有满足“到某一固定点与到某一固定直线的距离相等”的点 $P$ 的轨迹,会构成怎样的图形?显然,这样修改之后还需要修改$d$,否则如果直线 $l$ 正好通过定点 $O_1$,唯一满足条件的点就只有 $O_1$,轨迹会退化成一个点。当令$d$满足 $d(O_1,l)=p,(p\neq0)$时,情况就变得有趣起来。

\begin{example}{对定点 $F\left(0, \displaystyle\frac{p}{2}\right)$ 和 $l:y=-\displaystyle\frac{p}{2},\left(p>0\right)$,若点$P$满足$|PF|=d(P,l)$,求点 $P$ 的轨迹方程。}\label{ex_Para3_1}
解:

设点 $P$ 的坐标为 $(x, y)$,题设条件表示点 $P$ 到定点 $F$ 的距离等于它到直线 $l$ 的距离。写成数学表达式为:
\begin{equation}
\sqrt{x^2 + \left(y - \frac{p}{2}\right)^2} = \left|y + \frac{p}{2}\right|~.
\end{equation}

两边平方后,展开并整理,利用平方差公式,有:
\begin{equation}\label{eq_Para3_1}
x^2 = \left(y + \frac{p}{2}\right)^2 - \left(y - \frac{p}{2}\right)^2 = 2py~.
\end{equation}
\end{example}

可以注意到,原点正好是定点 $F$ 到直线 $l$ 所作垂线段的中点。将\autoref{eq_Para3_1} 改写为 $y$ 关于 $x$ 的函数形式,可得:
\begin{equation}\label{eq_Para3_2}
y = \frac{1}{2p}x^2~.
\end{equation}

这个式子正是一个开口向上的二次函数,其图像是顶点在原点、对称轴为 $y$ 轴的曲线。换句话说,从“点到定点的距离等于它到定直线的距离”这一条件出发,所得到的轨迹,恰好就是一条抛物线。事实上,这个条件本身也就是抛物线的定义。

\begin{definition}{抛物线}
在平面上,所有满足到一条直线 $l$ 与该直线外一固定点 $F$ 的距离相等的点 $P$ 的轨迹,构成一个几何图形,称为\textbf{抛物线(parabola)}。即,对于抛物线上的任意一点 $P$,都有:
\begin{equation}
|PF| = d(P, l)~.
\end{equation}
其中,$F$ 被称为抛物线的\textbf{焦点(focus)},$l$ 被称为抛物线的\textbf{准线(directrix)},而焦点与准线之间的距离 $d(F, l)$ 被称为抛物线的\textbf{焦准距(focal parameter)},记作 $p$。
\end{definition}

根据初中所学,抛物线具有一条对称轴和一个顶点。结合几何定义来看,\textbf{对称轴(axis of symmetry)}是指过焦点并垂直于准线的直线,而\textbf{顶点(vertex)}则是焦点到准线所作垂线段的中点。在抛物线所分割的平面中,约定包含焦点的一侧为抛物线的内部,而包含准线的一侧为其外部。

虽然所有的二次函数图像都是抛物线,但从几何的角度来看,抛物线的本质形状并不依赖于其在坐标系中的具体位置。图像如何放置只会影响其代数表达式,而不会改变其几何性质。事实上,任何抛物线都可以通过平移或旋转,化为顶点在原点、对称轴与 $x$ 轴或 $y$ 轴重合的标准形式。因此,为了研究的简便性,高中阶段通常将研究对象限定为这类标准位置的抛物线。

\subsection{抛物线的方程}

之前提到了标准位置,下面就根据\autoref{eq_Para3_1} 看看其他条件下的抛物线的方程。
首先,如果将\autoref{ex_Para3_1} 中的定点设为 $F_1(p,0)$,直线设为$x=p$,则相当于将原先的 $x$ 轴与 $y$ 轴互换,得到的表达式为:
\begin{equation}
y^2=2px~.
\end{equation}
其次,根据二次函数的知识,如果想要改变\autoref{eq_Para3_2} 的开口方向,只要在二次项前面添加一个负号即可,从而有:
\begin{equation}
y = -\frac{1}{2p}x^2\quad\implies\quad x^2=-2py~.
\end{equation}
综上,得到抛物线的标准方程。

\begin{theorem}{抛物线的标准方程}\label{the_Para3_1}
\begin{itemize}
\item 焦点在$y$轴正半轴上的抛物线标准方程:
\begin{equation}
x^2=2py,\qquad(p>0)~.
\end{equation}
\item 焦点在$y$轴负半轴上的抛物线标准方程:
\begin{equation}
x^2=-2py,\qquad(p>0)~.
\end{equation}
\item 焦点在$x$轴正半轴上的抛物线标准方程:
\begin{equation}
y^2=2px,\qquad(p>0)~.
\end{equation}
\item 焦点在$x$轴负半轴上的抛物线标准方程:
\begin{equation}
y^2=-2px,\qquad(p>0)~.
\end{equation}
\end{itemize}
\end{theorem}
\autoref{the_Para3_1} 中的四个方程,分别对应\autoref{fig_Para3_2} 中开口向上、下、右左的四条抛物线。
\begin{figure}[ht]
\centering
\includegraphics[width=6cm]{./figures/70f83278a603508e.png}
\caption{标准抛物线} \label{fig_Para3_2}
\end{figure}
\begin{theorem}{抛物线的参数方程}
	•	用参数表示抛物线上的点,如 $x=pt^2,,y=2pt$ 等(视教学安排可选讲)
\end{theorem}





\subsection{抛物线的几何性质}
对称性(关于轴对称)
顶点、焦点、准线的定义和关系
开口方向与参数正负有关

\subsubsection{抛物线的反射性质}(
光线从焦点发出反射后平行于轴)

\subsubsection{相似性}

\begin{figure}[ht]
\centering
\includegraphics[width=4.2cm]{./figures/c89771dd2fef516e.pdf}
\caption{抛物线的几何性质} \label{fig_Para3_1}
\end{figure}

在 $x$ 轴正半轴作一条与准线平行的直线 $L$, 则抛物线上一点 $P$ 到其焦点的距离 $r$ 与 $P$ 到 $L$ 的距离之和不变。

如\autoref{fig_Para3_1}, 要证明由焦点和准线定义的抛物线满足该性质, 只需过点 $P$ 作从准线到直线 $L$ 的垂直线段 $AB$, 由于 $r$ 等于线段 $PA$ 的长度, 所以 $r$ 加上 $PB$ 的长度等于 $AB$ 的长度, 与 $P$ 的位置无关。 证毕。
\subsubsection{切线}
	•	给定抛物线方程和点,求切线方程
	•	切线的几何意义(过点,与焦点、准线的关系)
