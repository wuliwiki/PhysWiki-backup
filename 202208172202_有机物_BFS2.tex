% 树与图的广度优先搜索
% 树与图|BFS|算法|搜索

树与图的广度优先遍历的大致思路和 广度优先搜索(BFS)\upref{BFS}差不多,实现 BFS 需要使用一个队列来维护结点,每次先把 $1$ 号结点入队,在依次遍历每一层中的结点,并把这些结点插入到队尾,队头的元素出队,直到队列为空,就遍历完了所有元素.

\begin{lstlisting}[language=cpp]
void bfs()
{
    int hh = 0, tt = -1;
    q[ ++ tt] = 1;
    memset(d, -1, sizeof d);  // 初始化距离,-1 表示没被遍历过
    d[1] = 0;
    
    while (hh <= tt)
    {
        auto t = q[hh ++ ];
        for (int i = h[t]; ~i; i = ne[i])
        {
            int j = e[i];
            if (d[j] == -1)  // 若 j 没被遍历过
            {
                d[j] = d[t] + 1;
                q[ ++ tt] = j;
            }
        }
    }
}
\end{lstlisting}

上面的代码完成了图的广度优先遍历,并记录了图中的每个结点的层次(从 $1$ 号结点走到当前结点最少经过几个点)或树中的深度.也可以理解为求出了 $1$ 号结点每个结点的最短距离.

\textbf{拓扑排序}

拓扑序列是指,在一个有向无环图中(在图论中,如果一个有向图无法从某个顶点出发经过若干条边回到该点,则这个图是一个有向无环图(DAG图)),若一个由图中所有点构成的序列 $A$ 满足:对于图中的每条边 $(x, y)$,$x$ 在 $A$ 中都出现在 $y$ 之前,则称 $A$ 是该图的一个拓扑序列.求拓扑序列的过程就被称为拓扑排序.如果一个有向图有环,必定有一条边从后指向前,所以有向有环图显然不会构成拓扑序列,因此有向无环图也被称为拓扑图.

具体的做法是,在读入的过程中并处理每个结点的入度(有多少条边指向自己,边数即为入度),首先将入度为 $0$ 的点入队,入度为 $0$ 说明没有任何一条边指向自己,则当前结点就是符合要求的点.取出队头