% 自治系统解的特点
% keys 自治系统|解的性质
% license Xiao
% type Tutor

\pentry{基本知识(常微分方程)\upref{ODEPr}}
本节将说明自治系统(\autoref{def_ODEPr_2}~\upref{ODEPr})解的两个性质。一般的自治系统的对应的标准方程组(\autoref{def_ODEPr_1}~\upref{ODEPr})可写为
\begin{equation}
\dv{y^i}{x}=f^i(y^1,\cdots,y^n),\quad i=1,\cdots,n~.
\end{equation}
采用矢量写法为
\begin{equation}
\dv{y}{x}=f(y)~,
\end{equation}
其中 $y=(y^1,\cdots,y^n),f=(f^1,\cdots,f^n)$。与“基本知识(常微分方程)\upref{ODEPr}”中类似,我们总假定 $f^i(y^1,\cdots,y^n)$ 及其一阶偏导数在其定义区间(记为 $\Delta$)上连续。


