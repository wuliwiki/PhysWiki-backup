% 第三方的
% license Usr
% type Test
\begin{equation}
( P(A) ): [ P(A) = \frac{3}{10} ] (2) ( P(B) )( \frac{2}{9} ),因为此时盒子里剩下8张奖券,其中2张是有奖的。 由于甲中奖和不中奖的概率各为 ( \frac{3}{10} ) 和 ( \frac{7}{10} ),所以乙抽到有奖的奖券的总概率 ( P(B) ) 为: [ P(B) = P(A) \cdot P(B|A) + P(\overline{A}) \cdot P(B|\overline{A}) ] [ P(B) = \frac{3}{10} \cdot \frac{2}{9} + \frac{7}{10} \cdot \frac{3}{9} ] [ P(B) = \frac{6}{90} + \frac{21}{90} = \frac{27}{90} = \frac{3}{10} ] (3) ( P(AB) ):甲和乙都抽到有奖的奖券的概率。这可以通过甲抽到有奖的奖券,然后乙接着抽到有奖的奖券的概率来计算: [ P(AB) = P(A) \cdot P(B|A) ] [ P(AB) = \frac{3}{10} \cdot \frac{2}{9} = \frac{6}{90} = \frac{1}{15} ] (4) ( P(A|B) ):在乙抽到有奖的条件下,甲抽到有奖的概率,这是一个条件概率。根据条件概率的定义,我们有: [ P(A|B) = \frac{P(AB)}{P(B)} [ P(A|B) = \frac{\frac{1}{15}}{\frac{3}{10}} ] [ P(A|B) = \frac{1}{15} \cdot \frac{10}{3} = \frac{2}{9} ] 以上就是这个问题的详细解答。
\end{equation}


