% 哈尔滨工业大学 2006 年 考研 量子力学
% license Usr
% type Note

\textbf{声明}:“该内容来源于网络公开资料,不保证真实性,如有侵权请联系管理员”

\subsection{二、(20 分)}
设$t=0$时氢原子处于如下的状态上
$$\Psi(r, \theta, \phi; 0) = -\frac{1}{2} R_{21}(r) Y_{10} (\theta, \phi) - \frac{1}{\sqrt{2}} R_{31}(r) Y_{10} (\theta, \phi) - \frac{1}{\sqrt{2}} R_{21}(r) Y_{1,-1} (\theta, \phi)~$$
求其能量、轨道角动量平方及轨道角动量Z 分量的可能取值、相应的取值概率与平均值,写出任意r时刻的波函数。
\subsection{三、(20 分)}
质量为$m$的粒子在一维势阱
\[
V(x) = \begin{cases} 
\infty, & x < 0 \\
-V_0, & 0 \leq x \leq a \\
\infty, & x > a
\end{cases}~
\] 
中运动,试求出其能量本征值与相应的本征波函数。
\subsection{四、(20 分)}
导出坐标算符在动量表象中的形式,$\hat{\vec{r}} = i\hbar\overline{\nabla}$
\subsection{五、(20 分)}
证明两个泡利算符 $\hat{\vec{\sigma}}_1$ 与 $\hat{\vec{\sigma}}_2$ 满足

$$(\hat{\sigma}_1 \cdot \hat{\sigma}_2)^2 = 3 - 2 (\hat{\sigma}_1 \cdot \hat{\sigma}_2)~$$

并由此求出 $\hat{\vec{\sigma}}_1 \cdot \hat{\vec{\sigma}}_2$ 的本征值。提示:

$$ \left( \hat{\sigma} \cdot \vec{A} \right) \left( \hat{\sigma} \cdot \vec{B} \right) = \vec{A} \cdot \vec{B} + i \hat{\sigma} \cdot (\vec{A} \times \vec{B}), ~$$

式中,$\hat{\vec{A}}, \hat{\vec{B}}$ 为任意两个与 $\hat{\vec{\sigma}}$ 对易的矢量算符。
\subsection{六、(20 分)}