% 半双线性形式
% license Xiao
% type Tutor

\begin{issues}
\issueDraft
\end{issues}

\footnote{参考 Wikipedia \href{https://en.wikipedia.org/wiki/Sesquilinear_form}{相关页面}。}\textbf{半双线性形式(sesquilinear form)}是双线性映射的一个变形。 它关于第一个变量是反线性(也叫\textbf{共轭线性 conjugate linear})的, 关于第二个变量是线性的。
\begin{definition}{}\label{def_sequil_1}
复数域上的线性空间 $V$ 中, 若映射 $f:V\times V\to \mathbb C$ 对任意 $u, v, w\in V$, $a,b\in \mathbb C$ 满足($a^*$ 表示 $a$ 的复共轭\upref{CplxNo})
\begin{equation}\label{eq_sequil_2}
f(au+bv, w) = a^*f(u, w) + b^*f(v, w)~,
\end{equation}
\begin{equation}\label{eq_sequil_1}
f(u, av+bw) = af(u, v) + bf(u, w)~,
\end{equation}
那么就说该映射是\textbf{半双线性的(sesquilinear)}。
\end{definition}
如果满足 $f(u, v) = f(v, u)^*$, 就说它是\textbf{对称的}。

可以发现, 若把\autoref{eq_sequil_2} 中的共轭符号去掉, 那么该定义就是双线性的定义(\autoref{eq_Tensor_2}~\upref{Tensor})。

任意半双线性形式 $f(u, v)$ 可以唯一地由 $g(v) = f(v, v)$ 确定:
\begin{equation}
f(u, v) =\frac{1}{2}[g(u+v)-g(u)-g(v)]
-\frac{\I}{2}[g(u+\I v)-g(u)-g(\I v)]~,
\end{equation}
这叫做\textbf{极化恒等式(polarization identity)}。 这可以由定义直接证明。

和双线性函数(\autoref{def_Tensor_1}~\upref{Tensor})一样, 半双线性形式也可以用矩阵表示, 若两个 $V$ 空间的基底分别为 $\{e_i\}$ 和 $\{e'_i\}$(当然也可以使用同一组基底), 那么表示为矩阵 $\mat F$ 后, 矩阵元就是 $F_{i,j} = f(e_i, e'_j)$, 且有
\begin{equation}
f(u, v) = \bvec u\Her \mat F \bvec v~.
\end{equation}

\begin{example}{半双线性型对应矩阵在不同基底下的转换规则}\label{ex_sequil_1}
和二次型同理, 若半双线性形式 $f(u, v)$ 在不同基底下的矩阵分别为 $\mat F$ 和 $\mat F'$, 即 $F_{i,j} = f(e_i, e_j)$, $F'_{i,j} = f(e'_i, e'_j)$, 且任意 $u, v\in V$ 关于基底 $\{e_i\}$ 的坐标列矢量为 $\bvec u, \bvec v$, 关于基底 $\{e'_i\}$ 的坐标列矢量记为 $\bvec u', \bvec v'$。 那么
\begin{equation}
f(u, v) = \bvec u\Her \mat F \bvec v = {\bvec u'}\Her \mat F' \bvec v'~.
\end{equation}
令两组基底的变换矩阵为 $\mat A$, 即 $\bvec u = \mat A\bvec u'$, $\bvec v = \mat A \bvec v'$, 那么代入上式得
\begin{equation}
f(u, v) = {\bvec u'}\Her (\mat A\Her \mat F \mat A) \bvec v' = {\bvec u'}\Her \mat F' \bvec v'~.
\end{equation}
由于这对任意 $\bvec u', \bvec v'$ 都成立, 所以对比得
\begin{equation}
\mat F' = \mat A\Her \mat F \mat A~.
\end{equation}
\end{example}
