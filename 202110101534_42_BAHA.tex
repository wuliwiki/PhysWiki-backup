% Baker-Hausdorff公式
% keys Baker|Hausdorff|定理


Baker-Hausdorff公式是一个相当有用的公式.在数学上,它可用于给出李群-李代数对应的深层结果的相对简单的证明;在量子力学中,它可实现系统哈密顿量在薛定谔绘景和海森堡绘景的转换,并在微扰论中也有诸多应用.本节将给出该公式的一个证明和由它导出的一些重要的结果.

Baker-Hausdorff公式是指
\begin{equation}
\begin{aligned}
\E ^{A}B\E^{-A}=\sum_{n=0}^{\infty}\frac{1}{n!}C_n\\
C_n\equiv\underbrace{[A,[A,\cdots,[A}_{n\text{个}},B]]\cdots]
\end{aligned}
\end{equation}

\subsection{证明}
\begin{lemma}{}
\begin{equation}
\underbrace{[A,[A,\cdots,[A}_{n\text{个}},B]]=
\end{equation}
\end{lemma}
\textbf{证明:}
由 $C_n$ 的定义
\begin{equation}
\begin{aligned}
[A,A]\
C_1=
\end{aligned}
\end{equation}
