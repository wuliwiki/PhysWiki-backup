% 协变矢量、协变矢量与指标升降
% keys 四矢量|协变矢量|逆变矢量|指标升降
% license Xiao
% type Tutor

\pentry{狭义相对论与洛伦兹对称性\upref{SRrefe}}

约定使用东海岸度规 $\eta_{\mu\nu}=\rm{diag}(-1,1,1,1)$ 和自然单位制 $c=1$。本文对于重复出现在上指标和下指标的希腊字母 $\mu,\nu,\rho,\sigma$ 等采用爱因斯坦求和约定\upref{EinSum},例如
\begin{equation}
A_\mu B^{\mu\nu}:= \sum_{\mu=0,1,2,3}A_\mu B^{\mu\nu} ~.
\end{equation}

\subsection{协变矢量变换法则}
四矢量 $A^\mu=(A^0,A^1,A^2,A^3)$ 被称为协变矢量,当且仅当它满足一下的相对论变换法则
\begin{equation}\label{eq_vecspr_1}
A^\mu\rightarrow {A'}^\mu = \Lambda^{\mu}{}_\nu A^\nu~.
\end{equation}
其中 $\Lambda^\mu{}_\nu$ 为洛伦兹变换矩阵,它满足\footnote{具体的推导见狭义相对论与洛伦兹对称性\upref{SRrefe}。}
\begin{equation}\label{eq_vecspr_2}
\Lambda^{\mu}{}_{\rho} \Lambda^{\nu}{}_{\sigma} \eta_{\mu\nu} = \eta_{\rho\sigma}~.
\end{equation}
常见的协变矢量有时空坐标四矢量
\begin{equation}
x^\mu = (x^0,x^1,x^2,x^3) = (t,\bvec x)~.
\end{equation}
协变矢量的特征是有一个上指标,一般用希腊字母 $\mu,\nu,\rho,\sigma$ 等表示。变换公式 $A^\mu\rightarrow \Lambda^{\mu}{}_\nu A^\nu$ 描述了协变矢量在洛伦兹变换下的行为。

\subsection{逆变矢量变换法则}
如果将矩阵 $\eta_{\mu\nu}$ 与 $x^\nu$ 相乘,对所有指标 $\nu=0,1,2,3$ 求和,得到的 $x_\mu =\eta_{\mu\nu} x^\nu = x^\nu\eta_{\nu\mu}$ 就被称为是逆变矢量。对所有指标 $\nu=0,1,2,3$ 进行求和的过程叫做“缩并”,这里采用了爱因斯坦求和约定\upref{EinSum}。

\subsubsection{逆变矢量的性质 1}
\begin{lemma}{}
协变矢量与逆变矢量的缩并 $A^\mu B_\mu=A_\mu B^\mu$ 是一个洛伦兹不变的量,也被称为洛伦兹标量。
\end{lemma}
\textbf{证明}:$A^\mu B_\mu = \eta_{\mu\nu}A^\mu B^\nu=A_\nu B^\nu$,在洛伦兹变换下 \begin{equation}
\begin{aligned}
&A^\mu B_\mu = \eta_{\mu\nu}A^\mu B^\nu ~,\\
\rightarrow &{A'}^\mu {B'}_\mu =\eta_{\mu\nu}{A'}^\mu {B'}^\nu = \eta_{\mu\nu}\Lambda^{\mu}{}_\rho\Lambda^\nu{}_\sigma A^\rho  B^\sigma~,\\
&=\eta_{\rho\sigma}A^\rho B^\sigma = A^\rho B_\rho
\end{aligned}
\end{equation}
最后一行利用了 \autoref{eq_vecspr_2},最终我们发现 ${A'}^\mu {B'}_\mu = A^\mu B_\mu $,这也就意味着 $A^\mu B_\mu$ 是一个洛伦兹不变的量。
\begin{lemma}{}
根据逆变矢量的定义公式 $A_\mu = \eta_{\mu\nu}A^\nu$,再定义一个矩阵 $\eta^{\rho\sigma}$,满足 $A^\mu = \eta^{\mu\nu} A_\nu$。那么 $\eta^{\mu\nu}$ 是 $\eta_{\mu\nu}$ 的逆矩阵:
\begin{equation}
\eta^{\mu\nu}\eta_{\nu\rho}=\delta^{\mu}_\rho~.
\end{equation}
\end{lemma}
\subsubsection{逆变矢量的性质 2}
\begin{lemma}{}
逆变矢量 $A_\mu$ 在洛伦兹变换下的行为是
\begin{equation}
A_\mu \rightarrow  A'_\mu = \Lambda_{\mu}{}^\nu A_\nu=\eta_{\mu\rho}\Lambda^{\rho}{}_\sigma \eta^{\sigma\nu} A_\nu~.
\end{equation}

\end{lemma}