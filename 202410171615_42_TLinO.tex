% 拓扑线性空间中的线性算子
% keys 线性算子
% license Usr
% type Tutor

\pentry{拓扑向量空间\nref{nod_tvs}}{nod_c6e1}
线性空间中由\enref{线性算子}{LiOper}的定义,拓扑线性空间是线性空间,因此在拓扑线性空间中就有线性算子的定义。此外,拓扑线性空间的拓扑性质使得其上的线性算子可以定义连续性,而算子的像和核有开闭性的讨论。

\begin{definition}{线性算子}
设 $E,E_1$ 是两个(定义在域 $\mathbb F$上)\enref{拓扑线性空间}{tvs},$D_A\subset E,$ 若映射 $A:D_A\rightarrow E_1$ 满足:
\begin{enumerate}
\item \textbf{可加性:}$A(x+y)=A(x)+A(y),\quad x,y\in E$;
\item \textbf{齐次性:}$A(\alpha x)=\alpha A(x),\quad x\in E,\alpha\in \mathbb F$
\end{enumerate}
则称 $A$ 是 $E$ 到 $E_1$ 的\textbf{线性算子}。
  
\end{definition}

由线性算子的定义,\enref{线性泛函}{Funal}显然是一类特殊的线性算子。

\textbf{注意:}一般地,不能假定 $D_A=E$,然而总可以假定 $D_A$ 是线性流形,即 $x,y\in D_A$,则 $\alpha x+\beta y\in D_A$ 对任意 $\alpha,\beta\in\mathbb F$ 恒成立(因为总可以通过 $A(\alpha x+\beta y)=\alpha A(x)+\beta A(y)$定义 $A$ 在 $\alpha x+\beta y$ 的值)。

对算子而言,通常记 $Ax:=A(x)$。

\subsection{性质}
\begin{definition}{连续}
设 $A:D_A\rightarrow E_1$ 是 $E$ 到 $E_1$ 的线性算子,若对 $y_0=Ax_0$ ($x_0\in D_A$) 的任意邻域 $V$,存在 $x_0$ 的邻域 $U$,使得 $A(U\cap D_A)\subset V$,则称 $A$ 在 $x_0$ 是\textbf{连续的}。若 $A$ 在 $D_A$ 上处处连续,则称 $A$ \textbf{连续}。
\end{definition}


\begin{definition}{核、象}
设 $A:D_A\rightarrow E_1$ 是 $E$ 到 $E_1$ 的线性算子。称 $\ker A:=\{x|Ax=0,x\in E\}$ 是 $A$ 的\textbf{核}(kernel),而 称 $\Im A:=\{y|y=Ax,x\in D_A\}$ 为 $A$ 的\textbf{象}(image)。 
\end{definition}

线性算子的象显然也是线性流形,因为 $Ax_1,Ax_2\in \Im A$,则 $\alpha Ax_1+\beta Ax_2=A(\alpha x_1+\beta x_2)$,而 $D_A$ 是线性流形,所以 $\alpha x_1+\beta x_2\in D_A$,从而 $\alpha Ax_1+\beta Ax_2\in\Im A$。 

\begin{theorem}{}
设 $A$ 是连续线性算子,则 $\ker A$ 是闭的。
\end{theorem}
\textbf{证明:}设 $x\in[\ker A]$,则存在 $\ker A$ 中的收敛到 $x$ 的序列 $\{x_n\}$。于是对所有的 $n$ 成立:
\begin{equation}
Ax_n=0.~
\end{equation}
 上面两边取极限,并由连续性,得 \begin{equation}
 Ax=A\lim_{n\rightarrow\infty}x_n\overset{A\text{连续}}{=}\lim_{n\rightarrow\infty}A x_n=\lim_{n\rightarrow\infty}0=0.~
 \end{equation}
 因此 $x\in\ker A$。即 $[\ker A]\subset\ker A$,从而 $[\ker A]=\ker A$。因此 $\ker A$ 闭。

\textbf{证毕!}

然而对 $\Im A$ ,即使当 $D_A=E$ 时,也不一定在 $E_1$ 中是闭的。


\subsection{例子}

\begin{example}{}
在闭区间 $[a,b]$ 上连续函数的空间内考虑由如下公式定义的算子
\begin{equation}
\phi(s)=\int_a^bK(s,t)\varphi(t)\dd t,~
\end{equation}
其中 $K(s,t)$ 是在 $[a,b]\times[a,b]$ 上固定的二元连续函数。由积分的定义,$\phi(s)$ 对任意连续的函数 $\varphi(t)$ 是连续的。所以该算子实际上把连续函数空间变到自身上。线性性质是显然的。为了讨论它的连续性,必须预先指出连续函数空间上有怎样的拓扑。建议读者证明下述情况下的连续性:
\begin{enumerate}
\item 考虑连续函数空间 $C[a,b]$ 其范数为 $\norm{\varphi}=\max{\abs{\varphi(t)}}$;
\item 考虑二阶连续函数空间 $C_2[a,b]$ 其范数为 $\norm{\varphi}=\qty(\int_a^b\varphi^2(t)\dd t)^{1/2}$。
\end{enumerate}

\end{example}




