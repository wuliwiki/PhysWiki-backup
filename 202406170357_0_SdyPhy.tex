% 如何自学物理
% license CCBYSA3
% type Tutor

要谈论这个话题,首当其冲的应该是荷兰诺贝尔物理学奖得主 Gerard 't Hooft 的文章了:

\href{https://webspace.science.uu.nl/~hooft101/theorist.html}{How to become a GOOD Theoretical Physicist}

这里附上一个版本的中文翻译: \href{https://xialab.pku.edu.cn/kytdyw1/zdylm.m.jsp?wbtreeid=1011&tstreeid=11956&_t_uid=2945&language=en&homepageuuid=BF649325C5584FC683CE0B601D21AC65&templateuuid=4CC182410BA14FF8B55ED726FB2087FB&producttype=0&_tmode_=99&tsitesapptype=zdylm}{《如何成为一名优秀的理论物理学家》}

我们不妨就这篇文章的核心内容详细讨论。

首先,标题说的虽然是 “理论物理学家”, 但大致来说也\textbf{同样适用于物理学其他研究方向}。 对于公众来说,可能一谈到物理学马上就会想到那些喜闻乐见的理论话题例如宇宙、黑洞、高能物理等。 这些话题的确非常引人入胜,但远非物理学的全部。 事实上只有很小一部分从事物理学研究的人会做这些领域。 

以下是物理学各主要方向研究者的大致比例(参考\href{https://ww2.aip.org/statistics/trends-in-physics-phds}{这个}):
\begin{itemize}
\item 凝聚态物理:约 40\% - 45\%
\item 高能物理(粒子物理):约 15\% - 20\%
\item 原子、分子和光物理:约 10\% - 15\%
\item 天体物理和宇宙学:约 10\% - 15\%
\item 等离子体物理:约 5\% - 10\%
\item 应用物理:约 5\% - 10\%
\end{itemize}


为什么需要学英语?英语是事实上的科研学术交流语言。在物理领域(以及其他大部分自然科学领域),正经的文章都是用英语发表的。哪怕是为数较少的国产 SCI 期刊,也多数是英语或双语的。 可以说一定程度的英语水平是进入物理领域的门槛。


