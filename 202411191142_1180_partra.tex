% 约化密度矩阵
% keys 偏迹|约化密度矩阵|部分迹
% license Xiao
% type Tutor

\pentry{密度矩阵\nref{nod_denMat},张量积空间的算符\nref{nod_ProdOp}}{nod_6acc}

对于孤立系统,我们知道其量子态是 Hilbert 空间的一个射线,而量子测量一个正交投影算符,系统的动力学演化是幺正的($U(t,0)=e^{-iHt}$)。现在我们转而研究开放系统(open systems),也就是说,当我们考察一个更大的量子系统中的一个\textbf{子系统},那么这个子系统与外部之间是有相互作用的。此时子系统的\textbf{态}不再是射线,对它的测量不再是正交投影算符,同时子系统的演化也不再是幺正的了。
\subsection{偏迹}
假设整个大系统的 Hilbert 空间可以表示为两个子 Hilbert 空间的张量积:$\mathcal{H}_A\otimes \mathcal{H}_B$,其中 $\mathcal A$ 是待研究的子系统的 Hilbert 空间。可以将该大系统的量子态表示为
\begin{equation}
\ket{\psi}_{AB}=\sum_{i,\mu}a_{i\mu}\ket{i}_A\otimes \ket{\mu}_B~.
\end{equation}
其中 $\ket{i}_A,\ket{\mu}_B$ 分别组成了 $\mathcal{H}_A, \mathcal{H}_B$ 的两组正交完备基,因此 $\ket{i}_A\ket{\mu}_B$ 组成了 $\mathcal{H}_A\otimes \mathcal{H}_B$ 的正交完备基。现在考虑对子系统 A 的测量操作 $M_A$(它是 $\mathcal H_A$ 的正交投影算符),它所对应的 $\mathcal{H}_A\otimes \mathcal{H}_B$ 上的测量算符应当是 $M_A\otimes I_B$。该可观测量的期望值为
\begin{equation}
\begin{aligned}
\ev{M_A}&={}_{AB}\bra{\psi}M_A\otimes I_B \ket{\psi}_{AB}\\
&=\sum_{i,j,\mu}a_{i\mu}^*a_{j\mu} \cdot [{}_A\bra{i} M_A \ket{j}_A]\\
&=\text{tr}(M_A \rho_A)~,
\end{aligned}
\end{equation}
其中 $\rho_A$ 被定义为
\begin{equation}\label{eq_partra_1}
\rho_A=\text{tr}_B(\ket{\psi}\bra{\psi})=\sum_{i,j,\mu}a_{i\mu}a_{j\mu}^*\ket{i}\bra{j}~.
\end{equation}
我们称该算符为\textbf{约化密度算符},$\text{tr}_B$ 为\textbf{偏迹}(partial trace),有时候也被称为部分迹,也就是说,我们只对 $\rho=\ket{\psi}\bra{\psi}$ 中 $\mathcal{H}_B$ 的指标部分取迹。

\subsection{约化密度算符的性质}
\pentry{正定矩阵\nref{nod_DefMat}}{nod_638b}
根据\autoref{eq_partra_1},我们很容易证明以下性质:
\begin{enumerate}
\item $\rho_A$ 是厄米的。
\item $\rho_A$ 是正定的,也就是说对于任意 $\ket{\varphi}$,$\bra{\varphi}{\rho_A}\ket{\varphi}\ge 0$。
\item $\text{tr}(\rho_A)=1$,这是由于 $\ket{\psi}_{AB}$ 是归一化的。
\end{enumerate}

因为密度矩阵的正定性,它总是可以在一组正交完备基上对角化,也就是说总是可以表示为
\begin{equation}
\rho_A=\sum_a p_a\ket{a}\bra{a}~.
\end{equation}
其中 $\ket{a}$ 组成了 $\mathcal{H}$ 的一组正交完备基;$p_a$ 是该正定算符的本征值,且 $\sum_a p_a=1,p_a>0$。如果 $\rho^2\neq \rho$,那么所有 $p_a$ 都小于 $1$,此时子系统 $A$ 不再是一个纯态,而是多个纯态组成的\textbf{系综},系综中每个态都被赋予了一个经典意义上概率的诠释,即都有一定的概率 $p_a$ 出现。这可以从以下的式子中看到:
\begin{equation}
\ev{M}=\text{tr} M\rho_A = \sum_a p_a \bra{a}M\ket{a}~.
\end{equation}


除了 $\rho^2\neq \rho$ 这个判据以外,我们也可以采用以下的判据
\begin{equation}\label{eq_partra_2}
\text{tr} \rho^2 <1~.
\end{equation}
若该判据成立,那么 $A$ 就不是纯态;否则 $A$ 是纯态。

若 $A$ 不是纯态,那么可以知道 $A$ 与 $B$ 间存在\enref{纠缠}{entang},以至于 $A$ 不能被视作孤立的系统。
