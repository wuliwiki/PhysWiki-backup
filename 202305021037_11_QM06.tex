% 相互作用绘景
% keys 动力学|微扰近似
\pentry{量子力学的基本原理\upref{Sample}}\pentry{薛定谔绘景和海森堡绘景\upref{Sample}}
\begin{definition}{}

不同于海森堡绘景和薛定谔绘景,在相互作用绘景里,态矢和算符都随时间而改变。哈密顿量可以分割为:
\begin{equation}
H=H_0+H_I
\end{equation}
下标I为相互作用的缩写。与海森堡绘景相仿,相互作用绘景里的态矢和算符通过薛定谔绘景(下标为S)导出如下:
\begin{equation}
\ket{s,t}_I=e^{iH_0 t}\ket{s,t}_S
\end{equation}
\begin{equation}
\mathcal O_I(t)=e^{iH_0 t}\mathcal O_Se^{-iH_0 t}
\end{equation}
可以验证,该定义是良定义,算符矩阵元不随绘景改变。
\end{definition}
对于一个绘景,我们首要关心的是态矢是如何随着时间演化的。根据薛定谔方程,我们可以导出相互作用绘景下的态矢动力学方程。
\begin{equation}
H\ket{s,t}_I=i\frac{\partial\ket{s,t}_I}{\partial t}= e^{iH_0 t}(-H_0)\ket{s,t}_S+e^{iH_0 t}H\ket{s,t}_S=e^{iH_0 t}H_Ie^{iH_0 t}
\end{equation}


