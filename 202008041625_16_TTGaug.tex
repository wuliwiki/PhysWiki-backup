% TT规范
在源的外面,我们有$T_{\mu\nu} = 0$,所以我们有如下方程
\begin{equation}\label{TTGaug_eq1}
\Box \bar h_{\mu\nu} = 0 ~.
\end{equation}
其中$\Box = - (1/c^2) \partial_0^2 +\nabla^2$. 这个方程说明了引力波以光速传播。因为\autoref{Geomet_eq3}没有完全确定规范,在没有源的地方我们能够极大地固定规范。

如果$\Box \xi_\mu = 0$,那么变换$x^\mu \rightarrow x^\mu+\xi^\mu$保持$\partial^\nu \bar h_{\mu\nu}$不变。从$\Box \xi_\mu = 0$可以推出
\begin{equation}
\Box \xi_{\mu\nu} = 0~, \quad \xi_{\mu\nu} \equiv \partial_{\mu} \xi_\nu +\partial_\nu \xi_\mu - \eta_{\mu\nu} \partial_\rho\xi^\rho~. 
\end{equation}
我们可以使用TT规范
\begin{equation}
h^{0\mu} = 0~, \quad h^i_i = 0~, \quad \partial^j j_{ij} = 0~.
\end{equation}
加上洛伦兹条件把十个自由度的对称矩阵$h_{\mu\nu}$变为六个自由度。剩余的四个满足$\Box \xi_\mu = 0$的$\xi^\mu$把它降到了两个自由度。我们把TT规范下的度规记作$h_{ij}^{TT}$. 

注意,TT规范在含有源的时候是不能取的,这是因为$\Box \bar h_{\mu\nu} \neq 0$. \autoref{TTGaug_eq1}有如下的平面波解
\begin{equation}
h_{ij}^{TT} (x) = e_{ij} (\mathbf k) e^{ikx} ~.  
\end{equation}
其中$k^\mu = (\omega/c,\mathbf k)$, $\omega/c = |\mathbf k|$。$e_{ij}(\mathbf k)$被称作极化张量。沿着z方向前进的度规张量是
\begin{equation}
h_{ij}^{TT} (t,z) = 
\begin{pmatrix}
h_+ & h_\times & 0 \\
h_\times & - h_+ & 0 \\
0 & 0 & 0
\end{pmatrix} \cos [\omega (t - z/c)]~. 
\end{equation}
更简要地,我们可以写成
\begin{equation}
h_{ab}^{TT} (t,z) = 
\begin{pmatrix}
h_+ & h_\times   \\
h_\times & - h_+   
\end{pmatrix} \cos [\omega (t - z/c)]~. 
\end{equation}
其中$a,b = 1,2$. 代入到度规中,我们可以写成
\begin{equation}
\begin{aligned}
ds^2 & = - c^2 dt^2 + dz^2 + \{ 1+ h_+ \cos [\omega(t-z/c)] \} dx^2 \\
& + \{ 1-h_+ \cos [\omega(t-z/c)] \} dy^2 + 2 h_\times \cos[\omega(t-z/c)] dx dy~.
\end{aligned}
\end{equation}
给定一个在源外面朝着$\hat{\bf{n}}$方向传播的平面波解$h_{\mu\nu}(x)$.假设我们已经加入了洛伦兹规范但还没有加上TT规范,我们可以通过如下步骤找到TT规范下的波。

首先,我们引入如下的张量
\begin{equation}
P_{ij} (\hat{\bf{n}} ) = \delta_{ij} - n_in_j~.
\end{equation}
这个张量是对称,横向的($ n^iP_{ij}(\hat{\bf{n}}) = 0 $ ),是一个投影。















