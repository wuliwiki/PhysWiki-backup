% 一种矢量算符的运算方法
% keys Nabla算子|矢量运算|Gibbs算符
% license Xiao
% type Tutor

\pentry{矢量算符\nref{nod_VecOp}, 连续叉乘的化简\nref{nod_TriCro}}{nod_c5b7}

声明: 本文中的方法是笔者原创, 使用的下标符号也是笔者自己定义的。

在用 $\grad$ 算符计算梯度, 散度和旋度时, 我们几乎可以将其看作一个矢量进行运算, 唯一的区别就是我们需要明确每一项中的偏微分是对哪些变量进行的。 例如
\begin{equation}\label{eq_MyNab_1}
\div(U\bvec A) = \pdv{x} (UA_x) + \pdv{y} (UA_y) + \pdv{z} (UA_z)~,
\end{equation}
\begin{equation}
(\bvec A \vdot \grad) U = A_x \pdv{U}{x} + A_y \pdv{U}{y} + A_z \pdv{U}{z}~.
\end{equation}
如果以上两式中把 $\grad$ 符号替换成一个普通的矢量, 两式将没有任何区别。 可见 $\grad$ 符号包含了另一层信息, 这个信息通过 $\grad$ 所在的位置来体现, 但我们希望能定义一种新的符号 $[\dots]_{\dots}$, 把偏导算符的作用对象在方括号的角标中声明, 而在方括号内的 $\grad$ 可以像普通矢量一样进行运算, 例如
\begin{equation}
 [\div(U\bvec A)]_{A\partial U}
 \equiv [\bvec A\vdot \grad U]_{A\partial U}
 \equiv [U\div \bvec A]_{A\partial U}
 \equiv \bvec A \vdot\grad U~.
\end{equation}
又如, 利用矢量公式 $\bvec A\cross(\bvec B\cross \bvec C)  = \bvec B (\bvec A\vdot \bvec C) - \bvec C(\bvec A\vdot\bvec B)$, 有
\begin{equation}
[\curl (\bvec A\cross\bvec B)]_{\partial (AB)} = [\bvec A (\div \bvec B) + \bvec B (\div \bvec A)]_{\partial (AB)}~.
\end{equation}
另外, 由乘法的求导法则,有
\begin{equation}
[\dots]_{\partial (AB)} = [\dots]_{B\partial A} + [\dots]_{A\partial B}~.
\end{equation}
使用这个新符号, 我们可以化简许多常用的矢量公式。

\begin{example}{}
证明 $\curl(U\bvec A) = (\grad U) \cross\bvec A + U \curl\bvec A$。
\begin{equation}\ali{
{}[\curl (U\bvec A)]_{\partial(UA)}
&= [\curl (U\bvec A)]_{A\partial U} + [\curl (U\bvec A)]_{U\partial A}\\
&= [(\grad U) \cross\bvec A]_{A\partial U} + [U \curl\bvec A]_{U\partial A}\\
&= (\grad U) \cross\bvec A + U \curl\bvec A~,
}\end{equation}
证毕。
\end{example}

\begin{example}{}
化简 $\curl(\curl \bvec E)$。
\begin{equation}
{}[\curl(\curl \bvec E)]_{\partial^2 E} = [\grad(\div\bvec E) - \laplacian \bvec E]_{\partial^2 E}
= \grad(\div\bvec E) - \laplacian \bvec E~.
\end{equation}
\end{example}

\begin{example}{}
证明 $\grad(\bvec F \vdot \bvec G) = \bvec F\cross(\curl \bvec G) + \bvec G\cross (\curl{\bvec F}) + (\bvec F\vdot\grad)\bvec G + (\bvec G\vdot\grad)\bvec F$。

从右向左证明, 上式等于
\begin{equation}\ali{
&\quad [\bvec F\cross(\curl \bvec G)]_{F\partial G} + [\bvec G\cross (\curl{\bvec F})]_{G\partial F} + [(\bvec F\vdot\grad)\bvec G]_{F\partial G} + [(\bvec G\vdot\grad)\bvec F]_{G\partial F}\\
&= [\grad (\bvec F\vdot\bvec G) - (\bvec F\vdot\grad)\bvec G]_{F\partial G} +[\grad(\bvec F\vdot\bvec G) - (\bvec G \vdot\grad)\bvec F]_{G\partial F} \\
& \qquad + [(\bvec F\vdot\grad)\bvec G]_{F\partial G} + [(\bvec G\vdot\grad)\bvec F]_{G\partial F}\\
&= [\grad(\bvec F\vdot\bvec G)]_{F\partial G} + [\grad(\bvec F\vdot\bvec G)]_{F\partial G}\\
&= [\grad(\bvec F\vdot\bvec G)]_{\partial (FG)} = \grad(\bvec F\vdot\bvec G)~,
}\end{equation}
证毕。
\end{example}
