% 泰勒级数 2
% keys 泰勒级数|博雷尔定理|导数|解析函数|光滑函数

\pentry{泰勒级数\upref{Taylor}}

\subsection{泰勒展开与近似}
事实上,泰勒展开可以看成是微分近似\upref{Diff}的一种高阶拓展。 微分近似中,在某点 $x_0$ 附近有
\begin{equation}
f(x) \approx f(x_0) + f'(x_0)(x - x_0)
\end{equation}
而这恰好是泰勒展开的前两项。 然而,这只是函数曲线在 $x_0$ 处的切线(见\autoref{Taylor_fig1}~\upref{Taylor} 中 $N=1$ 的情况),显然没有高阶的泰勒展开那么精确。 如果我们将 $f(x)$ 近似到其泰勒展开的 $x^n$ 项, 我们称这个近似精确到第 $n$ 阶, 因为它的误差小于或等于 $n + 1$ 阶无穷小\upref{Lim} $O(x^{n + 1})$。 在近似计算中, 可以使用(??) 来精确地估计近似多项式给出的误差。 例如, \autoref{Taylor_fig1}~\upref{Taylor} 中画出的正弦函数与其近似多项式的误差即可按此公式计算。

\subsection{泰勒展开的问题}
尽管泰勒展开式给出了函数在某点处的多项式近似, 但这里很可能会产生一些微妙的问题。 从余项公式(??) 并不能推出余项一定随着 $N$ 的增大而趋于零, 因为函数的 $N$ 阶导数可能会随着 $N$ 的增长而快速增长; 具体来说, 有博雷尔(E. Borel) 的一个定理:
\begin{theorem}{博雷尔定理}
对于任何复数序列 $\{c_n\}$, 都存在一个光滑函数 $f$ 使得其在 $x=0$ 处的泰勒展开式为 $\sum_{n=0}^\infty c_nx^n$.
\end{theorem}
因此, 显然不收敛的级数 $\sum_{n=0}^\infty n!x^n$ 实际上也是某个光滑函数的泰勒展开! 这说明由泰勒展开截断得到的多项式仅仅是一个渐近展开\upref{Asympt}, 仅仅在展开点那一处给出了最佳的多项式近似, 却并不能说明任何其它性质。 也就是说, 对于\textbf{固定的 $N$}, 当 $x$ 越来越接近 $x_0$ 时, $f(x)$ 同 $N$ 泰勒多项式之间的差是高于 $N$ 阶的高阶无穷小, 但对于固定的 $x$, 却不能断言当 $N$ 趋于无穷时 $f(x)$ 同其泰勒多项式的误差趋于零。 当然, 在许多情况下, 这对于近似计算已经足够用。

进一步, 即便泰勒级数收敛, 也不一定收敛到展开的函数本身, 例如函数 $f(x)=e^{-1/x^2}$ 在点 $x=0$ 处的所有导数都是零, 因此它在这点处的泰勒展开是 $0+0x+0x^2+...$, 它与函数 $f(x)$ 的误差永远是函数本身!

由此可见, 泰勒级数收敛且收敛到本身的函数实在是非常特殊的。 这样的函数称为\textbf{解析函数(analytic function)}。 详见幂级数\upref{anal}。
