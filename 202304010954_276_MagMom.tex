% 磁矩
% 磁矩|安培力|电荷|磁场|线圈

\begin{issues}
\issueTODO
\end{issues}

\pentry{力矩\upref{Torque}, 安培力\upref{FAmp}}

当我们把一个通有电流的线圈放置在磁场中时, 这个线圈往往回受到安培力产生的力矩。 本词条讨论如何计算一些简单的情况, 先看一道例题。

\begin{example}{匀强磁场中的长方形线圈}
假设一个粗细可以忽略不计的长方形电流环路被放置在匀强磁场 $\bvec B$ 中(图未完成), 两条边的边长分别为 $a, b$, 电流为 $I$。 线圈平面的法向量(由右手定则\upref{RHRul}定义)和磁场夹角为 $\theta$, 边长为 $a$ 边始终垂直于磁场。 求线圈所受力矩。

解: (未完成)力矩的方向使得线圈的法向磁场的方向转动, 大小正比于 $\sin\theta$。
\end{example}

为了更好地表示这个结果, 定义\textbf{磁矩(magnetic moment)}为
\begin{equation}\label{MagMom_eq1}
\bvec \mu = I \bvec A~.
\end{equation}
其中 $\bvec A$ 的方向是线圈的法向量, 模长等于线圈的面积, 我们不妨把它叫做面积矢量。 根据矢量叉乘\upref{Cross}的几何定义, 线圈所受力矩为
\begin{equation}
\bvec \tau = \bvec \mu \times \bvec B
\end{equation}
事实上, 可以证明这个结论与线圈的形状无关, 线圈甚至可以不共面(此时需要重新定义 $\bvec A$), 详见 “磁场中闭合电流的力矩\upref{EBTorq}”。 另外, 当我们由多匝线圈时, 只需将结果乘以匝数即可。

\subsection{旋转的电荷}
对于绕轴做圆周运动的点电荷, 令角速度为 $\omega$, 等效电流为 $q/T$, $T = 2\pi/\omega$ 是转动周期, 圆周面积 $A = \pi r^2$, 代入\autoref{MagMom_eq1} 得
\begin{equation}
\bvec \mu = \frac{1}{2} q\omega r^2
\end{equation}

\begin{example}{带电粒子经典圆周运动的磁矩}\label{MagMom_ex1}
设一个带电粒子的电荷为 $q$,并且以匀速 $v$ 绕 $z$ 轴上半径为 $r$ 的圆做圆周运动,那么它所产生的电流为:
\begin{equation}
I=\frac{qv}{2\pi r}
\end{equation}
那么,它的磁矩为:
\begin{equation}
\bvec\mu  = \frac{qv}{2\pi r}\pi r^2 \hat{\bvec z}=\frac{qvr}{2} \hat{\bvec z} 
\end{equation}
由于其角动量为:
\begin{equation}
\bvec L =mvr\hat{\bvec z}
\end{equation}
因此,磁矩与角动量的经典关系为:
\begin{equation}
\bvec\mu  = \frac{q}{2m}\bvec L
\end{equation}

\end{example}
% 未完成: 连续分布的情况
