% 实数

\subsection{从有理数到实数}

我们知道, 有理数集$\mathbb{Q}$是对四则运算封闭的最小的数系. 从正整数开始, 为了使得任意两个整数都能相减, 我们引入了零和负整数, 从而得到了整数集$\mathbb{Z}$; 而为了使得任意两个整数的除法都有意义 (当然, 要剔去除数为零的情形), 我们又引入了形如$m/n$的数, 从而得到了有理数集$\mathbb{Q}$. 有理数的英文 rational number 即来源于ratio (比例) 一词. 小学算术已经告诉我们, 有理数的和, 差, 积, 商都是有理数 (仍然需要假定除数不等于零), 而且对于任何有理数$r$都有$r+0=r$, $r\cdot1=r$. 用近代代数学的语言, 这表示有理数集构成了一个\textbf{域 (field)}.

但是我们也知道, 并非所有来自实际问题的度量对象都能用有理数来表示. 例如, 假若承认勾股定理 (在古希腊, 发现并证明它的是毕达哥拉斯), 那么直角边长为1的等腰直角三角形的斜边长$c=\sqrt{2}$满足$c^2=2$. 毕达哥拉斯的门徒西帕索斯发现, 这个奇特的数$c$不能表示为两个整数的比. 西帕索斯的发现打击了毕达哥拉斯学派的信条"万物皆 (有理) 数", 因而被试图维护教义的门徒们杀害. 不过, 随着越来越多类似的例子出现, 到了十世纪左右, 数学家们已经承认了这些不是有理数的平方根是同有理数一样真实的对象. 这些\textbf{无理数 (irrational number)} 是在求解整系数代数方程的过程中出现的, 它们是\textbf{代数数 (algebraic number)} 的例子.

\begin{exercise}{$\sqrt{2}$是无理数}
利用数论中的素因子分解定理 (每个正整数都可以唯一分解成它的素因子乘积; 这件事并不是显然的), 证明不存在整数$m,n$使得$m^2=2n^2$. 更一般地, 如果$p$是素数, 那么不存在整数$m,n,k>1$使得$m^k=pn^k$.
\end{exercise}

然而, 代数数仍然不能穷尽来自真实世界的度量对象. 圆周率$\pi=3.14159265...$这个几乎同$\sqrt{2}$一样古老的数, 在十九世纪末被证明不能表示成任何整系数代数方程的根. 它是一个\textbf{超越数 (transcendental number)}. 有理数, 无理代数数和超越数合起来构成了人们称之为"实数"的对象.

\subsection{戴德金分割}
我们按照戴德金 (Julius Wilhelm Richard Dedekind) 的方法来定义实数. 按照这种办法, 一个实数被定义为有理数集合的一个分割. 后面将会看到, 这远非唯一一种定义实数的方法, 但所有的方法给出的都是等价的对象. 之所以不用十进制小数, 是因为十进制小数的四则运算叙述起来甚至不一定有纯粹集合论的戴德金分割简洁.

\begin{definition}{戴德金分割}
一个\textbf{戴德金分割 (Dedekind cut)} 或一个实数被定义为将有理数集分划成两个部分的一种方式$\mathbb{Q}=L\cup R$, 其中不交的子集$L$和$R$满足

\begin{enumerate}
\item 如果$l\in L$, 那么任何小于$l$的有理数$l'$都属于$L$.
\item 如果$r\in R$, 那么任何大于$r$的有理数$r'$都属于$r$.
\item 如果$l\in L$, $r\in R$, 那么必有$l\leq r$.
\item $L$不包含最大的元素.
\end{enumerate}

这里的$L$被称为分割$L\cup R$的下类 (lower class), $R$被称作上类 (upper class). 两个实数相等当且仅当它们的上类和下类分别相等.
\end{definition}

显然, 对于有理数$q$, 它确定了一个分割$L:=\{l\in\mathbb{Q}:l<q\}$, $R:=\{r\in\mathbb{Q}:r\geq q\}$. 有理数集可以按这种方式视为实数集的子集.

乍看起来, 这种定义方式很违背直觉. 然而我们若以$\sqrt{2}$作为例子, 即可看出这个定义的确能够填补有理数之间的空隙. 定义$\sqrt{2}$的下类$L$为满足$l\leq0$或者$l^2<2$的有理数$l$的集合, 上类$R$为满足$r^2>2$的正有理数的集合. 当然不可能从有理数集中找出一个分点来标记这个分割 (这跟有理数对应的分割不同), 但我们可以设想这个分割本身就是一个"标志", 它标记的正是方程$x^2=2$所造成的有理数不能填补的空隙.
\begin{exercise}{}
证明由此确定的$\sqrt{2}$的分割的确符合戴德金分割的定义.
\end{exercise}

\subsection{对戴德金分割进行四则运算}
从有理数的定义出发, 我们可以对戴德金分割定义四则运算, 从而使得所有的戴德金分割组成通常意义下的"数". 

以下假设$x,y$是实数, $L_x\cup R_x$和$L_y\cup R_y$是它们对应的分割. 在这里作出一个约定: 对上类和下类进行集合运算时, 可能会使得下类里出现最大元素; 这时我们约定将这个最大元素 (如果当真有的话) 自动移到上类里, 而不再特别说明.

\subsubsection{全序关系}

首先可以很方便地定义两者之间的序关系: 说$x\leq y$, 也就是说$L_x\subset L_y$, 或者等价地$R_x\supset L_y$. 如果包含是真包含, 那么就有严格的不等号$x<y$. 两个实数$x,y$之间有且只有下列三种关系之一: $x<y$, $x=y$, $x>y$. 这是因为两个下类之间有且只有$L_x\subsetneqq L_y$, $L_x=L_y$, $L_x\supsetneqq L_y$三种关系之一.

\begin{exercise}{}
证明这一点. 提示: 回看下类的定义, 如果存在$l\in L_y$使得$L_x$里所有的元素都比$l$小, 那么可以说明什么? 如果不存在, 那又会怎样?
\end{exercise}

由此, 实数集是一个\textbf{全序集 (totally ordered set)}: 任何两个元素之间都一定可以比较大小. 这一点跟有理数区别不大.

\subsubsection{加法}

我们可以定义
$$
x+y=(L_x+L_y)\cup(R_x+R_y),
$$
其中$A+B$定义为$\{a+b:a\in A, b\in B\}$. 根据下类的定义, 不难验证$L_x+L_y$仍然是一个下类, 由此