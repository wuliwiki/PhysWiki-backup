% 南京理工大学 2009 年 研究生入学考试试题 普通物理(B)
% license Usr
% type Note

\textbf{声明}:“该内容来源于网络公开资料,不保证真实性,如有侵权请联系管理员”

\subsection{一、填空题(每题 2 分,共 30 分)}
1. 一半径为 $a$ 的金属球带电 $Q$,其周围充满介电常数为$ 的均匀无限大介质,则金属球内部的电场能量为__________,金属球外内部的电场能量为______________。

2. 一无限长载流直线 I 与一载流矩形回路共面,其尺寸如图所示,则载流线圈
受到的力矩大小为___________;电流 I 激发的磁场通过回路的磁通量________。

3. 互感系数的物理意义是________。

4. 油轮漏油(n=1.2)入海,在海面上形成一大片油膜,如有人在膜厚为 480nm的油膜上空的飞机上垂直往下看,能看到 =_________nm 的光;如有人在 45 度方向往油膜看,又能看到 =___________nm 的光。(海水的折射率为 1.33)

5. 一部分偏振光由线偏振光和自然光组成,让该部分偏振光经过一可旋转的线偏振片后,得到 Imax/Imin=5/2,则该部分偏振光中线偏振光的比例为________;如用自然光通过,则 Imax/Imin=______________。

6. 宽度为 a 的一维无限深势阱中,粒子的波函数为: ,则该粒子在势阱中出现的几率密度表达式为 P=______________,若 n=2 时,粒子在x=________________处出现的概率最大。

7. 电子的静止质量是 m0,当电子以 v=0.8c 的速度运动时,它的运动动能为____________,总能量为________________。

8. 在氢原子光谱的莱曼系(n=1)中,最短波长为_______nm,最长波长为_______nm。