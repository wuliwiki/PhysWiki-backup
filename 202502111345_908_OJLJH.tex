% 欧几里得几何(综述)
% license CCBYSA3
% type Wiki

本文根据 CC-BY-SA 协议转载翻译自维基百科\href{https://en.wikipedia.org/wiki/Euclidean_geometry#}{相关文章}。

\begin{figure}[ht]
\centering
\includegraphics[width=6cm]{./figures/29202c1c81a65fe8.png}
\caption{} \label{fig_OJLJH_1}
\end{figure}
欧几里得几何是归功于古希腊数学家欧几里得的数学体系,他在其几何学教材《几何原本》中对其进行了描述。欧几里得的方法是假设一小组直观上令人信服的公设(公理),并从这些公理中推导出许多其他命题(定理)。尽管欧几里得的许多结果早已被提出,[1] 但他是第一个将这些命题组织成一个逻辑系统的人,其中每个结果都是从公理和先前证明的定理推导出来的。[2]

《几何原本》以平面几何开始,至今仍在中学(高中)教授,作为第一个公理化系统和数学证明的初步例子。它接着讲解了三维的立体几何。《几何原本》中的许多内容阐述了现在被称为代数和数论的结果,用几何语言来表达。[1]

在超过两千年的时间里,“欧几里得”这个形容词是不必要的,因为欧几里得的公理似乎是如此直观明显(平行公设可能是唯一例外),以至于从这些公理中推导出的定理被认为是绝对正确的,因此没有其他类型的几何被认为是可能的。然而,今天许多自洽的非欧几里得几何已被发现,最早的几何形式是在19世纪初发现的。爱因斯坦的广义相对论理论的一个含义是,物理空间本身并非欧几里得空间,欧几里得空间仅在短距离内(相对于引力场的强度)对其进行良好近似。[3]