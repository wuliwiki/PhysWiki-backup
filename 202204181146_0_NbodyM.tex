% N 体问题的数值计算(Matlab)

\begin{issues}
\issueDraft
\end{issues}

\pentry{四阶龙格库塔法\upref{OdeRK4}}

数值解常微分方程组
\begin{equation}
\bvec a_i = \ddot{\bvec r_i} = \frac{\bvec F_i}{m_i}
\end{equation}
\begin{equation}
\bvec F_i = G\sum_{j\ne i} \frac{m_i m_j(\bvec r_j - \bvec r_i)}{\abs{\bvec r_j - \bvec r_i}^3}
\end{equation}

算法 “四阶龙格库塔法\upref{OdeRK4}” 中的 \verb|keplerRK4.m| 相似, 只是使用了 Matlab 自带的变步长龙格库塔解算器 \verb|ode45|, 效率更高. 另外当两个天体距离太近, 使得 $\bvec F_i$ 超过了双精度的最大值时, 程序将报错终止.

当设置 \verb|Ndim = 2|, 程序中所有的位置、速度、力都是二维矢量, 当 \verb|Ndim = 3| 时都是三维矢量.

\addTODO{该代码未完成}
\begin{lstlisting}[language=matlab]
% N 体问题(支持二维或三维)
function n_body
% ==== 参数设置 ====
Ndim = 2; % 2 维或 3 维运动
r0 = []; % 初始位置, Ndim*N
v0 = []; % 初始速度, Ndim*N
m = []; % 1*N
% =================
Y0 = [v0(:), r0(:)];
options = odeset('RelTol', RelTol);
[T, Y] = ode45(@(t,Y)ode_fun(t, Y, m, G), ...
        [tmin,tmax], Y0, options);
r = reshape(Y(1:Ndim*N), Ndim, N);
v = reshape(Y(Ndim*N+1:end), Ndim, N);
% TODO: 做动画
end

% 已知天体位置和速度, 计算加速度
function dY = ode_fun(~, Y, m, G, Ndim)
N = numel(Y)/6;
r = reshape(Y(1:Ndim*N), Ndim, N);
v = reshape(Y(Ndim*N+1:end), Ndim, N);
a = zeros(N, 1);
for i = 1:N
    F = 0;
    for j = 1:N
        if j == i, continue; end
        r_ij = r(:,i)-r(:,j); d_ij = abs(r_ij);
        F = F + G*m(i)*m(j)*r_ij/d_ij^3;
        if isinf(F)
            error('发生碰撞!');
        end
    end
    a(i) = F/m(i);
end
dY = [a; v(:)];
end
\end{lstlisting}
