% MacOS 编程笔记
% license Usr
% type Note

\begin{issues}
\issueDraft
\end{issues}

\begin{figure}[ht]
\centering
\includegraphics[width=12cm]{./figures/a119070580b69336.png}
\caption{动态和静态链接库拓展名} \label{fig_MacDev_1}
\end{figure}

\begin{itemize}
\item 命令行的许多命令不是完全相同的,会有很多兼容问题,脚本或者 makefile 是大坑
\item 要用 C++, 先安装 xcode, \item \verb`g++` 实际上是 Xcode 提供的 Clang++,而且不是满血的,例如不支持 OpenMP。 GPT 说用 OpenMP 需要装 brew 版的 llvm
\item \verb`CPATH` 和 \verb`LD_LIBRARY_PATH` 都有效
\item \verb`file 文件` 可以判断文件类型包括二进制可执行文件
\item \verb`otool -L 可执行文件` 可以查看它使用的动态链接库
\item Linux 二进制文件和动态库文件都是 EFL 格式, macOS 对应的是 \verb`mach-o`
\item \verb`BLAS` 和 \verb`LAPACK` 都是 apple acceleration framework 最专业
\item HomeBrew\upref{Homebr} 是必备
\item 用于管理服务的命令是 \verb`launchctl`
\item \verb`locate` 命令有自带,但不是 linux 的,初次运行提示运行一个 \verb`launchctl` 命令创建数据库
\end{itemize}
