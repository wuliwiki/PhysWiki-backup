% 线性算子代数
% 线性算子|算子代数

\begin{issues}
\issueTODO
\end{issues}

\subsection{线性算子}
\pentry{多重线性映射\upref{MulMap},矩阵与线性映射\upref{MatLS}}
域 $\mathbb{F}$ 上所有从 $n$ 维矢量空间 $V$ 到 $m$ 维矢量空间 $W$ 的线性映射 $f:V\rightarrow W$ 的集合用符号 $\mathcal{L}(V,W)$ (或者 $\mathrm{Hom}(V,W)$) 表示,它仍是一个矢量空间\upref{MulMap},其上的一个线性映射和一个 $m\times n$ 的矩阵一一对应\upref{MatLS}.在 $V=W$ 的情形,矢量空间 $\mathcal{L}(V,W)$ 简记为 $\mathcal{L}(V)$ (或 $\mathrm{End}(V)$),它的向量通常称为\textbf{线性算子}.

\textbf{符号约定:}在线性代数部分,线性算子将用拉丁字母 $\mathcal{A,B,C,\cdots}$ 表示,而在矢量空间 $V$ 的基底 $(\bvec e_i)$ 之下对应的矩阵用字母 $A,B,C,\cdots$ 表示,另一基底 $(\bvec e_i)$ 之下对应矩阵则用字母 $A',B',C',\cdots$ 表示.总是用 $\mathcal{E}=\mathrm{Id}$ 和 $E=(\delta_{ij})$ 表示恒等(单位)映射 $\bvec x\mapsto \bvec x$ .算子 $\mathcal{A}$ 作用在 $\bvec x$ 上的结果简写成 $\mathcal{A}\bvec x$ (代替 $\mathcal{A}(\bvec x)$ ).

线性算子 $\mathcal{B}$ 称为 $\mathcal{A}$ 的\textbf{逆算子},如果 $\mathcal{AB}=\mathcal{BA}=\mathcal{E}$ .算子 $\mathcal{A}$ 的逆算子通常记为 $\mathcal{A}^{-1}$.由\autoref{MatLS2_cor1}~\upref{MatLS2},$\mathcal{A}^{-1}$ 存在等价于 $\mathrm{Ker}\mathcal{A}=0$ 或者 $\mathrm{dim}\;V=\mathrm{dim\;Im}\mathcal{A}$ .$\mathrm{dim\;Ker}\mathcal{A}$ 称为 $\mathcal{A}$ 的\textbf{亏数}.
\begin{example}{零算子}
零算子 $\mathcal{O}$ 把每个向量 $\bvec v\in V$ 都变成零:$\mathrm{rank}\; \mathcal{O}=0$
\end{example}
\begin{example}{相似算子}
$\mathcal{A}\bvec x=\lambda\bvec x$,其中 $\lambda\in\mathbb{F}$.
\end{example}
\begin{example}{投影算子}
设 $V=U\oplus W$,则 $\bvec x=\bvec x_U+\bvec x_W$ 且 $\mathcal{P}\bvec x=\bvec x_U$,那么称 $\mathcal{P}$ 为\textbf{投影算子}或在子空间 $U$ 平行于 $W$ 的\textbf{投影}.显然 $\mathcal{P}^2=\mathcal{P}$
\end{example}
\subsection{算子代数}
根据线性映射的数乘,加法运算\upref{MulMap},及映射复合\autoref{map_def4}~\upref{map},可令
\begin{equation}\label{LiOper_eq1}
(\mathcal{A}+\mathcal{B})\bvec x=\mathcal{A}\bvec x+\mathcal{B}\bvec x,\quad (\lambda\mathcal A)\bvec x=\lambda(\mathcal A\bvec x),\quad (\mathcal{AB})\bvec x=\mathcal{A}(\mathcal{B}\bvec x)
\end{equation}
也就是说,复合 $\mathcal{A\circ B}$ 可直接表达为 $\mathcal{AB}$ .

由\autoref{LiOper_eq1} 可直接验证
\begin{equation}
\begin{aligned}
&\alpha(\mathcal{A+B})=\alpha\mathcal{A}+\alpha\mathcal{B}\\
&(\alpha+\beta)\mathcal{A}=\alpha\mathcal{A}+\beta\mathcal{A}\\
&(\alpha\beta)\mathcal{A}=\alpha(\beta\mathcal{A})\\
&1\cdot \mathcal{A}=\mathcal A\\
&\mathcal{A}(\mathcal{BC})=(\mathcal{AB})\mathcal C\quad(\text{结合律})\\
&\mathcal A(\mathcal{B+C})=\mathcal{AB+AC},\quad (\mathcal{A+B})\mathcal C=\mathcal{AC+BC}\quad(\text{分配律})\\
&\lambda(\mathcal{AB})=(\lambda\mathcal{A})\mathcal{B}=\mathcal{A}(\lambda \mathcal B)
\end{aligned}
\end{equation}
我们看到,$\mathcal{L}(V)$不仅是个矢量空间,同时也是个结合环\autoref{Ring_def2}~\upref{Ring} ,最后的关系式建立了纯量和算子之间乘法的补充定律.这样一个满足补充定律 $\lambda(ab)=(\lambda a)b=a(\lambda b)$ , 又是环又是域 $\mathbb{F}$ 上的向量空间 $K$ ,就称为域 $\mathbb{F}$ 上的\textbf{代数},\autoref{AlgFie_def1}~\upref{AlgFie}.$K$ 作为矢量空间的维数即称为代数 $K$ 的\textbf{维数}.

\begin{theorem}{}\label{LiOper_the1}
如果
\begin{equation}
\mathcal{A}:\bvec e_k\mapsto \mathcal{A}\bvec e_k=\sum_i^{n}a_{ik}\bvec e_i,\quad \mathcal{B}:\bvec e_j\mapsto  \mathcal{B}\bvec e_j=\sum_{k=1}^n b_{kj}\bvec e_k
\end{equation}
是线性空间 $V$ 在基底 $(\bvec e_i)$ 之下以 $A=(a_{ij}), B=(b_{kj})$ 为矩阵的线性算子,那么,算子 $\mathcal{AB}$ 在同一基底下的矩阵是 $C=AB$
\end{theorem}
\textbf{证明:}
\begin{equation}
\begin{aligned}
\sum_i c_{ij}\bvec e_i&=(\mathcal{AB})\bvec e_j=\mathcal{A}(\mathcal{B}\bvec e_j)=\mathcal{A}\qty(\sum_k b_{kj}\bvec e_k)=\sum_k b_{kj}\mathcal{A}\bvec e_k\\
&=\sum_k b_{kj}\sum_i a_{ik}\bvec e_i=\sum_{i,k}a_{ik}b_{kj}\bvec e_i=AB\bvec e_i
\end{aligned}
\end{equation}
\textbf{证毕}!
\subsection{不同基底下线性算子对应的矩阵}
\begin{theorem}{}\label{LiOper_the2}
若线性算子 $\mathcal A$ 在基底 $(\bvec e_1\cdots \bvec e_n)$ 下对应矩阵为 $A$,则在另一基底  $(\bvec e'_1\cdots \bvec e'_n)$ 之下对应的矩阵 $A'$ 为
\begin{equation}
A'=B^{-1}AB
\end{equation}
其中 $B$ 为基底 $(\bvec e_i)$ 向基底 $(\bvec e_j')$ 的过渡矩阵\upref{TransM}.
\end{theorem}
\textbf{证明:}
由定理条件,若设 $A=(a_{ij}), A'=(a_{kj}'),B=(b_{ij})$ ,则
\begin{equation}
\begin{aligned}
&\mathcal{A}\bvec e_i=\sum_{k} a_{ki}\bvec e_k
\\
&\mathcal{A}\bvec e_j'=\sum_{k} a'_{kj}\bvec e'_k\\
&\bvec e_j'=\sum_i b_{ij}\bvec e_i
\end{aligned}
\end{equation}
引入算子 $\mathcal{B}$ ,它在基底 $(\bvec e_1\cdots \bvec e_n)$ 下对应的矩阵为 $B$ ,那么
\begin{equation}
\mathcal{B}\bvec e_j=\sum_i b_{ij}\bvec e_i=\bvec e_j'
\end{equation}

由于线性算子与矩阵之间在固定基底之下一一对应,所以可定义一算子 $\mathcal{A'}$ ,它在基底 $(\bvec e_1\cdots \bvec e_n)$ 之下对应的矩阵为 $A'$ ,即
\begin{equation}
A'\bvec e_j=\sum_i a'_{ij}\bvec e_i
\end{equation}
于是
\begin{equation}
\mathcal{AB}\bvec e_j=\mathcal{A}\bvec e'_j=\sum_i a'_{ij} \bvec e'_i=\sum_i a'_{ij}\mathcal{B}\bvec e_i=\mathcal{B}\qty(\sum_i a'_{ij}\bvec e_i)=\mathcal{BA'}\bvec e_j
\end{equation}
于是 
\begin{equation}
\mathcal{A'}=\mathcal{B}^{-1}\mathcal{AB}
\end{equation}
由\autoref{LiOper_the1} ,上式对应的矩阵的形式就为
\begin{equation}
A'=B^{-1}AB
\end{equation}
\textbf{证毕}!

\autoref{LiOper_the2} 表明,每一个线性算子都对应一个相似矩阵的集合,而其中每一矩阵都相当于同一线性算子在不同基底下的矩阵.
