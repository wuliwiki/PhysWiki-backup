% 费马小定理(综述)
% license CCBYNCSA3
% type Wiki

本文根据 CC-BY-SA 协议转载翻译自维基百科\href{https://en.wikipedia.org/wiki/Fermat\%27s_little_theorem}{相关文章}。

在数论中,费马小定理指出:如果 $p$ 是一个质数,那么对于任意整数 $a$,数 $a^p - a$ 都是 $p$ 的整数倍。用模运算的记号表示,就是:
$$
a^p \equiv a \pmod{p}~
$$
例如,当 $a = 2$、$p = 7$ 时,有 $2^7 = 128$,而 $128 - 2 = 126 = 7 \times 18$,正好是 7 的倍数。

如果 $a$ 不能被 $p$ 整除,也就是说 $a$ 与 $p$ 互素,那么费马小定理等价于以下陈述:$a^{p-1} - 1$ 是 $p$ 的整数倍,或者用符号表示为:
$$
a^{p-1} \equiv 1 \pmod{p}~
$$
例如,当 $a = 2$、$p = 7$ 时,有 $2^6 = 64$,而 $64 - 1 = 63 = 7 \times 9$,也是 7 的倍数。

费马小定理是费马素性检验的理论基础,也是初等数论中的一个基本定理。该定理得名于皮埃尔·德·费马,他在 1640 年提出了这一结论。之所以称为“小定理”,是为了将其与费马大定理区分开来。
\subsection{历史}
