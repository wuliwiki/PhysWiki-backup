% LSZ约化公式
% LSZ约化公式|散射理论|编时格林函数

\subsubsection{入态和出态的构造}
定义 $\tilde{\phi}(p)$ 为海森堡场 $\phi(x)$ 的傅里叶分量.
\begin{equation}
\begin{aligned}
\tilde\phi(p)=\int \dd[4]{x} e^{-ipx} \phi(x)
\end{aligned}
\end{equation}
计算能量动量算符 $P^\mu$ 与它的对易子,可以得到

\begin{equation}
\begin{aligned}
[P^\mu, \tilde\phi(p)] &=\int \dd[4]{x} e^{-ipx} [P^\mu,\phi(x)]=-\int \dd[4]{x} e^{-ipx} i\partial^\mu \phi(x)\\
&=\int \dd[4]{x} \phi(x) i\partial^\mu e^{-ipx}=p^\mu \tilde\phi(p)
\end{aligned}
\end{equation}
这意味着 $P^\mu \tilde \phi(p)\ket{\Omega}=p^\mu \tilde{\phi}(p)\ket{\Omega}$,也就是说 $\tilde\phi (p^\mu) \ket{\Omega}$ 要么是 $0$,要么是能动量本征值为 $p^\mu$ 的态.进一步可以证明,由该算符构造的态矢量所构成的集合中,$p^2$ 相同的态矢量之间由洛伦兹变换相联系.我们在讨论物理态的性质时,已经对物理态的能谱有一系列假定,例如单粒子态为场的最低能激发,其能动量满足质壳条件 $p^2=m^2$;多粒子态的能谱位于 $p^2\ge (2m)^2$ 这一连续区域;可能存在束缚态的能谱在位于 $m$ 到 $2m$ 之间孤立的双曲面上.这一系列假定意味着上面构造的 $\phi(p^\mu)\ket{\Omega}$ 仅当 $p^2$ 满足一定条件时才是非 $0$ 的态.现在我们希望构造单粒子态,假定 $p^2$ 在 $m^2$ 附近,当 $p^2$ 恰好等于 $m^2$ 时 $\tilde{\phi}(p^\mu)\ket{\Omega}$ 才不为零.将它与单粒子态 $\ket{\bvec k}$ 作内积:
\[
\bra{\bvec k} \tilde\phi(p^\mu) \ket{\Omega}=\int \dd[4]{x} e^{-ikx}\bra{\bvec k}\phi(x)\ket{\Omega}
\]
利用 $\phi(x)=e^{iPx}\phi(0)e^{-iPx}$ 和 $U\phi(0)U^{-1}=\phi(0)$(其中 $U$ 是洛伦兹变换,$U^\dagger U=1,U |\bvec k\rangle = |\bvec k'\rangle$),可以得到
\begin{equation}\begin{aligned}
\bra{\bvec k}\phi(x)\ket{\Omega}&=
\bra{\bvec k}e^{iP\cdot x}\phi(0) e^{-i P\cdot x}\ket{\Omega} = \bra{\bvec k}\phi(0)\ket{\Omega} e^{ik\cdot x}|_{k^0=E_{\bvec k}} \\
&=\bra{\bvec k}U^\dagger U\phi(0)U^\dagger U\ket{\Omega} e^{ik\cdot x}|_{k^0=E_{\bvec k}}\\
&=\bra{\bvec k'}\phi(0)\ket{\Omega} e^{ik\cdot x}|_{k^0=E_{\bvec k}}
\end{aligned}\end{equation}
代入 $\bra{\bvec k}\phi(p^\mu)\ket{\Omega}$,可以得到
\[
\bra{\bvec p}\tilde\phi(p^\mu)\ket{\Omega}=(2\pi)^4\delta(p^0-E_{\bvec k})\delta(\bvec p-\bvec k) \sqrt{Z}
\]
其中 $\sqrt{Z}=\bra{\bvec k'}\phi(0)\ket{\Omega}$,由洛伦兹变换,当 $\ket{\bvec k'}$ 取任意单粒子态时它都是常数.可以调整单粒子态的相因子来使 $\sqrt{Z}$ 为正实数.将上式写成洛伦兹协变的形式:
\[
\bra{\bvec k}\tilde\phi(p^\mu)\ket{\Omega}=(2\pi)^4 \sqrt{Z} (2E_{\bvec k})\delta(p^2-m^2)\theta(p^0) \delta(\bvec p-\bvec k)
\]
因此当 $p^\mu$ 满足单粒子态的在壳条件 $p^2=m^2,p^0>0$ 时,$\tilde \phi(p^\mu)\ket{\Omega}$ 一定正比于 $\ket{\bvec p}$.具体地,可以由物理态的完备关系得到
\begin{equation}\begin{aligned}
\tilde \phi(p^\mu)\ket{\Omega}=&|\Omega\rangle \langle \Omega|\tilde \phi(p^\mu)\ket{\Omega} + \sum_{\lambda}\int \frac{\dd[3] \bvec p}{(2\pi)^3}\frac{1}{2E_{\bvec p}(\lambda)}|\lambda_{\bvec p}\rangle \langle \lambda_{\bvec p}|\tilde \phi(p^\mu)\ket{\Omega}\\ \overset{p^2\approx m^2}{=}&(2\pi)^4 \delta^4(p)\ket{\Omega}\bra{\Omega}\phi(0)\ket{\Omega}
+
\int \frac{\dd[3]{\bvec k}}{(2\pi)^3}\frac{1}{2 E_{\bvec k}} \ket{\bvec k}\bra{\bvec k}\tilde \phi(p^\mu)\ket{\Omega}\\
=&\eta(2\pi)^4\delta^4(p)\ket{\Omega}+2\pi \sqrt{Z} \delta(p^2-m^2)\theta(p^0) \ket{\bvec p}
\end{aligned}\end{equation}

现在我们希望构造一系列算符 $C_\alpha$ 来构造入态的产生算符.具体地,我们希望一个 $n$ 粒子的入态 $\ket{\psi^+}$ 可以写成 $C_1C_2\cdots C_n \ket{\Omega}$,出态 $\ket{\psi^-}$ 可以写成 $D_1D_2\cdots D_n \ket{\Omega}$.以入态为例,算符 $C_\alpha$ 是由 $\tilde \phi(k)$ 调制而成的波包,它作用于真空态产生动量为 $\bvec k_\alpha$ 的单粒子态.
\[
C_\alpha=\int \dd[4]{x_\alpha} u_\alpha(x_\alpha)\phi(x_\alpha) = \int \frac{\dd[4] l}{(2\pi)^4} \tilde{u}_\alpha(-l)\tilde\phi(l)
\]
$\tilde{u}_\alpha(-l)$ 为波包的调制函数,在 $l=(+E_{\bvec k_\alpha},\bvec k_\alpha)$ 的一个小邻域内非零.注意到 $(\tilde{\phi}(l))^\dagger=\tilde{\phi}(-l)$,因此 $C_\alpha^\dagger$ 的作用则是湮灭单粒子态.考虑用 $u_\alpha(x_\alpha)=F(x_\alpha)g(t_\alpha-T)$ 形式的调制函数来构造入态和出态,$g(t)$ 随 $|t|$ 增大而衰减,我们约定它的宽度远大于 $m^{-1}$,相应的其傅里叶变换 $\tilde g(\omega)$ 是 $\omega=0$ 附近宽度远小于 $m$ 的一个尖峰.$F(x)$ 是具有以下形式的相对论性波函数:
\[
F(x)=\int \frac{\dd[3]{\bvec l}}{(2\pi)^3} f(\bvec l) e^{-i\omega_{\bvec l}t+i\bvec l\cdot \bvec x}
\]
在物理意义上,入态是指在无穷远过去互相分离的波包,那么我们将算符定义在负无穷时刻,即 $g(t-T_-)$ 在 $t\approx T_-\rightarrow -\infty$ 处非零;出态是在无穷远未来互相分离的波包,那么 $g(t-T_+)$ 在 $t\approx T_+\rightarrow +\infty$ 处非零.约定 $T_+,T_-$ 远大于 $g(t)$ 的宽度.现在考虑波包调制函数的傅里叶分量 $\tilde{u}_\alpha(-l)$:
\begin{equation}\begin{aligned}
\int \dd[4]{x}F(x)g(t-T)e^{ilx}&=\int \dd t \dd[3]{x} e^{il^0t - i\bvec l\cdot \bvec x}\int \frac{\dd[3]{\bvec p}}{(2\pi)^3}f(\bvec p)e^{-i\omega_{\bvec p}t+i\bvec p\cdot \bvec x}\int \frac{\dd \nu}{2\pi}\tilde g(\nu) e^{-i\nu(t-T)}\\
&=\int \frac{\dd[3]\bvec p}{(2\pi)^3}\frac{\dd \nu}{2\pi} f(\bvec p)\tilde g(\nu) (2\pi)^3 \delta(\bvec p-\bvec l) 2\pi \delta(l^0-\omega_{\bvec p}-\nu)e^{i\nu T}\\
&=f(\bvec l) \tilde g(l^0-\omega_{\bvec l})e^{i(l^0-\omega_{\bvec l})T}
\end{aligned}\end{equation}
由此可以看到,$g(t)$ 的宽度远大于 $m^{-1}$ 导致了当 $l^0$ 在 $\omega_{\bvec l}$ 附近上式才不为零,由此构造的算符 $C_\alpha$ 或 $D_\alpha$ 不会贡献多粒子态和真空态.算符 $C_\alpha$ 或 $D_\alpha$ 作用到真空态后,只有当 $l$ 满足在壳条件,即 $l^0=\omega_{\bvec l}$ 时,$\tilde{u}_\alpha(-l)\tilde \phi(l) \ket{\Omega}$ 才会贡献对单粒子态的分量有贡献.最终该算符作用于真空态得到的就是,能壳 $p^2=m^2$ 上 $\bvec k_\alpha$ 附近所对应的单粒子态的光滑叠加.此外,对于不同的产生算符,我们可以假定其动量 $\bvec k_\alpha$ 互不相同,那么不同的波包在无穷远过去(或未来)空间距离间隔很远,因此不同的 $C_\alpha$(或不同的 $D_\alpha$)彼此对易,产生的单粒子态是互相独立的.由于这样我们就通过构造波包得到了入态和出态.
\begin{equation}\begin{aligned}
&C_\alpha=\int \dd[4]{x_\alpha} u_\alpha(x_\alpha)\phi(x_\alpha)=g(t_\alpha-T_-)\int \frac{\dd[3]{\bvec p}}{(2\pi)^3} f_\alpha(\bvec p)e^{-i\omega_{\bvec p}t_\alpha+i\bvec p\cdot \bvec x_\alpha},\ T_-\rightarrow -\infty\\
&D_\alpha=\int \dd[4]{x'_\alpha} u'_\alpha(x'_\alpha)\phi(x'_\alpha)=g(t'_\alpha-T_+)\int \frac{\dd[3]{\bvec p}}{(2\pi)^3} f'_\alpha(\bvec p)e^{-i\omega_{\bvec p}t'_\alpha+i\bvec p\cdot \bvec x'_\alpha},\ T_+\rightarrow +\infty\\
&\ket{\psi^+}=C_1C_2\cdots C_n\ket{\Omega},\ \ket{\psi^-}=D_1D_2\cdots D_{n'}\ket{\Omega}\\
\end{aligned}\end{equation}
\subsubsection{LSZ 约化公式}

S 矩阵被定义为入态和出态的内积,因此可以用 $C_\alpha,D_\alpha$ 表示为
\begin{equation}\label{LSZ_eq1} \begin{aligned}
\langle\psi^-|\psi^+\rangle=\langle\bvec k'_1 \cdots \bvec k'_{n'}|S|\bvec k_1\cdots\bvec k_n\rangle=\bra{\Omega} T D_1^\dagger \cdots D_{n'}^\dagger C_1\cdots C_n \ket{\Omega}
\end{aligned}\end{equation} 
上面插入了编时算符 $T$ 不改变表达式,因为 $D_\alpha$ 出现在无穷远未来而 $C_\alpha$ 出现在无穷远过去.定义两组新的算符 $\bar C_\alpha$ 和 $\bar D_\alpha$:
\begin{equation} \begin{aligned}
&\bar C_\alpha=g(t_\alpha-T_+)\int \frac{d^3{\bvec p}}{(2\pi)^3} f_\alpha(\bvec p)e^{-i\omega_{\bvec p}t_\alpha+i\bvec p\cdot \bvec x_\alpha},\ T_-\rightarrow -\infty\\
&\bar D_\alpha=g(t'_\alpha-T_-)\int \frac{d^3{\bvec p}}{(2\pi)^3} f'_\alpha(\bvec p)e^{-i\omega_{\bvec p}t'_\alpha+i\bvec p\cdot \bvec x'_\alpha},\ T_+\rightarrow +\infty
\end{aligned}\end{equation}
如果将 \autoref{LSZ_eq1}  中的某个 $C_\alpha$ 替换为 $\bar{C}_\alpha$,那么在编时算符的作用下它被移到最左侧,于是湮灭 $\bra{\Omega}$.如果将某个 $D_\alpha^\dagger$ 替换为 $\bar{D}_\alpha^\dagger$,则它被移到最右侧后湮灭 $\ket{0}$.因此可以将 \autoref{LSZ_eq1}  改写为
\[
\bra{\Omega} T (D_1^\dagger-\bar{D}_1^\dagger) \cdots (D_n^\dagger-\bar{D}_n^\dagger) (C_1-\bar{C}_1)\cdots (C_n-\bar{C}_n) \ket{\Omega}
\]
算符 $C_\alpha$ 与 $\bar{C}_\alpha$ 的波包调制函数,其 $f_\alpha(\bvec p)$ 分量是完全相同的.对于在壳的 $l$,两个波包调制函数的在壳傅里叶分量 $\tilde{u}_\alpha(-l)$ 是相同的.对于自由标量场论来说,满足自由 Klein-Gordon 方程的场算符只有在壳的傅里叶分量,那么 $C_\alpha$ 和 $\bar{C}_\alpha$ 将没有差别.这意味着两个算符 $C_\alpha$ 与 $\bar{C}_\alpha$ 的差别来自于理论的相互作用,可以对它作具体的计算:
\begin{equation} \begin{aligned}
&C_\alpha-\bar{C}_\alpha=\int d^4  x\left[g\left(t-T_{-}\right)-g\left(t-T_{+}\right)\right] \int \frac{d^3 {\bvec p}}{(2 \pi)^3} f_\alpha(\mathbf{p}) e^{-i \omega_{\mathbf{p}} t+i \bvec p \cdot \bvec x} \phi(x)\\
&=\int \frac{d^3{\bvec p}}{(2 \pi)^3} \frac{\dd \nu}{2 \pi} f_\alpha(\mathbf{p}) \tilde{g}(\nu) \int d^4  x\left[e^{-i \nu\left(t-T_{-}\right)}-e^{-i \nu\left(t-T_{+}\right)}\right] e^{-i \omega_{\mathrm{p}} t+i \bvec p \cdot \bvec{x}} \phi(x)\\
&=\int \frac{d^3{\bvec p}}{(2 \pi)^3} \frac{\dd \nu}{2 \pi} f_\alpha(\mathbf{p}) \tilde{g}(\nu) \frac{\left(e^{i \nu T_{-}}-e^{i \nu T_{+}}\right)}{m^2-\left(\omega_{\mathbf{p}}+\nu\right)^2+|\mathbf{p}|^2} \int d^4  x\left[\left(m^2+\partial^2\right) e^{-i\left(\omega_{\mathbf{p}}+\nu\right) t+i \mathbf{p} \cdot \mathbf{x}}\right] \phi(x)\\
&=\int \frac{d^3{\bvec p}}{(2 \pi)^3} \frac{\dd \nu}{2 \pi} f_\alpha(\mathbf{p}) \tilde{g}(\nu) \frac{\left(e^{i \nu T_{-}}-e^{i\nu T_{+}}\right)}{-2\omega_{\bvec p}\nu - \nu^2} \int d^4  x e^{-i\left(\omega_{\mathbf{p}}+\nu\right) t+i \mathbf{p} \cdot \mathbf{x}}\left[\left(m^2+\partial^2\right) \phi(x)\right]\\
&=\int \frac{d^3{\bvec p}}{(2 \pi)^3} \frac{\dd \nu}{2 \pi} f_\alpha(\mathbf{p}) \tilde{g}(\nu) \frac{2 \pi i \delta(\nu)}{2 \omega_{\mathbf{p}}} \int d^4  x e^{-i\left(\omega_{\mathbf{p}}+\nu\right) t+i \bvec p \cdot \bvec{x}}\left[\left(m^2+\partial^2\right) \phi(x)\right]\\
&=i \int \frac{d^3{\bvec p}}{(2 \pi)^3 2 \omega_{\mathbf{p}}} f_\alpha(\mathbf{p}) \int d^4 x e^{-ipx}\left[\left(m^2+\partial^2\right) \phi(x)\right]
\end{aligned}\end{equation}
其中第四行到第五行的推导利用了 $\tilde{g}(\nu)$ 宽度远小于 $m$ 的性质将分母中的 $\nu^2$ 当作无穷小量略去,当 $T_+\rightarrow +\infty,T_-\rightarrow -\infty$ 时 $(e^{i\nu T_-}-e^{i\nu T_+})/(-\nu)$ 可以改写为 $2\pi i\delta(\nu)$.在最后一步中我们取定了 $\tilde g(0)=\int \dd{t} g(t)=1$,如果它不为 $1$,我们可以将系数吸收进 $f_\alpha(\bvec p)$ 中.用类似的方法可以计算 $D_\alpha^\dagger-\bar{D}_\alpha^\dagger$.
\[
D_\alpha^\dagger-\bar{D}_\alpha^\dagger=i \int \frac{d^3{\bvec p}}{(2 \pi)^3 2 \omega_{\mathbf{p}}} f'^{*}_\alpha(\mathbf{p}) \int d^4 x e^{ipx}\left[\left(m^2+\partial^2\right) \phi(x)\right]
\]
现在离 LSZ 约化公式的导出只差一步之遥.S-矩阵依赖于归一化的选择,也就依赖于 $f_\alpha(\bvec p)$ 和 $f'\alpha(\bvec p)$,因此我们需要对 $C_\alpha\ket{\Omega}$ 和 $D_\alpha\ket{\Omega}$ 进行适当的归一化.利用 autoref47 和 autoref48 以及 $\tilde g(0)=1$,可以得到:
\begin{equation} \begin{aligned}
C_\alpha\ket{\Omega}&=\int \frac{d^4 {l}}{(2\pi)^4} \tilde{u}_\alpha(-l)\tilde{\phi}(l)\ket{\Omega}\\
&=\sqrt{Z}\int \frac{d^3{\bvec p}}{(2\pi)^3 2\omega_{\bvec p}} \left.\tilde{u}_\alpha(-p)
\right|_{p^0=\omega_{\bvec p}} \ket{\bvec{p}}\\
&=\sqrt{Z}\int \frac{d^3{\bvec p}}{(2\pi)^3 2\omega_{\bvec p}} f_\alpha(\bvec p) \ket{\bvec{p}}
\end{aligned}\end{equation}
$D_\alpha\ket{\Omega}$ 也是类似的.为了使这些构造的入态和出态具有和物理态一样的归一化条件,可以约定
\begin{equation} \begin{aligned}
&f_\alpha(\bvec p)=Z^{-1/2}(2\pi)^3 2\omega_{\bvec p}\delta(\bvec p-\bvec k_\alpha)\\
&f'_\alpha(\bvec p)=Z^{-1/2}(2\pi)^3 2\omega_{\bvec p}\delta(\bvec p-\bvec k'_\alpha)
\end{aligned}\end{equation}
那么由 $C_\alpha,D_\alpha$ 构造的入态和出态就是
\begin{equation} \begin{aligned}
\ket{\psi^+} &=C_1 \cdots C_n|\Omega\rangle=\left|\mathbf{k}_1, \cdots, \mathbf{k}_n\right\rangle_{\text {in }} \\
\ket{\psi^-} &=D_1 \cdots D_{n'}|\Omega\rangle=\left|\mathbf{k}_1^{\prime}, \cdots, \mathbf{k}_{n'}^{\prime}\right\rangle_{\text {out }}
\end{aligned}\end{equation}
最终我们得到 LSZ 约化公式:
\begin{equation} \begin{aligned}
\langle\psi^-|\psi^+\rangle&=\langle\bvec k'_1 \cdots \bvec k'_{n'}|S|\bvec k_1\cdots\bvec k_n\rangle= i^{n+n'} Z^{-\left(n+n'\right) / 2}
\prod_{i=1}^{n'}\int d^4 {x_i'} e^{i k_i' x_i'}\left(m^2+\partial_{x_i'}^2\right)  \\
& \prod_{j=1}^n\int d^4  x_j e^{-i k_j x_j}\left(m^2+\partial_{x_j}^2\right)
\left\langle \Omega\left|T \phi\left(x_1'\right) \cdots \phi\left(x_n'\right) \phi\left(x_{1}\right) \cdots \phi\left(x_{n}\right)\right| 0\right\rangle
\end{aligned}\end{equation} 