% 热动平衡判据
% keys 熵判据
% license Xiao
% type Tutor
\pentry{热平衡 热力学第零定律\nref{nod_TherEq}, 热力学第二定律\nref{nod_Td2Law}}{nod_f2de}

对于一个单元单相的孤立热力学系统(不考虑外场),它的平衡态意味着系统的\textbf{各个宏观性质在长时间内不发生变化},热平衡的判据为热学平衡、力学平衡、\enref{化学平衡}{TherEq}。为了能更好地将方法推广到更一般的系统(例如等温等压系统,例如单元复相系),来讨论相变和化学变化问题,我们需要具体地给出判据并在数学上进行分析。

\subsection{单元单相孤立系统的熵判据}

对于单元单相的孤立系统(体积 $V$,内能 $U$ 都不变),由\enref{熵增加定理}{Td2Law},我们可以利用\textbf{熵判据}判定孤立系统的某一状态是否为平衡态。也就是说,系统在任意的虚位移下,$\delta S=0$;熵还应当具有极大值,所以 $\delta^2S<0$,这是熵判据的稳定性条件。

虽然系统的内部可能宏大复杂的,但可以将系统“划分”成许多小部分,每一个小部分的内部 $P,V,T$ 近似处处相等,又仍有大量的微观粒子,这称为\textbf{子系统};对这样的子系统,熵是容易计算的。熵是广延量,将所有子系统的熵相加,可以得到整个孤立系统的熵。类似地,$U$ 是每个子系统内能之和,$V$ 是每个子系统体积之和。

现在设想系统发生一个“虚位移”,某两个子系统发生了变化:系统 $1$ 的内能变化为 $\delta U_1$,体积变化为 $\delta V_1$,系统 $2$ 的内能变化为 $\delta U_2$,体积变化为 $\delta V_2$。有 $\delta U_1+\delta U_2=0$,$\delta V_1+\delta V_2=0$。现在来看 $\delta S=\delta S_1+\delta S_2$:

\begin{align}
\delta S&=\delta S_1+\delta S_2=\frac{\delta U_1+P_1\delta V_1}{T_1}+\frac{\delta U_2+P_2\delta V_2}{T_2}
\\
&=\delta U_1\left(\frac{1}{T_1}-\frac{1}{T_2}\right)
+\delta V_1\left(\frac{P_1}{T_1}-\frac{P_2}{T_2}\right)=0~.
\end{align}
由于虚位移的 $\delta U_1$ 和 $\delta V_1$ 可以独立地改变,所以可以得到平衡条件:

\begin{equation}
T_1=T_2,\ \ \ P_1=P_2~,
\end{equation}
这恰好对应着热学平衡和力学平衡条件。下面我们对稳定性条件 $\delta^2 S<0$ 进行分析。经过一系列计算,可以得到平衡稳定性要求:
\addTODO{详细计算过程}
\begin{equation}
c_V>0,\ \ \ \left(\frac{\partial P}{\partial V}\right)_T<0~.
\end{equation}
这两条都是比较符合常理的——系统吸热后升温,压缩后增压。平衡的稳定性条件意味着,加入子系统由于涨落或某种外界影响而发生温度或体积变化,系统将自发地产生相应的过程以恢复平衡。

\subsection{单元系的复相平衡条件}
考虑单元两相系,用 $\alpha,\beta$ 表示两个相。$U^\alpha,V^\alpha,n^\alpha$ 表示 $\alpha$ 相的内能、体积、物质的量。用熵判据进行类似计算(具体的计算过程可以参考 \enref{相变平衡条件}{PhEquv}):
\begin{equation}
\delta S=\delta U^\alpha\left(\frac{1}{T^\alpha}-\frac{1}{T^\beta}\right)+\delta V^\alpha\left(\frac{P^\alpha}{T^\alpha}-\frac{P^\beta}{T^\beta}\right)
-\delta n^\alpha\left(\frac{\mu^\alpha}{T^\alpha}-\frac{\mu^\beta}{T^\beta}\right)~.
\end{equation}

由此我们可以得到三个平衡条件

\begin{equation}
T^\alpha=T^\beta,\ \ \ P^\alpha=P^\beta,\ \ \ \mu^\alpha=\mu^\beta~.
\end{equation}

其中第三个叫相变平衡条件,两相的\enref{{化学势}}{GibbsG}相等。例如,若一个冰水混合物系统处于相平衡,那么在这种温度压强情况下下冰的摩尔化学势一定等于水的摩尔化学势。利用这个原理,可以推出 $p-T$ 图上两相平衡曲线的斜率,得到\enref{{克拉伯龙方程}}{Clapey}。

化学势可以联系到态函数:\enref{吉布斯函数}{GibbsG}。吉布斯函数关于 $p,T$ 以及其他热力学参量的变化关系可以从其全微分公式中看出。通过相平衡条件就可以推导相平衡时各热力学参量之间的关系,这正是推导克拉伯龙方程(饱和蒸气压方程)和爱伦费斯特方程时采用的方法。详细的推导见 \enref{“克拉伯龙方程}{Clapey}”。

\subsection{热力学势判据}
除了熵判据外,还可以用热力学势判据,它们本质上都是从热力学第二定律出发得到的对热力学系统的判据,从而帮助我们判定系统是否处于平衡和稳定状态。

对于等温系统,可以利用亥姆霍兹自由能给出最大功原理或自由能判据(详见 “亥姆霍兹自由能\upref{HelmF}”)
\begin{theorem}{最大功原理}
对于一个恒温的封闭热力学系统,系统对外界做的功总是小于(不可逆过程)或等于(可逆过程)自由能 $F$ 的减少量。
\end{theorem}
\begin{theorem}{自由能判据}
等温等容系统处在稳定平衡状态的必要和充分条件为 $\dd F=0,\dd[2]{F}>0$。即对任意的可能发生的虚变动,$\Delta F>0$。
\end{theorem}

对于等温等压系统,可以利用吉布斯函数判据(见 \enref{“吉布斯自由能}{GibbsG}”):
\begin{theorem}{吉布斯判据}
反之而言,等温等压系统处在稳定平衡状态的必要和充分条件为 
$\dd G=0, \dd[2]{G}>0$。即对于任意系统可能发生的虚变动过程,其吉布斯自由能的改变量 $\Delta G>0$。
\end{theorem}
