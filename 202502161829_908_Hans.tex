% 汉斯·贝特(综述)
% license CCBYSA3
% type Wiki

本文根据 CC-BY-SA 协议转载翻译自维基百科\href{https://en.wikipedia.org/wiki/Hans_Bethe#Honors_and_awards}{相关文章}。


汉斯·阿尔布雷希特·贝特(Hans Albrecht Bethe,1906年7月2日–2005年3月6日)是一位德裔美籍物理学家,他在核物理学、天体物理学、量子电动力学和固态物理学方面做出了重要贡献,并因其在恒星核合成理论方面的工作获得了1967年诺贝尔物理学奖。[1][2][3][4] 贝特大部分职业生涯在康奈尔大学担任教授。[5]

1939年,贝特发表了一篇论文,确立了CNO循环作为更大质量恒星在主序星阶段的主要能量来源,这一贡献为他赢得了1967年的诺贝尔奖。[6] 在第二次世界大战期间,贝特担任洛斯阿拉莫斯国家实验室的理论部主任,该实验室研发了第一颗原子弹。他在计算武器的临界质量并开发用于“胖子”原子弹(在1945年8月投放到长崎)的内爆方法方面发挥了关键作用。

战后,贝特在氢弹的研发中发挥了重要作用,他还担任该项目的理论部门负责人,尽管他最初加入该项目的目的是证明氢弹无法制造。[7] 后来,他与阿尔伯特·爱因斯坦以及原子科学家紧急委员会一起,反对核试验和核军备竞赛。他帮助说服肯尼迪和尼克松政府分别签署了1963年的部分核试验禁令条约和1972年的反弹道导弹条约(SALT I)。1947年,他写了一篇重要论文,提供了对兰姆位移的计算,这一工作被认为彻底改变了量子电动力学,并进一步“为现代粒子物理学时代开辟了道路”。[8][9][10] 他对中微子的理解做出了贡献,[11] 并在解决太阳中微子问题中发挥了关键作用。[12] 他还为理解超新星及其过程做出了贡献。[13]

他的科学研究从未停止,直到九十多岁时他仍在发表论文,这使他成为少数几位在其职业生涯的每个十年中至少发表过一篇重要论文的科学家之一,而贝特的职业生涯几乎跨越了七十年。物理学家弗里曼·戴森(曾是他的博士生)称他为“20世纪的最高级问题解决者”,[14] 而宇宙学家爱德华·科尔布则称他为“物理学界最后的老大师”。[15]