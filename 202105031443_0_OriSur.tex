% 可定向曲面
% keys Jacobi矩阵|过渡矩阵|定向|曲面|orientation|surface

对于曲面$S$上一点$p$,其附近可能存在两个不同的局部坐标系$\bvec{x}:U_x\to V_x$和$\bvec{y}:U_y\to V_y$,其中$p\in V_x\cap V_y$.因此,这两个局部坐标系的交集非空.如果记$W=\bvec{x}^{-1}(U_x)\cap\bvec{y}^{-1}(U_y)$,$\bvec{x}$和$\bvec{y}$都是$W\to V_x\cap V_y$的局部坐标系.

由于局部坐标系是同胚,我们由此得到了两个$W$到自身的自同胚,$\bvec{x}\circ\bvec{y}$和$\bvec{y}\circ\bvec{x}$.这两个自同胚都是二维欧几里得空间之间的映射,因此可以计算其Jacobi矩阵.回忆Jacobi矩阵的几何意义,我们发现它可以用来描述区域的方向——就是说,当Jacobi矩阵为正的时候,映射不会“翻转”被映射的区域,但是Jacobi矩阵为负的时候,区域则被映射“翻转”了.

由此我们可以严格讨论什么是可定向曲面了.

\begin{definition}{可定向曲面}
给定一个正则曲面$S$,如果我们可以用一族局部坐标系$\{\bvec{x}_i\}$完全覆盖$S$\footnote{即对于任意$p\in S$,总存在一个$\bvec{x}_p$包含$p$.},且任意两个局部坐标系之间,如果交集非空,则Jacobi矩阵必恒正的,那么我们说这是一个\textbf{可定向曲面(oriented surface)}.如果不存在这样的一族局部坐标系,那么我们说这个曲面是\textbf{不可定向的(nonorientable)}.
\end{definition}







