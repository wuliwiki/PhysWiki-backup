% 西梅翁·德尼·泊松(综述)
% license CCBYSA3
% type Wiki

本文根据 CC-BY-SA 协议转载翻译自维基百科 \href{https://en.wikipedia.org/wiki/Sim\%C3\%A9on_Denis_Poisson}{相关文章}。

西缅·德尼·泊松男爵(Baron Siméon Denis Poisson,/pwɑːˈsɒ̃/,美式亦作 /ˈpwɑːsɒn/,法语发音:[si.me.ɔ̃ də.ni pwa.sɔ̃];1781年6月21日-1840年4月25日)是一位法国数学家和物理学家,研究领域包括统计学、复分析、偏微分方程、变分法、解析力学、电磁学、热力学、弹性力学与流体力学。此外,他在试图反驳奥古斯丁·让·菲涅耳的波动理论时,意外预测了“阿拉戈斑点”的存在。
\subsection{生平}
泊松出生于法国卢瓦雷省皮蒂维耶(今属卢瓦雷省),是法国陆军军官西缅·泊松的儿子。

1798年,他以年级第一的成绩进入巴黎综合理工学院,很快便引起校内教授们的注意,并被允许自由选择研究方向。在入学不到两年的最后一年学习中,他发表了两篇论文:一篇是关于埃蒂安·贝祖消元法的研究,另一篇是关于有限差分方程积分个数的探讨。这些成果令教授们极为赞赏,以至于他在1800年未参加毕业考试的情况下被特别准许毕业。\(^\text{[2][3]}\)第二篇论文由西尔维斯特-弗朗索瓦·拉克鲁瓦和阿德里安-玛丽·勒让德审阅,并推荐发表在《外国学者文集》上——这对当时年仅十八岁的泊松来说是史无前例的荣誉。

这一成功迅速使泊松进入了科学界。约瑟夫-路易·拉格朗日在综合理工学院开设函数论课程,泊松是他的听众之一,很早便发现了他的才华,并成为他的朋友。同时,泊松亦追随皮埃尔-西蒙·拉普拉斯的学术道路,后者几乎将他视为自己的儿子。

此后,泊松的余生一直在巴黎附近的索镇从事科研工作,撰写并发表了大量学术成果,并陆续担任了多个教育机构的重要职位,直至去世。\(^\text{[4]}\)

他在巴黎综合理工学院完成学业后,即被任命为该校的助教。实际上,他在还是学生时就已非正式地担任这一角色——因为每当某堂课特别难时,同学们常常会在课后去他宿舍听他复述和讲解课程内容。他于1802年被正式任命为代教授,1806年接替被拿破仑派往格勒诺布尔的让-巴普蒂斯特·约瑟夫·傅里叶,晋升为正教授。

1808年,他成为法国经度局的天文学家;1809年巴黎科学院理学院设立后,他被任命为理论力学教授。他于1812年成为科学院成员,1815年担任圣西尔军事学院的考试官,1816年起任综合理工学院的毕业考试官,1820年出任大学理事会委员,1827年继皮埃尔-西蒙·拉普拉斯之后担任法国经度局的几何学家。\(^\text{[4]}\)

1817年,泊松与南希·德·巴尔迪(Nancy de Bardi)结婚,两人育有四个孩子。他的父亲早年经历使其对贵族深恶痛绝,因此在第一共和国时期以严厉的共和主义信条教育他。在法国大革命、拿破仑帝国以及随后的王政复辟时期,泊松始终对政治不感兴趣,而是专注于数学研究。

1825年,他被授予男爵头衔的荣誉,\(^\text{[4]}\)但他既没有申请正式文凭,也未曾使用这一称号。1818年3月,他被选为英国皇家学会院士,\(^\text{[5]}\)1822年成为美国艺术与科学院的外籍荣誉会员,\(^\text{[6]}\)1823年当选为瑞典皇家科学院外籍院士。

1830年七月革命期间,他的所有荣誉一度面临被剥夺的威胁;但这一可能给路易-菲利普政府带来耻辱的事件,被弗朗索瓦·让·多米尼克·阿拉戈巧妙化解。当时政府正密谋撤销泊松的职位,而阿拉戈则促成他获邀前往王宫用餐,并在那里受到市民之王(路易-菲利普)公开而热情的接待,国王“想起了”他。有了这番公开的肯定,他的降职自然无法实施。七年后,他被授予法国贵族院议员的身份,这并非出于政治目的,而是作为法国科学界的代表。\(^\text{[4]}\)
\begin{figure}[ht]
\centering
\includegraphics[width=8cm]{./figures/37490114bee9a448.png}
\caption{} \label{fig_BSDN_1}
\end{figure}
据说,作为一位数学教师,泊松极为成功,这一点也许早在他担任巴黎综合理工学院助教时便已显露无疑。尽管他身兼多项官方职责,却仍设法发表了三百多篇著作,其中有若干篇为内容庞大的专著,还有许多论文探讨纯数学、应用数学、数学物理及理性力学中最为艰深的分支\(^\text{[4]}\)(阿拉戈曾将这样一句话归于泊松:“人生唯二有意义之事:一是做数学,二是教数学。”\(^\text{[7]}\)

阿拉戈的传记末尾列出了泊松亲自整理的著作目录。在此只能简要提及他最重要的部分作品。泊松对科学的最大贡献,或许就是将数学应用于物理领域。他最具原创性、同时也最具持久影响力的成果,可能就是那些关于电与磁理论的论文,这些论文实际上开创了数学物理学的一个全新分支。\(^\text{[4]}\)

在泊松的诸多成就中,仅次于(或在某些人看来甚至超过)他在电磁理论上的工作的重要成果,是他在天体力学领域的多篇论文,在这些论文中,他展现出自己作为皮埃尔-西蒙·拉普拉斯真正继承者的才能。其中最重要的包括:《论行星平均运动的长期不等式》,《论力学问题中任意常数的变化》,这两篇均发表于《综合理工学院学报》(Journal de l'École Polytechnique,1809);《论月球的纵摆》,载于《星历年鉴》(Connaissance des temps,1821);《论地球绕其质心的运动》,发表于《科学院纪要》(Mémoires de l'Académie,1827)等。在第一篇论文中,泊松探讨了著名的行星轨道稳定性问题。拉格朗日曾在摄动力的一阶近似下解决了这个问题,而泊松则将结果推广到了二阶近似,从而在行星理论上取得重要突破。这篇论文非比寻常,因为它激励了长时间沉寂的拉格朗日在晚年创作出他最伟大的论文之一:《论行星元素变化的理论,特别是轨道长半轴变化的理论》。拉格朗日对泊松的论文评价极高,亲手抄写了一份,并在他去世后被发现保存在遗稿中。泊松还对引力理论作出了重要贡献。\(^\text{[4]}\)

为表彰他横跨纯数、应用数与物理学的三百余篇科学著作,法国政府于1837年授予他贵族头衔。

他的名字也被镌刻在埃菲尔铁塔上的72位杰出科学家之中。
