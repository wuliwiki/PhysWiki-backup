% 二次型
% 二次型|规范型|对角型

\pentry{双线性型(2-线性函数)\upref{Tensor}}
\subsection{定义}
\begin{definition}{二次型}\label{QuaFor_def2}
\footnote{本文参考: 科斯特利金.代数学引论, 第二卷.}域 $\mathbb{F}$ 上有限维空间 $V$ 上的函数 $q:V\rightarrow\mathbb{F}$ ,若它满足如下两个性质:
\begin{enumerate}
\item $q(-{v})=q( v),\quad \forall v\in V$;
\item 由公式
\begin{equation}\label{QuaFor_eq1}
f(x, y)=\frac{1}{2}\qty[q(x+ y)-q(x)-q(y)]
\end{equation}
决定的映射 $f:V\times V\rightarrow\mathbb{F}$ 是 $V$ 上的双线性型\upref{Tensor}.
\end{enumerate}
则称 $q$ 是 $V$ 上的\textbf{二次型(quadratic form)},并称 $f$ 的秩为 $q$ 的秩:rank $q$=rank $f$ (链接未完成). 另外容易证明 $f$ 是\textbf{对称的}, 即 $f(x,y) = f(y,x)$.
\end{definition}
利用\autoref{QuaFor_eq1} ,由 $q$ 得到的对称的双线性型 $f$ 称为\textbf{极化的},或 $f$ 是与二次型 $q$ \textbf{配极}的双线性型.
\begin{example}{}
设 $f$ 是 $V$ 上任意一个对称的双线性型,令
\begin{equation}
q_f( x)=f( x, x)
\end{equation}
就得到一个满足二次型定义的函数 $q_f:V\rightarrow\mathbb{F}$.

\textbf{证明}:
\begin{equation}
q_f(-{x})=f(-{x},-{x})=f({x},{x})=q_f({x}) \qquad (\forall x\in V)
\end{equation}
\begin{equation}
f(x, y)=\frac{1}{2}\qty[f( x+ y, x+ y)-f( x, x)-f( y, y)]
=\frac{1}{2}\qty[q(x+ y)-q(x)-q(y)]
\end{equation}
证毕.
\end{example}

\begin{theorem}{}\label{QuaFor_the1}
每一个二次型 $q$ 都可以按着自己的配极双线性型 $f$ 唯一地恢复原型;换言之, $q=q_f$
\end{theorem}
\textbf{证明:}在\autoref{QuaFor_eq1} 中令 $y=-x$ :
\begin{equation}
-f(x,x)=\frac{1}{2}[q(0)-q(x)-q(-x)]
\end{equation}
从而
\begin{equation}
q(x)=f(x,x)+\frac{1}{2}q(0)
\end{equation}
因为 $f$ 是个双线性型,所以 $f(0,0)=0$ .因为,当 $x=0$ 时有 $q(0)=\frac{1}{2}q(0)$ ,即 $q(0)=0$,也就是说, $q(x)=f(x,x)$.

每一个二次型按\autoref{QuaFor_eq1} 定义一个与其配极对称双线性型 $f$ ,而由\autoref{QuaFor_the1} ,每一个对称的双线性型 $f$ 有唯一一个二次型 $q$ 与之对应,这就是说,\textbf{二次型和对称双线性型一一对应}.

\subsection{二次型的矩阵}
\begin{definition}{二次型的矩阵}
称与 $q$ 配极的双线性型 $f$ 在空间 $V$ 的基底 $(e_1,\cdots,e_n)$ 之下的矩阵 $\mat F$ 是二次型 $q=q_f$ 的矩阵. 即矩阵元为 $f_{ij} = f(e_i, e_j)$.
\end{definition}

若 $a = \sum_i a_i e_i$, $b = \sum_j b_j e_j$, 令对应的坐标列矢量为 $\bvec a = (a_1\ a_2\ \dots)\Tr$, $\bvec b = (b_1\ b_2\ \dots)\Tr$. 那么 $f(u, v)$ 可以表示为以下的矩阵运算\upref{Mat}.
\begin{equation}
f(a, b) = \bvec a\Tr \mat F \bvec b = \sum_{i,j} f_{ij}a_i b_j
\end{equation}
对应的二次型为
\begin{equation}
q(a) = \bvec a\Tr \mat F \bvec a = \sum_{i,j} f_{ij}a_i a_j
\end{equation}
因为 $f(e_i, e_j) = f(e_j, e_i)$, 所以 $\mat F$ 是一个对称矩阵.

\subsection{二次型的规范型}
\pentry{相似变换和相似矩阵\upref{MatSim}}
\begin{definition}{二次型的规范型(或对角型)}\label{QuaFor_def1}
称二次型 $q$ 在 $V$ 的基底 $(e_1,\cdots,e_n)$ 之下具有\textbf{规范型}或\textbf{对角型},如果对 $\forall x=\sum x_i e_i\in V$ ,$q(x)$ 的值可用公式
\begin{equation}
q(x)=\sum_{i}f_{ii}x_i^2
\end{equation}
计算.此时基底 $(e_i)$ 称为对 $q$ 的\textbf{规范基底}.
\end{definition}

由于二次型的矩阵是对称矩阵, 任何二次型矩阵都可以经过基底变换(即相似变换\upref{MatSim})变为对角矩阵. 令变换矩阵为 $\mat P$,则 $\bvec u = \mat P \bvec u'$, $\bvec v = \mat P \bvec v'$, 那么
\begin{equation}
\bvec u\Tr \mat F \bvec v = (\mat {Pu'})\Tr \mat F (\bvec {Pv'}) = \bvec u'\Tr \mat P\Tr\mat F \mat P \bvec v'=\bvec u'\mat F' \bvec v'
\end{equation}
其中 $\mat F'$ 就是 $\mat F$ 的相似矩阵.
