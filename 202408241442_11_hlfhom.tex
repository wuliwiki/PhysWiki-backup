% 半同态
% license Xiao
% type Tutor
\pentry{环\nref{nod_Ring}}{nod_5b17}
\begin{definition}{}
$f:R_1\rightarrow R_2$是两个环之间的映射。对于任意$a,b\in R_1$有:
\begin{equation}
\left\{\begin{array}{c}
f(a+b)=f(a)+f(b) \\
f(a b)=f(b)f(a)
\end{array}\right.
~,\end{equation}
则称$f$是\textbf{反同态}。

若映射$f$保环加法不变,$f(ab)=f(a)f(b)$或$f(b)f(a)$,则称$f$是\textbf{半同态}。
\end{definition}
\begin{theorem}{华罗庚半同态定理}
给定环$R_1,R_2$及半同态$f$,$f$要么是同态,要么是反同态。
\end{theorem}
\textbf{证明:}

设$r\in R_1$,$S_1=\{a\in R_1|f(ar)=f(a)f(r)\},S_2=\{a\in R_1|f(ar)=f(r)f(a)\}$,则$R_1=S_1\cup S_2$。

可证$S_1$是$R_1$的子群。设任意$a_1,a_2\in S_1$,则$f((a_1-a_2)r)=f(a_1r)-f(a_2r)=f(a_1-a_2)f(r)$,同理$S_2$也是$R_1$的子群。

若$a_1\in S_1-S_2,b_1\in S_2-S_1$,则$a_1-b_1\in R_1-S_1-S_2$,矛盾,因而$R_1=S_1$或$S_2$。称$R_1$是$r$“生成”的,使得$f$要么保同态,要么保反同态的环。

设$l_1$为环上的集合,其内所有元素都能生成保同态关系不变的$R_1$,$l_2$生成的$R_1$保反同态关系不变,易证这两个集合也是环上的子群。同上述过程类似,可证$R_1=l_1$或$l_2$。

