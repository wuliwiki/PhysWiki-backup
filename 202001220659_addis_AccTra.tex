% 加速度的坐标系变换

\pentry{速度的参考系变换\upref{Vtrans}}

\subsection{无相对转动}
类比\autoref{Vtrans_eq1}\upref{Vtrans}, 若两个参考系之间只有平移没有转动, 令某时刻点 $P$ 相对于 $S$ 系和 $S'$ 系的加速度分别为 $\bvec a_S$ 和 $\bvec a_{S'}$, 再令两坐标系中任意两个固定点(例如各自的原点)之间的加速度为 $\bvec a_r$, 那么有
\begin{equation}\label{AccTra_eq1}
\bvec a_S = \bvec a_{S'} + \bvec a_r
\end{equation}
同样地, 如果要将该式写成分量的形式, 三个矢量必须使用同一坐标系(见\autoref{Vtrans_ex2}\upref{Vtrans}).

\subsection{一般情况}
类比\autoref{Vtrans_eq2}\upref{Vtrans}, 若两参考系之间有可能存在转动, 我们将以上 $\bvec a_r$ 定义中的两个固定点取 $t$ 时刻点 $P$ 在两参考系中的坐标. 那么是否仍然有 $\bvec a_S = \bvec a_{S'} + \bvec a_r$ 呢? 答案是否定的, 正确的表达式需要添加一项
\begin{equation}\label{AccTra_eq2}
\bvec a_S = \bvec a_{S'} + \bvec a_r + 2 \bvec \omega \cross \bvec v
\end{equation}
其中 $\bvec \omega$ 是 $S'$ 系相对于 $S$ 系的瞬时角速度. 最有一项被称为\textbf{科里奥利加速度}
\begin{equation}
a_c = 2 \bvec \omega \cross \bvec v
\end{equation}
同样, 如果要将\autoref{AccTra_eq2} 写成分量的形式, 所有矢量必须使用同一坐标系.

\subsection{证明}
(未完成)% 画个三角形即可
