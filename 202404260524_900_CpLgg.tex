% 计算机语言
% keys 计算机语言|机器语言|汇编语言|标记语言|脚本语言|编程语言|高级语言|编译型语言|解释型语言
% license Usr
% type Tutor

\begin{issues}
\issueTODO
\end{issues}

计算机语言通常是一个能完整、准确和规则地表达人们的意图,并用以指挥或控制计算机工作的“符号系统”。

\begin{itemize}
\item \autoref{sub_CpLgg_1} 机器语言

\item \autoref{sub_CpLgg_2} 汇编语言

\item \autoref{sub_CpLgg_3} 标记语言

\item \autoref{sub_CpLgg_4} 脚本语言(解释型语言)

\item \autoref{sub_CpLgg_5} 编程语言

\item \autoref{sub_CpLgg_6} 高级语言

\item \autoref{sub_CpLgg_7} 编译型语言
\end{itemize}

\subsection{机器语言}\label{sub_CpLgg_1}

机器语言是机器能直接识别的程序语言或指令代码,无需经过翻译,每一操作码在计算机内部都有相应的电路来完成它,或指不经翻译即可为机器直接理解和接受的程序语言或指令代码。机器语言使用绝对地址和绝对操作码。不同的计算机都有各自的机器语言,即指令系统。从使用的角度看,机器语言是最低级的语言。

机器语言是用二进制代码表示的计算机能直接识别和执行的一种机器指令的集合。这种指令集,称机器码(machine code),是电脑的CPU可直接解读的数据。它是计算机的设计者通过计算机的硬件结构赋予计算机的操作功能。机器语言具有灵活、直接执行和速度快等特点。不同型号的计算机其机器语言是不相通的,按着一种计算机的机器指令编制的程序,不能在另一种计算机上执行。

一条指令就是机器语言的一个语句,它是一组有意义的二进制代码,指令的基本格式如,操作码字段和地址码字段,操作码(OperationCode,OP)用来表示该指令所要完成的操作(如加、减、乘、除、数据传送等),其长度取决于指令系统中的指令条数。地址码用来描述该指令的操作对象,它或者直接给出操作数,或者指出操作数的存储器地址或寄存器地址(即寄存器名)。

\subsection{汇编语言}\label{sub_CpLgg_2}

汇编语言,即为一种低级语言,它用人类容易记忆的语言和符号来表示一组0和1的代码,例如AND代表加法。

为了克服机器语言难读、难编、难记和易出错的缺点,人们就用与代码指令实际含义相近的英文缩写词、字母和数字等符号来取代指令代码(如用ADD表示运算符号“+”的机器代码),于是就产生了汇编语言。所以说,汇编语言是一种用助记符表示的仍然面向机器的计算机语言。汇编语言亦称符号语言。汇编语言由于是采用了助记符号来编写程序,比用机器语言的二进制代码编程要方便些,在一定程度上简化了编程过程。汇编语言的特点是用符号代替了机器指令代码, 而且助记符与指令代码一一对应,基本保留了机器语言的灵活性。使用汇编语言能面向机器并较好地发挥机器的特性,得到质量较高的程序。

汇编语言中由于使用了助记符号,用汇编语言编制的程序送入计算机,计算机不能象用机器语言编写的程序一样直接识别和执行(机器语言程序,它一经被安置在内存的预定位置上,就能被计算机的CPU处理和执行。)必须通过预先放入计算机的 “汇编程序“的加工和翻译,才能变成能够被计算机识别和处理的二进制代码程序。用汇编语言等非机器语言书写好的符号程序称源程序,运行时汇编程序要将源程序翻译成目标程序。

汇编语言像机器指令一样,是硬件操作的控制信息,因而仍然是面向机器的语言,使用起来还是比较繁琐费时,通用性也差。汇编语言是低级语言。但是,汇编语言用来编制系统软件和过程控制软件,其目标程序占用内存空间少,运行速度快,有着高级语言不可替代的用途。

\subsection{标记语言}\label{sub_CpLgg_3}

标记语言:是一种将文本以及文本相关的其他信息结合起来,展现出关于文档结构和数据处理细节的电脑文字编码。

与文本相关的其他信息(包括文本的结构和表示信息等)与原来的文本结合在一起,但是使用标记进行标识。标记语言不仅仅是一种语言,就像许多语言一样,它需要一个运行时环境,使其有用。

\subsection{脚本语言(解释型语言)}\label{sub_CpLgg_4}

脚本语言:是为了缩短编程语言的“编写、编译、链接、运行”等过程而创建的计算机编程语言。

是一种用来解释某些东西的语言,又被称为扩建的语言,或者动态语言,用来控制软件应用程序,脚本通常以文本保存,只在被调用时进行解释或编译。

脚本语言是为了缩短传统的编写-编译-链接-运行(edit-compile-link-run)过程而创建的计算机编程语言。它的命名起源于一个脚本“screenplay”,每次运行都会使对话框逐字重复。

早期的脚本语言经常被称为批量处理语言或工作控制语言。 一个脚本通常是解释运行而非编译。

脚本语言通常都有简单、易学、易用的特性,目的就是希望能让程序员快速完成程序的编写工作。

各种动态语言,如ASP、PHP、CGI、JSP、JavaScript、VBScript等,都是脚本语言。

\begin{itemize}
\item 脚本语法比较简单,比较容易掌握;

\item 脚本与应用程序密切相关,所以包括相对应用程序自身的功能;

\item 脚本一般不具备通用性,所能处理的问题范围有限。
\end{itemize}

\subsection{编程语言}\label{sub_CpLgg_5}

编程语言(programming language):是用来定义计算机程序的形式语言。它是一种被标准化的交流技巧,用来向计算机发出指令。一种计算机语言让程序员能够准确地定义
计算机所需要使用的数据,并精确地定义在不同情况下所应当采取的行动。

\subsection{高级语言}\label{sub_CpLgg_6}

高级语言,它是在低级语言的基础上,采用接近于人类自然语言的单词和符号来表示一组低级语言程序,使编程变得更加简单,易学,且写出的程序可读性强。

\subsection{编译型语言}\label{sub_CpLgg_7}

编译型语言:程序在执行之前需要一个专门的编译过程,把程序编译成 为机器语言的文件,运行时不需要重新翻译,直接使用编译的结果就行了。程序执行效率高,依赖编译器,跨平台性差些
