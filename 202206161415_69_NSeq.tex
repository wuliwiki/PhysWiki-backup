% Navier-Stokes 方程
% keys NS方程|流体力学动量方程

\pentry{流体力学守恒方程\upref{fluidC}}
让我们回顾流体力学的动量守恒方程:
\begin{equation}
\rho \frac{\partial \bvec u}{\partial t}+\rho(\bvec u\cdot \nabla)\bvec u=\rho \bvec g+\nabla\cdot \overleftrightarrow {\bvec T}
\end{equation}

为了进一步解这个方程,我们需要知道应力张量 $T_{ij}$ 的具体形式.$T_{ij}$ 表示在法线为 $x_i$ 方向的单位面元上,面外对面内的面力的 $x_j$ 分量,而且是二阶对称张量(为了保证角动量守恒),那么它有怎样的性质呢?

最简单的一种非粘性的各向同性流体,其应力张量的对角元都为 $-p$,这意味着每个面元上受到的力是垂直于面元的,单位面积上受到的力为流体在该处的压强,因此对角元 $-p$ 给出的就是压强 $p$.但实际情况中流体是有粘性的,例如下图
\begin{figure}[ht]
\centering
\includegraphics[width=10cm]{./figures/NSeq_1.png}
\caption{牛顿粘性实验} \label{NSeq_fig1}
\end{figure}
当 $y$ 方向相邻两侧流体的速度 $u_x$ 有梯度时,就会产生一个剪应力.