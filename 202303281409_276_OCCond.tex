% 斜截条件
% 斜截条件|变端问题

\pentry{端点可变问题\upref{EPQue}}
所谓的\textbf{斜截条件}是变端问题中\upref{EPQue}使得泛函 $J$ (\autoref{EPQue_eq1}~\upref{EPQue})取极值的曲线的端点所满足的条件。它是指下面两个方程
\begin{equation}\label{OCCond_eq1}
[F+(\varphi'-y')F_{y'}]^{(0)}=0,\quad [F+(\psi'-y')F_{y'}]^{(1)}=0~.
\end{equation}
其中,指标 ${(0)},{(1)}$ 在这里表示对应的函数值是取在弧的起点及终点上的,而 $\varphi(x),\psi(x)$ 是可取曲线的起点和终点所在的曲线函数。

于是变端问题的解答可用下述定理表述
\begin{theorem}{}
在一切连接已给曲线 $y=\varphi(x)$ 及 $y=\psi(x)$ 上的任意点的 $C_1$ 类的曲线中,如果曲线 $\gamma:y=y(x)$ 给出积分
\begin{equation}\label{OCCond_eq2}
J(\gamma)=\int_{\gamma}F(x,y,y')\dd x
\end{equation}
的极值,那么 $\gamma$ 是极端曲线,并且在它的端点上满足斜截条件\autoref{OCCond_eq1} 。
\end{theorem}
\subsection{证明}
由\autoref{EPQue_sub1}~\upref{EPQue}中的注意及\autoref{EPQue_the1}~\upref{EPQue}。给出积分\autoref{OCCond_eq2} 的曲线必定是极端曲线,且要在诸极端曲线中找到使 $J$ 达到极值的弧,对于任意的 $\delta x_0,\delta x_1$ ,有
\begin{equation}
\dd J=-[(F-y'F_{y'})^{(0)}\delta {x_0}+ F_{y'}^{(0)}\delta y_0]+[(F-y'F_{y'})^{(1)}\delta {x_1}+F_{y'}^{(1)}\delta y_1]=0~.
\end{equation}

 考虑到可取曲线端点在$y=\varphi(x),y=\psi(x)$ 上,那么
 \begin{equation}
 y_0=\varphi(x_0),y_1=\psi(x_1)~.
 \end{equation}
 所以 
 \begin{equation}
 \delta y_0=\varphi'\delta x_0,\quad \delta y_1=\psi'\delta x_1~,
 \end{equation}
 于是
 \begin{equation}
\dd J=-[F+(\varphi'-y')F_{y'}]^{(0)}\delta {x_0}+[F+(\psi'-y')F_{y'}]^{(1)}\delta {x_1}=0~.
\end{equation}
由于,$\delta x_0,\delta x_1$ 的任意性,所以
\begin{equation}
[F+(\varphi'-y')F_{y'}]^{(0)}=0,\quad [F+(\psi'-y')F_{y'}]^{(1)}=0~.
\end{equation}
定理得证!