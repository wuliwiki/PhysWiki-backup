% 路易·德布罗意(综述)
% license CCBYSA3
% type Wiki

本文根据 CC-BY-SA 协议转载翻译自维基百科\href{https://en.wikipedia.org/wiki/Louis_de_Broglie}{相关文章}。

\begin{figure}[ht]
\centering
\includegraphics[width=6cm]{./figures/34cce1e349a4d7bf.png}
\caption{德布罗意在1929年} \label{fig_Brogli_1}
\end{figure}
路易·维克托·皮埃尔·雷蒙德,第七代布罗意公爵(法语:[də bʁɔj] 或 [də bʁœj],1892年8月15日-1987年3月19日)是法国物理学家和贵族,他对量子理论做出了开创性贡献。在他1924年的博士论文中,他假设了电子的波动性质,并提出所有物质都有波动特性。这个概念被称为德布罗意假设,是波粒二象性的一个例子,并成为量子力学理论的核心部分。

德布罗意于1929年获得诺贝尔物理学奖,因为物质的波动行为在1927年首次得到了实验验证。

德布罗意发现的粒子波动行为被厄尔温·薛定谔用在他提出的波动力学中。德布罗意的导波概念于1927年在索尔维会议上提出,随后被放弃,转而支持量子力学,直到1952年被大卫·玻姆重新发现并加以完善。

路易·德布罗意于1944年当选为法兰西学院第16位成员,担任法兰西科学院的终身秘书。德布罗意是第一位呼吁建立多国实验室的高级科学家,这一提议最终促成了欧洲核子研究组织(CERN)的成立。
\subsection{传记}  
\subsubsection{家庭与教育}
路易·德布罗意出身于著名的布罗意贵族家族,几百年来,该家族的成员在法国担任重要的军事和政治职务。未来物理学家的父亲路易-阿尔方斯-维克多,第五代布罗意公爵,娶了波琳·达尔梅伊尔,她是拿破仑时代将军菲利普·保尔·塞吉尔伯爵的孙女,而塞吉尔伯爵的妻子是传记作家玛丽·塞勒斯廷·阿梅丽·达尔梅伊尔。他们有五个孩子,除了路易,还有:阿尔贝蒂娜(1872–1946),后来成为卢佩侯爵夫人;莫里斯(1875–1960),后成为著名的实验物理学家;菲利普(1881–1890),在路易出生前两年去世;波琳,潘日女伯爵(1888–1972),后成为著名作家。

路易·德·布罗意出生于法国塞纳-马尔姆地区的迪耶普。作为家中的小儿子,路易在相对孤独的环境中长大,阅读了大量书籍,并且特别喜欢历史,尤其是政治历史。从小他记忆力极好,能准确地背诵剧本中的片段,或者列出法兰西第三共和国的所有内阁部长。因此,人们预测他将来会成为一位伟大的政治家。

德·布罗意原本打算从事人文学科的职业,并获得了历史学的学士学位。此后,他转向数学和物理学,并获得了物理学的学位。第一次世界大战爆发后,他主动为军队提供服务,参与了无线电通讯的开发。
\subsubsection{军服务}
毕业后,路易·德·布罗意加入了工程部队,开始了强制性服役。服役开始于蒙·瓦莱里安堡,但不久后,在他哥哥的提议下,他被调到无线电通讯服务,并在埃菲尔铁塔工作,那里有无线电发射机。路易·德·布罗意在第一次世界大战期间一直服役,主要处理技术性问题。特别是,他与莱昂·布里渥因(Léon Brillouin)和哥哥莫里斯一起,参与了与潜艇的无线通信建设。路易·德·布罗意于1919年8月退役,晋升为上士。后来,这位科学家遗憾地表示,他不得不离开自己真正感兴趣的基础科学问题约六年之久。
\subsubsection{科学与教学生涯}  
他的1924年论文《Recherches sur la théorie des quanta》(《量子理论研究》)提出了他的电子波理论。这个理论包括物质的波粒二象性理论,基于马克斯·普朗克和阿尔伯特·爱因斯坦关于光的研究。这项研究最终得出了德布罗意假说,指出任何运动的粒子或物体都有一个相关的波。德布罗意由此创造了物理学的新领域——波动力学(mécanique ondulatoire),将能量(波)和物质(粒子)的物理学统一在一起。他因发现电子的波动性质而获得了1929年诺贝尔物理学奖。

在他后来的职业生涯中,德布罗意致力于发展波动力学的因果解释,反对主导量子力学理论的完全概率模型;该理论在1950年代由大卫·玻姆进一步完善。此理论后来被称为德布罗意–玻姆理论。

除了严格的科学工作,德布罗意还思考并写作关于科学哲学的内容,包括现代科学发现的价值。1930年,他创办了由埃尔曼出版社出版的书籍系列《Actualités scientifiques et industrielles》(《科学与工业新闻》)。