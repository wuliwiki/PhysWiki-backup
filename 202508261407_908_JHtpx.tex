% 几何拓扑学(综述)
% license CCBYSA3
% type Wiki

本文根据 CC-BY-SA 协议转载翻译自维基百科\href{https://en.wikipedia.org/wiki/Geometric_topology}{相关文章}

\begin{figure}[ht]
\centering
\includegraphics[width=6cm]{./figures/d945e3a478caa898.png}
\caption{由一组博罗米恩环所界定的赛弗特曲面;这些曲面可以作为几何拓扑中的研究工具。} \label{fig_JHtpx_1}
\end{figure}
在数学中,几何拓扑学研究的是流形及其之间的映射,尤其是一个流形嵌入到另一个流形中的情形。
\subsection{历史}
作为一个独立于代数拓扑的研究领域,几何拓扑学可以追溯到1935 年,当时通过Reidemeister 扭结对透镜空间进行了分类。这项工作首次要求区分那些同伦等价但不同胚的空间,也由此催生了简单同伦理论。“几何拓扑学”这一术语用来描述这一研究方向则是相对较近才出现的用法。\(^\text{[1]}\)
\subsection{低维拓扑与高维拓扑的区别}
流形在高维与低维中的行为存在显著差异。高维拓扑通常指维度 5 及以上的流形,或者从相对角度来看,指余维数3 及以上的嵌入问题。低维拓扑主要研究4 维及以下**的流形,或者余维数不超过 2 的嵌入问题。


四维流形具有特殊性:从某些角度(如拓扑结构)看,四维表现得像高维;而从其他角度(如可微结构)看,四维又表现得像低维。这种双重特性导致了许多四维独有的现象,例如 $\mathbb{R}^4$ 上的奇异可微结构。因此,四维流形的拓扑分类在理论上是可处理的,其核心问题包括:一个拓扑流形是否承认可微结构?如果承认,可能存在多少种不同的可微结构?值得注意的是,光滑四维情形仍然是广义庞加莱猜想的最后一个未解案例,可参见Gluck 扭转。

这种差异主要源于手术理论的适用范围:在五维及更高维度,手术理论适用;事实上,在某些情况下,它在四维的拓扑层面上也适用,但证明过程非常复杂。在四维及以下维度(拓扑意义上,三维及以下),手术理论则无法使用。

因此,研究低维流形时,一种常见的思路是:假设手术理论在低维情形下也成立,它会预测哪些结果?然后通过比较实际情况与这种预测的差异,来理解低维流形中那些偏离高维规律的特殊现象。
\begin{figure}[ht]
\centering
\includegraphics[width=8cm]{./figures/d454ad50d06c800e.png}
\caption{惠特尼技巧需要至少 2+1 个维度,因此手术理论需要 5 个维度才能适用。} \label{fig_JHtpx_2}
\end{figure}
造成5维与低维行为差异的精确原因在于:惠特尼嵌入定理—— 也是手术理论的关键技术 —— 需要至少2+1个维度才能发挥作用。简单来说,惠特尼技巧允许我们“解开”打结的球面(更准确地说,是去除浸入映射中的自交点)它的实现依赖以下步骤:借助一个2维的盘;通过一个增加1个维度的同伦对该盘进行变形。这样,在余维数大于2的情况下,就可以避免产生新的自交点,因此在余维数大于 2 的情形下,嵌入问题可以通过手术理论进行分析。在手术理论中,关键步骤发生在中间维度;当中间维度的余维数大于2(粗略说,超过 2.5 维即可,因此总维度至少5维即可),惠特尼技巧就能起作用。这一结论直接导出了Smale 的 h-同调定理,它适用于5维及更高维度,并构成了手术理论的基础。

在4 维中,惠特尼技巧的一个变形版本可以使用,这就是Casson把手。但由于维度不足,一个惠特尼盘会引入新的扭结,而这些扭结又需要通过新的惠特尼盘来消除,进而产生一个盘的序列(“塔”)。这个无限延伸的“塔”最终收敛到一个拓扑可行但不可微分的映射。因此,在4 维情况下,手术理论在拓扑意义上是成立的,但在可微意义上不成立。
\subsection{几何拓扑中的重要工具}
\subsubsection{基本群}
在所有维度中,流形的基本群都是一个极其重要的不变量,并决定了该流形的许多结构特性:
在1、2 和 3 维中,可能的基本群范围受到严格限制;而在4维及更高维度中,任何有限呈现群都可以作为某个流形的基本群。注意,只需证明 4 维或 5 维流形的情形成立,再将它们与球面相乘即可推广到更高维度。
\subsubsection{可定向性}
如果一个流形能够保持一致的方向选择,那么它就是可定向的;并且,一个连通的可定向流形恰好有两种不同的取向。在这一框架下,可定向性有多种等价表述,可以根据具体应用和泛化程度来选择:一般拓扑流形的可定向性,通常借助同调理论(的方法来描述;而在可微流形中,由于存在更丰富的结构,可以通过微分形式来表达。此外,可定向性的一个重要推广是纤维丛的可定向性:当一个空间族由另一个空间参数化时,需要在每个纤维空间中选择一个方向,并且这种方向选择需要随参数连续变化。
\subsubsection{柄分解}
一个 $m$ 维流形 $M$ 的柄分解是一个嵌套序列:
$$
\emptyset = M_{-1} \subset M_0 \subset M_1 \subset M_2 \subset \cdots \subset M_{m-1} \subset M_m = M~
$$
其中,每个 $M_i$ 都是通过在 $M_{i-1}$ 上附加一个 $i$-柄($i$-handle)得到的。柄分解对于流形的作用类似于CW 分解对拓扑空间的作用。柄分解的意义在于提供一种与CW 复形类似但适用于光滑流形的语言。因此,一个$i$-柄就是一个 $i$-胞腔在光滑情形下的类比。柄分解在流形中自然出现,尤其是在莫尔斯理论 中。而柄结构的修改则与塞尔夫理论密切相关。
\subsubsection{局部平坦性}
局部平坦性是指嵌入在高维拓扑流形中的子流形所具备的一种性质。在拓扑流形的范畴下,局部平坦的子流形与光滑流形范畴中的嵌入子流形扮演了类似的角色。

假设一个 $d$ 维流形 $N$ 被嵌入到一个 $n$ 维流形 $M$ 中(其中 $d < n$)。如果 $x \in N$,我们说 $N$ 在 $x$ 处是局部平坦的,当且仅当存在一个邻域 $U \subset M$,使得拓扑对 $(U, U \cap N)$ 同胚于 $(\mathbb{R}^n, \mathbb{R}^d)$,其中 $\mathbb{R}^d$ 是标准地嵌入 $\mathbb{R}^n$ 的子空间。换句话说,存在一个同胚 $U \to \mathbb{R}^n$,使得 $U \cap N$ 的像正好对应于 $\mathbb{R}^d$。
\subsubsection{Schönflies 定}理
广义 Schönflies 定理表明:如果一个 $(n-1)$ 维球面 $S$ 被局部平坦地嵌入到 $n$ 维球面 $S^n$ 中(即该嵌入可以扩展成一个“加厚的球面”),那么这对 $(S^n, S)$ 同胚于 $(S^n, S^{n-1})$,其中 $S^{n-1}$ 是 $n$ 维球体的赤道。布朗和马祖尔因为各自独立证明了该定理而获得了Veblen 奖\(^\text{[2][3]}\)。
