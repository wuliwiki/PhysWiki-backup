% 外导数
% 外代数|外导数|外微分|叉乘|梯度|散度|旋度

\pentry{外微分\upref{ExtDif}}

\subsection{外导数的定义}

外导数是一种流形上的\textbf{外微分代数}上的映射,其术语分为两部分,“外”和“导数”.“外”,指的是它把各$\Omega^k(M)$中的元素映射到$\Omega^k(M)$\textbf{之外};“导数”,指的是它具有和求导类似的性质.实际上,矢量分析中的求导就是外导数的一个特例——你可能会问,求导并不具有“外”的特点,怎么就是特例了呢?我们会在本节中解释这一点.

\begin{definition}{外导数}
给定流形$M$,其外微分代数是$\Omega (M)$.定义映射$\dd:\Omega (M)\rightarrow\Omega (M)$,满足:
\begin{itemize}
\item $\forall \omega\in\Omega^k(M)$,有$\dd \omega\in\Omega^{k+1}(M)$.
\item 对于光滑函数$f\in C^\infty(M)$,$\dd f$就是$f$的方向导数(1-形式).
\item \textbf{线性性}:任取$a, b\in \mathbb{R}$和$\omega, \mu\in\Omega(M)$,有$\dd(a\omega+b\mu)=a\dd\omega+b\dd\mu$.
\item \textbf{Leibniz性}:任取$\omega, \mu\in\Omega^1(M)$,有$\dd(\omega\wedge\mu)=\omega\wedge\dd\mu+\dd\omega\wedge\mu$\footnote{注意$\omega, \mu$都特指1-形式.}.
\item \textbf{幂零性}:任取$\omega\in\Omega(M)$,都有$\dd(\dd\omega)=0$.
\end{itemize}
称这个映射为$\Omega (M)$或者说$M$上的一个\textbf{外导数}.
\end{definition}

定义中的Leibniz性要特别注意.考虑到外积的\textbf{反对称性},我们也完全可以把这一条写成$\dd(\omega\wedge\mu)=\omega\wedge\dd\mu+\mu\wedge\dd\omega$.写成定义中减号的形式,是为了更方便计算出诸如$\dd(\dd\omega\wedge\mu\wedge\dd\nu)=-\dd\omega\wedge\dd\mu\wedge\dd\nu$的结果.

Leibniz性的另一个形式,可能更方便用于计算:对于$\omega\in\Omega^k(M), \mu\in\Omega^1(M)$,我们有$\dd(\omega\wedge\mu)=\omega\wedge\dd\mu+(-1)^k\mu\wedge\dd\omega$.

\subsection{三维欧几里得空间}

在\textbf{外代数}\upref{ExtAlg}中我们提到过,$\mathbb{R}^3$和$\bigwedge^2\mathbb{R}^3$同构.在流形$\mathbb{R}^3$上,2-形式构成的线性空间$\Omega^1(\mathbb{R}^3)\cong\mathbb{R}^3$,因为是由基$\{\dd x, \dd y, \dd z\}$张成的.这样,我们也可以定义$\Omega^2(\mathbb{R}^3$到$\Omega^1(\mathbb{R}^3)$之间的同构.这个同构的存在,意味着我们可以把\textbf{旋度}和\textbf{散度}视为外导数的特例.我们观察以下例子来说明这一点:

考虑$\mathbb{R}^3$中任意的1-形式$\omega_x\dd x+\omega_y\dd y+\omega_z\dd z$,其中各$\omega_i$是0-形式,即光滑函数.考虑到外导数对于光滑函数就是方向导数,我们可以得知,对于任意的$a\in\{x,y,z\}$,有$\dd\omega_a=\partial_x\omega_a\dd x+\partial_y\omega_a\dd y+\partial_z\omega_a\dd z$.这样,我们就可以计算出:
\begin{equation}\label{ExtDer_eq1}
\begin{aligned}
&\dd(\omega_x\dd x+\omega_y\dd y+\omega_z\dd z)\\=&(\partial_y\omega_z-\partial_z\omega_y)\dd y\wedge\dd z+\\&(\partial_z\omega_x-\partial_x\omega_z)\dd z\wedge\dd x+\\&(\partial_x\omega_y-\partial_y\omega_x)\dd x\wedge\dd y
\end{aligned}
\end{equation}

如果我们把$\dd y\wedge\dd z$看成$\dd x$、把$\dd z\wedge\dd x$看成$\dd y$、把$\dd x\wedge\dd y$看成$\dd z$,那么上




