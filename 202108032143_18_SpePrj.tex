% 谱投影

\pentry{有界算子的预解式\upref{BddRsv}}

\subsection{谱投影与空间分解}
\begin{definition}{谱投影}
设$X$是复巴拿赫空间, $T:X\to X$是有界算子. 设有一条简单闭道路$\gamma$将谱集$\sigma(T)$分成了不相交的两部分, 将包含在$\gamma$内部的部分记为$\Lambda$. 则算子
$$
P_\Lambda:=\frac{1}{2\pi i}\int_\gamma(z-T)^{-1}dz
$$
称为$T$在$\Lambda$上的谱投影.
\end{definition}

为何要像这样定义谱投影? 原来, 这其实是在推广线性代数中将空间分解为矩阵的不变子空间的操作. 对于矩阵的情形, 参见词条 例: 有限维方阵\upref{SpeMat}, 在那里谱投影的意义可以通过直接计算看出. 在一般的巴拿赫空间的情形, 我们首先有如下命题:

\begin{lemma}{}
如上定义的算子$P_\Lambda$的确是有界的投影算子, 即满足
$$
P_\Lambda^2=P_\Lambda.
$$
另外, $P_\Lambda$与$T$可交换.
\end{lemma}

由此可见, 空间$X$被分成了两个闭子空间$M_\Lambda:=\text{Ran}(P_\Lambda)$和$N_\Lambda:=\text{Ran}(1-P_\Lambda)$的直和. 由此即可得到不变子空间分解定理:

\begin{theorem}{不变子空间分解}
闭子空间$M_\Lambda$和$N_\Lambda$都是算子$T$的不变子空间, 而且有直和
$$
X=M_\Lambda\oplus N_\Lambda.
$$
若把算子$T$限制在$M_\Lambda$上, 并且视之为$M_\Lambda$上的算子, 则它的谱是$\Lambda$; 同样地, 若把算子$T$限制在$N_\Lambda$上, 并且视之为$N_\Lambda$上的算子, 则它的谱是$\sigma(T)\setminus\Lambda$.
\end{theorem}

这也就是说, 谱集分离成多个部分即意味着空间分解为算子的不变子空间的直和. 这就使得算子在空间上的作用更清楚了.

\subsection{孤立谱点}
如果$\lambda_0\in\sigma(T)$是孤立的谱点, 那么可以认为它是预解式$(z-T)^{-1}$的孤立奇点. 仿照复变函数论, 当然可以谈论它在孤立奇点处的留数和洛朗展开式. 显然, 预解式在$\lambda_0$处的留数就是谱投影
$$
P_{\lambda_0}=\frac{1}{2\pi i}\int_{|z-\lambda_0|=r}(z-T)^{-1}dz,
$$
这里$|z-\lambda_0|=r$是一个充分小的圆. 正像计算洛朗级数展开式那样, 也不难得出$(z-T)^{-1}$在$z=\lambda_0$附近的洛朗展开:
$$
\begin{aligned}
(z-T)^{-1}
&=\frac{1}{2\pi i}\int_{|\zeta-\lambda_0|=r}(\zeta-T)^{-1}\frac{1}{\zeta-z}d\zeta\\
&=\frac{1}{2\pi i}\int_{|\zeta-\lambda_0|=r}(\zeta-T)^{-1}\frac{1}{(\zeta-\lambda_0)-(z-\lambda_0)}d\zeta\\
\end{aligned}
$$