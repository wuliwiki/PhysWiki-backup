% 南京理工大学 2011 量子真题
% license Usr
% type Note

\textbf{声明}:“该内容来源于网络公开资料,不保证真实性,如有侵权请联系管理员”

\\section*{一、简答题(请考生在下列12题中任选10题,每题6分,其它0分):}

1. 一粒子的波函数为 $\\psi (\\mathbf{r}, t) = A e^{-\\alpha(r - x_0)^2} e^{-i \\omega t}$,写出粒子位于 $x = x_0 + \\Delta x$ 处的概率,用求导标示,解释波函数表示的意义。

2. 请将下波函数表示为 $\\psi (x, t) = (a e^{i \\omega x} + b e^{-i \\omega x}) e^{-i \\omega t}$ 写出粒子在类克尔(\\textit{Kr})中找到的几率。

3. 何谓量子纠缠?何谓量子隐形传态?何谓测量塌缩?

4. 写出狄拉克 $\\delta$ 函数的傅里叶变换及拉普拉斯变换的表示式。

5. 一体积分面两个空间的电偶极子 $ \\mathbf{d}_1$,粒子 $a$ 可用如下电量矩阵描述,给定两个可能的电量,分别为 $\\mathbf{d}_1 = (q_1, \\mathbf{r}_1)$ 和 $\\mathbf{d}_2 = (q_2, \\mathbf{r}_2)$,向你写出可能的表示为几人?

6. $\\mathcal{H}$ 力 $V$ 能的力矩如何单独相干叠加?

7. 对一个粒子保持近于某一力学空间的轨道,测量结果与求和方案单位有什么关系?两个力学空间时间具有确定值的条件是什么?

8. 量子系统的一些基本特征是什么?有啥区别?

9. 下列波函数所描写的状态是否为定态?并说明其理由。

\\begin{enumerate}
  \\item $\\varphi_1 (x, t) = \\varphi (x) e^{-i \\omega t} + \\varphi (x) e^{i \\omega t}$
  \\item $\\psi_2 (x, t) = u (x) e^{-i \\omega t} + v (x) e^{i \\omega t}$
\\end{enumerate}

10. 已知 $L \\cdot x = i \\hbar \\frac{\\partial}{\\partial t}$,量子 $L$ 符合条件是什么?请根据原因简要说明。

11. 与自由粒子相关联的变换是什么?写出表示有无偏移?

12. 谱学序列在实验量子物理的关系写为 [\\textbf{A} \\textit{B} \\textit{C}],写出形式定义和数学表示。

13. 写出迪拉克基态自旋的动力学关系,目标函数与自旋周期相等。
