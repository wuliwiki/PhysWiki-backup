% 匀加速直线运动
% keys 匀加速|加速度|速度|轨迹|抛物线
% license Xiao
% type Tutor

\pentry{速度 加速度(一维)\nref{nod_VnA1}}{nod_3cdf}

质点做匀速直线运动时, 我们延运动的直线建立坐标轴 $x$, 则最一般的运动方程
\begin{equation}\label{eq_CnstAL_1}
x(t) = x_0 + v_0 (t - t_0) +  \frac12 a_0 (t - t_0)^2~.
\end{equation}
其中 $x_0, v_0, a_0$ 分别是 $t_0$ 时刻的位置, 速度和加速度。 注意沿 $x$ 轴正方向的速度和加速度取正号, 沿反方向取负号。

匀加速直线运动速度变化为
\begin{equation}\label{eq_CnstAL_2}
v(t) = v_0 + a_0 (t - t_0)~,
\end{equation}
另外有一条不含时间的公式
\begin{equation}\label{eq_CnstAL_3}
v_2^2 - v_1^2 = 2a_0 (x_2 - x_1)~,
\end{equation}
其中 $v_1, v_2$ 分别是质点经过点 $x_1$ 和 $x_2$ 时的速度。

注意只有初速度矢量和加速度矢量方向相同(包括初速度为零)时匀加速运动的轨迹才是直线, 否则轨迹就是抛物线, 见 “\enref{匀加速运动}{ConstA}”。
\addTODO{把 “匀加速运动” 的自由落体移动过来}
\addTODO{插入 ep1 的飞机丢导弹例题}

\subsection{推导}
做匀速直线运动的质点在 $t_0$ 时的位置为 $x_0$, 速度为 $v_0$, 且加速度始终等于常数 $a_0$, 求任意时刻的速度和加速度 $x(t)$。

我们首先把 $a(t) = a_0$ 代入\autoref{eq_VnA1_8}~\upref{VnA1}
\begin{equation}
v(t) = v_0 + \int_{t_0}^t a_0 \dd{t'}~,
\end{equation}
马上得到\autoref{eq_CnstAL_2} 。 再次积分(\autoref{eq_VnA1_5}~\upref{VnA1})
\begin{equation}
x(t) = x_0 + \int_{t_0}^t [v_0 + a_0 (t' - t_0)] \dd{t'}~,
\end{equation}
得\autoref{eq_CnstAL_1} 。 这个过程相当于直接使用\autoref{eq_VnA1_2}~\upref{VnA1}。

要得到\autoref{eq_CnstAL_3}, 由\autoref{eq_CnstAL_1} 和\autoref{eq_CnstAL_2} 分别得
\begin{equation}
x_2 - x_1 = v_1 (t_2 - t_1) +  \frac12 a_0 (t_2 - t_1)^2~,
\end{equation}
\begin{equation}
t_2 - t_1 = \frac{v_2 - v_1}{a_0}~
\end{equation}
代入消去 $t$ 得\autoref{eq_CnstAL_3}。 直观来看, 如果画速度—时间图, $x_2 - x_1$ 可以表示成梯形的面积, 利用梯形公式可以得到该式。

\autoref{eq_CnstAL_3} 的另一种推导是, 在学习牛顿第二定律\upref{New3}和动能定理\upref{KELaw1}后, 我们可以直接写出
\begin{equation}
\frac{1}{2}mv_2^2 - \frac{1}{2}mv_1^2 = F (x_2 - x_1) = ma (x_2 - x_1)~,
\end{equation}
两边除以 $m/2$ 得到\autoref{eq_CnstAL_3} 。
