% 南京理工大学 2009 量子真题
% license Usr
% type Note

\textbf{声明}:“该内容来源于网络公开资料,不保证真实性,如有侵权请联系管理员”

请考生在下列13题中选作10题,每题15分,满分150分
\subsection{简要回答下列问题}
\begin{enumerate}
\item 最子力学中角动量是如何定义的?地球自转是否与量子力学中的自概念相对应?
\item 玻恩近似法的基本思想是什么?
\item 如果有心力场不是库仑场(即$V(r)$不与$\frac{1}{r}$成比例),则角分布函数将取什么形式?
\item 如何理解波函数必须满足的标准条件?
\item 在什么情况下力学量的测量具有确定值?两个不对易的力学量是否一定不能同时其有确定的测量值?
\end{enumerate}
二、定义 $[\hat{A}, \hat{B}]_{+} = \hat{A}\hat{B} + \hat{B}\hat{A}$ (反对易式),已知 $\hat{a}, \hat{b}$ 均与 $\hat{A}, \hat{B}$ 对易,证明:

\begin{enumerate}
    \item 
    \[
    [\hat{A}, \hat{B}\hat{C}] = \hat{A}[\hat{B}, \hat{C}]_{+} - [\hat{B}, \hat{C}]_{+}\hat{A} + \hat{C}[\hat{A}, \hat{B}]_{+} - [\hat{A}, \hat{C}]_{+}\hat{B}~
    \]
    \item 
    \[
    [\hat{a}\hat{A}, \hat{b}\hat{B}]_{+} = \frac{1}{2}[\hat{a}, \hat{b}]_{+}[\hat{A}, \hat{B}] + \frac{1}{2}[\hat{a}, \hat{b}][\hat{A}, \hat{B}]_{+}~
    \]
\end{enumerate}
三、计算受力 $F = -kx + k_e (k = m\omega^2)$ 作用的一个粒子的波函数和能量允许值。
