% 毕达哥拉斯(综述)
% license CCBYSA3
% type Wiki

本文根据 CC-BY-SA 协议转载翻译自维基百科\href{https://en.wikipedia.org/wiki/Pythagoras}{相关文章}。

\begin{figure}[ht]
\centering
\includegraphics[width=6cm]{./figures/76de0efc99539521.png}
\caption{萨摩斯的毕达哥拉斯雕像,位于罗马的卡比托利尼博物馆[1]} \label{fig_Pythag_1}
\end{figure}
萨摩斯的毕达哥拉斯(古希腊语:Πυθαγόρας;约公元前570年-约公元前495年),通常以单名“毕达哥拉斯”而为人熟知,是一位古代伊奥尼亚希腊哲学家、博学家,也是毕达哥拉斯学派的创始人。他的政治和宗教教义在大希腊地区广为人知,并且影响了柏拉图、亚里士多德的哲学思想,进而影响了整个西方。关于他生平的知识被传说所笼罩;现代学者对毕达哥拉斯的教育背景和影响存在分歧,但他们一致认为,在公元前530年左右,他前往意大利南部的克罗顿,在那里创立了一所学校,要求学员发誓保守机密,并过着共同体的禁欲主义生活。

在古代,毕达哥拉斯被誉为许多数学和科学发现的创立者,包括毕达哥拉斯定理、毕达哥拉斯调律、五种规则立体、比例理论、地球的球形学说,以及早晨星和晚星是金星的天体认知。据说他是第一个自称为“哲学家”(即“智慧之爱者”)的人,也是第一个将地球划分为五个气候带的人。古代历史学家对毕达哥拉斯是否做出了这些发现存在争议,许多归功于他的成就可能在他之前就已被发现,或者是由他的同事或继承者完成的。一些记载提到,与毕达哥拉斯相关的哲学与数学有关,数字在其中占有重要地位,但关于他在数学或自然哲学方面的实际贡献程度仍存在争议。

与毕达哥拉斯最为密切相关的教义是“灵魂的轮回”或“转世说”,即认为每个灵魂是永恒的,死后会进入另一个身体。他可能还创立了“宇宙音乐”(musica universalis)的学说,认为行星的运动遵循数学方程,因此产生一种不可听见的音乐交响曲。学者们争论毕达哥拉斯是否发展了与数字学和音乐相关的教义,或者这些教义是否由他的后继者,特别是克罗顿的菲洛劳斯(Philolaus)所创立。在公元前510年左右,克罗顿在与希巴里斯的决定性战役中取得胜利后,毕达哥拉斯的追随者与民主派支持者发生冲突,毕达哥拉斯学派的集会场所被焚毁。毕达哥拉斯可能在这次迫害中被杀害,或者他逃亡到梅塔庞图姆并在那里去世。

毕达哥拉斯对柏拉图产生了影响,柏拉图的对话录,特别是《蒂迈欧篇》,展示了毕达哥拉斯的教义。毕达哥拉斯关于数学完美的思想也影响了古希腊的艺术。他的教义在公元前1世纪经历了复兴,尤其是在中期柏拉图主义者中,并与新毕达哥拉主义的兴起相吻合。毕达哥拉斯在中世纪一直被视为伟大的哲学家,他的哲学对尼古拉斯·哥白尼、约翰内斯·开普勒和艾萨克·牛顿等科学家产生了重大影响。毕达哥拉斯的象征主义也在早期现代欧洲的神秘主义中广泛使用,而他在奥维德《变形记》中所描绘的教义后来还影响了现代素食主义运动。
\subsection{传记资料}  
没有毕达哥拉斯的真实著作流传下来,[2][3][4] 关于他的生活几乎没有确凿的资料。[5][6][7] 最早关于毕达哥拉斯生平的资料简短、模糊,并且常带有讽刺性。[4][8][9] 最早的关于毕达哥拉斯教义的资料是一首讽刺诗,可能是毕达哥拉斯去世后由他的一位同时代人——科洛丰的希腊哲学家克塞诺芬尼(约公元前570年-公元前478年)所写。[10][11] 诗中,克塞诺芬尼描述了毕达哥拉斯为一只被打的狗求情,声称在狗的哀嚎中听出了已故朋友的声音。[d][9][10][12] 克罗顿的医生阿尔克迈翁(约公元前450年,约与毕达哥拉斯生活在同一时期)将许多毕达哥拉斯的教义融入自己的著作中[13],并暗示自己可能与毕达哥拉斯有过私人接触。[13] 以弗所的诗人赫拉克利特(约公元前500年,出生地距离萨摩斯岛不远,可能与毕达哥拉斯同时代)讽刺毕达哥拉斯为一个狡猾的骗子,[8][14] 并评论道:“毕达哥拉斯,孟内萨尔库斯之子,比任何人都更擅长探求,他从这些著作中选择并制造了一种属于自己的智慧——博学多才,巧妙的伪善。”[8][14]
\begin{figure}[ht]
\centering
\includegraphics[width=6cm]{./figures/920ad764d7f392b0.png}
\caption{17世纪版画中的毕达哥拉斯虚构肖像} \label{fig_Pythag_2}
\end{figure}
古希腊诗人基俄斯的伊翁(约公元前480年—公元前421年)和阿克拉加斯的恩培多克勒斯(约公元前493年—公元前432年)都在他们的诗篇中表达了对毕达哥拉斯的钦佩。[15] 关于毕达哥拉斯的第一次简洁描述出自历史学家哈利卡纳苏斯的希罗多德(约公元前484年—公元前420年),他将毕达哥拉斯描述为最伟大的希腊教师之一,并且表示毕达哥拉斯教导他的追随者如何获得不朽。[16] 然而,希罗多德的著作准确性存在争议。[17][18][19][20][21] 被归于毕达哥拉斯哲学家克罗顿的菲洛劳斯(约公元前470年—公元前385年)的著作,是最早描述数字学和音乐理论的文本,这些理论后来归于毕达哥拉斯。[22] 雅典修辞学家伊索克拉底(约公元前436年—公元前338年)是第一个描述毕达哥拉斯曾访问埃及的人。[16] 亚里士多德(约公元前384年—公元前322年)曾写过一篇关于毕达哥拉斯的论文,现已失传。[23] 其中的一些内容可能被保存在《劝学篇》里。亚里士多德的弟子迪凯阿科斯、阿里斯托克修斯和赫拉克利德(他们都生活在公元前3世纪)也写过相关的著作。[24]

关于毕达哥拉斯生平的主要来源大多来自罗马时期,[25] 根据德国古典学者沃尔特·伯克特的说法,“毕达哥拉斯主义的历史已经是……一项艰难的重建,恢复已失去的东西。”[24] 三部来自晚期古代的古代传记仍然存世,[7][25] 它们主要充满了神话和传说。[7][25][26] 其中最早且最值得尊敬的传记出自狄俄尼修斯·拉尔修斯的《名哲学家列传》。[25][26] 另外两部传记是新柏拉图主义哲学家波尔菲里(Porphyry)和扬布利库斯(Iamblichus)所写,[25][26] 部分内容意在反对基督教的兴起。[26] 后来的来源比早期的传记更长,[25] 对毕达哥拉斯成就的描述也更加神奇。[25][26] 波尔菲里和扬布利库斯使用了亚里士多德弟子(迪凯阿科斯、阿里斯托克修斯和赫拉克利德)的失传著作中的材料,[24] 这些材料通常被认为是最可靠的。[24]
\subsection{生活}  
\subsubsection{早期生活}  
毕达哥拉斯的一生中没有一个细节没有受到争议。但是,借助对数据的或多或少批判性的选择,构建一个合理的叙述是可能的。

——沃尔特·伯克特,1972年[27]  
赫罗多托斯、伊索克拉底和其他早期作家一致认为,毕达哥拉斯是梅纳萨尔库斯的儿子,[16][29]并且他出生在爱琴海东部的希腊岛屿萨摩斯。[2][29][30][31] 根据这些传记作者的说法,毕达哥拉斯的父亲并非出生在萨摩斯岛,尽管他在那里取得了归化身份,[30]但根据扬布利库斯的说法,他是岛上的土生土长的人。[32] 他被认为是一名宝石雕刻家或富有的商人[33][34][35],但他的家族背景存在争议,尚不清楚。[f] 他的母亲是萨摩斯岛人,出自一个地理学家(geomoroi)家庭。[36] 泰阿那的阿波罗纽斯称她的名字为皮泰伊斯。[37][38] 扬布利库斯讲述了这样一个故事:毕达哥拉斯的母亲怀孕时,皮提亚女祭司预言她将生下一个极为美丽、智慧且有益于人类的人。[39] 关于他的出生日期,阿里斯托克修斯曾表示,毕达哥拉斯在波利克拉底斯统治时期离开萨摩斯,年约40岁,这样的话,出生日期大约是公元前570年。[40] 毕达哥拉斯的名字使他与皮提亚的阿波罗(Pūthíā)产生了联系;公元前4世纪的基里奈的阿里斯提普斯通过说:“他讲述真理的能力与皮提亚的阿波罗[πυθικός puthikós]一样”来解释他的名字。[39]

在毕达哥拉斯的成长时期,萨摩斯是一个繁荣的文化中心,以其先进的建筑工程壮举而闻名,包括尤帕利诺斯隧道的建造,并且因其热烈的节日文化而著称。[41] 它是爱琴海地区的重要贸易中心,商人们从近东带来商品。[2] 根据Christiane L. Joost-Gaugier的说法,这些商人几乎可以肯定将近东的思想和传统带到了萨摩斯。[2] 毕达哥拉斯的早期生活恰逢早期伊奥尼亚自然哲学的繁荣时期。[29][42] 他是哲学家阿那克西曼德、阿那克西门斯和历史学家赫卡泰俄斯的同时代人,他们都生活在米利都,米利都位于萨摩斯对面的海边。[42]
\subsubsection{传闻中的旅行}
传统上认为毕达哥拉斯在近东地区接受了大部分的教育。[43] 现代学术研究表明,古希腊的文化深受黎凡特和美索不达米亚文化的影响。[43] 和许多其他重要的希腊思想家一样,毕达哥拉斯被认为曾在埃及学习。[16][44][45] 到了公元前四世纪的伊索克拉底时代,毕达哥拉斯在埃及的学习已经被视为事实。[16][39] 作家安提丰(可能生活在希腊化时代)在他失传的作品《杰出人物论》中,作为波菲利引述的来源,声称毕达哥拉斯从法老阿马西斯二世本人那里学会了说埃及语,并且他曾在底比斯的埃及祭司那里学习,还是唯一一个被允许参与他们祭祀的外国人。[43][46] 中期柏拉图主义者、传记作家普鲁塔克(约公元46年–120年)在他的论文《伊希斯与欧西里斯论》中写道,在他的埃及之行中,毕达哥拉斯曾接受赫利奥波利斯的埃及祭司奥努菲斯的教导(与此同时,梭伦则听取了塞斯的索恩基斯的讲座)。[47] 根据基督教神学家亚历山大的克莱门特(约公元150年–215年)的说法,“毕达哥拉斯是埃及大先知索恩基斯的弟子,也是塞赫努菲斯的柏拉图。”[48] 一些古代作家声称,毕达哥拉斯从埃及人那里学到了几何学和再生转生学说。[44][49]

然而,其他古代作家声称,毕达哥拉斯是从波斯的魔术师或甚至是从琐罗亚斯德本人那里学到这些教义的。[50][51] 迪奥根尼斯·拉尔提乌斯(Diogenes Laërtius)断言,毕达哥拉斯后来访问了克里特岛,并与埃皮美尼得斯一起去了伊达山洞。[52] 据说腓尼基人教给毕达哥拉斯算术,迦勒底人则教给他天文学。[51] 到公元前三世纪,已有报告称毕达哥拉斯也曾向犹太人学习。[51] 与所有这些报告相反,公元前二世纪的小说家安东纽斯·迪奥根尼斯(Antonius Diogenes)写道,毕达哥拉斯通过解梦自己发现了所有的教义。[51] 公元三世纪的智者费洛斯特拉图斯(Philostratus)声称,除了埃及人之外,毕达哥拉斯还曾向印度的智者或裸体哲学家(gymnosophists)学习。[51] 亚姆布利库斯(Iamblichus)更进一步扩展了这个列表,声称毕达哥拉斯还曾与凯尔特人和伊比利亚人学习。[51]
\subsubsection{所谓的希腊教师}
\begin{figure}[ht]
\centering
\includegraphics[width=6cm]{./figures/bed39fddef188279.png}
\caption{梵蒂冈博物馆中的毕达哥拉斯雕像,位于梵蒂冈城,表现出他作为一位‘看起来疲惫的老人’的形象[1]。} \label{fig_Pythag_3}
\end{figure}
古代资料还记录了毕达哥拉斯曾向多位本土希腊思想家学习。[51] 有些人认为萨摩斯的赫尔莫达马斯(Hermodamas)可能是他的导师。[51][54] 赫尔莫达马斯代表了萨摩斯本土的吟游诗人传统,他的父亲克里欧斐洛斯(Creophylos)据说曾是他的对手诗人荷马的主人。[51] 另一些人则认为是普里埃奈的比阿斯(Bias)、泰勒斯(Thales)[55] 或阿那克西曼德(Anaximander,泰勒斯的学生)[51][55][56]。其他传统则认为神话中的诗人俄耳甫斯(Orpheus)是毕达哥拉斯的老师,从而代表了俄耳甫斯神秘教义。[51] 新柏拉图主义者曾写到毕达哥拉斯所写的一篇关于神祇的‘神圣话语’,用的是多利安希腊方言,他们认为这篇文章是在俄耳甫斯祭司阿格拉奥法摩斯(Aglaophamus)指导下,毕达哥拉斯在接受俄耳甫斯神秘教义启蒙时所写的。[51] 亚姆布利库斯(Iamblichus)认为俄耳甫斯是毕达哥拉斯言辞风格、精神态度和崇拜方式的榜样。[57] 亚姆布利库斯将毕达哥拉斯主义描述为毕达哥拉斯从俄耳甫斯、埃及祭司、厄琉希尼神秘仪式以及其他宗教和哲学传统中学到的一切的综合体。[57] 里德维格(Riedweg)指出,尽管这些故事充满幻想,但毕达哥拉斯的教义无疑在很大程度上受到了俄耳甫主义的影响。[58]
\begin{figure}[ht]
\centering
\includegraphics[width=6cm]{./figures/d6a619849b6b0a1e.png}
\caption{赫库兰尼姆的纸草别墅(Villa of the Papyri)出土的一尊铜质哲学家半身像,戴着缎带,可能是毕达哥拉斯的虚构雕像[53][1]。} \label{fig_Pythag_4}
\end{figure}
在众多被认为曾教导毕达哥拉斯的希腊智者中,来自锡罗斯的费雷西底斯被提及得最为频繁。[58][59] 关于毕达哥拉斯和费雷西底斯都有类似的奇迹故事,其中包括他预言一次船难、预言梅西纳的征服,以及预言通过井水喝下后发生地震的故事。[58] 生活在公元前二世纪的悖论学家阿波罗纽斯·悖论记述者(Apollonius Paradoxographus)认为毕达哥拉斯的神奇思想是受到费雷西底斯的影响。[58] 另一个故事,可能来自新毕达哥拉斯学派的哲学家尼科马库斯,讲述了当费雷西底斯年老且濒临死亡时,他在德洛斯岛上,毕达哥拉斯回去照顾他并表示敬意。[58] 萨摩斯的历史学家和暴君杜里斯被报道曾爱国地夸耀费雷西底斯所写的墓志铭,称毕达哥拉斯的智慧超越了他自己。[58] 基于这些将毕达哥拉斯与费雷西底斯联系在一起的记载,里德维格(Riedweg)得出结论,费雷西底斯可能真的是毕达哥拉斯的老师,这一传统或许具有某种历史依据。[58] 毕达哥拉斯和费雷西底斯似乎在灵魂和轮回转世的教义上有相似的看法。[58]

公元前520年之前,在毕达哥拉斯的一次访问埃及或希腊的过程中,他可能曾遇到过米利都的塔勒斯,塔勒斯比他大约五十四岁。塔勒斯是一位哲学家、科学家、数学家和工程师,[60] 还因其内接角定理的特殊情况而闻名。毕达哥拉斯的出生地,萨摩斯岛,位于爱琴海东北部,距离米利都不远。[61] 迪奥根尼·拉厄尔修斯引述了阿里斯托克斯(公元前四世纪)的一段话,称毕达哥拉斯从德尔菲女祭司忒米斯托克利亚那里学到了他大部分的道德教义。[62][63][64] 波尔斐里同意这一说法,[65] 但将这位祭司称为阿里斯托克利亚(Aristokleia)。[66] 古代文献还指出,毕达哥拉斯的宗教和禁欲特征与俄耳甫斯教派或克里特神秘宗教,[67] 或德尔菲神谕有许多相似之处。[68]
\subsubsection{在克罗顿}
\begin{figure}[ht]
\centering
\includegraphics[width=6cm]{./figures/63e9c54b2fde51e7.png}
\caption{显示与毕达哥拉斯相关地点的意大利地图} \label{fig_Pythag_5}
\end{figure}
波尔斐里重复了安提丰的记载,安提丰报告称,在毕达哥拉斯还在萨摩斯岛时,他创办了一所被称为“半圆形”的学校。[69][70] 在这里,萨摩斯人讨论公共事务。[69][70] 据说,这所学校声名鹊起,整个希腊最聪明的头脑纷纷前来萨摩斯聆听毕达哥拉斯的讲授。[69] 毕达哥拉斯自己则住在一个秘密的洞穴中,在那里他私下学习,并偶尔与一些亲密的朋友进行讨论。[69][70] 德国早期毕达哥拉斯学者克里斯托夫·里德维格(Christoph Riedweg)表示,毕达哥拉斯在萨摩斯岛教学完全是有可能的,[69] 但他提醒说,安提丰的记载提到了一座在他时代仍在使用的具体建筑,这似乎是出于萨摩斯的爱国兴趣。[69]

大约在公元前530年,当毕达哥拉斯大约40岁时,他离开了萨摩斯岛。[2][29][71][72][73] 后来的追随者声称他离开是因为不满波利克拉底斯在萨摩斯岛上的暴政,[71] 里德维格指出,这一解释与尼科马库斯对毕达哥拉斯声称的自由热爱的强调相符,但毕达哥拉斯的敌人则将他描绘成有暴政倾向的人。[71] 另一些记载则称毕达哥拉斯离开萨摩斯是因为他在萨摩斯担任过多的公共职务,且由于他在市民中的高评价,感到负担沉重。[74] 他来到了位于大希腊的希腊殖民地克罗顿(今天的卡拉布里亚的克罗托内)[29][73][75][76][77]。所有的记载都一致认为,毕达哥拉斯具有非凡的个人魅力,并迅速在新的环境中获得了巨大的政治影响力。[29][78][79] 他为克罗顿的精英提供建议,并经常给予他们指导。[80] 后来的传记作者讲述了他雄辩演讲所带来的奇幻故事,声称这些演讲使克罗顿人民放弃了奢华和腐化的生活方式,转而投身于他所引入的更纯洁的体系。[81][82]
\subsubsection{家庭与朋友}
\begin{figure}[ht]
\centering
\includegraphics[width=8cm]{./figures/0d4d4cc89586c62a.png}
\caption{1913年的插图,展示了毕达哥拉斯在教授一群女性。毕达哥拉斯学派的许多重要成员是女性[83][84],一些现代学者认为他可能认为女性也应当像男性一样接受哲学教育。[85]} \label{fig_Pythag_6}
\end{figure}
迪奥根尼·拉厄尔修斯指出,毕达哥拉斯“没有沉溺于爱情的享乐”[86],并且他告诫他人,只有在“愿意变得比自己更弱时”才应进行性行为。[87] 根据波尔斐里的记载,毕达哥拉斯娶了克里特的女性泰阿诺,她是皮瑟纳克斯的女儿[87],并与她生了几个孩子。[87] 波尔斐里写道,毕达哥拉斯有两个儿子,分别是特劳基斯和阿里尼奥特[87],还有一个女儿名叫米亚[87],她“在克罗顿的少女中居于领先地位,并且在结婚后,在已婚女性中也是如此。”[87] 而扬布里库斯并未提到这些孩子[87],只提到了一个以其祖父名字命名的儿子,名叫梅纳萨尔库斯。[87] 这个儿子由毕达哥拉斯指定的继任者阿里斯塔厄斯抚养长大,最终在阿里斯塔厄斯年老无法继续管理学派时接管了学派。[87] 《苏达》写道,毕达哥拉斯有四个孩子(特劳基斯、梅纳萨尔库斯、米亚和阿里尼奥特)。[88]

摔跤手米洛(Milo)来自克罗顿,据说是毕达哥拉斯的密友[89],并因在一座屋顶即将坍塌时救了哲学家的生命而被赞誉。[89] 这种关联可能是由于与另一位名叫毕达哥拉斯的运动教练混淆所致。[69] 迪奥根尼·拉厄尔修斯记录了米洛的妻子名叫米亚。[87] 扬布里库斯提到泰阿诺是克罗顿的布隆提努斯的妻子。[87] 迪奥根尼·拉厄尔修斯则称同样的泰阿诺是毕达哥拉斯的学生[87],并且毕达哥拉斯的妻子泰阿诺是她的母亲。[87] 迪奥根尼·拉厄尔修斯还记载了据称由泰阿诺所写的作品在他自己生前仍然存在[87],并引用了几个被归于她的观点。[87] 这些著作现在已知为伪名作品。[87]
\subsubsection{死亡}  
毕达哥拉斯对献身和禁欲的强调被认为帮助了克罗顿在公元前510年战胜邻近的锡巴里斯殖民地。[90] 胜利后,克罗顿的一些显赫市民提议制定一部民主宪法,但毕达哥拉斯的追随者拒绝了这一提议。[90] 由赛隆和尼农领导的民主支持者,前者据说因被排除在毕达哥拉斯的兄弟会之外而感到愤怒,煽动了民众反对他们。[91] 赛隆和尼农的追随者在一次毕达哥拉斯的会议中袭击了毕达哥拉斯的追随者,袭击发生在米洛的家中或其他某个聚会地点。[92][93] 对于这次袭击的记载常常互相矛盾,许多记载可能将其与后来的反毕达哥拉斯叛乱混淆,例如公元前454年在梅塔蓬图姆发生的叛乱。[91][94] 该建筑显然被放火焚烧,[92] 许多与会成员因此丧命;[92] 只有年轻且更为活跃的成员成功逃脱。[95]

关于毕达哥拉斯是否在袭击发生时在场以及如果在场,他是否成功逃脱,来源存在分歧。[27][93] 一些记载称,毕达哥拉斯并未出席被攻击的会议,因为他当时在德洛斯岛照顾临终的费雷西底斯。[93] 另一种来自狄卡尔库斯的记载则称,毕达哥拉斯当时在场并成功逃脱,[96] 带领一小群追随者前往附近的洛克里斯城,在那里他们请求庇护,但被拒绝。[96] 他们最终到达梅塔蓬图姆,在缪斯神庙中寻求庇护,并在没有食物的情况下,四十天后死于饥饿。[27][92][96][97] 波尔斐里记录的另一个故事称,当毕达哥拉斯的敌人在焚烧房屋时,他的忠实学生们躺在地上,为他铺设一条逃生的道路,让他能通过踩过他们的尸体,像走桥一样越过火焰。[96] 毕达哥拉斯成功逃脱,但因心痛于他深爱的学生们的死亡而自杀。[96] 另一种由迪奥根尼·拉厄尔修斯和扬布里库斯都记载的传说称,毕达哥拉斯几乎成功逃脱,但他来到了一片蚕豆田,拒绝穿越,因为这样做会违背他的教义,于是他停下来,最终被杀。[96][98] 这个故事似乎来源于作家尼安瑟斯,他讲述的是后来的毕达哥拉斯学派成员的故事,而非毕达哥拉斯本人。[96]
\subsection{教义}  
\subsubsection{轮回转生}
\begin{figure}[ht]
\centering
\includegraphics[width=8cm]{./figures/54ae434d74ddaf46.png}
\caption{在拉斐尔的壁画《雅典学派》中,毕达哥拉斯正坐着写书,一位年轻人向他呈上一块板,上面画着一个竖琴的示意图,竖琴上方是神圣四重三角形的图形。[99]} \label{fig_Pythag_7}
\end{figure}
尽管关于毕达哥拉斯教义的具体细节尚不确定,[100][101] 但可以重建出他主要思想的一般轮廓。[100][102] 亚里士多德详细讨论了毕达哥拉斯学派的教义,[12][103] 但并未直接提及毕达哥拉斯本人。[12][103] 毕达哥拉斯的主要教义之一似乎是轮回转生,[72][104][105][106][107][108] 即所有灵魂都是不朽的,死亡后,灵魂会转移到一个新的身体中。[104][107] 这一教义被色诺芬、基俄斯的伊翁和希罗多德提及。[104][109] 然而,关于毕达哥拉斯认为轮回转生如何发生的性质或机制,毫无所知。[110]

恩培多克勒斯在他的诗篇中提到,毕达哥拉斯可能声称自己拥有回忆前世转生的能力。[111] 迪奥根尼·拉厄尔修斯引用赫拉克利德斯·庞提库斯的说法,称毕达哥拉斯曾告诉人们,他曾生活过四个前世,并能详细记得每一世的经历。[112][113][114] 其中第一世是作为赫尔墨斯的儿子艾塔利德斯,赫尔墨斯赋予了他记住所有前世的能力。[115] 接着,他转生为尤佛博斯,一个在《伊利亚特》中短暂提到的特洛伊战争中的小英雄。[116] 然后,他成为了哲学家赫尔莫提穆斯,[117] 他在阿波罗神庙中认出了尤佛博斯的盾牌。[117] 最后的转生是作为德洛斯岛的渔夫皮路斯。[117] 据狄卡尔库斯所述,他的一个前世是作为一位美丽的妓女。[105][118]
\subsubsection{神秘主义}  
另一个归因于毕达哥拉斯的信仰是“天体和谐”[119][120],该理论认为行星和恒星的运动遵循数学方程式,这些方程式对应于音乐音符,从而产生一种无法听见的交响乐。[119][120] 根据波福里乌斯的说法,毕达哥拉斯教导人们七位缪斯女神实际上是七颗行星在一起歌唱。[121] 在他的哲学对话《劝学篇》中,亚里士多德让他笔下的双重角色说:

当毕达哥拉斯被问及[人类为何存在]时,他说,“是为了观察天界”,并且他常常声称自己是自然的观察者,正是为了这个目的,他才投身于人生。

—— 亚里士多德,《劝学篇》,第48页

据说毕达哥拉斯曾进行占卜和预言。[122] 在希腊文学中,关于通过数值占卜(isopsephy)的最早提及将这一实践与毕达哥拉斯联系在一起;他被视为这一实践的创始人。[123] 根据他的传记作者扬布里库斯的说法,他将自己的占卜方法传授给了一位名叫阿巴里斯的斯基泰人阿波罗祭司,这位祭司来自北极地区:[124]

阿巴里斯与毕达哥拉斯一起生活,并简要学习了生理学和神学;他不再通过动物的内脏进行占卜,而是向他揭示了通过数字进行预言的艺术,认为这种方法更加纯洁、更具神性,并且与神祇的天上数字更加契合。

—— 扬布里库斯,《毕达哥拉斯传》,第19.93节

这不应与今天所知的简化版本“毕达哥拉斯数字学”混淆,这种数字学涉及一种变体的数值占卜技术,被称为“毕达哥拉斯根”[125] 或“基数”[126],通过这种方法,单词中字母的基值通过加法或除法进行数学化简,以获得一个介于一到九之间的单一数值,代表整个名字或词语;[125] 然后,可以通过其他技术来解读这些“根”或“基数”,如传统的毕达哥拉斯归属法。[127] 这种后来的数字学形式在拜占庭时期兴盛,并首次出现在公元二世纪的诺斯替教徒中。[127] 到那时,数值占卜已经发展出了几种不同的技术,广泛用于各种目的,包括占卜、教义寓言、医学预后和治疗。[127]

在他被归因的希腊各地的访问中——例如德洛斯、斯巴达、菲吕斯、克里特等——他通常以宗教或祭司的身份出现,或者作为一位立法者。[128]
\subsubsection{数字学}  
所谓的毕达哥拉斯学派致力于数学,并且是第一个发展这一科学的学派;通过对数学的研究,他们开始相信数学的原理就是一切事物的原理。

—— 亚里士多德,《形而上学》 1, 985b
\begin{figure}[ht]
\centering
\includegraphics[width=6cm]{./figures/6d25fb7d466e77bd.png}
\caption{毕达哥拉斯被认为是发明了四重三角形(tetractys),[129][130] 这一符号在后期的毕达哥拉斯主义中具有重要的神圣意义。[131][132]} \label{fig_Pythag_8}
\end{figure}
根据亚里士多德的说法,毕达哥拉斯学派仅仅出于神秘主义的原因使用数学,缺乏实际应用。[133] 他们相信一切事物都由数字构成。[134][135] 数字一(单一数)代表一切事物的起源,[136] 数字二(二元数)代表物质。[136] 数字三是一个“理想数”,因为它有开始、中间和结束[137],并且是能够用来定义平面三角形的最小点数,他们将其视为阿波罗神的象征。[137] 数字四象征四季和四元素。[138] 数字七也被视为神圣的,因为它是行星的数量,也是竖琴弦的数量,[138] 还因为阿波罗的生日在每个月的第七天庆祝。[138] 他们认为奇数是阳性的,[139] 偶数是阴性的,[139] 并且数字五代表婚姻,因为它是二和三的和。[140][141]

十被视为“完美数”[133],毕达哥拉斯学派通过避免组成超过十人的团体来尊崇它。[142] 毕达哥拉斯被认为发明了四行三角形图形——四行的总和为完美数十,称为“四行图”(tetractys)。[129][130] 毕达哥拉斯学派将四行图视为极其神秘的重要象征。[129][130][131] 扬布里库斯在《毕达哥拉斯传》中指出,四行图“如此令人钦佩,并且被那些理解它的人赋予神性”,以至于毕达哥拉斯的学生会以它为誓言发誓。[99][130][131][143] 安德鲁·格雷戈里总结认为,将毕达哥拉斯与四行图联系起来的传统可能是真实的。[144]

现代学者对这些数字学教义是否由毕达哥拉斯本人发展,还是由后来的毕达哥拉斯哲学家克罗托的菲洛劳斯(Philolaus)发展存在争议。[145] 在他的开创性研究《古代毕达哥拉斯主义中的传统与科学》中,沃尔特·伯克特(Walter Burkert)认为,毕达哥拉斯是一个具有魅力的政治和宗教导师,[146] 但归于他名下的数字哲学实际上是菲洛劳斯的创新。[147] 根据伯克特的观点,毕达哥拉斯根本没有处理过数字,更不用说对数学作出任何值得注意的贡献了。[146] 伯克特认为,毕达哥拉斯学派实际上从事的唯一数学活动是简单的、没有证明的算术,[148] 但这些算术发现确实对数学的起步做出了重大贡献。[149]
\subsection{毕达哥拉斯主义}  
\subsubsection{共同生活方式}
\begin{figure}[ht]
\centering
\includegraphics[width=8cm]{./figures/2ca88a6b982f5593.png}
\caption{费奥多尔·布罗尼科夫(Fyodor Bronnikov)所作《毕达哥拉斯学派庆祝日出》(1869)} \label{fig_Pythag_9}
\end{figure}
柏拉图和伊索克拉底都表示,毕达哥拉斯最为人知的身份是他创立了一种全新的生活方式。[150][151][152] 毕达哥拉斯在克罗托创立的组织被称为“学校”[153][154],但在许多方面,这更像是一个修道院。[155] 成员们以誓言对毕达哥拉斯和彼此承担责任,目的是遵循宗教和禁欲的仪式,并研究他的宗教和哲学理论。[156] 该学派的成员共同分享所有财产[157],并专注于彼此之间的关系,排斥外人。[158][159] 古代资料记载,毕达哥拉斯学派的成员按照斯巴达人风俗共进餐。[160][161] 其中一条毕达哥拉斯的格言是“koinà tà phílōn”(“朋友之间的事物共享”)。[157] 扬布里库斯和波福里乌斯都详细记录了这个学校的组织结构,尽管他们的主要兴趣不在于历史的准确性,而是把毕达哥拉斯塑造为一个神圣的形象,认为他是由神派来造福人类的。[162] 尤其是扬布里库斯,将“毕达哥拉斯的生活方式”呈现为一种与他所处时代基督教修道社区相对立的异教选择。[155] 对毕达哥拉斯学派的人来说,人类能获得的最高奖赏是让灵魂与神灵的生活相融合,从而逃脱轮回的循环。[163] 早期毕达哥拉斯主义中存在两个群体:mathematikoi(“学习者”)和akousmatikoi(“听者”)。[61][164] akousmatikoi传统上被学者认为是“老信徒”,他们信仰神秘主义、数字学和宗教教义;[164] 而mathematikoi则被传统上认为是一个更具知识性、现代化的派别,更倾向于理性主义和科学。[164] 格雷戈里警告说,二者之间可能并没有明确的界限,许多毕达哥拉斯学派成员可能认为这两种方法是兼容的。[164] 数学和音乐的研究可能与阿波罗的崇拜有关。[165] 毕达哥拉斯学派认为音乐是灵魂的净化,正如医学是身体的净化一样。[121] 有一则关于毕达哥拉斯的轶事称,当他遇到一些醉酒的年轻人试图闯入一位贤良妇人的家时,他唱起了一首庄重的曲调,采用长音步格,结果这些男孩的“狂暴意志”被平息了。[121] 毕达哥拉斯学派也特别强调体育锻炼的重要性;[155] 治疗性舞蹈、每天早晨沿着风景优美的路线散步以及体育活动是毕达哥拉斯生活方式的主要组成部分。[155] 他们还建议每天开始和结束时进行冥想。[166]
\subsubsection{禁令和规定}
\begin{figure}[ht]
\centering
\includegraphics[width=6cm]{./figures/3bbaf0412a62f7e8.png}
\caption{1512/1514年的法国语手稿,展示了毕达哥拉斯因厌恶而将脸转向远离蚕豆。} \label{fig_Pythag_10}
\end{figure}
毕达哥拉斯的教义被称为“符号”(symbola)[83],成员们发誓保持沉默,不向非成员透露这些符号。[83][151][167] 那些不遵守社区规则的人会被驱逐[168],剩下的成员会为他们竖立墓碑,仿佛他们已经死去。[168] 一些归因于毕达哥拉斯的“口述格言”(akoúsmata)保存了下来,[12][169] 这些格言涉及毕达哥拉斯学派成员应该如何进行祭祀,如何尊崇神祇,如何“离开这里”,以及如何安葬自己。[170] 其中许多格言强调仪式纯洁性的重要性,并避免玷污。[171][108] 例如,列昂尼德·日穆德(Leonid Zhmud)认为,某句格言很可能可以追溯到毕达哥拉斯本人,这句话禁止他的追随者穿羊毛衣物。[172] 其他现存的口述格言则禁止毕达哥拉斯学派的人打破面包、用剑拨火或捡起碎屑[161],并教导人们应该总是先穿右脚的凉鞋再穿左脚的凉鞋。[161] 然而,这些格言的确切含义常常晦涩难懂。[173] 扬布里库斯保存了亚里士多德对其中一些格言的原始宗教性意图的描述,[174] 但这些显然在后来不再流行,因为波福里乌斯对这些格言提供了明显不同的伦理哲学解释:[175]
\begin{table}[ht]
\centering
\caption\label{Pythagoras}
\begin{tabular}{|c|c|c}
\hline
\textbf{毕达哥拉斯格言} & \textbf{根据亚里士多德/扬布里库斯的原始仪式目的} & \textbf{波福里乌斯的哲学解释}\\
\hline 
毕达哥拉斯格言 & 根据亚里士多德/扬布里库斯的原始仪式目的 & 波福里乌斯的哲学解释\\ 
\hline
不要在戒指上佩戴神祇的图像 [176] & 害怕佩戴这些图像而玷污它们 [176] & 不应让神祇的教义和知识随时显现出来,也不应与大众分享它们。[176]\\
\hline
为神祇从酒杯把手(‘耳’)处倒祭品 [176] & 努力保持神性与人性的严格区分 [176] & 他以隐晦的方式暗示,神祇应该通过音乐来崇拜和赞美,因为音乐是通过耳朵传达的。[176]\\
\hline
\end{tabular}
\end{table}
新入门者据说在完成五年的入会仪式期之前,不允许与毕达哥拉斯见面,[70] 在此期间,他们必须保持沉默。[70] 来源表明,毕达哥拉斯本人在对待女性的态度上异常进步,[85] 毕达哥拉斯学派的女性成员似乎在其运作中发挥了积极作用。[83][85] 扬布里库斯列出了235位著名的毕达哥拉斯学派成员,[84] 其中有17位是女性。[84] 后来的许多著名女性哲学家为新毕达哥拉斯主义的发展做出了贡献。[177]
 
毕达哥拉斯主义还包括一些饮食禁令。[108][161][178] 大多数人认为毕达哥拉斯禁止食用蚕豆[179][161] 和非祭祀性动物的肉类,如鱼和家禽。[172][161] 然而,这两项假设都遭到了反驳。[180][181] 毕达哥拉斯的饮食限制可能是基于对轮回转世教义的信仰。[151][182][183][184] 一些古代作家将毕达哥拉斯描绘为严格执行素食主义者。[g][151][183] 基尼都斯的欧多克索斯(Eudoxus),阿基塔斯的学生写道:“毕达哥拉斯因其纯洁而著称,他避免杀生和杀生者,不仅避免食用动物性食品,甚至远离厨师和猎人。”[185][186] 然而,其他权威资料对此说法提出了反驳。[187] 根据阿里斯托克赛纽斯的说法,[188] 毕达哥拉斯允许食用所有种类的动物食品,除了用于耕作的牛肉和公羊肉。[186][189] 根据赫拉克利德斯·庞提库斯的说法,毕达哥拉斯食用祭品的肉[186] 并为运动员制定了依赖肉类的饮食。[186]
\subsection{传说} 
\begin{figure}[ht]
\centering
\includegraphics[width=8cm]{./figures/834e21dd70ca3de2.png}
\caption{萨尔瓦托尔·罗萨(Salvator Rosa)所作《毕达哥拉斯从冥界归来》(1662)} \label{fig_Pythag_11}
\end{figure}
在毕达哥拉斯生前,他已经成为复杂的圣人传说的主题。[25][190] 亚里士多德将毕达哥拉斯描述为奇迹工作者和某种超自然人物。[191][192] 在一段片段中,亚里士多德写道,毕达哥拉斯有一条金色的大腿,[191][193][194] 他在奥林匹克运动会上公开展示过这条大腿[191][195],并将其展示给北方神阿巴里斯(Abaris the Hyperborean),作为自己是“北方阿波罗”的证明。[191][196] 据说,阿波罗的祭司给了毕达哥拉斯一支魔法箭,他用这支箭飞越长距离并进行仪式净化。[197] 据说他曾同时出现在梅塔庞图姆(Metapontum)和克罗顿(Croton)两个地方。[198][25][195][193][194] 当毕达哥拉斯渡过科萨斯河(现代的巴森托河)时,“几位见证人”报告说他们听到河流用他的名字向他问候。[199][195][193] 在罗马时代,有传说称毕达哥拉斯是阿波罗的儿子。[200][194] 根据穆斯林传统,毕达哥拉斯被认为是由赫尔墨斯(埃及的托特)引导入教的。[201]

据说毕达哥拉斯总是穿着全白的衣服。[191][202] 他还被说成是头戴金色的花环[191],并且穿着类似色雷斯人的裤子。[191] 迪奥根尼·拉尔修斯描述毕达哥拉斯具有非凡的自制力;[203] 他总是保持愉快的心情,[203] 但“完全避免笑声,避免所有像笑话和闲聊之类的放纵行为”。[87] 据说毕达哥拉斯在与动物打交道时取得了非凡的成功。[25][204][195] 亚里士多德的一段片段记载,当一条致命的蛇咬了毕达哥拉斯时,他回咬它并将其杀死。[197][195][193] 赫尔墨斯和扬布里库斯都报告说,毕达哥拉斯曾经劝说一头公牛不吃蚕豆[25][204],并且曾说服一只臭名昭著的破坏性熊发誓永不再伤害任何生物,而这只熊也信守了诺言。[25][204]

Riedweg 认为毕达哥拉斯可能亲自鼓励了这些传说,[190] 但 Gregory 表示没有直接的证据表明这一点。[164] 也有反毕达哥拉斯的传说流传。[205] 迪奥根尼·拉尔修斯转述了萨摩斯的赫里米普斯(Hermippus)讲的一个故事,故事说毕达哥拉斯曾进入一个地下室,告诉大家他要下到冥界。[206] 他在这个房间里呆了几个月,而他的母亲则悄悄记录下了他不在期间发生的一切。[206] 当毕达哥拉斯从这个房间回来后,他讲述了自己在离开期间发生的一切,[206] 使每个人都相信他确实去了冥界,[206] 并让他们信任他照顾他们的妻子。[206]
\subsection{归功于的发现}  
\subsubsection{在数学领域}
\begin{figure}[ht]
\centering
\includegraphics[width=6cm]{./figures/f5af127ddac94872.png}
\caption{毕达哥拉斯定理:直角三角形两条直角边(a 和 b)上的平方面积之和等于斜边(c)上的平方面积。} \label{fig_Pythag_12}
\end{figure}
尽管毕达哥拉斯今天最著名的是他据称的数学发现,[132][207] 但古典历史学家对他是否亲自做出过任何重要的数学贡献存在争议。[148][146] 许多数学和科学发现被归功于毕达哥拉斯,包括他的著名定理,[208] 以及在音乐,[209] 天文学,[210] 和医学领域的发现。[211] 自公元前至少一世纪起,毕达哥拉斯就常被认为是毕达哥拉斯定理的发现者,[212][213] 这是一个几何学定理,表示“在直角三角形中,斜边的平方等于两条直角边平方的和”[214]——即,\(a^{2} + b^{2} = c^{2}\)。 根据一个流行的传说,在发现这个定理后,毕达哥拉斯献祭了一头牛,或者甚至可能是整整一头祭牲给神祇。[214][215] 西塞罗否定了这个故事,认为它是伪造的,[214] 因为更广泛流传的信仰是毕达哥拉斯禁止流血祭献。[214] 波菲里尝试通过声称那头牛实际上是用面团做的来解释这个故事。[214]

毕达哥拉斯定理在毕达哥拉斯之前的几个世纪就被巴比伦人和印度人所知并使用,[216][214][217][218] 但他可能是第一个将其引入希腊的人。[219][217] 一些数学史学家甚至认为他——或他的学生——可能是第一个构造出该定理证明的人。[220] 布尔克特否定了这一观点,认为其不可信,[219] 并指出在古代,毕达哥拉斯从未被认为证明过任何定理。[219] 此外,巴比伦人使用毕达哥拉斯数的方式表明,他们知道这个原理是普遍适用的,并且掌握了一种某种形式的证明,但这一证明尚未在(仍然大量未发表的)楔形文字资料中找到。[h] 毕达哥拉斯的传记作者指出,他还是第一个识别出五种正多面体的人,[132] 也是第一个发现比例理论的人。[132]
\subsubsection{在音乐中}
\begin{figure}[ht]
\centering
\includegraphics[width=6cm]{./figures/ac1f8f28edd08521.png}
\caption{来自弗朗奇诺·加富里奥(Franchino Gafurio)《音乐理论》(Theoria musice,1492)的晚期中世纪木刻,展示了毕达哥拉斯与铃铛和其他按照毕达哥拉斯调音的乐器。[144]} \label{fig_Pythag_13}
\end{figure}
根据传说,毕达哥拉斯有一天经过铁匠铺,听到了铁匠用锤子敲击铁砧的声音。[221][222] 他认为这些锤子的声音除了一个之外都很美妙和和谐。[223] 于是他冲进铁匠铺,开始测试锤子。[223] 他随后意识到,当锤子敲击时,发出的音调与锤子的大小成正比,因此得出结论,音乐是数学的。[222][223]
\subsubsection{在天文学中}
在古代,毕达哥拉斯和他的同时代人厄尔阿的巴门尼德斯都被认为是第一个提出地球是球形的思想者,[224]第一个将地球划分为五个气候带的人,[224]以及第一个识别晨星和昏星为同一天体(即金星)的人。[225] 这两位哲学家中,巴门尼德斯的主张要更有力,[226] 而这些发现归功于毕达哥拉斯的说法可能源自一首伪命题的诗歌。[225] 生活在毕达哥拉斯和巴门尼德斯之后的恩培多克勒斯也知道地球是球形的。[227] 到了公元前五世纪末,这一事实已经在希腊知识分子中广泛接受。[228] 至于晨星和昏星的同一性,巴比伦人早在千年前就已经知道了。[229]
\subsection{古代后的影响}  
\subsubsection{在希腊哲学中}
\begin{figure}[ht]
\centering
\includegraphics[width=6cm]{./figures/cdf565b79b2ba423.png}
\caption{中世纪手稿,内容为卡尔基迪乌斯(Calcidius)拉丁语翻译的柏拉图《蒂迈欧篇》,这是柏拉图对话录中受到最明显毕达哥拉斯学派影响的作品之一[230]。} \label{fig_Pythag_14}
\end{figure}
在公元前四世纪初,毕达哥拉斯学派在大希腊地区、弗利乌斯和底比斯等地存在着相当规模的社区。[231] 大约在同一时期,毕达哥拉斯哲学家阿尔基塔斯在大希腊塔兰图姆城的政治中发挥了重要影响。[232] 根据后来的传统,阿尔基塔斯曾七次当选为战略指挥官("将军"),尽管其他人被禁止连任超过一年。[232] 阿尔基塔斯还是一位著名的数学家和音乐家。[233] 他是柏拉图的亲密朋友,[234] 并在柏拉图的《理想国》中有引用。[235][236] 亚里士多德指出,柏拉图的哲学在很大程度上依赖于毕达哥拉斯学派的教义。[237][238] 西塞罗重复了这一说法,并评论道:“他们说柏拉图学到了所有毕达哥拉斯的学问”(Platonem ferunt didicisse Pythagorea omnia)。[239] 查尔斯·H·卡恩(Charles H. Kahn)认为,柏拉图的中期对话录,包括《美诺篇》《斐多篇》和《理想国》,都有着强烈的“毕达哥拉斯色彩”,[240] 而他最后几部对话录(尤其是《菲雷布斯篇》和《蒂迈欧篇》)[230]则极具毕达哥拉斯的特点。[230]

根据R. M. Hare的说法,柏拉图的《理想国》可能部分基于毕达哥拉斯在克罗顿建立的“组织严密的志同道合的思想者社区”[241]。此外,柏拉图可能借用了毕达哥拉斯的观点,即数学和抽象思维是哲学、科学和道德的可靠基础[241]。柏拉图和毕达哥拉斯共享一种“神秘的灵魂观及其在物质世界中的位置”[241],而且两人可能都受到了奥尔菲教(Orphism)的影响[241]。哲学史学家弗雷德里克·科普尔斯顿(Frederick Copleston)表示,柏拉图可能从毕达哥拉斯学派借用了他的三分法灵魂理论[242]。伯特兰·罗素在他的《西方哲学史》中认为,毕达哥拉斯对柏拉图和其他人的影响如此深远,以至于他应当被视为史上最具影响力的哲学家[243]。他总结道:“我不知道还有哪个人能像他那样,在思想学派中产生如此深远的影响。”[244]

毕达哥拉斯教义在公元前一世纪复兴[245],当时中期柏拉图主义哲学家如厄多罗斯和亚历山大的斐洛赞扬了亚历山大出现的“新”毕达哥拉斯主义[246]。大约在同一时期,新毕达哥拉斯主义逐渐崭露头角[247]。公元一世纪的哲学家泰阿的阿波罗纽斯试图模仿毕达哥拉斯,按照毕达哥拉斯教义生活[248]。公元一世纪后期的新毕达哥拉斯主义哲学家迦底的摩德拉图斯扩展了毕达哥拉斯的数字哲学[248],并可能将灵魂理解为“一种数学和谐”[248]。新毕达哥拉斯主义的数学家和音乐学家尼科马科斯也扩展了毕达哥拉斯的数字学和音乐理论[247]。阿帕美亚的纽门纽斯则在理解柏拉图的教义时结合了毕达哥拉斯学说[249]。
\subsubsection{关于艺术与建筑}
\begin{figure}[ht]
\centering
\includegraphics[width=6cm]{./figures/3dfe7c9147be7ff3.png}
\caption{罗马的哈德良万神殿, Giovanni Paolo Panini 在十八世纪创作的这幅画中描绘的,按照毕达哥拉斯的教义建造[250]。} \label{fig_Pythag_15}
\end{figure}
希腊雕塑试图表现出表象背后的永恒现实[251]。早期的古风雕塑以简单的形式表现生命,并可能受到最早的希腊自然哲学的影响[ i ]。希腊人普遍认为,自然以理想形式自我表现,并由某种类型(εἶδος)来代表,这种类型是通过数学计算得出的[252][253]。当尺寸发生变化时,建筑师试图通过数学传达永恒性[254][255]。莫里斯·鲍尔认为,这些思想影响了毕达哥拉斯及其学生的理论,他们相信“万物皆为数字”[255]。

公元前六世纪,毕达哥拉斯的数字哲学引发了希腊雕塑的革命。[256] 希腊雕塑家和建筑师试图找出美学完美背后的数学关系(规范)。[253] 可能受到毕达哥拉斯思想的启发,[253] 雕塑家波利克利托斯在他的《规范》中写道,美丽在于比例,不是元素(材料)之间的比例,而是各部分之间与整体的相互关系。[253][j] 在希腊建筑的秩序中,每个元素都通过数学关系来计算和构建。赖斯·卡彭特(Rhys Carpenter)指出,2:1 的比例是“多利克柱式的生成比例,在希腊化时期,普通的多利克柱廊就像演奏节奏的音符一样”。[253]

已知最早按照毕达哥拉斯教义设计的建筑是波尔塔·马焦雷大教堂(Porta Maggiore Basilica),这是一座地下大教堂,建于罗马皇帝尼禄统治时期,作为毕达哥拉斯教徒的秘密崇拜场所。大教堂建在地下,是因为毕达哥拉斯强调保密,也因为有传说称毕达哥拉斯曾在萨摩斯岛的一个洞穴中隐居。大教堂的半圆形大殿位于东面,入口大厅则位于西面,以示对初升太阳的敬意。大教堂有一个狭窄的入口,通向一个小池塘,教徒可以在那里进行净化。建筑设计也遵循毕达哥拉斯的数字学原理,每张圣殿中的长椅可容纳七人。三条走廊通向一座祭坛,象征着三部分灵魂接近阿波罗的统一。大殿的半圆拱顶描绘了诗人萨福从莱乌卡底悬崖跳下,胸前抱着她的里拉,而阿波罗站在她下面,伸出右手作出保护的姿势,象征着毕达哥拉斯教义中关于灵魂不朽的理念。圣殿的内部几乎全是白色的,因为白色被毕达哥拉斯教徒视为神圣的颜色。

罗马哈德良皇帝的万神殿也是根据毕达哥拉斯数字学原则建造的。该神庙的圆形平面、中心轴线、半球形圆顶以及与四个主要方向的对齐,象征着毕达哥拉斯关于宇宙秩序的观点。圆顶顶部的单一天窗象征着一元(Monad)和太阳神阿波罗。圆顶上延伸出的二十八根肋条象征着月亮,因为二十八是毕达哥拉斯阴历中的月份数量。肋条下方的五个格状环代表着太阳与月亮的结合。
\subsubsection{在早期基督教时期}
许多早期基督徒对毕达哥拉斯怀有深深的敬意。凯撒利亚的尤西比乌斯(约260年-约340年)在《反对海洛克里斯》一书中称赞毕达哥拉斯,称其守口如瓶、节俭、拥有“非凡”的道德和智慧的教义。[270] 在另一部作品中,尤西比乌斯将毕达哥拉斯与摩西相提并论。[270] 教父杰罗姆(约347年-约420年)在他的书信中称赞毕达哥拉斯的智慧[270],并在另一封信中称毕达哥拉斯信仰灵魂不朽,并认为基督徒继承了这一信仰。[271] 坦比的奥古斯丁(354年-430年)虽然未明确提到毕达哥拉斯,但拒绝了他关于轮回转生的教义,并表达了对他的钦佩。[272] 在《论三位一体》一书中,奥古斯丁称赞毕达哥拉斯谦逊到自称为“哲学爱好者”(philosophos)而非“圣人”(sage)这一点。[273] 在另一段文字中,奥古斯丁为毕达哥拉斯辩护,认为毕达哥拉斯当然从未教授过轮回转生的教义。[273]
\subsection{古代之后的影响}  
\subsubsection{在中世纪}
\begin{figure}[ht]
\centering
\includegraphics[width=6cm]{./figures/486ec1e27ca50a22.png}
\caption{} \label{fig_Pythag_16}
\end{figure}
在中世纪,毕达哥拉斯被尊奉为数学和音乐的创始人,这两者是七艺之一。[274] 他出现在许多中世纪的艺术作品中,包括手抄本和沙特尔大教堂门户的浮雕。[274] 《蒂迈欧篇》是唯一一部以拉丁文翻译在西欧流传下来的柏拉图对话篇,[274] 这使得威廉·孔舍(约1080–1160年)宣称柏拉图是毕达哥拉斯学派的学者。[274] 在阿拔斯王朝期间,出现了一场大规模的翻译运动,将许多希腊文献翻译成阿拉伯文。被归于毕达哥拉斯的作品包括《金句》和他的科学与数学理论的片段。[275] 通过翻译和传播毕达哥拉斯的文本,伊斯兰学者确保了这些知识的保存和更广泛的传播,这些知识本可能因为罗马帝国的衰落和欧洲对古典学问的忽视而失传。[276] 在1430年代,卡马尔多会士修道士安布罗修·特拉弗萨里将狄奥根尼·拉厄尔修的《名哲生平》从希腊文翻译成拉丁文,[274] 而在1460年代,哲学家马尔西利奥·费奇诺也将波尔斐里和伊安布利库斯的《毕达哥拉斯传》翻译成拉丁文,[274] 使得这些作品得以为西方学者阅读和研究。[274] 1494年,希腊的新毕达哥拉斯学者康斯坦丁·拉斯卡里斯出版了《毕达哥拉斯金句》拉丁文版,并出版了他的《语法学》印刷本,[277] 从而使它们得以广泛传播。[277] 1499年,他出版了文艺复兴时期的首部毕达哥拉斯传记,收录在他在梅西纳出版的《西西里和卡拉布里亚著名哲学家传》中。[277]
\subsubsection{在现代科学中}
在他1543年出版的《天体运行论》前言中,尼古拉斯·哥白尼引用了各种毕达哥拉斯学派的学者,称他们为他提出日心模型的最重要影响来源,[274][278] 他故意未提及萨摩斯的阿里斯塔克斯——一位非毕达哥拉斯学派的天文学家,他在公元前四世纪已经提出了完全的日心模型,哥白尼这样做是为了将他的模型塑造为根本上属于毕达哥拉斯学派的。[278] 约翰内斯·开普勒认为自己是毕达哥拉斯学派的成员。[274][279][280] 他相信毕达哥拉斯的宇宙音乐学说[281],并且正是他对这一学说背后的数学方程式的探索,促使他发现了行星运动定律。[281] 开普勒将自己关于这一主题的书命名为《世界和声》(Harmonices Mundi),以此向启发他的毕达哥拉斯教义致敬。[274][282] 在书的结尾部分,开普勒描述了自己在天上的音乐声中入睡,“被喝了一大口...来自毕达哥拉斯杯中的美酒所温暖。”[283] 他还称毕达哥拉斯为所有哥白尼学派的“祖父”。[284]

艾萨克·牛顿坚信毕达哥拉斯关于宇宙数学和谐与秩序的教义。[285] 尽管牛顿以极少承认他人发现而著称,[286] 他还是将万有引力定律的发现归功于毕达哥拉斯。[286] 阿尔伯特·爱因斯坦认为,科学家也可以是“柏拉图主义者或毕达哥拉斯主义者,只要他认为逻辑简洁性视角是他研究不可或缺且有效的工具。”[287] 英国哲学家阿尔弗雷德·诺斯·怀特海德认为,“从某种意义上说,柏拉图和毕达哥拉斯比亚里士多德更接近现代物理科学。前两者是数学家,而亚里士多德是医生的儿子。”[288] 按照这一标准,怀特海德宣称爱因斯坦和其他像他一样的现代科学家“正在遵循纯粹的毕达哥拉斯传统。”[287][289]
\subsubsection{关于素食主义}
\begin{figure}[ht]
\centering
\includegraphics[width=8cm]{./figures/696a85918e7d2088.png}
\caption{彼得·保罗·鲁本斯(Peter Paul Rubens)创作的《毕达哥拉斯提倡素食主义》(1618–1630)受到了奥维德《变形记》中毕达哥拉斯演讲的启发。[290] 这幅画描绘了毕达哥拉斯学派成员拥有丰满的身躯,表明他们认为素食主义有益健康且富有营养。[290]} \label{fig_Pythag_17}
\end{figure}
毕达哥拉斯的虚构描绘出现在奥维德的《变形记》第十五卷中,[291] 在其中他发表了一篇演讲,恳求他的追随者遵守严格的素食饮食。[292] 通过阿瑟·戈尔丁于1567年翻译的奥维德《变形记》英语版,毕达哥拉斯在早期现代时期为讲英语的人所熟知。[292] 约翰·多恩的《灵魂的进程》讨论了演讲中阐述的教义的意义,[293] 米歇尔·德·蒙田在他的论文《论残忍》中引用了这篇演讲不止三次,以表达他对虐待动物的道德反对。[293] 威廉·莎士比亚在他的剧作《威尼斯商人》中提到这篇演讲。[294] 约翰·德赖登在他1700年的作品《古代与现代寓言》中翻译了这段毕达哥拉斯的场景,[293] 约翰·盖伊的1726年寓言《毕达哥拉斯与农民》重申了其主要主题,将食肉与暴政联系起来。[293] 切斯特菲尔德勋爵记录了他转为素食主义的原因,称其动机来源于阅读奥维德《变形记》中毕达哥拉斯的演讲。[293] 直到19世纪40年代“素食主义”一词被创造出来之前,英语中素食者常被称为“毕达哥拉斯学派成员”。[293] 珀西·比希·雪莱写了一首名为《献给毕达哥拉斯饮食》的颂歌,[295] 列夫·托尔斯泰自己也采用了毕达哥拉斯的饮食。[295]
\subsubsection{关于西方神秘主义}
早期现代欧洲的神秘主义深受毕达哥拉斯教义的影响。[274] 德国人文主义学者约翰内斯·罗伊赫林(Johannes Reuchlin,1455–1522)将毕达哥拉斯主义与基督教神学和犹太卡巴拉相结合,[296] 认为卡巴拉和毕达哥拉斯主义都受到摩西传统的启发,[297] 因此毕达哥拉斯可以被视为一位卡巴拉主义者。[297] 在他的对话录《神奇之言》(De verbo mirifico,1494)中,罗伊赫林将毕达哥拉斯的四重三角形与不可言喻的神圣之名YHWH进行比较,[296] 并根据毕达哥拉斯的神秘教义,赋予四个字母每个一个象征意义。[297]

海因里希·科尔内利乌斯·阿格里帕(Heinrich Cornelius Agrippa)的广受欢迎且具有影响力的三卷本著作《隐秘哲学》(De Occulta Philosophia)将毕达哥拉斯称为“一位宗教巫师”[298],并提出毕达哥拉斯的神秘数字学在超天界层面上运作[298],这是一个宗教术语,用来描述他那个时代所用的高天境界。自由石匠们故意将他们的社团模式化为毕达哥拉斯在克罗顿创立的社区。[299] 玫瑰十字教使用了毕达哥拉斯的象征,[274] 罗伯特·弗拉德(Robert Fludd,1574–1637)也是如此,[274] 他认为自己关于音乐的著作受到毕达哥拉斯的启发。[274] 约翰·迪(John Dee)深受毕达哥拉斯思想的影响,[300][298] 尤其是“万物皆由数字构成”的教义。[300][298] 光明会创始人亚当·韦肖普特(Adam Weishaupt)是毕达哥拉斯的崇拜者[301],在他的书《毕达哥拉斯》(1787)中,他主张社会应改革,模仿毕达哥拉斯在克罗顿的公社。[302] 沃尔夫冈·阿马德乌斯·莫扎特在他的歌剧《魔笛》中融入了共济会和毕达哥拉斯的象征。[303] 西尔万·马雷沙尔(Sylvain Maréchal)在他1799年的六卷本传记《毕达哥拉斯的航行》中宣称,所有时代的革命者都是“毕达哥拉斯的继承人”[304]。
\subsubsection{关于文学}
\begin{figure}[ht]
\centering
\includegraphics[width=6cm]{./figures/200ab24220ce0e3a.png}
\caption{但丁·阿利吉耶里的《天堂篇》(Paradiso)中对天堂的描述融入了毕达哥拉斯的数字学。[305]} \label{fig_Pythag_18}
\end{figure}
但丁·阿利吉耶里对毕达哥拉斯的数字学非常着迷[305],并将他的《地狱》、《炼狱》和《天堂》的描述基于毕达哥拉斯的数字。[305] 但丁写道,毕达哥拉斯将“一”视为善,将“多”视为恶[306],并在《天堂篇》第十五章56-57行中宣称:“五与六,若理解了,从一中发出光辉。”[307] 数字十一及其倍数贯穿《神曲》全书,每一卷都有三十三首歌诗,除了《地狱篇》,其有三十四首,第一首作为总介绍。[308] 但丁将地狱第八圈的第九和第十道波尔吉亚描述为分别为二十二英里和十一英里[308],这与分数 ⁠ 22/7⁠ 相对应,这是毕达哥拉斯对圆周率的近似值。[308]

超验主义者将古代的《毕达哥拉斯传》视为如何过上模范生活的指南。[309] 亨利·大卫·梭罗受托马斯·泰勒翻译的扬布里库斯的《毕达哥拉斯传》与斯托巴乌斯的《毕达哥拉斯格言》影响深远[309],他的自然观可能受到毕达哥拉斯关于图像与原型相对应的思想的影响。[309] 毕达哥拉斯的宇宙音乐学说(musica universalis)是梭罗的代表作《瓦尔登湖》中的一个反复出现的主题。[309]
\subsection{另见}  
\begin{itemize}
\item 毕达哥拉斯命名的事物列表  
\item Ex pede Herculem,"从他的脚,我们可以测量赫拉克勒斯"——这是一个格言,基于那个伪传说,毕达哥拉斯通过比萨的赛道长度来估算赫拉克勒斯的身高  
\item 毕达哥拉斯杯——一种恶作剧杯,内置隐藏的虹吸装置,归功于毕达哥拉斯  
\item 毕达哥拉斯均值——算术均值、几何均值和调和均值,据说是毕达哥拉斯研究过的
\end{itemize}
\subsection{注释}\
a./paɪˈθæɡərəs/(发音类似“py-THAG-ər-əs”),[310] 在美国也发音为 /pɪˈθæɡərəs/(类似“pih-THAG-ər-əs”)。[311]  
古希腊语:Πυθαγόρας ὁ Σάμιος,拉丁化为 Pythagóras ho Sámios,意为“萨摩斯的毕达哥拉斯”;或者在伊奥尼亚希腊语中写作 Πυθαγόρης(Pythagórēs)。\\  
b.他的生平日期无法精确确定,但如果亚里士多德的学生亚里士托克塞努斯(见波耳菲里《毕达哥拉斯传》,第9节)的说法大致正确,即他在40岁时为了逃避波利克拉底的暴政而离开萨摩斯,那么可以推测他的出生时间大约在公元前570年,或者稍早一些。他的寿命在古代有不同的估算,但普遍认为他活到了相当高的年龄,最可能是在75岁或80岁左右去世。” ——Guthrie(1967),第173页\\  
c.西塞罗在《图斯库卢姆问答》(Tusc. Qu,第431–433页,第5.3.8–5.3.9节,引用了赫拉克利德·彭提库斯的残篇88,编集者Wehrli),《第欧根尼传》(Diog I, 1.12 和 Diog VIII, §8.8),以及扬布利库斯的《毕达哥拉斯传》(Vit. Pyth, §58)中均有相关记载。Burkert(1960年)试图质疑这一古老的传统,但De Vogel(1966年,第97–102页)和Riedweg(2005年,第92页)为其辩护。\\
d.色诺芬尼关于毕达哥拉斯的诗歌(或哀歌)以下诗歌摘自《第欧根尼传》第八卷(Diog VIII, §1.36)所保存的内容:  
\begin{itemize}
\item 希腊文:  
περὶ δὲ τοῦ ἄλλοτε ἄλλον γεγενῆσθαι (Pythagoras) Ζενοφάνης ἐν ἐλεγείαι προσμαρτυρεῖ, ἧς ἀρχή 'νῦν ... κέλενθον'.  
ὃ δὲ περὶ αὐτοῦ (Pythagoras) φησιν, οὕτως ἔχει καί ... αἰών'.  
νῦν αὖτ' ἄλλον ἔπειμι λόγον, δείξω δὲ κέλευθον.  
καί ποτέ μιν στυφελιζομένου σκύλακος παριόντα  
φασὶν ἐποικτῖραι καὶ τόδε φάσθαι ἔπος  
"παῦσαι μηδὲ ῥάπιζε, ἐπεὶ ἡ φίλου ἀνέρος ἐστὶν ψυχή,  
τὴν ἔγνων φθεγξαμένης αὐδῆς." ——DK 21B7,第130页  
\item 英文翻译:  
“现在我将转向另一个故事,并指出其中的道理……  
据说有一次,他(毕达哥拉斯)路过时,看到一条狗正在被殴打,便说道:‘停下!不要打它!因为这是我朋友的灵魂,我从它的声音中认出来了。’”  
——Burnet(1920年),第118页  
\end{itemize}\\
e.“οὐ τῷ ἀσθενεστάτῳ σοφιστῇ” 这一短语直译为“不是最弱的智者”(见《希罗多德》,第4卷,第297页,第95.3节),在古希腊文学的语境中,这种表达方式常用来暗示相反的意思:即该对象非常强大或重要。Burnet(1920年),第97页。这种修辞手法被称为“曲言法”(litotes),是一种通过否定强调积极品质的委婉表达方式。因此,为避免歧义,A. D. Godley 将该句翻译为“希腊最伟大的教师之一,毕达哥拉斯”。\\
f.一些作者称他是萨摩斯人、勒姆诺斯的提雷尼亚人,或伯罗奔尼撒的菲利亚斯人,并记载其名字为 Marmacus 或 Demaratus(见《第欧根尼传》,第八卷,第1.1节;波耳菲里《毕达哥拉斯传》,第1节和第2节;《贾斯廷历史》,第20卷,第4节;《保萨尼亚斯游记》,第2卷,第13章;扬布利库斯《毕达哥拉斯传》,第2.4节)。波耳菲里还引用了尼安提斯的相互矛盾的记载,称毕达哥拉斯的父亲 Mnesarchus 生于叙利亚的推罗,或者是勒姆诺斯的提雷尼亚人。然而,这种混乱可能源于“推罗”(Tyre)和“提雷尼亚人”(Tyrrhenian)这两个名称的相似性;另有学者认为,波耳菲里自身出身于推罗,可能解释了为何他的传记是三部毕达哥拉斯传记中唯一一部将其父亲与推罗联系起来的。Ferguson(2008年),第11–12页、第198页。

由于这些模糊之处,一些现代学者认为更安全的结论是:“毕达哥拉斯和他的父亲是纯正的希腊血统,完全属于萨摩斯的本地人。”Jacoby & Bollansée(1999年),第256–257页,第73注释。\\
g.正如后来恩培多克勒所做的那样(见亚里士多德,《修辞学》第一卷第14章第2节;塞克斯图斯·恩披里库斯,第九卷第127节)。这也是奥尔甫斯教的教义之一(见阿里斯托芬,《蛙》,第1032行)。\\  
h.仅在大英博物馆,就有大约10万件尚未发表的楔形文字资料。巴比伦对毕达哥拉斯定理证明的认知已由 J. Høyrup 在论文《毕达哥拉斯的“规则”和“定理”——巴比伦与希腊数学关系的镜像》(收录于 J. Renger 主编的《巴比伦:美索不达米亚历史的焦点、早期学术的摇篮、现代的神话》(1999年))中讨论。\\ 
i.对于泰勒斯来说,万物的起源是水;而对于阿那克西曼德,起源是无限(apeiron),这必须被视为一种物质形式。” ——Homann-Wedeking(1968年),第63页。\\  
j.“每个部分(手指、手掌、手臂等)都将其独立存在传递给下一个部分,然后传递给整体。” ——波吕克列托斯的《比例法则》,以及普罗提诺,《九章集》第一卷第六篇第一节(见 Nigel Spivey,第290–294页)。【完整引文需补充】\\
\subsection{引文}
\begin{enumerate}
\item Joost-Gaugier(2006),第143页。  
\item Joost-Gaugier(2006),第11页。  
\item Celenza(2010),第796页。  
\item Ferguson(2008),第4页。  
\item Ferguson(2008),第3–5页。  
\item Gregory(2015),第21–23页。  
\item Copleston(2003),第29页。  
\item Kahn(2001),第2页。  
\item Burkert(1985),第299页。  
\item Joost-Gaugier(2006),第12页。  
\item Riedweg(2005),第62页。  
\item Copleston(2003),第31页。  
\item Joost-Gaugier(2006),第12–13页。  
\item Joost-Gaugier(2006),第13页。  
\item Joost-Gaugier(2006),第14–15页。  
\item Joost-Gaugier(2006),第16页。  
\item Marincola(2001),第59页。  
\item Roberts(2011),第2页。  
\item Sparks(1998),第58页。  
\item Asheri, Lloyd & Corcella(2007)。  
\item Cameron(2004),第156页。  
\item Joost-Gaugier(2006),第88页。  
\item 他自己在《亚里士多德·形而上学》1, 986a 中提及此事。  
\item Burkert(1972),第109页。  
\item Kahn(2001),第5页。  
\item Zhmud(2012),第9页。  
\item Burkert(1972),第106页。  
\item 《希罗多德》,第4卷,第297页,第95节。  
\item Kahn(2001),第6页。  
\item Ferguson(2008),第12页。  
\item Kenny(2004),第9页。  
\item Ferguson(2008),第11页。  
\item 波耳菲里,《毕达哥拉斯传》,第1节、第10节。  
\item 《诸教杂谈》,1.62(2),引自 Afonasin(2012),第15页。
  
\end{enumerate}