% 李代数的同态与同构
% keys Lie|Lie algebra|homomorphism|isomorphism|理想|ideal

\pentry{李代数的子代数、理想与商代数\upref{LieSub}}

和其它代数结构一样,李代数之间可以建立同态与同构映射,其基本思想依然是保持运算结构.

\begin{definition}{李代数的同态}
给定李代数$\mathfrak{g}$和$\mathfrak{h}$和\textbf{线性映射}$f:\mathfrak{g}\to\mathfrak{h}$,如果对于任意$g_1, g_2$还满足$f([g_1, g_2])=[f(g_1), f(g_2)]$,那么称$f$是$\mathfrak{g}$到$\mathfrak{h}$的一个\textbf{同态(homomorphism)}.
\end{definition}

\begin{definition}{李代数的同构}
双射的李代数同态,称为\textbf{同构(isomorphism)}.
\end{definition}

一个李代数$\mathfrak{g}$到自身的同态称为$\mathfrak{g}$上的一个\textbf{自同态(endomorphism)}.$\mathfrak{g}$上的全体自同态构成的集合,配合映射的复合作为二元运算,就可以构成一个\textbf{半群},记为$\opn{End}\mathfrak{g}$.它只是一个半群,因为有一些自同态不是双射\footnote{比如把所有元素都映射到$\bvec{0}$上的映射.},它们就没有逆映射.

$\mathfrak{g}$到自身的同构,称为$\mathfrak{g}$上的一个\textbf{自同构(automorphism)},全体自同构的集合配合映射的符合可以构成一个\textbf{群},记为$\opn{Aut}\mathfrak{g}$











