% 东南大学 2001 年 考研 量子力学
% license Usr
% type Note

\textbf{声明}:“该内容来源于网络公开资料,不保证真实性,如有侵权请联系管理员”

\subsection{(15分)}
质量为 $m$ 的粒子,被无限深方势阱束缚在 $0 < x < L$ 区域中作一维运动,粒子同时受到位于势阱中心、强度为 $\gamma$ 的$\delta$ 势作用。阱内区域的 Schrödinger 方程为:
\[
-\frac{\hbar^2}{2m}\frac{d^2 }{dx^2}\psi(x) + \gamma \delta \left(x - \frac{L}{2}\right)\psi(x) = E \psi(x)~
\]
求第一基态能量 $E$ 满足的超越方程(用 $m$、$\gamma$、$L$ 表达)。
\begin{figure}[ht]
\centering
\includegraphics[width=6cm]{./figures/3b31b18c8f0efb56.png}
\caption{} \label{fig_SEU01_1}
\end{figure}

\subsection{(15分)}
设$t=0$时,粒子的状态为
 \[\psi(x) = A \left( \sin^3 kx + \frac{1}{2} \cos kx \right)~\]
求此时粒子的平均动量和平均动能。

\subsection{(15分)}
考虑哈密顿量为 
\[H = \frac{p^2}{2m} + \frac{m \omega^2 x^2}{2}~\]
的一维谐振子
