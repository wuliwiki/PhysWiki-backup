% 南京航空航天大学 2005 量子真题
% license Usr
% type Note

\textbf{声明}:“该内容来源于网络公开资料,不保证真实性,如有侵权请联系管理员”

1. 一粒子在一维势 \( V(x) \) 中运动,试求确定其稳态能级的方程。
$$V(x) = \begin{cases} U_0, & x < -a \\\\0, & -a < x < 0 \\\\\infty, & x > 0 \end{cases}~$$
(30分)

2. 证明:在 $\left( I^2, I_z \right)$ 的共同本征态下,$\overline{I}_x = \overline{I}_y = 0$,并求 $\overline{\left( \Delta I_x \right)^2}$) 和 $\overline{\left( \Delta I_y \right)^2}$)。

(30 分)

3. 一维运动粒子的状态是 $\psi(x) = \begin{cases} Axe^{-\lambda x} &, x \geq 0 \\\\0 & ,x < 0\end{cases}$
其中 $\lambda > 0$,求:
(1) 粒子动量的几率分布函数;
(2) 粒子的平均动量。

(20 分)

4. 粒子在一维势 $U(x) = \begin{cases} 0 & x \leq 0 \\\\ \frac{1}{2}\mu \omega^2 x^2 & x > 0 \end{cases}$
(1) 不解方程,写出粒子能级与波函数的表达式(设已归一化),并说明理由:
(2) 加入微扰 $H' = \beta \cos \lambda x$,其中 $\beta$ 为常数,$\lambda \ll 1$,求能级至二级修正,波函数至一级修正。

(30 分)

 5. 有一个自旋 1/2,磁矩为 $\mu$,电荷为 0 的粒子,置于磁场 $\mathbf{B_0}$ 中。开始时 ($t=0$) 磁场沿 $z$ 方向,$\mathbf{B_0} = (0, 0, B_0)$,粒子处于 $\delta_z$ 的本征态 $\begin{pmatrix} 0 \\\\ 1 \end{pmatrix}$,即 $σ_z = -1$。$t > 0$ 时再加上沿 $x$ 方向的较弱的磁场 $\mathbf{B_1} = (B_1, 0, 0)$,从而 $\mathbf{B} = \mathbf{B_0} + \mathbf{B_1} = (B_1, 0, B_0)$。

求 $t > 0$ 时粒子的自旋态,以及测得自旋向上 ($\delta_z = 1$) 的几率。 (20 分)

