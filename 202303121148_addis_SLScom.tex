% SLISC 的编译和测试

\begin{issues}
\issueDraft
\end{issues}

\pentry{SLISC 库概述\upref{SLISC}, 在 Linux 上编译 C/C++ 程序\upref{linCpp}}

\subsection{直接使用头文件}
和 C++ 标准库一样, SLISC 一般无需编译。 如果你不想修改 SLISC 源码, 只需要把 \verb|SLISC| 文件夹复制到项目文件夹中, 并在 C++ 源码中使用相应的头文件即可。

下载 SLISC 源码, 创建 “SLISC 库概述\upref{SLISC}” 中的 \verb|intro.cpp|。 注意如果 \verb|intro.cpp| 和 \verb|SLISC| 文件夹不在同一个目录, 那么你需要修改 \verb|#include| 中的相对路径, 也可以用绝对路径。

要编译, 用 \verb|g++ -D SLS_USE_INT_AS_LONG intro.cpp -o intro|。 其中 \verb|-D SLS_USE_INT_AS_LONG| 用于预定义宏, 声明我们使用 4 字节的有符号整数作为 \verb|Long| 类型(一般用于数组索引)。 等效地, 你也可以在 \verb|intro.cpp| 的最开始插入 \verb|#define SLS_USE_INT_AS_LONG|, 注意要在第一个使用 SLISC 头文件的 \verb|#include| 之前。

编译成功后, 运行程序就得到输出结果
\begin{lstlisting}[language=none]
u = 
0   1   2   

a 有 2 行和 3 列, 共计 6 个元素。
1.1   0   4.4   
0   9.9   5.5
\end{lstlisting}
以及两个输出文件 \verb|data0.matb| 和 \verb|data.matb|。

\subsection{使用 64 位索引}
若使用上面的 \verb|SLS_USE_INT_AS_LONG| 选项, \verb|Long| 类型的最大值(也就是数组最多的元素个数)是 $2^{31}-1 = 2147483647$。 也就是 SLISC 只能处理大约 2.1GB 的 \verb|char| 数组或者 16.8GB 的 \verb|double| 数组。 这在一般情况下是够用的, 但如果要数组可能会更大, 那么就不能定义该选项, 且需要使用 64 位版本的头文件, 位于 \verb|SLISC-64| 目录下。 我们可以直接把 \verb|SLISC-64/*.h| 复制并覆盖到 \verb|SLISC| 中。

\subsection{生成头文件}

注意如果头文件 \verb|SLISC/xxx.h| 有对应的 \verb|SLISC/xxx.h.in| 模板文件, 那么这些头文件都是自动生成的。 要重新生成 64 位版本的头文件, 首先确保你的机器上安装了 \verb|octave 4.1| 或以上的版本以及 \verb|make|\upref{Make}, 然后在 \verb|SLISC0| 目录下运行 \verb|make [选项] h| 即可。 例如 \verb|SLISC| 中的被生成的头文件是使用 \verb|make h| 生成的, \verb|SLISC-64| 中的则是使用 \verb|make opt_long32=false h| 生成到 \verb|SLISC| 并复制过去的。 \verb|SLISC-64q| 则使用 \verb|make opt_long32=false opt_quadmath=true h| 生成, 支持 4 精度浮点数\upref{FltCpp}。

\subsection{编译测试}
SLISC 中几乎每个头文件都有对应的测试代码, 在 \verb|test/test_xxx.cpp| 中, 一般对应同名的 \verb|SLISC/xxx.h|。 为了测试代码在你使用的环境(CPU 架构和操作系统)中能正常运行, 强烈建议编译并运行测试。

最小测试的编译命令为 \verb|make opt_min=true -j8|, 然后运行使用 \verb|./main.x|。 这里的 \verb|-j8| 代表使用 8 个进程并行编译, 你也可以改为其他的数字, 匹配你电脑 cpu 的核数即可。 如果不使用 \verb|-j| 选项, 默认单线程编译,可能时间较长。

如果你想测试其他非默认的编译选项, 就先生成头文件 \verb|make [选项] h| 生成头文件到 \verb|SLISC|(也可以从对应的 \verb|SLISC-xxx| 把头文件覆盖过去, 然后用 \verb|make [选项] -j8|。 完成后同样运行 \verb|./main.x| 即可测试。

以下列出一些\textbf{常用选项}, 注意只有改变 \verb|opt_long32| 或 \verb|opt_quadmath| 需要重新生成头文件。
\begin{itemize}
\item \verb|opt_compiler|: 选择编译器, 支持 \verb|g++|(默认), \verb|clang++|, \verb|icpc|, \verb|icpx|。 非默认编译器不支持 SLISC 的所有功能(例如四精度类型)。
\item \verb|opt_long32|: \verb|true|(默认)使用 32 位的 \verb|Long| 类型, \verb|false| 使用 64 位的 \verb|Long| 类型。
\item \verb|opt_debug|: \verb|true|(默认)使用 debug 模式编译, 会检查索引是否超出数组边界, 矩阵尺寸是否合法等。 还会在可执行文件中生成调试信息, 以便使用 \verb|gdb|\upref{gdbNt} 调试。 \verb|false|: Release 模式, 开启编译器优化, 没有边界和尺寸检查等, 不生成调试信息。
\item \verb|opt_asan|: \verb|true|(默认) 开启 address sanitizer, 进一步检测内存泄漏。
\item \verb|opt_std|: C++ 标准, 支持 \verb|c++11|(默认) 及以上标准。
\item \verb|opt_static|: \verb|false|(默认)动态编译, 可执行文件体积较小, 但需要依赖动态链接库文件(取决于系统版本, 详见下文), 所以无法将可执行文件直接复制到其他系统使用。 \verb|true|: 静态编译, 不支持 address sanitizer, 但优点是编译后没有其他依赖, 一般复制到相似的系统中也能正常运行。
\item \verb|opt_min|: \verb|true| 兼容模式, 功能最少, 无需安装任何第三方库。 \verb|false| 使用默认的第三方库。
\end{itemize}

\subsubsection{直接下载依赖库(推荐)}
\addTODO{添加网盘链接}
\begin{itemize}
\item 依赖库取决于系统要查看 Ubuntu 系统的版本用 \verb|lsb_release -a| 命令。
\item 可以在 GitHub 上下载对应的版本(推荐): \href{https://github.com/MacroUniverse/SLISC0-libs-x64-ubuntu16.04}{Ubuntu16.04}, \href{https://github.com/MacroUniverse/SLISC0-libs-x64-ubuntu18.04}{Ubuntu18.04}, \href{https://github.com/MacroUniverse/SLISC0-libs-x64-ubuntu20.04}{Ubuntu20.04}, \href{https://github.com/MacroUniverse/SLISC0-libs-x64-ubuntu22.04}{Ubuntu22.04}, \href{https://github.com/MacroUniverse/SLISC0-libs-x64-centos7.9.2009}{CentOS7.9} (注意 \verb|Ubuntu16.04| 不支持 MPLAPACK 库)
\item 除了特殊说明, 解压后的文件夹包含 SLISC 所需的所有依赖, 无需任何其他安装。 加压出的文件夹 \verb|SLISC0-libs-*| 需要和 \verb|SLISC0| 文件夹放在同一目录。
\item 如果国内访问 GitHub 较慢, 可以从网盘下载(不保证及时更新)。
\item 注意文件夹中存在软链, 但 windows 不支持软链(WSL 支持), 所以应该用 Linux 命令行解压(命令 \verb|unzip 文件名.zip|)。
\item 在解压的文件夹中, 每个子文件夹是一个第三方库, 其中的 \verb|include| 中是头文件, \verb|lib| 中是库文件, 静态库的后缀名为 \verb|.a|, 用于静态编译; 动态库后缀名为 \verb|.so.版本号|。
\item 解压以后, 在解压的文件夹中使用 \verb|source setup.sh| 设置把库的路径添加到环境变量 \verb|CPATH, LIBRARY_PATH, LD_LIBRARY_PATH| 中。 他们分别是头文件的搜索路径, 编译时库文件的搜索路径, 运行时动态链接库的搜索路径。 除此之外, 该脚本也会把对一些超出 GitHub 尺寸限制的大文件进行拼接。
\item 要查看环境变量, 如 \verb`echo $CPATH | tr : \\n`。
\end{itemize}

可以用以下选项开始或关闭第三方库。
\begin{itemize}
\item \verb|opt_lapack|: \verb|reference|(默认)使用 Netlib 的基准版 BLAS\upref{BLAS} 和 LAPACK\upref{Lapack}, \verb|openblas| 使用 OpenBlas 和基于它的 LAPACK (性能较高), \verb|mkl| 使用 Intel 的 MKL (性能最高)。
\item \verb|opt_mplapack|: \verb|true| 或 \verb|false|, 默认取 \verb|opt_quadmath| 的值。 四精度版本的 BLAS 和 LAPACK。 如果编译时提示 \verb|_Float128| 未定义, 需要开启选项 \verb|opt_no__Float128|
\item \verb|opt_boost|: \verb|true|(默认) 或 \verb|false|, 使用 Boost 库\upref{Boost}。
\item \verb|opt_gsl|: \verb|true|(默认) 或 \verb|false|, 使用 GSL 库\upref{GSL}。
\item \verb|opt_eigen|: \verb|true|(默认) 或 \verb|false|, 使用 Eigen 库\upref{Eigen}。
\item \verb|opt_arb|: \verb|true|(默认) 或 \verb|false|, 使用 Arb 库\upref{ArbLib}。
\item \verb|opt_arpack|: \verb|true|(默认) 或 \verb|false|, 使用 Arpack++ 库\upref{Arpkpp}
\item \verb|opt_sqlite|: \verb|true|(默认) 或 \verb|false|, 使用 SqLite 3 库\upref{SQLite}。
\item \verb|opt_matfile|: \verb|true| 或 \verb|false|(默认), 读写 Matlab 文件(已停止维护)。
\end{itemize}

\subsubsection{用包管理程序安装依赖库(不推荐)}
除了直接下载, 也可以用包管理程序如 \verb|apt| 安装依赖库, 但有的库会确实 64 位版本或者静态库文件等。 例如在 \verb|Ubuntu| 中可以使用以下命令: 
\begin{lstlisting}[language=bash]
apt -y update && apt -y upgrade
apt install -y vim git make g++ gdb gfortran libarpack++2-dev \
liblapacke-dev libsqlite3-dev libgmp-dev libflint-arb-dev \
libflint-dev libgsl-dev libboost-filesystem-dev
\end{lstlisting}
