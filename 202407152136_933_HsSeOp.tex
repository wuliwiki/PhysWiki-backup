% 集合的基本运算(高中)
% keys 集合|基本运算|运算
% license Usr
% type Tutor
\pentry{集合\nref{nod_HsSet}}{nod_120e}

\begin{issues}
\issueDraft
\end{issues}

\subsection{交集与并集}
\begin{definition}{交集}
一般地,由既属于集合 $A$ 又属于集合 $B$ 的所有元素组成的集合叫做 $A$ 与 $B$ 的\textbf{交集(intersection set)},记作 $A \cap B$(读作“A交B”),即
\begin{equation}
A\cap B = \begin{Bmatrix} x|x\in A \wedge x\in B \end{Bmatrix}~.
\end{equation}
\end{definition}

根据交集定义,可得
\begin{equation}
\begin{aligned}
&A\cap B = B\cap A~, \\ 
&A\cap B \subseteq A ~,\\
&A\cap B \subseteq B~, \\
&A\cap A = A ~,\\
&A\cap \varnothing = \varnothing~.
\end{aligned}
\end{equation}

\begin{definition}{并集}
由属于集合 $A$ 或属于集合 $B$ 的所有元素组成的集合,叫作 $A$ 与 $B$ 的\textbf{并集(union set)},记作 $A\cup B$(读作“A并B”),即
\begin{equation}
A\cup B = \begin{Bmatrix}x|x\in A \vee x\in B\end{Bmatrix}~.
\end{equation}
\end{definition}

根据并集的定义,可得
\begin{equation}
\begin{aligned}
&A\cup B = B\cup A~, \\
&A\subseteq A\cup B ~,\\
&A\cup A = A ~,\\
&A\cup \varnothing = A~.
\end{aligned}
\end{equation}
\subsection{补集}
\begin{definition}{补集}
设 $U$ 是全集,$A$ 是 $U$ 的一个子集(即$A\subseteq U$),则由 $U$ 中所有不属于 $A$ 的元素组成的集合,叫作 $U$ 中子集 $A$ 的\textbf{补集(complementary set)}(或\textbf{余集}),记作$\complement_UA$,即
\begin{equation}
\complement_UA = \begin{Bmatrix}x|x\in U \wedge \notin A\end{Bmatrix}~.
\end{equation}
\end{definition}

由补集定义可得,
\begin{equation}
\begin{aligned}
&A\cup (\complement_UA) = U~, \\
&A\cap (\complement_UA) = \varnothing~.
\end{aligned}
\end{equation}
\subsection{*德摩根定律}

$$(\bigcup_{i\in I} A_i)^C=\bigcap_{i\in I} A^C_i$$
\begin{equation}

\end{equation}

\begin{equation}
(\bigcap_{i\in I} A_i)^C=\bigcup_{i\in I} {A_i^C}
\end{equation}

$$

$$

\subsection{*集合运算与逻辑运算的关系}
