% 平方反比定律
% license CCBYSA3
% type Wiki

(本文根据 CC-BY-SA 协议转载自原搜狗科学百科对英文维基百科的翻译)
\begin{itemize}
\item 在物理学中,如果一个物理定律中,某个物理量在空间中某点处的数值或强度与该点与该物理量源点的距离的平方成反比,则这个物理定律称为一个平方反比定律。其根本原因在于点源辐射形式的物理量因在三维空间中的扩散而按与点源距离的平方而衰减。
\item 雷达信号能量在传输和反射过程中都会在空间中扩散,因此在往返路径中均按照平方反比律衰减,这意味着雷达接收的反射信号能量将与距离的四次方成反比。
\item 为了防止在信号传播过程中的能量衰减,可以使用某些方法限制其在空间中的扩散,例如波导管可以让电磁波在其中传播,类似于排水管道之于水的作用,再如枪管将高温气体的膨胀限制在单一方向,以减少气体推进子弹加速过程中的能量损失。
\end{itemize}
\subsection{公式}
数学表示:
\begin{equation}
intensity \propto \frac{1}{distance^2}~
\end{equation}

也可以写为:\begin{equation}
\frac{intensity_1}{intensity_2}=\frac{distance_2^2}{distance_1^2}~ 
\end{equation}

或者写作守恒量的形式:\begin{equation}
intensity_1 * distance_1^2~
\end{equation}
