% 原码、反码、补码
% keys 编码|位运算|ALU|计算机组成原理
% license Xiao
% type Wiki
\begin{issues}
\issueDraft
\end{issues}

\subsection{原码(True form)}

原码即“未经更改”的码,是指一个二进制数左边加上符号位后所得到的码,且当二进制数大于0时,符号位为0;二进制数小于0时,符号位为1;二进制数等于0时,符号位可以为0或1(+0/-0)。

使用n位原码表示\textbf{有符号数}时,范围是 $-(2^{n-1}-1)\sim +(2^{n-1}-1)$。 当n=8时,这个范围就是$-127\sim +127 $;表示\textbf{无符号数}时,由于不需要考虑数的正负,就不需要用一位来表示符号位,n位机器数全部用来表示是数值,这时表示数的范围就是$0\sim 2^{n}-1$。当$n=8$时,这个范围就是$0\sim 255$。


\textbf{优点:}
简单直观,原码易于人类理解和计算(与真值转换容易)。

\textbf{缺点:}原码不能直接被用于运算。
对于加法运算,例如,数学上,1+(-1)=0,但用原码进行运算时:


\begin{example}{}
00000001+10000001=10000010
\end{example}


该结果对应数值为-2。显然出错了;
对于减法运算,原码减法需要先将减数取反加 1,才能得到正确的数学结果。

也就是说:\textbf{原码的运算,必须将符号位和其他位分开},这就增加了硬件的开销和复杂性。(也有人将该符号问题称作正负0现象)


\subsection{反码(1's complement)}

我们发现:
原码最大的问题就在于一个数加上它的相反数不等于0,于是反码的设计思想就是冲着解决这一点,既然一个负数是一个正数的相反数,那干脆用一个正数按位取反来表示负数。

在反码表示法中,正数和0的反码与其原码相同;负数的反码则是将原码(除符号位外)的每一位取反。

\textbf{缺点:}
虽然解决了相反数想加不等于0的问题,但是反码不能直接做减法,并且存在多余的负零 (eg. 1111_1111)

\begin{example}{}
0001+1110=1111,1+(-1)=-0;

1110+1100=1010,(-1)+(-3)=-5。
\end{example}

\subsection{补码(2's complement)}

正数和0的补码就是该数字本身再补上最高比特0。负数的补码则是将其绝对值按位取反再加1。


\textbf{优点:}
在加法或减法处理中,不需因为数字的正负而使用不同的计算方式。只要一种加法电路就可以处理各种有号数加法,而且减法可以用一个数加上另一个数的补码来表示,因此只要有加法电路及补码电路即可完成各种有号数加法及减法,在电路设计上相当方便。

另外,补码系统的0就只有一个表示方式,这和反码系统不同(在反码系统中,0有二种表示方式),因此在判断数字是否为0时,只要比较一次即可。




\subsection{补码的设计}

补码的规则看起来很怪,我们可以\textbf{换一种角度理解:}

在设计一个编码时,我们可以把编码后的数围成一个圆(类似一个时钟),我们希望在这个钟表上:
\begin{enumerate}
\item 
无冲突,每个数值的位置都是独一无二,有唯一的0,
\item 
连续性,每次运算(+1)相当于时钟顺时针移动一个单位,
\end{enumerate}

从上述角度看,补码是一种很和谐的编码。








% 参考 https://www.cnblogs.com/zhangziqiu/archive/2011/03/30/computercode.html remove by lzq
