% 线性引力
% 线性近似|线性爱因斯坦方程|弱引力场

\begin{issues}
\issueMissDepend
\issueDraft
\end{issues}



\subsection{线性引力理论}
爱因斯坦场方程\footnote{在广义相对论中,常常采用几何单位制,也即是$c=\hbar=G=1$}如下

\begin{equation}
G_{\mu \nu} = R_{\mu \nu} - \frac{1}{2}g_{\mu\nu}R = 8 G\pi T_{\mu\nu}
\end{equation}

由于采用几何量描述,而使其十分简洁而很难看出它包含着一系列的复杂的非线性微分方程.一方面,寻求严格满足爱因斯坦场方程的特定解是一个漫长而艰难的过程,许多数学天才也投入其中,取得了一些出色的结果.另一方面,在大多数情况中引力场都很微弱,我们可以采用近似处理使爱因斯坦场方程线性化,简而言之,我们实际上就是在对时空进行一阶线性微扰.



我们先考虑背景时空为闵氏时空的简单情况,之后也可将背景时空推广为一般时空.

\begin{equation}
g_{\mu\nu} = \eta_{\mu\nu} + h_{\mu\nu},\quad |h_{\mu\nu}|<<1
\end{equation}

例如,对于太阳系来说,$|h_{\mu\nu}| \sim 10^{-6}$.


\subsection{史瓦西时空解}

我们可以将史瓦西时空看作对于平直闵氏时空的围绕


\subsection{规范不变性}


\subsection{推广到一般时空}

背景时空的选择其实是任意的,我们同样可以对其进行线性微扰.

