% 范德蒙恒等式
% 排列组合|选择的展开定理

\pentry{二项式定理\upref{BiNor}}

\footnote{参考 Wikipedia \href{https://en.wikipedia.org/wiki/Vandermonde's_identity}{相关页面}}\textbf{范德蒙恒等式(Vandermonde's identity)}是指
\begin{equation}
\pmat{a + b\\ n} = \sum_i \pmat{a\\ i} \pmat{b\\ n-i} = \sum\limits_i \pmat{a\\n-i}\pmat{b\\i}
\end{equation}
其中求和是对所有使得表达式有意义的非负整数 $i$ 进行的.

\subsection{证明1}

假设有编了号的 $a+b$ 个小球. 不分顺序抓取 $n$ 个, 求总共有几种情况(用 $N$ 表示).

方法1:根据定义, 有 $N = \pmat{a+b\\n}$ 种情况.

方法2:先把球分成 $A$,  $B$ 两组, 分别有 $a$ 个和 $b$ 个. 如果在 $A$ 组中抽取 $i$ 个球(有 $\pmat{a\\i}$ 种情况), 在 $B$ 组中只能抽取  $n - i$ 个(有 $\pmat{b\\n-i}$ 种情况), 所以一个 $i$ 对应 $\pmat{a\\i} \pmat{b\\n-i}$ 种情况. 所有可能的 $i$ 一共有 $N = \sum_i \pmat{a\\i} \pmat{b\\n-i}$ 种情况.

由于这个问题只有一个答案, 所以有 $\pmat{a+b\\n} = \sum_i \pmat{a\\i} \pmat{b\\n-i}$. 

但 $i$ 的范围具体从多少取到多少, 由 $a$,  $b$ 是否大于 $n$ 来决定. 当 $a,b$ 都大于 $n$ 时, $i$ 可以从 0 取到 $n$,  如果其中至少有一个小于 $n$,  那么 $i$ 的取值不能使 $C$ 的上标大于下标.

证毕.

\subsection{证明2}
我们也可以通过二项式定理来证明
\begin{equation}
\begin{aligned}
(1 + x)^{m+n} &= \sum_{r=0}^{m+n} \pmat{m+n\\r}x^r\\
&=(1+x)^m (1+x)^n\\
&=\sum_0^m \pmat{m\\i}x^i  \sum_{j=0}^n \pmat{n\\j} x^j\\
&=\sum_{r=0}^{m+n} \sum_{k=0}^r \pmat{m\\k} \pmat{n\\r-k} x^r
\end{aligned}
\end{equation}
