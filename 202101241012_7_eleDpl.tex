% 电偶极子
% 偶极子|电场|电荷

% 这是一级词条, 先介绍两个点电荷

\pentry{电场\upref{Efield}}

\subsection{电偶极子}

在电磁学和基础的电动力学框架下,“电荷”可以被认为是一种粒子,它没有大小,同时具有一个特征,该特征用标量(数字)来刻画,也就是电荷.电荷能产生电场,同时由于电荷是一个标量,没有方向之分,这种电场也就是球对称的.

现在我们介绍的模型可以被认为是一种新的粒子,也没有大小,同时具有一个特征,称作电偶极矩,但是该特征用一个向量来刻画.电偶极子也能产生电场,但是由于电偶极矩是向量,有方向之分,这种电场就不再是球对称的了.我们从实际模型出发,来导入电偶极子的概念.

\begin{definition}{物理电偶极子}
在电磁学框架下,给定两个等量异号的电荷,假设它们的电荷量分别是$\pm q$,其中$q\geq 0$.如果负电荷到正电荷的位移向量是$\bvec{r}$,那么我们把这两个粒子构成的系统看成一个叫\textbf{物理电偶极子(physical electric dipole)}的对象.物理电偶极子电荷量为$0$,但是它有另一个性质叫\textbf{电偶极矩(electric dipole moment)}来取代电荷的概念.电偶极矩定义为$q\bvec{r}$,是一个向量.
\end{definition}

电磁学框架下的“粒子”,是没有大小但可以具有确定空间位置的一个存在.我们希望用同样的视角来看待电偶极子,把它也当成一个没有大小但是具有确定空间位置的一个存在.物理电偶极子显然不能达到这个效果,因为它必须有长度,正负电荷的位置是不一样的——如果一样,电偶极矩就为$0$了.不过,如果我们考虑问题的空间尺度远大于电偶极子的长度,那么我们可以近似地看成正负电荷的位置是一样的,这样就可以确定电偶极子的位置了.更进一步,为了形式上简洁,我们抽象出了一个原本不存在的新概念,理想电偶极子.

\begin{definition}{理想电偶极子}
在电磁学框架下引入新的粒子,称作\textbf{理想电偶极子(ideal electric dipole)}.理想电偶极子是一个点粒子,其标量电荷量为$0$,但是有一个向量\textbf{电偶极矩(electric dipole moment)}.理想电偶极子能产生电场,该电场的分布取决于电偶极矩的大小和方向.
\end{definition}

理想电偶极子是物理电偶极子的抽象,抹去了“大小”的概念.从引入理想电偶极子的角度考虑,我们希望的是,在可以忽略物理电偶极子长度的情形下,我们可以直接把它们视为理想电偶极子.这就使得理想电偶极子的性质需要从物理电偶极子来导出了.

这些性质包括理想电偶极子是如何产生电场的.导出的方法是,我们首先计算物理电偶极子的电场分布,这是电磁学框架下可以做到的;然后在\textbf{保持电偶极矩不变}的情况下\textbf{缩短两电荷间的距离}.正负电荷的间距趋于$0$时的电场分布,就是理想电偶极子的电场分布了.

我们接下来就应用这个思想来推导电偶极子产生的电场和电势.


\subsection{物理电偶极子的电场}

令空间中两个位置不同的点电荷具有等量的异号电荷, 则他们构成一对\textbf{物理电偶极子},令他们的电荷量分别为 $-q$ 和 $q$ , 位置矢量分别为 $\bvec r_1$ 和 $\bvec r_2$, 则它们的总电场为两个电荷各自电场的矢量和(见\autoref{Efield_eq2}~\upref{Efield})

\begin{equation}
\bvec E(\bvec r) = \frac{1}{4\pi\epsilon_0} \qty(\frac{-q(\bvec r - \bvec r_1)}{\abs{\bvec r - \bvec r_1}^3} + \frac{q(\bvec r - \bvec r_2)}{\abs{\bvec r - \bvec r_2}^3})
\end{equation}
总电势同样是两个点电荷的电势之和% 链接未完成: N 个点电荷的电势
\begin{equation}\label{eleDpl_eq4}
V(\bvec r) = \frac{1}{4\pi\epsilon_0} \qty(\frac{-q}{\abs{\bvec r - \bvec r_1}} + \frac{q}{\abs{\bvec r - \bvec r_2}})
\end{equation}

% 我们常常讨论的是偶极子远处的电势和电场分布, 即 $r \gg \abs{\bvec r_2 - \bvec r}_1$ 的情况. 定义\textbf{电偶极矩(electric dipole moment)}为
% \begin{equation}\label{eleDpl_eq1}
% \bvec p = -q \bvec r_1 + q \bvec r_2 = q (\bvec r_2 - \bvec r_1)
% \end{equation}

这个电偶极子的\textbf{电偶极矩(electric dipole moment)}为
\begin{equation}\label{eleDpl_eq1}
\bvec p = -q \bvec r_1 + q \bvec r_2 = q (\bvec r_2 - \bvec r_1)
\end{equation} 

如果改用电偶极矩来描述$\abs{\bvec{r}}\gg\\\abs{\bvec{r}_1}+\abs{\bvec{r}_2}$时的情况,即所谓的“远处”,则电势分布为
\begin{equation}\label{eleDpl_eq2}
V_d(\bvec r) = \frac{1}{4\pi\epsilon_0} \frac{\bvec p\vdot \bvec r}{r^3} = \frac{1}{4\pi\epsilon_0} \frac{\bvec p \vdot \uvec r}{r^2}
\end{equation}
对电势求梯度就能得到远处的电场分布
\begin{equation}\label{eleDpl_eq3}
\bvec E_d(\bvec r) = \frac{1}{4\pi\epsilon_0} \frac{1}{r^3} [3(\bvec p \vdot \uvec r) \uvec r - \bvec p]
\end{equation}
注意这两个量分别随 $r$ 的平方反比和三次方反比下降.

从远处,也就是$\abs{\bvec{r}}\gg\\\abs{\bvec{r}_1}+\abs{\bvec{r}_2}$的地方看来,电偶极子的长度是可以忽略的,所以我们可以把\autoref{eleDpl_eq2} 和\autoref{eleDpl_eq3} 看成是\textbf{理想电偶极子}的电势和电场分布.

换句话说,物理电偶极子在远处时的性质,就是理想电偶极子在任意位置的性质.这就是我们需要理想电偶极子所具有的性质.



\subsection{偶极子远处性质的推导}

\subsubsection{电势}

我们从\autoref{eleDpl_eq4} 出发,并且考虑到$\abs{\bvec{r}}\gg\\\abs{\bvec{r}_1}+\abs{\bvec{r}_2}$,$\abs{\bvec{a}-\bvec{b}}=\sqrt{\bvec{a}^2+\bvec{b}^2-2\bvec{a}\cdot\bvec{b}}$,以及$\lim_{\dd x\rightarrow 0}\sqrt{x+\dd x}=\sqrt{x}+\dd x/2\sqrt{x}$,从而得到\footnote{符号$\doteq$此处指约等于,或者取极限后的结果.}:
\begin{equation}
\begin{aligned}
V(\bvec{r})&=\frac{1}{4\pi\epsilon_0}\qty(\frac{-q}{\abs{\bvec{r}-\bvec{r}_1}}+\frac{q}{\abs{\bvec{r}-\bvec{r}_2}})\\
&=\frac{q}{4\pi\epsilon_0}\qty(\frac{\abs{\bvec{r}-\bvec{r}_1}-\abs{\bvec{r}-\bvec{r}_1}}{\abs{\bvec{r}-\bvec{r}_2}\abs{\bvec{r}-\bvec{r}_2}})\\
&=\frac{q}{4\pi\epsilon_0}\qty(\frac{\sqrt{\bvec{r}^2+\bvec{r}_1^2-2\bvec{r}\cdot\bvec{r}_2}-\sqrt{\bvec{r}^2+\bvec{r}_2^2-2\bvec{r}\cdot\bvec{r}_1}}{\abs{\bvec{r}-\bvec{r}_1}\abs{\bvec{r}-\bvec{r}_2}})\\
\doteq V_d(\bvec{r})&=\frac{q}{4\pi\epsilon_0}\qty(\frac{\sqrt{\bvec{r}^2}-\frac{2\bvec{r}\cdot\bvec{r}_1}{2\sqrt{\bvec{r}^2}}-\sqrt{\bvec{r}^2}+\frac{2\bvec{r}\cdot\bvec{r}_2}{2\sqrt{\bvec{r}^2}}}{\bvec{r}^2})\\
&=\frac{1}{4\pi\epsilon_0}\qty(\frac{q(\bvec{r}\cdot(\bvec{r}_2-\bvec{r}_1))}{\bvec{r}^2\abs{\bvec{r}}})\\
&=\frac{1}{4\pi\epsilon_0}\qty(\frac{\bvec{r}\cdot\bvec{p}}{r^3})
\end{aligned}
\end{equation}



\subsubsection{电场}

理想电偶极子的电场,或者物理电偶极子在远处的电场,可以通过求电势的梯度来得到.引用\autoref{VecAnl_eq15}~\upref{VecAnl},令$\bvec{a}=\bvec{p}$为常量,$\bvec{b}=\bvec{r}/r^3$,并考虑球坐标系下的nabla算子:$\nabla=\hat{r}\partial/\partial r+\hat{\theta}r\partial/\partial\theta+\hat{\varphi}r\sin\theta\partial/\partial\varphi$,则有:

\begin{equation}
\begin{aligned}
\bvec E_d(\bvec r) &= -\nabla V_d(\bvec r) =- \frac{1}{4\pi\epsilon_0} \nabla  \frac{\bvec p\vdot \bvec r}{r^3}\\
&=-\frac{1}{4\pi\epsilon_0}\qty((\bvec{p}\cdot\nabla)\frac{\bvec{r}}{r^3}+\bvec{p}\times(\nabla\times\frac{\bvec{r}}{r^3}))\\
&=-\frac{1}{4\pi\epsilon_0}(\bvec{p}\cdot\nabla)\frac{\bvec{r}}{r^3}\\
&=-\frac{1}{4\pi\epsilon_0}(p_x\frac{\partial}{\partial x}+p_y\frac{\partial}{\partial y}+p_z\frac{\partial}{\partial z})\pmat{\frac{x}{(x^2+y^2+z^2)^{3/2}}\\\frac{y}{(x^2+y^2+z^2)^{3/2}}\\\frac{z}{(x^2+y^2+z^2)^{3/2}}}\\
&=-\frac{1}{4\pi\epsilon_0}\pmat{p_x\frac{x^2+y^2+z^2-3x^2}{(x^2+y^2+z^3)^{5/2}}-p_y\frac{-3xy}{(x^2+y^2+z^2)^{5/2}}-p_z\frac{-3xz}{(x^2+y^2+z^2)^{5/2}}\\
    -p_x\frac{-3xy}{(x^2+y^2+z^2)^{5/2}}+p_y\frac{x^2+y^2+z^2-3y^2}{(x^2+y^2+z^3)^{5/2}}-p_z\frac{-3yz}{(x^2+y^2+z^2)^{5/2}}\\
    -p_y\frac{-3xz}{(x^2+y^2+z^2)^{5/2}}-p_y\frac{-3yz}{(x^2+y^2+z^2)^{5/2}}+p_z\frac{x^2+y^2+z^2-3z^2}{(x^2+y^2+z^3)^{5/2}})}
\end{aligned}
\end{equation}
