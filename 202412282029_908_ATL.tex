% 埃托雷·马约拉纳(综述)
% license CCBYSA3
% type Wiki

本文根据 CC-BY-SA 协议转载翻译自维基百科\href{https://en.wikipedia.org/wiki/Arthur_Compton}{相关文章}。

\begin{figure}[ht]
\centering
\includegraphics[width=6cm]{./figures/7dc439ef645ec5c1.png}
\caption{马约拉纳在1930年代} \label{fig_ATL_1}
\end{figure}
埃托雷·马约拉纳(Ettore Majorana,/maɪəˈrɑːnə/,[2] 意大利语:[ˈɛttore majoˈraːna];1906年8月5日出生——可能在1959年或之后去世)是意大利理论物理学家,曾研究中微子质量。1938年3月25日,他在购买了从那不勒斯到巴勒莫的船票后神秘失踪。

马约拉纳方程和马约拉纳费米子以他的名字命名。2006年,为了纪念他,设立了马约拉纳奖。
\subsection{生活与工作}
1938年,恩里科·费米曾这样评价马约拉纳:“世界上有几类科学家;第二或第三流的科学家尽最大努力,但永远不会走得太远。然后是第一流的科学家,他们做出了对科学进步至关重要的发现。但还有一些天才,比如伽利略和牛顿。马约拉纳就是其中之一。”
\subsection{数学天赋} 
马约拉纳出生在西西里岛的卡塔尼亚。他的叔叔奎里诺·马约拉纳也是一位物理学家。马约拉纳自幼展现出卓越的数学天赋,1923年开始在大学学习工程学,但在1928年,在埃米利奥·塞格雷的鼓励下转学物理。[4]: 69–72  他很年轻时就加入了恩里科·费米在罗马的研究小组,成为了著名的“潘尼斯佩尔纳街的男孩”之一,这个名字来源于他们实验室所在的街道地址。
\subsection{首次发表的学术论文}

马约拉纳的第一篇论文涉及原子光谱学中的问题。他的第一篇论文发表于1928年,当时他还是一名本科生,这篇论文是与罗马物理研究所的助理教授乔瓦尼·詹蒂尔(Giovanni Gentile Jr.)共同撰写的。这项工作是费米原子结构统计模型(现称为托马斯-费米模型)在原子光谱学中的早期定量应用。

在这篇论文中,马约拉纳和詹蒂尔在该模型的框架内进行了从头算的计算,成功地解释了实验观察到的镧和铀的核心电子能量,以及在光谱中观察到的铯线的精细结构分裂。1931年,马约拉纳发表了关于原子光谱中自电离现象的第一篇论文,他称其为“自发电离”;同年,普林斯顿大学的艾伦·申斯通(Allen Shenstone)也发表了一篇独立的论文,将这一现象称为“自电离”,这个术语最早由皮埃尔·奥热(Pierre Auger)提出。自那时以来,这一术语便成为了该现象的标准称呼。

马约拉纳于1929年在罗马大学(La Sapienza)获得物理学博士学位。1932年,他发表了一篇关于原子光谱学中对准原子在时变磁场中的行为的论文。这个问题也被I.I. 拉比等人研究过,后来发展成为原子物理学的一个重要分支——射频光谱学。同年,马约拉纳发表了关于具有任意内禀动量的粒子相对论理论的论文,他在其中发展并应用了洛伦兹群的无限维表示,并为基本粒子的质量谱提供了理论依据。和马约拉纳的大多数论文一样,这篇论文也是用意大利语写的,因此在几十年里一直鲜为人知。[5]

1932年,伊雷内·约里奥-居里(Irène Joliot-Curie)和弗雷德里克·约里奥(Frédéric Joliot)的实验表明存在一种未知粒子,他们认为这可能是伽马射线。马约拉纳是第一个正确解读这一实验的学者,认为它需要一种中性电荷且质量大约和质子相同的新粒子;这种粒子就是中子。费米建议他写一篇关于这一话题的文章,但马约拉纳没有这么做。詹姆斯·查德威克(James Chadwick)在同年通过实验证明了中子的存在,并因此获得了诺贝尔奖。[6]

马约拉纳以不追求个人功劳而著称,他认为自己的工作微不足道。他一生只写了九篇论文。