% 高斯波包
% keys 高斯分布|波包|光学|量子力学
% license Xiao
% type Tutor

\pentry{波包\nref{nod_WvPck}, 高斯分布\nref{nod_GausPD}}{nod_6aab}

\begin{figure}[ht]
\centering
\includegraphics[width=14.25cm]{./figures/98b376107396a32e.pdf}
\caption{高斯波包(\autoref{eq_GausPk_1} ),蓝色为实部,红色为虚部, $x_0 = 0$, $A = 1$, $a = 1/20$, $k_0 = 5$。} \label{fig_GausPk_1}
\end{figure}

\footnote{参考 Wikipedia \href{https://en.wikipedia.org/wiki/Wave_packet}{相关页面}。}\textbf{高斯波包(Gaussian wave packet)}是指轮廓为高斯分布的波包, 在光学和量子力学中有重要应用。 在光学中,它可以用作激光脉冲的的电场函数; 在量子力学中, 它常被作为波函数。

高斯波包用复函数表示为($A$ 为复数)
\begin{equation}\label{eq_GausPk_1}
f(x) = A \E^{-a x^2}\E^{\I k_0 x}~,
\end{equation}
由于机械波和电磁波的能量密度(光强)都正比于振幅平方,所以我们经常会讨论分布函数 $f(x)^2$ 的性质。

$f(x)^2$ 的方差为(对比\autoref{eq_GausPD_1}~\upref{GausPD})
\begin{equation}
\sigma^2= \frac{1}{4a}~.
\end{equation}

FWHMI (光强半高宽)为 $f(x)^2$
\begin{equation}
\mathrm{FWHMI} = \sqrt{\frac{2\ln 2}{a}} = 2\sqrt{2\ln 2}\ \sigma \approx 2.35482\sigma~.
\end{equation}
满足 $f^2(\text{FWHMI/2}) = f^2(0)/2$。

$f(x)$ 的积分为(用于求电场矢势)
\begin{equation}\label{eq_GausPk_3} % Mathematica 已验证
\int A \E^{-a x^2}\E^{\I k_0 x} \dd{x} = -\I\frac{A}{2} \sqrt{\frac{\pi}{a}} \exp(-\frac{k^2}{4a}) \erfi\qty(\frac{k + 2 \I a x}{2 \sqrt{a}}) + C~.
\end{equation}
$C$ 前面的部分在 $x = \pm \infty$ 处分别为 $\pm\frac{A}{2} \sqrt{\frac{\pi}{a}} \exp(-\frac{k^2}{4a})$。% 已数值验证

\subsubsection{频谱}
\pentry{傅里叶变换(指数)\nref{nod_FTExp}}{nod_c3d1}
% (未完成:哪里有介绍频谱的概念?)

要求\autoref{eq_GausPk_1} 的傅里叶变换 $g(k)$, 由\autoref{ex_FTExp_1}~\upref{FTExp}以及傅里叶变换性质\autoref{eq_FTExp_4}~\upref{FTExp}和\autoref{eq_FTExp_7}~\upref{FTExp}得
\begin{equation} % Mathematica 已验证
g(k) = A\sqrt{\frac{\pi}{a}} \exp[-\frac{(k-k_0)^2}{4a}]~.
\end{equation}
$g(k)^2$ 的半高全宽 FWHMI 为 $2\sqrt{2\ln 2}\sqrt{a} = 2.3548200 \sqrt{a}$。

\begin{figure}[ht]
\centering
\includegraphics[width=13cm]{./figures/925904dd88cd4821.pdf}
\caption{高斯波包和 cos2 波包的对比} \label{fig_GausPk_2}
\end{figure}