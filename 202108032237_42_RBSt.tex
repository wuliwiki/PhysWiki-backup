% 刚体的静力平衡
% 静力平衡|合外力|刚体|合力矩

\pentry{角动量定理\upref{AMLaw}}

\footnote{参考 Wikipedia \href{https://en.wikipedia.org/wiki/Mechanical_equilibrium}{相关页面}和 \cite{新力}.}在惯性系中, 如果刚体所受的所有合外力与合外力矩\upref{Torque}都为零, 则我们说它处于\textbf{静力平衡(static equilibrium)}. 其中合外力(矩)是指所有施加在刚体上的力(矩)的矢量和.

\begin{theorem}{刚体的静力平衡}
若一个刚体处于静力平衡, 那么它将保持静止或者做以下两种运动的组合: 1. 质心做匀速运动, 2. 绕质心做定轴匀速转动.
\end{theorem}

注意在非惯性系中, 若加入惯性力的修正, 该结论仍然成立. 另外注意当合外力为零时, 合外力矩与参考点(参考系)的选取无关(\autoref{Torque_eq5}~\upref{Torque}).

\textbf{证明}:把刚体看做由许多质点组成,合外力为零时刚体动量守恒\upref{PLaw}, 而动量等于 “质心的动量” \upref{SysMom}
$\bvec p_c = M_c \bvec v_c$,所以质心做匀速运动或不动.

刚体合外力矩为零时,其角动量守恒\upref{AMLaw},而刚体的角动量等于质心的角动量 $\bvec L_c =\bvec r_c\cross \bvec p_c$ 加上质心系中的角动量(\autoref{AngMom_eq3}~\upref{AngMom}). 当质心匀速直线运动或不动时 $\bvec L_c$ 不变,所以质心系中刚体的角动量也不变,所以刚体绕质心做匀速转动或不转动. 证毕.

\begin{example}{轻杆三力平衡}
如图, 一个长度为 $L$ 质量不计的细杆, 中间和两端受力分别为 $\bvec F_1, \bvec F_2, \bvec F_3$.
\addTODO{图, 选取不同受力点}
\end{example}

\addTODO{吊桥例题, 见 EP1 20201021, 缆绳受力与重物位置的关系.}

\begin{example}{}\label{RBSt_ex1}
如\autoref{RBSt_fig1}, 一个质量为 $m$ 的线轴被斜挂在墙上, 线轴与墙面的摩擦系数为 $\mu$,线轴的大圆半径为 $R$, 小圆半径为 $r$, 求当角 $\alpha$ 满足什么条件时, 线轴才能不滑落.
\begin{figure}[ht]
\centering
\includegraphics[width=5cm]{./figures/RBSt_1.pdf}
\caption{线轴的平衡} \label{RBSt_fig1}
\end{figure}

我们先来看线轴受哪几个力:重力 $mg$, 绳的拉力 $T$, 墙的支持力 $N$ 和摩擦力 $f$. 由摩擦系数的定义和刚体平衡条件可得
\begin{equation}
\begin{cases}
f \leqslant \mu N & \text{(摩擦系数)}\\
N - T\sin\alpha = 0 & \text{(水平方向受力平衡)}\\
T\cos\alpha + f - mg = 0 & \text{(竖直方向受力平衡)}\\
Tr - fR = 0 & \text{(力矩平衡)}
\end{cases}
\end{equation}
其中最后一条力矩平衡是以圆心为原点计算力矩, 虽然原则上我们可以取任意点计算力矩, 但取在圆心计算最为简单. 除了 $\alpha$ 我们有三个未知数 $T, f, N$, 用以上三条等式恰好可以把这三个未知数消去, 可得关于 $\alpha$ 的不等式
\begin{equation}
\sin\alpha \geqslant \frac{r}{\mu R}
\end{equation}

一个有趣的地方在于, 不等式中没有出现质量 $m$. 事实上, 我们不使用那条含有 $mg$ 的等式也可以顺利得到答案.
\end{example}
\begin{example}{二人抬物}
\begin{figure}[ht]
\centering
\includegraphics[width=7cm]{./figures/RBSt_2.pdf}
\caption{二人高低抬物}} \label{RBSt_fig2}
\end{figure}
如\autoref{RBSt_fig2} ,有二人一高一低抬起一物体缓慢移动,试比较二人出力的大小.

\textbf{解:}由于二人抬物缓慢移动,故物体可看成每一时刻处于平衡状态.物体受三个力:二人的支持力和物体自身的重力,三力平衡这三力必处于一平面上,并且二人对物体的力的作用点必和物体质心在一条直线上(否则两支持力作用点连线上必存在一点,使得两支持力对该点合力矩为零,此时物体重力对该点将有一力矩,而这是不可能的).

如\autoref{RBSt_fig3} 所示,以 $\bvec{F_1}$ 代表高处的力,$\bvec{F_2}$ 代表低处的力, $\bvec{G}$为物体重力.以物体质心为原点,水平向右为 $x$ 轴正方向,竖直向上为 $y$ 轴正方向.设 $\bvec{F_1}$  与 $x$ 轴正方向夹角为 $\theta_1$, $\bvec{F_2}$ 与 $x$ 轴正方向夹角 $\theta_2$,$\bvec{F_1}$作用点距离质心距离为 $l_1$,$\bvec{F_2}$距离质心位置为 $l_2$,作用点连线与水平方向夹角为 $\theta$.
\begin{figure}[ht]
\centering
\includegraphics[width=8cm]{./figures/RBSt_3.pdf}
\caption{二人高低抬物受力示意图} \label{RBSt_fig3}
\end{figure}
则
\begin{equation}
\begin{aligned}
&\bvec{F_1}=(F_1\cos\theta_1,F_1\sin\theta_1)\\
&\bvec{F_2}=(F_2\cos\theta_2,F_2\sin\theta_2)\\
&\bvec{G}=(0,-G)\\
&\bvec{r_1}=(-l_1\cos\theta,l_1\sin\theta)\\
&\bvec{r_2}=(l_2\cos\theta,-l_2\sin\theta)
\end{aligned}
\end{equation}
其中 $\bvec{r_1},\bvec{r_2}$ 为高,低作用点位矢.

由物体平衡条件有
\begin{equation}
\begin{aligned}
&\sum_i\bvec{F_i}=\bvec0:(F_1\cos\theta_1+F_2\cos\theta_2,F_1\sin\theta_1+F_2\sin\theta_2-G)=(0,0)\\
&\sum_i\bvec{M_i}=\bvec0:\bvec{r_1}\times \bvec F_1+\bvec{r_2}\times \bvec F_2=(-l_1F_1\cos\theta\sin\theta_1-l_1F_1\sin\theta\cos\theta_1+l_2F_2\cos\theta\theta_2+l_2F_2\sin\theta\cos\theta_2)
\end{aligned}
\end{equation}

\end{example}