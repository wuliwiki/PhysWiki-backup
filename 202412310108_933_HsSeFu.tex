% 数列(高中)
% keys 高中|数列的概念|数列的函数特性
% license Usr
% type Tutor

\begin{issues}
\issueDraft
\end{issues}
\pentry{函数\nref{nod_functi}}{nod_7191}

在之前的学习中,已经介绍过函数的概念。函数是一种描述变量之间关系的数学工具,高中阶段接触到的函数,其定义域通常是连续的,比如实数集 $\mathbb{R}$ 或区间,因此函数的图像往往是一条连续的曲线。然而,数学世界中并非所有的关系都具有连续性。一些仅定义在自然数集或其子集上的特殊函数,被称为\textbf{数列}。

数列是非常古老的数学内容,在某些方面,古代数学家们已经做了很深入的研究。最初,数列是人们将观测到的对象按顺序排列而形成的一种表示方式,那时甚至还没有函数的概念。尽管数列作为独立的数学领域有着丰富的研究内容和独特的性质,但从函数的视角观察数列,会发现它实际上继承了函数的许多特点。

\subsection{数列}

在小学阶段,常见的题目之一是类似这样的填空题:$1,3,7,(),31$。题目的目的是引导学生通过观察这些数字的排列规律,推断出空缺处的数字\footnote{然而,尽管这类题目通常有“标准答案”,但实际上,填写任何数字,都能够给出合理的规律。}。显然,在这里出题者希望隐含的规律是 $2^n - 1$,其中 $n$ 表示第几个出现的数。

另一个简单的例子是将一个月中的每一天按照星期的数字标记。假设一个月的第一天是星期日,那么接下来的数字标记会形成一个循环的数列:$7, 1, 2, 3, 4, 5, 6, 7, 1, 2, 3, \dots$。它的规律可以表示为 $(n-1) \mod 7$\footnote{$\mod$为取余函数,即$a \mod b$ 为 $a$ 除以 $b$ 得到的余数。},其中结果为 $0$ 时代表星期日,非零时则为对应的星期数。

最初研究数列的人,或许只是简单地将一些有关系的数字排列成一列,关注数字之间的直接关系,例如相邻数字的差值、比值或其他变化模式。然而,随着数学的发展,研究数列的视角逐渐发生了转变,人们引入函数的视角,将数字的位置视为自变量。例如,上述表达式若被看作函数,其中的 $n$ 就是表示位置的自变量。这种视角的改变突出了位置的重要性。通过这种方式,数列的研究不再局限于数字之间的关系,而是扩展到数字与其位置的对应关系,逐渐将重点转向提炼和揭示数字排列的内在规律,并为数列的表达和分析提供了更加系统化的工具。

\begin{definition}{数列}
将一些数按照一定的次序排列成一列,称为\textbf{数列(sequence)}或\textbf{序列},通常记作:
\begin{equation}
a_1, a_2, a_3, \cdots, a_n, \cdots~.
\end{equation}
简记为数列 $\{a_n\}$,字母$a$可以替换为其他字母。

其中,$n$ 是自然数\footnote{在数学中,自然数通常从 $0$ 开始,但在高中阶段,一般要求数列从 $1$ 开始。当然也有其他教材会从 $0$ 开始。},数列中的每一个数被称为该数列的\textbf{项(term)}。
\end{definition}

关于项,还有一些概念:
\begin{itemize}
\item 第一项 $a_1$ 称为\textbf{首项(first term)}。 
\item $a_n$ 表示数列的第 $n$ 项,也被称为\textbf{通项(general term)}。
\item 数列中包含项的个数称为\textbf{项数}。
\end{itemize}

根据数列中项数是否为有限值,可以将数列分为两类:若有限,例如 $1, 2, 3, 4$,则称为\textbf{有穷数列(finite sequence)};若无限,例如 $1, 2, 3, 4, \dots$,则称为\textbf{无穷数列(infinite sequence)}。在高中阶段,只研究有穷数列。对于有穷数列,其最后一项通常被称为\textbf{末项(last term)}。

为了更直观地理解数列,可以用一个排队的场景进行类比:设想一群人在排队,每个人依次站在某个位置上。可以用自然数 $n$ 给每个人编号(第一个人编号为 $1$,第二个人编号为 $2$,依此类推),并将每个人的身高记录下来。这样,排队中每个人的身高就形成了一个数列 ${a_n}$,其中 $a_n$ 表示排队中第 $n$ 个人的身高。例如:
\begin{equation}
a_1 = 170, \quad a_2 = 165, \quad a_3 = 180, \quad \dots~.
\end{equation}

在这个例子中,数列 ${a_n}$ 表示排队中每个人的身高,而 $n$ 表示他们在队伍中的位置。为了便于描述这群人排队的整体情况,经常会提到就会说“从某人到某人”,这对应的就是首项与末项的概念。

如果数列 ${a_n}$ 的第 $n$ 项 $a_n$ 与 $n$ 之间的关系可以通过一个表达式表示为 $a_n = f(n)$,就像前面提到的 $a_n = 2^n - 1$ 或 $a_n = (n-1) \mod 7$那样,那么这个表达式被称为该数列的\textbf{通项公式(general term formula)}。

从函数的角度来看,数列的通项公式可以视为相应函数的解析式,其中位置 $n$ 是自变量,项 $a_n$ 是该函数在 $n$ 处的函数值。也就是说,通项公式通过函数的形式明确了数列中每一项与其位置之间的关系,使得数列可以被视为一个定义在自然数集合上的函数。

\subsection{递推公式}

就像并非所有函数都能写出明确的解析式一样,也不是所有数列都能写出通项公式。然而,由于数列的特殊性,它还可以通过另一种表示方法来定义,这种方法称为递推公式。这与之前函数的部分稍有不同,可能会感到陌生。

\begin{definition}{递推公式}
用数列中已知的若干项,通过一定的数学关系推导出后续项的公式称为\textbf{递推公式(recurrence relation)}。递推公式定义了数列的生成规律,其形式通常包含当前项与之前若干项之间的关系。递推公式的通用形式为:

\begin{equation}
a_n = f(a_{n-1}, a_{n-2}, \dots, a_{n-k}),\qquad(n>k)~.
\end{equation}

其中:
\begin{itemize}
\item $a_n$ 表示数列的第 $n$ 项;
\item $a_{n-1}, a_{n-2}, \dots, a_{n-k}$ 表示数列中第$n$项前的$k$ 项;
\item $f$ 是一个表达式,用于描述项之间的关系;
\item 数列的初始若干项的具体值称作\textbf{起始条件(initial conditions)}。
\end{itemize}
\end{definition}

下面以著名的\textbf{斐波那契数列(Fibonacci sequence)}为例,介绍递推公式是如何作用的。斐波那契数列$\{a_n\}$的定义如下:

\begin{equation}
a_n = a_{n-1} + a_{n-2},\qquad(n>2)~.
\end{equation}
其中$a_1 =  a_2 =1$。

\begin{example}{求斐波那契数列的第5项。}
首先,明确数列的初始值。在斐波那契数列中,已知 $a_1 = 1$ 和 $a_2 = 1$。然后利用递推关系计算后续项,根据公式有第 $3$ 项$a_3 =a_1 +a_2= 1 + 1 = 2$,第 $4$ 项$a_4 = a_3 + a_2 = 2 + 1 = 3$和第 $5$ 项$a_5 = a_4 + a_3 = 3 + 2 = 5$。
\end{example}

事实上,按照递推公式不断推算,可以得到完整的斐波那契数列:$\{a_n\} = 1, 1, 2, 3, 5, 8, 13, \dots$。当然,使用递推公式不代表没有通项公式,经过数学家的努力,得到了斐波那契数列的通项公式:
\begin{equation}
a_n = \frac{\phi^n - \psi^n}{\sqrt{5}}~.
\end{equation}
其中:
\begin{itemize}
\item $\displaystyle\phi = \frac{1+\sqrt{5}}{2}$被称为\textbf{黄金分割比(golden ratio)},约为 $1.618$;
\item $\displaystyle\psi = \frac{1-\sqrt{5}}{2}$,约为 $-0.618$。
\end{itemize}

而有些问题则不那么幸运,尽管可以通过递推公式定义,却无法写出明确的通项公式,例如约瑟夫问题(Josephus problem)。约瑟夫问题描述了标号为从$0$到$N-1$的$N$个人围成一圈,他们从1开始报数,每数到k就淘汰一个人,然后下一个人再从1开始重新报数,直到剩下最后一人。
以$N=10,k=3$为例,初始队列为十个人,最后剩下一个人,因此要淘汰9个人,淘汰过程如\autoref{tab_HsSeFu_1} 所示。
\begin{table}\label{tab_HsSeFu_1}[ht]
\centering
\caption{$N=10,k=3$为例的淘汰过程}\label{tab_HsSeFu1}
\begin{tabular}{|c|c|c|c|c|c|c|c|c|c|c|}
\hline
淘汰人次 & 编号0 & 编号1& 编号2 & 编号3 & 编号4& 编号5 & 编号6& 编号7 & 编号8& 编号9\\
\hline
0 & 存活 & 存活 &存活 &存活 &存活 &存活 &存活 &存活 &存活 &存活\\
\hline
1 & 存活 & 存活 &淘汰 &存活 &存活 &存活 &存活 &存活 &存活 &存活\\
\hline
2 & 存活 & 存活 &淘汰 &存活 &存活 &淘汰 &存活 &存活 &存活 &存活\\
\hline
3 & 存活 & 存活 &淘汰 &存活 &存活 &淘汰 &存活 &存活 &淘汰 &存活\\
\hline
4 & 存活 & 淘汰 &淘汰 &存活 &存活 &淘汰 &存活 &存活 &淘汰 &存活\\
\hline
4 & 存活 & 淘汰 &淘汰 &存活 &存活 &淘汰 &淘汰 &存活 &淘汰 &存活\\
\hline
* & * & * & * & * & * & * & * & * & * & * \\
\hline
* & * & * & * & * & * & * & * & * & * & * \\
\hline
* & * & * & * & * & * & * & * & * & * & * \\
\hline
\end{tabular}
\end{table}
a_n表示第n个人被淘汰时,其递推公式为:

\begin{equation}
a_n = (a_{n-1} + k) \mod n, \qquad \text{其中 } a_1 = 0~.
\end{equation}

这个递推公式定义了每轮中剩余人的位置,而无法用简单的通项公式直接表示。

通过上面两个例子可以看出,相比于通项公式,递推公式特别适合定义具有明确生成规律但不易直接表达的数列。


例如:
	•	

递推公式的优势在于,即使数列没有明确的通项公式,仍然可以通过已知项递推出所有后续项,从而完整地描述数列的特性和规律。

\subsection{数列和}

无穷数列的数列和也称为级数,事实上


数列的基本性质和特点
如何根据数列规律求通项公式?
如何计算数列的前 $n$ 项和?
数列的规律性
每个数列都有独特的规律,例如:
等差数列的公差:相邻两项之差为 $d$。
等比数列的公比:相邻两项之比为 $r$。

常见问题


数列是按照某种规律排列的一列数字,具有明显的规律性。通项公式和递推公式是数列问题的重要工具。
\subsection{函数特性}

\subsection{增减性}

对于数列 $\{a_n\}$,从第 $2$ 项起,若满足:
\begin{enumerate}
\item 每一项都大于前一项,即$a_{n+1} > a_n$,则称为\textbf{递增数列(increasing sequence)}。
\item 每一项都小于前一项,即$a_{n+1} < a_n$,则称为\textbf{递减数列(decreasing sequence)}。
\item 有些项大于它的前一项,有些项小于它的前一项,则称为\textbf{摆动数列(oscillating sequence)}。
\end{enumerate}
就像前面提到的排队一样,这里分别对应的就是从小到大排列、从大到小排列和随意站立的情况。另外,如果数列的各项都相等,则称为\textbf{常数列(constant sequence)},即对应所有人高度相同,队伍呈现“整齐划一”的状态。



循环性


这个数列具有周期性,每隔 7 天就会重复一次,是一种典型的循环数列(cyclic sequence)。

在高中阶段,数列的研究相对狭隘,主要集中在古代早期就被发现的等差数列和等比数列。然而,建议读者不要因此局限视野,数列的概念在更高层次的数学学习中,将与许多重要概念密切相关,例如实数的构建、级数以及分析学中的广泛应用。因此,在学习数列时,需要多关注其规律和研究方法,而不仅仅停留于记忆一些常见公式。理解这些公式背后的原理尤为重要,这不仅有助于掌握数列的变化规律,也能培养更深层次的数学思维,扎实当前的知识体系,为未来更复杂的数学学习奠定坚实的基础。