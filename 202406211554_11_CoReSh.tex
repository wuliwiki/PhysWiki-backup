% 宇宙学红移
% keys FRW 度规|红移|哈勃常数
% license Usr
% type Tutor

\begin{issues}
\issueMissDepend
\end{issues}

\pentry{Friedmann-Robertson-Walker (FRW) 度规\nref{nod_FRW}}{nod_6c42}
由于物体在宇宙中传播的过程中,宇宙也在加速膨胀,所以为了准确的测量物体在宇宙传播过程中的物理量,我们需要引进宇宙学红移的概念。

\subsection{光子的红移}
假设位于共动坐标$r_1$处的光源在$t_1$时刻发出光子,原点处的我们在$t_0$时刻收到,不失一般性,假设$\dd\theta=\dd\phi=0$,那么根据光子走过的世界线为$0$这一事实,并结合FRW度规,我们有
\begin{equation}\label{eq_CoReSh_4}
s=\int^{t_0}_{t_1}\frac{c}{a(t')}\dd t'=\int_0^{r_1}\frac{1}{\sqrt{1-k(r/R)^2}}\dd r~.
\end{equation}

间隔一小段时间后,即在$t_1+\dd t_1$时,光源处再次发射出光子,而我们在$t_0+\dd t_0$时刻检测到,则
\begin{equation}\label{eq_CoReSh_3}
\int^{t_0+\dd t_0}_{t_1+\dd t_1}\frac{c}{a(t')}\dd t'=\int_0^{r_1}\frac{1}{\sqrt{1-k(r/R)^2}}\dd r~.
\end{equation}

由于间隔时间极短,用\autoref{eq_CoReSh_3} 减去\autoref{eq_CoReSh_4} ,可以解得:

\begin{equation}
\begin{aligned}
a(t_0)\dd t_0&=a(t_1)\dd t_1\\
\frac{a(t_0)}{a(t_1)}&=\frac{\dd t_1}{\dd t_0}~.
\end{aligned}
\end{equation}

假设我们检测的是连续传播的电磁波,$\dd t_0,\dd t_1$对应的是两个连续波峰的时间间隔,那么因为波长与该时间间隔成正比可知:
\begin{equation}\label{eq_CoReSh_1}
\lambda_0=\frac{a(t_0)}{a(t_1)}\lambda_1~,
\end{equation}
因为宇宙在加速膨胀 $a(t_0)>a(t_1)$, 所以容易得 $\lambda_0>\lambda_1$。也就是说,宇宙的加速膨胀使得我们测得的电磁波波长更长,这就是所说的红移效应。因此,观测到红移意味着星系都在远离我们。


\subsection{红移因子}
\begin{definition}{}
为衡量红移效应的大小,我们定义\textbf{红移因子(redshift parametre)}为
\begin{equation}
z=\frac{\lambda_0-\lambda_1}{\lambda_1}~,
\end{equation}
\end{definition}
显然从\autoref{eq_CoReSh_1} 我们可以推出
\begin{equation}
\begin{aligned}
1+z&=\frac{a(t_0)}{a(t)}=\frac{1}{a(t_1)}\\
a(t_1)&=\frac{1}{1+z}
\end{aligned}~,
\end{equation}
一般地我们设现在的宇宙尺度因子 $a(t_0)=1$。在有了红移因子的定义后,我们就多了一种描述宇宙内物体位置的方式,

\subsection{哈勃常数\footnote{参考Tong David的宇宙学讲义。}}

假设某星系在共动坐标系的轨迹为$r(t)$,那么其在物理坐标系下的轨迹为
\begin{equation}
r_{phy}(t)=a(t)r(t)~,
\end{equation}
且其物理速度为
\begin{equation}
\begin{aligned}
v_{phy}& =\frac{\dot a}{a}ar(t)+a\dv{r}{t}\\
&=H(t)r_{phy}+av_{pec}~,
\end{aligned}
\end{equation}
其中特殊速度$v_{pec}$为此星系在共动坐标系里的速度,由近邻星系的引力决定。比如我们银河系的特殊速度为$400\,km/s$,哈勃常数$H(T)\equiv \frac{\dot a}{a}$。在宇宙学里,常用下标0表示现在的值,比如$H_0$表示现今的哈勃常数。在今,哈勃常数的数值为
\begin{equation}H_0\approx70\text{ km s}^{-1}\operatorname{Mpc}^{-1}~.
\end{equation}
还有一种写法为
\begin{equation}H_0=100h \mathrm{km} \mathrm{s}^{-1} \mathrm{Mpc}^{-1}~,\end{equation}
其中 $h$ 测得的数值为\footnote{12Planck 2013 Results – Cosmological Parameters [arXiv:1303.5076]}
\begin{equation}
h\sim 0.67 \pm 0.01~.
\end{equation}显然,代入$h\approx 0.7$便是今天的哈勃常数。

我们把 $t_1$ 时刻的宇宙尺度因子 $a(t_1)$ 以现在的时刻 $t_0$ 为原点作泰勒展开,可得
\begin{equation}
a(t_1)=a(t_0)(1+(t-t_0)H_0+\cdots)~.
\end{equation}

若简单直白地令宇宙大爆炸时的尺度因子为零($a(t_{BB})=0$),那么便可知宇宙年龄的数量级必为$H_0^{-1}$,我们有
\begin{equation}t_0-t_{BB}=H_0^{-1}\approx4.4\times10^{17}\text{ s}\approx1.4\times10^{10}\text{ years}~,\end{equation}

与实际年龄138亿岁已是十分接近。
\subsection{光度距离}
物体的内秉亮度$L$是其在单位时间内发射出的能量,那么对于观测的星体,我们可以定义视亮度$l$为观测者在某一位置,\textbf{每单位时间和单位面积}接收到的能量。设观测者与星体的物理距离为$a(t_0)r_0$,在$\dd t_0$内接收到的能量为$E_0$,那么视亮度为
\begin{equation}
l=\frac{E_0}{4\pi a^2(t_0)r_0^2\dd t_0}~.
\end{equation}
若星体在共动参考系下,单位时间$\dd t_1$内发出能量为$E_1$,红移效应会使得
\begin{equation}
E_0=\hbar\nu_0=\hbar \frac{\nu_1}{1+z}\frac{E_1}{1+z};
\end{equation}
考虑到,