% Julia 变量的命名
% keys 变量 命名

本文授权转载自郝林的 《Julia 编程基础》. 原文链接:\href{https://github.com/hyper0x/JuliaBasics/blob/master/book/ch03.md}{第3章:变量与常量}.


\subsubsection{3.2 变量的命名}

\textbf{3.2.1 一般规则}

Julia 变量的名称是大小写敏感的.也就是说,\verb|y|和\verb|Y|并不是同一个标识符,它们可以代表不同的值.

变量名必须以大写的英文字母\verb|A-Z|、小写的英文字母\verb|a-z|、下划线\verb|_|,或者代码点大于\verb|00A0|的 Unicode 字符开头.代表数字的字符不能作为变量名的首字符,但是可以被包含在名称的后续部分中.当然,变量名中肯定不能夹杂空格符,以及像制表符、换行符这样的不可打印字符.

总之,大部分 Unicode 字符都可以作为变量名的一部分.即使你不知道什么是 Unicode 编码标准也没有关系.我们会在后面讨论字符和字符串的时候介绍它.

由此,Julia 允许我们把数学符号当做变量名,例如:

\begin{lstlisting}[language=julia]
julia> δ = 3
3

julia> 
\end{lstlisting}

你可能会问:怎么才能输入\verb|δ|?这又涉及到了 LaTeX 符号.简单来说,LaTeX 是一个排版系统,常被用来排版学术论文.因为这种论文中经常会出现复杂表格和数学公式,所以 LaTeX 自有一套方法去表现它们.我们没必要现在就去专门学习 LaTeX.你只要知道,如果需要输入数学符号的话,那么就可以利用 LaTeX 符号.

具体的做法是,先输入某个 LaTeX 符号(比如\verb|\delta|)再敲击 Tab 键,随后对应的数学符号(比如\verb|δ|)就可以显示出来了.如果你不知道一个数学符号对应的 LaTeX 符号是什么,那么可以在 REPL 环境的 help 模式下把那个数学符号复制、黏贴到提示符的后面,然后回车.比如这样:

\begin{lstlisting}[language=julia]
help?> δ
"δ" can be typed by \delta<tab>

search: δ

  No documentation found.

  δ is of type Int64.

  Summary
  ≡≡≡≡≡≡≡≡≡

  primitive type Int64 <: Signed

  Supertype Hierarchy
  ≡≡≡≡≡≡≡≡≡≡≡≡≡≡≡≡≡≡≡≡≡

  Int64 <: Signed <: Integer <: Real <: Number <: Any

julia> 
\end{lstlisting}

\textbf{3.2.2 变量名与关键字}

虽然变量命名的自由度很大,但还是有一些约束的.其中最重要的是,你不能使用 Julia 已有的单一的关键字作为变量名.更广泛地说,所有程序定义的名称都不能与任何一个单一的关键字等同.Julia 中单一的关键字目前一共有 29 个.我把它们分成了 7 个类别:

\begin{itemize}
\item 表示值的关键字:\verb|false|、\verb|true|
\item 定义程序定义的关键字:\verb|const|、\verb|global|、\verb|local|、\verb|function|、\verb|macro|、\verb|struct|
\item 定义(无名)代码块的关键字:\verb|begin|、\verb|do|、\verb|end|、\verb|let|、\verb|quote|
\item 定义模块的关键字:\verb|baremodule|、\verb|module|
\item 引入或导出的关键字:\verb|import|、\verb|using|、\verb|export|
\item 控制流程的关键字:\verb|break|、\verb|continue|、\verb|else|、\verb|elseif|、\verb|for|、\verb|if|、\verb|return|、\verb|while|
\item 处理错误的关键字:\verb|catch|、\verb|finally|、\verb|try|
\end{itemize}

其中,程序定义指的是变量、常量、类型、有名函数、宏或者结构体.所有的程序定义都是有名称的,或者说可以由某个标识符指代.其中的有名函数和宏也可以被称为有名的代码块.

所谓的无名代码块,与有名的代码块一样也是有明显边界的一段代码,但是并没有一个标识符可以指代它们.注意,我把关键字\verb|end|划分为了定义(无名)代码块的关键字.但实际上,我们在定义有名函数、宏、结构体和模块的时候也需要用到它.

另外,模块也是一个有名的代码块.并且,一个 Julia 程序的主模块(即入口程序所属的那个模块)就是它的最外层代码块.在 Julia 中,并没有一个可以囊括所有代码的全局代码块.这与很多主流的编程语言都是不同的.我们可以说,Julia 程序就是由一些具有依赖关系的模块组成的.换句话讲,Julia 程序中的代码要么在主模块中,要么在主模块依赖的那些模块中.不用着急,我会在后面专门用一章来讲解模块.

关于以上这些关键字的用法,我们在后面也都会讲到.所以你现在只要知道它们不能被作为变量名就可以了.

\textbf{3.2.3 变量名与作用域}

我们在前面说过,\verb|Int64|是一个代表了类型的标识符.又因为这个标识符的作用域相当于是全局的,所以我们设定的变量名就不能与它重复.更宽泛地讲,我们的变量名肯定不能与任何一个 Julia 预定义的类型的名称相同.

什么叫作用域呢?其含义是,标识符可以被其他代码直接引用的一个区域.一旦超出这个区域,这个标识符在默认情况下就是不可见的.比如,我们在第 1 章定义过一个叫做\verb|MyArgs|的模块,并且其中有一个名叫\verb|get_parameter|的函数.当时的情况是,我们无法在这个模块之外直接使用这个函数的本名来引用它.如果你翻回去看的话,就能见到我们的引用方式是\verb|MyArgs.get_parameter|.这被称为(针对这个\verb|get_parameter|函数的)限定名.

严格来讲,Julia 中没有任何一个标识符的作用域真正是全局的.但是,由于我们定义的所有模块都隐含了\verb|Core|模块,所以在该模块中直接定义的那些标识符的作用域就相当于是全局的.\verb|Int64|以及其他代表了某个类型的标识符都在其中.因此,我们设定的变量名肯定不能与\verb|Core|模块中那些基本的程序定义重名.

关于作用域,还有一些内容是我们必须要知道的.Julia 中的很多代码块都会自成一个作用域,比如模块就是这样.但由于这会涉及到一些我们还未曾讲到的重要知识,所以我把它们放到了流程控制的那一部分.那里会有关于变量作用域的更多内容.