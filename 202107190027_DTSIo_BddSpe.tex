% 有界算子的谱
\pentry{巴拿赫空间\upref{banach} 巴拿赫定理\upref{BanThm}} 
\textbf{线性算子的谱 (spectrum)} 推广了矩阵的本征值\upref{MatEig}这一概念. 它对于了解线性算子如何作用于线性空间有着重要意义. 在这一章中, 我们主要讨论复巴拿赫空间上有界线性算子的谱. 无界算子的谱将留待后续章节讨论.

\begin{definition}{有界算子的谱}
设$X$是复巴拿赫空间, $T:X\to X$是有界线性算子. 复数$\lambda\in\mathbb{C}$称为算子$T$的一个\textbf{谱点 (spectral point)}, 如果$T-\lambda$不是可逆映射. $T$的全体谱点的集合记为$\sigma(T)$, 称为\textbf{谱集 (spectrum)}, 而补集$\mathbb{C}\setminus\sigma(T)$称作\textbf{预解集 (resolvent set)}, 有时记为$\rho(T)$.
\end{definition}

既然$T-\lambda$是有界算子, 根据开映像原理, 如果逆映射$(T-\lambda)^{-1}$存在, 那么它也必然是连续的. 据此便可以证明如下基本命题:

\begin{theorem}{}
\begin{enumerate}
\item 如果$T:X\to X$是有界线性算子, 那么它至少有一个谱点.
\item 如果$T:X\to X$是有界线性算子, 那么预解集$\rho(T)$是开集, 而谱集$\sigma(T)$是紧集.
\end{enumerate}
\end{theorem}
\textbf{证明大意.} 
对于 1., 如果$f(z)=(z-T)^{-1}$对于所有复数$z$都存在, 那么它将是全复平面上的算子值全纯函数, 而且由于$|z|>2\|T\|$时有
$$
f(z)=\frac{1}{z}\sum_{n=0}\frac{T^n}{z^n},
$$
而右边级数的模显然小于2, 所以$f(z)$是有界的全纯函数. 按照刘维尔定理, 它只能是常值函数, 这当然不可能.

对于 2., 如果$\lambda_0\in\rho(T)$, 那么$A:=\lambda_0-T$是有界的可逆算子. 于是可作几何级数
$$
A^{-1}(\text{Id}+zA^{-1}+z^2A^{-2}+...+z^nA^{-n}+...);
$$
如果$|z|<\|A^{-1}\|^{-1}$, 那么这个级数收敛到$A^{-1}(\text{Id}-zA^{-1})^{-1}=(\lambda_0-z-T)^{-1}$. 这说明预解集的点的某个邻域还包含在预解集内, 从而预解集是开集, 而谱集是闭集. 另一方面, 从1. 中可见, 只要$|z|>\|T\|$, 几何级数
$$
\frac{1}{z}\sum_{n=0}\frac{T^n}{z^n}
$$
就收敛到$(z-T)^{-1}$, 所以谱集包含在圆$|z|\leq\|T\|$内. 所以谱集是紧集.

上面反复出现的几何级数
$$
\text{Id}+T+T^2+...
$$
称为\textbf{冯诺依曼级数 (von Neumann series)}. 在$\|T\|<1$时, 它绝对收敛到$(\text{Id}-T)^{-1}$, 这与通常的数值几何级数很类似.