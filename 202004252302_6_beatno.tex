% 拍频

\pentry{振动的指数形式\upref{VbExp},简谐振子\upref{SHO}}

\subsection{同一直线上两个不同频率的谐振动的合成}

设一质点在一直线上同时参与两个不同频率的谐振动,其振动表达式为
\begin{equation}
\begin{array}{l}x_{1}=A_{1} \cos \left(\omega_{1} t+\phi_{01}\right) \\ x_{2}=A_{2} \cos \left(\omega_{2} t+\phi_{02}\right)\end{array}
\end{equation}
根据叠加原理,合运动的位移为
\begin{equation}
x=x_{1}+x_{2}=A_{1} \cos \left(\omega_{1} t+\phi_{01}\right)+A_{2} \cos \left(\omega_{2} t+\phi_{02}\right)
\end{equation}

为方便计算,设$A_1=A_2=A,\phi_{01}=\phi_{02}=\phi_{0}$,则上式可化成
\begin{equation}
x=2 A \cos \left(\frac{\omega_{2}-\omega_{1}}{2} t\right) \cos \left(\frac{\omega_{2}+\omega_{1}}{2} t+\phi_{0}\right)
\end{equation}
对于通常实际遇到的情况而言,两个频率比较接近,且$\left|\omega_{2}-\omega_{1}\right|\ll \omega_1$,式中第一项因子随时间作缓慢地变化,第二项因子是角频率近于$\omega$,或$\omega_1,\omega_2$的简谐函数,因此合成运动可近似看成是角频率为$\dfrac{\omega_{1}+\omega_{2}}{2} \approx \omega_{1} \approx \omega_{2}$,振幅为$\left | 2 A \cos \dfrac{\omega_{2}-\omega_{1}}{2} t\right |$