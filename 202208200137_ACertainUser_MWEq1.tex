% 麦克斯韦方程组(介质)
% 麦克斯韦方程组|介质磁导率

% 麦克斯韦方程组(介质)
% 麦克斯韦方程组|介质磁导率
\footnote{本文参考了\cite{GriffE}}
\begin{issues}
\issueDraft
\end{issues}

\pentry{麦克斯韦方程组\upref{MWEq}}
介质中的麦克斯韦方程:
\begin{equation}\ali{
&\div\bvec D = \rho_f\\
&\curl\bvec E = -\pdv{\bvec B}{t}\\
&\div\bvec B = 0\\
&\curl\bvec H = J_f + \pdv{\bvec D}{t}
}\end{equation}

其中$\bvec D$为“电位移矢量”,$\bvec H$为“磁场强度”\footnote{有一些作者认为$\bvec D, \bvec H$仅仅是数学工具而没有实际的物理含义,因为电荷或电流最终只能感受到$\bvec E, \bvec B$这类“真正的场”.},$\rho_f$是自由电荷,$\bvec j_f$是自由电流. 根据"电介质 \upref{dieleS} \upref{Dielec}"与"磁介质\upref{MagMat}"的物理模型,他们被定义为\footnote{如此定义有一定的\textsl{历史原因}.}:
\begin{align}
\bvec D &= \epsilon_0 \bvec E + \bvec P\\
\bvec H &= \frac{\bvec B}{\mu_0} - \bvec M\\
\end{align}

在\textbf{各向同性、非铁磁性的均匀线性介质}中,有构成关系:
\begin{align}
\bvec D &= \epsilon \bvec E = \epsilon_0 \epsilon_r \bvec E = \epsilon_0(1 + \chi_E)\bvec E\\

\bvec H &= \frac{\bvec B}{\mu} = \frac{\bvec B}{\mu_0\mu_r} = \frac{\bvec B}{\mu_0(1 + \chi_B)}\\
\end{align}
其中$\epsilon_r$为相对介电常数,$\mu_r$为相对磁导率,是与物质种类有关的物理量.

\textbf{此时},麦克斯韦方程还可以写为(注意他的适用范围,以及和“真空”麦克斯韦方程组的相似与不同):
\begin{equation}\ali{
&\div\bvec E = \frac{\rho_f}{\epsilon}\\
&\curl\bvec E = -\pdv{\bvec B}{t}\\
&\div\bvec B = 0\\
&\curl\bvec B = \mu J_f + \mu\epsilon\pdv{\bvec E}{t}
}\end{equation}

% 在各向同性线性介质中,有 $\bvec P = \chi_E \epsilon_0 \bvec E$,  $\bvec M = \chi_B \bvec H$.  代入上式得 $\bvec D = (1 + \chi_E)\epsilon_0\bvec E$ 和  $\bvec H = \frac{\bvec B}{(1 + \chi_B)\mu_0}$. 

% 定义相对介电常数为 $\epsilon_r = 1 + \chi_E$, 相对磁导率为 $\mu_r = 1 + \chi_B$, 则 $\bvec D = \epsilon_r\epsilon_0\bvec E$, $\bvec H = \frac{\bvec B}{\mu_r\mu_0}$,  
