% 磁标势

\subsection{磁标势推导}
\begin{equation}
\oint_L \bvec H \cdot \dd  l = \int_S \bvec J \cdot \dd S

\end{equation}当S内无电流亦环路L内无电流时
\begin{equation}
\oint_ L\bvec H \cdot \dd \bvec l = 0
\end{equation}
在该区域,有$\bvec J=0$,则在该区域内,磁场方程为 
\begin{equation}
\curl \bvec H =0
\end{equation}
\begin{equation}
\div \bvec B=0
\end{equation}
\begin{equation}
\bvec B=\mu_0(\bvec H+\bvec M)=f(\bvec H)
\end{equation}
式(5)的写法为函数形式因为在如铁磁性物质中,线性关系$\bvec B=\mu_0 \bvec H$不成立.
而$\bvec H$ 与 $\bvec B$ 的关系可以由磁滞回线确定.
把(5)代入(4)得
\begin{equation}
\div \bvec H= - \div \bvec M
\end{equation}
把分子电流看作由一对假想的磁荷组成的磁偶极子,则和电场中的$\div \bvec P=-\rho _p$对应.
\begin{equation}
\rho_m=-\mu_0 \div \bvec M
\end{equation}
因而 在$\bvec J=0$区域内开始微风方程可以写为:




