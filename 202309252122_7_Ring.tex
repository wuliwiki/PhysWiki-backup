% 环
% keys 集合|运算|结合律|分配律
% license Xiao
% type Tutor

%未完成;理想是否需要单独开下一个词条来讲?

\pentry{群\upref{Group}}

\subsection{环的定义}
群\upref{Group}是代数学中研究的最简单的结构,只由一个集合配上一个运算构成,这个运算只有 $4$ 条公理进行限制。 通常,在群之后会介绍的一个更复杂一些的概念,是\textbf{环(ring)}。 一个环就是由一个集合配上两个运算构成的集合,通常把这两个运算叫做\textbf{加法}和\textbf{乘法};环上的加法和乘法分别构成群,不过乘法群不包括加法的单位元,而加法群是阿贝尔群。

% 为了简洁地定义环,先定义两个群论中并没有涉及到的概念。

% \begin{definition}{半群和幺半群}\label{def_Ring_1}
% 给定集合 $G$ 及其上的一个运算,运算符号忽略。如果运算满足:
% \begin{itemize}
% \item 封闭性
% \item 结合性
% \end{itemize}
% 那么称 $G$ 配合该运算构成一个\textbf{半群(semi-group)}。
% 如果半群中含单位元,则构成一个\textbf{幺半群(monoid)}。这里的“幺”是“一”的意思。
% \end{definition}

% 由此可见,群就是每个元素都有对应可逆元的幺半群。有了幺半群的概念,就可以很方便地定义环了。

\begin{definition}{环}\label{def_Ring_2}
一个\textbf{环(ring)}是一个配备了加法$+$和乘法$\cdot$两种运算的集合$R$,记作$(R, +, \cdot)$,并且满足以下公理:
\begin{enumerate}
    \item 加法$+$是封闭的。
    \item 加法$+$具有结合性。
    \item 加法$+$存在单位元,记为$0$,常称为“零元”。
    \item 任意元素在加法$+$下存在逆元。
    \item 加法$+$是交换的。
    \item 乘法对加法满足左右分配律:对于任意$x, y, z\in \mathbb{F}$,有$x\cdot(y+z)=x\cdot y+x\cdot z$(左分配律),且$(y+z)\cdot x=y\cdot x+z\cdot x$(右分配律)\footnote{分左右两种分配律,是因为环的乘法不一定交换。}。
    \item 乘法$\cdot$是封闭的。
    \item 乘法$\cdot$具有结合性。
    \item 对于非$0$元素,乘法$\cdot$存在单位元,记为$1$,常称为“幺元”\footnote{这是有争议的。有的数学家从形式上定义,不要求环有幺元,而把有幺元的特别称呼为“幺环”;但是有的数学家认为,我们很少研究不含幺元的环,所以不如直接定义环必须有幺元,简化讨论。本书使用后一种传统,但是有时也会为了强调而使用“幺环”的术语,也就是说我们会将“环”和“幺环”视作等价的术语来使用。}。
    \end{enumerate}
\end{definition}

通常,为了方便表示,我们也会省略环中乘法的符号,将 $a\cdot b$ 写为 $ab$。

由定义,环的加法必须是可交换的,但乘法却不一定。如果 $R$ 的乘法也交换的话,我们就称 $R$ 为一个\textbf{交换环(commutative ring)}。

\begin{example}{整数环}
整数集合配上通常的加法和乘法,构成一个交换环,记为 $\mathbb{Z}$。
\end{example}

\begin{example}{数域}
有理数集合配上通常的加法和乘法,构成一个交换环,记为 $\mathbb{Q}$。类似地,实数构成交换环 $\mathbb{R}$,复数构成交换环 $\mathbb{C}$。这三个环都是数域的例子,数域是指包含整数的域,而域\upref{field}是之后基于环而讨论的概念。
\end{example}

\begin{example}{多项式环}\label{ex_Ring_1}
设有交换环 $R$,$x$ 是一个自变量,那么集合 $\{\sum\limits_{i=0}^n a_ix^i|n\in\mathbb{Z}^+, a_i\in R\}$ 可以看成是 $x$ 的多项式构成的集合,其系数取遍 $R$。这些多项式之间的加法和乘法由 $R$ 的加法和乘法诱导(定义),并且构成了一个交换环,称为 $R$ 的多项式环。

记 $R$ 上的多项式环为 $R[x]$。
\end{example}

环的定义在一个细节上有争议,那就是乘法需不需要有幺元。有些书中的定义不要求有幺元,也就是说乘法只构成半群即可,这就使得对于任意正整数 $n$,$n$ 的全体倍数构成的集合 $n\mathbb{Z}$ 也是一个环;在这种定义里,会把含幺元的环称作\textbf{含幺环}或者\textbf{幺环}。由于不含幺元的结构一般不研究,所以主流数学界干脆将环定义为有幺元的。小时百科中为了方便表述,认为环都不一定有幺元,将幺环和环区分开,请读者注意。

\subsection{子环}

\begin{definition}{子环}
给定一个环 $R$,如果 $S$ 是 $R$ 的子集,并且在继承 $R$ 的两个运算后也构成环,那么称 $S$ 是 $R$ 的\textbf{子环(subring)}。
\end{definition}



