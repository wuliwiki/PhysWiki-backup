% 集合论(综述)
% license CCBYSA3
% type Wiki

本文根据 CC-BY-SA 协议转载翻译自维基百科\href{https://en.wikipedia.org/wiki/Set_theory}{相关文章}。

\begin{figure}[ht]
\centering
\includegraphics[width=6cm]{./figures/590e7a061f3044c8.png}
\caption{一个维恩图,展示了两个集合的交集。} \label{fig_Set1_1}
\end{figure}
\textbf{集合论}是数学逻辑的一个分支,研究集合,集合可以非正式地描述为对象的集合。尽管任何类型的对象都可以组成一个集合,但集合论——作为数学的一个分支——主要关注那些与整个数学相关的集合。

现代集合论的研究始于19世纪70年代,由德国数学家理查德·德德金德和乔治·康托尔发起。特别是,乔治·康托尔通常被认为是集合论的创始人。在这个早期阶段研究的非形式化系统被称为朴素集合论。在朴素集合论中发现悖论(如罗素悖论、康托尔悖论和布拉利-福尔蒂悖论)之后,20世纪初提出了各种公理化系统,其中泽梅洛–弗兰克尔集合论(无论是否包含选择公理)仍然是最著名和最研究的。

集合论通常被用作整个数学的基础系统,特别是以泽梅洛–弗兰克尔集合论与选择公理的形式。除了其基础性作用外,集合论还提供了一个框架,用于发展数学中的无穷大理论,并在计算机科学(如关系代数理论)、哲学、形式语义学和进化动力学等领域有着广泛的应用。它的基础性吸引力、与悖论的关系、以及对无穷大的概念及其多重应用的影响,使得集合论成为逻辑学家和数学哲学家关注的主要领域之一。当代集合论的研究涵盖了广泛的主题,从实数线的结构到大基数的一致性研究。
\subsection{历史}  
\subsubsection{早期历史}
\begin{figure}[ht]
\centering
\includegraphics[width=6cm]{./figures/d1970bff40e0e3fa.png}
\caption{普尔科修斯(Purchotius)于1730年所绘的波尔斐里树,展示了亚里士多德的《范畴》。} \label{fig_Set1_2}
\end{figure}
基本的对象分组概念自至少在数字的出现以来就已存在,而将集合视为自身对象的概念至少自公元3世纪的《波尔斐里树》以来就存在。集合的简单性和普遍性使得很难确定现在在数学中使用的集合的起源,然而,伯纳德·博尔扎诺的《无穷悖论》(*Paradoxien des Unendlichen*,1851年)通常被认为是集合首次严格引入数学的工作。在他的著作中,他(除了其他内容外)扩展了伽利略的悖论,并引入了无限集合的一对一对应,例如通过关系$5y = 12x$,在区间$[0, 5]$和$[0, 12]$之间建立了对应。然而,他拒绝称这些集合是等势的,他的工作通常被认为在他那个时代的数学中没有产生影响。\(^\text{[1][2]}\)

在数学集合论出现之前,无穷的基本概念被认为完全属于哲学领域(参见:无穷(哲学)和无穷 § 历史)。自公元前5世纪起,从西方的希腊哲学家芝诺开始(以及东方的早期印度数学家),数学家们就一直在与无穷的概念作斗争。随着17世纪末微积分的发展,哲学家们开始普遍区分实际无穷与潜在无穷,其中数学仅涉及后者。\(^\text{[3]}\) 卡尔·弗里德里希·高斯 famously 说道:“无穷不过是一个修辞,帮助我们讨论极限。完成的无穷这一概念不属于数学。”\(^\text{[4]}\)

数学集合论的发展受到了几位数学家的启发。伯恩哈德·黎曼在《几何基础上的假设》(1854年)的讲座中提出了关于拓扑学的新思想,并关于将数学(特别是几何学)建立在集合或流形的基础上,以类的方式(他称之为Mannigfaltigkeit),这现在被称为点集拓扑学。该讲座于1868年由理查德·德德金德出版,同时也出版了黎曼关于三角级数的论文(该论文提出了黎曼积分),后者成为实分析领域中研究“严肃的”不连续函数的起点。年轻的乔治·康托尔进入了这一领域,这引导他研究点集。大约在1871年,受黎曼的影响,德德金德开始在他的出版物中使用集合,这些出版物非常清晰和精确地处理了等价关系、集合的划分和同态。由此,20世纪数学中许多常见的集合论程序可以追溯到他的工作。然而,他直到1888年才出版了关于集合论的正式解释。
\subsubsection{朴素集合论}
\begin{figure}[ht]
\centering
\includegraphics[width=6cm]{./figures/6209bd2e6720530a.png}
\caption{乔治·康托尔,1894年} \label{fig_Set1_3}
\end{figure}
集合论,现代数学家所理解的集合论,通常被认为是由乔治·康托尔于1874年发表的论文《关于所有实代数数集合的一个性质》奠定的基础。\(^\text{[5][6][7]}\)在这篇论文中,他发展了基数的概念,通过一一对应来比较两个集合的大小。他的“革命性发现”是,所有实数的集合是不可数的,也就是说,不能将所有实数列在一张列表中。这个定理是通过康托尔的第一次不可数性证明来证明的,这与更为人熟知的使用对角线法的证明有所不同。

康托尔引入了集合论中的基本构造,如集合$A$的幂集,它是$A$的所有可能子集的集合。他后来证明,幂集$A$的大小严格大于$A$的大小,即使$A$是一个无限集合;这一结果很快被称为康托尔定理。康托尔发展了一种超限数的理论,称为基数和序数,扩展了自然数的算术。他为基数使用的符号是希伯来字母$\aleph$(ℵ,aleph)带有自然数下标;对于序数,他使用希腊字母$\omega$(ω,omega)。

集合论开始成为新“现代”数学方法的一个重要组成部分。最初,康托尔的超限数理论被认为是违反直觉的——甚至是令人震惊的。这导致它遭遇了数学当代人物如利奥波德·克罗内克和亨利·庞加莱的反对,后来也遭遇了赫尔曼·韦尔和L·E·J·布劳威尔的抵制,而路德维希·维特根斯坦提出了哲学上的异议(参见:康托尔理论的争议)。\(^\text{[a]}\)德德金德的代数风格直到1890年代才开始找到追随者。
\begin{figure}[ht]
\centering
\includegraphics[width=6cm]{./figures/81a7f93f8b63fcd5.png}
\caption{戈特洛布·弗雷格,大约1879年} \label{fig_Set1_4}
\end{figure}
尽管存在争议,康托尔的集合论在20世纪初得到了显著的发展,得益于几位著名数学家和哲学家的贡献。理查德·德德金德在同一时期开始在他的出版物中使用集合,并以著名的德德金德切割法构造实数。他还与朱塞佩·皮亚诺合作,发展了皮亚诺公理,这些公理使用集合论的思想形式化了自然数算术,并引入了用于集合成员关系的epsilon符号。可能最为重要的是,戈特洛布·弗雷格开始发展他的《算术基础》。

在他的著作中,弗雷格试图通过逻辑公理来构建所有数学,使用康托尔的基数概念。例如,“马棚里有四匹马”这句话意味着四个对象属于“马”这一概念。弗雷格尝试通过基数(“...的数量”,或$Nx: Fx$)来解释我们对数字的理解,依赖于休谟原则。
\begin{figure}[ht]
\centering
\includegraphics[width=6cm]{./figures/25b899959d5b0a2d.png}
\caption{伯特兰·罗素,1936年} \label{fig_Set1_5}
\end{figure}
然而,弗雷格的工作很短命,因为伯特兰·罗素发现他的公理导致了一个矛盾。具体而言,弗雷格的基本法则V(现在被称为无限制理解公理模式)。根据基本法则V,对于任何足够明确的属性,都存在一个集合,包含所有且仅有具备该属性的对象。矛盾,称为罗素悖论,证明过程如下:

令$R$为所有不包含自身的集合的集合(这个集合有时被称为“罗素集合”)。如果$R$不是它自己的成员,那么它的定义意味着它是它自己的成员;然而,如果它是它自己的成员,那么它就不是它自己的成员,因为它是所有不包含自身的集合的集合。由此产生的矛盾就是罗素悖论。用符号表示:

令  
$R = \{x \mid x \notin x\}$,那么$R \in R \iff R \notin R$  

这出现在多个悖论或违反直觉的结果发生的时期。例如,平行公理无法被证明,存在无法计算或明确描述的数学对象,存在无法通过皮亚诺算术证明的算术定理。其结果是数学的基础危机。
\subsection{基本概念和符号}  
集合论始于对象$o$与集合$A$之间的基本二元关系。如果$o$是$A$的成员(或元素),则使用符号$o \in A$。集合通过列出以逗号分隔的元素,或通过其元素的特征属性,在大括号{ }内描述。\(^\text{[8]}\)由于集合是对象,成员关系也可以涉及集合本身,即集合本身可以是其他集合的成员。

两个集合之间的一个派生二元关系是子集关系,也叫集合包含。如果集合$A$的所有成员也是集合$B$的成员,则$A$是$B$的子集,记作$A \subseteq B$。例如,$\{1, 2\}$是$\{1, 2, 3\}$的子集,$\{2\}$也是,但$\{1, 4\}$不是。根据这一定义,集合是它自己的子集。对于不适用或应当排除这种可能性的情况,定义了真子集,通常记作$A \subset B$、$A \subsetneq B$,或$A \subsetneqq B$(但请注意,符号$A \subset B$有时与$A \subseteq B$同义使用;即允许$A$和$B$相等的可能性)。我们称$A$是$B$的真子集,当且仅当$A$是$B$的子集,但$A$不等于$B$。此外,1、2和3是集合$\{1, 2, 3\}$的成员(元素),但不是它的子集;反过来,像$\{1\}$这样的子集不是集合$\{1, 2, 3\}$的成员。更复杂的关系也可以存在;例如,集合$\{1\}$既是集合$\{1, {1}\}$的成员,也是其真子集。

正如算术包含对数字的二元运算,集合论也包含对集合的二元运算。\(^\text{[9]}\)以下是它们的一部分:

\begin{itemize}
\item 集合A和B的并集,记作$A \cup B$,是所有属于A、B或两者的对象的集合。\(^\text{[10]}\)例如,集合$\{1, 2, 3\}$和$\{2, 3, 4\}$的并集是集合$\{1, 2, 3, 4\}$。
\item 集合A和B的交集,记作$A \cap B$,是所有同时属于A和B的对象的集合。\(^\text{[11]}\)例如,集合$\{1, 2, 3\}$和$\{2, 3, 4\}$的交集是集合$\{2, 3\}$。
\item 集合U和A的差集,记作$U \setminus A$,是所有属于U但不属于A的对象的集合。\(^\text{[12]}\)例如,集合$\{1, 2, 3\} \setminus \{2, 3, 4\}$是集合$\{1\}$,而反过来,集合$\{2, 3, 4\} \setminus \{1, 2, 3\}$是集合$\{4\}$。当$A$是$U$的子集时,差集$U \setminus A$也称为A在U中的补集。在这种情况下,如果$U$的选择在上下文中是明确的,有时使用符号$A^c$代替$U \setminus A$,特别是如果$U$是全集,如在维恩图的研究中。\(^\text{[13]}\)
\item 集合A和B的对称差,记作$A \triangle B$或$A \ominus B$,是所有只属于A或B其中一个集合的对象的集合(即属于其中一个集合但不属于两个集合的元素)。例如,集合$\{1, 2, 3\}$和$\{2, 3, 4\}$的对称差是集合$\{1, 4\}$。它是并集与交集的差集,$(A \cup B) \setminus (A \cap B)$或$(A \setminus B) \cup (B \setminus A)$。
\item 集合A和B的笛卡尔积,记作$A \times B$,是所有可能的有序对$(a, b)$的集合,其中$a$是A的成员,$b$是B的成员。例如,集合$\{1, 2\}$和$\{\text{red}, \text{white}\}$的笛卡尔积是$\{(1, \text{red}), (1, \text{white}), (2, \text{red}), (2, \text{white})\}$。\(^\text{[14]}\)
\end{itemize}
一些基本的集合,具有重要的中心地位,包括自然数集合、实数集合和空集合——唯一一个不包含任何元素的集合。空集合有时也被称为零集合,[15] 虽然这个名称是模糊的,可能导致多种解释。空集合可以用空的大括号表示$\{\}$,也可以用符号$\varnothing$或$\emptyset$表示。

集合$A$的幂集,记作${\mathcal {P}}(A)$,是一个集合,其中的成员是$A$的所有可能子集。例如,集合$\{1, 2\}$的幂集是$\{\{\}, \{1\}, \{2\}, \{1, 2\}\}$。值得注意的是,${\mathcal {P}}(A)$包含了$A$本身和空集合。
\subsection{本体论}
\begin{figure}[ht]
\centering
\includegraphics[width=6cm]{./figures/73913957a15a2238.png}
\caption{冯·诺依曼层次结构的初始段} \label{fig_Set1_6}
\end{figure}
如果一个集合的所有成员都是集合,并且它的成员的成员也是集合,依此类推,那么这个集合就是纯集合。例如,仅包含空集合的集合是一个非空的纯集合。在现代集合论中,通常将注意力限制在冯·诺依曼纯集合宇宙中,许多公理化集合论系统旨在仅公理化纯集合。这个限制有许多技术上的优点,并且几乎没有失去一般性,因为本质上所有数学概念都可以通过纯集合来建模。冯·诺依曼宇宙中的集合被组织成一个累积层次结构,基于它们的成员、成员的成员等的嵌套深度。这个层次结构中的每个集合通过超限递归被分配一个序数$\alpha$,称为它的秩。纯集合$X$的秩被定义为严格大于其任何元素的秩的最小序数。例如,空集合被分配秩0,而仅包含空集合的集合被分配秩1。对于每个序数$\alpha$,集合$V_{\alpha}$被定义为包含所有秩小于$\alpha$的纯集合的集合。整个冯·诺依曼宇宙用$V$表示。
\subsection{形式化集合论} 
基础集合论可以通过非正式和直观的方式进行研究,因此可以使用维恩图在小学进行教学。直观方法默认为一个集合可以由满足某一特定定义条件的所有对象的类组成。这一假设导致了悖论,其中最简单且最著名的是罗素悖论和布拉利-福尔蒂悖论。公理化集合论最初是为了将集合论中的这些悖论去除而设计的。\(^\text{[注1]}\)

最广泛研究的公理化集合论系统意味着所有集合构成一个累积层次结构。\(^\text{[b]}\)这些系统有两种类型,分别是:
\begin{itemize}
\item 仅包含集合的系统。这包括最常见的公理化集合论——泽梅洛–弗兰克尔集合论(ZFC),以及ZFC的片段,包括:
\item 泽梅洛集合论,它将替代公理模式替换为分离公理模式;
\item 一般集合论,泽梅洛集合论的一个小片段,足以支持皮亚诺公理和有限集合;
\item 克里普克-普拉特集合论,它省略了无限公理、幂集公理和选择公理,并弱化了分离公理和替代公理模式。
\item 集合与真类的系统。这些系统包括冯·诺依曼-伯奈-哥德尔集合论,它与ZFC在关于集合的定理方面具有相同的强度;以及莫尔斯-凯利集合论和塔尔斯基-格罗腾迪克集合论,它们都比ZFC更强
\end{itemize}。
上述系统可以被修改以允许原始元素,这些对象可以是集合的成员,但它们本身不是集合,也没有任何成员。

由威拉德·范·奥曼·奎因(Willard Van Orman Quine)提出的新基础(New Foundations,NF)系统(包括允许原始元素的NFU和不包含原始元素的NF)并不基于累积层次结构。NF和NFU包括一个“所有集合的集合”,相对于这个集合,每个集合都有一个补集。在这些系统中,原始元素是重要的,因为NF(而非NFU)会产生不满足选择公理的集合。尽管NF的本体论不反映传统的累积层次结构,并且违反了良基础性,托马斯·福斯特认为它确实反映了集合的迭代观念。\(^\text{[16]}\)

构造性集合论系统,如CST、CZF和IZF,将它们的集合公理嵌入直觉逻辑中,而不是经典逻辑。还有一些系统接受经典逻辑,但具有非标准的成员关系。这些系统包括粗集理论和模糊集理论,其中体现成员关系的原子公式的值不仅仅是“真”或“假”。ZFC的布尔值模型是一个相关的主题。

Edward Nelson在1977年提出了一种ZFC的扩展,称为内部集合论。\(^\text{[17]}\)
\subsubsection{应用}  
许多数学概念可以仅通过集合论的概念精确定义。例如,图、流形、环、向量空间和关系代数等多种数学结构都可以定义为满足各种(公理化)性质的集合。等价关系和顺序关系在数学中无处不在,数学关系的理论可以用集合论来描述。\(^\text{[18][19]}\)

集合论也是许多数学领域的有前景的基础系统。自《数学原理》第一卷出版以来,人们声称大多数(甚至所有)数学定理都可以使用恰当设计的集合论公理集来推导,并通过许多定义加以补充,使用一阶或二阶逻辑。例如,自然数和实数的性质可以在集合论中推导,因为每个数系都可以通过将其元素表示为特定形式的集合来定义。\(^\text{[20]}\)

集合论作为数学分析、拓扑学、抽象代数和离散数学的基础也是不具争议的;数学家们(原则上)接受这些领域的定理可以从相关定义和集合论的公理中推导出来。然而,至今为止,从集合论中正式验证的复杂数学定理的完整推导仍然很少,因为这样的形式化推导通常比数学家常用的自然语言证明要长得多。一个验证项目——Metamath,包含了从ZFC集合论、一阶逻辑和命题逻辑出发,通过人工编写并计算机验证的超过12,000个定理的推导。[21]
\subsection{研究领域}  
集合论是数学的一个主要研究领域,具有许多相互关联的子领域:
\subsubsection{组合集合论} 
组合集合论涉及有限组合学向无限集合的扩展。这包括基数算术的研究和拉姆齐定理的扩展研究,例如埃尔德什–拉多定理的研究。
\subsubsection{描述集合论 } 
描述集合论是研究实数线的子集,更一般地,是研究波兰空间的子集。它从研究Borel层次中的点类开始,并扩展到更复杂层次的研究,例如投影层次和瓦奇层次。许多Borel集合的性质可以在ZFC中建立,但证明这些性质对更复杂的集合成立需要额外的公理,这些公理与确定性和大基数相关。

有效描述集合论是集合论和递归理论之间的领域。它包括研究轻面点类,并与超算术理论密切相关。在许多情况下,经典描述集合论的结果都有有效的版本;在某些情况下,通过首先证明有效版本,然后扩展(“相对化”)它,使其更广泛适用,从而获得新的结果。
 
最近的研究领域涉及Borel等价关系和更复杂的可定义等价关系。这对许多数学领域中不变量的研究具有重要应用。