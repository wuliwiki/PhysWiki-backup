% 玻尔兹曼分布(统计力学)
% keys 玻尔兹曼分布|等概率原理|经典统计

\pentry{麦克斯韦—玻尔兹曼分布\upref{MxwBzm},理想气体的状态密度\upref{IdSDp},拉格朗日乘数法\upref{LagMul}}

根据等概率原理,对于平衡状态的孤立系统,每个可能的微观状态出现的概率是相等的.

根据统计力学中的量子力学假设,系统中单粒子态能级为分立的:$\epsilon_1,\epsilon_2,\cdots$.其中设第 $l$ 个能级的简并数为 $\omega_l$(意味着这个能级一共有 $\omega_l$ 种线性无关的态).设第 $l$ 个能级上共有 $a_l$ 个粒子,序列 ${a_l}$ 构成粒子系统的一种分布.微观状态数最多的分布出现的概率最大,称为\textbf{最概然分布}.

注意等概率原理中只涉及“可能出现”的微观状态,也就是说要满足孤立系统的粒子数守恒、能量守恒条件:
\begin{equation}\label{MBsta_eq1}
\begin{aligned}
\sum_l n_l=N\\
\sum_l \epsilon_l a_l=E
\end{aligned}
\end{equation}

我们下面要谈的是玻尔兹曼分布,它\textbf{不涉及全同粒子假设},即粒子之间是可区分的.我们将得到同经典统计中一样的结果\upref{MxwBzm}.

\subsection{推导}
\addTODO{文字说明}
\begin{equation}
\Omega=\frac{N!}{\Pi_l a_l!}\omega_l^{a_l}
\end{equation}

\begin{equation}
\ln \Omega=\ln N!-\sum_{l}a_l!+\sum_l a_l\ln \omega_l
\end{equation}

假设所有的 $a_l$ 都很大,根据近似等式 $\ln n! = n(\ln n-1)$,可以化简得到

\begin{equation}
\ln \Omega=N\ln N-\sum_l a_l\ln a_l+\sum_l a_l\ln \omega_l
\end{equation}

在约束条件