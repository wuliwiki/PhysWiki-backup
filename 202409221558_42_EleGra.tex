% 电磁力和引力
% keys 电磁力|引力|作用量
% license Usr
% type Tutor

\pentry{光与物质粒子的统一(相对论点粒子的作用量)\nref{nod_RAct}}{nod_e590}
本节将以一种“统一”的角度给出电磁力和引力的作用量。
\subsection{从自由粒子到势阱中的粒子}
在光与物质粒子的统一(相对论点粒子的作用量)\upref{RAct}一节,我们得到了自由粒子的作用量,其具有下面的形式
\begin{equation}\label{eq_EleGra_1}
S=-m\int\sqrt{-\eta_{\mu\nu}\dd x^\mu\dd x^\nu}=-m\int\sqrt{\dd t^2-\dd{\vec x}^2}~.
\end{equation}
现在考虑粒子处于势 $V(x)$ 中。尽管非相对论情形(Newton力学)时处于势为 $V(x)$ 的粒子作用量为
\begin{equation}
S_{NR}=\int\dd t\qty(\frac{1}{2}m\qty(\dv{\vec x}{t})^2-V(x)),~
\end{equation}
但是我们并不能理所当然的将 $V(x)$ 加入\autoref{eq_EleGra_1} 中得到相对论情形处于势阱 $V(x)$ 中的粒子。换言之,我们不知道如何将 $V(x)$ 放入\autoref{eq_EleGra_1} 中。

尽管如此,然而可以肯定的是,将 $V(x)$ 放入\autoref{eq_EleGra_1} 中,只有两种可能:a.根号外面,b.根号里面。因此,我们有下面的两种选择:
\begin{enumerate}
\item E:\begin{equation}
S=-\int
\end{equation}

\item
\end{enumerate}


