% Julia 简介

\begin{issues}
\issueDraft
\end{issues}

许多语法和 Matlab 很像, 另一些和 C/C++ 很像, 如果你两种语言都已经会了, 那么 Julia 是十分简单的.

安装方法
\begin{lstlisting}[language=bash]
sudo snap install Julia --classic
\end{lstlisting}

另外 Jupyter Notebook 也支持 Julia.

\subsection{命令行}
\begin{itemize}
\item 为了区分, 系统的命令行叫做 terminal, 而 Julia 的命令行叫做 \textbf{REPL (read-eval-print-loop)}
\item \verb|Ctrl + D| 退出或者 \verb|exit()| 退出
\item \verb|ans| 变量和 Matlab 一样
\item 用分号可以在一行中运行多个命令.
\item 要情况 workspace 的变量只能重启
\end{itemize}

\begin{lstlisting}[language=Julia]
println("hello world")
\end{lstlisting}

\subsection{计算器}
\begin{itemize}
\item \verb|b = 2a| 和 \verb|c = 2(b+3)| 都可以省略乘号
\item \verb|2/3| 返回 \verb|Float64| 而不是整数
\item \verb|2//3| 返回 \verb|Rational{Int64}|, 相当于用两个 \verb|Int64| 来表示分数, 分数之间的运算没有误差.
\item 表示复数如 \verb|1 + 2im|, 相当于 \verb|ComplexF64(0,1)|.
\end{itemize}

\subsection{变量}
\begin{itemize}
\item 用 \verb|typeof()| 查看某个变量的类型
\item 类型完全是动态的, 但可以在表达式或者函数定义后面加上 \verb|::类型| 限制变量类型
\item 字符串用双引号, 单个字符用单引号
\item 查看类型的最大和最小值如 \verb|typemax(Int64)|, \verb|typemin(Int64)|
\item 标准库提供任意精度类型 \verb|BigInt| 和 \verb|BigFloat| (底层是 GMP)
\item 变量名可以用 UTF-8, 在一些编辑器中可以用反斜杠 latex 命令打出对应的字符
\item 注意和 C++ 不同, Julia 中的变量名只是 object 的标签
\item \verb|//| 可以让整数除法返回 \verb|Rational| 类型, 例如 \verb|1//2 + (3//4)im| 的类型是 \verb|Complex{Rational{Int64}}|
\end{itemize}

\subsection{矩阵}
\begin{itemize}
\item 索引从 1 开始, 和 Matlab 一样
\item 矩阵使用列主元排序, 和 Matlab 一样
\item \verb|Array{T, 1}| 不区分行和列, 但是 \verb|Array{T, 2}| 可以区分行向量和列向量
\item 矩阵切割如 \verb|Psi[:, j, :]|
\item 矩阵尺寸 \verb|size(Psi, 维度)|
\item 零向量矩阵 \verb|zeros(整数)| 或 \verb|zeros(整数, 整数)| 分别返回 \verb|Array{Float64,1}| 和 \verb|Array{Float64,2}|
\item 64 位机器上, 整数 literal 的类型默认是 \verb|Int64|
\item \verb|a = [1 2; 3 4]| 得到 \verb|Array{Float64,2}|
\item \verb|a[:, 1]| 仍然得到 \verb|Array{Float64,2}|
\item \verb|a[:, 1] = [3; 4]| 可以修改 \verb|a| 的值
\item \verb|b = a[:, 1]| 却是复制数据.
\item \verb|a| 作为参数传给函数是 pass by reference
\item \verb|a[:, 1]| 作为参数传给函数是 pass by value, 如果需要 by reference, 使用 \verb|SubArray| 类
\item \verb|view(a, :, 1)| 得到的类型是 \verb|SubArray|, 具体是…… \verb|SubArray{Int64,1,Array{Int64,2},Tuple{Base.Slice{Base.OneTo{Int64}},Int64},true}|
\item \verb|SubArray| 不存数据, 可以用来给函数参数 pass by reference
\item \verb|SubArray <: AbstractArray{T,Ndim} <: Any|
\item \verb|Array{T,Ndim} <: DenseArray{T,Ndim} <: AbstractArray{T,Ndim} <: Any|
\item 所以只要函数参数接受 \verb|AbstractArray|, 就可以同时接受 \verb|Array, SubArray|.
\end{itemize}

\verb|SubArray| 的结构
\begin{lstlisting}[language=julia]
struct SubArray{T,N,P,I,L} <: AbstractArray{T,N}
    parent::P
    indices::I
    offset1::Int  # for linear indexing and pointer, only valid when L==true
    stride1::Int  # used only for linear indexing
    ...
end
\end{lstlisting}


随机矩阵
\begin{lstlisting}[language=Julia]
rand(ComplexF64, Nr1, Nr2, Npw)
\end{lstlisting}

\subsection{脚本}
\begin{itemize}
\item 在系统命令行用 \verb|Julia <file>| 运行脚本, 用 \verb|Julia <file> <arg1> <arg2>| 给出 arguments
\item 在 REPL 中运行脚本如 \verb|include("/Users/addis/Desktop/main.jl")|(路径不区分大小写), 也支持反斜杠, 但要转义成 \verb|\\|
\item 在文件中, 如果要确定当前文件是不是主文件, 用 \verb|abspath(PROGRAM_FILE) == @__FILE__|
\end{itemize}

\subsection{画图}
第三方画图包 \verb|Plots|, 无需安装!
\begin{lstlisting}[language=Julia]
using Plots
x = 1:10; y = rand(10); # These are the plotting data
plot(x,y)
\end{lstlisting}

\subsection{常用函数}
函数定义
\begin{lstlisting}[language=Julia]
function sphere_vol(r)
    return 4/3*pi*r^3
end
\end{lstlisting}

hash (把结果的 hash 输出可以保证计算过程不被优化掉)
\begin{lstlisting}[language=Julia]
hash(矩阵)
\end{lstlisting}

当前时间 \verb|time()|
