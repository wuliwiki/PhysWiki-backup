% 应力-应变曲线

\begin{table}[ht]
\centering
\caption{常用力学概念及其对应物理量}\label{SSCUR_tab1}
\begin{tabular}{|c|c|c|}
\hline
名称 & 英文名 & 描述\\
\hline
刚度 & Stiffness & 材料抵抗弹性变形的能力,弹性区间内应力-应变曲线的斜率,即杨氏模量$E$\\
\hline
屈服强度 & Yield Strength & 弹性区间内材料可承受的最大应力,$\sigma_y$\\
\hline
极限强度 & Tensile Strength & 材料可承受的最大应力,$\sigma_{TS}, \sigma_b$\\
\hline
硬度 & Hardness & “材料抵抗局域塑性变形”的能力,一般与强度呈正比 $Hardness \propto \sigma_{TS}$\\
\hline
延展性 & Ductility & 材料断裂时的应变,以百分比记。$\%EL = \frac{l-l_0}{l} $\\
\hline
韧性 & Resilience, Toughness & 断裂材料所需要的能量,即应力-应变曲线的面积。$U = \int_0^\varepsilon \sigma \dd \varepsilon$\\
\hline
\end{tabular}
\end{table}
