% 厦门大学 2008 年 考研 量子力学
% license Usr
% type Note

\textbf{声明}:“该内容来源于网络公开资料,不保证真实性,如有侵权请联系管理员”

\subsection{(25 分)简述题(每小题5分)}
(1) 解释微观粒子的波粒二象性,并写出德布罗意(de Broglie)关系式。

(2) 解释能级简并的概念并指出其起因。

(3) 经典力学和量子力学中,守恒量的含义有何不同?

(4) Clebsch-Gordan 系数(记为 $\langle  j_1 m_1 j_2 m_2|j m  \rangle$)不为零的条件。

(5) 微扰论的适用条件是什么?在库仑场中,微扰理论为什么不适用于计算高能级($n$ 较大)的修正?

\subsection{(25 分)}
质量为 \( m \) 的粒子在外场作用下作一维运动 \((- \infty < x < \infty)\),已知其处于束缚态 \(\varphi_1(x)\) 时,动能的平均值等于 \( E_1 \),并已知 \(\varphi_1(x)\) 是实函数且已归一化。试求当粒子处于态 \(\varphi_2(x) = \varphi_1(x) e^{ikx}\)(\(k\) 为实数)时动量平均值及动能平均值。
\subsection{(25 分)}
设已知在 \( L^2 \) 和 \( L_z \) 的共同表象中,算符 \( L_x \) 和 \( L_y \) 的矩阵分别为

$$L_x = \frac{\hbar}{\sqrt{2}}\begin{pmatrix}0 & 1 & 0 \\\\1 & 0 & 1 \\\\0 & 1 & 0\end{pmatrix}, \quad L_y = \frac{\hbar}{\sqrt{2}}\begin{pmatrix}0 & -i & 0 \\\\i & 0 & -i \\\\0 & i & 0\end{pmatrix}.~$$

(1)求\( L_z \)和\( L_y \),的本征值和归一化的本征函数:

(2)将矩阵\( L_z \)和\( L_y \),对角化。
\subsection{(25 分)}

\subsection{(25 分)}

\subsection{(25 分)}