% MacOS 基础笔记

\begin{issues}
\issueDraft
\end{issues}

\subsection{快捷键}
\begin{itemize}
\item Mac 的 Delete 键是 Win 的 backspace
\item Mac 的 Cmd 是 Win 的 Win 键
\item Mac 的 Opt 是 Win 的 Alt 键
\item \verb|Cmd + c|, \verb|Cmd + v| 复制粘贴
\item \verb|Cmd + c| 然后 \verb|Cmd + Alt + v| 移动
\item \verb|Cmd + Del| 把文件移动到回收站
\item \verb|Cmd + z| 撤销
\item \verb|Cmd + Tab| 切换程序
\item \verb|Cmd + Delete| 删除
\item \verb|Ctrl + 左右箭头| 切换屏幕
\item \verb|Ctrl + 上下箭头| 程序管理
\item \verb|Ctrl + Alt + eject| 睡眠
\item \verb|Ctrl + Alt + Cmd| 关机
\item \verb|Alt + Cmd + ESC| 强制退出
\item \verb|Ctrl + Shift + eject| 关闭屏幕
\item \verb|F11| 临时显示桌面
\end{itemize}

\subsection{常识}
\begin{itemize}
\item Mac 不能写入 NTFS 文件系统只能读取
\item account name 很难修改,最好新建一个(管理员)用户,然后把老用户删掉。
\item 要默认 terminal 的 shell (如 \verb|/bin/bash| 可以在 user 设置里面把左下角解锁, 然后右键用户名, 高级设置里面修改。
\item 修改 hostname 用: \verb|sudo scutil --set HostName 新名字|
\end{itemize}

\subsection{VMware MacOS 虚拟机}
\begin{itemize}
\item 据说 VMware Workstation Pro 和 MacOS 兼容比较好。 笔者测试 MacOS 12.4 没问题。
\end{itemize}

若显示性能较差(即用小分辨率的屏幕可以明显变得顺滑), 则修改 xm 文件的以下设置(如果没有则添加)。 但是亲测无效。
\begin{lstlisting}[language=none]
svga.vramSize = "268435456"
vmotion.checkpointFBSize = "1342177728"
vmotion.checkpointSVGAPrimarySize = "268435456"
vmotion.svga.mobMaxSize = "268435456"
vmotion.svga.graphicsMemoryKB = "262144"
svga.graphicsMemoryKB and set its value to "262144"
vmotion.svga.maxTextureSize and assign it the value "16384"
vmotion.svga.maxTextureAnisotropy, and change it to "2"
\end{lstlisting}

\subsection{命令行}
\begin{itemize}
\item 强制重启(用 GUI 重启可能会有一些软件不能正常退出) \verb|sudo shutdown -r now|
\end{itemize}
