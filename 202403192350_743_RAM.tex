% 随机存储器
% keys 随机存储器|随机存取存储器|内存
% license Xiao
% type Tutor

\begin{issues}
\issueDraft
\end{issues}

\subsection{随机存储器(Random Access Memory, RAM)}

随机存储器(Random Access Memory, RAM)是一种可读可写的存储器,其任何一个存储单元都是可以随机存取的,而且存取时间与单元的物理位置无关。

计算机系统的主存储器通常都是采用随机存储器。根据信息存储的物理原理,又可以分为静态随机存储器(SRAM)和动态存储器(DRAM)。


\subsection{SRAM}

\begin{figure}[ht]
\centering
\includegraphics[width=14cm]{./figures/bee5d823f761fcec.png}
\caption{一种简单SRAM存储单元示意图} \label{fig_RAM_2}
\end{figure}

静态随机访问存储器(Static random-access memory,SRAM)


SRAM的Static:只要通电就一直存储。

但是SRAM用了太多MOS管,而且总有两个MOS管饱和导通,占空间,功耗大。而且地址线多,导致不能做太大。

\subsection{DRAM}

通常我们俗称为内存,而我们使用的内存条全称叫做,


\subsection{NVRAM}

我们上述所说的内存介质都需要通电来维持数据



参考文献:
\begin{enumerate}
\item 唐朔飞。 计算机组成原理[M]. 高等教育出版社。 2008
\end{enumerate}
