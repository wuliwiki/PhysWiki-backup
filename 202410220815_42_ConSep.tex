% 凸集的可分离性
% keys 凸集|分离性
% license Usr
% type Tutor

\pentry{凸集和凸体\nref{nod_ConSet},泛函与线性泛函\nref{nod_Funal}}{nod_959a}

线性空间中集 $M,N$ 的分离性是指存在一个\enref{线性泛函}{Funal} $f$,使得泛函在这两个集上的值由一个常数区分开。凸集的分离性是指线性空间中两个凸集,若其中之一的\enref{核}{ConSet}非空且不与另一集相交,则必存在非零线性泛函将这两个凸集分离。

\begin{definition}{分离性}
设 $M,N$ 是实线性空间的两个子集,$f:L\rightarrow\mathbb R$ 是线性泛函,若存在常数 $C$ 使得任意 
\begin{equation}
f(x)\left\{\begin{aligned}\leq C,\quad x\in N\\
\geq C,\quad x\in M
\end{aligned}\right.~
\end{equation}
则称 $f$ 分离集 $M,N$。
即指
\begin{equation}
\inf_{x\in M}\geq
\end{equation}

\end{definition}
