% 南京航空航天大学 2013 量子真题
% license Usr
% type Note

\textbf{声明}:“该内容来源于网络公开资料,不保证真实性,如有侵权请联系管理员”

\subsection{简答题 (本题 30 分,每小题 15 分)}

① 如果波函数$\psi$不是力学量$F$的本征态,那么在态$\psi$中测量$F$会发生什么情况?

② 什么是定态?它有何特性?

\subsection{二}
$\hat{\vec A} = (A_x, A_y, A_z), \hat{\vec B} = (B_x, B_y, B_z)$ 是与泡利算符 $\hat{\sigma} = (\hat{\sigma}_x, \hat{\sigma}_y, \hat{\sigma}_z)$ 对易的任意矢量算符,证明:
$\hat{\vec A} \cdot \hat{\sigma}(\hat{\vec B} \cdot \hat{\sigma}) = \hat{\vec A} \cdot \hat{\vec B} + i \hat{\sigma} \cdot (\hat{\vec A} \times \hat{\vec B})$。(本题 30 分)

\subsection{三}
质量为 $m$ 的一维体系哈密顿量为 $H = p2 /2m+V(x),其中 V(x)= V0 x2n, V0 > 0 是常数、
n 为自然数。设 ψn(x) 为 H 的本征函数:
① 证明动量算符在 ψn(x) 态中的平均值为零;
② 求在态 ψn(x) 中动能平均值和势能平均值之间的关系。(本题 30 分,每小题 15 分)