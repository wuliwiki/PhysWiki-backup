% 相空间和相流
% keys 相空间|相流|相点|相曲线|相速度|静止点|扩张相空间|积分曲线|单参数变换群|推进映射
\pentry{映射\upref{map}}
相空间和相流是经典场论的基本概念,经典场论的几何图像以它们作为基石得以构建.在给出严格的数学定义之前,我们先研究一些例子,这些例子将有助于我们进一步的理解.
\subsection{唯象理解}
\subsubsection{相空间}
研究一个物理系统,可以说是研究它的发展过程,发展过程即是指它的过去、现在和未来系统所处的状态.

一个过程称为\textbf{确定的},如果它的整个过去和整个未来能由它现在的状态唯一确定.一个过程的所有可能状态称为它的\textbf{相空间},相空间的元素称为\textbf{相点}.

\begin{example}{}
经典力学研究系统的运动,这个运动的过去和未来由系统中所有点的初始位置和初始速度唯一确定.力学系统的相空间是由所有质点可能出现的位置和速度构成的集合,其上每一元素是系统所有质点的某一可能位置和可能速度的集合.具体来说,假设 $x_{i},y_i,z_i;\dot{x}_i,\dot{y}_i,\dot z_i$ 分别代表第 $i$ 个质点在三维空间中的位置和速度,且质点共有 $n$ 个,则相空间上的每一元素可以表示为 $(x_1,y_1,z_1\cdots,x_n,y_n,z_n;\dot x _1,\dot y_1,\dot z_1\cdots,\dot x_n,\dot y_n,\dot z_n)$.显然这一相空间处于一个 $6n$ 维的空间中.

量子力学中质点的运动不是由确定的过程描述的.
\end{example}

一个过程称为\textbf{有限维的},如果它的相空间是有限维的,即用来描述它的状态的参数个数是有限的.例如刚刚例子中的经典力学中 $n$ 个质点的运动对应的过程是有限维的,它的相空间维数是 $6n$.而流体运动的过程不能用有限维相空间来描述.

一个过程称为\textbf{可微的}.如果它的相空间具有可维流形(\autoref{Manif_def3}~\upref{Manif})的结构,并且它的状态随时间的变化由可微函数描述.例如:力学系统的质点坐标和速度随时间的变化以可微方式进行;而在量子力学中质点的位置则不能用可微函数来描述(否则它将同时具有确定的位置和速度,而这违背了不确定原理).
\subsubsection{相流}
\textbf{相流}是“确定过程”的数学模型.其直观理解如下:

设 $M$ 是相空间而 $x\in M$ 是过程的初始状态,又设 $g^t x$ 表示初始状态为 $x$ 的过程在时刻 $t$ 的状态.对每一实数 $t$ ,$g^t$ 确定相空间 $M$ 到它自身的一个映射:
\begin{equation}
g^t:M\rightarrow M
\end{equation}
此映射称为\textbf{推进映射},它将每一状态 $x\in M$ 映入新的状态 $g^t x\in M$.例如 $g^0$ 是恒等映射,它将 $M$ 的每一点留在它原来的位置.而且
\begin{equation}
g^{t+s}=g^tg^s
\end{equation}
因为 $x$ 经过时间 $s$ 后进入状态 $y=g^s x$,$y$ 经过时间 $t$ 后进入状态 $z=g^t y$ 和经过时间 $t+s$ 后进入状态 $z=g^{t+s} x$ 是相同的.

对任一相点 $x\in M$,将它作为过程的初始状态.则随着时间的流逝,此过程的状态将改变,而点 $x$ 在相空间中将描出一条\textbf{相曲线} $\{g^tx|t\in \mathbb R\}$.由于每个相点沿着它自己的相曲线运动,因此 $t$ 推进映射族 $g^t;M\rightarrow M$ 正好构成\textbf{相流}. 
\subsection{严格定义}
现在来进行严格的定义,在每种情形 $M$ 都是一个\textbf{任意的}集合.
\begin{definition}{单参数变化群}
一个由所有实数组成的集合 $(t\in \mathbb R)$ 所标记的,由集合 $M$ 到它自身的映射族 $\{g^t\}$ 称为 $M$ 的\textbf{单参数变换群},若对所有的 $s,t\in\mathbb R$ 满足
\begin{equation}
g^{t+s}=g^tg^s
\end{equation}
 且 $g^0$ 是恒等映射(它使每点固定).
\end{definition}
显然,单参数变换群是个\textbf{交换}的群(即 $g^tg^s=g^sg^t$),且每个映射 $g^t$ 是个\textbf{双射}.事实上,$g^tg^s=g^{t+s}=g^{s+t}=g^sg^t$.双射证明如下:若 $g^t x=g^t y$,则 $x=g^0x=g^{-t}g^tx=g^{-t}g^ty=g^0y=y$;任意 $x\in M$,$\exists g^{-t}x\in M$,使得 $g^t(g^{-t}x)=x$.
\begin{definition}{相流,相空间}
由集合 $M$ 和 $M$ 的单参数变换群 $\{g^t\}$ 组成的偶 $(M,\{g^t\})$ 称为\textbf{相流}.其中,$M$ 称为相流的\textbf{相空间},而它的元素称为\textbf{相点}.
\end{definition}
