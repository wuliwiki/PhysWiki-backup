% 惰性气体分子林纳德-琼斯势(量子力学)
% keys 林纳德琼斯势|伦纳德琼斯势|惰性气体分子
% license Xiao
% type Tutor

惰性气体分子是单原子分子,这类分子振动能往往用林纳德-琼斯势(Lennard-Jones Potential,又写作伦纳德-琼斯势)描述:
\begin{equation}
V(r) = 4 D\left[(\sigma / r)^{12} - (\sigma/r)^6\right] ~.
\end{equation}

其中 $r$ 是原子相对平衡位置的位移,$D$ 和 $\sigma$ 是不同分子的参数,一般 $D$ 有能量量纲、$\sigma$ 有长度量纲。林纳德-琼斯势在 $r \rightarrow 0$ 时 $V(r) \rightarrow +\infty$,而在 $r \rightarrow +\infty$ 时 $V(r) \rightarrow 0$。几个典型惰性气体分子的参数如下,
\begin{table}[ht]
\centering
\caption{经典分子参数}\label{tab_LenJoP1}
\begin{tabular}{|c|c|c|}
\hline
分子 & $D/\Si{meV}$ & $\sigma$/埃米 \\
\hline
Ne & 3.10 & 2.74 \\
\hline
Ar & 10.4 & 3.40 \\
\hline
Kr & 14.0 & 3.65 \\
\hline
Xe & 20.0 & 3.98 \\
\hline
\end{tabular}
\end{table}


\subsection{简谐近似}
一个有“平衡位置”的势能通常能展开为 $V(r) = \epsilon + \frac{1}2 k (r-r_0)^2$ 的形式。

为此首先考虑 $r_0$,$V(r)$ 在 $r=r_0$ 处有极值,故
$$\eval{\dv{V(r)}{r}} _{r=r_0} = 4D\left(-12 \frac{\sigma^{12}}{r_0 ^{13}} + 6 \frac{\sigma^6}{r_0^7}\right) = 0~.$$
可得 $r_0 = \sqrt[6]{2} \sigma$。

$V(r)$ 在 $r=r_0$ 处取极值是 $\epsilon$,故可以代入得到 $\epsilon = -D$。而:
$$k = \eval{\dv{^2 V(r)}{r^2}}_{r=r_0} = 4 D \left[156 (\sigma^{12} / r_0^{14}) - 42 (\sigma^{6}/r_0^{8})\right] = 36 \times 2^{2/3} D/\sigma^2 ~.$$

综上,展开到二阶项是:
\begin{equation}\label{eq_LenJoP_1}
V(r) = 18 \times 2^{2/3} \frac{D}{\sigma^2} (r - \sqrt[6]{2} \sigma)^2 - D ~.
\end{equation}

这展开到二阶项、仅展开到二阶导数的情况又被称为\textbf{简谐近似}。
\subsection{简谐近似势的量子化}\label{sub_LenJoP_1}
惰性气体分子的振动能级可以模仿谐振子、根据\autoref{eq_LenJoP_1} 表示为:
\begin{equation}
E_\nu = \hbar \omega \left(\nu + \frac12\right) - D,  \ (\nu = 0, 1, 2, \cdots) ~.
\end{equation}
其中 $\nu$ 是振动量子数。而其中 $\omega$ 由经典简谐振动可知
\begin{equation}
\omega = \sqrt{\frac km} = \frac{6 \times \sqrt[3]{2}}{\sigma}\sqrt{\frac Dm} ~,
\end{equation} 
$m$ 是单原子的质量。

系统能级间隔是
\begin{equation}
\hbar \omega = \frac{6\times\sqrt[3]2 \hbar}{\sigma} \sqrt{D/m} ~.
\end{equation}
