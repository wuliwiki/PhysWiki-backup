% 整数
\pentry{逻辑量词, 朴素集合论\upref{NSet},二元关系\upref{Relat}}

整数的概念是大家所熟悉的.严格的数学中对于整数的定义过于抽象,主要是为了逻辑基础的严谨性;严格定义依然是建立在我们熟知的整数概念上的,所以在物理学习中没必要深入到整数的定义上,读者只需要按照通常的理解方式来认识整数就可以.为了简化表达,数学家通常把整数构成的集合(\textbf{整数集(Set of Integers)})简记为$\mathbb{Z}$.

整数通常涉及的运算有加法和乘法.加法的逆运算被称为减法,且减法在整数集上是封闭的,即两个整数相减仍为整数;乘法的逆运算被称为除法,除法在整数集上就不封闭,例如$\frac{2}{3}\not\in\mathbb{Z}$,尽管$2$和$3$都属于$\mathbb{Z}$. 

在中小学中我们学过带余除法,用现代数学语言可以记成如下形式:当被除数为$a$,除数为$b$,余数为$c$的时候,存在一个整数$k$,使得$a=kb+c$. 作为练习,用逻辑量词来表达时,这句话可以写为:当被除数为$a$,除数为$b$,余数为$c$的时候,$\exists k\in \mathbb{Z}$, s.t.$ a=kb+c$. 

将除数固定为$b$,那么如果两个整数$a_1$和$a_2$除以$b$所得到的余数相同,我们就称$a_1$和$a_2$\textbf{模}$b$\textbf{同余}($a_1\equiv a_2$ mod $b$).这里\textbf{模}的意思类似于“除以”,\textbf{同余}的意思是“余数相同”.在这种情况下,我们也把$a_1$和$a_2$称为彼此的\textbf{模}$b$\textbf{的同余数}. 

由词条\upref{Relat}中对\textbf{等价关系}的讨论可以推知,给定整数$b$以后,模$b$同余的关系是一个整数集上的等价关系.也就是说,我们可以利用$b$把整数集划分为若干个等价类,称作\textbf{整数集模}$b$\textbf{的同余类}.在词条\upref{Relat}中所讨论的“两数的差是3的倍数”这一关系,其实就是一个模3同余的关系.