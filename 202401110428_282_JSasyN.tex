% JavaScript Async 笔记
% license Usr
% type Note

\pentry{JavaScript 入门笔记\upref{JS}}

\subsection{Promise}
\begin{lstlisting}[language=js]
console.log("Before Promise");

let myPromise = new Promise((resolve, reject) => {
    console.log("Inside Promise executor");
    resolve("Operation successful");
});

myPromise.then(
    (value) => { console.log(value); },  // Executed asynchronously
    (error) => { console.log(error); }
);

console.log("After Promise");
\end{lstlisting}
输出:
\begin{lstlisting}[language=none]
Before Promise
Inside Promise executor
After Promise
Operation successful
\end{lstlisting}

\begin{itemize}
\item \verb`Promise` 是内建类型, 生成时会马上执行第一个 arg 提供的函数(叫做 Promise executor), \verb`resolve, reject` 函数也同样由系统提供,用于承诺完成或出错时告知系统。 在这个例子中, 在生成器中, \verb`resolve()` 马上被调用了。 但有时候也可以延迟调用, 例如:
\end{itemize}
\begin{lstlisting}[language=js]
let myPromise = new Promise((resolve, reject) => {
        setTimeout(() => {
            resolve(`Completed after ${milliseconds} milliseconds`);
        }, milliseconds);
    });
\end{lstlisting}
\begin{itemize}
\item 在该例中, 
\end{itemize}
