% 衍射(综述)
% license CCBYSA3
% type Wiki

本文根据 CC-BY-SA 协议转载翻译自维基百科\href{https://en.wikipedia.org/wiki/Diffraction}{相关文章}。

\begin{figure}[ht]
\centering
\includegraphics[width=8cm]{./figures/c528f7b4a151de90.png}
\caption{红色激光束通过另一个板上的小圆孔后投射到板上的衍射图案} \label{fig_YS_1}
\end{figure}
不要与折射混淆,折射是指波从一种介质传递到另一种介质时,方向发生的变化。

衍射是波由于障碍物或通过孔径而偏离直线传播的现象。衍射物体或孔径实际上成为了传播波的二次源。衍射与干涉是相同的物理效应,但干涉通常应用于少数波的叠加,而当许多波叠加时,通常使用“衍射”一词。[1]: 433 

意大利科学家弗朗切斯科·玛丽亚·格里马尔迪(Francesco Maria Grimaldi)创造了“衍射”这个词,并在1660年首次准确记录了这一现象。

在经典物理学中,衍射现象由**惠更斯–弗涅耳原理**描述,该原理将传播波前的每个点视为一组独立的球面波。衍射的特征性图案在当来自相干光源(如激光)的波遇到与其波长相当大小的狭缝/孔径时最为明显,如插图所示。这是由于波前上不同点(或等效地,每个波面波)的叠加或干涉,它们以不同的路径长度传播到接收表面。如果存在多个间距较近的开口,则可能会产生复杂的强度变化图案。

这些效应也发生在光波通过折射率变化的介质,或声波通过具有变化声阻的介质时——所有波都发生衍射,[包括引力波](#),水波,以及其他电磁波如X射线和无线电波。进一步来说,量子力学也表明物质具有波动性质,因此也会发生衍射(这一现象可以在亚原子到分子层面进行测量)。