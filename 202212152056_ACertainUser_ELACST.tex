% 本构关系(弹性力学)

\begin{issues}
\issueDraft
\end{issues}
%工程量似乎有点巨大...先占位

之前我们已经分别探讨了应力\upref{STRESS}与应变\upref{Strain}.现在我们要探讨力如何使材料变形,即应力与应变之间的联系.这就是材料的本构关系所描述的.

以下是各向同性的线弹性材料的本构关系.本构关系一共包括$6$个独立的方程.为了构建本构关系,需要知道材料的两个力学性质,例如杨氏模量$E$与泊松比$\nu$.

\subsection{应力-应变本构关系}
\begin{equation}\label{ELACST_eq1}
\sigma_{ij}=\frac{E}{1+\nu}\varepsilon_{ij}+\delta_{ij}\frac{\nu E}{(1+\nu)(1-2\nu)}\sum_k\varepsilon_{kk}\qquad i,j=1,2,3
\end{equation}
或
\begin{equation}
\varepsilon_{ij}=\frac{1+\nu}{E}\sigma_{ij}-\delta_{ij}\frac{\nu}{E}\sum_k\sigma_{kk} \qquad i,j=1,2,3
\end{equation}

其中$\delta_{ij} = \left \{
\begin{aligned}
1 &\qquad i = j\\
0 &\qquad i \ne j\\
\end{aligned}
\right.
$

\subsection{应力-位移本构关系}
如果往\autoref{ELACST_eq1} 带入位移几何方程\upref{Strain},那么本构关系