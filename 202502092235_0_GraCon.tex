% 图的连通性
% keys 连通
% license Usr
% type Tutor

\pentry{链、路、圈、回\nref{nod_PatCyc}}{nod_11cb}
\cite{graph2}图上两点的连通性是指有以这两点为端点的\enref{路}{PatCyc}存在,而连通图是指任意两点都连通的图。

\begin{definition}{连通}
设 $G$ 是图,$x,y\in V(G)$。若存在连接 $x,y$ 的路,则称 $x,y$ 是\textbf{连通的}(connected)。
\end{definition}

\begin{theorem}{}\label{the_GraCon_1}
连通关系是图上的\enref{等价关系}{Relat}。
\end{theorem}
\textbf{证明:}
1.\textbf{自反性:} 设 $x,x$ 连通,那么 $x,x$ 连通。

2.\textbf{对称性:} 设 $x,y$ 连通。于是存在路 $xe_1\cdots e_m y$,而 $ye_m\cdots e_1 x$ 显然也是路,所以 $y,x$ 连通。

3.\textbf{传递性:}设 $x,y$ 连通,$y,z$ 连通。于是存在路 $xe_1\cdots e_m y$ 和 $ye_{m+1}\cdots e_{n}z$。于是
\begin{equation}
xe_1\cdots e_m ye_{m+1}\cdots e_{n}z~
\end{equation}
是连接 $x,z$ 的链。由\autoref{the_PatCyc_1},存在连接 $x,z$ 的链,即 $x,z$ 连通。

\textbf{证毕!}

由于连通关系是 $V(G)$ 的等价关系,因此其可以将 $V(G)$ 分成不相交的等价类 $V_1,\cdots V_m$。$G$ 在每一类 $V_i$ 的导图子图 $G[V_i]$ 称为 $G$ 的一个\textbf{连通分支}, $V_i$ 的个数 $m$ 称为 $G$ 的\textbf{连通分支数}。其可以\aref{商集}{def_Relat_2}的概念进行如下严格定义。

\begin{definition}{连通分支}\label{def_GraCon_1}
设 $G$ 是图,$\overset{c}{\sim}$ 是 $V(G)$ 上的连通关系,$V_i\in V(G)/\overset{c}{\sim}$ 。则称\aref{导出子图}{def_Graph_3} $G[V_i]$ 为 $G$ 的\textbf{连通分支}(connected component)。商集 $V(G)/\overset{c}{\sim}$ 的基数称为 $G$ 的连通分支数(number of components),记作 $\omega(G)$。
\end{definition}

只有一个连通分支的图称为连通图。

\begin{definition}{连通图}
若 $\omega(G)=1$,则称 $G$ 是\textbf{连通图}(connected graph),否则称为\textbf{非连通图}(disconnected graph)。
\end{definition}

\subsection{有向图的连通性}

\begin{definition}{强连通}
设 $G$ 是有向图,$x,y\in V(G)$。若 $G$ 中既存在从 $x$ 到 $y$ 的路,又存在从 $y$ 到 $x$ 的路,则称 $x,y$ 是\textbf{强连通的}(strongly connected)。
\end{definition}

\begin{exercise}{}
试证明强连通关系是等价关系。(提示:仿照\autoref{the_GraCon_1} 的证明)。
\end{exercise}

\begin{definition}{强连通分支,强连通图}
设 $G$ 是有向图,$\overset{sc}{\sim}$ 是 $V(G)$ 上的强连通关系,$V_i\in V(G)/\overset{sc}{\sim}$ 。则称\aref{导出子图}{def_Graph_3} $G[V_i]$ 为 $G$ 的\textbf{强连通分支}。商集 $V(G)/\overset{sc}{\sim}$ 的基数称为 $G$ 的强连通分支数,记作 $\bvec{\omega}(G)$。若 $\bvec{\omega}(G)=1$,则称 $G$ 是\textbf{强连通图},否则称为\textbf{非强连通图}。
\end{definition}

\begin{definition}{单连通图}
设 $G$ 是有向图。若 $\forall x,y\in V(G)$,$G$ 中都存在一条连接 $x,y$ 的\aref{有向路}{def_PatCyc_1}。则称 $G$ 是\textbf{单连通的}(unilateral connected)。
\end{definition}

显然,强连通图一定是单连通图。

\begin{theorem}{}
设 $G$ 是单连通图,则 $G$ 中有一条包含所有顶点的有向链。
\end{theorem}









