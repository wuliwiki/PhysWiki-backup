% QED的重整化理论-顶点函数的单圈修正
% QED的费曼规则|顶点函数|形状因子

\pentry{QED的重整化理论-电子自能和光子自能的单圈修正\upref{qedlop}}

\subsection{顶点函数的洛伦兹结构}
类似电子自能和光子自能的单圈修正\upref{qedlop}中的讨论,我们先在 OS 重整化方案下进行讨论,用维数正规化处理发散积分。我们常常将 $i\mathcal{M}(e^-(p)\gamma(q)\rightarrow e^-(p'))=\bar u(p') [-ie\Gamma^\mu(p,p')]u(p) \epsilon_\mu(q)$ 称作 QED 的正规顶点,它对应于两条电子外线、一条光子外线的所有截肢费曼图的贡献之和。将 $[-ie\Gamma^\mu(p,p')]$ 称为 QED 的顶点函数,它仍保持着四矢
量的洛伦兹结构,且根据重整化条件,我们希望当电子光子外线在壳、且光子四动量为 $0$ 时,有
\begin{equation}
-ie\Gamma^\mu(p,p')|_{q=0,p^2={p'}^2=m^2} = -ie\gamma^\mu
\end{equation}
这表明在充分大的空间尺度上测得的电荷量为 $e$。

让我们来先分析 $\Gamma^\mu$ 的洛伦兹结构。由于它服从洛伦兹四矢量的变换规则,且是 $4\times 4$ 矩阵,它一定具有以下的形式:
\begin{equation}
\begin{aligned}
\Gamma^\mu=\gamma^\mu\cdot A + ({p'}^\mu + p^\mu)\cdot B + ({p'}^\mu - p^\mu) \cdot C
\end{aligned}
\end{equation}
其中 $A,B,C$ 是 Dirac 代数中的元素,可以表示为 $1,\gamma^\mu,\sigma^{\mu\nu}=\frac{i}{2}[\sigma^\mu,\sigma^\nu],\gamma^\mu\gamma^5,\gamma^5$ 这 $16$ 个元素的线性组合,且它们是洛伦兹标量。又因为在旋量 QED 中,任何阶 Feynman 图的任何量中都没有出现 $\gamma^5$,所以我们只有 $1,\not p^{(1)},p_\mu^{(1)} p_\nu^{(2)} \sigma^{\mu\nu}=\frac{i}{2}[\not p^{(1)},\not p^{(2)}]$ 这几种可能。$A,B,C$ 作为洛伦兹标量矩阵,可以由这组基底展开。

现在假设电子外线在壳,$p^2=p'^2=m^2$,那么可以利用 $\bar u(p')(\not p'-m)= (\not p-m)u(p)=0$,将 $A,B,C$ 中所有 $\not p,\not p'$ 都可以被替换为 $m$,如果多个 $\not p,\not p'$ 相乘也可以通过等式 $\not p\not p'=2p\cdot p'-\not p'\not p$ 的方式交换位置最后被替换为单位矩阵。因此最终 $A,B,C$ 可以表示成洛伦兹标量函数乘以单位矩阵的形式。由于外动量只有 $p,p',q=p'-p$,洛伦兹标量一定是关于独立标量 $p^2,{p'}^2,p\cdot p'$ 的函数。$p^2={p'}^2=m^2$ 是常数,$q^2=(p'-p)^2 = -2p'p$,因此 $A,B,C$ 都可以表示为 $q^2$ 的函数。

我们再来考察 Ward 等式(或者规范对称性)对顶点函数的限制。Ward 等式要求
\begin{equation}
\begin{aligned}
&\bar u(p')\Gamma^\mu u(p) (p'_\mu-p_\mu) = 0\\
&\Rightarrow \bar u(p')\left[A(q^2)(\not p'-\not p)+ C(q^2)(p'-p)^2\right] u(p)=\bar u(p')C(q^2)q^2 u(p)=0
\end{aligned}
\end{equation}
Ward 等式对于不在壳的虚光子也成立,这说明 $C(q^2) = 0$。综合前面的讨论,$\Gamma^\mu$ 可以表示成
\begin{equation}\label{qedvet_eq1}
\begin{aligned}
\Gamma^\mu = \gamma^\mu\cdot A(q^2) + ({p'}^\mu + p^\mu)\cdot B(q^2),\quad p^2=(p')^2=m^2
\end{aligned}
\end{equation}
或者我们也可以利用 Gordon 等式:
\begin{equation}
\bar u(p')\gamma^\mu u(p) = \bar u(p') \left[\frac{{p'}^\mu+p^\mu}{2m}+\frac{i\sigma^{\mu\nu}q_\nu}{2m}\right]u(p)=\bar u(p')\gamma^\mu u(p)
\end{equation}
可以将\autoref{qedvet_eq1} 改写为另一种形式
\begin{equation}
\Gamma^\mu = \gamma^\mu F_1(q^2) + \frac{i\sigma^{\mu\nu}q_\nu}{2m} F_2(q^2),\quad p^2=(p')^2=m^2
\end{equation}
$F_1(q^2) = A,F_2(q^2)=-2mB$ 被称为形状因子。