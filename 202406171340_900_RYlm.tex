% 实球谐函数
% license Xiao
% type Tutor

\pentry{球谐函数\nref{nod_SphHar}}{nod_5992}

\footnote{参考 Wikipedia \href{https://en.wikipedia.org/wiki/Spherical_harmonics}{相关页面}。}球谐函数 $Y_{l,m}(\uvec r)$ 中唯一的复数因子是 $\E^{\pm\I m\phi}$(\enref{见球谐函数表}{YlmTab}), 如果我们需要一套实数的球谐函数作为基底(例如用于展开实函数), 可以通过欧拉公式(\autoref{eq_CExp_2})把该因子变为 $\sin(m\phi)$ 和 $\cos(m\phi)$
\begin{equation}
\E^{\pm\I m \phi} = \cos(m\phi) \pm \I \sin(m\phi)~.
\end{equation}
定义\textbf{实球谐函数}为
\begin{equation}\label{eq_RYlm_2}
\mathcal Y_{l,m}(\uvec r) = \leftgroup{
&(-1)^{m}\sqrt{2} A_{l,m} P_l^m(\cos \theta) \cos(m\phi) &&(m > 0)\\
&A_{l,m} P_l^m(\cos \theta) &&(m = 0)\\
&(-1)^{m}\sqrt{2} A_{l,m} P_l^m(\cos \theta) \sin(m\phi) &&(m < 0)~.
}\end{equation}
其中 $P_l^m$ 是连带勒让德函数,归一化系数为
\begin{equation}
A_{l,m} =  \sqrt{\frac{2l + 1}{4\pi }\frac{(l - m)!}{(l + m)!}}~.
\end{equation}
与球谐函数的关系为
\begin{equation}\label{eq_RYlm_1}
\mathcal Y_{l,m}(\uvec r) = \leftgroup{
&\frac{1}{\sqrt{2}}[(-1)^{m} Y_{l,m}(\uvec r) + Y_{l,-m}(\uvec r)]  &&(m > 0)\\
&Y_{l,0}(\uvec r)  &&(m = 0)\\
&\frac{1}{\sqrt{2}\ \I}[(-1)^{m} Y_{l,-m}(\uvec r) - Y_{l,m}(\uvec r)]  &&(m < 0)~.
}\end{equation}
上式两个方括号中第一项 $\sim \E^{\I \phi}$, 第二项 $\sim \E^{-\I \phi}$, $(-1)^m$ 是为了抵消 Condon–Shortley 相位(\autoref{sub_SphHar_1})。 所以 $m > 0$ 时 $\mathcal Y_{l,m} \sim \cos(m\phi)$, $m < 0$ 时 $\mathcal Y_{l,m} \sim  \sin(m\phi)$, $m = 0$ 时与 $\phi$ 无关。

把\autoref{eq_RYlm_2} 与球谐函数的定义(\autoref{eq_SphHar_1})相比可知,在\enref{球谐函数表}{YlmTab}中,要把复球谐函数变为实球谐函数, 只需要把奇数 $m$ 前面的 $\mp$ 去掉, 再把 $\E^{\pm\I m\phi}$ 分别替换为 $\sqrt{2}\cos(m\phi)$ 和 $\sqrt{2}\sin(m\phi)$ 即可。 故我们不再重复给出 $\mathcal Y_{l,m}$ 列表。

由于不同的 $Y_{l,m}$ 是正交归一的, 所以\autoref{eq_RYlm_1} 把两个球谐函数相加后,需要在前面乘以 $1/\sqrt{2}$ 保持正交归一条件:
\begin{equation}
\int \mathcal Y_{l',m'}\Cj(\uvec r) \mathcal Y_{l,m}(\uvec r) \dd{\Omega} = \delta_{l,l'}\delta_{m,m'}~.
\end{equation}
根据\autoref{eq_RYlm_1}, 复球谐函数的许多其他性质都容易类推到实球谐函数, 这里不再赘述。
