% 微积分学(综述)
% license CCBY4
% type Wiki

本文根据 CC-BY-SA 协议转载翻译自维基百科\href{https://en.wikipedia.org/wiki/Calculus}{相关文章}。

微积分是研究连续变化的数学分支,正如几何学研究形状,代数学研究算术运算的推广一样。

微积分最初被称为“无穷小微积分”或“无穷小量的演算”,它有两个主要分支:微分学和积分学。微分学关注的是瞬时变化率和曲线的斜率,而积分学则研究量的累积以及曲线下方或两条曲线之间的面积。这两大分支通过微积分基本定理相互联系。它们都依赖于无穷数列和无穷级数收敛于确定极限的基本概念。\(^\text{[1]}\)微积分是处理变量随时间或其他参照变量变化问题的“数学支柱”。\(^\text{[2]}\)

无穷小微积分在17世纪末分别由艾萨克·牛顿和戈特弗里德·威廉·莱布尼茨独立创立。\(^\text{[3][4]}\)后来的工作,包括对极限概念的形式化,使这些发展建立在更坚实的概念基础之上。如今,微积分被广泛应用于科学、工程、生物学,甚至在社会科学和其他数学分支中也有重要应用。\(^\text{[5][6]}\)
\subsection{词源}
在数学教育中,微积分是无穷小微积分和积分微积分的缩写,指的是初等数学分析的课程。

在拉丁语中,"calculus" 意为“小卵石”(是 "calx" 的 diminutive 形式,意为“石头”),这个含义在医学中仍然存在。由于这些小卵石被用来计算距离、计票和进行算盘运算,\(^\text{[7]}\)这个词逐渐成为拉丁语中表示“计算”的词汇。在这个意义上,它至少在1672年就已经在英语中使用了,早于莱布尼茨和牛顿的出版物,他们的数学著作是用拉丁语写的。\(^\text{[8]}\)

除了微分学和积分学,"calculus" 这个词也被用来命名一些特定的计算方法或理论,这些方法或理论暗示某种形式的计算。这种用法的例子包括命题微积分、里奇微积分、变分法微积分、\(\lambda\)微积分、序列微积分和过程微积分。此外,“微积分”一词也在伦理学和哲学中得到了不同的应用,如边沁的幸福计算法和伦理计算法。
\subsection{历史}  
现代微积分是在17世纪的欧洲由艾萨克·牛顿和戈特弗里德·威廉·莱布尼茨独立发展起来的(他们几乎在同一时期首次发表)。但是微积分的元素最早出现在古埃及,随后是古希腊,接着是中国和中东地区,最后在中世纪的欧洲和印度再次出现。
\subsubsection{古代先驱}
\textbf{埃及}  

体积和面积的计算(这是积分微积分的一个目标)可以在公元前1820年左右的埃及《莫斯科纸草书》中找到,但这些公式只是简单的指示,并未说明它们是如何得出的。\(^\text{[9][10]}\)

\textbf{希腊}

