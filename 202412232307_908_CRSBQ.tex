% 查尔斯·巴贝奇(综述)
% license CCBYSA3
% type Wiki

本文根据 CC-BY-SA 协议转载翻译自维基百科\href{https://en.wikipedia.org/wiki/Charles_Babbage}{相关文章}。

\begin{figure}[ht]
\centering
\includegraphics[width=6cm]{./figures/2f69cba1b5206a32.png}
\caption{巴贝奇在1860年} \label{fig_CRSBQ_1}
\end{figure}
查尔斯·巴贝奇(Charles Babbage,1791年12月26日-1871年10月18日)是一位英国博学家。[1] 他是一位数学家、哲学家、发明家和机械工程师,巴贝奇提出了数字可编程计算机的概念。[2]

巴贝奇被一些人认为是“计算机之父”。[2][3][4][5] 他被认为发明了第一台机械计算机——差分机,这为更复杂的电子设计奠定了基础,尽管现代计算机的所有基本思想都可以在他的分析机中找到,该机器是通过一个明确借鉴自雅卡尔织机的原则来编程的。[2][6] 除了计算机相关工作外,巴贝奇在他1832年出版的《制造与机械经济学》一书中还涉及了广泛的兴趣领域。[7] 他是伦敦社交圈中的重要人物,并且以其举办的周六晚会而闻名,被认为将“科学晚会”从法国引入英国。[8][9] 他在其他领域的多样工作使他被描述为其世纪中“最杰出”的博学家之一。[1]

巴贝奇虽然未能完成许多设计的成功工程实现,包括他的差分机和分析机,但他在计算机理念的提出上依然是一个重要人物。他未完成的部分机械装置如今被展示在伦敦的科学博物馆。1991年,一台根据原始设计图纸构建的功能性差分机完成了建造。按照19世纪可以实现的公差制造,最终完成的差分机成功运转,证明了巴贝奇的机器本应能够正常工作。


巴贝奇的出生地存在争议,但根据《牛津国家传记词典》,他最有可能出生在英国伦敦沃尔沃思路的44号克罗斯比街。[10] 在拉科姆街与沃尔沃思路交汇处有一块蓝色纪念牌,纪念这一事件。[11]

在《泰晤士报》对巴贝奇的讣告中,他的出生日期为1792年12月26日;但随后一位侄子写信表示,巴贝奇出生在一年之前,即1791年。伦敦纽宁顿圣玛丽教区的注册簿显示巴贝奇于1792年1月6日接受洗礼,支持他出生于1791年的说法。[12][13][14]
\subsection{早年生活}
巴贝奇是本杰明·巴贝奇和贝齐·普拉姆利·蒂普的四个孩子之一。他的父亲本杰明·巴贝奇是伦敦弗利特街普雷德银行的创始人之一,该银行与威廉·普雷德合伙,于1801年创办。[15] 1808年,巴贝奇一家搬到了东泰恩茅斯的老罗登斯宅邸。大约在八岁时,巴贝奇因患上威胁生命的高烧,被送到埃克塞特附近的阿尔菲顿乡村学校就读。短时间内,他曾在南德文托特尼斯的国王爱德华六世文法学校就读,但由于健康原因,他不得不转回私人家教。[16]

随后,巴贝奇进入了位于米德尔塞克斯恩菲尔德贝克街的霍尔姆伍德学院,学习数学,该校由史蒂芬·弗里曼牧师主持。[17] 学院里有一个图书馆,这激发了巴贝奇对数学的热爱。离开学院后,巴贝奇又请了两位私人家教。第一位家教是一位来自剑桥附近的牧师;通过这位家教,巴贝奇结识了查尔斯·西门和他的福音派追随者,但家教的内容并不适合他。[18] 他被带回家,回到托特尼斯的学校继续学习,那个时候他大约16或17岁。[19] 第二位家教是一位牛津的导师,在他的指导下,巴贝奇掌握了足够的古典学知识,并被剑桥大学录取。
\subsection{在剑桥大学}
Babbage 于 1810 年 10 月到达剑桥大学三一学院。[20] 他在当时的数学某些领域已经是自学成才;[21] 他曾阅读过罗伯特·伍德豪斯、约瑟夫·路易·拉格朗日和玛丽亚·盖塔纳·阿涅西的著作。因此,他对大学提供的标准数学教学感到失望。[10]

1812 年,Babbage、约翰·赫歇尔、乔治·皮科克和几位朋友共同成立了分析学会;他们也与爱德华·赖安关系密切。[22] 作为学生,Babbage 还是其他社团的成员,如“幽灵俱乐部”,该社团致力于调查超自然现象,以及“解救者俱乐部”,该社团旨在将其成员从精神病院中解救出来,如果有人被送入精神病院的话。[23][24]

1812 年,Babbage 转学到剑桥大学的彼得豪斯学院。[20] 他是那里最顶尖的数学家,但未能以优异的成绩毕业。相反,他在 1814 年获得了无需考试的学位。他曾在初步的公开辩论中为一篇被认为是亵渎性的论文辩护,但目前尚不清楚这一事实是否与他没有参加考试有关。[10]
\subsection{剑桥大学之后} 
考虑到他的声誉,Babbage 很快取得了进展。他于 1815 年在皇家学会讲授天文学,并在 1816 年当选为皇家学会会员。[25] 然而,毕业后,他申请职位未果,事业发展有限。1816 年,他曾申请海利伯里学院的教职,获得了詹姆斯·艾弗里和约翰·普莱费尔的推荐信,但最终败给了亨利·沃尔特。[26] 1819 年,Babbage 和赫歇尔访问了巴黎和阿尔库伊学会,拜访了法国的著名数学家和物理学家。[27] 同年,Babbage 在皮埃尔·西蒙·拉普拉斯的推荐下申请了爱丁堡大学的教授职位,但该职位最终授予了威廉·沃拉斯。[28][29][30]

Babbage 与赫歇尔一起研究了阿拉戈旋转的电动力学,并于 1825 年发表了相关研究。他们的解释只是过渡性的,后来由迈克尔·法拉第接手并加以扩展。现今这些现象已成为涡电流理论的一部分,而 Babbage 和赫歇尔未能抓住统一电磁理论的某些线索,他们仍然局限于安培的力学定律。[31]

Babbage 购买了乔治·巴雷特的精算表,巴雷特于 1821 年去世,留下未出版的工作,并在 1826 年通过《各种人寿保险机构的比较视角》对该领域进行了调查。[32] 他对此兴趣的起因是一个筹建保险公司的项目,该项目受到弗朗西斯·贝利的推动,并于 1824 年提出,但未能付诸实施。[33] Babbage 确实为这一计划计算了精算表,使用了自 1762 年起的“公正公司”死亡率数据。[34]

在整个这一时期,Babbage 在父亲的支持下生活,尽管他的父亲对他 1814 年的早婚态度颇为尴尬:他与爱德华·赖安一起娶了惠特莫尔姐妹。他在伦敦的玛丽尔本建立了家庭,并养育了众多子女。[35] 1827 年父亲去世后,Babbage 继承了一笔大财产(大约 10 万英镑,相当于今天的 1090 万英镑或 1500 万美元),使他变得经济独立。[10] 由于妻子在同年去世,他开始了旅行生活。在意大利,他遇到了托斯卡纳的大公利奥波德二世,预示着他之后将访问皮埃蒙特。[25] 1828 年 4 月,他在罗马,依赖赫歇尔来管理差分机项目时,得知自己成为了剑桥大学的教授,而这一职位是他三次未能获得的职位(分别在 1820 年、1823 年和 1826 年)[36]。
\subsubsection{皇家天文学会}  
Babbage 在 1820 年为创立皇家天文学会发挥了重要作用,该会最初名为伦敦天文学会。[37] 它的最初目标是将天文学计算简化为更标准的形式,并传播数据。[38] 这些目标与 Babbage 关于计算的理念密切相关,1824 年他因此获得了该会的金奖勋章,表彰他“发明了一个用于计算数学和天文表格的引擎”。[39]

Babbage 希望通过机械化克服表格中的错误的动机,在 1834 年由狄俄尼修斯·拉德纳(Dionysius Lardner)在《爱丁堡评论》中提到时已成为常见话题(在 Babbage 的指导下)。[40][41] 这些发展的背景仍然存在争议。Babbage 自述的差分机起源故事始于天文学会希望改进《航海年鉴》一书的愿望。Babbage 和赫歇尔被要求监督一个试点项目,重新计算该表格的某部分数据。在拿到结果后,发现了不一致之处。这发生在 1821 或 1822 年,也是 Babbage 提出机械计算想法的契机。[42] 《航海年鉴》的问题如今被描述为英国科学中的一种分裂遗产,这种分裂源于对已故约瑟夫·班克斯爵士(Sir Joseph Banks)的态度,班克斯于 1820 年去世。[43]

Babbage 与他的朋友托马斯·弗雷德里克·科尔比(Thomas Frederick Colby)研究了建立现代邮政系统的需求,得出结论认为应该设立统一的邮资费率。这个想法最终在 1839 年和 1840 年通过推出统一的四便士邮政系统和后来的统一便士邮政系统得以实施。[44] 科尔比是该学会创始小组的成员之一。[45] 他还负责爱尔兰的测量工作。赫歇尔和 Babbage 也参与了该测量工作中一次著名的行动——重新测量福伊尔湖基线。[46]
\subsubsection{英国拉格朗日学派}
分析学会最初不过是一个本科生的挑衅。在这一时期,它取得了一些更为实质性的成就。1816 年,Babbage、赫歇尔和皮科克共同出版了从法语翻译过来的西尔维斯特·拉克鲁瓦(Sylvestre Lacroix)讲义,这本讲义当时是最先进的微积分教材。[47]

在微积分领域提到拉格朗日,标志着现在所称的正规幂级数的应用。英国数学家大约从 1730 年到 1760 年期间就已经使用了这些方法。重新引入后,它们不仅仅作为微分学中的符号表示被应用。它们开启了函数方程(包括差分机基础的差分方程)和微分方程的算子(D-模)方法的研究。差分方程和微分方程的类比通过符号变化将Δ变为D,将“有限”的差异变为“无穷小”。这些符号化的方向成为了运算微积分,并推动到收益递减的程度。科西的极限概念被排除在外。[48] 伍德豪斯早已建立了这个第二个“英国拉格朗日学派”,并将泰勒级数作为正规化处理。[49]

在这个背景下,函数组合的表达变得复杂,因为链式法则不仅仅适用于二阶及更高阶导数。伍德豪斯在 1803 年就已知道这个问题,他从路易·弗朗索瓦·安托万·阿尔博戈斯特那里得到了如今称为法·迪·布鲁诺公式的东西。事实上,这个方法早在 1697 年就已为亚伯拉罕·德·莫伊夫(Abraham De Moivre)所知。赫歇尔对这一方法印象深刻,Babbage 也知道这一方法,后来 Ada Lovelace 认为它与分析机相兼容。[50] 在 1820 年之前的这段时间,Babbage 集中精力研究一般的函数方程,并抵制传统的有限差分法和阿尔博戈斯特的方法(在该方法中,Δ 和 D 通过指数映射的简单加法关系联系)。但通过赫歇尔,他在迭代问题上受到了阿尔博戈斯特思想的影响,即将一个函数与自身组合,可能会多次组合。[49] 在他于《哲学会刊》(Philosophical Transactions)上发表的关于函数方程的重要论文中(1815/6),Babbage 说他的出发点是加斯帕尔·蒙日(Gaspard Monge)的工作。[51]
\subsection{学术生涯} 
从 1828 年到 1839 年,Babbage 担任剑桥大学卢卡斯数学教授。他不是一位传统的常驻教授,也不太关心教学责任,在这段时间里,他写了三本专题书籍。1832 年,他被选为美国艺术与科学学院的外籍荣誉会员。[52] Babbage 与同事们的关系并不融洽:乔治·比德尔·艾里(George Biddell Airy),他作为卢卡斯数学教授的前任,认为应该对 Babbage 不热衷讲授的态度做出反应。Babbage 曾计划在 1831 年讲授政治经济学。他的改革方向旨在使大学教育更加包容,大学在研究方面做得更多,课程大纲更加广泛,且更加关注应用;但威廉·惠威尔(William Whewell)认为这个计划不可接受。Babbage 与理查德·琼斯(Richard Jones)之间的争论持续了六年。[54] 他最终并未进行讲授。[55]

在这段期间,Babbage 尝试进入政治领域。西蒙·谢弗(Simon Schaffer)写道,他在 1830 年代的观点包括解散英国国教、扩大选举权和将制造商纳入利益相关者。[56] 他曾两次作为候选人竞选芬斯伯里区的国会议员。1832 年,他在五名候选人中排名第三,在双议员选区中落后约 500 票,原因是两位其他改革派候选人,托马斯·沃克利(Thomas Wakley)和克里斯托弗·坦普尔(Christopher Temple),分裂了选票。[57][58] 在回忆录中,Babbage 讲述了这次选举如何让他与塞缪尔·罗杰斯(Samuel Rogers)建立了友谊:他的兄弟亨利·罗杰斯原本打算再次支持 Babbage,但几天后去世。[59] 1834 年,Babbage 在四名候选人中排最后一位。[60][61][62] 1832 年,Babbage、赫歇尔和艾弗里被任命为皇家格尔夫勋章骑士,但他们随后并未被授予骑士爵士头衔,因此不能使用“Sir”这一前缀,尽管赫歇尔后来被封为男爵。[63]