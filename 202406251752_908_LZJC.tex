% 量子纠缠
% license CCBYSA3
% type Wiki

(本文根据 CC-BY-SA 协议转载自原搜狗科学百科对英文维基百科的翻译)

量子纠缠(quantum entanglement),也译作量子缠结,由爱因斯坦、波多尔斯基、罗森于1935年提出,量子纠缠描述了两个或多个互相纠缠的粒子之间的一种 “神秘”的关联,即使各自相隔距离很遥远,之间也没有任何介质,但是其中一个粒子的行为将会影响到另一个粒子的状态,假设其中的一个粒子被操作而自身的状态发生了变化, 其中的另外一个粒子也会发生相应的变化。是经典力学无法解释的。

量子纠缠被认为是量子形式论中最非经典的特征,在量子信息科学中起着至关重要的作用。

\begin{figure}[ht]
\centering
\includegraphics[width=8cm]{./figures/b460e7eb4f1bc7ee.png}
\caption{自发参量下转换过程可以将光子分成具有相互垂直偏振的II型光子对。} \label{fig_LZJC_1}
\end{figure}

\subsection{历史}
\begin{figure}[ht]
\centering
\includegraphics[width=6cm]{./figures/9e92a3c1b8c51728.png}
\caption{关于EPR论文的文章标题,发表于1935年5月4日的《纽约时报》。} \label{fig_LZJC_2}
\end{figure}
1935 年,爱因斯坦、波多尔斯基和罗森三人联名在《物理学评论》杂志上发表了标志着第三次论战 的重要檄文: “能认为量子力学对物理实在的描述是完备的吗?”,并且将这篇论文发表于5月份的《物理评论》。[1] 这是最早探讨量子力学理论对于强关联系统所做的反直觉预测的一篇论文。这篇檄文是从经典实在论的立场出发,来论证量子力学对实在的描述是不完备的。论文发表不久,薛定谔就写信给爱因斯坦说,读到 EPR 论文非常高兴,认为这篇文章抓住了教条的量子力学的辫子。爱因斯坦在回信中写道,“你是惟一一个 我愿意与之交换意见的人。其他的同行在看问题时几乎都不是从现象到理论,而是从理论到现象,他们无法从已接受的概念网中跳出来,而只是在里面奇怪地蹦来蹦去。”薛定谔在 EPR 论文的激发下,不到一个月的时间,就在德国《自然科学》杂志上发表了标题为“量子力学的现状”的文章,英译版发表在《美国哲学学会进展》杂志。这篇文章的目标是基于对经典观念 与量子观念的比较,进一步从理论上加深对量子力学深层问题的理解。在德文版的“量子力学的现状”一文发表后不久,薛定谔越来越意识到,在量子测量中,“纠缠”概念 很重要,是量子力学的特征性质。不久之后,薛定谔发表了一篇重要论文,对于“量子纠缠”这术语给予定义,并且研究探索相关概念。薛定谔体会到这概念的重要性,他表明,量子纠缠不只是量子力学的某个很有意思的性质,而是量子力学的特征性质;量子纠缠在量子力学与经典思路之间做了一个完全切割。于是,1935 年 10 月,他又在《剑桥哲学学会的数学进展》杂志上发表 了一篇文章。这篇文章是用英文发表的,标题为“对分离系统之间的概率关系的讨论”。在这篇文章 中,薛定谔继续推广 EPR 论文的讨论,第一次明确地用“纠缠”概念来描述 EPR 思想实验中两个曾经耦 合的粒子,分开之后彼此之间仍然维持某种关联的现象,或者说,用“量子纠缠”这一概念来描述复合的 微观粒子系统存在的那种难以理解的特殊关联。薛定谔在“对分离系统之间的概率关系的讨论”一文中开门见山地指出,当两个系统由于受外力作 用,在经过暂时的物理相互作用之后,再彼此分开时,我们无法再用它们相互作用之前各自具有的表达 式来描述复合系统的态,两个量子态通过相互作用之后,已经纠缠在一起。不管这两个量子系统分离 之后相距多远,都始终会神秘地联系在一起,其中一方发生变化,都会立即引发另一方产生相应的变化。 薛定谔对这种特殊情境的另一种表达方式是: 一个整体的最有可能的知识不一定是它的所有部分的最 有可能的知识,即使这些部分可能是完全分离的,有能力拥有各自的“最有可能的认识”。这种知识的 缺乏决不是由于这种相互作用是不能够被认识的,而是由于这种相互作用本身。可见,薛定谔提出量 子纠缠概念是为了描述量子测量的不确定性,并不是为了突出意识对测量的决定作用。如同爱因斯坦一样,薛定谔对于量子纠缠的概念并不满意,因为量子纠缠似乎违反在相对论中对于信息传递所设定的速度极限。[2] 爱因斯坦后来对量子纠缠给出了著名的嘲笑:“spukhafte Fernwirkung”,[3] 即“鬼魅般的超距作用”。

       EPR论文很显然地引起了众多物理学者的兴趣,启发他们探讨量子力学的基础理论。但是除了这方面以外,物理学者认为这论题与现代量子力学并没有什么牵扯,在之后很长一段时间,物理学术界并没有特别重视这论题,也没有发现EPR论文可能有什么重大瑕疵。EPR论文试图建立定域性隐变量理论来替代量子力学理论。1964年,约翰·贝尔提出论文表明,对于EPR思想实验,量子力学的预测明显地不同于定域性隐变量理论。概略而言,假若测量两个粒子分别沿着不同轴向的自旋,则量子力学得到的统计关联性结果比定域性隐变量理论要强很多,贝尔不等式定性地给出这差别,做实验应该可以侦测出这差别。因此,物理学者做了很多检试贝尔不等式的实验。

       1972年,约翰·克劳泽与史达特·弗利曼(Stuart Freedman)首先完成这种检试实验。1982年,阿兰·阿斯佩的博士论文是以这种检试实验为题目。他们得到的实验结果符合量子力学的预测,不符合定域性隐变量理论的预测,因此证实定域性隐变量理论不成立。但是,每一个相关实验都存在有漏洞,这造成了实验的正确性遭到质疑,在作总结之前,还需要完成更多精确的实验。[4] 

       这些年来,众多研究结果促成了应用这些超强关联来传递信息的可能性,从而导致了量子密码学的成功发展,最著名的有查理斯·贝内特(Charles Bennett)与吉勒·布拉萨(Gilles Brassard)发明的BB84协议、阿图尔·艾克特(Artur Eckert)发明的E91协议。

       2005年, 中国科学技术大学潘建伟、彭承志等研究人员的小组在合肥创造了13公里的自由空间双向量子纠缠“拆分”、发送的世界纪录,同时验证了在外层空间与地球之间分发纠缠光子的可行性。

       2007年开始,中国科大——清华大学联合研究小组在北京架设了长达16公里的自由空间量子信道,并取得了一系列关键技术突破,最终在2009年成功实现了世界上最远距离的量子态隐形传输,证实了量子态隐形传输穿越大气层的可行性,为未来基于卫星中继的全球化量子通信网奠定了可靠基础。该成果已经发表在2010年6月1日出版的英国《自然》杂志子刊《自然·光子学》上,并引起了广泛关注。

       2017年6月16日,于2016年8月16日1时40分在酒泉用长征二号丁运载火箭成功发射升空量子卫星墨子号首先成功实现两个量子纠缠光子被分发到相距超过1200公里的距离后,仍可继续保持其量子纠缠的状态。
\begin{figure}[ht]
\centering
\includegraphics[width=6cm]{./figures/fa60b510f27d4052.png}
\caption{首张量子纠缠图像} \label{fig_LZJC_3}
\end{figure}
       2018年4月25日,芬兰阿尔托大学教授麦卡﹒习岚帕(Mika Sillanpää)领导的实验团队成功地量子纠缠了两个独自震动的鼓膜,这实验演示出宏观的量子纠缠。每个鼓膜的宽度只有15微米,约为头发的宽度,是由10个金属铝原子制成。通过超导微波电路,在接近绝对温度(-273.15摄氏度)下,两个鼓膜持续进行了约30分钟的互动。
       
\subsection{概念}
\subsubsection{2.1 纠缠的意义}
纠缠系统被定义为其量子态不能作为其局部成分的态的乘积来考虑的系统;也就是说,它们不是若干个独立的粒子,而是一个不可分割的整体。在纠缠中,系统的一个组成部分不能在不考虑其他部分的情况下被完全描述。一个复合系统的状态总是可以被表示为局部成分的状态的乘积的和(或者称为叠加);如果这个叠加必有超过一个的项,那么这个状态就是纠缠的。

       量子系统可以通过各种类型的相互作用纠缠在一起。对于一些可以达到实验目的的纠缠方式,请参见下面关于方法的章节。当被纠缠的粒子通过与环境的相互作用而退相干时,纠缠便被打破;例如,这可以发生在进行测量的时候。[5]

       纠缠的一个例子是:亚原子粒子衰变为其他粒子的纠缠对。衰变事件遵循各种守恒定律,因此,对一个子粒子的测量结果必定与对另一个子粒子的测量结果高度相关(从而使得总动量、总角动量、总能量等在这个过程前后大致保持不变)。例如,一个自旋为零的粒子可以衰变为一对自旋为1/2的粒子。由于衰变前后的总自旋必须为零(角动量守恒),所以当第一个粒子在某个轴上被测量为自旋向上时,另一个粒子在同一轴上总会被测量到自旋向下。(这被称为自旋反相关情况;如果测量到每个自旋的先验概率相等,则称这一对粒子出于单重态)。

如果我们把这两个粒子分开,就能更好地观察到纠缠的特殊性质。让我们把其中一个放在华盛顿的白宫,另一个放在白金汉宫(把这当成一个思想实验而不是一个真实的实验)。现在,如果我们测量其中一个粒子的某个特定特征(例如自旋)并得到一个结果,然后使用相同的标准去测量另一个粒子(沿着相同轴的自旋),我们发现第二个粒子的测量结果将与第一个粒子的测量结果(在互补的意义上)相匹配,因为它们的值将是相反的。

       上述结果可能让人感到惊讶,也可能不让人感到惊讶。基于经典力学和量子力学中的角动量守恒,经典系统将显示相同的性质,并且肯定需要一个隐变量理论来达到这个结果。区别在于,经典系统对所有可观察量都有明确的值,而量子系统没有。在下面将要讨论的意义上,这里考虑的量子系统似乎获得了在对第一个粒子进行了测量的情况下沿着另一个粒子的任何轴的自旋的测量结果的概率分布。这种概率分布通常不同于不测量第一个粒子的情况。对于空间上相互分离的纠缠粒子,这个结果肯定会被认为是令人惊讶的。

\subsubsection{2.2 悖论}