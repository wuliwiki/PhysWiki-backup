% 欧姆定律(综述)
% license CCBYSA3
% type Wiki

本文根据 CC-BY-SA 协议转载翻译自维基百科\href{https://en.wikipedia.org/wiki/Ohm\%27s_law}{相关文章}。

\begin{figure}[ht]
\centering
\includegraphics[width=6cm]{./figures/d4ea765673f5dcae.png}
\caption{欧姆定律的参数:V、I 和 R} \label{fig_OuMu_1}
\end{figure}
欧姆定律指出,通过导体的电流与其两端的电压成正比。引入比例常数“电阻”后,[1] 可以得到描述这种关系的三个数学公式:[2]  
\[
V = IR \quad \text{或} \quad I = \frac{V}{R} \quad \text{或} \quad R = \frac{V}{I}~
\]
其中,\(I\) 是通过导体的电流,\(V\) 是测量的导体两端的电压,\(R\) 是导体的电阻。更具体地说,欧姆定律指出,这个关系中的 \(R\) 是常数,独立于电流。[3] 如果电阻不是常数,那么之前的公式就不能称为欧姆定律,但它仍然可以作为静态/直流电阻的定义。[4] 欧姆定律是一个经验关系,准确描述了绝大多数电导材料在多个数量级的电流下的导电性。然而,一些材料不遵守欧姆定律,这些材料被称为非欧姆材料。

该定律以德国物理学家 Georg Ohm 的名字命名,他在 1827 年发表的一篇论文中,描述了通过包含不同长度电线的简单电路的施加电压和电流的测量。欧姆用一个比现代形式稍微复杂的方程来解释他的实验结果(参见下文的“历史”部分)。

在物理学中,欧姆定律一词也用于指代该定律的各种推广;例如,在电磁学和材料科学中使用的欧姆定律的矢量形式:
\[
\mathbf{J} = \sigma \mathbf{E},~
\]
其中 \( \mathbf{J} \) 是在某一位置的电流密度,\( \mathbf{E} \) 是该位置的电场,\( \sigma \)(希腊字母西格玛)是一个依赖于材料的参数,称为导电率,定义为电阻率 \( \rho \)(希腊字母罗) 的倒数。这个欧姆定律的重新表述是由古斯塔夫·基尔霍夫(Gustav Kirchhoff)提出的。[5]
\subsection{历史}
在1781年,乔治·欧姆的工作之前,亨利·卡文迪许进行了实验,使用了不同直径和长度的莱顿瓶和玻璃管,填充了盐溶液。他通过注意完成电路时身体感受到的电击强度来测量电流。卡文迪许写道,“速度”(电流)与“电气化程度”(电压)成正比。当时,他没有将结果与其他科学家分享,[6] 直到詹姆斯·克拉克·麦克斯韦在1879年发布了这些结果。[7]

弗朗西斯·罗纳兹在1814年使用金叶电计,划分了“强度”(电压)和“量”(电流)在干电堆(一种高电压源)中的关系。他发现,在某些气象条件下,干电堆中两者之间的关系并不是成正比的。[8][9]

欧姆于1825年和1826年进行电阻的研究,并于1827年将其结果出版为《Die galvanische Kette, mathematisch bearbeitet》(《电化学电路的数学分析》)。[10] 他在理论解释中受到了约瑟夫·傅里叶在热传导方面研究的很大启发。在实验中,欧姆最初使用了伏打电堆,但后来改用热电偶,因为热电偶在内阻和恒定电压方面提供了更稳定的电压源。他使用了电流计来测量电流,并知道热电偶接点之间的电压与接点温度成正比。然后,他加入了不同长度、直径和材料的测试导线以完成电路。他发现他的数据可以通过以下方程建模:
\[ x = \frac{a}{b + \ell}~\]
其中 \( x \) 是电流计的读数,\( \ell \) 是测试导体的长度,\( a \) 取决于热电偶接点温度,\( b \) 是整个系统的常数。由此,欧姆得出了他的比例定律并公布了他的结果。