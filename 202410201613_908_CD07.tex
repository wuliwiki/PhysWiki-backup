% 四川大学 2007 年硕士物理考试试题(933)
% license Usr
% type Tutor

\textbf{声明}:“该内容来源于网络公开资料,不保证真实性,如有侵权请联系管理员”

\begin{enumerate}
\item (各专业考生必作)(10分)\\
为什么在无电荷的空间里电场线不能相交?为什么静电场中的电场线不可能是闭合曲线?
\item (各专业考生必作)(8分)\\
用金属丝绕成的标准电阻要求无自感,怎样绕制才能达到这一要求?为什么?
\item (各专业考生必作)(5分)\\
光由密介质向疏介质入射时,其布儒斯特角能否大于全反射的临界角?
\item ( 理论物理、粒子物理与原子核物理、原子分子物理专业考生必作)
(5分)\\
光谱仪的后透镜焦面上获得的是时间频谱还是空间频谱?
\item ( 凝聚态物理、光学、生物医学物理、应用电子技术专业考生必作)
(5分)\\
用眼睛通过一单狭缝直接观察远处与缝平行的光源,看到的射图样是菲涅耳衍射图样还是夫琅禾费衍射图样?为什么?
\item (各专业考生、必作)'(8分)\\
有哪些方法可以使一束线偏振光的振动面旋转90°?(至少给出三种方法)
\item 各专业考生必作)(8分)\\
一天文望远镜的物镜直径为 $2.5m$,试求能被它分辨的双星对它所张的最小夹角(设人眼光波长为 $550om$)。设人眼瞳孔直径为$2.3mm$,求望远镜与人眼相比其分辨能力是人眼的多少倍?
\end{enumerate}
\subsection{(凝聚志物理、光学、生物医学物理、应用电子技术专业考生必作)(12分)}
如图所示,长直导线和矩形线圈共面,$AB$ 边与导线平行,$a=1cm,b=8cm,l=30.cm$.
\begin{enumerate}
\item 若直导线中的电流$i$在$1s$内均匀地从$10A$降为零,则线圈 $ABCD$ 中的感应电动势的大小和方向如何?
\item 长直导线和线圈的互感系数$M=?$
\end{enumerate}
\begin{figure}[ht]
\centering
\includegraphics[width=6cm]{./figures/1f277c6139bbf8bf.png}
\caption{} \label{fig_CD07_1}
\end{figure}
\subsection{(理论物理、粒子物理与原子核物理、原子分子物理专业必作)(12分)}
如图所示,将一无限大均匀载流平面放入均匀磁场中,(设均匀磁场方向沿$0x$轴正方向)且其电流方向与磁场方向垂直指向纸内,已知放入后平面两侧的总磁感强度分别为$\vec{B_1}$与$\vec{B_2}$·

求:该载流平面上单位面积所受的磁场力的大小及方向?
\begin{figure}[ht]
\centering
\includegraphics[width=6cm]{./figures/937fafd16c5c3ef2.png}
\caption{} \label{fig_CD07_2}
\end{figure}
\subsection{(各专业考生必作)(12分)}
\begin{enumerate}
\item 迈克尔孙干涉仪中的一块动镜移动$0.273mm$时,能数到移动1000条条纹。该光波长为多少?(6分)
\item 若该迈克尔孙干涉仪中的一动镜面以$\upsilon=14\mu m/s$的速度匀速推移,保证同样的入射光波长,而用透镜接收干涉条纹,并将它会聚到光电原件上把光信号转化为电信号。该电信号的时间频率是多少?(6分)
\end{enumerate}
\subsection{各专业考生必作)(10分)}
在双缝于涉实验中,波长$\lambda=550nm$ 的单色平行光垂直入射到缝间距$a=2\times10^{-4}m$的双缝上,屏到双缝的距离$D=2m$.求:
