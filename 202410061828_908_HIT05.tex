% 哈尔滨工业大学 2005 年 考研 量子力学
% license Usr
% type Note

\textbf{声明}:“该内容来源于网络公开资料,不保证真实性,如有侵权请联系管理员”

\subsection{一、(50 分)完成下列问题}
\begin{enumerate}
\item 试说明量子力学与经典力学的差异与关系。
\item 简述量子力学的基本原理。
\item 何为运动粒子的波粒两象性,写出德布罗意关系的表达式。
\item 分别说明什么样的状态是定态、束缚态、简并态与负字称态?
\item 证明坐标$x$与动量算符$\hat{p}_x$,的对易关系$[x,\hat{p}_x]=i\hbar$,并给出两者满足的不确定关系式。
\item 在坐标表象中,分别给出坐标算符$\hat{\vec{r}}$,与动量算符$\hat{\vec{p}}$的本征值和本征波函数。
\item 
\item 在量子力学中,一个可观测量与一个什么样的算符相对应?为什么?该算符的本征值与本征波函数分别具有什么性质。
\item 设体系由两个自旋皆为$\frac{\hbar}{2}$一的非全同粒子构成,分别在非耦合表象与耦合表象下写出该体系与自旋相关的波函数,并给出两者之间的关系。
\item 对于一维运动而言,试说明如下两种情况的本征解的异同\\
(1)处于宽度为$2l$的无限深对称方势阱中粒子;\\
(2)在$-l\leq x \leq1$区间做“箱归一化”的自由粒子。
\end{enumerate}
\subsection{二、(20 分)}
已知$t=0$时粒子处于状态
$$\psi(x,0) = \frac{1}{3} \varphi_2(x) 
\begin{pmatrix}
1 \\\\
0
\end{pmatrix}
-\frac{2}{3} \varphi_1(x) 
\begin{pmatrix}
0 \\\\
1
\end{pmatrix}
+ \frac{\sqrt{2}}{3} \varphi_3(x)
\begin{pmatrix}
1 \\\\
0
\end{pmatrix}~$$