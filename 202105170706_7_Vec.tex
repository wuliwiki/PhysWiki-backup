% 切向量场
% keys 光滑函数|流形|切向量|切空间|tangent space|tangent vector|方向导数|李括号|Lie bracket
\pentry{流形上的切空间\upref{tgSpa}}

\subsection{光滑切向量场的定义}

我们知道,集合上的一个实函数可以理解为给集合中的每个点都分配了一个实数的结果.流形上的一个实函数,有时也被称为“标量场”.类似地,如果给流形上每个点都分配一个该点的切向量,这种分配就被称为流形上的一个\textbf{切向量场(tangent vector field)}.更进一步,流形上每个点分配一个该点上的张量,就得到流形上的一个\textbf{张量场(tensor field)};切向量场和函数都是张量场的特例.


在流形$M$上某一点$p$附近取一个图$(U, \varphi)$,那么切空间$T_pM$中的每一个向量$\bvec{v}$都唯一对应一个坐标,即$\varphi(\bvec{v})$在$\mathbb{R}^n$中的坐标,其中$n=\opn{dim}M$是$M$的维度.

仅仅依靠“图”这一概念,足够我们定义什么是光滑切向量场.这一点很神奇,因为我们甚至还无法讨论怎么在流形上给切向量求导\footnote{流形上给切向量求(方向)导数的概念,见后续的\textbf{仿射联络(流形)}\upref{affcon}词条.},就已经能够定义流形上的光滑切向量场了,也就是“可以任意求导的切向量场”.

\begin{definition}{光滑切向量场}
给定实流形$M$,令$X:M\to TM$为$M$上的一个映射,其中对于任意$p\in M$,有$X(p)\in T_pM$.称$X$为$M$上的一个\textbf{切向量场(tangent vector field)},有时也直接将切向量简称为向量.如果对于任意$p\in M$,存在一个包含$p$的图$(U_p, \varphi_p)$,使得$\X\circ\varphi_p^{-1}$
\end{definition}





