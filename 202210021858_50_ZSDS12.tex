% 浙江理工大学 2012 年数据结构
% 浙江理工大学 2012 年数据结构

\subsection{一、 单选题(每题2分 ,共20分)}

1.不带头结点的单链表simpleList为空的判定条件是 \\
A. simpleList==null \\
B. simpleList->next==null \\
C. simpleList->next=simpleList \\
D. simpleList!=null

2.某线性表最常用的操作是在最后一个结点之后插入一个结点或删除第一一个结点 ,故采用(  )存储方式最节省运算时间. \\
A.单链表 \\
B.仅有头结点的单循环链表 \\
C.双链表 \\
D.仅有尾指针的单循环链表

3.向一个栈顶指针为top的链栈中插入一个$S$所指结点时,则执行(    ). \\
A. top->next=S; \\
B. S->next= top-> next;top-> next=S; \\
C. S->next=top;top=S; \\
D. S-> next=top;top=top->next;

4. 一维数组和线性表的区别是 \\
A.前者长度固定,后者长度可变 \\
B.后者长度固定,前者长度可变 \\
C.两者长度均固定 \\
D.两者长度均可变

5. 设矩阵A是-一个对称矩阵, 为了节省存储,将其下三角部分按行序存放在一维数组B[1,n(n-1)/2]中 , 对任一下三角部分中任-元素a:(i≥j),在一-组数组B的下标位置K的值是 \\
A. i(i-1)/2+j-1 \\
B. i(i-1)/2+j \\
C. i(i+1)/2+j-1 \\
D. i(i+1)/2+j

6. 在线索化二叉树中, $P$所指的结点没有左子树的充要条件是 \\
A. P->left==null \\
B. P->ltag=1 \\
C. P->Itag==1且P->left==null \\
D. 以上都不对

7. 如果Tree2是由有序树Tree1转换而来的二 叉树,那么Tree1中结点的后序就是Tree2中结点的____. \\
A.先序 $\qquad$ B.中序 $\qquad$ C.后序 $\qquad$ D.层次序

8. 判定一个有向图上是否存在回路除了可以利用拓扑排序方法外,还可以用(    )  \\
A.求关键路径的方法 \\
B.求最短路径的Dijkstra方法 \\
C.广度优先遍历算法 \\
D.深度优先遍历算法

