% 新基础集合论(综述)
% license CCBYSA3
% type Wiki

本文根据 CC-BY-SA 协议转载翻译自维基百科\href{https://en.wikipedia.org/wiki/New_Foundations}{相关文章}。

在数学逻辑中,新基础(New Foundations,简称 NF)是一种非良基、可有限公理化的集合论,由威拉德·范·奥曼·奎因构思,旨在简化《数学原理》中的类型论。
\subsection{定义}
NF 的良构公式是命题演算的标准公式,具有两个基本谓词:相等(\(=\))和成员关系(\(\in\))。NF 可以仅通过两个公理模式来表述:

\begin{itemize}
\item 外延性:具有相同元素的两个对象是相同的对象。形式化地说,给定任意集合 \( A \) 和任意集合 \( B \),如果对于任意集合 \( X \),\( X \) 是 \( A \) 的成员当且仅当 \( X \) 是 \( B \) 的成员,则 \( A \) 等于 \( B \)。
\item 受限的理解公理模式:对于每个分层公式\( \phi \),集合 \( \{x \mid \phi\} \) 存在。
\end{itemize}
一个公式 \( \phi \) 被称为分层的,如果存在一个从 \( \phi \) 的语法结构的各部分到自然数的函数 \( f \),使得:对于 \( \phi \) 中的任意原子子公式 \( x \in y \),满足 \( f(y) = f(x) + 1 \);对于 \( \phi \) 中的任意原子子公式 \( x = y \),满足 \( f(x) = f(y) \)。
\subsubsection{有限公理化}
NF 可以被有限公理化。[1] 这种有限公理化的一个优点是,它消除了分层性的概念。有限公理化中的公理对应于一些自然的基本构造,而分层理解公理虽然强大,但不一定直观。在其入门书籍中,Holmes 选择将有限公理化作为基本框架,并将分层理解公理作为一个定理来证明。[2] 

具体的公理集合可能有所不同,但通常包含以下大部分公理,而其余的可以作为定理证明:[3][1]
\begin{itemize}
\item 外延性:如果 \( A \) 和 \( B \) 是集合,并且对于每个对象 \( x \),\( x \) 是 \( A \) 的元素当且仅当 \( x \) 是 \( B \) 的元素,则 \( A = B \)。[4] 这一公理也可以视为对相等符号的定义。[5][6]
\item 单元素集:对于每个对象 \( x \),集合 \( \iota(x) = \{x\} = \{y \mid y = x\} \) 存在,并称为 \( x \) 的单元素集。[7][8]
\item 笛卡尔积:对于任意集合 \( A \) 和 \( B \),集合\(A \times B = \{(a, b) \mid a \in A \text{ 且 } b \in B\}\)称为 \( A \) 和 \( B \) 的笛卡尔积,并且它的存在性被保证。[9] 该公理可以限制为某个特定的交叉积,例如 \( A \times V \) 或 \( V \times B \) 的存在。[10][11]
\item 逆关系:对于每个关系 \( R \),集合\(R^{-1} = \{(x, y) \mid (y, x) \in R\}\)存在;可以观察到,\( x R^{-1} y \) 当且仅当 \( y R x \)。[12][13][14]
\item 单元素映像(Singleton Image)**:对于任意关系\( R \),集合\(R\iota = \{(\{x\}, \{y\}) \mid (x, y) \in R\}\)存在,并称为\( R \)的单元素映像。[15][16][17]
\item 定义域:如果 \( R \) 是一个关系,则集合\[\text{dom}(R) = \{x \mid \exists y . (x, y) \in R\}\]存在,并称为 \( R \) 的定义域。[12] 这一公理可以通过类型降维操作来定义。[18]

\item 包含关系(Inclusion)**:集合
   \[
   [\subseteq] = \{(x, y) \mid x \subseteq y\}
   \]
   存在。[19] 等价地,我们可以考虑集合
   \[
   [\in] = [\subseteq] \cap (1 \times V) = \{(\{x\}, y) \mid x \in y\}
   \]
   的存在性。[20][21]

8. **补集(Complement)**:对于每个集合 \( A \),其补集
   \[
   A^c = \{x \mid x \notin A\}
   \]
   存在。[22]

9. **(布尔)并集((Boolean) Union)**:如果 \( A \) 和 \( B \) 是集合,则它们的并集
   \[
   A \cup B = \{x \mid x \in A \text{ 或 } x \in B \text{ 或两者皆是}\}
   \]
   存在。[23]
\end{itemize}