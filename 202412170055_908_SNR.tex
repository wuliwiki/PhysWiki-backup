% 斯涅尔定律(综述)
% license CCBYSA3
% type Wiki

本文根据 CC-BY-SA 协议转载翻译自维基百科\href{https://en.wikipedia.org/wiki/Maxwell\%27s_equations}{相关文章}。

\begin{figure}[ht]
\centering
\includegraphics[width=6cm]{./figures/2be493390070c758.png}
\caption{光在两种折射率不同的介质界面处的折射示意图,其中 \( n_2 > n_1 \)。由于光在第二介质中的速度较低(\( v_2 < v_1 \)),折射角 \( \theta_2 \) 小于入射角 \( \theta_1 \);也就是说,光线在折射率较高的介质中更接近法线。} \label{fig_SNR_1}
\end{figure}

斯涅尔定律(也称为斯涅尔-笛卡尔定律、伊本·萨尔定律[^1],或折射定律)是一种描述入射角和折射角之间关系的公式,适用于光或其他波在两种不同各向同性介质(如水、玻璃或空气)的边界处传播时的情况。在光学中,该定律用于光线追踪以计算入射角或折射角,也用于实验光学中测定材料的折射率。该定律在具有负折射率的超材料中同样适用,这些材料允许光以“向后”折射的方式弯曲,形成负折射角。

定律指出,对于一对给定的介质,入射角(\( \theta_1 \))的正弦与折射角(\( \theta_2 \))的正弦之比,等于第二介质相对于第一介质的折射率(\( n_{21} \)),也等于两介质的折射率之比(\( \frac{n_2}{n_1} \)),或者等价于两介质中相位速度之比(\( \frac{v_1}{v_2} \))[^2]。
\[
\frac{\sin \theta_1}{\sin \theta_2} = n_{21} = \frac{n_2}{n_1} = \frac{v_1}{v_2}~
\]
斯涅尔定律可以从费马最短时间原理推导而来,而费马原理本身是基于光作为波传播的特性得出的。
\subsection{历史}
托勒密在埃及亚历山大的研究中发现了关于折射角的某种关系,但对于较大的角度来说,这一关系并不准确。托勒密确信自己找到了一个精确的经验定律,这部分是因为他对数据进行了轻微的修改以符合理论(参见:确认偏误)。