% 密度矩阵
% keys 量子力学|纯态|混合态|密度矩阵|测量
% license Xiao
% type Tutor

\pentry{矩阵的迹\upref{trace}, 投影算符\upref{projOp}, 量子力学的基本假设\upref{QMPos}}

\footnote{参考 Shankar, Principles of Quantum Mechanics 2ed, 以及 Wikipedia}若一个系综中的 $N$ 个系统中, 有 $n_i$ ($i = 1,2,\dots,k$) 个在状态 $\ket{i}$ (这里假设 $\ket{i}$ 是正交归一的)。 那么这个系综可以用\textbf{密度矩阵(density matrix)}(或算符)描述
\begin{equation}\label{eq_denMat_1}
\rho = \sum_i p_i\ket{i}\bra{i}   ~.
\end{equation}
其中 $p_i = n_i/N$ 是随机选一个系统, 处于状态 $\ket{i}$ 的概率。 若所有系统都处于同一个 $\ket{i}$, 那么这个系综就是\textbf{纯的(pure)}, 否则就是\textbf{混合的(mixed)}。

\subsection{非纯态的等效}
纯态 $\ket{\psi}$ 可以唯一地表示为密度矩阵 $\rho = \ket{\psi}\bra{\psi}$, 它的意义也相当明确。

然而对于非纯态, 我们可以使用不同的正交归一的(纯态)基底的概率组合来表示, 测量结果却是一样的。 因为测量并不能区分量子概率和经典概率。

例如有一束电子。 若这束电子是纯态的, 我们就可以通过多次测量得到这个纯态(除了一个总体相位因子)。 例如通过测量 $\ket{y+}$ 和 $\ket{y-}$ 的概率, 我们可以确定 $\ket{x+} + c\ket{x-}$ 中的复数 $c$。

但若这束电子的自旋方向是随机的, 我们既可将其等效为随机的一半 $\ket{x+}$ 和 $\ket{x-}$ 构成的, 也可以等效为随机的一半 $\ket{y+}$ 和 $\ket{y-}$ 构成的。 虽然 “实际” 上它们是不一样的, 但任何测量都无法区分这两种情况。 所以密度矩阵可以是 $\rho = (\ket{x+}\bra{x+} + \ket{x-}\bra{x-})/2$ 也可以是 $\rho = (\ket{y+}\bra{y+} + \ket{y-}\bra{y-})/2$。 事实上, 它们都是单位矩阵的一半。 注意如果概率不是均分的, 如 $\rho = (\ket{x+}\bra{x+}/3 + 2\ket{x-}\bra{x-})/3$ 就无法用 $\ket{y\pm}$ 基底表示(一般的对角矩阵经过相似变换后不是对角矩阵)。

纯态和非纯态都可以对应一个唯一密度矩阵。

\subsection{测量}
对于某个物理量对应的算符 $\Omega$, 它的\textbf{系综平均值(ensemble average)}为
\begin{equation}
\ev{\bar\Omega} = \sum_i p_i \mel{i}{\Omega}{i}~.
\end{equation}
这个平均值既包含了每个 $\ket{i}$ 的平均, 又包含了对每个系统的平均。

系综平均也可以用迹表示为 $\opn{tr}(\Omega\rho)$。 根据迹的定义,
\begin{equation}\label{eq_denMat_3}
\opn{tr}(\Omega\rho) = \sum_j \mel{j}{\Omega\rho}{j} = \sum_{i,j} p_i\mel{j}{\Omega}{i} \braket{i}{j} = \sum_{i} p_i\mel{i}{\Omega}{i} = \ev{\bar\Omega}~.
\end{equation}
证毕。

对于纯态, 获得测量值 $\omega$ 的概率可以看作投影算符 $\mathbb P_\omega$ 的平均值(满足 $\mathbb P_\omega^2 = \mathbb P_\omega$)
\begin{equation}
P(\omega) = \abs{\braket{\omega}{\psi}}^2 = \braket{\mathbb P_\omega \psi}{\mathbb P_\omega \psi} = \mel{\psi}{\mathbb P_\omega}{\psi}~,
\end{equation}
所以对于混合态, 测量值 $\omega$ 的概率为
\begin{equation}
\overline{P(\omega)} = \opn{tr}(\mathbb P_\omega\rho)~.
\end{equation}

\subsection{密度矩阵的性质}
\pentry{正定矩阵\upref{DefMat}}
\begin{itemize}
\item 密度矩阵算符是厄米算符(自伴算符)

\item 密度矩阵的迹为 1

\item 密度矩阵是正定矩阵。可以从\autoref{eq_denMat_1} 看到,对任意态 $\ket\varphi$,总是有 $\bra\varphi \rho \ket\varphi\ge 0$
\end{itemize}

密度矩阵厄米正定的性质使得我们总是将它在一组正交完备基 $\ket{\psi_1},\ket{\psi_2},\cdots$ 下对角化:
\begin{equation}
\rho = \sum_i \ket{\psi_i} p_{\psi_i}\bra{\psi_i}~.
\end{equation}
这表明系统处于 $\ket{\psi_i}$ 状态的概率为 $p_{\psi_i}$。

现在假设系综中每一个态都随时间发生演化,那么在薛定谔表象下
\begin{equation}
\ket{\psi_i(t)}= e^{-iHt}\ket{\psi_i(t)}~.
\end{equation}
密度算符随时间的演化为
\begin{equation}\label{eq_denMat_2}
\rho(t)= \sum_i p_{\psi_i}\ket{\psi_i(t)}\bra{\psi_i(t)}= e^{-iHt}\rho(0) e^{iHt}~.
\end{equation}

\subsection{密度矩阵矩阵元}

回顾在基矢量$\left\{ \ket{u_n}\right\}$下,密度矩阵$\rho$的表达形式:
$$\rho = \sum\limits_k p_k\rho_k = \sum\limits_{m,n}\rho_{mn}\ket{u_m}\bra{u_n} ~,$$
上式中$\rho_k = \ket{\psi_k}\bra{\psi_k}$,$\rho_{m,n}$是密度矩阵的矩阵元。

我们首先考虑对角元,即$\rho_{nn}$,显然有
$$\rho_{nn} = \sum\limits_{k} p_k \left[\rho_k\right]_{nn}~.$$
其中$\left[\rho_k\right]_{nn}$表示密度矩阵$\rho_k = \ket{\psi_k}\bra{\psi_k}$的第$n$个对角元。

假设:
$$\ket{\psi_k} = \sum\limits_n C_n^{k}\ket{u_n} = \sum\limits_n\ket{u_n}\braket{u_n}{\psi_k}~,$$
那么有:
$$\left[\rho_k\right]_{nn} = \left|\braket{u_n}{\psi_k}\right|^2 = \left| C_n^{k}\right|^2~.$$
则可以得到$\rho_{nn} = \sum\limits_k p_k \left|C_n^k\right|^2$,由此可以看到密度矩阵的对角元应当是一个非负实数。

现在考虑其物理意义,可以发现计算式中$\left| C_n^k \right|^2$表示的是量子态$\ket{\psi_k}$,在沿着基$\left\{\ket{u_n}\right\}$进行测量时,塌缩到$\ket{u_n}$的概率,经过经典概率$p_k$的叠加,$\rho_{nn}$表示的实际上是密度矩阵在沿着基$\left\{\ket{u_n}\right\}$测量时得到态$\ket{u_n}$的概率,因此我们称对角元


\subsection{统计系综与平衡态}
从 \autoref{eq_denMat_2} 中可以看到,如果密度算符与哈密顿量 $H$ 对易,那么 $\rho(t)=\rho(0)$,密度算符不随时间发生变化,系综中每个状态发生的概率不变,根据\autoref{eq_denMat_3} ,各个物理量的系综平均值也是不变的。系统处于平衡态。

因此统计系综处于平衡态等价于 $[\rho,H]=0$。由于两个算符对易,它们实际上可以被同时对角化。存在一组基底 $\ket{1},\cdots,\ket{n}$,满足 $\rho \ket{i} = p_i \ket{i}$,且 $H\ket{i}=E_i \ket{i}$。也就是说密度矩阵在哈密顿量的一组正交的本征态基底下可对角化:
\begin{equation}
\rho = \sum_i p_i \ket{i}\bra{i},\quad H\ket{i}=E_i \ket{i}~.
\end{equation}

