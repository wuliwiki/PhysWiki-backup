% 协变和逆变向量基

\pentry{正交归一基底\upref{OrNrB}}

在有内积的概念之前,线性空间的结构高度对称,任意线性无关的向量组都可以当一组基,而且各种基之间没有本质区别.但是有了内积的概念之后,一些特殊的基比其它的基更方便表达线性空间,这就是\textbf{正交归一基底},有时也称\textbf{标准正交基}或\textbf{单位正交基}.

但有的时候,更方便描述物理现象基也许不是标准正交基,这样一来就不得不牺牲分量描述的简洁性.比如说,如果取二维内积空间的一组基$\{\bvec{e}_1, \bvec{e}_2\}$,它不是标准正交的,那么任意向量$\bvec{v}=a\bvec{e}_1+b\bvec{e}_2$的分量就并不是$\bvec{v}\cdot\bvec{e}_1$和$\bvec{v}\cdot\bvec{e}_2$了.这个时候算出任意向量的分量麻烦了许多,因此我们引入新的方法来简化这一过程.

\begin{definition}{协变和逆变基向量、}
设线性空间$V$有一组基$\{\bvec{e}_i\}_{i=1}^n$,称其为一组\textbf{协变基(covariance base)},它对应一组\textbf{逆变基(contravariance base)}$\{\bvec{e}^i\}_{i=1}^n$,其中$\bvec{e}_i\cdot\bvec{e}^j=\delta_{ij}$.
\end{definition}


