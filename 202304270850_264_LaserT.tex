% 激光原理
% keys 光学|现代光学|原子物理

\begin{issues}
\issueDraft
\issueTODO
\end{issues}

激光器是现代光学的伟大成就之一,其拥有的窄频宽、单模式等高相干性的优势是许多近代实验,如迈克尔逊-莫雷的实验,成功的必要因素之一。激光器的核心原理是受激辐射。

\subsection{能级和光量子的概念}
量子力学告诉我们,在原子中的电子的能量并不是连续的。电子的能量总是一个又一个特定的能级中跳变。例如氢原子的能级分布就如下图:\begin{figure}[ht]
\centering
\includegraphics[width=5cm]{./figures/97c81e92cc27a211.png}
\caption{氢原子能级} \label{fig_LaserT_1}
\end{figure}

\subsection{受激辐射}
\subsection{谐振腔}
