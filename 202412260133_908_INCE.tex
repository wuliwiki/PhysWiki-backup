% 信息科学(综述)
% license CCBYSA3
% type Wiki

本文根据 CC-BY-SA 协议转载翻译自维基百科\href{https://en.wikipedia.org/wiki/Information_science}{相关文章}。

\begin{figure}[ht]
\centering
\includegraphics[width=10cm]{./figures/15c7d956dfa4ab1f.png}
\caption{从元数据领域获取洞察的各种方法论方法的可视化} \label{fig_INCE_1}
\end{figure}
信息科学[1][2][3] 是一门主要关注信息的分析、收集、分类、处理、存储、检索、传输、传播和保护的学术领域。[4] 该领域内外的从业者不仅研究知识在组织中的应用与使用,还研究人与组织以及任何现有信息系统之间的互动,旨在创建、替代、改进或理解信息系统。

从历史上看,信息科学与信息学、计算机科学、数据科学、心理学、技术、文献学、图书馆学、医疗保健和情报机构等领域有关联。[5] 然而,信息科学也包括了诸如档案学、认知科学、商业、法律、语言学、博物馆学、管理学、数学、哲学、公共政策和社会科学等多种领域的内容。
\subsection{基础} 
\subsubsection{范围与方法} 
信息科学专注于从利益相关者的角度理解问题,然后根据需要应用信息和其他技术。换句话说,它首先解决的是系统性问题,而不是系统内部的单个技术组件。在这一点上,可以将信息科学视为对技术决定论的回应,技术决定论认为技术“按照自身的规律发展,发挥自身的潜力,仅受到可用物质资源和开发者创造力的限制。因此,它必须被视为一个自治系统,控制并最终渗透社会的所有其他子系统。”[6]

许多大学拥有专门研究信息科学的学院、部门或学校,而众多信息科学学者则在传播学、医疗保健、计算机科学、法学和社会学等学科中工作。一些机构已成立信息学院联盟(参见信息学院列表),除此之外,许多其他机构也具有全面的信息研究方向。

在信息科学领域,2013年时的当前问题包括:
\begin{itemize}
\item 科学中的人机交互
\item 协同软件
\item 语义网
\item 价值敏感设计
\item 迭代设计过程
\item 人们生成、使用和寻找信息的方式
\end{itemize}
\subsubsection{定义}  
“信息科学”这一术语的首次已知使用是在1955年。[7] 信息科学的早期定义(追溯到1968年,美国文献学会将其名称更改为美国信息科学与技术学会的那一年)指出:

信息科学是研究信息的属性和行为、控制信息流动的力量,以及处理信息以实现最佳可访问性和可用性的方法的学科。它关注与信息的生成、收集、组织、存储、检索、解释、传输、转化和利用相关的知识体系。这包括自然和人工系统中信息表现的真实性,使用编码进行高效信息传递,以及研究信息处理设备和技术,如计算机及其编程系统。它是一门跨学科的科学,源自并与数学、逻辑学、语言学、心理学、计算机技术、运筹学、图形艺术、通讯、管理以及其他类似领域相关。它既有纯科学的成分,探索这一主题而不考虑其应用,也有应用科学的成分,开发服务和产品。(Borko 1968,第3页)[8]

\textbf{相关术语}

一些作者将信息学(Informatics)视为信息科学的同义词。这种说法尤其在与A. I. 米哈伊洛夫(A. I. Mikhailov)及其他苏联作者在1960年代中期提出的概念相关时尤为常见。米哈伊洛夫学派将信息学视为与科学信息研究相关的学科。[9] 由于信息学领域快速发展且跨学科的特点,使得信息学的定义很难精确定义。依赖于用于从数据中提取有意义信息的工具特性的定义,正在信息学的学术项目中出现。[10]

区域差异和国际术语使得这个问题更加复杂。有些人[哪一些?]指出,今天所称为“信息学”的许多内容曾经被称为“信息科学”——至少在医学信息学等领域中是这样。例如,当图书馆学家开始使用“信息科学”一词来指代他们的工作时,“信息学”这一术语便应运而生:
\begin{itemize}
\item 在美国,信息学作为计算机科学家为区分自己与图书馆学的工作所做的回应;
\item 在英国,信息学作为研究自然信息处理系统以及人工或工程化信息处理系统的科学术语。
\end{itemize}
另一个被讨论为“信息研究”的同义词的术语是“信息系统”。Brian Campbell Vickery的《信息系统》(1973)将信息系统纳入信息科学(IS)范畴。[11] 另一方面,Ellis、Allen和Wilson(1999)提供了一项文献计量学调查,描述了“信息科学”和“信息系统”这两个不同领域之间的关系。[12]
\subsubsection{信息哲学}   
信息哲学研究心理学、计算机科学、信息技术和哲学交汇处出现的概念性问题。它包括对信息的概念性质和基本原理的研究,包括信息的动态、利用及其科学,以及将信息理论和计算方法应用于其哲学问题的阐述与应用。[13]
\subsubsection{本体论}   
在科学和信息科学中,本体论正式地表示一个领域内的知识作为一组概念及这些概念之间的关系。它可以用来推理该领域中的实体,并且可以用来描述该领域。

更具体地说,本体论是描述世界的模型,由一组类型、属性和关系类型构成。围绕这些内容提供的具体内容有所不同,但它们是本体论的基本要素。通常,本体论的模型与现实世界之间应该有着密切的相似性。[14]

理论上,本体论是“对共享概念化的正式、明确规范”。[15] 本体论提供了共享的词汇和分类法,用来通过定义对象和/或概念及其属性和关系来建模某一领域。[16]

本体论是组织信息的结构框架,并在人工智能、语义网、系统工程、软件工程、生物医学信息学、图书馆学、企业书签和信息架构等领域中作为一种知识表示形式,用于表示世界或某一部分世界的知识。领域本体论的创建对于定义和使用企业架构框架也至关重要。
\subsubsection{科学还是学科?}  
像Ingwersen[17]这样的作者认为,信息学面临与其他学科界定自身边界的问题。根据波普尔(Popper)所说,“信息科学忙于处理一个包含大量常识性实际应用的海洋,这些应用越来越涉及计算机……而计算机科学在语言、沟通、知识和信息等常识性看法上几乎没有更好的状态。”[18] 其他作者,如Furner,否认信息科学是一门真正的科学。[19]
\subsection{职业}  
\subsubsection{信息科学家}  
信息科学家是指具有相关学科学位或高水平学科知识的个人,他们为工业界的科学和技术研究人员提供专门的信息,或者为学术界的学科教师和学生提供信息。工业界的信息专家/科学家和学术界的学科信息专家/图书馆员通常有相似的学科背景培训,但学术职位持有者通常需要拥有第二个高级学位(如信息与图书馆学的MLS/MI/MA学位),除了学科硕士学位外。这个职位也适用于从事信息科学研究的个人。
\subsubsection{系统分析师}  
系统分析师负责创建、设计和改进为特定需求提供的信息系统。系统分析师通常与一个或多个企业合作,评估和实施组织流程和技术,以改善组织内的信息访问,从而提高效率和生产力。
\subsubsection{信息专业人士}    
信息专业人士是指那些负责保存、组织和传播信息的个人。信息专业人士擅长组织和检索记录的知识。传统上,他们的工作涉及印刷材料,但这些技能现在越来越多地用于电子、视觉、音频和数字材料。信息专业人士在各种公共、私营、非营利和学术机构中工作,也可以在组织和工业环境中找到。他们的角色包括系统设计和开发、系统分析等。
\subsection{历史}  
\subsubsection{早期起源}
\begin{figure}[ht]
\centering
\includegraphics[width=6cm]{./figures/d3b8a79f2fbc4764.png}
\caption{戈特弗里德·威廉·莱布尼茨,一位德国博学家,主要用拉丁语和法语写作。他的研究领域包括形而上学、数学和神义学。} \label{fig_INCE_2}
\end{figure}
信息科学在研究信息的收集、分类、处理、存储、检索和传播时,其根源可以追溯到人类知识的共同积累。信息分析至少可以追溯到亚述帝国时期,当时文化存储库的出现,也就是今天所称的图书馆和档案馆。[20] 从机构上看,信息科学在19世纪与许多其他社会科学学科一同兴起。然而,作为一门科学,它的制度化根源可以追溯到科学史,始于1665年皇家学会(伦敦)出版的《哲学会刊》的第一期,这通常被认为是世界上第一本科学期刊。

科学的制度化过程贯穿了18世纪。1731年,本杰明·富兰克林建立了费城图书公司,这是由一群公民共同拥有的第一座图书馆,它很快超越了图书的范畴,成为了科学实验的中心,并举办了公开的科学实验展示。[21] 本杰明·富兰克林还将一批书籍捐赠给马萨诸塞州的一个小镇,镇上投票决定免费向所有人开放这些书籍,从而成立了美国第一座公共图书馆。[22] 1736年,巴黎的外科学院(Academie de Chirurgia)出版了《外科医师纪要》,被普遍认为是第一本医学期刊。模仿伦敦皇家学会的模式,美国哲学学会于1743年在费城成立。随着许多其他科学期刊和学会的成立,阿洛伊斯·塞内费尔德于1796年在德国开发了石版印刷的概念,用于大规模印刷工作。
\subsubsection{19世纪}
\begin{figure}[ht]
\centering
\includegraphics[width=6cm]{./figures/4bd649af023629f6.png}
\caption{约瑟夫·玛丽·雅卡尔} \label{fig_INCE_3}
\end{figure}
到19世纪,信息科学的雏形开始显现,成为与其他科学和社会科学分开且独立的学科,但同时与通信和计算相结合。1801年,约瑟夫·玛丽·雅卡尔发明了一种穿孔卡片系统,用于控制法国织布机的操作。这是首次使用“图案的记忆存储”系统。[23] 随着化学期刊在1820年代和1830年代的出现,[24] 查尔斯·巴贝奇于1822年开发了“差分机”,这是现代计算机的第一步,并于1834年完成了“分析机”。到1843年,理查德·霍伊发明了旋转印刷机,1844年,塞缪尔·莫尔斯发出了第一条公共电报消息。1848年,威廉·F·普尔开始编写《期刊文献索引》,这是美国第一本通用的期刊文献索引。

1854年,乔治·布尔出版了《思维法则研究...》,奠定了布尔代数的基础,后者在信息检索中得到了应用。[25] 1860年,在卡尔斯鲁厄技术高等学院举行了一次大会,讨论建立化学学科的系统和理性命名法的可行性。虽然大会没有得出结论,但几位关键参与者带回了斯坦尼斯劳·卡尼查罗(1858年)的提纲,最终说服他们接受他关于计算原子质量的方案。[26]

到1865年,史密森学会开始编制当前的科学论文目录,1902年该目录成为《国际科学论文目录》。[27] 次年,皇家学会开始在伦敦出版《论文目录》。1868年,克里斯托弗·肖尔斯、卡洛斯·格里登和S·W·苏尔共同发明了第一台实用的打字机。到1872年,开尔文勋爵设计了一种模拟计算机,用于预测潮汐,1875年,弗兰克·斯蒂芬·鲍德温获得了美国第一个实用计算机器的专利,该机器能执行四种算术运算。[24] 1876年和1877年,亚历山大·格雷厄姆·贝尔和托马斯·爱迪生分别发明了电话和留声机,美国图书馆协会在费城成立。1879年,《医学索引》首次由美国陆军外科总长图书馆出版,约翰·肖·比林斯担任馆长,后来该图书馆发行《索引目录》,并以其作为最完整的医学文献目录而享有国际声誉。[28]
\subsubsection{欧洲文献学}  
文献学这一学科,标志着现代信息科学的最早理论基础,出现在19世纪末的欧洲,并伴随着一些旨在组织学术文献的科学索引的出现。许多信息科学历史学家认为保罗·奥特莱(Paul Otlet)和亨利·拉·冯丹(Henri La Fontaine)是信息科学的奠基人,他们于1895年创立了国际文献学研究所(IIB)。第二代欧洲文献学家在二战后出现,其中最著名的是苏珊娜·布里耶(Suzanne Briet)。然而,“信息科学”这一术语直到20世纪后期才在学术界广泛使用。

文献学家强调将技术与技巧结合起来,服务于特定的社会目标。根据罗纳德·戴(Ronald Day)的说法,“作为一套有组织的技术和方法体系,文献学被理解为现代全球组织历史发展的一个重要组成部分——实际上是一个主要的组成部分,因为这个组织依赖于信息的组织与传递。”奥特莱和拉·冯丹(他在1913年获得诺贝尔奖)不仅预见了后来的技术创新,还为信息及信息技术的全球视野提供了前瞻性思考,这一视野直接影响了战后关于“信息社会”的全球设想。奥特莱和拉·冯丹建立了多个致力于标准化、书目学、国际协会和国际合作的组织,这些组织在确保国际商业、信息、通信和现代经济发展的国际生产方面发挥了基础性作用,后来它们的全球形式在国际联盟和联合国等机构中找到了体现。奥特莱设计了“国际十进制分类法”,这一分类法基于梅尔维尔·杜威(Melville Dewey)的十进制分类系统。

尽管奥特莱生活的年代远早于计算机和网络的出现,但他所讨论的内容预示着最终形成了万维网。他对知识大网络的设想聚焦于文献,并包含了超链接、搜索引擎、远程访问和社交网络等概念。

奥特莱不仅设想了所有世界知识应当相互连接,并远程提供给任何人,而且他还实际构建了一个结构化的文献集合。这个集合包括标准化的纸张和卡片,按照分层索引存放在定制设计的文件柜中(该索引从全球多种来源提取信息),并提供商业信息检索服务(通过复制相关的索引卡片信息来回答书面请求)。如果用户的查询结果可能超过50个,服务方甚至会提前提醒用户。到1937年,文献学正式得到制度化,体现之一就是美国文献学研究所(ADI)的成立,后者后来被更名为美国信息科学与技术学会(ASIST)。
\subsubsection{过渡到现代信息科学}
\begin{figure}[ht]
\centering
\includegraphics[width=6cm]{./figures/63654c07cba29630.png}
\caption{范尼瓦尔·布什(Vannevar Bush),著名信息科学家,大约在1940年至1944年间} \label{fig_INCE_4}
\end{figure}
随着1950年代的到来,人们越来越意识到自动化设备在文献检索和信息存储与检索方面的潜力。  
随着这些概念的规模和潜力不断增长,信息科学的兴趣领域也日益广泛。到了1960年代和1970年代,信息科学发生了从批处理到在线模式的转变,从大型计算机到小型计算机和微型计算机的变化。此外,学科之间的传统界限开始模糊,许多信息科学学者与其他学科的项目结合起来。他们通过将自然科学、人文学科、社会科学以及法律和医学等其他专业课程纳入到他们的课程中,进一步实现了跨学科发展。

在这些促进跨学科活动、促进科学交流的个人中,**福斯特·E·莫赫哈特(Foster E. Mohrhardt)尤为突出,他在1954年至1968年期间担任美国国家农业图书馆馆长。[32]

到了1980年代,像美国国立医学图书馆的“Grateful Med”大型数据库,以及如Dialog和Compuserve等面向用户的服务,第一次可以通过个人计算机进行访问。1980年代还出现了许多专门的兴趣小组,旨在应对这些变化。到这一年代末,涉及非印刷媒体、社会科学、能源与环境以及社区信息系统等领域的兴趣小组已经成立。今天,信息科学主要研究在线数据库的技术基础、社会影响和理论理解,广泛数据库在政府、工业和教育中的应用,以及互联网和万维网的发展。[33]
\subsection{21世纪的信息传播}
\subsubsection{定义的变化} 
信息传播历来被解读为单向的信息传递。随着互联网的出现,以及在线社区的流行,社交媒体在许多方面改变了信息传播的格局,创造了新的传播方式和信息类型,[34]从而改变了传播定义的解读。社交网络的性质使得信息的扩散速度比通过传统组织来源更为迅速。[35]互联网改变了我们查看、使用、创造和存储信息的方式;现在是重新评估我们如何共享和传播信息的时机。
\subsubsection{社交媒体对人们和行业的影响}
社交媒体网络为那些时间有限或无法接触传统信息传播渠道的大众提供了一个开放的信息环境,[35]这是一个“日益移动化和社交化的世界,[它]要求...新型的信息技能”。[34]社交媒体作为接入点的整合,对用户和提供者来说是一个非常有用且互利的工具。所有主要的新闻提供者都通过Facebook和Twitter等网络获得可见性和接入点,从而最大化他们的观众范围。通过社交媒体,人们可以通过认识的人被引导或提供信息。能够“分享、点赞和评论...内容”[36]使得信息的传播范围比传统方式更广、更远。人们喜欢与信息互动,他们喜欢将自己认识的人纳入自己的知识圈。社交媒体的分享变得如此有影响力,以至于出版商如果想要成功,必须“与社交媒体合作”。尽管对于出版商和Facebook来说,"分享、推广和发现新内容"通常是互利的,能够提升双方的用户体验。公众意见的影响力可以以难以想象的方式传播。社交媒体通过简单易学和易接触的工具促进互动;《华尔街日报》通过Facebook提供了一个应用程序,《华盛顿邮报》更进一步,提供了一个独立的社交应用程序,该应用在六个月内下载量达到了1950万次,[36]证明了人们对这种新的信息获取方式的高度兴趣。
\subsubsection{社交媒体在推动话题方面的力量} 
通过社交媒体维持的连接和网络帮助信息提供者了解人们关注的重要事项。人们在全球范围内建立的联系使得信息交换的速度达到了前所未有的水平。正因为如此,这些网络被意识到具有巨大的潜力。“大多数新闻媒体监控Twitter以获取突发新闻”[35],新闻主播也常常要求观众发布事件的图片。[36]分享信息的用户和观众已经获得了“意见领袖和议程设定的力量”[35]。这种渠道被认为是提供基于公众需求的目标信息的有用工具。
\subsection{研究领域和应用}
以下是信息科学研究和发展的部分领域。
\subsubsection{信息访问}  
信息访问是一个交叉领域,涉及信息学、信息科学、信息安全、语言技术和计算机科学。信息访问研究的目标是自动化处理大量繁杂的信息,并简化用户对信息的访问。如何分配权限并限制未授权用户的访问呢?访问的范围应根据为信息授予的权限级别来定义。相关技术包括信息检索、文本挖掘、文本编辑、机器翻译和文本分类。在讨论中,信息访问通常被定义为确保信息的自由访问与封闭访问或公共访问之间的平衡,并且常常在关于版权、专利法和公共领域的讨论中被提及。公共图书馆需要资源以提供信息保障的知识。
\subsubsection{信息架构}  
信息架构(IA)是组织和标记网站、内联网、在线社区和软件以支持可用性的艺术与科学。它是一个新兴的学科和实践社区,专注于将设计和建筑的原则应用到数字环境中。通常,信息架构涉及一种信息模型或概念,用于活动中,需要明确复杂信息系统的细节。这些活动包括图书馆系统和数据库开发。
\subsubsection{信息管理} 
信息管理(IM)是从一个或多个来源收集和管理信息,并将这些信息分发给一个或多个受众。这有时涉及到那些有利益关系或对这些信息有权的人。管理意味着对信息的结构、处理和传递的组织与控制。自1970年代以来,这主要局限于文件、文件维护以及基于纸质文件、其他媒体和记录的生命周期管理。随着1970年代信息技术的普及,信息管理的工作获得了新的意义,并开始包括数据维护领域。
\subsubsection{信息检索}  
信息检索(IR)是与搜索文档、文档中的信息以及文档的元数据相关的研究领域,同时也涉及搜索结构化存储、关系型数据库和万维网。自动化信息检索系统用于减少所谓的“信息过载”。许多大学和公共图书馆使用信息检索系统提供书籍、期刊和其他文档的访问。网络搜索引擎是最常见的信息检索应用。

信息检索过程开始时,用户向系统输入查询。查询是信息需求的正式陈述,例如在网络搜索引擎中的搜索字符串。在信息检索中,一个查询并不唯一标识集合中的单一对象。相反,多个对象可能匹配查询,可能具有不同的相关度。

对象是数据库中由信息表示的实体。用户查询与数据库信息进行匹配。根据应用的不同,数据对象可以是文本文档、图像、音频、思维导图或视频等。通常,文档本身不会直接存储在信息检索系统中,而是通过文档替代物或元数据在系统中表示。

大多数信息检索系统会计算一个数值评分,表示每个对象与查询的匹配度,并根据该值对对象进行排名。排名最高的对象会展示给用户。如果用户希望进一步细化查询,则该过程可以重复进行。

\subsubsection{信息寻求}  
信息寻求是试图获取信息的过程或活动,既包括人类环境,也包括技术环境。信息寻求与信息检索(IR)相关,但有所不同。

许多图书馆和信息科学(LIS)研究集中在各个专业领域实践者的信息寻求行为上。已有关于图书馆员、学者、医疗专业人员、工程师和律师等群体信息寻求行为的研究。许多研究借鉴了Leckie、Pettigrew(现为Fisher)和Sylvain的工作,他们在1996年进行了一项广泛的LIS文献回顾(以及其他学术领域的文献),探讨专业人士的信息寻求。作者提出了一个分析模型,用于概括各行各业的专业人士信息寻求行为,从而为该领域未来的研究提供平台。
\subsubsection{信息社会} 
信息社会是一个社会,其中信息的创造、分发、扩散、使用、整合和操控是重要的经济、政治和文化活动。信息社会的目标是通过创新和富有生产力地使用信息技术,获得国际竞争优势。知识经济是其经济对应体,通过对理解的经济开发创造财富。那些有能力参与这一形式社会的人有时被称为数字公民。

基本上,信息社会是信息从一个地方到另一个地方的传递手段(Wark 1997, p. 22)。随着技术的不断进步,我们在分享信息的方式上也有所适应。

信息社会理论讨论了信息和信息技术在社会中的作用,探讨了哪些关键概念应当用于描述当代社会,以及如何定义这些概念。它已成为当代社会学的一个具体分支。
\subsubsection{知识表示与推理}    
知识表示(KR)是人工智能研究的一个领域,旨在通过符号表示知识,以便从这些知识元素中推理出新的知识元素。KR可以独立于基础的知识模型或知识库系统(如语义网络)进行构建。

知识表示研究涉及分析如何准确有效地推理,以及如何最佳地使用符号集来表示知识领域中的事实。符号词汇和逻辑系统结合起来,使得可以对KR中的元素进行推理,生成新的KR句子。逻辑用于提供推理函数如何应用于KR系统中符号的正式语义。逻辑还用于定义操作符如何处理和重塑知识。操作符和操作的示例包括否定、合取、副词、形容词、量词和模态操作符。逻辑是解释理论。符号、操作符和解释理论这些元素共同赋予符号序列在KR中意义。
\subsection{另见}  
\begin{itemize}
\item 计算机与信息科学  
\item 类别:信息科学期刊  
\item 计算机科学奖项列表 § 信息科学奖项  
\item 信息科学大纲  
\item 信息技术大纲
\end{itemize}
\subsection{参考文献}  
\begin{enumerate}
\item Ibekwe, Fidelia; Aparac-Jelusic, Tatjana; Abadal, Ernest (2019). "信息科学中伞状术语的追寻:信息学和信息学的起源"。HAL Open Science. 10?: 1–17.  
\item Otten, Klaus; Debons, Anthony (1970). "信息的元科学:信息学"。美国信息科学学会期刊. 21: 89–94. doi:10.1002/asi.4630210115.  
\item Ingwersen, Peter. (1992). 《信息与信息科学的背景》Libri, 42(2): 99-135.  
\item Stock, W.G., & Stock, M. (2013). 《信息科学手册》档案保存于2023-05-10的Wayback Machine. 柏林,波士顿,MA: De Gruyter Saur.  
\item Yan, Xue-Shan (2011-07-23). "信息科学:它的过去、现在与未来"。信息。2 (3): 510–527. doi:10.3390/info2030510.  
\item "网络控制论与系统词典:技术决定论"。Principia Cibernetica Web. 于2011-11-12存档,2011-11-28查阅。  
\item "信息科学的定义"。www.merriam-webster.com. 于2017-09-25存档,2017-09-25查阅。  
\item Borko, H. (1968). "信息科学:它是什么?" 美国文献学期刊 19(1), 3–5.  
\item Mikhailov, A.I.; Chernyl, A.I.; Gilyarevskii, R.S. (1966). "信息学——科学信息理论的新名称"。科学技术信息期刊. 12: 35–39.  
\item Texas Woman's University (2015). "信息学"。于2016-02-24存档,2016-02-17查阅。  
\item Vickery, B. C. (1973). 《信息系统》。伦敦:Butterworth。  
\item Ellis, D.; Allen, D.; Wilson, T. (1999). "信息科学与信息系统:结合的主题,分离的学科"(PDF),JASIS,50卷,第12期,1095-1107,原文存档于2012-04-25(PDF)。  
\item Luciano Floridi, "什么是信息哲学?" 于2012-03-16存档,Metaphilosophy, 2002, (33), 1/2.  
\item Garshol, L. M. (2004). "元数据?词汇表?分类法?主题图!让我们搞清楚这一切"。2008年10月17日存档,2008年10月13日查阅。  
\item Gruber, Thomas R. (1993年6月). "一种便携式本体规范的翻译方法"。知识获取. 5 (2): 199–220. doi:10.1006/knac.1993.1008. S2CID 15709015.  
\item Arvidsson, F.; Flycht-Eriksson, A. "本体论 I"(PDF)。2008年12月17日存档,2008年11月26日查阅。  
\item Ingwersen, Peter. (1992). 《信息与信息科学的背景》Libri, 42(2): 99-135.  
\item Popper, Karl. (1973). 《客观知识:进化方法》。牛津:Clarendon Press. 信息与信息科学的背景。  
\item Furner, Jonathan (2015). "信息科学既不是"。《图书馆趋势》。63 (3): 362–377. doi:10.1353/lib.2015.0009. hdl:2142/89820. Project MUSE 579340.  
\item Clark, John Willis. 《图书馆的护理:从最早到十八世纪末的图书馆与其设备的发展》剑桥:剑桥大学出版社,1901年。  
\item Korty, Margaret Barton. "本杰明·富兰克林与十八世纪的美国图书馆"。美国哲学学会会刊 1965年12月,第55卷,第9期。  
\item "富兰克林市——富兰克林公共图书馆历史"。Franklinma.virtualtownhall.net. 2010-06-29. 于2011-07-24存档,2011-05-28查阅。  
\item Reichman, F. (1961). "刻痕卡片"。R. Shaw (编), 《图书馆艺术的现状》(第四卷,第1部分,第11–55页)。新不伦瑞克,新泽西州:罗格斯大学图书馆服务研究生院。  
\item Emard, J. P. (1976). "信息科学年表的透视"。美国信息科学学会公告. 2 (8): 51–56.  
\item Smith, E. S. (1993). "站在巨人的肩膀上:从布尔到香农,再到陶布:从19世纪中期到现在的计算机化信息的起源与发展"。信息技术与图书馆。12 (2): 217–226.  
\item Skolnik, H (1976). "化学信息科学的里程碑:化学文献(信息)部门对化学学会的贡献奖研讨会"。化学信息与计算机科学期刊. 16 (4): 187–193. doi:10.1021/ci60008a001.  
\item Adkinson, B. W. (1976). "联邦政府对信息活动的支持:历史概述"。美国信息科学学会公告. 2 (8): 24–26.  
\item Schullian, D. M., & Rogers, F. B. (1958). "美国国家医学图书馆"。《图书馆季刊》,28(1),1–17。  
\item Rayward, W. B. (1994). "国际信息与文献联合会"。在W. A. Wiegand和D. G. David Jr.(编),《图书馆历史百科全书》(第290–294页)。纽约:Garland Publishing, Inc.  
\item Maack, Mary Niles. (2004). "女士与羚羊:苏珊娜·布里埃对法国文献学运动的贡献"。《图书馆趋势》52, 第4期(2004年):719–47。  
\item Day, Ronald. 《信息的现代发明》。卡本代尔,伊利诺伊州:南伊利诺伊大学出版社,2001年:7。  
\item Cragin, Melissa H. (2004). "福斯特·莫哈德:连接传统图书馆世界与新兴的信息科学世界"。《图书馆趋势》52 (4): 833–52。  
\item "ASIST历史"。Asis.org. 1968-01-01. 于2012-10-18存档,2011-05-28查阅。  
\item Miller, R (2012). "社交媒体、真实学习与嵌入式图书馆服务:一项关于饮食学学生的案例研究"。信息素养期刊。6 (2): 97–109. doi:10.11645/6.2.1718. hdl:10919/19080.  
\item Zhang, B., Semenov, A., Vos, M. 和 Veijlainen, J. (2014). "理解社交媒体环境中信息的快速传播:两个案例的比较"。ICC 2014会议论文集,522–533。  
\item Thompson, M. (2012). "分享这个"。EContent. 14–19.  
\item "什么是信息架构?" 信息架构协会。IAinstitute.org 2007年07月26日存档。  
\item Morville, Peter; Rosenfeld, Louis (2006). 《为万维网设计的信息架构》。O'Reilly Media, Inc. ISBN 978-0-596-52734-1.  
\item Goodrum, Abby A. (2000). "图像信息检索:当前研究概述"。信息科学。3 (2).  
\item Foote, Jonathan (1999). "音频信息检索概述"。多媒体系统。7: 2–10. doi:10.1007/s005300050106.  
\item Beel, Jöran; Gipp, Bela; Stiller, Jan-Olaf (2009). "在思维导图上的信息检索——它能用来做什么?"(PDF)。第五届国际合作计算会议:网络、应用与工作分享(CollaborateCom'09)论文集。华盛顿DC:IEEE。  
\item Frakes, William B. (1992). 《信息检索数据结构与算法》。Prentice-Hall, Inc. ISBN 978-0-13-463837-9. 于2013年09月28日存档。  
\item Brown, C. M.; Ortega, L. (2007). "物理科学图书馆员的信息寻求行为:研究是否能指导实践"。《大学与研究图书馆》。66 (3): 231–247. doi:10.5860/crl.66.3.231. 于2014年06月12日存档,2012年04月29日查阅。  
\item Hemminger, B. M.; Lu, D.; Vaughan, K. T. L.; Adams, S. J. (2007). "学术科学家的信息寻求行为"。《美国信息科学与技术学会期刊》。58 (14): 2205–2225. doi:10.1002/asi.20686. S2CID 6142949.  
\item Davies, K.; Harrison, J. (2007). "医生的信息寻求行为:证据综述"。《健康信息与图书馆期刊》。24 (2): 78–94. doi:10.1111/j.1471-1842.2007.00713.x. PMID 17584211.  
\item Robinson, M. A. (2010). "工程师信息行为的实证分析"。《美国信息科学与技术学会期刊》。61 (4): 640–658. doi:10.1002/asi.21290. S2CID 15130260.  
\item Kuhlthau, C. C.; Tama, S. L. (2001). "律师的信息搜索过程:呼吁提供'专为我服务'的信息服务"。《文献学期刊》。57 (1): 25–43. doi:10.1108/EUM0000000007076.  
\item Solomon, Yosef; Bronstein, Jenny (2021年2月18日). "自由职业者在工作场所的信息收集实践:不仅仅是寻求信息"。《图书馆学与信息科学期刊》。54 (1). SAGE: 54–68. doi:10.1177/0961000621992810. ISSN 0961-0006. S2CID 233978764.  
\item "RDF/XML、KIF、Frame-CG和正式化英语中的知识表示" 于2012年03月26日存档,Philippe Martin, 分布式系统技术中心,昆士兰州,澳大利亚,2002年7月15–19日
\end{enumerate}
\subsection{来源} 
\begin{itemize}
\item Borko, H. (1968). "信息科学:它是什么?"。《美国文献学》19 (1)。Wiley: 3–5. doi:10.1002/asi.5090190103. ISSN 0096-946X.  
\item Leckie, Gloria J.; Pettigrew, Karen E.; Sylvain, Christian (1996). "专业人员信息寻求的建模:从对工程师、卫生保健专业人员和律师的研究中得出的通用模型"。《图书馆季刊》66 (2): 161–193. doi:10.1086/602864. S2CID 7829155.  
\item Wark, McKenzie (1997). 《虚拟共和国》。Allen & Unwin, St Leonards.  
\item Wilkinson, Margaret A (2001). "律师在解决问题中使用的信息来源:一项实证探索"。《图书馆与信息科学研究》23 (3): 257–276. doi:10.1016/s0740-8188(01)00082-2. S2CID 59067811.  
\end{itemize}
\subsection{进一步阅读 } 
\begin{itemize}
\item Khosrow-Pour, Mehdi (2005年3月22日). 《信息科学与技术百科全书》。Idea Group Reference. ISBN 978-1-59140-553-5.
\end{itemize}
\subsection{外部链接}