% 衍射(综述)
% license CCBYSA3
% type Wiki

本文根据 CC-BY-SA 协议转载翻译自维基百科\href{https://en.wikipedia.org/wiki/Diffraction}{相关文章}。

\begin{figure}[ht]
\centering
\includegraphics[width=8cm]{./figures/c528f7b4a151de90.png}
\caption{红色激光束通过另一个板上的小圆孔后投射到板上的衍射图案} \label{fig_YS_1}
\end{figure}
不要与折射混淆,折射是指波从一种介质传递到另一种介质时,方向发生的变化。

衍射是波由于障碍物或通过孔径而偏离直线传播的现象。衍射物体或孔径实际上成为了传播波的二次源。衍射与干涉是相同的物理效应,但干涉通常应用于少数波的叠加,而当许多波叠加时,通常使用“衍射”一词。[1]: 433 

意大利科学家弗朗切斯科·玛丽亚·格里马尔迪(Francesco Maria Grimaldi)创造了“衍射”这个词,并在1660年首次准确记录了这一现象。
\begin{figure}[ht]
\centering
\includegraphics[width=8cm]{./figures/922006afafe01bb9.png}
\caption{沿着长度 \(d\) 的无数个点(展示了三个点)从波前投射出相位贡献,产生在注册板上持续变化的强度 \(I(\theta)\)} \label{fig_YS_2}
\end{figure}
在经典物理学中,衍射现象由**惠更斯–弗涅耳原理**描述,该原理将传播波前的每个点视为一组独立的球面波。衍射的特征性图案在当来自相干光源(如激光)的波遇到与其波长相当大小的狭缝/孔径时最为明显,如插图所示。这是由于波前上不同点(或等效地,每个波面波)的叠加或干涉,它们以不同的路径长度传播到接收表面。如果存在多个间距较近的开口,则可能会产生复杂的强度变化图案。

这些效应也发生在光波通过折射率变化的介质,或声波通过具有变化声阻的介质时——所有波都发生衍射,[包括引力波](#),水波,以及其他电磁波如X射线和无线电波。进一步来说,量子力学也表明物质具有波动性质,因此也会发生衍射(这一现象可以在亚原子到分子层面进行测量)。
\subsection{历史}
\begin{figure}[ht]
\centering
\includegraphics[width=8cm]{./figures/b01610604cdbbed3.png}
\caption{托马斯·杨于1803年向皇家学会展示的水波双缝衍射图示} \label{fig_YS_3}
\end{figure}
光的衍射效应最早由弗朗切斯科·玛丽亚·格里马尔迪仔细观察并表征,他还创造了‘衍射’这个术语,源自拉丁语 diffringere,意思是‘打碎成碎片’,指的是光线分散到不同的方向。[6] 格里马尔迪的观察结果于1665年在他去世后发表。[7][8] 艾萨克·牛顿研究了这些效应,并将其归因于光线的弯曲。詹姆斯·格雷戈里(1638-1675)观察到由鸟羽引起的衍射图样,这实际上是首个被发现的衍射光栅。[9] 托马斯·杨在1803年进行了著名的实验,演示了两个紧密间隔的狭缝产生的干涉现象。[10] 他通过解释从两个不同狭缝发出的波的干涉结果,推导出光必须以波的形式传播。

1818年,支持光的粒子理论的学者提出巴黎科学院奖问题应涉及衍射,期望看到波动理论被击败。然而,奥古斯丁·让·弗涅尔凭借他的新波传播理论赢得了奖项,[11] 该理论将克里斯蒂安·惠更斯的思想与杨的干涉概念相结合。[12] 西门·德尼·泊松通过证明弗涅尔理论预测在圆形障碍物的阴影后有光线,挑战了弗涅尔的理论;而多米尼克-弗朗索瓦-让·阿拉戈随后通过实验演示了这种光是可见的,从而确认了弗涅尔的衍射模型。[13]: xxiii [14]
\subsection{机制}
\begin{figure}[ht]
\centering
\includegraphics[width=8cm]{./figures/deffc1f3b0ba65ef.png}
\caption{圆形涟漪水槽中的单缝衍射} \label{fig_YS_5}
\end{figure}
在经典物理中,衍射现象是由波传播的方式引起的;这可以通过惠更斯-菲涅尔原理和波的叠加原理来描述。波的传播可以通过将传播介质中波前上的每个粒子视为一个二次球面波的点源来进行可视化。任何后续点的波位移是这些二次波的总和。当波相加时,其总和由单个波的相位关系以及幅度决定,因此波的总幅度可以介于零和单个波幅度之和之间。因此,衍射图案通常具有一系列的极大值和极小值。

在现代量子力学的光传播理解中,通过一个缝隙(或多个缝隙)传播的每个光子都由其波函数描述,该波函数决定了光子的概率分布:光明和暗带是光子更或更不可能被检测到的区域。波函数由物理环境(如缝隙几何形状、屏幕距离和光子产生时的初始条件)决定。个别光子的波动性质(与仅由大量光子之间的相互作用引起的波动特性不同)是在1909年由G·I·泰勒首次进行的低强度双缝实验中暗示的。量子方法与惠更斯-菲涅尔原理有一些显著相似之处;基于该原理,当光通过缝隙和边界传播时,会在这些障碍物附近或沿着这些障碍物产生二次点光源,结果的衍射图案将是基于这些具有不同光学路径的所有光源的集体干涉的强度分布。在量子形式主义中,这类似于考虑围绕缝隙和边界的有限区域,光子更可能从这些区域起源,并计算与经典形式主义中结果强度成比例的概率分布。

有各种用于光子的解析模型,可以用来计算衍射场,包括基于波动方程的基尔霍夫衍射方程,[15]基尔霍夫方程的弗朗霍夫衍射近似(适用于远场),弗雷涅尔衍射近似(适用于近场)和费曼路径积分形式。大多数配置无法解析求解,但可以通过有限元和边界元方法获得数值解。对于物质波,也采用类似但稍有不同的方法,基于相对论修正的薛定谔方程形式,[16]首次由汉斯·贝特详细阐述。[17] 对于这些情况,也存在弗朗霍夫和弗雷涅尔极限,尽管它们更适用于薛定谔方程的物质波格林函数(传播子)的近似。[18][19][20] 更常见的是完整的多重散射模型,特别是在电子衍射中;[21] 在某些情况下,类似的多重散射模型也用于X射线。[22]

通过考虑个别二次波源的相对相位如何变化,特别是相位差等于半个周期的条件(此时波将相互抵消),可以定性地理解许多衍射现象。

衍射的最简单描述是那些可以简化为二维问题的情况。对于水波,情况已经是这样的;水波仅在水面上传播。对于光波,如果衍射物体在某一方向上延伸的距离远大于波长,我们通常可以忽略这一方向。在光通过小圆孔的情况下,我们需要考虑问题的完整三维特性。
\begin{figure}[ht]
\centering
\includegraphics[width=14.25cm]{./figures/ad5776f128350477.png}
\caption{} \label{fig_YS_6}
\end{figure}
\subsection{例子}
衍射效应在日常生活中经常可见。最引人注目的衍射例子是涉及光的情况;例如,CD或DVD上的紧密排列的轨道充当衍射光栅,当人们观察光盘时,便会形成熟悉的彩虹图案。
\begin{figure}[ht]
\centering
\includegraphics[width=8cm]{./figures/f4db40bd154b36c2.png}
\caption{智能手机屏幕上的像素充当衍射光栅} \label{fig_YS_7}
\end{figure}
\begin{figure}[ht]
\centering
\includegraphics[width=8cm]{./figures/762844ffc5621a76.png}
\caption{数据以凹坑和凸起的形式写入CD;表面的凹坑作为衍射元件。} \label{fig_YS_8}
\end{figure}
这个原理可以扩展用于设计具有特定结构的光栅,以产生任何所需的衍射图案;信用卡上的全息图就是一个例子。

小颗粒在大气中的衍射可以产生晕圈——在太阳或月亮等明亮光源周围出现明亮的圆盘和环状图案。在对面的地方,也可能观察到光环现象——观察者阴影周围的亮环。与晕圈不同,光环现象需要颗粒是透明的球体(如雾滴),因为形成光环的光的背向散射涉及雾滴内的折射和内部反射。
\begin{figure}[ht]
\centering
\includegraphics[width=8cm]{./figures/c1589b705d5fdf46.png}
\caption{月亮晕。} \label{fig_YS_9}
\end{figure}
\begin{figure}[ht]
\centering
\includegraphics[width=8cm]{./figures/fe435ae954bf008b.png}
\caption{从飞机上看到的太阳光环,出现在下方的云层上。} \label{fig_YS_11}
\end{figure}
使用紧凑光源的实心物体的阴影在边缘附近显示出小的干涉条纹。
\begin{figure}[ht]
\centering
\includegraphics[width=8cm]{./figures/ba65bba15578e4b0.png}
\caption{在圆形障碍物阴影中心看到的亮点(阿拉戈点)是由于衍射造成的。} \label{fig_YS_10}
\end{figure}
衍射尖峰是由于相机中的非圆形光圈或望远镜中的支撑支架引起的衍射图样;在正常视觉中,睫毛的衍射可能会产生这样的尖峰。
\begin{figure}[ht]
\centering
\includegraphics[width=8cm]{./figures/444c898c58469d47.png}
\caption{从千年桥的尽头看;月亮在南华桥上方升起。街灯在泰晤士河中倒影。} \label{fig_YS_12}
\end{figure}
\begin{figure}[ht]
\centering
\includegraphics[width=8cm]{./figures/9d3885d0fe627af0.png}
\caption{六角形望远镜镜面中的模拟衍射光芒} \label{fig_YS_13}
\end{figure}
当激光光线照射到光学粗糙的表面时,观察到的斑点图案也是一种衍射现象。当熟食肉类呈现彩虹色光泽时,那是光在肉纤维上的衍射。[24] 所有这些效应都是光作为波传播的结果。

衍射可以发生在任何类型的波上。海浪会在防波堤和其他障碍物周围发生衍射。
\begin{figure}[ht]
\centering
\includegraphics[width=8cm]{./figures/a04f17ceb22c969c.png}
\caption{从一个被淹没的海岸采石场狭窄入口衍射生成的圆形波} \label{fig_YS_14}
\end{figure}
声波可以绕过物体衍射,这就是为什么即使躲在树后面,仍然可以听到有人在叫的原因。[25]

衍射在一些技术应用中也可能成为一个问题;它为相机、望远镜或显微镜的分辨率设置了一个基本限制。

以下是一些其他的衍射示例。



一条极窄的狭缝在光照射下会使光发生衍射,形成一系列圆形波,且从狭缝中射出的波前是一个均匀强度的圆柱波,这符合惠更斯-费涅耳原理。

一个比波长宽的照明狭缝会在其下游空间产生干涉效应。假设狭缝行为如同在其宽度上均匀分布了许多点源,则可以计算出干涉效应。如果我们考虑单一波长的光源,这个系统的分析就会简化。如果入射光是相干的,这些点源的相位将是相同的。照射到狭缝下游某个点的光是由这些点源的贡献组成的,如果这些贡献的相位差为 \( 2\pi \) 或更大,我们可能会在衍射光中发现最小值和最大值。这样的相位差是由不同路径长造成的,贡献光线从狭缝到达该点的路径长度不同。

我们可以通过以下推理找到衍射光中第一个最小值的角度。来自狭缝顶端的光源与位于狭缝中间的光源发生破坏性干涉,当它们之间的路径差为 \( \lambda / 2 \) 时,发生破坏性干涉。类似地,狭缝顶部下方的光源将与位于狭缝中部下方的光源在相同角度上发生破坏性干涉。我们可以沿着狭缝的整个高度继续这一推理,从而得出,整个狭缝的破坏性干涉条件与两个窄缝之间的破坏性干涉条件相同,两个缝的间距为狭缝宽度的一半。路径差近似为 \( \frac{d \sin(\theta)}{2} \),因此最小强度出现在角度 \( \theta_{\text{min}} \) 处,满足以下关系:

\[
d \sin(\theta_{\text{min}}) = \lambda
\]

其中,\( d \) 是狭缝的宽度,\( \theta_{\text{min}} \) 是最小强度发生时的入射角,\( \lambda \) 是光的波长。

我们可以使用类似的论证来表明,如果我们把狭缝分成四个、六个、八个部分等,那么最小值将在以下角度 \( \theta_n \) 处出现:

\[
d \sin(\theta_n) = n\lambda~
\]

其中,\( n \) 是非零整数。

对于衍射图样的最大值,没有这样简单的推理方法。强度分布可以通过使用弗朗霍夫衍射方程来计算:
\[
I(\theta) = I_0 \operatorname{sinc}^2 \left( \frac{d\pi}{\lambda} \sin \theta \right)~
\]
其中,\( I(\theta) \) 是给定角度下的强度,\( I_0 \) 是中心最大值的强度(即 \( \theta = 0 \)),它也是强度分布的归一化因子,可以通过从 \( \theta = -\frac{\pi}{2} \) 到 \( \theta = \frac{\pi}{2} \) 的积分和能量守恒来确定;而 \( \operatorname{sinc} x = \frac{\sin x}{x} \),是未归一化的 sinc 函数。

此分析仅适用于远场(弗朗霍夫衍射),即远大于狭缝宽度的距离。

从上述强度分布中,如果 \( d \ll \lambda \),则强度几乎不依赖于 \( \theta \),因此从狭缝射出的波前将类似于具有方位对称性的圆柱波;如果 \( d \gg \lambda \),则只有 \( \theta \approx 0 \) 时强度才显著,因而从狭缝射出的波前将类似于几何光学中的波前。

当入射光的入射角 \( \theta_i \) 非零时(这会导致路径长度的变化),弗朗霍夫衍射区域(即远场)的强度分布变为:
\[
I(\theta) = I_0 \operatorname{sinc}^2 \left[ \frac{d\pi}{\lambda} (\sin \theta \pm \sin \theta_i) \right]~
\]
其中,\( \theta_i \) 是入射角,符号的选择取决于入射角 \( \theta_i \) 的定义。