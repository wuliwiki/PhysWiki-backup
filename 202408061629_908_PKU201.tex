% 北京大学 2001 年 考研 固体物理
% license Usr
% type Note

\textbf{声明}:“该内容来源于网络公开资料,不保证真实性,如有侵权请联系管理员”

\subsection{(16分)}
说明金铜$Cu$的品体结构,布拉伐格子,所属晶系、点群和空间群,每个单胞(Conventionalunit cell)中的钢$Cu$原子数;如果品格常数为$a$ ,求正格子空间 W-S原胞的体积和第一布里湖区的体积。
\subsection{(12 分)}
倒格子矢量为$K_h=h_1b_1+h_2b_2+h_3b_3$·

(1)求布里渊区边界方程。

(2)证明正格子中:族晶面($h_1h_2h_3$)和倒格矢$K_h$正交。

(3)倒格矢$K_h$长度正比于品而族($h_1h_2h_3$)面间距$d_{h_1h_2h_3}$的倒数。
\subsection{(16分)}
一维双原子链上最近邻原子间的办常数等于$\beta$,最近邻原子问的距离为$a$,令两种原子的质量分别为$M$和$m$,设$M>m$,试求色散关系$\omega(k)$。证明当$M=m$时,色散关系$\omega(k)$变成一维单原子链的色散关系。(要求推导过程)
\subsection{(16分)}
由泡利不相容原理,金属中费米面附近的自由电子容易被激发,费米能级以下很低能级上的自由电了很难激发,通常称为费米冻结。用此物理图象、

(1)估算在室温下金属中自由电子的比热。

(2)算T→0K金属中白由电子的泡利自旋顺性磁化率。