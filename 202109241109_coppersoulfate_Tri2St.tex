% Y-Δ 变换 星角变换

\begin{issues}
\issueDraft
\end{issues}

\footnote{参考 Wikipedia \href{https://en.wikipedia.org/wiki/Y-Δ_transform}{相关页面}.}三角形转星型电阻网络. 可以简化一些复杂的电阻网络.

\begin{figure}[ht]
\centering
\includegraphics[width=8cm]{./figures/Tri2St_1.png}
\caption{Δ 型(左)和 Y 型(右)电阻网络} \label{Tri2St_fig1}
\end{figure}

\subsubsection{Δ 转 Y}
\begin{equation}
R_1 = \frac{R_b R_c}{R_s} \qquad
R_2 = \frac{R_a R_c}{R_s} \qquad
R_3 = \frac{R_a R_b}{R_s}
\end{equation}
其中 $R_s = R_a + R_b + R_c$.

\subsubsection{Y 转 Δ}
\begin{equation}
R_a = \frac{R_0}{R_1} \qquad
R_b = \frac{R_0}{R_2} \qquad
R_c = \frac{R_0}{R_3}
\end{equation}
其中 $R_0 = R_1 R_2 + R_2 R_3 + R_3 R_1$.

\subsection{证明}
未完成.
\subsubsection{简单的推导}
在三端网络中有一端不接入电路时,应满足两端之间电阻相等.即:
\begin{equation}
\left\{\begin{matrix}R_1+R_2=\frac{1}{\frac{1}{R_c}+\frac{1}{R_a+R_b}}\\R_2+R_3=\frac{1}{\frac{1}{R_a}+\frac{1}{R_b+R_c}}\\R_3+R_1=\frac{1}{\frac{1}{R_b}+\frac{1}{R_c+R_a}}\end{matrix}\right.
\end{equation}

解以上方程即可得到Y-Δ 变换.
\subsubsection{比较严谨的证明}
要使得Y-Δ网络之间等价,应使得在三端接入任意电压时有对应一致的电流.
\begin{figure}[ht]
\centering
\includegraphics[width=9cm]{./figures/Tri2St_2.png}
\caption{等效图} \label{Tri2St_fig2}
\end{figure}
电流的正方向按图2定义.则应有:
\begin{equation}
\sum{i}=0
\end{equation}
在Δ网络中有
\begin{equation}
i_{AB}=\FRAC{U_{AB}}{R_{AB}},i_{CA}=\frac{U_{CA}}{R_{CA}},i_A=i_{AB}-i_{CA}
\end{equation}
即
\begin{equation}
i_A=\frac{U_{AB}}{R_{AB}}-\frac{U_{CA}}{R_{CA}}
\end{equation}
在Y网络中有
\begin{equation}
i_a R_a
\end{equation}
