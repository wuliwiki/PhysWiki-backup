% 高斯波包(量子)
% 薛定谔方程|波函数|高斯波包|傅里叶变换

% 未完成
% 插入 app 链接 http://wuli.wiki/apps/gausWP.html
% 介绍群速度和相速度
% 给出宽度和高度随时间的变化方式

\pentry{高斯分布\upref{GausPD},含时薛定谔方程,动量表象} %未完成

\subsection{结论}
设 $t = 0$ 时的波函数(已归一化)
\begin{equation}\label{GausWP_eq1}
\psi (x,0) = \frac{1}{(2\pi\sigma_x ^2)^{1/4}} \E^{-(x - x_0)^2/(2\sigma_x)^2} \E^{\I \frac{p_0}{\hbar}x}
\end{equation}
那么动量表象波函数具有对称的形式\footnote{也可以把\autoref{GausWP_eq1} 和\autoref{GausWP_eq2} 同时除以常数 $\E^{\I p_0 x_0}$ 使\autoref{GausWP_eq1} 最后的 $x$ 变为 $x-x_0$, \autoref{GausWP_eq2} 最后的 $p-p_0$ 变为 $p$. }
\begin{equation}\label{GausWP_eq2}
\psi (p,0) = \frac{1}{(2\pi\sigma_p^2)^{1/4}} \E^{-(p - p_0)^2/(2\sigma_p)^2} \E^{-\I\frac{x_0}{\hbar }(p - p_0)}
\end{equation}
%\begin{equation}
%\psi(k,0) = \qty(\frac{2\sigma_x ^2}{\pi})^{1/4} \E^{-\sigma_x^2 (k - k_0)^2} \E^{-\I x_0(k - k_0)}
%\end{equation}
其中 $\sigma_x$ 为位置的标准差, $\sigma_p$ 为动量的标准差,满足不确定原理
\begin{equation}
\sigma_x\sigma_p = \frac{\hbar}{2}
\end{equation}
含时波函数为
\begin{equation}\begin{aligned}
\psi (x,t) = &\frac{1}{(2\pi\sigma_x^2)^{1/4}} \frac{1}{\sqrt{1 + \frac{\I\hbar t}{2m \sigma_x^2}}}
\exp\qty[\frac{-(x - p_0 t/m)^2}{(2\sigma_x )^2 \qty(1 + \frac{\I\hbar t}{2m \sigma_x^2})}] \exp\qty[\frac{\I p_0}{\hbar } \qty(x - \frac{p_0 t}{2m})]
\end{aligned}\end{equation}

\subsection{推导}

如果我们想要一维波函数的概率分布为高斯分布\upref{GausPD},即
\begin{equation}
\abs{\psi (x)}^2 = \frac{1}{\sigma_x \sqrt{2\pi}} \E^{-\frac{x^2}{2\sigma_x^2}}
\end{equation}
先假设波函数为实数,有
\begin{equation}
\psi (x) = \frac{1}{(2\pi\sigma_x^2)^{1/4}} \E^{-\qty(\frac{x}{2\sigma_x})^2}
\end{equation}
变换到动量表象,% 链接未完成
得\footnote{可以用 Wolfram Alpha 或 Mathematica 计算积分.}
\begin{equation}
\psi(p) = \frac{1}{(2\pi\sigma_p^2)^{1/4}} \E^{-\qty(\frac{p}{2\sigma_p})^2}
\end{equation}
其中 $\sigma_p = \hbar/(2\sigma_x)$, 可见高斯波包一个独特的性质就是在位置和动量表象下都是高斯分布.

由于波函数为实数,动量平均值为零.%未完成
为了让波函数有一个动量 $p_0$,而维持 $\abs{\psi(x)}^2$ 和 $\abs{\psi(p)}^2$ %未完成: 应该区分函数名
的形状不变,我们可以直接将动量表象中的波函数平移 $p_0$, 得
\begin{equation}
\psi (p) = \frac{1}{(2\pi\sigma_p^2)^{1/4}} \E^{-\qty(\frac{p - p_0}{2\sigma_p})^2}
\end{equation}
由傅里叶变换的性质,% 未完成
对应的位置表象波函数需要乘以因子 $\exp(\I p_0 x/\hbar)$ 变为
\begin{equation}
\psi(x) = \frac{1}{(2\pi\sigma_x^2)^{1/4}} \E^{-\qty(\frac{x}{2\sigma_x})^2} {\E^{\I {p_0}x/{\hbar }}}
\end{equation}
类似地,也可以将 $\psi(x)$ 平移 $x_0$ 
\begin{equation}
\psi (x) = \frac{1}{(2\pi\sigma_x^2)^{1/4}} \E^{-\qty(\frac{x-x_0}{2\sigma_x})^2} \E^{\I p_0 (x-x_0)/\hbar}
\end{equation}
而 $\psi(p)$ 则需要乘以因子 $\exp (-\I x_0 p/\hbar)$
\begin{equation}
\psi(p) = \frac{1}{(2\pi\sigma_p^2)^{1/4}} \E^{-\qty(\frac{p - p_0}{2\sigma_p})^2} \E^{-\I x_0 p/\hbar}
\end{equation}
将以上两式同乘一个常数% 未完成:傅里叶变换中应该说明一个函数成一个常数,变换后的函数也乘以该常数.并在此引用
 ${\E^{\I{p_0}{x_0}/\hbar }}$ 就得到\autoref{GausWP_eq1} 和\autoref{GausWP_eq2}.

% 未完成: 时间演化的推导
