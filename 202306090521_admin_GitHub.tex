% GitHub Desktop 的简单使用

% GitHub Desktop 的简单使用
% 代码管理|Github|git|Github Desktop

\begin{issues}
\issueTODO
\end{issues}

\subsection{Git 和 GitHub}
要了解 GitHub 首先要知道什么是 Git。 Git 是一个文件版本控制程序, 通常用于管理程序代码, 但也可以用来管理任何文件。 从某种意义上来讲, Git 相当于一个强大的备份软件, 它可以保存一个文件夹(这个文件夹叫做 \textbf{仓库(repository)}, 简称 \textbf{repo})内所有文件在许多不同时间的快照(叫做 \textbf{commit}), 即所有文件某时刻的内容, 而且可以浏览,恢复某文件的某个版本。 也可以对比出当前文件夹中内容和上一次快照的任何改动, 或每个快照相对上一个的变化。 注意可查看变化的仅限文本文档(例如 txt 文件,以及代码源文件等), 二进制文件如 word 和 excel 等可以通过插件先自动转换为文本文档再做对比。

Git 还可以给仓库创造不同的\textbf{分支(branch)}, 例如一个程序写到一定的阶段后, 我们希望一边开发一个新功能, 一边维护现有的稳定版程序, 那么我们就给这个新功能创建一个分支, 开发该功能时对代码做出的改变不会影响稳定分支, 两个分支都可以有一些列快照(commit)。 等到新功能测试无误后, 再把新功能分支对代码做出的改变\textbf{合并(merge)}到稳定分支。

GitHub (\href{https://github.com}{github.com})是目前世界最大的 Git 仓库托管网站,主要用于编程项目的合作开发和下载(但也会有人用它分享整理一些文档和资料,或者作为个人网盘)。 虽然 Git 程序会把一个仓库的所有的数据保存在你的电脑上, 但它也可以设置一个或多个\textbf{服务器(remote)}, 并可以把该仓库的数据与服务器同步, 而 GitHub 就提供这样一个服务器。 把仓库托管到服务器上可以使数据在不同设备上同步, 进而可以方便不同的用户合作开发一个项目。 合作编辑时,每个设备上都保存该仓库, 且把服务器设置为 GitHub 上的同一仓库即可。 出于众所周知的原因, GitHub 在中国大陆经常无法正常访问, 但程序员通常有自己的方法。 另外国内也有类似的平台如\href{https://gitee.com/}{码云(Gitee)}, 但无论是开源代码的数量和质量都还有较大差距。

严格来说 Git 本身是一个基于命令行的程序, 只能手动输入命令操作(详见 Git 命令行教程\upref{Git0})。 虽然 Git 现在已经有了各种各样的图形界面(GUI), 但命令行版本的功能仍然是最齐全的。 \textbf{GitHub Desktop} 就是 GitHub 官方推出的一个 GUI 程序。 虽然目前官方只发布 Windows 和 Mac 的\href{https://desktop.github.com/}{安装包}, 但 Linux 上也可以找到第三方安装包。
 
\subsection{创建第一个仓库}
在\href{https://desktop.github.com/}{官网}下载 GitHub Desktop 并安装打开, 可以登录或注册 GitHub 账号。 注意 GitHub Enterprise 是企业账号, 个人账号是免费注册的。 注意免费账号只能建立 public repo(任何人都能访问或下载), 创建私密的的 private repo 需要开通会员。 目前使用 edu 结尾的学生邮箱注册可以获得免费会员。

\begin{figure}[ht]
\centering
\includegraphics[width=13cm]{./figures/175f00b22d270213.png}
\caption{GitHub Desktop 登录} \label{fig_GitHub_1}
\end{figure}

登录以后可以选择新建一个 repo 或者从 github 上 clone, 我们这里选择新建(如\autoref{fig_GitHub_2})。 根据提示输入 repo 的名称 \verb|test|, 描述 \verb|learn to use GitHub Desktop|, 文件夹储存目录(默认是 \verb|Documents/GitHub/|)。 勾选 Initialize this repository with a README。 然后就可以确定了。

\begin{figure}[ht]
\centering
\includegraphics[width=8cm]{./figures/7accfb368360ec66.png}
\caption{新建仓库} \label{fig_GitHub_2}
\end{figure}

创建完后界面如\autoref{fig_GitHub_3} 所示
\begin{figure}[ht]
\centering
\includegraphics[width=14cm]{./figures/3ce87c8e4529502d.png}
\caption{repo 主界面} \label{fig_GitHub_3}
\end{figure}

单击 History 可以看到左下角已经有一个 Initial commit 了, 所有的 commit 的历史可以在左上角的 History 找到。 点击 History 可以看到 Initial commit 中添加了两个文件, \verb|README.md| 和 \verb|.gitattributes|。 其中 \verb|README.md| 是一个文本文档, 可以添加 repo 的描述, 使用说明等。 \verb|.gitattributes| 文件用于对 Git 做一些设置, 例如如何处理换行\footnote{Windows 系统的换行使用 CR 和 LF 两个字符, 而 Linux 系统中只用一个 LF 字符。}。 如果新建 repo 的时候没有勾选 Initialize this repository with a README, 就不会有 Initial commit。 点击左上角的 Changes 返回刚才的界面。

注意左下角有一个 Undo 按钮。 如果按下, Initial commit 将被撤销。 再来看右上角的 Push 按钮, 这个按钮用于把新增的 commit 上传(即 push)到 GitHub。 由于我们从来没有 push 过, 现在打开 GitHub 网站是看不到刚刚创建的项目的。 按下 push, 完成后再刷新浏览器就可以在 GitHub 看到了。
\begin{figure}[ht]
\centering
\includegraphics[width=14cm]{./figures/28b0de3d116e69b6.png}
\caption{GitHub 网站上查看 repo} \label{fig_GitHub_5}
\end{figure}

\subsection{Commit}
\addTODO{基本操作和概念}
