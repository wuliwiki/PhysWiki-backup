% 一阶隐式常微分方程的存在唯一性定理
% keys 隐式方程|ODE|differential euqation|存在唯一
% license Usr
% type Wiki

\pentry{隐函数定理\upref{impli},皮卡定理\upref{PiLin}}

\subsection{一阶隐式常微分方程的存在唯一性定理}

对于一阶隐式常微分方程 $F(x, y, y')=0$,函数 $F$ 满足:
\begin{enumerate}\lable{lst_ODEa5_1}
\item 在 $(x_0, y_0, y_0')$ 的某个邻域内连续,且关于 $y$、$y'$ 有连续的一阶偏导数;
\item $F(x_0, y_0, y_0')=0$;
\item $F_{y'}'(x_0, y_0, y_0')\neq 0$。
\end{enumerate}
那么,$F(x, y, y') = 0$ 存在唯一的满足 $y(x_0) = y_0, y'(x_0) = y_0' $的、定义在 $[x_0-h, x_0+h]$ 上的函数,其中 $h$ 为一个充分小的正数。

\subsection{证明}

由隐函数定理,这方程为一确定了一个定义在点 $(x_0, y_0)$ 的某邻域 $S$ 上的隐函数 $y'=f(x, y)$,满足
$$F(x, y, f(x,y)) \equiv 0, y_0'=f(x_0, y_0) ~,$$
同时,$f(x, y)$ 在 $S$ 内连续,$f'_y$ 在 $S$ 内连续。其中
$$f'_y(x, y) = - \frac{F'_y(x, y, y')}{F'_{y'}(x, y, y')} ~.$$

接下来引入一个皮卡定理的推论:
\begin{corollary}{皮卡定理推论}
$f(x, y)$ 在区域 $D$ 上连续,关于 $y$ 有连续一阶偏导数 $f'_y(x, y)$。
那么
$\forall P(x_P, y_P) \in D$,初值问题
$$\frac{\dd y}{\dd x} = f(x, y), y(x_0) = y_0 ~,$$
存在唯一定义在含 $x_0$ 的某区间上的解。

由皮卡定理,又根据微分中值定理,容易证明在某区域内连续的函数,在区域内关于 $y$ 满足李氏局部条件的充分条件为 $f(x, y)$ 在区域内关于 $y$ 有连续一阶偏导数 $f'_y(x, y)$。容易证明这个推论。
\end{corollary}

由这个推论,方程 $y'=f(x, y)$ 存在唯一的满足 $y(x_0)=y_0$ 的解:
$$y=g(x), x \in [x_0-h, x_0+h] ~.$$
即 $g'(x) = f(x, g(x)), g(x_0) = y_0$。
将其带入 $F$ 有:
$$F(x,g(x),g'(x)) \equiv F(x, g(x), f(x,g(x))) \equiv 0 ~,$$
而 $g'(x_0)=f(x_0, g(x_0))=f(x_0,y_0)=y'_0$,故
$$y=g(x), x\in[x_0-h, x_0+h] ~,$$
为唯一解,即证。