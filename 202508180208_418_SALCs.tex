% 对称性匹配的线性组合
% keys 分子轨道理论|对称性匹配的线性组合|对称性匹配线性组合|SALC|SALCs
% license Usr
% type Tutor

\pentry{分子点群\nref{nod_MPG1}}{nod_52ba}
我们以 $\text{BH}_3$ 分子为例研究如何通过 SALCs 构建分子轨道。先确定中心原子,例如 $\text{H}_2\text{O}$ 就是中间的氧原子,$\text{BH}_3$ 则是中间的 $\text B$。将周围的 $\text H$ 原子的 $1s$ 轨道(让 $\text H$ 原子成键固然只能考虑 $1s$ 轨道)组合成\textbf{符合特定对称种类}的线性组合,再将上面的结果进一步按照对称性与 $\text B$ 原子的轨道合理线性组合,得到分子轨道。

我们将三个 $\text H$ 原子分别记为 $A$,$B$,$C$ 用以区分。$\text{BH}_3$ 分子应有 $C_{3v}$ 对称性,有 $C_3$ 和 $\sigma_v$ 两种对称操作以及恒等操作 $E$。计算特征标:
\begin{enumerate}
\item $s_A$、$s_B$、$s_C$ 均在 $E$ 下不变,$\chi(E)=3$;
\item $s_A$、$s_B$、$s_C$ 在 $C_3$ 一类操作(包括 $C_3^1$,$C_3^2$)下轮换,$\chi(C_3)=0$;
\item $s_A$、$s_B$、$s_C$ 在 $\sigma_v$ 一类操作(包括 $\sigma_v$,$\sigma_v'$,$\sigma_v''$)下,一者位置不变,另外两者交换,得到 $\chi(\sigma_v)=1$。
\end{enumerate}
考虑记分子的轨道为 $\Gamma$,以三个 $\text H$ 原子的轨道 $(s_A,s_B,s_C)$ 为基底,这显然就是一个可约表示,特征标表为
\begin{table}[ht]
\centering
\caption{$\Gamma$ 的特征标表}\label{tab_SALCs1}
\begin{tabular}{|c|c|c|c|}
\hline
 & $E$ & $2C_3$ & $3\sigma_v$ \\
\hline
$\Gamma$ & 3 & 0 & 1 \\
\hline
\end{tabular}
\end{table}
下面来看 $\Gamma$ 是由哪些不可约表示构成的,$C_{3v}$ 群的特征标表是
