% 计算机算数


要了解科学计算,首先要知道数据是如何在计算机中存储和表达的.在计算机基础中我们知道,所有的数据在计算机内存中都是以二进制数的形式存储的,但对于不同的数据类型,二进制数所代表的意义也不尽相同.下面我们来看两种最常见的数据类型:整数和浮点数.</p><h2>1. 整数</h2>很自然的,对于一个给定的十进制整数,可以将它转换为二进制数,从而在计算机中表示. 下图中的8位(8 bits)二进制数,</p>

表示

 $(01011101)2=0\cdot2^7+1\cdot2^6+0\cdot2^5+1\cdot2^4+1\cdot2^3+1\cdot2^2+0\cdot2^1+1\cdot2^0=(93)_{10}$  

从Python3.0起,可以表示的整数的最大值上限被移除, 这就意味着我们可以精确表示任何整数,也就是说只要将给定的整数转换为二进制数,然后占用相应长度的内存空间即可.理论上 16G 的内存可以存储的最大整数约为  $10^{5.17\times10^9}$  .

另外,由于两个不同整数之间的最小间隔为1(整数的机器精度),因此,与整数有关的加、减和乘法都可以被精确计算,并且没有任何舍入误差.

例子1: 大整数及其运算

\textbf{注1}:普通除法运算  \verb|/|  在Python中会默认转换为浮点数,因此并不能保证完全精确.

\textbf{注2}:整数除法 \verb|//| 可以保证精确,但是结果只有商,余数可以用 \verb|%|  求得.</p><div class="highlight"><pre><code class="language-python"><span class="n">a</span> <span class="o">=</span> <span class="mi">135791113151719</span>


输出:

2. 浮点数

不同于整数,实数是连续的,这就意味着两个不相同的实数之间的最小距离可以无限接近于0. 前文中对于整数的存储方式显然不再适用,我们并不能把所有实数都在计算机中表示出来. 甚至我们都不能把0到1之间的所有实数表达出来.

这就要求计算机对实数运算进行一定的近似,使得给定任意一个实数,我们都能找到一个与它接近的计算机浮点数表达.同时,计算机浮点数系统需要保证\textbf{运算精度}和\textbf{速度}.这就要求计算机浮点数系统需要尽可能满足下面的条件:</p><ol><li>每个浮点数占用的内存空间一致.

\begin{itemize}
\item 覆盖的范围尽可能大
\item 舍入误差或机器精度尽可能的小.关于舍入误差,参见我之前的文章
\end{itemize}

下面我们会一步一步的设计一个8位(8 bits)的浮点数系统,也就是用8位的256个二进制数字来表达浮点数,并逐步满足上述条件. 从而探究一下真正的IEEE浮点数标准是如何设计和指定的. 整个的这个设计过程,也可以在我的网络课件中找到: </p><h3>2.1 简单的二进制表达浮点数</h3>类似二进制以 \verb|2| 为基的方式,我们也可以用  $\frac{1}{2}$  为基,那么下图的这个8位二进制数就表示</p><figure data-size="normal"><noscript> $(01011101)_2=0\cdot2^{-1}+1\cdot2^{-2}+0\cdot2^{-3}+1\cdot2^{-4}+1\cdot2^{-5}+1\cdot2^{-6}+0\cdot2^{-7}+1\cdot2^{-8}=0.36328125$  

事实上,这个8位二进制数和前面整数一章用到的是同一个二进制数,但他们因为数据类型不同,代表的含义也就不同了.

由此,我们的8位浮点数系统可以表示 256 个在 0 到  $1-\frac{1}{2^8}=0.99609375$  之间的实数,它们的机器精度为  $\frac{1}{2^8}=0.00390625$.

\textbf{小结}: 此方法的机器精度是8位系统可达到的最小值,但覆盖范围仅为0到1</u>.</p><h3>2.2 扩大范围</h3>想要表达更大的数,可以从上面的方式出发,将得到的0到1之间的数都乘以一个固定的值,例如乘以8. 这样,我们的浮点数系统的覆盖范围扩大到了0到7.96875. 但是这个操作也会将机器精度乘以相应的值,使其下降为了  $\frac{1}{2^5}=0.03125$  .

\textbf{小结}: 等比例的扩大范围会大幅降低机器精度</u>.</p><h3>2.3 标准化</h3>延续上面的思路,我们可以用乘以多个不同的倍数的方法来控制所扩大的范围. 也就是说,从8位二进制中拿出2位(下面的例子中用的是最后两位,但也可以是前面两位),用来记录扩大的倍数,而前面的6位同之前一样用作表达0到1之间的实数,这6位被称作尾数位(mantissa). 

为了进一步扩大范围,取出的2位从二进制转化为十进制后,被放在以 \verb|2| 为基(basis)的指数位置上形成放大的倍数,因此这2位在这个系统中被称为指数位(exponent).

如下图,我们继续使用同样的8位二进制数,在经过标准化过程后,表达的数为 $\frac{1}{2^6}=0.015625$  .

\textbf{小结}:此方法在保持了与2.2相似的范围的同时,将机器精度提升了一倍.

\textbf{但是},我们注意到,这个方法有一个严重的问题在于\textbf{重复},例如 01011101 和 10111000 都表示了同一个数.

2.4 去掉重复

去掉重复数字的方法很巧妙,这个方法并不改变2.3中的标准化方式,只是在对尾数位求和过程中加上一个常数1,即

 $(01011101)_2=({\color{red}{1}}+0\cdot2^{-1}+1\cdot2^{-2}+0\cdot2^{-3}+1\cdot2^{-4}+1\cdot2^{-5}+1\cdot2^{-6})\times2^1=2.71875$  

使用这种改进方法,我们可以表达 1 至 15.875 之间的 256 个不同的实数, 我们把这些浮点数画在数轴上, 如下图 $\frac{1}{2^6}=0.015625$  .

\textbf{注}:数轴上小于1的范围称为下溢(underflow),而超过15.875的范围称为上溢(overflow).</p><h3>2.5 更多的</h3>下面的几个步骤可以进一步完善我们的浮点数系统: </p><ul><li>为了表达负数,我们需要从尾数位(mantissa)中取出一位,作为\textbf{符号位}. 

\begin{itemize}
\item 为了将下溢降低,我们将指数位(exponent)表达的十进制数(0,1,2,3)向负数平移,变成(-1,0,1,2).</li></ul><h3>3. IEEE 浮点数标准</h3>现在的计算机系统采用的是</p>

每一个浮点数(Python中的 \verb|float| 类型)占用64位.如下图所示,其中第1位为符号位,下面11位为指数位(exponent),最后52位为尾数位(mantissa).</p><figure data-size="normal"><noscript> $f=\frac{i}{2^{52}},\quad i=0,1,2,...,2^{52}-1$  .指数位  $e$  的标准范围定为  $-1022\le e\le 1023$  ,把  $e=1024$  作为特殊位,并把  $e=-1023$  作为保留位(我们放在最后分析).

由此,我们可以求得:</p><ul><li>机器精度 \verb|eps| 为  $\frac{1}{2^{52}}\approx 2.220446049250313\times10^{-16}$  .
\item 最小的正浮点数 \verb|realmin| 为  $2^{-1022}\approx2.2251\times10^{-308}$  ,在python中可以通过 \verb|numpy.finfo(float).tiny| 获得.
\item 最大的浮点数 \verb|realmax| 为 $(2-2^{-52})\times2^{1023}\approx1.7977\times10^{308}$  ,在python中可以通过 \verb|numpy.finfo(float).max| 获得.
\item 令  $e=1024$  和  $f=0$  ,表示无穷大  $\infty$  ,即任何大于 \verb|realmax| 的数,在python中可以通过 \verb|numpy.inf| 得到.
\item 令  $e=1024$  且  $f\neq0$  ,则表示这不是一个数(NaN,Not a Number).通常出作为 \verb|0/0| 或者 \verb|numpy.inf-numpy.inf| 的结果.我们也可以通过 \verb|numpy.nan| 表达它.</li></ul><h3>例子2: numpy中查看浮点数标准</h3><div class="highlight"><pre><code class="language-python"><span class="kn">import</span> <span class="nn">numpy</span> <span class="kn">as</span> <span class="nn">np</span>
\end{itemize}


输出

3.1 亚标准(subnormal)

在很多计算系统中,除了上面的标准浮点数以外,还存在着亚标准浮点数(subnormal floating point numbers). 

它们是比最小的标准浮点数 \verb|realmin| (  $2.2251\times10^{-308}$  )还要小的正实数.

我们先来看一下下面这个例子:

可以看到

\begin{itemize}
\item 我们能够表示比最小标准数 \verb|realmin| 更小的一些正实数;
\item 它们的精度随着我们远离 \verb|realmin| 而降低;
\item 最终,当这个数与 \verb|realmin| 的比小于约  $10^{-16}$  时,它变成了0.</li></ul>这是因为,我们是用保留位  $e=-1023$  来表示这类亚标准浮点数,它们范围为 \verb|realmin|   $\times$   \verb|eps| 到realmin(eps为机器精度).也就是说,我们能表示的最最小的正实数为  $2^{-(1022+52)}\approx 0.494\times10^{-323}$  ,但这个数的精度非常低.
\end{itemize}

\textbf{注}:这里尽管  $e=-1023$  ,但我们并不是用标准浮点数的转换规则进行运算的,同时在尾数前面加的1也并没有在这出现.
