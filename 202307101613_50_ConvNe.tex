% 卷积神经网络
% keys 深度学习 卷积 神经网络

\pentry{卷积\upref{Conv}}

\textbf{卷积神经网络}(简称卷积网络,Convolutional Neural Networks, CNN)是指那些至少有一层为卷积层的神经网络[1],卷积网络是一类最广泛使用的基本神经网络架构。\textbf{卷积层}(Convolutional Layer)是用卷积运算代替一般矩阵乘法运算的网络层。

卷积神经网络与以往的传统神经网络,比如全连接网络,有很多相似的地方。它们都有数据输入和输出,通常具有中间隐含层。在深度学习中,卷积网络的隐含层数量往往较多。卷积操作主要考虑相邻神经元之间的关系。由于使用卷积操作,卷积网络与全连接网络相比,参数数量大大减少。而此一特性恰好可以适用于图像处理。因为,在图像中,通常距离接近的像素之间具有较强的关系,而距离较远的像素之间可能没有较大的关系。

由于在实际应用中,大多数图像是二维数据,因为二维卷积网络最为常用。二维卷积层有多个超参数。一个二维卷积层的输入为一个矩阵,其宽为$W_i$、高为$H_i$、通道数$C_i$。卷积核个数为$K$,步长为$S$。补零的数量为$P$。经过卷积层的操作之后,输出一个矩阵,该矩阵的规格为:宽$W_o$、高$H_o$、通道数$C_o$。

输入和输出的超参数之间有如下关系: \\
\begin{itemize}
\item 输出的通道数取决于卷积核的个数,即:$C_o=K$。
\item 输出的宽取决于输入的宽、卷积步长和补零个数,即:$W_o=(W_i+2P)/S+1$
\item 输出的高取决于输入的高、卷积步长和补零个数,即:$H_o=(H_i+2P)/S+1$
\end{itemize}

有些应用场景,所须要处理的数据是三维图像块,此时可以使用三维卷积网络。三维卷积层的超参数如下:输入为一个矩阵,其宽为$W_i$、高为$H_i$、深为$D_i$、通道数$C_i$。卷积核个数为$K$,步长为$S$。补零的数量为$P$。经过卷积层的操作之后,输出一个矩阵,该矩阵的规格为:宽$W_o$、高$H_o$、深为$D_o$、通道数$C_o$。

输入和输出的超参数之间有如下关系: \\
\begin{itemize}
\item 输出的通道数取决于卷积核的个数,即:$C_o=K$。
\item 输出的宽取决于输入的宽、卷积步长和补零个数,即:$W_o=(W_i+2P)/S+1$
\item 输出的高取决于输入的高、卷积步长和补零个数,即:$H_o=(H_i+2P)/S+1$
\item 输出的深取决于输入的高、卷积步长和补零个数,即:$D_o=(D_i+2P)/S+1$
\end{itemize}

当前流行的深度学习开发框架都有相应的编程接口,能够很方便的构建卷积网络模型。
你如果使用的是TensorFlow v2.x版本,可以采用以下函数来建立一个二维卷积层:
\begin{lstlisting}[language=python]
tf.nn.convolution(
    input,
    filters,
    strides=None,
    padding='VALID',
    data_format=None,
    dilations=None,
    name=None
)
\end{lstlisting}
该函数的参数主要是:input是输入张量,filters是过滤器(即卷积核)的个数,strides表示的是步长,padding表示的是补零方式,data_format表示的是数据格式,dilations表示的是升采样和降采样率。其中data_format数据格式参数有两个取值:$NHWC$和$NCHW$。$N$表示的是批次大小,$H$表示的是高度,$W$表示的是宽度,$C$表示的是通道数。

你如果使用的是PyTorch,可以采用下面的代码来建立二维卷积层:
\begin{lstlisting}[language=python]
torch.nn.Conv2d(in_channels, 
                out_channels, 
                kernel_size,
                stride=1, 
                padding=0,
                dilation=1,
                groups=1,
                bias=True,
                padding_mode='zeros',
                device=None,
                dtype=None)
\end{lstlisting}
这是一个封装好的二维卷积层类,可以通过新建一个Conv2d对象来建立一个Pytorch的二维卷积层。该类的主要参数有:in_channels是输入数据的通道数,out_channels是输出结果的通道数,kernel_size是卷积核尺寸,stride是步幅,padding是补零数量,dilation是膨胀方式,groups是分组数,bias是偏置,padding_mode是补零方式,device是所采用的计算设备,可以是CPU,也可以是GPU,dtype是数据类型。


\textbf{参考文献:}
\begin{enumerate}
\item I. Goodfellow, Y. Bengio, and A. Courville, Deep learning. MIT press, 2016.
\end{enumerate}