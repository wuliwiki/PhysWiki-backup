% 类氢原子的 Stark 效应(抛物线坐标系)

\begin{issues}
\issueDraft
\end{issues}

\pentry{抛物线坐标系中的类氢原子定态波函数\upref{ParaHy}}

\footnote{参考 \cite{Bransden} Chap. 3.5 One-electron atoms in parabolic coordinates.}注意这里假设原子核不动, 而不是向之前一样采用约化质量 $\mu$。 哈密顿算符中的电场项为
\begin{equation}
H' = \mathcal Ez = \frac12 \mathcal E (\xi - \eta)
\end{equation}
薛定谔方程变为
\begin{equation}
\qty(-\frac12 \laplacian - \frac{Z}{r} + H')\psi = E\psi
\end{equation}
分离变量法得微分方程(对比\autoref{eq_ParaHy_1}~\upref{ParaHy})
\begin{equation}
\dv{\xi} \qty(\xi \dv{f}{\xi}) + \qty(\frac{E \xi}{2} - \frac{m^2}{4\xi} -\frac14\mathcal E \xi^2 + \nu_1) f = 0
\end{equation}
\begin{equation}
\dv{\eta}\qty(\eta \dv{g}{\eta}) + \qty(\frac{E\eta}{2} - \frac{m^2}{4\eta} +\frac14\mathcal E \eta^2 + \nu_2)g = 0
\end{equation}

可能由于求出精确解过于复杂, \cite{Bransden} 中仍使用一阶微扰理论, 得本征能量为
\begin{equation}
E_{n,n_1,n_2}^{(1)} = -\frac{Z^2}{2n^2} + \frac32 \mathcal E \frac{n}{Z}(n_1 - n_2)
\end{equation}
书中也没有给出波函数的解析解。

\subsection{平方 Stark 效应}
类氢原子基态不存在一阶微扰, 使用二阶微扰, 在抛物线坐标中计算积分有
\begin{equation}
E_{100}^{(2)} = e^2\mathcal E^2 \sum_{n\ne 1, l, m} \frac{\abs{\mel{\psi_{nlm}}{z}{\psi_{nlm}}}^2}{E_1 - E_n}
\approx - 2.25\frac{\mathcal E^2}{Z^4}
\end{equation}
例如氢原子基态能量 $-0.5$, 若静电场为 $0.01$, 则总能量为 $-0.500225$ (包括在静电场中的势能)。

一般公式
\begin{equation}
E_{n,n_1,n_2,m} = -\frac{Z^2}{2n^2} + \frac32 \mathcal E \frac{n}{Z} (n_1-n_2)
-\frac{1}{16}\mathcal E^2 \qty(\frac{n}{Z})^4 [17n^2-3(n_1-n_2)^2-9m^2+19]
\end{equation}
