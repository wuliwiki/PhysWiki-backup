% 柯西积分定理(综述)
% license CCBYSA3
% type Wiki

本文根据 CC-BY-SA 协议转载翻译自维基百科\href{https://en.wikipedia.org/wiki/Cauchy\%27s_integral_theorem}{相关文章}。

在数学中,柯西积分定理(也称为柯西–古尔萨定理)是复分析中的一个重要结论,以 奥古斯丁–路易·柯西(和爱德华·古尔萨的名字命名。该定理描述了复平面上全纯函数的路径积分性质。其核心内容是:如果函数$f(z)$在一个单连通域$\Omega$ 内是全纯的,那么对于 $\Omega$ 内的任何闭合路径$C$,沿着该路径的积分都为零:
$$
\int_{C} f(z)\, dz = 0.~
$$
\subsection{命题}
\subsubsection{复线积分的基本定理}
如果函数 $f(z)$ 在某个开区域 $U$ 上是全纯函数,且曲线 $\gamma$ 位于该区域内,从点 $z_0$ 延伸到点 $z_1$,则有:
$$
\int_{\gamma} f'(z)\,dz = f(z_1) - f(z_0).~
$$
此外,如果 $f(z)$ 在开区域 $U$ 内存在一个单值原函数,那么在该区域内,路径积分$\int_{\gamma} f(z)\,dz$对于所有路径来说都是路径无关的。

\textbf{在单连通区域上的表述}

设 $U \subseteq \mathbb{C}$ 是一个单连通开集,并且 $f: U \to \mathbb{C}$ 是一个全纯函数。如果 $\gamma: [a, b] \to U$ 是一条光滑的闭曲线,则有:
$$
\int_{\gamma} f(z)\,dz = 0.~
$$
其中,$U$ 是单连通集意味着它没有“洞”,换句话说,$U$ 的基本群是平凡的。

\textbf{一般形式}

设 $U \subseteq \mathbb{C}$ 是一个开集,且 $f: U \to \mathbb{C}$ 是一个全纯函数。如果 $\gamma: [a, b] \to U$ 是一条光滑的闭曲线,并且 $\gamma$同伦于一条常值曲线,则有:
$$
\int_{\gamma} f(z)\,dz = 0,~
$$
其中 $z \in U$。

一条曲线如果能通过在 $U$ 内的平滑同伦逐渐收缩到某一点(即常值曲线),则称这条曲线同伦于常值曲线。直观地说,这意味着可以在不离开区域 $U$ 的情况下,把闭合曲线“缩成一个点”。第一种表述是这一一般情况的特殊情形,因为在单连通区域中,任意闭曲线都可以同伦收缩为一点。
