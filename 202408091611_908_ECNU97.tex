% 东师范大学 1997 年 考研 量子力学
% license Usr
% type Note

\textbf{声明}:“该内容来源于网络公开资料,不保证真实性,如有侵权请联系管理员”

\subsection{一、10分}
求对易关系 $[x, e^{-\beta p_x}] = ? \beta$ 为常数
\subsection{二、10分}
若粒子在势场  V(x) 中运动,试证:
$$\frac{d \langle x \rangle}{dt} = \frac{\langle p_x \rangle}{m}~$$

$$\frac{d \langle p_x \rangle}{dt} = -\left\langle \frac{\partial V}{\partial x} \right\rangle~$$

$\langle \quad \rangle$ 表示物理量的平均值。
\subsection{三、15分}
利用角动量算符 $L_x$ 和 $L_y$ 组成升位算符 $L_+$ 和降位算符 $L_-$:
$L\pm = L_x + i L_y \quad L_+ = (L_-)^\dagger$

(1) 求解: $L_- L_+ = ? \quad L_+ L_- = ?$

(2) 已知 $L_+ |l m \rangle = C_+ |lm + 1 \rangle, \quad L_- |lm \rangle = C_- |lm - 1 \rangle$
求常数 C_+ 和 C_- 的值。
\subsection{四、15分}
带电粒子 (电荷为q) 在 B 场和 E 场的 xy 平面内运动,取参考系 (0, 8x, 0)。
\begin{enumerate}
\item 1.写出这一运动的哈密顿量。
2. 求出拉格朗日函数 L 对坐标的导数。
\end{enumerate}
