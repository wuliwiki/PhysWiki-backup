% 辛流形
% keys 哈密顿流|Hamiltonian flow|Symplectic Manifold|辛形式|哈密顿正则方程|理论力学|分析力学|经典力学|classical mechanics|流形

辛流形是经典力学研究中自然出现的几何结构.

要描述质点组成的系统所处的状态,只需要知道每个质点的速度和位置,或者动量和位置.自由质点组成的系统,其所有可能的状态构成的\textbf{相空间(phase space)}是一个欧几里得空间.具体地,$M$维空间中的$N$个自由质点构成的系统,其相空间是$M^N$维\textbf{欧几里得空间}.

但是如果存在约束,那么系统的相空间一般来说不是欧几里得空间.比如说,一个$3$维空间中的自由质点,其相空间就是$\mathbb{R}^3\times \mathbb{R}^3$,即描述其位置需要一个$\mathbb{R}^3$、描述动量又需要一个$\mathbb{R}^3$,合起来就是一个$6$维欧几里得空间$\mathbb{R}^6$.但如果此质点被固定在轻质杆的一端,而杆的另一端在某惯性系中固定不动,那么其相空间就是$S^2\times \mathbb{R}^2$,或者更准确地,$T^* S^2$.$S^2$描述的是质点的位置,这是一个\textbf{流形};而$\mathbb{R}^2$则是$S^2$的\textbf{余切空间}.

一般地,





























