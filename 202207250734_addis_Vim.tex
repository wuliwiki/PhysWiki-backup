% Vim 笔记

\begin{issues}
\issueDraft
\end{issues}




\subsection{基础}
\begin{itemize}
\item linux 命令行输入 `vimtutor` 进入教程
\item 在 Ubuntu 中, `vi` 就是 `vim`, 打开的都是 vim 程序.
\item 撤销: `u`, 重做: `Ctrl+r`
\item 复制: `y` (yank), 剪切: `d` (delete)
\item 用 `:w /<path>/<file>` 另存.
\item 命令行用 `ctrl+D` 可以查看所有候选的补全
\item 由于 ctrl 键过于重要, 建议仿照 hhkb, 将 caps loc 和左 ctrl 键位置调换, 用 sharp keys 软件就可以 
\end{itemize}

\subsection{移动}
\begin{itemize}
\item 使用 hjkl 按键可以上下左右移动, 不需要方向键(好远)
\item `w` 可以跳到下一个单词, 比 `l` 要快
\item `b` 跳到上一个单词, 比 `h` 要快
\item `e` 跳到下个单词末尾
\item `0` 使光标跳到行首
\item `^` 跳到第一个不为空格/tab 的字母
\item `$` 跳到行末
\item `ctrl+e` 将窗口向下移动一行, `ctrl+y` 向上移动一行
\item `ctrl+u` 向上翻半页, ctrl+d 向下翻半页, ctrl+b 上一页, ctrl+f 下一页
\item `ctrl+]` 可以进入一个 tag, 相当于点击链接, 用 ctrl+o 返回
\item `G` 可以跳到文档最后, `gg` 可以跳到文件开始
\item `123G` 跳到第 123 行
\item 用 `:<number>` 也可以直接跳到指定的行
\item 按 `$` 跳到行末
\item 按 `A` 跳到行末并编辑
\item ` `` ` 可以让光标回到上一个位置, `''` 让光标回到上一个位置的行首. 连续按多次将在两个位置间跳转
\item 要在 insert 或者 command 模式移动光标(包括上键查找历史), 可以设置快捷键如下
\begin{lstlisting}[language=bash]
" insert mode 快捷键
inoremap <C-h> <Left>
inoremap <C-j> <Down>
inoremap <C-k> <Up>
inoremap <C-l> <Right>
" command mode 快捷键
cnoremap <C-h> <Left>
cnoremap <C-j> <Down>
cnoremap <C-k> <Up>
cnoremap <C-l> <Right>
\end{lstlisting}
\item `z+回车` 用于将当前行移动到屏幕顶部, `zz` 移动到中间
\item `H` 让光标跳到屏幕第一行行首, `L` 跳到最后一行行首, `M` 跳到中间
\item `(` 和 `)` 把光标移动到上一句或者下一句. 下一句. 下一句. 下一句.
\item `{` 和 `}` 把光标移动到上一段或者下一段
\end{itemize}

\subsection{编辑}
\item 按 `i` 直接编辑
\item 用 `x` 键删除光标处的字符
\item 在单词首字母按 `dw` 可以删除当前字符到单词 `d` 开启 delete 模式, `w` 跳到下一个单词行首
\item `d$` 可以删除当前字符到行末 
\item `dw`, `d$` 这样的操作叫做 operator + motion
\item 在移动前面加数字, 可以重复移动指定的次数, 例如 `100w`, `d3w`(删除三个字)
\item `dd` 删除一行, `3dd` 删除三行
\item 编辑模式或者命令模式下 `ctrl+U` 可以删除光标左边的内容
\item `f` 加一个字符可以跳到同一行内下一个出现该字符的位置(话说为什么不直接用 `/`)

\subsection{vimtutor 笔记}
\item `set mouse=a` 可以使用鼠标!
\item 鼠标双击也可以进入 tag
\item `:help 内容` 可以进入帮助页面
\item 按 `esc` 进入 normal mode
\item `u` 撤销, ctrl+R 重做,`10u` 撤销 10 次. 注意每次撤销一个命令, 一次插入很多单词也算一个命令, `10x` 也算一个命令
\item `U` 用于撤销改行的所有修改
\item `p` (put)用于把 `dd` 删除的行插入到光标下方, 可以多次粘贴, 可以 `5p`
\item `r` (replace)可以替换光标处的单个字符, 例如 `ra` 替换为 `a`, `3ra` 连续三个字符替换为 `a`
\item `c` (change)相当于 `d` 再 `a`
\item `ctrl+G` 查看当前的位置和文件状态
\item `%` 可以找到对应的另一半括号 (), [], {}

\subsection{Visual}
\item 选择: `v`, 选择行: `V`, 选择块: `ctrl+v`.
\item 复制时用 `v` 粘贴(光标右侧): `p`, 粘贴(光标左侧): `P`
\item 剪切整行直接用 `dd` (事实上是剪切!)
\item 复制时用 `V` 或, 粘贴(光标下一行下移): `p`, 粘贴(光标所在行下移): `P`
\item 全选,用 `ggVG` 即可

\subsection{搜索替换}
\item 如果 search 的内容中包含 `/`, 用 `\/` 即可, 可能的转义的字符都用 `\` 试试就行
\item 用 `:/ *<string>*` 会向下搜索文档中所有包含 `string` 的位置, 如果再按 `n`, 会搜索下一处. 
\item 向下搜索 `/pattern`, 向上搜索 `?pattern`,下一个 `n`, 上一个 `N`.
\item `:noh` 可以取消搜索匹配的高亮(如果有设置)
\item `q/` 可以打开新窗口编辑搜索历史, 用回车执行, 按两次 `ctrl+c` 关闭
\item 不区分大小写搜索:  `/` 后面加 `\c`
\item `*` 可以搜索当前光标下的单词

\subsection{命令}
\item 用 `esc` 返回 Normal mode
\item 可以用 tab 键补全命令(包括查看文件目录)
\item 可以用上下方向键根据命令历史补全(同 Matlab)
\item 也可以用 `q:` 打开一个窗口编辑历史命令, 用回车执行,按两次 `ctrl+c` 关闭
命令模式下,`ctrl+r` 快捷键可以插入许多内容. `ctrl+r %` 可以插入当前文件名(含相对目录)

\subsection{多窗口}
\item 参考 `:help usr_08.txt`
\item `:split` 将当前窗口复制一份, 选中上方窗口
\item `:10split` 指定新增窗口行数
\item `:split [file]` 在新窗口打开文件
\item `:only` 关闭其他所有窗口
\item `:new` 创建新文件,在新窗口打开
\item `ctrl+w w` 切换窗口, `ctrl+w ctrl+w` 也一样(ctrl 按着连按两个 w)
\item 貌似按下 `ctrl+w` 以后, `ctrl` 都可以一直按着也没关系, 例如 `ctrl+w` 后 `ctrl` 不松手连续按 `wj, wk`
\item `ctrl+w +` 或 `ctrl+w -` 用于调整窗口大小, 例如 `10ctrl+w -`
\item `10 ctrl+w _` 指定大小为 10 行
\item `:q` 仅退出当前窗口, `:qa` 退出所有窗口
\item `:vsplit` 可以让窗口左右划分
\item `:vnew` 可以就是左右划分的 `new`
\item `ctrl+w` 后加上 `hjkl` 可以切换到指定方向的窗口
\item `ctrl+w` 后加上 `t` 或者 `b` 可以跳到最上方和最下方的窗口
\item 打开 vim 时用 `vim -o 1.txt 2.txt 3.txt` 可以每个文件开一个窗口

\subsection{多 Tab}
\item 用 `:tabnew` 新建一个 tab
\item `:tabf 文件名` 新建一个 tab 用于打开某文件
\item `q` 退出当前 tab
\item `:tabn` 和 `:tabp` 用于切换到后一个/前一个 tab

\subsection{设置}
\item Tab 默认占七个格, 可以在 .vimrc 文件中设置 `set ts=4`

\subsection{目录}
\item 当前目录可以用 `:pwd` 查看
\item `:cd` 可以更改当前目录

\subsection{文件搜索}
\item 用 `find: **/文件名` 可以搜索当前目录和子目录下的文件名并打开(但如果有太多 match 就会出错)

\subsection{vimrc}
\item 修改 vimrc 以后, 使用 `:source ~/.vimrc` 就可以生效, 不需要重启 vim

\subsection{旧}
\item 要养成经常保存的习惯,尤其是用 ssh
\item home, end, page up, page down 都可以正常使用
\item 用 `:e <filename>` 打开文件.
\item `gd` 跳到光标所在变量的声明处, 其实是搜索最早出现的地方.
\item vim 中可以查看正在被程序写入的文件, 但不要修改! `:edit` 命令可以刷新
\item linux 的 `file -bi <filename>` 命令可以检测文件的 encoding,注意 UTF8 文件首位要有 BOM (Byte Order Mark) 该命令才能检测到 UTF8.
\item vim 默认的 encoding 是 latin1,或者系统 environment variable 的设置. 如果是 UTF8, 有时候不会自动加 BOM, 需要在 vim 里面用 `:set bomb` 才行, BOM 有时候会产生一些问题, 例如 gcc (GNU compiler collection) 不接受 BOM, 又例如 cat 两个文件的时候第二个文件的 BOM 也会复制到第一个文件的结尾.
\item 要检查当前打开文件被识别的 encoding,用 `:set fileencoding`.
\item 要显示行号,用 `:set number`. 要隐藏,用 `:set nonumber`. 如果要默认显示行号,在 `~/.vimrc` 文件中输入 `set number` 即可

\subsection{NERDtree}
\item `?` 可以查看简单的说明
\item `u` 可以将 root 移动到上一层
\item `U` 相当于 `u` 但是保持打开的子目录
\item `C` 将选中的目录作为 root
\item `q` 退出
