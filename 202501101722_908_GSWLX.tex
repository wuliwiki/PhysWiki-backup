% 干涉 (物理学)(综述)
% license CCBYSA3
% type Wiki

本文根据 CC-BY-SA 协议转载翻译自维基百科\href{https://en.wikipedia.org/wiki/Wave_interference}{相关文章}。

对于无线电通信中的干扰,请参阅《干扰(通信)》。
“干涉图样”在此重定向。有关莫尔条纹,请参阅《莫尔条纹》。对于医学术语,请参阅《干涉图样(肌电图)》
\begin{figure}[ht]
\centering
\includegraphics[width=10cm]{./figures/1379a86de5dd6781.png}
\caption{两波的干涉。相位相同:两个较低的波组合(左侧面板), resulting in a wave of added amplitude(建设性干涉)。相位相反:(这里是180度),两个较低的波组合(右侧面板), resulting in a wave of zero amplitude(破坏性干涉)。} \label{fig_GSWLX_1}
\end{figure}
在物理学中,干涉是指两种相干波通过考虑它们的相位差,将它们的强度或位移相加的现象。 如果两波处于相位相同或相反的状态,所产生的波可能具有更大的强度(建设性干涉)或较小的振幅(破坏性干涉)。干涉效应可以在所有类型的波中观察到,例如光波、无线电波、声波、表面水波、引力波或物质波,以及扬声器中的电波。
\begin{figure}[ht]
\centering
\includegraphics[width=10cm]{./figures/f6e23b7411ed2318.png}
\caption{湖面上的干涉水波} \label{fig_GSWLX_2}
\end{figure}
\subsection{词源}  
“干涉”一词源自拉丁语单词 \textbf{inter},意思是“之间”,以及 \textbf{fere},意思是“撞击或打击”,该词在波的叠加上下文中由托马斯·杨于1801年首次使用。[1][2][3]
\subsection{机制}
波的叠加原理指出,当两个或更多相同类型的传播波在同一点相遇时,该点的合成振幅等于各个波的振幅的矢量和。[4] 如果一个波的波峰与另一个同频率波的波峰在同一点相遇,那么振幅就是各个振幅的总和——这就是建设性干涉。如果一个波的波峰与另一个波的波谷相遇,那么振幅等于各个振幅的差——这就是破坏性干涉。在理想介质中(如水、空气几乎是理想介质),能量始终是守恒的,在破坏性干涉的点,波的振幅互相抵消,能量会重新分布到其他区域。例如,当两颗小石子掉进池塘时,会观察到一定的波纹图案,但最终波动会继续传播,只有当波到达岸边时,能量才会从介质中被吸收。