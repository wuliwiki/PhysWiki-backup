% 路径积分(量子力学)
% 传播子|跃迁振幅|关联振幅|概率振幅|高等量子力学

\pentry{传播子(量子力学)\upref{PpgtQM},薛定谔绘景和海森堡绘景\upref{HsbPic}}

% 定义传播子$C_i(t) = \bra{\bvec{x}_{i+1}}\E^{-\I Ht}\ket{\bvec{x}_i}$.




\subsection{概念的引入}

为了方便,我们考虑二维时空的情况,即空间只有一维.

在初始时刻$t=0$时,一个粒子处于$x_0$位置,将它的态记为$\ket{x_0}$,其在位置空间的波函数为$\psi_0(x)=\braket{x}{x_0}=\delta(x-x_0)$.

时间过去$t_1$后,我们在$x_1$位置测量,发现粒子的概率振幅为$\bra{x_1}\E^{-\I H t_1}\ket{x_0}$.因此我们可以说,粒子在时间$t_1$后“出现”在$x_1$的概率密度是$\abs{\bra{x_1}\E^{-\I H t_1}\ket{x_0}}^2$.

同样地,时间过去$t_2>t_1$后,粒子在$x_2$位置的概率振幅为$\bra{x_2}\E^{-\I H t_2}\ket{x_0}$.

注意到$\int \ket{x_1}\bra{x_1}\dd x_1=1$,即恒等变换(其矩阵总是单位矩阵),因此我们可以把这个积分插入到任何位置,比如:
\begin{equation}\label{PIntQM_eq1}
\ali{
    \bra{x_2}\E^{-\I H t_2}\ket{x_0} &= \bra{x_2}\int \ket{x_1}\bra{x_1}\dd x_1 \E^{-\I H t_2}\ket{x_0}\\
    &= \int \braket{x_2}{x_1}\bra{x_1} \E^{-\I H t_2}\ket{x_0}\dd x_1\\
    &= \int \bra{x_2}\E^{-\I H t_2-t_1}\ket{x_1}\bra{x_1} \E^{-\I H t_1}\ket{x_0}\dd x_1\\
}
\end{equation}

\autoref{PIntQM_eq1} 数学上成立,但它有什么物理意义呢?

每个$\bra{x_2}\E^{-\I H t_2-t_1}\ket{x_1}\bra{x_1} \E^{-\I H t_1}\ket{x_0}$





















