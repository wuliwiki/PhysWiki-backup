% 格奥尔格·康托尔(综述)
% license CCBYSA3
% type Wiki

本文根据 CC-BY-SA 协议转载翻译自维基百科\href{https://en.wikipedia.org/wiki/Georg_Cantor}{相关文章}。

\begin{figure}[ht]
\centering
\includegraphics[width=6cm]{./figures/409c043d8a08ce28.png}
\caption{} \label{fig_Canto_1}
\end{figure}
乔治·费迪南德·路德维希·菲利普·康托尔(Georg Ferdinand Ludwig Philipp Cantor,/ˈkæntɔːr/ KAN-tor;德语发音:[ˈɡeːɔʁk ˈfɛʁdinant ˈluːtvɪç ˈfiːlɪp ˈkantoːɐ̯];1845年3月3日(旧历2月19日)-1918年1月6日)是一位数学家,他在集合论的创建中发挥了关键作用,集合论已成为数学中的一项基本理论。康托尔确立了两个集合成员之间一对一对应的重要性,定义了无限集合和良序集合,并证明了实数比自然数更多。康托尔证明该定理的方法意味着存在无数个不同大小的无限集合。他定义了基数和序数以及它们的算术运算。康托尔的工作在哲学上具有重大意义,他对此非常清楚。

最初,康托尔的超无限数理论被认为是反直觉的——甚至是震惊的。这使得它遭遇了数学界 contemporaries 的抵制,如利奥波德·克罗内克和亨利·庞加莱[3],以及后来的赫尔曼·外尔和L·E·J·布劳威尔,而路德维希·维特根斯坦则提出了哲学上的反对意见;参见康托尔理论的争议。康托尔是一个虔诚的路德宗基督徒[4],他认为这一理论是上帝传达给他的[5]。一些基督教神学家(尤其是新经院哲学家)认为康托尔的工作挑战了上帝本质中绝对无限的独特性[6]——曾有一次将超无限数理论与泛神论等同起来[7]——这一命题被康托尔坚决拒绝。然而,并非所有神学家都反对康托尔的理论;著名的新经院哲学家康斯坦丁·古特贝尔特支持这一理论,而枢机主教约翰·巴普蒂斯特·弗朗泽林在康托尔做出一些重要澄清后也接受了这一理论作为有效理论[8]。

对康托尔工作的反对有时非常激烈:利奥波德·克罗内克的公开反对和个人攻击包括将康托尔描述为“科学江湖医生”、“叛徒”和“青年堕落者”[9]。克罗内克反对康托尔证明代数数是可数的,以及超越数是不可数的,这些结果如今已被纳入标准数学课程中。维特根斯坦在康托尔去世几十年后写道,他感叹数学“完全被集合论的有害习语所支配”,并将其斥为“完全的胡说八道”,“可笑”且“错误”[10]。从1884年到他生命的尽头,康托尔反复遭遇抑郁症,这被归咎于许多同时代人对他的敌对态度[11],尽管也有人将这些症状解释为双相情感障碍的可能表现[12]。

激烈的批评与后来的赞誉相匹配。1904年,皇家学会授予康托尔西尔维斯特奖,这是它能授予数学工作者的最高荣誉[13]。大卫·希尔伯特为其辩护,宣称:“没有人能够将我们从康托尔创造的乐园中驱逐出去”[14][15]。
\subsection{传记}  
\subsubsection{青年时期与学业}
\begin{figure}[ht]
\centering
\includegraphics[width=6cm]{./figures/b8db46daef6392b0.png}
\caption{} \label{fig_Canto_2}
\end{figure}