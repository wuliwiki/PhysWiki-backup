% 利用复数方法证明三角恒等式
% 复数|三角恒等式

\begin{issues}
\issueTODO
\end{issues}

\pentry{复数\upref{CplxNo}}

\subsection{理论基础}
要借助复数证明三角恒等式,我们一般需要构造具有以下形式的复数:
$$w=\cos \alpha+i\sin\alpha$$

这类复数具有许多好的性质,我们熟知的有:
$$(\cos\alpha+i\sin\alpha)(\cos\beta+i\sin\beta)=\cos(\alpha+\beta)+i\sin(\alpha+\beta)\\\tag{1} $$
$$
(\cos\alpha+i\sin\alpha)^n=\cos n\alpha+i\sin n\alpha\\\tag{2}$$

此外,我们再引入另外两条常用的性质:

$\displaystyle{\begin{eqnarray*} 1-(\cos\alpha+i\sin\alpha)^n&=&1-\left(\cos{\frac{n\alpha}2+i\sin{\frac{n\alpha}2}}\right)^{2}\\ &=&1-\left(\cos^2{\frac{n\alpha}2}-\sin^2{\frac{n\alpha}2}+2i\sin{\frac{n\alpha}2}\cos{\frac{n\alpha}2}\right)\\ &=&2\sin^2{\frac{n\alpha}2}-2i\sin{\frac{n\alpha}2}\cos{\frac{n\alpha}2}\\ &=&-2i\sin{\frac{n\alpha}2}\left(\cos{\frac{n\alpha}2}+i\sin{\frac{n\alpha}2}\right) \end{eqnarray*}}$ 

故:
$$1-(\cos\alpha+i\sin\alpha)^n=-2i\sin{\frac{n\alpha}2}\left(\cos{\frac{n\alpha}2}+i\sin{\frac{n\alpha}2}\right)\\\tag{3}$$

同理可得:
$$
1+(\cos\alpha+i\sin\alpha)^n=2\cos{\frac{n\alpha}2}\left(\cos{\frac{n\alpha}2}+i\sin{\frac{n\alpha}2}\right)\\\tag{4}$$

\subsection{累加}
$$\begin{eqnarray*}\sum_{k=0}^n\sin(\alpha+k\beta)&=&\frac{\sin\left(\alpha+\frac{n\beta}2\right)\sin\left(\frac{n+1}2\beta\right)}{\sin\frac{\beta}2}\\ \sum_{k=0}^n\cos(\alpha+k\beta)&=&\frac{\cos\left(\alpha+\frac{n\beta}2\right)\sin\left(\frac{n+1}2\beta\right)}{\sin\frac{\beta}2} \end{eqnarray*}\\\tag{}$$

\textbf{证明:}设 $w_1=\cos \alpha+i\sin\alpha\,,w_2=\cos \beta+i\sin\beta$ ,则:
$$\begin{eqnarray*} \sum_{k=0}^n\cos(\alpha+k\beta)+i\sum_{k=0}^n\sin(\alpha+k\beta)&=&\sum_{k=0}^n\left[\cos(\alpha+k\beta)+i\sin(\alpha+k\beta)\right]\\&=&w_1\sum_{k=0}^nw_2^k \\&=&w_1\left(\frac{1-w_2^{n+1}}{1-w_2}\right)\\ &=&(\cos \alpha+i\sin\alpha)\frac{-2i\sin{\frac{(n+1)\beta}2}\left[\cos{\frac{(n+1)\beta}2}+i\sin{\frac{(n+1)\beta}2}\right]}{-2i\sin{\frac{\beta}2}\left(\cos{\frac{\beta}2}+i\sin{\frac{\beta}2}\right)}\\ &=&\frac{\cos\left(\alpha+\frac{n\beta}2\right)\sin\left(\frac{n+1}2\beta\right)}{\sin\frac{\beta}2}+i\frac{\sin\left(\alpha+\frac{n\beta}2\right)\sin\left(\frac{n+1}2\beta\right)}{\sin\frac{\beta}2} \end{eqnarray*}$$

对比虚实部,即证

令 $\alpha=\beta$ ,得:
$$\sum_{k=1}^n\sin k\alpha=\frac{\sin{\frac{(n+1)\alpha}2}\sin{\frac{n\alpha}{2}}}{\sin{\frac{\alpha}2}}$$ $$\sum_{k=1}^n\cos k\alpha=\frac{\cos{\frac{(n+1)\alpha}2}\sin{\frac{n\alpha}{2}}}{\sin{\frac{\alpha}2}}$$ 

可以推知:

$\displaystyle{\cos\frac{\pi}{7}+\cos\frac{3\pi}{7}+\cos\frac{5\pi}{7}=\frac{1}{2}}$

$\displaystyle{\cos\frac{\pi}{9}+\cos\frac{5\pi}{9}+\cos\frac{7\pi}{9}=0}$

