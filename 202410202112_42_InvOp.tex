% 逆算子
% keys 逆算子|可逆性
% license Usr
% type Tutor

\pentry{拓扑线性空间中的线性算子\nref{nod_TLinO}}{nod_d7fd}

逆算子是双射的算子的逆。当算子可逆时,在某些情形下逆算子和算子有很多相同的性质,比如,线性算子的逆算子时线性的,而完备赋范空间之间的线性有界算子的逆是有界的。


\begin{definition}{逆算子}
设 $D_A$ 是拓扑线性空间的子集,$A:D_A\rightarrow E_1$ 是其上的算子(映射),$\Im A$ 是 $A$ 的象。其对任意 $y\in\Im A$,方程
\begin{equation}
Ax=y~
\end{equation}
 有唯一解,则称算子 $A$ \textbf{可逆}。此时映射
 \begin{equation}
 \Im A\rightarrow E,Ax\mapsto x~  
 \end{equation}
 称为 $A$ 的\textbf{逆算子},记作 $A^{-1}$。

\end{definition}

\subsection{性质}
下面定理表明,线性算子的逆算子是线性的。

\begin{theorem}{}
\enref{线性算子}{TLinO} $A$ 的逆算子 $A^{-1}$ 是线性的。
\end{theorem}

\textbf{证明:}
任意 $\Im A$ 中的 $y_1=A x_1,y_2=Ax_2$,因为
\begin{equation}
A(\alpha x_1+\beta x_2)=\alpha Ax_1+\beta Ax_2=\alpha y_1+\beta y_2.~
\end{equation}
所以
\begin{equation}
A^{-1}(\alpha y_1+\beta y_2)=\alpha x_1+\beta x_2=\alpha A^{-1}y_1+\beta A^{-1}y_2.~
\end{equation}


\textbf{证毕!}

下面定理表明,在完备赋范空间(Banach空间)之间的线性有界算子的逆算子是有界的。

\begin{theorem}{逆算子的Banach定理}
设 $E,E_1$ 是\enref{完备赋范空间}{banach},$A$ 是 $E$ 到 $E_1$ 的可逆线性算子,则其逆算子 $A^{-1}$ 有界。
\end{theorem}

为证明它,需要如下引理。
\begin{lemma}{}
设 $M$ 是Banach空间 $E$ 中的\enref{处处稠密集}{MaDen}。则任意非零元 $y\in E$ 可展开成级数:
\begin{equation}
y=y_1+\cdots+y_n+\cdots,~
\end{equation}
其中 $y_k\in M$ 且 $\norm{y_k}\leq 3\norm{y}/2^k$。
\end{lemma}
\textbf{证明:}下面用逐次构造 $y_k$ 进行证明。选择 $y_1$ 使得
\begin{equation}
\norm{y-y_1}\leq\norm{y}/2.~
\end{equation}
这是一个半径为 $\norm{y}/2$ 中心在 $y$ 的球,由于 $M$ 在 $E$ 中处处稠密,因此在该球内任一点的邻域都有 $M$ 的点,因此这样的选择是可能的。

同样由于 $M$ 在 $E$ 中处处稠密,可选 $y_n$ 使得 $\norm{y-y_1-\cdots-y_n}\leq\norm{y}/2^n$。根据 $y^k$ 的选择,
\begin{equation}
\norm{y-\sum_{k=1}^n y_k}\rightarrow0.~
\end{equation}
即级数 $\sum_{k=1}^\infty y_k$ 收敛于 $y$。此外
\begin{equation}
\begin{aligned}
\norm{y_n}=&\norm{y_n+y_{n-1}+\cdots+y_1-y+y-y_1-\cdots-y_{n-1}}\\
\leq&\norm{y-y_1-\cdots-y_{n}}+\norm{y-y_1\cdots-y_{n-1}}\\
\leq&\norm{y}/2^n+\norm{y}/2^{n-1}=3\norm{y}/2^n.
\end{aligned}~
\end{equation}

\textbf{证毕!}




