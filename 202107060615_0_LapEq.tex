% 拉普拉斯方程
% 梯度|散度|拉普拉斯方程|偏微分方程|分离变量法

\begin{issues}
\issueDraft
\end{issues}

\pentry{梯度\upref{Grad}, 散度\upref{Divgnc}, 分离变量法解偏微分方程\upref{SepVar}}

设$\Omega\subset\mathbb{R}^n$是区域, $u$是$\Omega$上的实值函数. 如下偏微分方程称为\textbf{拉普拉斯方程(laplacian equation)}:
\begin{equation}\label{LapEq_eq6}
\Delta u = 0,
\end{equation}
这里$\Delta=\Nabla\vdot\Nabla=\partial_{x^1}^2+...+\partial_{x^n}^2$是梯度的散度, 也就是所谓的\textbf{拉普拉斯算子(Laplacian)}. 根据定义, 有分配律 $\Nabla \vdot (\Nabla u) = (\Nabla \vdot \Nabla) u$. 满足拉普拉斯方程的函数叫做\textbf{调和函数(harmonic function)}.

从物理上看, 二元函数的拉普拉斯方程可以理解为一片静止的, 不受外力的薄膜\upref{Wv2D}所满足的方程. 要得到方程的解, 我们需要规定一些边界条件. 常见的条件是给定一个区域, 然后给出 $u(\bvec r)$ 在边界上的函数值.

\begin{example}{}
在静电学问题中, 电势能满足泊松方程, 在没有电荷分布的区域, 泊松方程变为拉普拉斯方程
\addTODO{未完成: 引用相关【物理】词条}
\begin{equation}
\Delta V=0
\end{equation}
我们来计算$V$仅依赖于$r$($r$为到原点的距离)的情况.采用球坐标,拉普拉斯方程使用\autoref{SphLap_eq5}~\upref{SphLap} 写为
\begin{equation}
\frac{1}{r^2}\dv{}{r}\qty(r^2\dv{V}{r})=0
\end{equation}
则
\begin{equation}
r^2\dv{V}{r}=c\Rightarrow \dv{V}{r}=\frac{c}{r^2}
\Rightarrow V(r)=-\frac{c}{r}+c'
\end{equation}
其中,$c$、 $c'$为常数.
\end{example}
%\addTODO{使用分离变量法求解一些简单例题, 参考\cite{GriffE}. 1. 当边界值为常数时, $u$ 是常函数. 2. 长方形边界三边是 $c_1$, 一边是 $c_2$ 的解.}
\begin{example}{二维长方形边界问题}
\addTODO{图}
如图,二维平面上有一长方形区域,尺寸为 $a\times b$,该区域存在一标量场 $u(x,y)$,且满足
\begin{equation}\label{LapEq_eq1}
\pdv[2]{u}{x}+\pdv[2]{u}{y}=0
\end{equation}
边界条件为
\begin{equation}
u(0,y)=u(a,y)=u(x,0)=0,\quad u(x,b)=c
\end{equation}
求 $u(x,y)$

解:设 $u(x,y)=X(x)Y(y)$,代入\autoref{LapEq_eq1} ,得
\begin{equation}\label{LapEq_eq2}
\frac{X''}{X}+\frac{Y''}{Y}=0
\end{equation}
式中, $X''=\dv[2]{X}{x},Y''=\dv[2]{Y}{y}$.
\autoref{LapEq_eq2} 左边两项各为 $x$ 和 $y$ 的函数,而其和为常数0,这说明这两项只能是常数,令
\begin{equation}
\frac{X''}{X}=-k_x^2,\quad \frac{Y''}{Y}=-k_y^2
\end{equation}
由\autoref{LapEq_eq2} 
\begin{equation}\label{LapEq_eq3}
k_x^2+k_y^2=0
\end{equation}
首先解 $X(x)$
\begin{equation}
X''+k_x^2X=0
\end{equation}
其通解为
\begin{equation}
X(x)=a_1\sin k_x x+b_1\cos k_x x
\end{equation}
边界条件 $u(0,y)=u(a,y)=0$相当于 $X(0)=X(a)=0$,代入上式,得
\begin{equation}
b_1=0,\quad k_x=\frac{n\pi}{a} \qquad (n\in \mathbb{Z})
\end{equation}
所以
\begin{equation}
X_n(x)=a_1\sin\frac{n\pi x}{a} \qquad (n\in \mathbb{Z})
\end{equation}
再解 $Y(y)$
\begin{equation}
Y''+k_y^2Y=0
\end{equation}
同样,其通解为
\begin{equation}
Y(y)=a_2\sin k_y y+b_2\cos k_y y
\end{equation}
由\autoref{LapEq_eq3} ,$k_y=\I k_x$,代入上式,并由\autoref{CTrig_eq5}~\upref{CTrig} 和\autoref{CTrig_eq6}~\upref{CTrig},得
\begin{equation}
Y(y)=\I a_2\sinh k_x y+b_2\cosh k_x y
\end{equation}
同样,边界条件 $u(x,0)=0$ 相当于 $Y(0)=0$,代入上式,得
\begin{equation}
Y_n(y)=\I a_2\sinh\frac{n\pi y}{a} \qquad (n\in \mathbb{Z})
\end{equation}
所以
\begin{equation}
\begin{aligned}
u_n(x,y)&=X_n(x)Y_n(y)=\I a_1a_2\sin\frac{n\pi x}{a}\sinh\frac{n\pi y}{a}\\
&=C'_n\sin\frac{n\pi x}{a}\sinh\frac{n\pi y}{a} \qquad (n\in\mathbb{Z})
\end{aligned}
\end{equation}
由于 $k_x=0$ 的解没有意义(它意味着 $u(x,y)=0$),而且负数解不会给出新的解,因为 $\sin(-x)=-\sin x$,而负号可以合并到常数项去,所以可区分的解要求 $n\in\mathbb{Z^+}$.

由线性叠加原理
\begin{equation}\label{LapEq_eq5}
u(x,y)=\sum_n C''_n u_n(x,y)=\sum_n C_n\sin\frac{n\pi x}{a}\sinh\frac{n\pi y}{a} \qquad  (n\in\mathbb{Z^+})
\end{equation}
现在,我们还差边界条件 $u(x,b)=c$ 未用,代入上式
\begin{equation}
c=\sum_n C_n\sin\frac{n\pi x}{a}\sinh\frac{n\pi b}{a}=\sum_n B_n\sin\frac{n\pi x}{a}(n\in\mathbb{Z^+})
\end{equation}
其中,$B_n=C_n\sinh \frac{n\pi b}{a}$.
利用三角函数的正交性
\begin{equation}
\int_0^a\sin\frac{n\pi x}{a}\sin\frac{m\pi x}{a}\dd x=\left\{\begin{aligned}
\frac{a}{2},\quad  n=m\\
0,\quad n\neq m
\end{aligned}\right.
\end{equation}
得
\begin{equation}
\begin{aligned}
B_n&=\frac{2c}{a}\int_0^a\sin\frac{n\pi x}{a}\dd x=\frac{2c}{n\pi}(1+(-1)^{n+1})\\
&=\left\{\begin{aligned}
0,&\quad n=2,4,\cdots\\
\frac{4c}{n\pi},&\quad n=1,3,5,\cdots
\end{aligned}\right.
\end{aligned}
\end{equation}
所以
\begin{equation}\label{LapEq_eq4}
C_n=\frac{B_n}{\sinh \frac{n\pi b}{a}}=\frac{4c}{n\pi\sinh \frac{n\pi b}{a}}(n=1,3,5,\cdots)
\end{equation}
\autoref{LapEq_eq4} 代入\autoref{LapEq_eq5} ,得到
\begin{equation}
u(x,y)=\frac{4c}{\pi}\sum_{n=1,3,5,\cdots} \frac{1}{n\sinh \frac{n\pi b}{a}}\sin\frac{n\pi x}{a}\sinh\frac{n\pi y}{a}
\end{equation}
\end{example}
