% CMake 笔记
% license Xiao
% type Note


\pentry{Makefile 笔记\nref{nod_Make}}{nod_719b}

参考\href{https://cmake.org/cmake/help/latest/guide/tutorial/index.html}{官方教程}。

\subsection{常识}
\begin{itemize}
\item 在 Windows 下可直接使用 GUI, linux 命令行中使用 \verb`ccmake` 可以有 TUI。 否则就用 \verb`cmake`
\item 在 \verb`CMakeLists.txt` 的路径下, \verb`cmake .` 生成 \verb`Makefile`, 然后 \verb`make -j` 多线程编译。 \verb`make VERBOSE=1 -j` 输出编译命令。 cmake 会在当前路径生成 \verb`Makefile` 和一些临时文件和文件夹, 清理麻烦, 所以还是建议输出到子文件夹。
\item 如果想要 \verb`make` 命令默认 \verb`VERBOSE=1`,可以在之前用 \verb`cmake -DCMAKE_VERBOSE_MAKEFILE=ON`
\item \verb`CMakeLists.txt` 中, 注释如 \verb`#[[一些注释]]` 或者 \verb`#一些注释` 直到行末
\item 如果不想改变源码路径, 就用在别的文件夹 \verb`cmake CMakeList的路径` 即可。
\item \verb`cmake -G 生成器 CMakeList的路径` 可以指定生成器, \verb`生成器` 如 \verb`Unix Makefiles`, \verb`MinGW Makefiles`, \verb`Ninja`
\item ninja 的用法和 make 一样, \verb`ninja` 编译(默认使用当前路径的 \verb` build.ninja` 文件), \verb`ninja install` 安装
\item \verb`cmake -L CMakeList的路径` 列出编译选项, \verb`-LH` 列出选项以及帮助说明。 \verb`-LAH` 列出所有选项(会多出很多 \verb`CMAKE_` 开头的, 不是作者提供的选项), 用 \verb`cmake -D 选项1=值1 -D 选项2=值2 CMakeList的路径` 来设置选项(\verb`-D` 后面可以没有空格)。
\item 要指定路径, 用 \verb`cmake -S CMakeList的路径 -B 生成Makefile的路径`, 然后再到 Makefile 的路径 \verb`make -j12` 即可。 cmake 会自动生成 \verb`生成Makefile的路径`。 如果要清理 cmake 生成的文件以及所有编译出来的文件, 直接把第二个路径清空即可。
\item 当然, 也不是什么平台都可以用 \verb`make`, 所以也可以用 \verb`cmake --build 生成Makefile的路径 [--target 可执行文件] -- -j 12` 其中 \verb`--` 后面的选项会传给 \verb`make` 或者别的具体用于编译的程序。
\item Cmake 的基本语法是 \verb`命令(arg1 arg2 ...)` 其中 \verb`命令` 不区分大小写, \verb`arg` 用空格隔开也可以换行, 如果 \verb`arg` 本身有空格, 用双引号即可
\item \verb`cmake` 中的变量本质上都是字符串。 即使是 list, 也不过是特殊的字符串 \verb`str1;str2;str3`。 所谓的 \verb`bool` 类型不过是 \verb`ON` 和 \verb`OFF` 字符串。 数字也一样,并不是真正的数字。
\item 获取变量值用 \verb`${变量}`。 注意不能用 \verb`$变量`。
\item 如果字符串 literal 中没有空格或者特殊符号,两边的双引号可以省略。 例如 \verb`set(my_var abc123)`。
\item \verb`message(STATUS "...")` 可以输出到 stdout。 如果 \verb`message(STATUS ${变量})` 中的变量是 list, 那么输出中 \verb`;` 会消失。 应该用 \verb`message(STATUS "${变量}")`, 输出的格式为 \verb`str1;str2;str3`。 \verb`message(STATUS ${变量1} ${变量2} ...)` 相当于把若干变量合成 list。
\item \verb`message(FATAL_ERROR "字符串")` 会输出错误信息并立即退出。
\item Cmake 3.0 开始支持 raw string, 格式为 \verb`[=[字符串]=]` 等号可以有一个也可以有多个。
\item \verb`set` 可以对变量赋值, 例如 \verb`set(变量 123)`。
\item 变量是区分大小写的。
\item \verb`set` 也可以把许多变量变成一个 \textbf{list}, 如 \verb`set(Foo aaa bbb ccc)` 等效于 \verb`set(Foo "aaa" "bbb" "ccc")` 等效于 \verb`set(Foo "aaa;bbb;ccc")` 等效于 \verb`set(Foo aaa;bbb;ccc)`。
\item 把 list 作为 \verb`命令()` 的参数相当于多个 \verb`arg`。 例如 \verb`命令(${Foo})` 相当于 \verb`命令(aaa bbb ccc)`。
\item 所以 \verb`命令("${Foo}")` 相当于 \verb`命令("aaa;bbb;ccc")` 也就是说命令的第一个参数是一个 list。
\item linux 环境变量和 windows 注册表变量可以直接作为变量使用, 格式如 \verb`$EVN{变量}`。 例子: \verb`message(STATUS $ENV{PWD})`。 注意 cmake 看到的环境变量未必和 bash 中的相同, 可以用 \verb`cmake -E environment` 检查。
\item \verb`separate_arguments(变量)` 可以把一个含有若干空格的字符串拆分成 list (即把空格替换为分号)。
\end{itemize}

\subsection{常用命令}
\begin{itemize}
\item (可以学习一下 Eigen 的 CMakeList.txt)
\item \verb`cmake_minimum_required(VERSION x.x)` 要求 cmake 最小版本(必须有)
\item \verb`project(project_name)` 指定项目名称(必须有)
\item \verb`set(CMAKE_CXX_COMPILER "/usr/bin/g++")` 设置 compiler
\item \verb`set(CMAKE_CXX_FLAGS "-选项 -选项")` 设置 flag, 包括 C++ 标准。 注意这不是一个 list。 如果要在后面添加选项, 就用 \verb`set(CMAKE_CXX_FLAGS "${CMAKE_CXX_FLAGS} -选项1 -选项2")`。
\item 用 \verb`set(CMAKE_CXX_STANDARD 11)` 设置 C++ 标准
\item 用 \verb`set(CMAKE_CXX_STANDARD_REQUIRED ON)` (默认关)如果编译器不支持该标准就报错
\item \verb`set(CMAKE_CXX_EXTENSIONS OFF)` (默认开)不使用 C++ 标准拓展, 例如 \verb`-std=gnu11`。 后者并不完全兼容标准,一些情况会出错。
\item 添加 debug 模式的 flag 如 \verb`set(CMAKE_CXX_FLAGS_DEBUG "${CMAKE_CXX_FLAGS_DEBUG} -fsanitize=address")` 或者下面的 \verb`list(APPEND ...)`
\item 同理, \verb`CMAKE_CXX_FLAGS_RELEASE` 可以设置 release 模式下的 flag。
\item \verb`set(变量 str1 [str2] ...)` 对变量赋值(若有多个值就生成 list)
\item \verb`list(APPEND 变量 str1 [str2] ...)` 在变量(list)后面添加元素
\item \verb`list(REMOVE_ITEM 变量 str1 ...)` 从 \verb`变量` 的 list 中移除(如果不存在也不报错)
\item \verb`set(变量 "${变量} str1 str2")` 等效于 \verb`list(APPEND ...)`。
\item 使用一个不存在的变量相当于空字符串。
\item \verb`file(GLOB 变量 "folder/*.cpp" "fname2.cpp")` 可以列出所有 \verb`"folder/*.cpp"` 文件, 以及 \verb`fname2.cpp` (如果存在) 赋值给 \verb`变量`。 注意文件名包括绝对路径。 match 的结果也会存到 \verb`CMAKE_MATCH_编号`, 编号从 1 开始。
\item \verb`get_filename_component(变量 "文件名含路径" NAME或DIRECTORY)` 会从 \verb`"文件名含路径"` 中提取文件名或路径, 赋值给\verb`变量`。
\item \verb`file(READ 文件名 变量)` 读取文件内容到变量。
\item \verb`string(REGEX MATCH "...(...)..." 变量 字符串变量`。 在\verb`字符串变量`中匹配正则表达式, 并把 \verb`()` 中的匹配结果赋值给 \verb`变量`。
\item \verb`string(REPLACE "老词" "新词" 变量 "字符串")` 做字符串替换。
\item \verb`add_definitions(-D FOO -D BAR)` 给编译器添加宏定义
\item \verb`add_compile_options(-Wall -fopenmp)` 给编译器添加选项(注意不能用 \verb`-D` 定义宏), 该命令不受 \verb`CMAKE_BUILD_TYPE` 的影响。
\item 注意不同编译器有不同 option,所以可以判断语句给不同编译器 \verb`CMAKE_CXX_COMPILER_ID` 加不同 option
\begin{lstlisting}[language=cmake]
# Define flags for different compilers
if(CMAKE_CXX_COMPILER_ID STREQUAL "GNU"
    OR CMAKE_CXX_COMPILER_ID STREQUAL "Clang")
    # GCC and Clang specific flags
    add_compile_options(
        -Wall                 # Enable all warnings
        # Disable warning for constructor initialization order
        -Wno-reorder
        # Disable warning for misleading indentation
        -Wno-misleading-indentation
        -O3                   # High-level optimization
        -fopenmp              # Enable OpenMP
        -fmax-errors=1        # Limit the number of errors
        -D NDEBUG             # Disable debugging macros
    )
elseif(CMAKE_CXX_COMPILER_ID STREQUAL "MSVC")
    # MSVC specific flags
    add_compile_options(
        /W4                  # Enable high warning level
        /O2  # High-level optimization (equivalent to -O3 in g++)
        /openmp              # Enable OpenMP
        /DNDEBUG             # Disable debugging macros
    )
endif()
\end{lstlisting}
\item 另外还可以用 \verb`CMAKE_SYSTEM_NAME` 判断当前的系统
\begin{lstlisting}[language=cmake]
if(CMAKE_SYSTEM_NAME STREQUAL "Linux")
    ....
elseif(CMAKE_SYSTEM_NAME STREQUAL "Windows")
    ....
elseif(CMAKE_SYSTEM_NAME STREQUAL "Darwin")
    ....
endif()
\end{lstlisting}
\item cmake 还提供了关于系统的更多具体变量如 \verb`MINGW`,\verb`CYGWIN`, \verb`MSYS`, \verb`IOS`, 等(\href{https://cmake.org/cmake/help/latest/manual/cmake-variables.7.html}{文档})
\item \verb`add_executable(目标名 源码1 源码2)` 前者是可执行文件名, 后者是所有需要 link 的 \verb`c/C++` 文件名,不需要头文件。 cmake 会自动分析代码得到哪个 cpp 调用哪个头文件, 如果头文件改变了, 只有调用它的 cpp 会重新编译。 注意 \verb`目标名` 可以是任意名字不一定是可执行文件名。
\item 要指定可执行文件的名字, 用 \verb`set_target_properties(exe_targ_name PROPERTIES OUTPUT_NAME 可执行文件的名字)`
\item 一般来说 \verb`add_executable` 指定的源码不需要包含头文件, 但如果头文件需要由其他 rule 来生成, 那就需要包含。 所以最保险还是把所有头文件都包含。
\item \verb`configure_file(file_in file_out)` 将文件中 \verb`@变量@` 替换为变量 \verb`变量` 的字符串。
\item \verb`include_directories(dir1 [dir2] ...)` 添加头文件的搜索路径
\item \verb`add_library(exe_name source_name)` 单独编译一个 library (在 library 路径的 \verb`CMakeLists.txt` 中使用这个命令而不是 \verb`add_executable` 命令)
\item \verb`link_directories(路径)` 相当于编译器的 \verb`-L` 选项, 添加静态或动态 library 的搜索路径。
\item \verb`set(CMAKE_INSTALL_RPATH 路径)` 相当于设置 \verb`rpath` (到底是 \verb`RUNPATH` 还是 \verb`RPATH`?)
\item \verb`add_subdirectory(dir1 [dir2] ...)` 执行子路径中的 \verb`CMakeLists.txt`。 往往用于编译某个 library。
\item \verb`target_link_libraries(exe_name lib1 [lib2] ...)` link 阶段链接 library, 相当于 \verb`-l lib1 -l lib2`
\item \verb`option(opt_name description default)` 定义一个 option 开关(会在 cmake 的 GUI 中显示开关)以及默认值。 \verb`opt_name` 可以在 \verb`CMakeLists.txt` 中的 \verb`if` 语句中使用例如 \verb`if(opt_name) ... endif(opt_name)`。 \verb`default` 可以是 \verb`ON` 或 \verb`OFF`。
\item 注意更新 \verb`option` 的默认值后, 需要清空 cmake 的临时文件才可以生效。
\item 如果需要其他类型的 option, 可以用 cached variable: \verb`set(变量 "默认值" CACHE BOOL或STRING或FILEPATH或PATH或INTEGER "描述")`。 如果之前 \verb`变量` 已经被赋值(例如命令行 \verb`cmake -D 变量=...`), 则值不会被覆盖。 这种变量叫做 \textbf{cached variable}。
\item 以上两种方式产生的变量都可以在 \verb`cmake -LH .` 的帮助中看到说明。
\item 在模板文件 \verb`*.in` 中, 当 \verb`opt_name` 为 \verb`ON` 时, 使用 \verb`configure_file` 命令后 \verb`#cmakedefine opt_name` 会被替换为 \verb`#define opt_name`。 如果想要给宏定义一个值, 用 \verb`#cmakedefine USE_FEATURE_B @USE_FEATURE_B@`
\item 【cmake 3.16 的新功能】 \verb`target_precompile_headers(可执行文件 PRIVATE foo.h bar.h)` 可以编译头文件, 注意貌似不会自动包括 .h 包括的文件。 如果头文件(或者依赖的文件)改了, 那么将会自动重新编译头文件。
\item \verb`execute_process(COMMAND 命令 参数1 参数2 ...)` 可以执行 shell 命令, 其中命令和参数中都可以用 \verb`${变量}`。 其他功能(包括命令返回的内容, 文件输入输出), 详见\href{https://cmake.org/cmake/help/latest/command/execute_process.html}{文档}。 注意这个命令是在运行 \verb`cmake` 时执行而不是运行 \verb`make` 时。 后者可以用 \verb`add_custom_target`, 再用 \verb`add_dependencies(目标, target)` 把 target 作为目标的依赖。
\item \verb`include(文件或者模块)`, 详见\href{https://cmake.org/cmake/help/latest/command/include.html}{文档}。 模块可以是 \verb`*.cmake` 的文件, 也可以是 cmake 的内建模块如 \verb`CheckCXXCompilerFlag`
\item \verb`macro(宏名 [参数1 参数2]) 一些命令 endmacro()` 定义宏, 调用如 \verb`ei_add_cxx_compiler_flag(参数1 参数2)`
\item \verb`check_cxx_source_compiles(代码变量 输出变量)`\href{https://cmake.org/cmake/help/latest/module/CheckCXXSourceCompiles.html}{(文档)} 可以判断某个源码是否可以编译成功。 在调用前, 可以用 \verb`CMAKE_REQUIRED_LIBRARIES` 变量设置所需的库, \verb`CMAKE_REQUIRED_FLAGS` 指定额外的编译器选项, 等。
\item \verb`check_cxx_compiler_flag(编译器选项 输出变量)` 可以检查某个编译器选项是否可用。 \verb`输出变量` 是 True 或 False。
\item 循环如 \verb`foreach(变量 IN 列表) ... endforeach()` 或者 \verb`foreach(变量 IN 变量1 变量2 ...) ... endforeach()` 其中 \verb`IN` 在 cmake 2.4 以后可以省略。 \href{https://cmake.org/cmake/help/latest/command/foreach.html}{文档}。
\item \verb`find_program(变量 程序名 [路径1 路径2 ...])` 可以在指定路径或者 \verb`$PATH` 中寻找某个可执行文件, 并把绝对路径写入\verb`变量`。 如果没有找到, 会赋空值。
\item \verb`add_custom_command(OUTPUT output_files COMMAND 命令... [ARGS 参数...] [WORKING_DIRECTORY dir] [DEPENDS 依赖文件] [COMMENT 注释])` 相当于 Makefile 中的一个 rule, 指定依赖关系, 以及使用的命令。 其中 \verb`命令` 可以包含完整的命令而不需要额外用 \verb`参数`。 当 \verb`OUTPUT` 的文件缺失时或者 \verb`依赖文件` 更新时, 就会运行命令。 可以把 \verb`命令` 和所有的参数写成一整个字符串, 但是 cmake 可能会自动把空格 escape 比较恶心。 可以试试用 \verb`string(REPLACE " " ";" 命令变量 ${命令变量})` 然后 \verb`COMMAND ${命令变量}` 把命令拆成一个 list。 \verb`注释` 会在该命令被运行时输出到 stdout。
\item 为了更好的跨平台, \verb`命令` 可以使用 \verb`${CMAKE_COMMAND} -E 一些命令`。 其中 \verb`${CMAKE_COMMAND}` 就是当前使用的 \verb`cmake` 程序, \verb`一些命令` 如 \verb`echo 字符串`,\verb`copy 文件 文件`, \verb`make_directory 目录`, \verb`remove_directory 目录`, \verb`remove 文件` 等。
\item \verb`add_custom_target(目标名 [COMMAND 命令...] [ARGS arguments...] [DEPENDS depends...] [BYPRODUCTS byproducts...] [WORKING_DIRECTORY dir] [SOURCES sources...] [COMMENT 注释])` 会在生成的 Makefile 中添加一个 target。
\item \verb`add_dependencies(目标名 依赖1 [依赖2 ...])` 可以给 target(例如可执行文件)添加依赖。
\item \verb`check_symbol_exists("符号名字" "路径/头文件1;头文件2" 变量)` 可以检查 \verb`符号名字` 是否有定义, 包括函数名, 变量名, 以及宏。
\item \verb`check_symbol_exists` 的实现使用了 \verb`try_compile()` 和 \verb`try_run()` 函数, 试图编译和运行一个测试程序, 从运行结果来判断是否存在符号。
\end{itemize}

\subsection{链接库}
\begin{itemize}
\item \verb`find_library()` 的默认搜索路径如 \verb`/usr/lib`,\verb`/usr/local/lib`, \verb`/usr/[local/]share/cmake/Modules/`, \verb`/usr/lib/x86_64-linux-gnu/cmake/` 以及环境变量 \verb`LD_LIBRARY_PATH` 中的路径。
\item \verb`find_library(变量 NAMES 库名1 库名2 PATHS /usr/lib /usr/local/lib HINTS $ENV{LIBRARY_PATH} $ENV{LD_LIBRARY_PATH})` 可以寻找 \verb`lib库名.so或a[.版本号]`,然后把路径写入 \verb`变量`(不做其他事情)。 若要指定版本可以用例如 \verb`gfortran.3`, 也可以指定 \verb`gfortran.so或a`。 注意指定目录的子目录不会被搜索。
\item \verb`set(CMAKE_FIND_LIBRARY_SUFFIXES ".a或.so或.lib")` 可以让 \verb`find_library()` 只寻找某种类型的 lib。 也可以用 \verb`".a;.so"` 来指定优先级。
\item \verb`CMAKE_LIBRARY_PATH` 变量可以指定 \verb`find_library` 的更多默认搜索路径,且会优先搜索。
\end{itemize}

\subsection{函数}
自定义函数如
\begin{lstlisting}[language=cmake]
# 计算一个数的平方
function(compute_square number result_var)
    math(EXPR result "${number} * ${number}")
    set(${result_var} ${result} PARENT_SCOPE)
endfunction()

# 函数调用
set(my_number 5)
compute_square(${my_number} my_result)
message(STATUS "The square of ${my_number} is ${my_result}")
\end{lstlisting}

\subsection{CMake Package}
\begin{itemize}
\item \verb`find_package` 是一个更高级的命令, 找到某个 cmake 包的设置文件,这个文件负责把一切设置好,包括 include,如何 link 等,以及 import 一些变量。 而不是像 \verb`find_library` 那样简单地找到一个包所在的路径。
\item \verb`find_package(包名 [版本] [REQUIRED] [COMPONENTS 组件1 组件2...])`
\item \verb`REQUIRED`: 如果找不到就报错终止
\item \verb`包名` 会在默认路径或者 \verb`CMAKE_MODULE_PATH` 的路径里面搜索名为 \verb`Find包名.cmake`(Module Mode,该文件通常由 cmake 提供或用户自己写) 或者 \verb`包名Config.cmake`(Config Mode,该文件通常由 package 作者提供,更高级和智能)的文件。 该文件通常会设置一些变量, 可以在 \verb`find_package` 结束后访问。
\item Module Mode 的使用例子:
\begin{lstlisting}[language=cmake]
find_package(Boost 1.71 REQUIRED COMPONENTS filesystem system)
if(Boost_FOUND)
  include_directories(${Boost_INCLUDE_DIRS})
  add_executable(myapp main.cpp)
  target_link_libraries(myapp ${Boost_LIBRARIES})
endif()
\end{lstlisting}
\item Config Mode 的使用例子:
\begin{lstlisting}[language=cmake]
find_package(Boost 1.71 REQUIRED COMPONENTS filesystem)
if(Boost_FOUND)
    add_executable(my_app main.cpp)
    target_link_libraries(my_app Boost::filesystem)
endif()
\end{lstlisting}
\item 其中 \verb`Boost::filesystem` 就是 \verb`find_package` 导入的一个 \textbf{imported target},\verb`::` 只是名字的一部分并没有特别的 namespace 机制。 \verb`Boost_FOUND` 是导入的变量,显示 \verb`Boost` 是否被找到.
\end{itemize}

\subsubsection{如何写 \verb`包名Config.cmake` (Config Mode)}
如果要写 \verb`Find包名.cmake`,其实只需要使用 \verb`set()` 设置若干路径变量即可。 一般只有库的作者没有提供 config mode 的 cmake 时才需要用户写这个文件。 用 \verb`apt` 安装的包如果提供了该 cmake 一般也会被搜到。

要写 \verb`包名Config.cmake`,需要给 \verb`add_library` 定义要导出的 target, 以及用 \verb`set_target_properties` 指定库文件和头文件的路径,这样用户导入的时候就不需要再设置一次了。
\begin{lstlisting}[language=cmake]
# MyLibraryConfig.cmake
# Define the paths to the library and include directories
set(MyLibrary_LIBRARIES "${CMAKE_CURRENT_LIST_DIR}/lib/file.a")
set(MyLibrary_INCLUDE_DIRS "${CMAKE_CURRENT_LIST_DIR}/include")
# Create an imported target for MyLibrary
add_library(MyLibrary STATIC IMPORTED)
# Set the properties for the imported target
set_target_properties(MyLibrary PROPERTIES
  # set lib files location so that user can import the target
  IMPORTED_LOCATION "${MyLibrary_LIBRARIES}"
  # set header files location so that user can import the target
  INTERFACE_INCLUDE_DIRECTORIES "${MyLibrary_INCLUDE_DIRS}"
)
# Optionally, set the MyLibrary_FOUND variable
# to indicate the library was found
set(MyLibrary_FOUND TRUE)
\end{lstlisting}

\begin{lstlisting}[language=cmake]
# Minimum CMake version
cmake_minimum_required(VERSION 3.10)

# Project name
project(MyLibrary)

# Set the version (optional)
set(MYLIBRARY_VERSION_MAJOR 1)
set(MYLIBRARY_VERSION_MINOR 0)

# Define the library target
add_library(MyLibrary SHARED IMPORTED)

# Specify the directories for header files
target_include_directories(MyLibrary
    INTERFACE
    # Replace with actual path
    ${CMAKE_CURRENT_SOURCE_DIR}/include
)

# Link the library files (use separate statements
#     for different OSes if needed)
set_target_properties(MyLibrary PROPERTIES
    # Linux
    IMPORTED_LOCATION "${CMAKE_CURRENT_SOURCE_DIR}/lib/mylibrary.so"
    # Windows
    IMPORTED_IMPLIB "${CMAKE_CURRENT_SOURCE_DIR}/lib/mylibrary.dll"
)

# Define macros required by the library
target_compile_definitions(MyLibrary
    INTERFACE
    MYLIBRARY_DEFINE_1
    MYLIBRARY_DEFINE_2=VALUE  # Example with a value
)

# Export the library target for use in other projects
export(TARGETS MyLibrary FILE MyLibraryConfig.cmake)
\end{lstlisting}

\subsection{内置变量}
\begin{itemize}
\item \verb`PROJECT_NAME` 是 \verb`project(项目名)` 设置的\verb`项目名`
\item \verb`PROJECT_SOURCE_DIR` 是源文件的根路径, 就是传给 \verb`cmake` 的路径
\item \verb`PROJECT_BINARY_DIR` 是 Cmake 的输出路径, 临时文件, Makefile 等都在这个路径。 这就是运行 \verb`cmake` 命令时的当前路径
\item \verb`CMAKE_CURRENT_LIST_DIR` 当前在处理的 \verb`CMakeLists.txt` 的路径。
\item \verb`CMAKE_SOURCE_DIR` 和 \verb`CMAKE_BINARY_DIR` 和上面两个有什么区别?
\item \verb`CMAKE_BUILD_TYPE` 可以被用户设置为 \verb`Release`, \verb`Debug` 等。 例如 \verb`cmake -DCMAKE_BUILD_TYPE=Debug` 用 debug 模式编译。 \verb`Debug` 模式下,编译器优化会被自动关闭, \verb`-g` 选项会自动加上, \verb`NDEBUG` 不会。 在两种模式下, \verb`CMAKE_CXX_FLAGS_DEBUG` 和 \verb`CMAKE_CXX_FLAGS_RELEASE` 中的选项会分别传给编译器。 任何模式下 \verb`CMAKE_CXX_FLAGS` 的选项都会传给编译器。
\item \verb`WIN32` 可以判断是否在 windows 上(包括 32 和 64 位)
\item \verb`MSVC` 判断是否用 Visual C++ 编译器
\end{itemize}

\subsection{判断}
参考\href{https://cmake.org/cmake/help/latest/command/if.html}{这里}。
\begin{lstlisting}[language=bash]
if(1 或 ON 或 YES 或 TRUE 或 Y)
  ……
elseif(0 或 OFF 或 NO 或 FALSE 或 N 或 IGNORE 或 NOTFOUND)
  这里永远不会执行
elseif(条件 AND (条件 OR 条件) OR NOT 条件)
  ……
elseif(("bar" IN_LIST 变量) OR (file1 IS_NEWER_THAN file2))
  ……
elseif(变量)
  当变量有定义且不是 false constant
if(ENV{变量})
  这里永远不会执行
elseif (DEFINED <name>|CACHE{<name>}|ENV{<name>})
  若定义了变量
elseif (变量1 STREQUAL 变量2)
  比较字符串
elseif (IS_DIRECTORY 某路径)
  判断某路径是否存在
else()
  ……
endif()
\end{lstlisting}

\subsection{CMake 与 Visual Studio}
\begin{itemize}
\item 链接 \verb`*.lib`, 只需要在 \verb`add_executable()` 命令前面插入(必须在之前) \verb`abc.lib` 的路径 \verb`link_directories(path/to/lib)` 然后再 \verb`add_executable()` 之后插入 \verb`target_link_libraries(exe_name abs)` 即可
\item \verb`if(MSVC)...end(MSVC)` 可以专门给 Visual Studio 执行一些命令
\item 生成的 sln 文件中, 除了有 target 的工程, 还会有另外两个工程: \verb`ZERO_CHECK` 会重新运行/更新 \verb`CMakeLists.txt`, \verb`ALL_BUILD` 会编译所有工程。 如果看着不爽的话也可以把这两个 project 删掉。 如果直接 run without debug 的话, 会提示 \verb`ALL_BUILD` 不能 run, 所以要 run 只能右键某个 project 然后 run (所以还是把两个多余的 project 删掉好些)。
\end{itemize}

\subsection{ccmake 使用}
\begin{itemize}
\item \verb`sudo apt install cmake-curses-gui` 安装
\item 在需要生成 binary 的路径打开 \verb`ccmake`, 这个路径可以是源文件路径
\item 如果 binary 路径不是源文件路径, 那么指定路径即可, 例如 \verb`ccmake ../`
\item \verb`h` 帮助, \verb`q` 退出, \verb`c` configure, \verb`G` 生成 Makefile
\end{itemize}
