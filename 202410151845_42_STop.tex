% 共轭空间中的强拓扑
% keys 强拓扑|完备
% license Usr
% type Tutor
\pentry{共轭空间与代数共轭空间\nref{nod_ConSpa},线性算子的范数\nref{nod_ONorm}}{nod_0a2f}

\subsection{赋范空间中的强拓扑}

由\autoref{ex_tvs_1} 可知,赋范空间是一个拓扑线性空间,因此其上自然由\enref{线性连续泛函}{LinCon}的定义。而赋范空间上根据
\begin{equation}\label{eq_STop_1}
\norm{f}:=\sup_{x\neq0}\frac{\abs{f(x)}}{\norm{x}}~
\end{equation}
可引入线性连续泛函的范数,证明\autoref{eq_STop_1} 满足\enref{范数}{norm}的定义和\autoref{ex_ONorm_1} 的证明完全一样。因此赋范空间的共轭空间可赋予赋范空间的自然结构。范数可以用来定义度量,度量有一个自然定义开集的方式,即在赋范空间上有一个自然定义的拓扑,在赋范空间的共轭空间中这样定义的拓扑就是强拓扑。
\begin{definition}{强拓扑}
设 $E$ 是赋范空间,则其\enref{共轭空间}{ConSpa} $E^*$ 上由\autoref{eq_STop_1} 定义的范数相应的拓扑称为 $E^*$ 的\textbf{强拓扑}。
\end{definition}

当希望把 $E^*$ 当作赋范空间时,我们将其写作 $(E^*,\norm{*})$。

\begin{theorem}{}
设 $E$ 是赋范空间,则 $(E^*,\norm{*})$ 是\enref{完备}{ComSpa}的。
\end{theorem}

\textbf{证明:}设 $\{f_n\}$ 是线性连续泛函的柯西序列(\autoref{def_ComSpa_1})。那么对任意 $\epsilon>0$,存在 $N$,使得对所有的 $n,m>N$ 有 $\norm{f_n-f_m}<\epsilon$。由此,对任意 $x\in E$ 得到
\begin{equation}\label{eq_STop_2}
\abs{f_n(x)-f_m(x)}\leq\norm{f_n-f_m}\cdot\norm{x}<\epsilon\norm{x},~
\end{equation}
即对任意 $x\in E$,数列 $\{f_n(x)\}$ 收敛($\norm{x}$ 是某一确定的实数,而 $\epsilon$ 任意)。

令 $f(x):=\lim\limits_{n\rightarrow\infty}f_n(x)$。则由
\begin{equation}
\begin{aligned}
f(\alpha x+\beta y)=&\lim_{n\rightarrow\infty}(\alpha x+\beta y)\\
=&\lim_{n\rightarrow\infty}\qty[\alpha f_n(x)+\beta f_n(y)]\\
=&\alpha f(x)+\beta f(y),
\end{aligned}~
\end{equation}
得 $f$ 是线性的。此外,取\autoref{eq_STop_2} 的 $m\rightarrow\infty$,则 
\begin{equation}\label{eq_STop_3}
\abs{f(x)-f_n(x)}\leq \epsilon\norm{x}.~
\end{equation}
由此得 $f-f_n$ 有界($\norm{x}$ 有限,$\epsilon>0$ 任意)。因而 $f=f_n+(f-f_n)$ 有界($\{f_n(x)\}$ 对任意 $x$ 收敛)。由\autoref{the_LinCon_2} , $f$ 连续。 另外,由\autoref{eq_STop_3} 得 $\norm{f-f_n}\leq\epsilon$,即 $\{f_n\}$ 收敛于 $f$。


\textbf{证毕!}

该定理表明,无论赋范空间是否完备,其共轭空间都完备。

\begin{theorem}{}
若赋范空间 $E$ 不完备,而 $\bar{E}$ 是 $E$ 的完备化空间,则 $E^*,\bar{E}^*$ 同构。
\end{theorem}

\textbf{证明:} 因为 $\bar{E}$ 是 $E$ 的完备化空间,所以对任一 $x\in \bar{E}$,存在 $E$ 中的收敛于 $x$ 的序列 $\{x_n\}$ (由\autoref{the_cauchy_2},它是柯西序列)。设 $f$ 是 $E$ 上的线性连续泛函,则 $\{f(x_n)\}$ 是柯西数列,因而必收敛(实数集上的柯西序列必收敛),定义 $\bar f(x):=\lim\limits_{n\rightarrow\infty}f(x_n)$,则 $\bar f$ 就是 $f$ 在 $\bar E$ 上的连续延拓。

该延拓是唯一的,否则存在 $x\in \bar E\backslash E$,使得 $\bar f(x)\neq g(x)$,而由 $E$ 中收敛到 $x$ 的柯西序列 $\{x_n\}$,恒有 $\bar f(x_n)=g(x_n)$,取极限就是 $\bar f(x)=g(x)$,这一矛盾就说明连续延拓是唯一的。显然 $\bar f\in(\bar E)^*$,且 $\norm{\bar f}=\norm{f}$(由\autoref{eq_STop_1} 推得)。

另外,任一 $(\bar E)^*$ 中的线性连续泛函是其在 $E$ 上限制得到的泛函的延拓,因此,映射 $f\mapsto\bar f$ 是 $E$ 到 $(\bar E)^*$ 的同构映射。事实上,$f\mapsto\bar f$ 的线性性质可通过 
\begin{equation}
\begin{aligned}
&f(x_n)+g(x_n)-(\bar f(x)+\bar g(x))\\
&=(f(x_n)-\bar f(x))+(g(x_n)-\bar g(x))\\
&<\epsilon/2+\epsilon/2=\epsilon
\end{aligned}~
\end{equation}
获得。

\textbf{证毕!}



\subsection{拓扑线性空间的强拓扑}

我们看到,赋范空间中的强拓扑是由于\autoref{eq_STop_1} 定义了其共轭空间上的范数引进的。在强拓扑下,赋范空间的共轭空间也是拓扑线性空间。赋范空间的共轭空间的零邻域自然定义为满足条件
\begin{equation}
\norm{f}<\epsilon~
\end{equation}
的泛函的集合。

由\autoref{eq_STop_1} $\norm{f}\norm{x}\geq\abs{f(x)},x\neq0$。因此,零邻域的定义相当于说:当 $x$ 遍历单位球 $\norm{x}=1$ 时,我们把 $\abs{f(x)}<\epsilon$ 的线性泛函取为赋范空间的共轭空间的零邻域。取所有可能的 $\epsilon$,便得到确定零邻域系\footnote{即任一包含零的开集,都有该邻域系的一个包含在该开集中,由于范数定义的开集必然包含某一开球,所以文中所有可能的 $\epsilon$ 对应的系确实是确定邻域系}。由于拓扑线性空间的拓扑由零邻域系完全确定(\autoref{the_tvs_1}),所以这一确定零邻域系给出了赋范空间的共轭空间的拓扑。

对应到一般的拓扑线性空间 $E$,单位球面自然被任意有界集 $A$ 代替,而 $E^*$ 的零邻域则定义为满足 $\abs{f(x)}<\epsilon,x\in A$ 的线性泛函的集合。 

\begin{definition}{共轭空间的零邻域}
设 $E^*$ 是拓扑线性空间 $E$ 的共轭空间,$A\subset E$ 中的有界集,$\epsilon>0$ 是任一数,则称
\begin{equation}
U_{\epsilon,A}=\{f|f\text{是 $E$ 的线性连续泛函且}\abs{f(x)}<\epsilon\}~
\end{equation}
 是 $E^*$ 的(在 $A$ 上的 $\epsilon$)\textbf{零邻域}。
\end{definition}

类似的,采用不同的 $\epsilon,A$,变得到了共轭空间中的确定零邻域系,它们可以确定共轭空间的拓扑,这一拓扑称为共轭空间的强拓扑。 

\begin{definition}{强拓扑}
设 $E^*$ 是拓扑线性空间 $E$ 的共轭空间,$A\subset E$ 中的有界集。则称
\begin{equation}
\{U_{\epsilon,A}|A\text{有界},\epsilon>0\}~
\end{equation}
确定的 $E^*$ 上的拓扑为 $E^*$ 的\textbf{强拓扑}。
\end{definition}


\begin{theorem}{强拓扑满足第一分离公理与局部凸}
$E^*$ 的强拓扑必定满足\enref{第一分离公理}{DisAx} $T_1$和\enref{局部凸性}{tvs}。
\end{theorem}


