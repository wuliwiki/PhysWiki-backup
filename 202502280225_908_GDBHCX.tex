% 哥德巴赫猜想(综述)
% license CCBYSA3
% type Wiki

本文根据 CC-BY-SA 协议转载翻译自维基百科\href{https://en.wikipedia.org/wiki/Goldbach\%27s_conjecture}{相关文章}。

\begin{figure}[ht]
\centering
\includegraphics[width=6cm]{./figures/0b683c0c5cb7d58d.png}
\caption{哥德巴赫致欧拉的信,日期为1742年6月7日(拉丁语-德语)[1]} \label{fig_GDBHCX_1}
\end{figure}
哥德巴赫猜想是数论和整个数学中最古老且最著名的未解问题之一。它指出,所有大于2的偶数自然数都可以表示为两个质数的和。

尽管经过了大量的努力,这个猜想已经被证明对所有小于 \(4 \times 10^{18}\) 的整数成立,但仍未被完全证明。
\subsection{历史}  
\subsubsection{起源}  
1742年6月7日,普鲁士数学家克里斯蒂安·哥德巴赫写信给莱昂哈德·欧拉(信件编号XLIII),[2] 在信中他提出了以下猜想:

“dass jede Zahl, welche aus zweyen numeris primis zusammengesetzt ist, ein aggregatum so vieler numerorum primorum sey, als man will (die unitatem mit dazu gerechnet), bis auf die congeriem omnium unitatum”  
每个可以表示为两个质数和的整数,也可以表示为任意多个质数之和,直到所有的项都是单位数。  
哥德巴赫遵循了现在已被废弃的将1视为质数的传统[3],因此单位数的和也可以视为质数的和。然后,他在信的边缘提出了第二个猜想,暗示第一个猜想:[4]  

“Es scheinet wenigstens, dass eine jede Zahl, die grösser ist als 2, ein aggregatum trium numerorum primorum sey.”  
至少似乎每个大于2的整数可以表示为三个质数的和。

欧拉在1742年6月30日的信中回复,提醒哥德巴赫他们之前的对话(“… so Ew vormals mit mir communicirt haben …”),当时哥德巴赫提到,第一个猜想将从以下陈述中得到推导:

“每个正偶整数可以表示为两个质数的和。”  
这实际上等同于他的第二个边际猜想。在1742年6月30日的信中,欧拉写道:[6][7]  

“Dass ... ein jeder numerus par eine summa duorum primorum sey, halte ich für ein ganz gewisses theorema, ungeachtet ich dasselbe nicht demonstriren kann.”  
我认为“每个偶数都是两个质数的和”是一个完全确定的定理,尽管我无法证明它。
\subsubsection{笛卡尔的类似猜想}  
勒内·笛卡尔曾写道:“每个偶数都可以表示为最多三个质数的和。”[8] 这个命题等价于哥德巴赫猜想,保罗·厄尔德什曾说:“笛卡尔实际上在哥德巴赫之前发现了这个猜想……但把这个猜想命名为哥德巴赫更合适,因为在数学上,笛卡尔极其富有,而哥德巴赫非常贫穷。”[9]
\subsubsection{部分结果}  
强哥德巴赫猜想比弱哥德巴赫猜想要难得多,后者声称每个大于5的奇数是三个质数的和。利用维诺格拉多夫方法,尼古拉·丘达科夫、约翰内斯·范·德·科普特和西奥多·埃斯特曼(1937–1938年)证明了几乎所有偶数都可以表示为两个质数的和(意思是,能以这种方式表示的偶数占所有偶数的比例,随着\(N\)的增大趋近于1)。1930年,列夫·施尼雷尔曼证明了任何大于1的自然数都可以表示为不超过\(C\)个质数的和,其中\(C\)是一个有效可计算的常数;见施尼雷尔曼密度。[13][14] 施尼雷尔曼常数是具有这个性质的最小数字\(C\)。施尼雷尔曼本人得出\(C < 800000\)。此结果随后被许多作者加强,例如奥利维耶·拉马雷,他在1995年证明每个偶数\(n\ge4\)实际上是最多由6个质数的和。当前最著名的结果来自哈拉尔德·赫尔福特对弱哥德巴赫猜想的证明,[15] 它直接意味着每个偶数\(n\ge4\)是最多由4个质数的和。[16][17]

1924年,哈代和利特伍德在假设广义黎曼猜想的前提下,证明了违反哥德巴赫猜想的偶数的数量远小于\(X^{1/2}+ c\),其中\(c\)是一个小常数。[18]

1948年,阿尔弗雷德·雷尼使用筛法理论证明了每个足够大的偶数可以表示为一个质数和一个至多有\(K\)个因子的几乎质数之和。[19] 1973年,陈景润使用筛法理论证明了每个足够大的偶数可以表示为两个质数的和,或者一个质数和一个半质数(两个质数的乘积)的和。[20] 有关更多信息,请参见陈景润定理。

1975年,休·洛威尔·蒙哥马利和鲍勃·沃恩证明了“绝大多数”偶数可以表示为两个质数的和。更精确地说,他们证明了存在正常数\(c\)和\(C\),使得对于所有足够大的\(N\),所有小于\(N\)的偶数都可以表示为两个质数的和,且最多有\(CN^{1-c}\)个例外。特别地,不是两个质数和的偶数集合的密度为零。

1951年,尤里·利尼克证明了存在一个常数\(K\),使得每个足够大的偶数是两个质数和以及最多\(K\)个2的幂的和。贾诺什·平茨和伊姆雷·鲁扎在2020年发现\(K = 8\)是可行的。[21] 假设广义黎曼假设成立\(K = 7\)也是可行的,正如罗杰·希思-布朗和扬-克里斯托夫·施莱格-普赫塔在2002年所证明的。[22]

2013年,哈拉尔德·赫尔福特向《数学年刊研究系列》提交了弱哥德巴赫猜想的证明。尽管文章被接受,但赫尔福特决定进行审稿人建议的大幅修改。尽管经过几次修订,赫尔福特的证明尚未在同行评审的出版物中发表。[23][24][25] 弱哥德巴赫猜想由强哥德巴赫猜想所暗示,因为如果\(n-3\)是两个质数的和,那么\(n\)就是三个质数的和。然而,反向推理,因此强哥德巴赫猜想,如果赫尔福特的证明是正确的,仍然没有得到证明。
\subsubsection{计算结果}  
对于小的\(n\)值,强哥德巴赫猜想(因此弱哥德巴赫猜想)可以直接验证。例如,1938年,尼尔斯·皮平(Nils Pipping)费力地验证了该猜想直到\(n = 100000\)。[26] 随着计算机的出现,更多的n值被检查过;T. Oliveira e Silva进行了一项分布式计算机搜索,并且到2013年为止,已验证该猜想对于\(n\ge4\times10^{18}\)成立(并且对\(4\times10^{17}\)进行了双重检查)。这项搜索的一个记录是,3325581707333960528是不能写成两个质数和的最小数字,其中一个质数小于9781。[27]
\subsubsection{在流行文化中}
《哥德巴赫猜想》(中文:哥德巴赫猜想)是中国数学家和数论学家陈景润的传记,作者是徐迟。

该猜想是1992年希腊作家阿波斯托洛斯·多西亚迪斯(Apostolos Doxiadis)小说《佩特罗斯叔叔与哥德巴赫猜想》中的核心情节之一,此外还出现在艾萨克·阿西莫夫的短篇小说《六千万万亿种组合》以及2008年米歇尔·里士满(Michelle Richmond)所著的悬疑小说《你认识的任何人》中。[28]

哥德巴赫猜想是2007年西班牙电影《费尔马的房间》情节的一部分。

哥德巴赫猜想也出现在2023年法瑞联合制作的电影《玛格丽特定理》中,女演员艾拉·鲁姆夫(Ella Rumpf)所饰演的角色玛格丽特将其作为研究的主要课题。[29]
\subsection{正式陈述}
这三个猜想都有一个自然的类比,基于现代质数的定义,其中1被排除在外。第一个猜想的现代版本是:

每个可以表示为两个质数之和的整数,也可以表示为任意数量的质数之和,直到所有项都是二(如果该整数是偶数),或者一个项是三,其他所有项都是二(如果该整数是奇数)。

边缘猜想的现代版本是:

每个大于5的整数都可以表示为三个质数之和。

哥德巴赫的旧猜想(欧拉提醒他的那个猜想)的现代版本是:

每个大于2的偶数都可以表示为两个质数之和。

这些现代版本可能与相应的原始陈述并不完全等价。例如,如果存在一个偶整数\(N=p +1\)(其中\(p\)是质数),且大于4,那么它就不能用现代意义上的两个质数之和表示;在这种情况下,它将成为现代版本第三个猜想的反例(但不会是原始版本的反例)。因此,现代版本可能更强(但为了确认这一点,需要证明第一版本在自由地应用于任何正偶整数\(n\)时,不可能排除这种特定反例\(N\)的存在)。无论如何,现代版本之间的关系与旧版本之间的关系相同。也就是说,第二个和第三个现代陈述是等价的,且两者都能推导出第一个现代陈述。

第三个现代陈述(与第二个等价)是今天通常表达的形式,也称为“强”的、“偶数”的或“二元”哥德巴赫猜想。第二个现代陈述的较弱形式,称为“哥德巴赫的弱猜想”、“奇数哥德巴赫猜想”或“三元哥德巴赫猜想”,主张:

每个大于7的奇数整数都可以表示为三个奇质数之和。
\subsection{启发式证明}
\begin{figure}[ht]
\centering
\includegraphics[width=6cm]{./figures/581bf28466376757.png}
\caption{} \label{fig_GDBHCX_2}
\end{figure}
关注质数概率分布的统计考虑为猜想(无论是弱猜想还是强猜想)提供了非正式的证据,尤其是在足够大的整数情况下:整数越大,就有更多的方式可以将该数表示为两个或三个其他数的和,而其中至少有一个表示是完全由质数构成的“可能性”也就越大。
\begin{figure}[ht]
\centering
\includegraphics[width=6cm]{./figures/162f35f3dfa38008.png}
\caption{将一个偶数\(n\)表示为两个素数之和的方式数(OEIS 中的序列 A002375)} \label{fig_GDBHCX_3}
\end{figure}
一种非常粗略的启发式概率论证(适用于哥德巴赫猜想的强形式)如下。素数定理断言,随机选择一个整数\(m\),它是素数的概率大约为 \( \frac{1}{\ln m} \)。因此,如果 \(n\)是一个大的偶数,且\(m\)是介于 3 和 \( \frac{n}{2} \) 之间的数,那么可以预期 \(m\)和 \( n - m \) 同时为素数的概率大约为 \( \frac{1}{\ln m \cdot \ln(n - m)} \)。如果沿着这个启发式进行推理,可以预期将一个大的偶数\(n\)写成两个奇素数之和的方式总数大致为:
\[
\sum _{m=3}^{\frac {n}{2}}{\frac {1}{\ln m}}{\frac {1}{\ln(n-m)}} \approx \frac{n}{2(\ln n)^{2}}.~
\]
由于 \( \ln n \ll \sqrt{n} \),随着\(n\)增加,这个数量趋向于无穷大,因此可以预期每个大的偶数不仅有一个表示为两个素数之和的方式,实际上有非常多种这样的表示。

这个启发式论证实际上有些不准确,因为它假设\(m\)和\(n-m\)是相互独立的事件。例如,如果\(m\)是奇数,那么\(n-m\)也是奇数;如果\(m\)是偶数,那么\(n-m\)也是偶数,这是一个非平凡的关系,因为除了数字 2 以外,只有奇数可以是素数。类似地,如果 \(n\)能被 3 整除,并且\(m\)已经是 3 以外的素数,那么\(n-m\)也会与 3 互质,因此比一般数字更有可能是素数。仔细地进行这种类型的分析后,G. H. 哈迪和约翰·埃登索尔·利特伍德在 1923 年提出猜想(作为他们的哈迪-利特伍德素数元组猜想的一部分),即对于任何固定的\(c\ge2\),表示一个大整数\(n\)为\(c\)个素数之和\(n = p_1+\cdots +p_c\)(其中\(p_1 \leq \cdots \leq p_c\))的方式数应该渐近地等于
\[
\left(\prod _{p}{\frac {p\gamma _{c,p}(n)}{(p-1)^{c}}}\right)\int _{2\leq x_{1}\leq \cdots \leq x_{c}:x_{1}+\cdots +x_{c}=n}{\frac {dx_{1}\cdots dx_{c-1}}{\ln x_{1}\cdots \ln x_{c}}}~
\]
其中,乘积是对所有素数\(p\)求积,γc,\(p^{(n)}\)是方程\(n = q_1 + \cdots + q_c\) mod \(p\)的解的个数,满足约束条件 \(q_1,\cdots,\) \(q_c\ne0\) mod \(p\)。这个公式已经通过伊万·马特维耶维奇·维诺格拉多夫的工作严格证明了对\(c\ge3\) 是渐近有效的,但当\(c=2\) 时仍然只是一个猜想。[citation needed] 在后者的情况下,上述公式在\(n\)为奇数时简化为 0,而在\(n\)为偶数时简化为:
\[
2\Pi _{2}\left(\prod _{p\mid n;p\geq 3}{\frac {p-1}{p-2}}\right)\int _{2}^{n}{\frac {dx}{(\ln x)^{2}}}\approx 2\Pi _{2}\left(\prod _{p\mid n;p\geq 3}{\frac {p-1}{p-2}}\right){\frac {n}{(\ln n)^{2}}}~
\]
其中\(\Pi _{2}\)是哈迪-利特伍德的双素数常数:
\[
\Pi _{2}:=\prod _{p\;{\rm {prime}}\geq 3}\left(1-{\frac {1}{(p-1)^{2}}}\right)\approx 0.66016\,18158\,46869\,57392\,78121\,10014\dots~
\]
这有时被称为扩展的哥德巴赫猜想。强哥德巴赫猜想实际上与双素数猜想非常相似,且两者被认为在难度上大致相当。
\begin{figure}[ht]
\centering
\includegraphics[width=6cm]{./figures/766ae8532331674f.png}
\caption{哥德巴赫彗星;红色、蓝色和绿色的点分别对应数字对 3 的模值为 0、1 和 2 的值。} \label{fig_GDBHCX_4}
\end{figure}
哥德巴赫分解函数是一个将每个偶整数与它可以分解为两个素数之和的方式数关联起来的函数。它的图像像一颗彗星,因此被称为哥德巴赫彗星。[30]

哥德巴赫彗星暗示了关于偶数作为两个素数之和的表示数目具有严格的上下界,同时也表明这些表示数目在很大程度上取决于该数字对 3 的模值。
\subsection{相关问题}  
尽管哥德巴赫猜想意味着每个大于1的正整数可以表示为最多三个质数的和,但并不总是能够通过贪心算法找到这样的和,即每一步都使用可能的最大质数。Pillai序列跟踪需要最多质数的数字的贪心表示[31]。

与哥德巴赫猜想类似的问题存在,其中质数被其他特定的数集所替代,例如平方数:
\begin{itemize}
\item 拉格朗日证明了每个正整数都是四个平方数的和。见Waring问题及与Waring–Goldbach问题相关的质数幂的和。
\item 哈代和利特伍德将他们的猜想I列为:“每个大的奇数(\(n > 5\))都是一个质数和一个质数的两倍之和”[32]。这个猜想被称为莱莫因猜想,也叫做列维猜想。
\item 关于实践数的哥德巴赫猜想,一个类似质数的整数序列,由Margenstern于1984年提出[33],并由Melfi于1996年证明[34]:每个偶数都是两个实践数的和。
\item 哈维·杜布纳提出了一个加强版的哥德巴赫猜想,声明每个大于4208的偶整数是两个孪生质数的和(不一定属于同一对)[35][需要更好的来源]。小于4208的只有34个偶数不能表示为两个孪生质数的和;杜布纳已通过计算验证了这个列表在\(2\cdot10^{10}\)以内是完整的[36][验证需要]。证明这个更强的猜想不仅能推导出哥德巴赫猜想,还能推导出孪生质数猜想。
\item 根据伯特兰猜想,对于每个整数\(n>1\),总有至少一个质数\(p\)使得\(n<p <2n\)。如果这个猜想是错误的,就会存在某个整数\(n\),在\(n\)和\(2n\)之间没有质数,使得\(2n\)无法表示为两个质数的和。
\end{itemize}
哥德巴赫猜想在研究计算复杂性时被使用[37]。这种联系是通过忙碌海狸函数建立的,其中BB(n)是任何n状态图灵机停止时所采取的最大步骤数。如果知道27状态的图灵机会停止当且仅当哥德巴赫猜想是错误的[37],那么如果BB(27)已知,并且图灵机没有在这个步数内停止,那么就可以知道它永远不会停止,因此没有反例存在(从而证明猜想成立)。这是一种完全不切实际的方式来解决猜想;相反,它用于暗示BB(27)将非常难以计算,至少像解决哥德巴赫猜想一样困难。
