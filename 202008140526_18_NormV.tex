% 范数
% keys 线性代数|赋范空间|线性空间|三角不等式

\pentry{矢量空间\upref{LSpace}}

设$X$是实或复向量空间. $X$上的\textbf{范数 (norm)} 是满足如下条件的非负函数$\|\cdot\|$:
\begin{enumerate}
\item $\norm{x} \geqslant 0$ (正定)
\item $\norm{x} = 0$ 当且仅当 $x = 0$
\item $\|\lambda x\| = |\lambda|\| v\|$
\item $\|x_1+x_2\| \leqslant \|x_1\|+ \|x_2\|$ (三角不等式)
\end{enumerate}

如果一个矢量空间中定义了范数, 我们就把它称为\textbf{赋范空间(normed space)}\footnote{满足一定收敛条件的赋范空间也叫做 \textbf{巴拿赫空间(Banach space)\upref{banach}}, 有限维的赋范空间必定是巴拿赫空间。%链接未完成
}. 范数通常用双竖线表示, 如 $\norm{v}$. 赋范线性空间按照度量
$$
d(x,y):=\|x-y\|
$$
而成为度量空间\upref{Metric}. 

一个线性空间上可能可以定义许多个范数. 线性空间$X$上的两个范数$\|\cdot\|_1,\|\cdot\|_1$称为\textbf{等价 (equivalent)} 的, 如果有正实数$C>1$使得如下不等式对于任何$x\in X$都成立:
$$
C^{-1}\|x\|_{1}\leq\|x\|_2\leq C\|x\|_1.
$$

\subsection{一些常见的赋范线性空间}
设$p\geq1$. 定义 $\mathbb R^N$ 或 $\mathbb C^N$ 空间(即 $N$ 维实数或复数列矢量空间) 的 \textbf{$p$-范数}为
\begin{equation}
\norm{\bvec x}_p = \qty(\sum_{i=1}^N \abs{x_i}^p)^{1/p}
\end{equation}
物理中常见的是 \textbf{2-范数}, 也叫\textbf{欧几里得范数(Euclidean norm)} 即
\begin{equation}
\norm{\bvec x}_2 = \sqrt{\abs{x_1}^2 + \abs{x_2}^2 + \dots+|x_N|^2}
\end{equation}

在极限 $p \to \infty$ 之下, 绝对值最大的 $x_i$ 对求和的贡献将远大于其他分量, 所以可定义\textbf{无穷范数(infinity norm)}为
\begin{equation}
\norm{\bvec x}_\infty = \max \qty{\abs{x_i}}.
\end{equation}

有限维实或复线性空间上的任意两个范数必然彼此等价. 它们都给出空间上唯一的一个自然拓扑 (即使得所有线性泛函均连续的拓扑). 在任何范数之下, 有限维实或复线性空间都是巴拿赫空间\upref{banach}.

\subsection{函数的范数}
多元函数 $f(x_1, \dots x_N)$ 的范数在物理中常定义为
\begin{equation}
\norm{f} = \int \abs{f(x_1, \dots, x_N)}^2 \dd{x_1}\dots \dd{x_N}
\end{equation}
另一种简单定义是使用函数的最大值
\begin{equation}
\norm{f} = \max\qty{\abs{f(x_1, \dots, x_N)}}
\end{equation}
