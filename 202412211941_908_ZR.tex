% 自然语言处理(综述)
% license CCBYSA3
% type Wiki

本文根据 CC-BY-SA 协议转载翻译自维基百科\href{https://en.wikipedia.org/wiki/Natural_language_processing}{相关文章}。

自然语言处理(NLP)是计算机科学的一个子领域,特别是人工智能领域。它主要关注赋予计算机处理自然语言编码的数据的能力,因此与信息检索、知识表示和计算语言学(语言学的一个子领域)密切相关。通常,数据通过文本语料库收集,并使用基于规则、统计方法或基于神经网络的机器学习和深度学习方法进行处理。

自然语言处理的主要任务包括语音识别、文本分类、自然语言理解和自然语言生成。
\subsection{历史} 
更多信息:自然语言处理的历史  
自然语言处理的根源可以追溯到20世纪50年代。[1] 早在1950年,阿兰·图灵就发表了一篇名为《计算机器与智能》的文章,提出了现在被称为图灵测试的智能标准,尽管当时这并没有被表述为一个与人工智能分开的问题。该测试提议包括一个任务,涉及自动化地解释和生成自然语言。
