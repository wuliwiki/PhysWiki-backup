% 中科院2013普通物理
% 中科院|考研|普通物理
\subsection{选择题}
1. 质点沿一固定圆形轨道运动,如果速率均匀增大,下列物理量中不随时间变化的是\\
(A) 法向加速度大小;\\
(B) 切向加速度大小;\\
(C) 加速度大小;\\
(D) 加速度与速度间的夹角.

2. 某电梯的天花板上竖直悬挂着弹性系数为 $k$ 的弹簧振子, 弹簧下端挂有一质 量为 $m$ 的物块, 则当电梯以匀加速 $a_{1}$ 上升和匀减速 $a_{2}$ 上升时, 弹簧和物块组成的系统振动频率分别为\\
(A) $\frac{1}{2 \pi} \sqrt{\frac{k}{m}},\frac{1}{2 \pi} \sqrt{\frac{k}{m}}$;\\
(B) $\frac{1}{2 \pi} \sqrt{\frac{k}{m}\left(1+\frac{a_{1}}{g}\right)},\frac{1}{2 \pi} \sqrt{\frac{k}{m}\left(1-\frac{a_{2}}{g}\right)} ;$\\
(C) $\frac{1}{2 \pi} \sqrt{\frac{k}{m\left(g+a_{1}\right)}},\frac{1}{2 \pi} \sqrt{\frac{k}{m\left(g-a_{2}\right)}} ;$\\
(D) $\frac{1}{2 \pi} \sqrt{\frac{k}{m\left(g-a_{1}\right)}},\frac{1}{2 \pi} \sqrt{\frac{k}{m\left(g+a_{2}\right)}} .$

3. 核电站的原子能反应堆中需要用低速中子维持缓慢的链式反应, 反应释放的 却是高速快中子.在反应堆中, 快中子通过与石墨棒内静止的碳原子 $\left({ }_{6}^{12} \mathrm{C}\right)$ 发生弹性碰撞而减速.已知, 碳原子质量是中子质量的 12 倍, 则一次碰撞前、 后中子动能比为\\
(A) $\left(\frac{5}{6}\right)^{2}$;$\quad$
(B) $\left(\frac{6}{5}\right)^{2}$;$\quad$
(C) $\left(\frac{11}{13}\right)^{2}$;$\quad$
(D) $\left(\frac{13}{11}\right)^{2}$ .

4. 关于理想气体, 以下表述不正确的是\\
(A) 气体分子本身的体积可以忽略不计, 分子与容器壁以及分子与分子之间 的碰撞属于完全弹性碰撞;\\
(B) 在相同的温度下, 气体分子的平均平动动能相同而与气体的种类无关;\\
(C) 气体的压强与分子数密度成正比, 与平均平动动能无关;\\
(D) 在相同的温度和压强下,各种气体在相同的体积内所含的分子数相等.

5. 有一点电荷A带正电量 $Q$, 距其不远处放入一个不带电的金属导体小球B, 平衡后, 电荷A的电势为 $U_{\mathrm{A}}$, 导体球 $\mathrm{B}$ 的电势为 $U_{\mathrm{B}}$, 无穷远处电势为 $U$ .则以下关系正确的是\\
(A) $U_{A}>U_{B}>U$;\\
(B) $U_{B}>U_{A}>U$;\\
(C) $U_{A}>U>U_{B}$;\\
(D) $U_{B}>U>U_{A}$ .

6. 真空中,两靠近的平行金属板分别带均匀的等量异号电荷.若板间左侧一半 空间充入介电常数为 $\varepsilon$ 的电介质,则以下说法错误的是 \\
(A) 充入电介质后, 板间电压减少;\\
(B) 充入电介质后,板间电容增加;\\
(C) 充入电介质后, 板间电场总能量减\\
(D) 充入电介质后,板间左侧电介质中的电场强度小于右侧真空中电场强度.

7. 真空介电常数 $\varepsilon_{0}$ 在国际单位制中的量纲是\\
(A) $L^{-2} M^{-1} T^{4} I^{2}$;$\quad$
(B) $L^{-2} M^{1} T^{3} I^{2}$;$\quad$
(C) $L^{-3} M^{-1} T^{4} I^{2}$;$\quad$
(D) $L^{-3} M^{-1} T^{3} I^{2}$ .

8. 卢瑟福散射过程中 $\alpha$ 粒子以能量 $E$ 入射固定原子核靶, $\alpha$ 粒子电荷为 $Z^{\prime} e$, 原子核电荷为 $Z e, \alpha$ 粒子散射角为 $\theta$, 则以下卢瑟福散射截面 $\frac{\mathrm{d} \sigma}{\mathrm{d} \Omega}$ 与下列哪个物理量无关\\
(A) 核电荷$\quad$
(B) 能量 $E$ $\quad$
(C) 散射角$\quad$
(D) 靶原子密度.