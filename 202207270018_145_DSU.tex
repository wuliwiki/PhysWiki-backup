% 并查集
% keys 并查集|数据结构|C++

并查集是一个树形的数据结构,用于维护不相交的集合的数据结构.

对于并查集,主要有如下操作:
\begin{enumerate}
\item $\mathtt{merge}$ 合并两个集合;(“并”)

\item $\mathtt{find}$ 判断两个元素是否属于同一个集合.(“查”)
\end{enumerate}

定义集合的表示方法:“代表元”法,\textbf{即每个集合都会选择一个固定的元素,作为这个集合的代表}.其次需要定义归属关系的表示方法:\textbf{使用一个树形结构存储每个集合,树上的每个结点都是一个元素,树根是集合的代表元素.}我们用 \verb|p[i]| 表示每个结点的父结点,初始化 \verb|p[i]| 都指向自己,树根也指向自己.在合并两个集合时,只需要将其中一个树根的父节点指向另一个树根,即 \verb|p[root_1] = root_2|.在查询每个集合的树根时,朴素的办法就是通过 \verb|p[i]| 存储的值不断递归访问父节点,直至到达树根.为了提高效率,我们引入了\textbf{路径压缩}这种优化方法.

在进行合并和查询的操作中,我们只关心每个集合的根节点,\textbf{所以我们在 $\mathtt{find}$ 操作的时候,把访问过的每个结点都直接指向树根,}这种优化方法被称为\textbf{路径压缩},进行完路径压缩之后,每个结点的父节点都为根节点了.加上路径压缩的并查集每次 $\mathtt{find}$ 的操作均摊时间复杂度为 $O(\log N)$.

还有一种优化方法为\textbf{按秩合并},单独采用“按秩合并”的并查集每次 $\mathtt{find}$ 的操作均摊时间复杂度也为 $O(\log N)$,