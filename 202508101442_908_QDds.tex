% 乔丹代数(综述)
% license CCBYSA3
% type Wiki

本文根据 CC-BY-SA 协议转载翻译自维基百科\href{https://en.wikipedia.org/wiki/Jordan_algebra}{相关文章}。

在抽象代数中,乔丹代数是定义在一个域上的非结合代数,其乘法满足以下公理:
\begin{enumerate}
\item 交换律:$xy = yx$
\item 乔丹恒等式:$(xy)(xx) = x(y(xx))$
\end{enumerate}
在乔丹代数中,两个元素 $x$ 和 $y$ 的乘积也常记作 $x \circ y$,尤其是在需要避免与相关结合代数的乘法混淆时。

这些公理推出\(^\text{[[1]]}\)乔丹代数是幂结合的,即:$x^n = x \cdots x$的值与括号的放置方式无关。它们还推出\(^\text{[[1]]}\):$x^{m}(x^{n}y) = x^{n}(x^{m}y)$对所有正整数 $m$ 和 $n$ 都成立。因此,我们也可以等价地将乔丹代数定义为:一个交换的、幂结合的代数,并且对于任意元素 $x$,所有“与 $x^n$ 相乘”的运算彼此交换。

乔丹代数由帕斯夸尔·乔丹于 1933 年引入,目的是形式化量子电动力学中可观测量代数的概念。尽管很快就发现这种代数在该物理背景中并不实用,但此后它们在数学中找到了许多应用\(^\text{[[2]]}\)。这种代数最初被称为“r-数系统”,后来在 1946 年被亚伯拉罕·阿德里安·阿尔伯特改名为“乔丹代数”,并由他开始了对一般乔丹代数的系统研究。
\subsection{特殊乔丹代数}
首先注意,一个结合代数当且仅当它是交换的时才是乔丹代数。

给定任意结合代数 $A$(特征不为 2),可以利用相同的加法运算并引入一种新的乘法来构造一个乔丹代数 $A^+$,其乘法(乔丹积)定义为:
$$
x \circ y = \frac{xy + yx}{2}~
$$
这些乔丹代数及其子代数被称为特殊乔丹代数,而其他不是通过这种方式得到的乔丹代数称为例外乔丹代数。这一构造类似于与 $A$ 相关的李代数,其乘法(李括号)定义为交换子:$[x, y] = xy - yx$

Shirshov–Cohn 定理指出,任意由两个生成元构成的乔丹代数都是特殊的\(^\text{[[3]]}\)。与此相关,Macdonald 定理指出:任意在三个变量中的多项式,如果在其中一个变量上的次数为 1,并且它在每一个特殊乔丹代数中恒为零,那么它在所有乔丹代数中也恒为零\(^\text{[[4]]}\)。
\subsubsection{Hermitian 乔丹代数}
如果 $(A, \sigma)$ 是一个带有对合$\sigma$ 的结合代数,那么若 $\sigma(x) = x$ 且 $\sigma(y) = y$,就有:$\sigma(xy + yx) = xy + yx$因此,由对合所不变的所有元素(有时称为Hermitian 元素)组成的集合在 $A^+$ 中构成一个子代数,有时记作 $H(A, \sigma)$。
\subsection{例子}
\begin{enumerate}
\item 由所有自伴的实矩阵、复矩阵或四元数矩阵组成的集合,在乘法
$$
(xy + yx) / 2~
$$
下构成一个特殊乔丹代数。
\item 由所有 3×3 自伴的八元数矩阵组成的集合,在同样的乘法
$$
(xy + yx) / 2~
$$
\end{enumerate}
下构成一个 27 维的例外乔丹代数(之所以是例外的,是因为八元数不是结合的)。这是第一个Albert 代数的例子。它的自同构群是例外李群 $F_4$。在复数域上,这个代数在同构意义下是唯一的单纯例外乔丹代数\(^\text{[[5]]}\),因此它通常被称为“那个例外乔丹代数”。在实数域上,单纯例外乔丹代数共有三种同构类\(^\text{[[5]]}\)。
\subsection{导子与结构代数}
乔丹代数 $A$ 的一个导子是 $A$ 的一个自同态 $D$,满足:$D(xy) = D(x)y + xD(y)$所有导子构成一个李代数,记作 $\mathrm{der}(A)$。乔丹恒等式推出:如果 $x, y \in A$,则将 $z$ 映射为$x(yz) - y(xz)$的映射是一个导子。因此,$A$ 与 $\mathrm{der}(A)$ 的直和可以构成一个李代数,称为 $A$ 的结构代数,记作 $\mathrm{str}(A)$。

一个简单的例子由 Hermitian 乔丹代数 $H(A, \sigma)$ 提供。在这种情况下,若 $x \in A$ 且满足 $\sigma(x) = -x$,则 $x$ 定义了一个导子。在许多重要的例子中,$H(A, \sigma)$ 的结构代数就是 $A$ 本身。导子与结构代数也是蒂茨构造Freudenthal 魔方的一部分。
\subsection{形式实乔丹代数}
在实数域上的(可能是非结合的)代数,如果满足这样一个性质:若 $n$ 个平方的和为零,则每一个平方项都必须单独为零,就称为形式实。1932 年,乔丹尝试通过公理化的方式刻画量子理论,他提出:任何量子系统的可观测量代数都应当是一个形式实代数,并且是交换的($xy = yx$)且幂结合的(结合律在仅涉及 $x$ 的乘积中成立,从而任意元素 $x$ 的幂是无歧义的)。他证明了任何这样的代数都是一个乔丹代数。

并非所有乔丹代数都是形式实的,但乔丹、冯·诺伊曼和维格纳(Jordan, von Neumann & Wigner,1934)对有限维的形式实乔丹代数(也称为欧几里得乔丹代数)进行了分类。任何形式实乔丹代数都可以写成若干所谓单纯的代数的直和,而单纯代数本身不能以非平凡的方式再分解为直和。在有限维情形下,单纯形式实乔丹代数分为四个无限族,加上一个例外情形:
\begin{itemize}
\item $n \times n$ 自伴实矩阵的乔丹代数(同前所述)。
\item $n \times n$ 自伴复矩阵的乔丹代数(同前所述)。
\item $n \times n$ 自伴四元数矩阵的乔丹代数(同前所述)。
\item 由 $\mathbb{R}^n$ 自由生成,并满足关系式
$$
x^2 = \langle x, x \rangle~
$$
的乔丹代数,其中右侧是用 $\mathbb{R}^n$ 上的通常内积定义的。有时称为旋量因子或克利福德型乔丹代数。
\item $3 \times 3$ 自伴八元数矩阵的乔丹代数(同前所述,即例外乔丹代——Albert 代数)。在这些可能性中,到目前为止,自然界似乎只将 $n \times n$ 自伴复矩阵用作可观测量的代数。然而,旋量因子在狭义相对论中起着作用,而所有形式实乔丹代数都与射影几何有关。
\end{itemize}
