% SQL 入门(SQLite 为例)
% license Xiao
% type Tutor

\subsection{SQL简单介绍}
SQL数据库一般用于存储程序的各种数据。对于简单的程序来说,可能一个简单的文本文件或表格就足够了。但是如果数据量比较大,且他们之间联系比较多的时候,我们进行数据查询的时候就会很复杂。

简单来说,SQL数据库和办公软件的表格(Excel)有很多相似之处。在一个数据库中,可以创建许多个表格。表格的每一列可以有不同的数据类型,比如整数,小数,文本等。你可以使用SQL专有的命令,对表格中的数据进行插入,修改,删除,查询等操作。

\subsection{SQLite使用教程}
SQL有许多不同的软件,例如MySQL,ProstgreSQL,SQLite等,为了简单起见,我们以SQLite软件作为入门演示。

可以在\href{https://sqlitebrowser.org/dl/}{官方网站下载页面}下载,选择你需要的操作系统的安装包。安装完成如下图:

\begin{figure}[ht]
\centering
\includegraphics[width=14.25cm]{./figures/0efb331e4729b6f7.png}
\caption{SQLite主页} \label{fig_SQLint_1}
\end{figure}

根据需求创建数据库(相当于表格),我们以学生-课程数据库为例。

\begin{figure}[ht]
\centering
\includegraphics[width=14.25cm]{./figures/81626c61a8ee6c7d.png}
\caption{点击创建数据库} \label{fig_SQLint_2}
\end{figure}

\begin{figure}[ht]
\centering
\includegraphics[width=14.25cm]{./figures/013db632fb5ccad6.png}
\caption{保存在合适的位置} \label{fig_SQLint_3}
\end{figure}


\begin{figure}[ht]
\centering
\includegraphics[width=14.25cm]{./figures/15dfd0190d5b237c.png}
\caption{新增表} \label{fig_SQLint_5}
\end{figure}

\begin{itemize}
\item 填入表的名称(示例的表名为student)
\item 点击“增加”可增加字段(如图4),一个字段就是就相当于表格的一列,同理也可以进行删除。
\item 类型可以按需选择 INTEGER(整数类型),TEXT(文本类型),BLOB(二进制数据类型),BLOB(浮点类型),NUMERIC(用于高精度计算)。
\\例如:学号(Sno)属于整数类型  名字(Sname)属于文本类型。
\item 一个表格有一个或多个主键,它用于唯一标识表中的特定记录。例如Sno,学生的学号是可以唯一标识一位学生的,所以学号是一个学生的主键。简单来说就是,知道了学生的学号,就能找到对应的学生。如果只知道一个学生的性别Ssex,是不能找到的,这也就是为什么性别不能作为主键的原因。
\item 图4下方的语句属于SQL语句,基础SQL语句参考SQL 入门语法\upref{SQLgrm}。基础好的也可以直接用SQL语句建表。
\end{itemize}

建表成功后如图所示:
\begin{figure}[ht]
\centering
\includegraphics[width=12.5cm]{./figures/bd90e827cd376352.png}
\caption{建表成功} \label{fig_SQLint_4}
\end{figure}

表建立完成以后,我们需要网表里插入一些数据,具体操作如下:
\begin{figure}[ht]
\centering
\includegraphics[width=12.5cm]{./figures/784804ba9f7f24db.png}
\caption{插入数据} \label{fig_SQLint_6}
\end{figure}
% 插入“插入值”图片
这里我插入了四条数据:
\begin{figure}[ht]
\centering
\includegraphics[width=12.5cm]{./figures/e64feb77288146a0.png}
\caption{插入四条数据} \label{fig_SQLint_7}
\end{figure}

\begin{figure}[ht]
\centering
\includegraphics[width=12.5cm]{./figures/6e3ab89cfd4936cf.png}
\caption{} \label{fig_SQLint_8}
\end{figure}
% 插入“写入值”图片
% “四条数据”图片
当然这里也可以使用SQL语句进行写入值:
\begin{figure}[ht]
\centering
\includegraphics[width=12.5cm]{./figures/e0206e6e517b1325.png}
\caption{} \label{fig_SQLint_9}
\end{figure}
% “sql语句写入值”
% “运行sql语句”
作为示例,我建立了三个表格
\begin{itemize}
\item 学生表(Student):学号,姓名,性别,年龄。
\item 课程表(Course):课程号,课程名,学分。
\item 成绩表(SC):学号,课程号,成绩。
\end{itemize}

\begin{figure}[ht]
\centering
\includegraphics[width=12.5cm]{./figures/c71d921783017f84.png}
\caption{} \label{fig_SQLint_10}
\end{figure}
% “建立三个表格”

在这里,SC(成绩表)表中我们定义了一个外键约束,它引用了其他两个表(Student表中的学号,Course表中的课程号)。SQL语句如图11:
% “建立SC表”
\begin{figure}[ht]
\centering
\includegraphics[width=12.5cm]{./figures/1fdc442d2528b97b.png}
\caption{} \label{fig_SQLint_11}
\end{figure}

建立完成SC(成绩表)并插入数据后如图所示:
\begin{figure}[ht]
\centering
\includegraphics[width==12.5cm]{./figures/5a95b5a30d8e785b.png}
\caption{} \label{fig_SQLint_12}
\end{figure}

\subsection{数据库的增,删,改,查}
使用SQLite建立自己需要的数据库后,在后续的使用过程中,进行数据的插入,删除,修改和插入都是非常方便的。
基于上面建立的三个表,添加了一些数据用于给大家举例子。(具体规范的语句格式参考SQL 入门语法\upref{SQLgrm})
\subsubsection{1.插入数据}

\begin{itemize}
\item 假如班级里转来一位新同学小赵,我们需要将他的一些信息加入到student(学生表),只需要在SQL命令框输入sql语句:

\begin{lstlisting}[language=bash]
insert  into student(Sno,Sname,Ssex,Sage)
values('5','小赵','男','22');
\end{lstlisting}
执行成功后系统会自动添加一条记录到student表中:
\begin{figure}[ht]
\centering
\includegraphics[width=12.5cm]{./figures/48c40e7ef9fb7bd5.png}
\caption{} \label{fig_SQLint_13}
\end{figure}
\end{itemize}

\begin{itemize}
\item 对于每一门课程,求学生的平均成绩,并把结果存入数据库中:
\begin{lstlisting}[language=bash]
首先在数据库中建立一个新表,其中一列存放课程号,另一列存放学生对应该课程的平均成绩

Create Table Avg_Grade
(
    Cno INTEGER,
    Avg INTEGER
);

\end{lstlisting}
运行成功后,数据库会自动建立一个表:
\begin{figure}[ht]
\centering
\includegraphics[width=10cm]{./figures/e65ada344faf2d72.png}
\caption{} \label{fig_SQLint_18}
\end{figure}

\begin{lstlisting}[language=bash]
然后对成绩表(SC)按课程号分组求平均成绩,再把课程号和平均成绩存入新表中

insert into Avg_Grade(Cno,Avg_Grade)
select Cno,AVG(Grade)
from SC
group by Cno;
\end{lstlisting}
执行成功后如图:
\begin{figure}[ht]
\centering
\includegraphics[width=10cm]{./figures/de0f6f307c7bc556.png}
\caption{} \label{fig_SQLint_19}
\end{figure}
\end{itemize}





\subsubsection{2.查询数据}
\begin{itemize}
\item 查询指定列
\begin{lstlisting}[language=bash]
例如从成绩表中查询小李的数据库成绩:
select Grade
from SC
where Sno='1' and Cno='2001';
\end{lstlisting}执行成功如图:
\begin{figure}[ht]
\centering
\includegraphics[width=10cm]{./figures/57d3503e4f025a5e.png}
\caption{} \label{fig_SQLint_21}
\end{figure}
\end{itemize}

\begin{itemize}
\item 查询全部列
\begin{lstlisting}[language=bash]
例如查询全体学生的学号和姓名:
select Sno,Sname
from student;
\end{lstlisting}

执行成功如图:
\begin{figure}[ht]
\centering
\includegraphics[width=10cm]{./figures/7f363c66a962a5cb.png}
\caption{} \label{fig_SQLint_20}
\end{figure}
\end{itemize}

\begin{itemize}
\item 查询满足条件的元组
\begin{lstlisting}[language=bash]
li
\end{lstlisting}
\end{itemize}


\subsubsection{3.修改数据}
将小李的数据库成绩改为90
\begin{lstlisting}[language=bash]
由学生表可知小李的学号为1,数据库的课程号为2001
update SC
set Grade='90'
where Sno='1' and Cno='2001';
\end{lstlisting}
修改后如图所示:
\begin{figure}[ht]
\centering
\includegraphics[width=6cm]{./figures/442d9e6dd3684cc6.png}
\caption{} \label{fig_SQLint_15}
\end{figure}
当然,对于SQLite,还有一种很简便的修改方法,例如将小赵的年龄改为21岁:
\begin{figure}[ht]
\centering
\includegraphics[width=12.5cm]{./figures/3488f0411dab1e51.png}
\caption{} \label{fig_SQLint_16}
\end{figure}




