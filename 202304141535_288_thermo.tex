% 热学(高中)

\begin{issues}
\issueTODO
\end{issues}
% 分子动理论|气体等x定律|固体液体|热力学定律

%\pentry{机械运动基础\upref{HSPM01}, 相互作用\upref{HSPM02}}
%\pentry{词条示例\upref{Sample}}
\subsection{分子动理论}
\subsubsection{对于分子的直观认识}
%什么是分子|分子的特征(大小/数量)
当不需要区分分子、原子或者离子在化学变化中所起的作用不同时,而仅仅研究物体的热运动性质以及规律,可以将组成物体的微粒统称为\textbf{分子}。

我们不能直接观察分子,而必须借助高分辨率的显微镜,比如说扫描隧道显微镜\footnote{扫描隧道显微镜是一种可以个探测物质表面结构的仪器。其工作原理是利用探针和物质表面的相互作用来获得物质表面结构的图像信息,分辨尺度为原子尺度。}。这说明分子的尺度十分小,大概在$10^{-10}\mathrm{m}$数量级。直观来感受的话,分子的大小比之于弹珠,则相当于一颗苹果与地球的体积相比拟。

与此同时,由于物质是由分子组成的,所以我们还能合理得到一个结论,即:在宏观尺度中,物质中所包含分子的数目很多,为$10^{23}$数量级。事实上,$1\mathrm{mol}$的任何物质都含有相同的粒子数,这个数目为阿伏伽德罗常数,$N_A=6.02214076\times 10^{23}\mathrm{mol^{-1}}$。由于物体是由大量分子组成的,因此在后续我们研究中,认为体系均满足热力学极限\footnote{热力学极限是指粒子数(或者体积)趋近于无限大时的极限。}。
\subsubsection{分子热运动}
扩散|布朗运动|热运动
在生活中我们可以发现,不同种的物质能够相互进入彼此,这类现象叫做\textbf{扩散}。这种现象并非是由外力引起的,也即并非是由于重力作用、对流等造成,而是物质分子自身具有永不停息的无规则运动的体现,这种无规则的运动称之为\textbf{布朗运动}。

布朗运动最开始由英国植物学家布朗在显微镜下观察到,他发现悬浮在水中的花粉颗粒会进行移动,将其运动轨迹连接起来可以发现这样的运动是无规则的。布朗证明,这种移动并非由于花粉颗粒具有生命而产生。当花粉颗粒越小,则运动就越明显,这是由于花粉越小,单位时间内撞击花粉表面的水分子数目就越不平衡,另外,花粉越小,它的质量也就越小,运动状态就越容易发生改变,布朗运动也就越明显。

同样,升温也可以加剧花粉颗粒的运动。在扩散现象中,温度越高,扩散也就越快。由于这种分子运动和温度之间的密切相关关系,因此,我们将分子这种永不停息的无规则运动叫做\textbf{热运动}。温度是分子热运动剧烈程度的标志。

% 可以考虑画一张随机行走的图,来冒充花粉轨迹
\subsubsection{分子间作用力}
分子间空隙|分子间作用力
通过对日常生活的观察,我们知道气体很容易被压缩,这是因为气体分子之间存在很大的空隙。而两种不同液体相互混合之后,总体积会小于原先单独体积的加和;jinjinyazai
\subsubsection{分子动理论}
基本认识……

\subsection{……}
