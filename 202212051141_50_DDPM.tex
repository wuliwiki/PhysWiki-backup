% 去噪扩散概率模型
% 扩散 马尔科夫

\textbf{去噪扩散概率模型}(Denoising Diffusion Probabilistic Models, DDPM)是一种参数化的马尔科夫链,通过变分推理的方法来训练.去噪扩散概率模型(后文简称扩散模型)是深度生成模型的一种,通常包含两个过程:第一是前向扩散过程,第二是反向的逆扩散过程.正反两个方向的马尔科夫链均由有限个时间步组成.其中,前向扩散过程就是一个无参数的马尔科夫链,而反向的逆扩散过程须要学习算法来训练模型.模型结构如图1所示.
\begin{figure}[ht]
\centering
\includegraphics[width=14cm]{./figures/DDPM_1.png}
\caption{去噪扩散概率模型的结构} \label{DDPM_fig1}
\end{figure}

设源数据为$X_0$,$t$为时间变量,某个扩散过程中有$T$个时间步.

前向扩散过程会逐渐向原数据添加小幅的高斯噪音,时间序列为:$t=0$ -> $1$ -> $2$ -> ... -> $T-1$ -> $T$.每一步所添加的高斯噪音的方差序列为:$\beta_1$, $\beta_2$, ..., $\beta_T$.

从$t-1$到$t$的转换概率为$q(X_t|X_{t-1})$,则从源数据到扩散过程最后一步的转换概率为:
\begin{equation}
q(X_{1:T}|X_0)=\prod_{t=1}^{T}q(X_{t}|X_{t-1})
\end{equation}
其中,
\begin{equation}
q(X_{t}|X_{t-1})=N(X_t;\sqrt{1-\beta_t}X_{t-1},\beta_tI)
\end{equation}


参考文献:
\begin{enumerate}
\item J. Ho, A. Jain, and P. Abbeel, “Denoising Diffusion Probabilistic Models,” in Advances in Neural Information Processing Systems, 2020, vol. 33, pp. 6840–6851.
\end{enumerate}