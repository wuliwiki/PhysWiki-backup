% SLISC 的 Mcoo 矩阵类
% keys C++|SLISC|密矩阵|稀疏矩阵|非零元
% license Xiao
% type Tutor

\begin{issues}
\issueDraft
\end{issues}

\pentry{SLISC 的密矩阵类\nref{nod_SliMat}}{nod_b4e2}

当一个矩阵对象中有大量 0 元的时候, 用密矩阵数据结构会浪费内存和计算量。 COO 是一种常用的稀疏矩阵, 它的数据结构是把所有的非零元存放在一个矢量中, 并用另外两个矢量分别记录这些非零元的行标和列标。

在小时百科的 SLISC 库中, 我们根据这种数据结构定义了 \verb`McooDoub, McooComp` 等矩阵类, 它和密矩阵一样继承一个 \verb`Vbase` 类来储存非零矩阵元, 另外 \verb`m_N0, m_N1` 用于储存矩阵尺寸, \verb`m_Nnz` 储存非零元的长度, 该长度未必是 \verb`Vbase` 的长度, 后者通过 \verb`reserve(Long_I)` 成员函数来调整尺寸。 声明如下:
\begin{lstlisting}[language=cpp]
class McooDoub : public VbaseDoub
{
private:
    typedef VbaseDoub Base;
    using Base::m_p;
    using Base::m_N;
    Long m_N0, m_N1, m_Nnz;
    VecLong m_row, m_col;
public:
    using Base::p;
    McooDoub(): m_N0(0), m_N1(0), m_Nnz(0) {};
    McooDoub(Long_I N0, Long_I N1);
    McooDoub(Long_I N0, Long_I N1, Long_I Ncap);
    Long *row_p();
    const Long *row_p() const;
    Long *col_p();
    const Long *col_p() const;
    void push(Doub_I s, Long_I i, Long_I j);
    void set(Doub_I s, Long_I i, Long_I j);
    Long n0() const;
    Long n1() const;
    Long size() const;
    Long nnz() const;
    Long capacity() const;
    Long find(Long_I i, Long_I j) const;
    Doub &ref(Long_I i, Long_I j);
    Doub operator()(Long_I i, Long_I j) const;
    Doub &operator[](Long_I ind);
    Doub operator[](Long_I ind) const;
    Long row(Long_I ind) const;
    Long col(Long_I ind) const;
    void trim(Long_I Nnz);
    void resize(Long_I N);
    void reserve(Long_I N);
    void reshape(Long_I N0, Long_I N1);
};
\end{lstlisting}
