% 核合成
% license CCBYSA3
% type Wiki

(本文根据 CC-BY-SA 协议转载自原搜狗科学百科对英文维基百科的翻译)

核合成是从已有的核子(质子和中子)中创造新原子核的过程。第一个原子核是在大爆炸后大约三分钟,通过大爆炸核合成过程形成的。17分钟后,宇宙已经冷却到这些反应结束的温度,因此只发生了最快和最简单的反应,致使我们的宇宙中含有大约75\%的氢、24\%的氦和少量的其他元素,如锂和氢同位素氘。今天宇宙的成分仍然大致相同。

较重的原子核是通过几个过程从这些原子核中产生的。恒星形成后,开始在其核心将轻元素与重元素融合,在这个过程中释放能量,这就是所谓的恒星核合成。聚变过程产生了许多更轻的元素,包括铁和镍,当较小的恒星脱离它们的外壳,变成称为白矮星的较小恒星时,这些元素被喷射到太空(星际介质)中。它们喷射出的物质的残余物形成了行星状星云,可以在银河系中被观测到。

通过碳和氧的聚合,爆炸恒星中的超新星核合成是镁(原子序数12)和镍(原子序数28)之间元素丰度的原因。[1] 超新星核合成发生在第二类超新星事件的最后几秒钟,也被认为是产生比铁和镍重的稀有元素的原因。这些较重元素的合成吸收了超新星爆炸时产生的能量(吸热过程)。其中一些元素是由爆炸过程中几秒钟内吸收多个中子(r过程)产生的。超新星中形成的元素包括已知的最重元素,如长寿命元素铀和钍。

中子星的合并和碰撞也是产生许多重元素的主要原因,而这些重元素是通过r过程产生。中子星是超新星极其密集的残余物,顾名思义,它们由复杂的物质状态组成,主要由紧密堆积的中子组成。当两颗这样致密的恒星碰撞时,大量富含中子的物质可能在极高的温度和奇异的条件下被喷射出来,当喷射物开始冷却时,可能会形成重元素。2017年,GW170817的合并导致在较长时间内检测到大量金、铂和其他重元素的特征。

宇宙射线散裂是宇宙射线撞击星际介质并破碎较大的原子种类时引起的,是较轻原子核的重要来源,特别是$^{3}He$、$^{9}Be$和$^{10}$,$^{11}B$,它们不是由恒星核合成产生的。

除了导致宇宙中元素日益丰富的聚变过程之外,一些微小的自然过程继续在地球上产生非常少量的新核素。这些核素对其丰度贡献很小,但可能解释了特定新核的存在。这些核素是通过铀和钍等长寿命、重、原始放射性核素的放射性生成(衰变)产生的。宇宙射线轰击地球上的元素也促成了被称为宇宙成因核素的稀有、短寿命原子种类的出现。

\subsection{时间表}
\begin{figure}[ht]
\centering
\includegraphics[width=14.25cm]{./figures/45899aef79ecacb0.png}
\caption{元素周期表,显示出每种元素的宇宙起源。从碳到硫的元素可以通过阿尔法过程在小恒星中形成。铁以外的元素是在具有慢中子俘获(s过程)的大型恒星中形成的,然后在气体喷射中被驱逐到太空中(见行星状星云)。比铁重的元素可能是在中子星合并或超新星的r过程后形成的,包括中子的密集爆发和元素的快速捕获。} \label{fig_HHC_1}
\end{figure}
人们认为原始核子本身是由夸克-胶子等离子体在大爆炸期间冷却到了2万亿度以下形成的。爆炸发生几分钟后,开始只有质子和中子,直至后来形成了锂和铍(质量都是7)原子核,但是随着原子质量的增加,其他元素的丰度急剧下降。此时可能已经形成了一些硼,但是在形成大量碳之前,这一过程就停止了,因为这种元素需要比大爆炸的短核合成时期高得多的氦密度和时间的产物。由于宇宙继续膨胀,温度和密度下降,这一聚变过程基本上在大约20分钟后停止。这第一个过程,大爆炸核合成,是宇宙中发生的第一种类型的核合成。

较重元素的后续核合成需要恒星和超新星内的极端温度和压力。这些过程始于宇宙大爆炸产生的氢和氦,它们在5亿年前坍缩成第一颗恒星。从那时起,星系中恒星不断形成。在地球上自然发现的元素(所谓的原始元素)中,那些比硼重的元素是由恒星核合成和超新星核合成产生的。它们的原子序数从Z=6(碳)到Z=94(钚)不等。这些元素的合成要么通过核聚变(包括快速和慢速多次中子俘获),要么在较小程度上通过核裂变和随后的β衰变进行。

恒星通过结合其较轻的原子核、氢、氘、铍、锂和硼获得较重的元素,这些元素是在星际介质的初始成分中发现的,因而也是恒星。因此,星际气体中这些轻元素的含量不断下降,而这些轻元素只能借助于在大爆炸期间核合成而产生。所以,目前宇宙中大量较轻的元素被认为是通过数十亿年的宇宙射线(主要是高能质子)介导的星际气体和尘埃中较重元素的分解而恢复的。这些宇宙射线碰撞的碎片包括轻元素锂、铍和硼。

\subsection{核合成理论的历史}
关于核合成的第一个想法是简单地认为化学元素是在宇宙开始时产生的,但是没有合理的物理方案可以确定。渐渐地,氢和氦显然比任何其他元素都丰富得多。其余的都不到太阳系质量的2\%,其他恒星系统也是如此。与此同时,很明显,氧和碳是接下来两种最常见的元素,也有一个普遍的趋势,即轻元素的丰度很高,特别是那些由氦-4原子核组成的元素。

亚瑟·斯坦利·爱丁顿在1920年首次提出,恒星通过将氢融合成氦来获得能量,并提出了了较重元素也可能在恒星中形成的可能性。[2][3] 这个想法没有被普遍接受,因为核机制没有被理解。就在第二次世界大战前几年,汉斯·贝特首次阐明了氢融合成氦的核机制。

第二次世界大战不久之后,弗雷德·霍伊尔首次展开关于恒星中较重元素核合成的工作。[4] 他从氢开始解释了所有较重元素的产生。霍伊尔提出氢是由真空和能量在宇宙中不断产生的,不需要宇宙的开始。

霍伊尔的工作解释了随着银河系的老化,元素的丰度是如何随着时间而增加的。随后,60年代,威廉福勒、阿拉斯泰尔卡梅伦和唐纳德D克莱顿的贡献扩大了霍伊尔的工作,其他许多人紧随其后。布尔比吉夫妇、佛罗和霍伊尔[5] 在1957年发表的开创性评论文件是对该领域状况的著名总结。那篇论文定义了在恒星内部一个重核转变为另一个重核的新过程,天文学家可以将这些过程记录下来。

大爆炸这一概念早在1931年就由比利时物理学家乔治·勒梅特尔提出,他认为宇宙在时间上的明显膨胀性表明,如果宇宙在时间上收缩,那么宇宙将继续这样,直到它不能再收缩为止。这将把宇宙的所有质量带到一个单点,一个“原始原子”,带到一个时间和空间都不存在的状态。人们认为,霍伊尔在1949年英国广播公司的一次广播中创造了“大爆炸”一词,他说勒梅特尔(Lemaître)的理论是“基于‘宇宙中所有物质都是在遥远过去的某个特定时间,在一次大爆炸中产生的’这一假设”。据普遍报道,霍伊尔有意贬低勒梅特尔的理论,但霍伊尔明确否认了这一点,并表示这只是一个引人注目的形象,意在突出这两种模型之间的差异。勒迈特尔的模型被用来解释氘,氦与碳之间核素的存在,以及存在于恒星和星际空间中的大量氦。事实上,勒梅特尔和霍伊尔的核合成模型都能用来解释宇宙中的元素丰度。

核合成理论的目标是从自然过程的角度解释化学元素及其几种同位素的巨大差异。这一理论发展的主要推动力是元素丰度与原子序数的曲线图。当这些丰度作为原子序数的函数绘制在图表上时,会有锯齿状的锯齿结构,其变化系数高达一千万。汉斯·苏斯和哈罗德·尤里创造了一个丰度表,这是对核合成研究的一个非常有影响的刺激因素,该表是基于在未演化陨石中发现的非挥发性元素的未分类丰度。[6] 这样的丰度图以下面的对数标度显示,在这个标度上,显著参差不齐的结构在视觉上被这个图的垂直标度上跨越的十次幂所抑制。关于同位素丰度的更多数据和讨论,请参见《宇宙同位素手册》。[1]
\begin{figure}[ht]
\centering
\includegraphics[width=14.25cm]{./figures/448aa50d671559b5.png}
\caption{太阳系中化学元素的丰度。氢和氦是最常见的残留物,属于宇宙大爆炸的范例。[7] 接下来的三种元素(Li, Be, B)是稀有的,因为它们在宇宙大爆炸和恒星中很难合成。剩余恒星产生元素的两种普遍趋势是:(1)元素的丰度根据它们的原子序数是偶数还是奇数而变化;(2)随着元素变得更重,丰度普遍下降。 在这一趋势中,铁和镍的丰度达到峰值,这在对数图上表现得尤为明显,对数图的幂次小于10,比如在logA=2 (a =100)和logA=6 (a =1,000,000)之间。} \label{fig_HHC_2}
\end{figure}

\subsection{反应}
有许多天体物理过程被认为是核合成的原因。其中大多数发生在恒星内部,一些有名的链式核聚变反应包括有氢燃烧(通过质子-质子链或氯化萘循环)、氦燃烧、碳燃烧、氖燃烧、氧燃烧和硅燃烧。这些反应能够创造出铁和镍等元素。这是核合成的区域,在这个区域内,每个核子具有最高结合能的同位素被创造出来。较重的元素可以通过称为s过程的中子俘获过程在恒星内部组装,或者在爆炸环境中,例如超新星和中子星合并,通过许多其他过程组装。其他一些如r-反应,包含了快速中子俘获,RP-反应,和p-反应(有时称为γ过程),这些导致现有原子核的光分解。

\subsection{主要类型}
\subsubsection{4.1 大爆炸核合成}
大爆炸核合成[8] 发生在宇宙开始的前三分钟内,是$^1H$(氕)、$^2H$(氘)、$^3He$(氦-3)和$^4He$(氦-4)丰度的主要来源。宇宙中大多数同位素被认为是在大爆炸中产生的,尽管4He继由恒星聚变和$\alpha$衰变产生,微量的$^1H$由散裂和某些类型的放射性衰变产生。这些元素的原子核以及一些$^7Li$和$^7Be$被认为是在大爆炸后100到300秒之间形成的,当时原始夸克-胶子等离子冻结形成质子和中子。由于核合成在膨胀和冷却(约20分钟)停止之前发生的时间非常短,所以不会形成比铍(或可能是硼)重的元素。在此期间形成的元素处于等离子状态,直到很久以后才冷却到中性原子状态。
\begin{figure}[ht]
\centering
\includegraphics[width=14.25cm]{./figures/6d6a0f56146a5320.png}
\caption{} \label{fig_HHC_3}
\end{figure}
主要的核反应负责在整个宇宙中观察到的光
\subsubsection{4.2 恒星核合成}
恒星核合成是产生新核的核反应,它发生在恒星进化的过程中。它负责银河系中从碳到铁元素的丰度。恒星是热核熔炉,随着核心成分的演变,高温使氢和氦融合成更重的核。[9] 尤其重要的是碳,因为从氦中形成碳是整个反应的瓶颈。所有恒星的碳都是由三α反应产生的。碳也是导致恒星释放自由中子的主要元素,它会引发s过程,在这个过程中,中子的缓慢吸收将铁转化为比铁和镍重的元素。[10][11]

恒星核合成的产物通常通过质量损失阶段和低质量恒星的恒星风分散到星际气体中。今天,质量损失事件可以在低质量恒星演化的行星状星云阶段,以及那些质量超过太阳八倍的恒星(称为超新星)的爆炸式毁灭中看到。

恒星中发生核合成的第一个直接证据是天文观测,随着时间的推移,星际气体中的重元素越来越丰富。其结果是,从银河系的晚期诞生的恒星,形成的初始重元素丰度比早期形成的高得多。1952年通过光谱学在一颗红巨星的大气中检测到锝元素,这提供了恒星内部核活动的第一个证据。[12] 因为锝具有放射性,半衰期比恒星的年龄短得多,所以它的丰度一定能反映出它最近在恒星中的产生的物质。同样令人信服的重元素恒星起源的证据是,在渐近巨型分支恒星的大气中发现了大量特定稳定元素。对钡元素丰度的观测比在未演化的恒星中发现的钡元素丰度高20-50倍左右,这是此类恒星内部s反应的证据。恒星核合成的许多现代证据都是由星尘的同位素组成提供的,星尘是从单个恒星的气体中凝聚而成的固体颗粒,是从陨石中提取出来的。星尘是宇宙尘埃的一个组成部分,通常被称为前太阳系颗粒。星尘颗粒中测得的同位素组成显示了恒星内部核合成的许多方面,在恒星晚期质量损失事件中,颗粒发生凝聚。[13]
\subsubsection{4.3 爆炸性核合成}
超新星核合成发生在超新星的高能环境中,在这种环境中,硅和镍之间的元素是在快速聚变过程中建立的准平衡状态下合成的,[14] 通过对28Si往复平衡核反应联系在一起的。准平衡可以被认为是几乎平衡的,除了在炽热燃烧的混合物中有大量的28Si原子核。自霍伊尔1954年的论文以来,这个概念[11] 是量元素核合成理论中最重要的发现,因为它提供了对硅($A=28$)和镍($A=60$)之间丰富且化学上重要的元素的总体理解。它取代了B2FH论文中被大量引用的不正确的阿尔法过程,后者无意中模糊了霍伊尔1954年的理论。[15] 还可能发生进一步的核合成过程,特别是B2FH论文描述的r过程(快速过程),首先由西格、福勒和克莱顿计算,[16] 其中比镍重的元素中,最富中子的同位素是通过快速吸收自由中子产生的。超新星核心快速压缩过程中通过电子俘获产生自由中子,以及一些富含中子的种子核的组装,使得r过程成为一个主要过程,甚至在纯氢和氦的恒星中也可能发生。这与B2FH将该工艺指定为二级工艺形成对比。尽管这一前景光明的设想得到了超新星专家的普遍支持,但尚未实现对r过程丰度的令人满意的计算。天文学家们已经证实了主要的r过程,他们观察到了在星系金属性仍然很小的时候诞生的旧恒星,尽管它们仍然含有r过程原子核的补充,从而证明金属度是内部反应的产物。r过程负责我们天然的放射性元素队列,如铀和钍,以及每种重元素中最富中子的同位素。

Rp反应(快速质子)包括自由质子和中子的快速吸收,但是它的作用和存在还不太确定。

爆炸性核合成发生得太快,放射性衰变无法减少中子的数量,因此通过硅准平衡反应合成了许多质子和中子数量相等的同位素。[14] 在这个过程中,氧和硅的燃烧熔化了原子核,这些原子核本身具有相同数量的质子和中子,从而产生由多达15个(代表$^{60}Ni$)氦原子核组成的核素。这种多$\alpha$粒子核素在$^{40}Ca$(由10个氦核组成)时是完全稳定的,但是质子和中子数量相等且偶数的较重的核结合紧密但不稳定。准平衡产生放射性等压线$^{44}Ti,^{48}Cr,^{52}Fe$和$^{56}Ni$,这些等压线(除$^{44}Ti$外)大量产生,但在爆炸后衰变,并使相应元素的最稳定同位素保持相同的原子量。以这种方式生产的元素中最丰富和现存的同位素是$^{48}Ti,^{52}Cr$和$^{56}Fe$。这些衰变伴随着伽马射线的发射(来自原子核的辐射),伽马射线的光谱线可以用来识别衰变产生的同位素。这些放射线的探测是伽马射线天文学的早期重要成果。[17]

超新星爆发核合成的最令人信服的证据发生在1987年,当时从超新星1987A中发现了伽马射线线。鉴定$^{56}Co$和$^{57}Co$原子核的伽马射线线证明,它们的放射性半衰期将它们的年龄限制在一年左右,这证明它们的放射性钴母体创造了它们。这项核天文学观测是在1969年被预言为证实元素爆炸性核合成的一种方式[17]方式,这一预测在美国宇航局康普顿伽马射线天文台的规划中发挥了重要作用。

爆炸核合成的其他证据是在超新星膨胀和冷却时在超新星内部凝聚的星尘颗粒中找到的。星尘颗粒是宇宙尘埃的一个组成部分。特别是,在超新星膨胀过程中,当它们凝聚时,在超新星星尘颗粒中检测到放射性$^{44}Ti$非常丰富。[13] 这证实了1975年对超新星星尘(SUNOCONs)的预测,超新星星尘成为前太阳系颗粒万神殿的一部分。这些颗粒中其他不同寻常的同位素比率揭示了爆炸性核合成的许多具体方面。