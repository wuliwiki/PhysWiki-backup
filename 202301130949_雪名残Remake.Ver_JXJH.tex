% 解析几何
\begin{aligned}
解析几何初步
\end{aligned}
本章我将会在高中教材的基础上,探究在平面直角坐标系上,直线、圆、和圆锥曲线的一些性质,并解决部分中、高考问题
\begin{aligned}
1.一次函数与直线方程
\end{aligned}
\begin{example}{}
\begin{itemize}
\item 在平面直角坐标系中画出y=3x+1的图像
\item 用方程表示平面直角坐标系中过点(1,1),(3,2)的直线
\end{itemize}
\end{example}
初中课本曾用描点法刻画了一次函数的图像,在这里我将用另外的方法刻画一次函数图像(此时我们不知道一次函数图像是一条直线)

解:(1)令$y=0$得$x=-\frac{1}{3}$,因此图像过点A:($-\frac{1}{3}$,0)
设一点P:($x_0$,$y_0$)是图像上异于($-\frac{1}{3}$,0)的点,所以$y_0=3x_0+1\Rightarrow \frac{y_0}{x_0-\frac{1}{3}}=3$,当$y_0>0$时,$tan\angle PAO=3$,当$y_0<0$时,$tan\angle PAO=-3$由此可知,无论$x_0$取任何实数值,P都在一条固定的直线上,于是作过点($-\frac{1}{3}$,0)、(0,1)的直线即为y=3x+1的图像

(2)设点Q:($x_0$,$y_0$)在该直线上.点M:(1,1)、N:(3,2).由向量三点共线公式:$\overrightarrow{OP}=\lambda \overrightarrow{OA}+(1-\lambda)\over$