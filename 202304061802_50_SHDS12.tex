% 2012年沈阳航空航天大学818/数据结构专业综合考研真题
% 2012年沈阳航空航天大学818/数据结构专业综合考研真题


\subsection{一、选择题(每题2分,共30分)}
1从逻辑上可以把数据结构分为( )两大类。 \\
A.人动态结构、静态结构 \\
B.顺序结构、链式结构 \\
C.线性结构、非线性结构 \\
D.初等结构、构造型结构

2.在下面程序段中,对x的赋值语句的频度为() \\
\begin{lstlisting}[language=cpp]
for(k=1;k<n;k++)
{
    for(j=k; j<=n; j++)
    {
        x=x+1;
    }
}
\end{lstlisting}
A.$O(2n)$ $\qquad$ B.$O(n)$ $\qquad$ C.$O(n)$ $\qquad$ D.$O(log_2n)$

3. 算法的时间复杂度与(    )有关。 \\
A.问题规模 $\qquad$  B.计算机硬件的运行速度 \\
C.源程序的长度 $\qquad$ D.编译后可执行程序的质量

4. 在下列关于线性表的叙述中,正确的是() \\
A. 上线性表的逻辑顺序和物理顺序总是一致的.
B. 线性表的顺序存储结构优于链式存储结构
C. 就线性表的查找效率而言,链式存储结构比顺序存储结构高。
D. 就线性表的插入效率而言,链式存储结构比顺序存储结构高。

5. 已如循环链表的最后一个结点由ρ指针指向,若要在该结点后插入s指针指
向的新结点,则应执行下列( ) 操作。
L p->nextns; s->next=NULL;
B. p->next"s; s->next-p .
C. s->next*p->next; p->next-s;
D. s->nextep->next; p=s;

6.一个核的入找序列是a,b. c, d. e,则栈的不可能的输出序列是69 )。
A edcbe
B. decba
C. dceab
D. abede

7.若用一个大小为6的数组来实现循环队列, 且当前rear和front的值分别为10和3, 当从队列中删除一个元素, 再加入两个元素后,rear和front的值分别为多少? ()
A.上1和S
B.2和4
C.4和2
D.5和」