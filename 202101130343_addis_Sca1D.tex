% 一维散射(量子)

\begin{issues}
\issueDraft
\end{issues}

\pentry{一维自由粒子(量子)\upref{FreeP1}}

在量子力学中, 散射是一个十分重要的过程. 例如为了探索例如原子的结构, 一种重要的手段就是用其他粒子轰击原子, 并探测出射粒子在不同方向上的动量或能量分布等. 著名的卢瑟福散射就是用电子轰击原子的方法解开了原子结构之谜. 虽然卢瑟福最初使用经典力学来分析这类散射过程(当时量子力学还没有出现), 但原子尺度下量子力学比经典力学要精确得多.

在经典力学中, 散射\upref{Scater}过程考虑一个粒子从无穷远处入射, 经过一个势能 $V(\bvec r)$ 后发生偏折. 量子力学中, 这一过程可以用波包来描述: 一个波包 $\psi(x, t)$ 在初始时刻从远处入射, 我们想知道经过 $V(\bvec r)$ 散射后, $\psi(x, t)$ 会如何变化, 例如不同出射方向的概率流密度\upref{PrbJ}以及动量如何分布?

由于三维空间的量子散射所需的数学较为复杂, 我们先学习一维散射. 三维散射的许多性质都可以从一维情况类比.

我们已经学习了一维自由粒子\upref{FreeP1}如何随时间演化, 例如一维高斯波包\upref{GausWP}在自由演化过程中, 他的中心会像经典粒子一样以恒定速度移动, 但同时波包还会慢慢变宽. 在此基础上, 如果在波包前进方向上添加一个不为零的 $V(x)$ (例如一个势垒或势阱\autoref{QM0_fig3}~\upref{QM0}), 那么波包将会在各个方向上发生散射(一维运动只有左右两个方向). 我们希望能计算该过程中波函数如何变化.

原则上, 我们可以直接把初始波函数和势能代入薛定谔方程进行求解
\begin{equation}
\frac{1}{2m}\pdv[2]{t}\psi(x, t) + V(x) \psi(x, t) = \I \pdv{t} \psi(x, t)
\end{equation}
即使没有解析解, 也可以通过数值方法, 依次求出每个时间步长 $t_n$ ($n = 1, 2, \dots$)的波函数 $\psi(x, t_n)$.

但我们有更好的办法, 就是求出不同的动量本征态 $\psi_k$, 然后由于薛定谔方程是线性的(未完成).

\addTODO{然后, 各种散射势能引用该词条. 我们至少要先说明散射态(本征态)有什么用, 再去求解吧.}
