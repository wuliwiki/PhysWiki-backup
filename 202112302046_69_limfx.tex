% 函数的极限
% keys 函数极限

\pentry{序列的极限\upref{SeqLim}}

回顾:函数是一个非空集合 $A$ 到另一个集合 $B$ 的**对应法则**.

本词条中,函数 $f$ 是指从 $\mathbb R$ 的某个非空子集 $X$ 到 $\mathbb{R}$ 的映射  

\subsection{函数的极限}
\begin{figure}[ht]
\centering
\includegraphics[width=12cm]{./figures/limfx_1.png}
\caption{请添加图片描述} \label{limfx_fig1}
\end{figure}
\begin{definition}{邻域和去心邻域}
  定义\textbf{邻域}: $U(x_0,\delta)=\{x\in \mathbb{R}:|x-x_0|<\delta\}$.

  \textbf{去心邻域}: $U_0(x_0,\delta)= U(x_0,\delta) \backslash \{x_0\}=\{x\in \mathbb{R}:0<|x-x_0|<\delta\}$.
\end{definition}
\begin{definition}{极限}
 设函数 $f(x)$ 在 $U_0(x_0,\delta_0)(\delta_0>0)$ 内有定义.

  若存在实数 $A$ ,使得对任意 $\epsilon >0$,存在 $\delta>0$,使得当 $x\in U_0(x_0,\delta)$时,有 $|f(x)-A|<\epsilon$,则称\textbf{当 $x$ 趋于 $x_0$,函数 $f(x)$ 以 $A$ 为极限},记为 $\large \lim\limits_{x\rightarrow x_0}f(x)=A$ 或 $f(x)\rightarrow A\ (x\rightarrow x_0)$.
\end{definition}
PS:实际上有更宽泛的定义,只要 $x_0$ 是函数 $f(x)$ 定义域的\textbf{聚点},就可以定义在该点处的极限.

同序列极限的性质类似,函数极限也具有唯一性:
\begin{theorem}{}
  若函数 $f(x)$ 在 $x_0$ 处极限存在,证明在 $x_0$ 处极限唯一.
\end{theorem}
\begin{exercise}{}
\begin{enumerate}
\item  $f(x)=\left\{\begin{aligned} 0\ \ &(x<1)\\ 1\ \ &(x\ge 1) \end{aligned}\right.$ ,判断 $f(x)$ 在 $x_0=1$ 处极限是否存在.
\item $f(x)=\left\{\begin{aligned} 0\ \ &(x<1)\\ 1\ \ &(x= 1)\\2\ \ &(x>1) \end{aligned}\right.$,判断 $f(x)$ 在 $x_0=1$ 处极限是否存在.
\item $f(x)=x\cdot \sin(1/x)$,判断 $f(x)$ 在 $x_0=0$ 处极限是否存在.
\item $f(x)=\sin(1/x)$,判断 $f(x)$ 在 $x_0=0$ 处极限是否存在.
\item $f(x)=e^x$,证明 $\large \lim\limits_{x\rightarrow a}=e^a,\forall a\in\mathbb{R}$.
\end{enumerate}
\end{exercise}

\begin{figure}[ht]
\centering
\includegraphics[width=12cm]{./figures/limfx_2.png}
\caption{请添加图片描述} \label{limfx_fig2}
\end{figure}

我们来看一个有趣的函数 $f(x)$,它的定义域为 $[0,1]$:
\begin{equation}
f(x)=\left\{
\begin{aligned} &1/q, &x=\frac{p}{q}\ (p,q\in \mathbb{N},\frac{p}{q}\mbox{为既约真分数})\\
&0,&\mbox{x=0或x=1或x}\notin \mathbb{Q}
\end{aligned}
\right.
\end{equation}
我们称它为\textbf{黎曼 (Riemann) 函数}.

虽然在定义域内有无穷多个点的函数值不为 $0$,但 $f(x)$ 的极限却处处为 $0$,我们之后还将看到,$f(x)$ 在无理点处处连续,但 $f(x)$ 处处不可导.

\begin{exercise}{}
\begin{enumerate}
\item 证明:若函数 $f(x)$ 在 $U(a,\delta_0)(\delta_0>0)$ 上有定义,且满足 $\large\lim\limits_{x\rightarrow a}f(x)=f(a)$,那么对任意极限为 $a$ 的序列 $\{x_n\}$ ,序列 $\{f(x_n)\}$ 的极限也为 $f(a)$.
     上述命题反过来也成立.
\item 对于任意给定的序列 $\{a_n\}(0<a_n<1)$,构造定义域为 $[0,1]$ 的函数 $f(x)$,满足 $\forall x\in \{a_n\},f(x)\neq 0;\ \forall x \in [0,1]\backslash \{a_n\},f(x)=0$,且 $f(x)$ 在定义域上极限处处为 $0$.
\end{enumerate}
\end{exercise}
\subsection{函数的左右极限}
\begin{figure}[ht]
\centering
\includegraphics[width=12cm]{./figures/limfx_3.png}
\caption{请添加图片描述} \label{limfx_fig3}
\end{figure}
  如果把去心邻域 $U_0(x_0,\delta_0)$ 分成两块\textbf{单侧邻域}——

  左空心邻域:$U_0^+(x_0,\delta)=U_0(x_0,\delta)\cap (x_0,+\infty) \{x\in \mathbb{R} :x_0< x<x_0+\delta\}$

  右空心邻域:$U_0^-(x_0,\delta)=U_0(x_0,\delta)\cap (-\infty,x_0) \{x\in \mathbb{R} :x_0-\delta< x<x_0\}$

  那么就可以定义函数的左右极限:

  设 $f(x)$ 在 $U^+_0(x_0,\delta_0)(\delta_0>0)$ 上有定义.

  如果存在实数 $A$,使得对任意 $\epsilon >0$,存在 $\delta>0$,当 $x\in U_0^+(x_0,\delta)$ 时,有 $|f(x)-A|<\epsilon$,则称 $f(x)$ 在点 $x_0$ 的\textbf{右极限存在},而称 $A$ 为 $f(x)$ 在点 $x_0$ 的\textbf{右极限},记为 $\lim\limits_{x\rightarrow x_0^+}f(x)=A$ 或 $f(x_0^+)=A$.

  类似地可以定义\textbf{左极限存在}和\textbf{左极限}.

\begin{exercise}{}
\begin{enumerate}
\item $f(x)=\left\{\begin{aligned} 0\ \ &(x<1)\\ 1\ \ &(x= 1)\\2\ \ &(x>1) \end{aligned}\right.$,判断 $f(x)$ 在 $x_0=1$ 处的左极限与右极限.
\item  $f(x)=[x]$(取整函数),判断 $f(x)$ 在 $x_0=1$ 处的左极限与右极限.
\item  设函数 $f(x)$ 在 $U_0(x_0,\delta_0)$ 上有定义,证明: $\lim\limits_{x\rightarrow x_0} f(x)=A$ 当且仅当 $\lim\limits_{x\rightarrow x_0^-}f(x)=\lim\limits_{x\rightarrow x_0^+}f(x)=A$.
\end{enumerate}
\end{exercise}