% 牛顿—莱布尼兹公式(复变函数)

\begin{issues}
\issueDraft
\end{issues}

复变函数的积分为
\begin{equation}
\int_{C} f(z) \mathrm{d} z=\int_{C} u \mathrm{d} x-v \mathrm{d} y+\mathrm{i} \int_{C} v \mathrm{d} x+u \mathrm{d} y
\end{equation}
可以看作两个矢量场 $\bvec f_R(\bvec r), \bvec f_C(\bvec r)$ 在曲线 $C$ 上的线积分
\begin{equation}\label{AnaInt_eq1}
\int_{C} f(z) \dd{z} = \int_C \bvec f_R(\bvec r) \vdot \dd{\bvec r} + \I \int_C \bvec f_C(\bvec r) \vdot \dd{\bvec r}
\end{equation}
其中
\begin{equation}
\bvec f_R(\bvec r) = u\uvec x - v\uvec y
\qquad
\bvec f_I(\bvec r) = v\uvec x + u\uvec y
\end{equation}
而柯西—黎曼条件\upref{CauRie}恰好规定了这两个矢量场的旋度为零, 所以如果 $f(z)$ 在考虑的区域上解析, \autoref{AnaInt_eq1} 中的线积分结果只和起点和终点有关, 与路径无关(所有的路径必须在解析的区域内). 于是可以得到类似于牛顿—莱布尼兹公式\upref{NLeib}, 令 $\bvec f_R, \bvec f_I$ 的势函数分别为 $F_R, F_I$, 即
\begin{equation}
\grad F_R = f_R
\qquad
\grad F_I = f_I
\end{equation}
再令
\begin{equation}
F(\bvec r) = F_R + \I F_I
\end{equation}
有
\begin{equation}\label{AnaInt_eq2}
\int_{z_1}^{z_2} f(z) \dd{z} = F(z_2) - F(z_1)
\end{equation}
对于任意路径成立. 容易证明 $F(z)$ 就是 $f(z)$ 的原函数, 即
\begin{equation}
F'(z) = f(z)
\end{equation}
