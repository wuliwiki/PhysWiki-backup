% 横波与纵波
% license Xiao
% type Tutor

\subsection{横波 Transverse Wave}
\begin{figure}[ht]
\centering
\includegraphics[width=8cm]{./figures/9d04d16f5d3dce38.pdf}
\caption{横波示意图,图中可见波峰与波谷。背景灰色圆圈是各微元的平衡位置。\href{https://wuli.wiki/apps/waves.html}{动图}} \label{fig_LAT_1}
\end{figure}
横波中,各微元的振动方向垂直于波的传播方向。相邻微元间受剪切力并发生剪切变形。由于空气等介质不能发生剪切变形,因此横波不能在空气中传播。

\subsection{纵波 Longitudinal Wave}
\begin{figure}[ht]
\centering
\includegraphics[width=8cm]{./figures/3b259a45184518a5.pdf}
\caption{横波示意图,图中可见疏部与密部。\href{https://wuli.wiki/apps/waves.html}{动图}} \label{fig_LAT_2}
\end{figure}
纵波中,各微元的振动方向平行于波的传播方向。相邻微元间受拉力、压力力并发生拉伸、压缩变形。在空气中,“拉伸变形”、“压缩变形”可以理解为气体分子的疏散与聚集。

\subsection{附录:绘制相应图形的octave/matlab代码}
\begin{lstlisting}[language=matlab]
%绘制横波
T=2;
v=4;
t=0;
A=0.3;

w=2*pi/T;
k=w/v;

for t=0:0.1:5
  [x y]=meshgrid(-5:5);
  xo=x;
  yo=y;
  [a b] = size(x);

  wave = A*cos(k*x-w*t);
  y=y+wave;

  clf
  hold on
  scatter(xo,yo,'MarkerEdgeColor',[0.9 0.9 0.9]);
  scatter(x,y);
  axis equal
  axis([-6 6 -6 6])
  axis off
  hold off
  drawnow
  pause(0.1)
end

\end{lstlisting}

\begin{lstlisting}[language=matlab]
%绘制纵波
T=2;
v=5;
t=0;
A=0.5;

w=2*pi/T;
k=w/v;

for t=0:0.1:5
  [x y]=meshgrid(-5:5);
  xo=x;
  yo=y;
  [a b] = size(x);

  wave = A*cos(k*x-w*t);
  x=x+wave;

  clf
  hold on
  scatter(xo,yo,'MarkerEdgeColor',[0.9 0.9 0.9]);
  scatter(x,y);
  axis equal
  axis([-6 6 -6 6])
  axis off
  hold off
  drawnow
  pause(0.1)
end
\end{lstlisting}
