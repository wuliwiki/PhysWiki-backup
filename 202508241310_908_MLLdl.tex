% 莫雷拉定理(综述)
% license CCBYSA3
% type Wiki

本文根据 CC-BY-SA 协议转载翻译自维基百科\href{https://en.wikipedia.org/wiki/Morera\%27s_theorem}{相关文章}。

在数学的一个分支——**复分析**中,**莫雷拉定理**(Morera's theorem,以贾钦托·莫雷拉 *Giacinto Morera* 命名)提供了一个判断函数是否为全纯函数的**判据**。

---

**莫雷拉定理内容**

设 $f$ 是定义在复平面开集 $D$ 上的一个连续的复值函数,若它满足:

$$
\oint_{\gamma} f(z)\,dz = 0
$$

对 $D$ 中的每一条**分段 $C^1$** 的闭曲线 $\gamma$ 都成立,那么 $f$ 在 $D$ 上必然是**全纯函数**。

---

**推论与等价条件**

* 该定理的假设等价于 $f$ **在区域 $D$ 上存在一个原函数(反导函数)**。
* 该定理的逆命题一般情况下并不成立:
  一个全纯函数并不一定在其定义域上存在原函数,除非附加一些条件。
* 在区域 $D$ **单连通**的情况下,逆命题成立。这正是**柯西积分定理**的内容:
  全纯函数沿闭合曲线的积分为零。
