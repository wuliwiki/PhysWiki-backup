% 圆锥曲线的统一定义(高中)
% keys 准线|第二定义|焦点|圆锥曲线|焦点-准线定义
% license Usr
% type Tutor

\pentry{圆锥曲线与圆锥\nref{nod_ConSec}}{nod_55cd}

\subsection{圆锥曲线的焦点-准线定义}

利用准线与焦点得到的。

\textbf{圆锥曲线的焦点-准线定义(Focus-Directrix Definition of Conic Sections)}。

\begin{definition}{圆锥曲线的焦点-准线定义}
平面上到一个定点与到一条定直线的距离之比为常数$e$的点构成的图像称为\textbf{圆锥曲线}。
其中,定点称为圆锥曲线的\textbf{焦点},定直线称为圆锥曲线的\textbf{准线},二者互相对应。$e$称作圆锥曲线的\textbf{离心率}。特别地:
\begin{itemize}
\item 当 $e =0$ 时,轨迹称为\textbf{圆}。
\item 当 $0 < e < 1$ 时,轨迹称为\textbf{椭圆}。
\item 当 $e =1$ 时,轨迹称为\textbf{抛物线}。
\item 当 $e > 1$ 时,轨迹称为\textbf{双曲线}。
\end{itemize}
\end{definition}

显然,定点到定直线的垂线为圆锥曲线的对称轴。

\subsection{定义等价性}