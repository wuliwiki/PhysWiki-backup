% 2012 年计算机学科专业基础综合全国联考卷
% keys 计算机 考研 全国卷

\subsection{一、单项选择题} \\
第1~40小题,每小题2分,共80分.下列每题给出的四个选项中,只有一个选项最符合试题要求.

1.求整数n(n≥0)阶乘的算法如下,其时间复杂度是 . \\
\begin{lstlisting}[language=cpp]
int fact(int n){
  if (n<=1) return 1;
  return n*fact(n-1);
}
\end{lstlisting}
A. $O(log2n)$ $\quad$ B. $O(n)$ $\quad$ C. $O(nlog2n)$ $\quad$ D. $O(n2)$

2.已知操作符包括„+‟、„-‟、„*‟、„/‟、„(‟和„)‟.将中缀表达式a+b-a*((c+d)/e-f)+g转换为等价的后缀表达式ab+acd+e/f-*-g+时,用栈来存放暂时还不能确定运算次序的操作符,若栈初始时为空,则转换过程中同时保存在栈中的操作符的最大个数是 . \\
A.5 $\quad$ B.7 $\quad$ C.8 $\quad$ D.11

3.若一棵二叉树的前序遍历序列为a, e, b, d, c,后序遍历序列为b, c, d, e, a,则根结点的孩子结点 . \\
A. 只有e $\quad$ B. 有e、b $\quad$ C. 有e、c $\quad$ D. 无法确定

4.若平衡二叉树的高度为6,且所有非叶结点的平衡因子均为1,则该平衡二叉树的结点总数为 . \\
A. 10 $\quad$ B. 20 $\quad$ C. 32 $\quad$ D. 33

5.对有n个结点、e条边且使用邻接表存储的有向图进行广度优先遍历,其算法时间复杂度是 .\\
A.$O(n)$ $\quad$ B.$O(e)$ $\quad$ C.$O(n+e)$ $\quad$ D.$O(n*e)$

6.若用邻接矩阵存储有向图,矩阵中主对角线以下的元素均为零,则关于该图拓扑序列的结论是 .\\
A.存在,且唯一 $\quad$ B.存在,且不唯一\\
C.存在,可能不唯一 $\quad$ D.无法确定是否存在

7.对如下有向带权图,若采用迪杰斯特拉(Dijkstra)算法求从源点a到其他各顶点的最短路径,则得到的第一条最短路径的目标顶点是b,第二条最短路径的目标顶点是c,后续得到的其余各最短路径的目标顶点依次是 .\\
\begin{figure}[ht]
\centering
\includegraphics[width=5cm]{./figures/CSN12_1.png}
\caption{请添加图片描述} \label{CSN12_fig1}
\end{figure}



