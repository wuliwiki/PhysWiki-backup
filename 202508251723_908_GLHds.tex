% 格林恒等式(综述)
% license CCBYSA3
% type Wiki

本文根据 CC-BY-SA 协议转载翻译自维基百科\href{https://en.wikipedia.org/wiki/Green\%27s_identities}{相关文章}。

在数学中,格林恒等式是一组包含三条公式的向量分析恒等式,用于关联区域内部(体积部分)与其边界在微分算子作用下的关系。它们以发现格林定理的数学家乔治·格林的名字命名。
\subsection{格林第一恒等式}
这个恒等式可以通过将散度定理应用于向量场 $\mathbf{F} = \psi \nabla \varphi$ 并利用乘积法则的推广形式
$$
\nabla \cdot (\psi \mathbf{X}) = \nabla \psi \cdot \mathbf{X} + \psi \nabla \cdot \mathbf{X}~
$$
推导出来。
设 $\varphi$ 和 $\psi$ 是定义在某个区域 $U \subset \mathbb{R}^d$ 上的标量函数,其中 $\varphi$ 是二阶连续可微函数,$\psi$ 是一阶连续可微函数。令 $\mathbf{X} = \nabla \varphi$,并将 $\nabla \cdot (\psi \nabla \varphi)$ 在 $U$ 上积分,则有[1]:
$$
\int_U \left(\psi \,\Delta \varphi + \nabla \psi \cdot \nabla \varphi \right) \, dV
= 
\oint_{\partial U} \psi \, (\nabla \varphi \cdot \mathbf{n}) \, dS
=
\oint_{\partial U} \psi \, \nabla \varphi \cdot d\mathbf{S},~
$$
该定理是**散度定理**的一个特殊情形,本质上是**分部积分**在高维情况下的对应形式,其中 $\psi$ 和 $\varphi$ 的梯度分别代替了 $u$ 和 $v$。

需要注意的是,上述 **格林第一恒等式** 其实是一个更一般恒等式的特例,这个更一般的恒等式是通过在散度定理中代入 $\mathbf{F} = \psi \mathbf{\Gamma}$ 得到的:
$$
\int_U \left(\psi \, \nabla \cdot \mathbf{\Gamma} + \mathbf{\Gamma} \cdot \nabla \psi \right)\, dV
=
\oint_{\partial U} \psi \, (\mathbf{\Gamma} \cdot \mathbf{n}) \, dS
=
\oint_{\partial U} \psi \, \mathbf{\Gamma} \cdot d\mathbf{S}~
$$
\subsection{格林第二恒等式}
