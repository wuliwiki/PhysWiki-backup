% 数论函数 theta 与 psi 函数的阶
% license Usr
% type Tutor

\pentry{数论函数\nref{nod_NumFun},渐进估计与阶\nref{nod_asymeq}}{nod_5d92}
本文默认若对 $n$ 求和或求积,则范围均满足 $n \ge 1$。
\subsection{阶}
\begin{theorem}{}
函数 $\vartheta(x)$ 与 $\psi(x)$ 的阶是 $x$:

对于足够大的 $x$($x \ge 2$),
\begin{equation}
Ax < \vartheta(x) < Bx, ~ Cx < \psi(x) < Dx ~ ~.
\end{equation}

\end{theorem}

\subsection{证明}
我们首先引入一个引理并予以证明。
\begin{lemma}{}
\begin{equation}
\psi(x) = \vartheta(x) + \mathcal O\left( x^{1/2} (\ln x)^{2} \right) ~.
\end{equation}

\end{lemma}
\textbf{证明}:考虑 $\psi(x)$ 的定义,由于 $p^2 \le x$, $p^3 \le x$, $\dots$ 等价于 $p \le x^{1/2}$, $p \le x^{1/3}$, $\dots$, 故
\begin{equation}\label{eq_tpont_2}
\psi(x) = \vartheta(x) + \vartheta(x^{1/2}) + \vartheta(x^{1/3}) + \cdots = \sum_{m} \vartheta(x^{1/m}) ~.
\end{equation}
当 $x^{1/m} < 2$,也就是
\begin{equation}
m > \frac{\ln x}{\ln 2} ~~
\end{equation}
时,求和结束。由 $\vartheta(x)$ 的定义显然有 $x$ 足够大时 $\vartheta(x) < x \ln x$,故 $m \ge 2$ 时有
\begin{equation}
\vartheta(x^{1/m}) < x^{1/m} \ln x \le x^{1/2} \ln x~,
\end{equation}
且
\begin{equation}\label{eq_tpont_1}
\sum_{m \ge 2} \vartheta(x^{1/m}) = \mathcal O\left( x^{1/2} (\ln x)^{2} \right) ~.
\end{equation}
这等式成立是因为这级数仅有 $\mathcal O(\ln x)$ 项。

将\autoref{eq_tpont_1} 与 $\vartheta(x)$ 相加,就得到了\autoref{eq_tpont_2} ,就完成了证明。


根据这个引理,我们

