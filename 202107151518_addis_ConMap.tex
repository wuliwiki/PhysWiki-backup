% 巴拿赫不动点定理

\pentry{完备度量空间\upref{cauchy}}

\textbf{巴拿赫不动点定理 (Banach fixed point theorem)} 又称作\textbf{压缩映像原理 (contraction mapping principle)}. 它是完备度量空间理论中的基本定理, 在分析数学的诸多分支中均有应用.

\subsection{两个著名例子}
\begin{example}{落在地面上的地图}
这是数学科普中常见的命题: \textbf{将一座公园的地图铺开在公园地面上, 则地面上恰有唯一一点与地图上对应的点重合.} 

借助一点线性代数知识, 这个命题是不难验证的. 设公园可以用有界的面闭区域$\bar\Omega$表示. 设地图的压缩比是$\lambda$ (它当然介于0和1之间). 现在固定一个平面直角坐标系, 把地图铺在区域$\bar\Omega$内, 则从$\bar\Omega$内的点 (公园中的地点) 到地图上对应点的变换由下面的公式给出:
$$
x\to \lambda Rx+b,
$$
这里$R$是一个旋转变换, $b$是平移向量. 于是, 要找的重合点必然满足方程
$$
x=\lambda Rx+b.
$$
由于$\|\lambda R\|=\lambda<1$, 这方程就有唯一的解
$$
x=(1-\lambda R)^{-1}b=\sum_{n=0}^{\infty}\lambda^nR^nb.
$$
它就是要求的重合点.
\end{example}

\begin{example}{函数的迭代}
这是一个常见的数学实验: \textbf{在计算器中任意输入一个数, 而后不停地计算它的余弦值 (弧度制), 会得到什么结果? }

\begin{figure}[ht]
\centering
\includegraphics[width=9cm]{./figures/ConMap_1.png}
\caption{余弦函数的迭代} \label{ConMap_fig1}
\end{figure}

上图给出了一个结果; 迭代的结果越来越逼近对角线$y=x$与余弦曲线$y=\cos x$的唯一交点. 验算若干数值, 不难作出如下猜测: \textbf{不论实数的迭代序列
$$
x_{n+1}=\cos x_n
$$
开始于哪一个数, 它最后都会收敛到方程$x=\cos x$的唯一实数解$x=0.739...$.}

这个结论也可以严格证明. 不论起点$x_1$是何数, 都有$|x_2|=|\cos x_1|\leq 1$, 从而$\cos 1\leq x_3\leq 1$. 这样一来, 从$x_3$开始, 所有点都落在区间$[\cos 1,1]$内. 于是可以作出如下估计:
$$
\begin{aligned}
|x_{n+2}-x_{n+1}|
=\left|\int_{x_{n}}^{x_{n+1}}\sin tdt\right|
\leq (\sin 1)|x_{n+1}-x_n|.
\end{aligned}
$$
由于$|\sin1|<1$, 这表示序列$\{x_n\}$是柯西序列, 从而收敛到某个点$x_*$. 对等式$x_{n+1}=\cos x_n$取极限即看出这个极限点正是方程$x=\cos x$的解.
\end{example}


\subsection{巴拿赫不动点定理}
\subsubsection{定理的表述}
从上面所举的两个例子, 可以抽象出如下的定理:
\begin{theorem}{巴拿赫不动点定理}
设$(X,d)$是完备度量空间\upref{cauchy}, $T:X\to X$是连续映射. 如果存在一个数$q\in(0,1)$使得$d(Tx,Ty)\leq qd(x,y)$对于任意的$x,y\in X$都成立, 那么映射$T$有唯一的不动点, 即满足$x=Tx$的点. 而且, 对于任意$x_0\in X$, 点列$\{T^nx_0\}_{n\in\mathbb{N}}$都收敛到这个不动点.
\end{theorem}
证明是直接的计算: 根据度量空间中的三角不等式, 显然有
$$
\begin{aligned}
d(T^nx_0,T^{n+k}x_0)
&\leq \sum_{j=1}^k d(T^{n+j-1}x_0,T^{n+j}x_0)\\
&\leq \sum_{j=1}^k q^{n+j-1}d(x_0,Tx_0)\\
&\leq \frac{q^n}{1-q}d(x_0,Tx_0).
\end{aligned}
$$
于是点列$\{T^nx_0\}_{n\in\mathbb{N}}$是柯西序列, 在完备度量空间$X$之中自然收敛到某个$x_*\in X$. 在公式$T^{n+1}x_0=T(T^nx_0)$中令$n\to\infty$就立刻看出$x_*=Tx_*$. 不动点的唯一性则由$d(Tx,Ty)\leq qd(x,y)$立刻得到. \textbf{证毕.}

在上面的不等式中, 如果令$k\to\infty$, 就得到估计
$$
d(T^nx_0,x_*)\leq\frac{q^n}{1-q}d(x_0,Tx_0).
$$
这表示点列$\{T^nx_0\}_{n\in\mathbb{N}}$收敛到不动点$x_*$的速度很快, 是指数式的.

\subsubsection{辨析}
从许多个意义上来说, 巴拿赫不动点定理都是最优的, 因为取消任何一条假设都能够让定理不成立 (即存在反例).

\begin{example}{没有完备性}
这个例子非常简单. 考虑开区间$(0,1)$到自己的映射$f(x)=x/2$. 它的不动点$x=0$跑出了$(0,1)$的范围. 之所以会有这样的反例, 是因为开区间$(0,1)$在通常的距离函数之下并不是完备的度量空间.
\end{example}

\begin{example}{没有一致的压缩比}
巴拿赫不动点定理的表述要求一个无关于点$x,y$的数$q<1$, 这个数有时被称为压缩比. 如果没有一致的压缩比, 那么就能举出反例. 例如函数
$$
f(x)=\frac{3}{4}x-\frac{1}{4}\sqrt{x^2+1},\,x\in\mathbb{R}.
$$
如图所示, 它的图形是双曲线\upref{Hypb3}的一支, 以对角线$y=x$为渐近线. 方程$x=f(x)$当然无解. 另一方面, 通过微分中值定理, 却也容易证明不等式
$$
|f(x_1)-f(x_2)|<|x_1-x_2|,\,x_1,x_2\in\mathbb{R}.
$$
然而, 当$x_1,x_2\to-\infty$时, 不难看出$\frac{|f(x_1)-f(x_2)|}{|x_1-x_2|}$越来越接近1. 因此映射$f$没有一致的压缩比. 
\begin{figure}[ht]
\centering
\includegraphics[width=9cm]{./figures/ConMap_2.png}
\caption{函数$f(x)=\frac{3}{4}x-\frac{1}{4}\sqrt{x^2+1}$的图形} \label{ConMap_fig2}
\end{figure}

\end{example}

\subsection{应用举例}
巴拿赫不动点定理在数学中有诸多应用. 它可以用来证明方程存在唯一解, 而且因为它的证明是构造性的, 它还能够给出近似计算解的算法.

\begin{example}{平方根的计算}
对于给定的正数$a$,它的算术平方根$\sqrt{a}$一定满足方程$x^2=a$.不妨设$a>1$,将其改写为下列形式
\begin{equation}
\frac{1}{2}\qty(x+\frac{a}{x})=x
\end{equation}
将等式左边的部分视为$Tx$,易知$T$是完备度量空间\footnote{完备度量空间的紧子集必然是完备度量空间.} $[\frac{1}{2}a,a]$中的连续映射.

又对任意$x_1,x_2\in [\frac{1}{2}a,a]$
\begin{equation}
\abs{Tx_1 -  Tx_2}=\frac{1}{2}\abs{x_1 - x_2}\abs{1-\frac{a}{x_1x_2}}\leq\frac{1}{2}\abs{x_1 - x_2}
\end{equation}

根据巴拿赫不动点定理,我们可知算术平方根$\sqrt{a}$必然存在且唯一,且对任意$x\in[\frac{1}{2}a,a]$,点列$\{T^nx_0\}_{n\in\mathbb{N}}$均收敛到$\sqrt{a}$.

所以我们可以这么计算正数$a(a>1)$的算术平方根:先取$x_1=\frac{1}{2}a$,然后按下式迭代计算:
\begin{equation}
x_{n+1}=\frac{1}{2}\qty(x_n+\frac{a}{x_n})
\end{equation}

计算精度估计($n$为迭代计算次数)
\begin{equation}
\epsilon\leq\qty(\frac{1}{2})^{n+1}\abs{a-4}
\end{equation}
\end{example}
