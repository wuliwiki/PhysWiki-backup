% 北京大学 2009 年 考研 固体物理
% license Usr
% type Note

\textbf{声明}:“该内容来源于网络公开资料,不保证真实性,如有侵权请联系管理员”

\subsection{填空题(30分)}
\begin{enumerate}
\item 1,选图型:面心立方和体心立方分别对应的倒格子第一布里渊区

2,选图,区别不同的能带图:直接带隙,间接带隙,金属能带,半导体能带,绝缘体能带。

3,金属自由电子理论:朗道能级那一块有题,比如二维自由电子简并度等。

4,区别顺磁逆磁。

5,晶体结构基础知识
\end{enumerate}

\subsection{简答}
\begin{enumerate}
\item 自由电子近似和紧束缚近似方面的理解,考得很活了
\item 计算并比较xx 电子磁化率和包利自由电子磁化率(它们的比值应该是三倍),说明原因。
\item 碳 60 的键能
\end{enumerate}
\subsection{计算(40分)}
\begin{enumerate}
\item 计算$A,B$不同原子结合成分子后,能量的减少,
\item 考虑一维双原子链情况用紧束缚近似算其能带方程
\item 给出了能量方程(好像是二维情况),求电子空穴有效质量,费米半径,
费米能等(此题为送分题
\end{enumerate}
