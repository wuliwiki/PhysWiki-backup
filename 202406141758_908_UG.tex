% 万有引力
% license CCBYSA3
% type Wiki

(本文根据 CC-BY-SA 协议转载自原搜狗科学百科对英文维基百科的翻译)


牛顿的万有引力定律指出,每个粒子都会以一种与质量的乘积成正比且与它们的中心之间的距离的平方成反比的力吸引宇宙中的所有其他粒子。这是一个由艾萨克·牛顿从所谓的归纳推理经验观察中导出的一般物理定律。[1]它是经典力学的一部分,并详细地阐述在牛顿的自然哲学的数学原理 (“the Principia”) 著作之中,首次发表于1687年7月5日。当牛顿在1686年4月向皇家学会提交未出版文本的第一卷时,罗伯特·胡克声称牛顿从他那里获得了平方反比定律。

在今天的语言中,牛顿定律规定每个点的质量都受到沿与两个点相交的线的作用力吸引的其他点的质量。力与两个质量的乘积成正比,与它们之间距离的平方成反比。[2]

因此,万有引力方程的表达式为:

$F = G \frac{m_1 m_2}{r^2}$

其中 F 是作用在两个物体之间的重力,$m_1$ 和 $m_2$ 是物体的质量,$r$ 是它们的质心之间的距离,$G$ 是重力常数。

1798年,英国科学家亨利·卡文迪许在卡文迪什实验中进行了对牛顿质量间引力理论的第一次测试。[3] 它发生在牛顿的《原理》出版111年后,距他去世大约71年。

牛顿的引力定律类似于电力的库仑定律,用于计算两个带电体之间产生的电力的大小。两者都是平方反比定律,其中力与物体之间距离的平方成反比。库仑定律用两个电荷的乘积代替质量的乘积,用静电常数代替重力常数。

虽然牛顿定律后来被阿尔伯特·爱因斯坦的广义相对论所取代,但在大多数应用中,它仍然被用作为重力效应的一个极好的近似研究。只有当需要极端的准确性时,或者当处理非常强的引力场时才需要考虑相对性的影响因素,例如在极其巨大和致密的物体附近,或者在非常近的距离(例如水星绕太阳的轨道)。

\subsection{历史}

\subsubsection{1.1 早期历史}

最近,马尔迪和里奇奥利在1640年至1650年间证实了自由落体中物体的距离与所用时间之间的平方关系。他们还通过记录钟摆的振荡来计算重力常数。[4]

关于平方反比定律早期历史的现代评估是,“到17世纪70年代末”,关于“重力和距离平方成反比”的假设相当普遍,许多不同的人出于不同的原因提出了这一假设。[5]同一位作者认为罗伯特·胡克做出了重大而开创性的贡献,但是将胡克在反平方点上的优先权要求视为无关紧要,因为除了牛顿和胡克之外,还有其他几个人提出了这一点。相反,他指出“复合天体运动”的想法以及牛顿思想从“离心力”到“向心力”的转变是胡克的重要贡献。

牛顿在其《原理》中赞扬了两个人:布里亚杜斯(Bullialdus,他提出来地球有向太阳的力这一猜想)和博雷利(Borelli,他提出了所有行星都被吸引向太阳)。[6][7]对牛顿主要的影响可能来自于博雷利,他的书牛顿有一副本。[8]

\subsubsection{1.2 剽窃争议}

1686年,当第一本《牛顿原理》被提交给皇家学会时,罗伯特·胡克(Robert Hooke)指责牛顿剽窃,声称他从自己身上拿走了“重力下降规律,即与中心距离的平方相互作用”的“概念”。同时(根据埃德蒙多·哈雷的当代报告),胡克同意“由此产生的曲线的演示”完全是牛顿的。[9]

这样,就产生了牛顿欠胡克的问题,如果有的话。从那时起,这是一个广泛讨论的主题,下面概述的一些观点继续引起争议。

\subsubsection{1.3 胡克的作品和主张}

罗伯特·胡克在16世纪60年代发表了他关于“世界体系”的观点,当时他在1666年3月21日向英国皇家学会宣读了一篇论文“关于通过附带的吸引原理将直接运动转变为曲线”,并在1674年以更为详尽拓展的形式再次发表了这些思想,作为对“从观测中证明地球运动的尝试”的补充。[10]胡克在1674年宣布,他计划“解释一个在许多细节上不同于任何已知的世界系统”,其观点基于三个“假设”:即“任何天体,都有朝向自己中心的吸引力或引力”,以及“它们也吸引在其活动范围内的所有其他天体”;[11]“所有被置于直接和简单运动的物体,将继续沿着直线前进,直到它们被其他一些有效的力偏转和弯曲……”;而且“这些吸引人的力量远比身体靠近自己的中心要强大得多”。因此胡克清楚地假设了太阳与行星之间的相互吸引,其方式是随着与吸引体的距离增加以及线性惯性原理而增加。

然而,胡克直到1674年的声明没仍然有提到平方反比定律适用于或可能适用于这些吸引力。因此,胡克的引力理论也不是普遍的,尽管它比以前的假设更接近普遍性。[12]他也没有提供相关的证据或数学证明。关于后两个方面,胡克本人在1674年说过:“现在我还没有通过实验验证这几个(吸引)程度”。关于他的整个提议:“我现在仅暗示这一点”,“我手头上有许多我要首先完成的其他事情,因此不能很好地参加”(即“起诉本调查”)。[10]后来,在1679年1月6日|80写给牛顿的信中,[13] 胡克向牛顿传达了他的“假设”...引力与距离中心的倒数成正比,因此速度与引力成反比,因此,开普勒假设对距离的相互作用。”[14] (关于速度的推断是不正确的。) [15]

胡克在1679-1680年间与牛顿的通信不仅提到了引力随距离增加而下降的平方反比假设,而且在胡克1679年11月24日给牛顿的公开信中,提出了一种“通过切线将行星的直线运动和朝向中心体的吸引运动合成”的方法。[16]

\subsubsection{1.4 牛顿的工作和主张}

牛顿在1686年5月面对胡克对平方反比定律的主张,否认胡克被认为是这个想法的作者。其中一个原因是,牛顿回忆起在胡克1679年的信之前,他曾与克里斯托弗·雷恩爵士讨论过这个想法。[17]牛顿还指出并承认其他人以前的研究,[18]包括布礼奥多斯,[6](提出但未证明,来自太阳的吸引力与距离成反比),和波雷利[7](建议但也没有证明,在对太阳的重力吸引下,平衡会出现离心趋势,从而使行星运动成椭圆形)。D T .怀特塞德描述了博雷利的书对牛顿思想的贡献,他去世时,这本书的副本就在牛顿去世时的图书馆里。[8]

牛顿进一步为他的工作辩护说,如果他第一次从胡克那里听说平方反比定律,鉴于他对其准确性的证明,他仍然有一些权利。胡克没有支持这一假设的证据,只能猜测平方反比定律在离中心很远的地方近似有效。牛顿认为,《原理》还处于出版前阶段,有太多先验的理由怀疑平方反比定律的准确性(特别是在吸引区域附近),以至于“没有我(牛顿的)的证明, 胡克先生还只是一个陌生人,一个明智的哲学家不能相信它是准确的。”[19]

这句话主要指的是牛顿发现这个理论有数学证明支持,如果平方反比定律适用于微小的粒子,那么即使是一个大的球对称质量也会吸引其表面以外的质量,即使是靠近它,就好像它自己的质量都集中在它的中心一样。因此,牛顿给出了一个论据,否则就没有理由将平方反比定律应用于大的球形行星质量,就好像它们是微小的粒子一样。[20]此外,牛顿在第1本书的命题43-45[21]和第3本书的相关章节中制定了对平方反比定律准确性的敏感测试。他在书中指出,只有当力的定律被计算为距离的平方反比时,行星轨道椭圆的方向才会保持不变,因为行星间摄动造成的影响十分微小。

关于早期历史仍然存在的证据,牛顿在16世纪60年代写的手稿表明,到1669年,牛顿本人已经得出了证明,在行星运动的圆形情况下,“努力后退”(后来被称为离心力)与距中心的距离呈平方反比关系。[22]牛顿在1679-1680年与胡克通信后,采用了内向力或向心力的用词。根据牛顿学者布鲁斯·布莱肯里奇(J. Bruce Brackenridge)的说法,尽管在离心力和向心力之间的语言变化和观点差异方面做了很多工作,但实际计算和证明无论如何都保持不变。它们还涉及切线位移和径向位移的组合,这是牛顿在16世纪60年代提出的。胡克在这里给牛顿的观点虽然重要,但只是一个视角,并没有改变分析本质。[23]这一背景表明牛顿有理由否认从胡克定律推导出平方反比定律。

\subsubsection{1.5 牛顿的致谢}

另一方面,牛顿确实在《 原理》的所有版本中都承认并致谢,胡克(但不仅限于胡克)已经分别研究了太阳系中的平方反比定律。牛顿在注释中承认雷恩、胡克和哈雷在这方面的研究与第1卷中的命题4有关。[24]牛顿还向哈雷致谢,他在1679-1680年与胡克的通信激发了他对天文问题潜在的兴趣,但根据牛顿的说法,这并不意味着胡克告诉了牛顿任何新的或原创的东西:“然而,我不是因为他对那项事业有所贡献而感激他,只是因为他把我从其他研究中转移到思考这些事情上,而且因为他在写作中的教条主义,就好像他发现了椭圆运动的奥妙,这使我倾向于尝试……”[18]

\subsubsection{1.6 现代优先权争议}

自从牛顿和胡克时代以来,学术讨论也触及了胡克在1679年提到的“复合运动”是否为牛顿提供了新的和有价值的东西的问题,尽管当时胡克并没有真正提出这一主张。如上所述,牛顿在16世纪60年代的手稿确实显示了他实际上将切向运动与径向力或力的作用结合起来,例如在他推导圆形情况的平方反比关系时。它们还表明牛顿清楚地表达了线性惯性的概念——为此,他感谢笛卡尔在1644年发表的工作(胡克可能就是这样)。[25]牛顿似乎没有从胡克那里学到这些东西。

然而,许多作者对牛顿从胡克的思想中学到了什么有更多的看法,一些方面仍然有争议。[5]鉴于胡克的大多数私人文件已经被销毁或消失的事实,真相无法被确定。

牛顿在平方反比定律中的作用并不像它有时被描述的那样。他没有声称把它想成一个简单的想法。牛顿所做的是展示平方反比引力定律与太阳系中物体运动的可观测特征有许多必要的数学联系;并且它们之间的联系使得观测证据和数学证明结合起来,有理由相信平方反比定律不仅近似为真,而且完全为真 (达到牛顿时代和大约两个世纪后所能达到的精度,虽然还存在一些尚不能确定的松散点,一些理论的含义尚未充分确定或计算)。[26][27]

大约在牛顿于1727年去世三十年后,重力研究领域的数学天文学家亚历克西斯·克劳德·克莱罗在回顾胡克发表的文章后写道:“人们不能认为这个想法...胡克的存在削弱了牛顿的荣耀”;而“胡克的例子”用来“表明被瞥见的真理和被展示的真理之间的距离”。[28][29]

\subsection{现代形式}

在现代表述中,牛顿万有引力定律规定如下:

\begin{table}[ht]
\centering
\caption\label{tab_UG_1}
\begin{tabular}{|c|c|}
\hline
每个点质量都受到沿与两个点相交的线的作用力吸引的每个其他点质量。 力与两个质量的乘积成正比,与它们之间的距离的平方成反比:[2] \\
\hline
$F = G \frac{m_1 m_2}{r^2}$
其中:
\begin{itemize}
    \item $F$ 是质量之间的力;
    \item $G$ 是重力常数(\$6.674 \\times 10^{11} \\ \\text{N} \\cdot \\text{m}^2/\text{kg}^2$);
    \\item $m_1$ 是第一个物体的质量;
    \\item $m_2$ 是第二个物体的质量;
    \\item $r$ 是质心之间的距离。
\hline
\end{tabular}
\end{table}