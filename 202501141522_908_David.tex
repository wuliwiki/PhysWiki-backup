% 大卫·希尔伯特(综述)
% license CCBYSA3
% type Wiki

本文根据 CC-BY-SA 协议转载翻译自维基百科\href{https://en.wikipedia.org/wiki/David_Hilbert}{相关文章}。

\begin{figure}[ht]
\centering
\includegraphics[width=6cm]{./figures/9019106ae7482c98.png}
\caption{1912年的希尔伯特} \label{fig_David_1}
\end{figure}
大卫·希尔伯特(David Hilbert,发音:/ˈhɪlbərt/;德语:[ˈdaːvɪt ˈhɪlbɐt];1862年1月23日 – 1943年2月14日)是德国数学家和数学哲学家,是他那个时代最具影响力的数学家之一。

希尔伯特发现并发展了广泛的基础性思想,包括不变理论、变分法、交换代数、代数数论、几何学基础、算子谱理论及其在积分方程中的应用、数学物理学,以及数学基础(特别是证明理论)。他采纳并捍卫了乔治·康托尔的集合论和超限数理论。1900年,他提出了一系列问题,为20世纪的数学研究指明了方向。

希尔伯特及其学生为建立严格的数学理论做出了贡献,并发展了现代数学物理中重要的工具。他是证明理论和数学逻辑的共同创始人。