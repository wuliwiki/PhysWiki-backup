% 2013 年考研数学试题(数学一)
% keys 考研|数学
% license Copy
% type Tutor

\subsection{选择题}
\begin{enumerate}
\item 已知极限 $\lim_{n\to 0}\frac{x-\arctan x}{x^k}=c$ ,其中 $k,c$ 为常数,且 $c \neq 0$ ,则 ($\quad$)\\
(A) $\displaystyle k=2,c=-\frac{1}{2}$\\
(B) $\displaystyle k=2,c=\frac{1}{2}$\\
(C) $\displaystyle k=3,c=-\frac{1}{3}$\\
(D) $\displaystyle k=3,c=-\frac{1}{3}$
\item  曲面 $x^2+\cos (xy)+xyz+x=0$ 在点 $(0,1,-1)$ 处的切平面方程为 ($\quad$)\\
(A) $x-y+z=-2$\\
(B) $x+y+z=0$\\
(C) $x-2y+z=-3$\\
(D)  $x-y-z=0$
\item  设 $\displaystyle f(x)=\abs{x-\frac{1}{2}},b_n=2\int_{0}^{1}f(x)\sin n\pi x\dd{x} (n=1,2,\dots)$ ,令 $\displaystyle S(x)=\sum_{n=1}^\infty b_n \sin n\pi x$ ,则 $S(-\frac{9}{4})$ =($\quad$)\\
(A) $\displaystyle \frac{3}{4}$\\
(B) $\displaystyle \frac{1}{4}$\\
(C) $\displaystyle -\frac{1}{4}$\\
(D)  $\displaystyle -\frac{3}{4}$
\item  设 $L_1:x^2+y^2=1,L_2:x^2+y^2=2,  L_3:x^2+zy^2=2, L_4:2x^2+y^2=2$  为四条逆时针方向的平面曲线,记 $\displaystyle I_i=\oint (y+\frac{y_3}{6})\dd{x}+(2x-\frac{x^3}{3})\dd{y}\quad (i=1,2,3,4)$  ,则 $max\{I_1,I_2,I_3,I_4\}$ =($\quad$)\\
(A) $I_1$\\
(B)  $I_2$\\
(C)  $I_3$\\
(D)  $I_4$

\item 设 $\mat {A,B,C}$ 均为 $n$ 阶矩阵,若 $\mat {AB=C}$ ,且 $\mat B$ 可逆,则($\quad$)\\
(A) 矩阵 $C$ 的行向量组与矩阵 $A$ 的行向量组等价\\
(B) 矩阵 $C$ 的列向量组与矩阵 $A$ 的列向量组等价\\
(C) 矩阵 $C$ 的行向量组与矩阵 $B$ 的行向量组等价\\
(D) 矩阵 $C$ 的列向量组与矩阵 $B$ 的列向量组等价

\item  矩阵 $\pmat{1&a&1\\a&b&a\\1&a&1}$ 与 $\pmat{2&0&0\\0&b&0\\0&0&0}$ 相似的充分必要条件为 ($\quad$)\\
(A) $a=0,b=2$\\
(B) $a=0,b$为任意常数\\
(C) $a=2,b=0$\\
(D) $a=2,b$为任意常数
\item  设 $X_1,X_2,X_3$ 是随机变量,且 $X_1 $~ $N(0,1)$,$X_2 $~ $N(0,2^2)$,$X_3 $~ $N(5,3^2)$ ,$p_i=P\{-2 \le X_i \le2\}\quad  (i=1,2,3)$ 则($\quad$)\\
(A)  $p_1>p_2>p_3$\\
(B)  $p_2>p_2>p_3$\\
(C)  $p_3>p_1>p_2$\\
(D) $p_1>p_2>p_2$\\
\item 设随机变量 $X$~ $t(n)$,$Y$~ $F(1,n)$给定 $\alpha(0<a<0.5)$ ,常数 $c$ 满足 $P\{X>c\}=\alpha$  则  $P\{Y>c^2\}=$ ($\quad$)\\
(A) $\alpha$\\
(B) $1-\alpha$\\
(C) $2\alpha$\\
(D) $1-2\alpha $
\end{enumerate}
\subsection{填空题}
\begin{enumerate}
\item 设函数 $y=f(x)$ 由方程 $y-x=e^{x(1-y)}$ 确定,则  $\displaystyle \lim_{n\to\infty} n[f(\frac{1}{n})-1]$ = ($\quad$)
\item  已知 $y_1=e^{3x}-xe^{2x},y_2=e^x-xe^{2x},y_3=-xe^{2x}$  是某二阶常系数非齐次线性微分方程的三个解,则该方程的通解为 $y$ =($\quad$)
\item  设
$\leftgroup{&x=\sin t,\\&y=t\sin t+\cos t}$  (t为参数),则$\displaystyle \eval{ \dv[2]{y}{x}}_{t =\frac{\pi}{4}}$=($\quad$)
\item $\displaystyle \int_{1}^{+\infty }\frac{\ln x}{(1+x)^2}\dd{x}$=($\quad$)
\item 设 $\mat A $ 是三阶非零矩阵,$\abs{\mat A}$   为 $\mat A$ 的行列式,$\mat A_{ij}$  为 $a_{ij}$ 的代数余子式,若 $a_{ij}+A_{ij}=0 \quad (i,j=1,2,3)$ ,则 $\abs{\mat A}$=($\quad$)
\item 设随机变量 $Y$ 服从参数为1的指数分布, $a$ 为常数且大于零,则  $P\{Y \le a+1|Y>a\}$=($\quad$)
\end{enumerate}
\subsection{解答题}
\begin{enumerate}
\item 计算 $\displaystyle \int_{0}^{1}\frac{f(x)}{\sqrt{x}}$ ,其中  $\displaystyle f(x)=\int_{1}^{x}\frac{\ln(t+1)}{t}\dd{t}$.
\item 设数列 $\{a_n\}$ 满足条件 $a_0=3,a_1=1,a_{n-2}-n(n-1)a_n=0(n\ge 2)$ , $S(x)$ 是幂级数 $\displaystyle \sum_{n=0}^\infty a_nx^n$ 的和函数。\\
(1)证明 $S''(x)-S(x)=0$;\\
(2)求 $S(x)$ 的表达式.
\item 求函数 $\displaystyle f(x,y)=(y+\frac{x^3}{3})e^{x+y}$ 的极值.
\item 设奇函数 $f(x)$ 在 $[-1,1]$ 上具有二阶导数,且 $f(1)=1$ ,证明:\\
(1)存在 $\xi \in(0,1)$  ,使得 $f'(\xi)=1$;\\
(2)存在 $\eta \in (-1,1)$ ,使得 $f''(\eta)+f'(\eta)$.
\item 设直线 $L$ 过 $A(1,0,0),B(0,1,1)$ 两点,将 $L$ 绕 $z$ 轴旋转一周得到曲面 $\Sigma$ 与平面  所围成的立体为
\end{enumerate}