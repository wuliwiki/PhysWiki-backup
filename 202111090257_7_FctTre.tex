% 真因子树
% 环|因式分解|唯一析因环|因子|素理想|极大理想

\pentry{整环\upref{Domain}}

真因子树的概念,是笔者优化了“因子链”的概念而得出的一套描述因式分解理论的框架.

\subsection{概念的描述}

\begin{definition}{真因子}
给定整环 $R$,对于 $r, s\in R$,如果 $s|r$ 且 $r\not{|}s$,那么称 $r$ 是 $s$ 的真因子.
\end{definition}

\begin{definition}{单位}
给定整环 $R$,对于 $u\in R$,如果 $u^{-1}$ 是存在的,那么称 $u$ 是 $R$ 的一个\textbf{单位(unit)}.$R$ 中全体单位的集合,记为 $U$.
\end{definition}

显然,如果有单位 $u$ 使得 $r=us$,那么 $r$ 和 $s$ 互相不是真因子.我们将这样的 $r, s$ 视为等价的:

\begin{definition}{}
给定整环 $R$,定义集合 $R$ 上的一个等价关系:对于 $r, s\in R$,$r$ 等价于 $s$ 当且仅当存在单位 $u$ 使得 $r=su$.等价的元素视为同一个元素,或者说把每个等价类看成一个元素,得到的集合是 $R$ 模去该等价关系的商集\footnote{见\textbf{二元关系}\upref{Relat}.},记为 $R_u$.
\end{definition}

举例来说,在\textbf{整数环}$\mathbb{Z}$ 上,对于任意正整数 $n$,我们把它等价于 $-n$,于是 $\mathbb{Z}_u$ 也可以看成是非负整数的集合.

有了 $R_u$ 的概念,就可以定义本节核心的概念了:真因子树.

\begin{definition}{真因子树}
给定整环 $R$,对于 $r\in R$,如果存在非单位的 $a, b\in R$ 使得 $ab=r$,那么可以从 $r$ 画两个箭头分别指向 $a$ 和 $b$,而 $\{a, b\}$ 就是 $r$ 的一个因子分解;同样,如果 $a$ 和 $b$ 可以继续分解为其它非单位元素之积,那么也可以继续画出箭头指向它们对应的因子分解.如是反复,直到不能继续进行下去为止,所获得的整个结构称为 $r$ 的一棵\textbf{真因子树}.

每个从 $r$ 的\textbf{第一次分解}所得元素开始,出发一路指向末端的路径,称为 $r$ 的一个\textbf{枝条},枝条中涉及到的元素数量,称为枝条的\textbf{长度}.一棵真因子树中最长的枝条的元素数量,称为这棵树的\textbf{长度}.特别地,如果$r$无法进行分解,也就是说它的树只包含$r$本身,那么定义这棵树的长度为$0$.

一棵树中从元素$a$到元素$b$的路径,其长度定义为这条路径上包含的元素数量\textbf{减一}.

$r$ 的真因子树一般不止一棵.
\end{definition}

%要画图说明;说明等价关系是什么,以及真因子树长什么样.


\subsection{用真因子树进行描述}

以下讨论限制在整环 $R$ 的集合上.利用真因子树的语言来直接翻译各种概念的方式如下:

\begin{definition}{}\label{FctTre_def1}
\begin{itemize}
\item 不可约元素:在某一棵树中为末端.
\item 素元素:$p$ 是素元素,当且仅当对于任意 $a, b\in R$,如果 $p$ 在 $ab$ 的某个枝条上,那么 $p$ 必在 $a$ 或 $b$ 的某个枝条上.
\item 有限析因性:对于任意 $r$,存在一个正整数 $N_r$,使得 $r$ 的任意枝条长度不超过 $N_r$.
\item 唯一析因性:有限析因,且对于任意 $r$,其任何两棵树的末端元素构成的集合都是一样的.
\end{itemize}
\end{definition}

\begin{theorem}{素元素的等价定义}\label{FctTre_the1}
设$p$是整环$R$中的一个素元素,则对于任意$a$,如果$p|a$,那么$p$必在$a$的每一棵树上.
\end{theorem}

应用素元素的定义,不难证明该定理.


\begin{theorem}{素元素必是不可约元素}\label{FctTre_the2}
整环 $R$ 中的素元素都是不可约元素.
\end{theorem}

\textbf{证明}:

反设素元素 $p\in R$ 不是末端,那么就可以得到$p$的\textbf{长度不为零}的一棵树,其第一级分解为$p=ab$;而由于素元素的定义,它又必须在以$a$和$b$为起点的两根枝条中某一根的后面,从而在自己的后面,而这是不可能的.因此反设不成立,素元素必是末端.

\textbf{证毕}.

\begin{exercise}{}
利用\autoref{FctTre_the1} ,证明\autoref{FctTre_the2} .
\end{exercise}

今后我们也会引用真因子树的概念来方便阐释因式分解相关的问题.

