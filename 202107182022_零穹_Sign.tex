% 记号方法
% 常系数线性方程|记号方法

\begin{issues}
\issueDraft
\end{issues}

记号方法用于求解常系数线性方程(组),适当推广该方法,也可以用于比较复杂的问题.该方法要点在于把对自变量 $x$ 求微商的运算记号记作因子 $D$,写在需要求微商的函数的左边,于是若 $y$ 是 $x$ 的某一个函数,则
\begin{equation}
Dy=\dv{y}{x}
\end{equation}
\subsection{运算法则}
\begin{definition}{记号因子的方幂}\label{Sign_def1}
$D^ny=D(D^{n-1}y)\quad(n\in\mathbb{Z^{+}})$
\end{definition}
\begin{definition}{记号因子的负幂}
若 $D^sy=f(x)$ 且方程满足零初条件
\begin{equation}
y|_{x=x_0}=y'|_{x=x_0}=\cdots=y^{s-1}|_{x=x_0}=0\quad{s\in\mathbb{Z^{+}}}
\end{equation}
则记
\begin{equation}
D^{-s}f(x)=y
\end{equation}

\end{definition}
\begin{definition}{记号因子的加减法}
\begin{equation}
(D^{n_1}+D^{n_2})y=D^{n_1}y+D^{n_2}y\quad(n_1,n_2\in \mathbb{Z^{+
}})
\end{equation}
\end{definition}
\begin{definition}{记号因子的数乘}
\begin{equation}
(aD)y=a\cdot(Dy)\quad(a\in\mathbb{C})
\end{equation}
\end{definition}
利用上面的定义,容易证明以下几条性质
\begin{enumerate}
\item 
\begin{equation}
D^sy=\dv[s]{y}{x}
\end{equation}
\item 
\begin{equation}
D^{n_1}(D^{n_2}y)=D^{n_1+n_2}y
\end{equation}
\item 记号因子与任意常数因子可交换,即若 $a$ 为常数,则
\begin{equation}
aD^sy=D^s(ay)
\end{equation}
\item 若 $F(D)$ 是 $D$ 的具有常系数的多项式
\begin{equation}
F(D)=\sum_{i=0}^{n}a_iD^{n-i}
\end{equation}
则
\begin{equation}
F(D)y=\sum_{i=0}^{n}a_iD^{n-i}y,\quad F(D)ay=aF(D)y
\end{equation}
\item 若 $\varphi_1(D),\varphi_2(D)$ 是两个多项式,$\varphi(D)$是它们乘积,则
\begin{equation}
\varphi_1(D)\qty[\varphi_2(D)y]=\varphi(D)y,\quad \qty[\varphi_1(D)+\varphi_2(D)]y=\varphi_1(D)y+\varphi_2(D)y
\end{equation}
并且因子 $\varphi_1(D),\varphi_2(D)$ 可交换.
\item \begin{equation}
F(D)(e^{mx}y)=e^{mx}F(D+m)y
\end{equation}
\end{enumerate}

前5条性质是显然的,我们仅来证明最后一条性质.

\textbf{证明:}表达式 $F(D)(e^{mt}y)$ 由形为 $a_{i}D^{n-i}(e^{mx}y)$ 的项组成,于是只需证明对于每一个这样的项成立,ji




