% 有界集
% keys 有界集
% license Usr
% type Tutor

\pentry{拓扑向量空间\nref{nod_tvs}}{nod_84c5}
有界集的概念在拓扑线性空间起着重要的作用,它不仅是能够用来判定一个集合、空间是否有界,还是定义有界算子的基础。

\begin{definition}{有界集}\label{def_BudSet_1}
在\enref{拓扑线性空间}{tvs}中的集合 $M$ 称为\textbf{有界的}(bounded),是指对每一个零邻域(零矢量的邻域)$U$,存在 $n>0$,使得对所有 $\abs{\lambda}\geq n,M\subset \lambda U$。 
\end{definition}
有界集的概念和我们在赋范空间中按范数有界(即可把集置于某一球 $\norm{x}\leq R$ 内部)的理解是一致的。

\begin{definition}{局部有界}
拓扑线性空间 $E$ 称为\textbf{局部有界}的,若 $E$ 中至少存在一个非空有界开集。
\end{definition}


对于有界性有下面的定理成立。
\begin{theorem}{}\label{the_BudSet_1}
设 $E$ 是拓扑线性空间,则成立:
\begin{enumerate}
\item 集 $M\subset E$ 是有界的,当且将当对任何序列 $\{x_n\}\subset M$ 及任何趋于零的正数列 $\{\epsilon_n\}$,序列 $\epsilon_n x_n$ 趋于零;
\item 若 $\{x_n\}_{n=1}^\infty\subset E$ 且 $x_n\rightarrow x$,则 $\{x_n\}$ 是有界集;
\item 如果 $E$ 局部有界,则在 $E$ 中第一可数性公理成立。 
\end{enumerate}

\end{theorem}
这里只证明定理的一部分。


\begin{lemma}{}\label{lem_BudSet_1}
设 $\{x_n\}$ 是 $E$ 上\enref{收敛}{ConvTp}到 $x$ 的序列,$\{t_n\}$ 是收敛于0的数列。则 $\{t_nx_n\}$ 是 $E$ 中收敛于0(向量)的序列。
\end{lemma}

\textbf{证明:}任意零邻域 $U$,$U+x$ 是 $x$ 的邻域,因此存在 $N$,使得 $x_i\in U+x,i\geq N$,进而 $x_i-x\in U,i\geq N$,即 $\{x_n-x\}$ 是 $E$ 上收敛于0的序列。

由\enref{数乘的连续性}{tvs},任意零邻域 $U$,存在 $x$ 的邻域 $U_x$ 和 $\delta>0$,使得 $[-\delta,\delta]\cdot U_x\subset U$。由于 $\{x_n\}\rightarrow x,\{t_n\}\rightarrow0$,所以存在 $N_1,N_2>0$,使得只要 $n_1\geq N_1,n_2\geq N_2$,就有 $x_{n_1}\in U_x,t_{n_2}\in[-\delta,\delta]$。取 $N=\max\{N_1,N_2\}$,则对 $n>N$,成立

\begin{equation}
t_nx_n\in[-\delta,\delta]\cdot U_x\subset U~.
\end{equation}
于是 $\{t_nx_n\}$ 收敛于0。

\textbf{证毕!}

\textbf{\autoref{the_BudSet_1} 的证明:}1.\textbf{必要性:} 要证 $\epsilon_n x_n\rightarrow0$,就是要证对任一零邻域 $U$,存在正整数 $N$,使得 $n\geq N$,就有 $\epsilon_n x_n\subset U$。其证明如下:

由数乘的连续性,对任一零邻域 $U$,存在 $\delta>0$ 和零邻域 $V$,使得 $(-\delta,\delta)\cdot V\subset U$;由 $M$ 的有界性, 存在 $m>0$,使得只要 $\abs{\lambda}\geq m$,就有 $M\subset\lambda V$。 因此 $\epsilon_nx_n=\epsilon_n \lambda y_n$,其中 $y_n\in V$。

因为 $\{\epsilon_n\}$ 趋于零,所以存在 $N$,使得只要 $n\geq N$,就有 $\abs{\epsilon_n\lambda}<\delta$。从而 $\epsilon_nx_n=\epsilon_n \lambda y_n\in U$ 对 $n\geq N$ 恒成立。


2. 设 $\{x_n\}$ 无界。则存在零邻域 $U$,使得任意 $m>0$,存在 $t\geq m,y\in\{x_n\},x\notin tU$,或 $t^{-1}x\notin U$。选取 $\{m_n\}\rightarrow\infty$,于是存在 $\{t_n\}\rightarrow\infty$,使得 $t_n^{-1}y_n\notin U$ 对所有的 $n$ 成立 。下面将证明这是不可能的:

若点列 $\{t_n^{-1}y_n\}$ 包含 $\{x_n\}$ 的"无限"个点,即任意 $N>0$,有 $i>N$,使得 $x_i\in\{y_n\}$。那么对任意 $N>0$,存在 $N_2>0$, 使得 $i\geq N_2$, $y_i\in \{x_n|n\geq N\}$:否则,存在 $N>0$,使得对任意 $N_2>0,i\geq N_2$,都有 $y_i\notin\{x_n|n\geq N\}$,于是一切 $y_i$ 只能在 $\{x_1,\cdots,x_N\}$ 中取值而不能取 $N$ 后 $x_i$,这和假设矛盾。因此此时对从某项 $N$ 开始落入任一 $x$ 邻域的 $x_i$ 而言, 对充分大的 $N_2$,从该项开始 $y_i$ 落入该邻域,即 $\{y_n\}\rightarrow x$。由\autoref{lem_BudSet_1} ,这和 $\{t_n^{-1}y_n\}\rightarrow0$ ,这和所有的 $n$,  $t_n^{-1}y_n\notin U$ 矛盾。 

 设点 $\{y_n\}$ 只取 $\{x_n\}$ 上的有限个点 $x_1,\cdots,x_M$。那么由\autoref{lem_BudSet_1} ,每一序列 $\{t_n^{-1}x_i\},i=1\cdots,M$ 都收敛于0。即对任意零邻域 $U$,存在 $N_1,\cdots,N_M>0$,只要 $n_i\geq N_i,i=1,\cdots,M$,就有 $t_{n_i}x_{n_i}\in U$。取 $N=\max\{N_i|i=1,\cdots,M\}$,则只要 $n\geq N$,就有 $t_n y_n\in U$,即 $\{t_n y_n\}\rightarrow 0$,这也和所有的 $n$,  $t_n^{-1}y_n\notin U$ 矛盾。

 因此第2点得证。


\textbf{证毕!}











