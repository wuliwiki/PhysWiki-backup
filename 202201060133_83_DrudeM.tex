% 德鲁德模型

\begin{issues}
\issueDraft
\end{issues}

汤姆逊(J.J.Thomsom)在1987年发现了电子,这对物质结构的物理理论产生了直接且深刻的影响.显然的也就引出了金属导电性和其内部自由电子的存在息息相关.

在汤姆逊的发现三年之后,德鲁德(Drude)较为成功的借鉴了理想气体动力学理论的思想和假设,并且将其运用在对金属的研究上.

德鲁德的自由电子气体模型简单地将金属看作是由\textbf{价电子(Valence electron)}所构成的基本均匀的电子气体.
\begin{figure}[ht]
\centering
\includegraphics[width=5cm]{./figures/DrudeM_1.pdf}
\caption{德鲁德模型电子(此处以蓝色显示)不断在较重的静止晶体离子(以红色显示)之间反弹.} \label{DrudeM_fig1}
\end{figure}
参考:\href{https://en.wikipedia.org/wiki/Drude_model}{维基百科-Drude_model}
\subsection{假设}
\begin{enumerate}
\item \textbf{独立电子近似(Independent electron approximation)}:电子之间不会相遇,不存在任何相互作用.
\item \textbf{自由电子近似}:1.电子和离子之间不会相遇;2.电子在每次碰撞前后都是沿着直线运动;
\item \textbf{Jellium 近似}:正电荷,也就是原子核被假定均匀分布在空间中;电子密度在空间中也是一个均匀的量.由于正电荷的均匀分布,因此其对电子施加的电场为零.
\item 类似于理想气体的碰撞:电子会“忘记”碰撞前的速度,也就是电子碰撞前后的速度互不相关.电子碰撞之后的速度由能量(温度)决定:
\begin{equation}
\frac{3}{2}kT = \frac{1}{2}mv^2
\end{equation}
\subsection{模型的建立}
\end{enumerate}
首先,我们先来直接看到电子密度$n$的公式为:
\begin{equation}
n=N_A\frac{Z\rho_m}{A}
\end{equation}

其中$N_A=6.022\times 10^{23}$(阿伏伽德罗常数)表示每摩尔金属元素的原子个数,单位是$\rm{atoms/mole}$.$\rho_m$是元素的质量密度,单位是$\rm{g/cm^3}$.$A$是元素的相对原子量,单位是$\rm{g/mole}$.不难看出单位体积下物质的数量为$\rho_m/A$.由于$Z$是每个原子所提供的价电子数量,因此我们也就有了上述电子密度的公式.

接下来我们将每个电子平均的体积近似的看作成一个半径为$r_s$的球体,并且用其来表示电子密度的大小,也就是电子之间平均间隔的大小:

\begin{equation}
\frac{V}{N}=\frac{1}{n}=\frac{4}{3}\pi r_s^3 \Rightarrow r_s=\left(\frac{3}{4\pi n}\right)^\frac{1}{3}
\end{equation}
其中$V$是金属体积,$N$是总体导电的电子数量.由于$r_s$的大小大约在$0.1\rm{nm}$左右,因此我们习惯用Ångström单位$\buildrel _{\circ} \over {\mathrm{A}}=0.1\rm{nm}$,或者波尔半径$a_0=0.529\times 10^{-1}\rm{nm}$用作长度计量单位.

上述的模型不难让我直接就联想到了这似乎能够揭示金属电阻的某些性质.

\subsection{金属DC电路的导电率}

根据欧姆定律,穿过导线的电流$I$和电势差$V$成正比:$V=IR$,其中的电阻就是$R$.德鲁德模型就能够解释这一现象并且不通过实验测量估计出电阻的大小.

在考虑电阻的时候,一般我们不考虑导线的形状,而是定性的考虑导线的金属构成.我们将导线材质本身的电阻率$\rho$定义为一个在电场$\bvec E$和电流密度$\bvec j$的某一点上的正比常数:
\begin{equation}
\bvec E =\rho \bvec j
\end{equation}
