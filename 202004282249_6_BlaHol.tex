% 黑洞

\pentry{圆周运动的加速度\upref{CMAD},万有引力\upref{Gravty}}

\textbf{黑洞(black hole)}是天体物理学预言的一类天体,其特征是它的引力非常大,它“吞噬”周围的所有物质,甚至连光也无法逃逸出去,所以称为黑洞.早在1795 年,\textbf{拉普拉斯(P. S. M. Laplace)} 就预言过黑洞的存在.根据机械能守恒定律,一个质量为$m$的物体如果要从一个球状星体上逃逸,它的速度$u $至少要满足下列关系:
\begin{equation}
\frac{1}{2} m v^{2}=\frac{G m m_{\mathrm{C}}}{R}
\end{equation}
式中$G $为万有引力常量,$ m_{\mathrm{C}}$为星球质量,$ R $为星球半径,即其逃逸速度为$v \geqslant \sqrt{\dfrac{2 G m_{\mathrm{C}}}{R}}$.如果
\begin{equation}
\frac{2 G m_{\mathrm{c}}}{R} \geqslant c^{2}
\end{equation}
其中$c$为光速,那么这个星球就成为一个黑洞.此时,星球的半径与质量的关系为
\begin{equation}
R \leqslant \frac{2 G m_{\mathrm{C}}}{c^{2}}
\end{equation}