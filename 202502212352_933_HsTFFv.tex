% 函数视角下的三角函数(高中)
% keys 函数|三角函数|周期|性质
% license Usr
% type Tutor

\begin{issues}
\issueDraft
\end{issues}

\pentry{三角函数\nref{nod_HsTrFu},函数\nref{nod_functi},函数的性质\nref{nod_HsFunC},导数\nref{nod_HsDerv},导数的计算\nref{nod_HsDerB}}{nod_5a43}

在前面的内容中,已经接触过三角函数的定义,并基于这些定义推导出了诱导公式及同角三角函数之间的关系。这些推导主要依赖于任意角和三角函数的几何定义。然而,三角函数不仅仅是几何分析的工具,它们本质上也是一种函数,并具备一般函数的基本性质,如周期性、单调性和对称性。因此,本文将从函数的角度进一步分析三角函数,考察它们的性质、图像、变化趋势等。需要注意的是,这些视角本质上是等价的,它们都在描述同一数学对象。无论是几何定义还是函数分析,最终指向的都是相同的数学结构。这种多重视角的统一性,正是三角函数作为数学工具的强大之处。它不仅能够通过直观的几何形式展现对称性和变换规律,也能在函数的框架下揭示更广泛的性质,为各种数学应用提供坚实的基础。

另外,在三角函数的介绍中,有一个广为流传的动画:一个点在单位圆上运动,表示角度的变化,同时,在单位圆的右侧和上侧,将角度与对应的线段长度映射到另一坐标系,从而自然引出各个三角函数的图像。尽管这种动画能够直观展示三角函数的变化过程,更理想的方式是能够在脑海中主动演练这一过程。看到函数图像时,能够自动联想到单位圆上的点如何旋转;反之,观察圆周运动时,能够迅速在脑海中构建出相应的函数图像。这种能力不仅有助于理解三角函数的本质,也将在更深入的数学学习中提供帮助。本文内容主要关注正弦、余弦与正切函数,其余三角函数由于与它们存在倒数关系,将适当涉及,但不会展开详细推导。

\subsection{三角函数的性质}

按照之前分析其他函数的思路,仍旧要先讨论三角函数的性质,并在此基础上推导它们的图像。不过下面的分析过程,并未按照之前熟悉的顺序进行,而是基于定义及已研究的恒等关系,利用这些内容快速得出相关性质。事实上,由于函数的关系是确定的,因此在研究其性质时,可以根据分析的难易程度安排顺序,而不必拘泥于固定的讨论方式。

\subsubsection{定义域}

由于三角函数的自变量是任意角,因此理论上,它们的定义域应覆盖整个实数集。然而,在之前的讨论中\aref{提及}{eq_HsTrFu_13}过,某些三角函数在特定角度下无意义。例如,$\tan x$ 在 $\displaystyle x=\frac{\pi}{2}+k\pi, (k\in\mathbb{Z})$ 处没有定义。类似地,其他三角函数也存在某些不可取值的点。

综上所述:
\begin{itemize}
\item $\sin x,\cos x$ 的定义域为 $\mathbb{R}$;
\item $\tan x,\sec x$ 的定义域为 $\displaystyle\{x|x\neq\frac{\pi}{2}+k\pi,k\in\mathbb{Z}\}$,或写作$\displaystyle\{x|x\neq(2k+1)\frac{\pi}{2},k\in\mathbb{Z}\}$,即 $x$ 不能取 $\displaystyle\frac{\pi}{2}$ 的奇数倍;
\item $\cot x,\csc x$ 的定义域为 $\displaystyle\{x|x\neq k\pi,k\in\mathbb{Z}\}$,即 $x$ 不能取 $\pi$ 的整数倍。
\end{itemize}

\subsubsection{零点}

根据三角函数的定义,正弦函数和正切函数的零点出现在角 $x$ 的终边与 $x$ 轴重合的情况。按照弧度的定义,这对应于 $x = 0+2k\pi$ 和 $x = \pi+2k\pi$,合并后可得, $\sin x$ 和 $\tan x$ 的零点为:
\begin{equation}\label{eq_HsTFFv_1}
x = k\pi, \qquad (k\in\mathbb{Z})~.
\end{equation}

类似地,余弦函数和余切函数的零点出现在角 $x$ 的终边与 $y$ 轴重合的情况。对应的角度分别为 $\displaystyle{\frac{\pi}{2}} + 2k\pi$ 和 $\displaystyle{\frac{3\pi}{2}} + 2k\pi$,合并后可得 $\cos x$ 和 $\cot x$ 的零点为:
\begin{equation}
x = \frac{\pi}{2} + k\pi, \qquad (k\in\mathbb{Z})~.
\end{equation}

另一方面,由于 $\sec x$ 和 $\csc x$ 可理解为单位圆外某点到原点的连线长度,因此它们的值最小为 $1$,不会取到零值,即 $\sec x$ 和 $\csc x$ 不存在零点。

综上所述:
\begin{itemize}
\item $\sin x, \tan x$ 的零点为 $x = k\pi, (k\in\mathbb{Z})$;
\item $\cos x, \cot x$ 的零点为$\displaystyle x = \frac{\pi}{2} + k\pi, (k\in\mathbb{Z})$;
\item $\sec x, \csc x$没有零点。
\end{itemize}

值得注意的是,三角函数的零点和奇异点(即定义域中无法取到的点)与其周期性、对称性存在紧密联系。而这些特殊点也都出现在终边与坐标轴重合的情况下,即不是象限角所限定的范围。

\subsubsection{周期性}

根据\aref{之前的分析}{sub_HsTrFu_1},所有的三角函数都是周期函数,并且 $2\pi$ 是它们的一个周期。下面将分析 $2\pi$ 是否是它们的最小正周期。

对于 $\sin x$,设 $T$ 是其最小正周期,则根据周期函数的\aref{定义}{def_HsFunC_6},满足:
\begin{equation}\label{eq_HsTFFv_2}
\sin (x+T) = \sin x~.
\end{equation}
取 $x=0$,得到 $\sin T = \sin 0$。由于 $x=0$ 是 $\sin x$ 的一个零点,因此 $T$ 也必然是$\sin x$\aref{的一个零点}{eq_HsTFFv_1}。若 $T<2\pi$,则 $T=\pi$,但根据诱导公式 $\sin(\pi + x) = -\sin x$,可得:
\begin{equation}\label{eq_HsTFFv_3}
\sin (T + x) = -\sin x~.
\end{equation}
结合 \autoref{eq_HsTFFv_2} 和 \autoref{eq_HsTFFv_3},若 $T=\pi$,则要求 $\sin x$ 恒等于零,与实际情况矛盾。因此,$\sin x$ 的最小正周期是$T = 2\pi$。

对于 $\cos x$,同理,取 $x=\frac{\pi}{2}$ 代入类似的推导,也可以得出 $\cos x$ 的最小正周期为 $2\pi$。由于 $\sec x$ 和 $\csc x$ 分别是 $\cos x$ 和 $\sin x$ 的倒数,而$\displaystyle y={1\over x}$在定义域上都是单调的,也即存在一一映射,因此它们也具有相同的最小正周期。

对于 $\tan x$,根据诱导公式$\tan (\pi + x) = \tan x$可知 $\pi$ 是其一个周期。设 $T$ 是 $\tan x$ 的最小正周期,则:
\begin{equation}
\tan (T + x) = \tan x~.
\end{equation}
取 $x=0$,可知 $T$ 必然是 $\tan x$ 的零点。而根据 \autoref{eq_HsTFFv_1},$\tan x$ 不存在比$\pi$还小的零点,因此 $\tan x$ 的最小正周期为$T = \pi$。同理,由于 $\cot x$ 是 $\tan x$ 的倒数,它的最小正周期也为$T = \pi$。

综上所述:
\begin{itemize}
\item $\sin x, \cos x, \sec x, \csc x$ 的最小正周期为 $2\pi$;
\item $\tan x, \cot x$ 的最小正周期为 $\pi$。
\end{itemize}

\subsubsection{导数及单调性}



\subsubsection{奇偶性及对称性}

由于诱导公式本身是研究三角函数对称性的工具,并且在此前已经进行了深入分析,因此可以直接结合诱导公式和周期性来确定三角函数的奇偶性和对称性。

根据\aref{诱导公式}{eq_HsTrFu_14}以及奇偶性的\aref{定义}{def_HsFunC_7}可知:
\begin{itemize}
\item $\sin x$ 和 $\tan x$ 满足 $f(-x) = -f(x)$,因此它们是奇函数。
\item $\cos x$ 满足 $f(-x) = f(x)$,因此它是偶函数。
\end{itemize}

此外,根据三角函数的倒数关系可以推得:
\begin{itemize}
\item $\csc x$ 和 $\cot x$ 由于与 $\sin x$ 和 $\tan x$ 具有相同的符号变化特性,因此也是奇函数;
\item $\sec x$ 由于与 $\cos x$ 具有相同的符号变化特性,因此是偶函数。
\end{itemize}

从三角函数的奇偶性可以直接推得:
\begin{itemize}
\item $\cos x$ 和 $\sec x$的对称轴为 $x=0$。
\item $\sin x,\tan x,\csc x,\cot x$的对称中心为 $(0,0)$。
\end{itemize}
然而,对称性不仅限于此。根据\aref{诱导公式}{eq_HsTrFu_16}:
\begin{equation}
\cos(x+\pi) = -\cos x,\quad\sin(x+\pi) = -\sin x~.
\end{equation}
zhe yi wei zhe
\begin{itemize}
\item $\cos x$ 左移 $\pi$ 个单位变为 $-\cos x$,而 $-\cos x$ 仍是偶函数,因此 $x=\pi$ 也是 $\cos x$ 的对称轴。
\item $\sin x$ 左移 $\pi$ 个单位变为 $-\sin x$,而 $-\sin x$ 仍是奇函数,因此 $(\pi,0)$ 也是 $\sin x$ 的对称中心。
\end{itemize}
这表明 $\cos x$ 向左平移 $\pi$ 个单位后变为 $-\cos x$,而 $-\cos x$仍满足偶函数的定义,因此$x=\pi$ 也是 $\cos x$ 的对称轴; $\sin x$ 向左平移 $\pi$ 个单位后变为 $-\sin x$,而 $-\sin x$仍满足奇函数的定义,因此$(\pi,0)$ 也是 $\sin x$ 的对称中心。

而根据\aref{诱导公式}{eq_HsTrFu_14}:
\begin{equation}
\sin(x+\frac{\pi}{2}) = \cos x~.
\end{equation}
这意味着 $\sin x$ 向左平移 $\displaystyle\frac{\pi}{2}$ 得到 $\cos x$,从而可以推出 $\sin x$ 具有对称轴$\displaystyle x=\frac{\pi}{2}$和$\displaystyle x=\frac{3}{2}\pi$。当然,反过来也可以认为$\cos x$向右平移 $\frac{\pi}{2}$ 得到$\sin x$,即$\cos x$ 具有对称中心$\displaystyle (-\frac{\pi}{2},0)$和$\displaystyle (\frac{\pi}{2},0)$。

由于 $\tan x$ 和 $\cot x$ 都是奇函数,根据单调性,它们在一个周期内始终保持单调递增或递减。因此,不论如何进行平移或转换,都无法出现对称轴。但根据\aref{诱导公式}{eq_HsTrFu_14}:
\begin{equation}
\tan(x+\frac{\pi}{2}) = -\cot x~.
\end{equation}
这表明 $\tan x$ 向左平移$\displaystyle\frac{\pi}{2}$ 个单位后变为 $-\cot x$,而 $-\cot x$仍满足奇函数的定义,因此$\displaystyle ({\pi\over2},0)$ 也是 $\tan x$ 的对称中心。注意,尽管$\displaystyle x=\pi\over2$并不在$\tan x$的定义域内。这有点像$\displaystyle y={1\over x}$的情况。而正是由于其不在定义域内,很容易被忽略。当然,反之$\displaystyle (-{\pi\over2},0)$ 也是 $\cot x$ 的对称中心。


上面分析的都是一个周期内的情况,由于三角函数具有周期性,其对称中心和对称轴会随周期性重复,总结如下:
\begin{itemize}
\item $\sin x$ 是奇函数,对称中心为 $x = k\pi$,对称轴为 $x = \frac{\pi}{2} + k\pi$。
\item $\cos x$ 是偶函数,对称轴为 $x = k\pi$。
\item $\tan x$ 是奇函数,对称中心为 $x = k\pi$,无对称轴。
\item $\cot x$ 是奇函数,对称中心为 $x = \frac{\pi}{2} + k\pi$,无对称轴。
\item $\sec x$ 是偶函数,对称轴为 $x = k\pi$。
\item $\csc x$ 是奇函数,对称中心为 $x = \frac{\pi}{2} + k\pi$,无对称轴。
\end{itemize}

这些对称关系的背后,根本上来说是单位圆的对称性带来的。



\subsection{值域}

正弦和余弦函数的值域相对直观,它们对应于单位圆上点的纵坐标和横坐标,因此取值范围显然是 $[-1,1]$。

相比之下,正切函数的值域分析稍显复杂。根据几何定义,在锐角情况下,正切函数对应的线段长度受终边位置影响。参见\aref{几何示意图}{fig_HsTrFu_1},线段的一端是固定点 $X_0$ ,而长度取决于另一端$T$的移动情况。分析 $\displaystyle x\in\left[0,\frac{\pi}{2}\right)$ 时的情形:
\begin{itemize}
\item 当 $x=0$ 时,角的终边与 $x$ 轴重合,线段两个端点也重合,长度为 $0$;
\item 随着 $x$ 增大,终边与 $x$ 轴夹角增加,点 $T$ 沿着单位圆向上移动,可以取到直线$x=1$在第一象限中的所有点,使得对应的线段长度不断增加;
\item 当 $x=\frac{\pi}{2}$ 时,终边与 $y$ 轴重合,二者平行无交点。
\end{itemize}
同理,利用对称性,由 $\tan(-x) = -\tan x$ 可知,$\displaystyle x\in\left(-{\pi\over2},0\right]$ 时,正切函数的取值是第四象限中对应的所有点。综上所述,$\tan x$ 在 $\displaystyle \left(-\frac{\pi}{2},\frac{\pi}{2}\right)$ 内可以遍历所有实数。

这一点也可以由极限分析得到\footnote{下述分析较为严谨,但在高中阶段不作要求。}。由于 $\displaystyle\tan x = \frac{\sin x}{\cos x}$,而 $\cos x$ 出现在分母上,关键在于 $\cos x$ 在 $\displaystyle\frac{\pi}{2}$ 附近的变化:
\begin{itemize}
\item 当 $\displaystyle x \to \frac{\pi}{2}^-$(即 $x$ 从左侧逼近 $\frac{\pi}{2}$)时,$\cos x$ 逐渐趋近于 $0$ 且 $\cos x > 0$,导致 $\tan x$ 迅速增大,趋向 $+\infty$;
\item 当 $\displaystyle x \to \frac{\pi}{2}^+$(即 $x$ 从右侧逼近 $\frac{\pi}{2}$)时,$\cos x$ 依然趋向 $0$,但 $\cos x < 0$,因此 $\tan x$ 迅速变小,趋向 $-\infty$。根据诱导公式,$\tan x=\tan(x-\pi)$,因此,当$\displaystyle x \to -\frac{\pi}{2}^+$时,$\tan x$ 趋向 $-\infty$。
\end{itemize}

由此可见,$\tan x$ 在 $\displaystyle (-\frac{\pi}{2},\frac{\pi}{2})$ 内是连续变化的\footnote{也可以通过单调递增来说明连续性,但单调性的证明需要学习求导公式后才能得出。},因此它的值域为 $\mathbb{R}$。


这一点也可以由极限分析得到\footnote{注意下面的分析比较严谨,但在高中阶段不要求。},由于 $\cos x$在$\tan x={\sin x\over\cos x}$的分母上,而当 $x \to \frac{\pi}{2}$时,$\sin x $始终保持趋近$1$,为正值。因此需要研究 $\cos x$ 在 $\frac{\pi}{2}$ 附近的变化。当 $x \to \frac{\pi}{2}^-$(即 $x$ 从左侧逼近 $\frac{\pi}{2}$,或者说$x$是第一象限角)时,$\cos x$ 逐渐趋近于 $0$ 且 $\cos x > 0$。因此, $\tan x$ 迅速增大,趋于 $+\infty$。类似地,当 $x \to \frac{\pi}{2}^+$(即 $x$ 从右侧逼近 $\frac{\pi}{2}$,或者说$x$是第二象限角),$\cos x$ 依然趋向 $0$,但此时 $\cos x < 0$。所以 $\tan x$ 迅速变小,趋向 $-\infty$,根据诱导公式,$\tan x=\tan(x-\pi)$,因此,当$x \to -\frac{\pi}{2}^+$时,$\tan x$ 趋向 $-\infty$。而根据前面的分析$\tan x$ 在 $(-\frac{\pi}{2}, \frac{\pi}{2})$内是连续变化的\footnote{也可以用单调递增来说明连续性,不过单调递增需要后面学习求导公式后才能得出。},所以$\tan x$ 必然遍历整个 $\mathbb{R}$,即它的值域是 $\mathbb{R}$。

\subsection{图像}
根据前面的推导,可以得到基本三角函数的图像如下图。
总结一下:
可以看出正弦函数和余弦函数是定义域为 $R$ 值域为 $[-1,1]$ 最小正周期 $T = 2\pi$ 的周期函数。

\begin{figure}[ht]
\centering
\includegraphics[width=14.25cm]{./figures/14fd66d8d1e6e0b5.png}
\caption{$\sin x$和$\cos x$} \label{fig_HsTFFv_1}
\end{figure}

可以看到,二者的图像几乎一模一样,看上去就是$\sin x$向左平移了${\pi\over2}$个单位得到的。

\begin{figure}[ht]
\centering
\includegraphics[width=14.25cm]{./figures/6f97182187b36e36.png}
\caption{$\tan x$和$\cot x$} \label{fig_HsTFFv_3}
\end{figure}

作为扩展,下面也给出正割函数与余割函数的函数图像,他们的性质均可通过与正弦和余弦的关系分析得到,此处不予赘述。

\begin{figure}[ht]
\centering
\includegraphics[width=14.25cm]{./figures/56f93ee1a7fb0faa.png}
\caption{$\sec x$和$\csc x$} \label{fig_HsTFFv_2}
\end{figure}

\begin{table}[ht]
\centering
\caption{常用的三角函数值}\label{tab_HsTFFv1}
\begin{tabular}{|c|c|c|c|}
\hline
* & * & * & * \\
\hline
* & * & * & * \\
\hline
* & * & * & * \\
\hline
* & * & * & * \\
\hline
\end{tabular}
\end{table}

\subsection{正弦型函数}

\begin{definition}{正弦型函数}
形如
\begin{equation}
f(x)=A\sin(\omega x+\varphi)~.
\end{equation}
的函数称为\textbf{正弦型函数},其中$A,\omega,\varphi$为常数,且$A\omega\neq0$。
\end{definition}
其实在前面的介绍中已经接触过这种例子了,那就是$\cos x$。根据诱导公式有:
\begin{equation}
\cos x=\sin(x+{\pi\over2})~.
\end{equation}
他就是$\displaystyle A=\omega=1,\varphi={\pi\over2}$的正弦型函数。

\addTODO{五点法作图}

\subsection{导数的规律}