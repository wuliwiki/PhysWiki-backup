% Lippmann-Schwinger 方程

\begin{equation}
H = H_0 + V, \qquad H_0 = \frac{\bvec p^2}{2m}
\end{equation}
令 $H_0\ket{\phi} = E\ket{\phi}$, $(H_0+V)\ket{\psi}=E\ket{\psi}$. 那么形式上就有
\begin{equation}
\ket{\psi} = \frac{1}{E-H_0} V\ket{\psi} + \ket{\phi}
\end{equation}
但 $1/(E-H_0)$ 是奇异的, 要解决这个问题, 可以把它变得稍微复数一些
\begin{equation}
\ket*{\psi^{(\pm)}} = \ket{\phi} + \frac{1}{E-H_0\pm \I\varepsilon} V \ket*{\psi^{(\pm)}}
\end{equation}
放到位置表象中就是
\begin{equation}\label{LipSch_eq1}
\braket*{\bvec x}{\psi^{(\pm)}} = \braket{\bvec x}{\phi} + \int \dd[3]{x'} \mel{\bvec x}{\frac{1}{E-H_0\pm \I\varepsilon}}{\bvec x'} \mel{\bvec x'}{V}{\psi^{(\pm)}}
\end{equation}
这是一个积分方程. 如果 $\ket{\phi}$ 是平面波.

\begin{equation}
G_\pm (\bvec x, \bvec x') = \frac{1}{2m} \mel{\bvec x}{\frac{1}{E-H_0\pm \I\varepsilon}}{\bvec x'}
\end{equation}
可以证明
\begin{equation}
G_\pm (\bvec x, \bvec x') = -\frac{1}{4\pi} \frac{\E^{\pm\I k\abs{\bvec x-\bvec x'}}}{\abs{\bvec x-\bvec x'}}
\end{equation}
其中 $k = \sqrt{2mE}$. 这就是亥姆霍兹方程的格林函数
\begin{equation}
(\laplacian +k^2)G_\pm (\bvec x, \bvec x') = \delta(\bvec x-\bvec x')
\end{equation}

经过一番推导, \autoref{LipSch_eq1} 变为
\begin{equation}
\braket*{\bvec x}{\psi^{(\pm)}} = \braket{\bvec x}{\phi} - 2m\int \dd[3]{x'} \frac{\E^{\pm\I k\abs{\bvec x-\bvec x'}}}{4\pi\abs{\bvec x-\bvec x'}} V(\bvec x')\braket*{\bvec x'}{\psi^{(\pm)}}
\end{equation}

然后计算可以发现边界条件
\begin{equation}
\braket*{\bvec x}{\psi^{(\pm)}} \overset{r\to\infty}{\longrightarrow} \frac{1}{(2\pi)^{3/2}} \qty[\E^{\I \bvec k\vdot \bvec x} + f(\bvec k', \bvec k)\frac{\E^{\I kr}}{r}]
\end{equation}
其中
\begin{equation}
f(\bvec k', \bvec k) = -4\pi^2 m \mel{\bvec k'}{V}{\psi^{(+)}}
\end{equation}

