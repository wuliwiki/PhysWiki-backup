% 生日问题
% keys 生日问题|生日悖论|概率论
% license Usr
% type Wiki

\subsection{生日问题}

在概率论中,生日问题是最经典的概率论问题之一。生日问题探讨的是在一个集合中随机选择的人达到至少有两个人在同一天生日的概率。这个问题的常见形式是:

\begin{example}{生日问题}
在一个房间里需要多少人,才能使得至少有两个人生日相同的概率至少为50\%?
\end{example}

为了简化分析过程,我们假定:一年只有365天;每个人等可能地出生在一年中的任何一天,即出生日期服从均匀分布;客人的出生日期都是相互独立的;
 


\textbf{生日悖论}

生日问题也叫"生日悖论",此处的"悖论"并不是真正的悖论,而是描述一种现象: 这个问题的直观答案常常出乎人们的预料,即:"在集合相对较小的情况下,元素重复的概率就已经非常高了"。


参考:普林斯顿概率论读本