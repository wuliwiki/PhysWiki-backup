% 高斯单位制
% 厘米|克|秒|国际单位|CGS

\pentry{厘米—克—秒单位制\upref{CGS}}

高斯单位制是在 CGS 单位制\upref{CGS}的基础上添加了一些电磁学相关的单位。 以下先来定义各个物理量的转换常数, 
已知 CGS 的转换常数为
\begin{equation}\label{GaussU_eq3}
\begin{aligned}
&\beta_x = 0.01\Si{m/cm} &\qquad &\beta_m = 0.001\Si{kg}/\Si{g}\\
&\beta_t = 1 &\qquad  &\beta_F = \beta_x \beta_m = 10^{-5}\Si{N/dyn}
\end{aligned}
\end{equation}

\subsubsection{电荷}
为了满足\autoref{GaussU_eq1},与国际单位相比, 高斯单位制中电荷的量纲发生变化, 单位为 $\Si{\sqrt{cm^3 g}/s}$, 为了方便我们不妨记为\footnote{$C_g$ 是笔者发明的记号, 方便理解记忆。} $C_g$。 则电荷的转换常数为
\begin{equation}
\beta_q = \sqrt{4\pi\epsilon_0\beta_F} \beta_x \approx 3.3356409510736\e{-10} \Si{C/C_g}
\end{equation}
假如把 $\Si{C}$ 和 $\Si{C_g}$ 看成是相同的量纲,则以下除了磁场$\beta_B$ 外其他转换常数都是无量纲的 1, $\beta_B$ 量纲为速度分之一。

\subsubsection{电场}
为满足\autoref{GaussU_eq2} , 电场单位为 $\Si{\sqrt{g/cm}/s} = \Si{C_g/cm^2}$。 转换常数为
\begin{equation}
\beta_{\mathcal E} = \frac{\beta_F}{\beta_q} = \sqrt{\frac{\beta_m}{4\pi\epsilon_0 \beta_x}} \approx 2.997924580815998\e4 \Si{\frac{C_g\cdot m\cdot kg}{C\cdot cm\cdot g}}
\end{equation}
若采用 2020 年 5 月以前的国际单位标准\upref{SIunit}, 数值上等于 $10^{-4} c_{s}$, $c_{s}$ 是国际单位的光速。

\subsubsection{磁场}
为满足\autoref{GaussU_eq2} , 电场和磁场应具有相同的单位 $\Si{C_g/cm^2}$。 转换常数为\footnote{2020 新国际单位标准\upref{SIunit}以前, 这个数值精确等于 $1\e{-4}$, 新标准需要乘以 $\sqrt{\mu_0/(4\e{-7}\pi)}$ 的国际单位数值。}
\begin{equation}
\beta_B = \frac{\beta_m\beta_x}{c_s\beta_q} = \frac{\beta_{\mathcal E}}{c_s} \approx 1.000000000272\e{-4} \Si{\frac{C_g\cdot kg \cdot s}{C \cdot g \cdot m}}
\end{equation}
其中 $c_{s}$ 是国际单位的光速。

\subsection{高斯单位制公式}
高斯单位制下的电磁学公式比其国际单位要更简洁对称。 以下 $c$ 为 CGS 单位下的光速\footnote{如果高斯单位直接建立在国际单位制上(\autoref{GaussU_eq3} 中的转换常数全部改为 $1$), 以下公式同样成立($c$ 也要换成国际单位制)。}。

\begin{equation}\label{GaussU_eq4}
\begin{aligned}
&\div \bvec E = 4\pi\rho\\
&\curl \bvec E = -\frac{1}{c}\pdv{\bvec B}{t}\\
&\div \bvec B = 0 \\
&\curl \bvec B = \frac{4\pi}{c} \bvec j + \frac{1}{c}\pdv{\bvec E}{t}
\end{aligned}
\quad\text{(麦克斯韦方程组)}
\end{equation}

\begin{equation}\label{GaussU_eq1}
F = \frac{q_1 q_2}{r^2} \qquad\text{(库仑定律)}
\end{equation}
\begin{equation}\label{GaussU_eq2}
\bvec F = q\bvec E + \frac{q}{c} \bvec v \cross \bvec B \qquad\text{(广义洛伦兹力)}
\end{equation}
\begin{equation}
\curl \bvec A = \bvec B \qquad\text{(磁矢势)}
\end{equation}
\begin{equation}
\bvec E = -\grad \varphi - \frac{1}{c}\pdv{\bvec A}{t} \qquad\text{(标量势)}
\end{equation}
\begin{equation}
\bvec A = \bvec A' + \grad \chi \qquad
\varphi = \varphi' - \frac{1}{c}\pdv{\chi}{t} \qquad \text{(规范变换)}
\end{equation}
\begin{equation}
\laplacian \bvec E - \frac{1}{c^2} \pdv[2]{\bvec E}{t} = 0 \qquad \text{(波动方程)}
\end{equation}
真空中的平面电磁波\upref{VcPlWv}若用高斯单位表示, 有 $E_0 = B_0$。
\begin{equation}
\rho_E = \frac{1}{8\pi} (\bvec E^2 + \bvec B^2) \qquad\text{(场能量密度)}
\end{equation}

\begin{equation}
\bvec s = \frac{c}{4\pi} \bvec E \cross \bvec B \qquad\text{(坡印廷矢量)} 
\end{equation}

\begin{equation}
\epsilon_0 = \frac{1}{4\pi} \qquad
\mu_0 = 4\pi
\end{equation}
转换常数为 $\beta_\epsilon = 4\pi\epsilon_0$, $\beta_\mu = \mu_0/(4\pi)$。
