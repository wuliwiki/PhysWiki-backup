% 完备域
% keys complete domain|可分多项式|reducible polynomial|微商|导数|形式微商|形式导数|多项式|重根|自同构|形式求导|形式微分

\pentry{分裂域\upref{SpltFd}}

\addTODO{待加入目录.}

由\autoref{SpltFd_the1}~\upref{SpltFd}可见,不可约多项式在其分裂域中有无重根,决定了该分裂域的自同构数量.一个域的全体自同构配合映射的复合,能构成一个群,域的许多性质都蕴含在这个群的结构中.群的元素数量自然是其重要性质之一.

综上所述,研究多项式的重根是非常重要的课题.

本节中如无特别声明或定义,多项式都是指\textbf{首一多项式},即最高次项系数为$1$.这是合理的简化:我们讨论的是域上的多项式,因此任何多项式总可以乘以最高次项系数的乘法逆元来得到首一多项式.

\subsection{可分多项式}

为了研究重根,我们借用微积分的知识,引入\textbf{形式微商}的概念

\begin{definition}{形式微商}\label{CmplD_def1}

设$\mathbb{F}$是一个域,$f\in\mathbb{F}[x]$.若$f$表达为
\begin{equation}
f(x) = \sum_{i=0}^n a_ix^i
\end{equation}
其中各$a_i\in\mathbb{F}$,那么定义\textbf{形式微商}算子$\opn{D}:\mathbb{F}[x]\to\mathbb{F}[x]$为:
\begin{equation}
\opn{D}f = \sum_{i=0}^{n-1} (i+1)a_{i+1}x^i = \sum_{i=1}^{n} ia_{i}x^{i+1}
\end{equation}

形式微商也可称为“形式求导”、“形式微分”等.

\end{definition}

\autoref{CmplD_def1} 形式微商就是直接套用微积分中的求导操作,只不过这里没有求导的概念,而是就进行多项式变换,其形式就是求导或者求导的推广,因此才叫\textbf{形式}微商.

\begin{example}{}
考虑域$\mathbb{Z}_3$上的多项式$f(x)=x^5+2x^2-x-2$,则
\begin{equation}
\opn{D}f(x) = 2x^4+x-1
\end{equation}

显然,这和真正的求导不同.一方面我们没有在域$\mathbb{Z}_3$上定义极限的概念,另一方面实数域上$f$的导函数应该是$5x^4+4x-1$.
\end{example}

\begin{example}{一个很抽象的例子}

考虑$\mathbb{R}$上的集合$S=\{\text{非零函数}\}\cup\{f\mid f(x)\equiv 0\}$,则$S$配上函数之间逐点相加和逐点相乘的运算,构成一个域,记为$\mathbb{S}$.

取$f, g\in\mathbb{S}$,构成多项式$F(y)=(f+g)y^2+(fg)y$.则
\begin{equation}
\opn{D}F(y) = (2f+2g)y+fg
\end{equation}

\end{example}

注意,$\opn{D}f(x)$这一表示应理解为“多项式$\opn{D}f$和表示其自变量的抽象符号$x$”,而不是“对多项式$f(x)$进行$\opn{D}$操作”.换言之,$\opn{D}f$是整体,故应是$\opn{D}f$和$(x)$,而不是$\opn{D}$和$f(x)$.

容易验证,形式微商有以下性质:

\begin{theorem}{形式微商的性质}\label{CmplD_the1}

给定域$\mathbb{F}$,$\opn{D}$是$\mathbb{F}[x]$上的形式微商算子.则:

1. 任取$a\in\mathbb{F}$,则作为多项式,$\opn{D}a=0$.

2. 当$\opn{ch}\mathbb{F}=0$\footnote{概念见\autoref{field_def2}~\upref{field}. }时,对于$f\in\mathbb{F}[x]$有:$\opn{D}f=0 \iff \opn{deg}f=0$.

3. 对于$f, g\in\mathbb{F}[x]$,$a, b\in\mathbb{F}$,有$\opn{D}(af+bg)=a\opn{D}f+b\opn{D}g$.

4. 对于$f, g\in\mathbb{F}[x]$,有$\opn{D}(fg) = (\opn{D}f)g+f\opn{D}g$.

\end{theorem}

用\autoref{CmplD_the1} 中的性质可知,如果$a$是$f\in\mathbb{F}[x]$在$\mathbb{F}$上的重根,那么存在$g\in\mathbb{F}[x]$使得$f(x)=(x-a)^2g(x)$.这样一来,$\opn{D}f(x)=(x-a)\qty[2g(x)+(x-1)\opn{D}g(x)]$.也就是说,$a$还是$\opn{D}f$的根.

\autoref{CmplD_the1} 中这些性质都是由导数的性质自然启发而得的.但要注意的是,函数的导数和多项式的形式微商在概念上有所重叠,却不是互相包含的.函数的导数可以用来处理非多项式的函数,而多项式的形式微商又可以处理非实数域的多项式,所以二者都有对方所不能处理的领域,勿随意混为一谈.\autoref{CmplD_the1} 中的性质能成立,也是需要额外验证的,而不能说因为导数有这些性质所以形式微商就一定有.

下面这个定理说的就是导数所不具备的性质:

\begin{theorem}{}\label{CmplD_the2}
设$\mathbb{K}$是$f\in\mathbb{F}[x]$的分裂域,$a\in\mathbb{K}$是$f$的一个$k$重根.

1. 如果$\opn{ch}\mathbb{F}\nmid k$,那么$a$是$\opn{D}f$的$k-1$重根;

2. 如果$\opn{ch}\mathbb{F}\mid k$,那么$a$是$\opn{D}f$的\textbf{至少}$k$重根.
\end{theorem}

\textbf{证明}:

由题设可知,存在$g(x)\in\mathbb{K}[x]$使得$f(x) = (x-a)^k g(x)$,其中$g(a)\neq 0$.

1. 

\begin{equation}
\opn{D}f(x)=(x-a)^{k-1}\qty[kg(x)+(x-a)\opn{D}g(x)]
\end{equation}

显然,$[kg(x)+(x-a)\opn{D}g(x)]\mid_{x=a}\neq 0$.因此,$a$是$\opn{D}f$的$k-1$重根.


2. 

当$\opn{ch}\mathbb{F}\mid k$,则$k=0$.因此

\begin{equation}
\begin{aligned}
\opn{D}f(x)&=(x-a)^{k-1}\qty[kg(x)+(x-a)\opn{D}g(x)]\\
&=(x-a)^k\opn{D}g(x)
\end{aligned}
\end{equation}

故$a$至少是其$k$重根.如果$\opn{D}g(a)=0$,重数还会更高.

\textbf{证毕}.



\begin{corollary}{}\label{CmplD_cor1}
设$\mathbb{K}$是次数大于$1$的多项式$f\in\mathbb{F}[x]$的分裂域,则$f$在$\mathbb{K}$中无重根的充分必要条件是$(f, \opn{D}f)=1$\footnote{即最大公因子为$1$.}.
\end{corollary}

\textbf{证明}:

必要性:

由于$f$在$\mathbb{K}$中无重根,而域的特征不可能是$1$,因此适用\autoref{CmplD_the2} 中第一个情况.

因为$\mathbb{K}$是$f\in\mathbb{F}[x]$的分裂域,且$f$无重根,故在$\mathbb{K}$中有
\begin{equation}
f(x) = a_0(x-a_1)(x-a_2)\cdots(x-a_n)
\end{equation}
其中各$a_i\in\mathbb{K}$且各不相同.

则
\begin{equation}
\begin{aligned}
\opn{D}f(x) &= a_0[(x-a_2)(x-a_3)\cdots(x-a_n)\\
&+(x-a_1)(x-a_3)\cdots(x-a_n))\\
&\vdots\\
&+(x-a_1)(x-a_2)\cdots(x-a_{n-1}))
]
\end{aligned}
\end{equation}

$f$的因子必是若干个不同的$(x-a_i)$的积,但各$(x-a_i)$都不是$\opn{D}f$的因子.

充分性:

反设有重根,那至少是个二重根.于是$f$可以写成:

\begin{equation}
f(x) = a_0(x-a_1)(x-a_2)\cdots(x-a_n)
\end{equation}
其中各$a_i\in\mathbb{K}$,且$a_1=a_2$.

那么$(x-a_1)=(x-a_2)$就是$f$和$\opn{D}f$的公因子.

\textbf{证毕}.



\begin{corollary}{}\label{CmplD_cor2}
设$\mathbb{K}$是$f\in\mathbb{F}[x]$的分裂域,且$f$是$\mathbb{F}[x]$上次数大于$1$的\textbf{不可约}多项式,那么$f$在$\mathbb{K}$中无重根的充分必要条件是$\opn{D}f \neq 0$.
\end{corollary}

\textbf{证明}:

注意\autoref{CmplD_cor1} 和\autoref{CmplD_cor2} 的题设差异在于,后者多了“在$\mathbb{F}[x]$上不可约”这一要求.我们只需要证明加上这一要求时,$(f, \opn{D}f)=1 \iff \opn{D}f\neq 0$即可.

$\Rightarrow$:

如果$\opn{D}f=0$,则$(f, \opn{D}f)=f\neq 1$.

$\Leftarrow$:

设$g(x)=(f(x), \opn{D}f(x))\in\mathbb{F}[x]$,于是$g\mid f$.由于$f$不可约,故要么$g=1$,要么$g\propto f$\footnote{这里$\propto$是“正比”符号,意为$g$与$f$只相差$a$倍,其中$a\in\mathbb{F}$.但是考虑到本节开头的声明,无特别定义,则$g$也是首一多项式,这里实际上也可以写成$g=f$.}.

但$g\mid \opn{D}f$导致$\opn{deg}g\leq\opn{deg}\opn{D}f<\opn{deg}f$,因此必有$g(x)=1$.



\textbf{证毕}.

对于次数大于$1$的多项式$f$,其形式微商可以为$0$.比如,$\mathbb{Z}_3$上的多项式$x^3+1$就是这样.





\begin{example}{}

考虑这样一个域$\mathbb{F}$,其定义为:$\mathbb{F}$是全体以$t$为不定元(或称抽象符号)、以$\mathbb{Z}_2$为系数域的有理式\footnote{即要么是$0$,要么是两个非零多项式的比,如$(x^4+3)/(x-2)$.}.

\end{example}




















