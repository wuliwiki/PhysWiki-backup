% Linux 操作系统简介

\begin{issues}
\issueDraft
\end{issues}

Linux 系统简介
Linux 与 Unix 一样,最初是一个控制行系统,即不使用,没有窗口和按钮,只能用键盘打入命令进行一切操作(如图)。

\begin{figure}[ht]
\centering
\includegraphics[width=14cm]{./figures/e6089054ba32242e.png}
\caption{请添加图片描述} \label{fig_Linux0_1}
\end{figure}

Linux 有不同的发行版,不同发行版的特点不尽相同,例如著名的安卓手机系统就是 Linux 的一个发行版。计算机上比较常用的是 Ubuntu,目前最新的是 Ubuntu 16,官网见 https://www.ubuntu.com/。

Ubuntu 桌面版带有图形界面,与 Windows 系统的操作大同小异(如图),而 Ubuntu Server (服务器版)只有控制行,适合在服务器上使用。

\begin{figure}[ht]
\centering
\includegraphics[width=14cm]{./figures/3bc7da15f0c07d71.png}
\caption{请添加图片描述} \label{fig_Linux0_2}
\end{figure}

Linux 系统在图形界面上的确比 Windows 要逊色,所以即使在带有图形界面的 Ubuntu 桌面版,大家也经常会调出控制行(Terminal)。 Linux 系统真正强大的地方是稳定,常有人评论道 “Linux 就像一块砖”,因为它可以成年成年地运行而不出故障也不需要重启。也正是这个原因,这个系统非常适合用来做服务器,在上面搭建网站,VPN,等等。

远程控制行
远程控制行叫做 ssh (Secured Shell),是 Linux 服务器的一个重要功能 VPN 里面一般都是自动开启,只需要在另一个设备上安装一个客户端即可使用远程控制行。

Windows 系统中,常用的客户端是 putty(未完成)