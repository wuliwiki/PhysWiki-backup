% 仿射群
% 仿射群|正合列

\pentry{仿射空间\upref{AfSp}}
\textbf{仿射群}是在仿射空间利用仿射映射(\autoref{AfSp_def2}~\upref{AfSp})构建的一种群\upref{Group}。让我们进行如下思考:群首先要满足封闭性,即两个群元作用(或运算)得到的还是一个群元,对仿射群而言,群元是仿射映射,映射的运算很自然的用映射的复合(\autoref{map_sub2}~\upref{map})表示。那么封闭性要求对两仿射映射 $f,g$,$fg$还是仿射映射。映射的复合要求 $f$ 的定义域起码得包含 $g$ 的值域。记
 \begin{equation}
f:\mathbb A\rightarrow\mathbb A_1,\quad g:\mathbb A_2\rightarrow \mathbb A_3~,
 \end{equation}
 则 $\mathbb A_3\subset \mathbb A$。而任意两个群元都可以进行运算的,即还有 $gf$ 也是仿射映射,同理,这意味着 $\mathbb A_1\subset\mathbb A_2$。即
 \begin{equation}\label{AfQ_eq1}
 \mathbb A_3\subset \mathbb A,\quad \mathbb A_1\subset\mathbb A_2~.
 \end{equation}
 
 其次,群元必定得有逆元,对映射而言,就是逆映射得存在,这表明
\begin{equation}
 f^{-1}:\mathbb A_1\rightarrow\mathbb A,\quad g^{-1}:\mathbb A_3\rightarrow\mathbb A_2
\end{equation}
 存在,和前面一样,又有
 \begin{equation}\label{AfQ_eq2}
 \mathbb A_2\subset \mathbb A_1, \mathbb A\subset\mathbb A_3~.
 \end{equation}
 \autoref{AfQ_eq1} ,\autoref{AfQ_eq2} 联立,就有
 \begin{equation}
 \mathbb A=\mathbb A_1= \mathbb A_2=\mathbb A_3~.
 \end{equation}
 也就是说,仿射群是由仿射空间 $(\mathbb A,V)$ 上的所有自同构 $f:\mathbb A\rightarrow\mathbb A$ 实现的。由仿射映射的定义:
 \begin{equation}\label{AfQ_eq3}
 f(\dot p+v)=f(\dot p)+Df\cdot v~.
 \end{equation}
 显然,这里 $Df$ 是 $V\rightarrow V$ 上的可逆的线性映射(\autoref{AfSp_the1}~\upref{AfSp})。这意味着,$Df$ 是一个可逆的线性算子(\autoref{LiOper_sub4}~\upref{LiOper}),可记为 $\mathcal F=Df$。于是\autoref{AfQ_eq3} 变成
 \begin{equation}
 f(\dot p+v)=f(\dot p)+\mathcal F v~.
 \end{equation}
 \subsection{仿射群}
 \begin{definition}{仿射群}
 设 $n$ 维仿射空间 $(\mathbb A,V)$ 定义在域 $\mathbb F$ 上,则所有仿射自同构配上映射复合构成的集合 $\mathrm{Aff}(\mathbb A)=A_n(\mathbb F)$ 称为仿射空间 $\mathbb A$ 上的 $n$ 维\textbf{仿射群}。
 \end{definition}
 用 $e$ 表示仿射群的单位元,它是\textbf{单位仿射变换}(或\textbf{恒等变换}),其线性部分为 $V$ 上的单位算子 $\mathcal E$。

容易证明: $e$ 是单位仿射变换,当且仅当存在一点 $\dot q$ ,使得 $e(\dot o)=\dot o$,且 $e$ 的线性部分为 $\mathcal E$。事实上
 \begin{equation}
  e(\dot p)=e(\dot q+\overrightarrow{qp})=e(\dot q)+\mathcal E \overrightarrow{qp}=\dot q+\overrightarrow{qp}=\dot p~.
 \end{equation}
反过来,$e$ 是单位仿射变换,则 $\forall\dot q$,都有 $e(\dot q)=\dot q$。于是
\begin{equation}
\begin{aligned}
e(\dot p+v)=e(\dot p)+De\cdot v&=\dot p+De\cdot v=\dot p+v\\
&\Downarrow\\
De&=\mathcal E~,
\end{aligned}
\end{equation}
故证得结论。
\begin{example}{}
若 $f,g$ 是线性部分分别为$\mathcal F,\mathcal G$ 的两个仿射自同构,试证明,$fg$ 的线性部分为 $\mathcal {F,G}$.

\textbf{证明}:\begin{equation}
(fg)(\dot p+v)=f(g(\dot p+v))+f(g(\dot p)+\mathcal G v)=fg(\dot p)+\mathcal {FG}v~.
\end{equation}
\textbf{证毕!}
\end{example}
\begin{theorem}{}
所有保持点 $\dot o$ 不动的仿射自同构构成的集合是一个子群 $A_n(\mathbb F)_{\dot o}\in A_n(\mathbb F)$,且在 
\begin{equation}
D:f\rightarrow Df,\quad \forall f\in A_n(\mathbb F)_{\dot o}~,
\end{equation}
下同构于完全线性群 $GL(V)=GL_n(\mathbb F)$(\autoref{Group_ex5}~\upref{Group})。空间 $\mathbb A$ 的所有平移构成的子群 $T=\{t_v|v\in V\}$ 在群 $A_n(\mathbb F)$ 中是正规的(\autoref{NormSG_def1}~\upref{NormSG}),并且属于映射 $D$ 的核。
\end{theorem}
\textbf{证明:}先证明定理第1部分:

\textbf{封闭性:} $\forall f,g\in A_n(\mathbb F)_{\dot o}$ ,设 $\mathcal F,\mathcal G$ 分别是 $f,g$ 的线性部分,于是 
\begin{equation}
(fg)(\dot o+x)=f(g(\dot o+x))=f(g(\dot o)+\mathcal G x)=\dot o+\mathcal{FG}x~,
\end{equation}
故 $fg\in A_n(\mathbb F)_{\dot o}$ 。

\textbf{可逆性:}显然 $f^{-1}(\dot o+x)=\dot o+\mathcal F^{-1}x$ 是 $f(\dot o+x)=\dot o+\mathcal F x$ 的逆。

\textbf{单位元,结合性}显然。

所以,$A_n(\mathbb F)_{\dot o}$ 是 $A_n(\mathbb F)$ 的一个子群。

由同构的定义(\upref{homomo}), $D:f\rightarrow Df=\mathcal F$ 显然是 $A_n(\mathbb F)_{\dot o}$ 到 $GL(V)$ 上的同构。

第2部分的证明:

由\autoref{AfSp_exe1}~\upref{AfSp},所有平移构成的子群 $T$ 同构于空间 $V$ 的加法群。
 
对 $f\in A_n(\mathbb F)$ 且 $Df=\mathcal F$,则
\begin{equation}
\begin{aligned}
(f^{-1}t_vf)(\dot p)&=(f^{-1}t_v)f(\dot p)=f^{-1}(f(\dot p)+v)\\
&=f^{-1}(f(\dot p))+\mathcal F^{-1}v=\dot p+\mathcal F^{-1}v\\
&=t_{\mathcal F^{-1}v}(\dot p)
\end{equation}
因为 $\dot p$ 的任意性,从而 $f^{-1}t_vf=t_{\mathcal F^{-1}v}$ 。由正规子群的定义(\autoref{NormSG_def1}~\upref{NormSG}),$T$ 显然是 $A_n(\mathbb F)$ 的正规子群。

由 $\ker D=\{f\in A_n(\mathbb F)|\mathcal F=\mathcal E\}$, $A_n(\mathbb F)$ 中的平移显然都在 $\ker D$ 中,且
\begin{equation}
f\in \ker D\Rightarrow f(\dot p+v)=f(\dot p)+v=\dot p+v+\overrightarrow{pf(p)}
\end{equation}
由于 $u=\overrightarrow{(\dot p+v)(f(\dot p)+v)}=\overrightarrow{pf(p)}$ 与点 $\dot p$ 无关(由 $v$ 的任意性可见),所以 $f=t_u$ 是个用矢量 $u$ 做的平移。

\textbf{证毕!}

所谓的\textbf{正合列}\footnote{实际上这里叫做正合性,正合列还要求 $\{M_i\}$ 是环 $R$ 的模} 是指这样一个序列:
\begin{equation}
\cdots\rightarrow M_{i-1}\xrightarrow{\alpha_i} M_{i}\xrightarrow{\alpha_{i+1}}M_{i+1}\rightarrow\cdots
\end{equation}
其中 $\mathrm{Im}\,\alpha_{i}=\ker\alpha_{i+1}$ ,$\{\alpha_i\}$ 是从 $M_{i-1}$ 到 $M_i$ 的同态映射(\autoref{Group2_def1}~\upref{Group2})。由此,容易验证
\begin{equation}
e\rightarrow T\xrightarrow{\varphi}A_n(\mathbb F)\xrightarrow{D} GL_n(\mathbb F)\rightarrow \overline{e}
\end{equation}
是个正合列,其中 $\varphi$ 是恒等映射,$\overline{e}$ 就是只含有一个单位元 $\overline{e}$ 的群。 