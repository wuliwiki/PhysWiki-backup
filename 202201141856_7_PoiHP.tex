% 庞加莱半平面(微分几何计算实例)
% keys 微分几何|联络|Poincare Half-plane|度量|黎曼联络|黎曼度量|Riemannian connection|Riemannian metric

庞加莱半平面是历史上非常重要的一个模型.众所周知,欧几里得几何学中有五条公理,其中第五条“过直线外一点有且仅有一条直线与已知直线垂直”非常冗长而且绕口,因此历史上一直有不少数学家致力于通过其它四条来推出第五条,也就是将第五公理变成一个定理.在GTM 275\cite{GTM275}中将这种尝试评价为“英雄式”的(heroic).这是出于早期数学家们的一种朴素的直觉,即几何就应该是欧几里得空间那样子的,所以第五公理必须成立,哪怕只是作为定理.后来的人们逐渐意识到第五公理并不能被前四条所证明,并逐渐发展出了符合前四条但违反第五条的几何学,也就是所谓的\textbf{非欧几何学}.庞加莱半平面就是一个典型的例子,在本节的\textbf{测地线}小节我们会简单讨论这一点.

本节的主要目的是以庞加莱半平面为例,演示如何进行具体的计算.

\subsection{庞加莱半平面的定义}

\begin{definition}{庞加莱半平面}
设 $\mathbb{H}^2=\{(x, y)\in \mathbb{R}^2|y>0\}$,即二维实平面的上半平面(不包含 $x$ 轴).在 $\mathbb{H}^2$ 上定义\textbf{黎曼度量}$<*,*>$ 为,对于任意点 $(x, y)\in \mathbb{H}^2$ 处的\textbf{切向量}$(a_i, b_i)\in T_{(x, y)}\mathbb{H}^2$,有
\begin{equation}
<(a_1, b_1), (a_2, b_2)>=\frac{a_1a_2+b_1b_2}{y^2}
\end{equation}
则 $\mathbb{H}^2$ 配合该度量所得到的\textbf{黎曼流形}称为\textbf{庞加莱半平面(Poincaré half-plane)}.
\end{definition}

对于庞加莱半平面度量的描述,更简洁的表达是:$\frac{\dd x\otimes \dd x+ \dd x\otimes\dd x}{y^2}$.这里的 $\dd x$ 是指 $\mathbb{H}^2$ 上的函数 $f(x, y)=x$ 的微分 $\dd f$,同样地 $\dd y$ 是 $g(x, y)=y$ 的微分 $\dd g$,它们都是 $\mathbb{H}^2$ 上的 $1$-形式,而 $\otimes$ 是它们的张量积.

如果用通常的 $(x, y)\in \mathbb{R}^2$ 作为坐标来描述 $\mathbb{H}^2$,那么在 $\mathbb{R}^2$ 中的两个点,在保持 $x$ 坐标不变时同步向 $y$ 的正方向移动,那么它们的距离会缩短,并在 $y$ 坐标趋于正无穷时距离趋于零.反过来,如果两个点的 $y$ 坐标始终相同,$x$ 坐标不变,那么它们同步趋近于 $x$ 轴时,彼此距离会趋近于正无穷.

\subsection{联络形式}

\pentry{联络形式与结构定理\upref{ConFom}}

本小节先计算 $\mathbb{H}^2$ 上的一个\textbf{黎曼}联络形式.

由\textbf{反对称定理}\autoref{ConFom_the1}~\upref{ConFom},计算联络形式时选标准正交基来进行计算最为方便,因为 $\mathbb{H}^2$ 是二维的,标准正交基下的联络形式矩阵是一个反对称的 $2$ 阶方阵,也就是说可以被一个分量唯一确定.

第一步,选择标准正交基:
\begin{equation}
\begin{aligned}
\uvec{e}_1=y\frac{\partial}{\partial x}\\
\uvec{e}_2=y\frac{\partial}{\partial y}
\end{aligned}
\end{equation}

第二步,计算对偶基:

由于 $\dd x\cdot \frac{\partial }{\partial x}=\dd y\cdot \frac{\partial}{\partial y}=1$,$\dd x\cdot \frac{\partial }{\partial y}=\dd y\cdot \frac{\partial}{\partial x}=0$,我们可以得到对应的对偶基:

\begin{equation}\label{PoiHP_eq1}
\begin{aligned}
\theta^1=\frac{1}{y}\dd x\\
\theta^2=\frac{1}{y}\dd y
\end{aligned}
\end{equation}

因此我们可以计算出\footnote{注意外微分的幂零性,即 $\mathrm{d}^2=0$.}:
\begin{equation}\label{PoiHP_eq2}
\begin{aligned}
\dd\theta^1&=-\frac{1}{y^2}\dd y\wedge\dd x=\frac{1}{y^2}\dd x\wedge \dd y\\
\dd\theta^2&=0
\end{aligned}
\end{equation}

第三步,\textbf{设挠率}为 $0$,应用\textbf{结构定理}(\autoref{ConFom_the2}~\upref{ConFom})和\textbf{反对称定理}(\autoref{ConFom_the1}~\upref{ConFom})得:

\begin{equation}\label{PoiHP_eq3}
\begin{aligned}
&\dd\theta^1=-\omega^1_2\wedge\theta^2\\
&\dd\theta^2=-\omega^2_1\wedge\theta^1=\omega^1_2\wedge\theta^1
\end{aligned}
\end{equation}



最后,联立\autoref{PoiHP_eq1} 、\autoref{PoiHP_eq2} 和\autoref{PoiHP_eq3} ,得到:

\begin{equation}
\begin{aligned}
-\omega^1_2\wedge\theta^2&=\theta^1\wedge\theta^2\\
\omega^1_2\wedge\theta^1&=0
\end{aligned}
\end{equation}

因此易得 $\omega^1_2=-\theta^1=-\frac{1}{y}\dd x$.

因此联络形式矩阵为
\begin{equation}
\pmat{
    0&-\frac{1}{y}\dd x\\
    \frac{1}{y}\dd x&0
}
\end{equation}

\subsection{高斯曲率}

\pentry{高斯绝妙定理\upref{Egreg}}

由\textbf{高斯绝妙定理}\upref{Egreg},如果要计算 $\mathbb{H}^2$ 的高斯曲率,我们就要去计算某个基下的曲率形式 $\Omega^i_j$,从而有 $K=\Omega^1_2(\uvec{e}_1, \uvec{e}_2)$.

由上一小节计算出来的联络形式,结合\textbf{结构定理}\autoref{ConFom_the2}~\upref{ConFom},注意 $\omega^1_1=\omega^2_2=0$,我们有:
\begin{equation}
\begin{aligned}
\Omega^1_2&=\dd \omega^1_2+\omega^1_k\wedge\omega^k_2\\
&=\dd(-\frac{1}{y}\dd x)+\omega^1_1\wedge\omega^1_2+\omega^1_2\wedge\omega^2_2\\
&=\frac{1}{y^2}\dd y\wedge\dd x\\
&=-\frac{1}{y^2}\dd x\wedge \dd y
\end{aligned}
\end{equation}

于是
\begin{equation}
\begin{aligned}
K&=\Omega^1_2(y\frac{\partial}{\partial x}, y\frac{\partial}{\partial y})\\
&=-\frac{1}{y^2}\dd x\wedge \dd y(y\frac{\partial}{\partial x}, y\frac{\partial}{\partial y})\\
&=-\dd x\wedge \dd y(\frac{\partial}{\partial x}, \frac{\partial}{\partial y})\\
&=-(\dd x\cdot \frac{\partial}{\partial x})(\dd y\cdot \frac{\partial}{\partial y})+(\dd y\cdot \frac{\partial}{\partial x})(\dd x\cdot \frac{\partial}{\partial y})\\
&=-1+0\\
&=-1
\end{aligned}
\end{equation}

因此庞加莱半平面的高斯曲率处处为 $-1$,是典型的罗巴切夫斯基几何.




\subsection{Christoffel符号}
\pentry{Christoffel符号\upref{CrstfS}}

我们计算 $\mathbb{H}^2$ 在通常的 $(x, y)\in\mathbb{R}^2$ 坐标下的Christoffel符号.为方便计,将导子 $\frac{\partial}{\partial_a}$ 简记为 $\partial_a$.注意对于任何只依赖于 $y$ 的函数 $f(y)$,有 $\uvec{e}_1f(y)=y\frac{\partial}{\partial x}f(y)=0$,因此 $\nabla_{\uvec{e}_1}f(y)X=f(y)\nabla_{\uvec{e}_1}X$.

\begin{equation}\label{PoiHP_eq5}
\begin{aligned}
\nabla_{\partial_x}\partial_x&=\nabla_{\frac{1}{y}\uvec{e}_1}\frac{1}{y}\uvec{e}_1\\
&=\frac{1}{y^2}\nabla_{\uvec{e}_1}\uvec{e}_1\\
&=\frac{1}{y^2}\omega^k_1(\uvec{e}_1)\uvec{e}_k\\
&=\frac{1}{y^2}\omega^2_1(\uvec{e}_1)\uvec{e}_2\\
&=\frac{1}{y^2}(\frac{1}{y}\dd x\cdot y\partial_x)\uvec{e}_2\\
&=\frac{1}{y^2}\uvec{e}_2\\
&=\frac{1}{y}\partial_y
\end{aligned}
\end{equation}

类似地,可以计算出
\begin{equation}
\nabla_{\partial_x}\partial_y=-\frac{1}{y}\partial_x
\end{equation}
和
\begin{equation}\label{PoiHP_eq4}
\nabla_{\partial_y}\partial_y=-\frac{1}{y}\partial_y
\end{equation}

注意,在计算\autoref{PoiHP_eq4} 的时候会多出一项 $\frac{1}{y}(\uvec{e}_2\frac{1}{y})\uvec{e}_2$,这是因为 $\uvec{e}_2\frac{1}{y}$ 不为零,因此不能像\autoref{PoiHP_eq5} 一样直接略去.

最后,由于我们假设挠率为零,故有 $\Gamma^k_{ij}=\Gamma^k_{ji}$,因此无需重复计算 $\nabla_{\partial_y}\partial_x$.

由Christoffel符号的定义,$\nabla_{\partial_i}\partial_j=\Gamma^k_{ij}\partial_k$,就得到了Christoffel符号的每一个分量,列举如\autoref{PoiHP_fig1} 中的表格:

\begin{figure}[ht]
\centering
\includegraphics[width=5cm]{./figures/PoiHP_1.pdf}
\caption{庞加莱半平面在通常的 $\mathbb{R}^2$ 坐标系中的Christoffel符号} \label{PoiHP_fig1}
\end{figure}

% \begin{table}[ht]
% \centering
% \caption{庞加莱半平面在通常的 $\mathbb{R}^2$ 坐标系中的Christoffel符号}\label{PoiHP_tab1}
% \begin{tabular}{|l|c|c|c|}
% \hline
%  & 1 & 2 \\
% \hline
% 11 & 0 & $\frac{1}{y}$ \\
% \hline
% 12 & $-\frac{1}{y}$ & 0 \\
% \hline
% 21 & $-\frac{1}{y}$ & 0 \\
% \hline
% 22 & 0 & $-\frac{1}{y}$ \\
% \hline
% \end{tabular}
% \end{table}

\subsection{测地线}

有了Christoffel符号,我们就可以列出测地线方程了.

根据\autoref{geodes_eq1}~\upref{geodes},誊抄如下:
\begin{equation}\label{PoiHP_eq6}
\ddot{y}^k+\dot{y}^i\dot{y}^j\Gamma^k_{ji}=0
\end{equation}
改用我们在庞加莱半平面上使用的坐标,\autoref{PoiHP_eq6} 化为
\begin{equation}\label{PoiHP_eq7}
\leftgroup{
    \ddot{x}+2\dot{x}\dot{y}\Gamma^{1}_{12}=0\\
    \ddot{y}+\dot{x}^2\Gamma^{2}_{11}+\dot{y}^2\Gamma^{2}_{22}=0
}
\end{equation}

将\autoref{PoiHP_fig1} 代入\autoref{PoiHP_eq7} ,得到:
\begin{equation}\label{PoiHP_eq9}
\leftgroup{
    \ddot{x}-2\dot{x}\dot{y}\frac{1}{y}=0\\
    \ddot{y}+\frac{1}{y}\dot{x}^2-\frac{1}{y}\dot{y}^2=0
}
\end{equation}

\autoref{PoiHP_eq9} 的第二式较为难解,为了简化解答,将其用测地线的另一个性质,匀速性(\autoref{geodes_the1}~\upref{geodes}),来代替:
\begin{equation}\label{PoiHP_eq8}
\frac{\dot{x}^2+\dot{y}^2}{y^2}=1
\end{equation}
注意,这里我们设定测地线的速度总是 $1$,这是因为任意非零测地线都可以重新参数化,将速度变成 $1$.

\autoref{PoiHP_eq8} 没有提供超出\autoref{PoiHP_eq9} 的信息,但是它可以简化解答.

\subsubsection{$\dot{x}$ 不恒为零的情况}

{}%\subsubsection 中插入公式会导致后面的代码也显示为粗体黄字,解决办法是,只要后面多出现一个右大括号就行.但是这样会出现语法错误,因此我加了不会影响正文的双大括号.

当 $\dot{x}\ne 0$ 时,我们可以用它去除以\autoref{PoiHP_eq9} 中第一条,以完成变量分离:
\begin{equation}\label{PoiHP_eq10}
\frac{\ddot{x}}{\dot{x}}=\frac{2\dot{y}}{y}
\end{equation}

从\autoref{PoiHP_eq10} 容易解得
\begin{equation}\label{PoiHP_eq11}
\dot{x}=Ky^2
\end{equation}
其中 $K$ 为一常数.

将\autoref{PoiHP_eq11} 代回\autoref{PoiHP_eq8} ,得到
\begin{equation}\label{PoiHP_eq12}
\dot{y}^2=y^2-K^2y^4
\end{equation}

考虑到 $\frac{\dot{y}}{\dot{x}}=\frac{\dd y}{\dd x}$,用\autoref{PoiHP_eq12} 除以\autoref{PoiHP_eq11} 的平方,得:
\begin{equation}\label{PoiHP_eq13}
\frac{\dd y}{\dd x}=\pm\frac{\sqrt{1-K^2y^2}}{Ky}
\end{equation}

对\autoref{PoiHP_eq13} 变量分离后积分即得解:
\begin{equation}\label{PoiHP_eq14}
x=\mp\frac{\sqrt{1-K^2y^2}}{K}+C
\end{equation}
其中 $C$ 也是常数.

\autoref{PoiHP_eq14} 也可以换种写法,即
\begin{equation}
(x-C)^2+(y)^2=\frac{1}{K^2}
\end{equation}
这是圆的方程,圆心都在 $x$ 轴上.

因此,当 $\dot{x}$ 不恒为零时候,庞加莱半平面上的测地线是以 $x$ 轴上的点为圆心的半圆.

我们可以从给定起点画垂直于初始方向的直线,其交点即为对应测地线的圆心.

\subsubsection{$\dot{x}$ 恒为零的情况}

{}

此时我们只有一个有效的方程,即\autoref{PoiHP_eq9} 的第二式,现在化为:
\begin{equation}
\ddot{y}=\frac{\dot{y}^2}{y}
\end{equation}

但我们根本不需要解这个方程.因为 $\dot{x}$ 恒为零时,测地线在所给坐标系中就是一条平行于 $y$ 轴的直线,为保证其速度恒为 $1$,只需要 $\dot{y}^2/y^2=1$ 即可.

另外,对于 $\dot{x}$ 不恒为零的情况,当半径越来越大时,圆形测地线轨迹就越来越接近一条竖直的直线.所以竖直直线解也可以看成圆形解的极限.






