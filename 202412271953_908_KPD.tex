% 阿瑟·康普顿(综述)
% license CCBYSA3
% type Wiki

本文根据 CC-BY-SA 协议转载翻译自维基百科\href{https://en.wikipedia.org/wiki/Arthur_Compton}{相关文章}。

\begin{figure}[ht]
\centering
\includegraphics[width=6cm]{./figures/2b1294fc8f46ffcf.png}
\caption{1927年的康普顿} \label{fig_KPD_1}
\end{figure}
阿瑟·霍利·康普顿(Arthur Holly Compton,1892年9月10日—1962年3月15日)是美国物理学家,他与C.T.R. 威尔逊共同获得了1927年诺贝尔物理学奖,表彰他发现了康普顿效应,该效应证明了电磁辐射的粒子性质。这一发现当时轰动一时:虽然光的波动性已经得到了充分证明,但光既具有波动性又具有粒子性质的观点并不容易被接受。他还因在曼哈顿计划中领导芝加哥大学冶金实验室而闻名,并在1945年至1953年期间担任圣路易斯华盛顿大学的校长。

1919年,康普顿获得了最早的两项国家研究委员会奖学金之一,允许学生到国外学习。他选择前往英国剑桥大学的卡文迪许实验室,在那里他研究了伽玛射线的散射和吸收。进一步的研究导致了康普顿效应的发现。他利用X射线研究铁磁性,并得出结论,铁磁性是电子自旋排列的结果;他还研究了宇宙射线,发现它们主要由带正电的粒子组成。

在第二次世界大战期间,康普顿是曼哈顿计划中的关键人物,该计划开发了第一批核武器。他的报告对启动该项目起到了重要作用。1942年,他成为执行委员会成员,随后成为“X”项目的负责人,监督冶金实验室,负责生产核反应堆以将铀转化为钚,寻找将钚从铀中分离出来的方法,并设计原子弹。康普顿监督了恩里科·费米(Enrico Fermi)创建的芝加哥堆-1(Chicago Pile-1),这是世界上第一个核反应堆,于1942年12月2日实现临界。冶金实验室还负责设计和操作位于田纳西州橡树岭的X-10石墨反应堆。1945年,汉福德现场的反应堆开始生产钚。

战后,康普顿成为圣路易斯华盛顿大学的校长。在他的领导下,大学正式废除了本科部的种族隔离政策,任命了第一位女性终身教授,并在战后退伍军人返回美国后迎来了创纪录的学生入学人数。
\subsection{早年生活}