% 欧几里得群
% license CCBYSA3
% type Wiki

(本文根据 CC-BY-SA 协议转载自原搜狗科学百科对英文维基百科的翻译)

在数学中,欧几里德群是欧几里德空间𝔼n的(欧几里德)等距群;也就是说,欧几里德群是保持欧几里德空间中任意两点欧几里德距离不变的变换群(也称为欧几里德变换)。 该群仅取决于空间的维度n,并且通常表示为 $E$(n 或ISO(n)。

欧几里德群E(n)由欧几里德空间𝔼n所有的平移变换、旋转变换、反射变换以及它们的任意有限组合构成。欧几里德群可以被看作是空间自身的对称群,并且它包含该空间的任何图(子集)的对称群。

欧几里德等距可以是直接的或间接的,这取决于它是否保留了图形的旋向性。直接的欧几里德等距形成一个子群——特殊的欧几里德群,其中的元素被称为刚体运动或者欧几里德运动。它们包括平移和旋转的任意组合,但不包括反射。

这些群是最古老和被研究最多的群之一,至少在维度为2和维度为3的情况下是如此一一含蓄地说,这些群早在群的概念发明之前就有了。

\subsection{综述}

\subsubsection{1.1 维度}

E(n)的自由度为n(n + 1)/2,在n = 2的情况下自由度等于3,n = 3的情况下自由度等于6。其中,n个自由度归因于可用的平移对称,其余n(n-1)/2个自由度归因于旋转对称。

\subsubsection{1.2 直接和间接等距}

直接等距(即保持旋向性子集的旋向性等距)包括一个E(n)子群,称为特殊欧几里德群,通常用 $E^+(n)$或 SE$(n)$表示。它们包括平移和旋转变换及其组合;也包括单位变换,但不包括任何类型的反射变换。

反向旋性等距变换被称为“间接的”。对于任何固定的间接等距变换R,例如关于某个超平面的反射变换,其他间接等距变换都可以通过R与其他直接等距变换的组合得到。因此,间接等距变换是 $E^+(n)$,的陪集,可以用 $E^-(n)$.来表示。由此可见,子群 $E^+(n$)在E$(n)$中的指数为2。

\subsubsection{1.3 群的拓扑}

欧几里得空间$E^n$的自然拓扑诱导了欧几里德群$E(n)$的一个拓扑。也就是说,$E_n$的等距变换序列$f_i$是收敛的当且仅当对$E^n$的任意点$\rho$,对应的序列$p_i$($p_i = f_i(p)$)是在$\mathbb{N}$中收敛的($i \in \mathbb{N}$)。

根据这一定义,可以得出如下结论:函数$f: [0,1] \rightarrow E(n)$是连续的,当且仅当,对于$E^n$内的任意点$\rho$,由$f_{\rho}(t) = (f(t))(\rho)$定义的函数$f_{\rho}: [0,1] \rightarrow E^n$是连续的。这种函数在$E(n)$中被称为“连续轨道”。

可以证明,特殊的几里德群$SE(n) = E^+(n)$在此拓扑中是连通的。也就是说,给定的任意两个$E^n$内的直接等距变换A和B,在$E(n)$中存在一个连续的轨道,使得$g_0 = A$和$g_1 = B$。间接等距变换$E^-(n)$亦是如此。另一方面,群$E(n)$作为一个整体是不连通的:因为不存在从$E^+(n)$开始到$E^-(n)$结束的连续轨迹。

$E(3)$中的连续轨迹在经典力学中有着重要的作用,因为它们描述了刚体在三维空间中随时间推移可能发生的物理运动。假设$A(0)$是$E^3$的单位变换1,描述了刚体在任意时间$t$后的位置和方向可以从变换$f(t)$来描述,因为$f(0) = 1$在$E(3)$中,在任意时间$t$后的$f(t)$是连续的。因此,直接欧几里得等距变换也被称为“刚体运动”。


\subsubsection{1.4 李结构}



\subsubsection{1.5 和仿射群的关系}



\subsection{详细论述}



\subsubsection{2.1 子群的结构,矩阵和向量表达}

欧几里德群是仿射变换群的一个子群。

它具有平移群$T(n)$和正交群$O(n)$作为子群。$E(n)$的任何元素都是一个平移变换复合一个正交变换(等距的线性部分),以下面这种独特的方式:

\begin{align}
x & \mapsto A(x + b)~
\end{align}

其中 $A$ 是一个正交矩阵或者同样的正交变换,$b$ 是一个平移

\begin{align}
x & \mapsto Ax + c,~
\end{align}

其中 $c = Ab$

$t(n)$ 是 $E(n)$ 的正规子群:对于任何平移 $t$ 和任何等距 $u$,

\begin{align}
u^{-1} tu~
\end{align}

是一个平移(可以说,这是一个通过 $u$ 作用于 $t$ 的位移;因此等效地,位移是等距线性部分作用于 $t$ 的结果)

综上所述,这些事实意味着 $E(n)$ 是由 $T(n)$ 扩展的 $O(n)$ 半直积,可以写为
\begin{align}
E(n) &= T(n) \rtimes O(n)~
\end{align}

换句话说,$O(n)$(自然方式)也是 $E(n)$ 和 $T(n)$ 的商群:

\begin{equation}
O(n) \sim E(n)/T(n)~
\end{equation}

现在 $SO(n)$ 这个特殊正交群是指数为2的 $O(n)$ 的子群。因此,$E(n)$ 具有子群 $E^{+}(n)$ ,指数也为2,通过直接等距作用。在这些情况下,A的行列式是1。

它们可以看做是先平移后旋转,而不是先平移然后在做种种反射,(在维数为2和维数为3的情况下,这些是在镜像线上或平面中的一般的反射,可以包括原点,或者是在3D中的旋转反射)。

这个关系一般写成:

\begin{equation}
SO(n) \sim E^+(n)/T(n)~
\end{equation}

或等价地:

\begin{equation}
E^{+}(n) = SO(n) \times T(n)~
\end{equation}



\subsubsection{2.2 子群}

$E(n)$的子群类型

\textbf{有限群}

这些变换总有一个不动点。 在3D中,每个点在包含关系下在每个方向上在有限群众有两个极大群:$0_h$ 和$l_h$。群 $l_h$ 在包括下一类别的群中甚至是最大的。

没有任意小平移、旋转或其组合的可数无限群

也就是说,对于每个点,等距下的像集在拓扑上是离散的(例如,对于 1 ≤ m ≤ n 由独立方向上的 m 个平移产生的群,并且很可能是有限点群)。 这包括格。 比这些更一般的例子是离散空间群。

具有任意小平移、旋转或其组合的可数无限群

也就是说,在这种情形,存在某些点在等距变换下像集的不是闭合的。例如,在一维空间中,由1和 √2生成的平移群,在二维空间中,由围绕原点旋转1弧度生成的旋转群。
满足存在某些点在等距变换下像集的不是闭合的不可数群

(例如,在2D中,在一个方向上的所有平移,以及在另一个方向上的平移了有理距离的所有平移)。

满足所有点在等距变换下的像集是闭合的不可数群

例如:

\begin{itemize}
\item 保持原点固定的所有直接等距,或者更一般地说,保持某些点固定的所有直接等距(在3D中被称为旋转群)
\item 保持原点固定的所有等距,或者更一般地说,保持某些点固定的所有等距(正交群)
\item 所有直接等距$E^+(n)$
\item 整个欧几里得群$E^(n)$
\item m维子空间中的上述例子的某一个群与正交$(n-m)$维空间中的离散等距群的组合
\item m维子空间中的上述例子的某一个群与正交$(n-m)$维空间中的上述例子的某一个群的组合
\end{itemize}

3D中的组合示例:

\begin{itemize}
\item 围绕一个固定轴的所有旋转
\item 围绕一个固定轴的所有旋转与通过轴的平面和/或垂直于轴的平面的反射的组合
\item 围绕一个固定轴的所有旋转与沿轴的离散平移或沿轴的所有等距的组合
\item 平面中的离散点群、带状群或壁纸群与垂直方向上的任何对称群的组合
\item 所有围绕某个轴的旋转和沿轴的比例平移的组合的等距; 通常,这与围绕同一轴的k次旋转等距相结合$(k \geq 1)$; 等距变换下点的像集是k次螺旋线;另外,可能存在围绕垂直相交的轴的2次旋转,因此存在这些轴的$k$次螺旋。
\item 对于任何点群:所有等距群,它是点群中的等距和平移的组合; 例如,由在原点反向生成的群:它是所有平移的群和所有点的反向生成的群; 这就是$R^3$的广义二面群,$Dih(R^3)$。
\end{itemize}

\subsubsection{2.3 维数小于等于3的等距的概述}



\subsubsection{2.4 交换等距}

某些等距组合不依赖与顺序:

\begin{itemize}
\item 绕同一轴的两个旋转或者螺旋
\item 相对于平面的反射,以及该平面中的平移,围绕垂直于平面的轴的旋转,或相对于垂直平面的反射
\item 相对于平面的平移反射,以及在该平面中的一个点上的反向平移以及保持该点固定的任何等距
\item 绕轴180°的旋转和在平面中通过该轴的反射
\item 绕轴180°旋转和绕垂直轴180°的旋转(结果是绕垂直两者的轴线旋转180度)
\item 相对于同一平面,绕同一轴的两个旋转反射
\item 相对于同一平面的两个滑动反射
\end{itemize}


\subsubsection{2.5 共轭类}

在给定方向上移动给定距离的平移形成一个共轭类,平移群就是所有给定距离共轭类的并集。

在1D中,所有反射都在同一类中。

在2D中,在任一方向上旋转相同角度的旋转都在同一类中。 具有相同距离平移的滑动反射在同一类中。

在3D中:

\begin{itemize}
\item 关于所有点的反向属于同一类。
\item 相同角度的旋转属于同一类。
\item 如果角度相同且平移距离相同,则围绕轴的旋转与沿该轴的平移的组合在同一类中。
\item 平面中的反射属于同一类
\item 平面中的反射与该平面中相同距离的平移的组合属于同一类。
\item 围绕轴且不等于180°相同角度的旋转与平面中垂直于该轴的反射的组合属于同一类。
\end{itemize}


