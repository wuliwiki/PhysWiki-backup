% 磁矢势
% keys 旋度|磁场|矢势

\begin{issues}
\issueDraft
\issueOther{已知电流如何求磁矢势? 见表萨法尔定律的旋度形式}
\end{issues}

\pentry{亥姆霍兹分解\upref{HelmTh}}

由于磁场 $\bvec B(\bvec r)$ 任何情况都是一个无旋场\upref{MagGau}, 所以根据 “旋度的逆运算\upref{HlmPr2}” 的\autoref{HlmPr2_the1}, 必定存在一个矢量场 $\bvec A(\bvec r)$ 使得
\begin{equation}\label{BvecA_eq2}
\curl \bvec A = \bvec B
\end{equation}
且 $\bvec A$ 可以通过下式计算
\begin{equation}\label{BvecA_eq1}
\bvec A(\bvec r) = \frac{1}{4\pi} \int \bvec B(\bvec r') \cross \frac{\bvec R}{R^3} \dd{V'} + \bvec H(\bvec r)
\end{equation}
其中 $\bvec r, \bvec r'$ 分别是坐标原点指向三维直角坐标 $(x, y, z)$ 和 $(x', y', z')$ 的位置矢量, $\bvec R = \bvec r' - \bvec r$, $R = \abs{\bvec R}$, 体积分 $\int\dd{V'} = \int\dd{x'}\dd{y'}\dd{z'}$ 的区域是空间中 $\bvec B$ 不为零的区域, $\cross$ 表示矢量叉乘\upref{Cross}, $\bvec H(\bvec r)$ 是一个任意无旋场.

若已知恒定电流分布如何求空间某点的磁矢势呢? 当然我们可以先用比奥萨伐尔定律\upref{BioSav} 求出磁场分布再用\autoref{BvecA_eq1} 求出磁矢势, 但也而已直接求出, 使用比奥萨法尔定律的旋度形式(\autoref{BioSav_eq5}~\upref{BioSav})
\begin{equation}
\bvec B(\bvec r) = \frac{\mu_0}{4\pi} \curl \int \frac{\bvec j(\bvec r')}{\abs{\bvec r - \bvec r'}}\dd{V'}
\end{equation}
对比\autoref{BvecA_eq2} 可得
\begin{equation}\label{BvecA_eq3}
\bvec A(\bvec r) = \frac{\mu_0}{4\pi} \int \frac{\bvec j(\bvec r')}{\abs{\bvec r - \bvec r'}}\dd{V'} + \bvec F_{d}(\bvec r)
\end{equation}
其中 $\bvec F_d$ 是一个任意无旋场. 这是因为, 两个旋度相同的场只可能相差一个无旋场. 无旋场也可以记为任意函数的梯度 $\bvec F_d = \grad \varphi$.

可以证明静电学条件下\autoref{BvecA_eq3} 右边第一项是一个无散场, 对第一项的积分求g梯度
\begin{equation}
\begin{aligned}
&\div \int \frac{\bvec j(\bvec r')}{\abs{\bvec r - \bvec r'}}\dd{V'} = 
\int \qty(\grad\frac{1}{\abs{\bvec r - \bvec r'}}) \bvec j(\bvec r') \dd{V'}\\
&= -\int \qty(\grad' \frac{1}{\abs{\bvec r - \bvec r'}}) \bvec j(\bvec r') \dd{V'}
\end{aligned}
\end{equation}

使用多维分部积分\autoref{IntBP2_eq1}~\upref{IntBP2}, 令 $f(\bvec r') = 1/\abs{\bvec r - \bvec r'}$, $\bvec A(\bvec r') = \bvec j(\bvec r')$. 面积分取无穷大的球面, 积分为零; 最后一项中由于 $\grad' \vdot \bvec j(\bvec r') = 0$, 积分同样为零. 证毕.

\subsection{规范}
(详见 “规范变换\upref{Gauge}”)由于 $\bvec A(\bvec r)$ 不止一种, 我们有时候需要某种\textbf{规范(gauge)}来将其唯一确定下来. 例如在\textbf{库伦规范(Coulomb Gauge)}中, 我们要求
\begin{equation}
\div \bvec A = \bvec 0
\end{equation}
根据\autoref{HlmPr2_eq4}~\upref{HlmPr2}, 我们只需要令 $\bvec H(\bvec r)$ 是一个调和场即可, 事实上库伦规范直接规定 $\bvec H(\bvec r) = 0$.
