% 东南大学 2009 年 考研 量子力学
% license Usr
% type Note

\textbf{声明}:“该内容来源于网络公开资料,不保证真实性,如有侵权请联系管理员”

\textbf{1.(15 分)}以下叙述是否正确:(1)在定态下,任意不是含的力学量的平均值均不随时间变化:(2)若厄密算符与对易,则它们必有共同本征态:(3)一维谐振子的所有能级均是非简并的:(4)厄密算符的本征值必为正数(5)时间反演对称性导致能量守恒

\textbf{2.(15 分)}质量为 $m$ 的粒子作一维运动,几率守恒定理为
\[
\partial \rho/\partial t + \partial j/\partial x = 0,~
\]
其中,$\rho(x,t) = |\psi|^2$, $j(x,t) = -(i\hbar/2m)(\psi^*\partial \psi/\partial x - \psi \partial \psi^*/\partial x)$。

\begin{enumerate}
    \item 若粒子处于定态 $\psi = \phi(x) \exp(-iEt/\hbar)$,试证 $j = c$(与 $z,t$ 无关的常数);
    \item 若自由粒子处于动量本征态 $\psi(x,t) = \exp(ipx/\hbar - iEt/\hbar)$,试证 $j = p/m$。
\end{enumerate}

\textbf{3.(15 分)}试在坐标表象中写出:

\begin{enumerate}
    \item 位置算符 $\hat{x}$ 的本征函数;
    \item 动量算符 $\hat{p}_z$ 的本征函数;
    \item $\{\hat{x},\hat{y}, \hat{p}_z\}$ 的共同本征函数。
\end{enumerate}

\textbf{4.(15 分)}已知 $[\hat{A}, \hat{B}] = \hat{A}\hat{B} - \hat{B}\hat{A}$,$[\hat{A}, \hat{B}]_+ = \hat{A}\hat{B} + \hat{B}\hat{A}$,验证:

\begin{enumerate}
    \item $[\hat{A}\hat{B},\hat{C}] = \hat{A}[\hat{B}, \hat{C}]_{+} - [\hat{A}, \hat{C}]_{+}\hat{B}$;
    \item $[\hat{A}, \hat{B}\hat{C}] = [\hat{A}, \hat{B}]\hat{C} - \hat{B}[\hat{A}, \hat{C}]$。
\end{enumerate}

\textbf{5.(15 分)}假设体系有两个彼此不对易的守恒量 $F$ 和 $G$,即 $[F, H] = 0$,$[G, H] = 0$,$[F, G] \neq 0$。试证明该体系至少有一条能级是简并的。

\textbf{6.(15 分)}设粒子的波函数为 $\psi(0, \varphi) = aY_{11}(0, \varphi) + bY_{20}(0, \varphi)$ ($|a|^2 + |b|^2 = 1$),试求:
\begin{enumerate}
    \item  $\hat{L}_z$ 的可能测量值及平均值;
    \item  $\hat{L}^2$ 的可能测量值及相应的几率。
\end{enumerate}

\textbf{7.(15 分)}质量为 $m$ 的粒子以能量 $E > 0$ 从左入射,碰到势 $V(x) = \gamma \delta(x)$($\gamma > 0$)。

\begin{enumerate}
    \item 求入射几率流密度 $j_i$,反射几率流密度 $j_r$,透射几率流密度 $j_t$ 的表达式;
    \item 试证波函数 $\psi$ 满足 $\psi'(0^+) - \psi'(0^-) = (2m\gamma/\hbar^2)\psi(0)$;
    \item 求透射系数 $t$。
\end{enumerate}

\textbf{8.(15 分)}设体系由 2 个全同粒子组成,每个粒子可处于 2 个单粒子态 $\psi_i(r)$, $\psi_j(r)$ 中的任何一个,分别以下两种情况写出体系可能的波函数:

\begin{enumerate}
    \item 全同 Bose 子;
    \item 全同 Fermi 子。
\end{enumerate}

\textbf{9.(15 分)}两个电子的总角动量为 $\mathbf{S} = \mathbf{s}_1 + \mathbf{s}_2$,定义 $\hat{P} = (1 + \hat{\mathbf{s}}_1 \cdot \hat{\mathbf{s}}_2)/2$,试求:

\begin{enumerate}
    \item  $\hat{P}^2$;
    \item  $\hat{P} - \mathbf{S}^2/\hbar^2$;
    \item  $\hat{P}|SM\rangle$,其中 $|SM\rangle$ 为 $\mathbf{S}^2$ 和 $\hat{S}_z$ 的共同本征态。
\end{enumerate}

\textbf{10.(15 分)}体系未微扰哈密顿为 $\hat{H}_0$,微扰哈密顿 $\hat{H}'(t) = \hat{A}e^{-|t|/\tau}(\tau > 0)$,$\hat{H}_0|n\rangle = E_n |n\rangle$,$\langle n|n'\rangle = \delta_{nn'}$,$\sum_n |n\rangle \langle n| = 1$。已知 $t = -\infty$ 时体系处在 $\hat{H}_0$ 的非简并本征态 $|k\rangle$,即 $\psi(-\infty) = |k\rangle$。试利用一级近似下量子跃迁的几率幅公式
\[
C_{nk}^{(1)}(t) = \frac{1}{i\hbar} \int_{-\infty}^{t} dt' H'_{nk}(t') e^{i\omega_{nk}t'} \quad , \quad (\hbar\omega_{nk} = E_n - E_k, n \neq k)~
\]
\begin{enumerate}
    \item  求 $t = 0$ 时刻体系跃迁到本征态 $|n\rangle (n \neq k)$ 的几率幅 $C_{nk}^{(1)}(0)$;
    \item  求 $t = +\infty$ 时刻体系跃迁到本征态 $|n\rangle (n \neq k)$ 的几率幅 $C_{nk}^{(1)}(+\infty)$;
    \item  求 $t = +\infty$ 时刻的态 $\psi(+\infty)$(在此小题中取 $\tau \to \infty$)。
\end{enumerate}