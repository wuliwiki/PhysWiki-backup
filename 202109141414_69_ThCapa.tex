% 热容量
% 热容|比热容|等压热容|等体热容|理想气体

\begin{issues}
\issueTODO
\end{issues}
\pentry{热力学第一定律\upref{Th1Law}}
\subsection{热容}
一个系统在一定条件下的\textbf{热容量(heat capacity)}定义为\footnote{这个定义可以类比电容量\upref{Cpctor}}
\begin{equation}
C = \lim\limits_{\Delta T\rightarrow 0}\frac{\Delta Q}{\Delta T}
\end{equation}
热容可能跟温度压强等有关.

定义\textbf{比热容(specific heat capacity)}为热容除以质量
\begin{equation}
c = \frac{C}{m}
\end{equation}

定义摩尔热容为 $1 \rm{mol}$ 物质的热容
\begin{equation}
C_m=\frac{C}{n}
\end{equation}

\begin{example}{}
两份水初始温度分别为 $300\rm{K}$ 和 $360\rm{K}$,体积分别为 $1\rm{L}$ 或 $2 \rm{L}$.将它们放入绝热容器种混合均匀,求末温度.(注:水的比热容 $c$ 随温度的变化不大,可以近似看成一个常数)

设末温度为 $T$,那么第一份水吸收的热量为 $c m_1(T-300\rm{K})$,第二份水放出的热量为 $cm_2(360{\rm{K}}-T)$.由于在绝热容器中混合,且 $m_2=2m_1$,可以解得 $T=340\rm{K}$
\end{example}

我们可以定义等体热容为系统在等体过程\upref{EqVol}中的热容.根据热力学第一定律\upref{Th1Law} ,$\dd U=\delta Q-p\dd V$,可知等体过程中 $\delta Q=\dd U$(这里定义了态函数\textbf{焓} $H=U+pV$),所以
\begin{equation}\label{ThCapa_eq1}
C_V=\left(\frac{\partial U}{\partial T}\right)_V
\end{equation}

类似地,定义等压热容为系统在等压过程\upref{EqPre}中的热容.在等压过程中 $\delta Q=\dd U+p\dd V=\dd (U+pV)=\dd H$,所以
\begin{equation}
C_p=\left(\frac{\partial H}{\partial T}\right)_p
\end{equation}

\subsection{理想气体的等压热热容与等体热容}

根据理想气体的状态方程\upref{PVnRT},对于一定物质的量($n \rm{mol}$)的理想气体,内能 $U$ 只和温度有关,所以 \autoref{ThCapa_eq1} 的偏导数可以写为导数,即
\begin{equation}
C_V=\frac{\dd U}{\dd T}
\end{equation}

\subsubsection{等体热容的推导}
(未完成)
\begin{equation}
C_V = \dv{Q}{T} = \dv{E}{T} = \frac{i}{2} nR
\end{equation}

\subsubsection{等压热容的推导}
(未完成)
\begin{equation}
C_V = \dv{Q}{T} = \dv{E}{T} + \dv{W}{T} = \frac{i}{2} nR + nR = \frac{i+2}{2} nR
\end{equation}

