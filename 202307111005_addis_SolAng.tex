% 立体角
% 立体角|单位球

\begin{issues}
\issueTODO
\end{issues}

\pentry{重积分\upref{IntN}}

\footnote{参考 Wikipedia \href{https://en.wikipedia.org/wiki/Solid_angle}{相关页面}。}如果我们以某种锥体(例如圆锥, 三棱锥, 假设其无限高)的顶点作为圆心作一个半径为 1 的球(\textbf{单位球}), 那么这个锥体的\textbf{立体角(solid angle)}就是单位球的表面被锥体截出的面积, 通常用 $\Omega$ 表示。

我们知道半径为 $R$ 的球体的表面积为 $4\pi R^2$, 所以立体角的取值范围是 $[0, 4\pi]$。

\subsection{对立体角积分}
考虑 $\dd \theta$ 与 $\dd \phi$ 围成的立体角。在半径为 $R$ 上的球面上截出的面积为 $\sin\theta \dd \theta \dd \phi$。所以对对立体角积分的公式为
\begin{equation}
\int \dd \Omega=\int_0^\pi \sin\theta \dd \theta \int_0^{2\pi }\dd \phi =4\pi
\end{equation}
由此得到了三维球体的表面对应的立体角为 $4\pi$。


\begin{example}{圆锥的立体角}
顶角为 $2\theta$ 的圆锥在单位球面上可截出一个球盖。 在\autoref{ex_NLeib_3}~\upref{NLeib} 中我们知道球盖的面积, 所以该圆锥的立体角为
\begin{equation}
\Omega = 2\pi (1 - \cos\theta)~.
\end{equation}
\end{example}

\addTODO{例如一个质量分布不均匀的球壳质量}
