% 激光原理
% keys 光学|现代光学|原子物理

\begin{issues}
\issueDraft
\issueTODO
\end{issues}

激光器是现代光学的伟大成就之一,其拥有的窄频宽、单模式等高相干性的优势是许多近代实验,如迈克尔逊-莫雷的实验,成功的必要因素之一。激光器的核心原理是受激辐射。

\subsection{能级和光量子的概念}
量子力学告诉我们,在原子中的电子的能量并不是连续的。电子的能量总是一个又一个特定的能级中跳变。例如氢原子的能级分布就如下图:\begin{figure}[ht]
\centering
\includegraphics[width=7cm]{./figures/97c81e92cc27a211.png}
\caption{氢原子能级} \label{fig_LaserT_1}
\end{figure}
可以有很多中方法来让电子的能量发生跳变,例如使用高速电子轰击原子,但最常见的方法是使用光子。光子的能量为$E=h\nu$,其中$h$为普朗克常数,$\nu$为光子的频率。例如要让氢原子的电子从能级n=1跃迁到n=2,就需要一个频率为2.46PHz的光子,对应的波长是122nm,属于远紫外光,其他频率的光子不行。
\subsection{受激辐射}
受激辐射的概念是由爱因斯坦最先提出的,源自玻尔兹曼分布和普朗克分布之间矛盾。让我们现在来阐释一下这个矛盾。

想象一个理想黑体内有一个两能级系统。能量低的能级称为能级1,位于这个能级的电子的数量为$n_1$;为能量高者称为能级2,位于这个能级的电子数为$n_2$,能级之间的能量差为$\Delta E$。热平衡时,根据玻尔兹曼分布,应有:
\begin{equation}
\frac{n_2}{n_1}=\exp(-\frac{\Delta E}{k_B \ T})
\end{equation}

位于能级2的电子会自发跳转到能级1,并释放$\Delta E$能量的光子,称为自发辐射,电子发生自发辐射的概率是一个定值$A_{21}$,只与电子在两个能级中的状态有关,与时间、空间、电子数目等无关。

位于能级1的电子会吸收能量为$\Delta E$的光子,对应的频率为$\nu=\frac{\Delta E}{h}$,从而跃迁到能级2,称为受激吸收。根据隔壁词条“跃迁概率(一阶微扰)\upref{HionCr}”的介绍,电子发生受激吸收的概率与频率$\nu$附近的光子场的能量密度$\rho(\nu)$成正比,不妨称比例系数为$B_{12}$。$B_{12}$也是个常数,只与电子在两个能级中的状态有关,与时间、空间、电子数目等无关。

自发辐射和受激吸收是原子中的电子最常见的两个过程。但如果电子只发生这两种过程,则对于能级2上的电子数目$n_2$,有:
\begin{equation}
\frac{dn_2}{dt}=-n_2A_{21}+n_1B_{12}\rho(\nu)
\end{equation}
热平衡时,$dn_2/dt=0$,则有:
\begin{equation}
\rho(\nu)=\frac{A_{21}}{B_{12}}\frac{n_2}{n_1}=\frac{A_{21}}{B_{12}}\exp(-\frac{h\nu}{k_B \ T})
\end{equation}
但是根据普朗克黑体辐射公式,应有:
\begin{equation}\label{eq_LaserT_1}
\rho(\nu)=\frac{8\pi\nu^2}{c^3}\frac{h\nu}{\exp(\frac{h\nu}{k_B\ T})-1}
\end{equation}
两条式子存在显著的区别。说明原子中的电子存在除自发辐射和受激吸收以外的过程(注意$\nu$也是一个常数,变量只有$T$)。不妨猜测一种过程,即位于能级2的电子受到能量为$\Delta E$的光子的激发,向能级1跃迁并发射另一个能量为$\Delta E$的光子,称为受激辐射。由于受激辐射依赖光子的激发,所以不妨与受激吸收一样,认为电子发生受激激发的概率也与频率$\nu$附近的光子场的能量密度$\rho(\nu)$成正比,不妨称比例系数为$B_{21}$。

那么热平衡时就有:
\begin{equation}
\frac{dn_2}{dt}=-n_2A_{21}+n_1B_{12}\rho(\nu)-n_2B_{21}\rho(\nu)=0
\end{equation}
代入\autoref{eq_LaserT_1} ,即有:
\begin{equation}\label{eq_LaserT_2}
\frac{A_{21}}{B_{12}}=\frac{8\pi h\nu^3}{c^3},B_{12}=B_{21}
\end{equation}
\autoref{eq_LaserT_2} 被称为爱因斯坦关系式。这样就能解决玻尔兹曼分布和普朗克分布之间的矛盾了。

使用量子电动力学的二次量子化的计算方法可以证明,受激辐射中激发的光子和发射的光子具有同样的方向、频率和偏振状态。这是一个伟大的发现,这意味着我们可以利用增加电压等方式,手动制造一个高能级电子数偏多的系统,高能级电子自发辐射产生首个光子,而后该光子与其他高能级电子发生受激辐射产生2个几乎相同的光子,这两个光子再与其他高能级电子发生受激辐射,如此往复即可实现光子的放大、复制,也就是激光。
\subsection{谐振腔}、
