% 不动点迭代(Matlab 绘图)
% license Pub
% type Note

\begin{figure}[ht]
\centering
\includegraphics[width=8cm]{./figures/eb36329322fc836c.pdf}
\caption{不动点迭代求解$x=\varphi(x)=0.2x^2$的Cobweb图} \label{fig_FPIPLT_1}
\end{figure}

\begin{figure}[ht]
\centering
\includegraphics[width=8 cm]{./figures/41df98badc53c142.pdf}
\caption{对$x=\cos x$的不动点迭代更形象地展现了"Cobweb"(“蛛网”)\textsl{恐蜘蛛者慎入}} \label{fig_FPIPLT_2}
\end{figure}

不动点迭代法也是一种求解方程(或函数零点)$f(x)=0$的经典方法。为了使用不动点迭代,先得改写方程的形式:
\begin{equation}
f(x)=0\Rightarrow x = \varphi(x)~,
\end{equation}
使 $x = \varphi(x)$等式成立的$x$就是$\varphi(x)$的不动点,亦即$f(x)=0$的解。

不动点迭代的迭代公式是
\begin{equation}
x_{k+1} = \varphi(x_k) \qquad k=0,1,2,...~,
\end{equation}
其中$x_0$是人为选定的、对解的初始猜测。如果满足收敛条件\footnote{参考你的数值计算课本},那么迭代结果将收敛于不动点$x$:
$$\lim_{k\to+\infty} x_k = x~.$$

以下的Octave/matlab代码粗糙地实现了不动点迭代,同时做出了过程的Cobweb示意图:

\begin{lstlisting}[language=matlab]
clc
clear

f = @(x) 0.2*x.^2;
x(1) = 4;

hold on
axis equal

for i=1:100
  if i~=1
    x(i) = f(x(i-1));
  end

  line([x(i) x(i)],[0 f(x(i))],'Color','r');
  scatter(x(i),0);
  text(x(i),-0.1,[int2str(i)],'fontsize',12);

  if i~=1
    line([x(i - 1) x(i)],[f(x(i-1)) f(x(i-1))],'Color','b');
    line([x(i) x(i)],[f(x(i - 1)) f(x(i))],'Color','b');
  end

  if i~=1 && abs(x(i) - x(i-1)) < 0.01
    break;
  end
end

if x(i) < x(1)
  _x = x(i)-0.5:0.01:x(1)+0.5;
else
  _x = x(1)-0.5:0.01:x(i)+0.5;
end

_y = f(_x);
plot(_x, _y,'k');
plot(_x, _x,'k');
\end{lstlisting}
