\input{./others/format}

\renewcommand\thesubsubsection{\thesection.\arabic{subsection}.\arabic{subsubsection}}
\begin{document}
\renewcommand{\thelstlisting}{\arabic{lstlisting}}
\input{./others/MatlabStyle} % 设置 Matlab 颜色
\begin{titlepage}
\includepdf{./figures/frontcover.pdf}
\newpage
\includepdf{./figures/dedication.pdf}
\end{titlepage}
\frontmatter % 开始罗马数字页码
\input{./contents/FrontMatters} % 版权声明&关于本书
\setcounter{tocdepth}{1}
\tableofcontents % 生成目录
\mainmatter % 开始阿拉伯数字页码
\setcounter{secnumdepth}{3}


% \part{高中数理}
% %======================================
\part{高中数学}
% 待修改具体名称
%======================================
\chapter{高中数学}
% 待拆分
%---------------------------------------
\entry{高中数学导航}{HsMOv}
\entry{充分必要条件}{SufCnd}
\entry{集合(高中)}{HsSet}
\entry{数列的概念与函数特性(高中)}{HsSeFu}
\entry{等差数列(高中)}{HsAmPg}
\entry{等比数列(高中)}{HsGmPg}
% Giacomo:应当合并,删除(高中)的标注
\entry{等比数列}{GeoPrg}

\entry{角的概念(高中)}{HsAngl}
\entry{三角函数(高中)}{HsTrFu}
\entry{三角恒等变换(高中)}{HsAnTf}
% Giacomo:应当合并,标注超出高中的内容
\entry{三角恒等式}{TriEqv}

\entry{余弦定理}{CosThe}

% 排列组合
% \chapter{排列组合}
% %---------------------------------------
\entry{阶乘}{factor}
\entry{排列(高中)}{HsPm}
\entry{组合(高中)}{HsCb}
\entry{二项式定理(高中)}{HsBN}
\entry{隔板法(排列组合)}{BarCom}
% Giacomo:应当合并,标注超出高中的内容
\entry{排列}{permut}
\entry{组合}{combin}
\entry{二项式定理}{BiNor}

% 概率与统计
% \chapter{概率与统计}
% %---------------------------------------
\entry{离散型随机变量(高中)}{HsDRV}
\entry{条件概率与事件的独立性(高中)}{HsCpMi}
\entry{随机变量的数字特征(高中)}{HsRdNc}
\entry{概率习题(高中)}{HsPbEc}

% 向量与复数
% \chapter{向量与复数}
% %---------------------------------------
\entry{几何矢量}{GVec}
\entry{几何矢量的运算}{GVecOp}
\entry{线性相关性}{linDpe}
\entry{几何矢量的基底和坐标}{Gvec2}
\entry{几何矢量的内积}{Dot}

\entry{复数}{CplxNo}

% 高中代数
% Giacomo:缺韦达定理,二次方程的求根公式

% 平面几何,初中的欧几里得几何


\chapter{拓展}
%---------------------------------------
\entry{集合}{Set}
\entry{映射}{map}
\entry{函数}{functi}
\entry{反函数}{InvFun}
\entry{自然对数函数}{Ln}

\entry{反三角函数}{InvTri}
\entry{四象限 Arctan 函数(atan2)}{Arctan}

\entry{手动计算开根号(长除法)}{Hsqrt}

\entry{一元函数的对称与周期性}{shenry}
\entry{sinc 函数}{sinc}
\entry{双曲函数}{TrigH}

\entry{一元多项式}{OnePol}
\entry{多项式的结式与判别式}{RDPly}
\entry{带余除法}{DivAlg}
\entry{多项式的整除}{ExDiv}
\entry{辗转相除法}{SucDiv}

\entry{多项式的可约性质}{RedPol}
\entry{因式分解唯一性定理}{UniFac}

\chapter{拓展:几何}
% Giacomo:里面应该包括了不少高中几何的内容
%---------------------------------------
\entry{三角形面积、海伦—秦九韶公式}{Heron}
\entry{直线和平面的交点}{LPint}
\entry{点到直线的距离}{P2Line}
\entry{直线和球的交点}{LinSph}
\entry{三角形的外接圆}{SphTri}
\entry{圆锥曲线的极坐标方程}{Cone}
\entry{抛物线}{Para3}
\entry{椭圆}{Elips3}
\entry{双曲线}{Hypb3}
\entry{圆锥曲线和圆锥}{ConSec}
\entry{利萨茹曲线}{Lissaj}
\entry{极坐标系}{Polar}
\entry{阿基米德螺线}{ArcSpl}
\entry{柱坐标系}{Cylin}
\entry{柱坐标与直角坐标的转换}{CyCar}
\entry{球坐标系}{Sph}
\entry{球坐标与直角坐标的转换}{SphCar}
\entry{球坐标的旋转变换}{SphRot}
\entry{抛物线坐标系}{ParaCr}
\entry{椭圆坐标系}{EliCor}
\entry{圆锥曲线的光学性质}{ConOpt}
\entry{摆线}{cycloi}
\entry{解三棱锥顶角}{PrmSol}
\entry{足球顶点坐标的计算}{FootBl}
\entry{日晷的计算}{SunDia}


\part{高中物理}
%======================================
\chapter{高中物理}
% Giacomo:待细分
%---------------------------------------
\entry{机械运动基础(高中)}{HSPM01}
\entry{相互作用(高中)}{HSPM02}
\entry{牛顿运动定律(高中)}{HSPM03}
\entry{曲线运动(高中)}{HSPM04}
\entry{圆周运动(高中)}{HSPM05}
\entry{万有引力定律(高中)}{HSPM06}
\entry{功和机械能(高中)}{HSPM07}
\entry{动量(高中)}{HSPM08}
\entry{机械振动(高中)}{HSPM09}
\entry{静电场(高中)}{HSPE01}
\entry{静电场的应用(高中)}{HSPE02}
\entry{恒定电流(高中)}{HSPE03}

\chapter{科普:经典力学}
%---------------------------------------
\entry{经典力学及其他物理理论}{MecThe}
\entry{经典力学}{CM0}
\entry{牛顿第二定律的矢量形式}{New2}
\entry{动量和能量、一维势能曲线}{CM1}
% \entry{角动量(科普)}{AngMo}
% \entry{角动量}{CM2} % 未完成
% 万有引力 简介

\chapter{科普:电动力学}
%---------------------------------------
\entry{电动力学}{EM0}
\entry{静电的基本规律和性质}{EM1}
\entry{荷质比的测定}{Charge}
\entry{右手定则}{RHRul}
% \entry{电流产生磁场}{CurMag}
\entry{电路和水路的类比}{EleWat}

\chapter{科普:量子力学}
%---------------------------------------
\entry{原子的观念}{AtomId}
\entry{从天球的音乐到玻尔模型}{ClBohr}
\entry{量子力学的基本原理(科普)}{QM0}

\chapter{科普:其他}
%---------------------------------------
\entry{天文学常识}{Astro}
\entry{时间的计量}{TimeCa}
\entry{时间的计量 2}{time2}


\part{一元微积分}
%======================================
% Giacomo:一元标量函数的微积分
\chapter{极限}
%---------------------------------------
\entry{简明微积分导航}{Calc}
\entry{数列的极限(简明微积分)}{Lim0}
\entry{函数的极限(简明微积分)}{FunLim}
\entry{小角正弦极限(简明微积分)}{LimArc}
\entry{自然对数底(简明微积分)}{E}
\entry{切线与割线}{TanL}
\entry{幂级数(简明微积分)}{powerS}

\chapter{导数与微分}
%---------------------------------------
\entry{导数(简明微积分)}{Der}
\entry{基本初等函数的导数(简明微积分)}{FunDer}
\entry{高阶导数(简明微积分)}{HigDer}
\entry{求导法则(简明微积分)}{DerRul}
\entry{复合函数求导、链式法则(简明微积分)}{ChainR}
\entry{反函数求导(简明微积分)}{InvDer}
\entry{导数与函数极值(简明微积分)}{DerMax}
\entry{用极值点大致确定函数图像}{DerImg}
\entry{平面曲线的曲率和曲率半径(简明微积分)}{curvat}

\entry{一元函数的微分(简明微积分)}{Diff}

\entry{泰勒展开(简明微积分)}{Taylor}
\entry{泰勒级数 2}{Taylr2}
\entry{有限差分}{Diffen}
\entry{导数与差分}{DerDif}

\entry{二项式定理(非整数幂)}{BiNorR}

\chapter{不定积分与定积分}
%---------------------------------------
\entry{不定积分(简明微积分)}{Int}
\entry{换元积分法}{IntCV}
\entry{分部积分法}{IntBP}

\entry{定积分(简明微积分)}{DefInt}
\entry{曲线的长度}{CurLen}
\entry{牛顿—莱布尼兹公式(简明微积分)}{NLeib}

\chapter{常微分方程}
%---------------------------------------
\entry{常微分方程}{ODE}
\entry{一阶线性微分方程}{ODE1}
\entry{一维齐次亥姆霍兹方程}{HmhzEq}
\entry{二阶常系数齐次微分方程}{Ode2}
\entry{二阶常系数非齐次微分方程}{Ode2N}
\entry{欧拉方程(微分方程)}{Eulequ}

\chapter{未归类}
%---------------------------------------
\entry{线性最小二乘法}{LstSqr}
\entry{傅里叶级数(三角)}{FSTri}
\entry{傅里叶级数(指数)}{FSExp}

\entry{傅里叶变换(三角)}{FTTri}
\entry{傅里叶变换(指数)}{FTExp}


\part{向量与矩阵}
%======================================
\chapter{几何矢量}
%---------------------------------------
\entry{向量与矩阵导航}{Vector}

\entry{正交归一基底}{OrNrB}
\entry{施密特正交归一化}{SmdtOt}
\entry{矢量叉乘}{Cross}
\entry{矢量叉乘分配律的几何证明}{CrossP}
\entry{连续叉乘的化简}{TriCro}
\entry{三矢量的混合积}{TriVM}

\chapter{矩阵}
%---------------------------------------
\entry{矩阵}{Mat}
%\entry{代数矢量}{NumVec}
\entry{逆矩阵}{InvMat}
\entry{高斯消元法求逆矩阵}{InvMGs}

\entry{分块矩阵}{BlkMat}
\entry{块对角矩阵}{BlDiag}

\entry{矩阵指数}{MatExp}
\entry{相似变换和相似矩阵}{MatSim}
\entry{转移矩阵}{TransM}

\entry{正定矩阵}{DefMat}
\entry{二次多项式与二次型}{QuaPol}
\entry{对称矩阵}{SymMat}
\entry{厄米矩阵、自伴矩阵}{HerMat}
\entry{正交矩阵、酉矩阵}{UniMat}

\entry{矩阵的秩}{MatRnk}
\entry{矩阵的迹}{trace}
\entry{行列式}{Deter}
\entry{行列式的性质}{DetPro}
\entry{行列式与体积}{DetVol}

\entry{张量(向量与矩阵)}{TsrFst}

\chapter{旋转矩阵}
%---------------------------------------
\entry{平面旋转变换}{Rot2DT}
\entry{平面旋转矩阵}{Rot2D}

\entry{叉乘的矩阵形式}{CrosMt}
\entry{三维旋转矩阵}{Rot3D}

\entry{欧拉角}{EulerA}
\entry{四元数与旋转矩阵}{QuatN}
\entry{罗德里格旋转公式、定轴旋转矩阵}{RotA}
\entry{旋转矩阵的导数}{RotDer}

\chapter{矩阵的本征值}
%---------------------------------------
\entry{矩阵的本征方程}{MatEig}
\entry{对称矩阵的本征问题}{SymEig}
\entry{厄米矩阵的本征问题}{HerEig}
\entry{块对角厄米矩阵的本征问题}{BHeig}
\entry{对易厄米矩阵与共同本征矢}{Commut}

\chapter{线性方程组}
%---------------------------------------
\entry{线性方程组}{LinEqu}
\entry{高斯消元法解线性方程组}{GAUSS}
\entry{线性方程组解的结构}{LinEq}
\entry{超定线性方程组}{OvrDet}
\entry{线性方程组的仿射解释}{AS2LF}
\entry{克拉默法则}{kramer}

\chapter{未归类}
%---------------------------------------
\entry{范德蒙矩阵、范德蒙行列式}{VandDe}
\entry{Rayleigh-Ritz 方法}{RayRit}


\part{多元微积分}
%======================================

\chapter{多元标量函数的微积分}
%---------------------------------------
\entry{偏导数(简明微积分)}{ParDer}
\entry{全微分(简明微积分)}{TDiff}
\entry{海森矩阵}{Hesian}
\entry{多元函数的极值}{MulPlo}

\entry{方向导数}{DerDir}
\entry{二元函数的极值(简明微积分)}{F2Exm}
\entry{复合函数的偏导、链式法则(多元标量函数)}{PChain}
\entry{全导数(多元标量函数)}{TotDer}

\entry{偏导与差分}{ParDf}

\entry{多元泰勒展开}{NDtalr}
\entry{重积分、面积分、体积分(简明微积分)}{IntN}
\entry{重积分和宇称}{IntPry}
\entry{重积分的换序(简明微积分)}{Fubin0}

\entry{微分形式(简明微积分)}{DForm}
\entry{广义斯托克斯定理(简明微积分)}{Stoke2}
\entry{高阶微分(多元函数)}{MHDiff}

\chapter{一元矢量函数的微积分}
%---------------------------------------
\entry{导数(一元矢量函数)}{DerV}
\entry{积分(一元矢量函数)}{IntV}

\chapter{多元矢量值函数的微积分}
%---------------------------------------
\entry{偏导数(多元矢量值函数)}{VecPdv}
\entry{雅可比矩阵、雅可比行列式}{JcbDet}

\chapter{应用}
%---------------------------------------
\entry{极坐标系中单位矢量的偏导}{DPol1}
\entry{立体角}{SolAng}
\entry{高斯积分}{GsInt}
\entry{偏微分算符}{ParOp}
\entry{拉格朗日乘数法、条件极值}{LagMul}
\entry{多维球体的体积}{NSphV}

\entry{多元函数的傅里叶级数}{NdFuri}
\entry{齐次函数的欧拉定理}{Homeul}

\entry{一阶线性常微分方程组(简明微积分)}{ODEsys}
\entry{高阶线性微分方程的降阶}{ODEb4}


\chapter{矢量分析}
% Giacomo:我不懂矢量分析,就不做具体分类了
% Giacomo:应该专注于三维空间的特例,一般情况应当移动到上面的对应情况
%---------------------------------------
\entry{矢量场(矢量分析)}{Vfield}
\entry{曲面积分、通量}{SurInt}
\entry{线积分(矢量分析)}{IntL}
\entry{证明闭合曲面的法向量面积分为零}{CSI0}
\entry{矢量算符}{VecOp}
\entry{拉普拉斯算符}{Laplac}
\entry{一种矢量算符的运算方法}{MyNab}
\entry{矢量算符运算法则}{VopEq}
\entry{分部积分的高维拓展}{IntBP2}
\entry{梯度、梯度定理}{Grad}
\entry{用梯度求曲线和曲面的法向量}{GradNr}
\entry{散度、高斯散度定理}{Divgnc}
\entry{牛顿—莱布尼兹公式(矢量分析)}{NLext}
\entry{旋度(简明微积分)}{Curl}
\entry{正交曲线坐标系}{CurCor}
\entry{正交曲线坐标系中的重积分}{CrIntN}
\entry{正交曲线坐标系中的矢量算符}{CVecOp}
\entry{斯托克斯定理(矢量分析)}{Stokes}
\entry{调和场(无散无旋场)}{HarmF}

\entry{散度的逆运算}{DivInv}
\entry{旋度的逆运算}{HlmPr2}
\entry{亥姆霍兹分解}{HelmTh}

\entry{矢量分析总结}{VecAnl}

\part{数学基础}
%======================================
\chapter{逻辑}
%---------------------------------------
\entry{公理系统}{axioms}

\chapter{集合论}
%---------------------------------------
\entry{集合的基数}{CardiN}
\entry{集合(公理化)}{SetAxi}
\entry{无穷的概念}{infty}

\entry{整数}{intger}
% \entry{抽象}{Abstra}

\chapter{序论}
%---------------------------------------
\entry{二元关系}{Relat}

\chapter{范畴论}
%---------------------------------------
\entry{范畴论}{Cat}

\part{组合数学}
%======================================
\chapter{计数}
%---------------------------------------
\entry{范德蒙恒等式}{ChExpn}
\entry{逆序数}{InvNum}
\entry{列维—奇维塔符号}{LeviCi}

\chapter{数论}
%---------------------------------------
\entry{连分数}{ConFra}
\entry{数论函数}{NumFun}

\chapter{其他}
%---------------------------------------
\entry{克罗内克 delta 函数}{Kronec}
\entry{Clebsch–Gordan 系数}{SphCup}
\entry{Wigner 3j 符号}{ThreeJ}
\entry{Wigner 6j 符号}{SixJ}
\entry{Wigner 9j 符号}{NineJ}


\part{线性代数}
%======================================
\chapter{线性空间}
%---------------------------------------
\entry{矢量空间}{LSpace}
\entry{基(线性代数)}{VecSpn}
\entry{矢量空间的表示}{VecRep}

\entry{子空间}{SubSpc}
\entry{商空间}{QuoSpa}
\entry{直和(线性空间)}{DirSum}
\entry{线性映射}{LinMap}
\entry{对偶空间}{DualSp}
\entry{线性无关判别法}{LinInD}

\chapter{内积空间}
%---------------------------------------
% \entry{范数、赋范空间}{NormV}
\entry{内积、内积空间}{InerPd}
\entry{矢量的模和度量的关系}{SNadM}

% 实内积空间
\entry{欧几里得矢量空间}{EuVS}
\entry{欧几里得矢量空间的正交化、同构及正交群}{EVOIOG}
\entry{正交子空间}{OrthSp}

% 复内积空间
% Giacomo:需要统一一下名词
\entry{半双线性形式}{sequil}
\entry{埃尔米特型}{HeFor}
\entry{幺正变换}{Unitar}
\entry{埃尔米特矢量空间(酉空间)}{HVorUV}
\entry{酉群}{UQ}


\chapter{线性变换}
%---------------------------------------
\entry{线性映射的坐标表示}{LTrans}
\entry{矩阵与线性映射}{MatLS}
\entry{线性映射的结构}{MatLS2}
\entry{线性映射的结构 2}{LinEq2}

\entry{对易算符}{Commu}
\entry{厄米共轭算符的映射结构}{RCrank}

\chapter{线性算子(和谱定理)}
%---------------------------------------
\entry{线性算子代数}{LiOper}
\entry{单一算子生成的子代数}{SiOAlg}
\entry{不变子空间}{InvSP}
\entry{本征矢量与本征多项式}{EigVM}
\entry{线性算子对角化的充要条件}{LODia}

\chapter{二次型}
%---------------------------------------
\entry{二次型}{QuaFor}
\entry{二次型的规范型}{GuaOQu}
\entry{实二次型}{RQuaF}
\entry{正定二次型}{DeQua}
\entry{指数有限度量空间}{EFSp}
\entry{斜对称双线性型的规范型}{OBili}


\chapter{张量积}
%---------------------------------------
\entry{多重线性映射}{MulMap}
\entry{张量}{Tensor}
\entry{协变和逆变}{CoCon}
\entry{张量的分类}{CatTns}
\entry{爱因斯坦求和约定}{EinSum}
\entry{张量的坐标}{CofTen}
\entry{张量的坐标变换}{TrTnsr}
\entry{张量的张量积}{TsrPrd}
\entry{张量积空间}{DirPro}
\entry{张量积空间的算符}{ProdOp}
\entry{空间的张量积}{TPofSp}
\entry{线性算子的张量积}{TPofLO}


\entry{外代数}{ExtAlg}
% \entry{霍奇星算子}{HodgeO}

% Giacomo:仿射不应该放这里
\chapter{仿射空间}
%---------------------------------------
\entry{仿射空间}{AfSp}
\entry{仿射子空间}{SAfSp}
\entry{仿射群}{AfQ}
\entry{欧几里得空间}{EucSp}
\entry{保距群(欧氏空间)}{CDQ2Ec}


\chapter{超代数}
---------------------------------------
\entry{超线性空间}{SVecSp}


\part{代数基础}
%======================================
\chapter{群}
%---------------------------------------
% Giacomo:缺少交换群(阿贝尔群)的词条
\entry{群}{Group}
\entry{子群}{Group1}
\entry{陪集和同余}{coset}
\entry{正规子群和商群}{NormSG}
\entry{直积和半直积(群)}{GrpPrd}

\entry{单群}{SemGrp}
\entry{群的同态与同构}{Group2}

\entry{置换群}{Perm}
\entry{置换的奇偶性}{permu}
\entry{群作用}{Group3}
\entry{换位子群}{CmtGrp}

\entry{自由群}{FreGrp}
\entry{群的自由积}{FrePrd}
\entry{群论中的证明和习题解答}{GroupP}
\entry{群论笔记}{GroupN}

\chapter{环与域}
%---------------------------------------
\entry{环}{Ring}
\entry{环的理想和商环}{Ideal}
\entry{环同态}{RingHm}
\entry{整环}{Domain}
\entry{素理想与极大理想}{Ideals}
\entry{爱森斯坦判别式}{EsstCr}
\entry{真因子树}{FctTre}
\entry{唯一析因环}{UFD}
\entry{主理想整环}{PID}
\entry{欧几里得环}{EuRing}
\entry{多项式环}{RPlynm}
\entry{模}{Module}

\entry{环和域}{field}
\entry{分式域}{FrcFld}
\entry{四元数}{Quat}
\entry{域的扩张}{FldExp}
\entry{分裂域}{SpltFd}
\entry{有限域}{FntFld}

\entry{代数学基本定理}{BscAlg}
\entry{韦达定理}{VietaF}

\chapter{代数}
%-------------------------------------
\entry{域上的代数}{AlgFie}
\entry{结构张量代数}{STAlg}

\chapter{李代数}
%-------------------------------------
\entry{李代数}{LieAlg}
\entry{李代数的子代数、理想与商代数}{LieSub}
\entry{李代数的同态与同构}{LieMor}
% \entry{幂零李代数和可解李代数}{NSLie}

\part{拓扑学}
%=====================================
\chapter{点集拓扑}
%-------------------------------------
\entry{拓扑空间}{Topol}
\entry{点集的内部、外部和边界}{Topo0}
\entry{连续映射和同胚}{Topo1}
\entry{紧致性}{Topo2}
\entry{连通性}{Topo3}
\entry{道路连通性}{Topo4}
\entry{分离性}{Topo5}
\entry{积拓扑}{Topo6}
\entry{商拓扑}{Topo7}
\entry{拓扑空间之间的运算}{TopSpO}
\entry{映射空间}{Topo8}
\entry{空间偶和带基点空间}{Topo9}
% \entry{分离性(更多)}{Topo10}
\entry{拓扑群}{TopGrp}

\chapter{同伦论}
%-------------------------------------
\entry{映射的同伦和空间的同伦}{HomT1}
\entry{可缩空间}{HomT2}
\entry{基本群}{HomT3}
\entry{高阶同伦群}{HomT4}
\entry{球面的同伦群}{SphHmt}
\entry{覆叠空间}{CovTop}
\entry{基本群的计算}{HomT5}

\chapter{单纯同调论}
%-------------------------------------
\entry{单纯形与复形}{SimCom}
\entry{单纯剖分(三角剖分)}{Traglt}
\entry{复形的单纯同调群}{SimHml}
\entry{单纯同调群的计算}{SHCal}


% by Jier:微分几何应该适合放在拓扑学后头,有点集拓扑和群论知识就可以讲得足够深入了;李代数的引入可能得两条路同时走,“直接从代数引入”和“从流形到李群到李代数的步步抽象”两种方式.
% “函数芽”的概念该如何插入?
%======================================
\part{微分几何}
%======================================
\chapter{古典微分几何}
%-------------------------------------
\entry{三维欧几里得空间中的曲线}{Curv3D}
\entry{光滑映射(古典微分几何)}{SmthM}
\entry{切空间(古典微分几何)}{tgSpaE}
\entry{三维空间中的曲面}{RSurf}
\entry{基本型}{FForm}
\entry{可定向曲面}{OriSur}
\entry{高斯映射}{GMap}
\entry{高斯曲率和平均曲率}{GHcurv}
\entry{高斯绝妙定理}{Egreg}
\entry{保形映射}{Isomet}
\entry{直纹面(古典微分几何)}{RuSurf}


\chapter{流形}
%-------------------------------------
\entry{流形}{Manif}
\entry{子流形}{SubMnf}
\entry{积流形}{ManPro}
\entry{单位分割}{ParUni}
\entry{切空间(流形)}{tgSpa}
\entry{光滑映射(流形)}{DiffTg}
\entry{抽象指标}{AbsInd}
\entry{切向量场}{Vec}
\entry{纤维丛}{Fibre}
\entry{向量丛和切丛}{TanBun}
% \entry{余切丛}{CotBun}
\entry{流形上的张量场}{TenMan}
% \entry{向量丛}{VecBun}
% \entry{向量场的流}{VecFlo}
\entry{微分形式}{Forms}
\entry{外导数}{ExtDer}
% \entry{闭形式与恰当形式}{CloExa}
\entry{费罗贝尼乌斯定理}{FrobTh}


\chapter{李群和李代数}
%-------------------------------------
\entry{李群}{LieGrp}
\entry{李群的李代数}{LieGA}
\entry{流形上的代数结构}{MnfAlg}


\chapter{黎曼几何}
%-------------------------------------
\entry{黎曼度量与伪黎曼度量}{RiMetr}
% \entry{纳什嵌入定理}{NashEm}
\entry{仿射联络}{affcon}
\entry{协变导数}{CoDer}
\entry{黎曼联络}{RieCon}
\entry{Christoffel 符号}{CrstfS}
\entry{曲率张量场}{RicciC}
\entry{联络形式与结构定理}{ConFom}
\entry{测地线}{geodes}
\entry{庞加莱半平面(微分几何计算实例)}{PoiHP}


\chapter{向量丛上的联络}
%-------------------------------------
\entry{联络(向量丛)}{VecCon}
\entry{曲率(向量丛)}{VecCur}
\entry{平行性(向量丛)}{VecPar}
\entry{和乐群(向量丛)}{VecHol}


\part{代数进阶}
%======================================
\chapter{有限群论}
---------------------------------------
\entry{Sylow定理}{Sylow}

\chapter{Galois 理论}
---------------------------------------
\entry{完全域}{CmplD}
\entry{可分扩张}{SprbEx}
\entry{本原元定理}{PrmtEl}
\entry{可分元素的单扩张是可分扩张}{SprbE2}
\entry{纯不可分扩张}{PInsEx}
\entry{正规扩张}{NomEx}
\entry{Galois扩张}{GExt}
\entry{三次与四次多项式的根}{PlyRtS}
% Giacomo:这个应该放在“广义伽罗瓦理论”里面
\entry{无穷 Galois 扩张与Krull定理}{GExInf}

% \chapter{交换代数}
%--------------------------------------
% Giacomo:TODO
% radical ideal
% 局部化
% C[[x]]

\part{表示论}
%======================================
\chapter{有限群表示论}
---------------------------------------
\entry{群表示}{GrpRep}

\chapter{复半单李代数复表示论}
---------------------------------------
\entry{复化(实李代数)}{ClLA}
\entry{复半单李代数}{SSLA}

\part{概率与统计}
%======================================
\chapter{基础}
\entry{随机变量、概率密度函数}{RandF}
\entry{随机变量的变换}{RandCV}
\entry{多变量分布函数}{MulPdf}
\entry{高斯分布(正态分布)}{GausPD}
\entry{二项分布}{BiDist}
\entry{泊松分布}{PoisD}
\entry{中心极限定理}{CLT}
\entry{抛硬币实验进阶}{CoinEx}
\entry{高尔顿板}{Galton}
\entry{二维随机游走}{RW2D}
\entry{平均值的不确定度}{MeanS}
\entry{多项式定理}{PolyNm}
\entry{卡方分布}{Chi2}

\part{数学分析}
%======================================
\chapter{基础:实数与实数空间}
%---------------------------------------
\entry{集合的极限}{SetLim}
\entry{实数}{ReNum}
\entry{完备公理(戴德金分割)}{Cmplt}
\entry{上确界与下确界}{SupInf}
\entry{实数的完备公理}{RCompl}
\entry{实数集的拓扑}{ReTop}
\entry{序列}{seq}
\entry{序列的极限}{SeqLim}
\entry{极限存在的判据、柯西序列}{CauSeq}
\entry{子列极限、上极限与下极限}{SubLim}
\entry{自然对数的底数(数学分析)}{exp}
\entry{有限覆盖与紧性}{CptRe}
\entry{实数的表示}{ReRep} 
\entry{幂的定义}{RePw} 
%\entry{代数数}{AlgNum} \entry{超越数}{TrsNum} \entry{从集合论看实数}{ReSet} \entry{实数空间和欧几里得空间}{Euc} \entry{复数}{CplNum} \entry{开集与闭集}{OpClo} \entry{实数空间中的紧集}{CptEuc} 

\chapter{极限与连续}
%---------------------------------------
\entry{数项级数}{Series} 
\entry{正项级数的收敛性判别}{PosCov} 
\entry{绝对收敛与条件收敛}{Convg} 
\entry{黎曼重排定理}{RieRes} 
\entry{交错级数的收敛性判别}{AltCov}
\entry{函数的极限}{limfx}
\entry{函数极限的性质}{limff}
\entry{极限的一般观点 重极限与累次极限}{MulLim}
\entry{函数的连续与间断}{confun}
\entry{连续函数的性质}{conff}
%\entry{初等函数} \entry{上极限与下极限}{LimSI} \entry{连续映射}{ConMap} \entry{函数序列的收敛} \entry{连通与道路连通}{ConEuc} \entry{连续映射的性质}{ConPrp} \entry{连续延拓} \entry{一致连续}{UniCon} \entry{单调性}{Monot}


\chapter{微分学}
%---------------------------------------
\entry{导数(数学分析)}{Der2}
\entry{导数的运算法则}{Der3}
%\entry{求导法则}{DLaw} \entry{映射的微分}{MapDif} 
\entry{偏导数(微分学)}{ParDif}
\entry{一致收敛}{UniCnv}
\entry{一致收敛与极限换序}{UniCo2} 
\entry{微分中值定理}{MeanTh}
\entry{洛必达法则}{LHopiR}
\entry{泰勒公式}{Tayl}
\entry{施勒米希-洛希余项公式}{SchRo}
\entry{幂级数与解析函数}{anal}
%\entry{临界点与极值}{StExt} 
\entry{柯西—阿达玛公式}{CHF}
%\entry{隐函数定理} \entry{微分同胚} \entry{条件极值} \entry{拐点}{InflPt} \entry{凸函数}{ConvFu} 
\entry{莫尔斯引理}{Morse}
\entry{凸函数}{ConvFu}

\chapter{多元数量函数的微分学}
%---------------------------------------
\entry{多元数量函数的隐函数定理}{impli}


\chapter{多元向量函数的微分学}
%---------------------------------------
\entry{线性变换与矩阵的代数关系}{linmat}
\entry{向量函数的微分}{vecdif}


\chapter{扩展: 一般度量空间}
%---------------------------------------
\entry{度量空间}{Metric}
\entry{度量空间中的概念}{Metri2}
\entry{柯西序列、完备度量空间}{cauchy}
\entry{完备空间}{ComSpa} % 重复了!
\entry{巴拿赫不动点定理}{ConMap}
%\entry{贝尔纲定理}
%\entry{$p$-进数}
%\entry{有理数集的赋值}

% \chapter{积分学I: 一元黎曼积分}
%---------------------------------------
%\entry{黎曼和} \entry{黎曼积分} \entry{黎曼可积性} \entry{一致收敛与黎曼积分} \entry{牛顿-莱布尼兹公式(连续可微函数)} \entry{泰勒公式II} \entry{反常积分} \entry{绝对收敛与条件收敛(反常积分)}

% \chapter{测度论基础}
%---------------------------------------
%\entry{测度的概念; 测度空间}{MsIdea} \entry{开集与闭集的勒贝格测度}{OpClMs} \entry{博雷尔集}{BorSet} \entry{博雷尔测度}{BorMes} \entry{零测集}{NulSet} \entry{外测度: 卡拉泰奥多里构造}{CaraMe} \entry{测度扩张定理} \entry{勒贝格测度}{LeMes} \entry{行列式与体积}{DetVol} \entry{长度与面积}{LenAr} \entry{复测度与哈恩分解}

% \chapter{可测函数}
%---------------------------------------
%\entry{简单函数} \entry{可测函数} \entry{分布律} \entry{按测度收敛} \entry{几乎处处收敛} \entry{叶戈洛夫定理} \entry{可测函数按连续函数逼近}

% \chapter{积分学II: 勒贝格积分}
%---------------------------------------
%\entry{勒贝格积分} \entry{单调收敛定理} \entry{与黎曼积分的关系} \entry{控制收敛定理} \entry{傅比尼定理} \entry{赫尔德不等式与闵科夫斯基不等式} \entry{一致可积性} \entry{应用: 绝对收敛的含参数积分} \entry{换元公式} \entry{长度与面积的计算}

% \chapter{函数空间与测度空间}
%---------------------------------------
%\entry{$L^p$空间} \entry{扬氏不等式与逼近} \entry{里斯表示定理I: $L^p$空间} \entry{$L^2$空间} \entry{里斯表示定理II: 博雷尔测度} \entry{博赫纳定理} \entry{拉东-尼科蒂姆定理} \entry{有界变差函数} \entry{黑利选择原理} \entry{极大函数} \entry{牛顿-莱布尼兹公式(绝对连续函数)}


\entry{黎曼积分与勒贝格积分}{Rieman}
\entry{Rudin 数学分析笔记 1}{AnalNt}
\entry{Rudin 数学分析笔记 2}{AnalN2}
\entry{Rudin 数学分析笔记 3}{AnalN3}
\entry{Rudin 实分析与复分析笔记 1}{AnalN4}
\entry{Rudin 实分析与复分析笔记 2}{AnalN5}

% \chapter{单复分析}
%---------------------------------------
%\entry{全纯函数} \entry{复变函数求积分} \entry{柯西-古尔萨定理} \entry{柯西积分公式} \entry{全纯函数与解析函数}

\part{微积分与数学分析}
%======================================
\chapter{一元微积分}
%---------------------------------------
\entry{极限}{Lim}
\entry{函数的连续性}{contin}
\entry{莱布尼兹公式}{LeiEqu}

\entry{积分表}{ITable}
\entry{反常积分}{impro}
\entry{极坐标中的曲线方程}{PolCrd}
\entry{函数的算符}{DifOp}
\entry{微分方程 $y^{(N)}=f(x)$}{ynfx}
\entry{记号方法}{Sign}
\entry{正交函数系}{Fbasis}
\entry{正交函数系 2}{OFS}
\entry{狄拉克 delta 函数}{Delta}
\entry{多元狄拉克 delta 函数}{deltaN}
\entry{连续正交归一基底与傅里叶变换}{COrNoB}
\entry{傅里叶变换与矢量空间}{FTvec}
\entry{多元傅里叶变换}{NFTran}
% \entry{拉普拉斯变换}{LapTra}
% \entry{拉普拉斯变换的性质}{ProLap}%这两个拉普拉斯变换注释掉是因为挪到常微分方程中了.如果日后认为需要挪回微积分部分,那可以根据这两个注释的位置来调整
\entry{Gamma 函数}{Gamma}
\entry{Euler-Mascheroni 常数}{Masche}
\entry{Gamma 函数 2}{Gamma2}
\entry{余元公式}{Gama1}
\entry{不完全 Gamma 函数}{IncGam}
\entry{渐近展开}{Asympt}
\entry{拉普拉斯方法}{LapAsm}
% \entry{无穷级数}{infSer}
\entry{魏尔施特拉斯逼近定理}{Weiers}
\entry{狄拉克 delta 导函数}{delta2}
\entry{零函数(列)}{F0}
\entry{包络线}{Velope}
\entry{包络和奇解}{EnvSol}
\entry{隐函数}{ImpFun}
\entry{一元隐函数的存在及可微定理}{ImFED}
\entry{多元隐函数的存在定理}{Mulmp}

\chapter{复变函数}
%---------------------------------------
\entry{复变函数}{Cplx}
\entry{幂函数(复数)}{CPow}
\entry{指数函数(复数)}{CExp}
\entry{三角函数(复数)}{CTrig}
\entry{复变函数的导数、柯西—黎曼条件}{CauRie}
\entry{解析函数与散度旋度}{HolHar}
\entry{复变函数的积分}{CpxInt}
\entry{牛顿—莱布尼兹公式(复变函数)}{AnaInt}
\entry{柯西积分定理}{CauGou}
\entry{洛朗级数}{LaurSr}
\entry{留数定理}{ResThe}
\entry{Jordan 引理}{JdLem}

\chapter{其他}
%-------------------------------------
\entry{亥姆霍兹分解 2}{HelDe}
\entry{Euler-Maclaurin 求和公式}{EMSum}


\part{实变函数与广义函数论}
%======================================
\chapter{Lebesgue外测度与可测函数}

\entry{集合的测度(实变函数)}{SetMet}
% \entry{集合环上的测度}{MsExte}
\entry{可测集合}{MsbSet}
\entry{可测函数}{MsbFun}
\entry{Egoroff定理}{EgrfTh}
\entry{可测函数的Lusin定理}{MsbFSt}
\entry{依测度收敛}{LimMs}

\chapter{Lebesgue积分}
%-------------------------------------
\entry{Lebesgue 积分}{Lebes1}
\entry{Lebesgue积分的一些补充性质}{Lebes2}
\entry{Lebesgue可积的函数}{LIntFn}

\chapter{广义函数的基本概念}
%-------------------------------------
\entry{测度与广义函数}{GenFun}

\part{常微分方程}
%====================================

\chapter{一阶常微分方程}
%---------------------------------------
\entry{常微分方程简介}{ODEint}
\entry{基本知识(常微分方程)}{ODEPr}
\entry{化一般常微分方程组为标准方程组(常微分方程)}{GO2SOD}
\entry{一阶常微分方程解法:变量可分离方程}{ODEa1}
\entry{一阶常微分方程解法:常数变易法}{ODEa2}
\entry{一阶常微分方程解法:恰当方程}{ODEa3}
\entry{一阶隐式常微分方程}{ODEa4}

\chapter{高阶常微分方程和线性微分方程组}
%---------------------------------------
\entry{线性微分方程的一般理论}{ODEb1}
\entry{常系数线性齐次微分方程}{ODEb2}
\entry{一阶常系数线性微分方程组(常微分方程)}{ODEb3}
\entry{二阶齐次变系数线性微分方程的幂级数解法}{ODE2P}
\entry{拉普拉斯变换}{LapTra}
\entry{拉普拉斯变换的性质}{ProLap}
\entry{拉普拉斯变换与常系数线性微分方程}{ODELap}


\chapter{一般理论}
%---------------------------------------
\entry{皮卡-林德勒夫定理}{PiLin}
\entry{解对参数的连续依赖}{ConDep}
\entry{极大解}{MaxSol}
%\entry{皮亚诺存在定理}
%\entry{柯西-科瓦列夫斯卡娅定理}
\entry{施图姆—刘维尔理论}{SLthrm}

% \chapter{常系数方程(组)}
%---------------------------------------
%\entry{二维常系数线性常微分方程组的分类} \entry{基本解矩阵} \entry{特征方程与基本解系} \entry{共振与增益} \entry{渐近稳定性}

% \chapter{微分方程的动力学}
%---------------------------------------
%\entry{双曲不动点} \entry{哈特曼线性化定理} \entry{李雅普诺夫函数} \entry{极限圈} \entry{庞加莱-本迪克森定理} \entry{范德波尔振荡器} \entry{洛伦兹系统}

\part{偏微分方程和特殊函数}
%======================================
\chapter{偏微分方程和特殊函数}
%---------------------------------------
\entry{分离变量法解偏微分方程}{SepVar}
\entry{格林函数解线性非齐次微分方程}{GreenF}
\entry{拉普拉斯方程}{LapEq}
\entry{调和函数}{HarFun}
\entry{球坐标系中的矢量算符}{SphNab}
\entry{球坐标系中的拉普拉斯方程}{SphLap}
\entry{柱坐标系中的矢量算符}{CylNab}
\entry{柱坐标系中的拉普拉斯方程}{CylLap}
\entry{泊松方程}{PoiEqu}
\entry{三维直角坐标系中的亥姆霍兹方程}{RHM}
\entry{球坐标系中的亥姆霍兹方程}{SphHHz}
\entry{柱坐标中的亥姆霍兹方程}{CylHlm}
\entry{勒让德多项式}{Legen}
\entry{连带勒让德多项式}{AsLgdr}
\entry{Hermite 多项式}{HermiP}
\entry{贝塞尔函数}{Bessel}
\entry{球贝塞尔函数}{SphBsl}
\entry{球谐函数}{SphHar}
\entry{实球谐函数}{RYlm}
\entry{球谐函数表}{YlmTab}
\entry{连带拉盖尔多项式}{Laguer}
\entry{双 Gamma 函数}{digama}
\entry{Wigner D 矩阵}{WigDmt}
\entry{平面波的球谐展开}{Pl2Ylm}
\entry{库仑势能的球谐展开}{PChYlm}
\entry{球谐展开中径向函数的归一化}{FrNorm}
\entry{分离变量法与张量积空间}{SVarDP}
\entry{广义球谐函数}{GenYlm}
\entry{误差函数}{Erf}
\entry{黎曼 zeta 函数}{RiZeta}
\entry{虚误差函数}{Erfi}
\entry{超几何函数}{HypGeo}
\entry{Kummer 函数(1F1)}{Kummer}
\entry{椭圆积分}{EliInt}
\entry{库仑函数}{CulmF}
\entry{艾里函数}{AiryF}
\entry{三角积分}{TriInt}
\entry{赫尔德条件}{HolFun}

% \chapter{椭圆型偏微分方程}
%---------------------------------------
%\entry{赫尔德空间}{Holder} \entry{索伯列夫空间 I}{Sobo1} \entry{索伯列夫空间 II}{Sobo2} \entry{椭圆微分算子}{EllOp} \entry{流形上的椭圆微分算子}{EllOpM} \entry{位势积分}{NewPot} \entry{等温坐标}{IsoTh} \entry{椭圆正则性理论 I}{ElReg1} \entry{椭圆正则性理论 II}{ElReg2} \entry{椭圆正则性理论 III}{ElReg3} \entry{自伴椭圆微分算子的谱}{ElSpec} \entry{外尔律}{WeylSp} \entry{听音辨鼓问题}{Kac} \entry{椭圆复形}{ElComp} \entry{流形上的霍奇分解}{Hodge}

\part{泛函分析}
%====================================
\chapter{笔记}
\entry{希尔伯特空间}{Hilber}
%\entry{拉克斯-米尔格兰姆定理}{LaxMil}
% \entry{泛函分析笔记1}{FnalNt}
\entry{泛函分析笔记 1}{FnalNt}
\entry{泛函分析笔记 2}{FnalN2}
\entry{泛函分析笔记 3}{FnalN3}
\entry{泛函分析笔记 4}{FnalN4}
\entry{泛函分析笔记 5}{FnalN5}
\entry{装备希尔伯特空间}{RHS}
\entry{宇称算符}{Parity}


%\chapter{局部凸空间}
%\entry{半范数} %\entry{线性泛函} %\entry{哈恩-巴拿赫定理} %\entry{对偶空间} %\entry{弱拓扑} %\entry{弱紧与弱列紧} %\entry{弗雷歇空间} %\entry{巴拿赫定理(续)} %\entry{实例} 

%\chapter{广义函数}
%\entry{试验函数与施瓦兹函数} %\entry{广义函数与缓增广义函数} %entry{分布导数} %\entry{傅里叶变换} %\entry{帕塞瓦尔定理} %\entry{卷积}

%\chapter{巴拿赫空间中的紧性}
%\entry{弱拓扑与弱星拓扑} %\entry{可分空间中的弱拓扑} %\entry{$L^p$空间中的弱列紧性} %\entry{$L^1$空间中的弱列紧性} %\entry{测度族的胎紧性} %\entry{测度的弱收敛}

\chapter{赋范空间}
%---------------------------------------
\entry{范数、赋范空间}{NormV}
\entry{里斯引理(赋范空间)}{RiLem}
\entry{正交分解、投影算符}{projOp}
\entry{柯西—施瓦茨不等式}{CSNeq}
%\entry{里斯表示定理III: 希尔伯特空间}
\entry{巴拿赫空间}{banach}
\entry{巴拿赫定理}{BanThm}

\chapter{广义函数与Fourier变换}
\entry{广义函数}{GenFut}

\chapter{有界算子的谱论}
\entry{有界算子的谱}{BddSpe}
\entry{有界算子的预解式}{BddRsv}
\entry{谱半径}{SpeRad}
%\entry{算子的函数}{OpFunc}
\entry{谱投影}{SpePrj}
\entry{例: 有限维方阵}{SpeMat}
%\entry{紧线性算子}{CompOp} \entry{里斯-邵德尔理论}{RieSch} \entry{例: 弗雷德霍姆积分方程}{Fredh} \entry{例: 具有平方可积核的积分方程}{L2Ker} \entry{紧自伴算子}{CptAdj} \entry{希尔伯特-施密特算子}{HilSch}

%\chapter{扩展: 巴拿赫代数}
%\entry{巴拿赫代数}{BaAlg} \entry{元素的谱}{BaSpec} \entry{盖尔范德表示}{GelRep}

%\chapter{闭算子的谱论}
%\entry{无界算子}{UbddOp} \entry{闭算子}{CloOp} \entry{第一与第二预解公式}{ResvF} \entry{外尔序列}{WeylSq} \entry{离散谱与本质谱}{DisEss} \entry{伴随算子}{Adj} \entry{自伴算子}{SelfAd} \entry{半双线性形式}{sequil} \entry{弗里德里希扩张}{FrdExt} \entry{谱测度}{SpecMs} \entry{自伴算子的谱定理}{AdjSpe} \entry{瑞利-里斯定理}{RayRi}

%\chapter{谱微扰论}

\chapter{一元函数的变分学}
%---------------------------------------
\entry{绝对极值与相对极值(变分学)}{AbPol}
\entry{可取曲线(变分学)}{DesCur}
\entry{变分}{Varia}
\entry{极值的必要条件(变分学)}{PolReq}
\entry{变分的变换(变分学)}{VarCha}
\entry{变分的基本定理(变分学)}{VarDef}
\entry{欧拉方程(变分学)}{ElueEV}
\entry{二次变分}{SecVar}
\entry{极端曲线}{ExtCur}
\entry{端点可变问题}{EPQue}
\entry{斜截条件}{OCCond}
\entry{多元函数泛函的极值}{MulFP}

\part{经典力学}
%======================================
\chapter{质点运动学}
%---------------------------------------
\entry{物理量和单位转换}{Units}
\entry{无量纲的物理公式}{NoUnit}
\entry{位置矢量、位移}{Disp}
\entry{速度、加速度(一维)}{VnA1}
\entry{速度、加速度}{VnA}
\entry{圆周运动的速度}{CMVD}
\entry{圆周运动的加速度}{CMAD}
\entry{匀加速直线运动}{CnstAL}
\entry{匀加速运动}{ConstA}
\entry{曲线运动的加速度}{PCuvMo}
\entry{极坐标中的速度和加速度}{PolA}
\entry{速度的参考系变换}{Vtrans}
\entry{加速度的参考系变换}{AccTra}

\chapter{质点动力学}
%---------------------------------------
\entry{力的分解与合成}{Fdecom}
\entry{绳结的受力分析}{Knot}
\entry{牛顿运动定律、惯性系}{New3}
\entry{圆周运动的向心力}{CentrF}
\entry{重力、重量}{Weight}
\entry{功、功率}{Fwork}
\entry{动能、动能定理(单个质点)}{KELaw1}
\entry{力场、保守场、势能}{V}
\entry{状态量和过程量}{StaPro}
\entry{机械能守恒(单个质点)}{ECnst}
\entry{动量、动量定理(单个质点)}{PLaw1}
\entry{角动量、角动量定理、角动量守恒(单个质点)}{AMLaw1}
\entry{简谐振子}{SHO}
\entry{受阻落体}{RFall}
\entry{单摆}{Pend}
\entry{圆锥摆}{ConPen}
\entry{傅科摆}{Fouclt}
\entry{惯性力}{Iner}
\entry{滑块和运动斜面问题}{blkSlp}
\entry{离心力}{Centri}
\entry{科里奥利力}{Corio}
\entry{旋转参考系的 “机械能守恒”}{Rconst}
\entry{地球表面的科里奥利力}{ErthCf}

\chapter{质点系与刚体}
%---------------------------------------
\entry{自由度}{DoF}
\entry{质点系}{PSys}
\entry{质心、质心系}{CM}
\entry{木块堆叠问题(里拉斜塔)}{LireTo}
\entry{质点系的动量}{SysMom}
\entry{刚体}{RigBd}
\entry{轻杆模型}{rod}
\entry{动量定理、动量守恒}{PLaw}
\entry{质点系的动能、柯尼希定理}{Konig}
\entry{力矩}{Torque}
\entry{刚体的静力平衡}{RBSt}
\entry{系统的角动量}{AngMom}
\entry{角动量定理、角动量守恒}{AMLaw}
\entry{二体系统}{TwoBD}
\entry{二体碰撞}{TwoCld}
\entry{刚体定轴转动、转动惯量}{RigRot}
\entry{平行轴定理、垂直轴定理、可加性定理}{MIthm}
\entry{常见几何体的转动惯量}{ExMI}
\entry{刚体的平面运动方程}{RBEM}
\entry{惯性张量}{ITensr}
\entry{刚体的惯量主轴}{PrncAx}
\entry{刚体的瞬时转轴、角速度的矢量相加}{InsAx}
\entry{刚体定轴转动 2}{RBrot2}
\entry{纯滚动}{Pscrol}
\entry{刚体定轴转动的力矩做功、动能、动能定理}{RigEng}
\entry{刚体的动能、动能定理}{RBKE}
\entry{刚体的运动方程}{RBEqM}
\entry{刚体定点旋转的运动方程(欧拉角)}{RigEul}
\entry{刚体运动方程(四元数)}{RBEMQt}

\chapter{软体和流体力学}
%---------------------------------------
\entry{绳的法向压力}{RopeFP}
\entry{悬链线}{Catena}
\entry{杨氏模量}{YoungM}
\entry{流体和固体}{SLG}
\entry{流、流密度}{CrnDen}
\entry{浮力、阿基米德原理}{Buoy}
\entry{伯努利方程}{Bernul}
\entry{黏度}{viscos}
\entry{流体运动的描述方法}{fluid1}
\entry{物质导数(实质导数)}{fluid2}
\entry{流体力学守恒方程}{fluidC}
\entry{Navier-Stokes 方程}{NSeq}
\entry{流体力学方程组}{fluidE}
% \entry{流体力学的控制方程}{SCP42}
% \entry{流体的控制方程}{scp999}

\chapter{振动与波动}
%---------------------------------------
\entry{振动的指数形式}{VbExp}
\entry{能量法解谐振动问题}{EnerVi}
\entry{拍频}{beatno}
\entry{受阻简谐振子}{SHOf}
\entry{简谐振子的品质因数}{SHOq}
\entry{简谐振子受迫运动}{SHOfF}
\entry{弹簧的串联和并联}{Spring}
\entry{共振}{ResoN}
\entry{平面简谐波}{PWave}
% \entry{驻波}{StaWav}
\entry{波包}{WvPck}
\entry{高斯波包}{GausPk}
\entry{群速度}{GroupV}
\entry{多普勒效应(一维匀速)}{Dople1}
\entry{多普勒效应}{Dopler}
\entry{一维波动方程}{WEq1D}
% 未完成 边界条件 (两条密度不同的绳子)
\entry{二维波动方程}{Wv2D}
\entry{波的能量}{WaEner}
\entry{波的强度}{WaInte}
\entry{冲击波}{ShoWav}

\chapter{中心力场问题}
%---------------------------------------
\entry{万有引力、引力势能}{Gravty}
\entry{壳层定理}{SphF}
\entry{开普勒三定律}{Keple}
\entry{中心力场问题}{CenFrc}
\entry{开普勒问题}{CelBd}
\entry{开普勒问题的运动方程}{EqMoKp}
\entry{拉普拉斯—龙格—楞次矢量}{LRLvec}
\entry{轨道方程、比耐公式}{Binet}
\entry{开普勒第一定律的证明}{Keple1}
\entry{开普勒第二和第三定律的证明}{Keple2}
\entry{散射}{Scater}
\entry{卢瑟福散射}{RuthSc}
\entry{闭合轨道的条件}{ClsOrb}
\entry{Bohr-Sommerfeld 原子模型}{BohrEc}

\chapter{分析力学}
%---------------------------------------
\entry{欧拉—拉格朗日方程}{Lagrng}
\entry{广义力}{LagEqQ}
\entry{虚位移、虚功、虚功原理}{VirWrk}
\entry{拉格朗日方程和极值问题}{LagPrb}
\entry{贝尔特拉米等式}{Beltra}
\entry{最速降线问题}{Brachi}
\entry{单摆(大摆角)}{SinPen}
\entry{双摆和三摆}{Pendu3}
\entry{小振动}{Oscil}
\entry{拉格朗日方程的证明、达朗贝尔原理}{dAlbt}
\entry{最小作用量、哈密顿原理}{HamPrn}
\entry{运动积分}{motint}
\entry{二体问题(分析力学)}{twoobj}
\entry{勒让德变换}{TrLgdr}
\entry{哈密顿正则方程}{HamCan}
\entry{泊松括号}{poison}
\entry{正则变换 2}{ClsMec}

\chapter{轨道力学}
%---------------------------------------
\entry{二体问题综述}{ConDB}
\entry{轨道参数、时间变量}{OribP}
\entry{限制性三体问题}{TriLim}
\entry{雅可比常量}{JacCon}
\entry{雅可比常量2}{JConst}
\entry{拉格朗日点}{LPoint}
\entry{光子火箭}{PhRoc}
\entry{太空电梯}{SpcLad}

\part{光学}
%======================================
\chapter{几何光学}
%---------------------------------------
\entry{惠更斯原理}{Huygen}
\entry{光的折射、斯涅尔定律}{Snel}
\entry{相移}{PhaSft}
\entry{薄透镜}{ThnLen}

\chapter{波动光学}
%---------------------------------------
\entry{可见光谱}{VisSpt}
\entry{双缝干涉中一个重要极限}{SltLim}
\entry{干涉、光强的余弦平方分布}{IntCos}
\entry{杨氏双缝干涉实验}{Young}
% \entry{单缝衍射}{}, \entry{多缝衍射}{}, \entry{薄膜干涉}{} % 例如牛顿环、楔形薄膜
\entry{劳埃德镜实验}{Lloyd}
\entry{普通光源的发光机理}{LumiMe}
\entry{单色光}{MonoLi}
% \entry{偏振光}{PolLig}
\entry{高斯光束}{GausBm}
\entry{晶体衍射}{CrysDf}

\part{电动力学}
%======================================
\chapter{基础}
%---------------------------------------
\entry{电流}{I}
\entry{电流密度}{Idens}
\entry{库仑定律}{ClbFrc}
\entry{电场}{Efield}
\entry{磁场}{MagneF}
\entry{电势、电势能}{QEng}
\entry{电偶极子}{eleDpl}
\entry{电偶极子 2}{eleDP2}
\entry{导体}{Cndctr}
\entry{电压和电动势}{Voltag}
\entry{电介质的简单模型}{dieleS}
\entry{电介质的微观结构}{Dielec}
\entry{电极化强度}{ElecPo}
\entry{电极化强度与极化电荷的关系}{ElePAP}
\entry{极化电流}{PolCur}
\entry{介质中的静电场}{EFIDE}
\entry{电阻、欧姆定律、电阻率、电导率}{Resist}
\entry{电感}{Induct}
\entry{电场的高斯定律}{EGauss}
\entry{电场的高斯定律证明}{EGausP}
\entry{导体的静电平衡}{MetEqv}
\entry{静电势的泊松方程}{EPoiEQ}
\entry{磁场的高斯定律}{MagGau}
\entry{电场的能量}{EEng}
\entry{比奥萨伐尔定律}{BioSav}
\entry{安培环路定律}{AmpLaw}
\entry{洛伦兹力}{Lorenz}
\entry{霍尔效应}{Hallef}
\entry{磁场的能量}{BEng}
\entry{磁通量}{BFlux}
\entry{安培力}{FAmp}
\entry{磁矩}{MagMom}
\entry{磁偶极矩}{Bdipol}
\entry{磁场中闭合电流的合力}{EBLoop}
\entry{磁场中闭合电流的力矩}{EBTorq}
\entry{法拉第电磁感应定律}{FaraEB}
\entry{位移电流、广义安培环路定律}{DisCur}
\entry{分子电流和分子磁矩}{MoMaMo}
\entry{磁介质}{MagMat}
\entry{磁化强度}{MaInte}
\entry{顺磁质的磁化}{ParaMa}
\entry{抗磁质的磁化}{DiaMaM}
\entry{有磁介质时的安培环路定律}{MaAmpe}
\entry{厘米—克—秒单位制}{CGS}
\entry{高斯单位制}{GaussU}

\chapter{电路}
%---------------------------------------
\entry{电路}{Circ}
\entry{基尔霍夫电路定律}{Kirch}
\entry{电容}{Cpctor}
\entry{电阻的串联和并联}{Rcomb}
\entry{电感的串联和并联}{IndCmb}
\entry{电容的串联和并联}{Ccomb}
\entry{LC 振荡电路}{LC}
\entry{力电振动类比}{MeElec}
\entry{LC 受迫振荡电路}{EleRes}
\entry{Y-Δ 变换、星角变换}{Tri2St}
\entry{电容—电阻电路充放电曲线}{RCcurv}
\entry{惠斯通电桥}{WheBrg}
\entry{阻抗、电抗}{impeda}
\entry{电抗、容抗、感抗}{CapRea}

\chapter{电动力学 2}
%---------------------------------------
\entry{电荷守恒、电流连续性方程}{ChgCsv}
\entry{电多极展开}{EMulPo}
\entry{电磁场标势和矢势}{EMPot}
\entry{磁标势}{elecdy}
\entry{磁矢势}{BvecA}
\entry{磁多极矩}{edy33}
\entry{规范变换}{Gauge}
\entry{洛伦兹规范}{LoGaug}
\entry{库仑规范}{Cgauge}
\entry{格林函数与静电边值问题}{EleGr}
\entry{电磁场的能量守恒、坡印廷矢量}{EBS}
\entry{麦克斯韦方程组}{MWEq}
\entry{麦克斯韦方程组(介质)}{MWEq1}
\entry{麦克斯韦方程组(外微分形式)}{MWEq2}
\entry{介质的边界条件}{mbdy}
\entry{电磁场推迟势}{RetPt0}
\entry{磁单极子}{BMono}
\entry{恩绍定理}{earnsh}
\entry{非齐次亥姆霍兹方程、推迟势}{RetPot}
\entry{电场波动方程}{EWEq}
\entry{真空中的平面电磁波}{VcPlWv}
\entry{平面电磁波的能量叠加}{PwvAdd}
\entry{时谐电磁波}{TSEBW}
\entry{电磁波包的能谱}{WpEng}
\entry{电偶极子辐射}{DipRad}
\entry{介质中的波动方程}{MedWF}
\entry{导体中的电磁波}{MetalW}
\entry{菲涅尔公式、布儒斯特角、临界角、内反射与外反射}{Fresnl}
\entry{盒中的电磁波}{EBBox}
\entry{电磁场的动量守恒、动量流密度张量}{EBP}
\entry{磁旋比、玻尔磁子}{BohMag}

\chapter{电动力学 3}
%---------------------------------------
\entry{电磁场的参考系变换}{EMRef}
\entry{拉格朗日电磁势}{EMLagP}
\entry{电磁场中粒子的拉氏量}{ElecLS}
\entry{电磁场的作用量}{ElecS}
\entry{电磁场角动量分解}{EMAMSp}
\entry{电磁场张量}{EMFT}
\entry{电磁场的能动张量}{EMtens}
\entry{李纳维谢尔势}{LWP}
\entry{带电粒子的辐射}{chgrad}

\part{相对论}
%======================================
\chapter{狭义相对论}
%---------------------------------------
\entry{狭义相对论的基本假设}{SpeRel}
\entry{事件与尺缩效应}{SRsmt}
\entry{时间的变换与钟慢效应}{SRtime}
\entry{洛伦兹变换}{SRLrtz}
\entry{斜坐标系}{ObSys}
\entry{斜坐标系表示洛伦兹变换}{SROb}
\entry{自然单位制、普朗克单位制}{NatUni}
\entry{约化光速}{SRc}
\entry{洛伦兹变换的代数推导}{LornzT}
\entry{相对论速度变换}{RelVel}
\entry{相对论加速度变换}{SRAcc}
\entry{时空的四维表示}{SR4Rep}
\entry{闵可夫斯基空间}{MinSpa}
\entry{双生子佯谬}{Twins}
\entry{光的多普勒效应}{RelDop}
\entry{洛伦兹群}{qed1}
\entry{托马斯进动}{TmsPrs}

\chapter{相对论动力学}
%---------------------------------------
\entry{相对论动力学假设}{SRDyn}
\entry{闵可夫斯基时空中的能动张量}{SRFld}

\chapter{经典场论}
%---------------------------------------
\entry{从分析力学到场论}{CFa1}



\chapter{广义相对论}
%---------------------------------------
\entry{引力的弱场近似}{WeakG}
\entry{爱因斯坦场方程}{EinEqn}
\entry{因果结构}{Causal}
\entry{广义相对论中的对称性和 Killing 矢量场}{GR}
\entry{线性引力}{LinGra}
\entry{ADM形式}{ADMF}

\part{量子力学}
%======================================
% 未完成 束缚态的一般性质: 节点数, 对称性 (偶势能的基态是偶函数), 简并性(一维情况不简并)

\chapter{入门}
%---------------------------------------
\entry{量子力学导航}{QMmap}
\entry{量子力学的诞生}{QMborn}
\entry{玻尔原子模型}{BohrMd}
\entry{玻尔原子模型(约化质量)}{HRMass}
\entry{原子单位制}{AU}
\entry{指数衰减}{ExpDec}
\entry{精细结构常数}{FinStr}

\chapter{单粒子一维问题}
%---------------------------------------
\entry{狄拉克符号}{braket}
\entry{量子力学与矩阵}{QMmat}
\entry{量子力学的算符和本征问题}{QM1}
\entry{平均值(量子力学)}{QMavg}
\entry{守恒量(量子力学)}{QMcons}
\entry{概率流密度}{PrbJ}
\entry{不确定性原理}{Uncert}
\entry{平面波的的正交归一化}{EngNor}
\entry{薛定谔方程(单粒子一维)}{TDSE11}
\entry{量子散射(一维)}{Sca1D}
\entry{一维散射态的正交归一化}{ScaNrm}
\entry{定态薛定谔方程(单粒子一维)}{SchEq}
\entry{位置表象和动量表象}{moTDSE}
\entry{好量子数}{GoodQN}
\entry{算符的矩阵表示}{OpMat}

\chapter{单粒子多维问题}
%---------------------------------------
\entry{薛定谔方程(单粒子多维)}{QMndim}
\entry{薛定谔方程 2(单粒子多维)}{TDSE}
\entry{薛定谔方程的分离变量法}{SEsep}
\entry{球坐标系中的定态薛定谔方程}{RadSE}
\entry{球坐标中的薛定谔方程}{RYTDSE}
\entry{柱坐标系中的薛定谔方程}{CyliSE}
\entry{轨道角动量(量子力学)}{QOrbAM}
\entry{轨道角动量升降算符归一化}{QLNorm}
\entry{球坐标系中的轨道角动量算符}{SphAM}
\entry{自旋角动量}{Spin}
\entry{自旋角动量矩阵}{spinMt}
\entry{角动量的叠加(量子力学)}{AdAngM}
\entry{角动量的叠加 2(量子力学)}{AMAdd}
\entry{算符的指数函数、波函数传播子}{OpExp}
\entry{拉莫尔进动}{Larmor}

\chapter{多粒子问题}
%---------------------------------------
\entry{多体薛定谔方程}{NbdQM}
\entry{全同粒子}{IdPar}
\entry{泡利不相容原理}{PauliE}
\entry{粒子交换算符}{ExchOp}
\entry{量子态的对称化与反对称化}{symetr}
\entry{全同粒子的交换力}{ExchF}
\entry{平移算符}{tranOp}
\entry{旋转算符}{rotOp}

\chapter{定态问题}
%---------------------------------------
\entry{无限深方势阱}{ISW}
\entry{有限深方势阱}{FSW}
\entry{方势垒}{SqrPot}
\entry{无限深阶梯势阱}{StpPot}
\entry{有限深不对称方势阱}{AMW}
\entry{阶梯势能散射}{StepV}
\entry{升降算符}{RLop}
\entry{量子简谐振子(升降算符法)}{QSHOop}
\entry{简谐振子升降算符归一化}{QSHOnr}
\entry{量子简谐振子(级数法)}{QSHOxn}
\entry{一维自由粒子(量子)}{FreeP1}
\entry{一维 delta 势能散射}{Dsc1D}
\entry{一维 delta 势能晶格}{DelCry}
\entry{线性势能的定态薛定谔方程}{LinPot}
\entry{一阶不含时微扰理论(量子力学)}{TIPT}
\entry{WKB 近似}{WKB}
\entry{二维无限深方势阱}{ISW2D}
% \entry{二维有限深方势阱}{FSW2D}
\entry{无限深圆形势阱}{CirISW}
\entry{无限深球势阱}{ISphW}
\entry{有限深球势阱}{FiSph}
\entry{三维量子简谐振子(球坐标系)}{SHOSph}

\chapter{含时问题}
%---------------------------------------
\entry{拉比频率}{RabiF}
\entry{一维自由高斯波包(量子)}{GausWP}
\entry{含时微扰理论(束缚态)}{TDPT}
\entry{含时微扰理论}{TDPTc}
\entry{几种含时微扰}{TDPEx}
\entry{费米黄金法则}{FermGR}

\chapter{量子散射}
%---------------------------------------
\entry{量子散射(单粒子弹性)}{ParWav}
\entry{量子散射的波恩近似}{BornSc}
\entry{球面散射态与平面散射态的转换}{Scatt2}
\entry{Lippmann-Schwinger 方程}{LipSch}
\entry{含时散射的形式理论}{TDFSc}
\entry{库仑散射(量子)}{CulmWf}
\entry{量子散射的延迟}{tDelay}
\entry{光电离时间延迟}{HeAna2}
\entry{多通道散射}{MulSct}
\entry{R-矩阵法(量子力学)}{Rmat}

\chapter{量子力学 2}
%---------------------------------------
\entry{量子力学中的变分法、Rayleigh-Ritz 变分法}{QMVar}
\entry{密度矩阵}{denMat}
% \entry{一维散射的相移}{PhShi1}
\entry{质心系中的多粒子问题}{SECM}
\entry{量子力学的基本假设}{QMPos}
\entry{电磁场中的薛定谔方程及规范变换}{QMEM}
\entry{库仑规范(量子)}{CouGau}
\entry{长度规范}{LenGau}
\entry{速度规范}{LVgaug}
\entry{加速度规范}{AccGau}
\entry{Volkov 波函数}{Volkov}
\entry{海森堡绘景}{HsbPic}
\entry{Hartree-Fock 方法}{HarFor}
\entry{Baker-Hausdorff 公式}{BAHA}
\entry{Adiabatic 笔记}{Adibat}
\entry{超导唯象解释——伦敦方程}{edy34}
\entry{Hall 量子力学笔记}{HallQM}

\chapter{量子力学与量子场论}
%---------------------------------------
\entry{前言}{QFIntro}
\entry{基本概念}{Basics}
\entry{全同粒子的统计}{IdParS}
\entry{近似理论:微扰}{AprPtr}
\entry{角动量 2 (量子力学)}{QMAM}
% 第六章直接从 sakurai 翻译, 后面的一些内容非常零碎跳过
\entry{冷原子基本知识}{UCBas}
\entry{两个原子间的相互作用}{TwoAtF}
\entry{Feshbach 共振}{FeshRs}
\entry{BCS-BEC Crossover 的平均场描述}{BCSBEC}
\entry{BEC 超流}{BECSup}

\chapter{原子分子物理}
%---------------------------------------
\entry{氢原子基态的波函数}{HWF0}
\entry{类氢原子的束缚态}{HWF}
\entry{氢原子波函数分析}{Hanaly}
\entry{氢原子的精细能级结构}{HfineS}
\entry{氢线(21厘米线)}{HydroL}
\entry{类氢原子斯塔克效应(微扰)}{HStark}
\entry{塞曼效应}{ZemEff}
\entry{抛物线坐标系中的类氢原子定态波函数}{ParaHy}
\entry{类氢原子的 Stark 效应(抛物线坐标系)}{HStrk2}
\entry{电磁场中的类氢原子}{EMHydr}
\entry{跃迁概率(一阶微扰)}{HionCr}
\entry{单电子跃迁截面(一阶微扰)}{SIcros}
\entry{氢原子电离计算(一阶微扰)}{HyIon2}
\entry{氢原子的跃迁偶极子矩阵元和选择定则}{SelRul}
\entry{氢原子的跃迁偶极子矩阵元列表}{HDipM}
\entry{跃迁偶极子矩阵的三种形式}{DipEle}
\entry{康普顿散射}{Comptn}
\entry{电子轨道与元素周期表}{Ptable}
\entry{能项符号}{TrmSym}
\entry{兰姆位移}{LambSh}
\entry{ponderomotive 能量}{Ponder}
\entry{氢原子隧道电离}{Htunnl}
\entry{Keldysh 参数}{keldis}
\entry{单电子原子模型}{SAE}
\entry{氦原子中的对易算符与能项符号}{HeComu}
\entry{ADK 电离率}{ADKrat}
\entry{Floquet 理论}{Floque}
\entry{布洛赫理论}{Bloch}
\entry{FROG}{Frog}
\entry{双原子分子势能曲线}{dpecs}

\chapter{固体物理}
%---------------------------------------
\entry{晶格振动导论}{LatVib}
\entry{一维单原子链晶格}{onatom}
\entry{一维双原子链晶格}{twatom}
\entry{德鲁德模型}{DrudeM}
\entry{晶格热容的爱因斯坦理论}{EScap}
\entry{晶格热容的德拜理论}{Debye}
\entry{近自由电子模型}{egasmd}
\entry{紧束缚近似}{tbappx}


\part{热力学和统计力学}
%======================================
\chapter{热力学}
%---------------------------------------
\entry{热力学与统计力学导航}{StatMe}
\entry{理想气体状态方程}{PVnRT}
\entry{温度、温标}{tmp}
\entry{理想气体}{Igas}
\entry{理想气体的内能}{IdgEng}
\entry{压强体积图}{PVgraf}
\entry{热平衡、热力学第零定律}{TherEq}
\entry{热传导定律}{Heatco}
\entry{热力学第一定律}{Th1Law}
\entry{态函数}{statef}
\entry{盖斯定律与设计路径}{Hess}
\entry{准静态过程}{Quasta}
\entry{等压过程}{EqPre}
\entry{等体过程}{EqVol}
\entry{等温过程}{EqTemp}
\entry{热容量}{ThCapa}
\entry{绝热过程}{Adiab}
\entry{节流过程}{ttpro}
\entry{卡诺热机}{Carnot}
\entry{热力学第二定律}{Td2Law}
\entry{亥姆霍兹自由能}{HelmF}
\entry{吉布斯自由能}{GibbsG}
\entry{熵}{Entrop}
\entry{麦克斯韦关系}{MWRel}
\entry{熵的宏观表达式}{MacroS}
\entry{理想气体分压定律}{PartiP}
\entry{饱和蒸汽压}{VaporP}
\entry{大气密度和压强}{atmDen}
\entry{维恩位移定律}{WienDs}
\entry{斯特藩—玻尔兹曼定律}{SteBol}
\entry{热动平衡判据}{equcri}
\entry{相变平衡条件}{PhEquv}
\entry{范德瓦尔斯气体}{Vand}
\entry{克拉伯龙方程}{Clapey}
\entry{表面张力}{sftens}
\entry{沸腾}{ebull}
\entry{热传导定律与传递过程}{heatc}
\entry{多元系热力学导引}{mulTh}
\entry{理想气体化学平衡条件}{ICheEq}
\entry{吉布斯相律}{GBPL}
%---------------------------------------
\chapter{物质的微观经典理论}
% 近独立子系的最概然分布
% 熵与玻尔兹曼关系
% 能量均分定理
\entry{气体分子对容器壁的压强}{MolPre}
\entry{分子平均碰壁数}{AvgHit}
\entry{气体分子的速度分布}{VelPdf}
\entry{麦克斯韦—玻尔兹曼分布}{MxwBzm}
\entry{气体传递过程(微观理论)}{gasTra}
\chapter{统计力学}
%---------------------------------------
% 玻色气体
% 费米气体
% 实际气体状态方程(集团展开法)
% 玻尔兹曼方程和弛豫时间近似
\entry{黑体辐射定律}{BBdLaw}
\entry{相空间}{PhSpace}
\entry{理想气体的状态密度(相空间)}{IdSDp}
\entry{理想气体单粒子能级密度}{IdED1}
\entry{玻尔兹曼分布(统计力学)}{MBsta}
\entry{热力学量的统计表达式(玻尔兹曼分布)}{TheSta}
\entry{金属中的自由电子气体}{mfcgas}
\entry{理想气体(微正则系综法)}{IdNCE}
\entry{正则系综法}{CEsb}
\entry{理想气体(正则系宗法)}{IdCE}
\entry{理想气体(巨正则系综法)}{IdMCE}
\entry{等间隔能级系统(正则系宗)}{EqCE}
\entry{巨正则系综法}{MCEsb}
\entry{量子气体(单能级巨正则系综法)}{QGs1ME}
\entry{量子气体(巨正则系宗)}{QGsME}
\entry{光子气体}{PhoGas}
\entry{理想气体的熵:纯微观分析}{IdeaS}
\entry{伊辛模型}{Ising}
\entry{玻尔兹曼方程}{BolzEQ}
\entry{统计力学公式}{StatEq}
\entry{玻色爱因斯坦凝聚}{BEC}

\part{近代物理}
%======================================
\chapter{量子场论}
%---------------------------------------
\entry{引言}{QFT0}
\entry{经典场论基础}{classi}
\entry{标量场}{qed2}
\entry{标量场的量子化}{quanti}
\entry{因果律}{cau}
\entry{时空中的标量场}{scalar}
\entry{标量场的谱}{spectr}
\entry{克莱因-戈登传播子}{Klein}
\entry{Wick 定理}{wick}
\entry{粒子产生}{parti}
\entry{洛伦兹群覆盖群 SL(2,C) 的不可约表示}{qed3}
\entry{相互作用表象}{Ipic}
\entry{散射理论与 S 矩阵}{Smat}
\entry{狄拉克方程}{qed4}
\entry{狄拉克场}{Dirac}
\entry{狄拉克方程的自由粒子解}{diracs}
\entry{Weyl 旋量}{Weyl}
\entry{自旋求和}{spinsu}
\entry{狄拉克场的量子化}{Diracq}
\entry{量子化狄拉克场}{quandi}
\entry{狄拉克矩阵}{diracm}

\chapter{高能物理}
%---------------------------------------
\entry{基本粒子}{BasPar}

\chapter{弦理论}
%---------------------------------------
\entry{引力量子化}{QuanGR}
\entry{弦论概述}{STover}
\entry{弦论的种类}{TYPEst}
\entry{BRST 量子化}{BRST}
\entry{RNS 超弦}{RNS}
% \chapter{经典弦 1}
%---------------------------------------
% \chapter{经典弦 2}
%---------------------------------------
% \chapter{弦量子化}
%---------------------------------------
% \chapter{共形场论 1}或者叫规范场论?
%---------------------------------------
% \chapter{BRST 量子化}
%---------------------------------------
% \chapter{RNS 超弦}
%---------------------------------------
% \chapter{紧致化和T对偶}
%---------------------------------------
% \chapter{超弦理论(续)}
%---------------------------------------
% \chapter{超弦理论(总结)}
%---------------------------------------
% \chapter{第II型弦论}
%---------------------------------------
% \chapter{Heterotic 弦论}
%---------------------------------------
% \chapter{D-膜}
%---------------------------------------
% \chapter{黑洞}
%---------------------------------------
% \chapter{全息原理和AdS/CFT}
%---------------------------------------
% \chapter{弦理论和宇宙学}

\chapter{宇宙学}
%---------------------------------------
\entry{宇宙的演化}{UniEvo}
\entry{宇宙学红移}{CoReSh}
\entry{宇宙中的距离}{DisCos}
\entry{Friedmann-Robertson-Walker (FRW) 度规}{FRW}

\chapter{宇宙学扰动}
\entry{标量扰动}{ScaPT}
\entry{张量扰动}{TenPT}

\chapter{引力波}
\entry{引力波的几何描述}{Geomet}
\entry{TT 规范}{TTGaug}
\entry{引力波和测试质量的相互作用}{intera}

\part{科学计算}
%======================================
\chapter{Matlab 语言}
%---------------------------------------
\entry{科学计算导航}{NumPhy}
\entry{Matlab 简介}{Matlab}
\entry{Matlab 的变量与矩阵}{MatVar}
\entry{Matlab 的判断与循环}{MIfFor}
\entry{Matlab 的函数}{MatFun}
\entry{Matlab 画图}{MatPlt}
\entry{Matlab 的程序调试及其他功能}{MatOtr}
\entry{Matlab 性能优化(profiling)}{MLprof}
\entry{Matlab 画箭头矢量场图}{MQuivr}
\entry{Matlab 球坐标中的分布图}{MatPol}
\entry{Matlab 符号计算和变精度计算}{MatSym}
\entry{双精度和变精度浮点数测试(Matlab)}{FltMat}
\entry{Matlab 的稀疏矩阵}{MatSpa}
\entry{Julia 分形}{julias}
\entry{用 Matlab 手动提取图片中的曲线坐标}{plt2xy}
\entry{用 Matlab 制作 gif 动画}{MatGif}
\entry{用 Matlab 生成 mp4 视频}{MatMp4}
\entry{Matlab 的 Table 类型}{MatTab}

\chapter{Python 语言}
%---------------------------------------
\entry{Python 简介}{Python}
\entry{Python 基本变量类型}{PyType}
\entry{Python 字符串处理}{PyStr}
\entry{Python 数据类型}{PyData}
\entry{Numpy 库}{numpy}
\entry{Python 文件读写}{PyFile}
\entry{Python 判断与循环}{PyIfFr}
\entry{Python 函数}{PyFunc}
\entry{Python 画图}{PyPlot}
\entry{Python 的类}{PyClas}
\entry{Python 模块}{PyMod}
\entry{Python 异常处理}{PyExcp}
\entry{SciPy 数值微分与积分}{SciPy}
\entry{SciPy 最小二乘法}{PyFit}
\entry{SciPy 求解常微分方程组的初值问题}{PyIVP}
\entry{python符号计算}{SymPy}

\chapter{Mathematica 语言}
%--------------------------------------
% 未完成 Mathematica
% 未完成 Wolfram Alpha
\entry{Mathematica 入门笔记}{Mma}
\entry{Mathematica 文件操作}{mmaio}
\entry{Mathematica 脚本模式}{mmacmd}
\entry{Mathematica 制作和使用程序包}{mma123}

\chapter{Linux 系统}
%---------------------------------------
\entry{Linux 基础}{Linux}
\entry{Vim 笔记}{Vim}
\entry{在 Linux 上编译 C/C++ 程序}{linCpp}
\entry{g++ 编译器笔记}{gpp}
\entry{Makefile 简介}{Make0}
\entry{Makefile 笔记}{Make}
\entry{SSH 笔记}{SSH}
\entry{FTP/SFTP 笔记}{SFTP}
\entry{Shell 编程笔记}{Bash}
\entry{g++ 编译器创建静态和动态链接库}{gppLib}
\entry{Linux 创建网络文件夹(命令行)}{NFS}

\chapter{C++ 语言}
%---------------------------------------
\entry{C++ 基础}{Cpp0}
\entry{C++ 的整数}{cppInt}
\entry{C++ 函数}{CppFun}
\entry{C/C++ 多文件编译}{cppFil}
\entry{C++ 的 namespace}{cppNsp}
\entry{C++ 异常处理}{cppExc}
\entry{调试 C++ 程序}{gdbcpp}
\entry{CMake 笔记}{CMakeN}
\entry{OpenMP 笔记}{OpenMP}
\entry{MPI 笔记(C++)}{MPIcpp}
% \entry{C++ 的优化技巧}{} % 尽量避免分配内存, 使用 workspace
\entry{数据结构:密矩阵}{MatSto}
\entry{数据结构:带对角矩阵}{BanDmt}
\entry{C++ 矩阵类的实现}{CppMat}
\entry{SLISC 库概述}{SLISC}
\entry{SLISC 的密矩阵类}{SliMat}
\entry{SLISC 矩阵的基本运算}{SliAri}
\entry{SLISC 密矩阵的切割}{SliCut}
\entry{SLISC 的 Mcoo 矩阵类}{Mcoo}
\entry{SLISC 的 band 矩阵类}{SliBan}
\entry{SLISC 的 bit 操作工具}{SliBit}
\entry{SLISC 的文件读写}{Sfile}
\entry{SLISC 的计时工具}{SliTim}
\entry{SLISC 的 matt/matb 文件格式}{matb}
\entry{C++ 中的 SFINAE 技巧}{SFINAE}
\entry{Visual C++ 的简单画图库 MatPlot}{MtPlot}
\entry{C++ Boost 库笔记}{Boost}
\entry{BLAS 简介}{BLAS}
\entry{Lapack 笔记}{Lapack}
\entry{Eigen (C++ 线性代数库)笔记}{Eigen}
\entry{GNU Scientific Library}{GSL}
\entry{Arb 任意精度计算库}{ArbLib}
\entry{Arpack++2 大型本征方程库}{Arpkpp}
\entry{双精度和高精度浮点数测试(C++)}{FltCpp}
\entry{PETSc 库简介}{PETSc}

\chapter{数据结构 C++}
%---------------------------------------
\entry{单链表}{List}
\entry{双链表}{DList}

\chapter{Julia 语言笔记}
%---------------------------------------
\entry{Julia 入门笔记}{julia}
\entry{Julia 的数据类型}{JuType}
\entry{Julia 的函数}{JuFunc}

\chapter{Julia 编程基础}
%---------------------------------------
\entry{初识 Julia}{JuC1S1}
\entry{Julia 安装和启动}{JuC1S2}
\entry{编写第一个 Julia 程序}{JuC1S3}
\entry{改进第一个 Julia 程序}{JuC1S4}
\entry{Julia 第 1 章小结}{JuC1S5}
\entry{Julia 的 REPL 环境及其用法}{JuC2S1}
\entry{Julia 程序包与环境配置}{JuC2S2}
\entry{Julia 项目的创建与引入}{JuC2S3}
\entry{Julia 第 2 章小结}{JuC2S4}
\entry{Julia 的变量与常量}{JuC3S0}
\entry{Julia 变量的定义}{JuC3S1}
\entry{Julia 变量的命名}{JuC3S2}
\entry{Julia 变量的类型}{JuC3S3}
\entry{Julia 常量}{JuC3S4}
\entry{Julia 第 3 章小结}{JuC3S5}
\entry{Julia 的类型系统}{JuC4S0}
\entry{Julia 类型系统概述}{JuC4S1}
\entry{Julia 的类型与值}{JuC4S2}
\entry{Julia 的两个特殊类型}{JuC4S3}
\entry{Julia 的三种主要类型}{JuC4S4}
\entry{Julia 第 4 章小结}{JuC4S5}
\entry{Julia 的数值与运算}{JuC5S0}
\entry{Julia 的数值类型}{JuC5S1}
\entry{Julia 整数}{JuC5S2}
\entry{Julia 浮点数}{JuC5S3}
\entry{Julia 的复数和有理数}{JuC5S4}
\entry{Julia 常用的数学运算}{JuC5S5}
\entry{Julia 数值类型的提升}{JuC5S6}
\entry{Julia 数学函数速览}{JuC5S7}
\entry{Julia 第 5 章小结}{JuC5S8}
\entry{Julia Unicode 字符}{JuC6S1}
\entry{Julia 字符}{JuC6S2}
\entry{Julia 字符串}{JuC6S3}
\entry{Julia 非常规的字符串值}{JuC6S4}
\entry{Julia 第 6 章小结}{JuC6S5}
\entry{Julia 参数化类型}{JuC7S0}
\entry{Julia 类型的参数化}{JuC7S1}
\entry{Julia 参数化的更多知识}{JuC7S2}
\entry{Julia 容器:元组}{JuC7S3}
\entry{Julia 第 7 章小结}{JuC7S4}
\entry{Julia 字典与集合}{JuC8S0}
\entry{Julia 索引与迭代}{JuC8S1}
\entry{Julia 标准字典}{JuC8S2}
\entry{Julia 集合}{JuC8S3}
\entry{Julia 通用操作}{JuC8S4}
\entry{Julia 第 8 章 小结}{JuC8S5}
\entry{Julia 容器:数组(上)}{JuC9S0}
\entry{Julia 类型}{JuC9S1}
\entry{Julia 数组的表示}{JuC9S2}
\entry{Julia 数组的构造}{JuC9S3}
\entry{Julia 数组的基本要素}{JuC9S4}
\entry{Julia 访问数组元素值}{JuC9S5}
\entry{Julia 修改数组元素值}{JuC9S6}
\entry{Julia 第 9 章 小结}{JuC9S7}
\entry{Julia 广播式的修改}{JuCAS1}
\entry{Julia 元素值的排序}{JuCAS2}
\entry{Julia 数组的拷贝}{JuCAS3}
\entry{Julia 数组的拼接}{JuCAS4}
\entry{Julia 数组的比较}{JuCAS5}
\entry{Julia 再说数组的构造}{JuCAS6}
\entry{Julia 第10章 小结}{JuCAS7}
\entry{Julia 流程控制}{JuCBS0}
\entry{Julia 最简单的代码块}{JuCBS1}
\entry{Julia if 语句}{JuCBS2}
\entry{Julia for 语句}{JuCBS3}
\entry{Julia while 语句}{JuCBS4}
\entry{Julia let 语句}{JuCBS5}
\entry{Julia 错误的报告与处理}{JuCBS6}
\entry{Julia 第11章 小结}{JuCBS7}


\chapter{数值计算理论}
%---------------------------------------
% 转载自专栏 https://www.zhihu.com/column/c_1226443594048942080
% 未完成
\entry{数值计算的误差}{NumErr}
\entry{计算机算数}{CmArit}
\entry{数值解线性方程组(入门)}{NLinEq}
\entry{数值解线性方程组(进阶)}{NLinE2}
\entry{数值解线性方程组(高级)}{NLinE3}
\entry{数值解常微分方程(入门)}{NordEq}

\chapter{数值验证及常用算法}
%---------------------------------------
\entry{二项式定理(非整数幂)的数值验证}{BiNorM}
\entry{二分法}{Bisec}
\entry{多区间二分法}{MBisec}
\entry{冒泡法}{Bubble}
\entry{高斯消元法程序}{GauEli}
\entry{坐标定轴旋转程序(Matlab)}{turnM}
\entry{多项式插值}{InterP}
\entry{Nelder-Mead 算法}{NelMea}
\entry{Matlab 最小二乘法拟合函数}{CurFit}
\entry{Matlab 最小二乘法拟合多项式}{LSpoly}
\entry{数值积分(梯形法)}{NumInt}
\entry{稀疏矩阵}{SprMat}
\entry{函数求值}{SpcFun}
\entry{离散傅里叶变换}{DFT}
\entry{离散正弦变换}{DST}
\entry{Cholesky 分解}{Choles}
\entry{傅里叶变换的数值计算(Matlab)}{FFTft}
\entry{用傅里叶级数画曲线(Matlab)}{FFTdrw}
\entry{QR 分解}{QRdeco}
\entry{双共轭梯度法解线性方程组}{ConGra}

\chapter{计算机图形学}
%---------------------------------------
\entry{计算机图形学}{cg}
\entry{图像坐标系}{imgFrm}
\entry{建模坐标系(局部坐标系)}{cgmc}
\entry{世界坐标系}{Worcod}
\entry{三维投影}{proj3D}
\entry{相机模型}{CamMdl}
\entry{由图像坐标计算射线}{mn2lin}
\entry{计算 3D 艺术画}{art3D}
\entry{相机的定位}{CamLoc}
\entry{长方形相机定位法}{RecCam}
\entry{解三棱锥 1 (Matlab)}{Pmd1}
\entry{足球顶点坐标的计算程序(Matlab)}{foot60}
\entry{刚体的几何运算(Matlab)}{RigBMa}
% \entry{观察流水线}{ViPip}

\chapter{微分方程数值解}
%---------------------------------------
\entry{简谐振子受迫运动的简单数值计算}{SHOFN}
\entry{天体运动的简单数值计算}{KPNum0}
\entry{常微分方程(组)的数值解}{OdeNum}
\entry{中点法解常微分方程(组)}{OdeMid}
\entry{四阶龙格库塔法}{OdeRK4}
\entry{刚体转动数值模拟(Matlab)}{RBRNum}
\entry{陀螺的数值模拟(Matlab)}{TopNum}
\entry{洛伦兹吸引子}{LrzAct}
\entry{开普勒问题的数值计算(Matlab)}{KepNum}
\entry{双摆的数值计算(Matlab)}{DbPend}
\entry{天体物理中 N 体问题的数值计算(Matlab)}{NbodyM}
\entry{拉格朗日方程的数值解(Matlab)}{LagNum}

\chapter{偏微分方程数值解}
%---------------------------------------
\entry{一维波动方程的简单数值解(Matlab)}{W1dNum}
\entry{二维波动方程的简单数值解(Matlab)}{wav2dN}
\entry{单缝衍射的模拟(Matlab)}{DiffrN}
\entry{双缝干涉的模拟(Matlab)}{DbSliN}

\chapter{一维薛定谔方程数值解}
%---------------------------------------
\entry{一维有限深势阱束缚态数值解(试射法)}{BndSho}
\entry{无限深势阱中的高斯波包数值计算(Matlab)}{ISWmat}
\entry{自由高斯波包的动画绘制(Matlab)}{FreeGs}
\entry{无限深势阱中的高斯波包模拟(Matlab)}{wvISW}
\entry{简谐振子中的高斯波包模拟(Matlab)}{SHOgs}
\entry{有限深方势阱束缚态程序(Matlab)}{FSWmat}
\entry{方势垒定态波函数程序(Matlab)}{FSBplt}
\entry{高斯波包的方势垒散射数值计算(Matlab)}{FSBsct}
\entry{一维薛定谔方程不稳定的差分法数值解(Matlab)}{TDSE1N}
\entry{Crank-Nicolson 算法解一维含时薛定谔方程(Matlab)}{CraNic}
\entry{一维有限深方势阱中的光电离模拟(Matlab)}{FSWpi}

\chapter{氢原子薛定谔方程数值解}
%---------------------------------------
\entry{氢原子波函数 Matlab 画图程序}{Hplot}
\entry{球谐函数数值计算}{YlmNum}
\entry{Gauss-Lobatto 积分}{GLquad}
\entry{FEDVR 网格}{FEDVR}
\entry{Lanczos 算法}{Lanc}
\entry{指数格点}{ExpGrd}
\entry{虚时间法求基态波函数}{ImgT}
\entry{氢原子球坐标薛定谔方程数值解}{HyTDSE}
\entry{氢原子球坐标薛定谔方程数值解 2}{HTDSE}

\chapter{氦原子薛定谔方程数值解}
%---------------------------------------
\entry{氦原子数值解 TDSE 笔记}{HeTDSE}
\entry{氦原子波函数数值分析}{HeAnal}
\entry{Berkeley-ECS 方法}{BerECS}

\chapter{人工智能与机器学习}
%---------------------------------------
\entry{概率密度函数与人工智能概论}{AIstat}
\entry{欠拟合}{unfit}
\entry{过度拟合}{ovfit}
\entry{强化学习}{rl}
\entry{生成对抗网络}{GAN}

\chapter{其他}
%---------------------------------------
\entry{原码、反码、补码}{InvCom}
\entry{文本文件与字符编码}{encode}
\entry{正则表达式}{regex}
\entry{LaTeX 结构简介}{latxIn}
\entry{安装使用 TeXlive}{TeXliv}
\entry{JavaScript 入门笔记}{JS}
\entry{jQuery 笔记}{jQuery}
\entry{校验和}{chkSum}
\entry{GitHub Desktop 的简单使用}{GitHub}
\entry{Fortran 入门笔记}{Fortra}
\entry{Git 笔记}{Git}
\entry{Git 服务器搭建}{GitSer}
\entry{cuBLAS 库}{cublas}
\entry{库仑函数程序(Matlab 和 Mathematica)}{FlCode}
\entry{Blender 笔记}{Blendr}
\entry{数字货币简介}{crypto}
\entry{使用数字货币钱包}{CryWal}
% \entry{用百度网盘安全高效地备份文件}{PanBak}
\entry{字长}{WordLe}
\entry{算术逻辑单元}{ALU}
\entry{随机存储器}{RAM}
\entry{麦克斯韦—玻尔兹曼分布的简单数值模拟}{MBdisN}
\entry{麦克斯韦—玻尔兹曼分布的数值模拟}{MaxwD}
\entry{N 体问题软件(天体物理)}{Nbody}
\entry{SQLite 教程}{SQLite}
\entry{Docker 笔记}{Docker}

\part{考研}
%======================================
\chapter{普通物理}
%---------------------------------------
\entry{中国科学院 2012 年考研普通物理}{CAS12}
\entry{中国科学院 2013 年考研普通物理}{CAS13}
\entry{中国科学院 2014 年考研普通物理}{CAS14}
\entry{中国科学院 2017 年考研普通物理}{CAS17}
\entry{中国科学院 2018 年考研普通物理}{CAS18}
\entry{中国科学院 2020 年考研普通物理}{CAS20}
\entry{北京师范大学 2012 年考研普通物理}{BNU12}
\entry{北京师范大学 2013 年考研普通物理}{BNU13}
\entry{北京师范大学 2016 年考研普通物理}{BNU16}
\entry{中国科技大学 2013 年考研普通物理}{USTC13}
\entry{中国科技大学 2014 年考研普通物理}{USTC14}
\entry{中国科技大学 2015 年考研普通物理(B)}{USTC15}
\entry{中国科技大学 2016 年考研普通物理}{USTC16}
\entry{复旦大学 2015 年考研普通物理}{FDU15}
\entry{南京大学 2012 年考研普通物理}{NJU12}
\entry{南京大学 2013 年考研普通物理}{NJU13}
\entry{南京大学 2014 年考研普通物理}{NJU14}
\entry{南京大学 2015 年考研普通物理}{NJU15}
\entry{南京大学 2016 年考研普通物理}{NJU16}
\entry{南京大学 2017 年考研普通物理}{NJU17}
\entry{南京大学 2018 年考研普通物理}{NJU18}
\entry{南京大学 2019 年考研普通物理}{NJU19}

\chapter{量子力学}
%---------------------------------------------
\entry{天津大学 2011 年考研量子力学}{TJU11}
\entry{天津大学 2011 年考研量子力学答案}{TJU11A}
\entry{天津大学 2012 年考研量子力学}{TJU12}
\entry{天津大学 2012 年考研量子力学答案}{TJU12A}
\entry{天津大学 2013 年考研量子力学}{TJU13}
\entry{天津大学 2014 年考研量子力学}{TJU14}
\entry{天津大学 2014 年考研量子力学答案}{TJU14A}
\entry{天津大学 2015 年考研量子力学}{TJU15}
\entry{天津大学 2015 年考研量子力学答案}{TJU15A}
\entry{天津大学 2016 年考研量子力学}{TJU16}
\entry{天津大学 2016 年考研量子力学答案}{TJU16A}
\entry{天津大学 2017 年考研量子力学}{TJU17}
\entry{天津大学 2017 年考研量子力学答案}{TJU17A}

\chapter{计算机科学与技术}
%---------------------------------------------
\entry{2009 年计算机学科专业基础综合全国联考卷}{CSN09}
\entry{2010 年计算机学科专业基础综合全国联考卷}{CSN10}
\entry{2011 年计算机学科专业基础综合全国联考卷}{Na11}
\entry{2012 年计算机学科专业基础综合全国联考卷}{CSN12}
\entry{2013 年计算机学科专业基础综合全国联考卷}{CSN13}
\entry{2014 年计算机学科专业基础综合全国联考卷}{CSN14}
\entry{2015 年计算机学科专业基础综合全国联考卷}{CSN15}
\entry{2016 年计算机学科专业基础综合全国联考卷}{CSN16}
\entry{2017 年计算机学科专业基础综合全国联考卷}{CSN17}
\entry{2018 年计算机学科专业基础综合全国联考卷}{CSN18}

\part{附录}
%======================================
\chapter{附录}
%---------------------------------------
\entry{小时百科符号与规范}{Conven}
\entry{常见物理量}{PhyQty}
\entry{国际单位制}{SIunit}
\entry{物理学常数}{Consts}
\entry{国际单位制词头}{UniPre}
\entry{量类和单位}{QCU}
\entry{量类的延拓}{QCC}
\entry{单位制和量纲}{USD}
\entry{量纲式}{DIMF}
\entry{现象类}{PHEC}
\entry{量纲空间}{DimS}
\entry{小时百科图标}{xwLogo}

% \part{隐藏内容}
% %======================================

% \chapter{数学}
% %---------------------------------------
% \entry{Quaternion and Rotation}{QuaRot}

% \chapter{力学}
% %---------------------------------------
% \entry{角动量简介}{CM2}
% \entry{正则变换}{CanTra}
% \entry{重心}{CenG}
% \entry{经典系统的线性响应理论}{LiReC1}

% \chapter{光学}
% %---------------------------------------
% \entry{透镜的主平面和节平面}{LnsPln}

% \chapter{电动力学}
% %---------------------------------------
% \entry{电流产生磁场}{CurMag}
% \entry{用狄拉克 delta 函数表示点电荷的散度}{CEfDiv}

% \chapter{量子力学}
% %---------------------------------------
% \entry{伴随算符}{adjoin}
% \entry{自伴算符、厄米算符}{HerOp}

% \chapter{计算机}
% %---------------------------------------
% \chapter{其他}
% \entry{微分几何入门与广义相对论笔记}{DGGRNt}
% \entry{二极管}{Diode}
% \entry{偶极子近似(量子)}{DipApr}
% \entry{能均分定理}{EqEng}
% \entry{单缝的夫琅禾费衍射}{FD}
% \entry{分布函数的数值拟合}{FitPdf}
% \entry{有限深势阱中的双粒子}{FSWtwo}
% \entry{函数空间}{FunSpc}
% \entry{伽利略变换}{GaliTr}
% \entry{万有引力和天体运动}{Grav0}
% \entry{同构}{homomo}
% \entry{一维和二维氢原子模型势能}{Hy1D2D}
% \entry{L2 函数空间}{L2FunS}
% \entry{刘维尔定理}{LiouTh}
% \entry{代数矢量}{NumVec}
% \entry{用百度网盘安全高效地备份文件}{PanBak}
% \entry{命题及其表示法}{Propos}
% \entry{量子霍尔效应}{QHallE}
% \entry{给高中生的量子力学简介}{QMIntr}
% \entry{反射和折射、斯涅尔定律}{Reflec}
% \entry{环形电流的磁场}{RingB}
% \entry{TDSE Open Boundary Condition}{SEopBC}
% \entry{心脏螺旋波}{spiral}
% \entry{个人目录}{t}
% \entry{Tensor}{TestT}
% \entry{矢量积分}{VecInt}
% \entry{Wolfram Alpha 简介}{WolfAl}
% \entry{半波损失}{WvLost}

% \chapter{不属于百科}
% %---------------------------------------
% \entry{test}{test}
% \entry{公司简介}{Compny}
% \entry{关于小时百科}{about}
% \entry{小时百科用户协议}{EUL}
% \entry{小时百科创作协议}{licens}
% \entry{LaTeX 云笔记协议}{NtLcns}
% \entry{网盘协议}{PanLcs}
% \entry{小时百科编辑器简介}{editIn}
% \entry{小时百科创作指导}{WrGuid}
% \entry{小时百科编写规范}{rules}
% \entry{词条示例}{Sample}
% \entry{小时百科模板使用说明}{templt}
% \entry{编辑器使用说明}{editor}
% \entry{编辑器事项}{edTODO}
% \entry{小时百科 App 设计}{app}
% \entry{小时百科发展规划}{plan}
% \entry{小时百科程序框架简介}{myCode}

\bibli
\end{document}
