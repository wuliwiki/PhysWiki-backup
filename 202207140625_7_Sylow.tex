% Sylow定理
% keys 西罗定理|西罗子群|Sylow子群|群论|有限群

\pentry{群作用\upref{Group3}}

拉格朗日定理(\autoref{coset_the2}~\upref{coset})揭示了子群阶数的特点.可惜的是,其逆命题“如果$n\mid \abs{G}$则$G$总有阶数为$n$的子群”是不成立的.比如说,交错群$A_4$就没有$6$阶的子群——你可以动手验证这一点.

但是,挪威数学家Peter Ludwig Sylow于1872年发表的文章\footnote{L. Sylow, Th´eor`emes sur les groupes de substitutions, Mathematische Annalen 5 (1872), 584–594. 该文见 https://eudml.org/doc/156588. Robert Wilson 对此文的英文翻译见 http://www.maths.
qmul.ac.uk/~raw/pubs_files/Sylow.pdf.}指出,在$n$是素数或者素数的幂时,拉格朗日定理逆命题是成立的.同时他还发现了所谓的Sylow子群全都是彼此共轭的.

举个例子:考虑阶数为$300$的群$G$,对$300$进行素因子分解得$300=2^2\cdot 3^2\cdot 5^2$,那么阶数为$2^2$的子群总存在,且它们彼此共轭.不过,$2$阶子群虽存在,却不能总保证所有$2$阶子群彼此共轭.


\begin{definition}{Sylow子群}
给定群$G$和它的子群$H$.如果$\abs{H}=p^k$,其中$p$是素数,且$p^{k+1}\not\mid\abs{G}$,那么称$
\end{definition}


















