% 表面张力
% keys 表面张力|液体|表面能

\pentry{亥姆霍兹自由能\upref{HelmF}}

\subsection{表面张力}

液体与自己的蒸汽或另一种介质接触的交界面为液体的表面.实验表明,对液体表面的每一处,假想画一条线,则该线两侧的页面相互存在拉力的作用,该拉力垂直于此线.我们称这个\textbf{位于液面内处处与此线垂直的拉力}为\textbf{表面张力}.\textbf{表面张力系数} 的意思是:单位长度的直线两侧液面的相互拉力.表面张力系数用字母 $\sigma$ 表示.

例如:一个宽 $L$ 的导轨上,用可移动的铁丝拉出一块长方形肥皂泡,铁丝受肥皂泡液面的拉力为 $\sigma L$;由于肥皂泡有两个表面,所以铁丝受到肥皂泡的总拉力为 $2\sigma L$.如果将铁丝移动 $\Delta x$ 的距离使得肥皂泡的面积扩大 $L\Delta x$,就要对铁丝做功 $2\sigma L\Delta x$.注意肥皂泡的\textbf{表面总面积}为 $\Delta S=2 L \Delta x$,所以我们得到做功的表达式:
\begin{equation}
\Delta W=\sigma \Delta S
\end{equation}
该式不仅对上述例子成立,对任意形状的液面都成立.例如我们将肥皂泡吹大,受表面张力的影响,肥皂泡内部的气压大于外部的气压,在吹的过程中肥皂泡被做的功为 $\sigma\cdot 4\pi R^2$.

表面张力系数随温度的升高而降低.一般在分析表面张力的时候,我们通常假定温度不变.

\subsection{表面能}

在分析表面张力的时候,我们考虑的是恒温系统,而且通常我们再将整个大系统看成一个恒定体积的封闭容器.于是根据自由能判据\upref{HelmF},在稳定平衡状态下
\begin{equation}
\delta F=0,\delta^2 F>0
\end{equation}

通过这个判据我们可以推出许多结论.例如容器中充满液体 $A$,液体 $A$ 中悬浮着一滴半径为 $r$ 的液体 $B$.

\subsection{一些应用}
拉普拉斯公式:
\begin{equation}\label{sftens_eq1}
\Delta p = \sigma \left(\frac{1}{r_1}+\frac{1}{r_2}\right)
\end{equation}
待补充:
拉普拉斯公式
弯曲液面附加压强(对饱和蒸气压的影响)
对液滴的分析(临界半径),过饱和蒸汽,沸腾,过热液体