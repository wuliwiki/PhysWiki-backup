% Python C API 笔记
% license Usr
% type Note

% 参考 baltam_python_api 仓库

\begin{itemize}
\item \verb`PyObject *` 可以容纳任何 python 对象
\item \verb`Py_Initialize()` 和 \verb`Py_Finalize()` 开始和结束 python api (相当于一个 REPL session)。
\item \verb`PyObject *pModule = PyImport_Import(pName);` 可以加载 python 模块。
\item \verb`Py_DECREF(PyObject *obj);` 是一个宏,可以把 \verb`obj` 对象的引用计数减一, 当减少到零该对象就会自动销毁。\verb`Py_INCREF(PyObject *obj);` 把引用加一。
\item \verb`PyLong_Check(obj)` 检查一个对象是否为 \verb`Long`, \verb`PyFloat_Check(obj)` 检查是否为浮点数, \verb`PyUnicode_Check(obj)` 检查是否为字符串, 另外有 \verb`PyList_Check(obj)`, \verb`PyDict_Check(obj)`
\item \verb`PyObject* PyObject_GetAttrString(PyObject *obj, const char *attr_name);` 获取对象的属性(attribute)。 模块也是对象,如 \verb`math.pi` 就是模块 \verb`math` 对象的 \verb`pi` 属性。 它也可以用于获取函数如 \verb`math.sin`。 如果失败,返回 \verb`NULL`, 可以用 \verb`PyErr_Print()` 打印错误信息。
\item \verb`PyObject* PyObject_CallFunctionObjArgs(PyObject *callable, ...);` 可以调用函数,后面是函数的参数,以 \verb`NULL` 结尾。
\end{itemize}

\subsection{变量}
创建变量
\begin{itemize}
\item \verb`PyObject *pName = PyUnicode_DecodeFSDefault("my_python_module");` 可以创建一个 python 字符串变量。
\item \verb`PyObject *obj = PyLong_FromLong(42);` 创建 python 整数。
\item \verb`PyFloat_AsDouble()` 创建浮点数。
\item \verb`PyObject *int_obj = Py_BuildValue("i", 42);` 也可以创建 python 整数,该函数更动态更灵活,例如 \verb`PyObject *tuple_obj = Py_BuildValue("(is)", 42, "Hello");` 创建一个整数和字符串的 touple。
\item 创建 list: \verb`PyObject *arg = PyList_New(2);`, \verb`PyList_SetItem(arg, 0, PyLong_FromLong(1));`, \verb`PyList_SetItem(arg, 1, PyLong_FromLong(2));`
\end{itemize}

\begin{itemize}
\item \verb`PyArg_Parse(obj, "i", &result_c);` 从 \verb`PyObject *obj` 获取整数存到 \verb`int result_c` 中。
\end{itemize}

