% 域上的代数
% 代数|环|域|乘法|群

\pentry{矢量空间\upref{LSpace},环\upref{Ring}}

“代数(algebra)”一词,有两个含义.

第一个含义是指一个数学分支,代数学.代数学研究集合中各种各样的运算结构,我们从小学就开始涉及了.一次方程的移项、两边同乘等操作都是代数学研究的性质;理工科到了本科还必须研究线性代数(或称高等代数),作为大量理工学科的数学基础;此外,本部分“抽象代数”则研究了更为基础的一些代数学,但抽象代数本身也只是代数学这门广博学科的起点.

第二个含义,是代数学中研究的一种数学对象,代数.简单来说,代数就是一种定义了向量乘法的向量空间.当然,乘法的性质要具体讨论.



\begin{definition}{狭义的代数定义}
设$A$是域$\mathbb{K}$上的向量空间,若在$A$中再定义代数乘法$\circ$,使得$(A,+,\circ)$成为环,并且$\forall a\in \mathbb{K}, u, v\in A$有
\begin{equation}
a(u \circ v)=(a u) \circ v=u{\circ}(a v)
\end{equation}
则称$A$为\textbf{域$\mathbb{K}$上的代数},简称\textbf{(结合)代数}.

\end{definition}

如果没有特别说明,一个代数一般是指狭义的结合代数,也就是说,是在向量空间里定义向量的乘法,使得它还能构成一个环.当然,乘法只是一种运算,它不一定是结合的.如果所定义的乘法不是结合的,那么我们称这样的结构是一个\textbf{非结合代数}.


\subsection{代数的例子}
下面我们来看几个例子,加深对代数这一概念的理解.

\subsubsection{结合代数}

\begin{example}{矩阵代数}
域$\mathbb{K}$上的$n\times n$的矩阵的全体 $\mathrm{GL}(n, \mathbb{K})$ (\autoref{Group_ex5}~\upref{Group})在矩阵的加法,标量与矩阵的相乘运算下构成一个线性空间,在矩阵的加法和矩阵的乘法运算下构成一个环,因此它是一个结合代数.
\end{example}

\begin{example}{}
对于域$\mathbb{K} $及有限群$G=\{g_1,g_2,\cdots,g_n\}$,我们可构成\textbf{群代数}$\displaystyle A(G)=\left\{u | u=\sum a^{i} g_{i}, a^{i} \in \mathbb{K}\right\}$,其中加法为
\begin{equation}
u+v=\sum a^{i} g_{i}+\sum b^{i} g_{i}=\sum\left(a^{i}+b^{i}\right) g_{i} \notag
\end{equation}
数乘为
\begin{equation}
a u=\sum\left(a a^{i}\right) g_{i}, a \in \mathbb{K} \notag
\end{equation}
而代数乘法为
\begin{equation}
u \circ v=\left(\sum a^{i} g_{i}\right) \circ\left(\sum b^{i} g_{i}\right)=\sum_{i, j} a^{i} b^{j}\left(g_{i} g_{j}\right) \notag
\end{equation}
这是一个结合代数.
\end{example}

\subsubsection{李代数}

\begin{example}{三维实线性李代数}
在$3$维实向量空间中,以两个向量的叉积$\mathbf A\times \mathbf B$来定义它们的代数乘法运算,则它们构成一个代数,此时有
\begin{equation}
\begin{aligned}
&\mathbf{A} \times \mathbf{A}=0\\
&\mathbf{A} \times \mathbf{B}=-\mathbf{B} \times \mathbf{A}\\
&\mathbf{A} \times(\mathbf{B}+\mathbf{C})=\mathbf{A} \times \mathbf{B}+\mathbf{A} \times \mathbf{C}\\
&(\bvec B+\mathbf{C}) \times \mathbf{A}=\mathbf{B} \times \mathbf{A}+\mathbf{C} \times \mathbf{A}
\end{aligned}
\end{equation}
以及
\begin{equation} \label{AlgFie_eq1}
(\mathbf{A} \times \mathbf{B}) \times \mathbf{C}+(\mathbf{B} \times \mathbf{C}) \times \mathbf{A}+(\mathbf{C} \times \mathbf{A}) \times \mathbf{B}=0
\end{equation}
$3$维向量空间配上向量外积作为乘法,得到一个非结合代数,三维实线性李代数.
\end{example}

通常把\autoref{AlgFie_eq1} 称为\textbf{雅可比恒等式}. 具有这种性质的代数称为\textbf{李代数(Lie Algebra)}.我们会在将来详细讨论李代数.

\begin{example}{}
$\mathrm{gl}(n, \mathbb C)$除矩阵的加法及数乘外,再定义代数乘法$\mathbf A\circ \mathbf B=[\mathbf A, \mathbf B] = \mathbf A \mathbf B - \mathbf B\mathbf A, \mathbf A, \mathbf B\in \mathrm{gl}(n, \mathbb C)$.显然$\mathrm{gl}(n,\mathbb C)$构成一个李代数.
\end{example}


下面一个例子需要一些分析力学中的内容.如果不知道分析力学也没关系,记住即可.
\begin{example}{}
在分析力学中,正则变量记为$p_i,q_i, i=1,2,\cdots, s$.函数$u(p,q)$和$v(p, q)$的\textbf{泊松括号}定义为\begin{equation}
\{u, v\}=\sum_{i=1}^{s}\left(\frac{\partial u}{\partial q_{i}} \frac{\partial v}{\partial p_{i}}-\frac{\partial u}{\partial p_{i}} \frac{\partial v}{\partial q_{i}}\right)
\end{equation}
于是我们得到了一个李代数.
\end{example}

\subsubsection{外代数}
\pentry{直和\upref{DirSum}}

\begin{example}{外代数}
给定域$\mathbb{K}$上的$n$维线性空间$V$,任取其一组基$\{e_i\}^n_{1}$.我们进行以下构造:
\begin{itemize}
\item 对于每两个元素$\bvec{v}, \bvec{u}\in V$,我们定义一个新的元素$\bvec{v}\wedge \bvec{u}$,只要求它满足\textbf{线性性}和\textbf{反对称性}\footnote{线性性即$\forall a_1\bvec{v}_1+a_2\bvec{v}_2,b_1\bvec{u}_1+b_2\bvec{u}_2\in V$,必有$$}.这样,对于两个非零且不相等的$v,u\in V$,$v\wedge u$是一个不在$V$中的新元素;当$v=u$或其中之一为$0$时,$v\wedge u$是$0\in V$.我们用集合$\{e_i\wedge e_j\}_{i\not=j}$作为基底,在域$\mathbb{K}$上构造一个新的线性空间,记为$A^1(V)$.由于该运算得到的元素要么是$0$,要么是$V$之外的元素,因此我们把它称为\textbf{外积(exterior product)}.

\item 将以上步骤推广,对于每$k$个元素$v_i\in V$,我们定义一个新的元素$v_1\wedge v_2\wedge\cdots\wedge v_k$,要求它满足\textbf{反对称性}和\textbf{结合性},即$(v_1\wedge v_2)\wedge v_3=v_1\wedge(v_2\wedge v_3)$,这样就可以摆脱括号而直接将这三个元素的外积表示为$v_1\wedge v_2\wedge v_3$.$k$个$\{e_i\}$的外积中只有$C^k_n$个不为零,用这些不为零的外积作为基底在$\mathbb{K}$上构造一个线性空间,记为$A^k(V)$.特别地,$V=A^0(V)$.

\item 由于反对称性,若干元素中只要有两个相等,它们的外积就为零.这就意味着我们最多可以取到$A^n(V)$,而对于$m>n$,都有$A^m(V)=\{0\}$.这样,将各$A^k(V)$视为互相不同的线性空间,其中只有$n+1$个非平凡,它们作为向量空间的直和$\bigoplus\limits_{k=0}^n A^k(V)$也是一个向量空间,记为$A(V)$.由前面的步骤可知,任取
\end{itemize}


\end{example}
