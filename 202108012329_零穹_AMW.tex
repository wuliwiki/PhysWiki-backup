% 不对称势阱
\pentry{有限深方势阱\upref{FSW}}
\begin{figure}[ht]
\centering
\includegraphics[width=7cm]{./figures/AMW_1.pdf}
\caption{一维不对称势阱} \label{AMW_fig1}
\end{figure}
本节我们来解下面的一维不对称势阱的离散谱(束缚态),\autoref{AMW_fig1} .
\begin{equation}
V(x)=\left\{\begin{aligned}
&V_1\quad(x<0)\\
&0\\
&V_2\quad(x>L)
\end{aligned}\right.
\end{equation}

\textbf{解:}对于离散谱,能量 $E$ 需小于无穷远处势能,故 $E<V_1$. 

在 $x<0$ 区域内的其薛定谔方程为
\begin{equation}
\frac{\hbar^2}{2m}\dv[2]{\psi(x)}{x}+(E-V_1)\psi(x)=0
\end{equation}
波函数为
\begin{equation}
\phi=C_1 e^{}
\end{equation}
