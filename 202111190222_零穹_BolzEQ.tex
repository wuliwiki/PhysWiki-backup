% 玻尔兹曼方程
% 玻尔兹曼方程

\begin{issues}

\issueDraft
推导未完成
\end{issues}

\pentry{刘维尔定理\upref{LiouTh}}
\footnote{朗道.物理动理学.北京:高等教育出版社,2008.}玻尔兹曼方程是动理学理论的奠基者路德维希 $\cdot$ 玻尔兹曼于1872年首先推导出来的,该方程适用于气体分子在没有外场的情形.其可表示为下面的积分微分方程的形式:
\begin{equation}\label{BolzEQ_eq1}
\pdv{f}{t}+\bvec{v}\vdot\nabla f=\int\omega'\qty(f'f_1'-ff_1)\dd\Gamma_1\dd\Gamma'\dd\Gamma_1'
\end{equation}
式中,我们用$\Gamma$ 表示分布函数所依赖的变量中除分子质心坐标(和时间 $t$ )以外的一切变量总体. $f,f'$ 是气体分子在其相空间的分布函数 $f(t,\bvec r,\Gamma)$,本文规定函数 $f$ 的附标均对应于其变量 $\Gamma$ 的附标,即$f=f(t,\bvec r,\Gamma),f'=f(t,\bvec r,\Gamma')$,等等.$\omega=\omega(\Gamma',\Gamma_1';\Gamma,\Gamma_1)$是其所有变量的函数,其对应两分子初值为 $\Gamma$ 和 $\Gamma_1$ 而结果为 $\Gamma'$ 和 $\Gamma_1'$ 的碰撞(该碰撞简记为 $\Gamma,\Gamma_1\rightarrow\Gamma',\Gamma_1'$ ).相应的,式中的 $\omega'=\omega(\Gamma,\Gamma_1;\Gamma',\Gamma_1')$. 
\subsection{函数$\omega$ 的性质}
首先声明,为书写方便,我们这里的分布函数 $f(t,\bvec r,\Gamma)$ 代表相空间中单位体积元内的平均分子数,它等于通常的分布函数 $\rho(t,\bvec r,\Gamma)$(分子处于相空间中$(\bvec r,\Gamma)$ 附件单位体积元的概率) 乘以总分子数 $N_{total}$,这并不影响我们推导玻尔兹曼方程,这也可从 $f=N\rho$ 代入\autoref{BolzEQ_eq1} 和原方程等价看出\footnote{其实主要目的是为了贴合朗道的说法}.这就是说
\begin{equation}
f\dd \bvec r\dd\Gamma
\end{equation}
给出 $(\bvec r,\Gamma)$ 处微元 $\dd\bvec{r}\dd\Gamma$中的平均分子数.显然
\begin{equation}
\int f\dd \bvec r\dd\Gamma=N_{total}
\end{equation}
气体粒子的空间分布函数 $N(t,\bvec r)$ 是
\begin{equation}
\int f(t,\bvec r,\Gamma)\dd \Gamma=N(t,\bvec r)
\end{equation}
 $N\dd V$ 是体积元 $\dd V$ 中的平均分子数.

 对于碰撞 $\Gamma,\Gamma_1\rightarrow\Gamma',\Gamma_1'$,其中,$\Gamma,\Gamma_1$ 分别在区间 $\dd\Gamma,\dd\Gamma_1$ 中. 气体单位时间在每单位体积内的这种碰撞总数,可以写成每单位体积中的分子数 $f\dd\Gamma$ 与这其中的任一分子经受该类型碰撞的概率的乘积.这一概率总是正比于单位体积中 $\Gamma_1$ 处的分子数 $f_1\dd \Gamma_1$,并且正比于碰撞后两个分子 $\Gamma$ 值所在的区间 $\dd\Gamma'$ 和 $\dd \Gamma_1'$.这就是说,这一碰撞数可写成
 \begin{equation}\label{BolzEQ_eq5}
\omega(\Gamma',\Gamma_1';\Gamma,\Gamma_1)ff_1\dd\Gamma\dd\Gamma_1\dd\Gamma'\dd\Gamma_1'
 \end{equation}
 观察上式,$f\dd \Gamma$ 的单位为 $\mathrm{m^{-3}}$,而总的单位为 $\mathrm{m^{-3}s^{-1}}$,所以 $\omega(\Gamma',\Gamma_1';\Gamma,\Gamma_1)\dd\Gamma'\dd\Gamma_1'$ 单位为 $\mathrm{m^{-3}s^{-1}m^{3}m^{3}=m^3s^{-1}}$.这表明下式具有面积的量纲
 \begin{equation}\label{BolzEQ_eq2}
 \dd \sigma=\frac{\omega(\Gamma',\Gamma_1';\Gamma,\Gamma_1)}{\abs{\bvec v-\bvec v'}}\dd\Gamma'\Gamma_1'
 \end{equation}
 其中 $\bvec v-\bvec v'$ 为两粒子相对速度.$\dd\sigma$ 即为有效碰撞截面.\\

由于力学定律的具有时间反演对称性,用 $\Gamma^T$ 表 $\Gamma$ 时间反演所得的值,而时间反演使得碰撞“前”状态与碰撞“后”状态相交换,于是
 \begin{equation}
 \omega(\Gamma',\Gamma_1';\Gamma,\Gamma_1)=\omega(\Gamma^T,\Gamma_1^T;\Gamma'^T,\Gamma_1'^T)
 \end{equation}
 
$\omega$ 函数还有一个普遍关系,它不依赖于时间反演对称性,即
\begin{equation}\label{BolzEQ_eq7}
\int\omega(\Gamma',\Gamma_1';\Gamma,\Gamma_1)\dd\Gamma'\dd\Gamma_1'=\int\omega(\Gamma,\Gamma_1;\Gamma',\Gamma_1')\dd\Gamma'\dd\Gamma_1'
\end{equation}
这一关系可用量子力学清楚的推得.由量子力学知道,各种碰撞过程的概率幅形成么正矩阵 $\Q{S}$(所谓的\textbf{散射矩阵}),其矩阵元模平方 $\abs{S_{ni}}^2$ 确定跃迁 $i\rightarrow n$ 的碰撞概率 .么正条件即
\begin{equation}
\Q{S}\Her\Q{S}=\Q{S}\Q{S}\Her=I
\end{equation}
其中,$I$ 为单位矩阵.显然,上式即
\begin{equation}
\sum_n \abs{S_{ni}}^2=\sum_n \abs{S_{in}}^2=1
\end{equation}
去掉 $n=i$ 的项(无状态变化的跃迁),得
\begin{equation}
\sum_{n\neq i} \abs{S_{ni}}^2=\sum_{n\neq i} \abs{S_{in}}^2
\end{equation}
用函数 $\omega$ 表示上式便是所要证的\autoref{BolzEQ_eq7} 
\subsection{玻尔兹曼方程的推导}
 刘维尔定理\upref{LiouTh}告诉我们,如果分子间的碰撞可以完全忽略,刘维尔定理对于分子的分布函数成立,即
 \begin{equation}\label{BolzEQ_eq3}
 \dv{f}{t}=0
 \end{equation}
 这里,全导数对应于沿分子的相轨道(链接)所取的导数.

 而当考虑碰撞时,\autoref{BolzEQ_eq3} 不在成立,分布函数不在沿相轨道为恒定.代替\autoref{BolzEQ_eq3} 应写成
 \begin{equation}\label{BolzEQ_eq4}
 \dv{f}{t}=C(f)
 \end{equation}
 式中 $C(f)$ 表示分布函数由于碰撞引起的变化率.显然,$C(f)\dd V\dd\Gamma$ 是相空间体积元 $\dd V\dd\Gamma$ 中单位时间内由于碰撞引起的分子数的改变量.具有\autoref{BolzEQ_eq4} 形式的方程称为\textbf{动理方程},而量 $C(f)$ 称为\textbf{碰撞积分}.

 每一分子的 $\Gamma$ 在经受碰撞时都会使它转移出给定的区间 $\dd\Gamma$,这一碰撞称为\textbf{损失}.而初值在给定区间 $\dd \Gamma$ 外的分子经受碰撞进入该区间,这类碰撞称为\textbf{增益}.由\autoref{BolzEQ_eq5} ,单位时间内发生在体积 $\dd V$ 中,对于一切可能的损失碰撞,碰撞总数为
 \begin{equation}
 \dd V\dd \Gamma\int\omega(\Gamma',\Gamma_1';\Gamma,\Gamma_1)ff_1\dd\Gamma_1\dd\Gamma'\dd\Gamma_1'
 \end{equation}
  而对于一切可能的增益碰撞,碰撞总数为
  \begin{equation}
  \dd V\dd \Gamma\int\omega(\Gamma,\Gamma_1;\Gamma',\Gamma_1')f'f_1'\dd\Gamma_1\dd\Gamma'\dd\Gamma_1'
  \end{equation}
  显然,单位时间在体积 $\dd V$ 中的有关分子数增加是
  \begin{equation}\label{BolzEQ_eq6}
  \dd V\dd \Gamma\int(\omega'f'f_1'-\omega f f_1)\dd\Gamma_1\dd\Gamma'\dd\Gamma_1'
  \end{equation}
  式中,$\omega=\omega(\Gamma',\Gamma_1';\Gamma,\Gamma_1),\omega'=\omega(\Gamma,\Gamma_1;\Gamma',\Gamma_1')$.

显然,\autoref{BolzEQ_eq6} 便是 $C(f)\dd V\dd\Gamma$ .于是对于碰撞积分,有下面的表达式
\begin{equation}\label{BolzEQ_eq8}
C(f)=\int(\omega'f'f_1'-\omega f f_1)\dd\Gamma_1\dd\Gamma'\dd\Gamma_1'
\end{equation}
上式被积函数中的第二项,对 $\dd\Gamma'\dd\Gamma_1'$ 的积分只与 $\omega$ 有关,因为因子 $ff_1$ 不依赖于这些变量.对这部分积分可借助\autoref{BolzEQ_eq7} 将\autoref{BolzEQ_eq8} 变为
\begin{equation}\label{BolzEQ_eq9}
C(f)=\int\omega'(f'f_1'- f f_1)\dd\Gamma_1\dd\Gamma'\dd\Gamma_1'
\end{equation}

在没有外场的情况下,自由运动分子的$\Gamma$ 保持常量.由\autoref{BolzEQ_eq4} 
\begin{equation}\label{BolzEQ_eq10}
C(f)=\dv{f}{t}=\pdv{f}{t}+\bvec v\vdot\Nabla f
\end{equation}
\autoref{BolzEQ_eq9} 和\autoref{BolzEQ_eq10} 结合便证得玻尔兹曼方程\autoref{BolzEQ_eq1} 