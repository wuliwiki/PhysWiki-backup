% 对数与对数函数(高中)
% keys 对数|对数函数
% license Xiao
% type Tutor
\begin{issues}
\issueDraft
\end{issues}
\pentry{函数\nref{nod_functi},函数的性质\nref{nod_HsFunC},指数函数\nref{nod_HsExpF}}{nod_c094}
\subsection{对数运算}

与早已烂熟于胸的加、减、乘、除和乘方、开方,对数是一个陌生的运算。很多教材会从指数运算的前提下去介绍,说他是指数运算的逆运算。但对数运算本身是远早于指数运算出现的,直接从指数去介绍或许会降低陌生的感觉,但很容易造成理解上的困惑。下面会从对数运算诞生的思路去介绍它。

对数运算最早是为了算两个数的乘法而创造出来的。那时还没有计算器,小一点的数当然可以用乘法口诀,但如果想要计算两个位数特别多的数相乘,比如两个四位数相乘,列竖式的方法需要计算四次乘法和四次加法,这显然太麻烦,而且太容易出错了。

John Napier 考虑了这样一个场景:一个点沿着直线在匀速运动,另一个点在线段上运动,它运动的速度与剩下到终点的距离正相关。这样两个点是一一对应的,假设与点 $n$ 对应的另一个点是 $L_n$,线段的长度是$L$,认为在单位时间内点是匀速的,也就是单位时间内运动的长度树枝上等于这段时间内的速度$v_n$,即将到来的这一段的速度与此时剩下的距离成正比$\displaystyle v_{n}=\frac{L_{n-1}}{L}$,从而有:

\begin{equation}
L_{n}=L_{n-1}-v_{n}=L_{n-1}(1-\frac{1}{L})\implies\frac{L_{n}}{L_{n-1}}= \frac{L-1}{L}~.
\end{equation}

设初始速度为 $v_0=1$,则 $L_{0}=L,L_{1}=L-1$。可知 $L_n$ 始终满足按与第一次一样的比例减小:

\begin{equation}
\frac{L_{n}}{L_{n-1}}= \frac{L-1}{L}=\frac{L_1}{L_{0}}~.
\end{equation}


从而根据我们现在知道的幂运算有:

\begin{equation}
L_n={L_{n}\over L_{n-1}}\cdot{L_{n}\over L_{n-1}}\cdots{L_{2}\over L_{1}}{L_{1}\over L_{0}}\cdot L_0=L(1 - \frac{1}{L})^n~.
\end{equation}

John Napier根据自己的思路提出了他的对数表,顺带还正式使用了小数点。这个对数,其实是比例对数的意思,指的就是“按照\textbf{比例}制作的,\textbf{对}着查\textbf{数}的表”。也就是上表中$n$是$L_n$的对数。实际使用时,比如要计算两个数$L_n,L_m$的乘积,只需要根据他的表格找到对应的$n,m$,求和后再找到对应的$L_{n+m}$就可以得到计算的积了,根据幂运算的法则,也就是:

\begin{equation}
L_nL_m=(L_{0})^2(1 - \frac{1}{L})^{n+m}=L_{m+n}~.
\end{equation}

k只要取$L=1$就没问题了。当然,他实际制作表的考量很多,这里只是简化后的思路。另外

\begin{equation}
L_n=L((1-{1\over L})^{-L})^{-n\over L}~.
\end{equation}


对数运算有两个特殊的计算结果:$\log_a a=1,\log_a1=0$

\begin{theorem}{对数运算法则}
\begin{itemize}
\item 乘法法则$\log_a(xy)=\log_ax+\log_ay$
\item 换底公式:$\displaystyle \log_a b=\frac{\log_cb}{\log_ca}$
\item 幂法则:$\displaystyle \frac{m}{n}\log_a b=\log_{a^n} b^m$
\end{itemize}
\end{theorem}

\subsection{自然对数函数}
以 $\E$ 为底的对数函数 $\log_{\E} x$ 叫做\textbf{自然对数}, 通常记为
\begin{equation}
\ln x \qquad \text{或} \qquad \log x~.
\end{equation}
函数图如\autoref{fig_Ln_2}。
\begin{figure}[ht]
\centering
\includegraphics[width=7cm]{./figures/ce690bcbd8c28a93.png}
\caption{几种不同底的对数函数} \label{fig_Ln_2}
\end{figure}


\subsection{指数函数与对数函数的相似性}

\pentry{函数的变换(高中)\nref{nod_FunTra},导数(高中)\nref{nod_HsDerv}}{nod_a54a}

根据\aref{幂运算}{the_power_1}和对数运算的法则,任意$f(x)=a^x$都可以变形,得到:

\begin{equation}
f(x)=e^{x\ln a}~.
\end{equation}

即,所有的指数函数都可以由$\E^x$通过在$x$方向上伸缩或关于$y$轴对称($\ln a<0$时)得到,或者所有的指数函数$a^x$都可认为是$f(x)=e^x$与$g(x)=x\ln a$复合得到的$f(g(x))$。

同理,根据对数运算的性质,任意$f(x)=\log_ax$都可以变形,得到:

\begin{equation}
f(x)=\frac{1}{\ln a}\ln x\iff f(x)\ln a=\ln x~.
\end{equation}

即,所有的对数函数都可以由$\ln x$通过在$y$方向上伸缩或关于$x$轴对称($\ln a<0$时)得到,或者所有的对数函数$\log_ax$都可认为是$f(x)=\ln x$与$\displaystyle g(x)=\frac{x}{\ln a}$复合得到的$g(f(x))$。同时,这个结论也可由指数函数和对数函数在参数相同时互为反函数验证。

综上,所有的指数函数、对数函数之间都是相似的。根据上述关系,已知$\left(\E^x\right)'=\E^x,\left(\ln x\right)'=\frac{1}{x}$,根据复合函数求导法则,可知:
\begin{equation}
\left(a^x\right)'=\ln a\cdot\E^x~,\left(\log_a x\right)'=\frac{1}{x\ln a}.
\end{equation}
