% 可分元素的单扩张是可分扩张
% 可分扩张|分裂域|域单同态|域嵌入|可分闭包

\pentry{本原元素与单代数扩张\upref{PrmtEl}}

\subsection{定理的描述}

本词条是专门用来证明下述\autoref{the_SprbE2_1} 的,其自然语言的描述即是本词条的标题。本文部分节选自《代数学基础》。

\begin{theorem}{}\label{the_SprbE2_1}
域$\mathbb{F}$上的不可约可分多项式$f(x)$的分裂域$\mathbb{K}$是$\mathbb{F}$的可分扩张。
\end{theorem}

考虑Artin本原性定理(\autoref{the_PrmtEl_2}~\upref{PrmtEl}),以及可分扩张的定义(所有元素都是可分元),我们还可以得到上述\autoref{the_SprbE2_1} 的等价描述:

\begin{theorem}{}\label{the_SprbE2_2}
$\mathbb{F}(a)/\mathbb{F}$是可分扩张$\iff$ $a$是$\mathbb{F}$的可分元素。
\end{theorem}

本节我们就要证明\autoref{the_SprbE2_2} 。



\subsection{定理的证明}
该证明的思路取自 University of Connecticut 的 Keith Conrad 教授的讲义,但细节思路会有不同。这是因为本书有自己的逻辑体系,会大量引用小时百科中的内容来完成证明。
%证明思路来源:https://kconrad.math.uconn.edu/blurbs/galoistheory/separable1.pdf

\begin{lemma}{}\label{lem_SprbE2_1}
设$\mathbb{L}/\mathbb{K}$是一个域扩张,且$[\mathbb{L}:\mathbb{K}]=n$。设$\sigma:\mathbb{K}\to\mathbb{F}$是一个域单同态。

则:

1. $\sigma$开拓而得的域单同态$\mathbb{L}\to\mathbb{F}$的数量\textbf{小于等于}$n$。

2. 如果$\mathbb{L}/\mathbb{K}$是\textbf{不可分}扩张,则$\sigma$开拓而得的域单同态$\mathbb{L}\to\mathbb{F}$的数量\textbf{小于}$n$。

3. 如果$\mathbb{L}/\mathbb{K}$是\textbf{可分}扩张,则存在扩域$\mathbb{F}'/\mathbb{F}$,使得$\sigma$开拓而得的域单同态$\mathbb{L}\to\mathbb{F}'$的数量\textbf{等于}$n$。

\end{lemma}


\subsubsection{\autoref{lem_SprbE2_1} 中1. 的证明}

% 我们用数学归纳法来处理。

% 显然,当$n=1$时,$\mathbb{L}=\mathbb{K}$,定理成立。下设$n>1$,且对于任意域扩张$\mathbb{L}'/\mathbb{K}$,只要扩张次数小于$n$则定理成立。

注意,单同态就是定义域和象的同构。

取$a\in\mathbb{L}-\mathbb{K}$,$a$在$\mathbb{K}$上的最小多项式是$f$。则$f(x)\in\mathbb{K}[x]$的分裂域是$\mathbb{K}(a)\subseteq\mathbb{L}$。考虑\autoref{the_SpltFd_3}~\upref{SpltFd},可知$\sigma$开拓为$\mathbb{K}(a)$到$\mathbb{F}$的同态后,同态象最多只有一种可能。也就是说,$\mathbb{F}$中最多只有一个子域同构于$\mathbb{K}(a)$。

于是,如果$\sigma$开拓为$\mathbb{K}(a)\to\mathbb{F}$的同态存在,那么每个不同的开拓都对应一个$\mathbb{K}(a)$(或者$\sigma(\mathbb{K}(a))$)到自身的保$\mathbb{K}$(或者保$\sigma(\mathbb{K})$)自同构。

又据\autoref{the_SpltFd_1}~\upref{SpltFd},可知由$\sigma$开拓而来的域同态$\sigma:\mathbb{K}(a)\to\mathbb{F}$最多只有$[\mathbb{K}(a):\mathbb{K}]=\opn{deg}f$个。

上述讨论说明,定理对单扩张情况成立。再考虑\autoref{the_FldExp_3}~\upref{FldExp}和\textbf{中间域升链}\autoref{cor_FldExp_3}~\upref{FldExp},则得证。



\subsubsection{\autoref{lem_SprbE2_1} 中2. 的证明}

由于$\mathbb{L}/\mathbb{K}$是不可分扩张,故存在$a\in\mathbb{L}$是$\mathbb{K}$的不可分元素。再据\textbf{中间域升链}\autoref{cor_FldExp_3}~\upref{FldExp},可构造$\mathbb{L}/\mathbb{K}$的中间域升链,其中$\mathbb{K}$邻近的单扩张就是$\mathbb{K}(a)$。

不可分多项式的根的数目,小于其次数。但是据\autoref{the_FldExp_1}~\upref{FldExp},$[\mathbb{K}_2:\mathbb{K}_1]$正等于其次数。因此$\sigma$在一步步开拓的过程中,在第一步$\mathbb{K}(a)/\mathbb{K}$这里,开拓出的同态数目就要小于扩张次数。

由此得证。


\subsubsection{\autoref{lem_SprbE2_1} 中3. 的证明}

同上一条的证明,可知$\mathbb{L}/\mathbb{K}$的中间域中,任意相邻两个域之间都是\textbf{单可分}扩域的关系。

也就是说,每一步扩张都是一个\textbf{无重根的不可约多项式}的分裂域,因此据\autoref{the_SpltFd_1}~\upref{SpltFd},每一步中$\sigma$的开拓的数目都恰为扩张的次数。

除了一种情况:那就是某一步分裂域扩域无法映射入$\mathbb{F}$,那就在$\mathbb{F}$上取这个分裂域的多项式在$\mathbb{F}$上的映射的分裂域,即可\footnote{这么说很绕口。但如果采用“同构就是同一个”和“单同态就是定义域和象的同构”的理解,就会直白得多。}。这也是为什么定理中会说存在一个扩域$\mathbb{F}'/\mathbb{F}$。




\subsubsection{正式证明\autoref{the_SprbE2_2} }

注意,\autoref{lem_SprbE2_1} 的证明中,只出现了“可分单扩张”和“不可分单扩张”,但没有说\textbf{不可分单扩张}所用的元素是不是可分元素(即下面证明要讨论的),所以没有构成循环论证。

$\Rightarrow$:

由可分扩张的定义,显然。

$\Leftarrow$:

设$a$是域$\mathbb{K}$的可分代数元,其在$\mathbb{K}$上的最小多项式是$\opn{irr}(a, \mathbb{K})=f(x)\in\mathbb{K}[x]$。设$\opn{deg}f=n$,$f\in\mathbb{K}[x]$的分裂域是$\mathbb{L}$。

按可分的定义,知$f$是可分多项式,在$\mathbb{L}$上有$n$个根。据\autoref{the_FldExp_1}~\upref{FldExp},可知$[\mathbb{L}:\mathbb{K}]=n$。

取$\mathbb{L}$作为\autoref{lem_SprbE2_1} 中的$\mathbb{L}$和$\mathbb{F}$、$\mathbb{K}$作为\autoref{lem_SprbE2_1} 中的$\mathbb{K}$,且$\sigma=\opn{id}_{\mathbb{K}}$。

据\autoref{the_SpltFd_1}~\upref{SpltFd},$\mathbb{L}$到自身的保$\mathbb{K}$自同构有$n$个。也就是说,$\sigma$开拓而来的单同态$\mathbb{L}\to\mathbb{F}$的数量\textbf{等于}$n$。

这违反了\autoref{lem_SprbE2_1} 的第$2$条。因此,$\mathbb{L}/\mathbb{K}$必是可分扩张。



这样一来,只要有办法判断一个元素是否可分,就能判断其单扩张是否可分了。进一步,任何有限扩张都可以拆分成若干次单扩张的结果,于是也能讨论任何有限扩张的情形了。当然,这是目前的猜测,真实情况的性质更好,看下去就知道了。



\subsection{推论}


% \begin{lemma}{}\label{lem_SprbE2_2}
% 给定域$\mathbb{F}$和其上的两个可分代数元$a, b$,则$a+b$也是$\mathbb{F}$的可分代数元。
% \end{lemma}

% \textbf{证明}:

% 由于特征为零的域都是完备域(代数扩张皆为可分扩张),因此这里只考虑$\opn{ch}\mathbb{F}=p$的情况。

% \textbf{证毕}。

% 为了用本节结论得到\autoref{cor_SprbE2_1} ,我们还需要两个引理:


% \begin{lemma}{}

% \end{lemma}




% \begin{lemma}{}
% 设有域$\mathbb{F}$,$a$是$\mathbb{F}$的可分元素,$b$是$\mathbb{F}(a)$的可分元素。

% 则$\mathbb{F}(a, b)/\mathbb{F}$是可分扩张。
% \end{lemma}

\begin{corollary}{}\label{cor_SprbE2_2}
设$\mathbb{L}/\mathbb{K}$是一个\textbf{可分正规}扩张,且$[\mathbb{L}:\mathbb{K}]=n$。则$\mathbb{L}$的保$\mathbb{K}$自同构有$n$个。
\end{corollary}

\textbf{证明}:

取$\mathbb{F}=\mathbb{F}'=\mathbb{L}$,$\sigma=\opn{id}_{\mathbb{K}}$,套用\autoref{lem_SprbE2_1} 第$3$条即可。

\textbf{证毕}。





\begin{lemma}{可分元素的判定}\label{lem_SprbE2_2}
设域$\mathbb{F}$的特征为
\footnote{特征为$0$的域都是完美域,故无须讨论。}
$p$,则$\alpha$是$\mathbb{F}$的可分元素当且仅当$\mathbb{F}(\alpha)=\mathbb{F}(\alpha^p)$。
\end{lemma}





%挺有趣,必要性我思路走偏了,看了朱富海老师的证明写的。但是充分性虽然和朱老师一样,确实我自己想出来的,这个思路倒是好想,遇到不可分不可约多项式,第一反应就是用可分不可约多项式来表达嘛。
\textbf{证明}:

\textbf{必要性}:

令$f(x)=x^p-\alpha^p$,显然$f(\alpha)=0$,故$\opn{Irr}(\alpha, \mathbb{F}(\alpha))\mid f$。

又因为域特征为$p$,
\begin{equation}
    f(x) = x^p-\alpha^p = (x-\alpha)^p
\end{equation}
结合$\alpha$在$\mathbb{F}$上可分$\implies$在$\mathbb{F}(\alpha^p)$上可分,可知$\opn{Irr}(\alpha, \mathbb{F}(\alpha))=x-\alpha$。因此,$\alpha\in\mathbb{F}$,从而得证。

\textbf{充分性}:

记$m=\opn{Irr}(\alpha, \mathbb{F})$。反设$\alpha$在$\mathbb{F}$上不可分,即存在$\mathbb{F}$上\textbf{可分的不可约}多项式$h$和\textbf{正整数}$k$,使得$m(x)=h\qty(x^{p^k})$。

令$g(x)=h\qty(x^{p^{k-1}})$,则
\begin{equation}
    g(\alpha^p)= h\qty(\qty(\alpha^{p})^{p^{k-1}}) = h\qty(\alpha^{p^k}) = m(\alpha) = 0
\end{equation}
即$g(x)$是$\alpha^p$的零化多项式。

故$\opn{deg}\opn{Irr}(\alpha^p, \mathbb{F})\leq \opn{deg}g< \opn{deg}m$。由单扩张的次数定理(\autoref{the_FldExp_1}~\upref{FldExp})即可知$[\mathbb{F}(\alpha^p):\mathbb{F}]<[\mathbb{F}(\alpha):\mathbb{F}]$,从而$\mathbb{F}(\alpha^p))\subsetneq\mathbb{F}(\alpha)$。

\textbf{证毕}。





\begin{lemma}{}
给定域$\mathbb{F}$。若$\mathbb{F}(a)/\mathbb{F}$和$\mathbb{F}(a, b)/\mathbb{F}(a)$都是可分扩张,则$\mathbb{F}(a, b)/\mathbb{F}$也是。
\end{lemma}

\textbf{证明}:

不妨设$\opn{ch}\mathbb{F}=p$。

因为$b$在$\mathbb{F}(a)$上可分,故由\autoref{lem_SprbE2_2} 知,
\begin{equation}\label{eq_SprbE2_1}
    \mathbb{F}(a, b^p) = \mathbb{F}(a)(b^p) = \mathbb{F}(a)(b) = \mathbb{F}(a, b)
\end{equation}




记$f=\opn{Irr}(a, \mathbb{F}(b))$,$g=\opn{Irr}(a, \mathbb{F}(b^p))$,则由于$\mathbb{F}(b^p)\subseteq\mathbb{F}(b)$,可知$f\mid g$。

由引理\ref{Ch F is p, polynomial, sum, power p},$f(x)^p\in\mathbb{F}(b^p)[x]$,且显然是$a$的零化多项式,因此$g\mid f^p$。又因为$a$在$\mathbb{F}$上可分$\implies$在$\mathbb{F}(b^p)$上可分,故得$g\mid f$。

综上,$f=g$。

由单扩张的次数定理(定理\ref{Degree of Simple Extension, Minimal Polynomial, Theorem})可知
\begin{equation}
[\mathbb{F}(a, b^p):\mathbb{F}(b^p)] = [\mathbb{F}(a, b):\mathbb{F}(b)]
\end{equation}

再结合域扩张的次数乘积定理(定理\ref{Degree of Two Sequential Field Extensions is the Product of the Degrees of the Two Extensions})和\autoref{eq_SprbE2_1} 可知
\begin{equation}
\begin{\begin{aligned}

\end{aligned}
[\mathbb{F}(b):\mathbb{F}(b^p)] = \frac{[\mathbb{F}(b):\mathbb{F}(b^p)]}{[\mathbb{F}(b^p):\mathbb{F}]}
\end{equation}

\textbf{证毕}。











\begin{corollary}{可分扩张的传递性}\label{cor_SprbE2_1}
设$\mathbb{K}/\mathbb{F}$是域代数扩张,其有一中间域$\mathbb{M}$。如果$\mathbb{M}/\mathbb{F}$和$\mathbb{K}/\mathbb{M}$都是可分扩张,那么$\mathbb{K}/\mathbb{F}$是可分扩张。
\end{corollary}

\textbf{证明}:



\textbf{证毕}。





\begin{corollary}{}\label{cor_SprbE2_3}
设$\mathbb{F}(a_1, \cdots, a_n)$是域$\mathbb{F}$的代数扩张,且各$a_i$都是$\mathbb{F}$上的可分元素,则$\mathbb{F}(a_1, \cdots, a_n)/\mathbb{F}$是可分扩张。
\end{corollary}

\textbf{证明}:

利用可分扩张的传递性\autoref{cor_SprbE2_1} 和本节的核心\autoref{the_SprbE2_2} 即得证。

\textbf{证毕}。


%Woohoo! I finished this part at last! -Jier 2022 Jun. 11 Dawn



\begin{corollary}{可分元素的封闭性}\label{cor_SprbE2_4}
可分元素相加、相乘、取负和取逆的结果,仍然是可分元素。
\end{corollary}

\autoref{cor_SprbE2_4} 由\autoref{cor_SprbE2_3} 直接可得。由该推论还可知,一个域的全体可分元素构成一个域,称为其\textbf{可分闭包(separable closure)};$\mathbb{F}$的可分闭包与$\mathbb{K}$的交集,称为$\mathbb{F}$在$\mathbb{K}$上的可分闭包。











