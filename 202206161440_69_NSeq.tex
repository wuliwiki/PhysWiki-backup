% Navier-Stokes 方程
% keys NS方程|流体力学动量方程

\pentry{流体力学守恒方程\upref{fluidC},爱因斯坦求和约定\upref{EinSum}}
让我们回顾流体力学的动量守恒方程:
\begin{equation}
\rho \frac{\partial \bvec u}{\partial t}+\rho(\bvec u\cdot \nabla)\bvec u=\rho \bvec g+\nabla\cdot \overleftrightarrow {\bvec T}
\end{equation}

为了进一步解这个方程,我们需要知道应力张量 $T_{ij}$ 的具体形式.$T_{ij}$ 表示在法线为 $x_i$ 方向的单位面元上,面外对面内的面力的 $x_j$ 分量,而且是二阶对称张量(为了保证角动量守恒),那么它有怎样的性质呢?

最简单的一种非粘性的各向同性流体,其应力张量的对角元都为 $-p$,这意味着每个面元上受到的力是垂直于面元的,单位面积上受到的力为流体在该处的压强,因此对角元 $-p$ 给出的就是压强 $p$.但实际情况中流体是有粘性的,例如下图
\begin{figure}[ht]
\centering
\includegraphics[width=10cm]{./figures/NSeq_1.png}
\caption{牛顿粘性实验} \label{NSeq_fig1}
\end{figure}
当 $y$ 方向相邻两侧流体的水平速度 $u$ 有梯度时,就会产生一个剪应力 $\tau=\mu \dd u/\dd y$,$\mu$ 被称为第一粘性系数.继续假设流体是各向同性的,那么似乎应该有 $T_{ij}=\mu \dd v_j/\dd x_i$($i\neq j$).但这样实际上是有问题的,在我们上面的牛顿黏性实验中,应力张量的非对角元实际上不止有 $T_{21}$,否则考察其中的一个微元,上面受到的向右的力比下面受到的向右的力更大,取足够小的微元,它的单位体积角动量将会趋于无穷大.这也意味着应力张量必须是对称张量.在上面的牛顿粘性实验中,事实上每一水平层流体,沿 $x$ 方向相邻两个紧挨着的流体微元,将会有 $y$ 方向的面力.也就是说 $T_{12}=T_{21}$.最终,我们可以把 $T_{ij}$ 改写为 $\mu (\dd v_j/\dd x_i+\dd v_i/\dd x_j)=2\mu S_{ij}$($i\neq j$).

上面的讨论仍然存在问题.既然 $T_{ij}$ 对角元全相等(为 $-P$),那么对于各向同性的流体,应力张量矩阵 $T_{ij}$ 应当在正交相似变换下对角元应当仍是 $-P$.但从牛顿粘性实验的 $T_{ij}$ 出发很容易发现这是错误的.为了修正使得 $T_{ij}$ 各向同性,我们需要将第一粘性系数也考虑进 $T_{ij}$ 的对角元中:$T_{ij} = -p\delta_{ij}+2\mu S_{ij}$.

上面的讨论还没结束.注意到,对于可压缩流体,$S_{mm}=\nabla\cdot u\neq 0$,流体的压强($T_{ij}$ 的对角元的平均值)可能会变化.我们设第二粘性系数 $\mu_\nu$,并假设这个过程使得压强在 $p$ 基础上增大了 $\mu_\nu S_{mm}$,那么可以写出最终的应力张量

\begin{equation}
T_{ij}=-p\delta_{ij}+2\mu \qty(S_{ij}-\frac{2}{3}S_{mm}\delta_{ij})+\mu_\nu S_{mm} \delta_{ij}
\end{equation}
