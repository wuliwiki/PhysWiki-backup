% 布莱斯·帕斯卡(综述)
% license CCBYSA3
% type Wiki

本文根据 CC-BY-SA 协议转载翻译自维基百科\href{https://en.wikipedia.org/wiki/Blaise_Pascal}{相关文章}。

布莱兹·帕斯卡(Blaise Pascal,\(^\text{[a]}\) 1623年6月19日-1662年8月19日)是一位法国数学家、物理学家、发明家、哲学家及天主教作家。

帕斯卡是神童,由担任鲁昂税务官的父亲亲自教育。他最早的数学研究是投影几何,16岁时便撰写了一篇重要的关于圆锥曲线的论文。后来,他与皮埃尔·费马通信探讨概率论,对现代经济学与社会科学的发展产生了深远影响。1642年,他开始从事计算机的先驱性研究,发明了后来被称为“帕斯卡计算器”或“帕斯卡机”的装置,使他成为机械计算器的最早两位发明人之一\(^\text{[6][7]}\)。

与同时代的勒内·笛卡尔一样,帕斯卡也是自然科学和应用科学的先驱。他撰文为科学方法辩护,并提出了若干颇具争议的研究成果。他在流体研究方面作出了重要贡献,推广伊万杰利斯塔·托里拆利的研究成果,澄清了压力和真空的概念。国际单位制中压力单位“帕斯卡”正是以他命名的。1647年,他继托里拆利和伽利略之后,驳斥了亚里士多德与笛卡尔等人所持的“自然界厌恶真空”的观点。

他也被誉为现代公共交通的发明者,因为他在1662年去世前不久创立了“五苏之马车”,这是历史上第一种现代公共交通服务\(^\text{[8]}\)。

1646年,他与妹妹雅克琳一同接受了天主教内部一个被批评者称为詹森主义的宗教运动\(^\text{[9]}\)。1654年末经历一次宗教体验后,他开始撰写有深远影响的哲学与神学作品。他最著名的两部著作都诞生于这一时期:《省函集》和《思想录》。《省函集》以詹森主义者与耶稣会士之间的冲突为背景;而《思想录》中包含了著名的“帕斯卡赌注”,原名为《论机器的演说》\(^\text{[10][11]}\),这是一个以信仰主义为基础、具有概率论性质的论证,主张人应当相信上帝的存在。同年,他还撰写了一部关于“算术三角形”的重要论文。1658至1659年间,他又研究了摆线及其在求解立体体积中的应用。在多年疾病折磨之后,帕斯卡于39岁时在巴黎去世。
\subsection{早年生活与教育}
帕斯卡出生于法国奥弗涅地区的克莱蒙费朗,地处中央高原。他在三岁时失去了母亲安托瓦内特·贝贡。\(^\text{[12]}\)他的父亲艾蒂安·帕斯卡也是一位业余数学家,是当地的法官,同时是“法袍贵族”成员。帕斯卡有两个姐妹,妹妹叫雅克琳,姐姐叫吉尔贝特。
\begin{figure}[ht]
\centering
\includegraphics[width=6cm]{./figures/57329a3ddbf9ceaa.png}
\caption{} \label{fig_BLSpsk_1}
\end{figure}
迁居巴黎
1631年,也就是妻子去世五年后,\(^\text{[13]}\)艾蒂安·帕斯卡带着孩子们搬到了巴黎。这户新到的家庭很快雇佣了女仆路易丝·德福,后者最终成为了这个家庭的重要成员。艾蒂安终身未再婚,决定亲自教育自己的孩子们。

年幼的帕斯卡展现出非凡的智力,特别是在数学和科学方面展现出惊人的天赋。\(^\text{[14]}\)艾蒂安原本试图阻止儿子接触数学;然而在12岁时,帕斯卡凭借自己的努力,用木炭在瓷砖地板上重新推导出了欧几里得的前32条几何命题,因此他得到了《几何原本》的一本副本。\(^\text{[15]}\)

\textbf{关于圆锥曲线的论文}

帕斯卡尤其感兴趣的一本著作是德扎格关于圆锥曲线的研究。沿着德扎格的思路,年仅16岁的帕斯卡撰写了一篇短小的论文《圆锥曲线试作》(法语原名 Essai pour les coniques),用以证明一个被称为“神秘六边形”的命题,并将这篇他人生中第一篇严肃的数学论文寄给了巴黎的梅尔森神父。这一定理今天仍以“帕斯卡定理”之名为人熟知:它断言,若一个六边形内接于一个圆或一般的圆锥曲线中,则其对边延长线的交点三三成对后共线,这条直线称为“帕斯卡线”。

帕斯卡的工作极其早熟,以至于笛卡尔一度坚信是帕斯卡的父亲写下了这篇文章。当梅尔森向他确认确实是帕斯卡之子所作时,笛卡尔冷笑一声轻蔑地回应道:“我并不觉得奇怪,他在圆锥曲线方面给出的证明比古人更为得当”,并补充道:“但关于这一主题,还有些内容是一个16岁的孩子无论如何也不可能想到的。”\(^\text{[16]}\)
\subsubsection{离开巴黎}
在当时的法国,官职是可以买卖的。1631年,艾蒂安·帕斯卡以65,665里弗的价格出售了自己在辅助法院担任的二级主席职务\(^\text{[17]}\)。这笔资金被投资于一项政府债券,虽然不能说奢华,但足以为帕斯卡一家在巴黎提供一种安稳的生活。然而到了1638年,红衣主教黎塞留为了筹措继续打三十年战争的资金,违约了这批政府债券。艾蒂安·帕斯卡的财富于是骤减,从近66,000里弗跌到了不到7,300里弗。

像许多其他人一样,艾蒂安因反对黎塞留的财政政策最终不得不逃离巴黎,留下他的三个孩子由邻居圣托夫人照料。这位圣托夫人美貌动人,过往风流,却也经营着全法国最闪耀、最具文化气息的沙龙之一。直到某次雅克琳在一场儿童剧中表演出色,而黎塞留正好在场观看,艾蒂安才得以被赦免。不久之后,艾蒂安重新获得了红衣主教的青睐,并于1639年被任命为鲁昂市的国王税务专员——而当时该市的税务记录由于多次民变,已是一片混乱。
\subsubsection{帕斯卡计算器}
\begin{figure}[ht]
\centering
\includegraphics[width=6cm]{./figures/0a529a6a26267e4b.png}
\caption{} \label{fig_BLSpsk_2}
\end{figure}
1642年,为了减轻父亲在税务工作中那无休止的、令人精疲力竭的计算与重新计算(年轻的帕斯卡也参与了这项工作),帕斯卡在尚未满19岁时便设计并制造出一种能进行加法和减法运算的机械计算器,被称为“帕斯卡计算器”或“帕斯卡机”。在现存的八台帕斯卡机中,四台收藏于巴黎的工艺与技术博物馆,另有一台在德国德累斯顿的茨温格宫博物馆展出,均为他原始设计的机械计算器之一\(^\text{[18]}\)。

尽管这些机器是后续400年机械计算技术发展的先驱,从某种意义上说也可被视为计算机工程领域的前身,但这台计算器并未获得商业上的巨大成功。一方面是因为它在实际使用中仍显笨重,更主要的原因可能是其造价极其昂贵,帕斯卡机最终沦为法国乃至欧洲富人手中的玩具与地位象征。帕斯卡在接下来的十年中持续改进其设计,据他自己所述,有大约50台机器是按他的设计制造的\(^\text{[19]}\)。在随后的十年间,他共亲自制造了20台成品机器\(^\text{[20]}\)。
\subsection{数学}
\subsubsection{概率论}
1654年,在朋友谢瓦利耶·德·梅雷的启发下,帕斯卡与皮埃尔·德·费马就赌博问题展开通信讨论,由此诞生了数学概率论的雏形\(^\text{[21]}\)。他们讨论的具体问题是:两位玩家希望提前结束游戏,并希望根据当前局势公平分配赌注,即各自根据继续赢得比赛的概率来分配奖金。通过这场讨论,“期望值”这一概念首次被引入。约翰·罗斯写道:“概率论及其后续的发展改变了我们对不确定性、风险、决策,以及个人与社会影响未来事件走向的能力的看法”\(^\text{[22]}\)。帕斯卡在其《思想录》中使用了一种概率论论证——帕斯卡赌注,来为信仰上帝及过一种有德行的生活提供理由。然而,尽管帕斯卡与费马在概率论领域做出了重要的早期贡献,他们并未将这一理论发展得更为深入。荷兰科学家惠更斯通过阅读帕斯卡与费马的通信了解到这一新兴领域,并撰写了第一本关于概率论的书籍。随后,亚伯拉罕·德·莫阿弗尔和皮埃尔-西蒙·拉普拉斯等人继续推动了这一理论的发展。帕斯卡与费马在概率计算方面的研究,也为莱布尼茨后来的微积分构想奠定了重要基础\(^\text{[23]}\)。
\subsubsection{《算术三角形论》}
\begin{figure}[ht]
\centering
\includegraphics[width=6cm]{./figures/442c1dcc85f412f2.png}
\caption{帕斯卡三角形:每个数字都是其正上方两个数字之和。除了展示二项式系数外,这个三角形还体现了许多数学性质。} \label{fig_BLSpsk_3}
\end{figure}
帕斯卡于1654年撰写的《算术三角形论》,于他去世后在1665年出版,其中描述了一种排列二项式系数的简便表格形式,他称之为“算术三角形”,如今则被称作“帕斯卡三角形”。
该三角形也可以表示为:
\begin{figure}[ht]
\centering
\includegraphics[width=6cm]{./figures/adab679161e80ad4.png}
\caption{} \label{fig_BLSpsk_4}
\end{figure}
他通过递推关系来定义三角形中的数值:设第 $m+1$ 行、第 $n+1$ 列的数为 $t_{mn}$,那么有$t_{mn} = t_{m-1,n} + t_{m,n-1}$其中$m = 0, 1, 2, \ldots \text{ 且 } n = 0, 1, 2, \ldots$边界条件为 $t_{m,-1} = 0, \ t_{-1,n} = 0$,适用于 $m = 1, 2, 3, \ldots$ 和 $n = 1, 2, 3, \ldots$。生成元为 $t_{00} = 1$。帕斯卡最终给出了如下公式的证明:
$$
t_{mn} = \frac{(m+n)(m+n-1)\cdots(m+1)}{n(n-1)\cdots1}~
$$
在这篇论文中,帕斯卡还明确陈述了数学归纳法原理\(^\text{[24]}\)。1654年,他还证明了一个与幂和相关的恒等式(帕斯卡恒等式),该恒等式给出了前 $n$ 个正整数的 $p$ 次幂之和的关系,其中 $p = 0, 1, 2, \ldots, k$\(^\text{[26]}\)。

同年,帕斯卡经历了一次宗教体验,从此几乎放弃了数学方面的工作。
\subsubsection{摆线}
\begin{figure}[ht]
\centering
\includegraphics[width=6cm]{./figures/f9644e4e1481c829.png}
\caption{《帕斯卡研究摆线》,奥古斯丁·帕若作,1785年,卢浮宫藏} \label{fig_BLSpsk_5}
\end{figure}
1658年,帕斯卡在忍受牙痛时开始思考一些关于摆线的问题。他的牙痛随之消失,他将这视为来自上天的指引,决定继续研究。仅仅八天之后,他便完成了一篇论文\(^\text{[27]}\)。为了宣传研究成果,他发起了一项竞赛\(^\text{[28]}\)。

帕斯卡提出了三个与摆线有关的问题,分别关于其重心、面积与体积,优胜者将获得20或40枚西班牙达布隆金币作为奖品。评审包括帕斯卡本人、吉尔·德·罗贝瓦尔和皮埃尔·德·卡尔卡维。两份参赛作品(分别由约翰·沃利斯 John Wallis 和安托万·德·拉卢韦尔 Antoine de Lalouvère 提交)都被判定不合格\(^\text{[29]}\)。在比赛期间,克里斯托弗·雷恩向帕斯卡提交了一个关于摆线求长的证明提案;罗贝瓦尔立刻宣称他多年前就已知道该证明。沃利斯随后在他的《双论集》中发表了雷恩的证明,并明确将首创权归于雷恩。
\subsection{物理学}
\begin{figure}[ht]
\centering
\includegraphics[width=6cm]{./figures/3a06050aaec8b631.png}
\caption{帕斯卡木桶实验(可能是杜撰的)的插图} \label{fig_BLSpsk_6}
\end{figure}
帕斯卡在物理学多个领域作出了贡献,尤其是在流体力学和压力方面。为纪念他的科学贡献,国际单位制中压力的单位被命名为“帕斯卡”,他还提出了帕斯卡定律,这是流体静力学中的一个重要原理。他在研究永动机的过程中还发明了一种原始形式的轮盘赌和轮盘装置\(^\text{[30]}\)。
\subsubsection{流体动力学}
他在流体动力学和流体静力学方面的工作集中于液体的液压原理。他的发明包括液压机(利用液体压力来放大力量)和注射器。他证明了流体静压并不取决于流体的重量,而是取决于高度差。他通过一个著名的实验演示了这一原理:将一根细管插入盛满水的木桶中,并将水加至与三楼相同的高度。结果,木桶因压力而破裂,这一实验被称为“帕斯卡木桶实验”。
\subsubsection{真空}
1647年,帕斯卡得知埃万杰利斯塔·托里拆利关于气压计的实验。他复现实验:将一根充满水银的玻璃管倒插入水银盆中。他随即提出质疑:到底是什么力量让部分水银停留在管中?而管中水银上方的空间又是什么?当时大多数科学家(包括笛卡尔)都相信“充满说”,即某种看不见的物质充满了所有空间,而非存在真空——这源于亚里士多德的观念,认为世间一切运动都必须由一种物质推动另一种物质\(^\text{[31]}\)。此外,光能够穿过玻璃管,这也让人们倾向认为那是以太等物质,而非真空。

在此基础上,帕斯卡于1647年发表了《有关真空的新实验》,系统描述了空气压力可以支撑液体的高度规则,并阐明了气压计中液柱上方确实是一个真空。这项工作后来又以《液体平衡伟大实验记》在1648年发表。
\subsubsection{首次大气压与海拔高度关系实验}
\begin{figure}[ht]
\centering
\includegraphics[width=6cm]{./figures/f2bc0f3d6a03ddf2.png}
\caption{多姆山} \label{fig_BLSpsk_8}
\end{figure}
托里拆利真空表明空气压力相当于30英寸汞柱的重量。如果空气具有有限的重量,那么地球的大气层必定有一个最大高度。帕斯卡尔据此推理:如果这一点为真,那么高山上的气压就应该低于海拔较低处。他住在海拔4,790英尺(约1,460米)的多姆山附近,但因健康状况不佳无法亲自登山\(^\text{[32]}\)。1648年9月19日,在帕斯卡尔几个月友好而坚持不懈的督促下,他的姐夫弗洛林·佩里耶终于得以执行这项对帕斯卡尔理论至关重要的实地调查。佩里耶撰写的记录如下:
\begin{figure}[ht]
\centering
\includegraphics[width=6cm]{./figures/ccdf92d22ea29e2f.png}
\caption{弗洛兰·佩里耶在多姆山上} \label{fig_BLSpsk_9}
\end{figure}
“上周六天气不太稳定……[但]大约清晨五点……多姆山可见……于是我决定尝试登顶。克莱蒙市的几位要人曾要求我出发时告知他们……我很高兴能与他们共同参与这项伟大的工作……

……八点钟我们在米宁派修士花园集合,那是城中海拔最低之处……我先往一个容器中倒入16磅水银……然后取出几根玻璃管……每根四英尺长,一端密封,另一端敞开……将它们插入容器中的水银……我发现水银柱高出容器水银面26英寸又3又½行……我在同一地点又重复了两次实验……每次都得出相同的结果……

我将一根玻璃管固定在容器上,并标记了水银的高度……然后请米宁修士之一的沙斯坦神父在原地观察当日是否有任何变化……我则带着另一根玻璃管和一部分水银……登上比修道院高出约500寻的多姆山顶,在那里进行实验……发现水银柱高度仅为23英寸又2行……我在山顶不同地点重复实验五次……每次都得出相同的水银高度……\(^\text{[33]}\)”

帕斯卡尔在巴黎也重复了这一实验,他将气压计带上了圣雅克-德拉布舍里教堂钟楼,约50米高。水银柱下降了2行。他从这两组实验中得出:每上升7寻,汞柱高度就下降半行。【注】:帕斯卡尔使用的单位中,pouce 和 ligne 分别表示“英寸”和“行”,toise 表示“寻”\(^\text{[34]}\)。

帕斯卡尔在回应仍相信充盈说的艾蒂安·诺埃尔时写道,呼应了当代关于科学与可证伪性的观念:“为了证明一个假设是正确的,仅仅现象都可由其推出是不够的;相反,如果它得出与某一个现象相矛盾的结论,那就足以证明其错误。”\(^\text{[35]}\)

“布莱兹·帕斯卡尔讲席”授予国际杰出科学家,以便他们在法兰西岛大区进行研究工作\(^\text{[36]}\)。
\subsection{成年生活:宗教、文学与哲学}
\subsubsection{宗教皈依}
\begin{figure}[ht]
\centering
\includegraphics[width=6cm]{./figures/d6ba0557cb107cf0.png}
\caption{帕斯卡肖像} \label{fig_BLSpsk_7}
\end{figure}
1646年冬天,58岁的帕斯卡尔的父亲在鲁昂的冰街上滑倒摔断了髋骨;考虑到当时的医学水平和他年迈的身体,这种伤势可能会非常严重,甚至危及生命。鲁昂有两位法国最出色的医生,德朗德和布特耶里。老帕斯卡尔“除了这两位医生谁也不让碰他……这是个明智的选择,因为老人最终康复了,还能重新行走……”\(^\text{[37]}\)然而,治疗与康复花费了整整三个月,在这期间,德朗德和布特耶里成了帕斯卡尔家的常客。

这两位医生都是让·基耶贝尔的追随者,他是一个背离传统天主教教义的宗派——詹森主义的支持者。这个当时还算小众的派别却在法国天主教内部迅速扩展,主张严格的奥古斯丁主义。帕斯卡尔经常与两位医生交谈,在他们成功治愈他父亲之后,还向他们借阅了一些詹森主义作家的著作。就在这一时期,帕斯卡尔经历了一次“初次皈依”的转变,并在接下来的一年中开始撰写神学文章。

然而,帕斯卡尔很快便脱离了最初的宗教热情,度过了一段被一些传记作家称为他“世俗时期”(1648–1654)的岁月。他的父亲于1651年去世,遗产留给了帕斯卡尔和他的妹妹雅克琳,由帕斯卡尔作为财产监护人。此时雅克琳宣布她将加入詹森主义的圣地——波尔-罗亚尔修道院。帕斯卡尔对此深受打击与悲伤,不是因为妹妹的选择,而是因为他长期身体虚弱,非常需要她的照顾,正如她从前需要他一样。

帕斯卡尔家很快爆发了冲突。布莱兹恳求妹妹不要离开,但她态度坚决。他命令她留下,却也无济于事。根本的问题是……帕斯卡尔对被抛弃的恐惧……如果雅克琳进入波尔-罗亚尔,她将必须放弃自己的那份遗产……[但]她的决定不可动摇。\(^\text{[38]}\)

1651年10月底,兄妹之间终于达成妥协。作为交换条件,帕斯卡尔每年向妹妹支付一笔丰厚津贴,而雅克琳则将自己的遗产份额转让给哥哥。她的大姐吉尔贝特早已以嫁妆的形式得到了自己的份额。1652年1月初,雅克琳前往波尔-罗亚尔修道院。那天,据吉尔贝特回忆,“哥哥忧伤地回到房间,未去看望正在小客厅中等待的雅克琳……”\(^\text{[39]}\)到了1653年6月初,经过雅克琳长期不断地劝说,帕斯卡尔最终正式将她的全部遗产捐赠给波尔-罗亚尔修道院。在帕斯卡尔看来,这个地方“已经开始散发出邪教的气息”。\(^\text{[40]}\)至此,他父亲的遗产已有三分之二离他而去,29岁的帕斯卡尔不得不接受“体面贫穷”的生活状态。

在那之后的一段时间里,帕斯卡尔过着单身汉的生活。他曾多次探望在波尔-罗亚尔修道的妹妹,在这些探访中,他表现出对世俗事务的蔑视,但对上帝却仍未产生真正的信仰。\(^\text{[41]}\)

\textbf{纪念信}

1654年11月23日夜间10点半至12点半之间,帕斯卡尔经历了一次强烈的宗教体验,并立刻写下一张简短的纸条,开头写道:“火。亚伯拉罕之神,以撒之神,雅各之神,不是哲学家和学者之神……”结尾引用了《诗篇》119:16:“我必不忘记你的话。阿门。”据说他将这张纸条小心缝入自己的外套内,每次更换衣物时都随身携带;直到他去世后,一位仆人偶然才发现了这张纸条\(^\text{[42]}\)。这张纸条如今被称为《纪念信》。关于一场马车事故引发他在《纪念信》中描述的宗教体验的说法,部分学者对此提出了质疑\(^\text{[43]}\)。信仰与宗教热忱重新燃起之后,帕斯卡尔于1655年1月前往波尔-罗亚尔较年长的一所修道院,进行为期两周的隐修之旅。在随后的四年中,他在波尔-罗亚尔与巴黎之间频繁往返。正是在皈依之后的这段时期,他开始撰写自己的第一部重要宗教文学作品——《省函》。
\subsection{文学创作}
\begin{figure}[ht]
\centering
\includegraphics[width=6cm]{./figures/5f0bfd1fb7b13cb8.png}
\caption{帕斯卡} \label{fig_BLSpsk_10}
\end{figure}
在文学领域,帕斯卡被视为法国古典时期最重要的作家之一,至今仍被认为是法语散文最伟大的大师之一。他对讽刺与机智的运用影响了后来的政论家。
\subsubsection{《省函》}
从1656至1657年,帕斯卡发表了他那篇著名的对“权谋伦理”(casuistry)的猛烈抨击,这是一种在近代早期广受天主教学者(尤其是耶稣会士,特别是安东尼奥·埃斯科巴尔)采用的道德推理方法。帕斯卡将“权谋伦理”斥为用复杂的逻辑诡辩为道德宽纵甚至各种罪行开脱。这个18封信的系列在1656至1657年间以“路易·德·蒙塔尔特”(Louis de Montalte)的化名发表,激怒了路易十四。国王下令在1660年将这部作品销毁焚烧。1661年,正值“签署声明之争”(formulary controversy)期间,詹森主义的波尔-罗亚尔学派(Port-Royal)遭到谴责并被关闭;与该学派有关的人士被迫签署一项1656年的教皇诏书,谴责詹森的教义为异端。帕斯卡在1657年的最后一封信中甚至直接挑战了教皇亚历山大七世。尽管教皇在公开场合反对帕斯卡,但他也被帕斯卡的论证所折服。

除了宗教影响力外,《省函》作为文学作品也广受欢迎。帕斯卡在信中运用了幽默、嘲讽和尖刻的讽刺,使得这些书信易于被大众接受,并影响了后来的法国作家,如伏尔泰和卢梭的散文风格。

《省函》中也包含了帕斯卡那句被广泛引用的道歉:“我写这封信太长了,只因为我没有时间写得更短。”这一句出自第十六封信,由托马斯·麦克里(Thomas M'Crie)英译如下:

“尊敬的神父们,我的信向来既不如此冗长,也不如此频繁接连而至。这两项缺失都须以‘时间不足’为我辩解。这封信之所以如此冗长,仅仅是因为我没有空闲将它写得更短。”

夏尔·佩罗(Charles Perrault)在评论《省函》时写道:“其中应有尽有——语言的纯净、思想的高贵、推理的坚实、讽刺的机巧,而且始终洋溢着一种无可比拟的优雅。”\(^\text{[44]}\)
\subsubsection{哲学}
帕斯卡或许最为人所知的身份就是哲学家,有人认为他是仅次于笛卡尔的法国第二哲学天才。他在哲学上是笛卡尔式的二元论者。\(^\text{[45]}\)然而,他同时也因反对理性主义(如笛卡尔所代表的思潮)和经验主义这两大对立的认识论体系而被铭记,他更倾向于信仰主义。

在关于上帝的问题上,帕斯卡与笛卡尔意见相左。他写道:“我无法原谅笛卡尔。他的整个哲学本可以完全排除上帝,但他不得不让上帝启动宇宙运行;之后,他就再也不需要上帝了。”\(^\text{[46]}\)他反对像笛卡尔那样运用理性主义来证明上帝的存在,他认为信仰才是核心,因为“理性在此无从断定”。\(^\text{[47]}\)对帕斯卡而言,上帝的本性决定了理性无法揭示祂的存在。人类“活在黑暗中,与上帝疏离”,因为“祂隐藏了自己,使人无法通过知识认识祂”。\(^\text{[48]}\)

帕斯卡最关注的是宗教哲学。正如学者伍德所总结,帕斯卡的神学观点认为人类“生来便置身于一个虚伪的世界,这个世界将我们塑造成虚伪的主体,因此我们极易持续拒绝上帝,并在内心欺骗自己否认自己的罪性”。\(^\text{[49]}\)
\subsubsection{数学哲学}
帕斯卡在数学哲学方面的重要贡献,体现在他的《论几何精神》一书中。这本书最初是他为波尔-罗亚尔著名“小学”(Petites écoles de Port-Royal)编写的一本几何教科书撰写的序言,直到他去世一百多年后才正式出版。

在书中,帕斯卡探讨了如何发现真理的问题,主张理想的方法应是将所有命题建立在已确立的真理之上。然而,他也指出这是不可能的,因为这些已确立的真理还需要其他真理的支持,也就是说,第一原理是无法抵达的。因此,他认为几何学的方法已是尽可能完美的,即以若干公理为前提,从中演绎出其他命题。但即便如此,人类仍无法确认这些公理本身的真实性。

帕斯卡还在《论几何精神》中提出了定义理论。他区分了两种定义:一种是由作者人为规定的惯用标签;另一种则是自然语言中的定义,被所有人理解并能自然指向其所代表的对象。后者属于本质主义哲学的范畴。而帕斯卡认为,科学和数学中只有第一类定义才有意义,因此这两个领域应采纳笛卡尔提出的形式主义哲学。

在另一部作品《论说服术》中,帕斯卡更深入地探讨了几何学的公理化方法,尤其是人们如何被那些作为推理基础的公理所说服的问题。他与蒙田的观点一致,即人类无法通过理性手段在公理或结论上获得确定性。他认为,这些原理只能靠直觉才能掌握,这也恰恰突显了在追寻真理的过程中,必须谦卑地归服于上帝的必要性。
\subsection{《思想录》}
“人不过是一根苇草,自然界中最脆弱的,但他是一根会思考的苇草。”

——布莱兹·帕斯卡,《思想录》第200则

帕斯卡最具影响力的神学著作,即他死后被称为《思想录》(Pensées,“思想”),被广泛认为是一部杰作,同时也是法语散文的里程碑之作。评论家圣伯夫在谈到其中第72则时,称其为“法语中最精美的篇章”。\(^\text{[50]}\)威尔·杜兰特更是盛赞《思想录》为“法语散文中最雄辩的著作”。\(^\text{[51]}\)

帕斯卡在去世前未能完成该书。他原本打算写一部系统连贯的基督教信仰辩护书,原拟书名为《基督教宗教的辩护》。他去世后,人们在他身边找到了大量零散的笔记,首次整理出版是在1669年,书名为《帕斯卡先生关于宗教及其他问题的思想》,此后迅速成为经典。

《辩护》一书的主要策略之一,是借用皮浪主义与斯多葛主义这两种哲学体系的矛盾对立——前者以蒙田为代表,后者以爱比克泰德为代表——来将不信者引入绝望与困惑之中,最终促使其转向信仰上帝。
\subsection{晚年著作与逝世}
\begin{figure}[ht]
\centering
\includegraphics[width=6cm]{./figures/71bcb5b3e491050d.png}
\caption{} \label{fig_BLSpsk_11}
\end{figure}
T.S. 艾略特曾如此形容帕斯卡人生这一阶段:“他是在苦修者中间的一个世俗之人,又是在世俗之人中的一个苦修者。”帕斯卡的禁欲生活源于他一种信念:受苦是人之常情,也是必须的。1659年,帕斯卡病重。在他生命最后几年里,他常拒绝医生的治疗,并说:“别同情我,疾病是基督徒的自然状态,因为唯有如此,我们才能如其所是,处在苦难中,远离一切感官的享乐和安逸,脱离人生旅程中所有激烈的情绪,没有野心,没有贪欲,时时刻刻准备面对死亡。”\(^\text{[52][53]}\)

帕斯卡希望效法耶稣的贫灵精神,怀着热诚与仁爱之心,他说如果上帝让他康复,他将下定决心:“此生再无别的职业或事务,唯愿服侍贫者。”\(^\text{[54]}\)

1661年,路易十四镇压了位于波尔-罗亚尔的詹森主义运动。帕斯卡随即写下他最后一部重要作品之一:《签署表格一事论述》,劝诫詹森主义者不要屈服。同年,他的妹妹雅克琳去世,这一打击使帕斯卡决意停止再就詹森主义发表激烈言论。
\subsubsection{公共交通的发明者}
帕斯卡最后一项重要成就,重拾他在机械方面的天赋,是开创了最早的陆上公共交通系统之一——“五苏马车”,这是一套由马拉多座马车组成的网络,在五条固定线路上运营。他还制定了后世公共交通所沿用的运营原则:固定路线、固定票价(五苏 hence the name),即使没有乘客也照常发车。\(^\text{[55]}\)虽然这些线路最终未取得商业成功,最后一条于1675年停运,\(^\text{[56]}\)但他仍被称为“公共交通的发明者”。\(^\text{[57]}\)
\subsubsection{病逝}
1662年,帕斯卡病情加重,自从妹妹去世后他的情绪也严重恶化。他意识到自己来日无多,希望搬到“绝症医院”接受照料,但医生判定他身体状况太差,无法搬动。8月18日夜晚,帕斯卡在巴黎陷入痉挛状态并接受临终圣礼。第二天清晨去世,享年39岁。他的最后遗言是:“愿上帝永不离弃我。”他葬于巴黎圣艾蒂安·迪蒙教堂墓园。\(^\text{[52]}\)

死后解剖显示,他的胃及腹部器官严重受损,脑部也有损伤。尽管进行了尸检,他病情的确切原因仍未确定。人们推测可能是肺结核、胃癌,或两者兼有。\(^\text{[58]}\)而困扰他终生的头痛普遍被认为与其脑部病变有关。\(^\text{[59]}\)
\subsection{遗产与影响}
\begin{figure}[ht]
\centering
\includegraphics[width=6cm]{./figures/41ea94a879f0a1f7.png}
\caption{帕斯卡安葬于圣艾蒂安·迪蒙教堂的墓志铭} \label{fig_BLSpsk_12}
\end{figure}
法国克莱蒙费朗的大学之一——克莱蒙·费朗-布莱兹·帕斯卡大学以他命名。位于刚果民主共和国卢本巴希的法国学校“布莱兹·帕斯卡学校”亦以其为名。

1969年,埃里克·侯麦导演的电影《我在莫德家的一夜》便基于帕斯卡的作品改编。罗伯托·罗西里尼执导的电视电影《布莱兹·帕斯卡》于1971年在意大利电视台首播。\(^\text{[60]}\)1984年BBC第二台的纪录片《信仰之海》首集中,帕斯卡便是主题人物之一,由唐·库皮特主持。在动画电影《长发公主》中,那只变色龙也被命名为帕斯卡。

一种程序设计语言以帕斯卡命名。2014年,英伟达宣布其新的“Pascal”显卡微架构,亦是以他命名。首批采用Pascal架构的显卡于2016年发布。

2017年游戏《尼尔:机械纪元》中有多个以著名哲学家命名的角色,其中一位名为“帕斯卡”的角色是一台具有自我意识的和平主义机器,是重要的配角。他为希望与人类共处的机器创建了一个和平村落,并充当其他寻找个体性的机器的父亲般角色。

《动物森友会》系列中那只水獭也叫帕斯卡。\(^\text{[61]}\)

小行星4500号亦以帕斯卡命名。\(^\text{[62]}\)

1967年,教皇保禄六世在其通谕《民族发展》中引用了帕斯卡的《思想录》:

真正的人文主义指引人们走向上帝,并承认我们的使命,即赋予人类生活真实意义的使命。人不是衡量人的终极标准。只有超越自己,人才能真正成为人。如帕斯卡所言:“人无限地超越人自身。”

2023年,教皇方济各发布了宗座书信《人的崇高与悲惨》,纪念帕斯卡诞辰四百周年。

帕斯卡也影响了法国社会学家皮埃尔·布迪厄,后者将其1997年的著作命名为《帕斯卡式沉思录》,以及法国哲学家路易·阿尔都塞。
\subsection{作品}
\begin{itemize}
\item 《圆锥曲线论》(Essai pour les coniques,1639年)
\item 《关于真空的新实验》(Expériences nouvelles touchant le vide,1647年)
\item 《液体平衡大实验记述》(Récit de la grande expérience de l'équilibre des liqueurs,1648年)
\item 《算术三角形论》(Traité du triangle arithmétique,约1654年撰写,1665年出版)
\item 《省会书简》(Lettres provinciales,1656–1657年)
\item 《几何精神论》(De l'Esprit géométrique,1657或1658年)
\item 《关于签署表格的著作》(Écrit sur la signature du formulaire,1661年)
\item 《思想录》(Pensées,逝世时未完成,1670年出版)
\item 《论爱情激情的言论》(Discours sur les passions de l'amour,伪作)
\item 《论罪人悔改》
\item 《论恩典的著述》
\end{itemize}
\subsection{参见}
\begin{itemize}
\item 期望值
\item 期赌徒破产问题
\item 期计算机科学先驱列表
\item 期欧仁·纪尧姆作品列表
\item 期帕斯卡桶实验
\item 期帕斯卡分布
\item 期帕斯卡赌注难题
\item 期帕斯卡金字塔
\item 期帕斯卡单纯形
\item 期点数问题
\item 期科学革命
\end{itemize}
\subsection{注释}\\
a,/pæˈskæl/ 发音为“帕斯-卡尔”;英国发音还包括 /-ˈskɑːl, ˈpæskəl, -skæl/,美国发音为 /pɑːˈskɑːl/(帕斯-卡尔);法语发音:[blɛz paskal]\\
b,1 ligne = 2.256 毫米,1 toise = 1.949 米。水银密度为 13.534 克/立方厘米。根据帕斯卡的数据,空气的密度约为 1.1 千克/立方米。\\
\subsection{参考文献}
\begin{enumerate}
\item Wells, John (2008年4月3日).《朗文发音词典》(第3版). Pearson Longman. ISBN 978-1-4058-8118-0.
\item “Pascal”. 《兰登书屋韦氏未删节词典》. 2015年1月6日存档于Wayback Machine.
\item “Pascal, Blaise”. Lexico 英国英语词典. 牛津大学出版社. 存档于2021年12月5日.
\item “Pascal”. 《柯林斯英语词典》. HarperCollins. 存档于2019年8月14日. 于2019年8月14日检索.
\item “Pascal”. Merriam-Webster.com 词典. Merriam-Webster. 于2019年8月14日检索.
\item 参见 “Schickard versus Pascal: An Empty Debate?”(于2014年4月8日存档于Wayback Machine)以及 Jean Marguin 著,《计算工具与机器的历史:三百年的思维机械 1642–1942》(法文),Hermann出版社,1994年,第48页,ISBN 978-2-7056-6166-3.
\item d'Ocagne, Maurice(1893年).《简化计算》(法文). Gauthier-Villars et fils出版社,第245页. 存档于2018年8月9日. 于2010年5月14日检索.
\item Jarrett(2024年12月31日).“人类交通史上最伟大的发明之一”. 《人类交通》. 于2025年1月2日检索.
\item “布莱兹·帕斯卡”. 《天主教百科全书》. 存档于2009年3月10日. 于2009年2月23日检索.
\item Grumball, Kevin Shaun. “提交至诺丁汉大学以获取哲学博士学位的论文” (PDF). 诺丁汉大学. 存档于2020年6月5日 (PDF). 于2022年10月20日检索.
\item “互联网历史资料集”. sourcebooks.fordham.edu. 存档于2022年10月19日. 于2022年10月20日检索.
\item Devlin,第20页.
\item O'Connor, J.J.; Robertson, E.F.(2006年8月).“埃蒂安·帕斯卡”. 苏格兰圣安德鲁斯大学. 存档于2010年4月19日. 于2010年2月5日检索.

\end{enumerate}