% 等距变换
% license Usr
% type Tutor
本文用$(,)$表广义内积运算,即$(\bvec x,\bvec y)\equiv f(\bvec x,\bvec y) $。
\begin{definition}{}
$(V,f),(V',f')$是两个正交空间或辛空间。若线性映射$\sigma:V\rightarrow V'$是双射且保内积不变,即\begin{equation}
(x,y)=(\sigma x,\sigma y)~,
\end{equation}
则称$\sigma $是\textbf{等距映射}(isometry)。若$\opn{dim}V=\opn{dim}V'$,则称之为等距变换。
\end{definition}
由于等距变换是线性映射,因此保加法和数乘运算,是全体向量的自同构映射。由定义式可知,等距变换复合依然是等距变换。因此全体等距变换的集合可记作$\opn{Aut}(V,g)$,表明其自同构成群和保内积的性质。从正交变换的\autoref{def_ortho_1}~\upref{ortho}可知,正交变换是度量矩阵为$E$的等距变换,因此我们可以拓展正交变换的定义为\textbf{非退化}正交空间的等距变换,并称\textbf{非退化}辛空间的等距变换为\textbf{辛变换}。

等距变换的保内积性质