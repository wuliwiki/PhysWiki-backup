% 流体、固体和气体
% 流体力学|剪应力|正应力

我们都知道物质的三态有流体、固体和气体,这三者在我们生活的世界里无处不在.我们脚底下的土地,我们呼吸的空气,我们喝的水、吹的风、踏过的河流…… 如何在物理定义上给出这三种物质形态的一个划分呢?

\subsection{应力的角度}
\textbf{流体}是这样的一类物质,它能够在\textbf{剪应力}的作用下连续地发生形变.

\begin{figure}[ht]
\centering
\includegraphics[width=10cm]{./figures/SLG_1.png}
\caption{几种不同的施加应力的方式}} \label{SLG_fig1}
\end{figure}

图中的“剪切”和“扭曲”都是剪应力的一种形式.在数学上,剪应力有更简单的表达式定义,它体现了剪应力与正应力的明显的区分:
\begin{equation}
S_{ij}=\frac{\partial u_i}{\partial x_j}
\end{equation}
