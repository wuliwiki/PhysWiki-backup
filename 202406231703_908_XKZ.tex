% 相控阵
% license CCBYSA3
% type Wiki

(本文根据 CC-BY-SA 协议转载自原搜狗科学百科对英文维基百科的翻译)

\begin{figure}[ht]
\centering
\includegraphics[width=10cm]{./figures/67b1f779e41710d7.png}
\caption{相控阵的工作原理动态图它由一个由发射机(TX)供电的天线单元阵列(A)组成。每个天线的馈电电流通过由计算机(C)控制的移相器(φ)提供。移动的红线是每个天线单元发射的无线电波的波阵面示意图。单个波阵面是球形的,但是它们在天线前组合(叠加)形成一束在特定方向传播的平面波。移相器使无线电波在线路上依次延迟,因此每个天线发射波前的时间比它下面的天线晚。这导致产生的平面波与天线轴成θ角。通过改变相移,计算机可以立即改变光束的角度θ。大多数相控阵都有二维天线阵列,而不是上图的线性阵列,波束可以在二维方向上转向。无线电波的传播速度在视觉上减慢了。} \label{fig_XKZ_1}
\end{figure}

在天线理论中,相控阵通常指的是电子扫描阵列,这是一种由计算机控制的天线阵列,它产生的无线电波束可以在无需移动天线的条件下控制其指向不同的方向。[1][2][3][4][5][6][7][8] 在阵列天线中,来自发射机的射频电流以正确的相位关系馈送到各个天线,使得来自各个天线的无线电波相加在一起以增加预定方向的辐射增益,同时抵消以抑制其他方向的辐射。在相控阵中,来自发射机的功率通过被称为移相器的设备馈送到天线,移相器由计算机系统控制,计算机系统可以改变相位,从而将无线电波束导向不同的方向。由于阵列必须由许多小天线(有时几千个)组成才能获得高增益,相控阵主要适用于无线电频谱的高频端、超高频和微波波段,在这些波段中单个天线元件非常小。

相控阵被发明用于军事雷达系统,可以快速扫描雷达波束,以探测飞机和导弹。这些相控阵雷达系统现在得到了广泛的使用,相控阵正在向到民用领域扩展。相控阵原理也用于声学,相控阵声波传感器用于医学超声成像扫描仪(相控阵超声)、油气勘探(反射地震学)和军事声纳系统。[9]

术语“相控阵”在偶尔也用于馈电功率的相位以及天线阵列的辐射方向图固定的非阵列天线。[6][10] 例如,通过馈电产生特定的辐射模式的调幅AM广播无线电天线由多个主辐射器组成,也称为“相控阵”。

\subsection{类型}
无源相控阵或无源电子扫描阵(PESA)是一种相控阵,其中天线元件连接到单个发射机和/或接收机,如顶部动画所示。PESA是最常见的相控阵类型。

有源相控阵或有源电子扫描阵(AESA)是一种相控阵,其中每个天线单元都有自己的发射机/接收机单元,全部由计算机控制。有源阵列是一种更先进的第二代相控阵技术,广泛用于军事领域;与PESA不同,它们可以同时向不同方向辐射多个频率的多束无线电波。

共形天线是一种相控阵,其中各个天线不是布置在平面上,而是安装在曲面上。移相器补偿由于天线元件在表面上的不同位置而导致的不同路径长度的波,允许阵列辐射平面波。共形天线用于飞机和导弹,将天线集成到飞机的弯曲表面以减少空气动力阻力。

\subsection{历史}
\begin{figure}[ht]
\centering
\includegraphics[width=8cm]{./figures/37938c249e3db96d.png}
\caption{上图是费迪南·布劳恩(Ferdinand Braun)1905年的定向天线,采用相控阵原理,由3个单极天线组成一个等边三角形。一个天线馈线的四分之一波长延迟导致阵列以波束形式辐射。延迟可以手动切换到3个馈电中的任何一个,从而将天线波束旋转120°。} \label{fig_XKZ_2}
\end{figure}
\begin{figure}[ht]
\centering
\includegraphics[width=6cm]{./figures/a05705147b19f7da.png}
\caption{美国阿拉斯加PAVE PAWS有源相控阵弹道导弹探测雷达。它于1979年建成,是首批有源相控阵之一。} \label{fig_XKZ_3}
\end{figure}
相控阵传输最初是由诺贝尔奖获得者卡尔·费迪南德·布劳恩(Nobel laureate Karl Ferdinand Braun )在1905年展示的,他展示了无线电波在一个方向上的增强传输。[11][12] 二战期间,诺贝尔奖获得者路易斯·阿尔瓦雷斯(Luis Alvarez )在一个用于“地面控制方法”的快速可控雷达系统中使用相控阵传输,该系统有助于飞机着陆。与此同时,德国的GEMA建造了马穆特(Mammut 1)1号。[13]后来,在剑桥大学开发了几个大型相控阵之后,它被用于射电天文学,并为安东尼·休伊什和马丁·赖尔(Antony Hewish and Martin Ryle)赢得了诺贝尔物理学奖。这种设计也用于雷达,并在干涉测量天线中得到推广。
\begin{figure}[ht]
\centering
\includegraphics[width=6cm]{./figures/0721f1d47aad3ada.png}
\caption{组成平面阵列的一些2677交叉偶极天线单元的特写。这种天线产生的窄“铅笔”波束只有2.2度宽} \label{fig_XKZ_4}
\end{figure}
\begin{figure}[ht]
\centering
\includegraphics[width=6cm]{./figures/c54370eeb5d02cd3.png}
\caption{美国F-22猛禽战斗机机头内的有源相控阵雷达天线。几乎所有战斗机现在都使用相控阵雷达。} \label{fig_XKZ_5}
\end{figure}
2004年,加州理工学院的研究人员展示了第一台集成硅基相控阵接收机(24GHz,8个单元)。[14]随后,加州理工学院团队于2005年演示了一款24GHz的互补金属氧化物半导体(CMOS)相控阵发射机,[15]并于2006年演示了一款集成天线的完全集成的77GHz相控阵收发机。[16][17] 2007年,美国国防部高级研究计划局(DARPA)的研究人员宣布了一种16元相控阵雷达天线,它也与所有相关电路集成在一个硅片上,工作频率为30-50 GHz。[18][18]
\begin{figure}[ht]
\centering
\includegraphics[width=6cm]{./figures/238447fbbe4fbba4.png}
\caption{美国F-22猛禽战斗机机头内的有源相控阵雷达天线。几乎所有战斗机现在都使用相控阵雷达。} \label{fig_XKZ_6}
\end{figure}
BMEWS & PAVE PAWS雷达
单个天线辐射的信号相对幅度以及它们之间的叠加和相消干涉效应决定了阵列的有效辐射方向图。相控阵可以指向固定的辐射方向,或者在方位角或仰角上快速扫描。1957年,加利福尼亚休斯飞机公司(Hughes Aircraft Company, California)首次演示了方位和仰角的同时电子扫描的相控阵天线。[19]
\begin{figure}[ht]
\centering
\includegraphics[width=6cm]{./figures/dea4cc241f728c6a.png}
\caption{第二次世界大战的Mammut相控阵雷达} \label{fig_XKZ_7}
\end{figure}
\su