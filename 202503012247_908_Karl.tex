% 卡尔·威尔施特拉斯
% license CCBYSA3
% type Wiki

本文根据 CC-BY-SA 协议转载翻译自维基百科\href{https://en.wikipedia.org/wiki/Karl_Weierstrass}{相关文章}。
\begin{figure}[ht]
\centering
\includegraphics[width=6cm]{./figures/f068c4d82f5a5984.png}
\caption{} \label{fig_Karl_1}
\end{figure}
卡尔·西奥多·威尔斯特拉斯(/ˈvaɪərˌstrɑːs, -ˌʃtrɑːs/;德语:Weierstraß [ˈvaɪɐʃtʁaːs];1815年10月31日 – 1897年2月19日)是一位德国数学家,常被称为“现代分析学之父”。尽管他未获得大学学位,但他学习了数学并接受了教师培训,最终教授数学、物理学、植物学和体育学。[3] 他后来获得了荣誉博士学位,并成为柏林大学的数学教授。

威尔斯特拉斯做出了许多贡献,其中包括形式化函数连续性的定义和复分析,证明了中值定理和博尔查诺–威尔斯特拉斯定理,并利用后者研究闭合有界区间上连续函数的性质。
\subsection{传记}
威尔斯特拉斯出生于西法兰西省恩尼格尔洛附近的一个名为奥斯滕费尔德的村庄,来自一个天主教家庭。[4]

卡尔·威尔斯特拉斯是威廉·威尔斯特拉斯和西奥多拉·冯德福尔斯特的儿子,前者是政府官员,后者是天主教的莱茵兰人。他对数学的兴趣始于他在帕德博恩的提奥多里亚努姆中学时。毕业后,他被送往波恩大学,准备为政府工作;为此,他的学习方向应该是法律、经济和金融,这与他自己希望学习数学的愿望发生了直接冲突。他通过忽视计划中的课程,继续私下学习数学,最终导致他未能获得学位而离开了大学。

威尔斯特拉斯继续在明斯特学院(当时以数学闻名)学习数学,且他的父亲为他争取到了一所明斯特师范学校的位置;他在那里的努力最终使他获得了该市的教师资格。在这一学习期间,威尔斯特拉斯听取了克里斯托夫·古德曼的讲座,并对椭圆函数产生了兴趣。

1843年,他在西普鲁士的德意志克罗内教书,1848年起,他在布劳恩斯贝格的霍西亚努姆中学任教。[5] 除了数学,他还教授物理学、植物学和体育学。[4] 在某个时期,威尔斯特拉斯可能与他朋友卡尔·威廉·博尔哈特的寡妇有一个私生子("弗朗茨")。[6][有争议 – 讨论]