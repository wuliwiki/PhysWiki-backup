% 指数(复数)
% keys 复数|幂函数|反函数

\begin{issues}
\issueDraft
\issueTODO
\end{issues}


\pentry{复变函数\upref{Cplx}}

\subsection{复数幂函数}
我们再来将复数的幂函数分解为模长和相位的形式(令 $z = \abs{z} \E^{\I\phi(z)}$, $a = a_I + \I a_R$ )
\begin{equation}
z^a = \abs{z}^{a_R} \E^{-\phi(z) a_I} \E^{\I[\ln\abs{z}a_I + \phi(z)a_R]}
\end{equation}
可见 $z^a$ 的模长和幅角都分别与 $z$ 和 $a$ 有关. 一般情况下, 这是一个比较复杂的函数, 含有不同的分支(因为 $\phi(z)$ 可以加整数个 $2\pi$).% 未完成: 分支是什么?
当且仅当 $a$ 为整数时才不会出现分支. 在数值计算中, 分支切割线出现在 $\phi(z) = \pm\pi$ 处, 这是因为数值计算通常取 $\phi(z)\in(-\pi, \pi]$.
