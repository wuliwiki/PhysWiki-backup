% 南京理工大学 普通物理 B(845)模拟五套卷 第一套
% license Usr
% type Note

\textbf{声明}:“该内容来源于网络公开资料,不保证真实性,如有侵权请联系管理员”

\subsection{一、 填空题(24 分,每空 2 分)}
1. 一质点作直线运动,它的运动方程是$x=bt-ct^2$,$b,c$是常数,则此质点的
速度是______,加速度是__________。

2. 质量分别为 $200kg$ 和 $500kg$ 的甲、乙两船静止于湖中,甲船上一质量为 $50kg$
的人通过轻绳拉动乙船,经 5 秒钟乙船速度达到 $0.5m/s$,则人拉船的恒力为
______,甲船此时的速度为__________。

3. 花样滑冰运动员绕过自身的竖直轴运动,开始时两臂伸开,转动惯量为$I_0$ ,
角速度为$\omega_0$ 。然后她将两臂收回,使转动惯量减少为$\frac{I_0}{2}$ ,这时她转动的角速度变为_______。

4. 一弹簧振子作简谐振动,总能量为 E1,如果简谐振动振幅增加为原来的两倍,
重物的质量增为原来的四倍,则它的总能量 E2 变为_______。

5. 火车$A$以$20m/s$的速度向前行驶 , $A$车的司机听到本车的汽笛频率为$120Hz$,另一火车 $B$,以$25m/s$的速度向 $A$ 迎面驶来,则 $B$车司机听到 $A$ 车的频率是 ________。(设空气中声速为 3$40m/s$)

6. 静电场的环路定理的数学表达式为___________________。该式的物理意义
是______________。该定理表明,静电场是_____场。

7. 一瓶氢气和一瓶氧气温度相同,若氢气分子的平均平动动能为 $k$,则氧气分子的平均平动动能是____________,氧气的温度是________。
\subsection{二、 填空题(20 分,每空 2 分)}
1. 在双缝干涉实验中,两缝的间距为 $0.6mm$,照亮狭缝 $S$ 的光杠杆汞弧灯加上
绿色滤光片,在 $2.5m$ 远处的屏幕上出现干涉条纹,测得相邻两明条纹中心
的距离为 $2.27mm$,则计算入射光的波长为_________。

2. 若电子在垂直于磁场的平面内运动,均匀磁场作用于电子上的力为 $F$,轨道
的曲率为 $R$,则磁感应强度的大小为__________。

3. 波长 $600nm$ 的单色光垂直入射在一光栅上 , 第二级明条纹分别出现在$\sin\theta=02$处,第四级缺级,则光栅上相邻两缝的间距 $(a+b)$为__________,
光栅上狭缝可能的最小宽度为_________。

4. 如图所示,均匀磁场的磁感应强度为 $B=0.2T$,方向沿 $x$ 轴正方向,则通过
aefd 面的磁通量为________。

5. 一质量带有电荷 $q$,以速度 $v$ 在半径为 $R$ 的圆周上作匀速圆周运动,该运动
的带电质点在轨道中心所产生的磁感应强度 $B$ 为__________,该运动的带电
质点轨道运动的磁矩为__________

6. 处于激发态的钠原子,发出波长为 $589nm$ 的光子的时间平均约为_______$s$,
根据不确定度关系式,光子能量不确定量的大小为________,发射波长的不
确定度范围(即所谓谱线宽度)是__________。

7. 若粒子在均匀磁场中沿半径为 $R$ 的圆形轨道运动,磁场的磁感应强度为 $B$,
则粒子的德布罗意波长为_______。
\subsection{三、(14 分)}
一质量为 $0.20kg$ 的球,系在长为 $2m$ 的绳索上,绳索的另一端系在天花板
上,把小球移开,使绳索与铅直方向成一角度,然后从静止放开。求:( 1)在
绳索从角 30°到角 90°时,重力和张力所作的功。(2)物体在最低位置时的动
能和速率
\subsection{四、(14 分)}
气缸内有一定量的氧气(看成刚性分子理想气体),作如图所示的循环过
程,其中 $ab$ 为等温过程,$bc$ 为等体过程,$ca$ 为绝热方程。已知 $a$ 点的状态参量
为$P_a,V_a,T_a$,$b$ 点的体积$V_b=3V_a$,求该循环的效率。
\subsection{五、(12 分)}
两个半径分别为 $R_1$ 和 $R_2$ 的同心球壳,中间是空气,构成一球形电容器,
设所带电量分别为$+Q$ 和$-Q$ 均匀分布,求:
(1)两球壳之间的电场强度。(2)两球壳之间的电势差。(3)电容器
的电容。
\subsection{六、(12 分)}
半径为 $R=0.1m$ 的半圆形闭合线圈,载有电流 $I=10A$,放在均匀磁场中,
磁场方向与线圈平面平行,如图所示,已知 $B=0.5T$,求:\\\\
(1) 线圈所受力矩的大小和方向(以直径为转轴);\\\\
(2) 若线圈受上述磁场作用转到线圈平面与磁场垂直的位置,则力矩
作功为多少?
\subsection{七、(14 分)}
如图所示,$AB$ 和 $CD$ 为两根金属棒,长度嗾使 $1m$,电阻都是 $4\Omega$,放置在
均匀磁场中,已知磁场的磁感应强度 $B=2T$,方向垂直于纸面向里,当两根金属
棒在导轨上分别以 $v=4m/s$ 和 $v=2m/s$ 的速度向左运动时,忽略导轨的电阻,求:\\\\
(1) 两金属棒各自的动生电动势;\\\\
(2) 金属棒两端的电势差和;\\\\
(3) 金属棒中点和之间的电势差。\\\\
\subsection{八、(14 分)}
一平面简谐波沿 $x$ 轴正向传播,波的振幅 $A=10cm$,角频率为 $7\pi rad/s$。当
$t=1.0s$ 时,$x=10cm$ 处的 $a$ 质点正通过其平衡位置向 $y$ 轴负方向运动,而 $x=20cm$处的 $b$ 质点正通过 $y=5.0cm$ 向 $y$ 轴正方向运动,设该波波长>10cm,求该平面波的表达式。
\subsection{九、(13 分)}
波长为 $600nm$ 的单色光垂直入射在光栅上,已知第二级明纹出现在 $\sin \theta=0.20$ 处,第三级为缺级。求:(1)光栅常数。(2)光屏上可以看到的明纹数。