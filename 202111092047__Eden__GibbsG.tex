% 吉布斯自由能
% 吉布斯自由能|吉布斯函数化学势|自由焓

\pentry{热力学第一定律\upref{Th1Law}}

吉布斯自由能,也称自由焓,常用字母 $G$ 表示.$G$ 是一个热力学态函数.\textbf{对于一个恒温恒压过程,系统对外做的非体积功小于(不可逆过程)或等于(可逆过程)自由焓减小量}\footnote{非体积功是除去 $p\dd V$ 外的做功,例如 $-H\dd M$,$-E\dd P$ 等不同形式的功.}.我们把吉布斯自由能定义为 $G=U-TS+pV$,则在可逆过程中 $\dd G= -S\dd T+V\dd p$.对于不可逆过程, $\dd U<T\dd S-p\dd V$,所以 $\dd G<-S\dd T+V\dd p$.所以我们有\textbf{等温等压条件下热平衡的判据}:

自发的等温等压过程只能沿着吉布斯函数(自由焓)减小的方向进行.等温等压条件下热平衡的判据是
\begin{equation}
\Delta G>0
\end{equation}
写成一阶变分和二阶变分的形式即
\begin{equation}
\delta G=0,\delta^2 G>0
\end{equation}

\subsection{化学势}

对于粒子数固定为 $N$ 的封闭系统,$\dd G=-S\dd T+V\dd p$.对于单相物质,自由焓是广延量,每个分子占有的自由焓应该相等(忽略表面张力导致的表面能).我们称化学势为
\begin{equation}
\mu_{\text{单个粒子}}(p,T)=\frac{G(p,T)}{N}
\end{equation}

有时也用摩尔吉布斯函数来代表化学势,一般用符号 $\mu$ 或 $G_m$ 来表示:

\begin{equation}\label{GibbsG_eq1}
\mu(p,T)=G_m(p,T)=\frac{G}{\nu}
\end{equation}

\subsection{理想气体的吉布斯函数}
利用理想气体熵的表达式(\upref{MacroS}\autoref{MacroS_eq2})和 $G_m=-S_m\dd T+V\dd p$,可以得到理想气体吉布斯函数的表达式
\begin{equation}
G_m=\int C_{p,m} \dd T - T\int C_{p,m} \frac{\dd T}{T}+RT\ln p+G_{m0}
\end{equation}
可以写成
\begin{equation}
\begin{aligned}
&G_m=RT(\phi(T)+\ln p)\\
&\mu_{\text{单个粒子}} = kT(\phi(T)+\ln p)
\end{aligned}
\end{equation}

这个表达式在研究化学反应时非常重要.