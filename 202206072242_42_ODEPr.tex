% 基本知识(常微分方程)
% 常微分方程|小结

\pentry{常微分方程简介\upref{ODEint}}
在常微分方程简介\upref{ODEint}一节中,可以知道,如果一个方程未知的是函数,且方程含有未知函数的导数,则称这样的方程为\textbf{微分方程}.若微分方程中的未知函数是一元函数(即未知函数仅有一个自变量),则称着方程为\textbf{常微分方程};相反的,若未知函数是多元函数,则称\textbf{偏微分方程}.所以,在自变量为 $x$,且仅有一个未知函数 $y$ 时,一般的常微分方程形式为
\begin{equation}
F\qty(x,y,y',\cdots,y^{(n)})=0
\end{equation}
其中, $y^{(n)}=\dv[n]{y}{x}$.

常微分方程理论通常也研究更一般的方程组.通常,常微分方程组中方程的个数与其中出现的未知函数个数相同,而所有未知函数都是同一自变量的函数.所以,一般的常微分方程组形式为
\begin{equation}\label{ODEPr_eq1}
\left\{\begin{aligned}
&F_1\qty(x,y_1,y_1',\cdots,y_1^{(n_1)},\cdots,y_n,y_n',\cdots,y_n^{(n_m)})=0\\
\vdots\\
&F_n\qty(x,y_1,y_1',\cdots,y_1^{(n_1)},\cdots,y_n,y_n',\cdots,y_n^{(n_m)})=0
\end{aligned}\right.
\end{equation}
这里,$y_i,(i=1,\cdots,n)$ 都是 $x$ 的未知函数,函数 $F_i$ 是 $\qty(\sum\limits_{j=1}^{m}n_j+n+1)$ 个变量的函数.函数 $F_i$ 可能并不是对变量的所有值都有定义,所以要讨论 $F_i$ 的定义区域 $B$ (假定每个 $F_i$ 的定义区域都是 $B$).这里的区域是  $\qty(\sum\limits_{j=1}^{m}n_j+n+1)$ 个变量 
\begin{equation}
x,y_1,y_1',\cdots,y_1^{(n_1)},\cdots,y_n,y_n',\cdots,y_n^{(n_m)}
\end{equation}
 的$\qty(\sum\limits_{j=1}^{m}n_j+n+1)$ 维坐标空间中的区域\footnote{区域是指这样的集合,其中每一点都有一个邻域属于该集合}.其中,函数 $y_i$ 的最大阶数 $n_i$ 称维方程组\autoref{ODEPr_eq1} \textbf{关于 $y_i$ 的阶},而称数 $\sum\limits_{i=1}^m n_m$ 为\textbf{方程组\autoref{ODEPr_eq1} 的阶}.如果自变量 $x$ 的函数 $y_i=\varphi_i(x)$ 在区间 $r_1<x<r_2$ 上有定义,并把 $\varphi_i(x)$ 代入\autoref{ODEPr_eq1} 时,得到在区间 $r_1<x<r_2$ 上关于 $x$ 的恒等式,则称 $y_i=\varphi_i(x)$ 为方程组\autoref{ODEPr_eq1} 的\texbf{解};而称区间 $r_1<x<r_2$ 为解 $r_1<x<r_2$ 