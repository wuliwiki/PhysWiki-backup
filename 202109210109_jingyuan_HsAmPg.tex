% 等差数列(高中)
% 高中|等差数列

\subsection{定义}
从第2项起,每一项与前一项的差是同一个常数,我们称这样的数列为\textbf{等差数列},称这个常数为等差数列的\textbf{公差},通常用字母 $d$ 表示.

\textsl{注:常数列也是等比数列.}

\subsection{通项}
如果等差数列 $\begin{Bmatrix} a_n \end{Bmatrix}$的首项是 $a_1$,公差是 $d$,那么根据等差数列的定义可得
\begin{equation}
\begin{aligned}
&a_1 = a_1,\\
&a_2 = a_1 + d,\\
&a_3 = a_2 + d = a_1 + 2d,\\
&\cdots \\
&a_n = a_{n-1} + d = a_1 + (n - 1)d
\end{aligned}
\end{equation}

当 $n = 1$ 时
\begin{equation}
a_1 = a_1 + (1 - 1)d = a_1
\end{equation}
也就是说这个公式对 $n = 1$ 同样适用.

综上,等差数列通项公式为
\begin{equation}
a_n = a_1 + (n - 1)d
\end{equation}

\textsl{注:这里需要说明一下,带入$n = 1$验算的原因是,我们推算的是$n > 1$时的通项公式,不能说明对首项cheng'li}