% 激光原理
% 光学|现代光学|原子物理

\begin{issues}
\issueDraft
\issueTODO
\end{issues}

激光器是现代光学的伟大成就之一,其拥有的窄频宽、单模式等高相干性的优势是许多近代实验,如迈克尔逊-莫雷的实验,成功的必要因素之一。激光器的核心原理是受激辐射。

\subsection{能级和光量子的概念}
量子力学告诉我们,在原子中的电子的能量并不是连续的。电子的能量总是一个又一个特定的能级中跳变。例如氢原子的能级分布就如下图:\begin{figure}[ht]
\centering
\includegraphics[width=7cm]{./figures/97c81e92cc27a211.png}
\caption{氢原子能级} \label{fig_LaserT_1}
\end{figure}
可以有很多中方法来让电子的能量发生跳变,例如使用高速电子轰击原子,但最常见的方法是使用光子。光子的能量为$E=h\nu$,其中$h$为普朗克常数,$\nu$为光子的频率。例如要让氢原子的电子从能级n=1跃迁到n=2,就需要一个频率为2.46PHz的光子,对应的波长是122nm,属于远紫外光,其他频率的光子不行。
\subsection{受激辐射}
受激辐射的概念是由爱因斯坦最先提出的,源自玻尔兹曼分布和普朗克分布之间矛盾。让我们现在来阐释一下这个矛盾。

想象一个理想黑体内有一个两能级系统。能量低的能级称为能级1,位于这个能级的电子的数量为$n_1$;为能量高者称为能级2,位于这个能级的电子数为$n_2$,能级之间的能量差为$\Delta E$。热平衡时,根据玻尔兹曼分布,应有:
\begin{equation}
\frac{n_2}{n_1}=\exp(-\frac{\Delta E}{k_B \ T})
\end{equation}

位于能级2的电子会自发跳转到能级1,并释放$\Delta E$的能量,称为自发辐射。根据隔壁词条“跃迁概率(一阶微扰)\upref{HionCr}”的介绍,电子发生自发辐射的概率是一个定值$A_{21}$,与时间、空间、dianzu等无关。
\subsection{谐振腔}
