% 理想气体分压定律
% 理想气体|分压|状态方程|压强

\pentry{理想气体\upref{PVnRT}}

如果一个容器中有多种理想气体, 由于我们忽略分子间的相互作用, 每种气体分别满足理想气体状态方程\autoref{eq_PVnRT_1}~\upref{PVnRT}。
\begin{equation}\label{eq_PartiP_1}
P_i V = n_i R T~.
\end{equation}
其中 $n_i$ 是第 $i$ 种气体的摩尔数; $P_i$ 是该气体对容器壁撞击产生的压强,叫做该气体的\textbf{分压}; $V$是该气体能够运动的总的体积,即整个容器的大小。 由于平衡时所有气体的体积和温度相同, 所以每种气体的分压与它的分子个数成正比。


根据 “分子撞击对容器壁的压强\upref{MolPre}” 的压强产生原理, 当存在多种不同气体时, 总压强等于每种气体压强之和。
\begin{equation}\label{eq_PartiP_2}
P = \sum_i P_i~.
\end{equation}
另外, 总分子数等于每种气体的分子数之和
\begin{equation}\label{eq_PartiP_3}
n = \sum_i n_i~.
\end{equation}
所以将所有气体的\autoref{eq_PartiP_1} 相加, 再使用\autoref{eq_PartiP_2} 和\autoref{eq_PartiP_3}, 就重新得到了理想气体状态方程
\begin{equation}
P V = n RT~,
\end{equation}
可见该方程同样适用于混合气体。

\begin{figure}[ht]
\centering
\includegraphics[width=5cm]{./figures/1ed8fdf58ebcdbf4.pdf}
\caption{请添加图片描述} \label{fig_PartiP_3}
\end{figure}

\begin{figure}[ht]
\centering
\includegraphics[width=5cm]{./figures/da335bd819fc7868.pdf}
\caption{请添加图片描述} \label{fig_PartiP_4}
\end{figure}

\addTODO{需要例题}
