% 子空间的直和 补空间
% keys 向量空间|子空间|直和空间|直和

\begin{issues}
\issueTODO
\end{issues}

\pentry{子空间\upref{SubSpc}}
\begin{definition}{直和}\label{DirSum_def1}
令域 $\mathbb F$ 上的向量空间 $V_1$ 和 $V_2$ 为 $V$ 的两个子空间\upref{SubSpc}, 满足
\begin{equation}
V_1 \cap V_2 = \qty{0}
\end{equation}
且任意 ${v} \in V$ 都能表示为 $V_1$ 和 $V_2$ 中向量的线性组合, 即
\begin{equation}
{v} = c_1 {v_1} + c_2 {v_2}
\qquad
({v_1} \in V_1,\ {v_2} \in V_2,\ c_1, c_2 \in \mathbb F)
\end{equation}
那么空间 $V$ 就是 $V_1$ 和 $V_2$ 的\textbf{直和空间}, 用\textbf{直和(direct sum)}运算记为
\begin{equation}
V = V_1 \oplus V_2
\end{equation}
我们把这两个子空间叫做\textbf{互补的}, 即 $V_2$ 是 $V_1$ 在 $V$ 中的\textbf{补空间(complement space)}, 反之亦然.
\end{definition}

直和空间 $V_1 \oplus V_2$ 中的所有向量可以分为三组, 分别是 $V_1$ 中的向量, $V_2$ 的向量, 以及只能表示为 $V_1$ 和 $V_2$ 中非零向量之和的向量.

\subsection{多个子空间的直和}
同理, 也可以把向量空间 $V$ 表示为多个子空间的直和
\begin{equation}
V = V_1 \oplus \dots \oplus V_n
\end{equation}
这可以理解为先把 $V_1, V_2$ 做直和, 再把所得空间与 $V_3$ 做直和, 等. 事实上, 根据向量加法的分配律, 容易证明直和也满足分配律, 即
\begin{equation}
(V_1 \oplus V_2) \oplus V_3 = V_1 \oplus (V_2 \oplus V_3)
\end{equation}
所以多个空间的直和也无需添加括号.

\subsection{直和空间的基底}
从基底的角度来看, 若 $V_1$ 和 $V_2$ 中分别有一组基底 ${\alpha_i}$ $(i = 1, \dots, N_1)$ 和 ${\beta_i}$ $(i = 1, \dots, N_2)$, 那么直和空间中 $V$ 的任意向量可以表示为
\begin{equation}\label{DirSum_eq1}
{v} = \sum_i a_i {\alpha_i} + \sum_j b_j {\beta_j} \qquad (v\in V, a_i, b_i \in \mathbb F)
\end{equation}
\begin{theorem}{}\label{DirSum_the1}
$\alpha_1, \dots, \alpha_{N_1}, \beta_1, \dots, \beta_{N_2}$ 是 $V$ 的一组基底.
\end{theorem}

证明: 由于已经有\autoref{DirSum_eq1}, 我们只需要证明 $\alpha_1, \dots, \alpha_{N_1}, \beta_1, \dots, \beta_{N_2}$ 是线性无关的. 使用反证法, 若有不全为零的系数使
\begin{equation}
\sum_i a_i {\alpha_i} + \sum_j b_j {\beta_j} = 0
\end{equation}
令
\begin{equation}
u = \sum_i a_i {\alpha_i} = -\sum_j b_j {\beta_j}
\end{equation}
那么 $u \ne 0$ 且 $u \in V_1$ 且 $u \in V_2$. 这违反了\autoref{DirSum_def1}. 证毕.

\begin{corollary}{}\label{DirSum_cor1}
直和空间 $V = V_1 \oplus V_2$ 的维数等于 $V_1, V_2$ 的维数相加.
\end{corollary}
证明: \autoref{DirSum_the1} 中的基底有 $N_1 + N_2$ 个. 证毕.

注意 $N_1, N_2$ 可以等于零, 零维线性空间仅由零向量一个元素构成.

\begin{example}{}\label{DirSum_ex1}
若三维空间中有两个不共线的几何向量 $\bvec{v_1}, \bvec{v_2}$, 它们张成一个平面, 或二维子空间. 另有一个向量 $\bvec{v_3}$, 独自张成一条直线, 即一维空间.

若 $\bvec{v_3}$ 落在 $\bvec{v_1}, \bvec{v_2}$ 张成的平面内, 则三个向量的所有线性组合仍然在该平面内, 所以直和空间仍然是该平面.

若 $\bvec{v_3}$ 落在平面外, 则三个向量将会张成整个三维空间, 所以直和就是三维空间. 此时两个子空间在该三维空间中互补.
\end{example}

\addTODO{补是不唯一的(\autoref{DirSum_ex1})}
\addTODO{给出一个空间和子空间, 如何求一个补空间?}

\begin{theorem}{唯一分解}
直和空间 $V = V_1 \oplus V_2$ 中, 对任意 $v \in V$ 都存在唯一的 $v_1 \in V_1$ 和 $v_2 \in V_2$ 使得
\begin{equation}
v = v_1 + v_2
\end{equation}
\end{theorem}
