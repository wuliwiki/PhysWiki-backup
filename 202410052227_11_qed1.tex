% 洛伦兹群
% keys 洛伦兹变换|度规张量|四矢量|李群|李代数
% license Xiao
% type Tutor

\begin{issues}
\issueTODO
\end{issues}

\pentry{群\nref{nod_Group}, 李群, 算符对易与共同本征函数\nref{nod_Commut}, 闵可夫斯基空间\nref{nod_MinSpa},协变和逆变\nref{nod_CoCon}}{nod_6ce8}

\subsection{洛伦兹变换}
物质运动可以看成一连串事件的发展过程。在四维时空里,每个事件都有相应的坐标。对于同一事件,在惯性系 $A$ 里用 $(t, \bvec{x})$ 表示,在惯性系 $A'$ 里用 $(t', \bvec{x'})$ 表示。这就是四维时空的四矢量,洛伦兹变换就是作用在一个惯性系的四维矢量上,使之变为另一个惯性系的四矢量。洛伦兹变换必须是线性的,\textbf{譬如一个惯性系中的匀速运动,换了参考系也依然是匀速运动。}从 $(t, \bvec{x})$ 到 $(t', \bvec{x'})$ 的线性变换是保证一切物理定律在不同惯性系下\textbf{形式不变}的基本要求。

接下来我们考虑两个特殊事件。设惯性系 $A$ 和 $A'$ 在 $t=t'=0$ 时在原点重合,此时原点发出一股光信号,该事件为 $(0,0,0,0)$。在此后的某个时刻,P点接收一光信号。在惯性系 $A$ 里该事件为为 $(t,x,y,z)$,在惯性系 $A'$ 里该事件为 $(t',x',y',z')$。采取自然单位制 $c=1$。那么我们有
\begin{equation}\label{eq_qed1_6}
\begin{array}{c}
t^{2}-(x^{2}+y^{2}+z^{2})=0~,\\
t^{\prime 2}-(x^{\prime 2}+y^{\prime 2}+z^{\prime 2})=0~.
\end{array}
\end{equation}

对于一般的两个事件,它们之间可能并不是用光信号联系,甚至可以没有联系。所以以上两个二次式的结果取值是任意的。通过线性变换,可以把上述方程里参考系 $A'$ 中的二次式变为关于 $(t,x,y,z)$ 的二次式。然而由于初始事件在原点重合,这个线性变换只能是
\begin{equation}
t^{2}-(x^{2}+y^{2}+z^{2})=C\left[(t^{\prime 2}-(x^{\prime 2}+y^{\prime 2}+z^{\prime 2})\right]~,
\end{equation}
其中 $C$ 是常数因子。由于空间中不存在特殊方向,所以 $C$ 只与两惯性系的相对速度大小有关。反过来,我们也有
\begin{equation}
t'^{2}-(x'^{2}+y'^{2}+z'^{2})=C\left[(t^{\prime 2}-(x^{\prime 2}+y^{\prime 2}+z^{\prime 2})\right]~.
\end{equation}
解得 $C^2=1$,由变换的连续性我们取 $C=1$。
称\autoref{eq_qed1_6} 的二次式为\textbf{时空间隔},则可以定义\textbf{洛伦兹变换是不改变时空间隔的线性变换},即
\begin{equation}\label{eq_qed1_7}
s^2=t'^{2}-(x'^{2}+y'^{2}+z'^{2})=t^{2}-(x^{2}+y^{2}+z^{2})~.
\end{equation}

\subsection{洛伦兹群}
为了方便地表示时空间隔\autoref{eq_qed1_7} ,常引入闵可夫斯基空间的度规张量$g_{\mu\nu}$和时空坐标四矢量标记$x^{\mu}$,分别表示为
\begin{equation}
\begin{aligned}
g_{\mu \nu}&=(\pmat{1&0&0&0},\pmat{0&1&0&0},\pmat{0&0&1&0},\pmat{0&0&0&1})\\
x^{\mu} &=\left(x^{0}, x^{i}\right)^T \quad(i=1,2,3)\\
&=(t, \bvec{x})^T~,
\end{aligned}
\end{equation}
于是时空间隔可以表示为
\begin{equation}\label{eq_qed1_1}\begin{aligned}
s^{2} &=x^{0} x^{0}-x^{i} x^{i}=x^{\mu} x^{\nu} g_{\mu \nu} \\
&=x^{\prime 0} x^{\prime 0}-x^{\prime i} x^{\prime i}=x^{\prime \mu} x^{\prime \nu} g_{\mu \nu}~.\end{aligned}\end{equation}

通常,我们还把行矢量表示为$x_{\mu}$,于是你会发现$x_{\mu}=g_{\mu\nu}x^{\nu}$。当我们想用矩阵表示度规相关的一系列运算时,通常把闵氏度规表示为矩阵形式$g^{\mu}_{\nu}$,此时有
\begin{equation}
s^2=x^Tgx,\,g=\opn{diag(1,-1,-1,-1)}~.
\end{equation}



可以证明,洛伦兹变换满足群的四大定义:封闭性、结合性、单位元存在性与逆元存在性。于是我们把洛伦兹变换所在的集合称为\textbf{洛伦兹群(Lorentz Group)}。



设联系两个惯性系的洛伦兹矩阵为 $\Lambda$,矩阵元为 $\Lambda_{\nu}^{\mu}\quad(\mu,\nu=0,1,2,3)$。洛伦兹变换可以表示为
\begin{equation}\label{eq_qed1_2}x^{\prime \mu}=\Lambda_{\nu}^{\mu} x^{\nu}=\Lambda_{0}^{\mu} x^{0}+\Lambda_{i}^{\mu} x^{i}~,\end{equation}
代入\autoref{eq_qed1_1} 我们有
\begin{equation}
x^{\mu} x^{\nu} g_{\mu \nu}=x'^{\rho} x'^{\sigma} g_{\rho \sigma}~.
\end{equation}
代入\autoref{eq_qed1_2} ,保持时空距离不变要求
\begin{equation}\label{eq_qed1_3}
\begin{aligned}
g_{\rho \sigma}&=g_{\mu \nu} \Lambda_{\rho}^{\mu} \Lambda_{\sigma}^{\nu}\\
g&=\Lambda^Tg\Lambda
\end{aligned}~,
\end{equation}
所以洛伦兹变换实际上是\textbf{保闵氏度规不变}的合同变换。因此可以把洛伦兹变换看为$\mathbb R^{1,3}$空间的广义正交变换,写为$O(1,3)$,其中“3”表示闵氏度规里的3个负号,$O$表示\textbf{正交(orthogonal)}。

让我们具体写出一个洛伦兹矩阵如
\begin{equation}
\Lambda^{\mu}_{\nu}=\pmat{\Lambda^0_0&\Lambda^0_1&\Lambda^0_2&\Lambda^0_3\\\Lambda^1_0&\Lambda^1_1&\Lambda^1_2&\Lambda^1_3\\\Lambda^2_0&\Lambda^2_1&\Lambda^2_2&\Lambda^2_3\\\Lambda^3_0&\Lambda^3_1&\Lambda^3_2&\Lambda^3_3}~,
\end{equation}
二次型的要求使得
\begin{equation}
\begin{aligned}
(\Lambda^0_0)^2-(\Lambda^1_0)^2-(\Lambda^2_0)^2-(\Lambda^3_0)^2&=-1\\
(\Lambda^0_0)^2-(\Lambda^0_1)^2-(\Lambda^0_2)^2-(\Lambda^0_3)^2&=-1~,
\end{aligned}
\end{equation}
其他列(行)同理,只不过等式右边改为$1$。

经验事实告诉我们,一个物理事件的因果性不随坐标系的改变而改变。比如$A$点发射信号,$B$点接收信号,换一个坐标系描述也是一个信号从$A$至$B$的匀速方程。用光锥来表示便是,一个\textbf{有物理意义的}洛伦兹变换必然不改变类时矢量的\textbf{时间分量符号},使得上光锥内的矢量始终在上光锥内,此为绝对未来;下光锥的类时矢量同理,称之为绝对过去。可以证明,这样有物理意义的洛伦兹变换要求$\Lambda^0_0>0$。
\begin{theorem}{}
洛伦兹变换保持类时矢量的时间分量符号不变当且仅当$\Lambda^0_0>0$。
\end{theorem}
\textbf{证明:}

设$v^i$为任意类时矢量,$\Lambda^i_j$为任意洛伦兹变换且$v'=\Lambda^i_jv^j$。则
\begin{equation}
v'
\end{equation}

\begin{corollary}{}
设满足$\Lambda^0_0\ge1$的洛伦兹分支为$O^+(1,3)$,则这一分支是$O(1,3)$的子群。
\end{corollary}

\addTODO{正交群引用}

\subsection{分类}
对\autoref{eq_qed1_3} 两边取行列式后立刻得 $\det \Lambda=\pm1$。“+1”表示坐标系定向在洛伦兹变换后依然不变。%需要略谈一下意义吧

取洛伦兹矩阵的00分量,根据\autoref{eq_qed1_3} 有
\begin{equation}
1=g_{\mu \nu} \Lambda_{\rho}^{0} \Lambda_{\sigma}^{0}= (\Lambda_{0}^{0})^2- (\Lambda_{i}^{0})^2~,\\
\end{equation}
所以 $\left|\Lambda_{0}^{0} \right|\geqslant1$。 根据 $\Lambda_{0}^{0}$ 的范围($\Lambda_{0}^{0}\geqslant1$ 是orthochronous的洛伦兹变换)和 $\Lambda$ ($\det\Lambda=1$ 是proper的变换)的行列式可以将洛伦兹变换分为四类
\begin{enumerate}
\item 正规、正时(proper orthochronous)以$\Lambda_{+}^{\uparrow}$或$SO^+(1,3)$表示,对应 $\det\Lambda=1\quad \Lambda_{0}^{0}\geqslant1$。
\item 正规、非正时(proper non-orthochronous)以$\Lambda_{+}^{\downarrow}$表示,对应 $\det\Lambda=1\quad \Lambda_{0}^{0}\leqslant-1~$。
\item 非正规、正时(improper orthochronous)以$\Lambda_{-}^{\uparrow}$表示,对应 $\det\Lambda=-1 \quad\Lambda_{0}^{0}\geqslant1~$。
\item 非正规、非正时(improper non-orthochronous)以$\Lambda_{+}^{\downarrow}$表示,对应 $\det\Lambda=-1 \quad\Lambda_{0}^{0}\leqslant-1$。
\end{enumerate}

在这四个互不相交的分类中,$SO^+(1,3)$最为特殊,首先,容易证明任意洛伦兹变换都可以写成该分支与时间反演变换 $ \Lambda_{T}$ 以及宇称变换 $\Lambda_{P}$ 的乘积。即
\begin{equation}\label{eq_qed1_4}
\mathcal{O}(3,1)=\left\{SO^+(1,3), \Lambda_{P} SO^+(1,3), \Lambda_{T} SO^+(1,3), \Lambda_{P} \Lambda_{T} SO^+(1,3)\right\}~.
\end{equation}


实际上,这不仅是“看起来”大大减轻了对洛伦兹变换的研究负担。在进一步分析$SO^+(1,3)$的特殊性之前,我们先来看一下这四个分支的例子。
\subsection{举例}
\begin{enumerate}
\item \textbf{转动}: 纯粹的空间转动下,$x^{\prime 0}=x^{0}, x^{\prime i}=a^{i j} x^{j}$,这时洛伦兹矩阵写为
\begin{equation}       %开始数学环境
\Lambda=\left(                 %左括号
  \begin{array}{cccc}   %该矩阵一共3列,每一列都居中放置
   1& 0 & 0 & 0\\  %第一行元素
   0& a^{11} &  a^{12} &  a^{13}\\  %第二行元素
   0& a^{21} &  a^{22} &  a^{23}\\  %第三行元素
   0& a^{31} &  a^{32} &  a^{33}\\  %第四行元素
  \end{array}
\right)  ~.               %右括号
\end{equation}
 $a^{ij}$ 构成的子矩阵即为欧式空间的旋转矩阵,保证了 $\det\Lambda=1$。因此$\Lambda\in SO^+(1,3)$。
\item \textbf{平动(boost)}: 由双曲余弦的性质知,所有的平动都是正时的,实际上也都是正规的。例如对于沿着x轴的平动,变换为
\begin{equation}\begin{array}{c}
x^{\prime 0}=x^{0} \cosh \eta-x^{1} \sinh \eta~, \\
x^{\prime 1}=-x^{0} \sinh \eta+x^{1} \cosh \eta ~,\\
x^{\prime 2}=x^{2}, \quad x^{\prime 3}=x^{3}~.
\end{array}\end{equation}
参数定义为
\begin{equation}
\begin{aligned}
 \frac{v}{c} &=\tanh\eta ~,\\
\left(1-(\frac{v}{c})^2 \right)^{-\frac{1}{2}} &=\cosh\eta~,\\
\left   (1-(\frac{v}{c})^2 \right)^{-\frac{1}{2}}\cdot\frac{v}{c}&=\sinh\eta~. 
\end{aligned}
\end{equation}
那么对应的洛伦兹矩阵为
\begin{equation}\Lambda=\left(\begin{array}{cccc}
\cosh \eta & -\sinh \eta & 0 & 0 \\
-\sinh \eta & \cosh \eta & 0 & 0 \\
0 & 0 & 1 & 0 \\
0 & 0 & 0 & 1
\end{array}\right)~.\end{equation}
并有
\begin{equation}\begin{aligned}
&\det\Lambda=\cosh ^{2} \eta-\sinh ^{2} \eta=1~,\\
&\Lambda_{0}^{0}=\cosh \eta \geqslant 1~.
\end{aligned}\end{equation}
\item \textbf{时间反演}
此时 $\Lambda=\opn{diag}(-1,1,1,1)$。易见这是非正规、非正时的。
\item \textbf{空间反演}
 $\Lambda=\opn{diag}(1,-1,-1,-1)$。易见这是非正规、正时的。
\end{enumerate}

\subsection{狭义洛伦兹群的分解}
\subsection{狭义洛伦兹群的李代数}
\addTODO{李群子群的条件待补充}%不知道微分几何相关是作为前置知识还是简单叙述。

\subsubsection{$SO^+(1,3)$的特殊性}
$O(1,3)$群有两个特殊子群,一是能保持坐标系定向的正规群,二则是正时群,正时群能保证类时矢量的时间分量不变,这实际上是保证了事件的因果性。一个有物理意义的洛伦兹变换必然是正时的。

实际上,洛伦兹群的四个分支里只有$SO^+(1,3)$是子群。我们来简单证明一下。首先,非正规的洛伦兹分支总是违背封闭性的,因为两个行列式为$-1$的相乘结果是行列式为$1$的矩阵。其次,由于非正时的洛伦兹变换会改变类时矢量的时间分量符号,所以被两个非正时的洛伦兹变换作用后,类时矢量的时间分量符号会恢复。所以非正时的洛伦兹矩阵相乘结果必为正时矩阵。相较之下,由于子群相交结果是子群,作为正规与正时群的交集,$SO^+(1,3)$必为$O(1,3)$的子群,实际上还是一个李子群。%补充是李群的证明
因此我们可以用指数映射表示该李子群的“部分”群元。设任意$X\in \mathfrak{so}^+(1,3)\equiv\opn{Lie} SO^+(1,3),t\in \mathbb R$,则$\E^{tX_e}$表示一条从单位元切空间$T_eSO^+(1,3)$出发的左不变切场的积分曲线。那么该曲线经过的点都能被指数映射所表示。由此观之,其他连通分支的元素均不能被指数映射所表示,毕竟单位元$I$是正规正时矩阵,在$SO^+(1,3)$内,而不同连通分支是彼此不相交的。或者从另一个角度上看,$\E^{tX_e}$的行列式永远是正的,所以$O(1,3)$的李代数就是$SO^+(1,3)$的李代数。
%最好补充指数映射是满射的证明。

有意思的是,虽然指数映射并非总是满射,但可以证明在$SO^+(1,3)$上是满的。
\textbf{所以该群群元都可以被指数映射所表示}。因此,研究$SO^+(1,3)$上的李代数是非常有实践意义的,借由\autoref{eq_qed1_4} ,我们便得到$O(1,3)$的所有群元。由于$SO^+(1,3)$群包含了通俗的洛伦兹变换,即旋转和推促,所以也有学者称其为\textbf{狭义洛伦兹群(restricted lorentz group)}的。为了叙述简洁,下面我们沿用这一称呼。

\subsubsection{狭义洛伦兹群的群元形式}
对于任意$\Lambda\in SO^+(1,3)$,总存在$t\in \mathbb R,X\in \mathfrak so^+(1,3)$,使得$\Lambda=\E^{tX}$,因此找到$\mathfrak so^+(1,3)$的基尤为重要。

考虑到$SL(2,\mathbb R)$是$SO^+(1,3)$的双覆盖,因此它们的李代数同构,所以$\mathfrak{so}(1,3)$是六维线性空间。

又因为$SO^+(1,3)$的旋转部分同构于$SO(3)$,所以旋转部分包含了三个线性无关的生成元。分别是\footnote{推导见\href{https://wuli.wiki/online/lielot.html}{洛伦兹群的李代数}}
\begin{equation}
J_{1}=\begin{pmatrix}0&0&0&0\\0&0&0&0\\0&0&0&-1\\0&0&1&0\end{pmatrix},\quad J_{2}=\begin{pmatrix}0&0&0&0\\0&0&0&1\\0&0&0&0\\0&-1&0&0\end{pmatrix},\quad J_{3}=\begin{pmatrix}0&0&0&0\\0&0&-1&0\\0&1&0&0\\0&0&0&0\end{pmatrix}~,
\end{equation}
对于任意$t,a^i\in\mathbb R$,$\E^{ta^iJ_i}$便是$SO(3)$直和上时间分量的“1”。

同理,我们也可从推促变换反向得到三个线性无关的生成元。分别为
\begin{equation}
K_1=\begin{pmatrix}0&1&0&0\\1&0&0&0\\0&0&0&0\\0&0&0&0\end{pmatrix},\quad K_2=\begin{pmatrix}0&0&1&0\\0&0&0&0\\1&0&0&0\\0&0&0&0\end{pmatrix},\quad K_3=\begin{pmatrix}0&0&0&1\\0&0&0&0\\0&0&0&0\\1&0&0&0\end{pmatrix}~.
\end{equation}
所以任一狭义洛伦兹群的群元都可以写为$\E^{t(a^iJ_i+b^iK_i)}$,其中$t,a^i,b^i\in\mathbb R$。





