% Slurm 笔记

* 创建 gpu interactive session
```bash
srun -J srun -N 1 -n 32 -t 24:00:00 --mem=120G --partition=ksu-gen-gpu.q --gres=gpu:1 --pty bash
```
* 使用 `module load` 才能加载中 cuda 的 nvcc 和 nvprof。

## ===== 使用 cpu =======
* `sbatch` 命令用于提交任务
* `--mem-per-cpu=512M` 指定内存
* `--time=hours:minutes:seconds` 指定时间
* `--cpus-per-task=1`
* `--ntasks=1`
* `--nodes=1` 机器的数量
* 一个 batch script 的例子如
```
#!/bin/sh
srun hello.x < ./param.inp
```
* 文件 IO 目录是相对于提交任务时的 `pwd` 的, stdout 会自动生成 `slurm-任务编号.out`
* `kstat` 会列出所有人的所有任务
* `kstat --me` 可以查看我正在运行的所有任务, 或者用 `kstat | grep addis1` 也可以
* `kstat | grep USER` 也可以查看某个其他用户的所有任务
* `kstat -c` 可以看到所有用户的 cpu 使用情况
* `scancel 任务编号` 可以取消某个任务
* home 文件夹有 1T 的空间
* 要运行, 例如
```
sbatch --time=24:0:0 --mem-per-cpu=300M --cpus-per-task=10 --ntasks=1 --nodes=1 ./job.sh
```
* 要取消, 用 `scancel JOB_ID`, 要取消个人的所有任务, 用 `scancel -u addis1`
