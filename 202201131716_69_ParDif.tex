% 偏导数(微分学)
% keys 多元微积分|导数|偏导数|混合偏导|数学分析

\pentry{导数(数学分析)\upref{Der2},偏导数(简明微积分)\upref{ParDer}}

导数的几何意义是一元函数在某一点处的斜率,而我们可以将这个概念推广到多元函数.$n$ 元实函数是指从 $\mathbb{R} ^n$ 到 $\mathbb{R}$ 的映射:
\begin{equation}
\begin{aligned}
f:\mathbb{R} ^n&\rightarrow \mathbb{R}\\
(x_1,x_2,\cdots,x_n)&\mapsto f(x_1,x_2,\cdots,x_n)
\end{aligned}
\end{equation}
我们定义 $f$ 对 $x_i$ 的\textbf{偏导数}为
\begin{equation}
\lim\limits_{x'_i\rightarrow x_{i}}\frac{f(x_1,\cdots,x'_i,\cdots,x_n)-f(x_1,\cdots,x_i,\cdots,x_n)}{x'_i-x_i}
\end{equation}
如果该极限存在,那么函数在 $\bvec x_0=(x_1,\cdots,x_n)$ 处对 $x_i$ 的偏导数存在,记为
\begin{equation}
\frac{\partial f(x_1,\cdots,x_n)}{\partial x_i}=\left.\frac{\partial f}{\partial x_i}\right|_{(x_1,\cdots,x_n)}=f'_{x_i}(x_1,\cdots,x_n)
\end{equation}

在讨论多元函数时,我们约定用粗体字(例如 $\bvec x$)来表示 $\mathbb{R}^n$ 中的一个向量.对于给定的 $\bvec x=(x_1,\cdots,x_n)$,如果构造一元函数 $g(x)=f(x_1,\cdots,x_{i-1},x,x_{i+1},\cdots,x_n)$,那么根据一元函数导数的定义,容易发现
\begin{equation}
\left.\frac{\partial f}{\partial x_i}\right|_{\bvec x}=\left.\frac{\dd g}{\dd x}\right|_{x_i}
\end{equation}