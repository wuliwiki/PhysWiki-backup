% 量子简谐振子(级数法)
% 量子力学|微分方程|能级|厄密多项式|Hermite|薛定谔方程

\begin{issues}
\issueDraft
\end{issues}

\pentry{简谐振子(升降算符)\upref{QSHOop}}

量子力学中, 简谐振子的定态薛定谔方程为
\begin{equation}
-\frac{\hbar^2}{2m} \dv[2]{\psi}{x} + \frac12 m \omega^2 x^2\psi  = E\psi
\end{equation}
能级为
\begin{equation}
E_n = \qty(\frac12 + n)\hbar \omega 
\end{equation}
归一化的束缚态波函数为
\begin{equation}\label{QSHOxn_eq1}
\psi_n (x) = \frac{1}{\sqrt{2^n n!}} \qty(\frac{\alpha^2}{\pi })^{1/4} H_n(u) \E^{-u^2/2}
\end{equation}
其中
\begin{equation}
\alpha \equiv \sqrt{\frac{m\omega}{\hbar }}, \qquad
u \equiv \alpha x
\end{equation}
$H_n(u)$ 是 Hermite 多项式\upref{HermiP}.

\begin{figure}[ht]
\centering
\includegraphics[width=14cm]{./figures/QSHOxn_1.pdf}
\caption{简谐振子的前几个束缚态波函数(\autoref{QSHOxn_eq1}, $\alpha = 1$)} \label{QSHOxn_fig1}
\end{figure}

前4个波函数分别为(注意函数的奇偶性与角标的奇偶性相同)
\begin{equation}\ali{
\psi_0(x) &= \qty(\frac{\alpha^2}{\pi})^{1/4} \E^{-u^2/2}, &
\psi_2(x) &= \qty(\frac{\alpha^2}{\pi })^{1/4} \frac{1}{\sqrt 2 } (2u^2 - 1)\E^{-u^2/2},\\
\psi_1(x) &= \qty(\frac{\alpha ^2}{\pi })^{1/4} \sqrt2u \E^{-u^2/2}, \quad &
\psi_3(x) &= \qty(\frac{\alpha ^2}{\pi })^{1/4} \frac{1}{\sqrt 3} u(2u^2 - 3)\E^{-u^2/2}.
}\end{equation}

Matlab 代码如(使用原子单位制\upref{AU})
\begin{lstlisting}[language=matlab, caption=psi\_SHO.m]
function psi = psi_SHO(m, w, n, x)
alpha = sqrt(m*w); u = alpha * x;
psi = 1/sqrt(2^n*factorial(n)) *...
    (alpha^2/pi)^0.25 * hermiteH(n, u) .* exp(-u.^2/2);
end
\end{lstlisting}

\subsection{推导}

\addTODO{可以直接用级数解也可以用升降算符.}
