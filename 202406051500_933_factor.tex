% 阶乘(高中)
% keys 整数|gamma 函数|pi 符号
% license Xiao
% type Tutor

\begin{issues}
\issueDraft
\end{issues}

\pentry{\enref{求积符号(累乘)}{ProdSy}}{nod_fe64}

在排列组合的学习中,基于乘法原理,经常会出现多个连续的整数相乘的情形。

\begin{example}{求:从10名学生中,选出2名学生左右并排站立的方法数}
由于两人左右并排站立,所以对确定的两个人A和B,“AB”和“BA”是两种站法。因此,分别对两个位置进行考虑,第一个位置可以任选一个人,第二个位置可以从剩下的学生里再选一个人。由乘法原理,共有$10\times9=90$种方法

\end{example}

对自然数 $n$\footnote{拓展知识:阶乘的定义本身只限于自然数。但随着研究深入,偶尔会出现使用分数等数值的阶乘的场景,为此,欧拉推导了\enref{Gamma 函数}{Gamma}$\Gamma(x)$实现了对阶乘的解析延拓,即:1.保证它的函数值与阶乘对应$\Gamma(n)=(n-1)!$;2.保证自变量取其他实数值(如:部分负数、分数、和无理数等)时也可以有结果;3.函数的性质满足特定条件。于是,数学中也经常会用“Gamma函数”来代替需要表达“阶乘”概念的场合。}, \textbf{阶乘(factorial)}定义为所有小于等于 $n$ 的正整数的乘积,即
\begin{equation}
n! := \prod_{i = 1}^n i =1 \cdot 2 \cdot 3 \dots (n - 2) (n - 1)n~.
\end{equation}
其中,$\prod_{i = 1}^n i$为\enref{求积符号}{ProdSy}。特殊地,约定
\begin{equation}
0! := 1~.
\end{equation}

\begin{example}{求5的阶乘}
小于等于5的正整数有:1,2,3,4,5,因此:
$$5! = 1\times 2\times 3\times 4\times 5 = 120.~$$
\end{example}

