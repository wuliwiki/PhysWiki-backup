% 泡利矩阵(综述)
% license CCBYSA3
% type Sum

本文根据 CC-BY-SA 协议转载翻译自维基百科\href{https://en.wikipedia.org/wiki/Pauli_matrices}{相关文章}。

\begin{figure}[ht]
\centering
\includegraphics[width=6cm]{./figures/943b6a85e9584185.png}
\caption{沃尔夫冈·泡利(1900–1958),约摄于 1924 年。泡利因提出泡利不相容原理而在 1945 年获颁诺贝尔物理学奖**,提名人是阿尔伯特·爱因斯坦。} \label{fig_PLJZ_1}
\end{figure}
在数学物理和数学中,泡利矩阵是一组三个$2\times2$ 的复矩阵,它们具有迹为零、厄米特、自反和酉的性质。这些矩阵通常用希腊字母$\sigma$(σ)表示,在涉及同位旋对称性时,有时也用$\tau$(τ)表示。
$$
\sigma_1 = \sigma_x = 
\begin{pmatrix}
0 & 1 \\
1 & 0
\end{pmatrix},
\quad
\sigma_2 = \sigma_y = 
\begin{pmatrix}
0 & -i \\
i & 0
\end{pmatrix},
\quad
\sigma_3 = \sigma_z = 
\begin{pmatrix}
1 & 0 \\
0 & -1
\end{pmatrix}.~
$$
这些矩阵因物理学家 沃尔夫冈·泡利而得名。在量子力学中,它们出现在泡利方程中,用于描述粒子的自旋与外部电磁场相互作用的情况。它们还可以用来表示两种偏振滤光片的相互作用状态,例如水平/垂直偏振、45 度偏振(左/右)以及圆偏振(左/右)的状态。

每个泡利矩阵都是厄米特矩阵。再加上单位矩阵 $I$(有时被视为第零个泡利矩阵$\sigma_0$),这些矩阵在实数加法下构成了所有 $2 \times 2$ 厄米特矩阵的向量空间的基。这意味着,任何 $2 \times 2$ 厄米特矩阵都可以唯一地写成泡利矩阵的实数线性组合。

泡利矩阵满足以下有用的乘积关系:
$$
\sigma_i \sigma_j = \delta_{ij} + i \, \epsilon_{ijk} \, \sigma_k,~
$$
其中 $\delta_{ij}$ 是克罗内克 δ,$\epsilon_{ijk}$ 是三阶 Levi-Civita 符号。由于厄米特算符在量子力学中表示可观测量,因此泡利矩阵张成了复二维希尔伯特空间中所有可观测量的空间。在泡利的研究语境中,$\sigma_k$ 表示三维欧几里得空间 $\mathbb{R}^3$ 中沿第 $k$ 个坐标轴方向的自旋对应的可观测量。

此外,当泡利矩阵乘以 $i$ 使之成为反厄米特矩阵后,它们还可以生成李代数意义下的变换:矩阵 $i\sigma_1, i\sigma_2, i\sigma_3$ 构成了实李代数 $\mathfrak{su}(2)$ 的基,并通过指数映射生成特殊酉群 $SU(2)$。由三个位移矩阵 $\sigma_1, \sigma_2, \sigma_3$ 所生成的代数同构于 $\mathbb{R}^3$ 的克利福德代数;而由 $i\sigma_1, i\sigma_2, i\sigma_3$ 生成的带幺结合代数则与四元数代数$\mathbb{H}$ 完全同构。
\subsection{代数性质}
所有三个位相矩阵都可以压缩成一个表达式:
$$
\sigma_j =
\begin{pmatrix}
\delta_{j3} & \delta_{j1} - i\delta_{j2} \\
\delta_{j1} + i\delta_{j2} & -\delta_{j3}
\end{pmatrix},~
$$
其中,$\delta_{jk}$ 是克罗内克δ函数,当 $j = k$ 时取值为 $+1$,否则为 $0$。这个表达式的好处在于,可以通过代入 $j = 1, 2, 3$ 来“选择”三个位相矩阵中的任意一个,这在需要进行代数推导但不针对某个具体矩阵时非常实用。

这些矩阵具有自反性:
$$
\sigma_1^2 = \sigma_2^2 = \sigma_3^2 = -i\,\sigma_1\sigma_2\sigma_3 =
\begin{pmatrix}
1 & 0 \\
0 & 1
\end{pmatrix}
= I,~
$$
其中 $I$ 是单位矩阵。

位相矩阵的行列式和迹为:
$$
\det(\sigma_j) = -1, \quad
\operatorname{tr}(\sigma_j) = 0,~
$$
由此可知,每个矩阵 $\sigma_j$ 的特征值为 $+1$ 和 $-1$。

当将单位矩阵 $I$(有时记作 $\sigma_0$)包括在内时,位相矩阵与 $I$ 一起构成:实数域 $\mathbb{R}$ 上 $2 \times 2$ 厄米矩阵空间 $\mathcal{H}_2$的一个 Hilbert–Schmidt 正交基;复数域 $\mathbb{C}$ 上所有 $2 \times 2$ 矩阵空间 $\mathcal{M}_{2,2}(\mathbb{C})$的一个正交基。
\subsubsection{对易与反对易关系}
\textbf{对易关系}

保利矩阵满足如下对易关系:
$$
[\sigma_j, \sigma_k] = 2i \, \varepsilon_{jkl} \, \sigma_l,~
$$
其中 $\varepsilon_{jkl}$ 是Levi-Civita 符号。


这些对易关系表明,保利矩阵是以下李代数表示的生成元:
$$
(\mathbb{R}^3, \times) \;\cong\; \mathfrak{su}(2) \;\cong\; \mathfrak{so}(3),~
$$
即它们与三维实数空间 $\mathbb{R}^3$ 的叉积代数、$\mathfrak{su}(2)$ 李代数和 $\mathfrak{so}(3)$ 李代数互相同构。

\textbf{反对易关系}

保利矩阵还满足以下反对易关系:
$$
\{\sigma_j, \sigma_k\} = 2 \delta_{jk} I,~
$$
其中,$\{\sigma_j, \sigma_k\}$ 定义为:$\sigma_j \sigma_k + \sigma_k \sigma_j$,$\delta_{jk}$ 是 Kronecker δ 符号,$I$ 表示$2 \times 2$ 单位矩阵。


这些反对易关系说明,保利矩阵是三维实空间$\mathbb{R}^3$Clifford 代数$\mathrm{Cl}_3(\mathbb{R})$ 的一个表示的生成元。

利用 Clifford 代数的常规构造方式,可以定义 $\mathfrak{so}(3)$ 李代数的生成元:$\sigma_{jk} = \tfrac{1}{4}[\sigma_j, \sigma_k]$,从而重新得到上文的对易关系(仅相差一个数值因子)。


