% 中国科学院 2012 年考研数学(甲)
% keys 中科院|2023|数学
% license Xiao
% type Tutor

\begin{issues}
\issueDraft
\end{issues}

\subsection{选择题}(本题满分50分,每小题5分。请从题目所列的选项中选择一个正确项填充空格。每题的四个备选项中只有一个是正确的,不选、错选或多选均不得分。请将你的选择标清题号写在考场发的答题纸上,直接填写在试题上无效。)

1.函数$f(x)$的导数$f'(x)$在$(-\infty,\infty)$上是连续函数,$a>0$,则函数$F(x)=\leftgroup{
&a,&&f(x)\geqslant a\\ 
&f(x), &&-a<f(x)<a \\
&-a,&&f(x) \leqslant -a }$

一定是()\\
(A)有界可微函数$\quad$(B)有界连续函数 \\
(C)连续可微函数$\quad$ (D)以上结论都不正确

2.$\displaystyle \lim_{\substack{x \to 0}}(\frac{1}{n^2+2n+1}+\frac{2}{n^2+2n+2}+\dots+\frac{n}{n^2+2n+n})=$()\\
(A)1 $\qquad$ (A) $ \infty \qquad$  (A)$\frac{1}{2}$ $\qquad$   (D)0

3.函数$f(x)=(x+2cosx)^2$在区间$[0,\frac{\pi}{2}]$上的最大值是()\\
(A)$\frac{\pi ^2}{36}+\frac{\sqrt{3}\pi}{3}+1 \qquad$     (B)$\frac{\pi ^2}{36}+\frac{\sqrt{3}\pi}{3}+2 \qquad$  \\
(C) $\frac{\pi ^2}{36}+\frac{\sqrt{3}\pi}{3}+3\qquad$       (D)$\frac{\pi^2}{4} \qquad$  

4.设$f(x)=x(x+1)\dots(x+20)$,下面四个结论正确的是()\\
(A)$f'(-1)>0,f'(-2)>0\quad$ (B)$f'(-1)>0,f'(-2)<0\quad$ \\
(C)$f'(-1)<0,f'(-2)<0\quad$ (D)$f'(-1)<0,f'(-2)>0$

5.已知$\displaystyle g(x)\int_{0}^{2} f(x)\dd{x}=10$,则$\displaystyle \int_{0}^{2} f(x)\dd{x} \int_{0}^{2} g(x)\dd{x}=$()\\
(A)20 $\quad$ (B)10 $\quad$ (C)5 $\quad$ (D)不能确定

6.$ \displaystyle \lim_{\substack{x \to 0,y \to 0}} \frac{xy}{\sqrt[3]{x^4+y^{12}}}$=()\\
(A)0 $\quad$ (B)$\frac{1}{\sqrt{2}} \quad$ (C)$ \frac{1}{\sqrt[3]{2}}\quad$ (D)不存在  

7.za



(A)$\quad$ (B)$\quad$ (C)$\quad$ (D)




