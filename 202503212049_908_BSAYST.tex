% 玻色–爱因斯坦统计(综述)
% license CCBYSA3
% type Wiki

本文根据 CC-BY-SA 协议转载翻译自维基百科\href{https://en.wikipedia.org/wiki/Bose\%E2\%80\%93Einstein_statistics}{相关文章}。

在量子统计学中,玻色–爱因斯坦统计(B–E统计)描述了在热力学平衡下,一组非相互作用的相同粒子占据一组可用离散能级的两种可能方式之一。粒子聚集在同一状态中的现象是遵循\textbf{玻色–爱因斯坦统计}的粒子的特征,它解释了激光光束的凝聚流动和超流氦的无摩擦爬升。这一行为的理论由萨廷德拉·纳特·玻色于1924-25年提出,他认识到一组相同且不可区分的粒子可以以这种方式分布。这个想法后来被阿尔伯特·爱因斯坦与玻色合作进行了采纳和扩展。

玻色–爱因斯坦统计仅适用于不遵循泡利不相容原理限制的粒子。遵循玻色–爱因斯坦统计的粒子称为玻色子,它们具有整数自旋。与此相对,遵循费米–狄拉克统计的粒子称为费米子,具有半整数自旋。
\subsection{玻色–爱因斯坦分布}  

在低温下,玻色子与费米子(遵循费米–狄拉克统计)表现不同,玻色子可以“凝聚”到相同的能级中,数量没有限制。这个看似不寻常的性质也导致了物质的特殊状态——玻色–爱因斯坦凝聚态。费米–狄拉克统计和玻色–爱因斯坦统计在量子效应重要且粒子“不可区分”时适用。如果粒子浓度满足
\[
\frac{N}{V} \geq n_{\text{q}},~
\]
其中\( N \)是粒子数,\( V \)是体积,\( n_q \)是量子浓度,对于量子浓度,粒子间的距离等于热德布罗意波长,因此粒子的波函数几乎不重叠。

费米–狄拉克统计适用于费米子(遵循泡利不相容原理的粒子),而玻色–爱因斯坦统计适用于玻色子。由于量子浓度依赖于温度,大多数高温下的系统遵循经典(麦克斯韦–玻尔兹曼)极限,除非它们还具有非常高的密度,比如白矮星。无论是费米–狄拉克统计还是玻色–爱因斯坦统计,在高温或低浓度下都会变为麦克斯韦–玻尔兹曼统计。

玻色–爱因斯坦统计由玻色于1924年引入用于光子,并在1924–25年由爱因斯坦推广到原子。

玻色–爱因斯坦统计中,能级\( i \)中粒子的期望数为:
\[
\bar{n}_i = \frac{g_i}{e^{(\varepsilon_i - \mu) / k_B T} - 1}~
\]
其中\( \varepsilon_i > \mu \),\( n_i \) 是状态\( i \)中的占据数(粒子数),\( g_i \) 是能级\( i \)的简并度,\( \varepsilon_i \) 是第 \( i \)个状态的能量,\( \mu \)是化学势(对于光子气体为零),\( k_B \) 是玻尔兹曼常数,\( T \)是绝对温度。

这个分布的方差 \( V(n) \) 直接从上面关于平均数的表达式计算得出。\(^\text{[1]}\)
\[
V(n) = kT \frac{\partial}{\partial \mu} \bar{n}_i = \langle n \rangle (1 + \langle n \rangle) = \bar{n} + \bar{n}^2~
\]
为了比较,费米–狄拉克粒子-能量分布给出的能量为 \( \varepsilon_i \) 的费米子平均数具有类似的形式:
\[
\bar{n}_i(\varepsilon_i) = \frac{g_i}{e^{(\varepsilon_i - \mu) / k_B T} + 1}.~
\]
如上所述,在高温和低粒子密度的极限下,玻色–爱因斯坦分布和费米–狄拉克分布都趋向于麦克斯韦–玻尔兹曼分布,而无需任何特别的假设:

在低粒子密度的极限下,
\begin{itemize}
\item 
\[
\bar{n}_i = \frac{g_i}{e^{(\varepsilon_i - \mu) / k_B T} \pm 1} \ll 1,~
\]
因此,\(e^{(\varepsilon_i - \mu) / k_B T} \pm 1 \gg 1\)或者等效地,\(e^{(\varepsilon_i - \mu) / k_B T} \gg 1\).在这种情况下,
\[
\bar{n}_i \approx \frac{g_i}{e^{(\varepsilon_i - \mu) / k_B T}} = \frac{1}{Z} e^{-(\varepsilon_i - \mu) / k_B T},~
\]
这就是麦克斯韦–玻尔兹曼统计的结果。
\item 在高温的极限下,粒子分布在一个广泛的能量范围内,因此每个状态的占据数(特别是那些高能状态,其中\( \varepsilon_i - \mu \gg k_B T \)再次非常小,
\[
\bar{n}_i = \frac{g_i}{e^{(\varepsilon_i - \mu) / k_B T} \pm 1} \ll 1.~
\]
这再次简化为麦克斯韦–玻尔兹曼统计。
\end{itemize}
除了在高温\( T \)和低密度的极限下趋向于麦克斯韦–玻尔兹曼分布外,玻色–爱因斯坦统计还会在低能量状态下(当\(\varepsilon_i - \mu \ll k_B T\)时)简化为瑞利–金斯定律分布,即:
\[
\bar{n}_i = \frac{g_i}{e^{(\varepsilon_i - \mu) / k_B T} - 1} \approx \frac{g_i}{(\varepsilon_i - \mu) / k_B T} = \frac{g_i k_B T}{\varepsilon_i - \mu}.~
\]
\subsection{历史} 
瓦迪斯瓦夫·纳坦松在1911年得出结论,普朗克定律要求“能量单位”是不可区分的,尽管他并没有将这一点表述为爱因斯坦的光量子理论。\(^\text{[2][3]}\)

在达卡大学(当时是英国印度的一部分,现在是孟加拉国)讲授辐射理论和紫外灾难时,萨廷德拉·纳特·玻色打算向他的学生展示,当时的理论是如何不足的,因为它预测的结果与实验结果不符。在这次讲座中,玻色在应用理论时犯了一个错误,这个错误出乎意料地给出了与实验一致的预测。这个错误是一个简单的失误——类似于认为抛两枚公平的硬币时会有三分之一的概率出现两个正面——对任何具有基本统计学知识的人来说,这个错误显然是错误的(值得注意的是,这个错误与d'Alembert在其《十字或正反面》一文中著名的失误类似\(^\text{[4][5]}\))。然而,它预测的结果与实验一致,玻色意识到这可能并非错误。第一次,他认为麦克斯韦–玻尔兹曼分布并不适用于所有尺度上的所有微观粒子。因此,他研究了在相空间中找到粒子处于各种状态的概率,其中每个状态是一个小的区域,具有相空间体积为\( h^3 \),粒子的位置和动量并没有特别分开,而是被视为一个变量。