% 上确界与下确界
% 确界|limit

\begin{issues}
\issueDraft
\end{issues}

\addTODO{应整合到“六大实数完备性公理”的后续词条中.}

\pentry{完备公理\upref{Cmplt}} 

有了实数的完备性,我们就可以得到一个很重要的概念,确界.为了介绍确界,我们首先要熟悉“界”的概念.

\begin{definition}{界}
设$S$是实数集合$\mathbb{R}$的非空子集.

如果存在实数$a$,使得对于任意的$x\in S$,都有$a\geq x$,那么称$a$是$S$的一个\textbf{上界};如果不存在这样的$a$,则称$S$的上界是$+\infty$.

如果存在实数$b$,使得使得对于任意的$x\in S$,都有$b\leq x$,那么称$b$是$S$的一个\textbf{下界};如果不存在这样的$b$,则称$S$的下界是$-\infty$.
\end{definition}

\begin{example}{}
对于实数集上的区间$[a, b]$,$b$是它的一个上界,$b+1$也是它的一个上界.

对于“全体正偶数”的集合,它的下界可以是$2$,也可以是$0$、$-110$、$-507$等,但它的上界只有$+\infty$.
\end{example}

实数子集的界通常不是唯一的.比上界大的实数都是上界,比下界小的实数也都是下界.只有当上界是$+\infty$或者下界是$-\infty$,该上界或下界才是唯一的.

界无法保证唯一性,因此很难用来刻画集合本身的性质.比如说,区间$[0, 1]$和区间$[0, 2]$是不同的集合,但实数$2$都是它们的上界,于是光描述某些上界是完全无法体现这两个集合的区别的.但是我们也容易想到,一个集合的所有上界中,有一个上界是可以唯一确定的,这就是我们接下来要定义的\textbf{上确界}.

\begin{definition}{确界}
设$S$是实数集合$\mathbb{R}$的非空子集.

如果存在实数$a$,使得对于任意的$x\in S$,都有$a\geq x$,且对于任意实数$y<a$,$y$都不是$S$的上界\footnote{即存在$x\in S$,使得$x>y$.},那么称$a$是$S$的一个\textbf{上确界(supremum)},记为$\opn{sup} S=a$.

如果存在实数$b$,使得对于任意的$x\in S$,都有$b\leq x$,且对于任意实数$y>a$,$y$都不是$S$的下界,那么称$b$是$S$的一个\textbf{下确界(infimum)},记为$\opn{inf} S=b$.



\end{definition}


对于很多集合来说,上确界就是其中最大的元素,下确界就是其中最小的元素,比如区间$[a, b]$的上下确界就分别是$b$和$a$.那么我们为什么不用集合的最大最小值来讨论,而是非要定义个确界呢?这是因为不是所有集合都有最大最小元素的,比如区间$(a, b)$,它的上下确界依然是$b$和$a$,但它却没有最大最小值.

上面这段分析暗含了一个问题:如果不是所有集合都有最大最小值,那能否保证确界的存在呢?比如说,$S$的下确界是$S$的界所构成的集合中的最大值,那我们能不能保证这个界的集合一定有最大值呢?

答案是肯定的,我们称之为\textbf{确界原理}.

\begin{definition}{确界原理}
如果非空的实数子集$S$有上界,那么$S$必存在上确界.

由此可推论,如果$S$有下界,那么$S$必存在下确界.
\end{definition}

注意,我们将确界原理写成了一条定义,因为它是刻画实数完备性的六大公理之一.这六条公理是完全等价的,任何一条都能推出其它五条,任选其一作为公理来定义实数的完备性即可.其它的完备性公理请参见\addTODO{整理出所有完备性公理词条后引用.}.

























