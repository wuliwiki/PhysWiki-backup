% 线性泛函的延拓
% keys 延拓|泛函|Hahn-Banach定理
% license Usr
% type Tutor

\pentry{齐次凸泛函\nref{nod_ConFul}}{nod_4b3f}
在分析学中函数的延拓是一类重要的问题,而函数空间可以用线性空间来描述,因此函数的延拓是对线性泛函的延拓。本文将讲述在线性泛函延拓的一般概念,及在线性泛函延拓的整个问题中起着重要作用的定理——Hahn-Banach定理。

\begin{definition}{延拓}
设 $L$ 是实线性空间,而 $L_0$ 是它的某一子空间,$f_0$ 是在 $L_0$ 上定义的线性泛函。若 $f$ 是在 $L$ 上定义的线性泛函,且对 $\forall x\in L_0$,成立 $f(x)=f(x_0)$,则称 $f$ 是 $f_0$ (在全空间 $L$ 上)的\textbf{延拓}。
\end{definition}


\begin{theorem}{Hahn-Banach}
设 $p$ 是线性空间 $L$ 上的\enref{齐次凸泛函}{ConFul},$L_0$ 是 $L$ 中的线性子空间。如果 $f_0$ 是 $L_0$ 上的线性泛函,且从属于 $p$ 在 $L_0$ 上的限制,即在 $L_0$ 上
\begin{equation}\label{eq_ExLina_1}
f_0(x)\leq p(x),~
\end{equation}
则 $f_0$ 在 $L$ 上的延拓 $f$ 存在,且 $f$ 从属于 $p$。
\end{theorem}

\textbf{证明:}\textbf{首先证明},若 $L_0\neq L$,则 $f_0$ 可以从 $L_0$ 延拓到保持\autoref{eq_ExLina_1} 的某一更广的子空间上:事实上,设 $z\in L$ 且 $z\notin L_0$,$L'$ 是 $L_0$ 和 $z$ 生成的子空间(即包含 $L_0$ 和 $z$ 的最小子空间)。于是 $L'$ 中的每一元素都具有 $tz+x$ 的形式,其中 $x\in L_0$(试证明)。

若 $f'$ 是 $L'$ 上 $f_0$ 的延拓,则
\begin{equation}
f'(tz+x)=tf'(z)+f_0(x),~
\end{equation}
现在选取 $f'(z)$ 使得在 $L'$ 上\autoref{eq_ExLina_1} 成立,即 $tf'(z)+f_0(x)\leq p(x+tz)$ 对一切 $x\in L_0$ 及一切实数 $t$ 恒成立。这等价于
\begin{equation}\label{eq_ExLina_2}
f_0\qty(\frac{x}{t})+f'(z)\left\{\begin{aligned}
&\leq p\qty(\frac{x}{t}+z),\quad t\geq 0;\\
&\geq -p\qty(-\frac{x}{t}-z),\quad t\leq 0.
\end{aligned}\right.
~
\end{equation}
因此我们只要验证恒有满足\autoref{eq_ExLina_2} 的 $f'(z)$ 存在即可。设 $y,y'\in L_0$,则由\autoref{eq_ExLina_1} ,成立
\begin{equation}
-f_0(y')+p(y'+z)\geq -f_0(y)-p\qty(-y-z).~
\end{equation}
 令 
 \begin{equation}
 \begin{aligned}
 c'=\inf_{y'}\qty{f_0(y')+p(y'+z)},\\
 c=\sup_y\qty{-f_0(y)-p\qty(-y-z)}.
 \end{aligned}~
 \end{equation}
 



\textbf{证毕!}



