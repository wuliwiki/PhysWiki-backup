% 狄拉克之海
% license CCBYSA3
% type Wiki

(本文根据 CC-BY-SA 协议转载自原搜狗科学百科对英文维基百科的翻译)

\begin{figure}[ht]
\centering
\includegraphics[width=8cm]{./figures/21cd607e8e795fc4.png}
\caption{狄拉克之海为一个巨大的粒子。 • 粒子, • 反粒子。} \label{fig_DLKZH_1}
\end{figure}

狄拉克之海是真空的理论模型,它是一个由负能量粒子组成的无限海洋。它首先是由英国物理学家保罗狄拉克在1930年[1] 为了解释狄拉克方程的相对论电子解中出现的反常的负能量态而提出的真空理论假设。[2] 正电子,电子的反物质粒子,在1932年实验发现之前,就被认为是狄拉克海中的一个洞。

在求解3动量平面波动解的自由狄拉克方程时,
$$i \hbar \frac{\partial \Psi}{\partial t} = (c \alpha \cdot \hat{p} + mc^2 \beta) \Psi~,$$
人们发现
$$\Psi_{p \lambda} = N \left( \begin{array}{c}    U \\    \frac{(c \hat{\sigma} \cdot \mathbf{p})}{mc^2 + \lambda E_p} U \end{array} \right)\frac{\exp\left[ i ( \mathbf{p} \cdot \mathbf{x} - \epsilon t ) / \hbar \right]}{\sqrt{2 \pi \hbar}^3}~,$$
其中
$$\epsilon = \pm E_p, \quad E_p = +c \sqrt{\mathbf{p}^2 + m^2 c^2}, \quad \lambda = \text{sgn} \, \epsilon~$$
这是相对论能量动量关系的直接结果
$$E^2 = p^2 c^2 + m^2 c^4~$$

狄拉克方程就是建立在这个基础上的。量U是常数$2\times1$列向量,N是归一化常数。量$\epsilon$被称为时间演化因子,它在薛定谔方程的平面波解中的类似作用解释为波(粒子)的能量。这种解释在这里不能立即得到,因为它可能获得负值。克莱因-戈登方程也存在类似的情况。在这种情况下,$\epsilon$的绝对值在正则的形式中可以解释为波的能量,负ε的波实际上具有正能量Ep。[4] 但是狄拉克方程却不是这样。与负ε相关的正则形式中的能量是-Ep。[5]

在空穴理论中,具有负时间演化因子的解被重新解释为正电子的表示,这由卡尔·安德森发现。对这个结果的解释需要狄拉克之海(Dirac sea),表明狄拉克方程不仅仅是狭义相对论和量子力学的结合,还意味着粒子的数量不能守恒。[6]

\subsection{起源}
狄拉克之海源于狄拉克方程的能谱,狄拉克方程是薛定谔方程在狭义相对论下的延伸,狄拉克在1928年提出了这个方程。虽然这个方程在描述电子动力学方面非常成功,但它有一个相当奇特的特征:对于每个具有正能量E的量子态,都有一个对应的能量-E态。当考虑一个孤立的电子时,这并不是一个很大的困难,因为它的能量是守恒的,负能量电子可能会被遗漏。然而,当考虑电磁场的影响时,困难就出现了,因为正能量电子能够通过连续发射光子来释放能量,随着电子下降到越来越低的能量状态,这一过程可以无限制地继续下去。真正的电子显然不是这样。

狄拉克对此的解决方案是求助于泡利不相容原理。电子是费米子,遵守不相容原理,这意味着在一个原子中没有两个电子可以共享一个能量状态。狄拉克假设我们认为的“真空”实际上是所有负能量态都被填满的状态,而没有正能量态。因此,如果我们想引入一个电子,我们必须把它置于正能量状态,因为所有的负能量状态都被占据了。此外,即使电子因发射光子而失去能量,也将被不允许降到零能量以下。

狄拉克还指出,可能存在这样一种情况,即除了一个负能量态之外,所有的负能量态都被占据了。负能量电子海洋中的这个“空穴”会对电场做出反应,就好像它是一个带正电荷的粒子。最初,狄拉克认为这个洞是质子。然而,罗伯特·奥本海默(Robert Oppenheimer)指出,一个电子和它的空穴能够互相湮没,在高能光子的形式中以电子的静止能量的顺序释放能量;如果空穴是质子,稳定的原子就不存在。[7] 赫尔曼·韦勒还指出,一个空穴应该像电子一样具有相同的质量,而质子大约比它们重两千倍。终于,卡尔安德森于1932年发现了正电子,这个问题得到了解决,同时预测的狄拉克空穴所有物理性质得到了证实。

\subsection{狄拉克之海的丑陋}
尽管取得了成功,迪拉克之海的概念并没有给人以优雅的印象。海洋的存在意味着充满整个空间的无限负电荷。为了搞清楚这一点,我们必须假设“真空”必须具有无限的正电荷密度,而这恰好被狄拉克海抵消了。由于绝对能量密度是不可观测的——撇开宇宙常数不谈——真空的无限能量密度并不是一个问题。只要能观察到能量密度的变化。杰弗里·兰蒂斯(一部艰难的科幻短篇小说《狄拉克海的涟漪》的作者)也指出,泡利不相容原理并不一定意味着充满负电子的狄拉克之海不能接受更多的电子,因为正如希尔伯特所阐明的,即使充满了负电子,无限量的海洋也能接受新粒子。当我们有一个手性异常和一个标准瞬子时,就会发生这种情况。

量子场论(QFT)在20世纪30年代的发展使得重新表述狄拉克方程成为可能,QFT将正电子视为“真实的”粒子而不是虚幻的粒子,并使真空成为没有粒子而不是无限粒子海的状态。这种理论更有说服力,尤其是因为它再现了狄拉克之海的所有合理预测,如电子正电子互相湮没。另一方面,场公式并没有消除狄拉克之海带来的所有困难;特别是具有无限能量的真空问题。

\subsection{现代诠释}
狄拉克海解释和现代QFT解释是通过一个非常简单的博戈柳波夫变换联系在一起的,博戈柳波夫变换是两种不同自由场理论的产生和湮没算符之间的一种识别。在现代解释中,狄拉克自旋的场算符是产生算符和湮没算符的总和,用示意图表示:

\\psi(x) = \\sum a^\\dagger(k)e^{ikx} + a(k)e^{-ikx}
负频率的算符降低任何状态的能量,且降低量与频率成正比,而正频率的算符提高任何状态的能量。
