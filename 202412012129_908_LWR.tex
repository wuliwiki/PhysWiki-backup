% 刘维尔定理(综述)
% license CCBYSA3
% type Wiki

本文根据 CC-BY-SA 协议转载翻译自维基百科\href{https://en.wikipedia.org/wiki/Liouville\%27s_theorem_(Hamiltonian)}{相关文章}。

在物理学中,李维尔定理(Liouville's theorem)以法国数学家约瑟夫·李维尔(Joseph Liouville)命名,是经典统计力学和哈密顿力学中的一个关键定理。该定理断言,系统的相空间分布函数在系统的轨迹上是恒定的——即在相空间中,围绕给定系统点的系统点的密度随着时间变化保持不变。这种时间不变的密度在统计力学中称为经典的先验概率。[1]

李维尔定理适用于守恒系统,即在没有摩擦效应或摩擦效应可以忽略的系统。对于这种系统的普遍数学表述是保持度量不变的动力系统。当系统具有可以解释为位置和动量的自由度时,李维尔定理适用;并非所有保持度量不变的动力系统都有这些自由度,但哈密顿系统有。在共轭位置和动量坐标的数学框架中,李维尔定理在辛几何的数学设置下得到了描述。李维尔定理忽略了化学反应的可能性,在化学反应中,粒子的总数可能随时间变化,或能量可能转移到内部自由度。李维尔定理有扩展形式,可以涵盖这些广义的情形,包括随机系统。[2]