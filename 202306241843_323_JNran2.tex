% 贝叶斯公式
% keys 概率论

\pentry{条件概率 \upref{HsCpMi},从集合论角度看随机事件\upref{JNran1}}
\subsection{前言}
贝叶斯公式可常常用于概率模型中的统计推断。在开始讨论贝叶斯公式前,我们需要定义几个概念。
\begin{definition}{划分}
设S为一个样本空间,我们定义对一个样本空间的划分为一系列的事件$A_1,A_2,A_3...,A_n$,这些事件对于$\forall i,j $ 且$i\neq j$都满足$A_i \cap A_j = \emptyset$,且对于他们的并集满足$A_1\cup A_2\cup ...\cup A_n = S$。
\end{definition}
划分是对样本空间进行的两两互不相交的切割,当我们有了划分后,对于任意一个事件$B$发生的概率,就可以看做在每个划分中发生的概率