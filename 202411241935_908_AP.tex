% 安德烈-马里·安培(综述)
% license CCBYSA3
% type Wiki

本文根据 CC-BY-SA 协议转载翻译自维基百科\href{https://en.wikipedia.org/wiki/Andr\%C3\%A9-Marie_Amp\%C3\%A8re}{相关文章}。

安德烈-玛丽·安培(André-Marie Ampère ForMemRS,英国发音:/ˈɒ̃pɛər, ˈæmpɛər/,美国发音:/ˈæmpɪər/,[1] 法语:[ɑ̃dʁe maʁi ɑ̃pɛʁ];1775年1月20日-1836年6月10日)[2] 是法国物理学家和数学家,是经典电磁学学科的奠基人之一,他将其称为‘电动力学’。他还发明了众多应用,例如由他命名的螺线管和电报机。作为一位自学成才的科学家,安培是法国科学院的院士,并在巴黎综合理工学院和法兰西学院担任教授。

国际单位制中的电流单位安培(A)以他的名字命名。他的名字还被刻在埃菲尔铁塔上的72个名字之一。‘运动学’(kinematic)一词是他创造的法语‘cinématique’的英语版本,[3] 它来源于希腊语 κίνημα kinema(意为‘运动’),其本身衍生自 κινεῖν kinein(意为‘移动’)。[4][5]
\subsection{传记}
\subsubsection{早年生活} 
安德烈-玛丽·安培于1775年1月20日出生在里昂,他的父亲是成功的商人让-雅克·安培,母亲是让娜·安托瓦内特·德苏蒂耶尔-萨尔塞·安培。他出生于法国启蒙运动的鼎盛时期,童年和少年时期大部分时间都在靠近里昂的家族庄园——波莱米约-蒙多尔(Poleymieux-au-Mont-d'Or)度过。[6] 让-雅克·安培是一位成功的商人,同时也是让-雅克·卢梭哲学的仰慕者。卢梭在其著作《爱弥儿》中阐述的教育理论成为安培教育的基础。卢梭认为,男孩应该避免接受正式学校教育,而是从“自然中直接获得教育”。安培的父亲将这一理念付诸实践,允许儿子在家族藏书丰富的图书馆中自学。因此,像乔治-路易·勒克莱尔(乔治·路易·布封)的《自然史》(自1749年开始撰写)和丹尼斯·狄德罗与让·勒隆·达朗贝尔的《百科全书》(1751年至1772年间新增的卷册)等法国启蒙时期的杰作,成为了安培的“老师”。[需要引用]  

然而,年轻的安培很快重新开始了拉丁语课程,这使他能够深入研究莱昂哈德·欧拉和丹尼尔·伯努利的著作。[7]