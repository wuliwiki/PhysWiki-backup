% 普朗克黑体辐射定律(综述)
% license CCBYSA3
% type Wiki

本文根据 CC-BY-SA 协议转载翻译自维基百科\href{https://en.wikipedia.org/wiki/Michael_Faraday}{相关文章}。

\begin{figure}[ht]
\centering
\includegraphics[width=8cm]{./figures/70b29e3c83575e69.png}
\caption{普朗克定律准确描述了黑体辐射。这里展示的是不同温度下的一组曲线。经典(黑色)曲线在高频率(短波长)下与观测到的强度偏离。} \label{fig_HTFS_1}
\end{figure}
在物理学中,普朗克定律(也称为普朗克辐射定律)描述了在给定温度T下,黑体在热平衡状态下发射的电磁辐射的谱密度,当黑体与其环境之间没有物质或能量的净流动时。

在19世纪末,物理学家无法解释为什么已经准确测量的黑体辐射谱在高频率处与现有理论预测的谱有显著的偏离。1900年,德国物理学家马克斯·普朗克通过启发式推导得出了一个公式,解释了观测到的谱,假设在含有黑体辐射的腔体中,假设的电荷振荡器只能以最小增量E改变其能量,该增量与其相关电磁波的频率成正比。虽然普朗克最初认为将能量分成增量的假设只是为了得到正确答案的数学技巧,但其他物理学家,包括阿尔伯特·爱因斯坦,在他的基础上进行了进一步发展,普朗克的洞察力现在被认为对量子理论具有根本性的重要性。
\subsection{定律}
每个物体都会自发且持续地发射电磁辐射,物体的谱辐射强度 \( B_\nu \) 描述了特定辐射频率下,每单位面积、每单位立体角和每单位频率的谱发射功率。普朗克辐射定律给出的关系表明,随着温度的升高,物体辐射的总能量增加,且辐射谱的峰值会向短波长方向移动。根据普朗克分布定律,在给定温度下,谱能量密度(单位体积每单位频率的能量)由下式给出:
\[
u_\nu(\nu, T) = \frac{8 \pi h \nu^3}{c^3} \cdot \frac{1}{\exp \left( \frac{h \nu}{k_{\mathrm{B}} T} \right) - 1}~
\]
或者,该定律可以表示为物体在绝对温度 \( T \) 下,频率为 \( \nu \) 的谱辐射强度:
\[
B_\nu(\nu, T) = \frac{2 h \nu^3}{c^2} \cdot \frac{1}{\exp \left( \frac{h \nu}{k_{\mathrm{B}} T} \right) - 1}~
\]
其中 \( k_{\mathrm{B}} \) 是玻尔兹曼常数,\( h \) 是普朗克常数,\( c \) 是介质中的光速,无论是物质还是真空。谱辐射强度 \( B_\nu \) 的 cgs 单位是 \( \text{erg} \cdot \text{s}^{-1} \cdot \text{sr}^{-1} \cdot \text{cm}^{-2} \cdot \text{Hz}^{-1} \)。术语 \( B \) 和 \( u \) 通过因子 \( \frac{4 \pi}{c} \) 相关,因为 \( B \) 与方向无关且辐射以光速 \( c \) 传播。谱辐射强度也可以按单位波长 \( \lambda \) 表示,而不是按单位频率。此外,该定律还可以用其他术语表示,例如在某个波长下发射的光子数,或辐射体积内的能量密度。

在低频极限(即长波长)下,普朗克定律趋近于雷leigh–Jeans定律,而在高频极限(即短波长)下,它趋近于维恩近似。

马克斯·普朗克于1900年提出了这个定律,定律中只有经验确定的常数,后来他证明,将其表示为能量分布时,它是热力学平衡辐射的唯一稳定分布。作为能量分布,它是热平衡分布族中的一个成员,包括玻色–爱因斯坦分布、费米–狄拉克分布和麦克斯韦–玻尔兹曼分布。
\subsection{黑体辐射}
\begin{figure}[ht]
\centering
\includegraphics[width=8cm]{./figures/21e7152f5833fb75.png}
\caption{太阳近似为一个黑体辐射源。它的有效温度约为5777 K。} \label{fig_HTFS_2}
\end{figure}
黑体是一个理想化的物体,能够吸收和发射所有频率的辐射。在接近热力学平衡的状态下,发射的辐射可以通过普朗克定律精确描述。由于其依赖于温度,普朗克辐射被称为热辐射,这意味着物体的温度越高,它在每个波长上发射的辐射越多。

普朗克辐射在一个与物体温度相关的波长处具有最大强度。例如,在室温下(约300 K),物体发出的热辐射主要是红外线,并且是不可见的。在较高的温度下,红外辐射的数量增加并可以感受到热量,同时发射更多的可见辐射,物体呈现可见的红色光。在更高的温度下,物体变为明亮的黄色或蓝白色,并发射大量短波长的辐射,包括紫外线甚至X射线。太阳的表面(约6000 K)发射大量的红外线和紫外线辐射,其发射在可见光谱中达到峰值。由于温度变化引起的这种变化被称为维恩位移定律。

普朗克辐射是任何处于热平衡状态的物体从其表面发射的最大辐射量,无论其化学成分或表面结构如何。辐射穿过介质之间的界面时,可以通过界面的发射率来表征(实际辐射强度与理论普朗克辐射强度的比值),通常用符号 \( \epsilon \) 表示。发射率通常依赖于化学成分、物理结构、温度、波长、传输角度和偏振状态。自然界面上,发射率总是在 \( \epsilon = 0 \) 和 \( \epsilon = 1 \) 之间。

一个与另一个介质相接触的物体,如果该介质的发射率 \( \epsilon = 1 \) 且能够吸收所有射入的辐射,则被称为黑体。黑体的表面可以通过在一个大密封容器的墙壁上开一个小孔来建模,这个容器在均匀温度下保持不透明的墙壁,且在每个波长上都不是完全反射的。在平衡状态下,这个容器内部的辐射由普朗克定律描述,离开小孔的辐射也由此定律描述。

就像麦克斯韦–玻尔兹曼分布是物质粒子气体在热平衡状态下的唯一最大熵能量分布一样,普朗克分布也是光子气体的最大熵分布。与物质气体不同,物质气体的质量和粒子数起着重要作用,而光子气体在热平衡状态下的谱辐射强度、压强和能量密度完全由温度决定。

如果光子气体不是普朗克分布的,热力学第二定律保证了交互作用(光子与其他粒子之间的相互作用,甚至在足够高的温度下,光子之间的相互作用)会导致光子能量分布的变化,并最终趋近于普朗克分布。在热力学平衡的过程中,光子会以正确的数量和能量被创造或湮灭,直到它们填充整个腔体,并且达到普朗克分布,直到达到平衡温度。就像气体是由多个子气体组成的,每个子气体对应一个波长范围,每个子气体最终都会达到共同的温度。

量 \( B_\nu(\nu, T) \) 是谱辐射强度,作为温度和频率的函数。它在国际单位制(SI)中的单位为 \( \text{W} \cdot \text{m}^{-2} \cdot \text{sr}^{-1} \cdot \text{Hz}^{-1} \)。一个无穷小的功率 \( B_\nu(\nu, T) \cos \theta \, dA \, d\Omega \, d\nu \) 被辐射到由角度 \( \theta \) 描述的方向,从表面法线处的无穷小面积 \( dA \) 向无穷小的立体角 \( d\Omega \) 以及无穷小频率带宽 \( d\nu \) 辐射,其中心频率为 \( \nu \)。辐射到任何立体角的总功率是 \( B_\nu(\nu, T) \) 对这三个量的积分,且由斯特凡–玻尔兹曼定律给出。普朗克辐射的谱辐射强度对于每个方向和偏振角度都是相同的,因此黑体被称为兰伯特辐射源。
\subsection{不同形式}
普朗克定律可以根据不同科学领域的惯例和偏好,以多种形式出现。下表总结了谱辐射强度定律的不同形式。左侧的形式通常在实验领域中遇到,而右侧的形式则通常在理论领域中遇到。
\begin{figure}[ht]
\centering
\includegraphics[width=14.25cm]{./figures/a6e4d32ae252241b.png}
\caption{} \label{fig_HTFS_3}
\end{figure}
在分数带宽公式中,\(x = \frac{h \nu}{k_{\mathrm{B}} T} = \frac{hc}{\lambda k_{\mathrm{B}} T}\)积分是相对于\(\mathrm{d} (\ln x) = \mathrm{d} (\ln \nu) = \frac{\mathrm{d} \nu}{\nu} = -\frac{\mathrm{d} \lambda}{\lambda} = -\mathrm{d} (\ln \lambda)\)

普朗克定律也可以通过将 \( B \) 乘以 \( \frac{4 \pi}{c} \) 来用谱能量密度(\( u \))表示:[17]
\[
u_{i}(T) = \frac{4 \pi}{c} B_{i}(T)~
\]
\begin{figure}[ht]
\centering
\includegraphics[width=14.25cm]{./figures/ff7e3670c7064133.png}
\caption{} \label{fig_HTFS_4}
\end{figure}
这些分布表示黑体的谱辐射强度——即从辐射表面发射的功率,单位投影面积、单位立体角、单位谱量(频率、波长、波数或其角度等效物,或分数频率或波长)。由于辐射是各向同性的(即与方向无关),因此以某个角度发射的功率与投影面积成正比,按照兰伯特余弦定律,也与该角度的余弦成正比,并且是非偏振的。
\subsubsection{谱变量形式之间的对应关系}
不同的谱变量需要不同的普朗克定律表达形式。通常,不能通过简单地用一个变量替代另一个变量来转换普朗克定律的各种形式,因为这不会考虑到不同形式的单位不同。波长和频率的单位是倒数关系。

这些对应形式是相关的,因为它们表达的是同一个物理事实:对于特定的物理谱增量,相应的物理能量增量会被辐射出去。

无论是以频率增量 \( d\nu \) 的形式表达,还是以波长增量 \( d\lambda \) 的形式,或者是以分数带宽 \( \frac{d\nu}{\nu} \) 或 \( \frac{d\lambda}{\lambda} \) 的形式表达,结果都是相同的。引入负号可以表示频率增量与波长减小之间的对应关系。

为了转换相应的形式,使它们在相同单位下表达相同的量,我们通过谱增量来相乘。然后,对于特定的谱增量,特定的物理能量增量可以写成:
\[
B_{\lambda}(\lambda, T)\,d\lambda = -B_{\nu}(\nu(\lambda), T)\,d\nu~
\]
从而得到:
\[
B_{\lambda}(\lambda, T) = -\frac{d\nu}{d\lambda} B_{\nu}(\nu~(\lambda), T)~
\]
此外,\( \nu(\lambda) = \frac{c}{\lambda} \),所以 \( \frac{d\nu}{d\lambda} = -\frac{c}{\lambda^2} \)。替换后给出频率和波长形式之间的对应关系,并考虑它们不同的维度和单位。[15][18] 由此得出:
\[
\frac{B_{\lambda}(T)}{B_{\nu}(T)} = \frac{c}{\lambda^2} = \frac{\nu^2}{c}~
\]
显然,普朗克定律的谱分布峰值位置取决于所选择的谱变量。然而,从某种意义上讲,这个公式意味着根据维恩位移定律,谱分布的形状是温度独立的,如下面的 § 性质 §§ 百分位数部分所述。

分数带宽形式与其他形式之间的关系为:[16]
\[
B_{\ln x} = \nu B_{\nu} = \lambda B_{\lambda}~
\]
\subsubsection{第一和第二辐射常数}
在上述普朗克定律的不同形式中,波长和波数形式使用了包含物理常数的术语 \( 2hc^2 \) 和 \( \frac{hc}{k_{\mathrm{B}}} \)。因此,这些术语可以被视为物理常数[19],因此被称为第一辐射常数 \( c_{1L} \) 和第二辐射常数 \( c_2 \),其表达式为:
\[
c_{1L} = 2hc^2~
\]
和
\[
c_2 = \frac{hc}{k_{\mathrm{B}}}~
\]
利用这些辐射常数,普朗克定律的波长形式可以简化为:
\[
L(\lambda, T) = \frac{c_{1L}}{\lambda^5} \frac{1}{\exp\left(\frac{c_2}{\lambda T}\right) - 1}~
\]
波数形式也可以相应简化。

这里使用 \( L \) 而不是 \( B \),因为 \( L \) 是谱辐射强度的国际单位制符号。\( c_{1L} \) 中的 \( L \) 就指代此符号。这个引用是必要的,因为普朗克定律可以重新表述为给出谱辐射离开度 \( M(\lambda, T) \) 而不是谱辐射强度 \( L(\lambda, T) \),在这种情况下,\( c_1 \) 替代 \( c_{1L} \),并且:
\[
c_1 = 2\pi hc^2~
\]
因此,普朗克定律对于谱辐射离开度可以写成:
\[
M(\lambda, T) = \frac{c_1}{\lambda^5} \frac{1}{\exp\left(\frac{c_2}~{\lambda T}\right) - 1}~
\]
随着测量技术的提高,国际计量大会已经修订了对 \( c_2 \) 的估算;有关详情,请参见普朗克位置 § 国际温标部分。
\subsection{普朗克定律}
\begin{figure}[ht]
\centering
\includegraphics[width=10cm]{./figures/1576387588a8fb82.png}
\caption{高能振荡器的冻结} \label{fig_HTFS_5}
\end{figure}
普朗克定律描述了在热力学平衡中电磁辐射的独特和特征性的谱分布,当时物质或能量没有净流动。[2] 其物理学最容易通过考虑一个具有刚性不透明壁的腔体中的辐射来理解。壁的运动可能会影响辐射。如果墙壁不是不透明的,那么热力学平衡就不是孤立的。值得解释的是热力学平衡是如何达到的。主要有两种情况:(a)当热力学平衡的趋近发生在物质存在的情况下,腔体的墙壁对于每个波长都不完全反射,或者墙壁对于所有波长完全反射,而腔体中包含一个小黑体(这是普朗克主要考虑的情况);或(b)当热力学平衡的趋近发生在物质不存在的情况下,墙壁对于所有波长都完全反射,并且腔体中没有物质。对于不封闭在这种腔体中的物质,热辐射可以通过适当使用普朗克定律进行近似解释。

经典物理学通过能量均分定理引出了紫外灾难,预测黑体辐射的总强度是无限的。如果补充一个经典上无法辩解的假设,认为某些原因导致辐射是有限的,那么经典热力学可以解释普朗克分布的某些方面,例如斯特凡–玻尔兹曼定律和维恩位移定律。对于物质存在的情况,量子力学提供了一个良好的解释,正如下面“爱因斯坦系数”部分所述。这是爱因斯坦所考虑的情况,并且如今用于量子光学。[20][21] 对于物质不存在的情况,则需要量子场论,因为具有固定粒子数的非相对论量子力学无法提供足够的解释。
\subsubsection{光子}
普朗克定律的量子理论解释将辐射视为在热力学平衡中的无质量、无电荷的玻色子粒子气体,即光子。光子被视为电荷粒子之间电磁相互作用的载体。光子的数量不是守恒的。光子会在适当的数量和能量下被创造或湮灭,以使腔体充满普朗克分布。在热力学平衡中的光子气体,其内部能量密度完全由温度决定;此外,压力也完全由内部能量密度决定。这与物质气体的热力学平衡不同,在物质气体中,内部能量不仅由温度决定,而且还独立地由不同分子的数量决定,且不同分子具有不同的特性。在给定温度下,对于不同的物质气体,压力和内部能量密度可以独立变化,因为不同的分子可以独立地承载不同的激发能量。

普朗克定律是玻色–爱因斯坦分布的一个极限,后者是描述热力学平衡中非交互作用玻色子的能量分布。在无质量玻色子(如光子和胶子)的情况下,化学势为零,玻色–爱因斯坦分布就简化为普朗克分布。还有另一种基本的平衡能量分布:费米–狄拉克分布,它描述了热力学平衡中的费米子,如电子。两种分布的不同之处在于,多个玻色子可以占据相同的量子态,而多个费米子不能。在低密度的情况下,每个粒子可用的量子态数量较大,这种差异变得无关紧要。在低密度极限下,玻色–爱因斯坦分布和费米–狄拉克分布都会简化为麦克斯韦–玻尔兹曼分布。
\subsubsection{基尔霍夫的热辐射定律}
基尔霍夫的热辐射定律是对一个复杂物理情境的简洁概述。以下是该情境的一个介绍性概述,远不是严格的物理论证。此处的目的是仅总结该情境中的主要物理因素及其主要结论。

\textbf{热辐射的光谱依赖性}

导热和辐射热传递之间是有区别的。辐射热传递可以通过过滤器,只让特定频率范围的辐射通过。

通常知道,物体的温度越高,它在每个频率上辐射的热量越多。

在一个不透明的刚性墙体的腔体中,如果墙体在任何频率上都不是完全反射的,在热力学平衡下,腔体内只有一个温度,而且该温度必须是每种频率辐射的共同温度。

可以想象两个这样的腔体,每个腔体都处于各自独立的辐射和热力学平衡中。可以想象一个光学装置,它允许两腔体之间通过辐射热传递,但只通过一个特定频带的辐射频率。如果两个腔体在该频带的光谱辐射强度值不同,热量可能从较热的腔体传递到较冷的腔体。可以提出使用这种特定频带的热传递来驱动热机。如果两个腔体的温度相同,那么热力学第二定律不允许热机工作。由此可以推断,对于两个腔体共享的温度,所有频带的光谱辐射强度值也必须相同。这对于每个频带都成立。[22][23][24] 这一点首先被巴尔福·斯图尔特(Balfour Stewart)和后来基尔霍夫(Kirchhoff)明确地认识到。巴尔福·斯图尔特通过实验发现,在所有表面中,灯烟黑表面辐射出最多的热辐射,不论辐射的质量如何,经过不同的过滤器测量得出。

从理论上思考,基尔霍夫进一步指出,这意味着任何处于热力学平衡的腔体,其辐射频率作为光谱辐射强度的函数,必须是温度的唯一普遍函数。他假设了一个理想的黑体,它与周围环境的界面正好使其吸收所有射向它的辐射。根据亥姆霍兹互易原理,这样的物体内部的辐射将不受阻碍地直接传递到其周围环境,而不在界面处反射。在热力学平衡下,来自这样的物体的热辐射将具有作为温度函数的那个唯一普遍的光谱辐射强度。这个洞察力正是基尔霍夫热辐射定律的根源。

\textbf{吸收率与发射率的关系}

假设有一个小的均匀球形材料体 X,温度为 TX,位于一个大型腔体内,该腔体的壁材为 Y,温度为 TY。物体 X 会发出自身的热辐射。在某一特定频率 ν下,物体 X 在其中心的某个截面上向该截面法线方向发出的辐射可以表示为 Iν,X(TX),这是物体 X 的特征性辐射强度。对于该频率 ν,腔体壁的辐射功率可以表示为 Iν,Y(TY),即壁材 Y 在温度 TY 下的辐射强度。对于材料 X,定义吸收率 αν,X,Y(TX, TY) 为 X 吸收的该辐射的比例,那么入射能量的吸收速率为 αν,X,Y(TX, TY) Iν,Y(TY)。

在该截面上,物体 X 吸收辐射的能量累积速率 q(ν,TX,TY) 可以表示为:
\[ q(\nu ,T_{X},T_{Y}) = \alpha_{\nu ,X,Y}(T_{X},T_{Y}) I_{\nu ,Y}(T_{Y}) - I_{\nu ,X}(T_{X})~\]
基尔霍夫的开创性洞察,前面提到过,表明在热力学平衡下温度为 T 时,存在一个唯一的普遍辐射分布,现今称为 Bν(T),它与材料 X 和 Y 的化学特性无关,这为我们理解任何物体的辐射交换平衡提供了宝贵的视角。

当温度为 T 的热力学平衡存在时,腔体壁的辐射具有该唯一的普遍值,因此 Iν,Y(TY) = Bν(T)。进一步地,可以定义物体 X 的发射率 εν,X(TX),使得在温度 TX = T 时,Iν,X(TX) = Iν,X(T) = εν,X(T) Bν(T)。

当温度为 T = TX = TY 时,热平衡条件下能量的累积速率为零,即 q(ν,TX,TY) = 0。由此可以得出,在热力学平衡时,当 T = TX = TY:
\[ 0 = \alpha_{\nu ,X,Y}(T,T) B_{\nu}(T) - \epsilon_{\nu ,X}(T) B_{\nu}(T)~\]
基尔霍夫指出,由此可知,在热力学平衡时,当 T = TX = TY:
\[ \alpha_{\nu ,X,Y}(T,T) = \epsilon_{\nu ,X}(T) ~\]
引入特殊符号 αν,X(T),表示材料 X 在热力学平衡下温度为 T 时的吸收率(这一发现由爱因斯坦提出,后文将说明),可以进一步得出以下等式:
\[ \alpha_{\nu ,X}(T) = \epsilon_{\nu ,X}(T) ~\]
这里示范的吸收率和发射率相等的关系特指在热力学平衡下的温度 T,一般情况下如果不满足热力学平衡条件,这种关系不必然成立。发射率和吸收率是材料分子各自的性质,但它们依赖于分子激发态的分布,因为存在一种叫做“受激辐射”的现象,这一现象是由爱因斯坦发现的。当材料处于热力学平衡状态或处于被称为局部热力学平衡的状态时,发射率和吸收率才会相等。非常强的入射辐射或其他因素可能会破坏热力学平衡或局部热力学平衡。气体中的局部热力学平衡意味着分子碰撞对分子激发态分布的影响远大于辐射的吸收和发射。

基尔霍夫指出,他并不知道 Bν(T) 的精确形式,但他认为找出它是非常重要的。在基尔霍夫提出其存在性和特性的一般原理四十年后,普朗克的贡献是确定了这一平衡分布 Bν(T) 的精确数学表达式。
\subsubsection{黑体}
在物理学中,理想的黑体被定义为能够完全吸收所有频率 ν 的电磁辐射的物体(因此称为“黑体”)。根据基尔霍夫的热辐射定律,这意味着对于每一个频率 ν,在热力学平衡下温度为 T 时,黑体的吸收率和发射率都满足:αν,B(T) = εν,B(T) = 1,因此黑体的热辐射总是与普朗克定律所指定的总量相等。没有任何物体能够发出超过黑体的热辐射,因为如果物体与辐射场处于热平衡状态,它所发出的能量将大于它所接收的辐射能量。

尽管完美的黑体材料在自然界中不存在,但在实践中,可以较为准确地近似黑色表面。至于其物质内部,具有明确界面的凝聚态物体(如液体、固体或等离子体),如果完全不透明,它对于辐射而言就是完全黑色的。这意味着它会吸收所有穿透物体与周围环境接触界面的辐射,并进入物体。实际上,这一点并不难实现。另一方面,完美的黑色界面在自然界中并不存在。完美的黑色界面不会反射任何辐射,而是让所有辐射通过它,不论来自哪一侧。为了制造一个有效的黑色界面,最好的方法是通过在一个完全不透明、且在任何频率下都不会完美反射的刚性材料体的墙壁上打一个小孔,来模拟一个“界面”,同时保持其墙壁处于受控的温度状态。除此之外,墙壁的材料成分没有限制。进入小孔的辐射几乎没有可能在不被墙壁多次撞击吸收的情况下逃离腔体。
\subsubsection{兰伯特余弦定律}
如普朗克所解释,辐射体的内部由物质组成,并与其相邻的物质介质接触,通常辐射来自物体表面的辐射是从该介质中观察到的。界面并不是由物理物质组成的,而是一个理论构想,是一个二维的数学面,是两个相邻介质的共同属性,严格来说并不属于任何一个介质。这样的界面既不能吸收也不能发射辐射,因为它不是由物理物质组成的;但它是辐射反射和透射的地方,因为它是光学性质的间断面。界面上的辐射反射和透射遵循斯托克斯-亥姆霍兹互易原理。

在处于热力学平衡状态的腔体内的黑体内部,在温度 T 下,辐射是均匀的、各向同性的且无偏振的。黑体吸收所有照射到它的电磁辐射,并且不反射任何辐射。根据亥姆霍兹互易原理,来自黑体内部的辐射在其表面不会被反射,而是完全透过表面传递到外部。由于辐射在黑体内部是各向同性的,从其内部到外部通过表面传递的辐射的光谱辐射强度与方向无关。

这可以通过以下方式表示,即热力学平衡下,黑体表面发出的辐射遵循兰伯特余弦定律。这意味着,从黑体实际辐射表面上某个微小面积元素 dA 发出的光谱通量 dΦ(dA, θ, dΩ, dν),在由θ与该面积元素的法线方向夹角为θ的给定方向上被检测,进入以该方向为中心的固体角 dΩ 中,在频率带宽 dν 内,可以表示为:
\[
\frac{d\Phi (dA,\theta ,d\Omega ,d\nu )}{d\Omega } = L^{0}(dA,d\nu )\,dA\,d\nu \,\cos \theta~
\]
其中,L0(dA, dν) 表示该面积元素 dA 在法线方向 θ = 0 时测得的单位面积、单位频率、单位固体角的辐射通量。

存在 cos θ 因子是因为光谱辐射强度所指的面积是实际辐射表面面积投影到与θ方向垂直的平面上的面积。这就是余弦定律的名称由来。

考虑到热力学平衡下,黑体表面辐射的光谱辐射强度与方向无关,可以得出 L0(dA, dν) = Bν(T),因此有:
\[
\frac{d\Phi (dA,\theta ,d\Omega ,d\nu )}{d\Omega } = B_{\nu }(T)\,dA\,d\nu \,\cos \theta~
\]
因此,兰伯特余弦定律表达了热力学平衡下,黑体表面辐射强度 Bν(T) 与方向的独立性。
\subsubsection{斯特藩–玻尔兹曼定律}
总功率(P)是黑体表面单位面积辐射的功率,可以通过对兰伯特定律所给出的黑体光谱通量在所有频率和对应于半球(h)上的固体角上进行积分来求得。
\[
P = \int_0^\infty d\nu \int_h d\Omega \, B_{\nu} \cos(\theta)~
\]
无穷小的固体角可以用球坐标表示:
\[
d\Omega = \sin(\theta) \, d\theta \, d\phi~
\]
因此,功率可以表示为:
\[
P = \int_0^\infty d\nu \int_0^{\frac{\pi}{2}} d\theta \int_0^{2\pi} d\phi \, B_{\nu}(T) \cos(\theta) \sin(\theta) = \sigma T^4~
\]
其中,\(\sigma\) 是斯特藩–玻尔兹曼常数,定义为:
\[
\sigma = \frac{2 k_{\mathrm{B}}^4 \pi^5}{15 c^2 h^3} \approx 5.670400 \times 10^{-8} \, \mathrm{J\,s^{-1}m^{-2}K^{-4}}~
\]
斯特藩–玻尔兹曼常数 \(\sigma\) 描述了单位面积黑体在单位时间内辐射的功率与其温度的四次方成正比,即斯特藩–玻尔兹曼定律。
\subsubsection{辐射传输}
辐射传输方程描述了辐射在通过物质介质时如何受到影响。对于介质在某点附近处于热力学平衡的特殊情况,普朗克定律具有特别重要的意义。

为简化起见,我们可以考虑没有散射的线性稳态。在辐射传输方程中,对于一束穿过小距离 \(ds\) 的光束,能量是守恒的:该光束的(光谱)辐射强度 \(I_{\nu}\) 的变化等于被物质介质移除的部分和从物质介质获得的部分。如果辐射场与物质介质处于平衡状态,这两者将相等。物质介质将具有一定的发射系数和吸收系数。

吸收系数 \(\alpha\) 是光束强度在其传播的距离 \(ds\) 上的相对变化,单位为长度的倒数。它由两个部分组成:由于吸收造成的强度降低和由于受激辐射造成的强度增加。受激辐射是物质体由于和入射辐射成比例的作用而产生的辐射,它被包含在吸收项中,因为它与入射辐射的强度成比例。由于吸收量通常会随物质的密度 \(\rho\) 线性变化,我们可以定义“质量吸收系数”\(\kappa_{\nu} = \frac{\alpha}{\rho}\),它是物质本身的特性。光束由于吸收而强度变化的表达式为:
\[
dI_{\nu} = -\kappa_{\nu} \rho I_{\nu} \, ds~
\]
“质量发射系数”\(j_{\nu}\) 是单位体积的小体积元素的辐射强度除以其质量(因为和质量吸收系数一样,发射与发射物质的质量成正比),单位为功率·固体角\(^{-1}\)·频率\(^{-1}\)·密度\(^{-1}\)。像质量吸收系数一样,它也是物质本身的特性。光束由于发射而变化的表达式为:
\[
dI_{\nu} = j_{\nu} \rho \, ds~
\]
辐射传输方程是这两个贡献的总和:
\[
\frac{dI_{\nu}}{ds} = j_{\nu} \rho - \kappa_{\nu} \rho I_{\nu}~
\]
如果辐射场与物质介质处于平衡状态,那么辐射将是均匀的(与位置无关),从而 \(dI_{\nu} = 0\),并且:
\[
\kappa_{\nu} B_{\nu} = j_{\nu}~
\]
这再次是基尔霍夫定律的一个表达,描述了介质的两个物质特性,并且它在介质处于热力学平衡的点上得到了辐射传输方程:
\[
\frac{dI_{\nu}}{ds} = \kappa_{\nu} \rho (B_{\nu} - I_{\nu})~
\]
\subsubsection{爱因斯坦系数}
详细平衡原理指出,在热力学平衡状态下,每一个基本过程都由其逆过程达到平衡。

在1916年,阿尔伯特·爱因斯坦将这一原理应用于原子层面,研究了由于原子在两个特定能级之间跃迁而辐射和吸收辐射的情况,进而对辐射传输方程和基尔霍夫定律在这种辐射中的应用提供了更深的理解。如果能级1是能量为 \(E_1\) 的低能级,而能级2是能量为 \(E_2\) 的高能级,那么辐射或吸收的频率 \(\nu\) 将由玻尔频率条件决定:
\[
E_2 - E_1 = h\nu~
\]
如果 \(n_1\) 和 \(n_2\) 分别是原子处于状态1和状态2的数密度,那么这些数密度随时间变化的速率将由以下三种过程决定:

\textbf{自发辐射}
\[
\left(\frac{dn_1}{dt}\right)_{\mathrm{spon}} = A_{21} n_2~
\]
\textbf{受激辐射}
\[
\left(\frac{dn_1}{dt}\right)_{\mathrm{stim}} = B_{21} n_2 u_\nu~
\]
\textbf{光吸收} 
\[
\left(\frac{dn_2}{dt}\right)_{\mathrm{abs}} = B_{12} n_1 u_\nu~
\]
其中,\( u_\nu \) 是辐射场的光谱能量密度。三个参数 \( A_{21} \)、\( B_{21} \) 和 \( B_{12} \),称为爱因斯坦系数,与两个能级(状态)之间跃迁所产生的光子频率 \( \nu \) 相关。因此,光谱中的每一条线都有一组与之相关的系数。当原子与辐射场处于平衡状态时,辐射强度将遵循普朗克定律,并且根据详细平衡原理,这些速率的总和必须为零:
\[
0 = A_{21} n_2 + B_{21} n_2 \frac{4\pi}{c} B_\nu(T) - B_{12} n_1 \frac{4\pi}{c} B_\nu(T)~
\]
由于原子也处于平衡状态,两个能级的粒子数由玻尔兹曼因子关联:
\[
\frac{n_2}{n_1} = \frac{g_2}{g_1} e^{-h\nu / k_{\mathrm{B}} T}~
\]
其中 \( g_1 \) 和 \( g_2 \) 是相应能级的简并度。将上述两个方程结合,并要求它们在任意温度下有效,可得爱因斯坦系数之间的两个关系:
\[
\frac{A_{21}}{B_{21}} = \frac{8\pi h\nu^3}{c^3}~
\]
\[
\frac{B_{21}}{B_{12}} = \frac{g_1}{g_2}~
\]
因此,知道一个系数即可得出其他两个系数。

对于各向同性的吸收和辐射,上文辐射传输部分定义的辐射系数 (\( j_\nu \)) 和吸收系数 (\( \kappa_\nu \)) 可以通过爱因斯坦系数表示。爱因斯坦系数之间的关系将得出基尔霍夫定律的表达式,如上文辐射传输部分所述,即:
\[
j_\nu = \kappa_\nu B_\nu~
\]
这些系数适用于原子和分子。
\subsection{属性}
\subsubsection{峰值}
分布 \( B_\nu \)、\( B_\omega \)、\( B_{\tilde{\nu}} \) 和 \( B_k \) 在一个光子能量上达到峰值:
\[
E = \left[ 3 + W \left( -3e^{-3} \right) \right] k_{\mathrm{B}} T \approx 2.821 \, k_{\mathrm{B}} T~
\]
其中 \( W \) 是兰伯特 W 函数,\( e \) 是欧拉常数。

然而,分布 \( B_\lambda \) 在一个不同的能量上达到峰值:
\[
E = \left[ 5 + W \left( -5e^{-5} \right) \right] k_{\mathrm{B}} T \approx 4.965 \, k_{\mathrm{B}} T~
\]
这是因为,正如前面所提到的,不能通过简单地用 \( \lambda \) 替代 \( \nu \) 从 \( B_\nu \) 转换到 \( B_\lambda \)。此外,还必须乘以:\(| \d\nu/d\lambda| = \frac{c}{\lambda^2}\)这会将分布的峰值移动到更高的能量。以上的峰值是光子的模态能量,当频率或波长被均匀分成等大小的区间时。通过将 \( hc \)(14387.770 微米·K)除以这些能量表达式,可以得到峰值的波长。

在这些峰值处的光谱辐射度为:
\[
B_{\nu, {\text{max}}}(T) = \frac{2 k_{\mathrm{B}}^3 T^3 x^3}{h^2 c^2} \frac{1}{e^x - 1} \approx 1.896 \times 10^{-19} \, \frac{\text{W}}{\text{m}^2 \cdot \text{Hz} \cdot \text{sr}} \times \left( \frac{T}{\text{K}} \right)^3~
\]
其中
\[
x = 3 + W \left( -3e^{-3} \right)~
\]
而波长方向的辐射度为:
\[
B_{\lambda, {\text{max}}}(T) = \frac{2 k_{\mathrm{B}}^5 T^5 x^5}{h^4 c^3} \frac{1}{e^x - 1} \approx 4.096 \times 10^{-6} \, \frac{\text{W}}{\text{m}^2 \cdot \text{sr}} \times \left( \frac{T}{\text{K}} \right)^5~
\]
其中
\[
x = 5 + W \left( -5e^{-5} \right)~
\]
同时,黑体辐射的光子平均能量为:
\[
E = \left[ \frac{\pi^4}{30 \zeta(3)} \right] k_{\mathrm{B}} T \approx 2.701 \, k_{\mathrm{B}} T~
\]
其中 \( \zeta \) 是黎曼ζ函数。
\subsubsection{近似}
\begin{figure}[ht]
\centering
\includegraphics[width=10cm]{./figures/f2dc52cc65ed57d3.png}
\caption{计划定律(绿色)、瑞利-金斯定律(红色)和维恩近似(蓝色)在 8 毫开尔文温度下的黑体辐射与频率的对数-对数图。} \label{fig_HTFS_6}
\end{figure}
在低频极限(即长波长)下,普朗克定律变为瑞利–金斯定律:
\[
B_\nu (T) \approx \frac{2 \nu^2}{c^2} k_{\mathrm{B}} T~
\]
或者
\[
B_\lambda (T) \approx \frac{2c}{\lambda^4} k_{\mathrm{B}} T~
\]
辐射度随频率的平方增加,展示了紫外灾难。在高频极限(即短波长)下,普朗克定律趋近于维恩近似:
\[
B_\nu (T) \approx \frac{2 h \nu^3}{c^2} e^{-\frac{h\nu}{k_{\mathrm~{B}} T}}~
\]
或者
\[
B_\lambda (T) \approx \frac{2 h c^2}{\lambda^5} e^{-\frac{hc}{\lambda k_{\mathrm{B}} T}}~
\]
\subsubsection{百分位数}
\begin{figure}[ht]
\centering
\includegraphics[width=6cm]{./figures/a049e1ab8b8dd1d7.png}
\caption{} \label{fig_HTFS_7}
\end{figure}
维恩位移定律的较强形式指出,普朗克定律的形状与温度无关。因此,可以列出总辐射的百分位点以及波长和频率的峰值,以一种通过温度 T 除以得到波长 λ 的形式。下表的第二列列出了相应的 λT 值,即那些在第一个列中给定的辐射百分位点对应的 x 值,其中波长 λ 为 ⁠x/T μm。

即,0.01\% 的辐射波长小于 ⁠910/T μm,20\% 小于 ⁠2676/T μm,等等。波长和频率的峰值以粗体显示,分别出现在 25.0\% 和 64.6\% 处。41.8\% 点是波长-频率中性峰值(即波长或频率的对数变化单位功率的峰值)。这些是各自的普朗克定律函数 ⁠1/λ⁵、ν³ 和 ⁠ν²/λ² 分别除以 exp(⁠hν / kBT⁠) − 1 达到最大值的点。波长比率在 0.1\% 和 0.01\% 之间的差距(1110 比 910 多 22\%)比在 99.9\% 和 99.99\% 之间的差距(113374 比 51613 多 120\%)小,反映了在短波长(左端)的能量指数衰减和在长波长的多项式衰减。

选择哪个峰值取决于应用。传统上选择的是由维恩位移定律的较弱形式给出的 25.0\% 处的波长峰值。对于某些用途,将总辐射分为两半的中位数或 50\% 点可能更合适。后者更接近频率峰值而非波长峰值,因为在短波长处辐射呈指数衰减,而在长波长处则呈多项式衰减。中性峰值由于同样的原因出现在比中位数更短的波长处。
\begin{figure}[ht]
\centering
\includegraphics[width=10cm]{./figures/4add84c7283979af.png}
\caption{与5775 K黑体辐射相比的太阳光谱} \label{fig_HTFS_8}
\end{figure}
\textbf{与太阳光谱的比较}  

太阳辐射可以与大约5778 K的黑体辐射进行比较(但请参见图表)。右侧的表格展示了在此温度下黑体辐射的分布情况,并与太阳光的辐射分布进行了对比。为了进行对比,表格还展示了一个被建模为黑体的行星,其辐射温度为名义上的288 K(15°C)。
\begin{figure}[ht]
\centering
\includegraphics[width=10cm]{./figures/657553cb7e2518f9.png}
\caption{} \label{fig_HTFS_9}
\end{figure}
地球的温度变化较大,代表值为288 K(约15°C)。与太阳相比,其波长大约是太阳波长的二十倍,表格的第三列以微米为单位列出了这些数据(千纳米)。

也就是说,太阳辐射中只有1\%处于波长小于296 nm的范围,只有1\%处于波长大于3728 nm的范围。换算为微米,这意味着太阳辐射的98\%处于0.296至3.728 μm的波长范围内。而一个288 K的行星辐射出的98\%能量,则位于5.03至79.5 μm的波长范围,远高于太阳辐射的范围(如果使用频率 ν = c / λ 而非波长 λ 表示,则位于低频区)。

太阳辐射与行星辐射在波长上的这个数量级差异,导致了设计过滤器以允许通过一种辐射而阻挡另一种辐射变得相对容易。例如,普通玻璃或透明塑料制造的窗户可以通过至少80\%的来自5778 K太阳辐射,波长小于1.2 μm,而阻挡超过99\%的来自288 K热辐射,波长大于5 μm,而在这些波长下,大多数建筑级厚度的玻璃和塑料对热辐射是有效不透光的。

太阳辐射是到达大气顶层(TOA)的辐射。如表格所示,波长小于400 nm的紫外线辐射约占8\%,而波长大于700 nm的红外线辐射从约48\%的比例开始,占据总辐射的52\%。因此,只有40\%的TOA入射辐射对人眼是可见的。大气层会大大改变这些百分比,偏向可见光,因为它吸收了大部分紫外线和相当一部分红外线。