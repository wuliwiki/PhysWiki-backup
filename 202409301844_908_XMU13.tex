% 厦门大学 2013 年 考研 量子力学
% license Usr
% type Note

\textbf{声明}:“该内容来源于网络公开资料,不保证真实性,如有侵权请联系管理员”

\subsection{一 、}

(1)设$\varphi(x)_1, \varphi(x)_2$是体系的两个可能状态,有下面三种线性
叠加:

① $\phi_A = \varphi(x)_1 + \varphi(x)_2  e^{i\delta}$ 

② $\phi_B= \varphi(x)_1 + \varphi(x)_2 $

③ $\phi_C =e^{i\delta} (\varphi(x)_1 + \varphi(x)_2)$

式中$\delta$为实常数($\delta \ne 2n\pi$),问$\phi_A ,\phi_B , \phi_C$ 是否表示相同的态

(2)什么是厄米算符?它具有什么特征使得可观测量需要有厄米算符表示?试证明动量地 $x$ 分量是厄米算符。

(3)若体系的波函数为$Y_{lm} (\theta, \varphi)(l\ne 0)$,求其轨道角动量矢量与$z$轴的夹角。

(4)若两个力学量算符 $F$ 与 $G$ 的对易关系为$[F,G]=ik$,试写出 $F$与 $G$ 的测不准关系式,有哪些要求。

(5)粒子的总能量 $E=T+V$,若微观粒子处于经典禁区($E$ 小于 $V$),这是否意味着 $T$ 小于 0?为什么?