% LC 振荡电路
% keys 电路|振荡电路|电容|电感

电路中电压和电流的周期性变化称为\textbf{电磁振荡(electromagnetic oscillation)} , 电磁振荡与机械振动有类似的运动形式.产生电磁振荡的电路称为振荡电路.最简单的振荡电路是由一个电容器与一个自感线圈串联而成的,称为\textbf{ LC 电路(LC circuit)}.

电容公式
\begin{equation}\label{LC_eq1}
I = C \dv{U_c}{t}
\end{equation}
电感公式
\begin{equation}\label{LC_eq2}
U_L = L \dv{I}{t}
\end{equation}
由回路电压合为零,得
\begin{equation}\label{LC_eq3}
U_L + U_c = 0
\end{equation}
对\autoref{LC_eq1} 求导得 $\dv*{I}{t} = C \dv*[2]{U_c}{t}$, 代入\autoref{LC_eq2} 得 $U_L = LC \dv*[2]{U_c}{t}$. 代入  \autoref{LC_eq3}  得关于 $U_c$ 的微分方程
\begin{equation}
\dv[2]{U_c}{t} + \frac{1}{LC} U_c = 0
\end{equation}
根据二阶线性常系数齐次微分方程\upref{Ode2}的第3种情况,其通解为
\begin{equation}
U_c = U_0 \sin(\omega t + \phi_0) \qquad
\omega  = \frac{1}{\sqrt {LC}}
\end{equation}
\begin{equation}
I = C \dv{U_c}{t} = \sqrt{\frac{C}{L}} U_0 \cos(\omega t + \phi_0)
\end{equation}
 
