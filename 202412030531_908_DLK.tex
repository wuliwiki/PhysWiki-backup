% 狄拉克方程(综述)
% license CCBYSA3
% type Wiki

本文根据 CC-BY-SA 协议转载翻译自维基百科\href{https://en.wikipedia.org/wiki/Dirac_equation}{相关文章}。

在粒子物理学中,狄拉克方程是由英国物理学家保罗·狄拉克于1928年推导出的相对论波动方程。在其自由形式或包括电磁相互作用的情况下,它描述了所有自旋为1/2的有质量粒子,被称为“狄拉克粒子”,例如电子和夸克,这些粒子具有宇称对称性。它与量子力学原理和相对论的特殊理论一致,[1]并且是第一个在量子力学中完全考虑特殊相对论的理论。通过严格地解释氢谱的精细结构,它得到了验证。它在标准模型的构建中变得至关重要。[2]

该方程还暗示了一种新的物质形式——反物质,之前未曾被怀疑或观察到,几年的实验验证了这一点。它还为保罗的自旋现象学理论中引入多个分量波函数提供了理论依据。狄拉克理论中的波函数是四个复数值的向量(称为双自旋数),其中两个在非相对论极限下类似于保罗波函数,而与描述单一复数值波函数的薛定谔方程不同。此外,在零质量极限下,狄拉克方程简化为魏尔方程。

在量子场论的背景下,狄拉克方程被重新解释为描述与自旋1/2粒子相对应的量子场。

狄拉克没有完全意识到自己结果的重要性;然而,他关于自旋的解释——作为量子力学和相对论结合的结果——以及最终发现正电子,代表了理论物理学的伟大胜利之一。这一成就被认为与牛顿、麦克斯韦和爱因斯坦的工作相提并论。[3] 有些物理学家认为这方程是“现代物理学的真正种子”。[4] 该方程还被描述为“相对论量子力学的核心”,并且有人说“这方程可能是所有量子力学中最重要的方程”。[5]

狄拉克方程被刻在威斯敏斯特大教堂的地板上的一块纪念牌上。该纪念牌于1995年11月13日揭幕,纪念狄拉克的一生。[6]

\subsection{历史}
狄拉克方程在狄拉克最初提出的形式是:[7]:291 [8]
\[
\left( \beta mc^{2} + c \sum_{n=1}^{3} \alpha_{n} p_{n} \right) \psi(x,t) = i\hbar \frac{\partial \psi(x,t)}{\partial t}~
\]
其中,\(\psi(x,t)\) 是具有静止质量 \(m\) 的电子的波函数,\(x\) 和 \(t\) 为时空坐标,\(p_1, p_2, p_3\) 是动量的分量,被理解为薛定谔方程中的动量算符。\(c\) 是光速,\(\hbar\) 是约化普朗克常数;这些基本物理常数分别反映了特殊相对论和量子力学。

狄拉克提出此方程的目的是解释相对论性运动电子的行为,从而使得原子可以以与相对论一致的方式进行处理。他希望通过这种方式引入的修正可能对原子光谱的问题有所帮助。

在此之前,试图使旧量子理论与相对论理论兼容的努力——这些努力基于将电子在原子核周围可能非圆形轨道中存储的角动量离散化——都失败了,而海森堡、保利、约旦、薛定谔和狄拉克自己提出的新量子力学还未充分发展以处理这一问题。尽管狄拉克最初的目的已得到满足,但他的方程对物质结构有着更深远的影响,并引入了新的数学对象类,这些对象现在是基础物理学的重要元素。

方程中的新元素是四个 4 × 4 矩阵 \(\alpha_1, \alpha_2, \alpha_3\) 和 \(\beta\),以及四分量波函数 \(\psi\)。\(\psi\) 有四个分量,因为在配置空间中的任意一点处对其求值是一个双自旋数。它被解释为自旋向上的电子、自旋向下的电子、自旋向上的正电子和自旋向下的正电子的叠加。

这四个 4 × 4 矩阵 \(\alpha_k\) 和 \(\beta\) 都是厄米矩阵,并且是自反的:
\[
\alpha_i^2 = \beta^2 = I_4~
\]
它们互相反对易:
\[
\alpha_i \alpha_j + \alpha_j \alpha_i = 0 \quad (i \neq j)~
\]
\[
\alpha_i \beta + \beta \alpha_i = 0~
\]
这些矩阵及波函数的形式具有深刻的数学意义。伽马矩阵所表示的代数结构早在50年前由英国数学家 W. K. Clifford 创造。反过来,Clifford 的思想源于19世纪中期德国数学家赫尔曼·格拉斯曼在其《线性展开理论》(\textbf{Lineare Ausdehnungslehre})中的工作。
\subsubsection{使薛定谔方程相对论化}
狄拉克方程在表面上与描述自由粒子的薛定谔方程相似:
\[
- \frac{\hbar^2}{2m} \nabla^2 \phi = i\hbar \frac{\partial}{\partial t} \phi~
\]
左侧表示动量算符的平方除以两倍的质量,这是非相对论性动能。因为相对论将空间和时间视为一个整体,因此这一方程的相对论性推广要求空间和时间的导数必须对称地出现,就像麦克斯韦方程中描述光的行为一样——方程在空间和时间上的微分阶数必须相同。在相对论中,动量和能量是时空四矢量的空间和时间部分,称为四动量,它们通过相对论不变关系相互关联:
\[
E^2 = m^2 c^4 + p^2 c^2~
\]
这表示该四矢量的长度与静质量 \(m\) 成正比。将薛定谔理论中能量和动量的算符替代进入后,得到描述波动传播的克莱因–戈尔登方程,构建于相对论不变的物体基础之上:
\[
\left( -\frac{1}{c^2} \frac{\partial^2}{\partial t^2} + \nabla^2 \right) \phi = \frac{m^2 c^2}{\hbar^2} \phi~
\]
其中波函数 \(\phi\) 是相对论标量:它是一个复数,在所有参考系中具有相同的数值。空间和时间的导数都以二阶形式出现。这对方程的解释有重要影响。由于方程在时间导数上是二阶的,必须指定波函数本身和其一阶时间导数的初始值,才能求解具体问题。由于这两个初始值可以或多或少任意选择,波函数无法再像在薛定谔理论中那样,保持其决定电子在给定运动状态下的概率密度的作用。

在薛定谔理论中,概率密度由正定的表达式给出:
\[
\rho = \phi^* \phi~
\]
该密度根据概率流向量进行对流:
\[
J = - \frac{i \hbar}{2m} (\phi^* \nabla \phi - \phi \nabla \phi^*)~
\]
并且概率流和密度的守恒来自于连续方程:
\[
\nabla \cdot J + \frac{\partial \rho}{\partial t} = 0~
\]
密度为正定并且根据这个连续方程进行对流,意味着可以在某个区域内对密度积分并将总和设为 1,这个条件将由守恒定律保持。具有概率密度流的正确相对论性理论也必须具备这一特性。为了保持对流密度的概念,必须将薛定谔的密度和电流表达式推广,使得空间和时间的导数再次以对称的方式进入标量波函数的关系。薛定谔的表达式可以保留在电流中,但概率密度必须替换为对称形成的表达式(需要进一步解释):
\[
\rho = \frac{i \hbar}{2m c^2} \left( \psi^* \partial_t \psi - \psi \partial_t \psi^* \right)~
\]
这个表达式现在成为了时空四矢量的第四分量,整个概率四流密度具有相对论协变的表达式:
\[
J^\mu = \frac{i \hbar}{2m} \left( \psi^* \partial^\mu \psi - \psi \partial^\mu \psi^* \right)~
\]
连续方程依旧成立。现在,一切都与相对论兼容,但密度的表达式不再是正定的;\(\psi\) 和 \(\partial_t \psi\) 的初始值可以自由选择,因此密度可能变为负值,这对于合法的概率密度来说是不可能的。因此,在简单地假设波函数是相对论标量,并且它满足一个时间二阶方程的情况下,无法得到薛定谔方程的简单相对论性推广。

尽管它不是薛定谔方程的成功相对论性推广,这个方程在量子场论的背景下复兴,并被称为克莱因–戈尔登方程,用来描述无自旋粒子场(例如 \(\pi\) 介子或希格斯玻色子)。历史上,薛定谔本人在他的名字所命名的方程之前就得到了这个方程,但很快就放弃了它。在量子场论的背景下,这种不确定的密度被理解为对应于电荷密度,可以为正或负,而不是概率密度。
\subsubsection{狄拉克的突破}
因此,狄拉克想尝试一个在空间和时间上都一阶的方程。他假设了一个形式为:
\[
E\psi = (\vec{\alpha} \cdot \vec{p} + \beta m) \psi~
\]
其中,算符 \((\vec{\alpha}, \beta)\) 必须与 \((\vec{p}, t)\) 独立,以保证方程的线性性,并且与 \((\vec{x}, t)\) 独立,以保证时空均匀性。这些约束意味着 \((\vec{\alpha}, \beta)\) 算符将依赖于额外的动力学变量;根据这一要求,狄拉克得出结论,算符将依赖于与泡利矩阵相关的 4×4 矩阵。

例如,可以形式化地(即通过滥用符号)取相对论性的能量表达式:
\[
E = c \sqrt{p^2 + m^2 c^2}~
\]
将动量 \(p\) 替换为它的算符等价物,展开平方根为一个无限级数的导数算符,建立特征值问题,然后通过迭代法形式上解出方程。即使这种过程在技术上可能可行,大多数物理学家对这种方法并不抱有信心。

故事是这样的:狄拉克在剑桥大学盯着壁炉,思考这个问题时,突然想到取波动算符的平方根(参见半导数)来处理,形式如下:
\[
\nabla^2 - \frac{1}{c^2} \frac{\partial^2}{\partial t^2} = \left(A \partial_x + B \partial_y + C \partial_z + \frac{i}{c} D \partial_t \right) \left(A \partial_x + B \partial_y + C \partial_z + \frac{i}{c} D \partial_t \right)~
\]
展开右边可以看出,为了让所有交叉项(如 \(\partial_x \partial_y\))消失,必须假设:
\[
AB + BA = 0, \ldots~
\]
并且:
\[
A^2 = B^2 = \dots = 1~
\]
狄拉克当时刚刚专注于研究海森堡矩阵力学的基础,他立即理解到这些条件可以通过将 \(A\)、\(B\)、\(C\) 和 \(D\) 设为矩阵来满足,这意味着波函数有多个分量。这立刻解释了泡利的现象学自旋理论中出现的二维波函数,这在当时甚至对泡利本人来说也曾是个谜。然而,为了建立具有所需性质的系统,至少需要 4×4 矩阵——因此,波函数有四个分量,而不像泡利理论中的两个分量,或像薛定谔理论中的一个分量。这个四分量波函数代表了一类新的数学对象,这在物理理论中首次出现。

考虑到这种因子化,可以立即写出一个方程:
\[
\left(A \partial_x + B \partial_y + C \partial_z + \frac{i}{c} D \partial_t \right) \psi = \kappa \psi~
\]
其中 \(\kappa\) 待定。再将矩阵算符作用于方程两边,得到:
\[
\left( \nabla^2 - \frac{1}{c^2} \partial_t^2 \right) \psi = \kappa^2 \psi~
\]
令 \(\kappa = \frac{mc}{\hbar}\) 可得,波函数的所有分量都单独满足相对论的能量-动量关系。因此,寻求的方程,既在空间又在时间上是一阶的,最终是:
\[
\left( A \partial_x + B \partial_y + C \partial_z + \frac{i}{c} D \partial_t - \frac{mc}{\hbar} \right) \psi = 0~
\]
设:
\[
A = i \beta \alpha_1, \quad B = i \beta \alpha_2, \quad C = i \beta \alpha_3, \quad D = \beta~
\]
由于 \(D^2 = \beta^2 = I_4\),最终得到了上面写出的狄拉克方程。
\subsubsection{协变形式与相对论不变性}

为了证明方程的相对论不变性,将其转换成空间和时间导数处于平等地位的形式是有利的。为此,引入了新的矩阵,如下所示:
\[
D = \gamma^0, \quad A = i\gamma^1, \quad B = i\gamma^2, \quad C = i\gamma^3~
\]
此时方程的形式为(记住协变四梯度的定义,特别是 \(\partial_0 = \frac{1}{c} \partial_t\)):
\[
(i\hbar \gamma^\mu \partial_\mu - mc) \psi = 0~
\]
其中,\(\mu = 0, 1, 2, 3\) 是两次重复的指标,\(\partial_\mu\) 是四梯度。在实践中,常常将伽马矩阵表示为泡利矩阵和 2×2 单位矩阵的子矩阵。显式地,标准表示为:
\[
\gamma^0 = \begin{pmatrix} I_2 & 0 \\ 0 & -I_2 \end{pmatrix}, \quad 
\gamma^1 = \begin{pmatrix} 0 & \sigma_x \\ -\sigma_x & 0 \end{pmatrix}, \quad 
\gamma^2 = \begin{pmatrix} 0 & \sigma_y \\ -\sigma_y & 0 \end{pmatrix}, \quad 
\gamma^3 = \begin{pmatrix} 0 & \sigma_z \\ -\sigma_z & 0 \end{pmatrix}~
\]
整个系统使用时空上的闵可夫斯基度量(Minkowski metric)总结为:
\[
\{ \gamma^\mu, \gamma^\nu \} = 2 \eta^{\mu \nu} I_4~
\]
其中,括号表达式 \(\{ a, b \} = ab + ba\) 表示反对易子。这个关系是伪正交四维空间上克利福德代数(Clifford algebra)的定义关系,具有度量签名(+ − − −)。在狄拉克方程中使用的克利福德代数今天被称为狄拉克代数。尽管狄拉克在方程提出时并没有将其视为如此,但事后来看,这种几何代数的引入代表了量子理论发展的巨大进步。

现在,狄拉克方程可以被解释为特征值方程,其中静质量与四动量算符的特征值成正比,比例常数为光速:
\[
\operatorname{P}_{\mathsf{op}} \psi = m c \psi~
\]
使用 Feynman 划线符号定义的 \(\partial/\)(即 \(\partial/\mathrel{\stackrel{\mathrm{def}}{=}} \gamma^\mu \partial_\mu\))后,狄拉克方程变为:
\[
i\hbar \partial/\psi - mc \psi = 0~
\]
这里的 \(\partial/\) 被称为“d-斜线”,按照费曼斜线记法。

在实际应用中,物理学家常常使用自然单位,使得 \(\hbar = c = 1\)。此时,方程变为简单的形式:

\textbf{狄拉克方程(自然单位)}
\[
(i \partial / - m) \psi = 0~
\]
一个基础定理[哪个?]表明,如果给定两个不同的矩阵集合,它们都满足Clifford关系,那么它们通过相似变换相互连接:
\[
\gamma^{\mu \prime} = S^{-1} \gamma^{\mu} S~
\]
如果这些矩阵都是单位矩阵,正如狄拉克矩阵集所示,那么 \( S \) 本身也是单位的:
\[
\gamma^{\mu \prime} = U^{\dagger} \gamma^{\mu} U~
\]
变换 \( U \) 是唯一的,只会有一个绝对值为1的乘法因子。现在,假设空间和时间坐标以及导数算符(它们形成一个协变矢量)已经经历了洛伦兹变换。为了使算符 \(\gamma^{\mu} \partial_{\mu}\) 不变,\(\gamma\) 矩阵必须相对于它们的时空指标作为一个逆变矢量进行变换。由于洛伦兹变换的正交性,这些新的 \(\gamma\) 矩阵仍然会满足Clifford关系。根据之前提到的基础定理[哪个?],可以通过单位变换将新的矩阵集替换为旧的矩阵集。在新的参考系中,记住休止质量是一个相对论标量,狄拉克方程将呈现以下形式:
\[
\left( iU^{\dagger} \gamma^{\mu} U \partial_{\mu}^{\prime} - m \right) \psi (x^{\prime}, t^{\prime}) = 0~
\]
\[
U^{\dagger} (i \gamma^{\mu} \partial_{\mu}^{\prime} - m) U \psi (x^{\prime}, t^{\prime}) = 0~
\]
如果定义变换后的自旋量子比为
\[
\psi^{\prime} = U \psi~
\]
则变换后的狄拉克方程表现出显式的相对论不变性:
\[
\left( i \gamma^{\mu} \partial_{\mu}^{\prime} - m \right) \psi^{\prime} (x^{\prime}, t^{\prime}) = 0~
\]
因此,选择 \(\gamma\) 矩阵的任何单位表示都是最终确定的,前提是自旋量子比根据与给定洛伦兹变换相对应的单位变换进行变换。

不同的狄拉克矩阵表示将聚焦于狄拉克波函数中的特定物理内容。这里展示的表示法称为标准表示法——在此表示法中,波函数的前两个分量在低能量和小速度的极限下,变为保利自旋波函数。

以上的考虑揭示了 \(\gamma\) 矩阵在几何中的起源,回溯到格拉斯曼最初的动机;它们代表了时空中的一个固定单位向量基。同样,像 \(\gamma^{\mu} \gamma^{\nu}\) 这样的矩阵积代表了有向的面元,等等。考虑到这一点,我们可以找到时空中的单位体积元素的 \(\gamma\) 矩阵形式,如下所示。根据定义,它是:
\[
V = \frac{1}{4!} \epsilon_{\mu \nu \alpha \beta} \gamma^{\mu} \gamma^{\nu} \gamma^{\alpha} \gamma^{\beta}~
\]
为了保持不变性,\(\epsilon\) 符号必须是一个张量,因此它必须包含一个 \(\sqrt{g}\) 的因子,其中 \(g\) 是度规张量的行列式。由于这个因子是负的,因此该因子是虚数。因此:
\[
V = i \gamma^{0} \gamma^{1} \gamma^{2} \gamma^{3}~
\]
这个矩阵被赋予了特殊的符号 \(\gamma_5\),因为它在考虑时空不正当变换时(即那些改变基向量方向的变换)非常重要。在标准表示法中,它是:
\[
\gamma_5 = \begin{pmatrix} 0 & I_2 \\ I_2 & 0 \end{pmatrix}~
\]
这个矩阵还会与其他四个狄拉克矩阵反交换:
\[
\gamma_5 \gamma^{\mu} + \gamma^{\mu} \gamma_5 = 0~
\]
当涉及到宇称问题时,\(\gamma_5\) 作为一个有向大小的体积元素起到重要作用,因为它在时空反射下会改变符号。因此,上面取正平方根相当于在时空上选择一种手性约定。
\subsection{与相关理论的比较}
\subsubsection{保利理论}
引入半整数自旋的必要性可以追溯到斯特恩-格尔拉赫实验的实验结果。在实验中,一束原子通过一个强的不均匀磁场,这束原子会根据原子的内在角动量分裂成N个部分。研究发现,对于银原子,光束分裂成两部分;因此,基态不可能是整数的,因为即使原子的内在角动量最小为1,光束也应该分裂成三部分,分别对应于 \(L_z = -1, 0, +1\) 的原子。结论是,银原子的内在角动量为 \( \frac{1}{2} \)。保利建立了一个理论,通过引入一个二分量波函数,并在哈密顿量中加入相应的修正项,解释了这种分裂现象,该修正项表示波函数与施加的磁场之间的半经典耦合。其在国际单位制中的形式为:
\[
H = \frac{1}{2m} \left( \boldsymbol{\sigma} \cdot (\mathbf{p} - e\mathbf{A}) \right)^2 + e \phi~
\]
这里,\(\mathbf{A}\) 和 \(\phi\) 分别表示电磁四势的分量(标准国际单位制),三个 \(\sigma\) 是保利矩阵。展开第一项后,会发现一个与磁场的残余相互作用,以及经典哈密顿量的常见项,描述带电粒子与施加磁场的相互作用:
\[
H = \frac{1}{2m} (\mathbf{p} - e \mathbf{A})^2 + e \phi - \frac{e \hbar}{2m} \boldsymbol{\sigma} \cdot \mathbf{B}~
\]
这个哈密顿量是一个 \(2 \times 2\) 矩阵,因此基于它的薛定谔方程必须使用二分量波函数。在类似地引入外部电磁四势到狄拉克方程中,称为最小耦合,方程形式为:
\[
\left( \gamma^\mu \left( i\hbar \partial_\mu - e A_\mu \right) - mc \right) \psi = 0~
\]
再应用一次狄拉克算符,将会精确地再现保利项,因为空间的狄拉克矩阵乘以 \(i\) 后,具有与保利矩阵相同的平方和对易性质。而且,电子的旋磁比值,位于保利新项前面,可以从第一原理推导出来。这是狄拉克方程的一个重大成就,并且让物理学家对它的整体正确性充满信心。

然而,还有更多内容。保利理论可以被看作是狄拉克理论在低能量极限下的表现。首先,方程可以写成如下形式,表示二自旋子耦合的方程(恢复国际单位制):
\[
\begin{pmatrix} m c^2 - E + e \phi & + c \boldsymbol{\sigma} \cdot (\mathbf{p} - e \mathbf{A}) \\
- c \boldsymbol{\sigma} \cdot (\mathbf{p} - e \mathbf{A}) & m c^2 + E - e \phi \end{pmatrix}
\begin{pmatrix} \psi_+ \\ \psi_- \end{pmatrix}
= \begin{pmatrix} 0 \\ 0 \end{pmatrix}~
\]
因此:
\[
\begin{aligned}
(E - e \phi) \psi_+ - c \boldsymbol{\sigma} \cdot (\mathbf{p} - e \mathbf{A}) \psi_- &= m c^2 \psi_+ \\
c \boldsymbol{\sigma} \cdot (\mathbf{p} - e \mathbf{A}) \psi_+ - (E - e \phi) \psi_- &= m c^2 \psi_-
\end{aligned}~
\]
假设磁场较弱,并且电子的运动是非相对论性的,电子的总能量大致等于其静止能量,动量趋近于经典值:
\[
E - e \phi \approx m c^2, \quad \mathbf{p} \approx m \mathbf{v}~
\]
因此,第二个方程可以写成:
\[
\psi_- \approx \frac{1}{2m c} \boldsymbol{\sigma} \cdot (\mathbf{p} - e \mathbf{A}) \psi_+~
\]
该表达式是按 \( \frac{v}{c} \) 级别来估计的。这样,在典型的能量和速度下,标准表示中的狄拉克自旋子的底部分量相较于顶部分量被极大抑制。将此表达式代入第一个方程,通过一些整理后,得到:
\[
(E - m c^2) \psi_+ = \frac{1}{2m} \left[ \boldsymbol{\sigma} \cdot (\mathbf{p} - e \mathbf{A}) \right]^2 \psi_+ + e \phi \psi_+~
\]
左侧的算符表示粒子的总能量减去静止能量,即其经典动能,因此可以通过识别保利的二自旋子与狄拉克自旋子的顶部分量来恢复保利理论,前提是采用非相对论近似。进一步近似给出保利理论的薛定谔方程极限。

因此,薛定谔方程可以被视为狄拉克方程的非相对论极限,当忽略自旋并仅在低能量和低速度下工作时。这也是新方程的一个重大胜利,因为它追溯了方程中出现的神秘 \(i\) 以及复杂波函数的必要性,通过狄拉克代数联系到时空的几何。这还凸显了为什么薛定谔方程,尽管表面上看起来像是扩散方程,实际上代表的是波动传播。

需要强调的是,整个狄拉克自旋子表示的是一个不可约的整体。此处所做的将狄拉克自旋子分为大分量和小分量的做法依赖于低能量近似的有效性。上文中忽略的分量是必要的,它们产生了在相对论性范围内观察到的新现象——其中包括反物质以及粒子的创生与湮灭。
\subsubsection{韦尔理论}
在无质量的情况下 (\(m = 0\)),狄拉克方程简化为韦尔方程,描述了相对论性无质量的自旋-1/2 粒子。

该理论获得了第二个 \( U(1) \) 对称性:见下文。
\subsection{物理解释}  
\subsubsection{可观察量的识别}
量子理论中的一个关键物理问题是:该理论所定义的物理可观察量是什么?根据量子力学的公设,这些量是由自伴算符定义的,这些算符作用于系统可能状态的希尔伯特空间。这些算符的本征值则是测量相应物理量时可能得到的结果。

在薛定谔理论中,最简单的此类对象是整体哈密顿量,它代表系统的总能量。为了在狄拉克理论中保持这种解释,哈密顿量必须被取为:
\[
H = \gamma^{0}\left[ mc^{2} + c \gamma^{k}\left(p_k - qA_k\right)\right] + cqA^{0}~
\]
其中,像往常一样,有一个隐含的对重复两次的索引 \( k = 1, 2, 3 \) 求和。这个式子看起来很有希望,因为通过检查可以看到粒子的静能,并且在 \( A = 0 \) 的情况下,能量就是一个电荷在电势 \( cqA^{0} \) 中的能量。那么,涉及向量势的项呢?在经典电动力学中,一个电荷在施加电势中运动的能量是:
\[
H = c\sqrt{\left(\mathbf{p} - q\mathbf{A}\right)^{2} + m^{2}c^{2}} + qA^{0}~
\]
因此,狄拉克哈密顿量与其经典对应物在本质上有所区别,在识别此理论中的可观察量时必须小心。狄拉克方程所暗示的许多看似悖论的行为,其实是由于对这些可观察量的误识别所致。
\subsubsection{孔理论}
方程的负能量解存在问题,因为假设粒子具有正能量。然而,从数学角度来看,我们似乎没有理由排除负能量解。由于负能量解是存在的,它们不能简单地被忽视,因为一旦考虑到电子与电磁场之间的相互作用,任何处于正能量本征态的电子都会衰变到负能量本征态,并且能量会逐渐降低。显然,真实的电子并不以这种方式表现,否则它们会通过辐射光子而消失。

为了解决这个问题,狄拉克提出了孔理论的假设,认为真空是一个多体量子态,其中所有负能量的电子本征态都已被占据。这种将真空描述为电子“海洋”的观点被称为狄拉克海。由于泡利不相容原理禁止电子占据相同的态,因此任何额外的电子都会被迫占据正能量本征态,而正能量电子则不可能衰变为负能量本征态。

狄拉克进一步推测,如果负能量本征态没有完全填满,每一个未被占据的本征态——被称为孔——会像带正电的粒子一样表现。孔具有正能量,因为从真空中创造一个粒子-孔对需要能量。如上所述,狄拉克最初认为孔可能是质子,但赫尔曼·外尔指出,孔应该表现得像具有与电子相同质量的粒子,而质子则重约电子的1800倍。最终,孔被识别为正电子,正电子在1932年由卡尔·安德森实验发现。[12]

然而,使用无限的负能量电子海洋来描述“真空”并不完全令人满意。来自负能量电子海洋的无限负贡献必须通过一个无限大的正“裸”能量来抵消,同时,来自负能量电子海洋的电荷密度和电流的贡献也被一个无限大的正“果冻”背景完全抵消,以使得真空的净电荷密度为零。在量子场论中,布戈柳博夫变换(将占据的负能量电子态转变为未占据的正能量正电子态,而将未占据的负能量电子态转变为占据的正能量正电子态)使得我们能够绕过狄拉克海的形式主义,尽管从形式上讲,它与狄拉克海是等价的。

然而,在凝聚态物理的某些应用中,“孔理论”的基本概念是有效的。电导体中的导电电子海洋,称为费米海,包含能量直到系统的化学势的电子。费米海中的未填充态表现得像带正电的电子,虽然它也被称为“电子孔”,但它与正电子是不同的。费米海的负电荷通过材料的带正电的离子晶格得以平衡。
\subsubsection{在量子场论中的应用}
在量子场论中,例如量子电动力学,狄拉克场会经过二次量化的过程,这解决了方程中的一些悖论特性。
\subsection{数学表述}
在场论的现代表述中,狄拉克方程被写为一个狄拉克自旋场 \( \psi \) 的方程,该场取值于一个复向量空间 \( \mathbb{C}^4 \),并定义在平坦时空(闵可夫斯基空间) \( \mathbb{R}^{1,3} \) 上。其表达式还包含伽马矩阵和一个参数 \( m > 0 \),该参数被解释为质量,以及其他物理常数。狄拉克最初通过对爱因斯坦的能量-动量-质量等价关系进行因式分解得到他的方程,假设动量向量的标量积由度量张量确定,并通过将动量与相应的算符关联来量子化得到的关系。

用场 \( \psi : \mathbb{R}^{1,3} \rightarrow \mathbb{C}^4 \) 表示,狄拉克方程可以写为:

\textbf{狄拉克方程}  
\[
(i\hbar \gamma^\mu \partial_\mu - mc)\psi(x) = 0~
\]
在自然单位制下,使用费曼斜体符号,狄拉克方程为:

\textbf{狄拉克方程(自然单位制)}  
\[
(i \partial \!\!\!/- m)\psi(x) = 0~
\]
伽玛矩阵是一组四个 \( 4 \times 4 \) 复数矩阵(属于 \( \text{Mat}_{4 \times 4}(\mathbb{C}) \)),它们满足以下反对易关系:
\[
\{\gamma^\mu, \gamma^\nu\} = 2 \eta^{\mu \nu} I_4~
\]
其中 \( \eta^{\mu \nu} \) 是闵可夫斯基度规元素,索引 \( \mu, \nu \) 取值为 0, 1, 2, 和 3。这些矩阵可以在选择不同表示下显式实现。常见的两种表示是狄拉克表示和手征表示。

在狄拉克表示下,
\[
\gamma^0 = \begin{pmatrix} I_2 & 0 \\ 0 & -I_2 \end{pmatrix}, \quad \gamma^i = \begin{pmatrix} 0 & \sigma^i \\ -\sigma^i & 0 \end{pmatrix}~
\]
其中 \( \sigma^i \) 是泡利矩阵。

在手征表示下,\( \gamma^i \) 与狄拉克表示相同,但\(\gamma^0 = \begin{pmatrix} 0 & I_2 \\ I_2 & 0 \end{pmatrix}\)

斜体符号是一种紧凑的表示方法,
\[
A\!\!\!/ := \gamma^\mu A_\mu~
\]
其中 \( A \) 是四矢量(通常是四维微分算符 \( \partial_\mu \))。对索引 \( \mu \) 进行求和是隐含的。

或者,构成波函数的四个量的四个耦合一阶偏微分方程可以写成一个向量。在普朗克单位下,方程变为:
\[
i \partial_x \begin{bmatrix} +\psi_4 \\ +\psi_3 \\ -\psi_2 \\ -\psi_1 \end{bmatrix}
+ \partial_y \begin{bmatrix} +\psi_4 \\ -\psi_3 \\ -\psi_2 \\ +\psi_1 \end{bmatrix}
+ i \partial_z \begin{bmatrix} +\psi_3 \\ -\psi_4 \\ -\psi_1 \\ +\psi_2 \end{bmatrix}
- m \begin{bmatrix} +\psi_1 \\ +\psi_2 \\ +\psi_3 \\ +\psi_4 \end{bmatrix}
= i \partial_t \begin{bmatrix} -\psi_1 \\ -\psi_2 \\ +\psi_3 \\ +\psi_4 \end{bmatrix}~
\]
这使得它更清晰地展示为四个偏微分方程组,包含四个未知函数。(注意 \(\partial_y \) 项前没有 \( i \),因为 \( \sigma_y \) 是虚数。)
\subsubsection{狄拉克共轭和共轭方程}

狄拉克旋量场 \( \psi(x) \) 的共轭旋量定义为:
\[
\bar{\psi}(x) = \psi(x)^\dagger \gamma^0~
\]
利用伽玛矩阵的性质(这直接来自于 \( \gamma^\mu \) 的厄米性),即:
\[
(\gamma^\mu)^\dagger = \gamma^0 \gamma^\mu \gamma^0~
\]
可以通过对狄拉克方程取厄米共轭并右乘 \( \gamma^0 \) 来推导出共轭狄拉克方程:
\[
\bar{\psi}(x)(-i \gamma^\mu {\overleftarrow \partial}_\mu - m) = 0~
\]
其中,偏导数 \( {\overleftarrow \partial}_\mu \) 从右边作用于 \( \bar{\psi}(x) \)。用常规的左作用表示偏导数时,我们得到:
\[
-i \partial_\mu \bar{\psi}(x) \gamma^\mu - m \bar{\psi}(x) = 0~
\]
\subsubsection{克莱因–戈尔登方程}
将 \( i\partial \!\!\!/+m \) 作用于狄拉克方程得到:
\[
(\partial_\mu \partial^\mu + m^2)\psi(x) = 0~
\]
也就是说,狄拉克旋量场的每个分量都满足克莱因–戈尔登方程。
\subsubsection{守恒电流}
该理论的守恒电流为:
\[
J^\mu = \bar{\psi} \gamma^\mu \psi~
\]
\textbf{从迪拉克方程的守恒性证明}

将迪拉克方程和伴随迪拉克方程相加得到:
\[
i\left((\partial_{\mu} \bar{\psi}) \gamma^{\mu} \psi + \bar{\psi} \gamma^{\mu} \partial_{\mu} \psi \right) = 0~
\]
因此,应用莱布尼茨法则,有:
\[
i \partial_{\mu} (\bar{\psi} \gamma^{\mu} \psi) = 0~
\]
另一种推导这个表达式的方法是通过变分法,应用诺特定理(Noether's theorem)来推导全局 \( U(1) \) 对称性下的守恒电流 \( J^{\mu} \)。



















































































































































































































































































































































































































































































































































































































































