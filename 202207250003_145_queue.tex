% 队列
% keys 队列|数据结构|C++

队列是一种 “先进先出” 的数据结构.队列在左端(队头)弹出元素,在右端(队尾)插入元素.C++ STL 中的 $\mathtt{queue}$ 实现了队列.队列通常是实现广度优先搜索(BFS)的数据结构.队列还有多种变体,如两端都能插入和弹出元素的\textbf{双端队列}($\text{C++ STL deque}$),还有给元素赋予优先级,具有最高优先级的元素最先弹出,等价于一个二叉堆的\textbf{优先队列}($\text{C++ STL priority_queue}$).

队列通常有四种基本操作:

\begin{enumerate}
\item 向队尾插入一个数 $x$;
\item 从队头弹出一个数;
\item 判断队列是否为空;
\item 查询队头元素.
\end{enumerate}

C++ STL:

\begin{lstlisting}[language=cpp]
queue<int> q;
int hh = 0, tt = -1;  // hh 为队头,tt 为队尾

q.push(x);  // 在队尾插入新元素
q.pop();    // 删除队列首元素但不返回其值
q.empty();  // 如果队列为空返回 true,否则返回 false
q.front();  // 返回队首元素的值,但不删除该元素

// 还有如下几个操作
q.back();   // 返回队列尾元素的值,但不删除该元素
q.size();   // 返回队列中元素的个数

\end{lstlisting}

我们这里也是着重讲一下如何用数组实现队列.

定义一个数组 $q$