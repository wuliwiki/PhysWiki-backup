% 博苏克-乌拉姆定理(综述)
% license CCBYSA3
% type Wiki

本文根据 CC-BY-SA 协议转载翻译自维基百科\href{https://en.wikipedia.org/wiki/Borsuk\%E2\%80\%93Ulam_theorem}{相关文章}。

在数学中,博苏克–乌拉姆定理指出:每一个从 $n$ 维球面 $S^n$ 到 $n$ 维欧几里得空间 $\mathbb{R}^n$ 的连续函数,必定存在一对对踵点被映射到同一个点。这里,对踵点指的是位于球面上、从球心看方向完全相反的两点。

形式化表述:若$f: S^n \to \mathbb{R}^n$是一个连续函数,则存在$x \in S^n$使得$f(-x) = f(x)$。

特例说明:当 $ n = 1$:这意味着地球赤道上总存在一对正对着的点,它们的温度相同。这个结论也适用于任何圆周。注意这依赖于温度在空间中连续变化这一假设,而这在现实中不一定成立\(^\text{[1]}\)。

当 $ n = 2 $:可以解释为地球表面在任意时刻总存在一对对踵点,它们的温度和气压完全相同(假设这两个物理量都在空间中连续变化)。

与奇函数等价的其他表述:记$ S^n $为 $n$ 维球面,$ B^n $为 $n$维单位球体:
\begin{itemize}
\item 若  $g: S^n \to \mathbb{R}^n$是一个连续奇函数(即 $g(-x) = -g(x)$),则存在$x \in S^n$使得$g(x) = 0$。
\item 若 $g: B^n \to \mathbb{R}^n$是一个连续函数,且在边界 $S^{n-1}$ 上为奇函数,那么必存在$x \in B^n$使得$g(x) = 0$。
\end{itemize}
\subsection{历史}
据 Matoušek(2003年第25页)记载,博苏克–乌拉姆定理最早的历史性表述出现在 Lyusternik 与 Shnirel'man 于1930年的著作中。第一个正式的证明由卡罗尔·博苏克于1933年给出,他将这一问题的提出归功于斯坦尼斯瓦夫·乌拉姆。自那以后,许多作者都给出了该定理的不同证明方式,这些证明被 Steinlein(1985)汇总整理。
\subsection{等价表述}
以下命题与博苏克–乌拉姆定理是等价的\(^\text{[2]}\):
\subsubsection{关于奇函数的表述}
一个函数 $g$ 被称为奇函数(也称为对极函数或保对极函数),如果对于每一个 $x$,都有
$g(-x) = -g(x)$

博苏克–乌拉姆定理等价于以下两个命题:

1. 每一个连续的奇函数 $g : S^n \to \mathbb{R}^n$ 都有零点。\\
2. 不存在从 $S^n$ 到 $S^{n-1}$ 的连续奇函数。\\

证明:博苏克–乌拉姆定理等价于命题 (1)

($\Longrightarrow$) 如果博苏克–乌拉姆定理成立,那么它当然对奇函数也成立。而对于奇函数 $g$,若有$g(-x) = g(x)$则必须有$g(x) = 0$,因为又有 $g(-x) = -g(x)$。所以每个连续奇函数都有零点。

($\Longleftarrow$) 对于任意连续函数 $f : S^n \to \mathbb{R}^n$,可以构造一个连续奇函数$g(x) = f(x) - f(-x)$如果每个奇函数都有零点,那么 $g$ 存在某个零点 $x$,即$f(x) = f(-x)$,从而得出博苏克–乌拉姆定理成立。

为了证明命题 (1) 与命题 (2) 等价,使用以下连续奇函数:

\begin{itemize}
\item 明显的包含映射$i : S^{n-1} \to \mathbb{R}^n \setminus \{0\}$
\item 辐射投影映射$p : \mathbb{R}^n \setminus \{0\} \to S^{n-1}, \quad x \mapsto \frac{x}{|x|}$
\end{itemize}
证明如下:

((1) ⟹ (2))反设成立:若存在连续奇函数 $f : S^n \to S^{n-1}$,则复合映射 $i \circ f : S^n \to \mathbb{R}^n \setminus \{0\}$ 是一个连续奇函数,这与 (1) 每个连续奇函数都有零点相矛盾(因为值域不含零点)。

((1) ⟸ (2))同样使用反设:若存在连续奇函数 $f : S^n \to \mathbb{R}^n \setminus \{0\}$则 $p \circ f : S^n \to S^{n-1}$ 是连续奇函数,这与 (2) 不存在这样的映射相矛盾。因此,两者等价。
