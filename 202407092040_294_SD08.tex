% 苏州大学 2008 年硕士考试试题
% keys 苏州大学|考研|物理|2008年
% license Copy
% type Tutor

\begin{enumerate}
\item 有一同轴电缆,其长$L=1.50x10^3m$,内导体外径$R_1=1.0mm$,外导体内径$R_2=5.0mm$中间填充绝缘介质。由于受潮,测得绝缘介质的电阻率降低到$6.4*10^5\Omega .m$。若信号源是电动势$\varepsilon=24V$,内阻$R_1=3.0\Omega$的直流电源,求:\\
(1)同轴电缆的径向电电阻;\\
(2)在电缆末端的负载电阻$R_0=1.0k \Omega$上的信号电压。
\item 圆形线圈a由50匝细线绕成,横截面积为$4.0cm^2$,放在另一个半径为 20cm,匝数为100匝的另一圆形线圈b的中心,两线圈同轴共面。求:\\
(1)两线圈的互感系数;\\
(2)当线圈b中的电流以 50A/s的变化率减少时,线圈a内磁通量的变化率。\\
(3)线圈a中的感生电动势的大小。
\item 两无限长带异号电荷的同轴圆柱面,单位长度上的电量为$3.0*10^{-8}C/m$,内圆柱面半径为$2*10^{-2}m$,外圆面半径为$4*10^{-2}m$\\
(1)用高斯定理求内圆柱面内、两圆柱面间和外圆柱面外的电场强度;\\
(2)若一电子在两圆柱面之间垂直于轴线的平面内沿半径$3x10^{-2}m$的圆周匀速旋转,问此电子的动能为多少?
\item 某边长为20cm 的方形导体回路放在一圆形均匀磁场的正中,初始时磁场B为 0.5T,垂直圆面向内,现磁场B以0.1T/s减少,求:\\
(1)a、b、c三点感应电场$E_r$的大小,并标明方向;
(2)导体回路中的电流,已知该回路的电阻为2$\Omega$;\\
(3)点a和b之间的电势差。
\item 某油轮失事的海域,大量泄的石油(n=1.2)形成了一个很大的油膜。试求:\\
(1)如果从飞机上竖直向下看油膜,只监测到波长为 552nm 的蓝绿光,试求油膜的可能厚度?\\
(2)如果潜水员从水下竖直地向上看这油膜的厚度为460nm 区域,可看见哪些波长的可见光透射最强?(水的折射率为1.33)
\item 迈克耳逊干涉仪中一臂(反射镜),以速度v速推移,用透镜接收干涉条纹,将它会聚到光电元件上,把光强变化为电讯号。\\
(1)若测得电讯号强度变化的时间频率为v,求入射光的波长$\lambda$;\\
(2)若入射光波长为40$\mu$m,要使电讯号频率控制在100Hz,反射镜平移的速度应为多少?
\item 
\end{enumerate}