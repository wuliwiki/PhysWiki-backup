% 编程语言理论(综述)
% license CCBYSA3
% type Wiki

本文根据 CC-BY-SA 协议转载翻译自维基百科\href{https://en.wikipedia.org/wiki/Programming_language_theory}{相关文章}。

编程语言理论(PLT)是计算机科学的一个分支,研究编程语言这一形式语言的设计、实现、分析、特征描述和分类。编程语言理论与数学、软件工程和语言学等领域密切相关。该领域有许多学术会议和期刊。
\begin{figure}[ht]
\centering
\includegraphics[width=6cm]{./figures/7180d92ac9151835.png}
\caption{小写希腊字母 λ(lambda)是编程语言理论领域的一个非官方符号。[citation needed] 这一用法源自 λ 演算,这是一种由阿隆佐·丘奇(Alonzo Church)在1930年代提出的计算模型,并被编程语言研究人员广泛使用。它出现在经典著作《计算机程序的构造与解释》的封面上,以及1975至1980年间由杰拉尔德·杰伊·萨斯曼(Gerald Jay Sussman)和盖伊·斯蒂尔(Guy Steele)编写的《Lambda Papers》一书的标题中,后者是 Scheme 编程语言的开发者。[行话]} \label{fig_BCYY_1}
\end{figure}
\subsection{历史}  
在某些方面,编程语言理论的历史甚至早于编程语言本身的发展。λ 演算(lambda calculus)由阿隆佐·丘奇(Alonzo Church)和斯蒂芬·科尔·克莱尼(Stephen Cole Kleene)在1930年代提出,被一些人认为是世界上第一个编程语言,尽管它的初衷是用来建模计算,而不是作为程序员向计算机系统描述算法的工具。许多现代的函数式编程语言被描述为在 λ 演算之上提供了一个“薄薄的表面层”,[2] 并且许多语言可以通过 λ 演算来简单地描述。

第一个发明的编程语言是 Plankalkül,它由康拉德·祖泽(Konrad Zuse)在1940年代设计,但直到1972年才为公众所知(并且直到1998年才实现)。第一个广为人知且成功的高级编程语言是 FORTRAN(表示“公式翻译”),由IBM研究团队在1954至1957年间开发,约翰·巴克斯(John Backus)领导。FORTRAN的成功促成了一个科学家委员会的形成,旨在开发一种“通用”计算机语言;他们的努力结果是ALGOL 58。与此同时,麻省理工学院的约翰·麦卡锡(John McCarthy)开发了 Lisp,它是首个源自学术界且成功的编程语言。随着这些初步努力的成功,编程语言在1960年代及以后成为了一个活跃的研究主题。