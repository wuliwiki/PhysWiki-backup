% 布莱斯·帕斯卡(综述)
% license CCBYSA3
% type Wiki

本文根据 CC-BY-SA 协议转载翻译自维基百科\href{https://en.wikipedia.org/wiki/Blaise_Pascal}{相关文章}。

布莱兹·帕斯卡(Blaise Pascal,\(^\text{[a]}\) 1623年6月19日-1662年8月19日)是一位法国数学家、物理学家、发明家、哲学家及天主教作家。

帕斯卡是神童,由担任鲁昂税务官的父亲亲自教育。他最早的数学研究是投影几何,16岁时便撰写了一篇重要的关于圆锥曲线的论文。后来,他与皮埃尔·费马通信探讨概率论,对现代经济学与社会科学的发展产生了深远影响。1642年,他开始从事计算机的先驱性研究,发明了后来被称为“帕斯卡计算器”或“帕斯卡机”的装置,使他成为机械计算器的最早两位发明人之一\(^\text{[6][7]}\)。

与同时代的勒内·笛卡尔一样,帕斯卡也是自然科学和应用科学的先驱。他撰文为科学方法辩护,并提出了若干颇具争议的研究成果。他在流体研究方面作出了重要贡献,推广伊万杰利斯塔·托里拆利的研究成果,澄清了压力和真空的概念。国际单位制中压力单位“帕斯卡”正是以他命名的。1647年,他继托里拆利和伽利略之后,驳斥了亚里士多德与笛卡尔等人所持的“自然界厌恶真空”的观点。

他也被誉为现代公共交通的发明者,因为他在1662年去世前不久创立了“五苏之马车”,这是历史上第一种现代公共交通服务\(^\text{[8]}\)。

1646年,他与妹妹雅克琳一同接受了天主教内部一个被批评者称为詹森主义的宗教运动\(^\text{[9]}\)。1654年末经历一次宗教体验后,他开始撰写有深远影响的哲学与神学作品。他最著名的两部著作都诞生于这一时期:《省函集》和《思想录》。《省函集》以詹森主义者与耶稣会士之间的冲突为背景;而《思想录》中包含了著名的“帕斯卡赌注”,原名为《论机器的演说》\(^\text{[10][11]}\),这是一个以信仰主义为基础、具有概率论性质的论证,主张人应当相信上帝的存在。同年,他还撰写了一部关于“算术三角形”的重要论文。1658至1659年间,他又研究了摆线及其在求解立体体积中的应用。在多年疾病折磨之后,帕斯卡于39岁时在巴黎去世。
\subsection{早年生活与教育}
帕斯卡出生于法国奥弗涅地区的克莱蒙费朗,地处中央高原。他在三岁时失去了母亲安托瓦内特·贝贡。\(^\text{[12]}\)他的父亲艾蒂安·帕斯卡也是一位业余数学家,是当地的法官,同时是“法袍贵族”成员。帕斯卡有两个姐妹,妹妹叫雅克琳,姐姐叫吉尔贝特。
\begin{figure}[ht]
\centering
\includegraphics[width=6cm]{./figures/57329a3ddbf9ceaa.png}
\caption{} \label{fig_BLSpsk_1}
\end{figure}
迁居巴黎
1631年,也就是妻子去世五年后,\(^\text{[13]}\)艾蒂安·帕斯卡带着孩子们搬到了巴黎。这户新到的家庭很快雇佣了女仆路易丝·德福,后者最终成为了这个家庭的重要成员。艾蒂安终身未再婚,决定亲自教育自己的孩子们。

年幼的帕斯卡展现出非凡的智力,特别是在数学和科学方面展现出惊人的天赋。\(^\text{[14]}\)艾蒂安原本试图阻止儿子接触数学;然而在12岁时,帕斯卡凭借自己的努力,用木炭在瓷砖地板上重新推导出了欧几里得的前32条几何命题,因此他得到了《几何原本》的一本副本。\(^\text{[15]}\)

\textbf{关于圆锥曲线的论文}

帕斯卡尤其感兴趣的一本著作是德扎格关于圆锥曲线的研究。沿着德扎格的思路,年仅16岁的帕斯卡撰写了一篇短小的论文《圆锥曲线试作》(法语原名 Essai pour les coniques),用以证明一个被称为“神秘六边形”的命题,并将这篇他人生中第一篇严肃的数学论文寄给了巴黎的梅尔森神父。这一定理今天仍以“帕斯卡定理”之名为人熟知:它断言,若一个六边形内接于一个圆或一般的圆锥曲线中,则其对边延长线的交点三三成对后共线,这条直线称为“帕斯卡线”。

帕斯卡的工作极其早熟,以至于笛卡尔一度坚信是帕斯卡的父亲写下了这篇文章。当梅尔森向他确认确实是帕斯卡之子所作时,笛卡尔冷笑一声轻蔑地回应道:“我并不觉得奇怪,他在圆锥曲线方面给出的证明比古人更为得当”,并补充道:“但关于这一主题,还有些内容是一个16岁的孩子无论如何也不可能想到的。”\(^\text{[16]}\)
\subsubsection{离开巴黎}
在当时的法国,官职是可以买卖的。1631年,艾蒂安·帕斯卡以65,665里弗的价格出售了自己在辅助法院担任的二级主席职务\(^\text{[17]}\)。这笔资金被投资于一项政府债券,虽然不能说奢华,但足以为帕斯卡一家在巴黎提供一种安稳的生活。然而到了1638年,红衣主教黎塞留为了筹措继续打三十年战争的资金,违约了这批政府债券。艾蒂安·帕斯卡的财富于是骤减,从近66,000里弗跌到了不到7,300里弗。

像许多其他人一样,艾蒂安因反对黎塞留的财政政策最终不得不逃离巴黎,留下他的三个孩子由邻居圣托夫人照料。这位圣托夫人美貌动人,过往风流,却也经营着全法国最闪耀、最具文化气息的沙龙之一。直到某次雅克琳在一场儿童剧中表演出色,而黎塞留正好在场观看,艾蒂安才得以被赦免。不久之后,艾蒂安重新获得了红衣主教的青睐,并于1639年被任命为鲁昂市的国王税务专员——而当时该市的税务记录由于多次民变,已是一片混乱。
\subsubsection{帕斯卡计算器}
\begin{figure}[ht]
\centering
\includegraphics[width=6cm]{./figures/0a529a6a26267e4b.png}
\caption{} \label{fig_BLSpsk_2}
\end{figure}
1642年,为了减轻父亲在税务工作中那无休止的、令人精疲力竭的计算与重新计算(年轻的帕斯卡也参与了这项工作),帕斯卡在尚未满19岁时便设计并制造出一种能进行加法和减法运算的机械计算器,被称为“帕斯卡计算器”或“帕斯卡机”。在现存的八台帕斯卡机中,四台收藏于巴黎的工艺与技术博物馆,另有一台在德国德累斯顿的茨温格宫博物馆展出,均为他原始设计的机械计算器之一\(^\text{[18]}\)。

尽管这些机器是后续400年机械计算技术发展的先驱,从某种意义上说也可被视为计算机工程领域的前身,但这台计算器并未获得商业上的巨大成功。一方面是因为它在实际使用中仍显笨重,更主要的原因可能是其造价极其昂贵,帕斯卡机最终沦为法国乃至欧洲富人手中的玩具与地位象征。帕斯卡在接下来的十年中持续改进其设计,据他自己所述,有大约50台机器是按他的设计制造的\(^\text{[19]}\)。在随后的十年间,他共亲自制造了20台成品机器\(^\text{[20]}\)。
\subsection{数学}
\subsubsection{概率论}
1654年,在朋友谢瓦利耶·德·梅雷的启发下,帕斯卡与皮埃尔·德·费马就赌博问题展开通信讨论,由此诞生了数学概率论的雏形\(^\text{[21]}\)。他们讨论的具体问题是:两位玩家希望提前结束游戏,并希望根据当前局势公平分配赌注,即各自根据继续赢得比赛的概率来分配奖金。通过这场讨论,“期望值”这一概念首次被引入。约翰·罗斯写道:“概率论及其后续的发展改变了我们对不确定性、风险、决策,以及个人与社会影响未来事件走向的能力的看法”\(^\text{[22]}\)。帕斯卡在其《思想录》中使用了一种概率论论证——帕斯卡赌注,来为信仰上帝及过一种有德行的生活提供理由。然而,尽管帕斯卡与费马在概率论领域做出了重要的早期贡献,他们并未将这一理论发展得更为深入。荷兰科学家惠更斯通过阅读帕斯卡与费马的通信了解到这一新兴领域,并撰写了第一本关于概率论的书籍。随后,亚伯拉罕·德·莫阿弗尔和皮埃尔-西蒙·拉普拉斯等人继续推动了这一理论的发展。帕斯卡与费马在概率计算方面的研究,也为莱布尼茨后来的微积分构想奠定了重要基础\(^\text{[23]}\)。
\subsubsection{《算术三角形论》}
\begin{figure}[ht]
\centering
\includegraphics[width=6cm]{./figures/442c1dcc85f412f2.png}
\caption{帕斯卡三角形:每个数字都是其正上方两个数字之和。除了展示二项式系数外,这个三角形还体现了许多数学性质。} \label{fig_BLSpsk_3}
\end{figure}
帕斯卡于1654年撰写的《算术三角形论》,于他去世后在1665年出版,其中描述了一种排列二项式系数的简便表格形式,他称之为“算术三角形”,如今则被称作“帕斯卡三角形”。
该三角形也可以表示为:
\begin{figure}[ht]
\centering
\includegraphics[width=6cm]{./figures/adab679161e80ad4.png}
\caption{} \label{fig_BLSpsk_4}
\end{figure}
他通过递推关系来定义三角形中的数值:设第 $m+1$ 行、第 $n+1$ 列的数为 $t_{mn}$,那么有$t_{mn} = t_{m-1,n} + t_{m,n-1}$其中$m = 0, 1, 2, \ldots \text{ 且 } n = 0, 1, 2, \ldots$边界条件为 $t_{m,-1} = 0, \ t_{-1,n} = 0$,适用于 $m = 1, 2, 3, \ldots$ 和 $n = 1, 2, 3, \ldots$。生成元为 $t_{00} = 1$。帕斯卡最终给出了如下公式的证明:
$$
t_{mn} = \frac{(m+n)(m+n-1)\cdots(m+1)}{n(n-1)\cdots1}~
$$
在这篇论文中,帕斯卡还明确陈述了数学归纳法原理\(^\text{[24]}\)。1654年,他还证明了一个与幂和相关的恒等式(帕斯卡恒等式),该恒等式给出了前 $n$ 个正整数的 $p$ 次幂之和的关系,其中 $p = 0, 1, 2, \ldots, k$\(^\text{[26]}\)。

同年,帕斯卡经历了一次宗教体验,从此几乎放弃了数学方面的工作。
\subsubsection{摆线}
\begin{figure}[ht]
\centering
\includegraphics[width=6cm]{./figures/f9644e4e1481c829.png}
\caption{《帕斯卡研究摆线》,奥古斯丁·帕若作,1785年,卢浮宫藏} \label{fig_BLSpsk_5}
\end{figure}
1658年,帕斯卡在忍受牙痛时开始思考一些关于摆线的问题。他的牙痛随之消失,他将这视为来自上天的指引,决定继续研究。仅仅八天之后,他便完成了一篇论文\(^\text{[27]}\)。为了宣传研究成果,他发起了一项竞赛\(^\text{[28]}\)。

帕斯卡提出了三个与摆线有关的问题,分别关于其重心、面积与体积,优胜者将获得20或40枚西班牙达布隆金币作为奖品。评审包括帕斯卡本人、吉尔·德·罗贝瓦尔和皮埃尔·德·卡尔卡维。两份参赛作品(分别由约翰·沃利斯 John Wallis 和安托万·德·拉卢韦尔 Antoine de Lalouvère 提交)都被判定不合格\(^\text{[29]}\)。在比赛期间,克里斯托弗·雷恩向帕斯卡提交了一个关于摆线求长的证明提案;罗贝瓦尔立刻宣称他多年前就已知道该证明。沃利斯随后在他的《双论集》中发表了雷恩的证明,并明确将首创权归于雷恩。
\subsection{物理学}
\begin{figure}[ht]
\centering
\includegraphics[width=6cm]{./figures/3a06050aaec8b631.png}
\caption{帕斯卡木桶实验(可能是杜撰的)的插图} \label{fig_BLSpsk_6}
\end{figure}
帕斯卡在物理学多个领域作出了贡献,尤其是在流体力学和压力方面。为纪念他的科学贡献,国际单位制中压力的单位被命名为“帕斯卡”,他还提出了帕斯卡定律,这是流体静力学中的一个重要原理。他在研究永动机的过程中还发明了一种原始形式的轮盘赌和轮盘装置\(^\text{[30]}\)。
\subsubsection{流体动力学}
他在流体动力学和流体静力学方面的工作集中于液体的液压原理。他的发明包括液压机(利用液体压力来放大力量)和注射器。他证明了流体静压并不取决于流体的重量,而是取决于高度差。他通过一个著名的实验演示了这一原理:将一根细管插入盛满水的木桶中,并将水加至与三楼相同的高度。结果,木桶因压力而破裂,这一实验被称为“帕斯卡木桶实验”。
\subsubsection{真空}
1647年,帕斯卡得知埃万杰利斯塔·托里拆利关于气压计的实验。他复现实验:将一根充满水银的玻璃管倒插入水银盆中。他随即提出质疑:到底是什么力量让部分水银停留在管中?而管中水银上方的空间又是什么?当时大多数科学家(包括笛卡尔)都相信“充满说”,即某种看不见的物质充满了所有空间,而非存在真空——这源于亚里士多德的观念,认为世间一切运动都必须由一种物质推动另一种物质\(^\text{[31]}\)。此外,光能够穿过玻璃管,这也让人们倾向认为那是以太等物质,而非真空。

在此基础上,帕斯卡于1647年发表了《有关真空的新实验》,系统描述了空气压力可以支撑液体的高度规则,并阐明了气压计中液柱上方确实是一个真空。这项工作后来又以《液体平衡伟大实验记》在1648年发表。
\subsubsection{首次大气压与海拔高度关系实验}
托里拆利真空表明空气压力相当于30英寸汞柱的重量。如果空气具有有限的重量,那么地球的大气层必定有一个最大高度。帕斯卡尔据此推理:如果这一点为真,那么高山上的气压就应该低于海拔较低处。他住在海拔4,790英尺(约1,460米)的多姆山附近,但因健康状况不佳无法亲自登山\(^\text{[32]}\)。1648年9月19日,在帕斯卡尔几个月友好而坚持不懈的督促下,他的姐夫弗洛林·佩里耶终于得以执行这项对帕斯卡尔理论至关重要的实地调查。佩里耶撰写的记录如下:

“上周六天气不太稳定……[但]大约清晨五点……多姆山可见……于是我决定尝试登顶。克莱蒙市的几位要人曾要求我出发时告知他们……我很高兴能与他们共同参与这项伟大的工作……

……八点钟我们在米宁派修士花园集合,那是城中海拔最低之处……我先往一个容器中倒入16磅水银……然后取出几根玻璃管……每根四英尺长,一端密封,另一端敞开……将它们插入容器中的水银……我发现水银柱高出容器水银面26英寸又3又½行……我在同一地点又重复了两次实验……每次都得出相同的结果……

我将一根玻璃管固定在容器上,并标记了水银的高度……然后请米宁修士之一的沙斯坦神父在原地观察当日是否有任何变化……我则带着另一根玻璃管和一部分水银……登上比修道院高出约500寻的多姆山顶,在那里进行实验……发现水银柱高度仅为23英寸又2行……我在山顶不同地点重复实验五次……每次都得出相同的水银高度……\(^\text{[33]}\)”

帕斯卡尔在巴黎也重复了这一实验,他将气压计带上了圣雅克-德拉布舍里教堂钟楼,约50米高。水银柱下降了2行。他从这两组实验中得出:每上升7寻,汞柱高度就下降半行。【注】:帕斯卡尔使用的单位中,pouce 和 ligne 分别表示“英寸”和“行”,toise 表示“寻”\(^\text{[34]}\)。

帕斯卡尔在回应仍相信充盈说的艾蒂安·诺埃尔时写道,呼应了当代关于科学与可证伪性的观念:“为了证明一个假设是正确的,仅仅现象都可由其推出是不够的;相反,如果它得出与某一个现象相矛盾的结论,那就足以证明其错误。”\(^\text{[35]}\)

“布莱兹·帕斯卡尔讲席”授予国际杰出科学家,以便他们在法兰西岛大区进行研究工作\(^\text{[36]}\)。
\subsection{成年生活:宗教、文学与哲学}
\subsubsection{宗教皈依}
\begin{figure}[ht]
\centering
\includegraphics[width=6cm]{./figures/d6ba0557cb107cf0.png}
\caption{} \label{fig_BLSpsk_7}
\end{figure}