% 几种含时微扰
% 含时微扰|微扰理论|量子力学|脉冲

\pentry{含时微扰理论\upref{TDPT}}

令初态到末态能量差对应的光子频率为
\begin{equation}
\omega_{fi} = \frac{E_f - E_i}{\hbar}
\end{equation}

\begin{equation}
S = \frac{1}{\I\hbar} \int_{-\infty}^{+\infty} \mel{f}{\Q H'(t)}{i} \E^{\I\omega_{fi}} \dd{t}
\end{equation}
当 $\mel{f}{H'(t)}{i} = W_{fi} g(t)$ 时
\begin{equation}
P_{fi} = \abs{S}^2 = \frac{\abs{W_{fi}}^2}{\hbar^2} \abs{\int_{-\infty}^{+\infty} g(t) \E^{\I\omega_{fi} t} \dd{t}}^2
\end{equation}

\subsubsection{瞬时脉冲 $g(t) = \delta(t-t_0)$}
\begin{equation}
\abs{\int_{-\infty}^{+\infty} g(t) \E^{\I\omega_{fi} t} \dd{t}}^2
= \abs{\int_{t_0-\epsilon}^{t_0+\epsilon} \delta(t-t_0) \E^{\I\omega_{fi} t} \dd{t}}^2
= 1
\end{equation}
代入得
\begin{equation}
P_{fi} = \frac{\abs{W_{fi}}^2}{\hbar^2}
\end{equation}

\subsubsection{方形脉冲 $g(t)$(从 $t=t_1$ 到 $t=t_2$)}
\begin{equation}\begin{aligned}
\abs{\int_{-\infty}^{+\infty} g(t) \E^{\I\omega_{fi}t} \dd{t}}^2
&= \abs{\int_{t_1}^{t_2} \E^{\I\omega_{fi}t} \dd{t}}^2
= \abs{\frac{\E^{\I\omega_{fi}t_2} - \E^{\I\omega_{fi}t_1}}{\I\omega_{fi}}}^2\\
&= \frac{\sin^2[\omega_{fi}(t_2-t_1)/2]}{[\omega_{fi}(t_2-t_1)/2]^2} (t_2-t_1)^2 \\
&= \Delta t^2 \sinc^2[\omega_{fi}\Delta t/2]
\end{aligned}\end{equation}
概率为
\begin{equation}
P_{fi} = \frac{\abs{W_{fi}}^2}{\hbar^2} \Delta t^2 \sinc^2[\omega_{fi}\Delta t/2]
\end{equation}
于瞬时脉冲相比,主要跃迁到附近的 $E_2$ 能级.且时间越长能量变化越小.

\subsubsection{简谐振动 $g(t)= \E^{\I\omega t}$}
%未完成: 应该直接讨论三角函数的情况
与上面的推导类似,结果为
\begin{equation}
P_{fi} = \frac{\abs{W_{fi}}^2}{\hbar^2} \Delta t^2 \sinc^2[(\omega_{fi}+\omega)\Delta t/2]
\end{equation}
这说明,跃迁倾向于增加能量 $\hbar\omega$,时间越长,就越靠近 $\hbar\omega$.要注意真实的简谐微扰往往是 $\cos(\omega t)$, 分解为两项积分后,会有干涉效应,结果较为复杂.但若 $\omega \gg \omega_{fi}$ 时可以忽略干涉项.

考虑当 $\Delta t$ 非常大的情况,这时 $\sinc^2(x)/\sqrt{\pi}$ 可以近似看做 $\delta$ 函数.
