% 带电粒子的薛定谔方程
% 点电荷|薛定谔方程|哈密顿|量子化
% 参考 Brandsen 的原子物理

\pentry{点电荷的拉格朗日和哈密顿量\upref{EMLagP}}

经典点电荷在电磁场中的哈密顿量为
\begin{equation}
H = \frac{1}{2m}(\bvec p - q\bvec A)^2 + q\phi
\end{equation}
其中 $\bvec A$ 和 $\phi$ 都是位置和时间的函数. 注意这里的 $\bvec p$ 是广义动量
\begin{equation}
\bvec p = m\bvec v + q\bvec A
\end{equation}
算符仍然是 ${\bvec p} = -\I\hbar\grad$. 只有 $\bvec A = \bvec 0$ 时才有 $\bvec p = m\bvec v$.

所以哈密顿算符是
\begin{equation}
H = \frac{\bvec p^2}{2m} - \frac{q}{2m}(\bvec A\vdot \bvec p + \bvec p \vdot \bvec A) + \frac{q^2}{2m} \bvec A^2 + q\phi
\end{equation}

如果我们对波函数也进行一个相位变换, 这个方程在标势和矢势的规范变换下形式不变
\begin{equation}
\bvec A = \bvec A' + \grad \chi
\end{equation}
\begin{equation}
\phi = \phi' - \pdv{\chi}{t}
\end{equation}
\begin{equation}
\Psi = \Psi' \exp(\I q\chi/\hbar)
\end{equation}
其中 $\chi(\bvec r, t)$ 是一个任意可导函数. 将以上三式代入薛定谔方程, 只需要把不带撇的变量替换为带撇的变量.
