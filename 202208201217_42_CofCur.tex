% 曲线的切触
% keys 切触

研究二图形在某个点的贴近程度具有重要意义,它不仅反映图形几何位置的特点,更重要的是在研究在某点附近图形的性质时,利用在该点具有相当贴近程度的图形来代替原来的图形可使问题简化.例如,研究曲线在某点附近的性质时,我们往往用切线代替该曲线从而使问题简化.切触(Contact)便是刻画两图形在某一点附近贴近程度的重要概念,它属于局部微分几何,两图形切触阶越高,图形贴近的程度也就越高.本节主要介绍曲线的切触.
\subsection{曲线的切触}
设 $\Gamma_1,\Gamma_2$ 两曲线有公共点 $P_0$,而 $P_1,P_2$ 依次为曲线 $\Gamma_1,\Gamma_2$ 上与 $P_0$ 邻近的点,且弧长 $\overset{\Large\frown}{P_0P_1}$