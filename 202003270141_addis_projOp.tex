% 投影算符

\pentry{几何矢量\upref{GVec}}

本词条只介绍有限维矢量空间的投影算符.

\begin{theorem}{正交分解}
令 $M$ 为 $N$ 维矢量空间 $X$ 的子空间. 那么任意 $u\in X$ 有唯一的分解
\begin{equation}\label{projOp_eq1}
u = v + w \qquad (v\in M, w\in M^\bot)
\end{equation}
其中 $M^\bot$ 是 $M$ 在 $X$ 中的正交补.
\end{theorem}

对每个子空间 $M\subseteq X$, 我们定义对应的\textbf{投影算符} $P$, 根据\autoref{projOp_eq1} 将每个 $u\in X$ 映射到 $v$.

\subsection{证明投影算符是厄米算符}
令投影算符将空间 $X$ 中的矢量投影到空间 $A$ 中, $\qty{a_i}$ 是 $A$ 的一组正交归一基底. 那么投影算符可以表示为
\begin{equation}
P = \sum_i \ket{a_i}\bra{a_i}
\end{equation}
对任意 $u, v\in X$, 有
\begin{equation}
\braket{u}{Pv} = \sum_i \\braket{u}{a_i}\braket{a_i}{v}
\end{equation}
\begin{equation}
\braket{Pu}{v} = \overline {\braket{v}{Pu}} = \sum_i \overline{\\braket{v}{a_i}}\overline{\braket{a_i}{u}} = \sum_i \\braket{u}{a_i}\braket{a_i}{v} = \braket{u}{Pv}
\end{equation}
证毕.
