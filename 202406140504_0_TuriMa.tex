% 图灵机到真实计算机
% license Usr
% type Tutor

\begin{issues}
\issueDraft
\end{issues}

\pentry{}{nod_cab0}

\textbf{图灵机要素}:无限长纸带、有限字符集、有限的状态集、根据状态和当前字符决定下一步行为(写入、移动)、初始状态、停机状态。

图灵可以完成任何计算机可以完成的事情。如果一个编程语言可以做图灵机的任何事情(除了纸带不是无限长),那么他就是图灵完备的。

\subsubsection{真实计算机}
\textbf{纸带被划分为几个部分}:\textbf{代码段}(储存程序指令,一般只读)、\textbf{数据段}(分为只读和读写,只读部分是 literal,读写部分是全局变量和 static 变量)、\textbf{栈}(函数局部变量)、\textbf{堆}(动态分配内存)。

\begin{itemize}
\item \textbf{变量}没有类型,就对应一个地址和长度
\item 没有算符更没有重载,只有\textbf{函数},每个函数对应一个函数地址
\item 内存不区分 stack 和 heap,每个函数从内存固定几个地址读取固定长度的函数参数,输出到固定地址固定长度。
\item 这其实就是更具体的图灵机,纸带划分为指令部分和内存部分。每个周期,图灵机先从 program counter 读取下一个要执行的指令、该指令要操作的数据在内存中的固定地址,然后计算(“与非”或“或非”或NANA)。
\item 更具体的图灵机:可以认为图灵机本身包含了 cpu 的硬件运算能力,实现与、或、非、整数和浮点数加法等(其实最基本的只要掌握,其他硬件能力都可以用代码段来软实现),也包含了这些运算所需的寄存器。程序计数器。
\item \textbf{纸带被划分为几个部分}:\textbf{代码段}(储存程序指令,一般只读)、\textbf{数据段}(分为只读和读写,只读部分是 literal,读写部分是全局变量和 static 变量)、\textbf{栈}(函数局部变量)、\textbf{堆}(动态分配内存)。
\item \textbf{静态类型只是用于辅助编译器帮用户检查是否函数用对了,编译完成后静态类型丢失,所有的函数的参数都只是给定位置的一串 0 和 1}
\item \textbf{静态类型的唯一作用就是告诉编译器应该选择什么函数地址!否则用户完全可以所有变量都使用 \verb`const void *`。 甚至连每个变量的字节数都不需要用别的变量储存,因为每个函数是知道每个变量的长度的。}
\item \textbf{人类很喜欢 overload,机器不喜欢。 即使是 C 语言也有 overload 例如 \verb`+` 算符其实不是一个函数地址(或者 cpu 指令),而是根据变量类型决定具体用哪一个,只是不能用户定义罢了。 C++ 把 overload 推向了全新的境界。}
\item \textbf{多态(polymorphism)}主要指的是运行时的类型多态。 运行时多态和普通函数重载有本质区别。 前者根据运行时变量的值决定调用哪个地址的函数,而普通函数重载是编译时写死到代码段,运行时无法改变。另外,运行时也会有方法多态。
\end{itemize}
