% 庞加莱猜想(综述)
% license CCBYSA3
% type Wiki

本文根据 CC-BY-SA 协议转载翻译自维基百科\href{https://en.wikipedia.org/wiki/Poincar\%C3\%A9_conjecture}{相关文章}。

在几何拓扑的数学领域,庞加莱猜想(英国发音:/ˈpwæ̃kæreɪ/,美国发音:/ˌpwæ̃kɑːˈreɪ/,法语发音:[pwɛ̃kaʁe])是一个关于3-球体的定理,3-球体是四维空间中界定单位球的超球面。

该猜想最初由亨利·庞加莱于1904年提出,定理涉及到一些局部上看起来像普通三维空间,但在规模上是有限的空间。庞加莱假设,如果这样的空间具有一个额外的特性,即该空间中的每一条环路都可以连续地收缩到一个点,那么这个空间必定是一个三维球体。解决该猜想的努力推动了20世纪几何拓扑领域的许多进展。

最终的证明基于理查德·S·汉密尔顿使用Ricci流来解决该问题的方案。通过在Ricci流理论中发展一系列新的技术和结果,格里戈里·佩雷尔曼能够修改并完成汉密尔顿的方案。在2002年和2003年发布于arXiv的论文中,佩雷尔曼展示了他证明庞加莱猜想(以及威廉·瑟尔斯顿的更强的几何化猜想)工作的过程。随后几年,几位数学家研究了他的论文,并详细阐述了他的工作。

汉密尔顿和佩雷尔曼在该猜想上的工作被广泛认为是数学研究的一个里程碑。汉密尔顿因其贡献获得了邵逸夫奖和勒罗伊·P·斯蒂尔研究奠基奖。《科学》杂志将佩雷尔曼证明庞加莱猜想的成果评为2006年年度科学突破。[5] 克雷数学研究所将庞加莱猜想列为著名的千年奖问题之一,并为该猜想的解决提供了100万美元的奖赏。[6] 佩雷尔曼拒绝了这一奖项,称汉密尔顿的贡献与他自己的贡献相等。[7][8]
\subsection{概述}
\begin{figure}[ht]
\centering
\includegraphics[width=6cm]{./figures/bb28959dfa154fe6.png}
\caption{这个环面上的两个彩色环都不能连续地收缩到一个点。环面与球面不是同胚的。} \label{fig_PJLCX_1}
\end{figure}
庞加莱猜想是几何拓扑学领域的一个数学问题。用该领域的术语来说,它表述如下:

庞加莱猜想:  
每一个封闭、连通且基本群为平凡的三维拓扑流形,与三维球面是同胚的。

熟悉的形状,比如球体的表面(在数学中称为二维球面)或环面的表面,都是二维的。球体的表面具有平凡的基本群,意味着在表面上画出的任何环都可以连续地变形为一个点。相比之下,环面的表面具有非平凡的基本群,因为表面上存在一些环无法这样变形。两者都是拓扑流形,且都是闭合的(意味着它们没有边界,占据有限的空间区域)和连通的(意味着它们由一个整体组成)。当两个闭合流形的点能够通过连续的方式重新映射到彼此时,称它们是同胚的。由于基本群的(非)平凡性在同胚变换下是不变的,因此可以推断二维球面和环面不是同胚的。

庞加莱猜想的二维类比表明,任何闭合且连通的二维拓扑流形,如果不是二维球面同胚的,则必定包含一个无法连续收缩为一点的环。(这可以通过环面的例子来说明,如上所述。)通过对闭合且连通的二维拓扑流形的分类,已知这一类比是成立的,早在19世纪60年代就以各种形式被理解。在更高维度中,闭合且连通的拓扑流形没有简单的分类方法,这使得庞加莱猜想的解决变得更加复杂。
\subsection{历史}  
\subsubsection{庞加莱的问题}  
在19世纪,伯恩哈德·黎曼和恩里科·贝蒂开始研究流形的拓扑不变量。[9][10] 他们引入了贝蒂数,它将每个流形与一组非负整数关联起来。黎曼曾证明,闭合且连通的二维流形可以通过其贝蒂数完全表征。在他的1895年论文《分析位置》(于1892年公布)中,庞加莱证明了黎曼的结果无法扩展到更高维度。[11][12][13] 为此,他引入了基本群作为一种新的拓扑不变量,并能够展示一些具有相同贝蒂数但不同基本群的三维流形的例子。他提出了一个问题:基本群是否足以在拓扑上表征一个流形(给定维度),尽管他并没有继续追求答案,仅仅表示这将“需要漫长且困难的研究”。[12][13][14]

庞加莱论文的主要目的是通过他新引入的同调群来解释贝蒂数,并提出关于贝蒂数对称性的庞加莱对偶定理。在他遭到对论证完整性的批评后,他发布了若干后续的“补充”来增强和修正他的工作。他的第二篇补充材料,发表于1900年,最后写道:[15][13]

“为了避免使这项工作过于冗长,我仅限于陈述以下定理,证明将需要进一步的发展:

每个所有贝蒂数都等于1且所有其Tq表可定向的多面体都是简单连通的,即同胚于超球面。”

(用现代的语言来说,考虑到庞加莱在使用简单连通性的术语时有一种不常见的方式[16],这句话的意思是:一个具有球面同调群的闭合连通定向流形必须同胚于一个球面。[14])这两点对庞加莱对黎曼工作的否定性推广做出了两方面的修改。首先,他现在使用了完整的同调群,而不仅仅是贝蒂数。其次,他将问题的范围从询问任意流形是否由拓扑不变量来表征,缩小到询问球面是否可以这样被表征。

然而,在出版后,庞加莱发现他宣布的定理是错误的。在他1904年发布的第五篇也是最后一篇补充中,他通过庞加莱同调球的反例证明了这一点。庞加莱同调球是一个闭合连通的三维流形,它具有与球面相同的同调群,但其基本群有120个元素。这个例子清楚地表明,同调群不足以表征流形的拓扑结构。在第五篇补充的结束语中,庞加莱修改了他错误的定理,改用基本群代替同调群:[17][13]

“有一个问题仍需解决:是否有可能流形V的基本群缩减为单位元素,而V却不是简单连通的? [...] 然而,这个问题将把我们带得太远。”

在这个备注中,和第二篇补充的结束语一样,庞加莱使用了“简单连通”这个术语的方式与现代用法不一致,并且与他1895年对该术语的定义相矛盾。[12][16](根据现代用法,庞加莱的问题是自明的,问的是是否有可能一个流形是简单连通的却不是简单连通的。)然而,正如从上下文中可以推断的那样,[18]庞加莱的真正问题是:基本群的平凡性是否能够唯一地表征球面。[14]

在黎曼、贝蒂和庞加莱的工作中,所涉及的拓扑概念并没有以现代视角下精确的方式定义或使用。即便是“流形”这一关键概念,在庞加莱自己的工作中也没有一致的使用方式,并且常常混淆了拓扑流形、分段线性(PL)流形和平滑流形的概念。[16][19]因此,我们无法毫无歧义地解读庞加莱提出的问题。只有通过后来的数学家们对拓扑学的形式化和词汇发展,庞加莱的最后问题才被理解为“庞加莱猜想”,如前文所述。

然而,尽管通常将其表述为一个猜想,提出所有某类流形都是与球面同胚的,但庞加莱只是提出了一个开放性的问题,并未尝试猜测答案的方向。此外,没有证据表明他相信自己的问题会以何种方式得到回答。[14]
\subsubsection{解答}
在20世纪30年代,J.H.C.怀特海德声称已证明此猜想,但后来撤回了证明。在此过程中,他发现了一些简单连通的(实际上是收缩的,即同伦理等价于一个点的)非紧致3流形,它们与 \(\mathbb{R}^3\) 不同,现今称为怀特海德流形。

在20世纪50至60年代,其他数学家尝试证明该猜想,但都发现其证明存在缺陷。诸如乔治·德·拉姆(Georges de Rham)、R.H.宾(R.H. Bing)、沃尔夫冈·哈肯(Wolfgang Haken)、埃德温·莫伊斯(Edwin E. Moise)和基里斯托斯·帕帕基里亚科普洛斯(Christos Papakyriakopoulos)等有影响力的数学家也尝试过证明该猜想。1958年,R.H.宾证明了庞加莱猜想的一个弱版本:如果一个紧致的3流形的每一个简单闭合曲线都包含在一个3球内,那么该流形与3球同胚。[20]宾还描述了尝试证明庞加莱猜想时的一些陷阱。[21]

1978年,Włodzimierz Jakobsche证明,如果宾-博尔苏克猜想在3维下成立,那么庞加莱猜想也必须成立。[22]

随着时间的推移,庞加莱猜想被认为特别难以攻克。约翰·米尔诺(John Milnor)评论说,有时错误的证明中的漏洞“相当微妙,且难以察觉”。[23]对该猜想的研究提高了对3流形的理解。该领域的专家往往不愿宣布证明,并且倾向于对任何此类声明持怀疑态度。20世纪80至90年代,出现了一些广为宣传的错误证明(这些证明实际上并未以同行评审的形式发表)。[24][25]

关于尝试证明这一猜想的过程,可以在乔治·斯皮罗(George Szpiro)的非技术性著作《庞加莱的奖赏》(Poincaré's Prize)中找到详细的阐述。[26]

