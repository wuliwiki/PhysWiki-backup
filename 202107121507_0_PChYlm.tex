% 点电荷势能的球谐展

\begin{issues}
\issueDraft
\end{issues}

\pentry{球坐标系中的拉普拉斯方程\upref{SphLap}}

\begin{equation}
\frac{1}{\abs{\bvec r - \bvec r'}} = 4\pi \sum_{l=0}^{\infty} \frac{1}{2l+1} \frac{r_<^l}{r_>^{l+1}} \sum_{m = -l}^l Y_{l,m}^*(\uvec r') Y_{l,m}(\uvec r)
\end{equation}
根据\autoref{SphHar_eq10}~\upref{SphHar}, 令 $\alpha$ 为 $\uvec k, \uvec r$ 的夹角, 上式也可以记为
\begin{equation}\label{PChYlm_eq1}
\frac{1}{\abs{\bvec r - \bvec r'}} = \sum_{l=0}^{\infty} \frac{r_<^l}{r_>^{l+1}} P_l(\cos\theta)
\end{equation}

\subsection{证明}
和 “平面波的球谐展开\upref{Pl2Ylm}” 中的证明类似, 我们既可以直接积分得到径向函数, 根据\autoref{PChYlm_eq1} 还可以假设 $\bvec r'$ 在 $z$ 轴. 也可以在球坐标中解偏微分方程.

除 $\bvec r'$ 点外, $f(\bvec r) = \frac{1}{\abs{\bvec r - \bvec r'}}$ 符合拉普拉斯方程. 球坐标系中拉普拉斯方程\upref{SphLap}的通解为(\autoref{SphLap_eq6}~\upref{SphLap})
\begin{equation}
f(\bvec r) = f(r, \theta, \phi) = \sum_{l = 0}^\infty \sum_{m = -l}^l \qty(C_{l,m} r^l + \frac{C'_{l,m}}{r^{l+1}})P_l^m(\cos\theta)\E^{\I m\phi}
\end{equation}
把空间划分为 $r < r'$ 和 $r > r'$ 两个区域. $r = 0$ 时函数需要有定义, 所以 $r < r'$ 区间 $C'_{l,m} = 0$. 另外要求 $r \to \infty$ 时 $f(\bvec r) \to 0$, 所以当 $r > r'$, $C_{l,m} = 0$.

接下来对比系数即可求出 $C_{l,m}$ 和 $C_{l,m}'$.
\addTODO{具体计算}
