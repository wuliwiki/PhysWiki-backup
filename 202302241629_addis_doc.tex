% 学习


作者:超级懒的小周
链接:https://zhuanlan.zhihu.com/p/392284625
来源:知乎
著作权归作者所有。商业转载请联系作者获得授权,非商业转载请注明出处。



\begin{multicols}{2}
从世界范围看,国家、企业、资本、科技四大象限搅得天翻地覆

第四次金融危机正式展开,中美贸易摩擦持续,世界经济普遍陷入低迷,叠加债务危机、难民问题,再加上特朗普这位“朝三暮四”、到处惹祸的美国总统,全世界的国家都没了方向
科技发展日新月异,AI(人工智能)取代人工、生命科学的“上帝之手”等都成为正在进行时,底层人民的被剥夺感日渐加重,不安全,很慌张
国内社会看

改革开放40多年,发展成果斐然,社会生产力得到极大解放,主要矛盾也已经转化,但中国的发展具有非典型性,时代、互联网、地域、贫富等因素切割出太多群体
\end{multicols}

\columnseprule=1pt         % 实现插入分隔线

从世界范围看,国家、企业、资本、科技四大象限搅得天翻地覆

第四次金融危机正式展开,中美贸易摩擦持续,世界经济普遍陷入低迷,叠加债务危机、难民问题,再加上特朗普这位“朝三暮四”、到处惹祸的美国总统,全世界的国家都没了方向
科技发展日新月异,AI(人工智能)取代人工、生命科学的“上帝之手”等都成为正在进行时,底层人民的被剥夺感日渐加重,不安全,很慌张
国内社会看

改革开放40多年,发展成果斐然,社会生产力得到极大解放,主要矛盾也已经转化,但中国的发展具有非典型性,时代、互联网、地域、贫富等因素切割出太多群体

\begin{multicols}{3}
从世界范围看,国家、企业、资本、科技四大象限搅得天翻地覆

第四次金融危机正式展开,中美贸易摩擦持续,世界经济普遍陷入低迷,叠加债务危机、难民问题,再加上特朗普这位“朝三暮四”、到处惹祸的美国总统,全世界的国家都没了方向
科技发展日新月异,AI(人工智能)取代人工、生命科学的“上帝之手”等都成为正在进行时,底层人民的被剥夺感日渐加重,不安全,很慌张
国内社会看

改革开放40多年,发展成果斐然,社会生产力得到极大解放,主要矛盾