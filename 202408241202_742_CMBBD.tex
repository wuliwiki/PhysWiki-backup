% 暗物质属性的宇宙学限制
% license Usr
% type Tutor




大尺度结构和CMB数据允许测试暗物质的行为。全局拟合更倾向于最简单的冷非相互作用暗物质模型,具有高斯绝热扰动。下面,我们总结这意味着什么并提出当前的限制。

人们可以修改1.3.1和1.3.3节中描述宇宙学扰动的玻尔兹曼方程,用具有非零声速vs的更复杂的暗流体(使暗物质不再冷)和/或粘度(使暗物质相互作用)和/或修改后的状态方程w = ℘/ρ ≠ 0(使暗物质不再表现为非相对论性物质)来替代暗物质。全局拟合CMB数据给出了这些参数在10^−3数量级的限制[26]。

关于绝热性——1.3.3节中的玻尔兹曼方程必须针对每个组分和任何尺度k从远在物质/辐射平等之前的时刻开始求解初始条件。CMB数据很好地拟合了假设绝热初始条件:一个原始不均匀性对所有组分都是共同的δk = δbk = 3Θ0(k) = 3N0(k),……(1.35)

光子和中微子的因子3仅仅是因为Θ0和N0定义为温度中的不均匀性而不是密度中的不均匀性。方程(1.35)意味着每种组分的密度与总密度成比例,因此整体效应可以用几何原始曲率(或者等效地,引力势)来描述。这样的绝热初始条件出现在最小的膨胀模型中,其中所有的原始不均匀性都是由一个“膨胀子”标量场的量子涨落产生的。像往常一样,量子力学为这些涨落的统计分布提供了预测,而不是个别测量的具体结果。这就是为什么所有可观测量都是在空间上平均的,或者等效地,在具有不同取向的傅里叶模式上平均。

关于高斯性——我们在1.3.1和1.3.3节中专注于最简单的两点相关函数,⟨aℓmaℓ′m′⟩ = δℓℓ′δmm′ Cℓ在1.3.3节和⟨δkδk′⟩ = (2π)3 P(k) δ3(k − k′)在1.3.1节。原因是,具有可忽略相互作用的自由标量场(如具有其准平坦势的膨胀子)的量子涨落遵循高斯分布(因为这是非相互作用谐振子的基态)。高斯原始不均匀性由两点相关函数完全确定,紧凑地由单一功率谱描述。高斯不均匀性适合CMB和LSS数据。然而,这不是唯一的可能性。膨胀子可能有相互作用,导致曲率扰动ζ的概率分布中出现非高斯性,

\[ \phi(\zeta) \sim \exp \left( -\frac{\zeta^2}{2\langle\zeta^2\rangle} - \frac{\langle\zeta^3\rangle\zeta^3}{\langle\zeta^2\rangle^3} + \cdots \right) = \exp \left( -\frac{\zeta^2}{2\langle\zeta^2\rangle} (1 + f_{\text{NL}}\zeta + \cdots) \right) ~.\]

其中非零三点函数的种类被一个\( f_{\text{NL}} \sim \frac{\langle\zeta^3\rangle}{\langle\zeta^2\rangle^2} \)参数松散地描述。单场膨胀可以假设一个几乎平坦的膨胀子势,对应于一个小的慢滚参数 \( \epsilon \sim 0.01 \),这样几乎不相互作用的膨胀子产生 \( P_\zeta \sim \langle\zeta^2\rangle \sim \frac{H^2_{\text{inf}}}{M^2_{\text{Pl}}}\epsilon \) 和一个小的 \( f_{\text{NL}} \sim \epsilon \) [27]。这远低于当前由CMB数据暗示的非高斯性界限,\( f_{\text{NL}} \lesssim 10^{-2} \) [3]。未来的CMB和大尺度结构数据可以将对 \( f_{\text{NL}} \) 的灵敏度提高一个和两个数量级。甚至更低的值可能通过未来派的21厘米观测来探测。然而,仍然很难设计非最小膨胀模型来提供 \( f_{\text{NL}} \) 的信号。回到绝热性的问题——在膨胀背景下,额外的不均匀性可能产生,例如,来自其他一些场的独立量子涨落,这些场在膨胀期间恰好足够轻。从现象学角度来看,考虑所谓的等曲率涨落是方便的。这些违反了方程(1.35),通过将不同的不均匀性归因于不同的物种,但不影响总能量密度中的不均匀性,从而不影响曲率。我们对原始暗物质密度中的等曲率涨落感兴趣 \( \Delta_{\text{iso}}(k) = \frac{\delta\rho_k^{\text{DM}}}{\rho_{\text{DM}}} \) ,对应于功率谱 \( \Delta^2_{\text{iso}} = \frac{k^3P_{\text{iso}}}{2\pi^2} \) 。那么在1/k尺度上的暗物质密度的涨落将与标准模型粒子的涨落不同。全球拟合Planck数据要求绝热涨落占主导:

\[ \frac{P_{\text{iso}}(k)}{P_{\text{iso}}(k) + P_{\text{adi}}(k)} < 0.038 \text{ at 95\% C.L. at } k = 0.05/\text{Mpc} ~. \]

这个界限适用于在物质/辐射平等时期进入视界的尺度1/k的涨落,使得 \( \Delta^2_{\text{adi}} \approx 2.2 \times 10^{-9} \) 很好地被CMB探测。更小的、次宇宙学尺度上的等曲率涨落没有受到限制(例如,在星系尺度上)。粗略地说,DM具有与普通物质相同的涨落表明两者之间早期存在某种联系,并限制了可行的DM产生机制,见第3章的讨论。

到目前为止,我们假设DM是非相互作用的。如果DM粒子除了引力之外还受到其他额外的相互作用,那么描述DM引力聚类的方程(1.22)就会被修改。

- 如果DM与中微子相互作用[28]14,这通常会通过碰撞阻尼抑制DM原始密度涨落,从而在CMB各向异性功率谱和物质功率谱中留下明显的标志。粗略的总结是,当前数据将DM/ν相互作用截面限制在 \( \sigma_{\text{DM}/\nu} \lesssim 10^{-32} (\frac{M}{\text{GeV}})^2 \) cm^2(假设没有温度依赖性),不同研究发现了一个数量级较弱或更强的界限。

- 类似地,DM与光子的相互作用[29]只要足够弱就仍然是可行的。现象学后果与与中微子的相互作用相似(首先近似中微子和光子在CMB时期仅仅是两种形式的辐射)。粗略的总结是,当前数据再次将DM/γ相互作用截面限制在 \( \sigma_{\text{DM}/\gamma} \lesssim \text{few} \times 10^{-33} (\frac{M}{\text{GeV}})^2 \) cm^2(假设没有温度依赖性)。这并不特别限制性:在DM与光子相互作用是由于DM带有小电量的情况下,存在明显更强的界限,将在3.3.2节中讨论。

14不要与6.3.1节讨论的当前天体物理系统中DM/中微子在更高能量下的相互作用混淆。

\subsection{大爆炸核合成} 

从CMB数据的全局拟合得到的普通物质密度(第15页的Ωbh2)与从宇宙大爆炸核合成(BBN)得出的独立精确测定一致。BBN是一个理论框架,描述了宇宙中轻元素(大部分的氘和氦-4以及相当一部分的氦-3和锂-7)的合成,起始于它们的构建块:原始等离子体中存在的自由质子和中子。更准确地说,BBN对重子-光子比率η ≡ nb/nγ敏感,因为光子会破坏原子核,延迟它们的形成温度。测量给出[5]

\[ \eta = (6.2 \pm 0.4) \times 10^{-10} \Rightarrow \Omega_b h^2 = 0.022 \pm 0.002,~. \]

与CMB值(第15页的Ωbh2)一致。为了得到上述第二个关系,我们使用了 Ωb ≡ ρb/ρcrit = η n0 γmp/ρcrit,其中mp是质子质量,n0 γ是当前光子数密度。两个确定的重子密度的一致性被视为标准宇宙学模型的一个重大成功,为宇宙的演化在这个模型内被很好地理解提供了信心。BBN探测的宇宙阶段比CMB更早,即在显著更高的温度T ∼ MeV,相比之下CMB的温度是T ∼ eV。Ωb在(1.38)中满足 Ωb < Ωmatter ≃ 0.30,其中Ωmatter包括所有形式的物质,为暗物质提供了进一步的证据。这是作为非重子物质所必需的,并且对于如1.3.1节中讨论的结构形成是必需的。BBN在(1.38)中的结果在历史上特别重要,在CMB和LSS测量足够精确以能够单独确定重子和DM的相对比例之前。





