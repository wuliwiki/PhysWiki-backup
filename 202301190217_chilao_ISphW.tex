% 无限深球势阱

\begin{issues}
\issueDraft
\end{issues}

\pentry{球坐标系中的定态薛定谔方程\upref{RadSE}, 球贝塞尔函数\upref{SphBsl}}

球无限深势阱的势能项可以表示为
$$
V(r)=
\begin{cases}
0,  & \text{若 $0\leq r\leq a$} \\
\infty, & \text{若 $r>a$}
\end{cases}\tag{1}
$$

在球坐标系下,角度部分的解是球谐函数。对于径向方程,我们可以写出它的形式为:

\begin{equation}
-\frac{\hbar^2}{2mr^2} \dv[2]{\psi_l}{r} + \frac{l(l + 1)}{2mr^2}\psi_l = E\psi_l\tag{2}
\end{equation}

球面 $r = a$ 处的边界条件为 $\Psi = 0$。

势阱内部的径向波函数 (unscaled) 为球贝塞尔函数 $j_l(kr)$

证明波函数的正交归一性。
