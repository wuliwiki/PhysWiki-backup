% 函数(高中)
% 函数|定义域|值域|二元函数


\begin{issues}
\issueDraft
\issueTODO
\end{issues}

% 高中的 函数 不应该需要 映射 作为预备知识
本篇文章的预设读者是,对于函数的定义感到熟悉但又有些模糊,希望进一步了解函数的高中学生。

函数最开始是用于研究曲线的工具。高中时期函数往往出现在平面直角坐标系上,不同的函数就对应着不同的曲线,你也许有些困惑——说不清二者之间的区别和联系。比如$f(x)=x^2$,对于每一个$x_1$,都能够找到对应的$f(x_1)$,进而得到平面上对应的点$(x_1,f(x_1))$,这是它们的联系。而区别在于,不应认为曲线就是我们口中的函数,只能认为,一个函数可以确定一条曲线。

在高中时期,一个函数通常指的是一个计算式,输入一个数字,然后通过计算式计算以输出一个数字。数学家们在上述内涵中,进一步地抽象得到更加广泛的定义,以此让函数这一概念能够更加精确、更加广泛的描述事物。

首先,我们可以将函数认作是对输入和输出的关系的描述,我们所见到的\textsl{计算式}就是告诉我们如果将输入得到输出的方法。用生活中常见的事物来比喻这一角度所理解的函数的话,用工厂来比喻可能很合适,我们向自动化工厂的入料口中添加原材料,经过一系列复杂的加工得到了产品,其中的加工方法、流程就可以粗略地看作是一个\textbf{函数}。又或者更加具体的说,一份水果沙拉的制作方法+我们灵巧的双手,可以使得若干水果变为一份水果沙拉。这样,我们就在原有的计算式的理解上更进一步了,事实上,我们已经得到了计算机学科中对函数的理解。
\begin{enumerate}
\item 从内容上,我们将输入从数字拓宽到了任一事物
\item 从数量上,我们不局限于单一输入,而是能够同时输入多种事物
\end{enumerate}


\begin{figure}[ht]
\centering
\includegraphics[width=5cm]{./figures/functi_1.png}
\caption{函数} \label{functi_fig1}
\end{figure}



%广义来说, 任何映射都可以叫做\textbf{函数(function)}。 所以我们也可以用映射的记号表示函数。 例如 $f: \mathbb R \to \mathbb R$ 表示定义域为实数集, 值为实数的函数, 通常记为 $f(x)$。 又例如 $f: \mathbb R^2 \to \mathbb C$ 表示实变量和复值的二元函数 $f(x_1, x_2)$。 注意其中 $\mathbb R^2$ 表示笛卡尔积(\autoref{Set_eq1}~\upref{Set}) $\mathbb R \times \mathbb R$。

\subsection{复合函数}
\addTODO{复合函数即复合映射}

\subsection{函数的性质}
以后我们会看到一些用极限\upref{Lim}和导数\upref{Der}描述的性质。 例如连续性\upref{contin}, 一致连续 % 未完成
, 可导。
