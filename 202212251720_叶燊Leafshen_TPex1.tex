% 双指针法运用新

\begin{issues}
\issueDraft 请不要占用他人词条不作为
\end{issues}

\begin{example}{快慢指针——链表中点}
给定一个头结点为 \textsl{h} 的非空单链表,返回链表的中间结点。

如果有两个中间结点,则返回第二个中间结点。

\textsl{ex.1}

输入:[1,2,3,4,5] 

输出:此列表中的结点 3 (序列化形式:[3,4,5])
返回的结点值为 3 。

\textsl{ex.2}

输入:[1,2,3,4,5,6]

输出:此列表中的结点 4 (序列化形式:[4,5,6])
\end{example}

\subsection{解}
\subsubsection{方法一:}
数组

我们最常做的就是遍历法了,简单,暴力,将遍历到的元素依次放入数组 \textsl{A }中。如果我们遍历到了 \textsl{n }个元素,那么链表及数组的长度也为 \textsl{n},对应的中间节点即为 \textsl{A[n/2]}。

