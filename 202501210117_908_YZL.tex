% 杨振宁(综述)
% license CCBYSA3
% type Wiki

本文根据 CC-BY-SA 协议转载翻译自维基百科\href{https://en.wikipedia.org/wiki/Yang_Chen-Ning}{相关文章}。

\begin{figure}[ht]
\centering
\includegraphics[width=6cm]{./figures/0a203240e6e08647.png}
\caption{} \label{fig_YZL_1}
\end{figure}
杨振宁(英文名:Frank Yang,亦称C. N. Yang,简体字:杨振宁,繁体字:楊振寧,拼音:Yáng Zhènníng,生于1922年10月1日),是中国的理论物理学家,对统计力学、可积系统、规范理论以及粒子物理学和凝聚态物理学做出了重要贡献。他与李政道共同获得了1957年诺贝尔物理学奖,因其关于弱相互作用的宇称不守恒的工作。两人提出,宇称守恒定律在所有其他物理过程中都成立,但在所谓的弱核反应中被打破,弱核反应是指那些导致β粒子或α粒子发射的核反应。杨振宁还因与罗伯特·米尔斯合作发展了非阿贝尔规范理论(广为人知的杨–米尔斯理论)而闻名。
\subsection{早年生活与教育}
杨振宁出生于中国安徽省合肥市。他的父亲杨克纯(1896–1973)是一位数学家,母亲罗孟华是家庭主妇。

杨振宁在北京上过小学和中学,1937年秋,随着日本侵华,他的家庭搬回了合肥。1938年,他们搬到了云南昆明,国立西南联合大学(当时的国立西南联合大学设在昆明)也在那里。同年,杨振宁作为二年级学生,通过了入学考试,并进入了国立西南联合大学学习。他于1942年获得了理学学士学位,论文题目是关于群论在分子光谱中的应用,导师为吴大任教授。

杨振宁继续在该校攻读研究生课程两年,导师为王竹溪教授,研究方向为统计力学。1944年,他获得了清华大学的理学硕士学位,清华大学当时在抗日战争期间迁至昆明。随后,杨振宁获得了由美国政府设立的“庚子赔款奖学金”,该奖学金来自中国在庚子赔款中所支付的一部分资金。杨振宁前往美国的计划延迟了一年,在这一年里,他曾在一所中学担任教师,并学习了场论。

1946年1月,杨振宁进入芝加哥大学,并跟随爱德华·泰勒(Edward Teller)教授学习。1948年,他获得了哲学博士学位。
\subsection{职业生涯}
杨振宁在芝加哥大学继续待了一年,担任恩里科·费米的助理。1949年,他受邀到位于新泽西州普林斯顿的高级研究院进行研究,并开始与李政道的富有成效的合作。1952年,他成为该研究院的正式成员,1955年晋升为正教授。1963年,普林斯顿大学出版社出版了他的教科书《基本粒子》。1965年,杨振宁搬到斯托尼布鲁克大学,在那里他被任命为阿尔伯特·爱因斯坦物理学教授,并成为新成立的理论物理研究所的首任所长。今天,这个研究所被称为C. N. 杨理论物理研究所。

杨振宁于1999年从斯托尼布鲁克大学退休,并获得名誉教授的头衔。2010年,斯托尼布鲁克大学为了表彰杨振宁对学校的贡献,将其最新的宿舍楼命名为C. N. 杨楼。

杨振宁曾当选为美国物理学会、中国科学院、中央研究院、俄罗斯科学院和英国皇家学会的会士。他还是美国艺术与科学学院、美国哲学学会和美国国家科学院的会员。杨振宁获得了普林斯顿大学(1958年)、莫斯科国立大学(1992年)和香港中文大学(1997年)授予的名誉博士学位。

1971年,杨振宁在中美关系解冻后首次访问中国大陆,随后他为帮助中国物理界重建文化大革命期间被摧毁的研究氛围作出了努力。退休后,杨振宁回到清华大学,担任名誉所长,并在清华大学先进研究中心(CASTU)担任黄纪培-卢凯群教授。他还是两位邵逸夫奖创始成员之一,并且是香港中文大学的杰出教授。

杨振宁是1989年成立的亚太物理学会(AAPPS)的首任会长。1997年,AAPPS设立了C.N.杨奖,以表彰年轻研究人员。
\subsection{个人生活}
杨振宁于1950年与杜致礼(杜致礼,字:Dù Zhìlǐ)结婚,杜致礼是一位教师;他们有两个儿子和一个女儿。岳父是国民党将领杜聿明。杜致礼于2003年10月去世。2005年1月,杨振宁再婚,妻子是大学生翁帆(翁帆,拼音:Wēng Fān)。他们于1995年在一次物理学研讨会上相识,2004年2月重新建立了联系。杨振宁称翁帆(比他小54岁)是他“上帝给予的最终祝福”。杨振宁于2015年底正式放弃了美国国籍。

2022年10月1日,杨振宁迎来了自己的百岁生日。
\subsection{学术成就}

杨振宁在统计力学、凝聚态理论、粒子物理学以及规范理论/量子场论方面做出了重要贡献。

在芝加哥大学,杨振宁最初在加速器实验室工作了二十个月,但后来发现自己并不擅长实验工作,便转向理论研究。他的博士论文主要研究核反应中的角分布。

杨振宁最为人所知的成就是1953年与罗伯特·米尔斯共同发展了非阿贝尔规范理论,即广为人知的杨-米尔斯理论。这个想法主要由杨振宁提出,而米尔斯作为一位初学者在这项工作中提供了帮助,米尔斯回忆道:

在1953至1954学年,杨振宁是布鲁克海文国家实验室的访问学者……我也在布鲁克海文,并且被分配到与杨振宁相同的办公室。杨振宁多次表现出对刚开始职业生涯的物理学家的慷慨,他向我讲述了他推广规范不变性的想法,我们讨论了很长时间……我在讨论中能做出一些贡献,尤其是在量子化过程上,也在推导形式主义方面有所贡献;然而,关键的思想是杨振宁的。

杨-米尔斯理论被《科学家》杂志称为:

当今我们理解亚原子粒子相互作用的基础,这一贡献重塑了现代物理学和数学。

此后,在过去的三十年中,许多杰出的科学家在规范理论领域取得了关键突破,推动了这一理论的发展。[citation needed]

随后,杨振宁从事粒子物理现象学的研究;他的一个著名工作是费米-杨模型,将介子π视为一个束缚的核子-反核子对。1956年,他和李政道提出在弱相互作用中,宇称对称性不守恒,随后,吴健雄团队在华盛顿国家标准局通过实验验证了这一理论。杨振宁和李政道因此获得了1957年诺贝尔物理学奖,以表彰他们提出的宇称破坏理论,这一理论对粒子物理学领域产生了革命性影响。[3] 杨振宁还与李政道合作研究了中微子理论(1957年,1959年)、CT不守恒(与李政道和R. Oheme,1957年)、矢量介子的电磁相互作用(与李政道,1962年)、CP不守恒(与吴太遵,1964年)。

在1970年代,杨振宁研究了规范理论的拓扑性质,并与吴太遵合作阐明了吴-杨单极子。与狄拉克单极子不同,吴-杨单极子没有奇异的狄拉克弦。

杨振宁自本科时期起便对统计力学有浓厚兴趣。在1950年代和1960年代,他与李政道、黄可森等合作研究了统计力学和凝聚态理论。他研究了相变理论,并阐明了李-杨圆定理、量子玻色液体的性质、二维伊辛模型、超导体中的通量量子化(与N. Byers,1961年),并提出了“非对角长程有序”(ODLRO,1962年)的概念。1967年,他发现了一个一致的条件,适用于一维分解散射的多体系统,这个方程后来被命名为杨-巴克斯特方程,它在可积模型中发挥着重要作用,并对物理学和数学的多个分支产生了深远影响。
\subsection{奖项与荣誉}
\begin{itemize}
\item 诺贝尔物理学奖(1957年)[3]  
\item 十大杰出青年美国人(1957年)[15]  
\item 拉姆福德奖(1980年)[16]  
\item 国家科学奖章(1986年)[17]  
\item 奥斯卡·克莱因纪念讲座与奖章(1988年)[18]  
\item 本杰明·富兰克林奖章(美国哲学会授予的杰出科学成就奖,1993年)[19]  
\item 鲍尔奖(1994年)[20]  
\item 阿尔伯特·爱因斯坦奖章(1995年)[21]  
\item 拉尔斯·翁萨格奖(1999年)[22]  
\item 费萨尔国王国际奖(2001年)[23]  
\item C.N. 杨大厅,位于斯托尼布鲁克大学,是一座宿舍和活动中心,于2010年落成。[24]  
\item 马塞尔·格罗斯曼奖(2015年),因其“在保罗·狄拉克和赫尔曼·外尔的最佳传统中深化\item 爱因斯坦的几何物理方法”[25]
\end{itemize}
\subsection{精选出版物}
\begin{itemize}
\item Yang, C. N. (1983). 《Selected Papers, 1945–1980, with Commentary》. San Francisco, CA: W. H. Freeman & Co. ISBN 978-0-7167-1406-4.
\item Yang, Chen-Ning (2005) [1983]. 《Selected Papers, 1945–1980, with Commentary》. World Scientific Series in 20th Century Physics. 第36卷(2005年版)。新加坡:World Scientific. ISBN 981-256-367-9.
\item Yang, Chen-Ning (2013). 《Selected Papers of Chen Ning Yang II: With Commentaries》. 新加坡:World Scientific. ISBN 978-981-4449-00-7.
\item Yang, C. N.; Mills, R. L. (1954). "Conservation of Isotopic Spin and Isotopic Gauge Invariance". 《Physical Review》, 96 (1): 191–195. Bibcode:1954PhRv...96..191Y. doi:10.1103/PhysRev.96.191.
\item Mills, R. L.; Yang, C. N. (1966). "Treatment of Overlapping Divergences in the Photon Self-Energy Function". 《Progress of Theoretical Physics Supplement》, 37: 507. Bibcode:1966PThPS..37..507M. doi:10.1143/PTPS.37.507.
\item Yang, C. N. (2005). "Remembering Robert Mills". In 't Hooft, Gerardus (ed.). 《50 years of Yang-Mills theory》. 新加坡:World Scientific. ISBN 978-981-238-934-3.
\item Lee, T. D.; Yang, C. N. (1956). "Question of Parity Conservation in Weak Interactions". 《Physical Review》, 104 (1): 254–258. Bibcode:1956PhRv..104..254L. doi:10.1103/PhysRev.104.254. S2CID 2044624.
\item Lee, T. D.; Yang, C. N. (1952). "Statistical Theory of Equations of State and Phase Transitions. II. Lattice Gas and Ising Model". 《Physical Review》, 87 (3): 410–419. Bibcode:1952PhRv...87..410L. doi:10.1103/PhysRev.87.410. ISSN 0031-9007.
\item Byers, N.; Yang, C. N. (1961). "Theoretical Considerations Concerning Quantized Magnetic Flux in Superconducting Cylinders". 《Physical Review Letters》, 7 (2): 46–49. Bibcode:1961PhRvL...7...46B. doi:10.1103/PhysRevLett.7.46.
\end{itemize}
\subsection{另见}
\begin{itemize}
\item 杨–米尔斯理论
\item 吴–杨单极子
\item 杨–巴克斯特方程
\item 杨代数
\item 宇称违反
\item 吴实验
\item 李–杨定理
\item 拜尔斯–杨定理
\item C. N. 杨理论物理研究所
\item 清华大学高级研究中心
\item 中国诺贝尔奖获得者名单
\item 理论物理学家名单
\end{itemize}
\subsection{参考文献}  
\subsubsection{引用}
\begin{enumerate}
\item Li, Bing-An; Deng, Yuefan. "C.N. Yang 传记"(PDF)。检索于 2007年9月11日。其出生日期在1945年护照中错误记录为1922年9月22日。从那时起,他在所有后续官方文件中都使用了这个错误的日期。  
\item "陈省身 - 传记"。诺贝尔奖官网。检索于 2022年6月29日。  
\item "1957年诺贝尔物理学奖"。诺贝尔基金会。检索于 2014年11月1日。  
\item "独家:新宿舍楼或以诺贝尔奖得主命名"。Thinksb.com,2010年3月18日。存档于2011年7月17日。检索于2011年5月6日。  
\item "陈省身"。美国艺术与科学院。检索于 2022年10月11日。  
\item "APS 会员历史"。search.amphilsoc.org。检索于 2022年10月11日。  
\item "陈省身"。www.nasonline.org。检索于 2022年10月11日。  
\item "AAPPS"。www.aapps.org(韩文)。检索于2024年3月26日。  
\item "AAPPS:C.N. Yang奖"。www.aapps.org(韩文)。检索于2024年3月26日。  
\item "陈省身,82岁,娶28岁女子"。中国日报,2014年12月16日。  
\item "两位顶级中美科学家放弃美国国籍"。科学,2017年2月24日。  
\item "庆祝陈省身100岁"。www.ias.edu,2022年9月21日。检索于2022年9月30日。  
\item Gray, Jeremy; Wilson, Robin(2012年12月6日)。《数学对话:来自《数学智囊》的精选》。Springer Science & Business Media,第63页。ISBN 9781461301950。  
\item Ravo, Nick(1999年10月2日)。"Robert L. Mills, 72岁,亚原子物理领域的理论家"。《纽约时报》。  
\item Richards, Allen(1957年)。"陈省身和1957年美国青年杰出十人奖其他获得者"。高级研究所。检索于 2022年6月29日。
\item "往届奖项"。美国艺术与科学院。检索于2022年6月29日。  
\item "总统国家科学奖:获奖者详情 | NSF - 美国国家科学基金会"。www.nsf.gov。美国国家科学基金会。检索于2022年6月29日。  
\item "早期讲座 - 奥斯卡·克莱因中心"。www.okc.albanova.se。斯德哥尔摩大学。存档于2020年11月4日。检索于2022年6月29日。  
\item "本杰明·富兰克林奖:杰出科学成就奖获奖者"。美国哲学学会。检索于2011年11月26日。  
\item "陈省身"。富兰克林研究所,2014年1月15日。检索于2022年6月29日。  
\item "爱因斯坦学会"。www.einstein-bern.ch。检索于2022年6月29日。  
\item "1999年拉尔斯·昂萨格奖获奖者"。www.aps.org。美国物理学会。检索于2022年6月29日。  
\item "费萨尔王奖 - 陈省身教授"。kingfaisalprize.org。费萨尔王奖。检索于2022年6月29日。  
\item "C.N.杨厅落成典礼"。斯托尼布鲁克大学,2010年4月23日。检索于2022年11月27日。  
\item MG14,马赛尔·格罗斯曼奖,2015年罗马 ICRANet和ICRA。检索于2024年4月28日。
\end{enumerate}
\subsubsection{来源}  
\begin{itemize}
\item 杨振宁,C. N. (1952). 《统计力学的特殊问题》。华盛顿州西雅图:华盛顿大学出版社。ASIN B0007FZHH4。  
\item 李政道,T. D. 和 杨振宁,C. N. 《基本粒子与弱相互作用》,布鲁克海文国家实验室(BNL),美国能源部(通过前身机构原子能委员会),(1957)。  
\item 杨振宁,C. N. 《多体问题。物理学专著第6期》,里约热内卢。巴西物理研究中心,(1960)。  
\item 杨振宁,C. N. (1963) [1961]. 《基本粒子:原子物理学中的一些发现简史》。新泽西州普林斯顿:普林斯顿大学出版社。ASIN B000E1CBGG。  
\item 杨振宁,C. N. 《高能总横截面的一些规则的数学推导》,布鲁克海文国家实验室(BNL),美国能源部(通过前身机构原子能委员会),(1962)。  
\item 杨振宁,C. N. 《物理学中的对称性原理。布鲁克海文讲座系列第50期》,布鲁克海文国家实验室(BNL),美国能源部(通过前身机构原子能委员会),(1965年10月13日)。  
\item 杨振宁,C. N. (1983). 《选集 1945–1980,附评论》(杨振宁)。加利福尼亚州旧金山:W.H. Freeman出版社。ISBN 978-0-7167-1406-4。  
\item “C.N.杨理论物理研究所(YITP)”。检索于2008年1月5日。  
Sutherland, Bill (2004). 《美丽的模型》。新加坡:世界科技出版社。ISBN 978-981-238-859-9。
\end{itemize}
\subsection{外部链接}
\begin{itemize}
\item 陈宁杨教授访谈,约翰·以色列(John Israel)和拉里·施奈德(Larry Schneider)于1980年11月1日,尼尔斯·玻尔图书馆与档案馆,美国物理学会,马里兰州大学园区,美国  
\item 陈宁杨教授(主页 – 清华大学高级研究院)  
\item 陈宁杨(主页 – 斯托尼布鲁克大学)  
\item C.N.杨主页[永久失效链接](主页 – 香港中文大学)  
\item 陈宁杨在诺贝尔奖网站上的页面 编辑此条目于维基数据,包括1957年12月11日诺贝尔讲座《宇称守恒定律与其他物理学对称性法则》  
\item 肖奖,结构 于2016年10月31日在时光机中归档(主页 – 肖奖)  
\item 对称性与反射(C.N.杨退休座谈会,斯托尼布鲁克大学)  
\item CN杨学者项目 于2011年6月7日在时光机中归档,南洋理工大学新加坡 2011年4月22日归档  
\item 《陈宁杨访谈》. 西蒙斯基金会. 2011年12月20日
\end{itemize}