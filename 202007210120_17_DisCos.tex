% 宇宙中的距离

\pentry{FRW 度规,宇宙学红移}

\subsection{非可观测距离}
为了有利于定义可观测距离,我们首先定义非观测距离.也称为度规距离.FRW度规可以写成以下形式
\begin{equation}
ds^2=-dt^2+a(t)^2(d\chi^2+S^2_k (\chi) d\Omega^2),
\end{equation}
其中
[xxx].

\textbf{度规距离}可以被定义为空间度规中立体角的因子
\begin{equation}
d_m=S_k(\chi).
\end{equation}

当取平直宇宙的情况,即$k=0$时,度规距离可退化为\textbf{共动距离}(comoving distance)$\chi$. 共动距离可以用宇宙学红移因子$z$和哈勃常数$H(z)$来表达,具体可写成
\begin{equation}
\chi(z)=\int^{t_0}_{t_1} \frac{dt}{a(t)}=\int^z_0 \frac{dz}{H(z)}.
\end{equation}
这里需要强调,度规距离和共动距离都是非可观测距离.
