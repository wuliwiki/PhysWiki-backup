% Homebrew 笔记
% license Xiao
% type Note

\begin{issues}
\issueDraft
\end{issues}

\begin{itemize}
\item Homebrew 是 macOS 系统上类似于 apt 的包管理程序(另一个是 \verb|MacPorts|,但相对没那么流行)。
\item \href{https://brew.sh}{官网 brew.sh}
\item 安装命令: \verb|/bin/bash -c "$(curl -fsSL https://raw.githubusercontent.com/Homebrew/install/HEAD/install.sh)"|
\item \verb`brew --version` 查看版本
\item 安装完后根据提示运行 \verb`(echo; echo 'eval "$(/opt/homebrew/bin/brew shellenv)"') >> /Users/addis/.zprofile &&
eval "$(/opt/homebrew/bin/brew shellenv)"`
\item \href{https://github.com/Homebrew/homebrew-core/tree/master/Formula}{在线搜索}可用的包
\item \verb`brew update` 可以更新 app 索引
\item \verb`brew install 包名` 安装某个包
\item \verb`brew install coreutils` 安装 linux 的一些基础软件, 例如 \verb`sha1sum`,但为了区分 GNU 软件,实际命令是 \verb`gsha1sum`
\item \verb`brew info 包名` 可以查看包的信息(如版本)
\item \verb`brew --prefix 包名` 可用于查看某个包的安装路径。 
\item 每个包的所有版本都会存在 \verb`homebrew/Cellar/包名/版本号/` 中,也会有 symlink 在其他地方如 \verb`homebrew/opt/包名`。
\end{itemize}
