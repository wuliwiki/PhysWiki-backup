% 泰勒公式(综述)
% license CCBYSA3
% type Wiki

本文根据 CC-BY-SA 协议转载翻译自维基百科\href{https://en.wikipedia.org/wiki/Taylor\%27s_theorem}{相关文章}。

\begin{figure}[ht]
\centering
\includegraphics[width=10cm]{./figures/a2db9ae0e729b15e.png}
\caption{指数函数 \( y = e^x \)(红色)及其在原点附近的四阶泰勒多项式(绿色虚线)。} \label{fig_TLGS_1}
\end{figure}
在微积分中,泰勒定理给出了一个\( k \)次可导函数在某个给定点的近似,通过一个\( k \)次多项式,称为\( k \)阶泰勒多项式。对于一个光滑函数,泰勒多项式是该函数泰勒级数在 \( k \)阶的截断。一级泰勒多项式是该函数的线性近似,二级泰勒多项式通常称为二次近似\(^\text{[1]}\)。泰勒定理有多个版本,其中一些版本给出了函数通过其泰勒多项式近似的误差的明确估计。

泰勒定理以数学家布鲁克·泰勒命名,他在1715年提出了该定理的一个版本,\(^\text{[2]}\)尽管早在1671年,詹姆斯·格雷戈里就已提到过该结果的早期版本\(^\text{[3]}\)。

泰勒定理在初级微积分课程中教授,是数学分析中的一个核心基本工具。它提供了简单的算术公式,用于准确计算许多超越函数的值,如指数函数和三角函数。它是解析函数研究的起点,并在数学的各个领域、数值分析以及数学物理中具有基础性意义。泰勒定理也可以推广到多变量和向量值函数。它为一些开创性的早期计算机提供了数学基础:查尔斯·巴贝奇的差分机通过数值积分其泰勒级数的前七项来计算正弦、余弦、对数和其他超越函数。
\subsection{动机}
