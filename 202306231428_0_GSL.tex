% GNU Scientific Library
% 科学计算|GNU|GSL|C 语言

\begin{issues}
\issueDraft
\issueMissDepend
\end{issues}

GNU Scientific Library (GSL) 是一个 GNU 开源项目, 由 C 语言写成。

\subsection{Linux 安装使用}
在 Debian/Ubuntu 系统上可执行 \verb|sudo apt install libgsl-dev| 安装。

第二种方法是自己下载源码编译。

\href{https://coral.ise.lehigh.edu/jild13/2016/07/11/hello/}{参考资料}。 gsl 通常的使用方法是先安装,生成 include 文件夹 (含有所有头文件) 和 lib 文件夹 (.so 文件以及 .a 文件). 先下载最新版安装包, 目前是 \verb`gsl-2.5.tar.gz`, 解压命令 \verb`tar -xvzf gsl-2.5.tar.gz`, 得到 gsl-2.5 文件夹, \verb`cd gsl-2.5`, 然后 \verb`./configure --prefix=安装目录` 其中安装目录只能是绝对目录. 然后 \verb`make`, 安装好以后 \verb`make check` 检查, 然后 \verb`make install` 即可. 只有最后一步会在安装目录中生成文件.

\begin{lstlisting}[language=cpp]
// example.c
#include <stdio.h>
#include <gsl/gsl_sf_bessel.h>

int main (void) {
    double x = 15.0;
    double y = gsl_sf_bessel_J0 (x);
    printf ("J0(%g) = %.18e\n", x, y);
    return 0;
}
\end{lstlisting}

要编译, 用 \verb|gcc -Wall -I 安装目录/include/ -c example.c|

要 link, 用
\begin{lstlisting}[language=cpp]
gcc test.o -Wl,-rpath,安装目录/lib -L 安装目录/lib -lgsl -lgslcblas -lm
\end{lstlisting}
其中 gsl 包含所有 gsl 的函数, gslcblas 提供了一个 cblas, 但也可以用其他更优化的如 ATLAS 提供的, m 是编译器的 math library.

以下试图使用 apt 直接安装 gsl
\begin{itemize}
\item 使用 \verb`sudo apt-get install libgsl` 再按两次 tab 即可显示所有相关包
\item 安装完后用 \verb`locate libgsl.so` 或者 \verb`locate libgsl` 就能找到所有 \verb`so` 文件(貌似找不到 \verb`a` 文件), 只好放弃
\end{itemize}

\subsubsection{源码再利用}
gsl 完全是 c 语言写的, 所以 cuda 应该也是以用的, 但是不能用编译好的 a 和 so, 而是要重新由源码编译. 即使不用 cuda, 直接使用源码也可以让代码更 portable.

用 understand 分析了一下 \verb`bessel` 和 \verb`3j` symbol 两个函数, 发现如果不修改代码, 要找到编译某个函数需要的所有文件真的很难, 至少有几十个文件. 其中有一个比较特殊的文件是 \verb`config.h`, 应该是要运行安装包的 config 以后才会生成的.

我觉得可以把所有的头文件和 c 文件都放在一个文件夹里面, 然后做一个类似 \verb|SCID_TDSE| 的 makefile, 自动分析出 dependency. 不过 c 和 fortran 不同, 要做到自动还是很有难度的. 不知道 understand 有没有这个功能. 如果不行, 干脆就所有 c 文件都编译? 又或许安装包里的 makefile 已经有 dependency list 了.

然后又发现 gsl 安装的就是所谓的 two-step process: configure followed by make.

用 git 对比了一下安装过程的源码, 发现安装过程并不会改变任何源码(.c 和 .h 文件), 只会根据系统生成 "config.h", 其实也可以直接手动修改该文件, 主要是一些宏的定义和取消, 注释非常易懂. 所以理论上, 只要把所有源码文件都放到一起编译就可以了.


\subsection{Visual Studio 编译 GSL}
\begin{itemize}
\item 其实与其自己编译,可以看看 MSYS2\upref{Mingw} 里面安装 GSL 后的 dll 能不能拿来直接用
\item 也可以直接到 \href{https://github.com/MacroUniverse/GSL-bin}{GSL-bin repo} 中复制头文件, lib 和 dll 文件
\end{itemize}

以下介绍如何使用用 CMake 和 Visual Studio 2017 编译 lib 和 dll 以及测试文件(貌似测试的时候有一项会 fail)

\begin{itemize}
\item Windows 安装 Cmake
\item 首先下载 GSL \href{https://github.com/ampl/gsl}{源码}(含 CMake)
\item 打开 CMakeLists.txt 文件, 会提示各个编译器选项怎么设置, 如
\begin{lstlisting}[language=none]
    cmake -G"Visual Studio 15 2017 Win64" -DGSL_INSTALL_MULTI_CONFIG=ON \
          -DBUILD_SHARED_LIBS=ON -DMSVC_RUNTIME_DYNAMIC=ON \
          <path to gsl sources>
\end{lstlisting}
\item 在 PowerShell 里面运行这个貌似会出错, 最稳妥的办法还是用 CMake 的 GUI 界面。 输入源码路径和输出路径, 然后 Configure。 选择 \verb`x64`, \verb`Visual Studio 2017`, \verb`native generator`。 成功以后按照上面的选项勾选 \verb`GSL_INSTALL_MULTI_CONFIG=ON` \verb`BUILD_SHARED_LIBS=ON` \verb`MSVC_RUNTIME_DYNAMIC=ON`, 然后点 generate 即可。
\item 在输出路径找到 \verb`sln` 文件, 双击打开 Visual Studio 2017, 选 Release, 确认是 x64, 然后在菜单中 Build -> All 即可。

\item 成功之后, dll 文件在 \verb`bin/Release/gsl.dll`, lib 文件在 \verb`Release/gsl.lib`, 头文件在 \verb`gsl` 中。
\item 要测试, 在 Power Shell 里面 cd 到 CMake 输出目录下, \verb`CTest -VV` 即可 (\verb`-VV` 是 extra verbose)
\end{itemize}

\subsection{使用部分源码}
\begin{itemize}
\item \href{https://github.com/MacroUniverse/SLISC}{SLISC} 的源码中实现了自己从部分 GSL 源码直接编译。 供参考。
\end{itemize}

\subsection{异常处理}
如果要临时关闭一些 trivial 的异常(比如 exp(-x) 对于大 x 会 underflow 异常), 用 \verb|gsl_set_error_handler_off()|
要恢复默认的异常处理, 用 \verb|gsl_set_error_handler(NULL)|
详见\href{https://www.gnu.org/software/gsl/doc/html/err.html}{这里}。
