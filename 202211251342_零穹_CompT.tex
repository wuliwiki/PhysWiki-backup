% 比较定理
% 比较定理

比较定理可以形象的描述为:若甲乙两人在一直线上跑步,在直线上每一点,甲的速度都比乙的速度慢, 且在某一时刻 $t_0$,甲乙两人相遇,那么在相遇前,甲始终跑在乙前面,而在相遇后,甲则始终跑在乙的后面.

这一定理几乎是显然的,但是应该注意在给定的时刻,甲的速度可能比乙快.
\begin{theorem}{比较定理}
若 $v_1,v_2$ 是实轴区间 $U$ 上定义的两实连续函数,且 $v_1<v_2$.又设 $\varphi_1,\varphi_2$ 分别是微分方程
\begin{equation}
\dv{\varphi_1}{t}|_{t=\tau}=v_1(\varphi_1(\tau)), \quad \dv{\varphi_2}{t}|_{t=\tau}=v_2(\varphi_2(\tau))
\end{equation}
的解.其中 $\varphi_1,\varphi_2$ 是将区间 $(a,b)\;(-\infty\leq a<b\leq+\infty)$ 映射到 $U$ 上的函数.那么
\begin{equation}
\varphi_1(t)\leq\varphi_2(t)
\end{equation}
对于区间 $(a,b)$ 内的一切 $t\geq t_0$ 都成立.
\end{theorem} 
