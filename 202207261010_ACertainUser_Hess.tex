% 盖斯定律与设计路径
% 赫斯 赫士 盖斯 Hess 路径 状态量变化

\subsection{盖斯定律}
\begin{theorem}{盖斯定律}
(等压或等容系统中)一个反应,不管是一步完成的还是多步完成的,其热效应总是相同.
\end{theorem}
盖斯定律最早是源于实验观察的经验结论,可以理解为能量守恒在热力学中的另一种体现.

\begin{example}{碳的燃烧}
\begin{figure}[ht]
\centering
\includegraphics[width=8cm]{./figures/Hess_1.png}
\caption{C的两种氧化路径} \label{Hess_fig1}
\end{figure}
一定量的C被氧化为CO2,无论是直接被氧化为CO2,还是先被不完全氧化生成CO、再由CO被氧化为CO2,二者放出的总热量应该相同.
\end{example}

现在我们知道,盖斯定律有着更为基本、深刻的物理含义.根据状态量与路径无关的特性\upref{StaPro}\upref{statef},我们知道无论系统

