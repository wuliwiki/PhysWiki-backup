% 静电场的应用(高中)
% 电容|静电|带电粒子的运动

\begin{issues}
\issueDraft
\issueTODO
\end{issues}

\pentry{静电场\upref{HSPE01}}

\subsection{电容器}

\textbf{电容器}是用来储存电荷的电学元件.电容器由两个相互靠近、彼此绝缘的导体组成,两导体间可填充绝缘物质——\textbf{电介质}(空气也是一种电介质).最简单的电容器由两块相距很近的平行金属板(极板)组成,叫做平行板电容器.实际上,任何彼此绝缘又相隔很近的导体,都可以看成一个电容器.

电路中,电容器用字母$C$表示,符号如\autoref{HSPE02_fig1} .

\begin{figure}[ht]
\centering
\includegraphics[width=5cm]{./figures/HSPE02_1.png}
\caption{固定电容器(左)和可变电容器(右)} \label{HSPE02_fig1}
\end{figure}

\subsubsection{电容器的充电和放电}

使得两个极板分别带上等量异种电荷的过程,就是电容器的\textbf{充电}过程.充电过程中,流入正极板的电流逐渐减小,电容器所带的电荷量逐渐增加,电容器两极板之间的电场强度逐渐增强,板间电压也逐渐升高,电容器从电源获得的电能转化为电容器中的电场能.充电结束后,电容器两极板间电压与充电电压相等.

使电容器两极板的异种电荷中和的过程,就是电容器的\textbf{放电}过程.放电过程中,电流从正极板流出且逐渐减小,电容器所带的电荷量逐渐减少,电容器两极板间的电场强度逐渐减弱,板间电压也逐渐降低,电容器中的电场能转化为其他形式的能量.

\begin{figure}[ht]
\centering
\includegraphics[width=13cm]{./figures/HSPE02_2.png}
\caption{电容器的充电(左)和放电(右)} \label{HSPE02_fig2}
\end{figure}

\subsubsection{电容}

电容器所带电量Q与电容器两极板间电压之比,叫做电容器的\textbf{电容},用$C$表示,即
\begin{equation}
C=\frac{Q}{U}
\end{equation}

在国际单位中,电容的单位是\textbf{法拉},简称\textbf{法},符号为$\mathrm{F}$.

常用的电容单位有\textbf{微法}($\mathrm{\mu F}$)和\textbf{皮法}($\mathrm{pF}$).$1\mathrm{F}=10^6 \mathrm{\mu F}=10^{12} \mathrm{pF}$.

电容是表示电容器容纳电荷本领的物理量.

\subsubsection{电容的串联和并联\footnote{详细证明见“电容的串联和并联\upref{Ccomb}”}}

$n$个电容器并联时,有
\begin{equation}
C=\sum_{i=1}^{n}C_i
\end{equation}

$n$个电容器串联时,有
\begin{equation}
\frac{1}{C}=\sum_{i=1}^{n}\frac{1}{C_i}
\end{equation}

\subsubsection{平行板电容器}

平行板电容器的电容为
\begin{equation}\label{HSPE02_eq1}
C = \frac {\epsilon_r S}{4\pi kd}
\end{equation}

\autoref{HSPE02_eq1} 中的$S$和$d$分别为电容器两极板的正对面积和板间距离;$k$为静电力常量(\autoref{HSPE01_eq8}~\upref{HSPE01});$\epsilon_r$为相对介电常数,真空的$\epsilon_r$值为$1$.
