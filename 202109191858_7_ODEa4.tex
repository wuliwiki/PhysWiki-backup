% 一阶隐式常微分方程
% keys 隐式方程|ODE|differential euqation

\pentry{一阶常微分方程解法:常数变易法\upref{ODEa2},一阶常微分方程解法:恰当方程\upref{ODEa3}}

我们之前讨论的常微分方程都是显式写出导函数表达式的,即$\frac{\dd y}{\dd x}=F(x, y)$的形式.很多时候,一阶微分方程常被写为$F(x, y, \frac{\dd y}{\dd x})=0$ 的形式,如果这样的方程可以被改写为显式的形式,那么我们就可以尝试用预备知识中介绍过的方法来解方程;但如果难以改写或者解出来的形式极为复杂,那我们也可以尝试\textbf{换元}的方法.

本节介绍四种一阶隐式方程和它们的换元方法.

\subsection{第一种}

第一个要讨论的是形如
\begin{equation}
y=f(t, \frac{\dd y}{\dd t})
\end{equation}
的方程.这里自变量用的是通常代表时间的$t$,为的是提示该怎么换元——如果$y$是位移,那$\dd y/\dd t$就是速度,这就是我们要的变换.

令$v=\frac{\dd y}{\dd t}$,则原方程变为$y=f(t, v)$.在方程两边同时对$t$求导,得到
\begin{equation}
v=\frac{\partial}{\partial t}f(t, v)
\end{equation}




























