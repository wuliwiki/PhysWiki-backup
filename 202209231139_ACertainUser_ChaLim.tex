% 求极限的一些方法

\begin{issues}
\issueDraft
\end{issues}

求解极限是高数考试中的重要课题(虽然在实际中,这些工作往往交给电脑完成).本文简要介绍一些求解极限的基本方法与思路.当然,极限的问题千变万化,你不能指望机械地运用几种方法就能解决所有问题,还是需要你的一丝灵性与\textsl{多做题}

\subsection{预备知识}
这里列举一些关于极限计算的基本知识.
\begin{itemize}
\item 极限的四则运算法则
\item 极限与函数连续性
\item 极限的有理运算性质
\item 导数的定义
\item 有界函数*无穷小=0
\item 洛必达法则
\item 泰勒展开
\item 等价无穷小
\item 几个常用的等式与不等式
\end{itemize}

\subsection{拼凑与代换}
\textsl{有借有还,再借不难}

拼凑与代换的核心思路是变形待求的极限式,以配凑出容易解决的形式.例如,加一项再减去相同的一项、乘一项再除以相同的一项、提取一个公因式、代换变量...

当然,为了运用拼凑与代换,你先得有个目标,明确“我想凑出什么样的形式”.这需要你知晓各种常用形式.

\subsection{分式与根式}
\begin{itemize}
\item 有理化.常用于出现根式相加减的情况.
\item 上下同除以最高次的项
\end{itemize}

\subsection{不定式加减}
如果两个加数均为不定式($\infty+\infty$),将他们通分,并化为一个分式.

\subsection{幂指函数}
若待求极限含有幂指函数,则考虑对其运用指数与对数$f(x)=\E^{\ln(f(x))}$

\subsection{形式相近的项}
可以运用Lagrange中值定理与积分中值定理.