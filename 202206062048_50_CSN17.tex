% 2017 年计算机学科专业基础综合全国联考卷
% keys 2017 年计算机 全国 联考卷


\subsection{一、单项选择题}
1~40小题,每小题2分,共80分.下列每题给出的四个选项中,只有一个选项符合题目要求.

1.下列函数的时间复杂度是 \\
\begin{lstlisting}[language=cpp]
int func(int n)
{
    int i=0,sum=0;
    while(sum<n)
        sum+=++i;
    return i;
}
\end{lstlisting}
A. $O(log n)$  B.$O(n^{l/2})$    C.$0(n)$    D.$O(nlogn)$
    
2.下列关于栈的叙述中,错误的是 \\
I.采用非递归方式重写递归程序时必须使用栈 \\
II.函数调用时,系统要用栈保存必要的信息 \\
III.只要确定了入栈次序,即可确定出栈次序 \\
Ⅳ,栈是一种受限的线性表,允许在其两端进行操作 \\
A.仅I  $\quad$  B.仅I、II、III \\
C.仅I、Ⅲ、Ⅳ  $\quad$  D.仅II、III、Ⅳ

3.适用于压缩存储稀疏矩阵的两种存储结构是 \\
A.三元组表和十字链表 $\quad$ B.三元组表和邻接矩阵 \\
C.十字链表和二叉链表 $\quad$ D.邻接矩阵和十字链表

4.要使一棵非空二叉树的先序序列与中序序列相同,其所有非叶结点须满足的条件是 \\
A.只有左子树  $\quad$  B.只有右予树 \\
C.结点的度均为1 $\quad$ D.结点的度均为2

5.己知一棵二叉树的树形如下图所示,其后序序列为e,a,c,b.d,g,f,树中与结点a同层的结点是 \\
\begin{figure}[ht]
\centering
\includegraphics[width=5cm]{./figures/CSN17_1.png}
\caption{第5题图} \label{CSN17_fig1}
\end{figure}
A.C $\quad$ B.d $\quad$ C.f $\quad$ D.g

6.己知字符集{a,b,c,d,e,f,g,h},若各字符的哈夫曼编码依次是0100, 10, 0000, 0101, 001, 011, 11, 0001,则编码序列010001 100100101 1 1 10101的译码结果是 \\
A. acgabfh  $\quad$  B. adbagbb  \\
C. afbeagd  $\quad$  D. afeefgd

