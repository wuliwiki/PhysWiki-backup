% 蒙特卡洛树搜索算法
% keys 蒙特卡洛|搜索
% license Usr
% type Tutor

\begin{issues}
\issueDraft
\issueTODO
\end{issues}

% 暂且把 UCB 与 蒙特卡洛树搜索算法 合并,如果需要可以单开一篇详细介绍 UCB。

蒙特卡洛树是一种不同于 Alpha-Beta 剪枝的优化搜索方法。他与 Alpha-Beta 剪枝相比,更好的适用于诸如围棋一类的大型对弈游戏(即搜索空间过大,但搜索过程需要兼顾“探索”与“最优化”,尽量得到全局最优解,避免陷入局部最优解)。但与 Alpha-Beta 剪枝相比,由于过程中采用蒙特卡洛方法(可以理解为抽样检测),搜索的结果并不保证一定是最优解。

值得一提的是,\textbf{蒙特卡洛树搜索算法不是蒙特卡洛算法}。

蒙特卡洛树搜索限定于解决满足以下要求的博弈问题。
\begin{enumerate}
\item 场景必须是能分出输赢(不能同时赢);
\item 游戏的信息是完全公开的(不像打牌可以隐藏自己的手牌);
\item 确定性的(每一个操作结果没有随机因素;
\item 顺序的(操作都是按顺序执行的);
\item 离散的(没有操作是一种连续值)。
\end{enumerate}

\subsection{UCB 算法}
UCB,upper confidence bound,置信度上界,是蒙特卡洛树搜索中启发式搜索的一种常用方式,下面先来考察 UCB 算法。

简单来说采用 UCB 的蒙特卡洛树搜索是一种(类)启发式的搜索,大体思路是:在选择子节点的时候,
\begin{enumerate}
\item 优先考虑未搜索过的;
\item 如果都搜索过就根据给每个节点的“赋分”来选择,这个“赋分”是与这个节点的最终获胜概率正相关、与这个子节点访问的的次数负相关的。
\end{enumerate}
我们可以看到第 $2$ 条就是为了兼顾“探索”与“优化”。与最终获胜概率正相关为了保证“优化”、与访问次数负相关为了兼顾“探索”。

