% 牛顿—莱布尼兹公式(复变函数)

\begin{issues}
\issueDraft
\end{issues}

\pentry{柯西—黎曼条件\upref{CauRie}}


而柯西—黎曼条件\upref{CauRie}恰好规定了这两个矢量场的旋度为零, 所以如果 $f(z)$ 在考虑的区域上解析, \autoref{AnaInt_eq1} 中的线积分结果只和起点和终点有关, 与路径无关(所有的路径必须在解析的区域内). 于是可以得到类似于牛顿—莱布尼兹公式\upref{NLeib}, 令 $\bvec f_R, \bvec f_I$ 的势函数分别为 $F_R, F_I$, 即
\begin{equation}
\grad F_R = f_R
\qquad
\grad F_I = f_I
\end{equation}
再令
\begin{equation}
F(\bvec r) = F_R(\bvec r) + \I F_I(\bvec r)
\end{equation}
有
\begin{equation}\label{AnaInt_eq2}
\int_{z_1}^{z_2} f(z) \dd{z} = F(z_2) - F(z_1)
\end{equation}
对于任意路径成立. 容易证明 $F(z)$ 就是 $f(z)$ 的原函数, 即
\begin{equation}
F'(z) = f(z)
\end{equation}
这相当于对\autoref{AnaInt_eq2} 实部和虚部分别使用梯度定理\upref{Grad}.
