% 霍尔效应
% keys 霍尔效应|电势差|Hall|磁场|霍尔电势差
% license Xiao
% type Tutor

\begin{issues}
\issueDraft
\end{issues}

\footnote{参考 Wikipedia \href{https://en.wikipedia.org/wiki/Hall_effect}{相关页面}。}1879 年霍尔(E. C. Hall)首先观察到,把一载流导体薄片放在磁场中时,如果磁场方向垂直于薄片平面,则在薄片的上、下两侧面会出现微弱的电势差。这一现象称为\textbf{霍尔效应(Hall effect)}。此电势差称为\textbf{霍尔电势差}。实验测定,霍尔电势差的大小与电流 $I$ 及磁感应强度 $B$ 成正比,而与薄片沿 $\mathbf B$ 方向的厚度 $d$ 成反比。它们的关系可写成:
\begin{equation}
V = R_{\mathrm{H}} \frac{I B}{d}~.
\end{equation}
其中 $R_H$ 是\textbf{霍尔系数(Hall coefficient)}, 等于自由电子电荷量的体密度 $nq$ 的倒数(\autoref{eq_Hallef_1} )。

\begin{figure}[ht]
\centering
\includegraphics[width=8.5cm]{./figures/f6c43fb7857c45fc.png}
\caption{霍尔效应示意图(来自维基百科)} \label{fig_Hallef_1}
\end{figure}

\subsection{霍尔电阻的推导}
平衡时电场力与洛伦兹力相等
\begin{equation}
Eq = vBq~,
\end{equation}
令 $\rho$ 为自由电子的电荷密度
\begin{equation}
E = jB/(nq)~.
\end{equation}
两端电压为
\begin{equation}\label{eq_Hallef_1}
V = IB/(d n q)~,
\end{equation}
\textbf{霍尔电阻(Hall resistance)}
\begin{equation}
R = V/I = B/(d n q)~.
\end{equation}

\subsection{Drude 模型与经典霍尔效应}
下面我们从 Drude 模型的视角看待经典霍尔效应,并推导出它当对\textbf{二维材料}加 $z$ 方向的磁场时材料的电导率 $\sigma$ 与电阻率张量 $\rho$ \footnote{电阻率的定义见\autoref{eq_Resist_7}~\upref{Resist}。}的理论计算结果。 $\sigma$ 与 $\rho$ 的定义为
\begin{equation}
\begin{aligned}
&\bvec J=\sigma \bvec E,\quad \sigma=\pmat{
    \sigma_{xx} & \sigma_{xy}\\
    -\sigma_{yx} & \sigma_{yy}
}
~,\\
&\bvec E=\rho \bvec J,\quad \rho = \sigma^{-1} =
\pmat{\rho_{xx}&\rho_{xy}\\-\rho_{yx} & \rho_{yy}}~,\\
&\bvec E=\pmat{E_x\\ E_y},\quad \bvec J = \pmat{J_x\\J_y}~.
\end{aligned}
\end{equation}

霍尔效应表明材料可能存在横向的电阻 $\rho_{xy}$,即如果我们通 $x$ 方向的电流,材料 $y$ 方向会产生霍尔电势差。