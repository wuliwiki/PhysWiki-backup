% 黎曼-斯蒂尔吉积分
% keys 实分析|测度论|有界变差
% license Usr
% type Wiki

黎曼-斯蒂尔切斯积分(Riemann-Stieltjes Integral),是黎曼积分的一种推广形式,它允许对一个函数与另一个具有一定关系的函数之间的积分进行定义。

黎曼-斯蒂尔切斯积分是一种将两个在有限区间 $ [a, b] $ 上定义并有限的函数 $ f $ 和 $ \phi $ 结合起来的方法。设 $ \Gamma=\{a=x_{0}<x_{1}<\cdots<x_{m}=b\} $ 是 $ [a, b] $ 的一个划分,我们任意选择中间点 $ \{\xi_{i}\}_{i=1}^{m} $ 满足 $ x_{i-1}\leq\xi_{i}\leq x_{i} $,然后定义:

\begin{align}
R_{\Gamma}=\sum_{i=1}^{m}f(\xi_{i})[\phi(x_{i})\,-\,\phi(x_{i-1})]~.
\end{align}

$ R_{\Gamma} $ 被称为 $ \Gamma $ 的一个\textbf{黎曼-斯蒂尔切斯和},当然,它取决于点 $ \xi_{l} $、函数 $ f $ 和 $ \phi $ 以及区间 $ [a, b] $,尽管我们通常在符号中不显示这种依赖关系。
如果

\begin{align}
I=\lim_{|\Gamma|\to0}R_{\Gamma}~.
\end{align}

存在且有限,即,如果对于任意 $ \varepsilon>0 $ 存在 $ \delta>0 $ 使得对于任意满足 $ |\Gamma|<\delta $ 的 $ \Gamma $,都有 $ |I-R_{\Gamma}|<\varepsilon $,那么 $ I $ 关于 $ \phi $ 在 $ [a, b] $ 上对 $ f $ 的\textbf{黎曼-斯蒂尔切斯积分},并表示为
\begin{align}
I=\int_{a}^{b}\!f(x)\;d\phi(x)=\int_{a}^{b}\!f\,d\phi~.
\end{align}

黎曼-斯蒂尔切斯积分存在的一个必要且充分条件是,对于任意 $ \varepsilon>0 $ 存在 $ \delta>0 $,使得如果 $ |\Gamma|,|\Gamma^{\prime}|<\delta $,则 $ |R_{\Gamma}-R_{\Gamma}|<\varepsilon $。

\begin{theorem}{}
1. 如果 $\int_{a}^{b}f\ d\phi$ 存在,那么对于任意常数 $c$,$\int_{a}^{b}cf\ d\phi$ 和 $\int_{a}^{b}f\ d(c\phi)$ 也存在,且有

\begin{align} \int_{a}^{b}cf\ d\phi=\int_{a}^{b}f\ d(c\phi)=c\int_{a}^{b}f\ d\phi~. \end{align}

2.
如果 $\int_{a}^{b}f_{1},d\phi$ 和 $\int_{a}^{b}f_{2}\,d\phi$ 都存在,那么 $\int_{a}^{b}(f_{1}+f_{2})\,d\phi$ 也存在,且有

\begin{align} \int_{a}^{b}(f_{1}+f_{2})\,d\phi=\int_{a}^{b}f_{1}\,d\phi+\int_{a}^{b}f_{2}\,d\phi~. \end{align}

3.如果 $ \int_{a}^{b}fd\phi_{1} $ 和 $ \int_{a}^{b}fd\phi_{2} $ 都存在,那么 $ \int_{a}^{b}fd(\phi_{1}+\phi_{2}) $ 也存在,并且有 \begin{align} \int_{a}^{b}fd(\phi_{1}+\phi_{2})=\int_{a}^{b}fd\phi_{1}+\int_{a}^{b}fd\phi_{2}~. \end{align}
\end{theorem}

这里的证明比较简单,由定义可直接得出,于此仅证明2.为例:
根据黎曼-斯蒂尔切斯积分的定义,我们有:
   \[
   \int_{a}^{b}cf\,d\phi = \lim_{||\Gamma|| \to 0} \sum_{i=1}^{n} cf(c_i)(\phi(x_i) - \phi(x_{i-1}))~.
   \]
   \[
   \int_{a}^{b}f\,d(c\phi) = \lim_{||\Gamma|| \to 0} \sum_{i=1}^{n} f(c_i)(c\phi(x_i) - c\phi(x_{i-1}))~.
   \]
   由于 $c$ 是常数,我们可以将其从求和中提取出来,得到
\[
   = c\lim_{||\Gamma|| \to 0} \sum_{i=1}^{n} f(c_i)(\phi(x_i) - \phi(x_{i-1})) = c\int_{a}^{b}f\,d\phi~.
   \]
   因此,$\int_{a}^{b}cf\,d\phi=c\int_{a}^{b}f\,d\phi$。

\begin{theorem}{}
如果 $ \int_{a}^{b}fd\phi $ 存在并且 $ a<c<b $ ,则 $ \int_{a}^{c}fd\phi $ 和 $ \int_{c}^{b}fd\phi $ 都存在并且,

\begin{align}
\int_{a}^{b}fd\phi=\int_{a}^{c}fd\phi\,+\int_{c}^{b}fd\phi~.
\end{align}
\end{theorem}



\begin{theorem}{}
如果 $f$ 在 $[a,b]$ 上连续,$\phi$ 在 $[a,b]$ 上具有有界变差,那么 $ \int_{a}^{b}f,d\phi$ 存在。而且,
\begin{align}
  \left|\int_{a}^{b}f\,d\phi\right|\leq\,(\sup_{[a,b]}|f|)V[\phi\,;\,a,b]~.
\end{align}

\end{theorem}

\begin{theorem}{}
如果 $f\subset C[a,b]$,$g$ 在 $(a,b)$ 上可导,并且其导数在 $[a,b]$ 上可积(特别是有界),我们有
\begin{align}
\int_{a}^{b}fdg=\int_{a}^{b}fg^{\prime}~.
\end{align}
\end{theorem}

\begin{example}{假设 $f$ 在 $[a,b]$ 上连续,$\phi$ 在 $[a,b]$ 上具有有界变差。证明函数 $\psi(x)=\int_{a}^{x}f,d\phi$ 在 $[a,b]$ 上具有有界变差。如果 $g$ 在 $[a,b]$ 上连续。}

\textbf{证明:}由于 $ |\psi(x_i) - \psi(x_{i-1})| = \left|\int_{x_{i-1}}^{x_i}f , d\phi\right| $,然后由有界变差的性质,我们有 $ V(\phi, a, b) $ 是有限的。令 $ M = \sup_{[a,b]}|f| $ 为 $ f $ 在 $ [a, b] $ 上的上确界。因此, \begin{align} \sum_{i=1}^n |\psi(x_i) - \psi(x_{i-1})| \leq M \sum_{i=1}^n V(\phi,x_{i-1},x_i) = MV(\phi,a,b)~. \end{align} 这表明 $ \psi(x) $ 是有界变差函数。

证明 $\int_{a}^{b}gd\psi=\int_{a}^{b}gfd\phi$:

\begin{align}
\int_{a}^{b}g\,d\psi=\int_{a}^{b}g\,\psi'=\int_{a}^{b}gf\,d\phi~.
\end{align}
\end{example}

