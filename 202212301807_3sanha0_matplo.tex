% 多项式矩阵
% 矩阵论
\pentry{多项式\upref{OnePol},矩阵\upref{Mat}}
多项式矩阵也被称为$\lambda$-矩阵

\begin{definition}{多项式矩阵}
下列形式的矩阵
$$A(\lambda)=\begin{pmatrix}
a_{11}(\lambda) & a_{12}(\lambda) & \cdots & a_{1n}(\lambda) \\
a_{21}(\lambda) & a_{22}(\lambda) & \cdots & a_{2n}(\lambda) \\
\cdots          & \cdots          & \cdots & \cdots          \\
a_{m1}(\lambda) & a_{m2}(\lambda) & \cdots & a_{mn}(\lambda) \\
\end{pmatrix}$$
其中$a_{ij}(\lambda)$是数域$\mathbb{K}$上的多项式,称为多项式矩阵或$\lambda$-矩阵
\end{definition}


$\lambda$-矩阵的加减法、数乘以及乘法与数域上的矩阵的运算一样,只需在运算过程中将数的运算用一元多项式的运算替代即可。特征矩阵是一种特殊的$\lambda$-矩阵。

\begin{definition}{多项式矩阵的初等行变换}
对$\lambda$-矩阵$A(\lambda)$施行下列3种变换称为$\lambda$-矩阵的初等行变换:
\begin{enumerate}
\item 将$A(\lambda)$的两行对换
\item 将$A(\lambda)$的$i$行乘以常数$c$,$c$是数域$\mathbb{K}$上的\textbf{非零数}
\item 将$A(\lambda)$的$i$行乘以$\mathbb{K}$上的多项式$f(\lambda)$后加到第$j$行上
\end{enumerate}
同理,也可以这样定义3种$\lambda$-矩阵的初等列变换。
\end{definition}

注:第二种初等变换不能乘以$f(\lambda)$

多项式矩阵也能被表达为以(数值)矩阵为系数的多项式,所以也被称为矩阵系数多项式。令$d_{ij}=\deg(a_{ij}(\lambda))$,$d=max_{\forall{i,j}}(d_{ij})$,那么$A(\lambda)$也可以被表达为:
\begin{equation}
A=\sum_{k=0}^d\lambda^k[a_{ijk}]_{\forall{i,j}}
\end{equation}
