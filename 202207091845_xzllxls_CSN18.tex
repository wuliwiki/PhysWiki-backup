% 2018 年计算机学科专业基础综合全国联考卷
% 2018 年计算机学科专业基础综合全国联考卷

\subsection{一、单项选择题}
第1~40小题,每小题2分,共80分.下列每题给出的四个选项中,只有一个选项最符合试题要求.

1.若栈$S_1$中保存整数,栈$S_2$中保存运算符,函数$F()$依次执行下述各步操作: \\
(1) 从$S_1$中依次弹出两个操作数$a$和$b$; \\
(2) 从$S_2$中弹出一个运算符$op$; \\
(3) 执行相应的运算$b$ $op$ $a$; \\
(4) 将运算结果压人$S_1$中. \\
假定$S_1$中的操作数依次是$5$, $8$, $3$, $2$($2$在栈顶),$S_2$中的运算符依次是$*$, $-$, $+$ ($+$在栈顶).调用$3$次$F()$后, $S_1$栈顶保存的值是. \\
A. -15 $\quad$  B. 15 $\quad$ C. -20 $\quad$ D. 20

2. 现有队列$Q$与栈$S$,初始时$Q$中的元素依次是$1$, $2$, $3$, $4$, $5$, $6$($1$在队头),$S$为空.若仅允许下列$3$种操作:①出队并输出出队元素;②出队并将出队元素人栈;③出栈并输出出栈元素,则不能得到的输出序列是. \\
A . 1, 2, 5, 6, 4, 3 $\quad$ B. 2, 3, 4, 5, 6, 1 \\
C. 3, 4, 5, 6, 1, 2 $\quad$ D. 6, 5, 4, 3, 2, 1

3. 设有一个$12\times12$的对称矩阵M,将其上三角部分的元素$m_{i, j}$($1\leqslant i\leqslant j\leqslant12$)按$12$行优先存人$C$语言的一维数组$N$中,元素$m_{6,6}$在$N$中的下标是. \\
A . 50 $\quad$ B. 51 $\quad$ C. 55 $\quad$ D. 66

4. 设一棵非空完全二叉树$T$的所有叶结点均位于同一层,且每个非叶结点都有$2$个子结点.若$T$有$k$个叶结点,则$T$的结点总数是. \\
A . 2k-1 $\quad$ B. 2k $\quad$ C. k2 $\quad$ D. 2k-1

5. 已知字符集{a, b, c, d, e, f} ,若各字符出现的次数分别为6, 3, 8, 2, 10, 4 ,则对应字符集中
各字符的哈夫曼编码可能是.
A . 00, 1011, 01, 1010, 11, 100 B. 00, 100, 110, 000, 0010, 01
C. 10, 1011, 11, 0011, 00, 010 D. 0011, 10, 11, 0010, 01, 000
6. 已知二叉排序树如下图所示,元素之间应满足的大小关系是.