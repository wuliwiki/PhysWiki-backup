% 中山大学 2011 年913专业基础(数据结构)考研真题

\subsection{一、单项选择题(每题2分,共40分)}

1.算法复杂度通常是表达算法在最坏情况下所需要的计算量,$O(1)$的含义是( ) \\
(A).算法执行1步就完成 \\
(B).算法执行1秒钟就完成 \\
(C).解决执行常数步就完成 \\
(D).算法执行可变步数就完成

2.在数据结构中,按逻辑结构可把数据结构分为( ) \\
(A).静态结构和动态结构 \\
(B).线性结构和非线性结构 \\
(C).顺序结构和链式结构 \\
(D).内部结构和外部结构

3.在数据结构中,可用存储顺序代表逻辑顺序的数据结构为( ) \\
(A). Hash表 \\
(B).二叉搜索树 \\
(C).链式结构 \\
(D).顺序结构

4. 对链式存储的正确描述是( ) \\
(A).结点之间是连续存储的 \\
(B).各结点的地址由小到大 \\
(C).各结点类型可以不一致 \\
(D).结点内单元是连续存储的

5. 在下列关于“串”的陈述中,正确的说明是( ) \\
(A).串是一种特殊的线性表 \\
(B).串中元素只能是字母 \\
(C).串的长度必须大于零 \\
(D).空串就是空白串

6.关于堆栈的正确描述是( ) \\
(A). FILO \\
(B). FIFO \\
(C).只能用数组来实现 \\
(D).可以修改栈中元素的数据

7. 假设循环队列的长度为QSize. 当队列非空时,从其队列头取出数据后,其队头下标Front的变化为() \\
(A). Front = Front+ 1 \\
(B). Front = (Front + 1) \% 100 \\
(C). Front = (Front+ 1) \% QSize \\
(D). Front = Front \% Qsize + 1

8. 假设Head是带头结点单向循环链的头结点指针,判断其为空的条件是( ) \\
(A). Head.next = NULL \\
(B). Head~>next == Head \\
(C). Head->next = NULL \\
(D). Head = NULL

9.设A[)][m]为一个对称矩阵, 数组下标从[0[0)开始. 为了节省存储,将其下三角部分按行存放在一维数组B0.m-1], m=n(n+1)2, 对下三角部分中任一元素4.fi≥D, 它在一-维数组 B的下标k值是( ) \\
(A). i(i-1)/2+j \\
(B). (-1)2+j-1 \\
(C). (i+1)/2+j-1 \\
(D). i(+1)2+j

10. 假设二又树的根结点为第0层,那么,其第i层(20)的结点数最多为( ) \\
(A).$2i$ \\
(B).$2^i$ \\
(C).$2^{i+1}-1$ \\
(D).$2^{i+1}$

11.若一棵二叉树的后序和中序序列分别是dbefca和dbaef,则其先序序列是( )
(A). adbefc
(B). abdcfe
(C). adbcef
(D). abdcef
12.用一维数组来存储满二叉树,若数组下标从0开始,则元素下标为k的右子结点下标是
( )(不考虑数组 下标的越界问题)
(A). 2k+1
(B). 2k+2
(C).Lk2」
(D).[k21
13. 假设LTree和RTree是二叉搜索树Tree的左右子树,H(T)表示树T的高度.若树Tree是
AVL树,则( )
(A). H(LTree) - H(RTree)= 0
(B). HCLTree)- H(RTree) < 1
(C). H(LTree) - H(RTree) <= 1
(D). H(LTree) - H(RTree) <= 1
14.对n个结点和e条边的无向图(无环),其邻接矩阵中零元素的个数为( )
(A).e
(B). 2e 
(C).n-e
(D).n7-2e .
15.用邻接矩阵存储有n个顶点和e条边的有向图,则删除与某个顶点相邻的所有边的时间复杂
度是(\)
(A). 0(n)
(B). O(e)
(C). O(n+e)
(D). 0(ne)
16.下列排序算法中,时间复杂度最差的是( )
(A).选择排序
(B).桶(基数)排序
(C).快速排序
(D).堆排序
17.基于比较的排序算法对n个数进行排序的比较次数下界为( )
(A). 0(logn)
(B). 0(m) 
(C). 0(nlogn)
().0(n3)
18.在下列存储条件下,( )是最适合 使用折半查找算法来进行查找操作.
(A).顺序存储
(B).链式存储
(C).散列存储
(D).数据有序且顺序存储
19.在下列算法中,求图最小生成树的算法是( )
(A). DFS算法
(B). KMP算法
(C). Prim算法
(D). Djkstra算法
20.若结点的存储地址与其关键字之间存在某种映射关系,则称这种存储结构为( )
(A).顺序存储结构(B). 链式存储结构
(C).散列存储结构(D). 索引存储结构
