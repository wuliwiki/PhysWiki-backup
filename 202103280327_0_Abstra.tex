% 抽象
% keys 数学|一般|广义|抽象|概念

\begin{issues}
\issueTODO
\end{issues}

我们这里要讨论的是\textbf{抽象(abstract)} 一词, 意为 “抽出一个对象的部分特性做研究而忽略其它特性” 或者 “抽出若干对象的共同特征做研究而抛弃它们的特有特征”,其中“象”取“类比”、“相似”之意.当你在脑海中想象“树”的概念时,只关注树共有的一些关键特质,比如有树叶 有枝干有根系等,至于具体多少树叶、根系如何分布则被忽略了,因此抽象的结果往往是不能画出来的,因为画出来的树都有了具体的特征了,不再是抽象的树.抽象的对立面,是具象.abstract一词是由ab-(向外的)和-tract(拉、拔)构成的,词义和抽象完全一样.当我们说一个事物更抽象时,我们其实也在说它更为 “一般” 或者更 “广义”.

\addTODO{例子:从位移, 速度, 加速度等中抽象出矢量}
