% 比特币
% license CCBYSA3
% type Wiki

(本文根据 CC-BY-SA 协议转载自原搜狗科学百科对英文维基百科的翻译)

\textbf{比特币 (₿)}是一种加密货币,即一种电子现金。它是一种去中心化的数字货币,没有中央银行或单一管理员,可以在无需中介的情况下,在点对点比特币网络上从一个用户发送到另一个用户。

交易由网络节点通过加密进行验证,并记录在名为区块链的公共分散式账本中。比特币是由一个或一群未知晓的人用“中本聪”这个名字 发明的,并于2009年作为开源软件发布。 比特币是作为对采矿过程的奖励而创造的。它们可以被兑换成其他货币、产品和服务。[1] 剑桥大学开展的研究估计,在2017年里有290万至580万不同用户使用加密货币钱包,其中大多数使用比特币。

比特币因其在非法交易中的使用、高耗电量、价格波动性、交易中存在的偷窃以及其成为经济泡沫的可能性而受到批评。 比特币也被用作一种投资,尽管几个监管机构已经就比特币向投资者发出了警告。[2]

\subsection{历史}
\subsubsection{1.1 创造}
域名“bitcoin.org”于2008年8月18日被注册。[3] 2008年10月31日,中本聪撰写的一篇题为比特币:点对点电子现金系统 的论文的链接被邮寄到一个加密邮件列表上。[4] 中本聪将比特币软件作为开源代码实现,并于2009年1月发布。[5][6][7] 中本聪的身份仍然未知。[7]

2009年1月3日,当中本聪开采出这条链的第一个区块,即创世区块时,比特币网络就建立了。[7][8] 嵌在这一区块的货币基础中的是以下文字:“2009年1月3日《泰晤士报》财政大臣即将对银行进行第二次救助。”[7] 这张消息被解释为一个带有时间戳的对部分准备金银行业务造成的不稳定性的评论。[9]

第一笔比特币交易的接收者是密码朋克哈尔·芬尼,他在2004年创建了第一个可重复使用的工作量证明系统(RPOW)。[10] 芬尼在比特币软件发布之日下载了该软件,并于2009年1月12日收到了中本聪的10枚比特币。[11][12] 其他早期密码朋克的支持者是比特币前辈的创造者:bb-money的创造者戴伟和比特黄金的创造者尼克·萨伯。[7]2010年,第一次使用比特币的商业交易发生在程序员拉兹洛·汉尼茨花10,000比特币买了Papa John's的比萨饼的时候。[13]

据估计,中本聪在2010年消失之前已经开采了100万枚比特币,[14] 当时他将网络警报密钥和代码库的控制权交给了加文·安德列森。安德烈森后来成为比特币基金会的首席开发者。[15][16] 安德烈森随后寻求去中心化控制。这为比特币未来的发展道路留下了争议。[16]
\subsubsection{1.2 2011-2012年}
在早期的“概念验证”交易后,比特币的第一批主要用户是黑市,如丝绸之路。从2011年2月开始,丝绸之路在其存在的30个月中,只接受比特币作为支付,交易990万比特币,价值约2.14亿美元。

2011年,比特币的起价为每枚0.30美元,今年上涨至5.27美元。6月8日,价格升至31.50美元。不到一个月,价格降至11美元。第二个月跌至7.80美元,另一个月跌至4.77美元。[17]

Litecoin是比特币的早期衍生产品,于2011年10月问世。[18] 从那以后,许多另类硬币被创造出来。[19]

2012年,比特币价格从5.27美元开始上涨至13.30美元。[17] 截至1月9日,价格已升至7.38美元,但随后在接下来的16天内暴跌49\%,至3.80美元。8月17日,价格升至16.41美元,但在接下来的三天里下跌了57\%,至7.10美元。[20]

比特币基金会成立于2012年9月,旨在促进比特币的发展和认识。[21]
\subsubsection{1.3 2013-2016年}
2013年,价格从13.30美元开始,到2014年1月1日升至770美元。[17]

2013年3月,区块链暂时分裂成两条不同规则的独立链条。两个区块链同时运行了六个小时,每个小时都有自己版本的交易历史。当大多数网络降级到比特币软件的0.7版时,正常操作得以恢复。[22] 在接下来的几个小时里,比特币在恢复到之前的大约48美元的水平之前,Mt. Gox曾短暂停止比特币存款,导致比特币价格下跌了23\%,至37美元。[22][23] [24] 美国金融犯罪执法网(FinCEN)为比特币等“去中心化虚拟货币”制定了监管准则,将出售其生成的比特币的美国比特币矿商归类为货币服务企业(MSBs),这些企业需接受注册或其他法律义务。[25][26][27]4月份,由于容量不足,比特币交易所BitInstant和Gox经历了处理延迟,[28] 导致比特币价格从266美元跌至76美元,6小时内又回到160美元。[29] 比特币价格在4月10日升至259美元,但随后三天暴跌83\%,至45美元。[20] 2013年5月15日,美国当局在发现Mt. Gox没有在美国金融犯罪中心注册为汇款人后,查封了其相关账户。[30][31] 2013年6月23日,美国缉毒署根据《美国法典》第21篇第881条,在美国司法部扣押通知中将11.02枚比特币列为扣押资产。[32]这标志着政府机构首次没收比特币。[33] 2013年10月,在罗斯·威廉·乌尔布里切特被捕期间,美国联邦调查局从暗网丝绸之路缴获了约26,000枚比特币。[34][35][36]比特币的价格在11月19日升至755美元,当天暴跌50\%,至378美元。2013年11月30日,该价格在开始长期崩盘前达到1,163美元,2015年1月下跌87\%,至152美元。[20] 2013年12月5日,中国人民银行禁止中国金融机构使用比特币。[37] 宣布后,比特币的价值下跌,[38] 百度也不再接受比特币的某些服务。[39] 至少自2009年以来,在中国用任何虚拟货币购买现实世界的商品都是非法的。[40]

2014年,价格从770美元开始跌至314美元。[17]

2014年7月30日,维基媒体基金会开始接受比特币捐赠。[41]

2015年。价格从314美元开始上升到434美元。2016年,价格在2017年1月1日升至998美元。[17]
\subsubsection{1.4 2017-2018年}
价格从2017年的998美元开始,在2017年12月17日达到19,783.06美元的历史高点后,于2018年1月1日上涨到13,412.44美元。[17] [42]

中国于2017年9月开始禁止比特币交易,并于2018年2月1日开始全面禁止比特币交易。比特币价格随后在2018年2月5日从9,052美元跌至6,914美元。[20] 比特币在中国人民币交易中的比例从2017年9月的90\%以上降至2018年6月的不到1\%。[43]

在2018年上半年的剩余时间里,比特币的价格在11,480美元至5,848美元之间波动。2018年7月1日,比特币的价格为6,343美元。[44][45] 2019年1月1日的价格为3747美元,比2018年下降72\%,比历史最高水平下降81\%。[44][46]

比特币价格受到了几次涉及到加密货币的黑客攻击或盗窃的负面影响,包括分别对于2018年1月的coincheck、6月的coinrail和bithumb的盗窃以及7月的班科尔的盗窃。据报道,2018年前6个月,价值7.61亿美元的密码从交易被盗。[47] 即使其他加密货币在coinrail和bancor被盗,由于投资者担心密码货币交易所的安全性,比特币的价格也会受到影响。[48][49][50]

\subsection{设计}
\subsubsection{2.1 单位}
比特币系统的记账单位是比特币。用来代表比特币的股票代码是BTC 和XBT。[51] 它的唯一编码是₿.[52] 作为少量比特币的替代单位有毫比特币(mBTC)和satoshi(sat)。satoshi是为向比特币的创造者致敬而命名的,是比特币中最小的一种,代表0.00000001个比特币,相当于比特币的1亿分之一。[53] 一枚毫比特币等于0.001枚比特币;一枚比特币的千分之一或十万satoshis。[54]
\subsubsection{2.2 块状链}
\begin{figure}[ht]
\centering
\includegraphics[width=14.25cm]{./figures/e684c31b9a7ea6e2.png}
\caption{账本中区块的数据结构} \label{fig_BTC_1}
\end{figure}
比特币区块链是记录比特币交易的公共账本。[55] 它是作为一个块链来实现的,每个块包含一个直到链的起源块 的前一个块的散列。运行比特币软件的通信节点网络维护着区块链。[56]付款人X向收款人Z发送Y比特币的交易广播到使用现成的软件应用程序的网络。

网络节点可以验证交易,将其添加到账本的副本中,然后将这些账本的增量广播到其他节点。为了实现所有权链的独立验证,每个网络节点存储自己的区块链副本。[56] 大约每10分钟,一组新的被接受的事务,称为块,被创建、添加到区块链,并迅速发布到所有节点,而不需要中央监督。这使得比特币软件能够确定一个特定比特币何时被消费,而这是防止重复消费所必需的。传统的账本记录实际票据或本票的转移,但区块链是可以说比特币以未用交易产出的形式存在的唯一地方。
\subsubsection{2.3 处理}
事务是使用类似Forth的脚本语言定义的。 交易由一个或多个输入和一个或多个输出组成。当用户发送比特币时,用户在输出中指定每个地址以及发送到该地址的比特币数量。为了防止双重支出,每一项投入都必须参考区块链以前未用的输出。[57] 多种输入相当于在现金交易中使用多种比特币。由于交易可以有多个输出,用户可以在一次交易中向多个接收者发送比特币。与现金交易一样,输入(用于支付的比特币)的总和可能超过预期的支付总和。在这种情况下,使用额外的输出,将更改返回给付款人。[57] 任何未计入交易输出的输入satoshis都将成为交易费用。[57]\subsubsection{2.4 交易费用}
尽管交易费用是可选的,但矿工可以选择处理哪些交易,并优先处理那些支付较高费用的交易。[57] 矿工可以根据相对于其仓储规模支付的费用,而不是作为费用支付的绝对金额来选择交易。这些费用通常以每字节satoshi(sat/b)来衡量。事务的大小取决于用于创建事务的输入数量和输出数量。
\subsubsection{2.5 所有权}
\begin{figure}[ht]
\centering
\includegraphics[width=14.25cm]{./figures/b15d866e7b7bd6c2.png}
\caption{比特币白皮书中所述的所有权的简化链。[10] 实际中,一个交易可以有多个输入和输出。[9]} \label{fig_BTC_2}
\end{figure}
在区块链,比特币被注册到比特币地址。创建一个比特币地址只需要选择一个随机的有效私钥并计算相应的比特币地址。这个计算可以在一瞬间完成。但是相反,计算给定比特币地址的私钥在数学上是不可行的。用户可以告诉他人或公开比特币地址,而不会泄露其相应的私钥。此外,有效私钥的数量如此之大,以至于不太可能有人计算出已经在使用并有资金的密钥对。大量有效的私钥使得暴力破解私钥变得不可行。为了能够花掉他们的比特币,所有者必须知道相应的私钥并对交易进行数字签名。网络使用公钥验证签名;私钥永远不会泄露。

如果私钥丢失,比特币网络将无法识别任何其他所有权证据;[56] 然后比特币就不能用了,即丢失了。例如,2013年,一名用户声称丢失了7500枚比特币,当时价值750万美元,当时他意外丢弃了一个包含私人密钥的硬盘。[58] 据信,大约20\%的比特币丢失了。按2018年7月的价格计算,它们的市场价值约为200亿美元。[59]

为了确保比特币的安全性,私钥必须保密。如果私钥被泄露给第三方,例如通过数据泄露,第三方可以使用它来窃取任何相关联的比特币。截至2017年12月,约有98万枚比特币从密码货币交易所被盗。
\subsubsection{2.6 采矿}
\begin{figure}[ht]
\centering
\includegraphics[width=10cm]{./figures/66a5919eaf60a027.png}
\caption{早期的比特币矿工使用图形处理器(GPU)进行采矿,因为他们比中央处理器(CPU)更适合工作量证明的算法。} \label{fig_BTC_3}
\end{figure}
采矿是通过使用计算机处理能力完成的记录保存服务。 矿工通过反复将新广播的事务分组到一个块中来保持区块链的一致性、完整性和不可更改性,然后将该块广播到网络并由接收节点验证。[55] 每个块包含前一个块的SHA-256加密散列,[55] 因此将它链接到前一个块并给区块链命名。[55]

要被网络的其他节点接受,新块必须包含工作证明(PoW)。[55] 使用的系统是基于亚当·贝克1997年的反垃圾邮件方案哈斯查什。[58][60] 工作证明要求矿工找到一个称为随机数的数字,这样当块内容与随机数一起被散列时,结果在数字上小于网络的难度目标。 这种证明对于网络中的任何节点来说都很容易验证,但是生成这种证明非常耗时,因为对于安全的加密散列,矿工必须尝试许多不同的随机数(通常测试值的序列是升序自然数:0,1,2,3,...)在达到难度目标之前。
\begin{figure}[ht]
\centering
\includegraphics[width=10cm]{./figures/7c869435794a37a2.png}
\caption{后来的业余爱好者用特制的现场可编程门阵列(FPGA)和专用的集成电路(ASIC)芯片来挖矿。由于开采难度越来越大,图中的芯片已经过时。} \label{fig_BTC_4}
\end{figure}
每2016个块(大约14天,每个块大约10分钟),难度目标根据网络最近的性能进行调整,目的是将新块之间的平均时间保持在10分钟。这样,系统会自动适应网络上的总采矿电量。 在2014年3月1日至2015年3月1日期间,在创建新区块之前,挖掘随机数的矿工的平均人数从1640万亿亿增加到20050万亿亿。[61]

工作证明系统和区块链一起使得对区块链的修改变得极其困难,因为攻击者必须修改所有后续区块才使得对一个区块的修改被接受。[62] 随着新区块的不断开采,随着时间的推移,修改区块的难度会增加,因为随后区块的数量(也称为给定区块的确认)也会增加。[55]
\begin{figure}[ht]
\centering
\includegraphics[width=10cm]{./figures/e0821cf2bd77fd53.png}
\caption{如今,比特币矿业公司将用专门的设备来存放和运行大量的高性能采矿硬件。} \label{fig_BTC_5}
\end{figure}

\textbf{集合采矿}

计算能力通常被捆绑在一起或“集中起来”,以减少矿工收入的差异。单个采矿设备通常不得不等待很长一段时间来确认一笔交易并接收付款。在池中,每次参与的服务器解决一个块时,所有参与的矿工都会得到报酬。这笔钱取决于个体矿工帮助找到该区块的工作量。[63]