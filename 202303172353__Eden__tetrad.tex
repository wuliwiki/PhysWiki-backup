% 观测者的标架矢量
% 观测者|观测|狭义相对论|标架矢量|费米-沃克尔移动|Fermi-Walker移动

\pentry{洛伦兹群\upref{qed1}}

约定使用东海岸度规 $(-1,1,1,1)$ 和自然单位制 $c=1$。

在相对论中,我们需要很好地定义观测者及其观测。同一物理现象对于不同的观测者可能对于不同的观测结果,这是因为观测者在时空中有自己的世界线,我们需要通过世界线来确定相对于观测者的实验室,而所谓的观测是指相对于观测者的实验室(惯性系)中的钟和空间轴而进行的测量活动。

\subsection{标架矢量}

观测者的钟定义了时间方向:$\hat{e}_0=\hat{u}_\text{obs}$ 为观测者的4-速度。

那么可以定义观测者所在实验室的三个空间轴:$\hat{e}_a,a=1,2,3$,满足以下的正交归一关系:
\begin{equation}
\begin{aligned}
\hat{e}_i\cdot \hat{e}_0=0,\quad \hat{e}_i\cdot \hat{e}_j=\delta_{ij}
\end{aligned}
\end{equation}

$\hat{e}_a,a=1,2,3$ 被称为\textbf{标架(tetrad)矢量}。
\subsubsection{观测到的粒子的4-动量}
设粒子的4-动量为 $\hat p$,观测者的标架矢量由 $\hat{e}_0,\cdots,\hat{e}_3$ 描述。那么观测到的粒子的能量为 $E_\text{obs}=-\hat{e}_0\cdot\hat{p}$,观测到的粒子的动量为 $\boldsymbol P_\text{obs}^i=\hat{e}_i\cdot\hat{p}$。

我们可以将上述结果写成更加协变的形式。定义张量 $\hat{e}^{\mu\nu}$ 其中$\hat{e}^{0\nu}=-\hat{e}_0{}^\nu,\hat{e}^{a\nu}=\hat{e}_a{}^\nu$ 对应于前面定义的 4 个标架矢量,因此正交归一关系可以表示为 $e^{\mu\nu} e^\rho{}_\nu=g^{\mu\rho}$。那么观测者观测到的粒子的四动量实际上是 $P^\mu_\text{obs}=e^\mu{}_\nu P^\nu$。
\subsubsection{标架矢量与洛伦兹变换}
从静止惯性系到观测者参考系的变换可以由洛伦兹变换 $\Lambda^\mu{}_\nu$ 描述,四矢量的变换规则为 ${x'}^{\mu}=\Lambda^\mu{}_\nu x^\nu$。根据上面的分析,观测者参考系的标架矢量 $\hat{e}^\mu{}_\nu$ 正对应于洛伦兹变换 $\Lambda^\mu{}_\nu$。
\subsection{费米-沃克尔移动}
在前面的讨论中,我们可以发现空间基矢的选择不唯一,有很大的自由度。但重要的是,我们希望当一个时刻观测者的标架矢量确定下来以后,以后每一时刻观测者的标架矢量都是唯一的,这个过渡仅由观测者的运动决定。

设 $\hat{u},\hat{a}$ 是观测者的4-速度和4-加速度,那么标架矢量满足以下的要求:
\begin{enumerate}
\item 如果 $\hat{e}_a$ 在 $\hat u$ 和 $\hat a$ 定义的平面上,基矢的变换应当是某种线性变换;
\item 如果 $\hat{e}_a$ 在与 $\hat u$ 和 $\hat a$ 定义的平面正交的平面上,基矢不会有任何转动。
\end{enumerate}

满足这两个要求的唯一可能性是

\begin{equation}
\begin{aligned}
\frac{{\rm d} \hat{e}_a}{{\rm d} \tau}=-(\hat u\cdot\hat{e}_a)\hat a+ (\hat a\cdot \hat{e}_a)\hat u
\end{aligned}
\end{equation}

很显然要求2被满足了,当 $\hat{e}_a$ 与 $\hat u,\hat a$ 正交时上式右侧为 0。由上式右侧的第一项可以验证 $\frac{{\rm d}\hat{e}_0}{{\rm d}\tau}=\frac{{\rm d}\hat u}{{\rm d}\tau}=\hat a$ 成立。那么上式右侧的第二项是如何确定的呢?我们令 $\hat{e}_1$ 为与 $\hat{e}_0$ 正交的任意基矢,那么它的变化率由上式第二项可以得到:$\frac{{\rm d}\hat{e}_1}{{\rm d}\tau}=(\hat a\cdot \hat{e}_1)\hat u$。因此 $\frac{{\rm d}\hat{e}_1}{{\rm d}\tau}\cdot \hat{e}_0=-\hat a\cdot \hat{e}_1$,移项可以得到 $\frac{{\rm d}\hat{e}_1}{{\rm d}\tau}\cdot \hat{e}_0+ \frac{{\rm d}\hat{e}_0}{{\rm d}\tau} \cdot \hat{e}_1=0$,这实际上保证了 $\frac{{\rm d} (\hat{e}_1\cdot \hat{e}_0)}{{\rm d}\tau}=0$,也就是说 $\hat{e}_1$ 与 $\hat{e}_0$ 将永远保持正交,这符合我们的期望。
