% 复流形
% keys 复流形|流形|微分几何
% license Xiao
% type Tutor

\pentry{流形\nref{nod_Manif}}{nod_afd8}

\subsection{复流形}

将光滑流形\autoref{def_Manif_3} 定义中的 $\mathbb{R}^n$ 替换为 $\mathbb{C}^n$,图册中的“光滑映射”替换为“全纯映射”(复解析映射), 我们就得到复流形的定义。

\begin{definition}{单复变全纯函数}
若一个单复变函数 $f(z): \mathbb C \to C$,考虑为 $f(x+y \mathrm i)=u(x, y) + v(x, y) \mathrm i$,$x, y\in \mathbb R$,$u(x, y), v(x, y) \in \mathbb R$。则称 $f$ 是全纯的当且仅当满足柯西-黎曼条件:
\begin{equation}
    \pdv{u}{x} = \pdv{v}{y}~,\qquad \pdv{u}{y} = -\pdv{v}{x} ~.
\end{equation}
等价于说 $f$ 仅依赖于 $z$ 而不依赖于 $\bar{z}$,即
\begin{equation}
\pdv{f}{\bar{z}} = 0 ~.
\end{equation}

\end{definition}

\begin{definition}{多复变全纯函数(复值)}
对于一个复值多复变函数 $f(z_1, z_2, \dots, z_n): \mathbb C^n \to \mathbb C$ 是全纯函数当且仅当 $f$ 对每个变量 $z_i$ 而言都是全纯的。
\end{definition}

\begin{definition}{多复变全纯函数(向量值)}
对于一个向量值多复变函数 $f(z_1, z_2, \dots, z_n) = (f_1, f_2, \dots, f_m): \mathbb C^n \to \mathbb C^m$,称其是全纯的当且仅当对于每个复值多复变函数 $f_i$ 而言,其对每个变量 $z_j$ 都是全纯的。
\end{definition}



\begin{definition}{复图和复图册}
$N$ 是一个 $n$ 维拓扑流形,如果存在开集 $U \in \mathcal{T}_N$ 和拓扑同胚映射 $\varphi: U \rightarrow \tilde{U} \subseteq \mathbb{C}^n$,其中 $\tilde{U}$ 是 $\mathbb{C}^n$ 的一个开子集,那么称 $(U,\varphi)$ 是 $N$ 上的一张\textbf{复图}。如果图的一个集合 $\mathcal{A}=\{(U_\alpha, \varphi_\alpha)\}$ 覆盖了 $N$,即 $\bigcup\{U_\alpha\}=N$,那么称这个集合 $\mathcal{A}$ 是一个\textbf{复图册}。
\end{definition}

特别需要注意由于 $\mathbb{C}$ 和它的真开子集(比如开球)不是全纯同胚的(这和实数的情况不相同),不是所有的坐标图 $\varphi: U \rightarrow \tilde{U} \subseteq \mathbb{C}^n$ 都能改写成 $\varphi': U' \rightarrow \mathbb{C}^n$ 的形式的。实际上,$\mathbb{C}$无法嵌入到复环面(complex tori) $\mathbb{C} / \Lambda$ 中。

\begin{definition}{全纯相容}
考虑一个拓扑流形 $N$ 的两个复图 $(U, \varphi)$ 和 $(V, \phi)$。如果 $U \cap V \neq \varnothing$,且 $\varphi \circ \phi^{-1}: \phi(V) \rightarrow \varphi(U)$ 和 $\phi \circ \varphi^{-1}: \varphi(U) \rightarrow \phi(V)$ 都是全纯(复解析)映射,那么我们称这两个图是\textbf{全纯相容的(compatible)}。
\end{definition}

\begin{definition}{复(全纯)流形}\label{def_CMani_1}
一个拓扑流形 $N$ 和加上其上一组全纯相容的复图册 $\mathcal{A}$,被称为一个\textbf{复流形(complex (holomorphic) manifold)} $(N, \mathcal{A})$。
\end{definition}

\begin{theorem}{}
复流形都是可定向实流形。
\end{theorem}

\addTODO{证明}

\begin{definition}{全纯映射(复流形)}
给定复流形 $M, N$, 一个映射 $f: M \to N$ 在点 $p \in M$ 处\textbf{全纯}(holomorphic 或称\textbf{解析} analytic),如果存在 点$p$附近的坐标图 $\phi: U \ni p \to \tilde{U} \subseteq \mathbb{C}^m$,和点 $f(p)$ 附近的坐标图 $\psi: V \ni f(p) \to \tilde{V} \subseteq \mathbb{C}^n$,使得函数
$$
\psi \circ f \circ \phi^{-1}: \tilde{U} \subseteq \mathbb{C}^m \to \tilde{V} \subseteq \mathbb{C}^n~.
$$
在 $\phi(x)$ 点处全纯,如图;

$f$ 被称作\textbf{全纯函数},如果它在任意点处都全纯。
\end{definition}

\addTODO{交换图}

\addTODO{切向量和余切向量}

\subsection{解析集}

\begin{definition}{解析集}
给定复流形 $M, N$ 和一个全纯映射 $f: M \to N$,对任意的 $y \in N$,
$$
f^{-1}(y)~
$$
是一个 $M$ 上的解析集。
\end{definition}

解析集是复流形 $M$ 上的闭集合,但不一定是闭子流形。解析集的名字来自于,全纯映射又称“(复)解析映射”。

\begin{theorem}{隐函数定理(复几何)}
给定复流形 $M, N$ 和一个全纯映射 $f: M \to N$,如果 $y \in N$ 是一个正则值(regular value),即对任意的 $x \in f^{-1}(y)$,
$$
D f|_x: T_x M \to T_x N~
$$
是一个满射,那么解析集 $f^{-1}(y)$ 是 $M$ 的一个闭子流形。
\end{theorem}
这里的 $T_x M, T_x N$ 是复切空间(同构于实切空间,不同构域复化切空间)。


\subsection{近复流形}

\begin{definition}{(近)复结构(向量空间)}
一个实向量空间 $V$ 上的一个\textbf{(近)复结构}是一个(实)线性映射 $j: V \to V$ 满足 $j \circ j = - \text{id}_V$。
\end{definition}

\begin{theorem}{}
一个实向量空间上存在近复结构,当且仅当它的维度是偶数。
\end{theorem}

\begin{definition}{近复结构(实流形)/近复流形}
一个实流形 $M$ 上的一个\textbf{近复结构}是它的(实)切空间上的“近复结构场”,即一个 $(1, 1)$ 型张量场 $J \in \Gamma(M, T_1^1 M)$ 满足对任意一点 $p$, 
$$
J|_p: T_p(M) \to T_p(M)~
$$
是一个向量空间上的近复结构。

\textbf{近复流形}是一个带近复结构的实流形。
\end{definition}

\begin{theorem}{}
复流形都是近复流形。

可以考虑对于一个局部平凡化卡 $(U, \varphi)$, $\varphi = (z_1, z_2, \dots, z_n)$,并考虑 $z_i = x_i + \mathrm i y_i$,取 
\begin{equation}
J: \left\{\begin{aligned}
\pdv{x_i} &\mapsto \pdv{y_i} \\
\pdv{y_i} &\mapsto -\pdv{x_i} 
\end{aligned}\right. ~~
\end{equation}
即可。
\end{theorem}

\subsection{可积条件}
近复流形为复流形当且仅当其是可积的,当且仅当 Nijenhuis 张量恒为 $0$,类似于无挠的要求。
\begin{definition}{Nijenhuis 张量}
对于一个有近复结构 $J$ 的光滑流形 $M$,定义 Nijenhuis 张量为
\begin{equation}
N_J(X, Y) = [X, Y] + J[X, JY] + J[JX, Y] - [JX, JY] ~.
\end{equation}
有的地方把 $[X, Y]$ 写作 $-J^2[X, Y]$,是一样的。
\end{definition}

这里中括号是 Lie 括号,即有 
\begin{equation}
[X, Y] = (X^\mu \partial_\mu Y^\nu - Y^\mu \partial_\mu X^\nu) \partial_\nu ~.
\end{equation}


\begin{theorem}{}
带有近复结构 $J$ 的光滑流形 $M$ 是复流形当且仅当 $N_J$ 恒为 $0$,此时称 $J$ 为可积近复结构或复结构。
\end{theorem}



\addTODO{没有定义复流形的切空间没法往下写了}




