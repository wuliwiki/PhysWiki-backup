% 李导数
% license Usr
% type Tutor
\pentry{流\nref{nod_flow},拉回\nref{nod_PullBk},前推\nref{nod_pfw}}{nod_5fa4}

\begin{issues}
\issueTODO 场不变
\end{issues}
\subsection{对向量场作用}
区别于仿射联络,\textbf{李导数(Lie derivative)}$\mathcal{L}:\mathfrak{X}(M)\times\mathfrak{X}(M)\to\mathfrak{X}(M)$是对向量场微分的另一种思路。

\begin{definition}{}
设$V,W$为$M$上的任意光滑向量场,令$\theta_t:M_t\rightarrow M$为$V$的流,定义$W$关于$V$的李导数为
\begin{equation}(\mathcal{L}_VW)_p=\left.\frac{d}{dt}\right|_{t=0}(\theta_{-t})_*W_{\theta_t(p)}=\lim_{t\to0}\frac{(\theta_{-t})_*W_{\theta_t(p)}-W_p}{t}~.\end{equation}
\end{definition}
由前推定义可知,$\theta_{t*}W_q\in T_{\theta_t(q)}M$,因而$(\theta_{-t})_*W_{\theta_t(p)}\in T_p M$,所以该定义是合理的,我们利用前推得到了同一切空间的两个向量并进行微分。

其次,我们可以证明$(\mathcal{L}_VW)_p=[V,W]_p$,因李括号将两个光滑切场映射为光滑切场,从而可知李导数亦然如此,即$p\mapsto(\mathcal{L}_VW)_p$是光滑切场。结合李括号和前推的性质,我们有:
\begin{theorem}{}
\begin{itemize}
\item $\mathcal{L}_VW=-\mathcal{L}_WV.$
\item 如果$F:M\rightarrow N$是微分同胚,则$F_*(\mathcal{L}_VW)=\mathcal{L}_{F_*V}F_*W.$
\end{itemize}
\end{theorem}
在这里,我们只证明最后一条,即$F_*[V,W]_{p}=[F_*V,F_*W]|_{F(p)}$。

由于$F$是微分同胚,那么对于任意$X\in \mathcal J(M)$,我们都可以在$N$上找到唯一的向量场与之关联。因此我们可以设$Y_i=F_*X_i\in \mathcal J(N),f\in C^{\infty}N$,那么我们有
\begin{equation}\label{eq_lie_1}
\begin{aligned}
F_*[X_1,X_2]f&=[X_1,X_2](f\circ F)\\
&=(X_1X_2-X_2X_1)f\circ F~,
\end{aligned}
\end{equation}
又因为
\begin{equation}
\begin{aligned}
(X_1X_2)(f\circ F)&=X_1[(Y_2f)\circ F]\\
&=(Y_1Y_2f)\circ F~,
\end{aligned}
\end{equation}
同理可得:$(X_2X_1)(f\circ F)=(Y_2Y_1f)\circ F$。
代入\autoref{eq_lie_1} 便得:
\begin{equation}
\begin{aligned}
F_*[X_1,X_2]f&=(Y_1Y_2f-Y_2Y_1f)\circ F\\
&=([Y_1,Y_2]f)\circ F=([Y_1,Y_2]f)|_{F(p)}~,
\end{aligned}
\end{equation}
定理得证。

\subsection{对张量场作用}
同样设$\theta$是光滑切场$X$的流,对于任意$p\in M$,$\theta_t$是$p$点附近一邻域与$\theta_t(p)$附近一邻域的微分同胚——这意味着前推和拉回都是有意义的。于是根据拉回对张量场的定义可知,若设$\tau$是$k$阶张量场,则$(\theta_t^*\tau)_p=\theta_t^*\tau_{\theta_t(p)}$,依然得到$p$点上的一个$k$阶张量场。因此我们可以定义$\tau$关于$X$的李导数为:
\begin{equation}
\left(\mathcal{L}_X \tau\right)_p=\left.\frac{\partial}{\partial t}\right|_{t=0}\left(\theta_t^* \tau\right)_p=\lim _{t \rightarrow 0} \frac{\theta_t^* \tau_{\theta_t(p)}-\tau_p}{t}~.
\end{equation}

\textbf{可以证明,李导数对光滑张量场作用,依然得到光滑张量场。}

接下来,我们罗列一些李导数的常见性质。
\begin{theorem}{}\label{the_lie_1}
在光滑流形$M$上讨论。设$\sigma,\tau$为协变张量场,$f$为光滑函数,$\omega,\eta$为微分形式,$X,Y_i$是光滑向量场。
\begin{enumerate}
\item  $\mathcal{L}_X f=X f$.
\item  $\mathcal{L}_X(f \sigma)=\left(\mathcal{L}_X f\right) \sigma+f \mathcal{L}_X \sigma$.
\item  $\mathcal{L}_X(\sigma \otimes \tau)=\left(\mathcal{L}_X \sigma\right) \otimes \tau+\sigma \otimes \mathcal{L}_X \tau$.
\item  $\mathcal{L}_X(\omega \wedge \eta)=\mathcal{L}_X \omega \wedge \eta+\omega \wedge \mathcal{L}_X \eta$.
\item  $\left.\left.\left.\mathcal{L}_X(Y\lrcorner \omega\right)=\left(\mathcal{L}_X Y\right)\right\lrcorner \eta+Y\right\lrcorner \mathcal{L}_X \omega$.
\item 设 $Y_1, \ldots, Y_k$是光滑向量场,则
\begin{equation}
\begin{aligned}
\mathcal{L}_X\left(\sigma \left(Y_1, \ldots,\right.\right. & \left.\left.Y_k\right)\right)=\left(\mathcal{L}_X \sigma\right)\left(Y_1, \ldots, Y_k\right) \\
& +\sigma\left(\mathcal{L}_X Y_1, \ldots, Y_k\right)+\cdots+\sigma\left(Y_1, \ldots, \mathcal{L}_X Y_k\right)~.
\end{aligned}
\end{equation}
\end{enumerate}
\end{theorem}
\textbf{证明:}

$f$可看作$0$阶张量场,因此其李导数亦可使用本节定义,
\begin{equation}
\left(\mathcal{L}_X f\right)_p=\left.\frac{\partial}{\partial t}\right|_{t=0}\left(\theta_t^* f\right)_p=\left.\frac{\partial}{\partial t}\right|_{t=0}(f\circ\theta_t(p))=X_pf~.
\end{equation}
最后一个等号来自于$X$是$\theta$最小生成元的定义。

$3-5$条可视作李导数的莱布尼兹律,在这里只证第三条,,其余同理可得。

\begin{equation}\begin{aligned}
(\mathcal{L}_X(\sigma\otimes\tau))_p& =\lim_{t\to0}\frac{\theta_t^*(\sigma\otimes\tau)_{\theta_t(p)}-(\sigma\otimes\tau)_p}t \\
&=\lim_{t\to0}\frac{\theta_t^*\sigma_{\theta_t(p)}\otimes\theta_t^*\tau_{\theta_t(p)}-\sigma_p\otimes\tau_p}t \\
&=\lim_{t\to0}\frac{\theta_t^*\sigma_{\theta_t(p)}\otimes\theta_t^*\tau_{\theta_t(p)}-\theta_t^*\sigma_{\theta_t(p)}\otimes\tau_p}t \\
&+\lim_{t\to0}\frac{\theta_t^*\sigma_{\theta_t(p)}\otimes\tau_p-\sigma_p\otimes\tau_p}t \\
&=\lim_{t\to0}\theta_t^*\sigma_{\theta_t(p)}\otimes\frac{\theta_t^*\tau_{\theta_t(p)}-\tau_p}t+\lim_{t\to0}\frac{\theta_t^*\sigma_{\theta_t(p)}-\sigma_p}t\otimes\tau_p \\
&=\sigma_{p}\otimes(\mathcal{L}_{X}\tau)_{p}+(\mathcal{L}_{X}\sigma)_{p}\otimes\tau_{p}~.
\end{aligned}\end{equation}
\begin{theorem}{}
设$f \in C^{\infty}(M)$,则$ \mathcal{L}_X(\dd f)=\dd(\mathcal{L} _X f)$
\end{theorem}
$\dd f$是一阶张量场,因此我们可以利用\autoref{the_lie_1} 第六条。两边作用在任意光滑向量场$Y$上,得
\begin{equation}
\mathcal (L_X\dd f)(Y)=X(\dd f(Y))-\dd f([X,Y])~.
\end{equation}
并利用余切场的定义,即$\dd f(X)\equiv Xf$,代入上式得:
\begin{equation}
\begin{aligned}
X(\dd f(Y))-\dd f([X,Y])&=XYf-(XY-YX)f\\
&=YXf=\dd(Xf)(Y)\\
&=\dd(\mathcal L_Xf)(Y)~,
\end{aligned}
\end{equation}
因此得证。

\subsection{对微分形式作用}

\subsection{李导数的基本性质}