% 南京航空航天大学 2014 量子真题
% license Usr
% type Note

\textbf{声明}:“该内容来源于网络公开资料,不保证真实性,如有侵权请联系管理员”

\subsection{简答题 (本题 45 分,每小题 15 分)}
①写出氢原子、一维简谐振子、一维无限深势阱的能级,并用示意图表示。

②证明:定态波函数 $\psi(x)$总可以取作实数的。

③能量本征态有可能是角动量 $\hat{L}^2$ 的本征态吗?有可能是 $\hat{L}_z$ 的本征态吗?请回答为什么
并举例说明。

\subsection{二}
在一维无限深势阱中,一个粒子的初始波函数由前两个定态迭加而成:$\psi(x,0)=A[\psi_1 (x)
+\psi_2 (x)]$。为了简化计算可令 $\omega=\pi^2\hbar/2ma^2$

①归一化$\psi(x,0)$,并求 $\psi(x,t)$和$|\psi(x,t)|^2$,把后者用时间的正弦函数展开。

②计算〈$x$〉、〈$p$〉的值。它们是随时间振荡的,角频率是多少?振幅是多少?

③测量粒子的能量,可能得到什么值?得到各个值的几率是多少?求出 $\hat{H}$ 的期望值。并
与 $E_1$ 和 $E_2$ 比较。(本题 20 分)

\subsection{三}
质量为 $m$ 的粒子在一维线性谐振子势:$V(x)=m\omega^2x^2/2$ 中运动。在占有数表示中哈密顿量可写为$\hat{H} = \left(\hat{a}^\dagger \hat{a} + \frac{1}{2}\right)\hbar\omega$,这里 
$$\hat{a}^\dagger = \sqrt{\frac{m\omega}{2\hbar}} \left( \hat{x} - \frac{i}{m\omega} \hat{p} \right), \quad \hat{a} = \sqrt{\frac{m\omega}{2\hbar}} \left( \hat{x} + \frac{i}{m\omega} \hat{p} \right)~$$分别为升、降算符。已知谐振子基态波函数为:
$$\psi_0(x) = \sqrt{\frac{m\omega}{\pi \hbar}} e^{-\frac{m\omega x^2}{2 \hbar}}~$$
