% Gauss-Lobatto 积分
% 高斯积分|数值积分|多项式

\begin{issues}
\issueAbstract
\end{issues}

\pentry{定积分\upref{DefInt}}

\textbf{高斯积分(Gauss quadrature)}可以用求和来近似表示积分
\begin{equation}
\int_a^b f(x) \dd{x} \approx \sum_i w_i f(x_i)
\end{equation}
对于某区间的一组 $x_i, w_i$, 如果它是 $N$ 阶的, 那么当 $f(x)$ 是小于等于 $N$ 阶的多项式时, 上式精确成立。 % 未完成: 不确定是 N 阶还是 $N-1$ 阶。

\subsection{Gauss-Lobatto 积分}
Gauss-Lobatto 积分中令采样点(包括两个端点)的个数为 $N$, 如果被积函数是 $2N-3$ 阶多项式($f(x) = x^{2n-3} + \dots$), 则积分没有误差。

注意 Gauss-Lobatto 积分是对称的
\begin{equation}\label{GLquad_eq5}
x_j = -x_{N-j+1}, \qquad w_{0j} = w_{0,N-j+1}
\end{equation}

其中 $x_i$ 是 $P'_{n-1}(x)$ 的根\footnote{$P_n(x)$ 在 $[-1,1]$ 有 $l$ 个根。 $P'_n(x)$ 有 $n-1$ 个根, 所以 $P'_{n-1}(x)$ 有 $n-2$ 个根, 加上两个断点就是 $n$ 个。}, $P_n(x)$ 是勒让德多项式\upref{Legen}。 另见\autoref{Legen_eq6}~\upref{Legen}。
\begin{equation}
w_i = \frac{2}{n(n-1)[P_{n-1}(x_i)]^2}
\end{equation}
参考 Wikipedia \href{https://en.wikipedia.org/wiki/Gaussian_quadrature}{相关页面}。

\subsection{正交归一基底}
每个基底都是 $N-1$ 阶多项式, 由于阶数和零点数一样, 多项式可以唯一确定, 即拉格朗日插值多项式
\begin{equation}\label{GLquad_eq2}
\ali{
p_n(x) &= \prod_{i=1}^{n-1} \frac{x-x_i}{x_n-x_i} \prod_{i=n+1}^{N} \frac{x-x_i}{x_n-x_i}\\
&= \frac{x-x_1}{x_n-x_1} \times\dots\times\frac{x-x_{n-1}}{x_n-x_{n-1}}\frac{x-x-{n+1}}{x_n-x_{n+1}} \dots \frac{x-x_N}{x_n-x_N}
}\end{equation}
\begin{equation}
p_n(x_{n'}) = \delta_{n, n'}
\end{equation}

由于任意两个基底乘积只是 $2N-2$ 阶的多项式, 所以可以用求和精确代替其积分。
\begin{equation}
\int_{-1}^1 p_i(x) p_j(x) \dd{x} = \sum_k w_k p_i(x_k) p_j(x_k) = w_i \delta_{ij}
\end{equation}

所以 $N$ 个正交归一基底就是
\begin{equation}\label{GLquad_eq3}
f_n(x) = \frac{1}{\sqrt{w_n}} p_n(x)
\end{equation}
满足
\begin{equation}
f_i(x_j) = \frac{1}{\sqrt{w_i}} \delta_{ij}
\end{equation}

另一种等效的表示方法是利用 $N$ 阶 Gauss-Lobatto 数值积分对应的多项式, $N-1$ 阶勒让德多项式的导数, $P'_{N-1}(y)$,  来构建满足条件的多项式。根据定义,其 $N-2$ 个零点分别为 $y_2, y_3\dots y_{N-1}$, 为了加入 $y_1=-1$ 与 $y_n=1$ 这两个零点,将其变为 $N$ 阶多项式
\begin{equation}\label{GLquad_eq10}
(1-y^2)P'_{N-1}(y)
\end{equation}
然而,\autoref{GLquad_eq5} 要求 $p_n(y_n)=1$, 所以我们将\autoref{GLquad_eq10} 除以它自己在 $y_n$ 处的切线,在 $y=y_n$ 处形成极限类型 $0/0=1$ 即可得到 阶多项式 $u_n(y)$。 
\begin{equation}\label{GLquad_eq11}
u_n(y) = \frac{(1-y^2)P'_{N-1}(y)}{[(1-y^2)P'_{N-1}(y)]'_y} = y_n (y-y_n)
\end{equation}
该式与\autoref{GLquad_eq2} 事实上是完全相同的多项式,因为所有具有 $N-1$ 个零点的 $N-1$ 阶多项式都可以因式分解成\autoref{GLquad_eq2} 的形式乘以一个待定常数。用\autoref{GLquad_eq11} 便于快速地展开多项式(因为勒让德多项式的系数可以直接通过公式计算)。
