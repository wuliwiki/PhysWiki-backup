% 莱昂哈德·欧拉(综述)
% license CCBYSA3
% type Wiki

本文根据 CC-BY-SA 协议转载翻译自维基百科\href{https://en.wikipedia.org/wiki/Leonhard_Euler}{相关文章}。

**莱昂哈德·欧拉**(Leonhard Euler,发音:/ˈɔɪlər/ OY-lər;[b] 德语:[ˈleːɔnhaʁt ˈʔɔʏlɐ] ⓘ,瑞士标准德语:[ˈleɔnhard ˈɔʏlər];1707年4月15日 – 1783年9月18日)是瑞士数学家、物理学家、天文学家、地理学家、逻辑学家和工程师。他是图论和拓扑学的创始人,并在其他多个数学分支(如解析数论、复分析和微积分)中做出了开创性和深远的发现。他引入了许多现代数学术语和符号,包括数学函数的概念。他还以在力学、流体动力学、光学、天文学和音乐理论等领域的贡献而闻名。

欧拉被认为是历史上最伟大、最多产的数学家之一,也是18世纪最伟大的数学家。许多在欧拉去世后才产生的伟大数学家都承认他在这一领域的重要性,正如他们的名言所示:皮埃尔-西蒙·拉普拉斯曾通过一句话表达欧拉对数学的影响:“读欧拉,读欧拉,他是我们的导师。”卡尔·弗里德里希·高斯写道:“研究欧拉的作品将是学习数学各个领域的最佳学校,其他任何东西都无法替代它。”欧拉的866篇论文和他的信件正在被收集成《欧拉全集》(Opera Omnia Leonhard Euler),完成后将包含81卷四开本。欧拉大部分成年生活都在俄罗斯圣彼得堡和普鲁士首都柏林度过。

欧拉还被认为是第一个推广使用希腊字母π(小写pi)来表示圆的周长与直径的比率的人,以及第一个使用符号f(x)来表示函数值的人。他还使用字母i表示虚数单位√(-1),使用希腊字母Σ(大写sigma)表示求和,使用希腊字母Δ(大写delta)表示有限差分,使用小写字母表示三角形的边,使用大写字母表示角度。他给出了常数e的定义,它是自然对数的底数,现在被称为欧拉数。

欧拉还被认为是第一个发展图论的人(部分因为他解决了“柯尼斯堡七桥问题”,这也被认为是拓扑学的第一个实际应用)。他还因解决多个未解的数论和分析学问题而声名远播,包括著名的巴塞尔问题。欧拉还被誉为发现了多面体的顶点和面数之和减去边数等于2,这个数字现在被称为欧拉示性数。在物理学领域,欧拉将牛顿的物理定律重新表述为新的定律,并在他的两卷本著作《力学》中更好地解释了刚体的运动。他还为固体物体的弹性变形研究做出了重大贡献。