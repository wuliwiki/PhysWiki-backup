% 单电子跃迁截面(一阶微扰)
% 电磁波|光电离

\begin{issues}
\issueDraft
\end{issues}

\pentry{微分截面\upref{ParWav}, 跃迁概率(一阶微扰)\upref{HionCr}}

\subsection{束缚态到束缚态}
从\autoref{HionCr_eq5}~\upref{HionCr} 来看, 若电子从束缚态 $\ket{j}$ 跃迁到另一个束缚态 $\ket{i}$, 跃迁概率(即吸收能量)和光谱中的 $\omega_{ij}$ 的强度成正比. 所以自然地, 可以把\textbf{束缚态之间的跃迁截面}定义为
\begin{equation}
E = \omega_{ij}P_{j\to i} = \sigma_{j\to i} s(\omega_{ij})
\end{equation}
得 % 这和 \cite{Brandsen} eq 4.46 一样, 除了偶极子近似
\begin{equation}
\sigma_{j\to i} = \frac{4\pi^2 q^2}{c m^2 \omega_{ij}} \abs{\uvec e \vdot\mel{i}{\bvec p}{j}}^2
\end{equation}

\subsection{束缚态到连续态}
对于频率为 $\omega$ 的平面电磁波\upref{VcPlWv}, 截面可以想象成与电磁波传播方向垂直放置的一块面积为 $\sigma(\omega)$ 的面元, 使得原子从电磁波中吸收的平均功率恰好等于电磁波经过该面元的功率. 对于波包, 单位频率下原子吸收的能量为
\begin{equation}
\dv{E}{\omega} = \sigma(\omega) s(\omega)
\end{equation}
其中 $s(\omega)$ 是能量面密度的频率分布. 如果再对立体角微分得某频率的微分截面 % 链接未完成
\begin{equation}
\pdv{E}{\omega}{\Omega} = \pdv{\sigma}{\Omega} s(\omega)
\end{equation}
当波包经过以后, $\bvec A = 0$. 把 $\omega$ 看作 $k$ 的函数 $\omega(\bvec k) = k^2/(2m) + I_0$, $-I_0$ 是束缚态 $\ket{j}$ 的能量, 上式积分得总能量
\begin{equation}
E = \iint \pdv{\sigma}{\Omega} s(\omega) \dd{\omega}\dd{\Omega} = \iint \pdv{\sigma}{\Omega} s(\omega) \frac{k}{m}\dd{k}\dd{\Omega}
\end{equation}
另一方面, 又有
\begin{equation}
E = \iint \omega P_{j\to {\bvec k}} k^2 \dd{k}\dd{\Omega}
\end{equation}
\addTODO{$P_{j\to i}$ 是什么?引用和说明未完成}
其中 $\omega$ 是光子能量. 对比以上两式可以把微分截面表示为 $\bvec k$ 的函数, 也可以看成是单位矢量 $\uvec k$ 和 $\omega$ 的函数($\bvec k$ 的模长由 $\omega$ 决定)
\begin{equation}
\pdv{\sigma}{\Omega} = \frac{km \omega}{s(\omega)} P_{j\to \bvec k}
\end{equation}
其中 $\omega(\bvec k) = k^2/(2m) + I_0$, 下同.

长度规范下有(\autoref{HionCr_eq6}~\upref{HionCr})
\begin{equation}
\pdv{\sigma}{\Omega} = \frac{4\pi^2 m\omega k q^2}{c} \abs{\mel{\bvec k}{\uvec e \vdot\bvec r}{j}}^2
\end{equation}
速度规范下有(\autoref{HionCr_eq5}~\upref{HionCr})
\begin{equation}\label{SIcros_eq8} % 已验证与 Merzbaucher eq 19.86 完全相同
\pdv{\sigma}{\Omega} = \frac{4\pi^2 k q^2}{c m \omega} \abs{\mel{\bvec k}{\uvec e \vdot \bvec p}{j}}^2
\end{equation}
