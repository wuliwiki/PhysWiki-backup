% 隐函数定理的不动点证明

\pentry{隐函数定理\upref{ImpFun} 巴拿赫不动点定理\upref{ConMap}}

\subsection{隐函数定理的紧凑表述}
我们可以将隐函数定理用比较紧凑的形式表达出来. 在如下的版本中, 已知的"隐式关系"$F(x,y)$
\begin{theorem}{隐函数定理}
设$(x_0,y_0)\in\mathbb{R}^n\times\mathbb{R}^m$是给定的点. 设在某个开集$B(x_0,R)\times B(y_0,R)$上定义了映射$F:U\to\mathbb{R}^m$, 满足如下条件:
\begin{enumerate}
\item $F(x_0,y_0)=0$.
\item $F(x,y)$对$y$是连续可微的, 而且雅可比矩阵
$$
\frac{\partial F}{\partial y}(x_0,y_0)
$$
是可逆的.
\end{enumerate}
则存在$r<R$以及映射
\end{theorem}
