% 小时百科感谢墙说明

\subsection{要求}

\begin{figure}[ht]
\centering
\includegraphics[width=14.25cm]{./figures/1131f8b4944872fb.pdf}
\caption{两个最小单元格} \label{fig_thanks_1}
\end{figure}

\begin{itemize}
\item 感谢墙和小时百科共存,在百科首页放置链接,永不删除。 不可抗力以及网站关闭除外。
\item 感谢墙由方形的 svg 图片组成, 每张等分为 $100\times 100$ 的\textbf{单元格}。 若一张图片填满则向下加长 100 个单元格。
\item 同一个用户可以根据捐款数额获取一个或多个单元格的编辑权, 数量为捐款金额除以 $100$ 并向下取整。
\item 小时百科的突出贡献者,经内部讨论,可赠与一定数量的单元格。
\item 禁止使用违反法律法规或公序良俗的内容。
\item 禁止交易炒作编辑权,一经证实删除所有内容。
\item 单元格位置一经选定无法移动。 同一用户的单元格位置必须连通(即使多次捐款)(对角不属于连通),因其他用户的单元格阻挡不能连通的除外。
\item 每个用户平均每个单元格的数据不能超过 1000 字节。
\item 每个用户允许在自己所属的区域中给图形添加一个超链接。
\item 可以在 svg 中插入数量合理的 png 文件。
\item 每个单元格仅允许修改 1 次。
\item 也可以申请清空, 清空无法恢复, 可以被新的用户占用。
\item 原始模板中可以提供一些公共的 \verb`<style>` 帮助用户节约代码长度, 后来的用户可以使用其他用户定义的 \verb`<style>`。
\end{itemize}

\addTODO{搞一个简单的 svg 教程}
