% 安德鲁·怀尔斯(综述)
% license CCBYSA3
% type Wiki

本文根据 CC-BY-SA 协议转载翻译自维基百科\href{https://en.wikipedia.org/wiki/Andrew_Wiles}{相关文章}。

\begin{figure}[ht]
\centering
\includegraphics[width=6cm]{./figures/7082cdefe31fa317.png}
\caption{2005年的怀尔斯} \label{fig_Andrew_1}
\end{figure}
安德鲁·约翰·怀尔斯爵士(Sir Andrew John Wiles,1953年4月11日出生)是英国数学家,现任牛津大学皇家学会研究教授,专攻数论。他最著名的成就是证明了费马大定理,因此获得了2016年阿贝尔奖和2017年科普利奖,并于2000年被任命为英国帝国勋章骑士指挥官。2018年,怀尔斯被任命为牛津大学首任Regius数学教授。怀尔斯还是1997年麦克阿瑟学者奖得主。

怀尔斯出生于剑桥,父亲是神学家莫里斯·弗兰克·怀尔斯,母亲是帕特里夏·怀尔斯。在尼日利亚度过了大部分童年后,怀尔斯对数学产生了浓厚兴趣,特别是对费马大定理产生了兴趣。怀尔斯于1974年从牛津大学毕业后,开始致力于统一伽罗瓦表示、椭圆曲线和模形式,最初从巴里·马祖尔对岩泽理论的推广开始。1980年代初期,怀尔斯在剑桥大学工作了几年,然后前往普林斯顿大学,在那里他研究并应用了希尔伯特模形式。1986年,在阅读肯·里贝特关于费马大定理的开创性工作后,怀尔斯着手证明半稳定椭圆曲线的模性定理,这意味着费马大定理得以证明。到1993年,他已经能够说服一位知识渊博的同事相信他找到了费马大定理的证明,尽管随后发现了一个缺陷。1994年9月19日,怀尔斯和他的学生理查德·泰勒通过一番洞察力克服了这个缺陷,并于1995年发表了结果,广受赞誉。

在证明费马大定理的过程中,怀尔斯为数学家们发展了新的工具,使他们能够开始统一不同的思想和定理。他的前学生泰勒和另外三位数学家于2000年证明了完整的模性定理,使用了怀尔斯的工作。在2016年获得阿贝尔奖时,怀尔斯回顾了自己的遗产,并表示他不仅证明了费马大定理,还推动了整个数学领域朝着拉格朗日计划的方向发展,从而统一了数论。
\subsection{教育与早年生活}  
Wiles于1953年4月11日出生在英国剑桥,是Maurice Frank Wiles(1923–2005)和Patricia Wiles( née Mowll)的儿子。1952年至1955年间,他的父亲在剑桥的Ridley Hall担任牧师,后来成为牛津大学的神学讲座教授。[6]  

Wiles的正式学业从他和父母一起居住在尼日利亚时开始。然而,根据他父母写的信件,在他本应上学的最初几个月里,他拒绝去上学。从这个事实中,Wiles自己得出结论,认为在他最早的几年里,他并不热衷于待在学术机构。在2021年接受Nadia Hasnaoui采访时,他表示他相信这些信件,然而他自己却不记得有任何时候不喜欢解决数学问题。[7]  

Wiles曾就读于剑桥的King's College School[8] 和The Leys School, Cambridge[9]。Wiles在1999年接受WGBH-TV采访时提到,他在10岁时从学校回家的路上遇到了《费马最后定理》。他停在当地图书馆,发现了埃里克·坦普尔·贝尔的《最后的问题》一书,讲述了这个定理。[10] 因为这个定理表述如此简单,以至于他,一个10岁的孩子都能理解,但却没人能证明出来,他对它产生了浓厚的兴趣,决定成为第一个证明它的人。然而,他很快意识到自己的知识还不够,因此放弃了童年的梦想,直到33岁时,Ken Ribet在1986年证明了ε猜想,这一猜想与Gerhard Frey早前与费马方程的联系重新把这个梦想带回了他的注意。[11]
\subsection{早期生涯}
1974年,怀尔斯在牛津大学莫顿学院获得了数学学士学位。[6] 怀尔斯的研究生研究由约翰·科茨指导,始于1975年夏季。他们共同研究了椭圆曲线的算术问题,特别是通过岩泽理论的方法来研究复乘法的椭圆曲线。他还与巴里·马祖一起研究了岩泽理论在有理数上的主猜想,随后,他将这一结果推广到完全实数域。[12][13]

1980年,怀尔斯在剑桥大学克莱尔学院获得了博士学位。[3] 在1981年于普林斯顿大学高级研究院的停留之后,怀尔斯成为了普林斯顿大学数学教授。[14]

1985至1986年,怀尔斯成为古根海姆学者,先后在巴黎的高等科学研究所和巴黎高等师范学校工作。[14]

1989年,怀尔斯被选为皇家学会会员。根据他的选举证书,他当时正在“研究附加于希尔伯特模形式的ℓ-代数表示,并已将这些应用于证明完全实数域的环面扩展的‘主猜想’”。[12]
\subsection{费马大定理的证明}
1988年至1990年,怀尔斯是牛津大学皇家学会研究教授,随后他回到了普林斯顿大学。1994年至2009年,怀尔斯担任普林斯顿大学尤金·希金斯教授。

自1986年中期起,基于此前几年由格哈德·弗雷、让-皮埃尔·塞尔和肯·里贝特等人的持续进展,逐渐清楚了费马大定理(即没有三个正整数a、b、c能满足公式 \(a^n + b^n = c^n\) 对于任何大于2的整数n)可以作为模形式定理的一种有限形式的推论来证明(当时该定理尚未被证明,并且被称为“谷山–志村–维尔猜想”)。[15] 模形式定理涉及椭圆曲线,这是怀尔斯的专业领域,并且声明所有此类曲线都有与之相关联的模形式。[16][17] 这些曲线可以看作是数学对象,类似于一个圆环表面的解,如果费马大定理是错误的且存在解,“会产生一种特殊的曲线”。因此,证明该定理就涉及到证明不存在这种曲线。[18]

该猜想被当时的数学家认为是重要的,但极其困难,甚至可能是无法证明的。[19]: 203–205, 223, 226 例如,怀尔斯的前导师约翰·科茨表示,这看起来“几乎不可能证明”,[19]: 226 而肯·里贝特则认为自己是“绝大多数认为[它]完全无法触及的人的一员”,并补充道,“安德鲁·怀尔斯可能是地球上为数不多的几位敢于梦想你实际上可以去证明[它]的人之一。”[19]: 223 

尽管如此,怀尔斯在童年时便对费马大定理产生了浓厚兴趣,他决定接受挑战,至少证明到弗雷的曲线所需的程度。[19]: 226 他将自己所有的研究时间都投入到这个问题上,近乎完全保密,发布的工作只以小段落的形式发布为独立的论文,并且只与妻子分享。[19]: 229–230 

怀尔斯的研究涉及通过反证法证明费马大定理,里贝特在1986年的工作中发现,如果该定理为真,它将有一个椭圆曲线以及相关的模形式。怀尔斯首先假设该定理是错误的,然后基于这个假设,利用里贝特的定理(该定理指出,如果n是素数,则没有这样的椭圆曲线具有模形式,因此不存在费马方程的奇素数反例),矛盾地反驳了谷山–志村–维尔猜想。怀尔斯还证明了该猜想适用于与费马方程相关的特定案例——半稳定椭圆曲线。换句话说,怀尔斯发现谷山–志村–维尔猜想在费马方程的情况下是成立的,里贝特的发现(即如果猜想对半稳定椭圆曲线成立,则费马大定理为真)得到了验证,从而证明了费马大定理。[20][21][15]

1993年6月,怀尔斯在剑桥的一次会议上首次向公众展示了他的证明。纽约时报的吉娜·科拉塔总结了这一讲座如下:

他每周一、周二和周三进行一次讲座,标题为“模形式、椭圆曲线和伽罗瓦表示”。讲座标题没有暗示会讨论费马大定理,里贝特博士说... 最后,在第三次讲座的结束时,怀尔斯博士总结道,他已证明了谷山猜想的一般情况。然后,似乎是作为附带提及,他提到这意味着费马大定理为真。Q.E.D.[18]

1993年8月,发现证明中存在多个问题,涉及到塞尔默群的性质和使用一种称为欧拉系统的工具。[22][23] 怀尔斯尝试并未成功修正他的证明。根据怀尔斯的说法,绕过这一问题的关键想法出现在1994年9月19日,当时他几乎要放弃了。这个绕过方法使用了伽罗瓦表示来替代椭圆曲线,将问题简化为类数公式并解决了它,此外,还使用了维克托·科里瓦金的思想来修正马提亚斯·弗拉赫在岩泽理论中的方法。[23][22] 怀尔斯和他的前学生理查德·泰勒一起,发表了第二篇论文,其中包含了绕过部分,从而完成了证明。这两篇论文于1995年5月在《数学年刊》上专门刊登。[24][25]
\subsection{后期事业}
2011年,怀尔斯重新加入牛津大学,担任皇家学会研究教授。[14]

2018年5月,怀尔斯被任命为牛津大学的Regius数学教授,成为该校历史上的第一位此职位的教授。[4]
\subsection{遗产}  
怀尔斯的工作在数学的多个领域得到了应用。特别是,1999年,他的三位前学生理查德·泰勒、布莱恩·康拉德和弗雷德·戴蒙德,与克里斯托夫·布雷伊尔合作,在怀尔斯的证明基础上证明了完整的模形式定理。[26][15] 怀尔斯的博士生还包括曼朱尔·巴尔加瓦(2014年菲尔兹奖得主)、埃胡德·德·沙利特、瑞塔布拉塔·穆恩希(SSB奖和ICTP拉马努金奖得主)、卡尔·鲁宾(维拉·鲁宾的儿子)、克里斯托弗·斯金纳和维纳亚克·瓦特萨尔(2007年科克塞特–詹姆斯奖得主)。

2016年,在获得阿贝尔奖时,怀尔斯谈到了他对费马大定理的证明:“解决它的方法开辟了攻击当代数学中一个被称为Langlands计划的巨大猜想网络的新途径,这一宏大的愿景试图统一数学的不同分支。它为我们提供了一个全新的视角来看待这一问题。”[5]
\subsection{奖项与荣誉}
\begin{figure}[ht]
\centering
\includegraphics[width=6cm]{./figures/2dae0ef0581b2f6a.png}
\caption{1995年,安德鲁·怀尔斯在皮埃尔·德·费马雕像前合影,地点位于法国南部博蒙德洛曼热,费马的出生地。} \label{fig_Andrew_2}
\end{figure}
怀尔斯对费马大定理的证明经得起世界其他数学专家的严格审查。怀尔斯曾接受BBC纪录片系列《Horizon》关于费马大定理的一集采访。[27] 这集节目后来作为PBS科学电视系列《Nova》的一集播出,题为《The Proof》[10]。他的工作和生活也在西蒙·辛格的畅销书《费马大定理》中有详细描述。

1988年,怀尔斯获得伦敦数学学会的青年怀特黑德奖。[6] 1989年,他当选为皇家学会会员(FRS)[28][12]。

1994年,怀尔斯当选为美国艺术与科学院院士。[29] 1995年,怀尔斯完成费马大定理的证明后,获得了肖克奖、费马奖[30] 和沃尔夫数学奖。[14] 1996年,怀尔斯当选为美国国家科学院外籍院士[13],并获得美国国家科学院数学奖、皇家奖章和奥斯特罗夫斯基奖。[32] 1997年,他获得了美国数学学会的科尔奖[33]、麦克阿瑟基金会奖学金以及沃尔夫斯凯尔奖[34],同年当选为美国哲学学会会员。[35]

1998年,怀尔斯获得国际数学联盟的银奖牌,以表彰他的成就,作为代替菲尔兹奖的奖励,因为菲尔兹奖仅限于40岁以下的数学家(怀尔斯在1994年证明定理时已经41岁)。[36] 同年,他获得了费萨尔国王奖[37],并在1999年获得克莱研究奖,[14] 同年,小行星9999 Wiles以他的名字命名。[38]

2000年,他被授予大英帝国爵士指挥官勋章(2000年)[39]。2004年,怀尔斯获得了皮塔哥拉奖。[40] 2005年,他获得了肖奖。[30]

2016年,牛津大学的数学研究所大楼以怀尔斯的名字命名。[41] 同年,他获得了阿贝尔奖。[42][43][44][45][46] 2017年,怀尔斯获得了科普利奖章。[1] 2019年,他获得了德·摩根奖章。[47]
\subsection{另见}  
\begin{itemize}
\item 安德烈·魏尔
\end{itemize}
\subsection{参考文献}  
\begin{enumerate}
\item "数学家安德鲁·怀尔斯爵士获得皇家学会的著名科普利奖"。皇家学会。检索于2017年5月27日。  
\item 安德鲁·怀尔斯在数学家族树项目中的资料。  
\item Wiles, Andrew John (1978)。《互惠律与伯奇-斯温内顿-戴尔猜想》(博士论文)。剑桥大学。OCLC 500589130。EThOS uk.bl.ethos.477263 - 通过剑桥大学图书馆。  
\item "安德鲁·怀尔斯爵士被任命为牛津大学首任皇家数学教授"。新闻与事件。牛津大学。2018年5月31日。检索于2018年6月1日。  
\item Sample, Ian (2016年3月15日)。"牛津教授因证明费马大定理而获得阿贝尔奖"。《卫报》。检索于2023年11月20日。  
\item Anon (2017)。"Wiles, Sir Andrew (John)"。《谁是谁》(在线牛津大学出版社版)。牛津:A & C Black。doi:10.1093/ww/9780199540884.013.39819。(需要订阅或英国公共图书馆会员)。  
\item "与安德鲁·怀尔斯的访谈"。阿贝尔奖。2021年3月10日。检索于2023年11月15日 – 通过YouTube。  
\item "校友"。剑桥国王学院学校。检索于2022年2月1日。  
\item "剑桥莱斯学校老生,安德鲁·怀尔斯教授获得科普利奖"。莱斯与圣菲斯学校基金会。2017年11月2日。检索于2022年2月1日。  
\item "安德鲁·怀尔斯谈解费马大定理"。WGBH。2000年11月。检索于2016年3月16日。  
\item Chang, Sooyoung (2011)。《数学家学术家谱》。世界科学出版社。第207页。ISBN 9789814282291。  
\item "EC/1989/39: Wiles, Sir Andrew John"。皇家学会。2015年7月13日存档。检索于2016年3月16日。  
\item "安德鲁·怀尔斯"。美国国家科学院。检索于2016年3月16日。  
\item O'Connor, John J.; Robertson, Edmund F. (2009年9月)。"安德鲁·约翰·怀尔斯传记"。《MacTutor数学史档案》。检索于2022年2月1日。  
\item Darmon, Henri (1999年12月)。《宣布完整的Shimura-Taniyama-Weil猜想证明》(PDF)。《美国数学会通报》。46(11)。美国数学会:1397-1401。检索于2024年8月1日。  
\item Brown, Peter (2015年5月28日)。"数学最著名的证明如何差点破产"。《海洋》杂志。2016年3月15日存档。检索于2016年3月16日。  
\item Broad, William J. (2022年1月31日)。"科学中的人物档案——敢于挑战数学不可能之谜的德克萨斯石油继承人——詹姆斯·M·沃恩二世凭借财富,声称他促成了费马突破"。《纽约时报》。检索于2022年2月2日。  
\item Kolata, Gina (1993年6月24日)。"终于,‘尤里卡’的喊声响起,破解了古老的数学谜团"。《纽约时报》。2013年1月21日存档。  
\item Simon Singh (1997)。《费马大定理》。ISBN 1-85702-521-0  
\item Stevens, Glenn H.(无日期),《费马大定理证明概述》(PDF),波士顿大学
- Boston, Nick (2003年春),《费马大定理的证明》(PDF),威斯康星大学麦迪逊分校  
- Faltings, Gerd (1995年7月)。 "R. Taylor和A. Wiles证明费马大定理"(PDF)。《美国数学会通报》42(7)。美国数学会:743-746。检索于2024年8月1日。  
- Cipra, Barry Arthur (1995)。《"普林斯顿数学家回顾费马证明"》。《科学》268 (5214):1133–1134。Bibcode:1995Sci...268.1133C。doi:10.1126/science.268.5214.1133。PMID 17840622。  
- Wiles, Andrew (1995年5月)。《"第3期"》。《数学年刊》141:1–551。JSTOR i310703。  
- "数学家们是否终于满意安德鲁·怀尔斯的费马大定理证明?为什么这个定理如此难以证明?"。《科学美国人》。1999年10月21日。检索于2016年3月16日。  
- Devlin, Keith (1999年7月21日)。《"超越费马大定理"》。《卫报》。检索于2023年11月20日。  
- "BBC TWO,Horizon 费马大定理"。BBC。2010年12月16日。检索于2014年6月12日。  
- "安德鲁·怀尔斯爵士KBE FRS"。伦敦:皇家学会。2015年11月17日存档。检索于2022年2月1日。前述句子中的部分内容取自royalsociety.org网站,所有'传记'部分的内容均在知识共享署名4.0国际许可协议下发布。  
- "安德鲁·J·怀尔斯"。美国艺术与科学院。检索于2021年12月10日。  
- "怀尔斯获得2005年肖奖"。美国数学会。检索于2016年3月16日。  
- "NAS数学奖"。美国国家科学院。2010年12月29日存档。检索于2011年2月13日。  
- "怀尔斯获得奥斯特罗斯基奖"。美国数学会。检索于2016年3月16日。  
- "1997年科尔奖,AMS通报"(PDF)。美国数学会。2022年10月9日存档(PDF)。检索于2008年4月13日。  
- "保罗·沃尔夫斯凯尔与沃尔夫斯凯尔奖"。美国数学会。检索于2016年3月16日。  
- "APS会员历史"。search.amphilsoc.org。检索于2021年12月10日。  
- "安德鲁·J·怀尔斯获得'国际数学联盟银奖'"。美国数学会。1953年4月11日。检索于2014年6月12日。  
- "安德鲁·怀尔斯获得费萨尔奖"(PDF)。美国数学会。2022年10月9日存档(PDF)。检索于2014年6月12日。  
- "JPL小天体数据库浏览器"。美国宇航局。检索于2009年5月11日。  
- "编号55710"。《伦敦公报》(补充)。1999年12月31日,第34页。  
- "皮塔戈拉奖"(意大利语)。卡拉布里亚大学。2014年1月15日存档。检索于2016年3月16日。  
- "数学研究所"。牛津大学。2016年1月13日存档。检索于2016年3月16日。  
- Castelvecchi, Davide (2016)。 "费马大定理使安德鲁·怀尔斯获得阿贝尔奖"。《自然》531(7594):287。Bibcode:2016Natur.531..287C。doi:10.1038/nature.2016.19552。PMID 26983518。  
- "英国数学家安德鲁·怀尔斯获得阿贝尔数学奖"。《华盛顿邮报》。美联社。2016年3月15日。2016年3月15日存档。  
- McKenzie, Sheena (2016年3月16日)。 "300年历史的数学问题被解答,教授赢得70万美元奖学金"。《CNN》。  
- "一位英国数学家因22年前解开这个迷人的几个世纪历史的数学问题而获得70万美元奖学金"。《商业内幕》。检索于2016年3月19日。  
- Iyengar, Rishi。 "安德鲁·怀尔斯因费马大定理获得2016年阿贝尔奖"。《时代》。检索于2016年3月19日。  
- "LMS德摩根奖得主"。《MacTutor数学史档案》。圣安德鲁斯学院。检索于2024年1月29日。
\end{enumerate}