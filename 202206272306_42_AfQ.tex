% 仿射群
% keys 仿射群|正合列

\begin{issues}
\issueDraft
\end{issues}

\pentry{仿射空间\upref{AfSp}}
\textbf{仿射群}是在仿射空间利用仿射映射(\autoref{AfSp_def2}~\upref{AfSp})构建的一种群\upref{Group}.让我们进行如下思考:群首先要满足封闭性,即两个群元作用(或运算)得到的还是一个群元,对仿射群而言,群元是仿射映射,映射的运算很自然的用映射的复合(\autoref{map_sub2}~\upref{map})表示.那么封闭性要求对两仿射映射 $f,g$,$fg$还是仿射映射.映射的复合要求 $f$ 的定义域起码得包含 $g$ 的值域.记
 \begin{equation}
f:\mathbb A\rightarrow\mathbb A_1,\quad g:\mathbb A_2\rightarrow \mathbb A_3
 \end{equation}
 则 $\mathbb A_3\subset \mathbb A$.而任意两个群元都可以进行运算的,即还有 $gf$ 也是仿射映射,同理,这意味着 $\mathbb A_1\subset\mathbb A_2$.即
 \begin{equation}\label{AfQ_eq1}
 \mathbb A_3\subset \mathbb A,\quad \mathbb A_1\subset\mathbb A_2
 \end{equation}
 
 其次,群元必定得有逆元,对映射而言,就是逆映射得存在,这表明
\begin{equation}
 f^{-1}:\mathbb A_1\rightarrow\mathbb A,\quad g^{-1}:\mathbb A_3\rightarrow\mathbb A_2
\end{equation}
 存在,和前面一样,又有
 \begin{equation}\label{AfQ_eq2}
 \mathbb A_2\subset \mathbb A_1, \mathbb A\subset\mathbb A_3
 \end{equation}
 \autoref{AfQ_eq1} ,\autoref{AfQ_eq2} 联立,就有
 \begin{equation}
 \mathbb A=\mathbb A_1= \mathbb A_2=\mathbb A_3
 \end{equation}
 也就是说,仿射群是由仿射空间 $(\mathbb A,V)$ 上的所有自同构 $f:\mathbb A\rightarrow\mathbb A$ 实现的.由仿射映射的定义:
 \begin{equation}\label{AfQ_eq3}
 f(\dot p+v)=f(\dot p)+Df\cdot v
 \end{equation}
 显然,这里 $Df$ 是 $V\rightarrow V$ 上的可逆的线性映射(\autoref{AfSp_the1}~\upref{AfSp}).这意味着,$Df$ 是一个可逆的线性算子(\autoref{LiOper_sub4}~\upref{LiOper}),可记为 $\mathcal F=Df$.于是\autoref{AfQ_eq3} 变成
 \begin{equation}
 f(\dot p+v)=f(\dot p)+\mathcal F v
 \end{equation}
 \subsection{仿射群}
 \begin{definition}{仿射群}
 设 $n$ 仿射空间 $(\mathbb A,V)$ 定义在域 $\mathbb F$ 上,则所有仿射自同构配上映射复合构成的集合 $\mathrm{Aff}(\mathbb A)=A_n(\mathbb F)$ 称为仿射空间 $\mathbb A$ 上的 $n$ 维\textbf{仿射群}.
 \end{definition}
 用 $e$ 表示仿射群的单位元,它是\textbf{单位仿射变换}(或\textbf{恒等变换}),其线性部分为 $V$ 上的单位算子 $\mathcal E$.

容易证明: $e$ 是单位仿射变换,当且仅当存在一点 $\dot q$ ,使得 $e(\dot o)=\dot o$,且 $e$ 的线性部分为 $\mathcal E$.事实上
 \begin{equation}
  e(\dot p)=e(\dot q+\overrightarrow{qp})=e(\dot q)+\mathcal E \overrightarrow{qp}=\dot q+\overrightarrow{qp}=\dot p
 \end{equation}
反过来,$e$ 是单位仿射变换,则 $\forall\dot q$,都有 $e(\dot q)=\dot q$.于是
\begin{equation}
\begin{aligned}
e(\dot p+v)=e(\dot p)+De\cdot v&=\dot p+De\cdot v=\dot p+v\\
&\Downarrow\\
De&=\mathcal E
\end{aligned}
\end{equation}
故证得结论.
\begin{example}{}
若 $f,g$ 是线性部分分别为$\mathcal F,\mathcal G$ 的两个仿射自同构,试证明,$fg$ 的线性部分为 $\mathcal {F,G}$.

\textbf{证明}:\begin{equation}
(fg)(\dot p+v)=f(g(\dot p+v))+f(g(\dot p)+\mathcal G v)=fg(\dot p)+\mathcal {FG}v
\end{equation}
\textbf{证毕!}
\end{example}
\begin{theorem}{}
所以保持点 $\dot o$ 不动的仿射自同构构成的集合是一个子群 $A_n(\mathbb F)_{\dot o}\in A_n(\mathbb F)$,且同构于完全线性群 $GL(V)=GL_n(\mathbb F)$.空间 $\mathbb A$ 的所有平移构成的子群 $T=\{t_v|v\in V\}$ 在群 $A_n(\mathbb F)$ 中是正规的.
\end{theorem}
\textbf{证明:} $\forall f,g\in A_n(\mathbb F)_{\dot o}$ ,