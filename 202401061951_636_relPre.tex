% 相对论效应造成的近日进动
% keys 相对论|进动
% license Usr
% type Wiki

\pentry{开普勒问题\upref{CelBd},狭义相对论,分析力学}



对于开普勒问题,还需要考虑由于太阳的强引力对轨道产生的影响。这里仅讨论狭义相对论,不考虑广义相对论造成的修正。

\subsection{观测值与其他星体造成的影响理论值不符}
对于水星近日点的总进动值(约每世纪 $570''$),其他行星对水星的影响约仅有观测值的 $93 \%$,特别是木星、金星与地球(占约 $91\%$),人们发现有微小偏差,广义相对论的修正约是每世纪进动 $43''$,与其他星体造成的影响合并后恰好符合观测值。

特别的,广义相对论的修正包含了狭义相对论的修正,狭义相对论的修正结果只有广义相对论的约 $1/6$。

\subsection{狭义相对论修正的近日点进动}
这个问题一般被称为\textbf{相对论性开普勒问题},在经典力学中可以严格求解。第一个解决该问题的是索墨菲(Sommerfeld)。

\subsubsection{分析力学求近似解}
为简化计算,采用速度满足光速 $c=1$ 的单位制。

在狭义相对论中,粒子的拉氏量忽略常数 $-mc^2 = -m$ 后为:
$$L = -mc^2 \sqrt{1- \frac{{\bvec v}^2}{c^2}} = \frac{m}{2} {\bvec v}^2 \left( 1+\frac{1}{4} {\bvec v}^2 + \cdots \right) + \frac{\alpha}{r} ~.$$
其中省略的内容是 ${\bvec v}^2$ 的高次项,都是更高阶的相对论修正。现仅考虑近似解,故将其忽略。

开普勒问题仍然满足在一个二维平面上,采用极坐标表示速度 $\bvec v$,则:
$${\bvec v}^2 = \dot r^2 + r^2 \dot \phi^2 ~.$$
令 $E$ 为扣除粒子静止能量 $mc^2 = m$ 之后的能量。那么相对论性开普勒问题的系统能量 $E$、角动量 $p_\phi \equiv J$ 仍然为守恒量。对于角动量满足:
$$p_\phi = \frac{\partial L}{\partial {\dot \phi}} \equiv \mathcal J \approx mr^2 \dot \phi \left( 1+ {\bvec v}^2/2 \right)~.$$
而对于 $r$,其共轭动量为:
$$p_r = \frac{\partial L}{\partial {\dot r}} \approx m \dot r \left( 1 + {\bvec v}^2/2 \right)~.$$

那么粒子的能量可以写作:
\begin{equation}
\begin{aligned}
E &= \dot r p_r + \dot \phi p_\phi - L \\
&\approx m {\bvec v}^2(1+{\bvec v}^2/2)-L \\
&= \frac{1}{2} m {\bvec v}^2 \left( 1 + \frac{3}{4} {\bvec v}^2 \right) - \frac{\alpha}{r} ~.
\end{aligned}
\end{equation}

我们仅关心轨道形状,而非与时间依赖关系,因此考虑将上式用 $\dd{r}/\dd{\phi}$ 表示。其中 $\phi$ 考虑用守恒量 $\mathcal J$ 表示,这样不显含时。
$$\dot \phi \approx \frac{\mathcal J}{mr^2} \left( 1-{\bvec v}^2/2 \right) ~.$$

实际上这个公式并没有完全求解,因为 $\bvec v$ 中还含有 $\dot \phi$ 项。但仍可以将其考虑为一个近似解。其中相对论修正的部分应当用原来问题的零级近似,也就是无相对论修正时的开普勒问题的解代入。

由于:
$$\frac{\dd r}{\dd t} = \frac{\dd r}{\dd \phi} \frac{\dd \phi} {\dd t}~.$$

也就是 $\dot r = \dot \phi \cdot \dd{r}/\dd{\phi}$。仍然采用开普勒问题常用的代换 $u = 1/r$ 可以得到:
\begin{equation}
\dot r \approx - \frac{\mathcal J}{m} \frac{\dd u}{\dd \phi} \left(1-{\bvec v}^2/2\right)
~.\end{equation}


将 $\dot \phi$ 和 $\dot r$ 的表达式代入速度的表达式便可以得到粒子动能的近似表达式:
\begin{equation}
\frac{1}{2} m \bvec{v}^2 \approx \frac{\mathcal J^2}{2m} \left[\left(\frac{\dd u}{\dd \phi}\right) + u^2\right] (1-\bvec{v}^2)~.
\end{equation}

与粒子能量的表达式合并,可以得到粒子总能量的表达式:
\begin{equation}
E \approx \frac{\mathcal J^2}{2 m} \left[ \left(\frac{\dd u}{\dd \phi}\right) + u^2 \right] - \frac{1}{8} m \left(\bvec{v}^2\right)^2 -\alpha u ~.
\end{equation}

可以发现,其中仅有第二项为修正项,仍考虑用零级近似的 $\bvec{v}^2$ 来替代,利用:
$$\frac{1}{2} m \bvec{v}^2 \approx E + \alpha / r = E + \alpha u ~,$$
来替换相对论修正得到的 $\bvec{v}^2$,就得到了狭义相对论修正的
\begin{equation}
-\frac{m}{8} (\bvec v)^2 \approx - \frac{1}{2m} \left(E + \alpha u\right)^2 ~.
\end{equation}

将这个关系代入能量的表达式中,相比较零级近似,多了一个常数项仅改变能量的定义、一个线性依赖于 $u$ 的项仅影响轨道的尺度以及一个正比于 $u^2$ 的项影响轨道的闭合。

现在的轨道方程应当近似于满足:
\begin{equation}
E\left(1 + \frac{E}{2mc^2}\right) + \alpha u \left(1 + \frac{E}{mc^2}\right) = \frac{\mathcal J}{2m} \left[ \left(\frac{\dd u}{\dd \phi}\right)^2 + \left( 1-\frac{\alpha^2}{\mathcal J^2 c^2} \right) u^2 \right]~.
\end{equation}

解之,并于椭圆轨道 $l_0 u = 1 + e \cos \phi$ 对比可以发现狭义相对论修正后的星星运行一周进动角为:
\begin{equation}
\delta \phi = \frac{\pi \alpha^2}{\mathcal J^2c^2} = \frac{\pi G M}{a (1-e^2) c^2} ~.
\end{equation}


\subsubsection{哈密顿力学求精确解}
仍然取 $c=1$,此时粒子的哈密顿量可以表达为:
$$H = \sqrt{\bvec{p}^2 + m^2} - \alpha/r ~.$$

对于这个二维平面问题,仍考虑采用极坐标 $(r, \theta)$,有 $\bvec{p}^2 = p_r^2 + p_\theta^2/r^2$,从而有:
$$H = \sqrt{p_r^2+p_\theta^2/r^2 -m^2}-\alpha/r ~.$$

哈密顿量不显含时、也不显含 $\theta$,故哈密顿量的数值即机械能 $E$、角动量 $p_\theta \equiv  L$ 都是守恒量,有:
$$\left(E + \alpha/r\right) ^2 = p_r^2 + L^2/r^2 + m^2 ~.$$

另外,$p_r = \gamma m \dot r$、$p_\theta = L = \gamma m r^2 \dot \theta$。故有:
$$p_r = \frac{p_\theta}{r^2}\frac{\dd r}{\dd \theta} = \frac{L}{r^2} \frac{\dd r}{\dd \theta} = -L \frac{\dd u}{\dd \theta} ~.$$

其中 $u$ 仍为开普勒问题的常用代换 $u = 1/r$。

那么粒子径向运动的方程可以写为:
$$ (E+\alpha u)^2 = L^2 [(\dd u/\dd \theta)^2 + u^2] + m^2 ~,$$

这个方程看起来不是很明显,再对 $\theta$ 求导一次得到:
$$\frac{\mathrm{d}^2 u}{{\dd \theta ^2}} + \left(1-\frac{\alpha^2}{L^2}\right) u = \frac{\alpha E}{L} ~.$$

这是一个典型的简谐运动的微分方程,解为:
$$u \equiv 1/r = (1 + e \cos[\Omega (\theta - \theta_0)])/l_0 ~.$$

其中 $\theta_0$ 仅代表极轴的选取,剩余参数由下列式子给出:
$$\Omega^2 = 1-\frac{\alpha^2}{L^2}, l_0 = \frac{L^2}{\alpha E} \left( 1-\frac{\alpha^2}{L^2} \right), \\
e^2 = 1 + {\left(1-\frac{\alpha^2}{L^2}\right) \left(1-\frac{m^2}{E^2}\right)/(\alpha^2 / L^2) } ~,$$

特别的,这里的 $E$ 包含了粒子的静止能量,与分析力学求的近似解进行比较的话需要取:
$$E \approx m + E_0, |E_0| \ll m ~.$$

需要注意的是,这里 $E_0$ 可以小于 $0$,对应束缚的情形。同时,在非相对论极限下 $\alpha^2 \ll L^2 \equiv L^2c^2$。