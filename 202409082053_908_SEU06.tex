% 东南大学 2006 年 考研 量子力学
% license Usr
% type Note

\textbf{声明}:“该内容来源于网络公开资料,不保证真实性,如有侵权请联系管理员”

\subsection{10分}\\
(1) 对波长为5000埃的单色光,求其光子能量(用电子伏特表示)。\\
(2) 对动能为 $E_k$,质量为 $m$ 的非相对论性自由粒子,求其物质波波长。
\subsection{10分}
已知 $A = \begin{pmatrix} a & b+ci \\ 1 & 2 \end{pmatrix}$ 为厄米矩阵(其中 $ b, c$ 为实数),它的一个本征值为1。\\
(1) 求 $a, b, c$ 的值。\\
(2) 求矩阵A的另一个本征值。
\subsection{10分}\\
(1) 写出自旋算符 $S_x, S_y, S_z$ 之间的对易关系。\\
(2) 计算算符 $[L_x, p_y]$ 和 $[L_z, z]$。
\subsection{10分}\\
已知单个粒子在势场 $V(x)$ 中的能级,从低到高依次为 $E_0, E_1, E_2, \dots$。在该势场中放入2个相同的费米子,设其自旋量子数为1/2,忽略粒子的相互作用。\\
(1) 写出该系统基态和第一激发态的能量值。\\
(2) 若考虑粒子的相互作用,基态能量会如何变化?这种变化是正向的还是反向的?\\
(3) 如果是放入5个这样的费米子,写出你所能给出的能量值。
\subsection{10分}\\
(1) 写出电偶极辐射的角动量选择定则(不考虑电子自旋)。\\
(2) 考虑氢原子从基态($\psi_{100}$)至第一激发态($\psi_{200}, \psi_{210},\psi_{210}, \psi_{21-1}$)的跃迁。写出电偶极辐射角动量选择定则所允许的跃迁方式,并计算所需吸收的光子能量(用电子伏特表示)。
\subsection{20分}
在一维无限深方势阱
\[
V(x) = 
\begin{cases}
0, & 0 \leq x \leq L \\
\infty, & \text{其他}
\end{cases}~
\]
中,有一质量为 $m$ 的粒子处于基态。在某一时刻,势阱突然变为 :
\[
V_2(x) = 
\begin{cases}
0, & 0 \leq x \leq 2L \\
\infty, & \text{其他}
\end{cases}~
\]
(1) 分别求在 $V_1(x)$ 和 $V_2(x)$ 中粒子的基态能量和归一化波函数。\\
(2) 假设在势阱改变的瞬间,粒子的波函数并未发生变化,求粒子在 $V_2(x)$ 中仍处于基态的几率。
\subsection{20分}
一粒子在1维的波函数为 $C[\psi_1(x) + 2\psi_2(x)]$,这里 $\psi_1(x)$ 和 $\psi_2(x)$ 分别是粒子能量为 $K$ 和 $2K$ 的归一化定态波函数。
\begin{enumerate}
    \item 求归一化常数 $C$ 以及 t 时刻的波函数。
    \item 求粒子的平均能量,求测量值具有几率的能量值。
    \item 求粒子的能量涨落值 $ \vec E $ 以及能量的涨落 $\Delta E$。
\end{enumerate}
\subsection{20分}
氢原子基态波函数为 $\psi_{100}(r, \theta, \phi) = \frac{1}{\sqrt{\pi a^3}} \exp(-r/a)$,这里 $a$ 为玻尔半径。
\begin{enumerate}
    \item 求在 $r \to r+dr$ 球亮内找到电子的几率。
    \item 求玺原子基态的最可几半径(在该半径处,电子径向分几率密度最大)。
    \item 若氢原子基态,求 $\frac{1}{r}$ 的平均值。
\end{enumerate}
\subsection{20分}
已知有两个角量子数 $l=1$ 的粒子组成了系统,其总角量子数为 $H = K L_1 \cdot L_2$,这里 $L_1, L_2$ 分别是两个粒子的轨道角动量,$K$ 是表征系统特性的常量。指出合成这两个角动量的可能结果,写出结果的量子数 $l$ 取值,并求出相关的几率。
