% 电场的能量
% 电场|电场能|能量密度

\begin{issues}
\issueDraft
\end{issues}

\pentry{电容\upref{Cpctor}}

首先我们来回顾一下平行板电容器(\autoref{Cpctor_ex2}~\upref{Cpctor})的能量
\begin{equation}
E_p = \frac12 CV^2 = \frac12 \epsilon \frac Sd (Ed)^2 = \frac 12 \epsilon \tau E^2
\end{equation}
其中 $\tau = Sd$ 为平行板间长方体的体积. 这条公式容易让我们联想到电势能储存在电场之中. 在电动力学中, 这种理解是正确的, 我们不妨把电场含有的能量称为\textbf{电场能}. 通过上式, 我们假设空间中任意一点的电场能密度为 $\epsilon E^2/2$, 则总电场能可以通过体积分得到
\begin{equation}
E_p = \frac12 \epsilon \int \bvec E(\bvec r)^2 \dd[3]{r}
\end{equation}
可以证明该式定义的电场能与\autoref{QEng_eq8}~\upref{QEng} 定义的电势能是完全等效的
\begin{equation}
123
\end{equation}


\subsection{证明}
(未完成)
