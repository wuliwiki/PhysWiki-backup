% 循环群
% license Usr
% type Tutor


\begin{issues}
\issueTODO 插图,定理5
\end{issues}

\begin{definition}{}
若群$G$的元素都是由某个元素生成,即$G=<a>$,则称群$G$为\textbf{循环群}(cyclic group),称$a$是循环群的生成元。
\end{definition}

显然,循环群的群元都可以表示为生成元的整数次幂,因此循环群实际上是阿贝尔群。
\begin{example}{}
整数群$\mathbb Z$是有无限元素的循环群,群乘法为加法。
\end{example}
\begin{example}{}
模$n$同余类$\mathbb Z_n$。
\end{example}
\begin{example}{}
群$G=\{-1,-\mathrm i,1,\mathrm i\}=<\mathrm i>$。
\end{example}
循环群的形式看似没有什么规律,但我们可以借助同构来缩减研究对象。
\begin{theorem}{}
无限循环群同构于整数加群;$n$元有限循环群同构于$\{\mathbb Z_n;+\}$。
\end{theorem}
\textbf{proof.}\footnote{参考《抽象代数》,邓少强祝,朱富海著。}

设$G$是无限循环群,建立$\mathbb Z\rightarrow G$的同态映射,使得对于任意$n\in \mathbb Z$,都有$f(n)=a^n$。根据群同态基本定理\autoref{exe_Group2_1}~\upref{Group2},我们有$\mathbb Z/\opn{ker}f\cong G$。由于$\mathbb Z$的正规子群都是$n\mathbb Z,n\in \mathbb N$,因此模$n$同余类与$G$同构。当该$n=0$时,对应无限循环群;当$m\neq 0$时,$\mathbb Z_n$与n元循环群同构。

因为有限循环群可以继承整数群的乘法,因此还是一个环。可以证明,$\mathbb Z_n$环上的零因子是$n$的因子。所以,如果$n$是素数,那么这个环就是\textbf{无零因子交换幺环}了,我们一般简称其为\textbf{整环}。
\begin{theorem}{}
有限整环必是域。
\end{theorem}
只要证明任意环元都有逆在环内即可。
因为是有限整环,假设生成元为$a$,则由封闭性知对于每个非零同余类都必有$a^m=a^n,m\neq n$。设$m> n$,因为$a^{m-n}a^n=1\cdot a^n$,所以$a^{m-n}=1$\footnote{注意这是环上的乘法,由于乘法运算构成半群,消去律未必成立。若对于环上元素有$ab=cb$且$b\neq 0$,则$(a-c)b=0$。由于整环没有零因子,所以$a=c$,即消去律对整环必然成立。}。因此,$a^{m-n-1}$为$a$的逆元,$a^{k(m-n-1)}$是$a^k$的逆元,证毕。
\begin{exercise}{}
无限循环群的生成元只有两个。
\end{exercise}
\begin{exercise}{}
若$n$元有限循环群$G=<a>$,那任意元素$a^k,k\in N$的阶是多少。一个推论是:若$k$与$n$互素,则$<a>=<a^k>$,因此我们可以得知一个$n$元循环群的生成元个数。
\end{exercise}
\subsection{循环群的子群结构}
\begin{theorem}{}\label{the_cyclic_1}
循环群的子群必是循环群。
\end{theorem}
\textbf{proof.}
设循环群$G=<a>$,则其子群元素必定包含$a$的某次幂。设任意子群为$G_1$包含元素的最小次幂为$k$,则该子群包含$a^{kn},n\in \mathbb Z$。若该群不是循环群,必然包含元素形如$a^{kn+r},0<r<k$。由封闭性知也包含$a^{-kn}=(a^{kn})^{-1}$。则$a^{r}\in G^{1}$,与假设里$k$是最小正整数矛盾,所以循环群的子群必是循环群。
\begin{theorem}{}
设$G$是$n$元循环群,若$d|n$,则$G$内存在唯一一个$d$阶子群。
\end{theorem}
\textbf{proof.}

因为$a^{\frac{n}{d}d}=e$,因此$<a^{\frac{n}{d}}>$是一个$d$阶子群。由\autoref{the_cyclic_1} 知,$d$阶子群都是循环群,则都有对应的$d$阶生成元。
\begin{theorem}{}
若群$G$的不同子群阶数不同,则$G$是循环群。
\end{theorem}
\textbf{proof.}\footnote{引自《代数学基础》,Jier Peter著}

设$H$为$G$的任意子群,因为共轭子群的阶与原群相等,即对于任意$g\in G$都有$|gHg^{-1}|=|H|$,题设条件使得$G$的任意子群都是正规子群——$H\lhd G$。设$G'=G/H$且$H_1$是$G'$的任意子群,那么对于$G$而言,$H_1$是运算封闭的左陪集之并,也就是说,$H_1$也是原群的子群,题设及“任意子群都是正规子群”在商群意义上得以继承。

因此,我们可以利用循环子群来构造商群列。从$G$里选任意元素$x_1$,构造商群$G_1=G/<x_1>$,从$G_1$中选任意元素$x_2$,构造商群$G_2=G_1/<x_2>$,以此类推——
\begin{equation}
G>G_1>G_2...G_{s-1}>G_S=\{e\}~.
\end{equation}
我们知道,根据拉格朗日定理,商群的基数必是原群基数的因子,因此商群列总是有限的,最后终结于平凡群。又因为该平凡群是$G_{s-1}$商去循环群得到的,所以$G_{s-1}$必是循环群。

接下来我们只需要证明,对于$G_{n}=G_{n-1}/<x_n>$,若满足$G_{n}$为循环群,则$G_{n-1}$必是商群即可。这其实是从商群列逆向推导出$G$是循环群。

设$r$是$G_n$的生成元代表元素,即$r<x_n>$生成了这个循环群。设$|G_{n}|=d$,则对于任意$r_1,r_2\in r<x_n>$都有$r_1^d,r_2^d\in<x_n>$。$r_1\rightarrow r_1^d$是从左陪集$r<x_n>$到正规子群$<x_n>$的映射,下面证明这是一个双射。

设$r_1\neq r_2$,且$r_1^d=r_2^d$。在$G_{n-1}$中,$<r_1>$和$<r_2>$是两个阶数相同,但元素不同的子群,与题设矛盾。所以$r_1\rightarrow r_1^d$是单射。又因为正规子群和陪集的基数相同,所以该映射既单又满。

因此在$r<x_n>$中存在唯一的$r'$使得$r'^d=x_n$,把陪集表示为$r'^i<r'^d>,i=1,2...d$。可见$G_{n-1}$确实是循环群,证毕。