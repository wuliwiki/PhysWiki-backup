% 克里斯蒂安·惠更斯(综述)
% license CCBYSA3
% type Wiki

本文根据 CC-BY-SA 协议转载翻译自维基百科\href{https://en.wikipedia.org/wiki/Christiaan_Huygens}{相关文章}。

\begin{figure}[ht]
\centering
\includegraphics[width=6cm]{./figures/759a661ac1a7d67e.png}
\caption{惠更斯肖像,由卡斯帕·内彻绘于1671年,现藏于莱顿博尔哈夫博物馆[1]} \label{fig_HGS_1}
\end{figure}

克里斯蒂安·惠更斯,泽尔亨领主,英国皇家学会院士(/ˈhaɪɡənz/,音译‘海根斯’,[2] 美国亦发音为 /ˈhɔɪɡənz/,音译‘霍伊根斯’;[3] 荷兰语:[ˈkrɪstijaːn ˈɦœyɣə(n)s] ⓘ;也拼作 Huyghens;拉丁语:Hugenius;1629年4月14日-1695年7月8日),是一位荷兰数学家、物理学家、工程师、天文学家和发明家,被视为科学革命中的关键人物之一。[4][5] 在物理学领域,惠更斯在光学和力学方面做出了开创性的贡献;作为天文学家,他研究了土星的光环并发现了土星最大的卫星——泰坦。作为工程师和发明家,他改进了望远镜的设计,并发明了摆钟,这种时钟在近300年内是最精确的计时工具。他是一位才华横溢的数学家和物理学家,其著作首次通过一组数学参数对物理问题进行了理想化描述,[6] 并首次对一种无法直接观测的物理现象进行了数学和机械论的解释。[7]

惠更斯在其著作《De Motu Corporum ex Percussione》中首次正确地确定了弹性碰撞的定律,该书完成于1656年,但于1703年才在他去世后出版。[8] 1659年,惠更斯在其著作《De vi Centrifuga》中以几何方法推导出了经典力学中描述离心力的公式,这比牛顿早了十年。[9] 在光学领域,他因提出光的波动理论而闻名,这一理论发表于其1690年的《光论》(Traité de la Lumière)。惠更斯的光波理论最初被牛顿的光微粒理论所取代,直到1821年,奥古斯丁-让·菲涅耳(Augustin-Jean Fresnel)改进了惠更斯的原理,完整解释了光的直线传播和衍射现象。今天,这一原理被称为“惠更斯-菲涅耳原理”。

1657年,惠更斯发明了摆钟,并于同年获得专利。他对钟表的研究最终在《摆动时钟》(Horologium Oscillatorium,1673年)中发表,该书被认为是17世纪关于力学的重要著作之一。[6] 虽然书中包含了钟表设计的描述,但大部分内容是对摆动运动的分析和曲线理论。1655年,惠更斯与其兄弟康斯坦丁(Constantijn)开始研磨透镜,制作折射望远镜。他发现了土星最大的卫星——泰坦,并首次解释了土星的奇特外观是由于“一个薄而平坦的环,其不与土星接触,并倾斜于黄道面”。[10] 1662年,惠更斯开发了如今称为“惠更斯目镜”的装置,这是一种采用两个透镜的望远镜,能够减少色散现象。[11]

作为数学家,惠更斯发展了渐伸线的理论,并在《赌博中的计算》(Van Rekeningh in Spelen van Gluck)中研究了几何概率和点数问题。该书由弗朗斯·范·斯库腾(Frans van Schooten)翻译并以《赌博中的推理》(De Ratiociniis in Ludo Aleae,1657年)出版。[12] 惠更斯及其他人对期望值的使用后来启发了雅各布·伯努利(Jacob Bernoulli)对概率论的研究。[13][14]