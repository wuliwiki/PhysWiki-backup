% 延迟选择量子擦除实验
% 量子|干涉|擦除

\pentry{狄拉克符号\upref{braket},量子测量\upref{qmeas}}

如图\ref{fig:Experimental device}所示,左上角的光源发射光子,光子随后经过双缝。把红色路径标识的缝记为左缝,另一条则称为右缝。光子经过双缝后会进入$BBO$晶体。$BBO$将会吸收每一个入射的光子并发射出两个纠缠的光子。这两个光子经过棱镜后会分别射向不同的方向。我们把射向屏幕$D_0$的光子称为$A$光子,另一个称为$B$光子。$D_0$、$D_1$、$D_2$、$D_3$、$D_4$都是感光屏。$BS_a$、$BS_b$、$BS_c$是半透半反射镜,入射的光子有一半概率反射、一半概率透射。$M_a$、$M_b$是全反射镜。在经过$BS_c$后,从两条缝中射出的光子所产生的$B$光子的路径完全重合。从双缝到$D_1$、$D_2$、$D_3$、$D_4$光程远大于到$D_0$的光程,因此可以在观察到$D_0$上的现象后再对虚线框中的$B$光子观测系统进行调整。

在进行实验时,按如下的步骤操作:

(i)令光源每次只发射一个光子。

(ii)观察$D_0$屏幕上的光子落点,并记下另外四个屏幕中感光的屏幕编号。若$D_i(i=1,2,3,4)$屏幕感光,则将对应的光子落点记为$i$类点。重复进行多次。

(iii)只开放左狭缝,观察$D_0$屏幕上的光子落点,这类落点记为$L$类落点。重复进行多次。然后,在只开放右狭缝的情况下进行同样的工作,对应的落点记为$R$类落点。

实验完成后,将光子的落点画在图上,构成光强分布图。将$i(i=1,2,3,4,L,R)$类点的分布称为$i$类分布,所有光子落点的分布称为总分布。实验现象如下:

(i)总分布图中不存在干涉条纹。

(ii)$1$类分布与$2$类分布中存在干涉条纹。

(iii)$3$类分布与$4$类分布中不存在干涉条纹。

(iv)$3$类分布等于$R$类分布,$4$类分布等于$L$类分布。

(v)总分布等于$|\alpha|^2$倍的$L$类分布与$|\beta|^2$倍的$R$类分布之和。

\subsection{假设与约定}
为方便下面的讨论,这里做一些必要的简化,并对符号作出约定。

光子经过双缝后,其态矢量是确定的。显然,这个态矢可以写为如下的形式:
\begin{equation}
\ket{\Psi(t=0)} =\alpha\ket{\chi_L(t=0)}+\beta\ket{\chi_R(t=0)}~.
\end{equation}
其中$\ket{\chi_L(t=0)}$表示这样的一个态:$t=0$时刻,光子刚通过狭缝,此时对光子的位置进行测量,可以完全肯定它来自于左边的狭缝。若$t=0$时刻,从两边狭缝透射出的波函数没有任何重叠(即$\chi_L(t=0)\chi_R(t=0)$处处为零),这样的效果就可以达成。$\ket{\chi_R(t=0)}$的定义同理。所以,在光子刚经过狭缝时,对光子的位置进行测量,就可以完全肯定光子经过了哪条狭缝,且经过左边狭缝的概率为$|\alpha|^2$,经过右边狭缝的概率为$|\beta|^2$。这两个概率的和为$1$(这就保证了态矢量$\ket{\Psi}$的归一化),而它们的具体值取决于光源的位置和两条狭缝的宽度等参数,这里不讨论。通过以上的定义,还可以看出$\ket{\psi_L(t=0)}$正交于$\ket{\psi_R(t=0)}$,$\ket{\phi_L(t=0)}$正交于$\ket{\phi_R(t=0)}$,这是因为从两边狭缝透射出的波函数没有任何重叠。

接下来研究$BBO$晶体对态矢量的作用。简单起见,假设光子经过双缝后立刻受到$BBO$的作用。当一个光子射入$BBO$晶体时,将产生一对除自旋外各种属性和状态都相同的光子,且这对光子的属性和状态与入射的光子也相同。设$\mathscr{E}_A$为$A$光子的轨道态空间(即不考虑自旋的态空间),$\mathscr{E}_B$为$B$光子的轨道态空间,$\mathscr{E}_I$为入射光子的轨道态空间。经过$BBO$晶体后整个体系的轨道态空间即为$\mathscr{E}_A$和$\mathscr{E}_B$的张量积。为了保证属性相同,$\mathscr{E}_A$与、$\mathscr{E}_B$、$\mathscr{E}_I$互相自然同构。若入射光子的态矢量为$\ket{\chi}$,出射的$A$光子的态矢量为$\ket{\psi}$、出射的$B$光子的态矢量为$\ket{\phi}$,则$\ket{\chi}$、$\ket{\psi}$、$\ket{\phi}$应当在同构映射的意义下彼此相等(或相差一个无意义的相位因子)。从而,经过$BBO$晶体后,整个体系的态矢量应当为
\begin{equation}
\ket{\Psi(t=0)} =\alpha\ket{\psi_L(t=0)}\ket{\phi_L(t=0)}+\beta\ket{\psi_R(t=0)}\ket{\phi_R(t=0)}~.
\end{equation}
其中$\ket{\psi_L(t=0)}$是与$\ket{\chi_L(t=0)}$在同构映射的意义下相等的矢量,$\ket{\psi_R(t=0)}$、$\ket{\phi_L(t=0)}$、$\ket{\phi_R(t=0)}$的定义同理。
现在考虑体系的演化。在射出$BBO$晶体后,$A$光子和$B$光子将各自独立地演化,因此体系的哈密顿算符是这两者各自的哈密顿算符(的延伸算符)之和,即$H=H_A+H_B$。假定$\ket{\psi_L(t)}$是满足下列薛定谔方程的解:
\begin{equation}
\I\hbar\frac{\dd}{\dd t}\ket{\psi_L} =H_A\ket{\psi_L}~.
\end{equation}

该解的初态即为$|\psi_L(t=0)\rangle$。同理,对$|\psi_R(t)\rangle$、$|\phi_L(t)\rangle$、$|\phi_R(t)\rangle$也作同样的假定。于是,任意时刻整个体系的态可以写作
\begin{equation}
|\Psi(t)\rangle =\alpha|\psi_L(t)\rangle|\phi_L(t)\rangle+\beta|\psi_R(t)\rangle|\phi_R(t)\rangle.
\end{equation}
这是因为这样定义的$|\Psi(t)\rangle$满足整个体系的薛定谔方程:
\begin{align}
&i\hbar\frac{d}{dt}|\Psi(t)\rangle \notag \\
={}&\alpha((i\hbar\frac{d}{dt}|\psi_L(t)\rangle)|\phi_L(t)\rangle+|\psi_L(t)\rangle(i\hbar\frac{d}{dt}|\phi_L(t)\rangle)) \notag \\
&+\beta((i\hbar\frac{d}{dt}|\psi_R(t)\rangle)|\phi_R(t)\rangle+|\psi_R(t)\rangle(i\hbar\frac{d}{dt}|\phi_R(t)\rangle)) \notag \\
={}&\alpha(H_A|\psi_L(t)\rangle|\phi_L(t)\rangle+|\psi_L(t)\rangle H_B|\phi_L(t)\rangle) \notag \\
&+\beta(H_A|\psi_R(t)\rangle|\phi_R(t)\rangle+|\psi_R(t)\rangle H_B|\phi_R(t)\rangle) \notag \\
={}&H|\Psi(t)\rangle.
\end{align}
这样体系的演化就得到了求解。为简化讨论,再对体系的演化提出以下假设:
\begin{myquote}
$(i)$所有的态都是已归一化的。

$(ii)$$B$光子的波函数在$BS_c$之前的路径上是不重叠的,即$\phi_L(\boldsymbol{r_B},t)\phi_R(\boldsymbol{r_B},t)$在路径上恒为零。

$(iii)$$A$光子与$B$光子的波函数分布范围足够小,使得在屏幕后面的探测不到光子,也即不会出现光子从屏幕外面经过的情况。
\end{myquote}

屏幕对光子的作用相当于进行了一次位置的测量,这会导致态矢量发生跃变。假设在$t_A$时刻屏幕$D_0$感光,在$t_B$时刻$D_1$、$D_2$、$D_3$、$D_4$中的任意一个感光。若在$\boldsymbol{r_A}$处观测到了光子,则假设在$t=t_A^+$的时刻,$A$光子的态矢量跃变为$|\psi'(t_A)\rangle$,并且此后其随时间变化的规律为$|\psi'(t)\rangle(t>t_A)$。相应地,假设其对应的波函数为$\psi'(\boldsymbol{r_A},t)$。
