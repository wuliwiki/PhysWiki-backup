% 绝对极值与相对极值(变分学)
% 极值分类|曲线距离|邻区|强极大|弱极大

\pentry{可取曲线(变分学)\upref{DesCur}}

本节将对曲线 $\gamma$ 的函数 $J[\gamma]$ 的极值进行分类,本节内容将有助于引出下一节“变分”的概念.

\subsection{绝对极值}
定义在某可取曲线族中的曲线函数 $J[\gamma]$ 的\textbf{绝对极小值}由此族中曲线 $\gamma_0$ 实现,如果对于这族中任一曲线 $\gamma$ 有
\begin{equation}
J[\gamma]\geq J[\gamma_0]
\end{equation}
同样,可定义绝对极大值.
\subsection{相对极值}
\begin{figure}[ht]
\centering
\includegraphics[width=6cm]{./figures/AbPol_1.pdf}
\caption{相对极值示意图} \label{AbPol_fig1}
\end{figure}

先用一个直观的例子来理解相对极值的概念.如图,从 $A$ 点到 $B$ 点,有几条路可走,水平线上方和下方各有两条路,上方的最短路线 $S_0$ 小于下方的最短路线 $D_0$ .那么,上方路线 $S_0$ 就是连接 $A,B$ 两点的绝对极小值.下方最短路线 $D_0$ 虽然不是绝对极小值,然而它却比连接这两点下方的其它路线都短. 

为严格定义相对极值,需介绍几个必备概念.
\subsubsection{曲线间的距离}
\begin{definition}{曲线间距}
若在区间 $(a,b)$ 中定义有两曲线:
\begin{equation}
\begin{aligned}
y&=y(x)\\
y&=y_1(x)\quad(a\leq x\leq b)
\end{aligned}
\end{equation}
这两\textbf{曲线间的距离}是一非负数 $r$,它等于 $\abs{y_1(x)-y(x)}$ 在区间 $(a,b)$ 上的最大值,记为 $r=r[y_1(x),y_2(x)]$.
\end{definition}
由曲线间距的定义知,两条曲线间的距离恒为0的充要条件是这两曲线重合.

已知曲线序列:
\begin{equation}
y=y_1(x),y=y_2(x),\cdots,y=y_n(x),\cdots
\end{equation}
这些曲线到曲线 $y=y(x)$ 间的距离趋于0.就说这一函数系列\textbf{一致收敛}于 $y(x)$.

在曲线函数
\begin{equation}\label{AbPol_eq1}
J(y)=\int F(x,y,y')\dd x
\end{equation}
中,被积函数不仅依赖于函数值,而且依赖于它的微商.因此,对于两个距离很小的曲线,泛函\autoref{AbPol_eq1} 之值是可以很不同的.
\begin{example}{}
泛函 $\int_0\pi y'^2\dd x$ 对于曲线
\begin{equation}
\begin{aligned}
y&=\frac{1}{n}\sin nx\quad (0\leq x\leq \pi)\\
y&=0
\end{aligned}
\end{equation}
的值分别为 $\pi/2$ 和0,但当 $n\rightarrow 0$ 时,也就是说它们的距离趋于0时,其差也不变.
\end{example}


