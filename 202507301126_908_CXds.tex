% 抽象代数(综述)
% license CCBYNCSA3
% type Wiki

本文根据 CC-BY-SA 协议转载翻译自维基百科\href{https://en.wikipedia.org/wiki/Abstract_algebra}{相关文章}。

\begin{figure}[ht]
\centering
\includegraphics[width=6cm]{./figures/e5b4bfc6385c6286.png}
\caption{魔方的所有排列构成一个群,这是抽象代数中的一个基本概念。} \label{fig_CXds_1}
\end{figure}
在数学中,更具体地说,在代数学中,抽象代数或现代代数是研究代数结构的学科。代数结构是指带有特定运算作用于其元素的集合。\(^\text{[1]}\)代数结构包括群、环、域、模、向量空间、格以及域上的代数。抽象代数这一术语是在20世纪早期提出的,用来将其与代数学的旧分支区分开来,更具体地,是为了区别于初等代数,即使用变量来表示数进行计算和推理的部分。如今,抽象的代数观点已成为高等数学中如此根本的内容,以至于通常直接称为“代数”,而“抽象代数”这一术语除了在教学中很少再被使用。

代数结构及其相关的同态构成数学上的范畴。范畴论提供了一个统一的框架,用来研究各种结构中类似的性质和构造。

一般代数是一个相关学科,研究将不同类型的代数结构作为单一对象来对待。例如,在一般代数中,群的结构是一个单一的对象,被称为群的多类。
\subsection{历史}
在19世纪之前,代数被定义为对多项式的研究。\(^\text{[2]}\)随着更复杂的问题和解法的发展,抽象代数在19世纪逐渐产生。具体的问题和例子来自数论、几何、分析以及代数方程的解。如今被认为是抽象代数组成部分的大多数理论,最初只是来自数学各个分支的一些零散事实的集合,随后逐渐形成了一个共同的主题,作为核心将各种结果汇集起来,并最终在一套共同概念的基础上实现统一。这种统一发生在20世纪的前几十年,促成了对群、环、域等各种代数结构的形式公理化定义。\(^\text{[3]}\)这一历史发展过程几乎与流行教材中的处理方式相反,例如范德瓦尔登的 《现代代数》\(^\text{[4]}\),这些教材通常在每一章开头给出某种结构的形式定义,然后再给出具体的实例。\(^\text{[5]}\)
\subsubsection{初等代数}
对多项式方程或代数方程的研究有着悠久的历史。大约公元前 1700 年,巴比伦人已经能够解出以文字题形式给出的二次方程。这个“文字题”阶段被称为修辞代数,并且一直到 16 世纪都是主流方法。公元 830 年,花拉子米首次提出“algebra(代数)”一词,但他的工作完全属于修辞代数。完全符号化的代数直到弗朗索瓦·维埃特1591 年的《新代数》才出现,即便如此,其中仍有一些拼写出的词语,直到笛卡尔1637 年的《几何学》才被赋予统一的符号表示。\(^\text{[6]}\)对符号方程求解的正式研究促使莱昂哈德·欧拉在 18 世纪末接受当时被认为是“荒谬”的根,例如负数和虚数。\(^\text{[7]}\)然而,大多数欧洲数学家直到 19 世纪中叶仍然抗拒这些概念。\(^\text{[8]}\)

乔治·皮考克1830 年的《代数论》是第一次尝试将代数完全建立在严格的符号基础之上。他区分了新的符号代数与旧的算术代数。在算术代数中,$a - b$被限制为$a \geq b$,而在符号代数中,所有的运算规则在没有任何限制的情况下成立。利用这一点,皮考克能够证明类似$(-a)(-b) = ab$这样的法则:只需令$a = 0,\ c = 0$代入$(a - b)(c - d) = ac + bd - ad - bc$即可成立。皮考克使用他称为等价形式永久性原理来为他的论证辩护,但他的推理存在归纳法问题。\(^\text{[9]}\)例如,$\sqrt{a}\sqrt{b} = \sqrt{ab}$对于非负实数成立,但对一般复数却不成立。
\subsubsection{早期群论}
数学的几个领域共同促成了对群的研究。拉格朗日1770 年对五次方程解的研究,促成了多项式的伽罗瓦群的诞生。高斯1801 年对费马小定理的研究引出了模 $n$ 的整数环、模 $n$ 的整数乘法群,以及更一般的循环群和阿贝尔群的概念。克莱因1872 年的埃尔朗根纲领研究几何学,并引出了对称群,例如欧几里得群和射影变换群。1874 年,李引入了李群理论,旨在发展“微分方程的伽罗瓦理论”。1876 年,庞加莱和克莱因引入了莫比乌斯变换群,以及它的子群,例如模群和Fuchs 群,其基础是分析中自守函数的研究。\(^\text{[10]}\)

群的抽象概念在 19 世纪中叶缓慢发展起来。1832 年,伽罗瓦首次使用“群”一词\(^\text{[11]}\),表示在复合运算下封闭的一个置换集合。\(^\text{[12]}\)阿瑟·凯莱1854 年的论文《论群论》将群定义为一个带有结合性复合运算和单位元 $1$ 的集合,这在今天被称为幺半群。\(^\text{[13]}\)1870 年,克罗内克定义了一种抽象的二元运算,该运算是封闭的、交换的、结合的,并且具有左消去性质$b \neq c \to a \cdot b \neq a \cdot c$\(^\text{[14]}\)类似于有限阿贝尔群的现代运算规律。\(^\text{[15]}\)韦伯1882 年对群的定义是一个封闭的二元运算,它满足结合律并具有左右消去性。\(^\text{[16]}\)1882 年,沃尔特·冯·迪克首次要求逆元必须作为群定义的一部分。\(^\text{[17]}\)

一旦抽象群的概念形成,各种结果就被重新用这种抽象框架加以表述。例如,西罗定理在 1887 年被弗罗贝尼乌斯直接根据有限群的运算规律重新证明,尽管弗罗贝尼乌斯指出该定理可由关于置换群的柯西定理以及“每个有限群都是某个置换群的子群”这一事实推出。\(^\text{[18][19]}\)奥托·赫尔德在这一领域尤为高产:他在 1889 年定义了商群,1893 年定义了群自同构以及单群,并且完成了约旦–赫尔德定理。德德金和米勒独立地刻画了哈密顿群并引入了两个元素的换子的概念。伯恩赛德、弗罗贝尼乌斯和莫利安(Molien)在 19 世纪末建立了有限群的表示论。\(^\text{[18]}\)J. A. de Séguier 1905 年的专著《抽象群论要素》以抽象、一般的形式呈现了其中许多成果,把“具体的”群 relegated 到附录中,尽管它仅限于有限群。第一本关于有限与无限抽象群的专著是 O. K. Schmidt 1916 年的《抽象群论》。\(^\text{[20]}\)
\subsubsection{早期环论}
非交换环论起始于将复数扩展为超复数,尤其是威廉·罗恩·哈密顿在 1843 年提出的四元数。不久之后,许多其他数系陆续出现。1844 年,哈密顿提出双四元数,凯莱引入了八元数,格拉斯曼引入了外代数。[21] 1848 年,詹姆斯·科克尔提出四维复数,[22] 1849 年提出共四元数。[23] 威廉·金登·克利福德在 1873 年引入了分裂双四元数。此外,凯莱在 1854 年引入了实数和复数上的**群代数,并在 1855 年和 1858 年的两篇论文中引入了方阵。[24]

当足够多的例子被发现后,下一步就是对它们进行分类。在 1870 年的一部专著中,本杰明·皮尔斯对维数低于 6 的 150 多种超复数系进行了分类,并给出了结合代数的明确定义。他定义了幂零元和幂等元,并证明了任何代数必定包含其中之一。他还定义了皮尔斯分解。弗罗贝尼乌斯在 1878 年和查尔斯·桑德斯·皮尔斯(Charles Sanders Peirce)在 1881 年独立证明,唯一的有限维除环是$\mathbb{R}$上的实数、复数和四元数。在 1880 年代,基林和埃利·卡当证明了半单李代数可以分解为若干个单李代数,并对所有单李代数进行了分类。受此启发,在 1890 年代,卡当、弗罗贝尼乌斯和莫利安(Molien)独立证明:有限维结合代数在$\mathbb{R}$或$\mathbb{C}$上可以唯一分解为一个幂零代数和一个半单代数的直和,后者是若干个单代数的乘积,而这些单代数是除代数(division algebras)上的方阵代数。卡当是第一个定义**直和(direct sum)**和**单代数(simple algebra)\*\*概念的人,而这些概念后来证明影响极大。在 1907 年,韦德伯恩(Joseph Wedderburn)将卡当的结果推广到任意域,这些结果如今被称为**韦德伯恩主定理(Wedderburn principal theorem)**和**阿廷–韦德伯恩定理(Artin–Wedderburn theorem)**。[25]

对于*交换环,多个领域的研究共同促成了交换环理论的发展。[26] 高斯(Gauss)在 1828 年和 1832 年的两篇论文中提出了高斯整数(Gaussian integers),并证明它们构成一个唯一分解整环,同时证明了双二次互反律。大约在同一时期,雅可比和艾森斯坦为艾森斯坦整数证明了三次互反律。[25]

对费马大定理的研究引出了代数整数。1847 年,加布里埃尔·拉梅曾认为自己证明了 FLT,但他的证明是错误的,因为他假设所有圆分域都是 UFD。然而,正如库默尔所指出的,$\mathbb{Q}(\zeta_{23})$并不是一个 UFD。[27] 在 1846 年和 1847 年,库默尔引入了理想数,并证明了圆分域中对理想素数的唯一分解性。[28] 德德金在 1871 年推广了这一结果,证明了代数数域的整数环中任一非零理想可以唯一分解为若干素理想的乘积,这是德德金整环理论的前身。总体而言,德德金的工作奠定了代数数论这一学科。[29]

在 1850 年代,黎曼引入了黎曼曲面的基本概念。黎曼的方法依赖于他称之为狄利克雷原理的假设,[30] 但在 1870 年被魏尔斯特拉斯提出质疑。很久以后,在 1900 年,希尔伯特通过发展变分法中的直接方法,为黎曼的方法提供了严谨的依据。[31]在 1860 年代和 1870 年代,克莱布施、戈尔丹、布里尔,尤其是 M. 诺特(M. Noether)研究了代数函数和代数曲线。尤其是诺特研究了一个多项式成为多项式环$\mathbb{R}[x,y]$中由两条代数曲线生成的理想的元素所需的条件,尽管诺特当时并未使用这种现代语言。1882 年,德德金和韦伯(Weber)类比德德金早期在代数数论中的工作,创建了代数函数域理论,从而首次给出了黎曼曲面的严格定义,并严格证明了黎曼–罗赫定理。克罗内克在 1880 年代、希尔伯特在 1890 年、拉斯克在 1905 年以及麦考莱在 1913 年进一步研究了隐含在 E. 诺特工作中的多项式环的理想结构。拉斯克证明了拉斯克–诺特定理的一个特殊情形,即多项式环中的每个理想都是有限个初等理想的交集;麦考莱则证明了这种分解的唯一性。[32]总体而言,这些工作推动了代数几何的发展。[26]

