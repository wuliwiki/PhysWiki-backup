% 线性算子的张量积
% keys 线性算子|张量积
% license Xiao
% type Tutor

\pentry{向量空间的张量积\nref{nod_TPofSp}, 线性算子\nref{nod_LiOper}}{nod_530c}

\subsection{算子的张量积}
由\enref{空间的张量积}{TPofSp}一节知道,任意两个矢量空间的张量积都存在,且仍是个矢量空间,那么在其上就可研究线性算子。
\begin{theorem}{}
设 $V,W$ 是域 $\mathbb F$ 上的矢量空间,$\mathcal A,\mathcal B$ 分别是 $V,W$ 上的线性算子,那么在 $V\otimes W$ 上存在唯一的线性算子 $\mathcal C$ ,使得
\begin{equation}\label{eq_TPofLO_1}
\mathcal C(v\otimes w)=\mathcal A v\otimes \mathcal B w,\quad v(\in V,w\in W)~.
\end{equation}
\end{theorem}
\textbf{证明:}定义映射 $\varphi:V\times W\rightarrow V\otimes W$:
\begin{equation}
\varphi(v,w):=\mathcal A v\otimes \mathcal B w~\quad (v\in V,w\in W)~.
\end{equation}
显然, $\varphi$ 是个线性算子,只需利用 $\mathcal A,\mathcal B$ 是\enref{线性算子}{LiOper}和张量积的性质(\autoref{the_TsrPrd_1}~\upref{TsrPrd})即可。

由张量积的定义\autoref{def_vecTsr_1}~\upref{vecTsr},存在唯一的线性算子 $\mathcal C$,使得
\begin{equation}
\mathcal C(v\otimes w)=\varphi(v,w)=\mathcal A v\otimes \mathcal B w~.
\end{equation}

\textbf{证毕!}
该定理给出了定义线性算子张量积的依据。

\begin{definition}{算子的张量积}
设 $\mathcal A,\mathcal B$ 分别是矢量空间 $V,W$ 上的线性算子,则
\begin{equation}
\mathcal A\otimes \mathcal B:V\otimes W\rightarrow V\otimes W~
\end{equation}
称作算子 $\mathcal A,\mathcal B$ 的\textbf{张量积},其规则为
\begin{equation}
(\mathcal A\otimes \mathcal B)(v\otimes w)=\mathcal Av\otimes \mathcal Bw~.
\end{equation}
\end{definition}
由算子张量积的定义,容易验证线性算子的张量积具有下面的性质:
\begin{enumerate}
\item $(\mathcal A\otimes \mathcal B)(\mathcal C\otimes \mathcal D)=\mathcal AC\otimes \mathcal BD$
\item $(\mathcal A+\mathcal C)\otimes B=\mathcal A\otimes B+\mathcal C\otimes B$
\item $\mathcal A\otimes(\mathcal B+\mathcal D)=\mathcal A\otimes \mathcal B+\mathcal A\otimes \mathcal D$
\item $\mathcal A\otimes(\lambda\mathcal B)=(\lambda\mathcal A)\otimes \mathcal B=\lambda(\mathcal A\otimes \mathcal B)$
\end{enumerate}
\subsection{算子张量积的矩阵}
在\enref{线性算子代数}{LiOper} 文章开头说过,$\mathcal L(V,V)$ 上的线性算子与 $n$ 阶方阵一一对应( $n=\dim V$)。而 $V\otimes W$ 本身也是矢量空间,这样就可把算子的张量积 $\mathcal A\otimes \mathcal B$ 看成 $\mathcal L(V\otimes W,V\otimes W)$ 上的线性算子,那么 $\mathcal A\otimes \mathcal B$ 就对应 $nm$ 阶的方阵($m=\dim W$)。

设 $\{e_i\},\{f_j\}$ 分别是 $V,W$ 上的一组基。记
\begin{equation}
\mathcal Ae_i=\sum_{i'}\alpha_{i'i}e_{i'}~,\quad \mathcal Bf_j=\sum_{j'}\beta_{j'j}f_{j'}~,
\end{equation}
就得到
\begin{equation}
(\mathcal A\otimes \mathcal B)(e_i\otimes f_j)=\sum_{i',j'}\alpha_{i'i}\beta_{j'j}e_{i'}\otimes f_{j'}~.
\end{equation}
这就是说,由 $A=(a_{i'i}),B=(\beta_{j'j})$ ,可以得到
\begin{equation}\label{eq_TPofLO_2}
A\otimes B=(\alpha_{i'i}\beta_{j'j})=
\begin{pmatrix}
&\alpha_{11}B&\alpha_{12}B&\cdots&\alpha_{1n}B\\
&\alpha_{21}B&\alpha_{22}B&\cdots&\alpha_{2n}B\\
&\cdots&\cdots&\cdots&\cdots\\
&\alpha_{n1}B&\alpha_{n2}B&\cdots&\alpha_{nn}B\\
\end{pmatrix}~.
\end{equation}
注意,上面 $A\otimes B$ 的矩阵排列方式,和通常线性算子的矩阵排列一样,这里矢量空间 $V\otimes W$ 的第 $(i-1)+j$ 个基就是 $e_i\otimes f_j$。\autoref{eq_TPofLO_2} 的理解如\autoref{tab_TPofLO_1} (注意算子矩阵的排列表)

\begin{table}[ht]
\centering
\caption{ $A\otimes B$ 的矩阵排列方式理解表}\label{tab_TPofLO_1}
\begin{tabular}{|c|c|c|c|c|c|c|c|}
\hline
 & $e_1\otimes f_1$ & $\cdots$ & $e_1\otimes f_m$ & $\cdots$ & $e_n\otimes f_1$ & $\cdots$ & $e_n\otimes f_m$ \\
\hline
$e_1\otimes f_1$ & $\alpha_{11}\beta_{11}$ & $\cdots$ & $\alpha_{11}\beta_{1m}$ & $\cdots$ & $\alpha_{1n}\beta_{11}$ & $\cdots$ & $\alpha_{1n}\beta_{1m}$ \\
\hline
$\cdots$ & $\cdots$ & $\cdots$ & $\cdots$ & $\cdots$ & $\cdots$ & $\cdots$ & $\cdots$ \\
\hline
$e_1\otimes f_m$ & $\alpha_{11}\beta_{m1}$ & $\cdots$ & $\alpha_{11}\beta_{mm}$ & $\cdots$ & $\alpha_{1n}\beta_{m1}$ & $\cdots$ & $\alpha_{1n}\beta_{mm}$ \\
\hline
$\cdots$ & $\cdots$ & $\cdots$ & $\cdots$ & $\cdots$ & $\cdots$ & $\cdots$ & $\cdots$ \\
\hline
$e_n\otimes f_1$ &  $\alpha_{n1}\beta_{11}$ & $\cdots$ & $\alpha_{n1}\beta_{1m}$ & $\cdots$ & $\alpha_{nn}\beta_{11}$ & $\cdots$ & $\alpha_{nn}\beta_{1m}$ \\
\hline
$\cdots$ & $\cdots$ & $\cdots$ & $\cdots$ & $\cdots$ & $\cdots$ & $\cdots$ & $\cdots$ \\
\hline
$e_n\otimes f_m$ & $\alpha_{n1}\beta_{11}$ & $\cdots$ & $\alpha_{n1}\beta_{mm}$ & $\cdots$ & $\alpha_{nn}\beta_{m1}$ & $\cdots$ & $\alpha_{mn}\beta_{mm}$\\
\hline
\end{tabular}
\end{table}

对算子的基,有下面公式
\begin{equation}
\mathrm{tr}\,A\otimes B=\sum_{i}\alpha_{ii}\mathrm{tr}\,B=\mathrm{tr}\,A\cdot\mathrm{tr}\,B~,
\end{equation}
且
\begin{equation}
\begin{aligned}
\det A\otimes B&=\det((A\otimes E_m)(E_n\otimes B))\\
&=\det(A\otimes E_m)\cdot\det(e_n\otimes B)\\
&=(\det A)^m(\det B)^n~.
\end{aligned}
\end{equation}
其中第一式用到了算子张量积的性质1和算子运算与矩阵运算的对应关系(\autoref{the_LiOper_1}~\upref{LiOper}).
