% Baker-Hausdorff 公式
% keys Baker|Hausdorff|定理
\pentry{组合\upref{combin}}
\begin{issues}
\issueTODO
\end{issues}


Baker-Hausdorff公式是一个相当有用的公式.在数学上,它可用于给出李群-李代数对应的深层结果的相对简单的证明;在量子力学中,它可实现系统哈密顿量在薛定谔绘景和海森堡绘景的转换,并在微扰论中也有诸多应用.本节将给出该公式的一个证明和由它导出的一些重要的结果.

\textbf{Baker-Hausdorff公式}是指
\begin{equation}\label{BAHA_eq1}
\E ^{A}B\E^{-A}=\sum_{n=0}^{\infty}\frac{1}{n!}A^{(n)}
\end{equation}
其中,
\begin{equation}\label{BAHA_eq5}
\begin{aligned}
A^{(n)}&\equiv\underbrace{[A,[A,\cdots,[A}_{n\text{个}},B]]\cdots]\\
&=\sum_{m=0}^{n}(-1)^{n-m}C_{n}^{m}A^mBA^{n-m}
\end{aligned}
\end{equation}

\subsection{证明}
\subsubsection{纯数学证明}
在证明Baker-Hausdorff公式\autoref{BAHA_eq1} 之前,我们先证明\autoref{BAHA_eq5} 和一个引理.\\

\textbf{\autoref{BAHA_eq5} 的证明:}我们用数学归纳法来证明.
$A^{(0)},A^{(1)}$显然成立:
\begin{equation}
\begin{aligned}
A^{(0)}&=B=\sum_{m=0}^{0}(-1)^{0-m}C_{0}^{m}A^mBA^{0-m}\\
A^{(1)}=&[A,B]=AB-BA=\sum_{m=0}^{1}(-1)^{1-m}C_{1}^{m}A^mBA^{1-m}
\end{aligned}
\end{equation}

假设对 $n=k-1$ 时\autoref{BAHA_eq5} 成立,则
\begin{equation}
\begin{aligned}
A^{(k)}&=[A,A^{(k-1)}]=AA_{k-1}-A_{k-1}A\\
&=A\sum_{m=0}^{k-1}(-1)^{k-1-m}C_{k-1}^{m}A^mBA^{k-1-m}-\qty(\sum_{m=0}^{k-1}(-1)^{k-1-m}C_{k-1}^{m}A^mBA^{k-1-m})A\\
&=\sum_{m=0}^{k-1}(-1)^{k-1-m}C_{k-1}^{m}A^{m+1}BA^{k-1-m}-\sum_{m=0}^{k-1}(-1)^{k-1-m}C_{k-1}^{m}A^mBA^{k-m}\\
&=A^{k}B+\sum_{m=0}^{k-2}(-1)^{k-1-m}\qty(C_{k-1}^{m}+C_{k-1}^{m+1})A^{m+1}BA^{k-1-m}-(-1)^{k-1}BA^k\\
&=A^{k}B+\sum_{m=0}^{k-2}(-1)^{k-1-m}C_{k}^{m+1}A^{m+1}BA^{k-1-m}-(-1)^{k-1}BA^k\\
&=A^{k}B+\sum_{m=1}^{k-1}(-1)^{k-m}C_{k}^{m}A^{m}BA^{k-m}+(-1)^{k}BA^k\\
&=\sum_{m=0}^{k}(-1)^{k-m}C_{k}^{m}A^{m}BA^{k-m}
\end{aligned}
\end{equation}

由数学归纳法原理,\autoref{BAHA_eq5} 得证.

上面证明中 $C_n^m$ 为组合数\upref{combin}.
\begin{lemma}{}\label{BAHA_lem2}
\begin{equation}\label{BAHA_eq2}
A^nB=\sum_{m=0}^{n}C_{n}^mA^{(m)}A^{n-m}
\end{equation}
\end{lemma}
\textbf{证明:}这里同样用数学归纳法来证明.当 $n=1$ 时,\autoref{BAHA_eq2} 显然成立:
\begin{equation}
AB=BA+[A,B]
\end{equation}
假设 $n=k-1$ 时\autoref{BAHA_eq2} 成立,则
\begin{equation}\label{BAHA_eq3}
\begin{aligned}
A^{k}B=A\sum_{m=0}^{k-1}C_{k-1}^mA^{(m)}A^{k-1-m}=\sum_{m=0}^{k-1}C_{k-1}^mAA^{(m)}A^{k-1-m}
\end{aligned}
\end{equation}

因为
\begin{equation}
A^{(m+1)}=[A,A^{(m)}]\Rightarrow AA^{(m)}=A^{(m)}A+A^{(m+1)}
\end{equation}
所以\autoref{BAHA_eq3} 可改写为
\begin{equation}
\begin{aligned}
A^k B&=\sum_{m=0}^{k-1}C_{k-1}^mA^{(m)}A^{k-m}+\sum_{m=0}^{k-1}C_{k-1}^mA^{(m+1)}A^{k-1-m}\\
&=A^{(0)}A^{k}+\sum_{m=1}^{k-1}\qty[C_{k-1}^m+C_{k-1}^{m-1}]A^{(m)}A^{k-m}+A^{(k)}\\
&=A^{(0)}A^{k}+\sum_{m=1}^{k-1}C_{k}^mA^{(m)}A^{k-m}+A^{(k)}\\
&=\sum_{m=0}^{k}C_{k}^mA^{(m)}A^{k-m}
\end{aligned}
\end{equation}
由数学归纳法原理,证得\autoref{BAHA_lem2} 

现在,Baker-Hausdorff公式就呼之欲出了!

\textbf{Baker-Hausdorff公式的证明:}
由\autoref{BAHA_lem2} 
\begin{equation}\label{BAHA_eq4}
\E^A B=\sum_{n=0}^{\infty}\frac{A^nB}{n!}=\sum_{n=0}^{\infty}\frac{1}{n!}\sum_{i=0}^{n}C_{n}^iA^{(i)}A^{n-i}
\end{equation}

将上式求和符号交换顺序,注意指标要求满足 $i\leq n$,可将\autoref{BAHA_eq4} 改写为
\begin{equation}\label{BAHA_eq6}
\begin{aligned}
\E^A B&=\sum_{i=0}^{\infty}\sum_{n=i}^{\infty}\frac{1}{n!}C_{n}^iA^{(i)}A^{n-i}\\
&=\sum_{i=0}^{\infty}\frac{1}{i!}A^{(i)}\sum_{n=i}^{\infty}\frac{1}{(n-i)!}A^{n-i}\\
&=\sum_{i=0}^{\infty}\frac{1}{i!}A^{(i)}\E^A
\end{aligned}
\end{equation}
\autoref{BAHA_eq6} 两边作用于 $\E^{-A}$ ,证得\autoref{BAHA_eq1} 

\subsubsection{较物理的证明}
现在,我们给出一种量子力学语言的证明方式.选择 $A$ 表象为工作空间,其基右矢集合为 $\qty{\ket{a_n}}$,且 $A\ket{a_n}=a_n\ket{a_n}$.为方便起见,仅考虑离散态的情况.完备性条件为
\begin{equation}
\sum_{n}\ket{a_n}\bra{a_n}=1
\end{equation}
由\autoref{BAHA_eq5} ,关于 $A$ 算符的矩阵表达式为:
\begin{equation}
\begin{aligned}
A^{(n)}&=\sum_{i,j,k,l}\sum_{m=0}^{n}(-1)^{n-m}C_{n}^{m}\ketbra{a_i}{a_i}A^m\ketbra{a_k}{a_k}B\ketbra{a_l}{a_l}A^{n-m}\ketbra{a_j}{a_j}\\
&=\sum_{i,j}\sum_{m=0}^{n}(-1)^{n-m}C_n^m{a_i}^m{a_j}^{n-m}\bra{a_i}B\ket{a_j}\ketbra{a_i}{a_j}\\
&=\sum_{i,j}\sum_{m=0}^{n}(-1)^{n-m}C_n^m{a_i}^m{a_j}^{n-m}B^i_{\; j}\ketbra{a_i}{a_j}
\end{aligned}
\end{equation}
由此,有
\begin{equation}\label{BAHA_eq7}
\begin{aligned}
\qty(A^{(n)})^{i}_{\; j}&=\bra{a_i}A^{(n)}\ket{a_j}=\sum_{m=0}^{n}(-1)^{n-m}C_n^m{a_i}^m{a_j}^{n-m}B^i_{\; j}\\
A^{(n)}B&=\sum_{i,j,k}\ketbra{a_i}{a_i}A^{(n)}\ketbra{a_k}{a_k}B\ketbra{a_j}{a_j}\\
&=\sum_{i,j,k}\qty(A^{(n)})^{i}_{\; k}B^{k}_{\;j}\ketbra{a_i}{a_j}\\
&=\sum_{i,j,k}\sum_{m=0}^{n}(-1)^{n-m}C_n^m{a_i}^m{a_k}^{n-m}B^k_{\; j}\ketbra{a_i}{a_j}\\
BA^{(n)}&=\sum_{i,j,k}\ketbra{a_i}{a_i}B\ketbra{a_k}{a_k}A^{(n)}\ketbra{a_j}{a_j}\\
&=\sum_{i,j}B^{i}_{\;k}\qty(A^{(n)})^{k}_{\; j}\ketbra{a_i}{a_j}\\
&=\sum_{i,j}B^{i}_{\;k}a_n^iB^m_{\enspace n}\ketbra{a_j}{a_n}
\end{aligned}
\end{equation}
由\autoref{BAHA_eq7} ,有
\begin{equation}
[A^{(i)}]
\end{equation}

%由\autoref{BAHA_lem1} 
%\begin{equation}
%\E^A B=\sum_{n=0}^{\infty}\frac{1}{n!}\sum_{i=0}^{n}C_{n}^iA^{(i)}A^{n-i}=
%\end{equation}