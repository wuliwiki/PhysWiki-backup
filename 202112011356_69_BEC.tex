% 玻色爱因斯坦凝聚
% 玻色气体|玻色爱因斯坦凝聚|玻色分布

\pentry{玻尔兹曼分布(统计力学)\upref{MBsta}}

考虑由 $N$ 个\textbf{全同\footnote{需要考虑全同粒子假设,$\Omega=\Omega_{M.B.}/N!$.}、近独立\footnote{近独立的意思是,系统的总能量近似等于所有单粒子能量的总和,即忽略粒子相互作用势.这是一个极粗糙的近似,但我们可以以此简化计算得到系统可能的一些性质.}}的玻色子组成的系统.根据玻色分布,处在能级 $\epsilon_l$ 上的粒子数为
\begin{equation}
a_l=\frac{\omega_l}{e^{\frac{\epsilon_l-\mu}{kT}}-1}
\end{equation}
处于任意能级上的粒子数不能为负的,即 $a_l\ge 0$,这要求对任意能级 $\epsilon_l$都 $>\mu$.所以化学势 $\mu<0$.化学势可以由下式确定:
\begin{equation}\label{BEC_eq1}
\sum_la_l=\sum_l \frac{\omega_l}{e^{\frac{\epsilon_l-\mu}{kT}}-1} =N
\end{equation}
可以看出,化学势随温度的降低而升高.当温度降到某一临界温度 $T_C$ 时,$\mu$ 将趋于 $-0$,大量粒子将聚集在最低的单粒子态(即基态)上,直到绝对零度时,所有粒子都会凝聚到基态上.这种无相互作用系统中,宏观数量的玻色子凝聚到能量最低的单粒子态上的现象就被称为\textbf{玻色爱因斯坦凝聚}.

下面计算临界温度 $T_C$.将\autoref{BEC_eq1} 中的求和用积分代替,并利用\autoref{IdED1_eq2}~\upref{IdED1}化简.可以得到
\begin{equation}
\frac{2\pi V}{h^3}(2m)^{3/2}\int_0^\infty \frac{\epsilon^{1/2}\dd \epsilon}{e^{\frac{\epsilon-\mu}{kT}}-1}=N
\end{equation}
当 $T$ 降到 $T_C$ 时,$\mu$ 趋于 $-0$.所以临界温度 $T_C$ 由下式给出
\begin{equation}
\frac{2\pi V}{h^3}(2m)^{3/2}\int_0^\infty \frac{\epsilon^{1/2}\dd \epsilon}{e^{\frac{\epsilon-\mu}{kT}}-1}=N
\end{equation}
