% 线性引力
% 线性近似|线性爱因斯坦方程|弱引力场

\begin{issues}
\issueMissDepend
\issueDraft
\end{issues}



\subsection{线性引力理论}
爱因斯坦场方程\footnote{在广义相对论中,常常采用几何单位制\upref{NatUni},也即是$c=G=k_B=1$}如下

\begin{equation}
G_{\mu \nu} = R_{\mu \nu} - \frac{1}{2}g_{\mu\nu}R = 8 G\pi T_{\mu\nu}
\end{equation}

由于方程采用几何语言描述,十分简洁,并且很难看出它包含着一系列复杂的非线性微分方程.一方面,寻求严格满足爱因斯坦场方程的特定解是一个漫长而艰难的过程,许多数学天才也投入其中,取得了一些出色的结果.另一方面,在大多数情况中引力场都很微弱,我们可以采用近似处理使爱因斯坦场方程线性化,简而言之,实际上就是对背景时空进行一阶线性微扰.

\begin{equation}
g_{\mu\nu}=g^{(0)}_{\mu\nu}+\epsilon g^{(1)}_{\mu\nu} + \frac{1}{2}\epsilon^2 g^{(2)}_{\mu\nu}+\cdots
\end{equation}


\subsection{闵氏时空}

我们先考虑背景时空为闵氏时空的简单情况,之后也可将背景时空推广为一般时空.

\begin{equation}
g_{\mu\nu} = \eta_{\mu\nu} + h_{\mu\nu},\quad \abs{h_{\mu\nu}}<<1
\end{equation}

例如,对于太阳系来说,$\abs{h_{\mu\nu}} \sim 10^{-6}$.


\subsection{史瓦西时空}

我们可以将史瓦西时空看作对于平直闵氏时空的微扰.


\subsection{规范不变性}


\subsection{推广到一般时空}

背景时空的选择其实是任意的,我们同样可以对其进行线性微扰.

