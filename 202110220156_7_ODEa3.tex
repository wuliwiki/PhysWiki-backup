% 一阶常微分方程解法:恰当方程
% keys 恰当方程|微分方程|ODE|ordinary differential equation|exact equation|积分因子|integration factor

\pentry{常微分方程简介\upref{ODEint}}

\subsection{恰当方程的概念}

考虑一个二元函数$u(x, y)$,其全导数为
\begin{equation}
\dd u=\frac{\partial u}{\partial x}\dd x+\frac{\partial u}{\partial y}\dd y
\end{equation}

由多元微积分知识可知,$\frac{\partial^2 u}{\partial x\partial y}=\frac{\partial^2 u}{\partial y\partial x}$.因此,如果一个形如
\begin{equation}\label{ODEa3_eq1}
M(x, y)\dd x+N(x, y)\dd y=0
\end{equation}
的常微分方程满足
\begin{equation}\label{ODEa3_eq2}
\frac{\partial M}{\partial y}=\frac{\partial N}{\partial x}
\end{equation}
那么就可以存在一个$u(x, y)$,使得$M=\partial u/\partial x$和$N=\partial u/\partial y$.

这样一来,\autoref{ODEa3_eq1} 就相当于
\begin{equation}
\dd u=0
\end{equation}
其解就是$u=C$,$C$为积分常数.

也就是说,对于这样的方程,我们只需要求出$u$就能求解.

\begin{definition}{恰当方程}
将形如\autoref{ODEa3_eq1} 且满足\autoref{ODEa3_eq2} 的方程,称为\textbf{恰当方程(exact equation)}.
\end{definition}

我们研究一个例子,看看恰当方程是怎么解的.

\begin{example}{}\label{ODEa3_ex1}
考虑方程$\frac{\dd y}{\dd x}=\frac{y}{3y^2-x}$.

移项后得到$y\dd x+(x-3y^2)\dd y=0$.

记$M=y$,$N=x-3y^2$,则容易验证$\partial M/\partial y= 1 =\partial N/\partial x$,因此这是一个恰当方程.

我们希望找出一个$u$,使得$\partial u/\partial x=M$且$\partial u/\partial y=N$.

先用$M$关于$\dd x$积分,因为这样积分出来的结果再对$x$求偏微分就能得到$M$:
\begin{equation}\label{ODEa3_eq3}
u=\int M\dd x=\int y\dd x=xy+C_1(y)
\end{equation}
其中$C_1(y)$是一个关于$x$的常数——它完全可以是一个关于$y$的非常数函数,在对$x$求偏微分的时候不影响结果.

再用$N$关于$\dd y$积分,得到
\begin{equation}\label{ODEa3_eq4}
u=\int N\dd y=\int (x-3y^2)\dd y=xy-y^3+C_2(x)
\end{equation}
同样,$C_2(x)$是一个关于$y$的常数,它最多只和自变量$x$有关.

比较\autoref{ODEa3_eq3} 和\autoref{ODEa3_eq4} ,可见$C_1(y)=C_2(x)-y^3$,因此$C_2(x)$必须是一个和$x$也无关的常数,记为$C$,进而有$C_1(y)=C-y^3$.

代回\autoref{ODEa3_eq3} 或\autoref{ODEa3_eq4} ,得
\begin{equation}
u=xy-y^3+C
\end{equation}

于是方程的解就是$xy-y^3=K$,其中$K$为积分常数.

当$K\not=0$时,还可以写成$x-y^2=K$.

\end{example}


\subsection{积分因子}

如果给定的方程是恰当的,那按照\autoref{ODEa3_ex1} 的步骤就能很容易解出来.然而我们实际上遇到的方程往往乍一看不是恰当方程.不过,很多时候我们可以将一个非恰当方程转化为恰当方程,最常用的就是积分因子法.

将方程化为恰当方程的过程,在有些材料里也被称为\textbf{凑微分法}.

\begin{definition}{积分因子}
对于\textbf{非恰当}常微分方程$M(x, y)\dd x+N(x, y)\dd y=0$,如果存在一个函数$f(x, y)$,使得
\begin{equation}\label{ODEa3_eq5}
f(x, y)M(x, y)\dd x+f(x, y)N(x, y)\dd y=0
\end{equation}
是一个恰当方程,那么称$f(x, y)$是原方程的一个\textbf{积分因子(integration factor)}.
\end{definition}

既然\autoref{ODEa3_eq5} 是一个恰当方程,那就有
\begin{equation}
\frac{\partial (fM)}{\partial y}=\frac{\partial (fN)}{\partial x}
\end{equation}
展开后有
\begin{equation}\label{ODEa3_eq6}
f\qty(\frac{\partial M}{\partial y}-\frac{\partial N}{\partial x})=N\frac{\partial f}{\partial x}-M\frac{\partial f}{\partial y}
\end{equation}

\autoref{ODEa3_eq6} 是一个关于未知函数$f$的\textbf{偏微分方程}.虽然如果求出$f$,我们就能得到一个恰当方程,快速解出原方程,但一般情况下$f$的求解比原方程还难.

不过,在一些特殊情况下,我们确实是可以更容易地求出积分因子的.

最常见的一种情况,是限定$f$是一个一元函数,从而将\autoref{ODEa3_eq6} 变成常微分方程.

\subsubsection{一元积分因子}

设$M(x, y)\dd x+N(x, y)\dd y=0$不是恰当方程,而$f(x)$是它的一个积分因子.由于$\partial f/\partial y=0$,\autoref{ODEa3_eq6} 就化为
\begin{equation}
\frac{\partial M}{\partial y}-\frac{\partial N}{\partial x}=\frac{N}{f}\frac{\dd f}{\dd x}
\end{equation}
其中左边是已知函数.这是一个变量可分离方程,移项后可得
\begin{equation}
\frac{\frac{\partial M}{\partial y}-\frac{\partial N}{\partial x}}{N}\dd x=\frac{1}{f}\dd f
\end{equation}

故
\begin{equation}\label{ODEa3_eq7}
f=\pm \exp(\int \frac{\frac{\partial M}{\partial y}-\frac{\partial N}{\partial x}}{N}\dd x)
\end{equation}

观察\autoref{ODEa3_eq7} 式可知,$M(x, y)\dd x+N(x, y)\dd y=0$有一个自变量只有$x$的积分因子$f(x)$的充要条件是,$\frac{\frac{\partial M}{\partial y}-\frac{\partial N}{\partial x}}{N}$是一个和$y$无关的函数.

类似地,$M(x, y)\dd x+N(x, y)\dd y=0$有一个自变量只有$y$的积分因子$f(y)$的充要条件是,$\frac{\frac{\partial N}{\partial x}-\frac{\partial M}{\partial y}}{M}$是一个和$x$无关的函数.

\begin{example}{}
考虑方程$xy^2\dd x+xy\dd y=0$.令$M=xy^2$,$N=xy$,可知$\frac{\partial M}{\partial y}-\frac{\partial N}{\partial x}=2xy-y\not=0$,故它是非恰当的.

但是,$\frac{\frac{\partial M}{\partial y}-\frac{\partial N}{\partial x}}{N}=\frac{2xy-y}{xy}=2-\frac{1}{x}$只有$x$一个自变量,因此我们可以求积分因子$f(x)$\footnote{接下来的计算中我们没有加入正负号、积分常数和自然对数的绝对值符号,这是因为只要找出一个积分因子就够了,不必兼顾全面性.}:
\begin{equation}
\begin{aligned}
f&=\exp(\int \frac{\frac{\partial M}{\partial y}-\frac{\partial N}{\partial x}}{N}\dd x)\\
&=\exp(\int(2-\frac{1}{x})\dd x)\\
&=\exp(2x-\ln x)\\
&=\frac{\E^{2x}}{x}
\end{aligned}
\end{equation}

把$f$乘到原方程里,得到
\begin{equation}\label{ODEa3_eq8}
y^2\E^{2x}\dd x+y\E^{2x}\dd y=0
\end{equation}

\autoref{ODEa3_eq8} 就是一个恰当方程.仿照\autoref{ODEa3_ex1} 的方法,求出
\begin{equation}
u=\frac{1}{2}y^2\E^{2x}
\end{equation}

因此,原方程的解就是
\begin{equation}
\frac{1}{2}y^2\E^{2x}=C
\end{equation}



\end{example}





\begin{example}{}
在\textbf{一阶常微分方程解法:常数变易法}\upref{ODEa2}词条中,我们讨论了方程
\begin{equation}
\frac{\dd y}{\dd x}=P(x)y+Q(x)
\end{equation}
其中$P$和$Q$都是所考虑区间上的连续函数.

现在,我们尝试用积分因子法解这个方程.

首先把方程改写为
\begin{equation}\label{ODEa3_eq9}
(P(x)y+Q(x))\dd x-\dd y=0
\end{equation}

而
\begin{equation}
\frac{\frac{\partial (P(x)y+Q(x))}{\partial y}+\frac{\partial 1}{\partial y}}{-1}=-P(x)
\end{equation}

因此\autoref{ODEa3_eq9} 有积分因子$f(x)=\E^{-\int P(x)\dd x}
$.

将\autoref{ODEa3_eq9} 改写为
\begin{equation}\label{ODEa3_eq10}
\E^{-\int P(x)\dd x}(P(x)y+Q(x))\dd x-\E^{-\int P(x)\dd x}\dd y=0
\end{equation}

对\autoref{ODEa3_eq10} 左边两项求积分,比较后得

\begin{equation}
u(x, y)=-y\E^{-\int P(x)\dd x}+\int  \E^{-\int P(x)\dd x}Q(x)  \dd x
\end{equation}

故通解为
\begin{equation}
-y\E^{-\int P(x)\dd x}+\int  \E^{-\int P(x)\dd x}Q(x)  \dd x=C
\end{equation}

这和\textbf{一阶常微分方程解法:常数变易法}词条中\autoref{ODEa2_eq2}~\upref{ODEa2}的结论是一样的.

\end{example}














