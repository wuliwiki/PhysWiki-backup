% 连续函数的性质
% keys 连续函数|闭区间|最值定理|介值定理|零点存在定理|一致连续

\addTODO{修改格式}

\subsection{有界性}

  若函数 $f(x)\in C[a,b]$,则 $f(x)$ 在 $[a,b]$ 上有界.

  证明:若 $f(x)$ 在 $[a,b]$ 上无界,则将 $[a,b]$ 等分成两个闭区间,$f(x)$ 必在其中一个子区间上无界,取定这样的一个区间并将其设为 $[a_1,b_1]$ . 同理,取定 $[a_2,b_2]$ . 如此进行下去,我们就得到了一个闭区间列 $\{[a_n,b_n]\}$,满足:

  (1)$[a_n,b_n]\supset[a_{n+1},b_{n+1}]$;

  (2)$b_{n+1}-a_{n+1}=\dfrac{b_n-a_n}{2}$;

  (3)$f(x)$ 在每个 $[a_n,b_n]$ 上无界.

  于是,由闭区间套定理\autoref{RCompl_the3}~\upref{RCompl}知,存在唯一的 $\xi\in \opn{\bigcap}\limits^{+\infty}\limits_{n=1}[a_n,b_n]$. 由 $f(x)\in C[a,b]$ 和 $\xi\in[a,b]$,知 $\forall\epsilon>0$,$\exists\ \delta>0$,$\forall x\in B(\xi,\delta)\bigcap[a,b]$,有 $\vert f(x)-f(\xi)\vert<\epsilon$,即 $f(x)$ 在 $B(\xi,\delta)\bigcap[a,b]$ 上有界. 又由条件(2),当 $n$ 充分大时,有 $[a_n,b_n]\subset B(\xi,\delta)$,故 $[a_n,b_n]\subset B(\xi,\delta)\bigcap[a,b]$,从而 $f(x)$ 在 $[a_n,b_n]$ 上有界,与条件(3)矛盾. 故 $f(x)$ 在 $[a,b]$ 上有界.

\subsection{最值定理}

  若 $f(x)\in C[a,b]$ ,则 $f(x)$ 在 $[a,b]$ 上必有最小值和最大值,即存在 $\xi,\zeta\in [a,b]$,使得 $f(\xi)\leq f(x)\leq f(\zeta)$ 对一切的 $x\in[a,b]$ 成立.

  证明:令 $M=\sup\limits_{x\in[a,b]}\{f(x)\}$,则由上确界的定义知,存在一个序列 $\{x_n\}\subset[a,b]$,使得 $f(x)\rightarrow M(x\rightarrow \infty)$. 由于 $\{x_n\}$ 是有界序列,所以必有一收敛子列 $\{x_{n_k}\}$. 设 $\lim\limits_{k\rightarrow\infty}x_{n_k}=\zeta$,则 $\zeta\in[a,b]$. 再由 $f(x)\in C[a,b]$,知
  $$
  f(\zeta)=\lim\limits_{k\rightarrow\infty}f(x_{n_k})=M
  $$

  


  同理可证存在 $\xi\in[a,b]$,使得
$$
  f(\xi)=m=\inf\limits_{x\in[a,b]}\{f(x)\}
$$
  **注**:介值定理只在闭区间上成立,开区间上边界处 $f(x)$ 可能趋于无穷.

\subsection{介值定理}

  设 $f(x)\in [a,b]$,记 $m=\min\limits_{x\in[a,b]}\{f(x)\}$,$M=\max\limits_{x\in[a,b]}\{f(x)\}$,则 $f([a,b])=[m,M]$,即对 $\forall \eta\in[m,M]$,$\exists \xi\in[a,b]$,使得 $f(\xi)=\eta$.

  证明:由最值定理知,$\exists\ x_1,x_2\in[a,b]$,使得 $f(x_1)=m$,$f(x_2)=M$. 下证对 $\forall \eta\in (m,M)$,$\exists\  \xi\in[a,b]$,使得 $f(\xi)=\eta$.

  不妨设 $x_1<x_2$,并定义
  $$
  E=\{x\in[x_1,x_2]:f(x)>\eta\}.
  $$
  显然有 $x_1\notin E$,$x_2\in E$. 现令 $E$ 的下确界为 $\xi$,则有 $f(\xi)=\eta$.

  事实上,若不然,则 $f(\xi)>\eta$. 由连续函数的局部保号性知,存在 $\xi$ 的一个邻域 $U(\xi,\delta)$,使得 $f(x)>\eta$ 对一切 $x\in U(\xi,\delta)$ 成立. 取 $\xi'\in[x_1,x_2]\bigcap [\xi-\delta,\xi]$,则有 $f(\xi')\in E$ 及 $\xi'<\xi$,这与 $\xi$ 是 $E$ 的下确界矛盾.

\subsection{零点存在定理}

  设 $f(x)$ 在区间 $I$ 上连续,若 $\alpha,\beta\in I$,$\alpha<\beta$,满足 $f(\alpha)f(\beta)<0$,则存在 $\xi\in(\alpha,\beta)$,使得 $f(\xi)=0$.

  零点存在定理实质上是介值定理的一种特殊情况,常被用来证明非线性方程之解的存在性,或者用它来求方程的近似根. 如:

\subsection{Question}

  1.证明三次方程 $ax^3+bx^2+cx+d=0$ 至少有一实根.

  2.证明方程 $x^4+2x-1=0$ 在 $(0,1)$ 中必有一根,并求它的一个近似解.

\subsection{定义  一致连续}

  设函数 $f(x)$ 在区间 $I$ 上有定义.  若 $\forall \epsilon>0$,$\exists\ \delta>0$,当 $x_1,x_2\in I$ 且 $\vert x_1-x_2\vert<\delta$ 时,有 $\vert f(x_1)-f(x_2)\vert<\epsilon$,则称 $f(x)$ 在 $I$ 上**一致连续**. 

  由此,$f(x)$ 在 $I$ 上不一致连续与以下断言等价:存在 $\epsilon_0>0$ 及序列 $\{x'_n\}$,$\{x''_n\}\subset I$ 满足 $x'_n-x''_n\rightarrow0(n\rightarrow\infty)$ 及 $\vert f(x'_n)-f(x''_n)\vert\geq\epsilon_0$.

  显然,若 $f(x)$ 在 $I$ 上一致连续,则它必在 $I$ 上连续.

\subsection{Question}

1.证明 $f(x)=\sin x$ 在 $(-\infty,+\infty)$ 上一致连续.

2.试证明 $f(x)$ 在 $I$ 一致连续的充分必要条件是:对任意的两个序列 $\{x'_n\}\subset I$,$\{x''_n\}\subset I$,若它们满足 $\lim\limits_{n\rightarrow\infty}(x'_n-x''_n)=0$,必有 $\lim\limits_{n\rightarrow\infty}[f(x'_n)-f(x''_n)]=0$.

\subsection{康托尔定理}

设函数 $f(x)\in C[a,b]$,则 $f(x)$ 在闭区间 $[a,b]$ 上一致连续.

证明:定义
$$
F(x)=\left\{
\begin{matrix}
f(a),\ \ \ \ &x\in(-\infty,a),\\
f(x),\ \ \ \ &x\in[a,b],\\
f(b),\ \ \ \ &x\in(b,+\infty),
\end{matrix}
\right.
$$

则 $F(x)\in C(-\infty,+\infty)$. $\forall \epsilon>0$ 和 $x\in[a,b]$,$\exists\ \delta_x>0$,当 $x',x''\in U(x,2\delta_x)$ 时,有

$$
\vert F(x')-F(x'')\vert<\epsilon.
$$

显然,开区间簇 $\{U(x,\delta_x):x\in[a,b]\}$ 是闭区间 $[a,b]$ 的一个开覆盖. 由有限覆盖定理知,必有其中有限多个开区间

$$
U(x_1,\delta_{x_1}),U(x_2,\delta_{x_2}),\cdot\cdot\cdot,U(x_m,\delta_{x_m})
$$

它们完全覆盖了 $[a,b]$. 令 $\delta=\min\{\delta_{x_j}:j=1,2,\cdot\cdot\cdot,m\}$,则 $\delta>0$,且当 $x',x''\in[a,b]$ 且 $\vert x'-x''\vert<\delta$ 时,必定存在某一 $U(x_j,\delta_{x_j})$,使得 $x'\in U(x_j,\delta_{x_j})$,即 $\vert x'-x_j\vert<\delta_{x_j}$,从而 $\vert x''-x_j\vert\leq\vert x''-x'\vert+\vert x'-x_j\vert\leq\delta+\delta_{x_j}\leq2\delta_{x_j}$,于是有 $x',x''\in U(x_j,2\delta_{x_j})$. 这样,由 $\delta_{x_j}$ 的定义,有

$$
\vert f(x')-f(x'')\vert=\vert F(x')-F(x'')\vert<\epsilon
$$

注意到 $\delta$ 只与 $\epsilon$ 有关而与 $x',x''$ 无关,故 $f(x)$ 在 $[a,b]$ 上一致连续.

\subsection{Question}

1.设函数 $f(x)$ 在 $(a,b)$ 上连续. 证明:$f(x)$ 在 $(a,b)$ 上一致连续的充要条件是 $\lim\limits_{x\rightarrow a+0}f(x)$ 与 $\lim\limits_{x\rightarrow b-0}f(x)$ 都存在.

2.设 $f(x)$ 是定义在 $(-\infty,+\infty)$ 上的连续周期函数,证明 $f(x)$ 在 $(-\infty,+\infty)$ 上一致连续.
