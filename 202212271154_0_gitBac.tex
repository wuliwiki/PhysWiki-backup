% 用 git 备份文件夹

\begin{issues}
\issueDraft
\end{issues}

\pentry{Git 笔记\upref{Git}}

以下 bash 程序可以把若干个文件夹中的每个都用 git 备份到 \verb|备份目录|。 运行完后, 每个 \verb|文件夹*| 会保存为 \verb|备份目录/文件夹*.git|。 这个其实就是 git 仓库中的 \verb|.git| 文件夹(注意并不是 bare repo, 不可以用于上游)。

\begin{lstlisting}[language=bash, git-backup.sh]
repos="文件夹1 文件夹2 ..."
dir="备份目录"

for repo in $repos
do
  printf "\n\n\n===============================\n"
  echo "$repo"
  printf "===============================\n\n\n"
  if ! [ -d "$repo" ]; then
    echo "error: directory not found!"
    continue
  fi
  gitdir="${dir}${repo}.git"
  dirflag="--work-tree=\"$repo\" --git-dir=\"$gitdir\""
  echo "mkdir $gitdir ..."
  mkdir -p \"$gitdir\" &&
  echo "git init ..." &&
  git $dirflag init &&
  echo git add -A ... &&
  git $dirflag add -A --verbose &&
  echo "git commit..."
  # 其他有用的命令:
  # git $dirflag commit -m 'update'
  # git $dirflag status
done
\end{lstlisting}
