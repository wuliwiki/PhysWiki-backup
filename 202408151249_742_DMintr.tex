% 暗物质引言
% keys 暗物质
% license Usr
% type Tutor

在过去的世纪里,我们理解了日常物质的结构。量子力学和相对论教会我们,我们观察到的“物质”的本质常常与我们的直觉不同。这是否意味着物质的科学探索已经结束了呢?还没有。宇宙中还有更多的物质有待理解,这就是被称为暗物质的东西。我们对物质的探索从地球转向了太空。暗物质的存在已经通过从星系到宇宙学尺度的观测得到了证实。然而,这些观测只探测到了暗物质的引力耦合——总质量和其空间分布。为了真正理解暗物质是什么,我们还需要观察它与普通物质的其他可能的相互作用。因此,暗物质领域吸引了大量的关注:目前大约10\%的宇宙学、天体物理学和粒子物理学的论文提到了“暗物质”。不久以前,“中性微子”这个词经常被用作“弱相互作用质量粒子”(WIMP)的同义词,进而与“暗物质粒子”或简单地“暗物质”互换使用。这些都是极端和狭隘的简化。它们在某种程度上是由大约十年前支配粒子物理学领域的理论偏见所证明的,但现在它们不再有正当理由。重新思考这些理论假设的原因是,中性微子的理论基础——超对称性——并没有在对撞机上出现,而在大型强子对撞机(LHC)启动后,实验搜索取得了显著进展,但迄今为止没有积极的证据。此外,WIMP假设本身也受到了密切关注,在过去二十年中,其可行的参数空间显著减少。即使是曾经常见的隐含假设,即暗物质是一个新的基本粒子,现在也在被重新考虑。缺乏暗物质非引力相互作用的实验证据可能表明,暗物质是我们当前理论范式之外的东西,社区正在探索以前较少关注的可能性。在这个时候,回顾暗物质研究的现状似乎很有用,从更广泛的角度来看。这并不意味着否定“传统”研究方向,而是增加新的研究方向。在这样做时面临的挑战是,该领域正在分化,很难决定在这样一项努力中应该给予哪些主题或方面更多的权重。当前暗物质物理学的情况可能类似于古代哲学家讨论普通物质性质的时候:提出了模糊的原子概念以及其他可能性,好的想法积累起来,直到讨论停滞不前,等待“尤里卡”时刻的到来,之后一个连贯的图景出现了,合理的结果积累起来。表1总结了当前的状态,图1显示了科学的典型学习曲线。