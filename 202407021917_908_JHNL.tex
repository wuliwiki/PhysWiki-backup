% 结合能量
% license CCBYSA3
% type Wiki

(本文根据 CC-BY-SA 协议转载自原搜狗科学百科对英文维基百科的翻译)

在物理学中,结合能(也称为分离能)是将粒子系统分解成独立部分所需的最小能量。这能量等于质量亏损减去一个束缚系统(其势能通常低于其组成部分的势能总和)产生时释放的能量或质量,结合能也是使系统保持一体的原因。

\subsection{一般概念}
一般而言,结合能表示克服将物体固定在一起的力所必须做的机械功,把物体的组合部分分离到足够远的距离,使得进一步分离需要的额外功可以忽略不计。

在束缚系统中,如果结合能从系统中移除,它必须从非束缚系统的质量中减去,因为该能量具有质量。因此,如果能量在系统绑定时从系统中移除(或放出),这种能量损失也将导致系统质量的损失。[1]在这个过程中,系统的质量不守恒,因为在结合过程中,系统是“开放的”(即,从质量/能量输入或损失角度来说,并非孤立的系统)。

有几种类型的结合能,每一种都在不同的距离和能量尺度内运行。束缚系统的尺寸越小,其相关的束缚能就越高。

\subsection{结合能的类型}
\subsection{质量-能量关系}
束缚系统通常处于比其非束缚组分更低的能级,因为其质量必须小于未结合成分的总质量。对于具有低结合能的系统,结合后的“损失”质量可能占微小的比例,而对于具有高结合能的系统,损失的质量可能是容易测量的比例。这里缺失的质量可能在结合过程中以热或光的形式作为能量损失掉,其移除的能量与对应的移除的质量通过爱因斯坦方程$E = mc^2$相联系。在结合过程中,系统的成分可能进入原子核/原子/分子的较高能量状态,同时保持它们的质量,因此,在质量降低之前被从系统中去除是必要的。一旦系统冷却到正常温度并返回到与能级相关的基态,它的质量将比第一次结合时和处于高能状态时要小。热量的损失代表“质量不足”,热量本身保留了损失的质量(从初始系统的角度来看)。这个质量将出现在任何其他吸收热量并获得热能的系统中。[7]

例如,如果两个物体在空间中通过引力场相互吸引,吸引力会加速物体,增加它们的速度,从而将它们的(重力)势能转化为动能。当粒子在没有相互作用的情况下穿过彼此或者在碰撞过程中弹性排斥时,获得的动能(与速度相关)开始转化为势能,从而将碰撞的粒子分开。减速粒子将返回到初始距离,并继续运动到无穷远处,或者停止并重复碰撞(发生振荡)。这表明不会损失能量的系统,不会结合(束缚)成一个固体物体,该物体的一部分在短距离内振荡。因此,为了结合粒子,由于吸引力而获得的动能必须通过阻力耗散。碰撞中的复杂物体通常经历非弹性碰撞,将一些动能转化为内能(热量,即原子运动),再以光子的形式辐射出去——光和热。一旦逃离重力的能量在碰撞中耗散,这些部分将在更近的距离(可能是原子尺度)振荡,因此看起来像一个固体物体。这种失去的能量是结合能,是克服分离物体的势垒所必需的。如果这种结合能作为热量保留在系统中,它的质量不会降低,而作为热辐射从系统中损失的结合能本身就有质量。它直接代表了冷束缚系统的“质量亏损”。

化学和核反应中也有类似的考虑。封闭系统中的放热化学反应不会改变质量,但一旦反应热消失,质量会下降,尽管这种质量变化太小,无法用标准设备测量。在核反应中,可以作为光或热移除的质量分数,即结合能,通常是系统质量的更大比例。因此,它可以直接测量为反应物和(冷却的)产物的静质量之间的质量差。这是因为核力比库仑力相对更强,库仑力即电子和质子之间的相互作用,在化学中产生热量。

\subsubsection{3.1 质量变化}
束缚系统中的质量变化(减少),尤其是原子核的质量变化,也被称为质量缺失、质量亏损或质量堆积比。

用非束缚系统计算的质量和实验测量的原子核质量(质量变化)之间的差值表示为Δm.,可以按如下方式计算:

质量变化=(非束缚系统计算质量)-(系统测量质量)

即(质子和中子的质量之和)-(测量的原子核质量)

在产生激发态核的核反应发生后,为了衰变到未激发状态,作为结合能辐射或移除的能量可以是几种形式之一。这可能是电磁波,如伽马辐射;内部转换衰变中喷射粒子(如电子)的动能;或者部分作为一个或多个发射粒子的静质量,例如β衰变粒子。理论上,直到这种辐射或这种能量被释放出来,不再是系统的一部分,质量亏损才会出现。

当核子结合在一起形成原子核时,它们会失去少量的质量,即保持结合需要质量的变化。根据关系式$E = mc^2$,这种质量变化会被释放为如上所述的各种类型的光子或其他粒子能量。因此,移除结合能后,$\textbf{结合能=质量变化\times c^2}$。这种能量是将核子结合在一起的力的度量。它代表了为使原子核分裂成单个的核子必须从环境中重新提供的能量。

例如,对于氘原子,质量缺陷等于0.0023884 amu,其结合能几乎等于2.23 MeV。这意味着分裂氘原子需要这么多能量。[8]

核聚变或核裂变过程中释放的能量是“燃料”——即初始核素——的结合能与裂变或聚变产物的结合能之差。实际上,这种能量也可以从燃料和产物之间的巨大质量差异中计算出来,这种计算使用已知核素的原子质量的先前测量值,而同类核素总是具有相同的质量。一旦放出的热量和辐射被移除,这种质量差就会出现,这是这种计算中所涉及的测量(非激发的)核素的(静)质量所必需的。