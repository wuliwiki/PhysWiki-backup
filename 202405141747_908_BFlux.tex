% 磁通量
% keys 磁通量|闭合曲面|矢势|环路积分
% license Xiao
% type Tutor

\begin{issues}
\issueDraft
\issueOther{应该由浅入深,矢势放到后面。添加例题。}
\end{issues}

\pentry{曲面积分 通量\nref{nod_SurInt}}{nod_8836}

\footnote{参考 \cite{GriffE} 和 Wikipedia \href{https://en.wikipedia.org/wiki/Magnetic_flux}{相关页面}。}磁通量的一个直观的含义是,通过空间中一个曲面的磁感线的条数,对于匀强磁场而言,如果磁场 $\bvec B$ 与平面法向量 $\bvec S$ 的夹角为 $\theta$,且平面面积为 $S$,那么通过该平面的磁通量为 $|\bvec B|S \cos\theta = \bvec B\cdot \bvec S$。

将上述结果推广到任意曲面,以及非匀强磁场的情况。
令空间中磁感应强度为 $\bvec B(\bvec r)$ 我们可以通过曲面积分来定义一个通过某曲面的\textbf{磁通量(magnetic flux)}为
\begin{equation}
\Phi  = \int \bvec B(\bvec r) \vdot \dd{\bvec s}~.
\end{equation}
形象来说, 磁通量也可以看作是磁感线通过曲面的条数。 反方向的磁感线共线为负,和正方向磁感线抵消。

磁通量只与曲面的边界有关。设有两个曲面 $S_1$ 和 $S_2$ 的边界都是 $C$,而曲面 $S_1,S_2$ 围成的一个闭曲面 $S$(没有边界),穿过这个闭曲面的磁通量为
\begin{equation}
\int_{S_2} \bvec B(\bvec r)\cdot \dd {\bvec s }- \int_{S_1} \bvec B(\bvec r)\cdot \dd {\bvec s} = \int_{S} \bvec B(\bvec r)\cdot \dd {\bvec s}=\int_V \bvec \nabla \cdot \bvec B \dd V=0~.
\end{equation}
在最后我们利用了磁场的\enref{高斯定律}{MagGau}( $\bvec B$ 的散度恒为零), 以及\enref{高斯散度定理}{Divgnc} $\int_S \bvec V\cdot \dd{\bvec s}=\int_V \bvec \nabla\cdot \bvec B \dd V$。由此我们得到了
\begin{equation}
\int_{S_1} \bvec B(\bvec r)\cdot \dd {\bvec s}
=
\int_{S_2} \bvec B(\bvec r)\cdot \dd {\bvec s }~.
\end{equation}
由此我们利用了磁场的高斯定律证明了磁通量只与曲面的边界 $C$ 有关。
\subsection{穿过曲面的磁通量与磁矢势}
\pentry{磁矢势\nref{nod_BvecA},斯托克斯定理(矢量分析)\nref{nod_Stokes}}{nod_c8a3}
利用磁场矢势的定义 $\bvec B=\nabla\times \bvec A$ % \addTODO{链接}
及旋度定理, % \addTODO{链接}
磁通量变为
\begin{equation} \label{eq_BFlux_2}
\Phi  = \int \curl \bvec A \vdot \dd{\bvec s}  = \oint \bvec A \vdot \dd{\bvec r}~.
\end{equation}
磁矢势 $\bvec A$ 的定义正是由于磁场的散度为零% \addTODO{链接}
。因此从上式再次可以看出,如果选定一个闭合回路, 以该闭合回路为边界的任何曲面的磁通量都相等,只与沿闭合回路的磁矢势的积分有关。
\subsection{闭合线圈的磁通量}
如何计算一个通电闭合线圈对自己产生的磁通量呢?假设电流为 $I$,利用磁场矢势公式
\begin{equation}
\bvec A \qty(\bvec r) = \frac{\mu_0 I}{4 \pi} \oint \frac{\dd{\bvec r'}}{\abs{\bvec r - \bvec r'}}~.
\end{equation}
注意在该积分中, $\bvec r$ 视为常量, 积份完后, 积分变量 $\bvec r$ 消失。 现在根据\autoref{eq_BFlux_2} 再次将上式对 $\bvec r$ 进行同一环路积分得到磁通量
\begin{equation}\label{eq_BFlux_1}
\Phi  = \frac{\mu_0 I}{4\pi} \oint\oint \frac{\dd{\bvec r'} \dd{\bvec r}}{\abs{\bvec r - \bvec r'}}~.
\end{equation}

值得注意的是,对于理想导线(足够细的)组成的通电闭合回路,上式实际上是不良定义的,因为当 $\bvec r$ 与 $\bvec r'$ 距离趋于 $\bvec 0$ 时,积分是发散的。也可以从磁通量的角度理解,通电细导线附近的磁场强度与垂直导线方向的距离成反比,% \addTODO{计算}
离导线越近的地方磁场越强,当距离趋于零时磁场无穷大,这将导致闭合线圈对自己产生的磁通量为无穷大,从而导致许多电学磁学方面的悖论(比如感生电动势也将趋于无穷大)。

如何解决这个问题呢?上面的问题实际上反映了“理想导线”这一理想模型的缺陷,它并不能完备地给出所有问题的解释。现实中的导线都是有一定粗细的,电流通过导线时有一定的横截面积,而这保证了闭合线圈通电时对自己产生的磁通量是有限的,不会出现无穷大。也就是说,我们可以取“导线粗细”作为理论的一个位置空间上的“截断”,那么\autoref{eq_BFlux_1} 中 $\bvec r$ 与 $\bvec r'$ 距离趋于 $\bvec 0$ 的部分不会对结果有贡献,最终磁通量的计算结果是有限的。
