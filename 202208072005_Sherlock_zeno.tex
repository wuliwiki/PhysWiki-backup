% 芝诺时
% 芝诺

\begin{issues}

\end{issues}

\subsection{芝诺时与普通时}

\subsubsection{佯谬}
在一些定性的结论和定理未曾创立前,人们没有办法去理解和解决实际上的一些问题,在一些人的助推波澜下,人们的思想越来越混乱.在微积分发明之前,古代哲学家\textbf{芝诺}曾经提出过一系列的佯谬其中最为著名的就是他关于时间的悖论.

\subsubsection{芝诺的论点和论据}
论点:阿克琉斯(希腊神话中的英雄)的速度是乌龟的十倍,乌龟在他之前100米,但是他永远追不上乌龟.
论据:因为在他们开始赛跑时,乌龟在阿克琉斯前面100米,那么当阿克琉斯跑了100米时到乌龟原来的位置时,乌龟已经前进了10米;当阿克琉斯再前进10米时,乌龟又在他前面1米;而当他再跑1米时;乌龟又向前运动了  0.1米…… 如此下去,直至无穷.因此,阿克琉斯永远追不上乌龟.

\subsubsection{实际上的解释}
在我们现在看来,这个结论是错误的.因此,错误一定处在论证过程上.而错在哪里呢?仔细思考后,不难发现,芝诺实际上采取了与我们平时所用的时间量度不同的另一种量度.在芝诺时间里,本来有限的时间被分成了无限多份.但是这并不意味着时间有无限的数量.也就是说,在芝诺的时间取到无穷后,还是有时间存在的.换句话说,一种到达无限的时间量度,在其他量度上,有可能是有限的.

\subsubsection{推导和证明}
在思考到芝诺时间实际上是另一种时间量度后,物理学家们思考的是时间量度之间的转换:我们平时使用的时间量度与芝诺时间之间.有什么样的转换关系呢?

我们在这里,设我们日常使用的普通时间为$t$,芝诺时间为为$t'$,阿克琉斯与乌龟相距$L$,阿克琉斯的速度为$v_1$,乌龟的速度为$v_2$,且$v_1>v_1$
根据普通时间的运动学公式,不难得出:在普通的时间量度中,阿克琉斯将会在
\begin{equation}
t = \dfrac{L}{v_1 - v_2}
\end{equation}
时,赶上乌龟.
在上面的故事中,我们可以了解到:
当 $t'=1$ 时,阿克琉斯到达乌龟在 $t'=0$ 时的出发点,当$t'=2$ 时,他到达乌龟在 $t'=1$ 时的出发点……由此可得当他运动 $t'=n$ 时,他到达乌龟在 $t'=n-1$ 时的出发点.因此,只有当 $t'\to\infty$ 时,阿克琉斯才能无限接近乌龟.\begin{table}[ht]
\centering
\caption{芝诺时 $t'$ 与普通时 $t$ 之间的关系}\label{zeno_tab1}
\begin{tabular}{|c|c|}
\hline
芝诺时($t'$) & 普通时($t$) \\
\hline
0 & 0 \\
\hline
1 & $\dfrac{L}{v_1}$ \\
\hline
2 & $\dfrac{L}{V_1}+\dfrac{L}{v_1}\bullet\dfrac{V_2}{V_1}$ \\
\hline
3 & $\dfrac{L}{V_1}+\dfrac{L}{v_1}\bullet\dfrac{V_2}{V_1}+\dfrac{L}{v_1}\bullet(\dfrac{v_2}{v_1})^2$ \\
\hline
$\vdots$ & $\vdots$ \\
\hline
n & $\dfrac{L}{V_1}+\dfrac{L}{v_1}\bullet\dfrac{V_2}{V_1}+\dfrac{L}{v_1}\bullet(\dfrac{v_2}{v_1})^2+\cdots+\dfrac{L}{v_1}\bullet(\dfrac{v_2}{v_1})^{n-1}$ \\
\hline
\end{tabular}
\end{table}
