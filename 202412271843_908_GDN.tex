% 高德纳(综述)
% license CCBYSA3
% type Wiki

本文根据 CC-BY-SA 协议转载翻译自维基百科\href{https://en.wikipedia.org/wiki/Donald_Knuth}{相关文章}。

\begin{figure}[ht]
\centering
\includegraphics[width=6cm]{./figures/56ac8349febf8e6f.png}
\caption{2011年时的克努斯} \label{fig_GDN_1}
\end{figure}
唐纳德·厄尔文·克努斯(Donald Ervin Knuth,1938年1月10日出生)是美国计算机科学家和数学家,斯坦福大学荣誉教授。他是1974年图灵奖获得者,通常被视为计算机科学领域的“诺贝尔奖”[4]。克努斯被誉为“算法分析之父”[5]。

克努斯是《计算机程序设计的艺术》这一多卷本著作的作者。他为计算算法的复杂度分析做出了贡献,并系统化了相关的数学技巧。在此过程中,他还普及了渐近符号法。除了在多个计算机科学理论领域的基础性贡献外,克努斯还是TeX计算机排版系统的创造者,相关的METAFONT字体定义语言和渲染系统,以及Computer Modern字体系列的发明者。

作为作家和学者,克努斯创造了WEB和CWEB计算机编程系统,旨在鼓励和促进文献化编程,并设计了MIX/MMIX指令集架构。他坚决反对软件专利的授予,并已向美国专利商标局和欧洲专利组织表达过自己的看法。