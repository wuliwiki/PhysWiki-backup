% Legendre 符号(数论)
% keys 勒让德符号|Legendre符号|Legendre 符号
% license Usr
% type Tutor

\pentry{二次剩余\nref{nod_quares}}{nod_87de}
\subsection{定义}
\begin{definition}{Legendre 符号}
对于一个奇素数 $p$ 与一个不被 $p$ 整除的数 $a$,定义 \textbf{勒让德符号(Legendre 符号)} $\left( \frac{a}{p} \right)$ 是:
\begin{itemize}
\item 若 $a \opn R p$ 则 $\left( \frac{a}{p} \right) = +1$;
\item 若 $a \opn N p$ 则 $\left( \frac{a}{p} \right) = -1$。
\end{itemize}
\end{definition}

\begin{corollary}{}
由定义,立刻有
\begin{equation}
\left(\frac ap\right) ^2 = 1 ~.
\end{equation}

\end{corollary}

\begin{corollary}{}
若 $a \equiv b \pmod p$,而 $a$ 为不被奇素数 $p$ 整除的数,则
\begin{equation}
\left(\frac ap \right) = \left(\frac bp \right) ~.
\end{equation}

\end{corollary}

当 $p=2$ 时,不考虑其 Legendre 符号。

\subsection{应用}
\begin{theorem}{}\label{the_legnnt_1}
对于奇素数 $p$,若 $a$ 不是 $p$ 的倍数,则
\begin{equation}
(p-1)! \equiv -\left(\frac ap\right) a^{(p-1)/2} \pmod p ~.
\end{equation}
\end{theorem}
\textbf{证明}:若 $a \opn R p$ 且有 $x_1^2 \equiv a \pmod p$,则显然
\begin{equation}
x_2 = p-x_1 \equiv -x_1 \pmod p ~~
\end{equation}
也是 $x^2 \equiv a \pmod p$ 的解,此外若还有其他解 $x'^2 \equiv a \pmod p$,则
\begin{equation}
x^2 \equiv a \pmod p, x'^2 \equiv a \pmod p ~,
\end{equation}
这就指出,
\begin{equation}
x^2 - x'^2 = (x+x')(x-x') \equiv a - a = 0 \pmod p ~,
\end{equation}
从而 $(x+x') \equiv 0 \pmod p$ 或 $(x-x') \equiv 0 \pmod p$,这就指出了仅有 $x_1$ 与 $p-x_1$ 两个代表对应的模 $p$ 的剩余类是解。这就指出,仅有 $x_1$ 与 $p-x_1$ 的伴随数是自己。

此时可以立刻得出,各数 $1, 2, \dots, (p-1)$ 可以分成 $x_1, p-x_1$ 以及 $(p-3)/2$ 对不相等的伴随数对\footnote{可以证明这些数一定可以配对,可以利用费马小定理或\autoref{the_EulFun_1} 证明总可以构造,而由于 $1$ 至 $p-1$ 都与 $p$ 互素,故必定不重复。},其中
\begin{equation}
x_1(p-x_1) \equiv -x_1^2 \equiv -a \pmod p ~,
\end{equation}
而对任意一对伴随数 $x, x'$ 都有 $xx' \equiv a \pmod p$,故
\begin{equation}
(p-1)! = \prod x \equiv -a \cdot a^{(p-3)/2} \equiv -a ^{(p-1)/2} \pmod p ~.
\end{equation}
可以发现当 $a \opn R p$ 时,原定理成立。

若 $a \opn N p$,则 $x^2 \equiv a \pmod p$ 无解,故各数 $1, 2, \dots, (p-1)$ 可以分成 $(p-1)/2$ 对不相等的伴随数对\footnote{一样的,可以证明一定可以两两配对。},从而
\begin{equation}
(p-1)! = \prod x \equiv a^{(p-1)/2} \pmod p ~.
\end{equation}
故当 $a \opn N p$ 时,原定理也成立。证毕!