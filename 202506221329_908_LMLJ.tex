% 斯里尼瓦瑟·拉马努金(综述)
% license CCBYSA3
% type Wiki

本文根据 CC-BY-SA 协议转载翻译自维基百科 \href{https://en.wikipedia.org/wiki/Srinivasa_Ramanujan}{相关文章}。

斯里尼瓦瑟·拉马努金·艾扬加尔(Srinivasa Ramanujan Aiyangar)FRS(1887年12月22日-1920年4月26日)是一位印度数学家。他常被视为史上最伟大的数学家之一。尽管几乎没有接受过纯数学的正规训练,他仍在数学分析、数论、无穷级数和连分数等领域做出了重要贡献,并提出了当时被认为无法解决的数学问题的解法。

拉马努金最初是在孤立的环境中自行开展数学研究的。正如汉斯·艾森克所说:“他曾试图让当时最顶尖的职业数学家对他的研究产生兴趣,但大多数时候都失败了。他所展示的成果太新颖、太陌生,而且呈现方式也很不寻常;那些人懒得去理会。”\(^\text{[4]}\)为了寻找能真正理解他工作的数学家,1913年,他开始与英国剑桥大学的数学家G.H.哈代通信。哈代意识到拉马努金的研究非同寻常,便为他安排了赴剑桥的行程。在笔记中,哈代评论道,拉马努金提出了具有突破性的全新定理,其中一些“令我完全败下阵来;我从未见过任何类似的东西”,\(^\text{[5]}\)还有一些则是刚刚被证明、极为高深的成果。

在他短暂的一生中,拉马努金独立整理出了近 3900 条数学成果(主要是恒等式和方程)。\(^\text{[6]}\)其中许多都是前所未见的原创成果;他那些独特而极不寻常的发现,如“拉马努金素数”、“拉马努金θ函数”、“整数划分公式”以及“拟θ函数”等,不仅开辟了全新的研究领域,也激发了大量后续研究。\(^\text{[7]}\)在他成千上万的研究成果中,大多数后来都被证明是正确的。\(^\text{[8]}\)以他的名字命名的《拉马努金期刊》应运而生,专门发表受他研究影响的各类数学成果。\(^\text{[9]}\)他留下的笔记本——记录了他已发表和未发表成果的摘要——至今仍被数学家们分析和研究,成为不断涌现新数学思想的重要来源。直到 2012 年,研究者们仍不断发现,他笔记中那些仅以“简单性质”或“相似结果”带过的评论,其实暗藏着深奥而精妙的数论定理,且这些定理直到他去世近百年后才被真正识别出来。\(^\text{[10][11]}\)拉马努金是最年轻的英国皇家学会会士之一,是第二位印度籍成员,也是首位当选剑桥大学三一学院会士的印度人。

1919年,健康状况恶化——如今被认为是由多年前痢疾引发的并发症“肝阿米巴病”所致——迫使拉马努金返回印度。他于1920年去世,年仅32岁。他在1920年1月写给哈代的最后几封信表明,他在生命的最后时刻仍在不断提出新的数学思想和定理。他的“失落的笔记本”,记录了他生命最后一年中的诸多发现,于1976年被重新发现后,在数学界引起了极大的轰动。
\subsection{早年生活}
\begin{figure}[ht]
\centering
\includegraphics[width=6cm]{./figures/f833ac051ab76d62.png}
\caption{拉马努金的出生地位于埃罗德阿拉希里街18号,现属于泰米尔纳德邦。} \label{fig_LMLJ_1}
\end{figure}
拉马努金(意为“罗摩的弟弟”,罗摩是印度教神祇)于1887年12月22日出生在现今泰米尔纳德邦的伊罗德市的一个泰米尔婆罗门艾扬加尔家庭中。他的父亲库普苏瓦米·斯里尼瓦萨·艾扬加尔原籍坦贾武尔区,在一家纱丽店当职员。他的母亲科玛拉塔玛尔是家庭主妇,也在当地庙宇中唱诵圣歌。他们一家住在库姆巴科纳姆镇的萨兰加帕尼圣殿街上的一间传统小屋里,该住宅如今已被改为博物馆。

当拉马努金一岁半时,母亲生下了另一个儿子萨达戈潘,但他在出生不到三个月后夭折。1889年12月,拉马努金感染了天花,不过他幸运地康复了,而当时坦贾武尔区约有4000人死于这一年严重的天花疫情。此后,他随母亲搬到了她父母位于康契布勒姆(今切奈附近)的家中。母亲随后又分别于1891年和1894年生下两个孩子,但这两个孩子都未能活过一岁。

1892年10月1日,拉马努金被送入当地学校就读。\(^\text{[17]}\)后来,他的外祖父在康契布勒姆失去了担任法庭官员的职位,\(^\text{[18]}\)拉马努金和母亲便搬回了库姆巴科纳姆,并在那里就读于康伽延初级学校。\(^\text{[19]}\)随着祖父去世,他又被送回外祖父母位于马德拉斯(今金奈)的住所。但他不喜欢马德拉斯的学校,还经常试图逃课。他的家人甚至请了一位当地警察来监督他上学。不到六个月,拉马努金就再次回到了库姆巴科纳姆。\(^\text{[19]}\)
\begin{figure}[ht]
\centering
\includegraphics[width=6cm]{./figures/ee948be70794232c.png}
\caption{拉马努金位于库姆巴科纳姆萨兰加帕尼·桑尼迪街的故居} \label{fig_LMLJ_2}
\end{figure}
由于拉马努金的父亲大部分时间在外工作,照顾他的任务主要由母亲承担,因此母子之间关系十分亲密。他从母亲那里学习传统和《往世书》,学唱宗教歌曲,参加寺庙的祈祷仪式(puja),并遵守特定的饮食习惯——这些都是婆罗门文化的一部分。\(^\text{[20]}\)在康伽延初级学校,拉马努金表现出色。1897年11月,在即将年满10岁时,他以全区最高分通过了英语、泰米尔语、地理和算术的初等考试。\(^\text{[21]}\)同年,拉马努金进入了城镇高等中学,在那里他第一次正式接触到数学。\(^\text{[21]}\)

拉马努金在11岁时就已是神童,他很快掌握了家中两位大学生房客所掌握的全部数学知识。后来,他借到了一本由 S. L. Loney 编写的《高等三角学》教材。\(^\text{[22][23]}\)到13岁时,他不仅精通了书中的内容,还自主发现了一些复杂的定理。到了14岁,他已获得了多项优异成绩证书和学术奖项,并在整个求学阶段持续获奖。他还协助学校为大约1200名学生(每人需求不同)安排约35名教师的教学任务。\(^\text{[24]}\)他总能在规定时间的一半内完成数学考试,并表现出对几何学和无穷级数的熟练掌握。1902年,拉马努金学会了解三次方程的解法,后来又发展出一套解四次方程的方法。1903年,他尝试解五次方程,却不知道此类方程无法用根式来求解。\(^\text{[25]}\)

1903年,16岁的拉马努金从一位朋友那里借到了一本图书馆藏书——G. S. Carr 所编的《纯粹与应用数学初步成果概要》,这是一部包含5000个定理的合集。\(^\text{[26][27]}\)据说拉马努金详细研究了书中的内容。\(^\text{[28]}\)次年,他独立发展并研究了伯努利数,并将欧拉–马歇罗尼常数计算到小数点后15位。\(^\text{[29]}\)当时他的同龄人表示“很少能听懂他在讲什么”,并“对他怀有敬畏之心”。\(^\text{[24]}\)

1904年从城镇高等中学毕业时,拉马努金因数学成绩优异获得了校长克里希纳斯瓦米·艾耶尔颁发的K·兰加纳塔·拉奥数学奖。艾耶尔称拉马努金是“杰出的学生,应当得到超过满分的分数”。\(^\text{[30]}\)他获得奖学金进入库姆巴科纳姆的政府艺术学院学习,\(^\text{[31][32]}\)但由于他对数学过于专注,无法集中精力学习其他科目,最终大部分课程不及格,并因此失去了奖学金。\(^\text{[33]}\)1905年8月,拉马努金离家出走,前往维沙卡帕特南,随后在拉贾蒙德里停留了大约一个月。\(^\text{[33]}\)他后来在马德拉斯(今金奈)的帕查亚帕学院注册入学。在那里,他数学考试合格,只解答他感兴趣的问题,其他题目则空着不写,但在英语、生理学和梵文等其他科目表现不佳。\(^\text{[35]}\)拉马努金于1906年12月第一次参加文科士考试未通过,一年后再次失败。没有获得FA学位的他,离开了大学,继续独立从事数学研究,生活极度贫困,经常处于饥饿边缘。\(^\text{[36]}\)

1910年,23岁的拉马努金与印度数学学会创始人V·拉马斯瓦米·艾耶尔会面后,开始在马德拉斯的数学界获得认可,这也促成了他被马德拉斯大学接纳为研究人员。\(^\text{[37]}\)
\subsection{成年时期(在印度)}
1909年7月14日,拉马努金与贾纳基(Janaki,亦作 Janakiammal;1899年3月21日-1994年4月13日)结婚。\(^\text{[38]}\)贾纳基是他母亲一年前为他选定的新娘,当他们结婚时,她年仅十岁。\(^\text{[39][40][41]}\)当时为女孩在年幼时安排婚姻并不罕见。贾纳基来自拉詹德拉姆,这是位于马鲁杜尔(现属卡鲁尔县)火车站附近的一个村庄。拉马努金的父亲并未参加婚礼仪式。\(^\text{[42]}\)按照当时的习俗,婚后贾纳基仍在娘家居住了三年,直到她初潮来临。1912年,她与拉马努金的母亲一同前往马德拉斯与拉马努金团聚。\(^\text{[43]}\)
\begin{figure}[ht]
\centering
\includegraphics[width=6cm]{./figures/277959350c2d7195.png}
\caption{拉马努金独自坐着} \label{fig_LMLJ_3}
\end{figure}
婚后,拉马努金患上了睾丸鞘膜积液。\(^\text{[44]}\)这种病可以通过一个常规的外科手术来治疗,手术会引流阴囊囊袋中堵塞的液体,但他的家庭无力承担手术费用。1910年1月,一位医生自愿免费为他进行手术。\(^\text{[45]}\)

手术成功后,拉马努金开始寻找工作。他住在朋友家里,一边挨家挨户地在马德拉斯寻找文书职位。为了赚钱,他还为准备“文科学士”考试的总统学院学生辅导功课。\(^\text{[46]}\)

1910年末,拉马努金再次生病。他担心自己的健康,对朋友R·拉达克里希纳·艾耶说:“如果我不行了,就把这些[笔记本]交给Singaravelu Mudaliar教授(帕恰伊亚帕学院的数学教授),或是马德拉斯基督学院的英国教授爱德华·B·罗斯。”\(^\text{[47]}\)病愈后,拉马努金从艾耶手中取回笔记本,乘火车从库姆巴科纳姆前往维鲁普拉姆,这是一座当时处于法国控制下的城市。\(^\text{[48][49]}\)1912年,他与妻子和母亲搬到马德拉斯乔治镇的赛瓦·穆塔亚·穆达利街,在那里住了几个月。\(^\text{[50]}\)1913年5月,在获得马德拉斯大学的研究职位后,拉马努金携家人搬到了特里普利凯恩。\(^\text{[51]}\)
\subsubsection{追求数学事业}
1910年,拉马努金结识了副税务官V·拉马斯瓦米·艾耶,他是印度数学学会的创始人之一。\(^\text{[52]}\)拉马努金希望能在艾耶工作的税务部门谋得一份差事,于是将自己的数学笔记展示给他看。艾耶后来回忆道:

“我被那些笔记中非凡的数学成果所震撼。我实在不忍心把他的天才埋没在税务部门最底层的职位上。”\(^\text{[53]}\)

艾耶给了拉马努金几封介绍信,让他去拜访在马德拉斯的数学家朋友们。\(^\text{[52]}\)其中几位看过他的笔记后,给他写了推荐信,介绍他去见内洛尔的地方长官、同时也是印度数学学会秘书的R·拉马钱德拉·拉奥。\(^\text{[54][55][56]}\)拉奥对拉马努金的研究印象深刻,但一开始怀疑那些成果是否真是他本人所作。拉马努金提到他曾与孟买著名数学家萨尔达纳教授通信,对方虽然表示不太理解他的工作,但认定他并非骗子。\(^\text{[57]}\)拉马努金的朋友C·V·拉贾戈帕拉查里也出面说服拉奥打消对他学术诚实的怀疑。

拉奥决定再给拉马努金一次机会,并认真听他讲解关于椭圆积分、超几何级数和他提出的发散级数理论。拉奥表示,最终正是这些内容令他确信拉马努金才华卓绝。\(^\text{[57]}\)拉奥问他有什么要求,拉马努金答道,他需要的是工作和经济支持。拉奥答应资助他,并将他送往马德拉斯。拉马努金在拉奥的资助下继续进行研究。

在艾耶的帮助下,拉马努金的部分研究成果发表在《印度数学学会期刊》上。\(^\text{[58]}\)
\begin{figure}[ht]
\centering
\includegraphics[width=6cm]{./figures/c263599098bf421c.png}
\caption{} \label{fig_LMLJ_4}
\end{figure}
他在该期刊中提出的第一个问题之一是求以下表达式的值:\(^\text{[30]}\)
$$
\sqrt{1 + 2 \sqrt{1 + 3 \sqrt{1 + \cdots}}}~
$$
他在三期期刊中等待了六个月,希望有人能给出这个问题的解答,但始终没有收到回应。最终,拉马努金自己给出了一个不完整的解答。\(^\text{[59]}\)在他第一本笔记本的第105页中,他写下了一个可用于求解这个无限嵌套根式问题的等式:
$$
x + n + a = \sqrt{ax + (n + a)^2 + x \sqrt{a(x + n) + (n + a)^2 + (x + n) \sqrt{\cdots}}}~
$$
利用这个等式,只需设 $x = 2$、$n = 1$、$a = 0$,就可以得到期刊中所提出问题的答案为 3。\(^\text{[60]}\)拉马努金为该期刊撰写了第一篇正式论文,内容是关于伯努利数的性质。他发现的一个性质是:伯努利数分数的分母(在 OEIS 中为序列 A027642)始终可以被 6 整除。他还设计出一种基于已有伯努利数计算 $B_n$ 的方法,其中一种方法如下:

可以注意到,如果 $n$ 是偶数且不等于零:
\begin{enumerate}
\item $B_n$ 是一个分数,且在最简形式下 $\frac{B_n}{n}$ 的分子是一个质数;
\item $B_n$ 的分母中因子 2 和 3 各出现一次,且仅出现一次;
\item $2n(2n - 1)\frac{B_n}{n}$ 是一个整数,因此 $2(2n - 1)B_n$ 是一个奇整数。
\end{enumerate}
在他1911年发表的17页论文《伯努利数的一些性质》中,拉马努金给出了三个证明、两个推论和三个猜想。\(^\text{[61]}\)他最初的写作存在很多问题。正如期刊编辑 M. T. Narayana Iyengar 所指出的:

“拉马努金先生的方法过于简洁且新颖,他的表述缺乏清晰与严谨,普通的数学读者若不习惯如此高强度的思维体操,几乎无法理解他。”\(^\text{[62]}\)

拉马努金后来又写了一篇论文,并继续在《印度数学学会期刊》上提供数学问题。\(^\text{[63]}\)1912年初,他在马德拉斯审计长办公室找了一份临时工作,月薪为20卢比,但只工作了几个星期。\(^\text{[64]}\)在这份工作的末期,他申请了马德拉斯港务局总会计处的职位。

在1912年2月9日的一封信中,拉马努金写道:

尊敬的先生:

我听说贵办公室有一职位空缺,特此提出申请。我已通过中学毕业考试,并学习至F.A.(文凭),但由于若干不幸的原因未能继续深造。然而,我一直将全部时间用于数学研究与发展该学科。我可以有信心地说,如果我被录用,一定能胜任本职工作。因此,恳请您考虑予以录用为盼。\(^\text{[65]}\)

他在申请信中附上了总督学院数学教授 E. W. Middlemast 的推荐信,信中称拉马努金是“一位在数学方面具有非凡能力的青年”。\(^\text{[66]}\)三周后,即3月1日,拉马努金收到录用通知,成为三级四等会计文员,月薪30卢比。\(^\text{[67]}\)在工作中,他总是轻松迅速地完成任务,并将空闲时间用于数学研究。他的上司弗朗西斯·斯普林爵士以及同事、也是印度数学学会财务官的 S. Narayana Iyer 都非常鼓励他从事数学研究。\(^\text{[68]}\)
\subsubsection{联系英国数学家}
1913年春天,纳拉亚纳·艾耶、拉马钱德拉·拉奥和E·W·米德尔马斯特尝试将拉马努金的数学成果介绍给英国数学家。伦敦大学学院的M·J·M·希尔评论说,拉马努金的论文中充满了漏洞。\(^\text{[69]}\)他表示,尽管拉马努金“对数学有兴趣,也有一定的才能”,但他缺乏被数学界接受所需的教育背景与基础。\(^\text{[70]}\)尽管希尔并未提出要收拉马努金为学生,但他对拉马努金的工作给予了认真细致的专业建议。在朋友们的帮助下,拉马努金起草了信件,写给剑桥大学的几位著名数学家。\(^\text{[71]}\)

前两位教授,H. F. 贝克和E. W. 霍布森,在收到拉马努金的论文后没有发表评论便退还了。\(^\text{[72]}\)1913年1月16日,拉马努金写信给G. H. 哈代,他通过阅读《无穷阶的阶数》(Orders of Infinity,1910)而认识这位数学家。\(^\text{[73][74]}\)来自一位默默无闻数学家的九页数学公式起初让哈代怀疑这可能是一场骗局。\(^\text{[75]}\)虽然哈代认出了一些拉马努金的公式,但另一些却“几乎令人难以置信”。\(^\text{[76]: 494 }\)其中一个令哈代惊叹的定理位于第三页底部(在 $0 < a < b + \frac{1}{2}$ 的条件下有效):
$$
\int_0^{\infty} 
\frac{1 + \frac{x^2}{(b+1)^2}}{1 + \frac{x^2}{a^2}} 
\times \frac{1 + \frac{x^2}{(b+2)^2}}{1 + \frac{x^2}{(a+1)^2}} 
\times \cdots \, dx = \frac{\sqrt{\pi}}{2} \cdot 
\frac{ \Gamma\left(a + \frac{1}{2}\right) \Gamma(b + 1) \Gamma(b - a + 1) }{ 
\Gamma(a) \Gamma\left(b + \frac{1}{2}\right) \Gamma\left(b - a + \frac{1}{2}\right) }.~
$$
哈代还对拉马努金在无穷级数方面的其他工作印象深刻,例如:
$$
1 - 5\left(\frac{1}{2}\right)^3 + 9\left(\frac{1 \times 3}{2 \times 4}\right)^3 - 13\left(\frac{1 \times 3 \times 5}{2 \times 4 \times 6}\right)^3 + \cdots = \frac{2}{\pi}~
$$
以及:
$$
1 + 9\left(\frac{1}{4}\right)^4 + 17\left(\frac{1 \times 5}{4 \times 8}\right)^4 + 25\left(\frac{1 \times 5 \times 9}{4 \times 8 \times 12}\right)^4 + \cdots = \frac{2\sqrt{2}}{\sqrt{\pi}\, \Gamma^2\left(\frac{3}{4}\right)}~
$$
第一个结果早在1859年就已由G·鲍尔发现。而第二个结果对哈代来说是全新的,它来源于一类被称为超几何级数的函数,这类函数最早由欧拉和高斯研究。哈代认为这些结果“远比高斯在积分方面的工作更令人着迷”。\(^\text{[77]}\)当哈代看到拉马努金手稿最后一页关于连分数的定理时,他说道这些定理“完全把我打败了;我以前从未见过任何类似的东西”,\(^\text{[78]}\)并补充道,“它们一定是真的,因为如果不是真的,就不会有人有这样的想象力去凭空编造出来。”\(^\text{[78]}\)哈代让他的同事J·E·利特尔伍德也看了这些论文,利特尔伍德也对拉马努金的天才感到震惊。与利特尔伍德讨论后,哈代得出结论:这些信件“无疑是我所收到过的最非凡的”,拉马努金是“一位顶尖的数学家,一个极其具有独创性和力量的人”。\(^\text{[76]: 494–495}\)另一位同事E·H·内维尔后来回忆道:“当时在剑桥的数学圈里,没有人会忘记这封信引起的轰动……其中没有一个定理是世界上最先进的数学考试中可能出现的题目。”\(^\text{[63]}\)

1913年2月8日,哈代给拉马努金写了一封信,表达了对他研究的兴趣,并补充道:“我必须看到你某些论断的证明,这一点至关重要。”在这封信于2月第三周抵达马德拉斯之前,哈代已经联系了英属印度事务办公室,着手安排拉马努金前往剑桥的行程。\(^\text{[79]}\)印度学生咨询委员会秘书亚瑟·戴维斯会见了拉马努金,讨论出国事宜。\(^\text{[80]}\)但出于婆罗门的传统信仰,拉马努金拒绝“前往异国他乡”,他的父母也出于相同的原因表示反对。\(^\text{[81]}\)与此同时,拉马努金给哈代寄去了一封塞满定理的信,并写道:“我在您身上找到了一个能同情我的努力的朋友。”\(^\text{[82]}\)

为了补充哈代的推荐,剑桥大学三一学院前数学讲师吉尔伯特·沃克审阅了拉马努金的研究成果,并表示十分惊叹,强烈建议这位年轻人前往剑桥深造。\(^\text{[83]}\)得到沃克的支持后,工程学院的数学教授B·哈努曼塔·拉奥邀请拉马努金的同事纳拉亚纳·艾耶出席数学研究委员会的会议,以讨论“我们能为S·拉马努金做些什么”。\(^\text{[84]}\)委员会最终同意为拉马努金提供马德拉斯大学为期两年的每月75卢比的研究奖学金。\(^\text{[85]}\)

在担任研究生期间,拉马努金继续向《印度数学学会期刊》投稿。在一次投稿中,艾耶代为提交了拉马努金关于级数求和的一些定理,并附言写道:“以下定理出自马德拉斯大学的数学学生S·拉马努金。”同年11月,马德拉斯基督学院的英国教授爱德华·B·罗斯——几年前曾与拉马努金见过面——有一天突然神情激动地冲进教室,眼睛发亮地问学生们:“拉马努金懂波兰语吗?”原因是拉马努金在某篇论文中预见了一位波兰数学家的研究成果,而这位波兰学者的论文正好在当天的邮件中抵达。\(^\text{[86]}\)在他定期发表的论文中,拉马努金提出了使定积分更容易求解的定理。他以朱利亚诺·弗鲁拉尼(1821年的积分定理为基础,提出了可以用来计算原本难以求解的积分的一般化公式。\(^\text{[87]}\)

拉马努金拒绝前往英国后,他与哈代的通信关系一度恶化。为此,哈代请在马德拉斯任教的同事E.H. 内维尔(E. H. Neville)出面指导并劝说拉马努金前往英国。\(^\text{[88]}\)内维尔问拉马努金为何不愿去剑桥。此时拉马努金似乎已经接受了这个提议;内维尔说:“拉马努金并不需要被说服”,“他的父母也已不再反对”。\(^\text{[63]}\)据说,拉马努金的母亲做了一个生动的梦,梦见儿子被欧洲人环绕,家族的守护神——纳马吉里的女神在梦中命令她“不再阻碍儿子实现人生使命”。\(^\text{[63]}\)1914年3月17日,拉马努金乘船启程前往英国,\(^\text{[89]}\)将妻子留在印度与父母同住。\(^\text{[90]}\)
\subsection{英国生活}
\begin{figure}[ht]
\centering
\includegraphics[width=10cm]{./figures/4113055f26fb0a8d.png}
\caption{拉马努金(中)与同事G·H·哈代(最右),以及其他科学家一起,摄于剑桥大学参议院大楼外,约1914–1919年。} \label{fig_LMLJ_5}
\end{figure}
1914年3月17日,拉马努金从马德拉斯乘坐“内瓦萨号”轮船启程离境。\(^\text{[91][92]}\)4月14日他抵达伦敦时,内维尔已开车在港口等候。四天后,内维尔将他带到自己位于剑桥切斯特顿路的住所。拉马努金随即开始与利特尔伍德和哈代展开工作。六周后,拉马努金搬出内维尔家,搬进威尔庭院居住,那里距离哈代的房间仅五分钟步行路程。\(^\text{[93]}\)

哈代和利特尔伍德开始研究拉马努金的笔记本。哈代此前在拉马努金的前两封信中已经收到了120个定理,但笔记本中还有更多的结果和定理。哈代发现其中一些是错误的,一些早已被他人发现,而其余的则是全新的突破。\(^\text{[94]}\)拉马努金给哈代和利特尔伍德留下了深刻印象。利特尔伍德评论道:“我相信他至少可以媲美雅可比”,\(^\text{[95]}\)而哈代则说:“我只能把他拿来和欧拉或雅可比相比。”\(^\text{[96]}\)
\begin{figure}[ht]
\centering
\includegraphics[width=6cm]{./figures/1bd3ed4fc46ce279.png}
\caption{剑桥大学三一学院的惠威尔庭院} \label{fig_LMLJ_6}
\end{figure}
拉马努金在剑桥与哈代和利特尔伍德合作了将近五年,并在此期间发表了部分研究成果。哈代和拉马努金的性格截然不同,他们的合作是一场文化、信仰和工作方式的碰撞。在此前的几十年里,数学基础受到了质疑,对严格数学证明的需求也逐渐被认可。哈代是无神论者,是数学证明和严谨性的坚定信奉者,而拉马努金则是一个虔诚的宗教信徒,他极其依赖直觉与洞察力。哈代尽最大努力弥补拉马努金在教育上的空白,指导他理解用形式化证明来支撑结果的重要性,同时又不压抑他的灵感——这场张力对双方来说都并不轻松。
\begin{figure}[ht]
\centering
\includegraphics[width=6cm]{./figures/4ad46dddfe366e48.png}
\caption{} \label{fig_LMLJ_7}
\end{figure}
1916年3月,拉马努金因其关于“高度合成数”的研究被授予“研究型文学士学位”,这是如今博士学位的前身。该研究的第一部分已于前一年发表在《伦敦数学会会刊》上。论文长达50多页,证明了这类数的多种性质。尽管哈代本人并不喜欢这一研究方向,他仍评价说,虽然该研究涉及的是他称之为“数学的支流”领域,拉马努金在其中展现了“对不等式代数的非凡掌握力”。

1917年12月6日,拉马努金被选为伦敦数学学会会员。1918年5月2日,他当选为英国皇家学会院士,成为继1841年阿达西尔·柯塞吉之后第二位获此殊荣的印度人。当时年仅31岁的他,也成为皇家学会历史上最年轻的院士之一。他的当选理由是“在椭圆函数和数论方面的研究成果”。同年10月13日,他又成为剑桥大学三一学院的院士,成为首位获此殊荣的印度人。

