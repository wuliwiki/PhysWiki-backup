% ic 集成电路
% license CCBYSA3
% type Wiki

(本文根据 CC-BY-SA 协议转载自原搜狗科学百科对英文维基百科的翻译)

\begin{figure}[ht]
\centering
\includegraphics[width=6cm]{./figures/41b424041814090c.png}
\caption{可擦除可编程只读存储器(EPROM) 集成电路。这些封装具有透明窗口,显示内部的管芯。该窗口用于通过将芯片暴露于紫外光来擦除存储器。} \label{fig_icJCDL_1}
\end{figure}

\textbf{ic集成电路}或者\textbf{单片集成电路}(也称为\textbf{集成电路},\textbf{芯片},或\textbf{微芯片})是一组位于一小片(或“芯片”)半导体材料(通常为硅)上的电子电路。将大量微小的晶体管集成到一个小芯片中使得电路比由分立的电子元件构成的电路小几个数量级、更快、更便宜。集成电路的批量生产能力、可靠性和积木式方法的电路设计使得采用标准化集成电路迅速地取代了使用分立晶体管的设计。集成电路现在几乎被用于所有电子设备,并且已经彻底改变了电子学的世界。计算机、移动电话和其他数字家用电器现在是现代社会结构中不可分割的部分,集成电路的小尺寸和低成本使其成为可能。

20世纪中期半导体器件制造的技术进步使集成电路变得实用。自从20世纪60年代问世以来,芯片的尺寸、速度和容量都有了巨大的进步,这是由越来越多的晶体管安装在相同尺寸的芯片上的技术进步所推动的。现代芯片在人类指甲大小的区域内可能有数十亿个晶体管晶体管。这些进展大致跟随摩尔定律,使得今天的计算机芯片拥有上世纪70年代早期计算机芯片数百万倍的容量和数千倍的速度。

集成电路相对于分立电路有两个主要优势:成本和性能。成本低是因为芯片及其所有组件通过光刻作为一个单元印刷,而不是一次构造一个晶体管。此外,封装集成电路比分立电路使用的材料少得多。性能之所以高,是因为由于集成电路元件体积小且非常接近,它们的切换速度快,功耗相对较小。集成电路的主要缺点是设计它们和制造所需的光掩模的成本高。这种高初始成本意味着集成电路只有在预计到高产量时才是实用的。

\subsection{术语}



\subsection{发明}



\subsection{进展}



\subsection{设计}



\subsection{类型}



\subsection{制造业}



\subsubsection{6.1 制造}



\subsubsection{6.2 封装}



\subsection{知识产权}



\subsection{其他发展}



\subsection{世代}



