% 实数的表示

\pentry{完备公理(戴德金分割)\upref{Cmplt}}

我们最熟悉的实数表示方式是十进制小数或者二进制数. 在这里, 我们可以借助实数公理而给予这些表示以严格的逻辑基础. 特别地, 这能够解释一个古老的问题: $0.\dot{9}$ 到底等不等于 $1$?

我们固定一个正整数$q>1$作为进位制的基底. 取$q=2$或$q=10$当然是最熟知的.

我们从如下非常简单的命题开始. 
\begin{lemma}{}
设实数$x>0$. 则有唯一一个整数$n\in\mathbb{Z}$和唯一一个$k\in\{1,2,...,q-1\}$, 使得$kq^n\leq x<(k+1)q^{n}$.
\end{lemma}

实际上, 整个正实数轴被划分成彼此不相交的区间$[kq^n,(k+1)q^{n+1})$; 这里$n$跑遍所有整数, $k$跑遍$1,2,...,q-1$. 因此正实数$x$只能属于所有这些区间之中的一个. 

将对应于$x$的这两个整数记为$n_1$和$k_1$. 则$x-k_1q^{n_1}$是非负实数. 如果它等于零, 那么无需续行, 否则可以对它继续应用上述命题, 而得到$n_2\in\mathbb{Z}$和$k_2\in\{1,2,...,q-1\}$, 使得$k_2q^{n_2}\leq x-k_1q^{n_1}<(k_2+1)q^{n_2}$.