% 三角函数(高中)
% keys 高中|三角函数
% license Usr
% type Tutor
\pentry{弧度制与任意角\nref{nod_HsAngl},几何与解析几何初步\nref{nod_HsGeBa},函数\nref{nod_functi},函数的性质\nref{nod_HsFunC},导数\nref{nod_HsDerv}}{nod_1829}
\begin{issues}
\issueDraft
\end{issues}

在初中阶段,三角函数通常是在直角三角形的背景下\aref{引入}{eq_HsGeBa_1}的。在学习时,相信读者尚未学习“函数”这一概念,自然仅仅将其作为名称接受,并未意识到它与数学上的函数有何关联。而现在,在了解了函数表示的是输入与输出之间的确定关系的基础上,回顾初中的学习内容,可以发现:无论直角三角形的边长如何变化,只要其中一个锐角固定,三边之间的任意两者的比例始终不变。这一现象正是函数关系的体现——每个角都对应着唯一的比例,这也解释了三角函数名称的由来。

三角函数是一个历史悠久的数学主题,它的独特性在于,它不仅直接关联于几何图形,同时也符合函数的数学定义。这种双重属性使其从一开始就展现出复杂性。然而,复杂性往往伴随着强大的应用能力——事实上,几乎所有的周期函数都可以用三角函数表示。这一特性催生了一门重要的数学分支——\textbf{\enref{傅里叶分析}{FSTri}(Fourier analysis)},它构成了现代电子信息技术、信号处理等领域的基础,建立了时间与频率之间的数学联系。

随着角度的推广,三角函数的定义不再局限于直角三角形,而是扩展到任意角,并引入弧度制,使角度能够以实数的形式表示,从而与当前学习的实数函数体系结合,使角度成为三角函数的自变量。尽管在几何问题中仍然使用角度作为标记,但在涉及函数视角的导数等运算时,通常采用弧度制。例如,在弧度制下,$\sin x$ 在 $x=0$ 处的导数为 $1$,这一性质使得在微积分中的求导运算更加简洁自然。

\subsection{任意角下的三角函数}

\begin{figure}[ht]
\centering
\includegraphics[width=10cm]{./figures/fc460d9041b1fc1b.png}
\caption{三角函数定义示意图} \label{fig_HsTrFu_5}
\end{figure}

我们取单位圆上一点 $P(u,v)$,令 $OP$ 与 $x$ 轴夹角为 $\alpha$,则
如果不理解可以参考\autoref{fig_HsTrFu_3} 。
\begin{definition}{三角函数}
\begin{itemize}
\item \textbf{正弦函数}
\begin{equation}
\displaystyle\sin \alpha = \frac{v}{r}~.
\end{equation}
\item \textbf{余弦函数}
\begin{equation}
\displaystyle\cos \alpha = \frac{r}{r}~.
\end{equation}
\item \textbf{正切函数}
\begin{equation}
\displaystyle\tan \alpha = \frac{v}{u}~.
\end{equation}
\end{itemize}
\end{definition}

易得,正弦函数和余弦函数的\textbf{定义域为全体实数},正切函数的定义域为 $\begin{Bmatrix}\alpha|\alpha \neq \frac{\pi}{2}+k\pi,k\in Z\end{Bmatrix}~.$

三角函数除以上介绍的三种,还包括\footnote{其实还包括两个已经被弃用的概念:

\textbf{正矢(versine)}\begin{equation}
\mathrm{versin }\alpha=\displaystyle{r-y\over r}~.
\end{equation}

\textbf{余矢(vercosine)}
\begin{equation}
\mathrm{covers }\alpha=\displaystyle{r-x\over r}~.
\end{equation}
它们在早期的三角函数表中出现,本意是避免因正弦或余弦值过小造成的误差,但随着计算机的发展,这两个三角函数由于与其他三角函数关系不那么密切逐渐被弃用。}:
\begin{definition}{三角函数}
\begin{itemize}
\item \textbf{余切函数}
\begin{equation}
\displaystyle\cot \alpha = \frac{u}{v}~.
\end{equation}
\item \textbf{正割函数}
\begin{equation}
\displaystyle\sec \alpha = \frac{r}{u}~.
\end{equation}
\item \textbf{余割函数}
\begin{equation}
\displaystyle\csc \alpha = \frac{r}{v}~.
\end{equation}
\end{itemize}
\end{definition}
\begin{figure}[ht]
\centering
\includegraphics[width=14.25cm]{./figures/e9ea4e779f1c67a4.png}
\caption{任意角的三角函数(锐角)} \label{fig_HsTrFu_3}
\end{figure}

\subsection{*同角三角函数的基本关系}

根据现在给出的三角函数定义,很容易得到下面三组同角三角函数的恒等关系:
\begin{itemize}
\item 倒数关系:
\begin{equation}\label{eq_HsTrFu_1}
\begin{split}
\tan \alpha \cdot \cot \alpha = 1\\
\sin \alpha \cdot  \csc \alpha = 1\\
\sec \alpha  \cdot \cos \alpha = 1
\end{split}~.
\end{equation}
\item 乘积关系:
\begin{equation}\label{eq_HsTrFu_2}
\begin{split}
\tan \alpha \cdot\cos \alpha= \sin \alpha\\
\sin \alpha \cdot\cot \alpha= \cos \alpha\\
\cos \alpha \cdot\csc \alpha= \cot \alpha\\
\cot \alpha \cdot\sec \alpha= \csc \alpha\\
\csc \alpha \cdot\tan \alpha= \sec \alpha\\
\sec \alpha \cdot\sin \alpha= \tan \alpha\\
\end{split}~.
\end{equation}
\item 平方关系:
\begin{equation}\label{eq_HsTrFu_3}
\begin{split}
\sin ^{2} \alpha + \cos ^{2}\alpha =1\\
\tan  ^{2} \alpha + 1 =\sec ^{2}\alpha\\
1 + \cot ^{2}\alpha =\csc ^{2}\alpha\\
\end{split}~.
\end{equation}
\end{itemize}
上面的公式太多了,不好记怎么办?有人总结了如\autoref{fig_HsTrFu_4} 所示的方法来辅助记忆:
\begin{itemize}
\item 六边形对角线上的函数互为倒数。
\item 六边形顶点上的函数等于相邻两顶点乘积。
\item 三个倒立的三角(黄色标记)上方两顶点的平方和等于下方顶点。
\end{itemize}
\begin{figure}[ht]
\centering
\includegraphics[width=12cm]{./figures/6390d1e662067a9b.png}
\caption{同角三角函数的基本关系} \label{fig_HsTrFu_4}
\end{figure}

在上面的同角三角函数的恒等关系中,可以发现同一个数值往往有多种不同的表达方式。例如,$1$ 既可以表示为 $\sin^2\alpha+\cos^2\alpha$,也可以表示为 $\sec^2\alpha-\tan^2\alpha$。这种性质表明,由三角函数构成的表达式与代数中熟悉的幂表达式不同,其表示形式通常并不唯一。这种灵活性是三角函数的一个重要特点,也是高中阶段经常需要处理的问题之一。

其实,由于高中阶段不涉及$\cot,\sec,\csc$,因此,上面的关系中,常常使用的只有:

\begin{gather}
\sin ^{2} \alpha + \cos ^{2}\alpha =1~.\\
\tan \alpha= \frac{\sin \alpha}{\cos \alpha}~.
\end{gather}

前者是大名鼎鼎的勾股定理,后者则是初中时学习的$\tan\alpha$的定义。这里之所以给出这么多,一则是为了知识的完整,二则是在一些推导过程中,利用恒等关系可以化简避免错误。如果觉得记忆困难,完全可以放弃,提升熟练度一样可以。



\subsection{三角函数的性质}

\subsubsection{周期性}
正弦函数、余弦函数、正切函数的都是周期函数,根据定义易得,正弦函数和余弦函数,周期为 $2k\pi(k\in Z,k\neq0)$,正切函数的周期为 $k\pi(k\in Z,k\neq0)$.
\subsubsection{导数}

\subsubsection{图像}

\begin{figure}[ht]
\centering
\includegraphics[width=14.25cm]{./figures/bb986656153a547d.png}
\caption{正弦函数} \label{fig_HsTrFu_1}
\end{figure}
\begin{figure}[ht]
\centering
\includegraphics[width=14.25cm]{./figures/d0ae1e0d9c167f1e.png}
\caption{余弦函数} \label{fig_HsTrFu_2}
\end{figure}
可以看出正弦函数和余弦函数是定义域为 $R$ 值域为 $[-1,1]$ 最小正周期 $T = 2\pi$ 的周期函数。
