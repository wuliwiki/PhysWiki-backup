% 电磁场的作用量
% keys 电磁场|作用量|拉氏量

\pentry{电磁场中粒子的拉氏量\upref{ElecLS}, 经典场论基础\upref{classi}}

我们继续使用自然单位制,令 $\mu_0=\epsilon_0=c=1$ 来简化表达.依照习惯,上下标使用希腊字母如 $\mu, \nu$ 时,取值范围为 $\{0, 1, 2, 3\}$;使用拉丁字母如 $i, j$ 时,取值范围为 $\{1, 2, 3\}$.

\subsection{自由电磁场的作用量}
我们先考虑没有电流电荷密度分布的时空,仅考虑自由电磁场的作用量.我们要求电磁场的作用量满足洛伦兹不变性,是个洛伦兹标量,它可以由\textbf{电磁场张量}\upref{EMFT}来构造:
\begin{equation}\label{ElecS_eq1}
S=\int {\mathcal L} {\dd}^4 x=-\frac{1}{16\pi}\int F^{\mu\nu}F_{\mu\nu}{\dd}^4 x 
\end{equation}
其中 $\mathcal L=-\frac{1}{16\pi}F^{\mu\nu}F_{\mu\nu}$ 为自由电磁场的拉氏密度.由于 $F^{\mu\nu}=\partial^\mu A^\nu -\partial^\nu A^\mu$,对整个四维时空上的 $A^\mu$ 做一个变分 $\delta A^\mu$,
根据最小作用量原理,$\delta S=0$.由此可以推出自由电磁场的变化方程
\begin{equation}
\begin{aligned}
\delta S&=-\frac{1}{16\pi}\int (\delta F^{\mu\nu}F_{\mu\nu}+F^{\mu\nu}\delta F_{\mu\nu}){\dd}^4 x\\
&=-\frac{1}{16\pi}\int (2\delta (\partial^\mu A^\nu-\partial^\nu A^\mu)) F_{\mu\nu}){\dd}^4 x\\
&=-\frac{1}{4\pi}\int F_{\mu\nu}\partial^\mu \delta A^\nu {\dd}^4 x\\
&=\frac{1}{4\pi}\int \delta A^\nu\partial^\mu F_{\mu\nu} {\dd}^4 x \\
&=0\\
\Rightarrow &\partial_\mu F^{\mu\nu}=0
\end{aligned}
\end{equation}
上面的推导过程中用了分部积分,需要限制 $\delta A^\nu$ 在足够大的边界上消失.由最小作用量推出的 $\partial_\mu F^{\mu\nu}=0$ 对应着麦克斯韦方程组中的
\begin{equation}
\nabla\cdot \bvec E=0,\nabla\times \bvec B-\frac{\partial \bvec E}{\partial t}=0
\end{equation}
麦克斯韦方程组中的另两个方程 $\nabla\times \bvec E+\frac{\partial \bvec B}{\partial t}=0,\nabla\cdot \bvec B=0$ 已经蕴含在了 $F^{\mu\nu}=\partial^\mu A^\nu-\partial^\nu A^\mu$ 中\footnote{由$F^{\mu\nu}=\partial^\mu A^\nu-\partial^\nu A^\mu$ 可以推出 Bianchi 恒等式:$\partial_{\mu}F_{\nu\rho}+\partial_{\nu}F_{\rho\mu}+\partial_{\rho}F_{\mu\nu}=0$,由此可以导出另两个麦克斯韦方程.}.

\subsection{电磁场的作用量}
现在考虑带电粒子与电磁场的相互作用,需要引入场源的作用量\autoref{ElecLS_eq4}~\upref{ElecLS}.我们设想电荷是连续分布在空间的,为此引入电荷密度 $\rho(x^\mu)$,电流密度 $\bvec J(x^\mu)$.那么单个带电粒子的 $q(\dd x^\mu/\dd t)$ 就相当于 $\rho (\dd x^\mu / \dd t) {\dd} ^3 x=J^\mu {\dd}^3 x$,其中 $J^\mu=(\rho,J_x,J_y,J_z)$.电磁场中粒子的作用量可以改写为(略去质量项)
\begin{equation}
S=\int A_\mu J^\mu {\dd}^3 x\dd t=\int A_\mu J^\mu {\dd}^4 x
\end{equation}
再引入自由电磁场的作用量\autoref{ElecS_eq1} ,就可以得到电磁场的作用量的完整形式:
\begin{equation}
S=\int \qty(-\frac{1}{16\pi}F^{\mu\nu}F_{\mu\nu}+A_\mu J^\mu ){\dd}^4 x
\end{equation}

考虑固定四维时空上的 $J^\mu$ 不变,对 $A_\mu$ 做一个变分 $\delta A_\mu$,由最小作用量原理,$\delta S=0$,由此可以推出有场源的情况下的麦克斯韦方程组:
\begin{equation}
\begin{aligned}
\delta S&=\frac{1}{4\pi}\int (\partial^\mu F_{\mu\nu})\delta A^\nu {\dd}^4 x +\int J^\mu \delta A_\mu {\dd}^4 x\\
&=\frac{1}{4\pi}\int (\partial_\mu F^{\mu\nu}+4\pi J^\nu)\delta A_\nu {\dd}^4 x\\
\Rightarrow &\partial_\mu F^{\mu\nu}=-4\pi J^\nu
\end{aligned}
\end{equation}
由最小作用量原理推出的 $\partial_\mu F^{\mu\nu}=-4\pi J^\nu$ 对应着麦克斯韦方程组中的
\begin{equation}
\nabla \cdot \bvec E=4\pi \rho,\nabla\times \bvec B-\frac{\partial \bvec E}{\partial t}=4\pi \bvec J
\end{equation}
\subsection{电动力学的守恒量}
\addTODO{用诺特定理推电磁场的守恒量}
根据经典场论\upref{classi} 中能动张量 ${T^\mu}_\nu$ 的定义
\begin{equation}
T^\mu{}_\nu \equiv \frac{\partial \mathcal L}{\partial (\partial_\mu \phi)} \partial_\nu \phi - \mathcal L \delta^\mu{}_\nu
\end{equation}
可以写出电磁场的能动张量
\begin{equation}
T^\mu{}_\nu =\frac{\partial \mathcal L}{\partial (\partial_\mu A^\rho)} \partial_\nu A^\rho - \mathcal L \delta^\mu{}_\nu
\end{equation}
根据诺特定理,能动张量对应着四个守恒流:
\begin{equation}
\partial_\mu T^\mu{}_\nu=0
\end{equation}
其中 $T^0{}_0$ 对应着哈密顿量密度,或者说电磁场的能量密度;$T^0{}_i$ 对应着电磁场的动量密度.