% C++ 多线程笔记

\begin{itemize}
\item \verb|#include <thread>| \verb|#include <mutex>|
\item \verb|std::thread th(函数, arg1, arg2, ...)|; 创建一个线程, 调用\verb|函数|(可以是函数指针, 函数对象, lambda), \verb|arg| 是\verb|函数|的变量.
\item \verb|th.join()| 可以让主程序等待某个线程自己退出.
\item 声明一个全局变量 \verb|std::mutex m| 相当于 openmp 的 atomic 操作, 可以避免多个线程操作同意数据. \verb|m.lock()| 给 mutex 上锁, 如果已经被别人上锁就暂停并等待解锁. \verb|m.try_lock()| 试图上锁, 如果已经被别人上锁就返回 \verb|false|. \verb|m.unlock()| 解锁.
\item 为了避免在 \verb|m.lock()| 和 \verb|m.unlock()| 之间发生 throw, 通常不直接调用他们, 而是用 \verb|std::lock_guard<std::mutex> guard(m)| (相当于 lock), 当该变量被 destroy 时, 会自动 unlock.
\item \verb|std::this_thread::sleep_for(std::chrono::seconds(2));| 可以让某个线程暂停.
\end{itemize}
