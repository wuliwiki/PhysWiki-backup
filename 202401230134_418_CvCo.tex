% 约定式提交
% keys 约定式提交
% license Usr
% type Wiki

\begin{issues}
\issueDraft
\issueTODO
\end{issues}

约定式提交规范是一种在进行代码变更时的轻量级约定。约定式提交制定了一组简单的规则来创建清晰的提交历史,便于编写自动化工具。通过在提交信息中描述功能、修复和\textbf{破坏性变更}(Breaking Change)(下文会对破坏性变更做出解释),使这种惯例与语义化版本相互对应。

对于每次提交的代码变更,说明结构:
\begin{lstlisting}[language=none]
<type>[optional scope]: <description>

[optional body]

[optional footer(s)]
\end{lstlisting}
可以对应翻译为,
\begin{lstlisting}[language=none]
<类型>[(可选)范围]: <描述>

[(可选)正文]

[(可选)脚注]
\end{lstlisting}
提交说明描述了下列内容,可以向使用者与协同编辑者说明本次变更的内容:
\begin{enumerate}
\item \verb`fix`:这代表本次提交修复了一些 bug(对于修复多个 bug,建议多次提交)。这对应语义化版本的 \verb`PATCH`。
\item \verb`feat`:这代表本次提交实现了新的功能(对于实现多个功能,也建议多次提交)。这对应语义化版本的 \verb`MINOR`。
\item \verb`BREAKING CHANGE`:一般在类型后面加感叹号表示本次提交是一个破坏性变更,表示引入了\textbf{破坏性 API 变更},破坏性变更可以是任意类型的提交的一部分。这对应语义化版本的 \verb`MAJOR`。可以将破坏性变更理解为,进行了某些重大更新,使旧版本不再向上兼容。破坏性变更应在jiao'z
\item 除了 \verb`fix(!)` 与 \verb`feat(!)` 以外,类型也可以是 \href{https://github.com/conventional-changelog/commitlint/tree/master/%40commitlint/config-conventional}{@commitlint/config-conventional} 中推荐的任意一个等。
\end{enumerate}
