% 电荷在磁场中的圆周运动(电磁学入门)
% license Pub
% type Note


\begin{figure}[ht]
\centering
\includegraphics[width=10 cm]{./figures/1bd733e832e8dca7.pdf}
\caption{电荷在磁场中的圆周运动示意图} \label{fig_CBM_1}
\end{figure}

电荷在磁场中的圆周运动是一类经典问题。根据圆周运动以及磁场力始终垂直于速度方向的特性 ($\bvec F = q\bvec v \times \bvec B$),我们可以做出如下推论:
\begin{itemize}
\item 运动圆心在垂直入射速度的直线上
\item 出射速度沿轨迹的切线方向,同时垂直于与圆心的连线
\item 圆周运动中,电荷速度方向改变,但速率从不改变
\end{itemize}

此外,我们还有几个重要公式:
\begin{itemize}
\item $R=\frac{mv}{qB}~.$ 运动半径与初速度有关
\item $T=\frac{2\pi m}{qB}~.$ 运动周期与初速度无关
\item $t = \frac{\theta}{2\pi} T\text{(弧度制)}= \frac{\theta}{360^\circ } T\text{(角度制)}~.$ 运动所需时间与轨迹的圆心角有关
\end{itemize}
其中,$R$是圆周运动半径,$m$是电荷质量,$v$是电荷入射速率,$q$是电荷带电量,$B$是磁场强度,$T$是周期(电荷在磁场中完整运动一圈所需的时间),$t$是运动时长(电荷从入射到出射所需时间),$\theta$是圆心角。

相应的推导也很简单:
$$
\left \{
\begin{aligned}
F&=m\frac{v^2}{R}\\
F&=qvB\\
\end{aligned}
\Rightarrow
R=\frac{mv}{qB}
\right.
~,
$$
$$
\left \{
\begin{aligned}
v=R\cdot\omega\\
\omega = \frac{2\pi}{T}\\
R=\frac{mv}{qB}
\end{aligned}
\Rightarrow
R=\frac{mR\cdot\omega}{qB}=\frac{mR\cdot\frac{2\pi}{T}}{qB}
\Rightarrow
T=\frac{2\pi m}{qB}
\right.
~.
$$
