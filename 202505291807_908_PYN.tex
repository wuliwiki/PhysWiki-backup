% 皮亚诺公理(综述)
% license CCBYSA3
% type Wiki

本文根据 CC-BY-SA 协议转载翻译自维基百科 \href{https://en.wikipedia.org/wiki/Peano_axioms}{相关文章}。

在数学逻辑中,皮亚诺公理(Peano axioms,/piˈɑːnoʊ/,[peˈaːno]),也称为德德金–皮亚诺公理或皮亚诺假设,是一组用于描述自然数的公理体系,由19世纪意大利数学家朱塞佩·皮亚诺提出。这些公理在多项元数学研究中几乎未加改动地被广泛采用,其中包括关于数论是否一致与完备等基本问题的研究。

由皮亚诺公理所提供的算术公理化体系,通常被称为皮亚诺算术。

将算术进行形式化的重要性,在赫尔曼·格拉斯曼的工作之前并未被广泛重视。格拉斯曼在19世纪60年代指出,算术中的许多事实可以从关于后继运算和数学归纳的更基本事实中推导出来。1881年,查尔斯·桑德斯·皮尔士提出了一种自然数算术的公理化体系。1888年,理查德·德德金又提出了另一种自然数的公理体系,而皮亚诺则在1889年出版的著作《以新方法阐述的算术原理》(拉丁文:Arithmetices principia, nova methodo exposita)中,将其加以简化并作为一组公理发表。

皮亚诺公理共包括三类陈述:1. 第一条公理断言自然数集合中至少存在一个元素;2. 接下来的四条是关于等号的一般性陈述,在现代的处理方式中,这些往往被看作是“基础逻辑”的一部分,而非皮亚诺公理本身;3. 再接下来的三条公理是关于自然数及其后继运算的基本性质的一阶逻辑陈述;4. 第九条、也是最后一条公理则是一个二阶逻辑陈述,它表达了对自然数进行数学归纳法的原理,正是这一点使得原始皮亚诺公理体系接近二阶算术。如果显式引入加法和乘法两个运算符号,并将第九条二阶归纳公理替换为一阶公理模式,就可以得到一个较弱的一阶系统。术语“皮亚诺算术”有时特指这一限制后的系统。
\subsection{历史上的二阶表述}
当皮亚诺提出他的公理时,数理逻辑的语言还处于起步阶段。他为了表达这些公理而创造的逻辑符号体系并未广泛流行,尽管它成为现代集合成员关系符号(∈,源自皮亚诺的 ε)的起点。皮亚诺明确区分数学符号和逻辑符号,这在当时的数学中还不常见;这种区分最早由哥特洛布·弗雷格在其 1879 年出版的《概念文字》中引入。\(^\text{[7]}\)然而,皮亚诺并不知道弗雷格的工作,而是独立地基于布尔和施罗德的研究重建了自己的逻辑体系。\(^\text{[8]}\)

皮亚诺公理定义了自然数的算术性质,通常表示为集合 $N$ 或 $\mathbb{N}$。这些公理中的非逻辑符号包括一个常数符号 0 和一个一元函数符号 S(表示“后继”)。

第一条公理声明常数 0 是一个自然数:

1.0 是一个自然数。

皮亚诺最初在其公理表述中使用 1 而非 0 作为“第一个”自然数,\(^\text{[9]}\)而他在《数学公式集》中的公理则包括了零。\(^\text{[10]}\)

接下来的四条公理描述了等同性关系。由于这些内容在带有等号的一阶逻辑中是逻辑有效的,因此在现代的数学处理中,它们通常不被视为“皮亚诺公理”的一部分。\(^\text{[8]}\)\\\\
1.对于每一个自然数 $x$,有 $x = x$。即:等同性是自反的。\\
2.对于所有自然数 $x$ 和 $y$,如果 $x = y$,那么 $y = x$。即:等同性是对称的。\\
3.对于所有自然数 $x$、$y$ 和 $z$,如果 $x = y$ 且 $y = z$,那么 $x = z$。即:等同性是传递的。\\
4.对于所有 $a$ 和 $b$,如果 $b$ 是自然数并且 $a = b$,那么 $a$ 也是自然数。即:自然数在等同性下是封闭的。\\

其余的公理定义了自然数的算术性质。假设自然数在一个单值“后继”函数 $S$ 下是封闭的:\\\\
5.对于每一个自然数 $n$,$S(n)$ 也是自然数。即:自然数在后继函数 $S$ 下是封闭的。\\
6.对于所有自然数 $m$ 和 $n$,如果 $S(m) = S(n)$,那么 $m = n$。即:$S$ 是一个单射(注入函数)。\\
7.对于每一个自然数 $n$,命题 $S(n) = 0$ 是假的。即:不存在一个自然数,其后继是 0。
\begin{figure}[ht]
\centering
\includegraphics[width=10cm]{./figures/d58cef55c1c7abc7.png}
\caption{右图中从最近的一块开始的一连串浅色多米诺骨牌,可以用来表示自然数集合 $\mathbb{N}$。\(^\text{[1][11][12]}\)然而,公理 1–8 同样也被“所有多米诺骨牌的集合”(无论是浅色还是深色)所满足。\(^\text{[2]}\)第九条公理(归纳公理)将 $\mathbb{N}$ 限定为那一串浅色骨牌(即“无赘物”原则),因为只有浅色骨牌会在最前一块被推倒时依次倒下。\(^\text{[13]}\)指用继承关系(S)一块一块地排列这些骨牌。指如果不加第九公理,也可能包含不属于自然数的“赘余元素”。} \label{fig_PYN_1}
\end{figure}
公理 1、6、7 和 8 定义了自然数直观概念的一种一元表示方式:数字 1 可以被定义为 $S(0)$,2 为 $S(S(0))$,以此类推。然而,如果将自然数的概念完全建立在这些公理之上,则公理 1、6、7、8 并不蕴含后继函数能够生成所有非零的自然数。

我们直观上认为每一个自然数都可以通过对 0 反复应用后继函数而得到,这一想法需要额外添加一条公理来保证,这条额外的公理通常被称为归纳公理。

9.如果 $K$ 是一个集合,满足以下条件:

\begin{itemize}
\item $0 \in K$,并且
\item 对于每一个自然数 $n$,若 $n \in K$,则 $S(n) \in K$,
\end{itemize}

那么 $K$ 包含所有的自然数。

归纳公理有时也以如下形式表述:

9.如果 $\varphi$ 是一个一元谓词,满足:

\begin{itemize}
\item $\varphi(0)$ 为真,且
\item 对于每一个自然数 $n$,若 $\varphi(n)$ 为真,则 $\varphi(S(n))$ 也为真,
\end{itemize}
那么对于每一个自然数 $n$,$\varphi(n)$ 都为真。

在皮亚诺的最初表述中,归纳公理是一个二阶公理。而现在通常用一个较弱的一阶归纳公理模式来替代这个二阶原理。二阶与一阶的表述之间存在重要区别,这些区别将在下文《皮亚诺算术作为一阶理论》部分中进一步讨论。
