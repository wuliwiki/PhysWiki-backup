% 集合的测度(实变函数)
% 测度|实变函数|measure|集合|广义函数|泛函|可测集|Borel集

\pentry{集合\upref{Set},微积分或数学分析}

%放在实变函数还是广义函数里?或者广义函数本身作为实变函数的一个部分?

本节为实变函数的一部分,因此讨论中涉及的集合均为实数集合$\mathbb{R}^n$或者其子集.


\subsection{从Riemann积分到Lebesgue积分}

\pentry{黎曼积分与勒贝格积分\upref{Rieman}}

在微积分或数学分析中,介绍积分的时候通常都是指Riemann积分.Riemann积分的思路是\textbf{对定义域作分划},在每个分划区间里取一个代表的函数值作为“高度”,分划区间作为“底部”,构成许多“柱子”,计算这些柱子的“体积”并求和.这种思路的好处是逻辑上容易处理,只需要有极限的概念就能讨论清楚怎么积分.但是它有很多局限性,比如严重依赖函数的连续性,导致理论不完备,无法处理很多例外情况.比较典型的例子有Riemman函数和Dirichlet函数.

Lebesgue积分就是换了一种积分的思路,反过来\textbf{对值域作分划},计算各函数值对应的自变量集合的“面积”,以此来计算“柱子体积”并求和.上面提到的Riemann积分处理不了的病态函数,就可以用Lebesgue积分来处理,并且对于Riemann积分能处理的函数,两种积分算出来的结果是一样的.

考虑Riemann函数,函数值为整数$q$的点是$[0, 1]$上的全体形如$p/q$的既约真分数,记它们构成的集合为$A_q$.现在知道柱子的高度是$q$了,计算柱子的体积还需要底面积,也就是$A_q$的“面积”.从这里就能看出Lebesgue积分的特别之处,即需要考虑非开区间的集合的“面积”.

集合的“面积”,被称为“\textbf{测度(measure)}”.本节我们就来讨论测度的概念.



\subsection{点集的测度}

\subsubsection{开集的测度}

\begin{definition}{“开方块”的体积}\label{SetMet_def1}

设$I=I_1\times I_2\times I_3\times\cdots\times I_n$是$\mathbb{R}^n$上\textbf{开区间}的笛卡尔积,其中$I_i=(x_i, y_i)$,那么定义$I$的体积为$\abs{I} = \prod_{i=1}^n \abs{y_i-x_i}$.

\end{definition}

$\mathbb{R}^n$的任意开集都能表示为形如\autoref{SetMet_def1} 中$I$的“开方块”的\textbf{不交并},因此开集的体积就可以直接定义为这些开方块的体积之和.

开集的外测度具有如下性质:

\begin{enumerate}
\item \textbf{次可加性}:对于任意两个开集$U_1$和$U_2$,必有$\abs{U_1\cup U_2}\leq \abs{U_1}+\abs{U_2}$.
\end{enumerate}


\subsubsection{任意点集的外测度}

有了开集的体积概念,我们就可以定义一种测度了.

\begin{definition}{外测度}

\end{definition}




















