% 华东师范大学 2009 年 考研 量子力学
% license Usr
% type Note

\textbf{声明}:“该内容来源于网络公开资料,不保证真实性,如有侵权请联系管理员”

\subsection{简答题(每小题6分,共60分)}
\begin{enumerate}
\item 某能量为$E$和动量为$P$的非相对论粒子,问与该粒子对应的物质波的频率和波长。
\item 某能级是三重简并的,若已知体系处在该能级的某个纯态上,试问是否原则上总可以通过某种实验确定体系所处的具体态?为什么?
\item 设某一维体系的哈密顿算符为$H=\frac{1}{2m}p^2+x^4$,其中$x$为位置算符,$p$为其共轭动量,$m$为粒子质量,试写出$p$随时间的演化方程。
\item 简述海森堡绘景与薛定谔绘景的主要差异。
\item 波函数$\varphi=A\phi_P+B\phi_R$,其中$A,B$为常数,$\phi_P,\phi_R$分别是体系的动量本征态和位置本征态,问:体系是否可以处在$\varphi$态上?为什么?
\end{enumerate}
