% 产生和湮灭
% keys 多粒子体系|全同粒子|Fock空间
\pentry{标量场的量子化\upref{quanti}}
\begin{issues}
\issueDraft
\end{issues}

%需要修改K-G场一节,增加复形式.本文里的产生湮灭算符全文替换,注意量子化后的场符号
旧称量子场论的工作是“二次量子化”,但这种描述是不准确的。不仅仅是因为两套体系几乎是同时发展的,更是因为在概念上,量子场论仅仅是对经典场论中的场进行一次量子化,使得场不再是描述幅值随时空变换的函数,而是能描述粒子产生和湮灭的算符。为了算符化,我们需要利用量子力学中的产生和湮灭算符。这是量子场论中的基本概念,也是构筑多粒子体系不可或缺的砖瓦。


\subsection{定义}

与谐振子问题一样,用$\ket{0}$来定义能级最低,可视作没有粒子的真空态。用$ a, a^\dagger$来湮灭和产生粒子,而不再是具有使谐振子能级爬升或者降低的意义。即
\begin{equation}
\begin{aligned}
a_{\vec p}\ket{0}&=0\\
a^\dagger_{\vec p}\ket{0}&=\ket{\vec p}
\end{aligned}
\end{equation}
也就是说,在场论里,增加或者湮灭粒子对应于谐振子里的能级爬升或者降低。粒子数增加或者减少对应能量量子数的增加或者减少。
在扩展理论的过程中,我们需要格外留心原本理论中不变量的适用条件。在量子力学里,为了符合物理现实和数学理论,可观测量,比如力学算符本征值或者传播振幅是不随“参考系”而改变的,这里的参考系是不同表象或者绘景。然而量子场论里往往需要研究的是洛伦兹不变量,原本的$\braket{\vec q}{\vec p}$并不是洛伦兹不变量。

可以证明,洛伦兹不变测度$\int\mathrm d^4 p=\int\frac{\mathrm d^3 p}{2E_{\vec p}}$,因而$2E_{\vec p}\delta^3 (\vec p-\vec q)$才是洛伦兹不变量。于是在场论里,不变"传播振幅"可修改为
\begin{equation}
\braket{ q}{ p}=(2 \pi)^3 2 E_{\vec q}\delta^3 (\vec q-\vec p)~,
\end{equation}
其中$\ket{p}=\sqrt{2 E_\vec p}\ket{\vec p}$。
\subsection{因果律与对易关系}
在没有产生和湮灭算符之前,场方程的通解是函数的线性叠加。比如众所周知的$K-G$方程,其齐次形式为
$$(\partial_\mu\partial^\mu+m^2)\phi=0~,$$
该方程并不排斥负能解,所以我们可以把通解拆成两部分,
$$\phi(x)=\sum_pC(p)\mathrm e ^{\mathrm {i} px}+\sum_{-p}C(-p)\mathrm e ^{-\mathrm {i} px}~,$$
(注意,上式出现的矢量内积为四维形式。)
于是,在引入产生和湮灭算符后,我们可以把上式中的系数替换为算符,并把线性叠加修改为连续动量谱下的积分,积分测度需是洛伦兹不变。所以标量场量子化后为,
\begin{equation}
\phi (x)=\int\frac{\mathrm d^3 p}{(2\pi )^3}\frac {1}{2E_{\vec p}} a_\vec p\mathrm e^{-\mathrm ipx}+a_\vec p^\dagger\mathrm e^{\mathrm ipx}~,
\end{equation}
作用在态矢上,第一项表示湮灭正能粒子,第二项表示产生负能粒子(即反粒子),常系数是归一化系数。对于复$K-G$场,则为
\begin{equation}
\hat \phi (x)=\int\frac{\mathrm d^3 p}{(2\pi )^3}\frac {1}{2E_{\vec p}} a_\vec p\mathrm e^{-\mathrm ipx}+b_\vec p^\dagger\mathrm e^{\mathrm ipx}~,
\end{equation}




在量子化后,我们还可以得到这么一个东西:$[\hat{\phi}(x),\hat{\phi}^\dagger(y)]$,虽然暂时还不能得到具体值,但在直觉上,它需要满足类空下对易子为0——为了满足狭义相对论的因果律要求,类空下的算符必然是独立的,也就是说对类空的两个时间点进行测量,结果不受彼此影响,它们之间没有超光速的场传播。(虽然这个要求对场算符的限制不是很直观,但由于多粒子体系下的算符可以用场算符表示,我们可以证明,对于标量场,任意力学量算符类空下对易子为0,等价于对场算符的对易子要求)

现在我们来推导该对易子:
\begin{equation}
\begin{aligned}
[\hat{\phi}(x),\hat{\phi}^\dagger(y)]&=\int\frac{\mathrm d^3p}{(2\pi )^3}\frac {1}{\sqrt{2 E_{\vec p}}}\frac{\mathrm d^3p'}{(2\pi )^3}\frac {1}{\sqrt{2 E_{\vec p'}}}(\mathrm {e}^{\mathrm {i}py}\mathrm {e}^{-\mathrm {i}p'x}[a_{\vec {p'}},a^\dagger_{\vec p}]+ \mathrm e^{-\mathrm i py}\mathrm e^{\mathrm i p'x}[b^\dagger_{\vec {p'}},b_{{\vec p}}])\\
&=\int \frac {\mathrm d^3 p}{(2 \pi)^3}\frac {1}{2 E_p}(\mathrm e^{-\mathrm i p(x-y)}-\mathrm e^{\mathrm i p(x-y)})~,
\end{aligned}
\end{equation}
\begin{enumerate}
\item 可以看到,该对易子确实是洛伦兹不变的。
\item 当两个事件间隔是类空时,对易子为0.我们可以取$x_0=y_0=0$来保证类空。由于$p$取所有值,所以第二项中指数的负号不影响积分结果。总积分结果为0且洛伦兹不变。
\item 从计算过程中,我们可以发现,$[a_{\vec {p'}},a^\dagger_{\vec p}]$=,$[b_{\vec {p'}},b^\dagger_{\vec p}]=(2\pi)^3\delta(\vec p-\vec q)$保证了最后结果在类空间隔下为$0$。量子力学中的全同性原理导出了玻色子的对易关系,本质上来源于因果律的要求。对于旋量场,我们则采取费米子对易关系进行量子化。
\end{enumerate}
\subsection{Fock空间}
在量子力学里,我们对单粒子希尔伯特空间已经相当熟悉了。现在需要扩展到多粒子体系,最好能描述数量未知的多粒子系统。

假设我们现在有n粒子的玻色子系统。如果是2粒子,我们可以用对称化后的二粒子基矢的线性组合~$(\ket {k_1,k_2}+\ket {k_2,k_1})$来描述;三粒子亦然,任意n粒子态的基矢都是由n粒子希尔伯特空间的张量积构成。
\begin{align}
\ket 0 \longrightarrow \mathcal{H}^{(0)} \\
\ket k \longrightarrow \mathcal{H}^{(1)} \\
\ket {k_1,k_2}\longrightarrow \mathcal{H}^{(2)}\\
\mathcal{H}=\mathcal{H}^{(0)} \bigoplus \mathcal{H}^{(1)} \cdots \bigoplus \mathcal{H}^{(n)}~,
\end{align}
对于费米子体系,也是相应的构建思路,区别只是在对称化基矢上的搭建。
多粒子体系也有相应的完全性关系,

\subsection{多粒子体系的算符}
我们以离散谱为例,连续化的结果通过把求和替换为积分获得。

设$Q$为多粒子体系的总单体算符,对应的单粒子算符为$q$,那么总的测量结果为单粒子测量结果之和。引入粒子数算符$N(q_i)=a(q_i)^\dagger a(q_i)$,其中${q_i}$为单体算符的本征值谱,该粒子数算符统计处于该本征值的粒子数目。那么$Q=\sum q_i a(q_i)^\dagger a(q_i) $。产生和湮灭算符的定义是在动量表象的,现在需要得到$q$表象下的形式,利用真空态和完备性关系,我们有
\begin{equation}
\begin{aligned}
a^{\dagger}(p)\ket{0}&=\ket{p}=\sum^{q_i}\braket{q}{p}a^{\dagger}(q)\ket{0}\\
a(p)\ket{0}&=\ket{p}=\sum^{q}\braket{p}{q}a(q)\ket{0}~,
\end{aligned}
\end{equation}
因此,我们可以通过表象变换得到$Q$的最终形式,由于表象不影响观测结果,这里设$\lambda$为任意表象,易得
\begin{equation}
\begin{aligned}
Q=\sum q_i a^{\dagger}~,
\end{aligned}}
\end{equation}






双体算符
