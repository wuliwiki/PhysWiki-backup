% 小时百科符号与规范
% keys 符号|规范|单位制
% license Xiao
% type Tutor

这里列出小时百科中一些可能会产生歧义的符号以及规范, 并给出对应词条的链接, 如需补充, 请与管理员协商。

\subsection{数学}
\subsubsection{通用规范}
\begin{itemize}
\item 根据 ISO \href{https://en.wikipedia.org/wiki/ISO/IEC_80000}{国际标准}, 把\textbf{自然数}定义为非负整数($0,1,2\dots$)。 \textbf{非零自然数}定义为正整数($1,2,3,\dots$)。
\item 数域用双线字母表示: \textbf{实数} $\mathbb R$、 \textbf{复数} $\mathbb C$、 \textbf{有理数} $\mathbb Q$、 \textbf{整数} $\mathbb Z$、 \textbf{正整数(非零自然数)} $\mathbb Z^+$、 \textbf{负整数} $\mathbb Z^-$、 \textbf{自然数} $\mathbb N$、 \textbf{非负整数} $\mathbb Z^{0+}$、 \textbf{非正整数} $\mathbb Z^{0-}$、 \textbf{质数} $\mathbb Z_p$。
\item \textbf{定义}、\textbf{定理}等环境中的内容必须是严谨的, 否则不使用。 “证明” 也必须是严谨的, 否则用 “不严谨的证明” 或 “幼稚的证明”。
\item 较长的整数可以使用逗号对每三位进行分割, 如 $2,456,789$。
\item 集合\upref{Set} 中 $\subseteq$ 和 $\subset$ 表示\textbf{子集}, $\subsetneq$ 表示\textbf{真子集}, 但应该尽量避免使用 $\subset$。
\item 映射\upref{map} 中\textbf{单射}、\textbf{满射}、\textbf{双射}的定义使用\autoref{def_map_1}~\upref{map}。
\item \textbf{虚数单位}\upref{CplxNo} 用正体的 $\I$。
\item \textbf{四元数}中的三个虚数单位用 $\I, \mathrm{j}, \mathrm{k}$。
\item \textbf{自然对数底}\upref{E} 用正体的 $\E$。
\item 球坐标使用 “球坐标系\upref{Sph}” 中的定义。
\item 连带勒让德多项式\upref{AsLgdr} 和球谐函数\upref{SphHar}使用 Mathematica 或 Wolfram Alpha (二者相同)的定义, 即包含 Condon–Shortley 相位(\autoref{sub_SphHar_1}~\upref{SphHar})。 另见球谐函数表\upref{YlmTab}。
\item 空集符号用 \verb|\varnothing| $\varnothing$
\end{itemize}

\subsubsection{线性代数}
\begin{itemize}
\item 同义词: “矢量”、 “向量”\upref{GVec}; “矢量空间”、 “线性空间”\upref{LSpace}。 偏数学专业的内容尽量用 “向量”。
\item 同义词: “线性映射”、 “线性变换”、 “线性算符”\upref{LinMap}。
\item 使用\textbf{粗体加正体表示矩阵}\upref{Mat}(包括行矢量、列矢量)。 行向量或列向量一般用小写,其他矩阵一般用大写(如线性方程组记为 $\mat A \bvec x = \bvec b$)。 一个例外是当矩阵符号是希腊字母时, 使用粗斜体而不是粗正体(如 $\bvec \beta$)。 这与 Wikipedia 的规范一致(\href{https://en.wikipedia.org/wiki/Angular_acceleration}{例子})。 应该统一使用小时百科专门定义了命令 \verb|\bvec|。
\item 一般矢量空间中的矢量用普通标量的字体表示(如 $v$), 或区分矢量和坐标的不同。 这时内积\upref{InerPd}可以记为 $\ev{u, v}$。
\item 由于习惯原因, 在不需要与列向量区分的情况下\textbf{几何矢量}\upref{GVec}也可以采用粗体加正体(大小写均可, 如 $\bvec a, \bvec R$)。
\item 矩阵 $\mat M$ 的行列式表示为 $\opn{det}\mat M$ 或 $\abs{\mat M}$。
\item 当行矢量或列矢量需要声明基底时, 可以在外面加括号并用下表声明。 例如 $(\bvec v)_S = \pmat{x\\y\\z}_S$ 表示几何矢量 $\bvec v$ 在直角参考系 $S$ 中的坐标。 又例如 $(\bvec v)_{\qty{\uvec x_i}}$ 表示几何矢量 $\bvec v$ 关于基底 $\uvec x_1, \uvec x_2, \dots$ 的坐标。 具体例子如 “速度的参考系变换\upref{Vtrans}”。
\item 单位几何矢量在粗体与正体矢量上方加 $\hat{\phantom{x}}$ 表示, 如 $\uvec x$。
\item 几何矢量的点成(内积)记为 $\bvec u \vdot \bvec v$, 叉乘记为 $\bvec u \cross \bvec v$。
\item 矢量的坐标表示为实数或复数的笛卡尔积(\autoref{eq_Set_1}~\upref{Set}), 如 $\mathbb R^N, \mathbb C^N$ 或 $(c_1, c_2, c_3)$。 也可以表示为列向量(如 $\bvec c, \bvec x$)以便和矩阵相乘。
\item 为了排版美观, 列向量出现在正文中时可以记为行向量的转置如 $\pmat{1 & 2 & 3}\Tr$ 注意没有逗号。 注意与笛卡尔积区分(有逗号)。
\item 也可以用狄拉克符号表示任意矢量空间的矢量, 如 $\ket{v}$, 对偶矢量如 $\bra{v}$, 内积如 $\braket{u}{v}$。
\item eigenvector 和 eigenvalue 译作\textbf{本征矢}和\textbf{本征值}, 而不是特征矢和特征值。 characteristic 才译作\textbf{特征}。
\end{itemize}

\subsubsection{高等代数}
\begin{itemize}
\item 环不一定有幺元,将\textbf{环}和\textbf{幺环}区分开。
\end{itemize}

% \subsubsection{数学分析}

\subsubsection{微分几何}
\begin{itemize}
\item 小时百科中 “流形” 一词特指 “光滑流形”。
\item 同义词: “逆变向量(contravariant vectors)”、“切向量(tangent vectors)”、 “$(1, 0)$ 型张量”。
\item 同义词: “协变向量(covariant vectors)”、“余切向量(cotangent vectors)”、“切向量的对偶向量(dual vectors)”、 “线性映射”、 “线性1-形式(linear 1-forms)”、 “$(0, 1)$ 型张量”。
\item 同义词: “张量(tensors)”、“多重线性映射(multilinear maps)”。
\end{itemize}

\subsection{物理}
\subsubsection{通用规范}
\begin{itemize}
\item 如无声明默认使用最新国际单位制\upref{Consts}(2018 CODATA), 若使用其他单位制, 要在每个词条开头用脚注声明 “本文使用 xx 单位制”。 例如原子单位\upref{AU}, 厘米—克—秒\upref{CGS}, 或高斯单位制\upref{GaussU}, 自然单位制\upref{NatUni}。
\item 若要区分能量和电场, 用 $E$ 表示能量, $\mathcal E$ 表示电场。
\item 连带勒让德多项式\upref{AsLgdr} 和球谐函数\upref{SphHar}使用 Mathematica 或 Wolfram Alpha (二者相同)的定义, 也可以见球谐函数表\upref{YlmTab}。
\end{itemize}

\subsubsection{电动力学}
\begin{itemize}
\item \textbf{平面波(plane wave)}和\textbf{简谐波(sinusoidal wave)}是同义词, 或者\textbf{平面简谐波}更明确。
\item 把 $\bvec B$ 称为磁场\upref{MagneF}, 尽量不使用 “磁场强度” 和 “磁感应强度”。
\item 使用 $\epsilon$ 而不是 $\varepsilon$ 表示电介质常量。
\end{itemize}

\subsubsection{电路}
\begin{itemize}
\item 电路元件两端电压符号使用\textbf{被动符号规定}(\autoref{sub_Resist_1}~\upref{Resist}): 若电势延正方向降低,则电压为正,反之为负。
\end{itemize}
\addTODO{$\E^{-\I \omega t}$ 时间因子}

\subsubsection{量子力学}
\begin{itemize}
\item \textbf{平面波(plane wave)}和\textbf{简谐波(sinusoidal wave)}是同义词, 但尽量用前者。
\end{itemize}

\subsubsection{相对论}
\begin{itemize}
\item 用 $\gamma$ 表示 $1/\sqrt{1 - v^2/c^2}$。
\item 用 $\beta$ 表示 $v/c$。
\item 闵可夫斯基度规使用号差为 $2$ 的形式,即 $\opn{diag}(-1, 1, 1, 1)$。
\end{itemize}

\subsubsection{热力学}
\begin{itemize}
\item 压强使用大写 $P$。
\item 热量 $Q$ 以外界传给系统为正。
\item 做功 $\dd{W} = P\dd{V}$ 以系统对外为正。
\end{itemize}

\subsubsection{统计力学}
\begin{itemize}
\item $\Xi$ 表示配分函数。
\item $k_B$ 表示玻尔兹曼常数。
\item 如果温度使用能量单位, 意思是该温度(开尔文温标)乘以玻尔兹曼常数 $k_B T$。
\end{itemize}

\subsubsection{计算机}
\begin{itemize}
\item Linux 系统如无特殊说明使用 Ubuntu 22.04。
\item 对于某种网络问题, 说明如: “截至到 xxx 年 xxx 月,在中国大陆……的过程中可能会遇到某种网络问题”。如果能提供下载则提供,需要有校验和。
\end{itemize}
