% 序数算术(综述)
% license CCBYSA3
% type Wiki

本文根据 CC-BY-SA 协议转载翻译自维基百科\href{https://en.wikipedia.org/wiki/Ordinal_arithmetic}{相关文章}。

在数学中的集合论领域,序数算术描述了序数上的三种常见运算:加法、乘法和指数运算。每种运算基本上都可以通过两种不同的方式定义:一种是构造一个明确的良序集来表示运算结果;另一种是使用跨无限递归(transfinite recursion)来定义。康托范式(Cantor normal form)提供了一种标准化的序数表示方式。除了这些常见的序数运算之外,还有“自然”序数算术以及nimber运算。
\subsection{加法}  
两个良序集 \(S\) 和 \(T\) 的和是一个序数,它表示在直积集 \(S \times \{0\}\) 和 \(T \times \{1\}\) 的并集上定义的变体字典序(即最不重要的位置优先的字典序)。这种排序方式保证了以下几点:\(S\) 中的每个元素都小于 \(T\) 中的每个元素;\(S\) 内部的比较保持原来的顺序;\(T\) 内部的比较也保持原来的顺序。 

对于序数加法 \(\alpha + \beta\),也可以通过对\(\beta\)进行跨无限递归来定义:当右加数 \(\beta = 0\) 时,\(\alpha + 0 = \alpha\) 对任意 \(\alpha\)成立。当 \(\beta > 0\) 时,\(\alpha + \beta\)是所有\(\alpha + \delta\)(其中 \(\delta < \beta\))中最小的严格大于它们的序数。  

具体分成后继序数和极限序数的情况:
\begin{itemize}
\item \(\alpha + 0 = \alpha\)~
\item \(\alpha + S(\beta) = S(\alpha + \beta)\) 其中 \(S\) 表示后继函数。
\item \(\alpha + \beta = \bigcup_{\delta < \beta} (\alpha + \delta)\) 当 \(\beta\)是一个极限序数时。
\end{itemize}
在自然数上,序数加法与通常的加法是相同的。  
第一个超限序数是 \(\omega\),它是所有自然数的集合,接下来的序数是 \(\omega + 1\)、\(\omega + 2\) 等。  

\(\omega + \omega\) 表示两个按通常顺序排列的自然数副本,第二个副本完全排在第一个副本的右侧。  
用 \(0' < 1' < 2' < \dots\) 表示第二个副本,那么 \(\omega + \omega\) 看起来像:
\[
0 < 1 < 2 < 3 < \dots < 0' < 1' < 2' < \dots~
\]
这和 \(\omega\) 不同,因为在 \(\omega\) 中,只有 0 没有直接前驱,而在 \(\omega + \omega\) 中,0 和 0' 都没有直接前驱。
\subsubsection{性质}  
序数加法一般不是交换的。例如:\(3 + \omega = \omega\)因为 \(3 + \omega\) 的顺序关系是:\(0 < 1 < 2 < 0' < 1' < 2' < \dots\)
如果重新标记,就与 \(\omega\) 相同。而\(\omega + 3\)就不等于 \(\omega\),因为它的顺序关系是:\(0 < 1 < 2 < \dots < 0' < 1' < 2'\)其中 \(2'\) 是最大元素,而 \(\omega\) 没有最大元素。(\(\omega\) 和 \(\omega + 3\) 等势,但不是序同构的。)

序数加法是结合的。例如:\((\omega + 4) + \omega = \omega + (4 + \omega) = \omega + \omega\)

加法在右参数上是严格单调且连续的:
\[
\alpha < \beta \implies \gamma + \alpha < \gamma + \beta~
\]
但对于左参数不成立。对左参数,我们只有:
\[
\alpha < \beta \implies \alpha + \gamma \leq \beta + \gamma~
\]

序数加法具有左可消性:如果\(\alpha + \beta = \alpha + \gamma\) 
那么 \(\beta = \gamma\)。此外,对于满足 \(\beta \leq \alpha\) 的序数,可以定义左减法:存在唯一的 \(\gamma\) 使得 \(\alpha = \beta + \gamma\)。  

另一方面,右消去律则不成立:
但是,加法不满足右消去律:
\[3 + \omega = 0 + \omega = \omega \text{但}3 \neq 0~\]
即使在 \(\beta \leq \alpha\) 的情况下,右减法也不成立。例如,不存在任何 \(\gamma\) 满足:\(\gamma + 42 = \omega\)

如果小于 \(\alpha\) 的所有序数在加法下封闭且包含0,那么这种 \(\alpha\)有时被称为\(\gamma\)数(见:不可加分解序数)。这样的序数恰好是形如:\(\omega^\beta\)的序数。
\subsection{乘法}
\begin{figure}[ht]
\centering
\includegraphics[width=14.25cm]{./figures/ab2f43e19990cb97.png}
\caption{} \label{fig_Ordina_1}
\end{figure}
\begin{figure}[ht]
\centering
\includegraphics[width=14.25cm]{./figures/0f5d1a0373597675.png}
\caption{} \label{fig_Ordina_2}
\end{figure}
两个良序集 \(S\) 和 \(T\) 的笛卡尔积 \(S \times T\),可以通过一种变体字典序(即最不重要的位置优先的字典序)来良序化。  

实际上,这相当于把 \(T\) 中的每个元素替换为一个不相交的 \(S\) 副本。  

这个笛卡尔积的序型就是 \(S\) 和 \(T\) 的序型相乘所得到的序数。