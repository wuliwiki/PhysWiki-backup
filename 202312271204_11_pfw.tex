% 前推
% license Usr
% type Tutor
赋予流形以联络,便可以比较同一流形上不同点的切向量“大小”。那对于不同流形,我们能否找到一个定义去联系同一点上的不同切空间呢?

\begin{issues}
\issueDraft 缺图
\end{issues}

\subsubsection{前推}
\begin{definition}{前推}
设$M,N$都是光滑流形。对任意$p\in M,\,F:M\rightarrow N$为光滑映射。定义$F_*:T_p M\rightarrow T_{F(p)N}$为
\begin{equation}
(F_*X)(f)=X(f\circ F)~,
\end{equation}
称之为与$F$关联的\textbf{前推(push-forward)}。
\end{definition}
可记忆前推是顺着光滑映射的方向,对$M$上$p$点的切空间与$N$上$F(p)$点的切空间建立同构。在证明同构之前,我们需要先证明这样的定义是合理的。即$F_*X$确实是一个切向量,满足导子的性质:
\begin{equation}
\begin{aligned}
(F_*X)(fg)&=X((fg)\circ F)\\
&=X(f\circ F)(g\circ F)\\
&=~.
\end{aligned}
\end{equation}