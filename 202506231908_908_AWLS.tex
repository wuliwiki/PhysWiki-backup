% 埃瓦里斯特·伽罗瓦(综述)
% license CCBYSA3
% type Wiki

本文根据 CC-BY-SA 协议转载翻译自维基百科 \href{https://en.wikipedia.org/wiki/\%C3\%89variste_Galois}{相关文章}。

埃瓦里斯特·伽罗瓦(Évariste Galois,/ɡælˈwɑː/;法语发音:[evaʁist ɡalwa];1811年10月25日-1832年5月31日)是一位法国数学家和政治活动家。他在少年时期就成功找出了一个多项式是否可用根式求解的充要条件,从而解决了一个困扰数学界达350年的难题。他的工作奠定了伽罗瓦理论和群论的基础——这两个领域后来成为抽象代数的主要分支。

伽罗瓦是一位坚定的共和主义者,积极参与了围绕1830年法国革命的政治动荡。由于他的政治活动,他多次被捕,并服刑数月。出狱不久,他因某些至今仍不明的原因参加了一场决斗,并因伤重身亡。
\subsection{生平}
\subsubsection{早年生活}
\begin{figure}[ht]
\centering
\includegraphics[width=6cm]{./figures/709d53b4b96fb0d8.png}
\caption{} \label{fig_AWLS_1}
\end{figure}
伽罗瓦于1811年10月25日出生于尼古拉-加布里埃尔·伽罗瓦和阿德莱德-玛丽(Adélaïde-Marie,娘家姓德芒特,Demante)夫妇之家。\(^\text{[2][4]}\) 他的父亲是一位共和主义者,是布尔拉雷讷自由党派的领袖。1814年路易十八复辟后,他的父亲成为该村的市长。\(^\text{[2]}\)他的母亲是一位法学家的女儿,精通拉丁语和古典文学,并负责伽罗瓦前十二年的教育。

1823年10月,伽罗瓦进入路易大帝中学,他的老师路易·保罗·埃米尔·理查德识别出了他的非凡才华。\(^\text{[5]}\)14岁时,他开始对数学产生浓厚兴趣。\(^\text{[5]}\)

伽罗瓦找到了一本阿德里安-玛丽·勒让德的《几何原本》,据说他“像读小说一样”阅读这本书,并在第一次阅读时就掌握了其内容。15岁时,他已经在阅读约瑟夫-路易·拉格朗日的原始论文,如《代数方程解法的思考》,这很可能激发了他后来在方程理论方面的研究,\(^\text{[6]}\)以及《函数运算讲义》——这是专为专业数学家撰写的著作。然而,他在课堂上的表现并不出色,老师还指责他摆出一副天才的架势。\(^\text{[4]}\)
\subsubsection{初露锋芒的数学家}
1828年,伽罗瓦参加了法国当时最负盛名的数学学府——综合理工学院的入学考试,但由于缺乏数学方面的常规准备,在口试时未能充分说明自己的解题过程,因此落榜。同年,他进入了当时数学研究水平较低的正规师范学院(École Normale,当时称为“预备学院”l’École préparatoire),在那里他遇到了一些对他持同情态度的教授。[7]
\begin{figure}[ht]
\centering
\includegraphics[width=6cm]{./figures/ef1f974ae15f836d.png}
\caption{} \label{fig_AWLS_2}
\end{figure}
次年,伽罗瓦的第一篇论文《关于简单连分数》发表了。\(^\text{[8]}\)大约在同一时期,他开始在多项式方程理论方面取得根本性发现。他将两篇关于这一主题的论文提交给了科学院。著名数学家奥古斯丁-路易·柯西对这些论文进行了评审,但出于至今仍不明确的原因,拒绝将其发表。

尽管有许多相反的说法,广泛观点认为柯西确实认识到了伽罗瓦工作的重大意义,他只是建议将这两篇论文合并为一篇,以便参加科学院设立的数学大奖评选。柯西当时是享有盛誉的数学家,尽管在政治立场上与伽罗瓦截然相反,但他认为伽罗瓦的成果极有可能获奖。\(^\text{[9]}\)

1829年7月28日,伽罗瓦的父亲在与当地神父的一场激烈政治争执后自杀身亡。\(^\text{[10]}\)几天后,伽罗瓦第二次也是最后一次尝试报考巴黎综合理工学院(École Polytechnique),但再次落榜。\(^\text{[10]}\)毋庸置疑,伽罗瓦的能力远远超过入学标准;关于他落榜的原因,众说纷纭。较为可信的说法是,伽罗瓦在答题时逻辑跳跃过大,令水平不足的主考官无法理解,从而激怒了伽罗瓦。此外,父亲的突然去世可能也影响了他的情绪与表现。\(^\text{[4]}\)

在被综合理工学院拒之门外后,伽罗瓦改为参加学士考试,以进入巴黎师范学院(当时称为École Normale)。\(^\text{[10]}\)他顺利通过考试,并于1829年12月29日取得学位。\(^\text{[10]}\)他的数学考官评价道:“这名学生在表达自己的思想时有时显得晦涩,但他很聪明,展现出卓越的研究精神。”

伽罗瓦多次提交关于方程理论的论文稿,但在他生前从未获得发表。尽管他的第一次尝试被柯西拒绝了,但在1830年2月,他按照柯西的建议将论文递交给科学院秘书约瑟夫·傅里叶,\(^\text{[10]}\) 希望角逐科学院的大奖。不幸的是,傅里叶不久后去世,\(^\text{[10]}\)而那篇论文稿也随之遗失。\(^\text{[10]}\)当年的大奖最终追授给了尼尔斯·亨利克·阿贝尔,并与卡尔·雅可比共享。

尽管主要论文丢失,伽罗瓦当年还是发表了三篇论文。其中一篇奠定了伽罗瓦理论的基础;\(^\text{[11]}\)第二篇探讨了方程的数值解法(即现代所谓的“求根”问题);\(^\text{[12]}\)第三篇则是数论方面的重要成果,首次明确提出了有限域的概念。\(^\text{[13]}\)
