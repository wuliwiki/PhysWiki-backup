% 量子比特
% 未完成

\begin{issues}
\issueDraft
\issueOther{放一张布洛赫球的图片}
\end{issues}

\pentry{量子力学基本原理\upref{QMPrcp}}

在经典信息学中,一个比特(bit)代表着一个取值为0或者1的随机变量。比如一个电容器的状态可以离散表示为一个比特。当电容器处于高电平的时候,我们将其状态记为1,否则则记为0。在基于经典物理的信息论中,我们认为0和1这两种状态是可以被准确无误地区分开的。

在量子信息处理中,量子比特(qubit)是比特这个概念的量子对应。它描述了一个由$\ket{0},\ket{1}$表示的二能级量子系统的状态。我们仍然希望两种“量子状态”是可以被准确无误地区分开的,这自然要求着$\braket{0}{1}=0$。也就是说,$\ket{0},\ket{1}$张成了一个二维希尔伯特空间。

和经典比特不同的是,量子比特可以处于两种状态的叠加态上。也就是说,一个一般的量子比特可以处在状态
\begin{equation}
\ket{\psi}=a\ket{0}+b\ket{1},\quad a,b\in\mathbb{C},\quad |a|^2+|b|^2=1
\end{equation}
在$\{\ket{0},\ket{1}\}$基的测量下,有$|a|^2$的概率得到状态0,有$|b|^2$的概率得到状态1。



\subsection{物理实现}

有很多种不同的物理系统都可以实现量子比特。最简单的例子是自旋-1/2系统,它自然带有一个二维希尔伯特空间。不那么平凡的例子是光子的偏振自由度。虽然光子是自旋-1的系统,但是因为其没有静止质量,纵波方向的自由度被禁止。因此其偏振只能有两个取值,因此也可以作为量子比特的载体。此外,即使系统有着多于两个能级,比如很多类型的原子和介观量子电路,只要我们只考虑其中的两个能级,并保证有办法确定系统具体处在什么能级上,那么也可以将其作为量子比特的载体。

本部分不会花过多篇幅来讨论实际的物理载体,但是需要记住的是,量子信息科学并不是建立在抽象数学上的空中楼阁,而是有着扎实的物理根基。

\subsection{布洛赫(Bloch)球表示}

我们来数一数一个量子比特有多少自由的(实数)参数。$a,b$各有两个参数,$|a|^2+|b|^2=1$构成一个约束条件,看起来有三个约束,但是由于量子态的全局相位可以忽略,因此又会少一个自由度。因此描述一个量子比特只需要两个实数参数就够了。

根据归一化条件,一个一般的量子比特的描述也可以用
\begin{equation}
\ket{\psi}=e^{i\alpha}\left(\cos\frac{\theta}{2}\ket{0}+\sin\frac{\theta}{2}e^{i\psi}\ket{1}\right)
\end{equation}
来表示。

由于全局相位$\alpha$并不重要,因此我们总可以使用$\theta\in[0,\pi]$和$\phi\in[0,2\pi]$来表示一个任意的量子态。

可以看到在Bloch球表示当中,$\theta$和$\phi$刚好就是Bloch球上的球坐标。

\subsection{高维量子比特}

如果不加说明,量子比特都指的是包含两个能级的信息载体,不过在一些特殊情况下,我们也可以考虑更加高维的量子比特。为了做到这点只需要将希尔伯特空间的维度推广到$d$维。

当$d=3$时,我们称此时的量子比特为qutrit,在$d$为任意大于等于2的正数时,我们称此时的量子比特为qudit。