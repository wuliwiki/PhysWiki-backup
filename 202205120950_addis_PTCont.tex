% 含连续态的微扰理论
% 微扰理论|散射态|波函数|量子力学

\begin{issues}
\issueOther{合并到含时微扰理论\upref{TDPTc}}
\end{issues}

\pentry{不含时微扰理论}

一般的束缚+连续微扰理论. 假设我们有两个束缚态和连续态, 总的波函数可以写成
\begin{equation}
\ket{\psi} = \sum_n c_n\ket{n} + \int c_{\bvec k}\ket{\bvec k} \dd[3]{k}
\end{equation}

$\mat H'$  “矩阵” 可以想象成是这个样子的
\begin{figure}[ht]
\centering
\includegraphics[width=5cm]{./figures/PTCont_1.pdf}
\caption{$\mat H'$ 矩阵的结构} 
\end{figure}

方格子代表 $C_{ij} = \bra{i} H' \ket{j}$, 横条代表 $H_{\I\bvec k'} = \mel{i}{H'}{\bvec k'}$,  纵条代表 $H_{\bvec k j} = \mel{\bvec k}{H'}{j}$. 

\begin{equation}
c_i^{(n + 1)}(t) = \frac{1}{\I\hbar} \int \dd{t'} \qty(\sum_{j \ne i} H'_{ij} C_j^{(n)} + \int H_{\I\bvec k'} \phi ^{(n)} (\bvec k') \dd[3]{k'} )
\end{equation}
\begin{equation}
c^{(n+1)}_{\bvec k} = \frac{1}{\I\hbar} \int \dd{t'} \qty(\sum_j H'_{\bvec kj} C_j^{(n)} + \int H_{\bvec k\bvec k'} \phi ^{(n)}(\bvec k') \dd[3]{k'})
\end{equation}
