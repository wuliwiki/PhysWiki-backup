% 中国科学技术大学 2014 年考研普通物理考试试题
% keys 中国科学技术大学|考研|物理|2014年
% license Copy
% type Tutor


\textbf{声明}:“该内容来源于网络公开资料,不保证真实性,如有侵权请联系管理员”

\begin{enumerate}
\item 将一质量$m$,长为$l$的匀质柔软绳的两端合在一起悬挂于支点$O$。今使其中的一端脱离$O$而自由下落。求下落端下落距离$x$时,支点O所受的力F。
\item 一个质量为m,半径为R的匀质圆柱体放置在与水平面成$\theta$角的斜面上,如图所示。圆柱体和斜面直接的摩擦因数为$\mu$,若要让圆柱体无滑动地沿斜面滚下来,问$\theta$角的上限是多少。
(20分)设质量分别为m、m,的两个质点相距1,开始时均处于静止状态,其间仅有万有引力相互作用。
假设m固定不动,m将要经过多长时间1)后与m 相碰。
假设m也可动,两者将经过多少时间后相碰
一金属球带电量0,半径为,球外有一个半径为b的同心金属薄球壳,球与球壳间充满相对介电常数为ε=(K+r)/r的电介质,其中正为常数,是到球心的距离,求球与球壳的电势差
点电荷q与半径为a的接地导体球相距 d(d>a),求:
q所受静电力;q在感应电荷电场中的电势能;体系总静电能。
20分)一无限长圆筒内磁场均匀分布,且dB/d为大于零的常数,点P、Q坐标分别为(a,a)和(a,a),求下列三种情形下 P、Q两点间的电势差。(1)P、Q之间用圆弧导线连接;2P、Q之间用直导线连接;上述两根导线同时连接,单位长度电阻均为(3)常数 p。
考虑一个多电子原子,其电子组态为 1s“2s*2p°3s’3p“3d4s'4p4d:如果该原子遵循LS耦合,写出该电子组态耦合出的原子态符号;这个原子是否处于基态?如果不是基态,那么基态的电子组态又是什么样的?
写出该原子基态的原子态符号;如果该原子从 1s’2s’2p“3s’3p“3d"4s'4p4d 耦合出的原子态向基态跃迁,画出能级图及可能的允许跃迁。
(15分)假设将一部分质量数为3的氢同位素()通入含有正常氢气的放电管中,达到了足够在光谱仪中作观察的程度。(1)若不作波长测量,你能否直接从光谱中判断出哪些是H光谱,哪些是氚光谱?
确定应观察到的第一条巴耳末系谱线的间隔(用波长差表示)。已知质子和中子质量都是电子质量的1836倍,原子的电离能是13.6eV。
\end{enumerate}
