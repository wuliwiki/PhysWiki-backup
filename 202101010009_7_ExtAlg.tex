% 外代数
% 外代数|外微分|线性空间|格拉斯曼代数|Grassmann|外积|矢量分析|向量分析|外积|外乘|楔积|楔乘

\pentry{张成空间\upref{VecSpn}}

外代数是一种利用已有线性空间构造“代数”这一对象的通用方法,同时蕴含了对三维矢量分析中代数结构的本质解释.


给定线性空间$V$,任取$x, y\in V$,定义$x\wedge y\not\in V$是一个新的元素,其中符号$\wedge$称作\textbf{外积(exterior product)},有时也叫做\textbf{楔积(wedge product)},前者是因为这个运算得到的是$V$以外的新元素,后者是由于符号长得像个楔子.注意为了方便,我们没有使用线性代数中常见的粗体正体符号来表示向量.

利用各$x\wedge y$构造新的线性空间:定义$x\wedge y=-y\wedge x$对所有$x, y\in V$成立,这同时意味着$x\wedge x=0$.集合$\{x\wedge y|x, y\in V\}$张成的线性空间,记为$\bigwedge^2 V$.同时,为了统一考虑,记$V=\bigwedge^1 V$.

$\bigwedge^1 V$和$\bigwedge^2 V$之间也可以进行楔积,并且满足\textbf{结合律}:$x\wedge(y\wedge z)=(x\wedge y)\wedge z$,由此可以拿掉结合括号,定义$x\wedge y\wedge z=x\wedge(y\wedge z)=(x\wedge y)\wedge z$.集合$\{x\wedge y\wedge z|x, y, z\in V\}$张成的线性空间,记为$\bigwedge^3 V$.

同理,我们可以构造出任意阶的$\bigwedge^k V$.要注意的是,如果$k>\opn{dim} V$,那么$\bigwedge^k V=\{0\}$.

不同线性空间之间可以用直和组合在一起,因此以上这些空间也都可以作直和,得到一个$\bigwedge V=\bigoplus\bigwedge^k V$.这个$\bigwedge V$,就被称作$V$上的\textbf{外积空间(exterior product space)}或\textbf{楔积空间(wedge product space)}.

\begin{theorem}{外代数}
任给线性空间$V$,则$\bigwedge V$上的外积
\end{theorem}
