% 等比数列(高中)
% keys 高中|等比数列
% license Usr
% type Tutor

\begin{issues}
\issueDraft
\end{issues}

\pentry{数列(高中)\nref{nod_HsSeFu}}{nod_d6a2}

在介绍\aref{数列和}{sub_HsSeFu_1}时,曾提到国际象棋棋盘和麦粒的故事。故事中提到:“依此类推,每一格的麦粒数量是上一格的两倍。”尽管这一描述看似简单,但敏锐的读者可能已经注意到,如果将每一格的麦粒数量视为一个数列中的一项,那么这个“二倍”实际上就是数列生成的递推规则。换句话说,构成数列的每一项都是前一项的若干倍。

这种生成关系在自然界和生活中非常常见。比如,细胞的分裂遵循类似的规律,一个细胞分裂成两个细胞,两个分裂成四个,数量呈倍数增长;银行存款计算复利、声音传播中强度的衰减等场景也遵循类似的关系,每个周期都会在上一次的基础上按一定比例增长。尽管这些现象的具体数值不同,但它们的数学本质却是相同的。

人们以这种倍数特征将其命名为等比数列。与最简单的等差数列相比,等比数列的规则从“差”变成了“比”。这种看似微小的变化,使得数列的增长形式从线性变为指数。等比数列的这种特性不仅在基础数学中有广泛应用,其影响在大学数学中更为突出。例如,等比数列的延伸——等比级数,成为研究无限序列和是否收敛的核心工具。等比级数不仅是微积分的重要内容,还广泛应用于物理、经济学等领域,用于描述振荡、增长、衰减等复杂现象。

在麦粒的故事中,还提到过等比数列的数列和展现了指数的力量。这一力量背后隐藏着怎样的数学机制?理解这些问题,不仅对当前的学习有帮助,更会在未来的数学学习中发挥关键作用。接下来,将深入研究等比数列的特性和应用,揭示其中的规律与奥秘。

\subsection{等比数列}

前面提到的种种例子,无论是细胞分裂、银行存款计息,还是棋盘上的麦粒故事,其核心规律都是相同的,即“按比例增长或减少”的过程。本质上,这些现象都是通过相邻项的固定比值将整个变化过程紧密连接起来。基于这一规律,可以明确等比数列的递推定义:

\begin{definition}{等比数列}
如果数列 $\{a_n\}$ 满足对于 $n > 1$ 的所有项,每一项与前一项的比为同一个常数 $q(q\neq0)$,则称 $\{a_n\}$ 为\textbf{等比数列(geometric sequence)},$q$ 称为 $\{a_n\}$ 的\textbf{公比(common ratio)},即等比数列满足递推公式:
\begin{equation}
a_{n}=a_{n-1}\cdot q\qquad(n>1)~.
\end{equation}
\end{definition}

显然由于分母不能为0,所以等比数列中不会出现某一项的值为0。特别地,之前提到的常数列是 $q = 1$ 的等比数列。


下面根据定义,迭代递推公式来推导等比数列的通项公式。当$n>1$ 时,利用递推公式  $a_n = a_{n-1} \cdot q$ 逐步将数列的前一项用递推公式展开,可以得到:
\begin{equation}
\begin{aligned}
a_n &= a_{n-1} \cdot q\\
&= a_{n-2}\cdot q\cdot q\\
&\cdots \\
&= a_2\cdot q^{n-2}\\
&=  a_1 \cdot q^{n-1}~.
\end{aligned}
\end{equation}
检查边界条件,当$n=1$时,显然有:
\begin{equation}
a_1=a_1q{1-1}
\end{equation}

\begin{corollary}{等比数列通项公式}
对等比数列$\{a_n\}$,其通项公式为:
\begin{equation}
a_n = a_1 q^{n-1} \quad (a_1 \ne 0,\ q\ne 0,\ n=1,2,3\dots)~
\end{equation}
其中,$a_1$ 是首项,$d$ 是公差,$n$ 是项数。

\end{corollary}




\subsubsection{等比中项}
与等差数列类似,如果在 $a$ 和 $b$ 中插入一个数 $G$,使得 $a,G,b$ 成等比数列,那么根据等比数列的定义, $\frac{G}{a} = \frac{b}{G},G^2 = ab,G = \pm \sqrt{ab}$。我们称 $G$ 为 $a,b$ 的\textbf{等比中项}。

易得,在等比数列中,首末两项除外,每一项都是它前后两项的等比中项。

\subsection{等比数列的数列和}
等比数列求和与等差数列求和有相似之处,
\begin{equation}\label{eq_HsGmPg_1}
S = a_1 + a_2 + \cdots + a_n~,
\end{equation}
\begin{equation}
qS = qa_1 + qa_2 + \cdots + qa_n~,
\end{equation}
\begin{equation}\label{eq_HsGmPg_2}
qS= a_2 + a_3 + \cdots + qa_n~.
\end{equation}
\autoref{eq_HsGmPg_1} 与 \autoref{eq_HsGmPg_2} 做差
\begin{equation}
\begin{aligned}
(1 - q)S &= a_1 - qa_n\\
&= a_1(1 - q^n)~,
\end{aligned}
\end{equation}
\begin{equation}
S = \frac{a_1(1-q^n)}{1-q} \quad (q\neq 1)~.
\end{equation}
则等比数列前 $n$ 项和为
\begin{equation}
S = 
\begin{cases}
\begin{aligned}
&na_1 &\quad &(q = 1) \\
&\frac{a_1(1-q^n)}{1-q} &\quad &(q \neq 1)~.
\end{aligned}
\end{cases}
\end{equation}

当等比数列无穷递缩时,即 $0<q<1$, $n\rightarrow \infty$。 这里的极限详见 “\enref{数列的极限(简明微积分)}{Lim0}”。
\begin{equation}
S = \frac{a_1}{1 - q}~.
\end{equation}

\subsection{用数学归纳法证明求和公式}
要证明求和公式
\begin{equation}
1 + q + q^2 + \dots + q^{n-1} = \frac{q^n - 1}{q - 1}~.
\end{equation}
使用数学归纳法, 当 $n = 1$ 时
\begin{equation}
\frac{q^n - 1}{q - 1} = 1~,
\end{equation}
恰好是第 1 项。 第 $n + 1$ 项为
\begin{equation}
\frac{q^{n+1} - 1}{q - 1} - \frac{q^n - 1}{q - 1} = q^n~.
\end{equation}
证毕。
