% 新基础集合论(综述)
% license CCBYSA3
% type Wiki

本文根据 CC-BY-SA 协议转载翻译自维基百科\href{https://en.wikipedia.org/wiki/New_Foundations}{相关文章}。

在数学逻辑中,新基础(New Foundations,简称 NF)是一种非良基、可有限公理化的集合论,由威拉德·范·奥曼·奎因构思,旨在简化《数学原理》中的类型论。
\subsection{定义}
NF 的良构公式是命题演算的标准公式,具有两个基本谓词:相等(\(=\))和成员关系(\(\in\))。NF 可以仅通过两个公理模式来表述:

\begin{itemize}
\item 外延性:具有相同元素的两个对象是相同的对象。形式化地说,给定任意集合 \( A \) 和任意集合 \( B \),如果对于任意集合 \( X \),\( X \) 是 \( A \) 的成员当且仅当 \( X \) 是 \( B \) 的成员,则 \( A \) 等于 \( B \)。
\item 受限的理解公理模式:对于每个分层公式\( \phi \),集合 \( \{x \mid \phi\} \) 存在。
\end{itemize}
一个公式 \( \phi \) 被称为**分层的**(stratified),如果存在一个从 \( \phi \) 的语法结构的各部分到自然数的函数 \( f \),使得:对于 \( \phi \) 中的任意原子子公式 \( x \in y \),满足 \( f(y) = f(x) + 1 \);对于 \( \phi \) 中的任意原子子公式 \( x = y \),满足 \( f(x) = f(y) \)。