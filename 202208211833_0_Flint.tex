% FLINT 库笔记

\begin{issues}
\issueDraft
\end{issues}

\pentry{GNU Multiple Precision(GMP)库笔记\upref{GMP}}

\subsubsection{flint.h}
\begin{itemize}
\item \href{http://flintlib.org/sphinx/fmpz.html}{文档}
\item \verb|fmpz| 底层就是 \verb|slong|. 当第二最重要 bit 为 0 就是一个普通的 slong, 绝对值最多有 \verb|FLINT_BITS - 2| bits.
\item 当第二重要 bit 为 1 时, 它代表一个指针.
\end{itemize}
\begin{lstlisting}[language=cpp]
// COEFF_MIN, COEFF_MAX 是不 alloc 时能表示的最小和最大整数
#define COEFF_MAX ((WORD(1) << (FLINT_BITS - 2)) - WORD(1))
#define COEFF_MIN (-((WORD(1) << (FLINT_BITS - 2)) - WORD(1)))
/* fmpz x 是否是指针,存在 alloc */
#define COEFF_IS_MPZ(x) (((x) >> (FLINT_BITS - 2)) == WORD(1))

#define slong mp_limb_signed_t

#if GMP_LIMB_BITS == 64
    #define FLINT_BITS 64 // 一个 limb 或者 slong 的 bit 数
    #define FLINT_D_BITS 53
    #define FLINT64 1
#else 
    #define FLINT_BITS 32
    #define FLINT_D_BITS 31
#endif

typedef slong fmpz;
typedef fmpz fmpz_t[1];

// in fmpz_gc.c
__mpz_struct * _fmpz_promote(fmpz_t f)
{
    if (!COEFF_IS_MPZ(*f)) /* f is small so promote it first */
    {
        __mpz_struct * mf = _fmpz_new_mpz();
        (*f) = PTR_TO_COEFF(mf);
        return mf;
    }
    else /* f is large already, just return the pointer */
        return COEFF_TO_PTR(*f);
}

#define COEFF_TO_PTR(x) ((__mpz_struct *) (((ulong)x) << 2))

FMPZ_INLINE void
fmpz_set_si(fmpz_t f, slong val)
{
    if (val < COEFF_MIN || val > COEFF_MAX) /* val is large */
    {
        // __mpz_struct 是 GMP 的大整数数据结构
        __mpz_struct *mpz_coeff = _fmpz_promote(f);
        flint_mpz_set_si(mpz_coeff, val);
    }
    else
    {
        _fmpz_demote(f);
        *f = val;               /* val is small */
    }
}
\end{lstlisting}
