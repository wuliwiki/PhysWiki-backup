% MinGW-w64 和 MSYS2 笔记
% license Xiao
% type Note

\subsection{不同工具的比较}
\begin{itemize}
\item \href{https://www.msys2.org/docs/what-is-msys2/}{MSYS 和 WSL, Cygwin 等的对比}
\item 无论是 MinGW 还是 MSYS,二进制可执行文件都是 Windows 原生的 exe/dll 等文件。MinGW 工具链可以使用 \verb`Windows.h` 调用 Windows API,在 Linux 上无法运行。
\item WSL1/2 的可执行文件都是 Linux 的 ELF 文件,可以直接在 Linux 系统上运行,调用 Unix API (WSL1 相当于 Linux 和 Windows 之间的二进制兼容层,WSL2 是彻底的虚拟机),不能直接在 Windows 运行。
\item 所以选用哪个取决于你开发出来的软件最终要在 Windows 系统还是 Linux 系统或 WSL 上跑。
\item \verb`MinGW` 加 \verb`MSYS` 相当于一个轻量级的 \verb`Cygwin`,后者编译的软件需要链接到 \verb`cygwin1.dll`,它可以把对 Unix API 的调用转换成对 Windows API 的调用(但应该不在二进制层面)。
\end{itemize}

\subsubsection{POSIX API 是什么?}
\begin{itemize}
\item \textbf{文件系统}: Functions for creating, reading, writing, and closing files, as well as manipulating directories.
\item \textbf{进程管理}: Capabilities for process creation, execution, and synchronization, including fork, exec, and various forms of inter-process communication (IPC) like pipes and message queues.
\item \textbf{线程管理}: Standards for multithreading support, including thread creation, synchronization primitives like mutexes, and condition variables.
\item \textbf{网络接口}: APIs for socket-based communication, allowing for the implementation of client-server models and network communication.
\item \textbf{System Calls}: Standardized system calls for resource management, including memory allocation and device control.
\end{itemize}

\subsection{MinGW}
\begin{itemize}
\item MinGW 只是在 Windows 上提供 GCC 和 GDB 等工具, 不包括 bash shell,\verb`cat, echo` 等命令。 MSYS 提供 bash shell 和 \verb`cat, echo` 等命令但本身不提供编译器。 所以二者通常一起使用。 建议直接安装 MSYS,然后用 pacman 安装 MinGW 的 gcc 编译器等。
\item MinGW 仅支持 32-bit 程序, 现在一般用 \href{https://www.mingw-w64.org/}{MinGW-w64}, 既支持 32 也支持 64-bit
\item 一个\href{https://www3.ntu.edu.sg/home/ehchua/programming/howto/Cygwin_HowTo.html}{教程}。
\item \href{https://www.msys2.org/}{MSYS2} 是一个 windows 上的 bash shell 调用 MinGW-W64 来实现编译。 MSYS 最早是 Cygwin 的一个分支。
\item 双击 \verb`mingw64.exe` 即可打开 mingw 命令行。 \verb`mingw64.exe` 所在的目录就是命令行的根目录。
\item 可以检查 \verb`g++` 的版本。
\item 用 \verb`g++` 编译程序以后, 会出现 \verb`a.exe` 而不是 \verb`a.out`。 这个执行文件是可以双击执行的, 但是运行完会马上退出。 可以在程序最后用 \verb`getchar()`。 也可以打开一个 cmd 命令行然后运行 \verb`a.exe`。 当然在 \verb`mingw64` 的命令行也可以执行。 另外执行时 \verb`.exe` 拓展名可以省略。
\item mingw 不能使用 \verb`apt` 或类似的包管理软件, 一切库都要自己手动编译。 或者用 MinGW instellation 安装 32-bit 版本。 MSYS 上可以用 \verb`pacman`\upref{pacman}。
\item \verb`MinGW instellation manager` 是一个 GUI 界面,可以下载安装编译好的 mingw 程序。 例如安装 \verb`mingw32-make`, 安装以后就和 \verb`make` 的用法一样。 注意该 manager 并不是一个通用的包管理软件, 里面的包基本还是围绕 GCC 的。
\item \verb`c` 盘的目录为 \verb`/c`
\item 如果创建 symlink 会直接复制文件(夹)而不是真的 link。
\item \verb`g++` 中定义的宏有 \verb`__GNUC__`, \verb`__MINGW32__`, \verb`__MINGW64__`
\item \enref{编译动态链接库}{gppLib}的方法和 linux 中一样, 同样支持用 \verb`ldd` 查看动态程序的库依赖(\verb`dll` 和 \verb`so` 拓展名都会搜索到, 但建议用 \verb`dll`, 让其他 windows 程序也可以使用)。
\begin{lstlisting}[language=bash]
g++ -shared -o lib0.dll lib0.o
g++ -shared -o lib1.dll lib1.o -L . -l 0
g++ -c main.cpp
g++ -o main.x main.o -l1 -L./ -Wl,-rpath,./
\end{lstlisting}
\item MinGW 中(也就是 Windows 中),\verb`LD_LIBRARY_PATH` 变量是没用的,直接添加到 \verb`PATH` 就可以在其中的路径下自动搜索 dll
\end{itemize}

\subsection{MSYS2}
\begin{itemize}
\item \href{https://www.msys2.org/}{官网}
\item MSYS2 使用 \verb`pacman`\upref{pacman} 进行包管理, 和 Arch linux 一样。
\item 安装后无需另外安装 MinGW,直接用 pacman 就可以安装 mingw 版的 g++ 等编译工具。
\item \href{https://packages.msys2.org/queue}{所有可用的包在这里}查找。 名字一般和 \verb`apt` 命令中的不一样。
\item 这个页面里面有 4 种 Repository:\textbf{Clang、UCRT、MinGW、MSYS}(例如 gcc 编译器有这些不同 repo 的版本)我们稍后讨论。现在可以先安装 MinGW 版本的。
\item 根目录在 \verb`C:\msys64\`, 和 mingw 一样可以用 \verb`/c` 访问 C 盘。
\item 安装 gcc: \verb`pacman -S mingw-w64-x86_64-gcc` (UCRT 版的是 \verb`mingw-w64-ucrt-x86_64-gcc`,MSYS 版的是 \verb`gcc`)。
\item 安装目录是 \verb`/mingw64/bin`, 要把它默认添加到 PATH,用 \verb` echo 'export PATH=$PATH:/mingw64/bin' >> ~/.bashrc; source ~/.bashrc`
\item 现在应该就可以 \verb`g++ --version` 查看版本,愉快编译程序了。
\item 用 \verb`g++` 编译时, 宏 \verb`__MSYS__` 有定义, 而 \verb`__MINGW32__` 和 \verb`__MINGW64__` 没定义。
\item (看起来好像用处不大可以先不装)\verb`pacman -S base-devel` 包含\verb`binutils bison diffstat diffutils dos2unix flex m4 make patch pkgconf texinfo texinfo-tex base-devel`
\item 现在可以写一个简单的 c++ 程序并编译,默认可执行文件是 \verb`a.exe`。 直接在 MSYS2 中运行没问题,但是如果在 Windows 中双击就会找不到 dll。
\item 这时可以在 MSYS2 中用 \verb`ldd a.exe` 看看需要什么动态库。其中 \verb`/c/WINDOWS/SYSTEM32` 类似的目录不需要管,因为它们已经在 PATH 中且是 Windows 系统提供的。
\item 如果你装的是 UCRT 版本的 gcc,那额外 dll 将会在 \verb`/ucrt64/bin/` 中(也就是 \verb`C:\msys64\ucrt64\bin\` 中)。 UCRT 是 Windows 的 \textbf{Unified C Runtime}。
\item 如果你装的是 MinGW 非 UCRT 版本的 gcc, 那额外依赖的 dll 是 \verb`/mingw64/bin/` 中的 \verb`libgcc_s_seh-1.dll  libstdc++-6.dll  libwinpthread-1.dll`
\item 解决 Windows 找不到 dll 的一个简单的方法就是把额外需要的依赖库文件直接拷贝到和 exe 同一个目录中。
\item 双击 exe 成功运行后,窗口会迅速闪退,在 \verb`main()` 函数退出前用一个 \verb`getchar();` 即可实现任意键退出。
\item 是否可以在编译时使用 rpath 把依赖库的相对目录直接写进 exe 中(这样就可以把库放到子文件夹)? GPT 说这是不行的。 要么修改系统路径,要么创建一个 \enref{XML}{XML} 文件指定依赖路径,命名为 \verb`可执行文件名.exe.manifest`,和 exe 文件放在一起即可(文件具体内容可以问 GPT)。
\item 如果程序中使用了 \verb`system()`,那么当它在 MSYS2 命令行中运行时可以调用 Unix 命令例如 \verb`mkdir`,但双击运行时调用的应该是 \verb`cmd`。
\end{itemize}

\subsubsection{MSYS 的 Clang、UCRT、MinGW、MSYS 仓库}
\begin{itemize}
\item MinGW 仓库是最经典最成熟历史最久的仓库,编译出的 exe 只依赖于 Windows 提供的 dll
\item Clang 类似于 MinGW 但用的不是 gcc 编译器而是 clang
\item UCRT 仓库更现代兼容更多新的 Windows API,例如 Windows 10 或以上的,但生成的 exe 依赖于额外的 ucrt dll。
\item MSYS 版本编译出的 exe 运行时依赖于 Unix 兼容层 dll,不能直接发布。 Windows 原生软件,通途有点类似于 WSL。
\end{itemize}

\subsection{MinTTY}
\begin{itemize}
\item MinTTY 是 MinGW 和 MSYS2 默认的命令行窗口,Git Bash 用的也是这个。
\item \verb`Ctrl+Ins` 复制,\verb`Shift+Ins` 粘贴。
\item \verb`Ctrl+=` 放大 \verb`Ctrl+-` 缩小。滚轮也可以。
\item \verb`Alt+F3` 搜索
\end{itemize}
