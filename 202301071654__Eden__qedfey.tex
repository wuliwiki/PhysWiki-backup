% QED的费曼规则
% QED的费曼规则|量子电动力学|费曼图

\subsection{$n$ 点格林函数的费曼图表示}
QED 的 $n$ 点格林函数 $\bra{\Omega} \psi(x)\bar\psi(y)A_\mu(z) \cdots \ket{\Omega}$ 可以通过相互作用绘景与自由场的格林函数相联系,再利用 QED 的 Wick 定理,我们可以求得 $n$ 点格林函数在动量空间的费曼图表示。
\begin{theorem}{旋量 QED 的 Feynman 规则(动量空间)}

\begin{enumerate}
\item 画出所有\textbf{连通的}费曼图。
\item 给每一个传播子一个四动量,并在每个顶点要求动量守恒。
\item 对于动量为 $p$ 的费米子传播子,写下:$\frac{i(\not p+m_0)}{p^2-m_0^2 + i\epsilon}$。
\item 对于动量为 $q$、两端矢量指标为 $\mu,\nu$ 的光子传播子,写下:$\frac{-ig_{\mu\nu}}{q^2 + i\epsilon}$。
\item 对于相互作用顶点,写下:$-ie\gamma^\mu$;
\item 
对所有未知动量积分。
\item 
考察每个图中由于费米统计所可能造成的符号,例如一个费米子圈总会贡献一个负号。
\end{enumerate}
\end{theorem}

\subsection{S-矩阵元的费曼图表示}
\pentry{LSZ约化公式(旋量场)\upref{lszspn},LSZ约化公式(矢量场)\upref{lszqed}}

当外线的动量在壳时,外腿存在极点行为,而 $n$ 点格林函数的最奇异的多极点部分对 S-矩阵元有贡献。因此当我们计算 Feynman 矩阵元时,我们需要对连通的 Feynman 图进一步截肢,来消去外腿的极点。这是 LSZ 约化公式告诉我们的。

QED 是一个关于旋量、矢量场的相互作用理论。对旋量场、矢量场的 Wick 定理与 LSZ 约化公式作一个整理和总结,我们最终得到了旋量 QED 的 Feynman 规则:$i\mathcal{M}$ 可以由以下方式微扰计算:
\begin{theorem}{旋量 QED 的 Feynman 规则(动量空间)}

\begin{enumerate}
\item 画出所有\textbf{连通的、截肢的}费曼图。
\item 给每一个传播子一个四动量,并在每个顶点要求动量守恒。
\item 对于动量为 $p$ 的费米子传播子,写下:$\frac{i(\not p+m_0)}{p^2-m_0^2 + i\epsilon}$。
\item 对于动量为 $q$、两端矢量指标为 $\mu,\nu$ 的光子传播子,写下:$\frac{-ig_{\mu\nu}}{q^2 + i\epsilon}$。
\item 对于相互作用顶点,写下:$-ie\gamma^\mu$;
\item 对于外线的费米子和反费米子,写下它们对应的旋量:
\begin{align*}
&\overset{1}{\psi}\overset{1}{\ket{\bvec p,s,+}}=u^s(\bvec p),\quad \overset{1}{\bra{\bvec p',s',+}} \overset{1}{\bar\psi}=\bar u^{s'}(\bvec p')\\
&\overset{1}{\bar\psi}\overset{1}{\ket{\bvec k,r,-}}=v^{r}(\bvec k),\quad \overset{1}{\bra{\bvec k',r',-}} \overset{1}{\psi}=\bar v^{r'}(\bvec k')
\end{align*}
\item 对于外线的光子,写下它们对应的偏振矢量:
\[
\overset{1}{A^\mu}(x) \overset{1}{\ket{\bvec k,\lambda}}=\epsilon^{(\lambda)\mu}(\bvec k),\quad \overset{1}{\bra{\bvec k',\lambda'}}\overset{1}{A^\nu}(x)=\epsilon^{(\lambda)\nu*}(\bvec k')
\]
\item 
对所有未知动量积分。
\item 
考察每个图中由于费米统计所可能造成的符号,例如一个费米子圈总会贡献一个负号。
\end{enumerate}
\end{theorem}
\addTODO{规范场传播子的 $R_\xi$ 规范}
不同 $\xi$ 规范下的光子传播子是不同的,不过后面我们将通过 Ward 等式证明 S-矩阵和选取的 $\xi$ 是无关的。
