% 理想气体
% keys 热力学|理想气体

\begin{issues}
\issueDraft
\end{issues}

\subsection{理想气体状态方程}

理想气体的简要定义是:分子有质量无体积,分子间无作用里的气体体系.理想气体的性质与物质无关,而且是一般气体体系的良好近似,所以是一个非常值得研究的重要模型.

\subsection{微观模型}

理想气体是研究热力学系统的一种模型.它的微观模型有几种基本假设:

1.理想气体由大量运动的微观粒子组成.每个粒子都是质量为 $m$ 的质点,它的行为服从牛顿运动定律(注意,在标准状态下气体分子间平均距离和气体半径之比约为 $30$,所以可以近似地看作质点).

2. 粒子间无相互作用(这意味着分子势能是刚球势).粒子只与容器发生碰撞,所有碰撞都是弹性碰撞.

3. 组成理想气体的粒子的运动是完全无序的、各向同性的.完全无序体系无宏观运动.

由以上三条基本假设,可以得到热力学中非常重要的\textbf{细致平衡原理}:达到平衡态的气体中能实现的任一正向的元过程,必定有一逆元过程与之相平衡.例如:在理想气体容器中的任意一个面元,每当有一个气体分子从左边穿过它到右边,就一定有另一个同样的分子在同一时刻从右边穿过面元来到左边.理想气体在现实的物理世界中是不可能存在的,但它作为一个理想化的模型可以简化大量计算,而且是一些气体系统的良好近似.