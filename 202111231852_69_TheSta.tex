% 热力学量的统计表达式(玻尔兹曼分布)
% keys 热力学量|统计力学|配分函数|玻尔兹曼分布

\pentry{玻尔兹曼分布(统计力学)\upref{MBsta}}

满足经典极限\footnote{在玻尔兹曼分布\upref{MBsta} 词条中谈到了玻色分布和费米分布的表达式,式中如果 $e^\alpha\gg 1$,那么将过度到经典情况的玻尔兹曼分布.我们称这个条件为经典极限.}的大量粒子组成的系统中,粒子遵从玻尔兹曼分布.我们可以试图用统计力学中配分函数来推出一切热力学量.

配分函数表达式为:
\begin{equation}
Z_1=\sum_l \omega_l e^{-\beta \epsilon_l}
\end{equation}

式中 $\omega_l$ 为能级的简并度.根据玻尔兹曼分布,每个能级上的粒子数为 $e^{-\alpha-\beta\epsilon_l}$.于是有
\begin{equation}
\begin{aligned}
&N=\sum_l \omega_l e^{-\alpha-\beta\epsilon_l}=e^{-\alpha} Z_1\\
&E=\sum_l \epsilon_l \omega_l e^{-\alpha-\beta\epsilon_l}=-e^{-\alpha}\frac{\partial Z_1}{\partial \beta}=-\frac{N}{Z_1}\frac{\partial Z_1}{\partial \beta}=-N\frac{\partial \ln Z_1}{\partial \beta}
\end{aligned}
\end{equation}

对于一个粒子数 $N$ 和内能 $E$ 确定的系统,$\alpha,\beta$ 可以由上面两个表达式确定.