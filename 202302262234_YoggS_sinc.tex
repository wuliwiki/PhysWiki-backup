% sinc 函数
% 函数|连续|积分

\pentry{连续函数\upref{contin}}

$\sinc$ 函数的定义为
\begin{equation}
\sinc x = 
\leftgroup{
&\frac{\sin x}{x} &\quad & (x \ne 0)\\
&\quad 1 && (x = 0)
}\end{equation}

$\sinc$ 函数的图像如\autoref{sinc_fig1}, 可以证明, 该函数在 $x=0$ 处是连续的, 即
\begin{equation}
\lim_{x \to 0}\frac{\sin x}{x} = \sinc 0 = 1
\end{equation}

\begin{figure}[ht]
\centering
\includegraphics[width=10cm]{./figures/sinc_1.png}
\caption{sinc 函数(来自维基百科)} \label{sinc_fig1}
\end{figure}

\subsubsection{积分性质}
\begin{equation}\label{sinc_eq2}
\int_{-\infty}^{+\infty} \sinc x \dd{x} = \pi
\end{equation}

证明见\autoref{JdLem_ex1}~\upref{JdLem}。

\begin{equation}\label{sinc_eq1}
\int_{-\infty}^{+\infty} \sinc^2 x \dd{x} = \pi
\end{equation}

$\sinc$ 函数的原函数称为正弦积分函数, 详见 “三角积分”。
