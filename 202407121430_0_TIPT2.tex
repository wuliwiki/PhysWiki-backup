% 二阶不含时微扰理论(量子力学)
% license Xiao
% type Tutor

\begin{issues}
\issueDraft
\end{issues}

\pentry{一阶不含时微扰理论(量子力学)\nref{nod_TIPT}}{nod_ea9f}

\begin{equation}\label{eq_TIPT2_2}
E_n^2 = \sum_{m}^{E_m\ne E_n} \frac{\abs{\mel{\psi_m^0}{H^1}{\psi_n^0}}^2}{E_n^0-E_m^0}~.
\end{equation}
注意 $\psi_n^0$ 必须取好量子态。

\subsection{推导}
类似于一阶微扰的推导(\autoref{sub_TIPT_4}), 若\autoref{eq_TIPT_10} 展开括号后仅保留 $\lambda^2$ 的项,得
\begin{equation}
H^0\psi_n^2 + H^1\psi_n^1 = E_n^0\psi_n^2 + E_n^1\psi_n^1 + E_n^2\psi_n^0~.
\end{equation}
两边左乘任意 $\psi_m^0$ 得
\begin{equation}\label{eq_TIPT2_1}
\mel{\psi_m^0}{H^0}{\psi_n^2} + \mel{\psi_m^0}{H^1}{\psi_n^1} = E_n^0\braket{\psi_m^0}{\psi_n^2} + E_n^1\braket{\psi_m^0}{\psi_n^1} + \delta_{m,n}E_n^2~.
\end{equation}
其中
\begin{equation}
\mel{\psi_m^0}{H^0}{\psi_n^2} = \braket{H^0\psi_m^0}{\psi_n^2} = E_m^0\braket{\psi_m^0}{\psi_n^2}~.
\end{equation}
代入\autoref{eq_TIPT2_1} 得
\begin{equation}\label{eq_TIPT2_3}
\mel{\psi_m^0}{H^1}{\psi_n^1} = (E_n^0-E_m^0)\braket{\psi_m^0}{\psi_n^2} + E_n^1\braket{\psi_m^0}{\psi_n^1} + \delta_{m,n}E_n^2~.
\end{equation}
我们要求该式对所有可能的 $m,n$ 都成立。

\subsubsection{非简并}
考虑对角元($m=n$),\autoref{eq_TIPT2_3} 要求
\begin{equation}\label{eq_TIPT2_4}
E_n^2 = \mel{\psi_n^0}{H^1}{\psi_n^1} - E_n^1\braket{\psi_n^0}{\psi_n^1}~.
\end{equation}
把波函数的一阶微扰(\autoref{eq_TIPT_5})
\begin{equation}\label{eq_TIPT2_6}
\psi_n^1 = \sum_m^{E_m\ne E_n} \frac{\mel{\psi_m^0}{H^1}{\psi_n^0}}{E_n^0 - E_m^0}\psi_m^0~.
\end{equation}
代入\autoref{eq_TIPT2_4} 发现第二项为 0, 得二阶能量修正(\autoref{eq_TIPT2_2})
\begin{equation}\label{eq_TIPT2_5}
E_n^2 = \mel{\psi_n^0}{H^1}{\psi_n^1} = \sum_{m}^{E_m\ne E_n} \frac{\abs{\mel{\psi_m^0}{H^1}{\psi_n^0}}^2}{E_n^0-E_m^0}~.
\end{equation}

\autoref{eq_TIPT2_3} 要求所有非对角元($m\ne n$)满足
\begin{equation} % 验证:不同时间独立推导过两次,结果相同
\braket{\psi_m^0}{\psi_n^2} =
\frac{1}{E_n^0-E_m^0}\sum_{m'}^{E_{m'}\ne E_n}\frac{\mel{\psi_m^0}{H^1}{\psi_{m'}^0}\mel{\psi_{m'}^0}{H^1}{\psi_n^0}}{E_n^0 - E_{m'}^0} - \frac{E_n^1\mel{\psi_m^0}{H^1}{\psi_n^0}}{(E_n^0 - E_m^0)^2}~.
\end{equation}
同样假设 $m=n$ 时上式为 0,就得到了二阶波函数修正 $\psi_n^2$。

\subsubsection{二阶能量修正的意义}
根据\autoref{eq_TIPT_10} 有
\begin{equation}
\begin{aligned}
E_n &= E_n^0 + \lambda E_n^1 + \lambda^2 E_n^2 + \dots \\
&=\mel{\psi_n^0 + \lambda\psi_n^1 + \lambda^2\psi_n^2 + \dots}{H^0 + \lambda H^1}{\psi_n^0 + \lambda\psi_n^1 + \lambda^2 \psi_n^2 + \dots}~.
\end{aligned}
\end{equation}
右边展开后如果按 $\lambda$ 的幂合并同类项,是否会和左边的各项对应? 容易验证这对 $E_n^0$ 和 $E_n^1$ 来说都是正确的。 但对 $\lambda^2$ 有
\begin{equation}
E_n^2 = 2\mel{\psi_n^0}{H^1}{\psi_n^1} + \mel{\psi_n^1}{H^0}{\psi_n^1} + 2\mel{\psi_n^0}{H^0}{\psi_n^2}~.
\end{equation}
如果这成立, 那么对比\autoref{eq_TIPT2_5},就意味着
\begin{equation}
E_n^2 + \mel{\psi_n^1}{H^0}{\psi_n^1} + 2\mel{\psi_n^0}{H^0}{\psi_n^2} = 0~.
\end{equation}
这是否可以证明成立? 把\autoref{eq_TIPT2_6} 代入可得
\begin{equation}
\begin{aligned}
\mel{\psi_n^1}{H^0}{\psi_n^1} &= \sum_{m}^{E_m\ne E_n} \frac{\abs{\mel{\psi_m^0}{H^1}{\psi_n^0}}^2}{E_n^0 - E_m^0} \frac{E_m^0}{E_n^0 - E_m^0} = E_n^2 - E_n^0\braket{\psi_n^2}{\psi_n^2}~.
\end{aligned}
\end{equation}
最后一项
\begin{equation}
\mel{\psi_n^0}{H^0}{\psi_n^2} = E_n^0 \braket{\psi_n^0}{\psi_n^2}~.
\end{equation}
