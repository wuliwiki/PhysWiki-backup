% 圆锥曲线的极坐标方程
% keys 极坐标系|圆锥曲线|离心率
% license Xiao
% type Tutor

\pentry{极坐标的定义\nref{nod_Polar},圆锥曲线与圆锥}{nod_f498}

\begin{figure}[ht]
\centering
\includegraphics[width=11cm]{./figures/52670f52be70ae3b.pdf}
\caption{不同离心率 $e$ 的圆锥曲线($p = 1$)} \label{fig_Cone_2}
\end{figure}

圆锥曲线的极坐标方程是 $r$ 关于 $\theta$ 的函数(选取一个焦点作为原点)
\begin{equation}\label{eq_Cone_5}
r(\theta)  = \frac{l}{1 - e\cos \theta }~.
\end{equation}

由于$l=ep$,\autoref{eq_Cone_5} 也可以记为
\begin{equation}\label{eq_Cone_3}
r(\theta) = \frac{ep}{1 - e\cos \theta }~.
\end{equation}
其中 $e$ 是\textbf{离心率(eccentricity)}, $l$ 是半通径,$p$ 是焦点到\textbf{准线(directrix)}的距离, 叫做\textbf{焦准距(focal parameter)}。 极角 $\theta$ 的取值范围是所有使 $r>0$ 的值。 具体来说, 当 $0 < e < 1$ 时, 圆锥曲线称为\textbf{椭圆(ellipse)}, $\theta$ 可以取任意实数; 当 $e = 1$ 曲线称为\textbf{抛物线(parabola)}, $\theta$ 可以取任意不等于 $2\pi n$ 的实数($n$ 取任意整数); 当 $e > 1$ 曲线称为\textbf{双曲线(hyperbola)}, 要求 $\theta_0< \theta + 2\pi n < 2\pi-\theta_0$, 其中 $n$ 取任意整数,
\begin{equation}
\theta_0 = \arccos\frac{1}{e}~.
\end{equation}
下文为了方便表述, 把 $\theta$ 的取值范围限制在一个圆周内, 即 $(-\pi,\pi]$ 或 $[0, 2\pi)$。

在一些文献中, 也把\autoref{eq_Cone_5} 中的负号写为正号, 此需要把\autoref{fig_Cone_2} 中的曲线旋转 $180^\circ$, 因为 $-\cos\theta = \cos(\theta - \pi)$。 上述 $\theta$ 的取值范围也需要加上 $\pi$。

\subsection{双曲线的两支}
\begin{figure}[ht]
\centering
\includegraphics[width=6cm]{./figures/3ca758fbfc998b7f.pdf}
\caption{双曲线分为互不相连的左右两支} \label{fig_Cone_3}
\end{figure}
根据双曲线的其他定义\upref{Hypb3}, 对同一个 $e>1$, 双曲线事实上是两条曲线, 每条曲线称为一支。 \autoref{fig_Cone_2} 中仅画出了离焦点较近的一支。 上文已经提到 $\theta_0< \theta < 2\pi-\theta_0$。

事实上\autoref{eq_Cone_5} 也可以表示双曲线的另一支, 只需要取 $-\theta_0< \theta < \theta_0$, 此时 $r$ 恒为负值。 若我们在极坐标中定义 $(-r, \theta)$ 和 $(r, \theta + \pi)$ 表示同一点, 就可以画出另一支。 或者说, 把\autoref{eq_Cone_5} 中的  $r,\theta$ 分别替换为 $-r$ 和 $\theta+\pi$ 就得到了这支双曲线的正常极坐标方程($r > 0$) 和极角范围
\begin{equation}\label{eq_Cone_6}
r(\theta) = -\frac{l}{1 + e\cos\theta} \qquad (\pi - \theta_0<\theta < \pi + \theta_0)~.
\end{equation}

\subsection{推导}

\begin{figure}[ht]
\centering
\includegraphics[width=8cm]{./figures/f04d0ae1f160e70c.pdf}
\caption{由离心率定义圆锥曲线}\label{fig_Cone_1}
\end{figure}

圆锥曲线的一种定义(与其他定义等效)为(\autoref{fig_Cone_1}):
平面上有一点 $O$ 和一条直线 $L$, 相距为 $p$。 
平面上某一点到 $O$ 的距离为 $r$, 到 $L$ 的
(垂直)距离为 $d$, 令常数 $e > 0$, 则所有满足
\begin{equation}\label{eq_Cone_1}
r/d = e~
\end{equation}
的点组成的曲线就是圆锥曲线。 $e$ 是常数,叫做\textbf{离心率}, $O$ 是\textbf{焦点}, $L$ 是\textbf{准线}。 当 $e = 0$ 时曲线是圆\footnote{注意根据定义, 圆的准线为无穷远, 所以只能使用\autoref{eq_Cone_5} 而不能用\autoref{eq_Cone_3}。 所以在\autoref{fig_Cone_2} 中, 圆的半径为无穷小。}, $0 < e < 1$ 时是椭圆, $e = 1$ 时是抛物线, $e > 1$ 时是双曲线。

以 $O$ 点为原点,使极轴垂直于准线(如上图)。则 $$d = p + r \cos \theta ~$$, 代入\autoref{eq_Cone_1} 得
\begin{equation}\label{eq_Cone_2}
\frac{r}{p + r \cos \theta } = e~.
\end{equation}
变形,得
\begin{equation}
r(\theta) = \frac{ep}{1 - e\cos \theta }~,
\end{equation}
此即为 \autoref{eq_Cone_3}。

若定义圆锥曲线的\textbf{通径}为过焦点且平行于准线的直线被圆锥曲线截出的线段长度。 记\textbf{半通径}为 $l$,则通径为 $2l$, 那么有 $r(\pi /2) = l$。 代入\autoref{eq_Cone_3} 得 $l = ep$。 所以\autoref{eq_Cone_3} 又可以写为
\begin{equation}\label{eq_Cone_4}
r(\theta)  = \frac{l}{1 - e\cos \theta }~,
\end{equation}
此即为\autoref{eq_Cone_5}。

注意 $p$ 和 $e$ 分别控制圆锥曲线的大小和形状。由于抛物线的 $e = 1$ 不变, 所以所有抛物线的形状都相同。
