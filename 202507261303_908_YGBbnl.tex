% 雅各布·伯努利(综述)
% license CCBYSA3
% type Wiki

本文根据 CC-BY-SA 协议转载翻译自维基百科\href{https://en.wikipedia.org/wiki/Jacob_Bernoulli}{相关文章}。

\begin{figure}[ht]
\centering
\includegraphics[width=6cm]{./figures/2704fab0dd45f809.png}
\caption{} \label{fig_YGBbnl_1}
\end{figure}
雅各布·伯努利\(^\text{[a]}\)(Jacob Bernoulli,也被称为英语中的詹姆斯或法语中的雅克;1655年1月6日[旧历 1654年12月27日]—1705年8月16日)是一位瑞士数学家。在莱布尼茨与牛顿的微积分优先权之争中,他站在戈特弗里德·威廉·莱布尼茨一方,是莱布尼茨微积分法的早期支持者,并为其作出了诸多贡献。作为伯努利家族的一员,他与其兄约翰·伯努利一道,是变分法的奠基人之一。他还发现了基本数学常数 $e$。然而,他最重要的贡献是在概率论领域,在其著作《概率艺术》中首次推导出了大数法则的初步形式。
\subsection{生平简介}
\begin{figure}[ht]
\centering
\includegraphics[width=6cm]{./figures/71d5c1d90164a5d8.png}
\caption{来自《学者通报》(1682年)的插图,其中刊登了对伯努利《新彗星系统尝试》的批评。} \label{fig_YGBbnl_2}
\end{figure}
雅各布·伯努利出生于瑞士联邦的巴塞尔,父系是新教香料商人世家,\(^\text{[4][5]}\)母亲则出身于一个从事银行业与城市治理的家庭。\(^\text{[6]}\)

遵从父亲的意愿,他最初学习神学并成为牧师。但与父母的期望相反,\(^\text{[7]}\)他也私下研习数学和天文学。1676年至1682年间,他游历欧洲各地,向当时的著名学者学习最新的数学与科学成果,其中包括约翰内斯·胡德、罗伯特·波义耳以及罗伯特·胡克的研究。在此期间,他还提出了一种关于彗星的理论,但该理论被证明是错误的。下图为1682年《学者通报》刊登的对伯努利彗星新体系尝试的批评:

伯努利返回瑞士后,自1683年起在巴塞尔大学教授力学。他的博士论文《三重问题的解法》完成于1684年,\(^\text{[8]}\)并于1687年正式出版。\(^\text{[9]}\)

1684年,雅各布·伯努利与朱迪思·斯图帕努斯结婚,育有两名子女。在这一时期,他也开始了卓有成效的研究生涯。旅行使他得以与当时许多杰出的数学家和科学家建立通信联系,并保持终生往来。在此期间,他深入研究了数学领域的新发现,包括惠更斯的《论掷骰游戏中的推理》、笛卡尔的《几何学》以及范·斯霍滕为其所作的补编。他还研读了巴罗与沃利斯的著作,这激发了他对无穷小几何的浓厚兴趣。在1684年至1689年之间,他也发现了许多后来构成《概率艺术》的核心成果。

据认为,伯努利于1687年被任命为巴塞尔大学数学教授,并在该职位上一直工作到去世。此时,他已经开始指导其弟约翰·伯努利学习数学。他们兄弟俩共同研读了莱布尼茨在1684年发表于《学者通报》的论文《极大极小值新方法》中提出的微积分理论。他们还学习了齐恩豪斯的著作。需要指出的是,莱布尼茨早期发表的微积分论文对当时的数学家来说十分晦涩难懂,而伯努利兄弟是最早尝试理解并应用莱布尼茨理论的人之一。

雅各布曾与弟弟在微积分的诸多应用上合作,但随着约翰的数学才能日益成熟,两人之间的合作逐渐演变为激烈的竞争。他们不仅在著作中互相抨击,还频繁地互相出难题来考验对方的技巧。\(^\text{[10]}\)到了1697年,兄弟关系已彻底破裂。

月球上的伯努利环形山即是以两兄弟的名字共同命名的。
\subsection{重要著作}
雅各布·伯努利的第一批重要贡献包括:1685 年发表的一篇关于逻辑与代数对比的小册子,1685 年关于概率的研究,以及 1687 年的几何研究。他的几何成果提出了一种通过两条垂线将任意三角形分成四个相等部分的构造方法。

到 1689 年,他已发表了关于无穷级数的重要研究,并在概率论中提出了大数法则。1682 年至 1704 年间,伯努利共发表了五篇关于无穷级数的论文。其中前两篇包含了许多结果,例如基本结论$\sum \frac{1}{n}$发散,伯努利认为这是他的新发现,实际上意大利数学家彼得罗·门戈利早在 40 年前就已证明,而尼科尔·奥雷斯姆更是在 14 世纪就已得出这一结论\(^\text{[11]}\)。伯努利无法求出$\sum \frac{1}{n^2}$的闭式解,但他证明了该级数收敛于小于 2 的某个有限值。后来由欧拉在 1737 年首次求出了该级数的确切极限。伯努利还研究了由复利计算引出的指数级数。

1690 年 5 月,伯努利在《学者通报》发表论文指出,等时线问题等价于求解一个一阶非线性微分方程。等时线,也称恒时降线,是指一粒子在重力作用下,无论从曲线上的哪个点出发,都会在相同时间内到达最低点的那条曲线。该问题曾在 1687 年由惠更斯研究过,1689 年又被莱布尼茨研究。在建立了这个微分方程后,伯努利利用我们现在称为“变量分离法”的方法对其进行了解。他在 1690 年的这篇论文对于微积分史具有重要意义,因为“积分”一词首次以“求积分”的含义被使用。1696 年,伯努利求解了现今称为“伯努利微分方程”的:
$$
y' = p(x)y + q(x)y^n.~
$$
伯努利还发现了一种确定一条曲线的渐屈线的一般方法,即作为其曲率圆包络的方法。他还研究了焦线,特别研究了抛物线、对数螺线和本轮线所对应的焦线,时间大约在 1692 年左右。伯努利八字线是雅各布·伯努利于 1694 年首次构想出来的。在 1695 年,他研究了吊桥问题,即:设计一条曲线,使得沿缆绳滑动的重物始终能保持吊桥处于平衡状态。
\begin{figure}[ht]
\centering
\includegraphics[width=6cm]{./figures/6db15ab09df38a9a.png}
\caption{《概率艺术》,1713年出版(米兰,曼苏蒂基金会藏)。} \label{fig_YGBbnl_3}
\end{figure}
伯努利最具原创性的著作是《概率艺术》,该书于他去世八年后的1713年在巴塞尔出版。这部作品在他去世时尚未完成,但在概率论的发展史上仍具有极其重要的地位。书中还涉及了其他相关主题,包括对组合数学的回顾,特别是对范·斯胡滕、莱布尼茨和普雷斯泰等人工作的评述;以及在指数级数的讨论中对伯努利数的运用。受惠更斯作品的启发,伯努利还给出了许多关于在各种赌博游戏中可以期望赢得多少的实例。“伯努利试验”这一术语正是源自该书。

在书的最后部分,伯努利概述了许多数学概率的研究领域,包括:概率作为确定性的可度量程度;必然与偶然;道德期望与数学期望之区别;先验概率与后验概率;按技巧水平划分玩家时的胜利期望;如何考虑所有可得论据、评估其价值并进行可计算的推理;以及大数法则。

伯努利是形式化高等分析方法最重要的倡导者之一。他在表达和行文中虽少有华丽与优雅,但其方法却展现了极高的严谨性与诚实精神。
\subsection{数学常数 e 的发现}
1683 年,雅各布·伯努利在研究一个关于复利的问题时发现了常数 e。他需要求解如下表达式的极限值(该值实际上就是 e):\(^\text{[11]}\)[12][13]
$$
\lim_{n \to \infty} \left(1 + \frac{1}{n} \right)^n~
$$

例如,假设一个账户初始金额为 1 美元,年利率为 100\%。如果利息在年末结算一次,最终金额为 2 美元;但如果一年结算两次,则本金每次乘以 1.5,得到:
$\$1.00 \times 1.5^2 = \$2.25$
按季度结算则为:
$\$1.00 \times 1.25^4 = \$2.4414\ldots$
按月结算为:
$\$1.00 \times (1.0833\ldots)^{12} = \$2.613035\ldots$

伯努利注意到,随着复利结算间隔越来越短,这个数列趋近于一个极限值(即利率的“作用力”)。按周复利时金额为 \$2.692597…,按日复利时为 \$2.714567…,仅多出两分钱。设每年复利 n 次,每次利率为 100\%/n,则当 n 趋近无穷时,该表达式的极限即为后来欧拉命名为 e 的数值。使用连续复利时,账户金额将达到 \$2.7182818\ldots\$。更一般地说,若账户初始为 1 美元,利率为 R,则连续复利下最终金额为:$e^R$
