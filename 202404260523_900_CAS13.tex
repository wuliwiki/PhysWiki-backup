% 中国科学院 2013 年考研普通物理
% keys 中科院|考研|普通物理
% license Copy
% type Tutor
\subsection{选择题}
1. 质点沿一固定圆形轨道运动,如果速率均匀增大,下列物理量中不随时间变化的是\\
(A) 法向加速度大小;\\
(B) 切向加速度大小;\\
(C) 加速度大小;\\
(D) 加速度与速度间的夹角。

2. 某电梯的天花板上竖直悬挂着弹性系数为 $k$ 的弹簧振子, 弹簧下端挂有一质 量为 $m$ 的物块, 则当电梯以匀加速 $a_{1}$ 上升和匀减速 $a_{2}$ 上升时, 弹簧和物块组成的系统振动频率分别为\\
(A) $\frac{1}{2 \pi} \sqrt{\frac{k}{m}},\frac{1}{2 \pi} \sqrt{\frac{k}{m}}$;\\
(B) $\frac{1}{2 \pi} \sqrt{\frac{k}{m}\left(1+\frac{a_{1}}{g}\right)},\frac{1}{2 \pi} \sqrt{\frac{k}{m}\left(1-\frac{a_{2}}{g}\right)} ;$\\
(C) $\frac{1}{2 \pi} \sqrt{\frac{k}{m\left(g+a_{1}\right)}},\frac{1}{2 \pi} \sqrt{\frac{k}{m\left(g-a_{2}\right)}} ;$\\
(D) $\frac{1}{2 \pi} \sqrt{\frac{k}{m\left(g-a_{1}\right)}},\frac{1}{2 \pi} \sqrt{\frac{k}{m\left(g+a_{2}\right)}} .$

3. 核电站的原子能反应堆中需要用低速中子维持缓慢的链式反应, 反应释放的 却是高速快中子。在反应堆中, 快中子通过与石墨棒内静止的碳原子 $\left({ }_{6}^{12} \mathrm{C}\right)$ 发生弹性碰撞而减速。已知, 碳原子质量是中子质量的 12 倍, 则一次碰撞前、 后中子动能比为\\
(A) $\left(\frac{5}{6}\right)^{2}$ $\quad$
(B) $\left(\frac{6}{5}\right)^{2}$ $\quad$
(C) $\left(\frac{11}{13}\right)^{2}$ $\quad$
(D) $\left(\frac{13}{11}\right)^{2}$ 

4. 关于理想气体, 以下表述不正确的是\\
(A) 气体分子本身的体积可以忽略不计, 分子与容器壁以及分子与分子之间 的碰撞属于完全弹性碰撞;\\
(B) 在相同的温度下, 气体分子的平均平动动能相同而与气体的种类无关;\\
(C) 气体的压强与分子数密度成正比, 与平均平动动能无关;\\
(D) 在相同的温度和压强下,各种气体在相同的体积内所含的分子数相等。

5. 有一点电荷A带正电量 $Q$, 距其不远处放入一个不带电的金属导体小球B, 平衡后, 电荷A的电势为 $U_{\mathrm{A}}$, 导体球 $\mathrm{B}$ 的电势为 $U_{\mathrm{B}}$, 无穷远处电势为 $U$ 。则以下关系正确的是\\
(A) $U_{A}>U_{B}>U$;\\
(B) $U_{B}>U_{A}>U$;\\
(C) $U_{A}>U>U_{B}$;\\
(D) $U_{B}>U>U_{A}$ 。

6. 真空中,两靠近的平行金属板分别带均匀的等量异号电荷。若板间左侧一半 空间充入介电常数为 $\varepsilon$ 的电介质,则以下说法错误的是 \\
(A) 充入电介质后, 板间电压减少;\\
(B) 充入电介质后,板间电容增加;\\
(C) 充入电介质后, 板间电场总能量减\\
(D) 充入电介质后,板间左侧电介质中的电场强度小于右侧真空中电场强度。

7. 真空介电常数 $\varepsilon_{0}$ 在国际单位制中的量纲是\\
(A) $L^{-2} M^{-1} T^{4} I^{2}$;$\quad$
(B) $L^{-2} M^{1} T^{3} I^{2}$;$\quad$
(C) $L^{-3} M^{-1} T^{4} I^{2}$;$\quad$
(D) $L^{-3} M^{-1} T^{3} I^{2}$ 。

8. 卢瑟福散射过程中 $\alpha$ 粒子以能量 $E$ 入射固定原子核靶, $\alpha$ 粒子电荷为 $Z^{\prime} e$, 原子核电荷为 $Z e, \alpha$ 粒子散射角为 $\theta$, 则以下卢瑟福散射截面 $\frac{\mathrm{d} \sigma}{\mathrm{d} \Omega}$ 与下列哪个物理量无关\\
(A) 核电荷$\quad$
(B) 能量 $E$ $\quad$
(C) 散射角$\quad$
(D) 靶原子密度。
\subsection{简答题}
1.  某人用左、右手的食指水平托住一根半米长的均质尺子,手指分别放在尺子的两端, 然后彼此缓慢相向移动, 向中心靠拢。结果发现,两手指总是交替地滑动。比如开始时, 只有右手食指在尺子下方滑动, 而左手食指与 尺子没有相对滑动; 然后变为只有左手食指在尺子下方滑动, 而右食指没有; 如此循环, 直至某根食指到达尺子中心。请解释此现象。

2. 空间中存在随时间变化的磁场,但不存在任何导体。该空间是否有感应电场?有没有感应电动势?并简述理由。

3. 什么是双折射现象?并给出现实中双折射现象的一个例子。
\subsection{解答题}
1。甲乙两人在水平冰面上玩推车游戏, 甲和乙的质量均为 $M$, 小 车质量为 $m$ 。开始时,甲和小车静止在同一地点,乙静止在另一处。而后甲将小 车朝着乙推去,小车相对于甲的速度大小为 $u(u>0)$ 。乙接到小车后又将其推向甲,小车相对于乙的速度大小也为 $u$ 。甲接到小车后, 又再次将小车推向乙, 如此继续下去。设冰面光滑, 且足够大。\\
(1) 乙第一次接到小车后将其推出, 若小车一定能追上甲, 求 $M, m, u$ 之间需要满足的关系式;\\
(2) 若甲接到小车后再将其推出, 小车恰好与乙的速度相同,求此时三者各自的运动速度大小。

2。如图所示, 有一水平夫花板上竖直悬挂两轻质弹簧, 其弹性系数均为 $k$, 原长均为 $l$ 。一质量为 $m$ 、长度为 $L$ 的均质杆在左右两个端点分别与两弹簧联接。已知初始时刻,该系统仅在重力作用下保持平衡, 杆水平。\\
(1) 将杆的左右两端下拉 $\Delta h$, 然后从静止释放, 求杆两端点运动规律;\\
(2) 将杆左端下拉 $\Delta h$, 右端上倠 $\Delta h$, 然后从静止释放, 求杆两端点运动规律;\\
(3) 将杆的右端柆 $\Delta h$, 右端保持不动, 然后从静止释放, 求杆两端点运动规律。 注: $\Delta h$ 为小量。

3。一长为 $L$ 的电容器由圆柱形导体和同轴导体圆筒构成, 圆柱导体半径为 $R_{1}$, 圆筒内半径为 $R_{2}\left(R_{2}>R_{1}\right)$ 。圆柱导体与圆筒间充满了介电常数为 $\varepsilon$ 的
均匀介质。设沿轴线方向, 圆柱导体上单位长度电荷为 $\lambda(\lambda>0)$, 圆筒上单位长 度电荷为 $-\lambda$ 。假设 $L>>R_{2}$ 。 求:\\
(1) 介质内的电位移 $\bar{D}$, 电场强度 $\bar{E}$ 及电极化强度 $\bar{P}$ 的大小和方向;\\
(2) 介质内、外表面的极化面电荷密度;\\
(3) 圆筒与圆柱导体间的电势差和该电容器的电容。

4。如图所示, 某空场强度大小为 $E$, 磁感应强度大小为 $B$ 。一个质量为 $m$ 的电静止状态被释放。忽略重力。试求电子在 $y$ 方向的最友位移。
\begin{figure}[ht]
\centering
\includegraphics[width=4cm]{./figures/935c8e140d9f783a.pdf}
\caption{解答题第4题图示} \label{fig_CAS13_1}
\end{figure}
5。单缝的夫琅禾费衍射实验中, 单缝宽度为 $a$, 缝后透镜焦距 $f=20 \mathrm{~cm}$ 。波长 $\lambda_1= 600 \mathrm{~nm}$ 的平行单色光垂直入射单缝, 所产生的夫琅禾费衍射图样的中央明条纹的线宽度为 $\Delta x_{1}=10 \mathrm{~cm}$ 。另一束波长 $\lambda_{2}$ 的单色平行光垂直入射该单缝时, 一级亮条纹的线宽度为 $\Delta x_{2}=4 \mathrm{~cm}$ 。求\\
(1) 该单缝的宽度 $a$;\\
(2) 波长 $\lambda_{2}$ 。
