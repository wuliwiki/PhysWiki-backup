% 质子
% license CCBYSA3
% type Wiki

(本文根据 CC-BY-SA 协议转载自原搜狗科学百科对英文维基百科的翻译)

\textbf{质子}属于亚原子粒子,符号为p或者$p^+$,携带为一个基本电荷,其质量略小于中子。质子和中子的质量约为一个原子质量,它们都被称为“核子”。

每个原子的原子核内均存在一个或多个质子,它们是核子的必要组成部分。原子核中质子的数量被称为原子序数(由符号Z表示)。因为每种元素的质子数量不一样,因此每种元素都有自己独特的原子序数。

1917年,欧内斯特·卢瑟福做实验发现,使用$\alpha$粒子撞击氮原子核,可以提取氢原子核。卢瑟福因此推断,氢原子核是氮原子核与所有更重的原子核的基础材料。由于这重要结果,卢瑟福被公认为质子的发现者。

在粒子物理学的现代标准模型里,质子是由两个上夸克与一个下夸克组成的强子。夸克的静质量只贡献出大约1\%质子质量,剩余的质子质量主要源自于夸克的动能与捆绑夸克的胶子场的能量。

因为质子是由三个夸克组成,质子可视为基本粒子,质子具有物理尺寸,但这尺寸并不能完美良好定义,由于质子的表面很模糊,因为这表面是由作用力的影响来定义,而这作用力不会突然终止。质子的半径(更仔细地说,电荷半径)大约为0.84到0.87飞米即$0.84\times10^{-15}$到$0.87\times10^{-15}$ m。

在足够低的温度下,自由质子将和电子结合。然而,这种结合并没有改变质子的性质。当一个质子快速移动通过物质时,它会和电子以及原子核相互作用,进而速度变慢,直至被原子的电子云俘获。因此质子化原子将被产生,它是氢的化合物。当真空中存在自由电子时,足够慢的质子可以和单个自由电子结合,成为中性的氢原子,其在化学上是自由基。在足够低的能量下,这种“自由氢原子”倾向于与许多其他类型的原子发生化学反应。当自由氢原子相互反应时,它们形成中性氢分子$(H_2)$。因此氢分子是星际空间中分子云最常见的分子成分。

\subsection{物理性质}
质子是自旋为-½的费米子,由三个价夸克组成,[1]因此是一种重子(强子的子类型)。两个上夸克和一个下夸克由强力和胶子共同作用而结合在一起.[2]现代物理学观点认为质子由价夸克(上、上、下)、胶子和短暂的海夸克对组成。质子具有近似指数衰减的正电荷分布,均方差半径约为0.8飞米。[3]

质子和中子都是核子,它们可以通过核力结合在一起形成原子核。氢原子最常见的同位素(化学符号为“H”)是一个孤质子。氢原子具有两个重氢同位素,分别为氘和氚:其中氘的原子核包含一个质子和一个中子,氚的原子核包含一个质子和两个中子。其他类型的原子核由两个或多个质子以及不同数量的中子组成。

\subsection{历史}
类氢粒子作为其他原子组成部分的概念是经过长时间发展起来的。早在1815年,基于早期原子量的数值,威廉·普鲁特认为所有原子均由氢原子组成(他称之为“原生系”)。当类氢粒子的原子量被更精确测量时,他的解释被推翻。[4]

1886年,欧根·戈尔德斯坦发现极隧射线(也称为阳极射线)并证实它们是从气体中产生的带正电荷的粒子(离子)。然而,由于来自不同气体的粒子具有不同的荷质比(e/m),因此它们不像约瑟夫·汤姆孙发现的电子一样用单个粒子来表示。威廉·维恩于1898年确认氢离子是电离气体中荷质比最高的粒子。[5]
\begin{figure}[ht]
\centering
\includegraphics[width=6cm]{./figures/4aab6b70ce560c2f.png}
\caption{欧内斯特·卢瑟福在第一届索尔维会议,1911} \label{fig_Proton_4}
\end{figure}
在欧内斯特·卢瑟福于1911年发现原子核后,安东尼乌斯·范登布罗伊克提出周期表中每个元素的位置(其原子序数)等于其核子电荷数。亨利·莫塞莱在1913年使用X射线光谱通过实验证实了这一点。
\begin{figure}[ht]
\centering
\includegraphics[width=6cm]{./figures/1fc65365905b6605.png}
\caption{在异丙醇中检测到质子 威尔逊云雾室} \label{fig_Proton_5}
\end{figure}
1917年(实验于1919年被报道),卢瑟福证明了氢核存在于其他原子核中,这一结果通常被描述为质子的发现。[6]卢瑟福早期认识到氢核是α粒子和氮气相互作用的辐射产物,并可以通过它们在空气中独特的穿透特征以及在闪烁探测器中的表现来进行识别。卢瑟福注意到,当α粒子被射入空气(主要是氮气)时,闪烁探测器显示了典型氢核的特征。通过追踪空气中氮的反应,卢瑟福发现当α粒子进入纯氮气时,效果更明显。卢瑟福确定这种氢只能来自氮气,因此氮气必须含有氢核。一个氮核被α粒子撞击后产生了氧核。这是首次被报道的核反应,$^{14}N + \alpha \to ^{17}O + p$ (这种反应后来在1925年的威尔逊云雾室被直接观察到)。

受普鲁特假设的影响,卢瑟福认为氢是结构最简单质量最轻的元素,即氢是所有元素的组成部分。由于发现氢核存在于其他所有原子核中并怀疑氢核只包含一种粒子,卢瑟福因此给氢核一个特殊的名称, 即质子,在希腊语中意为“第一”。然而,卢瑟福也想到了这个词曾被普鲁特所使用的。在1920年8月24日举行的英国科学促进会上,[7]欧里佛·洛兹要求卢瑟福为正氢核取一个新名字,以避免与中性氢原子混淆。卢瑟福使用质子和普劳顿(源自普鲁特)来进行区分的建议被会议接受。 因此氢核被命名为“质子”,取自普鲁特的词“protyle”。[8]科学文献中首次使用“质子”一词是在1920年。[9]

最近的研究表明,雷暴可以产生能量高达几十兆电子伏的质子。[10][11]

质子通常用于加速器质子疗法或者各种各样的粒子物理学实验,最有力的例子是大型强子对撞机。

在2017年7月的一篇论文中,研究人员测量了质子的质量1.007276466583+15−29 atomic mass units(括号中的值分别是统计和系统不确定性),比来自CODATA 2014的测量值低三个标准差。[12][13]

\subsection{稳定性}
自由质子(不与核子或电子结合的质子)是一种稳定的粒子,尚未被观察到自发分解为其他粒子。自由质子在许多情况下都是天然存在的。在这些情况下,能量或温度足够高,可以将它们与电子分开。它们对电子有一定的亲和力。自由质子存在于温度过高以至于不能与电子结合的等离子体中。高能和高速的自由质子占宇宙射线的90\%。在一些罕见类型的放射性衰变中,自由质子可以直接从原子核发射。质子(连同电子和反中微子)也可在不稳定自由中子的放射性衰变中产生。

由于未观察到自由质子的自发衰变,根据标准模型,质子被认为是稳定的粒子。然而,一些粒子物理学的大统一理论 (GUTs)预测,质子衰变的寿命应该在$10^{31}$年至$10^{36}$年间。实验研究已经确定了质子平均寿命的下限以及可能的衰变产物。[14][15][16]

日本的超级神冈探测器通过实验测得,质子衰变成反μ子和中性介子的平均寿命下限为$6.6\times10^{33}$年,衰变为正电子和中性π介子的平均寿命下限为$8.2\times10^{33}$年。[17]另一个实验来自于加拿大的萨德伯里微中子观测站。 这个实验用来检测任何衰变后的产物,通过从氧16质子衰变后的残余核子中检测到γ射线,测得质子寿命的下限为$2.1\times10^{29}$年。[18]

然而,质子也可通过电子俘获转化为中子(称为逆$\beta$衰变)。对于自由质子,这个过程不是自发产生的,而是在有能量提供时才会发生。等式是:
$$p^+ + e^- \to n +\nu_e~$$
这个过程是可逆的;中子可以通过$\beta$衰变转换为质子。$\beta$衰变是放射性衰变的一种常见形式。事实上,自由中子以这种方式衰变,平均寿命约为15分钟。

\subsection{夸克和质子的质量}
在现代核力理论的量子色动力学中,狭义相对论解释了质子和中子的大部分质量。质子的质量大约是它内部所有夸克静止质量之和的80-100倍,同时胶子的静止质量为零。相比在量子色动力学真空中所有夸克的静止能量,质子内同一区域中的夸克和胶子的额外能量几乎占了质子总重量的99\%。因此,质子的静止质量是运动夸克和胶子系统的不变质量。在这样的系统中,无质量粒子的能量作为系统静止质量的一部分也同样能被测量。

两个术语用于指代组成质子的夸克质量:净夸克质量指夸克本身的质量,而组夸克质量指净夸克质量加上其周围胶子粒子场的质量。[19] [20]这两个质量通常具备不同的值。如上所述,质子的大部分质量来自将夸克结合在一起的胶子,而不是夸克本身。虽然胶子没有质量,但它们拥有能量,即量子色动力学结合能,对质子的总质量贡献很大(参见狭义相对论中的质量)。质子的质量约为$938 MeV/c^2$,其中三个价夸克中的静止质量只贡献了大约$9.4 MeV/c^2$,其余的大部分能量来自于胶子的量子色动力学结合能 。[21] [22]

质子的夸克模型波函数组成为
\begin{equation}
\left|p_{\uparrow}\right\rangle = \frac{1}{\sqrt{18}} \left[ 2 \left| u_{\uparrow} d_{\downarrow} u_{\uparrow} \right\rangle + 2 \left| u_{\uparrow} u_{\uparrow} d_{\downarrow} \right\rangle + 2 \left| d_{\downarrow} u_{\uparrow} u_{\uparrow} \right\rangle - \left| u_{\uparrow} u_{\downarrow} d_{\uparrow} \right\rangle - \left| u_{\uparrow} d_{\uparrow} u_{\downarrow} \right\rangle - \left| u_{\downarrow} d_{\uparrow} u_{\uparrow} \right\rangle - \left| d_{\uparrow} u_{\downarrow} u_{\uparrow} \right\rangle - \left| d_
{\uparrow} u_{\uparrow} u_{\downarrow} \right\rangle - \left| u_
{\downarrow} u_{\uparrow} d_{\uparrow} \right\rangle \right]~
\end{equation}
质子的内部动力学很复杂,因为它们是由夸克的交换胶子和各种真空冷凝物相互作用决定的。晶格量子色动力学提供了一种理论上计算质子质量的方法,可以计算到任何精度。最近的计算[23][24]声称其精度可以优于4\%,甚至到1\%(参见杜尔中的图S5等等。[24])。这些说法仍然有争议,因为夸克质量的计算还未能达到像实际期待的那样轻。这意味着,通过外推过程进行的计算预测会引入系统误差。[25]很难说这些误差是否得到了适当的控制,因为这些数量是通过与已知的强子质量进行比较而来的。

这些计算通常由大型超级计算机完成的,正如博菲和帕斯奎尼所指出的:“核子结构的详细描述仍然缺失,因为长距离行为需要非扰动和/或数值处理”[26]更多研究质子结构的理论方法有:源于Tony Skyrme 和更精确的 AdS/QCD的拓扑孤子方法,扩展到包括胶子的弦理论,[27]各种量子色动力学启发的模型,如20世纪80年代流行的袋模型和组夸克模型,以及允许粗略近似质量计算的 SVZ求和规则。[28]这些方法目前还没有达到与量子色动力学相同的精度。

\subsection{电荷半径}
原子核(质子)半径的问题与原子半径问题类似,因为原子和原子核都没有明确的边界。然而,在电子散射实验中,原子核被模拟为正电荷球体,通过对电子散射所获得的一系列截面面积进行均方根计算,其所获得数值即为原子核半径。因为原子核横截面积与其半径的平方成正比。

国际上质子的电荷半径公认值是0.8768 飞米。该值基于对质子和电子的相关测量(即电子散射测量和基于罗森布鲁特动量转移截面方程的散射截面复杂计算),以及氢和氘的原子能级的研究。

然而,在2010年,一个国际研究小组发表了通过μ子氢中(由质子和负电荷的$\mu$子组成的奇特原子)的兰姆位移测量的质子电荷半径的结果。由于$\mu$子比电子重200倍,其相应的德布罗意波长更短,因此其较小的原子轨道对质子的电荷半径更敏感,所以能够获得更精确的测量。它们对质子的均方根电荷半径的测量是0.84184(67) 飞米,与的CODATA的0.8768(69) 飞米相差5.0个标准差。[29]2013年1月,质子电荷半径的更新值,0.84087(39) 飞米,发表了。其精度提高了1.7倍,并将误差提高到7个希格玛。[30]2014年CODATA略微降低质子半径的推荐值(仅使用电子测量计算)至0.8751(61) 飞米,但这将误差扩大到了5.6个西格玛。

在菲利根的保罗·舍勒研究所获得这一结果的国际研究团队来自马克斯·普朗克量子光学研究所、路德维希-马克西米利安大学、斯图加特大学斯特拉斯韦克泽格研究所和葡萄牙科英布拉大学的科学家。[31][32]该团队正试图解释这种差异,并重新检查以前高精度测量和涉及散射横截面积的复杂计算。如果在测量或计算中没有发现错误,可能有必要重新检查世界上最精确和测试最好的基础理论:量子电动力学。[31]截至2017年,质子半径仍然是一个谜。[33]也许这种差异是由新物理学引起的,或者可以被一种忽略的普通物理效应所解释。[34]

质子半径与形状因子和动量传递截面有关。原子形状因子$G$改变了对应于点状质子的横截面。
\begin{equation}
R_e^2 = -6 \left. \frac{d G_e}{d q^2} \right|_{q^2 = 0}~
\end{equation}
\begin{equation}
\frac{d \sigma}{d \Omega} = \left. \frac{d \sigma}{d \Omega} \right|_{\text{point}} G^2(q^2)~
\end{equation}
原子形状因子与目标的波函数密度有关:
\begin{equation}
G(q^2) = \int e^{i \mathbf{q} \cdot \mathbf{r}} \psi(r)^2 \\, d^3r~
\end{equation}
形状因子可以分为电动和磁性形状因子。这些可以进一步写成狄拉克和泡利形状因子的线性组合。
\begin{equation}
G_m = F_D + F_P~
\end{equation}
\begin{equation}
G_e = F_D - \tau F_P~
\end{equation}
\begin{equation}
\frac{d \sigma}{d \Omega} = \left. \frac{d \sigma}{d \Omega} \right|_{\text{NS}} \frac{1}{1 + \tau} \left( G_e^2(q^2) + \frac{\tau}{\epsilon} G_m^2(q^2) \right)~
\end{equation}

\subsubsection{5.1 质子内部的压力}
因为质子由被胶子包围的夸克组成,所以作用在夸克上的等效压力可以被定义。这允许使用高能电子的康普顿散射从中心计算它们作为函数距离的分布(DVCS,对于深度虚拟康普顿散射)。质子中心的压力最大,其数值约为$10^{35}$帕且大于中子星内部压力。[35]在小于约0.6飞米的径向距离内,压力是正的(排斥的)。在更大的距离时,压力是负的(吸引的)。在距离超过大约2飞米时,压力变得非常弱。

\subsubsection{5.2 溶剂化质子中的电荷半径,水合氢离子}
水合质子的半径出现在玻恩方程用于计算水化焓水合氢离子。

\subsection{自由质子与普通物质的相互作用}
虽然质子对带相反电荷的电子有亲和力,但这是一种相对低能的相互作用,因此自由质子必须失去足够的速度(和动能),才能与电子紧密结合。高能质子在穿过普通物质时,通过与原子核的碰撞和原子的电离(除去电子)失去能量,直到它们被正常原子中的电子云俘获。

然而,在与电子的这种关联中,结合质子的性质没有改变,它仍然是一个质子。低能自由质子对正常物质中任何电子(如正常原子中的电子)的吸引力导致自由质子停止并与原子形成新的化学键。这种键发生在任何足够“冷”的温度(即相当于太阳表面的温度)和任何类型的原子。因此,在与任何类型的正常(非等离子体)物质相互作用时,低速自由质子被它们接触的任何原子或分子中的电子吸引,导致质子和分子结合。这样的分子被称为“质子化的”,并且在化学上它们经常因此变成所谓的布朗斯特酸。

\subsection{化学中的质子}
虽然质子对带相反电荷的电子有亲和力,但这是一种相对低能的相互作用,因此自由质子必须失去足够的速度(和动能),才能与电子紧密结合。高能质子在穿过普通物质时,通过与原子核的碰撞和原子的电离(除去电子)失去能量,直到它们被正常原子中的电子云俘获。

然而,在与电子的这种关联中,结合质子的性质没有改变,它仍然是一个质子。低能自由质子对正常物质中任何电子(如正常原子中的电子)的吸引力导致自由质子停止并与原子形成新的化学键。这种键发生在任何足够“冷”的温度(即相当于太阳表面的温度)和任何类型的原子。因此,在与任何类型的正常(非等离子体)物质相互作用时,低速自由质子被它们接触的任何原子或分子中的电子吸引,导致质子和分子结合。这样的分子被称为“质子化的”,并且在化学上它们经常因此变成所谓的布朗斯特酸。






