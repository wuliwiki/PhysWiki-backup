% 单群
% 单群|正规子群|有限单群|正规子群列|合成序列|composition sequence

\pentry{群的直积和半直积\upref{GrpPrd}}

\subsection{单群和各种序列}

单群可以和素数类比.我们知道,任何一个非平凡的整数(即除了 $0$ 和 $1$)都可以分解为素数的乘积,素数是正整数的“砖石”.类似地,任何一个具有正规子群的群都可以分解为其正规子群和另一个群的半直积,而不存在\textbf{非平凡}正规子群的群就可以类比为群的“砖石”.这样的砖石被称为\textbf{单群}.

\begin{definition}{单群}
如果一个非平凡群 $G$ 不存在正规子群,那么称其为一个\textbf{单群(simple group)}.
\end{definition}

正如我们不把 $1$ 视作素数,平凡群也不被视作单群. 

群的分解和数字的分解有很多类似之处,但依然有很多不同,因此本节中使用的类比更多地是为了方便建立直观印象,而非对号入座,请注意.

每一个非平凡整数都可以通过因子分解,一步步拆成素因子更少的数,直到得到素数.比如说,$120$ 可以拆成 $4 \times 30$, $30$ 可以拆成 $2 \times 15$, $15$ 再拆成 $3 \times 5$, $3$ 和 $5$ 就无法再拆了.群也类似,可以把群拆成自己的正规子群和子群的半直积.

\begin{definition}{正规子群列}
给定群 $G$,如果有 $n+1$ 个群 $\{G_i\}_{i = 1}^n$,满足 $\{e\} = G_0 \triangleleft G_1 \triangleleft \cdots \triangleleft G_n = G$,即满足 $G_{i}$ 都是 $G_{i+1}$ 的正规子群,称之为一个\textbf{正规子群序列(subnormal series)},$G_{i+1} / G_{i}$ 被称为该序列的\textbf{因子群(factor group)}, $n$被称为该序列的\textbf{长度(length)}.
\end{definition}

我们有 $G_{i + 1} = G_{i} \$

\addTODO{factor group的翻译待定}
\addTODO{统一一下“正规子群列”和“正规子群序列”,我更倾向序列,不过如果有官方翻译更好}

注意,正规子群列只要求相邻的两个群具有“正规子群”的关系,而 $G_{i+2}$ 不一定是 $G_i$ 的正规子群.

如果序列满足任意的 $G_{i} \neq G_{i+1}$,(等价的,所有因子群 $G_{i+1} / G_{i}$ 非平凡),该序列被称为\textbf{无重复的}.

\addTODO{“无重复的”不确定是否为官方翻译}

有时候,无重复的正规子群列的相邻两项中间可以再插入一个群,使之仍然成为一个正规子群列.这样的行为有时被称为“加细”.如果一个正规子群列已经无法加细,那么我们称之为群 $G$ 的一个\textbf{合成序列(composition sequence)}.

类比到非负整数的质因数分解,合成序列就好比是每次只拿走一个质因子,序列的每一项都比上一项少一个质因子.非负整数的“合成序列”不一定唯一,但是所有合成序列的长度一定是一样的.群的合成序列也类似,合成序列不唯一,但其长度必然是一样的.这就是以下Jordan-Holder定理在群论中的情形:

\begin{definition}{Jordan-Holder定理(有限群)}
若一个群 $G$ 有两合成列 $\{e\} = G_0 \triangleleft G_1 \triangleleft \cdots \triangleleft G_r = G$ 和 $\{e\} = H_0 \triangleleft  H_1 \triangleleft \cdots \triangleleft H_s = G$,那么必有 $r=s$.
\end{definition}

因此我们可以把群 $G$ 的\textbf{长度(length)}定义成它的合成序列的长度.
\addTODO{此处Jordan-Holder定理没有证明和引申}
\subsection{有限单群}

有限群的研究重点是有限单群,知道了所有有限单群的性质也就能相应推出任意有限群的性质.但我们是否知道所有有限单群了呢?答案是肯定的.事实上,目前的研究表明,全体有限单群都可以归入特定的几个类别里,如果这是真的,那么我们确实已经知道了所有有限单群.

\begin{definition}{有限单群分类定理\footnote{可参见http://www.ams.org/notices/200407/fea-aschbacher.pdf.}}
如果 $G$ 是一个单群,那么 $G$ 必属于以下几类群中的一个:
\begin{itemize}
\item $\mathbb{Z}_p$,其中 $p$ 是素数;
\item 交错群 $A_n$,其中 $n\geq 5$;
\item 典型李群;
\item 例外、缠绕李群;
\item 26个散在单群.
\end{itemize}
\end{definition}

有限单群的分类定理的证明历时长且极为复杂,汇聚了历代无数数学家的工作.这庞杂的证明在过去曾被指出有漏洞,尽管这些漏洞都已经被补上了,但过于复杂的证明和毫无美感的分类结果还是让人质疑其正确性.事实上,当代数学家仍然在尝试给出更简洁的证明,截至2019年已经有八册证明发表,预计完成后约5000页证明.












