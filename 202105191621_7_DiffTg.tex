% 微分(切空间之间的映射)
% 切空间|微分几何|流形|tangent space|manifold|浸入|浸没|submersion|immersion

\pentry{流形\upref{Manif}, 光滑映射\upref{SmthM}}


\subsection{微分}

给定两个流形间的光滑映射,也就给定了流形上的道路之间的映射.由于道路等同于切向量,我们还可以导出一个\textbf{切空间之间}的映射,这个映射就被称为\textbf{微分}.

\begin{definition}{微分}
对于两个流形$M$和$N$,给定它们之间的一个光滑映射$f:M\to N$.对于$p\in M$处出发的一条道路$v:I\to M$,我们可以导出$f(p)\in N$处出发的道路$f\circ v:I\to N$.

于是我们可以定义映射$\dd f_p: T_pM\to T_{f(p)}N$,使得对于任意$v\in T_pM$,有$\dd f_p(v)=f\circ v$.称$\dd f$为$f$在点$p$处的\textbf{微分(differentiatial)}\footnote{注意英文术语,不是数学分析中的differentiation.}.
\end{definition}

$\dd f_p$是切空间之间的线性映射,因此如果给定了$p\in M$和$f(p)\in N$的局部坐标系(图)以后,也就能顺便导出$T_pM$和$T_{f(p)}N$的坐标系,从而可以把$\dd f_p$表示成一个矩阵——它就是$f$的Jacobi矩阵.这也是为什么我们使用“微分”这个术语来称呼它.

由于可以用矩阵来表示微分,自然就有了\textbf{秩(rank)}的概念,参见线性代数中“矩阵的zhi'xu”















