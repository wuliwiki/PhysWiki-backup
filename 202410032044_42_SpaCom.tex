% 空间的完备化
% keys 完备化|度量空间
% license Usr
% type Tutor

有理数集构成的度量空间(度量为差的绝对值)是稠密的,但不是完备的,将有理数构成的度量空间完备化则得到实数集度量空间。类似的,在一般的度量空间中,就有这样的问题出现:若度量空间 $X$ 不完备,可否将它完备化?即可否在 $X$ 中加入一些点,而保留度量的定义,使得得到的新的度量空间是完备的。答案是肯定的,这由本节所谓的度量空间的完备化得到保证。

\subsection{完备化}
\begin{definition}{完备化}
设 $X$ 是度量空间,$X^*$ 是完备度量空间。如果
\begin{enumerate}
\item $X$ 是空间 $X^*$ 的子空间;
\item $X$ 在 $X^*$ 中处处稠密(\autoref{def_MaDen_2}),即 $[X]=X^*$。
\end{enumerate}
则称 $X^*$ 为 $X$ 的\textbf{完备化}(completion)。
\end{definition}

\begin{theorem}{}
任一度量空间 $X$ 都有完备化,并且,在不区分使 $X$ 中的点保持等距的映射的意义下,这个完备化是唯一的。
\end{theorem}

\textbf{证明:}
首先证明唯一性。需要证明:若 $X_1$ 和 $X_2$ 是空间 $X$ 的两个完备化,那么存在空间 $X_1$ 和 $X_2$ 的一一映射 $\phi$,使得
\begin{enumerate}
\item 对一切 $x\in X$,$\phi(x)=x$;
\item 对任意 $x_1,y_1\in X_1$,成立 $d_1(x_1,y_1)=d_2(\phi(x_1),\phi(y_1))$,其中 $d_1,d_2$ 分别是 $X_1,X_2$ 的距离。
\end{enumerate}
映射 $\phi$ 可以如下构造: $\forall y_1\in X_1$。根据完备化的定义,存在 $X$ 中的点列 $\{x_n\}$,其收敛于 $y_1$。因为收敛点列是柯西序列(\autoref{the_cauchy_2}),且 $X_2$ 完备,因此存在 $X_2$ 中的点 $y_2$,使得 $\{x_n\}$ 收敛于 $y_2$。显然 $y_2$ 和选择收敛到 $y_1$ 的点列 $\{x_n\}$ 无关(否则)。

\textbf{证毕!}

