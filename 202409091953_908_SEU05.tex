% 东南大学 2005 年 考研 量子力学
% license Usr
% type Note

\textbf{声明}:“该内容来源于网络公开资料,不保证真实性,如有侵权请联系管理员”

\subsection{[15分]}
试说明,对于任意势垒,粒子的反射系数 $R$ 及透射系数 $D$ 满足
\[    R + D = 1 \quad (\text{取} E > V_0) ~ \]
\subsection{[20分]}
利用归一化条件计算谐振子的基态能量。
\subsection{[10分]}
证明不存在非0的二维矩阵,能和三个泡利矩阵都反对易,即设 $\vec A \delta + 6\vec A = 0$, 则 $\vec{A} = 0$。
\subsection{[20分]}
设粒子处于范围 $[0, \alpha]$ 的一维无限深势阱中,已知用波函数
    \[    \psi(x) = \frac{4}{\sqrt{\alpha}} \sin\left(\frac{\pi x}{\alpha}\right) \cos^2\left(\frac{\pi x}{\alpha}\right) ~ \]
    描写,求粒子能量的可能测量值及相应几率。
\subsection{[25分]}
荷电子的谐振子,受到外电场 $\xi$ 的作用,
    \[    V(x) = \frac{1}{2} m \omega^2 x^2 - q E_0 x  ~\]
 试能量本征值和本征函数。
\subsection{[25分]}
设带电粒子在互相垂直的均匀磁场与电场中的运动, 求能量本征值和本征函数。
\subsection{[15分]}
设氢原子处于 $n=2$ 能级, 求它的 Stark 分裂(不考虑自旋)。
\subsection{[20分]}
荷电粒子的动量在平衡位置附近做小振动(简谐振动)。受到光照射而发生跃迁,设照射光的能量密度为 $P(\omega)$,说明该跃迁:
    \begin{enumerate}
        \item 跃迁选择定则;
        \item 设离子原来处于基态求每秒跃迁到第一级激发态的几率。
    \end{enumerate}

提示:

1. 教子公式 $$\cos\theta Y_{l,m} = \sqrt{\frac{(l+1)^2 - m^2}{(2l+1)(2l+3)}} Y_{l+1,m} + \sqrt{\frac{l^2 - m^2}{(2l-1)(2l+1)}} Y_{l-1,m}~$$

2. 跃迁速率公式 $$\omega_{n'l'nl} = \frac{4 \pi^2 e^2}{3 \hbar^2} \left| r_{n'l'nl} \right|^2 P(\omega_{n'l'nl})~$$