% SQL 入门语法

\begin{issues}
\issueDraft
\end{issues}

\subsection{SQL 基本语法}

\subsubsection{SQL 库操作}
\subsubsection{1.结构创建}
\begin{itemize}
\item CREATE 结构类型 结构名 结构描述;

\end{itemize}
\subsubsection{2.显示结构}
\begin{itemize}
\item 显示结构: show 结构类型(复数)
\item 显示结构创建详情: show create 结构类型 结构名;
\end{itemize}
\subsubsection{3.数据操作(数据表)}
\begin{itemize}
\item 新增数据: INSERT INTO 表名 VALUES
\begin{lstlisting}[language=bash]
例如:向学生表中插入学生数据:张三 231119001  男 23
INSERT INTO students
VALUES
('张三','231119001','男','北京');
\end{lstlisting}

\item 查看数据:SELECT FORM 表名 
\begin{lstlisting}[language=bash]
例如:从学生表查询名字为张三的学号(具体查询)
SElECT ID
FROM students
WHERE name='张三';


例如:从学生表查询姓名为张的学号(模糊查询)
// % 表示任意多个字符
//_ 表示任意多个任意字符

SELECT ID
FROM students
WHERE  name like '张%' ;
或者
SELECT ID
FROM students
WHERE  name like '张_' ;
\end{lstlisting}


\item 更新数据: UPDATE 表名 SET
\begin{lstlisting}[language=bash]
例如:在学生表中,将学生张三的id改为231119002
UPDATE students 
SET id='231119002'
WHERE name='张三';



另外,SET后面除了带文本之外,还可以是数学表达式

例如:将学生id为231119002的学生对应的年龄加2
UPDATE students 
SET age=age+2
WHERE id='231119019'
\end{lstlisting}

\item 删除数据:DELETE FROM 表名
\begin{lstlisting}[language=bash]
例如:删除students表中名字为张三的信息

DELETE FROM students 
WHERE name='张三';
\end{lstlisting}


\end{itemize}

\subsection{数据库操作}
\subsubsection{创建数据库:根据项目需求创建一个存储数据库的仓库}
\begin{itemize}

\item 使用create database 数据库名字创建
\item create database 数据库名字
\begin{lstlisting}[language=bash]
create database db_1
\end{lstlisting}

\end{itemize}
\begin{enumerate}
\item 数据库的创建是存储数据库的基础,数据库的创建通常是一次性的
\item 创建数据库的语法包含几个部分
\begin{itemize}
\item 关键字 create database
\item 数据库名字:自定义名字
\item  数字,字母和下划线组成
\item 数字不能开头
\end{itemize}
\item  创建好的数据库可以在数据存储指定地点看到



\end{enumerate}

\subsubsection{显示数据库:通过客户端指令来查看已有的数据库}
\begin{enumerate}
\item 数据库的查看是根据用户权限限定的
\item 数据库的查看分为两种查看方式:
\begin{itemize}
\item 查看全部数据库:show database;
\begin{lstlisting}[language=bash]
show  database;
\end{lstlisting}
\item 查看数据库创建指令:show crete database 数据库名字;
\begin{lstlisting}[language=bash]
show crete database db_1;
\end{lstlisting}
\end{itemize}

\end{enumerate}


\subsubsection{使用数据库}
\begin{itemize}
\item 使用数据库的指令是:use 数据库名字;
\begin{lstlisting}[language=bash]
use db_1;
\end{lstlisting}
\item 使用数据库的相关库选项

\end{itemize}
