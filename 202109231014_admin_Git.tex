% Git 笔记
% Git|GitHub

\pentry{GitHub Desktop 的简单使用\upref{GitHub}}

一个简单美观的 GUI 客户端是 GitHub Desktop\upref{GitHub}(未必需要使用 GitHub)

\addTODO{把我 git 笔记上的东西都放上来}

\begin{itemize}
\item 考虑到 `git status` 和 `git diff` 太常用, 可以用 `ctrl+r stat` 或者 `ctrl+r dif` 等. 如果不对就再按一次 `ctrl+r` 即可. 任何常用命令都可以这样.
\item 快速 add 并 commit 可以用 `git commit -am "标题"`
\end{itemize}

* 
* 

## 新安装设置
* 要设置用户名, 用 `git config --global user.name "myname"`
* 要设置 commit 邮箱, 用 `git config --global user.email "email"`  邮箱决定 commit 的作者.
* 用网址 clone 的话, 每次链接服务器需要输入用户名和密码(token), 要储存账户密码, 用 `git config --global credential.helper store` 如果 github 修改了密码, 再用一次该命令即可, 要测试是否需要密码, 用 `git pull`
* 清空账号密码用 `git config --global --unset credential.helper`. 如果还不行的话只好把下文的所有 config 文件删除并重装 git
* token 是命令行的密码, 和 GitHub 的密码是不同的, 创建 token 参考 https://docs.github.com/en/github/authenticating-to-github/keeping-your-account-and-data-secure/creating-a-personal-access-token#creating-a-token
* 要查看 token 登录 github 后用 https://github.com/settings/tokens/
* Windows 下设置 `git config --global core.autocrlf true`, 这样 commit 的时候会把所有 CRLF 换行变为 LF, 而 checkout 的时候会把所有 LF 变为 CRLF. linux 下设置 `git config --global core.autocrlf input`, 这样 commit 的时候会把所有 CRLF 变为 LF, 而 chekout 的时候不做任何操作. 如果这么设置了, GitHub Desktop 产生的 .gitattribute 文件就可以删掉了 (因为解决的是同一个问题).
* 如果中文路径显示乱码(反斜杠加数字), 用 `git config --global core.quotepath off`
*  `git config --global core.editor "vim"` 可以把用于 commit 的编辑器设为 vim

## 笔记
* Windows 的 Git Bash 相当于 linux 的命令行, 有当前路径 `pwd`, 更改路径用 `cd`.
安装后的设置
* 为了防止大文件传输失败, 使用 `git config --global http.postBuffer 2000000000` (字节)
* `git clone` 时在后面加上 `--depth N` 选项可以只下载前几次的 commit
* `git clone <url> newname` 会把本地 repo 文件夹命名成 `newname`
* git 会无视隐藏目录(`.` 开头的), `git status` 也不显示, `git clean` 也删不掉
* `git status` 只会显示 worktree 和 index 之间的区别, 而不一定是最新 commit 之间的(对于服务器上被 push 的 repo 而言)
* `git status 目录` 可以只显示某个目录的 status
* 每个 repo 的设置文件是 `.git/config`. 用户的 `.gitconfig` 文件可以在 home 目录 (~) 找到 (对当前用户生效), `git config --global` 的设置全部存在这里. 系统的设置文件可以在 linux 的 `/etc/gitconfig` 或 Windows 的 `/C/ProgramData/Git/config` 中找到 (对所有用户生效). repo 中的设置覆盖用户设置覆盖系统设置.
* `git config --list` 可以列出当前目录的所有 config, 注意如果有设置多次出现, 后出现的为准. `git config --show-origin <xxx.xxx>` 可以显示某个 config 所在的文件.
* `.gitattributes` 文件也有用户(而不是 repo) 的版本, 可以用 `~/.config/git/attributes` 文件设置.
* 要把某个路径变为一个 repository, 先 cd 到该路径, 然后 git init 即可. 该命令会在当前目录创建 .git 文件夹, 里面包含所有历史信息.
* 所有 config 和 attribute 文件对 GitHub Desktop 同样有效.
* 要从服务器 clone 一个 Repository, 先获取网址, 然后用 `git clone https://...` 即可. git 会 clone 到当前目录下.
* 如果 clone 了以后用 `git status` 发现所有的文件都改变了, 用 `git config core.fileMode false` 即可 (每个 repo 都要使用)
* `git clone <folder1> <folder2>` 把 folder1 中的 repo 复制到 folder2 中. 但注意路径要用 `/` 而不是 `\`. 如 `/C/Users/addis/Documents/GitHub/cn2D/`
* `git clone --no-checkout ...` 选项可以只复制 .git 文件夹.
* Git bash 第一次 push 到 GitHub 的时候会弹出登录框, 密码被存在 Windows 的 Credential Manager 中的 git:https://github.com.
* 检查当前目录的 repository 的状态, 用 `git status`. 
* 用 `git add <file>` 把变化添加到 staging area (也就是 index). 用 `git add <dir>/` 添加目录, 用 `git add .` 添加所有变化.
* 用 `git add -u` 可以只 add 修改的和删除的文件, 新文件不 add
* `.gitignore` 的例子: `folder/` 忽略整个文件夹, `folder/*.txt` 忽略文件夹中的所有 txt 文件, `folder/**/*.pdf` 忽略 folder 文件夹和所有子文件夹中的 pdf 文件.
* `git diff` 用于显示文件夹和 stage area 中的区别. `git diff --staged` 用于显示 stage area 和上一次 commit 的区别. git diff --name-only  可以只显示文件名.
* `git diff` 的 `--word-diff` 选项可以显示行内的细微差别, 当行比较长的时候使用(如 latex 段落). github 中有时候也不会显示具体位置. 如果还想更精确, 用 `--word-diff-regex=.`
* `git diff branchA branchB -- <filename>` 用于比较两个 branch 中同一个文件.
* `git commit` 用于自动打开文本编辑器信息并 commit. `git commit -m "message"` 直接把 message 作为 commit 信息. `git commit -a` 可以自动 git add 所有变化的文件, 也可以在后面再加 `-m "message"`.
* commit 时编辑器中第一行是 title, 空一行, 然后写 description
* 要删除一个文件, 用 `git rm <file>`. 这样做相当于手动删除文件, 再 stage 该变化. 如果文件已经 stage 了, 用 `git rm -f <file>`.
* 要重命名文件, 用 `git mv <file_from> <file_to>`, 相当于 `git rm <file_from>` 再 `git add <file>`.
* 要浏览所有 commit 历史, 用 `git log`. 要显示每次 commit 的 diff, 用 `git log -p`, 要只显示前两次, 用 `-2`. 用 `--stat` 可以显示基本统计(几个+几个-).
* `git log -- <filename>` 可以只显示一个一个文件的历史, `-p` 和 `--word-diff-regex` 同样有效
* 要浏览某个文件的历史, 用 `git log -- <filename>`, 
* `git commit --amend` 可以覆盖上一次 commit (如果上一次 commit 忘了做什么事情).
* `git reset` 可以 unstage 所有文件, `git reset HEAD <file>` 可以 unstage 一个文件.
* `git checkout -- <file>` 可以撤销一个文件从上次 commit 后的变化 (危险操作!). `git checkout .` 可以撤销所有文件从上次 commit 后的变化 (危险操作!) 但是他们不会影响 untracked files
* 注意 `git checkout folder` checkout 的是 index, 也就是 git add 保存的地方. 而对于服务器上被 push 的 repo, index 和 HEAD 的版本很可能是不同的. 这时用 `git checkout master -- ...` 的才是最新版本.
* `git checkout <hash> -- <file1> <file2>...` 可以将某个文件还原到某次 commit 的状态. 同样也适用于文件夹. 这里的 `<hash>` 可以是某个 branch 的名字
* `git --work-tree=/path/to/outputdir checkout ...` 可以指定 checkout 的路径, 对 `status` 等命令同样有效
* `git --git-dir /path/to/repo 命令` 可以指定 `.git` 所在的文件夹而无需先 cd 到那个文件夹
* 【不推荐】`GIT_WORK_TREE=/some/path/ git checkout ...` 可以把任何 `git checkout` 的内容输出到指定文件夹, 注意这里的 `GIT_WORK_TREE=/some/path/` 只有本次有效
* 【不推荐】(建议使用上一条) `git checkout-index -a -f --prefix=/some/path/` 可以把所有最新版本的所有内容从 `.git` 文件夹 checkout 到指定的文件夹, 注意最后的 `/` 是必须的, `-a` 表示全部文件, `-f` 表示覆盖已有文件. 如果要 checkout 某个文件, 在后面用 `-- 文件名` 即可, 要把 `-a` 去掉. 这种方法不能 checkout 某个文件夹
* `git mv <old_name> <new_name2>` 可以把某个文件重命名, 包括只改变大小写(windows 中也可以, 直接在文件夹中改变大小写是无效的)
* `git remote` 显示当前 repository 的 remote server shortname. 要显示网址, 用 `-v`.
* `git remote add <shortname> <url>` 添加一个 remote server (一个 repo 可以有多个 server).
* `git fetch <shortname>` 从服务器上下载本地没有的数据.
* `git pull` 的功能是 fetch 然后 merge.
* `git remote show <shortname>` 显示 remote 的信息.
* `git remote rename <oldname> <newname>` 可以修改服务器的名字.
* `git remote remove <shortname>` 删除和服务器的连接 (并不会删除服务器上的 repo)
* `git push` 可以把当前 branch 的变化上传到服务器. 后面可以加某次 commit 的编号如 `abcd:master`, 或者如 `HEAD~1:master`.
* 如果从服务器上 `git clone` 的是某个空 project, 第一次 commit 后上传用 `git push -u origin master`
* `git pull` 可以从服务器下载当前 branch 的变化.
* `git branch <name>` 创造一个 new branch.
* git show-branch 可以查看已有的 branch
* `git checkout <name>` 把文件夹中的文件切换成 `<name>` branch 中的文件. 但是新的修改和 staging area 中的修改也会移植到 `<name>` branch 中. 如果不想移植, 就先 commit.
* `git checkout <name>` 也可以用于从服务器上下载本地不存在的 branch.
* `git checkout <id>` 可以把工作区间的所有文件修改到某次 commit 的状态, 且不会影响当前的任何 branch. 这时会进入 'detached HEAD' 状态, 所做的任何修改和 commit 在退出该状态后都会消失.
* `git checkout master` 可以从 `git checkout <id>` 中恢复
* `git merge <name>` 把 `<name>` branch 并入到当前 branch 中.
* 如果 merge 的过程中出现 conflict, 用 `git merge --abort` 可以恢复到 merge 以前.
* 只要两个 branch 修改的不是同一个文件的同一行, merge 都不会有问题. 如果是, 就会产生 conflict
* `git branch -d <name>` 用于删除某个 branch.
* `git branch --set-upstream-to=origin/master` 用于把 pull 和 push 的默认 remote 设置为 origin (或其他)
* `git push <remote> --delete <branch>` 用于删除 server 上的 branch.
* `git push --set-upstream <server> <branch>` 用于把某个 branch 推送到 server 上并 track.
* `git branch -vv` 可以检查当前的默认 remote/branch, 用于 `push/pull`
* `git branch --set-upstream-to <remote>/<branch>` 可以将某个 remote 作为默认
* `git branch <name> <id>` 用于从指定的 commit 创建 branch.
* `git clone -b <branch> --single-branch <url>` 用于从 url 克隆一个指定的 branch.
* `git reset --hard` 用于把 working directory 和 staging area 还原到最近一次 commmit 的状态.
* `git reset HEAD~1` 等效于 `git reset --mixed HEAD~1`, 用于撤销最近一次 commit,文件夹中的文件不改变,改变的文件变为 changed not staged 的状态
* `git reset --soft HEAD~1` 用于撤销最近一次 commit,改变的文件变为 staged 的状态
* `git reset --hard HEAD~1` 用于撤销最近一次 commit (本地). 如果已经推送到了服务器, 继续用 `git push origin HEAD --force` 即可.
* 把某文件从所有历史中删除用 `git filter-branch -f --tree-filter 'rm -rf <path/file>' HEAD` 注意这样做本身并不一定会使 repo 变小. 如果要清空一个文件夹, 就用 `path/*` 就好
* `git gc --aggressive --prune=now` 会可以将 repo 缩小,PhysWiki 的 .git 大概能缩小到一半. 鼓励经常使用 `git gc`,但 `--aggressive` 选项会比较慢但也是安全的.
* `git gc` 命令只对本地 repo 有效,其实最有效的貌似还是直接把本地 repo 删了再 clone 一次.但是注意检查是否有不小心被 .gitignore 屏蔽的文件没有在服务器上
* 即使是二进制文件, git 也会尽量储存每个版本的增量, 而不是每个版本储存一次.
* `git clean -f` 可以清除 untracked file, `-d` 选项可以清除 untracked directory,  `git clean -n` 显示将要清除的文件.
* 要让 windows 中 git bash 支持中文, 在右键 option > text 中 local 选择 C, character set 选 UTF8, 另外输入以下命令
`export LESSCHARSET=utf-8`
`git config --global core.quotepath false`          # 显示 status 编码
`git config --global gui.encoding utf-8`            # 图形界面编码
`git config --global i18n.commit.encoding utf-8`    # 提交说明编码
`git config --global i18n.logoutputencoding utf-8`  # 输出 log 编码
在 linux 系统中, 只需要一条设置, 就是
`git config --global core.quotepath false`
* 用 `gitk` 可以弹出 GUI 显示每一次 commit 的文件目录, 不需要 checkout 就可以看到.
* `git push` 到一个 nonbare repo 一般是不允许的, 除非这个 repo 设置了 `git config receive.denyCurrentBranch warn` 或者 `git config receive.denyCurrentBranch ignore.` 这时如果 push, 并不会改变 remote 的 working directory. (其实这正是我增量备份 .git 需要的!)
* windows 文本编辑器默认是 ANSI 编码, 而 ANSI 在 git 中显示的是乱码. 而 Visual Studio Code 保存的时候默认会存成 UTF-8 编码, ANSI 编码在 Visual Studio Code 中也会是乱码, 这时千万不要保存, 否则就成了永久的乱码了.
* 要检查编码类型, 或转换编码类型,用 Visual Studio Code. 永远不要用 Windows 的 Notepad.
* ASCII 就是 UTF-8.所有的文本文件原则上都用 UTF-8.
