% 布劳威尔不动点定理(综述)
% license CCBYSA3
% type Wiki

本文根据 CC-BY-SA 协议转载翻译自维基百科\href{https://en.wikipedia.org/wiki/Brouwer_fixed-point_theorem}{相关文章}。

布劳威尔不动点定理是拓扑学中的一个不动点定理,以 L. E. J.(Bertus)布劳威尔命名。该定理指出:**对于任意一个将非空紧致凸集映射到其自身的连续函数 $f$,总存在一点 $x_0$,使得 $f(x_0) = x_0$**。
最简单的形式是将一个闭区间 $I$(实数集中的)映射到自身的连续函数,或将一个闭圆盘 $D$ 映射到自身的情形。比这更一般的形式是:将欧几里得空间中的非空紧致凸子集 $K$ 映射到自身的连续函数。

在众多不动点定理中,[1] 布劳威尔不动点定理尤为著名,部分原因是它在数学的众多领域中都有广泛应用。在其原始领域中,这一结果是刻画欧几里得空间拓扑性质的关键定理之一,与约旦曲线定理、毛球定理、维数不变性定理以及博苏克–乌拉姆定理并列为拓扑学的基本定理之一。[2] 它还被用于证明关于微分方程的重要结论,因此通常出现在微分几何的入门课程中。此外,它也出现在一些出人意料的领域,比如博弈论。在经济学中,布劳威尔不动点定理及其推广——卡库塔尼不动点定理,是 20 世纪 50 年代经济学诺贝尔奖得主肯尼斯·阿罗和热拉尔·德布鲁提出的一般均衡存在性证明的核心工具。

这一定理最初是在研究微分方程的背景下被法国数学家们提出的,代表人物有昂利·庞加莱和夏尔·埃米尔·皮卡尔。要证明诸如庞加莱–本迪克松定理这样的结果,需要运用拓扑方法。这一领域在 19 世纪末开启,催生了定理的若干版本。对于 $n$ 维闭球中可微映射的情形,最早由雅克·阿达马于 1910 年给出证明;而对连续映射的一般情形,则由布劳威尔于 1911 年完成证明。[5]
\subsection{陈述}

布劳威尔不动点定理有多种表述方式,取决于其应用的上下文以及推广的程度。最简单的形式如下:

\textbf{在平面上}

每一个从闭圆盘映射到其自身的连续函数至少有一个不动点。[6]

这个结论可以推广到任意有限维度:

\textbf{在欧几里得空间中}

**每一个从欧几里得空间中闭球映射到其自身的连续函数都有一个不动点。**\[7]

稍微更一般一点的版本是:

### 凸紧致集

**每一个从欧几里得空间中非空凸紧致子集 $K$ 映射到自身的连续函数都有一个不动点。**\[9]

一个更加广义的形式通常以另一个名称广为人知:

### 舍乌德尔不动点定理(Schauder fixed point theorem)

**每一个从巴拿赫空间中非空凸紧致子集 $K$ 映射到自身的连续函数都有一个不动点。**\[10]

### 先决条件的重要性

该定理仅对**自同态函数**(即定义域与值域相同的函数)成立,并且要求集合是**非空、紧致**(即有界并闭合)且**凸的**(或者与凸集同胚)。下面的示例将说明这些先决条件为何是必要的。
