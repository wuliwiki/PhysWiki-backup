% 塞曼效应
% 磁矩1|能级2|角动量3


在量子力学中,谱线的频率和波长的偏移意味着一种或两种状态的能级的跃迁.由单重态之间的在磁场作用下能级发生分裂,谱线分裂成间隔相等的3条谱线的塞曼效应(Zeeman Effect)被称为正常效应,而在一些实验中所观察到光谱线有时并非分裂成3条,其间隔也都不相同,这种当初始态或最终态或两者的总自旋为非零时所发生的塞曼效应被称为反常效应.尽管这两者并没有什么实质性区别,但在电子磁矩的值非常大时用反常效应解释会使问题复杂化,因此我们先来考虑由单重态的跃迁产生的赛曼正常效应.

\subsection{正常赛曼效应}
单重态的自旋为零,因此总的角动量$\mathbf{J}$等于轨道角动量$\mathbf{L}$.当原子被置于外磁场中时,原子的能量因其磁场中磁矩的能量而改变,
$$\Delta E = -\boldsymbol\mu\cdot \mathbf{B}=-\mu_z B$$反向$z$是磁场$B$的反向,在氢原子中角动量的磁矩为$$\mu=\frac{e}{2m_e}L=\frac{e\hbar}{2m_e}\sqrt{l(l+1)}=\sqrt{l(l+1)}\mu_B$$
其中$l$被称为角动量量子数或轨道量子数.因此,我们有$$\mu_z=-m_l\mu_B=-m_l\frac{e\hbar}{2m_e}$$
使得\begin{equation}
\Delta E = m_l\mu_B B
\end{equation}
由于$m_l$有$2l+1$个取值,因此每一个能级都被分裂成$2l+1$个能级.

对于$l=1$和$l=2$能级在正常赛曼效应下分裂\begin{figure}[ht]
\centering
\includegraphics[width=5cm]{./figures/ZemEff_1.pdf}
\caption{请添加图片描述} \label{ZemEff_fig1}
\end{figure}

\subsection{反常赛曼效应}

如上所述,反常塞曼效应发生在初始态或最状态,或两者的自旋都是非零的时候.由于自旋产生的磁矩是$1$而不是$\frac{1}{2}$玻尔磁子,所以能级分裂的计算很是复杂,因此总磁矩等同于总角动量.假设一个原子的轨道角动量为$\mathbf{L}$,自旋为$\mathbf{S}$,它的总角动量为
$$\mathbf{J=L+S}$$
当单个原子处于一均匀磁场$\mathbf{B}_{\rm{ext}}$中时,能级会产生跃迁.对于单个电子的微扰(perturbation)为
$$H_Z^{'} = -(\boldsymbol{\mu}_l+\boldsymbol{\mu_s})\cdot \mathbf{B}_{\rm{ext}},$$
其中电子自旋的磁偶极矩为$$\boldsymbol\mu _s =-\frac{e}{m}\mathbf{S}$$经典轨道运动的偶极矩为
$$\boldsymbol\mu _l =-\frac{e}{2m}\mathbf{L}$$因此就有$$H_Z^{'} = \frac{e}{2m}(\mathbf{L+2S})\cdot \mathbf{B}_{\rm{ext}}$$