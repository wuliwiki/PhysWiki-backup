% 中山大学 2015 年918专业基础(数据结构)考研真题
% 中山大学 2015 年918专业基础(数据结构)考研真题

\subsection{一、单项选择题(每小题3分,共45分)}
1. STL中的优先队列是采用什么数据结构来实现的? \\
A.堆  $\qquad$ B.队列 $\qquad$ C.栈 $\qquad$ D.图

2.数据结构中,与所使用的计算机无关的是数据的()结构。 \\
A.存储 $\qquad$ B.物理 $\qquad$ C.逻辑 $\qquad$ D.物理和存储

3.计算机算法指的是( )。 \\
A.计算方法 $\qquad$ B. 排序方法 \\
C.调度方法 $\qquad$ D. 解决问题的有限指令序列

4.给定一个有n个元素的数组(n为偶数)。如果要找出数组中的最大元素和最小元素,最少要进行( )次比较? \\
A.2n $\qquad$ B.3n/2-2 $\qquad$ C.2n-2 $\qquad$ D.4n/3

5.给定一个包含250个整数的数组,该数组中的整数已按从小到大的顺序排好序。假设用二分查找从该数组中寻找某个给定的整数y,最多只需要做( )次比较。 \\
A.8 $\qquad$ B.9 $\qquad$ C.10 $\qquad$ D.7

6. $T(n)$表示某个算法的时间复杂度。假设$T(n)=2T(n/2)+O(m)$,则$T(n)$为( ) \\
A. $O(log3n)$ $\qquad$ B. $O(n)$ $\qquad$ C. $O(nlog_3n)$ $\qquad$ D. $O(n^2)$

7.假设整数n>0,下面的程序的时间复杂度是( ). \\
\begin{lstlisting}[language=cpp]
x=2;
while (x<n/3) x=2*x;
\end{lstlisting}
A. $O(log_2n)$ $\qquad$ B. $O(m)$ $\qquad$ C. $O(nlog_2n)$ $\qquad$ D.$O(n)$

8.下列排序算法中,哪个是稳定的排序算法?
A.选择排序B. 快速排序C. 归并排序D. 希尔排序
9.假设小明用某-排序算法对整数序列(82, 45, 25, 15, 21)进行排序。以下为排序过程中序列状态的
变化过程:
输入:82 4525 15 21
第一步:45 82 251521
第二步:2545 82 1521
第三步:1525 45 82 21
.
