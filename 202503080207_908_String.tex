% 弦理论(综述)
% license CCBYSA3
% type Wiki

本文根据 CC-BY-SA 协议转载翻译自维基百科\href{https://en.wikipedia.org/wiki/String_theory}{相关文章}。

在物理学中,弦理论是一种理论框架,其中粒子物理学中的点状粒子被称为弦的一维物体所取代。弦理论描述了这些弦如何在空间中传播并相互作用。在大于弦尺度的距离范围内,弦表现得像一个粒子,其质量、电荷和其他属性由弦的振动状态决定。在弦理论中,弦的众多振动状态之一对应于引力子,一种携带引力的量子力学粒子。因此,弦理论是一种量子引力理论。

弦理论是一个广泛而多样的学科,试图解决许多深刻的基础物理学问题。弦理论为数学物理学做出了诸多贡献,这些贡献已应用于黑洞物理学、早期宇宙宇宙学、核物理学和凝聚态物理学的各种问题,并激发了纯数学领域的一些重大进展。由于弦理论可能提供一个统一的引力和粒子物理学的描述,它成为了万物理论的候选者,即一个自洽的数学模型,能够描述所有基本力和物质形式。尽管在这些问题上进行了大量研究,但目前尚不清楚弦理论在多大程度上描述了真实世界,或者该理论在选择细节时允许多少自由度。

弦理论最早在1960年代末期作为强核力的理论进行研究,但随后由于量子色动力学的兴起而被放弃。随后,人们意识到,正是使弦理论不适合作为核物理学理论的那些特性,使其成为量子引力理论的一个有前景的候选者。弦理论的最早版本是玻色子弦理论,它只包含了被称为玻色子的粒子类别。后来,它发展成了超弦理论,超弦理论假设玻色子和称为费米子的粒子类别之间存在一种叫做超对称的联系。在1990年代中期,人们推测出,超弦理论的五个一致版本其实是一个十一维的单一理论的不同极限情形,这个理论被称为M理论。1997年底,理论物理学家发现了一个重要的关系,叫做反德西特/共形场论对偶性(AdS/CFT对偶性),它将弦理论与另一种物理理论——量子场论——联系了起来。

弦理论的一个挑战是,完整的理论在所有情况下都没有一个令人满意的定义。另一个问题是,该理论被认为描述了一个庞大的可能宇宙的景观,这使得基于弦理论的粒子物理学理论发展变得复杂。这些问题导致物理学界一些人批评这些物理学方法,并质疑继续进行弦理论统一研究的价值。
\subsection{基本原理}
\subsubsection{概述}
\begin{figure}[ht]
\centering
\includegraphics[width=6cm]{./figures/b1d0a7d0dd142f3e.png}
\caption{弦理论的基本物体是开弦和闭弦。} \label{fig_String_1}
\end{figure}  
在20世纪,出现了两种理论框架来阐述物理学的基本法则。第一种是阿尔伯特·爱因斯坦的广义相对论,这一理论解释了引力的作用以及宏观层面上时空的结构。另一种是量子力学,这是一种完全不同的框架,利用已知的概率原理来描述微观层面的物理现象。到1970年代末,这两种框架已被证明足以解释大多数已观察到的宇宙特征,从基本粒子到原子,再到恒星和整个宇宙的演化。[1]

尽管取得了这些成功,但仍有许多问题亟待解决。现代物理学中最深刻的问题之一是量子引力问题。[1]广义相对论是在经典物理框架内 formul化的,而其他基本力则在量子力学框架内描述。为了将广义相对论与量子力学的原则统一起来,需要一个量子引力理论,但当尝试将量子理论的常规法则应用于引力时,便会遇到困难。[2]

弦理论是一种理论框架,试图解决这些问题。

弦理论的起点是这样一个观点:粒子物理学中的点状粒子也可以被建模为一种称为“弦”的一维物体。弦理论描述了弦如何在空间中传播并相互作用。在某个特定版本的弦理论中,只有一种类型的弦,它可能看起来像一个小的环或普通弦的一段,并且可以以不同的方式振动。在大于弦尺度的距离范围内,弦看起来就像是一个普通的粒子,符合非弦模型中的基本粒子,其质量、电荷以及其他特性由弦的振动状态决定。作为量子引力的应用,弦理论提出了一种振动状态,负责产生引力子——一种尚未被证实的量子粒子,理论上它承担引力的作用。[3]

过去几十年中,弦理论的主要发展之一是发现了某些“对偶性”,即将一种物理理论与另一种物理理论联系起来的数学变换。研究弦理论的物理学家发现了不同版本弦理论之间的一些对偶性,这导致了一个猜想:所有一致的弦理论版本都可以被统一在一个单一的框架中,称为\(M\)理论。[4]

弦理论的研究还在黑洞的性质和引力相互作用方面取得了一些成果。当人们试图理解黑洞的量子特性时,出现了一些悖论,弦理论的研究试图澄清这些问题。1997年底,这一领域的研究达到了顶峰,发现了反-德西特/共形场理论对应(AdS/CFT)。[5]这是一个理论结果,将弦理论与其他理论上更为清楚的物理理论联系起来。AdS/CFT对应对于黑洞和量子引力的研究具有重要意义,并且已被应用于其他领域,[6]包括核物理和凝聚态物理。[7][8]

由于弦理论包含了所有基本相互作用,包括引力,许多物理学家希望它最终能够发展到足以完全描述我们的宇宙,从而成为一种“万物理论”。目前弦理论研究的目标之一是找到一个能够再现已观察到的基本粒子谱、具有小的宇宙常数、包含暗物质并提供合理的宇宙膨胀机制的理论解。尽管在这些目标上已有一些进展,但目前尚不清楚弦理论在多大程度上能够描述现实世界,或者该理论在细节选择上允许多少自由度。[9]

弦理论的挑战之一是,完整的理论在所有情况下都没有一个令人满意的定义。弦的散射最直接的定义方法是使用微扰理论的技巧,但通常并不清楚如何非微扰地定义弦理论。[10]此外,是否存在某种原理来选择弦理论的真空态——即决定我们宇宙特性的物理状态——也尚不明确。[11]这些问题使得部分学者批评将物理学统一的这些方法,并质疑继续研究这些问题的价值。[12]
\subsubsection{弦}
\begin{figure}[ht]
\centering
\includegraphics[width=6cm]{./figures/fd6b90c488650f4c.png}
\caption{量子世界中的相互作用:点状粒子的世界线或弦论中闭合弦所扫过的世界面} \label{fig_String_2}
\end{figure}
将量子力学应用于像电磁场这样的在时空中扩展的物理对象,称为量子场论。在粒子物理学中,量子场论是我们理解基本粒子的基础,这些粒子被建模为基本场中的激发。[13]

在量子场论中,通常使用微扰理论的技术来计算各种物理事件的概率。微扰量子场论由理查德·费曼等人在二十世纪上半叶发展起来,使用一种叫做费曼图的特殊图形来组织计算。人们可以想象,这些图形描绘了点状粒子的路径及其相互作用。[13]

弦论的出发点是这样的观点:量子场论中的点状粒子也可以被建模为一维物体,称为弦。[14] 弦的相互作用通过推广在普通量子场论中使用的微扰理论来最直接地定义。在费曼图的层面上,这意味着用一个二维(2D)表面来替代表示点粒子路径的一维图形,从而表示弦的运动。[15] 与量子场论不同,弦论没有完整的非微扰定义,因此许多物理学家希望解答的理论问题仍然无法触及。[16]

在基于弦论的粒子物理学理论中,弦的特征长度尺度被假定为普朗克长度量级,即\(10^{-35}\)米,这是量子引力效应被认为变得显著的尺度。[15] 在更大尺度下,比如物理实验室中可见的尺度,这样的物体将与零维的点粒子无法区分,而弦的振动状态将决定粒子的类型。弦的一个振动状态对应于引力子,一种量子力学粒子,携带引力作用。[3]

原始版本的弦理论是玻色弦理论,但这个版本仅描述了玻色子——一种传递物质粒子之间相互作用力的粒子类别。玻色弦理论最终被称为超弦理论的理论所取代。这些理论既描述玻色子也描述费米子,并且它们包含了一个叫做超对称的理论概念。在具有超对称性的理论中,每个玻色子都有一个对应的费米子,反之亦然。[17]

超弦理论有多个版本:类型I、类型IIA、类型IIB和两种异质弦理论(\(SO(32)\)和\(E_8\times E_8\))。不同的理论允许不同类型的弦,并且在低能量下出现的粒子表现出不同的对称性。例如,类型I理论包含开放弦(具有端点的弦段)和闭合弦(形成闭环的弦),而类型IIA、IIB和异质弦理论仅包含闭合弦。[18]
\subsubsection{额外维度}
\begin{figure}[ht]
\centering
\includegraphics[width=6cm]{./figures/8815ea074c9ece9c.png}
\caption{紧致化的一个例子:在大尺度下,一个具有一个圆形维度的二维表面看起来像是一维的。} \label{fig_String_3}
\end{figure}
在日常生活中,我们熟悉的空间有三个维度(3D):高度、宽度和长度。爱因斯坦的广义相对论将时间视为与三个空间维度平等的维度;在广义相对论中,空间和时间并不是作为独立的实体来描述,而是统一为四维(4D)时空。在这个框架中,重力现象被视为时空几何的结果。[19]

尽管宇宙可以用四维时空很好地描述,但物理学家考虑其他维度的理论有几个原因。在某些情况下,通过以不同维度来建模时空,理论变得在数学上更易处理,从而可以更容易地进行计算并获得一般性的见解。[a] 还有一些情况,在二维或三维时空中的理论对于描述凝聚态物理中的现象非常有用。[13] 最后,存在一些情形,可能时空的维度实际上超过了四维,但这些额外的维度仍然未能被探测到。[20]

弦理论要求额外的时空维度以保持其数学一致性。在玻色弦理论中,时空是26维的,而在超弦理论中是10维的,在M理论中是11维的。因此,为了使用弦理论描述真实的物理现象,必须设想一些场景,其中这些额外的维度在实验中不会被观察到。[21]
\begin{figure}[ht]
\centering
\includegraphics[width=6cm]{./figures/cf09ca86ce3b51c2.png}
\caption{一个五次卡拉比–尤流形的截面} \label{fig_String_4}
\end{figure}
紧致化是修改物理理论中维度数量的一种方式。在紧致化过程中,假设一些额外的维度“自我闭合”,形成圆形。[22] 在这些卷曲的维度变得非常小的极限下,便得到一种有效维度较低的理论。对此的标准类比是考虑一个多维物体,比如花园水管。如果从足够远的地方看,水管似乎只有一个维度,即它的长度。然而,当靠近水管时,会发现它包含第二个维度,即其周长。因此,一只在水管表面爬行的蚂蚁将会在两个维度上移动。

紧致化可以用来构造有效四维时空的模型。然而,并非所有紧致化额外维度的方式都会产生具有正确性质的模型来描述自然界。在一个可行的粒子物理模型中,紧致的额外维度必须呈现卡拉比–尤流形的形状。[22] 卡拉比–尤流形是一种特殊的空间,在弦理论应用中通常被认为是六维的。它以数学家尤金尼奥·卡拉比和丘成桐的名字命名。[23]

减少维度的另一种方法是所谓的‘膜世界’场景。在这种方法中,物理学家假设可观察的宇宙是一个更高维空间的四维子空间。在这种模型中,粒子物理学中的传递力的玻色子来自于端点附着在四维子空间上的开放弦,而引力则来自于通过更大环境空间传播的闭合弦。这个思想在尝试基于弦理论开发现实世界物理学模型的过程中发挥着重要作用,并且为引力相比其他基本力的弱性提供了一个自然的解释。[24]
\subsubsection{对偶性}
\begin{figure}[ht]
\centering
\includegraphics[width=14.25cm]{./figures/58ea06ab2273f446.png}
\caption{弦理论对偶性的图示。蓝色边表示S-对偶性,红色边表示T-对偶性。} \label{fig_String_5}
\end{figure}
弦理论的一个显著特点是,不同版本的理论最终都以非常非平凡的方式相互关联。不同弦理论之间可能存在的一种关系被称为S-对偶性。这种关系表明,在某些情况下,一种理论中强相互作用的粒子集合可以被视为另一种完全不同理论中弱相互作用的粒子集合。粗略来说,如果一组粒子经常结合并衰变,则称其为强相互作用;如果这种过程发生得不频繁,则称其为弱相互作用。I型弦理论通过S-对偶性与\(SO(32)\)的异质弦理论是等价的。同样,IIB型弦理论也通过S-对偶性以非平凡的方式与自身相关联。[25]

不同弦理论之间的另一种关系是T-对偶性。在这里,考虑的是弦在一个圆形额外维度上传播。T-对偶性表明,一条在半径为\(R\)的圆圈上传播的弦,与一条在半径为\(1/R\)的圆圈上传播的弦是等价的,意思是一个描述中的所有可观察量与对偶描述中的量是对应的。例如,一条弦在绕圆圈传播时具有动量,它也可以绕圆圈一圈或多圈。弦绕圆圈缠绕的圈数称为缠绕数。如果在一种描述中,弦的动量为\(p\),缠绕数为\(n\),那么在对偶描述中,它的动量将是\(n\),缠绕数将是\(p\)。例如,IIA型弦理论通过T-对偶性与IIB型弦理论等价,异质弦理论的两个版本也通过T-对偶性相关联。[25]

一般来说,对偶性一词指的是两种看似不同的物理系统以非平凡的方式实际上是等价的情况。通过对偶性相关联的两种理论不一定是弦理论。例如,Montonen–Olive对偶性是量子场论之间S-对偶性关系的一个例子。AdS/CFT对应关系是一个将弦理论与量子场论联系起来的对偶性例子。如果两种理论通过对偶性相关联,意味着一种理论可以以某种方式被转换,使得它最终看起来与另一种理论完全相同。然后,这两种理论就被称为在变换下彼此对偶。换句话说,这两种理论是同一现象的数学上不同的描述。[26]
\subsubsection{膜}
\begin{figure}[ht]
\centering
\includegraphics[width=6cm]{./figures/94cdfc802840194e.png}
\caption{附着在一对D-膜上的开放弦} \label{fig_String_6}
\end{figure}
在弦理论及其他相关理论中,膜是一个物理对象,它将点粒子的概念推广到更高的维度。例如,点粒子可以被看作是一个零维的膜,而弦可以被看作是一个一维的膜。也可以考虑更高维度的膜。在\(p\)维度下,这些被称为p-膜。‘膜’一词来自于‘膜’(membrane)一词,后者指的是二维膜。[27]

膜是动态对象,可以根据量子力学的规则在时空中传播。它们具有质量,并可以具有其他属性,如电荷。一个p-膜在时空中扫过一个\((p+1)\)维的体积,称为它的世界体积。物理学家通常研究类似于电磁场的场,这些场存在于膜的世界体积上。[27]

在弦理论中,D-膜是一个重要的膜类别,当考虑开放弦时会出现。随着开放弦在时空中传播,它的端点必须位于D-膜上。D-膜中字母“\(D\)”指的是系统中一种称为狄利克雷边界条件的特定数学条件。弦理论中对D-膜的研究产生了重要的成果,例如AdS/CFT对应关系,这为量子场论中的许多问题提供了新的见解。[27]

膜经常从纯数学的角度进行研究,它们被描述为某些范畴中的对象,例如复代数簇上的相干层的导出范畴,或辛流形的福卡亚范畴。[28] 膜的物理概念与数学范畴概念之间的联系,导致了代数几何与辛几何[29]以及表示论[30]领域中的重要数学见解。
\subsection{M理论}  
在1995年之前,理论物理学家认为超弦理论有五种一致的版本(I型、IIA型、IIB型以及两种异质弦理论)。这一理解在1995年发生了变化,当时爱德华·威滕提出这五种理论仅仅是一个十一维理论——M理论的特殊极限情形。威滕的猜想基于其他物理学家的一些工作,包括阿肖克·森、克里斯·赫尔、保罗·汤森德和迈克尔·达夫等人。他的这一宣布引发了一场研究热潮,现被称为第二次超弦革命。[31]
\subsubsection{超弦理论的统一}
\begin{figure}[ht]
\centering
\includegraphics[width=10cm]{./figures/d43777704e0c6f56.png}
\caption{这是一张示意图,展示了M理论、五种超弦理论和十一维超引力之间的关系。阴影区域表示在M理论中可能的不同物理情景家族。在某些对应于拐点的极限情况下,使用其中一种标记的六种理论来描述物理现象是自然的。} \label{fig_String_7}
\end{figure}
在1970年代,许多物理学家开始对超引力理论产生兴趣,该理论将广义相对论与超对称结合起来。广义相对论在任何维度下都是合理的,而超引力则对维度的数量设定了上限。[32] 1978年,Werner Nahm的研究表明,能够制定一致的超对称理论的最大时空维度是十一。[33] 同年,来自巴黎高等师范学校的Eugene Cremmer、Bernard Julia和Joël Scherk证明,超引力不仅允许最多十一维,而且在这个最大维度下表现得最为优雅。[34][35]

