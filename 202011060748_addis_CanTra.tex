% 正则变换
% 哈密顿量|分析力学|拉格朗日|哈密顿原理|哈密顿正则方程|正则变换

\begin{issues}
\issueDraft
\end{issues}

\pentry{哈密顿正则方程\upref{HamCan}}

\footnote{参考 \cite{Goldstein}.}对于同一系统, 我们可以采用不同的广义坐标, 广义动量和哈密顿量. 那么如何在这些不同选择中做变换呢? 我们把一组广义坐标和广义动量记为 $q_1,\dots, q_N, p_1, \dots, p_N$, 哈密顿量为 $H(q, p, t)$. 另一组记为 $Q_1,\dots, Q_N, P_1, \dots, P_N$, 哈密顿量为 $K(Q, P, t)$. 那么他们满足什么关系,或者在二者之间进行变换呢?

首先, 根据勒让德变换\autoref{HamCan_eq1}~\upref{HamCan}, 拉格朗日量可以表示为
\begin{equation}
L = \sum_i \dot q_i p_i - H(q, p, t)
\end{equation}

如果对于不同的变量 $Q, P$, 以及哈密顿量 $K$, 那么

代入哈密顿原理

* 生成函数并不能生成所有的正则变换, 另一种方法是什么矩阵
