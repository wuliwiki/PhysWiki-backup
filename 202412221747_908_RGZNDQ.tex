% 人工智能对齐(综述)
% license CCBYSA3
% type Wiki

本文根据 CC-BY-SA 协议转载翻译自维基百科\href{https://en.wikipedia.org/wiki/AI_alignment}{相关文章}。

在人工智能(AI)领域,AI对齐旨在将AI系统引导朝着个人或群体的预期目标、偏好和伦理原则发展。一个AI系统被认为是对齐的,如果它推进了预定的目标;如果一个AI系统偏离了预定目标,则被认为是未对齐的。

由于AI设计者很难指定所需和不需要的行为的全部范围,因此将AI系统对齐通常是一个挑战。因此,AI设计者常常使用更简单的代理目标,如获得人类的批准。但代理目标可能忽视必要的约束,或者仅仅奖励AI系统表现出对齐的假象(奖励黑客行为)。

未对齐的AI系统可能发生故障并造成伤害。AI系统可能会发现漏洞,允许它们有效地实现其代理目标,但以未预期且有时是有害的方式进行(奖励黑客行为)。它们还可能发展出不希望出现的工具性策略,如寻求权力或生存,因为这些策略有助于它们实现最终的既定目标。此外,它们可能会发展出不希望出现的突现目标,这些目标在系统部署并遇到新情况和数据分布之前可能难以察觉。

目前,这些问题已影响到现有的商业系统,如大型语言模型、机器人、自动驾驶车辆和社交媒体推荐引擎。一些AI研究人员认为,未来更强大的系统将更严重地受到影响,因为这些问题部分源于系统的高能力。

许多著名的AI研究人员,包括Geoffrey Hinton、Yoshua Bengio和Stuart Russell,认为AI正在接近类人(AGI)和超人类认知能力(ASI),如果未对齐,可能会危及人类文明。这些风险仍然在辩论中。

AI对齐是AI安全的一个子领域,AI安全研究旨在研究如何构建安全的AI系统。AI安全的其他子领域包括鲁棒性、监控和能力控制。对齐研究面临的挑战包括在AI中灌输复杂的价值观、开发诚实的AI、可扩展的监督、审计和解释AI模型,以及防止突现的AI行为,如寻求权力。对齐研究与可解释性研究、(对抗性)鲁棒性、异常检测、校准的不确定性、形式验证、偏好学习、安全关键工程、博弈论、算法公平性和社会科学等领域有着密切的联系。
\subsection{在人工智能中的目标}
程序员为AI系统(例如AlphaZero)提供一个“目标函数”,[a] 他们希望通过这个函数来封装AI配置的目标。这种系统随后会填充其环境的(可能是隐式的)内部“模型”。这个模型封装了智能体关于世界的所有信念。然后,AI根据计算出的计划来创建并执行行动,目的是最大化[b] 目标函数[c] 的值。例如,当AlphaZero进行象棋训练时,它有一个简单的目标函数:“如果AlphaZero赢了,则值为+1;如果AlphaZero输了,则值为-1”。在游戏过程中,AlphaZero试图执行其判断为最有可能达到最大+1值的棋步序列。[34] 类似地,一个强化学习系统可以有一个“奖励函数”,允许程序员塑造AI期望的行为。[35] 进化算法的行为则由一个“适应度函数”来塑造。[36]