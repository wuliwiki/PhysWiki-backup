% 古斯塔夫·基尔霍夫(综述)
% license CCBYSA3
% type Wiki

本文根据 CC-BY-SA 协议转载翻译自维基百科\href{https://en.wikipedia.org/wiki/Gustav_Kirchhoff}{相关文章}。

\begin{figure}[ht]
\centering
\includegraphics[width=6cm]{./figures/ba04c3fbf9d952c3.png}
\caption{} \label{fig_JRHF_1}
\end{figure}
古斯塔夫·罗伯特·基尔霍夫(德语:[ˈgʊs.taf ˈkɪʁçhɔf];1824年3月12日–1887年10月17日)是德国的物理学家、数学家和化学家,他在电路学、光谱学以及加热物体的黑体辐射发射等基本理解方面做出了重要贡献。[1][2] 他还在1860年提出了“黑体”这一术语。[3]

有几个不同的概念集被称为“基尔霍夫定律”,包括基尔霍夫电路定律、基尔霍夫热辐射定律和基尔霍夫热化学定律。

“本生–基尔霍夫光谱学奖”是以基尔霍夫和他的同事罗伯特·本生的名字命名的。
\subsection{生平与工作}  
古斯塔夫·基尔霍夫于1824年3月12日出生在普鲁士的哥尼斯堡,父亲是律师弗里德里希·基尔霍夫,母亲是约翰娜·亨丽埃特·维特克。他的家庭属于普鲁士福音教会的路德宗信徒。他于1847年毕业于哥尼斯堡的阿尔贝图斯大学,在那里参加了由卡尔·古斯塔夫·雅各布·雅可比、弗朗茨·恩斯特·诺伊曼和弗里德里希·朱利乌斯·里舍洛特主持的数学物理研讨会。同年,他搬到了柏林,直到他获得布雷斯劳的教授职位。后来,在1857年,他与数学教授里舍洛特的女儿克拉拉·里舍洛特结婚,夫妻俩育有五个孩子。克拉拉于1869年去世。基尔霍夫于1872年再婚,妻子是路易莎·布勒梅尔。