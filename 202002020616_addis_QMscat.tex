% 量子散射
\pentry{散射\upref{Scater}, 球坐标中的薛定谔方程\upref{RYTDSE}}

\footnote{本文参考 Physics of Atoms and Molecules, Bransden}我们讨论一个粒子在\textbf{中心势能} $V(r)$ 下的散射. 核心思路: 我们需要 $E > 0$ 的所有本征波函数(也叫散射态), 入射的初始波包可以由这些散射态叠加而成, 之后的时间演化就是这些散射态分别乘以 $\exp(-\I E t)$ 再叠加.

\subsection{散射截面}

在三维情况下, 每个能量 $E$ 都是无穷维简并的. 且根据不同的边界条件我们可以获得不同的正交归一基底. 常见的边界条件如平面波入射, 即
\begin{equation}
\lim_{r \to \infty} \psi_{\bvec k}^{(+)}(\bvec r) = (2\pi)^{-3/2} \exp(\I \bvec k\bvec r) + f(k, \bvec r) \frac{\exp(\I k r)}{r}
\end{equation}
$f(k, \bvec r)$ 是\textbf{散射幅}\footnote{至于概率守恒对散射幅的约束, 见 Optical Theorem.}.

\textbf{微分散射截面}与概率流密度的关系
\begin{equation}
\dv{\sigma}{\Omega} = \frac{\bvec j_{out} r^2}{\bvec j_{in}}
\end{equation}
总\textbf{散射截面}可以形象理解为挡住概率流密度 $\bvec j$ 的面积
\begin{equation}
\sigma = \int\dv{\sigma}{\Omega} \dd{\Omega}
\end{equation}




另一种边界条件是
