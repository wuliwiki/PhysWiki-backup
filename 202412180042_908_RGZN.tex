% 人工智能史(综述)
% license CCBYSA3
% type Wiki

本文根据 CC-BY-SA 协议转载翻译自维基百科\href{https://en.wikipedia.org/wiki/History_of_artificial_intelligence}{相关文章}。

人工智能(AI)的历史可以追溯到古代,那个时候有关于由工匠们赋予智慧或意识的人工生命体的神话、故事和传闻。从古代到现代,逻辑学和形式推理的研究直接促成了1940年代可编程数字计算机的发明,这是一种基于抽象数学推理的机器。这个设备及其背后的理念启发了科学家们开始讨论构建电子大脑的可能性。

人工智能研究领域是在1956年于达特茅斯学院举行的一次研讨会上创立的。[1] 参加该研讨会的人成为了人工智能研究的领导者,并且在几十年里引领着这一领域的发展。许多人预测,在一代人之内,像人类一样智能的机器将会问世。美国政府也提供了数百万美元,希望能够将这一愿景变为现实。[2]

最终,研究人员明显低估了这一壮举的难度。[3] 1974年,詹姆斯·莱特希尔的批评以及美国国会的压力导致美国和英国政府停止资助无目标的人工智能研究。七年后,日本政府的远见性倡议和专家系统的成功重新激发了对人工智能的投资,到了1980年代末,人工智能产业已经成长为一个价值十亿美元的行业。然而,到了1990年代,投资者的热情减退,人工智能在媒体中受到批评,行业也开始回避这一领域(这一时期被称为“人工智能寒冬”)。尽管如此,研究和资金在其他名称下依然持续增长。

进入2000年代,机器学习被应用于学术和工业中的广泛问题。这一成功归功于强大计算机硬件的可用性、大规模数据集的收集以及扎实的数学方法的应用。很快,深度学习证明是一项突破性的技术,超越了所有其他方法。2017年,变换器架构的首次亮相带来了令人印象深刻的生成型人工智能应用,及其他多个应用场景。到2020年代,人工智能的投资呈现爆发式增长。
\subsection{前驱}
\subsubsection{神话、小说和推测性的前驱}
\textbf{神话与传说}

在希腊神话中,塔罗斯(Talos)是一个由青铜铸成的巨人,担任克里特岛的守护者。他会向入侵者的船只投掷大石块,并每天绕岛的周围完成三次巡逻。[4] 根据伪阿波罗多罗斯的《博物志》(Bibliotheke),赫淮斯托斯(Hephaestus)在一位独眼巨人的帮助下锻造了塔罗斯,并将这个自动装置作为礼物献给米诺斯(Minos)。[5] 在《阿尔戈英雄传》(Argonautica)中,杰森(Jason)和阿尔戈英雄们通过拔出塔罗斯脚旁的塞子,导致其体内的生命之液流出,从而使塔罗斯丧命。[6]

皮格马利翁(Pygmalion)是希腊神话中的一位传奇国王与雕刻家,著名的故事出自奥维德的《变形记》。在奥维德的叙事诗《变形记》第十卷中,皮格马利翁因目睹普罗波埃提德斯(Propoetides)自愿卖淫的行为而对女性感到厌恶。尽管如此,他还是在维纳斯(Venus)的神庙中献上祭品,请求女神赐予他一位像他雕刻的雕像一样的女子。[7]

\textbf{中世纪关于人工生命体的传说}
\begin{figure}[ht]
\centering
\includegraphics[width=6cm]{./figures/7e497a845b48fea1.png}
\caption{歌德《浮士德》中的人造小人描绘} \label{fig_RGZN_1}
\end{figure}
在《事物的本质》中,瑞士炼金术士帕拉塞尔苏斯(Paracelsus)描述了一种他声称能制造“人造人”的方法。他将“一名男子的精液”放入马粪中,并在40天后喂入“人血的秘方”,这种混合物将变成一个活生生的婴儿。[8]

关于制造“哥雷姆”(Golem)的最早书面记载出现在13世纪初沃尔姆斯的以利以撒·本·犹大(Eleazar ben Judah)的著作中。[9] 在中世纪,人们认为,通过将写有上帝名字的纸条放入泥人(哥雷姆)口中,可以使其复活。[10] 与像青铜头这样的传奇自动装置不同,哥雷姆是无法说话的。[11][12]

在伊斯兰教的炼金术手稿中,\textbf{塔克温}(Takwin,即人工生命的创造)是一个常见的主题,尤其是在那些归属于贾比尔·伊本·海扬(Jabir ibn Hayyan)的作品中。伊斯兰炼金术士尝试通过炼金术创造各种生命形式,从植物到动物不等。[13]

在约翰·沃尔夫冈·冯·歌德(Johann Wolfgang von Goethe)的《浮士德:悲剧的第二部分》中,炼金术制造的“人造小人”(homunculus)注定要永远生活在他被制造出来的瓶子里,但他努力想要变成一个完整的人类身体。然而,在这一转变开始时,瓶子破裂,人造小人也随之死去。[14]

\textbf{现代小说}

到19世纪,关于人工人和思考机器的理念成为小说中的一个流行主题。像玛丽·雪莱的《弗兰肯斯坦》和卡雷尔·恰佩克的《R.U.R.》(罗索姆的万能机器人)[15]等著名作品探讨了人工生命的概念。像塞缪尔·巴特勒的《机器中的达尔文》[16]和爱德加·爱伦·坡的《梅尔策尔的棋手》[17]等推测性文章反映了社会对具有人工智能的机器日益增长的兴趣。人工智能至今仍是科幻小说中的常见主题。[18]

\textbf{自动装置}

许多文明的工匠建造了现实中的类人自动装置,包括燕师[19]、亚历山大英雄[20]、阿尔·贾扎里[21]、哈鲁恩·拉希德[22]、雅克·德·沃康松[23][24]、莱昂纳多·托雷斯·伊·凯韦多[25]、皮埃尔·贾凯-德罗兹和沃尔夫冈·冯·肯普伦[26][27]。

已知最古老的自动装置是古埃及和古希腊的神像[28][29]。信徒们相信,工匠们赋予这些雕像非常真实的心智,能够表现出智慧和情感——赫尔墨斯·特里斯梅吉斯图斯曾写道:“通过发现神的真正本质,人类已能够复制它”[30]。英国学者亚历山大·内克哈姆(Alexander Neckham)主张,古罗马诗人维吉尔曾建造了一座拥有自动雕像的宫殿[31]。

在早期现代时期,这些传奇的自动装置据说具有回答提问的神奇能力。晚期中世纪的炼金术士和原始新教徒罗杰·培根据说曾制造过一颗青铜头,并发展出自己是巫师的传说[32][33]。这些传说与北欧神话中的米米尔之头相似。传说中,米米尔以智力和智慧著称,在 Æsir-Vanir 战争中被斩首。奥丁被认为“用草药”保存了米米尔的头,并对其念咒语,使米米尔的头仍能向奥丁传授智慧。奥丁随后将头放在身边,作为咨询之用[34]。