% 李普希茨条件
% keys 李普希茨条件|Lipschitz Condition|一致连续
% license Xiao
% type Wiki

\pentry{度量空间\upref{Metric}}
李普希茨条件(Lipschitz)条件描述的对象是度量空间中的映射,它描述那些像点的距离受到原点距离影响的映射。李普希茨条件的最初形式是由德国数学家李普希茨在其1864年关于周期函数的傅里叶级数收敛性的研究中提出的\cite{Li}。本文介绍的是一般度量空间中的李普希茨条件。李普希茨条件在证明常微分方程的存在及唯一定理中起到作用。
\begin{definition}{李普希茨条件}\label{def_LipCon_1}
设 $A$ 是度量空间 $(M_1,d)$ 到 $(M_2,d')$ 的映射,$L$ 是一个正实数。若 $A$ 满足
\begin{equation}\label{eq_LipCon_1}
d(Ax,Ay)\leq Ld'(x,y),\quad\forall x,y\in M_1~,
\end{equation}
则称 $A$ 为满足具有常数 $L$ 的\textbf{李普希茨条件},记作 $A\in Lip L$。满足\autoref{eq_LipCon_1} 的最小常数称为\textbf{李普希茨常数}。
\end{definition}
\begin{example}{压缩映射}
若\autoref{def_LipCon_1} 中的 $(M_2,d')=(M_1,d)$,且 $0<L<1$,则此时 $A$ 便是 $M_1$ 中的\textbf{压缩映射}\upref{ComMap}。
\end{example}

\begin{theorem}{满足李普希茨条件的映射必一致连续}
若 $A$ 是度量空间 $(M_1,d)$ 到 $(M_2,d')$ 的满足具有常数 $L$ 的李普希茨条件的映射,则 $A$ 必一致连续。
\end{theorem}

