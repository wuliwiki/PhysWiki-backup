% 狄拉克 delta 函数
% 微积分|定积分|狄拉克 delta 函数|\delta 函数

\begin{issues}
\issueTODO
\issueOther{这个定义并不适用于\autoref{Delta_eq10}, 所以也不适用于\autoref{Delta_ex1}. 更广义的定义是直接使用\autoref{Delta_eq11}, 参考 \cite{Arfken}. 但是如何证明\autoref{Delta_eq10} 呢?}
\end{issues}

\pentry{定积分\upref{DefInt}}
在物理中我们经常会遇到一些模型, 如质点和点电荷等, 这类模型使用了极限的思想(如令体积趋于无穷小). 如果考察质点的密度或点电荷的电荷密度, 将得到无穷大, 然而将其密度(电荷密度)在空间中积分却又能得到有限的质量与电荷. 为了描述这样的密度(电荷密度)分布, 我们引入\textbf{狄拉克 $\delta$ 函数(Dirac delta function)}.

\subsection{一维情况}

$\delta(x)$ 在泛函分析中并不是数学中一个严格意义上的函数, 而是被称为\textbf{广义函数(generalized function)}或\textbf{分布(distribution)}, 详见泛函分析教材如\cite{Zeidler}. 但本文中我们给出另一个严谨的定义: 把 $\delta$ 函数看作是满足一定条件的\textbf{函数序列}, 即无穷个函数按一定顺序排列.

\begin{figure}[ht]
\centering
\includegraphics[width=11cm]{./figures/Delta_1.pdf}
\caption{$\delta(x - x_0)$ 的几个例子} \label{Delta_fig1}
\end{figure}

我们来考虑一个函数(\autoref{Delta_fig1} 左)
\begin{equation}\label{Delta_eq6}
f_h(x) =
\begin{cases}
h & \qty( \abs{x - x_0} \leqslant \frac{1}{2h} )\\
0 & \qty( \abs{x - x_0} > \frac{1}{2h} )
\end{cases}
\end{equation}
其中 $h, x_0$ 是常数. 由函数图像易得函数曲线下面的面积为 $\int_{-\infty}^{+\infty} f(x) \dd{x} = 1$. 现在我们令 $h \to \infty$, 长方形的高将趋于无穷大, 宽将趋于零, 而定积分结果不变.

这样, 上面的 $f(x)$ 就可以表示为 $\delta(x - x_0)$. 当然, 我们还可以选取其他含有参数的 $f(x)$ 来逼近 $\delta$ 函数, 如\autoref{Delta_fig1} 中的另外两种情况.

\begin{definition}{狄拉克 $\delta$ 函数}\label{Delta_def1}
令 $\delta_\alpha(x)$ 为一个含参数 $\alpha$ 的一元实函数. 如果 $\alpha$ 是离散的, 我们可以将其看作一个由无穷多个函数组成的函数序列 $\delta_1(x), \delta_2(x), \dots$ 若 $\delta_\alpha(x)$ 满足以下两个条件, 那么我们把该函数列称为狄拉克 $\delta$ 函数:

对任意给定的 $\epsilon > 0$, 有
\begin{equation}\label{Delta_eq5}
\lim_{\alpha\to+\infty}\int_{-\epsilon}^{+\epsilon} \delta_\alpha(x) \dd{x} = 1
\end{equation}
\begin{equation}\label{Delta_eq4}
\lim_{\alpha\to+\infty} \qty[\int_{-\infty}^{-\epsilon} \abs{\delta_\alpha(x)} \dd{x}
+ \int_{\epsilon}^{+\infty} \abs{\delta_\alpha(x)} \dd{x}] = 0
\end{equation}
\end{definition}

形式上我们记
\begin{equation}\label{Delta_eq1}
\lim_{\alpha\to+\infty} \delta_\alpha(x) = \delta(x)
\end{equation}
注意\autoref{Delta_eq1} 中的等号并不是通常意义的两函数相等, 而是隐含了\autoref{Delta_eq4} 和\autoref{Delta_eq5} 两个条件. 另外在一些教材会把 $\delta(x)$ 的性质简单记为
\begin{equation}\label{Delta_eq2}
\delta(x) =
\begin{cases}
+\infty & (x = 0)\\
0 & (x \ne 0)
\end{cases}
\end{equation}
\begin{equation}\label{Delta_eq3}
\int_{-\infty}^{+\infty} \delta(x) \dd{x} = 1
\end{equation}
这些表达式都需要按照\autoref{Delta_def1} 来严格理解.

\begin{exercise}{高斯函数}
证明高斯分布函数\upref{GausPD}满足
\begin{equation}
\lim_{\alpha\to+\infty}\frac{\alpha}{\sqrt{\pi}} \E^{-{\alpha ^2}x^2} = \delta(x)
\end{equation}
\end{exercise}
\begin{exercise}{sinc 函数}\label{Delta_exe1}
证明 $\sinc$ 函数\upref{sinc}满足(提示: \autoref{sinc_eq1}~\upref{sinc}).
\begin{equation}
\lim_{\alpha\to\infty}\frac{\alpha}{\pi}\sinc^2(\alpha x) = \delta(x)
\end{equation}
\begin{equation}\label{Delta_eq10}
\lim_{\alpha\to\infty}\frac{\alpha}{\pi}\sinc(\alpha x) = \delta(x)
\end{equation}
注意\autoref{Delta_eq10} 不满足\autoref{Delta_def1} (从\autoref{Delta_fig2} 可知)却满足\autoref{Delta_eq11}. 所以根据某种更宽泛的定义也是 delta 函数.
\end{exercise}
\begin{figure}[ht]
\centering
\includegraphics[width=10cm]{./figures/Delta_2.pdf}
\caption{$\alpha\sinc(\alpha x)$ 的函数图} \label{Delta_fig2}
\end{figure}

\subsubsection{多元 delta 函数}
\addTODO{……}

\subsection{性质}
若一个等式中出现了 $\delta$ 函数, 那么其意义是先将 $\delta(x)$ 替换为\autoref{Delta_def1} 中的 $\delta_\alpha(x)$. 该\textbf{等式在 $\alpha\to+\infty$ 的极限时成立}.

例如 $\delta(x)$ 一个重要的性质是: 对任意在 $x = x_0$ 处连续函数 $f(x)$, 有
\begin{equation}\label{Delta_eq7}
\int_{-\infty}^{+\infty} f(x) \delta(x - x_0) \dd{x}= f(x_0)
\end{equation}
使用\autoref{Delta_def1} 中的 $\delta_\alpha(x)$ 表示, 该等式的严格意义是(再次注意极限和积分不可交换, 且极限必须在最外面)
\begin{equation}\label{Delta_eq11}
\lim_{\alpha\to+\infty}\int_{-\infty}^{+\infty} f(x) \delta_\alpha(x - x_0) \dd{x}= f(x_0)
\end{equation}

我们来简略地证明\autoref{Delta_eq7} : 然后令 $\epsilon\to 0$, 这样 $f(x)$ 在 $[x_0 - \epsilon, x_0 + \epsilon]$ 内的值就趋近于常数 $f(x_0)$. 所以有
\begin{equation}\ali{
\int_{-\infty}^{+\infty} f(x) \delta(x - x_0) \dd{x} &= \int_{x_0-\epsilon}^{x_0+\epsilon} f(x) \delta(x - x_0)\dd{x} \\
&= f(x_0)\int_{x_0-\epsilon}^{x_0+\epsilon} \delta(x - x_0)\dd{x} = f(x_0)
}\end{equation}
证毕.

$\delta$ 函数的另一个性质是
\begin{equation}
\delta[f(x)] = \delta[f'(0)x] = \frac{1}{\abs{f'(0)}} \delta(x)
\end{equation}
我们不妨从几何上来(不严谨地)证明这个性质: 首先在原点附近的无穷小区间内, 有 $f(x) = f'(0)x$, 第一个等号成立. 与 $\delta(x)$ 相比较, $\delta[f'(0)x]$ 的图像在 $x$ 方向变窄了 $\abs{f'(0)}$ 倍, 所以函数曲线下的面积变为原来的 $1/\abs{f'(0)}$ 倍, 故 $\abs{f'(0)}\delta[f(x)]$ 下的面积是 $1$, 所以有 $\abs{f'(0)}\delta[f(x)] = \delta(x)$. 证毕.

\subsection{其他应用}
再次强调我们不能 “按字面意思” 理解任何含有 $\delta$ 函数的等式.
\begin{example}{}\label{Delta_ex1}
在傅里叶分析中, 时常会看到
\begin{equation}\label{Delta_eq8}
\int_{-\infty}^{+\infty} \E^{\I kx}\dd{x} = 2\pi \delta(k)
\end{equation}
然而严格来说, 左边的积分并不收敛, 所以不能 “按字面意思” 理解该式. 要严格证明, 上式两边除以 $2\pi$, 令有限区间 $[-\alpha,\alpha]$ 内的积分为
\begin{equation}
\delta_\alpha(k) = \frac{1}{2\pi}\int_{-\alpha}^{\alpha} \E^{\I kx}\dd{x} = \frac{\sin(\alpha k)}{\pi k} = \frac{\alpha}{\pi} \sinc(\alpha k)
\end{equation}
由\autoref{Delta_exe1} 中的结论得
\begin{equation}
\lim_{\alpha\to+\infty} \delta_\alpha(k) = \delta(k)
\end{equation}
证毕.
\end{example}

\begin{exercise}{}
证明
\begin{equation}\label{Delta_eq9}
\begin{aligned}
&\int_{-\infty}^{+\infty} \sin(k'x) \sin(kx) \dd{x} = \int_{-\infty}^{+\infty} \cos(k'x) \cos(kx) \dd{x}\\
&= \pi\delta(k' - k) - \pi\delta(k' + k)
\end{aligned}
\end{equation}
提示:使用\autoref{CTrig_eq1}~\upref{CTrig},\autoref{CTrig_eq2}~\upref{CTrig} 和\autoref{Delta_eq8} .
\end{exercise}

\begin{example}{}
请证明
\begin{equation}
\int_{-\infty}^{+\infty} \delta(x-x_1) \delta(x-x_2) \dd{x} = \delta(x_1 - x_2)
\end{equation}
注意由此可得积分 $\int_{-\infty}^{+\infty}\delta(x)^2 \dd{x} = +\infty$, 即不收敛.

证明: 考虑和上文一样的含参函数 $\delta_\alpha(x)$, 令 $I_\alpha = \int_{-\infty}^{+\infty} \delta_\alpha(x) \dd{x}$, 有 $\lim_{\alpha\to\infty} I_\alpha = 1$. 再令
\begin{equation}
f_\alpha(x_1, x_2) = \int_{-\infty}^{+\infty} \delta_\alpha(x-x_1) \delta_\alpha(x-x_2) \dd{x}
\end{equation}
我们希望证明 $\lim_{\alpha\to\infty }f_\alpha(x_1, x_2) = \delta(x_1 - x_2)$. 首先对于给定的 $x_1 \ne x_2$ 显然有 $\lim_{\alpha\to\infty }f_\alpha(x_1, x_2) = 0$. 所以只需证明
\begin{equation}
\lim_{\alpha\to\infty }\int_{-\infty}^{+\infty} f_\alpha(x_1, x_2) \dd{x_2} = 1
\end{equation}
交换积分顺序得
\begin{equation}
\begin{aligned}
&\lim_{\alpha\to\infty }\int_{-\infty}^{+\infty}\int_{-\infty}^{+\infty} \delta_\alpha(x-x_1) \delta_\alpha(x-x_2) \dd{x}\dd{x_2}\\
&= \lim_{\alpha\to\infty }\int_{-\infty}^{+\infty} \delta_\alpha(x-x_1) \int_{-\infty}^{+\infty} \delta_\alpha(x-x_2)\dd{x_2} \dd{x}\\
&= \lim_{\alpha\to\infty } I_\alpha \int_{-\infty}^{+\infty} \delta_\alpha(x-x_1)\dd{x}\\
&= \lim_{\alpha\to\infty } I_\alpha^2 = 1
\end{aligned}
\end{equation}
证毕.
\end{example}
