% 点集拓扑学(综述)
% license CCBYSA3
% type Wiki

本文根据 CC-BY-SA 协议转载翻译自维基百科\href{https://en.wikipedia.org/wiki/General_topology}{相关文章}。

\begin{figure}[ht]
\centering
\includegraphics[width=10cm]{./figures/16b10920f52e4ba5.png}
\caption{} \label{fig_DJTP_1}
\end{figure}
在数学中,一般拓扑(或称点集拓扑)是拓扑学的一个分支,主要研究拓扑学中使用的基本集合论定义和构造。它是大多数其他拓扑学分支的基础,包括微分拓扑、几何拓扑和代数拓扑。

点集拓扑中的基本概念是连续性、紧致性和连通性:
\begin{itemize}
\item 连续函数直观上是将相邻的点映射到相邻的点。
\item 紧致集是指可以被有限多个任意小的集合覆盖的集合。
\item 连通集是指不能被分成两个彼此远离的部分的集合。
\end{itemize}
“附近”、“任意小”和“远离”这些术语都可以通过使用开集的概念来精确定义。如果我们改变“开集”的定义,就会改变连续函数、紧致集和连通集的定义。对于“开集”的每一种定义选择,都称为一种拓扑。具有拓扑的集合称为拓扑空间。

度量空间是拓扑空间中的一个重要类别,在这些空间中,可以为集合中的点对定义一个实数的非负距离,也称为度量。具有度量简化了许多证明,且许多最常见的拓扑空间都是度量空间。
\subsection{历史}
一般拓扑学起源于多个领域,其中最重要的包括:
\begin{itemize}
\item 对实数线子集的详细研究(曾被称为点集的拓扑;这种用法现已过时)
\item 流形概念的引入
\item 在功能分析的早期阶段对度量空间,特别是赋范线性空间的研究。
\end{itemize}
一般拓扑学在大约1940年形成了现在的形式。可以说,它几乎囊括了连续性直觉的所有内容,以一种技术上足够的形式,能够应用于数学的任何领域。

