% 泡利矩阵(综述)
% license CCBYSA3
% type Sum

本文根据 CC-BY-SA 协议转载翻译自维基百科\href{https://en.wikipedia.org/wiki/Pauli_matrices}{相关文章}。

\begin{figure}[ht]
\centering
\includegraphics[width=6cm]{./figures/943b6a85e9584185.png}
\caption{沃尔夫冈·泡利(1900–1958),约摄于 1924 年。泡利因提出泡利不相容原理而在 1945 年获颁诺贝尔物理学奖**,提名人是阿尔伯特·爱因斯坦。} \label{fig_PLJZ_1}
\end{figure}
在数学物理和数学中,泡利矩阵是一组三个$2\times2$ 的复矩阵,它们具有迹为零、厄米特、自反和酉的性质。这些矩阵通常用希腊字母$\sigma$(σ)表示,在涉及同位旋对称性时,有时也用$\tau$(τ)表示。
$$
\sigma_1 = \sigma_x = 
\begin{pmatrix}
0 & 1 \\
1 & 0
\end{pmatrix},
\quad
\sigma_2 = \sigma_y = 
\begin{pmatrix}
0 & -i \\
i & 0
\end{pmatrix},
\quad
\sigma_3 = \sigma_z = 
\begin{pmatrix}
1 & 0 \\
0 & -1
\end{pmatrix}.~
$$
这些矩阵因物理学家 沃尔夫冈·泡利而得名。在量子力学中,它们出现在泡利方程中,用于描述粒子的自旋与外部电磁场相互作用的情况。它们还可以用来表示两种偏振滤光片的相互作用状态,例如水平/垂直偏振、45 度偏振(左/右)以及圆偏振(左/右)的状态。

每个泡利矩阵都是厄米特矩阵。再加上单位矩阵 $I$(有时被视为第零个泡利矩阵$\sigma_0$),这些矩阵在实数加法下构成了所有 $2 \times 2$ 厄米特矩阵的向量空间的基。这意味着,任何 $2 \times 2$ 厄米特矩阵都可以唯一地写成泡利矩阵的实数线性组合。

泡利矩阵满足以下有用的乘积关系:
$$
\sigma_i \sigma_j = \delta_{ij} + i \, \epsilon_{ijk} \, \sigma_k,~
$$
其中 $\delta_{ij}$ 是克罗内克 δ,$\epsilon_{ijk}$ 是三阶 Levi-Civita 符号。由于厄米特算符在量子力学中表示可观测量,因此泡利矩阵张成了复二维希尔伯特空间中所有可观测量的空间。在泡利的研究语境中,$\sigma_k$ 表示三维欧几里得空间 $\mathbb{R}^3$ 中沿第 $k$ 个坐标轴方向的自旋对应的可观测量。

此外,当泡利矩阵乘以 $i$ 使之成为反厄米特矩阵后,它们还可以生成李代数意义下的变换:矩阵 $i\sigma_1, i\sigma_2, i\sigma_3$ 构成了实李代数 $\mathfrak{su}(2)$ 的基,并通过指数映射生成特殊酉群 $SU(2)$。由三个位移矩阵 $\sigma_1, \sigma_2, \sigma_3$ 所生成的代数同构于 $\mathbb{R}^3$ 的克利福德代数;而由 $i\sigma_1, i\sigma_2, i\sigma_3$ 生成的带幺结合代数则与四元数代数$\mathbb{H}$ 完全同构。
\subsection{代数性质}
所有三个位相矩阵都可以压缩成一个表达式:
$$
\sigma_j =
\begin{pmatrix}
\delta_{j3} & \delta_{j1} - i\delta_{j2} \\
\delta_{j1} + i\delta_{j2} & -\delta_{j3}
\end{pmatrix},~
$$
其中,$\delta_{jk}$ 是克罗内克δ函数,当 $j = k$ 时取值为 $+1$,否则为 $0$。这个表达式的好处在于,可以通过代入 $j = 1, 2, 3$ 来“选择”三个位相矩阵中的任意一个,这在需要进行代数推导但不针对某个具体矩阵时非常实用。

这些矩阵具有自反性:
$$
\sigma_1^2 = \sigma_2^2 = \sigma_3^2 = -i\,\sigma_1\sigma_2\sigma_3 =
\begin{pmatrix}
1 & 0 \\
0 & 1
\end{pmatrix}
= I,~
$$
其中 $I$ 是单位矩阵。

位相矩阵的行列式和迹为:
$$
\det(\sigma_j) = -1, \quad
\operatorname{tr}(\sigma_j) = 0,~
$$
由此可知,每个矩阵 $\sigma_j$ 的特征值为 $+1$ 和 $-1$。

当将单位矩阵 $I$(有时记作 $\sigma_0$)包括在内时,位相矩阵与 $I$ 一起构成:实数域 $\mathbb{R}$ 上 $2 \times 2$ 厄米矩阵空间 $\mathcal{H}_2$的一个 Hilbert–Schmidt 正交基;复数域 $\mathbb{C}$ 上所有 $2 \times 2$ 矩阵空间 $\mathcal{M}_{2,2}(\mathbb{C})$的一个正交基。
\subsubsection{对易与反对易关系}
\textbf{对易关系}

保利矩阵满足如下对易关系:
$$
[\sigma_j, \sigma_k] = 2i \, \varepsilon_{jkl} \, \sigma_l,~
$$
其中 $\varepsilon_{jkl}$ 是Levi-Civita 符号。


这些对易关系表明,保利矩阵是以下李代数表示的生成元:
$$
(\mathbb{R}^3, \times) \;\cong\; \mathfrak{su}(2) \;\cong\; \mathfrak{so}(3),~
$$
即它们与三维实数空间 $\mathbb{R}^3$ 的叉积代数、$\mathfrak{su}(2)$ 李代数和 $\mathfrak{so}(3)$ 李代数互相同构。

\textbf{反对易关系}

保利矩阵还满足以下反对易关系:
$$
\{\sigma_j, \sigma_k\} = 2 \delta_{jk} I,~
$$
其中,$\{\sigma_j, \sigma_k\}$ 定义为:$\sigma_j \sigma_k + \sigma_k \sigma_j$,$\delta_{jk}$ 是 Kronecker δ 符号,$I$ 表示$2 \times 2$ 单位矩阵。


这些反对易关系说明,保利矩阵是三维实空间$\mathbb{R}^3$Clifford 代数$\mathrm{Cl}_3(\mathbb{R})$ 的一个表示的生成元。

利用 Clifford 代数的常规构造方式,可以定义 $\mathfrak{so}(3)$ 李代数的生成元:$\sigma_{jk} = \tfrac{1}{4}[\sigma_j, \sigma_k]$,从而重新得到上文的对易关系(仅相差一个数值因子)。

下面给出一些显式的对易子和反对易子的示例:
\begin{figure}[ht]
\centering
\includegraphics[width=10cm]{./figures/5eeeab185b454847.png}
\caption{} \label{fig_PLJZ_2}
\end{figure}
\subsubsection{特征向量与特征值}
每一个(厄米的)泡利矩阵都有两个特征值:$+1$ 和 $-1$。相应的归一化特征向量为:
$$
\psi_{x+} = \frac{1}{\sqrt{2}}
\begin{bmatrix}
1\\
1
\end{bmatrix},
\quad
\psi_{x-} = \frac{1}{\sqrt{2}}
\begin{bmatrix}
1\\
-1
\end{bmatrix},~
$$
$$
\psi_{y+} = \frac{1}{\sqrt{2}}
\begin{bmatrix}
1\\
i
\end{bmatrix},
\quad
\psi_{y-} = \frac{1}{\sqrt{2}}
\begin{bmatrix}
1\\
-i
\end{bmatrix},~
$$
$$
\psi_{z+} =
\begin{bmatrix}
1\\
0
\end{bmatrix},
\quad
\psi_{z-} =
\begin{bmatrix}
0\\
1
\end{bmatrix}.~
$$
\subsection{泡利向量}
泡利向量定义为\(^\text{[b]}\):
$$
\vec{\sigma} = \sigma_1 \hat{x}_1 + \sigma_2 \hat{x}_2 + \sigma_3 \hat{x}_3,~
$$
其中 $\hat{x}_1$、$\hat{x}_2$、$\hat{x}_3$ 分别是更常见的 $\hat{x}$、$\hat{y}$、$\hat{z}$ 的等价记法。

泡利向量提供了一种将三维向量基底映射到泡利矩阵基底的机制\(^\text{[2]}\),如下所示:
$$
\vec{a} \cdot \vec{\sigma} 
= \sum_{k, l} a_k \sigma_\ell (\hat{x}_k \cdot \hat{x}_\ell) 
= \sum_k a_k \sigma_k
= 
\begin{pmatrix}
a_3 & a_1 - i a_2 \\
a_1 + i a_2 & -a_3
\end{pmatrix}.~
$$
更正式地说,该映射定义了从 $\mathbb{R}^3$ 到迹为零的厄米 $2\times 2$ 矩阵向量空间的映射。该映射通过矩阵函数编码了 $\mathbb{R}^3$ 作为赋范向量空间和李代数(其中向量叉乘作为李括号)的结构,使得该映射成为一个李代数同构。
从表示论的角度看,泡利矩阵因此可视为交织算子。

另一种视角下,泡利向量也可以看作是一个值为 $2\times 2$ 厄米迹为零矩阵的对偶向量,即:$\text{Mat}_{2\times 2}(\mathbb{C}) \otimes (\mathbb{R}^3)^*$,
它将向量 $\vec{a}$ 映射为:$\vec{a} \mapsto \vec{a} \cdot \vec{\sigma}$.
\subsubsection{完备性关系}
向量 $\vec{a}$ 的每一个分量都可以从矩阵中恢复出来(见下方的完备性关系):
$$
\frac{1}{2} \operatorname{tr} \bigl( (\vec{a} \cdot \vec{\sigma}) \vec{\sigma} \bigr) = \vec{a}.~
$$
这实际上是映射$\vec{a} \;\mapsto\; \vec{a} \cdot \vec{\sigma}$的逆映射,从而明确表明该映射是一个双射。
\subsubsection{行列式}
范数可以通过行列式(差一个负号)给出:
$$
\det(\vec{a} \cdot \vec{\sigma}) = -\vec{a} \cdot \vec{a} = -|\vec{a}|^2.~
$$
接着,考虑一个$SU(2)$矩阵 $U$ 在这类矩阵空间上的共轭作用:
$$
U * (\vec{a} \cdot \vec{\sigma}) := U\,(\vec{a} \cdot \vec{\sigma})\,U^{-1},~
$$
我们可以得到:$\det(U * (\vec{a} \cdot \vec{\sigma})) = \det(\vec{a} \cdot \vec{\sigma})$,并且 $U * (\vec{a} \cdot \vec{\sigma})$ 仍然是厄米且无迹的。因此我们可以定义:$U * (\vec{a} \cdot \vec{\sigma}) = \vec{a}' \cdot \vec{\sigma}$,其中,$\vec{a}'$ 与 $\vec{a}$ 具有相同的范数,于是我们可以把 $U$ 解释为三维空间中的一次旋转。事实上,矩阵 $U$ 的特殊限制还意味着该旋转是保持方向的。这就允许我们定义一个映射:$R: \mathrm{SU}(2) \;\longrightarrow\; \mathrm{SO}(3)$其具体形式为:
$$
U * (\vec{a} \cdot \vec{\sigma}) = \vec{a}' \cdot \vec{\sigma} := (R(U)\,\vec{a}) \cdot \vec{\sigma},~
$$
其中 $R(U) \in \mathrm{SO}(3)$。这个映射就是 $\mathrm{SU}(2)$ 对 $\mathrm{SO}(3)$ 的双覆盖的具体实现,因此说明了:$\mathrm{SU}(2) \;\cong\; \mathrm{Spin}(3)$.此外,$R(U)$ 的各个分量可以通过如下迹运算恢复出来:
$$
R(U)_{ij} = \frac{1}{2} \operatorname{tr}(\sigma_i U \sigma_j U^{-1})~
$$
\subsubsection{叉积}
叉积可以由矩阵对易子(差一个 $2i$ 的因子)表示:
$$
[\vec{a} \cdot \vec{\sigma},\; \vec{b} \cdot \vec{\sigma}] = 2i\,(\vec{a} \times \vec{b}) \cdot \vec{\sigma}.~
$$
实际上,范数的存在源于 $\mathbb{R}^3$ 本身是一个李代数(参见 Killing 型)。

这种叉积还可以用于证明上述映射保持空间方向的性质。
\subsubsection{特征值与特征向量}
矩阵$\vec{a} \cdot \vec{\sigma}$的特征值为$\pm |\vec{a}|$.这可以直接从矩阵的无迹性以及显式计算行列式得出。

从更抽象的角度来看,即便不计算行列式(而计算行列式需要用到泡利矩阵的具体性质),也能通过如下关系推导出该结果:$(\vec{a} \cdot \vec{\sigma})^2 - |\vec{a}|^2 = 0$,因为该式可以因式分解为:$(\vec{a} \cdot \vec{\sigma} - |\vec{a}|)(\vec{a} \cdot \vec{\sigma} + |\vec{a}|) = 0$.线性代数中的一个标准结论是:如果一个线性映射满足由若干不相同一次因式组成的多项式方程,那么该映射是可对角化的。因此,这意味着$\vec{a} \cdot \vec{\sigma}$是可对角化的,其特征值为$\pm |\vec{a}|$.此外,由于$\vec{a} \cdot \vec{\sigma}$是无迹矩阵,它必然有且只有一个正特征值 $+|\vec{a}|$ 和一个负特征值 $-|\vec{a}|$。

它的归一化特征向量是:
$$
\psi_{+} = 
\frac{1}{\sqrt{2\left|{\vec{a}}\right|\left(a_3+\left|{\vec{a}}\right|\right)}} 
\begin{bmatrix}
a_3 + \left|{\vec{a}}\right| \\
a_1 + i a_2
\end{bmatrix},
\qquad
\psi_{-} =
\frac{1}{\sqrt{2\left|{\vec{a}}\right|\left(a_3+\left|{\vec{a}}\right|\right)}} 
\begin{bmatrix}
i a_2 - a_1 \\
a_3 + \left|{\vec{a}}\right|
\end{bmatrix}.~
$$
这些表达式在 $a_3 \to -|{\vec{a}}|$ 时会出现奇异性。可以通过令${\vec{a}} = |{\vec{a}}|(\epsilon, 0, -(1-\epsilon^2/2))$并取极限 $\epsilon \to 0$ 来修正,这样可以得到 $\sigma_z$ 的正确特征向量 $(0,1)$ 和 $(1,0)$。

另一种方法是使用球坐标:${\vec{a}} = a(\sin\vartheta\cos\varphi, \sin\vartheta\sin\varphi, \cos\vartheta)$,这样可以得到特征向量:
$$
\psi_+ = \big(\cos(\vartheta/2), \ \sin(\vartheta/2)\exp(i\varphi)\big),~
$$
以及
$$
\psi_- = \big(-\sin(\vartheta/2)\exp(-i\varphi), \ \cos(\vartheta/2)\big)~
$$
\subsubsection{泡利四矢量}
在旋量理论中使用的泡利四矢量写作$\sigma^\mu$其分量为
$$
\sigma^\mu = (I, {\vec{\sigma}}).~
$$
这定义了一个从 $\mathbb{R}^{1,3}$ 到厄米矩阵向量空间的映射:
$$
x_\mu \;\mapsto\; x_\mu \sigma^\mu ,~
$$
该映射的行列式也编码了闵可夫斯基度规(采用“多数负号”的约定):
$$
\det(x_\mu \sigma^\mu) = \eta(x, x) .~
$$
该四矢量还满足一个完备性关系。为此,方便起见,引入第二个泡利四矢量:
$$
\bar{\sigma}^\mu = (I, -{\vec{\sigma}}) .~
$$
并允许使用闵可夫斯基度规张量进行指标升降。这样,关系可以写为:
$$
x_\nu = \frac{1}{2} \operatorname{tr} \left( \bar{\sigma}_\nu \, (x_\mu \sigma^\mu) \right).~
$$
与泡利三矢量的情形类似,我们也可以找到一个在$\mathbb{R}^{1,3}$上作为等距变换**作用的矩阵群。在这种情况下,该矩阵群是:$\mathrm{SL}(2, \mathbb{C})$,因此我们有:$\mathrm{SL}(2, \mathbb{C}) \;\cong\; \mathrm{Spin}(1,3)$.与前面的推导类似,对于$S \in \mathrm{SL}(2, \mathbb{C})$可以显式写出对应洛伦兹变换矩阵的分量为:
$$
\Lambda(S)^{\mu}{}_{\nu} = 
\frac{1}{2} 
\operatorname{tr}\!\left( \bar{\sigma}_\nu \, S \, \sigma^\mu \, S^\dagger \right).~
$$
实际上,行列式的这个性质可以从$\sigma^\mu$的迹运算特性抽象地推出。对于 $2\times 2$矩阵,成立如下恒等式:
$$
\det(A+B) = \det(A) + \det(B) + \operatorname{tr}(A)\operatorname{tr}(B) - \operatorname{tr}(AB).~
$$
也就是说,“交叉项”可以通过迹来表示。当$A, B$分别取不同的$\sigma^\mu$时,这些交叉项会消失。于是可以得到(显式写出求和):$\det\!\left( \sum_{\mu} x_\mu \sigma^\mu \right)= \sum_{\mu} \det\!\left( x_\mu \sigma^\mu \right).$由于这些矩阵都是$2\times 2$阶矩阵,所以该式进一步化简为:$\sum_{\mu} x_\mu^2 \det(\sigma^\mu)=\eta(x, x)$,即行列式自然反映了闵可夫斯基度规 $\eta(x,x)$ 的结构。
\subsubsection{与点积和叉积的关系}
泡利矢量优雅地将对易关系和反对易关系映射为相应的向量运算。将对易子与反对易子相加,可得:
$$
[\sigma_j, \sigma_k] + \{\sigma_j, \sigma_k\}
= (\sigma_j \sigma_k - \sigma_k \sigma_j) + (\sigma_j \sigma_k + \sigma_k \sigma_j)~
$$
$$
2 i \, \varepsilon_{jk\ell} \, \sigma_\ell + 2 \delta_{jk} I = 2 \sigma_j \sigma_k~
$$
因此:
$$
\sigma_j \sigma_k = \delta_{jk} I + i \varepsilon_{jk\ell} \, \sigma_\ell \;.~
$$
将方程两边分别与两个三维向量的分量 $a_p$ 和 $b_q$ 相缩并(它们与泡利矩阵可交换,即 $a_p \sigma_q = \sigma_q a_p$,同理 $b_q$ 也一样),对于每个矩阵 $\sigma_q$ 与向量分量 $a_p$,可得:
$$
a_j b_k \sigma_j \sigma_k 
= a_j b_k \left( i \varepsilon_{jk\ell} \sigma_\ell + \delta_{jk} I \right)~
$$
$$
a_j \sigma_j b_k \sigma_k
= i \varepsilon_{jk\ell} a_j b_k \sigma_\ell + a_j b_k \delta_{jk} I~
$$
最后,将点积与叉积的指标记号转化为向量形式,可以得到:
$$
(\vec{a} \cdot \vec{\sigma})(\vec{b} \cdot \vec{\sigma})
= (\vec{a} \cdot \vec{b}) I
+ i (\vec{a} \times \vec{b}) \cdot \vec{\sigma}~
$$
如果将 $i$ 识别为伪标量 $\sigma_x \sigma_y \sigma_z$,那么右边可以写成:$
a \cdot b + a \wedge b$这也正是几何代数中两个向量乘积的定义。
如果我们将自旋算符定义为$\mathbf{J} = \frac{\hbar}{2} \vec{\sigma}$,那么 $\mathbf{J}$ 满足如下对易关系:
$$
\mathbf{J} \times \mathbf{J} = i\hbar \mathbf{J}.~
$$
等价地,泡利向量满足:
$$
\frac{\vec{\sigma}}{2} \times \frac{\vec{\sigma}}{2} = i \frac{\vec{\sigma}}{2}.~
$$
\subsubsection{一些迹关系}
利用对易关系和反对易关系,可以推导出以下迹公式:
$$
\begin{aligned}
\mathrm{tr}(\sigma_j) &= 0,\\
\mathrm{tr}(\sigma_j \sigma_k) &= 2\delta_{jk},\\
\mathrm{tr}(\sigma_j \sigma_k \sigma_\ell) &= 2i\varepsilon_{jk\ell},\\
\mathrm{tr}(\sigma_j \sigma_k \sigma_\ell \sigma_m) &= 2(\delta_{jk}\delta_{\ell m} - \delta_{j\ell}\delta_{km} + \delta_{jm}\delta_{k\ell}).
\end{aligned}~
$$
如果将矩阵 $\sigma_0 = I$(单位矩阵)也包括进来,这些关系可以扩展为:
$$
\begin{aligned}
\mathrm{tr}(\sigma_\alpha) &= 2\delta_{0\alpha},\\
\mathrm{tr}(\sigma_\alpha \sigma_\beta) &= 2\delta_{\alpha\beta},\\
\mathrm{tr}(\sigma_\alpha \sigma_\beta \sigma_\gamma) &= 2\sum_{(\alpha\beta\gamma)} \delta_{\alpha\beta} \delta_{0\gamma} 
- 4\delta_{0\alpha}\delta_{0\beta}\delta_{0\gamma} 
+ 2i\varepsilon_{0\alpha\beta\gamma},\\
\mathrm{tr}(\sigma_\alpha \sigma_\beta \sigma_\gamma \sigma_\mu) &= 2(\delta_{\alpha\beta}\delta_{\gamma\mu} 
- \delta_{\alpha\gamma}\delta_{\beta\mu} 
+ \delta_{\alpha\mu}\delta_{\beta\gamma})\\
&\quad + 4(\delta_{\alpha\gamma}\delta_{0\beta}\delta_{0\mu} 
+ \delta_{\beta\mu}\delta_{0\alpha}\delta_{0\gamma}) 
- 8\delta_{0\alpha}\delta_{0\beta}\delta_{0\gamma}\delta_{0\mu} 
+ 2i\sum_{(\alpha\beta\gamma\mu)} \varepsilon_{0\alpha\beta\gamma} \delta_{0\mu}.
\end{aligned}~
$$
其中,希腊字母索引 $\alpha, \beta, \gamma, \mu$ 取值于 $\{0, x, y, z\}$,而符号$\sum_{(\alpha \ldots)}$ 表示对所列指标的循环排列求和。
\subsubsection{Pauli 向量的指数形式}
设
$$
\vec{a} = a \hat{n}, \quad |\hat{n}| = 1,~
$$
其中 $\hat{n}$ 是单位向量,则对于偶数次幂 $2p$($p = 0, 1, 2, 3, \ldots$),有:
$$
(\hat{n} \cdot \vec{\sigma})^{2p} = I,~
$$
这可以先通过 $p = 1$ 的情况结合反对易关系来验证。按照约定,$p = 0$ 的情况定义为单位矩阵 $I$。

对于奇数次幂 $2q+1$($q = 0, 1, 2, 3, \ldots$),有:
$$
(\hat{n} \cdot \vec{\sigma})^{2q+1} = \hat{n} \cdot \vec{\sigma}.~
$$
利用矩阵指数以及正弦和余弦的泰勒展开式,有:

$$
e^{ia(\hat{n} \cdot \vec{\sigma})}
= \sum_{k=0}^{\infty} \frac{i^k [a (\hat{n} \cdot \vec{\sigma})]^k}{k!}
= \sum_{p=0}^{\infty} \frac{(-1)^p (a \hat{n} \cdot \vec{\sigma})^{2p}}{(2p)!}
+ i \sum_{q=0}^{\infty} \frac{(-1)^q (a \hat{n} \cdot \vec{\sigma})^{2q+1}}{(2q+1)!}~
$$
由前面的结论,偶数次幂 $(\hat{n} \cdot \vec{\sigma})^{2p} = I$,奇数次幂 $(\hat{n} \cdot \vec{\sigma})^{2q+1} = \hat{n} \cdot \vec{\sigma}$,可以化简为:
$$
e^{ia(\hat{n} \cdot \vec{\sigma})}
= I \sum_{p=0}^{\infty} \frac{(-1)^p a^{2p}}{(2p)!}
+ i (\hat{n} \cdot \vec{\sigma}) \sum_{q=0}^{\infty} \frac{(-1)^q a^{2q+1}}{(2q+1)!}~
$$
最后一行中,第一个求和是余弦函数的展开式,第二个求和是正弦函数的展开式,因此最终得到:
$$
e^{ia(\hat{n} \cdot \vec{\sigma})}
= I \cos a
+ i (\hat{n} \cdot \vec{\sigma}) \sin a~
$$
这与欧拉公式类似,但扩展到了四元数的情形。具体来说:
$$
e^{ia\sigma_{1}}
= 
\begin{pmatrix}
\cos a & i\sin a \\
i\sin a & \cos a
\end{pmatrix},
\quad
e^{ia\sigma_{2}}
= 
\begin{pmatrix}
\cos a & \sin a \\
-\sin a & \cos a
\end{pmatrix},
\quad
e^{ia\sigma_{3}}
= 
\begin{pmatrix}
e^{ia} & 0 \\
0 & e^{-ia}
\end{pmatrix}.~
$$
注意:
$$
\det[ia(\hat{n}\cdot\vec{\sigma})] = a^2,~
$$
而指数矩阵本身的行列式为1,这意味着它正是$SU(2)$群的一个一般元素。

对于更抽象的推广,公式 (2) 对任意$2\times2$矩阵的版本可以参考“矩阵指数”条目。此外,对于在$a$和$-a$ 上解析的函数$f$,应用Sylvester 公式可以得到更一般的形式:
$$
f\big(a(\hat{n}\cdot\vec{\sigma})\big)
= 
I \cdot \frac{f(a) + f(-a)}{2}
+
(\hat{n}\cdot\vec{\sigma}) \cdot \frac{f(a) - f(-a)}{2}.~
$$
\textbf{SU(2) 群的群合成律}

直接应用公式 (2) 可以得到$SU(2)$群合成律的一种参数化形式。[c]我们可以通过解如下方程直接求出参数 $c$:
$$
\begin{aligned}
e^{ia(\hat{n}\cdot\vec{\sigma})} \; e^{ib(\hat{m}\cdot\vec{\sigma})}
&= I\big(\cos a\cos b - \hat{n}\cdot\hat{m}\,\sin a \sin b\big)+ i\big(\hat{n}\,\sin a \cos b + \hat{m}\,\sin b \cos a - \hat{n}\times\hat{m}\,\sin a \sin b\big)\cdot\vec{\sigma} \\
&= I\cos c + i(\hat{k}\cdot\vec{\sigma})\sin c \\
&= e^{ic(\hat{k}\cdot\vec{\sigma})},
\end{aligned}~
$$
该公式给出了SU(2)的一般群乘法关系,其中:

$c$ 满足:
  $$
  \cos c = \cos a \cos b - \hat{n} \cdot \hat{m} \, \sin a \sin b,~
  $$
这正是球面余弦定理。

而对应的单位向量$\hat{k}$则为:
  $$
  \hat{k} = \frac{1}{\sin c} 
  \big(
  \hat{n}\,\sin a \cos b 
  + \hat{m}\,\sin b \cos a 
  - (\hat{n} \times \hat{m}) \, \sin a \sin b
  \big) .~
  $$
因此,该群元素中的合成旋转参数(即该情况下 BCH 展开的闭合形式)可以简写为:
$$
e^{ic(\hat{k}\cdot\vec{\sigma})} 
= 
\exp \!\left(
  i \, \frac{c}{\sin c} 
  \big(
  \hat{n}\,\sin a \cos b 
  + \hat{m}\,\sin b \cos a 
  - \hat{n}\times\hat{m} \, \sin a \sin b
  \big) 
  \cdot \vec{\sigma}
\right).~
$$
当 $\hat{n}$ 与 $\hat{m}$ 平行时,$\hat{k}$ 也平行于它们,此时有:$c = a + b$.

\textbf{伴随作用}

同样可以很直接地推导出Pauli 向量的伴随作用,也就是任意角度$a$沿任意轴$\hat{n}$的旋转:
$$
R_n(-a)~\vec{\sigma}~R_n(a)
= e^{i\frac{a}{2}(\hat{n}\cdot\vec{\sigma})}~\vec{\sigma}~e^{-i\frac{a}{2}(\hat{n}\cdot\vec{\sigma})}
= \vec{\sigma}\cos(a)
+ (\hat{n} \times \vec{\sigma})\sin(a)
+ \hat{n}(\hat{n} \cdot \vec{\sigma})(1-\cos(a))\,.~
$$
该公式表明,$\vec{\sigma}$ 在$SU(2)$旋转下,表现为绕 $\hat{n}$ 轴的三维旋转。

将任意单位向量与上述公式点乘,可以得到任意单量子比特算符在任意旋转下的变换形式。例如,可以验证:$R_y\!\left(-\frac{\pi}{2}\right)\,\sigma_x\,
R_y\!\left(\frac{\pi}{2}\right)= \hat{x} \cdot (\hat{y} \times \vec{\sigma})= \sigma_z\,$这说明绕 $y$ 轴旋转 $\pi/2$ 角后,$\sigma_x$ 算符会变换为 $\sigma_z$。
\subsection{完备关系}
在另一种常见的Pauli 矩阵记号中,向量指标 $k$ 写作上标,矩阵的行、列指标作为下标。也就是说,第 $k$ 个 Pauli 矩阵第 $\alpha$ 行、第 $\beta$ 列的元素记作 $\sigma^k_{\alpha\beta}$。

在这种记号下,Pauli 矩阵的完备关系可以写为:
$$
\vec{\sigma}_{\alpha\beta} \cdot \vec{\sigma}_{\gamma\delta}
\;\equiv\;
\sum_{k=1}^{3} \sigma^k_{\alpha\beta}\,\sigma^k_{\gamma\delta}
= 2\,\delta_{\alpha\delta}\,\delta_{\beta\gamma}
- \delta_{\alpha\beta}\,\delta_{\gamma\delta}\,.~
$$

\textbf{证明}

由于Pauli 矩阵与单位矩阵$I$ 一起,构成了所有 $2 \times 2$ 复矩阵$\mathcal{M}_{2,2}(\mathbb{C})$的Hilbert 空间的一个正交基,因此,任意一个$2 \times 2$复矩阵$M$都可以表示为:
$$
M = c I + \sum_k a_k \sigma^k,~
$$
其中:$c$ 是一个复数,$\vec{a} = (a_1,a_2,a_3)$ 是一个三维复向量。利用Pauli矩阵的性质,可以很容易得出:
$$
\operatorname{tr}(\sigma^j \sigma^k) = 2\delta_{jk},~
$$
其中 “$\operatorname{tr}$” 表示矩阵的迹运算。据此,可以计算:
$$
c = \tfrac{1}{2} \operatorname{tr} M,  
\quad
a_k = \tfrac{1}{2} \operatorname{tr} (\sigma^k M).~
$$
于是可以得到:
$$
2M = I \, \operatorname{tr} M + \sum_k \sigma^k \, \operatorname{tr} (\sigma^k M).~
$$
将其改写为矩阵分量的形式:
$$
2 M_{\alpha \beta} 
= \delta_{\alpha \beta} M_{\gamma \gamma} 
+ \sum_k \sigma^k_{\alpha \beta} \, \sigma^k_{\gamma \delta} \, M_{\delta \gamma},~
$$
其中,$\gamma$ 和 $\delta$ 是重复指标,表示求和。由于这个式子对任意矩阵$M$都成立,因此直接推出前文给出的完备关系。证毕(Q.E.D.)。

正如前文所述,常见的做法是用$\sigma^0$表示$2 \times 2$单位矩阵,因此有$\sigma^0_{\alpha\beta} = \delta_{\alpha\beta}$.于是,完备关系也可以改写为:
$$
\sum_{k=0}^{3} \sigma^k_{\alpha\beta} \, \sigma^k_{\gamma\delta} = 2\, \delta_{\alpha\delta}\, \delta_{\beta\gamma}.~
$$
此外,任意厄米复 $2 \times 2$ 矩阵都可以用单位矩阵和 Pauli 矩阵的线性组合来表示。这一事实也直接导致了Bloch 球表示法,用于描述 $2 \times 2$混合态的密度矩阵(即迹为 1 的半正定 $2 \times 2$ 矩阵)。其推导思路如下:1. 将任意的厄米矩阵写成 $\{\sigma_0, \sigma_1, \sigma_2, \sigma_3\}$ 的实系数线性组合;2. 再在这个表达式的基础上施加半正定性和迹为 1的条件;由此就可以得到混合态密度矩阵在 Bloch 球上的表示。

对于纯态,在极坐标系下,向量$\vec{a}$表示为:
$$
\vec{a} =
\begin{pmatrix}
\sin\theta \cos\phi \\
\sin\theta \sin\phi \\
\cos\theta
\end{pmatrix},~
$$
对应的幂等密度矩阵为:
$$
\frac{1}{2} \left(\mathbf{1} + \vec{a} \cdot \vec{\sigma}\right)
=
\begin{pmatrix}
\cos^2\left(\frac{\theta}{2}\right) & e^{-i\phi}\sin\left(\frac{\theta}{2}\right)\cos\left(\frac{\theta}{2}\right) \\
e^{+i\phi}\sin\left(\frac{\theta}{2}\right)\cos\left(\frac{\theta}{2}\right) & \sin^2\left(\frac{\theta}{2}\right)
\end{pmatrix}.~
$$
该矩阵作用在以下态矢量上:
$$
\begin{pmatrix}
\cos\left(\frac{\theta}{2}\right) \\
e^{+i\phi}\sin\left(\frac{\theta}{2}\right)
\end{pmatrix},~
$$
其对应的特征值为 $+1$。因此,这个密度矩阵的作用等价于一个投影算符。
\subsubsection{与置换算符的关系}
设 $P_{jk}$ 表示作用在张量积空间$\mathbb{C}^2 \otimes \mathbb{C}^2$中的两个自旋 $\sigma_j$ 和 $\sigma_k$ 之间的换位(置换)算符,其作用为:
$$
P_{jk} \, |\sigma_j \, \sigma_k \rangle = |\sigma_k \, \sigma_j \rangle.~
$$
该算符也可以更明确地写成狄拉克的自旋交换算符:
$$
P_{jk} = \frac{1}{2} \left( \vec{\sigma}_j \cdot \vec{\sigma}_k + 1 \right).~
$$
因此,其本征值为$1$或$-1$。这意味着它可以作为哈密顿量中的一个相互作用项,用来区分对称与反对称本征态的能量本征值,从而实现能级的劈裂。
\subsection{SU(2)}
群$SU(2)$是所有行列式为1的$2\times2$幺正矩阵所构成的李群;它的李代数是所有迹为0的$2\times2$反厄米矩阵的集合。如上所示,通过直接计算可以发现,李代数$\mathfrak{su}_2$是由集合 $\{ i\sigma_k \}$ 张成的三维实代数。紧凑记法为:
$$
\mathfrak{su}(2) = \text{span} \{ i\sigma_1, i\sigma_2, i\sigma_3 \}.
~
$$
因此,每个$i\sigma_j$都可以视为$SU(2)$的一个无穷小生成元。$SU(2)$的任意元素都可以表示为这些三个生成元的线性组合的指数形式,并且它们的乘法规则如保利向量讨论中所示。不过,虽然这种方式足以生成$SU(2)$,但这并不是$SU(2)$的标准表示,因为保利矩阵的特征值在这里采用了非常规的标度。标准的归一化因子取为$\lambda = \frac{1}{2}$,
于是:
$$
\mathfrak{su}(2) = \text{span} \left\{ \frac{i\sigma_1}{2}, \frac{i\sigma_2}{2}, \frac{i\sigma_3}{2} \right\}.~
$$
由于$SU(2)$是紧群,它的Cartan 分解是平凡的。
\subsubsection{SO(3)}
李代数$\mathfrak{su}(2)$与$\mathfrak{so}(3)$同构,后者对应于三维空间的旋转群$SO(3)$。换句话说,可以认为$i\sigma_j$是三维空间中无穷小旋转的一种实现(事实上是最低维的实现)。然而,尽管$\mathfrak{su}(2)$和$\mathfrak{so}(3)$作为李代数是同构的,$SU(2)$和$SO(3)$作为李群却并不同构。实际上,$SU(2)$是$SO(3)$的双覆盖,这意味着从$SU(2)$到$SO(3)$存在一个二对一的群同态。关于两者之间更详细的关系,可参见$SO(3)$与$SU(2)$的关系。
\subsubsection{四元数}
由集合 $\{I,\, i\sigma_1,\, i\sigma_2,\, i\sigma_3\}$ 所张成的实线性空间,与四元数代数$\mathbb{H}$同构;四元数的标准基为$\{1,\,\mathbf{i},\,\mathbf{j},\,\mathbf{k}\}$.该同构可以通过如下映射给出(注意保利矩阵部分的符号是反向的):
$$
1 \;\mapsto\; I, \quad 
\mathbf{i} \;\mapsto\; -\sigma_2\sigma_3 = -i\sigma_1, \quad
\mathbf{j} \;\mapsto\; -\sigma_3\sigma_1 = -i\sigma_2, \quad
\mathbf{k} \;\mapsto\; -\sigma_1\sigma_2 = -i\sigma_3.~
$$
另一种等价的同构方式,是按照相反顺序使用保利矩阵构造映射:
$$
1 \;\mapsto\; I, \quad
\mathbf{i} \;\mapsto\; i\sigma_3, \quad
\mathbf{j} \;\mapsto\; i\sigma_2, \quad
\mathbf{k} \;\mapsto\; i\sigma_1.~
$$
由于四元数 $\mathbb{H}$ 中的单位元集合 $U \subset \mathbb{H}$ 构成一个与 $SU(2)$同构的群,集合 $U$ 为描述$SU(2)$提供了另一种方式。在这种表示下,从$SU(2)$到$SO(3)$的二对一同态也可以直接用保利矩阵来描述。
\subsection{物理学}
\subsubsection{经典力学}
在经典力学中,保利矩阵在 Cayley–Klein 参数的背景下非常有用。\(^\text{[6]}\)
空间中某一点的位置向量 $\vec{x}$ 所对应的矩阵 $P$,可以通过上述保利矩阵向量来定义:
$$
P = \vec{x} \cdot \vec{\sigma} = x\,\sigma_x + y\,\sigma_y + z\,\sigma_z.~
$$
因此,绕x轴旋转角度$\theta$的变换矩阵$Q_\theta$可用保利矩阵与单位矩阵表示为[6]:
$$
Q_{\theta} = \mathbf{1} \cos{\frac{\theta}{2}} + i \sigma_x \sin{\frac{\theta}{2}}.~
$$
类似地,对于一般方向的旋转,利用保利矩阵向量的公式,也可以得到类似的表达式,如前所述。
\subsubsection{量子力学}
在量子力学中,每一个保利矩阵都对应着一个角动量算符,用于描述自旋$1/2$粒子在三维空间中各个方向的自旋可观测量。由前文提到的Cartan 分解直接得出,矩阵 $i\sigma_j$ 是旋转群$SO(3)$在自旋$1/2$的非相对论粒子上投影表示(自旋表示)的生成元。这类粒子的量子态可用二维分量的旋量来表示。同样地,保利矩阵也与同位旋算符有关。

一个有趣的性质是,自旋$1/2$粒子必须旋转$4\pi$(而不是$2\pi$)角度才能回到原始状态。这源于$SU(2)$与$SO(3)$的二对一对应关系:虽然我们常把“自旋向上/向下”想象为 $S^2$ 球面上的南北两极,但在二维复希尔伯特空间中,它们实际上对应的是一对正交向量。

对于自旋$1/2$粒子,其自旋算符为:$\mathbf{J} = \frac{\hbar}{2} \, \vec{\sigma}$,这是$SU(2)$的基本表示。通过不断对这个表示进行Kronecker积,可以构造出所有更高阶的不可约表示。换句话说,三维空间中高自旋体系(任意大的 $j$ 值)的自旋算符,都可以利用这一基础自旋算符结合阶梯算符推导出来。相关内容可以在旋转群$SO(3)$§ 关于李代数的说明中找到。
对于群元素用自旋矩阵表达的形式(保利矩阵的欧拉公式推广版本),虽然可以求解,但形式上更为复杂。\(^\text{[7]}\)

此外,在多粒子量子力学中,广义保利群$G_n$也非常有用。它由所有$n$重张量积的保利矩阵组成,用于描述多粒子系统的对称性与运算结构。
\subsubsection{相对论性量子力学}
在**相对论性量子力学**中,四维空间的旋量(spinor)是 **4 × 1**(或 **1 × 4**)矩阵,因此作用在这些旋量上的保利矩阵(或称 Sigma 矩阵)也必须是 **4 × 4** 的形式。它们通过 **2 × 2 的保利矩阵**定义为:
$$
\Sigma_k =
\begin{pmatrix}
\sigma_k & 0 \\
0 & \sigma_k
\end{pmatrix}.~
$$
由此定义可以得出,这些 $\Sigma_k$ 矩阵与 $\sigma_k$ 矩阵具有**相同的代数性质**。

然而,在相对论理论中,**角动量并不是一个三维矢量,而是一个二阶四维张量**。因此,$\Sigma_k$ 需要扩展为 **$\Sigma_{\mu\nu}$**,即旋量上的**洛伦兹变换生成元**。
由于角动量具有**反对称性**,$\Sigma_{\mu\nu}$ 也必须是**反对称矩阵**,所以它们只有 **6 个独立矩阵**。

前三个矩阵定义为:$\Sigma_{k\ell} \equiv \epsilon_{jk\ell} \Sigma_j$其中 $\epsilon_{jk\ell}$ 是 Levi-Civita 符号。后三个矩阵定义为:$-i \, \Sigma_{0k} \equiv \alpha_k$其中Dirac $\alpha_k$ 矩阵为:

  $$
  \alpha_k =
  \begin{pmatrix}
  0 & \sigma_k \\
  \sigma_k & 0
  \end{pmatrix}.~
  $$

此外,相对论性的自旋矩阵 $\Sigma_{\mu\nu}$ 还可以用 **伽玛矩阵的对易子**以紧凑形式表示为:

$$
\Sigma_{\mu\nu} = \frac{i}{2} [\gamma_\mu, \gamma_\nu].~
$$
\subsubsection{量子信息}
在**量子信息**领域,单量子比特量子门是 **2 × 2 的酉矩阵**。保利矩阵是最重要的**单比特操作**之一。
在该领域,上文提到的 **Cartan 分解**通常称为 **单比特门的“Z–Y 分解”**。
如果选择不同的 Cartan 对,也可以得到类似的 **“X–Y 分解”**。
