% 几何与解析几何初步(高中)
% keys 解析几何|初中|几何|初步|坐标系|点斜式
% license Usr
% type Tutor

\begin{issues}
\issueDraft
\end{issues}

\subsection{几何基础}
点
线
角
jiao

\subsection{三角}
全等
相似
\subsection{三角函数}
正弦、余弦、正切的定义
\begin{table}[ht]
\centering
\caption{常见三角函数值}\label{tab_HsGeBa1}
\begin{tabular}{|c|c|c|c|}
\hline
角$\alpha$ & $30^{\circ}$ & $45^{\circ}$ & $60^{\circ}$ \\
\hline
$\sin\alpha$ & $\displaystyle\frac{1}{2}$ & $\displaystyle\frac{\sqrt{2}}{2}$ & $\displaystyle\frac{\sqrt{3}}{2}$ \\
\hline
$\cos\alpha$ & $\displaystyle\frac{\sqrt{3}}{2}$& $\displaystyle\frac{\sqrt{2}}{2}$ &  $\displaystyle\frac{1}{2}$ \\
\hline
$\tan\alpha$ & $\displaystyle\frac{1}{\sqrt{3}}$ & $1$ & $\sqrt{3}$ \\
\hline
\end{tabular}
\end{table}

\subsection{解析几何基础}

\subsubsection{坐标系}
将所有的实数和直线上的点一一对应,就形成了\textbf{数轴(number line)}。。数轴的定义基于一个确定的原点、单位长度和正方向,这三个因素唯一地确定了数轴在几何中的位置和方向。法国数学家勒内·笛卡尔(René Descartes)在数学研究中,将两条数轴的原点重叠,并将其正交(即相互垂直)放置,创造了\textbf{坐标系(coordinate system)}。这就是初中阶段学习过的\textbf{笛卡尔坐标系(Cartesian coordinate system)},也称为\textbf{直角坐标系(rectangular coordinate system)}。

引入坐标系后,平面上的任何一点都可以通过一个\enref{有序数对}{CartPr} $(x, y)$ 来表示。借助这种表示法,几何形状可以通过数对来分析和研究,这一方式称为\enref{解析几何}{JXJH}。而当数对中的值对应于函数的变量及其结果时,几何图形就成为了函数的图像。因此,坐标系不仅为函数的图像提供了清晰的视觉表达,还使得人们可以通过几何图形直观地观察函数的性质,例如其变化趋势、最大值和最小值等。

通常,直角坐标系中,两条数轴称为$x$轴和$y$轴,且向右的方向为$x$轴的正方向,向上为$y$轴的正方向。数轴将平面分为四个区域,称为\textbf{象限(quadrant)}。其中,第一象限是两个坐标都为正的区域,之后按逆时针方向依次为第二、第三和第四象限。

\subsubsection{常见表达式}

\begin{definition}{直线的点斜式}\label{def_HsGeBa_1}
经过点$(x_0,y_0)$,且斜率为$k$点直线表达式为:
\begin{equation}
y-y_0=k(x-x_0)~.
\end{equation}
\end{definition}

