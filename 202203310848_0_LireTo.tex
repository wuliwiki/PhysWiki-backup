% 木块堆叠问题(里拉斜塔)

\begin{issues}
\issueDraft
\end{issues}

\pentry{重心\upref{CenG}}

\footnote{参考 Wikipedia \href{https://en.wikipedia.org/wiki/Block-stacking_problem}{相关页面}.}木块推叠问题也称为里拉斜塔, 如\autoref{LireTo_fig1} 所示. 如何对方可以使最顶端的木块伸出最多.
\addTODO{如何证明这样的方式就是伸出最多的呢?}

可以证明 $N$ 个木块伸出的总长度为
\begin{equation}
L = \frac{1}{2} + \frac{1}{3} + \dots + \frac{1}{N}
\end{equation}
当 $N\to\infty$ 时, 这个级数称为调和级数(链接未完成), 可以证明它不收敛, 也就是 $L$ 会趋近于无穷.

\begin{figure}[ht]
\centering
\includegraphics[width=10cm]{./figures/LireTo_1.png}
\caption{里拉斜塔(来自 Wikipedia)} \label{LireTo_fig1}
\end{figure}

可以用递归法证明. 一个关键的思路在于, 上方 $n$ 个木块看成一个整体, 然后在下方添加一个木块后, $L/2$ 长度的两端质量比例为 $n:1$, 所以质心位置应该在 $1/(n+1)$ 处.

这里用到一个定理: 两个质点系的质心位置就是两个质点系各自的质心的质心, 见\autoref{CM_sub1}~\upref{CM}.


