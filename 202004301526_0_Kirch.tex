% 基尔霍夫电路定律

\pentry{电路\upref{Circ}}

\begin{theorem}{基尔霍夫电流定律}
\textbf{基尔霍夫电流定律}又称为\textbf{基尔霍夫第一定律},规定在电路中所有进入某节点的电流的总和等于所有离开这节点的电流的总和. 或者说,假设进入某节点的电流为正值,离开这节点的电流为负值,则所有涉及这节点的电流的代数和等于零.以方程式表达,对于电路的任意节点,有
\begin{equation}
\sum_{k=1}^n i_k =0
\end{equation}
其中,$i_k$是第$k$个进入或离开这节点的电流,是流过与这节点相连接的第$k$个支路的电流,可以是实数或复数.
\end{theorem}

\subsubsection{证明}
考虑电路的某节点,跟这节点相连接有$n$个支路.假设进入这节点的电流为正值,离开这节点的电流为负值,则这节点的总电流$i$等于流过支路$k$的电流$i_k$的代数和:
\begin{equation}
i=\sum_{k=1}^n i_k
\end{equation}
将这方程式对某段时间 $[t_1, t_2]$ 内积分,可以得到这段时间该节点电荷的增加
\begin{equation}
q=\sum_{k=1}^n q_k
\end{equation}
其中 $q = \int_{t_1}^{t_2} i(t) \dd{t}$, $q_k=\int_{t_1}^{t_2} i_k(t) \dd{t}$是流过支路$k$的电荷.

若 $q>0$, 则说明有正电荷会累积于该节点, $q < 0$ 表示负电荷会累积于节点. 在讨论电路时, 我们一般假设任意一点不存在净电荷\upref{Circ}, 所以 $q$ 和 $i$ 都恒为零.

\subsection{基尔霍夫电压定律}
\subsubsection{定理内容}
\begin{theorem}{基尔霍夫电压定律}
\textbf{基尔霍夫电压定律}又称为\textbf{基尔霍夫第二定律},表明沿着闭合回路所有元件两端的电势差(电压)的代数和等于零.或者,换句话说,沿着闭合回路的所有电动势的代数和等于所有电压降的代数和.以方程式表达,对于电路的任意闭合回路,
\begin{equation}
\sum_{k=1}^m V_k = 0
\end{equation}
其中,$m$是此闭合回路的元件数目,$V_k$是元件两端的电压,可以是实数或复数.
\end{theorem}

基尔霍夫电压定律不仅应用于闭合回路,也可以把它推广应用于回路的部分电路.

\subsubsection{证明}
根据电势差的定义(\autoref{Voltag_eq1}\upref{Voltag})
\begin{equation}
U_{21} = V(\bvec r_2) - V(\bvec r_1) = - \int_{\bvec r_a}^{\bvec r_b} \bvec E_0(\bvec r) \vdot \dd{\bvec r}
\end{equation}
如果路径起点为 $\bvec r_1$, 终点为 $\bvec r_N$, 中途有若干点 $\bvec r_2, \dots, \bvec r_{N-1}$. 那么有


 个不同的点 $$终点和起点相接, 即 $\bvec r_2 = \bvec r_1$. 有
\begin{equation}
\int_{\bvec r_1}^{\bvec r_2} \bvec E_0(\bvec r) \vdot \dd{\bvec r} = 0
\end{equation}


我们回顾一下电势的定义:
\begin{equation}
\phi(\mathbf{r})\stackrel{def}{=} - \int_{L} \mathbf{E} \cdot \mathrm{d} \mathbf{l}
\end{equation}

容易发现基尔霍夫电压定律可以等价表达为:
\begin{equation}
\oint _{{C} }\mathbf {E} \cdot d\mathbf {l} =0
\end{equation}
其中,${C}$是积分的闭合回路.

这正是环路定理. 而我们也知道,此方程式是法拉第电磁感应定律对于恒定磁场的的简化版本.假设通过闭合回路${C}$的磁通量为常数,则此方程式成立.此方程式指明,电场沿着闭合回路${C}$的线积分为零.将这线积分切割为几段支路,就可以分别计算每一段支路的电压.

由于要求闭合回路${C}$的磁通量为常数,基尔霍夫第二定律也存在着理论限制.由于含时电流会产生含时磁场,通过闭合回路${C}$的磁通量是时间的函数,根据法拉第电磁感应定律,会有电动势$\mathcal{E}$出现于闭合回路${C}$.所以,电场沿着闭合回路${C}$的线积分不等于零.这是因为电流会将能量传递给磁场;反之亦然,磁场亦会将能量传递给电流.

对于含有电感器的电路,必须将基尔霍夫电压定律加以修正.由于含时电流的作用,电路的每一个电感器都会产生对应的电动势$\mathcal{E}_k$,需要将这电动势纳入基尔霍夫电压定律,才能求得正确答案.
