% 虚时间法求基态波函数
% 数值解|薛定谔方程|束缚态|定态|虚时间

\begin{issues}
\issueDraft
\end{issues}

这里介绍一种求解基态波函数的数值方法. 如果薛定谔方程中势能不含时间, 用分离变量法解薛定谔方程的结果是
\begin{equation}
\Psi(\bvec r, t) = \sum_i \psi_i(\bvec r) \E^{-\I E_i t}
\end{equation}
其中 $\psi_i(\bvec r)$ 是能量为 $E_i$ 的能量本征态.

现在若要求基态, 我们可以用虚数时间, 即 $t' = \I t$, 使得含时波函数变为
\begin{equation}\label{ImgT_eq2}
\Psi(\bvec r, t') = \sum_i \psi_i(\bvec r) \E^{- E_i t'}
\end{equation}
这样, 激发态衰减的速度就都比基态要快, 当 $t \to +\infty$ 的时候, 就只剩下基态波函数了. 最后进行归一化即可.

假设我们有一个求解 TDSE 的数值方法, 那么我们只需要用其求解方程
\begin{equation}
\mat H\Psi = -\pdv{\Psi}{t}
\end{equation}
然后每个循环对波函数进行归一化即可. 因为该方程的分离变量解就是\autoref{ImgT_eq2}.
