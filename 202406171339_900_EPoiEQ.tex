% 静电势的泊松方程
% keys 静电学|静电势|泊松方程|电介质|镜像法|静电势的唯一性定理
% license Xiao
% type Tutor

\pentry{电场的高斯定律\nref{nod_EGauss},拉普拉斯方程\nref{nod_LapEq}}{nod_8aed}

\subsection{静电势的泊松方程}
在\textbf{静电学}或\textbf{静磁学}问题中,磁场是不随时间变化的,此时空间中的电势可以由\enref{库仑定律}{ClbFrc}轻易地表达为
\begin{equation}\label{eq_EPoiEQ_3}
\phi(\bvec x) = \int \dd[3]{\bvec x'} \frac{\rho(x')}{4\pi \epsilon_0|\bvec x-\bvec x'|}~.
\end{equation}
这里我们假设 $\rho(\bvec x')$ 为空间中的电荷分布(对于点电荷模型,每一个电荷都对应一处无穷小区域无穷大的电荷密度,即 $\delta$-函数),并且我们\textbf{常常假定} $\rho(\bvec x')$ 是一个连续函数,因此在三维积分的过程中 $\bvec x'\rightarrow \bvec x$ 时的积分是良定义的,不会出现无穷大的发散\footnote{注意到,空间中 $\bvec x$ 处一个电荷所激发的电场不会对当前位置的电势有贡献。而在此处我们假定了 $\rho$ 是连续函数,因此 $\bvec x'\rightarrow \bvec x$ 时 $\rho/|\bvec x-\bvec x'|$ 对积分的贡献是趋于 $0$ 的,因此与我们前面的论述不矛盾。}。那么接下来考虑 $\phi(\bvec x)$ 所满足的性质。我们有以下重要关系式:
\begin{equation}
\nabla^2 \frac{1}{|\bvec x-\bvec x'|} = \qty(\pdv[2]{x}+\pdv[2]{y}+\pdv[2]{z}) \frac{1}{|\bvec x-\bvec x'|}= -4\pi\delta^3(\bvec x-\bvec x')~.
\end{equation}
这一关系式的物理意义就是,空间中一个 $\bvec x'$ 处 $+q$ 的点电荷会在 $\bvec x$ 处造成 $1/(4\pi\epsilon |\bvec x-\bvec x'|)$ 的电势,$\nabla^2$ 被称为拉普拉斯算子。因此当我们将拉普拉斯算子作用于\autoref{eq_EPoiEQ_3},将会得到
\begin{equation}
\begin{aligned}
\nabla^2\phi(\bvec x) &= \int \dd[3]{\bvec x'}\frac{\rho(x')}{4\pi \epsilon_0} \nabla^2 \frac{1}{|\bvec x-\bvec x'|} \\
&= \int \dd[3]{\bvec x'}\frac{\rho(x')}{4\pi \epsilon_0} ( -4\pi\delta^3(\bvec x-\bvec x'))\\
&= -\frac{\rho(\bvec x)}{\epsilon_0}~.
\end{aligned}
\end{equation}

\subsection{从麦克斯韦方程组到泊松方程}
\pentry{麦克斯韦方程组(介质)\nref{nod_MWEq1}}{nod_3e8b}

\subsubsection{真空中泊松方程}
利用真空中的麦克斯韦方程组,电场满足以下的方程
\begin{equation}
\nabla\cdot \bvec E=\frac{\rho}{\epsilon_0},\  \nabla\times \bvec E=0~.
\end{equation}
由于 $\bvec E$ 无旋,它一定可以表示为一个静电势的梯度 $\bvec E=-\nabla \phi$。代入第一个方程,就可以得到\enref{泊松方程}{PoiEqu}
\begin{equation}
\nabla^2\phi=-\rho/\epsilon_0~.
\end{equation}
如果考虑的区域电荷密度 $\rho$ 恒等于 $0$,那么静电势满足拉普拉斯方程:
\begin{equation}
\nabla^2\phi=0~.
\end{equation}

\subsubsection{均匀介质中的泊松方程}
我们首先考虑均匀、各向同性的线性电介质中的静电问题。设其电容率为 $\epsilon$(即相对介电常数 $\epsilon_r$ 乘以 $\epsilon_0$)。根据介质中的\enref{麦克斯韦方程组}{MWEq1},电场与电极化强度需要满足以下方程:
\begin{align}
&\nabla \cdot \bvec D = \rho,\ \ \nabla \times \bvec E = 0,\\
&\bvec D=\epsilon \bvec E~.
\end{align}
式中 $\rho$ 表示空间的\textbf{自由电荷密度}。由于 $\bvec E$ 无旋,我们引入\enref{静电势}{QEng} $\phi$:
\begin{align}
E=-\nabla \phi~.
\end{align}
由此可以得到泊松方程:
\begin{align}
\nabla^2 \phi(\bvec x)=-\frac{\rho(\bvec x)}{\epsilon} \label{eq_EPoiEQ_1}~.
\end{align}
如果所考虑的区域自由电荷密度 $\rho(\bvec x)\equiv 0$,那么静电势满足拉普拉斯方程:
\begin{align}
\nabla^2 \phi(\bvec x)=0~.
\end{align}

