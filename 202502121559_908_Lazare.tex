% 拉扎尔·卡诺(综述)
% license CCBYSA3
% type Wiki

本文根据 CC-BY-SA 协议转载翻译自维基百科\href{https://en.wikipedia.org/wiki/Lazare_Carnot#References}{相关文章}。

拉扎尔·尼古拉·玛尔格里特·卡诺(法语:[lazaʁ nikɔla maʁɡəʁit kaʁno];1753年5月13日 – 1823年8月2日)是法国数学家、物理学家、军官、政治家,并在法国大革命期间成为公共安全委员会的主要成员之一。他的军事改革,包括实施全民征兵(levée en masse),在将法国革命军转变为一支有效的战斗力量方面发挥了重要作用。

卡诺于1792年当选为国民公会成员,次年成为公共安全委员会的成员,并在第一次反法同盟战争期间,作为战争部长之一,领导了法国的战争努力。他监督了军队的重组,实施了严格的纪律,并通过强制征兵大幅扩展了法国军队。卡诺被誉为1793年至1794年法国军事复兴的功臣,因而被称为“胜利的组织者”。

随着对蒙塔尼亚派激进政治的日益失望,卡诺与马克西米连·罗伯斯比尔决裂,并在1794年热月9日的政变中参与了后者的推翻及随后的处决。卡诺成为了最初的五名执政府成员之一,但在1797年9月18日的热月政变后被罢黜,随后流亡国外。

在拿破仑崛起之后,卡诺返回法国,并于1800年短暂担任战争部长。作为一名坚定的共和主义者,他在拿破仑加冕为皇帝后选择退出了公共生活。1812年,他重返职场,在拿破仑领导下负责安特卫普的防御,抵御第六次反法同盟的进攻;在百日复辟期间,他担任了拿破仑的内政部长。第二次波旁复辟后,卡诺被流放,并于1823年在普鲁士的马格德堡去世。

除了政治生涯,卡诺还是一位杰出的数学家。他于1803年出版的《位置几何学》被认为是投影几何领域的开创性著作。他还因发明了卡诺墙而闻名,这是一种防御工事系统,在19世纪广泛应用于欧洲大陆。
\subsection{教育与早期生活}  
\begin{figure}[ht]
\centering
\includegraphics[width=6cm]{./figures/9b9089a8019e7546.png}
\caption{} \label{fig_Lazare_1}
\end{figure}
卡诺(Lazare Carnot)于1753年5月13日出生在勃艮第的诺莱村,他是当地法官兼皇家公证员克劳德·卡诺(Claude Carnot)和妻子玛尔格丽特·波蒂耶(Marguerite Pothier)的儿子,家中有七个孩子,他是次子。14岁时,卡诺和他的哥哥一起进入奥吞学院(Collège d'Autun),专注于哲学和古典学的学习。他深信斯多葛哲学,并且深受罗马文明的影响。15岁时,他离开了奥吞的学校,前往圣苏尔皮斯神父会进一步强化他的哲学知识,并在比松神父(Abbé Bison)指导下学习逻辑、数学和神学。

卡诺的学术成就给奥蒙公爵(诺莱侯爵)留下了深刻印象,公爵建议他选择军事生涯。卡诺不久后被父亲送到奥蒙公爵的府邸,进一步接受教育。在这里,他于1770年被送入朗普雷先生(M. de Longpré)的私立学校,直到他准备好进入巴黎的两所著名的工程和炮兵学校。1771年2月,他在超过一百名考生中脱颖而出,排名第三,成功入选。这时,他进入了梅济耶尔皇家工程学院(École royale du génie de Mézières),并被任命为第二中尉。在梅济耶尔学院的学习内容包括几何学、力学、几何设计、地理学、水力学和材料准备等。1773年1月1日,他以第一中尉的身份毕业,年仅18岁。

卡诺获得了路易·约瑟夫·孔代亲王工程军团的中尉军衔。在此期间,他在理论工程学和防御工事方面都声名鹊起。在军队服役期间,驻扎在加来、瑟堡和贝图讷,他继续研究数学。1783年12月,他晋升为上尉。

1784年,他出版了他的第一部著作《机器学论文》,其中包含了预示着能量原理的表述,涉及到下落物体的能量,以及早期证明了动能在非完全弹性碰撞中的损失。这篇论文使他赢得了第戎科学、艺术与美学学会的荣誉。另一个转折点是他写的关于伏邦的论文,在这篇论文中,他赞扬了伏邦的工程成就,同时发展了自己作为作家/工程师的事业。伏邦的工作对他作为将军和工程师的职业生涯产生了深远影响。1786年,他在本地文学俱乐部结识了来自阿拉斯的律师罗伯斯比尔。1788年,他因未能履行与第戎一位女子结婚的承诺,被关押在贝图讷,并被用“密令信”监禁。释放后,他被调往阿尔尔-苏尔-拉利斯,并于1791年5月与圣奥梅尔的索菲·迪庞结婚。他在两个月内担任了当地文学社的社长。
\subsection{政治生涯}  
1791年9月,卡诺成为帕·德·卡莱省的立法会议代表。在立法会议期间,卡诺当选为公共教育委员会成员。他认为所有公民都应接受教育。作为该委员会的成员,他撰写了一系列关于教学和教育系统的改革方案,但由于革命时期暴力的社会和经济气候,这些改革未能得以实施。

立法会议解散后,卡诺于1792年9月当选为国民公会成员。他在1792年剩余的几个月里,前往巴约讷执行任务,组织军事防御,试图抵御西班牙可能的进攻。返回巴黎后,卡诺投票支持路易十六的死刑,尽管他未参与有关路易十六审判的辩论。  
到1793年2月中旬,卡诺提议,不论被并吞的人民是否愿意,都应以法国利益为名进行并吞。1793年5月27日,路易十六否决了立法会议关于镇压拒绝宣誓的神职人员的努力,卡诺和塞尔万在6月8日提议在议会中建立一支常备志愿民兵,6月18日布里索坦部长被重新任命后,君主制面临了6月20日的失败示威。

1793年8月14日,卡诺被选为公共安全委员会成员,并作为战争部长之一负责军事事务。他与约翰·瓦尔克纳(Johan Valckenaer)关系密切,后者试图加速对荷兰共和国的入侵。

1795年,随着目录政府的建立,卡诺成为最初的五位执政官之一。在第一年,五位执政官工作协调,彼此之间以及与议会合作得很好。然而,政治观点的分歧导致了卡诺与埃蒂安-弗朗索瓦·勒图尔努尔(Étienne-François Letourneur)及弗朗索瓦·德·巴尔泰尔米(François de Barthélemy)之间的裂痕,与此同时,保罗·弗朗索瓦·让·尼古拉·巴拉(Paul François Jean Nicolas, vicomte de Barras)、让-弗朗索瓦·雷贝尔(Jean-François Rewbell)和路易·玛丽·德·拉·雷维利埃尔-勒佩奥(Louis Marie de La Révellière-Lépeaux)三人组成的三人执政团则是另一方。卡诺和巴尔泰尔米支持通过妥协结束战争,并希望推翻三人执政团,用更保守的人取而代之。随着勒图尔努尔被卡诺的另一位亲密合作者弗朗索瓦·德·巴尔泰尔米替换,他们两人以及许多“国民议会五百人会议”的代表在1797年9月4日的“佛吕克蒂多18日政变”中被罢免,这场政变由拿破仑·波拿巴(最初是卡诺的门生)和皮埃尔·弗朗索瓦·查尔斯·奥热罗(Pierre François Charles Augereau)将军策划。卡诺随后逃亡至日内瓦,并于1797年在那里发表了他的《无穷小计算的形而上学》(La métaphysique du calcul infinitésimal)。
\subsubsection{军事成就}
\begin{figure}[ht]
\centering
\includegraphics[width=6cm]{./figures/20e65c18d1cc5c0c.png}
\caption{拉扎尔·卡诺(Lazare Carnot),在恐怖统治时期是公共安全委员会的一个极为高产的成员。他在发动全民动员(levée en masse)方面的作用,可能拯救了法国革命军免于在敌人数量上占优的情况下被击败。} \label{fig_Lazare_2}
\end{figure}
法国革命军的创建在很大程度上归功于他卓越的组织能力和执行纪律的能力。为了为战争征募更多的士兵,卡诺引入了征兵制度:由国民公会批准的“全民动员”(levée en masse)使法国的军队从1793年中期的64.5万人增加到1794年9月的150万人。他是第一个通过现代化的大规模军队与战略规划来进行战争的人,这一理念通过革命得以实现。作为一名军事工程师,卡诺偏爱堡垒和防御策略。他设计了创新的防御工事,其中包括以他名字命名的“卡诺墙”。然而,面对持续的入侵,他决定将自己的战略规划转向进攻。正是他的智慧催生了1793年至1794年战争中的战略机动和组织安排,扭转了战局。

基本思路是将一支庞大的军队分成多个小单位,这些单位能够比敌人更快地移动,并从侧翼进攻,而不是正面交锋,这在卡诺被选为公共安全委员会成员之前曾导致重大的失败。这一战术在与传统的欧洲军队作战时极为成功。卡诺的倡议是训练新兵掌握战争艺术,并将新兵与经验丰富的士兵配对,而不是建立一支没有实际战斗经验的大规模志愿军。

一旦兵员问题得到解决,卡诺便将他的行政才能转向了这支庞大军队所需的物资供应。许多军需物资严重短缺:由于缺乏铜料制造火炮,他命令没收教堂的钟铸成铜炮;火药中的硝石也不足,他便求助于化学技术;靴子所需的皮革短缺,于是他要求并确保了新的制革方法。他迅速组织起军队,并帮助扭转了战争的局势。这一切极大地激化了法国革命在仍然效忠波旁王朝的地区(如温代地区)产生的不满情绪,温代地区五个月前爆发了公开起义,但当时政府认为这是一次成功,卡诺也因此被誉为“胜利的组织者”。1793年秋,他接管了法国在北方战线的军队,并为让-巴普蒂斯特·朱尔丹在瓦蒂尼战役中的胜利做出了贡献。

\textbf{卡诺与马克西米连·罗伯斯比尔及雅各宾俱乐部的关系 } 

卡诺第一次遇见罗伯斯比尔是在阿拉斯,当时他被派往那里执行军事任务,而罗伯斯比尔则刚刚完成了法律学业。他们俩都是文学俱乐部的成员,并一同演唱了《玫瑰社》(Société des Rosati)。这个俱乐部成立于1778年,受到了查佩尔、拉封丹和肖吕的作品启发。在这里,他们成为了熟人,最终也成了朋友。罗伯斯比尔比卡诺早一步进入了阿拉斯学会,进入时间是1784年4月,而卡诺则是在1786年进入的。

尽管在1794年两人都是公共安全委员会的积极成员,但卡诺与罗伯斯比尔之间的紧张关系开始大幅升温。在公共安全委员会工作期间,委员会的立场极为激进,卡诺签署了43项法令并起草了其中的18项。这些法令大多数涉及军事战术和教育问题。尽管卡诺倾向于雅各宾派的信仰,但他被视为自己一方的“保守派”。他并不是该激进派的正式成员,因此在许多问题上持有自己的独立观点。其中一个重要问题就是罗伯斯比尔提出的平等主义社会制度,卡诺对此持有强烈反对意见。