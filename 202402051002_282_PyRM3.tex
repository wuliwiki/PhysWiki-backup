% Python RoboMaster EP 教程—初始化机器人
% keys Robomaster|机器人
% license Xiao
% type Tutor

\pentry{Python 基础\nref{nod_PyFi}}{nod_9172}

在进行与机器人相关的操作之前,需要先初始化机器人对象

首先从 robomaster 包中导入 robot 模块:

\begin{lstlisting}[language=python]
from robomaster import robot
\end{lstlisting}

当SDK运行在多网卡同时使用的设备时(自动获取的ip可能不是与机器人进行连接的ip)需要手动指定 RoboMaster SDK 的本地ip地址

指定ip使用以下语句:

\begin{lstlisting}[language=python]
robomaster.config.LOCAL_IP_STR = "ip地址"
\end{lstlisting}

创建 Robot 类的实例对象 ep_robot, ep_robot 即一个机器人的对象:

\begin{lstlisting}[language=python]
ep_robot = robot.Robot()
\end{lstlisting}

初始化机器人,如果调用初始化方法时不传入任何参数,则使用config.py中配置的默认连接方式(WIFI直连模式) 以及默认的通讯方式(udp通讯)对机器人进行初始化:

\begin{lstlisting}[language=python]
ep_robot.initialize(conn_type="连接方式", proto_type='通讯方式')
\end{lstlisting}

可以通过以下语句设置默认的连接方式与通讯方式:

\begin{lstlisting}[language=python]
config.DEFAULT_CONN_TYPE = "连接方式"
config.DEFAULT_PROTO_TYPE = "通讯方式"
\end{lstlisting}
