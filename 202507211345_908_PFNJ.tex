% 帕夫努季·切比雪夫(综述)
% license CCBYSA3
% type Wiki

本文根据 CC-BY-SA 协议转载翻译自维基百科\href{https://en.wikipedia.org/wiki/Pafnuty_Chebyshev}{相关文章}。

\begin{figure}[ht]
\centering
\includegraphics[width=6cm]{./figures/97a93cba399441cc.png}
\caption{} \label{fig_PFNJ_1}
\end{figure}
帕夫努季·利沃维奇·切比雪夫(俄语:Пафну́тий Льво́вич Чебышёв,发音:[pɐfˈnutʲɪj ˈlʲvovʲɪtɕ tɕɪbɨˈʂof],1821年5月16日[俄历5月4日]—1894年12月8日[俄历11月26日])\(^\text{3}\)是一位俄罗斯数学家,被认为是俄罗斯数学的奠基人。

切比雪夫以其在概率论、统计学、力学以及数论领域的基础性贡献而著称。许多重要的数学概念以他的名字命名,包括切比雪夫不等式(可用于证明大数定律的弱形式)、伯特兰-切比雪夫定理、切比雪夫多项式、切比雪夫连杆机构以及切比雪夫偏差。
\subsection{音译}
切比雪夫这个姓氏在翻译时出现了多种不同的拼写方式,如:Tchebichef、Tchebychev、Tchebycheff、Tschebyschev、Tschebyschef、Tschebyscheff、Čebyčev、Čebyšev、Chebysheff、Chebychov、Chebyshov(据以俄语为母语的人说,这种拼写在英语中最接近旧俄语中的正确发音),以及 Chebychev,这是一种混合了英语和法语音译的方式,通常被认为是错误的。

在数学文献中,这种拼写混乱被认为是最著名的数据检索噩梦之一。目前,“Chebyshev” 这一英语拼写已被广泛接受,唯独法语国家仍偏好使用 “Tchebychev”。根据 ISO 9 国际音译标准,该姓氏的标准转写形式是 “Čebyšëv”。美国数学会在其《数学评论》中采用了 “Chebyshev” 这一拼写方式\(^\text{4}\)。

他的名字 Pafnuty 源自希腊语 Paphnutius(Παφνούτιος),而后者又来自科普特语 Paphnuty(Ⲡⲁⲫⲛⲟⲩϯ),意为“属于上帝的人”或简而言之“上帝之人”。
