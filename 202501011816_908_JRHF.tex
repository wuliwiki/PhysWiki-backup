% 古斯塔夫·基尔霍夫(综述)
% license CCBYSA3
% type Wiki

本文根据 CC-BY-SA 协议转载翻译自维基百科\href{https://en.wikipedia.org/wiki/Gustav_Kirchhoff}{相关文章}。

\begin{figure}[ht]
\centering
\includegraphics[width=6cm]{./figures/ba04c3fbf9d952c3.png}
\caption{} \label{fig_JRHF_1}
\end{figure}
古斯塔夫·罗伯特·基尔霍夫(德语:[ˈgʊs.taf ˈkɪʁçhɔf];1824年3月12日–1887年10月17日)是德国的物理学家、数学家和化学家,他在电路学、光谱学以及加热物体的黑体辐射发射等基本理解方面做出了重要贡献。[1][2] 他还在1860年提出了“黑体”这一术语。[3]

有几个不同的概念集被称为“基尔霍夫定律”,包括基尔霍夫电路定律、基尔霍夫热辐射定律和基尔霍夫热化学定律。

“本生–基尔霍夫光谱学奖”是以基尔霍夫和他的同事罗伯特·本生的名字命名的。
\subsection{生平与工作}  
古斯塔夫·基尔霍夫于1824年3月12日出生在普鲁士的哥尼斯堡,父亲是律师弗里德里希·基尔霍夫,母亲是约翰娜·亨丽埃特·维特克。他的家庭属于普鲁士福音教会的路德宗信徒。他于1847年毕业于哥尼斯堡的阿尔贝图斯大学,在那里参加了由卡尔·古斯塔夫·雅各布·雅可比、弗朗茨·恩斯特·诺伊曼和弗里德里希·朱利乌斯·里舍洛特主持的数学物理研讨会。同年,他搬到了柏林,直到他获得布雷斯劳的教授职位。后来,在1857年,他与数学教授里舍洛特的女儿克拉拉·里舍洛特结婚,夫妻俩育有五个孩子。克拉拉于1869年去世。基尔霍夫于1872年再婚,妻子是路易莎·布勒梅尔。

基尔霍夫在1845年还是学生时,提出了如今在电气工程中广泛使用的基尔霍夫电路定律。这项研究起初是作为研讨课题完成的,后来成为他的博士论文。他于1854年被任命到海德堡大学,在那里与罗伯特·本生合作进行光谱学研究。1857年,他计算出电信号在无电阻的导线中以光速传播。[7][8] 1859年,他提出了热辐射定律,并于1861年提供了证明。他与本生共同发明了分光镜,基尔霍夫利用该仪器率先通过光谱辨别太阳中的元素,并于1859年显示太阳中含有钠。他与本生于1861年发现了铯和铷。[9] 在海德堡大学,他与数学家利奥·科尼斯贝格(Leo Koenigsberger)共同主持了一个以弗朗茨·恩斯特·诺伊曼的模式为蓝本的数学物理研讨会。参加该研讨会的有亚瑟·舒斯特(Arthur Schuster)和索菲娅·科瓦列夫斯卡娅(Sofia Kovalevskaya)。
\begin{figure}[ht]
\centering
\includegraphics[width=6cm]{./figures/d477792c3f3fd7f0.png}
\caption{基尔霍夫(左)与罗伯特·本生,约1850年} \label{fig_JRHF_2}
\end{figure}
基尔霍夫对光谱学领域贡献巨大,他形式化了描述炽热物体发射光的光谱成分的三条定律,大大扩展了大卫·阿尔特(David Alter)和安德斯·乔纳斯·安格斯特罗姆(Anders Jonas Ångström)的发现。1862年,他因研究太阳光谱的固定线以及人工光谱中明亮线条的反转,获得了伦福德奖章。[a] 1875年,基尔霍夫接受了柏林首个专门设立的理论物理学教授席位。

他还对光学作出了贡献,通过仔细求解波动方程,为惠更斯原理提供了坚实的基础,同时对其进行了修正。[11][12]

1864年,他当选为美国哲学会会员。[13] 

1884年,他成为荷兰皇家艺术与科学学院的外籍院士。[14]

基尔霍夫于1887年去世,葬于柏林舍嫩贝格的圣马修教堂公墓(距离格林兄弟的墓地仅几米远)。著名数学家利奥波德·克罗内克(Leopold Kronecker)也葬在同一墓地。
\subsubsection{基尔霍夫电路定律}  
基尔霍夫第一定律指出,在一个导体网络中汇集于某点(或节点)的电流代数和为零。第二定律则表明,在闭合电路中,系统中电压的定向代数和为零。
\subsubsection{基尔霍夫的三条光谱定律}
\begin{figure}[ht]
\centering
\includegraphics[width=6cm]{./figures/38cbec09c4ebc666.png}
\caption{基尔霍夫光谱学定律的可视化描述} \label{fig_JRHF_3}
\end{figure}
\begin{enumerate}
\item 固体、液体或高密度气体在受到激发发光时,会在所有波长范围内辐射,从而产生连续光谱。  
\item 低密度气体在受到激发发光时,只会在特定波长上辐射,从而产生发射光谱。  
\item 如果组成连续光谱的光通过冷却的低密度气体,结果将会是吸收光谱。
\end{enumerate}
基尔霍夫并不知道原子中能级的存在。自1814年弗劳恩霍夫发现离散光谱线以来,这些光谱线的存在已经为人所知。1885年,约翰·巴尔末描述了这些光谱线形成了离散的数学模式。约瑟夫·拉莫尔通过电子的振荡解释了在磁场中光谱线的分裂现象,即所谓的塞曼效应。[15][16] 然而,这些离散的光谱线直到1913年波尔的原子模型提出才被解释为电子跃迁的结果,这也为量子力学的发展铺平了道路。
\subsubsection{基尔霍夫的热辐射定律}
基尔霍夫在他的热辐射定律中提出了一种未知的辐射通用定律,这一理论最终引导了马克斯·普朗克发现了作用量子,从而开创了量子力学的道路。
\subsubsection{基尔霍夫的热化学定律} 
另见:反应标准焓 § 温度或压力的变化  
基尔霍夫在1858年表明,在热化学中,化学反应热的变化由产物和反应物的热容差异决定:  
\[
\left(\frac{\partial \Delta H}{\partial T}\right)_p = \Delta C_p~
\]
通过积分该方程,可以通过一个温度下的测量结果计算另一个温度下的反应热。[17][18]  
\subsubsection{基尔霍夫在图论中的定理} 
基尔霍夫还在图论这一数学领域做出了贡献,他证明了基尔霍夫矩阵树定理。
\subsection{作品} 
\begin{itemize}
\item 《论文集》(Gesammelte Abhandlungen,德文)  
  莱比锡:约翰·安布罗修斯·巴特出版社,1882年。  
\item 《电与磁的讲义》(Vorlesungen über Electricität und Magnetismus,德文) 
  莱比锡:贝内迪克图斯·戈特赫尔夫·陶布纳出版社,1891年。  
\item 《数学物理讲义》(Vorlesungen über mathematische Physik,德文)
  共四卷,B.G.陶布纳出版社,莱比锡,1876–1894年:
\end{itemize}  
\begin{itemize}
\item 第1卷:力学(Mechanik)  
    第一版,B.G.陶布纳出版社,莱比锡,1876年。([在线版本](http://example.com))  
\item 第2卷:数学光学(Mathematische Optik) 
    B.G.陶布纳出版社,莱比锡,1891年,由库尔特·亨塞尔编辑。([在线版本](http://example.com))  
\item 第3卷:电与磁(Electricität und Magnetismus) 
    B.G.陶布纳出版社,莱比锡,1891年,由马克斯·普朗克编辑。([在线版本](http://example.com))  
\item 第4卷:热理论(Theorie der Wärme) 
    B.G.陶布纳出版社,莱比锡,1894年,由马克斯·普朗克编辑。[19]
\end{itemize}
\subsection{另见} 
\begin{itemize}
\item 电路秩(Circuit rank)  
\item 计算空气声学(Computational aeroacoustics)  
\item 火焰发射光谱(Flame emission spectroscopy)  
\item 光谱仪(Spectroscope)  
\item 基尔霍夫物理研究所(Kirchhoff Institute of Physics) 
\item 德国发明家与发现者列表(List of German inventors and discoverers)
\end{itemize} 
\subsection{注释}  
a.基尔霍夫的银行家听说他确定了太阳中存在的元素时,讽刺道:“如果无法将金子带回地球,太阳中的金子又有什么用呢?” 基尔霍夫随后将奖金(金英镑)存入该银行,并说:“这是来自太阳的金子。”[10]
\subsection{参考文献}  
\begin{enumerate}
\item Marshall, James L.; Marshall, Virginia R. (2008). "Rediscovery of the Elements: Mineral Waters and Spectroscopy" (PDF). *The Hexagon*: 42–48. Retrieved 31 December 2019.  
\item Waygood, Adrian (19 June 2013). *An Introduction to Electrical Science*. Routledge. ISBN 9781135071134.  
\item Schmitz, Kenneth S. (2018). *Physical Chemistry*. Elsevier. p. 278. ISBN 9780128005996.  
\item Kondepudi, Dilip; Prigogine, Ilya (5 November 2014). *Modern Thermodynamics: From Heat Engines to Dissipative Structures*. John Wiley & Sons. p. 288. ISBN 9781118698709.  
\item Hockey, Thomas (2009). "Kirchhoff, Gustav Robert". *The Biographical Encyclopedia of Astronomers*. Springer Nature. ISBN 978-0-387-31022-0. Retrieved 22 August 2012.  
\item "Gustav Robert Kirchhoff – Dauerausstellung". Kirchhoff-Institute for Physics. Retrieved 18 March 2016. *Am 16. August 1857 heiratete er Clara Richelot, die Tochter des Königsberger Mathematikers ... Frau Clara starb schon 1869. Im Dezember 1872 heiratete Kirchhoff Luise Brömmel.*  
\item Kirchhoff, Gustav (1857). "On the motion of electricity in wires". *Philosophical Magazine*. 13: 393–412.  
\item Graneau, Peter; Assis, André Koch Torres (1994). "Kirchhoff on the motion of electricity in conductors" (PDF). *Apeiron*. 1 (19): 19–25. Archived (PDF) from the original on 8 January 2006.  
\item Weeks, Mary Elvira (1956). *The discovery of the elements* (6th ed.). Easton, PA: Journal of Chemical Education.  
\item Asimov, Isaac. *The Secret of the Universe*, Oxford University Press, 1992, p. 109  
\item Baker, Bevan B.; and Copson, Edward T.; *The Mathematical Theory of Huygens' Principle*, Oxford University Press, 1939, pp. 36–38.  
\item Miller, David A. B.; "Huygens's wave propagation principle corrected", *Optics Letters* 16, 1370–1372, 1991  
\item "APS Member History". search.amphilsoc.org. Retrieved 16 April 2021.  
\item "G. R. Kirchhoff (1824–1887)". Royal Netherlands Academy of Arts and Sciences. Retrieved 22 July 2015.  
\item Buchwald, Jed Z.; and Warwick, Andrew; editors; *Histories of the Electron: The Birth of Microphysics*  
\item Larmor, Joseph (1897), "On a Dynamical Theory of the Electric and Luminiferous Medium, Part 3, Relations with material media", *Philosophical Transactions of the Royal Society*, 190: 205–300, Bibcode:1897RSPTA.190..205L, doi:10.1098/rsta.1897.0020  
\item Laidler, Keith J.; and Meiser, J. H.; *Physical Chemistry*, Benjamin/Cummings 1982, p. 62  
\item Atkins, Peter; and de Paula, J.; *Atkins' Physical Chemistry*, W. H. Freeman, 2006 (8th edition), p. 56  
\item Merritt, Ernest (1895). "Review of *Vorlesungen über mathematische Physik. Vol. IV. Theorie der Wärme* by Gustav Kirchhoff, edited by Max Planck". *Physical Review*. American Physical Society: 73–75.
\end{enumerate}
\subsection{书目}
\begin{itemize}
\item Warburg, E. (1925). "Zur Erinnerung an Gustav Kirchhoff". *Die Naturwissenschaften*. 13 (11): 205. Bibcode:1925NW.....13..205W. doi:10.1007/BF01558883. S2CID 30039558.  
\item Stepanov, B. I. (1977). "Gustav Robert Kirchhoff (on the ninetieth anniversary of his death)". *Journal of Applied Spectroscopy*. 27 (3): 1099. Bibcode:1977JApSp..27.1099S. doi:10.1007/BF00625887. S2CID 95181496.  
\item Everest, A. S. (1969). "Kirchhoff-Gustav Robert 1824–1887". *Physics Education*. 4 (6): 341. Bibcode:1969PhyEd...4..341E. doi:10.1088/0031-9120/4/6/304. S2CID 250765281.  
\item Kirchhoff, Gustav (1860). "Ueber die Fraunhoferschen Linien". *Monatsberichte der Königliche Preussische Akademie der Wissenschaften zu Berlin*: 662–665. ISBN 978-1-113-39933-5. HathiTrust full text. Partial English translation available in Magie, William Francis, *A Source Book in Physics* (1963). Cambridge: Harvard University Press. p. 354-360.  
\item Kirchhoff, Gustav (1860). "IV. Ueber das Verhältniß zwischen dem Emissionsvermögen und dem Absorptionsvermögen der Körper für Wärme und Licht," *Annalen der Physik* 185(2), 275–301. (coinage of term “blackbody”) [On the relationship between the emissivity and the absorptivity of bodies for heat and light].
\end{itemize}
\subsection{进一步阅读}  
\begin{itemize}
\item Gustav Kirchhoff at the Mathematics Genealogy Project  
\item O'Connor, John J.; Robertson, Edmund F., "Gustav Kirchhoff", *MacTutor History of Mathematics Archive*, University of St Andrews  
\item Weisstein, Eric Wolfgang (ed.). "Kirchhoff, Gustav (1824–1887)". *ScienceWorld*  
\item Klaus Hentschel: *Gustav Robert Kirchhoff und seine Zusammenarbeit mit Robert Wilhelm Bunsen*, in: Karl von Meyenn (Hrsg.) *Die Grossen Physiker*, Munich: Beck, vol. 1 (1997), pp. 416–430, 475–477, 532–534.  
\item Klaus Hentschel: *Mapping the Spectrum. Techniques of Visual Representation in Research and Teaching*, Oxford: OUP, 2002.  
\item Kirchhoff's 1857 paper on the speed of electrical signals in a wire 
\end{itemize} 
\begin{itemize}
\item "Kirchhoff, Gustav Robert". *Encyclopedia Americana*. 1920.  
\item "Kirchhoff, Gustav Robert". *The New Student's Reference Work*. 1914.  
\item "Kirchhoff, Gustav Robert". *Encyclopædia Britannica* (11th ed.). 1911.  
\item "Kirchhoff, Gustav Robert". *New International Encyclopedia*. 1905.  
\item "Sketch of Gustav Robert Kirchhoff". *Popular Science Monthly*. Vol. 33. May 1888.  
\item "Kirchhoff, Gustav Robert". *The American Cyclopædia*. 1879. 
\end{itemize} 
\subsection{外部链接}  
\begin{itemize}
\item 与Gustav Kirchhoff相关的名言,参见Wikiquote 
\item 与Gustav Robert Kirchhoff相关的媒体,参见Wikimedia Commons  
\item Open Library
\end{itemize} 
