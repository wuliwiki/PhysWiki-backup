% 非结合代数(综述)
% license CCBYSA3
% type Wiki

本文根据 CC-BY-SA 协议转载翻译自维基百科\href{https://en.wikipedia.org/wiki/Non-associative_algebra}{相关文章}。

非结合代数[1](或称分配代数)是指一种域上的代数,其二元乘法运算不假定具有结合性。也就是说,一个代数结构 $A$ 若是域 $K$ 上的非结合代数,则它是一个 $K$-向量空间,并配备了一个 $K$-双线性的二元乘法运算 $A \times A \to A$,该运算可以是结合的,也可以不是。例子包括李代数、约当代数、八元数,以及带有叉乘运算的三维欧几里得空间。由于不假定乘法是结合的,因此必须使用括号来表示运算顺序。例如,$(ab)(cd)$、$(a(bc))d$ 和 $a(b(cd))$ 的结果可能完全不同。

这里的“非结合”意味着不要求结合律成立,但并不意味着结合律被禁止。换句话说,“非结合”就是“未必结合”的意思,正如“非交换”环中的“非交换”并不是绝对禁止交换律,而是指“未必交换”。

一个代数是幺代数(unital 或 unitary),如果它存在一个单位元 $e$,满足对代数中所有元素 $x$ 都有 $ex = x = xe$。例如,八元数是幺代数,但李代数从来都不是。

对 $A$ 的非结合代数结构,可以通过将其关联到其他结合代数来研究,这些结合代数是$A$ 作为 $K$-向量空间时其全体 $K$-自同态代数的子代数。其中有两个重要的例子:导子代数和包络代数(后者在某种意义上是“包含 $A$ 的最小结合代数”)。

更一般地,有些作者把非结合代数的概念扩展到交换环 $R$ 上:即一个带有 $R$-双线性二元乘法运算的 $R$-模[2]。如果一个结构满足除了结合律以外的所有环公理(例如任何 $R$-代数),那么它自然就是一个 $\mathbb{Z}$-代数,因此有些作者称非结合的 $\mathbb{Z}$-代数为非结合环。
\subsection{满足恒等式的代数}
具有两个二元运算、但没有其他限制的类环结构是一类非常广泛的对象,过于笼统而难以研究。出于这个原因,最为人熟知的非结合代数类型往往满足某些恒等式或性质,从而在一定程度上简化了乘法。这些性质包括以下几类。
\subsubsection{常见性质}
设 $x, y, z$ 为域 $K$ 上代数 $A$ 的任意元素。正整数次幂的递归定义为:$x^1 \coloneqq x$,对于 $n \geq 1$,有两种习惯定义:右幂:$x^{n+1} \coloneqq (x^n)x$[3],左幂:$x^{n+1} \coloneqq x(x^n)$[4][5],具体采用哪种定义取决于作者。
\begin{itemize}
\item 幺元(Unital):存在一个元素 $e$,使得 $ex = x = xe$。在这种情况下可定义 $x^0 \coloneqq e$。
\item 结合性:$(xy)z = x(yz)$。
\item 交换性:$xy = yx$。
\item 反交换性[6]:$xy = -yx$。
\item Jacobi 恒等式[6][7]:$(xy)z + (yz)x + (zx)y = 0$,或 $x(yz) + y(zx) + z(xy) = 0$,具体形式取决于作者。
\item Jordan 恒等式[8][9]:$(x^2y)x = x^2(yx)$,或 $(xy)x^2 = x(yx^2)$,具体形式取决于作者。
\item 可交替性[10][11][12]:左交替律:$(xx)y = x(xy)$,右交替律:$(yx)x = y(xx)$。
\item 柔性律[13][14]:$(xy)x = x(yx)$。
\item $n$ 次幂结合律(nth power associative,$n \geq 2$):对所有满足 $0 < k < n$ 的整数 $k$,有$x^{n-k}x^k = x^n$。
\item 三次幂结合律:$x^2x = xx^2$。
\item 四次幂结合律:$x^3x = x^2x^2 = xx^3$(可与下面的“四次幂交换律”比较)。
\item 幂结合律[4][5][15][16][3]:由任意单个元素生成的子代数是结合代数,即对所有 $n \geq 2$ 成立 $n$ 次幂结合律。
\item $n$ 次幂交换律(nth power commutative,$n \geq 2$)**:对所有满足 $0 < k < n$ 的整数 $k$,有$x^{n-k}x^k = x^kx^{n-k}$。
\item 三次幂交换律:$x^2x = xx^2$。
\item 四次幂交换律:$x^3x = xx^3$(可与上面的“四次幂结合律”比较)。
\item 幂交换律:由任意单个元素生成的子代数是交换代数,即对所有 $n \geq 2$ 成立 $n$ 次幂交换律。
\item 指数为 $n \geq 2$ 的幂零性(Nilpotent of index n)**:任意 $n$ 个元素的乘积(无论如何加括号)都为零,但存在某些 $n-1$ 个元素使得其在某种结合方式下乘积不为零:$x_1x_2\cdots x_n = 0, \quad \exists \; y_1,\dots,y_{n-1} \;\; \text{使得 } y_1y_2\cdots y_{n-1} \neq 0$
\item 指数为 $n \geq 2$ 的幂零:幂结合的代数,且 $x^n = 0$,并且存在某个元素 $y$ 使得 $y^{n-1} \neq 0$。
\end{itemize}