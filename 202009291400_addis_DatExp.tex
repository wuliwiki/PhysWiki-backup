% 机器学习数据探索
\begin{issues}
\issueTODO
\end{issues}

\pentry{皮尔逊相关系数\upref{PearsR},最大似然估计\upref{MaxLik},拟合优度检验和列联表独立性检验\upref{CoTaAn}}

首先要确定哪些数据是输入(Input)的变量,哪些数据是要预测的目标值(output),然后需要确定变量的类型,定类变量或者定比、变量等.
探索数据特征常采用以下方法:
\subsection{单变量分析}
单变量分析是数据分析中最简单的形式,其中被分析的数据只包含一个变量.因为它是一个单一的变量,它不处理原因或关系.单变量分析的主要目的是描述数据并找出其中存在的模式.  
\subsubsection{连续型特征分析方法}
连续特征变量可通过平均值,中位数,众数,最小值,最大值,范围,四分位数,IQR,方差,标准差,偏度,等来探索数据特征.此外,显示单变量数据的一些方法包括频率分布表、柱状图、直方图、频率多边形和饼状图.
\subsubsection{类别型特征分析方法}
对于类别特征的变量常通过频率,直方图来探索.
\subsection{双变量分析}
双变量分析目标是确定两个变量之间的相关性,测量它们之间的预测或解释的能力.
\subsubsection{散点图}
在笛卡尔平面上将一个变量对另一个变量进行绘图,从而创建散点图,如果数据似乎符合直线或曲线,那么这两个变量之间存在关系或相关性.
\begin{figure}[ht]
\centering
\includegraphics[width=12cm]{./figures/DatExp_1.png}
\caption{散点图} \label{DatExp_fig1}
\end{figure}
\subsubsection{热力图}
利用多个变量之间的相关性矩阵做出热力图,以确定两个变量之间的相关性.在建模的时候衡量变量相关性我们一般都是计算变量两两之间的皮尔逊相关系数\autoref{PearsR_def1}~\upref{PearsR}(Pearson correlation coefficient),并以热力图的方式展示.
\begin{figure}[ht]
\centering
\includegraphics[width=12cm]{./figures/DatExp_2.png}
\caption{热力图} \label{DatExp_fig2}
\end{figure}
\subsubsection{列联表}
列联表(contingency table)是观测数据按两个或更多属性(定性变量)分类时所列出的频数表.它是由两个以上的变量进行交叉分类的频数分布表.
若总体中的个体可按两个属性$A$与$B$分类,$A$有$r$个等级$A_1,A_2,\dots,A_r$,$B$有$c$个等级$B_1,B_2,\dots, B_c$,从总体中抽取大小为n的样本,设其中有$f_{ij}$个个体的属性属于等级$A_i$和$B_j$,$f_{ij}$称为频数,将$r\times c$个$f_{ij}$排列为一个$r$行$c$列的二维列联表,简称 $r\times c$表.
\begin{figure}[ht]
\centering
\includegraphics[width=12cm]{./figures/DatExp_3.png}
\caption{列联表} \label{DatExp_fig3}
\end{figure}
列联表分析的基本问题是,判明所考察的各属性之间有无关联,即是否独立.在$r\times c$表中,若以$p_i$、$p_j$和$p_{ij}$分别表示总体中的个体属于等级$A_i$,属于等级$B_j$和同时属于$A_i$、$B_j$的概率,“A、B两属性无关联”的假设可以表述为$H0: p_{ij}=p_i\cdot p_j, (i=1,2,\dots, r; \ j=1,2,\dots,c)$,未知参数 $p_{ij}$、$p_i$、$p_j$ 的最大似然估计分别为行和及列和为样本大小.当$H0$成立,且一切$p_i>0$和$p_j>0$时,统计量的渐近分布是自由度为$(r-1)(c-1)$ 的$\chi^2$分布,式中$E_{ij}=(n_i \cdot n_j)/n$称为期望频数.当$n$足够大,且表中各格的$E_{ij}$都不太小时,可以据此对$H0$作检验:若$\chi^2$值足够大,就拒绝假设$H0$,即认为$A$与$B$有关联.

\subsubsection{卡方检验}
卡方检验就是统计样本的实际观测值与理论推断值之间的偏离程度,实际观测值与理论推断值之间的偏离程度就决定卡方值的大小,如果卡方值越大,二者偏差程度越大;反之,二者偏差越小;若两个值完全相等时,卡方值就为0,表明理论值完全符合.卡方检验的常见应用即拟合优度检验和列联表独立性检验.
\subsubsection{Z检验}
\addTODO{关联词条}
Z检验(Z Test)又叫U检验.由于实际问题中大多数随机变量服从或近似服从正态分布,U作为检验统计量与X的均值是等价的,且计算U的分位数或查相应的分布表比较方便.通过比较由样本观测值得到的U的观测值,可以判断数学期望的显著性,我们把这种利用服从标准正态分布统计量的检验方法成为U检验(U-test).
\subsubsection{T检验}
\addTODO{关联词条}
t检验,亦称student t检验(Student's t test),主要用于样本含量较小(例如n < 30),总体标准差σ未知的正态分布. t检验是用t分布理论来推论差异发生的概率,从而比较两个平均数的差异是否显著.它与f检验、卡方检验并列.
\subsubsection{方差分析(ANOVA)}
\addTODO{关联词条}
方差分析(Analysis of Variance,简称ANOVA),又称“变异数分析”,是R.A.Fisher发明的,用于两个及两个以上样本均数差别的显著性检验. 由于各种因素的影响,研究所得的数据呈现波动状.造成波动的原因可分成两类,一是不可控的随机因素,另一是研究中施加的对结果形成影响的可控因素.

