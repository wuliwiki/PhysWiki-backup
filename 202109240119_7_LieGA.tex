% 李群的李代数
% 李群|李代数|Lie group|Lie algebra|切空间|tangent space|向量场|vector field|李括号|Lie bracket

\pentry{李群\upref{LieGrp},李代数\upref{LieAlg}}

\subsection{光滑向量场}

我们先回顾光滑向量场的两条性质.

\begin{lemma}{提升映射与向量场的交换性}\label{LieGA_lem1}
设$F:M\to N$是一个\textbf{微分同胚},那么对于任意光滑函数$f\in C^{\infty}(M)$和光滑向量场$X\in\mathfrak{X}(M)$,都有:
\begin{equation}
(Xf)\circ F^{-1}=F_*(X)(f\circ F^{-1})
\end{equation}
其中$F_*:TM\to TN$是$F$的微分.
\end{lemma}

\autoref{LieGA_lem1} 根据“向量场对光滑函数作用”的定义就可以证出.

\begin{lemma}{微分和李括号的交换性}\label{LieGA_lem2}
设$F:M\to N$是一个\textbf{微分同胚},那么对于任意光滑向量场$X, Y\in\mathfrak{X}(M)$,有$F_*([X, Y])=[F_*(X), F_*(Y)]$.
\end{lemma}

\textbf{证明}:

我们只需要证明$F_*(XY)(f\circ F^{-1})=F_*(X)F_*(Y)(f\circ F^{-1})$对于任意$f\in C^{\infty}(M)$成立即可.

由\autoref{LieGA_lem1} ,
\begin{equation}
\begin{aligned}
F_*(XY)(f\circ F^{-1})&=(XYf)\circ F^{-1}\\
&=F_*(X)(Yf\circ F^{-1})\\
&=F_*(X)(F_*(Y)(f\circ F^{-1}))\\
&=F_*(X)F_*(Y)(f\circ F^{-1})
\end{aligned}
\end{equation}

\textbf{证毕}.

\subsection{李群上的左不变向量场}

\begin{definition}{左不变向量场}\label{LieGA_def1}
给定李群$G$,对于任意$g\in G$,定义映射$l_g:G\to G$为\textbf{左平移映射},即对于任意$p\in G$,都有$l_g(p)=gp$.

根据李群的定义,$l_g$是一个光滑映射,因此可以求其微分$l_{g*}:TG\to TG$,该微分把一个$p\in G$处的切向量映射为$gp$处的切向量.

如果存在$G$上的切向量场$X$\footnote{注意,只要求是切向量场,没有要求连续性,更没有要求光滑性.但是稍后我们会看到,所谓的左不变向量场必然是光滑的.},使得对于\textbf{任意}$g\in G$,都有$l_g(X)=X$;换句话说,就是$l_{g*}(X_p)=X_g$,那么称$X$为$G$上的一个\textbf{左不变切向量场(left-invariant tangent vector field)}.
\end{definition}

\autoref{LieGA_def1} 已经直白地说明了,一个左不变向量场$X$唯一地由其在单位元处的取值$X_e$决定的,即$X_p=l_{p*}(X_e)$.同时,任意给定$X_e$,都能由此生成唯一的左不变向量场$X$.这样,左不变向量场的性质被其在单位元处的取值完全决定,单位元就好像存储了全息信息一样,局部就可以描述整体.

\begin{theorem}{}
左不变切向量场必为光滑向量场.
\end{theorem}

\textbf{证明}:

\addTODO{证明有问题.}

% 考虑李群$G$上的左不变切向量场$X$.

% 只需要证明对于任意$f\in C^{\infty}(G)$,$Xf$都是光滑函数即可.

% $Xf$在$p\in G$处的取值可以如下计算,其中取\textbf{光滑}道路$c:I\to G$使得$\frac{\dd}{\dd t}c(t)|_{t=0}=X_e$,$c(0)=e$,而$e$是$G$的单位元:
% \begin{equation}
% Xf|_p=X_pf=l_{p*}(X_e)f=\frac{\dd}{\dd t}f(pc(t))|_{t=0}
% \end{equation}

% 这里$p, c(t)$都是群元素,$pc(t)$是它们的群乘法.

% 我们只需要证明$\frac{\dd}{\dd t}f(pc(t))|_{t=0}=\frac{\dd}{\dd t}f\circ l_{p*}\circ c|_{t=0}$作为一个$G\to \mathbb{R}$的函数是光滑的即可.注意,这个函数的自变量是$p$.

% 为了证明上面这段话,我们又只需要证明$f\circ l_{p*}\circ c$是一个$G\times I\to \mathbb{R}$的光滑函数即可.

% 由于$f, l_{p*}, c$都是光滑函数,其组合自然也是光滑函数,由此得证.

\textbf{证毕}.

接下来这条性质是引入李代数的关键.

\begin{exercise}{}
如果$X, Y$是左不变切向量场,那么$[X, Y]=XY-YX$也是.

利用\autoref{LieGA_lem2} ,取左平移映射为所用的微分同胚,证明这一点.
\end{exercise}


\subsection{李群上的李代数}

前两节的结论让我们知道了如下事实:李群$G$上全体左不变切向量场的集合,$L(G)$,构成了光滑向量场$\mathfrak{X}(G)$的一个子线性空间.同时,由于左不变切向量场可以被局部描述,我们可以通过单位元上的切空间$T_eG$来描述$L(G)$;换句话说,$T_eG$和$L(G)$是同构的.

\begin{definition}{}
如果$X, Y\in L(G)$是李群$G$上的两个左不变向量场,那么定义$X_e, Y_e\in T_eG$的李括号为
\begin{equation}
[X_e, Y_e]=[X, Y]_e
\end{equation}
\end{definition}

%是否需要讨论矩阵李群上的李代数呢?恰好就是矩阵乘法的李代数.













