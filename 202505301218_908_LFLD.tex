% 列夫·朗道(综述)
% license CCBYSA3
% type Wiki

本文根据 CC-BY-SA 协议转载翻译自维基百科 \href{https://en.wikipedia.org/wiki/Lev_Landau}{相关文章}。

列夫·达维多维奇·朗道(俄语:Лев Дави́дович Ланда́у,1908年1月22日-1968年4月1日)是一位苏联物理学家,在理论物理的诸多领域作出了基础性的贡献。\(^\text{[1][2][3]}\)他被认为是最后一批在物理学各个分支都造诣深厚并做出开创性贡献的科学家之一。\(^\text{[4]}\)他被誉为20世纪凝聚态物理学的奠基人,\(^\text{[5]}\)同时也被广泛认为是苏联最杰出的理论物理学家。\(^\text{[6]}\)
\subsection{生平}
\subsubsection{早年时期}
朗道于1908年1月22日出生在俄罗斯帝国的巴库(今属阿塞拜疆),父母是犹太人[11][12][13][14]。他父亲达维德·列沃维奇·朗道是一位从事当地石油工业的工程师,母亲柳博芙·维尼亚米诺芙娜·加尔卡维-朗道是一名医生。两人都来自莫吉廖夫,并毕业于当地的文理中学[15][16]。朗道12岁学习微分学,13岁学习积分学,并在1920年13岁时从中学毕业。由于父母认为他年龄太小,不适合直接升入大学,他先在巴库经济技术学校学习了一年。
1922年,年仅14岁的朗道进入巴库国立大学,同时注册了两个系:物理与数学系以及化学系。后来他中止了化学的学习,但终其一生对化学始终保有兴趣。
\subsubsection{列宁格勒与欧洲时期}

1924年,朗道前往当时苏联物理学的主要中心——列宁格勒国立大学物理系,专注于**理论物理的学习**,并于1927年毕业。此后,他进入列宁格勒物理技术研究所攻读研究生,并最终于1934年获得**物理-数学科学博士学位**。[17]

1929年至1931年,朗道首次获得出国机会,依靠苏联政府(教育人民委员部)提供的**出国奖学金**,同时也得到了**洛克菲勒基金会**的资助。在这段时间里,他已能流利地使用**德语和法语**,并能以**英语**交流。\[18] 后来,他进一步提高了英语水平,并学习了**丹麦语**。\[19]

朗道曾短暂访问**哥廷根**和**莱比锡**,随后于**1930年4月8日**前往**哥本哈根**,在**尼尔斯·玻尔理论物理研究所**工作,直到同年**5月3日**离开。这次访问之后,朗道始终视自己为**尼尔斯·玻尔的学生**,而他的物理研究方法也受到玻尔深刻的影响。

离开哥本哈根后,朗道于1930年中期访问**剑桥**,与**保罗·狄拉克**合作研究,\[20] 同年**9月至11月**他再次回到哥本哈根,\[21] 接着于**1930年12月至1931年1月**在**苏黎世**与**沃尔夫冈·泡利**共事。\[20] 从苏黎世出发后,他第三次前往哥本哈根,\[22] 并于**1931年2月25日至3月19日**再次在那里停留,然后于同年返回列宁格勒。\[23]
