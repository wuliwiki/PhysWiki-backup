% 四维矢量
% 相对论
\subsection{概念}
设$K$与$K'$为两个坐标原点重合的惯性系。在开始计时后,$K'$相对$K$有沿着$x$轴的相对速度。从$0$时刻开始,光信号沿着x轴运动,设其在两个参考系的时空坐标分别为$(t,x,y,z)$与$(t',x',y',z')$,由光速不变,我们有:

$$ds^2=ct^2-x^2-y^2-z^2=ct'^2-x'^2-y'^2-z'^2$$

从直观上,这看起来是坐标矢量$(t,x,y,z)$“长度”的平方不随惯性系的改变而改变。准确来说,是其逆变矢量和协变矢量的对偶内积。形式化这个运算,该闵氏时空下的度规为$\eta_{\mu\nu}=diag(+1,-1,-1,-1) $ ,设光速为1,我们有
\begin{equation}\label{eq_vect2_1}
x^\mu x_\mu =\eta_{\mu\nu}x^\mu x^\nu=\eta_{\rho \sigma}x'^\rho x'^\sigma   
\end{equation}

这意味着改变惯性系相当于对原惯性系进行保距变换,即正交线性变换,我们把这个正交线性变换称之为洛伦兹变换。设洛伦兹变换矩阵为$\Lambda$,在矩阵下,这个线性变换表示为:$x^T x=x^T\Lambda^{-1}\Lambda x$,通常用$\Lambda^\mu_\nu $的形式表示矩阵,配合指标表示法进行运算。那么把该矩阵回代\autoref{eq_vect2_1} 我们有
\begin{equation}
\Lambda^\rho_\nu x^\nu=x^\rho
\end{equation}
\begin{equation}
\eta_{\mu\nu}=\eta_{\rho \sigma}\Lambda^\rho_\nu \Lambda^\sigma_\mu 
\end{equation}

利用坐标矢量的洛伦兹变换,我们可以把很多物理量扩展成四维形式。
\subsection{力学实例}
\subsection{电动力学实例}
\subsection{拓展:洛伦兹张量}
我们知道,张量实际上是多重线性映射,而洛伦兹张量则默认了基变换的过渡矩阵为洛伦兹矩阵。以二阶张量$F^{\mu\nu} $为例,
\begin{equation}
F'^{\mu\nu}=\Lambda_\rho^\mu \Lambda_\sigma^\nu F^{\rho\sigma}
F'_{\mu\nu}=\Lambda_\mu^\rho \Lambda_\nu^\sigma F_{\rho\sigma}
\end{equation}
指标的不同位置往往表明张量的作用对象,比如

