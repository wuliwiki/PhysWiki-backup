% 辗转相除法
% 互素|最大公因式|算法|辗转相除

\pentry{多项式的整除\upref{ExDiv}}
\begin{issues}
\issueOther{缺例题}
\end{issues}
\footnote{吴群。矩阵分析[M].上海:同济大学出版社}本节将证明最大公因式的存在性,并介绍最大公因式的算法程序——辗转相除法。简单来说,所谓的\textbf{辗转相除法}是指多次利用带余除法计算两个多项式的最大公因式的算法。


推荐的拓展阅读:\textbf{欧几里得环}\upref{EuRing}。简单来说,这是一种“能做辗转相除法”的环\footnote{环是一种数学结构,详见小时百科《代数学进阶》部分。}。


\subsection{最大公因式的存在性}
为证明最大公因式的存在,我们先来证明下面一个定理。
\begin{theorem}{}\label{SucDiv_the1}
设 $f(x),g(x),q(x),r(x)\in\mathbb{F}[x]$,并且
\begin{equation}\label{SucDiv_eq1}
f(x)=q(x)g(x)+r(x)
\end{equation}
则 $f(x),g(x)$ 与 $g(x),r(x)$ 有相同的公因式。
\end{theorem}
\textbf{证明:}1.$\phi(x)|g(x),\phi(x)|f(x)\Rightarrow \phi(x)|g(x),\phi(x)|r(x)$ 的证明:

因为 $\phi(x)|g(x),\phi(x)|f(x)$ ,由整除的性质3多项式的整除\upref{ExDiv} ,$\phi(x)$ 必整除 $f(x),g(x)$ 的组合
\begin{equation}
\phi(x)|f(x)-q(x)g(x)\Rightarrow \phi(x)|r(x)
\end{equation}

2.$\phi(x)|g(x),\phi(x)|r(x)\Rightarrow\phi(x)|g(x),\phi(x)|f(x)$ 的证明:同样有整除的性质3,立即证得。

下面利用\autoref{SucDiv_eq1} 来证明最大公因式的存在性,该证明方法便是\textbf{辗转相除法}。
\begin{theorem}{最大公因式存在定理}\label{SucDiv_the2}
设 $f(x),g(x)$ 是 $\mathbb{F}[x]$ 中的两个多项式,则在 $\mathbb{F}[x]$ 中, $f(x)$ 与 $g(x)$ 的最大公因式 $d(x)$ 存在,且有 $\mathbb{F}[x]$ 中的两个多项式 $u(x),v(x)$ 使得
\begin{equation}\label{SucDiv_eq2}
d(x)=u(x)f(x)+v(x)g(x)
\end{equation}

\end{theorem}
\textbf{证明:}如果 $f(x)$ 和 $g(x)$ 中有一个为0,比如 $g(x)=0$,那么另一个 $f(x)$ 就是 $f(x)$ 与 $g(x)$ 的最大公因式,且 $f(x)=1\cdot f(x)+1\cdot0$。

以下只需考虑 $g(x)\neq 0$。由带余除法\upref{DivAlg} ,有
\begin{equation}
f(x)=q_1(x)g(x)+r_1(x)
\end{equation}
只考虑 $r_1(x)\neq 0$,否则 $g(x)$ 便是最大公因式。再用 $r_1(x)$ 除 $g(x)$,得
\begin{equation}
g(x)=q_2(x)r_1(x)+r_2(x)
\end{equation}
同样只考虑 $r_2(x)\neq0$,再用 $r_2(x)$ 除 $r_1(x)$,得
\begin{equation}
r_1(x)=q_3(x)r_2(x)+r_3(x)
\end{equation}
如此不断重复该过程,所得余式的次数就不断降低,即
\begin{equation}
\mathrm{deg}\;g(x)>\mathrm{deg}\;r_1(x)>\mathrm{deg}\;r_2(x)>\cdots
\end{equation}
这样经过有限次相除之后,必然有余式为0。于是得到一等式:
\begin{equation}\label{SucDiv_eq3}
\begin{aligned}
f(x)&=q_1(x)g(x)+r_1(x)\\
g(x)&=q_2(x)r_1(x)+r_2(x)\\
r_1(x)&=q_3(x)r_2(x)+r_3(x)\\
&\vdots\\
r_{i-2}(x)&=q_i(x)r_{i-1}(x)+r_i(x)\\
&\vdots\\
r_{s-2}(x)&=q_s(x)r_{s-1}(x)+r_s(x)\\
r_{s-1}(x)&=q_{s+1}(x)r_s(x)
\end{aligned}
\end{equation}
其中 $r_s(x)$ 是 $r_s(x)$ 与 $r_{s-1}(x)$ 的一个最大公因式。由\autoref{SucDiv_the1} ,$r_s(x)$ 也是 $r_{s-1}(x)$ 与 $r_{s-2}(x)$ 的一个最大公因式。照此逐步推上去,便有 $r_s(x)$ 是 $f(x)$ 和 $g(x)$ 的一个最大公因式。这就证明了最大公因式的存在性。

下面来证明\autoref{SucDiv_eq2} .由\autoref{SucDiv_eq3} 倒数第二式
\begin{equation}
r_s(x)=r_{s-2}(x)-q_s(x)r_{s-1}(x)
\end{equation}
这就是说,每一个 $r_i$ 都可由只带 $r_{i-2}$ 和 $r_{i-1}$ 的项表示。若令 $g(x)=r_0(x),f(x)=r_{-1}(x)$ ,则显然所有的 $r_i$ 最终都可由只带 $g(x)$ 和 $f(x)$ 的项表示,即得\autoref{SucDiv_eq2} 。
\subsection{互素}
两个多项式的最大公因式的存在,使我们能够定义\textbf{互素} 的概念。这和整数中的互素的概念一样,都是指两个对象(这个对象在数论里就是整数)的最大公因式(对于整数叫公因子)为1.
\begin{definition}{互素}
设 $f(x)$ 和 $g(x)$ 为 $\mathbb{F}[x]$ 中的两个多项式,若 $(f(x),g(x))=1$,则称 $f(x)$ 与 $g(x)$ \textbf{互素}。
\end{definition}
由整除的性质1\upref{ExDiv},或由词条\upref{ExDiv}的\autoref{ExDiv_ex1}~\upref{ExDiv} 前一段指出的,可知若两个多项式互素,那么它们的公因式只能是零次多项式。

将互素的概念与最大公因式存在定理相结合,就有下面几个有用的推论。
\begin{theorem}{}\label{SucDiv_the3}
$\mathbb{F}[x]$ 中两个多项式 $f(x),g(x)$ 互素的充要条件是存在 $\mathbb{F}[x]$ 中的两个多项式 $u(x),v(x)$ 使得
\begin{equation}
u(x)f(x)+v(x)g(x)=1
\end{equation}
\end{theorem}
\textbf{证明}:\autoref{SucDiv_the2} 可以直接得出必要性。现在来证明充分性,设存在 $u(x),v(x)$ 使得 
\begin{equation}
u(x)f(x)+v(x)g(x)=1
\end{equation}
并且 $\phi(x)$ 是 $f(x)$ 与 $g(x)$ 的一个最大公因式。于是就有
\begin{equation}
\phi(x)|f(x),\quad \phi(x)|g(x)
\end{equation}
由整除性质3\upref{ExDiv},立即得到 $\phi(x)|1$ ,即 $f(x)$ 与 $g(x)$ 互素。
\begin{theorem}{}\label{SucDiv_the4}
若 $(f(x),g(x))=1$,且 $f(x)|g(x)h(x)$,则 $f(x)|h(x)$
\end{theorem}
\textbf{证明:}由 $(f(x),g(x))=1$ 和\autoref{SucDiv_the3} 可知,有 $u(x),v(x)$ 使
\begin{equation}
u(x)f(x)+v(x)g(x)=1
\end{equation}
上式两边乘 $h(x)$,得
\begin{equation}
u(x)f(x)h(x)+v(x)g(x)h(x)=h(x)
\end{equation}
因为 $f(x)|g(x)h(x)$,所以 $f(x)$ 整除等式左端,从而 $f(x)|h(x)$。
\begin{theorem}{}
若 $f_1(x)|g(x),f_2(x)|g(x)$,并且 $(f_1(x),f_2(x))=1$ ,则 $f_1(x)f_2(x)|g(x)$。
\end{theorem}
\textbf{证明:}因为 $(f_1(x),f_2(x))=1$ ,由\autoref{SucDiv_the3} ,存在 $u(x),v(x)$ ,使得
\begin{equation}
u(x)f_1(x)+v(x)f_2(x)=1
\end{equation}
上式两边乘 $g(x)$,则
\begin{equation}\label{SucDiv_eq5}
g(x)u(x)f_1(x)+g(x)v(x)f_2(x)=g(x)
\end{equation}
又 $f_1(x)|g(x),f_2(x)|g(x)$,则有 $q_1(x),q_2(x)$,使得
\begin{equation}\label{SucDiv_eq4}
g(x)=q_1(x)f_1(x)=q_2(x)f_2(x)
\end{equation}
\autoref{SucDiv_eq4} 代入 \autoref{SucDiv_eq5} ,得
\begin{equation}
u(x)q_2(x)f_2(x)f_1(x)+v(x)q_1(x)f_1(x)f_2(x)=g(x)
\end{equation}
所以 $f_1(x)f_2(x)|g(x)$。