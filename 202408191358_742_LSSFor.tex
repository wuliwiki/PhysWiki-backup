% 大尺度结构形成
% license Usr
% type Tutor


1.3.1 大尺度结构的形成

今天,宇宙在比目前视界更小的尺度上是非常不均匀的。例如,如星系调查结果所示,它是一个可观宇宙中所有星系的三维地图。仅凭肉眼就可以看出各种结构:团块、丝状、墙、空洞...,它们出现在不同的尺度上。其他天文观测,如Lyman-α森林、弱透镜测量、星系团计数等,也证实了宇宙是一个团块状的宇宙。定量地,从所有这些测量中可以提取出物质功率谱P(k)。它通过以下方式定义:
\begin{equation}
\langle \delta_k \delta_{k'} \rangle = (2\pi)^3 P(k) \delta^3(\mathbf k - \mathbf k')~.
\end{equation}
其中,$\delta_k$是密度对比$\delta(\mathbf r)$的傅里叶变换,而 
 表示对k方向的平均。狄拉克δ函数 
 不要与表示扰动的δ混淆,意味着具有不同k ≠ k'的模式在统计上是独立的。这一特性可以从膨胀中理解,膨胀预测了平均统计特性,这些特性编码在P(k)中。由于P(k)是密度自相关函数的傅里叶变换,它在傅里叶空间方便地表达了物理空间中物质分布的不均匀性。P(k)的大(小)值意味着存在许多(少)具有特征尺寸 
 的结构。图1.5(中间右侧)中的测量结果因此表明,宇宙在所有尺度上都有一定的功率。为了使宇宙的团块状更加明显,有时将P(k)(单位为长度的三次方)重新缩放为无量纲的方差 
 。小的
 对应于小的密度对比,而 
 ,例如,表示与平均密度相当的过密度。图1.5中的P(k)数据的快速操作表明,
 
  在大k区域确实上升到大值。换句话说,数据显示宇宙在小尺度(大k)上表现出大的不均匀性。在标准宇宙学模型中,原始的不均匀性是由具有小振幅的膨胀产生的,
 
 。这从CMB的近乎完美平滑中得到证实,CMB基本上提供了一张在光子最后一次散射时宇宙光子内容的照片。这提出了一个问题:这些微小的原始不均匀性如何从如此小的振幅增长到我们今天观察到的大对比度?答案如上所述,密度扰动的增长主要受到暗物质的驱动。为了定量理解这一过程,我们需要概述早期宇宙中这些扰动的演化。