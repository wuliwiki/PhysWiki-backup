% 长度规范和速度规范
% 长度规范|速度规范|波函数|规范变换|薛定谔方程|麦克斯韦方程组

\begin{issues}
\issueDraft
\end{issues}

\pentry{电磁场中的单粒子薛定谔方程\upref{QMEM}, 偶极子近似} % 未完成

我们已知库伦规范下, 电磁场中带电粒子的哈密顿量为(\autoref{QMEM_eq4}~\upref{QMEM})
\begin{equation}
\begin{aligned}
H &= \frac{\bvec p^2}{2m} - \frac{q}{m} \bvec A \vdot \bvec p + \frac{q^2}{2m} \bvec A^2 + q\varphi\\
&= -\frac{1}{2m} \laplacian + \I \frac{q}{m} \bvec A \vdot \Nabla + \frac{q^2}{2m} \bvec A^2 + q\varphi
\end{aligned}
\end{equation}
当我们用偶极子近似时, $\bvec A(t)$ 与位置无关而只是时间的函数. 我们可以利用这个性质方便地得到另外另种规范. 注意他们只有在偶极子近似时才成立.

\subsection{速度规范}
对库仑规范使用规范变换
\begin{equation}\label{LVgaug_eq3}
\Psi(\bvec r, t) = \exp(\I q\chi)\Psi^V(\bvec r, t)
\end{equation}
\begin{equation}\label{LVgaug_eq4}
\chi(t) = \frac{q}{2m} \int^t \bvec A^2(t') \dd{t'}
\end{equation}
注意\autoref{QMEM_eq5}~\upref{QMEM}中 $\grad \chi = \bvec 0$, 利用 $\varphi$ 的变换就可以将 $\bvec A^2$ 项消去
\begin{equation}
\I \pdv{t} \Psi^V = \qty(\frac{\bvec p^2}{2m} + q\varphi - \frac{q}{m} \bvec A \vdot \bvec p) \Psi^V
\end{equation}
这种规范叫做\textbf{速度规范(velocity gauge)}.

\subsection{长度规范}

\begin{equation}
\Psi(\bvec r, t) = \exp(\I q\chi)\Psi^L(\bvec r, t)
\end{equation}
\begin{equation}
\chi(t) = \bvec A(t) \vdot \bvec r
\end{equation}
$\grad X = \bvec A(t)$, $\pdv*{\chi}{t} = -\bvec {\mathcal E} \vdot r$. 变换后得薛定谔方程为
\begin{equation}
\I \pdv{t} \Psi^L = [H_0 + \bvec{\mathcal{E}} \vdot \bvec r] \Psi^L
\end{equation}
这种规范叫做\textbf{长度规范}.
