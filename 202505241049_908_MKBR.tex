% 麦克斯韦-玻尔兹曼分布(综述)
% license CCBYSA3
% type Wiki

本文根据 CC-BY-SA 协议转载翻译自维基百科\href{https://en.wikipedia.org/wiki/Maxwell\%E2\%80\%93Boltzmann_distribution}{相关文章}。

在物理学中(特别是在统计力学中),麦克斯韦–玻尔兹曼分布(Maxwell–Boltzmann distribution,或称麦克斯韦分布)是一种特定的概率分布,以詹姆斯·克拉克·麦克斯韦和路德维希·玻尔兹曼的名字命名。

该分布最初被定义并用于描述理想气体中粒子的速度分布,在这种理想化模型中,粒子在静止容器内自由运动,彼此之间没有相互作用,除了极短暂的碰撞,在这些碰撞中粒子与其他粒子或热环境交换能量和动量。在此语境下,“粒子”仅指气体粒子(即原子或分子),并假设该粒子系统已达到热力学平衡状态。[1]这类粒子的能量遵循麦克斯韦–玻尔兹曼统计,其速度的统计分布可通过将粒子能量与动能等同来推导得出。
