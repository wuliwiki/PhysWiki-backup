% 狄拉克 delta 导函数
% license Xiao
% type Tutor

\begin{issues}
\issueDraft
\end{issues}

\pentry{狄拉克 delta 函数\nref{nod_Delta}}{nod_4a17}

可以通过 $\delta$ 函数列构造 $\delta'$ 函数列:
\begin{equation}
\delta'_n(x) = \frac{\delta_n(x-h_n) - \delta_n(x+h_n)}{2h_n}~.
\end{equation}
其中
\begin{equation}
\lim_{n\to\infty} h_n = 0~.
\end{equation}

分部积分:
\begin{equation}
\begin{aligned}
&\quad \lim_{n\to\infty}\int_{-\infty}^\infty \delta'_n(x) f(x) \dd{x}\\
&= \lim_{n\to\infty}\eval{\delta_n(x) f(x)}_{-\infty}^\infty - \lim_{n\to\infty}\int_{-\infty}^\infty \delta_n(x) f'(x) \dd{x}\\
&= - f'(0)~.
\end{aligned}
\end{equation}
所以狄拉克 delta 导函数就是任意 delta 函数列的负导数。

可以拓展到
\begin{equation}
\lim_{n\to\infty}\int_{-\infty}^\infty -\delta'_n(x-x_0) f(x) \dd{x} = f'(x_0)~.
\end{equation}

性质: $\delta'(-x) = -\delta'(x)$ (但每个具体的 $\delta'_n$ 未必是奇函数)

\begin{example}{}
\autoref{ex_Delta_2}~\upref{Delta}
\begin{equation}
\delta_n(x) = \frac{n}{\pi}\sinc(n x) \qquad (n = 1, 2, \dots)~,
\end{equation}
\begin{equation}
-\delta'_n(x) = \frac{n}{\pi x} [\sinc(nx) - \cos(nx)]~.
\end{equation}
\end{example}
