% 量子力学诠释
% license CCBYSA3
% type Wiki

(本文根据 CC-BY-SA 协议转载自原搜狗科学百科对英文维基百科的翻译)

所谓\textbf{量子力学的解释}是试图用数学的方式将量子力学与现实“对应”起来。尽管量子力学在广泛的实验中经受住了严格和极其精确的考验(所有量子力学的预测都与实验相符),但是对于量子力学的解释存在着许多不同的思想流派。这些流派在诸如量子力学是确定的还是随机的、量子力学的哪些量可以被认为是“真实的”、测量的本质是什么等基本问题的观点上有所不同。

尽管进行了近一个世纪的辩论和实验,物理学家和哲学家们还没有就哪种解释最能“代表”现实达成共识。

\subsection{历史}

\begin{figure}[ht]
\centering
\includegraphics[width=6cm]{./figures/3ef8c507a5bfcd47.png}
\caption{薛定谔} \label{fig_QMinet_1}
\end{figure}

\begin{figure}[ht]
\centering
\includegraphics[width=6cm]{./figures/33a6c91157a45fe2.png}
\caption{玻姆} \label{fig_QMinet_2}
\end{figure}

量子力学解释中具有影响力的人物

薛定谔
玻姆
量子物理学家所使用的术语的定义,如\textbf{}波函数和矩阵力学,经历了许多变化阶段。例如,埃尔温·薛定谔最初认为电子的波函数是它在空间中的电荷密度,而马克斯·玻恩将波函数的绝对值的平方重新解释为电子在空间中的概率密度。

如尼尔斯·玻尔和维尔纳 ·海森堡这样的早期量子力学先驱的观点,经常被归为“哥本哈根解释”,尽管物理学家和物理史学家认为这个术语掩盖了观点之间的差异。虽然哥本哈根式的想法从未被普遍接受,但对哥本哈根正统观念的挑战在20世纪50年代随着大卫·玻姆的先导波解释和休·艾弗雷特三世的多世界解释的兴起而日益受到关注。[2][3] 此外,试图回避区分解释学派的严苛主义立场受到了可证伪实验的挑战,这些实验可能有朝一日可以利用诸如人工智能测量或量子计算来区分不同的解释学派。[4] [5]

物理学家大卫·梅尔明曾经打趣道:“新的解释学派每年都会出现。但是没有哪个解释学派会消失。”作为20世纪90年代至21世纪初主流观点发展的粗略指南,可以参考一下施洛斯豪尔等人在2011年7月“量子物理和现实本质”会议上的民意调查中收集的“快照”。[6]作者引用了马克斯·泰格马克在1997年8月“量子中的基本问题”会议上进行的一项类似的非正式投票调查。作者的主要结论是,“除了哥本哈根的解释仍然是占据很大优势,在他们的投票中获得最多的选票(42\verb|%| ),多世界解释也得到了显著上升:

“哥本哈根解释在这里仍然是有着巨大的优势,特别是如果我们把它和以此产生相关解释(如基于信息的解释和量子贝叶斯解释)算在一起。在泰格马克的民意测验中,埃弗雷特的解释获得了17\verb|%|的选票,这与我们民意测验中的票数(18\verb|%|)相似。"
值得注意的是,只有克莱默在1986年发表的事务性解释,才为梅克斯·玻恩的断言赋予物理基础,即波函数的绝对平方是概率密度。[7]
