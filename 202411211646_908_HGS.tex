% 克里斯蒂安·惠更斯(综述)
% license CCBYSA3
% type Wiki

本文根据 CC-BY-SA 协议转载翻译自维基百科\href{https://en.wikipedia.org/wiki/Christiaan_Huygens}{相关文章}。

\begin{figure}[ht]
\centering
\includegraphics[width=6cm]{./figures/759a661ac1a7d67e.png}
\caption{惠更斯肖像,由卡斯帕·内彻绘于1671年,现藏于莱顿博尔哈夫博物馆[1]} \label{fig_HGS_1}
\end{figure}

克里斯蒂安·惠更斯,泽尔亨领主,英国皇家学会院士(/ˈhaɪɡənz/,音译‘海根斯’,[2] 美国亦发音为 /ˈhɔɪɡənz/,音译‘霍伊根斯’;[3] 荷兰语:[ˈkrɪstijaːn ˈɦœyɣə(n)s] ⓘ;也拼作 Huyghens;拉丁语:Hugenius;1629年4月14日-1695年7月8日),是一位荷兰数学家、物理学家、工程师、天文学家和发明家,被视为科学革命中的关键人物之一。[4][5] 在物理学领域,惠更斯在光学和力学方面做出了开创性的贡献;作为天文学家,他研究了土星的光环并发现了土星最大的卫星——泰坦。作为工程师和发明家,他改进了望远镜的设计,并发明了摆钟,这种时钟在近300年内是最精确的计时工具。他是一位才华横溢的数学家和物理学家,其著作首次通过一组数学参数对物理问题进行了理想化描述,[6] 并首次对一种无法直接观测的物理现象进行了数学和机械论的解释。[7]