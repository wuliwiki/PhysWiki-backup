% 纤维丛
% keys 纤维丛|丛空间|拓扑学|拓扑|向量丛|乘积拓扑
% license Xiao
% type Tutor

\begin{issues}
\issueDraft
\end{issues}

\pentry{乘积拓扑\nref{nod_Topo6},矢量空间\nref{nod_LSpace}}{nod_88ff}

\subsection{定义}

直观来说,纤维丛是指在一个拓扑空间 $B$ 的每一个点都长出来另一个拓扑空间 $F$ 所得到的一个空间。每一个点 $x\in B$ 上的 $F$ 被称为一根\textbf{纤维(fibre)},这些纤维所在的 $B$ 称为\textbf{底空间(base space)},而整个结构 $(B, F)$ 就是一个\textbf{纤维丛(fibre bundle)}。

准确的定义如下所述,其中 $E$ 就是“$B$ 上每个点都长出一个 $F$ 的丛空间”:

\begin{definition}{纤维丛}
给定拓扑空间 $B$ 和 $F$,如果存在一个拓扑空间 $E$ 和一个连续满射 $\pi:E\rightarrow B$,使得对于任意的 $x\in B$,都有 $\pi^{-1}(x)$ 同胚于 $F$,那么称这个结构 $(E, F, B, \pi)$ 为一个\textbf{纤维丛(fibre bundle)}。

称 $E$ 是这个纤维丛的\textbf{全空间(total space)},$F$ 是其\textbf{纤维型(fibre type)},$\pi^{-1}(p)$是其在$p\in B$处的\textbf{纤维(fibre)},$B$ 是其\textbf{底空间(base space)},有时也译作\textbf{基空间}。
\end{definition}

如果把 $B$ 想象成一块土地,$F$ 想象成一棵草,那么 $E$ 就是“土地上长了一片草”这一概念,$E$ 的每个元素就是某棵草上的一个点。定义中的连续满射 $f$ 的作用是把这样的一个点映射到相应的草所在的地点。

每根纤维都是同胚的,即拓扑意义上都等价于纤维型。也就是说,纤维型是描述纤维的拓扑结构的,但不同点处的纤维不是同一个空间。

要注意的是,$E$ 不完全等同于 $B\times F$。对于 $B\times F$ 来说,任意给定两个 $x_1, x_2\in B$,我们自然可以找到 $x_1\times F$ 和 $x_2\times F$ 上的一一对应关系,这是由集合笛卡尔积的定义决定的。但是纤维丛 $E$ 上,如果上述 $x_1\not=x_2$,那么两个地方长出来的纤维是没有天然的双射对应的的\footnote{在微分几何中,我们研究的切丛是纤维丛的一种,而所谓的“联络”实际上就是指定了不同纤维间的双射。}。这就是“纤维丛”这一名称的深意,而乘积空间应该被想象纤维被粘在一起的情况,只是纤维丛的一个定义了额外联系的特例。

两个纤维丛之间可以有映射偶:

\begin{definition}{纤维丛的态射}
设 $(E_1, F_1, B_1, f_1)$ 和 $(E_2, F_2, B_2, f_2)$ 是两个纤维丛,
\end{definition}

\subsection{纤维丛的例子}

\subsubsection{平凡丛}


给定两个拓扑空间$X, Y$,则它们的积空间$X\times Y$可以看成纤维丛,称为\textbf{平凡丛(trivial bundle)}。$X\times Y$可以用$X$作底空间、$Y$作纤维型,取关于$X$的投影映射$\pi_X$构成纤维丛
\begin{equation}
(X\times Y, Y, X, \pi_X)~, 
\end{equation}
也可以反过来,用$Y$作底空间、$X$作纤维型,取关于$Y$的投影映射$\pi_Y$构成纤维丛
\begin{equation}
(X\times Y, X, Y, \pi_Y)~. 
\end{equation}


如上一小节所说,平凡丛的特点是,任意两根纤维之间都存在一个天然的同胚映射。下面所说的切丛则不存在这样的天然同胚。


\subsubsection{切丛}

\pentry{切空间(流形)\nref{nod_tgSpa}}{nod_1df5}


给定光滑实流形$M$,在其上每一点$p$处的全体切向量构成了一个线性空间,称为$p$处的\textbf{切空间}。所有点处的切空间维度相等,从而线性同构,从而同胚。这样,我们可以以各线性空间为纤维,构造流形上的纤维丛。


\begin{definition}{切丛}
给定$n$维光滑\textbf{实}流形$M$,记$p\in M$处的切空间为$\opn{T}_p M$,取线性拓扑\footnote{线性空间的拓扑取线性拓扑,即任意定义一个内积,取关于这个内积的度量拓扑。不同的内积定义的度量拓扑是等价的。}使之构成拓扑空间。显然,各$\opn{T}_pM$都线性同构于$n$维实线性空间,且同胚于$n$维欧几里得空间$\mathbb{R}^n$。

记\footnote{$\amalg$和$\coprod$都表示不交并。这里的$E$就是说把所有切空间都取不交并。}
\begin{equation}
\opn{T}M= \coprod_{p\in M}\opn{T}_p M~, 
\end{equation}
定义连续满射$\pi:\opn{T}M\to M$为:对于任意$v\in \opn{T}_p M$,$\pi(v)=p$。则可以得到一个纤维丛$(\opn{T}M, \mathbb{R}^n, M, \pi)$,称为$M$上的\textbf{切丛(tangent bundle)}。

\end{definition}





\subsubsection{向量丛}



向量丛是纤维丛的特例,即纤维都是向量空间的情况。显然,切丛是一种向量丛,但向量丛不止切丛这一种。

\begin{definition}{向量丛}\label{def_Fibre_1}
给定拓扑空间 $B$ 和$q$维线性空间 $V$,如果存在一个拓扑空间 $E$ ,一个连续满射 $\pi:E\rightarrow B$,$B$的一族开覆盖$\{U_\alpha\}$和一族映射$\varphi=\{\varphi_\alpha: U_\alpha\times V\to \pi^{-1}(U_\alpha)\}$,满足下列\textbf{相容条件(compatibility)}:
\begin{enumerate}
\item 各$\varphi_\alpha$是同胚映射,且对于任意$v\in V, p\in U_\alpha$有$\pi\circ\varphi_\alpha(p, v)=p$;
\item 对于\textbf{固定的}$p\in U_\alpha$,记$\phi_\alpha(v)=\varphi_\alpha(p, v)$。要求$\phi_\alpha(v)$是$V\to \pi^{-1}(p)$的同胚映射;且当$U_\alpha\cap U_\beta\not=\varnothing$时,$\phi_\beta^{-1}\circ\phi_\alpha$是$V$上的\textbf{线性自同构},
\end{enumerate}
则称$(E, B, \pi, \varphi)$是$B$上的$q$阶\textbf{向量丛(vector bundle)}。

\end{definition}






\begin{example}{经典力学}
经典力学中,可以把单个自由粒子的相空间看作是一维实流形上的三维实向量丛。作为底空间的一维实流形表示时间坐标,作为纤维的线性空间表示空间坐标。

更多讨论请参见\textbf{从分析力学到场论}\upref{CFa1}。
\end{example}





向量丛之间也有丛映射:

\begin{definition}{丛映射}
给定向量丛 $(E, V_E, M, \pi_E)$ 和 $(F, V_F, N, \pi_F])$,其中 $M$ 和 $N$ 是实流形。我们定义一个“\textbf{光滑丛映射($C^\infty$ bundle map)}”为 $E\rightarrow F$ 的映射偶 $\varphi: E\rightarrow F$ 和 $\overline{\varphi}: M\rightarrow N$,使得:
\begin{equation}
\overline{\varphi}\circ\pi_E=\pi_F\circ\varphi~.
\end{equation}
且在任意 $p\in M$ 处,$\varphi|_p$\footnote{即只考虑 $p$ 处纤维的映射 $\varphi$。}是从 $p\times V_E$ 到 $\overline{\varphi}(p)\times V_F$ 的映射,并且是一个线性映射。
\end{definition}

%虽然,一个向量丛 $(E, V, B, \phi)$ 不能简单等同于 $B\times V$,不过 $B\times V$ 本身也是一个纤维丛,称之为\textbf{平凡(trivial)}的纤维丛。




