% BLAS 简介
% BLAS|线性代数|科学计算|C 语言|接口

\pentry{C 语言基础} % 未完成

\textbf{BLAS(basic linear algebra subroutine)} 是一系列基本\textbf{线性代数运算}函数\footnote{编程中的函数, 不是数学上的函数, 在一些编程语言(如 Fortran)中也叫\textbf{子程序(subroutine)}.可以简单地认为 Fortran 函数有返回值,子程序没有,如下面的例子中,传进和传出的是指针}的\textbf{接口(interface)}标准.  这里的线性代数运算是指例如矢量的线性组合, 矩阵乘以矢量, 矩阵乘以矩阵等. 接口在这里指的是诸如哪个函数名实现什么功能, 有几个输入和输出变量, 分别是什么.

BLAS 被广泛用于科学计算和工业界, 已成为业界标准. 在更高级的语言和库中, 即使我们不直接使用 BLAS 接口, 它们也是通过调用 BLAS 来实现的(如 Matlab 中的各种矩阵运算).

BLAS 原本是用 Fortran 语言写的, 但后来也产生了 C 语言的版本 CBLAS, 接口与 Fortran 的略有不同(例如使用指针传递数组), 但大同小异.

注意 BLAS 是一个接口的标准而不是某种具体\textbf{实现(implementation)}. 简单来说, 就是不同的作者可以各自写出不同版本的 BLAS 库, 实现同样的接口和功能, 但每个函数内部的算法可以不同.  这些不同导致了不同版本的 BLAS 在不同机器上运行的速度也不同.

BLAS 的官网是 \href{http://www.netlib.org/blas/}{Netlib}, 可以浏览完整的说明文档以及下载源代码. 这个版本的 BLAS 被称为 reference BLAS, 运行速度较慢, 通常被其他版本用于衡量性能. 对于 Intel CPU 的计算机, 性能最高的是 Intel 的 \href{https://software.intel.com/en-us/mkl}{MKL (Math Kernel Library)} 中提供的 \href{https://software.intel.com/en-us/mkl-developer-reference-c-blas-and-sparse-blas-routines}{BLAS}. MKL 虽然不是一个开源软件, 但目前可以免费下载使用. 如果想要免费开源的版本, 可以尝试 \href{https://www.openblas.net/}{OpenBlas} 或者 \href{https://sourceforge.net/projects/math-atlas/}{ATLAS}\footnote{至于安装, 在 Windows 系统上作者推荐在 Visual Studio 的基础上安装 MKL 或者 Parallel Studio, 这些软件都比较大, 可能需要较长时间下载安装. 在 Linux 系统上, 可以直接用 apt-get 等安装开源版本, 也可以下载 intel MKL 的 deb 安装包按照提示安装.}. 另外, 无论是否使用 MKL, BLAS 的文档都推荐看 MKL 的\href{https://software.intel.com/en-us/mkl-developer-reference-c-blas-and-sparse-blas-routines}{相关页面}.
% 未完成: 要不要写安装教程?

\subsection{使用 CBLAS}
\pentry{矩阵的储存\upref{MatSto}}

我们这里介绍一个 CBLAS 的例子\footnote{要在 Linux 上编译, 见 “在 Linux 上编译第一个 C++ 程序\upref{linCpp}”}. 我们来测试矩阵和矢量的乘法函数 \verb|gemv|, 参考文档见\href{https://software.intel.com/en-us/node/834919#88940C4E-0889-46C3-B6CF-F8B6EA6CF4BC}{这里}.

BLAS 可以分为四种类型和三个级别

类型
\begin{itemize}
    \item single
    \item double
    \item complex
    \item double complex
\end{itemize}

级别(level)
\begin{enumerate}
    \item 向量与向量操作 (复杂度 $n$)
    \item 矩阵与向量操作 (复杂度 $n^2$)
    \item 矩阵与矩阵操作 (复杂度 $n^3$)
\end{enumerate}

根据上述的类型和级别,BLAS 有一套系统的命名规则,
我们需要的函数名格式为 \verb|cblas_?gemv|, 其中 \verb|cblas_| 是固定前缀, 问号表示 \verb|s|, \verb|d|, \verb|c|, \verb|z|, 中的一个, 分别代表单精度(\verb|float|), 双精度(\verb|double|), 单精度复数和双精度复数\footnote{注意 C 语言中内置的复数类型和 C++ 标准库中的 \lstinline|std::complex<>| 不同, 但由于 CBLAS 用指针 \lstinline|void *| 传递复数(任何指针都可以自动转换), 所以只要内存中用两个连续的 \lstinline|float| 或 \lstinline|double| 表示一个复数即可}. \verb|ge| 表示输入的矩阵是一个一般的矩阵\footnote{而不是对称矩阵或者厄米矩阵等, 后者可以使用专门的函数使性能提升, 例如 \href{https://software.intel.com/en-us/node/834934}{symv}}, 以行主序或者列主序\upref{MatSto}线性储存, \lstinline|mv| 表示矩阵与向量操作. 



我们接下来以 \verb|cblas_zgemv| 为例, 先来看函数声明.
\begin{lstlisting}[language=cpp]
void cblas_zgemv (const CBLAS_LAYOUT Layout, const CBLAS_TRANSPOSE trans,
const MKL_INT m, const MKL_INT n, const void *alpha, const void *a,
const MKL_INT lda, const void *x, const MKL_INT incx, const void *beta,
void *y, const MKL_INT incy);
\end{lstlisting}

BLAS 接口给人的第一感觉就是冗长, 为什么实现一个简单的功能需要这么多变量? 因为这个接口具有相当大的灵活性. 例如可以使用一个列主序(行主序)矩阵的子矩阵作为矩阵, 又例如可以使用一个列主序(行主序)矩阵的某一行(列)作为矢量; 例如可以在做乘法以前对矩阵进行转置\footnote{其实这个转置操作实现起来并不需要额外的运算, 在函数的代码中只需要把列主序(行主序)矩阵看作是行主序(列主序)的即可. 所以这么做比事先在内存中将矩阵元调换的方法快许多}; 又例如可以把相乘的结果累加到输出矢量已有的值上, 而不是直接覆盖.

作为一个最简单的例子, 我们可以用以下测试函数, 在这个例子中, 我们只需要关心几个变量, 即 \verb|Layout| (一个 \verb|enum| 类型)指定行主序(\verb|CblasRowMajor|)还是列主序(\verb|CblasColMajor|), \verb|m| 和 \verb|n| 分别指定矩阵的行数和列数, \verb|a, x, y| 分别是矩阵第一个元的指针.
\begin{lstlisting}[language=cpp, caption=blas\_demo.cpp]
// 计算矩阵—矢量乘法 y = a * x
#include <iostream>
#include <complex>
#include <cblas.h>
int main() {
    using namespace std;
    typedef complex<double> Comp; // 定义复数类型
    int Nr = 2, Nc = 3; // 矩阵行数和列数
    // 矩阵和矢量分配内存
    Comp *a = new Comp [Nr*Nc];
    Comp *x = new Comp [Nc];
    Comp *y = new Comp [Nr];
    Comp alpha(1, 0), beta(0, 0);
    // x 矢量赋值
    for (int i = 0; i < Nc; ++i) {
        x[i] = Comp(i+1., i+2.);
    }
    // a 矩阵赋值
    for (int i = 0; i < Nr*Nc; ++i) {
        a[i] = Comp(i+1., i+2.);
    }
    // 做乘法
    cblas_zgemv(CblasColMajor, CblasNoTrans, Nr, Nc, &alpha, a,
        Nr, x, 1, &beta, y, 1);
        
    // 控制行分别输出 x, a, y
    for (int i = 0; i < Nc; ++i) {
        cout << x[i] << "  ";
    }
    cout << "\n" << endl;
    for (int i = 0; i < Nr; ++i) {
        for (int j = 0; j < Nc; ++j) {
            cout << a[i + Nr*j] << "  ";
        }
        cout << endl;
    }
    cout << "\n" << endl;
    for (int i = 0; i < Nr; ++i) {
        cout << y[i] << "  ";
    }
}
\end{lstlisting}

编译与运行结果
\begin{lstlisting}[language=bash]
$ g++ test_cblas.cpp -cblas
$ ./a.out
(1,2)  (2,3)  (3,4)

(1,2)  (3,4)  (5,6)
(2,3)  (4,5)  (6,7)

(-18,59)  (-21,74)
\end{lstlisting}

\subsubsection{其他参数}
\begin{itemize}
\item \verb|trans| (也是 \verb|enum|)指定预先转置矩阵(\verb|CblasTrans|)或不转置(\verb|CblasNoTrans|)
\item \verb|alpha, beta| 用于计算 \verb|y = alpha * a * x + beta y|, 上例中我们使用了 \verb|alpha = 1, beta = 0|
\item \verb|lda| 叫做 leading dimension, 在列主元的情况下用于指定从某一列第 $i$ 个元的索引(index)减去上一列第 $i$ 元的索引. 在上例中 \verb|lda| 就是行数. 但如果我们需要在一个更大的列(行)主元矩阵中截取一个小矩阵进行计算, 那么 \verb|lda| 就是大矩阵的行(列)数.
\item \verb|incx| 和 \verb|incy| 用于指定两个矢量的步长(increment), 由于上例中两矢量在内存中是连续的, 所以步长为 1. 但若某个矢量是行(列)主元矩阵中的一列(行), 那么步长就是矩阵的列(行)数.
\end{itemize}

\subsection{XBLAS}
\begin{itemize}
\item XBLAS 文档在\href{https://netlib.org/xblas/}{这里}.
\item XBLAS 的一个功能是函数接口不变, 但在内部使用四精度浮点.
\item 第二个功能是提供混合精度的函数接口.
\end{itemize}

\subsection{编译}
一般用 (C)BLAS 也需要用 LAPACK(E), 所以可以直接用 LAPACK 的源码编译, 详见 “Lapack 笔记\upref{Lapack}”.

\subsection{OpenBlas}
\begin{itemize}
\item openblas 是一个比较快的实现, 也包含 lapack.
\item 源码和文档都见 \href{https://github.com/xianyi/OpenBLAS}{github}.
\item Ubuntu 中 \verb|apt install openblas| 会安装 \verb|cblas.h|, 但没有 \verb|lapacke.h|, 自己重新编译会更好.
\item 注意 CMakeList.txt 只处于实验阶段, 最好用 automake
\item 自己编译的命令: \verb|cmake -DBUILD_STATIC_LIBS=OFF -DBUILD_TESTING=OFF -DCMAKE_BUILD_TYPE=Release 源码路径|
\item 其中 \verb|BUILD_STATIC_LIBS| 只编译静态链接库, \verb|BUILD_TESTING| 安装测试. 用 \verb|cmake -LH| 查看所有编译选项和说明.
\item 自己编译后安装的文件: \verb|/usr/local/lib/libopenblas.a| 以及 \verb|/usr/local/include/openblas/| 中的 \verb|cblas.h|, \verb|lapack.h|, \verb|lapacke.h|, \verb|lapacke_config.h|, \verb|lapacke_mangling.h|, \verb|lapacke_utils.h|, \verb|openblas_config.h|, \verb|f77blas.h|
\end{itemize}
