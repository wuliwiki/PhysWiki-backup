% 电磁场中的狄拉克方程
% keys 电磁场|狄拉克方程|狄拉克场
% license Xiao
% type Tutor

\pentry{狄拉克方程\nref{nod_qed4}, 狄拉克场\nref{nod_Dirac}}{nod_abba}

我们继续使用自然单位制,令 $\hbar=c=1$ 来简化表达,不过在\autoref{sub_DiracE_1} 中我们会恢复 $\hbar,c$ 的使用以研究非相对论极限下狄拉克方程的情形。约定度规为 $(1,-1,-1,-1)$。下文中电磁单位制采用的是量子场论中最常用的洛伦兹-亥维赛单位制(Lorentz–Heaviside Units)\footnote{参考 \href{https://en.wikipedia.org/wiki/Heaviside\%E2\%80\%93Lorentz_units}{Wikipedia}}。

\subsection{量子电动力学的拉格朗日量}
\pentry{电磁场中的薛定谔方程及规范变换\nref{nod_QMEM}}{nod_f2bb}
让我们从\enref{狄拉克场}{Dirac}的拉氏量出发:
\begin{equation}
\mathcal{L}=\bar\psi (i\gamma^\mu \partial_\mu - m)\psi ~.
\end{equation}
我们已经知道了该方程在整体 $U(1)$ 规范变换下是不变的,即 $\psi\rightarrow e^{iq\alpha}\psi,\bar\psi \rightarrow \bar\psi e^{-iq\alpha}$ 变换下拉氏量保持不变。根据诺特定理\footnote{可以参考经典场论基础\upref{classi}。},整体规范变换对称性导致狄拉克场的电荷守恒。现在我们希望进一步地引入\enref{定域规范不变性}。即 $\psi\rightarrow e^{iq\alpha(x)}\psi $,其中 $\alpha(x)$ 为时空坐标的函数。然而,引入这样的规范变换以后,拉氏量不再保持不变,会多出一项 $-\bar\psi \gamma^\mu \qty[q\partial_\mu \alpha(x)]\psi$。所以为了抵消掉这一项,我们约定 $\partial_\mu$ 在定域规范变换下变为 $\partial_\mu+i[q\partial_\mu \alpha(x)]$。

因此,我们引入一个协变导数算子 $D_\mu = \partial_\mu +iqA_\mu(x)$,其中 $A_\mu(x)$ 是一个矢量场,也称它为规范场。在规范变换下,$A_\mu\rightarrow A_\mu - \partial_\mu \alpha(x)$。于是定域规范不变的拉氏量可以重新写为
\begin{equation}
\mathcal{L}=\bar\psi(i\gamma^\mu D_\mu-m)\psi ~.
\end{equation}

因此,我们的理论中出现了两组广义坐标,一组是物质场 $\psi$ 和规范场 $A_\mu$,我们当然还可以引入规范场的拉氏量。注意到 $F_{\mu\nu} = \partial_\mu A_\nu - \partial_\nu A_\mu$ 也是定域规范不变的,而它又是一个二阶张量。因此可以利用 $F_{\mu\nu}$ 组成一个新的拉氏量:
\begin{equation}
\mathcal{L}=-\frac{1}{4}F_{\mu\nu}F^{\mu\nu} + \bar\psi (i\gamma^\mu D_\mu - m)\psi~.
\end{equation}
这个拉氏量生成的理论被称为\textbf{旋量量子电动力学}。注意到它比自由旋量场和自由电磁场的拉氏量还多出一项
\begin{equation}
\mathcal{L}_{\rm int}=-q A_\mu \bar\psi \gamma^\mu \psi=-A_\mu J^\mu~.
\end{equation}
其中 $J^\mu = q\bar\psi \gamma^\mu \psi$ 正是整体 $U(1)$ 规范对称性的守恒流,它对应的是电荷电流密度。而这一项对应了两个场耦合的相互作用拉氏量。根据上述的拉氏量所导出的两个经典方程为
\begin{equation}\label{eq_DiracE_3}
\begin{aligned}
&(i\gamma^\mu D_\mu - m)\psi(x)=0~,\\
&\partial^\mu F_{\mu\nu}=J_\nu~.
\end{aligned}
\end{equation}
我们还可以将上述拉氏量和经典电动力学中单粒子的拉氏量作比较(参考\enref{电磁场的作用量}{ElecS}),除了自由粒子作用量部分不同之外,两者的形式是相似的。此外注意我们在电磁场的作用量\upref{ElecS}文章中采取的单位制是高斯单位制,在那里的电磁场拉氏量与此处的电磁场拉氏量有一个 $4\pi$ 因子的差异,这意味着我们在这里定义的旋量量子电动力学中采取的是一个新的单位制,也被称为 \textbf{洛伦兹-亥维赛单位制 (Lorentz–Heaviside Units)}。与高斯单位制不同的是,洛伦兹-亥维赛单位制将 $4\pi$ 吸收进电荷的定义中,因此\autoref{eq_DiracE_3} 中方程 $\partial^\mu F_{\mu\nu}=J_\nu$ 也不再有 $4\pi$ 的因子。此外要注意的是,与\enref{电磁场的作用量z{ElecS}文章不同的是这里采取的度规是 $(1,-1,-1,-1)$,因此拉氏量的符号上会有一定差异。
\subsection{非相对论极限下电磁场中的狄拉克方程}\label{sub_DiracE_1}
我们常常需要讨论一个缓变的电磁场中电子的行为。将 $A_\mu$ 视为一个缓变的外场,那么电子的波函数满足方程
\begin{equation}\label{eq_DiracE_4}
(i\gamma^\mu \partial_\mu -q\gamma^\mu A_\mu - m)\psi(x)=0~.
\end{equation}
利用 $\gamma$ 矩阵的 weyl 手性表示\autoref{eq_Dirac_9}~\upref{Dirac},并且设 $\psi = \pmat{\psi_L\\\psi_R}$,可以将上述方程改写为
\begin{equation}
\begin{aligned}
&(i\partial_0 - q\phi-\bvec \sigma\cdot (\bvec P-q\bvec A))\psi_R=m\psi_L~,\\
&(i\partial_0-q\phi+\bvec \sigma\cdot (\bvec P-q\bvec A))\psi_L = m\psi_R~,
\end{aligned}
\end{equation}
其中将 $A_\mu$ 拆成了 $(\phi,-\bvec A)^T$。或者也可以改用 Pauli-Dirac 表象下(从 \autoref{eq_qed4_6}~\upref{qed4} 出发得到 $\gamma$ 矩阵)。与 weyl 表示不同的是,这里的 $\gamma^0$ 为 $\pmat{&I&0\\&0&-I}$。设 Pauli-Dirac 表象下 $\psi=\pmat{\varphi\\ \chi}$,则
\begin{equation}\label{eq_DiracE_1}
\begin{aligned}
&(i\partial_0-q\phi-m)\varphi-\bvec \sigma\cdot (\bvec P-q\bvec A)\chi=0~,\\
&(i\partial_0-q\phi+m)\chi-\bvec \sigma\cdot (\bvec P-q\bvec A)\varphi=0~.
\end{aligned}
\end{equation}
两个表示下的电子波函数可以通过以下变换联系:
\begin{equation}
\varphi = \frac{1}{\sqrt{2}}(\psi_L+\psi_R),\chi = \frac{1}{\sqrt{2}}(\psi_R-\psi_L)~.
\end{equation}

为了求非相对论极限下的波函数,我们令 $\varphi\rightarrow \varphi e^{-imt},\chi\rightarrow \chi e^{-imt}$,对 \autoref{eq_DiracE_1} 化简可以得到
\begin{equation}\label{eq_DiracE_5}
\begin{aligned}
&(i\partial_0 - q \phi)\varphi = \bvec \sigma\cdot (\bvec P-q\bvec A) \chi~,\\
&(i\partial_0 - q \phi + 2m) \chi = \bvec \sigma\cdot (\bvec P-q\bvec A)  \varphi~,
\end{aligned}
\end{equation}
对比\autoref{eq_DiracB_1}~\upref{DiracB}可以发现两者的相似之处。在非自然单位制下,上式可以改写为
\begin{equation}\label{eq_DiracE_2}
\begin{aligned}
&i\hbar \pdv{t} \varphi = c\bvec \sigma\cdot \qty(\bvec P-\frac{q}{c}\bvec A)\chi + q \phi \varphi~,\\
&i\hbar\pdv{t} \chi = c \bvec \sigma\cdot \qty(\bvec P-\frac{q}{c}\bvec A) \varphi + (q \phi-2mc^2) \chi~.
\end{aligned}
\end{equation}
在非相对论极限下,上面第二行中不带 $c$ 的项可以略去,于是可以得到
\begin{equation}
\chi\approx \frac{1}{2mc} \bvec \sigma \cdot(\bvec P-\frac{q}{c}\bvec A)\varphi~.
\end{equation}
代入\autoref{eq_DiracE_2} 的第一行就得到了:
\begin{equation}
i\hbar\pdv{t} \varphi = \frac{1}{2m}\qty[\bvec \sigma \cdot\qty(\bvec P - \frac{q}{c}\bvec A)]^2 \varphi + q\phi  \varphi~.
\end{equation}
利用 $\bvec \sigma$ 的恒等式,经过一系列的推导(可以参考泡利方程\upref{Pauli}),最后可以得到\textbf{泡利方程}。
\begin{equation}
i\hbar\pdv{t} \varphi = \qty[\frac{(\bvec P - (q/c)\bvec A)^2}{2m}+e\phi-\frac{q}{mc}\bvec S\cdot \bvec B]\varphi~,
\end{equation}
其中 $\bvec S=(\hbar/2)\bvec \sigma$。
