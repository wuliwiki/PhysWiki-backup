% 阶与原根
% keys 阶|原根
% license Usr
% type Tutor

\pentry{费马小定理与欧拉定理\nref{nod_fermse},线性同余\nref{nod_linmod}}{nod_856e}
\begin{lemma}{}
对于命题 $P(n)$ 而言,若对于每个 $a$, $b$ 都有 $P(a)$ 与 $P(b)$ 蕴含 $P(a+b)$ 且至少在 $b \le a$ 时蕴含 $P(a-b)$,而 $r$ 是使 $P$ 成立的最小正正数,则:
\begin{enumerate}
\item 对于各个自然数 $k = 1, 2, \dots$,$P(kr)$ 也为真。
\item 任何使得 $P(m)$ 成立的非负整数 $m$ 都是 $r$ 的倍数。
\end{enumerate}
\end{lemma}
\textbf{证明}:对于第一点是显然的,由于 $P(r)$ 与 $P(r)$ 都为真,故 $P(r+r)$ 为真......

现在考虑第二点。根据 $r$ 的定义,$0 < r \le m$,记 $m = qr + s$,从而 $s = m - qr$,其中 $0 \le s < r$ 而 $q \ge 1$。根据第一点,$P(r) \rightarrow P(kr)$,而 $P(a)$ 与 $P(b)$ 蕴含 $P(a \pm b)$,故 $P(m)$ 与 $P(qr)$ 可以推导出 $P(s)$。而由于 $r$ 的定义,$s$ 必定为 $0$。故 $q = kr$。证毕!

