% 费米-狄拉克统计(综述)
% license CCBYSA3
% type Wiki

本文根据 CC-BY-SA 协议转载翻译自维基百科\href{https://en.wikipedia.org/wiki/Fermi\%E2\%80\%93Dirac_statistics}{相关文章}。

费米–狄拉克统计是一种适用于由许多不相互作用的相同粒子组成的系统的量子统计,这些粒子遵循泡利不相容原理。其结果是费米–狄拉克分布,描述了粒子在不同能量态上的分布。该统计方式以恩里科·费米和保罗·狄拉克命名,他们分别在 1926 年独立推导出了这一分布。\(^\text{[1][2]}\)费米–狄拉克统计属于统计力学的范畴,并且基于量子力学的基本原理。

费米–狄拉克统计适用于具有半整数自旋(\(1/2\)、\(3/2\) 等)的相同且不可区分的粒子,这些粒子被称为费米子,并处于热力学平衡状态。当粒子之间的相互作用可以忽略时,该系统可以用单粒子能级来描述。其结果是粒子在这些能级上的费米–狄拉克分布,其中任何两个粒子都不能占据相同的状态,这对系统的性质产生了重要影响。费米–狄拉克统计最常应用于电子,电子是一种自旋为\(1/2\)的费米子。

费米–狄拉克统计的对应理论是玻色–爱因斯坦统计,它适用于具有整数自旋(0、1、2 等)的相同且不可区分的粒子,这些粒子被称为玻色子。在经典物理中,麦克斯韦–玻尔兹曼统计用于描述相同但可区分的粒子。与费米–狄拉克统计不同,在玻色–爱因斯坦统计和麦克斯韦–玻尔兹曼统计中,多个粒子可以占据相同的量子态。
\subsection{历史}
\begin{figure}[ht]
\centering
\includegraphics[width=10cm]{./figures/ad1c8504edbe0ae0.png}
\caption{整数自旋粒子(玻色子,红色)、半整数自旋粒子(费米子,蓝色)以及经典(无自旋)粒子(绿色)的平衡热分布。图中显示了平均占据数\( \langle n \rangle \) 随能量\( \epsilon \)的变化情况,能量相对于系统的化学势\( \mu \) 而表示,其中\( T \) 为系统温度,\( k_B \) 为玻尔兹曼常数。} \label{fig_FermiD_1}
\end{figure}
在1926年费米–狄拉克统计被引入之前,理解电子行为的一些方面十分困难,因为存在看似矛盾的现象。例如,在室温下,金属的电子比热容似乎仅由比导电电子数目少100倍的电子贡献。\(^\text{[3]}\)此外,也难以解释为何在室温下对金属施加高电场所产生的发射电流几乎不依赖于温度。

当时的德鲁德模型,即金属的电子理论,之所以遇到困难,是因为它假设**所有电子在经典统计理论下是等效的。换句话说,认为每个电子对比热的贡献约为**玻尔兹曼常数\(k_B\)的量级。然而,这一问题直到费米–狄拉克统计**的发展才得以解决。  

费米–狄拉克统计首次由恩里科·费米\(^\text{[1]}\)和保罗·狄拉克\(^\text{[2]}\)于1926年发表。据马克斯·玻恩所述,帕斯库尔·约当在 1925年也提出了相同的统计方法,并将其称为“泡利统计”,但未能及时发表。\(^\text{[4][5][6]}\)据狄拉克所说,该统计方法最早是由费米研究的,因此他称其为“费米统计”,并将遵循该统计的粒子称为“费米子”\(^\text{[7]}\)  

1926年,拉尔夫·福勒率先将费米–狄拉克统计应用于描述恒星塌缩成白矮星的过程。\(^\text{[8]}\)1927年,阿诺德·索末菲将其应用于金属中的电子,并发展了自由电子模型。\(^\text{[9]}\)1928年,福勒和洛塔·诺德海姆又将其应用于金属的场电子发射。\(^\text{[10]}\)费米–狄拉克统计仍然是物理学的重要组成部分。
\subsection{费米–狄拉克分布}  
对于一个由相同费米子组成并处于热力学平衡的系统,单粒子态\( i \)上的平均费米子数由费米–狄拉克(F–D)分布给出:\(^\text{[[11][nb 1]]}\)   
\[
\bar{n}_{i} = \frac{1}{e^{(\varepsilon_{i} - \mu)/k_{\text{B}}T} + 1}~
\]
其中\(k_B\)为玻尔兹曼常数,\(T\)为绝对温度,\(\varepsilon_i\)为 单粒子态\(i\)的能量,\(\mu\)为总化学势。该分布由归一化条件
\[\sum_{i} \bar{n}_{i} = N~\]
约束,该条件可用于表示化学势\(\mu\)作为温度\(T\)和粒子数\(N\)的函数:\(\mu = \mu (T, N)\)其中\(\mu\)可能取正值或负值。\(^\text{[12]}\)

在绝对零温度下,化学势\(\mu\)等于费米能量加上每个费米子的势能,前提是该区域具有正的谱密度。在存在谱间隙(例如半导体中的电子)的情况下,对称点\(\mu\)通常被称为费米能级,对于电子而言,也被称为电化学势,它通常位于能隙的中心。\(^\text{[13][14]}\)

费米–狄拉克分布仅在系统中的费米子数足够大的情况下才有效,这样额外添加一个费米子对化学势\(\mu\)的影响可忽略不计。[15] 由于费米–狄拉克分布的推导基于泡利不相容原理,该原理规定每个可能的量子态至多只能被一个费米子占据,因此分布函数的值始终满足\(0 < \bar{n}_i < 1\)\(^\text{[nb 2]}\)

\begin{figure}[ht]
\centering
\includegraphics[width=8cm]{./figures/f19ab17e25000c3c.png}
\caption{能量依赖性。在较高温度下更加渐进。当 \( \varepsilon = \mu \) 时,\( \bar{n} = 0.5 \)。未显示的是,化学势\( \mu \) 随着温度 \( T \) 的升高而减小。\(^\text{[16]}\))} \label{fig_FermiD_2}
\end{figure}
\begin{figure}[ht]
\centering
\includegraphics[width=8cm]{./figures/674bc0e94481de69.png}
\caption{温度依赖性,当 \( \varepsilon > \mu \) 时。} \label{fig_FermiD_3}
\end{figure}
状态\(i\)中粒子数的方差可以通过以下公式从\(\bar{n}_i\)的表达式计算得到:\(^\text{[17][18]}\)
\[V(n_{i}) = k_{\text{B}}T \frac{\partial}{\partial \mu} \bar{n}_{i} = \bar{n}_{i} (1 - \bar{n}_{i}).~\]
\begin{figure}[ht]
\centering
\includegraphics[width=10cm]{./figures/ab2ba84b1ada0c40.png}
\caption{} \label{fig_FermiD_4}
\end{figure}
\subsubsection{粒子在能量上的分布}
从费米–狄拉克分布中,可以求出粒子在能量上的分布。\(^\text{[nb 3]}\)具有能量\( \varepsilon_i\) 的费米子的平均数目可以通过将费米–狄拉克分布\(\bar{n}_i\)与简并度\(g_i\)(即具有能量\(\varepsilon_i\)的态的数目)相乘得到:\(^\text{[19]}\) 
\[
\bar{n}(\varepsilon_i) = g_i \bar{n}_i = \frac{g_i}{e^{(\varepsilon_i - \mu) / k_B T} + 1}.~
\]
当\( g_i \geq 2 \)时,可能会出现\( \bar{n}(\varepsilon_i) > 1 \),因为存在多个能量为\( \varepsilon_i \)的状态可以被费米子占据。

当一个具有准连续能量\(\varepsilon\)的能量区间有一个相关的态密度\( g(\varepsilon)\)(即单位能量范围单位体积内的状态数目)\(^\text{[20]}\),则单位能量范围单位体积内的费米子平均数目为:
\[
\bar{\mathcal{N}}(\varepsilon) = g(\varepsilon) F(\varepsilon),~
\]
其中\(F(\varepsilon)\)被称为费米函数,它与费米–狄拉克分布\(\bar{n}_i\)中使用的函数相同:\(^\text{[21]}\)
\[
F(\varepsilon) = \frac{1}{e^{(\varepsilon - \mu) / k_B T} + 1},~
\]
因此:
\[
\bar{\mathcal{N}}(\varepsilon) = \frac{g(\varepsilon)}{e^{(\varepsilon - \mu) / k_B T} + 1}.~
\]
\subsection{量子与经典范畴}
在高温和低粒子密度的极限下,**费米–狄拉克分布**趋近于**麦克斯韦–玻尔兹曼分布**,无需任何临时假设:

在低粒子密度的极限下,\(\bar{n}_i = \frac{1}{e^{(\varepsilon_i - \mu) / k_B T} + 1} \ll 1\)因此,\(e^{(\varepsilon_i - \mu) / k_B T} + 1 \gg 1 \quad \text{或等效地} \quad e^{(\varepsilon_i - \mu) / k_B T} \gg 1\).在这种情况下,\(\bar{n}_i \approx \frac{1}{e^{(\varepsilon_i - \mu) / k_B T}} = \frac{N}{Z} e^{-\varepsilon_i / k_B T}\),这就是\textbf{麦克斯韦-玻尔兹曼统计}的结果。

在高温的极限下,粒子分布在一个较大的能量范围内,因此每个态的占据数(尤其是那些满足 \( \varepsilon_i - \mu \gg k_B T \) 的高能态)再次变得非常小,\[\bar{n}_i = \frac{1}{e^{(\varepsilon_i - \mu) / k_B T} + 1} \ll 1\.

这再次简化为**麦克斯韦-玻尔兹曼统计**。