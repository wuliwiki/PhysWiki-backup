% 实数

\subsection{从有理数到实数}

我们知道, 有理数集$\mathbb{Q}$是对四则运算封闭的最小的数系. 从正整数开始, 为了使得任意两个整数都能相减, 我们引入了零和负整数, 从而得到了整数集$\mathbb{Z}$; 而为了使得任意两个整数的除法都有意义 (当然, 要剔去除数为零的情形), 我们又引入了形如$m/n$的数, 从而得到了有理数集$\mathbb{Q}$. 有理数的英文 rational number 即来源于ratio (比例) 一词. 小学算术已经告诉我们, 有理数的和, 差, 积, 商都是有理数 (仍然需要假定除数不等于零), 而且对于任何有理数$r$都有$r+0=r$, $r\cdot1=r$. 用近代代数学的语言, 这表示有理数集构成了一个\textbf{域 (field)}.

但是我们也知道, 并非所有来自实际问题的度量对象都能用有理数来表示. 例如, 假若承认勾股定理 (在古希腊, 发现并证明它的是毕达哥拉斯), 那么直角边长为1的等腰直角三角形的斜边长$c$满足$c^2=2$. 毕达哥拉斯的门徒西帕索斯发现, 这个奇特的数$c$不能表示为两个整数的比. 西帕索斯的发现打击了毕达哥拉斯学派的信条"万物皆 (有理) 数", 因而被试图维护教义的门徒们杀害. 不过, 随着越来越多类似的例子出现, 到了十世纪左右, 数学家们已经承认了这些不是有理数的平方根是同有理数一样真实的对象.

\begin{exercise}{$\sqrt{2}$是无理数}
利用数论中的素因子分解定理 (每个正整数都可以唯一分解成它的素因子乘积; 这件事并不是显然的), 证明不存在整数$m,n$使得$m^2=2n^2$. 更一般地, 如果$p$是素数, 那么不存在整数$m,n,k>1$使得$m^k=pn^k$.
\end{exercise}

\subsection{戴德金分割}
我们按照戴德金的方法来定义实数. 按照这种办法, 一个实数被定义为有理数集合的一个分割.

\begin{definition}{戴德金分割}
一个\textbf{戴德金分割 (Dedekind cut)} 或一个实数被定义为将有理数集分划成两个部分的一种方式$\mathbb{Q}=L\cup R$, 其中不交的子集$L$和$R$满足

\begin{enumerate}
\item 如果$l\in L$, 那么任何小于$l$的有理数$l'$都属于$L$.
\item 如果$r\in R$, 那么任何大于$r$的有理数$r'$都属于$r$.
\item 如果$l\in L$, $r\in R$, 那么必有$l\leq r$.
\item $L$不包含最大的元素.
\end{enumerate}

这里的$L$被称为分割$L\cup R$的下类 (lower class), $R$被称作上类 (upper class).
\end{definition}

乍看起来, 这种定义方式很违背直觉. 然而我们若以$\sqrt{2}$作为例子, 即可看出这个定义的确能够填补有理数之间的空隙. 定义$sqrt{2}$的下类$L$为满足$l\leq0$或者$l^2<2$的有理数$l$的集合, 上类$R$为满足$r^2>2$的正有理数的集合. 于是对于$l\in L$, $r\in R$, 当然有$l<r$. 由于没有有理数能满足$l^2=2$, $L$