% 浮力 阿基米德原理
% 重力|液体|密度|浮力|等效法

\subsection{等效法}

\footnote{本文参考 Wikipedia \href{https://en.wikipedia.org/wiki/Archimedes'_principle}{相关页面}.}我们先用一个简单易懂的方式解释浮力. 假设在重力加速度为 $g$ 的环境中, 容器中密度为 $\rho_0$ 的液体完全静止. 这时令液体内部有一任意形状的闭合曲面, 体积为 $V_0$. 把曲面内部的液体作为一个整体做受力分析, 其质量为 $m = \rho_0 V_0$, 所受重力为 $mg = \rho_0 V_0 g$. 由于曲面中液体保持静止, 说明曲面外的液体对曲面内的液体施加了相同大小的浮力. 现在我们如果把曲面内的液体替换为一块密度为 $\rho$ 的物体, 由于曲面形状不改变, 外界液体对该物体的浮力仍然为
\begin{equation}
F = \rho_0 V_0 g
\end{equation}
注意 $V_0$ 为物体在水中部分的体积, 如果物体只有部分在水中, $V_0$ 将小于物体的体积.

\subsection{散度法}
\pentry{牛顿—莱布尼兹公式的高维拓展\upref{NLext}}

现在我们用面积分的方法表示浮力. 令 $z$ 轴竖直向上, 且水面处 $z = 0$, 则水面下压强为
\begin{equation}
P = -\rho_0 g z
\end{equation}
现在把上述的闭合曲面划分为许多个微面元, 第 $i$ 个面元用矢量 $\Delta \bvec s_i$, 表示, 其中模长为面元的面积, 方向为从内向外的法向. 这个面元受到外界液体的压力为
\begin{equation}
\Delta \bvec F_i = -P\Delta \bvec s_i = \rho_0 g z \Delta \bvec s_i
\end{equation}
现在把所有面元所受的压力求和, 并用曲面积分\upref{SurInt}表示为
\begin{equation}
\bvec F = \oint \rho_0 g z \dd{\bvec s}
\end{equation}
这就是物体所受的浮力. 使用\autoref{NLext_eq3}~\upref{NLext} 得
\begin{equation}
\bvec F = \int \grad(\rho_0 g z) \dd{V} = \rho_0 g V_0 \uvec z
\end{equation}
可见该结论与“等效法”中得出的一致.
