% 上海海事大学 2012 年数据结构
% keys 上海海事大学 2012 年数据结构

\subsection{一.判断题(本题20分, 每小题2分)}

1.线性的数据结构可以顺序存储,也可以链接存储;非线性的数据结构只能链接存储.

2. B+树中所有叶子结点都处在同一层次上,且每个叶子结点中关键字个数都是相等的.

3.单链表从任何一个结点出发,都能访问到所有结点.

4.排序的目的就是要将一组无序的记录序列按从小到大的顺序调整.

5.存储在顺序存储器上的顺序文件不能进行折半查找.

6.网络的最小代价生成树是唯一的.

7.磁带是顺序存取的外存储设备.

8.将一棵树转换成二叉树后,二叉树的根结点一定没有右子树.

9.所有叶子结点都处于同一层的二叉树一定是完全二叉树.

10.在AOE网中,任何一个关键活动提前完成,都将使整个工程提前完成.

\subsection{二、填空题(本题30分,每空2分) .}

1.分析下列程序段,其时间复杂度分别为:_ (1) ._ (2)_
\begin{lstlisting}[language=cpp]
i=1;
while(i<=n*n)
    i=i*2;

int i=0, s=0;
while (s<n*n){
    i++;
    s=s+i;
}
\end{lstlisting}

2.顺序表.栈和队列都是( (3) )结构, 顺序表可以在其( (4) )位置插入和删除元素;对于栈只能在( (5) )插入和删除元素; 对于队列只能在( (6) )插入元素和另一端删除元素.

3.广义表A=(a,b,(c ,d),(e,(f, g)))的长度是( (7) ).深度是( (8) ),取表头和表尾函数分别为head()和ail(),则head (head (tail (tail (A))))=( (9) ),而从表中取出原子项d的运算为( (10) ).

4.有一个二维数组A[1..6][0..7].每个数组元素占用6个存储单元,并且a[3][4]的存储地址为1280,若按行序为主序方式存储,数组元素A[2][3]的存储地址是( (11) );若按列序为主序方式存储,数组元素A[2][3]的存储地址是( (12) ).

5.在堆排序、快速排序和归并排序三种算法中,若仅从存储空间考虑,则应首先选取( (13) )方法:若只从平均情况下排序最快考虑.则应选取( (14) )方法;若只从排序结果的稳定性考虑,则应选取( (15) )方法.

\subsection{三、选择题(本题20分,每空2分)}

1. 算法分析的两个主要方面是( ). \\
A)时间复杂度和空间复杂度 $\qquad$ B)正确性和简单性 \\
C)可读性和健壮性 $\qquad$ D)数据复杂性和程序复杂性

2.线性表若采用顺序存储结构时,要求内存中可用存储单元的地址( ). \\
A)必须是连续的 $\qquad$ B)部分地址必须是连续的 \\
C)一定是不连续的 $\qquad$ D)连续不连续都可以

3.在一个单链表中,已知q所指结点是p所指结点的前驱结点,若在q和p之间插入S点,则执行( ). \\
A) p->next=S; S->nexl=q $\qquad$ B) S->next=p->next ; p->next=S \\
C) q->next=S; S->next=p $\qquad$ D) q->next=S->*next ; S->next=p

4.如果一个栈的进栈序列是ABCD(即,A先进栈,然后B、C和D依次进栈),,允许在进栈过程中可以退栈,且规定每个元素进栈和退栈各一次,那么不可能得到的退栈序列是( ) \\
A) DCBA $\qquad$ B) ACBD $\qquad$ C) CDBA $\qquad$ D) DBAC

5. 如果G是一个具有n(n>1)个顶点的连通无向图,T是G的一棵生成树,那么T有(、)条边.
A) n $\qquad$ B) n-1 $\qquad$ C) n+1 $\qquad$ D) n+2

6. 在对顺序表进行折半查找时,要求该表必须( ). \\
A)以顺序方式存储 \\
B)以链接方式存储 \\
C)以顺序方式存储,且结点按关键字有序排序 \\
D)以链接方式存储,且结点按关键字有序排序

7.一个有n个顶点的有向图最多有( )条边. \\
A) n $\qquad$ B) n(n-1) $\qquad$ C) n(n-1)/2 $\qquad$ D) 2n

8. 若某线性表中最常用的操作是在最后一个元素之后插入一个元素和删除第一个元素,则采用( ) 存储方式最节省运行时间. \\
A)单链表 $\qquad$ B)仅有头指针的单循环链表 \\
C)双向链表 $\qquad$ D)有尾指针的单循环链表

9.用一维数细Q[0..n-1]来存储一个循环队列,每个数组单元存放一 个队列元素,记f为队头指针,为队尾指针,都为数组下标,若空队列的初始状态f=τ=0,则计算队列中元素个数的公式(式中mod为取余)为( ). \\
A) r-f $\qquad$ B) (n+r-f) mod n $\qquad$ C) n-1 $\qquad$ D) n

10.若二叉树的叶子结点个数为n0.度数为2的结点个数为2,则n0=( ). \\
A) n2-1 $\qquad$ B) n2 $\qquad$ C) n2+1 $\qquad$ D) 2*n2

