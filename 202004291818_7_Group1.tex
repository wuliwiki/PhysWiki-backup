% 子群和正规子群
\pentry{群\upref{Group}}

\subsection{子群}

\begin{definition}{子群}

给定一个群$(G, \cdot)$,如果集合$G$有一个子集$H$,使得$e\in H$且$H$中的元素在运算$\cdot$下仍然封闭,那么显然\footnote{由于已经知道$(G,\cdot)$构成一个群了,群的四条公理中,结合性、单位元存在性以及逆元存在性都被满足了.}$(H,\cdot)$也构成一个群.称$H$是群$G$的\textbf{子群(subgroup)}.

\end{definition}

虽然群和子群的联系很紧密,但是我们通常还是把它们看作由完全无关的集合所构成的,只不过可以自然地应用已经存在的群运算和子集关系来定义子群的运算.这样,将已有的运算直接用在子集上,有时被称作在子集上\textbf{导出}或\textbf{诱导(induce)}了一个运算,有时也称子集上的运算是\textbf{限制在子集上的运算}.比如在定义里,群$H$的运算实际上被认为是和$G$的运算不一样,严格来说应该记为$\cdot|_H$,意思是“限制在$H$上的$\cdot$”. 但是不至于引起混淆的时候,也可以简单记为“$\cdot$”,并认为是同一个运算.

一般来说,当我们说$H$是一个子群时,强调的是$H$和$G$的关系;但如果我们说$H$是一个群,我们关心的是$H$本身作为群的性质,而没有强调它和其它群的关系.

\subsection{陪集}


群的运算有一个很棒的唯一性,算是我们目前遇到的第一个具体的结构性特征,由以下定理描述:

\begin{exercise}{群运算的唯一性}\label{Group1_exe1}
给定一个群$G$, 若 对于$ a, b\in G$ 有某个$x\in G$使得 $ax=bx$,那么必然有$a=b$;类似地,如果$xa=xb$,也必然有$a=b$.

\begin{itemize}
\item定理证明留作习题.提示:用$x^{-1}$去参与运算试试.
\end{itemize}
\end{exercise}

唯一性是所有群都有的性质,但是等到我们讨论环(ring)的时候,由于环的乘法不要求逆元一定存在,我们没法对环证明这个唯一性.事实上,很多环都没有唯一性.



\begin{definition}{左陪集}

给定一个群$G$和它的一个子群$H$.从$G$中任意挑一个元素$x$出来,用它来\textbf{左乘} $H$中的每一个元素,得到一个集合$\{x, xh_1, xh_2, \cdots\}$,这里的$h_n$要取遍每一个在$H$中的元素.这个集合,也可以写为$\{xh|h\in H\}$,被叫做\textbf{子群$H$关于元素$x$的左陪集(left coset)},记作$xH$. 

\end{definition}

类似地还可以定义右陪集,不过二者选其一作为代表来研究就可以了,我们习惯上主要研究左陪集.

注意这个表示方法:$xH=\{xh|h\in H\}$.这是一种非常简洁的表达方法,可以理解成$x$逐个左乘$H$中的元素得到的集合,也可以理解成是$x$和$H$之间的一个运算,结果是另一个集合.类似地,我们也可以用群里任意两个子集(不一定要求是子群)$A$和$B$来生成一个新的子集$AB=\{ab|a\in A, b\in B\}$,就是用$A$中的每一个元素去左乘$B$中的每一个元素,得到的所有结果的集合.

如果$h\in H$,而$H$是一个\textbf{有限群},那么由于封闭性,$hH=H$.这是因为,用$h$去左乘$H$中的一切元素(包括$h$自己),那么一方面由于封闭性,运算结果还是在$H$内部;另一方面由于群运算的唯一性,每一个左乘运算都不相同.这个论断不能简单地用于无限群.

特别地,考察这个形式的集合:$\{e, h, hh, hhh, hhhh, \cdots\}$,那么同样地由于封闭性和运算唯一性可知,这个集合还是群$H$的一个子集.特别地,$H$的群运算限制在这个集合上能构成一个循环群(\autoref{Group_ex2}\upref{Group}).只要我们把$n$个$h$相乘的结果记为$h^n$,$n$个$h^{-1}$相乘的结果记为$h^{-n}$,那么如此生成的循环群就可以用指数的加法运算来处理了.由单个元素可以生成循环群这一概念,我们引入以下定义:

\begin{definition}{阶}
给定一个群$G$和一个$g\in G$,如果存在一个整数$n$使得$g^n=e$,那么我们称$g$是一个有限阶的元素.特别地,所有满足条件的$n$中最小的那一个,被称为$g$的\textbf{阶(order)}.如果任何整数$n$都不能使$g^n=e$,那么我们称$g$的阶数为$\infty$.
\end{definition}

由于运算唯一性,有限群的元素都是有限阶的.

左陪集的意义是将群划分成互不相交的子集,这是一个等价类划分,也就是说,“$x$和$y$属于同一个左陪集”是一个等价关系\upref{Relat}.于是我们有了如下定理: 

\begin{theorem}{}

左陪集划分是一个等价类划分.

\end{theorem}

\textbf{证明}:

给定一个群$G$,两个元素$x, y\in G$,再给定它的一个子集$H$,那么“$x$和$y$属于$H$的同一个左陪集”的等价表述之一,可以是“$y\in xH$”\footnote{也等价于说“$x\in yH$”.};在这个属于关系两边同时左乘一个$x$\footnote{如果你不理解为什么可以像解方程一样两边同时乘以一个元素,请再琢磨琢磨上文中$xH$的定义是什么.},还能得到更常用的等价表述:$x^{-1}y\in H$.

我们用最后这个等价表述来检查,“在同一个左陪集中”这一关系,是否是等价关系.
\begin{itemize}
\item自反性:对于任意的$x\in G$,由于$e\in H$,故显然有$x=xe\in xH$.因此$x$和自己在同一个左陪集中.
\item对称性:如果$x^{-1}y\in H$,那么两边同时左乘以
\end{itemize}

%$y\in xH\iff e\in xHy^{-1}\iff \exists h_0$,使得$xh_0y=e\iff$\footnote{其中$xHy^{-1}=\{xhy|h\in H\}$.}


\subsection{正规子群}



