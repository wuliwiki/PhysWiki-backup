% 卷积
% 卷积

\textbf{卷积}(Convolution)是两个实变函数之间的运算,可以生成第三个函数。
设有两个函数$f(x)$和$g(x)$,$x \in R$,卷积后的结果函数为$s(t)$, $t \in R$。卷积运算通过定积分定义如下:
\begin{equation}
s(t) = (f*g)(t) = \int_{ - \infty }^{ + \infty } {f(x)g(t - x)dx}
\end{equation}
其中,星号*表示卷积运算。

从直观上说,卷积,顾名思义,就是相当于把一个函数水平翻转(即“卷”)之后,与另一个函数相乘。

上述是卷积的连续情况下的定义。在实际应用中,所测量到的物理量可能是离散的。可以将上式中的积分换成求和即可。卷积在离散情形下的定义如下:
\begin{equation}
s(t) = (f*g)(t) = \sum_{ x = - \infty }^{ + \infty } {f(x)g(t - x)}
\end{equation}

在深度学习中,卷积运算是卷积神经网络的基本操作。整个卷积网络是由大量卷积操作搭建而成。由于数据在计算机中均是离散形式,因此卷积神经网络中的卷积运算均是离散形式的操作。值得注意的是,真正的卷积神经网络中的卷积其实并非符合上述定义的真正的卷积,而是所谓\textbf{互相关函数}(Cross-correlation)。也就是说,实际网络中的卷积并没有翻转卷积核的操作。