% 线性泛函的几何意义
% keys 线性泛函|几何意义|超平面
% license Usr
% type Tutor

\pentry{余维数\nref{nod_Codim},泛函与线性泛函\nref{nod_Funal}}{nod_d70e}
在有限维线性空间中,一个线性方程和一个超曲面一一对应(\enref{线性方程组的仿射解释}{AS2LF}),这在wu'qiong。在任一的线性空间中,一个非平凡线性泛函(即不恒为零)和一个不通过坐标原点的超曲面一一对应。这便是线性泛函的几何意义。

\subsection{零子空间}
\begin{definition}{零子空间,核}
设 $f$ 是线性空间 $L$ 上不恒为零的线性泛函。则
\begin{equation}
\{x|f(x)=0,x\in L\}~,
\end{equation}
称为 $L$ 的(关于 $f$) 的\textbf{零子空间},
\end{definition}







