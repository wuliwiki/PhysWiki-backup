% 迈克尔·法拉第(综述)
% license CCBYSA3
% type Wiki

本文根据 CC-BY-SA 协议转载翻译自维基百科\href{https://en.wikipedia.org/wiki/Stokes\%27_theorem}{相关文章}。

迈克尔·法拉第(Michael Faraday,/ˈfærədeɪ, -di/;1791年9月22日-1867年8月25日)是一位英国物理学家和化学家,对电磁学和电化学的研究作出了重要贡献。他的主要发现包括电磁感应、抗磁性和电解等基本原理。尽管法拉第接受的正式教育很少,他作为一个自学成才的人,成为历史上最有影响力的科学家之一。[1] 通过研究直流电导体周围的磁场,法拉第在物理学中确立了电磁场的概念。他还发现了磁性可以影响光线,并揭示了两者之间的内在联系。[2][3] 他同样发现了电磁感应、抗磁性和电解定律的基本原理。他发明的电磁旋转装置奠定了电动机技术的基础,并且主要由于他的努力,使电的应用在技术中成为可能。[4]

作为一名化学家,法拉第发现了苯,研究了氯的包合水合物,发明了早期版本的本生灯以及氧化数系统,并推广了“阳极”、“阴极”、“电极”和“离子”等术语。最终,法拉第成为英国皇家学会的首任兼最重要的富勒讲座化学教授(Fullerian Professor of Chemistry),这一职位是终身制的。

法拉第是一位实验科学家,他以清晰简洁的语言表达自己的想法。他的数学能力并不高深,仅限于基础代数,甚至未涉猎三角学。詹姆斯·克拉克·麦克斯韦(James Clerk Maxwell)基于法拉第及其他人的工作,总结出一组方程,这些方程被认为是所有现代电磁现象理论的基础。关于法拉第使用“力线”的方法,麦克斯韦写道,这表明法拉第“实际上是一位非常高水平的数学家——未来的数学家可以从他那里获得宝贵而富有成果的方法。”[5] 国际单位制中电容的单位“法拉”(farad)就是以他的名字命名的。

阿尔伯特·爱因斯坦在书房的墙上挂着法拉第的画像,与艾萨克·牛顿和詹姆斯·克拉克·麦克斯韦的画像并列。[6] 物理学家欧内斯特·卢瑟福(Ernest Rutherford)曾说:“当我们考虑他的发现的广度与深度,以及它们对科学与工业发展的影响时,再大的荣誉也不足以表达对法拉第——这一有史以来最伟大的科学发现者之一——的敬意。”[1]
\subsection{传记}
\subsubsection{早年生活}
迈克尔·法拉第于1791年9月22日出生在萨里的纽因顿巴茨(Newington Butts),该地区现为伦敦南华克区的一部分。[7][8] 他的家境贫困。他的父亲詹姆斯·法拉第(James Faraday)是一个基督教格拉塞派(Glasite sect)的成员。1790年冬天,詹姆斯带着妻子玛格丽特(原姓哈斯特威尔,Margaret Hastwell)[9]和两个孩子从西摩兰郡的奥思吉尔(Outhgill)搬到伦敦。在奥思吉尔,詹姆斯曾是村庄铁匠的学徒。[10] 迈克尔是在次年秋天出生的,是家中四个孩子中的第三个。年幼的法拉第仅接受过最基本的学校教育,因此他不得不自学成才。[11]

14岁时,法拉第成为布兰德福德街(Blandford Street)当地书籍装订匠兼书商乔治·里博(George Riebau)的学徒。[12] 在七年的学徒期内,法拉第阅读了许多书籍,包括艾萨克·沃茨(Isaac Watts)的《提升心智》(*The Improvement of the Mind*)。他充满热情地实践了书中提出的原则和建议。[13] 在此期间,法拉第与同伴们在“城市哲学会”(City Philosophical Society)中进行讨论,并参加了关于各种科学主题的讲座。[14] 他对科学,特别是对电学产生了浓厚的兴趣。法拉第尤其受到简·马西特(Jane Marcet)所著的《化学对话》(*Conversations on Chemistry*)一书的启发。[15][16]