% 理查德·费曼
% license CCBYSA3
% type Wiki

(本文根据 CC-BY-SA 协议转载自原搜狗科学百科对英文维基百科的翻译)

理查德·菲利普斯·费曼(1918年5月11日-1988年2月15日),美国理论物理学家,以在量子力学的路径积分表述、量子电动力学理论、液体过冷氦的超流态物理学以及在粒子物理学领域提出的部分子模型等工作而闻名。由于对量子电动力学发展的贡献,费曼于1965年与朱利安·施温格和朝永振一郎共同获得诺贝尔物理学奖。

费曼为描述亚原子粒子行为的数学表达式开发了被广泛使用的图形表示方案,即后来的费曼图。在他一生中,费曼是世界上最著名的科学家之一。英国杂志 《物理世界》(Physics World )在1999年对全球130名顶尖物理学家进行了一项民意调查,费曼被列为有史以来十大物理学家之一。[1]

费曼在第二次世界大战期间协助设计了原子弹,并在1980年代作为罗杰斯委员会(调查航天飞机“挑战者号”事故的专家小组)的成员而广为人知 。随着在理论物理方面的工作,费曼被认为是量子计算领域的先驱,他还提出了纳米技术的概念。费曼在加州理工学院获得了理查德·C·托尔曼理论物理学教授职位。

费曼通过出书和讲座,包括1959年关于自上而下纳米技术,名为 There's Plenty of Room at the Bottom(底部有足够的空间)的演讲,以及他的三卷大学本科讲义《费曼物理学讲义》而成为物理学的热心普及者。除此之外,费曼还通过他的半自传体书籍《别闹了,费曼先生! 》、《你在乎别人怎么想吗? 》以及关于他的书籍如拉尔夫·莱顿的《 要么是图瓦要么是半身像!》 和詹姆斯·格雷克为他写的传记《天才:理查德·费曼的生活和科学 》而为人们所知。

\subsection{早年生活}
费曼于1918年5月11日出生在纽约皇后区,[2]母亲露西尔·菲利普斯是一位家庭主妇,父亲梅尔维尔·阿瑟·费曼[3]是一位来自白俄罗斯明斯克[4] (当时是俄罗斯帝国的一部分)的销售经理,两人都是立陶宛犹太人。[5] 费曼说话很晚,直到三岁生日后才学会说话。成年后,他说话还带有纽约口音[6][7] 以至于被认为十分做作和夸张[8][9]——这使得他的朋友沃尔夫冈·泡利和汉斯·贝特曾评论说费曼说话像个“流浪汉”。[8] 年轻的费曼深受父亲的影响,父亲鼓励他提出问题来挑战正统思想,并且总是准备教费曼一些新东西。而从母亲那里,他则获得了贯穿其一生的幽默感。费曼在孩童时代就很有工程天赋,他在家里有一个实验室,他还很喜欢修理收音机。费曼小学期间,在父母外出办事的时候,他还发明了一个家庭防盗报警系统。[10]

费曼五岁时,他的母亲生了弟弟亨利·菲利普斯,但仅存活了四周就夭折了。[11] 四年后,理查德的妹妹琼出生,一家人搬到了皇后区的法洛克威。[3] 尽管相差九岁,琼和费曼还是很亲近,他们都共同对这个世界有着好奇心。尽管母亲认为女性缺乏对天文学的理解能力,费曼还是鼓励琼对天文学的兴趣,琼最终成为了一名天体物理学家。[12]

\subsubsection{宗教}
费曼的父母没有宗教信仰,年轻时的费曼形容自己是“公开的无神论者”。[13]许多年后,在给蒂娜·莱维特的一封信中,他拒绝了蒂娜为写关于犹太人诺贝尔奖获得者的书而向他咨询信息的请求,他在信中表明,“为了认可来自某些所谓犹太人遗传性的特殊因素,要为种族理论上的各种谬论打开大门”,以及“13岁时,我不仅转向别的宗教观点,而且也不再相信犹太人在任何方面是‘天选之子’”。[14]晚年,在参观犹太神学院时,费曼第一次遇到了《塔木德》(犹太古法典),并评论该书包含了一种中世纪的思辨,是一本很好的书。[15]

\subsection{教育经历}
费曼就读于皇后区法洛克威的一所学校——法洛克威高中,和他同为诺贝尔奖获得者的伯顿·里克特和巴鲁克·塞缪尔·布隆伯格也曾就读于该校。 从高中开始,费曼很快被提拔到一个高等数学班。 根据其传记作者詹姆斯·格雷克的说法,一项由高中管理的智商测试估计他的智商为125——很高,但“还算不错”。 费曼的妹妹琼在这方面表现得更好,因此她声称自己比费曼更聪明。几年后,费曼拒绝加入门萨国际俱乐部,自称智商太低。 物理学家史蒂夫·许对这项测试的描述如下:

我怀疑这个测试强调的是语言能力,而不是数学能力。费曼在臭名昭著的普特南数学竞赛考试中获得了美国最高分......他在普林斯顿大学的数学/物理研究生入学考试中也获得了最高分......费曼的认知能力可能有点不平衡......我记得看过费曼在大学时保存的笔记本摘录......里面有许多拼写错误和语法错误。我怀疑费曼对这些事情其实十分关心。

费曼15岁时自学了三角学、高等代数、无穷级数、解析几何以及微积分。[16] 在进入大学之前,他已经在用自己的记数法实验并推导出数学课题,如半导数。[17] 他为对数、正弦、余弦和正切函数创造了特殊的符号,使其看起来不像三个变量相乘,他还为导数创造了特殊的符号。[18][19] 作为Arista荣誉协会的成员,费曼在高中的最后一年获得了纽约大学数学冠军。[20] 他直接描述事物的习惯有时会让更传统的思想家感到困惑,例如在学习猫科动物解剖学时,他的问道:“你有猫的地图吗?”(指解剖图)。[21]

费曼申请了哥伦比亚大学,但由于学校犹太人录取名额的限制而未被录取。这反而使他进入了麻省理工学院并在那里加入了Pi Lambda Phi 同好会。 尽管费曼最初主修数学,但后来却转到了电气工程专业,因为他认为数学太抽象了。由于觉得自己“走得太远”,他后来又投身物理,并声称物理“介于两者之间”。 费曼在大学本科期间就在《物理评论》杂志上发表了两篇论文。其中一篇是与曼努埃尔·瓦拉塔合作的,题为《银河系恒星的宇宙射线散射》。

瓦拉塔让他的学生(费曼)了解到师生共同发表论文的一个潜规则:这位资深科学家导师的名字必须排在第一位。几年后,费因曼对此进行了报复,海森堡在自己关于宇宙射线的书的结尾中说道:“根据瓦拉塔和费因曼的说法,这种影响很难预料。”当师生二人再次见面时,费曼高兴地问瓦拉塔是否看过海森堡的书。瓦拉塔知道费曼为什么对此发笑。“是的,”他回答。“你是宇宙射线中的最后一个词。”

另一篇是费曼的高级论文,关于“分子中的力”,[22] 该论文的发表是源于约翰·斯莱特的一个想法,他对这篇论文印象深刻,故将其发表。今天,它被称为“海尔曼-费曼定理”。[23]

1939年,费曼获得了学士学位,[24] 并获得普特南研究员的职称。[25] 他在普林斯顿大学的物理研究生入学考试中获得了满分,这是前所未有的壮举。同时他在数学方面也取得了优异的成绩,但历史和英语部分表现不佳。当时普林斯顿的物理系系主任亨利·史密斯有一个忧虑,他写信给菲利普·莫尔斯问:“费曼是犹太人吗?我们对犹太人没有明确的规定,但由于安置犹太人的困难性,我们不得不将他们在我们系的比例保持在很小的范围内。”[26] 莫尔斯承认费曼的确是犹太人,但让史密斯放心,费曼的“外貌和举止并没有显示出这种特征”。[26]

费曼参加的第一次研讨会是关于经典的惠勒-费曼吸收理论,与会者包括阿尔伯特·爱因斯坦、沃尔夫冈·泡利和约翰·冯·诺依曼。泡利预言这个理论极难量化,爱因斯坦认为人们可以尝试将这种方法应用于广义相对论中的引力。基于此,弗雷德·霍伊尔和贾扬·纳利卡尔展开了大量的研究,形成了后来的霍伊尔-纳利卡尔引力理论。 费曼于1942年在普林斯顿获得博士学位,他的论文导师是约翰·阿奇博尔德·惠勒。费曼的博士论文题为“量子力学中的最小作用量原理”。 出于对量子化电动力学中的惠勒-费曼吸收体理论的兴趣,费曼将静止作用原理应用于量子力学问题,并为路径积分表述和费曼图奠定了基础。其中一个重要观点是,正电子的行为就像电子在时间上向后移动。詹姆斯·格雷克写道:

这是理查德·费曼的巅峰时期。在他二十三岁时......现在地球上可能没有一个物理学家能像费曼一样对理论科学的既有知识如此掌握和了解。这些科学理论不仅仅是数学方面的才能(尽管人们已经很清楚......惠勒-费曼合作中出现的数学机制已经超出了惠勒自身的能力)。费曼似乎对方程背后的物质感到异常轻松,就像同龄的爱因斯坦,就像苏联物理学家列夫·兰道——但其他人很少能做到。

费曼获得普林斯顿大学奖学金的条件之一是他不能结婚,但他还是继续与高中时的恋人阿琳·格林鲍姆见面,尽管费曼知道她患有严重的肺结核,他还是决定一旦获得博士学位就娶她,。肺结核在当时是一种不治之症,人们预计她不会活过两年。1942年6月29日,他们乘渡船去斯塔滕岛并在那里的市政府办公室结婚。婚礼仪式既没有家人也没有朋友参加,仅有一对陌生人见证了仪式,婚礼上费曼也只能亲吻阿琳的脸颊。仪式结束后,费曼带阿琳去了黛博拉医院进行治疗,并在此后每到周末就去医院看望她。[27][28]

\subsection{曼哈顿计划}
\begin{figure}[ht]
\centering
\includegraphics[width=6cm]{./figures/a7213a5af965b6cb.png}
\caption{费曼在Los Alamos实验室的身份证。} \label{fig_Feynma_1}
\end{figure}
1941年,随着第二次世界大战在欧洲肆虐,但此时美国尚未涉入战争,费曼花了一个夏天在宾夕法尼亚州弗兰克福兵工厂研究弹道学问题。[29][30] 珍珠港事件使美国卷入战争后,费曼被罗伯特·威尔逊招募,他当时正在研究制造原子弹用浓缩铀的方法,这将成为曼哈顿计划的一部分。[31][32] 威尔逊在普林斯顿的团队正在研究一种名为isotron的同位素分离器装置,旨在电磁分离铀-235和铀-238。这与威尔逊的前导师欧内斯特·劳伦斯(Ernest O. Lawrence)在加州大学辐射实验室的一个团队开发的calutron电磁型同位素分离器完全不同。理论上,isotron的效率比calutron高很多倍,但费曼和保罗·奥卢姆很难确定它是否可行。最终,在劳伦斯的建议下,isotron项目还是被放弃了。[33]

就在此时,1943年初,罗伯特·奥本海默正在建立 Los Alamos实验室,这是一个位于新墨西哥州的方山,用来设计和制造原子弹的秘密实验室,普林斯顿大学的研究团队被重新部署到了该实验室。“像一群职业军人一样,”威尔逊后来回忆说,“我们一起报名去Los Alamos实验室。”[34] 费曼很快就像许多其他年轻物理学家一样被魅力非凡的奥本海默迷住了,奥本海默从芝加哥打长途电话给费曼,告诉费曼他在新墨西哥州的阿尔伯克基为阿琳找到了一个疗养院。他们于1943年3月28日乘火车离开,是第一批前往新墨西哥州的人。铁路公司为阿琳提供了轮椅,费曼为她支付了额外的私人房间费用。[35]

在Los Alamos实验室,费曼被分配到了汉斯·贝特的理论部门,[36] 他给贝特留下了深刻的印象并被任命为组长。[37] 他和贝特在罗伯特·塞伯先前工作的基础上发展了计算裂变式原子弹产量的贝特-费曼公式。[38] 作为一名年轻的物理学家,费曼还不是这个项目的核心,他主要负责管理理论部门的人类计算机计算组。费曼与斯坦利·弗兰克尔和尼古拉斯·梅特罗波利斯一起,帮助建立了一个使用国际商用机器公司(IBM)穿孔卡进行计算的系统。[39] 他还发明了一种新的计算对数的方法,后来用在连接机上。[40][41] 费曼在Los Alamos实验室的其他工作包括计算实验室小型核反应堆锅炉的中子方程,以测量裂变材料组件与临界状态的接近程度。[42]

费曼在完成这项工作后被派往位于田纳西州橡树岭的克林顿工程师工厂,曼哈顿计划在那里拥有铀浓缩设施。费曼帮助那里的工程师设计材料储存的安全程序,以避免临界事故的发生,特别是当浓缩铀与水接触时,水充当中子慢化剂,此时很容易发生事故。费曼还坚持给普通人做关于核物理的讲座,使他们能够居安思危。[43] 他解释说,虽然无论多少非浓缩铀都可以安全储存,但浓缩铀则必须小心处理。费曼还为不同等级的浓缩制定了一系列安全建议。[44] 他被告知,如果橡树岭的人对他的建议有任何异议,费曼会告诫他们Los Alamos实验室“否则无法对他们的安全负责”。[45]
\begin{figure}[ht]
\centering
\includegraphics[width=10cm]{./figures/29367573878adfd8.png}
\caption{1946年在Los Alamos实验室举行的“超级”座谈会上。费曼在第二排,左起第四位,奥本海默旁边。} \label{fig_Feynma_2}
\end{figure}
费曼回到Los Alamos实验室后被任命为对提出的氢化铀炸弹进行理论工作和计算的负责人,并最终证明这是不可行的。[37][46] 物理学家尼尔斯·玻尔曾找到他进行一对一的讨论。费曼后来发现了玻尔会来找他的原因:大多数其他物理学家对玻尔太敬畏了,不敢和他争论,而费曼不受这些限制,他极力指出自己认为玻尔思想中有缺陷的任何东西。费曼表示他和其他人一样尊重玻尔,但一旦有人让他谈论物理,他就会变得十分专注,乃至忽略社交细节。也许正因如此,玻尔从未对费曼产生好感。[47][48]

在出于安全考虑而被与世隔绝的Los Alamos实验室,费曼通过调查物理学家们的柜子和桌子上的密码锁来自娱自乐。他经常发现同事们喜欢把锁的密码留在工厂设备上,写下组合,或者使用像日期这样容易猜测的密码。[49] 通过尝试他认为物理学家可能使用的数字,费曼找到了一个柜子密码(经证实是自然对数的基数之后的27–18–28, e = 2.71828...),并发现同事保存研究笔记的三个文件柜都有相同的密码。他打开柜子后在里面留下纸条当做恶作剧,吓得他的同事弗雷德里克·德·霍夫曼一度以为间谍已经盯上他们了。[50]

费曼每月380美元的工资大约仅是他微薄的生活费和阿琳医疗费的一半,因此他们被迫动用了阿琳3300美元的积蓄。[51] 周末的时候费曼会开着从朋友克劳斯·福克斯那里借来的车去阿尔伯克基看阿琳。[52][53] 当被问及实验室谁最有可能是间谍时,福克斯提到费曼曾破解过实验室文件柜密码以及频繁的阿尔伯克基之行,[52] 不过福克斯本人后来承认在为苏联从事间谍活动。[54] 联邦调查局(FBI)汇编了一份关于费曼的庞大文件。[55]
\begin{figure}[ht]
\centering
\includegraphics[width=10cm]{./figures/4b4fa48b40702640.png}
\caption{曼哈顿计划期间,费曼(中)和罗伯特·奥本海默(费曼右侧第一位)在Los Alamos实验室的社交集会上。} \label{fig_Feynma_3}
\end{figure}
在得知阿琳即将离世时,费曼开车去阿尔伯克基,坐着陪了她几个小时,直到阿琳于1945年6月16日去世。[56] 随后,费曼就全身心地投入到曼哈顿项目的工作中,并致力于三位一体核试验。费曼声称自己是唯一一个不戴墨镜或焊工眼镜就能看到爆炸的人,他认为透过卡车挡风玻璃看是安全的,因为这样可以屏蔽有害的紫外线辐射。爆炸的巨大亮度使他不得不躲到卡车地板上,在那里他看到了一个暂时的“紫色斑点”余像。[57]

\subsection{康奈尔大学时期}
费曼名义上在威斯康星大学麦迪逊分校被任命为物理学助理教授,但在参与曼哈顿计划期间休了无薪假。[58] 1945年,他收到了文理学院院长马克·英格拉哈姆的一封信,信中要求他在下一学年回到大学任教。在费曼拒绝之后,他便不再在该学校任职。几年后在威斯康星大学麦迪逊分校的一次演讲中,费曼戏谑道:“回到唯一一所基于正确判断力而解雇了我的大学真是太棒了。”[59]

早在1943年10月30日,贝特就写信给他所在大学康奈尔大学的物理系的系主任,建议聘用费曼。1944年2月28日,这一建议得到了罗伯特·巴彻的认可,[60] 罗伯特·巴彻同样来自康奈尔大学,[61] 是Los Alamos实验室最资深的科学家之一。[62] 1944年8月,费曼接受了康奈尔大学的职位。奥本海默也曾希望招聘费曼加入加州大学,但物理系的负责人雷蒙德·T·比尔奇并不是很愿意。后来他在1945年5月表示愿意给费曼提供职位,但遭到了费曼的拒绝。康奈尔大学为费曼提供了与其相匹配的3900美元年薪。[60] 1945年10月,费曼成为Los Alamos实验室首批离开前往纽约州伊萨卡的小组领导者之一。[63]

由于费曼不再在Los Alamos实验室工作,他也无法再免除兵役。在费曼的入伍体检中,军队精神科医生诊断出费曼患有心理疾病,军队以此为由给予他4级豁免。[64][65] 费曼的父亲于1946年10月8日突然去世,此后费曼就一直患有抑郁症。[66] 1946年10月17日,他给阿琳写信表达了自己对她深深的爱和心痛,这封密封的信直到费曼死后才被公诸于世。“请原谅我没有寄出这封信,”信的结尾写道,“但是我不知道你的新地址。”[67] 由于无法专注于科研,费曼开始着手解决一些物理问题,不是为了实际应用,而是为了自我满足。[66] 费曼研究的内容之一涉及分析旋转中的物理学、章动圆盘在空气中运动,其灵感来自于他在康奈尔自助餐厅时见到有人将餐盘抛向空中。[68] 费曼阅读了威廉·罗恩·汉密尔顿爵士关于四元数的著作,并试图用它们来表述电子的相对论,但没有获得成功。费曼在此期间使用旋转方程来表达不同的旋转速度的工作,最终被证明对他获得诺贝尔奖的工作具有重要意义。但是由于他筋疲力尽,后来把注意力转移到不太直接的实际问题。当其他著名大学如普林斯顿高等研究院、加州大学洛杉矶分校和加州大学伯克利分校向他抛来教授职位的橄榄枝时,费曼甚至感到十分惊讶。[66]
\begin{figure}[ht]
\centering
\includegraphics[width=6cm]{./figures/f7fef9704a1adbc4.png}
\caption{表示电子/正电子湮灭的费曼图。} \label{fig_Feynma_4}
\end{figure}
费曼不是战后初期唯一碰壁的理论物理学家。当时量子电动力学深受微扰理论中无穷积分的影响,这些都是理论中明显的数学缺陷,费曼和惠勒试图解决这些问题,但未能成功。[69] 对此,默里·盖尔曼指出,“这是理论物理学家之耻。”[70] 1947年6月,美国主要物理学家在牛尾洲会议上会面。对费曼来说,这是他“第一次与大人物举行大型会议......和平时期我从参与过这样的会议。"[71] 科学家们在会上讨论了困扰量子电动力学界的问题,但是理论物理学家完全被实验物理学家的巨大成就所掩盖,实验物理学家们报道了发现兰姆位移,测量了电子的磁矩,以及提出了罗伯特·马沙克的双介子假说。[72]

贝特率先完成了汉斯·克拉默斯的工作,并为兰姆位移导出了重整化的非相对论量子方程,下一步是创建相对论量子方程。费曼觉得自己可以完成这一任务,但当他带着解决方案找到贝特时,两人的答案并不统一。[73] 费曼运用他在论文中使用的路径积分表述,再次仔细研究了这一问题。像贝特一样,他通过应用截止项使积分有限,最终得到了与贝特一致的结果。[74][75] 费曼在1948年的波科诺会议上向同行们介绍了自己的工作,但是进展并不顺利。朱利安·施温格对他在量子电动力学方面的工作做了长篇的介绍,费曼随后提出了他的版本,题为“量子电动力学的替代表述”。第一次被提出的陌生的费曼图令费曼的听众们迷惑不解,费曼也未能令他人理解自己的观点,而保罗·狄拉克、爱德华·泰勒和尼尔斯·玻尔都对此提出了反对意见。[76][77]

对弗里曼·戴森而言,至少有一件事是明确的:即使没有其他人明白,朝永振一郎、施温格和费曼也明白他们在说什么,尽管他们对此没有发表任何看法。他确信费曼的表述更容易理解,并最终说服奥本海默事实就是如此。[78] 戴森在1949年发表了一篇论文,在费曼的论文中增加了关于如何实现重整化的新规则。[79] 这促使费曼在20世纪60年代,三年多的时间内于杂志《物理评论》上发表了一系列论文来表述自己的观点。[80] 费曼在1948年发表的关于“经典电动力学的相对论截止点”的论文试图解释他在波科诺会议上没能理解的东西,[81] 在1949年发表的关于“正电子理论”的论文讨论了薛定谔方程和狄拉克方程,并介绍了现在所谓的费曼传播子。[82] 最后,在1950年关于“电磁相互作用的量子理论的数学表述”和1951年关于“应用于量子电动力学中的算符微积分”的论文中,他发展了自己这些想法的数学原理,推导出了熟悉的公式并提出了新的公式。[83]

人们最初都是引用施温格的论文,但是1950年开始出现了引用费曼论文并使用费曼图的论文,并且这一现象越来越广泛。[84] 学生们开始学习并使用费曼创造的有力的新工具,计算机程序后来也被用于计算费曼图,提供了一个前所未有的强大工具。费曼图用形式语法构成了一种形式语言,这使得编写这样的程序成为可能。马克·卡克提供了历史下求和的形式证明,表明抛物型偏微分方程可以重新表示为不同历史下的和(即期望算子),这就是现在所知的费曼-卡克公式,其使用范围已经超出了物理学,扩展到关于随机过程的诸多应用中。[85] 但对施温格来说,费曼图是“教育学,而不是物理学”。[86]

到1949年,费曼在康奈尔大学变得情绪焦躁不安。“直到这些安排变得性不稳定”,他才定居在固定的房子或公寓里,住在招待所或学生公寓里,或者和已婚朋友住在一起。[87] 费曼喜欢和大学生约会,雇佣妓女,和朋友的妻子睡觉。[88] 他不喜欢伊萨卡寒冷的冬季天气,渴望暖和的气候。[89] 最重要的是,在康奈尔大学,费曼一直生活在在汉斯·贝特的阴影下。[87] 尽管如此,费曼还是对他在康奈尔生涯中的居住了大部分时间的Telluride House进行了做了亲切的回顾。在一次采访中,他将这所房子描述为“一群男孩,因为他们的学识、聪明、头脑或其他原因,而被特别遴选出来,提供免费食宿等等。”他喜欢这套房子的便利,并说“正是在那里我做了一些基础工作”,而这帮助他获得了诺贝尔奖。[90][91]