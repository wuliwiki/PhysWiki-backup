% 放射源
% license CCBYSA3
% type Wiki

(本文根据 CC-BY-SA 协议转载自原搜狗科学百科对英文维基百科的翻译)

放射源是指发出电离辐射的已知数量放射性核素,典型的几种辐射类型有γ射线、$\alpha$粒子、$\beta$粒子和中子辐射。

放射源可用于辐照(这里辐射对目标材料起到了重要的电离作用),可以作为用于辐射测量过程和辐射防护仪器校准的辐射计量源,也用于工业过程测量,如造纸和钢铁工业中的厚度测量。放射源可以密封在容器中(高穿透性辐射)或者沉积在表面上(弱穿透性辐射),或者将它们置于流体中。

作为一种辐射源,它们被应用于放射治疗医学和工业中,例如工业射线成像、食品辐射、消毒、灭虫和聚氯乙烯的辐射交联。

放射性核素是根据它们发射的辐射类型和特征、发射强度和衰变半衰期来分类的。常见的放射性核素包括钴-60,[1] 铱-192,[2] 和锶-90。[3] 在国际单位制(SI)中,放射源活度的测量单位是贝克勒尔,但是在部分地区(比如美国)仍然使用历史上的单位:居里。尽管美国国家标准技术研究所(NIST)强烈建议使用国际单位制。[4] 出于健康卫生的目的,在欧盟国际单位制是强制使用的。