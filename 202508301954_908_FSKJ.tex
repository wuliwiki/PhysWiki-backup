% 仿射空间(综述)
% license CCBYSA3
% type Wiki

本文根据 CC-BY-SA 协议转载翻译自维基百科\href{https://en.wikipedia.org/wiki/Affine_space}{相关文章}

\begin{figure}[ht]
\centering
\includegraphics[width=6cm]{./figures/c472da6323adb22e.png}
\caption{在$\mathbb{R}^3$ 中,蓝色的上平面$P_2$不是一个向量子空间,因为:$\mathbf{0} \notin P_2$;$\mathbf{a} + \mathbf{b} \notin P_2$。因此,$P_2$是一个仿射子空间。它的方向(即与该仿射子空间关联的线性子空间)是绿色的下平面$P_1$,这个平面是一个向量子空间。虽然向量$\mathbf{a}$和$\mathbf{b}$都属于$P_2$,但它们的差向量是一个位移向量,它不属于$P_2$,而是属于向量空间 $P_1$。} \label{fig_FSKJ_1}
\end{figure}
在数学中,仿射空间是一种几何结构,它推广了欧几里得空间的一些性质,但这些性质与距离和角度测量无关,仅保留了平行性和平行线段长度比例等相关特性。仿射空间是仿射几何的基本背景。

与欧几里得空间类似,仿射空间中的基本对象称为点,它们可以看作空间中的“位置”,没有大小或形状,即零维对象。通过任意两点,可以画出一条无限延伸的直线(一维点集);通过任意不共线的三点,可以画出一个平面(二维点集);更一般地,任意 $k+1$ 个处于一般位置的点,可以确定一个$k$ 维平面或仿射子空间。仿射空间的一个显著特征是平行线的概念:同一平面内的两条平行直线永不相交;同一平面内的非平行直线则必定在某一点相交。并且,对于任意一条直线和空间中的任意一点,总能画出一条通过该点且与原直线平行的直线。所有相互平行的直线属于同一个方向的等价类。

与向量空间中的向量不同,仿射空间中没有一个特定的点作为原点。在仿射空间中:没有预定义的“点与点相加”或“点与数相乘”的概念。然而,对于任意一个仿射空间,可以通过两点之间的差来构造一个关联向量空间。这些差向量被称为:自由向量,位移向量,平移向量或简称平移[1]。同样,将一个位移向量加到某个点上是有意义的,这会生成一个新的点,即从原点沿该向量平移后的点。虽然点不能被随意相加,但可以取点的仿射组合:即系数和为 1 的加权和,这会产生另一个点。这些系数定义了经过这些点所在平面的重心坐标系。

任意向量空间都可以被看作一个仿射空间,这相当于“忘记零向量的特殊角色”。在这种情况下:向量空间中的元素既可以看作仿射空间中的点,也可以看作位移向量(平移)。当把零向量看作一个点时,它就被称为原点。将一个固定向量加到向量空间的某个线性子空间上,可以得到这个向量空间的一个仿射子空间。我们通常说,这个仿射子空间是通过某个平移向量将该线性子空间从原点平移得到的。在有限维情形下:这样的仿射子空间就是某个非齐次线性系统的解集;该仿射空间的位移向量则是对应齐次线性系统的解集,这是一个线性子空间。相比之下,线性子空间总是包含向量空间的原点。

仿射空间的维数定义为其平移向量空间的维数:一维仿射空间称为仿射直线;二维仿射空间称为仿射平面;在$n$维的仿射空间或向量空间中,维数为$n-1$的仿射子空间称为仿射超平面。
\subsection{非正式描述}
\begin{figure}[ht]
\centering
\includegraphics[width=8cm]{./figures/9b48b8a8cef561cb.png}
\caption{从Alice和Bob的视角来看原点的位置:Alice视角下的向量计算用红色表示;Bob视角下的向量计算用蓝色表示。} \label{fig_FSKJ_2}
\end{figure}
仿射空间就是一个向量空间,在“忘记哪个点是原点”之后所剩下的结构。正如法国数学家 Marcel Berger 所说:“仿射空间不过是一个向量空间,我们试图通过加入平移,把原点从线性映射的结构中抹去。”[2]想象这样一个情境:Alice知道某个点是真正的原点;而Bob认为另一个点$p$才是原点。现在要把两个向量$a$和$b$相加:Bob 从点 $p$ 各自画一条箭头指向$a$和$b$,再通过平行四边形法则求出他认为的$a + b$;但 Alice 知道,他实际上计算的是:
$$
p + (a - p) + (b - p)~
$$
同样地,Alice 和 Bob 也可以分别计算$a$和$b$的任意线性组合,或者某个有限向量集合的线性组合。一般情况下,他们会得到不同的结果。然而,如果某个线性组合的系数和等于 1,那么他们会得到完全相同的结果。

例如:如果 Alice 走到了:
$$
\lambda a + (1 - \lambda)b~
$$
那么 Bob 会走到:
$$
p + \lambda(a - p) + (1 - \lambda)(b - p) = \lambda a + (1 - \lambda)b~
$$
只要满足:$\lambda + (1 - \lambda) = 1$无论他们各自认为哪个点是原点,都会用相同的线性组合描述同一个点。

因此:只有Alice知道完整的线性结构;但Alice和Bob都知道相同的“仿射结构”,即“系数和等于 1 的线性组合”的取值。任何一个具备这种“仿射结构”的集合,就是一个仿射空间。
\subsection{定义}
虽然仿射空间可以用公理化的方式来定义(见下文 § 公理部分),这种方法类似于欧几里得在《几何原本》中对欧几里得空间的定义,但为了方便,现代的大多数资料都会基于成熟的向量空间理论来定义仿射空间。

一个仿射空间由以下部分组成:一个集合$A$;一个向量空间$\overrightarrow{A}$;以及 $\overrightarrow{A}$ 的加法群在$A$上的一个自由且传递的作用。[3]

其中:$A$ 中的元素称为点;与该仿射空间关联的向量空间 $\overrightarrow{A}$ 中的元素称为向量,也可以称为平移,有时也叫自由向量。

上述定义可以更明确地表示为一个映射(通常记作“加法”):
$$
\begin{aligned}
A \times \overrightarrow{A} &\;\longrightarrow\; A \\
(a, v) &\;\longmapsto\; a + v,
\end{aligned}~
$$
并且这个映射满足以下性质[4][5][6]:
\begin{enumerate}
\item \textbf{右单位元:}\\
$$
\forall a \in A, \quad a + 0 = a,~
$$
其中 $0$ 是关联向量空间 $\overrightarrow{A}$ 的**零向量**。
\item \textbf{结合律:}\\
$$
\forall v, w \in \overrightarrow{A},\; \forall a \in A, \quad (a + v) + w = a + (v + w),~
$$
这里最后的 “+” 是指在 $\overrightarrow{A}$ 中的向量加法。
\item \textbf{自由且传递的作用:}\\
对于任意 $a \in A$,映射
$$
\overrightarrow{A} \;\longrightarrow\; A: \quad v \;\longmapsto\; a + v~
$$
是一个双射。
前两个性质只是(右)群作用基本定义。第三个性质刻画了自由且传递的作用:传递性保证了映射是“满射”;自由性则保证了映射是“单射”。由前两个性质,还可以推出第四个性质:
\item 一一\textbf{对应的平移:}\\
对于任意 $v \in \overrightarrow{A}$,映射:
$$
A \;\longrightarrow\; A: \quad a \;\mapsto\; a + v~
$$
是一个双射。\\
性质 3 通常可以用以下等价形式(第 5 个性质)表述:\\
\item \textbf{减法:}\\
对于任意 $a, b \in A$,存在唯一的 $v \in \overrightarrow{A}$,记作:
$b - a$,使得:$b = a + v$.
\end{enumerate}
另一种表达这个定义的方式是:仿射空间是一个向量空间加法群作用下的主齐性空间。
按定义,齐性空间是带有一个传递群作用的空间;对于主齐性空间,这种传递作用还必须是自由的。
