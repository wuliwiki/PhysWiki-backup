% 哥德尔不完备定理(综述)
% license CCBYSA3
% type Wiki

本文根据 CC-BY-SA 协议转载翻译自维基百科\href{https://en.wikipedia.org/wiki/G\%C3\%B6del\%27s_incompleteness_theorems}{相关文章}。

哥德尔的不完全性定理是数学逻辑中的两个定理,涉及形式公理化理论中可证明性的极限。这些结果由库尔特·哥德尔在1931年发布,在数学逻辑和数学哲学中都具有重要意义。这些定理被广泛地,但并非普遍地解释为,证明了希尔伯特寻找一个完整且一致的公理集合来描述所有数学的计划是不可能实现的。

第一个不完全性定理声明,任何一个一致的公理系统,只要其定理可以通过有效程序(即算法)列出,都无法证明关于自然数算术的所有真理。对于任何这样的形式系统,总会存在一些关于自然数的陈述,这些陈述是正确的,但在该系统内无法证明。

第二个不完全性定理,是第一个定理的扩展,表明该系统无法证明自身的一致性。

通过使用对角线论证,哥德尔的不完全性定理是首批关于形式系统局限性的紧密相关定理之一。随后,塔尔斯基提出了真理的形式不可定义性定理,丘奇证明了希尔伯特的判定问题是不可解的,图灵的定理表明不存在可以解决停机问题的算法。
\subsection{形式系统:完整性、一致性和有效公理化} 
不完全性定理适用于那些足够复杂的形式系统,这些系统能够表达自然数的基本算术,并且是一致的且具有有效的公理化。特别是在一阶逻辑的背景下,形式系统也被称为形式理论。一般来说,形式系统是一个推理工具,由一组特定的公理以及符号操作规则(或推理规则)组成,这些规则允许从公理推导出新的定理。一个这样的系统的例子是一阶皮亚诺算术系统,这是一个所有变量都指代自然数的系统。在其他系统中,如集合论,只有一些形式系统中的句子表达关于自然数的陈述。不完全性定理涉及的是这些系统内的形式可证明性,而不是“非正式意义上的可证明性”。

形式系统可能具有几个属性,包括完整性、一致性和有效公理化的存在。不完全性定理表明,包含足够算术内容的系统无法同时具备这三种属性。
\subsubsection{有效公理化}  
如果一个形式系统的定理集合是递归可枚举的,则该系统被称为有效公理化(也称为有效生成)。这意味着存在一个计算机程序,理论上可以枚举该系统的所有定理,而不会列出任何非定理的陈述。有效生成的理论的例子包括皮亚诺算术和泽梅洛–弗兰克尔集合论(ZFC)。\(^\text{[1]}\)

被称为真算术的理论包括在皮亚诺算术语言中关于标准整数的所有真陈述。该理论是一致且完整的,并包含足够的算术内容。然而,它没有递归可枚举的公理集合,因此不满足不完全性定理的假设。