% 线性空间的同态与同构
% license Xiao
% type Tutor


与群运算的同态映射类似,线性空间的同态映射能保证运算结构不变。因而我们有
\begin{definition}{}
给定域$\mathbb F$上的两个线性空间$V_1,V_2$。如果存在一个映射$f:V_1\rightarrow V_2$,使得对于任意$a,b\in \mathbb F$及$任意\boldsymbol{x_1,x_2}\in V_1,V_2$,都有
\begin{equation}
f(a\boldsymbol {x_1}+b\boldsymbol {x_2})=af(\boldsymbol {x_1})+bf(\boldsymbol {x_2})~,
\end{equation}
则称$f$是$V_1$到$V_2$的一个同态(homomorphism)。
\end{definition}
如果该同态还是一个双射,则称之为同构(isomorphism)。此时记$V_1\cong V_2$
从上述定义我们可以知道,同构是特殊的同态,同态映射意味着“先线性运算再映射”和“先映射再线性运算”的结果是相同的。

那么怎么判断两个线性空间是否同构呢?