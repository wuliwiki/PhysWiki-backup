% 彭罗斯图形表示法
% Penrose Graphic Notation|张量|图像记号|图形|微分几何|线性代数|多重线性映射|彭罗斯|彭罗斯记号



\pentry{张量\upref{Tensor}}

彭罗斯图形表示法(Penrose graphical notation,或diagrammatic tensor notation),或者叫彭罗斯图像记号,是一种用二维图像的方式直观表示张量的性质和运算的系统,由彭罗斯于1971年提出\footnote{Roger Penrose, "Applications of negative dimensional tensors," in Combinatorial Mathematics and its Applications, Academic Press (1971). See Vladimir Turaev, Quantum invariants of knots and 3-manifolds (1994), De Gruyter, p. 71 for a brief commentary.}。

图像记号通常方便手写,可应用于自旋网络、多线性代数、量子计算和李群分类等诸多领域。




\subsection{概念的引入}


考虑一个线性空间$V$上的双线性函数$f:V\times V\to\mathbb{F}$,其中$\mathbb{F}$是$V$的基域。对$f$输入两个向量,能得到一个数字;但是如果只对$f$输入一个向量,则得到的是一个对偶向量。因为对于给定的$\bvec{x}\in V$,$f(\bvec{x}, \bvec{y})$可以理解为关于$\bvec{y}$的线性函数,这个线性函数随着$\bvec{x}$变化而变化,所以你可以理解为$f$把$\bvec{x}$映射为一个$V$上的线性函数$f(\bvec{x}, \cdot)$\footnote{这里的$\cdot$表示空位,即需要输入自变量的地方。},而线性函数正是对偶向量。

这么一看,$f$既可以理解为$V\times V\to\mathbb{F}$的映射,也可以理解为$V\to V^*$的映射,只取决于你输入的是什么。$f$作为二元映射,就像有两个接口,给它接入两个向量,它就返回一个数字;但是只接入一个向量的话,它就变成一个单自变量映射,只剩一个接口。

反过来,$V$也是$V^*$的对偶空间,所以$f$接入一个向量后所剩下的东西就是一个$V^*$中的元素——而这个东西的特点是“有一个接口”。

顺着这个思路,我们可以自然地创造出彭罗斯记号。



\begin{definition}{}

给定域$\mathbb{F}$上的线性空间$V$。将$\mathbb{F}$中的元素表示为没有接口的图形,$V$中的元素表示为有一个\textbf{上接口}的图形,$V^*$中的元素表示为有一个\textbf{下接口}的图形。$V$中不同的元素,用接口连接的图形的形状等特征作区分。

\end{definition}



\begin{example}{彭罗斯记号中的向量与对偶向量}\label{PnrsGN_ex1}


\begin{figure}[ht]
\centering
\includegraphics[width=14cm]{./figures/PnrsGN_1.pdf}
\caption{彭罗斯记号表示法示意图。} \label{PnrsGN_fig1}
\end{figure}


如\autoref{PnrsGN_fig1} 所示,指标记号所表示的主向量$v^a$用彭罗斯记号表示为有一个向上的接口的图像,它和其它主向量的区分在于接口一端连接的形状。也就是说,如果你看到一个同样有一个上接口的图像,但是接口一端的图形和这里不一样,那就说明这是不同于$v^a$的另一个主向量。对偶向量$\omega_a$表示为一个有向下的接口的图像。

$\omega_av^a$表示上述这一对对偶向量之间相互作用所得的数字,彭罗斯记号中表示为对应的两个接口连接到一起,形成一个没有接口的对象,那就是一个数字。

如果考虑二者的张量积$T_a^{\phantom{1}b}$,那么就不把它们的接口接到一起,而是简单合并到一起,变成一个具有一上一下两接口的对象。

你可以在接口旁标注这个接口的名字,但不至于混淆的时候可以不标。

\end{example}


由\autoref{PnrsGN_ex1} 可见,彭罗斯记号表示\textbf{缩并(contraction)}非常方便,只要把参与缩并的两个指标对应的接口接到一起就可以了。作缩并的时候要注意是哪些指标参与,因此彭罗斯记号中的接口也要注意区分。



\begin{example}{彭罗斯记号中的接口命名}

\begin{figure}[ht]
\centering
\includegraphics[width=14cm]{./figures/PnrsGN_2.pdf}
\caption{接口命名示意图。} \label{PnrsGN_fig2}
\end{figure}

如\autoref{PnrsGN_fig2} ,$(2, 0)$型张量被表示为一个具有两上接口的图形,我们在接口\textbf{末端}分别标注两个接口对应的指标名称。这样,入宫要计算$T^{ab}\omega_{a}$,就把$\omega_{a}$的下接口接到命名为$a$的接口上,而计算$T^{ab}\omega_{b}$时要把$\omega_b$的下接口接到命名为$b$的接口上。

如果调换指标的位置,则$T^{ba}$中两个接口的命名就和$T^{ab}$相反。只是命名改变,不影响$T^{ab}$的本质,但是很多时候我们为了方便描述,会默认接口$a$在左、接口$b$在右,于是$T^{ba}$可以表示成\autoref{PnrsGN_fig2} 中第一行最右边那个“交叉”的形式,保证$a$、$b$的默认位置不变。

如果指标位置已默认,那么就可以省去写接口命名,而用交叉的方式表示接口顺序不同的张量。比如,如果要表达$T^{ab}=T^{ba}$,就可以写成\autoref{PnrsGN_fig2} 第二行的样子。

\end{example}



既然不能忽略彭罗斯记号中的指标命名,就算省略指标命名也需要默认指标的位置顺序,所以换接口指标的时候要小心。

考虑$T^{ab}\omega_{c}$和$T^{ab}\delta^c_a\omega_{c}$的区别,前者有三个指标,后者只有一个。事实上,$T^{ab}\delta^c_a\omega_{c} = T^{ab}\omega_{a}$,于是彭罗斯记号里$\delta^c_a$就直接表示为一个没有连接图形的\textbf{导线},但是导线两端的接口分别命名为$c$和$a$。


有了图形表示法,我们一开始说的$f$就很容易直观理解了。如\autoref{PnrsGN_fig3} ,$f$本身表示为一个有两个下接口的图形,但是如果输入了一个自变量,即给它接入一个向量$v^a$(有一个上接口的图形),那么结果就是一个具有一个下接口的图形,也就是一个对偶向量。这个对偶向量的图形部分是$f$和$v^a$的形状,这意味着这个对偶向量由$f$和$v^a$共同决定。也就是说,$f(v^a, \cdot)$和$f(u^a, \cdot)$是不同的两个对偶向量,二者在彭罗斯记号中以总体图形的形状来区分。

\begin{figure}[ht]
\centering
\includegraphics[width=14cm]{./figures/PnrsGN_3.pdf}
\caption{双线性函数示意图。} \label{PnrsGN_fig3}
\end{figure}



\subsection{运算}










\subsection{一些重要的符号}


前面我们已经知道了,$\delta_b^a$在指标表示法中的作用是“改变指标名称”,因此在彭罗斯记号中表现为一根导线。类似地,内积$g^{ab}$是给定流形上最重要的一个双线性函数,因此特别地表示为




























