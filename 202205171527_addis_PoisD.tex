% 泊松分布

\begin{issues}
\issueDraft
\end{issues}

\begin{equation}
f(\lambda, k) = \frac{\E^{-\lambda}\lambda^k}{k!}
\end{equation}
平均值和方差都是 $\lambda$.

证明方差需要使用
\begin{equation}
\begin{aligned}
\sum_{k=0}^\infty \frac{k^2\lambda^{k}}{k!}
&= \sum_{k=0}^\infty \frac{(k+1)\lambda^{k+1}}{k!}
= \lambda \sum_{k=0}^\infty \frac{(k+1)\lambda^k}{k!}\\
&= \lambda\dv{\lambda} \sum_{k=0}^\infty \frac{\lambda^{k+1}}{k!}
= \lambda \dv{\lambda} (\lambda \E^{\lambda})
= \lambda(\lambda+1)\E^{\lambda}
\end{aligned}
\end{equation}

\subsection{推导}
若 $\dv*{P}{t} = \lambda$, 那么第一次出现的时间分布为
\begin{equation}
f(t) = \lambda\E^{-\lambda t}
\end{equation}
在时间 $[0,T]$ 内不发生的概率为
\begin{equation}
P_0 = \int_T^{\infty} \lambda\E^{-\lambda t} \dd{t} = \E^{-\lambda T}
\end{equation}
在时间 $[0,T]$ 内发生一次的概率为
\begin{equation}
P_1 = \int_{0}^{t'} \lambda\E^{-\lambda t} \dd{t}
\end{equation}

