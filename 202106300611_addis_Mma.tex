% Mathematica 基础

\begin{issues}
\issueDraft
\end{issues}

\pentry{复数\upref{CplxNo}, 自然对数底\upref{E}}

\subsection{常识}
\begin{itemize}
\item 调整字体大小 \verb|Ctrl + 鼠标滚轮|.
\item 快捷键 \verb|Shift + Enter| 执行选中的行.
\item 一行输入多个表达式用分号, 分号也可以用于抑制输出. \verb|a = 3; b = 5; a b|
\item 用 \verb|(*注释*)| 写注释. 注释不会被执行. 也可以选中后按 \verb|Alt + /|.
\item 可以用小数点产生数值解, 例如 \verb|Sqrt[2.]|.
\item 空格表示乘号, 相当于 \verb|*|. 数字和字母间相乘可以不需要空格或 \verb|*|.
\item 科学计数法如 \verb|2.3*^70| 表示 $2.3\e{70}$.
\item 中断执行 \verb|Alt + ,| (笔记本界面) 或者 \verb|Ctrl + C| (文本界面).
\end{itemize}

\subsubsection{常数}

\begin{itemize}
\item 圆周率: \verb|Pi| 或 \verb|\[Pi]| 或搜索 \verb|pi|.
\item 自然对数底: \verb|E| 或 \verb|\[ExponentialE]| 或搜索 \verb|ee|.
\item 虚数单位: \verb|I| 或 \verb|\[ImaginaryI]| 或搜索 \verb|ii|. \verb|a + I b| 表示复数, \verb|I| 不能用作其他变量名.
\item 虚数单位: \verb|\[ImaginaryJ]| 或搜索 \verb|jj|.
\item 1° 角的弧度: \verb|Degree| 或 \verb|\[Degree]| 或搜索 \verb|deg|.
\item 无穷: \verb|Infinity| 或 \verb|\[Infinity]| 或收货 \verb|inf|.
\end{itemize}

\begin{itemize}
\item 输入符号用 \verb|Esc| 键搜索, 例如搜索 \verb|pi| 按回车可以插入 \verb|π|. 这相当于输入 \verb|Pi| 或者 \verb|\[Pi]|. 如果你用过 TeX, 也可以按 \verb|Esc| 键后搜索 tex 命令, 例如 \verb|\varphi| 插入 \verb|φ|.
\item 可以在命令行直接粘贴 UTF-8 字符如 \verb|θ| 和 \verb|ϕ|.
\item 复制一个符号然后粘贴到一个 txt 文档中, 就会得到 \verb|\[...]| 形式的代码.
\item 复制一个表达式然后粘贴到 txt 文档中, 默认会得到含有空格和换行符的 plain text 命令, 相当于选中后右键 \verb|copy as -> input text|. 如果 \verb|copy as -> plain text|, 一些空格和换行符将不会插入.
\end{itemize}

\subsection{快捷键}
\begin{itemize}
\item \verb|Ctrl + 鼠标滚轮| 调整字体大小.
\item \verb|Shift + Enter| 执行选中的行.
\item \verb|Ctrl + L| 调用最近的输入.
\item \verb|Ctrl + Shift + L| 调用最近的输出.
\item \verb|Ctrl + -| 输入下标. 相当于 \verb|Subscript[x, y]|. 注意这不完全是符号, 有运算意义.
\item \verb|Ctrl + 6| 输入上标. 相当于 \verb|x^y|
\item \verb|Ctrl + /| 输入分式, 相当于 \verb|x/y|.
\item \verb|Ctrl + 2| 输入根式, 相当于 \verb|Sqrt[]|.
\item \verb|Alt + )| 或 \verb|Alt + ]| 或 \verb|Alt + }| 插入一对括号, 光标移动到中间.
\item \verb|Alt + 1| 到 \verb|Alt + 6| 不同级别的标题
\item \verb|Alt + 7| 正文
\item \verb|Alt + +| 和 \verb|Alt + -| 用于选中一段文字后改变大小.
\end{itemize}

\subsection{变量}
\begin{itemize}
\item \verb|?变量| 显示某个变量的定义
\item \verb|Clear[变量名]| 清除某个变量的定义.
\item 数组 \verb|data = {a, 2, c, d}|, \verb|T = {{1, 2}, {3, 4}}|
\item \verb|Tableform[T]| 用于显示矩阵(没有括号).
\item \verb|MatrixForm[T]| 用于显示矩阵(有圆括号).
\end{itemize}

\subsection{算符}
\begin{itemize}
\item \verb|=| 立即赋值(赋值时理科计算右边, 最终结果赋给左边)
\item \verb|:=| 延迟赋值(每次需要左边时都替换为右边)
\item \verb|!=| 不等号
\item 连接字符串 \verb|"abc" <> "defg" <> "hij"|.
\item 单变量的函数 \verb|f[...]| 也可以用后缀形式写成 \verb|... // f|.
\item \verb|/.| \verb|ReplaceAll[]| 的简写形式. (什么意思??)
\item \verb|Sin'[x]| 相当于 \verb|D[Sin[x], x]|
\item \verb|x =.| 清除符号 \verb|x| 的定义.
\item \verb|Remove[x]| 完全清除符号.
\end{itemize}

\subsection{常用函数}
\begin{itemize}
\item \verb|Print["a = ", a, " b = ", b]|.
\item 函数参数中, 下标在上标之前给出, 例如勒让德多项式 $P_n^m(x)$ 为 \verb|LegendreP[n, m, x]|.
\item 函数 \verb|N[表达式, 有效位数]| 把表达式的结果变为数值, 四舍五入到指定的有效数字, 第二个参数可省略. 例如 \verb|N[Pi]| 计算圆周率的前 5 位, \verb|N[Pi, 1000]| 计算 1000 位.
\item 定义函数 \verb|f[x_, y_] := x^2 + y^2|. 左边的自变量需要用下划线.
\item \verb|Re[z], Im[z]| 计算复数的实部和虚部.
\item \verb|Conjugate[z]| 复共轭, 也可以用 \verb|z\[Conjugate]| 显示为星号.
\item \verb|Abs[z]| 复数的绝对值
\item \verb|Arg[z]| 复数的幅角
\item \verb|Factor[x^25-1]| 因式分解
\item \verb|Expand[(a+b)^3]| 多项式展开
\item \verb|DSolve[{y''[x] + b y'[x] == 0, y[0] == 0, y'[0] == 1}, y[x], x]| 解常微分方程.
\item \verb|Piecewise[{{x^2, x < 0}, {x, x > 0}}]| 分段函数.
\item \verb|Table[i^2, {i,3}]| 输出 \verb|{1, 4, 9}|.
\item \verb|FullSimplify[表达式]| 简化表达式.
\item \verb|Clear["Global`*"]| 清除当前进程中的所有定义.
\item \verb|Range[5]|, \verb|Range[2, 5]| 输出 \verb|{1,2,3,4,5}|, \verb|{2,3,4,5}|.
\item \verb|Map[f, Range[3]]| 输出 \verb|{f[1], f[2], f[3]}|.
\item \verb|/@| 是 \verb|Map| 的简写形式 \verb|f /@ Range[3]|.
\item 分段函数 \verb|f[x_] := 0 /; x < -1| 以及 \verb|f[x_] := 1 /; x > -1 && x < 1| 等.
\item Listable 的函数可以直接 \verb|Sin[Range[3]]|. \verb|Atributes[Sin]|
\item \verb|LinearSolve[{{1, 2}, {3, 4}}, {5, 6}]| 解线性方程组.
\item \verb|Eigenvectors[{{1,2},{3,4}}]|, \verb|Eigenvalues[{{1,2},{3,4}}]| 计算矩阵本征值.
\end{itemize}

\subsubsection{微积分}
\begin{itemize}
\item 求和 \verb|Sum[f, {i, imin, imax}]|
\item \verb|Solve[左边==右边, x]| 求解关于 $x$ 的方程.
\item \verb|Limit[f, x->x0]| 极限 $\lim_{x\to x_0} f$.
\item \verb|Minimize[f, x]|, \verb|Maximize[f, x]| 求函数 $f(x)$ 的最小值/最大值.
\item \verb|Series[f, {x, x0, order}]| 把函数展开成泰勒级数.
\item \verb|D[Sin[x]]|, \verb|D[Sin[x+y],x]|, 导数和偏导数.
\item \verb|Integrate[1/(x^3-1),x]| 不定积分, \verb|Integrate[f,{x, xmin, xmax}]| 定积分, \verb|Integrate[f, {x, xmin, xmax},{y, ymin, ymax},...]| 重积分.
\item \verb|Integrate[(f[x])\[Conjugate] f[x], {x, -\[Infinity], \[Infinity]}, Assumptions -> {\[Sigma] > 0, Subscript[k, 0] > 0, x0 > 0}]| 中使用限制一些常数的范围, 这会大大缩短计算时间, 如果不规定范围, 则默认是任意复数.
\item \verb|FourierSeries[函数, 自变量, 项数]|.
\item \verb|FourierTransform[Exp[-x^2], x, k]| 傅里叶变换.
\item \verb|Plot[y[x], {x, -5, 5}]| 简单的画图.
\item \verb|Plot3D[Sin[y+Sin[3x]], {x,-3,3}, {y,-3,3}]| 简单的画图.
\end{itemize}

\subsubsection{数值运算}
\begin{itemize}
\item \verb|NIntegrate[f, {x, xmin, xmax}]| 数值积分.
\item \verb|NSolve[左边==右边, x]| 求解关于 $x$ 的方程.
\item \verb|NMinimize[f, x]|, \verb|NMaximize[f, x]| 求函数 $f(x)$ 的最小值/最大值.
\end{itemize}
