% 线性谐振⼦ (转载)
% license Usr
% type Art

(本文根据 CC-BY 协议转载自季燕江的《量子序曲》, 进行了重新排版和少量修改)

\subsection{胡克定律}

中学里我们都学过胡克定律(Hooke's Law),一个弹簧,下面挂个重物,假设弹簧原来的长度是$x_0$,弹簧被拉抻的长度$\Delta x = x - x_0$和重物的重量成正比。表示成数学式子就是:

\begin{equation}
F = - k \Delta x~
\end{equation}

这里$F$是弹簧的弹性力,它的方向和弹簧拉抻的方向相反,所以要有个负号“$-$”,$k$是弹性系数,表示弹簧的“软硬”,弹簧越“硬”,$k$就越大,所谓“强弓硬弩”,弹簧越硬,我们只需稍稍使弹簧有个拉抻,它就会有个很大的弹力。$\Delta x$就是拉抻,如果是在重力场中,弹簧会向下被拉长。

弹簧是人类使用的最早的储能器,比如在希腊-罗马时期,西方就懂得制作大型的弩炮,并懂得利用绷紧皮筋发出的声音校准弩炮射击的方向。维特鲁威是罗马时期的建筑师,同时他也是给罗马军团制造大型弩炮的工匠,在他的《建筑十书》中曾有这样的记录:

\begin{equation}
\text{将皮筋绳索穿入(弹索孔),并以绞车和杠杆将其绷紧,绷到弩炮制作者能听到皮筋绳索发出特定音高的弦音,方可用楔子将它固定住。将弩炮的双臂扣上扳机,一旦发射,两边皮筋就应释放出一致的推力。如果它们发出的音调不一致,弩炮射出的弹丸便不可能是直线的}~
\end{equation}

\begin{figure}[ht]
\centering
\includegraphics[width=6cm]{./figures/46e0cb460acf27b7.png}
\caption{ 罗马军团的小型弩炮} \label{fig_QMPre4_1}
\end{figure}
