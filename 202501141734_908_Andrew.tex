% 安德鲁·怀尔斯(综述)
% license CCBYSA3
% type Wiki

本文根据 CC-BY-SA 协议转载翻译自维基百科\href{https://en.wikipedia.org/wiki/Andrew_Wiles}{相关文章}。

\begin{figure}[ht]
\centering
\includegraphics[width=6cm]{./figures/7082cdefe31fa317.png}
\caption{2005年的怀尔斯} \label{fig_Andrew_1}
\end{figure}
安德鲁·约翰·怀尔斯爵士(Sir Andrew John Wiles,1953年4月11日出生)是英国数学家,现任牛津大学皇家学会研究教授,专攻数论。他最著名的成就是证明了费马大定理,因此获得了2016年阿贝尔奖和2017年科普利奖,并于2000年被任命为英国帝国勋章骑士指挥官。2018年,怀尔斯被任命为牛津大学首任Regius数学教授。怀尔斯还是1997年麦克阿瑟学者奖得主。

怀尔斯出生于剑桥,父亲是神学家莫里斯·弗兰克·怀尔斯,母亲是帕特里夏·怀尔斯。在尼日利亚度过了大部分童年后,怀尔斯对数学产生了浓厚兴趣,特别是对费马大定理产生了兴趣。怀尔斯于1974年从牛津大学毕业后,开始致力于统一伽罗瓦表示、椭圆曲线和模形式,最初从巴里·马祖尔对岩泽理论的推广开始。1980年代初期,怀尔斯在剑桥大学工作了几年,然后前往普林斯顿大学,在那里他研究并应用了希尔伯特模形式。1986年,在阅读肯·里贝特关于费马大定理的开创性工作后,怀尔斯着手证明半稳定椭圆曲线的模性定理,这意味着费马大定理得以证明。到1993年,他已经能够说服一位知识渊博的同事相信他找到了费马大定理的证明,尽管随后发现了一个缺陷。1994年9月19日,怀尔斯和他的学生理查德·泰勒通过一番洞察力克服了这个缺陷,并于1995年发表了结果,广受赞誉。

在证明费马大定理的过程中,怀尔斯为数学家们发展了新的工具,使他们能够开始统一不同的思想和定理。他的前学生泰勒和另外三位数学家于2000年证明了完整的模性定理,使用了怀尔斯的工作。在2016年获得阿贝尔奖时,怀尔斯回顾了自己的遗产,并表示他不仅证明了费马大定理,还推动了整个数学领域朝着拉格朗日计划的方向发展,从而统一了数论。
\subsection{教育与早年生活}  
Wiles于1953年4月11日出生在英国剑桥,是Maurice Frank Wiles(1923–2005)和Patricia Wiles( née Mowll)的儿子。1952年至1955年间,他的父亲在剑桥的Ridley Hall担任牧师,后来成为牛津大学的神学讲座教授。[6]  

Wiles的正式学业从他和父母一起居住在尼日利亚时开始。然而,根据他父母写的信件,在他本应上学的最初几个月里,他拒绝去上学。从这个事实中,Wiles自己得出结论,认为在他最早的几年里,他并不热衷于待在学术机构。在2021年接受Nadia Hasnaoui采访时,他表示他相信这些信件,然而他自己却不记得有任何时候不喜欢解决数学问题。[7]  

Wiles曾就读于剑桥的King's College School[8] 和The Leys School, Cambridge[9]。Wiles在1999年接受WGBH-TV采访时提到,他在10岁时从学校回家的路上遇到了《费马最后定理》。他停在当地图书馆,发现了埃里克·坦普尔·贝尔的《最后的问题》一书,讲述了这个定理。[10] 因为这个定理表述如此简单,以至于他,一个10岁的孩子都能理解,但却没人能证明出来,他对它产生了浓厚的兴趣,决定成为第一个证明它的人。然而,他很快意识到自己的知识还不够,因此放弃了童年的梦想,直到33岁时,Ken Ribet在1986年证明了ε猜想,这一猜想与Gerhard Frey早前与费马方程的联系重新把这个梦想带回了他的注意。[11]