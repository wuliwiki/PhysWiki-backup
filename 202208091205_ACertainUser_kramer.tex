% 克拉默法则
% 克拉默|线性方程组|线性代数|行列式|代数余子式

\begin{issues}
\issueTODO
\end{issues}

\pentry{行列式\upref{Deter}}

\footnote{参考 Wikipedia \href{https://en.wikipedia.org/wiki/Cramer's_rule}{相关页面}.}\textbf{克拉默法则(Kramer's rule)}是一种直接用行列式解线性方程组的方法. 把线性方程组记为矩阵乘法\upref{Mat}的形式
\begin{equation}\label{kramer_eq1}
\mat A \bvec x = \bvec b
\end{equation}
其中 $\mat A$ 为系数矩阵. 当 $\mat A$ 为 $N\times N$ 的方阵且行列式 $\abs{\mat A} \ne 0$ 时(即满秩矩阵\upref{MatRnk}), 方程有唯一解(见 “线性方程组解的结构\upref{LinEq}”). 该解可以用克拉默法则直接写出:
\begin{equation}\label{kramer_eq2}
x_i = \frac{\abs{\mat A_i}}{\abs{\mat A}} \qquad (i = 1, \dots, N)
\end{equation}
其中 $\mat A_i$ 是把 $\mat A$ 的第 $i$ 列替换为 $\bvec b$ 而来.

\begin{example}{解方程组}
令\autoref{kramer_eq1} 中 $\mat A = \pmat{2 & 1\\ -1 & 3}$, $\bvec b = \pmat{4\\5}$, 求解方程组.

解: $\abs{\mat A} = 7$, $\abs{\mat A_1} = \vmat{4 & 1\\ 5 & 3} = 7$, $\abs{\mat A_2} = \vmat{2 & 4\\ -1 & 5} = 14$. 代入\autoref{kramer_eq2} 得 $\bvec x = \pmat{1\\2}$.
\end{example}

在数值计算时, 克拉默法则解方程组效率较低, 直接用高斯消元法求逆矩阵高斯消元法求逆矩阵\upref{InvMGs}会更快.

\subsection{证明}
% 证明需要代数余子式 $\mat A^*$(未完成)
% \begin{equation}
% \bvec A^{-1} = \frac{\mat A^*}{\abs{\mat A}}
% \end{equation}
% \begin{equation}
% \bvec x = \frac{\mat A^* \bvec b}{\abs{\mat A}}
% \end{equation}
以下证明以三阶矩阵为例,但可以方便地推广至任意阶矩阵.

定义伴随矩阵
\begin{equation}
\mat A^*=
\begin{bmatrix}
A_{11}&A_{21}&A_{31}\\
A_{12}&A_{22}&A_{32}\\
A_{13}&A_{23}&A_{33}\\
\end{bmatrix}
\end{equation}
其中$A_{ij}$称为代数余子式,$A_{ij}$可以理解为去掉i行j列的$\mat A$的行列式乘以$(-1)^{ij}$.

伴随矩阵有一重要性质 $\mat A \mat A^* = det(\mat A)\mat I$.

先证明$\mat A \mat A^* = det(\mat A)\mat I$

\begin{equation}
\mat A \mat A^* =
\begin{bmatrix}
a_{11}&a_{12}&a_{13}\\
a_{21}&a_{22}&a_{23}\\
a_{31}&a_{32}&a_{33}\\
\end{bmatrix}
\begin{bmatrix}
A_{11}&A_{21}&A_{31}\\
A_{12}&A_{22}&A_{32}\\
A_{13}&A_{23}&A_{33}\\
\end{bmatrix}
=
\begin{bmatrix}
A_{11}a_{11}+A_{12}a_{12}+A_{13}a_{13}&A_{21}a_{11}+A_{22}a_{12}+A_{23}a_{13}&A_{31}a_{11}+A_{32}a_{12}+A_{33}a_{13}\\
A_{11}a_{21}+A_{12}a_{22}+A_{13}a_{23}&A_{21}a_{21}+A_{22}a_{22}+A_{23}a_{23}&A_{31}a_{21}+A_{32}a_{22}+A_{33}a_{23}\\
A_{11}a_{31}+A_{12}a_{32}+A_{13}a_{33}&A_{21}a_{31}+A_{22}a_{32}+A_{23}a_{33}&A_{31}a_{31}+A_{32}a_{32}+A_{33}a_{33}\\
\end{bmatrix}
\end{equation}

其中
$$
A_{11}a_{11}+A_{12}a_{12}+A_{13}a_{13}
$$,即为$\mat A$的行列式.

而
$$
A_{21}a_{11}+A_{22}a_{12}+A_{23}a_{13}
$$,即相当于计算
$$
\begin{bmatrix}
a_{11}&a_{12}&a_{13}\\
a_{11}&a_{12}&a_{13}\\
a_{31}&a_{32}&a_{33}\\
\end{bmatrix}
$$
的行列式,因此
$$
A_{21}a_{11}+A_{22}a_{12}+A_{23}a_{13}=0
$$
类似地,可以推知
\begin{equation}
\mat A \mat A^* =
\begin{bmatrix}
det(\mat A)&0&det(\mat A)\\
0&det(\mat A)&0\\
0&0&det(\mat A)\\
\end{bmatrix}
=det(\mat A) \mat I
\end{equation}

再证明$x_i=\frac{det(\mat A_i)}{det(\mat A)}$

$$
\bvec x = \mat A^{-1} \bvec b = \frac{1}{det(\mat A)}\mat A^* \bvec b = 
\frac{1}{det(\mat A)}
\begin{bmatrix}
A_{11}&A_{21}&A_{31}\\
A_{12}&A_{22}&A_{32}\\
A_{13}&A_{23}&A_{33}\\
\end{bmatrix}
\begin{bmatrix}
b_1\\
b_2\\
b_3\\
\end{bmatrix}
=
\frac{1}{det(\mat A)}
\begin{bmatrix}
A_{11}b_1+A_{21}b_2+A_{31}b_3\\
A_{12}b_1+A_{22}b_2+A_{32}b_3\\
A_{13}b_1+A_{23}b_2+A_{33}b_3\\
\end{bmatrix}
$$

$$
\therefore x_1=\frac{1}{det(\mat A)}(A_{11}b_1+A_{21}b_2+A_{31}b_3)
=\frac{1}{det(\mat A)}det\left(
\begin{bmatrix}
b_{1}&a_{12}&a_{13}\\
b_{2}&a_{22}&a_{23}\\
b_{3}&a_{32}&a_{33}\\
\end{bmatrix}
\right)
=\frac{det(\mat A_1)}{det(\mat A)}
$$

同理可证$x_2, x_3,...$
