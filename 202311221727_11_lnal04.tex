% 线性映射的核与象
% license Xiao
% type Tutor


\begin{issues}
\issueTODO
\end{issues}

线性映射是线性空间之间的同态映射,因此我们可以研究其核与象。
\begin{definition}{}
设$V,W$为域$\mathbb F$上的线性空间,$f:V\rightarrow W$为线性映射。

记$ker\,f=\{\boldsymbol x\in V|f(\boldsymbol x)=\boldsymbol 0\}$,称作线性映射$f$的核(kernel)。记$Im\,f=\{f(\boldsymbol x)|\boldsymbol x\in V\}$,称作线性映射$f$的象(Image)
\end{definition}
\begin{exercise}{}
$f,V,W$的定义同上。验证核与象分别是$V$及$W$的子空间。
\end{exercise}
关于核与象,有两个好用的结论。
\begin{itemize}
\item 核$ker\,f=\{\boldsymbol 0\}\Longleftrightarrow f$是单射。
\item 若象$Im\,f=W\Longleftrightarrow f$是满射
\end{itemize}
在此只证明第一个结论。

proof.
先验证充分条件。反证该映射并非单射,及至少存在两个向量映射到同一个向量,设为$\boldsymbol{x,y}$,那么我们有
\begin{equation}
f(\boldsymbol{x}-\boldsymbol{y})=f(\boldsymbol x)-f(\boldsymbol y)=\boldsymbol 0~,
\end{equation}
由于核只有向量$0$,因此$\boldsymbol {x}=\boldsymbol{y}$

再验证必要条件。假设存在一个非$0$向量映射到$0$,即$f(a^i\boldsymbol x_i)=0$,则$-f(a^i\boldsymbol x_i)=f(-a^i\boldsymbol x_i)=0$,与假设矛盾,证毕。
可见,第一条结论能成立多亏了该同态映射是线性的,这也是线性空间的一个好处。