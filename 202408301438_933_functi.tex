% 函数(高中)
% keys 函数|定义域|值域|二元函数
% license Usr
% type Tutor

\pentry{集合(高中)\nref{nod_HsSet},集合的基本关系(高中)\nref{nod_HsSeOp},函数回顾(高中) \nref{nod_HsFunB}}{nod_19dc}

\begin{issues}
\issueDraft
\end{issues}

% 高中的 函数 不应该需要 映射 作为预备知识
% 本篇文章的预设读者是,对于函数的定义感到熟悉但又有些模糊,希望进一步了解函数的高中学生。

在小学学习数学时,老师会先教你数数,然后引导你学习加减法等基本运算。到了初中,老师会让你理解边、角等几何的基本概念,再逐步带领你探索平面图形内部的线与角之间的关系。在高中阶段,主要的研究基础就是实数集。通过前面的学习,相信你已经掌握了集合和元素的知识,积累了一些理解新概念的经验,是时候开始探讨数集之间的关系——函数。函数将成为你理解和分析数学问题的重要工具。

\subsection{接触函数}

想象一下,有这样一间高度自动化的\aref{工厂}{fig_functi_1},我们将各种原材料投入到工厂的入料口,经过一系列复杂的加工和处理,最终得到了各种精致的产品。

\begin{figure}[ht]
\centering
\includegraphics[width=5cm]{./figures/720d887f9539bb73.png}
\caption{自动化工厂} \label{fig_functi_1}
\end{figure}

从原材料的视角来看,工厂代表着那个让原材料经过一系列操作最终变成产品的处理过程。原材料进入工厂后,经历了各种操作、转化,最终变成了某种特定的产品。而从外部的视角来看,工厂则起到了桥梁的作用:它将每一个原材料与特定的产品连接起来。无论原材料是什么,只要经过这个处理过程,都会对应生成一个特定的产品。

在数学上,函数同样扮演着这间工厂的角色,它将一个集合中的元素通过一定的规则对应到另一个集合中的元素。正如工厂里的每一道工序都有其明确的目的和结果,函数的每一个操作也都明确地将一个输入对应到一个输出。

其实,在初中时,你就已经接触过函数了。你或许还\enref{记得}{HsFunB}正比例函数、反比例函数以及一次函数、二次函数。想一想你最初接触函数概念的时候,是怎么做的。那时,老师给了你一个等式,等式左边是y,右边是含有x的一个式子,以正比例函数为例,就是$y=kx+b$的样子。然后提供了一张\aref{表格}{tab_functi_1},让你将不同的$x$值代入表达式中,计算出相应的$y$。接着,在平面直角坐标系上依次标出点$(x,y)$,并利用想象将它们连成一条曲线。称这个曲线就是函数的图象。

\begin{table}\label{tab_functi_1}[ht]
\centering
\caption{研究函数图象的表格}\label{tab_functi1}
\begin{tabular}{|c|c|c|c|c|}
\hline
x & -2 & -1 & 0 & 1 \\
\hline
y &   &   &   &   \\
\hline
\end{tabular}
\end{table}

在初中的数学学习中,主要关注这三个函数的图象的性质,比如样子、与坐标轴的交点等等。那个时候,我们通常把表达式本身与函数图象等同起来,认为好像那条曲线就是我们口中的函数,或者认为一个函数就是一个计算式,输入一个数字,然后通过计算式计算以输出一个数字。

现在回忆结束,让我们来看一看高中数学中的函数和初中的联系。初中的函数知识其实已经涵盖了函数的许多重要特点和性质,这些内容在未来的函数学习中仍然非常重要,在初中阶段主要是以函数图象的特质来描述,比如:
\begin{itemize}
\item 函数图象的斜率
\item 函数图象与$x$轴和$y$轴的交点
\item 两个函数图象的交点
\item 函数图象的对称轴
\item 函数图象是否过定点
\item 函数图象上某点与坐标轴形成的矩形的面积
\end{itemize}

同时,初中阶段也给出了一个函数定义,说的是:如果在一个变化过程中有两个变量$x,y$,对于每个确定的$x$值都有唯一的$y$值与它对应,就称$y$是$x$的函数。这里要求每一个$x$都能找到一个与之对应的$y$,这其实就是函数定义的精髓部分。而这个概念从变量的观点和解析式的方法来描述函数,在实际使用上有些局限,无法描述更复杂的问题。

在高中,由于函数的种类多样,每次都将函数完整写出来的方法过于麻烦,有时也不完全确定函数表达式的样子,我们开始用符号来标记表达式。以刚才说过的老师给你的等式为例,等式右边的那个含有$x$的式子,我们一般用下面的记号来表示:
\begin{equation}
f(x)~.
\end{equation}

其中,$f$是给这个式子起的名字,一般取的名字是$f,g,h$等,至于为什么是$f$,后面会提到。$x$代表式子中的变量,以刚才的正比例函数为例,就可以认为$f(x)=kx+b$,这时,$x$是变量\footnote{\textbf{变量}通常用来表示一个未知的数或在一个确定情况下可以取不同值的数。},$k,b$就是参数\footnote{\textbf{参数}是用来描述某个函数或方程的特性或行为的常数。参数的值通常在一个特定的问题或情境中被固定下来,但可以在不同的情境中改变。}。于是一个函数按照刚才的说法,右侧用$f(x)$代替后的样子,就写成了$y=f(x)$的形式。这样既明确了变量是什么,又可以直接表示具有某些特点的一类函数。现在套用集合的语言,将函数的定义给出如下:

\begin{definition}{函数}\label{def_functi_1}
对于两个非空数集$X$和$Y$,$f$是一个二元关系\footnote{和初中的认知一样,这里的二元指的是两个变量。而“关系”指的是一种特殊的数学概念,高中阶段不涉及,此处认为是两个变量之间存在联系即可。},且满足对于每个$x\in X$都存在唯一的$y\in Y$与它对应,则称$f$是一个定义在$X$上的或从$X$到$Y$的\textbf{函数}(function),记作:
\begin{equation}
y=f(x),x\in X~.
\end{equation}
其中:
\begin{itemize}
\item $X$称为\textbf{定义域}(domain)
\item 所有的函数值$y$构成的集合,即$\{y|y=f(x),x\in X\}$称为\textbf{值域}(range)。
\end{itemize}
\end{definition}

关于这个概念需要注意的几点:
\begin{itemize}
\item 定义域和值域可以像二次函数一样不同,也可以像一次函数一样相同。
\item 尽管函数的三个要素是:定义域、对应关系和值域,但函数的值域是由定义域和对应关系唯一确定的,因此一般不会特别提及。
\item 在不产生歧义的情况下,一般默认函数的定义域是使得表达式有意义的所有实数构成的集合,此时可以省略定义域不写。比如:$y={1\over x}$的定义域默认是$(-\infty,0)\cup(0,+\infty)$。
\item 两个函数如果定义域和对应关系完全相同,则两个函数不论表达式是否完全一致,都相同。比如:$y=\sqrt{x^2}$和$y=|x|$,在定义域都是$\mathbb R$的情况下,是同一个函数。
\end{itemize}

根据这些注意的内容,可以发现,在一般情况下,值域和定义域都不需要特别写明,因此对函数最重要的就是对应关系,因此很多时候就直接简称函数$f(x)$。而$f(x)$之所以一般用$f$就是因为f是函数单词function的首字母。

在某点$a$上(或称$x$取$a$时)的函数值记作
\begin{equation}
f(a)\qquad\text{或}\qquad y|_{x=a}\qquad\text{或}\qquad f(x)|_{x=a}~.
\end{equation}

我们研究函数的手段也更加丰富,看待函数的视角也发生了变化。

我们可以将函数认作是对输入和输出的关系的描述,我们所见到的\textsl{计算式}就是告诉我们如果将输入得到输出的方法。

在数学中,我们将函数的概念进一步抽象。抛开对象之间如何实现转换的过程,而仅仅在两个事物之间建立对应关系——将集合中的全体元素,向另一个集合中的元素建立对应法则,一个对应法则就称为一个函数。

看函数的视角:对应,处理过程。


\subsection{函数的表示方法}

你应该还记得集合的表示方法分为枚举法、描述法和图示法,这三者各有优势,需要根据情况使用。函数也一样。下面是函数的\addTODO{几}种表示方法

\subsubsection{解析法}

通常最关键的区别是f这个表达式。所以一般也会直接用f来代指函数,即直接记作f(x)。

\subsubsection{图象法}
存在一些函数的

\subsection{函数的性质}

函数具有一些性质,有一些在高中会接触到,有一些不会接触到。
以后我们会看到一些用\enref{极限}{Lim}和\enref{导数}{Der}描述的性质。 例如\enref{连续性}{contin}, 一致连续 % \addTODO{链接}
, 可导。
\subsection{反函数}

反函数(inverse function),又称为逆函数
\subsection{复合函数}

f(g(x))

\begin{exercise}{$f(x^2-1)$的定义域是$(2,+\infty)$,求$f(x-3)$的定义域。}
答案:

解析:

将原本的命题写出应为:对$f(g(x))$,$g(x)=x^2-1$,$g(x)$的定义域是$(2,+\infty)$,求$f(h(x))$中$h(x)=x-3$的定义域。

根据$g(x)$的定义域是$(2,+\infty)$,$g(x)=x^2-1$,可知$g(x)$的值域是$(3,+\infty)$。根据复合函数的要求$f(x)$的定义域就是$(3,+\infty)$,从而需要$h(x)$的值域是$(3,+\infty)$,根据$h(x)=x-3$,可知$h(x)$的定义域是$(6,+\infty)$。综上,也就是$f(x-3)$的定义域是$(6,+\infty)$。

\end{exercise}

\subsection{特殊的函数}

\subsubsection{初等函数}

性质好

基本初等函数:
\begin{itemize}
\item 常值函数
\item 幂函数
\item 指数函数
\item 对数函数
\item 三角函数
\end{itemize}

\subsubsection{分段函数}

绝对值函数

取整函数

狄利克雷函数

\subsection{*映射}

\begin{enumerate}
\item 从内容上,我们将输入、输出从数字拓宽到了更多事物
\item 从数量上,我们不局限于单一输入,而是能够同时输入多种事物
\end{enumerate}


请注意,这一部分的内容已经完全从高中数学中删除,但由于其地位重要及为了了解一些相关概念,这里介绍一下映射及相关的概念。其实,刚才学习的函数特指的是数集之间的关系,如果不限定集合中的元素是数字的话,就可以把“函数”的概念推广,得到“映射”。

\begin{definition}{映射}
对于两个非空集合$A$和$B$,$f$是一个二元关系,且满足对于每个$a\in A$都存在唯一的$b\in B$与它对应,则称$f$是一个定义在$A$上的或从$A$到$B$的\textbf{映射}(mapping),记作:
\begin{equation}
f:A\to B\qquad\text{或者}\qquad f:a\mapsto b~.
\end{equation}
其中:
\begin{itemize}
\item $A$称为\textbf{定义域}(domain)
\item $a$称为\textbf{原像}(preimage)
\item $B$称为\textbf{陪域}(codomain,也译作\textbf{到达域}、\textbf{上域})
\item $b$称为\textbf{像}(image)
\item 所有的像构成的集合称为\textbf{值域}(range),值域是陪域的子集。
\end{itemize}
\end{definition}

对比\aref{函数的定义}{def_functi_1}看到,映射和函数定义上的区别只是对于集合的限定,相当于映射是增加了函数概念的外延。由于所有的数集都是集合,所以函数都是映射,那么既可以用映射的记号表示函数,例如:$f: \mathbb R \to \mathbb R$表示一个定义域和值域都为实数集的函数;$f:x \mapsto y$表示一个自变量为$x$,函数值为$y$的函数;也可以使用上面域映射相关的概念,如像、原像等来描述函数的相关概念。请注意这些写法和概念在高中阶段均不使用。

根据定义,映射已经限定了所有的$x$都必须能找到对应的值,因此从映射的结果看,$Y$有四种可能:
\begin{itemize}
\item 存在一些$y$是没有对应(0个)的$x$的,剩下都只对应一个(单射)
\item 每个$y$都只对应一个(1个)$x$的(双射)
\item 存在一些$y$对应了好几个(多个)$x$的,剩下都只对应一个(满射)
\item 既存在一些$y$是没有对应的$x$的,又存在一些$y$对应了好几个$x$的
\end{itemize}

由于前三个比较容易研究,因此分别称为单射、双射、满射,理解清楚会对反函数的学习有帮助。下面给出这三个概念的具体定义,如果不理解,只参照刚才的文字或后面给出的图片形象记忆就可以了,高中这里并不要求掌握每个概念。

\begin{definition}{单射、满射、双射}
设$f:A\to{B}$,若:
\begin{itemize}
\item 对任意$a_1,a_2\in{A}$,$a_1\not={a_2}$,都有$f(a_1)\not={f(a_2)}$,则称$f$是一个\textbf{单射}(injective)。
\item 任意$b\in{B}$,都存在$a\in{A}$,使得$f(a)=b$。则称$f$是一个\textbf{满射}(surjective)。
\item $f$既是单射又是满射,则称$f$是一个\textbf{双射}或者\textbf{一一对应}。
\end{itemize}
\end{definition}

\begin{figure}[ht]
\centering
\includegraphics[width=3cm]{./figures/f6cfb71bb0c378ef.png}
\caption{单射}\label{fig_functi_2}
\end{figure}
\begin{figure}[ht]
\centering
\includegraphics[width=3cm]{./figures/3031fee516997db1.png}
\caption{满射} \label{fig_functi_3}
\end{figure}
\begin{figure}[ht]
\centering
\includegraphics[width=3cm]{./figures/9fef429051c64955.png}
\caption{双射} \label{fig_functi_4}
\end{figure}
\subsection{总结}

阅读完上面的内容,相信你已经开始对函数这个新朋友有一些熟悉的感觉了。实际上,函数的概念涵盖了非常广泛的领域,它不仅是一个数学工具,更代表了一种深刻的思维方式——找寻基本的单元,研究彼此的关系\footnote{更底层的抽象是两个元素之间或两个集合之间的“关系”。注意这里的“关系”并非语言上的泛指,而有严格的数学定义。函数是一种特殊的关系,而由于函数的概念大多数人都接触过,而且函数的英语本身也有功能的意思,有很多场合用函数来代指映射、变换等并非函数的关系,但这些概念高中不会接触到,此处只是提示,希望你在未来遇到时注意。}。而因为数字是我们最为熟悉的元素,数字构成的集合(数集)也是我们最为熟悉的集合,所以研究数集之间关系的函数在人们的认知中占据了特别重要的地位。函数这种概念不仅限于数学中的数集,它在各种领域中都有着广泛的应用。

从现在开始,直到你未来的大学、研究生学习,你都会遇到各种各样不同类型的函数。无论是在物理、经济学还是计算机科学中,函数都是用来描述复杂关系的核心。函数不仅仅是一个数学符号,它是一种让你能够用更抽象、更广阔的视角去理解世界的方法。通过将一个特定的过程或现象抽象成一个函数,并利用函数的性质去分析和预测其行为,你可以更深入地理解事物的动态变化。在生活中,你可能会发现,很多问题都可以通过函数的视角来处理和解决。这种方法将成为你未来学习和工作的基石,是帮助你分析、理解和解决问题的强大工具。
