% 2009 年计算机学科专业基础综合全国联考卷
% keys 2009年 计算机 考研 全国卷

\subsection{一、单项选择题}
第1~40 小题,每小题2 分,共80 分.下列每题给出的四个选项中,只有一个选项最符合试题要求.

1. 为解决计算机主机与打印机之间速度不匹配问题,通常设置一个打印数据缓冲区,主机将要输出的数据依次写入该缓冲区,而打印机则依次从该缓冲区中取出数据.该缓冲区的逻辑结构应该是. \\
A.栈 $\qquad$ B.队列 $\qquad$ C.树 $\qquad$ D.图

2. 设栈S和队列Q的初始状态均为空,元素a,b,c,d,e,f,g依次进入栈S.若每个元素出栈后立即进入队列Q,且7个元素出队的顺序是b,d,c,f,e,a,g,则栈S的容量至少是. \\
A.1 $\qquad$ B.2 $\qquad$ C.3 $\qquad$ D.4

3. 给定二叉树如图所示.设N 代表二叉树的根,L 代表根结点的左子树,R 代表根结点的右子树.若遍历后的结点序列是3,1,7,5,6,2,4,则其遍历方式是. \\
\begin{figure}[ht]
\centering
\includegraphics[width=5cm]{./figures/CSN09_1.png}
\caption{第3题图} \label{CSN09_fig1}
\end{figure}
A.LRN $\qquad$ B.NRL $\qquad$ C.RLN $\qquad$ D.RNL

4. 下列二叉排序树中,满足平衡二叉树定义的是______.\\
\begin{figure}[ht]
\centering
\includegraphics[width=14.25cm]{./figures/CSN09_2.png}
\caption{第4题图} \label{CSN09_fig2}
\end{figure}