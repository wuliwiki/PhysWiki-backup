% 圆锥曲线和圆锥

\pentry{平面旋转变换\upref{Rot2DT}, 椭圆的四种定义\upref{Elips3}, 抛物线的三种定义\upref{Para3}, 双曲线的三种定义\upref{Hypb3}}

圆锥曲线之所以叫做圆锥曲线, 是因为它们可以由平面截取圆锥面得到, 如\autoref{ConSec_fig1}. 然而由于这涉及较为繁琐的计算, 所以初学时

\begin{figure}[ht]
\centering
\includegraphics[width=6cm]{./figures/ConSec_1.png}
\caption{圆锥的有限截面是一个椭圆(来自 Wikipedia)}} \label{ConSec_fig1}
\end{figure}

在直角坐标系 $x$-$y$-$z$ 中, 为了方便我们使用顶角(两条母线的最大角度)为 $\pi/2$ 的圆锥, 其方程为
\begin{equation}\label{ConSec_eq4}
z_1^2 = x_1^2 + y_1^2
\end{equation}
对其他顶角的圆锥, 我们只需要把 $z$ 轴缩放一下即可.

\begin{figure}[ht]
\centering
\includegraphics[width=6cm]{./figures/ConSec_2.png}
\caption{\autoref{ConSec_eq4} 表示的圆锥面(修改自 Wikipedia)} \label{ConSec_fig2}
\end{figure}

我们可以再列出一个平面方程与\autoref{ConSec_eq4} 联立得到方程组, 但这样解出来的椭圆将与 $x$-$y$ 平面不平行. 所以更好的办法是先把圆锥旋转一下, 再用某个和 $x$-$y$ 平面平行的平面 $z = z_0$ 去截出椭圆. 关于 $y$ 轴的旋转变换\upref{Rot2DT}为
\begin{equation}
\pmat{x_1\\z_1} = \pmat{\cos\theta & -\sin\theta\\ \sin\theta & \cos\theta}\pmat{x\\z}
\end{equation}
\begin{equation}
y_1 = y
\end{equation}
代入\autoref{ConSec_eq4} 得
\begin{equation}
(\sin\theta\cdot x + \cos\theta\cdot z)^2 = (\cos\theta\cdot x - \sin\theta\cdot z)^2 + y^2
\end{equation}
这相当于把圆锥关于 $y$ 轴用右手定则\upref{RHRul}旋转了 $\theta$. 化成标准形式为
\begin{equation}
\frac{(x - \tan2\theta \cdot z)^2}{(z/\cos2\theta)^2} + \frac{y^2}{z^2/\cos2\theta} = 1
\end{equation}
离心率为 $\sqrt{2}\sin\theta$, 可见 $\theta \to \pi/4$ 的极限情况下, 离心率等于 $1$, 椭圆趋近于抛物线. 当 $\theta > \pi/4$ 时, 式中 $\cos2\theta < 0$, 上式变为双曲线方程.
