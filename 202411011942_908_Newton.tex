% 艾萨克·牛顿(综述)
% license CCBYSA3
% type Wiki

本文根据 CC-BY-SA 协议转载翻译自维基百科\href{https://en.wikipedia.org/wiki/Isaac_Newton}{相关文章})

\begin{figure}[ht]
\centering
\includegraphics[width=6cm]{./figures/0c83e6f3dfbe0a8c.png}
\caption{《46岁的牛顿肖像,1689年》} \label{fig_Newton_1}
\end{figure}
艾萨克·牛顿爵士,皇家学会会员(1642年12月25日-1726/27年3月20日[a]),是一位英国博学家,活跃于数学、物理学、天文学、炼金术、神学和写作领域,在他所在的时代被称为自然哲学家。他是科学革命及其后的启蒙运动中的关键人物。他的开创性著作《自然哲学的数学原理》首次出版于1687年,汇集了许多前人的研究成果,奠定了经典力学的基础。牛顿还在光学方面做出了开创性的贡献,并与德国数学家戈特弗里德·威廉·莱布尼茨共同被认为是微积分的创立者,尽管他在莱布尼茨之前几年就已发展了微积分。[10][11]