% 单位制和量纲
% keys 单位制|量纲|基本量类|导出量类

\subsection{单位制}
为明白制定单位制的动机,先来看一个例子.欧姆定律的数值表达式为
\begin{equation}\label{USD_eq1}
U=IR
\end{equation}
其中 $U,I,R$ 分别是以 \textbf{$\boldsymbol{V}$ }、\textbf{ $\boldsymbol{A}$ }和\textbf{ $\boldsymbol{\Omega}$ }为单位测量问题中的电压 $\boldsymbol{U},\boldsymbol{I},\boldsymbol{R}$ 所得的数.为明确起见,把\autoref{USD_eq1} 写为
\begin{equation}
U_{V}=I_{A}R_{\Omega}
\end{equation}
若以 \textbf{$\boldsymbol{mA}$} 测量电流,并把所得的数记作 $I_{mA}$ ,由\autoref{QCU_eq8}~\upref{QCU} 
\begin{equation}\label{USD_eq2}
I_{A}=\frac{\boldsymbol{mA}}{\boldsymbol{A}}I_{mA}=10^{-3}I_{mA}
\end{equation}
\autoref{USD_eq1} 代入 \autoref{USD_eq2} ,便得
\begin{equation}
U_{V}=10^{-3}I_{mA}R_{\Omega}
\end{equation}
为了简洁起见,通常都去掉下标,于是就有
\begin{equation}\label{USD_eq3}
U=10^{-3}IR
\end{equation}
\autoref{USD_eq1} 和\autoref{USD_eq3} 都称为欧姆定律,两者不同的原因在于采用不同的单位.