% 流形上的切空间
\pentry{流形\upref{Manif},切空间(欧几里得空间)\upref{tgSpaE}}

对于流形$N$,如果能将它嵌入到某个$\mathbb{R}^k$中,嵌入映射为$i:N\rightarrow\mathbb{R}^k$,那么根据\textbf{切空间(欧几里得空间)}\upref{tgSpaE}中关于曲面$S$的讨论,我们可以使用道路或者导子来算出特定嵌入$i$下流形$N$的切空间和切丛.但是和测试电荷、测试函数类似,特定嵌入也只是一个测试函数,我们讨论流形本身时不依赖特定的嵌入,这就体现出道路和导子定义的好处了.

和大多数教材不同的是,本书中使用道路的等价类来定义流形上的切向量,这样比起导子要更加容易可视化.

\subsection{流形上切空间的定义}

\begin{definition}{切向量}
给定流形$N$,则在其上一点$p\in N$处的一个切向量就是从$p$出发的一条道路$r$所在的等价类$[r]$.其中,两道路$r_1$和$r_2$等价当且仅当存在$p$处的一个图$(U, \varphi)$,使得道路$\varphi\circ r_1$和$\varphi\circ r_2$都收敛于$\varphi(U)$中的同一个切向量.向量的加法定义为$[r_1]+[r_2]=[\varphi^{-1}\circ(\varphi\circ r_1+\varphi\circ r_2)]$,即将代表元素用$\varphi$映射到欧几里得空间后相加,再映射回来的结果作为和的代表元素.向量的数乘类似,定义为$a[r]=[\varphi^{-1}\circ(a\varphi\circ r)]$
\end{definition}





