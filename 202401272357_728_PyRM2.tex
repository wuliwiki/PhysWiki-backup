% Python RoboMaster EP 教程—与机器人建立连接
% keys 机器人|Robomaster
% license Usr
% type Tutor

\begin{issues}
\issueDraft
\end{issues}

\pentry{Python 基础\upref{PyFi}}

Robomaster EP 支持3种与计算机连接的方式:WiFi 直连模式,WiFi 组网模式和 USB(RNDIS) 连接模式。

\begin{itemize}
\item \autoref{sub_PyRM2_1} WIFI直连
\item \autoref{sub_PyRM2_2} USB连接
\item \autoref{sub_PyRM2_3} 组网连接
\end{itemize}

\subsection{WIFI直连}\label{sub_PyRM2_1}

通过将机器人设置为直连模式,并连接机器人的 Wi-Fi 热点进行接入(智能中控顶部有热点名称与密码),Wi-Fi 直连模式下,机器人默认 IP 为 192.168.2.1。

开启机器人电源,切换智能中控的连接模式开关至 直连模式,如图1所示:

\begin{figure}[ht]
\centering
\includegraphics[width=4cm]{./figures/bde4a2d47ab0cf95.png}
\caption{直连模式} \label{fig_PyRM2_1}
\end{figure}

输入以下代码,可以测验是否连接

\begin{lstlisting}[language=python]
from robomaster import robot

import robomaster

if __name__ == '__main__':
    # 如果本地IP 自动获取不正确,手动指定本地IP地址
    # robomaster.config.LOCAL_IP_STR = "192.168.2.20"
    
    # 在程序内实例化Robot对象
    ep_robot = robot.Robot()

    # 程序与机器人联网并指定连接方式为AP 直连模式
    ep_robot.initialize(conn_type='ap')

    version = ep_robot.get_version()
    # 输出机器人固件版本号信息
    print("Robot version: {0}".format(version))
    # 关闭机器人
    ep_robot.close()
\end{lstlisting}

运行结果:
\begin{lstlisting}[language=pythonC]
Robot Version: xx.xx.xx.xx
\end{lstlisting}

\subsection{USB连接}\label{sub_PyRM2_2}

通过机器人的智能中控上的 USB 端口接入(机器人默认 IP 为 192.168.42.2)

USB 连接模式,实质上是使用 RNDIS 协议,将机器人上的 USB 设备虚拟为一张网卡设备, 通过 USB 发起 TCP/IP 连接更多 RNDIS 内容请参见\href{https://www.wikipedia.org/wiki/RNDIS}{RNDIS Wikipedia} 。

\begin{figure}[ht]
\centering
\includegraphics[width=12cm]{./figures/a8f72935358fa80b.png}
\caption{树莓派连接示意图} \label{fig_PyRM2_2}
\end{figure}

\begin{lstlisting}[language=python]
from robomaster import robot

import robomaster

if __name__ == '__main__':
    # 在程序内实例化Robot对象
    ep_robot = robot.Robot()

    # 程序与机器人联网并指定连接方式为USB RNDIS模式
    ep_robot.initialize(conn_type='rndis')

    version = ep_robot.get_version()
    # 输出机器人固件版本号信息
    print("Robot version: {0}".format(version))
    # 关闭机器人
    ep_robot.close()
\end{lstlisting}

运行结果:
\begin{lstlisting}[language=pythonC]
Robot Version: xx.xx.xx.xx
\end{lstlisting}

\subsection{组网连接}\label{sub_PyRM2_3}

通过将机器人设置为组网模式,并将计算设备与机器人加入到同一个局域网内,实现组网连接。

开启机器人电源,切换智能中控的连接模式开关至 组网模式,如图3所示:

\begin{figure}[ht]
\centering
\includegraphics[width=4cm]{./figures/6a3c7581be4b6cc2.png}
\caption{组网模式} \label{fig_PyRM2_3}
\end{figure}

安装myqr库生成二维码,按 win+r,在弹出窗口中输入 cmd 打开命令提示符界面,在命令行里面输入:

\begin{lstlisting}[language=bash]
pip install myqr
\end{lstlisting}

生成图片:

\begin{lstlisting}[language=python]
from robomaster import conn
from MyQR import myqr
from PIL import Image

import time
import robomaster

QRCODE_NAME = "qrcode.png"

if __name__ == '__main__':

    helper = conn.ConnectionHelper()
    info = helper.build_qrcode_strin(ssid="Wifi 名称", password="Wifi 密码")
    myqr.run(words=info)
    time.sleep(1)
    img = Image.open(QRCODE_NAME)
    img.show()
    if helper.wait_for_connection():
        print("Connected!")
    else:
        print("Connect failed!")
\end{lstlisting}

按下机器人智能中控上的扫码连接按键,扫描二维码进行组网连接。

\begin{figure}[ht]
\centering
\includegraphics[width=8cm]{./figures/8f3d943c7e5f948e.png}
\caption{} \label{fig_PyRM2_4}
\end{figure}

成功连接时,运行结果:
\begin{lstlisting}[language=pythonC]
Connected!
\end{lstlisting}

同时机器人的灯效变为白色呼吸变为青绿色常亮。

未完成

支持在组网模式下通过SN连接指定的机器人,用户通过在初始化时给 sn 参数赋值完成对机器人 sn 的输入。在不指定 sn 时,SDK默认与搜索到的第一台机器人建立连接。

\begin{lstlisting}[language=python]
from robomaster import robot
from robomaster import config


if __name__ == '__main__':
    ep_robot = robot.Robot()
    # 指定机器人的 SN 号
    ep_robot.initialize(conn_type="sta", sn="3JKDH2T001ULTD")

    ep_version = ep_robot.get_version()
    print("Robot Version: {0}".format(ep_version))

    ep_robot.close()
\end{lstlisting}
