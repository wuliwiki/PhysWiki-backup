% 运动学
% license CCBYSA3
% type Wiki

(本文根据 CC-BY-SA 协议转载自原搜狗科学百科对英文维基百科的翻译)

\textbf{运动学}是经典力学的一个分支,描述了点、体(对象)和体系统(对象组)的运动,而不考虑引起运动的力。[1][2][3]运动学作为一个研究领域,通常被称为“运动的几何”,偶尔也被视为数学的一个分支。[4][5][6]运动学问题首先描述系统的几何形状,并声明系统内任何已知点的位置、速度和/或加速度值的初始条件。然后,使用几何参数,可以确定系统任何未知部分的位置、速度和加速度。对力如何作用于物体的研究属于动力学范畴,而不是运动学。有关更多详细信息,请参见分析动力学。

运动学在天体物理学中用于描述天体的运动和这些天体的集合。在机械工程、机器人学和生物力学中[7]运动学用于描述由连接部件(多连杆系统)组成的系统的运动,例如发动机、机械臂或人体骨骼。

几何变换,也称为刚性变换,用于描述机械系统中部件的运动,简化了运动方程的推导。它们也是动态分析的核心。

运动学分析是测量用于描述运动的运动学量的过程。例如,在工程中,运动学分析可用于找到给定机构的运动范围,并反向工作,使用运动学综合来设计所需运动范围的机构。[8] 此外,运动学将代数几何学应用于研究机械系统或机构的机械效益。

\subsection{词源}



\subsection{非旋转参照系中粒子轨迹的运动学}



\subsubsection{2.1 速度和速度}



\subsubsection{2.2 加速度}



\subsubsection{2.3 相对位置向量}



\subsubsection{2.4 相对速度}



\subsubsection{2.5 相对加速度}



\subsection{恒定加速度下的粒子轨迹}



\subsection{圆柱-极坐标中的粒子轨迹}



\subsubsection{4.1 恒定半径}



\subsubsection{4.2 平面圆形轨迹}



\subsection{物体在平面内运动的点轨迹}



\subsubsection{5.1 位移和运动}



\subsubsection{5.2 矩阵表示}



\subsection{纯平移}



\subsection{ 物体围绕固定轴的旋转}



\subsection{物体三维运动的点轨迹}



\subsubsection{8.1 位置}



\subsubsection{8.2 速度}



\subsubsection{8.3 加速度}




