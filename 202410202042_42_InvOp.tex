% 逆算子
% keys 逆算子|可逆性
% license Usr
% type Tutor

\pentry{拓扑线性空间中的线性算子\nref{nod_TLinO}}{nod_d7fd}

逆算子是双射的算子的逆。当算子可逆时,在某些情形下逆算子和算子有很多相同的性质,比如,线性算子的逆算子时线性的,而完备赋范空间之间的线性有界算子的逆是有界的。


\begin{definition}{逆算子}
设 $D_A$ 是拓扑线性空间的子集,$A:D_A\rightarrow E_1$ 是其上的算子(映射),$\Im A$ 是 $A$ 的象。其对任意 $y\in\Im A$,方程
\begin{equation}
Ax=y~
\end{equation}
 有唯一解,则称算子 $A$ \textbf{可逆}。此时映射
 \begin{equation}
 \Im A\rightarrow E,Ax\mapsto x~  
 \end{equation}
 称为 $A$ 的\textbf{逆算子},记作 $A^{-1}$。

\end{definition}

\subsection{性质}
下面定理表明,线性算子的逆算子是线性的。

\begin{theorem}{}
线性算子 $A$ 的逆算子 $A^{-1}$ 是线性的。
\end{theorem}

\textbf{证明:}
任意 $\Im A$ 中的 $y_1=A x_1,y_2=Ax_2$,因为
\begin{equation}
A(\alpha x_1+\beta x_2)=\alpha Ax_1+\beta Ax_2=\alpha y_1+\beta y_2.~
\end{equation}
所以
\begin{equation}
A^{-1}(\alpha y_1+\beta y_2)=\alpha x_1+\beta x_2=\alpha A^{-1}y_1+\beta A^{-1}y_2.~
\end{equation}


\textbf{证毕!}

下面定理表明,在完备赋范空间(Banach空间)之间的线性有界算子的逆算子是有界的。

\begin{theorem}{逆算子的Banach}
设 $E,E_1$ 是\enref{完备赋范空间}{banach},则
\end{theorem}



