% 二次型
% 二次型|规范型|对角型

\begin{issues}
\issueTODO
\end{issues}

\pentry{双线性型(2-线性函数)\upref{Tensor}}
\begin{definition}{二次型}
域 $\mathbb{F}$ 上有限维空间 $V$ 上的函数 $q:V\rightarrow\mathbb{F}$ ,若它满足如下两个性质:
\begin{enumerate}
\item $q(-\bvec{v})=q(\bvec v),\quad \forall\bvec v\in V$;
\item 由公式
\begin{equation}\label{QuaFor_eq1}
f(\bvec x,\bvec y)=\frac{1}{2}\qty[q(\bvec x+\bvec y)-q(\bvec x)-q(\bvec y)]
\end{equation}
决定的映射 $f:V\times V\rightarrow\mathbb{F}$ 是 $V$ 上的双线性型(显然是对称的).
\end{enumerate}
则称 $q$ 是 $V$ 上的\textbf{二次型},并称 $f$ 的秩为 $q$ 的秩:rank $q$=rank $f$.
\end{definition}
利用\autoref{QuaFor_eq1} ,由 $q$ 得到的对称的双线性型 $f$ 称为\textbf{极化的},或 $f$ 是与二次型 $q$ \textbf{配极} 的双线性型.
\begin{example}{}
设 $f$ 是 $V$ 上任意一个对称的对称的双线性型,令
\begin{equation}
q_f(\bvec x)=f(\bvec x,\bvec x)
\end{equation}
就得到一个满足二次型定义的函数 $q_f:V\rightarrow\mathbb{F}$ ,因为
\begin{equation}
\begin{aligned}
q_f(-\bvec{x})&=f(-\bvec{x},-\bvec{x})=f(\bvec{x},\bvec{x})=q_f(\bvec{x}) \quad \forall\bvec x\in V\\
f(\bvec x,\bvec y)&=\frac{1}{2}\qty[f(\bvec x+\bvec y,\bvec x+\bvec y)-f(\bvec x,\bvec x)-f(\bvec y,\bvec y)]\\
&=\frac{1}{2}\qty[q(\bvec x+\bvec y)-q(\bvec x)-q(\bvec y)]
\end{aligned}
\end{equation}
\end{example}
\begin{theorem}{}
每一个二次型 $q$ 都可以按着自己的配极双线性型 $f$ 唯一地恢复原型;换言之, $q=q_f$
\end{theorem}
\textbf{证明:}在\autoref{QuaFor_eq1} 中令 $\bvec y=-\bvec x$ :
\begin{equation}
-f(\bevc x,\bvec x)=\frac{1}{2}[q(\bvec 0)-q(\bvec x)-q(-\bvec x)]
\end{equation}
从而
\begin{equation}
q(\bvec x)=f(\bvec x,\bvec x)+\frac{1}{2}q(\bvec 0)
\end{equation}
因为 $f$ 是个双线性型,所以 $f(\bvec 0,\bvec 0)=0$ .因为,当 $\bvec x=0$ 时有 $q(\bvec 0)=\frac{1}{2}q(\bvec 0)$ ,即 $q(\bvec 0)=0$,也就是说, $q()$