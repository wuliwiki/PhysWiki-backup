% 粒子物理学
% license CCBYSA3
% type Wiki

(本文根据 CC-BY-SA 协议转载自原搜狗科学百科对英文维基百科的翻译)

\begin{figure}[ht]
\centering
\includegraphics[width=10cm]{./figures/f8fb458743b7d92a.png}
\caption{Higgs粒子实验} \label{fig_LZWLX_2}
\end{figure}

\textbf{粒子物理(又称作“高能物理”)}是研究基本粒子之间相互作用、相互转化规律的科学,其主要目的是要找到一种既简单又普遍的物理原理来统一解释基本粒子之间五花八门的相互作用、相互转换现象[1],其研究对象就是物质的基本结构和基本相互作用。

\subsection{概论}
在过去的一百多年里,我们对基本粒子性质的认识有了长足的进步,建立发展并逐步完善了粒子物理标准模型,粒子物理标准模型的预测与实验测量达到了惊人的吻合程度。然而,时至今日,我们并没有找到一个能够统一描述所有粒子以及所有相互作用的理论。

自然界中已知的四种基本相互作用是引力相互作用、电磁相互作用、弱相互作用以及强相互作用,每一种相互作用的特点如表1.1所示。
\begin{table}[h]
\centering
\caption{四种基本相互作用性质比较}
\begin{tabular}{|c|c|c|c|c|c|}
\hline
\textbf{相互作用} & \textbf{强度} & \textbf{力程} & \textbf{媒介粒子} & \textbf{参与作用粒子} & \textbf{末端态 }\\
\hline
强相互作用 & 1 & $10^{-15}$ m & 胶子 & 夸克, 胶子 & 强子 \\
\hline
弱相互作用 & $10^{-5}$ & $<10^{-17}$ m & $W^{\pm}, Z^0$ & 夸克, 电子, 中微子等 & 无 \\
\hline
电磁相互作用 & $1/137$ & $F \propto 1/r^2$ & 光子 & 所有带电粒子 & 原子等 \\
\hline
引力 & $10^{-39}$ & $F \propto 1/r^2$ & 引力子? & 所有粒子 & 太阳系等 \\
\hline
\end{tabular}
\end{table}
从上表可以看出,强相互作用与弱相互作用是短程力,其有效范围不会超出原子核的尺度($10^{-15} m$ ),因此其效应并不会在宏观中有所体现,而在日常生活中有所体现的引力和电磁相互作用是长程力,粒子物理的主要研究对象就是除引力外的其它三种相互作用。2012年LHC发现Higgs玻色子后,有人把Higgs玻色子通过“汤川耦合”使费米子获得质量的作用成为“第五种相互作用”。[2]

\subsection{ 对称性和守恒律}
对称性在现代物理学的研究中起着至关重要的作用,而对称性的破缺在粒子物理的研究中尤其重要。比如,希格斯机制就与对称性自发破缺有着密切的联系。物理学在这方面探索的一个重要进展是建立了艾米·诺特定理,这个定理首先是在经典物理学中给出的,后来推广到量子物理范围内也得到了普遍证明。

Noether定理[2]:\textbf{如果运动规律在某一不明显依赖于时间的变换下具有不变性,则必然存在一种对应的守恒定律。}

比如说,经典力学中的动量守恒定律、角动量守恒定律以及能量守恒定律,分别源于空间平移不变性、空间转动不变性以及时间平移不变性。在量子物理的情况下也有相似的结论。

\subsection{粒子的分类}
按照是否参与强相互作用,粒子可以本分为两类:轻子和强子。前者不参与强相互作用而后者参与。

\subsubsection{3.1 轻子}
轻子是不直接参与强相互作用的粒子,它们可以直接参与电磁相互作用和弱相互作用,比如电子等。目前发现的轻子一共有六种:
\begin{table}[h!]
\centering
\caption{ 轻子}
\begin{tabular}{|c|c|c|} 
\hline
$e^-$ & $\mu^-$ & $\tau^-$ \\
\hline
$\nu_e$ & $\nu_\mu$ & $\nu_\tau$ \\
\hline
\end{tabular}
\end{table}
及上表中粒子的反粒子。这些粒子都是自旋为1/2的费米子。
\subsubsection{3.2 强子}
强子包含两类粒子,重子和介子。人们在20世纪30年代就已经知道了原子核是由带电的质子和不带电的中子构成,其中质子和中子属于重子。质子中子通过核力结合在一起构成原子核。随着对核力的研究,人们又发现了$\pi$介子、$K$介子、$\Lambda$奇异重子等。20世纪50年代以后,加速器建成,人们发现了更多的新粒子,包括一系列共振态粒子,如$\rho$介子等。
为了解释数量繁多、性质各异的强子,Gell-Mann和Zweig在1964年提出了夸克模型[3]。夸克模型认为,实验上发现的强子都是复合粒子,是由更基本的粒子夸克构成的。夸克模型在解释当时实验上观测到的强子性质上获得了巨大的成功。

夸克模型最开始只假设了三种夸克,分别是:上夸克、下夸克和奇异夸克。后来又发现了另外三种夸克:粲夸克、底夸克以及顶夸克。粲夸克的存在首先在1974年被丁肇中领导的小组在美国布鲁克海文国家实验室[4]和B. Richter领导的小组在美国斯坦福直线加速器中心[5]同时观测证实。1977年L.Lederman领导的小组在美国费米实验室的质子加速器上发现了两个长寿粒子[6],进而证实了底夸克的存在。1994年,费米实验室发现了顶夸克。至此,一共发现了上述六种夸克,并把每种夸克叫做“味”。注意,此处的“味”只是对夸克种类起的名字,与生活中的“味道”没有任何关系。

\subsection{标准模型}
粒子物理的标准模型理论包含了电弱统一理论和量子色动力学两个大部分。

\subsubsection{4.1 电弱统一理论}
电子以及其它带电轻子存在电磁相互作用,描述电磁相互作用的量子理论叫做\textbf{量子电动力学(QED, Quantum Electrodynamics)}。量子电动力学能够描述电子、正电子的产生和湮灭,也能描述光子的发射和吸收。这个理论是20世纪40年代后半叶只20世纪50年代由费曼(R. Feynman)、施温格(J. Schwinger)和朝永振一郎(S. Tomonaga)发展起来的。他们发展出重整化方法,消除了计算中出现的无穷大,不仅提高了计算精度,也给出了物理上的可观测效应。可以说,重整化是的QED的微扰计算变得有意义。

弱相互作用的存在是19世纪末首先在原子核的$\beta$衰变中发现的。此后,为了解释$\beta$能谱的连续性问题,泡利提出了中微子假设。1956年,李政道和杨振宁首次提出弱相互作用下宇称不守恒,并在1957年得到实验验证。这促使费曼和盖尔曼于1958年提出了关于弱相互作用的普适的V-A理论,并取得了成功。V-A理论的提出,使弱相互作用衰变过程的计算有了坚实的理论基础。

然而,上述的V-A理论在质心系能量达到大约300GeV时,计算的电子中微子散射过程的截面就将破坏几率守恒。为了克服这一困难,李政道、Rosenbluth和杨振宁类比电磁相互作用,假定弱相互作用也存在类似光子的中间玻色子$W$。弱相互作用和电磁相互作用表面是有明显的不同,但是却可以在Yang-Mills理论和Higgs机制下统一地来描述,即格拉肖-温伯格-萨拉姆弱电统一理论。

\subsubsection{4.2 量子色动力学}
量子色动力学(Quantum Chromodynamics,QCD)是描述强相互作用的理论,属于非阿贝尔规范理论,是在阿德勒、伯约肯以及其他人工作的基础上,在1972年由盖尔曼和弗里奇所创立。量子色动力学的基本自由度是夸克和胶子。QCD有两个重要特征:在高能量时夸克和胶子之间的相互作用非常弱,而且随着能量的提高,相互作用越来越弱,即所谓的渐进自由(渐进自由是非阿贝尔规范理论的一个独特性质);在低能量时夸克和胶子的相互作用较强,存在色禁闭现象。色禁闭的现象就是在实验上无法找到单独存在的夸克,只能发现夸克形成的强子。理论上通过QCD可以得到夸克相互作用的行为,但是由于数学上的复杂性,目前还无法通过QCD直接得到色禁闭这一特征。

\subsubsection{4.3 标准模型中的粒子}
\begin{figure}[ht]
\centering
\includegraphics[width=14.25cm]{./figures/88e847420d6dca68.png}
\caption{粒子物理标准模型} \label{fig_LZWLX_1}
\end{figure}
标准模型中共有四大类粒子,分别是:夸克、轻子、规范玻色子以及标量玻色子。
\begin{enumerate}
\item 夸克是强相互作用的基本自由度之一,一共有六种,分别叫做:上夸克、下夸克、粲夸克、奇异夸克、顶夸克以及底夸克。夸克带有分数电荷,比如说,上夸克带有电荷,而下夸克带有电荷。除此之外,夸克还带有“色”荷:色荷有三种,分别叫做“红”、“绿”、“蓝”,色荷与强相互作用的关系,就类似于电荷与电磁相互作用的关系。注意,此处的“色荷”以及“红绿蓝”只是给夸克的性质起的名字,跟宏观中的颜色没有任何关系。一对正反夸克可以组成介子,或者三个夸克(反夸克)组成重子,介子和统称强子,比如,质子就是由两个上夸克和一个下夸克构成的;
\item 轻子包含了电子、电子中微子、$\mu$子、$\mu$子中微子、$\tau$子、$\tau$子中微子,一共也是六种。轻子不参与强相互作用,只参与弱相互作用和电磁相互作用;
\item 规范玻色子,是相互作用中的媒介粒子,包括光子、$W^\pm/Z$以及胶子,其中光子是电磁相互作用的媒介粒子,$W^\pm/Z$是弱相互作用的媒介粒子,胶子传递强相互作用。
\item 标量玻色子,目前只有Higgs粒子,Higgs粒子是Higgs机制预言的粒子,轻子和夸克通过与Higgs粒子相互作用(并在自发对称性破缺的条件下)获得质量。Higgs粒子于2012年在欧洲核子中心被发现[7],是标准模型中最后一个被发现的粒子。
\end{enumerate}

\subsection{参考文献}
[1]
^杜东生 杨茂志 (2014). 粒子物理导论. 科学出版社. p. 1. ISBN 978-7-03-042935-3..

[2]
^肖振军 吕才典 (2016). 粒子物理学导论. 科学出版社. p. 1. ISBN 978-7-03-047551-0..

[3]
^Gell-Mann M..《A schematic model of baryons and mesons》.Phys. Rev. Lett,1964,214-215.

[4]
^Aubert J. J.《Experimental Observation of a Heavy Particle J》.Phys. Rev. Lett,1974,1404.

[5]
^J.-E. Augustin, et a.《Discovery of a Narrow Resonance in e+e− Annihilation》.Phys. Rev. Lett,1974,1406.

[6]
^Herb S W, et al.《Observation of a Dimuon Resonance at 9.5 GeV in 400-GeV Proton-Nucleus Collisions》.Phys. Rev. Lett,1977,252-255.

[7]
^New results indicate that new particle is a Higgs boson.CERN.