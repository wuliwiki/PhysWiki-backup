% 常微分方程解的存在、唯一及对参数的连续依赖定理
% keys 存在|唯一|连续依赖|常微分方程
% license Xiao
% type Tutor

\pentry{皮卡映射\upref{PicMap},基本知识(常微分方程)\upref{ODEPr}}
本节证明常微分方程的解的存在、唯一、及对参数的连续依赖定理。所谓 “解对参数的连续依赖” 是指微分方程
\begin{equation}\label{eq_ODEUC_2}
\dot x=v(x,t)~,
\end{equation}
 的解 $\varphi$ 也是某些参数 $\mu=(\mu_1,\cdots,\mu_m)$ 的函数,即 $\varphi(\mu,t)$。于是 \autoref{eq_ODEUC_2} 右边也因写成 $v(x,\mu,t)$。 即需要证明形如
\begin{equation}\label{eq_ODEUC_1}
\dot x=v(x,\mu,t)~
\end{equation}
的微分方程的存在唯一且对参数 $\mu$ 的连续依赖定理。然而,可以验证,对\autoref{eq_ODEUC_2} 证明存在唯一性和对初始点的连续依赖性,等价于证明\autoref{eq_ODEUC_1} 的存在唯一性和对参数 $\mu$ 的连续依赖性。事实上:
若微分方程\autoref{eq_ODEUC_2} 解 $\varphi$ 存在唯一且对初始点 $x$ 的连续依赖。记 
\begin{equation}
y=(x,\mu),\quad f=(v_x,v_{\mu})=(v,v_{\mu})~,
\end{equation}
则 \autoref{eq_ODEUC_1} 等价于(初始 $\mu$ 分量为 $\mu$ 的)微分方程
\begin{equation}\label{eq_ODEUC_3}
\begin{aligned}
\dot y=f(y,t)=(v(y,t),0)~.
\end{aligned}
\end{equation}
由假设,\autoref{eq_ODEUC_3} 的解 $\varphi(t)$存在唯一,且连续依赖于起始点 $y$(即解可写为 $\varphi(y,t)\equiv\varphi(t)$ 且 $\varphi(y,t_0)=y$),于是解也就连续依赖于 $\mu$;
反之,如果对\autoref{eq_ODEUC_1} 的微分方程存在唯一及对参数 $\mu$ 连续依赖的定理成立,则由 $\mu$ 是参数,可令 $v_\mu(x,t)=\equiv v(x,\mu,t)$ 则\autoref{eq_ODEUC_1} 等价于
\begin{equation}\label{eq_ODEUC_4} 
\dot x=v_{\mu}(x,t)~.
\end{equation}
由假定,其解存在唯一且对参数 $\mu$ 连续。于是设取参数 $\mu$ 对应起始点 $x_0$ 的情形,于是微分方程\autoref{eq_ODEUC_4} 的解存在唯一且对起始点连续。

一般的常微分方程都可以写为\autoref{eq_ODEUC_2} 的形式,并通过上面考虑,我们只需要证明对形为\autoref{eq_ODEUC_2} 的微分方程的解存在唯一且连续依赖于起始点即可。

\subsection{存在、唯一及对参数的连续依赖定理}
\begin{theorem}{存在、唯一及对参数的连续依赖定理}
设微分方程
\begin{equation}
\dot x=v(x,t)~
\end{equation}
中向量场 $v$ 及其关于变量 $x$ 的导数 $v_{*x}\equiv(\pdv{v^i}{x^j})$ 在扩张相空间(\autoref{def_PSaPF_2}~\upref{PSaPF})中的一区域 $U$ 上有定义且连续可微,则对任一点 $(t_0,x_0)\in U$,存在保持 $t_0$ 不变的充分接近点 $x_0$ 的任一给定点 $x$ $U$ 中 的一邻域 $M$,使得任一给定的 $x\in M$,存在点 $U$ 中点 $(t_0,x)$ 的一领域 $U'$, 满足初始条件 $\varphi(t_0)=x$ 的微分方程
\begin{equation}
\dot x=v(x,t)~
\end{equation}
的解 $\varphi(t)$,且这个解
\begin{equation}
\abs{}~
\end{equation}

\end{theorem}
