% 威廉·爱德华·韦伯(综述)
% license CCBYSA3
% type Wiki

本文根据 CC-BY-SA 协议转载翻译自维基百科\href{https://en.wikipedia.org/wiki/Wilhelm_Eduard_Weber}{相关文章}。

\begin{figure}[ht]
\centering
\includegraphics[width=6cm]{./figures/89cd851af1488861.png}
\caption{戈特利布·比尔曼(Gottlieb Biermann)所绘的韦伯(Weber)肖像。} \label{fig_Eduard_1}
\end{figure}
威廉·爱德华·韦伯(Wilhelm Eduard Weber,/ˈveɪbər/[1],德语发音:[ˈvɪlhɛlm ˈeːdu̯aʁt ˈveːbɐ];1804年10月24日—1891年6月23日),德国物理学家,与卡尔·弗里德里希·高斯共同发明了世界上第一台电磁电报。
\subsection{传记}  
\subsubsection{早年经历 } 
韦伯出生在维滕贝格的 Schlossstrasse(城堡街),他的父亲 迈克尔·韦伯(Michael Weber)是该地的神学教授。这栋建筑曾是 亚伯拉罕·瓦特(Abraham Vater)曾经的住所。[2]  

威廉是家中三兄弟中的老二,三兄弟都以出色的科学天赋闻名。1815年,随着维滕贝格大学的解散,他的父亲被调往哈雷。威廉最初由父亲教授基础课程,随后进入哈雷的孤儿院与文法学校(Orphan Asylum and Grammar School)学习。之后,他进入大学,专攻自然哲学。他在课堂上表现突出,并通过原创研究崭露头角。取得博士学位并成为 私人讲师(Privatdozent)后,他被任命为哈雷大学的 特别自然哲学教授(Professor Extraordinary of Natural Philosophy)。
\subsubsection{职业生涯}
1831年,在 卡尔·弗里德里希·高斯 的推荐下,年仅27岁的韦伯被 哥廷根大学 聘为物理学教授。他的讲座内容生动有趣,富有启发性和教育意义。韦伯认为,仅靠课堂讲授(即使配有实验演示)并不足以让学生真正掌握物理并将其应用于日常生活,因此,他鼓励学生免费使用学院实验室,亲自动手做实验。  

20岁时,他与哥哥 恩斯特·海因里希·韦伯(Ernst Heinrich Weber,时任莱比锡大学解剖学教授)合著了一本关于 波动理论与流体性质 的书,这本书为他们兄弟赢得了相当大的声誉。  

声学是韦伯特别热衷的研究领域,他在《波根多夫年鉴》(Poggendorffs Annalen)、《施魏格化学与物理年鉴》(Schweigger's Jahrbücher für Chemie und Physik)以及音乐期刊《Carcilia》上发表了大量相关论文。  

“人体步态机制”是他与弟弟爱德华·韦伯(Eduard Weber)共同研究的另一重要课题。这项重要的研究成果发表于1825年至1838年间。  

1833年,高斯与韦伯共同建造了世界上第一台电磁电报,将哥廷根的天文台与物理研究所连接起来。

1837年12月,由于政治原因,汉诺威政府将韦伯(作为哥廷根七教授之一)解除哥廷根大学的教职。之后,韦伯开始了一段时间的旅行,期间曾访问英国等国家。1843年至1849年,他担任莱比锡大学的物理学教授,直到1849年他被恢复哥廷根大学的职位。  

他最重要的作品之一是与卡尔·弗里德里希·高斯和卡尔·沃尔夫冈·本雅明·戈尔德施密特(Carl Wolfgang Benjamin Goldschmidt)共同完成的《地磁图集:根据理论要素设计》(Atlas des Erdmagnetismus: nach den Elementen der Theorie entworfen),[3][4] 这是一系列地磁图。正是由于他的努力,世界各地陆续建立了地磁观测站。  

他与高斯共同研究磁学,并于1864年发表了《电动力学比例测量》,提出了一套电流的绝对测量体系,这套体系成为今天电流测量的基础。  

韦伯逝世于哥廷根,并与马克斯·普朗克和马克斯·玻恩葬在同一座墓地。

他于1855年当选为瑞典皇家科学院的外籍院士。

1855年,他与鲁道夫·科尔劳施(Rudolf Kohlrausch,1809–1858)共同证明,静电单位与电磁单位之比,数值上与光速非常接近。[5] 这一发现促使麦克斯韦提出了光是电磁波的猜想。这一研究还促使韦伯发展了他的电动力学理论。此外,最早使用字母“c”表示光速,也是出现在1856年科尔劳施和韦伯合著的论文中。
\subsection{国际认可}  
国际单位制(SI)中磁通量的单位——韦伯(weber,符号:Wb),就是以他的名字命名的。
\subsection{作品}
\begin{figure}[ht]
\centering
\includegraphics[width=6cm]{./figures/7545fc2059ccaaee.png}
\caption{威廉·韦伯故居,维滕贝格 城堡街(Schlossstrasse)14号、15号。} \label{fig_Eduard_2}
\end{figure}
\begin{itemize}
\item 《电动力学测量:特别是将电流强度的测量归结为机械单位》(Elektrodynamische Maaßbestimmungen: insbesondere Zurückführung der Stromintensitäts-Messungen auf mechanisches Maass)(与威廉·韦伯合著),1857年。  
   “Electrodynamic Measurements, Especially Attributing Mechanical Units to Measures of Current Intensity”(德文原文及英文翻译)。
\item 《声学、力学、光学与热学》(Akustik, Mechanik, Optik und Wärmelehre),德文,柏林:Springer出版社,1892年。
\item 《波动理论》(Wellenlehre),德文,柏林:Springer出版社,1893年。
\item 《电流现象与电动力学》(Galvanismus und Elektrodynamik),德文,柏林:Springer出版社,1894年。
\item 《人体步态器官的力学》(Mechanik der menschlichen Gehwerkzeuge),德文,柏林:Springer出版社,1894年。
\end{itemize}
\begin{figure}[ht]
\centering
\includegraphics[width=6cm]{./figures/3f26078216f82e67.png}
\caption{《波动理论》,1893年} \label{fig_Eduard_3}
\end{figure}
\subsection{另见}  
\begin{figure}[ht]
\centering
\includegraphics[width=6cm]{./figures/b06625d561ce6e86.png}
\caption{维滕贝格邮局的 威廉·韦伯纪念碑} \label{fig_Eduard_4}
\end{figure}
\begin{itemize}
\item 德国发明家与发现者  
\item 国际电学与磁学单位制  
\item 双线圈  
\item 指针电报  
\item 矢量磁势  
\item 韦伯电动力学
\end{itemize}