% 函数视角下的三角函数(高中)
% keys 函数|三角函数|周期|性质
% license Usr
% type Tutor

\begin{issues}
\issueDraft
\end{issues}

\pentry{三角函数\nref{nod_HsTrFu},函数\nref{nod_functi},函数的性质\nref{nod_HsFunC},导数\nref{nod_HsDerv},导数的计算\nref{nod_HsDerB}}{nod_5a43}

在前面的内容中,已经接触过三角函数的定义,并基于这些定义推导出了诱导公式及同角三角函数之间的关系。这些推导主要依赖于任意角和三角函数的几何定义。然而,三角函数不仅仅是几何分析的工具,它们本质上也是一种函数,并具备一般函数的基本性质,如周期性、单调性和对称性。因此,本文将从函数的角度进一步分析三角函数,考察它们的性质、图像、变化趋势等。需要注意的是,这些视角本质上是等价的,它们都在描述同一数学对象。无论是几何定义还是函数分析,最终指向的都是相同的数学结构。这种多重视角的统一性,正是三角函数作为数学工具的强大之处。它不仅能够通过直观的几何形式展现对称性和变换规律,也能在函数的框架下揭示更广泛的性质,为各种数学应用提供坚实的基础。

另外,在三角函数的介绍中,有一个广为流传的动画:一个点在单位圆上运动,表示角度的变化,同时,在单位圆的右侧和上侧,将角度与对应的线段长度映射到另一坐标系,从而自然引出各个三角函数的图像。尽管这种动画能够直观展示三角函数的变化过程,更理想的方式是能够在脑海中主动演练这一过程。看到函数图像时,能够自动联想到单位圆上的点如何旋转;反之,观察圆周运动时,能够迅速在脑海中构建出相应的函数图像。这种能力不仅有助于理解三角函数的本质,也将在更深入的数学学习中提供帮助。本文内容主要关注正弦、余弦与正切函数,其余三角函数由于与它们存在倒数关系,将适当涉及,但不会展开详细推导。

\subsection{三角函数的性质}
下面照常tao l
\subsubsection{定义域}
由于三角函数的自变量是任意角,因此理论上,它们的定义域应覆盖整个实数集。然而,在之前的讨论中\aref{提及}{eq_HsTrFu_13}过,某些三角函数在特定角度下无意义。例如,$\tan x$ 在 $\displaystyle x=\frac{\pi}{2}+k\pi, (k\in\mathbb{Z})$ 处没有定义。类似地,其他三角函数也存在某些不可取值的点。因此,它们的定义域分别为:
\begin{itemize}
\item $\sin x,\cos x$ 的定义域为 $\mathbb{R}$;
\item $\tan x,\sec x$ 的定义域为 $\displaystyle\{x|x\neq\frac{\pi}{2}+k\pi,k\in\mathbb{Z}\}$,或写作$\displaystyle\{x|x\neq(2k+1)\frac{\pi}{2},k\in\mathbb{Z}\}$,即 $x$ 不能取 $\frac{\pi}{2}$ 的奇数倍;
\item $\cot x,\csc x$ 的定义域为 $\displaystyle\{x|x\neq k\pi,k\in\mathbb{Z}\}$,即 $x$ 不能取 $\pi$ 的整数倍。
\end{itemize}
值得注意的是,三角函数的这些奇异点(即定义域中无法取到的点)与后面将讨论的周期性存在紧密联系。
\subsubsection{奇偶性}

\subsubsection{周期性}
正弦函数、余弦函数、正切函数的都是周期函数,根据定义易得,正弦函数和余弦函数,周期为 $2k\pi(k\in Z,k\neq0)$,正切函数的周期为 $k\pi(k\in Z,k\neq0)$.

\addTODO{如何证明最小周期}
\subsubsection{导数及单调性}

\subsection{值域}

正弦和余弦函数的值域相对直观,它们对应于单位圆上点的纵坐标和横坐标,因此取值范围显然是 $[-1,1]$。

相比之下,正切函数的值域分析稍显复杂。根据几何定义,在锐角情况下,正切函数对应的线段长度受终边位置影响。参见\aref{几何示意图}{fig_HsTrFu_1},线段的一端是固定点 $X_0$ ,而长度取决于另一端$T$的移动情况。分析 $\displaystyle x\in\left[0,\frac{\pi}{2}\right)$ 时的情形:
\begin{itemize}
\item 当 $x=0$ 时,角的终边与 $x$ 轴重合,线段两个端点也重合,长度为 $0$;
\item 随着 $x$ 增大,终边与 $x$ 轴夹角增加,点 $T$ 沿着单位圆向上移动,可以取到直线$x=1$在第一象限中的所有点,使得对应的线段长度不断增加;
\item 当 $x=\frac{\pi}{2}$ 时,终边与 $y$ 轴重合,二者平行无交点。
\end{itemize}
同理,利用对称性,由 $\tan(-x) = -\tan x$ 可知,$\displaystyle x\in\left(-{\pi\over2},0\right]$ 时,正切函数的取值是第四象限中对应的所有点。综上所述,$\tan x$ 在 $\displaystyle \left(-\frac{\pi}{2},\frac{\pi}{2}\right)$ 内可以遍历所有实数。

这一点也可以由极限分析得到\footnote{下述分析较为严谨,但在高中阶段不作要求。}。由于 $\displaystyle\tan x = \frac{\sin x}{\cos x}$,而 $\cos x$ 出现在分母上,关键在于 $\cos x$ 在 $\displaystyle\frac{\pi}{2}$ 附近的变化:
\begin{itemize}
\item 当 $\displaystyle x \to \frac{\pi}{2}^-$(即 $x$ 从左侧逼近 $\frac{\pi}{2}$)时,$\cos x$ 逐渐趋近于 $0$ 且 $\cos x > 0$,导致 $\tan x$ 迅速增大,趋向 $+\infty$;
\item 当 $\displaystyle x \to \frac{\pi}{2}^+$(即 $x$ 从右侧逼近 $\frac{\pi}{2}$)时,$\cos x$ 依然趋向 $0$,但 $\cos x < 0$,因此 $\tan x$ 迅速变小,趋向 $-\infty$。根据诱导公式,$\tan x=\tan(x-\pi)$,因此,当$\displaystyle x \to -\frac{\pi}{2}^+$时,$\tan x$ 趋向 $-\infty$。
\end{itemize}

由此可见,$\tan x$ 在 $\displaystyle (-\frac{\pi}{2},\frac{\pi}{2})$ 内是连续变化的\footnote{也可以通过单调递增来说明连续性,但单调性的证明需要学习求导公式后才能得出。},因此它的值域为 $\mathbb{R}$。


这一点也可以由极限分析得到\footnote{注意下面的分析比较严谨,但在高中阶段不要求。},由于 $\cos x$在$\tan x={\sin x\over\cos x}$的分母上,而当 $x \to \frac{\pi}{2}$时,$\sin x $始终保持趋近$1$,为正值。因此需要研究 $\cos x$ 在 $\frac{\pi}{2}$ 附近的变化。当 $x \to \frac{\pi}{2}^-$(即 $x$ 从左侧逼近 $\frac{\pi}{2}$,或者说$x$是第一象限角)时,$\cos x$ 逐渐趋近于 $0$ 且 $\cos x > 0$。因此, $\tan x$ 迅速增大,趋于 $+\infty$。类似地,当 $x \to \frac{\pi}{2}^+$(即 $x$ 从右侧逼近 $\frac{\pi}{2}$,或者说$x$是第二象限角),$\cos x$ 依然趋向 $0$,但此时 $\cos x < 0$。所以 $\tan x$ 迅速变小,趋向 $-\infty$,根据诱导公式,$\tan x=\tan(x-\pi)$,因此,当$x \to -\frac{\pi}{2}^+$时,$\tan x$ 趋向 $-\infty$。而根据前面的分析$\tan x$ 在 $(-\frac{\pi}{2}, \frac{\pi}{2})$内是连续变化的\footnote{也可以用单调递增来说明连续性,不过单调递增需要后面学习求导公式后才能得出。},所以$\tan x$ 必然遍历整个 $\mathbb{R}$,即它的值域是 $\mathbb{R}$。




\subsection{图像}
根据前面的推导,可以得到基本三角函数的图像如下图。
总结一下:
可以看出正弦函数和余弦函数是定义域为 $R$ 值域为 $[-1,1]$ 最小正周期 $T = 2\pi$ 的周期函数。

\begin{figure}[ht]
\centering
\includegraphics[width=14.25cm]{./figures/14fd66d8d1e6e0b5.png}
\caption{$\sin x$和$\cos x$} \label{fig_HsTFFv_1}
\end{figure}

可以看到,二者的图像几乎一模一样,看上去就是$\sin x$向左平移了${\pi\over2}$个单位得到的。

\begin{figure}[ht]
\centering
\includegraphics[width=14.25cm]{./figures/6f97182187b36e36.png}
\caption{$\tan x$和$\cot x$} \label{fig_HsTFFv_3}
\end{figure}

作为扩展,下面也给出正割函数与余割函数的函数图像,他们的性质均可通过与正弦和余弦的关系分析得到,此处不予赘述。

\begin{figure}[ht]
\centering
\includegraphics[width=14.25cm]{./figures/56f93ee1a7fb0faa.png}
\caption{$\sec x$和$\csc x$} \label{fig_HsTFFv_2}
\end{figure}


\subsection{正弦型函数}

\begin{definition}{正弦型函数}
形如
\begin{equation}
f(x)=A\sin(\omega x+\varphi)~.
\end{equation}
的函数称为\textbf{正弦型函数},其中$A,\omega,\varphi$为常数,且$A\omega\neq0$。
\end{definition}
其实在前面的介绍中已经接触过这种例子了,那就是$\cos x$。根据诱导公式有:
\begin{equation}
\cos x=\sin(x+{\pi\over2})~.
\end{equation}
他就是$\displaystyle A=\omega=1,\varphi={\pi\over2}$的正弦型函数。

\addTODO{五点法作图}

\subsection{导数的规律}