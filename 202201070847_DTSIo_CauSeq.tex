% 极限存在的判据、柯西序列

\pentry{序列的极限\upref{SeqLim} 上极限与下极限\upref{SubLim}}

在之前的例子中, 我们都是在已经猜出序列极限的情况下来证明极限等式的. 但是, 对于比较复杂的序列, 又该如何判定它是否有极限?

回忆上极限与下极限\upref{SubLim}, 其中列出了几个简单的等价命题: 序列存在极限, 等价于其上下极限相等, 也等价于其所有子列极限均相等. 但一般来说, 这其实比判断序列本身是否有极限还要复杂. 我们需要一些简单的判据.

第一个判据如下:

\begin{theorem}{}
单调有界的实数序列必然有极限.
\end{theorem}
\textbf{证明.} 不妨设 $\{a_n\}$ 是单调递增的序列, 而且有上界. 按照确界原理\upref{SupInf}, 数集 $\{a_n:n\in\mathbb{N}\}$ 有唯一的上确界 $A$, 也就是说成立如下两件事: 第一, $a_n\leq A$ 对于任何 $n$ 都成立; 第二, 任给 $\varepsilon>0$, 数 $A-\varepsilon$ 都不是数集 $\{a_n:n\in\mathbb{N}\}$ 的上界. 于是, 存在一个脚码 $N_\varepsilon$ 使得 $A-\varepsilon<a_{N_\varepsilon}\leq A$. 根据单调性, 这表示对于 $n>N_\varepsilon$ 总有
\[A-\varepsilon<a_n\leq A,\]
即 $|a_n-A|<\varepsilon$. 于是 $A$ 是序列 $\{a_n\}$ 的极限. \textbf{证毕.}

\begin{exercise}{哪里用到了完备性?}
单调有界的有理数序列的极限不一定是有理数, 例如$\sqrt{2}$的不足近似值序列
$$
1.4,\,1.41,\,1.414,...
$$
这是因为有理数集不完备. 在上面的证明中, 哪里用到了实数集的完备性?
\end{exercise}

不过, 显然也有很多有极限的序列不是单调的. 