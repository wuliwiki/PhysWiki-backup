% 物理单位前缀

在物理中, 我们会在一些单位符号前面加上一个表示数量级的前缀(prefix)以方便书写. 如长度单位 $\Si{m}$ (米) 可以加不同的前缀拓展 $\Si{cm}$(厘米), $\Si{mm}$(毫米). 另外我们也会有一些前缀用来表示更大的数量级, 如频率单位 $\Si{Hz}$ (赫兹)可以拓展为 $\Si{MHz}$ (兆赫兹), $\Si{GHz}$ (千兆赫兹), 常见于无线电术语中. 每一个这样的前缀表示一个 $10^N$ 的整数, 下面我们来看常用前缀代表的数量级.

\begin{table}[ht]
\centering
\caption{请输入表格标题}\label{UniPre_tab1}
\begin{tabular}{|c|c|c|c|c|c|}
\hline
* & * & * & * & * & * \\
\hline
milli & $m$ & $10^{-3}$ & killo & $k$ & $10^3$ \\
\hline
micro & $\mu$ & $10^{-6}$ & $M$ & $10^6$ \\
\hline
nano & $n$ & $10^{-9}$ & $G$ & $10^9$ \\
\hline
pico & $p$ & $10^{-12}$ & $T$ & $10^{12}$ \\
\hline
femto & $f$ & $10^{-15}$ & $P$ & $10^{15}$ \\
\hline
atto & $a$ & $10^{-18}$ & $E$ & $10^{18}$ \\
\hline
zepto & $z$ & $10^{-21}$ & $Z$ & $10^{21}$ \\
\hline
yocto & $y$ & $10^{-24}$ & $Y$ & $10^{24}$ \\
\hline
* & * & * & * \\
\hline
\end{tabular}
\end{table}
