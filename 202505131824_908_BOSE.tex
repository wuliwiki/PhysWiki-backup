% 萨特延德拉·纳特·玻色(综述)
% license CCBYSA3
% type Sum

本文根据 CC-BY-SA 协议转载翻译自维基百科\href{https://en.wikipedia.org/wiki/Satyendra_Nath_Bose}{相关文章}。

萨特延德拉·纳特·玻色(Satyendra Nath Bose,FRS,印度国会议员)\(^\text{[1]}\)(/ˈboʊs/;\(^\text{[4]}[a]\)1894年1月1日-1974年2月4日)是印度理论物理学家和数学家。他最著名的成就是在20世纪20年代初对量子力学的研究,奠定了玻色–爱因斯坦统计的基础,并发展出玻色–爱因斯坦凝聚态理论。他是英国皇家学会院士,并于1954年获得印度政府颁发的印度第二高平民荣誉——帕德玛·毗布尚奖。\(^\text{[5][6][7]}\)

遵循玻色统计的粒子被称为玻色子,这个名称是由保罗·狄拉克以玻色的名字命名的。\(^\text{[8][9]}\)

玻色是一位百科全书式的学者,兴趣广泛,涵盖物理、数学、化学、生物、矿物学、哲学、艺术、文学与音乐等多个领域。印度独立后,他参与了许多科研和技术发展委员会的工作。\(^\text{[10]}\)
\subsection{早年生活}

**萨特延德拉·纳特·玻色出生地**
**萨特延德拉·纳特·玻色住所(加尔各答伊斯瓦尔米尔巷22号)入口及名牌**

玻色出生于加尔各答(今称加尔各答,Kolkata),是孟加拉卡雅斯特(Kayastha)家庭中七个孩子中的长子。他是家中唯一的儿子,下面还有六个妹妹。他的祖籍位于孟加拉省纳迪亚县(Nadia)的巴拉·贾古利亚(Bara Jagulia)村。

他五岁开始上学,学校就在家附近。后来家里搬到了果阿巴甘(Goabagan)地区,他便进入新印度学校(New Indian School)就读。在学业最后一年,他转入著名的**印度教学校(Hindu School)**。1909年,他通过入学考试(即中学毕业考试)并在成绩优异者中名列第五。

随后,他进入加尔各答**总督学院(Presidency College)**攻读中级理科课程。他在那里的老师包括**贾加迪什·钱德拉·玻色(Jagadish Chandra Bose)**、\*\*萨拉达·普拉萨纳·达斯(Sarada Prasanna Das)**以及**普拉富拉·钱德拉·雷(Prafulla Chandra Ray)\*\*等名师。
