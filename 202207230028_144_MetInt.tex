% 金属材料科普(草稿)

\subsection{原子与晶体}
如果你的视力足够好\footnote{事实上,光的波长一般大于此(可见光的波长约为300-700nm,因此凭光学显微镜是不可能看到如此细小的结构等.这也是为什么我们发明了电子显微镜)},可以看见纳米级别(大概$10^-9m =10^-6 mm$)的金属微观结构,那么你会发现金属是由大量原子层层叠叠有序堆积起来的.堆积的具体方式与金属的种类\footnote{有些金属有不止一种堆积方式,互称为同素异形体}有关.

\begin{figure}[ht]
\centering
\includegraphics[width=5cm]{./figures/MetInt_1.png}
\caption{晶体中原子的排列示意图} \label{MetInt_fig1}
\end{figure}

\begin{definition}{晶体}
原子(或分子)在三维空间按一定规律作周期性排列而形成的固体
\end{definition}

既然晶体中的原子排列是周期性重复的,我们自然就能找出其中一个最小的重复单元,以代表这种排列方式的特征.这种最小单元被称为晶胞.例如金属铁的晶胞是体心立方(BCC)结构,即正方体中心的一个原子与周围八个原子均相切,有点像这样:
\begin{figure}[ht]
\centering
\includegraphics[width=5cm]{./figures/MetInt_2.png}
\caption{铁的晶胞}} \label{MetInt_fig2}
\end{figure}
\begin{definition}{晶胞}
能够完全反应晶体几何特征的最小单元
\end{definition}

\subsection{缺陷}
如果金属都按这种理想的方式,整整齐齐地排列起来,那材料科学也未免太无趣了(同时我们能用材料科学能做的事也会少很多!).实际上,由于各种因素的影响,真实的金属晶体往往存在偏离理想结构的区域,称为缺陷.金属中的缺陷虽然占比较少,但对材料的性能起到了决定性影响.
\begin{definition}{缺陷}
晶体中原子偏离理想位置而出现的不完整区域
\end{definition}
根据缺陷在空间尺度上的不同,缺陷一般被分为点缺陷、线缺陷与面缺陷.
\subsubsection{点缺陷}
点缺陷指的是单独少数原子的错误排列.