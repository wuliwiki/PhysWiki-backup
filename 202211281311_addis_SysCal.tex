% Linux 的 System Call 笔记

\begin{issues}
\issueDraft
\end{issues}

\begin{itemize}
\item \href{https://en.wikipedia.org/wiki/User_space_and_kernel_space}{user-space}: 系统的虚拟内存被划分为 kernel space 和 user space. 前者跑的是系统内核和一些驱动, 剩下的所有程序和另一些驱动都在后者.
\item 参考 \href{https://en.wikipedia.org/wiki/Linux_kernel_interfaces}{Wikipedia}.
看这里的图, glibc 提供 kernel 和用户程序之间的 API 接口, 用户程序也可以直接进行 system call.
\item 内核大约一共提供 380 个 system call.
\item 参考\href{https://jameshfisher.com/2018/02/19/how-to-syscall-in-c/}{这篇教程}.
\item \verb|<unistd.h>| 头文件提供各种 system call, 例如 \verb|write(1, "hello, world!\n", 14);|
\item \verb|<sys/syscall.h>| 是更底层的工具, 例如提供函数 \verb|syscall(SYS_write, 1, "hello, world!\n", 14);|. 而 \verb|syscall()| 是用机器码定义的.
\end{itemize}
