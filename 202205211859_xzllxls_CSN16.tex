% 2016 年计算机学科专业基础综合全国联考卷
% 2016 计算机 考研 全国联考

\subsection{一、单项选择题}
1~40小题,每小题2分,共80分.下列每题给出的四个选项中.只有一个选项符合试题要求.

1.已知表头元素为c的单链表在内存中的存储状态如下表所示.
\begin{table}[ht]
\centering
\caption{第1题表}\label{CSN16_tab1}
\begin{tabular}{|c|c|c|}
\hline
地址 & 元素 & 链接地址 \\
\hline
1000H & a & 1010H \\
\hline
1004H & b & 100CH \\
\hline
1008H & c & 1000H \\
\hline
100CH & d & NULL \\
\hline
1010H & e & 1004H \\
\hline
1014H &   &  \\
\hline
\end{tabular}
\end{table}
现将f存放于1014H处并插入到单链表中,若f在逻辑上位于a和e之间,则a,e,f的“链接地址”依次是 \\
A.1010H,1014H,1004H $\qquad$ B.1010H,1004H,1014H \\
C.1014H,1010H,1004H $\qquad$ D.1014H,1004H,1010H

2.已知一个带有表头结点的双向循环链表L,结点结构为 \\
\begin{table}[ht]
\centering
\caption{第2题表}\label{CSN16_tab2}
\begin{tabular}{|c|c|c|}
\hline
prev & data & next \\
\hline
\end{tabular}
\end{table}
其中,prev和next分别是指向其直接前驱和直接后继结点的指针.现要删除指针p所指的结点,正确的语句序列是 \\
A. p->next->prev=p->prev; p->prev->next=p->prev; free (p); \\
B. p->next->prev=p->next; p->prey-> next=p->next; free (p); \\
C. p->next->prev=p->next; p->prev->next=p->prev; free (p); \\
D. p-> next-> prey=p->prey; p->prev->next=p->next; free (p);

3.设有如下图所示的火车车轨,入口到出口之间有n条轨道,列车的行进方向均为从左至右,列车可驶入任意一条轨道.现有编号为1~9的9列列车,驶入的次序依次是8,4,2,5,3,9,1,6,7.若期望驶出的次序依次为1~9,则n至少是 \\
\begin{figure}[ht]
\centering
\includegraphics[width=14.25cm]{./figures/CSN16_1.png}
\caption{第3题图} \label{CSN16_fig1}
\end{figure}
A. 2 $\qquad$ B.3 $\qquad$ C.4 $\qquad$ D.5

4.有一个$100$阶的三对角矩阵$M$,其元素$m_{i,j}$($1$≤$i$≤$100$,$1$≤$j$≤$100$)按行优先次序压缩存入下标从$0$开始的一维数组Ⅳ中.元素$m_{30}$,$30$在$N$中的下标是 \\
A.86 $\qquad$ B.87 $\qquad$ C.88 $\qquad$ D.89

5.若森林F有15条边、25个结点,则F包含树的个数是 \\
A.8 $\qquad$ B.9 $\qquad$ C.10 $\qquad$ D.11

6.下列选项中,不.是下图深度优先搜索序列的是 \\
\begin{figure}[ht]
\centering
\includegraphics[width=5cm]{./figures/CSN16_2.png}
\caption{请添加图片描述} \label{CSN16_fig2}
\end{figure}
A.$V_1$,$V_5$,$V_4$,$V_3$,$V_2$ \\
B.$V_1$,$V_3$,$V_2$,$V_5$,$V_4$ \\
C.$V_1$,$V_2$,$V_5$,$V_4$,$V_3$ \\
D.$V_1$,$V_2$,$V_3$,$V_4$,$V_5$

7.若将n个顶点e条弧的有向图采用邻接表存储,则拓扑排序算法的时间复杂度是 \\
A.$O(n)$ $\qquad$ B.$O(n+e)$ $\qquad$ C.$O(n^2)$ $\qquad$ D.$O(n\times e)$

8.使用迪杰斯特拉(Dijkstra)算法求下图中从顶点1到其他各顶点的最短路径,依次得到的各最短路径的目标顶点是 \\
A.5,2,3,4,6 $\qquad$ B.5,2,3,6,4 \\
C.5,2,4,3,6 $\qquad$ D.5,2,6,3,4

9.在有n(n>1000)个元素的升序数组A中查找关键字x.查找算法的伪代码如下所示. \\
\begin{lstlisting}[language=cpp]
k=0;
while(k<n且A[k]<x)k=k+3;
if(k<n且A[k]==x)查找成功;
else if(k-1<n且A[k-1]==x)查找成功;
  else if(k-2<n且A[k-2]==x)查找成功;
    else查找失败;
\end{lstlisting}
本算法与折半查找算法相比,有可能具有更少比较次数的情形是 \\
A.当x不在数组中 $\qquad$ B.当x接近数组开头处 \\
C.当x接近数组结尾处 $\qquad$ D.当x位于数组中间位置

10.$B^+$树\textbf{不同}于B树的特点之一是 \\
A.能支持顺序查找 \\
B.结点中含有关键字 \\
C.根结点至少有两个分支 \\
D.所有叶结点都在同一层上

11.对10TB的数据文件进行排序,应使用的方法是 \\
A.希尔排序 $\qquad$ B.堆排序 \\
C.快速排序 $\qquad$ D.归并排序

12.将高级语言源程序转换为机器级目标代码文件的程序是 \\
A.汇编程序 $\qquad$ B.链接程序 \\
C.编译程序 $\qquad$ D.解释程序

13.有如下C语言程序段: \\
\begin{lstlisting}[language=cpp]
short si=-32767;
unsigned short usi=si;
\end{lstlisting}
执行上述两条语句后,usi的值为
A.-32767 $\qquad$ B.32767 $\qquad$ C.32768 $\qquad$ D.32769

14.某计算机字长为32位,按字节编址,采用小端(Little Endian)方式存放数据.假定有一个double型变量,其机器数表示为1122 3344 5566 7788H,存放在0000 8040H开始的连续存储单元中,则存储单元0000 8046H中存放的是 \\
A.22H $\qquad$ B.33H $\qquad$ C.66H $\qquad$ D.77H

15.有如下C语言程序段: \\
\begin{lstlisting}[language=cpp]
for(k=0;k<1000;k++)
  a[k]=a[k]+32;
\end{lstlisting}
若数组a及变量k均为int型,int型数据占4 B,数据Cache采用直接映射方式、数据区大小为1 KB、块大小为16 B,该程序段执行前Cache为空,则该程序段执行过程中访问数组a的Cache缺失率约为 \\
A.1.25% $\qquad$ B.2.5% $\qquad$ C.12.5% $\qquad$ D.25%

16.某存储器容量为64 KB,按字节编址,地址4000H~5FFFH为ROM 区,其余为RAM区.若采用8 K×4位的SRAM芯片进行设计,则需要该芯片的数量是 \\
A.7 $\qquad$ B.8 $\qquad$ C.14 $\qquad$ D.16

17.某指令格式如下所示.
\begin{table}[ht]
\centering
\caption{第17题表}\label{CSN16_tab3}
\begin{tabular}{|c|c|c|c|}
\hline
OP & M & I & D \\
\hline
\end{tabular}
\end{table}
其中M为寻址方式,I为变址寄存器编号,D为形式地址.若采用先变址后间址的寻址方式,则操作数的有效地址是 \\
A.I+D $\qquad$ B.(I)+D $\qquad$ C.((I)+D) $\qquad$ D.((I))+D


