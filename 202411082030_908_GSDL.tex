% 高斯定律(综述)
% license CCBYSA3
% type Wiki

本文根据 CC-BY-SA 协议转载翻译自维基百科\href{https://en.wikipedia.org/wiki/Conservation_of_energy}{相关文章}。

本文讨论的是关于电场的高斯定律。关于其他场的类似定律,请参见磁场的高斯定律和重力的高斯定律。关于与这些定律相关的奥斯特罗格拉德斯基-高斯定理,请参见散度定理。  
请勿将其与高斯法则(Gause's law)混淆。

在物理学中(特别是电磁学),高斯定律(又称高斯通量定理,或有时称为高斯定理)是麦克斯韦方程组之一。它是散度定理的一个应用,将电荷分布与由此产生的电场联系起来。
\subsection{定义}
\begin{figure}[ht]
\centering
\includegraphics[width=8cm]{./figures/95db2b4380930eda.png}
\caption{当由于对称性原因可以找到一个电场沿其均匀的闭合曲面(GS)时,积分形式的高斯定律尤其有用。在这种情况下,电通量是表面积和电场强度的简单乘积,并且与曲面所包围的总电荷成正比。这里正在计算带电球体外部(\( r > R \))和内部(\( r < R \))的电场(见维基学院)。} \label{fig_GSDL_1}
\end{figure}
在积分形式下,高斯定律表述为:电场通过任意封闭曲面的通量与该曲面所包围的电荷成正比,而不论电荷如何分布。尽管仅凭此定律不足以确定包围任意电荷分布的表面上的电场,但在具有对称性导致电场均匀分布的情况下,这是可能的。在没有这种对称性的情况下,可以使用高斯定律的微分形式,其表述为电场的散度与电荷的局部密度成正比。

该定律最早由约瑟夫-路易·拉格朗日于1773年[1][2]提出,后由卡尔·弗里德里希·高斯于1835年在椭球引力的背景下提出。[3] 它是麦克斯韦方程组之一,构成经典电动力学的基础。[注1] 高斯定律可以用于推导库仑定律,[4] 反之亦然。
\subsection{定义}
在积分形式下,高斯定律表述为:电场通过任意封闭曲面的通量与该曲面所包围的电荷成正比,而不论电荷如何分布。尽管仅凭此定律不足以确定包围任意电荷分布的表面上的电场,但在对称性要求电场均匀的情况下,这是可能的。在不存在这种对称性的情况下,可以使用高斯定律的微分形式,其表述为电场的散度与电荷的局部密度成正比。

该定律最早由约瑟夫-路易·拉格朗日于1773年[1][2]提出,随后在1835年由卡尔·弗里德里希·高斯[3]提出,两者都是在椭球引力的背景下提出的。它是麦克斯韦方程组之一,构成经典电动力学的基础。[注1] 高斯定律可以用于推导库仑定律,[4] 反之亦然。