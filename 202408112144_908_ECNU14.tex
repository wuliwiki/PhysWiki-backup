% 华东师范大学 2014 年 考研 量子力学
% license Usr
% type Note

\textbf{声明}:“该内容来源于网络公开资料,不保证真实性,如有侵权请联系管理员”

\subsection{简答题(每小题6分,共42分)}
\begin{enumerate}
\item 什么是相速度?什么是群速度?对于有质量的微观粒子,其德布罗意波的相速度与群速度星何种关系?
\item 自由粒子的波函数是否一定是平面波?为什么?
\item 一维束缚定态是否存在简并?为什么?
\item 在什么条件下,厄密算符F(不显含时间)的平均值不随时间变化?
\item 设 $\vec{A}, \vec{B}$ 与泡利算符对应,证明:$\left( \vec{\sigma} \cdot \vec{A} \right) \left( \vec{\sigma} \cdot \vec{B} \right) = \vec{A} \cdot \vec{B} + i \vec{\sigma} \cdot \left( \vec{A} \times \vec{B} \right)$
\item 判断下面各符号中,哪些是算符?哪些是数?哪些是矢量?
$|\psi\rangle\langle\phi| : \langle\phi|\psi\rangle : \langle\phi|\hat{F}|\phi\rangle : \hat{F}|\alpha\rangle : \lambda |\alpha\rangle\langle\beta|\omega\rangle : \lambda \langle\alpha|\hat{A}|\psi\rangle\langle\mu|\gamma\rangle$
\item 斯塔克(Stark)效应是怎么回事?从量子力学的角度解释其产生的根源。
\end{enumerate}
\subsection{(木题16分)}
考虑一维体系 $\hat H = \frac{\dot p^2}{2m} + V(x),\ V(x) = V_0 x^\lambda,\ V_0 > 0,\ \lambda = 2, 4, 6, \ldots$。设其本征函数为 $\psi_n$。
\begin{enumerate}
\item 证明动量在态 $\psi_n$ 中的平均值为零;
\item 求在态 $\psi_n$ 中动能平均值与势能平均值之间的关系。
\end{enumerate}
\subsection{(木题16分)}
自旋为 $\frac{1}{2}$,质量为 $m$ 的粒子处于 $[0,a]$ 之间的无限深方势阱中,若 $t = 0$ 时粒子状态为:
$$\psi(x,s_z,0) = C \left( \varphi_1(x) \chi_{1/2}(s_z) - 2 \varphi_2(x) \chi_{-1/2}(s_z) + \sqrt{2} \varphi_2(x) \chi_{1/2}(s_z) \right)~$$
其中 $\varphi_n(x)$ 是一维无限深势阱的第 $n$ 个本征态,$\chi_{1/2}(s_z)$ 分别为自旋向上和向下的状态。
\begin{enumerate}
\item 求任意t时刻粒子的波函数以及能量的期望值。
\item 测量粒子自旋在z方向的可能取值和相应概率。
\end{enumerate}
\subsection{(木题16分)}