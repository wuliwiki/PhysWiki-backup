% 转动与平动的类比

以下的表格类比了刚体的平动与转动。形式上两者高度相似。
\begin{figure}[ht]
\centering
\includegraphics[width=8cm]{./figures/RATC_1.pdf}
\caption{刚体的平动} \label{RATC_fig1}
\end{figure}

\begin{figure}[ht]
\centering
\includegraphics[width=6cm]{./figures/RATC_2.pdf}
\caption{刚体的定轴转动} \label{RATC_fig2}
\end{figure}

\begin{table}[ht]
\centering
\caption{运动学量}\label{RATC_tab1}
\begin{tabular}{|c|c|}
\hline
平动&转动\\
\hline
位置 $r$ & 角度 $\theta$ \\
\hline
速度 $v=\dv{r}{t}$ & 角速度 $\omega = \dv{\theta}{t}$ \\
\hline
加速度 $a = \dv{r}{t}$ & 角加速度 $\alpha = \dv{\omega}{t}$ \\
\hline
\end{tabular}
\end{table}

\begin{table}[ht]
\centering
\caption{动力学量}\label{RATC_tab2}
\begin{tabular}{|c|c|}
\hline
平动&转动\\
\hline
力 $F$ & 力矩 $\tau=\abs{r}\abs{F} \sin \theta$\\
\hline 
功 $W = F \dd r$ & (力矩的)功 $W=\tau \dd \theta$\\
\hline
质量 $m$ & 转动惯量 $I = \int r_\perp^2 \dd m$ \\
\hline
动量 $p=mv$ & 角动量 $L=I\omega$ \\
\hline
动能 $E_k = \frac{1}{2}mv^2$ & (转动)动能 $E_k = \frac{1}{2} I\omega^2$ \\
\hline
\end{tabular}
\end{table}

\begin{table}[ht]
\centering
\caption{动力学定理}\label{RATC_tab3}
\begin{tabular}{|c|c|}
\hline
平动&转动\\
\hline 
牛二 $F=ma$& $\tau = I \alpha$\\
\hline
动量定理 $F=\dv{p}{t}$ & 角动量定理 $\tau=\dv{L}{t}$ \\
\hline
动能定理 $W = \Delta E_k$ & 动能定理 $W = \Delta E_k$ \\
\hline
\end{tabular}
\end{table}