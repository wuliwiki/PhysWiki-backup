% 等离子体
% license Usr
% type Note

\subsection{等离子体的基本概念:粒子的集体行为}

等离子体作为区别于固体、液体、气体而独立的物质第四态,其最大的特性就是内部相互作用力的改变——从固液气中的中性成分作用力占主导变为了电磁相互作用力占主导。等离子体由电子和离子组成,这些带电的粒子所携带的总电荷相等,也就是说体系整体上呈现电中性。等离子体中可以有中性成分吗?当然是可以的,但是考虑到“库仑力主导”的原则,电的作用要占到主导地位才能称得上等离子体,也就是说带电粒子占比要足够高或者说\textbf{电离度}足够高。

为了进一步地说明,我们不妨考察一个基本的物态相变的例子:对于某一固体,随着温度的升高,其内部分子的振动能量增加,原本的小尺度振荡被破坏,固体变成了可以自由流动变形的液体。液体在分子作用力作用下,其内部分子能被限制在一定体积中而不能随意逃离;但进一步加热液体,分子做热运动的动能进一步增大,突破分子作用力的限制而可以自由在空间中运动,这就成为了气体。不过这时,气体中的主导相互作用仍为分子或原子间的相互碰撞这样的中性作用。我们再对气体进一步加热,当其热运动的动能增加到其组分的第一电离能时,碰撞就能导致原子或分子的电离,产生正离子和电子。这一过程也就是\textbf{电离过程},而电离成分数密度与发生电离前中性成分数密度之比就是电离度。这是的气体可以称为部分电离的气体,如果温度进一步升高,电离度会随之增加,当库伦相互作用占绝对主导地位时,等离子态就出现了。

等离子体看上去只是强烈电离的气体,但是其物理性质却和气体截然不同,其最大的特点就是“牵一发而动全身”,库伦相互作用使原本过于松散随意的气体粒子变得更加循规蹈矩,也在它们之间建立了更加密切的联系,从而表现出一种\textbf{集体行为}。

综合以上,我们可以给出等离子体的定义如下:

\textbf{等离子体是带电粒子和中性粒子组成的表现出集体行为的一种准中性气体。}


接下来我们来展现这种集体行为的几个例子,并对准中性加以进一步的说明。

\subsubsection{等离子体振荡}

等离子体集体行为的第一个例子就是等离子体振荡。考虑振荡和波动,通常要明确两点——初始扰动和回复力。对等离子体来说,回复力通常是各种电磁力;对于这里讨论的基本的等离子振荡,回复力就是库仑力或者说静电力。我们要施加的初始扰动则是把等离子体中的一部分负电荷(一般为电子)挪离静电平衡位置,正电荷不动,然后我们来考虑系统在静电回复力下的振荡。

\begin{figure}[ht]
\centering
\includegraphics[width=6cm]{./figures/b8f74f69cab46d4c.png}
\caption{} \label{fig_Plasma_1}
\end{figure}

为简单起见,我们把问题抽象成这样的模型。初始时等离子体位于一个“等离子体平板”(即图中实线区域内)中保持平衡状态,由数密度相等的电子和质子构成,即有
			\begin{align}
				n_e=n_p=n_0~.
			\end{align}
			将其中的电子均匀向右移动一个小距离$\xi$到图\ref{fig_Plasma_1}中虚线位置,两端就会形成正负电荷区,从而产生电场。由于$\xi$很小,两端的正负电荷区可以看成带点平面,整个结构类似于一个平行板电容器,板上的面电荷密度为
			\begin{align}
			\sigma=en_0V/S=en_0\xi~,
			\end{align}
			从而中间部分的电场为
			\begin{align}
			E=\frac{\sigma}{\varepsilon_0}=\frac{en_0\xi}{\varepsilon_0}~.
			\end{align}
			电场力提供了“拉回”电子的回复力,则有
			$m_e\ddot{\xi}=-eE=\frac{n_0e^2}{\varepsilon_0}\xi~,$
			即
			\begin{align}
			\ddot{\xi}+\frac{n_0e^2}{m_e\varepsilon_0}\xi=0~.
			\end{align}
			从而得到电子的振荡频率
			\begin{equation}
			\boxed{\omega_p^2=\frac{n_0e^2}{m_e\varepsilon_0}}~.
			\end{equation}
			
			这一振荡最先由Langmiur发现,因此称为\textbf{Langmiur振荡},这一频率也称为\textbf{Langmiur频率}。我们可以再从物理上理解一下电子的振荡过程,电子先在向右的电场的作用下向左运动,回到原位之后继续向左,电子和离子交换相对位置,电场方向就会反向,电子开始受向右的力。这样循环往复的过程就是等离子的Langmiur振荡。
			
			将此结果稍作推广,对于密度、质量、电荷数为$(n_i,m_i,Z)~$的离子,其相应振荡频率为
			\begin{equation}
			\omega_{pi}^2=\frac{n_i Z^2e^2}{m_i\varepsilon_0}~.
			\end{equation}
			由于离子的质量远远大于电子质量,离子的振荡频率远远小于电子振荡频率。






