% S-L 边值问题本征函数零点数量的证明
% license Usr
% type Tutor


\begin{issues}
\issueDraft
\issueTODO
\end{issues}

\pentry{施图姆—刘维尔理论\nref{nod_SLthrm}}{nod_2f44}

首先把 S-L (边值)本征问题转化为另一常见形式:
\begin{definition}{三角函数形式边界条件的正则 S-L 边值问题}
对于微分算子
\begin{equation}
\hat L = \frac{1}{r(x)} \left[-\dv{}{x}\left(p(x)\dv{}{x}\right) + q(x)\right] ~,
\end{equation}
S-L 微分方程可化为 $\hat L (y) = \lambda y$。

边界条件为:
\begin{equation}
\left\{\begin{aligned}
y(a) \cos \alpha &= y'(a) p(a) \sin \alpha ~, \\
y(b) \cos \beta &= y'(b) p(b) \sin \beta ~.
\end{aligned}\right.
\end{equation}
其中由于 $\tan \theta$ 可以取遍 $\mathbb R$,故等价于原 S-L 边值问题的定义。
\end{definition}
