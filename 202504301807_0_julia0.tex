% Julia 开发笔记
% license Xiao
% type Note

本文是关于 Julia 解释器的原理: 它使用了哪些技术, 可以使得它作为一门动态语言能达到编译语言的性能。

\begin{itemize}
\item \href{https://docs.julialang.org/en/v1/devdocs/init/}{Julia 开发者文档}
\item C 语言这样的叫做 \textbf{Ahead of Time(AOT)} 编译。
\item Julia 完全使用 \textbf{Just-In-Time(JIT)}。 并不是把部分代码使用该优化,而是全部。Julia 默认不直接理解并执行高级语言(一个解释器见\href{https://juliadebug.github.io/JuliaInterpreter.jl/stable/}{这里})。 每个重载的函数第一次被调用的时候, julia 都会先给它生成专门的 llvm 机器码。 这个过程还有一个不那么正规的名字叫做 \textbf{Just Ahead-of-Time(JAOT)}。
\item \textbf{Multiple Dispatch (MD)}: 可以基于参数的类型生成不同的专门代码。
\item \textbf{Type Inference}: 自动推导变量类型。
\item 内建并行, 包括分布计算
\item 高效内存管理: 减少内存分配, 增加内存重复利用。
\item 基于 \textbf{LLVM}: Julia 编译器先把 Julia 语言变为 \textbf{Julia IR} (和 LLVM IR 相似,而且有若干层), 再变为 \enref{LLVM IR}{llvmIR}, 最后交给 LLVM 进行优化。
\item Julia IR 后期使用 \textbf{Single Static Assignment (SSA)} 形式, 把 julia 代码的控制流表示为 \textbf{directed acyclic graph (DAG)}。 Julia IR 中的变量类型都是明确的。
\item \verb`LLVM.jl` 包可以把 julia 代码直接生成 LLVM IR, 或者把 Julia IR 转为 LLVM IR。
\item 一些预备知识: 编译原理, 编译器设计, LLVM, Julia 背后原理。
\end{itemize}

\begin{figure}[ht]
\centering
\includegraphics[width=14.2cm]{./figures/7293eeb0d6cd9c43.png}
\caption{Julia JIT 流程} \label{fig_julia0_1}
\end{figure}

\subsection{把 Julia 代码编译成 LLVM IR 代码}
首先安装 \verb`using Pkg; Pkg.add("LLVM")` (其实没必要)

在 REPL 中输入如下代码
\begin{lstlisting}[language=julia,caption=julia]
function add(x::Int, y::Int)::Int
    return x + y
end
lowered_ir = @code_lowered add(1, 2) # 生成 Lowered Julia IR
typed_ir = @code_typed add(1, 2) # 生成静态类型的 Julia IR
llvm_ir = @code_llvm add(1, 2) # 生成 LLVM IR
\end{lstlisting}

也可以把函数定义写到 jl 脚本中,然后在 REPL 中 \verb`include("脚本名.jl")`, 再 \verb`@code_xxx add(1, 2)`。 \verb`include()` 相当于把脚本代码直接插入到该位置。

注意只会生成指定函数的 IR,不会递归生成它调用的其他函数的 IR。

\begin{lstlisting}[language=none,caption=julia ir]
julia> lowered_ir = @code_lowered add(1, 2)
CodeInfo(
1 ─ %1  = Main.Int
│   %2  = Main.:+
│   %3  = (%2)(x, y)   
│         @_4 = %3
│   %5  = @_4
│   %6  = %5 isa %1
└──       goto #3 if not %6
2 ─       goto #4
3 ─ %9  = @_4
│   %10 = Base.convert(%1, %9)
└──       @_4 = Core.typeassert(%10, %1)
4 ┄ %12 = @_4
└──       return %12
)
\end{lstlisting}

生成的 LLVM IR 代码如下
\begin{figure}[ht]
\centering
\includegraphics[width=14.25cm]{./figures/f34505db9ea28f8f.png}
\caption{LLVM IR} \label{fig_julia0_2}
\end{figure}

注意返回的 \verb`lowered_ir` 变量并不是一个字符串,而是结构化的信息。

\subsubsection{如何读懂 Julia IR}
如果 Julia IR 中的东西不懂,文档也不完善,那就直接执行试试即可, Julia IR 中的函数都是可以在 REPL 中执行的。 例如上面的 \verb`Main.Int` 执行出来就是 \verb`Int64::DataType`。 找许多 Julia 源码和 IR 对照下来基本就能明白。

\subsection{循环的 IR}
代码
\begin{lstlisting}[language=julia]
for n in 1:5
    c += n
end
\end{lstlisting}
对应的 IR 是
\begin{lstlisting}[language=none]
14 ┄ %77  = Main.:(:)                         ; for n in 1:5
                                              ;     % ==循环准备阶段==
│    %78  = (%77)(1, 5)                       ;     % 返回 ::UnitRange{Int64}
│           @_4 = Base.iterate(%78)           ;     % 初始化迭代器: (值, 循环指数)
                                              ;     % 类型 Tuple{Int64, Int64}
                                              ;     % , 当前为 (1, 1)
│    %80  = @_4
│    %81  = %80 === nothing
│    %82  = Base.not_int(%81)                 ;     % 非运算
└───        goto #17 if not %82
                                              ;     % ==循环体==
15 ┄ %84  = @_4
│           n = Core.getfield(%84, 1)
│    %86  = Core.getfield(%84, 2)             ;     % 循环指数(对 UnitRange{Int64}
                                              ;     %    未必从 1 开始也未必步长为 1)
│    %87  = Main.:+                           ;     c += n
│    %88  = c
│    %89  = n
│           c = (%87)(%88, %89)
│           @_4 = Base.iterate(%78, %86)      ;     % 迭代(nothing 表示结束)
│    %92  = @_4
│    %93  = %92 === nothing
│    %94  = Base.not_int(%93)
└───        goto #17 if not %94
16 ─        goto #15
                                              ; end
\end{lstlisting}

代码
\begin{lstlisting}[language=julia]
function foo()
a = 0
for i in [1, "123", 1+2im]
    a += 1
    println(i, a);
end
end
\end{lstlisting}
对应的 IR 是
\begin{lstlisting}[language=none]
CodeInfo(
1 ─       a = 0
                                         ; % ==循环准备==
│   %2  = Main.:+                        ; % 构造 [1, "123", 1+2im]
│   %3  = Main.:*
│   %4  = Main.im
│   %5  = (%3)(2, %4)
│   %6  = (%2)(1, %5)
│   %7  = Base.vect(1, "123", %6)
│         @_2 = Base.iterate(%7)         ; % 迭代器
│   %9  = @_2
│   %10 = %9 === nothing
│   %11 = Base.not_int(%10)
└──       goto #4 if not %11
2 ┄ %13 = @_2                            ; % ==循环体==
│         i = Core.getfield(%13, 1)      ; % 循环变量
│   %15 = Core.getfield(%13, 2)          ; % 循环指数
│   %16 = Main.:+
│   %17 = a
│         a = (%16)(%17, 1)
│   %19 = i
│   %20 = a
│         Main.println(%19, %20)
│         @_2 = Base.iterate(%7, %15)
│   %23 = @_2
│   %24 = %23 === nothing
│   %25 = Base.not_int(%24)
└──       goto #4 if not %25
3 ─       goto #2
4 ┄       return nothing
)
\end{lstlisting}


\subsection{Base 函数说明}
\begin{itemize}
\item \verb`Base.Pairs(("a","b"), (1,2))` 返回的类型是 \verb`Base.Pairs{Int64, String, Tuple{Int64, Int64}, Tuple{String, String}}`,但类似于 \verb`Dict([1=>"a", 2=>"b"])`。 可以使用 \verb`var[key]` 获取对应的值。
\end{itemize}
