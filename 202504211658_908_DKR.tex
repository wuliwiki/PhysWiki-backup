% 笛卡儿积(综述)
% license CCBY4
% type Wiki

本文根据 CC-BY-SA 协议转载翻译自维基百科\href{https://en.wikipedia.org/wiki/Cartesian_product}{相关文章}。

在数学中,特别是集合论中,两个集合\(A\)和\(B\)的笛卡尔积,记作\( A \times B \),是所有有序对\( (a, b) \)的集合,其中\( a \in A \),且\( b \in B \)\(^\text{[1]}\)。用集合构造符号表示为:  
\[
A \times B = \{ (a, b) \mid a \in A \text{ 且 } b \in B \}.^\text{[2][3]}~
\]
可以通过对“行的集合”与“列的集合”取笛卡尔积来创建一个表格。若取笛卡尔积 rows × columns,那么表格的每个单元格就包含一个形如(行值,列值)的有序对。

同样地,也可以定义 n 个集合的笛卡尔积,称为 n 重笛卡尔积,它可以表示为一个 n 维数组,其中每个元素是一个 n 元组(n-tuple)。有序对是 2 元组(2-tuple)或称为“偶对”。更一般地,还可以定义一个按索引排列的集合族的笛卡尔积。

笛卡尔积的名称来自于勒内·笛卡尔,他对解析几何的建立促成了这一概念的产生,该概念进一步推广后形成了“直积”的形式。