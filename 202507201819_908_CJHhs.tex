% 超几何函数(综述)
% license CCBYSA3
% type Wiki

本文根据 CC-BY-SA 协议转载翻译自维基百科\href{https://en.wikipedia.org/wiki/Hypergeometric_function}{相关文章}。

\begin{figure}[ht]
\centering
\includegraphics[width=6cm]{./figures/d7c2f81e451fe639.png}
\caption{超几何函数 ${}_2F_1(a, b; c; z)$ 在复平面上从 $-2 - 2i$ 到 $2 + 2i$ 的图像,其中参数取值为 $a = 2$、$b = 3$、$c = 4$,图像的颜色由 Mathematica 13.1 的函数 ComplexPlot3D 生成。} \label{fig_CJHhs_1}
\end{figure}
在数学中,高斯或普通超几何函数 ${}_2F_1(a,b;c;z)$ 是由超几何级数表示的特殊函数,它包含了许多其他特殊函数作为特例或极限情形。它是一个二阶线性常微分方程(ODE)的解。任何具有三个正规奇点的二阶线性常微分方程都可以转化为该方程。

关于超几何函数所涉及的成千上万条恒等式的系统整理,可以参见 Erdélyi 等人(1953年)和 Olde Daalhuis(2010年)的参考著作。至今尚无已知的体系可以组织所有这些恒等式;事实上,也没有已知的算法可以生成所有恒等式。目前已知的算法各自能生成不同系列的恒等式。发现恒等式的算法理论仍是一个活跃的研究课题。
\subsection{历史}
“超几何级数”这一术语最早由约翰·沃利斯在其1655年的著作《无穷算术》中提出。

超几何级数曾被莱昂哈德·欧拉研究过,但最早对其进行系统全面研究的是卡尔·弗里德里希·高斯,时间是在1813年。

19世纪的研究包括恩斯特·库默尔(Ernst Kummer,1836年)的工作,以及伯恩哈德·黎曼(Bernhard Riemann,1857年)通过超几何函数所满足的微分方程对其进行的基本刻画。

黎曼证明,对于 ${}_2F_1(z)$ 的二阶微分方程,在复平面上可以通过其在黎曼球面上的三个正规奇点来进行刻画。

而当超几何方程的解为代数函数的情形,则由赫尔曼·施瓦茨确定下来,这就是著名的“施瓦茨表”。
\subsection{超几何级数}
在 $|z| < 1$ 的范围内,超几何函数定义为幂级数:
$$
{}_2F_1(a, b; c; z) = \sum_{n=0}^{\infty} \frac{(a)_n (b)_n}{(c)_n} \frac{z^n}{n!} = 1 + \frac{ab}{c} \frac{z}{1!} + \frac{a(a+1)b(b+1)}{c(c+1)} \frac{z^2}{2!} + \cdots~
$$
当 $c$ 是非正整数时,函数未定义(或发散)。其中 $(q)_n$ 表示递增阶乘(Pochhammer 符号),定义如下:
$$
(q)_n =
\begin{cases}
1 & \text{若 } n = 0 \\
q(q+1)\cdots(q+n-1) & \text{若 } n > 0
\end{cases}~
$$
当 $a$ 或 $b$ 是非正整数时,级数终止,此时超几何函数退化为多项式,例如:
$$
{}_2F_1(-m, b; c; z) = \sum_{n=0}^{m} (-1)^n \binom{m}{n} \frac{(b)_n}{(c)_n} z^n~
$$
对于模长 $|z| \geq 1$ 的复数参数 $z$,该函数可以通过解析延拓定义于整个复平面上(避开分支点 1 和无穷大)。实际应用中,大多数计算机实现会在实轴上 $z \geq 1$ 的区间上引入一条分支切割线。

当 $c \to -m$($m$ 为非负整数)时,${}_2F_1(z)$ 会发散。但如果除以伽玛函数 $\Gamma(c)$,可以得到如下极限:
$$
\lim_{c \to -m} \frac{{}_2F_1(a, b; c; z)}{\Gamma(c)} = \frac{(a)_{m+1} (b)_{m+1}}{(m+1)!} z^{m+1} {}_2F_1(a + m + 1, b + m + 1; m + 2; z)~
$$
${}_2F_1(z)$ 是广义超几何级数 ${}_pF_q$ 中最常见的一类,通常简写为 $F(z)$。
