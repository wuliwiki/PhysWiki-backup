% 阿瑟·康普顿(综述)
% license CCBYSA3
% type Wiki

本文根据 CC-BY-SA 协议转载翻译自维基百科\href{https://en.wikipedia.org/wiki/Arthur_Compton}{相关文章}。

\begin{figure}[ht]
\centering
\includegraphics[width=6cm]{./figures/2b1294fc8f46ffcf.png}
\caption{1927年的康普顿} \label{fig_KPD_1}
\end{figure}
阿瑟·霍利·康普顿(Arthur Holly Compton,1892年9月10日—1962年3月15日)是美国物理学家,他与C.T.R. 威尔逊共同获得了1927年诺贝尔物理学奖,表彰他发现了康普顿效应,该效应证明了电磁辐射的粒子性质。这一发现当时轰动一时:虽然光的波动性已经得到了充分证明,但光既具有波动性又具有粒子性质的观点并不容易被接受。他还因在曼哈顿计划中领导芝加哥大学冶金实验室而闻名,并在1945年至1953年期间担任圣路易斯华盛顿大学的校长。

1919年,康普顿获得了最早的两项国家研究委员会奖学金之一,允许学生到国外学习。他选择前往英国剑桥大学的卡文迪许实验室,在那里他研究了伽玛射线的散射和吸收。进一步的研究导致了康普顿效应的发现。他利用X射线研究铁磁性,并得出结论,铁磁性是电子自旋排列的结果;他还研究了宇宙射线,发现它们主要由带正电的粒子组成。

在第二次世界大战期间,康普顿是曼哈顿计划中的关键人物,该计划开发了第一批核武器。他的报告对启动该项目起到了重要作用。1942年,他成为执行委员会成员,随后成为“X”项目的负责人,监督冶金实验室,负责生产核反应堆以将铀转化为钚,寻找将钚从铀中分离出来的方法,并设计原子弹。康普顿监督了恩里科·费米(Enrico Fermi)创建的芝加哥堆-1(Chicago Pile-1),这是世界上第一个核反应堆,于1942年12月2日实现临界。冶金实验室还负责设计和操作位于田纳西州橡树岭的X-10石墨反应堆。1945年,汉福德现场的反应堆开始生产钚。

战后,康普顿成为圣路易斯华盛顿大学的校长。在他的领导下,大学正式废除了本科部的种族隔离政策,任命了第一位女性终身教授,并在战后退伍军人返回美国后迎来了创纪录的学生入学人数。
\subsection{早年生活}
\begin{figure}[ht]
\centering
\includegraphics[width=6cm]{./figures/669c9eaa5f8f627c.png}
\caption{1929年康普顿与维尔纳·海森堡在芝加哥} \label{fig_KPD_2}
\end{figure}
阿瑟·霍利·康普顿(Arthur Compton)于1892年9月10日出生在俄亥俄州的伍斯特(Wooster),父母为伊莱亚斯(Elias)和奥特莉亚·凯瑟琳(Otelia Catherine, née Augspurger)·康普顿。奥特莉亚在1939年被评为“美国年度母亲”,她来自德国门诺派(Mennonite)家庭。康普顿家族是一个学术家庭,伊莱亚斯是伍斯特大学(后来为伍斯特学院)的院长,阿瑟也在该校就读。阿瑟的长兄卡尔(Karl)也在伍斯特大学就读,并于1912年从普林斯顿大学获得物理学博士学位,后成为麻省理工学院(MIT)的校长,任职时间为1930至1948年。阿瑟的次兄威尔逊(Wilson)同样在伍斯特大学就读,1916年在普林斯顿大学获得经济学博士学位,后来成为华盛顿州立大学(前身为华盛顿州立学院)的校长,任职时间为1944至1951年。三兄弟都是阿尔法·陶·欧米茄(Alpha Tau Omega)兄弟会的成员。

康普顿最初对天文学感兴趣,并在1910年拍摄了哈雷彗星的照片。大约在1913年,他描述了一个实验,通过观察水在圆管中的运动,证明了地球的自转,这个装置现在被称为康普顿发生器。同年,他从伍斯特大学获得理学学士学位,并进入普林斯顿大学,在那里他于1914年获得硕士学位。康普顿随后在赫尔沃德·L·库克(Hereward L. Cooke)的指导下攻读物理学博士学位,撰写了关于“X射线反射强度与原子内电子分布”的论文。

1916年,康普顿获得物理学博士学位,他、卡尔和威尔逊成为普林斯顿大学首批获得博士学位的三兄弟。后来,他们成为首个同时领导美国大学的三兄弟。康普顿的妹妹玛丽嫁给了传教士C·赫伯特·赖斯(C. Herbert Rice),赖斯后来成为拉合尔福门基督教学院的院长。1916年6月,康普顿与伍斯特大学的同班同学和毕业生贝蒂·查尔蒂·麦克洛斯基(Betty Charity McCloskey)结婚。他们育有两个儿子,阿瑟·艾伦·康普顿(Arthur Alan Compton)和约翰·约瑟夫·康普顿(John Joseph Compton)。

康普顿于1916-1917年间在明尼苏达大学担任物理学讲师,随后在匹兹堡的西屋灯泡公司(Westinghouse Lamp Company)担任研究工程师,工作两年,致力于钠蒸气灯的开发。在第一次世界大战期间,他为信号军团开发了飞机仪器。

1919年,康普顿获得了国家研究委员会(National Research Council)提供的首批两项奖学金之一,允许学生出国留学。他选择前往英国剑桥大学的卡文迪许实验室(Cavendish Laboratory)。在与乔治·帕吉特·汤姆森(George Paget Thomson,J·J·汤姆森的儿子)合作时,康普顿研究了伽马射线的散射和吸收现象。他观察到散射后的射线比原始射线更容易被吸收。康普顿深受卡文迪许实验室科学家的影响,特别是恩斯特·卢瑟福(Ernest Rutherford)、查尔斯·高尔顿·达尔文(Charles Galton Darwin)和阿瑟·爱丁顿(Arthur Eddington),他最终以J·J·汤姆森的名字为自己的第二个儿子命名。

1926至1927年间,康普顿曾在旁遮普大学的化学系教授,并获得了古根海姆奖学金。

康普顿曾一度是浸信会教堂的执事。他曾说:“科学与宗教之间没有冲突,”他认为,“宗教假定有一位上帝,人类如同祂的儿女。”
\subsection{职业生涯}
\subsubsection{康普顿效应}
\begin{figure}[ht]
\centering
\includegraphics[width=6cm]{./figures/e13502f45d9be8a4.png}
\caption{} \label{fig_KPD_3}
\end{figure}
返回美国后,康普顿于1920年被任命为圣路易斯华盛顿大学的威曼·克劳物理学教授,并成为物理系主任。1922年,他发现X射线量子与自由电子发生散射后,波长变长,并且根据普朗克关系,能量比入射X射线要低,能量的剩余部分转移给了电子。这一发现被称为“康普顿效应”或“康普顿散射”,证明了电磁辐射的粒子概念。

1923年,康普顿在《物理评论》上发表了一篇论文,解释了X射线波长变化,认为光子具有类似粒子的动量,这一概念是爱因斯坦在1905年诺贝尔奖获奖的光电效应解释中提出的。最早由马克斯·普朗克在1900年假设,这些光量子被概念化为具有特定能量的光“量子”,该能量仅与光的频率相关。在他的论文中,康普顿推导了波长变化与X射线散射角度之间的数学关系,假设每个散射的X射线光子只与一个电子相互作用。他的论文最后通过实验验证了他推导出的关系:
\[
\lambda' - \lambda = \frac{h}{m_e c}(1 - \cos \theta)~
\]
其中:
\(\lambda\) 是初始波长,\\
\(\lambda'\) 是散射后的波长,\\
\(h\) 是普朗克常数,\\
\(m_e\) 是电子的静止质量,\\
\(c\) 是光速,\\
\(\theta\) 是散射角度。

其中的 \(\frac{h}{m_e c}\) 被称为电子的康普顿波长,值为 \(2.43 \times 10^{-12} \, \text{m}\)。波长变化 \(\lambda' - \lambda\) 介于零(对于 \(\theta = 0^\circ\))和电子的康普顿波长的两倍(对于 \(\theta = 180^\circ\))之间。康普顿发现,尽管某些X射线经历了大角度散射,但它们并没有波长变化;在这些情况下,光子未能击出电子。因此,波长变化的大小不仅与电子的康普顿波长相关,而是与整个原子的康普顿波长相关,后者可能比前者小多达10,000倍。

“当我在1923年在美国物理学会的会议上展示我的结果时,”康普顿后来回忆道,“这引发了我所知的最激烈的科学争论。”光的波动性已被充分证明,认为它具有双重性质的想法并不容易被接受。尤其值得注意的是,晶格中的衍射现象只能通过光的波动性质来解释。这一发现使康普顿在1927年获得了诺贝尔物理学奖。康普顿与阿尔弗雷德·W·西蒙一起开发了观察单个散射X射线光子与反冲电子的同时方法。在德国,瓦尔特·博特(Walther Bothe)和汉斯·盖格(Hans Geiger)独立地开发了类似的方法。