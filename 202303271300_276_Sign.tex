% 记号方法
% keys 常系数线性方程|记号方法
\pentry{方程 $y^{(N)}=f(x)$\upref{ynfx}}

记号方法用于求解常系数线性方程(组),适当推广该方法,也可以用于比较复杂的问题。该方法要点在于把对自变量 $x$ 求微商的运算记号记作因子 $D$,写在需要求微商的函数的左边,于是若 $y$ 是 $x$ 的某一个函数,则
\begin{equation}\label{Sign_eq10}
Dy=\dv{y}{x}
\end{equation}
\subsection{运算法则}
\begin{definition}{记号因子的方幂}\label{Sign_def1}
$D^ny=D(D^{n-1}y)\quad(n\in\mathbb{Z^{+}})$
\end{definition}
\begin{definition}{记号因子的负幂}\label{Sign_def2}
若 $D^sy=f(x)$ 且方程满足零初条件
\begin{equation}\label{Sign_eq5}
y|_{x=x_0}=y'|_{x=x_0}=\cdots=y^{s-1}|_{x=x_0}=0\quad{s\in\mathbb{Z^{+}}}
\end{equation}
则记
\begin{equation}
D^{-s}f(x)=y
\end{equation}
\end{definition}
记号因子的负幂之所以这样定义,是为了使得记号 $D^{-s}f(x)$ 有确定的意义。

根据\autoref{ynfx_eq5}~\upref{ynfx}
\begin{equation}\label{Sign_eq9}
D^{-s}f(x)=\frac{1}{(s-1)!}\int_{x_0}^x(x-t)^{s-1}f(t)\dd t
\end{equation}
则方程 $D^sy=f(x)$ 一般解为\autoref{ynfx_eq7}~\upref{ynfx}
\begin{equation}
y=D^{-s}f(x)+P_{s-1}(x)=\frac{1}{(s-1)!}\int_{x_0}^x(x-t)^{s-1}f(t)\dd t+P_{s-1}(x)
\end{equation}
其中 $P_{s-1}(x)$ 是 $x$ 的具有任意系数的 $s-1$ 次多项式。
\begin{definition}{记号因子的加减法}
\begin{equation}
(D^{n_1}+D^{n_2})y=D^{n_1}y+D^{n_2}y\quad(n_1,n_2\in \mathbb{Z^{+
}})
\end{equation}
\end{definition}
\begin{definition}{记号因子的数乘}
\begin{equation}
(aD)y=a\cdot(Dy)\quad(a\in\mathbb{C})
\end{equation}
\end{definition}
显然,形如 $aD^n(n\in \mathbb{Z},a\in\mathbb{C})$ 的所有记号因子构成的集合 $\{aD^n|n\in \mathbb{Z},a\in\mathbb{C}\}$ 为域\upref{field} $\mathbb C$ 上的线性空间\upref{LSpace},而 集合中的元素 $aD^n$ 自然叫作该空间中的向量。

利用上面的定义,容易证明以下几条性质
\begin{enumerate}
\item 
\begin{equation}
D^sy=\dv[s]{y}{x}
\end{equation}
\item 
\begin{equation}
D^{n_1}(D^{n_2}y)=D^{n_1+n_2}y
\end{equation}
\item 记号因子与任意常数因子可交换,即若 $a$ 为常数,则
\begin{equation}
aD^sy=D^s(ay)
\end{equation}
\item 若 $F(D)$ 是 $D$ 的具有常系数的多项式
\begin{equation}
F(D)=\sum_{i=0}^{n}a_iD^{n-i}
\end{equation}
则
\begin{equation}
F(D)y=\sum_{i=0}^{n}a_iD^{n-i}y,\quad F(D)ay=aF(D)y
\end{equation}
\item 若 $\varphi_1(D),\varphi_2(D)$ 是两个多项式,$\varphi(D)$ 是它们乘积,则
\begin{equation}\label{Sign_eq11}
\varphi_1(D)\qty[\varphi_2(D)y]=\varphi(D)y,\quad \qty[\varphi_1(D)+\varphi_2(D)]y=\varphi_1(D)y+\varphi_2(D)y
\end{equation}
并且因子 $\varphi_1(D),\varphi_2(D)$ 可交换。
\item \begin{equation}\label{Sign_eq1}
F(D)(e^{mx}y)=e^{mx}F(D+m)y
\end{equation}
\end{enumerate}

前5条性质是显然的,我们仅来证明最后一条性质。

\textbf{证明:}表达式 $F(D)(e^{mx}y)$ 由形为 $a_{i}D^{n-i}(e^{mx}y)$ 的项组成,于是只需证明对于每一个这样的项\autoref{Sign_eq1} 成立,即只需证明
\begin{equation}\label{Sign_eq3}
D^{n-i}(e^{mx}y)=e^{mx}(D+m)^{n-i}y
\end{equation}
应用求乘积微商的莱布尼茨公式\autoref{LeiEqu_eq1}~\upref{LeiEqu}
\begin{equation}\label{Sign_eq2}
\begin{aligned}
D^{(n-i)}(e^{mx}y)&=\sum_{s=0}^{n-i}C_{n-i}^s\qty(e^{mx})^{(s)}y^{(n-i-s)}=e^{mx}\sum_{s=0}^{n-i}C_{n-i}^sm^sy^{(n-i-s)}\\
&=e^{mx}\sum_{s=0}^{n-i}C_{n-i}^sm^sD^{n-i-s}y
\end{aligned}
\end{equation}
应用二项式定理\autoref{BiNor_eq1}~\upref{BiNor},有
\begin{equation}\label{Sign_eq4}
\sum_{s=0}^{n-i}C_{n-i}^sm^sD^{n-i-s}=(D+m)^{n-i}
\end{equation}
\autoref{Sign_eq4}  代入\autoref{Sign_eq2} 即得\autoref{Sign_eq3} . 于是便证明了\autoref{Sign_eq1} 

\begin{example}{}
求解方程
\begin{equation}\label{Sign_eq7}
(D-\alpha)^sy=f(x) \quad (s\in \mathbb{Z}^{+})
\end{equation}
的一般解。
为求这个解,引用新的未知数 $z$ 以代替 $y$
\begin{equation}\label{Sign_eq6}
y=e^{\alpha t}z
\end{equation}
\autoref{Sign_eq6} 代入到\autoref{Sign_eq7} 并由性质6\autoref{Sign_eq1} 得
\begin{equation}\label{Sign_eq8}
D^sz=e^{-\alpha t}f(x)
\end{equation}
则由\autoref{Sign_def2} , $D^{-s}e^{-\alpha t}f(x)$ 确定这个方程\autoref{Sign_eq8} 的满足条件
\begin{equation}
z|_{x=x_0}=z'|_{x=x_0}=\cdots=z^{(s-1)}|_{x=x_0}=0
\end{equation}
的解,其可以由\autoref{Sign_eq9} 确定,只需将 $f(x)$ 用 $e^{-\alpha x}f(x)$ 替代
\begin{equation}
z=\frac{1}{(s-1)!}\int_{x_0}^x(x-t)^{s-1}e^{-\alpha t}f(t)\dd t
\end{equation}
将方程\autoref{Sign_eq8} 的一般解乘以 $e^{-\alpha t}$ , 就得到方程\autoref{Sign_eq7} 一般解。由\autoref{ynfx_eq7}~\upref{ynfx} ,这个方程的一般解是
\begin{equation}
\begin{aligned}
y&=(D-\alpha)^{-s}f(x)+e^{\alpha x}P_{s-1}(x)\\
&=\frac{e^{\alpha x}}{(s-1)!}\int_{x_0}^x(x-t)^{s-1}e^{-\alpha t}f(t)\dd t+e^{\alpha t}P_{s-1}(x)
\end{aligned}
\end{equation}
其中,$P_{s-1}(x)$ 是 $x$ 的具有任意系数的 $s-1$ 次多项式。

特别的, $f(x)=0$ 时,得到方程
\begin{equation}
(D-\alpha)^sy=0
\end{equation}
的一般解的形状
\begin{equation}
y=e^{\alpha t}P_{s-1}(x)
\end{equation}

\end{example}
利用该方法的例子,可参见欧拉方程\upref{Eulequ}的求解。
