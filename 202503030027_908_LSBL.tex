% 罗素悖论(综述)
% license CCBYSA3
% type Wiki

本文根据 CC-BY-SA 协议转载翻译自维基百科\href{https://en.wikipedia.org/wiki/Russell\%27s_paradox}{相关文章}。

在数学逻辑中,罗素悖论(也称为罗素反义命题)是由英国哲学家和数学家伯特兰·罗素于1901年提出的一个集合论悖论。罗素悖论表明,任何包含不受限制的理解原理的集合论都会导致矛盾。根据不受限制的理解原理,对于任何足够明确定义的属性,都存在一个集合,包含所有且仅包含具有该属性的对象。设\(R\)为所有不属于自身的集合的集合(这个集合有时被称为“罗素集合”)。如果\(R\)不属于自身,则根据其定义,它必须属于自身;然而,如果它属于自身,那么它就不属于自身,因为它是所有不属于自身的集合的集合。由此产生的矛盾就是罗素悖论。用符号表示如下:

设 
\[
R = \{ x \mid x \notin x \}~
\]
那么
\[
R \in R \iff R \notin R~
\]
罗素还展示了该悖论的一个版本可以在德国哲学家和数学家戈特洛布·弗雷格所构建的公理化系统中推导出来,从而破坏了弗雷格试图将数学归约为逻辑的尝试,并对逻辑主义的程序提出了质疑。避免该悖论的两种有影响力的方式都在1908年提出:罗素的类型理论和泽梅洛集合论。特别是,泽梅洛的公理限制了无限理解原理。随着亚伯拉罕·弗兰克尔的进一步贡献,泽梅洛集合论发展成了现在标准的泽梅洛–弗兰克尔集合论(当包含选择公理时,通常称为ZFC)。罗素和泽梅洛解决悖论的主要区别在于,泽梅洛修改了集合论的公理,同时保持了标准的逻辑语言,而罗素则修改了逻辑语言本身。ZFC的语言,在托拉夫·斯科勒姆的帮助下,最终被证明是第一阶逻辑的语言。[4]

该悖论早在1899年就由德国数学家恩斯特·泽梅洛独立发现。[5]然而,泽梅洛并未发表这一想法,这一想法仅为大卫·希尔伯特、埃德蒙德·胡塞尔和哥廷根大学的其他学者所知。在1890年代末,现代集合论的创始人乔治·康托尔就已经意识到他的理论会导致矛盾,他通过信件告诉希尔伯特和理查德·德德金德。[6]
\subsection{非正式的呈现}  
大多数常见的集合都不是它们自己的成员。我们称一个集合为“正常的”,如果它不是它自己的成员;如果它是它自己的成员,则称其为“非正常的”。显然,每个集合必须是正常的或非正常的。例如,考虑平面上的所有正方形的集合。这个集合本身不是平面上的一个正方形,因此它不是它自己的成员,因此是正常的。相比之下,包含所有非正方形的补集本身不是平面上的正方形,因此它是它自己的成员,因此是非正常的。

现在我们考虑所有正常集合的集合\(R\),并尝试确定R是正常的还是非正常的。如果\(R\)是正常的,那么它将包含在所有正常集合的集合中(即它自身),因此它是非正常的;另一方面,如果\(R\)是非正常的,它就不包含在所有正常集合的集合中(即它自身),因此它是正常的。这导致了一个结论:\(R\)既不是正常的,也不是非正常的:这就是罗素悖论。

\subsection{正式的呈现}  
“天真集合论”这个术语有多种用法。在一种用法中,天真集合论是一个形式理论,使用一阶语言并具有二元非逻辑谓词  
\(\in\),包括外延公理:
\[
\forall x\,\forall y\,(\forall z\,(z \in x \iff z \in y) \implies x = y)~
\]
以及不受限制的理解公理模式:
\[
\exists y \forall x (x \in y \iff \varphi(x))~
\]
对于任何谓词\(\varphi\),其中\(x\)是\(\varphi\)中的自由变量。将  
\(x \notin x\)代入\(\varphi(x)\)得到:
\[
\exists y \forall x (x \in y \iff x \notin x)~
\]
然后通过存在性实例化(重用符号\(y\))和全称实例化,我们得到:
\[
y \in y \iff y \notin y~
\]
这是一个矛盾。因此,这种天真集合论是不一致的。[7]
\subsection{哲学意义}  
在罗素悖论(以及同时期发现的其他类似悖论,如布拉利-福尔蒂悖论)之前,关于集合的一个普遍观念是“外延集合的概念”,正如冯·诺伊曼和莫根斯坦所叙述的:

集合是一个任意的对象集合,对于这些对象的性质和数量没有任何限制,这些对象是集合中的元素。集合的元素构成并决定了集合本身,元素之间没有任何顺序或关系。

特别地,集合和适当类作为对象的集合之间没有区分。此外,集合中每个元素的存在被视为该集合存在的充分条件。然而,像罗素悖论和布拉利-福尔蒂悖论这样的悖论通过一些对象集合的例子,展示了这种集合观念的不可行性,尽管所有这些对象都是存在的,但它们并不能形成集合。
\subsection{集合论的回应}  
根据经典逻辑的爆炸原理,从矛盾中可以证明任何命题。因此,像罗素悖论这样的矛盾在公理化集合论中的存在是灾难性的;因为如果任何公式都可以被证明为真,它就破坏了传统的真与假的意义。此外,由于集合论被视为所有其他数学分支公理化发展的基础,罗素悖论威胁到了整个数学的基础。这促使了20世纪初大量研究,旨在发展一种一致的(无矛盾的)集合论。

1908年,恩斯特·泽梅洛提出了一种集合论的公理化方法,通过用较弱的存在公理(如他的分离公理,Aussonderung)替换任意集合理解,从而避免了天真集合论中的悖论。(避免悖论并非泽梅洛的初衷,他的目的是记录他在证明良序定理时所使用的假设。)1920年代,亚伯拉罕·弗兰克尔、托拉夫·斯科勒姆和泽梅洛本人对这一公理化理论进行修改,最终得出了称为ZFC的公理化集合论。自泽梅洛的选择公理不再具有争议后,这一理论得到了广泛接受,ZFC一直保持到今天,成为公认的标准公理化集合论。

ZFC并不假设对于每一个属性,都存在一个包含所有满足该属性的事物的集合。相反,它断言,给定任何集合\(X\),\(X\)的任何子集,只要使用一阶逻辑可定义,就存在。由上文罗素悖论定义的对象\(R\)无法作为任何集合\(X\)的子集构造,因此在ZFC中它不是一个集合。在ZFC的某些扩展中,特别是在冯·诺伊曼–伯奈斯–哥德尔集合论中,像\(R\)这样的对象被称为适当类。

ZFC 对类型没有明确说明,尽管其累积层次结构有类似于类型的层次概念。泽尔梅洛本人从未接受斯科勒姆使用一阶逻辑语言来表述 ZFC。正如何塞·费雷罗斯所指出的,泽尔梅洛坚持认为,“用于区分子集的命题函数(条件或谓词)以及替代函数,可以是‘完全任意的’[ganz beliebig]”;现代对这一声明的解释是,泽尔梅洛希望引入高阶量化,以避免斯科勒姆悖论。大约在 1930 年,泽尔梅洛还引入了(显然是独立于冯·诺依曼的)基础公理,从而——正如费雷罗斯所观察到的——“通过禁止‘循环’和‘无根’集合,它 [ZFC] 融入了 TT [类型理论] 的一个关键动机——即‘论证类型的原则’”。泽尔梅洛偏好的这个包括基础公理的二阶 ZFC 允许了丰富的累积层次。费雷罗斯写道:“泽尔梅洛的‘层次’本质上与哥德尔和塔尔斯基提出的简单类型理论 [TT] 当代版本中的类型相同。我们可以将泽尔梅洛发展其模型的累积层次结构描述为一个允许无限类型的累积类型理论的宇宙。(一旦我们采用无定义的观点,放弃类是构造出来的这个想法,接受无限类型就不再是反自然的了。)因此,简单类型理论和 ZFC 现在可以看作是‘谈论’本质上相同的预期对象的系统。主要区别在于,类型理论依赖于强大的高阶逻辑,而泽尔梅洛使用了二阶逻辑,ZFC 也可以用一阶形式表述。累积层次结构的一阶‘描述’要弱得多,这一点通过可数模型的存在(斯科勒姆悖论)得到了体现,但它也享有一些重要的优势。”[10]

在ZFC中,给定一个集合\(A\),可以定义一个集合\(B\),它由\(A\)中恰好那些不属于自身的集合构成。\(B\)不能属于\(A\),因为按照罗素悖论中的相同推理,\(B\)无法在\(A\)中。这一变体的罗素悖论表明,没有任何集合包含一切。

通过泽梅洛等人的工作,特别是约翰·冯·诺依曼的贡献,ZFC所描述的某些人认为的“自然”对象的结构最终变得清晰:它们是冯·诺依曼宇宙\(V\)的元素,\(V\)是通过对空集进行跨无限次幂集运算迭代构建的。因此,现在可以再次以非公理化的方式推理集合,而不会陷入罗素悖论,即通过推理\(V\)的元素。是否适合以这种方式看待集合是数学哲学中不同观点之间的争议点。

其他解决罗素悖论的方案,其基本策略更接近类型理论,包括奎因的新基础理论和斯科特–波特集合论。另一个方法是通过定义多重成员关系,并采用适当修改的理解方案,例如在双重扩展集合论中所做的那样。
\subsection{历史}  
罗素于1901年5月或6月发现了这个悖论。[11][12] 根据他在1919年《数学哲学导论》中的自述,他“试图发现康托尔证明不存在最大基数中的某个缺陷。”[13] 在1902年的一封信中,[14] 他向戈特洛布·弗雷格通报了在弗雷格1879年《 Begriffsschrift》中的悖论,并且将这个问题框定在逻辑和集合论的术语中,特别是通过弗雷格对函数的定义:[a][b]

“我遇到的唯一困难点在于你指出(第17页 [上面第23页])一个函数也可以作为不确定的元素。我以前曾认为如此,但现在由于以下的矛盾,我对这一观点产生了怀疑。设\(W\)为谓词:是一个不能被自我断言的谓词。\(W\)能否被自我断言?从每个答案推导出相反的结论。因此我们必须得出结论,\(W\)不是一个谓词。同样,没有任何类(作为一个整体)包含那些每个被当作整体时不属于自己的类。从这个结论出发,我认为在某些情况下,一个可定义的集合[Menge]不能形成一个整体。”

罗素在1903年《数学原理》中详细讨论了这个悖论,他在书中重复了他首次遇到这一悖论的经历:[15]

“在离开基础性问题之前,有必要更详细地研究之前提到的关于不能自我断言的谓词的特殊矛盾……我可以提到,我是试图调和康托尔的证明时,才导致了这个悖论。”

罗素在弗雷格准备《算术原理》第二卷时给弗雷格写信,讨论了这个悖论。[16] 弗雷格迅速回应了罗素;他的回信日期为1902年6月22日,并且在1967年Heijenoort的评论中刊登:Heijenoort 1967:126–127。弗雷格随后写了一篇附录,承认了这一悖论,[17] 并提出了一个解决方案,这个方案后来被罗素在《数学原理》中采纳,[18] 但有些人后来认为这个解决方案不令人满意。[19] 至于罗素,他的书籍已进入排版阶段,并且他在书中增加了一个关于类型论的附录。[20]

恩斯特·泽梅洛在他1908年的《关于良序可能性的一个新证明》中(同时他也发布了“第一个公理化集合论”)宣称自己早于罗素发现了康托尔天真集合论中的反论证。他表示:“然而,即便是罗素所给出的集合论反论证的基本形式,也本应说服他们(J. König、Jourdain、F. Bernstein),这些困难的解决方案并非应通过放弃良序性来寻找,而应通过适当限制集合的概念来解决。”[22] 注释9中,他明确提出了他的主张:

“1903年,第366–368页。我自己实际上在独立于罗素的情况下发现了这个反论证,并且在1903年前已将其告知包括希尔伯特教授在内的其他人。”[23]

弗雷格将《算术原理》的一本副本寄给了希尔伯特;如上所述,弗雷格的最后一卷提到了罗素传达给弗雷格的悖论。在收到弗雷格的最后一卷后,希尔伯特于1903年11月7日写信给弗雷格,在信中提到,谈到罗素悖论时,他说:“我相信泽梅洛三四年前就发现了这个悖论。”泽梅洛实际论证的书面记录后来在埃德蒙德·胡塞尔的遗稿中被发现。[24]

1923年,路德维希·维特根斯坦提出了以下方法来“解决”罗素悖论:

“一个函数不能是它自己的参数的原因在于,函数的符号已经包含了其参数的原型,而它不能包含自身。假设函数\textbf{F(fx)}可以是它自己的参数:在这种情况下,就会出现命题\textbf{F(F(fx))},其中外部函数F和内部函数F必须有不同的含义,因为内部函数的形式是\textbf{O(fx)},而外部函数的形式是\textbf{Y(O(fx))}。只有字母‘F’是这两个函数的共同部分,但字母本身并不表示任何东西。如果我们把\textbf{F(Fu)}改为(做):\textbf{F(Ou)} . \textbf{Ou = Fu},这就解决了罗素的悖论。”(《逻辑哲学论》3.333)

罗素和阿尔弗雷德·诺思·怀特海德写了三卷本的《数学原理》,希望实现弗雷格未能做到的目标。他们试图通过采用一种为此目的而设计的类型论来驱除天真集合论中的悖论。尽管他们在某种程度上成功地将算术建立在逻辑基础上,但并不明显他们是通过纯粹的逻辑手段做到的。虽然《数学原理》避免了已知的悖论,并且允许推导出大量的数学内容,但其系统也引发了新的问题。

无论如何,库尔特·哥德尔在1930至1931年证明了,尽管《数学原理》中的大部分逻辑(现在被称为一阶逻辑)是完备的,但如果它是一致的,皮亚诺算术必定是不完备的。这在广泛的(但并非普遍的)看法中,被认为已经表明弗雷格的逻辑主义计划不可能完成。

2001年,在慕尼黑举办了庆祝罗素悖论发现一百周年的国际大会,其会议论文集已被出版。[12]
\subsection{应用版本}
有一些版本的这个悖论更接近现实生活的情况,可能对于非逻辑学家来说更容易理解。例如,理发师悖论假设有一个理发师,他剃所有不剃自己胡子的男人,而且只剃那些不剃自己胡子的男人。当人们思考理发师是否应该剃自己胡子时,一个类似的悖论开始显现。[25]

对“外行版本”如理发师悖论的一个简单驳斥似乎是,根本没有这样的理发师存在,或者理发师不是男人,因此可以存在而不产生悖论。罗素悖论的核心在于,“这样的集合不存在”意味着在给定理论中,集合的概念定义是不满意的。请注意“这样的集合不存在”和“它是一个空集合”之间的区别。这就像是“没有桶”和“桶是空的”之间的区别。

上述情况的一个显著例外可能是格雷林–纳尔逊悖论,其中单词和意义是情境中的元素,而不是人和理发。尽管可以通过说这样的理发师不存在(且不可能存在)来轻松驳斥理发师悖论,但关于一个有意义定义的词语,却无法做出类似的回答。

这个悖论的一种戏剧化表现方式如下:假设每个公共图书馆都必须编制所有书籍的目录。由于目录本身也是图书馆的书籍之一,一些图书馆员将其包含在目录中以确保完整性;而另一些图书馆员则将其排除在外,因为目录作为图书馆的一本书是显而易见的。现在,假设所有这些目录被送到国家图书馆。部分目录将自己包括在内,而其他则没有。国家图书馆员编制了两个主目录——一个列出了所有包含自己的目录,另一个列出了所有没有包含自己的目录。

问题是:这些主目录应该列出它们自己吗?“所有列出自己的目录的目录”没有问题。如果图书馆员不将它包括在自己的列表中,它仍然是那些列出自己的目录的真实目录。如果他将其包括在内,它仍然是那些列出自己的目录的真实目录。然而,正如图书馆员在第一个主目录中不会犯错一样,他注定会在第二个主目录中失败。当涉及到“所有不列出自己的目录的目录”时,图书馆员不能将其包括在自己的列表中,因为那样它就会列出自己,从而属于另一个目录——那些列出自己的目录的目录。然而,如果图书馆员将其排除在外,那么目录就不完整。无论哪种方式,它都不可能成为一个真实的、不列出自己的目录的主目录。
\subsection{应用及相关主题}  
\subsubsection{罗素式悖论}  
如上所述,罗素悖论并不难扩展。以以下为例:
\begin{itemize}
\item 一个及物动词 ⟨V⟩,可以应用于其名词形式。  
构成句子:
\end{itemize}
“⟨V⟩ 所有(且仅有那些)不 ⟨V⟩ 自己的人。”  
有时“所有”被替换为“所有 ⟨V⟩ 者”。

例如,“paint”(画):

“画家,画所有(且仅有那些)不画自己的人。”  
或“elect”(选举):

“选举者(代表),选举所有不选举自己的人。”

在《生活大爆炸》第八季的剧集《天行者的干扰》中,谢尔顿·库珀分析了歌曲《Play That Funky Music》,得出结论认为歌词展示了罗素悖论的一个音乐示例。[26]

符合这一模式的悖论包括:
\begin{itemize}
\item 理发师悖论(“刮胡子”)。
\item 原始的罗素悖论(“包含”):包含所有(不包含自己的)集合(容器)。
\item 格雷林–纳尔逊悖论(“描述者”):描述所有不描述自己的单词(描述者)。
\item 理查德悖论(“表示”):表示所有不表示自己的数字(表示者)的数字。
  (在这个悖论中,所有数字的描述都被赋予了一个编号。这里称为“表示所有不表示自己的数字的表示者”是理查德式的。)
\item “我在撒谎。”,即撒谎者悖论和埃皮门尼德悖论,它们的起源可以追溯到古代。
\item 罗素–迈希尔悖论。
\end{itemize}
\subsection{相关悖论}
\begin{itemize}
\item 布拉利-福尔蒂悖论,关于所有良序的顺序类型。
\item 克利尼–罗斯悖论,利用自我否定的陈述证明原始的λ演算是不一致的。
\item 库里的悖论(以哈斯凯尔·库里命名),不需要否定。
\item 最小无趣整数悖论。
\item 吉拉尔的悖论,类型理论中的悖论。
\end{itemize}
\subsection{另见}
\begin{itemize}
\item 基本法则V
\item 康托尔对角线论证 – 集合论中的证明
\item 哥德尔不完全性定理 – 数学逻辑中的限制性结果
\item 希尔伯特的第一问题 – 数学逻辑中的命题
\item 《论表示》
\item 集合论悖论
\item 奎因悖论
\item 自指
\item 自指悖论列表
\item 奇异循环 – 在分层系统中经过多个层级的循环结构
\item 普遍集合 – 包含所有对象的数学集合
\end{itemize}
\subsection{注释}\\
a.以下内容中,p. 17指的是原版《概念符号》中的一页,p. 23指的是同一页在 van Heijenoort 1967 版中的对应页。\\
b.值得注意的是,这封信直到 van Heijenoort 1967 年才出版——它与 van Heijenoort 的评论一起出现在 van Heijenoort 1967:124–125。
\subsection{参考文献}
\begin{enumerate}
\item 罗素,伯特兰,《与弗雷格的通信》,载于《戈特洛布·弗雷格的哲学与数学通信》。汉斯·卡尔译,芝加哥大学出版社,芝加哥,1980年。
\item 罗素,伯特兰,《数学原理》,第二版,重印,纽约:W.W. Norton & Company,1996年。(首次出版于1903年。)
\item 欧文,A.D., 德语,H.(2021)。“罗素悖论”。《斯坦福哲学百科全书》(2021年春季版),E.N. Zalta(编),[1]
\item A.A. 弗兰克尔;Y. 巴尔-希勒尔;A. 莱维(1973)。《集合论基础》。Elsevier出版社,第156-157页。ISBN 978-0-08-088705-0。
\item 伯恩哈德·兰格,沃尔夫冈·托马斯:《泽梅洛发现“罗素悖论”》,《数学史》8期。
\item 沃尔特·普尔克特,汉斯·J. 伊尔高德:《数学人生——乔治·康托尔》,Birkhäuser,1986年,ISBN 3-764-31770-1。
\item 欧文,安德鲁·大卫;德语,哈里(2014)。“罗素悖论”。载于 Zalta, Edward N.(编),《斯坦福哲学百科全书》。
\item R. Bunn, 《无限集合与数字》(1967年),第176-178页,博士论文,英属哥伦比亚大学。
\item P. Maddy,《相信公理 I》(1988年),符号逻辑学会。
\item José Ferreirós(2008年)。《思维的迷宫:集合论的历史及其在现代数学中的作用》(第二版)。Springer出版,§ 泽梅洛的累积层次,第374-378页。ISBN 978-3-7643-8350-3。

\end{enumerate}