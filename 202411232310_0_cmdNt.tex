% Windows cmd 命令行笔记
% license Usr
% type Note

\begin{issues}
\issueDraft
\end{issues}

\begin{itemize}
\item cmd 脚本文件拓展名是 \verb`.bat`
\item cmd 窗口用鼠标 + 滚轮即可
\item \verb`.ps1` 是 PowerShell 的脚本
\end{itemize}

以下对比 linux 命令行
\begin{itemize}
\item \verb`dir` 相当于 \verb`ls`
\item \verb`cd`, \verb`mkdir`, \verb`rmdir`,\verb`clear`, \verb`exit`, \verb`上下箭头` 和 linux 类似
\item \verb`cd` 不能改变盘符, 需要直接打 \verb`D:`
\item \verb`del` 删除文件
\item \verb`copy` 相当于 \verb`cp`
\item \verb`move` 相当于 \verb`mv`
\end{itemize}

\subsubsection{快捷键}
\begin{itemize}
\item \verb`F7` 搜索历史命令,但只能是当前 session 的,基本没办法持久保留。
\item 在标题栏右键 -> Edit, 可以看到一些基本操作以及快捷键
\item \verb`Enter` 可以复制选中的命令
\item \verb`Ctrl + V` 粘贴
\item \verb`Ctrl + A` 全选
\item \verb`Ctrl + F` 搜索(整个窗口)
\end{itemize}
