% 斯托克斯定理(矢量分析)
% keys 曲面|闭合曲线|环流量

\begin{issues}
\issueTODO
\end{issues}

\pentry{旋度\upref{Curl}}

\begin{figure}[ht]
\centering
\includegraphics[width=6cm]{./figures/Stokes_1.pdf}
\caption{斯托克斯定理} \label{Stokes_fig1}
\end{figure}

如\autoref{Stokes_fig1}, 我们选取一块曲面, 并规定一个正方向。 使用右手定则\upref{RHRul}, 我们也可以定义曲面边界的正方向。 空间中存在连续光滑的矢量场 $\bvec F(\bvec r)$, 则\textbf{斯托克斯定理}可以将矢量场在曲面边界上的环流量和矢量场的旋度在曲面上通量等同起来
\begin{equation}\label{Stokes_eq1}
\oint \bvec F(\bvec r) \vdot \dd{\bvec r} = \int \curl \bvec F(\bvec r) \vdot \dd{\bvec s}
\end{equation}

要证明这个定理, 我们将曲面划分为许多小面元 $\Delta \bvec s_i$, 其正方向与曲面一致, 边界的正方向同样由右手定则定义。 这样, 矢量场在曲面上的通量就等于在每个小面元上的通量之和。 当面元的面积趋于零时, 我们可以认为场的旋度在面元上是常矢量 $\bvec F(\bvec r_i)$, $\bvec r_i$ 为 $\Delta \bvec s_i$ 上任意一点。 由\autoref{Curl_eq3}~\upref{Curl} 可知面元的环流量为 $\curl\bvec F(\bvec r_i) \vdot \Delta \bvec s_i$(可类比\autoref{Diff_eq2}~\upref{Diff}), 所以根据积分的思想, 所有面元的环流量之和为
\begin{equation}
\lim_{\Delta s_i \to 0}\sum_i \curl\bvec F(\bvec r_i) \vdot \Delta \bvec s_i = \int \curl\bvec F(\bvec r) \vdot \dd{\bvec s}
\end{equation}
最后, 如何证明所有面元的环流量之和等于曲面边界的环流量呢? 类比散度定理(\autoref{Divgnc_eq13}~\upref{Divgnc})的证明, 考虑任意两块相邻的小面元, 矢量场在它们共同边界的线积分对一个面元的环流量贡献为正, 而对另一个面元的环流量贡献大小相同但符号为负, 所以在上式的求和中相加为零。 所以, 求和中唯一没有被抵消的环流量来自于曲面边界处的面元, 这些面元的边界与曲面边界重合且正方向一致, 对求和的贡献恰好等于曲面边界的环流量。 证毕。

\addTODO{例题}

\subsection{旋度的散度}
\begin{theorem}{}
任意矢量场旋度的散度恒为零。
\end{theorem}
证明: 在一个闭合曲面上任意取一闭合曲线。 这样闭合曲面被划分为两个以曲线为边界的非闭合曲面。 若两个曲面的正方向都向外, 那么对两个曲面来说曲线的正方向相反。 对两个曲面分别使用斯托克斯定理, 那么矢量场在曲线上的线积分也相反, 即矢量场的旋度在它们上的通量之和为零。 由于这里的曲面和曲线是任意选取的, 这说明旋度的散度恒为零。

\subsection{无旋场的线积分与势函数}
“任意回路的环积分为零” 的一种等效说法是 “线积分与路径无关”: 我们可以把任意回路拆成两条, 由于环积分为零, 延着回路正方向, 第一条的线积分与第二条的相反。 但若把第二条延反方向积分, 那么两个积分则必定相等。
\addTODO{画图说明}

那么 “线积分与路径无关” 的表述在理解上有什么好处呢? 答案就是可以定义\textbf{势函数}。 如果选取一点作为零势点, 那么矢量场中的每一点的势能可以定义为
\addTODO{未完成, 参考 “力场 保守场 势能\upref{V}”}
