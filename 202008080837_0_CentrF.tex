% 圆周运动的向心力
% 向心力|圆周运动

\pentry{圆周运动的加速度\upref{CMAD}, 牛顿运动定律\upref{New3}}

在惯性参考系中, 质点的运动符合牛顿第二定律 $\bvec F = m\bvec a$. 所以要使一个质点做半径为 $R$ 的圆周运动, 那么我们就可以把任意时刻圆周运动的加速度矢量(\autoref{CMAD_eq4}~\upref{CMAD})乘以质量 $m$ 得到它该时刻受到的合力. 我们把这个力叫做向心力
\begin{equation}
\bvec F_c = -m \omega^2 \bvec r
\end{equation}

\subsection{有趣的推导}
为了更好地理解为何向心力会产生圆周运动, 我们可以假设一个小球在运动的过程中被反弹若干次, 使得它的轨道称为闭合的正 $N$ 边形. 每次撞击后, 速度不变, 方向改变 $2\theta  = 2\pi/N$. % 图未完成
所以在每次碰撞时, 我们可以把速度垂直分解为径向(指向圆心)和角向速度. 碰撞后角向速度不变, 径向速度增量为 $2v\sin\theta$, 即动量改变为
\begin{equation}
\Delta p = 2mv\sin\theta
\end{equation}
正 $N$ 边形每条边的长度为 $L = 2R\sin(\theta)$, 所以相邻两次碰撞的时间为
\begin{equation}
\Delta t = \frac{2R\sin\theta}{v}
\end{equation}
所以平均来说, 力等于动量改变除以时间
\begin{equation}
F_c = \frac{\Delta p}{\Delta t} = \frac{mv^2}{R}
\end{equation}

