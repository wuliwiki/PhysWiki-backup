% 泡利不相容原理(综述)
% license CCBYSA3
% type Wiki

本文根据 CC-BY-SA 协议转载翻译自维基百科\href{https://en.wikipedia.org/wiki/Pauli_exclusion_principle}{相关文章}。

\begin{figure}[ht]
\centering
\includegraphics[width=6cm]{./figures/5713325387bd5c08.png}
\caption{沃尔夫冈·泡利在1929年哥本哈根的一次讲座中。[1] 沃尔夫冈·泡利提出了泡利不相容原理。} \label{fig_PLBXR_1}
\end{figure}

在量子力学中,泡利不相容原理(德语:Pauli-Ausschlussprinzip)指出,在遵循量子力学定律的系统中,两个或多个具有半整数自旋(即费米子)的相同粒子不能同时占据相同的量子态。奥地利物理学家沃尔夫冈·泡利于1925年首先为电子提出了这一原理,随后在1940年通过自旋-统计定理将其推广至所有费米子。

对于原子中的电子,不可违背原理可以表述如下:在一个多电子原子中,不可能有两个电子的所有四个量子数取值都相同。这四个量子数分别是:主量子数 \( n \),角量子数 \( \ell \),磁量子数 \( m_\ell \),以及自旋量子数 \( m_s \)。例如,如果两个电子处于同一个轨道中,那么它们的 \( n \)、\( \ell \) 和 \( m_\ell \) 值相等。在这种情况下,它们的自旋量子数 \( m_s \) 值必须不同。由于自旋量子数 \( m_s \) 仅能取 \( +1/2 \) 或 \( -1/2\),因此其中一个电子必须具有 \( m_s = +1/2 \),另一个必须具有 \( m_s = -1/2 \)。

具有整数自旋的粒子(玻色子)不受泡利不相容原理的约束。任意数量的相同玻色子可以占据同一量子态,例如激光产生的光子或玻色-爱因斯坦凝聚态中的原子。  

更严格的表述是:在交换两个相同粒子的情况下,总(多粒子)波函数对于费米子是反对称的,而对于玻色子是对称的。这意味着,如果交换两个相同粒子的空间和自旋坐标,则总波函数对费米子会改变符号,而对玻色子不会改变符号。

因此,假设两个费米子处于相同的状态——例如,在同一原子的同一轨道上且具有相同的自旋——那么交换它们后系统不会发生任何变化,总波函数也应保持不变。然而,对于费米子来说,总波函数必须在交换两个粒子时改变符号,而同时又保持不变的唯一可能性是该波函数在所有地方都为零,这意味着这种状态不可能存在。这种推理对玻色子不适用,因为对于玻色子而言,交换粒子时波函数的符号不会改变。

