% 2017 年计算机学科专业基础综合全国联考卷
% 2017 年计算机 全国 联考卷


\subsection{一、单项选择题}
1~40小题,每小题2分,共80分.下列每题给出的四个选项中,只有一个选项符合题目要求.

1.下列函数的时间复杂度是 \\
\begin{lstlisting}[language=cpp]
int func(int n)
{
    int i=0,sum=0;
    while(sum<n)
        sum+=++i;
    return i;
}
\end{lstlisting}
A. $O(log n)$  B.$O(n^{l/2})$    C.$0(n)$    D.$O(nlogn)$
    
2.下列关于栈的叙述中,错误的是 \\
I.采用非递归方式重写递归程序时必须使用栈 \\
II.函数调用时,系统要用栈保存必要的信息 \\
III.只要确定了入栈次序,即可确定出栈次序 \\
Ⅳ,栈是一种受限的线性表,允许在其两端进行操作 \\
A.仅I  $\quad$  B.仅I、II、III \\
C.仅I、Ⅲ、Ⅳ  $\quad$  D.仅II、III、Ⅳ

3.适用于压缩存储稀疏矩阵的两种存储结构是 \\
A.三元组表和十字链表 $\quad$ B.三元组表和邻接矩阵 \\
C.十字链表和二叉链表 $\quad$ D.邻接矩阵和十字链表

4.要使一棵非空二叉树的先序序列与中序序列相同,其所有非叶结点须满足的条件是 \\
A.只有左子树  $\quad$  B.只有右予树 \\
C.结点的度均为1 $\quad$ D.结点的度均为2

5.己知一棵二叉树的树形如下图所示,其后序序列为e,a,c,b.d,g,f,树中与结点a同层的结点是 \\
\begin{figure}[ht]
\centering
\includegraphics[width=5cm]{./figures/CSN17_1.png}
\caption{第5题图} \label{CSN17_fig1}
\end{figure}
A.C $\quad$ B.d $\quad$ C.f $\quad$ D.g

6.己知字符集{a,b,c,d,e,f,g,h},若各字符的哈夫曼编码依次是0100, 10, 0000, 0101, 001, 011, 11, 0001,则编码序列010001 100100101 1 1 10101的译码结果是 \\
A. acgabfh  $\quad$  B. adbagbb  \\
C. afbeagd  $\quad$  D. afeefgd

7.己知无向图G含有16条边,其中度为4的顶点个数为3,度为3的顶点个数为4,其他顶点的度均小于3.图G所含的顶点个数至少是 \\
A.10  $\quad$  B.11  $\quad$  C.13  $\quad$  D.15

8.下列二叉树中,可能成为折半查找判定树f不含外部结点1的是 \\
\begin{figure}[ht]
\centering
\includegraphics[width=14.25cm]{./figures/CSN17_2.png}
\caption{第8题图} \label{CSN17_fig2}
\end{figure}

9.下列应用中,适合使用B+树的是 \\
A.编译器中的词法分析 $\quad$ B.关系数据库系统中的索引 \\
C.网络中的路由表快速查找  $\quad$  D.操作系统的磁盘空闲块管理

10.在内部排序时,若选择了归并排序而没有选择插入排序,则可能的理由是 \\
I.归并排序的程序代码更短 \\
II.归并排序的占用空间更少 \\
III.归并排序的运行效率更高 \\
A.仅II  $\quad$  B.仅III  $\quad$  C.仅I、II  $\quad$  D.仅I、III

11.下列排序方法中,若将顺序存储更换为链式存储,则算法的时间效率会降低的是 \\
I.插入排序    II.选择排序  Ⅲ,起泡排序 \\
Ⅳ.希尔排序    V.堆排序 \\
A.仅I、II $\quad
$ B.仅II、III    C.仅Ⅲ、Ⅳ    D.仅Ⅳ、V

12.假定计算机Ml和M2具有相同的指令集体系结构(I SA),主频分别为1.5 GHz和1.2 GHz.在Ml和M2上运行某基准程序P,平均CP 1分别为2和1,则程序P在Ml和M2上运行时问的比值是 \\
A. 0.4    B. 0.625    C. 1.6    D. 2.5

13.某讣算机主存按字节编址,由4个64Mx8位的DRAM芯片采用交叉编址方式构成,并与宽度为32位的存储器总线相连,主存每次最多读写32位数据.若double型变量x的主存地址为804OOIAH,则读取x需要的存储周期数是 \\
A.1    B.2    C.3    D.4

14.某C语言程序段如下: \\
\begin{lstlisting}[language=cpp]
for(i=0;i<=9;i++1
    {temp=l;
    for(j=0;j<=i;j++)temp木=a[j];
    sum+ =temp;
    }
\end{lstlisting}
下列关于数组a的访问局部性的描述中,正确的是