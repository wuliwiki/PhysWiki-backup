% 厦门大学 2002 年 考研 量子力学
% license Usr
% type Note

\textbf{声明}:“该内容来源于网络公开资料,不保证真实性,如有侵权请联系管理员”

\subsection{(20分)回答和计算下列问题}
\begin{enumerate}
    \item  什么叫全同粒子?Pauli 原理是如何表述的?

    \item  什么是简并度?一维运动的束缚态所对应的能级是不是简并?(若是简并要说明是几重简并)

    \item  试计算:
    \[    \left[\hat{\vec{L}}^2_z, x\right] = ?  ~\]
    其中 $\hat{\vec{L}}$ 为轨道角动量。

    \item  体系的哈密顿量为:
    \[    \hat{H} = \frac{\hat{\vec{p}}^2}{2\mu} + A\hat{L}_z,  ~\]
    其中 $\hat{\vec{p}}^2$ 为动量算符的平方。

    $\hat{L}_z$ 为角动量$\hat{L}$的$Z$分量。给出 $ \mu, A$ 为常数。试判断下列力学量哪些是守恒量:
    \[  \hat{P}_x, \hat{P}_y, \hat{P}_z,\hat{\vec{p}}^2, \hat{L}_x, \hat{L}_y, \hat{L}_z \hat{\vec{L}}^2~\]

    \item  写出坐标 $Y$ 与动量 $\hat{P}_r$ 的测不准关系式。
\end{enumerate}
\subsection{(20分)}
一个质量为 $m$ 的微观粒子在一维无限深势阱中运动

\[U(x) = \begin{cases} 0 & 0 < x < a \\\\\infty & x \leq 0, x \geq a \end{cases}~\]

其本征方程为

\[\hat{H} \varphi_n (x) = E_n \varphi_n (x), \quad n = 1, 2, 3, \ldots~\]

\(\varphi_n (x)\), \(E_n\) 分别为本征函数,本征值。

设 \( t = 0 \) 时粒子的初态波函数为

\[\psi (0) = c_1 \varphi_1 (x) + c_2 \varphi_2 (x) + c_3 \varphi_3 (x) + c_4 \varphi_4 (x)~\]

\(c_1, c_2, c_3, c_4\) 均为常数。

问\begin{itemize}
  \item  对 \( \psi (0) \) 态,粒子的能量值,发现其值在 \( \frac{3\pi^2 \hbar^2}{ma^2} \) 以下的几率是多少?
  \item  对 \( \psi (0) \) 态,粒子能量的平均值是多少?
  \item  写出任意时刻 \( t \) 的波函数 \( \psi (t) \) 的表达式。
\end{itemize}
\subsection{(15 分)}
 慢速粒子受到势能为
\[U(r) = \begin{cases} U_0, & r \leq a \\\\0, & r > a \end{cases}~\]
的场的散射,若 $E > U_0, U_0 < 0,$ (即势阱), $U_0$ 为常数。试用分波法求散射截面(只考虑 $S$ 波项)
\subsection{(15 分)}
(1)两个自旋为$1/2$的粒子形成一个复合体系,自旋 $A$ 处于$S^\prime_z =+\frac{1}{2}$的本征态,自旋$B$处于$S^\prime_x=+\frac{1}{2}$的本征态,求总自旋为1且在$Z$方向投彩为0的测量几率。

(2)试求:$\hat{\vec{\sigma}_1},\hat{\vec{\sigma}_2}$对于自旋三重态波函数$X_t$,及自旋单态波函数$X_s$的本征值。$\hat{\vec{\sigma}_1},\hat{\vec{\sigma}_2}$分别为第一个粒子和第二个粒子的Paui矩
\subsection{(15 分)}
考虑一个无自旋的粒子, 其波函数为
\[\psi = k (x + y + 2z) e^{-\alpha r}~\]
其中$r = \sqrt{x^2 + y^2 + z^2}$, $ k$ 和 $a$ 是实常数, 试求 $\hat{L}_z$ 在此态上的平均值,$\hat{L}_z$ 为轨道角动量 $\hat{L}$ 的 $Z$ 分量。
\[Y_{1, \pm 1} = \mp \sqrt{\frac{3}{8\pi}} \sin \theta e^{\pm i\varphi}, \\ Y_{1,0} = \sqrt{\frac{3}{4\pi}} \cos \theta~\]
\subsection{(15 分)}
自旋为0的两个全同粒子在谐振子势场
\[U(x)=\frac{1}{2}m\omega^2x^2~\]
中作一维运动,粒子间相互作用
\[U_{int}(x_1,x_2)=U_0e^{-\beta^2(x_1-x_2)^2}~\]
其中$U_0$,$\beta$为实数。视$U_{int}(x_1,x_2$)为微扰,求修正至一级的体系基态能量。