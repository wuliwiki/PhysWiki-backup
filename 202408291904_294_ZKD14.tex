% 中国科学技术大学 2014 年考研普通物理考试试题
% keys 中国科学技术大学|考研|物理|2014年
% license Copy
% type Tutor


\textbf{声明}:“该内容来源于网络公开资料,不保证真实性,如有侵权请联系管理员”

\begin{enumerate}
\item 将一质量$m$,长为$l$的匀质柔软绳的两端合在一起悬挂于支点$O$。今使其中的一端脱离$O$而自由下落。求下落端下落距离$x$时,支点$O$所受的力$F$。
\begin{figure}[ht]
\centering
\includegraphics[width=5cm]{./figures/1c11f0ef67185d0d.png}
\caption{} \label{fig_ZKD14_1}
\end{figure}
\item 一个质量为$m$,半径为$R$的匀质圆柱体放置在与水平面成$\theta$角的斜面上,如图所示。圆柱体和斜面直接的摩擦因数为$\mu$,若要让圆柱体无滑动地沿斜面滚下来,问$\theta$角的上限是多少。
\begin{figure}[ht]
\centering
\includegraphics[width=5cm]{./figures/cebae5c17906a05e.png}
\caption{} \label{fig_ZKD14_2}
\end{figure}
\item 设质量分别为$m_1,m_2$,的两个质点相距l,开始时均处于静止状态,其间仅有万有引力相互作用。\\
(1)假设$m_1$固定不动,$m_2$将要经过多长时间后与$m_1$ 相碰。\\
(2)假设$m_1$也可动,两者将经过多少时间后相碰。
\item 一金属球带电量$Q$,半径为$a$,球外有一个半径为$b$的同心金属薄球壳,球与球壳间充满相对介电常数为$\varepsilon=(K+r)/r$的电介质,其中$K$正为常数,$r$是到球心的距离,求球与球壳的电势差。
\item 点电荷$q$与半径为$a$的接地导体球相距$d(d>a)$,求:\\
(1)q所受静电力;\\
(2)q在感应电荷电场中的电势能;\\
(3)体系总静电能。
\begin{figure}[ht]
\centering
\includegraphics[width=5cm]{./figures/4acb80001cf1f289.png}
\caption{} \label{fig_ZKD14_3}
\end{figure}
\item 一无限长圆筒内磁场均匀分布,且$dB/dt$为大于零的常数,点P、Q坐标分别为$(-a,a)$和$(a,a)$,求下列三种情形下 $P$、$Q$两点间的电势差。\\
(1)$P,Q$之间用圆弧导线连接;\\
(2)$P,Q$之间用直导线连接;\\
(3)上述两根导线同时连接,单位长度电阻均为常数 $\rho$。
\begin{figure}[ht]
\centering
\includegraphics[width=6cm]{./figures/ab3babf750123b0b.png}
\caption{} \label{fig_ZKD14_4}
\end{figure}
\item 考虑一个多电子原子,其电子组态为$1s^2 2s^2 2p^6 3s^2 3p^6 3d^{10}4s^2 4p4d$;\\
(1)如果该原子遵循$LS$耦合,写出该电子组态耦合出的原子态符号;\\
(2)这个原子是否处于基态?如果不是基态,那么基态的电子组态又是什么样的?\\
(3)写出该原子基态的原子态符号;\\
(4)如果该原子从 $1s^2 2s^2 2p^6 3s^2 3p^6 3d^{10}4s^2 4p4d$ 耦合出的原子态向基态跃迁,画出能级图及可能的允许跃迁。
\item 假设将一部分质量数为3的氢同位素(氚)通入含有正常氢气的放电管中,达到了足够在光谱仪中作观察的程度。\\
(1)若不作波长测量,你能否直接从光谱中判断出哪些是H光谱,哪些是氚光谱?\\
(2)确定应观察到的第一条巴耳末系谱线的间隔(用波长差表示)。已知质子和中子质量都是电子质量的1836倍,原子的电离能是13.6eV。
\end{enumerate}
