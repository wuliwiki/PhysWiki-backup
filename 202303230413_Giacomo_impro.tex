% 反常积分(简明微积分)
% 积分|定积分|微积分
\begin{issues}
\issueTODO
\issueDraft
\end{issues}

\pentry{牛顿—莱布尼兹公式(简明微积分)\upref{NLeib}}
\subsection{积分限为无穷的反常积分}

\subsubsection{引入}
在许多物理、数学问题中我们会遇到一些积分到无穷的积分。例如 势能(简介)\upref{POTENT} 中,我们定义势能是
$$E_P=\int ^\infty_a \bvec F_{in} \cdot \dd \bvec r$$

你可能会想,“功是力在位移上的作用量,如果物体在力的作用下运动了无穷远,那么力岂不做了无穷多的功?那这个结论还有什么意义?”结果真的是这样吗?

我们先看一个简单的例子:$\int^a_1 \frac{1}{x^2} \dd x$ (许多有心力遵循平方反比律),并令$a\to+\infty$。

\begin{figure}[ht]
\centering
\includegraphics[width=14cm]{./figures/impro_2.pdf}
\caption{$f(x)=\frac{1}{x^2} \quad x>1$的图像} \label{impro_fig2}
\end{figure}

\begin{table}[ht]
\centering
\caption{积分结果}\label{impro_tab1}
\begin{tabular}{|c|c|}
\hline
$a$的值 & $\int^a_1 \frac{1}{x^2} \dd x$的值 \\
\hline
$a=10$ & $0.9000$ \\
\hline
$a=100$ & $0.9900$ \\
\hline
$a=1000$ & $0.9990$ \\
\hline
$a=10000$& $0.9999$\\
\hline
\end{tabular}
\end{table}
看起来,随着$a$的增大,$\int^a_1 \frac{1}{x^2} \dd x$似乎并没有趋向无穷,而是趋向某一个确定的值$1$。这也就启发我们,即使积分区间趋向无穷大,积分结果仍然可能是有意义的。这就要求我们扩展定积分\upref{DefInt}的定义,\textsl{玩一种很新的积分}。

\subsubsection{定义}
\begin{definition}{反常积分}
设函数 $f(x)$ 在 $[a, +\infty)$ 上连续(或分段连续),对于任意 $t>a$,积分 $\displaystyle \int^t_af(x)\mathrm{d} x$ 存在,则定义 $\displaystyle \int ^{+\infty}_a f(x)\mathrm{d} x=\lim_{t\rightarrow+\infty }\int _a^{+\infty}f(x)\mathrm{d} x$,并称 $\displaystyle \int ^{+\infty}_a f(x)\mathrm{d} x $ 为 $f(x)$ 在 $[a, +\infty)$ 的反常积分。

如果 $\displaystyle \lim_{t\rightarrow+\infty }\int _a^{+\infty}f(x)\mathrm{d} x$ 存在,则称反常积分 $\displaystyle \int ^{+\infty}_a f(x)\mathrm{d} x$ \textbf{收敛},且该极限值称为反常积分的值;若此极限不存在,则称反常积分 $\displaystyle \lim_{t\rightarrow+\infty }\int _a^{+\infty}f(x)\mathrm{d} x$ \textbf{发散}。
\end{definition}

同样地,可以定义在 $(-\infty,b]$ 上的连续函数 $f(x)$ 的反常积分为 $\displaystyle \int ^b _{-\infty}f(x)\mathrm{d} x=\lim_{t\rightarrow-\infty }\int ^b _{-\infty}f(x)\mathrm{d} x$

对于定义在 $(-\infty,+\infty )$ 上的连续函数 $f(x)$ 的反常积分 $\displaystyle \int ^{+\infty}_{-\infty}f(x)\mathrm{d} x$ 作如下定义
$$\displaystyle \int ^{+\infty}_{-\infty}f(x)\mathrm{d} x=\displaystyle \int ^{+\infty}_c f(x)\mathrm{d} x+\displaystyle \int ^c _{-\infty}f(x)\mathrm{d} x$$
其中 $c$ 是任意实数。当且仅当等式右边的两个积分同时收敛时,称反常积分 $\displaystyle \int ^{+\infty}_{-\infty}f(x)\mathrm{d} x$ 收敛,且右端两个积分值的和称为反常积分 $\displaystyle \int ^{+\infty}_{-\infty}f(x)\mathrm{d} x$ 的值;否则称 $\displaystyle \int ^{+\infty}_{-\infty}f(x)\mathrm{d} x$ 发散。

\subsubsection{计算}
为了计算反常积分,我们仅需要简单地推广牛顿—莱布尼兹公式\upref{NLeib}。
\begin{equation}
\displaystyle \int ^{+\infty}_a f(x)\mathrm{d} x=\lim_{t\rightarrow+\infty }\int _a^{t}f(x)\mathrm{d} x
=\lim_{t\rightarrow+\infty }F(t) - F(a)
\end{equation}
其中$F(x)$是$f(x)$的原函数。

换句话说,如果$\lim_{t\rightarrow+\infty } F(t)$存在,那么原反常积分就收敛(趋近于一个确定的值)。

\begin{example}{计算$\int^{+\infty}_1 \frac{1}{x^2} \dd x$}
使用推广的牛顿—莱布尼兹公式计算上文中的$\int^{+\infty}_1 \frac{1}{x^2} \dd x$

很显然,$f(x)=\frac{1}{x^2}$的原函数是$F(x)=-\frac{1}{x}$,那么
$$\int^{+\infty}_1 \frac{1}{x^2} \dd x=\lim_{t\rightarrow+\infty } (-\frac{1}{t}) - (-\frac{1}{1})=1$$

因此,随着$a$增大、积分结果趋向于$1$并不是巧合。
\end{example}

\begin{example}{计算$\int^{+\infty}_1 \frac{1}{x} \dd x$}
\begin{figure}[ht]
\centering
\includegraphics[width=14cm]{./figures/impro_3.pdf}
\caption{$f(x) = \frac{1}{x}$(红)与$f(x) = \frac{1}{x^2}$(蓝)的图像} \label{impro_fig3}
\end{figure}

$$\int^{+\infty}_1 \frac{1}{x} \dd x=\lim_{t\rightarrow+\infty } {\ln(t)} - \ln(1)$$
很遗憾,$\lim_{t\rightarrow+\infty } {\ln(t)} $是发散的,因此这个反常积分也是发散的。

有人论证,这个结果似乎意味着\textsl{二次元}中引力势能是无穷大的,因此可怜的\textsl{二次元}宇航员永远没有足够的能量来摆脱母星的引力束缚。在三次元中,正如我们上文所见,引力势能是有限的。

\end{example}

\begin{exercise}{}
计算$\int^{+\infty}_1 \frac{1}{x^{0.9}} \dd x$与$\int^{+\infty}_1 \frac{1}{x^{1.1}} \dd x$。你发现了什么?
\begin{figure}[ht]
\centering
\includegraphics[width=14cm]{./figures/impro_4.pdf}
\caption{三者的函数图像} \label{impro_fig4}
\end{figure}
\end{exercise}

\subsection{瑕积分:被积函数趋向无穷的积分}
\begin{figure}[ht]
\centering
\includegraphics[width=8cm]{./figures/impro_1.pdf}
\caption{$f(x)=\frac{1}{x^{\frac{4}{5}}}$的图像} \label{impro_fig1}
\end{figure}
接下来,我们来处理一类被积函数趋近于无穷的问题。例如,积分$\int_0^1 \frac{1}{x^{\frac{4}{5}}} \dd x$。

注意到当$x=0$时,$\frac{1}{x^{\frac{4}{5}}}$发散,我们开始怀疑这个积分是否有意义。我们试着在$t\to0$时积分$\int_t^1 \frac{1}{x^{\frac{4}{5}}} \dd x$

\begin{table}[ht]
\centering
\caption{积分结果}\label{impro_tab2}
\begin{tabular}{|c|c|}
\hline
$t$的值 & $\int_t^1 \frac{1}{x^{\frac{4}{5}}} \dd x$的值 \\
\hline
$t=10^{-8}$ & $4.8744$ \\
\hline
$t=10^{-9}$  & $4.9208$ \\
\hline
$t=10^{-10}$  & $4.9500$ \\
\hline
\end{tabular}
\end{table}
与之前的例子类似,尽管$t\to0$时被积函数$\frac{1}{x^{\frac{4}{5}}}$发散,但$\int_t^1 \frac{1}{x^{\frac{4}{5}}}\dd x$正趋向一个有限的值。这种情况再次提醒我们需要继续扩展定积分的定义,即允许一个发散的函数具有有限的定积分值。于是,我们引入了瑕积分的概念。

%补充具体定义

计算瑕积分时,我们同样需要推广牛顿—莱布尼兹公式。
\begin{equation}
\displaystyle \int ^{b}_a f(x)\mathrm{d} x=\lim_{t\rightarrow a+}\int _t^{b}f(x)\mathrm{d} x
= F(b) - \lim_{t\rightarrow a+ }F(t)
\end{equation}
其中$F(x)$是$f(x)$的原函数,在$x=a$处$f(x)$发散。

\begin{example}{计算$\int_t^1 \frac{1}{x^{\frac{4}{5}}}\dd x$}
$\int_t^1 \frac{1}{x^{\frac{4}{5}}}\dd x = 5\times(1)^{\frac{1}{5}} - \lim_{t\rightarrow 0+ }5t^{\frac{1}{5}}=5$
\end{example}
