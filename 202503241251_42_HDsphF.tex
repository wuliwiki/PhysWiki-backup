% 高维球谐函数
% keys 球谐函数|高维
% license Usr
% type Tutor

\pentry{球谐函数\nref{nod_SphHar},高维弯曲空间中的拉普拉斯算符\nref{nod_GLapOp},高维空间球坐标及其度规\nref{nod_nDSM}}{nod_94ea}

在三维的情形,球谐函数是在球坐标下,求解拉普拉斯方程时,通过分离变量得到的。其是拉普拉斯方程角度部分的解。即下面的微分方程
\begin{equation}
\Nabla_\Omega^2 Y+\lambda Y=0.~
\end{equation}
其中 $\Omega$ 代表只含角 $(\theta,\phi)$ 的部分,$\lambda$ 是常数,且
\begin{equation}
\Nabla^2_\Omega=\pdv[2]{}{\theta}+\cot\theta\pdv{}{\theta}+\frac{1}{\sin^2\theta}\pdv[2]{}{\varphi}.~
\end{equation}
本词条将仿照三维情形推导球谐函数的方法,推导高维弯曲空间中的球谐函数。

\subsection{球坐标系下的高维拉普拉斯方程}
对一般高维空间的拉普拉斯方程,可由对应空间的\enref{拉普拉斯算符}{GLapOp}获得,即从 $\Delta u=0$ 一般高维空间的拉普拉斯方程如下:
\begin{equation}\label{eq_HDsphF_2}
\begin{aligned}
&\frac{1}{\sqrt{ \left\lvert g \right\rvert }} \frac{\partial }{\partial x^i} \left(\sqrt{ \left\lvert g \right\rvert }g^{ij} \frac{\partial u}{\partial x^j} \right)\\
=&\frac{1}{\sqrt{ \left\lvert g \right\rvert }} \qty(\frac{\partial }{\partial x^i} \sqrt{ \left\lvert g \right\rvert })g^{ij} \frac{\partial u}{\partial x^j}+\qty(\frac{\partial }{\partial x^i} g^{ij}) \frac{\partial u}{\partial x^j}+g^{ij}\frac{\partial^2 u}{\partial x^i\partial x^j}\\
=&0 .
\end{aligned}~
\end{equation}

而
\begin{equation}
\begin{aligned}
&\frac{1}{\sqrt{\abs{ g}}}\frac{\partial }{\partial x^i} \sqrt{ \left\lvert g \right\rvert }=\frac{1}{2g}\pdv{g}{g_{jk}}\pdv{g_{jk}}{x^i}\\
&=\frac{1}{2g}(gg^{jk})\pdv{g_{jk}}{x^i}\\
&=\frac{1}{2}g^{jk}\pdv{g_{jk}}{x^i}.
\end{aligned}~
\end{equation}
$N+1$ 维空间球坐标系 $x^i=\theta^i,i=1,\cdots,N,x^{N+1}=r$ 下的\enref{度规}{nDSM}为
\begin{equation}
\begin{aligned}
g_{ij}&=\mathrm{diag}(\eta_i),\\
\eta_{i}&=\left\{\begin{aligned}
& r^2\prod\limits_{k=i+1}^{N}\sin^2\theta^{k},i\leq N,\\
&1,i=N+1.
\end{aligned}\right.
\end{aligned}~
\end{equation}
由此,
\begin{equation}\label{eq_HDsphF_1}
\begin{aligned}
g^{ij}&=\mathrm{diag}(h^i),\\
h^{i}&=\left\{\begin{aligned}
& r^{-2}\prod\limits_{k=i+1}^{N}\sin^{-2}\theta^{k},i\leq N,\\
&1,i=N+1.
\end{aligned}\right.
\end{aligned}~
\end{equation}
所以
\begin{equation}
\begin{aligned}
&\frac{1}{2}g^{jk}\pdv{g_{jk}}{x^{N+1}}=\frac{1}{2}\sum_{j=1}^{N+1}h^{j}\pdv{\eta_{j}}{x^i}\\
=&\frac{1}{2}\qty(\sum_{j=1}^{N}r^{-2}\prod\limits_{k=j+1}^{N}\sin^{-2}\theta^{k}2r\prod\limits_{k=j+1}^{N}\sin^2\theta^{k})\\
=&Nr^{-1}\\
&\frac{1}{2}g^{jk}\pdv{g_{jk}}{x^i}=\frac{1}{2}\sum_{j=1}^{N+1}h^{j}\pdv{\eta_{j}}{x^i}\\
=&\frac{1}{2}\qty(\sum_{j=1}^{i-1}r^{-2}\prod\limits_{k=j+1}^{N}\sin^{-2}\theta^{k}2r^2\sin\theta^{i}\cos\theta^{i}\prod\limits_{k=j+1,k\neq i}^{N}\sin^2\theta^{k})\\
=&(i-1)\cot\theta^i,\quad i\leq N.
\end{aligned}~
\end{equation}
因此
\begin{equation}
\begin{aligned}
&\frac{1}{\sqrt{ \left\lvert g \right\rvert }} \qty(\frac{\partial }{\partial x^i} \sqrt{ \left\lvert g \right\rvert })g^{ij} \frac{\partial u}{\partial x^j}\\
=&Nr^{-1}\pdv{u}{r}+\sum_{i=1}^{N}h^{i}(i-1)\cot\theta^i\pdv{u}{\theta^i}\\
\end{aligned}~
\end{equation}
而由\autoref{eq_HDsphF_1} , $\pdv{g^{ij}}{\theta^i}=0$。且
\begin{equation}
g^{ij}\frac{\partial^2 u}{\partial x^i\partial x^j}=\sum_{i=1}^{N}h^{i}\frac{\partial^2 u}{\partial {\theta^i}^2}+\frac{\partial^2 u}{\partial r^2}~
\end{equation}
所以最后\autoref{eq_HDsphF_2} 变成
\begin{equation}
\begin{aligned}
\sum_{i=1}^{N}h^{i}\qty[(i-1)\cot\theta^i\pdv{u}{\theta^i}+\frac{\partial^2 u}{\partial {\theta^i}^2}]+r^{-N}\frac{\partial }{\partial r}\qty(r^{N}\frac{\partial u}{\partial r})=0.
\end{aligned}~
\end{equation}
若定义 
\begin{equation}\label{eq_HDsphF_6}
h^i_{(N)}:=r^2h^i,~
\end{equation}
则 $h^i_{(N)}$ 将仅仅依赖于角部变量。类似球坐标下\enref{三维拉普拉斯方程}{SphLap}的写法,我们有
\begin{equation}\label{eq_HDsphF_3}
r^{-2}\Nabla^{2}_{S^N}u+r^{-N}\frac{\partial }{\partial r}\qty(r^{N}\frac{\partial u}{\partial r})=0.~
\end{equation}
其中
\begin{equation}\label{eq_HDsphF_5}
\Nabla^{2}_{S^N}u:=\sum_{i=1}^N h^{i}_{(N)}\qty[(i-1)\cot\theta^i\pdv{u}{\theta^i}+\pdv[2]{u}{\theta^i}]~
\end{equation}
定义了单位 $N$ 球上的 \textbf{Laplace-Beltrami 算子}。
\subsection{高维球谐函数}
\autoref{eq_HDsphF_3} 没有出现混合导数,因此,若将 $u$ 写成变量分离的形式
\begin{equation}
u(r,\theta^1,\cdots,\theta^N)=R(r)W_N(\theta^1,\cdots,\theta^N),~
\end{equation}
\autoref{eq_HDsphF_3} 就可写为
\begin{equation}\label{eq_HDsphF_4}
\frac{1}{R}r^{-N+2}\frac{\partial }{\partial r}\qty(r^{N}\frac{\partial u}{\partial r}R)=-\frac{1}{W_N}\Delta_{S^N}W_N=\lambda_N.~
\end{equation}
其中 $\lambda_N$ 是分离常数,$\Delta_{S^N}:=\Nabla_{S^N}^2$。

将\autoref{eq_HDsphF_4} 中只含角度变量的部分提取出来,就得到如下角变量部分满足的方程:
\begin{equation}\label{eq_HDsphF_9}
\Delta_{S^N}W_N+\lambda_N W_N=0.~
\end{equation}
该方程的解 $W_N$ 便是所谓的 $N$ 维\textbf{球谐函数}(因为它是定义在 $N+1$ 维空间的 $N$ 维球上的)。

\subsection{将球谐函数转化为分离变量的形式}
三维空间的球谐函数可以继续分离为只关于 $\theta$ 和 $\varphi$ 的乘积形式。同样的,高维空间的球谐函数也能如此。

为了分离变量,我们先试图将 $\theta^N$ 提取出来。利用\autoref{eq_HDsphF_5} ,有
\begin{equation}\label{eq_HDsphF_8}
\sum_{i=1}^{N-1} h^{i}_{(N)}\qty[(i-1)\cot\theta^i\pdv{W_N}{\theta^i}+\pdv[2]{W_N}{\theta^i}]+h^{N}_{(N)}\qty[(N-1)\cot\theta^N\pdv{W_N}{\theta^N}+\pdv[2]{W_N}{\theta^N}]+\lambda_N W_N=0.~
\end{equation}
然而,由\autoref{eq_HDsphF_1} 和\autoref{eq_HDsphF_6} ,$h^i_{(N)}$ 是关于 $\theta^j,i+1\leq j\leq N$ 的函数,为使得上式左边第一式不含 $\theta^j$,只要将 $h^i_{(N)}$ 乘以 $\sin^2\theta^j$ 即可。因此,若定义
\begin{equation}\label{eq_HDsphF_7}
h^{i}_{(k-1)}=\sin^2\theta^k h^i_{(k)},\quad 1\leq i\leq k-1,2\leq k\leq N,~
\end{equation}
则 $h^i_{(N-1)}$ 只依赖与 $\theta ^{i+1},\cdots,\theta^{N-1}$。利用\autoref{eq_HDsphF_7} 的递推关系,可知,对于给定的 $k$, $\{h^i_{k}\}$ 仅仅依赖于变量 $(\theta^1,\cdots,\theta^k)$ 且 $h^k_{(k)}=1$,且 $h^k_{(k)}=1$。

现在,我们可以将\autoref{eq_HDsphF_8} 写为
\begin{equation}
\sum_{i=1}^{N-1} \sin^{-2}\theta^N h^{i}_{(N-1)}\qty[(i-1)\cot\theta^i\pdv{W_N}{\theta^i}+\pdv[2]{W_N}{\theta^i}]+h^{N}_{(N)}\qty[(N-1)\cot\theta^N\pdv{W_N}{\theta^N}+\pdv[2]{W_N}{\theta^N}]=-\lambda_N W_N.~
\end{equation}
利用\autoref{eq_HDsphF_5} ,上式可写为
\begin{equation}
\sin^{-2}\theta^N \Delta_{S^{N-1}}{W_N}+\qty[(N-1)\cot\theta^N\pdv{W_N}{\theta^N}+\pdv[2]{W_N}{\theta^N}]=-\lambda_N W_N.~
\end{equation}
显然,现在可以继续进行变量分离,令
\begin{equation}\label{eq_HDsphF_10}
W_N(\theta^1,\cdots,\theta^N)=W_{N-1}(\theta^1,\cdots,\theta^{N-1})y_N(\theta^N),~
\end{equation}
那么
\begin{equation}
-\frac{1}{W_{N-1}}\Delta_{S^{N-1}}W_{N-1}=\frac{\sin^2\theta^N}{y_N}\qty((N-1)\cot\theta^N\pdv{y_N}{\theta^N}+\pdv[2]{y_N}{\theta^N})+\lambda_N\sin^2\theta^N=\lambda_{N-1}.~
\end{equation}
其中 $\lambda_{N-1}$ 是一个新分离常数。因此,我们有下面方程
\begin{equation}
\begin{aligned}
&\Delta_{S^{N-1}}W_{N-1}+\lambda_{N-1}W_{N-1}=0,\\
&\qty((N-1)\cot\theta^N\pdv{y_N}{\theta^N}+\pdv[2]{y_N}{\theta^N})+\qty(\lambda_N-\lambda_{N-1}\sin^{-2}\theta^N)y_N=0.
\end{aligned}~
\end{equation}
注意第一个方程和回到了\autoref{eq_HDsphF_9} 的形式,因此可以继续重复该过程。最终可以获得关于 $W_k=W_k(\theta_1,\cdots,\theta_k)$ 的方程
\begin{equation}
\Delta_{S^{k}}W_{k}+\lambda_{k}W_{k}=0,\quad 1\leq k\leq N.~
\end{equation}
和关于 $y_k=y_k(\theta_k)$ 的方程:
\begin{equation}
\pdv[2]{y_k}{\theta^k}+(k-1)\cot\theta^k\pdv{y_k}{\theta^k}+\qty(\lambda_k-\frac{\lambda_{k-1}}{\sin^2\theta_k})y_k=0,2\leq k\leq N~
\end{equation}
并且有一组 $N$ 个变量的分离常数 $\{\lambda_k\}$。注意在 $k=2$ 时,由\autoref{eq_HDsphF_10} ,我们有
\begin{equation}
W_2(\theta^1,\theta^2)=W_1(\theta^1)y_N(\theta^N)
\end{equation}

\begin{equation}
\Delta_{S^{1}}W_{1}+\lambda_{1}W_{k}=0
\end{equation}









