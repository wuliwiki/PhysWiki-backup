% 子空间的正交关系
% keys 正交|子空间|内积|向量空间|高等代数
% license Xiao
% type Tutor

\begin{issues}
\issueTODO
\end{issues}

% Giacomo:本文仅仅对有限维度成立,需要修改

\pentry{内积\nref{nod_InerPd}, 直和\nref{nod_DirSum}, 基底(线性代数)\nref{nod_VecSpn}}{nod_aaeb}

\begin{definition}{正交}
一个内积空间 $V$ 中,如果两个向量 ${v_1}$ 和 ${v_2}$ 满足内积 $\ev{v_1, v_2} = 0$,我们就称它们相互\textbf{正交};

如果两个子空间 $V_1$ 和 $V_2$ 任意各选一个向量 ${v_1}$ 和 ${v_2}$ 都相互正交,那么我们就说者两个子空间相互\textbf{正交}。
\end{definition}

构造正交关系的一种简单的方法是, 在 $V$ 中找到两组向量 $x_1, \dots, x_m$ 和 $y_1, \dots, y_m$, 确保对任意 $x_i$ 和 $y_j$ 正交, 那么 $x_1, \dots, x_m$ 张成\upref{VecSpn}的子空间必定和 $y_1, \dots, y_m$ 张成的子空间正交。

从的角度来看, 两个空间正交的充分必要条件是:
\begin{theorem}{}
两个子空间 $V_1$ 和 $V_2$ 各选一组基底 $\{v_\alpha\}_\alpha$ 和 $\{w_\beta\}_\beta$,它们相互正交当且仅当对任意 $\alpha, \beta$ 都有 $\ev{v_\alpha, w_\beta} = 0$。
\end{theorem}

\begin{example}{}
三维几何向量空间中, 建立直角坐标系, 那么 $\uvec x$ 和 $\uvec y$ 张成的二维向量空间(平面)与 $\uvec z$ 张成的一维向量空间(直线)正交。

虽然 $xy$ 平面和 $xz$ 平面是两个垂直的平面, 但它们并不相互正交。 例如向量 $\uvec x$ 是两个平面共同的向量, 但 $\uvec x$ 和它本身不正交。
\end{example}

\subsection{相互正交的子空间的直和}

\begin{exercise}{}
证明两个相互正交的子空间中, 只有零向量是共同向量;换言之,两个子空间是正交的,意味着它们两个是线性无关的(参考基底\upref{VecSpn})。
\end{exercise}

若在两个相互正交的子空间 $V_1$ 和 $V_2$ 分别中取一组基底, 那么将他们合并起来就得到了母空间中的一组基底。特别地, 如果 $V_1$ 和 $V_2$ 各取一组正交归一基底\upref{OrNrB} $\{v_\alpha\}_\alpha$ 和 $\{w_\beta\}_\beta$,那么合并之后 $\{v_\alpha, w_\beta\}_{\alpha, \beta}$ 就是直和空间 $V_1 \oplus V_2$ 中的一组正交归一基底。 但注意直和空间中的任意一组正交归一基底未必可以划分为 $V_1$, $V_2$ 空间中的两组基底。

\subsection{正交补}
我们在 “直和\upref{DirSum}” 中已经定义了补空间的概念, 现在来定义一种特殊的补空间。
\begin{definition}{正交补空间}\label{def_OrthSp_1}
在 $V$ 空间中, 若 $V_1$ 和 $V_2$ 正交且 $V = V_1 \oplus V_2$, 那么 $V_1$ 和 $V_2$ 互为对方的\textbf{正交补空间(Orthogonal complement)}, 简称\textbf{正交补}。
\end{definition}

\begin{theorem}{}\label{the_OrthSp_1}
从基底的角度来看, 两个有限维空间 $V_1, V_2$ 是关于 $V$ 的正交的充分必要条件是: 如果从两空间各选一组基底 ${\alpha_i}$ $(i = 1, \dots, N_1)$ 和 ${\beta_i}$ $(i = 1, \dots, N_2)$, 那么基底 $\{\alpha_1, \dots, \alpha_{N_1}, \beta_1, \dots, \beta_{N_2}\}$ 就是 $X$ 的一组正交归一基底。
\end{theorem}

\addTODO{正交补是唯一的}
\addTODO{如何求正交补?}
