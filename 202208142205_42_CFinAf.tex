% 仿射空间中的曲线坐标系
% keys 曲线坐标系|仿射空间

\begin{issues}
\issueDraft
\end{issues}

\pentry{仿射空间\upref{AfSp}}
\subsection{曲线坐标系}
设 $\{\dot o;e_1,\cdots,e_n\}$ 是仿射空间 $\mathbb A$ 中的坐标系.则对其上任一点 $M$ ,其坐标为矢量 $\overrightarrow{OM}=x^i e_i$ 的坐标(采取坐标和基矢指标的对立约定\upref{CofTen}及爱因斯坦求和约定\upref{EinSum}).若选择一新坐标系 $\{\dot o',e'_1,\cdots,e'_n\}$,则点的坐标变换规则为(\autoref{AfSp_the3}~\upref{AfSp})
\begin{equation}
x'^i={B^i}_jx^j-{B^i}_j b^j
\end{equation}
其中,${B^i}_j$ 是基 $\{e_1,\cdots,e_n\}$ 到基 $\{e'_1,\cdots,e'_n\}$ 的转换矩阵 ${A^i}_j$ 的逆矩阵.而 $b^i$ 是 $\dot o'$ 在旧坐标系 $\{\dot o;e_1,\cdots,e_n\}$  中的坐标.

为描述方便,先引进仿射空间中领域和 $n$ 维区域的概念.
\begin{definition}{领域,$n$ 维区域}
设 $\mathbb A$ 是 $n$ 维仿射空间,点 $M=x^ie_i\in\mathbb A$ 的 $\delta$ \textbf{领域}是指坐标满足
\begin{equation}
\sum_{i}^n(x'^i-x^i)^2<\delta^2
\end{equation}
的所有点 $M'=x'^i e_i$ 构成的集合.

若 $\Omega$ 是 $\mathbb A$ 中这样一个集合:对 $\Omega$ 中任一点 $M$,必有 $M$ 的某个领域也属于 $\Omega$.则称 $\Omega$ 是 $\mathbb A$ 中的\textbf{ $n$ 维区域}.若区域 $\Omega$ 中任一点都可连续的变动到 $\Omega$ 中的另一点,即一点的坐标连续变化到另一点的坐标,则区域 $\Omega$ 称为\textbf{连通的}.
\end{definition}

凡提到区域总是指连通区域.

\begin{definition}{曲线坐标}
设 $\Omega\in\mathbb A$ 是一 $n$ 维连通区域,在其上给定仿射坐标(即对应基底 $\{e_i\}$ 的坐标)的 $n$ 个连续可微的单值函数 $f_k(x^1,\cdots,x^n)$ ($k=1,\cdots,n$),且 $\{f_i\}$ 是可逆的.则新定义的变量
\begin{equation}
x'^i=f_i(x^1,\cdots,x^n),\quad i=1,\cdots,n
\end{equation}
称为 $\Omega$ 上的\textbf{曲线坐标}.
\end{definition}

\textbf{注:}函数组 $\{f_i\}$ 的可逆性意味着
\begin{equation}
x^i=g_i(x'^1,\cdots,x'^n),\quad i=1,\cdots,n
\end{equation}
且 $g_i$ 也是连续可微的单值函数,且函数组 $\{g_i\}$ 也是可逆的.