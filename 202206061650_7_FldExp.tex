% 域的扩张
% 域|单扩张|超越扩域|最小多项式|minimal polynomial|超越元|超越数|代数元|扩张次数|代数闭包|algebraic closure|极小多项式|中间域升链|null polynomial|零化多项式

\pentry{分式域\upref{FrcFld},多项式环\upref{RPlynm}}
%%未完成

\subsection{域的单扩张}
如果在一个域中添加不属于域集合的元素,我们可以得到一个更大的集合.要让这个新集合成为域,我们就得定义新元素和原来域中元素相加和相乘的结果;无论怎么定义,这个结果必须满足域的公理.如果在集合中任何元素都无法成为某个运算结果,那么我们就必须再引入新的元素来作为这个结果.以此类推,不停地添加新元素,直到最后不需要添加新元素了,那最后这个集合就是一个新的域,它包含了原来的域.这个域是原来的域的\textbf{扩张},并且是包含最初那个新元素的最小的域,因此被称为\textbf{单元素扩张},简称\textbf{单扩张}.

域的单扩张具体是怎么进行的呢?我们将从例子开始说明,最后引入域的单扩张的定义.

\begin{exercise}{有理数域的 $\sqrt{2}$ 扩张}\label{FldExp_exe1}
给定有理数域 $\mathbb{Q}$,则 $\sqrt{2}$ 并不是 $\mathbb{Q}$ 的元素.将 $\sqrt{2}$ 添加进去,那么由于 $\sqrt{2}$ 是实数域 $\mathbb{R}$ 的元素,这提示我们可以把新的运算结果按照 $\mathbb{R}$ 中的运算来定义.这样,$\mathbb{Q}$ 中添加 $\sqrt{2}$ 的单扩域的集合就是 $\{a+b\sqrt{2}|a, b\in\mathbb{Q}\}$.

请验证,任意 $a+b\sqrt{2}$ 都必须在扩域中;而这个集合也满足了加法和乘法的封闭性与逆元存在性,从而构成了一个域.于是,这个集合是包含 $\sqrt{2}$ 和全体有理数的最小的域.
\end{exercise}

\begin{exercise}{超越扩域}\label{FldExp_exe2}
依然给定有理数域 $\mathbb{Q}$,这次添加的元素是 $\pi$.那么扩域应当是
\begin{equation}
\{\frac{\sum_{i=0}^N a_i\pi^i}{\sum_{j=0}^M b_j\pi^j}|a_i, b_j\in\mathbb{Q}, b_j\text{不全为零,} N, M\in\mathbb{Z}^+\}
\end{equation}
也就是 $\pi$ 的所有有理系数多项式的分式构成的集合,其中分母不是 $0$.

验证这是一个域,并且是包含全体有理数和 $\pi$ 的最小域.注意 $a_i$ 全为 $0$ 的情况意味着什么.

这是 $\mathbb{Q}$ 中添加 $\pi$ 后的单扩张.
\end{exercise}



借助\textbf{分式域}\upref{FrcFld}和\textbf{多项式环}\upref{RPlynm}的概念,我们可以简练地定义域的单扩张如下:


\begin{definition}{域的单扩张}\label{FldExp_def1}
给定域 $\mathbb{F}$ 和一个元素 $x_0$,其中 $x_0$ 在$\mathbb{F}$上的最小多项式是$f(x)$.定义 $\mathbb{F}$ 添加 $x_0$ 的\textbf{单扩张}是多项式环$\mathbb{F}[x_0]=\mathbb{F}[x]/<f(x_0)>$的分式域,记为$\mathbb{F}(x_0)$
\end{definition}



在\autoref{FldExp_def1} 中,如果 $x_0$ 在 $\mathbb{F}$ 中没有最小多项式,那么 $\mathbb{F}(x_0)$ 中的各元素彼此不同,此时我们称 $x_0$ 是 $\mathbb{F}$ 的\textbf{超越元(transcendental element)};



观察\autoref{FldExp_exe1} 和\autoref{FldExp_exe2} 中两个域扩张的特点,我们容易发现,由于加法和乘法的封闭性,添加某个元素后的单扩张,必须包含这个元素的所有多项式,系数取自域中;由于乘法逆元存在性,这些多项式的倒数也得包含进去.因此,域的单扩张可以简单理解为把新元素的多项式都添加进去的过程.

但是两个场景是不太一样的.$\mathbb{Q}$ 中添加 $\pi$ 的过程中,包含了所有 $\pi$ 的有理系数多项式,比如$3.5\pi^4+2\pi^2+0.1\pi-1/3$,但是添加 $\sqrt{2}$ 的时候却只需要最多 $1$ 阶的多项式,像$3.5(\sqrt{2})^3+1$就可以改写为$7\sqrt{2}+1$.这是因为 $\sqrt{2}$ 的 $2$ 次方是一个有理数,落入了 $\mathbb{Q}$ 中,因此 $\sqrt{2}$ 的 $2$ 阶及以上的有理系数多项式都可以表示为最多 $1$ 阶的多项式.

换一种更方便的表述,那就是 $x^2-2$ 这个多项式是阶数最小的能将 $\sqrt{2}$ 映射为 $0$ 的多项式.这就引出\textbf{最小多项式}的概念.

\begin{definition}{最小多项式}
给定一个环 $R$ 和任意元素 $x_0$,其中 $x_0$\textbf{不一定}是 $R$ 中的元素.记
$$R[x_0]=\{\sum_{i=0}^N r_ix_0^i|r_i\in R, N\in\mathbb{Z}^+\}$$称 $R[x_0]$ 是 $x_0$ 在 $R$ 中的\textbf{多项式环}.在 $R[x_0]$ 中,如果存在一个多项式 $f(x_0)=0$,那么称 $f(x_0)$ 是 $x_0$ 在 $R$ 上的\textbf{零化多项式(null polynomial)};阶数最低的零化多项式,称为$x_0$的\textbf{最小多项式(minimal polynomial)},或\textbf{极小多项式}.

如果不存在多项式$f$使得$f(x_0)=0$,那么称$x_0$ 在 $R$ 中\textbf{不具有最小多项式}.
\end{definition}

在\autoref{FldExp_exe1} 和\autoref{FldExp_exe2} 中,$\pi$ 和 $\sqrt{2}$ 的本质区别就是:在 $\mathbb{Q}$ 中,$\pi$ 没有最小多项式,但是 $\sqrt{2}$ 具有最小多项式 $x^2-2$.这决定了同一个域 $\mathbb{Q}$ 分别添加 $\pi$ 和 $\sqrt{2}$ 后的不同单扩张.

\begin{definition}{代数元素与超越元素}\label{FldExp_def2}
给定环$R$及其子环$S$,如果元素$a\in R$在$S$上有最小多项式,则称之为$S$的\textbf{代数元素(algebraic element)},否则称之为\textbf{超越元素(transcendental element)}.

特别地,整数环$\mathbb{Z}$上的代数元素,称为\textbf{代数数(algebraic number)},而不是其代数元素的实数则称为\textbf{超越数(transcendental number)}.
\end{definition}

\autoref{FldExp_def2} 中要给定$R$来讨论其子环$S$,是为了保证$a$已经定义好了与$S$中各元素的运算.如果有别的办法定义,那自然不需要那么麻烦.比如我们在\autoref{FldExp_exe1} 中讨论的$\mathbb{Q}(\sqrt{2})$,虽然$\sqrt{2}$不在$\mathbb{Q}$中,但我们已经定义好了它与各有理数的运算关系.这个运算关系是怎么确定的呢?那就是用其在$\mathbb{Q}$上的\textbf{最小多项式},$x^2-2$.




% 可是,我们把域抽象地看成满足一定条件的集合时,和所有集合一样,域中的元素叫什么名字并不重要.对于\autoref{FldExp_exe1} 和\autoref{FldExp_exe2} 中的情况,我们其实是借用了 $\mathbb{R}$ 中已有的元素名称来引出两个不同的单扩域的.如果不进行这样的类比引出,而只是单纯地说在 $\mathbb{Q}$ 中加入某个新的元素 $x_0$ 来进行单扩张,那么结果可能是 $\{a+bx_0|a, b\in\mathbb{Q}\}$,可能是 $\{\sum_{i=0}^N a_ix_0^i|a_i\in\mathbb{Q}, N\in\mathbb{Z}^+\}$,可能是 $\{a+bx_0+cx_0^2|a, b, c\in\mathbb{Q}\}$,甚至还可能形式上也是 $\{a+bx_0|a, b\in\mathbb{Q}\}$,只不过 $x_0^2=2/3$ 了.这时该怎么描述域的单扩张呢?



因此,关于$\sqrt{2}$的一切性质,其最小多项式$x^2-2$已经阐述清楚了.我们可以转而用最小多项式来描述扩域的过程,这和写出$\sqrt{2}$是等价的,只是用起来更方便,更广泛.比如说,要给$\mathbb{Q}$添加一个元素$a\not\in \mathbb{Q}$来进行扩域,而$a$在$\mathbb{Q}$上的最小多项式是$x^6+x^5-2x^3+7.5x^2-x+9$,那研究这个多项式自然比研究$a$到底是哪个实数要方便得多.


% \begin{definition}{域的单扩张}
% 给定域 $\mathbb{F}$ 和一个元素 $x_0$,其中 $x_0$ 不一定在 $\mathbb{F}$ 中.定义 $\mathbb{F}$ 添加 $x_0$ 的\textbf{单扩张}是集合
% \begin{equation}
% \{\frac{\sum_{i=0}^N a_ix_0^i}{\sum_{j=0}^M b_jx_0^j}|a_i, b_j\in\mathbb{F},\quad\sum_{j=0}^M b_jx_0^j\not=0,\quad N, M\in\mathbb{Z}^+\}
% \end{equation}
% 记为 $\mathbb{F}(x_0)$;容易验证,这是一个域.
% \end{definition}




 

% 如果 $x_0$ 在 $\mathbb{F}$ 中有最小多项式 $f(x)$,那么称 $x_0$ 是 $\mathbb{F}$ 的\textbf{代数元(algebraic element)},此时在 $\mathbb{F}(x_0)$ 中,若记 $g, h$ 为任意多项式,就有 $g(x_0)$ 和 $g(x_0)+f(x_0)h(x_0)$ 的值相等.这提示我们,在这种情形下,似乎可以把 $\mathbb{F}(x_0)$ 看成是某个\textbf{超越元} $x$ 的单扩张 $\mathbb{F}(x)$ 集合中,把 $g(x)$ 和 $g(x)+f(x)h(x)$ 认为是等价的,所得到的\textbf{商集}.实际上,严格来说不是商集,而是商集中再把类似$$\frac{\sum_{i=0}^N a_ix_0^i}{f(x)h(x)}$$的元素剔除以后剩下的集合.当然了,$0$ 不要剔除.

\begin{theorem}{}
给定域 $\mathbb{F}$ ,取其超越元素 $x_1$和$x_2$,则$\mathbb{F}(x_1)$与$\mathbb{F}(x_2)$同构.
\end{theorem}

证明思路是显然的,从$\mathbb{F}(x_1)$到$\mathbb{F}(x_2)$的同构映射只需要将所有出现$x_1$的地方都替换成$x_2$即可,比如$\frac{1}{x_1+x_1^2}$映射到$\frac{1}{x_2+x_2^2}$.

容易想到,$x_0$的最小多项式必然是给定域上的不可约多项式.因为一旦可约,那其因子的次数都小于它,但总得有一个因子是$x_0$的零化多项式,这就违背了“最小”的定义.




\subsection{域的扩张次数}

以上,我们讨论了如何添加一个元素,并根据元素的最小多项式来得到域的单扩张.对一个域进行单扩张以后,再进行单扩张,如此反复,我们可以获得域的多次扩张.更一般地,我们有关于域扩张的如下定义:

\begin{definition}{域的扩张}
设$\mathbb{K}$是一个域,$\mathbb{F}\subseteq \mathbb{K}$且在$\mathbb{K}$的运算下也构成域,那么称$\mathbb{F}$是$\mathbb{K}$的\textbf{子域},而$\mathbb{K}$是$\mathbb{F}$的\textbf{扩域(extension field)}或者\textbf{扩张(field extension)}.

将上述域的扩张记为$\mathbb{K}/\mathbb{F}$.
\end{definition}

考虑域$\mathbb{F}$及其扩域$\mathbb{K}$.$\mathbb{K}$中的元素\textbf{彼此之间}可以\textbf{相加},也可以\textbf{乘以}$\mathbb{F}$\textbf{中的元素},这两种运算分别对应线性空间中的向量加法与数乘.也就是说,$\mathbb{K}$可以视为$\mathbb{F}$上的一个\textbf{线性空间}.

\begin{definition}{域的扩张次数}
给定域的扩张$\mathbb{K}/\mathbb{F}$.$\mathbb{K}$作为$\mathbb{F}$上的线性空间的维数,称为$\mathbb{K}$对$\mathbb{F}$的\textbf{扩张次数},记为$[\mathbb{K}:\mathbb{F}]$.

$[\mathbb{K}:\mathbb{F}]<\infty$时,称之为\textbf{有限扩张},否则为\textbf{无限扩张}.
\end{definition}

$[\mathbb{Q}(\sqrt{2}):\mathbb{Q}]=2$,因此\autoref{FldExp_exe1} 中的扩域是一个二次扩张.你可以把$1$和$\sqrt{2}$分别想象为两个基向量,而$\mathbb{Q}(\sqrt{2})$就是这两个基向量在域$\mathbb{Q}$上张成的线性空间.

相对应地,$[\mathbb{Q}(2^{1/3}):\mathbb{Q}]$是一个三次扩张,其基向量是$1$、$2^{1/3}$和$2^{2/3}$.类似地,我们可以引申出以下定理:




\begin{theorem}{}\label{FldExp_the1}
设域$\mathbb{F}$有一个\textbf{单扩张}$\mathbb{F}(a)$,其中$a$在$\mathbb{F}$中的最小多项式为$f$,则$[\mathbb{F}(a):\mathbb{F}]=\opn{deg}f$.
\end{theorem}




\begin{definition}{代数扩张与超越扩张}\label{FldExp_def5}\label{FldExp_def4}\label{FldExp_def3}
给定域$\mathbb{F}$及其扩域$\mathbb{K}$.如果$\mathbb{K}$中元素都是$\mathbb{F}$的代数元素,则称$[\mathbb{K}:\mathbb{F}]$是一个\textbf{代数扩张(algebraic extension)},否则是一个\textbf{超越扩张(transcendental extension)}.
\end{definition}

下面这个定理给出的是\textbf{单}代数扩张的等价定义:

\begin{theorem}{}\label{FldExp_the2}
给定域$\mathbb{F}$及其单扩张$\mathbb{F}(a)$,则以下条件等价:

1. $a$是$\mathbb{F}$上的代数元;

2. $[\mathbb{F}(a):\mathbb{F}]<\infty$;

3. $[\mathbb{F}(a):\mathbb{F}]$是代数扩张.

\end{theorem}


\textbf{证明}:

1 $\to$ 2 :由\autoref{FldExp_the1} 直接可得.

2 $\to$ 3 :任取$x\in\mathbb{F}(a)$,由于$[\mathbb{F}(a):\mathbb{F}]=n<\infty$,故向量组$\{1, x, x^2, \cdots, x^n\}$必线性相关,因此存在一个$\mathbb{F}$上的非零多项式$f$使得$f(1, x, x^2, \cdots, x^n)=0$.因此任意$x$必是$\mathbb{F}$上的代数元.

3 $\to$ 1 :由\autoref{FldExp_def3} 直接可得.

\textbf{证毕}.

\begin{corollary}{}\label{FldExp_cor1}
如果$\mathbb{K}/\mathbb{F}$是有限扩张,则必是代数扩张.
\end{corollary}

证明思路类比\autoref{FldExp_the2} 中的 2 $\to$ 3.

要注意的是,\autoref{FldExp_cor1} 只说了是有限扩张,而不是单有限扩张,因此无法像\autoref{FldExp_the2} 那样反过来说“代数扩张必是有限扩张”.比如说,$\mathbb{Q}$添加所有整数的平方根以后得到的显然是代数扩张,但同时也是无限扩张.





\begin{theorem}{}\label{FldExp_the3}
考虑扩域$[\mathbb{K}:\mathbb{F}]$和$[\mathbb{L}:\mathbb{K}]$,则有

\begin{equation}
[\mathbb{L}:\mathbb{F}]=[\mathbb{K}:\mathbb{F}][\mathbb{L}:\mathbb{K}]
\end{equation}

\end{theorem}

\textbf{证明}:

只需要考虑$[\mathbb{K}:\mathbb{F}]$和$[\mathbb{L}:\mathbb{K}]$都有限的情况,证明该情况之后可自然推出无限的情况.

设$\{\alpha_i\}_{i=1}^{n}$是$\mathbb{K}$在$\mathbb{F}$上的基,$\{\beta_i\}_{i=1}^{m}$是$\mathbb{L}$在$\mathbb{K}$上的基.

任意$l\in\mathbb{L}$均可表示为
\begin{equation}
l = \sum_{i=1}^m k_ib_i
\end{equation}
其中各$k_i\in\mathbb{K}$.

同样,各$k_i\in\mathbb{K}$均可表示为
\begin{equation}
k_i = \sum_{j=1}^n f_{ij}a_j
\end{equation}
其中各$f_i\in\mathbb{F}$.

于是,任意$l\in\mathbb{L}$可以表示为
\begin{equation}
l = \sum_{i=1}^m \sum_{j=1}^n f_{ij}a_jb_i
\end{equation}

也就是说,$\{a_jb_i\}$构成了$\mathbb{L}$作为$\mathbb{F}$上线性空间的基.稍加计算即可得$\abs{\{a_jb_i\}}=\abs{\{a_j\}}\times\abs{\{b_i\}}$.





\textbf{证毕}.


\autoref{FldExp_the3} 中的$\mathbb{K}$通常称为扩张$\mathbb{L}/\mathbb{F}$的\textbf{中间域},扩张$\mathbb{K}/\mathbb{F}$称为$\mathbb{L}/\mathbb{F}$的\textbf{子扩张}.






\begin{corollary}{}
如果$[\mathbb{K}:\mathbb{F}]$是一个素数,那么不存在\textbf{域}$\mathbb{M}$使得$\mathbb{F}\subsetneq\mathbb{M}\subsetneq\mathbb{K}$.
\end{corollary}


\begin{corollary}{中间域升链}

如果$\mathbb{K}/\mathbb{F}$是有限扩张,那么存在下面一系列域:
\begin{equation}
\mathbb{F}=\mathbb{F}_0\subseteq\mathbb{F}_1\subseteq\mathbb{F}_2\subseteq\cdots\subseteq\mathbb{F}_n=\mathbb{K}
\end{equation}

其中各$\mathbb{F}_{i+1}$是$\mathbb{F}_i$的\textbf{单扩张}.

\end{corollary}





\subsection{同构的开拓}

\begin{definition}{开拓}\label{FldExp_def6}
设$R_1$和$R_2$分别为环$K_1$和$K_2$的子环.若$\sigma:R_1\to R_2$和$\eta:K_1\to K_2$都是环同构,且$\eta|_{R_1}=\sigma$\footnote{即对于任意$r\in R_1$,有$\eta(r)=\sigma(r)$.},那么称$\eta$是$\sigma$的\textbf{开拓}.
\end{definition}

特别地,如果有扩域$\mathbb{K}_1/\mathbb{F}$和$\mathbb{K}_2/\mathbb{F}$,那么$\mathbb{K}_1$到$\mathbb{K}_2$的保$\mathbb{F}$同构(如果存在的话)是$\opn{id}_\mathbb{F}$的开拓.

同构的开拓有以下性质:

\begin{theorem}{}\label{FldExp_the4}
设$\sigma:\mathbb{F}_1\to\mathbb{F}_2$是域同构,$\mathbb{K}_1/\mathbb{F}$和$\mathbb{K}_2/\mathbb{F}$是域扩张.

则:

1. $\sigma$可开拓为$\mathbb{F}_1[x]\to\mathbb{F}_2[x]$的环同构,为方便仍记为$\sigma$.$p(x)\in\mathbb{F}_1[x]$不可约当且仅当$\sigma(p)$不可约.

2. 设$a_1\in\mathbb{K}_1-\mathbb{F}_1$在$\mathbb{F}_1$上的最小多项式为$f(x)$,$a_2\in\mathbb{K}_2-\mathbb{F}_2$在$\mathbb{F}_1$上的最小多项式为$\sigma(f(x))$,则存在$\sigma$的开拓$\eta:\mathbb{F}_1(a_1)\to \mathbb{F}_2(a_2)$,使得$\eta(a_1)=a_2$
\end{theorem}

\textbf{证明}:

1. 

显然,对于任意多项式$f\in\mathbb{F}_1[x]$,若其表达式为$\sum_{i=0}^n c_ix^i$,那么$\sigma(f)$的表达式为$\sum_{i=0}^n \sigma(c_i)x^i$.

由于$\sigma$是环同构,即具有高度对称性,因此“当且仅当”只需要证明充分性或必要性其一即可.

设$p(x)$可约,即存在阶数都大于$1$的多项式$h(x),g(x)$,使得$p=hg$.那么$\sigma(p)=\sigma(h)\sigma(g)$,从而$\sigma(p)$也可约.必要性得证,定理得证.

2. 

定义$\eta$为:对于任意$\mathbb{F}_1$上的多项式$f$,有$\eta(f(a_1))=f(a_2)$.

由\textbf{多项式环}\upref{RPlynm}的\autoref{RPlynm_the2}~\upref{RPlynm},$\eta:\mathbb{F}_1(a_1)\to \mathbb{F}_2(a_2)$是环同构\footnote{本证明也可以这样来理解:由于同构,$\mathbb{F}_1[x]$和$\mathbb{F}_2[x]$本质上就是同一个环,$f$和$\sigma(f)$也就是同一个元素.于是命题化为:如果在一个域上$a_1$和$a_2$有相同的最小多项式,那么用它们分别进行单扩张得到的扩域是否同构?根据\autoref{RPlynm_the2}~\upref{RPlynm},答案显然是肯定的.}.



\textbf{证毕}.

\autoref{FldExp_the4} 的第1条保证了$f$不可约则$\sigma(f)$也不可约,这是第2条中“$a_2$的最小多项式为$\sigma(f)$”有意义的前提,因为最小多项式必是不可约的.








\subsection{代数闭包}

最后,借着本节中代数元与超越元的讨论,我们引出代数闭包的概念:


\begin{definition}{代数闭包}
考虑扩域$\mathbb{K}/\mathbb{F}$.$\mathbb{K}$中所有$\mathbb{F}$的\textbf{代数元}构成一个新的集合$\mathbb{K}_{\mathbb{F}}$,称之为域$\mathbb{F}$在域$\mathbb{K}$上的\textbf{代数闭包(algebraic closure)}.
\end{definition}

\begin{theorem}{}

域$\mathbb{F}$在域$\mathbb{K}$上的代数闭包$\mathbb{K}_{\mathbb{F}}$是一个域.

\end{theorem}

\textbf{证明}:

$\mathbb{K}_{\mathbb{F}}$已经天然满足加法和乘法的结合性、单位元存在性以及乘法对加法的分配性.因此,我们只需要证明\textbf{加法与乘法的封闭性}和\textbf{加法与乘法的逆元存在性}.

\textbf{逆元存在性}证明:任取$\alpha\in\mathbb{K}_{\mathbb{F}}-\{0\}$,设其最小多项式为$\sum_{i=0}^{n}a_ix^i$,那么$-\alpha$也有最小多项式$\sum_{i=0}^n(-1)^ia_ix^i$,$1/\alpha$也有最小多项式$\sum_{i=0}^na_ix^{n-i}$.于是按\autoref{FldExp_def2} ,$-\alpha$与$1/\alpha$都是$\mathbb{F}$的代数元素,逆元存在性得证.

\textbf{封闭性}证明:记$\mathbb{F}(\alpha, \beta)=\mathbb{F}(\alpha)(\beta)=\mathbb{F}(\beta)(\alpha)$.如果$\alpha,\beta\in\mathbb{K}_{\mathbb{F}}$,那么$\mathbb{F}(\alpha)/\mathbb{F}$是一个代数扩张,$\mathbb{F}(\alpha, \beta)/\mathbb{F}(\alpha)$也是一个代数扩张,故由\autoref{FldExp_the2} 知它们都是有限扩张\footnote{思考:为什么不能用\autoref{FldExp_cor1} ?},故由\autoref{FldExp_the3} 知$\mathbb{F}(\alpha, \beta)/\mathbb{F}$是有限扩张,再由\autoref{FldExp_the2} 或者\autoref{FldExp_cor1} 知它是代数扩张.

因此,$\alpha+\beta\in\mathbb{F}(\alpha, \beta)$和$\alpha\beta\in\mathbb{F}(\alpha, \beta)$必是代数元素,封闭性得证.

\textbf{证毕}.


由于域的代数闭包也是一个域,故也可称之为\textbf{代数闭域}.注意,光有$\mathbb{F}$是无法讨论代数闭包的,必须有扩域$\mathbb{K}/\mathbb{F}$才行.


\begin{corollary}{代数扩张的代数扩张还是代数扩张}\label{FldExp_cor2}
如果$\mathbb{L}/\mathbb{K}$和$\mathbb{K}/\mathbb{F}$都是代数扩张,那么$\mathbb{L}/\mathbb{F}$也是代数扩张.
\end{corollary}

\textbf{证明}:

任取$x_0\in\mathbb{L}$,其在$\mathbb{K}$上的最小多项式为$g(x) = \sum_{i=0}^n k_ix^i$.那么$x_0$在$\mathbb{F}(k_0, k_1, \cdots, k_n)$上的最小多项式也是$g(x)$.

因此,$\mathbb{K}(x_0)/\mathbb{F}(k_0, k_1, \cdots, k_n)$是单代数扩张,故是有限扩张.类似地,$\mathbb{F}(k_0, k_1, \cdots, k_n)/\mathbb{F}$也是有限扩张.由\autoref{FldExp_the3} ,这意味着$\mathbb{K}(x_0)/\mathbb{F}$是有限扩张,从而是代数扩张.

因此,$x_0$是$\mathbb{F}$的代数元.

由$x_0$的任意性,命题即得证.

\textbf{证毕}.





