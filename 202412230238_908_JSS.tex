% 计算机科学(综述)
% license CCBYSA3
% type Wiki

本文根据 CC-BY-SA 协议转载翻译自维基百科\href{https://en.wikipedia.org/wiki/Computer_science}{相关文章}。

计算机科学是关于计算、信息和自动化的研究。[1][2][3] 计算机科学涵盖了从理论学科(如算法、计算理论和信息理论)到应用学科(包括硬件和软件的设计与实现)等多个领域。[4][5][6]

算法和数据结构是计算机科学的核心。[7] 计算理论研究的是计算的抽象模型和可以通过这些模型解决的普遍问题。密码学和计算机安全领域涉及研究安全通信的手段以及防止安全漏洞的技术。计算机图形学和计算几何学关注图像的生成。编程语言理论探讨了描述计算过程的不同方式,数据库理论则关注数据存储库的管理。人机交互研究人类和计算机交互的接口,软件工程则专注于开发软件的设计和原则。操作系统、网络和嵌入式系统等领域研究复杂系统的原理和设计。计算机架构描述了计算机组件和计算机操作设备的构建方式。人工智能和机器学习旨在合成目标导向的过程,如人类和动物中出现的问题解决、决策制定、环境适应、规划和学习过程。在人工智能领域,计算机视觉旨在理解和处理图像和视频数据,而自然语言处理则专注于理解和处理文本和语言数据。

计算机科学的根本问题是确定什么可以被自动化,什么不能。[2][8][3][9][10] 图灵奖通常被认为是计算机科学领域的最高荣誉。[11][12]