% 复数和有理数
% 复数和有理数

本文授权转载自郝林的 《Julia 编程基础》. 原文链接:\href{https://github.com/hyper0x/JuliaBasics/blob/master/book/ch05.md}{第 5 章 数值与运算}.


\subsection{5.4 复数和有理数}

\subsubsection{5.4.1 复数}

Julia预定义的复数类型是\verb|Complex|.它是\verb|Number|的直接子类型.为了构造出复数的虚部,Julia还专门定义了一个常量\verb|im|.这里的im是imaginary的缩写.它使用起来是这样的:
\begin{lstlisting}[language=julia]
julia> 1 + 2im; typeof(1+2im)
Complex{Int64}

julia> 1.1 + 2.2im; typeof(1.1+2.2im)
Complex{Float64}

julia> 
\end{lstlisting}

可以看到,\verb|Complex|是一个参数化的类型.因为在其名称的右侧还有一个由花括号包裹的类型参数.这个类型参数会是一个代表了某个类型的标识符.关于参数化类型,我们在下下一章就会讲到.

为了使常见的数学公式和表达式更加清晰,Julia允许在变量之前紧挨一个数值字面量,以表示两个数相乘.比如,如果变量\verb|x|的值是整数\verb|8|,那么\verb|2x^3|就表示\verb|2|乘以\verb|8|的\verb|3|次方.又比如,\verb|2^3x|表示\verb|2|的\verb|24|次方.在这种情况下,变\verb|x|就被称为数值字面量系数(numeric literal coefficient).

正因为如此,我们才需要特别注意,上例中的\verb|2im|和\verb|2.2im|虽然看起来与这种表示法非常相似,但其含义却是完全不同的.整数或浮点数的字面量与常量\verb|im|共同组成的是一个复数的虚部.而且还要注意,在构造复数的虚部时,我们就不能再使用数值字面量系数了.因为这肯定会产生歧义.比如,\verb|1 + 2xim|就是不合法的,除非已经存在一个名为\verb|xim|的变量,但如此一来这表示的就不是一个复数了.如果必须有变量参与复数的构造,那么我们可以使用\verb|complex|函数,例如:\verb|complex(1, 2x)|.

Julia允许复数参与标准的数学运算.所以,下面的这些数学表达式是合法的:
\begin{lstlisting}[language=julia]
julia> (1 + 2im) + (3 + 4im)
4 + 6im

julia> (1 + 2im) - (3 + 4im)
-2 - 2im

julia> (1 + 2im) * (3 + 4im)
-5 + 10im

julia> (1 + 2im) / (3 + 4im)
0.44 + 0.08im

julia> 3(1 + 2im)^8
-1581 + 1008im

julia> 
\end{lstlisting}

例子中的圆括号代表着对运算次序的设定.这与它在数学中的一般含义是一致的.

要想分别得到一个复数的实部和虚部,我们就需要调用\verb|real|函数和\verb|imag|函数.示例如下:
\begin{lstlisting}[language=julia]
julia> com1 = 1 + 2im 
1 + 2im

julia> real(com1), imag(com1)
(1, 2)

julia> 
\end{lstlisting}

另外,我们还可以利用、\verb|conj|函数求出一个复数的共轭(conjugate),以及使用\verb|abs|函数计算出一个复数与\verb|0|之间的距离,等等.总之,Julia预定义的很多数学函数都可以应用于复数.

\subsubsection{5.4.2 有理数}

我们在前面说过,浮点数无法精确地表示所有小数.比如,\verb|1/3|是一个无限循环小数,但用浮点数表示的话只能是这样的:
\begin{lstlisting}[language=julia]
julia> 1/3
0.3333333333333333

julia> typeof(ans)
Float64

julia> 
\end{lstlisting}

严格来说,\verb|1/3|并不是一个浮点数.因为浮点数会对无限循环小数做舍入,这会损失精度.但是,它肯定是一个有理数.

在 Julia 中,有理数用于表示两个整数之间的精确比率.有理数的类型是\verb|Rational|.它的值可以由操作符\verb|//|来构造.代码如下:
\begin{lstlisting}[language=julia]
julia> 1//3
1//3

julia> typeof(ans)
Rational{Int64}

julia> 
\end{lstlisting}

在操作符\verb|//|左侧的被称为分子,而在它右侧的被称为分母.注意,这两个数都只能是整数,而不能是浮点数.

如果在分子和分母之间存在公因数,那么Julia会自动地把它们化为最小项并让分母变为非负整数.例如:
\begin{lstlisting}[language=julia]
julia> 3//9
1//3

julia> 3//-9
-1//3

julia> 42//126
1//3

julia> 
\end{lstlisting}

函数\verb|numerator|和\verb|denominator|可以让我们分别得到一个有理数的分子和分母:
\begin{lstlisting}[language=julia]
julia> rat1 = 1//3
1//3

julia> numerator(rat1)
1

julia> denominator(rat1)
3

julia>  
\end{lstlisting}

有理数可以参与标准的数学运算.比如,我们可以拿一个有理数与一个整数、浮点数或者其他有理数进行比较.又比如,我们可以对有理数进行加减乘数等运算.另外,有理数也可以很容易地被转换为浮点数.例如:
\begin{lstlisting}[language=julia]
julia> float(1//3) 
0.3333333333333333

julia> 
\end{lstlisting}

我在前面也说了,这实际上会存在精度上的损失.

最后,需要我们注意的是,\verb|0//0|是不合法的.该字面量会引发一个错误.相应的,表示浮点数的字面量\verb|0/0|等同于\verb|NaN|.从技术标准的角度讲,\verb|NaN|不与任何东西(包括它自己)相等.