% 控制理论
% license CCBYSA3
% type Wiki

(本文根据 CC-BY-SA 协议转载自原搜狗科学百科对英文维基百科的翻译)

控制系统工程中的控制理论是数学的一个子领域,用于控制工程过程和机器中连续运行的动力系统。其目的是开发一种控制模型,用于以最佳方式使用控制动作来控制这些系统,而没有延迟或过冲并确保控制的稳定性。

为此,需要具有纠正措施的 控制器。该控制器监控受控过程变量(PV),并将其与参考点或设定点进行比较。过程变量的实际值和期望值之差,称为误差信号或SP-PV误差,作为反馈,产生控制动作,使受控过程变量达到与设定值相同的值。值得研究的其他方面还包括可控性和可观察性。在此基础上,先进的自动化技术彻底改变了制造业、飞机、通信和其他行业。反馈控制,通常是连续的,包括使用传感器进行测量并进行计算调整,以通过“最终控制元件”(例如控制阀)将测量变量保持在设定范围内。[1]

广泛使用的通常是被称为框图的图解样式。其中传递函数也被称为系统函数或网络函数,是描述系统的微分方程的输入和输出之间关系的数学模型。

控制理论的起源可以追溯到19世纪,当时詹姆斯·克拉克·麦克斯韦首次描述了调速器运行的理论基础。[2]1874年,爱德华·劳斯,查尔斯·斯特姆及随后的1895年,阿道夫·胡尔维茨进一步完善了控制理论,他们都对控制稳定性标准的建立做出了贡献;从1922年起,尼古拉斯·米诺斯基发展了 PID控制理论。[3]虽然控制理论的一个主要应用是控制系统工程,它涉及工业过程控制系统的设计,但其应用远远不止这些。作为反馈系统的一般理论,控制理论对于存在反馈的任何地方都是有用的。

\subsection{历史}



\subsection{开环和闭环(反馈)控制}



\subsubsection{2.1 其他例子}



\subsection{经典控制理论}



\subsection{闭回路传递函数}



\subsection{PID反馈控制}



\subsection{线性和非线性控制理论}



\subsection{分析技术-频域和时域}



\subsection{系统接口- SISO和MIMO}



\subsection{控制理论中的主题}

\subsubsection{9.1 稳定性}



