% 电路

\pentry{电压\upref{Voltag}, 欧姆定律\upref{Resist}}

\subsection{零势能点}
虽然我们可以规定电路中某点电势为零, 但大多数情况下这么做没有什么意义, 因为在这些电路中我们只讨论两点间的电压(电势差). 例如电阻器两端的电压与电流和电阻值有关而与零势点的选取无关.

\subsection{不存在净电荷}
宏观上来说, 我们一般假设电路中的任意一点不能存在净电荷. 这是因为电路中通常讨论的电压所能产生的净电荷往往可以忽略不计.

\begin{example}{}
一个半径为 $1\Si{cm}$ 的导体小球与无穷远之间的电容约为 $1.1\times 10^{-12} \Si{F}$(见\autoref{Cpctor_ex1}\upref{Cpctor}). 如果一个 $2\Si{V}$ 电源的正负极分别通过长导线连接两个这样的小球, 每个小球上最终只能积累 $\pm 1.1\times 10^{-12} \Si{C}$ 的电荷, 仅相当于 $1\Si{A}$ 的电流流动 $1.1\times 10^{-12}$ 秒. 
\end{example}
通过上例我们可以看出为什么我们假设开路



例如一个 $5\Si{V}$ 电源在两端接两个 $1\Si{cm}$ 的小球. 每个

\subsection{接地}
在电路中如果我们要规定零势点, 可以用接地符号. 这时我们假设大地的电势为零, 且电流可以流入和流出.

在实际运用中, 接地符号往往并不需要真的接地, 只是一种用于简化电路图的手段, 相当于把所有接地的点都用导线连接起来. 例如
(图未完成: 电源负极接地, 正极并联一些电阻电容, 它们的另一端也接地)




\subsection{好}

为了说明这个假设是合理的, 我们来看一个例子: 假设电路中某节点



根据电荷守恒定律,$q$是个常数,不能够随着时间推移而改变.由于这节点是个导体,不能储存任何电荷.所以,$q=0$、$i=0$,基尔霍夫电流定律成立:
\begin{equation}
\sum_{k=1}^n i_k =0
\end{equation}

不过,电容器的两块导板可能会允许正电荷或负电荷的累积.这是因为电容器的两块导板之间的空隙,会阻止分别累积于两块导板的异性电荷相遇,从而互相抵消.对于这种状况,流向其中任何一块导板的电流总和等于电荷累积的速率,而不是零.但是,若将位移电流$\mathbf{J}_D$纳入考虑,则基尔霍夫电流定律依然有效.

从电动力学的角度怎么来考虑推导呢?其实对含电介质的安培定律取散度,然后与高斯定律相结合,即可得到基尔霍夫电流定律:
\begin{equation}
\nabla \vdot \mathbf {J} =-\epsilon _{0}\nabla \vdot {\frac {\partial \mathbf {E} }{\partial t}}=-{\frac {\partial \rho }{\partial t}}
\end{equation}
 
其中,$\mathbf{J}$是电流密度,$\epsilon_0$是电常数,$\mathbf{E}$是电场,$\rho$是电荷密度.

这是电荷守恒的微分方程式.以积分的形式表述,从封闭表面流出的电流等于在这封闭表面内部的电荷$Q$的流失率:
\begin{equation}
\oint _{S}\mathbf {J} \cdot \mathrm {d} \mathbf {a} =-{\frac {\mathrm {d} Q}{\mathrm {d} t}}
\end{equation}

基尔霍夫电流定律等价于电流的散度是零.所以我们可以看出,对于不含时电荷密度$\rho$,该定律成立.对于含时电荷密度,则必需将位移电流纳入考虑.

从上述推导可以看到,\textbf{只有当电荷量为常数时,基尔霍夫电流定律才会成立}.通常,这不是个问题,因为静电力相斥作用,会阻止任何正电荷或负电荷随时间演进而累积于节点,大多时候,节点的静电荷是零.