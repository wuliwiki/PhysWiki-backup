% 函数的性质(高中)
% keys 函数|变换|平移|旋转|伸缩|单调|对称|奇偶性|初等函数|周期
% license Xiao
% type Tutor

\begin{issues}
\issueDraft
\end{issues}

\pentry{函数(高中)\nref{nod_functi}}{nod_6f70}

在前面的学习中,我们已经接触了函数的概念,以及如何通过复合运算来将多个函数结合起来。现在,我们将深入探讨一个函数的各种性质,比如它的单调性、奇偶性和周期性。这些性质是理解函数行为和性质的关键,它们能够帮助我们更好地分析函数在不同情况下的表现。

然后,我们会大概介绍一下高中阶段需要掌握的几种主要函数类型。这些函数在数学和实际应用中扮演着重要角色,它们构成了函数世界的一大部分。这里会把它们作为两部分来整体介绍,后面的学习中会具体介绍每一个函数的细节。


\subsection{零点}

函数的零点是指使函数值为零的自变量 $x$,从几何角度来看,函数 $f(x)$ 的零点就是其图像与 $x$ 轴的交点。一般情况下\footnote{确实有一些函数的零点是连续的,例如 $f(x) = 0$ 以及由此复合得到的函数。},零点是孤立的,而零点之外的点通常连续构成一个区间。在这些区间中,函数往往表现出某种特性(例如取正值或负值),通常,函数在零点的两侧会发生符号变化\footnote{从正值变为负值,或从负值变为正值}。而零点则作为这些区间的端点,成为函数符号变化的分界线。因此,判断某个点是否属于某个区间时,通常需要将其与零点进行比较。事实上,函数的零点之所以具有特殊意义,正是因为它常常对应某种边界条件,标志着状态的转变或某种变化的界限,比如物体的平衡点以及工程问题中的最优设计或操作条件等。

\begin{definition}{零点}
对于函数 $f(x)$,若 $f(x_0) = 0$ ,则称$x_0$为 $f(x)$ 的\textbf{零点(zero point)},即:
\begin{equation}
x_0 \in \{ x \mid f(x) = 0 \}~.
\end{equation}
\end{definition}

函数的零点与方程有密切联系,零点沟通了函数与方程。任何方程总可以通过移项转换成形如 $f(x) = 0$ 的形式\footnote{高中阶段只涉及一元函数,因此此处指的是一元方程},而这个表达式所描述的,就是函数 $f(x)$ 的零点,零点集就是方程的解集。这一点在初中学习时,相信你就已经感受过了。

两个函数的交点也可以用零点的形式来表示。假如要求解两个函数 $f(x)$ 和 $g(x)$ 的交点,其实就是寻找满足 $f(x) = g(x)$ 的点。通过设 $F(x) = f(x) - g(x)$,可以将问题转化为求 $F(x)$ 的零点,这样 $F(x)$ 的零点对应的就是 $f(x)$ 和 $g(x)$ 的交点。特别地,当函数 $f(x)$ 取某个固定值 $a$ 时,这个问题可以看作是 $g(x) = a$ 的特殊情况。此时我们设 $F(x) = f(x) - a$。根据函数的\aref{平移性质}{sub_FunTra_1},这相当于将函数 $f(x)$ 向下平移 $a$ 个单位(如果 $a$ 是负值,就向上平移 $|a|$ 个单位)\footnote{这里如果看不懂可以先跳过,看完\enref{函数的变换}{FunTra}再回过头来理解。}。因此,函数取某个值的点,其实就是平移后的零点。自然,$f(x)$ 的零点就是 $g(x) = 0$ 的特殊情况了。

如果一个零点在方程中出现多次,称为重根。例如,对于 $f(x) = (x - 1)^2$,$x = 1$ 是零点,但它是一个重数为 2 的零点。重数的概念在多项式函数的分析中很重要,零点的重数还与函数在该点的图像行为有关,比如多项式曲线在某些重的重根处“接触”而不是“穿过” $x$ 轴。这是很关键的一点,在不等式部分会着重介绍。

另外,在后面会接触到的导数中,零点也有特殊的含义。一般而言,一阶导数的零点称为\textbf{驻点},可能是函数的\textbf{极值点}。二阶导数的零点往往对应于函数的\textbf{拐点},也就是曲率改变的点。此处只是提及,具体内容会在导数部分详细讲解。

\subsubsection{零点存在定理}

尽管,在高中阶段涉及到的零点通常可以通过代入某些特殊值来求解,但有些时候,并不是需要求解某个具体的零点值,这时只要能够证明在某个区间上存在零点就可以了。于是需要使用\textbf{零点存在定理(Existence Theorem of Zero Points)}。

\begin{theorem}{零点存在定理}
若函数 $f(x)$ 在$[a,b]$上连续,且满足 $f(a)$ 和 $f(b)$异号,即$f(a)f(b)<0$ ,则在 $(a, b)$ 上至少存在一个零点,即:
\begin{equation}
\exists x_0\in(a,b),f(x_0)=0~.
\end{equation}
\end{theorem}

通常,在使用零点存在定理时,首先需要找到一个区间的两个端点,然后通过证明函数在这些端点上的值符号相反(一个为正,一个为负),从而推断该区间内存在至少一个零点。接着,假设这个零点为 $x_0$,然后进一步研究 $x_0$ 的一些性质。

零点存在定理其实是一个更广泛定理——“\aref{介值定理}{sub_conff_4}”在取值为0时的特例。介值定理本质上反映了\enref{实数完备性}{RCompl}这一深层性质。因此,零点存在定理是一个非常基础的定理,但正因其基础性,证明过程相对复杂,在高中阶段不要求掌握其证明。在大学阶段,还会学习与零点存在定理相关的一组定理,被称为\enref{中值定理}{MeanTh}。这些定理构成了非常强大的数学工具,不仅能够判断零点或某些特殊取值的存在性,甚至还提供了高精度的估计方法。

\subsubsection{二分法}

基于零点存在定理,有一种在某个区间内逐步逼近函数零点的数值求解方法称为“二分法”。它的核心思路就是,始终保持区间端点函数值异号,这样就可以保证在区间上始终有零点,然后逐步缩小区间的范围,来保证精度。

\begin{theorem}{“二分法”算法}
初始条件:

给定函数 $f(x)$ 和一个函数连续的区间 $[a, b]$,且满足$f(a) \cdot f(b) < 0$,给出两个精度,一个是最小区间长度$d$,另一个是与零值接近的一个容许范围$\varepsilon$。

迭代过程:

\begin{enumerate}
\item 计算区间中点 $m = \frac{a + b}{2}$。
\item 计算 $f(m)$。
\item 判断 $f(m)$ 的符号:
\begin{itemize}
\item 如果 $f(m) = 0$,则 $m$ 就是零点。
\item 如果 $f(a) \cdot f(m) < 0$,则零点位于区间 $[a, m]$,将 $b$ 更新为 $m$。
\item 如果 $f(b) \cdot f(m) < 0$,则零点位于区间 $[m, b]$,将 $a$ 更新为 $m$。
\end{itemize}
\end{enumerate}

停止条件:

$|b - a|<d$ 或 $|f(m)|<\varepsilon$ 。此时,$m$ 可以作为零点的近似解。
\end{theorem}

迭代过程就是不断循环,直到符合停止条件为止。二分法的名称也来自于每次都要计算区间中点的行为。

二分法的收敛速度较为稳定,但相对较慢。当然,除此之外还有“牛顿法”等一系列方法,它们是使用计算机求解方程的重要工具,它们构成了一个名为\enref{数值分析}{NordEq}的学科。如果将求解零点的行为,看作搜索的话,二分法也是一种搜索方法。这两部分在高中并不涉及,提及只为扩展视野。

\subsection{变化率}
斜率
平均变化率
导数

\subsection{单调性}\label{sub_HsFunC_1}

\begin{definition}{单调性}
设$f(x)$是定义在$D$上的函数,若在$I\subseteq D$上,对$\forall x_1,x_2\in I,x_1< x_2$,函数均满足:
\begin{equation}
f(x_1)<f(x_2)~.
\end{equation}
则称函数$f(x)$在区间$I$上\textbf{单调递增(monotonically increasing})\footnote{此处采取高中教材上的定义。其实,满足这个条件时称作\textbf{严格单调递增 (Strictly increasing)},而单调递增则是指函数满足$f(x_1)\leq f(x_2)$时。递减也相同。这个概念会在大学阶段区分,目前给出作为提醒。},或函数$f(x)$在区间$I$上是\textbf{增函数};若满足
\begin{equation}
f(x_1)>f(x_2)~.
\end{equation}
则称函数$f(x)$在区间$I$上\textbf{单调递减(monotonically decreasing}),或函数$f(x)$在区间$I$上是\textbf{减函数}。

若函数在定义域上单调,则称为\textbf{单调函数 (monotonic functions)}。
\end{definition}

函数在某个区间上递增的充要条件是这个区间上任意两点的连线斜率为正,同理,递减则为负。


\subsection{对称性与奇偶性}

对称性分为两种,一种是轴对称性,一种是中心对称性,这两个性质在初中就有接触过。

\begin{definition}{轴对称}
如果函数$f(x)$对定义域$D$上的任意$x$均满足,
\begin{equation}
f(a-x)=f(a+x)~.
\end{equation}
则称,函数$f(x)$关于直线$x=a$\textbf{轴对称(Centrosymmetric)}。
\end{definition}


\begin{definition}{中心对称}
如果函数$f(x)$对定义域$D$上的任意$x$均满足,
\begin{equation}
f(a-x)+f(a+x)=2b~.
\end{equation}
则称,函数$f(x)$关于点$(a,b)$\textbf{中心对称(Centrosymmetric)}。
\end{definition}
定义要求定义域的区间需要关于$a$对称,也就是$\forall x\in D,2a-x\in D$。


注意这里要区分轴对称性和反函数的区别。

有两个比较特殊的对称性称为奇偶性,其中偶函数是指关于$y$轴对称的函数,奇函数是指关于原点中心对称的函数。

\begin{definition}{奇偶性}
设函数$f(x)$定义在$D$上,且$D$是关于$0$对称的。若对任意的$x\in D$,有:
\begin{itemize}
\item $f(x)=f(-x)$,则称$f(x)$是偶函数。
\item $f(x)=-f(-x)$或$-f(x)=f(-x)$,则称$f(x)$是奇函数。
\end{itemize}
\end{definition}

\subsection{周期性}

在生活中,很多事情都是有规律的,比如每天日出日落、四季轮回。数学中用周期性来描述这种“规律”。周期性的函数图象好比一首不断循环的旋律,它遵循着固定的步调,过一段时间就会“回到原点”,再继续以同样的方式变化。

\begin{definition}{周期}
设函数$f(x)$定义在$D$上,且满足
\begin{equation}
\forall x\in D,f(x+T)=f(x)~.
\end{equation}
则称,$T$是函数的一个周期。
\end{definition}


周期函数具有一些特性:
\begin{itemize}
\item 周期函数的图像会在每个周期$T$内重复。无论在$x$轴上平移多少个周期,函数的形状和取值都会保持不变。
\item 两个周期分别为$T_1,T_2$的周期函数,只有满足它们的周期之比$k={T_1\over T_2}$为有理数,即$k={p\over q},p,q\in\mathbb{N}^*$时,他们的和才是周期函数,周期为$T=pT_2=qT_1$。周期相同可以认为是二者周期之比为$1$的特殊情况,此时和仍然是周期函数且周期为原周期。
\item 周期函数在定义域内一定不是单调的,因此周期函数也没有反函数。
\end{itemize}

高中阶段涉及的周期函数主要是两类:一类是抽象函数,也就是不给出表达式,然后利用周期性的特性来等量替换;另一类是三角函数,这将在\enref{三角函数}{HsTrFu}的部分详细讲解。



\subsection{其他性质}


函数具有一些性质,有一些在高中会接触到,有一些不会接触到。
以后我们会看到一些用\enref{极限}{Lim}和\enref{导数}{Der}描述的性质。 例如 % \addTODO{链接}
, 可导。
还有一些性质是高中不会涉及到的,此处给出:
\begin{itemize}
\item \enref{连续性}{contin}, 一致连续
\end{itemize}

\subsection{特殊的函数}

在高中阶段会涉及到的两种特殊的函数包括初等函数和分段函数。

\subsubsection{初等函数}

高中研究的函数都是初等函数。初等函数指的是由基本初等函数经过基本运算(加减乘除)以及复合形成的函数。

初等函数之所以被称为初等函数就是因为它的性质很好,

基本初等函数:
\begin{itemize}
\item 常值函数
\item 幂函数
\item 指数函数
\item 对数函数
\item 三角函数
\end{itemize}

\subsubsection{分段函数}

绝对值函数

取整函数

狄利克雷函数