% 詹姆斯·克拉克·麦克斯韦(综述)
% license CCBYSA3
% type Wiki

本文根据 CC-BY-SA 协议转载翻译自维基百科\href{https://en.wikipedia.org/wiki/James_Clerk_Maxwell}{相关文章}。
\begin{figure}[ht]
\centering
\includegraphics[width=6cm]{./figures/86970eb17e0c5fea.png}
\caption{Maxwell, 大约 1870 年代} \label{fig_Clerk_1}
\end{figure}
詹姆斯·克拉克·麦克斯韦(James Clerk Maxwell FRS FRSE,1831年6月13日-1879年11月5日)是一位苏格兰物理学家和数学家[1],他提出了电磁辐射的经典理论,这是第一个将电、磁和光视为同一现象的不同表现形式的理论。麦克斯韦的电磁学方程实现了物理学中的第二次伟大统一[2],第一次统一则由艾萨克·牛顿实现。麦克斯韦在统计力学的创立中也起到了关键作用[3][4]。

1865年,麦克斯韦发表了《电磁场的动力学理论》,在该文中他证明了电场和磁场作为波动通过空间传播,传播速度与光速相同。他提出光是同一介质中的波动,这种介质也是电磁现象的来源[5]。光与电现象的统一促使他预言了无线电波的存在,并且这篇论文包含了他自1856年以来一直在研究的方程的最终版本[6]。由于他的方程以及他在解决网络问题和线性导体问题上提出的有效方法,他被视为现代电气工程学科的奠基人之一[7]。1871年,麦克斯韦成为第一任卡文迪许物理学教授,并一直担任此职直到1879年去世。

麦克斯韦是第一个推导出麦克斯韦–玻尔兹曼分布的人,这是一种描述气体动理论各个方面的统计方法,他在职业生涯中断断续续地研究了这一课题[8]。他还因于1861年展示了第一张持久的彩色照片而闻名,并且在分析杆接框架(如许多桥梁中的桁架)的刚性方面做出了基础性贡献。麦克斯韦帮助建立了CGS测量系统[9],并且他对现代尺寸分析的贡献也不可忽视[10][11]。麦克斯韦还因奠定了混沌理论的基础而受到认可[12][13]。他正确预测土星的环是由许多未附着的小碎片组成的[14]。他于1863年发表的《关于调速器》一文为控制理论和控制论提供了重要的基础,也为控制系统的早期数学分析[15][16]。在1867年,他提出了著名的思想实验——麦克斯韦妖[17]。

他的发现帮助开启了现代物理学的时代,为相对论等领域奠定了基础,他也是第一个将这一术语引入物理学的人[10],并对量子力学的发展做出了贡献[18][19]。许多物理学家认为,麦克斯韦是19世纪对20世纪物理学影响最大的科学家。他对科学的贡献被许多人认为与艾萨克·牛顿和阿尔伯特·爱因斯坦的贡献同样重要[20]。在麦克斯韦诞辰一百周年时,爱因斯坦曾形容他的工作是“自牛顿时代以来物理学所经历的最深刻且最富有成果的工作”[21]。当爱因斯坦于1922年访问剑桥大学时,他的主人告诉他,他之所以取得伟大的成就,是因为站在牛顿的肩膀上;爱因斯坦回答说:“不,我不是。我站在麦克斯韦的肩膀上。”[22] 汤姆·西格弗里德(Tom Siegfried)形容麦克斯韦是“那种百年一遇的天才,他比周围的人更敏锐地感知到物理世界。”[23]

