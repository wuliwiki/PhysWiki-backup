% 函数
% 函数|定义域|值域|二元函数

\pentry{映射\upref{map}}

\begin{issues}
\issueTODO
\end{issues}

广义来说, 任何映射都可以叫做\textbf{函数(function)}. 所以我们也可以用映射的记号表示函数. 例如 $f: \mathbb R \to \mathbb R$ 表示定义域为实数集, 值为实数的函数, 通常记为 $f(x)$. 又例如 $f: \mathbb R^2 \to \mathbb C$ 表示实变量和复值的二元函数 $f(x_1, x_2)$. 注意其中 $\mathbb R^2$ 表示笛卡尔积(\autoref{Set_eq1}~\upref{Set}) $\mathbb R \times \mathbb R$.

\subsection{复合函数}
\addTODO{复合函数即复合映射}

\subsection{函数的性质}
以后我们会看到一些用极限\upref{Lim}和导数\upref{Der}描述的性质. 例如连续性\upref{contin}, 一致连续 % 未完成
, 可导.
