% 北京大学 2008 年 考研 普通物理
% license Usr
% type Note

\textbf{声明}:“该内容来源于网络公开资料,不保证真实性,如有侵权请联系管理员”

力学(50分)
\subsection{(16分)}
质点作平面运动,已知其速度为$\vec{v} = (A \omega \sin \omega t) \hat{i} + (B \omega \cos \omega t) \hat{j}$,且$t=0$时,$x=0,y=0$,其中$A,B,\omega$为大于零的常量。(1)求质点的轨迹方程,画出轨迹曲线,并用箭头画出质点的运动方向;(2)求求任意时刻$t$时质点的法向加速度和切向
加速度。
\subsection{(16 分)}
水平面内的大圆环(半径为$R$,圆心为$O$’)绕过环上一点A的竖直轴以匀角速度$\omega_0$,转动。质量为$m$的小圆环套在大圆环上可以无摩擦地渭动,相对速度可表示为$\vec{v} =R\dot{\theta}\hat e_\theta$以大圆环为参考系(1)用$m,\omega_0,R,\theta,\dot{\theta},\hat e_r,$和$\hat e_\theta$等表示出质点所受的惯性力和大环对它的约束力:(2)试求出m在大环上的相对平衡位置和在平衡位置附近作小振动时的周期。






