% 刚体的惯量主轴
% 转动惯量|刚体|惯性张量|主轴|本征矢

\begin{issues}
\issueDraft
\end{issues}

\pentry{惯性张量\upref{ITensr}, 对称矩阵的本征问题\upref{SymEig}} % 角动量守恒\upref{AMLaw} 已经有了

刚体转动惯量和角速度的关系为
\begin{equation}
\bvec L = \mat I \bvec \omega~.
\end{equation}
我们下面来讨论什么情况下 $\bvec L$ 和 $\bvec \omega$ 会共线。

我们知道惯性张量是一个 $3\times 3$ 的对称矩阵, 即 $I_{ij} = I_{ji}$。 而对称矩阵必定具有三个相互垂直的本征矢量 $\bvec \omega_i$, 满足
\begin{equation}\label{eq_PrncAx_1}
\bvec I \bvec \omega_i = I_i \bvec \omega_i \qquad (i = 1,2,3)~.
\end{equation}
注意把 $\bvec \omega_i$ 乘以任意常数仍然满足上式。 也就是说, 如果刚体绕着三个主轴旋转, 那么它的转动惯量与角速度共线。 其中本征值 $I_i$ 称为刚体的三个\textbf{主转动惯量}。

\begin{example}{长方体的主轴}\label{ex_PrncAx_1}
在\autoref{ex_ITensr_1}~\upref{ITensr}中, 我们知道长方体的惯性张量为
\begin{equation}
\mat I = \frac{1}{12} M
\pmat{
   b^2 + c^2 & 0 & 0\\
   0 & a^2 + c^2 & 0\\
   0 & 0 & a^2 + b^2~,
}
\end{equation}
求本征方程就相当于把矩阵对角化, 但这已经是一个对角化矩阵, 所以本征矢就是三个单位矢量 $(1,0,0)$, $(0,1,0)$ 和 $(0,0,1)$, 即三个主轴分别沿长方体的三条边, 它们对应的转动惯量分别为
\begin{equation}
\begin{aligned}
&I_1 = \frac{1}{12}M(b^2+c^2)~,\\
&I_2 = \frac{1}{12}M(a^2+c^2)~,\\
&I_3 = \frac{1}{12}M(a^2+b^2~.
\end{aligned}
\end{equation}
\end{example}

\subsection{本征值简并}
\addTODO{应该在“对称矩阵的本征问题\upref{SymEig}” 中说明一下简并问题, 在这里引用}
一般情况下, 本征方程\autoref{eq_PrncAx_1} 的三个本征值 $I_i$ 互不相同, 这时三个主轴是唯一确定的。 否则, 我们把出现相同本征值的情况叫做简并。 但若其中两个本征值相同(例如 $I_1 = I_2$), 那么 $\bvec \omega_1, \bvec \omega_2$ 所在平面的任意矢量也都是主轴。 若三个本征值都相同, 那么任意方向的转轴都是主轴。 例如在\autoref{ex_PrncAx_1} 中, 若 $a,b,c$ 各不相同, 那么三个主轴只能沿三条边的方向。 若 $a = b$, 那么 $\bvec \omega_1, \bvec \omega_2$。

\subsection{自由旋转体}
这里来看主轴的一个重要应用
\begin{theorem}{}
如果一个刚体绕某点旋转且合外力矩为零, 那么它将绕主轴旋转。
\end{theorem}
我们可以用角动量守恒来证明。
