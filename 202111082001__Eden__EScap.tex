% 固体热容的爱因斯坦理论
% 固体热容|玻尔兹曼分布

经典理论中,常将固体视作 $N$ 个原子组成的体系,每个原子在平衡位置作经典的简谐微振动,共 $3N$ 个自由度,每一个自由度上有振动过程中的动能和势能,即
\begin{equation}
\epsilon=\frac{1}{2m}p^2+\frac{m\omega^2}{2}q^2
\end{equation}
经典理论中用能量均分定理讨论了固体热容,则固体的内能为
\begin{equation}
U=2\cdot 3N\cdot \frac{kT}{2}=3NkT
\end{equation}
所得结果在高温和室温范围内与实验结果符合,但是在低温附近却与实验结果不符.爱因斯坦首先用量子理论分析了固体热容问题,成功解释了固体热容随温度下降而下降的实验事实.

\subsection{固体热容的爱因斯坦理论}