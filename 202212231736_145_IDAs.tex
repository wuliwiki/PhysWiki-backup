% IDA-star 算法
% 搜索|算法

\pentry{迭代加深\upref{ID} A-star 算法\upref{Astar}}

IDA* 算法是加了估价函数的迭代加深。

A* 算法是在优先队列 BFS 上加了估价函数,估价函数也当然可以和 DFS 结合,但如果只是和最普通的 DFS 结合在一起很容易出现搜索深度很深,但答案深度很浅的情况,所以可以将迭代加深 DFS 和估价函数结合在一起,就形成了 IDA-star 算法。IDA* 算法的估价函数和 A* 非常类似,都是表示当前状态到目标状态的估计距离,IDA* 当然也要满足估计距离不大于真实距离这个前提。IDA* 使用迭代的框架,如果当前深度加估计距离大于深度限制,则直接回溯。

\textbf{IDA* 的框架:}
\begin{lstlisting}[language=cpp]
int f();   // 估价函数

// depth 当前搜索层数,max_depth 为迭代加深的搜索深度限制
bool dfs(int depth, int max_depth)  
{
    // 如果当前层数 + 估价函数 > 深度限制,则直接回溯
    if (depth + f() > max_depth) return false; 
    if (!f()) return true;  // 一般估价函数为 0 说明找到了答案
    
    /*
    以下为 dfs 内容
    */
    
    return false;   // 找不到答案就回溯
}

int main()
{
    int depth = 0;
    while (!dfs(0, depth)) depth ++ ;   // 迭代加深
    
    return 0;
}
\end{lstlisting}

\href{http://poj.org/problem?id=3460}{例题 $1$:排书}

题目大意:每次可以将打乱的图书的一段取出放到其他位置的后面,问最少需要多少次可以将打乱的图书按照递增的顺序排列。

首先确定搜索顺序:枚举序列中每一段的图书摆到哪些位置,对于一段图书摆到一个位置的前面或者一个位置的后面是等价的,所以只需枚举摆到后面,摆的位置从选的那一段图书的右端点的位置加一开始枚举。这样就可以不遗漏的枚举出每种状态。

设计估价函数:设定一个正确/错误后继的概念,若一个图书 $i$,如果按照递增的顺序排列,它的下一个位置应该是 $i + 1$,我们称 $i + 1$ 是 $i$ 的正确后继,错误后继显然就是图书 $i$ 后面不是 $i + 1$ 的情况。

想要将一个打乱的图书变为递增的序列,显然序列中错误的后继数为 $0$,所以我们统计一个图书序列中错误的后继数,记为 $\tt cnt$,可以发现每次操作最多更改三本书的后继,若每次操作是最理想的,将一个乱序变为递增的序列最少也需要 $\left\lceil\dfrac{\tt cnt}{3}\right\rceil$ 次操作,满足估价函数的前提。故可将估价函数设计为 $f(\tt state) = \left\lceil\dfrac{\tt cnt}{3}\right\rceil$。

