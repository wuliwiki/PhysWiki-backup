% 傅里叶变换的差分矩阵
% keys 差分矩阵|无穷阶|傅里叶变换
% license Usr
% type Tutor

\pentry{导数近似之差分矩阵算法\nref{nod_DifMa}}{nod_3681}
\enref{差分矩阵}{DifMa} $D$ 和数据点矢量 $u$ 矩阵乘法 $Du$ 给出数据点处的近似导数。本词条将给出基于傅里叶变换给出的差分矩阵。
\subsection{傅里叶变换回顾}
\subsubsection{连续区间上的傅里叶变换}
由\enref{正交函数系}{OFS}中的\autoref{ex_OFS_3} 可知,函数 $u(x),x\in\mathbb R$ 的傅里叶变换定义为
\begin{equation}\label{eq_InfDM_1}
\hat u(k):=\int_{-\infty}^{\infty} e^{-\I kx}u(x)\dd x,\quad k\in\mathbb R.~
\end{equation}
而从 $\hat u$ 的逆傅里叶变换可以重构 $u$:
\begin{equation}\label{eq_InfDM_2}
u(x)=\frac{1}{2\pi}\int_{-\infty}^{\infty} e^{\I kx}\hat u(k)\dd k.~
\end{equation}
这被称为\textbf{傅里叶合成}(Fourier synthesis),变量 $x$ 称为\textbf{物理变量}(physical variable),$k$ 称为\textbf{傅里叶变量}(Fourier variable)或\textbf{波数}(wavenumber)。

\subsubsection{离散点上的傅里叶变换}
\textbf{无界情形}

当限定 $x\in h\mathbb Z$时,即此时 $x$ 只能取离散点,那么只要 $k_1-k_2=\frac{2\pi}{h}$,就有 $e^{\I k_1x}=e^{\I k_2x}$ 对所有的 $x$ 成立。这使得对任意复指数 $e^{\I kx}$,有无穷多个复指数和它在 $h\mathbb Z$ 上具有相同的值。这一现象被称为\textbf{混叠}。为了保持\autoref{eq_InfDM_1} 中傅里叶变换 $\hat u(k)$ 的单值性,需要将 $k$ 限制在长为 $\frac{2\pi}{h}$ 的区间上。因此,我们可以选择 $k$ 的取值区间为 $[-\pi/h,\pi/h]$。注意到 $x$ 是物理变量,$k$ 是傅里叶变量,因此我们可以将这些结果总结为下面的图示:
\begin{equation}
\begin{aligned}
&\text{物理空间:}& \text{离散},&\qquad\text{无界:} &x\in h\mathbb Z\\
&&\updownarrow&\qquad\updownarrow&\\
&\text{傅里叶空间:}& \text{有界},&\qquad\text{连续:} &k\in [-\pi/h,\pi/h]
\end{aligned}~
\end{equation}
因此,在无界离散的情形,傅里叶变换及其逆变换\autoref{eq_InfDM_1} 和\autoref{eq_InfDM_2} 应当写为下面的级数和:
\begin{equation}\label{eq_InfDM_3}
\begin{aligned}
\hat u(k)&=h\sum_{i=-\infty}^{\infty}e^{-\I kx_i}u(x_i),\quad k\in[-\pi/h,\pi/h],\\
u(x_j)&=\frac{1}{2\pi}\int_{-\pi/h}^{\pi/h} e^{\I kx_j}\hat u(k)\dd k,\quad x_j=jh,j\in \mathbb Z.
\end{aligned}~
\end{equation}
这被称为\textbf{半离散傅里叶变换}。

\textbf{周期性情形}

当限制 $u(x_i)$ 对 $x_i$ 具有周期 $2L$ 时,即 $u(x_i+2L)=u(x_i)$,那么为了 $u$ 的单值性,限定 $x_i\in[0,2L]$。设 $x_i$ 有 $N$ 个(本词条将始终假定 $N$ 为偶数),于是 $x_j\in \{h,\cdots,Nh=2L\}$,其中 $h:=\frac{2L}{N}$。仍记 $x_j:=jh,j\in\{1,\cdots,N\}$。和无界情形一样的理由,$k$ 应当限制在 $[-\pi/h,\pi/h]=[-\frac{N\pi}{2L},\frac{N\pi}{2L}]$ 上。此外由于周期性,应当认为点 $x+2L$ 和点 $x$ 是同一个。因此 $e^{\I k(x+2L)}=e^{\I k x}$,从而
\begin{equation}
2Lk=2n\pi\Rightarrow k=n\frac{\pi}{L},\quad n\in\mathbb Z.~
\end{equation}
结合 $k\in[-\frac{N\pi}{2L},\frac{N\pi}{2L}]$,有
\begin{equation}
k\in\qty{\qty(-\frac{N}{2}+1)\frac{\pi}{L},\qty(-\frac{N}{2}+2)\frac{\pi}{L},\cdots,\frac{N}{2}\frac{\pi}{L}}.~
\end{equation}
注意这里用到了 $-\frac{N}{2}\frac{\pi}{L},\frac{N}{2}\frac{\pi}{L}$ 对应相同的 $e^{-\I kx}$。这些结果可以总结为下面的图示:
\begin{equation}
\begin{aligned}
&\text{物理空间:}& \text{离散},&\qquad\text{有界:} \qquad x\in \{h,\cdots,2L\}\\
&&\updownarrow&\qquad\updownarrow&\\
&\text{傅里叶空间:}& \text{有界},&\qquad\text{离散:} \qquad k\in \qty{\qty(-\frac{N}{2}+1)\frac{\pi}{L},\qty(-\frac{N}{2}+2)\frac{\pi}{L},\cdots,\frac{N}{2}\frac{\pi}{L}}.
\end{aligned}~
\end{equation}
相应的,傅里叶变换和逆变换的公式写为下面的级数和,
\begin{equation}\label{eq_InfDM_7}
\begin{aligned}
\hat u(k)&=h\sum_{i=1}^{N}e^{-\I k\frac{\pi}{L}x_i}u(x_i),\quad k\in\qty{\qty(-\frac{N}{2}+1),\qty(-\frac{N}{2}+2),\cdots,\frac{N}{2}},\\
u(x_j)&=\frac{1}{2L}\sum_{k=-\frac{N}{2}+1}^{\frac{N}{2}}e^{\I k\frac{\pi}{L}x_j}\hat u(k),\quad j\in\{1,\cdots,N\}.
\end{aligned}~
\end{equation}
这被称为\textbf{离散傅里叶变换}。
\subsection{插值函数}
\subsubsection{无界情形}
假若我们有定义在 $x_i\in h\mathbb Z$ 上的数据 $u_i=u(x_i)$,那么\autoref{eq_InfDM_3} 将会给我们插值函数:

1.通过\autoref{eq_InfDM_3} 的第一式我们可以获得 $\hat u(k)$,它关于 $k$ 是可微的;

2.而\autoref{eq_InfDM_3} 的第二式可以确定一个函数
\begin{equation}\label{eq_InfDM_4}
p(x)=\frac{1}{2\pi}\int_{-\pi/h}^{\pi/h} e^{\I kx}\hat u(k)\dd k,\quad x\in \mathbb R.~
\end{equation}
显然 $p(x)$ 满足 $p(x_i)=u_i$,并且 $p(x)$ 是可微的。因此可以用 $p(x)$ 的导数来近似 $u(x)$ 的导数。$p(x)$ 给了我们一个插值公式。

由\autoref{eq_InfDM_4} ,只需令 $\hat u(k)$ 仅在 $[-\pi/h,\pi/h]$ 上不为零。那么\autoref{eq_InfDM_4} 的积分下限和上限就可以变到 $-\infty,\infty$。因此由\autoref{eq_InfDM_1} 和\autoref{eq_InfDM_2} 可知 $p(x)$ 的傅里叶变换为
\begin{equation}
\hat p(k)=\left\{\begin{aligned}
&\hat u(k),&k\in[-\pi/h,\pi/h],\\
&0,&k\not\in[-\pi/h,\pi/h].
\end{aligned}\right.~
\end{equation}
由于这一点, \autoref{eq_InfDM_4} 的  $p(x)$ 被称为是 $v$ 的\textbf{有限带宽插值函数}。这一术语暗示着 $p$ 的傅里叶变换仅在 $[-\pi/h,\pi/h]$ 上不为0(仅在某一区域 $U$ 上不为0的函数往往称为具有支撑,$U$ 称为它的支撑集。若 $U$ 还是拓扑意义上的紧集,又被称为具有紧支撑)。

\subsubsection{周期情形}
假设数据定义在 $x_i=ih,i\in\{1,\cdots N\},h=\frac{2L}{N}$ 上,$u_i=u(x_i)$。那么:

1.通过\autoref{eq_InfDM_7} 的第一式我们可以获得 $\hat u(k)$.

2.而\autoref{eq_InfDM_7} 的第二式可以确定一个函数(为了更对称的处理,并考虑到 $e^{\I k\frac{\pi}{L}x}$ 在 $k=\frac{N}{2},-\frac{N}{2}$ 相等)  
\begin{equation}\label{eq_InfDM_8}
p(x)=\frac{1}{2L}{\sum'}\limits_{k=-\frac{N}{2}}^{\frac{N}{2}}e^{\I k\frac{\pi}{L}x}\hat u(k),~
\end{equation}
其中 $'$ 表示求和在 $-N/2,N/2$ 处需要乘 $1/2$。显然 $p(x)$ 满足 $p(x_i)=u_i$,并且 $p(x)$ 是可微的。因此可以用 $p(x)$ 的导数来近似 $u(x)$ 的导数。$p(x)$ 便是需要的插值公式。

\subsection{导数近似算法}
\subsubsection{无界情形}
有了\autoref{eq_InfDM_4} 的插值函数 $p(x)$,我们就可以进行导数近似了:

1.给定定义在 $h\mathbb Z$ 上的函数 $u$,利用\autoref{eq_InfDM_4} 确定它的有限带宽函数 $p(x)$;

2.令 $w_j=p'(x_j)$,则 $w_j$ 就是 $u$ 在 $x_j$ 处的近似导数。

当然,我们还有另一等价的方式来描述,它是在傅里叶空间中表述的:\autoref{eq_InfDM_2}  对 $x$ 微分得
\begin{equation}
u'(x)=\frac{1}{2\pi}\int_{-\infty}^{\infty} \I ke^{\I kx}\hat u(k)\dd k,\quad x\in \mathbb R.~
\end{equation}
由傅里叶正反变换的关系\autoref{eq_InfDM_1} 和\autoref{eq_InfDM_2} 可得
\begin{equation}
\hat{u'}(k)=ik\hat u(k).~
\end{equation}
因此我们得到另一等价的导数近似程序:

1.给定 $u$,利用\autoref{eq_InfDM_3} 的第一式计算它的离散傅里叶变换 $\hat u$;

2.定义 $\hat w(k)=\I k\hat u(k)$;

3.利用\autoref{eq_InfDM_3} 的第二式从 $\hat w$ 计算 $w$。这即得到在给定点的导数近似。
\subsubsection{周期情形}
周期情形的程序为:

1.给定定义在 $\{h,\cdots,Nh=2L\}$($x_i=ih,i=1,\cdots,N$) 上的函数 $u$,利用\autoref{eq_InfDM_8} 确定它的插值函数 $p(x)$;

2.令 $w_j=p'(x_j)$,则 $w_j$ 就是 $u$ 在 $x_j$ 处的近似导数。


\subsection{无穷阶差分矩阵}
\subsubsection{无界情形}
由\autoref{eq_InfDM_3}  可知,若 $u(x)=\sum\limits_{j}a_jf_j(x)$,$a_i\in\mathbb C$, 那么
\begin{equation}
\begin{aligned}
\hat u(k)&=h\sum_{i=-\infty}^{\infty}e^{-\I kx_i}\qty(\sum\limits_{j}a_jf_j(x_i))\\
&=\sum\limits_{j}a_j\qty(h\sum_{i=-\infty}^{\infty}e^{-\I kx_i}f_j(x_i))\\
&=\sum_{j}a_j\hat f_j(k).
\end{aligned}~
\end{equation}
即傅里叶变换是线性的。而任意的离散点 $u(x_j)$ 都可以用Kronecker Delta 函数($x_j=jh$) 
\begin{equation}
\delta(x)=\left\{\begin{aligned}
&1,&x=0,\\
&0,&x\neq0.
\end{aligned}\right.~
\end{equation}
表示为 $u(x_j)=\sum\limits_{m=-\infty}^{\infty}u(x_m)\delta(x_j-x_m)$。因此,若知道 $\delta(x-x_m)$ 的傅里叶变换 $\hat\delta_m(k)$,则\autoref{eq_InfDM_3} 的第一式就为 
\begin{equation}
\hat u(k)=\sum\limits_{m=-\infty}^\infty u(x_m)\hat\delta_m(k).~
\end{equation}
从而\autoref{eq_InfDM_4} 的 $p(x)$ 成为
\begin{equation}
\begin{aligned}
p(x)&=\frac{1}{2\pi}\int_{-\pi/h}^{\pi/h} e^{\I kx}\hat u(k)\dd k\\
&=\sum\limits_{m=-\infty}^\infty u(x_m)\frac{1}{2\pi}\int_{-\pi/h}^{\pi/h} e^{\I kx}\hat\delta_m(k)\dd k.
\end{aligned}~
\end{equation}

而
\begin{equation}
\begin{aligned}
\hat\delta_m(k)=h\sum_{i=-\infty}^{\infty}e^{-\I kx_i}\delta(x_i-x_m)=he^{-\I kx_m}.
\end{aligned}~
\end{equation}
所以
\begin{equation}
\begin{aligned}
p(x)&=\sum\limits_{m=-\infty}^\infty u(x_m)\frac{h}{2\pi}\int_{-\pi/h}^{\pi/h} e^{\I k(x-x_m)}\dd k\\
&=\sum\limits_{m=-\infty}^\infty u(x_m)\frac{h}{2\pi}\frac{e^{\I \frac{\pi}{h}(x-x_m)}-e^{-\I \frac{\pi}{h}(x-x_m)}}{\I (x-x_m)}\\
&=\sum\limits_{m=-\infty}^\infty u(x_m)\frac{h}{\pi}\frac{\sin(\frac{\pi}{h}(x-x_m))}{x-x_m}\\
&=\sum\limits_{m=-\infty}^\infty u(x_m)\sinc \qty(\frac{\pi}{h}(x-x_m)),
\end{aligned}~
\end{equation}
其中 $\sinc(x):=\frac{\sin x}{x}$。从而
\begin{equation}\label{eq_InfDM_5}
\begin{aligned}
w_j&=p'(x_j)=\sum\limits_{m=-\infty}^\infty u(x_m)\sinc '\qty(\frac{\pi}{h}(x-x_m))|_{x=x_j=jh}.
\end{aligned}~
\end{equation}
因为 $x\neq0$ 时,
\begin{equation}
\sinc'(x)=\qty(\frac{\sin x}{x})'=\frac{x\cos x-\sin x}{x^2},~
\end{equation}
而 $\sinc'0=0$,所以 $x\neq x_m$ 时
\begin{equation}
\begin{aligned}
&\sinc '\qty(\frac{\pi}{h}(x-x_m))|_{x=x_j=jh}=\frac{\pi}{h}\sinc '[(j-m)\pi]\\
&=\frac{(-1)^{j-m}}{(j-m)h}.
\end{aligned}~
\end{equation}
所以\autoref{eq_InfDM_5} 成为
\begin{equation}\label{eq_InfDM_6}
w_j=p'(x_j)=\sum\limits_{m=-\infty,m\neq j}^\infty \frac{(-1)^{j-m}}{(j-m)h}u(x_m).~
\end{equation}
该式表明,若已知数据点 $u_i=u(x_i),i\in\mathbb Z$,则其对应的导数可由上式求得。由\autoref{sub_DifMa_1}
中差分矩阵的概念,即差分矩阵和数据点矢量的矩阵乘法给出数据点的近似导数,因此我们得到定义在 $h\mathbb Z$ 上对应的差分矩阵,其是无穷阶的(\autoref{eq_InfDM_6} 表明对应指标(i,m) 的矩阵元为 $\frac{(-1)^{j-m}}{(j-m)h}$,容易验证其是\enref{Toeplitz矩阵}{ToepM} ):
\begin{equation}
h^{-1}
\begin{pmatrix}
&&&&&&&&\vdots&&&&&&&&\\
&&&&&&&&1/3&&&&&&&&\\
&&&&\ddots&&&&-1/2&&&&&&&&\\
&&&&\ddots&&&&1&&&&&&&&\\
&&&&\ddots&&&&0&&&&\ddots&&&&\\
&&&&&&&&-1&&&&\ddots&&&&\\
&&&&&&&&1/2&&&&\ddots&&&&\\
&&&&&&&&-1/3&&&&&&&&\\
&&&&&&&&\vdots&&&&&&
\end{pmatrix}.~
\end{equation}

\subsubsection{周期情形}
同样的,\autoref{eq_InfDM_8} 可以写为
\begin{equation}
\begin{aligned}
p(x)&=\frac{1}{2L}{\sum'}\limits_{k=-\frac{N}{2}}^{\frac{N}{2}}e^{\I k\frac{\pi}{L}x}\hat u(k)\\
&=\frac{1}{2L}\sum_{m=1}^{N}u(x_m){\sum'}\limits_{k=-\frac{N}{2}}^{\frac{N}{2}}e^{\I k\frac{\pi}{L}x}\hat \delta_m(k),~
\end{aligned}~
\end{equation}
而
\begin{equation}
\hat \delta_m(k)=h\sum_{i=1}^{N}e^{-\I k\frac{\pi}{L}x_i}\delta(x_i-x_m)=he^{-\I k\frac{\pi}{L}x_m}.~
\end{equation}
所以
\begin{equation}
\begin{aligned}
p(x)
&=\frac{h}{2L}\sum_{m=1}^{N}u(x_m){\sum'}\limits_{k=-\frac{N}{2}}^{\frac{N}{2}}e^{\I k\frac{\pi}{L}(x-x_m)}\\
&=\frac{h}{2L}\sum_{m=1}^{N}u(x_m)\qty(\frac{1}{2}\sum_{k=-\frac{N}{2}}^{\frac{N}{2}-1}e^{\I k\frac{\pi}{L}(x-x_m)}+\frac{1}{2}\sum_{k=-\frac{N}{2}+1}^{\frac{N}{2}}e^{\I k\frac{\pi}{L}(x-x_m)})\\
&=\frac{h}{2L}\sum_{m=1}^{N}u(x_m)\cos\qty(\frac{\pi}{2L}(x-x_m))\sum_{k=-\frac{N}{2}+1/2}^{\frac{N}{2}-1/2}e^{\I k\frac{\pi}{L}(x-x_m)}\\
&=\frac{h}{2L}\sum_{m=1}^{N}u(x_m)\cos\qty(\frac{\pi}{2L}(x-x_m))\frac{e^{\I (-N/2+1/2)\frac{\pi}{L}(x-x_m)}-e^{\I (N/2+1/2)\frac{\pi}{L}(x-x_m)}}{1-e^{\I \frac{\pi}{L}(x-x_m)}}\\
&=\frac{h}{2L}\sum_{m=1}^{N}u(x_m)\cos\qty(\frac{\pi}{2L}(x-x_m))\frac{e^{-\I \frac{N\pi}{2L}(x-x_m)}-e^{\I \frac{N\pi}{2L}(x-x_m)}}{e^{-\I \frac{\pi}{2L}(x-x_m)}-e^{\I \frac{\pi}{2L}(x-x_m)}}\\
&=\frac{h}{2L}\sum_{m=1}^{N}u(x_m)\cos\qty(\frac{\pi}{2L}(x-x_m))\frac{\sin\qty(\frac{N\pi}{2L}(x-x_m))}{\sin\qty(\frac{\pi}{2L}(x-x_m))}\\
&=\sum_{m=1}^{N}u(x_m)\frac{\sin\qty(\frac{N\pi}{2L}(x-x_m))}{\frac{2L}{h}\tan\qty(\frac{\pi}{2L}(x-x_m))}.
\end{aligned}~
\end{equation}

由
\begin{equation}
\dv{}{x}\qty(\frac{\sin\qty(\frac{N\pi}{2L}(x-x_m))}{\frac{2L}{h}\tan\qty(\frac{\pi}{2L}(x-x_m))})|_{x=x_j}=\left\{\begin{aligned}
&0,\quad j=m\\
&\frac{\pi}{2L}(-1)^{(j-m)}\cot[\pi(j-m)/N],j\neq m
\end{aligned}\right.~
\end{equation}

从而
\begin{equation}
\begin{aligned}
w_j=p'(x_j)=\frac{\pi}{2L}\sum_{m=1,m\neq j}^{N}(-1)^{(j-m)}\cot[\pi(j-m)/N]u(x_m)
\end{aligned}~
\end{equation}
因此,差分矩阵为
\begin{equation}
\frac{\pi}{2L}
\begin{pmatrix}
0&\cot(\pi/N)&-\cot(2\pi/N)&\cdots&\cot((N-1)\pi/N)\\
-\cot(\pi/N)&\ddots&\ddots&\ddots&\ddots\\
\cot(2\pi/N)&\ddots&\ddots&\ddots&\ddots\\
\vdots&\ddots&\ddots&\ddots&\ddots\\
-\cot((N-1)\pi/N)&\ddots&\ddots&\ddots&\ddots\\
\end{pmatrix}.~
\end{equation}




































