% 汉斯·贝特(综述)
% license CCBYSA3
% type Wiki

本文根据 CC-BY-SA 协议转载翻译自维基百科\href{https://en.wikipedia.org/wiki/Hans_Bethe#Honors_and_awards}{相关文章}。


汉斯·阿尔布雷希特·贝特(Hans Albrecht Bethe,1906年7月2日–2005年3月6日)是一位德裔美籍物理学家,他在核物理学、天体物理学、量子电动力学和固态物理学方面做出了重要贡献,并因其在恒星核合成理论方面的工作获得了1967年诺贝尔物理学奖。[1][2][3][4] 贝特大部分职业生涯在康奈尔大学担任教授。[5]

1939年,贝特发表了一篇论文,确立了CNO循环作为更大质量恒星在主序星阶段的主要能量来源,这一贡献为他赢得了1967年的诺贝尔奖。[6] 在第二次世界大战期间,贝特担任洛斯阿拉莫斯国家实验室的理论部主任,该实验室研发了第一颗原子弹。他在计算武器的临界质量并开发用于“胖子”原子弹(在1945年8月投放到长崎)的内爆方法方面发挥了关键作用。

战后,贝特在氢弹的研发中发挥了重要作用,他还担任该项目的理论部门负责人,尽管他最初加入该项目的目的是证明氢弹无法制造。[7] 后来,他与阿尔伯特·爱因斯坦以及原子科学家紧急委员会一起,反对核试验和核军备竞赛。他帮助说服肯尼迪和尼克松政府分别签署了1963年的部分核试验禁令条约和1972年的反弹道导弹条约(SALT I)。1947年,他写了一篇重要论文,提供了对兰姆位移的计算,这一工作被认为彻底改变了量子电动力学,并进一步“为现代粒子物理学时代开辟了道路”。[8][9][10] 他对中微子的理解做出了贡献,[11] 并在解决太阳中微子问题中发挥了关键作用。[12] 他还为理解超新星及其过程做出了贡献。[13]

他的科学研究从未停止,直到九十多岁时他仍在发表论文,这使他成为少数几位在其职业生涯的每个十年中至少发表过一篇重要论文的科学家之一,而贝特的职业生涯几乎跨越了七十年。物理学家弗里曼·戴森(曾是他的博士生)称他为“20世纪的最高级问题解决者”,[14] 而宇宙学家爱德华·科尔布则称他为“物理学界最后的老大师”。[15]
\subsection{早年生活}  
贝特于1906年7月2日出生在斯特拉斯堡,当时该地区是德国的莱茵省-阿尔萨斯-洛林的一部分。他是安娜(娘家姓库恩)和阿尔布雷希特·贝特(斯特拉斯堡大学生理学 Privatdozent)的独生子。[16] 尽管他的母亲是阿布拉罕·库恩(斯特拉斯堡大学教授)的女儿,具有犹太背景,[17] 贝特在成长过程中像父亲一样被抚养成新教徒;[18][19] 他后来成为无神论者。[20]
\begin{figure}[ht]
\centering
\includegraphics[width=6cm]{./figures/f8a1eabf841b3f3b.png}
\caption{12岁的汉斯·贝特与父母一起。} \label{fig_Hans_1}
\end{figure}
1912年,他的父亲接受了基尔大学生理学研究所教授兼所长的职位,家人搬进了研究所的所长公寓。最初,他由一位专业教师私下授课,和其他七个男孩女孩一起学习。[21] 1915年,当他的父亲成为新成立的法兰克福歌德大学生理学研究所的所长时,家人再次搬迁。[18]

贝特曾就读于德国法兰克福的歌德中学。1916年,他因患上结核病而中断学业,被送往巴德克罗伊茨纳赫休养。到1917年,他恢复得足够好,能够就读当地的实科学校。次年,他被送往欧登瓦尔德学校,这是一所私立的男女同校寄宿学校。[22] 1922年至1924年,他重新回到歌德中学,完成了中学最后三年的学业。[23]

贝特在通过了高中毕业考试(Abitur)后,于1924年进入法兰克福大学。他决定主修化学。由于物理教学质量较差,尽管法兰克福有像卡尔·路德维希·齐格尔和奥托·萨茨等杰出的数学家,但贝特并不喜欢他们的方法,这些方法将数学呈现得脱离其他学科。[24] 贝特发现自己是一个不太擅长实验的人,他曾因将硫酸洒在实验服上而把实验服弄坏,但他发现由副教授沃尔特·格拉赫教授的高阶物理更有趣。[24][25] 1925年,格拉赫离开了,接替他的卡尔·迈斯纳建议贝特应该去一所理论物理学更强的大学,尤其是慕尼黑大学,在那里他可以师从阿诺德·索末菲尔德。[26][27]

贝特于1926年4月进入慕尼黑大学,索末菲尔德在迈斯纳的推荐下接纳了他成为学生。[28] 索末菲尔德开设了一门关于物理中微分方程的高级课程,贝特非常喜欢这门课。由于索末菲尔德是如此著名的学者,他常常提前收到科学论文的稿件,并将这些论文作为讨论材料,在每周的晚间研讨会上进行讨论。当贝特到达时,索末菲尔德刚刚收到埃尔文·薛定谔关于波动力学的论文。[29]

对于贝特的博士论文,索末菲尔德建议他研究晶体中的电子衍射。作为起点,索末菲尔德建议他参考保罗·埃瓦尔德1914年关于晶体中X射线衍射的论文。贝特后来回忆说,他变得过于雄心勃勃,为了追求更高的准确性,他的计算变得不必要地复杂。[30] 当他第一次见到沃尔夫冈·泡利时,泡利对他说:“听了索末菲尔德讲的关于你的事,我原本期待你能做出比你论文更好的成果。”[31] 贝特后来回忆道:“我想那应该是泡利的夸奖。”[31]