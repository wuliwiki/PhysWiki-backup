% 切线与割线
% keys 微积分|极限|割线|切线
% license Xiao
% type Tutor

% \begin{issues}
% % \issueTODO
% % \issueOther{如果是两点趋近于一点, 就是双重极限有点复杂了, 应该改成 $A$ 点固定, $B$ 点从两个方向分别趋近于 $A$ 点, 且结果一致, 类似于左极限和右极限。}
% \end{issues}

\pentry{函数的极限(简明微积分)\nref{nod_FunLim}}{nod_f7a3}

如图一,在一段光滑曲线上任取两点,过这两点做直线,就是曲线过 $A$ 点与 $B$ 点的\textbf{割线}(当然直线与曲线还可以有其他交点)。
\begin{figure}[ht]
\centering
\includegraphics[width=6cm]{./figures/49e1d3dcd57839f5.pdf}
\caption{割线} \label{fig_TanL_1}
\end{figure}

当 $A$,  $B$ 两点逐渐向某点 $C$ 点靠近,割线的位置逐渐趋于不变。我们有很多种不同的方式使得$A, B$靠近$C$,比如先将$A$点移动到$C$点,再让$B$点慢慢靠近$C$点(反过来也可以);或者$A, B$以不同的速度靠近$C$点。如果用不同的方式能得到唯一确定的一条直线,我们就把它就叫做曲线在 $C$ 点的\textbf{切线},我们称割线的极限\upref{Lim}存在;如果不存在,那么称 $C$ 点没有切线。下面举一个简单的例子说明。

\begin{figure}[ht]
\vskip 0pt
\centering
\includegraphics[width=5cm]{./figures/194793dc8f4ad74c.pdf}
\caption{割线的极限是切线} \label{fig_TanL_2}
\end{figure}
例如要求正方形一角的切线,用
 $A$,  $B$ 两点接近 $C$ 点,则无论 $AB$ 点
有多么靠近 $C$, 切线的位置还要取决于 $AB$ 点的具体位置(如右图)。若 $B$ 更接近 $C$, 则直线就更接近竖直方向。反之直线就更接近水平方向。 用不同的方式得到的极限并不相同,因此 $C$ 点不存在切线。

\begin{figure}[ht]
\centering
\includegraphics[width=7cm]{./figures/3e99a18638471b5b.pdf}
\caption{C 点不存在切线} \label{fig_TanL_3}
\end{figure}

\subsubsection{用两点定义切线}
上面定义切线使用了三点。但注意到我们强调若切线存在,则 $A,B$ 接近 $C$ 的速度是不做要求的。 那我们能不能直接假设 $B$ 点始终与 $C$ 重合,而让 $A$ 接近二者呢?答案是可以的,但需要注意为了避免\autoref{fig_TanL_3} 的情况,此时 $A$ 必须从曲线的两个方向分别接近 $C$,并确保得到的极限是同一条直线。
