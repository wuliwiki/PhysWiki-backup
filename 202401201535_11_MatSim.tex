% 相似变换和相似矩阵
% keys 相似变换|相似矩阵|酉矩阵|对角化|本征值
% license Xiao
% type Tutor

\begin{issues}
\issueDraft
\end{issues}

\pentry{酉矩阵\upref{UniMat}}

矩阵是线性映射在特定基下的表示,因此我们可以改变空间的基向量组,从而得到线性映射的不同表示。
\begin{definition}{相似变换}\label{def_MatSim_1}
设$A,B$为$n$阶矩阵,若存在$n$阶可逆方阵$Q$,使
\begin{equation}
Q^{-1}AQ=B~,
\end{equation}
则称$A$\textbf{相似}于$B$,该运算称为对$A$进行相似变换,可逆矩阵$Q$称为过渡矩阵。
\end{definition}
相似关系是同阶矩阵群的等价关系(等价关系用$\sim $表示),这种关系具有下列性质:

\begin{enumerate}
\item 反身性 $A\sim A$ 
\item 对称性 若$A\sim B$,则$B\sim A$
\item 传递性 若$A\sim B\,,B\sim C$,则$A\sim C$
\end{enumerate}
由于相似变换是对同一空间的不同基向量组进行一一映射,因而相似变换是矩阵乘法群上的自同构,并具有如下性质(假设$A\sim B$):
\begin{enumerate}
\item $\opn{R}(A)=\opn{R}(B)$;
\item $A$与$B$的行列式相同:$|A|=|B|$;
\item $A$与$B$的迹相同:$\opn{Tr}A=\opn{Tr}B$;
\item 若$A$可逆,则$B$也可逆,且$A^{-1}\sim B^{-1}$;
\item $kA\sim kB\,,A^m\sim B^m$,其中$k$为任意常数,$m\in \mathbb Z^{+}$;
\item 若$f(x)$是任意多项式,则$f(A)\sim f(B)$。
\end{enumerate}
以上六点性质在线性映射的角度上看dou
\subsection{对角化}
\pentry{厄米矩阵的本征问题\upref{HerEig}}
\addTODO{该内容应该移动到 厄米矩阵的本征问题\upref{HerEig}}
若相似变换可以使矩阵变为对角矩阵, 我们把这个过程称为对角化。

一个 $N$ 维矩阵可以被对角化当且仅当它是(实)对称矩阵或厄米矩阵。
\begin{equation}\label{eq_MatSim_1}
\mat U\Her \mat A \mat U = \mat \Lambda~.
\end{equation}

对角化后, 对角矩阵 $\mat \Lambda$ 的对角元就是矩阵 $\mat A$ 的本征值 $\lambda_i$, $\mat U$ 的第 $i$ 列矢量就是 $\lambda_i$ 对应的本征矢。 所以我们时常把 “对角化” 作为 “解矩阵的本征方程” 的同义词。

(未完成)

\begin{example}{由本征值和本征矢求矩阵}
已知本征方程
\begin{equation}
\mat A \bvec v = \lambda \bvec v~.
\end{equation}
的 $N$ 个本征值和本征矢为 $\lambda_i$ 和 $\bvec v_i$, 求矩阵 $\bvec A$。

把\autoref{eq_MatSim_1} 两边分别左乘 $\mat U$, 右乘 $\mat U\Her$, 得
\begin{equation}
\mat A = \mat U \mat \Lambda \mat U\Her~.
\end{equation}
\end{example}
