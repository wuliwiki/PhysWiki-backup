% 电磁波包的能量
% 电磁波|波包|能量

对于真空中的平面波电磁波, 波速恒定为 $c$, 如果知道某点 $x_0$ 处的电场—时间关系 $g(t)$, 如何求波函数 $f(x - ct)$ 呢? 代入 $x = x_0$ 可知 $g(t) = f(x_0 - ct)$, 所以
\begin{equation}
f(x) = g\qty(\frac{x_0 - x}{c})
\end{equation}

当这个波包完整经过一个平面后, 穿过平面的能量密度为(积分上下限为 $\infty$)
\begin{equation}\label{WpEng_eq1}
\sigma_E = \epsilon_0 \int f(x)^2 \dd{x} = \epsilon_0  \int g^2\qty(\frac{x_0 - x}{c}) \dd{x} = c\epsilon_0 \int g^2(u) \dd{u}
\end{equation}

\subsection{能量的频率分布}
傅里叶变换可以保持\autoref{WpEng_eq1} 中的积分不变, 所以若令 $g$ 的傅里叶变换为 $\tilde g$ 则
\begin{equation}
\sigma_E = c\epsilon_0 \int {\tilde g}^2(\omega) \dd{\omega}
\end{equation}
所以面密度的频率分布为
\begin{equation}
s(\omega) = c\epsilon_0 {\tilde g}^2(\omega)
\end{equation}
光子能量分布为.
\begin{equation}
s(E) = \frac{c\epsilon_0}{\hbar} {\tilde g}^2\qty(\frac{E}{\hbar})
\end{equation}
