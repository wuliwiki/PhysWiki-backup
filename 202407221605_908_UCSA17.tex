% 中国科学院大学 2017 年考研 量子力学
% license Usr
% type Note

\textbf{声明}:“该内容来源于网络公开资料,不保证真实性,如有侵权请联系管理员”

一、已知一粒子处于一维无限深势阱时,势函数为
$$V = \begin{cases} 0 & 0 < x < a \\\\+\infty & x < 0, x > a \end{cases}~$$
能量本征函数数为 $\psi_n = \sqrt{\frac{2}{a}} \sin \frac{n\pi x}{a}$,$t = 0$ 时 $\psi(x, 0) = A x (a - x)$。

(1) 求归一化系数 $A$

(2) 求出 $t > 0$ 时的波函数 $\psi(x, t)$

(3) 该粒子处于基态和第一激发态的概率

(4) $t > 0$ 时,粒子坐标的平均值$x$.

二、如图,已知入射波函数为 $e^{ik_1x}$。
\begin{figure}[ht]
\centering
\includegraphics[width=10cm]{./figures/35ea0c2d3e63989d.png}
\caption{} \label{fig_UCSA17_2}
\end{figure}
(1) 求入射粒子的流密度

(2) 当$E = V_b$时,透射系数的取值为( )

A 0

B $0 < T < 1$

C 1

(3) 为了怎发透射系数,可以采取的办法有( )

A 保持$E = V_b$,降低Va,但保证$V_a > V_b$

B 在满足$V_a > V_b$的条件下,使$E > V_b$

C 保持$E = V_b$,降低Va,保证$V_a < V_b$

(4) 若已知入射波函数数为$ e^{ik_1x}$ ,反射波函数数为$R e^{ik_1x}$,透射波函数数为$S e^{ik_2x}$,其中 $k_1^2 = \frac{2mE}{\hbar^2}, k_2^2 = \frac{2m(E - V_0)}{\hbar^2} $,透射系数为( )

A. $|S|^2$ \\\\
B. $1 - |R|^2$ \\\\
C. 以上都不是