% 艾萨克·巴罗(综述)
% license CCBYSA3
% type Wiki

本文根据 CC-BY-SA 协议转载翻译自维基百科\href{https://en.wikipedia.org/wiki/Isaac_Barrow}{相关文章}。

\begin{figure}[ht]
\centering
\includegraphics[width=6cm]{./figures/3f0be7137842d59d.png}
\caption{} \label{fig_ASKbl_1}
\end{figure}
艾萨克·巴罗(Isaac Barrow,1630年10月-1677年5月4日)是一位英国基督教神学家和数学家,通常被认为在微积分早期的发展中扮演了重要角色,特别是在证明微积分基本定理方面作出了贡献。\(^\text{[1]}\)他的研究主要集中于切线的性质;巴罗是第一个计算出卡帕曲线切线的人。他还因担任著名的卢卡斯数学教授讲席第一任教授而备受瞩目,该职位后来由他的学生艾萨克·牛顿继承。
\subsection{生平}
\subsubsection{早年生活与教育}
\begin{figure}[ht]
\centering
\includegraphics[width=6cm]{./figures/37d75e02495b220e.png}
\caption{《剑桥大学公开讲座讲义》(Lectiones habitae in scholiis publicis academiae Cantabrigiensis AD 1664)} \label{fig_ASKbl_2}
\end{figure}
巴罗出生于伦敦。他的父亲托马斯·巴罗是一名亚麻布商人。1624年,托马斯娶了来自肯特郡北克雷的威廉·巴金之女安,他们的儿子艾萨克出生于1630年。看起来巴罗是这段婚姻中唯一的孩子——至少是唯一存活至幼年之后的孩子。安约在1634年去世,守寡的父亲于是将年幼的艾萨克送往他祖父、剑桥郡治安官艾萨克处抚养,祖父当时住在斯宾尼修道院\(^\text{[2]}\)。然而不到两年,托马斯便再婚,新妻是来自肯特郡梅德金的亨利·奥克辛登之妹凯瑟琳·奥克辛登。这段婚姻至少育有一女伊丽莎白(生于1641年)和一子托马斯。小托马斯曾拜皮革商爱德华·米勒为师,于1647年完成学徒任务,1680年移民至巴巴多斯\(^\text{[3]}\)。
\subsubsection{早期生涯}
艾萨克最初就读于查特豪斯公学,在那里他性格顽劣、好斗,甚至父亲曾被人听到祈祷说,如果上帝要带走他的一位孩子,那么他最愿意牺牲的就是艾萨克。之后他转学至费尔斯特学校,在那里他终于安定下来,并在杰出的清教徒校长马丁·霍尔比奇指导下学习。霍尔比奇在十年前曾教过约翰·沃利斯\(^\text{[4]}\)。在费尔斯特学校,他学习了希腊语、希伯来语、拉丁语和逻辑,为进入大学做准备\(^\text{[5]}\)。随后他进入剑桥大学三一学院继续深造;他选择三一学院的原因是沃波尔家族某位成员向他提供了资助,“这项资助可能出于沃波尔家族对巴罗忠于王党事业的同情”\(^\text{[6]}\)。他的叔叔、同名的艾萨克·巴罗,后来成为圣阿萨夫主教,是彼得学院的院士。

巴罗刻苦学习,在古典学和数学方面表现出色;1648年取得学位后,他于1649年当选为三一学院的院士\(^\text{[7]}\)。1652年,他在导师詹姆斯·杜波特指导下获得剑桥大学文学硕士学位(MA),随后在学院中居住了几年,并曾竞选希腊语教授职位。但在1655年,由于拒绝签署效忠英格兰共和国的《誓言》,他获得了一笔旅行资助,开始出国游历\(^\text{[8]}\)。

\textbf{旅行经历}

接下来的四年中,他游历了法国、意大利和土耳其。在土耳其,他曾居住在伊兹密尔,并在伊斯坦布尔(当时分别称为士麦那和君士坦丁堡)求学。历经多次冒险后,他于1659年返回英格兰。他以勇敢著称,最为人称道的是,他曾凭借个人英勇之举,拯救了自己所乘的船只免遭海盗劫掠。他被描述为“身材矮小、体型瘦削、肤色苍白”,衣着邋遢,并有长期抽烟的习惯,是一位“根深蒂固的烟民”。在宫廷活动中,他因机智风趣而深得查理二世赏识,也赢得了其他朝臣的尊重。他的著作风格延续并展现出某种庄重典雅的修辞。作为当时令人印象深刻的人物之一,他一生清白,行事谨慎而勤勉\(^\text{[9]}\)。
\subsubsection{后期生涯}
\textbf{工作经历}

1660年王政复辟后,他受封为牧师,并被任命为剑桥大学希腊语御用教授。1662年,他出任格雷沙姆学院几何学教授;1663年,又被选为剑桥大学卢卡斯数学教授席位的首任教授。在任期间,他发表了两部学术水平极高且风格优雅的数学著作,一部探讨几何学,另一部论述光学。1669年,他将卢卡斯教授职位让予艾萨克·牛顿【10】。大约在此期间,巴罗还撰写了《信经释义》、《主祷文释义》、《十诫释义》与《圣礼释义》等神学作品。此后,他专心研究神学。1670年,他被国王钦命授予神学博士学位;两年后(1672年),出任三一学院院长,并创立了该学院的图书馆,直至去世一直担任此职。
\begin{figure}[ht]
\centering
\includegraphics[width=6cm]{./figures/24531eeb77ba9bab.png}
\caption{剑桥大学三一学院教堂内的艾萨克·巴罗雕像} \label{fig_ASKbl_3}
\end{figure}
他最早的著作是欧几里得《几何原本》的完整版本,分别于1655年以拉丁文出版,1660年又出版了英文版;随后在1657年,他又出版了《已知量》的版本。他在1664年、1665年和1666年所讲授的数学讲义,于1683年以《数学讲义》为题发表,这些讲义主要探讨数学真理的形而上学基础。他1667年的讲义也在同年出版,其中提出了阿基米德获得其主要结果所使用的分析方法。1669年,他出版了《光学与几何讲义》。据说牛顿对这些讲义进行了校订和修正,并加入了自己的一些内容,但从牛顿在“流数法争议”中的评论来看,这些新增内容大概只涉及光学部分。这部著作是巴罗在数学领域中最重要的作品之一,并在1674年作了少许修改后再版。1675年,他又出版了对佩加的阿波罗尼乌斯《圆锥曲线论》前四卷的评注版本,并对阿基米德和比提尼亚的狄奧多修斯现存的著作进行了整理与注释。

在光学讲义中,他以巧妙的方法处理了许多与光的反射与折射相关的问题。他首次定义了通过反射或折射所看到的一个点的“几何焦点”,并解释说,一个物体的像是该物体上每一个点的几何焦点的轨迹。他还推导出了一些薄透镜的基本性质,并大大简化了笛卡尔对彩虹形成的解释。

此外,巴罗是第一个以封闭形式求出正割函数积分的人,从而证明了当时一个广为人知的猜想。

\textbf{逝世与遗产}

巴罗于46岁时在伦敦去世,终身未婚,葬于西敏寺。约翰·奥布里在《简略传记》中将他的死归因于他在土耳其居住期间染上的鸦片瘾。

除上述著作外,他还撰写了其他重要的数学论文;但在文学领域,他的地位主要依靠其布道文来维持,这些布道文堪称雄辩推理的杰作;\(^\text{[11]}\)而他所著的《论教皇至上权》更被视为现存最完美的论战作品之一。巴罗的人格在各方面都配得上他的非凡才华,尽管他性格中也带有强烈的怪癖色彩。
\subsection{计算切线}
几何讲义中包含了一些求曲线面积与切线的新方法。其中最著名的是求曲线切线的方法,这一点非常重要,值得详细说明,因为它展示了巴罗、赫德和斯吕兹如何沿着费马所提出的思路,朝着微分法的发展方向努力。

费马曾指出,如果除了曲线上一点 $P$ 外,还知道另一点,则可以确定该点的切线;因此,如果能求出子切线 $MT$ 的长度(从而确定点 $T$),那么连线 $TP$ 就是所要求的切线。巴罗注意到:如果作出点 $P$ 邻近点 $Q$ 的横坐标与纵坐标,并连接得到小三角形 $PQR$,则这个三角形的边 $QR$ 与 $RP$ 就分别是 $P$ 与 $Q$ 的横坐标和纵坐标的差,因此他称其为“差分三角形”,从而有:
$$
\frac{TM}{MP} = \frac{QR}{RP}~
$$
为了求出 $\frac{QR}{RP}$,他设点 $P$ 的坐标为 $(x, y)$,点 $Q$ 的坐标为 $(x - e, y - a)$(巴罗本人使用 $p$ 表示 $x$,$m$ 表示 $y$,但本文采用现代标准记号)。将 $Q$ 的坐标代入曲线方程,并忽略 $e$ 与 $a$ 的平方及更高次幂,仅保留一次项,即可得到比值 $e : a$。

后来,按照斯吕兹的建议,这一比值的倒数 $a/e$ 被称为该点切线的角系数。
