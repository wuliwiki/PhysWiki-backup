% 高斯绝妙定理
% Theorema Egregium|高斯曲率|Gaussian curvature|微分几何

\pentry{黎曼联络\upref{RieCon}}

本节我们来讨论高斯绝妙定理.

在最初的古典微分几何研究中,常常需要将流形理解为某个$\mathbb{R}^n$空间中的超平面,进行具体的、复杂的计算,从而得到其性质.比如说,很多二维的流形都可以表示为三维空间中的一个曲面,比如球、平面、双曲面等等;也有的二维流形没法在三维空间中表示,比如Klein瓶,但是在四维空间中一定可以表示的\footnote{题外话:任意$n$维实流形,都可以嵌入到$\mathbb{R}^{2n}$中.}.

高斯第一个发现“曲率”这一\textbf{内蕴}量,并把该发现命名为\textbf{绝妙定理(Gauss's Theorema Egregium)}\footnote{注意“绝妙定理(Theorema Egregium)”是拉丁语.}.连高斯都觉得绝妙的发现到底是什么呢?这就需要解释何为“内蕴”了:它是指,曲率的计算不依赖于具体的嵌入、图等具体表示,而是流形本身具有的性质.这就逐渐进入了现代微分几何更为抽象但优雅的大门了.

当然,我们这里不会使用高斯一开始的语言去描述该定理,而是用更为现代的语言.毕竟我们是在学习数学,而非数学史,站在巨人的肩膀上当然是更合适的.

\subsection{一点预备}

首先是一个引理:

\begin{lemma}{曲面上的黎曼联络}\label{Egreg_lem1}
$\mathbb{R}^n$中的曲面$M$上的仿射联络$\nabla$,就是$\mathbb{R}^n$的方向导数$D$的投影:
\begin{equation}
\nabla_XY=\opn{pr}(D_XY)
\end{equation}
其中$\opn{pr}$表示\textbf{投影(projection)}映射.我们也可以表示为$\opn{pr}(D_XY)=(D_XY)_\perp$.
\end{lemma}

引理较为直观,证明留给读者,只需要挨个验证黎曼联络所需要满足的线性性和Leibniz律即可.

接下来是两个有用的式子.

\begin{theorem}{}
设$(M, \nabla)$是一个$\mathbb{R}^3$中的子黎曼流形,且$X, Y, Z\in\mathfrak{X}(M)$.则我们有以下定理:
\begin{enumerate}
\item \textbf{(高斯曲率方程)}$R(X, Y)Z=\ev{L(X), Z}L(Y)-\ev{L(Y), Z}L(X)$;
\item $\nabla_XL(Y)-\nabla_YL(X)=L([X, Y])$
\end{enumerate}
\end{theorem}

以下分别证明这两个式子.

\subsubsection{第一个式子的证明}

回忆形状算子$L$的定义:对于$\mathbb{R}^3$中一曲面上的\textbf{切向量场}$X$,记$N$为该曲面的\textbf{单位法向量场},那么$L(X)=-D_XN$,其中$D$为$\mathbb{R}^3$中的\textbf{方向导数},也即其天然的仿射联络.

由于$X, Y$和$N$垂直,因此我们有$\ev{L(X), Y}=\ev{D_XY, N}$.

考虑到$\abs{N}\equiv 1$,因此$(D_XY)_\perp=\ev{Y, L(X)}N$;又由\autoref{Egreg_lem1} ,$\nabla_XY=(D_XY)_\parallel=D_XY-(D_XY)_\perp$.

于是我们有:
\begin{equation}
D_XY=\nabla_XY+\ev{D_XY, N}N
\end{equation}

于是:
\begin{equation}
\begin{aligned}
D_XD_YZ&=D_X\nabla_YZ+D_X(\e)
\end{aligned}
\end{equation}



\subsubsection{第二个式子的证明}






















