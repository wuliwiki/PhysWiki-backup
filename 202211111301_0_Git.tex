% Git 笔记
% keys Git|GitHub

\pentry{GitHub Desktop 的简单使用\upref{GitHub}}

一个简单美观的 GUI 客户端是 GitHub Desktop\upref{GitHub}(未必需要使用 GitHub), 然而许多 GUI 的功能比起命令行来还是要少得多, 例如 GitHub Desktop 不支持 \verb|stage|, 不支持把仓库或指定文件恢复到以前某个版本等. 要安装 Git 命令行程序, Windows 用户可以下载 \href{https://gitforwindows.org/}{Git for Windows}, Debian 或 Ubuntu 可以直接用 \verb|sudo apt install git| 安装.Git for Windows 的 Git Bash 相当于 linux 的命令行, 有当前路径 \verb|pwd|, 更改路径用 \verb|cd|.

\subsection{新安装设置}
\begin{itemize}
\item 要设置用户名, 用 \verb|git config --global user.name "用户名"|.
\item 要设置 commit 邮箱, 用 \verb|git config --global user.email "邮箱"|  邮箱决定 commit 的作者.
\item Github 的 token 是命令行使用的密码, 和 GitHub 的密码是不同的, GitHub 创建 token 参考\href{https://docs.github.com/en/github/authenticating-to-github/keeping-your-account-and-data-secure/creating-a-personal-access-token#creating-a-token}{这里}.
\item 要查看 token 登录 github 后用 \href{https://github.com/settings/tokens/}{https://github.com/settings/tokens/}. 注意每个 token 只能查看一次, 要注意保存, 如果丢失只能新建一个.
\item 私有仓库用网址如果 clone, 每次连接服务器需要输入用户名和 token, 包括 \verb|clone, pull| 等. 要让 Git 记住账户密码, 用 \verb|git config --global credential.helper store| 如果 github 修改了密码, 再用一次该命令即可, 要测试是否成功可用 \verb|git pull|, 如果成功则不需要密码.
\item 清空账号密码用 \verb|git config --global --unset credential.helper|. 如果还不行的话只好把下文的所有 config 文件删除并重装 git.
\item Windows 下设置 \verb|git config --global core.autocrlf true|, 这样 commit 的时候会把所有 CRLF 换行变为 LF, 而 checkout 的时候会把所有 LF 变为 CRLF. Linux 下设置 \verb|git config --global core.autocrlf input|, 这样 commit 的时候会把所有 CRLF 变为 LF, 而 chekout 的时候不做任何操作. 如果这么设置了, GitHub Desktop 自动产生的 .gitattribute 文件就可以删掉了 (因为解决的是同一个问题).
\item 如果仓库中的中文路径显示乱码(反斜杠加数字), 用 \verb|git config --global core.quotepath off|
\item 如果有时候在 windows 上 pull 提示文件名太长, 用 \verb|git config --global core.longpaths true|
\item 如果在 windows git bash 上 pull 显示 invalid path, 用 \verb|git config core.protectNTFS false|. 但这么做可能仍会有一些问题(例如 windows 的文件夹中最后的 . 会自动删除)
\item \verb|git config --global core.editor "vim"| 可以把用于 commit 的编辑器设为 vim
\item on linux, don't show any safe directory warning: \verb|git config --global --add safe.directory "*"|
\item 为了防止大文件传服务器失败, 使用例如 \verb|git config --global http.postBuffer 2000000000| (字节)
\end{itemize}

\subsection{基础}
\begin{itemize}
\item \verb|git init| 可以把某个文件夹变为仓库, 也可以现在 GitHub 上新建仓库再 \verb|git clone| 到本地.
\item \verb|git status| 用于查看仓库的当前状态, 例如有哪些新增、 改变和删除的文件.
\item \verb|git diff [文件或路径]| 用于显示上次 \verb|commit| 之后的改变, 如果有文件被 \verb|add|, 则显示其 \verb|add| 之后的改变.
\item \verb|git add [文件或路径]| 用于把改变的文件添加到 staging area, commit 的时候只会 commit 这些文件. 如果文件又发生了改动, 需要再次 \verb|git add|.
\item 快速 add 并 commit 可以用 \verb|git commit -am "标题"|
\item 考虑到 \verb|git status| 和 \verb|git diff| 太常用, 可以用 \verb|ctrl+r stat| 或者 \verb|ctrl+r dif| 等. 如果不对就再按一次 \verb|ctrl+r| 即可. 这是命令行的功能不是 Git 功能.
安装后的设置
\item \verb|git fsck| 可以检查数据库是否出现错误(硬盘坏点等)
\end{itemize}

\begin{itemize}
\item 要把某个路径变为一个 repository, 先 cd 到该路径, 然后 \verb|git init| 即可. 该命令会在当前目录创建 .git 文件夹, 里面包含所有历史信息.
\item \verb|git clone| 时在后面加上 \verb|--depth N| 选项可以只下载最后 \verb|N| 次的 commit
\item \verb|git clone <url> newname| 会把本地 repo 文件夹命名成 \verb|newname|
\item git 会无视隐藏目录(\verb|.| 开头的), \verb|git status| 也不显示, \verb|git clean| 也删不掉
\item \verb|git status| 只会显示 worktree 和 index 之间的区别, 而不一定是最新 commit 之间的(对于服务器上被 push 的 repo 而言)
\item \verb|git status 目录| 可以只显示某个目录的 status
\item 每个 repo 的设置文件是 \verb|.git/config|. 用户的 \verb|.gitconfig| 文件可以在 home 目录 (~) 找到 (对当前用户生效), \verb|git config --global| 的设置全部存在这里. 系统的设置文件可以在 linux 的 \verb|/etc/gitconfig| 或 Windows 的 \verb|/C/ProgramData/Git/config| 中找到 (对所有用户生效). repo 中的设置覆盖用户设置覆盖系统设置.
\item \verb|git config --list| 可以列出当前目录的所有 config, 注意如果有设置多次出现, 后出现的为准. \verb|git config --show-origin <xxx.xxx>| 可以显示某个 config 所在的文件.
\item \verb|.gitattributes| 文件也有用户(而不是 repo) 的版本, 可以用 \verb|~/.config/git/attributes| 文件设置.
\item 所有 config 和 attribute 文件对 GitHub Desktop 同样有效.
\item 要从服务器 clone 一个 Repository, 先获取网址, 然后用 \verb|git clone https://...| 即可. git 会 clone 到当前目录下.
\item 如果 clone 了以后用 \verb|git status| 发现所有的文件都改变了, 用 \verb|git config --global core.fileMode false| 即可. 如果还是不行就去掉 \verb|--global|.
\item \verb|git clone <folder1> <folder2>| 把 folder1 中的 repo 复制到 folder2 中. 但注意路径要用 \verb|/| 而不是 \verb|\|. 如 \verb|/C/Users/addis/Documents/GitHub/cn2D/|
\item \verb|git clone --no-checkout ...| 选项可以只复制 .git 文件夹.
\item Git bash 第一次 push 到 GitHub 的时候会弹出登录框, 密码被存在 Windows 的 Credential Manager 中的 git:https://github.com.
\item 检查当前目录的 repository 的状态, 用 \verb|git status|. 
\item 用 \verb|git add <file>| 把变化添加到 staging area (也就是 index). 用 \verb|git add <dir>/| 添加目录, 用 \verb|git add .| 添加所有变化.
\item 用 \verb|git add -u| 可以只 add 修改的和删除的文件, 新文件不 add
\item \verb|.gitignore| 的例子: \verb|folder/| 忽略整个文件夹, \verb|folder/*.txt| 忽略文件夹中的所有 txt 文件, \verb|folder/**/*.pdf| 忽略 folder 文件夹和所有子文件夹中的 pdf 文件.
\item \verb|git diff| 用于显示文件夹和 stage area 中的区别. \verb|git diff --staged| 用于显示 stage area 和上一次 commit 的区别. \verb|git diff --name-only| 可以只显示文件名.
\item \verb|git diff| 的 \verb|--word-diff| 选项可以显示行内的细微差别, 当行比较长的时候使用(如 latex 段落). github 中有时候也不会显示具体位置. 如果还想更精确, 用 \verb|--word-diff-regex=.|
\item \verb|git commit| 用于自动打开文本编辑器信息并 commit. \verb|git commit -m "message"| 直接把 message 作为 commit 信息. \verb|git commit -a| 可以自动 git add 所有变化的文件, 也可以在后面再加 \verb|-m "message"|.
\item commit 时编辑器中第一行是 title, 空一行, 然后写 description
\item 要删除一个文件, 用 \verb|git rm <file>|. 这样做相当于手动删除文件, 再 stage 该变化. 如果文件已经 stage 了, 用 \verb|git rm -f <file>|.
\item 要重命名文件, 用 \verb|git mv <file_from> <file_to>|, 相当于 \verb|git rm <file_from>| 再 \verb|git add <file>|. (在 Windows 中, 要改变文件名的大小写必须这么做而不能直接给文件重命名)
\item 要浏览所有 commit 历史, 用 \verb|git log|. 要显示每次 commit 的 diff, 用 \verb|git log -p|, 要只显示前两次, 用 \verb|-2|. 用 \verb|--stat| 可以显示基本统计(几个+几个-).
\item \verb|git log ... --name-only| 只显示改变的文件名
\item \verb|git log -- <filename>| 可以只显示一个一个文件的历史, \verb|-p| 和 \verb|--word-diff-regex| 同样有效
\item 要搜索 commit 的标题和描述, 用 \verb|git log --grep=<pattern>|
\item 要浏览某个文件的历史, 用 \verb|git log -- <filename>|, 
\item \verb|git commit --amend| 可以覆盖上一次 commit (如果上一次 commit 忘了做什么事情).
\item \verb|git reset| 可以 unstage 所有文件, \verb|git reset HEAD <file>| 可以 unstage 一个文件.
\item \verb|git checkout -- <file>| 可以撤销一个文件从上次 commit 后的变化 (危险操作!). \verb|git checkout .| 可以撤销所有文件从上次 commit 后的变化 (危险操作!) 但是他们不会影响 untracked files
\item 注意 \verb|git checkout folder| checkout 的是 index, 也就是 git add 保存的地方. 而对于服务器上被 push 的 repo, index 和 HEAD 的版本很可能是不同的. 这时用 \verb|git checkout master -- ...| 的才是最新版本.
\item \verb|git checkout <hash> -- <file1> <file2>...| 可以将某个文件还原到某次 commit 的状态. 同样也适用于文件夹.
\item \verb|git --work-tree=/path/to/outputdir checkout ...| 可以指定 checkout 的路径, 对 \verb|status| 等命令同样有效
\item \verb|git --git-dir /path/to/repo 命令| 可以指定 \verb|.git| 所在的文件夹而无需先 cd 到那个文件夹
\item 【不推荐】\verb|GIT_WORK_TREE=/some/path/ git checkout ...| 可以把任何 \verb|git checkout| 的内容输出到指定文件夹, 注意这里的 \verb|GIT_WORK_TREE=/some/path/| 只有本次有效
\item 【不推荐】(建议使用上一条) \verb|git checkout-index -a -f --prefix=/some/path/| 可以把所有最新版本的所有内容从 \verb|.git| 文件夹 checkout 到指定的文件夹, 注意最后的 \verb|/| 是必须的, \verb|-a| 表示全部文件, \verb|-f| 表示覆盖已有文件. 如果要 checkout 某个文件, 在后面用 \verb|-- 文件名| 即可, 要把 \verb|-a| 去掉. 这种方法不能 checkout 某个文件夹
\item \verb|git mv <old_name> <new_name2>| 可以把某个文件重命名, 包括只改变大小写(windows 中也可以, 直接在文件夹中改变大小写是无效的)
\item \verb|git grep| 可以在仓库中快速搜索关键词, \verb|grep --exclude-dir='.git'| 则往往较慢. 注意 \verb|git grep| 不能搜到没有 \verb|add| 的文件, 除非加了 \verb|--no-index|.
\item \verb|git remote| 显示当前 repository 的 remote server shortname. 要显示网址, 用 \verb|-v|.
\item \verb|git remote add <shortname> <url>| 添加一个 remote server (一个 repo 可以有多个 server).
\item \verb|git fetch <shortname>| 从服务器上下载本地没有的数据.
\item \verb|git pull| 的功能是 fetch 然后 merge.
\item \verb|git remote show <shortname>| 显示 remote 的信息.
\item \verb|git remote rename <oldname> <newname>| 可以修改服务器的名字.
\item \verb|git remote remove <shortname>| 删除和服务器的连接 (并不会删除服务器上的 repo)
\item \verb|git push| 可以把当前 branch 的变化上传到服务器.
\item 如果从服务器上 \verb|git clone| 的是某个空 project, 第一次 commit 后上传用 \verb|git push -u origin master|
\item \verb|git checkout <id>| 可以把工作区间的所有文件修改到某次 commit 的状态, 且不会影响当前的任何 branch. 这时会进入 'detached HEAD' 状态, 所做的任何修改和 commit 在退出该状态后都会消失.
\item \verb|git checkout master| 可以从 \verb|git checkout <id>| 中恢复
\item \verb|git reset --hard| 用于把 working directory 和 staging area 还原到最近一次 commmit 的状态.
\item \verb|git reset HEAD~1| 等效于 \verb|git reset --mixed HEAD~1|, 用于撤销最近一次 commit,文件夹中的文件不改变,改变的文件变为 changed not staged 的状态
\item \verb|git reset --soft HEAD~1| 用于撤销最近一次 commit,改变的文件变为 staged 的状态
\item \verb|git reset --hard HEAD~1| 用于撤销最近一次 commit (本地). 如果已经推送到了服务器, 继续用 \verb|git push origin HEAD --force| 即可.
\item \verb|git stash| 可以临时保存修改(不包括 untracked 的文件) \verb|git stash -u| 包括 untracked 的文件. \verb|git stash pop| 可以恢复这些修改(相当于 merge). 注意 \verb|pop| 以后的文件并不一定和 stash 的时候一模一样, 例如 stash 之后 pop 之前 pull 了一些变化, 那么 pop 的时候相当于在这些变化的基础上 merge.
\item 把某文件从所有历史中删除用 \verb|git filter-branch -f --tree-filter 'rm -rf <path/file>' HEAD| 注意这样做本身并不一定会使 repo 变小. 如果要清空一个文件夹, 就用 \verb|path/*| 就好
\item \verb|git gc --aggressive --prune=now| 会可以将 repo 缩小,PhysWiki 的 .git 大概能缩小到一半. 鼓励经常使用 \verb|git gc|,但 \verb|--aggressive| 选项会比较慢但也是安全的. gc 表示 garbage collection. 该命令还可以让 \verb|.git| 中的文件数量大大减小(copy 文件夹会快很多)
\item \verb|git gc| 命令只对本地 repo 有效,其实最有效的貌似还是直接把本地 repo 删了再 clone 一次.但是注意检查是否有不小心被 .gitignore 屏蔽的文件没有在服务器上
\item 即使是二进制文件, git 也会尽量储存每个版本的增量, 而不是每个版本储存一次.
\item \verb|git clean -f| 可以清除 untracked file, \verb|-d| 选项可以清除 untracked directory,  \verb|git clean -n| 显示将要清除的文件.
\item 要让 windows 中 git bash 支持中文, 在右键 option > text 中 local 选择 C, character set 选 UTF8, 另外输入以下命令
\verb|export LESSCHARSET=utf-8|
\verb|git config --global core.quotepath false|     显示 status 编码
\verb|git config --global gui.encoding utf-8|    图形界面编码
\verb|git config --global i18n.commit.encoding utf-8|   提交说明编码
\verb|git config --global i18n.logoutputencoding utf-8| 输出 log 编码
在 linux 系统中, 只需要一条设置, 就是
\verb|git config --global core.quotepath false|
\item 用 \verb|gitk| 可以弹出 GUI 显示每一次 commit 的文件目录, 不需要 checkout 就可以看到.
\item \verb|git push| 到一个 nonbare repo 一般是不允许的, 除非这个 repo 设置了 \verb|git config receive.denyCurrentBranch warn| 或者 \verb|git config receive.denyCurrentBranch ignore.| 这时如果 push, 并不会改变 remote 的 working directory. (其实这正是我增量备份 .git 需要的!)
\item windows 文本编辑器默认是 ANSI 编码, 而 ANSI 在 git 中显示的是乱码. 而 Visual Studio Code 保存的时候默认会存成 UTF-8 编码, ANSI 编码在 Visual Studio Code 中也会是乱码, 这时千万不要保存, 否则就成了永久的乱码了.
\item 要检查编码类型, 或转换编码类型,用 Visual Studio Code.  不要用 Windows 的 Notepad.
\end{itemize}

\subsection{分支}
\begin{itemize}
\item \verb|git checkout <branch> -- <file1> <file2>...| 可以把某个 branch 的文件还原到当前 branch.
\item \verb|git pull| 可以从服务器下载当前 branch 的变化.
\item \verb|git fetch <remote> <branch>| 可以下载一个指定的 branch.
\item \verb|git diff branchA branchB -- <filename>| 用于比较两个 branch 中同一个文件.
\item \verb|git checkout <branch> -- <filename>| 可以从别的 branch 里面 checkout 一个文件到当前 branch.
\item \verb|git branch <name>| 创造一个 new branch.
\item \verb|git show-branch| 可以查看已有的 branch 
\item \verb|git checkout <branch>| 把文件夹中的文件切换成某个 branch 中的文件. 但是新的修改和 staging area 中的修改也会移植到该 branch 中. 如果不想移植, 就先 commit.
\item \verb|git checkout <name>| 也可以用于从服务器上下载本地不存在的 branch.
\item \verb|git merge <branch>| 把某个 branch 并入到当前 branch 中.
\item 如果 merge 的过程中出现 conflict, 用 \verb|git merge --abort| 可以恢复到 merge 以前.
\item 只要两个 branch 修改的不是同一个文件的同一行, merge 都不会有问题. 如果是, 就会产生 conflict
\item \verb|git branch -m 新名字| 重命名当前的 branch.
\item \verb|git branch -d <name>| 用于删除某个 branch.
\item \verb|git branch --set-upstream-to=<remote>/<branch>| 用于设置 pull 和 push 的默认 remote 和 branch
\item \verb|git push <remote> --delete <branch>| 用于删除 server 上的 branch.
\item \verb|git push --set-upstream <server> <branch>| 用于把某个 branch 推送到 server 上并 track.
\item \verb|git branch -vv| 可以检查当前的默认 remote/branch, 用于 \verb|push/pull|
\item \verb|git branch --set-upstream-to <remote>/<branch>| 可以将某个 remote 作为默认
\item \verb|git branch <name> <id>| 用于从指定的 commit 创建 branch.
\item \verb|git clone -b <branch> --single-branch <url>| 用于从 url 克隆一个指定的 branch.
\item \verb|git push| 后面可以加某次 commit 的 hash 和 branch 如 \verb|abcd:master|, 或者如 \verb|HEAD~1:master|.
\end{itemize}
