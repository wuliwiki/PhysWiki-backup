% 指数衰减
% keys 指数函数|放射|衰变
% license Xiao
% type Tutor

\begin{issues}
\issueDraft
\end{issues}

\pentry{一阶线性微分方程\upref{ODE1}}

指数衰减是十分常见的一种现象,例如在以下问题中:

一种原子核发生 $\alpha$ 或 $\beta$ 等衰变后成为另一种原子核。
这种转变对同一种核素的许多原子核并非同时发生,有的先衰变,有的后衰变。放射性物质中的原有核素的数量随时间推移逐渐减少,而整体的衰变速率与该时刻的衰变前的核素数量成正比。
所以任何放射性物质在单独存在时都遵从指数衰减规律式。

即:\footnote{参考 Wikipedia \href{https://en.wikipedia.org/wiki/Exponential_decay}{相关页面}。}衰变的速率和总量成正比, 或者说每个粒子单位时间衰变的概率一样。

指数衰减的常见原因正是由于其微分方程形式的常见性:
$$y' = \alpha y ~.$$

下面进行推导:
\begin{equation}
\dv{N}{t} = -\lambda N~.
\end{equation}
方程的解
\begin{equation}
N(t) = N_0 \E^{-\lambda t}~.
\end{equation}
将半衰期 $T_h$ 定义为,原有这种粒子的一半发生衰变所需要的时间。 满足
\begin{equation}
N(T_h) = \frac{N_0}{2} \iff \E^{-\lambda T_h} = \frac{1}{2}~,
\end{equation}
解得
\begin{equation}
T_h = \frac{\ln 2}{\lambda}~.
\end{equation}
