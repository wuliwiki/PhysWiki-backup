% 一维薛定谔方程失败的差分法数值解(Matlab)

\begin{issues}
\issueDraft
\end{issues}

\pentry{薛定谔方程(单粒子一维)\upref{TDSE11}, 一维波动方程的简单数值解(Matlab)\upref{W1dNum}}

本文使用原子单位制\upref{AU}. 本文演示如何用差分法解一维薛定谔方程.
\begin{equation}\label{TDSE1N_eq1}
-\frac{1}{2m}\pdv[2]{x}\psi + V(x,t)\psi = \I\pdv{t}\psi
\end{equation}

乍看之下, 类比 “一维波动方程的简单数值解(Matlab)\upref{W1dNum}”, 我们似乎也能用简单的差分法求解\autoref{TDSE1N_eq1}, 但事实证明这是行不通的, 迭代几步以后, 数值误差就会盖过波函数本身. 但为了教学,还是把公式和代码给出来. 一些改进算法如 Crank-Nicolson 算法解一维含时薛定谔方程\upref{CraNic}, 以及 Lanczos 算法\upref{Lanc}.

把波函数取离散值, 令 $\psi_{i,n} = \psi(x_i,t_n)$. 用有限差分表示二阶导数(\autoref{DerDif_eq5}~\upref{DerDif}), 得
\begin{equation}
-\frac{1}{2m}\frac{\psi_{i-1,n} - 2\psi_{i,n} + \psi_{i+1,n}}{\Delta x^2} + V_{i,n}\psi_{i,n} = \I \frac{\psi_{i, n+1} - \psi_{i, n}}{\Delta t}
\end{equation}
于是
\begin{equation}
\psi_{i, n+1} = \frac{\I\Delta t}{2m\Delta x^2} (\psi_{i-1,n} - 2\psi_{i,n} + \Delta t\psi_{i+1,n}) + (1-\I \Delta t V_{i,n})\psi_{i,n}
\end{equation}

失败的差分法程序如下
\begin{lstlisting}[language=matlab]
% 差分法解一维薛定谔方程

function TDSE_1d_simple
% ==== 参数设置 ======
m = 1; % 质量,角频率
xmin = -10; xmax = 30; Nx = 1000; % x 网格
tmin = 0; tmax = 10; Nt = 300; % 时间网格
Nplot = 1; % 画图步数
t0 = 0; % 高斯波包的初始时间
p0 = 5; % 初始动量
x0 = 0; sig_x = 2; % 初始位置, 波包宽度
V = @(x,t) zeros(size(x));
% ===================
close all;
psi_gs = @(x) 1/(2*pi*sig_x^2)^0.25/sqrt(1 + 1i*t0/(2*m*sig_x^2))...
      *exp(-(x-x0-p0*t0/m).^2/(2*sig_x)^2/(1 + 1i*t0/(2*m*sig_x^2)))...
      .*exp(1i*p0*(x-p0*t0/(2*m)));
x = linspace(xmin, xmax, Nx); dx = (xmax-xmin)/(Nx-1);
t = linspace(tmin, tmax, Nt); dt = (tmax-tmin)/(Nt-1);
psi = psi_gs(x);
figure; plot(x, real(psi)); hold on;
plot(x, imag(psi));
for it = 1:Nt
    d2psi = [0, psi(1:end-2) - 2*psi(2:end-1) + psi(3:end), 0];
    psi = 1i*dt/(2*m*dx^2)*d2psi + (1 - 1i*dt*V(x,t(it))).*psi;
    psi(1) = 0; psi(end) = 0;
    if mod(it, Nplot) == 0
        clf;
        plot(x, real(psi)); hold on;
        plot(x, imag(psi)); axis([xmin,xmax,-0.5,0.5]);
    end
end
end
\end{lstlisting}
