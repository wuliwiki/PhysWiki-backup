% 吉布斯自由能
% 吉布斯自由能|吉布斯函数化学势|自由焓

\begin{issues}
\issueOther{建议创建单独的化学势词条}
\end{issues}

\pentry{热力学第一定律\upref{Th1Law}, 热力学第二定律\upref{Td2Law}}

\begin{definition}{吉布斯自由能}
吉布斯自由能(也称自由焓)是一个热力学态函数,常用字母 $G$ 表示,定义为
\begin{equation}
G=H-TS=U+pV-TS
\end{equation}
\end{definition}

对于一个系统,虽然原则上可以定义它的吉布斯自由能,但其物理意义不是很明晰.只有当系统等温等压变化时,前后吉布斯自由能的变化量才有物理意义,见下.

\subsection{吉布斯能判据}
可以与亥姆霍兹自由能\upref{HelmF}类比,亥姆霍兹自由能所分析的系统是恒温的封闭系统.而此处,我们用吉布斯自由能分析\textbf{恒温恒压}下封闭的系统(封闭的意思是,没有物质进出于系统).

\begin{theorem}{}
\textbf{对于一个恒温恒压过程,系统对外做的非体积功\footnote{非体积功是除去 $p\dd V$ 外其余形式的做功,例如电场、磁场做功$-H\dd M$,$-E\dd P$ 等.}小于(不可逆过程)或等于(可逆过程)吉布斯自由能减小量}.
\begin{equation}
\delta w \le -\Delta G
\end{equation}
\end{theorem}
\subsubsection{推导}

由亥姆霍兹自由能判据\upref{HelmF},得$\delta w \le -dA\Rightarrow \delta w \le -d(U-TS)$

将 $w$ 分为体积功 $pdV$ 与非体积功 $w'$ 两部分,即
$\delta w' + pdV \le -d(U-TS)$

即 $\delta w'  \le -d(U-TS+pV) = -d(H-TS) = -\Delta G$
\subsubsection{吉布斯自由能与过程的自发性}
当我们假设系统与外界并没有非体积功形式的能量交换时,$\delta w=0$,因此一个自发的热力学过程中,$\Delta G\le 0$.我们自然而然地有以下推论:
\begin{corollary}{}
\textbf{等温等压条件下,只有吉布斯自由能降低的过程能自发发生},即
\begin{equation}
\Delta G \le 0
\end{equation}
\end{corollary}
这个推论非常有用,可以帮助我们评估一个热力学过程的自发性.例如,化学实验通常是在等温等压状况下进行,由此我们可以判断化学反应会朝着吉布斯能减少的方向自发进行.

除此以外,还可以用吉布斯自由能判据得到等温等压系统处于稳定平衡状态的条件,系统不能以任何方式的变动引起自身的吉布斯自由能的下降,从而它将维持在这一恒定的状态.这可以总结为以下推论:

\begin{corollary}{}
反之而言,等温等压系统处在稳定平衡状态的必要和充分条件为:对于任意系统可能发生的虚变动过程,其吉布斯自由能的改变量 $\Delta G$都满足
\begin{equation}
\Delta G>0
\end{equation}
或写成一阶变分和二阶变分的形式,即
\begin{equation}\label{GibbsG_eq2}
\delta G=0,\quad \delta^2 G>0
\end{equation}
\end{corollary}

吉布斯能判据是物理化学中最为重要的结论之一.


\subsection{化学势}

对于粒子数固定为 $N$ 的封闭系统,$\dd G=-S\dd T+V\dd p$.对于单相物质,自由焓是广延量,每个分子占有的自由焓应该相等(忽略表面张力导致的表面能).我们称化学势为
\begin{equation}
\mu_{\text{单个粒子}}(p,T)=\frac{G(p,T)}{N}
\end{equation}

有时也用摩尔吉布斯函数来代表化学势,一般用符号 $\mu$ 或 $G_m$ 来表示:

\begin{equation}\label{GibbsG_eq1}
\mu(p,T)=G_m(p,T)=\frac{G}{\nu}
\end{equation}

\subsection{理想气体的吉布斯函数}
利用理想气体熵的表达式(\upref{MacroS}\autoref{MacroS_eq2}),我们有理想气体摩尔熵公式:
\begin{equation}
S_m=\int \qty(\frac{C_{p,m}}{T}\dd T-\frac{R}{p}\dd p)+S_{m,0}
\end{equation}
我们知道理想气体的摩尔焓可以表示为
\addTODO{缺少焓的词条}
\begin{equation}
H_m=\int C_{p,m}\dd T+H_{m,0}
\end{equation}
再利用摩尔吉布斯函数的热力学公式 $G_m=H_m-TS_m$,可以得到理想气体吉布斯函数的表达式
\begin{equation}
\begin{aligned}
G_m&=H_m-TS_m\\
&=\int C_{p,m} \dd T - T\int C_{p,m} \frac{\dd T}{T}+\int \frac{RT}{p}\dd p +H_{m,0}-TS_{m,0}
\\
&=\int C_{p,m} \dd T - T\int C_{p,m} \frac{\dd T}{T}+RT\ln p+G_{m0}
\end{aligned}
\end{equation}
可以写成
\begin{equation}\label{GibbsG_eq3}
\begin{aligned}
&G_m=RT(\phi(T)+\ln p)\\
&\mu_{\text{单个粒子}} = kT(\phi(T)+\ln p)
\end{aligned}
\end{equation}

这个表达式在研究化学反应时非常重要.我们也用符号 $\mu$ 来表达摩尔吉布斯函数 $G_m$,称它为化学势.

如果具体地把 $\phi(T)$ 写出来,那么有
\begin{equation}\label{GibbsG_eq4}
\begin{aligned}
\phi(T)&=\frac{1}{RT}\int C_{p,m}\dd T-\frac{1}{R}\int C_{p,m}\frac{\dd T}{T}+\frac{G_{m,0}}{RT}\\
&=-\int \frac{\dd T}{RT^2}\int C_{p,m}\dd T+\frac{H_{m,0}}{RT}-\frac{S_{m,0}}{R}
\end{aligned}
\end{equation}

上式第一行到第二行用了分部积分公式.有时人们为了书写方便,用小写字母来表示摩尔的热力学量,例如 $h_0,c_p,s_0$ 表示每摩尔的焓常量、定压热容、熵常量.$\phi(T)$ 可写为
\begin{equation}
\phi(T)=-\int\frac{\dd T}{RT^2}\int c_{p}\dd T+\frac{h_{0}}{RT}-\frac{s_{0}}{R}
\end{equation}

\begin{example}{理想气体化学势的近似}
假设理想气体的温度在一个小范围内变动,那么其摩尔定压热容可近似为一个常量 $c_p$.求它的化学势近似表达式 $\mu(P,T)$.

根据 \autoref{GibbsG_eq4} 中的 $\phi(T)$ 可近似为(考察分部积分前的公式)
\begin{equation}
\phi(T)=-\frac{c_p\ln T}{R}+\frac{h_0}{RT}+\frac{c_p-s_0}{R}
\end{equation}
因此化学势 $\mu$ 为
\begin{equation}\label{GibbsG_eq5}
\mu(P,T)=-c_p\ln T+h_0+(c_p-s_0)T+RT\ln P
\end{equation}

\end{example}
