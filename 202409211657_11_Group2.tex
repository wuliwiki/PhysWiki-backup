% 群的同态与同构
% keys 同态|同构|商群|群同态基本定理
% license Xiao
% type Tutor

\pentry{正规子群\nref{nod_Group1}}{nod_fef5}

\subsection{同构}

让我们来观察两个群 $(\mathbb{Z}, +)$ 和 $(2\mathbb{Z},+)$。如果我们把 $2\mathbb{Z}$ 中的 $2$ 都看成 $1$,$4$ 都看成 $2$,以此类推,将 $2k$ 都看成 $k$,那么两个群的运算规则是一模一样的。比如说,$2\mathbb{Z}$ 中有 $2+4=6$,对应的是 $\mathbb{Z}$ 中 $1+2=3$ 的等式。

我们研究集合和群的时候,元素叫什么名字并不重要,重要的是元素之间是否相同以及运算规则是怎样的。那么,如果我们真的将 $2\mathbb{Z}$ 中的元素 $2k$ 都重命名为 $k$,它就和 $\mathbb{Z}$ 没什么区别了。所以在群的意义上,如果不考虑子群关系,单独把 $\mathbb{Z}$ 和 $2\mathbb{Z}$ 拿出来的时候,我们就认为它们是不可区分的,完全相同的两个群。

如果我们建立一个映射 $f:\mathbb{Z}\rightarrow2\mathbb{Z}$,定义为 $f(k)=2k$,那么这个 $f$ 就是一个双射,它在两个群的元素之间一一对应地建立了联系。这样,对于任意整数 $m, n$,有 $f(m)+f(n)=f(m+n)$,也就是说“先运算再映射”和“先映射再运算”结果是相同的。

类似地,对于任意的两个群 $G$ 和 $K$,如果存在一个\textbf{双射} $f:G\rightarrow K$,使得对于任意的 $x, y\in G$ 都满足 $f(x)f(y)=f(xy)$,那么这两个群的运算结构就是一模一样的。这时我们说这两个群是\textbf{同构(isomorphic)}的,而这个使得它们同构的双射就被称为 $G$ 和 $K$ 之间的\textbf{同构映射(isomorphic mapping)},也可以简称\textbf{同构(isomorphism)}这里加粗的两个“同构”,前者是形容词,后者是名词。

\begin{definition}{自同构}
称群到自身的同构为一个\textbf{自同构(automorphism)}\footnote{这个词是用词根“auto(自身的)”和单词“isomorphism(同构)”组合而成的。}。


群 $G$ 的全体自同构配合映射的复合,又构成一个群,称为 $G$ 的\textbf{自同构群},记为 $\opn{Aut}(G)$。


\end{definition}

由于同构使得两个群各方面表现一模一样,研究同构其实没有太大意义,我们甚至直接把同构的两个群看成同一个群,不管元素具体怎么命名的。有意思的结构,是以下定义的“同态映射”。

\subsection{同态}

同构映射是一个双射。如果把这个要求拿掉,我们就得到同态的概念:

\begin{definition}{同态映射}\label{def_Group2_1}
对于两个群 $G$ 和 $K$,如果映射\textbf{(不一定是双射)}$f:G\rightarrow K$ 使得 $\forall x, y\in G, f(x)f(y)=f(xy)$,那么称 $G$ 和 $K$ 是\textbf{同态(homomorphic)}的,称 $f$ 是\textbf{同态映射(homomorphic mapping)}或\textbf{同态(homomorphism)}。
\end{definition}

\begin{definition}{像和核}
沿用\autoref{def_Group2_1} 的设定。$K$ 中被映射到的元素构成的集合,称为 $f$ 的\textbf{像(image)},记作 $f(G)$。$G$ 中映射到 $K$ 的单位元 $e_K$ 的元素构成的集合,称为 $f$ 的\textbf{核(kernal)},记为 $\ker(f)$。
\end{definition}

注意,$f(G)\subset K$,$\ker(f)\subset G$。

同态的两个群,运算结构很相似但又不完全一样。在以上定义的例子中,$K$ 的行为就像是一个弱化版的 $G$,可能会丢失一些细节,但保留的方面和 $G$ 是一模一样的。这么说可能不够具体,我们用\autoref{exe_Group2_2} 和\autoref{exe_Group2_1} 来理解同态的“似而不同”。


\begin{exercise}{}\label{exe_Group2_2}
设两个群 $G$ 和 $K$,$f:G\rightarrow K$ 是一个同态,$x\in G$,求证 $f(x^{-1})=(f(x))^{-1}$。
\end{exercise}


\begin{exercise}{群同态基本定理}\label{exe_Group2_1}
设两个群 $G$ 和 $K$,$f:G\rightarrow K$ 是一个\textbf{满}同态。求证:
\begin{enumerate}
\item $\ker(f)$ 是 $G$ 的一个正规子群\footnote{这保证了群 $G/\ker(f)$ 存在。}。
\item 对于 $x_1, x_2, y_1, y_2\in G$,如果\footnote{即$x_1$与$x_2$模$\ker(f)$同余,$y_1$与$y_2$模$\ker(f)$同余。}$x_1^{-1}x_2\in \ker(f)$ 和 $y_1^{-1}y_2\in \ker(f)$,那么 $f(x)=f(x_1)$,$f(y)=f(y_1)$。
\item 由前两条的结论,证明可以用 $f$ 来导出一个映射 $f': G/\ker(f)\rightarrow K$ ,它是一个群同构。

\end{enumerate}
\end{exercise}

由\autoref{exe_Group2_1},同态的实质就是商群 $G/\ker(f)$ 和 $K$ 之间的同构。$G/\ker(f)$ 继承了 $G$ 的运算,但是由于把同余的元素全都当作同一个了,也就丢失了一部分细节。因此我们说同态的两个群也是“似而不同”的。

对于两个存在\textbf{满}同态关系的群,我们可以利用群同态定理,进一步找到结构上的相似之处。
\begin{theorem}{}\label{the_Group2_3}
设$f:G_1\to G_2$是满同态,$N=\opn{ker}f$,则
\begin{enumerate}
\item 若$B$是$G_1$的子群,则$f(B)$是$G_2$的子群。反之,若$K$是$G_2$的子群,则$f^{-1}(K)=\{x\in G_1\mid f(x)\in K\}$是$G_1$的子群\textbf{且$f^{-1}(K)\supseteq N$}。
\item 对于任意$H\supseteq N,f:H\to f(H)$建立了子群之间的一一对应;在该双射下,$H\lhd G_1$当且仅当$f(H)\lhd G_2$。
\item $H$同上设。$G_1/H\cong G_2/f(H)$。
\end{enumerate}
\end{theorem}
\textbf{证明:}
\begin{enumerate}
\item 设任意$f(x),f(y)\in f(B)$,则$f(x)f(y)^{-1}=f(xy^{-1})$。又因为$xy^{-1}\in B$,所以$f(xy^{-1})\in f(H)$。所以$f(B)$是$G_2$的子群。反之也同理易证$f^{-1}(K)$是$G_1$的子群\footnote{需要注意映射的逆和群元的逆的区别,$[f^{-1}(a)]^{-1}=f^{-1}(a^{-1})$。}。设$e'=f(e)$,由于$e'\in K$,所以$f^{-1}(e')=N\subseteq f^{-1}(K) $。
\item 
由前文讨论可知,$f(H)$必然是$G_2$的子群,因此$f^{-1}f(H)$必然是包含$N$的子群。所以$f$确实建立了包含$N$的$G_1$\textbf{子群集合}与$f(H)$构成的\textbf{子群集合}之间的映射。在该集合意义上,$f$是满射($f^{-1}f(H)\supseteq H,N$),所以接下来我们只需要证明这是单射即可。\\


假设这不是单射,即至少存在两个包含$N$的子群$A,B$使得$f(A)=f(B)$。对于任意$a\in A$,存在$b\in B$使得$f(a)=f(b)$,则$f(ab^{-1})=e'$,所以$ab^{-1}\in N\subseteq B$,则$a\in B, A\subseteq B$。反之,对于任意$b'\in B$,存在$a'\in A$使得$f(a'b'^{-1})=e'$,则$a'b'^{-1}\subseteq A$,则$b'\in A,B\subseteq A$,所以$A=B$,映射是单射;\\
下面证最后一个性质。设$H\lhd G_1$。对于任意$f(a)\in G_2$有$f(a)f(H)f(a)^{-1}=f(aHa^{-1})=f(H)$,因此$f(H)\lhd G_2$。反之由$f$是满同态,易证若$K\lhd G_2$,则$f^{-1}(K)\lhd G_1$。\\


\item 由上文知$f(H)\lhd G_2$。现设$f'=\pi \circ f:G_1\to G_2\to G_2/f(H)$\footnote{$\pi$为自然同态,即若$N\lhd G,\pi(g)=gN$,易见这是一个满同态。}。这是满同态的复合,因此$f'$也是满同态。因为$\opn{ker}f'=f^{-1}f(H)=H$,由群同态基本定理得证此条性质。
\end{enumerate}
应用\autoref{the_Group2_3} 在自然同态$\pi:G\to G/N$上,可以清楚看到,商群是如何继承原群结构的,具体如下图所示。

\begin{figure}[ht]
\centering
\includegraphics[width=8cm]{./figures/5801f65a98b86ce0.png}
\caption{正规子群的一一对应。红色圆代表正规子群$N$,所有圆都是群$G$关于$N$的陪集分解。紫色部分加上红色圆代表在群$G$意义上的正规子群$N_2$,$N_2\supseteq N$。黄色圆加上红色圆则代表在商集意义上的正规子群,即$f(N_2)\lhd G/N$。} \label{fig_Group2_2}
\end{figure}

\begin{corollary}{}
$G$是群,若$H,N$都是$G$的正规子群且$H\supseteq N$,我们有\footnote{对任意$h\in H$有$\pi(h)=hN$,$h_1N=h_2N$当且仅当$h_1h_2^{-1}\in H$}
\begin{equation}G/H\cong (G/N)/(H/N)~.\end{equation}
\end{corollary}


\subsection{内自同构和外自同构}



回顾线性代数中的知识:给定线性空间的基以后,线性变换和矩阵就一一对应(我们称之为给定基下用矩阵表示线性变换),而改变基以后同一个线性变换的基也会变。因此,群 $(\{\text{线性变换}\}, \text{映射的复合})$ 与群 $(\{\text{矩阵}\}, \text{矩阵乘法})$ 之间可以建立同构。这样的同构不是唯一的,而是依赖于基的选择。

不同的矩阵可以看成同一个线性变换在不同基下的表示,也可以看成两个不同的线性变换在同一个基下的表示。因此,我们可以用一个基来将矩阵对应到线性变换上,再用另一个基将线性变换对应到另一个矩阵上,由此就得到了矩阵之间的对应,这个对应就是矩阵乘法群到自身的同构。同一个线性变换在不同基之间的矩阵表示的关系是\textbf{相似},具体参见\enref{过渡矩阵}{TransM}小节。

由上段论述可知,给定可逆矩阵 $\bvec{Q}$,则矩阵到自身的映射 $f$ 是一个自同构,其中 $f(\bvec{M})=\bvec{Q}^{-1}\bvec{MQ}$。这提示我们一种构建群自同构的方法。

\begin{definition}{内自同构}
给定群 $G$。取 $g\in G$,定义映射 $\opn{Ad}_g:G\to G$ 如下:对于任意 $x\in G$,都有 $\opn{Ad}_g(x)=gxg^{-1}$。

称 $\opn{Ad}_g$ 是 $G$ 上的\textbf{内自同构(inner automorphism)},或者\textbf{共轭自同构(cogredient automorphism)}。


群 $G$ 的全体内自同构构成 $\opn{Aut}(G)$ 的一个子群,称为 $G$ 的\textbf{内自同构群}, 记为 $\opn{Inn}(G)$。

\end{definition}

\begin{theorem}{}\label{the_Group2_1}
内自同构群是自同构群的正规子群。
\end{theorem}

\textbf{证明}:

给定群 $G$,设 $f\in \opn{Aut}(G)$。任取 $g, x\in G$,则由\autoref{exe_Group2_2} 可得,
\begin{equation}\label{eq_Group2_1}
\begin{aligned}
f^{-1}\qty(gf(x)g^{-1})&=f^{-1}\qty(g)f^{-1}\qty(f(x))f^{-1}\qty(g^{-1})\\
&=f^{-1}\qty(g)xf^{-1}\qty(g^{-1})\\
&=f^{-1}\qty(g)x\qty(f^{-1}\qty(g))^{-1}
\end{aligned}~
\end{equation}

由 $x$ 的任意性,\autoref{eq_Group2_1} 意味着 $f\circ \opn{Ad}_g\circ f^{-1}=\opn{Ad}_{f^{-1}(g)}$。因此 $f\circ\opn{Inn}(G)\circ f^{-1}\subseteq\opn{Inn}(G)$,故 $\opn{Inn}(G)$ 是 $\opn{Aut}(G)$ 的正规子群。

\textbf{证毕}。

由\autoref{the_Group2_1},我们可以计算商群 $\opn{Aut}(G)/\opn{Inn}(G)$。

\begin{definition}{外自同构}
给定群 $G$,称 $\opn{Aut}(G)/\opn{Inn}(G)$ 为 $G$ 的\textbf{外自同构群(outer automorphism)},记为 $\opn{Out} (G)$。
\end{definition}

\begin{example}{不是内自同构的自同构}
取复数乘法群 $(\mathbb{C}, \times)$,则取共轭映射 $f(z)=\bar{z}$ 是其上一个自同构,但它显然不是内自同构。由于复数乘法的交换性,复数乘法群上的内自同构只有恒等映射一种。
\end{example}

对于任意群而言,内自同构群的结构是清楚的,由下述定理给出。这个定理其实是群同态基本定理的一个简单应用。

\begin{theorem}{}\label{the_Group2_2}
$\opn{Inn}(G)$同构于$G/C(G)$。
\end{theorem}

\textbf{证明}:

我们只需要给出$G$到$\opn{Inn}(G)$的一个满同态并证明其核为$C(G)$即可。

定义映射$f:G\to \opn{Inn} G$为$f(x)=\opn{Ad}_x$,容易知道这是个满射,又因为对任意$g\in G$,有
\begin{equation}
\begin{split}
(f(x)f(y))(g)&=(\opn{Ad}_x\opn{Ad}_y)(g)=\opn{Ad}_x((\opn{Ad}_y)(g))\\
&=xygy^{-1}x^{-1}=(xy)g(xy)^{-1}\\
&=\opn{Ad}_{xy}(g)\\
&=(f(xy))(g),
\end{split}~
\end{equation}
即$f(x)f(y)=f(xy)$,所以这是一个同态映射。

其核$\ker(f)=C(G)$,因为若$x\in\ker(f)$,则对任意$g\in G$,有$xgx^{-1}=g$.

所以由\autoref{exe_Group2_1}, $\opn{Inn}(G)$同构于$G/C(G)$。

\textbf{证毕}。

特别地,由于当$n\geq 3$时$C(S_n)=\{e\}$,故$\opn{Inn} S_n$与$S_n$同构。

相比内自同构群,自同构群的结构复杂得多。这里单就置换群给出结果。

\begin{theorem}{}
当$n\neq 2,6$时,置换群$S_n$的自同构群同构于自身。

$S_2$的自同构群是平凡群。

$\opn{Out} S_6$是2阶循环群。
\end{theorem}
