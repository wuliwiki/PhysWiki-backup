% 群的矩阵表示及实例
% keys 群表示|矩阵表示|群线性表示|忠实表示

\pentry{群表示\upref{GrpRep},群乘法表及重排定理\upref{groupt}}

在群表示一节中曾提到群元可与线性变换建立同态关系,由此可以给出群的线性表示。\textbf{矩阵表示}是一种特殊的有限维度表示 $(\mathbb{F}^n, \rho)$,其中 $\rho: G \to \opn{GL}(n; \mathbb{F})$\footnote{间一般线性群\upref{GL}},即将群元素表示成矩阵的形式。

对于一般的群表示 $(V, \rho)$,在向量空间 $V$ 中选取合适的基后,我们有同构 $\opn{GL}(V) \cong \opn{GL}(\mathbb{F}^n) = \opn{GL}(n; \mathbb{F})$,因此我们可以把它转化成一个矩阵表示。

\begin{example}{循环群$C_n$的一维复表示}\label{gprep_ex1}
对于群 $C_n = \{e, a, a^2, \dots, a^{n - 1}\}$ 而言,有表示 $D: C_n \to \opn{GL}(1; \mathbb{C}) = \mathbb{C}^*$,定义为$D(e)=1$,$D(a^m) = \mathrm{e}^{\frac{2\pi i}{n}m}$。显然,矩阵群$D(C_n) = \{\mathrm{e}^{\frac{2\pi i}{n}m} \in \mathbb{C}^* \mid m \in \mathbb{Z}\}$乘法关系与$C_n$乘法关系相同。
\end{example}

\begin{example}{$C_n$群的二维实表示}\label{gprep_ex2}
对于群$C_n$而言,有表示 $D: C_n \to \opn{GL}(2; \mathbb{R})$,满足

$D(e)=\begin{pmatrix}
 1 & 0 \\
 0 & 1
\end{pmatrix}$,
$D(a^m)=\begin{pmatrix}
 \cos{\frac{2\pi m}{n}} & -\sin{\frac{2\pi m}{n}}\\
 \sin{\frac{2\pi m}{n}} &\cos{\frac{2\pi m}{n}}
\end{pmatrix}$

这实际上是在平面上转动$\frac{2\pi m}{n}$角所对应的旋转矩阵,这与$C_n$群的几何含义相符,通过几何含义给出表示矩阵也是我们常用的方法。
\end{example}

\begin{definition}{平凡表示}
\textbf{平凡表示}(trivial representation)指的是是所有群元都对应于恒等线性映射(对应矩阵表示中的1或单位阵)的表示。

特别的,群 $G$ 的\textbf{零表示}(zero representation)指的是表示 $\rho: G \to \opn{GL}(0; \mathbb{F}) = \{0\}, \rho(g) = \opn{id}$。
\end{definition}

\begin{definition}{忠实表示}
若群元与线性变换之间的映射是单射则称该表示为\textbf{忠实表示}。
\end{definition}

显然,\autoref{gprep_ex1},\autoref{gprep_ex2} 为群的忠实表示,而对于非平凡群来说,平凡表示不是忠实表示。

\begin{definition}{矩阵群}
\textbf{矩阵群}是群元素为矩阵的群,即某个一般线性群 $\opn{GL}(n; \mathbb{F})$ 的子群。
\end{definition}

\begin{definition}{矩阵群的自身表示}
对于矩阵群 $G \subseteq \opn{GL}(n; \mathbb{F})$ 那么其自身的矩阵形式给出表示叫做\textbf{自身表示},即包含映射 $\rho: G \hookrightarrow \opn{GL}(n; \mathbb{F})$。
\end{definition}
\begin{example}{$SO2$群的自身表示}
SO2群是平面转动群其矩阵形式为:$D(\alpha)=\begin{pmatrix}
 \cos{\alpha} & -\sin{\alpha}\\
 \sin{\alpha} & \cos{\alpha}
\end{pmatrix}$。
则这是其自身表示。
\end{example}

\begin{corollary}{}
一个群表示的共轭、取逆后转置、取逆后取复共轭均还是群的一个表示,称为共轭表示、逆步表示和逆步复共轭表示。
\end{corollary}

证明:

若$D_{g_\gamma}=D_{g_\alpha}D_{g_\beta}$,那么$D_{g_\gamma}^*=D_{g_\alpha}^*D_{g_\beta}^*$

若$D_{g_\gamma}=D_{g_\alpha}D_{g_\beta}$,那么$(D_{g_\gamma}^{-1})^\mathrm{T}=(D_{g_\beta}^{-1}D_{g_\alpha}^{-1})^\mathrm{T}=(D_{g_\alpha}^{-1})^\mathrm{T}(D_{g_\beta}^{-1})^\mathrm{T}$

若$D_{g_\gamma}=D_{g_\alpha}D_{g_\beta}$,那么$(D_{g_\gamma}^{-1})^\dagger=(D_{g_\beta}^{-1}D_{g_\alpha}^{-1})^\dagger=(D_{g_\alpha}^{-1})^\dagger(D_{g_\beta}^{-1})^\dagger$

注:单纯的取逆或转置或复共轭得到的不一定是群的表示,原因如上,单纯取逆或转置及复共轭无法完成如上的两次调换顺序。