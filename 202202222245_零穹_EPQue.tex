% 端点可变问题
% 端点可变|泛函极值|全微分

\begin{issues}
\issueTODO
\end{issues}

\pentry{极端曲线\upref{ExtCur}}
\subsection{问题的提出}在前面的问题中,总是取以两个固定点为端点的曲线作为可取曲线\upref{DesCur}.现在,我们研究以更广泛的曲线作为可取曲线的泛函极值问题.

设函数 $F(x,y,y')$ 满足普通的连续性及可微性条件;此外,在 $xOy$ 平面上,给定两个 $C_1$ 类的曲线 $\varphi$ 及 $\psi$:
\begin{equation}
y=\varphi(x),\quad y=\psi(x)
\end{equation}
在此规定下,我们的问题可叙述如下:

取端点分别在曲线 $\varphi$ 及曲线 $\psi$ 上的 $C_1$ 类中全体曲线 $\gamma$,作为可取曲线族.现在要求泛函
\begin{equation}\label{EPQue_eq1}
J(\gamma)=\int_\gamma F(x,y,y')\dd x
\end{equation}
的极值,其中积分是沿曲线 $\gamma$ 而取的.

注意到:若某一以 $A$ 和 $B$ 为端点的曲线 $\gamma_0$ 是本问题的解,即 $\gamma_0$ 给积分\autoref{EPQue_eq1} 以极值,那么这一曲线 $\gamma_0$ 也在连接 $A_0$ 及 $B_0$ 的所有 $C_1$ 类曲线中,给 $J$ 以极值.由最简单问题时的欧拉定理\upref{ElueEV},曲线 $\gamma_0$ 满足欧拉方程
\begin{equation}
F_y-\dv{}{x}F_{y'}=0
\end{equation}

这就是说,若以可取曲线族中的极端曲线构成新的可取曲线族,得到的使 $J$ 取极值的曲线仍是同一条曲线.
\subsection{$J(\gamma)$ 对极端曲线的微分} 
\begin{theorem}{}
泛函 $J$ \autoref{EPQue_eq1} 对极端曲线(其起点和终点坐标分别为 $x_0,y_0;x_1,y_1$)的微分为
\begin{equation}
\begin{aligned}
&\pdv{J}{x_0}=-(F-y'F_{y'})^{(0)},\quad \pdv{J}{y_0}=-F_{y'}^{(0)}\\
&\pdv{J}{x_1}=(F-y'F_{y'})^{(1)},\quad \pdv{J}{y_1}=F_{y'}^{(1)}\\
&\dd J=-[(F-y'F_{y'})^{(0)}\delta_{x_0}+ F_{y'}^{(0)}\delta y_0]+[(F-y'F_{y'})^{(1)}\delta_{x_1}+F_{y'}^{(1)}\delta y_1]
\end{aligned}
\end{equation}
其中,\textbf{指标 ${(0)},{(1)}$ 在这里及以后(特指变分学相关词条)都表示对应的函数值是取在弧的起点及终点上}.
\end{theorem}
\subsubsection{证明}
设 $\gamma_0$ 是起点为 $A$ 终点为 $B$ 的极端曲线,考虑临近于它的极端曲线族 $\{\gamma\}$ ,它们的端点在 $A,B$ 的某二邻区内.假设 $\{\gamma\}$ 中只有唯一的弧通过每二端点,则 $\{\gamma\}$ 中每一条弧将由它们的起点和终点坐标 $x_0,y_0;x_1,y_1$ 决定,即在族 $\{\gamma\}$ 上的泛函 $J(\gamma)$ 变成弧 $\gamma$ 的端点的坐标函数:$J(\gamma)=J(x_0,y_0,x_1,y_1)$.于是当变数微分为 $\delta x_0,\delta y_0,\delta x_1,\delta y_1$ 时,就有
\begin{equation}
\dd J=\qty(\pdv{J}{x_0}\delta x_0+\pdv{J}{y_0}\delta y_0)+\qty(\pdv{J}{x_1}\delta x_1+\pdv{J}{y_1}\delta y_1)
\end{equation}
1.\textbf{首先},考虑 $x_0,x_1$ 固定的族 $\{\gamma\}$ 中的曲线构成的部分族(弧 $\gamma$ 的端点在两条平行于 $y$ 轴的直线上流动),于是这个部分族只含两参变数 $y_0,y_1$,且在该部分族上微分与变分重合.设 $y=y(x)$ 与 $y=y(x)+\delta y(x)$ 是这个部分族中的两条无限接近的极端曲线弧.于是 $y_0=y(x_0),y_1=y(x_1)$,并令 $\delta y_0=\delta y(x_0),\delta y_1=\delta y(x_1)$.于是,\textbf{从弧 $y=y(x)$ 到 $y=y(x)+\delta y(x)$ 的变分 $\delta J$等于取在前一弧上的积分\autoref{PolReq_the1}~\upref{PolReq}}
\begin{equation}\label{EPQue_eq2}
\delta J=\int_{x_0}^{x_1}(F_{y}\delta y+F_{y'}\delta y')\dd x
\end{equation}
注意到\autoref{VarCha_eq5}~\upref{VarCha},并利用分部积分\upref{IntBP},得
\begin{equation}
\int_{x_0}^{x_1}F_{y'}\delta y' \dd x=F_{y'}^{(1)}-F_{y'}^{(0)}-\int_{x_0}^{x_1}\qty(\dv{}{x}F_{y'})\delta y\dd x
\end{equation}
于是\autoref{EPQue_eq2} 化为
\begin{equation}
\begin{aligned}
\delta J&=F_{y'}^{(1)}\delta y_1-F_{y'}^{(0)}\delta y_0+\int_{x_0}^{x_1}\qty(F_y-\dv{}{x}F_{y'})\delta y\dd x\\
&=F_{y'}^{(1)}\delta y_1-F_{y'}^{(0)}\delta y_0
\end{aligned}
\end{equation}
其中,积分项为0是因为弧 $y=y(x)$ 为极端曲线,满足欧拉方程.

同时,因为对该部分族有
\begin{equation}
\delta J=\dd J=\pdv{J}{y_0}\delta y_0+\pdv{J}{y_1}\delta y_1
\end{equation}
比较上面两式得
\begin{equation}
\pdv{J}{y_0}=-F_{y'}^{(0)},\quad \pdv{J}{y_1}=F_{y'}^{(1)}
\end{equation}
2.\textbf{现在},考虑另一部分族,它由同一条极端曲线 $y$
