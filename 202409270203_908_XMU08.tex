% 厦门大学 2008 年 考研 量子力学
% license Usr
% type Note

\textbf{声明}:“该内容来源于网络公开资料,不保证真实性,如有侵权请联系管理员”

\subsection{(25 分)简述题(每小题5分)}
(1) 解释微观粒子的波粒二象性,并写出德布罗意(de Broglie)关系式。

(2) 解释能级简并的概念并指出其起因。

(3) 经典力学和量子力学中,守恒量的含义有何不同?

(4) Clebsch-Gordan 系数(记为 $\langle  j_1 m_1 j_2 m_2|j m  \rangle$)不为零的条件。

(5) 微扰论的适用条件是什么?在库仑场中,微扰理论为什么不适用于计算高能级($n$ 较大)的修正?

\subsection{(25 分)}
质量为 \( m \) 的粒子在外场作用下作一维运动 \((- \infty < x < \infty)\),已知其处于束缚态 \(\varphi_1(x)\) 时,动能的平均值等于 \( E_1 \),并已知 \(\varphi_1(x)\) 是实函数且已归一化。试求当粒子处于态 \(\varphi_2(x) = \varphi_1(x) e^{ikx}\)(\(k\) 为实数)时动量平均值及动能平均值。
\subsection{(25 分)}
设已知在 \( L^2 \) 和 \( L_z \) 的共同表象中,算符 \( L_x \) 和 \( L_y \) 的矩阵分别为

$$L_x = \frac{\hbar}{\sqrt{2}}\begin{pmatrix}0 & 1 & 0 \\\\1 & 0 & 1 \\\\0 & 1 & 0\end{pmatrix}, \quad L_y = \frac{\hbar}{\sqrt{2}}\begin{pmatrix}0 & -i & 0 \\\\i & 0 & -i \\\\0 & i & 0\end{pmatrix}.~$$

(1)求\( L_z \)和\( L_y \),的本征值和归一化的本征函数:

(2)将矩阵\( L_z \)和\( L_y \),对角化。
\subsection{(25 分)}
一电子处于自旋态
\[\chi = A\begin{pmatrix}3i \\\\4\end{pmatrix},~\]
求:

(1) 归一化常数 \( A \);

(2) 自旋 \( S_x \), \( S_y \), \( S_z \) 的平均值;

(3) \( S_x \), \( S_y \), \( S_z \) 的不确定度 \(\Delta_i = \sqrt{\langle (S_i^2  - \langle S_i \rangle)^2\rangle}~\)。

\subsection{(25 分)}
设在 \( t=0 \) 时氢原子的波函数为
\[\psi(r, 0) = \frac{1}{3} \left( \psi_{100} + \psi_{210} + 2 \psi_{211} + \sqrt{3} \psi_{21-1} \right)~\]
式中下标是量子数 \( n, l, m \) 的值。试求:

(1) 体系能量的期望值;

(2) \( t \) 时刻的波函数 \( \psi(r,t) \);

(3) \( t \) 时刻体系处于 \( l=1, m=1 \) 态的几率。

[提示]利用演化算符
\[\psi(r,t) = e^{-\frac{i }{\hbar}^{\hat{H}t}} \psi(r,0)~\]
\subsection{(25 分)}
转动惯量为 $I$,电偶极距离为 $D$(常矢量)的空间转子,处在匀匀强电场 $\mathcal{E}$ 中,用微扰论求转子基态能量的二级修正。

[提示] 选取外电场 $\mathcal{E}$ 的方向为球坐标的极轴方向,且注意到

$$H_0 = \\frac{L^2}{2I},$$

$$Y_{10} = \\sqrt{\\frac{3}{4\\pi}} \cos \theta$$