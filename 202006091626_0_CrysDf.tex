% 晶体衍射
% 晶体|衍射|布拉格衍射|米勒指数

\subsection{米勒指数}

这里只讨论长方体晶格,假设三条边分别为 $a,b,c$.  晶格面是通过许多格点的平面.取一个格点为原点,平面过三条坐标轴的截距分别定义为 $a/h, b/k, c/l$.其中 $h,k,l$ 必须是整数.这样,平面方程为
\begin{equation}
\frac{x}{a/h} + \frac{y}{b/k} + \frac{z}{c/l} = 1
\end{equation}
法向量为 $(h/a,\;k/b,\;l/c)/\sqrt{(h/a)^2 + (k/b)^2 + (l/c)^2}$.现在来看相邻的两平面相距多少.截点有仍然需要落在格点上,所以只能是所有截距变为两倍.两平面的距离为法向量与任何一个截距的增量矢量内积
\begin{equation}
d = \frac{1}{\sqrt{(h/a)^2 + (k/b)^2 + (l/c)^2}}
\qquad
\frac{1}{d^2} = \frac{h^2}{a^2} + \frac{k^2}{b^2} + \frac{l^2}{c^2}
\end{equation}

\subsection{布拉格衍射公式}

两个平面,不管格点在上面如何分布,若入射光和出射光和平面夹角都为 $\theta$,那么光程差为 $\delta  = 2d\sin \theta$, 干涉条件为
\begin{equation}
2d\sin \theta  = n\lambda
\end{equation}

\subsection{Crystalline Scattering Factor}

晶格中各个原子的位置用 $(x,y,z)$ 表示,坐标为 $(ax,by,cz)$. 在进行布拉格衍射时,同一个 cell 里面的不同格点会产生不同平面组,即不同相位.例如两点 $(x_1, y_1, z_1)$ 和 $(x_2, y_2, z_2)$ 所在的两个平面
\begin{equation}\ali{
d_{12} &= \frac{(h/a,\;k/b,\;l/c)}{\sqrt{(h/a)^2 + (k/b)^2 + (l/c)^2}} \vdot [a(x_2 - x_1), b(y_2 - y_1),c(z_2 - z_1)] \\
&= \frac{(h\Delta x + k\Delta y + l\Delta z)}{\sqrt{(h/a)^2 + (k/b)^2 + (l/c)^2}} = d(h\Delta x + k\Delta y + l\Delta z)
}\end{equation}
然而发生衍射时, $d$ 对应的相位差为 $2\pi$,所以 $d_{12}$ 对应的相位差为
\begin{equation}
\delta  = 2\pi (h\Delta x + k\Delta y + l\Delta z)
\end{equation}
如果一个晶格有多个原子,每个原子的散射振幅为 $f_i$,那么总振幅为
\begin{equation}
F = \sum_i f_i \E^{2\pi\I(hx_i + ky_i + lz_i)}
\end{equation}


 



