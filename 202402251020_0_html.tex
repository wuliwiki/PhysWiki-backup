% html 笔记
% keys html
% license Usr
% type Note

\begin{issues}
\issueDraft
\end{issues}

% html
\begin{lstlisting}[language=none]
<!DOCTYPE html>
<html>
<head>
    <meta charset="UTF-8">
</head>
<body>
    <!-- 内容 -->
</body>
</html>
\end{lstlisting}




可以用 base64 来表示图片,代替单独的图片文件
\begin{lstlisting}[language=none]
<img src="data:image/jpeg;base64,一个base64字符串" alt="...">
\end{lstlisting}
其中 \verb|一个base64字符串| 是直接把整个文件转换成 base64, 在 linux 命令行中可以用 \verb|base64 文件| 做到。该命令默认给每 64 个字符换行,最后一行即使每填满也换行。 若不需要换行用 \verb|base64 -w 0 文件|
