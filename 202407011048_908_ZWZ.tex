% 中微子
% license CCBYSA3
% type Wiki

(本文根据 CC-BY-SA 协议转载自原搜狗科学百科对英文维基百科的翻译)

中微子(/nuːˈtriːnoʊ/或/njuːˈtriːnoʊ/)(由希腊字母ν表示)是费米子(一种具有半整数自旋的基本粒子),它仅通过弱力和引力参与相互作用。[1][2]中微子之所以如此命名,是因为它是电中性的,并且因为它的静止质量非常小,以致于人们长期以来认为它为零。中微子的质量比其他已知的基本粒子小得多。弱力的范围非常短,引力相互作用非常弱,中微子作为轻子不参与强相互作用。因此,中微子通常能畅通无阻地穿过普通物质,无法被探测到。[1][2]

弱相互作用产生三味轻子类型之一的中微子:电子中微子($\nu_e$)、μ子中微子($\nu_\mu$)、或$\tau$中微子($\nu_\tau$)并伴随相应的带电轻子。虽然中微子长期以来被认为是无质量的,但现在已经知道有三个不同微小值的离散中微子质量,但它们并不与这三种味道唯一对应。一个带有特定味道的中微子是所有三种质量状态的特定量子叠加。因此,中微子在飞行中在不同的味道之间振荡。例如,在衰变反应中产生的电子中微子可能在远处的探测器中作为$\mu$子或$\tau$中微子相互作用。尽管截至2016年,只有三个质量值的平均差是已知的,但宇宙学观察表明,三个质量的总和必须小于电子质量的百万分之一。

对于每一个中微子,也存在一个相应的反粒子,称为反中微子,它也有半整数自旋,没有电荷。它们与中微子的区别在于轻子数和手性符号相反。为了使总的轻子数守恒,在核$\beta$衰变中,电子中微子只与正电子(反电子)或电子反中微子一起出现,电子反中微子与电子或电子中微子一起出现。

中微子是由各种放射性衰变产生的,包括原子核或强子的$\beta$衰变、核反应,如发生在恒星核心或核反应堆、核弹或粒子加速器中的核反应、超新星爆发期间、中子星自旋期间以及加速粒子束或宇宙射线撞击原子时。地球附近的大多数中微子来自太阳的核反应。在地球附近,垂直于太阳方向,每平方厘米每秒大约有650亿($6.5\times10^{10}$)太阳中微子穿过。[3]

为了研究中微子,可以用核反应堆和粒子加速器人工制造中微子。有大量涉及中微子的研究活动,目标包括确定三个中微子质量值,测量轻子区的CP破坏程度(导致轻子发生);并寻找粒子物理标准模型之外的物理证据,如无中微子双$\beta$衰变,这将是轻子数守恒破坏的证据。中微子也可以用于地球内部的断层摄影。[4][5]

\subsection{历史}
\subsubsection{1.1 泡利的提议}
中微子是沃尔夫冈泡利在1930年首次提出的,用来解释$\beta$衰变如何使能量、动量和角动量(自旋)守恒。与尼尔斯·玻尔相反,他提出了守恒定律的统计版本来解释在$\beta$衰变中观察到的连续能谱,泡利假设了一个未被发现的粒子,他称之为“中子”,使用与命名质子和电子相同的-on结尾来命名。他认为新粒子是在β衰变过程中与电子或β粒子一起从原子核中发射出来的。[6]

詹姆斯·查德威克在1932年发现了一种质量更大的中性核粒子,并将其命名为中子,留下了两种同名的粒子。早些时候(1930年),泡利用“中子”一词既指在$\beta$衰变中保存能量的中性粒子,也指原子核中假定的中性粒子;起初,他并不认为这两种中性粒子彼此不同。[6]中微子一词是通过恩利克·费米引入科学词汇的,他在1932年7月巴黎的一次会议上和1933年10月的苏威会议上使用了中微子,泡利也在那次会议上使用了中微子。这个名字(意大利语中相当于“小中性粒子”)是由爱德华多·阿马尔迪(Edoardo Amaldi)在罗马via Panisperna物理研究所与费米(Fermi)交谈时开玩笑地创造出来的,目的是为了将这个轻中性粒子与查德威克的重中子区分开来。[7]

在费米的β衰变理论中,查德威克的大中性粒子可以衰变为质子、电子和较小的中性粒子(现在称为电子反中微子):
$$n^0 \to p^+ +e^- +\nu_e~$$
费米的论文写于1934年,将泡利中微子与保罗·狄拉克正电子和维尔纳海森堡中子质子模型统一起来,为未来的实验工作提供了坚实的理论基础。《自然》杂志拒绝了费米的论文,称该理论“离现实太远”。他将论文提交给一家意大利杂志,该杂志接受了论文,但在早期对他的理论普遍缺乏兴趣,导致他转向实验物理学。[8]

到1934年,有实验证据反对玻尔关于能量守恒对$\beta$衰变无效的观点:在那一年的索尔维会议上,报道了对β粒子(电子)能谱的测量,表明每种$\beta$衰变的电子能量都有严格的限制。如果能量守恒是无效的,这种限制是不可预期的,在这种情况下,在至少几个衰变中,统计上任意数量的能量都是有可能的。1934年首次测量到的β衰变谱的自然解释是,只有有限的(和守恒的)能量可用,一个新粒子有时会吸收有限能量的不同部分,剩下的留给β粒子。泡利利用这个机会公开强调仍然未被发现的“中微子”一定是一个真实的粒子。

\subsubsection{1.2 直接检测}
\begin{figure}[ht]
\centering
\includegraphics[width=6cm]{./figures/474e24bba6d42f22.png}
\caption{克莱德·科温进行中微子实验大约1956} \label{fig_ZWZ_1}
\end{figure}
1942年,王淦昌首次提出使用β俘获来实验检测中微子。[9]在1956年7月20日的《科学》杂志上,克莱德·考恩、弗雷德里克·莱因斯、哈里森、克鲁斯和麦奎尔发表了他们已经探测到中微子的证据,[10][11]这一结果在近40年后获得了1995年诺贝尔奖。[12]

在这个现在被称为考恩-雷恩中微子实验的实验中,通过在核反应堆中的β衰变产生的反中微子与质子反应产生中子和正电子:
$$





























































