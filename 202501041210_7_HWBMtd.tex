% 菲涅尔半波带法
% keys half wave band|Fresnel|optics|光学
% license Usr
% type Tutor


\pentry{振动的指数形式\nref{nod_VbExp}}{nod_67f8}

菲涅尔半波带法是一种处理连续分布的波源时,简化的计算方法。



\subsection{波的干涉}


设介质或空间中存在一个波源,如果要研究此波源在整个空间中产生的波是怎样随时间变化的,只需要任取空间中的一个点$P$,研究清楚$P$上的波函数如何变化,则由$P$的任意性,相当于已经得到了整个空间中波的分布。

产生波的点,我们称之为\textbf{源点};被选中用于研究的点,我们称之为\textbf{场点}。

当空间中只有一个源点时,场点的波可以理解为源点的波弱化并时间延迟后的结果。设源点为$S$,场点$P$到$S$的距离为$L$,波速为$c$,则在时间$t_0$时$P$点处的波函数,就是时间$t_0-L/c$时$S$点处的波函数乘以一个系数,此系数正比于$1/L^{n/2}$,其中$n$是波的维度\footnote{波的维度为描述波函数所需要的空间维度;这也正是介质的维度。}减一。此系数表达的是能量守恒,波随着传递而减弱\footnote{比如二维波,如水波。波源的振动在时间$t$后传播到整个半径为$ct$的圆上,振动能量均匀分布到这个圆上,因此圆上的能量密度正比于$1/ct$。由于振动所含的能量正比于振幅的平方,因此圆上振幅正比于$1/\qty(ct)^{1/2}$。}。

比如说,若源点的振动为$A\cos \omega (t+\varphi_0)$,那么场点的振动则为$\frac{AC}{L^n/2}\cos \omega(t+\varphi_0-L/c)$。当然,表示

当空间中有多个源点时,任意时刻场点的波即为所有源点在场点单独产生的波函数相加。为了表达干涉,最好用复数来表示简谐振动,即将源点$S_i$的振动$A_i\cos\omega_i(t-\varphi_i)$写为$A_i\exp\qty(\I\omega_i(t-\varphi_i))$。此时$S_i$在场点$P$产生的振动为$\frac{A_iC}{L_i^{n/2}}\exp\qty(\I\omega_i(t-\varphi_i))\frac{A_iC}{L_i^{n/2}}\exp\qty(\I\omega_i(t-\varphi_i))$,而所有源点在$P$产生的合振动即为
\begin{equation}
\sum_i \frac{A_iC}{L_i^{n/2}}\exp\qty(\I\omega_i(t-\varphi_i))~. 
\end{equation}

这个式子本身看起来并不直观,我们可以用复数的几何表示来直观理解计算过程:每一项$\frac{A_iC}{L_i^{n/2}}\exp\qty(\I\omega_i(t-\varphi_i))$都表示一个箭头,$\frac{A_iC}{L_i^{n/2}}$是其长度,$\omega_i(t-\varphi_i)$是其角度;每个箭头都随着时间逐渐旋转,角速度为$\omega_i$;最终的振动便是所有箭头相加的结果,随着时间变化。



\begin{example}{一个合振动的计算实例}

\begin{figure}[ht]
\centering
\includegraphics[width=10cm]{./figures/7ce4e9662e078e39.pdf}
\caption{三个波源干涉的计算例子} \label{fig_HWBMtd_1}
\end{figure}

如图\autoref{fig_HWBMtd_1} 

\end{example}




现实中,波源并不是理想的一个点,而是占据了一定区域,包含无穷多个源点。此时干涉的计算就从\textbf{求和}改为\textbf{积分},






















