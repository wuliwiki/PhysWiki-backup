% 狭义相对论的运动学(有质量粒子)
% 狭义相对论|运动学|有质量粒子|质能关系

\pentry{世界线和固有时\upref{wdline}}

约定使用东海岸度规 $\eta_{\mu\nu}=\rm{diag}(-1,1,1,1)$ 和自然单位制 $c=1$。

\subsection{四速度、四动量、四加速度}
为了计算的方便,我们可以定义一系列四矢量,它们在洛伦兹变换下都是协变的,即与 $x'^\mu=\Lambda^\mu{}_\nu x^\nu$ 的变换规则相同\footnote{参考时空的四维表示\upref{SR4Rep}。}。

\begin{definition}{四速度}
有质量粒子的世界线 $x^\mu(\tau)$ 是类时的,$\tau$ 为固有时\footnote{参考世界线和固有时\upref{wdline}。}。那么可以定义其四速度为
\begin{equation}
u^\mu = \dv{x^\mu(\tau)}{\tau}
\end{equation}
根据固有时的定义\footnote{参考世界线和固有时\upref{wdline}。},$u^\mu u_\mu = -1$。
\end{definition}
粒子的四速度总是满足 $u^\mu u_\mu = -1$,因此 $\dd s = \dd \tau \sqrt{- u^\mu u_\mu} = \dd \tau$ 就是世界线上的时空间隔。

\begin{definition}{四动量}
定义 $\hat p=m\hat u$,其中 $m$ 为粒子的静质量。那么
\begin{equation}
p^\mu = m u^\mu = (E,\bvec P)
\end{equation}
\end{definition}
设粒子的运动速度为 $\bvec v$,那么粒子的四速度为 $(\gamma_{\bvec v},\gamma_{\bvec v}\bvec v)$,四动量为 $(m\gamma_{\bvec v}, m\gamma_{\bvec v} \bvec v)=(E,\bvec P)$。由此可以看到
\begin{equation}
E^2-|\bvec P|^2=-p^\mu p_\mu=-m^2 u^\mu u_\mu = m^2
\end{equation}
这就是著名的爱因斯坦\textbf{质能关系}。如果取消自然单位制回归国际单位制(带上 $c$),那么上式改写为 $E^2-|\bvec P|^2c^2=m^2c^4$,且容易证明在非相对论近似($v\ll c$)下,$\bvec P\approx m\bvec v,\ E\approx mc^2+\frac{1}{2}mv^2$。
而在相对于粒子精止的参考系(被称为\textbf{共动参考系})中,
\begin{equation}
E=mc^2
\end{equation}
最后可以定义四加速度。
\begin{definition}{四加速度}
\begin{equation}
a^\mu = \dv{u^\mu}{\tau}
\end{equation}
\end{definition}
它与我们通常所定义的加速度 $\bvec a=\dd {\bvec v}/\dd t$ 不同,但这种定义的好处是它满足洛伦兹协变性。