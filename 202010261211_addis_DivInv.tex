% 散度的逆运算
% 散度|库伦定律|不定积分

\pentry{散度\upref{Divgnc}}

\begin{theorem}{}
令 $V(\bvec r)$ 为任意标量函数, 则 $V(\bvec r)$ 总能表示为一个矢量场 $\bvec F(\bvec r)$ 的散度, 即
\begin{equation}\label{DivInv_eq3}
V(\bvec r) = \div \bvec F(\bvec r)
\end{equation}
且 $\bvec F(\bvec r)$ 可以通过以下公式计算:
\begin{equation}\label{DivInv_eq1}
\bvec F(\bvec r) \equiv \frac{1}{4\pi}\int \frac{V(\bvec r') \bvec R}{R^3} \dd{V'}
\end{equation}
其中 $\bvec r, \bvec r'$ 分别是坐标原点指向三维直角坐标 $(x, y, z)$ 和 $(x', y', z')$ 的位置矢量, $\bvec R = \bvec r' - \bvec r$, $R = \abs{\bvec R}$, 体积分 $\int\dd{V'} = \int\dd{x'}\dd{y'}\dd{z'}$ 的区域是空间中 $\bvec F$ 不为零的区域, $\cross$ 表示矢量叉乘\upref{Cross}.

若在\autoref{DivInv_eq1} 右边也可以加上任意无散场 $\bvec H(\bvec r)$, 定理同样成立.
\end{theorem}

\autoref{DivInv_eq1} 可以看做是散度运算的逆运算, 类似于不定积分\upref{Int}是求导的逆运算. $\bvec H(\bvec r)$ 则相当于不定积分中的任意常数. 旋度也有类似的逆运算\upref{HlmPr2}.

该定理在物理中可应用于电场和电荷的关系, 定理中的 $\bvec F$ 可看做电场 $\bvec E$, 而 $V(\bvec r)$ 可看做电荷密度 $\rho(\bvec r)$ 除以真空介电常数 $\epsilon_0$. 于是\autoref{DivInv_eq3} 就是电场的高斯定理(\autoref{EGauss_eq1}~\upref{EGauss}), \autoref{DivInv_eq1} 就是库伦定律(\autoref{Efield_eq9}~\upref{Efield}).

\begin{theorem}{}
令 $V(\bvec r)$ 为一个标量场, 则\autoref{DivInv_eq1} 得到的 $\bvec F(\bvec r)$ 是一个无旋场, 即 $\curl \bvec F = \bvec 0$.
\end{theorem}

\subsection{证明定理 1}
对\autoref{DivInv_eq1} 求散度, 得
\begin{equation}\label{DivInv_eq2}
\div \bvec F(\bvec r) = \frac{1}{4\pi}\int \div \frac{V(\bvec r') \bvec R}{R^3} \dd{V'}
\end{equation}
使用\autoref{VecOp_eq2}~\upref{VecOp}, 得
\begin{equation}
\div \frac{V(\bvec r') \bvec R}{R^3} = [\div V(\bvec r')] \frac{ \bvec R}{R^3} + V(\bvec r') \qty(\div \frac{\bvec R}{R^3})
\end{equation}
由于 $V(\bvec r')$ 是 $x', y', z'$ 而不是 $x, y, z$ 的函数, 右边第一项中 $\div V(\bvec r') = 0$. 第二项等于 $4\pi \delta(\bvec R)$ (链接未完成). 上式代入\autoref{DivInv_eq2} 得
\begin{equation}
\div \bvec F(\bvec r) = \int \div V(\bvec r') \delta(\bvec r - \bvec r') \dd{V'} = V(\bvec r)
\end{equation}
证毕.

\subsection{证明定理 2}
对\autoref{DivInv_eq1} 求旋度, 得
\begin{equation}
\curl \bvec F(\bvec r) \equiv \frac{1}{4\pi}\int \curl \frac{V(\bvec r') \bvec R}{R^3} \dd{V'}
\end{equation}
由\autoref{VecOp_eq3}~\upref{VecOp}得