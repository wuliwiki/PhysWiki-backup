% 群代数与正则表示
% keys 群代数|群空间|正则表示|类空间
\pentry{群矩阵表示及实例\upref{gprep}共轭与共轭类\upref{gpcon}}


\subsection{群代数}

注:本文主要以复表示为例。

\begin{definition}{群空间}
对于有限群$G=\{g_1,g_2...g_m\}$,设$V_G$为群元在复数域$\mathbb{C}$是的所有线性叠加的集合:

\begin{equation}
V_G=\{\displaystyle\sum_\nu x_\nu g_\nu|x_\nu \in \mathbb{C},g_\nu \in G\}
\end{equation}

在这个基础上我们可以定义加法和数乘。

设$x=\displaystyle\sum_\nu x_\nu g_\nu$,$y=\displaystyle\sum_\mu y_\mu g_\mu$,$a\in \mathbb{C}$,则有:
\begin{align}
x+y&=\displaystyle\sum_\nu x_\nu g_\nu+\displaystyle\sum_\mu y_\mu g_\mu=\displaystyle\sum_\nu(x_\nu+y_\nu)g_\nu\\
ax&=\displaystyle\sum_\nu (ax_\nu) g_\nu
\end{align}

这样显然构成了一个$m$维的线性空间,$m$为群$G$的阶数。

线性空间的一组基为$\{g_1,g_2...g_m\}$,称为自然基底。

\end{definition}

在定义完群空间后我们进一步定义群空间中的代数乘法,使其构成一个代数,也就是本节标题——群代数。

\begin{definition}{群代数}
设$V_G$为群$G$的群空间,且有$x=\displaystyle\sum_\nu x_\nu g_\nu$,$y=\displaystyle\sum_\mu y_\mu g_\mu$,我们定义其乘法规则为:

\begin{equation}
x * y = \displaystyle\sum_{\nu} x_\nu g_\nu \sum_{\mu} y_\mu g_\mu =
\displaystyle\sum_{\nu\mu}(x_\nu y_\mu) (g_\nu g_\mu)
\end{equation}

其中$g_\nu g_\mu$一项依照群乘法表的乘法规则给出结果。

在这样的乘法规则下$V_G$构成了复数域$\mathbb{C}$上的一个结合代数,称为$A_G$。

\end{definition}

注:群代数的结合律来自于群元的结合律。

我们同样可以在这个线性空间中定义内积
$$(g_\alpha,g_\beta)=\delta_{\alpha,\beta}$$

在群代数的视角下,群元的相乘可以视作一个算符,由群元映射出的算符给出的表示被称为正则表示,由算符定义不同可以分为\textbf{左正则表示}和\textbf{右正则表示}。

这样给出的算符可以验证其的确符合群乘法规则:

$$L(g_\alpha)x=g_\alpha x,R(g_\alpha)x=xg_\alpha^{-1}$$

容易验证这两种方法定义的算符均符合群的乘法规则,可以构成一个表示。

表示矩阵的形式也极为简单,仅仅是标定了群元相乘的结果,我们以$C_3$群为例(以$\{e,a,a^2\}$为基)。

$$D(e)=\begin{pmatrix}
 1& 0& 0\\
 0 &1 & 0 \\
 0&0&1
 \end{pmatrix},
 D(a)=\begin{pmatrix}
 0&0& 1\\
 1&0 & 0 \\
 0&1&0
 \end{pmatrix},
 D(a^2)=\begin{pmatrix}
 0& 1& 0\\
 0 &0 & 1 \\
 1&0&0
\end{pmatrix}$$

\subsection{类空间}

\begin{definition}{}
设集合$C_\alpha=\{s_1,s_2,...s_{n(\alpha)}\}$为群$G$的一个共轭类。

那么定义群代数空间$A_G$中矢量$c^\alpha$为\textbf{类算符}。
$$c^\alpha=\displaystyle\sum_{s_i\in C_\alpha}s_i$$
\end{definition}

根据共轭类的定义可以看出类算符与所有群元对易。

完全类似上文的,可以定义:
\begin{definition}{类空间}
以群$G$中所有类算符构成的线性空间$V_C$:
$$V_C=\{x=\displaystyle\sum_\alpha x_\alpha c^\alpha,x_\alpha \in \mathbb{C}\}$$

称为群$G$的\textbf{类空间},类空间是群空间的一个子空间。
\end{definition}