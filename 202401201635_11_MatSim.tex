% 相似变换和相似矩阵
% keys 相似变换|相似矩阵|酉矩阵|对角化|本征值
% license Xiao
% type Tutor

\begin{issues}
\issueDraft 欠缺向量的变化,和线性映射角度上的详细解释。
\end{issues}

\pentry{酉矩阵\upref{UniMat}}

矩阵是线性映射在特定基下的表示,每一列表示基向量被映射后的结果。因此我们可以改变空间的基向量组,从而得到线性映射的不同表示。
\begin{definition}{相似变换}\label{def_MatSim_1}
设$A,B$为$n$阶矩阵,若存在$n$阶可逆方阵$Q$,使得
\begin{equation}
Q^{-1}AQ=B~,
\end{equation}
则称$A$\textbf{相似}于$B$,该运算称为对$A$进行相似变换,可逆矩阵$Q$称为过渡矩阵。
\end{definition}
相似关系是同阶矩阵群的等价关系(等价关系用$\sim $表示),这种关系具有下列性质:

\begin{enumerate}
\item 反身性 $A\sim A$ 
\item 对称性 若$A\sim B$,则$B\sim A$
\item 传递性 若$A\sim B\,,B\sim C$,则$A\sim C$
\end{enumerate}
由于相似变换是对同一空间的不同基向量组进行一一映射,因而相似变换是矩阵乘法群上的自同构,并具有如下性质(假设$A\sim B$):
\begin{enumerate}
\item $\opn{R}(A)=\opn{R}(B)$;
\item $A$与$B$的行列式相同:$|A|=|B|$;
\item $A$与$B$的迹相同:$\opn{Tr}A=\opn{Tr}B$;
\item 若$A$可逆,则$B$也可逆,且$A^{-1}\sim B^{-1}$;
\item $kA\sim kB\,,A^m\sim B^m$,其中$k$为任意常数,$m\in \mathbb Z^{+}$;
\item 若$f(x)$是任意多项式,则$f(A)\sim f(B)$。
\end{enumerate}
以上六点性质在线性映射的角度上看都是很好理解的,只要抓住相似矩阵是同一线性映射在不同基下的表示这一意义即可。在矩阵角度上也非常好证,比如第五点的$A^m\sim B^m$:\begin{equation}
B^m=(Q^{-1}AQ)^m=Q^{-1}AQQ^{-1}AQ...Q^{-1}AQ=Q^{-1}A^mQ~.
\end{equation}

在量子力学中,相似变换也无处不在,我们一般称之为\textbf{表象变换}。由于量子力学中的矩阵定义在复数域上,且基向量组都满足正交归一关系,因此相似变换是保距变换(正交变换),且过渡矩阵$Q$是酉矩阵,满足:$Q^{\dagger}=Q^{-1}$\footnote{由复数域上的保距关系:$\bvec x^{\dagger}Q^{\dagger}Q\bvec x=\bvec x^{\dagger}\bvec x$得。}。

由于相似变换改变了空间的基,因此向量的坐标表示也发生了变化。设相似变换前后的基向量组分别为$\{\bvec e_i\}$与$\{\bvec {\theta}_i\}$,且对于任意向量有:
\begin{equation}
\bvec x=x^i\bvec e_i=x'^i\bvec {\theta}_i~.
\end{equation}
设过渡矩阵为$Q$,即$\bvec{\theta}_i=Q^j_i\bvec e_j$,代入上式得:
\begin{equation}
x'^i\bvec {\theta}_i=x'^iQ^j_i\bvec e_j=x^i\bvec e_i~.
\end{equation}
即$\bvec x'=Q^{-1}\bvec x$。由这个关系,我们可以进一步证明一个直觉上的常识:\textbf{相似变换不改变矩阵的特征值和特征向量}。
\subsection{对角化}
\pentry{矩阵的本征问题\upref{MatEig}}

若相似变换可以使矩阵变为对角矩阵, 我们把这个过程称为对角化,并称$A$\textbf{可对角化}。从\autoref{the_MatEig_1}~\upref{MatEig}可知,$A$可对角化的充要条件是$A$有$n$个线性无关的特征向量。结合\autoref{the_MatEig_2}~\upref{MatEig}“不同特征值的特征向量线性无关”和\autoref{cor_MatEig_1}~\upref{MatEig}“不同特征值的特征向量组线性无关”可知,如果$A$满足以下两种情况,则$A$可对角化。
\begin{corollary}{}
如果$n$阶矩阵$A$有$n$个不同的特征值,则$A$一定可以对角化。
\end{corollary}
\begin{corollary}{}
$n$阶方阵$A$可对角化当且仅当$A$有$n$个特征值(包括重根),且每个特征值的几何重数等于代数重数。
\end{corollary}



可以证明,实对称矩阵一定可以对角化。\footnote{参见合同变换一节}

