% 斯里尼瓦瑟·拉马努金(综述)
% license CCBYSA3
% type Wiki

本文根据 CC-BY-SA 协议转载翻译自维基百科 \href{https://en.wikipedia.org/wiki/Srinivasa_Ramanujan}{相关文章}。

斯里尼瓦瑟·拉马努金·艾扬加尔(Srinivasa Ramanujan Aiyangar)FRS(1887年12月22日-1920年4月26日)是一位印度数学家。他常被视为史上最伟大的数学家之一。尽管几乎没有接受过纯数学的正规训练,他仍在数学分析、数论、无穷级数和连分数等领域做出了重要贡献,并提出了当时被认为无法解决的数学问题的解法。

拉马努金最初是在孤立的环境中自行开展数学研究的。正如汉斯·艾森克所说:“他曾试图让当时最顶尖的职业数学家对他的研究产生兴趣,但大多数时候都失败了。他所展示的成果太新颖、太陌生,而且呈现方式也很不寻常;那些人懒得去理会。”\(^\text{[4]}\)为了寻找能真正理解他工作的数学家,1913年,他开始与英国剑桥大学的数学家G.H.哈代通信。哈代意识到拉马努金的研究非同寻常,便为他安排了赴剑桥的行程。在笔记中,哈代评论道,拉马努金提出了具有突破性的全新定理,其中一些“令我完全败下阵来;我从未见过任何类似的东西”,\(^\text{[5]}\)还有一些则是刚刚被证明、极为高深的成果。

在他短暂的一生中,拉马努金独立整理出了近 3900 条数学成果(主要是恒等式和方程)。\(^\text{[6]}\)其中许多都是前所未见的原创成果;他那些独特而极不寻常的发现,如“拉马努金素数”、“拉马努金θ函数”、“整数划分公式”以及“拟θ函数”等,不仅开辟了全新的研究领域,也激发了大量后续研究。\(^\text{[7]}\)在他成千上万的研究成果中,大多数后来都被证明是正确的。\(^\text{[8]}\)以他的名字命名的《拉马努金期刊》应运而生,专门发表受他研究影响的各类数学成果。\(^\text{[9]}\)他留下的笔记本——记录了他已发表和未发表成果的摘要——至今仍被数学家们分析和研究,成为不断涌现新数学思想的重要来源。直到 2012 年,研究者们仍不断发现,他笔记中那些仅以“简单性质”或“相似结果”带过的评论,其实暗藏着深奥而精妙的数论定理,且这些定理直到他去世近百年后才被真正识别出来。\(^\text{[10][11]}\)拉马努金是最年轻的英国皇家学会会士之一,是第二位印度籍成员,也是首位当选剑桥大学三一学院会士的印度人。

1919年,健康状况恶化——如今被认为是由多年前痢疾引发的并发症“肝阿米巴病”所致——迫使拉马努金返回印度。他于1920年去世,年仅32岁。他在1920年1月写给哈代的最后几封信表明,他在生命的最后时刻仍在不断提出新的数学思想和定理。他的“失落的笔记本”,记录了他生命最后一年中的诸多发现,于1976年被重新发现后,在数学界引起了极大的轰动。
\subsection{早年生活}
\begin{figure}[ht]
\centering
\includegraphics[width=6cm]{./figures/f833ac051ab76d62.png}
\caption{} \label{fig_LMLJ_1}
\end{figure}
拉马努金(意为“罗摩的弟弟”,罗摩是印度教神祇)于1887年12月22日出生在现今泰米尔纳德邦的伊罗德市的一个泰米尔婆罗门艾扬加尔家庭中。他的父亲库普苏瓦米·斯里尼瓦萨·艾扬加尔原籍坦贾武尔区,在一家纱丽店当职员。他的母亲科玛拉塔玛尔是家庭主妇,也在当地庙宇中唱诵圣歌。他们一家住在库姆巴科纳姆镇的萨兰加帕尼圣殿街上的一间传统小屋里,该住宅如今已被改为博物馆。

当拉马努金一岁半时,母亲生下了另一个儿子萨达戈潘,但他在出生不到三个月后夭折。1889年12月,拉马努金感染了天花,不过他幸运地康复了,而当时坦贾武尔区约有4000人死于这一年严重的天花疫情。此后,他随母亲搬到了她父母位于康契布勒姆(今切奈附近)的家中。母亲随后又分别于1891年和1894年生下两个孩子,但这两个孩子都未能活过一岁。

1892年10月1日,拉马努金被送入当地学校就读。\(^\text{[17]}\)后来,他的外祖父在康契布勒姆失去了担任法庭官员的职位,\(^\text{[18]}\)拉马努金和母亲便搬回了库姆巴科纳姆,并在那里就读于康伽延初级学校。\(^\text{[19]}\)随着祖父去世,他又被送回外祖父母位于马德拉斯(今金奈)的住所。但他不喜欢马德拉斯的学校,还经常试图逃课。他的家人甚至请了一位当地警察来监督他上学。不到六个月,拉马努金就再次回到了库姆巴科纳姆。\(^\text{[19]}\)
\begin{figure}[ht]
\centering
\includegraphics[width=6cm]{./figures/ee948be70794232c.png}
\caption{} \label{fig_LMLJ_2}
\end{figure}
由于拉马努金的父亲大部分时间在外工作,照顾他的任务主要由母亲承担,因此母子之间关系十分亲密。他从母亲那里学习传统和《往世书》,学唱宗教歌曲,参加寺庙的祈祷仪式(puja),并遵守特定的饮食习惯——这些都是婆罗门文化的一部分。\(^\text{[20]}\)在康伽延初级学校,拉马努金表现出色。1897年11月,在即将年满10岁时,他以全区最高分通过了英语、泰米尔语、地理和算术的初等考试。\(^\text{[21]}\)同年,拉马努金进入了城镇高等中学,在那里他第一次正式接触到数学。\(^\text{[21]}\)

拉马努金在11岁时就已是神童,他很快掌握了家中两位大学生房客所掌握的全部数学知识。后来,他借到了一本由 S. L. Loney 编写的《高等三角学》教材。\(^\text{[22][23]}\)到13岁时,他不仅精通了书中的内容,还自主发现了一些复杂的定理。到了14岁,他已获得了多项优异成绩证书和学术奖项,并在整个求学阶段持续获奖。他还协助学校为大约1200名学生(每人需求不同)安排约35名教师的教学任务。\(^\text{[24]}\)他总能在规定时间的一半内完成数学考试,并表现出对几何学和无穷级数的熟练掌握。1902年,拉马努金学会了解三次方程的解法,后来又发展出一套解四次方程的方法。1903年,他尝试解五次方程,却不知道此类方程无法用根式来求解。\(^\text{[25]}\)
