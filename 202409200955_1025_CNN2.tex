% 深度学习 CNN 入门 2
% keys 卷积
% license Usr
% type Wiki


\pentry{卷积\nref{nod_Conv},神经网络\nref{nod_NN},全连接网络\nref{nod_FCNN},Python 导航\nref{nod_PyFi}}{nod_c889}

本文从更基础的部分来解释CNN神经网络,上文是说的神经网络的基本结构,这篇解释的是CNN的底层逻辑。

首先需要了解一下卷积的概念。
定义:在泛函分析中,卷积是通过两个函数$f$和$g$生成第三个函数的一种数学运算,其本质是一种特殊的积分变换,表征函数$f$与$g$经过翻转和平移的重叠部分函数值乘积对重叠长度的积分。
数学表达式:对于连续函数,卷积的数学表达式通常为
\begin{equation}
(f*g)(t)=\int_{-\infty}^{+\infty}f(\tau)g(t-\tau)~.
\label{juanji}
\end{equation}
在CNN中,卷积操作主要用于特征提取。它通过滑动一个称为“卷积核”(或“滤波器”)的小型矩阵窗口在输入数据(如图像)上,进行元素级别的乘法并求和,从而生成新的特征图(Feature Map)。

假设我们有一个简单的3x3的输入矩阵(图像的一个局部区域)和一个2x2的卷积核:
输入矩阵(Input Matrix):
\begin{table}[ht]
\centering
\caption{输入矩阵}\label{tab_CNN21}
\begin{tabular}{|c|c|c|}
\hline
1 & 0 & 1 \\
\hline
2 & 1 & 0 \\
\hline
0 & 1 & 1 \\
\hline
\end{tabular}
\end{table}
卷积核(Kernel):
\begin{table}[ht]
\centering
\caption{卷积核(Kernel}\label{tab_CNN22}
\begin{tabular}{|c|c|}
\hline
1 & 0 \\
\hline
0 & 1 \\
\hline
\end{tabular}
\end{table}



​
