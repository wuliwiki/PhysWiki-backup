% 经典形核理论

在实际中,常常观察到液体凝固时,往往是一部分液体先凝固,然后凝固的部分逐渐长大,直至液体完全凝固(又称形核-长大过程);同时,如果温度只略微低于理想的热力学凝固点(熔点),液体也往往不凝固.以下简要介绍关于形核的经典理论.

\subsection{均匀形核}
液体凝固时,虽然液-固相变降低了体积自由能,但相变时新产生的相界面又提高了表面自由能,二者间存在竞争.
\begin{figure}[ht]
\centering
\includegraphics[width=14cm]{./figures/NCLT_1.png}
\caption{形核时,虽然体积自由能降低,但表面自由能升高}} \label{NCLT_fig1}
\end{figure}

假设过冷液体中形成一半径为r的固体晶胚.前后总自由能变化: 
\begin{equation}
\Delta G  = \Delta G_V +\Delta G_S
\end{equation}

\begin{itemize}
\item $\Delta G_V = \frac{4}{3}\pi r^3 \Delta G_B$是体积自由能变
\item $\Delta G_B = \Delta H \frac{\Delta T}{T_M}$ 是单位体积自由能变,其中$\Delta H$是相变焓变,$\Delta T=T_M-T$是过冷度(实际温度与理想凝固点的差值),$T_M$是熔点
\item $\Delta G_S = 4\pi r^2 \gamma$是表面自由能变
\end{itemize}

\subsubsection{临界形核半径}
随后,系统将沿自由能减少的方向自发运动.如图所示,只有当晶胚的初始半径r大于某个值时,自由能减少的方向才是晶胚自发长大的方向,否则晶胚将自发衰亡.此半径称为\textbf{临界形核半径}.
\begin{figure}[ht]
\centering
\includegraphics[width=14cm]{./figures/NCLT_2.png}
\caption{临界形核半径.\href{https://www.geogebra.org/m/prktxhhk}{一个可交互模型}(站外链接)} \label{NCLT_fig2}
\end{figure}

\begin{figure}[ht]
\centering
\includegraphics[width=8cm]{./figures/NCLT_3.png}
\caption{“生存还是毁灭,这是一个问题”} \label{NCLT_fig3}
\end{figure}

为求解临界形核半径,令 $\dv{\Delta G}{r} = 0$.解得$r_k=-\frac{2\gamma}{\Delta G_B}=-\frac{2\gamma T_M}{\Delta H \Delta T}$.可见,临界形核半径与表面张力、体积焓变、过冷度等均有关.
