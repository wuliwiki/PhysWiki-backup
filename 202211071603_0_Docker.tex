% Docker 笔记

\begin{issues}
\issueDraft
\end{issues}

\subsubsection{apt 安装}
\begin{itemize}
\item ubuntu x86/64 安装 Docker Engine 参考\href{https://docs.docker.com/install/linux/docker-ce/ubuntu/}{这里}.
\item 测试安装成功, 用 \verb`sudo docker run hello-world`
\end{itemize}

\subsubsection{deb 安装包安装}
\begin{itemize}
\item 但是如果 apt 不能用的话, 也可以下载 deb 文件离线安装
\item 安装包\href{https://download.docker.com/linux/ubuntu/dists/}{下载链接}, 选择系统版本代号, 然后 pool/stable/amd64/
\item 选好了以后需要下载三个安装包, 依次安装(如果次序不对也问题不大,会提示 dependency 找不到的错误)
\item 三个安装包依次是 \verb|containerd.io|, \verb|docker-ce-cli|, \verb|docker-ce|, 用 \verb`dpkg -i xxx.deb` 安装即可
\end{itemize}

\subsection{基础}
\begin{itemize}
\item 参考\href{https://docs.docker.com/get-started/}{官方入门}.
\item \href{https://devops.stackexchange.com/questions/2826/difference-between-chroot-and-docker}{docker 和 chroot 的对比}.
\item docker 在功能上基本和虚拟机一样, 但是占用资源要少得多, 因为共享 host 系统的内核
\item docker 本质上就是一个进程
\item windows 和 mac 上有 docker desktop, 建议使用
\item 用了 docker 就不能用 virtualbox 了, 必须要在 eufi 把 Hyper V 关掉或打开才可以
\item 最新的 windows docker 默认使用 WSL2 实现.
\item 重启以后启动 docker, 登录, 用 powershell \verb`docker --version` 检查版本
\item \verb`sudo docker run hello-world` 测试最简单的 image (镜像).
\item 所有 \verb|docker| 命令前面都默认需要 \verb|sudo|, 以下省略. 如果不加, 会有错误 \verb|Cannot connect to the Docker daemon...|. 也可以通过用户组来设置不需要 sudo 的 user.
\item 注意区分容器(container)和镜像(image)和仓库(repo). 仓库相当于 VirtualBox 虚拟机, 镜像相当于增量备份点(snapshot), 而容器相当于虚拟机的目前状态.
\item \verb`docker images` 可以检查本地所有镜像
\item \verb`docker ps -a` 可以检查本地所有 container/process
\item \verb|docker container ls| 也可以, 但区别是什么?
\item 要下载镜像用 \verb`docker pull 用户名/repo名:TAG` 其中 \verb|用户名/| 对于一些官方镜像可以忽略, 例如 \verb`docker pull ubuntu:22.04` 中, \verb|ubuntu| 是 repo, \verb|22.04| 是 TAG. \verb|docker pull hello-world| 中, tag 也忽略了, 默认的 tag 是 \verb|latest|
\item \verb`docker run -it -d 镜像` 从镜像创建容器. 反复使用会创建多个. 这里的 \verb|镜像| 是指 \verb|用户/repo名:TAG| 或者 \verb|ps| 命令里面的 \verb|IMAGE ID|, 下同.
\item \verb|docker run --name 容器名 ...| 可以指定容器的名字, 如果不指定则会随机生成一个(用 \verb|ps| 查看). 下文的 \verb|容器| 都是指容器名或者容器 ID(用 \verb|ps| 查看).
\item \verb`docker exec -it 容器 bash` 进入某个 container 的 bash. 如果该镜像的默认 user 不是 root, 那么在 \verb|-it| 前面加上 \verb|-u 0| 选项可以以 root 的身份 login.
\item docerk 镜像没有简单的上锁设置,就算修改了 passwd. 能执行 \verb`sudo docker...` 命令的人都可以自由访问本地的任何 container
\item \verb`docker stop 容器ID` 停止 container
\item \verb`docker start 容器ID` 开始 container (注意区分 \verb|start| 和 \verb|run|)
\item \verb`docker rm [-f] 容器ID` 删除 container
\item \verb`docker image rm [-f] 镜像` 删除镜像
\item \verb`docker commit 容器ID 用户名/repo名` 会将 container commit, 也可以在后面加 \verb`:TAG` 指定 tag
\item \verb`sudo docker login` 用于登录 docker hub
\item \verb`docker push 用户名/repo名` 可以直接将 commit 的 image push 到 docker hub, 也可以用 \verb`repo名:TAG` 指定 tag
\item \verb|docker save -o image.tar 镜像名| 可以把镜像保存为文件. 也可以用 \verb|镜像名:TAG|
\item \verb|docker load -i image.tar| 可以恢复保存的镜像文件
\item \verb|docker rename 旧镜像名字 新镜像名字|
\item 从正在运行的 container 立面拷贝文件到外面 \verb|docker cp 容器ID:/容器内某路径/ 本地某路径|
\item \verb|docker save -o 备份文件.tar 镜像|
\end{itemize}

\subsection{Dockerfile}
\begin{itemize}
\item dockerfile 是一系列创建 docker image 的命令. 当然也可以写一个 bash 脚本直接在 container 中运行, 但前者更灵活方便.
\item \verb|docker build -t 容器名:tag名 .| 使用当前路径下的 \verb|Dockerfile| 创建 image 并命名.
\item 如果想指定 Dockerfile 文件名, 用 \verb|-f 文件|
\item 一个例子是 mplapack 的 \verb|sudo docker build -t mplapack:ubuntu2204  .|. 其中 \verb|-t| 命名一个 tag, \verb|-f| 指定 dockerfile, \verb|.| 指定某个路径为当前路径, \verb|tee| 把 stdout 输出到命令行以及指定的文件.
\item 每个命令会建立一个 \textbf{layer}, 应该相当于 git 的一次 commit. 每个 layer 保存两个 image 之间的差别, 可以从任意 layer 建立 container.
\item 不一定每个 layer 都有 tag, 但都会有 id.
\item \verb|CMD["命令", "参数1", "参数2", ...]| 或者 \verb|CMD 命令 参数1 参数2, ...| 可以在每次 container 运行时都执行一次某个命令.
\item \verb|docker run| 执行 \verb|CMD| 中命令时, 会把命令行输出转到 host 的命令行.
\end{itemize}

具体命令
\begin{itemize}
\item \verb|# Comment| 写注释
\item dockerfile 命令一般用大写(但严格来说不区分大小写)
\item 第一个除了 \verb|ARG| 的命令必须是 \verb|FROM|, 例如 \verb|FROM image:tag|, 这和 \verb|docker pull repo:tag| 一样. 例如 \verb|FROM ubuntu:22.04|.
\item \verb|RUN 命令| 可以在 container 的 shell 中执行命令.
\item 用 \verb|\| 给命令换行
\item 也可以用 \verb|RUN 命令1 && 命令2 ...| 执行多个命令
\item \verb|COPY 本地文件/文件夹 容器内路径| 可以从外面到里面复制文件
\item \verb|ARG| 命令基本是定义 container 的环境变量. 例如 \verb|ARG ver="2.0.1"|, 之后可以用例如 \verb|myapp-${ver}| 来获替换成值(双引号不会包含).
\end{itemize}


\subsection{docker 文件夹映射}
\begin{itemize}
\item 如果想从 docker 中访问本地文件夹, 就用 \verb|docker run -v 本地目录:docker中的目录|
\end{itemize}

\subsection{docker 网络端口映射}
\begin{itemize}
\item \href{https://www.freecodecamp.org/news/how-to-get-a-docker-container-ip-address-explained-with-examples/}{简单介绍}
\item container 自己的 ip 默认为 \verb`172.17.xxx.xxx`
\item 如果要把 container 的 port 映射到 host 的 port, 使用 \verb`docker run -p HOST_PORT:CONTAINER_PORT IMAG_NAME`, 这样如果 docker 有一个 web server 就可以从外面访问了
\end{itemize}

\subsection{ubuntu image 缺少的功能}
\begin{itemize}
\item \verb|docker pull ubuntu| 安装官方 ubuntu docker 镜像, 这是一个非常精简的系统, 比 ubuntu server 还精简, 只有 30 多 M.
\item \verb|apt update|
\item \verb`apt install sudo`
\item \verb`apt install bash-completion`, 然后在 bashrc 中加上
\begin{lstlisting}[language=bash]
if ! shopt -oq posix; then
  if [ -f /usr/share/bash-completion/bash_completion ]; then
    . /usr/share/bash-completion/bash_completion
  elif [ -f /etc/bash_completion ]; then
    . /etc/bash_completion
  fi
fi
\end{lstlisting}
\end{itemize}

\subsection{X11 转发}
\begin{itemize}
\item 参考\href{https://opendata-forum.cern.ch/t/x11-forwarding-with-docker/31}{这篇文章}.
\end{itemize}
