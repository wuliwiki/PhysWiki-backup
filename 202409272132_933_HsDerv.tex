% 导数(高中)
% keys 导数|高中|变化|求导
% license Usr
% type Tutor

\begin{issues}
\issueDraft
\end{issues}

一点的导数
\begin{equation}
f'(x_0)=\lim_{x_1\to x_0}{f(x_1)-f(x_0)\over x_1-x_0}~.
\end{equation}

导数也是一个对应关系,即每个自变量都对应一个导数,因此他也是一个函数,这个函数称为\textbf{导函数,简称为导数}。导函数和原本的函数是一一对应的,因此可以根据定义或求导方法,来求一个函数的导函数,这个过程就是\textbf{求导}。