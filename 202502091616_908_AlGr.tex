% 亚历山大·格罗滕迪克(综述)
% license CCBYSA3
% type Wiki

本文根据 CC-BY-SA 协议转载翻译自维基百科\href{https://en.wikipedia.org/wiki/Alexander_Grothendieck#Mathematical_work}{相关文章}。

\begin{figure}[ht]
\centering
\includegraphics[width=6cm]{./figures/a1640137f4bbaf24.png}
\caption{1970年,亚历山大·格罗滕迪克在蒙特利尔。} \label{fig_AlGr_1}
\end{figure}
亚历山大·格罗滕迪克(后来的法语名为亚历克斯·格罗滕迪克,发音:/ˈɡroʊtəndiːk/;德语发音:[ˌalɛˈksandɐ ˈɡʁoːtn̩ˌdiːk] ⓘ;法语发音:[ɡʁɔtɛndik]),1928年3月28日出生,2014年11月13日去世,是一位出生于德国的法国数学家,他在现代代数几何的创立中成为了主要人物。他的研究拓展了该领域的范围,并将交换代数、同调代数、层理论和范畴理论等元素融入了其基础中,而他所谓的“相对”视角则在纯数学的许多领域带来了革命性的进展。许多人认为他是二十世纪最伟大的数学家。

格罗滕迪克于1949年开始了他富有成效且公开的数学家生涯。1958年,他被任命为高等科学研究院(IHÉS)的研究教授,并一直在那里工作,直到1970年,由于个人和政治信念,他因与军事资金的争执而离开。1966年,他因在代数几何、同调代数和K理论方面的突破而获得了菲尔兹奖。他后来成为蒙彼利埃大学的教授,并在继续进行相关数学研究的同时,逐渐退出了数学界,投身于政治和宗教事务(最初是佛教,后来转向更为天主教的基督教观点)。1991年,他搬到了位于比利牛斯山脉的法国小村庄拉塞尔,在那里他过上了隐居生活,仍然致力于数学及其哲学和宗教思想,直至2014年去世。
\subsection{生平}  
\subsubsection{家庭与童年}  
格罗滕迪克出生于柏林,父母为无政府主义者。他的父亲亚历山大·“萨沙”·沙皮罗(也叫亚历山大·塔纳罗夫)有哈西德犹太血统,曾在俄罗斯被囚禁,后于1922年移居德国;他的母亲约翰娜·“汉卡”·格罗滕迪克来自汉堡的一个新教德国家庭,并且是一名记者。[a] 两位父母在青少年时期都脱离了他们的早期背景。[16] 在格罗滕迪克出生时,他的母亲与记者约翰内斯·拉达茨结婚,最初,他的出生名字被记录为“亚历山大·拉达茨”。该婚姻在1929年解除了,沙皮罗承认了自己的父亲身份,但并未与汉卡·格罗滕迪克结婚。[16] 格罗滕迪克有一位母亲那边的兄弟姐妹——同父异母的妹妹麦迪。

格罗滕迪克与父母一起生活在柏林,直到1933年底,父亲为了躲避纳粹主义而搬到巴黎,母亲随之而后。格罗滕迪克被交由威廉·海多恩照料,海多恩是一位路德教牧师和汉堡的教师。[17][18] 据温弗里德·沙尔劳称,在此期间,格罗滕迪克的父母作为非战斗辅助人员参与了西班牙内战。[19][20] 然而,也有其他人表示沙皮罗曾在无政府主义民兵中作战。[21]
\subsubsection{第二次世界大战}  
1939年5月,格罗滕迪克被从汉堡送上前往法国的火车。不久后,他的父亲被关押在勒维尔内(Le Vernet)集中营。[22] 他和母亲随后在1940年至1942年间作为“危险的外国人”被关押在不同的集中营。[23] 第一个营地是里厄克罗斯营地(Rieucros Camp),在那里,他的母亲感染了结核病,这种疾病最终导致她在1957年去世。在那里,格罗滕迪克设法上了当地的学校——孟德尔学校(Mendel)。有一次,他设法从营地逃脱,打算刺杀希特勒。[22] 后来,他的母亲汉卡被转移到居尔斯集中营,直到第二次世界大战结束。[22] 格罗滕迪克被允许与母亲分开生活。[24]

在勒尚邦-sur-Lignon村,他在当地的寄宿家庭或旅馆中得到了庇护和隐藏,尽管有时他必须在纳粹突袭期间躲进树林里,几天没有食物和水也能活下来。[22][24]

他的父亲在维希政府的反犹法令下被逮捕,并被送到德朗西集中营,随后由维希政府交给德国人,被送往奥斯维辛集中营,在1942年被杀害。[8][25]

在勒尚邦,格罗滕迪克就读于塞文学院(Collège Cévenol,现在被称为Le Collège-Lycée Cévenol International),这是一所由当地新教和平主义者和反战活动家于1938年创办的独特中学。许多在勒尚邦藏匿的难民儿童都在塞文学院就读,正是在这所学校里,格罗滕迪克显然第一次对数学产生了浓厚的兴趣。[26]

1990年,因冒着生命危险拯救犹太人,整个村庄被认定为“国际义人”(Righteous Among the Nations)。
\subsubsection{学业与接触研究数学}  
战争结束后,年轻的格罗滕迪克在法国学习数学,最初在蒙彼利埃大学,起初他的表现不佳,甚至在天文学等课程上不及格。[27] 他开始独立学习,重新发现了勒贝格测度。在那里经过三年的越来越独立的学习后,他于1948年继续前往巴黎深造。[17]

最初,格罗滕迪克参加了亨利·卡尔坦(Henri Cartan)在巴黎高等师范学校(École Normale Supérieure)的研讨会,但由于缺乏必要的背景,他无法跟上这个高水平研讨会的进度。在卡尔坦和安德烈·韦伊(André Weil)的建议下,他转到南锡大学(University of Nancy),那里有两位领先的专家在研究格罗滕迪克感兴趣的领域——拓扑向量空间:让·迪厄多内(Jean Dieudonné)和洛朗·施瓦茨(Laurent Schwartz)。后者刚刚获得了菲尔兹奖。迪厄多内和施瓦茨向这位新来的学生展示了他们最新的论文《空间(F)与(LF)中的对偶性》(La dualité dans les espaces (F) et (LF));论文末尾列出了14个开放问题,涉及局部凸空间。[28] 格罗滕迪克引入了新的数学方法,使他能够在几个月内解决所有这些问题。[29][30][31][32][33][34][35]

在南锡,他在这两位教授的指导下写了他的博士论文,内容是关于泛函分析,从1950年到1953年。[36] 在此期间,他成为了拓扑向量空间理论的领先专家。[37] 1953年,他移居巴西圣保罗大学,凭借难民护照(Nansen护照),因为他拒绝加入法国国籍(因为那样会要求他服兵役,而这与他的信念相悖)。他在圣保罗呆到1954年底(除了1953年10月到1954年3月期间在法国的长时间访问)。他在巴西期间发表的工作仍然集中在拓扑向量空间理论;正是在那里,他完成了关于这一主题的最后一项重要工作——关于巴拿赫空间的“度量”理论。