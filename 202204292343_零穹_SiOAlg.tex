% 单一算子生成的子代数
% 子代数|极小多项式|幂零算子

\pentry{线性算子代数\upref{LiOper}}
已经知道,定义在域 $\mathbb{F}$ 上的向量空间 $K$,如果它同时是一个环,它就是一个代数\autoref{LiOper_sub3}~\upref{LiOper}.如果这个向量空间 $K$ 的子空间 $L$ 对于 $K$ 作为环的乘法封闭,那么 $L$ 就称为代数 $K$ 的\textbf{子代数}.

现在要研究包含算子 $\mathcal{A}$ 的最小子代数(且含环的单位元 $\mathcal{E}$),这个子代数记作 $\mathbb{F}[\mathcal A]$ (之所以这样记,是因为这个子代数是域 $\mathbb{F}$ 上的多项式环\autoref{Ring_ex1}~\upref{Ring}).

要找包含算子 $\mathcal{A}$ 的最小子代数 $\mathbb{F}[\mathcal A]$ ,可以这样思考:首先考虑这个子代数是个向量空间并且含有 $\mathcal{E}$,那么由算子 $\mathcal{A,E,O}$ 进行向量空间的加法和数乘得到的元素形为 $a_0\mathcal{E},a_1\mathcal{A}\;\forall a_0,a_1\in\mathbb{F}$ .现在,这个子代数至少包含 $a_0\mathcal{E},a_1\mathcal{A}\;\forall a_0,a_1\in\mathbb{F}$ .考虑子代数也是个环,那么由算子 $a_0\mathcal{E},a_1\mathcal{A}\;\forall a_0,a_1\in\mathbb{F}$ 进行环的乘法得到的元素形为 $a_0\mathcal{E},a_1\mathcal{A},a_2\mathcal{A}^2\;\forall a_0,a_1,a_2\in\mathbb{F}$,所以现在这个子代数至少包含算子 $a_0\mathcal{E},a_1\mathcal{A},a_2\mathcal{A}\;\forall a_0,a_1,a_2\in\mathbb{F}$;再考虑它是个向量空间及环,如此重复可得到这个子代数至少包含所有下面形式的元素(注意 $\mathcal{A}^0=\mathcal{E}$)
\begin{equation}\label{SiOAlg_eq1}
f(\mathcal{A})=\sum_{i=0}^na_i\mathcal{A^i},\;n\in\mathbb{N}
\end{equation}
易验证,所有形如\autoref{SiOAlg_eq1} 的元素构成一个子代数,这个子代数便是要找的 $\mathbb{F}[\mathcal A]$ ,称为由\textbf{算子 $\mathcal{A}$ 生成的子代数}.
\subsection{代数 $\mathbb{F}[\mathcal{A}]$ 的交换性}
代数 $\mathbb{F}[\mathcal A]$ 是交换的,因为 $\mathcal A^{k}\cdot \mathcal{A}^l=\mathcal{A}^{k+l}=\mathcal A^{l}\cdot \mathcal{A}^k$(利用 $\mathcal{A}$ 本身的交换性和结合性易证该性质),易验证 $f(\mathcal{A})g(\mathcal{A})=g(\mathcal{A})f(\mathcal{A})$.

与线性算子作用在向量上的方式\upref{LiOper}一样, $f(\mathcal{A})$ 通过以下方式作用在 $\bvec x\in V$ 上:
\begin{equation}
f(\mathcal{A})\bvec x=a_0\bvec x+\sum_{i=1}^n a_i\mathcal{A}^i\bvec x
\end{equation}
