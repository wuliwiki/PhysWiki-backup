% 序关系
% 序|偏序|全序|良序|关系

\begin{issues}
\issueAbstract
\end{issues}

\pentry{二元关系\upref{Relat}}

法国的布尔巴基学派曾对数学里的结构进行梳理,并且提炼出了三大基本结构:代数结构,拓扑结构以及序结构,序结构通常是一个装备了序关系的集合,它在组合数学,分析学中有着奠基的作用。

\subsection{1.偏序关系}
\textbf{偏序关系(partially ordering relation)}是一类二元关系。称在非空集合 $A$ 上关系$\leq$ 是一个偏序关系,如果$\leq$满足条件:
\begin{enumerate}
\item \textbf{自反性}:$\forall x\in A,x\leq x$;
\item \textbf{反对称性}:$\forall x,y\in A,\ x\leq y,\ y\leq x \Rightarrow x = y $;
\item \textbf{传递性}:$\forall x,y,z\in A,\ x\leq y,\ y\leq z \Rightarrow x\leq z $。
\end{enumerate}
我们称$\leq$是$A$上的偏序关系,并且把赋予偏序结构$\leq$的集合$A$,即$(A;\leq)$称为偏序集(半序集,有序集,序集)
\begin{example}{}
各种数集(包括 $\mathbb{N},\mathbb{Z},\mathbb{Q},\mathbb{R}$)按通常的序关系(大小关系)构成偏序集。

注解:一般我们不提及$\mathbb{C}$上的序关系,因为从$\mathbb{R}$中拓展得到的$\mathbb{C}$上的序关系性质并不良好,但有必要时也可以在$\mathbb{C}$上定义一些特别的序关系来便于研究
\end{example}

这个例子说明序关系实际上是数与数大小关系的抽象。

\begin{example}{整除序}
$\mid$是自然数集$\mathbb{N}$上的一个可除关系, 即$m \mid n$表示 $ m$整除$n$.则$(\mathbb{N};\mid)$是一个偏序集.
\end{example}

\begin{example}{包含序}
设  $E$  是一个集合,  $\mathbb{P}(E)$  表示 $ E  $的所有子集 (包括空集 $ \varnothing  $) 组成 的集合, 称为 $ E $ 的幂集.定义  $\mathbb{P}(E)$  上的包含关系  $\subseteq $ 为序关系. 则  $(\mathbb{P}(E) ; \subseteq) $ 是 一个有序集.
\end{example}

\begin{example}{有限偏序集的例子}
设集合$A = \{a,b,c\}$,关系$\leq\!= \{(a,a),(a,b),(a,c),(b,b),(c,c)\}$,可以验证 $\leq$ 是 $A$ 上的偏序关系。
%我们可以用下列的图示来表示 $A$ 及其上的偏序关系 $\leq$
\end{example}

\begin{example}{字典序}\label{OrdRel_ex1}
已知 $(A,\prec),(B,\leq)$ 是两个偏序集,那么笛卡尔积 $A\times B$ 按某个序关系 $\leqslant$ 构成偏序集。这个序关系 $\leqslant$ 满足:
\begin{enumerate}
\item $\forall(a_1,b_1),(a_2,b_2) \in A\times B, a_1\prec a_2 \Rightarrow (a_1,b_1)\leqslant(a_2,b_2)$;
\item $\forall(a_1,b_1),(a_2,b_2) \in A\times B, a_1=a_2, b_1\leq b_2 \Rightarrow (a_1,b_1) \leqslant (a_2, b_2)$。
\end{enumerate}

由于字典通常按照这样的顺序编排\footnote{比如,字典中单词in在a后,在it前。},因此这种序关系称为字典序。
\end{example}

\begin{example}{小时百科词条序}
小时百科的知识树(目录树)可以看做一个偏序集,"词条预备知识"是这个偏序集的序关系,词条A在词条B前意味着词条A是词条B的预备知识
\end{example}

\subsection{1.exp。序的对偶原理}
从偏序的定义来看,我们发现它是有方向性的,如果我们把方向取反,自然而然能得到一个和原偏序有着千丝万缕关系的新偏序,也就是偏序的对偶关系。

\begin{lemma}{序对偶}
  设$A$是一个非空集合. 若$R$是$E$的一个序关系, 则$R$的对偶关系$R^{d}$也是$A$的一个序关系
\end{lemma}
证明留作读者自行完善。

我们将以上引理换一种方式叙述就得到了这条神奇的玩意儿:
\begin{theorem}{对偶原理}
 设$A$是一个非空集合. 则关于$(A;\leq)$的任意一个命题,都存在与之相对应的$(A;\geq)$上的一个命题
\end{theorem}

于是大多数情况下,我们考虑一对互为对偶的序关系时只用考虑其中一个(如同考虑非交换结构时只考虑左一样)

(以下内容待修整,修整原因:部分符号与序论符号不一致,顺序不合理)
\subsection{全序集}

\begin{definition}{全序集}
偏序集 $(A;\leq)$ 称为\textbf{全序集(totally ordered set)}或\textbf{有序集(ordered set)},当且仅当对任意 $a,b \in A$,$a \leq b$ 或 $b \leq a$。
\end{definition}
全序集上的序关系称为\textbf{全序关系(totally ordered relation)}。
\begin{example}{}
数集 $\mathbb{N},\mathbb{Z},\mathbb{Q},\mathbb{R}$ 按通常的序关系构成全序集。
\end{example}
\begin{example}{}
将\autoref{OrdRel_ex1} 中的偏序集 $A,B$ 换成全序集,则按同样的序关系,$A\times B$ 也构成全序集。
\end{example}

一个显然但重要的事实是,任何偏序集的非空子集按原来的序仍是偏序集,任何全序集的非空子集按原来的序仍是全序集。

\begin{definition}{极小元、极大元}
已知偏序集 $(A;\leq)$:
\begin{enumerate}
\item 如果存在某个 $a \in A$,使得对任意 $x \in A, x\leq a$ 都有 $x = a$,那么称 $a$ 为偏序集 $A$ 的\textbf{极小元(minimal element)}。
\item 如果存在某个 $b \in A$,使得对任意 $x \in A, b\leq x$ 都有 $x = b$,那么称 $b$ 为偏序集 $A$ 的\textbf{极大元(maximal element)}。
\end{enumerate}
\end{definition}

\begin{definition}{最小元、最大元}
已知偏序集 $(A;\leq)$ 的某个非空子集 $B$:
\begin{enumerate}
\item 如果存在某个 $a \in B$,使得 $\forall x \in B, a \leq x$。那么 $a$ 被称为 $B$ 的\textbf{最小元(least element)};
\item 如果存在某个 $b \in B$,使得 $\forall x \in B, x \leq b$。那么 $b$ 被称为 $B$ 的\textbf{最大元(greatest element)}。
\end{enumerate}
\end{definition}

极大元、极小元与最大元、最小元极易混淆。偏序集的极大元、极小元不一定唯一,但最大元、最小元只要存在必然唯一。在全序集中两者统一。\footnote{读者自证。}

\begin{example}{}
集合 $A=\{0,1,2\}$,在上面定义偏序关系 $\leq$ 为 $0\leq 0$, $1\leq 1$, $1\leq 2$, $2\leq 2$。那么子集 $\{0,1\}$ 的极大元为0和1,不存在最大元。
\end{example}

\begin{exercise}{}
证明:如果偏序集的某个子集存在最大元,那么它的极大元必然存在且唯一,并且两者相等。
\end{exercise}

\begin{exercise}{}
证明:对于全序集的任意非空子集,如果存在极大元,则最大元必然存在,且两者相等。
\end{exercise}

\subsection{良序集}

\begin{definition}{良序集}
全序集 $(A,\leq)$ 称为\textbf{良序集(well-ordered set)},当且仅当它的任意非空子集都有极小元。
\end{definition}

良序集上的序关系称为\textbf{良序关系(well-ordered relation)}。

任何良序集的非空子集仍是良序集。