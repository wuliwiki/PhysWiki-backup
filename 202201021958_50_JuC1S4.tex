% 改进第一个程序
% 改进 程序 初步

本文授权转载自郝林的 《Julia 编程基础》. 原文链接:\href{https://github.com/hyper0x/JuliaBasics/blob/master/book/ch01.md}{第1章:起步}.

我们应该对上述程序的功能稍作改进.因为它现在只能向Julia打招呼,不论执行它的人是谁.我们需要让它根据执行人给定的参数值来自定义它打招呼的对象.顺便说一句,我会把这一程序的改进版本放在\verb|src/ch01/args|路径下.

首先,我们要改变一下调用\verb|println|函数时传给它的那个参数.修改后的调用表达式如下:
\begin{lstlisting}[language=julia]
println("Hey, $(name)!")
\end{lstlisting}
我只改动了几个字符,即:把Julia改成了\verb|$(name)|.后者代表了一个插值(interpolation).对于插值来说,前缀\verb|$|(和后缀)之间的内容可以是一个变量的名称,也可以是一个表达式.在这里,我放入的是变量name的名字.在\verb|println|函数向目的地输送内容之前,它会把name替换成该变量在那一刻时的值.
