% 域上的代数
% keys 代数|环|域|乘法|群

\pentry{矢量空间\upref{LSpace},环\upref{Ring}}

“代数(algebra)”一词,有两个含义.

第一个含义是指一个数学分支,代数学.代数学研究集合中各种各样的运算结构,我们从小学就开始涉及了.一次方程的移项、两边同乘等操作都是代数学研究的性质;理工科到了本科还必须研究线性代数(或称高等代数),作为大量理工学科的数学基础;此外,本部分“抽象代数”则研究了更为基础的一些代数学,但抽象代数本身也只是代数学这门广博学科的起点.

第二个含义,是代数学中研究的一种数学对象,代数.简单来说,代数就是一种定义了向量乘法的向量空间.当然,乘法的性质要具体讨论.



\begin{definition}{狭义的代数定义}
设$A$是域$K$上的向量空间,若在$A$中再定义代数乘法$\circ$,使得$(A,+,\circ)$成为环,并且$\forall a\in K, u, v\in A$有
\begin{equation}
a(u \circ v)=(a u) \circ v=u{\circ}(a v)
\end{equation}
则称$A$为\textbf{域$K$上的代数},简称\textbf{(结合)代数}.

\end{definition}

如果没有特别说明,一个代数一般是指狭义的结合代数,也就是说,是在向量空间里定义向量的乘法,使得它还能构成一个环.当然,乘法只是一种运算,它不一定是结合的.如果所定义的乘法不是结合的,那么我们称这样的结构是一个\textbf{非结合代数}.

下面我们来看几个例子,加深对代数这一概念的理解.

\begin{example}{矩阵代数}
域$\mathbb{F}$上的$n\times n$的矩阵的全体 $\mathrm{GL}(n, \mathbb F)$ (\autoref{Group_ex5}~\upref{Group})在矩阵的加法,标量与矩阵的相乘运算下构成一个线性空间,在矩阵的加法和矩阵的乘法运算下构成一个环,因此它是一个结合代数.
\end{example}

\begin{example}{非结合代数}
在$3$维向量空间中,以两个向量的叉积$\mathbf A\times \mathbf B$来定义它们的代数乘法运算,则它们构成一个代数,此时有
\begin{equation}
\begin{aligned}
&\mathbf{A} \times \mathbf{A}=0\\
&\mathbf{A} \times \mathbf{B}=-\mathbf{B} \times \mathbf{A}\\
&\mathbf{A} \times(\mathbf{B}+\mathbf{C})=\mathbf{A} \times \mathbf{B}+\mathbf{A} \times \mathbf{C}\\
&(\bvec B+\mathbf{C}) \times \mathbf{A}=\mathbf{B} \times \mathbf{A}+\mathbf{C} \times \mathbf{A}
\end{aligned}
\end{equation}
以及
\begin{equation} \label{AlgFie_eq1}
(\mathbf{A} \times \mathbf{B}) \times \mathbf{C}+(\mathbf{B} \times \mathbf{C}) \times \mathbf{A}+(\mathbf{C} \times \mathbf{A}) \times \mathbf{B}=0
\end{equation}
\end{example}

通常把\autoref{AlgFie_eq1} 称为\textbf{雅可比恒等式}. 具有这种性质的代数称为\textbf{李代数(Lie Algebra)}.我们会在将来详细讨论李代数.

\begin{example}{}
$\mathrm{gl}(n, \mathbb C)$除矩阵的加法及数乘外,再定义代数乘法$\mathbf A\circ \mathbf B=[\mathbf A, \mathbf B] = \mathbf A \mathbf B - \mathbf B\mathbf A, \mathbf A, \mathbf B\in \mathrm{gl}(n, \mathbb C)$.显然$\mathrm{gl}(n,\mathbb C)$构成一个李代数.
\end{example}

\begin{example}{}
对于域$K $及有限群$G=\{g_1,g_2,\cdots,g_n\}$,我们可构成\textbf{群代数}$\displaystyle A(G)=\left\{u | u=\sum a^{i} g_{i}, a^{i} \in K\right\}$,其中加法为
\begin{equation}
u+v=\sum a^{i} g_{i}+\sum b^{i} g_{i}=\sum\left(a^{i}+b^{i}\right) g_{i} \notag
\end{equation}
数乘为
\begin{equation}
a u=\sum\left(a a^{i}\right) g_{i}, a \in K \notag
\end{equation}
而代数乘法为
\begin{equation}
u \circ v=\left(\sum a^{i} g_{i}\right) \circ\left(\sum b^{i} g_{i}\right)=\sum_{i, j} a^{i} b^{j}\left(g_{i} g_{j}\right) \notag
\end{equation}
这是一个结合代数.
\end{example}

下面一个例子需要一些分析力学中的内容.如果不知道分析力学也没关系,记住即可.
\begin{example}{}
在分析力学中,正则变量记为$p_i,q_i, i=1,2,\cdots, s$.函数$u(p,q)$和$v(p, q)$的\textbf{泊松括号}定义为\begin{equation}
\{u, v\}=\sum_{i=1}^{s}\left(\frac{\partial u}{\partial q_{i}} \frac{\partial v}{\partial p_{i}}-\frac{\partial u}{\partial p_{i}} \frac{\partial v}{\partial q_{i}}\right)
\end{equation}
于是我们得到了一个李代数.
\end{example}
