% 2011 年考研数学试题(数学一)
% keys 考研|数学
% license Copy
% type Tutor

\textbf{声明}:“该内容来源于网络公开资料,不保证真实性,如有侵权请联系管理员”

\begin{enumerate}
\item 曲线 $y=(x-1)(x-2)^2(x-3)^2(x-4)^2$ 的拐点是  ($\quad$)\\
(A) $(1,0)$\\
(B) $(2,0)$\\
(C) $(3,0)$\\
(D) $(4,0)$
\item 设数列 $\{a_n\}$ 单调减少, $\displaystyle \lim_{n\to\infty} a_n=0,S_n=\sum_{k=1}^{n}a_k(n=1,2,\dots)$  无界,则幂级数 $\displaystyle \sum_{n=1}^\infty a_n(x-1)^n$ 的收敛域为($\quad$)\\
(A) $(-1,1]$\\
(B) $[-1,1)$\\
(C) $[0,2)$\\
(D) $(0,2]$
\item  设函数 $f(x)$ 具有二阶连续导数,且 $f(x)>0,f'(0)=0$  ,则函数  $z=f(x)\ln f(y)$ 在点 $(0,0)$ 处取得极小值的一个充分条件是 ($\quad$)\\
(A) $f(0)>1,f''(0)>0$\\
(B) $f(0)>1,f''(0)<0$\\
(C) $f(0)<1,f''(0)>0$\\
(D) $f(0)<1,f''(0)<0$
\item 设 $\displaystyle I=\int_{0}^{\frac{\pi}{4}}\ln (\sin x)\dd{x},J=\int_{0}^{\frac{\pi}{4}}\ln (\cot x)\dd{x},K=\int_{0}^{\frac{\pi}{4}}\ln (\cos x)\dd{x}$ ,则 $I,J,K$ 的大小关系为 ($\quad$)\\
(A) $I<J<K$\\
(B) $I<K<J$\\
(C) $J<I<K$\\
(D) $K<J<I$
\item 设 $\mat A$ 为3阶矩阵,将 $\mat A$ 的第2列加到第1列得到矩阵 $\mat B$ ,再交换 $\mat B$ 的第2行与第3行得单位矩阵。记 $\mat P_1=\pmat{1&0&0\\1&1&0\\0&0&1},\mat P_2=\pmat{1&0&0\\0&0&1\\0&1&0}$ ,则 $\mat A$=($\quad$)\\
(A) $\mat {P_1,P_2}$\\
(B) $\mat {P_1^{-1},P_2}$\\
(C) $\mat {P_2,P_1}$\\
(D) $\mat {P_2,P_1^{-1}}$
\item  设 $\mat{A=(\alpha_1,\alpha_2,\alpha_3,\alpha_4)}$ 是4阶矩阵,$A^*$  为 $A$ 的伴随矩阵。若 $(1,0,1,0) \Tr$ 是方程组 $\mat {Ax=0}$ 的一个基础解系,则 $\mat {A^*=0}$ 的基础解系可为 ($\quad$)\\
(A) $\mat {\alpha_1,\alpha_3}$\\
(B) $\mat {\alpha_1,\alpha_2}$\\
(C) $\mat {\alpha_1,\alpha_2,\alpha_3}$\\
(D) $\mat {\alpha_2,\alpha_3,\alpha_4}$
\item  设 $F_1(x)$ 与 $F_2(x)$ 为两个分布函数,其相应的概率密度 $f_1(x)$ 与 $f_2(x)$ 是连续函数,则必为概率密度的是 ($\quad$)\\
(A) $f_1(x)f_2(x)$\\
(B)  $2f_2(x)F_2(x)$\\
(C) $f_1(x)F_2(x)$\\
(D) $f_1(x)F_2(x)+f_2(x)F_1(x)$
\item 设随机变量 $X$ 与 $Y$ 相互独立,且 $E(X)$ 与 $E(Y)$ 存在,记 $U=\max \{X,Y\},V=\min {X,Y}$ ,则 $E(UV)$=($\quad$)\\
(A) $E(U)$· $E(V)$\\
(B)  $E(X)$· $E(Y)$\\
(C) $E(U)$· $E(Y)$\\
(D) $E(X)$· $E(V)$\\
\end{enumerate}
\subsection{填空题}
\begin{enumerate}
\item 曲线 $y=\int_{0}^{x} \tan t\dd{t} \quad (0 \le x \le \frac{\pi}{4})$  的弧长 $s$=($\quad$) 
\item 微分方程 $y'+y=e^{-x}\cos x$ 满足条件 $y(0)=0$  的解为 $y$=($\quad$) 
\item  设函数 $\displaystyle F(x,y)=\int_0^{xy} \frac{\sin t}{1+t^2}\dd{t}$ ,则 $\displaystyle {\eval{ \pdv[2]{F}{x}}_{x=y=2}}$=($\quad$)
\item  设 $L$ 是柱面 $x^2+y^2=1$ 与平面 $z=x+y$ 的交线,从 $z$ 轴正向往 $z$ 轴负向看去为逆时针方向,则曲线积分 $\displaystyle \oint_L xz\dd{x}+x\dd{y}+\frac{y^2}{2}\dd{z}$ =($\quad$)
\item  若二次曲面的方程  $x^2+3y^2+z^2+2axy+2xz+2yz=4$ 经正交变换为 $y_1^2+4z_1^2=4$ ,则 $a$=($\quad$)
\item  设二维随机变量 $(X,Y)$ 服从正态分布 $N(\mu,\mu;\sigma^2,\sigma^2;0)$ ,则 $E(XY^2)$= ($\quad$)
\end{enumerate}
\subsection{解答题}
\begin{enumerate}
\item 求极限  $\displaystyle \lim_{n\to\infty}[\frac{\ln (1+x)}{x}]^{\frac{1}{e^x-1}}$.
\item 设函数 $z=f(xy,yg(x))$ ,其中函数 $f$ 具有二阶连续偏导数,函数 $g(x)$ 可导,且在 $x=1$ 处取得极值 $g(1)=1$ 。求 $\displaystyle \eval{\pdv{z}{x}{y}}_{x=1,y=1}$。
\item 求方程 $k\arctan x -x=0$ 不同实根的个数,其中$ k$为参数。
\item (1) 证明:对任意的正整数 $n$,都有 $\frac{1}{n+1}<\ln(1+\frac{1}{n})<\frac{1}{n}$ 成立;\\
(2)设 $\displaystyle a_n=1+\frac{1}{2}+\dots+\frac{1}{n}-\ln n\quad (n=1,2,\dots )$  ,证明数列 $\{a_n\}$ 收敛。
\item 已知函数 $f(x,y)$ 具有二阶连续偏导数,且 $f(1,y)=0,f(x,1)=0,\iint_D f(x,y)\dd{x}\dd{y}=a$  ,其中 $D={(x,y)|0\le x \le 1,0 \le y \le 1}$ ,计算二重积分 $\displaystyle I=\iint_D xyf''_{xy}\dd{x}\dd{y}$
\item 设向量组  $\mat \alpha_1=(1,0,1) \Tr,\mat \alpha_2=(0,1,1)\Tr,\mat \alpha_3=(1,3,5)\Tr$  不能由向量组 $ \mat \beta_1=(1,1,1) \Tr,\mat \beta_2=(1,2,3) \Tr,\mat \beta_1=(1,1,1) \Tr$ 线性表示。
\end{enumerate}