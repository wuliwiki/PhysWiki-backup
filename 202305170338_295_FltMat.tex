% 双精度和变精度浮点数测试(Matlab)

\begin{issues}
\issueDraft
\end{issues}

\pentry{计算机算数\upref{CmArit}, Matlab 符号计算和变精度计算\upref{MatSym}}

另见 “双精度和高精度浮点数测试(C++)\upref{FltCpp}”。

\begin{itemize}
\item 双精度的小数部分有 52 bit(整数部分默认取 1), 正负号 1 bit。 大概范围是 $[-2,2]$
\item 2 为底的指数部分有 11 bit。 范围是 $-1022,\dots,1023$, 因为 $-1023, 1024$ 用来表示 \verb|inf, nan| 等。
\end{itemize}

所以理论上, 双精度类型能精确表示绝对值小于等于 \verb|2^53| 的所有整数。 更大的浮点数也都精确表示整数, 但不连续。 这包括所有 32 bit 整数, 但不包括所有 64 bit 整数, 无论 signed 还是 unsigned。

Matlab 默认用 double 作为数组 index, 而 53 bit 完全足够了, 如果一个元素占一个字节, 那么可以最大支持 9007TB 的数组。

\begin{lstlisting}[language=matlab, caption=digits2.m]
% 求当前 vpa 变精度计算的有效数字
% digi_dec 是十进制有效数字
% digi_bin 是二进制有效数字
% eps 是最大相对精度
function [digi_dec, digi_bin, eps] = digits2
n = 0;
while true
    n = n + 1;
    if (1 + 2^(-vpa(n))) - 1 == 0
        break;
    end
end
digi_dec = n * log10(2) + 1;
digi_bin = n + 1;
eps = 2^(-n);
end
\end{lstlisting}

\begin{lstlisting}[language=matlab, caption=num2bin.m]
% x = a*2^Npw (exactly)
% 1 <= a < 2, 精度为 digits2() 输出的 eps
% 若 x 是双精度, 则 a 的绝对精度为 2^(-52), 也就是 eps
% 若 x 是单精度, 则 a 的绝对精度为 2^(-23), 也就是 eps('single')
% 最小的双精度为 2^(-52)*2^(-1022), 最大的双精度为 (2-eps)*2^1023
function [a, Npow] = num2bin(x)
if isnan(x) || isinf(x)
    a = x; Npow = x; return;
end
sgn = sign(x); x = abs(num2sym(x));
Npow = floor(log2(x));
a = sgn * (x * 2^(-Npow));
if a*2^Npow - sgn*x ~= 0
    error('something wrong!');
end
end
\end{lstlisting}

\begin{lstlisting}[language=matlab, caption=num2bin2.m]
% 精确解出 x = N*2^Npow, 其中 N 是奇数和整数, Npow 是整数
function [N, Npow] = num2bin2(x)
if isnan(x) || isinf(x)
    N = x; Npow = x; return;
end
if x == 0
    N = 0; Npow = 0; return;
end
N = x;
Npow = 0;
for e = [32, 16, 8, 4, 2, 1]
    fac = 2^e;
    while true
        if mod(N, 1) == 0
            break;
        end
        N = N * fac;
        Npow = Npow - e;
    end
    while true
        if mod(N, fac) == 0
            N = N / fac;
            Npow = Npow + e;
        else
            break;
        end
    end
end
end
\end{lstlisting}

\begin{lstlisting}[language=matlab, caption=num2sym.m]
% 双精度转换为符号表达式
% sym(x) 会对 x 进行猜测和近似,该函数不会
function s = num2sym(x)
[N, Npow] = num2bin2(x);
s = sym(N)*2^sym(Npow);
end
\end{lstlisting}

\begin{lstlisting}[language=matlab, caption=num2vpa.m]
% 双精度转换为变精度浮点数
% vpa(x) 会对 x 进行猜测和近似,该函数不会
function ret = num2vpa(x, digi)
[N, Npow] = num2bin2(x);
if nargin == 1
    ret = vpa(N)*2^vpa(Npow);
elseif nargin == 2
    ret = vpa(N, digi)*2^vpa(Npow, digi);
end
end
\end{lstlisting}

该函数参考 “计算机算数\upref{CmArit}”
\begin{lstlisting}[language=matlab, caption=double2bin.m]
% 把一个双精度数转换为 '0' '1' 字符串
% convert a double to bit string
function str = double2bin(x)
if abs(x) < 2^-1022 || isnan(x) || isinf(x)
    error('not implemented!');
end
% 1 bit sign
if x > 0
    str = '0';
else
    str = '1';
end
x = abs(x);
% 11 bit exponent
[~, e] = num2bin(x);
str = [str dec2bin(e + 1023, 11)];
% 52 bit mantessa
i = int64(num2bin2(x));
mstr = dec2bin(i);
str = [str mstr(2:end)];
if numel(str) < 64
    str = [str repmat('0', [1,64-numel(str)])];
elseif numel(str) > 64
    error('something wrong!');
end
end
\end{lstlisting}
