% 碱金属原子(量子力学)
% keys 碱金属|量子力学
% license Xiao
% type Tutor

\pentry{薛定谔方程(单粒子多维)\nref{nod_QMndim},球谐函数 \nref{nod_SphHar},类氢原子的束缚态\nref{nod_HWF},球坐标系中的定态薛定谔方程\nref{nod_RadSE}}{nod_060d}

碱金属原子本质是由价电子(最外层电子)来决定性质的。价电子所受到原子实(原子核、内层电子)作用的势函数描述为
\begin{equation}
V(r) = -e^2/r - \lambda a e^2/r^2, \ (0 < \lambda \le 1/8) ~.
\end{equation}
是一个中心势场。则薛定谔方程是可分离变量的,使得波函数可以表示为球谐函数与径向解的乘积的形式——$\psi = Y_{l, m}(\theta, \phi) R(r)$。考虑求解径向波函数,应满足微分方程
\begin{equation}\label{eq_AlkmQM_1}
\frac{1}{R} \left(\frac{1}{r^2} \dv{R}{r}\right) + \frac{2mr^2}{\hbar^2} \left[E - V(r)\right] = l(l+1), \ (l = 0, 1, 2, \cdots)~.
\end{equation}

观察到第一项,不难想到常规技巧,设 $R(r) = u(r) /r$,从而可化为
\begin{equation}
\dv{^2 u}{r^2} + \left[\frac{2m}{\hbar^2}\left(E + e^2/r\right) + \left(\frac{2 \lambda}{r^2} - \frac{l(l+1)}{r^2}\right)\right]u = 0 ~.
\end{equation}
若取 $l'(l'+1) = l(l+1) - 2\lambda$,即
$$l' = -\frac12 + \left(l + \frac12\right) \sqrt{1 - \frac{8\lambda}{(2l+1)^2}} ~,$$
就有
\begin{equation}
\dv{^2 u}{r^2} + \left[\frac{2m}{\hbar^2}\left(E + e^2/r\right) - \frac{l'(l'+1)}{r^2}\right]u = 0 ~.
\end{equation}
这类似于(类)氢原子的情况\autoref{eq_HWF_11}~\upref{HWF}。

类比于类氢原子,对于碱金属原子有价电子
$$n' = n_r + l' + 1, \ (n_R = 0, 1, 2, \cdots) ~.$$
其中 $n' \ge 1/2$ 一般不是整数。

从而类比(类)氢原子有径向波函数的解
\begin{equation}
R_{n', l'}(r) = \frac{2}{a^{3/2} n'^2 \Gamma(2l'+2)}\sqrt{\frac{\Gamma(n'+l'+1)}{(n'-l'-1)!}} \left(\frac{2r}{an'}\right)^{l'} \exp(-r/(an')) {_1F_1}(-n'+l'+1, 2l'+2; 2r/(an'))~.
\end{equation}

价电子能级
\begin{equation}
E_{n'} = -\frac{me^4}{2\hbar^2} \left(\frac{1}{n'}\right)^2 ~.
\end{equation}

\subsection{基态性质}
价电子基态应有 $n=1, l=0$,从而 $n' = 1/2, l'=-1/2$。此时能量
$$E_{1/2} = -\frac{2me^4}{\hbar^2}~.$$
对于基态 $n'=1/2, l'=-1/2$ 而言,有 $_1F_1(0, c; \rho) = 1$。从而有完整的解
\begin{equation}
R_{1/2,-1/2}(r) Y_{0,0}(\theta, \phi) = \frac{1}{\sqrt{4\pi}} \frac{4}{a^{3/2}} \left(\frac{r}{a}\right)^{-1/2} \exp\left(\frac{-2r}{a}\right) ~.
\end{equation}
概率密度是模的平方为
\begin{equation}
|R_{1/2, -1/2}(r) Y_{0, 0}(\theta, \phi)|^2 = \frac{4}{\pi a^3}\frac{a}{r}\exp\left(\frac{-4r}{a}\right)~.
\end{equation}
则在 $r$ 出薄球壳出现的概率是
\begin{equation}
P(\rho) = P(r/a) = (|R_{1/2, -1/2}(r) Y_{0, 0}(\theta, \phi)|^2 \cdot 4\pi r^2 \dd r)/(\dd r) = 16 \rho \exp(-4\rho)~.
\end{equation}
最可几时 $P(\rho)$ 取极值,即 $\rho=1/4$ 处。