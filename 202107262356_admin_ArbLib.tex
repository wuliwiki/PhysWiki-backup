% Arb 任意精度计算库

\pentry{C++ 基础\upref{Cpp0}}

Arb 是一款支持任意精度计算的 C/C++ 程序, 支持对计算进行严谨的误差估计, 即每个任意精度浮点数 $z$ 都会伴随一个误差半径 $r$, 使得精确结果必定落在复平面上以 $z$ 为圆心半径为 $r$ 的圆盘中. 通过增加浮点数的精度, 就可以用数值方法无限逼近理论值. Arb 还提供了许多特殊函数的计算, 例如 $\Gamma$ 函数\upref{Gamma}, 超几何函数\upref{HypGeo} 等, 以及常用的线性代数功能和离散傅里叶变换\upref{DFT}等. 官方主页 \href{https://arblib.org/}{arblib.org} 包含详细的文档.

\subsection{安装}
以 Ubuntu 为例, 最方便的安装方式就是使用 \verb|apt| 安装. 首先安装 dependency
\begin{equation}
sudo apt install libmpfr-dev libflint-dev
\end{equation}
其中 \verb|MPFR| 和 \verb|flint| 两个包分别用于任意精度浮点数以及数论. 然后安装 Arb
\begin{lstlisting}[language=bash]
sudo apt install libflint-arb-dev libflint-arb2
\end{lstlisting}

但目前这并不是 Arb 的最新版本(例如没有实现库仑函数\upref{CulmF}). 要获得最新版本, 可以直接从 \href{https://github.com/fredrik-johansson/arb/}{GitHub} 下载源码编译即可(默认使用 gcc 编译器). 目前笔者使用的版本是 release 2.19.0.

可以用 \verb|./configure --help| 查看编译选项, 若所有的包都安装在默认目录则不需要编译选项.
\begin{lstlisting}[language=bash]
./configure [编译选项];
make -j4;
sudo make install;
\end{lstlisting}
其中 \verb|-j4| 是使用 4 线程进行编译, 也可以改成其他数字.

\subsection{编译}
在 Ubuntu 中如果你使用 \verb|apt| 安装, 在编译程序是需要加 \verb|-lflint-arb| 选项. 如果你直接从源码编译, 则需要加 \verb|-larb| 选项.

一个 C++ 例程:
\begin{lstlisting}[language=cpp, caption=test.cpp]
#include "arb_hypgeom.h"
#include "acb_hypgeom.h"
#include <complex>
#include <iostream>

using namespace std;

typedef double Doub, Doub_O;
typedef const double Doub_I;
typedef complex<double> Comp, Comp_O;
typedef complex<double> Comp_I;
typedef int Int;
typedef const int Int_I;
typedef int & Int_O, Int_IO;

// 1F1 hypdergeometric function with real arguments
Doub hypergeom1F1(Doub_I a, Doub_I b, Doub_I z)
{
	slong prec = 64; // set precision bit (log10/log2 = 3.322)
	Doub res;
	arb_t a1, b1, z1, res1;
	arb_init(a1); arb_init(b1); arb_init(z1); arb_init(res1);
	arb_set_d(a1, a); arb_set_d(b1, b);	arb_set_d(z1, z);
	arb_hypgeom_1f1(res1, a1, b1, z1, 0, prec);
	// arb_printn(res1, 100, 0); printf("\n");  // print result
	int digits = arb_rel_accuracy_bits(res1)/3.321928;
	if (digits < 15) {
		cout << "warning: hypergeom1F1 error too large" << endl;
	}
	res = arf_get_d(arb_midref(res1), ARF_RND_NEAR);
	arb_clear(a1); arb_clear(b1); arb_clear(z1); arb_clear(res1);
	return res;
}

// 1F1 hypdergeometric function with complex arguments
Comp hypergeom1F1(Comp_I a, Comp_I b, Comp_I z)
{
	slong prec = 64; // set precision bit (log10/log2 = 3.322)
	Comp res;
	arb_t temp1;
	arb_init(temp1);
	acb_t a1, b1, z1, res1;
	acb_init(a1); acb_init(b1); acb_init(z1); acb_init(res1);
	acb_set_d_d(a1, real(a), imag(a));
	acb_set_d_d(b1, real(b), imag(b));
	acb_set_d_d(z1, real(z), imag(z));
	acb_hypgeom_1f1(res1, a1, b1, z1, 0, prec);
	// acb_printn(res1, 100, 0); printf("\n"); // print result
	int digits = acb_rel_accuracy_bits(res1)/3.321928;
	if (digits < 15) {
		cout << "warning: hypergeom1F1 error too large" << endl;
	}
	acb_get_real(temp1, res1);
	res.real(arf_get_d(arb_midref(temp1), ARF_RND_NEAR));
	acb_get_imag(temp1, res1);
	res.imag(arf_get_d(arb_midref(temp1), ARF_RND_NEAR));
	acb_clear(a1); acb_clear(b1); acb_clear(z1);
    acb_clear(res1); arb_clear(temp1);
	return res;
}

// Coulomb function F
Doub coulombF(Doub_I l, Doub_I eta, Doub_I x)
{
	slong prec = 64; // set precision bit (log10/log2 = 3.322)
	Doub F;
	arb_t l1, eta1, x1, F1;
	arb_init(l1); arb_init(eta1); arb_init(x1); arb_init(F1);
	arb_set_d(l1, l); arb_set_d(eta1, eta);
	arb_set_d(x1, x); arb_set_d(F1, F);

	arb_hypgeom_coulomb(F1, NULL, l1, eta1, x1, prec);
	
	Int digits = arb_rel_accuracy_bits(F1)/3.321928;
	if (digits < 15) {
		cout << "warning: hypergeom1F1 error too large" << endl;
	}
	F = arf_get_d(arb_midref(F1), ARF_RND_NEAR);
	arb_clear(l1); arb_clear(eta1); arb_clear(x1); arb_clear(F1);
	return F;
}

int main()
{
	cout.precision(16);
    cout << "hypergeom1F1(1.23,1.23,1.23) = " << endl;
	cout << hypergeom1F1(1.23,1.23,1.23) << endl;
	printf("Mathematica: 3.421229536289674\n");

	cout << "hypergeom1F1(1.23+1.23I, 1.23+1.23I, 1.23+1.23I) = " << endl;
	cout << hypergeom1F1(Comp(1.23,1.23),Comp(1.23,1.23),Comp(1.23,1.23))
    cout << endl;
	printf("Mathematica: 1.143503984180676 + 3.224470526790991i\n");

	cout << "coulombF(203, -0.1, 6) = " << endl;
	cout << coulombF(203, -0.1, 6) << endl;
	printf("Mathematica: 4.5882301380287637784e-286\n");
}
\end{lstlisting}
编译:
\begin{lstlisting}[language=makefile]
g++ -o test.x test.cpp -larb -lflint
\end{lstlisting}
