% 四川大学 2009 年研究生入学物理考试试题
% keys 四川大学|2009年|考研|物理
% license Copy
% type Tutor

\textbf{声明}:“该内容来源于网络公开资料,不保证真实性,如有侵权请联系管理员”


科目代码:947


适用专业:光学、无线电物理、物理电子学


可能用到的物理常数:真空中的介电常数:$8.85*10^{-12}F/m$,真空中的导磁率:$4\pi*10^{-7}H/m$,一个电子的电量:$1.62*10^{-19}C$
\begin{enumerate}
\item  处于静电平衡的理想导体,导体内部电场强度为 $(\qquad)$ 随曲率半径增大,导体外表面的电场强度$(\qquad)$ 
\item 利用万用表测量市电的交流电压,读数为220V,是指交流电压的$(\qquad)$ 值,对应的峰值电压为$(\qquad)$ V。
\item 相隔距离为 d 的等量同号点电荷+q 和+q,二者中点处的电势为$(\qquad)$ v,电场强度的大小为$(\qquad)$ V/m。
\item 两根无限长的均匀带电直线相互平行,相距为2a,线电荷密度分别为+$\rho$和-$\rho$则每单位长度上的带电直线受的库仑力为$(\qquad)$N,两根直线相
互$(\qquad)$。
\item 两块平行金属板间充满电容率为$\varepsilon_1=2\varepsilon_0$的均匀介质,当维持两块金属板上电压V不变,每块平行金属板的电荷为Q。1)如果将介质换为$\varepsilon_2=2\varepsilon_1$的介质则每块平行金属板的电荷为$(\qquad)$
。2)如果将介质$\varepsilon_1$ 移去,则每块平行金属板的电荷为$(\qquad)$。
\item 一波长为 550nm的黄绿光入射到间距为0.2mm的双缝上,则离缝2m远处观察屏上干涉条纹的间距是$(\qquad)$mm;若缝间距增大为2mm,则
\end{enumerate}