% 集合(综述)
% license CCBYSA3
% type Wiki

本文根据 CC-BY-SA 协议转载翻译自维基百科\href{https://en.wikipedia.org/wiki/Set_(mathematics)}{相关文章}。

\begin{figure}[ht]
\centering
\includegraphics[width=6cm]{./figures/c3177a16637a20f4.png}
\caption{欧拉图中的一组多边形} \label{fig_JHSX_1}
\end{figure}
在数学中,集合是不同事物的集合;这些事物被称为集合的元素或成员,通常是任何类型的数学对象:数字、符号、空间中的点、线条、其他几何形状、变量,甚至是其他集合。集合可以是有限的或无限的,具体取决于其元素的数量是否有限。存在一个没有元素的唯一集合,称为空集;只有一个元素的集合称为单集合。

集合在现代数学中无处不在。实际上,集合理论,特别是泽尔梅洛-弗兰克尔集合理论,自20世纪上半叶以来,已经成为为所有数学分支提供严谨基础的标准方法。
\begin{figure}[ht]
\centering
\includegraphics[width=6cm]{./figures/7cd2858c49f58e9d.png}
\caption{这个集合等于上面所示的集合,因为它们具有完全相同的元素。} \label{fig_JHSX_2}
\end{figure}
\subsection{背景}
在19世纪末之前,集合并没有被专门研究,也没有与数列明确区分开来。大多数数学家认为无穷大是潜在的——意味着它是一个无尽过程的结果——因此他们不愿意考虑无限集合,即那些成员数量不是自然数的集合。具体来说,一条线并没有被看作是其点的集合,而是看作一个点可能位于其中的轨迹。

无限集合的数学研究始于乔治·康托尔(Georg Cantor,1845-1918)。这带来了一些违反直觉的事实和悖论。例如,数轴上有一个无限多个元素,其数量严格大于自然数的无限集合,而任何线段的元素数量与整个空间相同。此外,拉塞尔悖论意味着“所有集合的集合”这一短语是自我矛盾的。

这些违反直觉的结果,加上其他的悖论,导致了数学的基础危机,最终通过广泛采纳泽尔梅洛-弗兰克尔集合理论作为集合论和所有数学的坚实基础得以解决。

与此同时,集合开始在所有数学领域广泛应用。特别是,代数结构和数学空间通常是通过集合来定义的。此外,许多较早的数学成果也以集合的形式重新表述。例如,欧几里得的定理常常被表述为“素数集合是无限的”。大卫·希尔伯特曾预言,集合在数学中的广泛使用:“没有人会把我们从康托尔为我们创造的天堂中赶出去。”

通常,数学中对集合的常见使用并不需要泽尔梅洛-弗兰克尔集合理论的完整能力。在数学实践中,集合可以独立于该理论的逻辑框架进行操作。

本文的目的是总结在数学中常用的集合操作规则和属性,而不涉及任何逻辑框架。对于研究集合的数学分支,请参见集合论;对于对应逻辑框架的非正式介绍,请参见朴素集合论;对于更正式的介绍,请参见公理化集合论和泽尔梅洛-弗兰克尔集合理论。
\subsection{基本概念}
在数学中,集合是不同事物的集合。这些事物被称为集合的元素或成员,通常是任何类型的数学对象,如数字、符号、空间中的点、线条、其他几何形状、变量、函数,甚至是其他集合。集合有时也被称为集合或族,特别是当其元素本身是集合时;这样可以避免集合与其成员之间的混淆,并使阅读更加清晰。集合可以通过列出其元素或通过一个描述其元素的性质来指定,例如素数集合或某一班级所有学生的集合。

如果\( x \)是集合\( S \)的元素,则称\( x \)属于 \( S \)或\( x \)在\( S \)中,表示为\( x \in S \)。声明 " \( y \)不在\( S \)中" 表示为\( y \notin S \),也可以读作 " \( y \)不在\( S \) 中"。例如,如果 \( \mathbb{Z} \)是整数的集合,那么\( -3 \in \mathbb{Z} \)和\( 1.5 \notin \mathbb{Z} \)。

每个集合通过其元素唯一地定义。特别地,两个集合如果具有完全相同的元素,它们是相等的(它们是相同的集合)。这个性质叫做外延性,可以用公式写为:
\[
A = B \iff \forall x\; (x \in A \iff x \in B)~
\]
这意味着只有一个没有元素的集合,称为空集(或零集),表示为\( \varnothing \)或\( \emptyset \),或者\( \{\} \)。单集合是一个只有一个元素的集合。如果\( x \)是这个元素,则单集合表示为\( \{x\} \)。如果\( x \)本身是一个集合,它不能与 \( \{x\} \)混淆。例如,\( \emptyset \)是一个没有元素的集合,而 \( \{\emptyset\} \)是一个以\( \emptyset \)为唯一元素的单集合。
\begin{figure}[ht]
\centering
\includegraphics[width=6cm]{./figures/73964688f127400d.png}
\caption{所有标准数系都是无限集合。} \label{fig_JHSX_3}
\end{figure}
如果存在一个自然数\( n \),使得前\( n \)个自然数可以与集合的元素一一对应,则该集合是有限的。在这种情况下,称\( n \) 为集合的元素个数。如果不存在这样的\( n \),则集合是无限的。空集是一个有限集合,具有\( 0 \)个元素。

自然数构成一个无限集合,通常表示为 \( \mathbb{N} \)。其他无限集合的例子包括包含自然数的数集、实数向量空间、曲线和大多数类型的空间。
\subsection{指定一个集合}
外延性意味着,在指定一个集合时,必须么列出其元素,要么提供一个唯一描述集合元素的性质。
\subsubsection{列举法符号}
列举法符号是由恩斯特·泽尔梅洛于1908年引入的符号,通过在花括号内列出集合的元素并用逗号分隔来指定一个集合。例如,集合\( \{4, 2, 1, 3\} \)和\( \{\text{blue, white, red}\} \)表示的是集合而不是元组,因为它们被花括号包围。

上面的符号\( \{\} \)和\( \{x\} \)分别表示空集和单集合,这些都是列举法符号的例子。

在指定集合时,唯一重要的是每个不同的元素是否属于该集合;这意味着如果元素重复或顺序改变,集合本身不会发生变化。例如:
\[
\{1, 2, 3, 4\} = \{4, 2, 1, 3\} = \{4, 2, 4, 3, 1, 3\}~
\]
当存在明确的生成所有集合元素的模式时,可以使用省略号来简化符号,如表示不大于1000的正整数集合:
\[
\{1, 2, 3, \ldots, 1000\}~
\]
省略号也可以用来扩展列举法符号,表示一些无限集合。例如,所有整数的集合可以表示为:
\[
\{\ldots, -3, -2, -1, 0, 1, 2, 3, \ldots\}~
\]
或者:
\[
\{0, 1, -1, 2, -2, 3, -3, \ldots\}~
\]
\subsubsection{集合描述符号}
集合描述符号通过指定满足某些逻辑公式的所有元素来定义一个集合。更精确地说,如果\( P(x) \)是一个依赖于变量\( x \)的逻辑公式,该公式根据\( x \)的值评估为真或假,那么:
\[
\{x \mid P(x)\} \quad \text{或} \quad \{x : P(x)\}~
\]
表示所有使得\( P(x) \)为真的\( x \)的集合。例如,可以如下指定集合\( F \):
\[
F = \{n \mid n \text{ 是整数,且 } 0 \leq n \leq 19\}~
\]
在这种符号中,竖线符号 "|" 读作 "使得",整个公式可以读作 "\(F\)是所有\( n \)的集合,使得\( n \)是一个在 0 到 19 范围内的整数"。

某些逻辑公式,如\( S \)是一个集合 或\( S \)是一个集合且\( S \notin S \),不能用于集合描述符号,因为没有一个集合的元素能够由该公式来描述。为了解决这个问题,可以采取几种方法。可以证明公式定义了一个集合;这通常是显而易见的,但也可能非常困难。

还可以引入一个更大的集合\( U \),它必须包含指定集合的所有元素,并将符号写作:
\[
\{x \mid x \in U \text{ 且 ...}\} \quad \text{或} \quad \{x \in U \mid \text{ ...}\}~
\]
也可以事先定义\( U \),并约定每个出现在符号竖线左边的变量代表\( U \)的元素。这意味着在集合描述符号中\( x \in U \)是隐含的。在这种情况下,\( U \)通常被称为讨论域或宇宙。

例如,假设小写拉丁字母表示一个实数且仅此而已,则表达式
\[
\{x \mid x \notin \mathbb{Q}\}~
\]
是
\[
\{x \in \mathbb{R} \mid x \notin \mathbb{Q}\}~
\]
的简写,定义了无理数集合。
\subsection{子集}
集合\( B \)的子集是一个集合 \( A \),使得\( A \)的每个元素也是\( B \)的元素。如果\( A \)是\( B \)的子集,通常说\( A \)包含于\( B \),\( B \) 包含\( A \),或\( B \)是\( A \) 的超集。这表示为\( A \subseteq B \) 和 \( B \supseteq A \)。然而,许多作者使用\( A \subset B \)和\( B \supset A \) 来表示。子集的定义可以用符号表示为:
\[
A \subseteq B \quad \text{当且仅当} \quad \forall x \; (x \in A \implies x \in B)~
\]
如果集合\( A \)是集合\( B \)的子集并且\( A \neq B \),则称\( A \)是\( B \)的真子集。这表示为\( A \subset B \)和\( B \supset A \)。当使用\( A \subset B \)表示子集关系时,或者在可能产生歧义的情况下,通常使用\( A \subsetneq B \)和\( B \supsetneq A \)。

通过符号\( \subseteq \)建立的集合之间的关系称为包含关系或包含性。集合之间的相等可以通过子集来表达。两个集合当且仅当它们彼此包含时才相等:即\( A \subseteq B \)和\( B \subseteq A \)等价于\( A = B \)。空集是每个集合的子集:\( \emptyset \subseteq A \)。

示例:
\begin{itemize}
\item 所有人的集合是所有哺乳动物集合的真子集。
\item \( \{1, 3\} \subset \{1, 2, 3, 4\} \)
\item \( \{1, 2, 3, 4\} \subseteq \{1, 2, 3, 4\} \)
\end{itemize}
\subsection{基本操作}
有几个标准操作可以从给定的集合中生成新的集合,类似于加法和乘法可以从给定的数字中生成新的数字。本节讨论的操作是那些生成的集合中的所有元素都属于先前定义的集合的操作。这些操作通常通过欧拉图和维恩图来说明。

集合的主要基本操作如下。
\subsubsection{交集}
\begin{figure}[ht]
\centering
\includegraphics[width=6cm]{./figures/149744294b6fff3d.png}
\caption{} \label{fig_JHSX_4}
\end{figure}
两个集合\( A \) 和 \( B \)的交集是一个集合,表示为\( A \cap B \),其元素是既属于\( A \) 又属于\( B \)的元素。即:
\[
A \cap B = \{x \mid x \in A \land x \in B\}~
\]
其中\( \land \)表示逻辑“与”运算。

交集是结合律和交换律的;这意味着在进行一系列交集运算时,可以任意顺序进行,无需括号来指定操作顺序。交集没有通用的单位元素。然而,如果将交集限制为给定集合\( U \)的子集,那么交集的单位元素是\( U \)。

如果\( \mathcal{S} \)是一个非空的集合族,则其交集,表示为\( \bigcap_{A \in \mathcal{S}} A \),是一个集合,其元素是属于\( \mathcal{S} \)中所有集合的元素。即:
\[
\bigcap_{A \in \mathcal{S}} A = \{x \mid (\forall A \in \mathcal{S})\; x \in A\}~
\]
当\( \mathcal{S} \)只有两个元素时,这两个交集的定义是等价的。
\subsubsection{并集}
\begin{figure}[ht]
\centering
\includegraphics[width=6cm]{./figures/5f0b23253648df40.png}
\caption{} \label{fig_JHSX_5}
\end{figure}
两个集合\( A \)和 \( B \)的并集是一个集合,表示为\( A \cup B \),其元素是属于\( A \)、\( B \)或者两者的元素。即:
\[
A \cup B = \{x \mid x \in A \lor x \in B\}~
\]
其中\( \lor \)表示逻辑“或”运算。

并集是结合律和交换律的;这意味着在进行一系列并集运算时,可以任意顺序进行,无需括号来指定操作顺序。空集是并集运算的单位元素。

如果\( \mathcal{S} \)是一个集合族,则其并集,表示为\( \bigcup_{A \in \mathcal{S}} A \),是一个集合,其元素是属于\( \mathcal{S} \)中至少一个集合的元素。即:
\[
\bigcup_{A \in \mathcal{S}} A = \{x \mid (\exists A \in \mathcal{S})\; x \in A\}~
\]
当\( \mathcal{S} \)只有两个元素时,这两个并集的定义是等价的。
\subsubsection{集合差}
\begin{figure}[ht]
\centering
\includegraphics[width=6cm]{./figures/3e7235ce75db5c90.png}
\caption{} \label{fig_JHSX_6}
\end{figure}
两个集合\( A \)和\( B \)的集合差是一个集合,表示为\( A \setminus B \)或\( A - B \),其元素是属于\( A \)但不属于\( B \)的元素。即:
\[
A \setminus B = \{x \mid x \in A \land x \notin B\}~
\]
其中 \( \land \) 表示逻辑“与”运算。
\begin{figure}[ht]
\centering
\includegraphics[width=6cm]{./figures/480817247a571f2b.png}
\caption{A 在 U 中的补集} \label{fig_JHSX_7}
\end{figure}
当\( B \subseteq A \) 时,差集\( A \setminus B \)也称为\( B \)在\( A \) 中的补集。当所有考虑的集合都是固定的全集\( U \)的子集时,补集\( U \setminus A \)通常称为\( A \)的绝对补集。
\begin{figure}[ht]
\centering
\includegraphics[width=6cm]{./figures/a6ff1c72275eb57a.png}
\caption{A 和 B 的对称差} \label{fig_JHSX_8}
\end{figure}
两个集合\( A \)和\( B \)的对称差,表示为\( A \, \Delta \, B \),是一个集合,其中的元素属于\( A \)或\( B \)但不属于两者。即:
\[
A \, \Delta \, B = (A \setminus B) \cup (B \setminus A)~
\]
\subsubsection{子集代数}
集合\( U \)的所有子集构成的集合称为\( U \)的幂集,通常表示为\( \mathcal{P}(U) \)。幂集是一个代数结构,其主要运算包括并集、交集、集合差、对称差和绝对补集(即在 \( U \) 中的补集)。

幂集是一个布尔环,其中对称差作为加法,交集作为乘法,空集作为加法单位,\( U \) 作为乘法单位,补集作为加法逆元。

幂集也是一个布尔代数,其中联合(\( \lor \))是并集(\( \cup \)),交(\( \land \))是交集(\( \cap \)),而否定是集合补集。

像所有布尔代数一样,幂集也是一个部分有序集合,按照集合包含关系排序。它也是一个完全格。

这些结构的公理引出了许多与子集相关的恒等式,详细内容请参见相关链接文章。