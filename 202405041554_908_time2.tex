% 时间的计量 2
% keys 世界时|ICRF|儒略日|视太阳时|平太阳时|均时差
% license Xiao
% type Tutor

\pentry{时间的计量\nref{nod_TimeCa}}{nod_f0dd}

\subsection{世界时}
通过遥远天体的观测等方法决定一个参考系, 即 \textbf{ICRF(International Celestial Reference Frame)}, 决定地球在 ICRF 中的转角\textbf{地球旋转角(Earth Rotation Angle, ERA)} $\theta$, 这和 \textbf{UT1 儒略日} 的关系为
\begin{equation}\label{eq_time2_1}
\theta = 2\pi(0.7790572732640 + 1.00273781191135448 T_u)~,
\end{equation}
其中 $T_u = \text{UT1 儒略日} - 2451545.0$, 且 $2.7378\e{-3}$ 约等于 $1/365.25$ 即一回归年的太阳日个数。
\addTODO{儒略日期是通过 \enref{SI 的秒来定义的\upref{TimeCa}, 所以 UT1 儒略日和儒略日到底是什么关系呢?}

把 UT1 儒略日的小数部分除以 86400 ($24\times 60\times 60$), 就可以定义 UT1 时间, 即上一个 UT1 正午经过的 UT1 秒数(注意儒略日是从中午开始计算的)。 注意这里的 UT1 秒是由\autoref{eq_time2_1} 定义的而不符合国际单位。 由于地球自转速度会在潮汐力的作用下发生改变, 所以 UT1 中一秒的长度也会在几天到几十天的周期内变化。 UT2 就是在 UT1 的基础上把周期性潮汐力对地球自转的影响过滤掉了。 测量得出, 现在地球自转一圈的平均用时(UT2)约比 100 年前的 1 天慢 1.7 毫秒。

\subsection{两种太阳时}
\textbf{地方视太阳时(local apparent solar time)}可以定义为一个理想日晷(把太阳看作一个点光源)显示的时间。 视太阳时的问题在于它是不均匀的而是随季节周期性变化。 太阳相对于地球的行经角速度由两部分组成, 一个是上文的 ERA 角速度, 另一个是在 ICRF 参考系中太阳直射点相对地轴的角速度。 前者在变化非常缓慢几乎可以忽略不计, 而后者会发生显著的季节性变化。 改变化由两个因素决定: 一是地球的公转旋转轨道是椭圆的, 当地球远离太阳时公转角速度变慢, 靠近太阳时角速度变快; 二是由于赤道平面相对于公转平面(黄道平面)是倾斜的, 当太阳直射点纬度大时行经角速度快(行纬角速度慢), 接近赤道时行经角速度慢(行纬角速度快)。

如果把一年中的视太阳时做一个线性拟合, 就得到了\textbf{平太阳时(mean solar time)}。 平太阳时和 UT1 几乎相等(除了每百年增加 1.7ms 误差)。 当前的平太阳日约为国际单位的 $86400.002\mathrm{s}$。 视太阳时和平太阳时之间的差距叫做\textbf{均时差(equation of time)}, 由\autoref{fig_time2_1} 可得在一月中旬慢 15 分钟左右, 而在 11 月初则快 17 分钟左右。

\begin{figure}[ht]
\centering
\includegraphics[width=10cm]{./figures/79fd0b8eacbe2adc.pdf}
\caption{均时差, 横轴表示日期, 纵轴为分钟(来自 Wikipedia)} \label{fig_time2_1}
\end{figure}
