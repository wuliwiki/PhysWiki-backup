% 栈(综述)
% license CCBYSA3
% type Wiki

(本文根据 CC-BY-SA 协议转载自原搜狗科学百科对英文维基百科的翻译)

在计算机科学中,堆栈作为元素的集合是一种抽象数据类型,有两个主要操作:
\begin{itemize}
\item 推送,它将元素添加到集合中。
\item pop,移除最近添加的尚未移除的元素。
\end{itemize}
元素从堆栈中取出的顺序产生了它的另一个名字,叫做后进先出。此外,查看操作可以在不修改堆栈的情况下访问顶部。[1] 这种结构的名称“堆栈”来源于一组堆叠在一起的物理项目的类比,这使得从堆栈顶部取出一个项目变得容易,而到达堆栈中更深处的项目可能需要先取出多个上层的其他项目。[2]

推送和弹出操作被视为线性数据结构,或者更抽象地说是顺序集合,只发生在结构的一端,即堆栈的顶部。这使得将堆栈实现为单个链表和指向顶部元素的指针成为可能。堆栈可以被实现为具有有限容量的空间。如果堆栈已满,并且没有足够的空间接受要推送的实体,则堆栈被认为处于溢出状态。弹出操作从堆栈顶部移除一个项目。

\subsection{历史}
斯塔克斯(Stacks)于1946年进入计算机科学文献,当时艾伦·图灵(Alan M. Turing)使用“埋葬”和“不埋葬”("bury" and "unbury")这两个术语作为调用子程序和从子程序返回的手段。[3] 子程序已经于1945年在康拉德·楚泽的Z4( Konrad Zuse's Z4 )中实现。

慕尼黑理工大学(Technical University Munich)的克劳斯·萨姆森(Klaus Samelson)和弗里德里希·鲍尔(Friedrich L. Bauer)于1955年提出了这个想法,并于1957年申请了专利,[4] 鲍尔(Bauer)于1988年3月因堆栈原理的发明获得了计算机先锋奖。[5] 澳大利亚人查尔斯·伦纳德·汉布林(Charles Leonard Hamblin)在1954年上半年独立发展了同样的概念。[6]

堆栈通常被比喻成自助餐厅中弹簧加载的一堆盘子。[7][2][8] 干净的盘子放在堆叠的顶部,将任何已经在那里的盘子向下推。当一个盘子从堆叠中取出时,它下面的那个就会弹出来成为新的顶部。

\subsection{ 非必要操作}
在许多实现中,堆栈比“推送”和“弹出”有更多的操作。一个例子是“栈顶”,即“读取数据”,它观察最顶层的元素而不将其从栈中移除。[9] 由于这可以通过使用相同数据的“弹出”和“推送”来完成,所以这不是必需的。如果堆栈为空,“堆栈顶部”操作可能会出现下溢情况,与“弹出”操作相同。此外,实现通常有一个函数用于返回堆栈是否为空。

\subsection{软件栈}
\subsubsection{3.1 实现}
堆栈可以通过数组或链表轻松实现。在这两种情况下,将数据结构标识为堆栈的不是实现,而是接口:只允许用户将项目弹出或推送到数组或链表上,几乎没有其他辅助操作。下面将使用伪代码演示这两种实现。