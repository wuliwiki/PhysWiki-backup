% 柯西-施瓦茨不等式(综述)
% license CCBYNCSA3
% type Wiki

本文根据 CC-BY-SA 协议转载翻译自维基百科 \href{https://en.wikipedia.org/wiki/Cauchy\%E2\%80\%93Schwarz_inequality}{相关文章}。

柯西–施瓦茨不等式(也称为柯西–布尼亚科夫斯基–施瓦茨不等式)是对内积空间中两个向量的内积绝对值的一个上界,其上界由这两个向量范数的乘积给出。它被认为是数学中最重要且应用最广泛的不等式之一。

向量的内积可以用于描述有限和(通过有限维向量空间)、无穷级数(通过序列空间中的向量)以及积分(通过希尔伯特空间中的向量)。柯西于1821年首次发表了关于求和形式的不等式。相应的积分形式的不等式由布尼亚科夫斯基于1859年发表,赫尔曼·施瓦茨于1888年发表了积分形式的现代证明。
\subsection{不等式的表述}
柯西–施瓦茨不等式指出,对于内积空间中的任意两个向量$\mathbf{u}$和$\mathbf{v}$,都有:
$$
|\langle \mathbf{u}, \mathbf{v} \rangle|^2 \leq \langle \mathbf{u}, \mathbf{u} \rangle \cdot \langle \mathbf{v}, \mathbf{v} \rangle~
$$
其中,$\langle \cdot , \cdot \rangle$ 表示内积运算。例如,实数或复数的点积就是常见的内积形式。每一个内积都对应一个欧几里得 $\ell_2$ 范数,也叫做“标准范数”或“诱导范数”,记作 $|\mathbf{u}|$,其定义为:
$$
|\mathbf{u}\| := \sqrt{\langle \mathbf{u}, \mathbf{u} \rangle},~
$$
其中 $\langle \mathbf{u}, \mathbf{u} \rangle$ 总是一个非负实数(即使内积是复值的)。对上述不等式两边取平方根,就可以得到柯西–施瓦茨不等式更常见的形式,用范数表示为:
$$
|\langle \mathbf{u}, \mathbf{v} \rangle| \leq |\mathbf{u}| \cdot |\mathbf{v}|~
$$
当且仅当 $\mathbf{u}$ 和 $\mathbf{v}$ 线性相关时,上述不等式取等号。\(^\text{[8][9][10]}\)
\subsection{特殊情形}
\subsubsection{Sedrakyan 引理 —— 正实数情形}
Sedrakyan 不等式,又称为 Bergström 不等式、Engel 形式、Titu 引理(或 T2 引理),表述如下:

对于实数 $u_1, u_2, \dots, u_n$ 和正实数 $v_1, v_2, \dots, v_n$,有:

$$
\frac{(u_1 + u_2 + \cdots + u_n)^2}{v_1 + v_2 + \cdots + v_n} \leq \frac{u_1^2}{v_1} + \frac{u_2^2}{v_2} + \cdots + \frac{u_n^2}{v_n},~
$$

或者用求和符号表示为:

$$
\left( \sum_{i=1}^{n} u_i \right)^2 \bigg/ \sum_{i=1}^{n} v_i \leq \sum_{i=1}^{n} \frac{u_i^2}{v_i}.~
$$

这个不等式是柯西–施瓦茨不等式的直接推论,具体地,可以将其看作是在欧几里得空间 $\mathbb{R}^n$ 中对向量点积应用柯西–施瓦茨不等式得到的。

方法是令:

$$
u_i' = \frac{u_i}{\sqrt{v_i}}, \quad v_i' = \sqrt{v_i},~
$$

将其代入向量内积后即可得出上述不等式。这种形式在处理分式型不等式(尤其是分子为完全平方形式)时尤其有用。
