% 仿射子空间
% keys 仿射子空间|平面|超平面|方向子空间|平行

\pentry{仿射空间\upref{AfSp}}
本节将引入仿射子空间的概念,仿射子空间也称为仿射空间中的平面.0维的仿射子空间是个点,1维的是直线,$n-1$ 维的则是超平面($n$ 为仿射空间的维数).本节将证明,仿射空间中的平面本身也是一个仿射空间,且任何的平面包含通过平面上两不同点的直线.此外,平面作为仿射空间,其装备了一个方向子空间(即与其相配备的矢量空间),若两平面的方向子空间相同,就称它们平行,平行的平面必能通过相互平移得到.抛去仿射空间的内容,这些都与我们通常的几何直觉相一致.以上内容都能在本节得到.
\subsection{仿射子空间}
\begin{definition}{}
设 $(\mathbb A,V)$ 是个 $n$ 维的仿射空间,$U$ 是 $V$的 矢量子空间.在 $\mathbb A$ 中固定一点 $\dot p$,称集合
\begin{equation}
\Pi=\dot p+U=\{\dot p+u|u\in U\}
\end{equation}
是 $\mathbb A$ 的一个 $m=\dim U$ 维的\textbf{平面}(或\textbf{仿射子空间}).当 $m=0$ 是,$\Pi$ 称为\textbf{点};$m=1$ 称为\textbf{直线};$m=n-1$ 称为\textbf{超平面}.$U$ 称为 $\Pi$ 的\textbf{方向子空间}.显然,由于 $\dot p=\dot p+0\in \Pi$ \footnote{这里0是 $V$ 中的矢量},故也称 $\Pi$ 是经过点 $\dot p$ 的在方向子空间 $U$ 上的仿射子空间.
\end{definition}