% 正交函数系 2
% 正交;函数系;贝塞尔不等式

\begin{issues}
\issueDraft
\end{issues}

若函数系 
\begin{equation}\label{OFS_eq1}
\qty{\varphi_n(x)}\quad(n\in \mathbb{N})
\end{equation}
中各函数及其平方在区间 $\qty[a,b]$ 上皆可积分,且满足
\begin{equation}
\int_a^b\varphi_n(x)\varphi_m(x)\dd x=
\left\{
    \begin{aligned}
    &0\quad &m\neq n\\
   &\lambda_n>0\quad &m=n
    \end{aligned}
\right.
\end{equation}
则称函数系$\qty{\varphi_n(x)}$ 为\textbf{正交函数系}.当 $\lambda_n=1(n\in\mathbb{N})$ 时,该函数系称为\textbf{正交规范系}(或\textbf{正交标准系}).显然,任意的正交函数系都可化为正交规范系 
\begin{equation}
\qty{\frac{\varphi_n(x)}{\sqrt{\lambda_n}}}\quad(n\in \mathbb{N})
\end{equation}

设 $f(x)$ 是任一实函数,在区间 $[a,b]$ 内是连续的,则数值
\begin{equation}
c_k=\frac{1}{\lambda_k}\int_a^b f(x)\varphi_k(x)\dd x\quad (k\in \mathbb{N})
\end{equation}
称为函数 $f(x)$ 关于函数系\autoref{OFS_eq1} 的\textbf{傅里叶系数}.由 $c_k$ 的定义,我们有等式
\begin{equation}\label{OFS_eq2}
\int_a^b\qty[f(x)-\sum_{k=1}^nc_k\varphi_k(x)]^2\dd x=\int_a^b\qty[f(x)]^2\dd x-\sum_{k=1}^nc_k^2\lambda_k
\end{equation}

根据\autoref{OFS_eq2} ,可得
\begin{equation}\label{OFS_eq3}
\sum_{k=1}^nc_k^2\lambda_k\leq\int_a^b\qty[f(x)]^2\dd x
\end{equation}
对于正交规范系, $\lambda_k=1$,此时\autoref{OFS_eq3} 便是所谓的\textbf{贝塞尔(bessel)不等式}. 

若对任何在区间 $[a,b]$ 定义的连续函数 $f(x)$,\autoref{OFS_eq3} 中的等号成立,则称函数系 $\qty{\varphi_n(x)}$ 是\textbf{完整的}(或\textbf{完备的}).



