% 整数模与裴蜀定理
% keys 整数模|最大公约数|裴蜀定理|算术基本定理
% license Usr
% type Tutor

\begin{definition}{数的模}
数的\textbf{模(module)}是指,对于一个数的集合 $S$,若任意两个 $S$ 中的元素 $x, y$,他们的和与差 $(x\pm y)$ 都是 $S$ 中的元素,即 $\forall x, y \in S$,$(x \pm y) \in S$。
\end{definition}
显然单独一个数 $0$ 也构成一个模,称为\textbf{零模(null module)}。

由数的模的定义可知,若 $a \in S$,则 $(a-a)=0 \in S$,即 $0$ 总在数的模中。

而对于整数的模,若 $a \in S$,则 $a + a = 2a\in S$,类似的, $2a + a = 3a\in S$,依此类推。而 $0 \in S$,故 $0 - a \in S$。综合以上两个性质,就有,若 $a \in S$,则 $\forall n \in \mathbb Z$,$na \in S$。

\begin{theorem}{}
除零模外,任何整数模都是某正整数 $d$ 的整数倍构成的集合。
\end{theorem}
\textbf{证明}:由 $\forall a \in S$, $\forall n \in \mathbb Z, na \in S$,可知 $\forall a, b \in S$, $\forall n, m \in \mathbb Z, (na+mb) \in S$。我们考虑 $S$ 中的最小正整数 $d$,而对于正数 $n \in S$,则对于所有的整数 $d$ 都有 $n - zd \in S$。