% 庞加莱猜想(综述)
% license CCBYSA3
% type Wiki

本文根据 CC-BY-SA 协议转载翻译自维基百科\href{https://en.wikipedia.org/wiki/Poincar\%C3\%A9_conjecture}{相关文章}。

在几何拓扑的数学领域,庞加莱猜想(英国发音:/ˈpwæ̃kæreɪ/,美国发音:/ˌpwæ̃kɑːˈreɪ/,法语发音:[pwɛ̃kaʁe])是一个关于3-球体的定理,3-球体是四维空间中界定单位球的超球面。

该猜想最初由亨利·庞加莱于1904年提出,定理涉及到一些局部上看起来像普通三维空间,但在规模上是有限的空间。庞加莱假设,如果这样的空间具有一个额外的特性,即该空间中的每一条环路都可以连续地收缩到一个点,那么这个空间必定是一个三维球体。解决该猜想的努力推动了20世纪几何拓扑领域的许多进展。

最终的证明基于理查德·S·汉密尔顿使用Ricci流来解决该问题的方案。通过在Ricci流理论中发展一系列新的技术和结果,格里戈里·佩雷尔曼能够修改并完成汉密尔顿的方案。在2002年和2003年发布于arXiv的论文中,佩雷尔曼展示了他证明庞加莱猜想(以及威廉·瑟尔斯顿的更强的几何化猜想)工作的过程。随后几年,几位数学家研究了他的论文,并详细阐述了他的工作。

汉密尔顿和佩雷尔曼在该猜想上的工作被广泛认为是数学研究的一个里程碑。汉密尔顿因其贡献获得了邵逸夫奖和勒罗伊·P·斯蒂尔研究奠基奖。《科学》杂志将佩雷尔曼证明庞加莱猜想的成果评为2006年年度科学突破。[5] 克雷数学研究所将庞加莱猜想列为著名的千年奖问题之一,并为该猜想的解决提供了100万美元的奖赏。[6] 佩雷尔曼拒绝了这一奖项,称汉密尔顿的贡献与他自己的贡献相等。[7][8]
\subsection{概述}
\begin{figure}[ht]
\centering
\includegraphics[width=6cm]{./figures/bb28959dfa154fe6.png}
\caption{这个环面上的两个彩色环都不能连续地收缩到一个点。环面与球面不是同胚的。} \label{fig_PJLCX_1}
\end{figure}
庞加莱猜想是几何拓扑学领域的一个数学问题。用该领域的术语来说,它表述如下:

庞加莱猜想:  
每一个封闭、连通且基本群为平凡的三维拓扑流形,与三维球面是同胚的。

熟悉的形状,比如球体的表面(在数学中称为二维球面)或环面的表面,都是二维的。球体的表面具有平凡的基本群,意味着在表面上画出的任何环都可以连续地变形为一个点。相比之下,环面的表面具有非平凡的基本群,因为表面上存在一些环无法这样变形。两者都是拓扑流形,且都是闭合的(意味着它们没有边界,占据有限的空间区域)和连通的(意味着它们由一个整体组成)。当两个闭合流形的点能够通过连续的方式重新映射到彼此时,称它们是同胚的。由于基本群的(非)平凡性在同胚变换下是不变的,因此可以推断二维球面和环面不是同胚的。

庞加莱猜想的二维类比表明,任何闭合且连通的二维拓扑流形,如果不是二维球面同胚的,则必定包含一个无法连续收缩为一点的环。(这可以通过环面的例子来说明,如上所述。)通过对闭合且连通的二维拓扑流形的分类,已知这一类比是成立的,早在19世纪60年代就以各种形式被理解。在更高维度中,闭合且连通的拓扑流形没有简单的分类方法,这使得庞加莱猜想的解决变得更加复杂。
\subsection{历史}  
\subsubsection{庞加莱的问题}  
在19世纪,伯恩哈德·黎曼和恩里科·贝蒂开始研究流形的拓扑不变量。[9][10] 他们引入了贝蒂数,它将每个流形与一组非负整数关联起来。黎曼曾证明,闭合且连通的二维流形可以通过其贝蒂数完全表征。在他的1895年论文《分析位置》(于1892年公布)中,庞加莱证明了黎曼的结果无法扩展到更高维度。[11][12][13] 为此,他引入了基本群作为一种新的拓扑不变量,并能够展示一些具有相同贝蒂数但不同基本群的三维流形的例子。他提出了一个问题:基本群是否足以在拓扑上表征一个流形(给定维度),尽管他并没有继续追求答案,仅仅表示这将“需要漫长且困难的研究”。[12][13][14]

