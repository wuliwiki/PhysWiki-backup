% 快速排序
% 快速排序|算法|排序|C++

排序算法:

排序算法有很多种,例如:快速排序、归并排序、插入排序、冒泡排序、堆排序等等.

这里只介绍几种常用的排序算法:快速排序、归并排序和堆排序.

\subsection{快速排序:}

快排的主要思想是基于分治.分治的大概流程可以分为三步:分解 -> 解决 -> 合并,快排大致也是这么实现的,我们这里着重讨论一下排序算法.

快速排序的期望时间复杂度为 $\mathcal{O}(n \log_2 n)$,最坏时间复杂度为 $\mathcal{O}(n^2)$,因为快速排序的常数非常小,所以实际效率很高.

快速排序一般分为 $3$ 步:
\begin{enumerate}
\item 找分界点
\item 划分区间
\item 递归处理左右两段
\end{enumerate}

快排与归并排序唯一一点不同的是不用合并,因为子数组都是原址排序的,所以不需要合并操作,数组已经有序.但归并排序的核心是合并操作.

分界点一般确定为 $\dfrac{l+r}{2}$,然后维护两个指针,第一个指针从第一个位置开始走,如果指向的值如果小于分界点,那么第一个指针就往后走,第二个指针从最后一个位置开始往前走,如果第二个指针指向的值大于分界点,那么就往前走.最终使得对于分界点左边的数都 $\leq \text{mid}$,分界点右边的数都 $\geq \text{mid}$,排序成功两个指针一定会相遇,如果最后两个指针没办法再继续走下去的时候,第一个的指针在第二个指针的前面,这种情况一般是:第一个指针指向的元素大于等于分界点了,需要放到右半边、或者是第二个指针指向的元素小于等于分界点,需要放到左半边.那么就需要交换一下两个指针所指向的值,最后再递归左右两边继续做上述的操作.


\begin{lstlisting}[language=cpp]
void quick_sort(int q[], int l, int r)
{
    if (l >= r) return;  // 已经排好序了,直接返回
		
    int i = l - 1, j = r + 1;
    int mid = q[l + r >> 1];  // mid 为分界点
    while (i < j)
    {
        do i ++ ; while (q[i] < mid);
        do j -- ; while (q[j] > mid);
        if (i < j) swap(q[i], q[j]);
    }
    quick_sort(q, l, j), quick_sort(q, j + 1, r); // 递归左右两边
    
    return;
}

\end{lstlisting}


这里使用算法导论中的\textbf{循环不变式}简单的证明一下快速排序的正确性:

待证问题:\verb|while| 循环结束后,\verb|q[l..j] <= mid| 并且 \verb|q[j + 1..r] >= mid|.

循环不变式:\verb|q[l...i] <= x| 并且 \verb|q[j ... r] >= x|.

需要证明这个循环不变式在第一轮迭代之前是成立的,并且在每一轮迭代后也是成立的.在循环结束之后,可以证明出待证问题.

\textbf{初始化:} 循环的第一轮迭代开始之前,\verb|i = l - 1, j = r + 1|.所以 \verb|q[l...i]| 和 \verb|q[j ... r]| 都为空,循环不变式显然成立.

\textbf{保持:}需要考虑两种情况:
当对于 $i$ 指针的 $\text{do while}$ 循环结束后,有:\verb|q[l...i - 1] <= x, q[i] >= x|.当对于 $j$ 指针的 $\text{do while}$ 循环结束后,有:\verb|q[j + 1...r] >= x, q[j] <= x|.进行交换操作后,会使得:\verb|q[l...i] <= x, q[j...r] >= x|.所以再下次迭代开始之前,循环不变式同样得到保持.

\textbf{终止:}

当终止时,\verb|l >= r|,由于最后的一轮迭代中的 if 判断不一定执行