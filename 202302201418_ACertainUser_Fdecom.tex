% 力的分解与合成
% 平行四边形法则|几何矢量|三角形法则|合力|分力

\pentry{几何矢量\upref{GVec}}

\subsection{力的合成}
\begin{figure}[ht]
\centering
\includegraphics[width=10cm]{./figures/Fdecom_2.pdf}
\caption{一组力的作用效果等效于他们的合力} \label{Fdecom_fig2}
\end{figure}

在经典力学中, 力可以用几何矢量\upref{GVec}表示. 力的分解与合成可以看作一个\textbf{基本假设}. 这个假设是牛顿运动定律\upref{New3}的基础, 因为牛顿三定律中的 “力” 都是指质点所受的合力.

当若干个力 $\bvec F_i$ ($i = 1, 2, \dots, N$)作用在同一个质点上时, 等效于一个力 $\bvec F$ 作用在同一个质点上.
\begin{equation}\label{Fdecom_eq1}
\bvec F_1 + \bvec F_2 + \dots + \bvec F_N = \sum_{i=1}^N \bvec F_i =  \bvec F 
\end{equation}

注意这里的加号表示几何矢量\upref{GVec}的加法而不是数的加法. 我们把 $\bvec F$ 叫做 $N$ 个 $\bvec F_i$ 的\textbf{合力}, 每个 $\bvec F_i$ 叫做一个\textbf{分力}. 

这里所说的 “等效” 可以指这个质点受力后的运动情况,例如我们基于牛顿第二定律\upref{New3} $\bvec F = m \bvec a$计算粒子的运行轨迹,那么 $\bvec F$ 指作用在粒子上的合力$\bvec F_\text{合}$,即有 $\bvec F_\text{合}=m \bvec a$。“等效”也可以指物体发生的形变, 例如该质点固定在弹簧上, 弹簧发生的形变。

\subsubsection{力的加法法则}
\begin{figure}[ht]
\centering
\includegraphics[width=5cm]{./figures/Fdecom_3.pdf}
\caption{力的合成法则} \label{Fdecom_fig3}
\end{figure}
回顾两个几何矢量的加法\upref{GVecOp}, 我们就得到了所谓的\textbf{平行四边形法则}或者\textbf{三角形法则}. 若 $N > 2$, 将所有力 “首尾相接” 即可得到合力. 注意这个过程不需要坐标系的概念. 若建立了直角坐标系, 我们也可以先计算这些矢量的坐标, 然后使用坐标计算矢量加法(\autoref{Gvec2_eq8}~\upref{Gvec2}).

\subsection{力的分解}
\begin{figure}[ht]
\centering
\includegraphics[width=10cm]{./figures/Fdecom_1.pdf}
\caption{实际常基于坐标轴、切面、粒子运动方向等,将力分解为一组正交(互相垂直)的力} \label{Fdecom_fig1}
\end{figure}
我们还可以反向运用\autoref{Fdecom_eq1} ,将一个力分解为多个力:
\begin{equation}\label{Fdecom_eq2}
\bvec F = \sum_i \bvec F_{i}
\end{equation}
可见公式完全相同,力的合成与分解只是同一个现象的两面。\autoref{Fdecom_eq2} 从左到右的过程叫做\textbf{力的分解}, 从右到左的过程叫做\textbf{力的合成}.

我们甚至可以进行多次分解, 即继续令某个力等于若干力相加:
\begin{equation}
\bvec F_i = \sum_j \bvec F_{i,j}
\end{equation}

那么 合力 $\bvec F$ 就可以最终分解为:
\begin{equation}
\bvec F = \sum_{i,j} \bvec F_{i,j}
\end{equation}


\begin{example}{小滑块}
\textsl{不管你喜不喜欢},小滑块模型是展示力合成与分解的经典模型。我们吃醋假定小滑块是一个质点。
\begin{figure}[ht]
\centering
\includegraphics[width=8cm]{./figures/Fdecom_4.pdf}
\caption{小滑块} \label{Fdecom_fig4}
\end{figure}

\begin{figure}[ht]
\centering
\includegraphics[width=5cm]{./figures/Fdecom_10.pdf}
\caption{上图的放大} \label{Fdecom_fig10}
\end{figure}

我们可以将运用力的合成法则,找到三个力的合力:
\begin{figure}[ht]
\centering
\includegraphics[width=5cm]{./figures/Fdecom_5.pdf}
\caption{力的合成} \label{Fdecom_fig5}
\end{figure}

也可以将$G$分解至垂直与平行于斜面的方向。这种分解往往有助于我们做题,因为$\bvec N$与$\bvec f$往往是未知的。
\begin{figure}[ht]
\centering
\includegraphics[width=5cm]{./figures/Fdecom_6.pdf}
\caption{力的分解} \label{Fdecom_fig6}
\end{figure}

\end{example}
% 这仍然符合分解的定义, 即一个力矢量表示为多个力矢量相加, 本质上并无不同.

\begin{example}{刚体的复杂情况}
%感觉需要单独开一章放这个内容?
以上的例子和论述中,我们探讨的都是一个质点。如果对象是一个刚体,那么受力问题会复杂的多。此处只是告诉你要谨慎地对待刚体,并不涉及过多具体细节。

\begin{figure}[ht]
\centering
\includegraphics[width=10cm]{./figures/Fdecom_7.pdf}
\caption{即使合力为零,刚体也会“运动”} \label{Fdecom_fig7}
\end{figure}
如 \autoref{Fdecom_fig7} 所示,尽管这个刚体受的合力为零,刚体也不会静止不动,而是发生转动。这是因为此时力的力矩和不为零。\upref{RBEqM}

\begin{figure}[ht]
\centering
\includegraphics[width=10cm]{./figures/Fdecom_8.pdf}
\caption{即使合力、力矩和为零,刚体的内部受力也不同} \label{Fdecom_fig8}
\end{figure}
如 \autoref{Fdecom_fig8}  所示,即使(材料力学中的)刚体的所受合力、力矩和为零,他内部的受力\upref{INTFRC} 也不同。在材料力学看来,外力会导致刚体如同弹簧一样发生微小形变,因此处于静力平衡的刚体与不受力的刚体是不同的。不过,在非材料力学的理想刚体中,一般不需要考虑这么多。
\end{example}
