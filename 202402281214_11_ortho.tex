% 正交变换与对称变换
% license Usr
% type Tutor


\begin{issues}
\issueDraft 全文替换向量
\end{issues}

本文用“$(x,y)$”表示对任意两个向量作内积。
\subsection{正交变换}
\begin{definition}{}
定义实内积空间$V$上的满射$\mathcal A$。对于任意$(x,y)\in V$,若有:
\begin{equation}
(x,y)=(\mathcal A x,\mathcal A y)~,
\end{equation}
则称$\mathcal A$是$V$上的\textbf{正交变换}。也就是说,正交变换是保内积不变的满射,从而保向量长度和向量之间的夹角不变。
\end{definition}
实际上,正交变换是线性变换。
\begin{theorem}{}
实内积空间$V$上的正交变换$\mathcal A$一定是线性映射。
\end{theorem}
\textbf{证明:}
\begin{equation}
\begin{aligned}
|\mathcal A(x+y)-(\mathcal Ax+\mathcal Ay)|^2&=(x+y)^2+x^2+y^2-2\left(\mathcal A(x+y),\mathcal Ax+\mathcal Ay\right)\\
&=(x+y)^2+x^2+y^2-2(x+y,x)-2(x+y,y)\\
&=0
\end{aligned}
~,\end{equation}

因而$\mathcal A(x+y)=\mathcal Ax+\mathcal Ay$。同理可得$\mathcal A(kx)=k\mathcal A(x)$。

设$\{\bvec e_i\}^k_{i=1}$是$V$上的一组基,由线性性可知,$\{\mathcal A\bvec e_i\}^k_{i=1}$也是线性无关组。又因为$\mathcal A$是满射,所以$\{\mathcal A\bvec e_i\}^k_{i=1}$是$V$上的一组基,则该线性映射既单又满,是“同构映射”。由正交变换保内积可知,$\mathcal A$\textbf{把标准正交基映射为标准正交基。}

总结上述讨论,易证:$\mathcal A$是正交变换$\Longleftrightarrow \mathcal A$\textbf{把标准正交基映射为标准正交基。}后者是正交矩阵的定义,可见正交矩阵是正交变换的矩阵表示。\textbf{正交矩阵保二次型不变的性质等价于正交变换的内积定义,}\footnote{广义内积即二次型决定的对称双线性函数:$(x,y)\equiv \frac{1}{2}(q(x+y)-q(x)-q(y))$}或者说——

$\mathcal A$是正交变换$\Longleftrightarrow $对任意$x\in V,|\mathcal Ax|=|x|$。

令$O(n)$表示$n$维线性空间$V$上全体正交变换的集合。由定义可知,该集合具有如下性质:
\begin{enumerate}
\item 封闭性。若$A,B\in O(n)$,则$AB\in O(n)$;
\item 结合性。若$A,B,C\in O(n)$,则$A(BC)=(AB)C$;
\item 单位元存在性。$I\in O(n)$;
\item 逆元存在性。若$A\in O(n)$,则$A^-1\in O(n)$;
\end{enumerate}
所以,正交变换构成一个群。

在欧几里得空间中,正交矩阵的行列式为$1$或者$-1$,常称行列式为$1$的正交变换为\textbf{第一类的(旋转)};行列式为$-1$的正交变换则是\textbf{第二类的}。
\subsubsection{正交变换的特征值}
由上述讨论可知,正交变换若有特征向量,则特征值的模长为$1$,从而保证向量的模长不变。
\begin{corollary}{}
 设$\lambda$为正交变换$A$的特征值,则$|\lambda|=1$。若$V$定义在实数域上,$\lambda=\pm 1$。
\end{corollary}
\begin{corollary}{}
若$V$是奇数维线性空间,$A$必有特征值$1$。
\end{corollary}
\textbf{证明:}
实数域上易证推论2,下面讨论复数域的情况。由于特征值是特征多项式的根,且一元多项式方程里复根成对出现(若$\lambda_0$为特征值,则$\lambda_0^{*}$也为特征值),所以$A$有偶数个模为$1$的复根,这些特征值连乘为$1$。结合行列式等于特征值连乘这一\autoref{the_MatEig_3}~\upref{MatEig},可知剩下奇数个连乘结果为1的实特征值,得证。
\subsubsection{正交变换的块对角形式}
为了简化正交矩阵的形式,我们先来证明几个定理。
\begin{theorem}{}
设$f$为\textbf{实线性空间}$V$上的线性映射,存在$f$的不变子空间$W\subseteq V$,且$\opn{dim}W\in\{1,2\}$。
\end{theorem}
\textbf{证明:\footnote{参考Jier Peter的《代数学基础》}}

若$f$有若干个实特征值,则其对应的特征向量是$V$上的一维不变子空间。

将$n$维$V$复化为$U$。若$\{\bvec e_i\}$为$V$上的一组基,则任意$x\in U$可表示为:
\begin{equation}
x=a^i\bvec e_i+\mathrm i b^j\bvec e_j=\bvec u+\mathrm i\bvec v~.
\end{equation}
其中$a^i,b^i\in \mathbb R,\bvec u,\bvec v\in V$。显然$U\supseteq V$。
在复数域上,$f$有$n$个特征值。设有$k$个实特征值,则对应的特征向量是$V$上的一维不变子空间。设复特征值表示为$a+\mathrm ib$,对应特征向量表示为$x+\mathrm i y\,(x,y\in V)$。利用$f$的线性可得:
\begin{equation}
\begin{aligned}
f(x+\mathrm i y)&=(a+\mathrm i b)(x+\mathrm i y)\\
&=(ax-by)+\mathrm i(ay+bx)\\
&=f(x)+\mathrm if(y)
\end{aligned}~.
\end{equation}
即:
\begin{equation}
\begin{aligned}
f(x)&=ax-by\\
f(y)&=ay+bx
\end{aligned}~,
\end{equation}
所以$\opn{Span}\{x,y\}$为$f$在$V$上的一个二维不变子空间。
\begin{theorem}{}
设$A$为$V$上的正交变换,若$W$为其不变子空间,则$W^{\bot}$也是其不变子空间。
\end{theorem}
设$\{\bvec e_i}$为$W$上的一组基。
\subsection{对称变换}


