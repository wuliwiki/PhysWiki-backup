% Docker 笔记

\begin{issues}
\issueDraft
\end{issues}

\subsection{Linux 安装}
\subsubsection{apt 安装}
\begin{itemize}
\item ubuntu x86/64 安装参考\href{https://docs.docker.com/install/linux/docker-ce/ubuntu/}{这里}.
\begin{lstlisting}[language=bash]
sudo apt-get remove docker docker-engine docker.io containerd runc
sudo apt-get update
sudo apt-get install \
    apt-transport-https \
    ca-certificates \
    curl \
    gnupg-agent \
    software-properties-common
sudo add-apt-repository \
   "deb [arch=amd64] https://download.docker.com/linux/ubuntu \
   $(lsb_release -cs) \
   stable"
sudo apt-get update
sudo apt-get install docker-ce docker-ce-cli containerd.io
\end{lstlisting}
\item 测试安装成功, 用 \verb`sudo docker run hello-world`
\end{itemize}

\subsubsection{deb 安装包安装}
\begin{itemize}
\item 但是如果 apt 不能用的话, 也可以下载 deb 文件离线安装
\item 安装包下载 https://download.docker.com/linux/ubuntu/dists/ , 选择系统版本代号, 然后 pool/stable/amd64/
\item 选好了以后需要下载三个安装包, 依次安装(如果次序不对也问题不大,会提示 dependency 找不到的错误)
\item 三个安装包依次是 containerd.io, docker-ce-cli, docker-ce, 用 \verb`dpkg -i xxx.deb` 安装即可
\end{itemize}

\subsection{基础}
\begin{itemize}
\item docker 在功能上基本和虚拟机一样, 但是占用资源要少得多, 因为共享内核
\item docker 本质上就是一个进程
\item windows 和 mac 上有 docker desktop, 建议使用
\item 用了 docker 就不能用 virtualbox 了, 必须要在 eufi 把 Hyper V 关掉或打开才可以
\item 最新的 windows docker 需要 WSL2 才能使用了
\item 重启以后启动 docker, 登录, 用 powershell \verb`docker --version` 检查版本
\item \verb`docker run hello-world` 测试最简单的 image
\item \verb`docker images` 可以检查本地所有 image
\item \verb`docker ps -a` 可以检查本地所有 container/process
\item 要下载 image, 用 \verb`docker pull` 例如 \verb`docker pull [image_name]`
\item 注意区分 container 和 image, image 相当于 VirtualBox 里面的 snapshot, 而 container 相当于现在在运行的虚拟机, \verb`container commit` 以后可以生成 image
\item \verb`docker run -it -d IMAGE_NAME/ID` 从 image 创建 container
\item \verb`docker exec -it CONTAINER_ID bash` 进入某个 container 的 bash
\item docerk image 没有简单的上锁设置,就算修改了 passwd. 能执行 \verb`sudo docker...` 命令的人都可以自由访问本地的任何 container
\item \verb`docker stop CONTAINER_ID` 停止 container
\item \verb`docker start CONTAINER_ID` 开始 container
\item \verb`docker rm [-f] CONTAINER_ID` 删除 container
\item \verb`docker image rm [-f] IMAGE_NAME/ID` 删除 image
\item \verb`docker commit CONTAINER_ID USR_NAME/REPO_NAME` 会将 container commit, 也可以用 \verb`REPO_NAME:TAG_NAME` 指定 tag
\item \verb`docker login` 用于登录 docker hub, 如果 \verb`permission denied`, 就加 \verb`sudu`
\item \verb`docker push USR_NAME/REPO_NAME` 可以直接将 commit 的 image push 到 docker hub, 也可以用 \verb`REPO_NAME:TAG_NAME` 指定 tag
\end{itemize}


\subsection{docker 网络}
\begin{itemize}
\item \href{https://www.freecodecamp.org/news/how-to-get-a-docker-container-ip-address-explained-with-examples/}{简单介绍}
\item container 自己的 ip 默认为 \verb`172.17.xxx.xxx`
\item 如果要把 container 的 port 映射到 host 的 port, 使用 \verb`docker run -p HOST_PORT:CONTAINER_PORT IMAG_NAME`, 这样如果 docker 有一个 web server 就可以从外面访问了
\end{itemize}

\subsection{ubuntu image 缺少的功能}
\begin{itemize}
\item \verb`apt install sudo`
\item \verb`apt install bash-completion`, 然后在 bashrc 中加上
\begin{lstlisting}[language=bash]
if ! shopt -oq posix; then
  if [ -f /usr/share/bash-completion/bash_completion ]; then
    . /usr/share/bash-completion/bash_completion
  elif [ -f /etc/bash_completion ]; then
    . /etc/bash_completion
  fi
fi
\end{lstlisting}
\end{itemize}
