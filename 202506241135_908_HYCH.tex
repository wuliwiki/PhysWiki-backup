% 海因茨·霍普夫(综述)
% license CCBYSA3
% type Wiki

本文根据 CC-BY-SA 协议转载翻译自维基百科 \href{https://en.wikipedia.org/wiki/Heinz_Hopf}{相关文章}。

海因茨·霍普(Heinz Hopf,1894年11月19日-1971年6月3日)是德国数学家,研究领域包括动力系统、拓扑学和几何学。
\subsection{早期生活与教育}
霍普出生于德国帝国的格雷布申(现波兰弗罗茨瓦夫的格拉比辛),父亲是威廉·霍普,母亲是伊丽莎白(原姓基尔赫纳)。他的父亲出生为犹太人,霍普出生一年后父亲皈依了新教;母亲来自一个新教家庭。

霍普于1901年至1904年就读于卡尔·米特尔豪斯高年级男子学校,随后进入布雷斯劳的科尼格·威廉中学。他从小便显示出数学天赋。1913年,他进入上西里西亚的弗里德里希·威廉大学,聆听了恩斯特·施泰尼茨、阿道夫·克内塞尔、马克斯·德恩、厄尔哈德·施密特和鲁道夫·斯图尔姆的讲座。第一次世界大战爆发后,霍普积极参军,曾两次受伤,并于1918年获得铁十字勋章(一级)。

战争结束后,霍普继续在海德堡(1919/20冬季和1920年夏季)和柏林(从1920/21年冬季开始)继续他的数学教育。他在路德维希·比伯巴赫的指导下学习,并于1925年获得博士学位。
\subsection{生涯}
在他的博士论文《流形的拓扑与度量之间的联系》(德文原题:Über Zusammenhänge zwischen Topologie und Metrik von Mannigfaltigkeiten)中,霍普证明了:任何单连通的、完全的、具有常截面曲率的黎曼三维流形,在全局上等距于欧几里得空间、球面或双曲空间。他还研究了超曲面上向量场零点的指标,并将其总和与曲率联系起来。大约六个月后,他给出了一个新的证明,说明流形上向量场零点指标的总和与所选向量场无关,并等于该流形的欧拉示性数。这个定理现在被称为庞加莱–霍普定理(Poincaré–Hopf 定理)。

霍普在获得博士学位后的那一年曾在哥廷根大学工作,当时大卫·希尔伯特、理查德·柯朗特、卡尔·龙格和埃米·诺特都在该校任教。在那里,他结识了帕维尔·亚历山德罗夫,并与他建立了终身友谊。

1926 年,霍普回到柏林,在那里讲授组合拓扑课程。1927/28 学年,他与亚历山德罗夫一同获得洛克菲勒基金会奖学金,在普林斯顿大学度过。那时,所罗门·勒夫谢茨、奥斯瓦尔德·维布伦以及 J. W. 亚历山大都在普林斯顿任教。也正是在这一时期,霍普发现了从
$S^3 \to S^2$ 的映射的霍普不变量(Hopf invariant),并证明霍普纤维化(Hopf fibration)具有不变量 1。

1928 年夏天,霍普回到柏林,在柯朗特的建议下与帕维尔·亚历山德罗夫合作编写一本拓扑学著作,计划分三卷出版,但最终只完成了一卷,于 1935 年出版。

1929 年,他拒绝了普林斯顿大学的职位邀请。1931 年,霍普接替赫尔曼·外尔在苏黎世联邦理工学院(ETH)的职位。1940 年他再次收到普林斯顿的邀请,但再次拒绝。然而,两年后,由于纳粹没收了他的财产,他不得不申请瑞士国籍——他父亲的基督教皈依并未说服德国当局将他们视为“雅利安人”。

1946/47 年和 1955/56 年,霍普访问美国,在普林斯顿大学逗留,并在纽约大学和斯坦福大学讲学。他曾于 1955 年至 1958 年担任国际数学联盟(IMU)主席。
