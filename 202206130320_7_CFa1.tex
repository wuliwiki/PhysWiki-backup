% 从分析力学到场论
% keys 经典场|狭义相对论|作用量|拉格朗日函数

%在相对论中创建目录:经典场论

\addTODO{预备知识待确定.}

\pentry{流形上的张量场\upref{TenMan},欧拉—拉格朗日方程\upref{Lagrng}}



所谓场论,就是研究场的运动,或者说变化的理论.

牛顿力学认为物质是由粒子构成的,而粒子是一种无大小的数学对象,并且被赋予了一个描述其特征的标量,称为“质量”.牛顿动力学所研究的,就是“粒子”这种对象运动的规律.最古典的处理方式,是用\textbf{牛顿三定律}来描述这种规律,相当于描述每时每刻粒子的运动状态和其改变;拉格朗日力学则提出了另一种研究范式,即从整体着手,研究哪些粒子运动轨迹是允许的,此时描述规律的方式变成了\textbf{最小作用量原理}.角度虽有不同,但两种处理方式都是描述“粒子”的运动.

现在,我们要从拉格朗日力学的方法出发,讨论如何用场论来处理牛顿力学.


\subsection{一个粒子的运动:从粒子观点到场论观点}

考虑一个粒子在三维空间中的运动.粒子的位置由一个三维向量函数$\bvec{r}(t)$描述,它是时间$t$的函数.定义拉格朗日函数
\begin{equation}
\mathcal{L}=\frac{1}{2}\qty(\dot{\bvec{r}}(t))^2-V(\bvec{r})
\end{equation}
那么就可以得到其所关联的作用量,从给定的时间$a$到$b$:
\begin{equation}
\mathcal{S}=\int ^b_a \mathcal{L} \dd t
\end{equation}
于是,粒子的合法运动就是满足“在任意时间段中的作用量取最小值”的运动.用一点变分的技巧,我们就能得出粒子运动的欧拉拉格朗日方程:
\begin{equation}\label{CFa1_eq1}
\frac{\dd}{\dd t}\frac{\partial\mathcal{L}}{\partial \frac{\dd}{\dd t}\bvec{r}} = \frac{\partial\mathcal{L}}{\partial \bvec{r}}
\end{equation}

\autoref{CFa1_eq1} 中没有写$\dot{\bvec{r}}$,而是写了$\frac{\dd}{\dd t}\bvec{r}$,是为了方便接下来引出场论的观点.

换一种视角来理解这个




















