% 加速度规范

\pentry{速度规范\upref{LVgaug}}

本文使用原子单位制\upref{AU}. 首先注意加速度规范并不是一种规范而只是薛定谔方程的一种变换, 说它是规范只是习惯上的叫法. 改变换也叫做 \textbf{Kramers-Henneberger frame} 或 \textbf{K-H frame}.

一个静止粒子在电磁波到来之前处于静止, 那么接下来它在电磁波作用下的位移为
\begin{equation}
\bvec \alpha(t) = q\int_{-\infty}^t \bvec A(t') \dd{t'}
\end{equation}
由于电磁波没有 DC 分量

\begin{equation}
\chi = \bvec \alpha \vdot \bvec p
\end{equation}

\begin{equation}
\Psi_V(\bvec r, t) = \E^{-\I \bvec \alpha \vdot \bvec p} \Psi_A(\bvec r, t)
\end{equation}
这里的 $\bvec p$ 应该是广义动量算符.
