% 量子力学的诞生
% keys 量子力学|黑体辐射|原子模型

量子力学的提出是为了解决 19 世纪末期经典物理学无法解释的几个实验事实.不同于爱因斯坦以一己之力发现的狭义相对论,量子力学是许许多多物理学家的智慧凝结而成的结果.

经典物理学无法解释的问题有黑体辐射、光电效应、原子的结构模型、固体比热……在学习的过程中,我们将体会到 20 世纪初的物理学家们是如何被迫一点一点地放弃他们曾经钟爱的物理学概念,以及那些了不起的大师们,诸如海森堡、薛定谔和狄拉克等,除去他们在早期的一些不正确的起始点和错误的转折点外,最终怎样成功地阐明了我们今天所熟知的量子力学.

\subsection{黑体辐射}
黑体指的是这样一种理想空腔,它能将设在其上的电磁波完全吸收.自然,它也会以电磁波的形式辐射出自身的能量,最后达到热力学平衡.

1896 年,维恩从热力学的普适理论出发,并结合实验数据,提出了如下半经验公式(\textbf{维恩公式})
\begin{equation}
E_\nu \dd \nu=C_1\nu^3 \exp(-C_2 \nu /T) \dd \nu
\end{equation}
其中 $E_\nu$ 为单位体积内频率在 $\nu$ 到 $\nu+\dd \nu$ 内的辐射能量密度.维恩因此获得了 1911 年的诺贝尔奖.

然而进一步实验表明,维恩公式在长波极限下与实验符合的不好.为此,瑞利和金斯提出了\textbf{瑞利金斯公式}:
\begin{equation}\label{QMborn_eq1}
E_\nu \dd \nu=\frac{8\pi}{c^3}kT\nu^2 \dd \nu
\end{equation}

瑞利金斯公式是基于经典统计中的\textbf{能量均分定理}\upref{EqEng}提出的.其思路是,考虑空腔内的电磁波模式,每一个振动模式上的平均能量为 $kT/2$,而对于一个有限立方体而言,每种振动模式下的电磁波具有如下的一般形式
\begin{equation}
\bvec E=\bvec E_0 \sin\frac{n_1\pi x}{a} \sin\frac{n_2\pi y}{b}\sin \frac{n_3\pi z}{c}
\end{equation}
其中 $n_i=0,\pm 1,\pm 2,\cdots$.每个模式占据的相空间体积为 $\Delta k=\pi^3/abc=\pi^3/V$.波数和频率的关系式是 $|\bvec k| = 2\pi\nu/c$,所以频率在 $\nu$ 到 $\nu+\dd \nu$ 内的电磁波模式有 $(16\pi \nu^2V/c^3 )\dd \nu$ 个.根据能量均分定理,就可以得到\autoref{QMborn_eq1}.

瑞利金斯公式在低频区域与实验符合得很好,但在高频极限下辐射能量将趋于无穷大,这被称为\textbf{紫外灾难}.这暗示了在量子统计中,能量均分定理需要被重新考量.

1900年,普朗克利用插值技巧得出了黑体辐射公式,并赋予了它能量量子化的解释,迈出了量子力学发展的第一步.对于一个特定的电磁波模式,能量只能取分立的值而非连续的值,并且服从玻尔兹曼分布:每种值出现的概率与 $e^{-E/kT}$ 成正比(也就是说,服从玻尔兹曼分布).
\begin{equation}
E_\nu \dd \nu=\frac{8\pi h}{c^3}\frac{\nu^3}{\E^{h\nu/(k T)} - 1}
\end{equation}

在量子力学中,光子是电磁场的激发态,光子作为一种准粒子,具有类似玻色子的性质,因此普朗克公式是玻色-爱因斯坦统计的结果.

\subsection{原子的玻尔模型}
1911年,卢瑟福通过 $\alpha$ 粒子散射实验\upref{RuthSc},提出了电子绕原子核运动的模型.然而
\addTODO{未完成}