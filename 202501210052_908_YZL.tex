% 杨振宁(综述)
% license CCBYSA3
% type Wiki

本文根据 CC-BY-SA 协议转载翻译自维基百科\href{https://en.wikipedia.org/wiki/Yang_Chen-Ning}{相关文章}。

\begin{figure}[ht]
\centering
\includegraphics[width=6cm]{./figures/0a203240e6e08647.png}
\caption{} \label{fig_YZL_1}
\end{figure}
杨振宁(英文名:Frank Yang,亦称C. N. Yang,简体字:杨振宁,繁体字:楊振寧,拼音:Yáng Zhènníng,生于1922年10月1日),是中国的理论物理学家,对统计力学、可积系统、规范理论以及粒子物理学和凝聚态物理学做出了重要贡献。他与李政道共同获得了1957年诺贝尔物理学奖,因其关于弱相互作用的宇称不守恒的工作。两人提出,宇称守恒定律在所有其他物理过程中都成立,但在所谓的弱核反应中被打破,弱核反应是指那些导致β粒子或α粒子发射的核反应。杨振宁还因与罗伯特·米尔斯合作发展了非阿贝尔规范理论(广为人知的杨–米尔斯理论)而闻名。
\subsection{早年生活与教育}
杨振宁出生于中国安徽省合肥市。他的父亲杨克纯(1896–1973)是一位数学家,母亲罗孟华是家庭主妇。

杨振宁在北京上过小学和中学,1937年秋,随着日本侵华,他的家庭搬回了合肥。1938年,他们搬到了云南昆明,国立西南联合大学(当时的国立西南联合大学设在昆明)也在那里。同年,杨振宁作为二年级学生,通过了入学考试,并进入了国立西南联合大学学习。他于1942年获得了理学学士学位,论文题目是关于群论在分子光谱中的应用,导师为吴大任教授。

杨振宁继续在该校攻读研究生课程两年,导师为王竹溪教授,研究方向为统计力学。1944年,他获得了清华大学的理学硕士学位,清华大学当时在抗日战争期间迁至昆明。随后,杨振宁获得了由美国政府设立的“庚子赔款奖学金”,该奖学金来自中国在庚子赔款中所支付的一部分资金。杨振宁前往美国的计划延迟了一年,在这一年里,他曾在一所中学担任教师,并学习了场论。

1946年1月,杨振宁进入芝加哥大学,并跟随爱德华·泰勒(Edward Teller)教授学习。1948年,他获得了哲学博士学位。
\subsection{职业生涯}
杨振宁在芝加哥大学继续待了一年,担任恩里科·费米的助理。1949年,他受邀到位于新泽西州普林斯顿的高级研究院进行研究,并开始与李政道的富有成效的合作。1952年,他成为该研究院的正式成员,1955年晋升为正教授。1963年,普林斯顿大学出版社出版了他的教科书《基本粒子》。1965年,杨振宁搬到斯托尼布鲁克大学,在那里他被任命为阿尔伯特·爱因斯坦物理学教授,并成为新成立的理论物理研究所的首任所长。今天,这个研究所被称为C. N. 杨理论物理研究所。

杨振宁于1999年从斯托尼布鲁克大学退休,并获得名誉教授的头衔。2010年,斯托尼布鲁克大学为了表彰杨振宁对学校的贡献,将其最新的宿舍楼命名为C. N. 杨楼。

杨振宁曾当选为美国物理学会、中国科学院、中央研究院、俄罗斯科学院和英国皇家学会的会士。他还是美国艺术与科学学院、美国哲学学会和美国国家科学院的会员。杨振宁获得了普林斯顿大学(1958年)、莫斯科国立大学(1992年)和香港中文大学(1997年)授予的名誉博士学位。

1971年,杨振宁在中美关系解冻后首次访问中国大陆,随后他为帮助中国物理界重建文化大革命期间被摧毁的研究氛围作出了努力。退休后,杨振宁回到清华大学,担任名誉所长,并在清华大学先进研究中心(CASTU)担任黄纪培-卢凯群教授。他还是两位邵逸夫奖创始成员之一,并且是香港中文大学的杰出教授。

杨振宁是1989年成立的亚太物理学会(AAPPS)的首任会长。1997年,AAPPS设立了C.N.杨奖,以表彰年轻研究人员。