% 银行家算法
% license CCBYSA3
% type Wiki

(本文根据 CC-BY-SA 协议转载自原搜狗科学百科对英文维基百科的翻译)

\textbf{银行家算法}算法(有时称为\textbf{检测算法})是由Edsger Dijkstra开发的资源分配和死锁避免算法,它通过模拟所有资源的预定最大可能量的分配来测试安全性,然后在决定是否允许继续分配之前进行“s状态” ,检查以测试所有其他待处理活动的可能死锁条件。该算法是在THE操作系统的设计过程中开发的,最初在EWD108中以荷兰语描述。[1] 当新进程进入系统时,它必须声明它可能需要的每种资源类型的最大实例数; 显然,该数字可能不会超过系统中的资源总数。 此外,当进程获取其所有请求的资源时,它必须在有限的时间内返回它们。

\subsection{资源}
银行家算法要发挥作用,需要知道三件事:
\begin{itemize}
\item 每个进程可以请求每个资源的最大值
\item 每个进程目前拥有多少分配的资源
\item 系统目前有多少资源可用
\end{itemize}
只有当请求的资源量小于或等于可用的资源量时,才能将资源分配给进程;否则,该过程将一直等到资源可用。

在实际系统中跟踪的一些资源是内存,信号量和接口访问。

银行家算法的名称源于这样一个事实,即该算法可用于银行系统,以确保银行不会耗尽资源,因为银行永远不会以不再满足资金的方式分配资金。 所有客户的需求[2]。 通过使用银行家的算法,银行确保当客户要求资金时,银行永远不会离开安全状态。 如果客户的请求不会导致银行离开安全状态,则会分配现金,否则客户必须等到其他客户存款足够。

实施银行家算法需要维护的基本数据结构:

设 n 是系统中的进程数, m 是资源类型的数量。那么我们需要以下数据结构:

\begin{itemize}
    \item \textbf{可用}: 长度为 m 的向量表示每种资源类型的可用资源数。如果 $\text{Available}[j] = k$, 则有 k 个资源类型为 $R_j$ 的实例可用。
    \item \textbf{最多}: $n \times m$矩阵定义每个进程的最大需求。如果 $\text{Max}[i, j] = k$, 则 $P_i$ 可以请求最多 k 个资源类型 $R_j$ 的实例。
    \item \textbf{分配}: $n \times m$ 矩阵定义当前分配给每个进程的每种类型的资源数。如果 $\text{Allocation}[i, j] = k$, 则过程 $P_i$ 当前被分配 k 个资源类型 $R_j$ 的实例。
    \item \textbf{需求}: $n \times m$ 矩阵表示每个进程的剩余资源需求。如果 $\text{Need}[i, j] = k$, 则 $P_i$ 可能需要 k 个或更多的资源类型 $R_j$ 实例来完成任务。注意: 需要 $\text{Need}[i, j] = \text{Max}[i, j] - \text{Allocation}[i, j]$。N = M - A。
\end{itemize}

