% 一维薛定谔方程的简单数值解(Matlab)

\begin{issues}
\issueDraft
\end{issues}

\pentry{薛定谔方程(单粒子一维)\upref{TDSE11}, 一维波动方程的简单数值解(Matlab)\upref{W1dNum}}

本文使用原子单位制\upref{AU}. 本文演示如何用差分法解一维薛定谔方程. 注意这是一种误差较大的方法,仅用于教学. 一些改进算法如 Crank-Nicolson 算法解一维含时薛定谔方程\upref{CraNic}, 以及Lanczos 算法\upref{Lanc}.
\begin{equation}
-\frac{1}{2m}\pdv[2]{x}\psi + V(x,t)\psi = \I\pdv{t}\psi
\end{equation}
把波函数取离散值, 令 $\psi_{i,n} = \psi(x_i,t_n)$. 用有限差分表示二阶导数(\autoref{DerDif_eq5}~\upref{DerDif}), 得
\begin{equation}
-\frac{1}{2m}\frac{\psi_{i-1,n} - 2\psi_{i,n} + \psi_{i+1,n}}{\Delta x^2} + V_{i,n}\psi_{i,n} = \I \frac{\psi_{i, n+1} - \psi_{i, n}}{\Delta t}
\end{equation}
于是
\begin{equation}
\psi_{i, n+1} = \frac{\I\Delta t}{2m\Delta x^2} (\psi_{i-1,n} - 2\psi_{i,n} + \Delta t\psi_{i+1,n}) + (1-\I \Delta t V_{i,n})\psi_{i,n}
\end{equation}
