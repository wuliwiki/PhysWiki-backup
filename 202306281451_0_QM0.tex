% 量子力学的基本原理(科普)
% keys 量子力学|动量|能量|波函数|本征态|本征函数|波包|有限深势阱|叠加|干涉|衍射

\pentry{动量和能量\upref{CM1}, 角动量\upref{AngMo}}

我们试图在科普部分以高中生能理解的方式简单介绍量子力学的基本假设, 有了这些基本假设, 一切其他结论都可以被推导出来。 注意这里介绍的是非相对论量子力学, 只适用于低速微观粒子。 另外量子力学中有不同的\textbf{诠释}, 本书使用常见的\textbf{哥本哈根诠释}。 本文只讨论一维的情况,即粒子沿直线运动。 另外我们不讨论粒子的自旋。

面向物理专业的量子力学基本原理,请参见\textbf{量子力学的基本原理(量子力学)}\upref{QMPrcp}。%量子力学的基本假设\upref{QMPos}

\subsection{波函数}

\subsubsection{德布罗意波}

爱因斯坦用解释光电效应时,提出了光子能量与其波频率的关系:$E=h\nu$,其中$E$是光子的能量,$\nu$是频率,$h$是普朗克常数。而狭义相对论认为光子的能量$E$与动量大小$p$之间的关系是$E=pc$,其中$c$是光速。由此还可以推知,光子的动量大小应该是$p=E/c=h\nu/c=h/\lambda$,其中$\lambda$是波长。

1924年,德布罗意(de Broglie)在他的博士学位论文中提出,一切物质都和光一样具有波粒二象性,每个粒子都关联于一个波,称之为\textbf{物质波}。他将爱因斯坦的光量子理论推广,认为上述波长、波频与动量、能量的关系适用于一切粒子。后来,每个粒子所关联的这个“物质波”,被正式描述为量子力学中的\textbf{波函数}。注意,波函数是理论中的不可观测量,即并非物理实在。


\subsubsection{量子力学中的波函数}

经典力学中可以用位置关于时间的函数 $x(t)$ 完整描述一个质量为 $m$ 的粒子的直线运动过程, 它在某个瞬间的状态则由位置 $x$ 和动量 $p$ 唯一决定。 在经典力学中, 我们假设粒子的状态本身就是可以直接测量的, 且测量不会改变粒子的状态。

在量子力学中一个粒子的状态完全由波函数描述。 波函数是一个关于位置和时间的函数, 一般记为 $\Psi(x, t)$, 也就是说在某个 $t_0$ 时刻, 波函数只是关于 $x$ 的一元函数 $\Psi(x, t_0)$。 量子力学和经典力学一个重要的不同在于, 量子力学中需要区分粒子的状态和粒子的测量。 一个粒子的波函数不能被一次性完整测量, 我们一次只能测量某个物理量如位置或动量, 且测量会对波函数造成不可逆转的改变。 我们下面先来简单介绍波函数以及它如何随时间变化, 然后再详细介绍如何对波函数测量某个物理量。

严格来说, 波函数的值是复数,但是为了生动起见,我们不妨暂时认为它是实数, 或者说只考虑它的实部。 我们可以想象做直线运动的粒子的波函数是一条紧绷的橡皮绳(或弦)上的波动, 绳子的高度 $y$ 是关于位置 $x$ 和时间 $t$ 的函数。 注意波函数和绳子的运动方式(波动方程)相似但不完全相同。

量子力学用波函数表示粒子的状态是因为许多实验发现微观粒子(如电子,质子,中子)在运动时具有一些经典物理中波的性质(例如声波,水波,光波), 这就是著名的\textbf{波粒二象性}。

\begin{figure}[ht]
\centering
\includegraphics[width=12cm]{./figures/5d40369d1d066173.pdf}
\caption{蓝线是一维波函数/绳子的形状,橙线是振幅。某个位置的振幅越大,粒子就越有可能出现在该位置(与振幅平方成正比)。图中的形状叫做波包,一般来说波函数可以是几乎任何形状。} \label{fig_QM0_1}
\end{figure}




\subsubsection{测量位置}
和质点相比较, 波函数的一个显著特点就是具有一定的空间分布。 但如果我们在某时刻测量该粒子的位置, 理论上仍然会得到一个确定的坐标点。 那么这点在哪里呢? 答案是: 测得的位置是随机的! 只要波函数振幅不为零的地方都可能测到该粒子。 这种随机性是根植于量子力学的原理中的, 而不是因为技术不够而测不准。

虽然测得的位置是随机的, 但却可以根据测量时刻的波函数得到位置的\textbf{概率分布}。 简单来说, 在测量的时刻, 波函数振动幅度越大的地方, 粒子越有可能出现。 具体来说, 粒子在某个位置出现的概率正比于测量时刻波函数在该位置振动的幅度的平方\footnote{量子力学中的波函数的值可以是复数\upref{CplxNo}, 粒子出现在某点的概率是复数模长的平方。这里为了简单暂时使用实数,可以看做是复数的实部}, 这叫做波函数的\textbf{概率诠释}。 例如在\autoref{fig_QM0_1} 中, 橙色曲线代表振幅, 它越高的地方, 粒子被测量到的概率越大。

\subsubsection{波包}
即使粒子的位置一般不能精确确定,我们往往也能确定其大致范围(例如实验物理学家往往知道某粒子在实验室而不在月球),所以波函数往往只有在一定范围内不为零, 即我们只能在一定空间范围内测到检测到粒子。将振幅关于位置的函数画出来,其形状往往像一个包(越靠近中间,检测到粒子的概率越大),我们称为\textbf{波包}(\autoref{fig_QM0_1})。 物理上实际存在的波函数都是波包, 但波包的长度和形状可以非常不同。 需要强调,波包不是粒子本身,而是对粒子状态的描述。 波包的大小并不是粒子本身的大小,而是给出了粒子大概的活动范围。

\subsection{波函数的演化}
\subsubsection{薛定谔方程}
经典力学中,给出粒子初始的位置和速度和势能,它将按照牛顿运动定律运动。 于是牛顿运动定律决定了粒子状态随时间演化的规律。 对应地, 在量子力学中, 给出粒子的初始波函数, 波函数将按照\textbf{薛定谔方程}\upref{TDSE}演化。 把势能和某个时刻的波函数代入薛定谔方程,就可以解出任意时刻的波函数。 注意薛定谔方程并不是数的方程,而是函数的方程(偏微分方程),所以解出的是波函数关于时间的变化而不是一个数关于时间的变化。 薛定谔方程是量子力学的公设\upref{MecThe}之一,就像牛顿运动定律是牛顿力学的公设。

\subsubsection{自由粒子}

\begin{figure}[ht]
\centering
\includegraphics[width=6cm]{./figures/f38e233554fdd7c2.pdf}
\caption{对应自由粒子的波包。注意图中波包仅画出了\autoref{fig_QM0_1} 中的振幅, 没画出波函数。 一个典型的例子是高斯波包(动画见\href{https://wuli.wiki/apps/free_gauss.html}{这里}, 互动见\href{https://wuli.wiki/apps/gausWP.html}{这里})} \label{fig_QM0_2}
\end{figure}

先看不受力的粒子(叫自由粒子),在经典力学中,粒子不受力时静止或者做匀速运动。量子力学中,若粒子不受力(势能处处为常数)且初始时波函数为波包且具有初速度,波包中心就会匀速运动,但同时波包会扩散(越变越宽,越变越矮)。这就好比一根紧绷的绳子的一端突然迅速上下抖了几下又停下来,那么将产生一个波包将从绳的一段传到另一端。 如果波包开始是静止,那么波包只会在原地扩散\footnote{小时百科的图标就是自由高斯波包, 详见 “小时百科图标\upref{xwLogo}”}。

在一般的量子力学问题中, 波包的长度非常小(原子尺度)。 当我们回到宏观尺度中看它, 波包的大小就可以忽略不计, 测量其位置的不确定性也可以忽略不计, 从而把波包近似成一个固定位置的质点, 波包的整体速度就是质点的速度, 这样就得到了匀速运动的宏观质点。 可见量子力学和经典力学是不矛盾的, 只是描述的尺度不同。

\subsubsection{散射}
% 新词条: 一维势能
在经典力学\upref{CM0}的科普中我们已经讨论了一些势能曲线, 以下通过几个例子来定性说明在一些特定势能曲线下量子力学中的粒子如何运动(即非自由粒子)。

\begin{figure}[ht]
\centering
\includegraphics[width=6cm]{./figures/be3eaeefc1de8f3b.pdf}
\caption{波包遇到三角势垒后的散射。} \label{fig_QM0_3} % 未完成: 图中改用方势垒
\end{figure}

经典力学中,若粒子匀速入射到如\autoref{fig_QM0_3} 所示的势能 $V(x)$(势垒)上, 如果粒子的初始动能小于势垒的最大值,则粒子将会原路返回,若粒子初始动能大于势垒的最大值,则粒子将继续前进。 而在量子力学中,无论波包以什么初速度入射, 波包都会被分为两个不同方向运动的波包, 但两个波包的相对大小取决于初始波包的形状和动能。 定性来说, 入射波包的能量越大, 粒子就越有可能穿过势垒, 所以向右运动的波包就相对越大。 反之, 如果入射波包的能量越小, 粒子就越有可能被反弹, 所以向左运动的波包就越大。

这个问题中最不可思议的地方在于当粒子的初始动能小于势垒的最大值时, 它仍会有一定的概率穿过势垒, 这在经典力学中时不可能的。 顾名思义, 我们把这种量子力学中特有的现象叫做\textbf{隧道效应(tunneling effect)}。 隧道效应带给我们的一个显而易见的疑惑是: 如果我们在势垒内部检测到粒子, 那它的动能岂不是小于零? 如果再求速度或动量(需要开方), 岂不是会得到虚数? 毫无疑问, 虚数是没有物理意义的, 但思考一下可以发现在这个问题隐藏了一个假设就是我们可以同时准确地测量粒子的位置和动能, 也就是同时准确测量位置和动量。 而根据下面将要介绍的不确定性原理, 这是不可能的。

定性来说, 粒子进入或穿过势垒的概率与两个因素有关: 势垒的高度和宽度。 势垒比动能大得越多, 概率就越小, 势垒越宽, 概率也越小(波函数的振幅呈指数下降)。

\subsubsection{有限深势阱}
经典力学中的有限深势阱见 “动量和能量\upref{CM1}”。 在量子力学中, 被困在势阱中的波包在有限深势阱中同样一边来回碰撞一边扩散。 但由于隧道效应, 波函数会稍微渗透到势阱的外部

对经典的粒子(经典力学中的粒子)来说, 当粒子被困在势阱内时, 两种势阱带来的效果是一样的。

\addTODO{无限深势阱并没有提到!一维光电效应也可以讲一下,放动画}

\subsection{波的叠加}
为什么要用波函数来描述粒子?因为波的一个重要特点就是具有空间分布且可以叠加。 如果一个波包 $\Psi_1(x, t)$ 是薛定谔方程的解,且另一个波包 $\Psi_2(x, t)$ 也是薛定谔方程的解,那这两个波包叠加(即把两个波函数相加)后 $\Psi(x, t) = \Psi_1(x, t) + \Psi_2(x, t)$ 仍然是薛定谔方程的解。例如, 两个不同方向运动的波包相遇然后穿过对方。注意这里两个波包并不代表两个粒子,而是代表一个粒子有可能在一个波包的位置被观测到也有可能在另一个波包的位置被观测到。
\addTODO{这里可以放一个动画,两个波包相遇然后离开,产生干涉条纹}

一个关于波的可叠加性的经典实验就是粒子的双缝干涉, 当两个点状波源各自发出相同频率的圆形波时, 在远处放一个观测屏, 就可以观测到干涉条纹(见\autoref{fig_QM0_6} 和\autoref{fig_QM0_7})。 这类干涉现象在量子物理被发现很久以前就已经有了详细的研究。 然而当物质波的概念提出后, 物理学家竟然也观测到了微观粒子产生的干涉(如电子的双缝干涉), 进而产生了类似波函数和波动方程(即薛定谔方程)这类的数学工具。

% 未完成: 这两张图来自搜索引擎, 可能存在版权问题
\begin{figure}[ht]
\centering
\includegraphics[width=7cm]{./figures/2f217ff8544288fd.png}
\caption{水波的干涉} \label{fig_QM0_6}
\end{figure}

\begin{figure}[ht]
\centering
\includegraphics[width=7cm]{./figures/ca7aafac21174c66.png}
\caption{光波的干涉} \label{fig_QM0_7}
\end{figure}

% 量子力学(测量理论)

\subsection{测量的另一个例子:能量}
作为可以测量的物理量的第二个例子, 我们现在来讨论粒子总能量的测量。 在经典力学中,要测量一个粒子的能量,我们只需要先观测其速度,则能量等于动能($mv^2/2$)加上该位置的势能\upref{CM1}。量子力学中,测量波包的能量会得到什么结果呢?这时候我们需要薛定谔方程的另一个用途:对某个势能函数,我们能解出一些(往往是无穷多个)特殊的波函数,称为能量的\textbf{本征函数}或者\textbf{本征态}。 这些波函数特殊在哪里呢? 一般来说, 我们对一个粒子测某个物理量(位置, 动量, 能量), 测得的值都是不确定的(这与仪器的误差无关, 而是原理上的不确定)。 但如果粒子的波函数恰好是一个能量本征态, 那么测量其能量会得到唯一确定的值(先不讨论用什么方法测)叫能量的\textbf{本征值}。 每个本征态对应唯一一个本征值。

物理学家最早探索的能量本征态是氢原子中电子的波函数的本征态。 氢原子一般由一个质子和一个电子组成, 是最简单的原子。 由于质子的质量远大于电子,我们一般近似认为它不动,所以该问题可以看作是一个单粒子在三维空间中运动的问题。 氢原子能量最低的能量本征态叫做\textbf{基态}, 约等于 $-13.6\Si{eV}$。 也就是说如果该电子的波函数是基态波函数, 那么测量能量时结果就肯定是 $-13.6\Si{eV}$, 不会有任何不确定的因素。 同理, 更高能量的本征态叫做激发态。 若电子的波函数是第 $n$ 个激发态, 那么测得的能量就是该激发态的能量, 等于 $-13.6\Si{eV}/n^2$。

但是我们上面说波函数可以是任意的, 那么如果一个波函数不处于能量的任何本征态, 能量的测量结果会如何呢? 要测量任意波包(波函数)的能量,我们可以想办法用不同的能量本征态叠加来凑出所需波函数, 例如我们可以把第一个本征态 $\Psi_1$ 乘以常数 $C_1$ 加上第二个本征态 $\Psi_2$ 乘以常数 $C_2$, 等等, 即\textbf{线性组合}
\begin{equation}
\Psi = C_1\Psi_1 + C_2\Psi_2 + \dots~
\end{equation}
就得到了一个不处于任何能量本征态的波函数, 可以把它称为(本征波函数的)\textbf{叠加态}。 量子力学告诉我们: 如果对这个叠加态的波包测量能量, 那么测到第 $n$ 个能量本征值的概率等于\footnote{上文提到波函数一般是复数,这里的系数事实上也同样是复数,而这里的绝对值符号代表复数的模长} $\abs{C_n}^2$。 也就是说,某个能量本征态占总波函数的比例越多,越有可能测出对应的能量本征值。

由于氢原子电子的三维波函数较为复杂, 我们还是先看一维的例子。

\begin{example}{无限深势阱}
\begin{figure}[ht]
\centering
\includegraphics[width=4cm]{./figures/d6776d51b077d721.pdf}
\caption{无限深势阱} \label{fig_QM0_10}
\end{figure}
如\autoref{fig_QM0_10} ,若粒子(质量为 $m$ )处于这样的势能当中:除了两个端点 $(x=0,x=a)$ 外势能为0,而在端点处势能为无穷。这种情形便是无限深势阱。 可以解得粒子的能量本征态在势阱内是一些正弦函数(\autoref{fig_QM0_4} )。 第 $n$ 个本征态对应的能量本征值为 $E_n=\frac{\pi^2\hbar^2}{2ma^2}n^2$。

\begin{figure}[ht]
\centering
\includegraphics[width=7.5cm]{./figures/197f3970f51a81a9.pdf}
\caption{无限深势阱的一些波函数能量本征态,每个能量 $E_i$ 叫做一个能级,共有无穷多个能级,将这些本征态叠加就能获得任何波函数例如波包。} \label{fig_QM0_4}
\end{figure}

在势阱中, 粒子任意的波函数 $\psi(x)$ 都可以表示为本征函数叠加态的线性组合:
\begin{equation}
\psi(x) = C_1\psi_1(x) + C_2\psi_2(x) + \dots~
\end{equation}
若对这个叠加态测量能量,那么测得 $E_i$ 的概率为 $|C_i|^2$。
\end{example}


\subsection{一般测量理论}
我们上面已经涉及了两个物理量的测量:位置和能量。 现在我们来讨论总结如何测量量子力学中任意一个可观测的物理量。 我们将要介绍量子力学的另一组重要公设, 这里将其统称为\textbf{测量理论}。 注意这里讨论的\textbf{测量}是一个抽象的概念, 我们先不讨论用什么仪器进行测量,也不讨论现实中这种测量可不可行, 只是假设存在 “测量某个物理量” 这种操作。

上面介绍的能量测量的步骤就是测量理论的一个特例。 量子力学中的任何物理量的测量都可以由类似的方法求出, 只是不同的物理量都有各自的一组本征态, 可以通过物理量对应的\textbf{本征方程}求出。 以下我们先不讨论如何列出本征方程以及解出本征态, 而是直接给出本征态。 如果要对一个粒子测量某物理量, 就先计算出这个物理量的各个本征态, 再计算它们占波包的比例 $C_i$, 测量到第 $i$ 个本征值的概率就是 $\abs{C_i}^2$。

这里有一个前提条件是任何可能真实存在的波包都可以由任何测量量的本征态叠加而成, 这叫做本征态的\textbf{完备性}。

\begin{example}{位置和动量的本征态}
位置的本征态是一个无限窄无限高的波包, 称为 $\delta$ 函数\footnote{如果你想了解更多,详见“狄拉克 delta 函数\upref{Delta}”。}。 $\delta$ 函数描述的粒子只可能出现在某点。 
\begin{figure}[ht]
\centering
\includegraphics[width=9cm]{./figures/e0430a149ae0c451.pdf}
\caption{$\delta$ 函数示意图。 注意其宽度为无穷窄。 $\delta(x)$ 的中心在坐标原点, 如果需要任意 $x_0$ 处的 $\delta$ 函数, 可以使用 $\delta(x - x_0)$。} \label{fig_QM0_9} % 未完成, 标记原点, 解释无限窄
\end{figure}

动量的本征态是一个无限长的波, 即某个频率的 $\sin$ 函数\footnote{事实上是 $\exp(\I kx)$, 但为了简单就不那么严谨了。}, 称为\textbf{平面简谐波}(区别于波包), 见\autoref{fig_QM0_8}。 平面简谐波简称\textbf{平面波}, 这是量子力学中习惯的叫法, 一维的情况下和 “平面” 没什么关系。 在量子力学中, 波也叫\textbf{德布罗意波}。

\begin{figure}[ht]
\centering
\includegraphics[width=9cm]{./figures/a4e990f8acd71612.pdf}
\caption{一维平面波 $\sin(kx)$, 由负无穷延伸至正无穷。} \label{fig_QM0_8}
\end{figure}

% 是不是把这个放到力学里面介绍?
注意图中画出的是波动关于位置 $x$ 的函数, 而不是关于时间 $t$ 的函数, 所以我们把 $k$ 叫做\textbf{空间频率}\footnote{严格来说应该是\textbf{角频率} 而不是频率}(或者\textbf{波数}), % 未完成: exam 环境中没有粗体的问题真的应该解决一下了!
与波长\footnote{也叫\textbf{德布罗意波长}}(即空间周期) $\lambda$ 的关系是 $k = 2\pi/\lambda$。 动量的本征值为
\begin{equation}\label{eq_QM0_1}
p = \hbar k = \frac{h}{\lambda}~.
\end{equation}
其中 $\hbar = h / (2\pi)$ 是\textbf{约化普朗克常数}, $h$ 是\textbf{普朗克常数}。 所以和空间频率成正比, 和波长成反比。
\end{example}

\subsubsection{测量改变粒子状态}
在经典力学中, 我们完全可以假设测量不会对粒子的状态产生影响(例如给粒子拍照不会改变它的运动状态)。 而在量子力学中, 由于考虑的粒子质量非常小, 测量会不可避免地改变粒子的状态。

测量理论假设, 在对某个物理量进行测量后, 如果得到第 $i$ 个本征值, 粒子的状态会立即\textbf{坍缩}为这个本征值对应的本征态, 然后继续按照薛定谔方程演化。

那么我们应该如何理解测到某个本征值的概率 $\abs{C_i}^2$ 呢? 这里显然不是指对处于某状态的粒子连续测量 $N$ 次, 就大概有 $\abs{C_i}^2 N$ 次会测到本征值。 因为由于测量带来的坍缩, 只有第一次测量的概率是 $\abs{C_i}^2$ , 之后由于波函数会坍缩到第 $i$ 个本征态并继续按薛定谔方程继续演化, 再次测量时, 我们就要根据新的波函数重新计算每个 $C_i$, 而不是使用第一次测量前波函数。

正确的理解应该是, 如果我们有 $N$ 个处于同样状态(具有相同波函数)的相同粒子, 对这些粒子进行测量, 大概会有 $\abs{C_i}^2 N$ 个次测量得到第 $i$ 个本征值。 或者是, 我们把同一个粒子测量 $N$ 次, 但每次测量后都进行某种操作, 将其恢复到测量前的状态, 再进行下一次测量。

这就意味着, 除非粒子在测量前就已经处于该本征态, 测量操作必定会改变粒子的状态。 例如粒子的波函数如为\autoref{fig_QM0_1} 中的波包, 当我们测量其位置, 如果得到的值为 $x_0$, 那么波函数将会马上变为 $x_0$ 处的一个无限窄的波包($\delta$ 函数)。

\subsubsection{连续本征值与离散本征值}
上面我们提到一个物理量往往有无穷多个本征波函数。一些情况下我们求出的本征波函数对应的值是不连续的(如无限深势阱中的能量本征态),而另一些情况下则是连续的(如位置和动量的本征值总是连续的, 即它们可以取任意实数)。 经典物理世界中这些物理量都是连续的, 而量子中的物理量会出现离散的情况, 这就是“\textbf{量子}”这个词的来源:表示一份一份的,不连续的。

离散本征值的情况在数学上是容易处理的, 我们上面使用的符号 $C_i$ 就是假设了本征值是离散的(因为这里 $i$ 是整数), 连续的情况有类似的理解, 但数学形式有所不同(比如求和变为积分)。 我们再举一个可观测量的例子:

\begin{example}{角动量}\label{ex_QM0_2}
量子力学中, 另一个著名的具有离散本征值的物理量就是角动量。 % 引用经典力学科普
例如一个电子围绕固定的原子核旋转, 在经典力学中它就可能具任意方向的角动量。 注意这是一个三维空间中的问题, 在直角坐标系中, 我们可以将角动量的三个分量看作三个不同的物理量分别为 $L_x$, $L_y$ 和 $L_z$。 为了简单起见我们假设这三个量各自只有两个本征值, 即\footnote{以后我们会知道这个例子描述的是自旋为 1/2 的粒子的自旋角动量而并不是电子轨道角动量,这里为了书写方便使用约化普朗克常数 $\hbar$ 作为角动量的单位。} $L_x = -1/2$ 或 $+1/2$, $L_y = -1/2$ 或 $+1/2$, $L_z = -1/2$ 或 $+1/2$。 对应的 6 个状态记为\footnote{这里我们使用 $\ket{\dots}$ 来表示波函数, 这种符号叫做\textbf{狄拉克符号}, 如果你不习惯, 完全可以把它们记为 $\Psi_{x+}(\dots)$, $\Psi_{x-}(\dots)$ 等形式。 事实上, 本例讨论的自旋态, 并不能用波函数表示, 但为了方便理解, 我们姑且假设它们可以。} $\ket{x-}$, $\ket{x+}$,  $\ket{y-}$, $\ket{y+}$,  $\ket{z-}$, $\ket{z+}$。

为了简单起见, 我们下面只讨论 $x$ 和 $z$ 两个方向。 $L_x$ 本征态与 $L_z$ 本征态之间的关系为
\begin{equation}\label{eq_QM0_2}
\leftgroup{
&\ket{x+} = \frac{1}{\sqrt{2}} \ket{z+} + \frac{1}{\sqrt{2}} \ket{z-}\\
&\ket{x-} = \frac{1}{\sqrt{2}} \ket{z+} - \frac{1}{\sqrt{2}} \ket{z-}~.
}\end{equation}

现在, 如果我们对处于 $\ket{x+}$ 态的粒子测量 $x$ 方向的角动量, 根据测量理论中对本征态的定义, 得到的只可能是其对应的本征值 $L_x = 1/2$。 同理, 如果对 $\ket{z-}$ 态的粒子测量 $z$ 方向的角动量, 只可能得到 $L_z = -1/2$。 

但如果我们对 $\ket{x+}$ 态的粒子测量 $z$ 方向的角动量会发生什么呢? 根据测量理论, 我们首先需要将 $\ket{x+}$ 表示成 $\ket{z+}$ 和 $\ket{z-}$ 的线性组合, 而\autoref{eq_QM0_2} 中的第一条已经给出了这个关系。 于是我们马上得到 $C_1 = 1/\sqrt{2}$, $C_2 = -1/\sqrt{2}$。 所以测得 $L_z = -1/2$ 和 $L_z = 1/2$ 的概率分别是 $\abs{C_1}^2 = 1/2$, $\abs{C_2}^2 = 1/2$, 即二者概率相同。

根据同样的过程, 我们也从\autoref{eq_QM0_2} 中的第二条得出, 若粒子处于 $\ket{x-}$ 态, 测量 $z$ 方向角动量得到 $L_z = 1/2$ 和 $L_z = -1/2$ 的概率同样也都是 $1/2$。
\end{example}

\subsection{薛定谔的猫}
注意我们在测量理论中并没有指明什么样的事件会构成测量。 还是用最开始波包的例子, 假设一个粒子的波函数是两个反方向运动的波包, 且两个波包的形状相同。 我们在其中一个方向放一个屏幕用于测量位置, 检测到粒子后, 有且只有一个屏幕会发光。 那么根据测量理论, 这个粒子有 $1/2$ 的概率会落到屏幕上(朝着屏幕方向运动的波包)。 按照通常的理解, 这似乎就对粒子的位置构成了测量。

但是这并不严谨, 因为屏幕也是由微观粒子(质子,电子等)构成的, 也可以由波函数描述。 如果将屏幕与粒子共同作为考察的系统, 我们可以用一个多粒子的波函数来描述这整个系统(多粒子的波函数这里作不介绍)。 当一个波包传播到屏幕的位置时, 波函数描述下的屏幕的也变为包含发光的本征态与不发光的本征态的叠加, 正如粒子的初始状态是向左传播的波函数和向右传播的波函数的叠加。 

如果假设“人眼看屏幕” 这个操作对屏幕的波函数进行了测量, 那么只有当实验者去看屏幕的瞬间, 波函数才会坍缩到屏幕发光或不发光的本征态。

这个论述可以一直进行下去, 人眼也是由微观粒子构成的, 那么如果也纳入考察的系统,共同用波函数描述, 那人眼的状态就会是“看到发光屏幕的人眼”和“没有看到发光屏幕” 的人眼这两种本征态的叠加, 直到人脑对人眼进行测量后才会得到确定的结果。 实验者最终也会处于“看到发光屏幕的人”和“没看到发光屏幕的人” 这两种本征态的叠加态。

\textbf{薛定谔的猫}就是这类佯谬中的一个具体例子。 在这个假想实验中, 一瓶放射性物质旁边放有一个粒子检测器(盖革计数器), 当检测器检测到放射性出的粒子时, 就会触发一个释放毒药的开关, 将箱子里的猫毒死。 由于放射出的粒子是由波函数描述的, 对这个实验的理解取决于这一系列过程中的哪一步构成测量。 如果我们认为检测器对放射粒子构成了测量, 那么从测量到结果的那一刻, 一切都是确定的, 也就不存在佯谬。 但如果我们认为只有当人打开箱子看到猫的那一刻才构成测量, 那么在打开箱子前猫都处于死和活两个本征态的叠加态。 更不可思议的理解是, 根本不存在测量, 当人打开箱子检查后, 人也成了“看到活猫的人” 和“看到死猫的人” 的叠加态。

关于什么构成测量这个问题至今仍然没有公认的解释, 但在现实中研究者往往假设测量仪器对粒子构成了测量\footnote{这一方面也是由于多粒子波函数的计算极其复杂, 目前我们还无法精确计算多于两个粒子的波函数的薛定谔方程, 即使是用超级计算机做数值解。 量子力学中所有涉及多粒子波函数的理论几乎都是用了不同程度的近似。}, 且得到了一些实验结果的高度吻合。

\subsection{不确定性原理}

\begin{figure}[ht]
\centering
\includegraphics[width=14cm]{./figures/e0ac7772fcd6f319.pdf}
\caption{波包越短,频率就变得越模糊而位置变得更精确,波包越长,频率就变得更精确而位置越模糊。} \label{fig_QM0_5}
\end{figure}

首先注意\textbf{不确定原理}并不是量子力学的基本假设, 而是从测量理论推导出的一些结论。 不确定性原理有多个, 最常见的一个不确定性原理说的是, 我们\textbf{无法既准确测量粒子的位置又准确测量粒子的动量}。 还有一种常见的不确定性原理指的是\textbf{无法既准确测量能量又准确测量时间}。 这里只讨论前者。

对不确定原理的一个理解是, 假设我们准备许多处于相同状态(波函数相同)的粒子, 将其分为两组, 对第一组的每个粒子测量位置, 对第二组测量动量, 然后计算所有测得位置的不确定度(例如标准差)$\Delta x$, 以及所有测得动量的不确定度 $\Delta p$。 此时会发现, 无论初始时粒子处于什么状态, 二者的乘积 $\Delta x \Delta p$ 总会大于一个常数。
\begin{equation}
\Delta x \Delta p \geqslant \frac{\hbar}{2} \approx 5.3\e{-35} \Si{Js}~.
\end{equation}

上文提到位置的本征态是一个 $\delta$ 函数,对应本征值为该波包的位置(坐标)。 动量的本征态恰恰相反,是平面波。 动量的本征值为平面波的空间频率乘以一个常数。 如果粒子的波包很窄(但不是无限窄),那么要测量位置,我们只需要使用波包中心附近的一些 $\delta$ 函数就可以叠加出待测的波包,所以测量结果也只可能离波包中心较近。 但在测量动量时,由于这个波包和平面波一点都不像,我们需要许多不同频率的平面波叠加才能得到这个波包(为什么无限长的平面波相加能得到有限长的波包?这是数学上一个非常有趣的结果,叫做傅里叶变换)。

另一种情况是,如果波包很长但不是无限长(想象绳子的一段连续上下抖了许多下才停下来),那么将波包看起来与某个频率的平面波十分相似。 在测量位置时,我们需要将许多不同坐标的位置本征函数叠加得到波包,那测得的坐标就可能在很大的范围内出现。 而测量动量时,我们只需要使用某个频率附近的一些平面波,所以测得的动量(与频率成正比)也都很接近中心频率对应的动量。

对不确定性原理的另一种理解来自测量理论中的波函数坍缩。 假设我们先对处于某个状态的粒子准确地测量位置, 那么根据测量理论, 无论测到什么值, 测量完后波函数都会坍缩为该位置的本征态, 即 $\delta$ 函数。 如果这时再马上测量动量, 我们测到的就不是粒子原来状态的动量了, 所以即使测到一个精确的动量值也并没有意义。 反之如果我们先精确地测量某状态下的粒子的动量, 那么测量完后波函数将会坍缩为某空间频率的平面波, 再马上测量其位置, 得到的值也同样没有意义了。

事实上对不确定性原理的这种理解对任何两个本征函数不相同的物理量都适用, 因为当测量完第一个物理量后, 波函数发生坍缩, 再测量第二个物理量就没有意义了。 但幸运的是有一些物理量有一组相同的本征函数(例如动量和动能的本征函数都是平面波), 这样我们只需要测量其中一个, 无需再次测量就可以直接计算出另一个物理量的值。
