% 浙江大学 2011 年 考研 量子力学
% license Usr
% type Note

\textbf{声明}:“该内容来源于网络公开资料,不保证真实性,如有侵权请联系管理员”

\subsection{第一题(42 分)简答题:}\\
(1)什么是光电效应?什么是受激辐射?\\
(2)什么是斯达克效应?什么是塞曼效应?\\
(3)一维问题的能级的简并度最大是多少?\\
(4)写出测不准关系,并简要说明其物理含义。\\
(5)写出非简并微扰论的能级修正公式(到二阶)。\\
(6)放射性指的是束缚在某些原子核中的更小粒子有一定的概率逃逸出来,你
认为这与什么量子效应有关?\\
(7)由正则对易关系 $[\hat{x}, \hat{p}] = i\hbar$ 导出角动量的三个分量
\[
L_x = y \frac{\partial}{\partial z} - z \frac{\partial}{\partial y}, \quad 
L_y = z \frac{\partial}{\partial x} - x \frac{\partial}{\partial z}, \quad 
L_z = x \frac{\partial}{\partial y} - y \frac{\partial}{\partial x}~
\]
的对易关系。

\subsection{第二题(28 分):}
有一个质量为 $m$ 的粒子处在如下势阱中(这里 $V_0 > 0$):

\[
V(x) =
\begin{cases}
\infty, & x < 0 \\
-V_0, & 0 < x < a \\
0, & x > a
\end{cases}~
\]

\begin{enumerate}
    \item  求其能级,求其波函数。
    \item  确定至少存在一个束缚态的条件。
\end{enumerate}
\subsection{第三题(20分):}

\subsection{第四题(20分):}

\subsection{第五题(20分):}
