% 交错代数(综述)
% license CCBYSA3
% type Wiki

本文根据 CC-BY-SA 协议转载翻译自维基百科\href{https://en.wikipedia.org/wiki/Alternative_algebra}{相关文章}。

在抽象代数中,交替代数是一类代数,其乘法运算不必满足结合律,只需满足交替性。也就是说,对于代数中的任意元素 $x$ 和 $y$,必须有:
\begin{itemize}
\item $x(xy) = (xx)y$
\item $(yx)x = y(xx)$
\end{itemize}
显然,所有结合代数都是交替代数,但也存在一些严格非结合的代数(如八元数)同样满足交替性。
\subsection{结合子}
交替代数之所以得名,是因为它们的结合子是交替的。结合子是一个三线性映射,定义为:
$$
[x, y, z] = (xy)z - x(yz)~
$$
按定义,如果一个多线性映射在其任意两个自变量相等时取零,则称其为交替的。对于一个代数,左交替恒等式与右交替恒等式等价于\(^\text{[1]}\):
$$
[x, x, y] = 0~
$$
$$
[y, x, x] = 0~
$$
将这两个恒等式结合起来,可以推出:
$$
[x, y, x] = [x, x, x] + [x, y, x] - [x, x+y, x+y]~
$$
$$
= [x, x+y, -y]~
$$
$$
= [x, x, -y] - [x, y, y] = 0~
$$
对于所有的 $x$ 和 $y$ 都成立。这等价于柔性恒等式\(^\text{[2]}\):
$$
(xy)x = x(yx)~
$$
因此,交替代数的结合子是交替的。反过来,任何结合子是交替的代数显然也是交替代数。由于对称性,任何满足以下三个恒等式中任意两个的代数:
\begin{itemize}
\item 左交替恒等式:$x(xy) = (xx)y$
\item 右交替恒等式:$(yx)x = y(xx)$
\item 柔性恒等式:$(xy)x = x(yx)$
\end{itemize}
都是交替代数,并且因此满足这三个恒等式全部成立。

一个交替的结合子总是完全反对称的,即:
$$
[x_{\sigma(1)}, x_{\sigma(2)}, x_{\sigma(3)}] = \operatorname{sgn}(\sigma) [x_1, x_2, x_3]~
$$
对于任意排列 $\sigma$ 都成立。只要基域的特征不为 2,反之亦然。
\subsection{例子}
\begin{itemize}
\item 每一个结合代数都是交替代数。
\item 八元数构成一个非结合的交替代数,它是实数域上的 8 维赋范除代数\(^\text{[3]}\)。
\item 更一般地,任意八元数代数都是交替代数。
\end{itemize}
\subsubsection{非例子}
\begin{itemize}
\item 十六元数、三十二元数以及所有更高阶的凯莱–迪克森代数都失去了交替性。
\end{itemize}