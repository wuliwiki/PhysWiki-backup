% 点电荷系统的静电场 (MATLAB绘图)


\pentry{电势、电势能\upref{QEng}}
\begin{figure}[ht]
\centering
\includegraphics[width=10cm]{./figures/d3cbf69dab372ca8.pdf}
\caption{随机生成的点电荷电场\upref{Efield}} \label{fig_MLTEPD_1}
\end{figure}
\begin{lstlisting}[language=matlab]


本代码基于CC-BY发布

clc
clear

[x y z]=meshgrid(-10:0.5:10,0,-10:0.5:10);
[sx sy sz] = size(x);

N=2; %设定电荷个数

chargeLocation=zeros(3,N);
q = zeros(N,1);

for n = 1:N %随机生成电荷的位置与电量
  chargeLocation(:,n) = 20*rand(3,1) - 10;
  chargeLocation(2,n) = 0;
  q(n)=4*rand() - 2;
end

%你也可以手动指定各电荷的位置与电量
%chargeLocation(:,1) = [0;0;0]; %x;y;z
%chargeLocation(:,2) = [0;0;5];
%q(1)=1;
%q(2)=-1;

for i=1:sx
 for j=1:sy
   for k=1:sz
      r = [x(i,j,k);y(i,j,k);z(i,j,k)];

      E = [0;0;0];

      for n = 1:N
        R = r-chargeLocation(:,n);
        E = E + 30*q(n)/norm(R)^3 * R;
      end

      if isnan(norm(E))
        E=[0;0;0];
      elseif norm(E) >= 0.5;
        E = E/norm(E) * 0.5;
      end

      u(i,j,k)=E(1);
      v(i,j,k)=E(2);
      w(i,j,k)=E(3);
   end
 end
end

hold on
axis equal
axis([-10 10 -10 10 -10 10])

quiver3(x,y,z,u,v,w,0);

for n=1:N
  %正电荷记为红色,负电荷记为黑色,大小与电荷量有关。
  s = scatter3(chargeLocation(1,n),
  chargeLocation(2,n),chargeLocation(3,n),'r');
  if q(n) > 0 
    set(s,'MarkerEdgeColor','r')
  else
    set(s,'MarkerEdgeColor','k')
  end
  set(s,'sizedata',36*log(abs(q(n))+1));
end

xlabel('X','fontsize',15)
ylabel('Y','fontsize',15)
zlabel('Z','fontsize',15)
view(0,0)
hold off

U=0; 
%计算该点电荷系统的相互作用能。
%由于没有正确地选择常数,因此结果只有定性的演示意义。
for i=1:N
  for j = i+1:N
    r1 = chargeLocation(:,i);
    r2 = chargeLocation(:,j);
    phi = 30*q(j)/norm(r1-r2);
    U = U + q(i)*phi;
  end
end
U
\end{lstlisting}
