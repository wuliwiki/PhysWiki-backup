% 洛朗级数
% 复变函数|留数|residue|全纯函数|级数

\pentry{全纯函数,柯西积分定理\upref{CauGou}}

如果一个单复变函数处处解析,那它处处都可以展开成泰勒级数的形式.但我们常遇到的很多函数并不是处处解析的,从而在非解析点处无法泰勒展开.但是我们依然有一种类似级数的办法来展开其中一些函数,这就是\textbf{洛朗级数}.

\begin{definition}{洛朗级数}
形如
\begin{equation}
f(z)=\sum\limits_{n=-\infty}^{\infty} a_n(z-c)^n
\end{equation}
的单复变函数,被称为一个关于点$c\in\mathbb{C}$的\textbf{洛朗级数(Laurent series)},其中$a_n$都是常数.


\end{definition}

容易看到,洛朗级数就是将泰勒级数的幂次拓展到了负的情况.在$z\neq c$的点处,$f(z)$可以看成两个级数$\sum\limits_{n=0}^{\infty} a_n(z-c)^n$和$\sum\limits_{n=1}^{\infty} a_{-n}\qty(\frac{1}{z-c})^n$的和;在$z=c$处则是发散的.

最简单的一个非泰勒级数的洛朗级数例子,大概就是$f(z)=1/z$.

\begin{example}{}
考虑$f(z)=1/z$的围道积分.

令积分路径为$\Gamma(t)=\rho\E^{\I t}$,其中$\rho$是一个常数.$\Gamma$就是一个绕着一个半径为$\rho$的圆的逆时针路径.$f$沿$\Gamma$的围道积分就是:
\begin{equation}
\begin{aligned}
\int_{\Gamma}\frac{1}{z}\dd z&=\int^{2\pi}_0\frac{1}{\rho\E^{\I t}}\cdot\frac{\dd\rho\E^{\I t}}{\dd t}\dd t\\
&=\int_0^{2\pi}\frac{1}{\rho\E^{\I t}}\cdot (\I \rho \E^{\I t})\dd t\\
&=\int_0^{2\pi}\I\dd t\\
&=2\pi\I
\end{aligned}
\end{equation}

对于任意逆时针环绕原点的路径$P(t)$,我们可以求$\int_{\Gamma}\frac{1}{z}\dd z-\int_{P}\frac{1}{z}\dd z$,其中$\Gamma$完全包裹在$P$中.这相当于求$\Gamma$和$P$之间区域的围道积分.这个中间区域是不包含原点的,因此由于$1/z$在原点之外处处解析,由\textbf{柯西积分定理}\upref{CauGou}知,中间区域的围道积分为$0$.从而我们可以推知,$\int_{P}\frac{1}{z}\dd z=2\pi\I$.

更一般地,对于任意环绕原点的路径$P$,有$\int_{P}\frac{1}{z}\dd z=\pm 2\pi\I$,其中$P$为逆时针时取正号,反之取负号.

\end{example}

















