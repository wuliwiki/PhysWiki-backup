% 高斯绝妙定理
% keys Theorema Egregium|高斯曲率|Gaussian curvature|微分几何

\pentry{黎曼联络\upref{RieCon}}

本节我们来讨论高斯绝妙定理.

在最初的古典微分几何研究中,常常需要将流形理解为某个$\mathbb{R}^n$空间中的超平面,进行具体的、复杂的计算,从而得到其性质.比如说,很多二维的流形都可以表示为三维空间中的一个曲面,比如球、平面、双曲面等等;也有的二维流形没法在三维空间中表示,比如Klein瓶,但是在四维空间中一定可以表示的\footnote{题外话:任意$n$维实流形,都可以嵌入到$\mathbb{R}^{2n}$中.}.

高斯第一个发现“曲率”这一\textbf{内蕴}量,并把该发现命名为\textbf{绝妙定理(Gauss's Theorema Egregium)}\footnote{注意“绝妙定理(Theorema Egregium)”是拉丁语.}.连高斯都觉得绝妙的发现到底是什么呢?这就需要解释何为“内蕴”了:它是指,曲率的计算不依赖于具体的嵌入、图等具体表示,而是流形本身具有的性质.这就逐渐进入了现代微分几何更为抽象但优雅的大门了.

当然,我们这里不会使用高斯一开始的语言去描述该定理,而是用更为现代的语言.毕竟我们是在学习数学,而非数学史,站在巨人的肩膀上当然是更合适的.

\subsection{一点预备}

我们首先需要介绍两个很有用的公式.

\begin{theorem}{}
设$(M, \nabla)$是一个$\mathbb{R}^3$中的子黎曼流形,且$X, Y, Z\in\mathfrak{X}(M)$.则我们有以下定理:
\begin{enumerate}
\item \textbf{(高斯曲率方程)}$R(X, Y)Z=\ev{L(X), Z}L(Y)-\ev{L(Y), Z}L(X)$;
\item $\nabla_XL(Y)-\nabla_YL(X)=L([X, Y])$
\end{enumerate}
\end{theorem}

以下分别证明这两个式子.

\subsubsection{第一个式子的证明}

形状算子$L$




\subsubsection{第二个式子的证明}






















