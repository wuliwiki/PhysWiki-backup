% 氢原子电离计算(微扰)
% 氢原子|偶极子|跃迁|微扰

\pentry{跃迁概率(含时微扰)\upref{HionCr}}

基态与平面波的积分

\begin{equation}
\bvec d_{\bvec k} =  \mel{\bvec k}{\bvec r}{0}
=  \frac{\uvec k}{\sqrt{2\pi}\sqrt{\pi}} \int_0^{+\infty} \int_0^\pi \E^{-r} \E^{-\I k r \cos\theta} r \cos\theta \cdot 2\pi r^2 \sin\theta \dd{\theta} \dd{r}
\end{equation}
换元, 令 $u = \cos\theta$, 得
\begin{equation}\ali{% 已检查多次
\bvec d_{\bvec k} &= \sqrt 2 \uvec k \int_0^{+\infty} r^3 \E^{-r} \int_{-1}^1 \E^{-\I k r u} u  \dd{u} \cdot \dd{r}\\
&=  \I\frac{2\sqrt2 \uvec k}{k}  \int_0^{+\infty} r^2 \E^{-r} \qty[\cos(kr) - \frac{1}{kr}\sin(kr)] \dd{r}\\
&= - 16 \I\sqrt2 \frac{\bvec k}{(k^2+1)^3}
}\end{equation}

\subsection{速度规范}
% Merzbacher 19.87 上面一条
\begin{equation}
\int \exp(-\I \bvec k \vdot \bvec r) \exp(-Zr) \dd[3]{r} = \frac{8\pi Z}{(k^2 + Z^2)^2}
\end{equation}

\begin{equation}
\begin{aligned}
&\int \exp(-\I \bvec k \vdot \bvec r) \grad \exp(-Zr) \dd[3]{r}
= -\int [\grad \exp(\I \bvec k \vdot \bvec r)]^* \exp(-Zr) \dd[3]{r}\\
&= \I \bvec k \int \exp(-\I \bvec k \vdot \bvec r) \exp(-Zr) \dd[3]{r}
= \I \frac{8 \pi Z \bvec k}{(k^2 + Z^2)^2}
\end{aligned}
\end{equation}