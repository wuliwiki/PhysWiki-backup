% 中国科学院 2012 年考研数学(甲)
% keys 中科院|2012|数学
% license Xiao
% type Tutor

\begin{issues}
\issueDraft
\end{issues}


\subsection{选择题}
1.函数$f(x)=xcosx^2$, 正确结论是()\\
(A).在($-\infty~,+\infty$)内有界\\
(B).当$x\to\infty$时$f(x)$为无穷大\\
(C)在($-\infty~,+\infty$)内无界\\
(D).当$x\to\infty$时$f(x)$极限存在\\

2.函数$f(x)$在($-\infty~,+\infty$)上是连续函数,且$0<m<f(x)<M<\infty$。则
$ \frac{1}{m} \int_{-m}^{m}(f(t)-M \dd{t})$的最大取值区间是()\\
(A).$(-M-m,m-M) \quad$( B).$ (2m-2M,0)$\\ (C).$(m-M,0)\qquad \qquad $(D).$(0,M+m)$

3.微分方程$y y''-(y')^2=0$的一个特解是()\\
(A).$y=xe^x$ $\quad$ (B).$y=x\ln x$ $\quad$ (C).$y=\ln x$  $\quad$ (D).$y=e^x$

4.已知$n,m$是正整数,且$n<m$,如果
\begin{equation}
A=\int_{0}^{1} x^m(1-x)^n \dd{x},B=\int_{0}^{1}x^n(1-x)^{m+1} \dd{x}~,
\end{equation}则下面结论正确的一个是()\\
(A).$A>B$ $\quad$ (B).$A=B$ $\quad$ (C).$A<B$ $\quad$ (D).$A,B$的大小关系不确定

5.函数$f(x)=e^x-x^2-4x-3$在其定义域内零点的个数是()\\
(A).1 $\quad$(B).2 $\quad$(C).3 $\quad$ (D).多于3

6.若函数$f(x)=\leftgroup{
    &e^x\\
    &abx^2
    }$