% 约翰·沃利斯(综述)
% license CCBYSA3
% type Wiki

本文根据 CC-BY-SA 协议转载翻译自维基百科\href{https://en.wikipedia.org/wiki/John_Wallis}{相关文章}。

\begin{figure}[ht]
\centering
\includegraphics[width=6cm]{./figures/7dd5979fcc742ec6.png}
\caption{约翰·沃利斯} \label{fig_YHALS_1}
\end{figure}
约翰·沃利斯(/ˈwɒlɪs/;[2]拉丁语:Wallisius;1616年12月3日 [旧历11月23日] – 1703年11月8日 [旧历10月28日])是英国的一位牧师和数学家,部分功劳归于他对微积分发展的贡献。

1643年至1689年间,沃利斯担任议会及后来王室的首席密码学家。[3] 他被认为引入了符号 ∞ 来表示无穷的概念。[4] 他同样使用 1/∞ 表示无穷小。他是牛顿的同时代人,也是数学早期复兴时期最伟大的知识分子之一。[5]
\subsection{传记}
\subsubsection{教育背景}  
\begin{itemize}
\item 剑桥大学,硕士学位;牛津大学,神学博士学位。  
\item 1625–1631年,就读于肯特郡滕特登的文法学校。  
\item 1631–1632年,就读于埃塞克斯郡费尔斯特的马丁·霍尔比奇学校。  
\item 1632–1640年,剑桥大学伊曼纽尔学院;1637年获学士学位,1640年获硕士学位。  
\item 1654年获牛津大学神学博士学位。
\end{itemize}