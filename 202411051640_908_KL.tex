% 库仑定律(综述)
% license CCBYSA3
% type Wiki

本文根据 CC-BY-SA 协议转载翻译自维基百科\href{https://en.wikipedia.org/wiki/Coulomb\%27s_law}{相关文章}。

\begin{figure}[ht]
\centering
\includegraphics[width=6cm]{./figures/e58a5f04bd7289ad.png}
\caption{两个点电荷 q1 和 q2 之间的静电力 F 的大小与它们电荷大小的乘积成正比,且与它们之间距离的平方成反比。相同电荷相互排斥,异性电荷相互吸引。} \label{fig_KL_1}
\end{figure}
库仑的反比平方定律,简称库仑定律,是一条物理学的实验定律[1],用于计算静止的两个带电粒子之间的相互作用力。这种电力通常称为静电力或库仑力[2]。尽管这一定律早有认识,但它是在1785年由法国物理学家查尔-奥古斯丁·库仑首次公布的。库仑定律对于电磁学理论的发展至关重要,甚至可能是其起点[1],因为它使得有意义的讨论粒子电荷量成为可能[3]。

该定律指出,两个点电荷之间的吸引或排斥静电力的大小(或绝对值)与它们电荷大小的乘积成正比,且与它们之间距离的平方成反比[4]。库仑发现,相同电荷的物体会相互排斥:

因此,从这三个实验中可以得出结论:两个带有相同类型电荷的球体相互排斥的力,遵循与距离的平方成反比的规律[5]。

库仑还表明,带相反电荷的物体会按照反比平方定律相互吸引:
\[
|F| = k_{\text{e}} \frac{|q_1| |q_2|}{r^2}~
\]
其中,\( k_{\text{e}} \) 是常数,\( q_1 \) 和 \( q_2 \) 是两个电荷的电量,\( r \) 是它们之间的距离。

力的方向沿着连接两个电荷的直线。如果电荷具有相同的符号,静电力使它们相互排斥;如果电荷符号不同,静电力使它们相互吸引。

作为一个反比平方定律,它类似于艾萨克·牛顿的万有引力反比平方定律,但引力总是使物体相互吸引,而静电力则使电荷相互吸引或排斥。此外,万有引力远弱于静电力[2]。库仑定律可以用来推导高斯定律,反之亦然。在静止的单个点电荷情况下,这两个定律是等效的,以不同的方式表达相同的物理规律[6]。这一定律已经被广泛验证,观测结果在从 \( 10^{-16} \) 米到 \( 10^8 \) 米的尺度上都符合该定律[6]。
\subsection{历史}
地中海周围的古代文化知道某些物体,如琥珀棒,可以通过与猫毛摩擦来吸引轻物体,如羽毛和纸片。米利都的泰利斯大约在公元前600年首次记录了静电现象[7],当时他注意到摩擦可以使一块琥珀吸引小物体[8][9]。

1600年,英国科学家威廉·吉尔伯特对电学和磁学进行了仔细研究,将磁铁效应与摩擦琥珀产生的静电区分开来[8]。他创造了新拉丁词“electricus”(意为“像琥珀一样”或“来自琥珀”,源自希腊语单词 ἤλεκτρον [elektron],意为“琥珀”),用来指代摩擦后能吸引小物体的性质[10]。这一联系促成了英语单词“electric”和“electricity”的诞生,它们首次出现在1646年托马斯·布朗的《伪常识》一书中[11]。

18世纪早期的研究者怀疑电力与重力一样会随距离衰减(即按距离的平方反比衰减),包括丹尼尔·伯努利[12]和亚历山德罗·伏打,他们都测量了电容器板之间的力,以及弗朗茨·艾皮努斯,他在1758年提出了反比平方定律[13]。

基于电荷球体的实验,英国的约瑟夫·普里斯特利是最早提出电力遵循反比平方定律的科学家之一,类似于牛顿的万有引力定律。然而,他并未进一步推广或详细阐述这一点[14]。在1767年,他猜测电荷之间的力随距离的平方反比变化[15][16]。