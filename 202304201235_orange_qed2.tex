% 标量场

\pentry{洛伦兹群\upref{qed1}, 欧拉—拉格朗日方程\upref{Lagrng}}
在量子场论里,我们把量子力学应用到场这样的动力学系统里面。相对应的是在量子力学里面,我们是把量子力学应用到粒子这样的动力学系统里面。这个课程是理解基本粒子物理学的关键。除此之外,量子场论还在原子物理,原子核物理,以及凝聚态物理中有诸多的应用。在这一章里面,我们主要考虑基本粒子,那些\textbf{相对论性}的粒子。

为什么我们必须学习场的量子化,而不能像我们之前在量子力学里面量子化非相对论粒子一样去量子化相对论性的粒子呢?

那么这个问题的回答有好几个层次。最好的方法是写下单个粒子的相对论性波动方程,比如克莱因-戈登方程,或者是狄拉克方程。之后,我们会发现这回导致负能态以及其他的一些不自洽。这个课程在本科生的高等量子力学中会学到。一般是学完量子力学课之后学习。那么在这儿我们就不再重复了。

另一方面,任何一个相对论性的过程并不能通过单个粒子来解释。因为在真空中会不断产生粒子-反粒子对。这种过程涉及到多粒子态,是量子力学所不能描述的。

从因果性方面的考虑来说,我们也是需要这种多粒子态的。考虑一个自由粒子从 $\mathbf x_0$ 传播到 $\mathbf x$ 的幅度
\begin{equation}
U(t) = \langle \mathbf x | e^{i H t} | \mathbf x_0 \rangle
\end{equation}
在非相对论性的量子力学里,我们有 $E = \mathbf p^2/2m$, 所以
\begin{equation}\label{eq_qed2_2}
\begin{aligned}
U(t) & = \langle \mathbf x | e^{-i(\mathbf p^2/2m)t}|\mathbf x_0\rangle \\
& = \int \frac{d^3 p}{(2\pi)^3} \langle x | e^{-i(\mathbf p^2/2m)t} | \mathbf p \rangle \langle \mathbf p | \mathbf x_0 \rangle \\
& = \frac{1}{(2\pi)^3} \int d^3 p e^{- i (\mathbf p^2/2m)t}\cdot e^{i\mathbf p \cdot (\mathbf x - \mathbf x_0) } \\
& = \bigg( \frac{m}{2\pi it} \bigg)^{3/2} e^{im(\mathbf x - \mathbf x_0)^2/2t}~.
\end{aligned}
\end{equation}
这个表达式对所有的 $x$ 和 $t$ 都成立,也就是说,一个例子可以在任意两点间,以任意短时间传播。

但是,在相对论性的理论里面,这个结论会导致因果性的违反。推导如下:

对于一个相对论的系统,能量是 $E = \sqrt{p^2+m^2}$,则
\begin{equation}
\begin{aligned}
U(t) & = \langle \mathbf x | e^{-i\sqrt{p^2+m^2}t}|\mathbf x_0\rangle \\
& = \int \frac{d^3 p}{(2\pi)^3} \langle x | e^{-i\sqrt{p^2+m^2}t} | \mathbf p \rangle \langle \mathbf p | \mathbf x_0 \rangle \\
& = \frac{1}{(2\pi)^3} \int d^3 p e^{- i \sqrt{p^2+m^2} t}\cdot e^{i\mathbf p \cdot (\mathbf x - \mathbf x_0) } \\
& =  \frac{1}{2\pi^2|\mathbf x - \mathbf x_0|}
\int _0^\infty d p p \sin(p|\mathbf x - \mathbf x_0|) e^{-it \sqrt{p^2+m^2}}
\end{aligned}
\end{equation}
这个积分可以积出来,答案是贝塞尔函数。我们可以研究一下它在 $x^2\gg t^2$ 时的表现。

相位函数 $px-t\sqrt{p^2+m^2}$ 在 $p=imx/\sqrt{x^2-t^2}$ 点出有一个静态点。我们可以把积分围道向上推,穿过这个点。我们把这个 $p$ 值代入到\autoref{eq_qed2_2} 中,得
\begin{equation}
U(t)\sim e^{-m\sqrt{x^2-t^2}}
\end{equation}
这个答案说明,传播的幅度很小,因为是指数压低的。但仍然在光锥外面非零,因果性仍然被违反了。

量子场论解决了这个问题,我们将看到,粒子沿着类空间个传播跟反粒子沿着反方向传播是不可区分的。这两个过程的振幅互相抵消。所以,量子场论是遵守因果律的。

量子场论不仅提供了多粒子态的一种自然的处理方法,还能处理不同粒子数的态之间的转换。它引入了反粒子,解决了因果律的问题。然后解释了自旋和统计之间的关系。最重要的是,它提供了计算散射截面,粒子寿命,以及很多别的可观测量的工具。

\begin{definition}{}
标量场即洛伦兹变换下不变的场。设在两个惯性系中,标量场分别为 $\phi(x)$ 和 $\phi'(x')$,那么有
\begin{equation}
\phi^{\prime}\left(x^{\prime}\right)=\phi(x)
\end{equation}
图景是这样的,对于一个固定点P而言,洛伦兹变换前后对应的坐标不同,场函数形式不同,结果是P的数值不变。
\end{definition}
\begin{corollary}{}
做一个无穷小变换
\begin{equation}
x^{\rho} \rightarrow x^{\prime \rho}=x^{\rho}+\delta x^{\rho}
\end{equation}
并且
\begin{equation}\label{eq_qed2_1}
\delta x^{\rho}=\omega_{\sigma}^{\rho} x^{\sigma}=-\frac{i}{2} \omega_{\mu \nu}\left(J^{\mu \nu}\right)_{\sigma}^{\rho} x^{\sigma}
\end{equation}
其中,$\left(J^{\mu \nu}\right)_{\sigma}^{\rho}=i\left(\eta^{\mu \rho} \delta_{\sigma}^{\nu}-\eta^{\nu \rho} \delta_{\sigma}^{\mu}\right)$
显然,令 $J^{\mu \nu}=0$, 该标量表示 $(0,0,)$ 能满足标量场的定义。这是个平庸的结果,在场表示里,我们能做到更多。 对于固定点 $x$,我们做场的无穷小变换
\begin{equation}
\delta_{0} \phi \equiv \phi^{\prime}(x)-\phi(x)
\end{equation}
为了找到该表示下的生成元,我们做一阶泰勒展开
\begin{equation}
\delta_{0} \phi=\phi^{\prime}\left(x^{\prime}-\delta x\right)-\phi(x)=-\delta x^{\rho} \partial_{\rho} \phi(x)
\end{equation}
代入\autoref{eq_qed2_1} ,我们有
\begin{equation}
\delta_{0} \phi=\frac{i}{2} \omega_{\mu \nu}\left(J^{\mu \nu}\right)_{\sigma}^{\rho} x^{\sigma} \partial_{\rho} \phi \equiv-\frac{i}{2} \omega_{\mu \nu} L^{\mu \nu} \phi
\end{equation}
在这里,我们定义
\begin{equation}
L^{\mu \nu}=-\left(J^{\mu \nu}\right)_{\sigma}^{\rho} x^{\sigma} \partial_{\rho}=i\left(x^{\mu} \partial^{\nu}-x^{\nu} \partial^{\mu}\right)
\end{equation}
我们可以验证,$L^{\mu \nu}$ 满足洛伦兹群的李代数,从而确实是洛伦兹变换的生成元。所以在这里我们给出了无限维的洛伦兹群的标量场表示。由于场函数随着坐标不同而不同,所以是无限维的。

实际上,如果定义 $p^{\mu}=+i \partial^{\mu}$,会有 $L^{\mu \nu}=x^{\mu} p^{\nu}-x^{\nu} p^{\mu}$,若上标取空间坐标i,j,我们会发现这就是轨道角动量。由于 $J^{\mu \nu}=L^{\mu \nu}+S^{\mu \nu}$,所以标量场表示是自旋为 0 的。

综上所述,我们得到了洛伦兹群的标量场表示,标量场在该表示下进行洛伦兹变换,由于该表示自旋为 0,所以标量场代表的是自旋为 0 的场。
\end{corollary}

\subsection{举例}
克莱因—戈登场即为标量场。克莱因—戈登方程为
\begin{equation}
\frac{1}{c^{2}} \frac{\partial^{2}}{\partial t^{2}} \psi-\nabla^{2} \psi+\frac{m^{2} c^{2}}{\hbar^{2}} \psi=0
\end{equation}
取自然单位,$c=\bar{h}=1$,得到协变形式
\begin{equation}
\left(\square^{2}+m^{2}\right) \psi=0
\end{equation}
达朗贝尔算符 $\square^{2}=\frac{1}{c^{2}} \frac{\partial^{2}}{\partial t^{2}}-\nabla^{2}$
在经典场论,它的拉格朗日量为 $L=\frac{1}{2} \partial_{\mu} \phi \partial^{\mu} \phi-\frac{1}{2} m^{2} \phi^{2}$,由欧拉-拉格朗日方程可解得克莱因-戈尔登方程方程。根据能量本征值解,我们知道这是一个相对论性的自旋为0的标量场。
\begin{corollary}{}
以上我们得到了经典场论下的标量场,接着我们将标量场进行正则量子化。所谓量子化,实际上是把场看做无数个简谐振子,这由产生和湮灭算符体现。

对于量子力学,我们有坐标-动量基本对易关系
$\left[\hat{q^{i}}, \hat{p^{j}}\right]=i \delta^{i j}$
若在海森堡绘景,该对易子则是等时性的。
同样的,在量子场论里,我们也可以有类似的操作。场 $\hat{\phi}(t, \mathbf{x})$ 对应于坐标算符 $\hat{q^{i}}$,场的共轭动量定义为\begin{equation}
\hat{\Pi}_{\hat{\phi}}=\frac{\partial \hat{\mathcal{L}}}{\partial\left(\partial_{0} \phi\right)}=\partial_{0} \hat{\phi}
\end{equation},其中 $\hat{\mathcal{L}}$ 为拉氏量密度,\textbf{Action} $S=\int d t L=\int d^{4} x \mathcal{L}\left(\phi, \partial_{\mu} \phi\right)$

$\hat{\Pi}(t, \mathbf{x})$ 对应于动量算符 $\hat{q^{i}}$,则有
\begin{equation}
[\hat{\phi}(t, \mathbf{x}), \hat{\Pi}(t, \mathbf{y})]=i \delta^{(3)}(\mathbf{x}-\mathbf{y})
\end{equation}
等时下
\begin{equation}
[\hat{\phi}(t, \mathbf{x}), \hat{\phi}(t, \mathbf{y})]=[\hat{\Pi}(t, \mathbf{x}), \hat{\Pi}(t, \mathbf{y})]=0
\end{equation}
实标量场是厄米算符。由此我们可得到克莱因-戈登方程的算符解,即:
\begin{equation}
\hat{\phi}(x)=\int \frac{d^{3} p}{(2 \pi)^{3} \sqrt{2 E_{\mathbf{p}}}}\left(\hat{a}_{\mathbf{p}} e^{-i p x}+\hat{a}_{\mathbf{p}}^{\dagger} e^{i p x}\right)
\end{equation}
可以验证算符的厄米性。

其中 $ipx=ip_\mu x^\mu=iEt-i\vec{p}\vec{x}$,可以证明,场算符的对易关系等价于
\begin{equation}
\left[\hat{a}_{\mathbf{p}}, \hat{a}_{\mathbf{q}}^{\dagger}\right]=(2 \pi)^{3} \delta^{(3)}(\mathbf{p}-\mathbf{q})
\end{equation}
\begin{equation}
\left[\hat{a}_{\mathbf{p}}, \hat{a}_{\mathbf{q}}\right]=0, \quad\left[\hat{a}_{\mathbf{p}}^{\dagger}, \hat{a}_{\mathbf{q}}^{\dagger}\right]=0
\end{equation}
自然而然的,从对易关系我们可以看出,场算符的两项由产生算符和湮灭算符线性叠加而成。
\end{corollary}
\subsection{能量}
我们需要知道量子化标量场后的哈密顿密度形式。由于
\begin{equation}
\hat{\mathcal{H}}=\hat{\Pi}_{\hat{\phi}} \partial_{0} \hat{\phi}-\hat{\mathcal{L}}=\frac{1}{2}\left[\hat{\Pi}_{\hat{\phi}}^{2}+(\nabla \hat{\phi})^{2}+m^{2} \hat{\phi}^{2}\right]
\end{equation}
\begin{equation}
\hat{H}=\int d^{3} x \hat{\mathcal{H}}=\int \frac{d^{3} p}{(2 \pi)^{3}} E_{p} \frac{1}{2}\left(\hat{a}_{\mathbf{p}}^{\dagger} \hat{a}_{\mathbf{p}}+\hat{a}_{\mathbf{p}} \hat{a}_{\mathbf{p}}^{\dagger}\right)=\int \frac{d^{3} p}{(2 \pi)^{3}} E_{p}\left(\hat{a}_{\mathbf{p}}^{\dagger} \hat{a}_{\mathbf{p}}+\frac{1}{2}\left[\hat{a}_{\mathbf{p}}, \hat{a}_{\mathbf{p}}^{\dagger}\right]\right)
\end{equation}
第二项是所有谐振子的真空零点能,由于我们实际上的测值是能量差,所以这里出现的发散情况并无大碍。

对于这种操作,我们引入\textbf{normal ordering}这一概念,对于算符 $\hat{\mathcal{O}}$,其 normal ordered form表示为 $:\mathcal{O}:$,把产生算符都放左边,湮灭算符都放右边,从而得到 $:\mathcal{O}:$。

所以
\begin{equation}
\hat{H}=\frac{1}{2} \int d^{3} x: \hat{\Pi}^{2}+(\nabla \hat{\phi})^{2}+m^{2} \hat{\phi}^{2}:
\end{equation}

\subsection{因果性}
我们需要验证这个理论是否符合因果性,即类空距离的算符是否都对易
\begin{equation}
\left[\mathcal{\hat{O}}_{1}(x), \mathcal{\hat{O}}_{2}(y)\right]=0 \quad \forall \quad(x-y)^{2}<0
\end{equation}
我们定义
\begin{equation}
\Delta(x-y)=[\hat{\phi}(x), \hat{\phi}(y)]
\end{equation}
可求得
\begin{equation}
\Delta(x-y)=\int \frac{d^{3} p}{(2 \pi)^{3}} \frac{1}{2 E_{\vec{p}}}\left(e^{-i p \cdot(x-y)}-e^{i p \cdot(x-y)}\right)
\end{equation}
从上式中我们可以获知:
\begin{enumerate}
\item 这是洛伦兹不变的。积分测度的不变性可以由delta函数的性质证明,指数函数由于是数,所以也是洛伦兹不变的。
\item 类时下,该函数不为0,类空反之。我们可以通过举特例证明这一点。(由洛伦兹不变性知特例的结果取值不变)
\end{enumerate}
因而我们证明了标量场理论是符合因果律的。

\subsection{传播子}
在量子力学里,传播子即为跃迁振幅,$\left\langle\mathbf{x}^{\prime \prime}, t \mid \mathbf{x}^{\prime}, t_{0}\right\rangle$ 为 $t_0$ 时制备的具有本征值\textbf{x'}的粒子在t时刻,\textbf{x''}处被发现的概率振幅。也就是粒子从一个时空点到另一个时空点的概率振幅。(当然啦,这只是其中一种解释)场论中亦然,
由于场算符由产生算符和湮灭算符线性叠加而成,其作用于真空态则意味着在某处产生粒子。所以传播子的表达可为
$\begin{aligned}\langle 0|\hat{\phi}(x) \hat{\phi}(y)| 0\rangle &=\int \frac{d^{3} p d^{3} p^{\prime}}{(2 \pi)^{6}} \frac{1}{\sqrt{4 E_{\vec{p}} E_{\vec{p}^{\prime}}}}\left\langle 0\left|a_{\vec{p}} a_{\vec{p}^{\prime}}^{\dagger}\right| 0\right\rangle e^{-i p \cdot x+i p^{\prime} \cdot y} \\ &=\int \frac{d^{3} p}{(2 \pi)^{3}} \frac{1}{2 E_{\vec{p}}} e^{-i p \cdot(x-y)} \equiv D(x-y) \end{aligned}$

可以验证
\begin{equation}
[\hat{\phi}(x), \hat{\phi}(y)]=D(x-y)-D(y-x)=0 \quad \text { if } \quad(x-y)^{2}<0
\end{equation}

\subsection{费曼传播子}
在相互作用场论中,费曼传播子相当重要。它是这么定义的
\begin{equation}
\Delta_{F}(x-y)=\langle 0|T \hat{\phi}(x) \hat{\phi}(y)| 0\rangle=\left\{\begin{array}{ll}
D(x-y) & x^{0}>y^{0} \\
D(y-x) & y^{0}>x^{0}
\end{array}\right.
\end{equation}
其中T是\textbf{time ordering} 编时乘积,作用是把时间靠后的算符放在前面
\begin{equation}
T \hat{\phi}(x) \hat{\phi}(y)=\left\{\begin{array}{ll}
\hat{\phi}(x) \hat{\phi}(y) & x^{0}>y^{0} \\
\hat{\phi}(y) \hat{\phi}(x) & y^{0}>x^{0}
\end{array}\right.
\end{equation}
可以用留数定理证明,标量场的费曼传播子是
\begin{equation}
\Delta_{F}(x-y)=\int \frac{d^{4} p}{(2 \pi)^{4}} \frac{i}{p^{2}-m^{2}} e^{-i p \cdot(x-y)}
\end{equation}

\subsection{格林函数} 
费曼传播子即 Klein-Gordon方程的格林函数解。
\begin{equation}
\begin{aligned}
\left(\partial_{t}^{2}-\nabla^{2}+m^{2}\right) \Delta_{F}(x-y) &=\int \frac{d^{4} p}{(2 \pi)^{4}} \frac{i}{p^{2}-m^{2}}\left(-p^{2}+m^{2}\right) e^{-i p \cdot(x-y)} \\
&=-i \int \frac{d^{4} p}{(2 \pi)^{4}} e^{-i p \cdot(x-y)} \\
&=-i \delta^{(4)}(x-y)
\end{aligned}
\end{equation}
