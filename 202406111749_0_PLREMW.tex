% 简单的偏振电磁波
% license Usr
% type Tutor

\begin{issues}
\issueDraft
\end{issues}

\pentry{真空中的平面电磁波\nref{nod_VcPlWv}}{nod_5991}
\footnote{本文参考了周磊教授的《电动力学》课程及讲义}

\begin{figure}[ht]
\centering
\includegraphics[width=5cm]{./figures/c3735376741092d8.pdf}
\caption{沿z轴传播的电磁波只有x,y分量} \label{fig_PLREMW_1}
\end{figure}

让我们想象一束沿$z$轴传播的电磁波。由于电磁波是横波,所以$E_z=0$。为简明起见,我们假定电场$x,y$两个分量的振幅相同,且$E_x$分量的相位因子为0.\footnote{重要的是分量间的相位差,而不是具体的初相位}此时,电场的波函数就可以写为
\begin{equation}
\bvec E = 
\begin{pmatrix}
E_{0} \cos(kz - \omega t)\\
E_{0} \cos(kz - \omega t+\varphi_{0})\\
0\\
\end{pmatrix}~.
\end{equation}

根据$\varphi_0$的取值,电磁波也就呈现不同的偏振类型。这让我们联想到\enref{利萨茹曲线}{Lissaj}。

\subsubsection{$\varphi_0=n\pi, n=0,\pm1,\pm2,...$:线偏振}
\begin{figure}[ht]
\centering
\includegraphics[width=10cm]{./figures/ed70e9f1e9516632.pdf}
\caption{线偏振。右侧图像为俯视图,下同} \label{fig_PLREMW_2}
\end{figure}

\subsubsection{$\varphi_0=\frac{\pi}{2}n, n=\pm1,\pm3,\pm5,...$:圆偏振}
\begin{figure}[ht]
\centering
\includegraphics[width=10cm]{./figures/428036d78c042354.pdf}
\caption{圆偏振,\href{https://www.geogebra.org/m/hj6qsfdu}{一个可动的模型}(站外链接)} \label{fig_PLREMW_3}
\end{figure}

\subsubsection{其余情况:椭圆偏振}
\begin{figure}[ht]
\centering
\includegraphics[width=10cm]{./figures/80c2537d5635065b.pdf}
\caption{椭圆偏振} \label{fig_PLREMW_4}
\end{figure}

\subsection{附录:绘制图像的 Matlab 代码}
\begin{lstlisting}[language=matlab]
t=0; 
T=1; %周期
v=1; %波速
phi0=pi/3; %相位差
E0=1; %振幅

w=2*pi/T;
k=w/v;

E = [];
z=0:0.01:5;
E(1,:)=E0*cos(k*z-w*t);
E(2,:)=E0*cos(k*z-w*t+phi0);
E(3,:)=z;

subplot(1,2,1)
hold on
axis equal
axis off
line([0 0],[0 0],[0,5],'color','r');
line([0 2],[0 0],[0,0],'color','r');
line([0 0],[0 2],[0,0],'color','r');
plot3(E(1,:),E(2,:),E(3,:),'b')
for i=1:20:size(z,2) %绘制电场向量
  quiver3(0,0,E(3,i),E(1,i),E(2,i),0,'b');
end
view(30,30)

subplot(1,2,2)
hold on
axis equal
axis off
plot3(E(1,:),E(2,:),E(3,:),'b')
view(0,90)

\end{lstlisting}
