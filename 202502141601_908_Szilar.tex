% 利奥·西拉德(综述)
% license CCBYSA3
% type Wiki

本文根据 CC-BY-SA 协议转载翻译自维基百科\href{https://en.wikipedia.org/wiki/Leo_Szilard}{相关文章}。

\begin{figure}[ht]
\centering
\includegraphics[width=6cm]{./figures/1949e65970ed96d0.png}
\caption{西拉德,约1960年} \label{fig_Szilar_1}
\end{figure}
莱奥·西拉德(Leo Szilard,/ˈsɪlɑːrd/;匈牙利语:Szilárd Leó [ˈsilaːrd ˈlɛoː];原名Leó Spitz;1898年2月11日 – 1964年5月30日)是一位匈牙利出生的物理学家、生物学家和发明家,他在核物理学和生物学领域做出了许多重要发现。他在1933年构思了核链式反应,并于1936年为此申请了专利。1939年底,他为阿尔伯特·爱因斯坦签署的信件起草了内容,这封信促成了曼哈顿计划的启动,最终制造出了原子弹。随后在1944年,他撰写了《西拉德请愿书》,要求杜鲁门总统在没有将原子弹投放到平民身上的情况下展示其威力。根据吉尔吉·马克斯的说法,他是被称为“火星人”的匈牙利科学家之一。[1]

西拉德最初在布达佩斯的约瑟夫皇帝技术大学(Palatine Joseph Technical University)学习,但他的工程学学业在第一次世界大战期间因服役于奥匈帝国军队而中断。他于1919年离开匈牙利前往德国,进入柏林查理滕堡的技术大学(现为柏林工业大学)就读,但他对工程学感到厌倦,转学到弗里德里希·威廉大学,开始学习物理学。他的博士论文是关于麦克斯韦妖的,这是热力学和统计物理哲学中的一个长期难题。西拉德是第一个认识到热力学与信息理论之间联系的著名科学家。

西拉德为电子显微镜(1928年)、回旋加速器(1929年)提出了最早的专利申请并发布了相关的首批论文,还参与了线性加速器(1928年)在德国的发展。1926年到1930年间,他与爱因斯坦合作开发了爱因斯坦冰箱。

1933年阿道夫·希特勒成为德国总理后,西拉德敦促他的家人和朋友趁还可以时逃离欧洲。他移居英国,并帮助成立了学术援助委员会,这是一个致力于帮助难民学者找到新工作的组织。在英国期间,西拉德与托马斯·A·查尔默斯一起发现了一种同位素分离的方法,这就是著名的西拉德-查尔默斯效应。

预见到欧洲将爆发另一场战争,西拉德于1938年移居美国,在那里他与恩里科·费米和沃尔特·津恩合作研究如何制造核链式反应。1942年12月2日,他亲眼见证了这一目标在芝加哥堆-1中实现。他在芝加哥大学的曼哈顿计划冶金实验室工作,参与了核反应堆设计的相关工作,并担任首席物理学家。他起草了《西拉德请愿书》,主张以非致命方式展示原子弹,但临时委员会最终决定将其用于军事打击。

他与恩里科·费米一起,于1944年申请了核反应堆的专利。他公开警告可能会发展出盐化热核弹,这是一种新型的核武器,可能会灭绝人类。

他在生物科学方面的发明、发现和贡献同样重要;其中包括反馈抑制的发现和化学反应池的发明。根据西奥多·帕克和菲利普·I·马库斯的说法,西拉德提供了至关重要的建议,使得人类细胞的最早克隆成为现实。

1960年,他被诊断为膀胱癌,并接受了他自己设计的钴-60治疗。他帮助创立了索尔克生物研究所,并成为该研究所的常驻学者。西拉德于1962年创立了“适宜生存世界委员会”,旨在向国会、白宫和美国公众传递关于核武器的“理性之声”。他于1964年因心脏病发作在睡梦中去世。
\subsection{早年生活}  
他于1898年2月11日出生在匈牙利王国布达佩斯,原名Leó Spitz。他的父母是中产阶级犹太人,父亲Lajos(Louis)Spitz是一名土木工程师,母亲Tekla Vidor将Leó抚养在佩斯的Városligeti Fasor街。[2] 他有两个弟妹,哥哥Béla出生于1900年,妹妹Rózsi出生于1901年。1900年10月4日,家族将姓氏从德语的“Spitz”改为匈牙利语的“Szilárd”,这个名字在匈牙利语中意味着“坚固”。[3] 尽管有宗教背景,西拉德后来成为了不可知论者。[4][5] 从1908年到1916年,他就读于布达佩斯第六区的Föreáliskola(高中)。他从小对物理学表现出浓厚兴趣,并且在数学方面很有天赋,1916年,他获得了厄尔托奖,这是一个国家级的数学奖项。[6][7]
\begin{figure}[ht]
\centering
\includegraphics[width=6cm]{./figures/63e9374f9b39cc50.png}
\caption{西拉德,约1915年[8]} \label{fig_Szilar_2}
\end{figure}