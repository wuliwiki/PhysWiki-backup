% nginx 笔记

\begin{issues}
\issueDraft
\end{issues}

\begin{itemize}
\item 一个\href{https://zhuanlan.zhihu.com/p/80600540}{知乎教程}
\item \href{https://nginx.org/en/docs/}{官方文档}
\item \verb|sudo apt install nginx|
\item 要重启 nginx 服务用 \verb|sudo systemctl restart nginx|
\item 要查看是否连接成功, 用 \verb|curl localhost| (默认访问 80 接口)。 如果打印出一个 html 文本, 包含 \verb|Welcome to nginx!|, 就是成功了(当然也可以用浏览器访问, 只是有时候只有命令行)。
\item 如果要限制 nginx 只监听某个网卡, 编辑配置文件 \verb|sudo vim /etc/nginx/sites-enabled/default|, 然后在 \verb|listen 80 default_server;| 的 \verb|80| 改成 \verb|ip地址:80|, 然后重启 \verb|nginx| 服务即可生效。
\item 事实上不光是本机, 监听的网卡所在的所有机器访问该网卡的 ip 的 80 端口都会收到
\end{itemize}

\subsection{静态网站}
\begin{itemize}
\item 配置文件: \verb|/etc/nginx/nginx.conf|。 在 \verb|http| section 里面加入
\begin{lstlisting}[language=none]
server {
    listen 80;
    server_name ip地址或者域名:80;
    
    location / {
        root /静态网页根目录;
    }
}
\end{lstlisting}
\end{itemize}

