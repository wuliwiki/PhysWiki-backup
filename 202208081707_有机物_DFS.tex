% 深度优先搜索(DFS)
% DFS|算法|C++|搜索

深度优先搜索(DFS,Depth First Search)简称深搜或者爆搜,DFS 的搜索顺序是按照深度优先搜索,简单来说就是“一条路走到黑”,搜索是把所有方案都试一遍,再判断是不是一个可行解.搜索与“递归”和“栈”有很大的联系,递归是实现搜索的一种方式,而栈则是计算机实现递归的方式.每个搜索过程都对应着一棵\textbf{递归搜索树},递归搜索树可以让我们更加容易的理解 DFS.
整个搜索过程就是基于该搜索树完成的,为了不重复遍历每个结点,会对每个节点进行标记,也可以对树中不可能是答案的分支进行删除,从而更高效的找到答案,这种方法被称为\textbf{剪枝}.如果搜索树在某个子树中搜索到了叶结点,想继续搜索只能返回上个或多个状态,返回的过程称为\textbf{回溯},回溯要记得\textbf{恢复状态},才能保证接下来的搜索过程可以正常进行.


\subsection{普通搜索}
来看一道\href{https://www.luogu.com.cn/problem/P1706}{具体例题}学习 DFS

题意:输出 $n$ 的全排列

思路:以 $n$ 为 $3$ 举例,枚举每个位置上该填什么数,但是每一位上的数不能和其他位置上的数一样,填满了 $3$ 位就输出,然后回溯继续搜索.

\begin{figure}[ht]
\centering
\includegraphics[width=14.25cm]{./figures/DFS_1.png}
\caption{递归搜索树} \label{DFS_fig1}
\end{figure}

上图则是一棵递归搜索树,就是搜索的过程形象化的显示出来.从第一个数字开始填,在填第二个数字,填过的数字记得要标记一下“使用过了”,不然会导致三个数字有两个数字是重复的.上文提到的回溯,就是填完了三个数字,先输出答案,无法在继续搜索下去就要回溯,回溯记得要恢复状态,不然接下来无法填数.

\textbf{算法思路:}
\begin{enumerate}
\item 用 \verb|path| 数组保存排列.

\item 用 \verb|st| 数组保存每个数的状态,\verb|st[i]| 为 $\mathtt{true}$ 则表示数字 $i$ 被用过,反之没被用过.

\end{enumerate}
\begin{lstlisting}[language=cpp]
int path[10], st[10], n;

void dfs(int u)
{
    if (u == n) // u == n 则说明填满了三个数
    {
        for (int i = 0; i < n; i ++ )
            cout << path[i] << ' ';     // 输出答案
        cout << endl;
        return;     // 返回,进行回溯操作
    }
    
    for (int i = 1; i <= n; i ++ )
        if (!st[i])
        {
            st[i] = true;   // 标记数字 i 被使用过
            path[u] = i;    // 第 u 个位置上的数是 i
            
            dfs(u + 1);     // 搜下一个位置
            
// 如果开始执行这段代码了,说明已经填满了 3 个数,正在进行回溯操作,则需要恢复状态
            st[i] = false;  
// 第 u 个位置变成空了,但这句话其实没必要写,在回溯完毕准备填数的时候则会被覆盖
            path[u] = 0;    
        }
        
    return;     // 回溯
}
\end{lstlisting}

DFS 的思想和代码很容易理解,这里不再赘述,但初学者学不懂 DFS 的原因主要是不理解递归而理解不了 DFS.这里来详细的讲解一下递归的执行过程.

\begin{figure}[ht]
\centering
\includegraphics[width=14.25cm]{./figures/DFS_3.png}
\caption{递归调用的过程} \label{DFS_fig3}
\end{figure}


进入入口则是主函数调用,然后在执行一段代码然后递归调用(对应上面的代码就是 \verb|dfs(u+1)|),如果某一次递归的过程的中触发了判断条件,则返回(回溯),回溯到上次执行我这次的递归的代码的下面再继续执行代码,如图中的第一根红线,对应上面的代码是先输出了 $1 2 3$,然后是第一次触发判断条件,这是 $u$ 为 $3$,就返回到 $u$ 为 $2$ 的这次递归上,然后不执行 \verb|dfs(u+1)|,直接执行下面的代码.然后没代码可以执行了,就只能执行第 $27$ 行的代码继续返回(回溯),然后再继续做接下来的操作.


\subsection{搜索的剪枝}

搜索的剪枝是优化搜索的很好的一种方式,即及时排除不可能的答案,从而少搜索点结点,就可以优化时间.在递归搜索树中则是删除一个子树,则被成为“剪枝”.

\textbf{剪枝的常见方法:}

\begin{enumerate}
\item 优化搜索顺序\\
在某些问题中,顺序怎么来枚举都差不多,但对于一些特殊的问题,搜索的顺序会直接影响到整个 DFS 的效率,可见,优化搜索顺序是 DFS 剪枝中的一个常用方法.

\item 排除等效冗余 \\
简单来讲,就是不搜索重复的结点,比如搜索 $123$ 和搜索 $321$ 是同一个效果(特殊问题除外),一遍都是人为定义一个搜索的顺序.只按照这个顺序搜索下去,因为改变一下顺序搜索对答案的影响是一样的.

\item 可行性剪枝\\
如果搜索的当前节点无法搜索到答案,则提前返回,这也是 DFS 剪枝中最常用的剪枝.

\item 最优化剪枝 \\
如果搜索的当前节点不可能作为答案输出,则提前返回,比如再往下搜索对答案的影响都不可能再变小,则直接返回.
\end{enumerate}

同样的,让我们来看\href{https://loj.ac/p/2148}{一道例题}来看看剪枝是怎么应用到题目中的.

\textbf{题目大意:}
给定若干个长木棍和若干的短木棍,让我们从长木棍中裁剪出若干个短木棍,问我们最多可以裁出多少根短木棍.每个长木棍和短木棍只能使用一次.

\textbf{思路:}
数据范围只有 $m\leq50$ 所以直接深搜就行了, 但需要加一点剪枝优化.


\textbf{剪枝 $1$}

排序 $+$ 二分 

先将长木棍和短木棍从小到大排序.

显然,我们裁的时候一定是从小到大裁剪的前 $k$ 个,不妨设我们裁剪的不是最小的 $k$ 个,那么我们裁剪最小的 $k$ 个,方案数不会变少.

所以只需要裁剪从小到大的前 $k$ 的短木棍就可以了,所以需要存一个短木棍的前缀和 \verb|S[i]|.

我们还可以发现,如果裁剪的是前 $k$ 个短木棍的话,那么在 $k$ 之前的一定可以裁剪,在 $k$ 之后的一定不能裁剪,所以具有二段性,所以就可以二分了.	

\textbf{剪枝 $2$(优化搜索顺序)}

我们在判断前 $k$ 个短木棍能不能裁剪出来的时候,直接搜索就可以了,可以从后往前枚举(即从大到小)短木棍.

从大到小枚举可以使搜索树的分支更少,从而更快的触发剪枝条件.

\textbf{剪枝 $3$ (可行性剪枝)}

我们开一个变量 \verb|total| 来记录所有长木棍的总长度,如果 \verb|total < S[k]| 那么一定无解.

\textbf{剪枝 $4$ }

如果某个长木棍的长度都要 $<$ 最短的木棍的长度的话,那么也一定无解.

\textbf{剪枝 $5$(排除等效冗余)}
\begin{enumerate}
\item 如果有几个长度相等的短木棍的话,改变裁剪顺序都是等价的,所以我们可以定一个搜索顺序,避免重复搜索.

\item 如果我们发现有两个相等的长木棍,如果在第一个长木棍无解,那么与它相等的长木棍也一定无解.

\end{enumerate}

\textbf{完整代码:}
\begin{lstlisting}[language=cpp]
#include <iostream>
#include <cstring>
#include <algorithm>

using namespace std;

const int N = 55, M = 2 << 10;

int n, m;
int Board[N];      // 老板提供的每一块木板的长度(长木板)
int board[M];     // 约翰需要的每一块木板的长度(短木板)
int total;              // 长木板的总长度
int sum[M];             // 短木板的前缀和
int mid;


bool dfs(int u, int start)
{
    if (!u) return true;
    if (total < sum[u]) return false; // 剪枝 3
    if (u + 1 > mid || board[u] != board[u + 1]) start = 1;
    
    for (int i = start; i <= n; i ++ )
    {
        if (i > start && Board[i] == Board[i - 1]) continue;  // 剪枝 5.2
        
        if (Board[i] >= board[u])
        {
            total -= board[u], Board[i] -= board[u];
            if (Board[i] < board[1]) total -= Board[i]; // 剪枝 4
            
            if (dfs(u - 1, i))
            {   
                // 由于要回溯,所有需要恢复现场
                if (Board[i] < board[1]) total += Board[i];
                total += board[u], Board[i] += board[u];
                return true;
            }
            
            if (Board[i] < board[1]) total += Board[i];
            total += board[u], Board[i] += board[u];
        }
    }
    
    return false;
}

int main()
{
    cin >> n; 
    for (int i = 1; i <= n; i ++ ) 
        cin >> Board[i], total += Board[i];    // 记录所有长木棍的总长度
    sort(Board + 1, Board + n + 1);
    
    cin >> m;
    for (int i = 1; i <= m; i ++ ) 
        cin >> board[i];
    sort(board + 1, board + m + 1);
    
    // 短木棍的前缀和
    for (int i = 1; i <= m; i ++ ) 
        sum[i] = sum[i - 1] + board[i];
    
    // 二分答案
    int l = 0, r = m;
    while (l < r)
    {
        mid = l + r + 1 >> 1;
        if (dfs(mid, 1)) l = mid;
        else r = mid - 1;
    }
    
    cout << r << endl;
    
    return 0;
}
\end{lstlisting}