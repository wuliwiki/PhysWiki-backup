% 核裂变
% license CCBYSA3
% type Wiki

(本文根据 CC-BY-SA 协议转载自原搜狗科学百科对英文维基百科的翻译)

在核物理和核化学中,\textbf{核裂变}是指核反应或放射性衰变过程中原子核分裂成更小、更轻的核的现象。裂变过程通常产生自由中子和γ 光子,同时释放出大量的能量。

重元素核裂变于1938年12月17日由德国人奥托·哈恩和他的助手弗里茨·施特拉斯曼发现,1939年1月莉泽·迈特纳和她的侄子奥托·弗里施给出了理论解释。弗里希将这一过程比喻为生物细胞的分裂。重核素的裂变是一个放热反应,会以电磁辐射和裂变碎片的动能形式释放大量能量。为了使裂变过程释放能量,裂变产物元素的总结合能应该大于起始元素的结合能。

裂变是一种核嬗变,因为裂变产物与初始原子属于不同的元素。裂变产生的两个核的质量通常比较接近,对于一般的可裂变同位素,其裂变产物的质量比约为3比2。[1][2]大多数裂变是二元裂变(产生两个带电碎片),但偶尔(每1000次事件发生2到4次)会发生三元裂变,产生三个带正电荷的碎片。三元裂变过程中最小的裂变碎片大小可位于质子到氩核之间。

除了已经被人类开发利用的中子诱导裂变之外,还存在另一种形式的裂变,称为自发放射性衰变(不需要中子诱导),该现象易发生在具有较高质量数的同位素中。自发裂变于1940年由弗廖罗夫、彼得扎克和库尔恰托夫[3]在莫斯科发现,当时他们决定通过实验验证尼尔斯·玻尔作出的一个预测,即没有中子轰击时,铀几乎不发生裂变,然而实验结论正相反。[3]

产物组成的不可预测性(产物可能的种类很多且无规律性)将裂变与量子隧穿过程区分开来,如质子发射、α衰变和团簇衰变等量子隧穿过程每次的产物都是相同的。核裂变是核电站及核武器的能量来源。作为核燃料的物质在被裂变中子撞击时会发生裂变,而在它们裂变过程中又会发射中子。这使得自持的核链式反应成为可能,这种核链式反应可以在核反应堆中以受控的速率释放能量,或者在核武器中以非常快速、不受控制的速率释放能量。

核燃料中包含的自由能是同等质量的化学燃料(如汽油)的数百万倍,这使得核裂变成为一种非常高效的能源。然而,核裂变的产物的放射性通常比作为裂变燃料的重元素高得多,并且其半衰期相当长,导致了核废料的问题。对核废料积聚和核武器的潜在破坏力的担忧影响了人们和平利用裂变作为能源的愿望。

\subsection{ 物理概述}
\subsubsection{1.1 机制}

\textbf{放射性衰变}

核裂变可以在没有中子轰击的情况下发生,这是一种放射性衰变。这种类型的裂变(称为自发裂变)除了少数重同位素之外很罕见。
\begin{figure}[ht]
\centering
\includegraphics[width=8cm]{./figures/b2a24667fa77a86b.png}
\caption{一种中子诱导核裂变事件的视觉演示,其中速度较慢的中子被铀-235原子的原子核吸收,原子核分裂成两种快速运动的较轻元素(裂变产物)和额外的中子。释放的大部分能量以裂变产物和中子的动能的形式存在。} \label{fig_HLB_1}
\end{figure}

\textbf{核反应}

在工程核设施中,基本上所有的核裂变都是以“核反应”形式发生的,其产生于轰击过程中亚原子的碰撞。在核反应中,亚原子粒子与原子核发生碰撞并使其发生变化。核反应是由轰击机制驱动的,与自发放射性衰变过程不同,后者具有相对稳定的指数衰减规律和特征半衰期。

人们目前已经发现许多种核反应。核裂变与其他类型的核反应有很大不同,因为它可以通过核裂变链式反应被放大并控制。在核裂变链式反应中,每个裂变事件释放的自由中子可以触发更多的裂变过程,这反过来又释放更多的中子并导致更多的裂变。

能够维持裂变链式反应的元素同位素被称为核燃料,我们称该种核素是可裂变的。最常见的核燃料是235U(铀的同位素,原子量为235,常用于核反应堆中)和239Pu(钚的同位素,原子量为239)。这些燃料的裂变产物的原子质量呈成双峰分布,峰值分布在95u和135u附近。大多数核燃料的自发裂变非常缓慢,而是主要通过α-β衰变链,其过程可长达千年至数万年甚至更久。在核反应堆或核武器中,绝大多数裂变事件是由中子轰击引起的,这些中子本身是由先前的裂变事件产生的。

裂变燃料中的核裂变是易裂变核素俘获中子时产生的核激发能的结果,这些能量来自于中子和原子核之间相互吸引的核力,该能量使原子核变形为双瓣状的“液滴”,原子核的“两叶”均带正电荷,当两叶之间的距离超过核力能够维持其不分离的范围时,两个裂变随便就完成了分离,进而被相互排斥的电荷进一步分开距离越来越远,因此这是一个不可逆过程。可裂变同位素(如铀-238)中也会发生类似的过程,但这些同位素需要由快中子(如热核武器中核聚变产生的中子)提供额外的能量才可以发生裂变。

根据原子核的液滴模型,核裂变产物应具有相同的原子量。通过更复杂的核壳层模型可以从机理上解释为何通常一种裂变产物比另一种稍小。玛丽亚·格佩特·梅耶提出了一种基于核壳层模型的裂变理论。

最常见的裂变过程是二元裂变,如上文所述,两个裂变产物的原子量通常分布在在95±15和135±15 u 区间。二元裂变发生的概率最大,而在核反应堆中,每1000次裂变事件中,还会发生2到4次三元裂变,三元裂变的过程产生三个带正电荷的碎片(加上中子),其中最小的裂变碎片的质量范围可以在质子(原子质量Z=1)至氩(原子质量Z=18)之间。最常见的小碎片由90\%的氦-4核组成,其能量高于α衰变产生的α粒子(即所谓“长程α粒子”,能量约16MeV),加上氦-6和氚核。三元裂变不太常见,但最终仍会在核反应堆的燃料棒中产生大量氦-4和气体氚。[4]
\begin{figure}[ht]
\centering
\includegraphics[width=8cm]{./figures/6ba36d8f2d7b9973.png}
\caption{U-235 、Pu-239 (当前核电反应堆中两种典型)和 U-233 (用于钍增殖循环)热中子诱导裂变产物的质量分布。} \label{fig_HLB_2}
\end{figure}

\subsubsection{1.2 能量学}
\textbf{能量输入}

