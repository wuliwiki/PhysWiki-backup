% 分离变量法解偏微分方程
% 偏微分方程|分离变量

\begin{issues}
\issueDraft
\end{issues}

在求解各种物理问题中的偏微分方程时, 我们经常使用\textbf{分离变量法}. 不夸张地说, 分离变量法在解各种物理学的偏微分方程是最重要最常用的方法. 在分离变量法中, 我们假设微分方程的解可以表示为
\begin{equation}\label{SepVar_eq2}
f(x_1, \dots , x_N) = \sum_{i_1, \dots, i_N} c_{i_1, \dots, i_N} f_{i_1}(x_1) f_{i_2}(x_2) \dots f_{i_N}(x_N)
\end{equation}
即每个变量都具有一组一元函数, 这些一元函数的乘积的线性组合可以表示方程的解. 若将该式代入偏微分方程, 可以分别得到关于每个变量 $x_i$ 的常微分方程, 我们就说这个偏微分方程式\textbf{可分离变量}的. 我们通过一个例子来说明.

% ======= 回收 ======

% 以二元偏微分方程为例,要解该方程,先假设…乘积,
% 由于等号左边只含有 ,右边只含有y,所以两边只能等于一个常数.原因如下:

% 若 是方程的一个解,则在定义域中任意改变 $x, y$ 上式都成立.现在令 不变(等号左边为常数),当y在定义域内任意变化时,由于等式必须成立,等式右边必须是同一常数.现在,当x变化是,等号左边也是同一常数.


% 这样,一个偏微分方程就变成了两个常微分方程.解这两个方程,得到偏微分方程的一个特解.

% 在大部分情况下,所有特解的线性组合就是通解….

% ==================


\subsection{弦上的驻波}

\pentry{一维波动方程\upref{WEq1D}, 多元函数的傅里叶级数\upref{NdFuri}}

一维的波动方程(\autoref{WEq1D_eq3}~\upref{WEq1D})为
\begin{equation}
\pdv[2]{x} f(x, t) - \frac{1}{v^2} \pdv[2]{t}f(x, t) = 0
\end{equation}
假设弦长为 $a$, 两端固定, 则边界条件为
\begin{equation}\label{SepVar_eq3}
f(0, t) = f(a, t) = 0
\end{equation}
显然, $f(x, t) \equiv 0$ 是方程的一个解, 代表一根静止的弦, 但这对我们并没有什么用. 那么有哪些些非零的解呢? 我们现在只会解一些常微分方程, 而 $f(x, t)$ 是一个二元函数, 让人有点无从下手, 所以我们可以先猜测某个解具有
\begin{equation}\label{SepVar_eq1}
f(x, t) = X(x) T(t)
\end{equation}
的形式, 即分别含有两个变量的两个一元函数相乘. 把\autoref{SepVar_eq1} 代入原方程\autoref{SepVar_eq2} , 得
\begin{equation}
\pdv[2]{x}[X(x)T(t)] - \frac{1}{v^2}\pdv[2]{t} [X(x) T(t)] = 0
\end{equation}
但考虑到偏微分的计算法则, 有
\begin{equation}
\pdv[2]{XT}{x} = T \dv[2]{X}{x}
\end{equation}
\begin{equation}
\pdv[2]{XT}{t} = X \dv[2]{T}{t}
\end{equation}
所以方程变为
\begin{equation}
T\dv[2]{X}{x} = \frac{X}{v^2} \dv[2]{T}{t}
\end{equation}
现在把方程两边同时除以 $f_n$, 即同时除以 $X_n T_n$, 使等式左边只是 $x$ 的函数, 右边只是 $t$ 的函数
\begin{equation}
\frac{1}{X}\dv[2]{X}{x} = \frac{1}{v^2 T} \dv[2]{T}{t}
\end{equation}
但为什么要这么做呢? 因为真的存在一个 这样的解, 在某个时刻 $t$, $X(x)$ 必定会随 $x$ 变化(因为我们要找的是非零解, 弦不可能是一条直线). 反之, 如果保持 $x$ 不变(观察弦上某个点的运动情况), $T(t)$ 也必须随 $t$ 改变(因为弦上的任意一点一般会做某种运动, 例如振动, 不可能都静止). 把这样的推理用到上式, 就会得出, 唯一可能让等式成立的方法是方程两边分别等于一个常数, 因为 $x$ 不变时方程左边是常数,  $t$ 不变时方程右边是常数. 若设这个常数为(一个任意实数, 这么表示是为了下文书写方便), 得
\begin{equation}
\frac{1}{X}\dv[2]{X}{x} = \pm k^2
\qquad
\frac{1}{v^2 T}\dv[2]{T}{t} = \pm k^2
\end{equation}
或
\begin{equation}
\dv[2]{X}{x} mk^2 X = 0
\qquad
\dv[2]{T_n}{t} mv^2k^2 T = 0
\end{equation}
这两条都是一维齐次亥姆霍兹方程, 常数为 时, 方程的解为指数函数
\begin{align}
X &= C_1 \E^{kx} + C_2 \E^{-kx}\\
T &= D_1 \E^{kvt} + D_2 \E^{-kvt}
\end{align}
$f = XT$ 不可能满足边界条件\autoref{SepVar_eq3}  (但有可能满足其他边界条件). 当常数为 $-k^2$ 时, 方程的解为
\begin{align}\label{SepVar_eq4}
X &= C_1 \cos(kx) + C_2 \sin(kx)\\
\label{SepVar_eq5}T &= D_1 \cos(kvt) + D_2 \sin(kvt)
\end{align}
要使 $t$ 取任何值时 $f = XT$ 都满足边界条件, 只能令 $X$ 满足 $X(0) = 0$ 且 $X(a) = 0$. 前者代入\autoref{SepVar_eq4} 得 $C_1 = 0$, 所以 $X = C_2\sin(kx)$. 再代入后者得 $C_2\sin(ka) = 0$. 为了获得非零解, 我们不可能让 $C_1, C_2$ 都为零, 所以只能令 $\sin(ka) = 0$, 解出常数 $k_n = n\pi/a$ 只能取离散的值, $n$ 为任意正整数.

现在, 我们可以令整数 $n$ 取任意值, 得到无穷多个方程的解
\begin{equation}
f_n(x, t) = X_n T_n = C_{1n} \sin(\frac{n\pi}{a}x)\cos(\frac{n\pi}{a} vt) + C_{2n}\sin(\frac{n\pi}{a}x)\sin(\frac{n\pi}{a}vt)
\end{equation}
\autoref{SepVar_eq5} 也可以记为 $D\cos(kvt+\phi_0)$, 所以上式也可记为
\begin{equation}
f(x, t) = XT = C_n\sin(\frac{n\pi}{a}x)\cos(\frac{n\pi}{a}vt + \phi_0)
\end{equation}

