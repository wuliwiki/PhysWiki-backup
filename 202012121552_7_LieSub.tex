% 李代数的子代数、理想与商代数
% keys 李代数|子代数|理想|正规化子




\pentry{李代数\upref{LieAlg}}
和抽象代数中的子群、子环等类比,李代数也可以有次级结构,即子代数.

设$\mathfrak{m}, \mathfrak{n}$是李代数$\mathfrak{g}$的非空子集,定义子集间的运算为$\mathfrak{m}+\mathfrak{n}=\{M+N|M\in\mathfrak{m}, N\in\mathfrak{n}\}$,以及$[\mathfrak{m}, \mathfrak{n}]=\{[M, N]|M\in\mathfrak{m}, N\in\mathfrak{n}\}$.那么如果$\mathfrak{m}, \mathfrak{n}$和$\mathfrak{p}$都是$\mathfrak{g}$作为线性空间的子空间,我们容易证明以下性质:

\begin{itemize}
\item $[\mathfrak{m}+\mathfrak{n}, \mathfrak{p}]\subseteq[\mathfrak{m}, \mathfrak{p}]+[\mathfrak{n}, \mathfrak{p}]$;
\item $[\mathfrak{m},\mathfrak{n}]=[\mathfrak{n}, \mathfrak{m}]$;
\item $[\mathfrak{m}, [\mathfrak{n}, \mathfrak{p}]]\subseteq[\mathfrak{n}, [\mathfrak{m}, \mathfrak{p}]]+[\mathfrak{p}, [\mathfrak{n}, \mathfrak{m}]]$.
\end{itemize}



