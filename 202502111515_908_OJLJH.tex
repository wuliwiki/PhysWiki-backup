% 欧几里得几何(综述)
% license CCBYSA3
% type Wiki

本文根据 CC-BY-SA 协议转载翻译自维基百科\href{https://en.wikipedia.org/wiki/Euclidean_geometry#}{相关文章}。

\begin{figure}[ht]
\centering
\includegraphics[width=6cm]{./figures/29202c1c81a65fe8.png}
\caption{拉斐尔的《雅典学派》中的细节,展示了一位希腊数学家——可能代表欧几里得或阿基米德——正在使用圆规绘制几何构图。} \label{fig_OJLJH_1}
\end{figure}
欧几里得几何是归功于古希腊数学家欧几里得的数学体系,他在其几何学教材《几何原本》中对其进行了描述。欧几里得的方法是假设一小组直观上令人信服的公设(公理),并从这些公理中推导出许多其他命题(定理)。尽管欧几里得的许多结果早已被提出,[1] 但他是第一个将这些命题组织成一个逻辑系统的人,其中每个结果都是从公理和先前证明的定理推导出来的。[2]

《几何原本》以平面几何开始,至今仍在中学(高中)教授,作为第一个公理化系统和数学证明的初步例子。它接着讲解了三维的立体几何。《几何原本》中的许多内容阐述了现在被称为代数和数论的结果,用几何语言来表达。[1]

在超过两千年的时间里,“欧几里得”这个形容词是不必要的,因为欧几里得的公理似乎是如此直观明显(平行公设可能是唯一例外),以至于从这些公理中推导出的定理被认为是绝对正确的,因此没有其他类型的几何被认为是可能的。然而,今天许多自洽的非欧几里得几何已被发现,最早的几何形式是在19世纪初发现的。爱因斯坦的广义相对论理论的一个含义是,物理空间本身并非欧几里得空间,欧几里得空间仅在短距离内(相对于引力场的强度)对其进行良好近似。[3]

超过两千年来,“欧几里得”这个形容词并不必要,因为欧几里得的公理看起来非常直观明显(平行公设可能是个例外),从这些公理推导出来的定理被认为是绝对正确的,因此没有其他形式的几何被认为是可能的。然而,今天我们知道许多其他自洽的非欧几里得几何,最早的发现是在19世纪初。阿尔伯特·爱因斯坦的广义相对论理论的一个含义是,物理空间本身并非欧几里得的,欧几里得空间只有在短距离(相对于引力场的强度)内才是一个很好的近似。

欧几里得几何是合成几何的一个例子,因为它从描述几何对象(如点和线)的基本属性的公理出发,逻辑地推导出关于这些对象的命题。这与近两千年后由勒内·笛卡尔引入的解析几何形成对比,后者通过坐标使用代数公式来表达几何属性。
\subsection{《几何原本》}
《几何原本》主要是对早期几何知识的系统化。它在比早期的几何处理方法中取得了显著的进步,这一点很快被认可,因此几乎没有人再对保留早期的几何作品感兴趣,而这些作品如今几乎都已遗失。

《几何原本》共包含13卷:

第I至IV卷和第VI卷讨论平面几何。许多关于平面图形的定理被证明,例如“任意三角形中,任意两个角之和小于两个直角。”(第I卷命题17)以及“在直角三角形中,斜边的平方等于两直角边的平方和。”(第I卷命题47)

第V卷和第VII至X卷涉及数论,数被几何地处理,作为线段的长度或平面区域的面积。介绍了素数、有理数和无理数等概念。并且证明了素数的个数是无限的。

第XI至XIII卷涉及立体几何。一个典型的定理是圆锥体和底面和高度相同的圆柱体的体积比为1:3。并且构造了柏拉图立体。
\subsubsection{公理}
\begin{figure}[ht]
\centering
\includegraphics[width=6cm]{./figures/8f816c81e5e3a080.png}
\caption{平行公设(公设 5):如果两条直线与第三条直线相交,并且在一侧的内角之和小于两个直角,那么这两条直线如果延伸足够远,必定会在该侧相交。} \label{fig_OJLJH_2}
\end{figure}
欧几里得几何是一种公理化系统,其中所有定理(“真命题”)都从少数简单的公理推导出来。在非欧几里得几何出现之前,这些公理被认为在物理世界中显而易见,因此所有的定理也都被认为是同样真实的。然而,欧几里得从假设到结论的推理依然独立于物理现实有效。[4]

在《几何原本》第一卷的开头,欧几里得给出了五个平面几何公设(公理),用构造的方式来表述(根据托马斯·希斯的翻译):[5]

假设如下:
\begin{enumerate}
\item 从任意一点到任意一点画一条直线。
\item 将一条有限的直线继续延伸成一条直线。
\item 以任意中心和任意距离(半径)画一个圆。
\item 所有的直角都相等。
\item [平行公设]:如果一条直线与两条直线相交,并且在同一侧形成的内角小于两个直角,那么这两条直线如果无限延伸,必定会在角小于两个直角的那一侧相交。
\end{enumerate}

尽管欧几里得明确地只主张构造物体的存在,但在他的推理中,他也隐含地假设这些物体是唯一的。

《几何原本》还包括以下五个“共通命题”:
\begin{enumerate}
\item 与同一事物相等的事物也彼此相等(欧几里得关系的传递性)。
\item 如果相等的东西相加,那么整体也相等(相等加法公理)。
\item 如果相等的东西相减,那么差也相等(相等减法公理)。
\item 与彼此重合的事物相等(反身性公理)。
\item 整体大于部分。
\end{enumerate}
现代学者一致认为,欧几里得的公设并未提供欧几里得为其论述所要求的完整逻辑基础。[6] 现代的研究使用了更广泛和完整的公理集合。
\subsubsection{平行公设}  
对古代人来说,平行公设似乎不像其他公设那样显而易见。古人力图建立一个绝对确定的命题体系,而平行线公设似乎需要从更简单的命题中证明出来。现在我们知道,这种证明是不可能的,因为可以构造出一致的几何系统(遵循其他公理),其中平行公设为真,也可以构造出平行公设为假的系统。[7] 欧几里得本人似乎认为平行公设与其他公设在性质上有所不同,这一点可以从《几何原本》的组织方式中看出:他前28个命题是那些可以在没有平行公设的情况下证明的。

许多可以与平行公设逻辑等价的替代公理可以被提出(在其他公理的背景下)。例如,普莱费尔公理指出:

在平面上,经过一个不在给定直线上的点,最多只能画一条与给定直线永不相交的直线。  
“最多”这一条件是必要的,因为可以从其余公理中证明,至少存在一条平行线。
\subsubsection{证明方法} 
\begin{figure}[ht]
\centering
\includegraphics[width=6cm]{./figures/a7ea5743fea08888.png}
\caption{给定一条线段,可以构造一个包含该线段作为其中一边的等边三角形:通过在点 Α 和 Β 上分别以 Δ 和 Ε 为圆心画圆,并取两个圆的交点之一作为三角形的第三个顶点,从而构造出一个等边三角形 ΑΒΓ。} \label{fig_OJLJH_3}
\end{figure}
欧几里得几何是构造性的。公设 1、2、3 和 5 断言了某些几何图形的存在性和唯一性,这些断言具有构造性:也就是说,我们不仅被告知某些事物存在,而且还给出了用圆规和无标尺直尺来构造它们的方法。在这一意义上,欧几里得几何比许多现代公理化系统(如集合论)更加具体,因为后者通常仅仅断言对象的存在,而没有说明如何构造它们,甚至有时断言某些无法在该理论内构造的对象的存在。严格来说,纸上的直线是形式系统中定义的对象的模型,而不是这些对象的实例。例如,欧几里得的直线没有宽度,但任何实际绘制的直线都有宽度。尽管几乎所有现代数学家认为非构造性证明与构造性证明一样有效,但它们通常被认为不如构造性证明优雅、直观或在实践中有用。欧几里得的构造性证明常常取代了错误的非构造性证明,例如一些假设所有数都是有理数的毕达哥拉斯证明,通常需要像“找出…的最大公约数”这样的语句。

欧几里得常常使用反证法。
\subsection{符号和术语}  
\subsubsection{点和图形的命名}  
点通常使用字母表的大写字母来命名。其他图形,如直线、三角形或圆,通常通过列出足够数量的点来命名,以便从相关图形中明确区分它们。例如,三角形 ABC 通常指的是一个顶点分别位于点 A、B 和 C 的三角形。
\subsubsection{互补角和补角}  
角度和为直角的角叫做互补角。互补角形成于当一条射线与原有两条射线共享同一个顶点,并且指向位于两条原射线之间的方向时。两条原射线之间的射线数量是无限的。

角度和为平角的角叫做补角。补角形成于当一条射线与原有两条射线共享同一个顶点,并且指向位于两条原射线之间的方向时,这两条原射线形成一个平角(180度角)。两条原射线之间的射线数量也是无限的。
\subsection{欧几里得符号的现代版本}  
在现代术语中,角度通常以度数或弧度来衡量。

现代的教科书通常定义了不同的图形,称为直线(无限长)、射线(半无限长)和线段(有限长度)。欧几里得并不像现代那样将射线视为一个在一个方向上延伸至无限的对象,他通常会使用类似“如果直线延长到足够的长度”这样的表达方式,尽管他偶尔也提到“无限直线”。对于欧几里得来说,“直线”可以是直的,也可以是曲的,必要时他会使用更具体的术语“直线”。
\subsection{一些重要或著名的结果}
\begin{figure}[ht]
\centering
\includegraphics[width=6cm]{./figures/31b84b2faa941374.png}
\caption{驴桥定理(Pons Asinorum)指出,在一个等腰三角形中,角 α = 角 β 且角 γ = 角 δ。} \label{fig_OJLJH_4}
\end{figure}
\begin{figure}[ht]
\centering
\includegraphics[width=6cm]{./figures/230550cbb7372788.png}
\caption{“三角形内角和定理指出,任何三角形的三个角的和,无论是角α、β还是γ,总是等于180度。”} \label{fig_OJLJH_5}
\end{figure}
\begin{figure}[ht]
\centering
\includegraphics[width=6cm]{./figures/be3cecb54b8c3ef1.png}
\caption{“勾股定理指出,直角三角形两条直角边(a和b)上方的两个正方形的面积之和等于斜边(c)上方正方形的面积。”} \label{fig_OJLJH_6}
\end{figure}
\begin{figure}[ht]
\centering
\includegraphics[width=6cm]{./figures/ff2445419aa66ec9.png}
\caption{泰勒斯定理指出,如果AC是直径,那么角B是直角。} \label{fig_OJLJH_7}
\end{figure}
\subsubsection{驴桥定理}  
驴桥定理(Pons asinorum)指出,在等腰三角形中,底角相等,并且如果相等的直线延长,底下的角也相等。[12] 这个名字可能来源于它在《几何原本》中经常作为检验读者智慧的第一个真正考验,以及作为通向后面更难命题的桥梁。也有可能因为这个几何图形像一座陡峭的桥,只有脚步稳健的驴子才能通过,故得此名。[13]”