% 计算机史(综述)
% license CCBYSA3
% type Wiki

本文根据 CC-BY-SA 协议转载翻译自维基百科\href{https://en.wikipedia.org/wiki/History_of_computing}{相关文章}。

计算历史的时间比计算硬件和现代计算技术的历史更为悠久,其中包括了为笔和纸或黑板和粉笔而设计的方法的历史,无论是否借助表格的辅助。
\subsection{具体设备}  
数字计算与数字的表示密切相关。[1] 但在像数字这样的抽象概念出现之前,已经有了为文明服务的数学概念。这些概念隐含于具体的实践中,例如:
\begin{itemize}
\item 一一对应,[2] 一种用于计数物品数量的规则,例如在计数棒上,最终被抽象为数字。
\item 与标准比较,[3] 一种假定测量可重复性的方法,例如,硬币的数量。
\item 3-4-5 直角三角形是一种确保直角的设备,使用带有 12 个均匀间距结的绳子,例如。[4][未验证]
\end{itemize}
\subsection{数字}  
最终,数字的概念变得具体且熟悉,足以用于计数,有时还伴随着歌谣式的记忆法来教别人记住数列。所有已知的人类语言,除了皮拉哈语(Piraha),都有表示至少“一个”和“两个”的词汇,甚至一些动物,如黑鸟,也能够区分出令人惊讶的物品数量。[5]

数字系统和数学符号的进步最终导致了数学运算的发现,例如加法、减法、乘法、除法、平方、平方根等。最终,这些运算被形式化,关于这些运算的概念也得到了足够清晰的理解,可以被正式表述,甚至被证明。例如,可以参考欧几里得算法,用于找出两个数的最大公约数。

到了中世纪晚期,位值的印度-阿拉伯数字系统传入了欧洲,这使得数字的系统化计算成为可能。在这一时期,计算在纸上表示出来,允许数学表达式的计算,并且能够列出数学函数,如平方根、常用对数(用于乘法和除法),以及三角函数。到艾萨克·牛顿的研究时期,纸张或羊皮纸已成为重要的计算资源,甚至在我们今天的时代,像恩里科·费米这样的研究人员会在随机的纸片上进行计算,来满足他们对方程的好奇心。[6] 甚至在可编程计算器的时代,理查德·费曼也毫不犹豫地手动计算任何超出计算器内存限制的步骤,只为了解答案;到1976年,费曼购买了一台HP-25计算器,具有49步程序容量;如果微分方程需要超过49步来解答,他就继续手工计算。[7]
\subsection{早期计算}  
数学陈述不必仅限于抽象;当一个陈述可以用实际数字加以说明时,这些数字就可以被传递,从而形成一个社区。这使得可重复、可验证的陈述成为数学和科学的标志。这类陈述存在了数千年,并且跨越了多个文明,如下所示:

已知最早用于计算的工具是苏美尔的算盘,据信它是在公元前2700年至2300年左右的巴比伦发明的。其最初的使用方式是通过在沙地上划线并使用小石子。[需要引用]

公元前1050至771年左右,古代中国发明了指向南方的战车。这是已知的第一个使用差动齿轮的齿轮机制,后来被用于模拟计算机。中国人还发明了一种更为复杂的算盘,大约在公元前2世纪左右,称为中国算盘。[需要引用]

公元前3世纪,阿基米德使用平衡的机械原理(参见《阿基米德的手稿》 § 机械定理法)来解决数学问题,例如计算宇宙中的沙粒数量(《沙粒计算器》),这也需要一种递归的数字表示法(例如,万万)。  

安提基特拉机制被认为是已知最早的齿轮计算设备。它被设计用来计算天文位置。它于1901年在希腊安提基特拉岛附近的沉船中被发现,估计制作于公元前100年左右。[8]

根据西蒙·辛格的说法,穆斯林数学家在密码学方面也做出了重要的进展,如阿尔金都斯(Alkindus)发展了密码分析和频率分析。[9][10] 穆斯林工程师还发明了可编程机器,如班努·穆萨兄弟的自动笛子演奏机。[11]

在中世纪,几位欧洲哲学家试图制造模拟计算设备。在阿拉伯人和经院哲学的影响下,马略卡哲学家拉蒙·吕尔(Ramon Llull,1232–1315)将他的一生大部分时间都投入到定义和设计几种逻辑机器,这些机器通过结合简单且无可否认的哲学真理,可以产生所有可能的知识。这些机器实际上并未建造出来,因为它们更多的是一种思想实验,旨在以系统化的方式产生新知识;尽管它们可以进行简单的逻辑操作,但仍需要人工来解读结果。此外,它们缺乏多功能的架构,每个机器只能用于非常具体的目的。尽管如此,吕尔的工作对戈特弗里德·莱布尼茨(18世纪初)产生了强烈影响,后者进一步发展了他的思想,并根据这些思想制造了几种计算工具。

这一早期机械计算时代的顶峰可以从查尔斯·巴贝奇的差分机及其后继的解析机中看到。巴贝奇虽然未能完成这两台机器的建造,但在2002年,伦敦科学博物馆的多伦·斯韦德(Doron Swade)和其他工程师们使用1840年代可得的材料,完成了巴贝奇的差分机。[12] 通过遵循巴贝奇的详细设计,他们成功地建造了一个能够运行的机器,使得历史学家可以有信心地说,如果巴贝奇能够完成他的差分机,它本应该是可以工作的。[13] 更加先进的解析机结合了他之前工作的概念以及其他人的研究成果,创造了一种设备,如果按照设计建造,便会具备许多现代电子计算机的特性,例如等同于RAM的内部“暂存记忆”、多种输出形式,包括铃声、图形绘制仪和简单打印机,及可编程的输入输出“硬”记忆(即穿孔卡片),它不仅能够读取信息,还可以修改这些信息。巴贝奇的设备超越前人之处在于,每个组件都是独立于其他部分的,就像现代电子计算机的各个组件一样。这是一种思维上的根本转变;之前的计算设备只能服务于单一的目的,且在解决新问题时,最好的做法是拆解并重新配置机器。巴贝奇的设备则可以通过输入新数据重新编程来解决新问题,并且在同一指令序列中对之前的计算进行操作。阿达·洛夫莱斯将这一概念进一步发展,她为解析机编写了一个程序来计算伯努利数,这一复杂的计算需要递归算法。这被认为是第一个真正的计算机程序的例子,一个指令序列,在程序运行之前并不完全知道操作的数据。

在巴贝奇之后,虽然并不知晓他早期的工作,珀西·拉德盖特(Percy Ludgate)于1909年发布了历史上仅有的两个机械解析机设计中的第二个。[14][15] 另外两位发明家,莱昂纳多·托雷斯·凯维多(Leonardo Torres Quevedo)[17] 和范尼瓦尔·布什(Vannevar Bush)[18],也基于巴贝奇的工作进行了后续研究。在他的《自动化论文集》(Essays on Automatics,1914年)中,托雷斯提出了一种机电计算机的设计,并引入了浮动点算术的概念。[19][20] 1920年,为了庆祝算术仪发明100周年,托雷斯在巴黎展示了机电算术仪,它是一种算术单元,可以连接远程打字机,用户可以输入命令,结果会自动打印出来。[21][22] 布什的论文《仪器分析》(Instrumental Analysis,1936年)讨论了如何利用现有的IBM穿孔卡机器来实现巴贝奇的设计。同年,他开始了快速算术机器项目,研究构建电子数字计算机的问题。

一些模拟计算的例子一直延续到最近时代。示意计(planimeter)是一种使用距离作为模拟量来进行积分的设备。直到1980年代,暖通空调(HVAC)系统仍然使用空气作为模拟量和控制元素。与现代数字计算机不同,模拟计算机的灵活性较差,需要手动重新配置(即重新编程)才能将其从解决一个问题切换到另一个问题。模拟计算机相较于早期的数字计算机有一个优势,它们能够利用行为类比来解决复杂问题,而最早的数字计算机尝试则相当有限。
\begin{figure}[ht]
\centering
\includegraphics[width=6cm]{./figures/542f2c87bce40253.png}
\caption{史密斯图(Smith Chart)是一种著名的标度图(nomogram)。} \label{fig_JSJS_1}
\end{figure}
由于在那个时代计算机非常稀缺,解决方案通常被硬编码到纸质表单中,如标度图(nomograms),这些表单可以为这些问题提供模拟解答,例如加热系统中压力和温度的分布。
\subsection{数字电子计算机}   
‘大脑’[计算机]或许有一天会走到我们普通人的身边,帮助我们计算所得税和做账。但这只是猜测,到目前为止还没有任何迹象表明这一点。

—— 1949年6月《星报》关于EDSAC计算机的新闻文章,远在个人计算机时代之前。
 
在1886年的一封信中,查尔斯·桑德斯·皮尔斯描述了如何通过电气开关电路进行逻辑操作。[25] 在1880到1881年期间,他展示了仅使用NOR门(或仅使用NAND门)可以重现所有其他逻辑门的功能,但这项工作直到1933年才出版。[26] 第一个已发布的证明是亨利·M·谢弗在1913年所做的,因此NAND逻辑运算有时被称为谢弗符号(Sheffer stroke);而逻辑NOR有时被称为皮尔斯箭头(Peirce's arrow)。[27] 因此,这些门有时被称为通用逻辑门。[28]  

最终,真空管取代了继电器来进行逻辑操作。1907年,李·德·福雷斯特对弗莱明阀的改进可以用作逻辑门。路德维希·维特根斯坦在《逻辑哲学论》(1921年)中引入了16行真值表版本,作为命题5.101。沃尔特·博特,巧合电路的发明者,因1924年发明的第一个现代电子与门电路获得了1954年诺贝尔物理学奖的一部分。康拉德·楚泽为他的计算机Z1(从1935年到1938年)设计并建造了机电逻辑门。  

使用数字电子进行计算的第一个记录想法出现在1931年C·E·温·威廉姆斯的论文《高速度自动计数物理现象的使用阀门》[29]。从1934年到1936年,NEC工程师中岛明、克劳德·香农和维克托·谢斯塔科夫发布了论文,介绍了开关电路理论,利用数字电子进行布尔代数运算。[30][31][32][33]  

1936年,艾伦·图灵发表了他的重要论文《可计算数及其在决策问题中的应用》[34],他用一维存储带的模型描述了计算,提出了图灵机的概念,并提出了图灵完备系统的思想。[需要引用]  

第一台数字电子计算机是在1936年4月到1939年6月间,由阿瑟·霍尔西·迪金森在IBM专利部(位于纽约恩迪科特)开发的。[35][36][37] 在这台计算机中,IBM推出了一种带有键盘、处理器和电子输出(显示)的计算设备。IBM的竞争者是1939年4月到1939年8月,由约瑟夫·德什和罗伯特·穆马在NCR公司(位于俄亥俄州代顿市)开发的数字电子计算机NCR3566。[38][39] IBM和NCR的机器是十进制的,使用二进制位置码执行加法和减法运算。  

1939年12月,约翰·阿塔纳索夫和克利福德·贝瑞完成了他们的实验模型,证明了阿塔纳索夫-贝瑞计算机(ABC)的概念,该计算机于1937年开始开发。[40] 这个实验模型是二进制的,使用八进制二进制码执行加法和减法运算,是第一台二进制数字电子计算设备。阿塔纳索夫-贝瑞计算机的目标是解决线性方程组,尽管它不是可编程的。由于阿塔纳索夫于1942年离开爱荷华州立大学前往美国海军工作,这台计算机并未完全完成。[41][42] 许多人将ABC视为早期电子计算机发展的许多想法的源头。[43]  

1941年,德国发明家康拉德·楚泽设计并建造了Z3计算机,它是第一台可编程、全自动计算机,但它不是电子的。  

在第二次世界大战期间,弹道计算由女性完成,她们被雇用为“计算员”。直到1945年,“计算员”这一术语仍然主要指代女性(现在称为“操作员”),之后它才获得了现代机械定义。[44]  

ENIAC(电子数值积分计算机)是第一台电子通用计算机,于1946年向公众发布。它是图灵完备的,[45] 数字的,并且能够通过重新编程解决各种计算问题。像ENIAC这样的机器的编程工作由女性实现,而硬件则由男性开发。[44]  

曼彻斯特宝宝是第一台电子存储程序计算机。它是在曼彻斯特大学由弗雷德里克·C·威廉姆斯、汤姆·基尔本和杰夫·图蒂尔建造的,并于1948年6月21日运行了它的第一个程序。[46]  

威廉·肖克利、约翰·巴尔丁和沃尔特·布拉顿于1947年在贝尔实验室发明了第一台工作晶体管——点接触晶体管,随后在1948年发明了双极结晶体管。[47][48] 1953年,在曼彻斯特大学,汤姆·基尔本领导的团队设计并建造了第一台晶体管计算机——晶体管计算机,这台机器使用新开发的晶体管而不是电子管。[49] 第一台存储程序晶体管计算机是由日本电气技术实验室(ETL)开发的Mark III,于1954年到1956年间建成。[50][51][52] 然而,早期的结型晶体管相对笨重且难以大规模生产,这限制了它们仅能应用于一些特定领域。[54]  

1954年,95\%的计算机用于工程和科学目的。[55]
\subsubsection{个人计算机}  
金属–氧化物–硅场效应晶体管(MOSFET),也称为MOS晶体管,是在1955到1960年间由贝尔实验室发明的。[56][57][58][59][60][61][62] 它是第一款真正紧凑的晶体管,可以进行小型化和大规模生产,广泛应用于多种领域。[54] MOSFET使得高密度集成电路芯片的制造成为可能。[63][64] MOSFET是计算机中使用最广泛的晶体管,[65][66] 也是数字电子学的基本构建块。[67]

硅栅MOS集成电路是在1968年由费德里科·法金(Federico Faggin)在费尔柴尔德半导体公司(Fairchild Semiconductor)开发的。[68] 这一进展促成了第一款单芯片微处理器——英特尔4004的诞生。[69] 英特尔4004从1969年到1970年由英特尔的费德里科·法金、马尔西亚·霍夫(Marcian Hoff)、斯坦利·梅佐尔(Stanley Mazor)以及武藏小山(Busicom)的岛正俊(Masatoshi Shima)领导开发。[70] 该芯片主要由法金设计和实现,采用了他所开发的硅栅MOS技术。[69] 这一微处理器催生了微计算机革命,进而发展出了后来被称为个人计算机(PC)的微计算机。

大多数早期的微处理器,如英特尔8008和英特尔8080,都是8位的。德州仪器于1976年6月发布了第一款完整16位微处理器——TMS9900处理器。[71] 该微处理器被用于TI-99/4和TI-99/4A计算机中。

1980年代,微处理器技术取得了显著进展,对工程学和其他科学领域产生了深远影响。摩托罗拉68000微处理器的处理速度远远优于当时使用的其他微处理器。因此,拥有更新、更快的微处理器使得之后出现的微计算机更加高效,能够处理更多的计算任务。这在1983年发布的苹果Lisa计算机中得到了体现。Lisa是其中一款首批具备图形用户界面(GUI)的商业化个人计算机。它搭载了摩托罗拉68000 CPU,使用了双软盘驱动器和一个5MB的硬盘进行存储。该机器还配备了1MB的RAM,用于从磁盘上运行软件而不需要反复读取磁盘。[72] 然而,由于销售失败,苹果随后发布了第一款Macintosh计算机,依然采用摩托罗拉68000微处理器,但只有128KB的RAM,一台软盘驱动器,且没有硬盘,以降低价格。

在1980年代末期和1990年代初期,计算机开始在个人和工作用途上变得更加实用,例如文字处理。[73] 1989年,苹果发布了Macintosh Portable,它重7.3公斤(16磅),且非常昂贵,售价为7300美元。发布时,它是最强大的笔记本计算机之一,但由于价格高昂和重量大,市场反响并不热烈,仅两年后便停产。同年,英特尔推出了Touchstone Delta超级计算机,该计算机拥有512个微处理器。这一技术进步具有重要意义,因为它成为了世界上一些最快的多处理器系统的模型,并且被加州理工学院(Caltech)的研究人员用作原型,进行实时卫星图像处理和模拟分子模型等项目的研究。
\subsubsection{超级计算机}  
在超级计算领域,首个广泛认可的超级计算机是1964年由塞门·克雷(Seymour Cray)设计和制造的控制数据公司(CDC)6600。[74] 其最大速度为40 MHz,或每秒3百万次浮点运算(FLOPS)。CDC 6600于1969年被CDC 7600取代;[75] 尽管其正常时钟速度不比6600更快,但7600的峰值时钟速度约为6600的30倍,因此仍然更快。尽管CDC在超级计算机领域处于领先地位,但与塞门·克雷的关系(已经开始恶化)最终完全崩溃。1972年,克雷离开了CDC,创办了自己的公司——克雷研究公司(Cray Research Inc.)。[76] 在华尔街投资者的支持下,并且没有了在CDC时的种种限制,克雷制造了Cray-1超级计算机。其时钟速度为80 MHz,或136百万次FLOPS,克雷因此在计算机界声名鹊起。到1982年,克雷研究公司推出了具备多处理能力的Cray X-MP,并于1985年发布了Cray-2,继续推动多处理趋势,并且时钟速度为1.9吉FLOPS。克雷研究公司于1988年开发了Cray Y-MP,但此后面临着继续生产超级计算机的困境。这主要是因为冷战的结束,大学和政府对尖端计算的需求急剧下降,而对微处理单元的需求则大幅上升。

1998年,David Bader使用商用零件开发了第一台Linux超级计算机。[77] 在新墨西哥大学时,Bader希望构建一台运行Linux的超级计算机,使用商用现货零件和高速低延迟的互联网络。原型机采用了Alta Technologies的“AltaCluster”,由八台双核心、333 MHz的英特尔奔腾II计算机组成,运行经过修改的Linux内核。Bader将大量软件移植到Linux上,提供对所需组件的支持,并将国家计算科学联盟(NCSA)成员的代码移植过来,以确保互操作性,因为之前这些代码从未在Linux上运行过。[78] 使用成功的原型设计,他领导开发了“RoadRunner”,这是第一台为国家科学和工程界开放使用的Linux超级计算机,运行在国家科学基金会的国家技术网格上。RoadRunner于1999年4月投入生产使用。当时,它被认为是全球100台最快的超级计算机之一。[78][79] 尽管在Bader的原型和RoadRunner开发之前,基于Linux的集群(如Beowulf)就已经使用商用零件存在,但它们缺乏扩展性、带宽和并行计算能力,因此无法被视为“真正的”超级计算机。[78]

今天,超级计算机仍然被世界各国政府和教育机构用于进行如自然灾害模拟、与疾病相关的种群遗传变异搜索等计算任务。截至2023年4月,全球最快的超级计算机是Frontier。
\subsection{导航与天文学}  
从已知的特殊情况开始,计算对数和三角函数可以通过查阅数学表格中的数字,并在已知的情况之间插值来完成。对于足够小的差异,这种线性操作在探索时代的导航和天文学中足够准确。插值的使用在过去的500年里得到了广泛应用:到20世纪,Leslie Comrie和W.J. Eckert将插值法系统化,应用于数字表格中的数据,供打孔卡片计算使用。
\subsection{天气预测}  
求解微分方程的数值解,特别是Navier-Stokes方程,是推动计算机发展的重要动力之一,尤其是Lewis Fry Richardson采用数值方法求解微分方程。1950年,首个计算机天气预报由一支团队完成,该团队由美国气象学家Jule Charney、Philip Duncan Thompson、Larry Gates、挪威气象学家Ragnar Fjørtoft、应用数学家John von Neumann以及ENIAC程序员Klara Dan von Neumann组成。[80][81][82] 直到今天,地球上一些最强大的计算机系统仍然被用于天气预报。[83]
\subsection{符号计算}  
到1960年代末,计算机系统已经足够能够执行符号代数操作,甚至能够通过大学级微积分课程的测试。[citation needed]
\subsection{重要女性及其贡献}  
与男性同行相比,女性在STEM(科学、技术、工程和数学)领域往往代表性不足。[84] 在1960年代之前,计算通常被视为“女性的工作”,因为它与制表机操作和其他机械化的办公工作相关联。[85][86] 这种看法的准确性因地区而异。在美国,Margaret Hamilton 回忆起一个由男性主导的环境,[87] 而Elsie Shutt 回忆起她惊讶地发现雷神公司(Raytheon)中有一半的计算机操作员是男性。[88] 在英国,直到1970年代初,机器操作员大多是女性。[89] 随着这些观念的变化,计算领域成为一种高地位的职业,男性在该领域的主导地位逐渐加强。[90][91][92] Janet Abbate教授在她的书《Recoding Gender》中写道:

然而,女性在计算机学科的早期几十年里占据了重要地位。她们在第二次世界大战期间构成了第一批计算机程序员的主体;她们在早期计算机行业中担任了重要的责任和影响职位;她们的就业人数虽然占总数的少数,但与许多其他科学和工程领域女性的代表性相比,仍然表现良好。20世纪50至60年代的一些女性程序员可能会嘲笑编程永远不会被视为男性的职业,然而这些女性的经历和贡献很快就被遗忘了。[93]

计算机历史中的一些著名女性例子包括:
\begin{itemize}
\item \textbf{Ada Lovelace}:为巴贝奇的分析机写下附录,详细描述了第一个计算机算法,诗意地阐述了分析机如何根据设计原理工作。
\item \textbf{Grace Murray Hopper}:计算机领域的先驱之一,她与霍华德·A·艾肯(Howard H. Aiken)一起工作,参与了IBM的Mark I的开发。Hopper还创造了“调试”(debugging)这一术语。
\item \textbf{Hedy Lamarr}:发明了“频率跳变”技术,海军在第二次世界大战期间利用该技术通过无线电信号控制鱼雷。这项技术今天仍然被用于蓝牙和Wi-Fi信号的创建。
\item \textbf{Frances Elizabeth "Betty" Holberton}:发明了“断点”(breakpoints),即在计算机代码中设置的迷你暂停,帮助程序员轻松检测、故障排除和解决问题。
\item \textbf{最初编程ENIAC的女性}:Kay McNulty, Betty Jennings, Marlyn Meltzer, Fran Bilas, Ruth Lichterman 和 Betty Holberton(见上文)。
\item \textbf{Jean E. Sammet}:共同设计了COBOL,这是一种广泛使用的编程语言。
\item \textbf{Frances Allen}:计算机科学家,优化编译器领域的先驱,第一位获得图灵奖的女性。
\item \textbf{Karen Spärck Jones}:负责“逆文档频率”(inverse document frequency)这一概念,这一概念最常被搜索引擎使用。
\item \textbf{Dana Angluin}:对计算学习理论作出了基础性贡献。
\item \textbf{Margaret Hamilton}:麻省理工学院软件工程部门的主任,该部门开发了阿波罗太空任务的航天软件。
\item \textbf{Barbara Liskov}:开发了“Liskov替代原则”。
\item \textbf{Radia Perlman}:发明了“生成树协议”(Spanning Tree Protocol),这一关键的网络协议广泛应用于以太网网络。
\item \textbf{Stephanie "Steve" Shirley}:创办了F International,这是一家非常成功的自由职业软件公司。
\item \textbf{Sophie Wilson}:帮助设计了ARM处理器架构,该架构广泛应用于智能手机和视频游戏等许多产品中。
\item \textbf{Ann Hardy}:开创了计算机时间共享系统。
\item \textbf{Lynn Conway}:通过共同引入结构化VLSI设计等发明,革命性地改变了微芯片的设计和生产。
\item \textbf{Bletchley Park的女性}:大约8,000名女性在第二次世界大战期间以多种身份参与英国的密码分析工作。许多女性来自英国皇家海军服务队(被称为“wrens”)和女子辅助空军("WAAFs")。她们在破解“恩尼格码”密码和帮助盟军赢得战争方面发挥了重要作用。
\end{itemize}
\subsection{参考文献}
\begin{enumerate}
\item "Digital Computing - Dictionary definition of Digital Computing | Encyclopedia.com: FREE online dictionary". www.encyclopedia.com. 检索于2017-09-11。
\item "One-to-One Correspondence: 0.5". 维多利亚州教育与早期儿童发展部。原文存档于2012年11月20日。
\item Ifrah, Georges (2000), 《数字的普遍历史:从史前时代到计算机的发明》,John Wiley & Sons,第48页,ISBN 0-471-39340-1
\item W., Weisstein, Eric. "3, 4, 5 Triangle". mathworld.wolfram.com. 检索于2017-09-11。
\item Lorenz, Konrad (1961). 《所罗门王的戒指》. Marjorie Kerr Wilson翻译。伦敦:Methuen出版社。ISBN 0-416-53860-6。
\item "DIY: Enrico Fermi's Back of the Envelope Calculations"。
\item "Try numbers" 是费曼的一种问题解决技巧。
\item Ouellette, Jennifer (2021年3月12日). "科学家解开安提基特拉机制的另一个谜题"。Ars Technica。
\item Simon Singh, 《密码本》,第14-20页。
\item "Al-Kindi, Cryptography, Codebreaking and Ciphers". 2003年6月9日。检索于2022年7月3日。
\item Koetsier, Teun (2001), "论可编程机器的史前时代:音乐自动机、织布机、计算器",《机械与机器理论》,36 (5),Elsevier出版社:589–603,doi:10.1016/S0094-114X(01)00005-2。
\item "A 19th-Century Mathematician Finally Proves Himself"。NPR.org。检索于2022-10-24。
\item "A Modern Sequel | Babbage Engine | 计算机历史博物馆"。www.computerhistory.org。检索于2022-10-24。
\item Randell 1982,第4-5页。
\item "Percy Ludgate's Analytical Machine"。fano.co.uk。检索于2018年10月29日。
\item "Percy E. Ludgate Prize in Computer Science"(PDF)。约翰·加布里埃尔·伯恩计算机科学收藏。检索于2020-01-15。
\item Randell 1982,第6、11-13页。
\item Randell 1982,第13、16-17页。

L. Torres Quevedo (1914). "关于自动化的论文——其定义。应用的理论扩展"。《精确科学学报》,第12期:391–418。

Torres Quevedo, L. (1915). "自动化的论文——其定义。应用的理论扩展"。《纯粹与应用科学总览》。第2期:601–611。

"计算机先驱者 J.A.N. Lee著 - 莱昂纳多·托雷斯·耶·凯韦多"。检索于2018年2月3日。

Bromley 1990。

Steinhaus, H. (1999). 《数学快照》(第3版)。纽约:Dover出版社,第92–95页,301页。

"EDSAC模拟器教程指南"(PDF)。检索于2020-01-15。

Peirce, C. S.,"信件,皮尔斯致A. Marquand",1886年,查尔斯·S·皮尔斯文集,第5卷,1993年,第421–423页。见Burks, Arthur W.(1978年9月)。“评论:查尔斯·S·皮尔斯,《数学新元素》”。美国数学会公告,84(5):917。doi:10.1090/S0002-9904-1978-14533-9。

Peirce, C. S.(1880–81年冬季手稿),"具有一个常数的布尔代数",1933年在《查尔斯·S·皮尔斯文集》第4卷,第12–20段中出版。1989年在《查尔斯·S·皮尔斯文集》第4卷,第218–212页重新出版。见Roberts, Don D.(2009),《查尔斯·S·皮尔斯的存在图》,第131页。

Büning, Hans Kleine;Lettmann, Theodor (1999). 《命题逻辑:推理与算法》。剑桥大学出版社,第2页。ISBN 978-0-521-63017-7。

Bird, John (2007). 《工程数学》。Newnes出版社,第532页。ISBN 978-0-7506-8555-9。

Wynn-Williams, C. E.(1931年7月2日),"利用气体放电管进行高速自动计数物理现象",《皇家学会会刊A》,132(819):295–310,Bibcode:1931RSPSA.132..295W,doi:10.1098/rspa.1931.0102。

Yamada, Akihiko (2004). "日本开关理论研究史"。《电气工程学会事务:基础与材料》,124(8)。日本电气工程师学会:720–726。Bibcode:2004IJTFM.124..720Y。doi:10.1541/ieejfms.124.720。
\end{enumerate}