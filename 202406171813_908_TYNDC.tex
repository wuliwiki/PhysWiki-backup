% 太阳能电池
% license CCBYSA3
% type Wiki

(本文根据 CC-BY-SA 协议转载自原搜狗科学百科对英文维基百科的翻译)

\begin{figure}[ht]
\centering
\includegraphics[width=6cm]{./figures/af8830ce3bab8b15.png}
\caption{传统的晶体硅太阳能电池。由汇流条(较大的银色带)和手指(较小的带)制成的电触点印刷在硅片上。} \label{fig_TYNDC_1}
\end{figure}

太阳能电池,或称光伏电池,是一种通过光伏效应将光能直接转化为电能的电力设备,光伏效应是一种物理和化学现象。 它是光电池的一种形式,被定义为当暴露在光线下时,其电特性如电流、电压或电阻会发生变化的装置。单个太阳能电池装置可以组合成模块,也称为太阳能电池板。 基本上,单结硅太阳能电池可以产生大约0.5至0.6伏的最大开路电压。[1]

太阳能电池被描述为光伏电池,无论其来源是阳光还是人造光。它们被用作光电探测器(例如红外探测器),探测可见光范围内的光或其他电磁辐射,或测量光强。

光伏电池的运行需要三个基本属性:
\begin{itemize}
\item 光的吸收,产生电子空穴对或激子。
\item 相反类型电荷载流子的分离。
\item 将这些载流子单独提取到外部电路中。
\end{itemize}

相比之下,太阳能集热器是通过吸收阳光来提供热量,用于直接加热或间接发电。另一方面,“光电分解电池”(光电化学电池)是指一种光伏电池(像亚历山大·爱德蒙·贝克勒尔和现代染料敏化太阳能电池开发的),或者是指一种仅利用太阳能光照将水直接分解成氢和氧的装置。

\subsection{应用}

\begin{figure}[ht]
\centering
\includegraphics[width=6cm]{./figures/a66af93ccc61e938.png}
\caption{光伏系统的组件图,从太阳能电池到光伏系统。} \label{fig_TYNDC_17}
\end{figure}

太阳能电池组件被用来制造太阳能模块,太阳能模块利用阳光产生电能,区别于“太阳能热能模组”或“太阳能热水器”。

\subsubsection{1.1 电池、模块、面板和系统}

集成的多个太阳能电池,全部定向在一个平面上,构成太阳能光伏板或模块。光伏模块通常在面向太阳的一侧有一片玻璃,允许光线通过,同时保护半导体晶片。太阳能电池通常串联和并联电路连接,或模块串联,产生附加电压。并联电池产生更高的电流;然而,阴影效应等问题可能会关闭较弱(光照较少)的部分串连电池(许多串联的单元),因为受到光照的电池对阴影电池施加反向偏置,从而导致大量功率损失和可能的损坏。串联电池串通常独立处理,不并联,但从2014年开始,每个模块通常都有单独的电源盒,并且可以并联连接。虽然模块可以互连以创造具有所需峰值直流电压和负载电流容量的阵列,但是使用独立的最大功率点跟踪器是优选的。此外,并联二极管可以减少串联/并联单元阵列中的遮蔽功率损耗。

\begin{table}[ht]
\centering
\caption{2013年一些国家的典型光伏系统的价格 (\$/W)}\label{tab_TYMDZ}
\begin{tabular}{|c|c|c|c|c|c|c|c|c}
\hline
USD/W&澳大利亚&中国	&法国 & 德国 & 意大利 & 日本 & 英国 & 美国 \\
\hline
居民用途 & 1.8 & 1.5 & 4.1 & 2.4 & 2.8 & 4.2 & 2.8 & 4.9 \\
\hline
商业用途 & 1.7 & 1.4& 2.7 & 1.8 & 1.9 & 3.6 & 2.4 & 4.5 \\
\hline
公用事业用途 & 2.0 & 1.4 & 2.2 & 1.4 & 1.5 & 2.9 & 1.9 & 3.3 \\
\hline
\end{tabular}
\end{table}


\subsection{历史}

法国物理学家亚历山大·爱德蒙·贝克勒尔首先通过实验证明了光伏效应。1839年,19岁时,他在父亲的实验室里建造了世界上第一个光伏电池。威洛比·史密斯在1873年2月20日的《自然》杂志上首次描述了“电流通过时光对硒的影响”。1883年,查尔斯·弗里茨通过在半导体硒上镀一薄层金来形成结,从而制造了第一个固态光伏电池;该设备的效率只有1\verb|%|左右。其他里程碑包括:

\begin{itemize}
\item 1888年,俄罗斯物理学家亚历山大·斯托尔托夫基于1887年海因里希·赫兹发现的外光电效应制造了第一个电池。[2]
\item 1905年,阿尔伯特·爱因斯坦提出了一个新的光量子论,并在一篇里程碑式的论文中解释了光电效应,并因此获得了1921年的诺贝尔物理学奖。[3]
\item 1941年,瓦迪姆·拉什卡耶夫发现 p-n-铜结$2^o$和$Ag_2S$光伏电池。[4]
\item 1946年,拉塞尔·欧尔获得了现代结型半导体太阳能电池的专利,[5] 同时致力于晶体管的发展。
\item 1954年,第一个实用的光伏电池在贝尔实验室公开展示。[6] 发明者是卡尔文·沙无然·富勒、达里尔·查平和杰拉尔德·皮尔逊。[7]
\item 1958年,太阳能电池与先锋一号卫星的结合使其变得引人注目。
\end{itemize}


\subsubsection{2.1 空间应用}

\begin{figure}[ht]
\centering
\includegraphics[width=6cm]{./figures/279f1ab220154bd3.png}
\caption{美国宇航局(NASA)从一开始就在航天器上使用太阳能电池。例如,1959年发射的探索者6号,有四个阵列,在轨道上折叠一次。它们在太空中提供了数月的电力。} \label{fig_TYNDC_16}
\end{figure}

1958年,作为主电池电源的替代电源,太阳能电池首次在先锋卫星上被提出并应用于一个突出的领域。通过在航天器外添加电池,延长了任务时间,而不会对航天器或其动力系统造成重大改变。1959年,美国发射了“探索者6号”(Explorer 6),其特点是巨大的翼状太阳能阵列,这成为卫星的一个常见特征。这些阵列由9600个霍夫曼太阳能电池组成。

到20世纪60年代,太阳能电池已经是(现在仍然是)大多数地球轨道卫星和许多太阳系探测器的主要能源,因为它们提供了最佳的功率重量比。然而,这一成功在空间应用是可能的,因为在空间应用中电力系统成本可能很高,空间用户几乎没有其他电力选择,并且愿意为尽可能最好的电池付费。空间电力市场推动了太阳能电池效率的提高,直到美国国家科学基金会的“应用于国家需求的研究”项目开始推动太阳能电池在陆地上的应用。

20世纪90年代初,用于空间太阳能电池的技术不同于用于地面太阳能板的硅技术,航天器的应用转移到砷化镓基III-V半导体材料,然后演变成航天器上使用的现代III-V多结光伏电池。

近年来,研究已经转向设计和制造轻质、柔性和高效的太阳能电池。陆地太阳能电池技术通常使用层压有玻璃层的光伏电池来增强和保护。太阳能电池的空间应用要求电池和阵列既高效又极其轻量化。一些在卫星上实现的新技术是多结光伏电池,它由不同的PN结组成,带隙不同,以便利用更宽的太阳能量谱。此外,大型卫星需要使用大型太阳能电池阵列来发电。这些太阳能电池阵列需要被分解,以适应卫星进入轨道前运载火箭的几何约束。历史上,卫星上的太阳能电池由几块折叠在一起的小型陆地面板组成。卫星在轨道上部署后,这些小面板将展开成一个大面板。较新的卫星旨在使用柔性的可卷曲太阳能电池阵列,这种阵列重量很轻,可以装入很小的体积。由于有效载荷重量和运载火箭发射成本之间的直接关系,这些柔性阵列的较小尺寸和重量大大降低了发射卫星的总成本。[8]

\subsubsection{2.2 降价}

在20世纪60年代,进步是缓慢的。这也是成本居高不下的原因,因为空间用户愿意为尽可能最好的电池付费,没有理由投资于成本更低、效率更低的解决方案。价格主要由半导体行业决定;20世纪60年代,他们转向集成电路,导致较大的晶锭以较低的相对价格出现。随着价格的下降,最终太阳能电池的价格也随之下降。这些因素将1971年的电池成本降低到大约每瓦100美元。[9]

1969年末,埃利奥特·伯曼加入了埃克森美孚的特别工作组,该工作组正在寻找30年后的项目,1973年4月,他成立了太阳能公司,当时是埃克森的全资子公司。[10][11][12] 该小组的结论是,到2000年,电力将更加昂贵,并认为价格的上涨将使替代能源更具吸引力。他进行了一项市场研究,得出的结论是,每瓦20美元左右的价格将创造巨大的需求。[10] 该团队取消了抛光晶片和用抗反射层涂覆晶片的步骤,而是依靠粗糙的晶片表面。该团队还用背面的印刷电路板、正面的丙烯酸塑料和两者之间的硅胶胶水“灌注”电池来代替空间应用中使用的昂贵材料和手工布线。[13] 太阳能电池可以用电子市场的废弃材料制成。到1973年,他们宣布了一种产品,太平洋共同体秘书处说服Tideland Signal使用其面板为美国海岸警卫队的导航浮标供电。[11]

\subsubsection{2.3 研究和工业生产}

从1969到1977,美国国家科学基金会的高级太阳能研究开发部门在“应用于国家需求”的研究项目中对太阳能应用进行了研究,[14] 资助了为地面电力系统开发太阳能的研究。1973年的“樱桃山会议”提出了实现这一目标所需的技术目标,并概述了实现这些目标的雄心勃勃的项目,启动了将持续几十年的应用研究项目。[15] 该项目最终被能源研究与发展局(ERDA)接管,[16] 后来被并入美国能源部。

继第一次石油危机之后,石油公司利用其更高的利润创办(或收购)太阳能公司,几十年来一直是最大的生产商。埃克森、ARCO、壳牌、阿莫科(后来被英国石油公司收购)和美孚在20世纪70年代和80年代都有主要的太阳能部门。一些技术公司也参加了,包括通用电气、摩托罗拉、IBM、泰科和RCA。[17]

\subsection{成本下降和指数增长}

\begin{figure}[ht]
\centering
\includegraphics[width=6cm]{./figures/b2a729ff8abde0c0.png}
\caption{自1977年以来,传统晶体硅太阳能电池的每瓦历史价格。} \label{fig_TYNDC_12}
\end{figure}

\begin{figure}[ht]
\centering
\includegraphics[width=6cm]{./figures/0a8e5b9fd24d3224.png}
\caption{斯旺森定律——太阳光伏的学习曲线。} \label{fig_TYNDC_13}
\end{figure}

考虑到通货膨胀,在20世纪70年代中期,一个太阳能组件的成本是每瓦特96美元。彭博社新能源金融的数据显示,工艺改进和产量的大幅提升使这一数字下降了99\verb|%|,至2016年的68美分每瓦特。[18]斯旺森定律是一个类似于穆尔定律的发现,它指出太阳能电池价格每翻一倍工业产能就下降20%。[19]

进一步的改进将生产成本降至每瓦1美元以下,批发成本远低于2美元。当时系统成本的价格高于面板的价格。从2010年开始,大型商用阵列可以以低于每瓦3.40美元的价格建成,并全面投入使用。[20][21]

随着半导体工业转移到越来越大的晶锭,旧设备变得越来越便宜。随着剩余市场上设备的出现,电池尺寸也在增加;ARCO太阳能公司最初的太阳能板使用 2 to 4 inches (50 to 100 millimetres) 直径的电池。20世纪90年代和21世纪初的面板通常使用125 mm晶片;自2008年以来,几乎所有新面板都使用156 毫米太阳能电池。平板电视在20世纪90年代末和21世纪初的广泛应用导致了覆盖面板的大块高质量玻璃板的广泛供应。

在20世纪90年代,多晶硅(“多晶硅”)电池变得越来越流行。这些电池提供的效率比它们的单硅(mono)电池低,但它们在大缸中生长,降低了成本。到2000年代中期,多晶低成本面板市场占据主导地位,但最近单晶又开始广泛使用。

2004-2008年,硅片电池制造商对较高的硅价格做出了反应,硅消费量迅速下降。2008年,根据有机微电子中心和太阳能部门主任杰夫·波尔特曼的说法,目前的电池使用 8–9 grams (0.28–0.32 ounces) 硅每瓦特发电量,晶片厚度约为200 微米。 晶体硅面板主导着全球市场,主要在中国和台湾制造。到2011年底,欧洲需求的下降将晶体太阳能模块的价格降至1.09美元左右[21] 比2010年大幅下降。价格在2012年继续下跌,到2012年第四季度达到0.62美元/瓦。[22]

\begin{figure}[ht]
\centering
\includegraphics[width=6cm]{./figures/aaea6c795018e773.png}
\caption{光伏技术的增长——全球光伏总装机容量。} \label{fig_TYNDC_14}
\end{figure}

太阳能光伏发电在亚洲发展最快,中国和日本目前占全球部署的一半。[23] 2016年,全球光伏装机容量至少达到301千兆瓦,到2016年增长至全球发电量的1.3%。[24]

硅太阳能电池和人类每美元利用的石油的能量容量,一些关键发电技术的碳强度。[8]
事实上,自2004年以来,成本为1美元的硅太阳能电池所利用的能量已经超过了石油。[25] 预计光伏发电将与整个欧洲的批发电力成本竞争,到2020年,晶体硅模块的能源回收期将降至0.5年以下。[26]

\begin{figure}[ht]
\centering
\includegraphics[width=6cm]{./figures/c9a2e769008f2461.png}
\caption{硅太阳能电池和人类每美元利用的石油的能量容量,一些关键发电技术的碳强度。[8]} \label{fig_TYNDC_15}
\end{figure}


\subsubsection{3.1 补贴和电网平价}

太阳能特定的上网电价因国家和国内而异。这种关税鼓励太阳能项目的发展。广泛的电网平价,即光伏电力等于或低于没有补贴的电网电力,可能需要在所有三个方面取得进展。太阳能的支持者希望首先在阳光充足、电力成本高的地区实现电网平价,比如加利福尼亚和日本。[27] 2007年,英国石油公司宣称夏威夷和其他使用柴油发电的岛屿的电网平价。乔治·布什将2015年定为美国电网平价的日期。[28][29] 光伏协会在2012年报告说,澳大利亚已经实现了电网平价(忽略了上网电价)。[30]

太阳能电池板的价格持续下跌了40年,在2004年中断,当时德国的高补贴大幅增加了那里的需求,并大幅提高了提纯硅的价格(提纯硅用于电脑芯片和太阳能电池板)。2008年的经济衰退和中国制造业的兴起导致价格恢复下跌。在2008年1月之后的四年里,德国太阳能电池组件的价格从每峰值每瓦特€3降至€1。与此同时,产能的年增长率超过了50\verb|%|。中国的市场份额从2008年的8\verb|%|增加到2010年最后一个季度的55\verb|%|以上。[31] 2012年12月,中国太阳能电池板的价格降至0.60美元/Wp(晶体模块)。[32] (缩写Wp代表瓦特峰值容量,或最佳条件下的最大容量。[33])

截至2016年底,据报道组装太阳能板的现货价格 (不是电池)已降至0.36美元/Wp的创纪录低点。第二大供应商Canadian Solar公司报告称,2016年第三季度的成本为0.37美元/Wp,比前一季度下降了0.02美元,因此可能仍然至少收支平衡。许多生产商预计,到2017年底,成本将降至0.30美元左右。[34] 另据报道,在世界上一些地区,新的太阳能装置比燃煤火力发电厂便宜,预计十年内世界上大多数地区都会出现这种情况。[35]

\subsection{理论}

\begin{figure}[ht]
\centering
\includegraphics[width=6cm]{./figures/8c1ba3a0fdec8ee0.png}
\caption{太阳能电池收集电荷的示意图。 光透过透明导电电极产生电子空穴对,电子空穴对由两个电极收集。[1]} \label{fig_TYNDC_10}
\end{figure}

太阳能电池分几个步骤工作:

\begin{itemize}
\item 阳光中的光子撞击太阳能电池板,被硅等半导体材料吸收。
电子从它们当前的分子/原子轨道被激发,一旦被激发,电子可以以热的形式耗散能量并返回轨道,或者穿过电池直到到达电极。电流流过材料以抵消电势,这种电被捕获。这种材料的化学键对这一过程的进行至关重要,通常硅分为两层,一层掺杂硼,另一层掺杂磷。这些层具有不同的化学电荷,并随后驱动和引导电子流。[36]
\item 太阳能电池阵列将太阳能转换成可用的直流电(DC)。
\item 逆变器可以将电能转换成交流电。
\end{itemize}

\begin{figure}[ht]
\centering
\includegraphics[width=6cm]{./figures/0931d8e8db851fc8.png}
\caption{太阳能电池工作机理} \label{fig_TYNDC_11}
\end{figure}

最常见的太阳能电池是由硅制成的大面积pn结。其他可能的太阳能电池类型有有机太阳能电池、染料敏化太阳能电池、钙钛矿太阳能电池、量子点太阳能电池等。太阳能电池的光照侧通常具有透明导电膜,用于允许光进入活性材料并收集产生的电荷载流子。典型地,具有高透射率和高电导率的膜例如氧化铟锡、导电聚合物或导电纳米线网络被用于该目的。[37]

\subsection{效率}

\begin{figure}[ht]
\centering
\includegraphics[width=6cm]{./figures/bba09f1845145ba2.png}
\caption{Shockley-Queisser限制了太阳能电池的理论最大效率。 带隙在1和1.5eV之间或近红外光的半导体最有可能形成有效的单结电池。 (多结太阳能电池可以超过此处所示的效率“极限”。)} \label{fig_TYNDC_9}
\end{figure}

太阳能电池效率可以分为反射效率、热力学效率、电荷载流子分离效率和导电效率。总体效率是这些指标的乘积。

太阳能电池的功率转换效率是一个由入射功率转换成电能的百分比定义的参数。[36]

太阳能电池具有电压相关的效率曲线、温度系数和允许的阴影角。

由于难以直接测量这些参数,其他参数被用来替代: 热力学效率、量子效率、积分量子效率、$V_{OC}$ 比率和填充系数。反射损耗是“外部量子效率”下量子效率的一部分。复合损失构成量子效率的另一部分,$V_{OC}$ 比率和填充系数。电阻损耗主要归入填充因子,但也构成量子效率$V$的一小部分$_{OC}$ 比率。

填充系数是实际最大可获得功率与开路电压和短路电流乘积的比值。这是评估性能的一个关键参数。2009年,典型的商用太阳能电池的填充系数> 0.70。B级电池通常在0.4和0.7之间。[38] 具有高填充因子的电池具有低等效串联电阻和高等效并联电阻,因此电池产生的电流较少被消耗在内部损耗中。

单个pn结晶体硅器件现在接近理论极限功率效率33.16\verb|%|,[39] 1961年被称为肖克利-奎伊瑟极限。在层数无限和使用聚焦阳光的极端情况下,相应的效率极限是86\verb|%|。[40]

2014年,三家公司打破了硅太阳能电池25.6\verb|%|的纪录。松下的效率最高。该公司将前触点移到面板后部,消除了阴影区域。此外,他们在(高质量硅)晶片的正面和背面涂上薄硅膜,以消除晶片表面或附近的缺陷。[41]

2015年,在法国和德国的夫琅和费太阳能系统研究所、CEA-LETI和SOITEC的合作下,一个4结的GaInP/GaAs//GaInAsP/GaInAs太阳能电池实现了46.1\verb|%|的新实验室记录效率(阳光浓度比= 312)。[42]

2015年9月,夫琅和费ISE宣布外延片电池的效率达到20\verb|%|以上。优化大气压化学气相沉积(APCVD)生产线的工作是与NexWafe GmbH合作完成的,NexWafe GmbH是从Fraunhofer ISE剥离出来的一家公司,旨在将生产商业化。

三结薄膜太阳能电池的世界纪录是13.6\verb|%|,创于2015年6月。[43]

2016年,夫琅和费ISE的研究人员宣布了一种GaInP/GaAs/硅三结太阳能电池,其两个终端的效率在不聚焦光的情况下达到30.2\verb|%|。[44]

2017年,国家可再生能源实验室(NREL)、EPFL和CSEM(瑞士)的一组研究人员报告称,双结GaInP/GaAs太阳能电池器件的单太阳效率达到创纪录的32.8\verb|%|。此外,双结器件与硅太阳能电池机械堆叠,三结太阳能电池的单太阳效率达到创纪录的35.9\verb|%|。[45]

\subsection{材料}

\begin{figure}[ht]
\centering
\includegraphics[width=6cm]{./figures/941c451fed4be7bb.png}
\caption{自1990年以来光伏技术年产量的全球市场份额。} \label{fig_TYNDC_8}
\end{figure}

太阳能电池通常以它们所用的半导体材料命名。为了吸收阳光,这些材料必须具有一定的特性。一些电池被设计用来处理到达地表的阳光,而另一些电池被优化用于太空。太阳能电池可以仅由单层光吸收材料(单结)制成,或者使用多种物理配置(多结)来利用各种吸收和电荷分离机制。

太阳能电池可分为第一代、第二代和第三代电池。第一代电池——也称为传统电池或晶片电池——由晶体硅制成,晶体硅是商业上占主导地位的光伏技术,包括多晶硅和单晶硅等材料。第二代电池是薄膜太阳能电池,包括非晶硅、碲化镉和氯化锗电池,在公用事业规模的光伏电站、建筑集成光伏或小型独立电力系统中具有重要的商业意义。第三代太阳能电池包括许多薄膜技术,这些技术通常被描述为新兴的光伏技术——其中大部分尚未在商业上应用,仍处于研发阶段。许多使用有机材料,通常是有机金属化合物以及无机物。尽管它们的效率很低,并且吸收材料的稳定性对于商业应用来说通常太短,但是对这些技术进行了大量的研究,因为它们承诺实现生产低成本、高效率太阳能电池的目标。

\subsubsection{6.1 结晶硅}

迄今为止,太阳能电池最普遍的块体材料是晶体硅(c-Si),也称为“太阳能级硅”。根据所得晶锭、晶带或晶片中的结晶度和晶体尺寸,晶体硅被分成多种类型。这些电池完全基于pn结的概念。由晶体硅制成的太阳能电池由160到240微米厚的晶片制成。

\textbf{单晶硅}

\begin{figure}[ht]
\centering
\includegraphics[width=6cm]{./figures/713954c7775c9116.png}
\caption{顶部,发动机罩和Sion外壳的大部分都配备了高效的单晶硅电池。} \label{fig_TYNDC_7}
\end{figure}

单晶硅太阳能电池比大多数其他类型的电池更有效,也更昂贵。电池的角看起来像八边形,因为晶片材料是从圆柱形铸锭上切下来的,而这些铸锭通常是由柴可拉斯基法生长的。使用单晶硅电池的太阳能电池板显示出独特的白色小钻石图案。

\textbf{外延硅发展}

晶体硅外延片可通过化学气相沉积(CVD)在单晶硅“种子”晶圆上生长,然后分离成某些标准厚度(例如250微米)的自支撑晶圆,可手动操作,并直接替代从单晶硅锭切割的晶片单元。用这种“切口”技术制造的太阳能电池可以具有接近晶片切割电池的效率,但是如果化学气相沉积可以在大气压力下以高通量在线工艺进行,成本明显更低。[46][46] 外延晶片的表面可以被纹理化以增强光吸收。[46][47]

2015年6月,据报道,在n型单晶硅晶片上外延生长的异质结太阳能电池在243.4厘米${^2}$的总电池面积上达到了22.5\verb|%|的效率 。[48]

\textbf{多晶硅}

多晶硅或多晶硅电池由铸造的方形铸锭制成——大块熔融硅经过仔细冷却和固化。它们由小晶体组成,赋予材料典型的金属片效果。多晶硅电池是光伏电池中最常见的类型,比单晶硅电池便宜,但效率也低。

\textbf{带状硅}

带状硅是多晶硅的一种,它是通过从熔融硅中提取平坦的薄膜而形成的,并形成多晶结构。由于硅废料的大量减少,这些电池比多硅电池制造更便宜,因为这种方法不需要从铸锭上锯下来。[49] 然而,它们的效率也较低。

\textbf{类单晶硅多硅(MLM)}

这种形式是在2000年代开发的,并在2009年左右引入商业。这种设计也被称为铸造单体,它使用了多晶铸造室,其中含有单体材料的小“种子”。其结果是一种大块的单体材料,其外部是多晶的。当切片进行加工时,内部部分是高效的单晶电池(但是是正方形而不是“修剪”),而外部边缘作为传统的多晶电池。这种生产方法以类似多晶的价格生产出类似单晶的电池。[50]

\subsubsection{6.2 薄膜}

薄膜技术减少了电池中活性物质的数量。大多数设计将活性材料夹在两片玻璃之间。由于硅太阳能电池板只使用一块玻璃,薄膜板的重量大约是晶体硅板的两倍,尽管它们的环境影响较小(由生命周期分析确定)。[51] [52]

\textbf{碲化镉}

碲化镉是迄今为止唯一一种在成本/瓦特方面与晶体硅相媲美的薄膜材料。然而,镉是剧毒的,碲(阴离子:“碲化物”)的供应有限。电池中存在的镉如果释放出来将是有毒的。然而,在电池正常运行期间释放是不可能的,在住宅屋顶着火期间也是不可能的。[53] 一平方米的碲化镉薄膜电池包含与单个镍镉蓄电池大约相同量的镉,并且更稳定和更难溶解的。[53]

\textbf{铜铟硒化镓薄膜电池}

铜铟硒化镓薄膜电池是一种直接带隙材料。在所有具有商业意义的薄膜材料中,它具有最高的效率(约20\verb|%|)。传统的制造方法包括真空工艺,包括共蒸发和溅射。IBM和纳米太阳能的最新发展试图通过使用非真空解决方案来降低成本。[54]

\textbf{硅薄膜}

硅薄膜电池主要由硅烷气体和氢气通过化学气相沉积(通常为等离子体增强等离子体化学气相沉积)来制造。根据沉积参数,这可以产生非晶硅(a-Si或a-Si:H)、原晶硅或纳米晶硅(nc-Si或nc-Si:H),也称为微晶硅。[55]

非晶硅是迄今最发达的薄膜技术。非晶硅太阳能电池由非晶硅或微晶硅制成。非晶硅具有比晶体硅(c-Si) (1.1 eV)更高的带隙(1.7 eV),这意味着它比高功率密度的红外光谱部分更强烈地吸收太阳光谱的可见光部分。非晶硅薄膜太阳能电池的生产使用玻璃作为衬底,并通过离子强化气相沉积来沉积非常薄的硅层。

具有低体积分数纳米晶体硅的原晶硅是高开路电压的最佳选择。[56] 纳米硅具有与晶体硅和纳米硅大约相同的带隙,并且晶体硅和纳米硅可以有利地组合成薄层,从而产生称为串联电池的分层电池。非晶硅中的顶部电池吸收可见光,并为非晶硅中的底部电池留下光谱的红外部分。

\textbf{砷化镓薄膜}

半导体材料砷化镓(GaAs)也用于单晶薄膜太阳能电池。尽管GaAs电池非常昂贵,但它们在单结太阳能电池效率方面保持着世界纪录,达到28.8\verb|%|。[57] GaAs更常用于聚光光伏电池(CPV、HCPV)和航天器上的太阳能电池板的多结光伏电池,因为该行业更青睐基于空间的太阳能发电的效率而非成本。根据以前的文献和一些理论分析,GaAs具有如此高的功率转换效率有几个原因。首先,GaAs带隙是1.43ev,这几乎是太阳能电池的理想选择。其次,因为镓是其他金属冶炼的副产品,GaAs电池对热相对不敏感,当温度相当高时,它可以保持高效率。第三,GaAs有广泛的设计选择。利用GaAs作为太阳能电池的活性层,工程师们可以有多种其他层的选择,这些层可以更好地在GaAs产生电子和空穴。

\subsubsection{6.3 多结电池}

\begin{figure}[ht]
\centering
\includegraphics[width=6cm]{./figures/7733a9ee4733d5b0.png}
\caption{黎明公司的完全延伸的10千瓦三结砷化镓太阳能电池阵列。} \label{fig_TYNDC_6}
\end{figure}

多结电池由多个薄膜组成,每个薄膜本质上是一个生长在另一个之上的太阳能电池,通常使用有机金属气相外延。每一层都有不同的带隙能量,允许它吸收光谱不同部分的电磁辐射。多结电池最初是为卫星和空间探索等特殊应用而开发的,但现在越来越多地用于陆地聚光光伏(CPV),这是一种新兴技术,使用透镜和曲面镜将阳光集中到小型高效多结太阳能电池上。通过将阳光聚集一千倍, 高能聚光光伏技术(HCPV) 有潜力在未来超越传统的太阳能光伏发电。[58]

尽管存在成本压力,基于单片串联镓磷化铟(GaInP)、砷化镓(GaAs)和锗(GE)pn结的串联太阳能电池的销量仍在增长。[59] 从2006年12月到2007年12月,4N镓金属的价格从每公斤350美元左右上升到每公斤680美元。此外,锗金属价格今年大幅上涨至每公斤1000至1200美元。这些材料包括镓(4N、6N和7N Ga)、砷(4N、6N和7N)和锗、用于生长晶体的热解氮化硼坩埚和氧化硼,这些产品对整个衬底制造业至关重要。

三结电池可以由半导体组成,例如GaAs、锗和 $GaInP_2$。[60] 三结砷化镓太阳能电池曾在2003、2005和2007年被荷兰四次世界太阳能挑战赛优胜者努纳(Nuna)和荷兰太阳能汽车解决方案(2005)、Twente One(2007)和21Revolution(2009)用作电源。基于GaAs的多结器件是迄今为止最高效的太阳能电池。2012年10月15日,三结太阳能电池达到创纪录的44\verb|%|。[61]

\textbf{GaInP/Si双结太阳能电池}

2016年,一种生产混合光伏晶片的新方法被引入,该方法将三伏多结太阳能电池的高效率与硅相关的经济性和丰富经验相结合。离子强化气相沉积在低温下在GaAs外延生长硅避免了在所需高温下在硅上生长ⅲ-ⅴ族材料的技术复杂性,这是30年来的研究课题。[62]

硅单结太阳能电池已经被广泛研究了几十年,并且在1标准太阳条件下达到了约26\verb|%|的实用效率。[63] 提高这种效率需要向硅电池添加更多带隙能量大于1.1 eV的电池,从而允许转换短波长光子以产生额外的电压。带隙为1.6-1.8 eV的双结太阳能电池作为顶部电池,可以降低热化损失,产生高的外部辐射效率,并实现超过45\verb|%|的理论效率。[64] 串联电池可以通过生长GaInP和硅电池来制造。分别生长它们可以克服硅和最常见的III-V层之间4\verb|%|的晶格常数失配,这种失配阻碍了直接集成到一个单元中。因此,两个单元被透明载玻片分开,因此晶格失配不会对系统造成应变。这产生了一个具有四个电触点和两个结的电池,效率为18.1\verb|%|。当填充系数(FF)为76.2\verb|%|时,串联装置中的硅底部电池的效率达到11.7\verb|%| ( 0.4),导致累计串联电池效率为29.8\verb|%|。[65] 这一效率超过了29.4\verb|%|的理论极限[66] 和创纪录的1标准太阳太阳能电池实验效率值,也高于创纪录的1标准太阳GaAs装置。然而,使用GaAs衬底是昂贵且不实用的。因此,研究人员试图制造一种具有两个电接触点和一个结的电池,这种电池不需要GaAs衬底。这意味着GaInP和Si将直接集成。

\subsection{太阳能电池研究}

\subsubsection{7.1 钙钛矿太阳能电池}

钙钛矿太阳能电池是包含钙钛矿结构材料作为活性层的太阳能电池。最常见的是,这是一种溶液处理的有机-无机锡或卤化铅基混合材料。效率已经从2009年首次使用时的不到5\verb|%|提高到2014年的20\verb|%|以上,这使得它们成为一项迅速发展的技术和太阳能电池领域的热门话题。[67] 据预测,钙钛矿太阳能电池规模扩大的成本极低,这使得它们成为非常有吸引力的商业化选择。到目前为止,大多数类型的钙钛矿太阳能电池还没有达到足够的运行稳定性进而商业化,尽管许多研究小组正在研究解决这个问题的方法。[68]

\subsubsection{7.2 双面太阳能电池}

\begin{figure}[ht]
\centering
\includegraphics[width=6cm]{./figures/d07b6898018d35a6.png}
\caption{1988年,诺托(塞内加尔)的双面太阳能电池厂,地板漆成白色以增强反照率。} \label{fig_TYNDC_5}
\end{figure}

双面太阳能电池具有透明的背面,可以从正面和背面吸收光线。因此,它们比传统的单电池能产生更多的电能。双面太阳能电池的第一项专利是由日本研究员Hiroshi Mori在1966年提出的。[69] 后来,20世纪70年代,据说俄罗斯是第一个在太空计划中部署双面太阳能电池的国家。 1976年, 马德里理工大学的太阳能研究所开始了由安东尼奥·卢克教授领导的双面太阳能电池开发研究项目。基于卢克1977年的美国和西班牙专利,提出了一种实用的双面电池,其正面为阳极,背面为阴极;在以前公开的建议和尝试中,两面都是阳极的,电池之间的互连复杂且昂贵。[70][71][72] 1980年,卢克团队的博士生安德烈斯·奎瓦斯通过实验证明,当提供白色背景时,双面太阳能电池的输出功率相对于相同取向和倾斜的单面太阳能电池增加了50\verb|%|。[73] 1981年,Isofoton公司在马拉加成立,生产开发双面电池,从而成为这种光伏电池技术的第一次工业化。最初的产能是300千瓦/年,在双面太阳能电池中,Isofoton生产的早期里程碑是1986年为西班牙伊维尔德罗拉公司建造的圣阿古斯丁-德瓜达利斯(San agustín de Guadalix)20kWp发电厂,以及1988年由西班牙国际援助与合作计划资助的诺托古耶迪亚马村(古耶迪亚马)20k WP的离网安装。

由于制造成本降低,自2010年以来,公司再次开始生产商用双面模块。截至2017年,北美至少有八家认证光伏制造商提供双面模块。《国际光伏技术路线图》预测,双面技术的全球市场份额将从2016年的不到5\verb|%|扩大到2027年的30\verb|%|。[74]

由于对双面技术的极大兴趣,最近的一项研究调查了全世界双面太阳能组件的性能和优化。[75][76] 结果表明,在全球范围内,地面安装的双面模块与地面反照率系数为25\verb|%|的单面模块相比,每年只能提供约10\verb|%|的发电量增益(针对典型的混凝土和植被地被)。然而,通过将模块提升到地面以上1 m,并将地面反照率系数提高到50\verb|%|,增益可以增加到30\verb|%|左右。Sun 等人 还导出了一组可以解析优化双面太阳能电池组件的经验方程。[75]

在线仿真工具可用于模拟双面模块在全球任意位置的性能。它还可以根据倾斜角、方位角和离地高度优化双面模块。[77]

\subsubsection{7.3 中间带}

太阳能电池研究中的中间带光伏技术提供了超过肖克利-奎伊瑟极限电池效率的方法。它在价带和导带之间引入了一个中间带(IB)能级。理论上,引入一个中间带允许两个能量小于带隙的光子将电子从价带激发到导带。这增加了感应光电流,从而提高了效率。[78]

卢克和玛蒂首先使用详细的平衡推导出一个中间间隙能级器件的效率理论极限。他们假设中间带没有收集到载流子,并且设备处于完全聚光光照状态,他们发现对于1.95eV的带隙,中间带为0.71eV,来自价带或导带,电池最大效率为63.2\verb|%|。在一个标准太阳光照下,极限效率为47\verb|%|。[79]

\subsubsection{7.4 液体墨水}

2014年,加州纳米系统研究所的研究人员发现使用黄锡矿和钙钛矿提高了太阳能电池的电能转化效率。[80]

\subsubsection{7.5 上转化和下转化}

光子上转换是利用两种低能光子(例如红外线)来产生一个高能光子的过程;下转换是使用一个高能光子(例如,紫外线)来产生两个低能光子的过程。这两种技术中的任何一种都可以被用来生产效率更高的太阳能电池,从而使太阳能光子得到更有效的利用。然而,困难在于现有的表现出上转换或下转换的磷光体的转换效率低,并且通常是窄带的。

一种上转换技术是结合镧系元素掺杂材料($Er^{3+}$, $Yb^{3+}$, $Ho^{3+}$ 或者它们的组合),利用其发光特性将红外辐射转换成可见光。当两个红外光子被稀土离子吸收产生(高能)可吸收光子时,就会发生上转换过程。例如,能量转移上转换过程包括近红外激发离子之间的连续转移过程。上转换器材料可以放置在太阳能电池的下面,以吸收通过硅的红外光。有用的离子最常见于三价状态。$Er^{+}$ 离子是最常用的。$Er^{3+}$ 离子吸收大约 1.54 微米的太阳辐射。两个吸收这种辐射的 $Er^{3+}$ 离子可以通过上转换过程相互作用。被激发的离子发射出接带阈以上的光,被太阳能电池吸收,并产生一个额外的电子-空穴对,可以产生电流。然而,提高的效率很小。此外,含氟玻璃有低声子能量,并认为是掺杂 $Ho^{3+}$ 离子的合适基质。[81]

\subsubsection{7.6 吸光染料}

染料敏化太阳能电池(DSSCs)由低成本材料制成,不需要复杂的制造设备,因此可以自己动手制造。整体而言,它应该比旧的固态电池设计便宜得多。染料敏化太阳能电池可以被设计成柔性片,虽然它的转换效率低于最好的薄膜电池,但是它的价格/性能比可能足够高,以使它们能够与化石燃料发电竞争。

典型地,钌金属有机染料(钌中心)被用作光吸收材料的单层。染料敏化太阳能电池依靠纳米二氧化钛的介孔层来大大扩大表面积(200-300米$^2/g$ $TiO_2$与大约 $10 m^2$/克扁平单晶相比)。吸光染料产生的光生电子被传递到n型 $TiO_2$ ,并且孔被染料另一侧的电解质吸收。该电路由电解质中的氧化还原对完成,电解质可以是液体或固体。这种类型的电池允许更灵活地使用材料,并且通常通过丝网印刷或超声波喷嘴制造,具有比用于大块太阳能电池更低加工成本的潜力。然而,这些电池中的染料在热和紫外光下会降解,并且电池外壳由于组装中使用的溶剂而难以密封。染料敏化太阳能电池模块的第一批商业运输发生在2009年7月,由G24i创新公司提供。[82]

\subsubsection{7.7 量子点}

量子点太阳能电池(QDSCs)基于Gratzel电池,或染料敏化太阳能电池架构,但采用低带隙半导体纳米粒子,其微晶尺寸足够小以形成量子点(例如CdS、CdSe、 $Sb_2S_3$、PbS等。),而不是使用有机或有机金属染料作为光吸收剂。由于与镉和铅基化合物相关的毒性,还有一系列“绿色”量子点敏化材料正在开发中(如氯化萘$_2$, $CuInSe_2$ 和胸针)。[83] 量子点的尺寸量子化允许通过简单地改变粒子大小来调节带隙。它们也有很高的消光系数,并显示出产生多重激子的可能性。[84]

在量子点太阳能电池中,二氧化钛纳米颗粒的介孔层形成了电池的骨架,很像在染料敏化太阳能电池中。然后,通过使用化学浴沉积、电泳沉积或连续离子层吸附和反应,用半导体量子点涂覆,可以使TiO2 层具有光活性。接下来 ,通过使用液体或固体氧化还原对完成电路。[85] 对于液体结[86] 和固态电池,量子点太阳能电池的效率了超过5\verb|%|,[87] 报道的峰值效率为11.91\verb|%|。[88] 为了降低生产成本,普拉山特·卡迈特研究小组[89] 展示了一种太阳能涂料 $TiO_2$ 和CdSe,其可以使用一步方法应用于任何导电表面,效率超过1\verb|%|。[90] 然而,量子点在室温下的吸收很弱。[91] 等离子体纳米粒子可用于解决量子点(例如纳米焦油)的弱吸收。[92] 另一种解决方案是添加外部红外泵浦源来激发量子点的带内和带间跃迁。[91]

\subsubsection{7.8 有机/聚合物太阳能电池}

有机太阳能电池和聚合物太阳能电池由有机半导体薄膜(通常为100 纳米)构成,包括聚合物,如聚苯乙烯和小分子化合物,如铜酞菁(蓝色或绿色有机颜料)和碳富勒烯和富勒烯衍生物,如PCBM。

它们可以用液体溶液中加工,提供了简单的卷到卷印刷工艺的可能性,潜在地导致廉价的大规模生产。此外,这些电池对于力学柔性和一次性应用可能是有益的。然而,目前的电池效率非常低,实际设备基本上不存在。

与无机材料相比,迄今为止使用导电聚合物实现的能量转换效率非常低。然而,科纳卡电力塑料达到了8.3\verb|%|的效率,[93] 有机串联电池在2012年达到11.1\verb|%|。

有机器件的有源区由两种材料组成,一种是电子供体,一种是电子受体。当光子被转换成电子-空穴对时,通常在供体材料中,电荷倾向于以激子形式保持束缚,当激子扩散到供体-受体界面时,电荷就会分离,这与大多数其他类型的太阳能电池不同。大多数聚合物系统的短激子扩散长度往往会限制这种器件的效率。纳米结构界面,有时以体异质结的形式,可以提高性能。[94]

2011年,麻省理工学院和密歇根州立大学的研究人员开发了太阳能电池,其能效接近2\verb|%|,对人眼的透明度大于65\verb|%|,这是通过用小分子化合物选择性吸收光谱的紫外线和近红外线部分实现的。[95][96] 加州大学洛杉矶分校的研究人员最近开发了一种类似的聚合物太阳能电池,采用同样的方法,即70\verb|%|透明,功率转换效率为4\verb|%|。[97][98][99] 这些轻质、柔性的电池可以以低成本批量生产,并可用于制造发电窗户。

2013年,研究人员宣布聚合物电池的效率约为3\verb|%|。他们使用嵌段共聚物,这种自组装的有机材料将自己排列成不同的层。这项研究集中在3-己基噻吩的聚合物-B-全氟辛烷磺酸,波段宽约为16nm。[100][101]

\subsubsection{7.9 自适应电池}

自适应电池根据环境条件改变它们的吸收/反射特性。适应性材料对入射光的强度和角度做出反应。在电池中光线最强的部分,电池表面从反射变为适应,允许光线穿透电池。电池的其他部分保持反射,增加了吸收光在电池内的保留。[102]

2014年,人们开发了一种系统,该系统将自适应表面与玻璃基板相结合,将吸收的光重新导向板边缘的光吸收器。该系统还包括固定透镜/反射镜阵列,以将光集中到自适应表面上。随着时间的推移,集中的光线沿着电池表面移动。当光线最集中时,该表面从反射型转换为自适应型,当光线移走后,该表面又转换为反射型。[102]

\subsubsection{7.10 表面纹理}

\begin{figure}[ht]
\centering
\includegraphics[width=6cm]{./figures/6efa341e247c76f2.png}
\caption{Solar Impulse飞机是瑞士设计的单座单翼飞机,完全由光伏电池供电。} \label{fig_TYNDC_4}
\end{figure}

过去几年来,研究人员一直试图降低太阳能电池的价格,同时最大限度地提高效率。薄膜太阳能电池是一种具有成本效益的第二代太阳能电池,其厚度大大减小,但以牺牲光吸收效率为代价。人们已经努力在减小厚度的情况下最大化光吸收效率。表面纹理化是减少光损失以最大化光吸收的技术之一。目前,硅光伏电池表面纹理化技术正受到广泛关注。表面纹理可以通过多种方式完成。使用各向异性蚀刻剂蚀刻单晶硅衬底可以在表面上产生随机分布的基于正方形的金字塔。[103] 最近的研究表明,晶体硅晶片可以被蚀刻下来形成纳米级的倒金字塔。多晶硅太阳能电池,由于较差的晶体质量,不如单晶太阳能电池有效,但多晶硅太阳能电池仍在广泛使用,因为制造难度较小。据报道,通过各向同性蚀刻或光刻技术,多晶硅太阳能电池可以被表面纹理化以产生与单晶硅电池相当的太阳能转换效率。[104][105] 与入射到平坦表面上的光线不同,入射到纹理表面上的光线不会反射回空气。相反,由于表面的几何形状,一些光线会被反射回该表面。由于光吸收增加,这一过程显著提高了光电转换效率。这种纹理效应以及与光伏模块中其他界面的交互是一项具有挑战性的光学模拟任务。一种特别有效的建模和优化方法是OPTOS形式化。[106] 2012年,麻省理工学院的研究人员报告称,具有纳米级倒金字塔纹理的晶体硅薄膜可以实现相当于30倍厚的平面晶体硅的光吸收。[107] 结合抗反射涂层,表面纹理技术可以有效地将光线捕获在薄膜硅太阳能电池中。因此,太阳能电池所需的厚度随着光线吸收的增加而减少。

\subsubsection{7.11 包装}

太阳能电池通常封装在透明的聚合树脂中,以保护精密的太阳能电池区域,使其在运行期间或在室外使用时与湿气、灰尘、冰和其他预期条件接触。密封剂通常由聚醋酸乙烯酯或玻璃制成。大多数密封剂在结构和组成上是均匀的,这增加了光收集,因为树脂内光的全内反射引起光的捕获。未来进一步提供光收集,对封装材料的结构进行了研究。这种密封剂包括粗糙的玻璃表面,[108] 衍射元件,[109] 棱镜阵列,[110] 空气棱镜,[111] v形凹槽,[112] 漫射元件以及多方向波导阵列。[113] 棱镜阵列显示总的太阳能转换增加了5\verb|%|。[111] 垂直排列的宽带波导阵列在垂直入射时增加了10\verb|%|,广角收集增强了4\verb|%|,[114] 优化的结构使短路电流增加了20\verb|%|。[115] 将红外光转换成可见光的活性涂层显示出30\verb|%|的增长。[116] 诱导等离子体光散射的纳米粒子涂层将广角转换效率提高到3\verb|%|。在封装材料中也产生了光学结构,以有效地“遮盖”金属前触点。[117][118]

\subsection{制造}

\begin{figure}[ht]
\centering
\includegraphics[width=6cm]{./figures/204042fc46ffd08b.png}
\caption{早期的太阳能供电计算器。} \label{fig_TYNDC_3}
\end{figure}

太阳能电池与其他半导体器件有一些相同的加工和制造技术。然而,太阳能电池比半导体制造的清洁度和质量控制的严格要求更加宽松,降低了成本。

多晶硅晶片是通过将块状硅锭线锯成180至350微米的晶片来制造的。晶片通常是轻度$p$型掺杂的。在晶片的正面进行$n$型掺杂剂的表面扩散。这在表面下几百纳米处形成一个$pn$结。

然后通常应用抗反射涂层来增加耦合到太阳能电池中的光量。氮化硅已经逐渐取代二氧化钛作为优选材料,因为它具有优异的表面钝化质量。它防止电池表面的载流子复合。使用PECVD施加几百纳米厚的层。一些太阳能电池的前表面有纹理,像防反射涂层,增加了到达晶圆的光的数量。这种表面首先应用于单晶硅,随后是多晶硅。

整个区域的金属接触是在背面进行的,一个网格状的金属接触是由细的“手指”和较大的“汇流条”组成的,用银浆在正面进行丝网印刷。这是拜耳公司在1981年提交的美国专利中首次描述的应用电极的所谓“湿”工艺的演变。[119] 后触点通过丝网印刷金属膏形成,通常是铝。通常这种接触覆盖整个后部,尽管有些设计采用网格图案。然后在几百摄氏度下烧制该浆料,以形成与硅欧姆接触的金属电极。一些公司使用额外的电镀步骤来提高效率。金属接触完成后,太阳能电池通过扁线或金属带相互连接,并组装成模块或“太阳能电池板”。太阳能电池板正面有一块钢化玻璃,背面有聚合物封装。

\subsection{制造商和认证}

\begin{figure}[ht]
\centering
\includegraphics[width=6cm]{./figures/3086e1d44fcf5d1c.png}
\caption{不同地区的太阳能电池产量。[1]} \label{fig_TYNDC_2}
\end{figure}

国家可再生能源实验室测试和验证太阳能技术。三个可靠的团体认证太阳能设备:UL和IEEE(两个美国标准)和IEC。

日本、德国、中国、台湾、马来西亚和美国大量生产太阳能电池,而欧洲、中国、美国和日本在已安装系统中占主导地位(截至2013年为94\verb|%|或更多)。[120] 其他国家正在获得大量太阳能电池生产能力。

根据欧盟委员会联合研究中心发布的年度“光伏状况报告”,尽管太阳能投资下降了9\verb|%|,但2012年全球光伏电池/模块产量增长了10\verb|%|。2009年至2013年间,电池产量翻了两番。[120][121][122]

\subsubsection{9.1 中国}

由于政府的大力投资,中国已经成为太阳能电池制造业的主导力量。2013年,中国公司生产的太阳能电池/模块的产能约为23 GW(占全球产量的60\verb|%|)。[120]

\subsubsection{9.2 马来西亚}

2014年,马来西亚是仅次于中国和欧盟的世界第三大光伏设备制造商。[123]

\subsubsection{9.3 美国}

美国的太阳能电池生产因全球金融危机而受损,但由于优质硅价格下跌而部分恢复。[124][125]

\subsection{处理}

太阳能电池随着时间的推移会退化并失去效率。在极端气候条件下,如沙漠或极地,太阳能电池更容易因暴露在严酷的紫外线和雪中而退化。[126] 太阳能电池板在退役前通常有25-30年的寿命。[127]

国际可再生能源机构估计,2016年太阳能电池板产生的废物量为43500-250000吨。据估计,到2030年,这一数字将大幅增加,到2050年,估计废物量将达到6000万至7800万吨。[128]

\subsubsection{10.1 回收利用}

2018年,大多数退役的太阳能电池板被送往垃圾填埋场。回收利用是有限的,因为处理太阳能电池板废料的成本太高了。[129] 然而,太阳能电池含有有毒物质,如铅和镉,一旦被破坏,会渗入土壤并污染环境。随着太阳能电池板废物量的增加,在垃圾填埋场处置太阳能电池板的安全性成为一个大问题。许多制造商转而转向回收太阳能电池板。[130]

第一家太阳能电池板回收工厂于2018年在法国罗塞特开业。该公司计划每年回收1300吨太阳能电池板废料,并将产能提高到4000吨。[131]

\subsection{参考文献}

\begin{enumerate}
\item "Solar cells – performance and use". solarbotics.net..
\item Gevorkian, Peter (2007). Sustainable energy systems engineering: the complete green building design resource. McGraw Hill Professional. ISBN 978-0-07-147359-0..
\item "The Nobel Prize in Physics 1921: Albert Einstein", Nobel Prize official page.
\item Lashkaryov, V. E. (1941) Investigation of a barrier layer by the thermoprobe method Archived 28 9月 2015 at the Wayback Machine, Izv. Akad. Nauk SSSR, Ser. Fiz. 5, 442–446, English translation: Ukr. J. Phys. 53, 53–56 (2008).
\item "Light sensitive device" 美国专利 2,402,662 Issue date: June 1946.
\item "April 25, 1954: Bell Labs Demonstrates the First Practical Silicon Solar Cell". APS News. American Physical Society. 18 (4). April 2009..
\item Tsokos, K. A. (28 January 2010). Physics for the IB Diploma Full Colour. Cambridge University Press. ISBN 978-0-521-13821-5..
\item Garcia, Mark (2017-07-31). "International Space Station Solar Arrays". NASA. Retrieved 2019-05-10..
\item Perlin 1999, p. 50..
\item Perlin 1999, p. 53..
\item Williams, Neville (2005). Chasing the Sun: Solar Adventures Around the World. New Society Publishers. p. 84. ISBN 9781550923124..
\item Jones, Geoffrey; Bouamane, Loubna (2012). "Power from Sunshine": A Business History of Solar Energy (PDF). Harvard Business School. pp. 22–23..
\item Perlin 1999, p. 54..
\item The National Science Foundation: A Brief History, Chapter IV, NSF 88-16, 15 July 1994 (retrieved 20 June 2015).
\item Herwig, Lloyd O. (1999). "Cherry Hill revisited: Background events and photovoltaic technology status". AIP Conference Proceedings. National Center for Photovoltaics (NCPV) 15th Program Review Meeting. AIP Conference Proceedings. 462. p. 785. Bibcode:1999AIPC..462..785H. doi:10.1063/1.58015..
\item Deyo, J. N., Brandhorst, H. W., Jr., and Forestieri, A. F., Status of the ERDA/NASA photovoltaic tests and applications project, 12th IEEE Photovoltaic Specialists Conf., 15–18 Nov. 1976.
\item Reed Business Information (18 October 1979). The multinational connections-who does what where. Reed Business Information. ISSN 0262-4079..
\item Buhayar, Noah (28 January 2016) Warren Buffett controls Nevada’s legacy utility. Elon Musk is behind the solar company that’s upending the market. Let the fun begin. Bloomberg Businessweek.
\item "Sunny Uplands: Alternative energy will no longer be alternative". The Economist. 21 November 2012. Retrieved 28 December 2012..
\item 1/W Photovoltaic Systems DOE whitepaper August 2010.
\item Solar Stocks: Does the Punishment Fit the Crime?. 24/7 Wall St. (6 October 2011). Retrieved 3 January 2012..
\item Parkinson, Giles. "Plunging Cost Of Solar PV (Graphs)". Clean Technica. Retrieved 18 May 2013..
\item "Snapshot of Global PV 1992–2014" (PDF). International Energy Agency — Photovoltaic Power Systems Programme. 30 March 2015. Archived from the original on 30 March 2015..
\item "Solar energy – Renewable energy – Statistical Review of World Energy – Energy economics – BP". bp.com..
\item Yu, Peng; Wu, Jiang; Liu, Shenting; Xiong, Jie; Jagadish, Chennupati; Wang, Zhiming M. (2016-12-01). "Design and fabrication of silicon nanowires towards efficient solar cells". Nano Today. 11 (6): 704–737. doi:10.1016/j.nantod.2016.10.001..
\item Mann, Sander A.; de Wild-Scholten, Mariska J.; Fthenakis, Vasilis M.; van Sark, Wilfried G.J.H.M.; Sinke, Wim C. (2014-11-01). "The energy payback time of advanced crystalline silicon PV modules in 2020: a prospective study". Progress in Photovoltaics: Research and Applications. 22 (11): 1180–1194. doi:10.1002/pip.2363. ISSN 1099-159X..
\item "BP Global – Reports and publications – Going for grid parity". Archived from the original on 8 June 2011. Retrieved 4 August 2012.. Bp.com. Retrieved 19 January 2011..
\item BP Global – Reports and publications – Gaining on the grid. Bp.com. August 2007..
\item The Path to Grid Parity. bp.com.
\item Peacock, Matt (20 June 2012) Solar industry celebrates grid parity, ABC News..
\item Baldwin, Sam (20 April 2011) Energy Efficiency & Renewable Energy: Challenges and Opportunities. Clean Energy SuperCluster Expo Colorado State University. U.S. Department of Energy..
\item ENF Ltd. (8 January 2013). "Small Chinese Solar Manufacturers Decimated in 2012 | Solar PV Business News | ENF Company Directory". Enfsolar.com. Retrieved 1 June 2013..
\item "What is a solar panel and how does it work?". Energuide.be. Sibelga. Retrieved 3 January 2017..
\item Martin, Chris (30 December 2016). "Solar Panels Now So Cheap Manufacturers Probably Selling at Loss". Bloomberg View. Bloomberg LP. Retrieved 3 January 2017..
\item Shankleman, Jessica; Martin, Chris (3 January 2017). "Solar Could Beat Coal to Become the Cheapest Power on Earth". Bloomberg View. Bloomberg LP. Retrieved 3 January 2017..
\item Solar Cells. chemistryexplained.com.
\item  Kumar, Ankush (2017-01-03). "Predicting efficiency of solar cells based on transparent conducting electrodes". Journal of Applied Physics. 121 (1): 014502. Bibcode:2017JAP...121a4502K. doi:10.1063/1.4973117. ISSN 0021-8979..
\item "T.Bazouni: What is the Fill Factor of a Solar Panel". Archived from the original on 15 April 2009. Retrieved 17 February 2009..
\item Rühle, Sven (2016-02-08). "Tabulated Values of the Shockley-Queisser Limit for Single Junction Solar Cells". Solar Energy. 130: 139–147. Bibcode:2016SoEn..130..139R. doi:10.1016/j.solener.2016.02.015..
\item Vos, A. D. (1980). "Detailed balance limit of the efficiency of tandem solar cells". Journal of Physics D: Applied Physics. 13 (5): 839. Bibcode:1980JPhD...13..839D. doi:10.1088/0022-3727/13/5/018..
\item Bullis, Kevin (13 June 2014) Record-Breaking Solar Cell Points the Way to Cheaper Power. MIT Technology Review.
\item Dimroth, Frank; Tibbits, Thomas N.D.; Niemeyer, Markus; Predan, Felix; Beutel, Paul; Karcher, Christian; Oliva, Eduard; Siefer, Gerald; Lackner, David; et al. (2016). "Four-Junction Wafer Bonded Concentrator Solar Cells". IEEE Journal of Photovoltaics. 6 (1): 343–349. doi:10.1109/jphotov.2015.2501729..
\item Zyg, Lisa (4 June 2015). "Solar cell sets world record with a stabilized efficiency of 13.6\verb|%|". Phys.org..
\item 30.2\verb|%| Efficiency – New Record for Silicon-based Multi-junction Solar Cell — Fraunhofer ISE. Ise.fraunhofer.de (2016-11-09). Retrieved 2016-11-15..
\item Essig, Stephanie; Allebé, Christophe; Remo, Timothy; Geisz, John F.; Steiner, Myles A.; Horowitz, Kelsey; Barraud, Loris; Ward, J. Scott; Schnabel, Manuel (September 2017). "Raising the one-sun conversion efficiency of III–V/Si solar cells to 32.8\verb|%| for two junctions and 35.9 \verb|%| for three junctions". Nature Energy. 2 (9): 17144. Bibcode:2017NatEn...217144E. doi:10.1038/nenergy.2017.144. ISSN 2058-7546..
\item Janz, Stefan; Reber, Stefan (14 September 2015). "20\verb|%| Efficient Solar Cell on EpiWafer". Fraunhofer ISE. Retrieved 15 October 2015..
\item Chen, Wanghua; Cariou, Romain; Foldyna, Martin; Depauw, Valerie; Trompoukis, Christos; Drouard, Emmanuel; Lalouat, Loic; Harouri, Abdelmounaim; Liu, Jia; Fave, Alain; Orobtchouk, Régis; Mandorlo, Fabien; Seassal, Christian; Massiot, Inès; Dmitriev, Alexandre; Lee, Ki-Dong; Cabarrocas, Pere Roca i (2016). "Nanophotonics-based low-temperature PECVD epitaxial crystalline silicon solar cells". Journal of Physics D: Applied Physics. 49 (12): 125603. Bibcode:2016JPhD...49l5603C. doi:10.1088/0022-3727/49/12/125603. ISSN 0022-3727..
\item Kobayashi, Eiji; Watabe, Yoshimi; Hao, Ruiying; Ravi, T. S. (2015). "High efficiency heterojunction solar cells on n-type kerfless mono crystalline silicon wafers by epitaxial growth". Applied Physics Letters. 106 (22): 223504. Bibcode:2015ApPhL.106v3504K. doi:10.1063/1.4922196. ISSN 0003-6951..
\item Kim, D.S.; et al. (18 May 2003). String ribbon silicon solar cells with 17.8\verb|%| efficiency (PDF). Proceedings of 3rd World Conference on Photovoltaic Energy Conversion, 2003. 2. pp. 1293–1296. ISBN 978-4-9901816-0-4..
\item Wayne McMillan, "The Cast Mono Dilemma" Archived 5 11月 2013 at the Wayback Machine, BT Imaging.
\item Pearce, J.; Lau, A. (2002). "Net Energy Analysis for Sustainable Energy Production from Silicon Based Solar Cells" (PDF). Solar Energy. p. 181. doi:10.1115/SED2002-1051. ISBN 978-0-7918-1689-9. Archived from the original (PDF) on 2010-06-22..
\item Edoff, Marika (March 2012). "Thin Film Solar Cells: Research in an Industrial Perspective". AMBIO. 41 (2): 112–118. doi:10.1007/s13280-012-0265-6. ISSN 0044-7447. PMC 3357764. PMID 22434436..
\item Fthenakis, Vasilis M. (2004). "Life cycle impact analysis of cadmium in CdTe PV production" (PDF). Renewable and Sustainable Energy Reviews. 8 (4): 303–334. doi:10.1016/j.rser.2003.12.001..
\item "IBM and Tokyo Ohka Kogyo Turn Up Watts on Solar Energy Production", IBM.
\item Collins, R. W.; Ferlauto, A. S.; Ferreira, G. M.; Chen, C.; Koh, J.; Koval, R. J.; Lee, Y.; Pearce, J. M.; Wronski, C. R. (2003). "Evolution of microstructure and phase in amorphous, protocrystalline, and microcrystalline silicon studied by real time spectroscopic ellipsometry". Solar Energy Materials and Solar Cells. 78 (1–4): 143. doi:10.1016/S0927-0248(02)00436-1..
\item Pearce, J. M.; Podraza, N.; Collins, R. W.; Al-Jassim, M. M.; Jones, K. M.; Deng, J.; Wronski, C. R. (2007). "Optimization of open circuit voltage in amorphous silicon solar cells with mixed-phase (amorphous+nanocrystalline) p-type contacts of low nanocrystalline content" (PDF). Journal of Applied Physics. 101 (11): 114301–114301–7. Bibcode:2007JAP...101k4301P. doi:10.1063/1.2714507. Archived from the original (PDF) on 13 June 2009..
\item Yablonovitch, Eli; Miller, Owen D.; Kurtz, S. R. (2012). "The opto-electronic physics that broke the efficiency limit in solar cells". 2012 38th IEEE Photovoltaic Specialists Conference. p. 001556. doi:10.1109/PVSC.2012.6317891. ISBN 978-1-4673-0066-7..
\item "Photovoltaics Report" (PDF). Fraunhofer ISE. 28 July 2014. Archived (PDF) from the original on 31 August 2014. Retrieved 31 August 2014..
\item Oku, Takeo; Kumada, Kazuma; Suzuki, Atsushi; Kikuchi, Kenji (June 2012). "Effects of germanium addition to copper phthalocyanine/fullerene-based solar cells". Central European Journal of Engineering. 2 (2): 248–252. Bibcode:2012CEJE....2..248O. doi:10.2478/s13531-011-0069-7..
\item Triple-Junction Terrestrial Concentrator Solar Cells. (PDF) Retrieved 3 January 2012..
\item Clarke, Chris (19 April 2011) San Jose Solar Company Breaks Efficiency Record for PV. Optics.org. Retrieved 19 January 2011..
\item Cariou, Romain; Chen, Wanghua; Maurice, Jean-Luc; Yu, Jingwen; Patriarche, Gilles; Mauguin, Olivia; Largeau, Ludovic; Decobert, Jean; Roca i Cabarrocas, Pere (2016). "Low temperature plasma enhanced CVD epitaxial growth of silicon on GaAs: a new paradigm for III-V/Si integration". Scientific Reports. 6: 25674. Bibcode:2016NatSR...625674C. doi:10.1038/srep25674. ISSN 2045-2322. PMC 4863370. PMID 27166163..
\item Smith, David D.; Cousins, Peter; Westerberg, Staffan; Jesus-Tabajonda, Russelle De; Aniero, Gerly; Shen, Yu-Chen (2014). "Toward the Practical Limits of Silicon Solar Cells". IEEE Journal of Photovoltaics. 4 (6): 1465–1469. doi:10.1109/JPHOTOV.2014.2350695..
\item Almansouri, Ibraheem; Ho-Baillie, Anita; Bremner, Stephen P.; Green, Martin A. (2015). "Supercharging Silicon Solar Cell Performance by Means of Multijunction Concept". IEEE Journal of Photovoltaics. 5 (3): 968–976. doi:10.1109/JPHOTOV.2015.2395140..
\item Essig, Stephanie; Steiner, Myles A.; Allebe, Christophe; Geisz, John F.; Paviet-Salomon, Bertrand; Ward, Scott; Descoeudres, Antoine; Lasalvia, Vincenzo; Barraud, Loris; Badel, Nicolas; Faes, Antonin; Levrat, Jacques; Despeisse, Matthieu; Ballif, Christophe; Stradins, Paul; Young, David L. (2016). "Realization of GaInP/Si Dual-Junction Solar Cells with 29.8\verb|%| 1-Sun Efficiency". IEEE Journal of Photovoltaics. 6 (4): 1012–1019. doi:10.1109/JPHOTOV.2016.2549746..
\item Richter, Armin; Hermle, Martin; Glunz, Stefan W. (2013). "Reassessment of the Limiting Efficiency for Crystalline Silicon Solar Cells". IEEE Journal of Photovoltaics. 3 (4): 1184–1191. doi:10.1109/JPHOTOV.2013.2270351..
\item "NREL effiiciency chart". Archived from the original on 22 January 2016..
\item Kosasih, Felix Utama; Ducati, Caterina (May 2018). "Characterising degradation of perovskite solar cells through in-situ and operando electron microscopy". Nano Energy. 47: 243–256. doi:10.1016/j.nanoen.2018.02.055..
\item "Radiation energy transducing device". Mori Hiroshi, Hayakawa Denki Kogyo KK. 1961-10-03..
\item (A1) ES 453575 (A1)A. Luque: "Procedimiento para obtener células solares bifaciales" filing date 05.05.1977.
\item (A) US 4169738 (A)A. Luque: "Double-sided solar cell with self-refrigerating concentrator" filing date 21.11.1977.
\item Luque, A.; Cuevas, A.; Eguren, J. (1978). "Solar-Cell Behavior under Variable Surface Recombination Velocity and Proposal of a Novel Structure". Solid-State Electronics. 21 (5): 793–794. Bibcode:1978SSEle..21..793L. doi:10.1016/0038-1101(78)90014-X..
\item Cuevas, A.; Luque, A.; Eguren, J.; Alamo, J. del (1982). "50 Per cent more output power from an albedo-collecting flat panel using bifacial solar cells". Solar Energy. 29 (5): 419–420. Bibcode:1982SoEn...29..419C. doi:10.1016/0038-092x(82)90078-0..
\item "International Technology Roadmap for Photovoltaic (ITRPV) – Home". www.itrpv.net. Retrieved 2018-02-20..
\item Sun, Xingshu; Khan, Mohammad Ryyan; Deline, Chris; Alam, Muhammad Ashraful (2018). "Optimization and performance of bifacial solar modules: A global perspective". Applied Energy. 212: 1601–1610. arXiv:1709.10026. doi:10.1016/j.apenergy.2017.12.041..
\item Khan, M. Ryyan; Hanna, Amir; Sun, Xingshu; Alam, Muhammad A. (2017). "Vertical bifacial solar farms: Physics, design, and global optimization". Applied Energy. 206: 240–248. doi:10.1016/j.apenergy.2017.08.042..
\item Zhao, Binglin; Sun, Xingshu; Khan, Mohammad Ryyan; Alam, Muhammad Ashraful (2018-02-19). "Purdue Bifacial Module Calculator". doi:10.4231/d3542jb3c..
\item Luque, Antonio; Martí, Antonio (1997). "Increasing the Efficiency of Ideal Solar Cells by Photon Induced Transitions at Intermediate Levels". Physical Review Letters. 78 (26): 5014–5017. Bibcode:1997PhRvL..78.5014L. doi:10.1103/PhysRevLett.78.5014..
\item Okada, Yoshitaka, Tomah Sogabe, and Yasushi Shoji (2014). "Ch. 13: Intermediate Band Solar Cells". In Arthur J. Nozik, Gavin Conibeer, and Matthew C. Beard. Advanced Concepts in Photovoltaics. Energy and Environment Series. Vol. 11. Cambridge, UK: Royal Society of Chemistry. pp. 425–54. doi:10.1039/9781849739955-00425. ISBN 978-1-84973-995-5.CS1 maint: Multiple names: authors list (link).
\item Researchers use liquid inks to create better solar cells, Phys.org, 17 September 2014, Shaun Mason.
\item Hernández-Rodríguez, M.A.; Imanieh, M.H.; Martín, L.L.; Martín, I.R. (September 2013). "Experimental enhancement of the photocurrent in a solar cell using upconversion process in fluoroindate glasses exciting at 1480nm". Solar Energy Materials and Solar Cells. 116: 171–175. doi:10.1016/j.solmat.2013.04.023..
\item Dye Sensitized Solar Cells. G24i.com (2 April 2014). Retrieved 20 April 2014..
\item Sharma, Darshan; Jha, Ranjana; Kumar, Shiv (2016-10-01). "Quantum dot sensitized solar cell: Recent advances and future perspectives in photoanode". Solar Energy Materials and Solar Cells. 155: 294–322. doi:10.1016/j.solmat.2016.05.062. ISSN 0927-0248..
\item Semonin, O. E.; Luther, J. M.; Choi, S.; Chen, H.-Y.; Gao, J.; Nozik, A. J.; Beard, M. C. (2011). "Peak External Photocurrent Quantum Efficiency Exceeding 100\verb|%| via MEG in a Quantum Dot Solar Cell". Science. 334 (6062): 1530–3. Bibcode:2011Sci...334.1530S. doi:10.1126/science.1209845. PMID 22174246..
\item Kamat, Prashant V. (2012). "Boosting the Efficiency of Quantum Dot Sensitized Solar Cells through Modulation of Interfacial Charge Transfer". Accounts of Chemical Research. 45 (11): 1906–15. doi:10.1021/ar200315d. PMID 22493938..
\item Santra, Pralay K.; Kamat, Prashant V. (2012). "Mn-Doped Quantum Dot Sensitized Solar Cells: A Strategy to Boost Efficiency over 5\verb|%|". Journal of the American Chemical Society. 134 (5): 2508–11. doi:10.1021/ja211224s. PMID 22280479..
\item Moon, Soo-Jin; Itzhaik, Yafit; Yum, Jun-Ho; Zakeeruddin, Shaik M.; Hodes, Gary; GräTzel, Michael (2010). "Sb2S3-Based Mesoscopic Solar Cell using an Organic Hole Conductor". The Journal of Physical Chemistry Letters. 1 (10): 1524. doi:10.1021/jz100308q..
\item Du, Jun; Du, Zhonglin; Hu, Jin-Song; Pan, Zhenxiao; Shen, Qing; Sun, Jiankun; Long, Donghui; Dong, Hui; Sun, Litao; Zhong, Xinhua; Wan, Li-Jun (2016). "Zn–Cu–In–Se Quantum Dot Solar Cells with a Certified Power Conversion Efficiency of 11.6\verb|%|". Journal of the American Chemical Society. 138 (12): 4201–4209. doi:10.1021/jacs.6b00615. PMID 26962680..
\item Solar Cell Research || The Prashant Kamat lab at the University of Notre Dame. Nd.edu (22 February 2007). Retrieved 17 May 2012..
\item Genovese, Matthew P.; Lightcap, Ian V.; Kamat, Prashant V. (2012). "Sun-BelievableSolar Paint. A Transformative One-Step Approach for Designing Nanocrystalline Solar Cells". ACS Nano. 6 (1): 865–72. doi:10.1021/nn204381g. PMID 22147684..
\item Yu, Peng; Wu, Jiang; Gao, Lei; Liu, Huiyun; Wang, Zhiming (2017-03-01). "InGaAs and GaAs quantum dot solar cells grown by droplet epitaxy". Solar Energy Materials and Solar Cells. 161: 377–381. doi:10.1016/j.solmat.2016.12.024..
\item Wu, Jiang; Yu, Peng; Susha, Andrei S.; Sablon, Kimberly A.; Chen, Haiyuan; Zhou, Zhihua; Li, Handong; Ji, Haining; Niu, Xiaobin (2015-04-01). "Broadband efficiency enhancement in quantum dot solar cells coupled with multispiked plasmonic nanostars". Nano Energy. 13: 827–835. doi:10.1016/j.nanoen.2015.02.012..
\item Konarka Power Plastic reaches 8.3\verb|%| efficiency. pv-tech.org. Retrieved 7 May 2011..
\item Mayer, A.; Scully, S.; Hardin, B.; Rowell, M.; McGehee, M. (2007). "Polymer-based solar cells". Materials Today. 10 (11): 28. doi:10.1016/S1369-7021(07)70276-6..
\item Lunt, R. R.; Bulovic, V. (2011). "Transparent, near-infrared organic photovoltaic solar cells for window and energy-scavenging applications". Applied Physics Letters. 98 (11): 113305. Bibcode:2011ApPhL..98k3305L. doi:10.1063/1.3567516..
\item Rudolf, John Collins (20 April 2011). "Transparent Photovoltaic Cells Turn Windows Into Solar Panels". green.blogs.nytimes.com..
\item "UCLA Scientists Develop Transparent Solar Cell". Enviro-News.com. 24 July 2012. Archived from the original on 27 July 2012..
\item Lunt, R. R.; Osedach, T. P.; Brown, P. R.; Rowehl, J. A.; Bulović, V. (2011). "Practical Roadmap and Limits to Nanostructured Photovoltaics". Advanced Materials. 23 (48): 5712–27. doi:10.1002/adma.201103404. PMID 22057647..
\item Lunt, R. R. (2012). "Theoretical limits for visibly transparent photovoltaics". Applied Physics Letters. 101 (4): 043902. Bibcode:2012ApPhL.101d3902L. doi:10.1063/1.4738896..
\item Guo, C.; Lin, Y. H.; Witman, M. D.; Smith, K. A.; Wang, C.; Hexemer, A.; Strzalka, J.; Gomez, E. D.; Verduzco, R. (2013). "Conjugated Block Copolymer Photovoltaics with near 3\verb|%| Efficiency through Microphase Separation". Nano Letters. 13 (6): 2957–63. Bibcode:2013NanoL..13.2957G. doi:10.1021/nl401420s. PMID 23687903..
\item "Organic polymers create new class of solar energy devices". Kurzweil Accelerating Institute. 31 May 2013. Retrieved 1 June 2013..
\item Bullis, Kevin (30 July 2014) Adaptive Material Could Cut the Cost of Solar in Half. MIT Technology Review.
\item Campbell, Patrick; Green, Martin A. (Feb 1987). "Light Trapping Properties of Pyramidally textured surfaces". Journal of Applied Physics. 62 (1): 243–249. Bibcode:1987JAP....62..243C. doi:10.1063/1.339189..
\item Zhao, Jianhua; Wang, Aihua; Green, Martin A. (May 1998). "19.8\verb|%|efficient "honeycomb" textured multicrystalline and 24.4\verb|%|monocrystalline silicon solar cells". Applied Physics Letters. 73 (14): 1991–1993. Bibcode:1998ApPhL..73.1991Z. doi:10.1063/1.122345..
\item Hauser, H.; Michl, B.; Kubler, V.; Schwarzkopf, S.; Muller, C.; Hermle, M.; Blasi, B. (2011). "Nanoimprint Lithography for Honeycomb Texturing of Multicrystalline Silicon". Energy Procedia. 8: 648–653. doi:10.1016/j.egypro.2011.06.196..
\item Tucher, Nico; Eisenlohr, Johannes; Gebrewold, Habtamu; Kiefel, Peter; Höhn, Oliver; Hauser, Hubert; Goldschmidt, Jan Christoph; Bläsi, Benedikt (2016-07-11). "Optical simulation of photovoltaic modules with multiple textured interfaces using the matrix-based formalism OPTOS". Optics Express. 24 (14): A1083–A1093. Bibcode:2016OExpr..24A1083T. doi:10.1364/OE.24.0A1083. PMID 27410896..
\item Mavrokefalos, Anastassios; Han, Sang Eon.; Yerci, Selcuk; Branham, M.S.; Chen, Gang. (June 2012). "Efficient Light Trapping in Inverted Nanopyramid Thin Crystalline Silicon Membranes for Solar Cell Applications". Nano Letters. 12 (6): 2792–2796. Bibcode:2012NanoL..12.2792M. doi:10.1021/nl2045777. PMID 22612694..
\item Jaus, J.; Pantsar, H.; Eckert, J.; Duell, M.; Herfurth, H.; Doble, D. (2010). "Light management for reduction of bus bar and gridline shadowing in photovoltaic modules". 2010 35th IEEE Photovoltaic Specialists Conference. p. 000979. doi:10.1109/PVSC.2010.5614568. ISBN 978-1-4244-5890-5..
\item Mingareev, I.; Berlich, R.; Eichelkraut, T. J.; Herfurth, H.; Heinemann, S.; Richardson, M. C. (2011-06-06). "Diffractive optical elements utilized for efficiency enhancement of photovoltaic modules". Optics Express. 19 (12): 11397–404. Bibcode:2011OExpr..1911397M. doi:10.1364/OE.19.011397. PMID 21716370..
\item Uematsu, T; Yazawa, Y; Miyamura, Y; Muramatsu, S; Ohtsuka, H; Tsutsui, K; Warabisako, T (2001-03-01). "Static concentrator photovoltaic module with prism array". Solar Energy Materials and Solar Cells. PVSEC 11 – PART III. 67 (1–4): 415–423. doi:10.1016/S0927-0248(00)00310-X..
\item Chen, Fu-hao; Pathreeker, Shreyas; Kaur, Jaspreet; Hosein, Ian D. (2016-10-31). "Increasing light capture in silicon solar cells with encapsulants incorporating air prisms to reduce metallic contact losses". Optics Express. 24 (22): A1419–A1430. Bibcode:2016OExpr..24A1419C. doi:10.1364/oe.24.0a1419. PMID 27828526..
\item Korech, Omer; Gordon, Jeffrey M.; Katz, Eugene A.; Feuermann, Daniel; Eisenberg, Naftali (2007-10-01). "Dielectric microconcentrators for efficiency enhancement in concentrator solar cells". Optics Letters. 32 (19): 2789. Bibcode:2007OptL...32.2789K. doi:10.1364/OL.32.002789..
\item Hosein, Ian D.; Lin, Hao; Ponte, Matthew R.; Basker, Dinesh K.; Saravanamuttu, Kalaichelvi (2013-11-03). Enhancing Solar Energy Light Capture with Multi-Directional Waveguide Lattices. Renewable Energy and the Environment. pp. RM2D.2. doi:10.1364/OSE.2013.RM2D.2. ISBN 978-1-55752-986-2..
\item Biria, Saeid; Chen, Fu Hao; Pathreeker, Shreyas; Hosein, Ian D. (2017-12-22). "Polymer Encapsulants Incorporating Light-Guiding Architectures to Increase Optical Energy Conversion in Solar Cells". Advanced Materials. 30 (8): 1705382. doi:10.1002/adma.201705382. PMID 29271510..
\item Biria, Saeid; Chen, Fu-Hao; Hosein, Ian D. (2019). "Enhanced Wide-Angle Energy Conversion Using Structure-Tunable Waveguide Arrays as Encapsulation Materials for Silicon Solar Cells". Physica Status Solidi A. 0 (2): 1800716. Bibcode:2019PSSAR.21600716B. doi:10.1002/pssa.201800716..
\item Huang, Zhiyuan; Li, Xin; Mahboub, Melika; Hanson, Kerry M.; Nichols, Valerie M.; Le, Hoang; Tang, Ming L.; Bardeen, Christopher J. (2015-08-12). "Hybrid Molecule–Nanocrystal Photon Upconversion Across the Visible and Near-Infrared". Nano Letters. 15 (8): 5552–5557. Bibcode:2015NanoL..15.5552H. doi:10.1021/acs.nanolett.5b02130. PMID 26161875..
\item Schumann, Martin F.; Langenhorst, Malte; Smeets, Michael; Ding, Kaining; Paetzold, Ulrich W.; Wegener, Martin (2017-07-04). "All-Angle Invisibility Cloaking of Contact Fingers on Solar Cells by Refractive Free-Form Surfaces". Advanced Optical Materials. 5 (17): 1700164. doi:10.1002/adom.201700164..
\item Langenhorst, Malte; Schumann, Martin F.; Paetel, Stefan; Schmager, Raphael; Lemmer, Uli; Richards, Bryce S.; Wegener, Martin; Paetzold, Ulrich W. (2018-08-01). "Freeform surface invisibility cloaking of interconnection lines in thin-film photovoltaic modules". Solar Energy Materials and Solar Cells. 182: 294–301. doi:10.1016/j.solmat.2018.03.034..
\item Fitzky, Hans G. and Ebneth, Harold (24 May 1983) 美国专利 4,385,102, "Large-area photovoltaic cell".
\item Jäger-Waldau, Arnulf (September 2013) PV Status Report 2013. European Commission, Joint Research Centre, Institute for Energy and Transport..
[121]
^PV production grows despite a crisis-driven decline in investment. European Commission, Brussels, 30 September 2013.
\item PV Status Report 2013 | Renewable Energy Mapping and Monitoring in Europe and Africa (REMEA). Iet.jrc.ec.europa.eu (11 April 2014). Retrieved 20 April 2014..
\item "Solar Rises in Malaysia During Trade Wars Over Panels". New York Times. 12 December 2014..
\item Plunging Cost Of Solar PV (Graphs). CleanTechnica (7 March 2013). Retrieved 20 April 2014..
\item Falling silicon prices shakes up solar manufacturing industry. Down To Earth (19 September 2011). Retrieved 20 April 2014..
\item Jordan, Dirk C.; Kurtz, Sarah R. (June 2012). "Photovoltaic Degradation Rates – An Analytical Review" (PDF). Progress in Photovoltaics: Research and Applications. Retrieved 2019-03-06..
\item How long do solar panels last?. CleanTechnica (4 February 2019). Retrieved 2019-03-06..
\item End-of-Life Management: Solar Photovoltaic Panels. International Renewable Energy Agency (June 2016). Retrieved 2019-03-06..
\item More solar panels mean more waste and there’s no easy solution. The Verge (25 October 2018). Retrieved 2019-3-6..
\item If Solar Panels Are So Clean, Why Do They Produce So Much Toxic Waste?. Forbes (23 May 2018). Retrieved 2019-03-06..
\item Europe's First Solar Panel Recycling Plant Opens in France. Reuters (25 June 2018). Retrieved 2019-03-06..
\end{enumerate}
