% 泊松括号
% keys 哈密顿量|泊松括号|守恒量

\begin{issues}
\issueDraft
\end{issues}

\pentry{哈密顿正则方程\upref{HamCan}}

给定函数 $u(q, p)$ 和 $v(q, p)$, 定义泊松括号为
\begin{equation}
\pb{u}{v} = \sum_i \pdv{u}{q_i}\pdv{v}{p_i} - \pdv{v}{q_i}\pdv{u}{p_i}
\end{equation}

对任意物理量 $\omega (q,p)$,   都有
\begin{equation}\label{poison_eq1}
\dot \omega  = \sum_i \qty[ \pdv{\omega}{q_i} \dot q_i + \pdv{\omega}{p_i} \dot p_i ]  = \sum_i \qty[\pdv{\omega}{q_i} \pdv{H}{p_i} - \pdv{H}{q_i} \pdv{\omega}{p_i} ]  =  \pb{\omega}{H} 
\end{equation}
量子力学中的对易算符对应泊松括号. 注意该物理量不能显含时间, 即不能是 $\omega (q,p,t)$.  所以若泊松括号消失, 则该物理量守恒. 当 $\omega$ 显含时间时
\begin{equation}
\dot \omega  =  \pb{\omega}{H}  + \pdv{\omega}{t}
\end{equation}
对应量子力学中的算符平均值演化方程. 注意若调换泊松括号里面的物理量, 结果取相反数.
