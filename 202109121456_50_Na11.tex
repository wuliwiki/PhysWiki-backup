% 2011 年计算机学科专业基础综合全国联考卷
% keys 2011 | 计算机学科专业基础综合 | 全国联考卷

\subsection{一、单项选择题}
第1-40小题,每小题2分,共80分,下列每小题给出的四个选项中,只有一项符合题目要求的.请在答题卡上将所选项的字母涂黑.

1. 设$n$是描述问题规模的非负整数,下面程序片段的时间复杂度是  \\
$\qquad$ x = 2; \\
$\qquad$ while(x < n/2) \\
$\qquad$ $\quad$ x = 2*x; \\
A. $O(log_2n)$ $\quad$ B. $O(n)$ $\quad$ C. $O(nlog_2n)$ $\quad$ D. $O(n^2)$

2. 元素$a$,$b$,$c$,$d$,$e$依次进入初始为空的栈中,若元素进栈后可停留、可出栈,直到所有元素都出栈,则在所有可能的出栈序列中,以元素$d$开头的序列个数是  \\
A. 3  $\quad$  B. 4  $\quad$  C. 5  $\quad$  D. 6

3. 已知循环队列存储在一维数组$A[0...n-1]$中,且队列非空时front和rear分别指向队头元素和队尾元素.若初始时队列为空,且要求第1个进入队列的元素存储在$A[0]$处,则初始时front和rear的值分别是 \\
A. $0$,$0$  $\quad$ B. $0$,$n-1$ $\quad$ C. $n-1$,$0$ $\quad$ D. $n-1$,$n-1$ 

4. 若一棵完全二叉树有768个结点,则该二叉树中叶结点的个数是 \\
A. 257 $\quad$ B. 258 $\quad$ C. 384 $\quad$ D. 385

5. 若一棵二叉树的前序遍历序列和后序遍历序列分别为1,2,3,4和4,3,2,1,则该二叉树的中序遍历序列不会是 \\
A.1,2,3,4 $\quad$ B.2,3,4,1 $\quad$ C.3,2,4,1 $\quad$ D.4,3,2,1

6. 已知一棵有2011个结点的树,其叶结点个数为116,该树对应的二叉树中无右孩子的结点个数是 \\
A.115 $\quad$ B.116 $\quad$ C.1895 $\quad$ D.1896

7. 对于下列关键字序列,不可能构成某二叉排序树中一条查找路径的序列是 \\
A.95,22,91,24,94,71 $\quad$ B.92,20,91,34,88,35 \\
C.21,89,77,29,36,38 $\quad$ D.12,25,71,68,33,34

8. 下列关于图的叙述中,正确的是 \\
Ⅰ. 回路是简单路径 \\
Ⅱ.存储稀疏图,用邻接矩阵比邻接表更省空间 \\
Ⅲ.若有向图中存在拓扑序列,则该图不存在回路 \\
A.仅Ⅱ $\quad$ B.仅Ⅰ、Ⅱ $\quad$  C.仅Ⅲ $\quad$ D.仅Ⅰ、Ⅲ

9. 为提高散列(Hash)表的查找效率,可以采取的正确措施是 \\
Ⅰ. 增大装填(载)因子 \\
Ⅱ.设计冲突(碰撞)少的散列函数 \\
Ⅲ.处理冲突(碰撞)时避免产生聚集(堆积)现象 \\
A.仅Ⅰ $\quad$ B.仅Ⅱ $\quad$ C.仅Ⅰ、Ⅱ $\quad$ D.仅Ⅱ、Ⅲ

10.为实现快速排序算法,待排序序列宜采用的存储方式是 \\
A.顺序存储 $\quad$ B.散列存储 $\quad$ C.链式存储 $\quad$ D.索引存储

11. 已知序列25,13,10,12,9是大根堆,在序列尾部插入新元素18,将其再调整为大根堆,调整过程中元素之间进行的比较次数是 \\
A.1 $\quad$  B.2 $\quad$ C.4 $\quad$ D.5

12.下列选项中,描述浮点数操作速度指标的是 \\
A.MIPS $\quad$ B.CPI $\quad$ C.IPC $\quad$ D.MFLOPS

13.float型数据通常用IEEE 754单精度浮点数格式表示.若编译器将float型变量$x$分配在一个$32$位浮点寄存器FR1中,且$x$=$-8.25$,则FR1的内容是 \\
A.C104 0000H $\quad$ B.C242 0000H $\quad$ C.C184 0000H $\quad$ D.C1C2 0000H

14.下列各类存储器中,不采用随机存取方式的是 \\
A.EPROM $\quad$ B.CDROM $\quad$ C.DRAM $\quad$ D.SRAM

15.某计算机存储器按字节编址,主存地址空间大小为64MB,现用4M×8位的RAM芯片组成32MB的主存储器,则存储器地址寄存器MAR的位数至少是 \\
A.22位 $\quad$ B.23位 $\quad$ C.25位 $\quad$ D.26位

16.偏移寻址通过将某个寄存器内容与一个形式地址相加而生成有效地址.下列寻址方式中,\textbf{不}属于偏移寻址方式的是 \\
A.间接寻址 $\quad$ B.基址寻址 $\quad$ C.相对寻址 $\quad$ D.变址寻址

17.某机器有一个标志寄存器,其中有进位/借位标志CF、零标志ZF、符号标志SF和溢出标志OF,条件转移指令bgt(无符号整数比较大于时转移)的转移条件是 \\
A.$CF+OF=1$ $\quad$ B.$\overline{SF}+ZF=1$ $\quad$ C.$\overline{CF+ZF}=1$ $\quad$ D.$\overline{CF+SF}=1$

18.下列给出的指令系统特点中,有利于实现指令流水线的是 \\
Ⅰ. 指令格式规整且长度一致 \\
Ⅱ.指令和数据按边界对齐存放 \\
Ⅲ.只有Load/Store指令才能对操作数进行存储访问 \\
A.仅Ⅰ、Ⅱ $\quad$ B.仅Ⅱ、Ⅲ $\quad$ C.仅Ⅰ、Ⅲ $\quad$ D.Ⅰ、Ⅱ、Ⅲ

19.假定不采用Cache和指令预取技术,且机器处于“开中断”状态,则在下列有关指令执行的叙述中,\textbf{错误}的是 \\
A.每个指令周期中CPU都至少访问内存一次 \\
B.每个指令周期一定大于或等于一个CPU时钟周期 \\
C.空操作指令的指令周期中任何寄存器的内容都不会被改变 \\
D.当前程序在每条指令执行结束时都可能被外部中断打断

20.在系统总线的数据线上,不可能传输的是 \\
A.指令 $\quad$ B.操作数 $\quad$ C.握手(应答)信号 $\quad$ D.中断类型号

21.某计算机有五级中断$L4$~$L0$,中断屏蔽字为$M_4M_3M_2M_1M_0$,$M_i=1$(0≤i≤4)表示对$L_i$级中断进行屏蔽.若中断响应优先级从高到低的顺序是$L4$→$L0$→$L2$→$L1$→$L3$,则$L_1$的中断处理程序中设置的中断屏蔽字是 \\
A.11110 $\quad$ B.01101 $\quad$ C.00011 $\quad$ D.01010

22.某计算机处理器主频为50MHz,采用定时查询方式控制设备A的I/O,查询程序运行一次所用的时钟周期数至少为500.在设备A工作期间,为保证数据不丢失,每秒需对其查询至少200次,则CPU用于设备A的I/O的时间占整个CPU时间的百分比至少是 \\
A.0.02\%  $\quad$ B.0.05\%  $\quad$ C.0.20\%  $\quad$ D.0.50\%

23.下列选项中,满足短任务优先且不会发生饥饿现象的调度算法是 \\
A.先来先服务 $\quad$ B.高响应比优先 $\quad$ C.时间片轮转 $\quad$ D.非抢占式短任务优先

24.下列选项中,在用户态执行的是
A.命令解释程序 $\quad$ B.缺页处理程序 $\quad$ C.进程调度程序 $\quad$ D.时钟中断处理程序

25.在支持多线程的系统中,进程P创建的若干个线程不能共享的是
A.进程P的代码段 $\quad$ B.进程P中打开的文件 $\quad$ C.进程P的全局变量 $\quad$ D.进程P中某线程的栈指针

26.用户程序发出磁盘I/O请求后,系统的正确处理流程是 \\
A.用户程序→系统调用处理程序→中断处理程序→设备驱动程序 \\
B.用户程序→系统调用处理程序→设备驱动程序→中断处理程序 \\
C.用户程序→设备驱动程序→系统调用处理程序→中断处理程序 \\
D.用户程序→设备驱动程序→中断处理程序→系统调用处理程序

27.

28. 在缺页处理过程中,操作系统执行的可能是 \\
Ⅰ. 修改页表 $\quad$ Ⅱ.磁盘I/O $\quad$ Ⅲ.分配页框 \\
A.仅Ⅰ、Ⅱ $\quad$ B.仅Ⅱ $\quad$ C.仅Ⅲ $\quad$ D.Ⅰ、Ⅱ和Ⅲ

29 .当系统发生抖动( .当系统发生抖动( thrashing thrashing )时,可用采取的有效措施是 ) \\
Ⅰ. 撤销部分进程 \\
Ⅱ.增加磁盘交换区的容量 \\
Ⅲ.提高用户进程的优先级 \\
A.仅Ⅰ $\quad$ B.仅Ⅱ $\quad$ C.仅Ⅲ $\quad$ D.仅Ⅰ、Ⅱ

30.在虚拟内存管理中,地址变换机构将逻辑变换为物理地址,形成该逻辑地址的阶段是 \\
A.编辑 $\quad$ B.编译 $\quad$ C.链接 $\quad$ D.装载

31.某文件占$10$个磁盘块,现要把该文件磁盘块逐个读入主存缓冲区,并送用户区进行分析,假设一个缓冲区与一个磁盘块大小相同,把一个磁盘块读入缓冲区的时间为$100\mu s$,将缓冲区的数据传送到用户区的时间是$50\mu s$,CPU对一块数据进行分析的时间为$50\mu s$.在单缓冲区和双缓冲区结构下,读入并分析完该文件的时间分别是 \\
A.$1500\mu s$、$1000\mu s$ $\quad$ B.$1550\mu s$、$1100\mu s$ \\
C.$1550\mu s$、$1550\mu s$ $\quad$ D.$2000\mu s$、$2000\mu s$

32.有两个并发执行的进程$P1$和$P2$,共享初值为$1$的变量$x$.$P1$对$x$加$1$,$P2$对$x$减$1$.加$1$和减$1$操作的指令序列分别如下所示. \\

// 加1操作 \\
load R1,x  // 取$x$到寄存器$R1$中 \\
inc R1 \\
store x,R1  // 将$R1$的内容存入$x$ \\

// 减1操作 \\
load R2,x // 取$x$到寄存器$R2$中 \\
dec R2 \\
store x,R2 // 将$R2$的内容存入$x$ \\

两个操作完成后,$x$的值 \\
A.可能为-1或3 \\
B.只能为1 \\
C.可能为0、1或2 \\
D.可能为-1、0、1或2

33.TCP/IP参考模型的网络层提供的是 \\
A.无连接不可靠的数据报服务 $\quad$ B.无连接可靠的数据报服务 \\
C.有连接不可靠的虚电路服务 $\quad$ D.有连接可靠的虚电路服务

34.若某通信链路的数据传输速率为2400bps,采用4相位调制,则该链路的波特率是 \\
A.600波特 $\quad$ B.1200波特 $\quad$ C.4800波特 $\quad$ D.9600波特








\subsection{二、综合应用题}
第41~47小题,共70分.请将答案写在答题纸指定位置上.

41.(8分)已知有$6$个顶点(顶点编号为为$0$~$5$)的有向带权图$G$,其邻接矩阵$A$为上三角矩阵,按行为主序(行优先)保存在如下的一维数组中.
\begin{table}[ht]
\centering
\caption{第41题图}\label{Na11_tab1}
\begin{tabular}{|c|c|c|c|c|c|c|c|c|c|c|c|c|c|c|}
\hline
$4$ & $6$ & $\infty$ & $\infty$ & $\infty$ & $5$ & $\infty$ & $\infty$ & $\infty$ & $4$ & $3$ & $\infty$ & $\infty$ & $3$ & $3$ \\
\hline
\end{tabular}
\end{table}
要求:  \\
(1)写出图$G$的邻接矩阵$A$.  \\
(2)画出有向带权图$G$.  \\
(3)求图$G$的关键路径,并计算该关键路径的长度.




\subsection{参考答案}
\subsection{一、单项选择题}
1. 解答:A.程序中,执行频率最高的语句为“$x=2*x$”.设该语句执行了$t$次,则$2t+1=n/2$, 故$t=log2(n/2)-1=log2n-2= O(log2n)$.

2. 解答:B.出栈顺序必为d_c_b_a_,e的顺序不定,在任意一个“_”上都有可能.

3. 解答:B.插入元素时,front不变,rear+1.而插入第一个元素之后,队尾要指向尾元素,显然,rear初始应该为$n-1$,front为$0$.

4. 解答:C.叶结点数为$n$,则度为2的结点数为$n-1$,度为1的结点数为0或1,本题中为1(总结点数为偶数),故而即$2n=768$.

5. 解答:C.由前序和后序遍历序列可知3为根结点,故(1,2)为左子树,(4)为右子树,C不可能.或画图即可得出结果.

6. 解答:D.本题可采用特殊情况法解.设题意中的树是如下图所示的结构,则对应的二叉树中仅有前115个叶结点有右孩子.\\
(还有一个插图需要添加)

7. 解答:A.选项A中,当查到91后再向24查找,说明这一条路径之后查找的数都要比91小,后面的94就错了.

8. 解答:C.Ⅰ.回路对应于路径,简单回路对应于简单路径;Ⅱ.刚好相反;Ⅲ.拓扑有序的必要条件.故选C.

9. 解答:B.III错在“避免”二字.

10. 解答:A.内部排序采用顺序存储结构.

11. 解答:B.首先与10比较,交换位置,再与25比较,不交换位置.比较了二次.

12. 解答:D.送分题.

13. 解答:A.$x$的二进制表示为$-1000.01$﹦$-1.000 01$×$2^{11}$ 根据IEEE754标准隐藏最高位的“$1$”,又$E-127=3$,所以$E=130=1000 0010_{(2)}$数据存储为$1$位数符+$8$位阶码(含阶符)+$23$位尾数.
故FR1内容为1 10000 0010 0000 10000 0000 0000 0000 000,即1100 0001 0000 0100 0000 0000 0000 0000,即C104000H.

14. 解答:B.光盘采用顺序存取方式.

15. 解答:D.64MB的主存地址空间,故而MAR的寻址范围是64M,故而是26位.而实际的主存的空间不能代表MAR的位数.

16. 解答:A.间接寻址不需要寄存器,EA=(A).基址寻址:EA=A+基址寄存器内同;相对寻址:EA﹦A+PC内容;变址寻址:EA﹦A+变址寄存器内容.

17. 解答:C.无符号整数比较,如A>B,则A-B无进位/借位,也不为0.故而CF和ZF均为0.

18. 解答:D.指令定长、对齐、仅Load/Store指令访存,以上三个都是RISC的特征.均能够有效的简化流水线的复杂度.

19. 解答:C.会自动加1,A取指令要访存、B时钟周期对指令不可分割.

20. 解答:C.握手(应答)信号在通信总线上传输.

21. 解答:D.高等级置0表示可被中断,比该等级低的置1表示不可被中断.

22. 解答:C.每秒200次查询,每次500个周期,则每秒最少200×500﹦10 0000个周期,100000÷50M=0.20\%.

23. 解答:B.响应比=作业响应时间/作业执行时间=(作业执行时间+作业等待时间)/作业执行时间.高响应比算法,在等待时间相同情况下,作业执行时间越少,响应比越高,优先执行,满足短任务优先.随着等待时间增加,响应比也会变大,执行机会就增大,所以不会产生饥饿现象.先来先服务和时间片轮转不符合短任务优先,非抢占式短任务优先会产生饥饿现象.

24. 解答:A.缺页处理程序和时钟中断都属于中断,在核心态执行.进程调度属于系统调用在核心态执行,命令解释程序属于命令接口,它在用户态执行.

25. 解答:D.进程中某线程的栈指针,对其它线程透明,不能与其它线程共享.

26. 解答:B.输入/输出软件一般从上到下分为四个层次:用户层、与设备无关软件层、设备驱动程序以及中断处理程序.与设备无关软件层也就是系统调用的处理程序.所以争取处理流程为B选项.

27. 

28. 解答:D.缺页中断调入新页面,肯定要修改页表项和分配页框,所以I、III可能发生,同时内存没有页面,需要从外存读入,会发生磁盘I/O.

29. 解答:A.在具有对换功能的操作系统中,通常把外存分为文件区和对换区.前者用于存放文件,后者用于存放从内存换出的进程.抖动现象是指刚刚被换出的页很快又要被访问为此,又要换出其他页,而该页又快被访问,如此频繁的置换页面,以致大部分时间都花在页面置换上.撤销部分进程可以减少所要用到的页面数,防止抖动.对换区大小和进程优先级都与抖动无关.

30. 解答:B.编译过程指编译程序将用户源代码编译成目标模块.源地址编译成目标程序时,会形成逻辑地址.

31. 解答:B.单缓冲区下当上一个磁盘块从缓冲区读入用户区完成时下一磁盘块才能开始读入,也就是当最后一块磁盘块读入用户区完毕时所用时间为150×10=1500.加上处理最后一个磁盘块的时间50为1550.双缓冲区下,不存在等待磁盘块从缓冲区读入用户区的问题,也就是100×10+100=1100.

32. 解答:C.将P1中3条语句变为1,2,3,P2中3条语句编为4,5,6.则依次执行1,2,3,4,5得结果1,依次执行1,2,4,5,6,3得结果2,执行4,5,1,2,3,6得结果0.结果-1不可能得出,选C.

33. 解答:A.TCP/IP的网络层向上只提供简单灵活的、无连接的、尽最大努力交付的数据报服务.此外考察IP首部,如果是面向连接的,则应有用于建立连接的字段,但是没有;如果提供可靠的服务,则至少应有序号和校验和两个字段,但是IP分组头中也没有(IP首部中只是首部校验和).因此网络层提供的无连接不可靠的数据服务.有连接可靠的服务由传输层的TCP提供

34. 解答:B.有4种相位,则一个码元需要由log24=2个bit表示,则波特率=比特率/2=1200波特.


\subsection{二、综合应用题}
