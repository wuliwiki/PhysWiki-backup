% 希尔伯特旅馆悖论(综述)
% license CCBYSA3
% type Wiki

本文根据 CC-BY-SA 协议转载翻译自维基百科\href{https://en.wikipedia.org/wiki/Hilbert\%27s_paradox_of_the_Grand_Hotel}{相关文章}。

希尔伯特的大酒店悖论(口语化:无限酒店悖论或希尔伯特的酒店)是一个思想实验,用来说明无限集合的一个违反直觉的性质。实验表明,即使一个完全占满的酒店有无限多的房间,它仍然可以容纳更多的客人,甚至是无限多的客人,而且这一过程可以无限次地重复。这个想法由大卫·希尔伯特在1925年的讲座《关于无限》中提出,并在《希尔伯特2013年论文集》(第730页)中重印,后来通过乔治·伽莫夫1947年的书籍《一二三……无限》得到了广泛传播。[1][2]
\subsection{悖论}  
希尔伯特设想了一个假想的酒店,房间按1、2、3等顺序编号,且没有上限。这被称为可数无限数量的房间。最初,每个房间都已被占满,但新客人到来,每个人都希望拥有自己的房间。一个正常的、有限的酒店在每个房间都已满员时无法容纳新客人。然而,可以证明,现有的客人和新来的客人——即使是无限多的——都可以在这个无限酒店里各自拥有一个房间。
\subsubsection{有限数量的新客人}  
如果有一个额外的客人,酒店可以容纳他们和现有的客人,只需要让所有现有客人同时换房。现在在1号房间的客人移到2号房间,在2号房间的客人移到3号房间,依此类推,把每个客人从他们当前的房间n移到房间n+1。由于无限酒店没有最终房间,所以每个客人都有新的房间可以去。这样一来,1号房间就空了,新客人可以被安排到这个房间。通过重复这个过程,就可以为任何有限数量的新客人腾出房间。一般来说,当有k个客人需要房间时,酒店可以应用相同的程序,把每个客人从房间n移到房间n+k。
\subsubsection{无限多的新客人}
\begin{figure}[ht]
\centering
\includegraphics[width=10cm]{./figures/6d84c164267a5248.png}
\caption{通过将每个客人移动到他们之前房间号的两倍位置,可以容纳无限数量的新客人。} \label{fig_HPGH_1}
\end{figure}
同样也可以容纳可数无限多的新客人:只需将占用1号房间的人移到2号房间,将占用2号房间的人移到4号房间,通常,将占用n号房间的人移到2n号房间(n的两倍),这样所有奇数编号的房间(它们是可数无限的)就会空出来,供新客人使用。
\subsubsection{无限多的车厢,每个车厢有无限多的乘客 } 
可以通过几种不同的方法容纳可数无限多的车厢,每个车厢有可数无限多的乘客。大多数方法依赖于车厢中的座位已经编号(或使用可数选择公理)。一般来说,任何配对函数都可以用来解决这个问题。对于这些方法中的每一种,考虑乘客在车厢中的座位号为 \( n \),车厢号为 \( c \),然后将 \( n \) 和 \( c \) 输入配对函数的两个参数。

\textbf{质数幂法}  

将房间 \( i \) 的客人送到房间 \( 2^i \),然后将第一车厢的乘客放入房间 \( 3^n \),第二车厢的乘客放入房间 \( 5^n \);一般来说,对于车厢号 \( c \),我们使用房间 \( p_c^n \),其中 \( p_c \) 是第 \( c \) 个奇质数。这个解决方案会留下某些空房间(这些空房间可能对酒店有用,也可能无用);具体来说,所有不是质数幂的数字,比如15或847,将不再被占用。(因此,严格来说,这表明到达的客人数量小于或等于创造的空房间数量。通过独立的方式更容易证明,到达的客人数量也大于或等于空房间数量,因此它们是相等的,而不是修改算法使其精确匹配。)  
(如果交换 \( n \) 和 \( c \),该算法同样有效,但无论选择哪个,都必须在整个过程中统一应用。)

\textbf{质因数分解法 } 

每个乘客根据其座位 \( s \) 和车厢 \( c \) 可以被安排到房间 \( 2^s 3^c \)(假设对于已经在酒店的客人,\( c = 0 \),对于第一车厢的乘客,\( c = 1 \),依此类推)。由于每个数字都有唯一的质因数分解,可以很容易地看出所有人都会有自己的房间,且没有两个人会被安排到同一个房间。例如,房间2592(\( 2^5 3^4 \))的客人原本坐在第四车厢的第五个座位。像质数幂方法一样,这种解决方案也会留下某些空房间。

这种方法也可以很容易地扩展到无限晚、无限次进出等情况(\( 2^s 3^c 5^n 7^e \dots \))。