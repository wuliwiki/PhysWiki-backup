% 自旋与有限转动
% license Xiao
% type Tutor

\begin{issues}
\issueMissDepend
\issueTODO
\end{issues}

\subsubsection{空间转动与角动量生成元}
在物理里,常有系统“主动旋转”与坐标系“被动旋转”之分。如下图所示,系统$P$和附着在坐标系上的$Q$点到原点的距离相同。因此,若系统要到达$Q$点,可以绕原点顺时针转动$\phi$角,或者坐标系逆时针转动$\phi$。
\begin{figure}[ht]
\centering
\includegraphics[width=6cm]{./figures/41ed15d994cd1884.png}
\caption{} \label{fig_spinqm_1}
\end{figure}
以右手定则确定$z$轴,现在我们将系统$P$逆时针转动$\phi$。设原坐标为$\bvec v=a\bvec i+b\bvec j+c\bvec k$,新坐标为$\bvec {v'}=a\bvec {i'}+b\bvec {j'}+c\bvec k$。稍加计算可知$\bvec {i'}=\pmat{\cos \phi &\sin {\phi}}^T,\bvec {j'}=\pmat{-\sin \phi &\cos {\phi}}^T$。所以
\begin{equation}
\bvec{v'}\equiv R_z(\phi)\bvec v=\pmat{\cos\phi&-\sin\phi&0\\\sin\phi&\cos \phi&0\\0&0&1}\bvec v~,
\end{equation}
同理可得
\begin{equation}
R_x(\phi)=\pmat {1&0&0\\0&\cos\phi&-\sin\phi\\0&\sin\phi&\cos\phi},R_y(\phi)=\pmat {\cos\phi&0&\sin\phi\\0&1&0\\-\sin\phi&0&\cos\phi}~.
\end{equation}

\subsubsection{量子力学的角动量生成元}
以自旋$1/2$的粒子为例,其自旋期望值为$(\overline{\hat S_x},\overline{\hat S_y},\overline{\hat S_z})$。设该粒子的初始态矢为$\ket{a}$,态矢绕$z$轴“转动”后变为$\mathrm e^{- \I\hat S_z\phi}\ket{a}$。

则期望值变化为:

\begin{equation}
\bra{a}\hat S_i\ket{a}\rightarrow \bra{a}\mathrm e^{ \I\hat S_z\phi}\hat S_i\mathrm e^{- \I\hat S_z\phi}\ket{a}~.
\end{equation}
在$\hat S_z$表象下计算$\mathrm e^{ \I\hat S_z\phi}\hat S_x\mathrm e^{- \I\hat S_z\phi}$得:

\begin{equation}
\begin{aligned}
\mathrm e^{ \I\hat S_z\phi}\hat S_x\mathrm e^{- \I\hat S_z\phi}&=\mathrm e^{ \I\hat S_z\phi}\left(\frac{1}{2}(\ket{-}\bra{+}+\ket{+}\bra{-})\right)\mathrm e^{- \I\hat S_z\phi}\\
 &=\frac{1}{2}\left(\mathrm e^{-\mathrm i t}\ket{-}\bra{+}+\ket{+}\bra{-}\mathrm e^{\mathrm i t}\right)\\
 &=\frac{1}{2}\left[\cos(\phi)(\ket{-}\bra{+}+\ket{+}\bra{-})+\mathrm i\sin(\phi)(\ket{+}\bra{-}-\ket{-}\bra{+})\right]\\
 &=\cos(\phi) \hat S_x-\sin(\phi) \hat S_y~.
\end{aligned}
\end{equation}
因此,$\hat S_x$的期望值变化为:
\begin{equation}
\overline{\hat S_x}\rightarrow  \overline{\hat S_x}\cos(\phi)-\overline{\hat S_y}\sin(\phi)~.
\end{equation}
同理可以计算出其他分量的期望值变化:
\begin{equation}
\overline{\hat S_y}\rightarrow \overline{\hat S_y}\cos(\phi)+\overline{\hat S_x}\sin(\phi)~,
\end{equation}
\begin{equation}
\overline{\hat S_z}\rightarrow \overline{\hat S_z}~.
\end{equation}
因此,自旋期望值可看作经典矢量,态矢绕自旋$z$分量“旋转”相当于该矢量绕自旋$z$分量“旋转”:
\begin{equation}
\begin{pmatrix}
 \cos(\phi) &-\sin(\phi)  &0 \\
  \sin(\phi) & \cos(\phi)  & 0\\
  0& 0 &1
\end{pmatrix}
\begin{pmatrix}
 \overline{\hat S_x}\\
  \overline{\hat S_y}\\
 \overline{\hat S_z}
\end{pmatrix}
=
\begin{pmatrix}
  \overline{\hat S'_x}\\
  \overline{\hat S'_y}\\
 \overline{\hat S'_z}
\end{pmatrix}~.
\end{equation}
可以利用贝克-豪斯多夫(Baker-Hausdorff)公式计算$\mathrm e^{ \I\hat S_z\phi}\hat S_x\mathrm e^{- \I\hat S_z\phi}$。
计算过程表明自旋期望值的变化适用于任意角动量期望值的变化(即也适用于轨道角动量算子期望值)。



