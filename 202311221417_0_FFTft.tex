% 傅里叶变换的数值计算、快速傅里叶变换(Matlab)
% license Xiao
% type Tutor

\pentry{离散傅里叶变换\upref{DFT}, Matlab 画图\upref{MatPlt}}

\subsection{直接数值积分}
作为下文 FFT 方法的参照, 我们先实现直接用数值积分计算傅里叶变换(\autoref{eq_FTExp_6}~\upref{FTExp})。 调用时需要提供一元函数(句柄) \verb|f|, 而不是一系列离散函数值。 \verb|xspan| 是对 \verb|f| 积分的区间, 而 \verb|kspan| 是输出中 \verb|k| 的区间, \verb|Nk| 是 \verb|k| 的长度。 这么做虽然直观且精确, 但计算量较大, 所以一般还是用下一节中的 FFT 方法。

尤其是如果 \verb|f| 并不是通过函数给出, 而只是一系列等间距的散点, 那么与其先插值再做数值积分, FFT 方法是最适合的, 因为 FFT 已经相当于对散点进行了 sinc 插值(\autoref{sub_DFT_2}~\upref{DFT})。
\addTODO{搞几个具体的例子, 对比数值积分和 FFT 的结果。}

\begin{lstlisting}[language=matlab, caption=CFT.m]
% Continuous Fourier Transform by Integration
% f must be a function handle
% gh is function handle, g = gh(linspace(kmin,kmax,Nk))
% input the 7th argument to plot spectrum
function [k,g,gh] = CFT(f,xspan,kspan,Nk,~)
k = linspace(kspan(1),kspan(2),Nk);
g = zeros(1,Nk);
for ii = 1:Nk
    integrand = @(x)  f(x).*exp(-1i*k(ii)*x);
    g(ii) = integral(integrand,xspan(1),xspan(2), 'AbsTol',1e-8);
end
g = g/sqrt(2*pi);

if nargin == 5
    figure; plot(k,g);
end

if nargout == 3
    gh = @(kq) interp1(k,g,kq,'spline');
end
end
\end{lstlisting}
同理, 可以用数值积分计算反傅里叶变换
\begin{lstlisting}[language=matlab, caption=iCFT.m]
% Continuous Fourier Transform by Integration
% g must be a function handle
% fh is function handle, f = fh(linspace(xmin,xmax,Nx))
% input the 7th argument to plot spectrum

function [x,f,fh] = iCFT(g,kspan,xspan,Nx,~)
x = linspace(xspan(1),xspan(2),Nx);
f = zeros(1,Nx);
for ii = 1:Nx
    integrand = @(k)  g(k).*exp(1i*k*x(ii));
    f(ii) = integral(integrand,kspan(1),kspan(2), 'AbsTol',1e-8);
end
f = f/sqrt(2*pi);

if nargin == 5
    figure; plot(x,f);
end

if nargout == 3
    fh = @(xq) interp1(x,f,xq,'spline');
end
end
\end{lstlisting}

\subsection{用 FFT 近似傅里叶变换}
这里使用的算法见\autoref{sub_DFT_1}~\upref{DFT}。 给出任意等间距的 $x$ 坐标格点 \verb|[x0, x0+dx, x0+2*dx, ...]|, 以及对应的函数值 \verb|f = [f(1), f(2), ...]|, 那么该代码可以通过 Matlab 提供的快速傅里叶变换(FFT) 计算傅里叶变换(\autoref{eq_FTExp_6}~\upref{FTExp})。 输入中 \verb|Nk| 是可选的,默认等于 \verb|x| 的个数。 若 \verb|Nk| 大于 \verb|f| 的个数, 输出中 \verb|k| 的步长将会相应变小使 \verb|k| 的长度为 \verb|Nk|, 但区间不会变。 \verb|k| 的区间是由 \verb|dx| 决定的。 在实现上, 当 \verb|Nk > numel(f)| 时会预先在 \verb|f| 两边添加 0 使其先具有 \verb|Nk| 个元。
\begin{lstlisting}[language=matlab, caption=FFT.m]
% fft approximation of the analytical fourier transform from f(x) to g(k)
% x and k are both equally spaced, x starts from x0 equally spaced by dx
% norm(g) = norm(f)
% numel(g) = Nk

function [g, k] = my_FFT(f, x0, dx, Nk, k_mid)
N = numel(f);
if ~isvector(f)
    error('f must be a vector!');
end
if ~exist('k_mid', 'var')
    k_mid = 0;
end
if ~exist('Nk', 'var')
    Nk = N;
end
f = reshape(f, 1, N);
x_mid = x0 + ceil((N-1)/2)*dx;
if k_mid ~= 0
    x = linspace(x0, x0+dx*(N-1), N);
    f = f .* exp(-1i*k_mid*(x-x_mid));
end
if Nk > N
    N = Nk;
    f = fftresize(f, N);
end
g = sffts(f)*(dx/sqrt(2*pi));
k = fftlinspace(2*pi/dx, N) + k_mid;
if (abs(x_mid/x0) > 2*eps)
    g = g .* exp(-1i*k*x_mid);
end
end
\end{lstlisting}
对应的反傅里叶变换如下
\begin{lstlisting}[language=matlab, caption=iFFT.m]
% ifft approximation of the analytical inverse Fourier transform
%      from g(k) to f(x)
% x and k are both equally spaced, k starts from k0 equally spaced by dk
% norm(g) = norm(f)
% numel(f) = Nx

function [f, x] = iFFT(g, k0, dk, Nx, x_mid)
N = numel(g);
if ~isvector(g)
    error('g must be a vector!');
end
if ~exist('x_mid', 'var')
    x_mid = 0;
end
if ~exist('Nx', 'var')
    Nx = N;
end
g = reshape(g, 1, N);
k_mid = k0 + ceil((N-1)/2)*dk;
if x_mid ~= 0
    k = linspace(k0, k0+dk*(N-1), N);
    g = g .* exp(1i*x_mid*k);
end
if Nx > N
    N = Nx;
    g = fftresize(g, N);
end
f = siffts(g)*(N*dk/sqrt(2*pi));
x = fftlinspace(2*pi/dk, N) + x_mid;
if (abs(k_mid/k0) > 2*eps)
    f = f .* exp(1i*k_mid*(x-x_mid));
end
end
\end{lstlisting}

下面是一些依赖程序
\begin{lstlisting}[language=matlab, caption=fftresize.m]
% resize vector/matrix length for ftt by zero padding on both ends
function y = fftresize(x, newN)
% === x is row vector ===
if size(x, 1) == 1 
    N = numel(x);
    Ndiff = abs(newN - N);
    if newN > N % 0-padding
        if mod(Ndiff,2) == 0
            Ndiff = 0.5*Ndiff;
            y = [zeros(1, Ndiff), x, zeros(1, Ndiff)];
        else
            Ndiff = 0.5*(Ndiff-1);
            if mod(N, 2) == 0
                y = [zeros(1, Ndiff), x, zeros(1, Ndiff+1)];
            else
                y = [zeros(1, Ndiff+1), x, zeros(1, Ndiff)];
            end
        end
    elseif newN < N % shrink
        y = shrink(x, N, Ndiff);
    else
        y = x;
    end

% === x is column vector ===
elseif size(x, 2) == 1
    N = numel(x);
    Ndiff = abs(newN - N);
    if newN > N % 0-padding
        if mod(Ndiff,2) == 0
            Ndiff = 0.5*Ndiff;
            y = [zeros(Ndiff, 1); x; zeros(Ndiff, 1)];
        else
            Ndiff = 0.5*(Ndiff-1);
            if mod(N, 2) == 0
                y = [zeros(Ndiff, 1); x; zeros(Ndiff+1, 1)];
            else
                y = [zeros(Ndiff+1, 1); x; zeros(Ndiff, 1)];
            end
        end
    elseif newN < N % shrink
        y = shrink(x, N, Ndiff);
    else
        y = x;
    end

% === x is matrix ===
else
    [N, Ncol] = size(x);
    Ndiff = abs(newN - N);
    if newN > N % 0-padding
        if mod(Ndiff,2) == 0
            Ndiff = 0.5*Ndiff;
            y = [zeros(Ndiff, Ncol); x; zeros(Ndiff, Ncol)];
        else
            Ndiff = 0.5*(Ndiff-1);
            if mod(N, 2) == 0
                y = [zeros(Ndiff, Ncol); x; zeros(Ndiff+1, Ncol)];
            else
                y = [zeros(Ndiff+1, Ncol); x; zeros(Ndiff, Ncol)];
            end
        end
    elseif newN < N % shrink
        if mod(Ndiff,2) == 0
            Ndiff = 0.5*Ndiff;
            y = x(Ndiff+1:end-Ndiff, :);
        else
            Ndiff = 0.5*(Ndiff-1);
            if mod(N, 2) == 0
                y = x(Ndiff+2:end-Ndiff, :);
            else
                y = x(Ndiff+1:end-Ndiff-1, :);
            end
        end
    else
        y = x;
    end
end
end


function y = shrink(x, N, Ndiff)
    if mod(Ndiff,2) == 0
        Ndiff = 0.5*Ndiff;
        y = x(Ndiff+1:end-Ndiff);
    else
        Ndiff = 0.5*(Ndiff-1);
        if mod(N, 2) == 0
            y = x(Ndiff+2:end-Ndiff);
        else
            y = x(Ndiff+1:end-Ndiff-1);
        end
    end
end
\end{lstlisting}

\begin{lstlisting}[language=matlab, caption=sffts.m]
% shifted fft
function y = sffts(x, dim)
    if nargin < 2
        y = fftshift(fft(ifftshift(x)));
    else
        y = fftshift(fft(ifftshift(x, dim),[], dim), dim);
    end
end
\end{lstlisting}

\begin{lstlisting}[language=matlab, caption=fftlinspace.m]
% generate N grid points from bandwidth
% input 2 or 3 arguments
function x = fftlinspace(L, N, x0)
if mod(N, 2) == 0
    Lh = 0.5*L; dx = L/N;
    if nargin == 3
        x = linspace(-Lh+x0, Lh-dx+x0, N);
    else
        x = linspace(-Lh, Lh-dx, N);
    end
else
    a = (N-1)*L/(2*N);
    if nargin == 3
        x = linspace(-a+x0, a+x0, N);
    else
        x = linspace(-a, a, N);
    end
end
end
\end{lstlisting}

\subsubsection{傅里叶级数}
根据\autoref{eq_DFT_10}~\upref{DFT}, 傅里叶级数与傅里叶变换值相差一个常数:
\begin{lstlisting}[language=matlab, caption=FS.m]
% Fourier series by FFT
function [C, k] = FS(f, x0, dx)
[g, k] = FFT(f, x0, dx, Nk, dim);
C = sqrt(2*pi)/(N*dx) * g;
end
\end{lstlisting}

\subsection{sinc 插值}
根据采样定理, 可以使用 FFT 对函数的离散值进行 \verb|sinc| 插值, \verb|dx| 是可选的。
\begin{lstlisting}[language=matlab, caption=fftinterp.m]
% approximate sinc interpolation by fft
% N is optional, used to zero-pad f
function [f1, x1] = fftinterp(f, N1, dx, N)
if ~exist('N','var') || isempty(N)
    N = numel(f);
else
    f = fftresize(f, N);
end
f1 = siffts(fftresize(sffts(f), N1))*N1/N;
if nargout == 2
    x1 = fftlinspace(dx*N, N1);
end
end
\end{lstlisting}

为了对比验证, 我们也可以直接实现 \verb|sinc| 插值, 但该代码的效率较低
\begin{lstlisting}[language=matlab, caption=sinc\_interp.m]
% sinc_interp
function y = sinc_interp(x, x0, y0)
    N0 = numel(x0);
    y = zeros(size(x));
    dx0 = (max(x0)-min(x0))/(numel(x0)-1);
    a = pi/dx0;
    for ii = 1:N0
        y = y + y0(ii).*sinc(a*(x-x0(ii)));
    end
end

function y = sinc(x)
    mask = (x~=0);
    y(mask) = sin(x(mask))./x(mask);
    y(~mask) = 1;
end
\end{lstlisting}
