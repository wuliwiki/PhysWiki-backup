% 质心参考系
% license Usr
% type Tutor

\pentry{质心的定义\nref{nod_CM}}{nod_bcf5}

\subsection{质心参考系}\label{sub_CM_2}
定义\enref{质点系}{PSys}的\textbf{质心参考系}(或\textbf{质心系})为原点固定在其\enref{质心}{CM}上且没有转动的参考系(平动参考系)。% \addTODO{链接:平动是相对的, 转动是绝对的。}
根据质心的唯一性(\autoref{eq_CM_4}~\upref{CM})。根据该定义,在质心系中,系统的质心(\autoref{eq_CM_1}~\upref{CM})始终固定在原点,即
\begin{equation}\label{eq_CM_7}
\sum_i m_i \bvec r_{ci} = \bvec 0~,
\end{equation}
其中 $\bvec r_{ci}$ 是质心系中质点 $i$ 的位矢。

注意质心系不一定是\enref{惯性系}{New3},以后我们会看到\upref{PLaw},在任意惯性系中,只有当系统所受合外力为零时,质心才会做匀速直线运动,此时质心系才是惯性系(因为相对惯性系做匀速平移的都是惯性系)。在非惯性系中,需要考虑\enref{惯性力}{Iner}。

\subsection{质心系中总动量}
把\autoref{eq_CM_7} 两边对时间求导,得
\begin{equation}\label{eq_CM_8}
\sum_i m_i \bvec v_{ci} = \bvec 0~.
\end{equation}
注意到等式左边是质心系中质点系的总动量,所以我们得到质心系的一个重要特点,\textbf{质心系中总动量为零}。 但注意总动量为零的惯性系未必是质心系(把\autoref{eq_CM_7} 右边改成常矢量也可以)。

% \addTODO{举例,人船模型等}
