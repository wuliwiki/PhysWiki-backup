% 小时云笔记协议
% keys LaTeX|云笔记|协议
% license Xiao
% type Note

% 未完成: 两个链接
本协议仅适用于本站的 \href{http://wuli.wiki/note/}{小时云笔记服务}, 另见\href{http://wuli.wiki/online/licens.html}{《百科创作协议》}以及\href{http://www.example.com}{《网盘协议》}。 若您继续使用网站的笔记功能, 将视为同意以下条款。

\subsection{创作内容}
\begin{itemize}
\item 笔记内容的著作权完全归原作者所有, 未经作者允许, 本站无权使用或改动。
\item 本站所有数据设有定期多重备份, 对于极罕见的情况下由不可抗原因导致用户创作内容部分或全部丢失, 本站恕不负责, 请自行对重要内容进行备份。
\item 用户编辑或上传的内容如有侵权行为, 引发的后果由用户本人承担。
\item 若发现笔记内容存在违法违规现象(不包含侵犯著作权), 管理员可以不经用户同意直接删除。 同时会根据严重程度采取注销或冻结账号等措施。
\end{itemize}

\subsection{隐私策略}
\begin{itemize}
\item 请勿将任何隐私信息存于笔记中。
\item 由于技术原因, 目前所有的笔记内容都是公开的(如果知道网址)。 笔记目录的网址为 \verb|http://wuli.wiki/user/用户名/online/|, 单篇笔记的网址为 \verb|http://wuli.wiki/user/用户名/online/文件名.html| (已发布的内容)。
\item 以上两个网址中 \verb|online| 换成 \verb|changed| 就可以显示未发布的内容(要发布,使用编辑器的发布按钮)。
\item 如果一篇笔记不在目录中, 那么网站其他\textbf{普通访客}无法直接找到该笔记, 除非被搜索引擎收录或者 \verb|文件名| 被猜出。
\item 网站的管理员和技术人员可以在后台看到每个用户的所有笔记。 原则上技术人员会遵守保密协议。
\end{itemize}
