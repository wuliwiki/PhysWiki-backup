% 库仑规范(量子力学)
% keys 库仑规范|电磁场|薛定谔方程|矢量算符|量子力学
% license Xiao
% type Tutor

\begin{issues}
\issueDraft
\end{issues}

\pentry{电磁场中的单粒子薛定谔方程\nref{nod_QMEM},库仑规范(电动力学)\nref{nod_Cgauge}}{nod_8fc8}

\footnote{本文参考 \cite{Bransden}。}规范不变的哈密顿量为(\autoref{eq_QMEM_1}~\upref{QMEM})
\begin{equation}\label{eq_CouGau_1}
H = \frac{1}{2m} (\bvec p - q\bvec A)^2 + q\varphi + V(\bvec r)~.
\end{equation}

库仑规范\upref{Cgauge}推导起来较为简单, 用角标 $C$ 表示库仑规范, 则矢势满足 $\div \bvec A_C = 0$, 标势 $\varphi_C$ 满足(\autoref{eq_Cgauge_1}~\upref{Cgauge})
\begin{equation}
\varphi_C(\bvec r, t) = \frac{1}{4\pi\epsilon_0} \int \frac{\rho(\bvec r', t)}{\abs{\bvec r - \bvec r'}} \dd[3]{r'}~.
\end{equation}
但原子核的库仑势能一般包含在\autoref{eq_CouGau_1} 的 $V(\bvec r)$ 而不是 $\varphi$ 中(根据\autoref{sub_QMEM_1}~\upref{QMEM}这没问题,其实包含在 $\varphi$ 中也没问题),空间中也无其他电荷,所以有(\autoref{eq_Cgauge_5}~\upref{Cgauge},\autoref{eq_Cgauge_2}~\upref{Cgauge})
\begin{equation}\label{eq_CouGau_2}
\varphi_C \equiv 0~.
\end{equation}
\begin{equation}\label{eq_CouGau_3}
\bvec {\mathcal E}(t) = -\pdv{\bvec A_C}{t}~.
\end{equation}

结合矢量算符法则\autoref{eq_VopEq_1}~\upref{VopEq} 得
\begin{equation}
\div (\bvec A_C \Psi) = (\div \bvec A_C) \Psi + \bvec A_C \vdot (\grad \Psi) = \bvec A_C \vdot (\grad \Psi)~,
\end{equation}
所以\autoref{eq_CouGau_1}(\autoref{eq_QMEM_2}~\upref{QMEM})简化为
\begin{equation}\label{eq_CouGau_4}
H_L = -\frac{1}{2m} \laplacian + \I \frac{q}{m} \bvec A_C \vdot \Nabla + \frac{q^2}{2m} \bvec A_C^2~.
\end{equation}
