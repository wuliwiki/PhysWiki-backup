% Python求解常微分方程组的初值问题
% 常微分方程组|Python|数值解
这里我们再次以“天体运动的简单数值计算\upref{KPNum0}” 中的问题为例,利用\verb |Python|语言实现该微分方程组的求解.
\begin{equation}\label{PyIVP_eq}
\begin{cases}
x' = v_x\\
y' = v_y\\
v'_x = -GMx/(x^2 + y^2)^{3/2}\\
v'_y = -GMy/(x^2 + y^2)^{3/2}
\end{cases}
\end{equation}
中的 $x, y, v_x, v_y$.

在对该问题求解之前,我们先简单介绍\verb|Python|中关于微分方程数值求解库极相关函数的用法.一般,最常用的数值计算库为\verb|scipy|,而微分方程求解对应的模块为\verb|scipy.integrate.solve_ivp|.

 \verb|solve_ivp|的一般格式为
 \begin{lstlisting}[language=python]
 solve_ivp(fun, t_span, y0, method='RK45', t_eval=None,
        dense_output=False, events=None, 
        vectorized=False, args=None, **options)
 \end{lstlisting}
 其中,
\begin{itemize}
\item \verb|fun|是微分方程(组)的右端;
\item  \verb|t_span=(t0,tf)|代表积分区间为\verb|t0|到\verb|tf|;
\item  \verb|y0|为初始条件;
\item \verb|method|为数值积分方法,这里可以使用的积分方法有\verb|RK45|、\verb|RK23|、\verb|DOP853|(8阶显式龙格库塔法)、\verb|Radau|(5阶Radau IIA族的隐式Runge-Kutta方法)、\verb|BDF|(隐式多步变阶)、\verb|LSODA|(具有自动刚度检测和切换的Adams/BDF方法)等.需要主要的是显式Runge-Kutta方法(\verb|RK23、RK45、DOP853|)应用于非刚性问题,隐式方法(\verb|Radau、BDF|)应用于刚性问题.
\item \verb|t_eval|是可选参数,需要数组类型数据.如果给定就在这些时间点进行求解,否则求解器自己选择时间点进行求解.
\item \verb|dense_output|为布尔类型,默认为假,是否计算连续解.
\item \verb|args|为元祖类型的参数,用于向微分方程传递必要的参数
\item 后面还有许多可选参数,很少会使用.
\end{itemize}


基于对\verb|solve_ivp|的使用说明,我们接下来对微分方程组\ref{PyIVP_eq} 进行数值计算.具体代码如下:
\begin{lstlisting}[language=python]

\end{lstlisting}