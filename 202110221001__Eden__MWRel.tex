% 麦克斯韦关系
% 热力学|态函数

\pentry{热力学第一定律\upref{Th1Law},亥姆霍兹自由能\upref{HelmF},吉布斯自由能\upref{GibbsG}}

\subsection{麦氏关系}

\begin{align}
&\left(\frac{\partial T}{\partial V}\right)_S=-\left(\frac{\partial P}{\partial S}\right)_V\label{MWRel_eq1}
\\
&\left(\frac{\partial T}{\partial P}\right)_S=\left(\frac{\partial V}{\partial S}\right)_p
\\
&\left(\frac{\partial S}{\partial V}\right)_T=\left(\frac{\partial P}{\partial T}\right)_V
\\
&\left(\frac{\partial S}{\partial P}\right)_T=-\left(\frac{\partial V}{\partial S}\right)_V
\end{align}

\subsection{推导}
根据 $\dd U=T\dd S-P\dd V$,可得 $\Big(\partial U/\partial S\Big)_V=T$,$\Big(\partial U/\partial V\Big)_S=-P$.再根据 $\partial^2 U/(\partial V\partial S)=\partial^2 U/(\partial S\partial V)$,就可以推出 \autoref{MWRel_eq1}.

对热力学函数焓 $H=U+pV$,自由能 $F=U-TS$,吉布斯函数 $G=U-TS+pV$ 都可以列出微分表达式,于是类似地可以推出剩余 $3$ 个麦克斯韦关系.

\begin{example}{内能方程}
对于某个热力学系统(例如某种气体),我们想知道在恒温条件下,它的内能随体积的改变会如何变化.这个物理量在实验上不能直接测量(测量系统吸收了多少热量是很困难的),但我们可以期待将它用其他物理量表达出来.

\begin{equation}
\dd U=T\dd S-P\dd V
\end{equation}

其中 $T\dd S$ 也就是说 $\dd S$ 部分是无法直接测量的.我们考虑将 $S$ 看成状态参量 $T,V$ 的函数,那么 
\begin{equation}
\dd S=\left(\frac{\partial S}{\partial T}\right)_V \dd T+\left(\frac{\partial S}{\partial V}\right)_T \dd V
\end{equation}
代入内能微分式,就可以得到
\begin{equation}\label{MWRel_eq2}
\dd U=T\left(\frac{\partial S}{\partial T}\right)_V\dd T+\left[T\left(\frac{\partial S}{\partial V}\right)_T-P\right]\dd V
\end{equation}
所以在等温条件下,内能关于体积的变化率为

\begin{equation}
\left(\frac{\partial U}{\partial V}\right)_T=T\left(\frac{\partial S}{\partial V}\right)_T-P
\end{equation}
利用麦克斯韦关系,可以得到
\begin{equation}
\left(\frac{\partial U}{\partial V}\right)_T=T\left(\frac{\partial P}{\partial T}\right)_V-P=PT\beta-P
\end{equation}
式中 $\beta$ 为定容压强系数.这就是内能方程,通过在实验中测量压强、温度,计算定容压强系数,就可以求出内能随时间的变化率.

从上面的 \autoref{MWRel_eq2} 还可以得出等体热容的另一表达式:
\begin{equation}
\frac{C_V}{T}=\left(\frac{\partial S}{\partial T}\right)_V
\end{equation}

对于状态函数\textbf{焓} $H=U+PV$,也存在类似的方程:
\begin{equation}\label{MWRel_eq3}
\left(\frac{\partial H}{\partial P}\right)_T=T\left(\frac{\partial S}{\partial P}\right)_T+V
=V-T\left(\frac{\partial V}{\partial T}\right)_P=V-V\alpha T
\end{equation}

\end{example}
