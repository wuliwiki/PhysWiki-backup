% 磁矢势
% 旋度|磁场|矢势

\begin{issues}
\issueDraft
\end{issues}

\pentry{旋度的逆运算\upref{HlmPr2}}

由于磁场 $\bvec B(\bvec r)$ 任何情况都是一个无旋场\upref{MagGau}, 所以根据\autoref{HlmPr2_the1}~\upref{HlmPr2} 必定存在一个矢量场 $\bvec A(\bvec r)$ 使得
\begin{equation}
\curl \bvec A = \bvec B
\end{equation}
且 $\bvec A$ 可以通过下式计算
\begin{equation}
\bvec A(\bvec r) = \frac{1}{4\pi} \int \bvec B(\bvec r') \cross \frac{\bvec R}{R^3} \dd{V'} + \bvec H(\bvec r)
\end{equation}
其中 $\bvec r, \bvec r'$ 分别是坐标原点指向三维直角坐标 $(x, y, z)$ 和 $(x', y', z')$ 的位置矢量, $\bvec R = \bvec r' - \bvec r$, $R = \abs{\bvec R}$, 体积分 $\int\dd{V'} = \int\dd{x'}\dd{y'}\dd{z'}$ 的区域是空间中 $\bvec B$ 不为零的区域, $\cross$ 表示矢量叉乘\upref{Cross}, $\bvec H(\bvec r)$ 是一个任意无旋场.

\subsection{规范}
由于 $\bvec A(\bvec r)$ 不止一种, 我们有时候需要某种\textbf{规范(gauge)}来将其唯一确定下来. 例如在\textbf{库伦规范(Coulomb Gauge)}中, 我们要求
\begin{equation}
\div \bvec A = \bvec 0
\end{equation}
根据\autoref{HlmPr2_eq4}~\upref{HlmPr2}, 我们只需要令 $\bvec H(\bvec r)$ 是一个调和场即可.
