% 四川大学 2009 年研究生入学物理考试试题
% keys 四川大学|2009年|考研|物理
% license Copy
% type Tutor

\textbf{声明}:“该内容来源于网络公开资料,不保证真实性,如有侵权请联系管理员”


科目代码:947


适用专业:光学、无线电物理、物理电子学


可能用到的物理常数:真空中的介电常数:$8.85*10^{-12}F/m$,真空中的导磁率:$4\pi*10^{-7}H/m$,一个电子的电量:$1.62*10^{-19}C$
\subsection{填空题}
\begin{enumerate}
\item  处于静电平衡的理想导体,导体内部电场强度为 $(\qquad)$ 随曲率半径增大,导体外表面的电场强度$(\qquad)$ 
\item 利用万用表测量市电的交流电压,读数为220V,是指交流电压的$(\qquad)$ 值,对应的峰值电压为$(\qquad)$ V。
\item 相隔距离为 d 的等量同号点电荷+q 和+q,二者中点处的电势为$(\qquad)$ v,电场强度的大小为$(\qquad)$ V/m。
\item 两根无限长的均匀带电直线相互平行,相距为2a,线电荷密度分别为+$\rho$和-$\rho$则每单位长度上的带电直线受的库仑力为$(\qquad)$N,两根直线相
互$(\qquad)$。
\item 两块平行金属板间充满电容率为$\varepsilon_1=2\varepsilon_0$的均匀介质,当维持两块金属板上电压V不变,每块平行金属板的电荷为Q。1)如果将介质换为$\varepsilon_2=2\varepsilon_1$的介质则每块平行金属板的电荷为$(\qquad)$
。2)如果将介质$\varepsilon_1$ 移去,则每块平行金属板的电荷为$(\qquad)$。
\item 一波长为 550nm的黄绿光入射到间距为0.2mm的双缝上,则离缝2m远处观察屏上干涉条纹的间距是$(\qquad)$mm;若缝间距增大为2mm,则干涉条纹间距变为$(\qquad)$mm。
\item 一玻璃劈尖,折射率为1.52,波长 $\lambda=589.3nm$的鈉光垂直入射时,测得相邻条纹间距$L=5.0mm$,则劈尖夹角约为$(\qquad)$弧度。
\item 用迈克耳孙干涉仪测量光的波长过程中,测得可动反射镜移动了$\Delta L=0.3220mm$,等倾条纹在中心处缩进$1092.8$个条纹,则所采用的光波长$(\qquad)$nm。
\item 当人眼瞳孔直径为2mm 时,此时人眼的最小分辨角约为$(\qquad)$分(设照明的光波长为550nm)。
\item 自然光通过两个偏振化方向间成60°的偏振片,透射光强为$I_1$。今在这两个偏振片之间再插入另一偏振片,它的偏振化方向与前两个偏振片均为30°角,则透射光强为$(\qquad)$
\end{enumerate}
\subsection{选择题}
\begin{enumerate}
\item 有一导体置于有一导体置于静电场中,如图所示,其中L>2a,位置1代表尖端处,位置 2代表柱体部分,位置3代表底面,请问该导体电位最高的位置是$(\qquad)$\\
(A)位置1\\
(B)位置2\\
(C)位置3\\
(D)所有位置电位相同
\begin{figure}[ht]
\centering
\includegraphics[width=10cm]{./figures/e7386c3f47b4b580.png}
\caption{} \label{fig_CD09_2}
\end{figure}
\item 下列说法正确的是$(\qquad)$\\
(A)理想导体表面电场强度为零\\
(B)理想导体的电容与形状无关\\
(C)两个理想导体所带的电荷量相同,则其电位也一定相同\\
(D)上述说法都不对
\item 下列关于超导体与理想导体的说法,哪一个不正确$(\qquad)$
(A)超导体的电阻为零\\
(B)理想导体的电阻为零\\
(C)超导体内可以存在磁感应强度$\vec B$\\
(D)理想导体内可以有不随时间变化的磁感应强度$\vec B$
\item 如图所示,一封闭的导体壳A内有两个导体B和C,所带电荷为$Q_A=Q_0,Q_B=Q_c=0$。请问下列说法正确的是$(\qquad)$\\
(A)$U_B=U_C>U_A$\\
(B)$U_C=U_A=U_B$\\
(C)$U_A>U_B=U_C$\\
(d)$U_B>U_A>U_C$
\begin{figure}[ht]
\centering
\includegraphics[width=8cm]{./figures/c8669877560a985b.png}
\caption{} \label{fig_CD09_3}
\end{figure}
\item 四个电阻联接如图所示,已知$R_1=R_2=R_3=R_4=9\Omega$,则a,b间的电阻为$(\qquad)$\\
(A)12$\Omega$\\
(B)9$\Omega$/4\\
(C)6$\Omega$\\
(D)18$\Omega$
\begin{figure}[ht]
\centering
\includegraphics[width=8cm]{./figures/f3ca2d1eaf1a3715.png}
\caption{} \label{fig_CD09_1}
\end{figure}
\item 杨氏实验中,光源波长为0.64$\mu m$ ,缝间距为0.4mm,光屏离缝的距离为50cm,屏上第一亮纹与中央亮纹之间的距离为:$(\qquad)$\\
(A)0.8mm\\
(B)8mm\\
(C)0.4mm\\
(D)4mm
\item 白光垂直入射每厘米有 4000 条缝的光栅,利用该光栅能产生多少级完整的光谱?$(\qquad)$\\
(A)0°\\
(B)18.44°\\
(C)15°\\
(D)无法确定
\item 一光栅宽度为6.0cm,每厘米有6000条刻线,问在第三季谱线中,对$\lambda=500nm$处,可分辨的最小波长间隔是$(\qquad)$\\
(A)0.0046nm\\
(B)0.028nm\\
(C)0.1667nm\\
(D)以上答案均不对
\item 水的折射率为1.33,玻璃的折射率为1.5,当光由水中射向玻璃而反射时,起偏角为$(\qquad)$\\
(A)41.56°\\
(B)48.44°\\
(C)45°\\
(D)以上答案都不对
\end{enumerate}
\subsection{简答题}
\begin{enumerate}
\item 请写出真空中麦克斯韦方程组的微分形式。
\item 两个矢量场分别满足如下两个积分方程。1)试写出两个矢量场满足的微
分方程。2)从电磁学角度举例说明两个矢量场。
\begin{equation}
\oint_s \vec F_1d\vec S=0 \quad \text{S为任意闭合面}~
\end{equation}
\begin{equation}
\oint \vec F_2d\vec I=0 \quad \text{C任意闭合曲线}~
\end{equation}
\item 何谓光的干涉?如果用白光作光源进行杨氏双缝于涉实验,可观察到什么
现象?
\item 一束光可能是:(1)自然光:(2)线偏振光(3)部分振光。如何用实
验确定这束光究竟是哪一种光?如何利用单色自然光产生圆偏振光和圆
偏振光?
\end{enumerate}
\subsection{计算题}
\begin{enumerate}
\item 内外半径分别为$R_2$和$R_3$的金属球壳B带有电荷量Q=0。在它里面放一个带
有电荷量q、半径为$R_1$;的同心金属球A,如图所示。\\
1)试求离球心为r处的、电场强度$\vec E$和电势U。\\
2)试求金属球A与金属球壳B的电势差。\\
3)试求该系统的电场能量。\\
\begin{figure}[ht]
\centering
\includegraphics[width=9cm]{./figures/c2e3f18fd9b48ef3.png}
\caption{} \label{fig_CD09_4}
\end{figure}
\item 一均匀介质球在均匀极化后,极化强度为$\vec p$。试求:\\
1)介质球表面上极化电荷量的面密度;\\
2)这些极化面电荷在球心产生的电场强度。
\item 在双缝夫琅和费衍射实验中,所用波长为$\lambda=632.8nm$ ,透镜焦距
$f=500mm$,测得两亮纹之间的距离为$1.5mm$,并且第四级亮纹缺级,试求双缝的间距和缝宽。
\item 已知一光栅的光栅常数$d=2.5um$,缝数为$N=20000$条。求:\\
(1)光的一、二、三级光谱的分本领;\\
(2)波长$\lambda=0.69\mu m$ 红光的二级、三级光谱的角度。
\item 一折射率为1.5、厚度为4.0$\mu$ m 的薄玻璃片,用白光垂直照明,在可见光范
围内(0.4~0.7$\mu$m),问哪些波长的光在反射中加强?哪些波长的光在透射中加强?
\end{enumerate}
