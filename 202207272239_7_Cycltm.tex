% 分圆域
% keys 分圆域|cyclotomy|域扩张|单位根|本原根|原根|primitive element|切圆域|割圆域|分圆多项式|切圆多项式|割圆多项式

\addTODO{预备知识待确定}

\pentry{本原多项式(线性代数)\upref{PPlyR}}



$n$次\textbf{单位根}即形如$x^n-1$的多项式在$\mathbb{C}$中的根,可以理解为$1$的$n$次根.$\pm 1$都是$1$的$2^n$次根,$\pm \I$都是$1$的$4$次单位根,而$\omega=\frac{1}{2}\qty(-1+\I\sqrt{3})$是$1$的$3$次根.

本节将使用Galois理论来处理单位根及其最小多项式对有理数域的扩张,作为Galois理论的应用.


\subsection{分圆多项式}

任意$n$次单位根都形如$\exp{(2\pi\I\cdot  k/n)}$,其中$k$取从$1$到$k$的全体整数,即包括了所有$n$次单位根.如$4$次单位根构成的集合是$\{1, -1, \I, -\I\}$,而这四个元素又可以两两分组:$\pm 1$都可以是$2$次单位根\footnote{更细节些,$1$也是$1$次单位根.},$\pm \I$是$4$次单位根.显然两组性质是不同的:$\mathbb{Q}(\pm 1)=\mathbb{Q}\subsetneq \mathbb{Q}(\pm\I)$.我们将这分类表述为以下定义:

\begin{definition}{本原单位根}\label{Cycltm_def2}

当$n$与$k$互素时,称单位根$\exp{(2\pi\I\cdot  k/n)}$是\textbf{本原(primitive)}的.

\end{definition}

由互素的概念易知,$n$次\textbf{本原单位根}的全体幂,包含了全体$n$次\textbf{单位根},即$n$次本原单位根至少要自乘$n$次才能变成$1$;而非本原的单位根则不然,如$3$次单位根自乘$3$次即可得到$1$,但它同时也是$6$次单位根.因此由分裂域的知识可知,非本原的单位根,其最小多项式不可能是$x^n-1$.为此,我们需要明确本原根所属的最小多项式.

\begin{definition}{分圆多项式}\label{Cycltm_def1}

设$\{\varepsilon_i\}_{i\in S}$是全体$n$次\textbf{本原单位根}的集合,则称
\begin{equation}
\Phi_n(x)\in \mathbb{C}[x] = \prod_{i\in S}(x-\varepsilon_i)
\end{equation}
为$n$次\textbf{分圆多项式(cyclotomic polynomials)},有时也译作“\textbf{切圆多项式}”或“\textbf{割圆多项式}”.

\end{definition}

取“分圆”之名,是指其将单位圆分割.英文cyclotomy取自“cycl-”和“-tomy”的组合,“cycl-”指“圆形的、环形的或旋转的”,而-tomy源自古希腊语“tomia”或其变形“tomnein”,指“切割”.




\begin{example}{分圆多项式的例子}\label{Cycltm_ex1}

\begin{equation}
\Phi_1(x) = x-1
\end{equation}

\begin{equation}\label{Cycltm_eq2}
\Phi_2(x) = x+1
\end{equation}

\begin{equation}
\Phi_3(x) = (x^3-1)/(x-1) = x^2+x+1
\end{equation}

\begin{equation}
\Phi_4(x) = (x^4-1)/(x^2-1) = x^2+1
\end{equation}

\begin{equation}
\Phi_5(x) = (x^5-1)/(x-1) = x^4+x^3+x^2+x+1
\end{equation}

% 注意后三个例子里,都除以了一个多项式.为什么这么做?

\end{example}

\begin{exercise}{}\label{Cycltm_exe1}
请尝试算几个\autoref{Cycltm_ex1} 中未列出的分圆多项式的例子.
\end{exercise}

更多的例子,可以参见维基百科\href{https://en.wikipedia.org/wiki/Cyclotomic_polynomial#Examples}{分圆域},这也是\autoref{Cycltm_exe1} 的部分答案.





\subsubsection{分圆多项式的基本性质}


下面列出分圆多项式的几个基本性质.

\begin{theorem}{}
任取正整数$n\geq 2, q\geq 2$,则$\Phi_n(x)\in\mathbb{R}[x]$,且$\Phi_n(q)>q-1$.
\end{theorem}

\textbf{证明}:

首先证明$\Phi_n(x)\in\mathbb{R}[x]$.

任取小于$n$的正整数$k$,则$k$与$n$互素当且仅当$n-k$与$n$互素.因此,$\exp{(2\pi\I\cdot  k/n)}$是本原的,当且仅当$\exp{(2\pi\I\cdot  (n-k)/n)}$是本原的.所以本原元素$\varepsilon_k$的共轭$\varepsilon_{n-k}$还是本原元素,从而
\begin{equation}\label{Cycltm_eq1}
(x-\varepsilon_k)(x-\varepsilon_{n-k})=x^2-(\varepsilon_k+\varepsilon_{n-k})x+\varepsilon_n\varepsilon_{n-k}
\end{equation}
是\textbf{实系数}多项式.于是分圆多项式是由若干形如\autoref{Cycltm_eq1} 或者形如$x+1$\footnote{即本原单位元的共轭就是自身的情况,而该情况只有\autoref{Cycltm_eq2} 一种,所以$n>3$的情况都是本原元素成对出现的.}的多项式相乘而得,从而$\Phi_n(x)\in\mathbb{R}[x]$.

由于$q\geq 2$,且各$\Phi_n$的本原根$\varepsilon_k$满足$\abs{\varepsilon_k}=1$,$\varepsilon_k\neq 1$.因此$\abs{q-\varepsilon_k}>\abs{q-1}=q-1$,从而得$\Phi_k(q)>q-1$.


\textbf{证毕}.




\begin{theorem}{}\label{Cycltm_the1}
\begin{equation}
x^n-1 = \prod_{d\mid n}\Phi_d(x)
\end{equation}

等价表述:不同次的\textbf{本原单位根}之间无交集,且全体满足$d\mid n$的$d$次\textbf{本原单位根}构成的集合,恰为全体$n$次\textbf{单位根}之集合.
\end{theorem}

\textbf{证明}:

\textbf{一个}$d_i$次\textbf{本原单位根}的幂遍历所有$d_i$次单位根,因此“一个$d_1$的本原单位根也是$d_2$的本原单位根”,当且仅当“$d_1=d_2$”,当且仅当“$d_1$和$d_2$的本原单位根完全一样”.于是得证“不同的次的\textbf{本原单位根}之间无交集”.

任取一个$n$次单位根$\varepsilon_k=\exp{(2\pi\I\cdot  k/n)}$,取$d=\opn{gcd}(n, k)$\footnote{即$n$与$k$的最大公因子(greatest common divisor).},则$\varepsilon_k$是一个$d$次本原单位根.从而任意$n$次单位根都是某个$d$次本原单位根,满足$d\mid n$,结合上一段的结论得证定理.

\textbf{证毕}.



\begin{theorem}{}\label{Cycltm_the2}
$\Phi_n(x)$是首一整系数多项式.
\end{theorem}

\textbf{证明}:

由\autoref{Cycltm_def1} ,首一是显然的.接下来用归纳法证明“整系数”.

由\autoref{Cycltm_ex1} ,$\Phi_1$和$\Phi_2$都是整系数的.假设对于正整数$d$,所有满足$k<d$的$\Phi_k$都是整系数的,那么由\autoref{Cycltm_the1} ,
\begin{equation}
x^{2d}-1 = f(x)\Phi_d(x)
\end{equation}
其中$f(x)$是首一整系数多项式.

由于$x^{2d}-1$和$f(x)$是\textbf{本原多项式}(\autoref{PPlyR_def1}~\upref{PPlyR}),故$\Phi_d(x)$也是本原多项式,从而是整系数的.

\textbf{证毕}.




\begin{theorem}{}
各$\Phi_n$皆是$\mathbb{Q}$上的不可约多项式.
\end{theorem}

\textbf{证明}:

反设$\Phi_n=f(x)g(x)$,其中$f$和$g$都是次数大于$0$的有理系数多项式.不妨\footnote{总可以选择$f$使之不可约,并且可以乘以一个正整数,将它化为本原多项式.}设$f$是\textbf{不可约本原多项式},首项系数是$s$,则$g$的首项系数是$1/s$.

首先讨论$f$和$g$的关系.

由\autoref{Cycltm_the2} ,$\Phi_n$也是本原多项式.于是由\autoref{PPlyR_lem2}~\upref{PPlyR},知$g(x)$应为本原多项式,故$1/s$也是整数,故$s=1$.因此$g$和$f$都是\textbf{首一}整系数多项式.

任取$\Phi_n$的一根$\varepsilon_i$,则由\autoref{Cycltm_def2} 的互素条件可知,$\varepsilon_i$的全体幂包含了全体$n$次单位根.于是,任取$f$的一根$\alpha$和$g$的一根$\beta$,总存在正整数$k$使得$\beta=\alpha^k$,且$k$与$n$互素(因为$\beta$是本原单位根).遍历$f$和$g$的所有根,挑出其中使得$k$\textbf{最小}的情况,设该情况为“取$f$的根$\alpha_0$和$g$的根$\beta_0$,则$\beta_0=\alpha_0^k$”.由于$\alpha_0\neq \beta_0$,知$k>1$.

设素数$p$满足$p\mid k$,则$p\not\mid n$\footnote{这一条会在证明的最后一段用到.},因此$\alpha_0^p$是$n$次本原单位根.显然,$\Phi_n(\alpha_0^p)=0$,即$f(\alpha_0^p)$和$g(\alpha_0^p)$中至少一个为零.

设$\alpha_0^p$是$f$的根,那么$\beta_0=\alpha_0^k=(\alpha_0^p)^{k/p}$.由于$k/p<k$,故这违反了“$k$最小”的假设,因此$f(\alpha_0^p)\neq 0$.因此$\alpha_0^p$是$g$的根.又因为“$k$最小”的假设,知$k=p$.

由于$f$不可约,可知$f=\opn{Irr}(\alpha_0, \mathbb{Q})$;又因为$g(\alpha_0^p)=0$,可知$f(x)\mid g(x^p)$.设$g(x^p)=f(x)h(x)$.由于$g(x)$是首一整系数的,故$g(x)$是本原多项式,故$g(x^p)$是本原多项式.由\autoref{PPlyR_lem2}~\upref{PPlyR},$h(x)$也是本原多项式.

接下来利用上述讨论所得,完成证明.

考虑商映射$\pi:\mathbb{Z}\to\mathbb{Z}/p\mathbb{Z}$,其中$\pi(a)=\bar{a}$.于是$\pi$可以扩展为$\mathbb{Z}[x]\to\mathbb{Z}/p\mathbb{Z}[x]$的映射,也用$\bar{f}$来表示$\pi(f)$.设$g(x)=\sum_i c_ix^i$,记$\bar{g}(x)=\pi(g)(x)=\sum_i \bar{c_i}x^i$.

由于$\mathbb{Z}/p\mathbb{Z}$上任意元素$\bar{a}$都满足$\bar{a}^p=\bar{a}$,故$\qty(\bar{g}(x))^p=\sum_i\bar{c_i}^px^{pi}=\sum_i \bar{c_i}(x^p)^i=\bar{g}(x^p)$.

由于$f(x)\mid g(x^p)$,故$\bar{f}(x)\mid \bar{g}(x^p)$,故$\bar{f}(x)\mid \qty(\bar{g}(x))^p$.所以,$\bar{\Phi}=\bar{f}\bar{g}$在\textbf{其}在$\mathbb{Z}/p\mathbb{Z}$的分裂域上\textbf{有重根}.

已知$\Phi$是可分多项式,无重根,因此由\autoref{SprbEx_cor2}~\upref{SprbEx},知$\opn{D}\Phi\neq 0$.由于$p\not\mid n$,而$\Phi(x)$的最高次单项式为$x^n$,故$\opn{D}x^n=nx^{n-1}\implies \opn{D} \bar{1}x^n=\bar{n}x^{n-1}\neq 0$,故$\opn{D}\bar{\Phi}\neq 0$.因此$\bar{\Phi}$也\textbf{无重根}.这和上一段的结论矛盾,因此整个反设不成立.

\textbf{证毕}.



\begin{theorem}{}
对于素数$p$,$\Phi_p(x)=x^{p-1}+x^{p-2}+\cdots+x+1$.
\end{theorem}

\textbf{证明}:

由于$p$是素数,因此除了$1$以外的所有$p$次单位根都是本原单位根.因此
\begin{equation}
\Phi_p(x) = (x^p-1)/(x-1) = x^{p-1}+x^{p-2}+\cdots+x+1
\end{equation}

\textbf{证毕}.






\begin{theorem}{}
$\Phi_n(0)=\pm 1$,且对于$n>1$,总有$\Phi(0)=1$.
\end{theorem}

\textbf{证明}:

$\Phi_n(0)$就是全体$n$次本原单位根的积.记$\varepsilon=\exp{2\pi\I/n}$,则$\varepsilon^k$是$n$次本原单位根,当且仅当$k$与$n$互素.

由于单位根的模都是$1$,故其积的模也是$1$.全体$n$次本原单位根之积是$\Phi_n(x)$的常数项,由\autoref{Cycltm_the2} ,是整数.因此必为$\pm 1$.

由于$k$与$n$互素当且仅当$n-k$与$n$互素,故本原单位根的还是本原单位根.当$n>1$时,要么$n$为奇数,则$(-1)^n\neq 0$;要么$n$为偶数,则$2$和$n$不互素,故$-1$不是$n$次本原单位根.所以$n>1$时$-1$不是$n$次本原单位根.于是本原单位根以共轭形式成对出现,每对的乘积都是$1$.

\textbf{证毕}.















为了进一步讨论分圆多项式的性质,我们需要回顾\textbf{Möbius函数(数论)}\upref{MbusF}中的一些概念:




\begin{definition}{Möbius函数\footnote{见\autoref{MbusF_def1}~\upref{MbusF}.}}
Möbius函数$\mu:\mathbb{N}\to\{-1, 0, 1\}$定义为:
\begin{equation}
\mu(\prod_{k=1}^r p_k^{f_k})=
\leftgroup{
    1, \quad\text{如果}\prod_{k=1}^r p_k^{f_k}=1\\
    (-1)^r, \quad\text{如果}f_k=1\text{恒成立}\\
    0, \quad\text{如果有一个}f_k>1
}
\end{equation}

其中各$f_k$为正整数,$p_k$为互不相同的素数.

\end{definition}




\begin{lemma}{Möbius反演公式\footnote{见\autoref{MbusF_the3}~\upref{MbusF}}}\label{Cycltm_lem1}
取映射$f:\mathbb{Z}^+\to\mathbb{Z}^+$,令$S(n)=\sum_{d\mid n}f(d)$,$P(n)=\prod_{d\mid n}f(d)$.则有
\begin{equation}
f(n) = \sum_{s\mid n}\mu(s)S\qty(\frac{n}{s})
\end{equation}
和
\begin{equation}
f(n) = \prod_{s\mid n}\qty(P(\frac{n}{s}))^{\mu(s)}
\end{equation}
\end{lemma}






有了Möbius函数的概念,我们可以得到一个重要的性质:



\begin{theorem}{Möbius函数表示分圆多项式}
\begin{equation}
\Phi_n(x) = \prod_{d\mid n}(x^d-1)^{\mu(n/d)}
\end{equation}
\end{theorem}


\textbf{证明}:

由\autoref{Cycltm_the1} ,
\begin{equation}
x^n-1 = \prod_{d\mid n}\Phi_d(x)
\end{equation}

又由\autoref{Cycltm_lem1} ,令$f(n)=\Phi_n(x)$,其中$x$可以是\textbf{任意}正整数,于是有
\begin{equation}\label{Cycltm_eq3}
\begin{aligned}
\Phi_n(x)&=\prod_{s\mid n}\qty(\prod_{k\mid \frac{n}{s}}\Phi_k(x))^{\mu(s)}\\
&=\prod_{s\mid n}\qty()^{\mu(s)}\\
&=
\end{aligned}
\end{equation}

其中\autoref{Cycltm_eq3} 第二个等号即为

\textbf{证毕}.











\begin{theorem}{}\label{Cycltm_the3}
设$n$是一正整数,其质因数分解为$n=\prod_{k=1}^r p_k^{f_k}$,其中$f_k$是正整数,$p_k$是互不相同的素数.

任取$m=\prod_{k=1}^r p_k^{g_k}$,其中$1\leq g_k\leq f_k$,则有
\begin{equation}
\Phi_n(x) = \Phi_m(x^{n/m})
\end{equation}
\end{theorem}

\textbf{证明}:

由此定义,任取正整数$d\mid n$,如果$d\not\mid m$,那么必有$\mu(d)=0$.

\textbf{证毕}.








由\autoref{Cycltm_the3} 直接可得以下推论:

\begin{corollary}{}\label{Cycltm_cor1}
对于素数$p$,$\Phi_{p^k}(x) = \Phi_p(x^{p^{k-1}})$
\end{corollary}




\begin{example}{\autoref{Cycltm_cor1}  的实例}
\begin{equation}
\leftgroup{
    \Phi_2(x) &= x+1\\
    \Phi_4(x) &= x^2+1\\
    \Phi_8(x) &= x^4+1\\
    \Phi_{16}(x) &= x^8+1
}
\end{equation}
\begin{equation}
\leftgroup{
    \Phi_3(x) &= x^2+x+1\\
    \Phi_9(x) &= x^6+x^3+1\\
    \Phi_{27}(x) &= x^{18}+x^9+1
}
\end{equation}
\end{example}










































