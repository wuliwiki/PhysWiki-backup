% 归并排序
% 算法|排序|归并排序|C++

上文介绍了快速排序\upref{qsort},本文将介绍\textbf{归并排序}.

归并排序也是基于分治实现的.

归并排序的算法步骤:
\begin{enumerate}
\item 确定分界点为 $\dfrac{l + r}{2}$,把一个序列分成两个大小为 $\dfrac{n}{2}$ 的子序列;
\item 先递归排序两个子序列;
\item 合并两个已排好序的子序列.
\end{enumerate}

归并排序的核心就是\textbf{归并}这一步,首先把一个序列分成两个子序列,然后维护两个指针,第一个指针指向第一个子序列的开头,第二个指针指向第二个子序列的开头.其实归并操作就是把两个子序列的值存到一个序列中输出出来,具体的做法是:每次判断一下两个指针所指向的值哪一个更小,把较小的的值插入到答案数组中,如果发现其中一个序列的指针已经直到末尾了,那么就退出循环,直接把另一个子序列的后面的值接到答案数组中(如图 $1$ 所示).前提得保证两个子序列中的值都已排好序.

\begin{figure}[ht]
\centering
\includegraphics[width=10.25cm]{./figures/Msort_2.png}
\caption{终止} \label{Msort_fig2}
\end{figure}

\textbf{时间复杂度:}

归并排序的期望时间复杂度为 $\mathcal{O}(n \log_2 n)$,最坏时间复杂度也为 $\mathcal{O}(n \log_2 n)$,可以发现,归并的这一步操作是维护两个指针,两个指针会遍历完整个序列,所以时间复杂度为 $\mathcal{O}(n)$,然后递归每一层,每层有 $\frac{n}{2}$、$\frac{4}{n}$...,一共有 $\log_2 n$ 层,每层时间复杂度为 $\mathcal{O}(n)$,所以总共时间复杂度为 $\mathcal{O}(n \log_2 n)$.

\begin{figure}[ht]
\centering
\includegraphics[width=14.25cm]{./figures/Msort_1.png}
\caption{算法导论中的递归树} \label{Msort_fig1}
\end{figure}

