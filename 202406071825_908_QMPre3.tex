% 投影和表示 (转载)
% license Usr
% type Art


\subsection{太阳比喻}

在晴朗的日子里出去走一走,我们看不见自己,但却能看见自己的影子。这是很有意思的事情。

如果太阳不是很晒的话,我们可以站在一个空旷平坦的地上观察我们的影子,它和阳光射来的方向相对,在地上留下一个阴影,如果时间早的话,太阳升的不是很“高”,光线会斜斜地在地上投下一个较长的阴影,随着时间的流逝,太阳会沿着自己的轨道在天空中划出一个圆弧,随着太阳的升“高”,阴影会越来越短,当太阳升到最“高”的时候,阴影也最短。

但说高并不精确,我们可以把眼睛眯起,朝太阳的方向看,所谓“高”就是我们要仰起脖子才能“追踪”到太阳,我们仰起脖子的角度越大、太阳越高,我们可以把这个仰角定义为“太阳-观察者”连接线与地面的夹角$\theta$。当这个角度为$90^o$的时候,太阳在天顶,光线垂直地射下来,此时我们在地上的影子会“消失”\footnote{阴影之内没有光线是暗的,而阴影之外会被阳光照亮,光在这里更多地体现出“粒子性”,它以直线传播,绝对不会绕过障碍物。光从$\theta$方向照射到物体上,在地面上留下一个影子,假设物体的高度是$H$,影子的长度将是$H \cdot \frac{\cos \theta}{\sin \theta } = H \cdot \cot \theta$。}。

\begin{figure}[ht]
\centering
\includegraphics[width=6cm]{./figures/2f63172810481ba4.png}
\caption{太阳光⼊射,与竖直⽅向成 α 角。} \label{fig_QMPre3_4}
\end{figure}

有时我们也以竖直的方向为基准,定义太阳光与竖直方向的夹角为$\alpha$($\alpha = \frac{\pi}{2} - \theta $),当$\alpha = 0$时,阳光笔直地照射在地面上,这时照射到单位面积上太阳光的能量最大,当角度$\alpha$逐渐增大时,照射到单位面积上太阳光的能量会变小,变小的比例正比于$\cos \alpha$。
