% LSFC
% 最小二乘法|Frog

LSFC 的主要思想就是 GPA (基本就是 Gerchberg–Saxton algorithm) 在一些情况下, 这个收敛得很慢, 远远没有直接优化快. 所以我想尝试直接用最小二乘法单向优化来解 SFA 公式.

VTGPA 算法计算论文中的例子需要 1.5 小时, 而我自己发明的 LSFC(Least Square Frog Crab) 只需要短短几分钟即可达到同样的效果.

\subsection{参数选取}
LSFC 的主要的思路是用一组参数确定 XUV, 一组参数确定 IR. XUV 波包必须是完整的波包, 即时域的采样范围要包含任何非零点. IR 的波包不必是完整的, 只需要满足 XUV 的长度和 time delay 的要求即可.

目前 XUV 的参数选取其实数波函数的 DFT 非负频谱(第一个频率为零的格点必定是实数, 其他格点为复数). 如果带宽比正半频谱宽度要窄, 那就排除带宽以外的点. 要从参数还原任何密度的 XUV, 只需在参数两边适当添零再做反 DFT 即可. 一般来说, 只需要在左边添零就能满足条件.

由于 IR 在时域上并不需要是完整的波包(如果需要, 就用与 XUV 相同的参数), 用 XUV 同样的参数效率并不高或者可能有边缘效应. 我尝试用 VTGPA 论文中的方法以振幅, 初相位, 频率和各阶 chirp 作为参数也并不理想, 因为优化时很有可能出现很离谱的 local minimum 解.

现在的方法是, 假设 IR 只需要知道一两个周期, 倒不如直接用 $E_{IR}(t)$ (或者直接 $A(t)$)的散点作为参数, 用 cubic spline 插值还原 $E_{IR}(t)$. 使用越直接的参数, 优化时 converge 越较快.

IR 其实还有一个重要的参数就是 $t = +\infty$ 处矢势 $A$ 的值. $A(t)$ 是 $E_{IR}(t)$ 的反原函数, 但却包含一个任意常数(对 trace 的影响不可忽略). 如果 $E_{IR}$ 是完整的波包, 我们或许可以令 $A(\pm\infty)$ 的其中一个为零, 然而若我们只知道一小段 $E_{IR}$, 任意常数就无从得知, 所以我们必须将 $A(+\infty)$ 放到参数中. 虽然同理对 $A(t)$ 积分后得到的 $\phi(t)$ 也有一个任意常数, 但在对 trace 取绝对值时, global phase 并没有任何影响.

\subsection{优化算法}
最简单的算法就是把所有参数放在一起优化, 但这么做的确很慢. 对 Frog-Crab trace 的公式稍加观察就会发现, 如果 IR 固定, trace 和 XUV 波函数离散点的关系可以用 generalized Phase-Retrieval 来描述
\begin{equation}
\abs{\sum_j A_{ij} x_j} = b_i
\end{equation}
其中矩阵 $\mat A$ 只与 IR 有关. 如果事先计算好 $\mat A$, 使用这个公式比从头计算 trace 要快得多. 所以我们可以交替优化 XUV 和 IR 的参数, 也可以偶尔优化所有参数.

目前的算法是, 规定三种优化的最小循环次数. 先衡量三种优化的效率(单位时间的 trace 误差减小), 然后选取效率最高的来优化一轮, 重新计算它的优化效率, 再次比较三者, 再选取效率最高的来优化第一轮, 重新计算它的优化效率, 再比较三者……

目前三个优化都是使用 Matlab 的 lsqnonlin() 函数中的 Levenberg-Marquadt 算法, 用差分计算梯度.

之前有考虑过用专门的 Phase-Retrieval 算法(见 github 的 phase-pack-Matlab, 如 Wirt Flow)来优化, 但 Phase-Retrieval 一般假设 $x_j$ 是复数, 而离散的 XUV 波函数是实数, 所以测试得到了十分荒谬的结果. 但如果有 $x_j$ 是实数的 Phase-Retrieval 算法, 相信效率会比 LS 还要高许多. 但总时间应该不会有太大改进, 因为测试中, 单独优化 XUV 的时间并不占大比例.

也有考虑过用之前的 Exact-FrogCrab 算法来优化 XUV, 但是计算 $\mat A$ 的时间就会更长(因为还要做许多 DFT), 也是没太大必要.
