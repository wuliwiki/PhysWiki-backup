% 自由粒子拉格朗日函数(狭义相对论)
% 场论|经典场论|相对论|分析力学|理论力学|作用量原理

\pentry{从分析力学到场论\upref{CFa1}}

本文中$c=1$,闵可夫斯基空间度规为$\opn{diag}(1, -1, -1, -1)$.

\subsection{自由粒子和自由场}

拉格朗日函数能描述场和粒子的运动规律.如果宇宙中只存在一个场或者粒子,我们就说它是自由的,因为不存在任何其它东西与它相互作用.

\subsubsection{粒子}

自由粒子的拉格朗日函数如何确定?注意到拉格朗日的积分,即作用量,作为运动轨迹的泛函,和坐标的选取无关,因此我们可以猜想用只和轨迹相关的某个泛函作为作用量.最直接的想法是什么呢?轨迹的“长度”:
\begin{equation}\label{FPLSR_eq1}
S(\Gamma) = \int_\Gamma \dd s
\end{equation}
这里的$\Gamma$为给定的轨迹,$\dd s$为轨迹上两点的微分\footnote{即任取轨迹上两点,求它们的间隔,然后令两点趋于同一点$\bvec{x}_0$,则所求的间隔趋于$\bvec{x}_0$处的$\dd s$.}.有的材料里,也把间隔$\dd s$记为\textbf{固有时}$\dd \tau$.

如果你更习惯把作用量看成对时间的积分,那么\autoref{FPLSR_eq1} 也可以写为
\begin{equation}\label{FPLSR_eq2}
S(\Gamma) = \int_{t_1}^{t_2} \sqrt{1-v^2} \dd t
\end{equation}
其中$v$是三维速度,由$\Gamma$给定.

将\autoref{FPLSR_eq2} 代入\textbf{欧拉—拉格朗日方程}\upref{Lagrng}方程,得这个作用量所确定的粒子运动:
\begin{equation}
\frac{\dd}{\dd t}\frac{-v}{\sqrt{1-v^2}} = 0
\end{equation}
即$v$为常量.这确实符合四动量守恒定律,因此是狭义相对论的合法运动.

在经典力学中我们知道,同一运动轨迹可以由不同形式的拉格朗日函数确定.比如,两个拉格朗日函数如果只差一个常数倍数,那它们确定的运动规律完全相同.因此,我们应设自由粒子的拉格朗日函数为
\begin{equation}
L(t, \bvec{x}, \dot{\bvec{x}}) = \alpha\sqrt{1-v^2}
\end{equation}

当$v\ll 1$,近似有$\sqrt{1-v^2}=1-\frac{v^2}{2}$,而经典力学的自由粒子作用量为$\frac{mv^2}{2}$.又考虑到我们需要的是\textbf{最小}作用量原理,即粒子的合法轨迹应该是作用量极小而非极大的情况,故需要设$\alpha=-m$以满足这两个条件.

综上可得自由粒子的拉格朗日函数
\begin{equation}\label{FPLSR_eq3}
L(t, \bvec{x}, \dot{\bvec{x}}) = -m\sqrt{1-v^2}
\end{equation}

\begin{exercise}{}
如果不设$c=1$,而需要带着$c$计算,那么\autoref{FPLSR_eq3} 应该是什么样,才能和经典力学统一?
\end{exercise}