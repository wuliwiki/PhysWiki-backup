% Visual Studio Code 笔记
% keys 
% license Usr
% type Note

% 留在百科

\subsection{基本设置}

注意 windows 中需要 \enref{git for windows}{Git} 才能显示修改位置以及显示修改内容。

所有的设置都在 \verb`setting.json`, 路径是:
\begin{itemize}
\item Windows \verb`%APPDATA%\Code\User\settings.json`
\item Linux \verb`$HOME/.config/Code/User/settings.json`
\item macOS \verb`$HOME/Library/Application\ Support/Code/User/`
\end{itemize}

也可以在设置菜单中搜索 \verb`Edit in settings.json`

\begin{lstlisting}[language=json,caption=settings.json]
{
	"editor.minimap.enabled": false,
	"files.hotExit": "off",
	"editor.wordWrap": "on",
    "editor.insertSpaces": false,
	"files.autoGuessEncoding": false,
	"window.zoomLevel": 1,
	"editor.largeFileOptimizations": false,
	"julia.enableTelemetry": true,
	"julia.enableCrashReporter": true,
	"files.eol": "\n",
	"breadcrumbs.enabled": true,
	"editor.detectIndentation": true,
	"security.workspace.trust.enabled": false
	"files.associations": {
		"*.mak": "makefile",
		"*.inp": "c",
		"*.in": "cpp"
	},
	"git.ignoreWindowsGit27Warning": true,
	"window.zoomLevel": 1
}
\end{lstlisting}


\subsection{linux 远程 gui}
\begin{itemize}
\item 要安装可以在官网下载安装包, 然后 \verb`sudo apt install ./name.deb`
\item 【不要用】也可以用 snap 安装 \verb`snap install code --classic` (在 bandwagon 成功, Uwe 机器失败, 另外似乎不支持搜狗输入法)
\item 完了以后在命令行输入 \verb`code` 就可以打开远程 gui, 直接打开文件如 \verb`code file.txt`, 打开文件夹如 \verb`code ./`
\end{itemize}
