% 粒子物理学(综述)
% license CCBYSA3
% type Wiki

本文根据 CC-BY-SA 协议转载翻译自维基百科\href{https://en.wikipedia.org/wiki/Particle_physics}{相关文章}。

粒子物理学或高能物理学是研究构成物质和辐射的基本粒子和力的学科。该领域还研究从基本粒子到质子和中子尺度的粒子组合,而研究质子和中子组合的学科称为核物理学。

宇宙中的基本粒子在标准模型中被分类为费米子(物质粒子)和玻色子(传递力的粒子)。费米子有三代,然而普通物质仅由第一代费米子构成。第一代包括构成质子和中子的上夸克和下夸克,以及电子和电子中微子。已知由玻色子介导的三种基本相互作用是电磁相互作用、弱相互作用和强相互作用。

夸克不能单独存在,而是形成强子。含有奇数个夸克的强子称为重子,而含有偶数个夸克的强子称为介子。两个重子,质子和中子,构成了普通物质的大部分质量。介子是不稳定的,最长寿命的介子也只有几微秒。介子发生在由夸克组成的粒子之间的碰撞后,例如在宇宙射线中快速运动的质子和中子。介子也可以在回旋加速器或其他粒子加速器中产生。

粒子有对应的反粒子,具有相同的质量,但电荷相反。例如,电子的反粒子是正电子。电子带负电,正电子带正电。这些反粒子理论上可以形成一种对应的物质形式,称为反物质。一些粒子,如光子,是它们自身的反粒子。

这些基本粒子是量子场的激发,这些量子场也支配着它们的相互作用。解释这些基本粒子和场及其动力学的主导理论被称为标准模型。引力与当前粒子物理学理论的统一尚未解决;许多理论已经试图解决这个问题,如环量子引力理论、弦理论和超对称理论。

实际粒子物理学是研究这些粒子在放射性过程和粒子加速器中的行为,如大型强子对撞机。理论粒子物理学则是在宇宙学和量子理论的背景下研究这些粒子。这两者是紧密相关的:希格斯玻色子是由理论粒子物理学家提出的,并通过实际实验确认了它的存在。
\subsection{历史}
\begin{figure}[ht]
\centering
\includegraphics[width=10cm]{./figures/f59246dd62ba0e0e.png}
\caption{盖革–马斯登实验观察到,当α粒子撞击金箔时,一小部分α粒子发生了剧烈的偏转。} \label{fig_Partic_1}
\end{figure}
所有物质基本上由基本粒子构成的思想至少可以追溯到公元前6世纪。[1] 在19世纪,约翰·道尔顿通过对化学计量学的研究得出结论,认为自然界的每种元素都由一种独特的粒子组成。[2] “原子”一词来源于希腊语“atomos”,意为“不可分割”,此后该词一直用来表示化学元素中最小的粒子,但物理学家后来发现,原子实际上并不是自然界的基本粒子,而是由更小的粒子(如电子)组成的聚合物。20世纪初,核物理学和量子物理学的研究导致了1939年莉泽·迈特纳(基于奥托·哈恩的实验)证明了核裂变,并且汉斯·贝特在同一年发现了核聚变;这两项发现也促成了核武器的发展。贝特1947年计算的兰姆位移被认为“开启了粒子物理学现代时代的道路”。[3]

在1950年代和1960年代,随着越来越高能量的粒子束碰撞,发现了种类繁多的粒子,这种现象被非正式地称为“粒子动物园”。詹姆斯·克罗宁和瓦尔·费奇的CP破坏等重要发现提出了物质和反物质不平衡的新问题。[4] 在1970年代标准模型的制定之后,物理学家澄清了粒子动物园的来源。大量粒子的出现被解释为(相对较小数量的)更基本粒子的组合,并被框架化在量子场理论的背景下。这种重新分类标志着现代粒子物理学的开始。[5][6]
\subsection{标准模型}  
目前所有基本粒子的分类状态由标准模型解释,该模型在1970年代中期获得广泛接受,此时实验确认了夸克的存在。标准模型描述了强相互作用、弱相互作用和电磁相互作用,并通过介质规范玻色子来实现这些相互作用。规范玻色子的种类包括八种胶子、W⁻、W⁺和Z玻色子,以及光子。标准模型还包含24种基本费米子(12种粒子及其对应的反粒子),这些费米子是所有物质的组成部分。最后,标准模型还预测了一种名为希格斯玻色子的玻色子的存在。2012年7月4日,欧洲核子研究组织(CERN)的“大型强子对撞机”实验的物理学家宣布,他们发现了一种新粒子,其行为与预期的希格斯玻色子相似。

目前的标准模型包含61种基本粒子。这些基本粒子可以组合成复合粒子,解释了自1960年代以来发现的数百种其他粒子种类。标准模型与迄今为止进行的几乎所有实验测试一致。然而,大多数粒子物理学家认为,标准模型是对自然的一个不完整描述,且更为基本的理论有待发现(参见万物理论)。近年来,中微子质量的测量提供了与标准模型的首个实验偏差,因为标准模型中认为中微子没有质量。
\subsection{亚原子粒子}
\begin{figure}[ht]
\centering
\includegraphics[width=10cm]{./figures/3f36f74204fb82e2.png}
\caption{} \label{fig_Partic_2}
\end{figure}
现代粒子物理学研究的重点是亚原子粒子,包括原子成分,如电子、质子和中子(质子和中子是由夸克组成的复合粒子,称为重子),这些粒子通过放射性过程和散射过程产生;这些粒子包括光子、中微子、缪子以及各种异域粒子。迄今为止观察到的所有粒子及其相互作用几乎可以完全通过标准模型来描述。

粒子的动态也由量子力学支配;它们表现出波粒二象性,在某些实验条件下表现出粒子特性,而在其他条件下则表现出波动特性。更技术化的说法是,它们通过量子态矢量描述,这些矢量位于希尔伯特空间中,量子场论也对其进行了处理。遵循粒子物理学家的惯例,术语“基本粒子”用于描述那些根据当前理解,被假定为不可分割且不由其他粒子组成的粒子。
\subsubsection{夸克和轻子}
普通物质由第一代夸克(上夸克、下夸克)和轻子(电子、电子中微子)构成。[13] 夸克和轻子统称为费米子,因为它们具有半整数的量子自旋(−1/2,1/2,3/2等)。这使得费米子遵循泡利不相容原理,即任何两个粒子不能占据相同的量子态。[14] 夸克具有分数电荷(−1/3或2/3),而轻子具有整数电荷(0或−1)。[15][16] 夸克还具有色荷,这个色荷是任意标记的,与实际的光色没有关联,通常标记为红色、绿色和蓝色。[17] 由于夸克之间的相互作用会储存能量,而当夸克足够远离时,这些能量可以转化为其他粒子,因此夸克无法独立观测到。这种现象称为色禁闭。[17]

已知夸克有三代(上夸克和下夸克、奇夸克和魅夸克、顶夸克和底夸克),轻子也有三代(电子及其中微子、μ子及其中微子、τ子及其中微子),有强有力的间接证据表明不存在第四代费米子。[18]