% 类氢原子斯塔克效应(微扰)
% license Xiao
% type Tutor

\begin{issues}
\issueTODO
\end{issues}

\pentry{不含时微扰理论\nref{nod_TIPT}}{nod_8e0d}

微扰理论($\mathcal{E_z}$ 是 $z$ 方向电场):
\begin{equation}
H' = \mathcal{E_z} z~.
\end{equation}
矩阵元为
\begin{equation}\label{eq_HStark_1}
H'_{l',l} = \mathcal{E_z}\mel{\psi_{n,l',m}}{z}{\psi_{n,l,m}}~.
\end{equation}

但事实上氢原子加上匀强电场后是不存在数学上严格的束缚态的,因为无论电场多弱,在电场反方向的某个距离外,势能都会小于基态能量,使波函数变为散射态。在含时薛定谔方程中,波函数可能会有长时间处于微扰理论给出的 “束缚态”,但这只能算是一种\textbf{亚稳态(metastable state)},仍会有不为零的隧道电离概率。

\begin{example}{类氢原子 $n=2$ 的斯塔克效应}
先考虑 $n=2$, $m=0$ 的情况, 这是一个二维希尔伯特子空间,基底为 $\ket{2,0,0}$ 和 $\ket{2,1,0}$。 根据\autoref{tab_HDipM_1}~\upref{HDipM}, \autoref{eq_HStark_1} 为
\begin{equation}
\mat H' = -\frac{3}{Z}\mathcal{E_z}\pmat{0 & 1\\ 1 & 0}~.
\end{equation}
本征值为 $E_{\pm}^1 = \mp \frac{3}{Z}\mathcal{E}_z$, 好本征态为 $\ket{2\pm 0} = (\ket{200} \pm \ket{210})/{\sqrt 2}$, 也被称为 \textbf{Stark 态}。

\begin{figure}[ht]
\centering
\includegraphics[width=8cm]{./figures/cf2a4c9b0548dbae.png}
\caption{$\ket{2+}$ 的概率密度函数的 $x$-$z$ 切面, 可见电子向下偏移, 电场向上为正, 所以本征能量变小。 $\ket{2-}$ 态是此图上下翻转, 本征能量变大。} \label{fig_HStark_1}
\end{figure}

不要以为\autoref{fig_HStark_1} 是外电场扭曲波函数的结果, $\ket{2\pm}$ 本身就是无电场的氢原子 $n=2$ 本征态。 施加了电场后波函数反而需要进一步修正。

从经典电磁学角度来理解, 电偶极子在电场中的能量(\autoref{eq_eleDP2_1}~\upref{eleDP2})等于 $-d_z \mathcal{E}_z$, 其中 $d_z$ 是 $z$ 方向电偶极子
\begin{equation}
d_z^{(\pm)} = \mel{2\pm}{z}{2\pm} = \pm 3~.
\end{equation}
\end{example}

\begin{example}{氢原子斯塔克效应(能级截断)}\label{ex_HStark_2}
事实上,在\autoref{ex_HStark_1} 中使用\autoref{eq_HStark_2} 计算二阶微扰十分繁琐,通常需要编程来数值计算。 除了按不同阶来计算微扰外,还有一个更直接的近似方法就是直接把总哈密顿 $H = H^0+H^1$ 用有限个 $\ket{n,l,m}$ (可以用 $n_\text{max}$ 作为截断条件)完整表示(例如\autoref{tab_HDipM_1}~\upref{HDipM}),然后数值将其对角化。 当 $n_\text{max}$ 无穷大,结果就是精确的(包含任意阶微扰)。若使用编程,该算法将比\autoref{ex_HStark_1} 更容易实现。

首先可以用 “氢原子的跃迁偶极子矩阵元列表\upref{HDipM}” 中的 \verb`h_dipole_z.m` 程序生成 $\mat H^1$ 矩阵, 然后用 Matlab 自带的 \verb`eigen()` 函数求 $\mat H^0+\mat H^1$ 的本征值, 画出每个能级关于电场强度的曲线。

\end{example}

\begin{example}{氢原子的极化率(二阶微扰)}\label{ex_HStark_1}
若氢原子处于某个好量子态,使用一阶微扰求其极化率(polarizability) $\alpha$,定义为 $\bvec p_z = \alpha \bvec{\mathcal E_z}$,其中 $\bvec p_z$ 为 $z$ 方向的电偶极子, $\bvec{\mathcal E_z}$ 为电场强度。(实验数据参考\href{https://physicspages.com/pdf/Electrodynamics/Polarizability\%20of\%20hydrogen.pdf}{这篇})

该问题中, $H^1 = \mathcal{E}_z z$, $p_z = -\mel{\psi_n}{z}{\psi_n}$。 使用波函数的一阶修正 $\psi_n \approx \psi_n^0 + \psi_n^1$, 有
\begin{equation}\ali{
p_z &\approx -\mel{\psi_n^0 + \psi_n^1}{z}{\psi_n^0 + \psi_n^1}\\
&= -\mel{\psi_n^0}{z}{\psi_n^0} - 2\Re[\mel{\psi_n^0}{z}{\psi_n^1}] - \mel{\psi_n^1}{z}{\psi_n^1}~.
}\end{equation}
其中第一项就是 $E_n^1/\mathcal{E}_z$。 第二项第三项分别是二阶和三阶小量,所以第三项可忽略。 根据\autoref{eq_TIPT2_4}~\upref{TIPT2}, 第二项的 $\mel{\psi_n^0}{z}{\psi_n^1}$ 对应二阶能量修正 $E_n^2/\mathcal{E}_z$,所以
\begin{equation}
p_{n,z} \approx p_{n,z}^1 + p_{n,z}^2 = -\frac{1}{\mathcal{E}_z}(E_n^1 + 2E_n^2)~.
\end{equation}
其中一阶能量修正
\begin{equation}
E_n^1 = \mel{\psi_n^0}{H^1}{\psi_n^0}~
\end{equation}
和电场成正比,而二阶能量修正(\autoref{eq_TIPT2_2}~\upref{TIPT2})
\begin{equation}\label{eq_HStark_2}
E_n^2 = \sum_{m}^{E_m\ne E_n} \frac{\abs{\mel{\psi_m^0}{H^1}{\psi_n^0}}^2}{E_n^0-E_m^0}~.
\end{equation}
正比于电场平方。

可见 $p_{n,z}^1$ 是一个常数, 而 $p_{n,z}^2$ 正比于电场,并决定极化率 $\alpha_n$。
\addTODO{计算 $E_n^2$,参考\autoref{sub_HDipM_1}~\upref{HDipM},但太复杂了,还是应该用\autoref{ex_HStark_2} 的方法}
\end{example}

\subsection{含时问题}
以上我们讨论的都是不含时问题。 但是在真正的斯塔克效应实验中, 电场是对时间慢慢增加的。 如果初始时, 波函数处于 $n=2$ 子空间的任意状态, 例如 $\ket{20}$, 那么当缓慢施加电场后, 波函数会如何变化? 这可以参考 “绝热近似(量子力学)\upref{AdiaQM}”。

\begin{example}{$\ket{n,l,m}$ 的极化率}
那么如何确定非好本征态如 $\ket{n,l,m}$ 的极化率呢? 考虑 TDSE, 根据绝热近似\upref{AdiaQM},若波函数初始处于 $\ket{n,l,m}$,需要先分解为好本征态的线性组合,
\begin{equation}
\ket{n,l,m} = \sum_\alpha c_{n,l,\alpha,m} \ket{n,\alpha,m}~.
\end{equation}
然后再考虑每个好本征态在电场中的变化。 把经过电场扭曲后的 $\ket{n,l,m}$ 记为 $\ket{n,l,m,*}$,则
\begin{equation}
\ket{n,l,m,*} = \sum_\alpha c_{n,l,\alpha,m} \ket{n,\alpha,m,*}~.
\end{equation}
\begin{equation}
p_{(n,l,m),z} = -\mel{n,l,m,*}{z}{n,l,m,*} = -\sum_\alpha \abs{c_{n,l,\alpha,m}}^2\mel{n,\alpha,m,*}{z}{n,\alpha,m,*}~.
\end{equation}
其中近似认为 $\alpha\ne\alpha'$, $\mel{n,\alpha',m,*}{z}{n,\alpha,m,*} = 0$(\autoref{sub_TIPT_2}~\upref{TIPT})。在电场为零时, 任意 $p_{(n,l,m),z} = 0$, 所以只有二阶修正起作用:
\begin{equation}
p_{(n,l,m),z} = \sum_\alpha \abs{c_{n,l,\alpha,m}}^2 p_{(n,\alpha,m),z}^2~.
\end{equation}
注意和电场成正比。
\end{example}
