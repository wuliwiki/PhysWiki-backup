% 电磁场的参考系变换0

\begin{issues}
\issueDraft
\end{issues}

\pentry{相对论速度变换\upref{RelVel}, 麦克斯韦方程组\upref{MWEq}, 洛伦兹力\upref{Lorenz}}

在 $S_0$ 参考系中有一个电荷密度为 $\lambda$ 的无限长直导线于 $x$ 轴重合并静止. 距离导线 $r_0$ 处有一个电荷为 $q$ 的粒子沿 $\uvec x$ 方向以速度 $v_0$ 运动. 另外两个参考系 $S_1, S_2$ 相对 $S_0$ 沿 $\uvec x$ 方向运动, 速度分别为 $u_1, u_2$. 在这三个参考系中, 粒子所受的电磁力是否相同?

在 $S_0$ 参考系中, 假设粒子在导线上方($\uvec z$ 方向), 那么粒子处的电场只有 $z$ 分量, 磁场只有 $y$ 分量(如果导线中存在电流)
\begin{equation}
E_{z,0} = \frac{1}{2\pi\epsilon_0 r_0} \lambda
\qquad
B_{y,0} = 0
\end{equation}
在 $S_i$ ($i=1,2$)参考系中, 线电荷密度变为 $\lambda_i = \gamma_i \lambda$, 电流为 $I_i = \gamma_i \lambda u_i$, 电磁场分别为
\begin{equation}
E_{z,i} = \frac{1}{2\pi\epsilon_0 r_0} \gamma_i \lambda
\qquad
B_{y,i} = -\frac{\mu_0}{2\pi r_0}\gamma_i \lambda u_i
\end{equation}
这个结论既符合直接计算也符合场变换公式.

\begin{equation}
\gamma_{u+v} = \gamma_u\gamma_v(1 + uv/c^2)
\end{equation}
