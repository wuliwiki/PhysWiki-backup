% 晶格振动导论
% 声子;晶格振动

我们常常对固体模型作一定程度上的简化,并将它分为两部分来研究,一部分是占据在格点处的离子实,相邻两个格点上的离子实之间存在相互作用势,因此它们在平衡位置附近作振动\pentry{词条示例\upref{Sample}};另一部分是价电子在晶格中的行为,电子与离子、电子与电子之间的库仑势能共同形成了晶格中的势能分布,从而影响了电子的波函数。前者被称为\textbf{“晶格振动”}。根据量子力学,晶格振动模式的能量本征值是分立的,而晶格振动的激发可以被视作“声子”,研究晶格的振动的方法于是转化为对声子气体的研究。

研究晶格振动,我们可以从最简单的模型(一维晶格)出发。通过对\textbf{一维单原子链晶格\upref{onatom}}的分析,我们得到了一维晶体的声子的\textbf{色散关系},它反映了各种晶格振动模式的频率、波数对应关系。我们可以进一步讨论\textbf{一维双原子链晶格\upref{twatom}},其中两个原子构成一个新的原胞。通过对色散关系的计算,我们发现双原子链晶格的声子能带存在光学支,也因此存在红外吸收峰,这在固体物理中是非常重要的现象。随后我们可以将模型推广到\textbf{三维晶格的振动}。在长波极限下,一般而言,沿一个方向传播的声子有三种,其中两个是横波,一个是纵波,在色散关系图上对应三个声学支。

利用\textbf{玻色-爱因斯坦分布}可以得到不同频率的声子分布函数和温度之间的关系。再通过计算声子的能态密度,可以得到晶格振动的总能量。用这种方法可以计算声子气体对\textbf{晶格热容}的贡献。有两个简化版的近似理论,一个是\textbf{晶格热容的爱因斯坦理论\upref{EScap}},另一个是\textbf{晶格热容的德拜理论\upref{Debye}},它们都对声子色散关系作了简化处理,从而得到晶格热容的一个简化表达式。

最后,我们还要研究晶格中相互作用的非谐项,即拉格朗日量中的三次方项或更高阶项。它们对应着声子之间的碰撞,或多个声子之间的相互作用。这些非谐项来自于库仑势本身,以及非理想晶格中的杂质、边界等。这些非谐项并不是可以忽略的,而是相当重要的,它直接“导致”了晶格的\textbf{热传导}现象。