% 欧几里得(综述)
% license CCBYSA3
% type Wiki

本文根据 CC-BY-SA 协议转载翻译自维基百科\href{https://en.wikipedia.org/wiki/Euclid}{相关文章}。

\begin{figure}[ht]
\centering
\includegraphics[width=6cm]{./figures/7848a3bb491b282e.png}
\caption{《欧几里得》,由朱塞佩·德·里贝拉(Jusepe de Ribera)创作,约1630–1635年。} \label{fig_Euclid_1}
\end{figure}
欧几里得(/ˈjuːklɪd/;古希腊语:Εὐκλείδης;约公元前300年活跃)是一位古希腊数学家,主要从事几何学和逻辑学的研究。被称为“几何学之父”,他最著名的成就是《几何原本》这部著作,它奠定了几何学的基础,直到19世纪初期,这些基础一直主导着该领域。他的体系,现被称为欧几里得几何学,结合了之前希腊数学家的理论创新与综合,包括厄多克索斯(Eudoxus of Cnidus)、希波克拉底(Hippocrates of Chios)、泰勒斯(Thales)和提阿托斯(Theaetetus)等人的理论。与阿基米德和佩尔加的阿波罗尼斯一起,欧几里得通常被认为是古代最伟大的数学家之一,也是数学史上最具影响力的人物之一。

关于欧几里得的生平知之甚少,绝大多数资料来源于几百年后学者普罗克洛斯(Proclus)和帕普斯(Pappus)的记载。中世纪的伊斯兰数学家创造了一个富有幻想色彩的生平,而中世纪拜占庭学者和早期文艺复兴学者则误将他与早期哲学家梅加拉的欧几里得混淆。现在普遍认为,欧几里得大部分时间都在亚历山大里亚度过,约生活在公元前300年,介于柏拉图的学生与阿基米德之间。也有一些猜测认为,欧几里得曾在柏拉图学园学习,后来在穆萨尤姆教授数学;他被认为是将早期的柏拉图学派传统与后来的亚历山大里亚学派传统连接起来的桥梁。

在《几何原本》中,欧几里得通过少数公理推导出定理。他还撰写了关于透视学、圆锥曲线、球面几何、数论和数学严谨性的著作。除了《几何原本》,欧几里得还写了光学领域的基础性著作《光学》,以及其他较不为人知的作品,如《数据》和《现象》。关于《几何分割》和《镜面反射学》是否为欧几里得所作,学术界仍有争议。欧几里得还被认为撰写了许多已经失传的作品。
\subsection{生命}  
\subsubsection{传统叙述}
\begin{figure}[ht]
\centering
\includegraphics[width=8cm]{./figures/0d7a20b6f91e0fa9.png}
\caption{拉斐尔在《雅典学派》(1509–1511)中的欧几里得形象细节,展示了他在教授学生。} \label{fig_Euclid_2}
\end{figure}
“Euclid”这个英语名字是古希腊名字Eukleídes(Εὐκλείδης)的英文化版本。[4][a] 它来源于“eu-”(εὖ;意为“好”)和“klês”(-κλῆς;意为“名声”),意思是“著名的,荣耀的”。[6] 在英语中,"Euclid"通过转喻有时指代他最著名的作品《几何原本》,或其副本,[5] 有时也被当作“几何”的同义词。[2]

与许多古希腊数学家一样,关于欧几里得的生平细节大多未知。[7] 他被认为是四部大部分存世的著作的作者——《几何原本》、 《光学》、 《数据》、 《现象》——但除此之外,关于他的确切信息几乎没有。[8][b] 传统的叙述主要依赖于公元5世纪普罗克洛斯在其《欧几里得《几何原本》第一卷注释》中的记载,以及公元4世纪初亚历山大的帕普斯的一些轶事。[4][c]

根据普罗克洛斯的说法,欧几里得生活在柏拉图(公元前347年去世)的几位追随者之后,并且在数学家阿基米德(公元前287年–公元前212年)之前;[d] 具体来说,普罗克洛斯将欧几里得置于托勒密一世统治时期(公元前305/304–282年)。[7][8][e] 欧几里得的出生日期不详;一些学者估计大约在公元前330年[11][12] 或公元前325年,[2][13] 但其他学者则避免做出推测。[14] 假设他是希腊血统,[11] 但他的出生地未知。[15][f] 普罗克洛斯认为欧几里得遵循柏拉图的传统,但没有确凿的证据可以证实这一点。[17] 他不太可能与柏拉图同代,因此通常推测他是柏拉图的学生,在雅典的柏拉图学园接受教育。[18] 历史学家托马斯·希思支持这一理论,指出大多数有能力的几何学家都生活在雅典,包括许多欧几里得依赖的前人的工作;[19] 历史学家米哈利斯·西亚拉罗斯认为这只是一个猜测。[4][20] 无论如何,欧几里得的著作内容表明他熟悉柏拉图几何学的传统。[11]

在《帕普斯集》中,帕普斯提到阿波罗尼乌斯曾与欧几里得的学生一起在亚历山大学习,这表明欧几里得曾在那里工作并创立了一个数学传统。[8][21][19] 该城市由亚历山大大帝于公元前331年建立,[22] 托勒密一世自公元前306年起统治,使其在亚历山大帝国分裂后的混乱战争中拥有相对的稳定性。[23] 托勒密开始了希腊化进程,并委托建造许多建筑,建立了庞大的穆塞翁学术机构,这是当时的一个领先教育中心。[15][g] 推测欧几里得是穆塞翁最早的学者之一。[22] 欧几里得的死亡日期不详;有学者推测他大约在公元前270年去世。[22]
\subsubsection{身份与历史性}
\begin{figure}[ht]
\centering
\includegraphics[width=8cm]{./figures/9a263cfe27f2006e.png}
\caption{多梅尼科·马罗利(Domenico Maroli)在1650年代的画作《厄尔克里德·梅加拉(Euclid of Megara Dressing as a Woman to Hear Socrates Teach in Athens)》描绘了厄尔克里德(Euclid)以女性装扮前往雅典聆听苏格拉底的讲授。当时,哲学家厄尔克里德与数学家厄尔克里德被错误地认为是同一个人,因此这幅画中桌子上放置了数学物品。[25]} \label{fig_Euclid_3}
\end{figure}
欧几里得通常被称为“亚历山大城的欧几里得”,以区分他与早期的哲学家梅加拉的欧几里得(苏格拉底的弟子,曾出现在柏拉图的对话录中),两者在历史上常常被混淆。[4][14] 公元1世纪的罗马编年史家瓦勒留斯·马西穆斯(Valerius Maximus)错误地将欧几里得的名字与欧多克索斯(公元前4世纪的数学家)互换,误将他作为柏拉图派遣给询问如何立方倍增的人们的数学家。[26] 由于这一提到大约百年之前的数学欧几里得,欧几里得与梅加拉的欧几里得在中世纪拜占庭文献中(现在已失传)混淆,最终导致欧几里得这位数学家被附上了两位人物生平的细节,并被描述为“梅加拉人”(Megarensis)[4][28]。拜占庭学者西奥多·梅托基特斯(约1300年)明确地将这两位欧几里得混为一谈,印刷商埃尔哈德·拉特多尔特(Erhard Ratdolt)也在其1482年版的《元素》拉丁语版中沿用了这种说法。[27] 在数学家巴托洛梅奥·赞贝尔蒂(Bartolomeo Zamberti)于1505年翻译《元素》时,将有关两位欧几里得的现存生平片段附在了前言中,此后相关出版物都沿用了这一辨识方法。[27] 后来的文艺复兴学者,尤其是彼得·拉姆斯(Peter Ramus),重新评估了这一观点,并通过年代学问题和早期文献中的矛盾证明其错误。[27]

中世纪阿拉伯文献提供了大量关于欧几里得生平的信息,但这些资料完全无法验证。[4] 大多数学者认为这些资料的真实性存疑;[8] 赫斯特别指出,这些虚构的内容是为了加强这位受人尊敬的数学家与阿拉伯世界的联系。[17] 也有许多关于欧几里得的轶事,虽然它们的历史性尚不确定,这些故事“将他描绘为一位和蔼可亲、温和的老人”。[29] 最著名的故事是普罗克卢斯(Proclus)讲述的关于托勒密问欧几里得是否有比读《元素》更快捷的几何学习方式,欧几里得回答说:“没有王道可以走捷径。”[29] 这一轶事值得怀疑,因为在斯托巴乌斯(Stobaeus)中也记录了梅内克莫斯(Menaechmus)与亚历山大大帝之间非常相似的互动。[30] 这两段记载均写于公元5世纪,且没有注明来源,且都未出现在古希腊文献中。[31]

关于欧几里得约公元前300年的活动的任何确定性日期都受到当时缺乏直接提及的质疑。[4] 欧几里得的最早原始提及出现在阿波罗尼乌斯(Apollonius)写给《圆锥曲线》序言中的信中(公元前2世纪初):“《圆锥曲线》第三卷包含许多令人惊讶的定理,这些定理对于解方程和求解空间位置的解的数目非常有用。其中大多数,特别是最精妙的,是新的。当我们发现这些定理时,我们意识到欧几里得并没有完成三线和四线的定位,只完成了一个偶然的片段,而且即使那个片段也做得不太好。”[26] 《元素》推测至少部分在公元前3世纪就已流传,因为阿基米德和阿波罗尼乌斯都理所当然地使用了其中的一些命题;[4] 然而,阿基米德采用了与《元素》不同的比例理论的早期版本。[8] 《元素》中所含材料的最古老物理副本可追溯至公元100年左右,在埃及奥克里恩库斯的废墟中发现的纸草纸片上。[26] 直接引用《元素》且日期明确的最早文献出现在公元2世纪,由盖伦(Galen)和阿弗罗狄西亚的亚历山大(Alexander of Aphrodisias)提及;到那时,《元素》已经成为标准的学校教材。[26] 一些古希腊数学家提到欧几里得时,通常称他为“ὁ στοιχειώτης”(“《元素》的作者”)。[32] 在中世纪,一些学者认为欧几里得不是历史人物,而他的名字来源于希腊数学术语的误传。[33]
\subsection{作品}  
\subsubsection{《几何原本》}
\begin{figure}[ht]
\centering
\includegraphics[width=8cm]{./figures/592c992974e918fd.png}
\caption{一块公元约75–125年的《几何原本》纸莎草 fragment,发现于奥克斯里欣库斯(Oxyrhynchus),该图表伴随于第二卷第五命题。[34]} \label{fig_Euclid_4}
\end{figure}
欧几里得以他的十三卷著作《几何原本》最为人知(古希腊文:Στοιχεῖα;Stoicheia),被认为是他的代表作。[3][35] 其中大部分内容来源于早期的数学家,包括欧多克索斯、希奥斯的希波克拉底、塔勒斯和西阿图斯,而其他一些定理则被柏拉图和亚里士多德提及。[36] 由于《几何原本》本质上取代了许多早期并且现已失传的希腊数学,因此很难区分欧几里得的工作与其前辈的工作。[37][h] 古典学者马克斯·阿斯珀(Markus Asper)总结道:“显然,欧几里得的成就在于将公认的数学知识整理成一个有条理的结构,并增加新的证明以填补空白”,而历史学家塞拉菲娜·库莫(Serafina Cuomo)则将其描述为一个“结果的宝库”。[38][36] 尽管如此,西阿拉罗斯(Sialaros)进一步指出:“《几何原本》令人瞩目的紧密结构展示了超越单纯编辑的作者控制。”[9]

《几何原本》并不仅仅讨论几何学,正如有时人们所认为的那样。[37] 它通常被分为三个主题:平面几何(第1–6卷)、基础数论(第7–10卷)和立体几何(第11–13卷)——尽管第5卷(关于比例)和第10卷(关于无理线)并不完全符合这一划分。[39][40] 该书的核心是散布在其中的定理。[35] 使用亚里士多德的术语,这些定理大致可以分为两类:“第一原理”和“第二原理”。[41] 第一类包括被标记为“定义”(古希腊文:ὅρος 或 ὁρισμός)、“公设”(αἴτημα)或“共同概念”(κοινὴ ἔννοια)的陈述;[41][42] 只有第一卷包括公设——后来的公理——和共同概念。[37][i] 第二类由命题组成,呈现时伴有数学证明和图示。[41] 不清楚欧几里得是否将《几何原本》视为一本教科书,但其呈现方式使其成为一种自然的教科书选择。[9] 整体而言,作者的语气保持一般性和非个人化。[36]

\textbf{目录}

\begin{table}[ht]
\centering
\caption{欧几里得的公设和共同概念[43]}\label{Euclid}
\begin{tabular}{|c|c}
\hline \textbf{NO}.& \textbf{公设}\\
\hline 设定如下公设:& \\
\hline 1. & 从任何一点到任何一点画一条直线[j]\\ 
\hline 2. & 将一条有限的直线在同一方向上延伸 \\ 
\hline 3. & 以任意点为中心,任意距离画一个圆 \\ 
\hline 4. & 所有直角都相等\\  
\hline 5. & 如果一条直线与两条直线相交,使得同一侧的内角小于两个直角,那么这两条直线如果无限延长,将在内角小于两个直角的一侧相交\\
\hline \textbf{NO}.& \textbf{共同概念设}\\
\hline 1. & 等于同一事物的事物也相互相等\\  
\hline 2. & 如果相等的东西加上相等的东西,则整体相等\\  
\hline 3. & 如果相等的东西从相等的东西中减去,则剩余部分相等\\  
\hline 4. & 相互重合的事物相等\\  
\hline 5. & 整体大于部分\\
\hline 
\end{tabular}
\end{table}
《几何原本》第1卷是整本书的基础。[37] 它以一系列20个定义开始,涉及基本的几何概念,如直线、角度和各种规则的多边形。[44] 然后,欧几里得提出了10个假设(见右表),分为五个公设(公理)和五个共同概念。[45][k] 这些假设旨在为随后的每个定理提供逻辑基础,即作为一个公理系统。[46][l] 共同概念仅涉及量的比较。[48] 虽然公设1到4相对简单,[m] 第5个公设被称为平行公设,并且特别著名。[48][n] 第1卷还包括48个命题,可以大致分为以下几类:关于平面几何的基本定理和三角形全等的命题(1–26);平行线(27–34);三角形和梯形的面积(35–45);以及毕达哥拉斯定理(46–48)。[48] 最后一项包括现存的最早的毕达哥拉斯定理证明,西阿拉罗斯将其描述为“非常精妙”。[41]\begin{figure}[ht]

第2卷传统上被理解为讨论“几何代数”,尽管自1970年代以来这一解释一直备受争议;批评者认为这种表述具有时代错位性,因为即使是初步的代数学基础也是在几个世纪后才出现的。[41] 第二卷的范围较为集中,主要提供了代数定理来配合各种几何形状。[37][48] 它关注于矩形和正方形的面积(参见“求方”),并引出了余弦定律的几何前身。[50] 第3卷关注圆形,第4卷讨论规则多边形,特别是五边形。[37][51] 第5卷是该作品中最重要的部分之一,呈现了通常称为“比例的一般理论”的内容。[52][o] 第6卷在平面几何的背景下运用了“比率理论”。[37] 它几乎完全由其第一个命题构成:[53] “在同一高度下的三角形和梯形,它们的比率等于其底边的比率。”[54]
\centering
\includegraphics[width=10cm]{./figures/b812504681999b70.png}
\caption{五种柏拉图立体,是立体几何的基础组成部分,出现在第11至13卷中。} \label{fig_Euclid_5}
\end{figure}
从第7卷开始,数学家本诺·阿特曼(Benno Artmann)指出:“欧几里得重新开始,前面各卷的内容都没有被使用。”[55] 第7至10卷涵盖了数论,第7卷首先提出了22个定义,涉及偶数、质数以及其他与算术相关的概念。[37] 第7卷包括了欧几里得算法,这是一个用于求两个数最大公约数的方法。[55] 第8卷讨论了几何级数,而第9卷包括了现在称为欧几里得定理的命题,即质数的数量是无限的。[37] 《几何原本》中,第10卷是最大且最复杂的一卷,讨论了在量的背景下的无理数。[41]

最后三卷(第11至13卷)主要讨论立体几何。[39] 第11卷通过引入37个定义,为接下来的两卷提供了背景。[56] 尽管它的基础性质类似于第1卷,但与后者不同,它没有公理系统或公设。[56] 第11卷的三个部分包括关于立体几何的内容(1–19),立体角(20–23)和平行六面体(24–37)。[56]
\subsubsection{其他著作}
\begin{figure}[ht]
\centering
\includegraphics[width=6cm]{./figures/ce59cf9f21c85370.png}
\caption{欧几里得构造正十二面体} \label{fig_Euclid_6}
\end{figure}
除了《几何原本》之外,至少有五部欧几里得的著作保存至今。这些作品遵循与《几何原本》相同的逻辑结构,包括定义和已证明的命题。
\begin{itemize}
\item 《镜面学》(Catoptrics)涉及镜子的数学理论,特别是平面和球面凹面镜中形成的图像,尽管其归属有时受到质疑。[57]  
\item 《已知》(The Data)(古希腊文:Δεδομένα)是一部相对简短的文本,讨论了几何问题中“已知”信息的性质和意义。[57]  
\item 《分割》(On Divisions)(古希腊文:Περὶ Διαιρέσεων)仅部分保存在阿拉伯文翻译中,涉及几何图形的分割,将其分为两个或多个相等的部分,或按照给定的比例进行分割。它包括三十六个命题,并且与阿波罗尼乌斯的《圆锥曲线》相似。[57]  
\item 《光学》(The Optics)(古希腊文:Ὀπτικά)是现存最早的希腊透视学著作。它包括对几何光学的介绍性讨论和基本透视规则。[57]  
\item 《现象》(The Phaenomena)(古希腊文:Φαινόμενα)是一部关于球面天文学的著作,保存有希腊文;它与公元前310年左右活跃的皮塔内的奥托律库斯的《运动的球体》相似。[57]
\end{itemize}
\subsubsection{失传的著作}  
有四部其他著作可信地归于欧几里得,但已失传。[9]

《圆锥曲线》(The Conics)(古希腊文:Κωνικά)是一部四卷本的圆锥曲线综述,后来被阿波罗尼乌斯同名的更为全面的著作所取代。[58][57] 该作品的存在主要通过帕普斯的记载得知,他声称阿波罗尼乌斯的《圆锥曲线》前四卷大部分基于欧几里得的早期作品。[59] 历史学家亚历山大·琼斯[de]对这一说法提出质疑,认为证据稀少且没有其他资料可以佐证帕普斯的说法。[59]

《伪命题》(The Pseudaria)(古希腊文:Ψευδάρια;字面意思为“谬误”)根据普罗克鲁斯的记载(70.1–18),这是一部关于几何推理的著作,旨在向初学者提供避免常见谬误的建议。[58][57] 除了它的范围和一些现存的内容外,关于它的具体内容知之甚少。[60]

《命题集》(The Porisms)(古希腊文:Πορίσματα;字面意思为“推论”)根据帕普斯和普罗克鲁斯的记载,可能是三卷本的著作,包含大约200个命题。[58][57] 在这个语境中,“命题集”一词并不指推论,而是指“第三类命题——介于定理和问题之间——其目的是发现现有几何实体的特征,例如,寻找圆的中心”。[57] 数学家米歇尔·查尔斯(Michel Chasles)推测,这些现已失传的命题可能涉及现代横截线理论和投影几何的内容。[58][p]

《表面位置》(The Surface Loci)(古希腊文:Τόποι πρὸς ἐπιφανείᾳ)几乎没有已知内容,除了基于作品标题的推测。[58] 后来的记载表明,它可能讨论了圆锥和圆柱体等主题。[57]
\subsection{遗产}
\begin{figure}[ht]
\centering
\includegraphics[width=6cm]{./figures/f4a2a1a88fb6ea9f.png}
\caption{“奥利弗·伯恩(Oliver Byrne)1847年彩色版《几何原本》的封面”} \label{fig_Euclid_7}
\end{figure}
欧几里得通常与阿基米德和佩尔加的阿波罗尼乌斯一起,被认为是古代最伟大的数学家之一。[11] 许多评论家认为他是数学历史上最具影响力的人物之一。[2] 《几何原本》所建立的几何体系长期主导了这一领域;然而,今天这一体系常被称为“欧几里得几何”,以便将其与19世纪初发现的其他非欧几里得几何区分开来。[61] 以欧几里得命名的事物包括欧洲航天局(ESA)的欧几里得号宇宙飞船,[62] 月球上的欧几里得陨石坑,[63] 以及小行星4354欧几里得。[64]

《几何原本》常被认为是仅次于《圣经》的西方世界历史上最频繁翻译、出版和研究的书籍。[61] 与亚里士多德的《形而上学》一样,《几何原本》也许是最成功的古希腊著作,并且在中世纪的阿拉伯和拉丁世界中是主要的数学教科书。[61]

《几何原本》的首个英文版由亨利·比林斯利(Henry Billingsley)和约翰·迪(John Dee)于1570年出版。[27] 数学家奥利弗·伯恩(Oliver Byrne)于1847年出版了一个著名版本,标题为《欧几里得几何原本的前六卷:使用彩色图表和符号代替字母,以便学习者更易理解》,其中包括旨在增强教学效果的彩色图表。[65] 大卫·希尔伯特(David Hilbert)编写了《几何原本》的现代公理化版本。[66] 爱德娜·圣文森特·米雷(Edna St. Vincent Millay)曾写道:“只有欧几里得曾直视赤裸的美。”[67]
\subsection{参考文献} 
\subsubsection{注释} \\
a. 在现代英语中,“Euclid”的发音为 /ˈjuːklɪd/。[5]\\  
b. 欧几里得的作品还包括《分割论》,该作品在后来的阿拉伯文献中以残篇的形式保存下来。[9] 他还创作了许多失传的作品。[9]\\  
c. 来自亚历山大帕帕斯的有关欧几里得的信息现在已失传,并通过普罗克鲁斯的《欧几里得几何原本第一卷注释》得以保存。[10]\\  
d. 普罗克鲁斯可能是依据(现已失传的)公元前4世纪的数学历史著作进行工作的,这些著作由泰奥弗拉斯图斯和罗德岛的欧德穆斯编写。普罗克鲁斯明确提到赫拉克利亚的阿米克拉斯、梅那赫穆斯和他的兄弟狄诺斯特拉图斯、马格尼西亚的塞奥迪乌斯、基济库斯的阿瑟那乌斯、科洛丰的赫莫提穆斯以及门德的菲利普斯,并表示欧几里得是在这些人之后“不过很久”来到的。\\  
e. 请参阅希思1981年,第354页,查阅普罗克鲁斯关于欧几里得生平的英文翻译。\\  
f. 后来的阿拉伯文献称他是出生于现代黎巴嫩的提尔的希腊人,尽管这些说法被认为是可疑和推测性的。[8][4] 请参阅希思1981年,第355页,查阅阿拉伯文献的英文翻译。他长期被认为出生于梅加拉,但到文艺复兴时期,学者们得出结论认为他与梅加拉的哲学家欧几里得混淆了,[16] 参见 §身份与历史性。\\  
g. 穆塞翁后来会包括著名的亚历山大图书馆,但它可能是在托勒密二世菲拉德尔弗斯(公元前285-246年)统治时期成立的。[24]\\
h.今天可用的《几何原本》版本也包括了“后欧几里得”数学内容,这些内容可能是后来的编辑所添加的,例如公元4世纪的数学家亚历山大城的提安(Theon)。\\  
i.“公理”一词代替“公设”的使用,源自普罗克鲁斯在他极具影响力的《几何原本》注释中所做的选择。普罗克鲁斯还将“公理”替换为“常识”,但保留了“公设”一词。\\
j.另见:欧几里得关系\\  
k.这些类别之间的区别并不立即显现;公设可能仅指代几何学, 而常识则具有更广泛的范围。\\  
l.数学家杰拉德·维尼马(Gerard Venema)指出,这一公理化体系并不完备:“欧几里得假设了比他在公设中所陈述的更多的内容。”\\
m.请参见希思(Heath 1908,第195-201页)关于公设1至4的详细概述\\  
n.自古以来,关于第五公设的学术研究数量庞大,通常来自试图证明该公设的数学家——这将使其与其他四个无法证明的公设有所不同。\\  
o.《几何原本》第五卷的内容很可能是从早期数学家那里获得的,可能是欧多克索斯(Eudoxus)提出的。\\  
p.欲了解更多关于“旁理”(Porisms)的信息,请参见琼斯(Jones 1986,第547-572页)。
\subsubsection{引用文献}  
\begin{enumerate}
\item Getty.  
\item 布鲁诺(Bruno 2003,第125页)。  
\item 西亚拉罗斯(Sialaros 2021),§“总结”。  
\item 西亚拉罗斯(Sialaros 2021),§“生平”。  
\item 牛津英语词典a(OEDa)。  
\item 牛津英语词典b(OEDb)。  
\item 希思(Heath 1981,第354页)。  
\item 阿斯珀(Asper 2010),§段落1。  
\item 西亚拉罗斯(Sialaros 2021),§“作品”。  
\item 希思(Heath 1911,第741页)。  
\item 鲍尔(Ball 1960,第52页)。  
\item 西亚拉罗斯(Sialaros 2020,第141页)。  
\item 古尔丁(Goulding 2010,第125页)。  
\item 斯莫林斯基(Smorynski 2008,第2页)。  
\item 博耶(Boyer 1991,第100页)。  
\item 古尔丁(Goulding 2010,第118页)。  
\item 希思(Heath 1981,第355页)。  
\item 古尔丁(Goulding 2010,第126页)。  
\item 希思(Heath 1908,第2页)。  
\item 西亚拉罗斯(Sialaros 2020,第147–148页)。  
\item 西亚拉罗斯(Sialaros 2020,第142页)。  
\item 布鲁诺(Bruno 2003,第126页)。  
\item 鲍尔(Ball 1960,第51页)。  
\item 崔西(Tracy 2000,第343–344页)。  
\item 西亚拉罗斯(Sialaros 2021),§“生平”和注释5。  
\item 琼斯(Jones 2005)。
\item 古尔丁(Goulding 2010,第120页)。
\item 泰斯巴克与范德瓦登(Taisbak & Van der Waerden 2021),§“生平”。
\item 博耶(Boyer 1991,第101页)。
\item 博耶(Boyer 1991,第96页)。
\item 西亚拉罗斯(Sialaros 2018,第90页)。
\item 希思(Heath 1981,第357页)。
\item 鲍尔(Ball 1960,第52–53页)。
\item 福勒(Fowler 1999,第210–211页)。
\item 阿斯珀(Asper 2010),§段落2。
\item 阿斯珀(Asper 2010),§段落6。
\item 泰斯巴克与范德瓦登(Taisbak & Van der Waerden 2021),§“《几何原本》的来源与内容”。
\item 库莫(Cuomo 2005,第131页)。
\item 阿特曼(Artmann 2012,第3页)。
\item 阿斯珀(Asper 2010),§段落4。
\item 西亚拉罗斯(Sialaros 2021),§“《几何原本》”。
\item 扬克(Jahnke 2010,第18页)。
\item 希思(Heath 1908,第154–155页)。
\item 阿特曼(Artmann 2012,第3–4页)。
\item 沃尔夫(Wolfe 1945,第4页)。
\item 皮科弗(Pickover 2009,第56页)。
\item 维尼马(Venema 2006,第10页)。
\item 阿特曼(Artmann 2012,第4页)。
\item 希思(Heath 1908,第202页)。
\item 卡茨与米哈洛维茨(Katz & Michalowicz 2020,第59页)。
\item 阿特曼(Artmann 2012,第5页)。  
\item 阿特曼(Artmann 2012,第5–6页)。  
\item 阿特曼(Artmann 2012,第6页)。  
\item 希思(Heath 1908b,第191页)。  
\item 阿特曼(Artmann 2012,第7页)。  
\item 阿特曼(Artmann 2012,第9页)。  
\item 西亚拉罗斯(Sialaros 2021),§“其他作品”。  
\item 泰斯巴克与范德瓦登(Taisbak & Van der Waerden 2021),§“其他著作”。  
\item 琼斯(Jones 1986,第399–400页)。  
\item 阿切尔比(Acerbi 2008,第511页)。  
\item 泰斯巴克与范德瓦登(Taisbak & Van der Waerden 2021),§“遗产”。  
\item “NASA为欧空局的欧几里得号太空探测器提供探测器”。喷气推进实验室,2017年5月9日。  
\item “行星命名法地名册 | 欧几里得(Euclides)”。美国地质调查局(usgs.gov)。国际天文学联合会。检索日期:2017年9月3日。  
\item “小行星4354欧几里得(Euclides)(2142 P-L)”。小行星中心。检索日期:2018年5月27日。  
\item 霍斯与科尔帕斯(Hawes & Kolpas 2015)。  
\item 哈尔与彼得斯(Hähl & Peters 2022),§段落1。  
\item 米雷(Millay, Edna St. Vincent)。欧几里得独自一人看见裸露的美丽。
\end{enumerate}