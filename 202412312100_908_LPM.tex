% 加布里埃尔·李普曼(综述)
% license CCBYSA3
% type Wiki

本文根据 CC-BY-SA 协议转载翻译自维基百科\href{https://en.wikipedia.org/wiki/Gabriel_Lippmann}{相关文章}。

乔纳斯·费迪南德·加布里埃尔·李普曼(1845年8月16日-1921年7月12日)是一位出生于卢森堡的法国物理学家和发明家,因其“基于干涉现象的色彩摄影再现方法”而于1908年获得诺贝尔物理学奖。[2]
\subsection{早年生活和教育}
\begin{figure}[ht]
\centering
\includegraphics[width=6cm]{./figures/2135552cea5927ab.png}
\caption{1908年的李普曼} \label{fig_LPM_1}
\end{figure}
加布里埃尔·李普曼于1845年8月16日出生在卢森堡的博讷沃伊(卢森堡语:Bouneweg)。当时,博讷沃伊是霍勒里奇(卢森堡语:Hollerech)市镇的一部分,这个地方常被认为是他的出生地。(如今,博讷沃伊和霍勒里奇都是卢森堡市的区。)他的父亲伊萨耶(Isaïe)是一位出生在梅斯附近恩内里的法国犹太人,经营着位于博讷沃伊的旧修道院内的家族手套制造生意。1848年,李普曼一家搬到了巴黎,在那里李普曼最初由母亲米里亚姆·罗斯(Lévy)辅导,之后进入拿破仑中学(现为亨利四世中学)。据说他是一个注意力不集中但富有思考的学生,对数学特别感兴趣。1868年,他进入巴黎高等师范学院学习,但未通过能够使他进入教师职业的聚集考试,而是选择专攻物理学。1872年,法国政府派他前往海德堡大学,在古斯塔夫·基尔霍夫的鼓励下,他专攻电学,并于1874年获得“优等”博士学位。李普曼于1875年回到巴黎,继续学习,直到1878年成为索邦大学的物理学教授。在索邦大学,他教授声学和光学课程。