% 隐函数定理的不动点证明

\pentry{多元隐函数的存在定理\upref{Mulmp}, 巴拿赫不动点定理\upref{ConMap}}

\subsection{隐函数定理的紧凑表述}
我们可以将隐函数定理用比较紧凑的形式表达出来. 在如下的版本中, 已知的"隐式关系"$F(x,y)$是一个映射, 其中$x$是$n$维的, $y$是$m$维的, 映射的取值也是$m$维的. 这样一来, 对于给定的$x$, 求解隐式方程$F(x,y)=0$就相当于从$m$个方程求解$m$个未知量($y$的$m$个分量). 为了这个方程能够求解, 自然期望方程之间需要是"独立"的.
\begin{theorem}{隐函数定理}
设$(x_0,y_0)\in\mathbb{R}^n\times\mathbb{R}^m$是给定的点. 设在某个开集$B(x_0,R)\times B(y_0,R)$上定义了映射$F:U\to\mathbb{R}^m$, 满足如下条件:
\begin{enumerate}
\item $F(x_0,y_0)=0$.
\item $F(x,y)$对$y$是连续可微的, 而且雅可比矩阵
$$
\frac{\partial F}{\partial y}(x_0,y_0)
$$
是可逆的.
\end{enumerate}
则存在$r<R$以及映射$f:B(x_0,r)\to B(y_0,R)$, 使得$f(x_0)=y_0$, 而且$F(x,f(x))=0$. 换句话说, 在$x_0$的某个邻域内, 给定了$x$就可以唯一求解$y$, 从而$x\to y$确定了一个函数关系.
\end{theorem}

\subsection{证明}
除了词条 多元隐函数的存在定理\upref{Mulmp} 中给出的归纳证明之外, 还可以用不动点定理给出一个简洁的, 而且是构造性的证明.

如果要求解未知量$y$的方程$F(x,y)=0$, 就需要对它进行适当的变换, 变成不动点方程的形状. 最简单的变换方式, 就是乘上一个$m\times m$的常值可逆方阵$A$, 将方程重写为
$$
y=y+A\cdot F(x,y).
$$
那么该选择怎样的方阵$A$, 才能使得右边(视为$y$的映射)有不动点呢?

将右边记为$G_x(y)$, 其中$x$接近$x_0$, 是已知的. 则$G_x(y)$对$y$的微分是
\begin{equation}\label{IFTFix_eq1}
\frac{\partial G_x(y)}{\partial y}=\mathrm{Id}_m+A\cdot\frac{\partial F}{\partial y}(x,y).
\end{equation}
根据有限增量定理, 我们应该希望$\frac{\partial G_x(y)}{\partial y}$尽可能小, 这样就能保证$y\to G_x(y)$是压缩映射. 因此, 最合适的选择当然是
$$
A=-\left(\frac{\partial F}{\partial y}(x_0,y_0)\right)^{-1}.
$$
根据定理的条件, 雅可比矩阵$\frac{\partial F(x_0,y_0)}{\partial y}$可逆, 所以$A$是良好定义的; 进一步, 根据定理的条件, 如果$(x,y)$足够地接近$(x_0,y_0)$, 那么\autoref{IFTFix_eq1} 的矩阵范数就会小于1/2. 因此, 存在$r<R$, 使得只要$|x-x_0|\leq r$, $|y_1-y_0|\leq r$, $|y_2-y_0|\leq r$, 就有
$$
|F(x,y_1)-F(x,y_2)|\leq\frac{1}{2}|y_1-y_2|.
$$  
因此对于$x\in \bar B(x_0,r)$, 映射$y\to G_x(y)$将闭球$\bar B(y_0,r)$映射到它自己, 而且是压缩映射. 根据巴拿赫不动点定理, 它有唯一一个不动点, 而这正是方程$F(x,y)=0$的解. 给定了$x$后, 解$y$就唯一确定了, 它当然给出了$x$的函数. 证毕.

由于巴拿赫不动点定理的证明是由迭代序列给出的, 这个证明实际上给出了隐函数$y=f(x)$的近似计算方法: 对于接近$x_0$的$x$, 函数值$y=f(x)$由迭代序列
$$
y_{k+1}=y_k-\left(\frac{\partial F}{\partial y}(x_0,y_0)\right)^{-1}\cdot F(x,y_k)
$$
所逼近.