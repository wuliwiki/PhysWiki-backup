% 正则系综(综述)
% license CCBYSA3
% type Wiki

本文根据 CC-BY-SA 协议转载翻译自维基百科\href{https://en.wikipedia.org/wiki/Canonical_ensemble}{相关文章}。

在统计力学中,正则系综是表示与在固定温度下与热库处于热平衡的机械系统可能状态的统计系综。\(^\text{[1]}\)系统可以与热库交换能量,因此系统的状态将在总能量上有所不同。

正则系综的主要热力学变量,决定状态的概率分布,是绝对温度(符号:\(T\))。该系综通常还依赖于机械变量,如系统中的粒子数(符号:\(N\))和系统的体积(符号:\(V\)),这些变量都会影响系统内部状态的性质。具有这三个参数的系综,假设这些参数对于被认为是正则的系综是恒定的,有时被称为\(NVT\)系综。

正则系综为每一个不同的微观状态分配一个概率\(P\),其公式为以下指数形式:
\[
P = e^{(F - E) / (kT)},~
\]
其中\(E\)是微观状态的总能量,\(k\)是玻尔兹曼常数。

数值\(F\)是自由能(具体来说是赫尔姆霍兹自由能),并假设对于特定的系综,\(F \)是常数,才认为该系综是正则的。然而,如果选择不同的\(N\)、\(V\)、\( T\),概率和\(F\)将发生变化。自由能\(F\)承担两个角色:首先,它为概率分布提供了一个归一化因子(所有微观状态的概率总和必须为1);其次,许多重要的系综平均值可以直接通过函数\(F(N, V, T)\)计算。

对于同一概念,另一种等效的表述将概率写为
\[
P = \frac{1}{Z} e^{-E / (kT)},~
\]
其中使用了正则系综的分配函数
\[
Z = e^{-F / (kT)},~
\]
而不是自由能。下面的方程(以自由能为单位)可以通过简单的数学操作转化为正则系综分配函数的形式。

从历史上看,正则系综最早由玻尔兹曼于1884年在一篇相对不为人知的论文中描述(他称之为全像体)。\(^\text{[2]}\)该理论后来由吉布斯在1902年重新表述并进行了广泛研究。\(^\text{[1]}\)
\subsection{正则系综的适用性}  
正则系综是描述与热库处于热平衡的系统可能状态的系综(这一事实的推导可以在吉布斯的著作中找到\(^\text{[1]}\))。

正则系综适用于任何规模的系统;尽管需要假设热库非常大(即,取宏观极限),但系统本身可以是小型的或大型的。

系统必须是机械隔离的,以确保它不与热库以外的任何外部物体交换能量。\(^\text{[1]}\)一般来说,正则系综适用于直接与热库接触的系统,因为正是这种接触确保了平衡。在实际情况中,通常通过以下两种方式之一来证明正则系综的适用性:1)假设接触是机械弱的,或者 2)将热库连接的适当部分纳入所分析的系统,从而在系统内建模接触对系统的机械影响。

当总能量固定但系统的内部状态未知时,适当的描述不是正则系综,而是微正则系综。对于粒子数可变的系统(由于与粒子库的接触),正确的描述是巨正则系综。在涉及相互作用粒子系统的统计物理教材中,通常假设三种系综在热力学上是等价的:宏观量围绕其平均值的波动变得很小,并且随着粒子数趋向于无穷大,这些波动趋于消失。在这种极限下,称为热力学极限,平均约束实际上变为硬约束。系综等价性的假设可以追溯到吉布斯,并且已经在一些具有短程相互作用且受到少量宏观约束的物理系统模型中得到了验证。尽管许多教材仍然传达着系综等价性适用于所有物理系统的信息,但在过去几十年中,已经发现了一些物理系统的例子,在这些系统中,系综等价性被打破了。\(^\text{[3][4][5][6][7][8]}\)