% 托马斯进动
% 狭义相对论|相对论|洛伦兹变换|洛伦兹群|进动|托马斯进动|复合

\pentry{洛伦兹群\upref{qed1}}

当两个洛伦兹平动复合的时候,其结果通常不是一个洛伦兹平动,而是一个平动加上一个转动.

从物理直觉上来说,考虑三个参考系$K_1$,$K_2$和$K_3$.每个参考系都架设三把标尺,彼此垂直,各标尺分别测量相应参考系里的$x$,$y$和$z$坐标.令$K_2$以速度$\bvec{v}$相对$K_1$运动,同时保证在$K_1$和$K_2$看来,双方对应的标尺都相互平行;再令$K_3$以速度$\bvec{w}$相对$K_2$运动,同时保证在$K_2$和$K_3$看来,双方对应的标尺都相互平行,那么这个时候$K_3$的标尺还和$K_1$的标尺对应平行吗?

答案是否定的.这是因为尺缩效应只发生在沿着参考系相对速度的方向,而垂直于相对速度的方向不会发生尺缩效应.如图

\begin{figure}[ht]
\centering
\includegraphics[width=8cm]{./figures/TmsPrs_1.pdf}
\caption{托马斯进动的示意图.左图表示$K_3$眼中自己的一根标尺,它是物理存在;垂直的线段表示这个标尺,下面的长箭头表示$K_3$相对$K_1$的运动方向,而上面的短箭头表示垂直于该运动的方向.在$K_1$看来,沿着运动方向发生了尺缩效应,下面的箭头变短,而垂直运动方向的箭头没有变短,结果就是标尺的长度缩短,并且其方向转动了一个角度.} \label{TmsPrs_fig1}
\end{figure}



