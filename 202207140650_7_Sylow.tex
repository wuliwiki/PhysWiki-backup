% Sylow定理
% keys 西罗定理|西罗子群|Sylow子群|群论|有限群

\pentry{群作用\upref{Group3}}

\addTODO{加入目录}

拉格朗日定理(\autoref{coset_the2}~\upref{coset})揭示了子群阶数的特点.可惜的是,其逆命题“如果$n\mid \abs{G}$则$G$总有阶数为$n$的子群”是不成立的.比如说,交错群$A_4$就没有$6$阶的子群——你可以动手验证这一点.

但是,挪威数学家Peter Ludwig Sylow于1872年发表的文章\footnote{L. Sylow, Th´eor`emes sur les groupes de substitutions, Mathematische Annalen 5 (1872), 584–594. 该文见 https://eudml.org/doc/156588. Robert Wilson 对此文的英文翻译见 http://www.maths.
qmul.ac.uk/~raw/pubs_files/Sylow.pdf.}指出,在$n$是素数或者素数的幂时,拉格朗日定理逆命题是成立的.同时他还发现了所谓的Sylow子群全都是彼此共轭的.

举个例子:考虑阶数为$300$的群$G$,对$300$进行素因子分解得$300=2^2\cdot 3^2\cdot 5^2$,那么阶数为$2^2$的子群总存在,且它们彼此共轭.不过,$2$阶子群虽存在,却不能总保证所有$2$阶子群彼此共轭.


\begin{definition}{Sylow子群}
给定群$G$和它的子群$H$.如果$\abs{H}=p^k$,其中$p$是素数,且$p^{k+1}\not\mid\abs{G}$,那么称$H$是$G$的一个$p$\textbf{-Sylow 子群},或\textbf{Sylow-}$p$\textbf{子群},或直接统称为\textbf{Sylow子群}.
\end{definition}



\begin{exercise}{}
考虑循环群$C_{12}$,求它的所有Sylow子群.
\end{exercise}

\begin{example}{}
考虑置换群$S_4$.则
\begin{equation}
C=\{1, \pmat{1&2}\pmat{3&4}, \pmat{1&3}\pmat{2&4}, \pmat{1&4}\pmat{2&3}, \pmat{1&2}, \pmat{3&4}, \pmat{1&3&2&4}, \pmat{1&4&2&3}\}
\end{equation}
是它的一个Sylow-$2$子群.

$C$是$\pmat{1&2}\pmat{3&4}$的中心,因此我们可以类似构造出$S_4$的剩下两个Sylow-$2$子群,分别是$\pmat{1&3}\pmat{2&4}$和$\pmat{1&4}\pmat{2&3}$的中心.这三个Sylow-$2$子群的交集是$V_4=\{1, \pmat{1&2}\pmat{3&4}, \pmat{1&3}\pmat{2&4}, \pmat{1&4}\pmat{2&3}\}$.

如果把一个正方形的四个顶点顺时针依次编号为$1, 3, 2, 4$,则不难看出$C$实际上就是正方形的对称群$D_4$.类似地,$S_4$的每个Sylow-$2$子群都同构于$D_4$.

\end{example}

Sylow定理通常被拆分成三个部分来表述.

\begin{theorem}{Sylow第一定理}
取\textbf{有限群}$G$.如果素数$p\mid\abs{G}$,那么$G$一定有一个Sylow-$p$子群$H_p$.
\end{theorem}

\textbf{证明}:



\textbf{证毕}.





\begin{theorem}{Sylow第二定理}
取\textbf{有限群}$G$,它的所有Sylow-$p$子群彼此共轭.
\end{theorem}

\textbf{证明}:



\textbf{证毕}.




\begin{theorem}{Sylow第三定理}
取\textbf{有限群}$G$,设$n_p$是它的所有Sylow-$p$子群的数目.令$\abs{G}=p^km$,其中$p\not\mid m$.

那么$n_p\equiv 1\opn{mod} p$,或者说$p\mid n_p-1$;且$n_p\mid m$.
\end{theorem}

\textbf{证明}:



\textbf{证毕}.




















