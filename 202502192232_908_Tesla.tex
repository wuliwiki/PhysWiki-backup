% 尼古拉·特斯拉(综述)
% license CCBYSA3
% type Wiki

本文根据 CC-BY-SA 协议转载翻译自维基百科\href{https://en.wikipedia.org/wiki/Nikola_Tesla}{相关文章}。


\begin{figure}[ht]
\centering
\includegraphics[width=6cm]{./figures/4da9af8e4f34bc2b.png}
\caption{} \label{fig_Tesla_1}
\end{figure}
尼古拉·特斯拉(/ˈnɪkələ ˈtɛslə/;塞尔维亚西里尔字母:Никола Тесла,[nǐkola têsla];1856年7月10日 – 1943年1月7日)是塞尔维亚裔美国工程师、未来学家和发明家。他以对现代交流电(AC)电力供应系统设计的贡献而闻名。

特斯拉出生并成长于奥斯曼帝国,在1870年代,他首先学习了工程学和物理学,但并未获得学位。随后,他在1880年代初期,在电话通信和大陆爱迪生公司(Continental Edison)新兴的电力行业中积累了实践经验。1884年,他移民到美国,并成为美国公民。他在纽约市的爱迪生机械厂工作了短暂时间后,便开始独立创业。在合作伙伴的帮助下,为了融资和推广自己的创意,特斯拉在纽约设立了实验室和公司,开发各种电气和机械设备。他的交流电感应电动机和相关的多相交流电专利,于1888年获得了西屋电气公司的许可,这使他赚得了可观的财富,并成为该公司最终推广的多相电系统的基石。

为了开发可以申请专利并商业化的发明,特斯拉进行了多种实验,包括机械振荡器/发电机、电气放电管和早期的X射线成像。他还制造了一艘无线控制的船,是最早展出的一批之一。特斯拉作为发明家广为人知,并在他的实验室向名人和富有的赞助人展示自己的成就,他的公共讲座也因其表演性质而备受关注。整个1890年代,特斯拉在纽约和科罗拉多斯普林斯进行高电压、高频率的电力实验,追求无线照明和全球无线电力传输的构想。1893年,他宣布了使用自己设备进行无线通信的可能性。特斯拉试图将这些想法付诸实践,通过未完成的沃登克利夫塔项目,这是一个跨洲的无线通信和电力传输塔,但在资金耗尽之前他未能完成该项目。

在沃登克利夫塔之后,特斯拉在1910年代和1920年代进行了一系列发明实验,取得了不同程度的成功。由于花费了大部分的钱,特斯拉在一系列纽约酒店中居住,并留下了未付的账单。他于1943年1月在纽约市去世。特斯拉的工作在他去世后逐渐被遗忘,直到1960年,国际计量大会将国际单位制(SI)中磁通密度的单位命名为“特斯拉”,以此向他致敬。自1990年代以来,特斯拉的公众兴趣重新兴起。
\subsection{早年时期}
\begin{figure}[ht]
\centering
\includegraphics[width=6cm]{./figures/4e958945d0c8af76.png}
\caption{特斯拉重建的出生地(教区大厅)和他父亲曾服务的教堂,位于克罗地亚的斯米扬。该地点已被改建为博物馆,以纪念他。} \label{fig_Tesla_2}
\end{figure}
尼古拉·特斯拉于1856年7月10日出生在奥斯曼帝国(今克罗地亚)军事边境的斯米扬村,来自一个塞尔维亚裔家庭。他的父亲米卢廷·特斯拉(1819–1879)是东正教的牧师。他父亲的兄弟约瑟夫是军事学院的讲师,撰写了几本数学教材。

特斯拉的母亲,乔治娜“杜卡”曼迪奇(1822–1892),她的父亲也是一位东正教牧师,具有制作家用工具和机械设备的天赋,并能背诵塞尔维亚史诗诗篇。杜卡从未接受过正式教育。特斯拉将自己的过目不忘的记忆力和创造力归功于母亲的遗传和影响。

特斯拉是五个孩子中的第四个。他有三个姐妹,分别是米尔卡、安杰丽娜和马里察,还有一个名叫丹尼的哥哥,他在特斯拉六七岁时因马术事故去世。1861年,特斯拉在斯米扬的初级学校上学,学习德语、算术和宗教。1862年,特斯拉一家搬到附近的戈斯皮奇镇,特斯拉的父亲在那儿担任教区牧师。尼古拉完成了小学学业后,继续上了中学。1870年,特斯拉搬到卡尔洛瓦茨[19],在那里他进入高等实科中学(Higher Real Gymnasium)上学,课堂上讲授的是德语,这是奥匈帝国军事边境地区学校的常规语言。[20][21] 后来,在申请专利时,在获得美国国籍之前,特斯拉会将自己标识为“来自奥匈帝国边界地区的斯米扬,利卡”。[22]
\begin{figure}[ht]
\centering
\includegraphics[width=6cm]{./figures/f94da1546174f1e0.png}
\caption{特斯拉的父亲米卢廷是斯米扬村的东正教牧师。} \label{fig_Tesla_3}
\end{figure}
特斯拉后来写道,他对物理教授展示的电学实验产生了兴趣。[a] 特斯拉指出,这些“神秘现象”的展示让他想“了解更多关于这种奇妙力量的知识”。[25] 特斯拉能够在脑海中进行积分计算,这使得他的老师们认为他在作弊。[26] 他在三年内完成了四年的学业,并于1873年毕业。[27]

毕业后,特斯拉返回了斯米扬,但很快感染了霍乱,卧床不起九个月,并多次濒临死亡。在绝望时刻,特斯拉的父亲(原本希望他进入神职)[28]承诺,如果他从病中恢复,将送他去最好的工程学院。[29] 特斯拉后来表示,在恢复期间,他读了马克·吐温的早期作品。[30][31]

次年,特斯拉通过逃避征召,避开了奥匈帝国军队的征兵,他从斯米扬逃到利卡东南部的托米纳吉,靠近格拉查茨。在那里,他穿着猎人的衣服探索山脉。特斯拉表示,这种与大自然的接触让他在身体和心理上都变得更强壮。1875年,特斯拉凭借军事边境奖学金入读格拉茨的帝国皇家技术学院。在那里,特斯拉通过了九门考试(几乎是所需考试数量的两倍[33]),并收到了技术系院长写给他父亲的表扬信,信中写道:“您的儿子是一级明星。”[33] 在格拉茨,特斯拉提到他对雅各布·佩什尔教授关于电学的详细讲座产生了浓厚的兴趣,并描述了他如何提出改进教授展示的电动机设计的建议。[30][需要更好的来源][34] 但到第三年时,他在学校成绩不佳,最终未能毕业,并于1878年12月离开了格拉茨。某位传记作家认为,特斯拉并没有认真学习,可能因为赌博和花心被开除。[35]
\begin{figure}[ht]
\centering
\includegraphics[width=6cm]{./figures/6c6ed982c821e12f.png}
\caption{大约在1879年,23岁的特斯拉} \label{fig_Tesla_4}
\end{figure}
特斯拉离开学校后,他的家人再也没有收到他的消息。[36] 他同学中有一个谣言,说他在附近的穆尔河里溺水身亡,但到了1月,其中一位同学在马里博尔镇遇到了特斯拉,并将这一遭遇报告给了特斯拉的家人。[37] 结果发现,特斯拉当时在那里做草图员,每月赚取60弗罗林。[35] 1879年3月,米卢廷终于找到了他的儿子,并试图说服他回家继续在布拉格接受教育。[37] 特斯拉在同月晚些时候回到了戈斯皮奇,但因没有居住许可被驱逐出境。[37] 特斯拉的父亲在次月去世,享年60岁,死于一种不明的疾病。[37] 在那一年剩余的时间里,特斯拉在戈斯皮奇的老学校教授了一个大班的学生。

1880年1月,特斯拉的两位叔叔凑够了足够的钱,帮助他离开戈斯皮奇前往布拉格,那里他计划继续学习。然而,他到达时已经错过了查理-费迪南大学的报名时间,而且他从未学习过希腊语,这是一门必修科目;同时,他也不懂捷克语,另一门必修科目。然而,特斯拉确实以听课生的身份参加了大学的哲学讲座,但他并未为这些课程获得成绩。[38][39]
\subsubsection{在布达佩斯电话交换公司工作}
特斯拉于1881年移居匈牙利布达佩斯,在蒂瓦达尔·普什卡什的领导下,加入了一家电报公司——布达佩斯电话交换公司。到达后,特斯拉发现该公司当时正在建设中,并未投入使用,因此他改为在中央电报局担任草图员。几个月后,布达佩斯电话交换公司开始投入使用,特斯拉被分配为首席电工。在任职期间,特斯拉对中央站的设备进行了许多改进,并声称自己完善了一种电话中继器或放大器,但这一发明从未获得专利,也未公开描述。[30]\subsection{在爱迪生公司工作}  
1882年,蒂瓦达尔·普什卡什为特斯拉在巴黎的大陆爱迪生公司(Continental Edison Company)找到了另一份工作。[40] 特斯拉开始从事当时全新的行业——在大规模电力公用事业中,安装室内白炽灯,覆盖整个城市。公司有多个子公司,特斯拉在位于巴黎伊夫里-苏尔-塞纳郊区的爱迪生电气公司(Société Electrique Edison)工作,负责安装照明系统。在那里,他获得了大量电气工程的实践经验。管理层注意到他在工程学和物理学方面的深厚知识,很快就让他设计和建造改进版的发电机和电动机。[41] 他们还派他去法国和德国其他爱迪生公用事业公司解决工程问题。
\subsubsection{移居美国}
\begin{figure}[ht]
\centering
\includegraphics[width=6cm]{./figures/e2a69211c5447039.png}
\caption{位于纽约哥尔克街的爱迪生机器厂。特斯拉发现,从充满国际化气息的欧洲转到这家位于曼哈顿下东区贫民区的工厂,令他感到“痛苦的惊讶”。[42]} \label{fig_Tesla_5}
\end{figure}
1884年,负责巴黎安装项目的爱迪生经理查尔斯·巴切洛尔被召回美国,接管位于纽约市的爱迪生机器厂,并要求将特斯拉也带到美国。[43] 1884年6月,特斯拉移民美国[44],并几乎立刻开始在位于曼哈顿下东区的机器厂工作,这是一家拥挤的工厂,拥有数百名机械师、工人、管理人员和20名“现场工程师”,他们在努力建造该市的大型电力公用事业。[45] 和在巴黎一样,特斯拉主要负责排查安装问题并改进发电机。[46] 历史学家W·伯纳德·卡尔森指出,特斯拉可能只见过公司创始人托马斯·爱迪生几次。[45] 其中一次在特斯拉的自传中提到,特斯拉在整夜修理海轮“俄勒冈号”上的损坏发电机后,遇到了巴切洛尔和爱迪生,爱迪生打趣道他们的“巴黎人”整夜未归。特斯拉告诉他们自己整夜都在修理“俄勒冈号”,爱迪生对巴切洛尔评论道:“这是个了不起的人。”[42] 特斯拉负责的一个项目是开发基于弧光灯的街道照明系统。[47][48] 弧光照明是最流行的街道照明方式,但它需要高电压,与爱迪生的低电压白炽灯系统不兼容,这导致公司在一些城市失去了合同。特斯拉的设计未能投入生产,可能是由于白炽灯街道照明的技术改进,或者是因为爱迪生与一家弧光照明公司达成了安装协议。[49]

特斯拉在爱迪生机器厂工作了六个月后辞职。[45] 促使他离开的事件并不明确。可能是因为他未能获得某个奖金,奖金可能是为重新设计发电机或为被搁置的弧光照明系统设计所支付的。[47] 特斯拉此前曾因未能获得他认为自己应得的奖金而与爱迪生公司发生过冲突。[50][51] 在自传中,特斯拉提到,爱迪生机器厂的经理曾提供5万美元奖金,让他设计“24种不同类型的标准机器”,但“结果证明这是一个恶作剧”。[52][需要更好的来源] 后来的版本中,这个故事变成了托马斯·爱迪生本人提出这个奖金,并且在后来反悔,开玩笑说:“特斯拉,你不懂我们的美国幽默”。[53][54] 无论是哪种版本,奖金的数额都被认为很奇怪,因为机器厂的经理巴切洛尔对工资非常吝啬,且公司手头并没有这么多现金(今天相当于1,695,556美元)。[56][57] 特斯拉的日记中只包含了一条关于他离开工作时的评论,那是他在1884年12月7日至1885年1月4日的两页上写下的一句话:“告别爱迪生机器厂”。[48][58]
\subsection{特斯拉电光与制造公司}  
离开爱迪生公司后不久,特斯拉开始致力于为一个弧光照明系统申请专利,[59] 可能就是他在爱迪生公司开发的那个系统。[45] 1885年3月,他与专利律师莱穆埃尔·W·塞雷尔(Lemuel W. Serrell)会面,塞雷尔是爱迪生公司使用的同一律师,以寻求帮助提交专利申请。[59] 塞雷尔将特斯拉介绍给了两位商人——罗伯特·莱恩和本杰明·韦尔,他们同意为特斯拉名下的弧光照明制造和公用事业公司提供资金,成立了特斯拉电光与制造公司(Tesla Electric Light and Manufacturing Company)。[60] 特斯拉在接下来的时间里,致力于获得专利,其中包括一种改进的直流发电机,这是特斯拉在美国获得的首批专利,并在新泽西州的拉威安装和建造了该系统。[61] 特斯拉的新系统引起了技术媒体的关注,评论称其具有先进的特点。

投资者对特斯拉的新型交流电机和电力传输设备的想法兴趣不大。1886年,当公用事业公司运营起来后,他们决定制造业务竞争过于激烈,选择仅仅经营电力公用事业。[62] 他们成立了一个新的公用事业公司,抛弃了特斯拉的公司,使得这位发明家陷入贫困。[62] 特斯拉甚至失去了他所创造的专利的控制权,因为他将这些专利转让给了公司,换取了股票。[62] 他不得不从事各种电力修理工作,并且作为一名挖沟工,每天挣2美元。特斯拉在晚年回忆起1886年那段日子时,称之为艰难的时光,他写道:“我在各个科学、机械和文学领域的高等教育,仿佛对我来说是一种讽刺。”[62][c]
\subsection{交流电与感应电动机}
\begin{figure}[ht]
\centering
\includegraphics[width=6cm]{./figures/6cdded301a8bf03d.png}
\caption{来自美国专利381,968的图示,展示了特斯拉交流电感应电动机的原理} \label{fig_Tesla_6}
\end{figure}
1886年底,特斯拉遇到了西联电报公司的主管阿尔弗雷德·S·布朗和纽约律师查尔斯·弗莱彻·佩克。[64] 这两人有丰富的经验,在为发明和专利提供资金支持并推广方面颇有建树。[65] 基于特斯拉的新电气设备构想,包括热磁电动机的想法,[66] 他们同意为这位发明家提供资金支持,并处理他的专利事务。1887年4月,他们共同成立了特斯拉电气公司,并达成协议:专利产生的利润将按三分之一分配给特斯拉,三分之一分配给佩克和布朗,三分之一用于资金开发。[65] 他们在曼哈顿自由街89号为特斯拉建立了一个实验室,特斯拉在那里改进并开发新型电动机、发电机及其他设备。

1887年,特斯拉开发了一种使用交流电(AC)运行的感应电动机,这种电力系统在欧洲和美国迅速扩展,因为它在长距离高电压传输方面具有明显优势。该电动机使用多相电流,产生旋转磁场来驱动电动机(特斯拉声称这一原理是在1882年构思的)。[67][68][69] 这种创新的电动机于1888年5月获得专利,采用了简单的自启动设计,不需要换向器,从而避免了火花和频繁更换机械刷子的高维护成本。[70][71]

除了获得电动机专利,佩克和布朗还安排了让电动机广为宣传的工作,首先进行独立测试以验证其功能性改进,随后向技术出版物发布新闻稿,以便在专利发布时同时刊登相关文章。[72] 物理学家威廉·阿诺德·安东尼(测试电动机的人)和《电气世界》杂志主编托马斯·科默福德·马丁安排了特斯拉在1888年5月16日于美国电气工程师学会展示他的交流电动机。[72][73] 西屋电气制造公司的一些工程师向乔治·西屋报告,特斯拉的交流电动机及其相关电力系统是可行的——这是西屋在推广交流电系统时所需要的。西屋考虑过为意大利物理学家伽利略·费拉里斯在1885年开发的、基于旋转磁场的无换向器感应电动机申请专利,这一成果已在1888年3月的论文中展示,但他最终决定,特斯拉的专利可能会主导市场。[74][75]
\begin{figure}[ht]
\centering
\includegraphics[width=6cm]{./figures/1d473daee5879fca.png}
\caption{特斯拉的交流电发电机(AC电动发电机)在1888年美国专利390,721中的描述} \label{fig_Tesla_7}
\end{figure}
1888年7月,布朗和佩克与乔治·威斯汀豪斯(George Westinghouse)就特斯拉的多相感应电动机和变压器设计达成了许可协议,金额为6万美元现金和股票,并且每台电动机生产的每千瓦交流功率将支付2.50美元的版税。威斯汀豪斯还以每月2,000美元(按今天的通货膨胀计算相当于67,800美元)的高薪聘请特斯拉作为西屋电气制造公司匹兹堡实验室的顾问,为期一年。

在这一年里,特斯拉在匹兹堡工作,帮助创建一个交流电系统,用于为该市的电车供电。他认为这段时间很沮丧,因为与西屋的其他工程师在如何最佳实施交流电供电系统的问题上发生了冲突。他们最终采纳了特斯拉提出的60赫兹交流电系统(与特斯拉电动机的工作频率匹配),但他们很快发现,这个系统无法用于电车,因为特斯拉的感应电动机只能以恒定的速度运转。最后,他们改用了直流牵引电动机。
\subsubsection{市场动荡}  
特斯拉展示他的感应电动机以及威斯汀豪斯随后的专利许可,都是发生在1888年,这一时期恰逢电力公司之间的极端竞争。[80][81] 三大公司——威斯汀豪斯、爱迪生和汤姆森-休斯顿电力公司——都在资本密集型行业中争夺市场份额,同时通过相互压低价格来削弱对方的财务状况。甚至还爆发了“电流战争”的宣传战,爱迪生电气宣称他们的直流电系统比威斯汀豪斯的交流电系统更好、更安全,而汤姆森-休斯顿有时也支持爱迪生。[82][83] 在这样的市场竞争中,威斯汀豪斯没有足够的现金或工程资源来立即开发特斯拉的电动机和相关的多相系统。[84]

两年后,西屋电气因财务问题陷入困境。1890年,伦敦贝林银行(Barings Bank)濒临破产,引发了1890年的金融恐慌,导致投资者纷纷收回对西屋电气的贷款。[85] 资金突然短缺迫使公司不得不重新融资。新的贷款方要求西屋削减他们认为过度的支出,包括收购其他公司、研究投资以及特斯拉合同中每台电动机的专利使用费。[86][87] 此时,特斯拉的感应电动机仍未取得成功,开发进度受阻。[84][85] 尽管这种电动机的实际应用案例极少,运行它所需的多相电力系统更是罕见,西屋仍需支付每年15,000美元的保证专利使用费。[70][85]  

1891年初,乔治·西屋直言不讳地向特斯拉解释了公司的财务困难,表示如果无法满足贷款方的要求,他将失去对西屋电气的控制权,特斯拉将不得不“与银行家们交涉”以获取未来的专利使用费。[89] 考虑到让西屋继续推广感应电动机的优势,特斯拉可能认为这一点显而易见,因此同意免除合同中的专利使用费条款。[89][90]  

六年后,作为与通用电气(由1892年爱迪生公司与汤姆森-休斯顿公司合并而成)签署专利共享协议的一部分,西屋公司以一次性支付216,000美元的方式购买了特斯拉的专利。[91][92][93]
\subsection{纽约实验室}
\begin{figure}[ht]
\centering
\includegraphics[width=6cm]{./figures/74e75cf8225146ef.png}
\caption{1894年,马克·吐温在特斯拉位于南第五大道的实验室中} \label{fig_Tesla_8}
\end{figure}
特斯拉通过授权他的交流电专利赚取的财富使他变得独立富有,并为他提供了时间和资金来追求自己的兴趣。[94] 1889年,特斯拉搬出了佩克和布朗租用的自由街商店,接下来的十多年里,他在曼哈顿的多个工作室/实验室中工作。这些实验室包括位于175号大街的实验室(1889–1892年),位于南第五大道33–35号的四楼(1892–1895年),以及位于东休斯顿街46号和48号的六楼和七楼(1895–1902年)。[95][96] 特斯拉和他的雇员在这些工作室中进行了他一些最重要的工作。
\subsubsection{特斯拉线圈} 
1889年夏,特斯拉前往巴黎参加1889年世界博览会,并了解了海因里希·赫兹(Heinrich Hertz)1886年至1888年间的实验,这些实验证明了电磁辐射的存在,包括无线电波。[97] 在重复并扩展这些实验时,特斯拉尝试使用他为改进弧光照明系统而开发的高速交流发电机为鲁姆科夫线圈供电,但发现高频电流使铁芯过热,并且融化了线圈中初级和次级绕组之间的绝缘材料。为了解决这个问题,特斯拉设计了他的“振荡变压器”,在初级和次级绕组之间使用空气间隙代替绝缘材料,并设计了一个可以在变压器内部或外部移动的铁芯。[98] 后来被称为特斯拉线圈,它能够产生高电压、低电流、高频率的交流电。[99] 他在后来的无线电力传输工作中使用了这种共振变压器电路。[100][101]
\subsubsection{国籍} 
1891年7月30日,35岁的特斯拉成为美国的自然化公民。[102][103] 同年,他获得了特斯拉线圈的专利。[104]
\subsubsection{无线照明}
\begin{figure}[ht]
\centering
\includegraphics[width=6cm]{./figures/7724cea168491c9f.png}
\caption{特斯拉在1891年哥伦比亚学院的讲座中演示通过“静电感应”进行无线照明,他手中持有两根长型盖斯勒管(类似霓虹管)。} \label{fig_Tesla_9}
\end{figure}
1890年后,特斯拉开始尝试通过电感耦合和电容耦合传输电力,使用他设计的特斯拉线圈产生高电压交流电。他试图开发一种基于近场电感和电容耦合的无线照明系统,并进行了多次公开演示,在这些演示中,他成功地使盖斯勒管甚至白炽灯泡在舞台上远程点亮。特斯拉花了大部分时间与各种投资者合作,探索这一新型照明方式的不同变种,但这些尝试都未能将他的发现商业化。

1893年,在美国密苏里州圣路易斯市的演示中,特斯拉在费城的富兰克林学会和美国电气光照协会的观众面前表示,他确信像他所设计的系统,最终可以通过地球传导“可理解的信号,甚至可能传输电力到任何距离,而无需使用电线”。

从1892年到1894年,特斯拉担任美国电气工程师学会(即今天的IEEE前身之一,另一个前身是无线电工程师学会)的副主席。
\subsubsection{多相系统与哥伦比亚博览会}
\begin{figure}[ht]
\centering
\includegraphics[width=6cm]{./figures/acd22d5d3f319d24.png}
\caption{西屋公司在1893年芝加哥哥伦比亚博览会上的“特斯拉多相系统”展示} \label{fig_Tesla_10}
\end{figure}
到1893年初,西屋公司工程师查尔斯·F·斯科特和本杰明·G·拉梅在特斯拉的感应电动机的高效版本上取得了进展。拉梅找到了使其所需的多相系统与旧的单相交流电和直流电系统兼容的方法,通过开发旋转变流器。[111] 西屋电气公司现在有了一种为所有潜在客户提供电力的方法,并开始将其多相交流电系统品牌化为“特斯拉多相系统”。他们认为特斯拉的专利使他们在多相交流电系统方面拥有优先权。[112]

西屋电气公司邀请特斯拉参与1893年芝加哥哥伦比亚博览会,公司在“电力大楼”中有一个专门展示电气设备的大展区。西屋电气公司赢得了为博览会提供交流电照明的合同,这是交流电历史上的一个重要事件,展示了该公司向美国公众证明其多相交流电系统的安全性、可靠性和效率,并能够为博览会上的其他交流电和直流电展品提供电力。[113][114][115]

特设的展区展示了特斯拉感应电动机的各种形式和模型。驱动这些电动机的旋转磁场通过一系列演示来解释,其中包括使用感应电动机中的双相线圈来旋转铜蛋,使其竖立起来的“哥伦布的蛋”演示。[116]

特斯拉在博览会为期六个月的展期中,曾访问过一周,参加了国际电气大会,并在西屋展区进行了一系列演示。[117][118] 一个特别的昏暗房间被布置出来,特斯拉展示了他的无线照明系统,使用他此前在美国和欧洲演示过的实验;[119] 这些演示包括使用高电压、高频率交流电来点亮无线气体放电灯。[120]

有一位观察者注意到:

在房间内,悬挂着两块硬橡胶板,表面覆盖着锡箔。它们大约相距十五英尺,作为来自变压器的电线端子。当电流开启时,那些没有接线的灯泡或管子,放置在悬挂板之间的桌子上,或几乎可以被手持在房间的任何地方,都被点亮了。这些实验和设备与特斯拉大约两年前在伦敦展示的相同,“当时它们引起了极大的惊奇和震撼”。[121]
\subsubsection{蒸汽驱动振荡发电机}  
在哥伦比亚博览会农业厅的国际电气大会上,特斯拉介绍了他当年获得专利的蒸汽驱动往复电力发电机,他认为这是一种更好的交流电发电方式。[122] 蒸汽被压入振荡器,通过一系列的开口迅速排出,推动一个与电枢连接的活塞上下运动。磁性电枢以高速上下振动,产生交流磁场。这在相邻的线圈中感应出交流电流。它去掉了蒸汽机/发电机的复杂部件,但作为发电的可行工程解决方案始终未能普及。[123][124]
\subsubsection{尼亚加拉顾问工作}  
1893年,负责尼亚加拉大瀑布建设公司的爱德华·迪安·亚当斯(Edward Dean Adams)寻求特斯拉的意见,关于如何选择最佳的系统来传输瀑布发电的电力。几年来,关于如何最佳地实现这一目标,已经有一系列提案和公开竞赛。美国和欧洲几家公司提出的方案包括两相和三相交流电、高压直流电和压缩空气等。亚当斯向特斯拉询问了所有竞争系统的当前状态。特斯拉建议亚当斯,二相系统将是最可靠的,并指出西屋电气公司曾在哥伦比亚博览会上展示过使用二相交流电点亮白炽灯的系统。根据特斯拉的建议以及西屋电气在哥伦比亚博览会上的展示,公司向西屋电气授予了在尼亚加拉大瀑布建设二相交流电发电系统的合同。同时,通用电气公司也获得了建设交流电配电系统的合同。[125]
\subsubsection{尼古拉·特斯拉公司} 
1895年,爱德华·迪安·亚当斯(Edward Dean Adams)在参观特斯拉实验室时对所见印象深刻,同意帮助成立尼古拉·特斯拉公司,该公司旨在资助、开发和营销特斯拉的多项专利和发明,包括一些新发明。阿尔弗雷德·布朗(Alfred Brown)也签约,带来了在佩克和布朗公司开发的专利。董事会成员还包括威廉·伯奇·兰金(William Birch Rankine)和查尔斯·F·科尼(Charles F. Coaney)。然而,由于90年代中期的经济困境,投资者寥寥无几,而公司计划推广的无线照明和振荡器专利未能取得成功。该公司继续处理特斯拉的专利,直到多年以后。
\subsubsection{实验室火灾}  
1895年3月13日凌晨,特斯拉实验室所在的南第五大道大楼发生火灾。火灾从大楼的地下室起火,火势非常猛烈,以至于特斯拉位于四楼的实验室被烧毁并坍塌到二楼。火灾不仅使特斯拉的正在进行的项目受到严重影响,还摧毁了他的一大批早期笔记和研究资料、模型以及演示作品,其中许多曾在1893年世界哥伦比亚博览会上展出。特斯拉在接受《纽约时报》采访时表示:“我太伤心了,无法言语。我还能说什么呢?”[127] 火灾后,特斯拉搬到东休斯顿街46号和48号,并在6楼和7楼重建了他的实验室。
\subsubsection{X射线实验}
从1894年开始,特斯拉开始研究他所称之为“不可见”种类的辐射能量,在之前的实验中,他注意到实验室中的胶片受损(后来被确认是“伦琴射线”或“X射线”)。他早期的实验使用了克鲁克斯管,这是一种冷阴极电气放电管。特斯拉可能无意中拍摄到了X射线图像——这比威廉·伦琴在1895年12月宣布X射线发现的时间早了几周——当时他试图用盖斯勒管(一种早期的气体放电管)拍摄马克·吐温的照片,然而在图像中,唯一拍摄到的东西是相机镜头上的金属锁螺丝。

1896年3月,在得知伦琴发现X射线及X射线成像(放射摄影)后,[130] 特斯拉开始进行自己的X射线成像实验。他设计了一种高能量单端真空管,该管没有靶电极,并且通过特斯拉线圈的输出工作(现代对此设备所产生现象的术语是“制动辐射”或“刹车辐射”)。在研究过程中,特斯拉设计了几种实验装置来产生X射线。他认为,利用他的电路,“这种仪器将……能够产生比普通设备更强大的伦琴射线”。[131]

特斯拉注意到使用他的电路和单节点X射线产生设备时的危险。在他关于这一现象的早期研究笔记中,他将皮肤损伤归因于多种原因。他早期认为皮肤损伤不是由伦琴射线引起的,而是由与皮肤接触时产生的臭氧所致,较小程度上是由硝酸引起的。特斯拉错误地认为X射线是纵波,就像等离子体中的波动那样。这些等离子体波可以出现在无力磁场中。[132][133]

1934年7月11日,《纽约先驱论坛报》刊登了一篇关于特斯拉的文章,其中回顾了他在实验单电极真空管时偶尔发生的一个事件。一个微小的粒子从阴极上脱落,穿过管子并物理撞击到他身上:[134]

特斯拉说他能感觉到粒子进入身体的地方传来一阵尖锐的刺痛感,然后在它穿出身体的地方再次感到疼痛。在将这些粒子与他“电枪”发射出的金属片进行比较时,特斯拉说:“这些力束中的粒子……将比这些粒子旅行得更快……而且它们会以集中状态旅行。”
\subsubsection{无线电遥控}
\begin{figure}[ht]
\centering
\includegraphics[width=6cm]{./figures/abee86456e2de99e.png}
\caption{1898年,特斯拉展示了一艘无线电控制的船只,他希望将其作为一种引导型鱼雷销售给世界各国的海军。[135]} \label{fig_Tesla_11}
\end{figure}
1898年,特斯拉展示了一艘使用基于同调器的无线电控制(他称之为“远程自动机”)的船只,公开演示是在麦迪逊广场花园的一场电气展览上进行的。[136] 特斯拉试图将这一创意卖给美国军方,作为一种无线电控制的鱼雷,但他们表现得兴趣不大。[137] 无线电遥控在第一次世界大战之前仍然是一个新奇的概念,直到战后,多个国家才开始将其用于军事项目。[138] 1899年5月13日,在前往科罗拉多斯普林斯的途中,特斯拉在芝加哥商业俱乐部的会议上进一步展示了“远程自动机”。
\subsection{无线电能}
从1890年代到1906年,特斯拉将大量时间和财富投入到一系列项目中,试图开发无线电能传输。这是他利用线圈进行无线照明实验的延伸。他认为,这不仅是将大量电力传输到全球的一种方式,而且,正如他在早期的演讲中所指出的,这也是实现全球通信的一种方式。

到1890年代中期,特斯拉开始着手研究一个想法,即他可能通过地球或大气层长距离传导电力,并开始进行实验来验证这个想法,包括在他位于东休斯顿街的实验室中设置一个大型共振变压器放大发射器。[144][145][146] 他似乎借鉴了当时流行的一个观点,即地球的大气层具有导电性,[147][148] 他提出了一个由气球悬浮的电极组成的系统,用来传输和接收信号,这些电极位于海拔30,000英尺(约9,100米)以上的高空,特斯拉认为在低气压环境下,他能够将高电压(数百万伏特)传输到远距离。