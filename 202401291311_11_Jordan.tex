% 幂零线性变换的 Jordan(若尔当)标准形
% license Usr
% type Tutor


尽管复数域可以保证$n$阶矩阵的特征多项式都有$n$个解,但依然不是所有矩阵都有$n$个线性无关的特征向量从而可以对角化。为了简化问题,我们需要“简化”矩阵(找一组基,使得矩阵在该基下有比较简单的形式,比较多$0$)。\autoref{the_nullpl_1}~\upref{nullpl}保证我们线性变换都有分块对角矩阵的形式。更进一步,通过在根子空间的相似变换,我们可以把每一个“对角块”上三角化。

与之比较,“Jordan标准形”是更加简化的形式。为拓展至任意线性变换,本节先从较为简单的幂零线性变换入手。
\subsection{循环子空间}
设矩阵$B$为线性空间$V$上的\textbf{幂零线性变换},即对于任意\textbf{非零}向量$\bvec x$,总存在非负整数$k$使得$B^{k}(\bvec x)=\bvec 0$且$B^{k-1}(\bvec x)\neq \bvec 0$。可以证明,$\{\bvec x,B\bvec x,B^2\bvec x,...B^{k-1}\bvec x\}$线性无关。
\begin{exercise}{}
证明:$\opn{Span}\{\bvec x,B\bvec x,B^2\bvec x,...B^{k-1}\bvec x\}$张成线性空间,并证明这是$B$的不变子空间。
\end{exercise}
设$W=\opn{Span}\{\bvec x,B\bvec x,B^2\bvec x,...B^{k-1}\bvec x\}$,将幂零变换$B$限制在该不变子空间上,记为$B|_W$。则每一列可表示为:$(B\bvec x,B^2\bvec x,...B^{k-1}\bvec x,B^k\bvec x)$,即:
\begin{equation}\label{eq_Jordan_1}
\left(\begin{array}{cccccc}
0 & 0 & 0 & \cdots & 0 & 0 \\
1 & 0 & 0 & \cdots & 0 & 0 \\
0 & 1 & 0 & \cdots & 0 & 0 \\
\vdots & \vdots & \vdots & \ddots & \vdots & \vdots \\
0 & 0 & 0 & \cdots & 1 & 0
\end{array}\right)~.
\end{equation}

假设该子空间为4维。输入$\bvec x=(1\,0\,0\,0)^T$,得到$B\bvec x=(0\,1\,0\,0)^T$。输入$B\bvec x=(0\,1\,0\,0)^T$,得到$B^2\bvec x=(0\,0\,1\,0)^T$。显然,$B$可以对基进行循环,因此把这种形式的基向量组$\opn{Span}\{\bvec x,B\bvec x,B^2\bvec x,...B^{k-1}\bvec x\}$称为\textbf{循环基(cyclic basis)},$W$为$B$的\textbf{循环子空间(cyclic subspace)}。

称形如\autoref{eq_Jordan_1} 的矩阵为\textbf{Jordan(若当)块},由若当块直和而成的矩阵为\textbf{ Jordan(若当)形矩阵}。本节的主要目的便是证明:复数域上的幂零变换总可以表示为Jordan矩阵。下面的讨论默认在复数域上。


\subsection{幂零变换的循环子空间分解}
\pentry{不变子空间\upref{InvSP}}

由于Jordan表示是分块对角矩阵,对角块(Joradan块)是循环子空间,则$B$有Jordan形表示等价于$V$可分解为$B$的循环子空间之直和。因此我们需要证明:

\begin{theorem}{}\label{the_Jordan_1}
若$B$是$V$上的幂零线性变换,则$V$总可以分解为$B$的循环子空间之直和。
\end{theorem}
Proof.

用归纳法证明\footnote{引自Jier Peter的《代数学基础》}。当线性空间维度为1时,由$B$的定义可知定理显然成立。现设定理对维度小于$n$时成立,设$\opn{dim}V=n$,需要利用该假设证明定理在此维度下成立。

由于$B$可以在$\opn{Im}B$上分解为循环子空间的直和,即:
\begin{equation}
\opn{Im}B=W_1\oplus W_2\oplus...\oplus W_k~,
\end{equation}

令$W_i=\opn{Span}\{\bvec x_i,B(\bvec x_i),B^2(\bvec x_i)...B^{l_i-1}(\bvec x_i)\}$,有$B^{l_i}|_{W_i}=0,i=1,2...k$。由于$B^{l_i-1}(\bvec x_i)\in \opn{ker}B$,因此可以扩充$\{B^{l_i-1}(\bvec x_i)\}^k_{i=1}$至$\opn{ker}B$的一组基:

\begin{equation}
\opn{ker}B=\opn{Span}\{\bvec v\}^m_{i=1}\cup \{B^{l_i-1}(\bvec x_i)\}^k_{i=1}~,
\end{equation}

则对于任意$\bvec x_i\in V$,都可以由$\{\bvec v\}^m_{i=1}\cup \{\bvec x_i,B(\bvec x_i),B^2(\bvec x_i)...B^{l_i-1}(\bvec x_i)\}^{k}_{i=1}$表示。接下来我们证明:
\begin{equation}
V=\bigoplus ^m_{i=1}\opn{Span}\bvec v_i\oplus W_1\oplus W_2\oplus...\oplus W_n~.
\end{equation}


设$\{a_i\}^m_{i=1},\{b^i_1,b^i_2...b^i_{l_{i-1}}\}^{k}_{i=1}\in\mathbb F$,且
\begin{equation}\label{eq_Jordan_2}
a_i\bvec v_i+b^i_jB^{j}(\bvec x_i)=\bvec 0~.
\end{equation}

如前所叙,$j\in[1,l_{i-1}],i\in[1,k]$。两边各乘以$B$,则有:
\begin{equation}
b^i_jB^{j}(\bvec x_i)=\bvec 0~,
\end{equation}
此时$j\in[1,l_{i-2}]$。由题设知$B^j(\bvec x_i)$线性无关,则系数为$0$,代入\autoref{eq_Jordan_2} 得:
\begin{equation}
a_i\bvec v_i+b^i_{l_{i-1}}B^{l_{i-1}}(\bvec x_i)=\bvec 0~.
\end{equation}

这是$\opn{ker}B$的basis,所以系数为$0$。因此\autoref{eq_Jordan_2} 的所有系数为0,所证向量组线性无关,和为直和。
由于$B\bvec v_i=0$,则$\opn{Span}\bvec v_i$都是$B$的一维循环子空间,定理得证。


