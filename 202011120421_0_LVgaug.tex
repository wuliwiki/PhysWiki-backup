% 速度规范
% 长度规范|速度规范|波函数|规范变换|薛定谔方程|麦克斯韦方程组

\pentry{长度规范\upref{LenGau}}

和长度规范中的思路一样, 我们只在使用偶极子近似时讨论\textbf{速度规范(velocity gauge)}. 用角标 $V$ 表示速度规范, 先从规范不变的哈密顿算符(\autoref{LenGau_eq2}~\upref{LenGau})出发
\begin{equation}\label{LVgaug_eq2}
H_V = H_0 - \frac{q}{2m} (\bvec A_V \vdot \bvec p + \bvec p \vdot \bvec A_V)
+ \frac{q^2}{2m} \bvec A_V^2 + q \varphi_V
\end{equation}
长度规范的思路是把上式中的 $\bvec A_V^2$ 消去. 对库仑规范使用规范变换(\autoref{QMEM_eq5}~\upref{QMEM})
\begin{equation}\label{LVgaug_eq3}
\Psi_C(\bvec r, t) = \exp(\I q\chi_V)\Psi_V(\bvec r, t)
\end{equation}
\begin{equation}\label{LVgaug_eq4}
\chi_V(t) = -\frac{q}{2m} \int_{-\infty}^t \bvec A_C^2(t') \dd{t'}
\end{equation}
得
\begin{equation}\label{LVgaug_eq1}
\bvec A_V = \bvec A_C - \grad \chi_V = \bvec A_C
\end{equation}
可见\textbf{速度规范下的矢势和库仑规范的相同}, 以下统一记为 $\bvec A_C$. 这使得广义动量(\autoref{QMEM_eq6}~\upref{QMEM})也和库伦规范的相同, 
\begin{equation}
\bvec p_V = \bvec p_C =  m \bvec v + q\bvec A_{C} = -\I \grad
\end{equation}
再看标势的变换:
\begin{equation}\label{LVgaug_eq5}
\varphi_V = \varphi_C + \pdv{\chi_V}{t} = - \frac{q}{2m} \bvec A_C^2
\end{equation}
\autoref{LVgaug_eq1} 和\autoref{LVgaug_eq5} 带入\autoref{LVgaug_eq2} 可以消去 $\bvec A_V^2$ 项得
\begin{equation}
H_V = H_0 - \frac{q}{m} \bvec A \vdot \bvec p
\end{equation}
薛定谔方程为
\begin{equation}
H_V \Psi_V = \I \pdv{t} \Psi_V
\end{equation}

长度规范与速度规范中的波函数转换关系为为(\autoref{LenGau_eq3}~\upref{LenGau})
\begin{equation}
\Psi_V = \exp[\I q(\chi_L - \chi_V)]\Psi_L = \exp[-\I\frac{q^2}{2m}\int_{-\infty}^t \bvec A^2(t')\dd{t'} + \I q \bvec A\vdot \bvec r] \Psi_L
\end{equation}
