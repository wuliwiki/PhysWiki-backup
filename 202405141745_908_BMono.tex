% 磁单极子
% keys 磁单极子|麦克斯韦方程组|高斯单位
% license Xiao
% type Tutor

\pentry{磁场的高斯定律\nref{nod_MagGau}}{nod_6a71}

\footnote{参考 Wikipedia \href{https://en.wikipedia.org/wiki/Magnetic_monopole}{相关页面}。}\textbf{磁单极子(magnetic monopole)}是电动力学中的一类假想的粒子, 至今未被真实观测到,但是其理论的提出背后有丰富且深刻的数学和物理内涵。

\subsection{麦克斯韦方程组}
\pentry{麦克斯韦方程组\nref{nod_MWEq}, 洛伦兹力\nref{nod_Lorenz}}{nod_4dde}

 在\enref{麦克斯韦方程组}{MWEq} 出现后,人们注意到若假设磁单极子存在并能像带电粒子产生电场那样产生磁场以及像电流产生磁场那样产生电场, 那么电场 $\bvec E$ 和磁场 $\bvec B$ 的地位就完全平等了。

磁单极子是一类假想粒子的名称, 类似于把质子和电子等微观带电粒子称为 “电单极子”。 类比电荷, 我们说磁单极子中带有\textbf{磁荷(magnetic charge)}。 磁荷的国际单位是 $\Si{Am}$(安培·米), 以下把磁荷记为 $q_m$, \textbf{磁荷密度(magnetic charge density)}记为 $\rho_m$, 则麦克斯韦方程组变为
\begin{equation}
\begin{aligned}
&\div \bvec E = \frac{\rho}{\epsilon_0}~,\\
&\curl \bvec E = - \mu_0 \bvec j_m -\pdv{\bvec B}{t}~,\\
&\div \bvec B = \mu_0 \rho_m~,\\
&\curl \bvec B = \mu_0 \bvec j + \mu_0\epsilon_0 \pdv{\bvec E}{t}~.
\end{aligned}
\end{equation}
电荷和磁荷的总洛伦兹力(\autoref{eq_Lorenz_1}~\upref{Lorenz})变为
\begin{equation}
\bvec F = q \qty(\bvec E + \bvec v \cross \bvec B) +
q_m \qty(\bvec B - \bvec v \cross \bvec E)~.
\end{equation}


\subsubsection{高斯单位}
高斯单位制\upref{GaussU}下, 麦克斯韦方程组和洛伦兹力具有更对称的形式
\begin{equation}\label{eq_BMono_1}
\begin{aligned}
&\div \bvec E = 4\pi\rho~,\\
&\curl \bvec E = -\frac{1}{c}\pdv{\bvec B}{t}  - \frac{4\pi}{c}\bvec j_m~,\\
&\div \bvec B = 4\pi\rho_m~,\\
&\curl \bvec B = \frac{1}{c}\pdv{\bvec E}{t} + \frac{4\pi}{c} \bvec j~.
\end{aligned}
\end{equation}
\begin{equation}
\bvec F = q \qty(\bvec E + \frac{\bvec v}{c}\cross \bvec B) + q_m \qty(\bvec B - \frac{\bvec v}{c}\cross \bvec E)~.
\end{equation}

\subsection{磁单极子的磁场与磁矢势}
\pentry{磁通量\nref{nod_BFlux}}{nod_4974}
下面我们在高斯单位制\upref{GaussU}下讨论问题。假设一个静止在原点的磁单极子,磁荷为 $e_M$,则磁场满足类似于电场高斯定律的方程,可以写出磁场的散度与磁荷成正比的方程:
\begin{equation}\label{eq_BMono_4}
\div \bvec B(\bvec x)= 4\pi e_M\delta^3(0)~.
\end{equation}
解上述微分方程可以得到
\begin{equation}
\bvec B(\bvec x)=\frac{e_M}{|\bvec x|^3}\bvec x~.
\end{equation}
它对应的磁矢势为
\begin{equation}\label{eq_BMono_5}
\bvec A(\bvec x)=e_M \frac{1-\cos\theta}{r\sin\theta} \hat e_\phi~.
\end{equation}
可以验证 $\curl \bvec A(\bvec x) = \bvec B(\bvec x)$。要注意的是上式在 $\theta=\pi$ 即 $z$ 轴负半轴处奇异。所以 $\curl \bvec A(\bvec x)$ 在 $z$ 轴负半轴上奇异。从原点出发沿着 $z$ 轴负半轴一直到无穷远处的弦被称为\textbf{狄拉克弦}。从下面的推导我们将看出狄拉克弦是没有物理意义的,可以利用 $\bvec A$ 的规范变换从一根弦变为另一根弦。
\subsubsection{stokes 定理产生的矛盾}
对于场 $\curl \bvec A(\bvec x)$,它在闭球面上面积分(不考虑狄拉克弦处的奇异的部分)为
\begin{equation}\label{eq_BMono_3}
\int \bvec B(\bvec x)\cdot \dd{\bvec s} =\int \curl \bvec A(\bvec x) \cdot \dd{\bvec s}=0~. 
\end{equation}
上式最终利用了 stokes 定理,由于闭曲面是没有边界的,所以磁矢势的旋度在闭曲面上的积分为 $0$。这与我们之前对磁单极子产生磁场的期待相违背。我们期待 $\bvec B$ 在球面上的积分应当是
\begin{equation}\label{eq_BMono_2}
\int \bvec B(\bvec x)\cdot \dd{\bvec s}=|\bvec B(|\bvec x|=R)|\cdot 4\pi R^2=4\pi e_M~.
\end{equation}
产生矛盾的原因正是狄拉克弦的存在。stokes 定理并不适用于积分区域内存在非奇异点的情况。或者也可以将狄拉克弦理解为一根具有无穷大磁场的磁感线,沿着狄拉克弦一直延伸到原点;它导致了一个有限的朝内的磁通量通过闭曲面,与\autoref{eq_BMono_2} 的磁单极子产生的朝闭曲面外的那部分磁通相抵消,即加起来为 $0$。
\subsection{规范变换与 WuYang 单极子}
\pentry{规范变换\nref{nod_Gauge}}{nod_afbd}

如果我们将\autoref{eq_BMono_5} 作一个 $\theta\rightarrow \pi-\theta$ 的代换(即关于 $xy$ 平面作镜像对称),可以得到另一种满足\autoref{eq_BMono_4} 的磁矢势的解为
\begin{equation}\label{eq_BMono_6}
\bvec A(\bvec x)=-e_M\frac{1+\cos\theta}{r\sin\theta}
\hat e_\phi~.
\end{equation}
可以看到狄拉克弦的位置变为 $\theta=0$,即 $z$ 轴正半轴的位置。与之前的解的狄拉克弦的位置不同。对于一个磁单极子它所激发出的磁场虽然是确定的,但是它的磁矢势可以有无穷多种解的情况,而狄拉克弦也可以有任意多种情况。虽然狄拉克弦上磁矢势的取值是奇异的,但是它是没有物理意义的,无论它处于什么位置都对物理结果没有影响。

为了避免讨论狄拉克弦的情况,我们可以将\autoref{eq_BMono_5} 作为北半球磁矢势的取值,将\autoref{eq_BMono_6} 作为南半球磁矢势的取值,则磁矢势在赤道附近会有两种不同取值 $\bvec A^S(\bvec x)=-e_M\frac{1+\cos\theta}{r\sin\theta}
\hat e_\phi$ 和 $\bvec A^N(\bvec x)=e_M \frac{1-\cos\theta}{r\sin\theta} \hat e_\phi$,这种构造磁矢势的方式被称为 Wu-Yang 单极子,Wu-Yang 利用这种思想提出了用纤维丛\upref{Fibre}去研究磁单极子问题背后的几何和拓扑性质。

由于 Wu-Yang 单极子所构造的磁矢势没有奇异的狄拉克弦,所以我们可以对北半球和南半球分别利用 stokes 定理求出磁通量。磁场在闭球面上的磁通量为 $4\pi e_M$(根据\autoref{eq_BMono_2} )可以被表示为沿赤道 $C$ 的两个磁矢势的线积分之差(注意到 $\bvec A^N,\bvec A^S$ 在赤道处的指向都是沿切向方向的):
\begin{equation}\label{eq_BMono_7}
\Phi = \int \bvec B(\bvec x) \cdot \dd{\bvec x} = 4\pi  e_M = \oint_C (\bvec A^N(\bvec x)-\bvec A^S(\bvec x)) \cdot \dd{\bvec l} ~.
\end{equation}

根据\autoref{eq_Gauge_3}~\upref{Gauge} ,规范变换下磁矢势的变换行为是
\begin{equation}
\bvec A\rightarrow \bvec A+\nabla \Lambda~.
\end{equation}
可以通过规范变换将一组磁矢势的解转换为另一组磁矢势的解。我们可以构造一个赤道上的\textbf{连续的标量函数} $\Lambda$ 使得赤道上的两种取值可以通过规范变换相联系: $\bvec A^N = \bvec A^S + \nabla\Lambda$ 。那么\autoref{eq_BMono_7} 转化为
\begin{equation}\label{eq_BMono_8}
\Phi = 4\pi e_M = \oint_C \nabla\Lambda \cdot \dd{\bvec l} ~.
\end{equation}

\subsection{电荷量子化}
\pentry{电磁场中的薛定谔方程及规范变换\nref{nod_QMEM}}{nod_9af5}
考虑物质场(电子)与电磁场的耦合,考虑一个电荷量为 $q$ 的粒子,它在电磁场中的薛定谔方程如下:
\begin{equation}
\left(\frac{1}{2m}(\bvec p-\frac{q}{c}\bvec A)^2 +q\phi\right)\psi(\bvec r) = E\psi(\bvec r)~.
\end{equation}
其中 $\bvec p=-\I\hbar\bvec \nabla$,那么在规范变换 $\bvec A\rightarrow \bvec A+\nabla \Lambda$ 下,波函数以 $\psi(\bvec r)\rightarrow \psi(\bvec r)\exp(\I\frac{q}{\hbar c}\Lambda(\bvec r))$ 变换时,薛定谔方程仍然成立。因此 $\Lambda(\bvec r)$ 的取值范围可以约定为 $[0,2\pi \hbar c/q]$ ,并且是首尾相接的(周期性边界条件),可以从 $0$ 连续地“跃变”到 $2\pi \hbar c/q$。那么\autoref{eq_BMono_8} 的结果一定是 $2\pi \hbar c/q$ 的倍数:
\begin{equation}
\Phi = 4\pi e_M=\oint_C \nabla\Lambda\cdot \dd{\bvec l} = n\cdot 2\pi \hbar c/q,\quad n=0,\pm 1,\pm 2,\cdots ~.
\end{equation}
由此我们可以得到
\begin{equation}
e_M \cdot q= \frac{\hbar c}{2}\cdot n,\quad n\in \mathbb{Z}~.
\end{equation}
这被称为\textbf{Dirac 量子化条件}。意味着如果世界上的某个地方存在磁荷,那么某个粒子的带电量与磁荷大小 $e_M$ 一定具有\autoref{eq_BMono_6} 的关系,电荷一定是量子化的。同样的,磁荷也是量子化的。