% 维尔纳·海森堡(综述)
% license CCBYSA3
% type Wiki

本文根据 CC-BY-SA 协议转载翻译自维基百科\href{https://en.wikipedia.org/wiki/Werner_Heisenberg}{相关文章}。

\textbf{维尔纳·卡尔·海森堡}(Werner Karl Heisenberg,/ˈhaɪzənbɜːrɡ/;[2] 德语:[ˈvɛʁnɐ ˈhaɪzn̩bɛʁk] ⓘ;1901年12月5日-1976年2月1日)[3] 是德国理论物理学家,量子力学理论的主要奠基人之一,也是二战期间纳粹核武器计划中的核心科学家。他于1925年发表了他的《重新诠释》论文,对旧量子理论进行了重要的重新解释。同年,他与马克斯·玻恩和帕斯库尔·约尔丹合写了一系列论文,详细阐述了他的矩阵量子力学表述。他以1927年发表的不确定性原理而闻名。海森堡因“创立量子力学”而获得1932年诺贝尔物理学奖。[4][a]  

海森堡还对湍流流体动力学、原子核、铁磁性、宇宙射线和亚原子粒子的理论作出了贡献。他在1957年协助规划了卡尔斯鲁厄的第一个西德核反应堆,以及慕尼黑的一个研究反应堆。  

二战后,他被任命为\textbf{威廉皇帝物理研究所}(Kaiser Wilhelm Institute for Physics)所长,不久后该研究所更名为马\textbf{克斯·普朗克物理研究所}。他一直担任所长,直到研究所于1958年迁至慕尼黑。1960年至1970年间,他担任\textbf{马克斯·普朗克物理与天体物理研究所}所长。  

此外,海森堡还担任\textbf{德国研究委员会}主席、\textbf{原子物理委员会}\textbf{主席}、\textbf{核物理工作组}主席以及\textbf{洪堡基金会}主席。[1]  
\subsection{早年生活与教育 }
\subsubsection{早年}  
维尔纳·卡尔·海森堡(Werner Karl Heisenberg)于德国维尔茨堡出生,父亲是卡斯帕·恩斯特·奥古斯特·海森堡(Kaspar Ernst August Heisenberg),母亲是安妮·维克莱因(Annie Wecklein)。[6] 他的父亲是一名古典语言的中学教师,后来成为德国大学体系中唯一一位中世纪和现代希腊研究的正式教授(ordinarius professor)。[7]  

海森堡在信仰上是一名路德宗基督徒。[8] 在青少年晚期,他曾在巴伐利亚阿尔卑斯山徒步旅行时阅读柏拉图的《蒂迈欧》。他回忆起在慕尼黑、哥廷根和哥本哈根接受科学训练时,与同学和老师进行的关于理解原子的哲学对话。[9] 海森堡后来表示:“我的思想是通过学习哲学、柏拉图以及类似的内容形成的。”[10] 他还说:“现代物理学明确支持柏拉图的观点。实际上,物质的最小单位并不是普通意义上的物理对象;它们是形式、理念,只能通过数学语言清晰地表达。”[11]  

1919年,海森堡作为自由军团(Freikorps)的一员来到慕尼黑,参与对一年前成立的巴伐利亚苏维埃共和国的斗争。五十年后,他回忆起那段日子,说那是年轻人的乐趣,就像“玩警察抓小偷一类的游戏;完全没有任何严肃的意义。”[12] 他的职责仅限于从“红色”行政建筑中没收自行车或打字机,以及看守被怀疑的“红色”囚犯。[13]  
\subsubsection{大学学习}
\begin{figure}[ht]
\centering
\includegraphics[width=6cm]{./figures/481ce511fb21e6cb.png}
\caption{1924年的海森堡} \label{fig_Heisen_1}
\end{figure}
从1920年至1923年,海森堡在慕尼黑的路德维希-马克西米利安大学师从阿诺德·索末菲和威廉·维恩学习物理和数学,同时在哥廷根的乔治-奥古斯特大学师从马克斯·玻恩和詹姆斯·弗兰克**学习物理,并跟随大卫·希尔伯特学习数学。他于1923年在慕尼黑大学获得索末菲指导的博士学位。  

1922年6月,索末菲带海森堡前往哥廷根参加玻尔节(Bohr Festival),因为索末菲对学生非常关心,并了解海森堡对尼尔斯·玻尔原子物理理论的兴趣。在活动中,玻尔作为客座讲师发表了一系列关于量子原子物理的全面演讲,海森堡首次见到了玻尔,这对他产生了深远的影响。[14][15][16]  

海森堡的博士论文由索末菲建议题目,研究湍流问题;[17] 论文讨论了层流的稳定性和湍流的性质。他通过**奥尔-索末菲方程**(Orr-Sommerfeld equation)研究稳定性问题,这是一种描述层流微小扰动的四阶线性微分方程。二战后,他曾短暂地回到这一课题的研究中。[18]  

在哥廷根,海森堡在马克斯·玻恩指导下完成了关于反常塞曼效应的资格论文(Habilitationsschrift),并于1924年取得资格证书。[19][3][20][21]  

青年时期,海森堡是新探路者运动(Neupfadfinder,一支德国童子军协会,也是德国青年运动的一部分)的成员和童子军领袖。[22][23][24] 1923年8月,罗伯特·霍泽尔与海森堡共同组织了一个来自慕尼黑的童子军团队前往芬兰旅行。[25]  
\subsubsection{个人生活}  
海森堡热爱古典音乐,是一位出色的钢琴家。[3] 他对音乐的兴趣促成了与未来妻子的相遇。1937年1月,海森堡在一场私人音乐会中结识了伊丽莎白·舒马赫(Elisabeth Schumacher,1914–1998)。伊丽莎白是柏林一位著名经济学教授的女儿,她的兄弟是《小即是美》的作者、经济学家E. F. 舒马赫。海森堡于同年4月29日与她结婚。  

1938年1月,他们的双胞胎玛丽亚(Maria)和沃尔夫冈(Wolfgang)出生。沃尔夫冈·泡利因此祝贺海森堡的“对生成”(pair creation),这是一种来自基本粒子物理学的术语“对产生”的文字游戏。此后12年间,他们又育有五个孩子:芭芭拉(Barbara)、克里斯蒂娜(Christine)、尤赫恩(Jochen)、马丁(Martin)和弗蕾娜(Verena)。[26][27]  

1939年,海森堡在德国南部瓦尔兴湖的乌尔费尔德为家人购置了一处夏季住宅。  

海森堡的一个儿子马丁·海森堡(Martin Heisenberg)成为维尔茨堡大学的神经生物学家,另一个儿子尤赫恩·海森堡(Jochen Heisenberg)则成为新罕布什尔大学的物理学教授。[28]  
\subsection{学术生涯}  
\subsubsection{哥廷根、哥本哈根与莱比锡}  
从1924年至1927年,海森堡在哥廷根担任私人讲师(Privatdozent),意味着他具备独立教学和考试的资格,但没有教授职位。1924年9月17日至1925年5月1日期间,海森堡获得国际教育委员会洛克菲勒基金会奖学金,前往哥本哈根大学理论物理研究所,与所长尼尔斯·玻尔开展研究。他的重要论文《关于运动学与力学关系的量子理论重新诠释》(Über quantentheoretische Umdeutung kinematischer und mechanischer Beziehungen),即所谓的重新诠释论文(Umdeutung paper),于1925年9月发表。[29]  

他随后返回哥廷根,与马克斯·玻恩和帕斯库尔·约尔丹合作,在大约六个月内发展了量子力学的矩阵力学表述。1926年5月1日,海森堡开始在哥本哈根担任大学讲师,并成为玻尔的助理。1927年,海森堡在哥本哈根工作期间,研究量子力学的数学基础,提出了著名的不确定性原理。2月23日,他在写给物理学家沃尔夫冈·泡利的信中首次描述了这一新原理。[30] 在关于该原理的论文中,[31] 海森堡使用了“Ungenauigkeit”(意为“模糊”或“不精确”),而非“不确定性”来描述这一概念。[3][32][33]  

1927年,海森堡被任命为莱比锡大学理论物理学的正式教授(ordentlicher Professor)兼物理系主任。他于1928年2月1日发表了就职演讲。在莱比锡发表的第一篇论文中,[34] 海森堡利用泡利不相容原理解决了铁磁性的谜题。[3][20][32][35]  

25岁的海森堡成为德国最年轻的全职教授,并担任莱比锡大学理论物理研究所所长。[36] 他讲授的课程吸引了包括爱德华·泰勒和罗伯特·奥本海默在内的学生,[36] 后者后来参与了美国的曼哈顿计划。[37]  

在海森堡于莱比锡任职期间,与他一起学习和工作的博士生、研究生以及研究合作者的高水平可见一斑,从他们后来获得的广泛赞誉便可证明。他的学生和合作者包括:埃里希·巴格(Erich Bagge)、费利克斯·布洛赫(Felix Bloch)、乌戈·法诺(Ugo Fano)、齐格弗里德·弗吕格(Siegfried Flügge)、威廉·维尔米利恩·休斯顿(William Vermillion Houston)、弗里德里希·洪德(Friedrich Hund)、罗伯特·S·穆利肯(Robert S. Mulliken)、鲁道夫·佩耶尔斯(Rudolf Peierls)、乔治·普拉茨克(George Placzek)、伊西多·艾萨克·拉比(Isidor Isaac Rabi)、弗里茨·绍特(Fritz Sauter)、约翰·C·斯莱特(John C. Slater)、爱德华·泰勒(Edward Teller)、约翰·哈斯布鲁克·范弗莱克(John Hasbrouck van Vleck)、维克托·弗雷德里克·韦斯科普夫(Victor Frederick Weisskopf)、卡尔·弗里德里希·冯·魏茨泽克(Carl Friedrich von Weizsäcker)、格雷戈尔·温策尔(Gregor Wentzel)以及克拉伦斯·齐纳(Clarence Zener)。[38]  

1929年初,海森堡和泡利共同发表了两篇奠定相对论量子场论基础的论文中的第一篇。[39] 同年,海森堡展开了讲学之旅,访问了中国、日本、印度和美国。[32][38] 1929年春,他在芝加哥大学担任访问讲师,讲授量子力学课程。[40]  

1928年,英国数学物理学家保罗·狄拉克(Paul Dirac)推导出了他的相对论量子力学波动方程,这一方程预示了正电子的存在。1932年,美国物理学家卡尔·大卫·安德森(Carl David Anderson)通过宇宙射线的云室照片,确认了一条轨迹是由正电子产生的。1933年中期,海森堡提出了他的正电子理论,并在两篇论文中详细阐述了他对狄拉克理论的思考和该理论的进一步发展。第一篇论文《对狄拉克正电子理论的评论》(Bemerkungen zur Diracschen Theorie des Positrons)于1934年发表,[41] 第二篇《狄拉克正电子理论的推论》(Folgerungen aus der Diracschen Theorie des Positrons)于1936年发表。[32][42][43]  

在这些论文中,海森堡首次将狄拉克方程重新解释为描述任意自旋为 ħ/2 的点粒子的“经典”场方程,并将其置于涉及反对易子的量子化条件下。通过这一重新诠释,海森堡将狄拉克方程作为一种准确描述电子的(量子)场方程,将物质与电磁学置于同一理论基础上:均由允许粒子产生和湮灭的相对论量子场方程所描述。(赫尔曼·外尔(Hermann Weyl)早在1929年的一封写给爱因斯坦的信中已经描述了这一点。)
\subsubsection{《矩阵力学与诺贝尔奖》}
海森堡的《重新解释》论文,奠定了现代量子力学的基础,这篇论文让物理学家和历史学家都感到困惑。他的方法假定读者已经熟悉克拉默-海森堡过渡概率的计算。论文中的主要新思想——非对易矩阵——仅通过拒绝不可观测的量来得到证明。它通过基于对应原理的物理推理,引入了矩阵的非对易乘法,尽管海森堡当时并不熟悉矩阵的数学理论。麦金农(MacKinnon)已经重构了通向这些结果的路径,艾奇森(Aitchison)及其合著者则完成了详细的计算。

在哥本哈根,海森堡和汉斯·克拉默斯合作撰写了一篇关于色散的论文,讨论了波长大于原子尺寸的辐射与原子的散射。他们证明了克拉默斯早期发展出的成功公式不能基于波尔轨道,因为过渡频率是基于不恒定的能级间距。相比之下,经典锐化轨道的傅里叶变换中的频率是等间隔的。但这些结果可以通过半经典的虚拟态模型来解释:入射辐射使得价电子或外层电子激发到一个虚拟态,随后从这个虚拟态衰减。在后来的论文中,海森堡展示了这个虚拟振荡器模型也能解释荧光辐射的偏振现象。

这两项成功,以及波尔-索末菲模型在解释异常塞曼效应的突出问题上不断失败,促使海森堡利用虚拟振荡器模型尝试计算光谱频率。尽管这个方法在应用于现实问题时过于复杂,海森堡转而考虑一个更简单的例子,即非谐振荡器。

偶极振荡器由一个简单的谐振荡器组成,可以将其视为一个充电粒子在弹簧上振荡,受到外部力的扰动,例如外部电荷。振荡电荷的运动可以用振荡器频率的傅里叶级数来表示。海森堡通过两种不同的方法解决了量子行为问题。首先,他采用虚拟振荡器方法处理该系统,计算由外部源产生的能级间的跃迁。

然后,他通过将非谐势项视为对谐振荡器的扰动,并使用他和玻恩开发的扰动方法来解决同样的问题。这两种方法得出了相同的结果,尤其是对于第一项和非常复杂的二阶修正项。这表明,在这些复杂的计算背后,存在一个一致的方案。

因此,海森堡着手在不依赖虚拟振荡器模型的情况下,公式化这些结果。为此,他用矩阵代替了空间坐标的傅里叶展开,这些矩阵对应于虚拟振荡器方法中的跃迁系数。他通过引用波尔的对应原理和泡利的观点——即量子力学必须局限于可观察量——来为这种替代方法提供理论依据。

7月9日,海森堡将这篇论文交给玻恩审阅并提交出版。当玻恩阅读论文时,他认出了这种公式化方法可以被转录并扩展为矩阵的系统语言,而这正是他在布雷斯劳大学跟随雅各布·罗萨内斯学习时所学的内容。玻恩在他的助手兼前学生帕斯夸尔·乔丹的帮助下,立即开始进行转录和扩展,并将结果提交出版;这篇论文在海森堡论文提交后的60天内就被接收了。三位作者在年底前共同提交了后续论文。

直到此时,物理学家们很少使用矩阵;它们被认为属于纯数学的领域。古斯塔夫·米耶(Gustav Mie)在1912年关于电动力学的论文中使用过矩阵,玻恩在1921年关于晶体格理论的研究中也使用过矩阵。尽管在这些案例中使用了矩阵,但矩阵的代数运算和乘法并未像在量子力学的矩阵形式化中那样发挥作用。

1928年,阿尔伯特·爱因斯坦提名海森堡、玻恩和乔丹为诺贝尔物理学奖的候选人。1932年诺贝尔物理学奖的宣布推迟到了1933年11月。那时,宣布海森堡获得了1932年诺贝尔物理学奖,“因其创造了量子力学,其应用,尤其是,导致了氢的同素异形体形式的发现”。
\subsubsection{量子理论的解释}
量子力学的发展及其在“现实”方面的看似矛盾的含义,具有深刻的哲学影响,特别是关于科学观察真正意义的讨论。与阿尔伯特·爱因斯坦和路易·德布罗意等现实主义者不同,他们相信粒子在任何时刻都有客观存在的动量和位置(即使无法同时测量这两个量),海森堡则是一位反现实主义者,他认为直接了解“现实”是什么超出了科学的范围。在他的著作《物理学家对自然的理解》中,海森堡认为,归根结底,我们只能谈论描述粒子某些特征的知识(如表格中的数字),但我们永远无法获得对粒子本身的“真实”接触:

“我们再也不能独立于观察过程谈论粒子的行为。最终的结果是,量子理论中数学形式化的自然法则不再处理基本粒子本身,而是处理我们对它们的认知。我们也不再可能问这些粒子是否在空间和时间中客观存在……当我们谈论当今时代精确科学中的自然图景时,我们所指的不再是自然的图景,而是我们与自然之间关系的图景……科学不再作为客观观察者面对自然,而是把自己视为人类与自然之间互动中的一个行动者。科学分析、解释和分类的方法已经意识到它的局限性,这些局限性源于这样一个事实:科学通过介入改变并重塑了研究对象。换句话说,方法和对象不再能够分开。”
\subsubsection{SS调查}
在詹姆斯·查德威克于1932年发现中子后不久,海森堡提交了关于他核子中子-质子模型的三篇论文中的第一篇。1933年,阿道夫·希特勒上台后,海森堡在媒体中被攻击为“白犹太人”(即表现得像犹太人的雅利安人)。支持德国物理学(也称雅利安物理学)的团体对包括阿诺德·索末菲和海森堡在内的领先理论物理学家发起了激烈的攻击。从1930年代初起,反犹太主义和反理论物理学运动“德国物理学”开始关注量子力学和相对论理论。政治因素在大学环境中的影响超过了学术能力,尽管该运动的两位最著名支持者是诺贝尔物理学奖得主菲利普·伦纳德和约翰内斯·斯塔克。

曾有许多失败的尝试,希望海森堡能被任命为多所德国大学的教授。他尝试接替阿诺德·索末菲的职位未果,因为遭到了德国物理学运动的反对。1935年4月1日,海森堡的博士导师、著名的理论物理学家索末菲在慕尼黑大学获得了名誉教授职称。然而,索末菲在选举继任者的过程中继续担任其职位,这一过程直到1939年12月1日才结束。这个过程之所以如此漫长,是因为慕尼黑大学教职工的选拔与德国教育部及德国物理学支持者之间存在学术和政治上的分歧。

1935年,慕尼黑大学教职工委员会拟定了替代索末菲担任理论物理学常任教授和理论物理学研究所所长的候选人名单。三位候选人都曾是索末菲的学生:海森堡,曾获得诺贝尔物理学奖;彼得·德拜,1936年获得诺贝尔化学奖;以及理查德·贝克尔。慕尼黑大学教职工委员会坚定支持这些候选人,将海森堡列为首选。然而,德国物理学的支持者和帝国教育部(REM)内部的部分力量有自己的候选人名单,这场争斗持续了四年多。在此期间,海森堡受到了德国物理学支持者的激烈攻击。其中一次攻击被刊登在SS的报纸《黑色军团》上,该报由海因里希·希姆莱领导。在文章中,海森堡被称为“白犹太人”,应该让他“消失”。这些攻击被严肃对待,因为犹太人当时遭到了暴力攻击和监禁。海森堡通过一篇社论和一封给希姆莱的信进行反击,试图解决这一问题并恢复自己的名誉。

有一次,海森堡的母亲拜访了希姆莱的母亲。这两位妇女相识,因为海森堡的外祖父和希姆莱的父亲是巴伐利亚登山俱乐部的成员,并且曾分别担任过教堂牧师。最终,希姆莱通过两封信解决了海森堡的争议,分别于1938年7月21日向SS集团军指挥官海德里希和海森堡本人发出了信件。在给海德里希的信中,希姆莱表示德国不能失去或沉默海森堡,因为他对培养一代科学家有重要价值。在给海森堡的信中,希姆莱说,这封信是根据海森堡家庭的推荐发出的,并警告海森堡要区分专业物理学研究成果与相关科学家的个人和政治态度。

威廉·穆勒取代索末菲,成为慕尼黑大学的教授。穆勒并非理论物理学家,未曾在物理学期刊上发表过论文,也不是德国物理学会的成员。他的任命被视为一种荒谬的做法,并且对理论物理学的教育造成了不利影响。

三位领导SS调查海森堡的调查员都具备物理学背景。实际上,海森堡曾参与其中一位调查员在莱比锡大学的博士考试。三人中最有影响力的是约翰内斯·朱尔夫。在调查过程中,他们转而支持海森堡,并支持他反对德国物理学运动在理论物理学和学术界的意识形态政策。”
\subsection{德国核武器计划}    
\subsubsection{战前的物理学工作}  
1936年中期,海森堡在两篇论文中提出了他的宇宙射线暴理论。[73] 接下来两年里,又发表了四篇论文[74][75][76][77]。[32][78]

1938年12月,德国化学家奥托·哈恩和弗里茨·斯特拉斯曼向《自然科学》杂志提交了一篇稿件,报告称他们在用中子轰击铀时发现了元素钡,这使得哈恩得出铀核爆裂的结论;[79] 同时,哈恩将这些结果传达给了他的朋友莉泽·迈特纳(Lise Meitner),她在同年7月已逃亡到荷兰,随后又去了瑞典。[80] 迈特纳和她的侄子奥托·罗伯特·弗里施正确地将哈恩和斯特拉斯曼的结果解释为核裂变。[81] 弗里施于1939年1月13日进行了实验确认。[82]

1939年6月和7月,海森堡前往美国,拜访了密歇根大学的塞缪尔·亚伯拉罕·古德斯密特(Samuel Abraham Goudsmit)。然而,海森堡拒绝了移民美国的邀请。直到六年后,二战结束时,海森堡才再次见到古德斯密特,当时古德斯密特是美国阿尔索斯行动(Operation Alsos)的首席科学顾问。[32][83][84]