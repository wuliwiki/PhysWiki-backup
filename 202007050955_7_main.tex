\input{./others/format}

\begin{document}
\renewcommand{\thelstlisting}{\arabic{lstlisting}}
\input{./others/MatlabStyle} % 设置 Matlab 颜色
\begin{titlepage}
\includepdf{./figures/frontcover.pdf}
\newpage
\includepdf{./figures/dedication.pdf}
\end{titlepage}
\frontmatter % 开始罗马数字页码
\input{./contents/FrontMatters} % 版权声明&关于本书
\setcounter{tocdepth}{1}
\tableofcontents % 生成目录
\mainmatter % 开始阿拉伯数字页码

\part{科普}
%=======================================
\chapter{经典力学}
%---------------------------------------
\entry{经典力学及其他物理理论}{MecThe}
\entry{经典力学}{CM0}
% \entry{动量和能量}{CM1} % 未完成
% \entry{角动量}{CM2} % 未完成

\chapter{电动力学}
%---------------------------------------
\entry{电动力学}{EM0}
\entry{静电的基本规律和性质}{EM1}
\entry{荷质比的测定}{Charge}

\chapter{量子力学}
%---------------------------------------
\entry{原子的观念}{AtomId}
\entry{从天球的音乐到玻尔模型}{ClBohr}
\entry{量子力学}{QM0}

\chapter{其他}
%---------------------------------------
\entry{天文学常识}{Astro}

\part{微积分}
%=======================================
\chapter{基础}
\entry{公理系统}{axioms}
\entry{集合}{Set}
\entry{整数}{intger}
\entry{映射}{map}
\entry{无穷的概念}{infty}
\entry{二项式定理}{BiNor}
\entry{二项式定理(非整数幂)}{BiNorR}
\entry{直线和平面的交点}{LPint}
\entry{等比数列}{GeoPrg}
\entry{三角恒等式}{TriEqv}
\entry{双曲函数}{TrigH}
\entry{sinc 函数}{sinc}
\entry{充分必要条件}{SufCnd}
\entry{四象限 Arctan 函数}{Arctan}
\entry{极坐标系}{Polar}
\entry{阿基米德螺线}{ArcSpl}
\entry{柱坐标系}{Cylin}
\entry{球坐标系}{Sph}
\entry{一元函数的对称与周期性}{shenry}
\entry{球坐标与直角坐标的转换}{SphCar}
\entry{圆锥曲线的极坐标方程}{Cone}
\entry{椭圆的三种定义}{Elips3}
\entry{双曲线的三种定义}{Hypb3}
\entry{抛物线的三种定义}{Para3}
\entry{圆锥曲线的光学性质}{ConOpt}
\entry{复数}{CplxNo}
\entry{复变函数}{Cplx}
\entry{幂函数(复数)}{CPow}
\entry{指数函数(复数)}{CExp}
\entry{三角函数(复数)}{CTrig}
\entry{复变函数的导数 柯西—黎曼条件}{CauRie}
\entry{复变函数的积分}{CpxInt}
\entry{柯西积分定理}{CauGou}
\entry{立体角}{SolAng}

\chapter{一元微积分}
%---------------------------------------
\entry{微积分导航}{Calc}
\entry{极限}{Lim}
\entry{小角正弦极限}{LimArc}
\entry{自然对数底}{E}
\entry{切线与割线}{TanL}
\entry{函数的连续性}{contin}
\entry{导数}{Der}
\entry{求导法则}{DerRul}
\entry{反函数求导}{InvDer}
\entry{基本初等函数的导数}{FunDer}
\entry{高阶导数}{HigDer}
\entry{莱布尼兹公式}{LeiEqu}
\entry{导数与函数极值}{DerMax}
\entry{用极值点确定函数图像}{DerImg}
\entry{一元函数的微分}{Diff}
\entry{复合函数求导\ 链式法则}{ChainR}
\entry{泰勒展开}{Taylor}
\entry{导数与差分}{DerDif}
\entry{不定积分}{Int}
\entry{换元积分法}{IntCV}
\entry{分部积分法}{IntBP}
\entry{积分表}{ITable}
\entry{定积分}{DefInt}
\entry{牛顿—莱布尼兹公式}{NLeib}
\entry{极坐标中的曲线方程}{PolCrd}
\entry{函数的算符}{DifOp}
\entry{常微分方程}{ODE}
\entry{一阶线性微分方程}{ODE1}
\entry{二阶常系数齐次微分方程}{Ode2}
\entry{一维齐次亥姆霍兹方程}{HmhzEq}
\entry{二阶常系数非齐次微分方程}{Ode2N}
\entry{正交函数系}{Fbasis}
\entry{傅里叶级数(三角)}{FSTri}
\entry{傅里叶级数(指数)}{FSExp}
\entry{狄拉克 delta 函数}{Delta}
\entry{傅里叶变换(三角)}{FTTri}
\entry{傅里叶变换(指数)}{FTExp}
\entry{拉普拉斯变换}{LapTra}
\entry{拉普拉斯变换的性质}{ProLap}
\entry{Gamma 函数}{Gamma}

\chapter{多元微积分与矢量分析}
%---------------------------------------
\entry{偏导数}{ParDer}
\entry{最小二乘法}{LstSqr}
\entry{全微分}{TDiff}
\entry{复合函数的偏导\ 链式法则}{PChain}
\entry{全导数}{TotDer}
\entry{矢量的导数\ 求导法则}{DerV}
\entry{偏微分算符}{ParOp}
\entry{一元矢量函数的积分}{IntV}
\entry{方向导数}{DerDir}
\entry{二元函数的极值}{F2Exm}
\entry{重积分}{IntN}
\entry{极坐标系中单位矢量的偏导}{DPol1}
\entry{正交曲线坐标系}{CurCor}
\entry{多元函数的傅里叶级数}{NdFuri}
\entry{曲线坐标系中的重积分}{CrIntN}
\entry{矢量场}{Vfield}
\entry{线积分}{IntL}
\entry{曲面积分\ 通量}{SurInt}
\entry{证明闭合曲面的法向量面积分为零}{CSI0}
\entry{矢量算符}{VecOp}
\entry{拉普拉斯算符}{Laplac}
\entry{一种矢量算符的运算方法}{MyNab}
\entry{梯度\ 梯度定理}{Grad}
\entry{散度\ 散度定理}{Divgnc}
\entry{旋度\ 斯托克斯定理}{Curl}
\entry{矢量分析总结}{VecAnl}
\entry{拉格朗日乘数法}{LagMul}
\entry{一阶线性常微分方程组}{ODEsys}
\entry{齐次函数的欧拉定理}{Homeul}
\entry{多元泰勒展开}{NDtalr}
\entry{雅可比行列式}{JcbDet}
\entry{高斯积分}{GsInt}
\entry{多维球体的体积}{NSphV}
\entry{多元函数积分和宇称}{IntPry}

\part{线性代数}
\chapter{线性代数 1}
%---------------------------------------
\entry{线性代数导航}{Vector}
\entry{几何矢量}{GVec}
\entry{矢量内积}{Dot}
\entry{正交归一基底}{OrNrB}
\entry{施密特正交归一化}{SmdtOt}
\entry{右手定则}{RHRul}
\entry{矩阵}{Mat}
\entry{行列式}{Deter}
\entry{矢量叉乘}{Cross}
\entry{矢量叉乘分配律的几何证明}{CrossP}
\entry{叉乘的矩阵形式}{CrosMt}
\entry{连续叉乘的化简}{TriCro}
\entry{高斯消元法解线性方程组}{GAUSS}
\entry{线性相关\ 线性无关}{LinDep}
\entry{克莱姆法则}{Cramer}
\entry{三矢量的混合积}{TriVM}
\entry{平面旋转变换}{Rot2DT}
\entry{线性变换}{LTrans}
\entry{逆矩阵}{InvMat}
\entry{平面旋转矩阵}{Rot2D}
\entry{三维旋转矩阵}{Rot3D}
\entry{过渡矩阵}{TransM}
\entry{绕轴旋转矩阵}{RotA}
\entry{旋转矩阵的导数}{RotDer}
\entry{厄米矩阵}{HerMat}
\entry{矩阵的本征方程}{MatEig}
\entry{对称矩阵的本征问题}{SymEig}
\entry{厄米矩阵的本征问题}{HerEig}
\entry{相似变换和相似矩阵}{MatSim}
\entry{矩阵的迹}{trace}

\chapter{线性代数 2}
%---------------------------------------
\entry{矢量空间}{LSpace}
\entry{矢量空间的表示}{VecRep}
\entry{内积}{InerPd}
\entry{狄拉克符号}{braket}
\entry{线性相关\ 线性无关}{LinInd}
\entry{子空间}{SubSpc}
\entry{直和}{DirSum}
\entry{对易算符}{Commu}
\entry{代数矢量}{NumVec}
\entry{矩阵与线性映射}{MatLS} % 未完成:把 MatSpc 合并到这里
\entry{正交子空间}{OrthSp}
\entry{矩阵的秩}{MatRnk}
\entry{证明矩阵行秩等于列秩}{RCrank}
% 未完成: 线性变换与矢量空间
\entry{线性方程组与矢量空间}{LinEq}
\entry{酋矩阵}{UniMat}
\entry{超定线性方程组}{OvrDet}
\entry{傅里叶变换与矢量空间}{FTvec}
\entry{海森矩阵}{Hesian}
\entry{张量积空间}{DirPro}
\entry{四元数与旋转矩阵}{QuatN}

\chapter{张量分析}
%-------------------------------------------
\entry{张量}{Tensor}

\part{抽象代数基础}
%=======================================
\chapter{群论}
%---------------------------------------
\entry{群}{Group}
\entry{子群}{Group1}
\entry{陪集和同余}{coset}
\entry{正规子群}{NormSG}
\entry{群同态}{Group2}
\entry{群作用}{Group3}
\entry{商空间}{QuoSpa}
\entry{群的表示}{GrpRep}
\entry{群论中的证明和习题解答}{GroupP}

\chapter{环论}
%---------------------------------------
\entry{环}{Ring}

\chapter{域论}
\entry{域}{field}
\entry{四元数}{Quat}
\entry{域的扩张}{FldExp}
\entry{域上的代数}{AlgFie}


\part{拓扑学}
%=====================================
\chapter{点集拓扑}
%-------------------------------------
\entry{拓扑空间}{Topol}
\entry{点集的内部、外部和边界}{Topo0}
\entry{连续映射和同胚}{Topo1}
\entry{紧致性}{Topo2}
\entry{连通性}{Topo3}
\entry{道路连通性}{Topo4}
\entry{分离性}{Topo5}
\entry{乘积拓扑}{Topo6}
\entry{商拓扑}{Topo7}
\entry{映射空间}{Topo8}
\entry{空间偶和带基点空间}{Topo9}

\chapter{同伦论}
%-------------------------------------
\entry{映射的同伦和空间的同伦}{HomT1}
\entry{可缩空间}{HomT2}


% \part{李群和李代数} 
%======================================

\part{概率与统计}
%=======================================
\chapter{基础}
\entry{随机变量\ 概率分布函数}{RandF}
\entry{随机变量的变换}{RandCV}
\entry{多变量分布函数}{MulPdf}
\entry{高斯分布(正态分布)}{GausPD}
\entry{二项分布}{BiDist}
\entry{中心极限定理}{CLT}
\entry{二维随机走动}{RW2D}
\entry{平均值的不确定度}{MeanS}

\part{偏微分方程和特殊函数}
%=======================================
\chapter{基础}
%---------------------------------------
\entry{分离变量法}{SepVar}
\entry{拉普拉斯方程}{LapEq}
\entry{柱坐标系中的拉普拉斯方程}{CylLap}
\entry{球坐标系中的梯度散度旋度及拉普拉斯算符}{SphNab}
\entry{球坐标系中的拉普拉斯方程}{SphLap}
\entry{泊松方程}{PoiEqu}
\entry{球坐标系中的亥姆霍兹方程}{SphHHz}
\entry{勒让德多项式}{Legen}
\entry{连带勒让德多项式}{AsLgdr}
\entry{贝赛尔函数}{Bessel}
\entry{球贝塞尔函数}{SphBsl}
\entry{球谐函数}{SphHar}
\entry{球谐函数表}{YlmTab}
\entry{Wigner D 矩阵}{WigDmt}
\entry{平面波的正交归一}{PwOrNr}
\entry{平面波的球谐展开}{Pl2Ylm}
\entry{球谐波的归一化}{FrNorm}
\entry{分离变量法与张量积空间}{SVarDP}
\entry{广义球谐函数}{GenYlm}
\entry{误差函数}{Erf}
\entry{虚误差函数}{Erfi}
\entry{超几何函数}{HypGeo}
\entry{椭圆积分}{EliInt}
\entry{库仑函数}{CulmF}

\part{其他数学}
%=======================================
\chapter{数学分析}
%---------------------------------------
\entry{集合的极限}{SetLim}
\entry{度量空间}{Metric}
\entry{范数}{NormV}
\entry{投影算符}{projOp}
\entry{柯西—施瓦茨不等式}{CSNeq}
\entry{巴拿赫空间}{banach}
\entry{希尔伯特空间}{Hilber}
\entry{一致收敛}{UniCnv}
\entry{数学分析笔记}{AnalNt}
\entry{黎曼积分与勒贝格积分}{Rieman}
\entry{泛函分析笔记1}{FnalNt}
\entry{泛函分析笔记2}{FnalN2}
\entry{泛函分析笔记3}{FnalN3}
\entry{泛函分析笔记4}{FnalN4}
\entry{泛函分析笔记5}{FnalN5}


\chapter{其他}
%---------------------------------------
\entry{堆放排列组合}{StackC}
\entry{选择的展开定理}{ChExpn}
\entry{解三棱锥顶角}{PrmSol}
\entry{足球顶点坐标的计算}{FootBl}
\entry{CG 系数}{SphCup}
\entry{Wigner 3j 符号}{ThreeJ}
\entry{Wigner 6j 符号}{SixJ}
\entry{Wigner 9j 符号}{NineJ}
\entry{二元关系}{Relat}
\entry{流形}{Manif}
\entry{群论笔记}{GroupN}
\entry{范畴论}{Cat}
\entry{余元公式}{Gama1}

\part{力学}
%=======================================
\chapter{运动学}
%---------------------------------------
\entry{物理量和单位转换}{Units}
\entry{无单位的物理公式}{NoUnit}
\entry{位置矢量\ 位移}{Disp}
\entry{速度\ 加速度(一维)}{VnA1}
\entry{速度\ 加速度}{VnA}
\entry{圆周运动的速度}{CMVD}
\entry{圆周运动的加速度}{CMAD}
\entry{匀加速运动}{ConstA}
\entry{极坐标中的速度和加速度}{PolA}
\entry{速度的坐标系变换}{Vtrans}
\entry{加速度的坐标系变换}{AccTra}

\chapter{质点}
%---------------------------------------
\entry{牛顿运动定律\ 惯性系}{New3}
\entry{功\ 功率}{Fwork}
\entry{动能\ 动能定理(单个质点)}{KELaw1}
\entry{力场\ 势能}{V}
\entry{机械能守恒(单个质点)}{ECnst}
\entry{动量\ 动量定理(单个质点)}{PLaw1}
\entry{角动量\ 角动量定理\ 角动量守恒(单个质点)}{AMLaw1}
\entry{简谐振子}{SHO}
\entry{受阻落体}{RFall}
\entry{单摆}{Pend}
\entry{傅科摆}{Fouclt}
\entry{惯性力}{Iner}
\entry{离心力}{Centri}
\entry{科里奥利力}{Corio}
\entry{地球表面的科里奥利力}{ErthCf}

\chapter{质点系与刚体}
%---------------------------------------
\entry{常见物理量}{PhyQty}
\entry{国际单位制}{Consts}
\entry{自由度}{DoF}
\entry{质点系}{PSys}
\entry{质心\ 质心系}{CM}
\entry{质点系的动量}{SysMom}
\entry{刚体}{RigBd}
\entry{轻杆模型}{rod}
\entry{动量定理\ 动量守恒}{PLaw}
\entry{质点系的动能\ 柯尼希定理}{Konig}
\entry{力矩}{Torque}
\entry{刚体的静力平衡}{RBSt}
\entry{角动量}{AngMom}
\entry{角动量定理\ 角动量守恒}{AMLaw}
\entry{二体系统}{TwoBD}
\entry{二体碰撞}{TwoCld}
\entry{刚体的绕轴转动\ 转动惯量}{RigRot}
\entry{平行轴定理与垂直轴定理}{MIthm}
\entry{常见几何体的转动惯量}{ExMI}
\entry{刚体的平面运动方程}{RBEM}
\entry{惯性张量}{ITensr}
\entry{瞬时转轴}{InsAx}
\entry{刚体绕轴转动 2}{RBrot2}
\entry{刚体的动能定理}{RBKE}
\entry{刚体的运动方程}{RBEqM}
\entry{刚体运动方程(四元数)}{RBEMQt}

\chapter{软体和液体}
%---------------------------------------
\entry{绳的法向压力}{RopeFP}
\entry{流密度}{CrnDen}
\entry{浮力}{Buoy}

\chapter{振动与波动}
%---------------------------------------
\entry{振动的指数形式}{VbExp}
\entry{能量法解谐振动问题}{EnerVi}
\entry{拍频}{beatno}
\entry{受阻简谐振子}{SHOf}
\entry{简谐振子的品质因数}{SHOq}
\entry{简谐振子受迫运动}{SHOfF}
\entry{平面波}{PWave}
\entry{多普勒效应(一维匀速)}{Dople1}
\entry{多普勒效应}{Dopler}
\entry{一维波动方程}{WEq1D}
% 未完成 边界条件 (两条密度不同的绳子)
\entry{二维波动方程}{Wv2D}
\entry{波的能量}{WaEner}
\entry{波的强度}{WaInte}
\entry{冲击波}{ShoWav}

\chapter{中心力场问题}
%---------------------------------------
\entry{万有引力\ 引力势能}{Gravty}
\entry{球体的引力场}{SphF}
\entry{中心力场问题}{CenFrc}
\entry{开普勒问题}{CelBd}
\entry{开普勒三定律}{Keple}
\entry{拉普拉斯—龙格—楞次矢量}{LRLvec}
\entry{轨道方程\ 比耐公式}{Binet}
\entry{开普勒第一定律的证明}{Keple1}
\entry{开普勒第二和第三定律的证明}{Keple2}
\entry{反开普勒问题}{InvKep}
\entry{散射}{Scater}
\entry{卢瑟福散射}{RuthSc}
\entry{闭合轨道的条件}{ClsOrb}

\chapter{分析力学}
%---------------------------------------
\entry{拉格朗日方程}{Lagrng}
\entry{单摆(大摆角)}{SinPen}
\entry{双摆和三摆}{Pendu3}
\entry{达朗贝尔定理}{dAlbt}
\entry{哈密顿原理}{HamPrn}
\entry{运动积分}{motint}
\entry{哈密顿正则方程}{HamCan}
\entry{分析力学笔记}{ClsMec}

\chapter{轨道力学}
%---------------------------------------
\entry{二体问题综述}{ConDB}
\entry{轨道参数 时间变量}{OribP}
\entry{限制性三体问题}{TriLim}
\entry{雅可比常量}{JConst}
\entry{拉格朗日点}{LPoint}


\part{光学}
%=======================================
\chapter{几何光学}
%---------------------------------------
\entry{惠更斯原理}{Huygen}
\entry{光的折射\ 斯涅尔定律}{Snel}
\entry{薄透镜}{ThnLen}

\chapter{波动光学}
%---------------------------------------
\entry{可见光谱}{VisSpt}
\entry{双缝干涉中一个重要极限}{SltLim}
\entry{杨氏双缝干涉实验}{Young}
\entry{劳埃德镜实验}{Lloyd}
\entry{普通光源的发光机理}{LumiMe}
\entry{单色光}{MonoLi}

\part{电动力学}
%=======================================
\chapter{基础}
%---------------------------------------
\entry{电流}{I}
\entry{电流密度}{Idens}
\entry{库仑定律}{ClbFrc}
\entry{电场}{Efield}
\entry{电势\ 电势能}{QEng}
\entry{电偶极子}{eleDpl}
\entry{电偶极子2}{eleDP2}
\entry{导体}{Cndctr}
\entry{电压}{Voltag}
\entry{电容}{Cpctor}
\entry{电介质}{Dielec}
\entry{电极化强度}{ElecPo}
\entry{电极化强度与极化电荷的关系}{ElePAP}
\entry{介质中的静电场}{EFIDE}
\entry{电阻 欧姆定律}{Resist}
\entry{电感}{Induct}
\entry{电场的高斯定理}{EGauss}
\entry{电场的能量}{EEng}
\entry{LC 振荡电路}{LC}
\entry{力电振动类比}{MeElec}
\entry{LC 受迫振荡电路}{EleRes}
\entry{比奥萨伐尔定律}{BioSav}
\entry{安培环路定理}{AmpLaw}
\entry{洛伦兹力}{Lorenz}
\entry{磁场的能量}{BEng}
\entry{磁通量}{BFlux}
\entry{安培力}{FAmp}
\entry{磁场中闭合电流的合力}{EBLoop}
\entry{磁场中闭合电流的力矩}{EBTorq}
\entry{法拉第电磁感应定律}{FaraEB}
\entry{分子电流和分子磁矩}{MoMaMo}
\entry{磁介质}{MagMat}
\entry{磁化强度}{MaInte}
\entry{顺磁质的磁化}{ParaMa}
\entry{有磁介质时的安培环路定理}{MaAmpe}
\entry{电路}{Circ}
\entry{基尔霍夫电路定律}{Kirch}
\entry{惠斯通电桥}{WheBrg}

\chapter{电动力学 2}
%---------------------------------------
\entry{电荷守恒\ 电流连续性方程}{ChgCsv}
\entry{电多极展开}{EMulPo}
\entry{电磁场标势和矢势}{EMPot}
\entry{磁标势}{elecdy}
\entry{磁多极矩}{edy33}
\entry{规范变换}{Gauge}
\entry{洛伦兹规范}{LoGaug}
\entry{库仑规范}{Cgauge}
\entry{电磁场的能量守恒\ 坡印廷矢量}{EBS}
\entry{麦克斯韦方程组}{MWEq}
\entry{麦克斯韦方程组(介质)}{MWEq1}
\entry{非齐次亥姆霍兹方程\ 推迟势}{RetPot}
\entry{电场波动方程}{EWEq}
\entry{真空中的平面电磁波}{VcPlWv}
\entry{介质中的波动方程}{MedWF}
\entry{菲涅尔公式}{Fresnl}
\entry{盒中的电磁波}{EBBox}
\entry{电磁场的动量守恒\ 动量流密度张量}{EBP}
\entry{磁旋比\ 玻尔磁子}{BohMag}

\chapter{电动力学 3}
%---------------------------------------
\entry{电磁场的参考系变换}{EMRef}
\entry{拉格朗日电磁势}{EMLagP}
\entry{电磁场角动量分解}{EMAMSp}

\part{相对论}
%=======================================
\chapter{狭义相对论}
%---------------------------------------
\entry{狭义相对论的基本假设}{SpeRel}
\entry{事件与尺缩效应}{SRsmt}
\entry{时间的变换与钟慢效应}{SRtime}
\entry{洛伦兹变换}{SRLrtz}
\entry{斜坐标系}{ObSys}
\entry{斜坐标系表示洛伦兹变换}{SROb}
\entry{约化光速}{SRc}
\entry{洛伦兹变换的代数推导}{LornzT}
\entry{相对论速度变换}{RelVel}
\entry{相对论加速度变换}{SRAcc}
\entry{时空的四维表示}{SR4Rep}
\entry{闵可夫斯基空间}{MinSpa}
\entry{光的多普勒效应}{RelDop}
\entry{爱因斯坦求和约定}{EinSum}
\entry{洛伦兹群}{qed1}
\entry{相对论动力学}{SRDyn}
\entry{相对论流体}{SRFld}

\part{量子力学}
%=======================================
% 未完成 讲讲为什么定态波函数一定是实数函数? 实数波函数为什么会有动量平均值为零?
% 未完成 x,p 不确定原理,一般的不确定原理, 高斯波包时取等号.
% 未完成 束缚态的一般性质: 节点数, 对称性 (偶势能的基态是偶函数), 简并性(一维情况不简并)

\chapter{量子力学 1}
%---------------------------------------
\entry{玻尔原子模型}{BohrMd}
\entry{原子单位}{AU}
\entry{量子力学与矩阵}{QMmat}
\entry{算符和本征问题}{QM1}
\entry{定态薛定谔方程}{SchEq}
\entry{薛定谔方程}{TDSE}
\entry{不确定原理}{Uncert}
\entry{多维空间中的量子力学}{QMndim}
\entry{高斯波包}{GausWP}
\entry{无限深势阱}{ISW}
\entry{升降算符}{RLop}
\entry{简谐振子(升降算符)}{QSHOop}
\entry{简谐振子升降算符归一化}{QSHOnr}
\entry{简谐振子(级数)}{QSHOxn}
\entry{有限深球势阱}{FiSph}
\entry{一维 delta 势能散射}{Dsc1D}
\entry{方势垒}{SqrPot}
\entry{拉比频率}{RabiF}
\entry{二维无限深势阱}{ISW2D}
\entry{平均值(量子力学)}{QMavg}
\entry{守恒量(量子力学)}{QMcons}
\entry{薛定谔方程的分离变量法}{SEsep}
\entry{对易厄米算符与共同本征矢}{Commut}
\entry{类氢原子的约化质量}{HRMass}
\entry{概率流密度}{PrbJ}
\entry{算符的矩阵表示}{OpMat}
\entry{轨道角动量}{QOrbAM}
\entry{轨道角动量升降算符归一化}{QLNorm}
\entry{自旋角动量}{Spin}
\entry{自旋角动量矩阵}{spinMt}

\chapter{量子力学 2}
%---------------------------------------
\entry{算符的指数函数\ 波函数传播子}{OpExp}
\entry{平移算符}{tranOp}
\entry{旋转算符}{rotOp}
\entry{角动量加法(量子力学)}{AMAdd}
\entry{能量归一化}{EngNor}
\entry{氢原子基态的波函数}{HWF0}
\entry{球坐标和柱坐标中的定态薛定谔方程}{RadSE}
\entry{球坐标中的薛定谔方程}{RYTDSE}
\entry{量子力学中的变分法}{QMVar}
\entry{含时微扰理论}{TDPT}
\entry{几种含时微扰}{TDPEx}
\entry{含连续态的微扰理论}{PTCont}
\entry{量子散射}{ParWav}
\entry{波恩近似(散射)}{BornSc}
\entry{质心系中的多粒子问题}{SECM}
\entry{三维简谐振子(球坐标)}{SHOSph}
\entry{球面散射态与平面散射态的转换}{Scatt2}
\entry{库仑波函数}{CulmWf}
\entry{量子力学的基本假设}{QMPos}
\entry{共振}{ResoN}
\entry{带电粒子的薛定谔方程}{EMTDSE}
\entry{电磁场中的单粒子薛定谔方程}{QMEM}
\entry{电磁场中的类氢原子}{EMHydr}
\entry{氢原子电离截面}{HionCr}
\entry{Volkov 波函数}{Volkov}
\entry{氢原子的选择定则}{SelRul}
\entry{康普顿散射}{Comptn}
\entry{全同粒子}{IdPar}
\entry{密度矩阵}{denMat}
\entry{Hartree-Fork 方法}{HarFor}
\entry{多通道散射}{MulSct}
\entry{Adiabatic 笔记}{Adibat}
\entry{量子力学中的数学笔记}{QMmath}
\entry{晶体衍射}{CrysDf}

\chapter{量子力学与量子场论}
%---------------------------------------
\entry{前言}{QFIntro}
\entry{基本概念}{Basics}
\entry{全同粒子的统计}{IdParS}
\entry{近似理论:微扰}{AprPtr}
\entry{角动量}{QMAM}
% 第六章直接从 sakurai 翻译, 后面的一些内容非常零碎跳过
\entry{冷原子基本知识}{UCBas}
\entry{两个原子间的相互作用}{TwoAtF}
\entry{Feshbach 共振}{FeshRs}
\entry{BCS-BEC Crossover 的平均场描述}{BCSBEC}
\entry{BEC 超流}{BECSup}

\part{弦理论}
%=======================================
\chapter{基础}
%---------------------------------------
\entry{引力量子化}{QuanGR}
% \chapter{经典弦 1}
%---------------------------------------
% \chapter{经典弦 2}
%---------------------------------------
% \chapter{弦量子化}
%---------------------------------------
% \chapter{共形场论 1}
%---------------------------------------
% \chapter{BRST 量子化}
%---------------------------------------
% \chapter{RNS 超弦}
%---------------------------------------
% \chapter{紧致化和T对偶}
%---------------------------------------
% \chapter{超弦理论 续}
%---------------------------------------
% \chapter{超弦理论 总结}
%---------------------------------------
% \chapter{第II型弦论}
%---------------------------------------
% \chapter{Heterotic 弦论}
%---------------------------------------
% \chapter{D-膜}
%---------------------------------------
% \chapter{黑洞}
%---------------------------------------
% \chapter{全息原理和AdS/CFT}
%---------------------------------------
% \chapter{弦理论和宇宙学}


\part{原子分子物理}
%=======================================
\chapter{原子}
%---------------------------------------
\entry{类氢原子的定态波函数}{HWF}
\entry{氢原子波函数分析}{Hanaly}
\entry{电子轨道与元素周期表}{Ptable}
\entry{原子符号}{TrmSym}
\entry{Keldysh 参数}{keldis}
\entry{FROG}{Frog}
\entry{Frog-Crab}{FrogCr}

\part{热力学和统计力学}
%=======================================
\chapter{热力学}
%---------------------------------------
\entry{气体分子对容器壁的压强}{MolPre}
\entry{理想气体状态方程}{PVnRT}
\entry{温度\ 温标}{tmp}
\entry{理想气体的内能}{IdgEng}
\entry{压强体积图}{PVgraf}
\entry{热平衡\ 热力学第零定律}{TherEq}
\entry{热传导定律}{Heatco}
\entry{热力学第一定律}{Th1Law}
\entry{等压过程}{EqPre}
\entry{等体过程}{EqVol}
\entry{等温过程}{EqTemp}
\entry{热容量}{ThCapa}
\entry{绝热过程}{Adiab}
\entry{熵}{Entrop}
\entry{准静态过程}{Quasta}
\entry{卡诺热机}{Carnot}
\entry{分子平均碰壁数}{AvgHit}
\entry{气体分子的速度分布}{VelPdf}
\entry{麦克斯韦—玻尔兹曼分布}{MxwBzm}
\entry{理想气体分压定律}{PartiP}
\entry{饱和蒸汽压}{VaporP}

\chapter{统计力学}
%---------------------------------------
\entry{统计力学公式大全}{StatEq}
\entry{相空间}{PhSpace}
\entry{理想气体的状态密度(相空间)}{IdSDp}
\entry{理想气体单粒子能级密度}{IdED1}
\entry{理想气体(微正则系综法)}{IdNCE}
\entry{正则系宗法}{CEsb}
\entry{理想气体(正则系宗法)}{IdCE}
\entry{理想气体(巨正则系综法)}{IdMCE}
\entry{等间隔能级系统(正则系宗)}{EqCE}
\entry{巨正则系综法}{MCEsb}
\entry{量子气体(单能级巨正则系综法)}{QGs1ME}
\entry{量子气体(巨正则系宗)}{QGsME}
\entry{光子气体}{PhoGas}

\part{计算物理}
%=======================================
\chapter{Matlab 简介}
% 未完成 Mathematica
% 未完成 Wolfram Alpha
%---------------------------------------
\entry{计算物理导航}{NumPhy}
\entry{Matlab 简介}{Matlab}
\entry{Matlab 的变量与矩阵}{MatVar}
\entry{Matlab 的判断与循环}{MIfFor}
\entry{Matlab 的函数}{MatFun}
\entry{Matlab 画图}{MatPlt}
\entry{Matlab 的程序调试及其他功能}{MatOtr}

\chapter{Python 简介}
%---------------------------------------
\entry{Python 简介}{Python}
\entry{Python 基本变量类型}{PyType}
\entry{Python 数据类型}{PyData}
\entry{Numpy 库}{numpy}
\entry{Python 字符串处理}{PyStr}
\entry{Python 文件读写}{PyFile}
\entry{Python 判断与循环}{PyIfFr}
\entry{Python 函数}{PyFunc}
\entry{Python 画图}{PyPlot}
\entry{Python 的类}{PyClas}
\entry{Python 模块}{PyMod}
\entry{SciPy 数值微分与积分}{SciPy}
\entry{SciPy 最小二乘法}{PyFit}

\chapter{Linux 简介}
%---------------------------------------
\entry{Linux 基础}{Linux}
\entry{在 Linux 上编译 C/C++ 程序}{linCpp}
\entry{Makefile 简介}{Make0}
\entry{Makefile 笔记}{Make}

\chapter{C++ 简介}
%---------------------------------------
\entry{C++ 基础}{Cpp0}
\entry{C++ 的整数}{cppInt}
\entry{C/C++ 多文件编译笔记}{cppFil}
\entry{调试 C++ 程序}{gdbcpp}
% \entry{SLISC0 库简介}{SLISC}
\entry{BLAS 简介}{BLAS}
\entry{矩阵的数据结构}{MatSto}
\entry{带对角矩阵}{BanDmt}
\entry{GNU Scientific Library}{GSL}
\entry{简单的矢量和矩阵类}{SLSvec}
\entry{C++ 中的 SFINAE 技巧}{SFINAE}

\chapter{其他}
%---------------------------------------
\entry{文本文件与字符编码}{encode}
\entry{LaTeX 结构简介}{latxIn}
\entry{GitHub Desktop 的简单使用}{GitHub}
\entry{cuBLAS 库}{cublas}
% \entry{用百度网盘安全高效地备份文件}{PanBak}

\chapter{数值验证及常用算法}
%---------------------------------------
\entry{二项式定理(非整数幂)的数值验证}{BiNorM}
\entry{二分法}{Bisec}
\entry{多区间二分法}{MBisec}
\entry{冒泡法}{Bubble}
\entry{高斯消元法程序}{GauEli}
\entry{Nelder-Mead 算法}{NelMea}
\entry{最小二乘法的数值计算}{CurFit}
\entry{数值积分(梯形法)}{NumInt}
\entry{稀疏矩阵}{SprMat}
\entry{函数求值}{SpcFun}
\entry{离散傅里叶变换}{DFT}
\entry{离散正弦变换}{DST}

\chapter{计算机图形学}
%---------------------------------------
\entry{图像坐标系}{imgFrm}
\entry{三维投影}{proj3D}
\entry{相机模型}{CamMdl}
\entry{由图像坐标计算射线}{mn2lin}
\entry{计算 3D 艺术画}{art3D}
\entry{相机的定位}{CamLoc}

\chapter{微分方程数值解}
%---------------------------------------
\entry{简谐振子受迫运动的简单数值计算}{SHOFN}
\entry{天体运动的简单数值计算}{KPNum0}
\entry{常微分方程(组)的数值解}{OdeNum}
\entry{中点法解常微分方程(组)}{OdeMid}
\entry{四阶龙格库塔法}{OdeRK4}
\entry{刚体转动数值模拟}{RBRNum}

\chapter{偏微分方程数值解}
%---------------------------------------
\entry{一维波动方程的数值解}{W1dNum}

\chapter{一维薛定谔方程数值解}
%---------------------------------------
\entry{一维势能束缚态数值解(试射法)}{BndSho}
\entry{无限深势阱中的高斯波包}{wvISW}

\chapter{氢原子薛定谔方程数值解}
%---------------------------------------
\entry{球谐函数数值计算}{YlmNum}
\entry{Crank-Nicolson 算法(一维)}{CraNic}
\entry{Gauss-Lobatto 积分}{GLquad}
\entry{FEDVR 算法}{FEDVR}
\entry{Lanczos 算法}{Lanc}
\entry{指数格点}{ExpGrd}
\entry{虚时间法求基态波函数}{ImgT}
\entry{氢原子薛定谔方程数值解}{HyTDSE}
\entry{氢原子球坐标数值解 TDSE}{HTDSE}

\chapter{氦原子薛定谔方程数值解}
%---------------------------------------
\entry{氦原子数值解 TDSE 笔记}{HeTDSE}
\entry{氦原子波函数数值分析}{HeAnal}

\end{document}
