% 热学初步(高中)
% license Xiao
% type Tutor

% 分子动理论|气体等x定律|固体液体|热力学定律
% 缩减一部分,把第二章的前两小节合并,第二章整体作为一个新的小节
% 或者直接拆分成分子动力学和热力学初步算了,麻烦
% 第二章和第三章作为热力学初步内容

\pentry{功和机械能\nref{nod_HSPM07},分子动力学\nref{nod_thermo} }{nod_55f4}

\subsection{温度和温标}
\subsubsection{状态参量与平衡态}
以研究容器中气体的热学性质为例,我们的研究对象是一个由大量分子所组成的系统,称之为热力学系统,简称\textbf{系统}。在我们的经验常识中,一小罐热的气体会在室温下逐渐变凉,最终和室温温度保持一致,这个过程中,这一小罐热的气体即是系统,而系统之外与之产生相互作用的其他物体统称为\textbf{外界}。外界影响系统,导致系统的某些物理量发生变化。在这个例子中,热空气温度逐渐下降,最终和外界空气保持一致,热空气(系统)的状态也随之改变。在热学中,为了确定系统的状态,需要用一些物理量来进行描述,这些物理量叫做系统的状态参量,比如体积$V$是描述系统空间范围的几何参量,压强$p$是描述系统之间或内部力的作用的力学参量,温度$T$是确定系统冷热程度的热学参量……

但是,往往在任意时刻确定系统的状态是困难的,只有在系统处于\textbf{平衡态},也即系统经过了足够长的时间的演化,内部的各个部分的状态参量达到稳定状态不再改变时,我们才可以比较准确的描述系统的状态。

\subsubsection{温度和热平衡}
在上述的例子中,可以发现温度逐渐下降,说明系统的热学性质发生了改变,并在足够长的时间之后罐中空气温度和外界保持一致。在这个过程中,系统和外界相互接触,并发生了热传导,最终各自的状态参量不再发生改变,达到了平衡状态,这种平衡叫做\textbf{热平衡}。如果两个系统之间处于热平衡,那它们必然可以被某一相等的状态参量所描述,这里的状态参量则是\textbf{温度}。换言之,温度是决定一个系统是否处于热平衡的物理量。实验表明,如果两个系统分别与第三个系统达到热平衡,则两个系统之间一定也是处于热平衡的,因为它们都具有相同的温度,这个结果被称为\textbf{热平衡定律}。

\subsubsection{温度计}
为了测量温度,人们发明了温度计。首先,温度计需要有一种测温物质,这种测温物质的某些物理性质会随着温度的改变而发生改变,比如说水银的热胀冷缩可以制成水银温度计,气体的压强随温度变化可以制成气体温度计,电阻随温度变化可以制成电阻温度计,由不同金属随温度升高膨胀程度不同制成的双金属温度计……需要注意的是,每一种温度计都有其工作范围,在工作范围之外原来的变化关系未必成立,不能够继续测量温度。为了使用方便,人们尽可能的采用具有线性变化的物质来制作温度计。另一方面,人们需要定义该特性和温度之间的对应关系,每一种不同的定义方式对应于不同的\textbf{温标},同时,还需要定义温度的零点和分度方法。例如,我们常见的摄氏度定义一个标准大气压下冰水混合物的温度为$0^\circ \mathrm{C}$,水沸腾的温度为$100^\circ \mathrm{C}$,并将其间的刻度平均分成$100$份,每份为$1^\circ \mathrm{C}$。

但是当我们进行对热力学的学习时,我们将采用热力学温标,对应的就是\textbf{热力学温度},单位是\textbf{开尔文},简称\textbf{开},符号是$\mathrm{K}$。摄氏温度$t$和热力学温度$T$之间的换算关系由国际计量大会所确定:
$$T=t+273.15\mathrm{K}~.$$
可以发现,摄氏温度和热力学温度的温度差都是相同的,也即$1^\circ \mathrm{C}=1\mathrm{K}$。

\subsection{气体的等温变化}
为了探索气体的热学性质,首先先考察一种特殊的情况。在温度不变的条件下,一定质量的气体,它的压强和体积将满足什么样的变化关系呢?这种变化被称之为气体的\textbf{等温变化}。英国科学家玻意耳和法国科学家马洛特各自通过实验发现,一定质量的气体,在温度不变的情况下压强$p$与体积$V$成反比,即
$$p\propto\dfrac{1}{V}~$$
稍作整理可以得到
$$pV=C~.$$
其中$C$是常数,该定律也称\textbf{玻意耳定律}。但是实际应用中我们并不是特别关系常数$C$的具体取值,而更关系在初末状态时系统状态参量的关系,因此上式可以继续整理得到
$$p_1V_1=p_2V_2~,$$
其中$p_1$、$V_1$和$p_2$、$V_2$可以分别看成初末状态下系统的压强和体积。

\subsection{气体的等压变化}
同理,我们也将研究气体在压强不变的情况下,气体体积和温度之间的变化关系,这种变化叫气体的\textbf{等压变化}。实验表明,在压强不变时,一定质量的某种气体的体积随温度$T$线性变化,当温度采用热力学温度$T$时,所绘制的$V-T$图像中,等压线是一条过原点的直线。

法国科学家盖-吕萨克首先发现了这一线性关系,并将其表述为:一定质量的某种气体,在压强不变的情况下,其体积$V$与热力学温度$T$成正比,即
$$V=CT~.$$
其中$C$是常量,称之为\textbf{盖-吕萨克定律}。同上所述,我们更关心体系初末状态之间的关系,即有$$\dfrac{V_1}{T_1}=\dfrac{V_2}{T_2}~,$$
其中$V_1$、$T_1$和$V_2$、$T_2$可以分别看成初末状态下系统的体积和热力学温度。

\subsection{气体的等容变化} 
现在,我们只剩下体积不变这一个选项了。一定质量的气体,在体积不变的情况下,压强随着温度变化的过程叫做气体的\textbf{等容变化}。实验表明,在等容变化中,压强$p$和温度成正比,如果选用热力学温度$T$,则呈简单的正比例关系。当气体压强不太大、温度不太低时,可以发现$p-T$图像连线的延长线过原点,此时对应的就是热力学温度$T$等于$0$的状态,也称\textbf{绝对零度}。

法国科学家查理通过实验第一次发现这个定律,即一定质量的某种气体,在体积不变的情况下,压强$p$和热力学温度$T$成正比,即$p\propto T$,写成等式有
$$p=CT~.$$
其中$C$为常数,称之为\textbf{查理定律}。我们依旧更关心体系初末状态之间的关系,由此可以整理得到$$\dfrac{p_1}{T_1}=\dfrac{p_2}{T_2}~.$$
其中$p_1$、$T_1$和$p_2$、$T_2$可以分别看成初末状态下系统的压强和热力学温度。

\subsection{理想气体}
\subsubsection{理想气体假设}
事实上,我们上述得到的结论都需要满足理想气体的条件。\textbf{理想气体}指的是气体的分子大小和相互作用力可以忽略不计,也不计气体分子和容器壁碰撞时的动能损失。换句话说,理想气体中的分子可以看成质点,且彼此之间是完全独立的,不需要考虑我们先前介绍的分子间相互作用力\upref{thermo},且只发生弹性碰撞。现实中的气体,其分子具有大小,且具有相互作用力。但是在压强不太大(相比于一个大气压)、温度不太低(相比于室温)的前提下,分子大小远远小于分子间距,且分子间作用力很小,因此可以近似成理想气体处理。

当温度很低、压强很大时,由于气体分子大小和气体分子间相互作用的影响,我们不再可以使用玻意耳定律、盖-吕萨克定律以及查理定律进行处理,而是采用更为精确的近似方程\footnote{以后会在大学热学的学习中有更深入的说明。}。

\subsubsection{气体实验定律的微观解释}
 % 图2.3-3等,可选
 回顾分子动理论的\upref{thermo}内容,可以从微观的角度定性的解释气体实验定律。一定质量的某种气体,在温度不变时,气体的平均动能是一定的。这时候,如果减小体积,则单位时间单位面积内,碰撞到容器壁上的气体分子数目增多,而气体分子动量的改变量和之前相等,由此,压强便会增大,这对应的是玻意耳定律。

 同理,如果升高温度,则气体分子的平均动能更大,且由于分子速度变快,单位时间单位面积内和容器壁碰撞的分子数目也会更多,因此如果维持体积不变,压强就会增大,这就是查理定律的微观解释。如果想保持压强不变,只能通过降低气体分子数密度的方式来使得单位时间单位面积内和容器壁碰撞的分子数目减小,因此需要增加气体的体积,这对应的就是盖-吕萨克定律。

\subsection{固体和液体简介}
\subsubsection{固体}
生活中常见的物质状态除了气态以外,还有固态和液态。对于固体,我们可以将其分为\textbf{晶体}和\textbf{非晶体}。常见的晶体有石英、云母、食盐、硫酸铜等等,它们都具有规则的几何形状。
非晶体有玻璃、松香、沥青、橡胶等,它们通常没有规则的几何外形。

我们知道,晶体具有确定的熔点,而非晶体的熔化温度是不确定的,事实上,晶体和非晶体在物理性质上也有诸多不同。晶体往往具有\textbf{各向异性},比如说玻璃片在各个方向上导热性能相同,而云母片某一个轴的导热性能要大于另一个轴;有些晶体具有不同的导电特性;有些晶体各个方向的光学性质不同,比如说方解石是会形成双折射现象,可以将光分解为两束沿着不同方向折射的光。这些物理性质还包括弹性、硬度、磁性等等。而对于非晶体,它的物理性质往往是\textbf{各向同性}的。

晶体和非晶体物理性质的区别可以在微观上找到解释:通过X射线在晶体上的衍射实验,以及后续的电子显微镜和扫描隧道显微镜,人们逐渐观察到了物质表面上原子的排列方式。经由观察,人们发现,在各种晶体中,原子(或者分子、离子)都是规则排布的,具有空间上的周期性,这也就解释了为什么晶体具有规则的几何外形。以及由于排列规则,因此晶体不同方向上的原子(也可能是分子、离子)的数目和排布方式往往是不同的,因此会产生各向异性。至于非晶体,组成它们的原子(或者分子、离子)排布是随机的,由此,非晶体不具有规则的几何外形,而且各向同性。

需要注意的是,同一种原子有可能形成不同的晶体,比如说碳原子既可以构成石墨、也可以构成金刚石。同种原子乃至同种物质也未必一直都是晶体或者非晶体,比如说石英是一种晶体,但如果高温熔化之后再重新凝聚,则会变成一种非晶体——石英玻璃。

另外,我们也会发现某些物质具有确定的熔点,但是自然状态下不具有规则的几何外形,比如说金属,这类称之为\textbf{多晶体}。多晶体顾名思义,可以理解为由多个细小的晶粒随机排布所组成的物质,宏观上的随机排布使得它不具备规则的几何外形和各向异性,但是每一个晶粒小单元却是一个小的单晶体,具有规则的几何形状和各向异性。

% 将晶体、非晶体、多晶体的性质稍加总结,形成下表:
% 插不了合并行列吗
\subsubsection{液体}
液体具有表面张力,对于生活中的观察,我们可以发现,昆虫可以停在水面上,肥皂泡可以在空气中形成 球形,这些现象告诉我们液体是具有表面张力的。

由于液体表面的分子数密度较液体内部较小,在液体内部,分子间平均间距$r$略小于平衡位置$r_0$,分子表现为斥力,但是表面层的分子平均距离$r$略大于$r_0$,所以液体表面的分子之间呈现相互吸引力,这种力的方向总是和液面相切的,且与分界面垂直,叫做液体的\textbf{表面张力},表面张力让液体表面具有收缩趋势,这种收缩的趋势让液体表面绷紧。

当然,对于不同的界面,对于液体表面分子间距的影响是不同的。水滴可以在荷叶上形成近乎于完美的球形,但是滴在玻璃上却会散开并逐渐附着在玻璃表面上,这种区别可以定义浸润和不浸润。液体如果可以润湿某种固体并附着在固体的表面上,这种现象就叫做\textbf{浸润},与之相反,如果不会附着,那就是\textbf{不浸润}。浸润现象由于液体和固体表面分子之间的引力大于液体内部分子引力,所以分界面上的分子数密度将会大于液体内部的,分界面上的液体分子之间相互表现为排斥力。反之对于不浸润现象,固体分子对液体分子的引力小于液体分子内部的引力,就和我们讨论的表面张力现象一样,分界面上的分子数密度小于液体内部的分子数密度,因此界面上分子表现为相互吸引,液面会有收缩趋势。

生活中也可以在毛细现象中观察到浸润和不浸润的区别,水可以浸润玻璃,因此插入到水中的玻璃细管可以观察到水面上升,并且水面呈现出向下凹的形态;相反,由于水银不浸润玻璃,因此如果将玻璃细管插入到水银当中,玻璃管中的水银液面会下降并且向上鼓起。% 补充示意图
实验证明,如果管直径越小,则这种上升和下降的现象就越明显,这种现象称之为\textbf{毛细现象}。这在理论上也是很好理解的,当管直径越小,固液界面处的分子数目比之于细管内的液体数目就越少,界面处分子间作用力的影响表现得就更明显。

\subsubsection{其他物质形态}
除了我们最熟悉的固态、液态和气态以外,自然界中存在更多的物态,比如\textbf{等离子态}。火就是自然界中常见的一种等离子态,在这个状态下,气体的原子核分子发生电离,电子不再紧紧围绕在原子核周围,而是变成了游离态。此时,带有正电荷的原子核和游离出来的电子组成了等离子态,其整体呈现电中性,但有各种奇妙的物理性质。等离子态也有诸多应用,稍微举例说明,玩具中的等离子球就是利用等离子体在人触摸时被打破了电场、电势的分布而产生漂亮的电弧;可控核聚变也是以高温下等离子体中原子核被完全裸露为前提条件,来研究原子核之间聚变的。

在高中知识的学习中,我们还需要对\textbf{液晶}态进行介绍。 液晶广泛应用于手机、电视等屏幕中,以此来进行画面显示。液晶是介于固体和液体之间的一种物态,既具有一定的流动性,也有规律的晶格排布。在低温时,液晶会凝结为结晶态,这种固体结构具有有序的晶格排列;但是当温度升高时,由于分子热运动的增加,这种有序排列被稍稍打散,分子取向的有序程度也有所下降;当温度进一步升高时,分子的取向有序性完全小时,进入无序状态,这时候,液晶态也变成了液态。

从微观的角度来看,构成液晶的分子为有机分子,大多数呈现棒状,因此如果没有外界扰动,这些长棒状结构倾向于按照分子相互平行的方式进行排列,因此液晶有比较有序的分子结构。另外,分子取向排列的液晶具有光学各向异性,长棒状的分子在长轴上和短轴上对于光的折射率是不同的,因此当入射光的偏振方向和液晶分子的长轴方向成不同夹角时,由于液晶折射率的各向异性,会产生两束光程不同的光,而它们会在一起发生干涉,因此会在出射点处呈现出来某种特定的颜色,这也是液晶广泛应用于显示器的原理。

\subsection{功、热、内能}
\subsubsection{焦耳实验}
%对于之前的学习,我们清楚,自然界的能量既不会增加也不会减少,对于某一个封闭系统,能量是守恒的\upref{HSPM07}。
我们
知道物体具有内能\upref{thermo},其内能是由分子动能和分子势能所决定的,那么在考虑热学变化下,一个系统的能量又将如何改变呢?

随着实验经验的积累,人们逐渐意识到,为了使系统的热学状态发生变化,可以采用做功和热传导两种方法。对于这些变化,英国物理学家焦耳利用实验精准的测量外界对系统做功和传热时对系统状态的影响,以及其中能量的转化关系。这个规律后来被总结和提炼为热力学第一定律。

\textbf{焦耳的实验一:}在一个绝热性能良好的容器中倒入一定量的水,水中安装着一个搅拌器,搅拌器连接着机械装置,外界机械装置中由于有重物下落而带动搅拌器转动。
在搅拌器的转动中,会扰动和摩擦容器内部的水,从而使得水的温度升高。焦耳尝试了多组实验,发现,只要重力做的功相同,不论重物的质量、高度、下落时间有任何变化,水温上升的数值都是相同的,也就是说这一定量水的内能增加是一个定值。%进一步的测量发现,这一定值等于重力所做的功的大小。

\textbf{焦耳实验二:}同样在一个绝热系统中,如果采用更为直接的办法来加热这一定量的水,比如说让重物做功带动发电机发电,而所产生的电流通过浸没于液体中的电阻丝,由电流的热效应来加热液体。实验表明,不管电阻的大小、电流的大小如何,以及通电时间的长短,只要所做的电功是相等的,则一定量水的水温变化也是相同的,内能增加仍为定值。

这两个实验告诉我们,不管做功的过程怎样,只要外界所输入的功的大小相同,那么系统的初末状态就一定。也就是说,系统的这个状态可以由某一个状态量\footnote{所谓状态量,其实是和过程量相对的一个词。状态量即是描述物体在某个状态所具有的性质的物理量,状态量的变化只和系统的状态有关,而和达成这个状态的过程无关,比如说位置、速度、能量、温度、压强、体积都是状态量。}来进行描述。之前学习的重力势能\upref{HSPM07}就是一个状态量,物体具有重力势能的大小和物体的位置有关,重力势能的改变量只和物体在两个不同位置处所具有的重力势能的差值有关,而和物体如何从位置$1$改变到位置$2$的方式无关。与之类比,在热力学系统的绝热过程中,外界对系统所做的功会决定系统的初末状态,这和外界做功的具体形式和方式无关。所以说,任何一个热力学系统必然存在一个只依赖于系统本身状态的物理量,这个物理量的差值和外界做功大小相联系。根据简单的量纲分析\footnote{量纲分析法是研究物理问题的一个常见方法,在这里的应用简单来说就是表达相同物理含义的量应该具有同样的单位,反之亦然。}经验,
% 握草量纲分析居然没有词条,啊啊单位制没写完啊喂
我们知道这个物理量必然用来描述系统的能量,称之为系统的\textbf{内能}。这里提到的内能和分子动力学\upref{thermo}的内能是一样的,后者给出了微观解释。

\subsubsection{功和内能的改变}
在知道外界做功和系统内能的定性关系之后,更进一步的,考虑两者之间定量的关系。

刚才我们提到,水上升的温度只和外界对系统所做的功有关,经过更为精密的实验探究得知,在热力学系统的绝热过程中,当系统从状态$1$转变为状态$2$时,其中内能的变化量$\Delta U=U_2-U_1$等于外界对系统所做的功$W$,也即
$$\Delta U=W~.$$
当外界对系统做工的时候,内能$U$随之增加,系统对外界做功时,内能$U$减小。

\subsubsection{热与内能的改变}
当然,做功并非是改变系统内能的唯一方法,另一种常见的方法就是热传递。所谓热传递,就是两个温度不同的物体相互接触时会发生的一个现象,温度高的物体会降温,而温度低的物体会升温,这时候其实就是热从高温物体向低温物体有了一个传输。

和对系统做功与系统内能变化做类比,假设在传热前后,系统内能的变化量$\Delta U=U_2-U_1$,则系统与外界交换的热量$Q=\Delta U$,和做功情形下的方程形式是完全类似的。需要指出的是,虽然两者在方程上形式相同,但是做功的时候,是内能和其他形式的能量发生了转化,但是热传递造成的内能转变过程中,只是不同物体之间的内能传递,不涉及到能量的转换。

\subsection{热力学第一定律}
根据上述的介绍,我们可以清楚地看到做功和热传递所对系统内能变化的影响在方程形式上是完全等价的,那么,它们在物理上是否也是等价的呢?

想象一个场景,一个系统经由外界做功,从状态$1$变到状态$2$,此时$\Delta U=U_2-U_1=W$,而同时,如果我们采取热传递的方法也让一个完全相同的系统从状态$1$改变到状态$2$,则$\Delta U=U_2-U_1=Q$,可以发现对于引起系统发生相同的内能变化,采用做功和热传递的方法并没有本质区别。

上述内容可以总结为,做功和热传递对改变系统内能是等价的,我们再进一步进行推理,如果一个系统既受到外界的做功,也同时受到热传递,那系统的内能变化量就应该是
$$\Delta U=W+Q~.$$
直接为两者的加和。这也就是\textbf{热力学第一定律}的内容,更为严谨的表述为:一个热力学系统的内能增量等同于外界对它传递的热量与外界对它所做功的和。

\subsection{能量守恒定律}
回顾我们曾经学过的知识,我们学过机械能守恒定律\upref{HSPM07},也对自然界的各种能量形式有了一些认识。实际上,在对于各种能量形式之间的转换中,比如说动能和势能可以相互转换、电和磁可以相互转换、电和热可以相互转换,人们逐渐意识到有一种更为普世的守恒律可以用来对自然界各种能量形式之间的关系进行描述,这就是\textbf{能量守恒定律}。

以刚刚学习的热力学第一定律为例,内能和其他能之间满足能量守恒定律,因此会有内能的改变量等于外界对系统做功的大小或者外界对系统传递热量的多少。结合我们之前学过的机械能守恒定律、焦耳定律等,可以对\textbf{能量守恒定律}有更为准确的概括:能量既不会凭空产生,也不会凭空消失,而只能从一种形式转化为其他形式,或者从一个物体转移到另一个物体,在所有的这些变化中,能量的总和保持不变。

热力学第一定律告诉我们,第一类永动机\footnote{第一类永动机:不消耗能量却可以源源不断的输出动力。}是不可能被制造出来的。

\subsection{热力学第二定律}
能量守恒定律似乎暗示了某种“对称性”,热量从物体$A$可以传输到物体$B$,也完全可以从物体$B$传递给物体$A$;从山上滚下来的小球会获得一定的速度,而具有速度的小球可以滚上山峰。由于能量守恒律定律的保护,从这个角度来看,似乎这些变化的方向都是完全等价的,可是根据我们日常生活的经验观察,热量只会从高温物体传递给低温物体,实际条件下的小球的运动高度只会越发降低,因此,一定存在某种\textbf{不可逆}的现象,使得原有的对称性被打破,使得自然界的演化呈现出某种方向性。

这种演化的方向性由\textbf{热力学第二定律}指出,我们先从两个表述来认识这种方向性。德国物理学家克劳修斯指出:热量不能自发的从低温物体传到高温物体。这个表述被称之为\textbf{热力学第二定律的克劳修斯表述}。对于这个表述,需要强调热量传递的“自发”性,也就是说在没有外界做功的情况下,热量会从高温物体传递到低温物体,但是如果有外界做功的话,不一定完全按照这个方向来进行热量的传输,冰箱和空调就是这方面的例子。

那么,我们知道做功有可能会改变热量传输的方向,那热量是不是也可以变成功呢?我们来观察热机的工作原理:首先,热机中会有燃料燃烧,使得工作物质温度升高,内能增大;之后,工作物质可以对外做功,让自己的内能变成机械能。需要注意的是,热机做功除了可以对外做功的高温物质,也必须有一个承接剩下热量的低温物质,根据我们先前学习的热力学第一定律,$\Delta U=W+Q$,$Q>0$,因此有
$$W<Q~.$$
也就是热机用于做功的热量一定小于它从高温物质吸收的热量,这和体系是否理想(有无损耗)无关。这显然是不满足时间反演对称性的一个情形,对此可以得到\textbf{热力学第二定律的开尔文表述}:不可能从单一热库吸收热量,使之完全变成功,而不产生其他影响。这里所谓的“其他影响”就是必须要对低温物质传输一定的热量,或者有其他的能量输入。

热力学第二定律的开尔文表述实际上作出了某种暗示,即虽然自然界的能量总和是不变的,不同的能量形式之间可以相互转化,但是不同的能量形式并非是完全等价的。可以很不严谨的认为,某些能量比其他的能量更没有“价值”,而自然的演化过程只能将更有价值的能量换成价值更低的,比如说小球会在摩擦力的作用下逐渐停下来,机械能转化为热能,但是小球不能吸收环境的热量来开始运动。

在大学的学习中我们可以使用卡诺热机证明,热力学第二定律的克劳修斯表述和开尔文的表述是等价的。

实际上,热力学第二定律的各个表述都阐述了一个原理,即\textbf{熵增原理}。所谓\textbf{熵},是对系统混乱程度的表述量,若一个系统包含了更多的微观状态,则这个系统是更加混乱的,熵也是更大的。熵增原理说明,一个封闭系统中的熵只会自发增加而不会自发减小。因此,在没有外界影响的前提下,系统的自发演化总会从有序过渡到无序。

%% 画图时间
% 无
%% 错别字纠正
% 17:33-17:49
