% 中微子
% license CCBYSA3
% type Wiki

(本文根据 CC-BY-SA 协议转载自原搜狗科学百科对英文维基百科的翻译)

中微子(/nuːˈtriːnoʊ/或/njuːˈtriːnoʊ/)(由希腊字母ν表示)是费米子(一种具有半整数自旋的基本粒子),它仅通过弱力和引力参与相互作用。[1][2]中微子之所以如此命名,是因为它是电中性的,并且因为它的静止质量非常小,以致于人们长期以来认为它为零。中微子的质量比其他已知的基本粒子小得多。弱力的范围非常短,引力相互作用非常弱,中微子作为轻子不参与强相互作用。因此,中微子通常能畅通无阻地穿过普通物质,无法被探测到。[1][2]

弱相互作用产生三味轻子类型之一的中微子:电子中微子($\nu_e$)、μ子中微子($\nu_\mu$)、或$\tau$中微子($\nu_\tau$)并伴随相应的带电轻子。虽然中微子长期以来被认为是无质量的,但现在已经知道有三个不同微小值的离散中微子质量,但它们并不与这三种味道唯一对应。一个带有特定味道的中微子是所有三种质量状态的特定量子叠加。因此,中微子在飞行中在不同的味道之间振荡。例如,在衰变反应中产生的电子中微子可能在远处的探测器中作为$\mu$子或$\tau$中微子相互作用。尽管截至2016年,只有三个质量值的平均差是已知的,但宇宙学观察表明,三个质量的总和必须小于电子质量的百万分之一。

对于每一个中微子,也存在一个相应的反粒子,称为反中微子,它也有半整数自旋,没有电荷。它们与中微子的区别在于轻子数和手性符号相反。为了使总的轻子数守恒,在核$\beta$衰变中,电子中微子只与正电子(反电子)或电子反中微子一起出现,电子反中微子与电子或电子中微子一起出现。

中微子是由各种放射性衰变产生的,包括原子核或强子的$\beta$衰变、核反应,如发生在恒星核心或核反应堆、核弹或粒子加速器中的核反应、超新星爆发期间、中子星自旋期间以及加速粒子束或宇宙射线撞击原子时。地球附近的大多数中微子来自太阳的核反应。在地球附近,垂直于太阳方向,每平方厘米每秒大约有650亿($6.5\times10^{10}$)太阳中微子穿过。[3]

为了研究中微子,可以用核反应堆和粒子加速器人工制造中微子。有大量涉及中微子的研究活动,目标包括确定三个中微子质量值,测量轻子区的CP破坏程度(导致轻子发生);并寻找粒子物理标准模型之外的物理证据,如无中微子双$\beta$衰变,这将是轻子数守恒破坏的证据。中微子也可以用于地球内部的断层摄影。[4][5]

\subsection{历史}
\subsubsection{1.1 泡利的提议}
中微子是沃尔夫冈泡利在1930年首次提出的,用来解释$\beta$衰变如何使能量、动量和角动量(自旋)守恒。与尼尔斯·玻尔相反,他提出了守恒定律的统计版本来解释在$\beta$衰变中观察到的连续能谱,泡利假设了一个未被发现的粒子,他称之为“中子”,使用与命名质子和电子相同的-on结尾来命名。他认为新粒子是在β衰变过程中与电子或β粒子一起从原子核中发射出来的。[6]

詹姆斯·查德威克在1932年发现了一种质量更大的中性核粒子,并将其命名为中子,留下了两种同名的粒子。早些时候(1930年),泡利用“中子”一词既指在$\beta$衰变中保存能量的中性粒子,也指原子核中假定的中性粒子;起初,他并不认为这两种中性粒子彼此不同。[6]中微子一词是通过恩利克·费米引入科学词汇的,他在1932年7月巴黎的一次会议上和1933年10月的苏威会议上使用了中微子,泡利也在那次会议上使用了中微子。这个名字(意大利语中相当于“小中性粒子”)是由爱德华多·阿马尔迪(Edoardo Amaldi)在罗马via Panisperna物理研究所与费米(Fermi)交谈时开玩笑地创造出来的,目的是为了将这个轻中性粒子与查德威克的重中子区分开来。[7]

在费米的β衰变理论中,查德威克的大中性粒子可以衰变为质子、电子和较小的中性粒子(现在称为电子反中微子):
$$n^0 \to p^+ +e^- +\nu_e~$$
费米的论文写于1934年,将泡利中微子与保罗·狄拉克正电子和维尔纳海森堡中子质子模型统一起来,为未来的实验工作提供了坚实的理论基础。《自然》杂志拒绝了费米的论文,称该理论“离现实太远”。他将论文提交给一家意大利杂志,该杂志接受了论文,但在早期对他的理论普遍缺乏兴趣,导致他转向实验物理学。[8]

到1934年,有实验证据反对玻尔关于能量守恒对$\beta$衰变无效的观点:在那一年的索尔维会议上,报道了对β粒子(电子)能谱的测量,表明每种$\beta$衰变的电子能量都有严格的限制。如果能量守恒是无效的,这种限制是不可预期的,在这种情况下,在至少几个衰变中,统计上任意数量的能量都是有可能的。1934年首次测量到的β衰变谱的自然解释是,只有有限的(和守恒的)能量可用,一个新粒子有时会吸收有限能量的不同部分,剩下的留给β粒子。泡利利用这个机会公开强调仍然未被发现的“中微子”一定是一个真实的粒子。

\subsubsection{1.2 直接检测}
\begin{figure}[ht]
\centering
\includegraphics[width=6cm]{./figures/474e24bba6d42f22.png}
\caption{克莱德·科温进行中微子实验大约1956} \label{fig_ZWZ_1}
\end{figure}
1942年,王淦昌首次提出使用β俘获来实验检测中微子。[9]在1956年7月20日的《科学》杂志上,克莱德·考恩、弗雷德里克·莱因斯、哈里森、克鲁斯和麦奎尔发表了他们已经探测到中微子的证据,[10][11]这一结果在近40年后获得了1995年诺贝尔奖。[12]

在这个现在被称为考恩-雷恩中微子实验的实验中,通过在核反应堆中的β衰变产生的反中微子与质子反应产生中子和正电子:
$$\nu_e+p^+\to n^0+e^+~$$
正电子很快找到一个电子,然后它们互相湮灭。产生的两种伽马射线($\gamma$)是可探测的。中子可以通过被适当的原子核俘获并释放出伽马射线来探测。这巧合的两个事件——正电子湮没和中子俘获——给出了反中微子相互作用的独特标志。

1965年2月,包括弗雷德尔·塞尔肖普在内的一个团体在南非的金矿中发现了自然界中第一个中微子。[13]实验是在博克堡附近ERPM矿3公里深处的一个专门准备的小室中进行的。主楼的一块匾纪念了这一发现。实验还实现了原始中微子天文学,并研究了中微子物理和弱相互作用问题。[14]

\subsubsection{1.3 中微子味道}
Cowan和Reines发现的反中微子是电子中微子的反粒子。

1962年,利昂·M·莱德曼、梅尔文·施瓦茨和杰克·斯坦伯格通过首先探测$\mu$子中微子(已经用neutretto这个名字做了假设)的相互作用,[15]证明了不止一种中微子的存在,这使他们获得了1988年诺贝尔物理学奖。

当第三种轻子$\tau$于1975年在斯坦福线性加速器中心被发现时,它也有一个相关的中微子(τ中微子)。第三种中微子类型的第一个证据来自对τ衰变中丢失能量和动量的观察,$\tau$衰变类似于导致发现电子中微子的β衰变。费米实验室的DONUT合作组于2000年宣布首次探测到τ中微子相互作用;从大型正负电子对撞机的理论一致性和实验数据已经推断出它的存在。[16]

\subsubsection{1.4 太阳中微子问题}
在20世纪60年代,现在著名的豪斯休实验首次测量了来自太阳核心的电子中微子通量,发现了一个介于标准太阳模型预测的三分之一到二分之一之间的数值。这种差异,后来被称为太阳中微子问题,在大约三十年的时间里一直没有得到解决,尽管对实验和太阳模型可能存在的问题进行了研究,但没有发现任何问题。最终人们意识到两者都是正确的,而且它们之间的差异是由于中微子比以前假设的要复杂。据推测,这三个中微子的质量非零且略有不同,因此在飞往地球的过程中会振荡成不可察觉的味道。这一假设被一系列新的实验所研究,从而开启了一个新的主要研究领域,这个领域还在继续。中微子振荡现象的最终证实导致了两项诺贝尔奖,一项是小雷蒙德·戴维斯,他构思并领导了Homestake实验,另一项是阿特·麦克唐纳,他领导了SNO实验,该实验可以探测所有中微子的味道,并且没有发现任何缺陷。

\subsubsection{1.5 振荡}
Bruno Pontecorvo在1957年首次提出了一种研究中微子振荡的实用方法,这种方法与kaon振荡相似;在随后的10年里,他发展了真空震荡的数学形式和现代公式。1985年斯坦尼斯拉夫·米哈耶夫和阿列克谢·斯米尔诺夫(扩展了林肯·沃尔夫斯坦1978年的工作)指出,当中微子在物质中传播时,味道振荡可以被改变。这种所谓的米基耶夫-斯米尔诺夫-沃尔夫斯坦效应(MSW effect)很重要,因为许多太阳聚变产生的中微子在到达地球探测器的途中会穿过太阳核心的致密物质(基本上所有的太阳聚变都发生在这里)。

从1998年开始,实验开始表明太阳和大气中微子会改变味(见超级卡米康德和萨德伯里中微子天文台)。这解决了太阳中微子问题:太阳中产生的电子中微子部分变成了实验无法检测到的其他味道。

尽管单个实验,如太阳中微子实验,与中微子味道转换的非振荡机制相一致,但中微子实验暗示中微子振荡的存在。在这方面特别相关的是反应堆实验KamLAND和加速器实验,如MINOS。KamLAND实验确实将振荡确定为太阳电子中微子中涉及的中微子味道转换机制。类似地,MINOS证实了大气中微子的振荡,并给出了更好地确定了质量平方劈裂。[17] 日本的梶田隆章和加拿大的亚瑟·麦克唐纳获得了2015年诺贝尔物理学奖,因为他们在理论和实验上的里程碑式发现,中微子可以改变味道。

\subsubsection{1.6 宇宙中微子}
小雷蒙德·戴维斯和小柴昌俊共同获得了2002年诺贝尔物理学奖。两人都在太阳中微子探测方面进行了开创性的工作,小柴昌俊的工作还促使了对附近大麦哲伦星云中SN 1987A超新星中微子的首次实时观测。这些努力标志着中微子天文学的开始。[18]

SN 1987A代表了超新星中微子的唯一被证实的探测。然而,在我们的星系中,许多恒星已经变成超新星,留下了一个理论上的扩散超新星中微子背景。同样的概念延伸到整个宇宙,给出宇宙中微子背景,或遗迹中微子。

\subsection{ 性质和反应}
中微子具有半整数自旋(½ h),因此是费米子。中微子也是轻子,已经被观察到只通过弱力相互作用,尽管假设它们也在引力作用下相互作用。

\subsubsection{2.1 味道、质量及其混合}
弱相互作用产生三种轻子类型之一的中微子:电子中微子($\nu e$)、$\mu$子中微子($\nu \mu$)或$\tau$中微子($\nu \tau$),分别与相应的电子、$\mu$子和$\tau$带电轻子相关联。[19]

虽然中微子长期以来被认为是无质量的,但现在我们知道也有三个离散的中微子质量,但它们并不唯一对应于这三种味道。尽管截至2016年,只有这三个质量值的平方差是已知的,[19]但实验表明,这些质量很小。根据宇宙学测量,已经计算出三个中微子质量的总和必须小于电子质量的百万分之一。[19][19]

更正式地说,中微子味本征态不同于中微子质量本征态(简称为1,2,3)。截至2016年,还不知道这三个中哪一个最重。与带电轻子的质量等级相似,质量2比质量3轻的构型通常被称为“正常等级”,而在“反向等级”中,情况正好相反。几个主要的实验努力正在进行,以帮助确定哪个是正确的。[19]

一个中微子以一种特定的味道本征态产生,它是所有三种质量本征态的相关特定量子叠加。这是可能的,因为由于不确定性原理,这三个质量相差很小,以至于在任何实际飞行路径中都无法通过实验加以区分。已经发现,在产生的纯味状态中,每种质量状态的比例强烈依赖于该味。味和质量本征态之间的关系编码在PMNS矩阵中。实验已经确定了矩阵元素的值。[19]

中微子存在质量,这使得中微子有一个微小的磁矩成为可能,在这种情况下中微子也可以参与电磁相互作用;还没有发现这种相互作用。[20]

\subsubsection{2.2 味振荡}
中微子在飞行中在不同的味道之间振荡。例如,在$\beta$衰变反应中产生的电子中微子可能在远处的探测器中作为$\mu$子或$\tau$中微子相互作用,这由探测器中产生的带电轻子的味道来定义。这种振荡的发生是因为所产生的味的三种质量状态成分以稍微不同的速度行进,因此它们的量子机械波包产生相对相移,改变它们结合方式以产生三种味的不同叠加。因此,随着中微子的传播,每种味成分都呈正弦振荡,味在相对强度上有所不同。中微子相互作用时的相对味道比例代表了相互作用的味道产生相应带电轻子味道的相对概率。[21][21]

中微子还有其他可能振荡,即使它们没有质量。如果洛仑兹对称性不是一个精确的对称性,中微子可能会发生违反洛仑兹的振荡。[21]

\subsubsection{2.3 米哈耶夫-斯米尔诺夫-沃尔夫斯坦效应}
中微子穿过物质,一般来说,经历一个类似光穿过透明材料的过程。这一过程不能直接观察到,因为它不会产生电离辐射,但会产生MSW效应。中微子的能量只有一小部分被转移到物质中。[22]

\subsubsection{2.4 反中微子}
对于每一个中微子,也存在一个相应的反粒子,称为反中微子,它也没有电荷,有半整数自旋。它们与中微子的区别在于轻子数和手性的符号相反。截至2016年,没有发现任何其他差异的证据。在迄今为止对轻子过程的所有观察中(尽管对例外情况进行了广泛和持续的搜索),轻子数没有总体变化;例如,如果总轻子数在初始状态为零,电子中微子在最终状态下只与正电子(反电子)或电子反中微子一起出现,电子反中微子与电子或电子中微子一起出现。[23][23]

反中微子与β粒子一起在核β衰变中产生,例如,中子衰变为质子、电子和反中微子。迄今为止观察到的所有反中微子都具有右旋螺旋度(即,只观察到两种可能自旋态中的一种),而中微子是左旋的。尽管如此,由于中微子具有质量,它们的螺旋度是依赖于参考系的,所以与此相关的是与参考系无关的手性性质。

反中微子最初是由于在一个大水箱中与质子相互作用而被发现的。它被安装在核反应堆旁边,作为反中微子的可控来源。世界各地的研究人员已经开始研究在防止核武器扩散的背景下使用反中微子进行反应堆监测的可能性。[23][24][25]

\subsubsection{2.5 马约拉纳质量}
因为反中微子和中微子是中性粒子,所以它们有可能是同一个粒子。具有这种性质的粒子被称为马约拉纳粒子,以首次提出这一概念的意大利物理学家埃托雷·马约拉纳的名字命名。对于中微子来说,这一理论已经获得了广泛的应用,因为它可以结合跷跷板机制来解释为什么中微子的质量比其他基本粒子(如电子或夸克)的质量小。马约拉纳中微子的性质是中微子和反中微子只能通过手性来区分;实验观察到中微子和反中微子之间的差异可能仅仅是由于一个粒子具有两种可能的手性。

到目前为止还不知道中微子是马约拉纳粒子还是狄拉克粒子;有可能通过实验测试这一特性。例如,如果中微子确实是马约纳粒子,那么轻子数破坏过程如无中微子双β衰变将被允许,而如果中微子是狄拉克粒子,它们就不会被允许。已经并正在进行几项实验来寻找这一过程,例如, GERDA,[26] EXO,[27]和SNO+。[28]宇宙中微子背景也是对中微子是否是马约纳粒子的一种探索,因为在狄拉克或马约拉纳情况下应该检测到不同数量的宇宙中微子。[29]

\subsubsection{2.6 核反应}
中微子可以与原子核相互作用,把它变成另一个原子核。这个过程用于放射化学中微子探测器。在这种情况下,必须考虑靶核内的能级和自旋状态来估计相互作用的概率。一般来说,相互作用的概率随着原子核内中子和质子的数量而增加。[30][30]

很难在放射性的自然背景中唯一识别中微子的相互作用。因此,在早期的实验中,选择了一个特殊的反应通道来促进识别:一种反中微子与水分子中一个氢核的相互作用。氢原子核是一个质子,所以在探测实验中不需要考虑在较重的原子核内同时发生的核相互作用。在核反应堆外的一立方米水中,只能记录到相对较少的这种相互作用,但是现在该装置被用于测量反应堆的钚生产率。

\subsubsection{2.7 诱发裂变}
就像核反应堆中的中子一样,中微子可以在重核中引发裂变反应。[31]到目前为止,这种反应还没有在实验室进行测量,但是预计会在恒星和超新星内部发生。这个过程影响了宇宙中同位素的丰度。[30]氘原子核的中微子裂变已经在萨德伯里中微子天文台观测到,该天文台使用重水探测器。

\subsubsection{2.8 没有自相互作用}
对宇宙微波背景的观察表明中微子没有自相互作用。[32]

\subsubsection{2.9 类型}
\begin{figure}[ht]
\centering
\includegraphics[width=8cm]{./figures/f5d357f9f3f9a02d.png}
\caption\label{fig_ZWZ_2}
\end{figure}
中微子有三种已知类型(味):电子中微子$\nu e$、$\mu$介子中微子$\nu\mu$和$\tau$中微子$\nu \tau$,以标准模型中它们的伙伴轻子命名(见右图)。目前对中微子类型数量的最佳测量来自观察Z玻色子的衰变。这种粒子可以衰变为任意轻中微子和它的反中微子,而且可用的轻中微子类型越多,Z玻色子的寿命就越短。对Z寿命的测量表明,耦合到Z的轻中微子的数量是3。[19]标准模型中的六个夸克和六个轻子之间的对应关系,其中包括三个中微子,向物理学家的直觉表明,应该正好有三种中微子。证明中微子只有三种仍然是粒子物理学难以实现的目标。

\subsection{研究}
中微子有几个活跃的研究领域。有些人关心中微子行为的预测测试。其他研究集中在中微子未知性质的测量上;人们对实验特别感兴趣,这些实验决定了它们的质量和CP破坏率,而这是目前的理论无法预测的。

\subsubsection{3.1 人造中微子源附近的探测器}
国际科学合作组在核反应堆附近或粒子加速器的中微子束中安装大型中微子探测器,以更好地限制中微子质量以及中微子味道之间振荡的幅度和速率。因此,这些实验是在寻找中微子部分中CP破坏的存在;也就是说,物理定律对中微子和反中微子是否不同。[33]

德国的KATRIN 实验已于2018年6月开始获取数据[33]以确定电子中微子的质量值,并在计划阶段采用其他方法解决这一问题。

\subsubsection{3.2 中微子振荡测试}
2013年7月19日,在瑞典斯德哥尔摩举行的欧洲物理学会高能物理会议上提出的T2K实验结果证实了中微子振荡理论。[34]

\subsubsection{3.3 引力效应}
尽管中微子质量很小,但它们数量众多,以至于它们的引力可以影响宇宙中的其他物质。

已知的三味中微子是暗物质候选者中唯一确定的基本粒子,特别是热暗物质,尽管这种可能性似乎在很大程度上被宇宙微波背景的观测所排除。如果较重的无菌中微子存在,它们可能充当温暖的暗物质,这似乎仍然是合理的。[35]

\subsubsection{3.4 搜索惰性中微子}
其他努力寻找惰性中微子的证据——第四种中微子味道像三味已知的中微子那样不与物质相互作用。[36][37][38][39]惰性中微子的可能性不受上述玻色子衰变测量的影响:如果它们的质量大于玻色子质量的一半,它们就不是衰变产物。因此,重的中微子的质量至少为45.6 GeV。

LSND实验的实验数据实际上暗示了这种粒子的存在。另一方面,目前正在运行的 (MinBoOne)MiniBooNE 实验表明,解释实验数据,[40]惰性中微子不是必要的,尽管对这一领域的最新研究正在进行, MiniBooNE 数据中的异常可能允许奇异中微子类型,包括惰性中微子。[41]劳厄-朗之万研究所最近对参考电子光谱数据的重新分析[42]也暗示了第四个惰性中微子。[43]

根据2010年发表的一项分析,来自宇宙背景辐射威尔金森微波各向异性探测器的数据与三种或四种中微子兼容。[44]

\subsubsection{3.5 无中微子双β衰变搜索}
另一个假设与“无中微子双$\beta$衰变”有关,如果它存在的话,将违反轻子数守恒,并意味着传统上被称为“中微子”的物理质量与其轻子数符号相反的相应“反中微子”之间的微小分裂(或差异)。对这一机制的研究正在进行中,但尚未找到有力的证据。如果是的话,那么现在所谓的反中微子不可能是真正的抗粒子。由此产生的六个截然不同的中微子没有截然不同的反粒子伙伴。[45] 宇宙射线中微子实验检测来自太空的中微子,研究中微子的性质和产生中微子的宇宙来源。[46]

\subsubsection{3.6 速度}
在中微子被发现振荡之前,它们通常被认为是无质量的,以光速传播。根据狭义相对论,中微子速度的问题与它们的质量密切相关:如果中微子没有质量,它们必须以光速行进,如果它们有质量,它们就不能达到光速。由于质量很小,在所有实验中,预测的速度都非常接近光速,而目前的探测器对预期的差异不敏感。

另外,一些违反洛伦兹定律的量子引力变体可能允许比光速更快的中微子。洛伦兹破坏的综合框架是标准模型扩展(SME)。

在20世纪80年代早期,中微子速度的首次测量是使用脉冲$\pi$介子束(由撞击目标的脉冲质子束产生的)。π介子衰变产生中微子,在一定距离的探测器中在时间窗内观察到的中微子相互作用与光速一致。2007年,使用MINOS探测器重复了这一测量,发现3 GeV中微子的速度在99\%置信水平下,在0.999976c和1.000126c之间。1.000051c的中心值高于光速,但也与恰好$c$或甚至稍小的速度一致。这种测量在99\%的置信度下确定了50兆电子伏$\mu$子中微子质量的上限。[47][48]2012年项目探测器升级后,MINOS完善了初始结果,发现与光速一致,中微子和光的到达时间相差0.0006\% ( 0.0012\%)。[49]

超新星1987A (SN 1987A)在更大的范围内也进行了类似的观察。在时间窗口内探测到的从超新星发出的10兆电子伏反中微子与中微子光速一致。迄今为止,中微子速度的所有测量都与光速一致。[50][51]

2011年9月,OPERA合作组发布的计算结果显示,在他们的实验中,17千兆电子伏和28中微子的速度超过了光速(见超光速中微子异常)。2011年11月,OPERA重复了实验并做了一些改变,这样就可以为每个探测到的中微子单独确定速度。结果显示同样的超光速。2012年2月,有报道称,这一结果可能是由一根松散的光缆引起的,光缆连接在一个原子钟上,用来测量中微子的离开和到达时间。ICARUS 在同一实验室独立地重现了这个实验,发现中微子速度和光速之间没有明显的差异。[52]

2012年6月,欧洲粒子物理研究所(CERN)宣布,由所有四个Gran Sasso实验(OPERA、ICARUS、Borexino和LVD)进行的新测量发现光速和中微子速度一致,最终驳斥了OPERA最初的说法。[53]

\subsubsection{3.7 质量}
粒子物理的标准模型假设中微子是无质量的。中微子振荡现象是实验确定的,它将中微子味态和中微子质量态混合在一起(类似于CKM混合),要求中微子具有非零质量。[54]大质量中微子最初是由布鲁诺·庞德科沃在20世纪50年代构想出来的。通过增加右手拉氏量,可以很简单地增强基本框架,容纳它们的质量。

提供中微子质量有两种方法, 有些人建议两者都用:
\begin{enumerate}
\item 如果像其他基本的标准模型粒子一样,质量是由狄拉克机制产生的,那么这个框架将需要SU(2)单态。这个粒子会有与希格斯双重态中性成分的汤川相互作用,但否则不会与标准模型粒子相互作用,所以被称为“中微子。
\item 或者,质量可以由马约拉纳机制产生,这将要求中微子和反中微子是同一个粒子。
\end{enumerate}
中微子质量的最大上限来自宇宙学:宇宙大爆炸模型预测在宇宙微波背景中中微子的数量和光子的数量之间有一个固定的比率。如果所有三种中微子的总能量超过平均每中微子50 eV,宇宙中就会有如此多的质量以至于坍缩。[55]这个极限可以通过假设中微子是不稳定的来规避,但是标准模型中有一些极限使得这变得困难。一个更严格的限制来自对宇宙数据的仔细分析,如宇宙微波背景辐射、星系测量和莱曼-阿尔法森林。这表明三个中微子的总质量必须小于0.3电子伏。[56]

2015年诺贝尔物理学奖授予了梶田隆章和亚瑟·麦克唐纳,因为他们在实验中发现了中微子振荡,这证明中微子具有质量。[57][58]

1998年,超级神冈仑中微子探测器的研究结果确定中微子可以从一种味道振荡到另一种味道,这就要求它们必须具有非零质量。 尽管这表明中微子有质量,但绝对中微子质量尺度仍然未知。这是因为中微子振荡只对 $8m^2_{21} = 0.000079 \\, \text{eV}^2$ 敏感。2006年,MINOS实验测量了强子中微子的振荡,确定了中微子质量本征态2和3之间质量平方的差异。初步结果表明 $8m^2_{32} = 0.0027$,与超级神冈仑之前的结果一致。 因为 $8m^2_{32}$ 是两个平方质量的差,其中至少有一个不低于这个值的平方根。因此,至少存在一个质量至少为 $0.05 \\, \text{电子伏}$ 的中微子质量本征态。

2009年,分析了星系团的透镜数据,预测中微子质量约为1.5电子伏。[64]这个惊人的高值要求三个中微子质量几乎相等,中微子振荡约为毫电子伏特。2016年,这一质量被更新为1.85电子伏。[65]它预测了3个相同质量的惰性中微子,源于普朗克暗物质部分和无中微子双β衰变的观测。对于电子反中微子,质量位于Mainz-Troitsk上限2.2电子伏以下。[66]后者自2018年6月以来一直在KATRIN实验中进行验证,该实验寻找的质量在0.2电子伏和2电子伏之间。[33]