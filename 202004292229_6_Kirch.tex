% 基尔霍夫定律

\subsection{基尔霍夫电流定律}

\subsubsection{定理内容}
\begin{theorem}{基尔霍夫电流定律}
\textbf{基尔霍夫电流定律}又称为\textbf{基尔霍夫第一定律},表明\textbf{所有进入某节点的电流的总和等于所有离开这节点的电流的总和.}.或者,更详细描述,假设进入某节点的电流为正值,离开这节点的电流为负值,则所有涉及这节点的电流的代数和等于零.以方程式表达,对于电路的任意节点,有
\begin{equation}
\sum_{k=1}^n i_k =0
\end{equation}

其中,$i_k$是第$k$个进入或离开这节点的电流,是流过与这节点相连接的第$k$个支路的电流,可以是实数或复数.
\end{theorem}


由于累积的电荷是电流与时间的乘积,从电荷守恒定律可以推导出这条定律.其实质是稳恒电流的连续性方程,即根据电荷守恒定律,流向节点的电流之和等于流出节点的电流之和.

\subsubsection{定理证明}
思考电路的某节点,跟这节点相连接有$n$个支路.假设进入这节点的电流为正值,离开这节点的电流为负值,则经过这节点的总电流$i$等于流过支路$k$的电流$i_k$的代数和:
\begin{equation}
i=\sum_{k=1}^n i_k
\end{equation}

将这方程式对时间积分,可以得到累积于这节点的电荷的方程式:
\begin{equation}
q=\sum_{k=1}^n q_k
\end{equation}

其中,$\displaystyle q=\int_0^t i(t') \mathrm{d}t'$是在此节点积累的总电荷,$\displaystyle q_k=\int_0^t i_k(t') \mathrm{d}t'$是流过支路$k$的电荷,$t$是检验时间,$t'$是积分时间变量.

假设$q>0$,则正电荷会累积于节点;否则,负电荷会累积于节点.根据电荷守恒定律,$q$是个常数,不能够随着时间推移而改变.由于这节点是个导体,不能储存任何电荷.所以,$q=0$、$i=0$,基尔霍夫电流定律成立:
\begin{equation}
\sum_{k=1}^n i_k =0
\end{equation}

从上述推导可以看到,\textbf{只有当电荷量为常数时,基尔霍夫电流定律才会成立}.通常,这不是个问题,因为静电力相斥作用,会阻止任何正电荷或负电荷随时间演进而累积于节点,大多时候,节点的静电荷是零.

不过,电容器的两块导板可能会允许正电荷或负电荷的累积.这是因为电容器的两块导板之间的空隙,会阻止分别累积于两块导板的异性电荷相遇,从而互相抵消.对于这种状况,流向其中任何一块导板的电流总和等于电荷累积的速率,而不是零.但是,若将位移电流$\mathbf{J}_D$纳入考虑,则基尔霍夫电流定律依然有效.

从电动力学的角度怎么来考虑推导呢?其实对含电介质的安培定律取散度,然后与高斯定律相结合,即可得到基尔霍夫电流定律:
\begin{equation}
\nabla \vdot \mathbf {J} =-\epsilon _{0}\nabla \vdot {\frac {\partial \mathbf {E} }{\partial t}}=-{\frac {\partial \rho }{\partial t}}
\end{equation}

其中,$\mathbf{J}$是电流密度,$\epsilon_0$是电常数,$\mathbf{E}$是电场,$\rho$是电荷密度.

这是电荷守恒的微分方程式.以积分的形式表述,从封闭表面流出的电流等于在这封闭表面内部的电荷$Q$的流失率:
\begin{equation}
\oint _{S}\mathbf {J} \cdot \mathrm {d} \mathbf {a} =-{\frac {\mathrm {d} Q}{\mathrm {d} t}}
\end{equation}

基尔霍夫电流定律等价于电流的散度是零.所以我们可以看出,对于不含时电荷密度$\rho$,该定律成立.对于含时电荷密度,则必需将位移电流纳入考虑.

\subsection{基尔霍夫电压定律}
\subsubsection{定理内容}
\begin{theorem}{基尔霍夫电压定律}
\textbf{基尔霍夫电压定律}又称为\textbf{基尔霍夫第二定律},表明沿着闭合回路所有元件两端的电势差(电压)的代数和等于零.或者,换句话说,沿着闭合回路的所有电动势的代数和等于所有电压降的代数和.以方程式表达,对于电路的任意闭合回路,
\begin{equation}
\sum_{k=1}^m V_k = 0
\end{equation}
其中,$m$是此闭合回路的元件数目,$V_k$是元件两端的电压,可以是实数或复数.
\end{theorem}




基尔霍夫电压定律不仅应用于闭合回路,也可以把它推广应用于回路的部分电路.clarify

===电场与电势===
在静电学里,电势定义为电场的负线积分:
:$\phi(\mathbf{r})\stackrel{def}{=} - \int_\mathbb{L} \mathbf{E} \cdot \mathrm{d} \boldsymbol{\ell}\,\!$;

其中,$\phi(\mathbf{r})$是电势,$\mathbf{E}$是电场,$\mathbb{L}$是从参考位置到位置$\mathbf{r}$的路径,$\mathrm{d} \boldsymbol{\ell}$是这路径的微小线元素.

那麽,基尔霍夫电压定律可以等价表达为:
:$\oint_{\mathbb{C \mathbf{E} \cdot d\mathbf{l} = 0$;

其中,$\mathbb{C}$是积分的闭合回路.

这方程式乃是法拉第电磁感应定律对于一个特殊状况的简化版本.假设通过闭合回路$\mathbb{C}$的磁通量为常数,则这方程式成立.

这方程式指明,电场沿着闭合回路$\mathbb{C}$的线积分为零.将这线积分切割为几段支路,就可以分别计算每一段支路的电压.

===理论限制===
由于含时电流会产生含时磁场,通过闭合回路$\mathbb{C}$的磁通量是时间的函数,根据法拉第电磁感应定律,会有电动势$\mathcal{E}$出现于闭合回路$\mathbb{C}$.所以,电场沿着闭合回路$\mathbb{C}$的线积分不等于零.这是因为电流会将能量传递给磁场;反之亦然,磁场亦会将能量传递给电流.

对于含有电感器的电路,必需将基尔霍夫电压定律加以修正.由于含时电流的作用,电路的每一个电感器都会产生对应的电动势$\mathcal{E}_k$.必需将这电动势纳入基尔霍夫电压定律,才能求得正确答案.