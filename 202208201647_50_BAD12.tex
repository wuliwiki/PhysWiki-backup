% 北京航空航天大学 2013 年数据结构与 C 语言程序设计
% 北京航空航天大学 2013 年数据结构与 C 语言程序设计


考生注意:所有答题务必书写在考场提供的答题纸上,写在本试题单上 的答题一律无效(本题单不参与阅卷).

\subsection{一、填空题}
(本题共20分,每小题各2分)

1. 从总体上说,“数据结构”课程主要研究(    )三个方面的内容.

2. 若对某线性表及常用的操作是在表中插入元素或者删除表中元素,则对于顺序存储结构和链式存储结构这两种存储结构而言,线性表应该采用(    ).

3. 在长度为$n$的非空队列中进行插入或者删除操作的时间复杂度用大$O$符号表示为(    ).

4. 若一棵度为4的树中度为1、2、3和4的结点个数分别为4、2、1和1,则该树中叶结点的个数为(    ).

5. 若某一叉树的中序遍历序列为B,A,F,D,G,C,E,按层次遍历序列为A,B,C,D,E,F,G,则该二义树的后序遍历序列为(    ).

6. 将一棵结点总数为$n$、且具有$m$个叶结点的树转换为一棵二叉树以后,该二叉树中右子树为空的结点有(    )个.

7. 对于图$G=(V,E)$与$G'=(V'E)$,若有$V'\subseteq V$,$E'\subseteq E$,则称$G'$是$G$的(    ).

8. 在顺序表(6,15,30,37,65,68,70,72,89,99)中采用折半查找法看找元素37,与表中进行过比较的元素依次是(    ).

9. 若已知$n$个关键字值具有相同的散列函数值,并且采用线性探测再散列法处理冲突,那么,将这$n$个关键字值全部散列到初始为空的地址空间中,发生散列冲突的次数是(    ).

10. 若长度为$n$的序歹ij K=(k|,k2,…,L)当且仅当满足kWk*并旦k,^k21+1(l ^i<n/2j) 时,则称该序列为一个小顶堆积(Heap).根据该定义,序列(26,5,77,1,61」1,59,48,15,19) 对应的小顶堆积是.

\subsection{二、简答题}
(本题共20分,每小题各5分)

1	.如果一个具有100个顶点、200条边的有向图采用邻接矩阵存储,该邻接矩阵是 否是稀疏矩阵?为什么?(这里我们假设:当矩阵中非零元索的数门小「整个知阵总元 素的数目的5%时认为该如阵为稀疏矩阵)
2	. 一般情况F,建立散列表时难以避免出现散列冲突,常用处理散列冲突的方法之 是开放定址法,该方法的基木思想是什么?
3	.若对序列(2,12,列,88510)按值从小到大进行排序,前三趟排序的结果分别为:
第一趟排序的结果:(2J 2.16.5.10.88)
第二趟排序的结果:(2.12,5,10,16,88)
笫三趟排序的结果:(2,5,10」2,16,88)
请问:该结果是采用了选择排序法还是采用了(起)泡排序法得到的?为付么?
4	.快速排序法的排序过程是递归的.若待排序序列的长度为n,则快速排序的最小 递归深度与最大递归深度分别是多少?

三、综合题(本题共20分,每小题各5分)
1	.若非空双向循环链表中链结点结构为llink data rlink ,则依次执行下列4
条语句的目的是在该链表中由q指的结点后面插入一个由p指的结点,其中1条语句有
错误,请找出该语句,并写出正确的语句.
p->llink=q;
p->rlink=q->rlink: q_>rlink=p: q->rlink->llink=p;
/*笫1条语句*/ /*第2条语句*/ /*第3条语句*/ /*第4条语句*/
2	.已知某完全二叉树的第7层有10个叶结点,请求出该完全二叉树的结点总数的 最大值.(要求写出结论的求解过程)
第991—2页
3	.证明:具有n个顶点的无向图最多有nx(n-l)/2条边.
4	.清分别写出对数据元素序列(80.30.50.10.90.20)按值从大到小进行选择排序时每
一趟的排序结果.

四、算法设计题(本题15分)
已知某具有n个顶点的有向图采用邻接表方法存储,其中,用以存储有向边信息的 边结点类型为
typedef struct edge{ int adj vex;
struct edge *next:
/*某有向边的终止顶点在顶点结点中的位置*/
/*指向下,个边结点*/
} ELink;
用以存储顶点信息的顶点结点类型为
typedef struct ver{ int indegree;
vertype vertex;
ELink ♦link;
/♦某顶点的入度*/
/♦某顶点的数据信息*/
/*指向以该顶点为出发点的第一个边结点*/
JVLink;
并且n个顶点结点构成一个数组结构请写一个异法,该算法判断给定的顶点 序列丫[0.5-1)-{ww2股八..N}是否是该有向图的一个拓扑序列,若是该有向图的一个拓 扑序列,算法返回1,否则,兑法返回0.

五、单项选择题(木题共20分,每小题各2分)
1	.在C语言中,标识符只能由字母、数字和卜划线三种字符组成,并R第♦个字
符.
A.必须是字母	B.必须是下划线
C.必须是字母或者下划线	D.可以是字母、数字和卜划线之
2	.若整型变量x的初值为6,则计算表达式“x+=x-=x*x”之后,x的值是.
A. 50	B. 60	C. -50	D. -60
第991—3页
3	.下列4个程序段中,不是无限循环的是.
A. fbr(b=0,a=I; a>-+-+b; a=k++) k=a: B. Ibr(;; a++=k);
C. while(l) { a++; }	D. fbr(k~IO;; k—) loial+=k;
4.说明 “double (*ptr)[N];“ 中的标识符 ptr 是•
A. N个指向double类型变量的指针
B.指向N个double类型变电的函数指针
C. 一个指向由N个double类型元素组成的•维数组的指针
D.具有N个指针元素的•维指针数组,其每一个元素都只能指向double类
型变量
5. F列4个叙述中,正确的是.
A.	char *『“china";等价于 char *r; *r=“china'';
B.	char *ptr二“china”;等价 T char *ptr: ptr^kchina'';
C.	char string^0]={‘'china"};等价于 char string]lO];string[ ]={"china”};
D.	char str[4]="abc”,temp[4]="abc”;等价于 char str(4]=temp[4j=*kabc'';

6	.在 C 程序中,语句 “char *func(int x,int y);” 表示•
A.时函数fhnc的定义	B.对函数func的调用
C.对函数func返回值类型的说明 D.对函数func的原型说明
7	.对于下列程序,若从键盘上输入:abcdef<回车〉,则输出结果是
^include <stdio.h>
r/include <malloc.h>
main()
{ char *p,*q;
p=(char *)malloc(sizeof(char)*20);
q=p;
scanf(4t%s%sn,p,q);
printf(“%s%s\n”,p,q);
}
A. defdef	B. abcdef	C. abc d	D. dd
笫991—4页
8.当说明一个结构体变量时系统分配给它的内存是
A.结构中最后一个成员所需的内存量
B.结构中第一个成员所需的内存量
C.成员中占内存是最大者所需的容量
D.各成员所需内存后的总和