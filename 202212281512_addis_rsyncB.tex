% 用 cp 或 rsync 备份文件夹

\begin{issues}
\issueDraft
\end{issues}

事实上, 如果要备份一些不常改动的大文件(例如视频、电影), 要防止 bitrot, 只需要生成一个哈希表: \verb`find . -type f -exec sha1sum {} \; | sort > sha1sum.txt`。 然后把文件夹复制粘贴到备份文件夹即可。 要更新, 可以用 rsync(同样具有对比 hash 的功能)也可以自己写一个 bash 脚本或者简单的程序做这件事情。 要检查 bitrot, 就再运行一次输出到 \verb|sha1sum2.txt|, 然后用 \verb|git diff --no-index| 对比两个文件即可。

最后, 至于网络备份, 国内的主流网盘基本都支持秒传功能。 所谓秒传就是通过检测文件 hash 和大小等来避免重复上传任何人已经传过的文件。 所以如果同样的文件夹用网盘第二次备份, 那么所有文件都将秒传。 一些网盘(如某度)还支持删除重复文件的功能。 不过为了防和谐, 可以自己写一个程序把文件的少数一些 bit 就地打乱, 以及把文件名 flit 一下等等, 等传完再就地复原, 这样就完美了。

\subsection{用 hash 和 cp 备份}

\addTODO{未完成!}
\begin{lstlisting}[language=bash]
#! /usr/bin/bash

dest="$1" # backup directory
dest=${dest%/}

for repo in */ ; do
  repo=${repo%/}
  if ! [ -f "$repo/sha1sum.txt" ]; then
    continue
  fi

  printf "\n\n\n===============================\n"
  echo "$repo"
  printf "===============================\n\n\n"

  if [ -d "$dest/${repo}.backup" ]; then
    echo "$dest/${repo}.backup folder exists! skip!" 1>&2
    continue
  fi

  if ! [ -s "$repo/sha1sum.txt" ]; then
    echo "sha1sum.txt is empty! hasing..." 1>&2
    cd $repo
    find . -type f -exec sha1sum {} \; | sort > sha1sum.txt
    cd - > /dev/null
  else
    echo "sha1sum.txt not empty! rehasing..."
    cd $repo
    find . -type f -exec sha1sum {} \; | sort > sha1sum-new.txt
    cd - > /dev/null
  fi

  # # copy folder
  # echo "copying to $dest/${repo}.backup ..."
  # mkdir "$dest/${repo}.backup"
  # cp -a $repo -t "$dest/${repo}.backup"
  # # verify if needed
  # cd "$dest/${repo}.backup"
  # find . -type f -exec sha1sum {} \; | sort > sha1sum.txt
  # cd - > /dev/null
done
\end{lstlisting}
