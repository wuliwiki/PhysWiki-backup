% 北京师范大学 2013 年考研普通物理
% keys 北师大|普通物理
% license Copy
% type Tutor

\textbf{声明}:“该内容来源于网络公开资料,不保证真实性,如有侵权请联系管理员”

\begin{issues}
相关题目画图
\end{issues}

1。一质量为 $m$ ,长度为 $l$ 的匀质细杆铅直的放置在光滑水平面,杆自静止倒下,当杆与铅直线成为 $\theta$ 时,求地面对杆端的支撑力。

2。一个圆柱形大桶, 高为 $\mathrm{H}$, 底面积为 $\mathrm{S}$ 。桶的侧壁有一个小孔, 小孔的面积为 $\mathrm{A}$, 已知 $S$ 比 $A$ 大得多。初始时小孔用活塞塞紧,桶内盛满了水,水面到小孔的高度差为 $L$ 。\\
(1) 活塞受到的水的作用力有多大? \\
(2) 当拔除活塞时,水从小孔流出的速度有多大? \\
(3) 如果拔掉塞子后,马上再以速度 $v$ 塞回去, 忽略水面高度的变化, 请估计在此过程中塞子受到的来自水的作用力。

3。在直径为 $D$ 的管中传播的平面简谐声波,频率为 $f$,波的相速为 $v$ ,平均能流密度 $I$。\\
(1)求波的振幅。\\
(2)求相位差为 $2\pi$ 的两个波面间的总能量。\\
(3)如果它与另一支同频率平面简谐声波在此管内迎面相遇,造成有些位置声强始终为零,这些位置点之间的最小距离是多少?

4。解释下面物理概念:\\
(1)准静态过程,(2)理想气体,(3)可逆过程,(4)熵。

5。一容器与外界隔绝,中间用绝热壁隔开,一侧体积为 $V_1$,内有 $1\mathrm{mol}$ 温度 $T_1$ 的刚性双原子分子理想气体:另一侧体积为 $V_2$ ,内有 $1\mathrm{mol}$ 温度为 $T_2$ 的单原子分子理想气体。现在将绝热壁抽掉,使两边气体混合,最后达到平衡。忽略绝热壁的体积,求平衡时的温度和过程中气体的总熵变。(气体普适常数记为 $R$ )

6。如\autoref{fig_BNU13_1} 所示,把焦距为 $10\mathrm{cm}$ 的会聚透镜的中央部分 $C$ 切去, $C$ 的宽度为 $1\mathrm{cm}$ ,把余下的两部分 $A,B$ 粘起来。在粘合透镜的左侧,其对称轴上距透镜 $5\mathrm{cm}$ 处放置一个点物。\\
(1)求像的位置。
(2)光屏放在粘合透镜的右侧,与透镜相距 $490\mathrm{cm}$,用波长为 $500\mathrm{nm}$ 的单色光入射,求光屏上的条纹间隔。\\
(3)要使光屏上的条纹间隔与光屏到透镜的距离无关,点物应放在粘合透镜对称轴上的何处?此时光屏上的条纹间隔是多少?
\begin{figure}[ht]
\centering
\includegraphics[width=5cm]{./figures/e32fa07fc7773e31.pdf}
\caption{第6题图示} \label{fig_BNU13_1}
\end{figure}
7.如\autoref{fig_BNU13_2} 所示,波长为 $600\mathrm{nm}$ 的单色平行光垂直照射一个置于空气中的玻璃劈尖,在劈尖表面观察到干涉条纹的条纹间隔为 $0.1\mathrm{mm}$ ,已知玻璃折射率为 $1.5$,空气折射率近似为 $1$。\\
(1)劈尖处(膜厚为0处)为亮条纹还是暗条纹?\\
(2)求劈尖的劈角 $\alpha$。
(3)假设入射光为自然光,如果只考虑上表面的反射光与经下表面第一次反射后到空气中的光束之间的干涉,求干涉条纹的可见度。(已知正入射时的菲涅尔公式: $r_s=r_p=\frac{n_1-n_2}{n_1+n_2},t_s=t_p=\frac{2n_1}{n_1+n_2}$ )
\begin{figure}[ht]
\centering
\includegraphics[width=3cm]{./figures/7039ef608c297434.pdf}
\caption{第7题图示} \label{fig_BNU13_2}
\end{figure}
8。如\autoref{fig_BNU13_3} 所示,自然光沿垂直偏振片表面方向依次通过偏振片和 $1/4$ 波片,从 $1/4$ 波片出射的光(光束1)垂直入射到光滑玻璃表面上,部分光被反射回来(光束2),反射光再次入射到 $1/4$ 波片上,已知偏振片的透振方向和 $1/4$ 波片的光轴方向的夹角为 $45^\circ$。\\
(1)说明光束1和光束2的偏振态。\\
(2)光束2经过 $1/4$ 波片后能否透过偏振片?为什么?
\begin{figure}[ht]
\centering
\includegraphics[width=8cm]{./figures/2c4b3f7c32d651a4.pdf}
\caption{第8题图示} \label{fig_BNU13_3}
\end{figure}
9。如下\autoref{fig_BNU13_4} 所示,在金属球 $A$ 内有两个球形空腔,此金属球整体不带电,在两空腔中心各自放置一个点点电荷 $q_b$ 和 $q_c$ 。此外在金属球 $A$ 之外远处放置一个点电荷 $q_d$ ($q_d$ 至 $A$ 中心的距离为 $r$ 远大于球 $A$ 的半径 $R$)。求:\\
(1)作用在 $A,q_b,q_c,q_d$ 四物体上的静电力是多少?\\
(2)点电荷 $q_d$ 和点电荷 $q_c$ 之间的静电力是多少?\\
(3)如果 $r=3R$,$q_d$ 电荷在金属球上感应电荷在球心产生的电场强度是多少,此时金属球的的电势是多少?
\begin{figure}[ht]
\centering
\includegraphics[width=5cm]{./figures/9df7813d5ab19c6d.pdf}
\caption{第9题图示} \label{fig_BNU13_4}
\end{figure}
10。如\autoref{fig_BNU13_5} 所示:半径为 $R$ 的金属球浮在相对介电常数为 $\epsilon_r$ 的大油槽中,球的一半浸在油中,球的上半部在真空中,已知球上净电荷为 $Q$。求:
(1)金属球的下半球极化面点荷密度。\\
(2)空间的电场强度分布。\\
(3)油中计划体电荷密度分布。
\begin{figure}[ht]
\centering
\includegraphics[width=5cm]{./figures/a61abb8c6495e475.pdf}
\caption{第10题图示} \label{fig_BNU13_5}
\end{figure}
11。如下\autoref{fig_BNU13_6} 所示,半径为 $R$ 的圆线圈绕其直径 $PQ$ 匀角速转动,角速度为 $\omega$ 。在线圈中,放置一个尺寸很小的磁体,它的磁矩为 $
\bvec{M}$,方向沿 $z$ 轴正方向。求 $PC$ 之间的感应电动势。( $C$ 点是 $PQ$ 弧的中点)。
\begin{figure}[ht]
\centering
\includegraphics[width=2.5cm]{./figures/4ec0bd9227482885.pdf}
\caption{第11题图示} \label{fig_BNU13_6}
\end{figure}
