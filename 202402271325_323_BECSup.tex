% BEC 超流
% keys 玻色爱因斯坦|凝聚态|超流
% license Usr
% type Tutor

首先,我们回顾一下理想玻色气体的BEC。 假设这种玻色子的色散是 $\epsilon_{\bvec k} = \hbar^2{\bvec k}^2/2m$ (实际上长什么样在我们这里一般的分析中并没有什么关系)。 我们可以看到,对于玻色-爱因斯坦分布, $n_{\bvec k}=1/(\E^{\beta(\epsilon_{\bvec k}+\mu)}-1)$。为保证粒子占据数为正,则要求对任意动量$\bvec k$都有 $e^{\beta (\epsilon_{\bvec k}-\mu)}-1>0$。 因此当$\bvec k = 0$时, 就要求$\mu \leq 0$。 在热力学极限下可以发现如果 $\mu<0$,换句话说温度处于 $T_c$, $N\to\infty$, $n_{\bvec k}/N\to 0$。 而相变的时候 $T_c: \mu=0$,这时候物理图像是de Brogilie波长 $\approx$ 粒子间距。 在 $<T_c$ 时,
\begin{equation}
N=N_0+\frac{V}{2\pi^3}\int \text{d}^3{\bvec k}\frac{1}{\E^{\beta\epsilon_{\bvec k}}-1},\quad N_0/N\neq0~
\end{equation}
呈现宏观占据。 看得出来我们虽然很熟悉什么是理想玻色气体的BEC,但是对于一般的情况,如何规定它是否是处于凝聚的态呢?

首先,我们可以写出某个多体($N$ 体)波函数 $\Psi({\bvec r}_1,{\bvec r}_2,\cdots,{\bvec r}_N)$ 密度矩阵 $\rho$。 它的矩阵元有 $2N$ 个指标, $\langle x_1,\cdots,x_N|\rho|y_1,\cdots,y_N\rangle$。 把它trace掉 $N-1$ 个指标,得到
\begin{equation}
\rho({\bvec r},{\bvec r'}) = N\int\sum_s p_s\psi^*({\bvec r},{\bvec r}_2,\cdots,{\bvec r}_N)p_s\psi({\bvec r'},{\bvec r}_2,\cdots,{\bvec r}_N)\text{d}^3{\bvec r}_2\cdots\text{d}^3{\bvec r}_N~.
\end{equation}
二参数函数可以按照某一个基展开,
\begin{equation}
\rho({\bvec r},{\bvec r'}) = \sum_i N_i \varphi_i^*({\bvec r})\varphi_i({\bvec r'})~.
\end{equation}
对于 $N_i$,如果所有 $N_i/N=0$, 那么就称这个态是normal phase。 如果有一个不等于 $0$ 的,那就是一般的BEC,如果有多个 $N_i\neq0$ 叫 fragment BEC。 后两者称之存在非对角长程序(off-diagonal long-range order,ODLRO)。

对于正常的BEC的情况, $\exists N_0\gg N_{i\neq0}$,如果我们忽略两体相互作用的细节,我们就可以写出
\begin{equation}
\rho({\bvec r,\bvec r'}) \approx N_0\varphi^*({\bvec r})\varphi({\bvec r'})~.
\end{equation}
其中,满足要求的系统的态是
\begin{equation}\label{eq_BECSup_5}
\Psi({\bvec r}_1,{\bvec r}_2,\cdots,{\bvec r}_N) = \prod_{\otimes}\varphi({\bvec r}_i)~,
\end{equation}


我们考虑系统的Hamiltonian。 利用赝势,我们写出
\begin{equation}
\hat{H} = \sum_i^N \left(\frac{\nabla_i^2}{2m}+V({\bvec r}_i)\right)\sum_{\langle i,j\rangle}\frac{4\pi\hbar^2 a_s}{m}\delta^3({\bvec r}_{ij})\frac{\partial}{\partial r_{ij}}r_{ij}~.
\end{equation}


显然,我们的波函数\autoref{eq_BECSup_5} 在这里面并不存在我们上一章讨论的 $r_i\to r_j$ 的那种发散行为,因为我们实际上忽略了两体相互作用的细节。 当然,这种情况下我们很容易发现,没有奇异性的时候,赝势仅仅就是一个 $\delta$ 而已,因为其形如 $r_{ij}\partial_{r_{ij}}$ 的成分在 $\delta^3({\bvec r}_{ij})$ 下不作任何贡献:
\begin{equation}
\delta^3({\bvec r}_{ij})\frac{\partial}{\partial r_{ij}}r_{ij} = \delta^3({\bvec r}_{ij})~,
\end{equation}
我们可以通过对 $\langle \varphi|H+\mu N|\varphi\rangle$ 取极值得到本征值方程。
\begin{equation}
\varepsilon = N\int \text{d}^3{\bvec r}\varphi^*({\bvec r})\left(-\frac{\hbar^2\nabla^2}{2m}+V({\bvec r})\right)\varphi({\bvec r})+\frac{N(N-1)}{2}\int \text{d}^3{\bvec r}\frac{4\pi\hbar^2a_s}{m}|\varphi({\bvec r})|^4~.
\end{equation}
以 $\varphi^*$ 作为变量, $N\to\infty$ 而且 $\phi = \sqrt{N}\varphi, U\equiv \dfrac{4\pi\hbar^2a_s}{m}$, 我们有
\begin{equation}
\Rightarrow \left(-\frac{\hbar^2\nabla^2}{2m}+V({\bvec r})\right)\phi + U|\phi|^2\phi = \mu\phi~,
\end{equation}
这就是著名的\textbf{不含时 Gross-Pitaevskii 方程}。

如果我们考虑的不是一个定态问题,而是 $\phi = \phi({\bvec r},t)$, 那么我们的这个方程实际上要稍作改动。 符合直觉的我们可以得到 \textbf{含时 Gross-Pitaevskii 方程}
\begin{equation}\label{eq_BECSup_10}
\I\hbar\frac{\partial\phi}{\partial t} = \left(-\frac{\hbar^2\nabla^2}{2m}+V({\bvec r})\right)\phi + U|\phi|^2\phi~,
\end{equation}

接下来我们做两个魔法操作。 首先将 $\phi^*\times$ \autoref{eq_BECSup_10} 减去它的共轭,显然其左手边为
\begin{equation}
\I\hbar\phi^*\frac{\partial\phi}{\partial t} + \I\hbar\phi\frac{\partial\phi^*}{\partial t} = \I\hbar\frac{\partial\phi^*\phi}{\partial t} = \I\hbar\frac{\partial|\phi|^2}{\partial t}~,
\end{equation}
而右手边则为
\begin{equation}
\begin{split}
&\left(\phi^* \left(-\frac{\hbar^2\nabla^2}{2m}+V({\bvec r})\right)\phi + U|\phi|^4\right) -\left(\phi \left(-\frac{\hbar^2\nabla^2}{2m}+V({\bvec r})\right)\phi^* + U|\phi|^4\right) \\
=& \frac{\hbar^2}{2m}\nabla\cdot\left(\phi\nabla\phi^* - \phi^*\nabla\phi\right)~.
\end{split}
\end{equation}
于是,就得到
\[\I\hbar\frac{\partial|\phi|^2}{\partial t} = \frac{\hbar^2}{2m}\nabla\cdot\left(\phi\nabla\phi^* - \phi^*\nabla\phi\right) ~.\]
定义 $\phi({\bvec r}) = \sqrt{\rho}\E^{\I\theta}$, 其中 $\rho = \rho({\bvec r}),\theta = \theta({\bvec r})$。
我们从而有
\begin{equation}\frac{\partial\rho}{\partial t} + \nabla\cdot(\rho{\bvec v}),\quad {\bvec v}\overset{\text{def}}{\equiv}\frac{\hbar}{m}\frac{\partial\theta}{\partial t}~, \end{equation}
这就得到了连续性方程。 当然,还得补充一个无旋方程
\begin{equation}
\nabla\times{\bvec v}=0~,
\end{equation}
另一个魔法是将 $\phi^*\times$ \autoref{eq_BECSup_10} % 未完成:这里应该有一个动词
它的共轭。 左手边为
\begin{equation}
\I\hbar\phi^*\frac{\partial\phi}{\partial t} - \I\hbar\phi\frac{\partial\phi^*}{\partial t} = \I\hbar\phi^{*2}\frac{\partial \phi/\phi^*}{\partial t} = -2\hbar\rho\frac{\partial\theta}{\partial t}~,
\end{equation}
而右边就有点复杂了,下面的最后一项会很复杂:
\begin{equation}
\begin{split}
&\left(\phi^* \left(-\frac{\hbar^2\nabla^2}{2m}+V({\bvec r})\right)\phi + U|\phi|^4\right) + \left(\phi \left(-\frac{\hbar^2\nabla^2}{2m}+V({\bvec r})\right)\phi^* + U|\phi|^4\right) \\
=& 2V\rho + 2U\rho^2 - \frac{\hbar^2}{2m}\left(\phi^*\nabla^2\phi + \phi\nabla^2\phi^*\right)~.
\end{split}
\end{equation}

我们显然希望能够提取出来一个 $\rho$,所以我们给出详细的计算过程:
\begin{equation}
\begin{split}
\nabla^2\phi &= \nabla^2\left(\sqrt\rho \E^{\I\theta}\right)\\
&=\nabla\cdot\left(\frac{\E^{\I\theta}}{2\sqrt\rho}\nabla\rho+i\sqrt{\rho} \E^{\I\theta}\nabla\theta\right)\\
&=\frac{\E^{\I\theta}}{2\sqrt\rho}\nabla^2\rho+\frac{\I \E^{\I\theta}}{\sqrt\rho}\nabla\theta\nabla\rho - \frac{\E^{\I\theta}}{4\rho^{3/2}}(\nabla\rho)^2-\sqrt\rho \E^{\I\theta}(\nabla\theta)^2+i\sqrt\rho \E^{\I\theta}\nabla^2\theta~.
\end{split}
\end{equation}
类似有
\begin{equation}
\begin{split}
\nabla^2\phi^* &= \nabla^2\left(\sqrt\rho \E^{-\I\theta}\right)\\
&=\nabla\cdot\left(\frac{\E^{-\I\theta}}{2\sqrt\rho}\nabla\rho-i\sqrt{\rho} \E^{-\I\theta}\nabla\theta\right)\\
&=\frac{\E^{-\I\theta}}{2\sqrt\rho}\nabla^2\rho-\frac{\I \E^{-\I\theta}}{\sqrt\rho}\nabla\theta\nabla\rho - \frac{\E^{-\I\theta}}{4\rho^{3/2}}(\nabla\rho)^2-\sqrt\rho \E^{-\I\theta}(\nabla\theta)^2-i\sqrt\rho \E^{-\I\theta}\nabla^2\theta~,
\end{split}
\end{equation}
从而有
\begin{equation}
\phi^*\nabla^2\phi+\phi\nabla^2\phi^* = \nabla^2\rho - \frac{1}{2\rho}(\nabla\rho)^2-2\rho(\nabla\theta)^2 = 2\sqrt\rho \nabla^2\sqrt\rho - 2\rho(\nabla\theta)^2~.
\end{equation}
从而我们得到原本的式子右边等于
\begin{equation}
-\frac{\hbar^2}{2m}\frac{1}{\sqrt\rho}+\frac{\hbar^2}{2m}(\nabla\theta)^2+V+U\rho~.
\end{equation}
将 $\nabla$ 作用到等式两边,得到
\begin{equation}
m\frac{\partial\bvec v}{\partial t} = -\nabla\cdot\left[-\frac{\hbar^2}{2m}\frac{1}{\sqrt\rho}\nabla^2\sqrt\rho + \frac{1}{2}m{\bvec v}^2+V+U\rho\right]~.
\end{equation}

以上的推导都是零温情况,只包含超流成分;有限温的问题可以加入正常流体的成分,从而变成二流体模型。 考虑一个均匀系统 $V(\bvec r) = 0$,基态有均匀的密度 $\rho_0$,且 ${\bvec v} = 0$。 我们可以对非基态的态的密度函数作展开: $\rho = \rho_0+\delta\rho$。 只取 $\delta\rho,{\bvec v}$ 和 $k$ 的领头阶,我们可以简化刚刚得到的方程:
\begin{equation}
m\frac{\partial\bvec v}{\partial t} = -U\nabla\cdot\delta\rho,\quad \frac{\partial\rho}{\partial t} = -\nabla(\rho{\bvec v}) \approx-\rho_0\nabla\bvec v~.
\end{equation}
联合起来就得到
\begin{equation}
\frac{\partial^2\delta\rho}{\partial t^2} = \frac{U\rho_0}{m}\nabla^2\delta\rho~,
\end{equation}
即得到了声子色散
\begin{equation}
\omega = \sqrt{\frac{U\rho_0}{m}}k~.
\end{equation}
声速得到为 $c=\sqrt{\dfrac{U\rho_0}{m}}$,是一个低能的线性 gapless 激发(从而无色散)。 这与 $\text{U}(1)$ 对称性是联系起来的。

最后我们来说一下超流与临界速度。 这个临界速度就和我们BCS里面的临界磁场有点像:超过这个速度,超流就不再是超流了。
考虑一个质量为 $m_0$,速度 ${\bvec v}_i$ 的杂质,在与某处于BEC态的物体相互作用。 如果有摩擦,也就是说,这个杂质被散射到某一速度 ${\bvec v}_f$ 了,我们可以在初始BEC质心系写能动量守恒
\begin{equation}
m_0{\bvec v}_i + {\bvec q} = m_0{\bvec v}_f,\quad \frac{m_0{\bvec v}_i^2}{2} = \frac{m_0{\bvec v}_i^2}{2} + c|{\bvec q}|~.
\end{equation}
得到
\begin{equation}
{\bvec v}_i\cdot{\bvec q}-c|{\bvec q}| = \frac{{\bvec q}^2}{2m}~.
\end{equation}
可见右边等式大于 $0$,如果 ${\bvec v}_i<c$,则左边无法满足也为正,即无法发生散射,也就是真正意义的“超流”(不会产生摩擦作用),这个临界速度也就是声速 $c$。
