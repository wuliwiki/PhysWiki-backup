% Toeplitz 矩阵
% keys Toeplitz矩阵
% license Usr
% type Tutor

\pentry{矩阵及其运算\nref{nod_Mat}}{nod_51f9}

Toeplitz矩阵是这样的矩阵,处于位置 $(i,j)$ 的矩阵元和 $(i+1,j+1)$ 的矩阵元具有相同值,直观上相当于“从左上到右下的 $45^\circ$ 斜线上”的矩阵元具有相同值。因此,只要给出矩阵第一列和第一行的元素,其它元则根据“斜线”规则确定。

\begin{definition}{Toeplitz矩阵}\label{def_ToepM_1}
设 $A$ 是 $m\times n$ 的矩阵,若对任意 $i\in m,j\in n$,矩阵元 $a_{ij}=a_{i+1,j+1}$ 恒成立,则称 $A$ 为\textbf{Toeplitz矩阵}。
\end{definition}

\begin{theorem}{}\label{the_ToepM_1}
若 $A$ 是 $m\times n$ 的Toeplitz矩阵,其元记作 $a_{ij}$。则
\begin{equation}
a_{ij}=\left\{\begin{aligned}
&a_{1,j-i+1},\\
&a_{i-j+1,1}.
\end{aligned}\right.~
\end{equation}
\end{theorem}

\textbf{证明:}由Toeplitz矩阵\autoref{def_ToepM_1} ,可知:若 $i\leq j$,则
\begin{equation}
a_{ij}=a_{i-1,j-1}=\cdots=a_{i-(i-1),j-(i-1)}=a_{1,j-i+1}.~
\end{equation}
而若 $i\geq j$,同理有 $a_{ij}=a_{i-j+1,1}$。

\textbf{证毕!}

若把行指标当 $x$ 轴,列指标当 $y$ 轴,原点取作 $(1,1)$ 构建的平面坐标系。则Toeplitz矩阵是说坐标位于过点 $(1,i)$ 或 $(i,1)$ 的斜率为 $-45^\circ$ 的直线上的矩阵元,取相同值。这些线就是开头提到的“从左上到右下的 $45^\circ$ 斜线”。

\begin{example}{}\label{ex_ToepM_1}
矩阵
\begin{equation}
(1),
\begin{aligned}
\left(
\begin{array}{cc}
 1 & 3 \\
 2 & 1 \\
 3 & 2 \\
\end{array}\right)
\end{aligned},
\left(
\begin{array}{ccccc}
 4 & \frac{1}{2} & 3 & 4 & 5 \\
 0 & 4 & \frac{1}{2} & 3 & 4 \\
 -3 & 0 & 4 & \frac{1}{2} & 3 \\
 4 & -3 & 0 & 4 & \frac{1}{2} \\
\end{array}
\right).~
\end{equation}
都是Toeplitz矩阵的例子。
\end{example}

\subsection{Mathematica实操}
Mathematica软件提供了构建Toeplitz矩阵的基本函数,其语法为 ToeplitzMatrix[a,b],其中 $a$ 告诉Mathematica矩阵的第一列元素,$b$ 则是第一行元素。例如下面的例子给出了\autoref{ex_ToepM_1} 中的第二个矩阵。
\begin{lstlisting}[language=mathematica, caption=Mathematica构造Toeplitz矩阵]
ToeplitzMatrix[{1, 2, 3}, {1, 3}]
\end{lstlisting}
当然,有了Toeplitz矩阵的定义,和\autoref{the_ToepM_1} ,我们完全可以自己写出构造Toeplitz矩阵的代码。这可以通过For循环嵌套For循环实现。
\begin{lstlisting}[language=mathematica, caption=Mathematica自写Toeplitz矩阵代码]
(*l,r分别代表第一列和第一行*)
Tope[l_, r_] := 
  Module[{ll = Length[l], lr = Length[r]},
  (*若第一列和第一行元素不同,则打印警告,并取第一个元素为第一行的第一个元素*)
    If[l[[1]] != r[[1]], Print["Warning:the column element ",
        l[[1]] , " and row element ", 
        r[[1]], " at positions 1 and 1
        are not the same. Using row element."], {}
      ];
   (*构建矩阵A,使得第一列为l,第一行为r*)
    A = Table[0, {i, 1, ll}, {j, 1, lr}];
    (*利用定理1构造矩阵其它元素*)
    A[[All, 1]] = l; A[[1]] = r;
    For[i = 2, i <= ll, {
        For[j = 2, j <= lr,
            {If[j > i, A[[i, j]] = A[[1, j - i + 1]], 
            A[[i, j]] = A[[i - j + 1, 1]]]; j++ } 
           ];
         i++}
        ];
    A
    ]
\end{lstlisting}
执行,将给出自带函数 TopelitzMatrix相同的结果


