% 线性映射

\begin{issues}
\issueOther{重整中: 线性变换和线性映射改为同义词}
\end{issues}

% 未完成:从几何向量出发, 给出 “线性” 的定义, 给出一些平面线性变换的例子(包括投影变换), 然后再总结出一般的代数形式, 给出矩阵表示

\pentry{子空间\upref{SubSpc}}

\subsection{线性映射}
\begin{definition}{线性映射}\label{LinMap_def1}
给定域\upref{field} $\mathbb F$ (通常取实数域 $\mathbb R$ 或复数域 $\mathbb C$)上的线性空间 $V$ 和 $U$. 如果有映射\upref{map} $f:V\rightarrow U$满足: 对于任意的向量 ${v}_1, {v}_2\in V$ 和标量 $c_1, c_2 \in \mathbb{F}$,都有
\begin{equation}
f(c_1 {v}_1+c_2 {v}_2)=c_1f({v}_1)+c_2f({v}_2)
\end{equation}
则称$f$是$V$到$U$的一个\textbf{线性映射(linear map)}, 也叫\textbf{线性变换(linear transform)}, 或者\textbf{线性算符(linear operator)}.
\end{definition}

\autoref{LinMap_def1} 的内涵比看上去广一些, 不仅仅是对$V$中两个向量的线性组合成立. 容易证明对于任意的一组有限个向量$\{{v}_i\}\subseteq V$和一组对应的标量$\{c_i\}\subseteq\mathbb{F}$,都有
\begin{equation}\label{LinMap_eq1}
f\qty(\sum_i c_i {v}_i)=\sum_i c_i f({v}_i)
\end{equation}
注意一些教材中只把定义域和值域相同的线性映射 $f:V\to V$ 称为线性变换, 本书中不做这种区分.

\begin{exercise}{几何矢量的线性变换}
请证明 $N$ 维几何矢量空间 $V_N$ 中, 以下变换 $f:V\to V$ 都是线性变换:
\begin{itemize}
\item 缩放: 把矢量乘以一个固定标量 $\lambda$.
\item 旋转: 把矢量绕任意固定轴逆时针旋转固定角度 $\theta$.
\item 镜像: 把矢量关于某条过原点的固定直线做镜像对称.
\item 投影: 把二维矢量垂直投影到过原点的固定直线, 或者把三维矢量垂直投影到过原点的平面(也可以拓展到 $N$ 维空间中的矢量投影到 $N-1$ 维平面).
\item 以上几种变换的任意组合.
\end{itemize}
\end{exercise}
注意在以上映射中, 除了投影外都是双射\upref{map}, 且都把零向量映射到零向量. 多对一映射也会把一些非零向量映射到零向量.

\begin{exercise}{}
请证明线性变换的线性变换仍然是线性变换.
\end{exercise}

在映射\upref{map}中, 我们把一个映射的像定义为值域中所有被映射到的元素的集合.
\begin{theorem}{值空间}
线性映射 $f:V\to U$ 的像 $f(V) \subseteq U$ 是一个线性空间, 即 $U$ 的子空间\upref{SubSpc}. 称为 $f$ 的\textbf{值空间(range space)}.
\end{theorem}
证明留作习题, 使用线性空间的定义\upref{LSpace}即可.

\addTODO{这两段应该放到后面展开讲解, 太浓缩}
如果$\{{e}_i\}_{i=1}^n$是$V$的一组基底, 那么任意${v}\in V$都可以唯一地表示为${v}=\sum_i c_i {e}_i$的形式,其中$c_i\in\mathbb{F}$. 这样,由于线性性,$f({v})=\sum_ic_if({e}_i)$.也就是说,只需要知道了基向量被 $f$ 映射到哪里,也就可以计算出任意向量映射到哪里.于是,和线性函数一样,确定一个线性映射的时候,我们最多只能自由选择基向量映射到哪里,只不过这里的函数值不再是数字,而是$U$中的向量.

在向量空间的表示\upref{VecRep}中我们还会看到,选定两个空间的基以后,一个线性映射也可以看成是多个线性函数的排列,因此线性映射和线性函数性质很相似. 线性函数是一种特殊的线性映射, $V$ 上的所有线性函数组成的向量空间叫做\textbf{对偶空间}\upref{DualSp}.

\subsection{线性变换}

\addTODO{以下需要改:线性变换和映射是同义词!}
要知道一个具体的线性变换的性质,我们只需要知道它把基向量都映射到哪里去了.我们知道,基向量张成线性空间$V$本身,而任意一组基向量变换后所得到的向量组,张成\upref{VecSpn}了线性变换的\textbf{像空间(image space)},即$V$中全体向量变换后的结果之集合.如果基向量变换后的向量组还是线性无关的,那么这个向量组也是一组基,从而像空间就是$V$本身;但有的时候变换后的向量组线性相关了,这时候的像空间就是$V$的一个真子集了.极端情况下甚至把所有向量都变换为零向量,这样的线性变换的像空间就只有零向量了.

如果一个线性变换的像空间是$V$的一个真子集,那么这个真子集一定是$V$中的一个“过原点的平面”,我们把这种线性变换称为\textbf{退化(degenerate)}的.反过来,$V$中的一个“过原点的平面”可以是某线性变换的像空间.一个线性变换唯一对应一个像空间,但是一个像空间总是对应无穷多个线性变换.

如果${v}\in V$在基$\{{e}_i\}$中表示为$\sum c_i {e}_i$,那么它变换后的向量就是$T {v}=T\sum c_i {e}_i=\sum c_iT {e}_i$.如果向量组$\{T {e}_i\}$ 是线性相关的,那么就意味着存在并非全为零的 $\{c_i\}$使得$T {v}=\sum c_iT {e}_i=0$;而$\{c_i\}$ 并非全为零意味着${v}$不是零向量.这就是说,退化的线性变换会把一些非零向量变换成零向量.

\begin{exercise}{}
如果线性变换非退化,那么非零向量有可能被变换为零向量吗?提示:退化和非退化的区别在于基做了线性变换后得到的向量组还是不是线性无关的,或者说还是不是一组基.
\end{exercise}

退化的线性变换会把非零向量变成零向量这一事实,引申出了零化子的概念:

\begin{theorem}{零空间}\label{LinMap_the1}
给定线性映射 $f:V\to U$, $V$ 中被变映射到 $U$ 中零向量的全体向量所构成的集合 $\{{v}\in V|f(v)= 0\}$ 构成 $V$ 的子空间, 称为 $f$ 的\textbf{零空间(null space)}, 也称为\textbf{零化子空间}或\textbf{零化子}.
\end{theorem}

\subsection{总结与拓展}

到目前为止,我们对于向量的描述依然是抽象的,并未涉及许多具体的性质,如向量垂直、向量长度、向量坐标等概念.

在将来的词条中,我们会看到如何在给定了具体的基时用矩阵来描述向量和线性变换,以及当基的选择变化时,这些矩阵该如何变化.

零化子空间是一种子空间,而我们也会在将来的词条中解释子空间的概念.
