% 深度学习 CNN 入门
% keys 深度学习
% license Usr
% type Tutor
神经网络是一种受到生物神经系统启发的人工智能模型,它重现了大脑中神经元之间相互连接的方式。神经网络在诸多领域中取得了显著成就,如图像识别、自然语言处理和语音识别等。这篇博客将为您解释神经网络的构造,让您能够理解这个令人着迷的领域的基本工作原理。
\subsection{第一部分:神经元 }
首先需要了解神经元,这是神经网络的基本构建块。

1.神经元的结构:每个神经元都由细胞体、树突和轴突组成。细胞体包含核心部分,树突接收来自其他神经元的信号,而轴突将信号传递给其他神经元。
2.信号传递:神经元之间的通信是通过电化学信号完成的。当信号通过树突传递到细胞体时,如果达到一定阈值,神经元就会触发并将信号传递给下一个神经元。
\begin{figure}[ht]
\centering
\includegraphics[width=12cm]{./figures/c714d3a47bfcc267.png}
\caption{神经元} \label{fig_CNN1_1}
\end{figure}

\subsection{第二部分:神经元数学模型 }
