% 最大似然估计
\begin{issues}
\issueTODO
\end{issues}
\pentry{随机变量 概率分布函数\upref{RandF},定积分\upref{DefInt}}
在统计学中,最大似然估计(maximum likelihood estimation,缩写为MLE),也称极大似然估计,是用来估计一个概率模型的参数的一种方法.

下边的讨论要求读者熟悉概率论中的基本定义,如概率分布、概率密度函数、随机变量、数学期望等.读者还须先熟悉连续实函数的基本技巧,比如使用微分来求一个函数的极值(即极大值或极小值).
最大似然估计的原理
给定一个概率分布$D$,已知其概率密度函数(连续分布)或概率质量函数(离散分布)为$f_D$,以及一个分布参数$\theta$ ,我们可以从这个分布中抽出一个具有$n$个值的采样$X_{1},X_{2},... ,X_{n}X_1, X_2,..., X_n$,利用$f_D$计算出其似然函数:
$$
L(\theta|x_1,...x_n) = f_{\theta}(x_1,...x_n)
$$
若{\displaystyle D}D是离散分布,{\displaystyle f_{\theta }}{\displaystyle f_{\theta }}即是在参数为{\displaystyle \theta }\theta 时观测到这一采样的概率.若其是连续分布,{\displaystyle f_{\theta }}{\displaystyle f_{\theta }}则为{\displaystyle X_{1},X_{2},\ldots ,X_{n}}X_1, X_2,\ldots, X_n联合分布的概率密度函数在观测值处的取值.一旦我们获得{\displaystyle X_{1},X_{2},\ldots ,X_{n}}X_1, X_2,\ldots, X_n,我们就能求得一个关于{\displaystyle \theta }\theta 的估计.最大似然估计会寻找关于{\displaystyle \theta }\theta 的最可能的值(即,在所有可能的{\displaystyle \theta }\theta 取值中,寻找一个值使这个采样的“可能性”最大化).从数学上来说,我们可以在{\displaystyle \theta }\theta 的所有可能取值中寻找一个值使得似然函数取到最大值.这个使可能性最大的{\displaystyle {\widehat {\theta }}}\widehat{\theta}值即称为{\displaystyle \theta }\theta 的最大似然估计.由定义,最大似然估计是样本的函数.

注意
这里的似然函数是指{\displaystyle x_{1},x_{2},\ldots ,x_{n}}x_1,x_2,\ldots,x_n不变时,关于{\displaystyle \theta }\theta 的一个函数.
最大似然估计不一定存在,也不一定唯一.
