% 蒙特卡洛树搜索算法(实现 TicTacToe 机-机对战)
% keys 蒙特卡洛树搜索算法
% license Usr
% type Tutor

\pentry{蒙特卡洛树搜索算法(理论)\nref{nod_MCTS}}{nod_d3cd}

前面已经讨论过蒙特卡洛树搜索算法的理论,下面通过讲解例题进行实战练习,这利于我们更深入地理解这算法。首先回顾例题:
\begin{example}{}
使用蒙特卡洛树搜索算法实现一个\textbf{机器 vs. 机器}的 Tic-Tac-Toe 井字棋对战游戏。Tic-Tac-Toe 有以下规则:
\begin{enumerate}
\item 井字棋棋盘是 $n \times n$ 的正方形网格棋盘,例如下面是一个有一些棋子的 $7 \times 7$ 的棋盘:
\begin{figure}[ht]
\centering
\includegraphics[width=6cm]{./figures/4d8426da153f2ff1.png}
\caption{棋盘例} \label{fig_MCTSci_2}
\end{figure}
\item 游戏的下法是有 $\cross$、$\bigcirc$ 两方,每次都可以在没有棋子的正方形内部落子。输赢定义为:最先有\textbf{连续} $m$ 个我方棋子出现的一方获胜。
\item \textbf{连续}的定义是:横向、纵向,或两个 $45^\circ$ 对角线方向。
\end{enumerate}

你需要使得以下内容是\textbf{可以自定义}的:
\begin{enumerate}
\item $n$ ,即棋盘大小可以自定义。
\item $m$ ,即输赢的(棋子连续数)条件可以自定义。
\item 先下棋的一方可以自定义。也就是谁第一步下棋可以自定义。
\end{enumerate}

这是一个工程问题,不用考虑时间限制。
\end{example}
在实现蒙特卡洛树搜索这个算法前,我们需要先做一些准备,定义好游戏的各种内容类。
