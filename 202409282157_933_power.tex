% 幂函数(高中)
% keys 幂函数
% license Usr
% type Tutor

\begin{issues}
\issueDraft
\end{issues}

\pentry{函数\nref{nod_functi},函数的性质\nref{nod_HsFunC},}{nod_9bbd}

在初中阶段,已经学过的几类典型的函数形式:正比例函数$f(x) = ax$、反比例函数$f(x) = \frac{k}{x}$和二次函数$f(x) = ax^2+bx+c$,其实都是幂函数的特例。可以这么说,幂函数是初中唯一接触过的函数。在高中阶段,将扩展这些知识,进一步研究幂函数的更一般形式。

\begin{definition}{幂函数}
形如
\begin{equation}
f(x) = x^a~.
\end{equation}
的函数称作\textbf{幂函数},其中 $a\in\mathbb Q$。
\end{definition}
幂函数的参数在指数位置,自变量在底数位置,幂函数的名称指的就是自变量的幂次是函数值,注意不要与指数函数相混淆。$a$会影响函数的性质的各个方面,在学习时需要时刻注意。本文除最后一章,全都默认$a\in\mathbb Q$。

由于幂函数的样式很多,直接研究每个具体的形式会很混乱。但对$y=x^a$,在$x>0$时,总是有定义的,而且此时$y$一定为正。因此,函数一定会在第一象限有图象。本文会先讨论定义域以及奇偶性,奠定好研究基础,然后再专注研究其在第一象限的规律。这也反映了一般分析函数的过程。

\subsection{一个特例}

在正式的研究开始之前,需要先提及一个特例,当 $a = 0$ 时,幂函数会退化为常值函数 $f(x) = 1(x\neq0)$,该函数在所有非$0$处值恒为 $1$,图像是一个在$0$处无定义的水平直线。它是在定义域上恒为正的偶函数。

这里有必要讨论一下$0^0$。尽管在$x\neq0$时,$x^0=1$和$0^x=0$都成立,严格来说,$0^0$在高中阶段是未定义的,但在大部分场合(例如幂级数展开)会默认其值为$1$。事实上,如果从幂运算的最基础定义来讲,一般会先定义$0^0=1$来保证归纳得到的其他幂次不会出现多余的参数。这里也可以理解为,幂运算是比指数运算更底层的运算,因此要优先保证幂运算$x^0=1$成立。

\subsection{定义域与奇偶性}

取$\displaystyle a=\frac{n}{m}$($n,m$互质,$m>0$)的形式进行讨论。

\subsubsection{定义域}

通常来讲,幂函数的定义域是$x\in\mathbb{R}$,但是在一些特殊情况时,定义域需要调整。

首先,$0$当分母的运算是非法的,为了逻辑一致性,幂运算要求$0^{-k},k\in \mathbb{N}$时不存在。因此,如果$n<0$,则幂函数的定义域为$({-\infty},0)\cup(0,{+\infty})$。

其次,在实数域上负数的偶数次方根均无意义,为了逻辑一致性,幂运算要求$\displaystyle(-x)^\frac{1}{2k}$,在$k\in \mathbb{N},x\in\mathbb{R}^+$时不存在。因此如果${m}$为偶数,则幂函数的定义域为$[0,{+\infty})$。

\subsubsection{奇偶性讨论}

根据定义域的讨论,只有$m$为奇数时,定义域才是关于$0$对称的,因此,针对奇偶性的讨论将会在$m$是奇数的前提下进行。

由幂运算的定义始终有$(-x)^2=x^2,(-x)^1=-(x)^1$,即$a=2$时是偶函数,$a=1$时是奇函数。下面以此为基础讨论其他形式的幂次的奇偶性:

\begin{itemize}
\item 若$a$为正整数,则取$a=2k+b$,$k$为任意自然数,$b=0$时$a$为偶数,$b=1$时$a$为奇数。从而$(-x)^{2k+b}=(-1)^b(x)^{2k+b}$,结果取决于$b$的取值。代入可知,$a$为偶数时是偶函数,$a$为奇数时是奇函数。
\item 若$a$为正分数,则取$t=x^\frac{1}{m}$,有$x^\frac{n}{m}=t^{n}$,结论与之前相同。
\item 若$a$为负数,则取$t=x^{-1}$,有$x^a=t^{|a|}$,结论与之前相同。
\end{itemize}

整理一下,幂函数的奇偶性分为三种情况:

\begin{itemize}
\item 偶函数:$n$是偶数,$m$是奇数。
\item 奇函数:$n$是奇数,$m$是奇数。
\item 非奇非偶:$m$是偶数。
\end{itemize}

事实上,这也是奇偶性名称最直观的体现,它与$n$的奇偶性相同。$n=0$时也符合这里的条件。

总结本章的内容可知,幂函数的形态与$n,m$是否同号以及奇偶关系有关,因此在遇到幂函数时,首先要关注的就是这两个数字的性质。

\subsection{在第一象限的幂函数}

在第一象限时,$x>0$。下面的讨论是在这个基础上的。

分别研究$x$在$0$附近、$x$趋向于无穷的性质,$x<1$和$x>1$的性质单调性,反函数关系,还有比较$a_1,a_2$与$-1,0,1$的不同关系时的函数之间的关系。

将定义域限定在$(0,+\infty)$时,由于$\displaystyle(x^a)^{\frac{1}{a}}=x$,所以$\displaystyle x^{\frac{1}{a}}$与$x^a$ 互为反函数。也即二者的图象关于$y=x$对称,而这条对称轴本身也是$a=1$的情况,即它与自身互为反函数。

如果将第一象限以$x=1,y=x,y=1$三条线分成六个区域。这三条线的交点是$(1,1)$,由于不论$a$取何值,$1^a=1$,因而幂函数一定会过点$(1,1)$。如果将$x=1,y=x$之间大于$1$的部分记作区域$\rm{I}$,则可顺时针得到$\rm{I, II, III, IV, V, VI}$一共六个区域。

$x^a$运算的几个性质:
x>1,a>0



当$a_1,a_2>1$时,对x>1有x^a_1>x^a_2

单调性:
a>0递增,a<0递减。

\autoref{fig_power_1} 展示了幂函数 $f(x) = x^a$ 第一象限的函数图象。

\begin{figure}[ht]
\centering
\includegraphics[width=8cm]{./figures/86604297d1436480.pdf}
\caption{实参数的幂函数(相同颜色的函数互为反函数)}\label{fig_power_1}
\end{figure}


\subsection{*复数域扩展}

\pentry{复数\nref{nod_CplxNo},三角函数\nref{nod_HsTrFu}}{nod_b678}

在实数域上,无理数次幂和偶次方根只定义在正数上。因此,函数只在第一象限有图象,但是根据欧拉公式,引入复数后将函数拓展到复数域中可以有一些新知,下面略窥这些新想法。请注意,下面的内容在高中阶段完全不需要理解,甚至在本科基础阶段都不会涉及。此处给出只是为了扩展视野,如果想要具体了解需要学习\enref{复变函数}{Cplx}。

首先,不加证明地给出\textbf{欧拉公式}\footnote{如果代入$x=\pi$,则会得到恒等式$e^{i\pi}+1=0$,这被很多人称为数学中最优美的等式之一。}:

\begin{theorem}{欧拉公式}
\begin{equation}
\forall x\in\mathbb{R},e^{ix} = \cos(x) + i\sin(x)~.
\end{equation}
\end{theorem}

注意到,等号右侧是一个模为$1$的复数。因此,任意复数可以表示为 $z = r (\cos \theta + i \sin \theta)$,称$r$ 是复数$z$的\textbf{模},$\theta$ 是复数与实轴的夹角,称为\textbf{辐角},由于正弦与余弦都是周期函数,所以$\theta+2k\pi,k\in\mathbb{Z}$都是辐角,称呼$\theta\in[0,2\pi)$为\textbf{辐角主值}。于是,通过欧拉公式有\textbf{复数的指数形式}:

\begin{equation}
z = r e^{i(\theta+2k\pi)},k\in\mathbb{Z}~.
\end{equation}

\subsubsection{负数偶次方根的情况}

假设要计算负负实数的偶次方根 $(-x)^{\frac{1}{4}},x>0$,由于 $-x$的辐角为 $\pi$,模为$|x|$,将 $-x$ 写成指数形式为:
\begin{equation}
-x = |x| e^{i\pi(1+2k)},k\in\mathbb{Z}~.
\end{equation}

接下来,使用幂运算公式:

\begin{equation}
\displaystyle
(-x)^{\frac{1}{4}} = \left( |x| e^{i\pi(1+2k)} \right)^{\frac{1}{4}} =|x|^{\frac{1}{4}} \cdot e^{i\pi\frac{1+2k}{4}}=|x|^{\frac{\pi}{4}}\cos(\frac{1+2k}{4}\pi) + i|x|^{\frac{\pi}{4}}\sin(\frac{1+2k}{4}\pi),k\in\mathbb{Z}~.
\end{equation}

由于正、余弦函数的周期性,与无理数部分不同,这只会产生四个不同的解,分别为:

\begin{itemize}
\item $\displaystyle k = 0,\quad |x|^{\frac{1}{4}} \left( \cos\frac{\pi}{4} + i\sin\frac{\pi}{4} \right) = |x|^{\frac{1}{4}} \cdot \frac{\sqrt{2}}{2} \left( 1 + i \right)$
\item $\displaystyle k = 1,\quad |x|^{\frac{1}{4}} \left( \cos\frac{3\pi}{4} + i\sin\frac{3\pi}{4} \right) = |x|^{\frac{1}{4}} \cdot \frac{\sqrt{2}}{2} \left( -1 +i \right)$
\item $\displaystyle k = 2,\quad |x|^{\frac{1}{4}} \left( \cos\frac{5\pi}{4} + i\sin\frac{5\pi}{4} \right) = |x|^{\frac{1}{4}} \cdot \frac{\sqrt{2}}{2} \left( -1 - i \right)$
\item $\displaystyle k = 3,\quad |x|^{\frac{1}{4}} \left( \cos\frac{7\pi}{4} + i\sin\frac{7\pi}{4} \right) = |x|^{\frac{1}{4}} \cdot \frac{\sqrt{2}}{2} \left( 1 - i \right)$
\end{itemize}

其实,有理数次幂函数 $x^{n/m}$ ($x\in \mathbb R$, $n$ 为整数, $m$ 为正整数) 总是有 $m$ 个可能的值
\begin{equation}
x^{n/m} = \leftgroup{
&\abs{x^n}^{1/m}\E^{\I 2\pi k/m} & (x^n > 0)\\
&\abs{x^n}^{1/m}\E^{\I 2\pi (k+1/2)/m} & (x^n < 0)
}\qquad (k = 0,1,\dots, m-1)~.
\end{equation}

\subsubsection{无理数次幂}

假设要计算负实数的无理数($\pi$)次幂 $(-x)^\pi,(x>0)$。由于 $-x$的辐角为 $\pi$,模为$|x|$,将 $-x$ 写成指数形式为:
\begin{equation}
-x = |x| e^{i\pi(1+2k)},k\in\mathbb{Z}~.
\end{equation}

接下来,使用幂运算公式:

\begin{equation}
(-x)^\pi = \left( |x| e^{i\pi(1+2k)} \right)^\pi =|x|^\pi \cdot e^{i\pi^2(1+2k)}=|x|^\pi\cos(\pi^2(1+2k)) + i|x|^\pi\sin(\pi^2(1+2k)),k\in\mathbb{Z}~.
\end{equation}

可以看出,负实数的$\pi$次幂是复数,且是由于$\pi^2(1+2k)$与正、余弦函数的周期$2\pi$没有最小公倍数,使得得到的复数有无穷多个。其他无理数次幂也依此计算。事实上,在这样的观点下,正实数的无理数次幂,除了幅角主值对应的是一个实数之外,其余的都是复数。

上面的内容带你对复变函数初窥门径,希望能够让你感受到数学的魅力。