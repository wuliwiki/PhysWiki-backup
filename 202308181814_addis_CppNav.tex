% C++ 导航
% license Usr
% type Map

\subsection{环境设置}
首先你需要选择一个用于编译和运行你的 C++ 程序的环境。 首先你需要选择一个命令行环境。 对于小时百科的绝大部分使用 C++ 的场景来说,强烈推荐你使用 Linux\upref{Linux} 系统。 如果你不想装一个完整的操作系统,也可以在 Windows 中安装 WSL\upref{WSLnt}(推荐 WSL2) 或者 Mingw-w64\upref{Mingw}, 又或者 VirtualBox\upref{VirBox} 等虚拟机。

如果你属于比较硬核的用户,那么你进行 C++ 的整个过程中需要的就只是一个文本编辑器(如 Visual Studio Code\upref{VScode} 甚至 vim\upref{Vim})以及一个类 Linux 命令行窗口。 你需要在上面安装一个 C++ 编译器, 比较流行的有 g++\upref{gpp}, clang++\upref{clangp} 和 icpc\upref{icpcNt}(前两个开源,icpc 闭源)。 该做法的一个入门教程见 “在 Linux 上编译 C/C++ 程序\upref{linCpp}”。

另一个学习 C++ 过程中强烈推荐的工具是 Cling\upref{Cling}。 它可以让你像使用 Python 一样互动地运行一些简单的 C++ 代码,可以逐行运行。 它提供一个命令行版本,也可以在 Jupyter Notebook 中运行。

如果你不想在本地部署环境,那么网上也会有一些在线的 C++ 编译器, 例如 \href{https://wandbox.org/}{Wandbox}。 也可以使用一些算法题网站如 LeetCode 或者牛客网,它们甚至还提供逐行调试功能(下面会介绍)。

如果你还想要更多的功能,例如定义的跳查,用 GUI 进行调试, 那么你需要一个 IDE (集成开发环境, Integrated Development Environment)。 Visual Studio 是 Windows 上的一个常用的 IDE, 但是无法在其他系统中使用, 且社区版的 Visual Studio 免费。 老版本的 Visual Studio 只能使用 Windows 专有的 Visual C++ 编译器进行 C++ 开发, 但新版本的 Visual Studio 支持使用 WSL 中的其他编译器。 由于不能跨平台,我们不推荐使用 Visual Studio。 另一个流行的 IDE 是 JetBrain 推出的 CLion。 CLion 是一个跨平台的 IDE, 功能较多较完善,学生可以免费使用,但对其他用户收费。 我们推荐 CLion。 另外也有一些免费的 IDE, 但功能上相对没那么完善,外观也有所欠缺。

IDE 的部署通常要比直接使用命令行麻烦得多。 它们通常需要通过 “工程” 来进行项目管理。

\subsubsection{调试程序}
如果你选择简单的文本编辑器和命令行的环境, 那么命令行调试器几乎是必备的, 详见 “调试 C++ 程序\upref{gdbcpp}”。 当然有许多不那么专业的用户选择完全不使用调试器。 g++ 和 icpc 编译器对应的调试器是 gdb\upref{gdbNt}, clang++ 编译器对应的是 lldb\upref{LLDB}, 二者的用法几乎一样。

\subsection{}