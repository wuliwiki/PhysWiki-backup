% 大尺度结构形成
% license Usr
% type Tutor


1.3.1 大尺度结构的形成

今天,宇宙在比目前视界更小的尺度上是非常不均匀的。例如,如星系调查结果所示,它是一个可观宇宙中所有星系的三维地图。仅凭肉眼就可以看出各种结构:团块、丝状、墙、空洞...,它们出现在不同的尺度上。其他天文观测,如Lyman-α森林、弱透镜测量、星系团计数等,也证实了宇宙是一个团块状的宇宙。定量地,从所有这些测量中可以提取出物质功率谱P(k)。它通过以下方式定义:
\begin{equation}
\langle \delta_k \delta_{k'} \rangle = (2\pi)^3 P(k) \delta^3(\mathbf k - \mathbf k')~.
\end{equation}
其中,$\delta_k$是密度对比$\delta(\mathbf r)$的傅里叶变换,而$\langle\rangle$表示对k方向的平均。狄拉克δ函数不要与表示扰动的δ混淆,意味着具有不同k ≠ k'的模式在统计上是独立的。这一特性可以从膨胀中理解,膨胀预测了平均统计特性,这些特性编码在P(k)中。由于$P(k)$是密度自相关函数的傅里叶变换,它在傅里叶空间方便地表达了物理空间中物质分布的不均匀性。P(k)的大(小)值意味着存在许多(少)具有特征尺寸的结构。测量结果因此表明,宇宙在所有尺度上都有一定的功率。为了使宇宙的团块状更加明显,有时将$P(k)$(单位为长度的三次方)重新缩放为无量纲的方差。小的对应于小的密度对比,而,例如,表示与平均密度相当的过密度。$P(k)$数据的快速操作表明,在大$k$区域确实上升到大值。换句话说,数据显示宇宙在小尺度(大$k$)上表现出大的不均匀性。在标准宇宙学模型中,原始的不均匀性是由具有小振幅的膨胀产生的。这从CMB的近乎完美平滑中得到证实,CMB基本上提供了一张在光子最后一次散射时宇宙光子内容的照片。这提出了一个问题:这些微小的原始不均匀性如何从如此小的振幅增长到我们今天观察到的大对比度?答案如上所述,密度扰动的增长主要受到暗物质的驱动。为了定量理解这一过程,我们需要概述早期宇宙中这些扰动的演化。

让我们考虑一个充满一般物质流体的宇宙,我们暂时不指定其确切性质。通过将完整的广义相对论计算近似为其牛顿极限,可以理解这种介质中密度扰动的演化主要特征。这是一个有效的描述,适用于在视界尺度上非相对论性物质的长度尺度。非相对论性流体完全由其密度 \( \rho(r, t) \),速度场 \( v(r, t) \) 以及压力的状态方程 \( \varphi(\rho) \) 所表征。它的引力相互作用由牛顿势 \( \Phi \) 描述。这些量由以下演化(“欧拉”)方程描述:
\begin{equation}\label{eq_LSSFor_1} \begin{cases} 
\frac{\partial \rho}{\partial t} + \nabla \cdot (\rho v) = 0, & \text{连续性} \\
\frac{\partial v}{\partial t} + (v \cdot \nabla)v = -\frac{\nabla \varphi}{\rho} - \nabla \Phi, & \text{牛顿定律} \\
\nabla^2 \Phi = 4\pi G \rho, & \text{泊松方程}
\end{cases} ~.
\end{equation} 
这些方程构成了一组非线性微分方程。对于准均匀宇宙,通过对上述量展开为零阶均匀背景的一阶扰动,可以获得有用的解析信息:
\begin{equation} 
\rho = \rho_0(t) + \rho_1(x, t), \quad \varphi = \varphi_0 + \varphi_1, \quad v = v_0 + v_1, \quad \Phi = \Phi_0 + \Phi_1~. 
\end{equation}
我们首先考虑静态宇宙(无膨胀)的情况。\autoref{eq_LSSFor_1} 简化为一组描述扰动的耦合方程:
\begin{equation}
\begin{cases} 
\frac{\partial \rho_1}{\partial t} + \rho_0 \nabla \cdot v_1 = 0, & \text{(静态宇宙)} \\
\frac{\partial v_1}{\partial t} + \frac{v^2_s}{\rho_0} \nabla \rho_1 + \nabla \Phi_1 = 0, \\
\nabla^2 \Phi_1 = 4\pi G \rho_1,
\end{cases} ~.
\end{equation}
其中我们定义了量 \( v^2_s = \frac{\partial \varphi}{\partial \rho} = \frac{\varphi_1}{\rho_1} \),这被解释为流体中的速度(如我们片刻后将看到的)。通过对第一个方程进行时间导数操作,并使用第二和第三个方程进行适当的代换,可以得到密度扰动 \( \rho_1 \) 的演化方程,即Jeans方程:
\begin{equation}
\frac{\partial^2 \rho_1}{\partial t^2} - v^2_s \nabla^2 \rho_1 = 4\pi G \rho_0 \rho_1~.
\end{equation}
忽略重力(即,设G=0),解是密度波(即声音),以速度$v_s$传播。包括重力,完整的方程表达了压力项(左侧)与坍缩项(右侧)之间的竞争。Jeans长度 \( \lambda_J = \sqrt{\frac{v^2_s}{4\pi G \rho_0}} \) 区分了哪个项占主导:大尺度 \( \lambda > \lambda_J \) 的扰动随时间增长并增长,而小尺度 \( \lambda < \lambda_J \) 的扰动则得到压力的支持。随时间增长的坍缩模式是指数级的,\( \rho_1 \propto e^{\sqrt{4\pi G \rho_0} t} \),通过解微分方程(1.9)可以很容易地验证这一点,当$v_s \rightarrow 0$时。这是Jeans不稳定性,当应用于正常物质时,解释了气体云如何坍缩形成紧凑的天体,例如恒星。Jeans尺度的直观含义是,当云层足够大,其流体静压 \( \tau_{\text{pressure}} \sim \lambda_J / v_s \) 太慢而无法阻止引力吸引时,它就会坍缩。引力吸引在时间尺度 \( \tau_{\text{gravity}} \sim (G\rho_0)^{-1/2} \) 上聚集它。人们也可以定义Jeans质量为 \( M_J = \frac{4\pi \rho_0 \lambda^3_J}{3} \),即包围在半径 \( \lambda_J \) 的球体内的质量。质量 \( M > M_J \) 的扰动是“Jeans不稳定”的并且会坍缩。

 

