% 麦克斯韦方程组(综述)
% license CCBYSA3
% type Wiki

本文根据 CC-BY-SA 协议转载翻译自维基百科\href{https://en.wikipedia.org/wiki/Maxwell\%27s_equations}{相关文章}。

\textbf{麦克斯韦方程组},或称\textbf{麦克斯韦–赫维赛德方程组},是一组耦合偏微分方程,与洛伦兹力定律一起构成了经典电磁学、经典光学、电路和磁路的基础。这些方程为电学、光学和无线电技术(如发电、电动机、无线通信、透镜、雷达等)提供了数学模型。它们描述了电场和磁场如何由电荷、电流及场的变化产生。[注1] 这些方程以物理学家和数学家詹姆斯·克拉克·麦克斯韦的名字命名,他在1861年和1862年首次发表了包含洛伦兹力定律的早期方程形式。麦克斯韦最早使用这些方程提出光是一种电磁现象。方程的现代形式及其最常见的表述归功于奥利弗·赫维赛德。[1]

麦克斯韦方程组可以组合起来,展示电磁场的波动(波)如何以恒定速度 \( c \)(真空中为 299792458 m/s[2])传播。这种波动称为**电磁辐射**,其以不同的波长出现,从而产生从无线电波到伽马射线的辐射谱。

在偏微分方程形式和一致的单位制中,麦克斯韦的微观方程可以写成:
\[
\begin{aligned}
\nabla \cdot \mathbf{E} &= \frac{\rho}{\varepsilon_0} \\
\nabla \cdot \mathbf{B} &= 0 \\
\nabla \times \mathbf{E} &= -\frac{\partial \mathbf{B}}{\partial t} \\
\nabla \times \mathbf{B} &= \mu_0 \left(\mathbf{J} + \varepsilon_0 \frac{\partial \mathbf{E}}{\partial t}\right)
\end{aligned}~
\]
其中,\(\mathbf{E}\) 表示电场,\(\mathbf{B}\) 表示磁场,\(\rho\) 为电荷密度,\(\mathbf{J}\) 为电流密度,\(\varepsilon_0\) 为真空电容率,\(\mu_0\) 为真空磁导率。
\begin{figure}[ht]
\centering
\includegraphics[width=6cm]{./figures/fc8212bde3089bd9.png}
\caption{刻在麦克斯韦爱丁堡雕像上的麦克斯韦方程。} \label{fig_MAXS_1}
\end{figure}
这些方程有两个主要变体:
\begin{enumerate}
\item \textbf{微观方程}具有普遍适用性,但在常规计算中不便使用。它们将电场和磁场与总电荷和总电流联系起来,包括原子尺度上复杂的材料内部电荷和电流。
\item \textbf{宏观方程}定义了两个新的辅助场,用于描述物质的宏观行为,而不必考虑原子尺度的电荷和量子现象(如自旋)。但其使用需要实验确定的参数,以表征材料对电磁的响应。
\end{enumerate}
“麦克斯韦方程”一词通常也用于指等效的替代形式。基于电标势和磁标势的麦克斯韦方程版本常用于显式求解边值问题、分析力学或量子力学。协变形式(在时空中,而非分离的空间和时间)使得麦克斯韦方程与狭义相对论的兼容性显而易见。在高能物理和引力物理中常用的曲时空中的麦克斯韦方程与广义相对论相容。[注2] 实际上,阿尔伯特·爱因斯坦发展了狭义和广义相对论,以容纳麦克斯韦方程中光速不变的结果,遵循只有相对运动才具有物理意义的原则。

这些方程的发表标志着对之前分别描述的现象——磁、静电、光及相关辐射——的理论统一。从20世纪中期以来,人们理解到麦克斯韦方程并未精确描述电磁现象,而是更精确的量子电动力学理论的经典极限。
\subsection{方程的历史}  
主条目:麦克斯韦方程的历史
\subsection{概念描述} 
\subsubsection{高斯定律} 
\begin{figure}[ht]
\centering
\includegraphics[width=6cm]{./figures/c46a6d0d0e00cff7.png}
\caption{电场从正电荷指向负电荷} \label{fig_MAXS_2}
\end{figure}
高斯定律描述了电场与电荷之间的关系:电场从正电荷向外指向负电荷,电场通过闭合表面的净流出量与所包围的电荷成正比,包括由于材料极化产生的束缚电荷。该比例系数为真空介电常数。
\subsubsection{磁场的高斯定律}
\begin{figure}[ht]
\centering
\includegraphics[width=6cm]{./figures/528b4e2e05dd6692.png}
\caption{磁场的高斯定律:磁场线没有起点或终点,而是形成闭合回路或延伸至无穷大,正如图中电流环所产生的磁场所示。} \label{fig_MAXS_3}
\end{figure}
磁场的高斯定律指出,电荷没有磁性对应物,称为磁单极子;不存在孤立的北极或南极。[3] 相反,材料的磁场归因于偶极子,磁场通过闭合表面的净流出量为零。磁偶极子可以表示为电流回路或不可分离的相等且相反的“磁荷”对。具体来说,通过高斯面的总磁通量为零,磁场是一个螺线型矢量场。
\subsubsection{法拉第定律}
\begin{figure}[ht]
\centering
\includegraphics[width=10cm]{./figures/053be297f99c6caf.png}
\caption{在地磁风暴中,太阳风等离子体冲击地球磁场,引起磁场的时间变化,从而在地球的大气层和导电的岩石圈中感应出电场,这可能会导致电网不稳定。(非按比例绘制。)} \label{fig_MAXS_4}
\end{figure}
法拉第感应定律的麦克斯韦-法拉第形式描述了一个随时间变化的磁场如何与电场的旋度相关联。[3] 其积分形式表明,将电荷绕闭合回路移动所需的单位电荷功等于穿过所围表面的磁通量变化率。

电磁感应是许多电力发电机的工作原理:例如,旋转的条形磁铁会产生变化的磁场,并在附近的导线上生成电场。
\subsubsection{安培–麦克斯韦定律}
