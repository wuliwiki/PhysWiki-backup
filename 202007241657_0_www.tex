% www

教师:[邱锡鹏](http://nlp.fudan.edu.cn/xpqiu)  微博:[@邱锡鹏](http://weibo.com/xpqiu)

签如词性,也可以是一个标签如文本的类别
$$
x^i = (w_1...w_t...w_T)\\ 
y^i=(p_1...p_t...p_T)
$$
使用机器学习方法-即找到这样一个映射f,维数.
$$
\phi(x) = \left(
		\begin{array} {ccc} \phi_1(x) \\ \phi_2 (x) \\ .\\ .\\ \phi_m(x)
		\end{array}
	      \right)
$$
​	自然语言处理中类问题

$$
\hat y =sign((f(z))) = sign(\theta^Tz+\theta_0)
$$

sign 为符号函数,取判别函数f(z)的正负号,为方便,简写判别函数为
$$
f(z) = \theta^Tz+\theta_0 = \sum_{i=1}^{k}\theta_iz_i + \theta_0 = \sum_{i=0}^{k} = \hat \theta^T \hat z
$$
其和增光特征向量.

$$
\hat z = \left( 
		        \begin{array} {ccc} 1 \\  z_1\\ .\\ .\\  z_k
		        \end{array}
        	    \right) 
                =  \left( 
        	    \begin{array}{ccc}
		        1 \\ \\ z \\ \\  \\  
		        \end{array}
        	    \right)
$$

$$
\hat \theta = \left( 
		\begin{array} {ccc} \theta_0 \\  \theta_1\\ .\\ .\\  \theta_k
		\end{array}
        	\right) =  \left( 
        	\begin{array}{ccc}
		\theta_0 \\ \\ \theta \\ \\  \\  
		\end{array}
        	\right)
$$

后面的分类器用如下形式:
$$
\hat y = \mathop{argmax}_yf(\phi(x,y),\theta)  \hspace{5cm} 式3.2
$$
这含了样本x和类别y混合信息的特征向量例子:
$$
\phi_1(x,y) =\left\{
\begin{array}{11}
1 &  \textrm{if x contains 'stock' and y is 'eco'}\\
0 & \textrm{otherwise} 
\end{array}
\right.
$$

$$
\phi_2(x,y)=\left\{
\begin{array}{11}
1 &  \textrm{if x contains 'stock' and y is 'sport'}\\
0 & \textrm{otherwise} 
\end{array}
\right..
$$

其中大小的向量.

总之:
$$
\phi(x,y) = \left( 
		\begin{array} {ccc} \phi_1(x,y) \\  \phi_2(x,y) \\.\\.\\ \phi_{|V|*k}(x,y) 
		\end{array}
        	\right)=\phi(x) \otimes \phi(y)
$$
其中:
$$
\phi_k(x) =\left\{
\begin{array}{11}
1 &  \textrm{if x contains c}_k\\
0 & \textrm{otherwise} 
\end{array}
\right.
$$
词典中第k个词.
$$
\phi_m(y) =\left\{
\begin{array}{11}
1 &  \textrm{if y is label}_m\\
0 & \textrm{otherwise} 
\end{array}
\right.
$$
表示第m个标签.件随机
$$
\phi(x,y) = \sum_{i=1}^{L}\theta^T_i\phi(\vec{x},y_i)+\sum_{i=2}^{L}\theta_i^T\phi^{*}(\vec{x},y_{i-1},y_i)
$$
L为序列化输出y的长度.

其中征抽取方式如下-假定窗口为1:
$$
\phi_1(x,y_i) =\left\{
\begin{array}{11}
1 &  if\ x_{i+1}\ =\ 'a'\ and\ x_{i-1}\ =\ 'give'\ and \ y_{i}=P\\
0 & \textrm{otherwise} 
\end{array}
\right.
$$
其中取方式如下:
$$
\phi^*(x,y_{i-1},y_{i}) \approx \phi^*(y_{i-1},y_{i})=\left\{
\begin{array}{11}
1 &  if\ y_{i-1}\ =\ V\ and \ y_{i}=P\\
0 & \textrm{otherwise} 
\end{array}
\right.
$$

### 5 感知器
