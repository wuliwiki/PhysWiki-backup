% 相控阵
% license CCBYSA3
% type Wiki

(本文根据 CC-BY-SA 协议转载自原搜狗科学百科对英文维基百科的翻译)

\begin{figure}[ht]
\centering
\includegraphics[width=10cm]{./figures/67b1f779e41710d7.png}
\caption{相控阵的工作原理动态图它由一个由发射机(TX)供电的天线单元阵列(A)组成。每个天线的馈电电流通过由计算机(C)控制的移相器(φ)提供。移动的红线是每个天线单元发射的无线电波的波阵面示意图。单个波阵面是球形的,但是它们在天线前组合(叠加)形成一束在特定方向传播的平面波。移相器使无线电波在线路上依次延迟,因此每个天线发射波前的时间比它下面的天线晚。这导致产生的平面波与天线轴成θ角。通过改变相移,计算机可以立即改变光束的角度θ。大多数相控阵都有二维天线阵列,而不是上图的线性阵列,波束可以在二维方向上转向。无线电波的传播速度在视觉上减慢了。} \label{fig_XKZ_1}
\end{figure}

在天线理论中,相控阵通常指的是电子扫描阵列,这是一种由计算机控制的天线阵列,它产生的无线电波束可以在无需移动天线的条件下控制其指向不同的方向。[1][2][3][4][5][6][7][8] 在阵列天线中,来自发射机的射频电流以正确的相位关系馈送到各个天线,使得来自各个天线的无线电波相加在一起以增加预定方向的辐射增益,同时抵消以抑制其他方向的辐射。在相控阵中,来自发射机的功率通过被称为移相器的设备馈送到天线,移相器由计算机系统控制,计算机系统可以改变相位,从而将无线电波束导向不同的方向。由于阵列必须由许多小天线(有时几千个)组成才能获得高增益,相控阵主要适用于无线电频谱的高频端、超高频和微波波段,在这些波段中单个天线元件非常小。

相控阵被发明用于军事雷达系统,可以快速扫描雷达波束,以探测飞机和导弹。这些相控阵雷达系统现在得到了广泛的使用,相控阵正在向到民用领域扩展。相控阵原理也用于声学,相控阵声波传感器用于医学超声成像扫描仪(相控阵超声)、油气勘探(反射地震学)和军事声纳系统。[9]

术语“相控阵”在偶尔也用于馈电功率的相位以及天线阵列的辐射方向图固定的非阵列天线。[6][10] 例如,通过馈电产生特定的辐射模式的调幅AM广播无线电天线由多个主辐射器组成,也称为“相控阵”。

\su