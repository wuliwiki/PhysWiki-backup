% 状态量和过程量

\begin{issues}
\issueDraft
\end{issues}

系统关于时间的变化可以看作状态空间中一点经过的轨迹. 若定义一个量只和该点在状态空间的位置有关, 那么它就是状态量, 若它和状态空间中的运动轨迹有关, 它就是过程量.

一个常见的例子是, 一个系统的状态用矢量 $\bvec x = (x_1, x_2, \dots, x_N)$ 描述, 在一个特定的过程中, $x_i$ 都是时间的函数. 定义一个量为
\begin{equation}\label{StaPro_eq1}
Q_{12} = \int_{\mathcal L} \sum_i f_i(x_1, \dots, x_N) \dd{x_i} = \int_{t_1}^{t_2} \sum_i f_i(x_1, \dots, x_N) \dv{x_i}{t} \dd{t}
\end{equation}
$\mathcal L$ 表示状态点的 “运动方程” $x_i(t)$ ($i = 1,\dots, N$) 以及起点终点 $t_1, t_2$.

那么从定义上来说, $Q$ 是一个过程量, 但如果它可以表示为两个状态量之差, 那么\textbf{对这个系统}区分状态量和过程量将没有太大意义. 也就是说, \autoref{StaPro_eq1} 积分的结果不取决于路径, 只取决于初末状态. 此时我们可以另外定义一个状态量 $V(\bvec x)$, 那么总有
\begin{equation}
Q_{12} = V(\bvec x(t_2)) - V(\bvec x(t_1))
\end{equation}

但事实上远非\autoref{StaPro_eq1} 的积分都可以表示为两个状态量之差. 也就是积分的结果与路径的xing.
