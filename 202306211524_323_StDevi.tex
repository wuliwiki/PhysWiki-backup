% 标准差与方差
% keys 期望


\addTODO{本词条处于草稿阶段。}
\addTODO{加入目录}

%如果本词条定位是高中,那么建议不使用求和符号$\sum$。我没有细写,只是顺手写了一些别的词条要用到的知识。by Jier
\begin{figure}[ht]
\centering
\includegraphics[width=10cm]{./figures/abb95ec079618c45.pdf}
\caption{经典的正态分布示意图。三组数据的平均值虽然相同,但是他们的离散程度不同。} \label{fig_StDevi_1}
\end{figure}

标准差和方差用于衡量一组数据的\textbf{离散程度}。

直观地,如果所有数据都是相等的,我们就认为这组数据没有一点离散;但若数据平均值不变,各数据离平均值越远,我们就认为数据离散程度越高。因此,我们可以给出衡量离散程度的式子:

\begin{definition}{方差}\label{def_StDevi_1}
设有一组数据$\{x_i\}_{i=1}^n$,它们的平均值是$\mu=\frac{x_1+x_2+\cdots+x_n}{n}$。则定义这组数据的\textbf{方差}为
\begin{equation}\label{eq_StDevi_1}
\sigma^2 = \frac{(x_1-\mu)^2+(x_2-\mu)^2+\cdots+(x_n-\mu)^2}{n}~.
\end{equation}
\end{definition}

定义方差时使用平方,是为了避免向不同方向离开平均值的数据相互抵消。

\begin{definition}{总体标准差}
设有一组数据$\{x_i\}_{i=1}^n$,它们的平均值是$\mu=\frac{x_1+x_2+\cdots+x_n}{n}$。则定义这组数据的\textbf{总体标准差}为
\begin{equation}
\sigma = \sqrt{\frac{(x_1-\mu)^2+(x_2-\mu)^2+\cdots+(x_n-\mu)^2}{n}}~.
\end{equation}
\end{definition}

\begin{definition}{标准误差}
数据的\textbf{标准误差}定义为$\sigma_n=\sigma/\sqrt{n}$。
\end{definition}



%解释一下为什么会有样本标准差。


\begin{definition}{样本标准差}
设有一组数据$\{x_i\}_{i=1}^n$,它们的平均值是$\mu=\frac{x_1+x_2+\cdots+x_n}{n}$。则定义这组数据的\textbf{样本标准差}为
\begin{equation}
S = \sqrt{\frac{(x_1-\mu)^2+(x_2-\mu)^2+\cdots+(x_n-\mu)^2}{n-1}}~.
\end{equation}
\end{definition}


\subsection{与期望值的关系}

给定一组数据$\{x_i\}_{i=1}^n$,记它的平均值(期望值)为$\langle x_i \rangle$,即\autoref{def_StDevi_1} 里的$\mu$。有了这个符号,就可以方便地表示$\{x_i^2\}$的平均值了:$\langle x_i^2 \rangle$。

根据\autoref{eq_StDevi_1} ,
\begin{equation}
\begin{aligned}
\sigma^2 &= \frac{x_1^2+x_2^2+\cdots+x_n^2}{n}+\frac{n\mu^2}{n}-\frac{2\mu(x_1+x_2+\cdots+x_n)}{n}\\
&=\langle x_i^2 \rangle+\mu^2-2\mu\langle x_i \rangle\\
&=\langle x_i^2 \rangle-\langle x_i \rangle^2~.
\end{aligned}
\end{equation}
