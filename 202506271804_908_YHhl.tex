% 约翰·惠勒(综述)
% license CCBYSA3
% type Wiki

本文根据 CC-BY-SA 协议转载翻译自维基百科 \href{https://en.wikipedia.org/wiki/John_Archibald_Wheeler}{相关文章}。

约翰·阿奇博尔德·惠勒(John Archibald Wheeler,1911年7月9日-2008年4月13日)是美国理论物理学家。他在二战后基本上推动了美国对广义相对论研究兴趣的复兴。惠勒还曾与尼尔斯·玻尔合作,解释了核裂变的基本原理。他与格雷戈里·布赖特共同提出了“布赖特–惠勒过程”的概念。他最广为人知的成就是推广了“黑洞”这一术语,\(^\text{[1]}\)用以描述在20世纪早期就已被预测会因引力坍缩形成的天体,并且他还创造了“量子泡沫”“中子慢化剂”“虫洞”和“从比特中产生物质”等术语,\(^\text{[2]}\)并提出了“一电子宇宙”假说。斯蒂芬·霍金称惠勒为“黑洞故事的英雄”。\(^\text{[3]}\)

惠勒在21岁时于约翰斯·霍普金斯大学获得博士学位,导师是卡尔·赫茨费尔德。随后,他在国家研究委员会奖学金支持下,师从格雷戈里·布赖特(和尼尔斯·玻尔(学习。1939年,他与玻尔合作,利用液滴模型解释核裂变机制,并发表了一系列论文。二战期间,他在芝加哥的曼哈顿计划冶金实验室工作,帮助设计核反应堆;之后又前往华盛顿州里奇兰的汉福德基地,协助杜邦公司建造核反应堆。战争结束后,他回到普林斯顿任教,但在1950年代初再次进入政府体系,参与氢弹的设计与建造。他和爱德华·泰勒是热核武器的主要平民倡导者。\(^\text{[4]}\)

惠勒的大部分职业生涯都在普林斯顿大学物理系度过,他于1938年加入该校,并一直任职至1976年。在普林斯顿任教期间,他指导了46名博士生,是指导博士生数量最多的物理教授。

65岁时,惠勒离开普林斯顿大学。1976年,他被任命为德克萨斯大学奥斯汀分校理论物理中心主任,并一直任职至1986年退休,随后成为名誉教授。
\subsection{早年与教育}
约翰·阿奇博尔德·惠勒于1911年7月9日出生在佛罗里达州杰克逊维尔,父母是图书馆员约瑟夫·L·惠勒和梅布尔·阿奇博尔德(Mabel Archibald,昵称 Archie)惠勒。\(^\text{[5]}\)他是四个孩子中最年长的一个。他的弟弟约瑟夫获得了布朗大学的博士学位和哥伦比亚大学的图书馆学硕士学位;弟弟罗伯特在哈佛大学获得地质学博士学位,并在多家石油公司及多所大学担任地质学家;妹妹玛丽在丹佛大学学习图书馆学,后成为一名图书馆员。\(^\text{[6]}\)他们在俄亥俄州扬斯敦长大,但在1921至1922年间曾在佛蒙特州本森的农场居住一年,惠勒曾在那里的一所单间教室学校上学。回到扬斯敦后,他进入了雷恩高中学习。\(^\text{[7]}\)

1926年从巴尔的摩城市学院高中毕业后,\(^\text{[8]}\)惠勒凭借马里兰州提供的奖学金进入约翰斯·霍普金斯大学。\(^\text{[9]}\)1930年,他在美国国家标准局暑期工作期间发表了第一篇科学论文。\(^\text{[10]}\)1933年,他获得博士学位。他在导师卡尔·赫兹费尔德的指导下完成的论文题目是《氦的色散与吸收理论》。\(^\text{[11]}\)他获得了国家研究委员会奖学金,并于1933至1934年在纽约大学师从格雷戈里·布赖特学习,\(^\text{[12]}\)随后于1934至1935年前往哥本哈根师从尼尔斯·玻尔学习。\(^\text{[13]}\)在1934年发表的一篇论文中,布赖特和惠勒提出了“布赖特–惠勒过程”,即光子可能转化为物质(电子–正电子对)的机制。\(^\text{[9][14]}\)
\subsection{早期职业生涯}
1937年,北卡罗来纳大学教堂山分校聘请惠勒担任副教授,但他希望能与粒子物理领域的专家有更密切的合作。[15] 1938年,他拒绝了约翰斯·霍普金斯大学提供的副教授职位,选择了普林斯顿大学的助理教授职位。虽然职位较低,但他认为正在扩充物理系的普林斯顿是更好的职业选择。[16] 他一直在普林斯顿任教至1976年。[17]

在1937年的论文《用共振组结构方法对光核的数学描述》中,惠勒提出了S矩阵(scattering matrix,散射矩阵)的概念,即“一个连接任意特解积分方程渐近行为与标准形式解渐近行为的酉矩阵”。[18][19] 虽然惠勒本人并未继续深入发展这一概念,但在1940年代,维尔纳·海森堡将S矩阵理论发展成为粒子物理中的重要工具。[18]

1938年,惠勒与爱德华·泰勒一起研究玻尔的原子核液滴模型;[20] 他们在纽约举行的美国物理学会会议上展示了研究成果。惠勒在教堂山的研究生凯瑟琳·韦也在会上展示了论文,并在后续文章中详细说明了液滴模型在某些条件下的不稳定性。由于液滴模型本身的局限性,他们都错过了预言核裂变的机会。[21][22] 1939年,玻尔将莉泽·迈特纳和奥托·弗里施发现核裂变的消息带到美国。玻尔告诉了莱昂·罗森菲尔德,由他告知惠勒。[16]

随后,玻尔与惠勒开始合作,利用液滴模型来解释核裂变的机制。[23] 随着实验物理学家对核裂变的研究,他们发现了一些令人困惑的实验结果。乔治·普拉兹克(George Placzek)曾询问玻尔为何铀在快中子和慢中子作用下都能发生裂变。在与惠勒一同去开会的路上,玻尔突然意识到:在低能量下引发裂变的是铀-235同位素,而在高能量下发生裂变的主要是丰度更高的铀-238同位素。[24] 他们随后又合作撰写了两篇关于裂变的论文。[25][26] 他们的第一篇论文发表于1939年9月1日的《物理评论》,就在德国入侵波兰、第二次世界大战爆发的当天。[27]

1940年,惠勒考虑到正电子是向后穿越时间的电子这一观点,提出了“一电子宇宙”假说):宇宙中实际上只有一个电子,在时间中来回穿梭。他的研究生理查德·费曼(起初难以相信这一观点,但正电子是向后穿越时间的电子这一想法引起了费曼的兴趣,费曼后来在其费曼图中融入了时间可逆性的概念。[28]
\subsection{核武器}
\subsubsection{曼哈顿计划}
在日本轰炸珍珠港并将美国卷入第二次世界大战后不久,亚瑟·康普顿请求惠勒加入位于芝加哥大学的曼哈顿计划冶金实验室,惠勒接受了请求,并于1942年1月前往芝加哥。[27] 他加入了尤金·维格纳的团队,参与核反应堆设计研究。[29] 他与罗伯特·F·克里斯蒂合作撰写了题为《溶液中纯裂变材料的链式反应》的论文,这对钚的提纯过程具有重要意义。[30] 该论文于1955年12月解密。[31] 他为“中子慢化剂”这一术语命名,用以取代恩里科·费米原先使用的“slower downer”。[32][33]
\begin{figure}[ht]
\centering
\includegraphics[width=8cm]{./figures/c3bbba4d4051c3b3.png}
\caption{} \label{fig_YHhl_1}
\end{figure}
在美国陆军工兵部接管曼哈顿计划后,他们将核反应堆的详细设计和建造任务交给了杜邦公司(DuPont)。[34] 惠勒加入了杜邦的设计团队。[35] 他与杜邦工程师紧密合作,在芝加哥和杜邦总部所在地特拉华州威尔明顿之间频繁往返。1943年3月,他将家人搬到了威尔明顿。[36] 杜邦的任务不仅是建造核反应堆,还要在华盛顿州汉福德基地(Hanford Site)建造一个完整的钚生产综合设施。[37] 随着项目推进,1944年7月,惠勒再次搬家,将家人迁往华盛顿州里奇兰(Richland),在那里他在被称为“300区”的科研大楼工作。[30][36]

早在汉福德基地于1944年9月15日启动三座反应堆中的第一座B反应堆之前,惠勒就已经担心某些核裂变产物可能会成为“核毒物”,它们的积累会通过吸收大量链式反应所需的热中子而阻碍持续的核链式反应。[38] 在1942年4月的一份报告中,他预测只要没有裂变产物的中子俘获截面超过100,000巴恩,反应性的降低不会超过百分之一。[39] 当反应堆在运行中意外停堆,又在大约15小时后意外重新启动时,他怀疑是半衰期为6.6小时的碘-135及其衰变产物、半衰期为9.2小时的氙-135造成了问题。事实证明,氙-135的中子俘获截面远超过两百万巴恩。最终通过增加额外的燃料棒来烧尽这些“毒物”解决了这一问题。[40]

惠勒参与曼哈顿计划还有个人原因。他在意大利参战的弟弟乔(Joe)曾给他寄来一张明信片,简单写着:“快点。”\[41] 然而一切已为时已晚:乔在1944年10月战死。\[40] 惠勒后来写道:“我们当时离创造出结束战争的核武器如此接近。我当时无法停止思考,直到今天也未曾停止思考,战争本可能在1944年10月就结束。”乔去世时留下了遗孀和年幼的女儿玛丽·乔(Mary Jo),她后来嫁给了物理学家詹姆斯·哈特尔(James Hartle)。\[42]
