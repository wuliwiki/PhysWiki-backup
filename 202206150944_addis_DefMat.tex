% 正定矩阵

\begin{issues}
\issueTODO
\end{issues}

\pentry{厄米矩阵的本征问题\upref{HerEig}}

\footnote{参考 Wikipedia \href{https://en.wikipedia.org/wiki/Definite_matrix}{相关页面}.}\textbf{正定矩阵(positive definite matrix)}定义如下.
\begin{definition}{}
正定矩阵是一个对称矩阵对称矩阵\upref{SymMat}或厄米矩阵\upref{HerMat} $\mat A$, 对任意列向量 $\bvec v$ 满足 $\bvec v\Her \mat A \bvec v > 0$.
\end{definition}
其中 $\bvec v\Her$ 表示 $\bvec v$ 的厄米共轭\upref{HerMat}. 类似地, 也可以定义\textbf{半正定矩阵}(把 $>$ 替换为 $\geqslant$), \textbf{负定矩阵}(把 $>$ 替换为 $<$), \textbf{半负定矩阵}.

当 $\mat A$ 是对称矩阵时, 它对应一个二次型\upref{QuaFor} $q(v) = \bvec v\Tr \mat A \bvec v$.

\begin{theorem}{}
一个矩阵 $\mat A$ 是正定矩阵当且仅当其本征值都大于零. 半正定矩阵和(半)负定矩阵的定义也类似.
\end{theorem}

证明: 令 $\mat A$ 的本征矢为 $\{\uvec u_i\}$ (一组正交归一基底), 对应本征值为 $\lambda_i$(实数), 令 $\bvec v = \sum_i c_i \uvec u_i$. 那么
\begin{equation}
\bvec v\Her \mat A \bvec v = \sum_i \lambda_i \abs{c_i}^2
\end{equation}
可见若所有 $\lambda_i > 0$, 结果必然是正的. 若要求对任意 $c_1,c_2,\dots$, 等式都大于零, 那么也能得到所有 $\lambda_i > 0$.

\begin{example}{}
求二维厄米矩阵
\begin{equation}
H = \pmat{a & b\\ b^* & d}
\end{equation}
正定的充分必要条件.

用特征多项式直接求本征值
\begin{equation}
(\lambda - a)(\lambda - d) - \abs{b}^2 = 0
\end{equation}
$\lambda$ 必定有解, 利用求根公式, 两个解大于零的充要条件是
\begin{equation}
ad > \abs{b}^2, \qquad
a > 0
\end{equation}
\end{example}
