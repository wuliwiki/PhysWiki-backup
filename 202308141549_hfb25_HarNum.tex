% 调和数(基础)
% keys 调和数|数论
% license Usr
% type Tutor

\pentry{数论函数\upref{NumFun},Euler-Mascheroni 常数\upref{Masche}}

调和级数是有名的发散级数,也是最简单的发散级数之一。既然我们不能求出发散级数的值,那我们总能计算出它的部分和,或是它趋向无穷大的阶吧。但是实际上它的部分和没有特别简单的表达式,于是,一个简单粗暴的方法是,直接把调和级数的部分和定义为一个数论函数。

\begin{definition}{调和数}
第$n$调和数$H_n$定义为调和级数的第$n$部分和,即
\begin{equation}
H_n = \sum_{i=1}^n\frac{1}{n}~.
\end{equation}
\end{definition}

可以通过积分的方法给出$H_n$的阶。

\begin{theorem}{调和数的阶}
当$n\to\infty$时,
\begin{equation}
H_n = \log n + \gamma + O(1/n) ~.
\end{equation}
其中 $\gamma$ 称为 Euler-Mascheroni 常数。
\end{theorem}

\textbf{证明}:

\begin{equation}
\begin{aligned}
H_n & = \sum_{i=1}^n\frac{1}{n}\\
& = \sum_{i=1}^n\int_n^{n+1}\frac{1}{n}\dd{t}\\
& = \sum_{i=1}^n\int_n^{n+1}\qty(\frac{1}{n}-\frac{1}{t})\dd{t}+\int_1^{t+1}\frac{1}{t}\dd{t}\\
& = \log n+\sum_{i=1}^n\int_n^{n+1}\qty(\frac{1}{\lfloor t\rfloor}-\frac{1}{t})\dd{t}+\log\qty(1+\frac{1}{n})\\
& = \log n+\int_1^{n+1}\qty(\frac{1}{\lfloor t\rfloor}-\frac{1}{t})\dd{t}+O(1/n)~.
\end{aligned}
\end{equation}

我们来考察次项这个积分趋近无限时的表现,
