% 一致收敛与极限换序
% 一致收敛|换序|limit|微积分|数学分析

\pentry{一致收敛\upref{UniCnv}}

\subsection{极限换序}

函数、数列等有多个自变量存在时,我们有可能会需要考虑多重极限.比如说对于一个二元函数$f(x, y)$,设它的定义域是$x>0, y>0$,那么我们如何计算$x\to 0, y\to 0$时$f$的极限值呢?我们可以用以下式子来计算:
\begin{equation}\label{UniCo2_eq1}
\lim\limits_{y\to 0}\lim\limits_{x\to 0}f(x, y)
\end{equation}

其中,对于任意$y>0$,$f(x, y)$可以认为是关于$x$的一元函数,这样我们就可以计算出$\lim\limits_{x\to 0}f(x, y)$.对于每个$y>0$,可以定义$g(y)=\lim\limits_{x\to 0}f(x, y)$,因此对二元函数$f$进行第一个求极限后,得到的是一个一元函数$g$.这样,我们同样也能计算出$\lim\limits_{y\to 0}g(y)$,这也就是\autoref{UniCo2_eq1} .

从直观的几何角度来理解,$\lim\limits_{x\to 0}f(x, y)$就像是求出了$x$轴上的一个函数,然后$\lim\limits_{y\to 0}\lim\limits_{x\to 0}f(x, y)$就是这个函数的一个极限.

我们也可以反过来,用
\begin{equation}\label{UniCo2_eq2}
\lim\limits_{x\to 0}\lim\limits_{y\to 0}f(x, y)
\end{equation}
来计算$f$的二重极限,这个时候就相当于先求了$y$轴上的一个函数,再求它的极限.

随之而来的问题是,\autoref{UniCo2_eq1} 和\autoref{UniCo2_eq2} 的值一样吗?换个说法就是,$f(x, y)$的二重极限可以交换次序吗?答案时“不一定”,取决于函数的性质.\autoref{UniCo2_ex1} 就是一个反例.

\begin{example}{}\label{UniCo2_ex1}
考虑函数$f(x, y)=x^y$.我们有:
\begin{equation}
\lim\limits_{x\to 0}f(x, y)=0
\end{equation}
和
\begin{equation}
\lim\limits_{y\to 0}f(x, y)=1
\end{equation}

这就导致
\begin{equation}
\begin{aligned}
\lim\limits_{y\to 0}\lim\limits_{x\to 0}f(x, y)&=0\\
&\not = 1\\
&=\lim\limits_{x\to 0}\lim\limits_{y\to 0}f(x, y)
\end{aligned}
\end{equation}

因此,对于这个$f(x, y)$,极限是不可以随便换序的.

\end{example}



\subsection{一致收敛的极限换序}

可以极限换序的函数,性质非常良好.一致收敛的函数列就拥有这个良好的性质.

\begin{theorem}{}
设在开区间$(a, b)$上,函数列$\{f_n(x)\}$一致收敛.如果对于每个$n$,右极限$\lim\limits_{x\to 0^+}f_n(x)$都存在,那么就有
\begin{equation}
\lim\limits_{n\to\infty}\lim\limits_{x\to 0^+}f_n(x)=\lim\limits_{x\to 0^+}\lim\limits_{n\to\infty}f_n(x)
\end{equation}
\end{theorem}

\textbf{证明}:

$\lim\limits_{x\to 0^+}f_n(x)$关于编号$n$构成一个数列;$\lim\limits_{n\to\infty}f_n(x)$是一个函数.我们首先要证明$\lim\limits_{x\to 0^+}f_n(x)$收敛,这样才能保证$\lim\limits_{n\to\infty}\lim\limits_{x\to 0^+}f_n(x)$是有意义的.

首先定义$f_n(a)=\lim\limits_{x\to 0^+}f_n(x)$.考虑一致收敛的柯西收敛原理\autoref{UniCnv_the6}~\upref{UniCnv},可知当$f_n(x)$一致收敛时,

\textbf{证毕}.






















