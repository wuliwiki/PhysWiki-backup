% 集合的运算
% keys 交|并|对偶原理|de Morgan定理
% license Xiao
% type Tutor

\pentry{集合\upref{Set}}
通过 “集合\upref{Set}” 的预备知识,我们应该知道,集合是由元素构成的。这是说,在一开始讨论集合的时候,必须先行给出一个集合,才能继续进行有关的讨论。这是说当你说集合 $A$ 时,我们已经知道了它的元素。具体而言,当谈论集合 $A$ 时,你应该马上将其理解成 $A=\{a_1,\cdots,a_n,\cdots\}$,其中花括号里的是 $A$ 的元素。既然提到集合相当于指出它的元素,那同时就能判断哪些使它的元素,哪些不是它的元素,$a$ 是 $A$ 的元素记作 $a\in A$,$a$ 不是 $A$ 的元素记作 $a\notin A$,这些都是在提到集合的时候就同时表明了的。有了集合后,一个重要的任务是其上运算的定义及运算规则,这便是本节要介绍的。
\subsection{与 $\land$ 和 或 $\lor$}
同样在“集合\upref{Set}”里,已经知道集合间可以进行运算 $\cap$(交)、$\cup$(并)。现在来给出它们的具体定义。在此之前,需要用到两个逻辑学概念 $\land$(读作“与”) 和 $\lor$。
