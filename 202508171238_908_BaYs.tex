% 八元数(综述)
% license CCBYSA3
% type Wiki

本文根据 CC-BY-SA 协议转载翻译自维基百科\href{https://en.wikipedia.org/wiki/Octonion}{相关文章}。

在数学中,八元数是一种实数域上的赋范除代数,是一种超复数系统。八元数通常用大写字母 $\mathbf{O}$ 表示,也可以写作黑板粗体 $\mathbb{O}$。八元数具有 8 个维度,是四元数的 2 倍维数,而它们正是四元数的扩展。八元数是非交换的、非结合的,但满足一种较弱的结合性,即所谓的可交替性。此外,它们还具有幂结合性。

八元数不像四元数和复数那样广为人知,后者在研究和应用上更为广泛。八元数与数学中的一些例外结构(exceptional structures,需要进一步澄清)有关,其中包括例外李群。八元数在弦理论、狭义相对论和量子逻辑等领域都有应用。将 Cayley–Dickson 构造应用于八元数,可以得到十六元数。
\subsection{历史}
八元数是在 **1843 年 12 月**由 \*\*约翰·T·格雷夫斯(John T. Graves)\*\*发现的,他的灵感来自好友 \*\*威廉·罗恩·哈密顿(William Rowan Hamilton)\*\*发现四元数。就在格雷夫斯发现八元数前不久,他在 **1843 年 10 月 26 日**写给哈密顿的一封信中写道:

“如果凭借你的炼金术,你能炼出三磅黄金,为什么要止步于此呢?”

格雷夫斯把他的发现称为 **“octaves”(八度数)**,并在 **1843 年 12 月 26 日**写给哈密顿的信中提到这一点。他最早发表研究结果的时间,比 \*\*阿瑟·凯莱(Arthur Cayley)\*\*的文章稍晚一些。八元数也被 **凯莱**独立发现,有时被称为 **Cayley 数(Cayley numbers)**或**凯莱代数(Cayley algebra)**。哈密顿后来描述过格雷夫斯发现八元数的早期经过。
定义

八元数可以被看作是实数的 **八元组(octets 或 8-元组)**。
每一个八元数都是单位八元数的实线性组合:

$$
\{e_{0}, e_{1}, e_{2}, e_{3}, e_{4}, e_{5}, e_{6}, e_{7}\},
$$

其中,\$e\_{0}\$ 是标量或实数元,可以与实数 \$1\$ 对应。

也就是说,每个八元数 \$x\$ 都可以写成以下形式:

$$
x = x_{0} e_{0} + x_{1} e_{1} + x_{2} e_{2} + x_{3} e_{3} 
  + x_{4} e_{4} + x_{5} e_{5} + x_{6} e_{6} + x_{7} e_{7},
$$

其中系数 \$x\_{i} \in \mathbb{R}\$。

---

要不要我帮你直接排版成一份完整的 **LaTeX 源码**(带 `\section` 和公式环境),方便你直接编译成 PDF?
