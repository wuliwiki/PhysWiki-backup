% 包晶合金
包晶转变:由一个液相与一个固相生成另一种固相. $L + \alpha \rightarrow \beta$

在热力学中,往往只关心相的变化;但由于动力学因素,实际冷却时,各相往往形成一定的有序组织结构.本文一并简要讨论,以Pt-Ag合金的平衡冷却为例.

\subsection{包晶相图}
\subsubsection{$wAg=42.4\%$的合金} 

\begin{figure}[ht]
\centering
\includegraphics[width=14cm]{./figures/PERITC_1.png}
\caption{$w_{Ag}=42.4\%$的合金} \label{PERITC_fig1}
\end{figure}

全程的相转变:$L \rightarrow \alpha+\beta$

全程的组织转变:$L \rightarrow \beta + \alpha_{II}$

D以上,先发生匀晶反应 L→α

D处,发生包晶转变:固相α与液相L转为一个固相β L+α→β.此时相变完成后,L与α被均完全消耗,系统由β组成
\begin{itemize}
\item 包晶转变是恒成分转变,即包晶相变中,先后结晶的部分的成分一致
\item 包晶转变是恒温转变: f=2-3+1=0,系统没有自由度
\end{itemize}
D以下,随后发生脱熔反应.$\beta \rightarrow \alpha_{II}$

\subsubsection{$10.5\%<wAg<42.4\%$的合金} 
\begin{figure}[ht]
\centering
\includegraphics[width=14cm]{./figures/PERITC_2.png}
\caption{请添加图片描述} \label{PERITC_fig2}
\end{figure}
全程的相转变:$L \rightarrow \alpha+\beta$

全程的组织转变:$L \rightarrow \beta + \alpha_{II}$

2.  成分属于CD线的合金的转变
	 
	L⟶α_初+β_包+α_II+β_II
	先发生匀晶反应L→α
	2点处, 发生了包晶反应L+α→β.
	相变完成后,匀晶反应形成的α没有被完全消耗,而L被完全消耗,系统由α+β组成“馒头多汤少”
	随后脱溶

3.  成分属于PD线的合金的转变
	 
	L⟶β+α_II.β相有两个来源(包晶转变+匀晶匀晶反应)
	先发生匀晶反应 L→α
	3点处,发生包晶反应α+L→β.
	相变完成后,匀晶反应形成的α被完全消耗,但L没有被完全消耗,系统由β+L组成“汤多馒头少”
	随后发再生匀晶体反应 L→β.在4处,L被完全消耗,系统完全由β组成
	随后脱溶

4. 其余成分合金的转变
	简单的匀晶-脱溶转变
