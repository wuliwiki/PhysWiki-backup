% 蒙特卡罗方法(综述)
% license CCBYSA3
% type Wiki

本文根据 CC-BY-SA 协议转载翻译自维基百科\href{https://en.wikipedia.org/wiki/Monte_Carlo_method}{相关文章}。

蒙特卡罗方法(Monte Carlo methods),又称**蒙特卡罗实验**,是一类依赖**重复随机抽样**以获得数值解的广义计算算法。其基本思想是利用**随机性**来解决一些**本质上可能是确定性的问题**。该名称源自摩纳哥的**蒙特卡罗赌场**,该方法的主要发展者、数学家斯坦尼斯瓦夫·乌拉姆(Stanisław Ulam)受到他叔叔赌博习惯的启发而命名。

蒙特卡罗方法主要用于三类问题:优化问题**
2. **数值积分**
3. **从概率分布中抽样生成样本**

此外,它还可以用于对输入存在高度不确定性的现象进行建模,例如**评估核电站故障的风险**。蒙特卡罗方法通常通过**计算机模拟**实现,能为那些**无法解析求解或过于复杂的问题**提供近似解。

蒙特卡罗方法广泛应用于多个科学与工程领域,包括:
**物理、化学、生物学、统计学、人工智能、金融、密码学**,也被应用于社会科学,如:**社会学、心理学与政治学**。它被认为是**20世纪最重要且最具影响力的思想之一**,并推动了众多科学与技术突破的实现。

不过,蒙特卡罗方法也存在一些局限性和挑战,例如:

* **精度与计算成本之间的权衡**
* **维度灾难(curse of dimensionality)**
* **随机数生成器的可靠性**
* **结果的验证与确认**
