% 郑州大学 2003 年 考研 量子力学
% license Usr
% type Note

\textbf{声明}:“该内容来源于网络公开资料,不保证真实性,如有侵权请联系管理员”

\textbf{1.}在泡利矩阵 $\sigma$ 表象中,
$$\sigma_x = \begin{pmatrix}
0 & 1 \\
1 & 0
\end{pmatrix}, \quad
\sigma_y = \begin{pmatrix}
0 & -i \\
i & 0
\end{pmatrix}, \quad
\sigma_z = \begin{pmatrix}
1 & 0 \\
0 & -1
\end{pmatrix}~$$

求 $\sigma_x$, $\sigma_y$, $\sigma_z$ 的归一化本征函数。

\textbf{2.}$\hat{J}$ 为角动量算符,其分量为 $\hat J_x$, $\hat J_y$ 和 $\hat J_z$,引入 $\hat J_\pm = \hat J_x + i\hat J_y$,根据角动量基本对易关系:

\begin{enumerate}
    \item (1) 证明 $[\hat J^2, \hat J_{\pm}] = 0$, $[\hat J_z, \hat J_{\pm}] = \pm \hbar \hat J_{\pm}$。
    \item (2) 设 $\hat J_z |j, m\rangle = m\rangle = \hbar\sqrt{(j\mp m)(j\pm m+1)}|j, m\pm 1\rangle$,(其中,$|j,m\rangle$ 为 $\hat J^2$ 与 $\hat J_z$ 的共同本征态),写出 $j=1$ 时的 $\langle j, m | J_z | j, m' \rangle \quad \text{及} \quad \langle j, m | J_y | j, m' \rangle $的矩阵表示式。
\end{enumerate}
