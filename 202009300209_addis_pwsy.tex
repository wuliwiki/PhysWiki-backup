% 普物实验
\date{}
\makeatother
\begin{document}
	\pagestyle{fancy}
	\lhead{普通物理实验报告} 
	\chead{}
	\rhead{}
	\thispagestyle{fancy}
	\section{\Large{数据处理}}
		\begin{table}[ht]
			\centering
			\caption{生物显微镜放大倍数的测定}
			\label{倍数测定}
			\begin{tabular}{cccccc}
				\toprule
				试验次数 & n &起始$x_1$(mm) & 终止$x_2$(mm) & $ny_{1}^{'}$(mm) & $y_{1}^{'}$(mm)\\
				\midrule
				1 & 4 & 0.288 & 4.949 & 4.661 & 1.165\\
				2 & 5 & 0.280 & 6.060 & 5.780 & 1.156\\
				3 & 6 & 0.285 & 7.220 & 6.935 & 1.156\\
				平均 & & & & & 1.159\\
				\bottomrule
			\end{tabular}
		\end{table}
		\par
		学习了测微目镜的原理后我们得知,测微目镜中观察到的像为经过物镜放大后成在叉丝双线板上的像,因此我们
		根据双线板上的刻度就可以测量出标准测微尺的像每一格的平均宽度为$y_{1}^{'} = $ 1.159mm,而实际中标准测微尺的最小分度为$y = $ 0.100mm,可以计算出物镜的放大倍数为:
		$$\beta_o = \frac{y_{1}^{'}}{y} = 11.59$$
		\par
		从表\ref{倍数测定}可以看出,第一次的实验较第二、三次实验的数据有较明显的偏差,个人推测原因为测量
		器具的回程差引起的,因为我是在作实验途中才听到老师提醒,要注意仪器的回程差,而后才在操作中注意不在
		同一次测量中将旋钮向不同的方向拧.
		\begin{table}[ht]
			\centering
			\caption{改装了测微目镜的生物显微镜测量一待测光栅的空间频率}
			\label{测微目镜测量}
			\begin{tabular}{ccccccc}
				\toprule
				试验次数 & n & 起始$x_1$(mm) & 终止$x_2$(mm) & $ny^{'}$(mm) & $y^{'}$(mm) & $y = \frac{y^{'}}{\beta_o}$/mm\\
				\midrule
				1 & 5 & 2.600 & 5.500 & 2.900 & 0.580 & 0.050\\
				2 & 6 & 2.618 & 6.090 & 3.472 & 0.579 & 0.050\\
				3 & 7 & 2.620 & 6.668 & 4.048 & 0.578 & 0.050\\
				平均 & & & & & 0.579 & 0.050\\
				\bottomrule
			\end{tabular}
		\end{table}
		\par
		根据表\ref{倍数测定}数据计算得到的物镜倍数,再依据表\ref{测微目镜测量}中的测量数据,我们可以计算
		此待测光栅的空间频率为:
		$$\frac{1}{y} = 20.0mm^{-1}$$
		\begin{table}[ht]
			\centering
			\caption{读数显微镜测量一待测光栅的空间频率}
			\label{读数显微镜测量}
			\begin{tabular}{cccccc}
				\toprule
				试验次数 & n &起始$x_1$(mm) & 终止$x_2$(mm) & $ny$(mm) & $y$(mm)\\
				\midrule
				1 & 10 & 9.349 & 10.189 & 0.840 & 0.084\\
				2 & 20 & 10.273 & 11.939 & 1.666 & 0.083\\
				3 & 25 & 12.024 & 14.098 & 2.074 & 0.083\\
				平均 & & & & & 0.083\\
				\bottomrule
			\end{tabular}
		\end{table}
		\par
		根据表\ref{读数显微镜测量}可以计算得到此光栅的空间频率为:
		$$\frac{1}{y} = 12mm^{-1}$$\\
		\\
		\section{\Large{分析与讨论}}
			\subsection{改装了测微目镜的显微镜与读数显微镜的差异}
			\par
			从测量原理来看,测微目镜测量的是经过显微镜物镜放大过后的像,并不是物体本身的大小,因此需要标定
			(通过测量一个用作长度基准的标准测微尺来实现)物镜的放大倍数,再根据目镜中的读数计算出物体实际的
			大小;而读数显微镜相当于一个带着放大功能、分度值更小的刻度尺,其可以通过目镜本身的移动距离大小
			直接反应被测物体的真实大小.由于测微目镜需要一次物镜倍数标定,这样会引入额外的误差,因此测量
			精度不如读数显微镜.
			\par
			从测量方式来看,测微目镜是通过调节视野中的叉丝位置实现距离的测量的,因此视野范围限制了其量程
			,同时由于视野边缘会有较大的像差,因此对于视野边缘物体的测量可能并不准确;读数显微镜是通过目镜
			的移动来实现测量的,因此其有较大的量程,可以对视野外的物体实现连续的测量.但是值得注意的是,
			读数显微镜的物镜放大倍数并不高,对于较小的物体测量时会使观测者比较费力(比如说作实验的我),
			如果物体足够小,那么由于机械加工精度的限制,我们不可能直接使用螺杆的方式将物体的大小与实际距离
			直接对应(这里的加工精度有很多种情况,螺纹表面的光洁程度,螺纹之间的间距,摩擦面之间的实际距离
			等等),那么就必须要有一个放大测量其像大小,再计算出实际大小的过程了.
			\subsection{实验中测量误差的来源分析}
			\par
			从操作者角度来看:使用测微目镜时,如果没有固定好测微目镜,导致测量过程中光学间隔发生了变化,就
			会使得物镜的放大倍数变化而导致测量误差;如果测量中使用者不小心移动了待测光栅,就会导致明显的测
			量误差(感觉这个已经不能叫做误差了,应该叫错误);如果测量中没有保持叉丝移动的方向与光栅刻痕的
			方向垂直,就会导致测量结果偏大.
			\par
			从仪器的构造特点来看:测微目镜的误差很可能来自目镜与镜筒的接口,微小的位移或者旋转会使得测量出
			现较大的偏差;还有一个很重要的误差来源是做实验之前完全没有考虑到的,就是回程差,由于两种仪器都
			是通过旋转鼓轮来进行测量,而鼓轮内部一定会有微小的缝隙填充有润滑物质,导致鼓轮可以在不旋转的情
			况之下可以左右移动,从而造成比较明显的回程差.\\
			\\
		\section{\Large{收获与感想}}
			\par
			第一次的实验比我想象中的要简单一些,但是也发现了很多问题:书本上的内容过于理想化,只有实际操作
			一下才会知道问题比想象中的要多很多,一个很现实的问题就是在最初测定物镜倍数的时候,由于刻度刻在
			某一个面上,所以调焦距的时候并不一定能一步到位,当时为此还花费了很长时间.
			\par
			还有就是数光栅真的好累啊,心疼自己的高度近视的眼睛.
\end{document}