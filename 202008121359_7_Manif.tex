% 流形
% keys 拓扑|流形|欧几里得

\pentry{拓扑空间\upref{Topol}}

\textbf{实流形 (real manifold)} 是一种拓扑空间, 其每个点都有一邻域与与欧几里得空间中的开集同胚 (homomorphic).如果这些欧几里得空间是$n$维的, 那么就叫做$n$维流形. 因此,一个实流形可以看成是我们熟知的欧几里得空间“拼接”而成的.如果把欧几里得空间换成复数空间,即用复数平面取代每一根实数轴,那么所得到的对象称为一个\textbf{复流形(complex manifold)}.

\subsection{流形的定义和例子}

\begin{definition}{实流形}

设$N$是一个拓扑空间,满足Hausdorff分离性以及第二可数性\footnote{拓扑空间如果有一个可数拓扑基(即一个拓扑基,包含最多$\aleph_0$个基本开集),则称之为第二可数的.},且对于任意$x\in N$都存在$U\in\mathcal{T}_N$使得对于某个正整数$n$,有$U\approx\mathbb{R}^n$,那么我们称$N$是一个\textbf{实拓扑流形(real topological manifold)}.如果记$U$到$\mathbb{R}^n$的同胚映射是$\varphi$,那么称$(U, \varphi)$是流形$N$的一张\textbf{图(chart)}.如果图的一个集合$\{U_\alpha\}$覆盖了$N$,即$\bigcup\{U_\alpha\}$

\end{definition}

将以上定义中的$\mathbb{R}^n$替换为$\mathbb{C}^n$,我们就得到复流形的定义:

\begin{definition}{复流形}

设$N$是一个拓扑空间,满足Hausdorff分离性以及第二可数性,,且对于任意$x\in N$都存在$U\in\mathcal{T}_N$使得对于某个正整数$n$,有$U\approx\mathbb{C}^n$,那么我们称$N$是一个\textbf{复拓扑流形(complex topological manifold)}.

\end{definition}

从这些定义可看到,流形是“局部地”和欧几里得空间同胚的数学对象.低维欧几里得空间可以很方便地用我们的几何直觉来理解,大大方便了建立对于流形的直觉.

\begin{example}{$S^n$流形}

$n$维度球面是

\end{example}


