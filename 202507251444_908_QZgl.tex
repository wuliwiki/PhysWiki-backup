% 乔治·格林(综述)
% license CCBYSA3
% type Wiki

本文根据 CC-BY-SA 协议转载翻译自维基百科\href{https://en.wikipedia.org/wiki/George_Green_(mathematician)}{相关文章}。

乔治·格林(George Green,1793年7月14日-1841年5月31日)是一位英国数学物理学家,他于1828年撰写了《数学分析在电学与磁学理论中的应用论文》\(^\text{[2][3]}\)。这篇论文引入了多个重要概念,包括一个类似于现代“格林公式”的定理、如今在物理学中广泛使用的势函数思想,以及现在被称为“格林函数”的概念。格林是第一个构建电学与磁学数学理论的人,他的理论为詹姆斯·克拉克·麦克斯韦、威廉·汤姆逊(开尔文勋爵)等科学家的研究奠定了基础。他在势理论方面的工作与卡尔·弗里德里希·高斯的研究是并行发展的。

格林的人生经历极为非凡,因为他几乎完全是自学成才。他在童年时期仅接受过大约一年的正规教育,年龄在8岁到9岁之间。
\subsection{早年生活}
\begin{figure}[ht]
\centering
\includegraphics[width=6cm]{./figures/e426677496c6ac27.png}
\caption{} \label{fig_QZgl_1}
\end{figure}
格林出生并大部分时间生活在英格兰诺丁汉郡的斯尼顿镇,如今该地已划入诺丁汉市。他的父亲也叫乔治,是一位面包师,还建造并拥有一座砖结构的风车,用于碾磨谷物\(^\text{[1]}\)。

在年轻时,格林被形容为体质虚弱,不喜欢在父亲的面包房里干活。然而,他并没有选择的余地;就像那个时代许多孩子一样,他很可能从五岁起便开始每天劳作谋生。
\subsubsection{罗伯特·古达克学院}
在那个时代,诺丁汉只有大约25\%至50\%的儿童能接受任何形式的教育【需要引用】。大多数学校是由教会运营的主日学校,孩子们通常只会在其中学习一到两年。格林的父亲由于面包店生意成功,经济状况良好,注意到年幼的格林智力超群,于是在1801年3月将他送入位于上议会街的罗伯特·古达克学院就读。罗伯特·古达克是当时著名的科普作家和教育家,他出版了《论青少年教育》一书,其中写道,他关注的不是“男孩的利益,而是未来成年人的塑造”。对外行人来说,他似乎在科学与数学方面知识渊博,但细读他的论文和教学大纲便会发现,他的数学教学范围仅限于代数、三角和对数。因此,格林日后在数学领域所作出的那些展现出极其先进数学知识的贡献,不可能是在古达克学院期间获得的。他仅在该校待了四个学期(相当于一个学年),他的同时代人曾猜测,他之所以离开,是因为学院已无可教之物。
\subsubsection{从诺丁汉搬到斯尼顿}
1773年,格林的父亲搬到了诺丁汉,当时该地以环境宜人、道路宽敞著称。然而到了1831年,受工业革命初期的影响,城市人口已增长近五倍,诺丁汉成了英格兰最糟糕的贫民区之一。饥饿的工人经常爆发骚乱,面包师和磨坊主则因被怀疑囤粮抬高物价而成为众矢之的。

基于这些原因,1807年,乔治·格林的父亲在斯尼顿购置了一块土地,并在那里建造了一座“砖结构风力磨坊”,如今被称为“格林风车”。这座风车在当时的技术水平上相当先进,但却需要几乎全天候的维护,而这成为了格林此后二十年的负担。
\subsection{成年生活}
\subsubsection{磨坊主}
与面包烘焙一样,格林也觉得操控风车的工作烦人而单调。农田的谷物源源不断地运到磨坊门口,而风车的帆必须不断根据风速调整,以防强风损毁,同时在微风中尽可能提高转速。用于碾磨的磨石会持续互相摩擦,若缺粮则可能磨损过快甚至引发火灾。每月,重达一吨的磨石还需更换或修复一次。
\subsubsection{家庭生活}
1823年,格林与简·史密斯建立了关系。简是威廉·史密斯的女儿,后者被格林的父亲聘为磨坊经理。尽管格林与简·史密斯从未正式结婚,但简最终被称为简·格林,两人共同育有七个孩子,除第一个孩子外,其余在受洗时均使用了“格林”作为姓氏。最小的孩子出生于格林去世前13个月。格林在遗嘱中为其事实婚姻妻子和孩子们留有遗产\(^\text{[4]}\)。
\subsubsection{诺丁汉订阅图书馆}
格林三十岁时,成为了诺丁汉订阅图书馆的一员。该图书馆至今仍然存在,很可能是格林获得高等数学知识的主要来源。与传统图书馆不同,订阅图书馆仅对大约一百名订阅者开放,订阅者名单上的首位便是纽卡斯尔公爵。该图书馆根据订阅者的具体兴趣,提供他们所需的专业书籍与期刊。
\subsubsection{1828年论文}
\begin{figure}[ht]
\centering
\includegraphics[width=6cm]{./figures/659dbb6b00e6f629.png}
\caption{} \label{fig_QZgl_2}
\end{figure}