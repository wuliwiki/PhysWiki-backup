% 指数衰减
% keys 指数函数|放射|衰变
% license Xiao
% type Tutor

\begin{issues}
\issueDraft
\end{issues}

\pentry{一阶线性微分方程\upref{ODE1}}

\addTODO{举例说明在什么问题中出现:核衰变}

随着海平面以上高度的增加,大气压力大致呈指数级下降,约为每1000米12\% 。

\footnote{参考 Wikipedia \href{https://en.wikipedia.org/wiki/Exponential_decay}{相关页面}。}衰变的速率和总量成正比, 或者说每个粒子单位时间衰变的概率一样。
\begin{equation}
\dv{N}{t} = -\lambda N~.
\end{equation}
方程的解
\begin{equation}
N(t) = N_0 \E^{-\lambda t}~.
\end{equation}
半衰期 $T_h$ 定义为, 一半粒子发生衰变所需要的时间。 满足
\begin{equation}
N(T_h) = \frac{N_0}{2} \iff \E^{-\lambda T_h} = \frac{1}{2}~,
\end{equation}
解得
\begin{equation}
T_h = \frac{\ln 2}{\lambda}~.
\end{equation}
