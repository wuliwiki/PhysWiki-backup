% 开映射和闭映射
% keys 开映射|闭映射
% license Usr
% type Tutor

\pentry{拓扑空间\nref{nod_Topol}}{nod_dab5}
拓扑空间中,映射是\enref{连续}{Topo1}的当且将当对像空间的每一开集的原象是开集,或者像空间的每一闭集的原象是闭集。即若 $(X,\mathcal T_X),(Y,\mathcal T_Y)$ 是两个拓扑空间, $f:\mathcal X\rightarrow\mathcal Y$ 是连续的当且将当每一 $O_Y\in\mathcal T_Y$,$f^{-1}(O_Y)\in\mathcal T_X$。然而,我们可以问:在连续映射下,开集的像是否一定是开集?闭集的象是否是闭集?一些例子告诉我们,一般情况下回答是否定的。这就引出了开映射和闭映射的概念。而对理解这两个概念本身来说,我们无需知道连续映射的定义。

\subsection{连续映射下开(闭)集的像不是开(闭)集}



