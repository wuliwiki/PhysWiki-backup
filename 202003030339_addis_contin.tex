% 连续函数

\pentry{极限\upref{Lim}}

简单来说, \textbf{连续函数}定义为: 在某个区间内, 函数曲线是连续的. 例如常见的 $\sin x$, $\exp x$, $x^2$ 都在任意


函数 $f(x)$ 在某点 $x = x_0$ 处\textbf{连续}的定义是
\begin{equation}
\lim_{x \to x_0} f(x) = f(x_0)
\end{equation}
注意这里要求 $x$ 从左边和右边趋近于 $x_0$ 时的极限(即左极限和右极限)都成立. 如果一个函数在某区间的所有点都连续, 我们就说它\textbf{在这个区间连续}.

% 举例: 分段函数在断点不满足该要求
