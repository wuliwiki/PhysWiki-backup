% 蒙特卡洛树搜索算法
% keys 蒙特卡洛|搜索
% license Usr
% type Tutor

\begin{issues}
\issueDraft
\issueTODO
\end{issues}

% 暂且把 UCB 与 蒙特卡洛树搜索算法 合并,如果需要可以单开一篇详细介绍 UCB。

蒙特卡洛树是一种不同于 Alpha-Beta 剪枝的优化搜索方法。他与 Alpha-Beta 剪枝相比,更好的适用于诸如围棋一类的大型对弈游戏(即搜索空间过大,但搜索过程需要兼顾“探索”与“最优化”,尽量得到全局最优解,避免陷入局部最优解)。但与 Alpha-Beta 剪枝相比,由于过程中采用蒙特卡洛方法(可以理解为抽样检测),搜索的结果并不保证一定是最优解。

值得一提的是,\textbf{蒙特卡洛树搜索算法不是蒙特卡洛算法}。
\subsection{UCB 算法}
UCB,upper confidence bound,置信度上界,是蒙特卡洛树搜索中启发式搜索的一种常用方式,下面先来考察 UCB 算法。

简单来说 UCB 算法也是一种启发式搜索的方法,在选择子节点的时候优先考虑没有探索过的,如果都探索过就根据得分来选择,得分不仅是由这个子节点最终赢的概率来,而且与这个子节点玩的次数成负相关,也就是说这个子节点如果平均得分高就约有可能选中(因为认为它比其他节点更值得利用),同时如果子节点选中次数较多则下次不太会选中(因为其他节点选择次数少更值得探索),因此MCTS根据配置探索和利用不同的权重,可以实现比随机或者其他策略更有启发式的方法。