% 集合的极限
\pentry{集合\upref{Set},极限\upref{Lim}}

\subsection{特征函数}
如果我们有可数多个集合构成的族$\{A_n\}$,每个$A_n$都是编了号的集合,那么对于给定的元素,我们可以观察它是否存在于各$A_n$中.为了表达方便,我们用一个\textbf{特征函数},$\mathcal{X}$,来表达属于关系:对于元素$x$和集合$A_n$,当$x\in A_n$时$\mathcal{X}_{A_n}(x)=1$;当$x\not\in A_n$时$\mathcal{X}_{A_n}(x)=0$.

有了特征函数,我们就可以套用数列的极限来讨论集合的极限.

\subsection{上极限和下极限}
给定一列编好了号的集合:$\{A_n\}$.为了方便之后的讨论,先定义两列集合:$U_k=\bigcup_{i\ge k}A_i$,$J_k=\bigcap_{i\ge k}A_i$.