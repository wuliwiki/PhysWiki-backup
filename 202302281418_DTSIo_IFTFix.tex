% 隐函数定理的不动点证明

\pentry{隐函数定理\upref{ImpFun} 多元隐函数的存在定理\upref{Mulmp}
 巴拿赫不动点定理\upref{ConMap}}

\subsection{隐函数定理的紧凑表述}
我们可以将隐函数定理用比较紧凑的形式表达出来. 在如下的版本中, 已知的"隐式关系"$F(x,y)$是一个映射, 其中$x$是$n$维的, $y$是$m$维的, 映射的取值也是$m$维的. 这样一来, 对于给定的$x$, 求解隐式方程$F(x,y)=0$就相当于从$m$个方程求解$m$个未知量($y$的$m$个分量). 为了这个方程能够求解, 自然期望方程之间需要是"独立"的.
\begin{theorem}{隐函数定理}
设$(x_0,y_0)\in\mathbb{R}^n\times\mathbb{R}^m$是给定的点. 设在某个开集$B(x_0,R)\times B(y_0,R)$上定义了映射$F:U\to\mathbb{R}^m$, 满足如下条件:
\begin{enumerate}
\item $F(x_0,y_0)=0$.
\item $F(x,y)$对$y$是连续可微的, 而且雅可比矩阵
$$
\frac{\partial F}{\partial y}(x_0,y_0)
$$
是可逆的.
\end{enumerate}
则存在$r<R$以及映射$f:B(x_0,r)\to B(y_0,R)$, 使得$f(x_0)=y_0$, 而且$F(x,f(x))=0$. 换句话说, 在$x_0$的某个邻域内, 给定了$x$就可以唯一求解$y$, 从而$x\to y$确定了一个函数关系.
\end{theorem}

\subsection{证明}
除了词条 多元隐函数的存在定理\upref{Mulmp} 中给出的归纳证明之外, 还可以用不动点定理给出一个简洁的, 而且是构造性的证明.

