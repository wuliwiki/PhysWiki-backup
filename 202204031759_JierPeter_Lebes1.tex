% 非负函数的Lebesgue积分
% 实变函数|勒贝格积分

\pentry{可测函数\upref{MsbFun}}

Lebesgue积分的思路是对函数的值域进行分划,以相应值域的逆映射作为“柱底”.归根到底,Lebesuge积分还是要对定义域作分划的,但相比Riemann积分的直接对定义域作分划,Lebesgue积分的分划方式更任意.对于可测函数,Lebesgue积分的分划得到的“柱底”都是可测集.

我们就从将可测集划分为两两不交的可测子集入手,先研究这种分划的性质.

\subsection{可测集的分划}



\begin{definition}{可测分划}
设$E\in\mathbb{R}^n$是可测集.如果有限\textbf{族}$\{E_1, E_2, \cdots, E_n\}$中各$E_i$\textbf{两两不交}、都是$E$的子集、\textbf{可测},且$E=\bigcup^n_{i=1}E_i$,那么称集族$\{E_i\}_{i=1}^n$为可测集$E$的一个\textbf{分划},或者\textbf{可测分划}.
\end{definition}

如果$A=\{E_i\}_{i=1}^n$和$B=\{F_i\}_{i=1}^m$都是$E$的分划,那么易证$C=\{E_i\cap F_j|E_i\in A, F_j\in B\}$也是$E$的分划.称$C$是分划$A$和$B$的\textbf{合并}.

容易看到,$C$中存在每一个$E_i$的分划$\{E_i\cap F_j\}_{j=1}^m$,类似地也存在每一个$F_j$的分划,像是更细一层地进行分划.因此,如果分划$C$是$A$和另一个分划的合并,我们就称$C$是比$A$\textbf{更细}的分划,反过来$A$比$C$\textbf{更粗}.

\subsubsection{$\opn{m}E$<+\infty的情况}























