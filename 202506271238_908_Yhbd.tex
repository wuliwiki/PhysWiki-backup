% 约翰·巴丁(综述)
% license CCBYSA3
% type Wiki

本文根据 CC-BY-SA 协议转载翻译自维基百科 \href{https://en.wikipedia.org/wiki/John_Bardeen}{相关文章}。

\begin{figure}[ht]
\centering
\includegraphics[width=6cm]{./figures/c261931e347b9e06.png}
\caption{巴丁于1956年} \label{fig_Yhbd_1}
\end{figure}
约翰·巴丁(John Bardeen,1908年5月23日-1991年1月30日)\(^\text{[1]}\)是一位美国物理学家。他是唯一一位两次获得诺贝尔物理学奖的人:第一次是在1956年,与威廉·肖克利和沃尔特·布拉顿因共同发明晶体管而获奖;第二次是在1972年,与利昂·库珀和约翰·施里弗因提出超导性的微观理论——BCS理论——而再次获奖。\(^\text{[4][5]}\)

巴丁出生并成长于威斯康星州,在威斯康星大学获得电气工程的学士和硕士学位,随后在普林斯顿大学取得物理学博士学位。在参加第二次世界大战后,他曾在贝尔实验室从事研究工作,之后担任伊利诺伊大学教授。

晶体管的发明彻底改变了电子工业,使得从电话到计算机几乎所有现代电子设备的出现成为可能,也由此开启了信息时代。而巴丁在超导性领域的研究成果——为他赢得第二次诺贝尔奖——则被广泛应用于核磁共振光谱(NMR)、医学磁共振成像(MRI)以及超导量子电路等领域。

巴丁是仅有的三位在同一领域获得两次诺贝尔奖的人之一(另外两位是弗雷德里克·桑格和卡尔·巴里·夏普利斯,均为化学奖获得者),也是全球仅有的五位两度获诺贝尔奖的人之一。1990年,《生活》杂志将他评为“20世纪最具影响力的一百位美国人”之一。\(^\text{[6]}\)
\subsection{教育与早年生活}
巴丁于1908年5月23日出生在威斯康星州麦迪逊市。\(^\text{[7]}\)他是查尔斯·巴丁的儿子,后者是威斯康星大学医学院的首任院长。

巴丁就读于麦迪逊的威斯康星大学附属中学。他于1923年毕业,时年仅15岁。\(^\text{[7]}\)他本可以更早毕业,但由于他在另一所高中修读课程以及母亲去世,毕业时间被推迟了。1923年,巴丁进入威斯康星大学。在大学期间,他加入了齐达赛兄弟会,并靠打台球赚取部分会费。他还被接纳为工程荣誉学会Tau Beta Pi的成员。由于不想像父亲那样走学术路线,巴丁选择了工程专业。他还认为工程专业具有良好的就业前景。\(^\text{[8]}\)

巴丁于1928年在威斯康星大学获得电气工程学士学位。\(^\text{[9]}\)尽管他曾休学一年在芝加哥工作,但仍于1928年顺利毕业。\(^\text{[10]}\)他选修了所有令他感兴趣的研究生层次的物理与数学课程,因此用了五年时间完成学业,而不是通常的四年。这也为他提供了时间完成硕士论文,该论文由里奥·J·彼得斯(Leo J. Peters)指导。他于1929年在威斯康星大学获得电气工程硕士学位。\(^\text{[11][9]}\)

巴丁随后继续在威斯康星大学深造,但最终进入了位于匹兹堡的海湾石油公司旗下的研究部门——海湾研究实验室工作。\(^\text{[6]}\)从1930年到1933年,巴丁在该实验室从事磁力和重力勘测数据解释方法的研究工作。\(^\text{[7]}\)他的职位是一名地球物理学家。但随着这项工作逐渐失去吸引力,他申请并被普林斯顿大学的数学研究生项目录取。\(^\text{[8]}\)

作为研究生,巴丁学习了数学和物理。在物理学家尤金·维格纳的指导下,他撰写了一篇关于固体物理问题的论文。尚未完成论文时,他于1935年获得哈佛大学学者协会的初级会员职位。他随后在哈佛度过了三年时间(1935年至1938年),与未来的诺贝尔物理学奖得主约翰·哈斯布鲁克·范弗莱克\(^\text{[12]}\)和珀西·威廉姆斯·布里奇曼合作,研究金属的内聚力与电导问题,并参与了一些关于原子核能级密度的研究工作。他于1936年获得普林斯顿大学数学物理学博士学位。\(^\text{[7]}\)
\subsection{职业与研究经历}
\subsubsection{二战服役}
1941年至1944年间,巴丁在海军军械实验室领导一个研究小组,负责磁性水雷与鱼雷的研发,以及针对水雷和鱼雷的反制技术。在此期间,他的妻子简先后生下一子一女:儿子比尔出生于1941年,女儿贝齐出生于1944年。\(^\text{[13]}\)
\subsubsection{贝尔实验室}
\begin{figure}[ht]
\centering
\includegraphics[width=8cm]{./figures/73f53791c62eaf0d.png}
\caption{约翰·巴丁、威廉·肖克利和沃尔特·布拉顿在贝尔实验室,1948年} \label{fig_Yhbd_2}
\end{figure}
1945年10月,巴丁开始在贝尔实验室工作,加入了一个由威廉·肖克利和化学家斯坦利·摩根领导的固态物理研究小组。小组其他成员包括沃尔特·布拉顿、物理学家杰拉尔德·皮尔森、化学家罗伯特·吉布尼、电子专家希尔伯特·摩尔以及若干名技术人员。巴丁将家人搬到了新泽西州的萨米特市。\(^\text{[14]}\)

该小组的任务是寻找一种固态替代品,以取代易碎的玻璃真空管放大器。最初的尝试基于肖克利提出的设想,即利用外部电场作用于半导体,从而改变其导电性。然而,无论采用何种材料和配置,这些实验始终神秘地失败了。小组的工作陷入停滞,直到巴丁提出一个理论,认为表面态阻止了电场深入半导体内部。于是小组转向研究这些表面态,并几乎每天召开会议讨论进展。团队成员之间配合默契,思想交流非常自由。\(^\text{[15]}\)到1946年冬,他们取得了足够的实验结果,巴丁将关于表面态的论文提交至《物理评论》。布拉顿开始通过照射强光在半导体表面进行实验,以研究表面态的表现。这一系列实验带来了更多论文(其中一篇由肖克利合作署名),论文估算的表面态密度足以解释此前实验失败的原因。当他们开始使用电解质包围半导体与导线之间的点接触时,研究进展显著加快。摩尔设计了一个可以轻松调节输入信号频率的电路,并建议使用甘醇硼酸盐(glycol borate,简称gu)这种不易挥发的粘性化学物质。最终,在皮尔森根据肖克利的建议\(^\text{[16]}\),在p–n结上的gu液滴施加电压后,他们首次获得了放大电信号的证据。
\subsubsection{晶体管的发明}
\begin{figure}[ht]
\centering
\includegraphics[width=6cm]{./figures/fe9c8020374976ee.png}
\caption{1947年12月23日于贝尔实验室发明的首个晶体管的仿制模型(艺术化复刻)} \label{fig_Yhbd_3}
\end{figure}
1947年12月23日,在没有肖克利参与的情况下,巴丁和布拉顿成功制造出一种点接触晶体管,并实现了信号放大。次月,贝尔实验室的专利律师开始着手准备专利申请。\(^\text{[17]}\)

不久,贝尔实验室的律师发现,肖克利所依据的场效应原理早在1930年就已被尤利乌斯·李利恩菲尔德提出并申请了专利。他在1925年10月22日于加拿大申请了一项类似于MESFET(金属-半导体场效应晶体管)的专利。\(^\text{[18]}\)

肖克利在公开场合将晶体管的发明功劳大部分归于自己,这导致他与巴丁的关系恶化。\(^\text{[19]}\)然而,贝尔实验室的管理层始终将三位发明者作为一个团队进行对外宣传。最终,肖克利惹怒并疏远了巴丁和布拉顿,基本阻止他们参与结型晶体管的后续研究。巴丁随后转向研究超导理论,并于1951年离开了贝尔实验室。布拉顿也拒绝再与肖克利合作,被调往另一个研究小组。巴丁和布拉顿在晶体管发明后的第一年之后,基本没有再参与其进一步的发展工作。\(^\text{[20][21]}\)

“晶体管”这个词是“跨导”和“电阻器”的合成词。与当时用于电视和收音机的真空管相比,晶体管体积仅为其1/50,耗电更少、可靠性更高,并使得电子设备得以小型化。\(^\text{[6]}\)
\subsubsection{伊利诺伊大学厄本那-香槟分校}
\begin{figure}[ht]
\centering
\includegraphics[width=6cm]{./figures/9c5efef362b3409a.png}
\caption{伊利诺伊大学厄本那-香槟分校纪念约翰·巴丁及其超导理论的纪念牌匾} \label{fig_Yhbd_4}
\end{figure}
到1951年,巴丁开始寻找新的工作机会。他的朋友弗雷德·塞茨(Fred Seitz)说服伊利诺伊大学厄本那-香槟分校向巴丁提供一份年薪一万美元的职位。巴丁接受了这一聘请并离开了贝尔实验室,\(^\text{[17]}\)于1951年加入伊利诺伊大学,担任电气工程系和物理系的教授。\(^\text{[22]}\)

在伊利诺伊大学,他建立了两个主要的研究项目,一个设在电气工程系,另一个设在物理系。电气工程系的研究项目涵盖半导体的实验和理论方面,物理系的研究项目则集中于宏观量子系统的理论研究,特别是超导性和量子液体。\(^\text{[23]}\)

从1951年到1975年,巴丁一直是伊利诺伊大学的在职教授,之后成为荣休教授。\(^\text{[6]}\)在晚年,巴丁仍积极参与学术研究,他的研究重点转向探索电子在电荷密度波(CDW)中穿越金属线性链化合物的流动行为。他提出\(^\text{[24][25][26]}\)CDW电子输运是一种集体量子现象(见“宏观量子现象”),起初这一观点受到质疑。\(^\text{[27]}\)然而,2012年发表的实验结果\(^\text{[28]}\)显示,在三硫化钽环中,CDW电流随磁通量呈现出振荡行为,这与超导量子干涉器(见 SQUID 和阿哈罗诺夫–玻姆效应)中的现象相似,从而增强了CDW电子输运本质上属于量子集体现象这一理论的可信度。\(^\text{[29][30]}\)(参见“量子力学”。)巴丁在整个1980年代持续从事研究,并在去世前不到一年仍在《物理评论快报》\(^\text{[31]}\)和《今日物理》\(^\text{[32]}\)上发表论文。

巴丁的个人文献资料现由伊利诺伊大学档案馆收藏。\(^\text{[33]}\)
\subsubsection{1956年诺贝尔物理学奖}
1956年,约翰·巴丁与贝克曼仪器公司半导体实验室的威廉·肖克利以及贝尔电话实验室的沃尔特·布拉顿共同获得诺贝尔物理学奖,以表彰他们“在半导体研究及晶体管效应发现方面的贡献”。\(^\text{[34]}\)

在斯德哥尔摩举行的诺贝尔奖颁奖典礼上,布拉顿和肖克利当晚从古斯塔夫六世·阿道夫国王手中接过了奖章。巴丁只带了三个孩子中的一个出席典礼,古斯塔夫国王对此打趣了他一番,巴丁则向国王保证,下次一定会带上全部孩子。后来他确实信守了承诺。\(^\text{[35]}\)
\subsubsection{BCS理论}
1957年,巴丁与利昂·库珀以及他的博士生约翰·施里弗合作,提出了超导性的标准理论——BCS理论(以三人姓氏首字母命名)。\(^\text{[6]}\)
\subsubsection{约瑟夫森效应争议}
1960年夏天,巴丁因在纽约斯克内克塔迪的通用电气研究实验室担任顾问而对超导隧穿现象产生兴趣。在那里,他了解到伊瓦尔·贾维尔在伦斯勒理工学院所做的实验,这些实验表明,来自普通材料的电子可能会隧穿进入超导材料。\(^\text{[36]: 222–223 }\)

1962年6月8日,当时年仅23岁的布赖恩·约瑟夫森(Brian Josephson)向《物理快报》提交了一篇论文,预言了通过势垒产生超电流的现象,这一现象后来被称为“约瑟夫森效应”。\(^\text{[37]}\)十天后,巴丁也向《物理评论快报》提交了一篇论文,在附注中对约瑟夫森的理论提出质疑。\(^\text{[36]: 222–225 [38]}\):

在一则近期的注释中,约瑟夫森使用了某种类似的表述来讨论超流体在隧穿区域中流动的可能性,且在此过程中不会产生准粒子。然而,正如作者(参考文献3)指出的那样,电子对不会延伸到势垒中,因此不可能存在此类超流体流动。

这场争论在1962年9月16日至22日于伦敦玛丽女王大学举行的第八届低温物理国际会议上进一步展开。当约瑟夫森正在介绍他的理论时,巴丁起身陈述了自己的反对意见。双方展开激烈辩论,但始终未能达成共识。在辩论过程中,约瑟夫森多次质问巴丁:“你算过了吗?没有?我算过。”\(^\text{[36]: 225–226 }\)

1963年,菲利普·W·安德森与约翰·罗威尔在贝尔实验室发表的一篇论文提供了对约瑟夫森效应的实验证据和进一步的理论澄清,证明了该效应的正确性。\(^\text{[39]}\)此后,巴丁接受了约瑟夫森的理论,并在1963年8月的一次会议上公开撤回了自己的反对意见。他还邀请约瑟夫森于1965–1966学年在伊利诺伊大学担任博士后,随后更提名约瑟夫森与贾维尔获得诺贝尔物理学奖——他们于1973年共同获奖。\(^\text{[36]: 226 }\)
\subsubsection{1972年诺贝尔物理学奖}
1972年,约翰·巴丁与布朗大学的利昂·N·库珀以及宾夕法尼亚大学的约翰·罗伯特·施里弗(共同获得诺贝尔物理学奖,以表彰他们“共同发展出的超导理论,通常称为BCS理论”。\(^\text{[40]}\)这是巴丁第二次获得诺贝尔物理学奖,他也由此成为首位在同一领域两次获得诺贝尔奖的人。\(^\text{[41]}\)

这次,巴丁带着三个孩子一起出席了在斯德哥尔摩举行的诺贝尔奖颁奖典礼。\(^\text{[35]}\)他将大部分奖金捐赠给杜克大学,用于设立弗里茨·伦敦纪念讲座。\(^\text{[42]}\)

20世纪60年代末,巴丁认为库珀和施里弗应当因为BCS理论而获得诺贝尔奖。他担心由于自己已获得过一次诺贝尔奖,而作为该理论的合作者,评奖委员会可能会因此不愿再次授予他。这一顾虑促使他在1967年提名其他在超导隧穿效应方面取得重要成果的科学家(如约瑟夫森效应)作为获奖候选人,包括利奥·埃萨基、伊瓦尔·贾维尔和布赖恩·约瑟夫森。他认识到,隧穿效应的研究成果依赖于超导性原理,因此若这些工作先被授奖,也能提高BCS理论后续获奖的可能性。他还推断,诺贝尔委员会偏好多国合作的研究团队,而他所提名的隧穿效应科学家正好分别来自不同国家。巴丁在1971年和1972年再次提名这些人,BCS理论就在1972年获奖,而隧穿效应研究则于1973年获得诺贝尔奖。\(^\text{[36]: 230–231}\) 

巴丁是唯一一位获得两次诺贝尔物理学奖的人,也是仅有的三位在同一奖项中获得双料诺贝尔奖的科学家之一,另外两位是分别于1958年与1980年两度获得诺贝尔化学奖的弗雷德里克·桑格以及于2001年与2022年获得诺贝尔化学奖的卡尔·巴里·夏普利斯。\(^\text{[43]}\)
\subsubsection{其他奖项}
除了两次获得诺贝尔奖之外,巴丁还荣获了众多其他奖项,包括:
\begin{itemize}
\item 1952年:富兰克林研究所斯图尔特·巴兰坦奖章
\item 1954年:当选为美国国家科学院院士\(^\text{[44]}\)
\item 1958年:当选为美国哲学学会会员\(^\text{[45]}\)
\item 1959年:当选为美国文理科学院院士\(^\text{[46]}\)
\item 1965年:美国国家科学奖章\(^\text{[47]}\)
\item 1971年:IEEE荣誉奖章,以表彰他“在固体导电性理解、晶体管发明和超导微观理论方面的深远贡献”
\item 1973年:当选为英国皇家学会外籍院士\(^\text{[1][48]}\)
\item 1975年:富兰克林奖章
\item 1977年1月10日:由杰拉尔德·福特总统授予总统自由勋章(由其子威廉·巴丁代为出席领奖)
\item 1990年:获得乔治·H·W·布什总统颁发的“第三个世纪奖”,以表彰其对美国社会的卓越贡献;1988年还获得苏联科学院颁发的金质奖章
\item 1987年:美国成就科学院“金盘奖”
\end{itemize}
\subsubsection{施乐公司}
巴丁还是施乐公司的重要顾问。尽管他性格内敛,但他却罕见地主动劝说施乐高层保留其位于加利福尼亚的研究中心——施乐帕洛阿尔托研究中心,当时施乐总部对该研究中心的前景持怀疑态度,担心其研究成效甚微。
\subsection{个人生活}
1938年7月18日,巴丁与简·麦克斯韦结婚。在普林斯顿期间,他在一次回匹兹堡拜访老朋友时结识了简。

巴丁是一位性格极为低调的科学家。尽管他在伊利诺伊大学担任教授近40年,但在邻居们眼中,他最令人印象深刻的却是那些在家中举办的烧烤聚会——他会亲自为朋友们准备食物,而许多来访者甚至并不知道他在大学里的非凡成就。他总会问客人是否喜欢烤过的汉堡面包(因为他自己喜欢那样)。他喜欢打高尔夫球,也常和家人一起去野餐。莉莲·霍德森曾说,巴丁“完全不同于人们对‘天才’的刻板印象,也不在意展现出与众不同的样子,因此常常被公众和媒体所忽视。”\(^\text{[22]}\)

在1988年的一次采访中,当被问及个人信仰时,巴丁回答说:“我不是一个宗教信仰者,所以并不太去思考这类问题。”但他也曾表示:“我认为科学无法回答关于生命意义和目的这些终极问题。”巴丁确实相信一套道德价值和行为准则。\(^\text{[50]}\)他的孩子们由妻子带去教堂,简不仅教授主日学,也是一位教会长老。\(^\text{[36]: 168–169 }\)尽管如此,夫妻俩都明确表示,他们并不相信来世或其他宗教观念。\(^\text{[51]}\)

巴丁是詹姆斯·M·巴丁、威廉·A·巴丁和女儿伊丽莎白的父亲。
\subsubsection{逝世}
1991年1月30日,巴丁因心脏病在马萨诸塞州波士顿的布莱根和妇女医院逝世,享年82岁。\(^\text{[52]}\) 尽管他常年居住在尚佩恩-厄本那地区,但当时是为了就医而前往波士顿的。\(^\text{[6]}\) 巴丁与其妻子简(1907–1997)安葬于威斯康星州麦迪逊的森林山公墓。\(^\text{[53]}\)他们去世后留下了三位子女:詹姆斯、威廉和伊丽莎白·巴丁·格雷塔克,以及六位孙辈。\(^\text{[6]}\)
\subsubsection{遗产与影响}
在本世纪即将结束之际,当人们开始列举那些对20世纪影响最深远的人物时,上周逝世的约翰·巴丁这个名字,必定会位居前列,甚至可以说,有理由将他排在榜首……巴丁先生曾两次获得诺贝尔奖,并荣获无数其他荣誉。但还有什么比这更高的荣誉呢?当我们环顾四周,处处都能看到他的智慧所留下的痕迹——正是这些成就,让我们的生命更长久、更健康、更美好。

——《芝加哥论坛报》社论,1991年2月3日

