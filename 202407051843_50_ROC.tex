% 接收者操作特征曲线
% keys 接收|操作|特徵曲線|信号检测论|分类
% license Xiao
% type Tutor

\pentry{分类\nref{nod_Class},模型评估\nref{nod_MoEva},混淆矩阵\nref{nod_ConMat}}{nod_9fcc}

\textbf{接收者操作特征曲线}(Receiver operating characteristic curve, ROC)是一种用于评价二分类器的分类性能的图表。该方法来源于信号检测论,是在第二次世界大战中,由电子信号工程师发明的。在心理学领域中也有广泛应用。

接收者操作特征曲线能够评估分类模型在不同阈值下的性能,通过绘制\textbf{真阳性率}(True positive rate, TPR)与\textbf{假阳性率}(False positive rate, FPR)的关系来衡量模型的区分能力。

\begin{definition}{真阳性率}
真正阳性率(TPR):也称为灵敏度(Sensitivity)或召回率(Recall),表示在实际为正类的样本(阳性样本)中,被分类模型正确判断为正类(阳性)的比例。数学表达式为:
\begin{equation}
TPR=\frac{TP}{P}=\frac{TP}{TP+FN}~,
\end{equation}
其中,$P$表示实际为阳性的样本数,$TP$表示真阳的样本数,$FN$表示假阴性的样本数。
\end{definition}

\begin{definition}{假阳性率}
假正率(FPR):表示在所有负类样本(阴性样本)中,被分类模型错误判断为正类(阳性)的样本比例。数学表达式为:
\begin{equation}
FPR=\frac{FP}{N}=\frac{FP}{TN+FP}~,
\end{equation}
其中,$N$表示实际为阴性的样本数,$FP$表示假阳性的样本数,$TN$表示真阴性的样本数,$FP$表示假阳性的样本数。
\end{definition}

曲线中点的坐标是真阳率和假阳率。随着分类器阈值的变化,真阳率和假阳率分别随之改变,由此产生一系列的点,然后将相邻两点连接起来,即构成接收者操作特征曲线。

\begin{figure}[ht]
\centering
\includegraphics[width=14.25cm]{./figures/74304970d1216023.png}
\caption{ROC示意图} \label{fig_ROC_1}
\end{figure}
上图中蓝色曲线即为ROC曲线。横坐标为假阳率,纵坐标为真阳率。红色对角线表示一个完全随机分类器的ROC曲线。如果一个分类器的ROC曲线大体在红色虚线的上方,则表示性能优于随机分类器。

ROC曲线分析可以帮助我们做模型选择,选择最优模型,抛弃次优模型。曲线下面积(Aera under curve, AUC)是ROC曲线最基本的评估指标,顾名思义,表示的是曲线下方,横轴上方的面积。通常把整图的面积定义为1,则AUC的值在$0$和$1$之间。该面积值越大,则表示分类模型效果越好;反之,则越差。
