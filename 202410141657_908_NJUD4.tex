% 南京理工大学 普通物理 B(845)模拟五套卷 第四套
% license Usr
% type Note

\textbf{声明}:“该内容来源于网络公开资料,不保证真实性,如有侵权请联系管理员”

\subsection{一、 填空题 I(24 分,每空 2 分)}
1.一质点作圆周运动,设半径为 $R$,运动方程$s=v_0t-\frac{1}{2}bt^2$ ,其中 $s$ 为弧长,$v_0$为初速度,$b$ 为常数。则任一时刻 $t$ 质点的法向加速度为_____,切向加速度为__________。

2. 质量为 $4.25Kg$ 的质点,在合力$F=5i-3j(N)$的作用下由静止从原点运动到$r=5i-3j(m)$时,合力所做的功为_________;此时质点的运动速度大小为_______________。

3. 花样滑冰运动员绕通过自身的竖直轴转动,开始时两臂伸开,转动惯量为 $J$,角速度为$\omega$ ,然后她将两臂收回,使转动惯量减少为 $J/2$,这时她转动的角速度变为____________。

4. 质量为 2kg 的质点,按方程 $x=0.2\sin[5t-(\pi/6)]$沿着 $x$ 轴振动,则 $t=0$ 时,作用于质点的力的大小为__________;作用于质点的力的最大值为________,此时质点的位置________。

5. 设平面简谐波沿 $x$ 轴传播时在 $x=0$ 处发生反射 ,反射波的表达式为,已知反射点为一自由端,则由入射波和反射波形成驻波波节的位置坐标为__________。

6. 如图,真空中一长为$L$ 的均匀带电细直杆,总电量为 $q$,则在直杆延长线上
距杆一端距离为 $d$ 的 $P$ 点的电场强度为___________。

7. 一气缸内储有 $10mol$ 单原子分子理想气体,在压缩过程中,外力做功 $209J$,
气体温度升高 $1K$,则气体内能的增量$\Delta E$ 为____________$J$,吸收的热量 $Q $____________$J$。