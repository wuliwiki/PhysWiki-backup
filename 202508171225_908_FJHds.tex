% 非结合代数(综述)
% license CCBYSA3
% type Wiki

本文根据 CC-BY-SA 协议转载翻译自维基百科\href{https://en.wikipedia.org/wiki/Non-associative_algebra}{相关文章}。

非结合代数[1](或称分配代数)是指一种域上的代数,其二元乘法运算不假定具有结合性。也就是说,一个代数结构 $A$ 若是域 $K$ 上的非结合代数,则它是一个 $K$-向量空间,并配备了一个 $K$-双线性的二元乘法运算 $A \times A \to A$,该运算可以是结合的,也可以不是。例子包括李代数、约当代数、八元数,以及带有叉乘运算的三维欧几里得空间。由于不假定乘法是结合的,因此必须使用括号来表示运算顺序。例如,$(ab)(cd)$、$(a(bc))d$ 和 $a(b(cd))$ 的结果可能完全不同。

这里的“非结合”意味着不要求结合律成立,但并不意味着结合律被禁止。换句话说,“非结合”就是“未必结合”的意思,正如“非交换”环中的“非交换”并不是绝对禁止交换律,而是指“未必交换”。

一个代数是幺代数(unital 或 unitary),如果它存在一个单位元 $e$,满足对代数中所有元素 $x$ 都有 $ex = x = xe$。例如,八元数是幺代数,但李代数从来都不是。

对 $A$ 的非结合代数结构,可以通过将其关联到其他结合代数来研究,这些结合代数是$A$ 作为 $K$-向量空间时其全体 $K$-自同态代数的子代数。其中有两个重要的例子:导子代数和包络代数(后者在某种意义上是“包含 $A$ 的最小结合代数”)。

更一般地,有些作者把非结合代数的概念扩展到交换环 $R$ 上:即一个带有 $R$-双线性二元乘法运算的 $R$-模[2]。如果一个结构满足除了结合律以外的所有环公理(例如任何 $R$-代数),那么它自然就是一个 $\mathbb{Z}$-代数,因此有些作者称非结合的 $\mathbb{Z}$-代数为非结合环。
\subsection{满足恒等式的代数}
具有两个二元运算、但没有其他限制的类环结构是一类非常广泛的对象,过于笼统而难以研究。出于这个原因,最为人熟知的非结合代数类型往往满足某些恒等式或性质,从而在一定程度上简化了乘法。这些性质包括以下几类。
\subsubsection{常见性质}
设 $x, y, z$ 为域 $K$ 上代数 $A$ 的任意元素。正整数次幂的递归定义为:$x^1 := x$,对于 $n \geq 1$,有两种习惯定义:右幂:$x^{n+1} := (x^n)x$[3],左幂:$x^{n+1} := x(x^n)$[4][5],具体采用哪种定义取决于作者。
\begin{itemize}
\item 幺元(Unital):存在一个元素 $e$,使得 $ex = x = xe$。在这种情况下可定义 $x^0 := e$。
\item 结合性:$(xy)z = x(yz)$。
\item 交换性:$xy = yx$。
\item 反交换性[6]:$xy = -yx$。
\item Jacobi 恒等式[6][7]:$(xy)z + (yz)x + (zx)y = 0$,或 $x(yz) + y(zx) + z(xy) = 0$,具体形式取决于作者。
\item Jordan 恒等式[8][9]:$(x^2y)x = x^2(yx)$,或 $(xy)x^2 = x(yx^2)$,具体形式取决于作者。
\item 可交替性[10][11][12]:左交替律:$(xx)y = x(xy)$,右交替律:$(yx)x = y(xx)$。
\item 柔性律[13][14]:$(xy)x = x(yx)$。
\item $n$ 次幂结合律(nth power associative,$n \geq 2$):对所有满足 $0 < k < n$ 的整数 $k$,有$x^{n-k}x^k = x^n$。
\item 三次幂结合律:$x^2x = xx^2$。
\item 四次幂结合律:$x^3x = x^2x^2 = xx^3$(可与下面的“四次幂交换律”比较)。
\item 幂结合律[4][5][15][16][3]:由任意单个元素生成的子代数是结合代数,即对所有 $n \geq 2$ 成立 $n$ 次幂结合律。
\item $n$ 次幂交换律(nth power commutative,$n \geq 2$)**:对所有满足 $0 < k < n$ 的整数 $k$,有$x^{n-k}x^k = x^kx^{n-k}$。
\item 三次幂交换律:$x^2x = xx^2$。
\item 四次幂交换律:$x^3x = xx^3$(可与上面的“四次幂结合律”比较)。
\item 幂交换律:由任意单个元素生成的子代数是交换代数,即对所有 $n \geq 2$ 成立 $n$ 次幂交换律。
\item 指数为 $n \geq 2$ 的幂零性(Nilpotent of index n)**:任意 $n$ 个元素的乘积(无论如何加括号)都为零,但存在某些 $n-1$ 个元素使得其在某种结合方式下乘积不为零:$x_1x_2\cdots x_n = 0, \quad \exists \; y_1,\dots,y_{n-1} \;\; \text{使得 } y_1y_2\cdots y_{n-1} \neq 0$
\item 指数为 $n \geq 2$ 的幂零:幂结合的代数,且 $x^n = 0$,并且存在某个元素 $y$ 使得 $y^{n-1} \neq 0$。
\end{itemize}
\subsubsection{性质之间的关系}
对于任意特征的域 $K$:

\begin{itemize}
\item 结合性 $\implies$ 可交替性。
\item 左交替性、右交替性和柔性律三者中,任意两个 $\implies$ 第三个。
\item 因此,可交替性 $\implies$ 柔性律。
\item 可交替性 $\implies$ Jordan 恒等式[17][a]。
\item 交换性 $\implies$ 柔性律。
\item 反交换性 $\implies$ 柔性律。
\item 可交替性 $\implies$ 幂结合性[a]。
\item 柔性律 $\implies$ 三次幂结合律。
\item 二次幂结合律 $\equiv \text{恒成立}$。
\item 二次幂交换律 $\equiv \text{恒成立}$。
\item 三次幂结合律 $\iff$ 三次幂交换律。
\item $n$ 次幂结合律 $\implies$ $n$ 次幂交换律。
\item 指数为 $2$ 的幂零性 $\implies$ 反交换性。
\item 指数为 $2$ 的幂零性 $\implies$ Jordan 恒等式。
\item 指数为 $3$ 的幂零代数 $\implies$ Jacobi 恒等式。
\item 指数为 $n$ 的幂零代数 $\implies$ 指数为 $N$ 的幂零代数,且 $2 \leq N \leq n$。
\item 幺代数 $\wedge$ 指数为 $n$ 的幂零性两者不相容。若 $K \neq \mathrm{GF}(2)$ 或 $\dim(A) \leq 3$
\item Jordan 恒等式 与 交换性 一起 $\implies$ 幂结合性[18][19][20][citation needed]。若 $\mathrm{char}(K) \neq 2$:
\item 右交替性 $\implies$ 幂结合性[21][22][23][24]。
\item 类似地,左交替性 $\implies$ 幂结合性。
\item 幺代数 $\wedge$ Jordan 恒等式 $\implies$ 柔性律[25]。
\item Jordan 恒等式 $\wedge$ 柔性律 $\implies$ 幂结合性[26]。
\item 交换性 $\wedge$ 反交换性 $\implies$ 指数为 $2$ 的幂零性。
\item 反交换性 $\implies$ 指数为 $2$ 的幂零。
\item 幺代数 $\wedge$ 反交换性不相容。若 $\mathrm{char}(K) \neq 3$:
\item 幺代数 $\wedge$ Jacobi 恒等式不相容。若 $\mathrm{char}(K) \notin \{2,3,5\}$:
\item 交换性 $\wedge$ $x^4 = x^2x^2$(定义四次幂结合律的两个恒等式之一) $\implies$ 幂结合性[27]。若 $\mathrm{char}(K) = 0$:
\item 三次幂结合律 $\wedge$ $x^4 = x^2x^2$(定义四次幂结合律的两个恒等式之一) $\implies$ 幂结合性[28]。
若 $\mathrm{char}(K) = 2$:交换性 $\iff$ 反交换性。
\end{itemize}
\subsubsection{结合子}
在代数 $A$ 上的结合子是一个 $K$-多线性映射:$[\cdot,\cdot,\cdot] : A \times A \times A \to A$其定义为
$$
[x,y,z] = (xy)z - x(yz).~
$$
它衡量了代数 $A$ 的“非结合性程度”,并且可以方便地表达一些 $A$ 可能满足的恒等式。

设 $x, y, z$ 为代数中的任意元素:

\begin{itemize}
\item 结合性:$[x,y,z] = 0$
\item 可交替性:$[x,x,y] = 0 \quad \text{(左交替)}, \qquad [y,x,x] = 0 \quad \text{(右交替)}$
\item 这意味着交换任意两个变量会改变符号:$[x,y,z] = -[x,z,y] = -[z,y,x] = -[y,x,z]$其逆命题仅在 $\mathrm{char}(K) \neq 2$ 时成立。
\item 柔性律:$[x,y,x] = 0$
\item 这意味着交换首尾两项会改变符号:$[x,y,z] = -[z,y,x]$其逆命题同样仅在 $\mathrm{char}(K) \neq 2$ 时成立。
\item Jordan 恒等式[29]:$[x^2,y,x] = 0 \quad \text{或} \quad [x,y,x^2] = 0$取决于不同作者的定义。
\item 三次幂结合性:$[x,x,x] = 0$
\end{itemize}
核是与所有其他元素都满足结合律的元素集合[30],即所有满足
  $$
  [n,A,A] = [A,n,A] = [A,A,n] = \{0\}~
  $$
的 $n \in A$。

核构成 $A$ 的一个结合子环。
\subsubsection{中心}
代数 $A$ 的中心是指在 $A$ 中既与所有元素可交换、又与所有元素满足结合律的元素集合。它等于以下两个集合的交集:
$$
C(A) = \{\, n \in A \mid nr = rn \;\; \forall r \in A \,\}~
$$
与核的交集。

事实表明,对于 $C(A)$ 的元素来说,只需在下面三个集合中的两个成立为零集:$([n, A, A], \quad [A, n, A], \quad [A, A, n],)$那么第三个集合也必然是零集。
\subsection{例子}
\begin{itemize}
\item 欧几里得空间 $\mathbb{R}^3$:以向量叉积为乘法所形成的代数是一个反交换但非结合的代数。叉积同时满足 Jacobi 恒等式。
\item 李代数:满足反交换律与 Jacobi 恒等式的代数即为李代数。
\item 向量场代数:
  可微流形上的向量场代数(当 $K=\mathbb{R}$ 或复数域 $\mathbb{C}$ 时),或代数簇上的向量场代数(当 $K$ 是一般域时)。
\item 约当代数:满足交换律与 Jordan 恒等式的代数[9]。
\item 结合代数与李代数的关系:每一个结合代数都能通过交换子定义出一个李代数。事实上,每一个李代数都可以这样构造,或者是这样构造出的李代数的一个子代数。
\item 结合代数与约当代数的关系:在特征不为 2 的域上,每一个结合代数都能通过新定义的乘法$x * y = (xy + yx)/2$构造出一个约当代数。与李代数的情形不同,并非所有约当代数都能通过这种方式得到。那些可以这样得到的称为特殊约当代数。
\item 可交替代数:满足可交替性的代数。最重要的例子是八元数(实数域上的代数),以及其在其他域上的推广。所有结合代数都是可交替代数。在有限维实可交替除代数中,按同构分类,只有实数、复数、四元数和八元数(见下文)。
\item 幂结合代数:满足幂结合恒等式的代数。例子包括所有结合代数、所有可交替代数、特征不为 2 的域上的所有约当代数(见前文),以及十六元数。
\item 双曲四元数代数:定义在 $\mathbb{R}$ 上的双曲四元数代数,是狭义相对论采用闵可夫斯基空间之前的一种实验性代数。
\end{itemize}
更多类别的代数
\begin{itemize}
\item 分次代数:包含了多重线性代数中最重要的代数,例如张量代数、对称代数和外代数(定义在给定向量空间上)。分次代数可以推广为滤过代数。
\item 除代数:其中每个非零元素都存在乘法逆元。有限维实数域上的可交替除代数已经被分类,它们分别是:实数(维数 1)、复数(维数 2)、四元数(维数 4)和八元数(维数 8)。其中四元数和八元数不满足交换律;除八元数外,其余都满足结合律。
\item 二次代数:要求满足 $xx = re + sx$,其中 $r, s$ 属于底域,$e$ 是代数的单位元。例子包括所有有限维可交替代数,以及实数 $2 \times 2$ 矩阵代数。在同构意义下,唯一既是可交替的、二次的、实的且没有零因子的代数是:实数、复数、四元数和八元数。
\item Cayley–Dickson 代数(Cayley–Dickson algebras, $K=\mathbf{R}$):由下列代数依次构造:
\item 复数 $\mathbf{C}$(交换且结合的代数);
\item 四元数 $\mathbf{H}$(结合代数);
\item 八元数 $\mathbf{O}$(可交替代数);
\item 十六元数(Sedenions, $\mathbf{S}$);
\item 三十二元数(Trigintaduonions, $\mathbf{T}$)以及一列无限延伸的 Cayley–Dickson 代数(它们是幂结合代数)。
\item 超复数代数:所有有限维的幺实代数,因而包括 Cayley–Dickson 代数以及更多类型。
\item Poisson 代数:出现在几何量子化中,带有两种不同的乘法,从而同时构成交换代数和李代数。
\item 遗传代数:一类用于数学生物学遗传学研究的非结合代数。
\item 三元系统

参见:代数列表。
\end{itemize}
\subsection{性质}
有一些在环论或结合代数中熟悉的性质,在非结合代数中并不总是成立。与结合代数的情况不同,带有(双边)乘法逆元的元素也可能是零因子。例如,十六元数的所有非零元素都有双边逆元,但其中某些同时也是零因子。
\subsection{自由非结合代数}
在域 $K$ 上,集合 $X$ 的自由非结合代数定义为:其基由所有非结合单项式构成,即由 $X$ 中元素的有限形式积组成,且保留括号。两个单项式 $u, v$ 的积就是 $(u)(v)$。若将空积看作单项式,则该代数是幺代数[31]。

Kurosh 证明了:自由非结合代数的任一子代数仍然是自由的[32]。
\subsection{相关代数}
域 $K$ 上的代数 $A$,本身就是一个 $K$-向量空间,因此可以考虑其所有 $K$-线性自同态所构成的结合代数 $\mathrm{End}_K(A)$。与 $A$ 的代数结构相关,可以在 $\mathrm{End}_K(A)$ 中引出两个重要的子代数:导子代数(结合的)包络代数
\subsubsection{导子代数}
代数 $A$ 上的一个导子是满足以下性质的映射 $D$:$D(x \cdot y) = D(x) \cdot y + x \cdot D(y)$,$A$ 上的所有导子构成了 $\mathrm{End}_K(A)$ 的一个子空间,记作 $\mathrm{Der}_K(A)$。两个导子的交换子仍然是一个导子,因此李括号使 $\mathrm{Der}_K(A)$ 具有李代数的结构[33]。
\subsubsection{包络代数}
对于代数 $A$ 的每个元素 $a$,可以定义两个线性映射[34]:
$$
L(a): x \mapsto ax ,\quad R(a): x \mapsto xa.~
$$
这里的每个 $L(a), R(a)$ 都被视为 $\mathrm{End}_K(A)$ 的元素。代数 $A$ 的结合包络代数(或称乘法代数)是 $\mathrm{End}_K(A)$ 的一个结合子代数,由所有左、右线性映射 $L(a), R(a)$ 所生成[29][35]。$A$ 的中心核定义为包络代数在自同态代数 $\mathrm{End}_K(A)$ 中的中心化子。若一个代数的中心核仅由单位映射的 $K$-数倍构成,则称该代数为中心代数[16]。

一些非结合代数可能满足的恒等式,可以方便地用线性映射来表达[36]:
\begin{itemize}
\item 交换性:每个 $L(a)$ 等于相应的 $R(a)$。
\item 结合性:任意 $L$ 与任意 $R$ 可交换。
\item 柔性律:每个 $L(a)$ 与相应的 $R(a)$ 可交换。
\item Jordan 恒等式:每个 $L(a)$ 与 $R(a^2)$ 可交换。
\item 可交替性:每个 $L(a)^2 = L(a^2)$,右作用同理。
\end{itemize}
二次表示定义为[37]:
$$
Q(a): x \mapsto 2a \cdot (a \cdot x) - (a \cdot a)\cdot x,~
$$
或等价地:
$$
Q(a) = 2L^2(a) - L(a^2).~
$$
关于普遍包络代数的条目,描述了其典范构造方法,以及类似 PBW 定理(Poincaré–Birkhoff–Witt 型定理)。对于李代数,这样的包络代数具有普遍性质,但一般的非结合代数并不具备这一性质。最著名的例子也许是 Albert 代数 —— 一种特殊的 Jordan 代数,它并不能通过 Jordan 代数的典范包络代数构造来得到。
\subsection{参见}
\begin{itemize}
\item 代数列表
\item 交换的非结合幺半群,它们可以生成非结合代数
\end{itemize}
\subsection{引用文献}
\begin{enumerate}
\item Schafer 1995,第 1 章。
\item Schafer 1995,第 1 页。
\item Albert 1948a,第 553 页。
\item Schafer 1995,第 30 页。
\item Schafer 1995,第 128 页。
\item Schafer 1995,第 3 页。
\item Okubo 2005,第 12 页。
\item Schafer 1995,第 91 页。
\item Okubo 2005,第 13 页。
\item Schafer 1995,第 5 页。
\item Okubo 2005,第 18 页。
\item McCrimmon 2004,第 153 页。
\item Schafer 1995,第 28 页。
\item Okubo 2005,第 16 页。
\item Okubo 2005,第 17 页。
\item Knus 等 1998,第 451 页。
\item Rosenfeld 1997,第 91 页。
\item Jacobson 1968,第 36 页。
\item Schafer 1995,第 92 页。
\item Kokoris 1955,第 710 页。
\item Albert 1948b,第 319 页。
\item Mikheev 1976,第 179 页。
\item Zhevlakov 等 1982,第 343 页。
\item Schafer 1995,第 148 页。
\item Bremner, Murakami & Shestakov 2013,第 18 页。
\item Bremner, Murakami & Shestakov 2013,第 18–19 页,事实 6。
\item Albert 1948a,第 554 页,引理 4。
\item Albert 1948a,第 554 页,引理 3。
\item Schafer 1995,第 14 页。
\item McCrimmon 2004,第 56 页。
\item Rowen 2008,第 321 页。
\item Kurosh 1947,第 237–262 页。
\item Schafer 1995,第 4 页。
\item Okubo 2005,第 24 页。
\item Albert 2003,第 113 页。
\item McCrimmon 2004,第 57 页。
\item Koecher 1999,第 57 页。
\end{enumerate}
\subsection{注释}

a.这可由 Artin 定理 推出。
\subsection{参考文献}
\begin{itemize}
\item Albert, A. Adrian (2003) [1939]. Structure of algebras(《代数的结构》)。美国数学学会专题出版,第 24 卷(1961 年修订版的勘误重印本)。纽约:American Mathematical Society。ISBN 0-8218-1024-3。Zbl 0023.19901。
\item Albert, A. Adrian (1948a). “Power-associative rings”(《幂结合环》)。Transactions of the American Mathematical Society,64: 552–593。doi:10.2307/1990399。ISSN 0002-9947。JSTOR 1990399。MR 0027750。Zbl 0033.15402。
\item Albert, A. Adrian (1948b). “On right alternative algebras”(《关于右可交替代数》)。Annals of Mathematics,50: 318–328。doi:10.2307/1969457。JSTOR 1969457。
\item Bremner, Murray; Murakami, Lúcia; Shestakov, Ivan (2013) [2006]. “Chapter 86: Nonassociative Algebras”(《第 86 章:非结合代数》)(PDF)。见 Hogben, Leslie (编),Handbook of Linear Algebra(《线性代数手册》,第 2 版)。CRC Press。ISBN 978-1-498-78560-0。
\item Herstein, I. N., 编 (2011) [1965]. Some Aspects of Ring Theory(《环论的一些方面》):1965 年 8 月 23–31 日于意大利科莫瓦伦纳举行的 C.I.M.E. 暑期学校讲义。C.I.M.E. Summer Schools,第 37 卷(重印版)。Springer-Verlag。ISBN 3-6421-1036-3。
\item Jacobson, Nathan (1968). Structure and representations of Jordan algebras(《约当代数的结构与表示》)。美国数学学会专题出版,第 39 卷。普罗维登斯(罗得岛):American Mathematical Society。ISBN 978-0-821-84640-7。MR 0251099。
\item Knus, Max-Albert; Merkurjev, Alexander; Rost, Markus; Tignol, Jean-Pierre (1998). *The book of involutions*(《对合之书》)。专题出版,第 44 卷。序言作者 J. Tits。普罗维登斯(罗得岛):American Mathematical Society。ISBN 0-8218-0904-0。Zbl 0955.16001。
\item Koecher, Max (1999). Krieg, Aloys; Walcher, Sebastian (编). *The Minnesota notes on Jordan algebras and their applications*(《明尼苏达笔记:约当代数及其应用》)。Lecture Notes in Mathematics,第 1710 卷。柏林:Springer-Verlag。ISBN 3-540-66360-6。Zbl 1072.17513。
\item Kokoris, Louis A. (1955). “Power-associative rings of characteristic two”(《特征为 2 的幂结合环》)。Proceedings of the American Mathematical Society,6 (5): 705–710。doi:10.2307/2032920。
\item Kurosh, A.G. (1947). “Non-associative algebras and free products of algebras”(《非结合代数与代数的自由积》)。Mat. Sbornik,20 (62)。MR 0020986。Zbl 0041.16803。
\item McCrimmon, Kevin (2004). A taste of Jordan algebras(《约当代数初探》)。Universitext。柏林、纽约:Springer-Verlag。doi:10.1007/b97489。ISBN 978-0-387-95447-9。MR 2014924。Zbl 1044.17001。附勘误。
\item Mikheev, I.M. (1976). “Right nilpotency in right alternative rings”(《右可交替环中的右幂零性》)。*Siberian Mathematical Journal*,17 (1): 178–180。doi:10.1007/BF00969304。
\item Okubo, Susumu (2005) [1995]. Introduction to Octonion and Other Non-Associative Algebras in Physics(《物理中的八元数与其他非结合代数导论》)。Montroll Memorial Lecture Series in Mathematical Physics,第 2 卷。剑桥大学出版社。doi:10.1017/CBO9780511524479。ISBN 0-521-01792-0。Zbl 0841.17001。
\item Rosenfeld, Boris (1997). Geometry of Lie groups(《李群几何》)。Mathematics and its Applications,第 393 卷。多德雷赫特:Kluwer Academic Publishers。ISBN 0-7923-4390-5。Zbl 0867.53002。
\item Rowen, Louis Halle (2008). Graduate Algebra: Noncommutative View(《高等代数:非交换视角》)。Graduate Studies in Mathematics。American Mathematical Society。ISBN 0-8218-8408-5。
\item Schafer, Richard D. (1995)[1966]. An Introduction to Nonassociative Algebras(《非结合代数导论》)。Dover。ISBN 0-486-68813-5。Zbl 0145.25601。
\item Zhevlakov, Konstantin A.; Slin'ko, Arkadii M.; Shestakov, Ivan P.; Shirshov, Anatoly I. (1982) [1978]. Rings that are nearly associative(《近似结合的环》)。Harry F. Smith 翻译。ISBN 0-12-779850-1。
\end{itemize}