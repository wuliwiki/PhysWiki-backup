% 核裂变
% license CCBYSA3
% type Wiki

(本文根据 CC-BY-SA 协议转载自原搜狗科学百科对英文维基百科的翻译)

在核物理和核化学中,\textbf{核裂变}是指核反应或放射性衰变过程中原子核分裂成更小、更轻的核的现象。裂变过程通常产生自由中子和γ 光子,同时释放出大量的能量。

重元素核裂变于1938年12月17日由德国人奥托·哈恩和他的助手弗里茨·施特拉斯曼发现,1939年1月莉泽·迈特纳和她的侄子奥托·弗里施给出了理论解释。弗里希将这一过程比喻为生物细胞的分裂。重核素的裂变是一个放热反应,会以电磁辐射和裂变碎片的动能形式释放大量能量。为了使裂变过程释放能量,裂变产物元素的总结合能应该大于起始元素的结合能。

裂变是一种核嬗变,因为裂变产物与初始原子属于不同的元素。裂变产生的两个核的质量通常比较接近,对于一般的可裂变同位素,其裂变产物的质量比约为3比2。[1][2]大多数裂变是二元裂变(产生两个带电碎片),但偶尔(每1000次事件发生2到4次)会发生三元裂变,产生三个带正电荷的碎片。三元裂变过程中最小的裂变碎片大小可位于质子到氩核之间。

除了已经被人类开发利用的中子诱导裂变之外,还存在另一种形式的裂变,称为自发放射性衰变(不需要中子诱导),该现象易发生在具有较高质量数的同位素中。自发裂变于1940年由弗廖罗夫、彼得扎克和库尔恰托夫[3]在莫斯科发现,当时他们决定通过实验验证尼尔斯·玻尔作出的一个预测,即没有中子轰击时,铀几乎不发生裂变,然而实验结论正相反。[3]

产物组成的不可预测性(产物可能的种类很多且无规律性)将裂变与量子隧穿过程区分开来,如质子发射、α衰变和团簇衰变等量子隧穿过程每次的产物都是相同的。核裂变是核电站及核武器的能量来源。作为核燃料的物质在被裂变中子撞击时会发生裂变,而在它们裂变过程中又会发射中子。这使得自持的核链式反应成为可能,这种核链式反应可以在核反应堆中以受控的速率释放能量,或者在核武器中以非常快速、不受控制的速率释放能量。

核燃料中包含的自由能是同等质量的化学燃料(如汽油)的数百万倍,这使得核裂变成为一种非常高效的能源。然而,核裂变的产物的放射性通常比作为裂变燃料的重元素高得多,并且其半衰期相当长,导致了核废料的问题。对核废料积聚和核武器的潜在破坏力的担忧影响了人们和平利用裂变作为能源的愿望。

\subsection{ 物理概述}
\subsubsection{1.1 机制}

\textbf{放射性衰变}

核裂变可以在没有中子轰击的情况下发生,这是一种放射性衰变。这种类型的裂变(称为自发裂变)除了少数重同位素之外很罕见。
\begin{figure}[ht]
\centering
\includegraphics[width=8cm]{./figures/b2a24667fa77a86b.png}
\caption{一种中子诱导核裂变事件的视觉演示,其中速度较慢的中子被铀-235原子的原子核吸收,原子核分裂成两种快速运动的较轻元素(裂变产物)和额外的中子。释放的大部分能量以裂变产物和中子的动能的形式存在。} \label{fig_HLB_1}
\end{figure}

\textbf{核反应}

在工程核设施中,基本上所有的核裂变都是以“核反应”形式发生的,其产生于轰击过程中亚原子的碰撞。在核反应中,亚原子粒子与原子核发生碰撞并使其发生变化。核反应是由轰击机制驱动的,与自发放射性衰变过程不同,后者具有相对稳定的指数衰减规律和特征半衰期。

人们目前已经发现许多种核反应。核裂变与其他类型的核反应有很大不同,因为它可以通过核裂变链式反应被放大并控制。在核裂变链式反应中,每个裂变事件释放的自由中子可以触发更多的裂变过程,这反过来又释放更多的中子并导致更多的裂变。

能够维持裂变链式反应的元素同位素被称为核燃料,我们称该种核素是可裂变的。最常见的核燃料是235U(铀的同位素,原子量为235,常用于核反应堆中)和239Pu(钚的同位素,原子量为239)。这些燃料的裂变产物的原子质量呈成双峰分布,峰值分布在95u和135u附近。大多数核燃料的自发裂变非常缓慢,而是主要通过α-β衰变链,其过程可长达千年至数万年甚至更久。在核反应堆或核武器中,绝大多数裂变事件是由中子轰击引起的,这些中子本身是由先前的裂变事件产生的。

裂变燃料中的核裂变是易裂变核素俘获中子时产生的核激发能的结果,这些能量来自于中子和原子核之间相互吸引的核力,该能量使原子核变形为双瓣状的“液滴”,原子核的“两叶”均带正电荷,当两叶之间的距离超过核力能够维持其不分离的范围时,两个裂变随便就完成了分离,进而被相互排斥的电荷进一步分开距离越来越远,因此这是一个不可逆过程。可裂变同位素(如铀-238)中也会发生类似的过程,但这些同位素需要由快中子(如热核武器中核聚变产生的中子)提供额外的能量才可以发生裂变。

根据原子核的液滴模型,核裂变产物应具有相同的原子量。通过更复杂的核壳层模型可以从机理上解释为何通常一种裂变产物比另一种稍小。玛丽亚·格佩特·梅耶提出了一种基于核壳层模型的裂变理论。

最常见的裂变过程是二元裂变,如上文所述,两个裂变产物的原子量通常分布在在95±15和135±15 u 区间。二元裂变发生的概率最大,而在核反应堆中,每1000次裂变事件中,还会发生2到4次三元裂变,三元裂变的过程产生三个带正电荷的碎片(加上中子),其中最小的裂变碎片的质量范围可以在质子(原子质量Z=1)至氩(原子质量Z=18)之间。最常见的小碎片由90\%的氦-4核组成,其能量高于α衰变产生的α粒子(即所谓“长程α粒子”,能量约16MeV),加上氦-6和氚核。三元裂变不太常见,但最终仍会在核反应堆的燃料棒中产生大量氦-4和气体氚。[4]
\begin{figure}[ht]
\centering
\includegraphics[width=8cm]{./figures/6ba36d8f2d7b9973.png}
\caption{U-235 、Pu-239 (当前核电反应堆中两种典型)和 U-233 (用于钍增殖循环)热中子诱导裂变产物的质量分布。} \label{fig_HLB_2}
\end{figure}

\subsubsection{1.2 能量学}
\textbf{能量输入}

在最常见的过程二元裂变过程中(产生两个带正电荷的裂变产物+中子),重核的裂变需要大约7-8MeV来克服将原子核保持为球形或近似球形的核力,并使其变形为双瓣形(类似“花生”),接着两瓣在正电荷的排斥力下继续彼此分离,一旦裂变碎片被推到临界距离外,短程强相互作用核力就不能再将它们保持在一起,它们的分离过程从碎片之间的(远程)电磁排斥开始,最终以两个高能裂变碎片相互远离结束。

裂变输入能量中约6MeV是由一个中子通过强力与重元素原子核结合的过程提供;然而,在许多可裂变核素中,这一数量的能量不足以产生裂变。例如,对于能量小于1MeV的中子,铀-238的裂变反应截面接近零。如果没有任何其他机制提供额外的能量,原子核不会裂变,而只会吸收中子,就像铀-238可以吸收慢中子的甚至部分快中子而变成铀-239。引发裂变所需的剩余能量可以由另外两种机制提供:其中一种是提高入射中子的动能,中子动能超过1MeV(称为快中子)时,随着中子能量增加可裂变重核素发生裂变的可能性随之增加。这种高能中子能够直接使铀-238发生裂变。然而,这一过程在核反应堆中不可能大量发生,因为任何类型裂变产生的裂变中子中只有很小一部分具有足够的能量使铀-238有效裂变(裂变中子的最概然能量为2 MeV,但能量中值仅为0.75 MeV,这意味着其中一半的中子小于这一能量)。[5]

在锕系元素中,那些具有奇中子数的核素(例如具有143个中子的铀-235)与具有偶中子数的相同元素的同位素(例如具有146个中子的铀-238)相比,俘获一个额外中子释放的结合能要多1到2 MeV。这些额外的结合能来自中子配对效应,泡利不相容原理允许俘获中子占据与原子核中最后一个中子相同的核轨道,从而使两者形成一对。在这样的核素中,裂变对中子动能没有要求,因为所有必要的能量都可以通过吸收中子的结合能提供,无论是慢中子还是快中子(前者用于慢中子核反应堆,后者用于快中子反应堆和武器)。如上所述,可裂变元素中有一部分可以利用它们自己的裂变中子产生裂变(从而可以使用相对少量的物质即可发生核裂变链式反应),这些元素被称为“易裂变元素”。易裂变同位素的例子有铀-235和钚-239。
\begin{figure}[ht]
\centering
\includegraphics[width=6cm]{./figures/6eaac8000a9e515f.png}
\caption{液滴模型解释二元裂变过程。能量输入使原子核变形为“雪茄”形,然后是“花生”形,接着由于两部分之间的距离超过了短程核力吸引范围而分裂,然后被电荷斥力推开。液滴模型预言两个裂变碎片具有相同大小。而如通常实验观察到的,核壳层模型理论允许它们大小不同。} \label{fig_HLB_3}
\end{figure}

\textbf{能量输出}

典型的裂变事件释放大约两亿电子伏特(200 MeV),相当于大约2万亿开尔文。无论是可裂变元素还是易裂变元素,发生裂变的同位素种类对对释放的能量只有很小的影响。这可以通过结合能曲线(下图)看出,注意到锕系核素的平均结合能约为每个核子7.6 MeV,从曲线左侧可以看出裂变产物的结合能趋于每个核子8.5 MeV左右。因此,在锕系元素质量范围内的任何同位素的裂变事件中,起始元素的每个核子大约释放0.9 MeV能量。铀-235被慢中子诱发裂变释放的能量几乎与铀-238被快中子诱发裂变释放的能量相同。这种能量释放规律对钍和各种次锕系元素也适用。[6]

相比之下,大多数化学物质氧化反应(如煤或TNT的燃烧)单个反应最多释放几个eV 的能量。所以,核燃料的单位质量可用能量比普通化学燃料大至少一千万倍。核裂变的能量以裂变产物和碎片的动能及电磁辐射(γ射线)的形式释放。在核反应堆中,这些能量通过热粒子、γ射线与反应堆工作物质的原子碰撞而转化为热能,工作物质通常为水,部分为重水或者熔盐.
\begin{figure}[ht]
\centering
\includegraphics[width=6cm]{./figures/65e0bb7b5fd8acc1.png}
\caption{库仑爆炸动画,演示正电荷原子核簇的情况,类似于裂变碎片簇。色调与原子核电荷成正比。在这个时间尺度上的电子(较小的点)只能通过频闪观测到,色调水平与其动能成正比。} \label{fig_HLB_4}
\end{figure}
当一个铀核分裂成两个子核碎片时,铀核质量的0.1\%[7]转化为裂变能量,约200MeV。对于铀-235(总平均裂变能量202.79 MeV[8]),通常约为169 MeV表现为子核的动能,由于库仑排斥作用,子核分离的速度约为光速的3\%。此外,每次裂变平均发射2.5个中子,每个中子的平均动能约为2 MeV(总计4.8 MeV)。[9]裂变反应也释放出约7MeV的瞬发伽马射线,也就是说核裂变爆炸或临界事故中约3.5\%的能量以伽马射线形式释放,不到2.5\%的能量以快中子动能形式释放(两种辐射站总能量的约6\%),其余能量作为裂变碎片的动能(碎片的能量几乎瞬间就通过与周围物质的撞击转化为热能)。[10][11]在原子弹中,这种热量可能会使炸弹核心的温度升高到1亿开尔文,并通过产生次级软X射线将部分能量转换为电离辐射,而在核反应堆中,裂变碎片动能主要产生低温热,本身很少或部产生电离辐射。

所谓的中子弹(增强辐射武器)以电离辐射(特别是中子)的形式释放更大比例的能量,但这些都是依靠核聚变阶段产生额外辐射的热核装置,在纯核裂变武器中,辐射能量占总能量的比例总是在6\%附近。

每次核裂变反应瞬时释放能量约181 MeV,为全时段释放总能量的89\%。剩余约11\%的能量以裂变产物的具有各种半衰期的β衰变以及与这些β衰变相关联的延迟γ射线形式释放。例如,在铀-235中,这些延迟能量包括约6.5MeV的β射线,8.8MeV的反中微子(与β射线同时产生),以及约6.3MeV的处于激发态的β衰变产物的延迟伽马射线中(平均每次裂变约产生10次伽马射线发射)。因此,裂变总能量的大约6.5\%在事件发生后的某个时间作为非即时或延迟电离辐射释放,延迟电离能量几乎被伽马射线和β射线平分。

在已经运行一段时间的反应堆中,放射性裂变产物将积累到稳态浓度,使得它们的衰变率等于它们的形成率,从而它们对反应堆热量(通过β衰变)的贡献比例与这些辐射能量占裂变能量的比例相同。在这些条件下,延迟电离辐射(来自放射性裂变产物的延迟γ和β辐射)对反应堆稳态热功率的贡献约为6.5\%。当反应堆突然关闭(急停)时,剩余的热量输出就来源于此。因此,一旦反应堆关闭时,反应堆衰变热输出的初始值为全稳态裂变功率的6.5\%。然而,在几小时内,由于这些同位素的衰变,衰变功率会变小很多。

剩余的延迟能量(8.8MeV/202.5MeV=总裂变能量的4.3\%)以反中微子形式释放,反中微子不被认为是“电离辐射”,原因是反中微子不会被反应堆捕获而产生热量,而是以接近光速直接穿过所有材料(包括地球)逃逸到行星际空间(吸收的量极小)。中微子辐射通常不被归类为电离辐射,因为它几乎完全不被吸收,因此不会产生电离效应(尽管有极小概率发生中微子电离事件)。而几乎所有其余的辐射(6.5\%延迟的β和γ辐射)最终都在反应堆堆芯或其屏蔽层中转化为热量。

一些涉及中子吸收或最终产生能量的过程非常重要,如当中子被铀-238原子捕获并产生钚-239时,中子动能不会立即产生热量,但是如果钚-239后续发生裂变,则会释放这些能量。另一方面,从裂变子产物发射的半衰期最高达几分钟的缓发中子对反应堆控制非常重要,因为当核反应在缓发临界区进行时,要依赖这些中子进行超临界链式反应(其中每个裂变循环产生的中子多于吸收的中子),因此这些中子会确定一个核反应规模翻倍的特征时间。如果没有它们的存在,核链式反应将是瞬发临界的,其反应规模的增加比人类干预所能控制的要快。在这种情况下,第一个实验性原子反应堆在操作人员能够手动关闭之前,就可能已经跑到了危险和混乱的“瞬发临界反应”状态(为此,设计者恩利克·费密引入了由电磁驱动的的辐射计数触发控制棒,它可以自动落入芝加哥一号堆的中心)。如果这些缓发中子被俘获而不产生裂变,它们最终也会产生热量。[12]

\subsubsection{1.3 裂变产物和结合能}
在裂变中,倾向于产生质子数为偶数的碎片,这被称为裂变碎片电荷分布的奇偶效应。然而,裂变碎片\textbf{质量数}分布没有奇偶效应。这一结果归因于核子对断裂。

在核裂变事件中,原子核可以分裂成较轻原子核的不同组合,但最常见的不是分裂成质量大约相等的两个质量数约为$120\mathbf{u}$的原子核,而是裂变为质量稍有区别的两个部分(取决于裂变核素和过程),其中一个子核的质量约为90至$100\mathbf{u}$, 另一个约为130到$140\mathbf{u}$。[13]不相等的裂变在能量上更有利,因为这允许一个产物的质量数更接近能量最低的60 $\mathbf{u}$(约为裂变元素质量的四分之一),而另一个原子核的质量为135 $\mathbf{u}$仍然很接近最紧密结合原子核的质量范围(换句话说,原子质量120处左侧的结合能曲线比右侧稍陡)。

\subsubsection{1.4 活化能和结合能曲线的来源}
重元素的核裂变能够产生可利用的能量是因为原子序数和原子量接近Ni-62和铁-56附近的中等质量原子核具相较更重的核素具有更大的平均结合能(每个核子结合能的平均值),所以当重的原子核被分裂时能量被释放出来。裂变产物的质量和(Mp)的质量小于初始裂变燃料核的质量(M)。根据质能等效性公式E=mc2,超过部分的质量δm = M –Mp以光子(伽马射线)和裂变碎片的动能的形式转化为能量释放。
