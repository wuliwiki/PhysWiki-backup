% 新基础集合论(综述)
% license CCBYSA3
% type Wiki

本文根据 CC-BY-SA 协议转载翻译自维基百科\href{https://en.wikipedia.org/wiki/New_Foundations}{相关文章}。

在数学逻辑中,新基础(New Foundations,简称 NF)是一种非良基、可有限公理化的集合论,由威拉德·范·奥曼·奎因构思,旨在简化《数学原理》中的类型论。
\subsection{定义}
NF 的良构公式是命题演算的标准公式,具有两个基本谓词:相等(\(=\))和成员关系(\(\in\))。NF 可以仅通过两个公理模式来表述:

\begin{itemize}
\item 外延性:具有相同元素的两个对象是相同的对象。形式化地说,给定任意集合 \( A \) 和任意集合 \( B \),如果对于任意集合 \( X \),\( X \) 是 \( A \) 的成员当且仅当 \( X \) 是 \( B \) 的成员,则 \( A \) 等于 \( B \)。
\item 受限的理解公理模式:对于每个分层公式\( \phi \),集合 \( \{x \mid \phi\} \) 存在。
\end{itemize}
一个公式 \( \phi \) 被称为分层的,如果存在一个从 \( \phi \) 的语法结构的各部分到自然数的函数 \( f \),使得:对于 \( \phi \) 中的任意原子子公式 \( x \in y \),满足 \( f(y) = f(x) + 1 \);对于 \( \phi \) 中的任意原子子公式 \( x = y \),满足 \( f(x) = f(y) \)。
\subsubsection{有限公理化}
NF 可以被有限公理化。[1] 这种有限公理化的一个优点是,它消除了分层性的概念。有限公理化中的公理对应于一些自然的基本构造,而分层理解公理虽然强大,但不一定直观。在其入门书籍中,Holmes 选择将有限公理化作为基本框架,并将分层理解公理作为一个定理来证明。[2] 

具体的公理集合可能有所不同,但通常包含以下大部分公理,而其余的可以作为定理证明:[3][1]
\begin{itemize}
\item 外延性:如果 \( A \) 和 \( B \) 是集合,并且对于每个对象 \( x \),\( x \) 是 \( A \) 的元素当且仅当 \( x \) 是 \( B \) 的元素,则 \( A = B \)。[4] 这一公理也可以视为对相等符号的定义。[5][6]
\item 单元素集:对于每个对象 \( x \),集合 \( \iota(x) = \{x\} = \{y \mid y = x\} \) 存在,并称为 \( x \) 的单元素集。[7][8]
\item 笛卡尔积:对于任意集合 \( A \) 和 \( B \),集合\(A \times B = \{(a, b) \mid a \in A \text{ 且 } b \in B\}\)称为 \( A \) 和 \( B \) 的笛卡尔积,并且它的存在性被保证。[9] 该公理可以限制为某个特定的交叉积,例如 \( A \times V \) 或 \( V \times B \) 的存在。[10][11]
\item 逆关系:对于每个关系 \( R \),集合\(R^{-1} = \{(x, y) \mid (y, x) \in R\}\)存在;可以观察到,\( x R^{-1} y \) 当且仅当 \( y R x \)。[12][13][14]
\item 单元素映像:对于任意关系\( R \),集合\(R\iota = \{(\{x\}, \{y\}) \mid (x, y) \in R\}\)存在,并称为\( R \)的单元素映像。[15][16][17]
\item 定义域:如果 \( R \) 是一个关系,则集合\(\text{dom}(R) = \{x \mid \exists y . (x, y) \in R\}\)存在,并称为 \( R \) 的定义域。[12] 这一公理可以通过类型降维操作来定义。[18]
\item 包含关系:集合\([\subseteq] = \{(x, y) \mid x \subseteq y\}\)存在。[19] 等价地,我们可以考虑集合\([\in] = [\subseteq] \cap (1 \times V) = \{(\{x\}, y) \mid x \in y\}\)的存在性。[20][21]
\item 补集:对于每个集合 \( A \),其补集\(A^c = \{x \mid x \notin A\}\)存在。[22]
\item (布尔)并集:如果\(A\)和\(B\)是集合,则它们的并集\(A\cup B=\{x \mid x \in A\text{ 或 }x \in B\text{或两者皆是}\}\)存在。[23]
\item 全集:全集\(V = \{x \mid x = x\}\)存在。显然,对于任何集合\( \),都有\(x \cup x^c = V\)其中\( x^c \)表示\( x \)的补集。[22]
\item 有序对:对于任意对象 \( a \) 和 \( b \),有序对\((a, b)\)存在,并且\((a, b) = (c, d)\iff a = c \text{ 且 } b = d\)这种定义也可以推广到更大的元组。如果使用某种有序对的构造方法,则它可以被定义,而不作为一个独立的公理。[24]
\item 投影:集合\(\pi_1 = \{((x,y), x) \mid x, y \in V\}\)和\(\pi_2 = \{((x,y), y) \mid x, y \in V\}\)存在。这些集合对应于有序对的第一和第二分量的投影关系。[25]
\item 对角线关系:集合\([=] = \{(x, x) \mid x \in V\}\)存在,并被称为相等关系。[25]
\item 集合并:如果\( A \)是一个集合,并且\( A \)的所有元素都是集合,则集合并\(\bigcup [A] = \{x \mid \exists B, x \in B \text{ 且 } B \in A\}\)存在。[26]
\item 相对积:如果\( R \) 和 \( S \)是关系,则相对积\((R | S) = \{(x, y) \mid \exists z, x R z \text{ 且 } z S y\}\)存在。[12]
\item 反交:集合\(x | y = \{z \mid \neg (z \in x \land z \in y)\}\)存在。这一运算等价于补集和并集的组合,其中\(x^c = x | x\)和\(x\cup y =x^c | y^c\)[27]
\item 基数为 1 的集合:所有单元素集的集合\(1 = \{x \mid \exists y : (\forall w, w \in x \leftrightarrow w = y)\}\)存在。[28][29]
\item 元组插入:对于一个关系 \( R \),集合\(I_2(R) = \{(z, w, t) \mid (z, t) \in R\}\)和\(I_3(R) = \{(z, w, t) \mid (z, w) \in R\}\)存在。[30][31]
\item 类型降维:对于任何集合 \( S \),集合\(TL(S) = \{z \mid \forall w, (w, \{z\}) \in S\}\)存在。[32][33]
\end{itemize}
\subsubsection{类型化集合论}
新基础集合论与拉塞尔型的非分层类型化集合论密切相关。TST 是《数学原理》中类型论的简化版本,采用线性类型层次结构。在这种多排序理论(many-sorted theory)中,每个变量和集合都被赋予一个类型(type)。通常,类型指数使用上标表示:\(x^n\)表示类型为\( n \)的变量。类型\textbf{0}由未进一步描述的个体组成。对于每个(元)自然数\( n \),类型\( n+1 \)的对象是类型\( n \)对象的集合。相等关系仅适用于相同类型的对象。类型\( n \)的集合仅包含类型\( n-1 \)的成员。TST 的公理包括:外延性:适用于相同(正)类型的集合。理解公理:如果\( \phi(x^n) \)是一个公式,则集合\(\{x^n \mid \phi(x^n)\}^{n+1}\)存在,即:\(\exists A^{n+1}\forall x^n [x^n \in A^{n+1} \leftrightarrow \phi(x^n)]\)是一个公理,其中\( A^{n+1} \)代表集合\(\{x^n \mid \phi(x^n)\}^{n+1}\),并且\( A^{n+1} \)在\( \phi(x^n) \)中不是自由变量。这种类型论比《数学原理》中最初提出的类型论要简单得多,后者包括了关系的类型,而这些关系的参数不一定具有相同的类型。

NF 和 TST 之间存在一种类型标注的添加或删除的对应关系:在 NF 的理解模式中,公式是“分层的”当且仅当该公式可以按照 TST 的规则赋予类型。这意味着 NF 公式可以映射到 TST 公式的集合,其中每个 TST 公式都带有不同的类型索引。这种映射是一对多的,因为 TST 允许多个相似的公式。例如,在 TST 公式中,将所有类型索引提升 1,仍然会得到一个新的、有效的 TST 公式。
\subsubsection{纠缠类型论}
纠缠类型论(TTT)是TST的扩展,其中每个变量的类型由\textbf{序数}而非\textbf{自然数}标注。其良构的原子公式包括:相等关系:\( x^n = y^n \)成员关系:\( x^m \in y^n \)(其中 \( m < n \))TTT 的公理与TST的公理相同,但其中每个类型为\( i \)的变量都会被映射到一个变量\( s(i) \),其中 \( s \)是一个递增函数。

TTT 被认为是一种“怪异的”理论,因为它的每个类型都以相同的方式与所有较低类型相关。例如:类型 2 的集合既可以包含类型 1 的成员,也可以包含类型 0 的成员。\textbf{外延性公理}声明,类型 2 的集合仅由其类型 1 成员或类型 0 成员唯一确定。与 TST 不同:在 TST 中,自然模型满足每个类型\( i+1 \)都是类型\( i \)的\textbf{幂集}。在 TTT 中,每个类型同时被解释为所有较低类型的幂集。尽管如此:NF 的模型可以很容易地转换为 TTT 的模型,因为在NF 中,所有类型本质上都是相同的。反过来,经过更复杂的论证,可以证明 TTT 的一致性(Consistency of TTT)能够推出 NF 的一致性。[34]