% C++ 原子变量原子操作笔记
% keys 原子变量|原子操作|C++|多线程|并发
% license Usr
% type Tutor

\begin{issues}
\issueDraft
\issueTODO
\end{issues}

C++ 中原子变量(atomic)是在多线程编程中的常用同步机制。它可以避免竟态条件(race condition)和死锁(dead lock)问题。它可以保证对共享变量的操作在执行时不会被其他线程的操作干扰。

举个例子,以下 C++ 代码:
\begin{lstlisting}[language=cpp]
#include <iostream>
#include <thread>
using namespace std;
int n(0);

void cnt1e4() {
	for(int i(1); i <= 1e4; ++i) ++ n;
}

int main() {
	thread ths[100];
	for(thread &thr: ths) thr = thread(cnt1e4);
	for(thread &thr: ths) thr.join();
	cout << n << endl;
	return 0;
}
\end{lstlisting}
理论输出应为 $10^4 \times 100 = 1000000$,而实际输出总小一些,例如在某品牌轻薄本上运行三次结果分别为:
\begin{figure}[ht]
\centering
\includegraphics[width=10cm]{./figures/a6a955b5d96c1353.png}
\caption{第一次运行结果} \label{fig_atomVO_1}
\end{figure}
\begin{figure}[ht]
\centering
\includegraphics[width=10cm]{./figures/5000b1ff19d8f7a3.png}
\caption{第二次运行结果} \label{fig_atomVO_2}
\end{figure}
\begin{figure}[ht]
\centering
\includegraphics[width=10cm]{./figures/bf4fe703627db114.png}
\caption{第三次运行结果} \label{fig_atomVO_3}
\end{figure}
这类问题尤其在大量不同线程对于同一变量进行操作时出现,geng'duo'de'xian'c