% 环的理想
\pentry{环\upref{Ring}}

\subsection{概念的来源}

环$R$的理想$I$,是一种“正规子环”,即它是$R$的子环,同时使得商集$R/I$能自然成环,就像正规子群的作用一样.之所以不叫正规子环,是因为理想最初来自代数数论,库默尔(Ernst Eduard Kummer)定义了一个他称为“理想数”的概念,证明了费马大定理在$n<100$时大多数情况成立.后来,戴德金(Julius Wilhelm Richard Dedekind)发现,库默尔定义的理想数,正是能诱导出商环的“正规子环”,于是直接借用了“理想数”的名字,将其命名为“理想”.

给定环$R$和它的子环$I$,那么$I$要满足什么条件才能使得$R/I$成环呢?

显然,$I$关于环的加法,得构成一个正规子群,而这是天然满足的,因为环的加法群是阿贝尔群,而阿贝尔群的一切子群都是正规子群.

$R/I$中的每个元素,被定义为$I$作为子群的陪集.元素$a\in R$所在的陪集就是$a+I$.陪集之间显然可以进行加法运算:
\begin{equation}
(a+I)+(b+I)=a+b+I+I=(a+b)+I
\end{equation}

这满足诱导运算的要求:$a$的陪集加$b$的陪集,等于$a$加$b$的陪集.

为了让$R/I$诱导一个环乘法,我们还需要:$a$的陪集乘$b$的陪集,等于$a$乘$b$的陪集.也就是说,
\begin{equation}
(a+I)(b+I)=ab+aI+bI+I=ab+I
\end{equation}

这就意味着$aI+bI\in I$,对于任意的$a, b\in R$都成立.那就是说,对于任意的$a\in R$,都有$aI\in I$.

如果$I$是$R$的子环,并且对于任意的$a\in R$,都有$aI\in I$,那么我们就可以利用$R$的环运算,诱导出$R/I$上的环运算.这样的$I$就是我们要的\textbf{理想}.

\subsection{理想的定义}

\begin{definition}{理想}
给定环$R$,则$R$的一个子集$I$是$R$的一个理想,当且仅当$I$是$R$的子环,且满足吸收律:$\forall r\in R$,有$rI\in I$.
\end{definition}

\begin{example}{多项式环}
定义多项式环$R[x]$的时候,我们说$x$是一个自变量,言下之意就是,$R[x]$中的多项式$f(x)$都表示函数;两个多项式$f(x)$和$g(x)$只有每个系数都相等才被认为相等,如果任何系数不相等,那么认为$f(x)\not=g(x)$.

如果取某个元素$a$(不一定属于$R$),那么$f(a)$就不是一个函数,而是一个固定的元素了(不一定属于$R$).比如说,取$R$是整数环,那么$2x$就是一个函数,图像表示为一条直线;但$2\sqrt{2}$就是一个元素,尽管它不在整数环中.


\end{example}
