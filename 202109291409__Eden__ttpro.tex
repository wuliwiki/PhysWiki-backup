% 节流过程
% 焓|焦汤系数|焦耳-汤姆孙效应

\pentry{热力学第一定律\upref{Th1Law}}

绝热条件下,气体从高压的一边经过多孔塞不断地流到低压的一边,并达到定常状态;多孔塞两边各维持着较高的气压 $p_1$ 和较低的气压 $p_2$,这个过程叫做节流过程.

\begin{figure}[ht]
\centering
\includegraphics[width=7cm]{./figures/ttpro_1.png}
\caption{节流过程} \label{ttpro_fig1}
\end{figure}

由于整个过程是绝热的,所以 $\Delta Q=\Delta U+p\Delta V=0$.设气体在通过多孔塞前内能为 $U_1$,体积为 $V_1$,通过多孔塞后内能为 $U_2$,体积为 $V_2$,那么 $U_2-U_1+p_2V_2-p_1V_1=0$,即 $U_2+p_2V_2=U_1+p_1V_1$.我们可以定义系统的另一状态函数:\textbf{焓} $H=U+pV$.那么 $H_1=H_2$,在节流过程前后气体的焓值相等.

现在我们要研究在节流过程中,气体降温