% 天津大学 2012 年考研量子力学答案
% keys 考研|天津大学|量子力学|2012|答案

\begin{issues}
\issueDraft
\issueTODO
\end{issues}

\subsection{ }
\begin{enumerate}
\item 频率为$\omega$的简谐振子出于状态$\psi(x)=\frac{1}{\sqrt{5}}\psi_0(x)-\sqrt{\frac{2}{5}}\psi_1(x)-A\psi_2(x) $,将其归一化,并求能量平均值.\\
解:由$\abs{\frac{1}{\sqrt{5}}}^2+\abs{-\sqrt{\frac{2}{5}}}^2+\abs{A}^2=1$可得$A=\sqrt{\frac{2}{5}}$\\
因为$E_n=(n+\frac{1}{2})\hbar \omega$,所以$E_0=\frac{1}{2}\hbar\omega$,$E_1=\frac{3}{2}\hbar\omega$,$E_2=\frac{5}{2}\hbar\omega$.\\
$\bar{E}=(\frac{1}{\sqrt{5}})^2\cdot\frac{1}{2}\hbar\omega+(-\sqrt{\frac{2}{5}})^2\cdot\frac{3}{2}\hbar\omega+(\sqrt{\frac{2}{5}})^2\cdot\frac{5}{2}\hbar\omega=\frac{17}{10}\hbar\omega $
\item 波函数为什么可以归一化?\\
答:由于粒子必定要在空间中的某一点出现,所以粒子在空间各点出现的概率总和为1.因而粒子在空间各点出现的概率只决定了波函数在空间各点的相对强度,而不决定强度的绝对大小.
\item 三维空间转子的哈密顿量是$H=\frac{L^2}{2I}$,能量简并度是多少?
答:
\end{enumerate}
\subsection{ }
质量为$m$的粒子,处在区间$(0,a)$无限深势阱中运动,归一化函数为$\psi(x)=x(x-a)\sqrt{\frac{30}{a^5}} $.
\begin{enumerate}
\item 计算粒子处在某个本征态的概率.
\item 写出任一时刻的波函数.
\end{enumerate}
解:
\begin{enumerate}
\item 一维无限深势阱的本征函数和本征值分别为:
\begin{align}
&\psi_n(x)=\sqrt{\frac{2}{a}}\sin{\frac{n\pi}{a}x}\\
&E_n=\frac{n^2\pi^2\hbar^2}{2ma^2}
\end{align}
\begin{equation}
\begin{aligned}
C_n &=\int\psi^{*}_{n}(x)\psi(x)\dd{x}\\
&=\sqrt{\frac{2}{a}}\int^{a}_{0}\sin{\frac{n\pi}{a}x}\cdot(x-a)x\sqrt{\frac{30}{a^5}}\dd{x}\\
&=\frac{4\sqrt{15}}{n^3\pi^2}[(-1)^n-1]
\end{aligned}
\end{equation}
\begin{equation}
w_n=\abs{C_n}^2=\leftgroup{}
\end{equation}
\item 
\begin{equation}
\psi(x)=\sqrt{\frac{2}{a}}\sum^{\infty}_{n=1}C_n\cdot\sin{\frac{n\pi}{a}x}\cdot e^{-\frac{n^2\pi^2\hbar^2}{2ma^2}t}
\end{equation}
\end{enumerate}
\subsection{ }
一个体系的哈密顿量是$H=\frac{P^2_x+(P_y-qB_x)^2+P^2_z}{2m}$,其中$P_x$,$P_y$,$P_z$分别为三个方向的动量算符,求体系的能级和波函数.\\
解:
\begin{equation}
\hat{H}=\frac{P^2_x+(P_y-qB_x)^2+P^2_z}{2m}
\end{equation}
$\qty{P_y,P_z,H}$相互对易有共同本征函数.\\
\begin{equation}
\begin{aligned}
\psi(x,y,z)&=\psi_{P_y}(y)\psi_{P_z}(z)\psi(x)\\
\psi_{P_y}(y)&=\frac{1}{\sqrt{2\pi\hbar}}e^{\frac{i}{\hbar}P_{y} \cdot y}\\
\psi_{P_z}(z)&=\frac{1}{\sqrt{2\pi\hbar}}e^{\frac{i}{\hbar}P_{z} \cdot z}
\end{aligned}
\end{equation}
本征方程$\hat{H}\psi(x,y,z)=E\psi(x,y,z) $\\
\begin{equation}
\begin{aligned}
\frac{P^2_x+(P_y-qB_x)^2+P^2_z}{2m}\psi(x,y,z)&=E\psi(x,y,z)\\
\qty[\frac{P^2_x}{2m}+\frac{1}{2}m\qty(\frac{qB}{m})^2\cdot \qty(x-\frac{P_y}{qB})^2]\psi(x)&=\qty[E-\frac{P^2_z}{2m}]\\
\end{aligned}
\end{equation}
令$\omega=\frac{qB}{m}$,$x_0=\frac{P_y}{qB}$,$\epsilon=E-\frac{P^2_z}{2m}$
\begin{equation}
\qty[\frac{P^2_x}{2m}+\frac{1}{2}m\omega^2(x-x^2_0)]\psi(x)=\epsilon\psi(x) 
\end{equation}
$\epsilon=(n+\frac{1}{2}\hbar\omega),n=0,1,2\dots$
\begin{equation}
\psi_n(x-x_0)=N_ne^{-\frac{\alpha^2(x-x_0)^2}{2}}H_n[\alpha(x-x_0)]
\end{equation}
所以$\hat{H}$体系能级为$E=\epsilon+\frac{P^2_z}{2m}=(n+\frac{1}{2})\hbar\omega ,\quad (n=0,1,2\dots)$.\\
体系的波函数为:\\
\begin{equation}
\begin{aligned}
\psi(x,y,z)&=\frac{1}{2\pi\hbar}e^{\frac{i}{\hbar}(\hat{P}_{y}\hat{y}+\hat{P}_{z}\hat{z})}\psi_n(x-x_0)\\
&=\frac{1}{2\pi\hbar}e^{\frac{i}{\hbar}(\hat{P}_{y}\hat{y}+\hat{P}_z\hat{z})}\cdot \psi_n(x-\frac{P_y}{qB})\\
&=\frac{1}{2\pi\hbar}e^{\frac{i}{\hbar}(\hat{P}_y\hat{y}+\hat{P}_z\hat{z})}\cdot\psi_n(x-\frac{P_y}{qB})
\end{aligned}
\end{equation}
\subsection{ }
解:在$H_0=\frac{p^2}{2m}+\frac{1}{2}m\omega^2x^2$体系下,波函数为$\psi(x)=N_ne^{-\frac{\alpha^2x^2}{2}}H_n(\alpha x) $,能量为$E_n=(n+\frac{1}{2})\hbar\omega \quad n=1,2,3\dots$.\\
能量一级近似为:
\begin{equation}
E^{(1)}_n=\int \psi^*_n(x)H'\psi_n(x)
\end{equation}