% 四川大学 2006 年硕士物理考试试题
% keys 四川大学|考研|物理|2006年
% license Copy
% type Tutor


\textbf{声明}:“改内容来源于网络公开资料,不保证真实性,如有侵权请联系管理员”

\subsection{简答题}
\begin{enumerate}
\item 将一个带电$+q$、半径为$R$的大导体球$B$移近一个半径为$r$而不带电的小导体球$A$,试判断下列说法是否正确?并说明理由。\\
(1)$B$球电势高于$A$球。\\
(2)以无限远为电势零点,$A$球的电势小于0。\\
(3)在$B$球表面附近任一点的场强等于$\displaystyle \frac{q}{4\pi R^2}$
\item 指出下列有关电场强度与电势$U$的关系的说法是否正确?并简要说明原因。\\
(1)已知某点的就可以确定该点的$U$。\\
(2)已知某点的$U$就可以确定该点的$\vec E$。\\
(3)$\vec E$不变的空间,$U$也一定不变。\\
(4)$U$值相等的曲面上,$\vec E$值不一定相等。
\item 一根通有 $20A $电流的无限长细直导线,放在磁感应强度为$B=10^{-3}T$的均匀外磁场中,导线与外磁场正交。试确定磁感应强度为零的各点的位置。
\end{enumerate}
\subsection{应用题}
\begin{enumerate}
\item 如图1所示,有一弯成$\theta$角的金属架 COD放在磁场中,磁感强度$\vec B$的方向垂直于金属架 COD所在平面。一导体杆MN垂直于 OD边,并在金属架上以恒定速度$\bar v$向右滑动,$\bar v$与MN垂直.设$t=0$时,$x=0$.分别求下列两种情形下,框架内的感应电动势$E_i$:\\
(1)磁场分布均匀,且$\vec B$不随时间改变。\\
\begin{figure}[ht]
\centering
\includegraphics[width=8cm]{./figures/6071f7b6ca21b4a5.png}
\caption{} \label{fig_CD06_1}
\end{figure}
(2)磁场随时间和空间都变化,且磁场强度$B=Kx\sin \omega t$。
\item (\textbf{凝聚杰物理、光学、生物医学物理、应用电子技术专业考生必作})\\
如图2所示,一半径为$R$的均匀带正电圆环,其电荷线密度为$\lambda$。在其轴线上有A、B两点,它们与环心的距离分别为$\displaystyle \bar{OA}=\sqrt{3}R,\bar{OB}=\sqrt{8}R$,一质量为 $m$、电荷为$q$的粒子从A点运动到B点,求在此过程中电场力所作的功。
\begin{figure}[ht]
\centering
\includegraphics[width=8cm]{./figures/0e9d1c146b9fedfc.png}
\caption{} \label{fig_CD06_2}
\end{figure}
\item (\textbf{凝聚态物理、光学、生物医学物理、应用电子技术专业考生必作})\\
如图3所示,一半径为$R$的均匀带电无限长直圆筒,面电荷密度为$\sigma$.该筒以角速度$\omega$绕其轴线匀速旋转。试求圆筒内部的磁感强度。
\begin{figure}[ht]
\centering
\includegraphics[width=8cm]{./figures/a90dd2c1cf450c6c.png}
\caption{} \label{fig_CD06_3}
\end{figure}
\item (\textbf{通论物理、粒子物理与原子核物理、原子分子物理专业必作})\\
如图4所示,一平行板电容器,极板面积为$S$,两板间距离为$d$,其中充有两种各向同性均匀电介质,相对介电常量分别为$\varepsilon_{r1}$和$\varepsilon_{r2}$,且各占一半体积。试证该电容器的电容为\begin{equation}
C=\frac{s_0S}{2d}(\varepsilon_{r1}+\varepsilon_{r2})~
\end{equation}
并说明该电容器相当于左、右两部分作为单独的电容器的并联。
\begin{figure}[ht]
\centering
\includegraphics[width=8cm]{./figures/219519ee81ee28bf.png}
\caption{} \label{fig_CD06_4}
\end{figure}
\item (\textbf{理论物理、粒子物理与原子核物理、原子分子物理专业必作})\\
载有稳恒电流$I_1$的无限长直导线(看成刚体)下用一劲度系数为$K$的轻质弹簧挂一载有稳恒电流$I_2$的矩形线圈。设长直导线通电前弹簧长度为$L_0$,通电后矩形线圈将向下移动一段距离,求当磁场对线圈作的功满足$A=\mu_0I_1I_2a/2 \pi$时,线圈、弹簧、地球组成的系统的势能变化(忽略感应电流对$I_2$的影响)。
\begin{figure}[ht]
\centering
\includegraphics[width=6cm]{./figures/4cfb277eed3abea3.png}
\caption{} \label{fig_CD06_5}
\end{figure}
\item (\textbf{各专业必作})\\
薄钢片上有两条紧靠的平行细缝,用波长$\lambda=546.1nm(1nm=10^-9m)$的平面光波正入射到钢片上。屏幕距双缝的距离为$D=2.00m$,测得中央明条纹两侧的第五级明条纹间的距离为$\delta x=12.0nm$。\\
(1)求两缝间的距离。\\
(2)从任一明条纹(记作 0)向一边数到第 20条明条纹,共经过多大距离?\\
(2)如果使光波斜入射到钢片上,条纹间距将如何改变?
\item (\textbf{凝聚态物理、光学、生物医学物理、应用电子技术专业考生必作})\\
用每毫米$ 300 $条刻痕的衍射光栅来检验仅含有属于红和蓝的两种单色成分的光谱,已知红谱线波长$\lambda_R$在$0.63-0.76\mu m$ 范围内,蓝谱线波长$\lambda_B$名在 $0.43-0.49 \mu m$范围内,当光垂直入射到光栅时,发现在衍射角为24.46°处,红蓝两谱线同时出现。\\
(1)在什么角度下红蓝两谱线还会同时出现?\\
(2)在什么角度下只有红谱线出现?
\item (\textbf{各专业必作})\\
双星之间的角距离为$1*10^{-6}rad$,其幅射波长为5770A和5790A两个波长。\\
(1)望远镜的口径需要多大才能分辨此双星的像?\\
(2)若要分辨此两波长,光栅条数应为多少?
\item ( \textbf{凝聚态物理、光学、生物医学物理、应用电子技术专业考生必作})\\
一块厚度为 $0.04mm$的方解石晶片,其光轴平行于表面,将它插入正交偏振片之间,且使主截面与第一偏振片的透振方向成$\theta$($\theta \neq 0,\theta \neq 90$°)角。试问哪些光不能透过该装置?已知方解石的$n_0=1.658,n_e=1.486$。
\item (\textbf{理论物理、粒子物理与原子核物理、原子分子物理专业必作})\\
正弦光栅的屏函数为$\widetilde t(x,y)=t_0+t_1\cos(2\pi f_x x+2\pi f_y y)$,现将它沿斜方向平移$\Delta r=(\Delta x,\Delta y)$。写出移动后的屏函数表达式。
\item (\textbf{理论物理、粒子物理与原子核物理、原子分子物理专业必作})\\
在偏振光干涉的装置中,两偏振片的透光方向夹角为60°,两者之间插入一个顶角$\alpha=30'$的石英尖劈,其光轴平行于表面,尖劈的主截面与两偏振片的透光方向都成 30°角。以波长 589.3nm 的钠黄光垂直入射。求:\\
(1)透射光的光强分布;
(2)干涉条纹的反衬度。已知石英的折射率$n_0=1.54424,n_e=1.55335$。
\end{enumerate}