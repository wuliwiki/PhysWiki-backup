% 相对论
% license CCBYSA3
% type Wiki

(本文根据 CC-BY-SA 协议转载自原搜狗科学百科对英文维基百科的翻译)


相对论通常包含阿尔伯特·爱因斯坦(Albert Einstein)的两个相互关联的理论:狭义相对论和广义相对论。[1]狭义相对论适用于基本粒子及其相互作用,描述了除引力以外的所有物理现象。广义相对论解释了引力定律及其与其他自然力的关系,[2]适用于宇宙学和天体物理学领域,包括天文学。[3]

该理论在20世纪改变了理论物理学和天文学,取代了主要由艾萨克·牛顿(Isaac Newton)创立的有200年历史的力学理论。[3][4][5]它引入了一些概念,包括作为时间和空间统一实体的时空、同时性的相对性、运动学和引力时间膨胀、以及长度收缩。在物理学领域,相对论改进了基本粒子及其基本相互作用的科学,同时迎来了核时代。借助相对论,宇宙学和天体物理学预测了非同寻常的天文现象,如中子星、黑洞和引力波。[3][4][5]

\begin{figure}[ht]
\centering
\includegraphics[width=6cm]{./figures/1aba0c5b7b223588.png}
\caption{广义相对论中时空曲率三维类比的二维投影} \label{fig_XDL_1}
\end{figure}

\subsection{发展和认可}


\subsection{狭义相对论}


\subsection{广义相对论}


\subsection{实验证据}


\subsubsection{4.1 狭义相对论的实验验证}





