% 相对论
% license CCBYSA3
% type Wiki

(本文根据 CC-BY-SA 协议转载自原搜狗科学百科对英文维基百科的翻译)


相对论通常包含阿尔伯特·爱因斯坦(Albert Einstein)的两个相互关联的理论:狭义相对论和广义相对论。[1]狭义相对论适用于基本粒子及其相互作用,描述了除引力以外的所有物理现象。广义相对论解释了引力定律及其与其他自然力的关系,[2]适用于宇宙学和天体物理学领域,包括天文学。[3]

该理论在20世纪改变了理论物理学和天文学,取代了主要由艾萨克·牛顿(Isaac Newton)创立的有200年历史的力学理论。[3][4][5]它引入了一些概念,包括作为时间和空间统一实体的时空、同时性的相对性、运动学和引力时间膨胀、以及长度收缩。在物理学领域,相对论改进了基本粒子及其基本相互作用的科学,同时迎来了核时代。借助相对论,宇宙学和天体物理学预测了非同寻常的天文现象,如中子星、黑洞和引力波。[3][4][5]

\begin{figure}[ht]
\centering
\includegraphics[width=6cm]{./figures/1aba0c5b7b223588.png}
\caption{广义相对论中时空曲率三维类比的二维投影} \label{fig_XDL_1}
\end{figure}

\subsection{发展和认可}

1905年,阿尔伯特·爱因斯坦在阿尔伯特·迈克耳孙、亨德里克·洛伦兹、儒勒·昂利·庞加莱等人的理论成果和实证结果的基础上,发表了狭义相对论。马克斯·普朗克、赫尔曼·闵可夫斯基等人做了后续研究。

爱因斯坦在1907年至1915年间发展了广义相对论,并在1915年后得到许多其他人的贡献。广义相对论的最终形式发表于1916年。[3]

“相对论”一词是基于普朗克于1906年使用的“相对理论”(德语:Relativtheorie)这一表达,他强调了相对论是如何运用相对论原理的。在同一篇论文的讨论部分,阿尔弗雷德·布切勒首次使用了“相对论”这个表述(德语:Relativitätstheorie)。[6][7]

截止20世纪20年代,物理界理解并接受了狭义相对论。[8]它迅速成为理论家和实验家在原子物理学,核物理学和量子力学等新领域的重要和必要的工具。

相比之下,广义相对论似乎没有那么有用,只是对牛顿重力理论的预测做了一些微小的修正。[3]它似乎没有什么被实验论证的潜力,因为它的大部分理论都是基于天文的范围。它的数学论看似很复杂,只有少数人能完全理解。约在1960年,广义相对论成为物理学和天文学的核心。应用于广义相对论的新数学技术简化了计算,使其概念更容易可视化。随着天文现象的发现,如类星体(1963年),3-开尔文微波背景辐射(1965年),脉冲星(1967年)和第一批黑洞候选者(1981年),[3]该理论解释了它们的属性,并对它们进行的测量进一步证实了这一理论。

\subsection{狭义相对论}

狭义相对论是关于时空结构的理论。这是在爱因斯坦1905年的论文《论动体的电动力学》中提出的,其他许多物理学家对狭义相对论也做出了贡献。狭义相对论基于两个在经典力学中相互矛盾的假设:

\begin{enumerate}
\item 对于所有相对于彼此匀速运动的观察者来说,物理定律是相同的(相对论原理)。
\item 无论观察者的相对运动或光源的运动如何,真空中的光速对他们来说都是一样的。
\end{enumerate}


\subsection{广义相对论}


\subsection{实验证据}


\subsubsection{4.1 狭义相对论的实验验证}



\subsubsection{4.2 广义相对论的实验验证}


\subsection{现代应用}


\subsection{参考文献}
