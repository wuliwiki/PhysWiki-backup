% BRST量子化
弦理论的量子化方案有三种:协变量子化,光锥量子化和BRST量子化.三种方案的优劣比较如下:
\begin{itemize}
\item 协变量子化:洛伦兹不变能明显表现出来.有鬼.时空维数是26维难以证明.
\item 光锥量子化:洛伦兹不变不再明显.没有鬼.时空维数是26维容易证明.
\item BRST量子化:洛伦兹不变能明显表现出来.有鬼.时空维数是26维容易证明.
\end{itemize}
\subsubsection{BRST算符}
首先我们来考虑李代数.考虑算符$K_i$,这些算符满足如下的李代数
\begin{equation}
[K_i,K_j] = f_{ij}{}^k K_k~.
\end{equation}
其中$f_{ij}{}^k$被称作理论的结构常数.这些结构常数满足
\begin{equation}
f_{ij}{}^m f_{mk}{}^i + f_{jk}{}^m f_{mi}{}^l+f_{ki}{}^m f_{mj}{}^l = 0 ~. 
\end{equation}
现在我们引入两个鬼场,记作$b_i$和$c_j$,它们满足反对易关系
\begin{equation}
\{ c^i, b_j \} = \delta^i_j~.
\end{equation}
我们现在回忆一下,一个场$\phi(z,\bar z)$在共形变换$z\rightarrow w(z)$下具有如下变换的时候
\begin{equation}
\phi(z,\bar z) = \bigg( \frac{\partial w}{\partial z} \bigg)^h \bigg( \frac{\partial w}{\partial \bar z} \bigg)^{\bar h} \phi (w,\bar w)~.
\end{equation}
我们就说,这个场具有共形维度$(h,\bar h)$. $b$和$c$场的共形维度分别是$2$和$-1$. 我们现在来用$b$和$c$这两个鬼场以及$K_i$来构造两个算符.第一个是
\begin{equation}
Q = c^i K_i - \frac{1}{2} f_{ij}{}^k c^i c^j b_k~.
\end{equation}
我们假设$Q = Q^\dagger$.$Q$满足如下性质
\begin{equation}
Q^2 = 0~.
\end{equation}
这个关系也可以写作$\{Q,Q\}=0$.我们把BRST的算符记作$Q$是为了暗示它是这个系统的守恒荷.我们把这叫做BRST荷.

第二个完全由鬼场组成的算符叫做鬼场算符U.它的表达式如下
\begin{equation}
U = c^i b_i~.
\end{equation}
这个算符具有整数的特征值.如果李代数的维度是$n$,那么$U$的特征值就是$0,\ldots ,n$.如果一个态$|\Psi\rangle$满足$U|\Psi\rangle=m|\Psi\rangle$,我们就说这个态具有鬼数$m$.









