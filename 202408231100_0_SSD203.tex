% 首都师范大学 2003 年硕士入学物理考试试题(二)
% keys 首都师范大学|考研|2003年|物理
% license Copy
% type Tutor

\textbf{声明}:“该内容来源于网络公开资料,不保证真实性,如有侵权请联系管理员”
\begin{enumerate}
\item 做微幅振动的单摆悬吊在电梯的天花板上。求电梯加速上升和匀速上升两种情况下的单摆振动周期差。
\item 小球自一斜面上运动到底部,下落高度为h。分两种情况考点:(1)如果将小球视为不受摩擦力的质点,(2)将小球视为无滑滚动的刚体,求在这两种情况下的小球质心位于底部时的速度差。
\item 孤立系中有两个质量任意的质点,它们的初始速度不在一直线上,它们相互间的作用力满足万有引力定律。说明系统满足什么守恒定律,以及两个质点的运动情况。
\item 一均匀带电细棒弯成半径为R的半圆环,棒上总电量为+Q。求:圆心o处的电场强度$\vec E$和电位U。
\item 一个正点电荷Q放在一个内半径为$R_1$,外半径为$R_2$的电介质球壳中心(如图所示),电介质的相对介电常数为$\varepsilon_r$,求:\\
(1)$r<R_1,R_1<r<R_2,r<R_2$,各空间的电位移矢量$\vec D$、电场强度矢量$\vec{E}$、极化强度矢量 $\vec P$ 分布。\\
(2)$r<R_1$空间的点位分布及 $r=R_2$ 面上的极化电荷密度 $\sigma$'。\\
(3)储存在 $R_1<r<R_2$ ,空间的静电场能量。
\begin{figure}[ht]
\centering
\includegraphics[width=6cm]{./figures/39280c599d24c023.png}
\caption{} \label{fig_SSD103_2}
\end{figure}
\item 一无限长载流导线弯成图示形状,电流为I。$\frac{2}{3}$圆弧的半径为 $R$,圆心在o点。求:该载流导线在圆心处产生的磁感应强度 $\vec B$。
\begin{figure}[ht]
\centering
\includegraphics[width=8cm]{./figures/60eed05276edf8d8.png}
\caption{} \label{fig_SSD103_3}
\end{figure}
\item 一根长直导线载有电流I,另有两根平行放置的导体棒,一端接有电阻R,另一端有可滑动的导体ab ,当ab以匀速 $\vec V$向右滑动时(如图)求:\\
(1)感应电动势的大小及回路中的电流的方向。\\
(2)作用在 ab 上的磁场力。
\begin{figure}[ht]
\centering
\includegraphics[width=8cm]{./figures/976e2695e25477b5.png}
\caption{} \label{fig_SSD103_5}
\end{figure}
\item (1)画出氢原子n=3时的能级图(一般结构与精细结构),并标出相应的原子态;\\
(2)画出锂原子中价电子处于n=3时原子的能级图(一般结构与精细结构),并标出相应的原子态;\\
(3)试比较(1)与(2)的能级特点,并说明能级分裂的原因。
\item 写出锌原子(Z=30)的基态电子组态,其基态时最外层为两个s电子,若当其中一个电子被激发到5s态,按L-S耦合可形成哪些原子态?当从激发态同到基态时,可能产生老、多少条光谱线?画出能级跃迁图,并标出相应的原子态。
\item (1)写出$\beta$衰变的三种类型及衰变方程;\\
(2)试解释:原子核具有能级,为什么$\beta$射线的能谱会是连续的?\\
(3)试解释:在原子核中发生了什么实质性的变化,才能释放出原子核内并不存在的电子或正电子?
\end{enumerate} 